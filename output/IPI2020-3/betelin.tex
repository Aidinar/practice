\def\stat{betelin}

\def\tit{ОСНОВНЫЕ ПОНЯТИЯ ПРОГРАММИРОВАНИЯ 
В~ИЗЛОЖЕНИИ~ДЛЯ~ДОШКОЛЬНИКОВ$^*$}

\def\titkol{Основные понятия программирования в~изложении для 
дошкольников}

\def\aut{В.\,Б.~Бетелин$^1$, А.\,Г.~Кушниренко$^2$, А.\,Г.~Леонов$^3$}

\def\autkol{В.\,Б.~Бетелин, А.\,Г.~Кушниренко, А.\,Г.~Леонов}

\titel{\tit}{\aut}{\autkol}{\titkol}

\index{Бетелин В.\,Б.}
\index{Кушниренко А.\,Г.}
\index{Леонов А.\,Г.}
\index{Betelin V.\,B.}
\index{Kushnirenko A.\,G.}
\index{Leonov A.\,G.}
 

{\renewcommand{\thefootnote}{\fnsymbol{footnote}} \footnotetext[1]
{Работа выполнена по теме 0065-2019-0010 госзадания 2020~года 
в~отделе учебной информатики Федерального 
научного центра <<На\-уч\-но-ис\-сле\-до\-ва\-тель\-ский институт системных исследований>> Российской 
академии наук.}}


\renewcommand{\thefootnote}{\arabic{footnote}}
\footnotetext[1]{Федеральный научный центр <<На\-уч\-но-ис\-сле\-до\-ва\-тель\-ский институт системных исследований>> 
Российской академии наук, \mbox{betelin@niisi.msk.ru}}
\footnotetext[2]{Федеральный научный центр <<На\-уч\-но-ис\-сле\-до\-ва\-тель\-ский  институт системных исследований>> 
Российской академии наук, \mbox{agk\_@mail.ru}}
\footnotetext[3]{Московский государственный университет им.\ М.\,В.~Ломоносова; Федеральный научный центр  
<<На\-уч\-но-ис\-сле\-до\-ва\-тель\-ский институт системных исследований>> Российской академии наук; 
Московский педагогический государственный университет, \mbox{dr.l@vip.niisi.ru}}

%\vspace*{-6pt}



\Abst{Развитие информационных технологий сформировало социально-экономический 
запрос на снижение возраста знакомства детей с~программированием. В~результате 
шестилетних усилий авторам удалось разработать и~массово внедрить годовой курс 
программирования для дошкольников, построенный на метафоре программного управления. 
В~процессе развития курса удалось отобрать и~сформулировать набор основных понятий 
программирования, который может быть освоен дошкольниками возраста~6+  
в~дея\-тель\-ност\-но-иг\-ро\-вой форме. Этот набор понятий вводится на примерах программ 
управления движущимися и~неподвижными объектами с~интуитивно понятными, 
обозримыми системами команд. Курс строится на базе беcтекстовой пиктографической 
системы программирования <<ПиктоМир>> разработки ФНЦ НИИСИ РАН. Разработанное 
для курса про\-грам\-мно-ме\-то\-ди\-че\-ское наполнение позволяет каждому дошкольнику 
к~концу курса получить опыт составления и~отладки 120--150~простейших программ.}
  
  \KW{информатика; робот; программа; компьютер; язык программирования; дошкольник; 
<<ПиктоМир>>; пиктограмма}

\DOI{10.14357/19922264200308} 
 
%\vspace*{-6pt}


\vskip 10pt plus 9pt minus 6pt

\thispagestyle{headings}

\begin{multicols}{2}

\label{st\stat}
  
\section{Введение}

  Законодательными структурами власти России федерального уровня 
выдвигаются предложения по понижению возраста знакомства детей 
с~информатикой и~программированием на уровень системы дошкольного 
образования~[1]. Настоящая статья посвящена описанию конкретной методики 
такого\linebreak понижения и~суммирует шестилетний опыт разработки и~массового 
внедрения годового курса <<\mbox{Алгоритмика} для дошколят>>, проводимого 
совместными усилиями ФНЦ НИИСИ РАН и~Департамента образования 
администрации г.~Сур\-гу\-та. Начиная с~осени 2018~г.\ годовой курс проходят 
все выпускники всех подготовительных групп всех до\-школь\-но-об\-ра\-зо\-ва\-тель\-ных 
организаций г.~Сургу\-та~--- 
около 6~тыс.\ детей. В~курсе используется\linebreak
 бестекстовая учебная система 
программирования <<ПиктоМир>>, разработка которой была начата в~ФНЦ 
НИИСИ РАН около 10~лет назад~[2] и~продолжается сегодня в~работах по теме 
госзадания Минобрнауки РФ <<Разработка, реализация и~внедрение семейства 
интегрированных многоязыковых сред программирования$\ldots$>>~[3]. 
Система <<ПиктоМир>> и~методическое обеспечение курса~--- свободно 
распространяемые. Их можно загрузить с~сайта НИИСИ РАН или работать 
с~ними с~по\-мощью браузера через веб-ин\-тер\-фейс~[4].
  
  Информатизация дошкольного образования~--- очень широкая и~важная для 
современной цивилизации тема~[5]. В~данной статье, в~рамках более общей 
работы над 4-лет\-ним курсом программирования для дошкольников и~младших 
школьников, упор делается на конкретной задаче~--- организации первых 
занятий курса программирования для дошкольников. Этот вопрос особенно 
важен, поскольку именно на первых занятиях закладывается фундамент курса, 
осваивается набор основных понятий. 
  
  Накопленный опыт позволил авторам предложить набор основных понятий 
программирования, который может быть органично освоен дошкольниками 
в~дея\-тель\-ност\-но-иг\-ро\-вой форме. Этот набор понятий призван раскрыть 
одну из <<больших идей>> нашей цивилизации~--- {\bfseries\textit{принцип 
программного управ\-ления}}. 
  
  Согласно рубрикации А.\,Л.~Семенова~\cite[п.~12]{6-bet}, одной из целей 
курса информатики должно быть <<освоение информационной картины 
мира>>. Принцип программного управления и~строит такую картину 
в~терминах и~образах, понятных шестилетнему ребенку. Схематично принцип 
может быть объяснен так: \textbf{любую работу, которую человек может 
выполнить, командуя механическим по\-мощ\-ни\-ком-ро\-бо\-том, можно 
перепоручить компьютеру, если человеку удастся составить программу 
выполнения той деятельности, которую роботу надлежит выполнить}. 
  
  Методически правильным представляется двухэтапное изложение принципа 
программного управ\-ле\-ния, при котором на первом этапе излагается 
простейший вариант принципа~--- без обратной связи. Практика работы 
с~дошкольниками показала, что именно этот этап оказывается 
ос\-но\-во\-по\-ла\-гаю\-щим и~более трудным, чем второй этап, на котором вводится 
обратная связь.
  
\section{Принцип программного управления без~обратной связи}

  \textbf{Программа}~--- это план будущей деятельности,\linebreak
   в~процессе которой 
один объект~--- \textbf{компьютер}~-- управляет другим объектом~--- 
\textbf{роботом}~--- по программе, заранее составленной человеком~--- 
\textbf{программистом}~--- по заранее известным \textbf{правилам\linebreak 
составления программ} (\textbf{язык программирования}). Процесс 
\textbf{выполнения} программы компьютером состоит в~том, что компьютер, 
следуя программе, некоторым заранее установленным способом дает роботу 
\textbf{команды}, которые тот \textbf{исполняет}, докладывая компьютеру об 
окончании исполнения каждой команды и~готовности к~приему следующей 
команды. Чтобы компьютер мог выполнить программу, она должна быть ему 
предварительно сообщена (\textbf{загружена в~память} компьютера).
  
  Это описание принципа явно вводит 12~понятий: 
6~{\bfseries\textit{объектов}}, 1~{\bfseries\textit{субъект}} 
и~5~{\bfseries\textit{взаимодействий}} между объектами и~субъектом.
   
\textit{Объекты}: 

  \textbf{программа}; \textbf{компьютер}; \textbf{память компьютера}; 
\textbf{робот};  \textbf{правила составления программ} (\textbf{язык 
программирования}); \textbf{команда}.

\textit{Субъект}: 

  \textbf{программист}.

\textit{Взаимодействия}: 
\begin{itemize}
  \item программист \textbf{составляет} программу;
  \item компьютер \textbf{выполняет} программу, \textbf{давая} роботу 
команды;
\item получив команду, робот ее \textbf{исполняет} и~ждет поступления 
следующей   команды;
  \item компьютер \textbf{загружает в~свою память} сообщенную ему 
программу.
  \end{itemize}
  
  Разумеется, выбранный выше набор понятий и~акценты, сделанные 
в~объяснении принципа, могли бы быть другими. Авторы выбрали указанный 
набор из~12~понятий, исходя из сугубо практических соображений. Дело 
в~том, что на первых занятиях с~дошкольниками педагог должен одновременно 
решить две задачи:
  \begin{enumerate}[(1)]
  \item добиться интуитивного понимания детьми <<правил игры>>, 
интуитивного осознания детьми предложенной системы понятий;\\[-15pt]
  \item пополнить словарный запас детей, научить их использовать в~речевой 
практике термины, выражающие освоенные понятия.
  \end{enumerate}
  
    Предлагаемый выбор понятий позволяет педагогу вчерне решить обе 
задачи примерно за 10--15~получасовых групповых занятий и~добиться 
твердого усвоения системы понятий к~концу годового курса. Важно, что 
данные понятия образуют некоторую замкнутую систему. Скорее всего, для 
ребенка 6~лет, пришедшего на занятия алгоритмикой, данная система понятий 
окажется первой в~его жизни изученной взаимосвязанной \textbf{системой 
научных понятий}. И~это изучение должно быть организовано так, чтобы 
система понятий была понята до конца каждым ребенком.   
  
  Согласно Л.\,С.~Выготскому~\cite{7-bet}, осознание любого общего 
принципа требует комплексного освоения ребенком некоторой 
{\bfseries\textit{системы научных понятий}}: <<Научные понятия являются 
воротами, через которые осознанность входит в~царство детских понятий>>. 
  
  \textbf{Принцип программного управления может быть понят, осознан 
ребенком только после усвоения изложенной выше достаточно сложной 
системы из~12 научных понятий.}
   
  Разумеется, доведение перечисленных понятий до ребенка возраста 6--7~лет 
в~вербальной форме, путем устных объяснений взрослого, невозможно. 
Осознанное усвоение этих понятий станет возможным, только если ребенку 
будут предложены виды деятельности, позволяющие ему в~игровой форме 
<<вжиться>> \textbf{во все} эти 12~понятий, <<пропустить их через себя>>. 


  
  Подчеркнем, что принцип программного управ\-ле\-ния можно вводить 
 по-раз\-но\-му, варьируя набор понятий, которые преподносятся как основные, и~понятий, которые вводятся неявно. Предложенный выше набор 
из~12~основных понятий был подобран так, чтобы максимально облегчить 
ребенку освоение \textbf{каждого} из этих понятий и~\textbf{всей системы} 
понятий в~дея\-тель\-ност\-но-иг\-ро\-вой форме. 

  
  
  Авторам удалось построить первые занятия курса так, 
  чтобы каждый ребенок в~группе смог 
  про-\linebreak\vspace*{-12pt}
  
  \pagebreak
  
  \end{multicols}
  
  \begin{figure*} %fig1
\vspace*{1pt}
 \begin{center}
 \mbox{%
 \epsfxsize=163mm 
 \epsfbox{bet-1.eps}
 }
 \end{center}
   \vspace*{-9pt}
  \Caption{Виртуальные роботы~(\textit{а}) и~реальный робот-игрушка~(\textit{б})}
  %\end{figure*}
  %\begin{figure*} %fig2
\vspace*{18pt}
 \begin{center}
 \mbox{%
 \epsfxsize=149.668mm 
 \epsfbox{bet-3.eps}
 }
 \end{center}
   \vspace*{-9pt}
\Caption{Программа управления роботом, выложенная дошкольником 
из кубиков~(\textit{а}),
и~та же программа, составленная семиклассником~(\textit{б})}
\end{figure*}

\begin{multicols}{2}
  
  \noindent 
  ими\-ти\-ро\-вать выполнение \textbf{всех} пяти перечисленных
выше действий-взаимодействий, выполняя поочередно роль робота, 
компьютера и~программиста, и~смог поработать с~материальным воплощением 
программы, составляя ее и~загружая в~память компьютера и~сталкиваясь 
с~трудностями в~случае нарушения правил составления программы. Это 
удалось сделать за счет использования на первых занятиях следующих 
методических и~технических решений.
  
  Как было предложено Пейпертом еще полвека назад~\cite{8-bet}, дети 
работают не только с~виртуальными (экранными) роботами, но и~с~реальными  
ро\-бо\-та\-ми-иг\-руш\-ка\-ми, которые перемещаются по полу игровой 
комнаты, имитируя перемещения виртуальных роботов на экране планшета 
(рис.~1). 
  
  
   
Реальные роботы управляются звуковыми командами. Эти команды 
<<наблюдаемы>> (слышимы) детьми. 

  Программы составляются из материальных объектов, кубиков, 
с~нанесенными на их грани пикто\-грам\-ма\-ми команд, повторителей и~других 
конст\-рук\-ций языка программирования (рис.~2). \mbox{Функции} компьютера 
выполняет планшет.
  
    




  Загрузка программы в~память компьютера (планшета) состоит в~явно 
проводимом ребенком фотографировании камерой планшета выложенной на 
столе программы, т.\,е.\ некоторой конфигурации кубиков. За этим явным 
действием невидимо для ребенка, скрытно, следует <<понимание>> программы 
компьютером~--- распознавание фотографии с~кубиками процессором 
планшета с~помощью нейронных сетей. Результат этого понимания ребенок 
видит на экране планшета, а как это понимание происходит~--- с~ребенком не 
обсуждается.
  
  
  
  
  Первое синтаксическое правило составления программы из кубиков требует, 
чтобы кубики были выложены в~достаточно ровные ряды, ряды располагались 
друг под другом. Первое семантическое правило выполнения программы 
гласит: при выполнении программы пиктограммы в~рядах читаются слева 
направо, а~ряды читаются сверху вниз. 
  %
  При этом ребенок сталкивается с~тем, что компьютеру не удается <<понять>> 
программы, со\-став\-лен\-ные с~нарушением правил, т.\,е.\ компьютеру 
<<непонятны>> расположения кубиков на столе, в~которых трудно или 
невозможно мысленно разбить выложенные кубики на ряды 
(рис.~3)\footnote{Невыровненность кубиков может рассматриваться как непрерывный аналог 
<<синтаксической неправильности>> программы.}.
  
  Еще две группы правил описывают две конст\-рукции языка 
программирования: числовой повторитель и~подпрограмма с~однобуквенным 
именем.\linebreak Эти правила описывают и~синтаксис, и~семантику и~применяются 
в~ситуациях, когда ребенок, имитируя компьютер, пытается <<понять>> 
выложенную другим ребенком программу и~далее пытается <<понятую>> 
программу выполнить. Эти правила требуют, чтобы пиктограммы 
располагались в~рядах в~определенном порядке. Дети осваивают эти правила 
без затруднений.
  
  Рассматриваемый курс построен так, что на первых занятиях ребенок играет 
со сверстниками и~учебными пособиями (роботом и~набором кубиков 
с~пиктограммами команд) в~сюжетно-ролевые игры. \textbf{Компьютер на 
первых порах используется только по его главному назначению, для 
исполнения загруженных в~его память программ,} и~не используется для 
других целей: ни для составления программ, ни для генерации изображений 
виртуальных роботов и~виртуальных обстановок на экране. На первых занятиях 
курса \textbf{программа, робот и~обстановка, в~которой робот действует, 
являются реальными, а не виртуальными объектами,} и~все 
взаимодействия представляют собой реальные процессы с~участием 
материальных объектов. И~ребенок может осваивать роли объектов в~игре. 
Ребенок может выступать в~роли робота, исполняя звуковые команды, 
поступающие от компьютера или от другого ребенка, выступающего в~роли 
компьютера; ребенок может выступать в~роли компьютера, выполняя 
составленную другим ребенком программу и~командуя при этом третьим 
ребенком, играющим роль робота; ребенок может поработать программистом, 
составляя самостоятельно программу путем перемещения материальных 
кубиков на столе и~переходя позднее\linebreak\vspace*{-12pt}

{ \begin{center}  %fig3
 \vspace*{-1pt}
    \mbox{%
 \epsfxsize=79mm 
 \epsfbox{bet-5.eps}
 }

\end{center}

\noindent
{{\figurename~3}\ \ \small{
Иллюстрация художника Михаила Гладковского к~докладу А.\,П.~Ершова 
<<Программирование~--- вторая грамотность>>
}}}

\vspace*{9pt}




\noindent
 к~составлению программ путем 
псевдоматериального перемещения рукой пиктограмм на сенсорном экране 
планшета.
  
  Важно, чтобы по мере того, как основные по\-нятия программирования 
осваиваются детьми на\linebreak интуитивном уровне при работе с~реальными роботами 
на ковриках, кубиками на столе и~виртуальными роботами, ковриками 
и~программами на экранах планшетов, на бескомпьютерных половинах занятий 
происходил перевод этих интуитивных представлений в~вербальную форму. 
Под руководством воспитателя дети должны обсуждать значения слов 
\textit{программист}, \textit{робот}, \textit{программа}, значения фраз типа 
<<я~выполнил программу, которую составил Коля>>, <<программа составлена 
из 6~пиктограмм>>, <<робот выполнил 10~команд>>. Это пополнение 
словарного запаса детей и~развитие навыков монолога и~диалога 
с~использованием накопленного <<профессионального>> словарного запаса 
является столь же важной целью курса, как и~обретение навыков 
самостоятельного составления простейших программ в~учеб\-но-иг\-ро\-вой 
системе программирования.
  
\section{Принцип программного управления с~обратной связью}

  Б$\acute{\mbox{о}}$льшую часть курса <<Алгоритмика для дошколят>> 
занимает составление программ управления без обратной связи. Каждая такая 
программа решает одну задачу: обеспечивает нужное поведение робота  
в~од\-ной-един\-ст\-вен\-ной внешней обстановке, предъявляемой ребенку на 
полу в~игровой комнате или в~графической форме на экране планшета. 
Программы без обратной связи составляются с~использованием всего 
\textbf{двух явных} управляющих конструкций: числовой повторитель 
и~подпрограмма с~однобуквенным именем и~\textbf{одной неявной} 
конструкцией~--- последовательного выполнения линейного участка 
программы.
  
  В <<поминутной>> методичке годового курса <<Алгоритмика для 
дошколят>> описаны 30~занятий, еще 4~занятия предусмотрены как 
резервные. Последовательное выполнение линейного участка программы 
появляется на первом же занятии. Конструкция \textit{повторитель} впервые 
появляется на занятии №\,10 и~вводится как способ сокращения размера 
программы. Конструкция \textit{подпрограмма} впервые появляется на занятии 
№\,15 и~вводится как способ <<шифровки>> фрагментов программы. Позднее 
эта конструкция рассматривается еще и~как способ сокращения размера 
программы. Разумеется, выразительная сила двух выбранных управляющих 
конструкций невелика. Все программы, которые можно составить 
с~использованием этих конструкций, являются линейными. Однако эти 
линейные программы могут иметь достаточно сложную структуру управления. 
  
  Практика показала, что набора содержательных задач, решаемых 
с~использованием этих двух управ\-ля\-ющих конструкций, достаточно для 
удержания внимания детей в~течение года (первые 25~занятий из~30) при 
условии создания достаточного разнообразия роботов и~их графических 
представлений. В~настоящее время в~курсе <<Алгоритмика для дошколят>> 
используются 5~виртуальных роботов и~1~реальный робот (Ползун), 
имитирующий одного из виртуальных. Несмотря на весьма малую 
продолжительность компьютерной части каждого занятия курса~--- 
от~15~мин в~первом полугодии до~20~мин во втором~--- удается добиться 
того, что каждый ребенок на каждом занятии выполняет 4--5~заданий, т.\,е.\ 
\textbf{в~годовом курсе каждый дошкольник самостоятельно составляет 
120--150~простейших программ}. Самостоятельное выполнение более сотни 
упражнений представляется необходимым условием устойчивого освоения 
теоретического и~практического материала курса. 
  
  В конце первого года обучения, на последних занятиях, начинается (но не 
завершается) переход от управления без обратной связи к~управлению с~ее 
использованием:
  \begin{itemize}
  \item в~системы команд роботов вводятся ко\-ман\-ды-во\-про\-сы, 
и~в~предоставляемые системой <<ПиктоМир>> конструкции языка 
программирования добавляются ветвление и~повторение;
  \item в~систему основных понятий вводятся новый вид команды~--- 
\textbf{ко\-ман\-да-во\-прос} и~новый вид взаимодействия: робот 
\textbf{отвечает} на ко\-ман\-ду-во\-прос компьютера \textbf{да} или 
\textbf{нет}.
  \end{itemize}
  
  Параллельно с~введением понятия <<обратная связь>> в~систему основных 
понятий вводится и~понятие {\bfseries\textit{число}} (неотрицательное целое 
число). Для <<материализации>> понятия <<число>> вводится виртуальный 
исполнитель <<Волшебный кувшин с~камнями>>, играющий роль счетчика. 
Наличие счетчика позволяет с~помощью подсчета числа шагов решать задачи 
управления роботом типа <<дойти до ближайшей стены и~вернуться 
в~исходную точку>>. 
{\looseness=1

}
  
  Разумеется, введение новых понятий со\-про\-вож\-да\-ет\-ся играми. Исполняя роль 
робота, дети отвечают на ко\-ман\-ды-во\-про\-сы <<да>> или <<нет>>; 
исполняя роль кув\-ши\-на-счет\-чи\-ка, ребенок по команде до\-бав\-ля\-ет или 
удаляет камешек из реального кувшина и~отвечает на ко\-ман\-ды-во\-про\-сы 
<<кувшин пуст?>>, <<кувшин не пуст?>> и~<<сколько камней в~кувшине?>>. 
На введение обратной связи в~курсе <<Алгоритмика для дошколят>> требуется 
не менее~5~занятий. На устойчивое освоение этих понятий на следующем году 
обучения необходимо еще~15~занятий. 
{\looseness=1

}
  
   После введения обратной связи становится возможным давать задачи на 
составление <<универсальных>> программ, работающих не в~одной, 
а~в~нескольких разных обстановках. Эти задачи также даются в~графической 
форме, без словесного описания класса обстановок, в~которых должна работать 
программа. Просто на очередном уровне игры требуется составить программу, 
которая работает не в~одной, как раньше, а~в~двух или трех заданных 
обстановках. 

\vspace*{-10pt}
  
\section{Выводы}

\vspace*{-2pt}

  Опыт показал, что дети возраста 6--7~лет без труда и~с~энтузиазмом 
осваивают азы программирования с~использованием описанного выше подхода 
и~готовы продолжать занятия программированием в~школе. Авторы считают  
раннее освоение основ программирования в~описанном выше объеме
необходимым.

%\vspace*{-12pt}

{\small\frenchspacing
 {%\baselineskip=10.8pt
 \addcontentsline{toc}{section}{References}
 \begin{thebibliography}{9}
 
 %\vspace*{-4pt}
 
\bibitem{1-bet}
Глава профильного комитета Думы считает нужным ввести информатику в~дошкольную 
программу:  Мат-лы XVIII съезда <<Единой России>>~// ТАСС, 8~декабря 2018. {\sf 
https://tass.ru/obschestvo/5888487}.
\bibitem{2-bet}
\Au{Rogozhkina I., Kushnirenko~A.} PiktoMir: Teaching programming concepts to preschoolers with 
a~new tutorial environment~// Procd.  Soc. Behv., 2011. Vol.~28.  
P.~601--605. doi: 10.1016/j.sbspro.2011.11.114.
\bibitem{3-bet}
\Au{Бесшапошников Н.\,О., Кушниренко~А.\,Г., Леонов~А.\,Г., Малый~А.\,А.} Проект 
двуязыковой пик\-то\-грам\-мно-текс\-то\-вой учебной среды программирования ПиктоМир-К~// 
Свободное программное обеспечение в~высшей школе: Сб. тезисов XIV конф.~--- 
М.: МАКС Пресс, 2019. С.~64--66.
\bibitem{4-bet}
ПиктоМир: Стартовая страница проекта на сайте Федерального научного центра  
<<На\-уч\-но-ис\-сле\-до\-ва\-тель\-ский институт системных исследований>> Российской 
академии наук. {\sf https://www.niisi.ru/piktomir}.
\bibitem{5-bet}
\Au{Калаш И.} Возможности информационных и~коммуникационных технологий 
в~дошкольном образовании:\linebreak\vspace*{-12pt}

\columnbreak

\noindent
 Аналитический обзор~/ Пер. с~англ. под ред. 
А.\,Л.~Семенова.~--- Институт ЮНЕСКО по информационным технологиям 
в~дошкольном образовании, 2010. 176~с. {\sf 
https://iite.unesco.org/pics/publications/ru/files/\linebreak 3214673.pdf}.
(\Au{\mbox{Kaba{\!\!\ptb{\v{s}}}}, I.} 
Recognizing the potential of ICT in early childhood education: Analytical survey.~---
UNESCO Institute for Information Technologies in Education, 2010. 148~p. 
Available at: {\sf 
https://unesdoc.\linebreak unesco.org/ark:/48223/pf0000190433.pdf} 
(accessed September~2, 2020).)
\bibitem{6-bet}
\Au{Семёнов А.\,Л.} Концептуальные проблемы информатики, алгоритмики и~программирования 
в~школе~// Вестник кибернетики, 2016. №\,2(22). С.~12--16.
\bibitem{7-bet}
\Au{Выготский Л.\,С.} Мышление и~речь.~--- Изд. 5-е, испр.~--- М.: Лабиринт, 1999. Гл.~6. 
\bibitem{8-bet}
\Au{Пейперт С.} Переворот в~сознании: Дети, компьютеры и~плодотворные идеи~/ Пер. 
с~англ.~--- М.: Педагогика, 1989. 224~с.
(\Au{Papert~S.} Mindstorms: Children, computers and powerful ideas.~---  New York, NY, USA: Basic Books, 
1980. 252~p.)
\end{thebibliography}

 }
 }

\end{multicols}

\vspace*{-3pt}

\hfill{\small\textit{Поступила в~редакцию 20.08.19 (последняя правка 21.07.20)}}

\vspace*{10pt}

%\pagebreak

%\newpage

%\vspace*{-28pt}

\hrule

\vspace*{2pt}

\hrule

\vspace*{2pt}

\def\tit{BASIC CONCEPTS OF~PROGRAMMING EXPOUNDED FOR~PRESCHOOLERS}


\def\titkol{Basic concepts of~programming expounded for~preschoolers}

\def\aut{V.\,B.~Betelin$^1$, A.\,G.~Kushnirenko$^1$, and~A.\,G.~Leonov$^{1,2,3}$}

\def\autkol{V.\,B.~Betelin, A.\,G.~Kushnirenko, and~A.\,G.~Leonov}

\titel{\tit}{\aut}{\autkol}{\titkol}

\vspace*{-6pt}


\noindent
$^1$Federal Research Center ``Scientific Research Institute for System Analysis of the Russian Academy of 
Sciences,''\linebreak
$\hphantom{^1}$36-1~Nakhimovsky Prosp., Moscow 117218, Russian Federation

\noindent
$^2$M.\,V.~Lomonosov Moscow State University, 1~Leninskie Gory, GSP-1, Moscow 119991, Russian 
Federation

\noindent
$^3$Moscow Pedagogical State University, 1-1~Malaya Pirogovskaya Str., Moscow 119991, Russian 
Federation

\def\leftfootline{\small{\textbf{\thepage}
\hfill INFORMATIKA I EE PRIMENENIYA~--- INFORMATICS AND
APPLICATIONS\ \ \ 2020\ \ \ volume~14\ \ \ issue\ 3}
}%
 \def\rightfootline{\small{INFORMATIKA I EE PRIMENENIYA~---
INFORMATICS AND APPLICATIONS\ \ \ 2020\ \ \ volume~14\ \ \ issue\ 3
\hfill \textbf{\thepage}}}

\vspace*{6pt} 

\Abste{The development of information technology has formed 
a~socioeconomic demand for reducing the age of acquaintance 
of children with programming. As a~result of 6~years of efforts, 
the authors managed to develop and massively introduce an annual 
programming course for preschoolers built on the metaphor of program 
control. In the process of developing the course, the authors were 
able to select and formulate a~set of basic programming concepts 
which fully reveals this metaphor and, at the same time, can be 
mastered by preschool children age 6+ in an active-play form. 
This set of concepts is introduced using examples of control 
programs for moving and stationary objects with an intuitive, 
visible command system. At the beginning of the course, control 
without feedback is introduced, the concept of feedback is 
introduced and used only at the end of the course. As a~basic 
pedagogical software product, the PictoMir text-free pictographic 
system developed by the Federal Research Center 
``Scientific Research Institute for System Analysis
 of the Russian Academy of Sciences'' 
and its programmatic and methodological content is used, allowing 
each preschooler to gain experience in writing and debugging at 
least 120--150~simplest programs by the end of the course.}

\KWE{informatics; robot; program; computer; programming language; preschooler; 
PiktoMir; pictogram}

\DOI{10.14357/19922264200308} 

%\vspace*{-20pt}

\Ack
\noindent
The work was completed on the subject of the Government order 0065-2019-0010 
in 2020 in the Department 
of Educational Informatics, SRISA RAS.

%\vspace*{6pt}

 \begin{multicols}{2}

\renewcommand{\bibname}{\protect\rmfamily References}
%\renewcommand{\bibname}{\large\protect\rm References}

{\small\frenchspacing
 {%\baselineskip=10.8pt
 \addcontentsline{toc}{section}{References}
 \begin{thebibliography}{9}
\bibitem{1-bet-1}
Materialy TASS XVIII s''ezda ``Edinoy Rossii'' [TASS Materials of 
the 18th All-Russian political party 
``UNITED RUSSIA'' Congress]. Available at: {\sf https://tass.ru/
obschestvo/5888487} (accessed July~24, 
2020).
\bibitem{2-bet-1}
\Aue{Rogozhkina, I., and A.~Kushnirenko.} 2011. PiktoMir: Teaching programming concepts to 
preschoolers with a~new tutorial environment. 
\textit{Procd. Soc. Behv.}  28:601--605.  doi: 10.1016/j.sbspro.2011.11.114.
\bibitem{3-bet-1}
\Aue{Besshaposhnikov, N.\,O., A.\,G.~Kushnirenko,
 A.\,G.~Leonov, and A.\,A.~Malyy.} 2019. Proekt 
dvuyazykovoy piktogrammno-tekstovoy uchebnoy sredy programmirovaniya 
PiktoMir-K [The project of the 
bilingual pictogram-text educational environment for programming PictoMir-K]. 
\textit{Sbornik tezisov 
XIV konf. ``Svobodnoe 
programmnoe obespechenie v~vysshey shkole''} [14th Conference 
``Free Software in Higher Education'' Proceedings]. Moscow: MAKS Press. 64--66.
\bibitem{4-bet-1}
PiktoMir: Startovaya stranitsa proekta na sayte Fe\-de\-ral'\-no\-go 
nauchnogo tsentra  
``Nauchno-issledovatel'skiy\linebreak\vspace*{-12pt}

\columnbreak

\noindent
 institut sistemnykh issledovaniy'' Rossiyskoy akademii nauk [The start page of 
the PictoMir project on the website of the SRISA/NIISI RAS]. Available at: {\sf 
https://www.niisi.ru/piktomir/} (accessed July~24, 2020).
\bibitem{5-bet-1}
\Aue{\mbox{Kaba{\!\ptb{\v{s}}}}, I.} 2010.
Recognizing the potential of ICT in early childhood education: Analytical survey.
\mbox{UNESCO} Institute for Information Technologies in Education. 148~p. 
Available at: {\sf 
https://unesdoc.unesco.org/ark:/ 48223/pf0000190433.pdf} 
(accessed September~2, 2020).
\bibitem{6-bet-1}
\Aue{Semyonov, A.\,L.} 2016. Kontseptual'nye problemy informatiki, algoritmiki i~programmirovaniya 
v~shkole [Conceptual problems of teaching
 computer science, algorithm studies, and programming at school]. 
\textit{Vestnik kibernetiki} [Proceedings in Cybernetics] 2(22):11--15.
\bibitem{7-bet-1}
\Aue{Vygotskiy, L.\,S.} 1999. \textit{Myshlenie i~rech'}  
[Thinking and saying]. Moscow: Labirint. Ch.~6. 
\bibitem{8-bet-1}
\Aue{Papert, S.} 1980. \textit{Mindstorms: Children, computers and powerful 
ideas.} New York, NY: Basic 
Books. 252~p.
\end{thebibliography}

 }
 }

\end{multicols}

\vspace*{-6pt}

\hfill{\small\textit{Received August 20, 2019 (last revision July~21, 2020)}}

%\hfill{\small\textit{(last revision July~21, 2020)}}

%\pagebreak

%\vspace*{-24pt}

\Contr

\noindent
\textbf{Betelin Vladimir B.} (b.\ 1946)~--- 
Doctor of Science in physics and mathematics, professor, Academician of 
RAS, research advisor, Federal Research Center 
``Scientific Research Institute for System Analysis of the 
Russian Academy of Sciences,'' 36-1~Nakhimovsky Prosp., Moscow 117218, Russian Federation; 
\mbox{betelin@niisi.msk.ru}

\vspace*{3pt}

\noindent
\textbf{Kushnirenko Anatoliy G.} (b. 1944)~--- Candidate of Science (PhD) in 
physics and mathematics, Head of 
Department, Federal Research Center 
``Scientific Research Institute for System Analysis of the Russian 
Academy of Sciences,'' 36~b1~Nakhimovsky Prosp., Moscow 117218, Russian Federation; agk\_@mail.ru.

\vspace*{3pt}

\noindent
\textbf{Leonov Aleksandr G.} (b. 1961)~--- Candidate of Science (PhD) in physics and mathematics, associate 
professor, leading scientist, Department of Mechanics and Mathematics, 
M.\,V.~Lomonosov Moscow State 
University, 1~Leninskie Gory, GSP-1, Moscow 119991, Russian Federation; head of laboratory, Federal 
Research Center ``Scientific Research Institute for System Analysis of the Russian Academy of Sciences,''
 36-1~Nakhimovsky Prosp., Moscow 117218, Russian Federation; 
 professor, senior scientist, Moscow 
Pedagogical State University, 1-1~Malaya Pirogovskaya Str., Moscow 119991, Russian Federation; 
\mbox{dr.l@vip.niisi.ru}


\label{end\stat}

\renewcommand{\bibname}{\protect\rm Литература} 
    