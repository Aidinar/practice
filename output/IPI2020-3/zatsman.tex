\def\stat{zatsman}

\def\tit{ПРОБЛЕМНО-ОРИЕНТИРОВАННАЯ ВЕРИФИКАЦИЯ ПОЛНОТЫ 
ТЕМПОРАЛЬНЫХ ОНТОЛОГИЙ И~ЗАПОЛНЕНИЕ ПОНЯТИЙНЫХ 
ЛАКУН$^*$}

\def\titkol{Проблемно-ориентированная верификация полноты 
темпоральных онтологий и~заполнение понятийных 
лакун}

\def\aut{И.\,М.~Зацман$^1$}

\def\autkol{И.\,М.~Зацман}

\titel{\tit}{\aut}{\autkol}{\titkol}

\index{Зацман И.\,М.}
\index{Zatsman I.\,M.}
 

{\renewcommand{\thefootnote}{\fnsymbol{footnote}} \footnotetext[1]
{Исследование выполнено в~Институте проб\-лем информатики ФИЦ ИУ РАН при 
финансовой поддержке РФФИ в~рамках научного
проекта №\,18-07-00192.}


\renewcommand{\thefootnote}{\arabic{footnote}}
\footnotetext[1]{Институт проблем информатики Федерального исследовательского центра <<Информатика и~управление>> 
Российской академии наук,  \mbox{izatsman@yandex.ru}}

%\vspace*{-6pt}

   
    
   \Abst{Предлагается подход к~проверке полноты онтологии и~заполнению обнаруженных в~ней понятийных лакун с~применением следующих симбиотических информационных 
процессов: це\-ле\-на\-прав\-лен\-ное извлечение нового знания из данных, его представление 
в~онтологии и~применение для решения некоторой проблемы. В~процессе ее решения 
осуществляется верификация полноты онтологии, ре\-гист\-ри\-ру\-ют\-ся и~заполняются ее лакуны, 
выявленные при решении именно этой проблемы. Разделяются личностный, коллективный 
и~конвенциональный уровни представления знания в~онтологии. Этот подход позволяет 
обнаружить лакуны на конвенциональном уровне онтологии и~заполнить их на ее 
личностном и/или коллективном уровне, если для извлечения нового знания доступны его 
потенциальные источники. Цель статьи~--- рассмотреть модель перечисленных 
симбиотических процессов. Разработанная модель представляет собой обобщенную 
блок-схе\-му реализации предлагаемого подхода. Блок-схе\-ма служит основой компьютеризации 
симбиотических процессов. Описание модели иллюстрируется примером обнаружения 
и~заполнения понятийных лакун в~лингвистической типологии концептами нового знания, 
извлекаемого из текстовых данных.}
   
   \KW{трехуровневое представление знания; темпоральная онтология; понятийная лакуна; 
генерация нового знания; симбиотические информационные процессы}

\DOI{10.14357/19922264200317} 
 
%\vspace*{-6pt}


\vskip 10pt plus 9pt minus 6pt

\thispagestyle{headings}

\begin{multicols}{2}

\label{st\stat}
   
\section{Введение}
    
  Разработка принципов создания информационных систем, обеспечивающих 
симбиоз процессов извлечения из текстовых данных нового знания о~языке, его 
представления в~лингвистической типологии и~его применения, является одной 
из задач проекта по гранту РФФИ, который в~настоящее время выполняется 
в~Институте проблем информатики ФИЦ ИУ РАН. Отметим, что 
лингвистические типологии как частный случай онтологий принадлежат левой 
части их линейного спектра (см.\ рис.~2 в~[1]), где расположены глоссарии 
и~тезаурусы.
  
  В результате выполнения проекта был разработан подход к~проверке 
полноты лингвистической типологии и~применению симбиотических процессов 
для заполнения на ее личностном и~коллективном уровнях тех понятийных 
лакун, которые обнаружены на конвенциональном уровне. Для решения задач 
компьютерной лингвистики на основе этого подхода была предложена модель 
для описания проб\-лем\-но-ори\-ен\-ти\-ро\-ван\-ной верификации полноты 
типологии, обнаружения и~регистрации лакун, а также их заполнения 
концептами нового знания, извлекаемого из текстовых данных~[2, 3]. В~статье 
предлагаемая модель описывается в~обобщенной форме, без привязки 
к~лингвистическим типологиям и~задачам компьютерной лингвистики, что 
позволяет применять ее в~других предметных областях как инструмент  
проб\-лем\-но-ори\-ен\-ти\-ро\-ван\-ной верификации полноты онтологий 
и~заполне\-ния их понятийных лакун.
  
  Обобщенная модель позиционируется как развитие следующих трех 
моделей:
  \begin{enumerate}[(1)]
  \item  спиральная модель генерации нового знания, предложенная  
Нонака~[4--6];
  \item модель Веж\-биц\-ко\-го--На\-ка\-мо\-ри (далее~--- WN-мо\-дель)~[7--9];
  \item модель Ниссена~\cite[с.~36]{10-zac}.
  \end{enumerate}
  
  В спиральной модели генерируемое новое знание подразделяется на два вида 
и, соответственно, на два уровня представления (личностный и~коллективный). 
В~WN-мо\-де\-ли к~этим двум уровням добавлен третий уровень представления 
конвенционального знания. В~модели Ниссена коллективный уровень делится 
на два подуровня: новое знание группы и~организации. Предлагаемая в~\mbox{статье} 
модель дает возможность учесть степень социализации нового знания 
в~коллективе, изменение которой фиксируется с~единичным шагом. 
  
Кроме уровней представления знания в~этих моделях для описания процессов генерации нового знания явно 
или неявно используются, как правило, три разнородных измерения пространства\footnote{Отметим, что 
термин <<пространство>> здесь не является математическим. Авторами статей~[7--10] пространство 
трактуется как сочетание некоторого числа осей (которые могут быть и~номинативными шкалами), 
используемых для описания процессов генерации нового знания. Примеры таких сочетаний приводятся далее 
в~статье.}. При этом процессы генерации охватывают объекты двух сред: ментальной, включающей концепты 
знания человека, и~информационной, содержащей перцептивные 
формы их представления, например в~виде 
слов естественного языка (как сочетаний графических символов).
  
  Во всех трех перечисленных моделях в~явном виде не специфицированы 
источники нового знания, не эксплицирована цель его генерации и~не даны 
алгоритмы извлечения концептов нового знания. В~предлагаемой модели цель 
эксплицируется следующим образом: обнаружение понятийных лакун на 
конвенциональном уровне онтологии и~их заполнение на ее личностном 
и~коллективном уровнях концептами, извлеченными из специфицированного 
множества потенциальных источников нового знания. Поэтому для нее 
предлагается название~--- \textit{модель обнаружения и~заполнения лакун}  
(ОЗЛ-мо\-дель, англоязычное название~--- finding and filling model).
  
  Цель статьи~--- описать ОЗЛ-мо\-дель как обобщенную блок-схе\-му на 
основе трех моделей (спиральной, WN и~Ниссена), развивающую их по пяти 
позициям:
  \begin{enumerate}[(1)]
\item целенаправленность процессов обнаружения лакун в~онтологии и~их 
заполнения, ориентированных на решение явно сформулированной 
проблемы;
\item симбиоз следующих четырех процессов: 
(а)~проб\-лем\-но-ори\-ен\-ти\-ро\-ван\-ное применение онтологии; 
(б)~регистрация понятийных лакун, препятствующих решению проблемы, 
с~по\-мощью тегов неполноты; (в)~извлечение концептов нового знания для 
заполнения лакун; (г)~представление новых концептов руб\-ри\-ка\-ми 
\textit{понятий онтологии} на ее личностном и/или коллективном уровне;
\item специфицирование потенциальных источников концептов нового 
знания для обеспечения проб\-лем\-но-ори\-ен\-ти\-ро\-ван\-ной ве\-ри\-фи\-кации 
полноты онтологии, обнаружения\linebreak и~заполнения ее понятийных лакун;
\item расширение числа сред до трех, включая циф\-ро\-вую среду компьютеров 
и~сетей, ментальную и~информационную среды;
\item использование количественных осей для двух из трех измерений 
пространства для описания процессов обнаружения и~заполнения лакун, а 
также для определения числовых характеристик этих процессов и~их 
динамики.
  \end{enumerate}
  
  До начала выполнения четырех процессов, перечисленных выше в~п.~2, 
заранее не известно, где и~когда в~онтологии будут обнаружены лакуны, из 
каких именно источников будут извлечены концепты нового знания для их 
заполнения. По этой причине элементы специфицированного массива данных 
названы \textit{потенциальными источниками концептов нового знания}. 
В~проведенных экспериментах размеры предварительно специфицированного массива 
текстовых данных как минимум на порядок превышали те текстовые 
фрагменты массива, из которых потом действительно были  извлечены концепты нового знания. В~одном 
из экспериментов только~59 из 2503~аннотированных источников (2,36\%) 
включали теги неполноты и~из них были извлечены четыре концепта нового 
знания~[11]. При этом процесс их извлечения можно охарактеризовать, скорее 
всего, как \textit{вероятностно предсказуемый}.
  
\section{Исторический контекст создания модели обнаружения и~заполнения лакун}
  
  Прежде чем дать описание ОЗЛ-мо\-де\-ли, приведем краткий обзор 
спиральной модели, на основе которой были созданы WN-мо\-дель и~модель 
Ниссена. В~настоящее время спиральная модель генерации нового знания стала 
одной из самых популярных качественных моделей~\cite{4-zac, 6-zac}. В~ней 
определены две категории знания: индивидуальное и~коллективное. Каждая из 
этих категорий делится, в~свою очередь, на две подкатегории: выраженное, или 
явное (explicit), знание и~неявное (tacit) знание, т.\,е.\ она включает следующие 
четыре вида знания: выраженное и~неявное индивидуальное, выраженное 
и~неявное коллективное. Помимо этих видов знания в~модели определены 
следующие \textit{четыре вида} процессов трансформации знания и~форм его 
представления: социализация, экстернализация, комбинирование, 
интернализация, а~также понятие <<виток спирали>>.

  \begin{table*}[b]\small
  \vspace*{6pt}
  \begin{center}
  \begin{tabular}{|p{78mm}|p{78mm}|}
  \multicolumn{2}{c}{Два из 16\,268 потенциальных источников концептов нового знания}\\
  \multicolumn{2}{c}{\ }\\[-6pt]
  \hline
\multicolumn{1}{|c|}{Оригинальный текст}&\multicolumn{1}{c|}{Перевод}\\
\hline
  --- <<Gut>>, sagte ich, <<soll er dich verehren, aber soviel kostbare Blumen, das ist 
aufdringlich$\ldots$>>\newline
  \textit{Генрих Бёлль. Ansichten eines Clowns (1963)}&--- Очень мило,~--- сказал я,~--- 
поклонник поклонником, но дарить такой большой букет дорогих цветов~--- значит 
навязываться$\ldots$\newline
  \textit{Глазами клоуна. Пер.\  
Р.~Райт-Ковалева (1964)}\\
  \hline
  Soll Wilke sich irgendeinen Reim darauf machen! Wenn Werners Sarg de-swegen nicht fertig 
wird, so ist das kein Ungl$\ddot{\mbox{u}}$ck~--- der Bankier hat Dutzende von kleinen 
Hausbesitzern mit Inflationsgeld um ihr bisschen-Besitz gebracht.\newline
  \textit{Эрих Мария Ремарк. Der schwarze Obelisk (1956)}
  &Пусть Вильке вообразит себе ка\-кую-ни\-будь небылицу! Если гроб Вернера из-за всего 
этого не будет закончен~--- не беда: этот банкир, пользуясь инфляцией, лишил десятки 
мелких домовладельцев их жалкой собственности.\newline
  \textit{Черный обелиск. Пер.\ В.~Станевич (1961)}\\
  \hline
  \end{tabular}
  \end{center}
  \end{table*}
  
  
  По определению~\cite{4-zac, 6-zac}, каждый из витков включает 
последовательность: \textit{социализация} (формирование коллективного 
знания на основе  
ин\-ди\-ви\-ду\-аль\-но\-го)\;$\to$\;\textit{экс\-тер\-на\-ли\-за\-ция} (выражение 
коллективного знания перцептивными формами его\linebreak представления, например  
сло\-ва\-ми)\;$\to$\;\textit{ком\-би\-ни\-ро\-ва\-ние} (создание индивидуальных 
форм пред\-став\-ле\-ния знания на основе  
кол\-лек\-тив\-ных)\;$\to$\;\textit{ин\-тер\-на\-ли\-за\-ция} (личностное 
понимание, т.\,е.\ \mbox{трансформация} перцептивных форм в~индивидуальное  
зна\-ние)\;$\to$\;\textit{со\-циа\-ли\-за\-ция} (начало следующего витка спирали). 

Согласно Братиану~\cite{12-zac}, <<спиральная модель не содержит время 
в~качестве явной переменной$\ldots$ Модель содержит время неявно, так как 
любое преобразование требует времени, но это время \textit{без\linebreak ка\-кой-ли\-бо 
возможности его измерения} (курсив мой~--- ИЗ)>>. 

Таким образом, в~этой 
модели есть три измерения:
  \begin{enumerate}[(1)]
\item измерение \textit{социализации}, на шкале которого определены два 
номинативных значения (индивидуальное и~коллективное знание);
\item \textit{темпоральное} измерение (присутствует неявно);
\item \textit{эксплицитное} измерение, на шкале которого также 
определены два номинативных значения (неявное знание и~явное, т.\,е.\ 
выраженное).
\end{enumerate}

  Ниже будет показано, что три из четырех видов процессов трансформации 
знания и~форм его представления, а также эти три измерения, присутствующие 
также в~WN-мо\-де\-ли и~модели Ниссена, используются в~ОЗЛ-мо\-де\-ли 
в~явном виде и~при этом первые два измерения являются числовыми. Чис\-ло 
видов процессов трансформации в~ОЗЛ-мо\-де\-ли будет увеличено, так как она 
включает цифровую среду и~ее границы с~ментальной и~информационной 
средами.


  
\section{Модель обнаружения и~заполнения лакун}

\vspace*{-10pt}
    
  Сначала сформулируем критерий новизны знания в~рамках ОЗЛ-мо\-де\-ли. 
Концепт знания, извлеченный из множества потенциальных источников 
в~некоторый момент времени, считается новым, если он не был описан 
в~онтологии до этого момента времени. Предполагается, что эксперты до 
начала ими процесса верификации полноты онтологии уже отразили в~ней 
конвенциональное знание, относящееся к~ее предметной области.

\begin{figure*}[b] %fig1
\vspace*{1pt}
\begin{minipage}[t]{80mm}
 \begin{center}
 \mbox{%
 \epsfxsize=66.206mm 
 \epsfbox{zac-1.eps}
 }
 \end{center}
   \vspace*{-9pt}
\Caption{Четыре этапа первого примера обнаружения концептов}
\end{minipage}
%\end{figure*}
\hfill
%\begin{figure*} %fig2
\vspace*{1pt}
\begin{minipage}[t]{80mm}
 \begin{center}
 \mbox{%
 \epsfxsize=79mm 
 \epsfbox{zac-2.eps}
 }
 \end{center}
   \vspace*{-9pt}
\Caption{Семь этапов второго примера обнаружения концептов}
\end{minipage}
\end{figure*}
  
  Пять позиций, сформулированных в~первом разделе, по которым 
 ОЗЛ-мо\-дель развивает спиральную модель, WN-мо\-дель и~модель Ниссена, 
были получены в~процессе наблюдения за работой лингвистов, исследующих 
смысловое содержание немецких модальных глаголов, которые могут иметь 
более десяти значений. Цель лингвистов заключалась в~верификации полноты 
существующей типологии конвенциональных значений глаголов, обнаружении в~ней лакун и~их заполнении концептами нового знания, извлекаемого 
в~процессе лингвистического аннотирования фрагментов текстов 
с~модальными глаголами (см.\ таблицу)~\cite{13-zac, 14-zac, 15-zac, 16-zac}. Работу 
лингвистов можно описать с~помощью четырех симбиотических процессов:\\[-15pt]
  \begin{enumerate}[(1)]
\item применение рубрик типологии значений модальных глаголов при 
лингвистическом аннотировании~[17] фрагментов параллельных текс\-тов 
(конвенциональный уровень типологии был создан на основе словаря~[18]);\\[-17pt]
\item регистрация понятийной лакуны, обнаруженной на конвенциональном 
уровне, с~по\-мощью тега неполноты, если в~типологии нет руб\-ри\-ки, 
соответствующей значению глагола, найден\-но\-му в~процессе аннотирования;\\[-17pt]
\item извлечение из фрагмента параллельного текста и~его аннотации с~тегом 
неполноты кон\-цеп\-та нового знания о значении того модального глагола, 
которое встретилось во фрагменте, но отсутствует в~типологии;
\item включение в~типологию новых рубрик и~соответствующих им 
дефиниций, создаваемых на основе обнаруженных новых значений глаголов, 
что позволяет заполнить обнаруженные при аннотировании понятийные 
лакуны.
\end{enumerate}

  Для формирования множества потенциальных источников концептов нового 
знания лингвисты использовали параллельные не\-мец\-ко-рус\-ские текс\-ты, 
полученные из немецкого подкорпуса Национального корпуса русского 
языка~[19]. Эти текс\-ты включали 2,6~млн словоупотреблений (1,4~млн 
словоупотреблений в~оригинальных текстах на немецком языке и~1,2~млн 
словоупотреблений в~их переводах на русский язык). В~результате 
выравнивания оригиналов и~переводов было сформировано множество, которое 
состояло из~83\,190~потенциальных источников нового знания. Из 
них~16\,268~соответствовали цели исследования, т.\,е.\ содержали немецкие 
модальные глаголы (см.\ таблицу с~двумя примерами источников), в~том числе 
и~с~новыми их значениями, но число которых станет известным только после 
аннотирования всех 16\,268~источников~\cite{14-zac, 16-zac}.
  

  До описания ОЗЛ-мо\-де\-ли в~форме обобщенной блок-схе\-мы 
проиллюстрируем использование трех измерений (социализации, темпоральное и~эксплицитное) на двух частных примерах. В~первом примере (рис.~1) один 
лингвист анализирует источник и~обнаруживает в~нем концепт нового знания 
(обозначено как этап~$A_1$, который завершается в~момент времени~$t_1$), 
формирует для него дефиницию личностной рубрики в~типологии (этап~$A_2$ 
завершается в~момент времени~$t_2$), а затем согласовывает его с~($M-1$) 
лингвистами, участвующими кроме него в~пополнении типологии новыми 
рубриками (этап~$A_3$ завершается в~момент времени~$t_3$) 
и~затем~$M$~лингвистов совместно формируют для него дефиницию 
коллективной рубрики в~типологии (этап~$A_4$ завершается в~момент 
времени~$t_4$).
  


  Во втором примере (рис.~2) сначала один лингвист анализирует источник 
и~обнаруживает концепт нового знания (этап~$A_1$ завершается в~момент 
времени~$t_1$), формирует для него дефиницию личностной рубрики 
типологии (этап~$A_2$ завершается в~момент времени~$t_2$). Другой лингвист 
анализирует \textit{тот же источник}, интерпретирует его иначе, чем первый 
(этап~$A_3$ завершается в~момент времени~$t_3$), формирует дефиницию другой 
личностной рубрики типологии (этап~A$_4$ завершается в~момент 
времени~$t_4$). Два лингвиста по-раз\-но\-му интерпретируют один и~тот же 
источник, и~поэтому две новые рубрики будут разными. Тогда они совместно 
формируют согласованный между ними концепт (этапы~$A_{5.1}$ и~$A_{5.2}$ 
завершаются в~момент времени~$t_5$), а~затем создают для него дефиницию 
согласованной рубрики типологии (этап~$A_6$ завершается в~момент 
времени~$t_6$).
  


  На рис.~1 и~2 показаны три измерения ОЗЛ-мо\-де\-ли: два числовых 
(темпоральное и~степень социализации) и~одно номинативное 
(эксплицитность). В~измерении эксплицитности показаны две номинативные 
подкатегории знания, определенные в~спиральной модели: неявное и~явное 
(выраженное). На рис.~2 в~измерении социализации цифра~1 указывает на 
первого из двух лингвистов, а~циф\-ра~2 соответствует двум лингвистам, 
согласовавшим концепт, дефиницию и~соответствующую ему коллективную 
рубрику типологии. Формирование коллективных концептов любым числом 
лингвистов обсуждается ниже. Для примера с~двумя лингвистами рис.~2 
показывает первые~6~точек для моментов времени окончания~7~этапов 
обнаружения и~представления концептов нового знания ($t_1$--$t_6$).

\begin{figure*}[b] %fig3
\vspace*{6pt}
 \begin{center}
 \mbox{%
 \epsfxsize=163mm 
 \epsfbox{zac-3.eps}
 }
 \end{center}
   \vspace*{-9pt}
\Caption{Модель обнаружения и заполнения лакун
в~форме блок-схемы процессов обнаружения лакун и~представления 
в~онтологии концептов нового знания}
\end{figure*}

  
  Из приведенных двух примеров видно, что в~трех измерениях удается 
отобразить только последовательность этапов и~моменты времени, но не 
процессы трансформации, поэтому на рис.~1 и~2 не показаны вход и~выход 
каждого процесса. При этом удается проиллюстрировать только одно из пяти 
направлений развития (пятое). Для описания еще четырех направлений 
развития, приведем блок-схе\-му ОЗЛ-мо\-де\-ли в~общем виде без привязки 
к~лингвистическим задачам (рис.~3). Частный лингвистический вариант этой 
блок-схемы подробно описан в~работе~\cite{20-zac} и~реализован в~технологии 
работы с~базой данных немецких модальных глаголов, которая используется 
для верификации полноты типологии их значений и~ее  
пополнения~\cite{2-zac, 3-zac, 16-zac}.
{\looseness=1

}
  
  Сначала дадим краткое описание блок-схе\-мы верификации полноты 
онтологии и~заполнения ее\linebreak лакун при решении некоторой проблемы 
аналитиками (например, аннотирование значений и~контекста употреблений 
немецких модальных глаголов). Если в~процессе применения онтологии\linebreak 
аналитики обнаруживают ее неполноту, то они передают описание понятийной 
лакуны экспертам, которые отвечают за развитие онтологии. Эксперты 
анализируют описание лакуны и~пытаются заполнить ее на личностном уровне 
онтологии (левая часть рис.~3), извлекая концепты из потенциальных 
источников нового знания. Затем эксперты начинают обсуждать созданные ими 
личностные кон\-цеп\-ты и~пытаются сформировать концепты (понятия) 
и~дефиниции их рубрик для коллективного уровня онтологии (правая часть 
рис.~3). Блок-схе\-ма на этом рисунке включает~7~видов процессов 
трансформации, для каждого из которых указаны вход и~выход.
  
  Теперь рассмотрим подробнее этапы одной итерации, на которой может быть 
извлечен концепт 
 для заполнения понятийной лакуны (левая часть рис.~3):
  \begin{enumerate}[(1)]
\item поиск в~базе данных записи (которая может стать источником нового 
знания), необходимой для решения некоторой проблемы с~применением 
онтологии (например, поиск предложения, содержащего немецкий глагол 
\textit{sollen}, если целью обработки найденной записи ставится 
аннотирование его значения в~контексте этого предложения с~указанием 
рубрики этого значения, обозначающего понятие на конвенциональном 
уровне~--- этап \textit{поиска}, выполняемый аналитиком, обозначен 
стрелкой с~литерой~$R$;
\item преобразование найденной записи из цифровой формы 
в~перцептивную~---  этап обозначен словом <<\textit{Визуализация}>> над 
стрелкой с~литерой~$R$;
\item выявление аналитиком семантического свойства найденной записи, 
соответствующего цели ее обработки, и~его описание с~использованием 
\textit{рубрики} этого свойства, обозначающей \textit{понятие} (\textit{класс 
понятий}) на конвенциональном уровне онтологии (например, 
в~предложении анализируется такое его семантическое свойство, как 
значение глагола \textit{sollen} в~этом предложении; тогда ищется рубрика, 
которая соответствует именно этому значению глагола; если рубрика такого 
понятия в~онтологии найде\-на, то она включается в~описание записи)~--- этап 
<<\textit{Концептуализация}>> найденной записи;
\item если в~процессе решения проблемы аналитиком с~применением 
онтологии для некоторого семантического свойства найденной записи нет 
релевантной рубрики на конвенциональном уровне, то это может говорить 
о~лакуне (например, в~аннотируемом предложении для значения глагола 
\textit{sollen}, с~точки зрения аналитика, нет релевантной рубрики); тогда 
в~описании найденной записи аналитик проставляет специальную метку 
неполноты онтологии (икс-тег~--- x-tag), которая фиксирует потенциальную 
понятийную лакуну; найденная запись и~описание встретившегося в~ней 
семантического свойства передаются для их анализа одному из экспертов, 
отвечающих за ведение онтологии,~--- этап понимания экспертом 
семантического свойства найденной записи обозначен словом 
<<\textit{Интернализация}>>; результат этого процесса~--- 
\textit{индивидуальное понимание} этого свойства экспертом;


\item если аналитик отразил семантическое свойство найденной записи в~ее 
описании релевантной рубрикой конвенционального уровня (тогда 
в~описании записи отсутствует метка неполноты онтологии, т.\,е.\ нет  
икс-те\-га), то эта найденная запись не может служить источником нового знания; 
ее описание с~про\-став\-лен\-ной рубрикой преобразуется в~цифровую форму 
и~вводится в~базу данных описаний~--- этот этап обозначен словом 
(<<\textit{Оцифровка}>>);
\item если же в~описании записи аналитик проставил икс-тег для некоторого 
семантического свойства найденной записи, то это свойство эксперт 
сопоставляет с~концептами\linebreak рубрик онтологии, чтобы убедиться, что оно не 
пред\-став\-ле\-но рубрикой на конвенциональном уровне; если эксперт, 
в~отличие от аналитика, находит релевантную рубрику, то в~описании 
найденной записи икс-тег он заменяет на эту рубрику и~затем оно вводится 
в~базу данных описаний (этап <<\textit{Экстернализации}>> семантического 
свойства найденной записи рубрикой);
\item если эксперт не находит на конвенциональном уровне релевантную 
рубрику, то найденная\linebreak запись, с~его точки зрения, обладает семантическим 
свойством, которое не может быть представлено ни одной руб\-ри\-кой; для его 
пред\-став\-ле\-ния в~описании новой руб\-ри\-ки на личностном уровне эксперт 
анализирует это семантическое свойство, на основе его анализа создает 
дефиницию новой рубрики (<<\textit{Экстернализация}>>), добавляет ее 
в~свой личностный уровень онтологии (на рис.~3 добавление руб\-ри\-ки не 
показано);
\item в~описании найденной записи икс-тег заменяется на эту новую 
личностную руб\-ри\-ку, оно вводится в~базу данных описаний 
(<<\textit{Оцифровка}>>) и~осуществляется переход к~п.~1 (поиск 
следующей записи, релевантной решаемой проблеме, например другое 
предложение с~\textit{sollen}, и~так до их исчерпания).
    \end{enumerate}
    
    Описание восьми перечисленных этапов включает шесть процессов 
трансформации (см.\ рис.~3): \textit{интернализация}, \textit{экстернализация}, 
\textit{поиск}, \textit{визуализация}, \textit{оцифровка} 
и~\textit{концептуализация} (первые два есть в~витке спирали), две базы 
данных и~базу знаний с~темпоральной онтологией и~ретроспективой ее 
состояний.
    
    
    Если итерационно формировать описания найден\-ных записей, которые 
могут стать источниками нового знания, то в~онтологии могут появляться 
новые индивидуальные рубрики, созданные экспертами. В~общем случае для 
некоторых записей может быть создано по несколько разных описаний 
с~разными рубриками. Рассмотрим в~правой части рис.~3 этапы одной 
итерации, на которой~$M$~экспертов пытаются согласовать~$L$~описаний 
одного источника с~$L$~разными индивидуальными рубриками понятий 
онтологии ($1\hm\leq L\hm\leq M$, так как в~общем случае не каждый из~$M$ 
экспертов обязательно создает свою рубрику на личностном уровне):
    \begin{enumerate}[(1)]
\item для некоторого источника нового знания из базы данных описаний 
извлекаются~$L$~описаний, созданных на его основе, но с~разными новыми 
индивидуальными рубриками понятий онтологии (если $L\hm>1$; при 
$L\hm=1$ ищется одно описание, релевантное источнику, с~одной новой 
рубрикой), а также сам источник из базы данных источников нового знания 
(<<\textit{Поиск}>>);
\item выполняется преобразование источника и~$L$ его описаний с~новыми 
рубриками из цифровой формы в~перцептивную (<<\textit{Визуализация}>>);
\item эксперты проводят семантический анализ личностных дефиниций 
рубрик онтологии, проставленных в~$L$~описаниях источника 
(<<\textit{Интернализация}>>);
\item эксперты анализируют семантическое свойство источника нового 
знания, сопоставляют с~ним~$L$ его описаний (с~разными 
индивидуальными рубриками) и~пытаются согласовать дефиницию 
коллективной рубрики для семантического свойства этого источника 
(<<\textit{Социализация}>>);
\item если $K$ экспертов ($2\hm\leq K\hm\leq M$) смогут согласованно 
создать дефиницию новой коллективной рубрики 
(<<\textit{Экстернализация}>>), то они\linebreak вмес\-те составляют коллективное 
описание источника с~проставлением новой рубрики; в~противном случае 
выполнение этой итерации прекращается (<<\textit{Останов}>>);
\item коллективная рубрика и~ее дефиниция до\-бав\-ля\-ют\-ся на коллективный 
уровень онтологии \textit{именно этих~$K$~экспертов}, а новое описание 
источника с~этой коллективной рубрикой вводится в~базу данных описаний 
(<<\textit{Оцифровка}>>); затем осуществляется переход к~п.~1 
(извлекаются описания, созданные на основе другого источника нового 
знания).
\end{enumerate}
  
  В описании блок-схе\-мы указаны аналитики и~эксперты. В~операции 
согласования (<<\textit{Социализация}>>) принимают участие только 
эксперты. В~блок-схе\-му включено семь процессов трансформации: 
\textit{интернализация}, \textit{экстернализация}, \textit{социализация}, 
\textit{поиск}, \textit{визуализация}, \textit{оцифровка} 
и~\textit{концептуализация}. Первые три из них уже были в~моделях из разд.~2. 
Наличие еще четырех процессов трансформации позволяет говорить 
о~существенной степени новизны ОЗЛ-мо\-дели.

\section{Заключение}

%\vspace*{-6pt}

  В ОЗЛ-модели заполнение понятийных лакун на конвенциональном уровне 
онтологии обуслов\-ле\-но выполнением проб\-лем\-но-ори\-ен\-ти\-ро\-ван\-ной 
верификации ее полноты. В~результате формируются еще два уровня 
в~онтологии базы знаний. По завершении верификации появляется 
возможность выбирать уровень представления знания в~процессе решения 
проблемы, а также сопоставлять ее решения, полученные на конвенциональном 
уровне и~разными коллективами экспертов. 

Отметим, что применение  
ОЗЛ-мо\-де\-ли предполагает использование темпоральных онтологий, что дает 
возможность фиксировать и~отображать динамику извлечения нового знания.
  
  Выделим в~ОЗЛ-мо\-де\-ли те характерные черты,\linebreak которые обеспечивают 
первые четыре направления ее развития, перечисленные в~первом разделе\linebreak 
\mbox{статьи}. 
%
Целенаправленность обнаружения лакун\linebreak в~онтологии обеспечивается 
процессом поиска (первое направление развития трех моделей из разд.~2). 
%
В~ОЗЛ-мо\-де\-ли процессы применения и~пополнения онтологии являются 
\textit{симбиотическими} (второе направление): без применения онтологии 
нельзя обнаружить лакуны (при концептуализации потенциальных источников) и~заполнить их, а без регулярного пополнения онтологии новыми рубриками 
сфера ее применения всегда будет ограничена только уже существующим 
\textit{конвенциональным знанием}.
  
  В процессе поиска лакун используются потенциальные источники концептов 
нового знания, соответствующие проб\-лем\-но-ори\-ен\-ти\-ро\-ван\-ной цели 
применения онтологии (третье направление). 
%
Расширение числа сред до трех 
(четвертое направление) включает ментальную и~информационную среды, 
а~также цифровую среду компьютеров и~сетей, что дает возможность 
проводить поиск источников нового знания в~данных большого объема.
  
  В заключение отметим, что перспективным направлением применения  
ОЗЛ-мо\-де\-ли может стать проб\-лем\-но-ори\-ен\-ти\-ро\-ван\-ная 
верификация и~пополнение существующих онтологий систем искусственного 
интеллекта на личностном и~коллективном уровнях в~процессе 
целенаправленного поиска источников нового знания в~данных большого 
объема и~их концептуализации. В~настоящее время накопление таких данных 
наблюдается в~широком спектре предметных областей. Использование данных 
большого объема для верификации онтологий и~извлечения нового знания для  
их пополнения экспертами может быть реализовано на основе этой модели.
  
{\small\frenchspacing
 {%\baselineskip=10.8pt
 \addcontentsline{toc}{section}{References}
 \begin{thebibliography}{99}
\bibitem{1-zac}
\Au{McGuinness D.\,L.} Ontologies come of age~// Spinning the Semantic Web: Bringing the 
World Wide Web to its full potential~/ Eds. D.~Fensel, J.~Hendler, H.~Lieberman,  
W.~Wahlster.~--- Cambridge, MA, USA: MIT Press, 2003. P.~171--194.
\bibitem{2-zac}
\Au{Zatsman~I.} Goal-oriented creation of individual knowledge: Model and information 
technology~// 19th European Conference on Knowledge Management Proceedings.~--- Reading: 
Academic Publishing International Ltd., 2018. Vol.~2. P.~947--956.
\bibitem{3-zac}
\Au{Zatsman I.} Finding and filling lacunas in knowledge systems~// 20th European Conference 
on Knowledge Management Proceedings.~--- Reading: Academic Publishing International Ltd., 
2019. Vol.~2. P.~1143--1151.
\bibitem{4-zac}
\Au{Nonaka I.} The knowledge-creating company~// Harvard Bus. Rev., 1991. Vol.~69. 
No.\,6. P.~96--104.
\bibitem{5-zac}
\Au{Nonaka I.} A~dynamic theory of organizational knowledge creation~// Organ.
Sci.,  1994. Vol.~5. No.\,1. P.~14--37.
\bibitem{6-zac}
\Au{Нонака И., Такеучи~Х.} Компания~--- создатель знания~/
Пер. c~англ.~--- М.: Олимп-биз\-нес, 2003. 
384~с. (\Au{Nonaka~I., Takeuchi~H.} The knowledge-creating company.~--- Oxford, NY, USA: 
Oxford University Press, 1995. 284~p.)
\bibitem{7-zac}
\Au{Wierzbicki A.\,P., Nakamori~Y.} Basic dimensions of creative space~// Creative space: Models 
of creative processes for knowledge civilization age~/ Eds. A.\,P.~Wierzbicki, Y.~Nakamori.~--- 
Berlin: Springer Verlag, 2006. P.~59--90.
\bibitem{8-zac}
\Au{Wierzbicki A.\,P., Nakamori~Y.} Knowledge sciences: Some new developments~// 
Z.~Betriebswirt., 2007. Vol.~77. No.\,3. P.~271--295.
\bibitem{9-zac}
\Au{Nakamori Y.} Knowledge and systems science~--- enabling systemic knowledge synthesis.~--- 
London\,--\,New York: CRC Press, 2013. 234~p.
\bibitem{10-zac}
\Au{Nissen M.\,E.} Harnessing knowledge dynamics: Principled organizational knowing \& 
learning.~--- London: IRM Press, 2006. 278~p.
\bibitem{11-zac}
\Au{Zatsman~I., Buntman~N., Coldefy-Faucard~A., Nuriev~V.} WEB Knowledge Base for 
asynchronous brainstorming~// 17th European Conference on Knowledge Management 
Proceedings.~--- Reading: Academic Publishing International Ltd., 2016. Vol.~1. P.~976--983.
\bibitem{12-zac}
\Au{Bratianu C.} A~strategic view on the knowledge dynamics models used in knowledge 
management~// 20th European Conference on Knowledge Management Proceedings  
Proceedings.~--- Reading: Academic Publishing International Ltd., 2019. 
Vol.~1. P.~185--192.
\bibitem{13-zac}
\Au{Добровольский Д.\,О., Зализняк Анна~А.} Немецкие конструкции с~модальными 
глаголами и~их русские соответствия: проект надкорпусной базы данных~//\linebreak
 Компьютерная 
лингвистика и~интеллектуальные технологии: По мат-лам Междунар. конф. <<Диалог>>.~--- 
М.: РГГУ, 2018. Вып.~17(24). С.~172--184.
\bibitem{14-zac}
\Au{Зацман И.\,М.} Стадии целенаправленного извлечения знаний, имплицированных 
в~параллельных текстах~// Системы и~средства информатики, 2018. Т.~28. №\,3.  
С.~175--188.

\bibitem{16-zac} %15
\Au{Зацман И.\,М.} Целенаправленное развитие систем лингвистических знаний: выявление 
и~заполнение лакун~// Информатика и~её применения, 2019. Т.~13. Вып.~1. С.~91--98.

\bibitem{15-zac} %16
\Au{Гончаров А.\,А., Зацман~И.\,М.} Информационные трансформации параллельных текстов 
в~задачах извлечения знаний~// Системы и~средства информатики, 2019. Т.~29. №\,1.  
С.~180--193. 

\bibitem{17-zac}
Handbook of linguistic annotation~/ Eds. N.~Ide, J.~Pustejovsky.~--- Dordrecht, The Netherlands: 
Springer Science\;+\;Business Media, 2017. 1459~p.
\bibitem{18-zac}
Немецко-русский словарь: актуальная лексика~/ Под ред. Д.\,О.~Добровольского.~--- М.: 
Лексрус, 2020 (в~печати).
\bibitem{19-zac}
Параллельный немецкий подкорпус Национального корпуса русского языка. {\sf 
http://www.ruscorpora.ru/\linebreak search-para-de.html}.
\bibitem{20-zac}
\Au{Zatsman I.} Three-dimensional encoding of emerging meanings in AI-systems~// 21st 
European Conference on Knowledge Management Proceedings.~--- Reading: Academic 
Publishing International Ltd., 2020 (in press).
\end{thebibliography}

 }
 }

\end{multicols}

\vspace*{-9pt}

\hfill{\small\textit{Поступила в~редакцию 15.07.20}}

\vspace*{8pt}

%\pagebreak

%\newpage

%\vspace*{-28pt}

\hrule

\vspace*{2pt}

\hrule

%\vspace*{-2pt}

\def\tit{PROBLEM-ORIENTED VERIFYING THE~COMPLETENESS 
OF~TEMPORAL ONTOLOGIES AND~FILLING~CONCEPTUAL LACUNAS}

\def\titkol{Problem-oriented verifying the~completeness 
of~temporal ontologies and~filling conceptual lacunas}

\def\aut{I.\,M.~Zatsman}

\def\autkol{I.\,M.~Zatsman}

\titel{\tit}{\aut}{\autkol}{\titkol}

\vspace*{-15pt}


\noindent
Institute of Informatics Problems, Federal Research Center ``Computer Sciences and Control'' of 
the Russian Academy of Sciences; 44-2~Vavilov Str., Moscow 119133, Russian Federation


\def\leftfootline{\small{\textbf{\thepage}
\hfill INFORMATIKA I EE PRIMENENIYA~--- INFORMATICS AND
APPLICATIONS\ \ \ 2020\ \ \ volume~14\ \ \ issue\ 3}
}%
 \def\rightfootline{\small{INFORMATIKA I EE PRIMENENIYA~---
INFORMATICS AND APPLICATIONS\ \ \ 2020\ \ \ volume~14\ \ \ issue\ 3
\hfill \textbf{\thepage}}}

\vspace*{3pt} 
   
     
    
\Abste{An approach to verify the completeness of ontologies and fill 
conceptual lacunas found in them is proposed. The approach is 
based on the following symbiotic information processes: goal-oriented 
discovering new knowledge using data, its representation in an ontology, 
and application of the ontology to solve a~problem. In
the process of 
its solving, the completeness of the ontology is verified and its 
conceptual lacunas found during
solving the problem are registered 
and filled. The personal, collective, and conventional levels of 
knowledge\linebreak\vspace*{-12pt}}

\Abstend{representation in ontologies are discussed. The approach 
allows one to find conceptual lacunas at the conventional   level of 
ontologies and fill them at their personal and/or collective levels, 
if for discovering new knowledge its potential sources are available. 
The purpose of the paper is to consider the model of symbiotic 
information processes. The developed model is a~generalized flowchart 
that implements the proposed approach. The flowchart serves as the 
basis for computerization of symbiotic processes. The model description 
is illustrated by an example of finding conceptual lacunas in a~linguistic 
typology and filling them with the concepts of new knowledge discovered 
using text data.}

\KWE{three-level representation of knowledge; temporal ontology; 
conceptual lacuna; generation of new knowledge; symbiotic processes}
    
\DOI{10.14357/19922264200317} 

%\vspace*{-20pt}

 \Ack
    \noindent
     The reported study was funded by RFBR, project 
No.\,18-07-00192.

\vspace*{6pt}

 \begin{multicols}{2}

\renewcommand{\bibname}{\protect\rmfamily References}
%\renewcommand{\bibname}{\large\protect\rm References}

{\small\frenchspacing
 {%\baselineskip=10.8pt
 \addcontentsline{toc}{section}{References}
 \begin{thebibliography}{99}
    \bibitem{1-zac-1}
    \Aue{McGuinness, D.\,L.} 2003. Ontologies come of age. \textit{Spinning the 
Semantic Web: Bringing the World Wide Web to its full potential}. Eds. D.~Fensel, 
J.~Hendler, H.~Lieberman, and W.~Wahlster. Cambridge, MA: MIT Press.  
171--194.
    \bibitem{2-zac-1}
    \Aue{Zatsman, I.} 2018. Goal-oriented creation of individual knowledge: Model 
and information technology. \textit{19th European Conference on Knowledge 
Management Proceedings}. Reading: Academic Publishing International Ltd.  
2:947--956.
    \bibitem{3-zac-1}
    \Aue{Zatsman, I.} 2019. Finding and filling lacunas in knowledge systems. 
\textit{20th European Conference on Knowledge Management Proceedings}. 
Reading: Academic Publishing International Ltd. 2:1143--1151.
    \bibitem{4-zac-1}
    \Aue{Nonaka, I.} 1991. The knowledge-creating company. \textit{Harvard 
Bus. Rev.} 69(6):96--104.
    \bibitem{5-zac-1}
    \Aue{Nonaka, I.} 1994. A~dynamic theory of organizational knowledge 
creation. \textit{Organ. Sci.} 5(1):14--37.
    \bibitem{6-zac-1}
    \Aue{Nonaka, I., and H.~Takeuchi.} 1995. \textit{The knowledge-creating 
company}. Oxford, NY: Oxford University Press. 284~p.
    \bibitem{7-zac-1}
    \Aue{Wierzbicki, A.\,P., and Y.~Nakamori.} 2006. Basic dimensions of creative 
space. \textit{Creative space: Models of creative processes for knowledge civilization 
age.} Eds. A.\,P.~Wierzbicki and Y.~Nakamori. Berlin: Springer Verlag. 59--90.
    \bibitem{8-zac-1}
    \Aue{Wierzbicki, A.\,P., and Y.~Nakamori.} 2007. Knowledge sciences: Some 
new developments. \textit{Z.~Betriebswirt.} 77(3):271--295.
    \bibitem{9-zac-1}
    \Aue{Nakamori, Y.} 2013. \textit{Knowledge and systems science~--- enabling 
systemic knowledge synthesis}. London\,--\,New York: CRC Press. 234~p.
    \bibitem{10-zac-1}
    \Aue{Nissen, M.\,E.} 2006. \textit{Harnessing knowledge dynamics: Principled 
organizational knowing \& learning}. London: IRM Press. 278~p.
    \bibitem{11-zac-1}
    \Aue{Zatsman, I., N.~Buntman, A.~Coldefy-Faucard, and V.~Nuriev.} 2016. 
WEB Knowledge Base for asynchronous brainstorming. \textit{17th European 
Conference on Knowledge Management Proceedings}. Reading: Academic 
Publishing International Ltd. 1:976--983.
    \bibitem{12-zac-1}
    \Aue{Bratianu, C.} 2019. A~strategic view on the knowledge dynamics models 
used in knowledge management. 
    \textit{20th European Conference on Knowledge Management Proceedings}. 
Reading: Academic Publishing International Ltd. 1:185--192.
    \bibitem{13-zac-1}
    \Aue{Dobrovol'skiy, D.\,O., and Anna~A.~Zalizniak.} 2018. Nemetskie 
konstruktsii s~modal'nymi glagolami i~ikh russkie sootvetstviya: proyekt 
nadkorpusnoy bazy dannykh [German constructions with modal verbs and their 
Russian correlates: A~supracorpora database project]. \textit{Komp'yuternaya 
lingvistika i~intellektual'nye tekhnologii: po mat-lam Mezhdunar. konf. ``Dialog'}' 
[Computer Linguistic and Intellectual Technologies: Conference (International) 
``Dialog'' Proceedings]. Moscow. 17(24):172--184.
    \bibitem{14-zac-1}
    \Aue{Zatsman, I.} 2018. Stadii tselenapravlennogo izvlecheniya znaniy, 
implitsirovannykh v~parallel'nykh tekstakh [Stages of goal-oriented discovery of 
knowledge implied in parallel texts]. \textit{Sistemy i~Sredstva Informatiki~--- 
Systems and Means of Informatics} 28(3):175--188.
    
    \bibitem{16-zac-1} %15
    \Aue{Zatsman, I.\,M.} 2019. Tselenapravlennoe razvitie sistem 
lingvisticheskikh znaniy: vyyavlenie i~zapolnenie lakun [Goal-oriented 
development of linguistic knowledge systems: Identifying and filling of lacunae]. 
\textit{Informatika i~ee Primeneniya~--- Inform. Appl.} 13(1):91--98.

\bibitem{15-zac-1} %16
    \Aue{Goncharov, A.\,A., and I.\,M.~Zatsman.} 2019. In\-for\-ma\-tsi\-on\-nye 
transformatsii parallel'nykh tekstov v~zadachakh izvlecheniya znaniy [Information 
transformations of parallel texts in knowledge extraction]. \textit{Sistemy i~Sredstva 
Informatiki~--- Systems and Means of Informatics} 29(1):180--193.

    \bibitem{17-zac-1}
    Ide, N., and J.~Pustejovsky, eds. 2017. \textit{Handbook of linguistic 
annotation}. Dordrecht, The Netherlands: Springer Science\;+\;Business Media. 
1459~p.

%\pagebreak

    \bibitem{18-zac-1}
    Dobrovol'skiy, D.O., ed. 2020 (in press). \textit{Nemetsko-russkiy slovar': 
aktual'naya leksika} [German--Russian dictionary: Actual vocabulary]. Moscow: 
Leksrus. 
    \bibitem{19-zac-1}
    Parallel'nyy nemetskiy korpus [Parallel German corpus]. Available at: {\sf 
http://www.ruscorpora.ru/search-para-de.html} (accessed July~20, 2020).

\pagebreak

    \bibitem{20-zac-1}
    \Aue{Zatsman, I.} 2020 (in press). Three-dimensional encoding of emerging 
meanings in AI-systems. \textit{21st European Conference on Knowledge 
Management Proceedings}. Reading: Academic Publishing International Ltd.
{ %\looseness=1

}
\end{thebibliography}

 }
 }

\end{multicols}

\vspace*{-6pt}

\hfill{\small\textit{Received July 15, 2020}}

%\pagebreak

%\vspace*{-24pt}
    
    \Contrl
    
    \noindent
    \textbf{Zatsman Igor M.} (b.\ 1952)~--- Doctor of Science in technology, Head 
of Department, Institute of Informatics Problems, Federal Research Center 
``Computer Science and Control'' of the Russian Academy of Sciences, 44-2~Vavilov 
Str., Moscow 119333, Russian Federation; \mbox{izatsman@yandex.ru}
   
\label{end\stat}

\renewcommand{\bibname}{\protect\rm Литература} 
    
       