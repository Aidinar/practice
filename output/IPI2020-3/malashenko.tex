
%\def\de{\defeq}

%\def\fa{\forall}

%\def\se{\subseteq}
%\def\bs{\backslash}

%\def\ul{\underline}
%\def\wl{\widetilde}
%\def\ge{\geq}
%\def\vn{\varnothing}
%\def\sm{\setminus}
%\def\w{\widetilde}
%\def\v{\vert}
%\def\V{\Vert}
%\def\dm{d^{max}}
%\def\d{{\rm d}}
%\def\zm{z^{max}}

%\def\ups{\upsilon}
%\def\sig{\sigma}
%\def\vs{\varsigma}
%\def\del{\delta}
%\def\ta{\tau}
%\def\io{\iota}
%\def\vt{\vartheta}
%\def\al{\alpha}
%\def\ka{\kappa}
%\def\lm{\lambda}
%\def\vph{\varphi}
%\def\t{\theta}
%\def\i#1,#2,#3 {{#1}^{#2}_{#3}}
%\def\c{\cdot}
%\def\la{\langle}
%\def\ra{\rangle}
%\def\beq{\begin{equation}}
%\def\eeq{\end{equation}}
%\def\r(#1){({\ref{#1}})}
%%\def\L{{\rm I}\!{\rm L}}
%\def\st{\stackrel{\rightarrow}}
%\def\defeq{\stackrel{\mbox{\scriptsize {\rm def}}}=}
%\def\ba{{\bar a}}
%\def\bp{{\bar p}}
%\def\pr{{\bf pr}}

\def\stat{malashenko}

\def\tit{АППРОКСИМАЦИЯ МНОЖЕСТВА ДОСТИЖИМЫХ ПОТОКОВ МНОГОПОЛЬЗОВАТЕЛЬСКОЙ СЕТИ}

\def\titkol{Аппроксимация множества достижимых потоков многопользовательской сети}

\def\aut{Ю.\,Е.~Малашенко$^1$, И.\,А.~Назарова$^2$}

\def\autkol{Ю.\,Е.~Малашенко, И.\,А.~Назарова}

\titel{\tit}{\aut}{\autkol}{\titkol}

\index{Малашенко Ю.\,Е.}
\index{Назарова И.\,А.}
\index{Malashenko Yu.\,E.}
\index{Nazarova I.\,A.}
 

%{\renewcommand{\thefootnote}{\fnsymbol{footnote}} \footnotetext[1]
%{Работа выполнена при финансовой поддержке РФФИ (проект 19-07-00352);
%исследования проводились в~рамках программы Московского Цент\-ра 
%фундаментальной и~прикладной математики.}}


\renewcommand{\thefootnote}{\arabic{footnote}}
\footnotetext[1]{Федеральный исследовательский центр <<Информатика и~управление>> 
Российской академии наук, \mbox{malash09@ccas.ru}}
\footnotetext[2]{Федеральный исследовательский центр <<Информатика и~управление>> 
Российской академии наук, \mbox{irina-nazar@yandex.ru}}

%\vspace*{-6pt}

 

\Abst{Рассматривается метод приближенного описания  
выпуклого многогранного множества до\-пус\-ти\-мых потоков,  передаваемых 
между всеми узлами сети одновременно. Предлагается способ построения 
внутреннего выпуклого аппроксимирующего каркаса. Каркас формируется 
на основе векторов предельно достижимых потоков между парами вершин 
ис\-точ\-ник--при\-ем\-ник. Система  опорных векторов определяется точками, 
лежащими на внешних гранях исходного множества. Любая выпуклая комбинация  
базовых векторов задает допустимое распределение потоков. Разработанные 
алгоритмические схемы допускают  распараллеливание вычислительных 
процедур на гетерогенных многопроцессорных комплексах. Полученное 
агрегированное описание можно использовать при диспетчеризации   
интенсивных входных информационных потоков, превышающих функциональные 
возможности сети.}

\KW{многопродуктовая потоковая модель; множество допустимых потоков; 
внутренний опорный каркас}

\DOI{10.14357/19922264200312} 
 
%\vspace*{-6pt}


\vskip 10pt plus 9pt minus 6pt

\thispagestyle{headings}

\begin{multicols}{2}

\label{st\stat}

\section{Введение}

В работе рассматривается общая схема по\-строения 
приближенного описания множества\linebreak достижимых потоков~\cite{lot} 
в~многопользовательской сетевой системе~\cite{Yen, Dan}. В~рамках 
формальной математической записи системы ограничений ис\-следуются 
векторы различных потоков, которые могут одновременно передаваться 
между всеми\linebreak парами ис\-точ\-ник--при\-ем\-ник. Вычисляются граничные точки,
 которые соответствуют предельно достижимому значению потока для некоторой 
 пары узлов при фиксированных значениях для всех остальных. 
 Полученные наборы точек на грани\-це множества достижимости определяют
  опорные векторы      внутреннего каркаса. Любая вы\-пук\-лая комбинация 
  базисных векторов задает до\-пус\-ти\-мое распределение потоков в~исходной сети. 
  Предло\-жен\-ную алгоритмическую схему можно рас\-сматри\-вать как 
  многоэтапную динамическую процедуру диспетчеризации в~условиях 
  не\-опре\-де\-лен\-ности~\cite{JCM-19, Mal19-6}. 
  
  \vspace*{-6pt}

\section{Математическая модель}

  \vspace*{-2pt}

Для описания многопользовательской сетевой системы  используется 
математическая запись\linebreak модели передачи многопродуктового потока~\cite{Mal19-6}.
Структура сети~$G(\mathbf{d})$ задается множествами  
$\langle V,\ R, \ U\rangle$:
узлов (вершин) сети  $V \hm= \{ v_1, v_2, \ldots\linebreak
\ldots, v_n, \ldots, v_N \}$;
неориентированных ребер $R \hm= \{ r_1, r_2, \ldots, r_k, \ldots, r_E \} 
\hm\subset V \times V$.  Ребро $r_k = (v_{n_k}, v_{j_k})$ 
соединяет вершины~$v_{n_k}$, $v_{j_k}$ (инцидентно вершинам~$v_{n_k}$, 
$v_{j_k}$), которые для него служат концевыми. Предполагается, что 
в~сети отсутствуют петли и~сдвоенные ребра.
Ребру~$r_k$ ставятся в~соответствие две ориентированные дуги~$u_k$ и~$u_{k+E}$ 
прямого и~обратного направления из множества
ориентированных дуг  $U \hm= \{ u_1, u_2,\ldots, u_k, \ldots, u_{2E}\}$.
Дуги~$\{u_k, u_{k+E}\}$ определяют направление передачи потока по  
ребру~$r_k$ между концевыми вершинами~$v_{n_k}$ и~$v_{j_k}$.

Обозначим через~$S(v_n)$ множество номеров исходящих дуг, 
по которым поток покидает узел~$v_n$;
$T(v_n)$~--- множество номеров входящих дуг, по которым поток поступает 
в~узел~$v_n$.
Состав множеств $S(v_n)$ и~$T(v_n)$ однозначно 
определяется в~ходе выполнения следующей процедуры. Пусть некоторое ребро
 $r_k \hm\in R$ соединяет вершины с~номерами~$n$ и~$j$, такими что $n \hm< j$. 
 Тогда ориентированная дуга $u_k \hm= (v_n, v_j)$
считается \textit{исходящей} из вершины~$v_{n}$ и~ее номер~$k$ 
заносится в~множество~$S(v_n)$, ориентированная дуга $u_{k+E} 
\hm= (v_j, v_n)$~--- \textit{входящей} для~$v_{n}$ и~ее номер~$k\hm+E$ 
помещается в~список~$T(v_n)$.
Соответственно дуга~$u_k$ является \textit{входящей} для~$v_j$ и~ее номер~$k$ 
попадает в~$T(v_j)$, а~дуга~$u_{k+E}$~--- \textit{исходящей} и~номер~$k\hm+E$ 
вносится в~список исходящих дуг~$S(v_j)$.

Предполагается, что в~сети
между любой парой узлов могут  передаваться потоки разных видов.
Пара уз\-лов-кор\-рес\-пон\-ден\-тов~$p_m$ определяется упорядоченной 
парой вершин $p_m \hm= (v_{s_m}, v_{t_m})$,
где вершина с~номером~$s_m$ называется источником, а~с~номером~${t_m}$~--- 
стоком, или приемником, потока $m$-го вида.
В~настоящей работе в~сети из~$N$~узлов рассматривается $M \hm= 
N (N\hm-1)$ независимых, невзаимозаменяемых и~равноправных потоков 
различных видов, которые передаются между уз\-ла\-ми-кор\-рес\-пон\-ден\-та\-ми 
из множества $P \hm= \{ p_1, p_2, \ldots, p_M\}$.

Пусть $x_{mk}$~--- величина потока $m$-го вида, который передается
по ребру~$r_k$, $x_{mk}\hm \ge 0$, $m \hm= \overline{1, M}$, 
$k \hm= \overline {1, E}$. Каждому ребру $r_k \hm\in R$ 
приписывается неотрицательное число~$d_k$, определяющее 
суммарный предельно допустимый поток, который можно передать по ребру~$r_k$ 
в~обоих направлениях. В~исходной сети компоненты вектора пропускных 
способностей $\mathbf{d} \hm= (d_1, d_2, \ldots, d_k, \ldots, d_E)$~--- 
наперед заданные положительные числа $d_k \hm> 0$. Вектором~$\mathbf{d}$
 определяются следующие ограничения на все  потоки, передаваемые по ребру~$r_k$:
\begin{multline}
\sum\limits_{m=1}^{M} \left(x_{mk}+ x_{m(k+E)}\right) \le d_k,  
\\
 x_{mk} \ge 0, \ \ x_{m(k+E)} \ge 0, \ \  
k =\overline{1, E}. 
\label{e1-mal}
\end{multline}

Обозначим через~$z_m$ величину потока $m$-го вида, который 
поступает в~сеть в~узле с~номером~${s_m}$ и~покидает из узла с~номером~${t_m}$.

Во всех узлах сети $v_n \hm\in V$, $n \hm= \overline{1,N}$,  
для всех видов потоков должны выполняться условия сохранения потоков:
\begin{multline}
\sum\limits_{i \in S(v_n)}{x_{mi}} - \sum\limits_{i \in T(v_n)}{x_{mi}} ={}\\
{}=
\begin{cases}
 z_m, & \mbox{если } v_n = v_{s_m}; \\
- z_m, & \mbox{если } v_n = v_{t_m}; \\
 0 & \mbox{в\ остальных\ случаях,}
\end{cases}
\\
n = \overline {1, N}, \ m = \overline {1, M}, \ x_{mi} \ge 0, \ z_m \ge 0.
\label{e2-mal}
\end{multline}
Величина $z_m$ равна входному потоку $m$-го вида, который пропускается 
от источника к стоку пары~$p_m$ при распределении потоков~$x_{mi}$ 
по дугам сети.

Ограничения~(1) и~(2) задают множество  допустимых значений компонент 
векторов потоков
 $\mathbf{z} \hm= (z_1, z_2, \ldots, z_m, \ldots, z_M)$:
\begin{equation*}
\mathcal{Z}(x,\mathbf{ d}) = \{\mathbf{z} \ge 0 | \
(\mathbf{z}, x)\  \mbox{удовлетворяют } (1), (2)\}. 
%\label{e3-mal}
\end{equation*}

\section{Аппроксимация множества допустимых потоков}

Для приближенного описания множества допустимых потоков 
строится внутренний опорный каркас~\cite{Dan}. Каркас формируется 
на основе набора точек (векторов), лежащих на границе 
$\mathcal{Z}(x,\mathbf{ d})$. На начальном этапе рассматривается 
произвольная пара узлов корреспондентов $p_a \hm\in P$ 
и~определяется максимальный поток~\cite{Yen}, который можно 
независимо передать в~сети из вершины с~номером~$s_a$ в~вершину 
с~номером~${t_a}$ без учета всех остальных потоков.

\smallskip

\noindent
\textbf{Задача~1.} Найти
$$ 
z_a^0 (1) = \max\limits_{(\mathbf{z}, x)  \in \mathcal{Z}(x, 
\mathbf{d})} z_a 
$$
при дополнительных условиях 
$$
z_m = 0,\ m \not = a,\ m = \overline {1, M}\,.$$

Вектор решений $\mathbf{z}_a^0(1) = (0, 0, \ldots, z_a^0 (1), \ldots, 0)$ 
определяет угловую точку выпуклого множества
$\mathcal{Z}(x,\mathbf{ d})$, которая лежит на пересечении грани 
$\mathcal{Z}(x,\mathbf{ d})$ с~осью координат, соответствующей паре~$p_a$. 
Последовательное решение задачи~1 для всех $p_m \hm\in P$ позволяет 
построить множество $\mathbf{Z}^0(1) \hm= \{\mathbf{z}_1^0(1),
 \mathbf{z}_2^0(1), \ldots, \mathbf{z}_m^0(1), \ldots, 
 \mathbf{z}_M^0(1) \}$ опорных векторов $C(1)$-се\-че\-ния 1-го слоя.

Векторы из множества $\mathbf{Z}^0(1)$ можно также рас\-смат\-ри\-вать как 
направляющие векторы конуса возмож\-ных направлений к внешней 
границе множества~$\mathcal{Z}(x,\mathbf{ d})$.
Компоненты вектора $\mathbf{z}(0) \hm= \left(
{z}_1^0(1), {z}_2^0(1), \ldots, {z}_m^0(1), \ldots, 
{z}_M^0(1)\right)$\linebreak численно равны предельно достижимым потокам для каждой 
пары $p_m \hm\in P$. Рассмотрим  луч $\mathbf{z}(\beta)
\hm = \beta \mathbf{z}(0)$, на\-прав\-лен\-ный из начала координат 
в~точку~$\mathbf{z}(0)$. Длину шага по лучу определим как коэффициент 
разложения в~выпуклой комбинации опорных векторов из~$\mathbf{z}^0(1)$.

\smallskip

\noindent
\textbf{Задача~2.} Найти~$\beta^*$ и~коэффициенты~$\gamma_m^*$,  
$m \hm= \overline {1, M}$,
$$
\beta^* \mathbf{z}(0) = \sum\limits_{m=1}^M \gamma_m^* {z}_m^0(1) = 
\sum\limits_{m=1}^M \gamma_m^* \mathbf{z}^0(1) 
$$
при условии 
$$
\sum\limits_{m=1}^M \gamma_m^* = 1, \  \gamma_m^* \ge 0, \ 
 m = \overline {1, M}\,. 
 $$

Из покомпонентной записи
$$
\beta^* {z}_m^0(1) = \gamma_m^* {z}_m^0(1), \ 
 \sum\limits_{m=1}^M \gamma_m^* = 1, \  
 \gamma_m^* \ge 0, \  m = \overline {1, M},
 $$
следует, что $\beta^* \hm= \gamma_m^*  \hm= 1/M$. 
Координаты точки~$\mathbf{z}(\beta^*)$ 
в~множестве $ \mathcal{Z}(x,\mathbf{ d})$ обозначим
$$
z_m^*(1) = \fr{1}{M}\, z_m^0(1),\enskip  m = \overline {1, M}\,. 
$$

Вектор $\mathbf{z}^*(1) \hm= \left(z_1^*(1), z_1^*(1), \ldots, z_m^*(1), 
z_M^*(1)\right)$ определяет распределение потоков при одновременной 
передаче потока между всеми парами узлов ис\-точ\-ник--при\-ем\-ник. 
Каждая компонента вектора~$\mathbf{z}^*(1)$  пропорциональна максимальному 
потоку для соответствующей пары $p_m \hm\in P$.

На следующем этапе в~множестве $\mathcal{Z}(x, \mathbf{d})$ 
осуществляется поиск вектора потоков, каждая компонента которого 
больше или равна достигнутым значениям~${z}_m^*(1)$.

\smallskip

\noindent
\textbf{Задача~3.} Для некоторой пары узлов~$p_a$ определить 
максимальный поток
$$ 
z_a^0 (2) = \max\limits_{(\mathbf{z}, x)  \in \mathcal{Z}(x, \mathbf{d})}  z_a  
$$
при дополнительных условиях 
$$ z_a \ge z_a^*(1)\,;\enskip z_m \ge z_m^*(1)\  \forall p_m \in P,  \ 
m\not = a. $$
Решение задачи~3~--- вектор 
$$
\mathbf{z}_a^0 (2) = \left(z_1^*(1), z_2^*(1), \ldots, z_a^0 (2), 
\ldots, z_M^*(1)\right), 
$$
который лежит на границе множества $\mathcal{Z}(x, \mathbf{d})$. 
На основе решения цепочки задач~3 последовательно для всех $p_m \hm\in P$ 
формируется множество
$$
\mathbf{Z}^0(2)= \left\{\mathbf{z}_1^0(2), \mathbf{z}_2^0(2), 
\ldots, \mathbf{z}_m^0(2), \ldots, \mathbf{z}_M^0(2)\right\}
$$ 
опорных векторов $C(2)$-се\-че\-ния 2-го слоя или направляющих 
векторов конуса возможных\linebreak направлений из точки 
с~координатами~$\left(z_1^*(1)\right.$,\linebreak  $\left.z_2^*(1), \ldots, z_m^*(2), \ldots, z_M^*(1)\right)$.

Любой вектор потоков~$\mathbf{z}$, который можно представить 
как выпуклую комбинацию векторов из~$\mathbf{Z}^0(2)$, 
является допустимым и~принадлежит множеству $\mathcal{Z}(x, \mathbf{d})$.
Век\-тор-луч~$\Delta \mathbf{z}(2)$ с~компонентами 
$$
\Delta z_m(2) = 
z_m^0 (2) - z_m^*(1)\,,\enskip m = \overline {1, M}\,, 
$$
определяет возможное направление увеличения достигнутых величин потоков. 
При этом коэффициент~$\alpha$~--- длина шага в~возможном направлении 
при переходе из~$\mathbf{z}^*(1)$ в~точку~$\mathbf{z}^*(2)$~--- к~новому 
распределению потоков:
$$
 \mathbf{z}^*(2) = \mathbf{z}^*(1) + \alpha \Delta \mathbf{z}(2)\,.
 $$


\noindent
\textbf{Задача~4.} Найти~$\alpha^*$ и~коэффициенты выпуклой 
комбинации~$\gamma_m^*$,  $m \hm= \overline {1, M}$, такие что
\begin{multline*}
\mathbf{z}^*(2) = \mathbf{z}^*(1) + \alpha^* \Delta \mathbf{z}(2) = 
\sum\limits_{m=1}^M \gamma_m^* \mathbf{z}_m^0(2)\,, 
\\
  \sum\limits_{m=1}^M \gamma_m^* = 1, \enskip  \gamma_m^* \ge 0,  \enskip 
  m = \overline {1, M}\,,
\end{multline*}
при покомпонентной записи:
\begin{multline*} 
{z}_m^*(2) = z_m^*(1) + \alpha^*\left( z_m^0(2)  -  z_m^*(1)\right) = {}\\
{}=
\gamma_m^* z_m^0(2) + \sum\limits_{j=1}^M \gamma_j^* z_m^*(1)  - 
\gamma_m^* {z}_m^*(1) = {}\\
{}  = \gamma_m^* z_m^0(2) + 
\left(1 - \gamma_m^*\right) {z}_m^*(1);
\end{multline*}
$$  
\sum\limits_{m=1}^M \gamma_m^* = 1, \  \gamma_m^* \ge 0, \ 
 m = \overline {1, M}\,.
 $$

Решение задачи~4~--- координаты разложения $\alpha^* \hm= \gamma_m^* 
\hm= ({1}/{M})$, $ m \hm= \overline {1, M}$,~--- 
определяет вектор допустимых потоков~$\mathbf{z}^*(2)$ с~координатами:
\begin{multline*}
z_m^*(2) = \fr{M - 1}{M}\, z_m^*(1) + \fr{1}{M}\, z_m^0(2) = {}\\
{}=
\left(\fr{1}{M} - \fr{1}{M^2}\right) z_m^0(1) + \fr{1}{M} \,z_m^0(2) = {}\\
{} = \fr{1}{M} \left[\fr{M - 1}{M} \,z_m^0(1) + z_m^0(2) \right],  \ 
m = \overline {1, M}.
\end{multline*}
После сравнения найденных~$\mathbf{z}^*(1)$ и~$\mathbf{z}^*(2)$ 
формируется множество пар корреспондентов для поиска вектора~$\mathbf{z}^*(3)$.

Пусть задано малое положительное число $\delta\hm > 0$. Обозначим  $P^-(2)$~--- 
множество пар, для которых на всех последующих этапах  фиксируется 
достигнутый поток~$z_m^*(2)$,
$$ 
P^-(2) = \left\{ p_m \ | \ p_m \in P, \ |z_m^*(2) - z_m^*(1)| \le \delta \right\}, 
$$
$M^-(2)$ --- множество индексов пар, входящих в~$P^-(2)$,
$M^-(2)\hm = \{ m \ | \ p_m \hm\in P^-(2)\}$, $\mu^-(2)$~--- 
число элементов в~$P^-(2)$ или~$M^-(2)$. При этом
$P^+(2)$~--- множество пар, для которых на сле\-ду\-ющем, треть\-ем, этапе 
определяется возможное направление увеличения потоков,
$$ 
P^+(2) = \left\{ p_m \ | \ p_m \in P, \ |z_m^*(2) - z_m^*(1)| > \delta \right\}, 
$$
$M^+(2)$~--- множество индексов пар, входящих в~$P^+(2)$,
$\mu^+(2) \hm= \{ m \ | \ p_m \in P^+(2)\}$, $\mu^+(2)$~--- 
число элементов в~$P^+(2)$ или~$M^+(2)$.

\smallskip

\noindent
\textbf{Задача~5.} Для некоторой пары узлов-кор\-рес\-пон\-ден\-тов $p_a \hm\in P^+(2)$ 
определить максимальный поток
$$ 
z_a^0 (3) = \max\limits_{(\mathbf{z}, x)  \in \mathcal{Z}(x, \mathbf{d})} 
 z_a  
 $$
при дополнительном условии $z_a \hm\ge z_a^*(2)$
и~сохранении значений всех остальных потоков $z_m \hm\ge z_m^*(2)$ 
для всех пар $p_m \hm\in P$,  $m \not= a$.

Вектор-решение задачи~5 
$$
\mathbf{z}_a^0 (3) = \left(z_1^*(2), z_2^*(2), \ldots, z_a^0 (3), \ldots, z_M^*(2)\right)
$$ 
находится на границе множества $\mathcal{Z}(x, \mathbf{d})$. 
Решение цепочки задач~5 для каждого $p_m \hm\in P^+(2)$ позволяет 
построить множество опорных векторов
$$
\mathbf{Z}^0(3) = \left\{\mathbf{z}_1^0(3), \mathbf{z}_2^0(3), 
\ldots, \mathbf{z}_m^0(3), \ldots, \mathbf{z}_{M^+(2)}^0(3)\right\}
$$ 
 $C(3)$-се\-че\-ния 3-го слоя.

Компоненты вектора $\mathbf{z}^*(3)$ определяются по следующему правилу:
\begin{multline*}
\!\!\!z_j^*(3) ={}\\
\!{}=\!
\begin{cases}
\!
z_j^*(2) + \displaystyle \fr{1}{\mu^+(2)}\left(z_j^0(3) - z_j^*(2)\right), & 
\!\!\!\mbox{если } j  \in M^+(2); \\
z_j^*(2), & \!\!\!\mbox{если } j  \in M^-(2).
\end{cases}\hspace*{-9.9pt}
\end{multline*}

Предположим, что на некотором этапе~$T$ найден вектор~$\mathbf{z}^*(T)$ 
и~при попытке формирования множества~$P^+(T)$ оказалось, что 
$P^+(T) \hm= \emptyset$, поскольку для всех $p_m \hm\in P$ выполняется 
неравенство:
$$ \left\vert z_m^0(T) - z_m^*(T) \right\vert \le \delta,
\ m = \overline {1, M}\,. 
$$

Вектор $\mathbf{z}^*(T)$ лежит  \textit{вблизи} 
границы множества $\mathcal{Z}(x, \mathbf{d})$ и~определяет 
\textit{приближенное} предельное  распределение потоков, 
при котором нельзя увеличить ни одно из достигнутых значений~$z_m^*(T)$, 
не уменьшив при этом поток хотя бы для одной пары уз\-лов-кор\-рес\-пон\-ден\-тов.

На основе вектора~$\mathbf{z}^*(T)$ и~всех векторов 
из~$\mathbf{Z}^0(t)$, $t \hm= \overline{1, T}$, формируется опорный 
внутренний  каркас:
$$ 
\mathrm{SiF}\left (\cdot\right) =\left\{\textbf{z}^*(T), \mathbf{Z}^0(1), \mathbf{Z}^0(2), 
\ldots, \mathbf{Z}^0(T)\right\}
$$
(от \textit{англ}. Support-internal-Frame~--- опорный внут\-рен\-ний каркас). 
$\mathrm{SiF}\,(\cdot)$-кар\-кас можно рас\-смат\-ривать как приближенное описание, 
или ап\-про\-к\-симацию, множества $\mathcal{Z}(x, \mathbf{d})$. 
Любая выпуклая комбинация базисных векторов $\mathrm{SiF}\,(\cdot)$-кар\-ка\-са 
задает допустимое распределение потоков, которые могут одновременно 
передаваться между всеми парами уз\-лов-кор\-рес\-пон\-ден\-тов. 

\section{Заключение}

Разработанную схему вычислений можно использовать как  составную 
часть многоэтапной  динамической  процедуры диспетчеризации в~условиях 
отсутствия данных о~состоянии  передающих\linebreak
 подсистем сети  и~при  
интенсивных  входных потоках~\cite{Mal19-6}.  Вектор потоков~$\mathbf{z}^*(1)$, 
вычисленный на первом шаге, можно трактовать как априорное  допустимое  
распределение,  полученное  на основе  \mbox{нормативных} показателей 
работоспособности сис\-те\-мы. На втором шаге решается цепочка задач 
поиска максимального увеличения потока  независимо для каждой пары 
корреспондентов, но при фиксированных достигнутых  значениях для 
всех остальных. Каждая   найденная точка  лежит на границе множества 
достижимости и~определяет   предельно до\-пус\-ти\-мые потоки.  Любой набор 
до\-пус\-ти\-мых   потоков приближенно  представим выпуклой комбинацией  
векторов предельно допустимых потоков.

Значение коэффициента разложения меньше  единицы будет означать, 
что пропускные возможности системы не позволяют одновременно передать 
все запрашиваемые  потоки.  В~результате,  согласно диспетчерским правилам, 
будут введены  ограничения на величины передаваемых потоков, 
а~через несколько   итераций  вновь будут определены новые возможные   
распределения.  Пред\-ла\-га\-емый метод позволяет в~процессе принятия решений  
использовать как нормативную, так и~текущую информацию о~состоянии сети.  

{\small\frenchspacing
 {%\baselineskip=10.8pt
 \addcontentsline{toc}{section}{References}
 \begin{thebibliography}{9}
    
\bibitem{lot}  
\Au{Лотов А.\,В., Поспелова И.\,И.}  
Многокритериальные задачи принятия решений.~--- М.: Макс Пресс, 2008. 197~с.



\bibitem{Dan} \Au{Данциг Дж.} Линейное программирование, 
его применения и~обобщения~/ Пер. с~англ.~--- М.: Прогресс, 1966. 589~c. 
(Dantzig~G.  Linear programming and extensions.~--- 
Princeton, NJ, USA: Princeton University Press, 1963. 600~p.)


\bibitem{Yen} \Au{Йенсен П., Барнес~Д.} 
Потоковое программирование~/ Пер. с~англ.~--- М.: Радио и~связь, 1984. 392~с. 
(Jensen~P.\,A., Barnes~J.\,W. Network flow programming.~--- 
New York, NY, USA: Wiley, 1980. 408~p.)


\bibitem{JCM-19} \Au{Малашенко~Ю.\,Е., Назарова~И.\,А., Новикова~Н.\,М.} 
Один подход к анализу возможных  структурных  повреждений  в~многопродуктовых  
сетевых системах~// Ж.~вычисл. мат. мат. физ., 2019. Т.~59.  №\,9. С.~1626--1638.

\bibitem{Mal19-6} \Au{Малашенко~Ю.\,Е., Назарова~И.\,А., Новикова~Н.\,М.} 
Экс\-пресс-ана\-лиз и~агрегированное представление множества достижимых 
потоков многопродуктовой сетевой системы~// Изв. РАН. ТиСУ, 2019. №\,6. 
С.~63--72.
\end{thebibliography}

 }
 }

\end{multicols}

\vspace*{-6pt}

\hfill{\small\textit{Поступила в~редакцию 23.06.20}}

\vspace*{8pt}

%\pagebreak

\newpage

\vspace*{-28pt}

%\hrule

%\vspace*{2pt}

%\hrule

%\vspace*{-2pt}

\def\tit{APPROXIMATION OF~THE~MULTIUSER NETWORK FEASIBLE FLOWS SET}

\def\titkol{Approximation of~the~multiuser network feasible flows set}

\def\aut{Yu.\,E.~Malashenko and I.\,A.~Nazarova}

\def\autkol{Yu.\,E.~Malashenko and I.\,A.~Nazarova}

\titel{\tit}{\aut}{\autkol}{\titkol}

\vspace*{-9pt}


\noindent
Federal Research Center ``Computer Science and Control'' 
of the Russian Academy of Sciences, 40~Vavilov Str., 
Moscow 119333, Russian Federation

\def\leftfootline{\small{\textbf{\thepage}
\hfill INFORMATIKA I EE PRIMENENIYA~--- INFORMATICS AND
APPLICATIONS\ \ \ 2020\ \ \ volume~14\ \ \ issue\ 3}
}%
 \def\rightfootline{\small{INFORMATIKA I EE PRIMENENIYA~---
INFORMATICS AND APPLICATIONS\ \ \ 2020\ \ \ volume~14\ \ \ issue\ 3
\hfill \textbf{\thepage}}}

\vspace*{3pt} 




\Abste{A~method for approximate description of 
a~convex polyhedral set of feasible flows transmitted between all 
network nodes simultaneously is considered. A~method for constructing 
an internal convex approximating frame is proposed. The frame is 
formed based on the vectors of maximum feasible flows between pairs 
of source--receiver vertices. The system of support vectors is 
determined by the points lying on the outer edges of the baseline set. 
Any convex combination of base vectors sets the feasible flow 
distribution. The developed algorithmic schemes allow parallelization 
of computational procedures on heterogeneous multiprocessor complexes. 
The resulting aggregated description can be used for dispatching 
intensive input information flows that exceed the network's capability.}

\KWE{multicommodity flow model; feasible flows set; internal support frame}

\DOI{10.14357/19922264200312} 

%\vspace*{-20pt}

%\Ack
%\noindent


%\vspace*{6pt}

 \begin{multicols}{2}

\renewcommand{\bibname}{\protect\rmfamily References}
%\renewcommand{\bibname}{\large\protect\rm References}

{\small\frenchspacing
 {%\baselineskip=10.8pt
 \addcontentsline{toc}{section}{References}
 \begin{thebibliography}{9}
\bibitem{1-mal-1}
\Aue{Lotov, A.\,V., and I.\,I.~Pospelova.} 
2008. \textit{Mnogokriterial'nye zadachi prinyatiya resheniy} 
[Multicriteria decision making tasks]. Moscow: Maks Press. 197~p.

\bibitem{3-mal-1}
\Aue{Dantzig, G.} 1963. 
\textit{Linear programming and extensions}. 
Princeton, NJ: Princeton University Press. 600~p.

\bibitem{2-mal-1}
\Aue{Jensen, P.\,A., and J.\,W.~Barnes.}
 1980. \textit{Network flow programming}. New York, NY: Wiley. 408~p.
 
\bibitem{4-mal-1}
\Aue{Malashenko, Yu.\,E., I.\,A.~Nazarova, and N.\,M.~Novikova.}
 2019. An approach to the analysis of possible structural damages 
 in multicommodity network systems. 
 \textit{Comp. Math. Math. Phys.} 59(9):1562--1574. 
\bibitem{5-mal-1}
\Aue{Malashenko, Yu.\,E., I.\,A.~Nazarova, and N.\,M.~Novikova.}
 2019. Express analysis and aggregated representation of the set of 
 reachable flows for a multicommodity network system. 
 \textit{J.~Comput. Sys. Sc. Int.} 58(6):889--897.  
 \end{thebibliography}

 }
 }

\end{multicols}

\vspace*{-6pt}

\hfill{\small\textit{Received June 23, 2020}}

%\pagebreak

%\vspace*{-24pt}


\Contr

\noindent
\textbf{Malashenko Yuri E.} (b.\ 1946)~--- 
Doctor of Science in physics and mathematics, principal scientist, 
Federal Research Center ``Computer Science and Control'' 
of the Russian Academy of Sciences, 40~Vavilov Str., Moscow 119333, Russian Federation; 
\mbox{malash09@ccas.ru} 

\vspace*{3pt}

\noindent
\textbf{Nazarova Irina A.} (b.\ 1966)~--- 
Candidate of Science (PhD) in physics and mathematics, scientist, 
Federal Research Center ``Computer Science and Control''
 of the Russian Academy of Sciences, 40~Vavilov Str., Moscow 119333, Russian Federation; 
 \mbox{irina-nazar@yandex.ru}


\label{end\stat}

\renewcommand{\bibname}{\protect\rm Литература} 