
\def\stat{sopin}

\def\tit{АНАЛИЗ МЕХАНИЗМОВ НАРЕЗКИ СЕТИ С УЧЕТОМ ГАРАНТИЙ ДЛЯ РАЗЛИЧНЫХ 
ТИПОВ ТРАФИКА$^*$}

\def\titkol{Анализ механизмов нарезки сети с~учетом гарантий для различных 
типов трафика}

\def\aut{К.\,А.~Агеев$^1$, Э.\,С.~Сопин$^2$, Н.\,В.~Яркина$^3$, 
К.\,Е.~Самуйлов$^4$, С.\,Я.~Шоргин$^5$}

\def\autkol{К.\,А.~Агеев, Э.\,С.~Сопин, Н.\,В.~Яркина и др.} 
%К.\,Е.~Самуйлов$^4$, С.\,Я.~Шоргин$^5$ и~др.}

\titel{\tit}{\aut}{\autkol}{\titkol}

\index{Агеев К.\,А.}
\index{Сопин Э.\,С.}
\index{Яркина Н.\,В.} 
\index{Самуйлов К.\,Е.}
\index{Шоргин С.\,Я.}
\index{Ageev K.\,A.}
\index{Sopin E.\,S.}
\index{Yarkina N.\,V.}
\index{Samouylov K.\,Е.}
\index{Shorgin S.\,Ya.}
 

{\renewcommand{\thefootnote}{\fnsymbol{footnote}} \footnotetext[1]
{Исследование выполнено при поддержке Программы РУДН <<5-100>> и~при частичной финансовой 
поддержке РФФИ в~рамках научных проектов №\,19-07-00933 и~№\,19-37-90147.}}


\renewcommand{\thefootnote}{\arabic{footnote}}
\footnotetext[1]{Российский университет дружбы народов, ageev-ka@rudn.ru}
\footnotetext[2]{Российский университет дружбы народов; Институт проблем информатики Федерального исследовательского 
центра <<Информатика и~управ\-ле\-ние>> Российской академии наук, \mbox{sopin-es@rudn.ru}}
\footnotetext[3]{Российский университет дружбы народов, yarkina-nv@rudn.ru}
\footnotetext[4]{Российский университет дружбы народов; Институт проб\-лем информатики Федерального 
исследовательского центра <<Информатика и~управ\-ле\-ние>> 
Российской академии наук,  
\mbox{samouylov-ke@rudn.university}}
\footnotetext[5]{Институт проблем информатики Федерального исследовательского центра <<Информатика  
и~управ\-ле\-ние>> Российской академии наук, \mbox{sshorgin@ipiran.ru}}

\vspace*{-5pt}

 
  
  \Abst{Нарезка радиоресурсов сети (network slicing)~--- это одна из ключевых 
возможностей современных сетей, позволяющая нескольким виртуальным мобильным 
операторам использовать ресурсы одной базовой станции. Это дает возможность 
операторам, владельцам ресурсов, предоставлять в~аренду и~управлять несколькими 
выделенными логическими сетями с~определенной функциональностью, реализуемой поверх 
общей инфраструктуры. Каждая из этих логических сетей называется слайсом сети и~может 
быть адаптирована для обеспечения определенного поведения системы, чтобы наилучшим 
образом поддерживать определенные показатели качества услуг. В~работе построена модель 
механизма нарезки радиоресурсов, распределяющего ресурс по слайсам, и~проведен анализ 
этой модели методом имитационного моделирования.}
  
  
  \KW{имитационное моделирование; система массового обслуживания; ограниченные 
ресурсы; нарезка сети}

\DOI{10.14357/19922264200314} 
 
%\vspace*{-6pt}


\vskip 10pt plus 9pt minus 6pt

\thispagestyle{headings}

\begin{multicols}{2}

\label{st\stat}
  
\section{Введение}

\vspace*{-2pt}

  Нарезка радиоресурсов сети (англ.\ \textit{network slicing}) дает возможность 
оператору мобильной связи предоставлять выделенные логические сети 
в~аренду виртуальным сетевым операторам в~виде сетевых слайсов 
с~функциями, специфичными для клиента. Слайс сети, который охватывает все 
сегменты сетевой инфраструктуры, может быть выделен для конкретных видов 
услуг нескольким виртуальным операторам, предоставляющим схожие услуги, 
либо отдельно для каждого виртуального оператора~[1,~2].
  
  Для каждого слайса сети выделяются ресурсы (например, 
виртуализированные сетевые функции, пропускная способность сети и~др.), 
и~ошибки или неисправность, возникающие в~одном слайсе, не влияют на 
обеспечение показателей качества обслуживания QoS (Quality of Service) 
в~других слайсах; иными словами, гарантируется изоляция слайса для 
обеспечения гарантированного качества обслуживания. При этом алгоритм 
нарезки радиоресурсов должен обеспечивать эффективное использование 
ресурсов соты с~учетом гарантированного объема ресурсов, выделенного для 
каждого слайса~[3].
  
  Данная тематика в~последнее время привлекает повышенное внимание 
исследователей. В~\cite{4-sop} представлена гибкая модель нарезки сети 
радиодоступа (Radio Access Network, RAN). Основные цели заключаются 
в~определении уровня изоляции производительности между операторами 
виртуальных сетей (Virtual Network Operator, VNO), которые выступают 
в~качестве арендаторов сети, с~тем чтобы гарантировать, что их соглашения об 
уровне обслуживания (Service Level Agreement, SLA) не будут затронуты 
изменением различных параметров сети, и~в~то же время оптимизировать 
использование инфраструктуры RAN путем динамического распределения 
радиоресурсов между различными сегментами справедливым образом. 
  
  В~\cite{5-sop} также рассматривается система управления виртуальными 
радиоресурсами (Virtual radio resource management, VRRM), которая 
обеспечивает оптимальное использование виртуализированных ресурсов 
поставщика инфраструктуры между несколькими операторами виртуальной 
сети. В~статье представлена архитектура инструмента моделирования VRRM 
в~терминах систем массового обслуживания (СМО). С~по\-мощью разработанного 
инструмента проводится анализ практического сценария с~тремя поставщиками 
и~различными типами SLA и~исследуются показатели производительности при 
изменении нагрузки на трафик и~SLA.
  
  В~работах~\cite{6-sop, 7-sop} рассматривается теоретическая основа для 
многооператорного планирования (Multi-Operator Scheduling, MOS). Благодаря 
динамической адаптации к~каналу и~нагрузке централизованный подход 
максимизирует спектральную эффективность для нескольких операторов 
с~полным контролем над гарантиями совместного использования. 
  
  В данной работе рассматривается сценарий функционирования одной соты 
беспроводной сети связи, в~которой активировано~$S$~слайсов, модуль 
нарезки делит между ними~$C$~единиц радиоресурсов. Каждый слайс 
предоставляет пользователям услугу связи, предполагающую непрерывную 
передачу данных с~определенным выделенным ресурсом, скоростью передачи, 
не менее $a_s\hm\geq 0$ и~не более $b_s\hm\geq a_s$, $s\hm\in S$. 
При этом скорость передачи является переменной: в~каждый момент времени 
она пересчитывается и~зависит от числа активных сессий в~каждом слайсе. 
Предполагается, что ресурс в~рамках одного слайса распределяется поровну 
между пользователями. В~работе описана модель в~виде 
СМО, предложен алгоритм разделения радиоресурсов, описана 
работа средства имитационного моделирования, проведен численный 
эксперимент и~анализ полученных результатов.

\section{Математическая модель}

  Пусть в~многолинейную СМО
поступает~$S$~потоков заявок, соответствующих запросам на передачу 
данных от пользователей~$S$~различных слайсов. Потоки являются 
пуассоновскими с~интенсивностями~$\lambda_s$, $s\hm\in S$. Объемы  
заявок~--- независимые случайные величины, распределенные по 
экспоненциальному закону с~параметрами~$1/\mu_s$, $s\hm\in S$, а~скорость 
обслуживания заявки определяется объемом выделенного ей ресурса. 
  
  Пусть общий объем ресурсов СМО для обслуживания заявок равен~$C$. 
Количество ресурсов, выделяемых заявке, принятой на обслуживание, зависит 
от состояния системы и~может варьироваться в~диапазоне $[a_s, b_s]$, $s\hm\in 
S$. При этом после каж\-до\-го поступления либо ухода заявки происходит 
перераспределение ресурсов между слайсами.
  
  Определим случайный процесс $X(t)\hm= \{ m_1(t), m_2(t), \ldots , m_S(t)\}$, 
где~$m_s$, $s\hm\in S$,~--- число заявок в~слайсе в~момент времени~$t$, 
причем 
  $$
  m_s\in \left\{ 0,1,\ldots , \left\lceil \fr{C}{a_s}\right\rceil \right\}\,,\enskip s\in S\,.
  $$
  %
  Тогда пространство возможных состояний процесса имеет вид:
  $$
  \mathrm{X}=\left\{ \left( m_1, m_2, \ldots , m_s\right)\,,\ \sum\limits^S_{s=1} 
m_s a_s\leq C\right\}\,.
  $$
  
  Обозначим через~$r_s$, $s\hm\in S$, количество выделенного ресурса одной 
заявке в~слайсе~$s$, $s\hm\in S$. Тогда интенсивности обслуживания заявок 
соответствующих слайсов определятся как~$r_s\mu_s$, $s\hm\in S$.
  
  Для обеспечения изоляции слайсов обозначим через~$\overline{R}_s$ объем 
ресурсов, который гарантированно выделен слайсу~$s$, $s\hm\in S$, а~через 
$\overline{M}_s\hm= \overline{R}_s/a_s$~--- число заявок, которое 
гарантированно может быть принято в~слайсе~$s$, $s\hm\in S$. Слайсы, число 
заявок в~которых превышает гарантированное значение, будем называть 
нарушителями. В~случае нехватки ресурсов и~наличия  
слай\-сов-на\-ру\-ши\-те\-лей поступившая заявка другого слайса может 
вытеснить одну или несколько заявок слай\-сов-на\-ру\-ши\-те\-лей. В~случае 
нехватки ресурсов и~отсутствия нарушителей, а~также в~случае когда все 
слайсы нарушают, поступающая заявка будет сброшена.
  
  Рассмотрим подробнее возможные события при поступлении сессий 
в~состоянии $(m_1, m_2, \ldots , m_s)\hm\in \mathrm{X}$. Пусть в~систему 
поступает заявка слайса~$s$, $s\hm\in S$. Тогда возможны следующие случаи:
  \begin{itemize}
  \item $(m_1, m_2, \ldots , m_s+1, \ldots , m_S)\hm\in \mathrm{X}$, т.\,е.\ 
предо\-став\-ле\-ние минимального количества ресурса для поступающей заявки 
возможно; в~этом случае заявка встает на обслуживание, а~случайный процесс 
переходит в~состояние  $(m_1, m_2, \ldots , m_s+1,\ldots , m_S)$;
  \item  $(m_1, m_2, \ldots , m_s+1, \ldots , m_S)\not= \mathrm{X}$, т.\,е.\ не 
гарантируется предоставление минимально тре\-бу\-емо\-го количества ресурса, 
тогда:
\begin{itemize}
  \item  если $m_s\hm< \overline{M}_s$ и~$m_k\hm> \overline{M}_k$, $k\hm\in S$, $k\not= 
s$, то выполняется освобождение ресурсов слай\-са-на\-ру\-ши\-те\-ля. 
Алгоритм сброса заявок приведен в~разд.~3;
  \item  в~остальных случаях поступающая заявка будет сброшена.
  \end{itemize}
  \end{itemize}
  
  Таким образом, множество состояний сброса заявок при поступлении:
  \begin{multline*}
  D_s=\left\{\vphantom{\left(\overline{M}_k\right)}\left (m_1, m_2, \ldots , m_S\right): {}\right.\\
{}:  \left( \left(m_1, m_2, \ldots , m_s+1, \ldots , m_S\right)\not= 
\mathrm{X}\right)\cap{}\\ 
\left.{}\cap\left( \left( m_s\geq \overline{M}_s\right) \cup \left( m_k\leq 
\overline{M}_k,\ k\in S,\ k\not=s\right)\right)
  \right\}\,.
\end{multline*}
  
  Множество состояний прерывания обслуживания:
\begin{multline*}
  B_k=\left\{ \vphantom{\left(\overline{M}_k\right)}
  \left( m_1, m_2, \ldots , m_S\right) :{}\right.\\
  {}: \left( \left( m_1, m_2, \ldots , m_s+1, \ldots , 
m_S\right) \not=\mathrm{X}\right) \cap {}\\
\left.{}\cap\left( \left( m_s<\overline{M}_s\right) \cup 
\left( m_k>\overline{M}_k,\ k\in S\,,\ k\not= s\right)\right)\right\}\,.
\end{multline*}

\section{Алгоритм выбора сбрасываемой заявки}

  Для выбора сбрасываемой заявки при наличии нескольких  
слай\-сов-на\-ру\-ши\-те\-лей введем понятие веса слайса, показывающего, 
насколько сильно слайс нарушает границы других слайсов. Ниже приведены 
три возможные формулы для вычисления весов: 
\begin{align*}
 w_s^{(1)}&=\begin{cases}
  1\,, & m_s\leq \overline{M}_s\,;\\
  \fr{1}{m_s-\overline{M}_s+1}\,, & m_s>\overline{M}_s\,;
  \end{cases}\\
  w_s^{(2)}&=\begin{cases}
  1\,, & m_s\leq \overline{M}_s\,;\\
  \fr{\overline{M}_s}{m_s}\,, & m_s>\overline{M}_s\,;
  \end{cases}\\
  w_s^{(3)}&=\begin{cases}
  1\,, & m_s\leq \overline{M}_s\,;\\
  w=const<1, & m_s>\overline{M}_s\,.
  \end{cases} %\label{e3-sop}
\end{align*}
 
  Освобождение ресурсов требуется в~случае, если заявка поступает в~слайс, 
который не является нарушителем, при этом она не может быть принята 
в~сис\-те\-му из-за нехватки свободного ресурса: 
\begin{multline*}
\left\{ \left( \left( m_1, m_2, \ldots , m_s+1, \ldots , m_S\right) \notin 
\mathrm{X}\right) :{}\right.\\
\left.{}: \left( m_s+1\right)a_s\leq R_s\right\}\,.
\end{multline*}

 \textbf{Шаг 1:} ищем слайс с~наименьшим весом, т.\,е.\ $s^*\hm\in S$: $w_{s^*}\hm= 
\min \{ w_r, r\hm\in S\}$.
  
  \textbf{Шаг 2:} сбрасываем заявку найденного слайса $(m_1, m_2, \ldots , m_{s^*}-1, 
\ldots , m_S)$.
  
  Шаги 1 и~2 выполняются, пока не будет удовле\-тво\-ре\-но условие $C\hm- 
\sum\nolimits^S_{i=1} m_i a_i\hm\geq a_s$. Отметим, что в~случае, когда 
у~нескольких слайсов вес минимален, выбор слайса для сброса заявки 
происходит случайным образом.
  
  
  \section{Распределение ресурсов}
  
  В состоянии избытка ресурсов 
  $$
  \mathrm{X}_0= \left(\! \left(m_1, m_2, \ldots , 
m_s+1, \ldots, m_S\right): \ \!\sum\limits^S_{s=1}\! m_s b_s \hm\leq C\!\right)
\hspace*{-0.94673pt}
$$ 
всем  заявкам во всех слайсах будет выделен максимальный объем ресурса~$b_s$.
  
  В состоянии ограниченных ресурсов $\mathrm{X}_1\hm= 
\mathrm{X}\backslash \mathrm{X}_0$ для справедливого и~эффективного 
распределения ресурсов решается задача оптимизации. Пусть скорости 
передачи данных в~слайсах имеют функцию полезности $U_s(r_s)\hm= \ln \left( 
r_s\right)$. Тогда задача выглядит следующим образом:
  \begin{gather}
  \sum\limits_{s\in S} w_s(m_s) m_s U_s(r_s)\to \max\,,\enskip s\in S\,;\notag\\
  \sum\limits_{s\in S} m_s r_s=C\,;\label{e5-sop}\\
  P=\left\{ \mathbf{r}\in \mathbf{R}^S:\ a_s\leq r_s\leq b_s\,,\ s\in S\right\}\,,
  \label{e6-sop}
  \end{gather}
где вектор~$\mathbf{r}$ имеет вид $\mathbf{r}\hm= (r_1, r_2, \ldots , r_S)$. 
Решение задачи выполняется с~по\-мощью при\-бли\-жен\-но\-го метода 
проецирования градиента, в~котором итеративная процедура поиска максимума 
описывается соотношениями:
\begin{gather*}
\mathbf{d}_k=\mathbf{P}\nabla f(\mathbf{x}_k)\,;\\
\mathbf{x}_{k+1} =\mathbf{x}_k+\tau_k \mathbf{d}_k\,;\\
\tau_k>0:\ \mathbf{x}_{k+1}\in P\,,
\end{gather*}
где вектор~$\mathbf{x}$ является решением, а~$\mathbf{P}$~--- матрица 
проецирования на гиперплоскость~(\ref{e5-sop}):
$$
\mathbf{P}=\mathbf{I}-\mathbf{m}\left(\mathbf{m}
\mathbf{m}^{\mathrm{T}}\right)^{-1}\mathbf{m} = \mathbf{I} -\fr{1}{\sum\nolimits_{s\in S} 
m_s^2}\,\mathbf{m}^{\mathrm{T}}\mathbf{m}\,.
$$
    Здесь вектор $\mathbf{m}\hm= (m_1, m_2, \ldots , m_S)$~--- текущее 
состояние системы. Градиент функции полезности представляет собой  
век\-тор-стол\-бец:
  $$
  \nabla f(\mathbf{x}_k)=\left( \fr{w_s m_s}{r_s}\right)_{s\in S}\,.
  $$
  
  Длину шага~$\tau_k$ выбираем таким образом, чтобы не выйти за пределы 
области~$P$, задаваемой прямыми ограничениями задачи~(\ref{e6-sop}). 
В~качестве начального приближения~$\mathbf{x}_0$ удобно взять точку 
пересечения диагонали координатного параллелограмма~$P$, соединяющей 
точки $(a_1, a_2, \ldots , a_S)$ и~$(b_1, b_2, \ldots , b_S)$ 
с~гиперплоскостью~(\ref{e5-sop}). Данная точка находится при решении 
системы линейных уравнений:
  \begin{equation*}
  \left.
  \begin{array}{c}
  (b_S-a_S)x_1-(b_1-a_1)x_S=a_1b_S-a_Sb_1\,;\\[6pt]
  (b_S-a_S)x_2-(b_2-a_2)x_S=a_2b_S-a_Sb_2\,;\\[6pt]
  \cdots\\[6pt]
  \hspace*{-8mm}(b_S-a_S)x_{S-1}-(b_{S-1}-a_{S-1})x_S={}\\[6pt]
  \hspace*{32mm}{}=a_{S-1}b_S - a_Sb_{S-1}\,;\\[6pt]
  \hspace*{-5mm}m_1x_1+\cdots + m_S x_S=C\,.
  \end{array}
  \right.
  \end{equation*}
  
  \begin{figure*}[b] %fig1
\vspace*{-4pt}
 \begin{center}
 \mbox{%
 \epsfxsize=159.488mm 
 \epsfbox{sop-1.eps}
 }
 \end{center}
   \vspace*{-9pt}
\Caption{Схема работы инструмента имитационного моделирования}
\end{figure*}
  
  Итеративная процедура обеспечивает движение по  
гиперплоскости~(\ref{e5-sop}) в~направлении возрастания функции полезности 
до границы области~$P$, где и~будет найдено решение.

\vspace*{-6pt}
  
  \section{Описание работы средства имитационного моделирования}
  
  \vspace*{-3pt}
  
  Общая схема работы имитатора изображена на рис.~1. На шаге 
инициализации входных па\-ра\-мет\-ров задаются начальные параметры: 
$\lambda_s$, $\mu_s$, $\overline{M}_s$, $a_s$, $b_s$, $s\hm\in S$, $C$, $\max$~--- 
максимальное число принятых заявок в~системе. Затем выполняется запуск 
имитационного моделирования. Далее определяется ближайшее событие: 

\begin{enumerate}[(1)]
\item если это поступление, то выполняется проверка на достаточность ресурсов:
\begin{enumerate}[({1}.1)] 
\item если ресурсов хватает, то добавляется заявка, пересчитываются ресурсы, 
и~определяется продолжение моделирования; 
\item если ресурсов не хватает, то 
проверяется, является ли слайс нарушителем: 
\begin{enumerate}[({1.2.}1)]
\item если является, то входящая 
заявка считается заблокированной; 
\item если не является, то запускается 
процесс поиска нарушителя и~освобождение ресурсов, после чего происходит 
добавление заявки, пересчитываются ресурсы и~определяется продолжение 
моделирования;
\end{enumerate}
\end{enumerate}
\item если ближайшее событие~--- это обслуживание, то 
выполняется освобождение и~перерасчет ресурсов.
\end{enumerate}

\vspace*{-8pt}

  
  \section{Численный эксперимент}
  
  \vspace*{-2pt}
  
  Для численного эксперимента рассматривается диапазон 10~МГц в~сети LTE
  (long-term evolution). 
Минимально допустимая скорость в~таком диапазон составляет 0,75~Мбит/с 
$(a_s\hm=0{,}75$, $s\hm\in S)$,\linebreak максимальная~--- 79,9~Мбит/с ($b_s\hm= 
79{,}9$, $s\hm\in S$). Максимальный объем ресурса, который может 
единовременно предоставлять базовая станция в~диапазоне 10~МГц,~--- 
450~Мбит/с, для расчетов используются несколько значений объема ресурсов: 
$$
C= [325,350,375,400,425,450]\,.
$$
 Данный ресурс будет разделен на трех 
($S\hm=3$) виртуальных операторов с~гарантированным выделенным ресурсом 
$\overline{R}_1\hm=225$, $\overline{R}_2\hm= 150$ и~$\overline{R}_3\hm= 75$. 
Интенсивности поступлений $\lambda_1\hm= \lambda_2\hm= \lambda_3\hm=30$, 
интенсивности обслуживания $\mu_1\hm= \mu_2\hm= \mu_3\hm= 30$ оставим 
равными для всех операторов.
  
  На рис.~2,\,\textit{а} видно, что когда ресурсов не хватает для обеспечения 
гарантированной скорости для всех операторов, то и~вероятность блокировки 
выше. Далее при увеличении ресурса видно, что для оператора, которому 
предоставлено больше гарантированных ресурсов, уменьшение вероятности 
блокировки происходит быстрее.
  

  На рис.~2,\,\textit{б} видно, что для оператора, которому предоставляется больший 
гарантированный ресурс, увеличивается средний размер слайса, в~то время как 
для оператора, которому гарантируется меньший ресурс, этот параметр 
меняется незначительно. Это связано со способом решения оптимального 
распределения ресурса, описанного в~данной работе.
  
  На рис.~2,\,\textit{в} отображен график, показывающий изменение доли времени, когда 
оператор является нарушителем. В~связи с~тем что общий размер \mbox{ресурса} 
растет, гарантированный объем остается неизменным, а~интенсивности 
поступления~--- фиксированные для каждого оператора, можно наблюдать 
резкое увеличение значений этого параметра для оператора~3.
  
   
\vspace*{-8pt}

  
  \section{Заключение}
  
  \vspace*{-2pt}
  
  В работе описана СМО с~ограниченными 
ресурсами с~распределением ресурсов в~зависимости от
веса слайса сети. 
Предложен и~реализован алгоритм освобождения ресурсов  
слай\-са\-ми-на\-ру\-ши\-те\-ля\-ми. Предложен и~реализован алгоритм 
распределения\linebreak\vspace*{-12pt}

{ \begin{center}  %fig2
 \vspace*{-3pt}
   \mbox{%
 \epsfxsize=79.374mm 
 \epsfbox{sop-2.eps}
 }

\end{center}

\noindent
{{\figurename~2}\ \ \small{
Вероятность блокировки~(\textit{а}),  
средний размер слайса~(\textit{б}) 
и~доля времени, когда слайс находится в~состоянии нарушителя~(\textit{в})
в~зависимости от объема ресурсов: \textit{1}~--- VNO~1; 
\textit{2}~--- VNO~2; \textit{3}~--- VNO~3
}}}

\vspace*{12pt}





\noindent
 ресурсов на основе весов слайсов. Разработано средство 
имитационного моделирования механизма распределения ресурсов. Проведен 
численный эксперимент для трех слайсов. В~дальнейшем планируется 
разработать модель, в~которой перераспределение слайсов происходит не при 
каждом изменении состояния процесса, а~при выполнении некоторого 
критерия.

\vspace*{-8pt}
  
{\small\frenchspacing
 {%\baselineskip=10.8pt
 \addcontentsline{toc}{section}{References}
 \begin{thebibliography}{9}
 
 \vspace*{-1pt}
 
 
\bibitem{1-sop}
3GPP TS 23.501 V15.4.0. System architecture for the 5G System, 2018. {\sf 
http://www.3gpp.org/ftp//Specs/ archive/23\_series/23.501/23501-f40.zip}.
\bibitem{2-sop}
ITU-T Rec. Y.3101. Requirements of the IMT-2020 network, 2018. {\sf 
https://www.itu.int/rec/dologin\_pub.asp?\linebreak lang=e\&id=T-REC-Y.3101-201801-I!!PDF-E\&type= items}.
\bibitem{3-sop}
\Au{P$\acute{\mbox{e}}$rez-Romero~J., 
Sallent~O., Ferr$\acute{\mbox{u}}$s~R., 
\mbox{Agust{\!\!\ptb{$\acute{\mbox{\i}}$}}}~R.} On 
the configuration of radio resource management in a~sliced RAN~// IEEE/IFIP 
Network Operations and Management Symposium.~--- IEEE, 2018. P.~1--6. doi: 
10.1109/ NOMS.2018.8406280.
\bibitem{4-sop} 
\Au{Rouzbehani B., Correia~L.\,M., Caeiro~L.} An SLA-based method for radio resource slicing 
and allocation in virtual %\linebreak\vspace*{-12pt}
%\columnbreak 
%\noindent
 RANs~// IRACON 7th MC and Technical Meeting.~--- 
Cartagena, Spain, 2018. Cost Action 15104 TD(18)07034. P.~1--7.
\bibitem{5-sop} 
\Au{Ageev K., Garibyan~A., Golskaya~A., Gaidamaka~Yu., Sopin~E., Samouylov~K., Correia~L.} 
Modelling of virtual radio resources slicing in 5G networks~// 
Information technologies and 
mathematical modelling. Queueing theory and applications~/
Eds. A.~Dudin, A.~Nazarov, A.~Moiseev.~--- Communications in 
computer and information science ser.~--- Cham: Springer, 2019.  
Vol.~1109. P.~150--161. doi: 10.1007/978-3-030-33388-1\_13.
\bibitem{6-sop} 
\Au{Malanchini I., Valentin~S., Aydin~O.} An analysis of\linebreak gen\-er\-al\-ized 
resource sharing for multiple 
operators in cel\-lu\-lar networks~// 25th Annual Symposium
 (In\-ter\-na\-tion\-al) on Personal, Indoor, 
and Mobile Radio Com\-mu\-ni\-ca\-tion.~--- IEEE, 2014. P.~1157--1162. doi: 
10.1109/\linebreak \mbox{PIMRC}.2014.7136342.
\bibitem{7-sop}
\Au{Malanchini I., Valentin~S., Aydin~O.} Wireless resource sharing for multiple operators: 
Generalization, fairness, and the value of prediction~// Comput. Netw., 2016. Vol.~100. 
P.~110--123. doi: 10.1016/j.comnet.2016.02.014.
\end{thebibliography}

 }
 }

\end{multicols}

\vspace*{-6pt}

\hfill{\small\textit{Поступила в~редакцию 15.07.20}}

\vspace*{8pt}

%\pagebreak

%\newpage

%\vspace*{-28pt}

\hrule

\vspace*{2pt}

\hrule

%\vspace*{-2pt}

\def\tit{ANALYSIS OF~THE~NETWORK SLICING MECHANISMS WITH~GUARANTEED ALLOCATED 
RESOURCES\\ FOR~VARIOUS TRAFFIC TYPES}


\def\titkol{Analysis of~the~network slicing mechanisms with~guaranteed allocated 
resources for~various traffic types}

\def\aut{K.\,A.~Ageev$^1$, E.\,S.~Sopin$^{1,2}$, N.\,V.~Yarkina$^1$, K.\,E.~Samouylov$^{1,2}$, 
and~S.\,Ya.~Shorgin$^2$}

\def\autkol{K.\,A.~Ageev, E.\,S.~Sopin, N.\,V.~Yarkina, 
%K.\,E.~Samouylov$^{1,2}$, and~S.\,Ya.~Shorgin$^2$ 
et al.}

\titel{\tit}{\aut}{\autkol}{\titkol}

\vspace*{-9pt}


\noindent
$^1$Peoples' Friendship University of Russia (RUDN University), 6~Miklukho-Maklaya Str., Moscow 
117198, Russian\linebreak
$\hphantom{^1}$Federation

\noindent
$^2$Institute of Informatics Problems, Federal Research Center ``Computer Sciences and Control'' of the 
Russian\linebreak
$\hphantom{^1}$Academy of Sciences; 44-2~Vavilov Str., Moscow 119133, Russian Federation

\def\leftfootline{\small{\textbf{\thepage}
\hfill INFORMATIKA I EE PRIMENENIYA~--- INFORMATICS AND
APPLICATIONS\ \ \ 2020\ \ \ volume~14\ \ \ issue\ 3}
}%
 \def\rightfootline{\small{INFORMATIKA I EE PRIMENENIYA~---
INFORMATICS AND APPLICATIONS\ \ \ 2020\ \ \ volume~14\ \ \ issue\ 3
\hfill \textbf{\thepage}}}

\vspace*{3pt} 




\Abste{Network slicing is one of the key capabilities of
 modern networks, allowing several virtual mobile operators 
 to use the physical resources of one base station. 
 This allows operators and resource owners (tenants) to 
 lease and manage several dedicated logical networks with 
 specific functionality working on top of a~common infrastructure. 
 Each of these logical networks is called a~network slice and can 
 be adapted to provide certain system behavior to maintain 
 a~specified level of quality of service indicators. The paper
  describes the developed mathematical framework of the network
   slicing mechanisms 
and analyzes it by means of extensive simulations.}

\KWE{simulation modeling; queuing system; limited resources; network slicing}

\DOI{10.14357/19922264200314} 

\vspace*{-20pt}

\Ack
\noindent
The reported study was funded by the ``RUDN University Program 5-100'' and in
part by RFBR, projects  
Nos.\,19-07-00933 and 19-37-90147.

\vspace*{4pt}

 \begin{multicols}{2}

\renewcommand{\bibname}{\protect\rmfamily References}
%\renewcommand{\bibname}{\large\protect\rm References}

{\small\frenchspacing
 {%\baselineskip=10.8pt
 \addcontentsline{toc}{section}{References}
 \begin{thebibliography}{9}
 
 \vspace*{-2pt}
 
\bibitem{1-sop-1}
3GPP TS 23.501 V15.4.0. 2018. System architecture for the 5G System. 
Available at: {\sf 
http://www.3gpp.org/ftp// Specs/archive/23\_series/23.501/23501-f40.zip} 
(accessed June~15, 2020).
\bibitem{2-sop-1}
ITU-T Rec. Y.3101. 2018. Requirements of the IMT-2020 network. Available at: {\sf 
https://www.itu.int/rec/ dologin\_pub.asp?lang=e\&id=T-REC-Y.3101-201801-I!!PDF-E\&type=items} 
(accessed June~15, 2020).
\bibitem{3-sop-1}
\Aue{P$\acute{\mbox{e}}$rez-Romero, J., O.~Sallent,
 R.~Ferr$\acute{\mbox{u}}$s, and 
R.~\mbox{Agust{\!\!\ptb{$\acute{\mbox{\i}}$}}}}. 2018. On the configuration of radio resource management in a sliced 
RAN. \textit{IEEE/IFIP Network Operations and Management Symposium}.
 IEEE. 1--6.  doi: 10.1109/ NOMS.2018.8406280.
\bibitem{4-sop-1}
\Aue{Rouzbehani, B., L.\,M.~Correia, and L.~Caeiro.} 2018. An SLA-based method for radio resource 
slicing and allocation in virtual RANs. \textit{IRACON 7th MC and Technical Meeting}. Cartagena. 
TD(18)07034. 1--7.
\bibitem{5-sop-1}
\Aue{Ageev, K., A.~Garibyan, A.~Golskaya, Yu.~Gaidamaka, E.~Sopin, K.~Samouylov, and L.~Correia.} 
2019. Modelling of virtual radio resources slicing in 5G networks. 
\textit{Information technologies and 
mathematical modelling. Queueing theory and applications}. Eds. 
A.~Dudin, A.~Nazarov, and A.~Moiseev. 
Communications in computer and information science ser. Cham:
Springer. 1109:150--161. 
\bibitem{6-sop-1}
\Aue{Malanchini, I., S.~Valentin, and O.~Aydin.} 2014. An analysis of generalized resource sharing for 
multiple operators in cellular networks. \textit{25th Annual Symposium (International)
on Personal,  Indoor, and Mobile Radio Communication}. IEEE. 1157--1162. 
doi:  10.1109/PIMRC.2014.7136342.
\bibitem{7-sop-1}
\Aue{Malanchini, I., S.~Valentin, and O.~Aydin.} 2016. Wireless resource sharing for multiple operators: 
Generalization, fairness, and the value of prediction. \textit{Comput. Netw.} 100:110--123.
doi: 10.1016/j.comnet.2016.02.014.

\end{thebibliography}

 }
 }

\end{multicols}

\vspace*{-6pt}

\hfill{\small\textit{Received July 15, 2020}}

%\pagebreak

%\vspace*{-24pt}



\Contr

\noindent
\textbf{Ageev Kirill A.} (b.\ 1993)~--- PhD student, Peoples' Friendship University of Russia (RUDN 
University), 6~Miklukho-Maklaya Str., Moscow 117198, Russian Federation; \mbox{ageev-ka@rudn.ru}

\vspace*{3pt}

\noindent
\textbf{Sopin Eduard S.} (b.\ 1987)~--- Candidate of Science in physics and mathematics, associate 
professor, Peoples' Friendship University of Russia (RUDN University), 6~Miklukho-Maklaya Str., 
Moscow 117198, Russian Federation; senior scientist, Institute of Informatics Problems, Federal Research 
Center ``Computer Sciences and Control'' of the Russian Academy of Sciences; 44-2 Vavilov Str., Moscow 
119133, Russian Federation; \mbox{sopin-es@rudn.ru}

\vspace*{3pt}

\noindent
\textbf{Yarkina Natalia V.} (b.\ 1979)~--- Candidate of Science in physics and mathematics, associate 
professor, Peoples' Friendship University of Russia (RUDN University), 6~Miklukho-Maklaya Str., 
Moscow 117198, Russian Federation; \mbox{yarkina-nv@rudn.ru}

\vspace*{3pt}

\noindent
\textbf{Samouylov Konstantin E.} (b.\ 1955)~--- Doctor of Science in technology, professor, Head of 
Department, Peoples' Friendship University of Russia (RUDN University), 6~Miklukho-Maklaya Str., 
Moscow 117198, Russian Federation; senior scientist, Institute of Informatics Problems, Federal Research 
Center ``Computer Sciences and Control'' of the Russian Academy of Sciences; 44-2~Vavilov Str., Moscow 
119133, Russian Federation; \mbox{samuylov\_ke@rudn.university}

\vspace*{3pt}

\noindent
\textbf{Shorgin Sergey Ya.} (b.\ 1952)~--- Doctor of Science in physics and mathematics, professor, 
principal scientist, Institute of Informatics Problems, Federal Research Center ``Computer Sciences and 
Control'' of the Russian Academy of Sciences, 44-2~Vavilov Str., Moscow 119333, Russian Federation; 
\mbox{sshorgin@ipiran.ru}
\label{end\stat}

\renewcommand{\bibname}{\protect\rm Литература} 