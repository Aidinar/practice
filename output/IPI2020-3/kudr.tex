\newcommand{\G}{{\sf Ge}}
\renewcommand{\Re}{{\sf Re} }

\def\stat{kudr}

\def\tit{МЕТОД ЛОГАРИФМИЧЕСКИХ МОМЕНТОВ ДЛЯ ОЦЕНИВАНИЯ ПАРАМЕТРОВ 
ГАММА-ЭКСПОНЕНЦИАЛЬНОГО РАСПРЕДЕЛЕНИЯ$^*$}

\def\titkol{Метод логарифмических моментов для оценивания параметров 
гамма-экспоненциального распределения}

\def\aut{А.\,А.~Кудрявцев$^1$, О.\,В.~Шестаков$^2$}

\def\autkol{А.\,А.~Кудрявцев, О.\,В.~Шестаков}

\titel{\tit}{\aut}{\autkol}{\titkol}

\index{Кудрявцев А.\,А.}
\index{Шестаков О.\,В.}
\index{Kudryavtsev A.\,A.}
\index{Shestakov O.\,V.}
 

{\renewcommand{\thefootnote}{\fnsymbol{footnote}} \footnotetext[1]
{Работа выполнена при частичной 
финансовой поддержке РФФИ (проект 20-07-00655) и в соответствии с программой Московского центра
 фундаментальной и~прикладной математики.}}


\renewcommand{\thefootnote}{\arabic{footnote}}
\footnotetext[1]{Московский государственный университет 
им.~М.\,В.~Ломоносова, факультет вычислительной математики и~кибернетики, 
\mbox{nubigena@mail.ru}}
\footnotetext[2]{Московский государственный университет 
им.~М.\,В.~Ломоносова, факультет вычислительной математики и~кибернетики; 
Институт проблем информатики Федерального исследовательского центра 
<<Информатика и~управление>> Российской академии наук, \mbox{oshestakov@cs.msu.su}}

\vspace*{-8pt}




\Abst{Рассматривается модифицированный метод моментов для оценивания 
параметров гам\-ма-экс\-по\-нен\-ци\-аль\-но\-го распределения. Обосновывается сильная 
состоятельность полученных оценок. Гам\-ма-экс\-по\-нен\-ци\-аль\-ное распределение 
представляет собой удобный механизм при моделировании процессов и~явлений 
с~по\-мощью мас\-штаб\-ных смесей обобщенных гам\-ма-рас\-пре\-де\-ле\-ний. Подобные задачи 
возникают во многих естественно-научных областях в~предположении 
о~рандомизированности рас\-смат\-ри\-ва\-емых па\-ра\-мет\-ров и~могут быть единообразно описаны 
в~терминах байесовских моделей баланса. Полученные результаты могут применяться 
в~широком классе задач, использующих для моделирования распределения 
с~положительным неограниченным носителем, без дополнительных предположений 
о~представлении исследуемого объекта в~терминах масштабной смеси, ввиду большого 
разнообразия видов плотности пятипараметрического гам\-ма-экс\-по\-нен\-ци\-аль\-но\-го 
распределения.}

\KW{оценивание параметров; гамма-экс\-по\-нен\-ци\-аль\-ное 
распределение; смешанные распределения;  обобщенное гам\-ма-рас\-пре\-де\-ле\-ние; 
метод  моментов; состоятельная оценка}

\DOI{10.14357/19922264200307} 
 
\vspace*{-6pt}


\vskip 10pt plus 9pt minus 6pt

\thispagestyle{headings}

\begin{multicols}{2}

\label{st\stat}

\section{Введение}

В~различных областях жизнедеятельности присутствуют характеристики систем, 
выражаемые при помощи отношения двух  величин. 

Так, в~демографии уровень 
младенческой смертности определяется как отношение чис\-ла умерших в~возрасте до 
года к~числу родившихся за некоторый период времени~\cite{Borisov2001}, а~индекс 
раз\-во\-ди\-мости~--- как отношение коэффициента суммарной разводимости к~коэффициенту 
суммарной брач\-ности~\cite{Rybako2005}; в~физике коэффициент трансформации~--- 
это  отношение выходного напряжения к~входному, 
а~универсальная функция Кирхгофа~--- 
отношение излучательной способности тела к~поглощательной~\cite{KuzRog2012}; 
в~тео\-рии массового обслуживания отношение интенсивности входящего потока 
к~ин\-тен\-сив\-ности обслуживания определяет коэффициент загрузки 
сис\-те\-мы~\cite{BoPe1995}; при моделировании чрезвычайных ситуаций пожароопасность объекта 
определяется отношением угрозы возникновения пожара к~фактору пожарозащиты~\cite{ShaRaSha2010}; 
в~тео\-рии на\-деж\-ности ожи\-да\-емое время безотказной работы 
представимо в~виде отношения среднего времени безотказной работы к~среднему 
времени восстановления~\cite{ZdRo2004}. 

Ряд примеров можно продолжить. Каждую из 
таких характеристик можно рассматривать как индекс баланса системы~\cite{Ku2018}.
Применение рандомизационного байесовского подхода к~описанным моделям дает 
возможность изучать характеристики\linebreak индекса баланса как масштабной смеси 
вероятностных законов. 

В~случае когда для описания процессов и~явлений 
используются распределения с~неограниченным неотрицательным носителем, широкую 
популярность приобрел гам\-ма-класс распределений. В~работе~\cite{Ku2019_2} было 
описано так называемое гам\-ма-экс\-по\-нен\-ци\-аль\-ное распределение, в~терминах 
которого, в~част\-ности, выражаются характеристики масштабных смесей обобщенных 
гамма-рас\-пре\-де\-ле\-ний, применяемых в~байесовских моделях баланса~\cite{Ku2018}.

На практике исследователь имеет дело с~наблюдаемыми величинами, отражающими 
про\-яв\-ление анализируемого реального процесса, в~отношении которых делаются 
некоторые модельные предположения о~виде их распределения. При этом 
пред\-полагается, что истинное значение параметров\linebreak распределения неизвестно 
и~требует оценки.
Заметим, что нередко при моделировании используются частные случаи обобщенного 
гам\-ма-рас\-пре\-де\-ле\-ния, такие как
экспоненциальное распределение;
$\chi^2$-рас\-пре\-де\-ле\-ние;
распределение Эрланга;
гам\-ма-рас\-пре\-де\-ле\-ние;
полунормальное распределение,\linebreak или
распределение максимума процесса броуновского движения;
распределение Рэлея;
распределение Макс\-вел\-ла--Больц\-ма\-на;
$\chi$-распределение;\linebreak
m-распределение Накагами;
распределение Виль\-со\-на--Хиль\-фер\-ти;
распределение Вей\-бул\-ла--Гне\-ден\-ко
и многие другие, включая масштабированные и~обратные аналоги перечисленных.

Дадим необходимые определения.
Обозначим обобщенное гам\-ма-рас\-пре\-де\-ле\-ние, или распределение 
Криц\-ко\-го--Мен\-ке\-ля~\cite{KrMe1946,KrMe1948}, с~плот\-ностью
$$
f(x)=\fr{|v| x^{vq -1}e^{-(x/\theta)^v}}{\theta^{vq}\Gamma(q)},\ \ v\neq0,\ 
\ q>0,\ \ \theta>0,\ \ x>0,
$$
через $\mathrm{GG}\,(v,q,\theta)$.

\smallskip

\noindent
\textbf{Определение~1.}
Назовем функцию вида
\begin{multline}
\label{GEF}
\G_{\alpha,\, \beta} (x) = \sum\limits_{k=0}^{\infty}\fr{x^k}{k!}\,
 \Gamma(\alpha k + 
\beta), \\
 x\in\mathbb{R}, \enskip 0\le\alpha<1, \enskip \beta> 0,
\end{multline}
гамма-экс\-по\-нен\-ци\-аль\-ной функцией~\cite{KuTi2017}.


Функция~(\ref{GEF}) обобщает на случай $\beta\hm\neq1$ преобразование, введенное 
Леруа~\cite{LeRoy1900_1} для исследования производящих функций специального 
вида. Кроме того, функцию~(\ref{GEF}) можно рас\-смат\-ри\-вать (при некоторых 
допущениях) как частный случай функции Сри\-ва\-ста\-ва--То\-мов\-ски~\cite{SrTo2009}, 
обобщающей функ\-цию Мит\-таг-Леф\-фле\-ра~\cite{GoKiMaRo2014}.

\smallskip

\noindent
\textbf{Определение~2.}
Будем говорить, что случайная величина~$\zeta$ имеет гам\-ма-экс\-по\-нен\-ци\-аль\-ное 
распределение $\mathrm{GE}\,(r,\nu,s,t,\delta)$ с~па\-ра\-мет\-ра\-ми изгиба $0\hm\le r\hm<1$, формы 
$\nu\hm\neq0$, концентрации $s,t\hm>0$ и~масштаба $\delta\hm>0$, если ее плотность при 
$x\hm>0$ задается соотношением:
$$
g_E(x) =
\fr{|\nu|x^{t\nu-1}}{\delta^{t\nu}\Gamma(s)\Gamma(t)}
   \G_{r,\, tr+s}\left(-\left(\fr{x}{\delta}\right)^{\nu}\right),
$$
где $E=(r,\nu,s,t,\delta)$~\cite{Ku2019_2}.

\smallskip

В работе~\cite{Ku2019_2} было показано, что гам\-ма-экс\-по\-нен\-ци\-аль\-ное распределение 
обладает следующими свойствами.

\smallskip

\noindent
\textbf{Лемма~1.}
%\label{GG_to_GE}
\begin{enumerate}[1.]
\item \textit{Пусть независимые случайные величины~$\lambda$ и~$\mu$ имеют соответственно 
распределения $\mathrm{GG}\,(v,q,\theta)$ и~$\mathrm{GG}\,(u,p,\alpha)$, $uv\hm>0$. Тогда
распределение~$\lambda$ совпадает с~$\mathrm{GE}\,(0,v,\cdot,q,\theta)$;
распределение~$\lambda/\mu$ при $|u|\hm>|v|$ совпадает с}
$\mathrm{GE}\,(v/u,v,p,q,\theta/\alpha)$;
\textit{распределение~$\lambda/\mu$ при $|v|\hm>|u|$ совпадает с~$\mathrm{GE}\,(u/v,-
u,q,p,\theta/\alpha)$}.
\item
 \textit{При $0<r<1$ плотность $g_E(x)$, $E\hm=(r,\nu,s,t,\delta)$, совпадает 
с~плот\-ностью отношения независимых случайных величин, имеющих обобщенные 
гам\-ма-рас\-пре\-де\-ле\-ния}
$\mathrm{GG}\,(\nu,t,\delta)$ и~$\mathrm{GG}\,(\nu/r,s,1)$.
\end{enumerate}


В случае моделирования реального процесса при помощи гам\-ма-экс\-по\-нен\-ци\-аль\-но\-го 
распределения неизбежно возникает вопрос оценивания неизвестных параметров по 
реальным данным. Ввиду представления плот\-ности~$g_E(x)$ в~терминах специальной 
гам\-ма-экс\-по\-нен\-ци\-аль\-ной функции~(\ref{GEF}) метод максимального правдоподобия 
представляется затруднительным. То же можно сказать и~о~прямом методе моментов. 
По этой причине предлагается оценивать параметры гам\-ма-экс\-по\-нен\-ци\-аль\-но\-го 
распределения при помощи модифицированного метода, основанного на 
логарифмических моментах.

\section{Логарифмические моменты гамма-экспоненциального распределения}

Введем обозначение для ди\-гам\-ма-функ\-ции: $ \psi(z)\hm={\Gamma'(z)}/{\Gamma(z)}$.

Найдем первые два логарифмических момента случайной величины $\zeta\hm\sim 
\mathrm{GE}\,(r,\nu,s,t,\delta)$. Рас\-смот\-рим преобразование Меллина
$$
\mathcal{M}_\zeta(z) = \int\limits_{0}^{\infty}x^z \, dF_\zeta(x)\,,\enskip 
z\in\mathbb{C}\,,
$$
распределения~$\zeta$. Воспользуемся леммой~1 и~представлением 
$\zeta\stackrel{d}{=}\lambda/\mu$, где независимые случайные величины~$\lambda$ 
и~$\mu$ имеют соответственно распределения $\mathrm{GG}\,(\nu,t,\delta)$ и~$\mathrm{GG}\,(\nu/r,s,1)$.
Поскольку для $\lambda\hm\sim \mathrm{GG}\,(\nu,t,\delta)$ преобразование Меллина имеет вид:
$$
\mathcal{M}_\lambda(z) = 
\fr{\delta^z}{\Gamma(t)}\,\Gamma\left(t+\fr{z}{\nu}\right), \enskip
 t+\fr{\Re  (z)}{\nu}>0\,,
 $$
для отношения $\lambda\sim \mathrm{GG}\,(\nu,t,\delta)$ к~$\mu\sim 
\mathrm{GG}\,(\nu/r,s,1)$ 
выполнено
\begin{multline*}
\mathcal{M}_{\lambda/\mu}(z) = 
\fr{\delta^z}{\Gamma(t)\Gamma(s)}\,\Gamma\left(t+\fr{z}{\nu}\right)\Gamma\left
(s-\fr{rz}{\nu}\right), \\
 t+\fr{\Re (z)}{\nu}>0, \enskip
 s-\fr{r\Re  (z)}{\nu}>0\,,
\end{multline*}
откуда получаем вид моментов $\zeta\stackrel{d}{=}\lambda/\mu$:
\begin{multline}
\label{moments_zeta}
\e \zeta^k = 
\fr{\delta^{k}}{\Gamma(t)\Gamma(s)}\,\Gamma\left(t+\fr{k}{\nu}\right)\Gamma
\left(s-\fr{rk}{\nu}\right), \\
 \ t+\fr{k}{\nu}>0\,, \ \ s-\fr{rk}{\nu}>0\,,
\end{multline}
и характеристической функции логарифма~$\zeta$:
\begin{multline}
\label{ChF_zeta}
\e e^{iy\ln\zeta}= 
\fr{\delta^{iy}}{\Gamma(t)\Gamma(s)}\,\Gamma\left(t+\fr{iy}{\nu}\right)\Gamma
\left( s-\fr{iry}{\nu}\right), \\
 y\in\mathbb{R}\,.
\end{multline}
Дважды продифференцировав соотношение~(\ref{ChF_zeta}), получаем:
\begin{align*}
\e\ln\zeta&=\fr{\nu\ln\delta+\psi(t)-r\psi(s)}{\nu}\,;
\\
\e\ln^2\zeta&=
\frac{\left[\nu\ln\delta+\psi(t)-
r\psi(s)\right]^2}{\nu^2}+\fr{\psi'(t)}{\nu^2}+\fr{r^2\psi'(s)}{\nu^2}.
\end{align*}

\section{Оценивание параметров изгиба и~масштаба}

Зафиксируем параметры формы~$\nu$ и~концентрации~$t$ и~$s$. Заметим, что, 
поскольку функция, стоящая в~правой части~(\ref{moments_zeta}), определена не 
для всех значений параметров и~необратима, использовать классический метод 
моментов напрямую не представляется возможным. Оценим при помощи 
модифицированного метода моментов параметры изгиба~$r$ и~масштаба~$\delta$ 
распределения $\mathrm{GE}\,(r,\nu,s,t,\delta)$ с~учетом соотношения 
$\zeta\stackrel{d}{=}\lambda/\mu$, где независимые случайные величины~$\lambda$ 
и~$\mu$ имеют соответственно распределения~$\mathrm{GG}\,(\nu,t,\delta)$ 
и~$\mathrm{GG}\,(\nu/r,s,1)$.

Введем обозначение для выборочных логарифмических моментов~$\zeta$:
$$
L_k(X)=\fr{1}{n}\sum\limits_{i=1}^n\ln^k X_i,
$$
где $X=(X_1,\ldots,X_n)$~--- выборка из распределения~$\zeta$.

Составим систему из двух уравнений с~двумя неизвестными~$r$ и~$\delta$:
$$
\e\ln\zeta=L_1(X);\enskip \e\ln^2\zeta=L_2(X)\,.
$$
Очевидно, что исходная система уравнений эквивалентна следующей:
\begin{align}
\label{first_eq}
\fr{\nu\ln\delta+\psi(t)-r\psi(s)}{\nu}&=L_1(X)\,;
\\
\label{second_eq}
\fr{\psi'(t)+r^2\psi'(s)}{\nu^2}&=L_2(X)-L_1^2(X)\,.
\end{align}
Из последнего уравнения с~учетом $0\hm\le r\hm<1$ получаем, что оценка параметра 
изгиба имеет вид:
\begin{equation}
\label{r_est}
{\hat r}(X)=
\sqrt{\fr{\nu^2(L_2(X)-L_1^2(X))-{\psi'(t)}}{\psi'(s)}}\,,
\end{equation}
откуда следует вид оценки параметра масштаба:
\begin{multline}
\label{delta_r_est}
{\hat \delta}_r(X)=\exp\Bigg\{
L_1(X)-\fr{\psi(t)}{\nu}+{}\\
{}+\psi(s)\sqrt{\fr{\nu^2(L_2(X)-L_1^2(X))-
{\psi'(t)}}{\nu^2\psi'(s)}}\Bigg\}\,.
\end{multline}

Из усиленного закона больших чисел и~непрерывности функций, стоящих в~правых 
частях~(\ref{r_est}) и~(\ref{delta_r_est}), следует свойство полученных оценок.

\smallskip

\noindent
\textbf{Теорема~1.}
\textit{При фиксированных параметрах~$\nu$, $t$ и~$s$ распределения 
$\mathrm{GE}\,(r,\nu,s,t,\delta)$ оценки~${\hat r}(X)$ и~${\hat \delta}_r(X)$ параметров~$r$ 
и~$\delta$ обладают свойством сильной состоятельности}.


\section{Оценивание параметров формы и~масштаба}

Зафиксируем параметры изгиба~$r$ и~концентрации~$t$ и~$s$. Оценим при помощи 
модифицированного метода моментов параметры формы~$\nu$ и~масштаба~$\delta$ 
распределения $\mathrm{GE}\,(r,\nu,s,t,\delta)$.

Проводя рассуждения, аналогичные приведенным в~предыдущем разделе, можно 
заметить, что двух уравнений~(\ref{first_eq}) и~(\ref{second_eq}) недостаточно 
для решения поставленной задачи ввиду невозможности определения
 знака параметра~$\nu$ исходя из равенства
$$
\nu^2=\fr{\psi'(t)+r^2\psi'(s)}{L_2(X)-L_1^2(X)}\,.
$$

Рассмотрим дополнительное уравнение:
$$
\e\ln^3\zeta=L_3(X)\,.
$$
Поскольку выполнено~(\ref{ChF_zeta}), имеем:
\begin{multline*}
\e\ln^3\zeta=
\fr{\left[\nu\ln\delta+\psi(t)-
r\psi(s)\right]^3}{\nu^3}+{}\\
{}+\fr{3\left[\nu\ln\delta+\psi(t)
-r\psi(s)\right]\psi'(t)}{\nu^3}+{}
\\
{}+\fr{3r^2\left[\nu\ln\delta+\psi(t)
-r\psi(s)\right]\psi'(s)}{\nu^3}+{}\\
{}+\fr{\psi''(t)}{\nu^3}-
\fr{r^3\psi''(s)}{\nu^3},
\end{multline*}
откуда с~учетом~(\ref{first_eq}) и~(\ref{second_eq}) получаем
$$
\fr{\psi''(t)}{\nu^3}-\fr{r^3\psi''(s)}{\nu^3}=L_3(X)-
3L_1(X)L_2(X)+2L_1^3(X)\,.
$$

Таким образом, оценка параметра формы~$\nu$ имеет вид:
\begin{equation*}
%\label{nu_est}
{\hat \nu}(X)=\sqrt[3]{\fr{\psi''(t)-r^3\psi''(s)}{L_3(X)-
3L_1(X)L_2(X)+2L_1^3(X)}}\,.
\end{equation*}
Из~(\ref{first_eq}) следует вид оценки параметра масштаба $\delta$:
\begin{multline*}
%\label{delta_nu_est}
{\hat \delta}_\nu(X)=\exp\left\{
\vphantom{\sqrt[3]{\fr{L_3(X)-3L_1(X)L_2(X)+2L_1^3(X)}{\psi''(t)-r^3\psi''(s)}}}
L_1(X)-(\psi(t)-r\psi(s))\times{}\right.\\
\left.{}\times
\sqrt[3]{\fr{L_3(X)-3L_1(X)L_2(X)+2L_1^3(X)}{\psi''(t)-r^3\psi''(s)}}\right\}.
\end{multline*}

Аналогично заключению разд.~3 справедливо следующее утверждение.

\smallskip

\noindent
\textbf{Теорема~2.} \textit{При фиксированных параметрах $r$, $t$ и~$s$ 
распределения 
$\mathrm{GE}\,(r,\nu,s,t,\delta)$ оценки ${\hat \nu}(X)$ 
и~${\hat \delta}_\nu(X)$ 
параметров $\nu$ и~$\delta$ обладают свойством сильной состоятельности}.


\section{Заключение}

При получении сильно состоятельных оценок\linebreak двух пар параметров
 гам\-ма-экс\-по\-нен\-ци\-аль\-но\-го распределения, описанных в~статье, активно использовалось 
представление гам\-ма-экс\-по\-нен\-ци\-аль\-но\-го распределения 
как масштабной смеси 
обобщенных гам\-ма-рас\-пре\-де\-ле\-ний. При этом изначально предполагалось, что 
исследуемое рас\-пределение пред\-став\-ля\-ет собой удобный механизм анализа 
байесовских моделей баланса, в~основе которых лежит представление целевого 
объекта как частного двух независимых случайных величин. Однако 
пятипараметрическое гам\-ма-экс\-по\-нен\-ци\-аль\-ное распределение обладает широким 
набором возможных видов плот\-ности и~может оказаться\linebreak полезным при изучении 
и~других моделей, ис\-поль\-зу\-ющих для описания процессов и~явлений распределения 
с~неограниченным положительным носителем.


{\small\frenchspacing
 {%\baselineskip=10.8pt
 \addcontentsline{toc}{section}{References}
 \begin{thebibliography}{99}
\bibitem{Borisov2001}
\Au{Борисов~В.\,А.\/}
Демография.~--- М.: NOTABENE, 2001. 272~с.

\bibitem{Rybako2005}
\Au{Волгин~Н.\,А., Рыбаковский~Л.\,Л., Калмыкова~Н.\,М. и~др.} 
%Архангельский~В.\,Н., Иванова~Е.\,И., Захарова~О.\,Д., Иванова~А.\,Е., 
%Денисенко~М.\,Б., Тихомиров~Н.\,П., Тихомирова~Т.\,М.}
Демография.~--- М.: Логос, 2005. 280~с.

%/ Под ред. Н.\,А.~Волгина, Л.\,Л.~Рыбаковского.


\bibitem{KuzRog2012}
\Au{Кузнецов~С.\,И., Рогозин~К.\,И.}
Справочник по физике.~--- Томск: ТПУ, 2012. 224~с.

\bibitem{BoPe1995}
\Au{Бочаров~П.\,П., Печинкин~А.\,В.}
Теория массового обслуживания.~--- М.: РУДН, 1995. 529~с.

\bibitem{ShaRaSha2010}
\Au{Шаптала~В.\,Г., Радоуцкий~В.\,Ю., Шаптала~В.\,В.}
Основы моделирования чрезвычайных ситуаций.~--- 
Белгород: БГТУ, 2010. 166~с.
%/ Под общ. ред. В.\,Г.~Шапталы.


\bibitem{ZdRo2004}
\Au{Здоровцов~И.\,А., Королев~В.\,Ю.}
Основы теории надежности волоконно-оптических линий передачи железнодорожного 
транспорта.~--- М.: МАКС Пресс, 2004. 308~с.

\bibitem{Ku2018}
\Au{Кудрявцев~А.\,А.}
Байесовские модели баланса~// Информатика и~её применения, 2018. 
Т.~12. Вып.~3. С.~18--27.

\bibitem{Ku2019_2}
\Au{Кудрявцев~А.\,А.}
О~пред\-став\-ле\-нии гам\-ма-экс\-по\-нен\-ци\-аль\-но\-го 
и~обобщенного отрицательного биномиального распределений~// Информатика 
и~её применения, 2019. Т.~13. Вып.~4. С.~78--82.

\bibitem{KrMe1946}
\Au{Крицкий~С.\,Н., Менкель~М.\,Ф.}
О приемах исследования случайных колебаний речного стока~// 
Труды НИУ ГУГМС. Сер.~IV, 1946. Вып.~29. С.~3--32.
%. Труды Гос. гидрологического ин-та, вып. 29, 1946.

\bibitem{KrMe1948}
\Au{Крицкий~С.\,Н., Менкель~М.\,Ф.}
Выбор кривых распределения вероятностей для расчетов речного стока~// 
Известия АН СССР. Отд. техн. наук, 1948. №\,6. С.~15--21.

\bibitem{KuTi2017}
\Au{Кудрявцев~А.\,А., Титова~А.\,И.}
Гам\-ма-экс\-по\-нен\-ци\-аль\-ная функция в~байесовских моделях массового обслуживания~// 
Информатика и~её применения, 2017. Т.~11. Вып.~4. С.~104--108.

\bibitem{LeRoy1900_1}
\Au{Le~Roy\ \,\,$\acute{\mbox{\!\!E}}$.}
Sur les s$\acute{\mbox{e}}$ries divergentes et les fonctions 
d$\acute{\mbox{e}}$finies par un d$\acute{\mbox{e}}$veloppement de Taylor~// 
Ann. Facult$\acute{\mbox{e}}$ Sci.
Toulouse 2 Ser., 1900. 
Vol.~2. No.\,3. P.~317--384.

\bibitem{SrTo2009}
\Au{Srivastava~H.\,M., Tomovski~{\ptb{\v{Z}}}.}
Fractional calculus with an integral operator containing a generalized 
Mittag-Leffler function in the kernel~// 
Appl. Math. Comput., 2009. Vol.~211. P.~198--210.

\bibitem{GoKiMaRo2014}
\Au{Gorenlo~R., Kilbas~A.\,A., Mainardi~F., Rogosin~S.\,V.}
Mittag-Leffler functions, related topics and applications.~--- 
Berlin--Heidelberg: Springer-Verlag, 2014. 443~p.
\end{thebibliography}

 }
 }

\end{multicols}

\vspace*{-6pt}

\hfill{\small\textit{Поступила в~редакцию 04.07.20}}

\vspace*{8pt}

%\pagebreak

\newpage

\vspace*{-28pt}

%\hrule

%\vspace*{2pt}

%\hrule

%\vspace*{-2pt}

\def\tit{METHOD OF LOGARITHMIC MOMENTS FOR~ESTIMATING 
THE~GAMMA-EXPONENTIAL DISTRIBUTION PARAMETERS}

\def\titkol{Method of logarithmic moments for~estimating 
the~gamma-exponential distribution parameters}

\def\aut{A.\,A.~Kudryavtsev$^1$ and~O.\,V.~Shestakov$^{1,2}$}

\def\autkol{A.\,A.~Kudryavtsev and~O.\,V.~Shestakov}

\titel{\tit}{\aut}{\autkol}{\titkol}

\vspace*{-15pt}


\noindent
$^1$Department of Mathematical Statistics, Faculty of Computational 
Mathematics and Cybernetics, M.\,V.~Lomo-\linebreak 
$\hphantom{^1}$nosov Moscow State University, 
1-52~Leninskiye Gory, GSP-1, Moscow 119991, Russian Federation

\noindent
$^2$Institute of Informatics Problems, Federal Research Center 
``Computer Science and Control'' of the Russian\linebreak
$\hphantom{^1}$Academy of Sciences, 
44-2~Vavilov Str., Moscow 119333, Russian Federation

\def\leftfootline{\small{\textbf{\thepage}
\hfill INFORMATIKA I EE PRIMENENIYA~--- INFORMATICS AND
APPLICATIONS\ \ \ 2020\ \ \ volume~14\ \ \ issue\ 3}
}%
 \def\rightfootline{\small{INFORMATIKA I EE PRIMENENIYA~---
INFORMATICS AND APPLICATIONS\ \ \ 2020\ \ \ volume~14\ \ \ issue\ 3
\hfill \textbf{\thepage}}}

\vspace*{3pt} 

\Abste{The article discusses a~modified method of moments for 
estimating the parameters of gamma-exponential distribution. 
The strong consistency of the estimates obtained is proved. 
Gamma-exponential distribution is a~convenient mechanism for 
modeling the processes and phenomena using scale mixtures of generalized 
gamma distributions. Such problems arise in many fields of science 
under the assumption that the considered parameters are randomized 
and can be described in terms of Bayesian balance models. 
The obtained results can be applied in a~wide class of problems that 
use for modeling the distribution with  positive unlimited support, 
without additional assumptions about the representation of the 
studied object in terms of a~scale mixture, due to the wide variety 
of density types of the five-parameter gamma-exponential distribution.}

\KWE{parameter estimation; gamma-exponential distribution; 
mixed distributions; generalized gamma distribution; method of moments; 
consistent estimate}

\DOI{10.14357/19922264200307} 

\vspace*{-20pt}

\Ack

\vspace*{-6pt}

\noindent
The work was partly supported by the Russian Foundation for Basic Research 
(project 20-07-00655). The research was conducted in accordance with the 
Program of the Moscow Center for Fundamental and Applied Mathematics.

%\vspace*{6pt}

 \begin{multicols}{2}

\renewcommand{\bibname}{\protect\rmfamily References}
%\renewcommand{\bibname}{\large\protect\rm References}

{\small\frenchspacing
 {%\baselineskip=10.8pt
 \addcontentsline{toc}{section}{References}
 \begin{thebibliography}{99}
\bibitem{1-kudr}
\Aue{Borisov, V.\,A.}
 2001. \textit{Demografiya} [Demography]. Moscow: NOTABENE. 272~p.
\bibitem{2-kudr}
\Aue{Volgin, N.\,A., L.\,L.~Rybakovskiy, N.\,M.~Kalmykova, 
\textit{et al.}} 2005. \textit{Demografiya} [Demography]. Moscow: Logos. 280~p.
\bibitem{3-kudr}
\Aue{Kuznetsov, S.\,I., and K.\,I.~Rogozin.} 2012. 
\textit{Spravochnik po fizike} [Handbook of physics]. Tomsk: TPU. 224~p.
\bibitem{4-kudr}
\Aue{Bocharov, P.\,P., and A.\,V.~Pechinkin.} 1995. 
\textit{Teoriya massovogo obsluzhivaniya} [Queueing theory]. Moscow: RUDN. 529~p.
\bibitem{5-kudr}
\Aue{Shaptala, V.\,G., V.\,Yu.~Radoutskiy, and V.\,V.~Shaptala.}
2010. \textit{Osnovy modelirovaniya chrezvychaynykh situatsiy} 
[Basics of modeling of emergency situations]. Belgorod: BGTU. 166~p.
\bibitem{6-kudr}
\Aue{Zdorovtsov, I.\,A., and V.\,Yu.~Korolev.}
2004. \textit{Osnovy teorii nadezhnosti volokonno-opticheskikh liniy peredachi 
zheleznodorozhnogo transporta} [Fundamentals of reliability theory of fiber 
optic transmission lines for railway transport]. Moscow: MAKS Press. 308~p.
\bibitem{7-kudr}
\Aue{Kudryavtsev, A.\,A.}
 2018. Bayesovskie modeli balansa [Bayesian balance models]. 
 \textit{Informatika i~ee Primeneniya~--- Inform. Appl.} 12(3):18--27.
\bibitem{8-kudr}
\Aue{Kudryavtsev, A.\,A.} 2019. O~predstavlenii gamma-eksponentsial'nogo 
i~obobshchennogo otritsatel'nogo binomial'nogo raspredeleniy 
[On the representation of gamma-exponential and generalized negative 
binomial distributions]. \textit{Informatika i~ee Primeneniya~--- 
Inform. Appl.} 13(4):78--82.
\bibitem{9-kudr}
\Aue{Kritsky, S.\,N., and M.\,F.~Menkel.} 1946. 
O~priemakh issledovaniya sluchaynykh kolebaniy rechnogo stoka 
[Methods of investigation of random fluctuations of river flow]. 
\textit{Trudy NIU GUGMS Ser.~IV} 
[Proceedings of GUGMS research institutions. Ser. IV] 29:3--32.
\bibitem{10-kudr}
\Aue{Kritsky, S.\,N., and M.\,F.~Menkel.}
 1948. Vybor krivykh raspredeleniya veroyatnostey dlya raschetov rechnogo 
 stoka [Selection of probability distribution curves for river flow calculations].
 \textit{Izvestiya AN SSSR. Otd. tekhn. nauk}
  [Herald of the USSR Academy of Sciences. Technical Sciences] 6:15--21.
\bibitem{11-kudr}
\Aue{Kudryavtsev, A.\,A., and A.\,I.~Titova.}
2017. Gamma-eksponentsial'naya funktsiya v~bayesovskikh modelyakh massovogo 
obsluzhivaniya [Gamma-exponential function in Bayesian queuing models]. 
\textit{Informatika i~ee Primeneniya~--- Inform. Appl.} 11(4):104--108.
\bibitem{12-kudr}
\Aue{Le Roy,~$\acute{\mbox{E}}$.} 1900. Sur les series divergentes 
et les fonctions definiesparun developpement de Taylor. 
\textit{Ann. Facult$\acute{\mbox{e}}$ Sci. 
Toulouse 2~Ser.} 2(3):317--384.
\bibitem{13-kudr}
\Aue{Srivastava, H.\,M., and Z.~Tomovski.}
 2009. Fractional calculus with an integral operator containing 
 a~generalized Mittag-Leffler function in the kernel. 
 \textit{Appl. Math. Comput.} 211:198--210.
\bibitem{14-kudr}
\Aue{Gorenlo, R., A.\,A.~Kilbas, F.~Mainardi, and S.\,V.~Rogosin.}
 2014. \textit{Mittag-Leffler functions, related topics and applications.}
  Berlin--Heidelberg: Springer-Verlag. 443~p.
\end{thebibliography}

 }
 }

\end{multicols}

\vspace*{-12pt}

\hfill{\small\textit{Received July 4, 2020}}

\pagebreak

%\vspace*{-24pt}

 

\Contr


\noindent
\textbf{Kudryavtsev Alexey A.} (b.\ 1978)~--- 
Candidate of Science (PhD) in physics and mathematics, associate professor, 
Department of Mathematical Statistics, Faculty of Computational Mathematics 
and Cybernetics, M.\,V.~Lomonosov Moscow State University, 
1-52~Leninskiye Gory, GSP-1, Moscow 119991, Russian Federation; 
\mbox{nubigena@mail.ru}

\vspace*{6pt}

\noindent
\textbf{Shestakov Oleg V.} (b.\ 1976)~--- 
Doctor of Science in physics and mathematics, professor, 
Department of Mathematical Statistics, Faculty of Computational Mathematics 
and Cybernetics, M.\,V.~Lomonosov Moscow State University, 1-52~Leninskiye 
Gory, GSP-1, Moscow 119991, Russian Federation; senior scientist, 
Institute of Informatics Problems, Federal Research Center 
``Computer Science and Control'' of the Russian Academy of Sciences, 
44-2~Vavilov Str., Moscow 119333, Russian Federation; \mbox{oshestakov@cs.msu.su}
\label{end\stat}

\renewcommand{\bibname}{\protect\rm Литература} 

\renewcommand{\figurename}{\protect\bf Рис.}