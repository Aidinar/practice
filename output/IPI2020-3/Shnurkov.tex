\def\stat{shnurkov}

\def\tit{О КОНЦЕПЦИИ СТОХАСТИЧЕСКОЙ МОДЕЛИ С~УПРАВЛЕНИЕМ В~МОМЕНТЫ ВЫХОДА 
ПРОЦЕССА НА~ГРАНИЦУ ЗАДАННОГО ПОДМНОЖЕСТВА МНОЖЕСТВА СОСТОЯНИЙ}

\def\titkol{О концепции стохастической модели с~управлением в~моменты выхода 
процесса на~границу заданного подмножества} % множества состояний}

\def\aut{П.\,В.~Шнурков$^1$, Д.\,А.~Новиков$^2$}

\def\autkol{П.\,В.~Шнурков, Д.\,А.~Новиков}

\titel{\tit}{\aut}{\autkol}{\titkol}

\index{Шнурков П.\,В.}
\index{Новиков Д.\,А.}
\index{Shnurkov P.\,V.}
\index{Novikov D.\,A.}
 

%{\renewcommand{\thefootnote}{\fnsymbol{footnote}} \footnotetext[1]
%{Работа выполнена при финансовой поддержке РФФИ (проект 19-07-00352);
%исследования проводились в~рамках программы Московского Цент\-ра 
%фундаментальной и~прикладной математики.}}


\renewcommand{\thefootnote}{\arabic{footnote}}
\footnotetext[1]{Национальный исследовательский университет 
<<Высшая школа экономики>>, \mbox{pshnurkov@hse.ru}}
\footnotetext[2]{Национальный исследовательский университет <<Высшая школа 
экономики>>, \mbox{even.he@yandex.ru}}

\vspace*{-14pt}


\Abst{Работа посвящена созданию и~анализу общей концепции 
специальной стохастической модели с~управлениями. Основная особенность модели 
заключается в~том, что управляющие воздействия осуществляются в~моменты времени, 
когда случайный процесс, описывающий исследуемую систему, достигает границы 
некоторого заданного подмножества множества состояний. Само управляющее 
воздействие заключается в~переводе процесса из граничного в~одно из внутренних 
состояний заданного подмножества. При этом внутренние состояния интерпретируются 
как допустимые, а граничные~--- как недопустимые. Управляющие воздействия 
описываются набором дискретных вероятностных распределений, зависящих от номера 
граничного состояния. Такой набор определяет стратегию управления. Проблема 
оптимального управления формализуется как задача нахождения стратегии 
управления, доставляющей глобальный экстремум некоторому стационарному 
стоимостному показателю эффективности, который по своему экономическому 
содержанию представляет собой среднюю удельную прибыль, возникающую при 
длительной эволюции системы. Поставленную проблему оптимального управления 
предлагается называть задачей о~настройке.
Отмечается, что данная стохастическая модель и~соответствующая задача 
о~настройке могут быть использованы для исследования многих реальных явлений, 
происходящих в~экономических и~технических системах. В~качестве примера такого 
явления рассматривается проведение интервенций на валютном рынке Российской 
Федерации.}

\KW{управление в~стохастических системах; марковские 
управ\-ля\-емые процессы; полумарковские управляемые процессы; стохастическая задача 
о~настройке}


\DOI{10.14357/19922264200315} 
 
\vspace*{-6pt}


\vskip 10pt plus 9pt minus 6pt

\thispagestyle{headings}

\begin{multicols}{2}

\label{st\stat}

\section{Введение}


При анализе многих процессов стохастического характера, происходящих 
в~технических и~экономических системах, наблюдается следующее явление. Основной 
случайный процесс, описывающий исследуемую систему, в~некоторые моменты времени 
выходит из заданного подмножества состояний, которые считаются допустимыми. Для 
того чтобы возвратить процесс в~подмножество допустимых состояний, 
осуществляется внешнее воздействие, которое можно считать управлением 
в~рассматриваемой стохастической модели. В~результате управ\-ля\-юще\-го воздействия 
процесс возвращается в~одно из состояний допустимого множества. При этом 
параметром управления, или решением, служит само состояние (или номер этого 
состояния), в~которое будет переведен процесс в~результате внешнего воздействия.
Предполагается, что состояние, в~которое переводится процесс, определяется как 
результат случайного эксперимента в~соответствии с~некоторым вероятностным 
распределением, которое можно назвать управляющим.
После перевода основного случайного процесса в~одно из допустимых состояний этот 
процесс вновь начинает эволюционировать независимо от прошлого; его поведение 
будет зависеть только от состояния, в~которое он был переведен в~результате 
внешнего воздействия (управления). В~некоторый момент времени основной процесс 
вновь покидает множество допустимых состояний, после чего осуществляется внешнее 
управляющее воздействие, которое организуется по указанным выше правилам. 
Проблема оптимизации управления заключается в~нахождении управляющих 
вероятностных распределений, доставляющих экстремум некоторому показателю 
эффективности управления.

Изложенные выше идеи, основанные на особенностях функционирования реальных 
технических и~экономических систем, образуют общую концепцию стохастической 
модели с~управлениями, осуществляемыми в~моменты выходов основного процесса на 
границу заданного подмножества множества состояний. Создание соответствующей 
математической модели и~решение возникающей при этом задачи оптимального 
управления является актуальной проблемой прикладной математики. В~дальнейшем 
будем называть такую теоретическую проблему задачей о настройке, имея в~виду 
аналогию с~настройкой некоторого важного параметра технической системы.

В качестве примера реальной экономической системы, для описания которой можно 
использовать предлагаемую стохастическую модель с~управ\-ле\-ни\-ями, происходящими 
в~моменты выходов процесса на границу заданного подмножества \mbox{множества} состояний, 
рассмотрим валютный рынок Российской Федерации и~так называемые интервенции, 
проводившиеся на этом рынке Цент\-раль\-ным банком РФ в~течение ряда лет.

\vspace*{-6pt}

\section{Общее описание явления интервенции в~экономике}

\vspace*{-3pt}

Для создания модели необходимо точное описание реального явления. В~связи с~этим начнем исследование с~научного определения понятия интервенции
и некоторых связанных с~ним понятий, принятых в~экономической теории.

<<Интервенция~--- экономическое воздействие одного субъекта на
дела и~действия другого, проводимое посредством проникновения в~сферу
этих действий, вложения и~размещения в~ней собственных денежных средств.


Обычно интервенционные операции проводятся центральными
банками, казначейством путем массовой продажи или скупки валюты, ценных
бумаг, предоставления кредитов в~целях нормализации состояния финансовой
системы>>~\cite{SN2}.

Как следует из данного определения, по своему содержанию интервенции
подразделяются на те, основное действие которых заключается в~продаже
какого-либо товара, валюты или ценных бумаг, и~те, основным действием
которых является закупка таких товаров или ценных бумаг. Эти действия и~составляют сущность явления, называемого интервенцией.

Приведем еще одно определение интервенции.
При проведении закупочной интервенции осуществляется покупка товара, в~случае 
если его цена опускается ниже установленной (интервенционной), чтобы повысить 
рыночную цену.
Товарная интервенция проводится на внутреннем рынке, если внутренняя цена 
превышает интервенционную цену~\cite{SN2.5}.

Понятия закупочной и~товарной интервенций, проводимых на рынке продукции
сельского хозяйства Российской Федерации, определены не только как экономические 
категории, но также и~законодательно. Приведем описание указанных понятий, 
следуя
опубликованному тексту статьи~14 Федерального закона
<<О~развитии сельского хозяйства>>:

<<1.\ Государственные закупочные интервенции, товарные интервенции
проводятся в~целях стабилизации цен на рынке сельскохозяйственной
продукции, сырья и~продовольствия и~поддержания уровня доходов
сельскохозяйственных товаропроизводителей.

2.\ Государственные закупочные интервенции проводятся при
снижении цен на реализуемую сельскохозяйственную продукцию ниже
минимальных расчетных цен путем закупки, в~том числе на биржевых торгах,
у сельскохозяйственных товаропроизводителей произведенной ими
сельскохозяйственной продукции или путем проведения залоговых операций в~отношении данной продукции.

3.\ Государственные товарные интервенции проводятся при росте
цен на реализуемую сельско\-хозяйственную продукцию свыше максимальных\linebreak
расчетных цен путем продажи закупленной сельскохозяйственной продукции,
в том числе на биржевых торгах>> \cite{SN3}.


Итак, в~данном разделе на основе авторитетных источников определена сущность 
экономического явления,
называемого интервенцией. Эта сущность заключается в~целенаправленном
воздействии на рынок ка\-ко\-го-ли\-бо товара, валюты или ценных бумаг в~форме
их продажи или закупки. Проведение закупочной интервенции связывается с~моментом времени, когда рыночная цена на некоторый продукт опускается
ниже заданного минимального уровня. Товарные интервенции должны
осуществляться в~моменты времени, когда рыночная цена продукта становится выше 
заданного максимального уровня. Отметим, что указанные
признаки, т.\,е.\ достижение ценой минимального и~максимального
допустимого уровня, считаются основными причинами проведения
интервенций. Как следует из приведенных выше описаний, цель проведения
интервенции состоит в~достижении такого состояния рассматриваемой
экономической системы (товарного или валютного рынка), в~котором значение 
некоторого основного параметра, а~именно: цены соответствующего товара или 
валюты, находилось бы в~заданных допустимых пределах. Математическая модель, 
описывающая интервенции,
должна прежде всего отражать указанные основные особенности этого
реального явления.


\begin{figure*}[b] %fig1
\vspace*{-6pt}
 \begin{center}
 \mbox{%
 \epsfxsize=151.139mm 
 \epsfbox{shn-1.eps}
 }
 \end{center}
   \vspace*{-9pt}
\Caption{Интервенции Центрального банка на внутреннем валютном рынке~(\textit{1})
 и~динамика 
рублевой стоимости бивалютной корзины
(\textit{2}~--- стоимость
бивалютной корзины, рассчитанная по официальным курсам рублей;
\textit{3} и~\textit{4}~--- нижняя и~верхняя границы операционного интервала): 
(\textit{а})~2010--2011~гг.; (\textit{б})~2011--2012~гг. 
}
\label{fig1}
\end{figure*}
\vspace*{-6pt}

\section{Примеры интервенций на~валютном рынке Российской~Федерации}

\vspace*{-3pt}

Приведем конкретные примеры закупочных и~товарных интервенций, проводившихся 
Цент\-раль\-ным банком на валютном рынке Российской Федерации в~течение 2010--2012~гг. 
Основной случайный процесс, управление которым осуществлялось при помощи 
интервенций, представлял собой стоимость так называемой бивалютной корзины, 
включающей в~себя доллар и~евро в~определенной пропорции.

В 2011~г.\ политика Центрального банка была направлена на сглаживание курса 
рубля. В~начале года проводились закупочные интервенции для замедления снижения 
стоимости бивалютной корзины.

С августа по сентябрь наблюдался отток частного капитала из России, в~результате 
чего спрос на иностранную валюту на внутреннем валютном рынке значительно возрос и~превысил предложение. В~ответ на сложившуюся ситуацию Центральный банк принял 
меры в~виде товарных интервенций. Таким образом, путем продажи иностранной 
валюты Центральный банк обеспечил снижение стоимости бивалютной корзины для 
стабилизации ситуации на внутреннем валютном рынке.




В январе--апреле 2012~г.\ на внутреннем валютном рынке наблюдалось устойчивое 
превышение предложения иностранной валюты над спросом на нее, что создавало 
предпосылки для укрепления рубля и~обусловило проведение Центральным банком 
операций по покупке иностранной валюты на внутреннем валютном рынке.


В мае--июне 2012~г.\ на мировых валютных рынках наблюдалось увеличение спроса 
на доллар США, сопровождавшееся ослаблением российского рубля. В~этих условиях 
в~рамках действующего механизма курсовой политики Центральным банком в~конце мая 
осуществлялась продажа иностранной валюты на внутреннем валютном рынке, что 
позволило сгладить колебания курса рубля.

Иллюстрации, демонстрирующие эволюцию стоимости бивалютной корзины и~объемы 
интервенций Центрального банка в~течение 2010--2012~гг., приведены на рис.~1.

\vspace*{-6pt}

\section{Конструктивное описание стохастической управляемой модели}

\vspace*{-3pt}


Перейдем к~построению стохастической модели с~управлениями, происходящими 
в~моменты выходов процесса на границу заданного подмножества множества состояний. 
В данном разделе будет приведено конструктивное описание эволюции   
соответствующих моделей с~дискретным и~непрерывным временем. Для сохранения 
последовательности изложения будем предполагать, что рас\-смат\-ри\-ва\-емая система 
имеет экономический характер, хотя, как уже отмечалось во введении, аналогичные 
явления имеют место и~в технических системах.

В рамках данного исследования под экономической системой будем понимать 
ка\-кой-ли\-бо товарный или финансовый рынок, на котором объективно действуют случайные 
факторы. На этом рынке формируется некоторый основной параметр, изменяющийся во 
времени. Этот параметр представляет собой случайный процесс, который будет\linebreak 
считаться математической моделью функ\-цио\-ни\-рования исследуемой системы. Состояние 
этого процесса в~произвольный момент времени будет характеризовать состояние 
системы.
В~системе товарного рынка (например,  рынка зерновых культур) таким параметром 
выступает текущая цена единицы соответствующего товара.
В~системе финансового рынка таким параметром может служить текущая стоимость 
ка\-кой-ли\-бо иностранной валюты. На финансовых рынках России используется понятие 
бивалютной корзины, в~которую входят доллар и~евро. Для таких рынков параметром 
состояния можно считать цену этой корзины.

Эволюция этого процесса в~целом
состоит из двух основных этапов, последовательно сме\-ня\-ющих 
друг друга и~обра\-зу\-ющих повторяющийся цикл. На первом этапе эволюция проходит без
внеш\-не\-го воздействия, по внутренним закономерностям данной рыночной
системы. В~множестве возможных значений процесса выделяется заданное
подмножество допустимых значений. В~период пребывания процесса в~этом
допустимом подмножестве внешние воздействия не используются. В~моменты,
когда процесс (основной параметр) выходит на границу или за пределы
до\-пус\-ти\-мо\-го подмножества значений, осуществляется внешнее воздействие,
которое и~пред\-став\-ля\-ет собой интервенцию. Проведение такого воздействия образует 
второй этап эволюции процесса. Цель этого воздействия~--- возвращение значения 
процесса в~одно из состояний допустимого
подмножества. Такое внешнее воздействие является управлением в~данной
вероятностной модели.
Математическая проблема оптимального управления заключается в~нахождении такого 
вероятностного распределения управляющего воздействия, которое доставляет 
глобальный экстремум некоторому стоимостному показателю эффективности 
функционирования данной экономической системы.

Введем важное предположение, связанное с~осо\-бен\-но\-стя\-ми предлагаемой 
стохастической\linebreak модели. Математическое содержание этого предположения 
заключается в~том, что основной случайный процесс, описывающий 
функционирование исследуемой 
системы, обладает марковским свойством в~моменты достижения граничных состояний, 
а~также в~моменты перехода из граничных состояний во внутренние. 
В~соответствии с~идеей марковского свойства данное предположение означает, 
что начиная 
с~каждого такого момента времени процесс продолжает эволюционировать независимо от 
прошлого, а его вероятностные характеристики зависят только от состояния 
в~указанный момент. Заметим, что такое предположение выполняется для многих 
технических и~экономических систем.

Иллюстрацией приведенного описания эволюции модели служит рис.~2,
на котором изображена некоторая траектория основного процесса и~возможные
 управ\-ля\-ющие воздействия.
 
 Перейдем к~непосредственному описанию стохастических процессов, характеризующих 
упо\-мя-\linebreak\vspace*{-12pt}
 
 { \begin{center}  %fig2
 \vspace*{18pt}
     \mbox{%
 \epsfxsize=77.802mm 
 \epsfbox{shn-3.eps}
 }

\end{center}

\vspace*{-3pt}

\noindent
{{\figurename~2}\ \ \small{
Возможная траектория случайного процесса 
$\left\{ {\widehat{\xi}}_{k}\right\}$, представляющего собой 
стохастическую модель поведения основного параметра
}}}

%\vspace*{6pt}





\noindent
ну\-тую экономическую систему. Однако пред\-ва\-ри\-тель\-но примем определенные 
соглашения об обозначениях состояний.

Предположим, что множество состояний рассматриваемого основного процесса 
конечно:  $X=\lbrace0,1,2,\ldots ,N\rbrace $. Для удобства применения 
ре\-зуль\-татов теории поглощающих цепей Маркова~\cite{SN16}\linebreak проведем нумерацию 
состояний таким образом, что состояния $\{0\}$ и~$\{1\}$ будут граничными, 
недопустимыми и~поглощающими, а остальные состояния $\lbrace 2,3,\ldots 
,N\rbrace$~--- внутренними, допустимыми и~невозвратными.

Начнем с~описания стохастической модели с~дискретным временем.
Предположим, что задана последовательность независимых поглощающих марковских 
цепей
$$
\xi^{(n)}=\left\{\xi_{k}^{(n)}\right\}_{k=0}^{\infty},\enskip n=0,1,2,\dots,
$$
с множеством состояний $X$ и~одинаковыми матрицами переходных вероятностей.
В~начальный момент времени $t_{0}=0$ процесс стартует из некоторого допустимого 
состояния
$l_0\hm\in\lbrace 2,3,\ldots ,N\rbrace $. Эволюция процесса описывается
поглощающей марковской цепью $\vphantom{\int\limits_{m}}\left\{ \xi _{k}^{(0)}\right\}^{\infty}_{k=0}$, 
в~которой граничные состояния являются поглощающими, а~внут\-рен\-ние
до\-пус\-ти\-мые состояния~--- невозвратными. Как известно из тео\-рии марковских
процессов, в~некоторый конечный момент времени~$k_{0}$
процесс попадает в~одно из по\-гло\-ща\-ющих со\-сто\-яний. В~примере, который 
иллюстрируется рис.~2, $\xi_{k_0}^{(0)}\hm=1$. 
После этого проводится внешнее воздействие
(управ\-ле\-ние), в~результате которого процесс переводится в~некоторое
внутреннее допустимое состояние $l_1\hm\in\lbrace 2,3,\ldots ,N\rbrace $ с~вероятностью $\alpha _{l_1}^{(1)}$; 
$\sum\nolimits_{l=2}^{N}{\alpha_{l}^{(1)}}\hm=1$. 
Аналогичное внешнее воздействие, осуществ\-ля\-емое при
поглощении в~со\-сто\-янии~0, т.\,е.\ когда $\xi _{k_0}^{(0)}\hm=0$,
описывается дискретным вероятностным распределением $\left( \alpha_{l}^{(0)}, 
l\hm=2,3,\ldots ,N\right) $. После такого воздействия
процесс независимо от прошлого начинает эволюционировать из состояния $
l_1$ по траектории новой поглощающей марковской цепи $\left\{ \xi_{k}^{(1)}
\right\}_{k=0}^{\infty}$, вероятностные характеристики которой 
совпадают с~характеристиками цепи $\left\{ \xi _{k}^{(0)}\right\}_{k=0}^{\infty}$. 
В~момент
достижения одного из граничных погло\-щающих состояний вновь проводится
внешнее управляющее воздействие, которое описывается теми же дискретными
распределениями ве\-ро\-ят\-ностей 
\begin{align*}
\alpha^{(0)} &= \left( \alpha_{l}^{(0)},l\hm=
2,3,\ldots
,N\right);\\
\alpha^{(1)}&=\left( \alpha _{l}^{(1)},
 l\hm=2,3,\ldots ,N\right). 
 \end{align*}
 
Дальнейшая эволюция процесса осуществляется аналогично.

Перейдем к~модели с~непрерывным временем. Возьмем за основу этой модели 
полумарковский случайный процесс с~дискретным множеством состояний. Общая теория 
таких процессов изложена в~работе~\cite{KorolukTurbin}. Из современных изданий 
укажем на работу~\cite{JanssenManca}.

Модели управляемых полумарковских процессов подробно рассмотрены в~классических 
работах~\cite{Jewell, Mine}.

Пусть $\xi^{(n)}(t)$, $n\hm=0,1,2,\ldots,$~--- последовательность независимых 
полумарковских процессов с~поглощением и~с~одинаковыми вероятностными 
характеристиками. Обозначим через 
$$
X=\{0,1,\ldots ,N\},\enskip N\hm<\infty\,, 
$$
множество состояний данных процессов.

В дальнейшем будем предполагать, что для полумарковских процессов 
$\xi^{(n)}(t)$, $n\hm=0,1,2,\dots$, за-\linebreak даны необходимые исходные вероятностные 
характеристики. Различные формы задания таких\linebreak характеристик приведены 
в~\cite{KorolukTurbin}. В~част\-ности, могут быть заданы полумарковские функции 
$Q_{ij}(t),~i,j\hm\in X$, которые представляют собой совместные распределения 
переходов вложенных цепей Маркова и~длительностей пребывания процесса 
в~различных состояниях.

Введем также две системы независимых не\-от\-ри\-ца\-тель\-ных случайных величин 
$\{\Delta_n^{(0)},$ $n\hm=0,1,2,\ldots\}$ 
и~$\{\Delta_n^{(1)},$ $n\hm=0,1,2,\ldots\}$, 
распределения которых для каждого фиксированного~$n$ будут зависеть от 
дополнительного условия, связанного с~данной моделью.

Установим, что в~структуре предлагаемой модели полумарковские процессы  
$\xi^{(n)}(t)$, $n\hm=0,1,2,\dots$, будут описывать эволюцию исходной системы 
в~течение периодов времени между управляющими воздействиями. Каждый процесс 
$\xi^{(n)}(t)$ начинает эволюционировать в~одном из допустимых состояний  
$i\hm\in\{2,3,\ldots,N\}$.
Через конечное время после начала эволюции процесс~$\xi^{(n)}(t)$ 
с~вероятностью, равной~$1$, оказывается в~одном 
из граничных поглощающих состояний~$\{0\}$ или~$\{1\}$. 
После поглощения процесса~$\xi^{(n)}(t)$ модель 
подвергается так называемому управляющему воздействию, которое заключается 
в~переходе из граничного состояния в~одно из допустимых (внутренних) состояний. 
Вероятности перехода из граничного состояния~$0$ во внутренние допустимые 
состояния задаются вектором 
$$
\alpha^{(0)}\hm=\left(\alpha_k^{(0)},\ k=2,3,\dots,N\right),\
\sum\nolimits_{k=2}^N\alpha_k^{(0)}=1\,,$$
 вероятности перехода 
из граничного состояния~$1$ во внутренние 
допустимые состояния задаются вектором 
$$
\alpha^{(1)}=\left(\alpha_k^{(1)},\ k=2,3,\dots,N\right),\
\sum\nolimits_{k=2}^N\alpha_k^{(1)}\hm=1\,.
$$
 Распределения 
вероятностей~$\alpha^{(0)}$ и~$\alpha^{(1)}$ 
описывают внешние управляющие воздействия. Время данного перехода является 
случайной величиной~$\Delta_n^{(0)}$, если переход происходит из 
состояния~$\{0\}$, и~случайной величиной~$\Delta_n^{(1)}$, 
если переход осуществляется из 
состояния~$\{1\}$. При этом распределение случайной величины $\Delta_n^{(0)}$ 
или~$\Delta_n^{(1)}$ может, вообще говоря, зависеть от номера того допустимого 
состояния $k \hm\in \{2,3,\ldots,N\}$, в~которое осуществляется переход. После 
перехода в~допустимое состояние $k \hm\in \{2,3,\ldots,N\}$ эволюция системы будет 
проходить независимо от прошлого (при фиксированном начальном состоянии~$k$) 
и~описываться полумарковским процессом~$\xi^{(n+1)}(t)$, вероятностные 
характеристики которого совпадают с~соответствующими характеристиками процесса~$\xi^{(n)}(t)$.

Задача оптимального управления в~обеих версиях описанной модели будет формально 
ставиться как экстремальная задача на множестве пар дискретных вероятностных 
распределений $\left(\alpha^{(0)}, \alpha^{(1)}\right)$ по отношению 
к~некоторому стационарному стоимостному показателю эффективности, имеющему смысл 
средней удельной прибыли.

Можно доказать~\cite{SN1, SN29}, что по своей аналитической форме указанный 
показатель эффективности представляет собой дробно-линейный интегральный 
функционал, заданный на множестве пар дискретных вероятностных распределений 
$\left(\alpha^{(0)}, \alpha^{(1)}\right)$. Задача безусловного экстремума таких 
функционалов решена в~работах~\cite{SN26, qwe2}.
На основании указанных результатов можно утверждать, что решения поставленных 
задач оптимального управления достигаются на вырожденных вероятностных 
распределениях. Таким образом, решения о~проведении управ\-ля\-ющих воздействий 
должны приниматься детерминированно. Сами оптимальные решения определяются точкой максимума 
заданной функции двух дискретных переменных.

\vspace*{-6pt}

\section{Основные особенности стохастической управляемой модели}

\vspace*{-3pt}

Выделим несколько ключевых особенностей построенной стохастической управляемой 
модели.
\begin{enumerate}
\item Наличие в~множестве состояний некоторого заданного подмножества так 
называемых до\-пус\-ти\-мых состояний и~конечного подмножества состояний, которые 
можно условно называть внешними, или граничными по отношению к~допустимым. При 
этом множество до\-пус\-ти\-мых состояний необязательно должно быть дискретным
\item Выполнение некоторого общего условия, связанного с~поведением исходного 
случайного процесса~$\xi^{(n)}(t)$ (непрерывное время) или~$\xi^{(n)}$ 
(дискретное время). Сущность этого условия состоит в~том, что данный случайный 
процесс, начиная эволюцию из произвольного допустимого состояния, должен за 
конечное время достигать одного из внешних (граничных) состояний с~вероятностью, 
равной единице.
\item Выполнение марковского свойства в~моменты перехода во внешние (граничные) 
состояния, а~также в~моменты выхода из внешних состояний и~перехода в~одно из 
состояний допустимого подмножества.
\item Наличие возможности аналитического определения вероятностных 
характеристик, описывающих переход из произвольного фиксированного допустимого 
состояния в~одно из внешних состяний (аналог вероятностей поглощения).
\item Наличие возможности аналитического определения математического ожидания 
времени, прошедшего от момента выхода из произвольного фиксированного 
допустимого состояния до попадания в~одно из внешних состояний, 
и~математического ожидания дохода, накопленного за указанное время.
\end{enumerate}

Заметим, что условия~1--3 из сформулированного набора условий связаны 
с~основными стохастическими свойствами модели. Условия~4 и~5 призваны обеспечить 
возможность нахождения \mbox{аналитических} представлений для необходимых вероятностных 
характеристик основного процесса, связанных со временем его пребывания 
в~множестве допустимых состояний.

Указанные условия можно использовать при построении аналогичных моделей 
с~управ\-ле\-ни\-ями, осуществляемыми в~моменты выхода процесса на границу заданного 
подмножества множества состояний. В~таких моделях могут использоваться иные виды 
основных процессов, описывающих эволюцию системы без управляющего воздействия. 
При этом можно поставить и~решить соответствующие задачи о настройке.

\vspace*{-6pt}

\section{Заключение}

\vspace*{-3pt}

Формулировки теоретических утверждений, опре\-де\-ля\-ющих решения задач управления 
в~описанных выше моделях с~дискретным и~непрерывным временем, приведены 
в~\cite{SN1}. В~работе~\cite{SN29}\linebreak приведен полный анализ задачи управления 
с~дискретным временем, включая доказательства основных утверждений.

\vspace*{-12pt}

{\small\frenchspacing
 {%\baselineskip=10.8pt
 \addcontentsline{toc}{section}{References}
 \begin{thebibliography}{99}
\bibitem{SN2} 
\Au{Лопатников Л.\,И.} Эко\-но\-ми\-ко-ма\-те\-ма\-ти\-че\-ский словарь: 
Словарь современной экономической науки.~--- М.: Дело, 2003. 520~c.
\bibitem{SN2.5} 
OECD~--- Organisation for Economic Co-operation and Development: 
OECD Glossary of Statistical Terms (OECD Glossaries).~--- OECD Publishing, 2008. 
288~p.

\bibitem{SN3} 
О~развитии сельского хозяйства: Федеральный закон от 29.12.2006 
№\,264-ФЗ (ред.\ от 12.02.2015 с~изм. и~доп., вступ. в~силу 
с~13.08.2015). {\sf http://www.\linebreak kremlin.ru/acts/bank/24837}.

\bibitem{SN16} 
\Au{Кемени Дж.\,Дж., Снелл Дж.\,Л.} Конечные цепи Маркова~/ 
Пер. с~англ.; под ред.\ А.\,А.~Юшкевича.~---  М.: Наука, 1970. 272~с.
(\Au{Kemeny~J.G., Snell~J.\,L.} 
Finite Markov chains.~--- London: D.~van Nostrand Co. Ltd., 1960. 210~p.)



\bibitem{KorolukTurbin} 
\Au{Королюк В.\,С., Турбин А.\,Ф.} 
Полумарковские 
процессы и~их приложения.~--- Киев: Наукова думка, 1976. 184~с.

\bibitem{JanssenManca}  %6
\Au{Janssen J., Manca~R.} 
Applied semi-Markov processes.~--- New York, NY, USA: Springer, 2006. 309~p.

\bibitem{Jewell} %7
\Au{Джевелл В.} 
Управляемые полумарковские процессы~// 
Кибернетич. сборник. Нов. серия.~--- М.: Мир, 1967. 
 Вып.~4. С.~97--134.
 (\Au{Jewell~W.\,S.} Markov-renewal programming. 
I, II~//  {Oper. Res.}, 1963. Vol.~11. No.\,6. P.~938--971.)

\bibitem{Mine}  %8
\Au{Майн Х., Осаки~С.} 
Марковские процессы принятия решений~/
Пер. с~англ.~--- М.: Наука, 1977. 176~с.
(\Au{Mine~H., Osaki~S.}
 {Markovian decision processes}.~--- New York, NY, USA: Elsevier,
 1970. 142~p.)

\bibitem{SN1} 
\Au{Shnurkov P.\,V.}
 Optimal control problem in a stochastic 
model with periodic hits on the boundary of a~given subset of the state set 
(tuning problem). arXiv.org, 2017. arXiv: 1709.03442v1. 16~p.
\bibitem{SN29} 
\Au{Shnurkov P.\,V., Novikov D.\,A.} Analysis of the problem 
of intervention control in the economy on the basis of solving the problem of 
tuning. arXiv.org, 2018. arXiv: 1811.10993. 15~p.    
\bibitem{SN26} %11
\Au{Шнурков П.\,В.} 
О~решении задачи безусловного экстремума 
для дроб\-но-ли\-ней\-но\-го интегрального функционала на 
множестве вероятностных мер~//
Докл. Акад. наук, 2016. Т.~470. №\,4. С.~387--392.
\bibitem{qwe2} \Au{Шнурков П.\,В., Горшенин~А.\,К., Белоусов~В.\,В.} 
Аналитическое решение задачи оптимального управ\-ле\-ния полумарковским процессом 
с~конечным множеством со\-сто\-яний~// Информатика и~её применения, 2016. Т.~10. 
Вып.~4. С.~72--88.
\end{thebibliography}

 }
 }

\end{multicols}

\vspace*{-9pt}

\hfill{\small\textit{Поступила в~редакцию 15.07.20}}

\vspace*{8pt}

%\pagebreak

%\newpage

%\vspace*{-28pt}

\hrule

\vspace*{2pt}

\hrule

\vspace*{-2pt}

\def\tit{ON THE CONCEPT OF~A~STOCHASTIC MODEL WITH~CONTROL AT~THE~MOMENTS 
OF~THE~PROCESS AT~THE~BORDER OF~A~PRESENTED SUBSET OF~MULTIPLE STATES}


\def\titkol{On the concept of~a~stochastic model with~control at~the~moments 
of~the~process at~the~border of~a~presented subset of~multiple states}

\def\aut{P.\,V.~Shnurkov and D.\,A.~Novikov}

\def\autkol{P.\,V.~Shnurkov and D.\,A.~Novikov}

\titel{\tit}{\aut}{\autkol}{\titkol}

\vspace*{-9pt}


\noindent
National Research University Higher School of Economics, 
34~Tallinskaya Str., Moscow 123458, Russian Federation

\def\leftfootline{\small{\textbf{\thepage}
\hfill INFORMATIKA I EE PRIMENENIYA~--- INFORMATICS AND
APPLICATIONS\ \ \ 2020\ \ \ volume~14\ \ \ issue\ 3}
}%
 \def\rightfootline{\small{INFORMATIKA I EE PRIMENENIYA~---
INFORMATICS AND APPLICATIONS\ \ \ 2020\ \ \ volume~14\ \ \ issue\ 3
\hfill \textbf{\thepage}}}

\vspace*{3pt} 

\Abste{The work is devoted to the creation and analysis of the general 
concept of a special stochastic model with controls. The main feature of 
the model is that the control actions are carried out at times when 
a~stochastic process describing the system under research reaches the 
boundary of a given subset of the set of states. The control action 
itself consists in transferring the process from the boundary to one 
of the internal states of a given subset. In this case, the internal 
states are interpreted as acceptable and the boundary ones as unacceptable. 
Control actions are described by a set of discrete probability distributions 
depending on the boundary state number. Such a set defines a control strategy. 
The problem of optimal control is formalized as the problem of finding 
a~control strategy that delivers a global extremum to a certain stationary 
cost-effectiveness indicator, which in terms of its economic content 
represents the average specific profit arising from a long evolution 
of the system. The posed problem of optimal control is proposed to be 
alled the tuning problem. The paper notes that this stochastic model and 
the corresponding setup problem can be used to study many real phenomena 
occurring in economic and technical systems. As an example of such 
a~real phenomenon, interventions in the foreign exchange market of the 
Russian Federation are considered.}

\KWE{control in stochastic systems; Markov controlled processes; 
semi-Markov controlled processes; stochastic tuning problem}

\DOI{10.14357/19922264200315} 

%\vspace*{-20pt}

%\Ack
%\noindent


%\vspace*{6pt}

 \begin{multicols}{2}

\renewcommand{\bibname}{\protect\rmfamily References}
%\renewcommand{\bibname}{\large\protect\rm References}

{\small\frenchspacing
 {%\baselineskip=10.8pt
 \addcontentsline{toc}{section}{References}
 \begin{thebibliography}{99}
\bibitem{1-shn}
\Aue{Lopatnikov, L.\,I.}
 2003.  \textit{Ekonomiko-matematicheskiy slovar': slovar' 
 sovremennoy ekonomicheskoy nauki} [Economic and mathematical dictionary: 
 Modern economic science dictionary]. Moscow: Delo. 520~p.
\bibitem{2-shn}
OECD~--- Organisation for Economic Co-operation and Development. 2008.
OECD glossary of statistical terms. OECD Publishing. 288~p. Available at: 
{\sf https://stats.oecd.org/glossary/} (accessed July~30, 2020).
\bibitem{3-shn}
264-FZ. 2015. O~razvitii sel'skogo khozyaystva: 
fe\-de\-ral'\-nyy zakon [About the development of agriculture: The 
Russian Federal Law]. Available at:
{\sf http://www.kremlin.\linebreak ru/acts/bank/24837} (accessed July~30, 2020).
\bibitem{4-shn}
\Aue{Kemeny, J., and J.~Snell.}
 1976.  \textit{Finite Markov chains}. Prentice-Hall. 232~p.
\bibitem{5-shn}
\Aue{Korolyuk, V.\,S., and A.\,F.~Turbin.}
 1976.  \textit{Polumarkovskie protsessy i~ikh prilozheniya} [Semi-Markov 
 processes and their applications]. Kiev: Naukova dumka. 184~p.
\bibitem{6-shn}
\Aue{Janssen, J., and R.~Manca.} 2006. 
 \textit{Applied semi-Markov processes.} New York, NY: Springer. 309~p.
\bibitem{7-shn}
\Aue{Jewell, W.\,S.} 1963. Markov-renewal programming. 
I, II.  \textit{Oper. Res.} 11(6):938--971.
\bibitem{8-shn}
\Aue{Mine, H., and S.~Osaki.}
 1970.  \textit{Markovian decision processes}. New York, NY: Elsevier. 142~p.
\bibitem{9-shn}
\Aue{Shnurkov, P.\,V.} 2017. 
Optimal control problem in a~stochastic model with periodic hits 
on the boundary of a~given subset of the state set (tuning problem). 
 {arXiv.org}. 16~p. Available at: 
 {\sf https://arxiv.org/abs/1709.03442v1} (accessed July~30, 2020).
\bibitem{10-shn}
\Aue{Shnurkov, P.\,V., and D.\,A.~Novikov.} 2018. 
Analysis of the problem of intervention control in the economy on the 
basis of solving the problem of tuning. 
{arXiv.org}. 15~p. Available at: 
 {\sf https://arxiv.org/abs/1811.10993} (accessed July~30, 2020).
\bibitem{11-shn}
\Aue{Shnurkov, P.\,V.}
 2016. Solution of the unconditional extremum problem for 
 a~linear fractional integral functional on a set of probability measures. 
  \textit{Dokl. Math.} 94(2):550--554.
\bibitem{12-shn}
\Aue{Shnurkov, P.\,V., A.\,K.~Gorshenin, and V.\,V.~Belousov.}
 2016. Analiticheskoe reshenie zadachi optimal'nogo upravleniya 
 polumarkovskim protsessom s~konechnym mnozhestvom sostoyaniy 
 [An analytic solution of the optimal control problem for 
 a~semi-Markov process with a~finite set of states]. 
 \textit{Informatika i~ee Primeneniya~--- Inform. Appl.} 10(4):72--88.
 \end{thebibliography}

 }
 }

\end{multicols}

\vspace*{-6pt}

\hfill{\small\textit{Received July 15, 2020}}

%\pagebreak

%\vspace*{-24pt}

\Contr

\noindent
\textbf{Shnurkov Peter V.} (b.\ 1953)~--- 
Candidate of Science (PhD) in physics and mathematics, associate professor, 
National Research University Higher School of Economics, 34~Tallinskaya Str., 
Moscow 123458, Russian Federation; \mbox{pshnurkov@hse.ru}

\vspace*{3pt}

\noindent
\textbf{Novikov Daniil A.} (b.\ 1993)~--- 
PhD student, National Research University Higher School of Economics, 
34~Tallinskaya Str., Moscow 123458, Russian Federation; \mbox{even.he@yandex.ru}


\label{end\stat}

\renewcommand{\bibname}{\protect\rm Литература} 

\renewcommand{\figurename}{\protect\bf Рис.}