\def\stat{popov}

\def\tit{АППРОКСИМАЦИЯ МНОЖЕСТВА РЕШЕНИЙ СИСТЕМ НЕЛИНЕЙНЫХ НЕРАВЕНСТВ 
С~ИСПОЛЬЗОВАНИЕМ ГРАФИЧЕСКИХ УСКОРИТЕЛЕЙ$^*$}

\def\titkol{Аппроксимация множества решений систем нелинейных неравенств 
с~использованием графических ускорителей}

\def\aut{М.\,В.~Попов$^1$, М.\,А.~Посыпкин$^2$}

\def\autkol{М.\,В.~Попов, М.\,А.~Посыпкин}

\titel{\tit}{\aut}{\autkol}{\titkol}

\index{Попов М.\,В.}
\index{Посыпкин М.\,А.}
\index{Popov M.\,V.}
\index{Posypkin M.\,A.}
 

{\renewcommand{\thefootnote}{\fnsymbol{footnote}} \footnotetext[1]
{Работа выполнена при частичной поддержке РНФ (проект 16-19-00148).}}


\renewcommand{\thefootnote}{\arabic{footnote}}
\footnotetext[1]{Федеральный исследовательский центр <<Информатика и~управление>> Российской академии наук, 
\mbox{alvopim@gmail.com}}
\footnotetext[2]{Федеральный исследовательский центр <<Информатика и~управление>> Российской академии наук, 
\mbox{mposypkin@gmail.com}}

\vspace*{-6pt}

\Abst{Существует множество задач, сводящихся к~решению системы неравенств. Точное 
получение множества решений в~подобных задачах не всегда возможно, из-за чего 
прибегают к~различным методам аппроксимации данного множества. При повышении 
точности аппроксимации искомого множества увеличивается объем необходимых 
вычислений и, соответственно, время работы алгоритмов. В~работе для увеличения 
быстродействия алгоритмов поиска аппроксимируемого множества применяются 
параллельные вычисления на графических ускорителях. Приводится описание и~реализация 
последовательного метода аппроксимации системы неравенств и~предлагается параллельный 
гибридный алгоритм, сочетающий перебор на равномерной сетке и~идеи метода ветвей 
и~границ. Этот алгоритм хорошо подходит для реализации на графических ускорителях и~не 
приводит к~избыточному перебору. Приведено сравнение эффективности работы 
последовательного и~двух вариантов параллельного алгоритмов на примере прикладной 
задачи аппроксимации рабочей области робота. Рабочая область состоит из множества 
возможных положений инструмента и~служит одной из ключевых характеристик робота.}

\KW{оптимизация; параллельные вычисления; графический ускоритель, GPU; CUDA; 
нелинейные неравенства}

\DOI{10.14357/19922264200303} 
 
%\vspace*{-6pt}


\vskip 10pt plus 9pt minus 6pt

\thispagestyle{headings}

\begin{multicols}{2}

\label{st\stat}

\section{Введение}

     В статье рассматривается задача построения внутренней и~внешней 
аппроксимаций множества, задаваемого системой нелинейных неравенств. Для 
решения этой задачи разработаны подходы~[1], основной недостаток которых 
заключается в~высокой трудоемкости. В~данной работе показано, что 
упомянутая проблема может быть преодолена с~помощью методов 
высокопроизводительных вычислений. 
     
     В качестве платформы для высокопроизводительных вычислений 
рассматриваются современные графические ускорители (GPU, graphics processing units)~[2]. 
Данный тип 
устройств получил широкое распространение в~последнее время благодаря 
наилучшему соотношению це\-на/про\-из\-во\-ди\-тель\-ность. В~то же время 
разработка программ для этой платформы является нетривиальной задачей 
в~силу особенностей ее архитектуры. Современные графические ускорители 
обладают большим числом ядер, но возможности взаимодействия между 
потоками существенно ограничены по отношению к~центральному про\-цес\-сору.
      
     В работе предлагается гибридный алгоритм, сочетающий перебор на 
равномерной сетке и~идеи метода ветвей и~границ. Предложенный алгоритм 
хорошо подходит для реализации на графических ускорителях и~не приводит 
к~избыточному пе\-ре\-бору. 
     
\section{Последовательный алгоритм}

     Рассматривается задача нахождения множества решений системы 
неравенств вида:
     \begin{equation}
     \left.
     \begin{array}{rl}
     f_j(x)\leq 0\,, & j\in [1,m]\,;\\[6pt]
     a_i\leq x_i\leq b_i\,, & i\in [1,n]\,.
     \end{array}
     \right\}
     \label{e1.1-pos}
     \end{equation}
      
     Для поиска решения необходимо задать изначальные границы, внутри 
которых будет находиться искомая область~--- $n$-мер\-ный параллелепипед 
$X\hm\in R^n$. Далее задается точность аппроксимации~$d$ получаемого 
решения, представляющая собой минимальный размер аппроксимирующего 
параллелепипеда. Под размером будем понимать длину максимального 
измерения. После этого параллелепипед~${X}$ разбивается на 
меньшие параллелепипеды~$P_k$ равномерной сеткой. Число точек сетки по 
измерению~$j$  задается формулой:
 $$
 k_j= \left\lceil \fr{b_j- a_j}{d}\right\rceil\,.
 $$
 
  Для 
каждого из полученных параллелепипедов проверяется выполнение следующих 
условий:
     \begin{equation}
     \max\limits_{j\in [1,m]} \max\limits_{x\in P_k} f_j(x)<0\,;
     \label{e1.2-pos}
     \end{equation}
\begin{equation}
\max\limits_{j\in [1,m]} \min\limits_{x\in P_k} f_{x\in P_k}(x)>0\,.
\label{e1.3-pos}
\end{equation}

Можно выделить три основных случая:
\begin{enumerate}[(1)]
\item выполнено условие~(\ref{e1.2-pos})~--- параллелепипеды полностью 
лежат внутри искомой области;
\item выполнено условие~(\ref{e1.3-pos})~--- параллелепипед лежит вне 
искомой области;
\item оба условия не выполнены~--- параллелепипед классифицируется как 
<<граничный>>.
\end{enumerate}

     Метод перебора на равномерной сетке удобен тем, что можно заранее 
рассчитать число операций, необходимых для получения аппроксимирующего 
покрытия фигуры, образованной функциями из~(\ref{e1.1-pos}). 
     
     Для последовательного алгоритма перебора на равномерной сетке число 
рассматриваемых параллелепипедов определяется по формуле 
$\prod\nolimits^m_{j=1} k_j$. Для каждого параллелепипеда $P_k\hm\in X$ 
проводится проверка выполнения условий~(\ref{e1.2-pos}) и~(\ref{e1.3-pos}) для 
каждой функции~$f_j$. Время проверки~$t_c$ не отличается для разных 
параллелепипедов. Таким образом, общее время составляет $t_c  
\prod\nolimits^m_{j=1} k_j$. 
      
     Недостаток данного алгоритма состоит в~необходимости полного 
перебора всех параллелепипедов, число которых существенно возрастает 
с~уменьшением значения точности аппроксимации. Для преодоления 
указанного недостатка может применяться метод <<ветвей и~границ>>, 
адаптированный для данной задачи. Отличие заключается в~том, что сетка не 
задается сразу, а параллелепипеды формируются рекурсивно. Приведем 
описание алгоритма.
     \begin{enumerate}[I.]
\item В текущий список помещается $m$-мерный параллелепипед $X\hm\in 
R^m$, который полностью содержит в~себе искомую область.
\item Из текущего списка извлекается параллелепипед и~делится на две части 
по наибольшему из измерений.
\item Выполняется проверка условий~(2) и~(3) для полученных делением 
параллелепипедов и~в~за\-ви\-си\-мости от результата проверки проводится одно 
из следующих действий:
\begin{enumerate}[(1)]
\item если выполнено условие~(\ref{e1.2-pos}), то параллелепипед 
помещается в~список полученных прямоугольников для последующего 
отоб\-ра\-же\-ния;
 \item если выполнено условие~(\ref{e1.3-pos}), то параллелепипед 
далее не рассматривается;
\item если параллелепипед оказывается граничным, но его максимальное 
измерение превосходит величину точности аппрокси\-мации, то он помещается 
в~текущий список и~выполняется переход к~шагу~2. В~противном случае 
параллелепипед добавляется к~множеству граничных параллелепипедов.
 \end{enumerate}
 \end{enumerate}
     
     Данный метод не позволяет заранее рассчитать число получаемых 
параллелепипедов для покрытия искомой области, но, как показывает 
эксперимент, оно будет много меньше, чем при методе равномерной сетки. 
Минус данного метода~--- его рекурсивность, из-за чего данный алгоритм 
сложен для распараллеливания. 

\section{Параллельный алгоритм}

     Реализация метода равномерной сетки для параллельных вычислений на 
графическом ускорителе позволяет сократить число выполняемых операций 
и~время вычислений в~десятки раз. Главное\linebreak отличие параллельного алгоритма 
от последовательного состоит в~том, что обработка параллелепипедов 
выполняется одновременно разными потоками графической карты. 
     
     Алгоритм равномерной сетки (рис.~1) для параллельной аппроксимации 
множества решений системы неравенств состоит в~разбиении параллелепипеда, 
содержащего исследуемую область, на параллелепипеды меньшего размера 
с~их последующей параллельной обработкой. Параллельная\linebreak\vspace*{-12pt}

{ \begin{center}  %fig1
 \vspace*{12pt}
    \mbox{%
 \epsfxsize=79mm 
 \epsfbox{pos-1.eps}
 }

\vspace*{6pt}

\noindent
{{\figurename~1}\ \ \small{
Алгоритм равномерной сетки
}}
\end{center}}

%\vspace*{6pt}

{ \begin{center}  %fig2
 \vspace*{-2pt}
    \mbox{%
 \epsfxsize=79mm 
 \epsfbox{pos-2.eps}
 }

\vspace*{6pt}

\noindent
{{\figurename~2}\ \ \small{
Алгоритм двойной сетки
}}
\end{center}}


\vspace*{6pt}


\noindent
 обработка 
выполняется на графическом ускорителе~--- один поток обрабатывает 
несколько ячеек сетки. При этом координаты ячеек не пересылаются на GPU, 
а~генерируются исходя из номера потока.
     
      

     
     Алгоритм равномерной сетки идеально подходит для реализации на 
графических ускорителях, так как обработка ячеек равномерной сетки может 
выполняться абсолютно независимо. Независимость потоков создает 
предпосылки для массивного\linebreak параллелизма без синхронизации. Недостаток 
данного алгоритма~--- высокая ресурсоемкость: при\linebreak увеличении точности 
аппроксимации количество обрабатываемых параллелепипедов существенно 
возрастает и~превышает пределы графической памяти. 
     
     Преодолеть указанный недостаток позволяет алгоритм двойной сетки 
(рис.~2), сочетающий элементы метода ветвей и~границ и~перебора на 
равномерной сетке. Сначала на центральном процессоре (CPU,
central processing unit) проводится 
предварительная обработка ячеек крупной равномерной сетки. В~результате 
часть таких параллелепипедов классифицируется как внешние или внутренние, 
сохраняется и~далее не рассматривается. Граничные параллелепипеды 
передаются графическому ускорителю по одному, где они обрабатываются так 
же, как исходный параллелепипед в~алгоритме равномерной сетки.
     

     
     Результатом обработки равномерной сетки на GPU служит массив, 
содержащий значения~0, 1, 2 в~зависимости от принадлежности 
параллелепипеда к~искомому множеству аппроксимации. Далее в~зависимости 
от индекса элемента массива для него вычисляются координаты 
параллелепипеда.
     
     Для графического отображения полученного множества используется 
язык Python и~библиотека Matplotlib. При большом числе отображаемых 
параллелепипедов сильно возрастает время вывода изображения 
и~задействуется большой объем оперативной памяти. Для оптимизации 
процесса было принято решение объединять параллелепипеды, лежащие 
внутри искомого множества, в~крупные. Таким образом удалось существенно 
уменьшить затраты времени и~оперативной памяти. 

\section{Реализация и~экспериментальные исследования}

     Предложенные методы были реализованы с~помощью языка~C 
и~технологии CUDA (Compute Unified Device Architecture). 
В~качестве примера прикладного применения 
рассмотренных методов выбрана задача аппроксимации рабочей области 
(множества всех возможных положений рабочего инструмента) плоского 
параллельного робота. Роботы параллельной структуры характеризуются 
замкнутой кинематической цепью, сочетают высокую точность 
позиционирования и~жесткость конструкции~\cite{3-pos}. Для экспериментов 
выбран планарный робот с~двумя степенями свободы (\mbox{2-RPR}). Схематично 
данное устройство представлено на рис.~3. Рабочий инструмент робота, 
закрепленный в~точке~$X$, приводится в~движение двумя призматическими 
двигателями, которые изменяют длины штанг~$AX$ и~$BX$.



Рабочая область данного робота определяется следующей системой неравенств:
\begin{align*}
f_1\left( x_1, x_2\right) &=x_1^2+x_2^2-\left( l_1^{\max}\right)^2\leq0\,;\\
f_2\left(x_1, x_2\right)&= \left( l_1^{\min}\right)^2 -x_1^2-x_2^2\leq0\,;\\
f_3\left( x_1, x_2\right)&= \left(x_1-l_0\right)^2 +x_2^2-\left( 
l_2^{\max}\right)^2\leq0\,;\\
f_4\left(x_1, x_2\right)& = \left( l_2^{\min}\right)^2 -\left( x_1-l_0\right)^2-x_2^2\leq 
0\,,
\end{align*}
где
$$
-l_1^{\max}\leq x_1 \leq l_0+l_2^{\max}\,;\
0\leq x_2\leq \min\left( l_1^{\max}, l_2^{\max}\right)\,.
$$

\setcounter{figure}{3}
\begin{figure*}[b] %fig4
\vspace*{1pt}
 \begin{center}
 \mbox{%
 \epsfxsize=162.252mm 
 \epsfbox{pos-4.eps}
 }
 \end{center}
   \vspace*{-9pt}
\Caption{Искомые множества: (\textit{а})~точ\-ность $d\hm=0{,}1$;
(\textit{б})~точность $d\hm=0{,}01$}

\vspace*{6pt}

{\small
\begin{center}
\begin{tabular}{|c|c|c|c|c|}
\multicolumn{5}{c}{Время выполнения методов в~зависимости от точности аппроксимации 
(мс)}\\
\multicolumn{5}{c}{\ }\\[-6pt]
\hline
\tabcolsep=0pt\begin{tabular}{c}Точность\\ аппроксимации\end{tabular}&
\tabcolsep=0pt\begin{tabular}{c}Последовательный\\ метод равномерной\\ сетки\end{tabular}&
\tabcolsep=0pt\begin{tabular}{c}Метод\\ ветвей\\ 
и~границ\end{tabular}&
\tabcolsep=0pt\begin{tabular}{c}Параллельный\\ метод равномерной\\ сетки
\end{tabular}&
\tabcolsep=0pt\begin{tabular}{c}Параллельный\\ метод двойной\\ сетки\end{tabular}\\
\hline
0,1\hphantom{99}&\hphantom{9}13&\hphantom{99}8&47&54\\
0,05\hphantom{9}&\hphantom{9}41&\hphantom{9}27&47&54\\
0,01\hphantom{9}&945&889&92&54\\
0,005&4\,003\hphantom{9}&3\,950\hphantom{9}&140\hphantom{9}&54\\
0,001&99\,100\hphantom{99}&132\,300\hphantom{999}&510\hphantom{9}&102\hphantom{9}\\
\hphantom{9}0,0001&$>$2,5~ч&$>$2,5~ч&---&4\,400\hphantom{99}\\
\hline
\end{tabular}
\end{center}
}
%\end{table*}
\end{figure*}


{ \begin{center}  %fig3
 \vspace*{-2pt}
    \mbox{%
 \epsfxsize=79mm 
 \epsfbox{pos-3.eps}
 }

\vspace*{6pt}

\noindent
{{\figurename~3}\ \ \small{
Схема плоского робота 2-RPR
}}
\end{center}}


\vspace*{6pt}


\noindent
Значения $l_{1,2}$ изменяются в~диапазоне $8\hm\leq l_{1,2}\hm\leq 12$, т.\,е.\ 
$l_1^{\min}\hm= 8$; $l_1^{\max}\hm=12$. Расстояние между точками 
закрепления штанг составляет $l_0\hm=5$. Требуется аппроксимировать 
область, в~которой, исходя из конструктивных особенностей длин звеньев, 
может находиться точка $X(x_1,x_2)$ закрепления рабочего инструмента. Для 
получения гарантированных оценок минимума и~максимума функций в~левых 
частях неравенств применялись методы интервального  
анализа~\cite{4-pos, 5-pos}.
     
     Реализованные методы тестировались на персональном компьютере 
с~Intel Core~I7 с~базовой тактовой частотой 2,8~ГГц, 16~ГБ ОЗУ, видеокартой 
Nvidia GeForce~1060: 3~ГБ оперативной памяти, 1280~ядер с~базовой тактовой 
частотой от 1506~МГц. Тестирование проводилось для поставленной задачи со 
значениями точности аппроксимации, равными 0,1; 0,05; 0,01; 0,005; 0,001 
и~0,0001.
     
     Аппроксимации рабочей области для разных значений точности, 
полученные методом двойной сетки, приведены на рис.~4.




На рис.~4,\,\textit{б} убраны линии, выделяющие получаемые параллелепипеды, 
поскольку при высокой точности аппроксимации толщина отображаемых 
линий становится соизмерима с~высотой параллелепипедов, из-за чего линии 
сливаются. Таким образом, для точности $d\hm=0{,}01$ высота 
параллелепипедов, которые не объединены друг с~другом, слишком мала для 
графического отображения на целой фигуре. 
     
Время работы алгоритмов в~миллисекундах представлено в~таблице.




Проведенные эксперименты позволяют сделать следующие выводы.
\begin{enumerate}[1.]
\item При низкой точности аппроксимации (0,1 или~0,05) параллельные 
варианты работают с~невысокой эффективностью, что означает 
недостаточную загруженность вычислительных мощностей видеокарты. 
Применение GPU при таком объеме вычислений нецелесообразно.
\item При повышении точности возрастает объем вычислений 
и~загруженность ядер графической карты, что приводит к~увеличению 
эффективности расчетов. С~по\-мощью параллельного алгоритма 
равномерной сетки не удается получать решения при высокой точ\-ности 
аппроксимации в~силу превышения максимально допустимых значений 
числа обрабатываемых потоков или пределов локальной памяти. 
\item Принцип дополнительного разбиения исходной области, заложенный 
в~параллельном алгоритме двойной сетки, позволяет снизить объем 
вычислений и~данных, обрабатываемых на графической карте, благодаря 
чему возможно получение более высокой точности аппроксимации.
\end{enumerate}

\section{Заключение}
     
     В данной работе разработаны два параллельных алгоритма для численной 
аппроксимации решения системы нелинейных неравенств, ориентированные на 
графические ускорители: алгоритм равномерной сетки и~алгоритм двойной 
сетки. Результаты экспериментов показали, что параллельные методы 
с~вычислениями на графическом ускорителе оказались существенно 
эффективнее последовательных однопоточных методов на центральном 
процессоре. Параллельный алгоритм двойной сетки оказался наиболее 
эффективным по скорости работы и~наилучшим по точности аппроксимации. 
Этот метод сочетает в~себе аппроксимацию на равномерной сетке, которая 
хорошо распараллеливается, и~метод ветвей и~границ, позволяющий сократить 
число операций за счет эффективного отсева <<неперспективных>> областей 
на начальном этапе.
     
     В~дальнейшем планируется рассмотреть роботов более сложной 
конструкции~\cite{6-pos}, что неизбежно приведет к~необходимости 
привлечения большей вычислительной мощности для расчетов. Предполагается 
использование серверов, оснащенных несколькими графическими 
ускорителями, установленных в~центре обработки данных ФИЦ ИУ 
РАН~\cite{7-pos}.


{\small\frenchspacing
 {%\baselineskip=10.8pt
 \addcontentsline{toc}{section}{References}
 \begin{thebibliography}{9}
  \bibitem{1-pos}
  \Au{Evtushenko Yu., Posypkin~M., Rybak~L., Turkin~A.} Approximating a~solution set of 
nonlinear inequalities~// J.~Global Optim., 2018. Vol.~71. Iss.~1. P.~129--145. 
  \bibitem{2-pos}
  \Au{Cheng J., Grossman~M., McKercher~T.} Professional Cuda~C programming.~--- New 
York, NY, USA: John Wiley \& Sons, 2014. 499~p.
  \bibitem{3-pos}
  \Au{Merlet J.\,P.} Parallel robots.~--- Solid mechanics and its  
applications ser.~--- Springer Science \& Business Media, 2006. Vol.~128. 402~p.
  \bibitem{4-pos}
  \Au{Moore R.\,E., Bierbaum~F.} 
  Methods and applications of interval analysis.~---  SIAM studies 
in applied and numerical mathematics ser.~--- Soc. for 
Industrial \& Applied Math., 1979. 201~p.
  \bibitem{5-pos}
  \Au{Hansen~E., Walster~G.\,W.} Global optimization using interval 
  analysis: Revised and 
expanded.~--- Pure and applied mathematics ser.~--- CRC Press, 2003. 
Book~264. 728~p.
  \bibitem{6-pos}
  \Au{Malyshev D., Rybak~L., Behera~L., Mohan~S.} Workspace modelling of a~parallel robot 
with relative manipulation mechanisms based on optimization methods~// 
Robotics and mechatronics~/ Eds. C.\,H.~Kuo, P.\,C.~Lin, T.~Essomba, 
G.\,C.~Chen.~--- Cham: Springer, 2019. Vol.~78. P.~151--163.
  \bibitem{7-pos}
  \Au{Zatsarinny A.\,A., Gorshenin~A.\,K., Kondrashev~V.\,A., Volovich~K.\,I., Denisov~S.\,A.} 
Toward high performance solutions as services of research digital platform~// 
Procedia Comput. Sci., 2019. Vol.~150. P.~622--627.
\end{thebibliography}

 }
 }

\end{multicols}

\vspace*{-6pt}

\hfill{\small\textit{Поступила в~редакцию 08.10.19}}

\vspace*{8pt}

%\pagebreak

%\newpage

%\vspace*{-28pt}

\hrule

\vspace*{2pt}

\hrule

%\vspace*{-2pt}

\def\tit{APPROXIMATION OF~THE~SET OF~SOLUTIONS OF~SYSTEMS OF~NONLINEAR 
INEQUALITIES USING~GRAPHIC ACCELERATORS}


\def\titkol{Approximation of~the~set of~solutions of~systems of~nonlinear 
inequalities using~graphic accelerators}



\def\aut{M.\,V.~Popov and M.\,A.~Posypkin}

\def\autkol{M.\,V.~Popov and M.\,A.~Posypkin}

\titel{\tit}{\aut}{\autkol}{\titkol}

\vspace*{-9pt}


\noindent
Federal Research Center ``Computer Science and Control'' of the Russian Academy of Sciences, 
44-2~Vavilov Str., Moscow 119133, Russian Federation


\def\leftfootline{\small{\textbf{\thepage}
\hfill INFORMATIKA I EE PRIMENENIYA~--- INFORMATICS AND
APPLICATIONS\ \ \ 2020\ \ \ volume~14\ \ \ issue\ 3}
}%
 \def\rightfootline{\small{INFORMATIKA I EE PRIMENENIYA~---
INFORMATICS AND APPLICATIONS\ \ \ 2020\ \ \ volume~14\ \ \ issue\ 3
\hfill \textbf{\thepage}}}

\vspace*{3pt} 



\Abste{Solutions of certain problems can be reduced to the solution 
of some systems of inequalities. But the computation of the set of 
exact solutions may not be feasible. Thus, various methods for 
approximation of the solution set have been developed. The more 
accurate approximation is required, the bigger number of 
calculations must be performed and, consequently, the runtime of 
the algorithms increases. Nowadays, it is common to speed up algorithms 
by paralleling computations on graphics accelerators. The paper 
describes the serial method for approximation of the solution of systems 
of inequalities and proposes the parallel hybrid algorithm that combines 
iterations on the uniform grid and the branch and bound method. This 
algorithm is suited for direct implementation on graphics accelerators 
and does not suffer from the excessive enumeration of possible solution 
candidates. The\linebreak\vspace*{-12pt}}

\Abstend{sequential algorithm and the two versions of the 
parallel algorithm are compared through one example: the problem of 
approximation of the working area of the robot which consists of 
the set of robot's tool positions and is the key robot's characteristic.}

\KWE{optimization; parallel computing; graphics accelerator, GPU; CUDA; 
nonlinear inequalities}



\DOI{10.14357/19922264200303} 

%\vspace*{-20pt}

\Ack
\noindent
The paper was partially supported by the Russian Science Foundation  (project  16-19-00148).

%\vspace*{6pt}

 \begin{multicols}{2}

\renewcommand{\bibname}{\protect\rmfamily References}
%\renewcommand{\bibname}{\large\protect\rm References}

{\small\frenchspacing
 {%\baselineskip=10.8pt
 \addcontentsline{toc}{section}{References}
 \begin{thebibliography}{9}
\bibitem{1-pos-1}
\Aue{Evtushenko, Yu.\,G., M.\,A.~Posypkin, L.\,A.~Rybak, and A.\,V.~Turkin.} 2018. Approximating 
a~solution set of nonlinear inequalities. \textit{J.~Global Optim.} 71(1):129--145. 
\bibitem{2-pos-1}
\Aue{Cheng, J., M.~Grossman, and T.~McKercher.} 2014. \textit{Professional Cuda~C programming}. 
New York, NY: John Wiley \&~Sons. 499~p.
\bibitem{3-pos-1}
\Aue{Merlet, J.\,P.} 2006. \textit{Parallel robots}. 
Solid mechanics and its  
applications ser.
Springer Science \&~Business Media. Vol.~128. 402~p.
\bibitem{4-pos-1}
\Aue{Moore, R.\,E., and  F.~Bierbaum.} 1979. 
\textit{Methods and applications of interval analysis}.
 SIAM studies in applied and numerical mathematics ser.
{Soc. for Industrial \&~Applied Math.} 201~p.
\bibitem{5-pos-1}
\Aue{Hansen, E., and G.\,W.~Walster.} 2003. 
\textit{Global optimization using interval analysis: Revised 
and expanded.} Pure and applied mathematics ser.
CRC Press. Book~264. 728~p.
\bibitem{6-pos-1}
\Aue{Malyshev, D., L.\,A.~Rybak, L.~Behera, and S.~Mohan.} 2019. Workspace modelling of a parallel 
robot with relative manipulation mechanisms based on optimization methods. 
\textit{Robotics and 
mechatronics}. Eds. C.\,H.~Kuo, P.\,C.~Lin, T.~Essomba, 
and G.\,C.~Chen. Cham: Springer. 78:151--163.
\bibitem{7-pos-1}
\Aue{Zatsarinny, A.\,A., A.\,K.~Gorshenin, V.\,A.~Kondrashev, K.\,I.~Volovich, and S.\,A.~Denisov.} 
2019. Toward high performance solutions as services of research digital
 platform. \textit{Procedia Comput. 
Sci.} 150:622--627.
\end{thebibliography}

 }
 }

\end{multicols}

\vspace*{-6pt}

\hfill{\small\textit{Received October 8, 2019}}

%\pagebreak

%\vspace*{-24pt}

\Contr

\noindent
\textbf{Popov Mikhail V.} (b.\ 1995)~--- PhD student, Federal Research Center 
``Computer Science and Control'' of the Russian Academy of Sciences, 44-2~Vavilov Str., Moscow 119333, 
Russian Federation; \mbox{alvopim@gmail.com}

\vspace*{3pt}

\noindent
\textbf{Posypkin Mikhail A.} (b.\ 1974)~--- 
Doctor of Science in physics and mathematics, associate professor, principal scientist, 
Federal Research Center ``Computer Science 
and Control'' of the Russian Academy of Sciences, 44-2~Vavilov 
Str., Moscow 119333, Russian Federation; \mbox{mposypkin@gmail.com}

\label{end\stat}

\renewcommand{\bibname}{\protect\rm Литература} 