
\def\a{\overline a}
\def\r{\overline r}
\def\b{\overline b}

\def\tp{\widetilde p}

\def\tb{\tilde b}
\def\tw{\tilde w}

\def\stat{pech}


\def\tit{СИСТЕМА \boldmath$Geo/G/1/\infty$ С ОДНОЙ <<НЕСТАНДАРТНОЙ>>
ДИСЦИПЛИНОЙ ОБСЛУЖИВАНИЯ$^*$}
\def\titkol{Система $Geo/G/1/\infty$ с одной <<нестандартной>>
дисциплиной обслуживания}
\def\autkol{А.\,В.~Печинкин, С.\,Я.~Шоргин}
\def\aut{А.\,В.~Печинкин$^1$, С.\,Я.~Шоргин$^2$}

\titel{\tit}{\aut}{\autkol}{\titkol}

{\renewcommand{\thefootnote}{\fnsymbol{footnote}}\footnotetext[1]{Работа выполнена при
поддержке РФФИ, гранты 06-07-89056 и 08-07-00152.}}

\renewcommand{\thefootnote}{\arabic{footnote}}
\footnotetext[1]{Институт проблем
информатики Российской академии наук, apechinkin@ipiran.ru}
\footnotetext[2]{Институт проблем информатики Российской академии наук, 
sshorgin@ipiran.ru}

\Abst{Рассматривается система массового обслуживания (СМО) 
$Geo/G/1/\infty$
с дисциплиной обслуживания, при которой в момент поступления в
систему новой заявки ее длина сравнивается с
(остаточной) длиной заявки на приборе и та из них, длина которой
меньше, занимает прибор, а другая становится первой в
очереди, сдвигая остальную очередь на единицу.
Для этой системы найдены основные стационарные характеристики
функционирования.
В частности, показано, что, в отличие от непрерывного времени, в
дискретном времени стационарное распределение числа заявок в системе
не является инвариантным относительно загрузки.}

\KW{система массового обслуживания; дискретное время; <<нестандартная>>
дисциплина обслуживания}

      \vskip 24pt plus 9pt minus 6pt

      \thispagestyle{headings}

      \begin{multicols}{2}

      \label{st\stat}

\section{Введение}

Хорошо известно, что применение различных дисциплин
обслуживания может существенно улучшить пользовательские
характеристики СМО.

Для СМО с ожиданием в классе консервативных дисциплин
(дисциплина называется
консервативной, если процесс обслуживания не зависит от
процесса поступления заявок и в любой момент времени,
когда в системе имеются заявки, суммарная скорость
обслуживания равна единице) абсолютным чемпионом
в смысле минимальной \mbox{длины}
 очереди является дисциплина
преимущественного обслуживания заявки минимальной
остаточной длины, или SRPT (Shortest Remaining Processor
Time)~[1--4].

Отметим, что математические соотношения для расчета
показателей функционирования (в частности, стационарного
распределения числа заявок) системы $M/G/1/\infty$
с дисциплиной SRPT~[4--9] весьма сложны, демонстрируют,
скорее всего, возможности современных математических
конструкций и вряд ли могут быть использованы для
практических расчетов.
Поэтому важно было найти грани\-цы изменения
показателей функционирования СМО $M/G/1/\infty$ с
дисциплиной SRPT при фикси\-ро\-ван\-ных значениях параметров
обслуживания, например загрузке системы~[10, 11].

Интересная гипотеза была высказана А.\,Д.~Соловьевым:
при заданной загрузке в стационарном режиме функционирования
число заявок в СМО $M/G/1/\infty$ с дисциплиной SRPT
максимально в смысле упорядочения <<$\prec$>> функций
распределения (т.\,е.\ такого упорядочения, при котором
$\xi\prec\eta$, если $F_\xi(x)\ge F_\eta(x)$ для всех $x$)
при постоянной длине заявок.
Гипотезу Соловьева удалось доказать в~[12] с помощью введения
специальной дисциплины обслуживания, при которой так же,
как и при дисциплине SRPT, поступающая новая заявка
ставится на прибор, если ее длина меньше остаточной длины
обслуживаемой заявки, вытесняя последнюю на первое место в
очереди.
Однако если ее длина больше остаточной длины обслуживаемой
заявки, то новая заявка сама становится на первое место в
очереди (вне зависимости от того, есть ли в очереди заявки
более короткой длины).
Иными словами, место на приборе и первое место в очереди
разыгрывают между собой не все заявки, а только новая
заявка и заявка на приборе.
Доказательство гипотезы Соловьева основывалось на том, что
при введенной дисциплине стационарное распределение числа
заявок в СМО $M/G/1/\infty$ зависит только от загрузки
системы.

В дальнейшем появилось много работ, обобщающих данную
дисциплину в разных направлениях~[13--28].
Так, в~[13] было показано, что постоянная длина заявок
является не только худшей при дисциплине SRPT, но и лучшей
при обычной дисциплине обслуживания заявок в порядке
поступления (FCFI~--- First-Come-First-In).

Обратимся теперь к несколько иным задачам современной
теории массового обслуживания.
Применение в инфотелекоммуникационных сетях современных
технологий привело к новому всплеску исследований СМО,
функционирующих в дискретном времени
(см., например,~[29--35]).
Заметим, что при кажущемся упрощении методов исследования
СМО в дискретном времени по сравнению со СМО в непрерывном
в некоторых случаях результаты анализа значительно
усложняются.
Это связано, в первую очередь, с тем, что, в отличие от
непрерывного времени, в дискретном времени может
одновременно происходить несколько изменений состояний
(в частности, окончание обслуживания заявки на приборе
и поступление в систему новой заявки).

В настоящей работе для системы $Geo/G/1/\infty$ с
введенной в~[12] дисциплиной, являющейся дискретным
аналогом СМО $M/G/1/\infty$ с той же самой дисциплиной,
найдены основные стационарные характеристики
функционирования.
В частности, показано, что в дискретном времени стационарное
распределение числа заявок в системе зависит не только от
загрузки, т.\,е.\ не является инвариантным относительно
загрузки.

\section{Описание системы}

Рассмотрим СМО в дискретном времени $Geo/G/1/\infty$, в
которую поступает геометрический поток заявок с вероятностью
$a$ поступления заявки на одном такте
(далее будем называть тактом как интервал времени между соседними
изменениями состояния системы, так и сами моменты, в которые
происходят эти изменения).
Распределение времени обслуживания заявки является
произвольным дискретным с вероятностью $b_i$, $i\ge 0$,
того, что обслуживание заявки продлится $i$ тактов
(предполагается, что $b_0=0$).

Далее будем использовать следующие обозначения:

$\a=1-a$~--- вероятность непоступления заявки на такте;

$B_i=\sum_{j=i}^\infty b_j$, $i\ge0$,~--- вероятность того,
что обслуживание заявки продлится не менее $i$ тактов;

$\overline B_i=\sum_{j=0}^{i-1} b_j = 1-B_i$,  $i\ge 1$,~---
вероятность того, что обслуживание заявки продлится менее $i$
тактов;

$\b = \sum_{i=0}^\infty i b_i = \sum_{i=1}^\infty B_i$~---
среднее время обслуживания заявки.

Будем предполагать, что загрузка системы $\rho\;=$ $=\;a\b$ меньше единицы. Это 
условие является необходимым и достаточным для существования стационарного 
режима функционирования системы.



Дисциплина обслуживания заключается в следующем.
Предполагается, что в любой момент времени известна остаточная
длина (далее будем говорить просто длина) каждой заявки в
системе, т.\,е.\ число тактов, необходимое для окончания
обслуживания данной заявки.
В момент поступления в систему новой заявки ее длина
сравнивается с (остаточной) длиной заявки на приборе, и та
из них, длина которой меньше, занимает прибор, а другая
заявка становится на первое место в очереди, сдвигая остальную
очередь на единицу.
Для определенности будем считать, что при равенстве длин
обеих заявок заявка на приборе продолжит обслуживаться.
Если же в некоторый момент одновременно оканчивается
обслуживание одной заявки и в систему поступает другая,
то независимо от длин всех находящихся в очереди заявок
на прибор становится вновь поступившая заявка.
Заявки с прерванным обслуживанием дообслуживаются.

\section{Стационарное распределение очереди}

Найдем стационарное распределение числа заявок в системе.

Введем обозначения:

$p_{0}$~--- стационарная вероятность того, что непосредственно
после очередного такта система будет пуста;

$p_{n}(i)$,  $i\ge 1$,~--- стационарная вероятность того,
что непосредственно после очередного такта в сис\-те\-ме
будет $n$ заявок и до окончания обслуживания заявки на приборе
останется $i$ тактов.

Положим
\begin{align*}
\overline P_{n}(j) & =
\sum\limits_{i=j+1}^\infty
p_{n}(i)\,, \ \ n\ge 1\,,\ \ j\ge 0\,;
\\
P_{n}(j) & = \sum\limits_{i=1}^j
p_{n}(i)     =
\overline P_{n}(0) - \overline P_{n}(j),
\ \ n\ge 1\,,\ \ j\ge 1\,;
\\
p_{n} &=
\sum\limits_{i=1}^\infty
p_{n}(i) = \overline P_{n}(0)\,
\ \ n\ge 1\,.
\end{align*}

Выпишем систему уравнений равновесия (СУР), которой
удовлетворяют введенные функции.
Для этого удобно поступить следующим образом~[12].
Введем новую СМО $Geo/G/1/n$ с конечным числом $n$ мест
ожидания, отличающуюся от исходной только тем, что если
в очереди находится $n$ заявок и поступает новая заявка,
то после сравнения длин этой заявки и заявки на приборе
в системе (на приборе) остается заявка большей
длины, а вторая заявка покидает систему.
Используя прием, введенный в~[12] и подробно изложенный
в~[4], нетрудно показать, что стационарные вероятности
состояний в исходной и новой СМО отличаются лишь на
постоянный множитель.
Это дает возможность записать следующую СУР:
\begin{align}
p_{0} &= p_{0} \a + p_{1}(1) \a\,; \label{e1p}\\
p_{1}(i) & = p_{0} a b_i + p_{1}(i+1) \a + P_{1}(i+1) a b_i +{}\notag\\
&\ \ \ \ \ \ \ \ \ \ \ \ \ {}+ p_{1}(i+1) a \overline B_i\,, \ \ i\ge 1\,; \label{e2p} \\
p_{n}(i) & = p_{n-1}(i+1) a B_i + \overline P_{n-1}(i+1) a b_i +{}\notag\\
&\ \ \ \ \ \ \ \ \ \ \ \ \ {}+ p_{n}(i+1) \a + P_{n}(i+1) a b_i +{}\notag\\
&\ \ \ \ \ \ \ \ \ \ \ {}+ p_{n}(i+1) a \overline B_i\,,
\ \ n\ge 2\,,\ \ i\ge 1\,, \label{e3p}
\end{align}
к которой необходимо добавить условие норми\-ровки
\begin{multline}
p_0 + \sum\limits_{n=1}^\infty \sum\limits_{i=1}^\infty
p_n(i) = p_0 + \sum\limits_{n=1}^\infty \overline P_{n}(0) ={}\\
{}=p_0 + \sum\limits_{n=1}^\infty p_{n} = 1\,. \label{e4p}
\end{multline}

Для того чтобы решить СУР~(\ref{e1p})--(\ref{e4p}), просуммируем
сначала уравнения~(\ref{e2p}) и~(\ref{e3p}) по $i$ от $j$ до $\infty$.
Тогда
\begin{align}
\overline P_1(j-1) &= [p_0 + p_1]aB_j +{}\notag\\
&{}+ \overline P_{1}(j) (1-a B_{j})\,,
\ \ j\ge 1\,; \label{e5p}\\
\overline P_n(j-1) & = [p_n + \overline P_{n-1}(j)] a B_j +{}\notag\\
&{}+
\overline P_{n}(j) (1-a B_{j})\,,  \ \ n\ge 2\,,\ \ j\ge 1\,.   \label{e6p}
\end{align}
Полагая $j=1$ в соотношении~(\ref{e6p}), имеем
\begin{equation}
p_n(1) \a = [p_{n-1} - p_{n-1}(1)] a\,, \ \ n\ge 2\,.    
\label{e7p}
\end{equation}

В терминах производящей функции (ПФ)
$$
\overline P(z,j) = \sum_{n=1}^\infty z^n \overline P_{n}(j)\,,
\ \ j\ge 0\,,
$$
из~(\ref{e5p}) и~(\ref{e6p}) получаем уравнение
\begin{multline}
\overline P(z,j-1) = [z p_0 + P(z)] a B_j +\\
{}+
\overline P(z,j) [1-a(1-z) B_{j}]\,,
\ \ j\ge 1\,,  \label{e8p}
\end{multline}
где
\begin{equation}
P(z) = \sum_{n=1}^\infty z^n p_n = \overline P(z,0)\,.
\label{e9p}
\end{equation}
Решение уравнения~(\ref{e8p}) имеет вид
\begin{multline}
\overline P(z,j) =
 \fr{zp_0 + P(z)}{ 1-z}\times{}\\
{} \times
\Bigg(
1 - \prod_{i=j+1}^\infty (1 - a(1-z) B_{i})
\Bigg)\,,
\ \ j\ge 0\,, \label{e10p}
\end{multline}
где $P(z)$ определяется из соотношений~(\ref{e10p}) при $j=0$
и~(\ref{e9p}), которые приводят к формуле
\begin{multline}
P(z) = z p_0 \Bigg(
1 - \prod_{i=1}^\infty (1 - a(1-z) B_{i})
\Bigg)\times{}\\
{}\times
\Bigg(
\prod_{i=1}^\infty      (1 - a(1-z) B_{i}) - z
\Bigg)^{-1}\,.
\label{e11p}
\end{multline}
Наконец, вероятность $p_0$ получается с помощью правила
Лопиталя из условия нормировки~(\ref{e4p}) и имеет обычный вид
$$
p_0=1-\rho\,.
$$

Дифференцируя формулу~(\ref{e11p}) в точке $z=1$ соответствующее
число раз, можно найти моменты любого порядка стационарного
распределения числа заявок в системе.
Так, математическое ожидание определяется выражением
\begin{equation}
\overline m =   P^\prime (1) = \rho + \fr{1}{ 1-\rho} \Sigma^{(2)}\,,
\label{e12p}
\end{equation}
где
\begin{equation}
\Sigma^{(2)} =  \sum^\infty_{\substack{{i,k=1}\\{k>i}}}
a^2 B_i B_k\,.
\label{e13p}
\end{equation}
Последняя формула показывает, в частности, что, в отличие от
непрерывного времени, в дискретном времени для СМО $Geo/G/1/\infty$
с рас\-смат\-ри\-ва\-е\-мой дисциплиной обслуживания отсутствует
 свойство
инвариантности стационарного распределения очереди
от распределения длины заявки при фиксированной загрузке $\rho$.

Решения уравнений~(\ref{e5p}) и (\ref{e6p}) можно записать без
использования ПФ.
Например,
\begin{align}
\overline P_{1}(j) &= P_1 \left ( 1 - \Pi_j\right )\,,
\ \ j\ge 0\,; \label{e14p}\\
\overline P_{2}(j) & = P_2 \left ( 1 - \Pi_j\right ) - P_1  \Pi_{j}
\widetilde \sigma_j^{(1)}\,,
\ \ j\ge 0\,; \label{e15p}\\
\overline P_{3}(j) & = P_3 \left ( 1 - \Pi_{j}\right ) - P_2 \Pi_{j}
\widetilde \sigma_j^{(1)} -{}\notag\\
&\hspace*{50pt}\ \ \ \ \ \ \ \ \ \ {}- P_1 \Pi_{j} \widetilde \sigma_j^{(2)}\,,
\ \ j\ge 0\,, \label{e16p}
\end{align}
где
\begin{align*}
P_n & = \sum_{i=0}^n p_i\,, \ \ n\ge 0\,;\\
\Pi_j & = \prod_{i=j+1}^\infty (1 - a B_{i})\,, \ \ j\ge 0\,;\\
\widetilde \sigma_j^{(1)} & = \sum_{i=j+1}^\infty \fr{aB_i }{ 1 - aB_i}\,,
\ \ j\ge 0\,;\\
\widetilde \sigma_j^{(2)} & =
\sum^\infty_{\substack{{i,k=j+1}\\{k>i}}}
\fr{a^2 B_i B_k }{ (1 - aB_i)(1-aB_k)}\,,
\ \ j\ge 0\,;
\end{align*}
\begin{align*}
p_{1} & = \overline P_1(0) = p_0 (1 - \Pi_0) \Pi_0^{-1}\,;\\
p_{2} & = \overline P_2(0) = P_1 \left ( 1 - \Pi_0 +
\Pi_0 \widetilde \sigma_j^{(1)}\right ) \Pi_0^{-1}\,;\\
p_{3} & = \overline P_3(0) = \bigg [
P_2 \left ( 1 - \Pi_0 + \Pi_0 \widetilde \sigma_j^{(1)}\right ) +{}\\
&\ \ \ \ \ \ \ \ \ \ \ \ \ \ \ \ \ \ \ \ \ \ \ \ \ \ \ \ \ \ \ \ \ \ \ \ \ \ \ {}+P_1 \Pi_0 \widetilde \sigma_j^{(2)}
\bigg ]
\Pi_0^{-1}\,.
\end{align*}
Впрочем, формулы~(\ref{e14p})--(\ref{e16p}) можно получить и из
ПФ~(\ref{e10p}) и~(\ref{e11p}), дифференцируя их в точке $z=0$.

Отметим, что для численных расчетов удобно воспользоваться
исходными соотношениями~(\ref{e1p})--(\ref{e3p}), которые вкупе с
равенством~(\ref{e7p}) приводят к следующей рекуррентной
по $n$ и по $j$ процедуре:
\begin{align*}
p_{1}(1) & = \fr{a}{\a}\, p_{0}\,, \\
p_{1}(i+1) & = \fr{1}{1 - a B_{i+1}}
\left \{
p_1(i)\right. -{}\notag\\
&\hspace*{40pt}\ \ \ \ \ \left.{}-\left [ p_0 + P_1(i)\right ]
ab_i \right \}\,,\  \ i\ge 1\,,\\
p_n(1) & = \fr{a}{\a}
\left [ p_{n-1} - p_{n-1}(1)\right ]\,,
\ \ n\ge 2\,,  \\
p_{n}(i+1) &=  \fr{1}{1 - a B_{i+1}}
\left \{
p_{n}(i) - p_{n-1}(i+1) a B_i  -{}\right.\\
&\left.\!\!\!\!\!\!\!\!\!\!\!{}- \left [
\overline P_{n-1}(i+1) + P_{n}(i)\right ]
a b_i \right \}\,,
\ \ n\ge 2\,,\ \ i\ge 1\,.
\end{align*}
Программная реализация приведенной процедуры не вызывает
никаких затруднений.

\vspace*{-3pt}
\section{Стационарное распределение времени
пребывания заявки в~системе}
\vspace*{-3pt}

Определим теперь стационарное распределение времени
пребывания заявки в рассматриваемой сис\-теме.

Заметим сначала, что период занятости системы (ПЗ),
открываемый заявкой длины $i$, имеет~ПФ
\begin{equation}
\gamma_i(z)     = (z[a\gamma(z)+\a])^i\,,
\ \ i\ge 1\,,
\label{e17p}
\end{equation}
где $\gamma(z)$~--- ПФ ПЗ, открываемого произвольной
заявкой, удовлетворяющая уравнению
\begin{equation}
\gamma(z) = \beta(z[a\gamma(z)+\a])\,.
\label{e18p}
\end{equation}
Здесь $ \beta(z) = \sum_{i=1}^\infty z^i b_i $~---
ПФ времени обслуживания заявки.

Стационарное распределение времени ожидания начала
обслуживания заявки длины $i$ имеет~ПФ

\noindent
\begin{multline}
\psi_i(z) = p_0 + \sum_{n=1}^\infty
\Bigg(
p_n(1) +{}\\
{}+ \sum_{j=2}^{i+1} p_n(j) \gamma_{j-1}(z) + \overline P_n(i+1)
\Bigg)\,,
\ \ i\ge 1\,.
\label{e19p}
\end{multline}

Распределение полного времени обслуживания заявки длины
$i$ (т.\,е.\ времени от момента первого поступления заявки
на прибор и до момента ухода ее из системы) имеет ПФ
\begin{align}
\delta_1(z) &= z\,;\notag\\
\delta_i(z) &  =z \Bigg(
a \Bigg[
\sum_{j=1}^{i-2} \gamma_{j}(z) b_j + B_{i-1}
\Bigg ] +{} \label{e20p}\\
&\hspace*{60pt}{}+ \a
\Bigg)
\delta_{i-1}(z)\,,
\ i\ge 2\,.\notag
\end{align}

Наконец, стационарное распределение времени пребывания в
системе заявки длины $i$ имеет ПФ
\begin{equation}
\varphi_i(z)    = \psi_i(z) \delta_i(z)\,.
\label{e21p}
\end{equation}

Соответствующие безусловные характеристики определяются
усреднением по длине заявки.
В~частности, стационарное распределение времени пребывания
в системе произвольной заявки имеет ПФ
\begin{equation}
\varphi(z)      = \sum_{i=1}^\infty
\varphi_i(z) b_i\,.
\label{e22p}
\end{equation}

Дифференцируя формулы~(\ref{e17p})--(\ref{e22p}), можно получить выражения
для моментов соответствующих стационарных характеристик.
В частности, средняя длина ПЗ и средняя длина ПЗ, открываемого
заявкой длины $i$, определяются формулами
$$
\overline g = \fr {\overline b }{ 1-\rho}\,,
\qquad \overline g_i = \fr{i }{ 1-\rho}\,;
$$
математическое ожидание стационарного распределения времени
ожидания начала обслуживания заявки длины $i$ и математическое
ожидание полного времени обслуживания заявки длины $i$~---
формулами
\begin{align*}
\overline v_i & = \fr{1}{1-\rho}
\sum_{n=1}^\infty \sum_{j=2}^{i+1} (j-1) p_n(j)\,,
\ \ i\ge 1\,;\\
\overline d_1 & = 1\,;\\
\overline d_i & = 1 +  \fr{a}{1-\rho}
\sum_{j=1}^{i-2} j b_j + \overline d_{i-1}\,,
\ \ i\ge 2\,;
\end{align*}
математические ожидания стационарных распределений времен
пребывания в системе заявки длины~$i$ и произвольной заявки~---
формулами

\noindent
\begin{align*}
\overline w_i & = \overline v_i + \overline d_i\,,
\ \ i\ge 1\,;\\
\overline w & = \sum_{i=1}^\infty \overline w_i b_i\,.
\end{align*}

\section{Неравенства для~числа заявок в~системе
при~некоторых дисциплинах обслуживания}

В этом разделе будут приведены границы, в которых изменяется
среднее число заявок в рас\-смат\-ри\-ва\-емой системе при фиксированной
загрузке, а также неравенства для длины очереди в системе
$Geo/G/1/\infty$ при некоторых дисциплинах обслуживания.

Можно показать, что для любых функций
$B_i=\sum_{j=i}^\infty b_j$, $i\ge0$,
определяемых распределением $\{b_i,\ \ i\ge 0\}$
положительной целочисленной случайной величины, справедливы
неравенства
$$
1 < \sum_{i=1}^{\infty} B_i^2 \le \overline b - \alpha(1-\alpha)\,,
$$
где через $\alpha$ обозначена дробная часть $\overline b$.
Тогда из~(\ref{e12p}) и~(\ref{e13p}) имеем
\begin{multline}
\rho + \fr{a^2}{2(1-\rho)} \left [ \overline b^2 -
\overline b + \alpha\left ( 1-\alpha\right ) \right ]
\le {}\\
{}\le
\overline m < \rho + \fr{a^2}{2(1-\rho)}
\left [ \overline b^2 - 1\right ]\,.
\label{e23p}
\end{multline}

Левая часть неравенства~(\ref{e23p}) превращается в равенство при
времени обслуживания заявки, принимающем всего 2~значения:
$[\beta]$ с вероятностью $(1-\alpha)$ (здесь $[\beta]$~---
целая часть числа $\overline b$) и $[\beta]+1$~--- с дополнительной
вероятностью $\alpha$.

Правая часть неравенства~(\ref{e23p}) превращается в равенство в пределе
на последовательности <<расплывающихся>> распределений времен
обслуживания заявок,
например на последовательности распределений длин
заявок, принимающих такие 2~значения:
1 с вероятностью $(1-1/n)$ и
$n$ с вероятностью $(b-1+1/n)/n$.

Если увеличивать среднюю длину $\overline b$ заявки и
одновременно уменьшать вероятность $a$ поступления заявки
таким образом, что загрузка $\rho=a\overline b$ остается
постоянной, то в пределе нижняя и верхняя границы изменения
среднего числа заявок в системе совпадут и будут равны
$\rho + \rho^2 /\left [ 2 (1-\rho)\right ]$.
Это соответствует тому факту, что в пределе получается система
$M/G/1/\infty$ с рассматриваемой дисциплиной обслуживания,
стационарные вероятности чис\-ла заявок в которой при фиксированной
загрузке инвариантны относительно распределения длины заявки,
о чем уже говорилось выше.

Введем бесконечномерный процесс
$\nu(t) =$\linebreak $= (\nu_1(t),\nu_2(t),\ldots)$,
где $\nu_n(t)$~--- суммарная (остаточная) длина всех заявок в
системе после $t$-го такта, за исключением заявок, находящихся
на последних $(n-1)$ местах в очереди (заявка на приборе считается
находящейся на первом месте в очереди).
В частности, $\nu_1(t)$~--- суммарная длина всех заявок после
$t$-го такта.
Если в системе в момент $t$ меньше $n$ заявок, то $\nu_n(t)=0$.
Естественно, при заданных параметрах системы значение процесса
$\nu(t)$ зависит от дисциплины обслуживания.

Сравним значения процессов $\nu(t)$ при сле\-ду\-ющих дисциплинах:
\begin{enumerate}[(1)]
\item при рассматриваемой дисциплине;
\item
при инверсионном порядке обслуживания без прерывания
обслуживания (с точки зрения распределения числа заявок в
системе эта дисциплина совпадает со стандартной дисциплиной
обслуживания заявок в порядке поступления);
\item
 при дисциплине SRPT~--- наилучшей в смысле минимизации
числа заявок в системе.
\end{enumerate}

Значение процесса $\nu(t)$ при $i$-й дисциплине снабдим верхним
индексом~$(i)$.

Справедливы неравенства:
\begin{equation}
\nu^{(3)}(t)    \le \nu^{(1)}(t) \le \nu^{(2)}(t)\,.
\label{e24p}
\end{equation}
Неравенства~(\ref{e24p}) доказываются математической индукцией по времени
$t$ на основе покоординатного сравнения процессов.

Поскольку число $\eta(t)$ заявок в системе в момент~$t$ определяется
выражением
$$
\eta(t) = \max\{n: \nu_n(t)>0\}\,,
$$
то из неравенств~(\ref{e24p}) получаем неравенства
$$
F_{\eta^{(3)}(t)}(x) \prec F_{\eta^{(1)}(t)}(x) \prec F_{\eta^{(3)}(t)}(x)\,,
$$
связывающие распределения чисел заявок в системах с дисциплинами~1--3
в смысле упорядочения <<$\prec$>>.
В частности, такое упорядочение справедливо и для стационарных
распределений числа заявок в этих же системах.

%\vspace*{6pt}

\section{Вариант дисциплины}

%\vspace*{4pt}

Обратимся теперь к варианту исследуемой дисциплины,
отличающемуся от рассмотренного ранее только лишь
правилом постановки заявки на прибор в том случае,
когда одновременно оканчивается обслуживание одной
заявки и в систему поступает новая заявка.
В этой ситуации сравниваются длины новой заявки и
первой заявки из очереди, кратчайшая из них
становится на прибор, а оставшаяся занимает первое
место в очереди.
Здесь также для
 определенности будем считать, что
при равенстве длин заявок на прибор становится
заявка из оче\-реди.

Введем обозначения:

$p_{0}$~--- стационарная вероятность того, что непосредственно
после окончания очередного такта система будет пуста;

$p_{n}(i,j)$, $i,j\ge 1$, $n\ge 2$,~--- стационарная
вероятность того, что непосредственно после окончания
очередного такта в системе будет $n$ заявок и до окончания
обслуживания заявки на приборе осталось~$j$~тактов, а первой
заявки из очереди~--- $i$ тактов;

$p_{n}(j)$, $j\ge 1$, $n\ge 1$,~--- стационарная
вероятность того, что непосредственно после окончания
очередного такта в системе будет $n$ заявок и до окончания
обслуживания заявки на приборе осталось $j$ тактов,
в частности
$p_{n}(j) = \sum\limits_{i=1}^\infty p_{n}(i,j)$,
$j\ge 1$, $n\ge 2$.

Положим
\begin{align*}
P_{n}(i,j) &= \sum\limits_{k=1}^j
p_{n}(i,k)\,,  \ n\ge 2\,,\  j\ge 1\,;\\[2pt]
\overline P_{n}(j) &=  \sum\limits_{i=j+1}^\infty p_{n}(i)\,,
\  n\ge 1\,,\  j\ge 0\,;\\[2pt]
P_{n}(j) & =  \sum\limits_{i=1}^j
p_{n}(i) = \overline P_{n}(0) - \overline P_{n}(j)\,,
\  n\ge 1\,,\  j\ge 1\,;\\[2pt]
p_{n} & = \sum\limits_{i=1}^\infty
p_{n}(i)  = \overline P_{n}(0)\,,
 \ n\ge 1\,.
\end{align*}

Введенные функции удовлетворяют СУР
\begin{align}
p_{0} &= p_{0} \a + p_{1}(1) \a\,;\label{e25p}\\[3pt]
p_{1}(j) &  = p_{0} a b_j + p_{1}(j+1) \a + P_{1}(j+1) a b_j +{}\notag\\[1pt]
&\hspace*{40pt}\ \ \ \ \ \ \ \ {}+
p_{1}(j+1) a \overline B_j\,;  \ \ j\ge 1\,,\label{e26p}\\[3pt]
p_{n}(i,j) & = p_{n}(i,j+1) \a + [P_n(i,j+1) -{}\notag\\[1pt]
&\hspace*{20pt}{}- p_{n}(i,1)] a b_j  +
p_{n}(i,j+1) a \overline B_j\,;\notag\\
&\hspace*{40pt}\ \ \ \ \ \ \ \ \ \ \ \ \ \ n\ge 2\,,\ \ j>i\ge 1\,;\label{e27p}\\[3pt]
p_{n}(j,j) & = p_{n-1}(j+1) a b_j + p_{n}(j,j+1) \a +{}\notag\\[1pt]
&\hspace*{20pt}{}+[P_n(j,j+1) - p_{n}(j,1)] a b_j +{}\notag\\[1pt]
&\hspace*{30pt}{}+ p_{n}(j,j+1) a \overline B_j +
p_n(j,1)ab_j\,;\notag\\
&\hspace*{40pt}\ \ \ \ \ \ \ \ \ \ \ \  \ \ n\ge 2\,,\ \ j\ge 1\,; \label{e28p}\\[3pt]
p_{n}(i,j) & = p_{n-1}(i+1) a b_j + p_{n-1}(j+1) a b_i +{}\notag\\[1pt]
&\hspace*{20pt}{}+p_{n}(i,j+1) \a + [P_n(i,j+1) - {}\notag %\\[1pt]
\end{align}
\begin{align}
&\hspace*{30pt}{}-p_{n}(i,1)] a b_j +
p_{n}(i,j+1) a \overline B_j +{}\notag\\[1pt]
&{}+ p_n(i,1)ab_j +
p_n(j,1)ab_i\,, \  n\ge 2\,,\  i>j\ge 1\,, \label{e29p}
\end{align}
с условием нормировки
\begin{multline}
p_0 + \sum\limits_{n=1}^\infty
\sum\limits_{i=1}^\infty p_n(i)
= p_0 + \sum\limits_{n=1}^\infty \overline P_{n}(0)
=\\
{}= p_0 + \sum\limits_{n=1}^\infty p_{n} = 1\,.
\label{e30p}
\end{multline}

Произведя  элементарные преобразования соотношений~(\ref{e25p})--(\ref{e30p}),
получим следующие формулы:
\begin{align}
p_0 &= 1-\rho\,; \label{e31p}\\
p_{1}(1) &=\fr{a}{\a}\, p_0\,;\label{e32p}\\
p_{1}(j+1) & = \fr{1}{1-aB_{j+1}}
 \{
p_{1}(j) -{}\notag\\
&\ \ \ \ \ \ {}- \left [ p_0 + P_{1}(j)\right ] ab_j
 \}\,,  \ \ j\ge 1\,,\label{e33p}\\
p_{n}(1,1) & = \fr{ab_1}{1 - ab_1}\, p_{n-1}(2)\,,
\ \ n\ge 2\,; \label{e34p}\\
p_{n}(i,1) & = \fr{1}{1 - a \overline B_{i+1}}
 \{
\left [p_n(1,1)+\ldots \right.\notag\\
&\!\!\!\!\!\!\!\!\left.\ldots {}+p_n(i-1,1)\right ] ab_i +
p_{n-1}(i+1) a \overline B_{i+1} +{}\notag\\
&\!\!\!\!\!\!\!\!\!\!\!\!\!\!\!\!\!\!\!\! {}+[P_{n-1}(i) - p_{n-1}(1)] a b_i  \}\,,
\ \ n\ge 2\,,\ \ i\ge 2\,; \label{e35p}\\
p_{n}(1) & = \sum\limits_{i=1}^\infty p_{n}(i,1)\,,
\ \ n\ge 2\,;                               \label{e36p}\\
p_n(j+1) & = \fr{1}{1 - aB_{j+1}}
\big\{
p_{n}(j) - P_n(j) a b_j - {}\notag\\
&\hspace*{-20pt}{}-\left [ p_{n-1} - p_{n-1}(1)\right ] a b_j -
p_{n-1}(j+1) a +{}\notag\\
&\hspace*{20pt}{}+ p_{n}(j,1) \a \big\}\,,\ \ 
n\ge 2\,,\  j\ge 1\,.                   \label{e37p}
\end{align}
Формулы~(\ref{e31p})--(\ref{e37p}) позволяют рекуррентно вы\-чис\-лять
$p_0$, $p_{n}(i,1)$,\  $n\ge 2$,\  $i\ge 2$, и
$p_n(j)$,\ $n\ge 1$,\ $j\ge 1$.
При этом сначала по формулам~(\ref{e31p})--(\ref{e33p}) 
находятся~$p_0$ и $p_{1}(j)$, $j\ge 1$,
а затем по формулам~(\ref{e34p})--(\ref{e37p}) последовательно по $n$,
начиная с $n=2$, определяются
$p_{n}(i,1)$, $i\ge 1$,
$p_{n}(1)$, $p_n(j+1)$, $j\ge 1$.

Формулы для вероятностей $p_{n}(i,j)$,
$n\ge 2$, $i\ge 1$, $j\ge 2$, здесь не
приводятся, поскольку эти вероятности не используются
при вычислении основных стационарных показателей
функционирования сис\-темы.

Однако следует отметить, что, несмотря на прос\-то\-ту
приведенных выше формул, не удалось, в от\-личие от
первоначального варианта, получить аналитические выражения
даже для стационарного среднего числа заявок в системе.

Вычисление стационарных распределений, связанных с временем
пребывания заявки в системе, мало отличается от вычисления
их при дисциплине, введенной в начале статьи.
Приведем здесь только те формулы, в которых имеются
отличия от формул разд.~4.

Формула для ПФ стационарного распределения времени ожидания
начала обслуживания заявки длины $i$ (аналог формулы~(\ref{e19p}))
принимает вид
%\pagebreak
%\noindent
%\vspace*{-18pt}
\begin{multline*}
\!\!\!\!\psi_i(z)    = p_0 + \sum_{n=1}^\infty
\Bigg(
\sum_{k=1}^{i} p_n(k,1) \gamma_{k}(z)
+  \sum_{k=i+1}^{\infty} p_n(k,1) +{}\\
{}+  \sum_{j=2}^{i+1}
p_n(j) \gamma_{j-1}(z)  + \overline P_n(i+1)
\Bigg)\,,
\ \ i\ge 1\,.
\end{multline*}

Для того чтобы записать аналог формулы~(\ref{e20p}),
введем <<укороченный>> ПЗ, открываемый заявкой длины $i$,
под которым будем понимать интервал времени от момента
поступления в свободную сис\-те\-му заявки длины $i$ и до того
первого момента (такта),
перед которым в системе окажется единственная заявка,
обслуживаемая на последнем такте,
т.\,е.\ до того такта, когда система впервые освободится от
заявок, если не считать той заявки, которая, возможно,
поступит на этом такте.
Производящая функция <<укороченного>> ПЗ, открываемого заявкой длины~$i$,
имеет вид
$$
\tilde\gamma_{i}(z) = z^i (a \gamma(z) + \a)^{i-1}\,,
\ \ i\ge 1\,.
$$
Тогда ПФ времени, на которое прервется обслуживание заявки,
имеющей в начале такта длину $i$, определяется
выражением
$$
\hat\gamma_{i}(z) = a \Bigg(
\sum_{j=1}^{i-1}      \tilde\gamma_{j}(z) b_j
\hat\gamma_{i}(z)     + B_i
\Bigg) + \a\,, \ \ i\ge 1\,,
$$
откуда находим:
$$
\hat\gamma_{i}(z) = (a B_i + \a)
\Bigg(
1 - a \sum_{j=1}^{i-1} \tilde\gamma_{j}(z) b_j
\Bigg)^{-1}\,, \ \ i\ge 1\,.
$$

Наконец, распределение полного времени обслуживания заявки
длины $i$ имеет ПФ
\begin{align*}
\delta_1(z)  & = z\,;\\
\delta_i(z) & = z \hat\gamma_{i-1}(z) \delta_{i-1}(z)\,, \ \ i\ge 2\,.
\end{align*}

Обозначая через $\nu^{(4)}(t)$ введенный в предыдущем
разделе процесс при рассматриваемой в этом разделе
дисциплине, нетрудно получить неравенства
$$
\nu^{(3)}(t) \le \nu^{(4)}(t) \le \nu^{(1)}(t)
\le \nu^{(2)}(t)\,.
$$

\section{Заключение}

Итак, в настоящей работе рассмотрены два варианта
функционирующей в дискретном времени СМО
$Geo/G/1/\infty$ со следующей дисциплиной обслуживания.
Предполагается, что в любой момент времени известна остаточная
длина каждой заявки в системе, т.\,е.\ число тактов, необходимое для
окончания обслуживания данной заявки.
В момент поступления в систему новой заявки ее длина сравнивается
с (остаточной) длиной заявки на приборе, и та из них, длина
которой меньше, занимает прибор, а другая заявка становится
первой в очереди, сдвигая остальную очередь на единицу.

Для этой системы найдены основные стационарные характеристики
функционирования: распределение числа заявок в системе и
распределения времени ожидания начала обслуживания и времени
пребывания заявки в системе.
В частности, показано, что, в отличие от непрерывного времени,
в дискретном времени стационарное распределение числа заявок в
системе зависит не только от загрузки, т.\,е.\ не является
инвариантным относительно загрузки.

Приведены границы, в которых изменяется среднее число заявок в
рассматриваемой системе при фиксированной загрузке, а также
неравенства для длины очереди в системе $Geo/G/1/\infty$ при
некоторых дисциплинах обслуживания.

{\small\frenchspacing
{%\baselineskip=10.8pt
\addcontentsline{toc}{section}{Литература}
\begin{thebibliography}{99}

\bibitem{1p}
\Au{Schrage~L., Miller~L.}
The queue $M/G/1$ with the shortest
remaining processing time discipline~//
Oper.\ Res., 1966. Vol.~14. P.~670--684.

\bibitem{2p}
\Au{Schrage~L.}
A proof of the optimality of the shortest remaining
processing time discipline~// Oper.\ Res., 1968. Vol.~16. P.~687--690.

\bibitem{3p}
\Au{Козлов В.\,В.}
Оптимальная дисциплина обслуживания для систем
массового обслуживания~// Научные труды Кубанского университета,
1977. Т.~247. С.~33--37.

\bibitem{4p}
\Au{Бочаров~П.\,П., Печинкин~А.\,В.}
Теория массового обслуживания.~---
М.: Изд-во РУДН, 1995.

\bibitem{5p}
\Au{Печинкин~А.\,В., Соловьев~А.\,Д., Яшков~С.\,Ф.}
О системе с дисциплиной
обслуживания первым требования с минимальной оставшейся длиной~//
Изв.\ АН СССР. Технич.\ кибернет., 1979. №\,5. С.~51--58.

\bibitem{6p}
\Au{Schassberger~R.} The steady-state appearance of the $M/G/1$
queue under the discipline of shortest remaining processing
time // Adv.\ Appl.\ Probab., 1990. Vol.~22. P.~456--479.

\bibitem{7p}
\Au{Grishechkin~S.\,A.}
On a relationship between processor-sharing
queues and Cramp--Mode--Jagers branching proc\-esses~//
Adv.\ Appl.\ Probab., 1992. Vol.~24. P.~653--698.

\bibitem{8p}
\Au{Печинкин А.\,В.}
Нестационарные характеристики сис\-те\-мы обслуживания с
дисциплиной SRPT~// Вестник РУДН. Сер.\ Прикл.\ матем.\ и информ.,
1996. №\,1. С.~77--85.

\bibitem{9p}
\Au{Печинкин А.\,В.}
Система $MAP/G/1/\infty$ с дисциплиной SRPT~//
Теория вероятностей и ее прим., 2000. Т.~45. Вып.~3. С.~589--595.

\bibitem{10p}
\Au{Печинкин А.\,В.}
О верхней и нижней оценках средней очереди в системе
с дисциплиной Шраге~// Техника средств связи. Сер.\ Системы связи,
1980. Вып.~3. С.~24--28.

\bibitem{11p}
\Au{Печинкин А.\,В.}
Границы изменения стационарной очереди в системах
обслуживания с различными дисциплинами обслуживания~// Проблемы
устой\-чи\-вости стохастических моделей. Труды семинара.~--- М.: ВНИИСИ, 1985.
С.~118--121.

\bibitem{12p}
\Au{Печинкин А.\,В.}
Об одной инвариантной системе массового обслуживания~//
Math.\ Operationsforsch.\ und Statist. Ser.\ Optimization, 1983. Vol.~14.
No.\,3. P.~433--444.

\bibitem{13p}
\Au{Нагоненко В.\,А.}
О характеристиках одной нестандартной системы массового
обслуживания.~I~// Изв.\ АН СССР. Технич.\ кибернет., 1981.
№\,1. С.~187--195.

\bibitem{14p}
\Au{Нагоненко В.\,А.}
О характеристиках одной нестандартной системы массового
обслуживания.~II~// Изв.\ АН СССР. Технич.\ кибернет., 1981.
№\,3. С.~91--99.

\bibitem{15p}
\Au{Нагоненко В.\,А., Печинкин~А.\,В.}
О большой загрузке в системе с инверсионным
обслуживанием и вероятностным приоритетом~// Изв.\ АН СССР.
Технич.\ кибернет., 1982. №\,1. С.~86--94.

\bibitem{16p}
\Au{Нагоненко~В.\,А., Печинкин~А.\,В.}
О малой загрузке в системе с инверсионным
порядком обслуживания и вероятностным приоритетом~//
Изв.\ АН СССР. Технич.\ кибернет., 1984. №\,6.

\bibitem{17p}
\Au{Печинкин А.\,В.}
Инверсионный порядок обслуживания с вероятностным
приоритетом в системе обслуживания с неординарным входящим потоком~/
Случайные процессы и их приложения. Математические исследования.~---
Кишинев: Штиинца, 1989. Вып.~109.

\bibitem{18p}
\Au{Печинкин А.\,В.}
Нестационарные характеристики СМО с инверсионным порядком
обслуживания и вероятностным приоритетом~/ Вероятностные задачи
дискретной математики: Межвузовский сборник.~--- М.: МИЭМ, 1990.

\bibitem{19p}
\Au{Таташев А.\,Г.}
Многоканальная система массового обслуживания с потерей
кратчайших требований~// Автоматика и телемеханика, 1991. №\,7. С.~187--189.

\bibitem{20p}
\Au{Таташев А.\,Г.} Одна система массового обслуживания с инвариантной
дисциплиной~// Автоматика и телемеханика, 1992. №\,7. С.~92--96.

\bibitem{21p}
\Au{Таташев~А.\,Г.}
Одна система массового обслуживания с инвариантными
вероятностями состояний~// Кибернетика и системный анализ, 1993.
№\,5. С.~183--186.

\bibitem{22p}
\Au{Таташев А.\,Г.}
Система массового обслуживания с групповым поступлением
и инверсионной дисциплиной~// Кибернетика и системный анализ,
1995. №\,6. С.~163--165.

\bibitem{23p}
\Au{Таташев А.\,Г.}
Одна инверсионная дисциплина обслуживания в системе
с групповым поступлением~// Автоматика и вычисл.\ техника, 1995.
№\,1. С.~53--59.

\bibitem{24p}
\Au{Таташев А.\,Г.}
Одноканальная система массового обслуживания с потерей
заявки наибольшей длины~// Кибернетика и системный анализ, 1997.
№\,3. С.~187--188.

\bibitem{25p}
\Au{Таташев А.\,Г.}
Одна инверсионная дисциплина обслуживания в одноканальной
системе с разнотипными заявками~// Автоматика и телемеханика, 1999.
№\,7. С.~177--181.

\bibitem{26p}
\Au{Таташев А.\,Г.}
Одноканальная система с инверсионной дисциплиной
обслуживания и разнотипными заявками~//
Кибернетика и системный анализ, 2000. №\,3. С.~170--174.

\bibitem{27p}
\Au{Печинкин А.\,В., Свищева~Т.\,А.}
Система $M\!AP/G/1/\infty$ с инверсионным порядком обслуживания и вероятностным
приоритетом~// Вестник Российского университета дружбы народов.
Сер.\ Прикладная математика и информатика, 2003. №\,1. С.~109--118.

\bibitem{28p}
\Au{Pechinkin~A., Svischeva~T.}
The stationary state probability in the $BMAP/G/1/r$ queueing system
with inverse discipline and probabilistic priority~//
Transactions of XXIV International Seminar on Stability Problems for
Stochastic Models. Jurmala, Latvia. September 10--17, 2004. P.~141--147.

\bibitem{29p}
\Au{Chaudhry~M.}
Invited talk: Queue-length and waiting-time distributions of discrete-time
$GI(X)/Geom/1$ queueing systems with early and late arrivals~//
Queueing systems: Theory and applications,
1997. Vol.~25. No.\,1--4. P.~307--324.

\bibitem{30p}
\Au{Desert~B., Daduna~H.}
Discrete time tandem networks of queues effects of different regulation
schemes for simultaneous events~//
Performance Evaluation, 2002. Vol.~47. No.\,2. P.~73--104.

\bibitem{31p}
\Au{He~J., Sohraby~Kh.}
An extended combinatorial analysis framework for discrete-time queueing
systems with general sources~//
IEEE/ACM Transactions on Networking (TON), 2003. Vol.~11. No.\,1. P.~95--110.

\bibitem{32p}
\Au{Chaudhry M., Gupta~U.\,C., Goswami~V.}
On discrete-time multiserver queues with finite buffer: $GI/Geom/m/N$~//
Computers and Operations Research, 2004. Vol.~31. No.\,13. P.~2137--2150.

\bibitem{33p}
\Au{Fiems~D., Steyaert~B., Bruneel~H.}
Discrete-time queues with generally distributed service times and
renewal-type server interruptions~//
Performance Evaluation, 2004. Vol.~55. No.\,3--4. P.~277--298.

\bibitem{34p}
\Au{Atencia~I., Moreno~P.}
A discrete-time $Geo/G/1$ retrial queue with general retrial times~//
Queueing Systems, 2004. No.\,48. P.~5--21.

\bibitem{35p}
\Au{Akar~N.}
A matrix analytical method for the discrete time Lindley equation using
the generalized Schur decomposition~//
ACM International Conference Proceeding Series, 2006.
Vol.~201. No.\,12.

\end{thebibliography}
\label{end\stat}
} 
}
\end{multicols}