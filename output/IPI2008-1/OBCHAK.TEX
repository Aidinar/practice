\def\stat{abstr}
{%\hrule\par
%\vskip 7pt % 7pt
\raggedleft\Large \bf%\baselineskip=3.2ex
A\,B\,S\,T\,R\,A\,C\,T\,S \vskip 17pt
    \hrule
    \par
\vskip 21pt plus 6pt minus 3pt }

\def\tit{THE MIDDLEWARE ARCHITECTURE OF THE SUBJECT 
MEDIATORS FOR PROBLEM SOLVING OVER A SET OF INTEGRATED 
HETEROGENEOUS DISTRIBUTED INFORMATION RESOURCES IN~THE 
HYBRID GRID-INFRASTRUCTURE OF VIRTUAL OBSERVATORIES }

%1
\def\aut{D.\,O.~Briukhov$^1$, A.\,E.~Vovchenko$^2$, V.\,N.~Zakharov$^3$, 
 O.\,P.~Zhelenkova$^4$, L.\,A.~Kalinichenko$^5$, D.\,O.~Martynov$^6$,
N.\,A.~Skvortsov$^7$, and~S.\,A.~Stupnikov$^8$}

\def\auf{$^1$IPI RAN, brd@ipi.ac.ru\\[1pt]
$^2$IPI RAN, itsnein@gmail.com\\[1pt]
$^3$IPI RAN, vzakharov@ipiran.ru\\[1pt]
$^4$Special Astrophysical Observatory RAN, zhe@sao.ru\\[1pt]
$^5$IPI RAN, leonidk@synth.ipi.ac.ru\\[1pt]
$^6$IPI RAN, domartynov@gmail.com\\[1pt]
$^7$IPI RAN, nskv@ipi.ac.ru\\[1pt]
$^8$IPI RAN, ssa@ipi.ac.ru
 }

\def\leftkol{\ } % ENGLISH ABSTRACTS}

\def\rightkol{\ } %ENGLISH ABSTRACTS}

\titele{\tit}{\aut}{\auf}{\leftkol}{\rightkol}


\noindent The middleware architecture of subject mediators for scientific 
problem solving over a set of heterogeneous distributed information resources 
in the hybrid grid-infrastructure of virtual observatory (VO) is considered. 
The VO hybrid architecture is implemented as a binding of the AstroGrid VO 
system developed in the U.K.\ and of the facilities supporting subject 
mediators developed at the Institute of Informatics Problems of the Russian 
Academy of Sciences. An approach is implemented according to which for a class 
of applications, a specification of subject domain is formed independently of 
preexisting information resources. The hybrid architecture is implemented as 
the binding of execution engines of two infrastructures (AstroGrid and subject 
mediators).  That is why, in the paper, the main attention is drawn to the 
problems of rewriting  the mediator queries into the plans of their 
implementation over specific information resources, to the brief description of 
the hybrid architecture of execution engines of the AstroGrid and subject 
mediators. An example of implementation in the hybrid architecture of a subject 
mediator for solving distant galaxies discovery problem is revealed. The 
distinguishing features of the presented approach comparing to the well-known 
prototypes of database integration in VOs are overviewed. The middleware 
architecture of subject mediators is planned to be used for solving the Russian 
VO problems. 

\label{st\stat}

 \KWN{subject mediator; canonical information model; virtual observatory; 
unifier of the information models; refinement, formulae rewriting; 
semantic integration of heterogeneous information resources; 
resource registration at the mediator; ontological model; 
resources relevant to a mediator; middleware; 
specification of the problem's subject domain driven by an application}

\vskip 18pt plus 6pt minus 3pt


\def\tit{QUASI-LINEAR METHODS FOR THE INFORMATION MODEL BUILDING
FOR THE EARTH TIDAL IRREGULAR ROTATION}

%1
\def\aut{I.\,N.~Sinitsyn}
\def\auf{IPI RAN, sinitsin@dol.ru}

\def\leftkol{\ } % ENGLISH ABSTRACTS}

\def\rightkol{\ } %ENGLISH ABSTRACTS}

\titele{\tit}{\aut}{\auf}{\leftkol}{\rightkol}


\noindent
Modern stochastic information technologies for scientific research
(using \textit{a priori} and  \textit{a posteriori} data) are based on the stochastic
correlational model building methods. Off-line and on-line quasi-linear
methods based on equivalent statistical linearization of nonlinear
stochastic differential equations of the Earth tidal irregular rotation
(on half year intervals) are considered. Ten testing examples for MATLAB
software from informational resources on RAS fundamental problem
``Statistical Dynamics of the Earth Motion'' are presented.

%\label{st\stat}

\KWN{Earth tidal irregular rotation; informational model; correlational 
characteristics; correlational methods; equavalent linearization; stochastic 
differential equations}

\pagebreak

% \thispagestyle{headings}

\vskip 18pt plus 6pt minus 3pt

%\vfil

%2
\def\tit{PORTALS FOR e-GOVERNMENT SYSTEMS}

\def\aut{A.\,V.~Bosov}
\def\auf{IPI RAN, AVBosov@ipiran.ru}

\titele{\tit}{\aut}{\auf}{\leftkol}{\rightkol}

\noindent Nowadays, different companies, organizations, and associations use 
informational portal technologies for various purposes. Most likely portal 
greatest utility are in such practice as cooperation and coordination of 
business and science. But in other information spheres one can find a lot of 
problems for successful decision with portal technologies. The subject of the 
paper is an investigation of current results and perspectives of web-portal 
technologies applications for government institutions (e-Government). The 
potency and effectiveness of web-portals in e-Government activity, potentially 
decided problems are discussed and a lot of examples are briefly considered.

\KWN{e-Government; Internet technologies; portal; Internet standards}
%\pagebreak


%\vfil
 \vskip 18pt plus 6pt minus 3pt
% \vskip 24pt plus 9pt minus 6pt

%3
\def\tit{\boldmath$Geo/G/1/\infty$-QUEUE WITH ONE ``NONSTANDARD''
DISCIPLINE OF SERVICE}

\def\aut{A.\,V.~Pechinkin$^1$ and S.\,Ya.~Shorgin$^2$}
\def\auf{$^1$IPI RAN, apechinkin@ipiran.ru\\[1pt]
$^2$IPI RAN, sshorgin@ipiran.ru}


%\def\leftkol{ENGLISH ABSTRACTS}

%\def\rightkol{ENGLISH ABSTRACTS}

\titele{\tit}{\aut}{\auf}{\leftkol}{\rightkol}

\noindent
The object of consideration is s queueing system
$Geo/G/1/\infty$ with the service discipline according
to which upon the arrival of a new customer, its length
is compared to the length of (remaining) length of the
customer on the server. The customer with the minimum
length occupies the server
whereas the other becomes the first in the queue thus
shifting the remaining queue for one place.
For this system, main nonstationary characteristics are
found. In particular, it is demonstrated that, unlike the
continuous time case, for the discrete time case, the
stationary distribution of the number of customers in
the system is not invariant with respect to the loading.


\KWN{queueing system; discrete time;
``nonstandard'' discipline of the service}
%\pagebreak

%\vful

 \vskip 18pt plus 6pt minus 3pt
 
% \vskip 24pt plus 9pt minus 6pt
%\vskip 6pt plus 3pt minus 3pt
%\vspace*{12pt}

%4
\def\tit{QUEUEING SYSTEMS ALLOCATIONS MINIMIZING EXPECTED QUEUE LENGTH}

\def\aut{T.\,V.~Zakharova}

\def\auf{Moscow State University,
Faculty of Computational Mathematics and Cybernetics, lsa@cs.msu.su}

\def\leftkol{ENGLISH ABSTRACTS}

\def\rightkol{ENGLISH ABSTRACTS}

\titele{\tit}{\aut}{\auf}{\leftkol}{\rightkol}

\noindent
A class of queueing systems is considered with claims emerging on the plane.
The problem of optimal allocation of servers is solved with respect to the
criterion of expected total queue length. The optimal allocations are compared
according to the expected total queue length and expected total waiting time.

\KWN{queueing system; claims on a plane; queue length; waiting time;
optimal allocation}

%\vskip 18pt plus 6pt minus 3pt

 \vskip 18pt plus 6pt minus 3pt
 
 \pagebreak

\def\tit{NEW STAGE OF THE SOCIETY INFORMATIZATION AND ACTUAL PROBLEMS OF EDUCATION}

\def\aut{I.\,А.~Sokolov$^1$ and K.\,K.~Kolin$^2$}

\def\auf{$^1$IPI RAN, isokolov@ipiran.ru\\[1pt]
$^2$IPI RAN, kolinkk@mail.ru}

\titele{\tit}{\aut}{\auf}{\leftkol}{\rightkol}

\noindent 
The basic features of the present stage of a society informatization 
process and corresponding actual problems of the education modernization 
are analyzed. It is shown that the new conditions for a person in a global 
information society create essentially new opportunities and problems, which 
are still insufficiently considered in an education system and consequently 
demand its essential modernization. The current stage of the information 
society in Russia and the actual problems of the education in Russia 
connected with the transition to innovative strategy of development are 
considered.  

\KWN{global informatization of a society; information and communication technologies; 
an information society in Russia; a new paradigm of education; advancing 
education; strategy of innovative development}

\vskip 18pt plus 6pt minus 3pt

%5
\def\tit{PREREQUISITES AND FACTORS OF THE
INFORMATION AND COMPUTER SCIENCES CONVERGENCE}
\def\aut{I.\,M.~Zatsman$^1$ and O.\,S.~Kozhunova$^2$}

\def\auf{$^1$IPI RAN, im@a170.ipi.ac.ru\\[1pt]
$^2$IPI RAN, okozhunova@ipiran.ru}

%\def\leftkol{ENGLISH ABSTRACTS}

%\def\rightkol{ENGLISH ABSTRACTS}

\titele{\tit}{\aut}{\auf}{\leftkol}{\rightkol}

\noindent Analytical review is devoted to a problem of convergence of 
information and computer sciences, and also to interrelations of these sciences 
with information and communication technologies (ICT). Interest in the 
convergence problem arouse more than forty years ago. One of its 
formulations~--- computer and information sciences: a new basic discipline~--- 
became the title of the work by S.~Gorn published in~1963. At present, the 
urgency of a convergence problem has essentially increased. Considered in the 
review priority guidelines of research and development on ICT within 7th Frame 
Program of the European Union accepted for the period of~2007--2013 are the 
evidence of it. In the review, priority ICT guidelines are positioned as 
external factors of convergence. Except for external factors, historical 
preconditions of convergence of information and computer sciences are viewed. 
Factors and preconditions of convergence are considered in a context of 
development of scientific basis of new ICT generations' creation.


\KWN{information science; computer science; information and computer science; 
information and communication technologies (ICT); correlation between ICT and 
computer science}

 \label{end\stat}
 %\pagebreak