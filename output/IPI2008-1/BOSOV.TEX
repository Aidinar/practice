\def\stat{bosov}

\def\tit{ПОРТАЛЫ В СИСТЕМАХ ОРГАНОВ ГОСУДАРСТВЕННОЙ
ВЛАСТИ}
\def\titkol{Порталы в системах органов государственной власти}
\def\autkol{А.\,В.~Босов}
\def\aut{А.\,В.~Босов$^1$}

\titel{\tit}{\aut}{\autkol}{\titkol}


{\renewcommand{\thefootnote}{\arabic{footnote}}}
\footnotetext[1]{Институт проблем информатики Российской академии наук, AVBosov@ipiran.ru}

\Abst{Технологии информационных порталов применяют многие организации,
учреждения и сообщества для достижения самых разных целей. Наверное, наиболее полно
порталы нашли свое применение в таких областях деятельности, как поддержка
взаимодействия в коммерческой (e-Business) и в научной (e-Science) среде. Но и в других
информационных сферах можно обнаружить множество задач, которые успешно
решаются или могут решаться с помощью порталов. Предметом обсуждения данной
работы является исследование состоявшихся результатов и перспектив применения
технологий web-порталов в органах государственной власти (e-Government).
Результативность и эффективность использования порталов в интересах органов
государственной власти естественнее всего просматривается с позиций задач, стоящих
перед популярной и поддерживаемой в большинстве развитых стран концепции
<<электронного правительства>>.}

\KW{<<электронное правительство>>; Интернет-технологии; порталы; стандарты
Интернета}

      \vskip 24pt plus 9pt minus 6pt

      \thispagestyle{headings}

      \begin{multicols}{2}


      \label{st\stat}


\section{Портал~--- инструмент государственного управления}

\subsection{Информационные технологии для~государственных учреждений} %1.1.

На данный момент уже окончательно состоявшимся можно считать факт проникновения
информационных технологий (ИТ)
на все уровни государственного управления. В любом
государственном учреждении от нижнего уровня конечных исполнителей до руководства
любого уровня сегодня обязательно найдутся компьютеры и, что более важно,
специфические для данного учреждения про\-грам\-мные средства и информационные
ресурсы, обеспечивающие функции назначения именно этого уч\-реж\-дения. К~очевидным
достоинствам применения ИТ в государственном учреждении
можно отнести:
\begin{itemize}
\item сокращение числа разного рода ошибок;
\item возможность получения доступа к различным инструментам автоматизации
делопроизводственной, аналитической и коллективной деятельности, повышающим
оперативность и качество принимаемых решений;
\item улучшенное качество подготовки документов, в частности по причине большей
информированности при принятии решений;
\item укрепление исполнительской дисциплины, в частности в делопроизводстве
организации;
\item повышение уровня защищенности конфиденциальной информации граждан,
которой располагает учреждение;
\item быстрый обмен информацией как между сотрудниками уч\-реж\-де\-ния, так и с
другими уч\-реж\-де\-ни\-ями, в~том числе возможность использования интегрированных
межведомственных ресурсов;
\item возможность более эффективного предо\-став\-ле\-ния ориентированных на граждан
услуг за счет использования инструментов автоматизации обработки клиентских
запросов.
\end{itemize}

Кроме того, к не менее очевидным преимуществам можно отнести и повышение
<<коэффициента интеллекта>> и авторитетности учреждения~\cite{1bos}, сотрудниками
которого активно используются компьютерные технологии, как в глазах общества, так и с
точки зрения международной позиции государства в целом.

Ясно, что такая перспектива однозначно свидетельствует о стратегической
перспективности информатизации любого госучреждения. Однако не все просто на пути
достижения этой перспективы. Есть и не менее очевидные, чем достоинства, недостатки.
Обобщая опыт многих проектов внедрения ИТ-решений в органах государственной
влас\-ти, можно утверждать, что основные проблемы, которые приходится решать в рамках
обсуждаемых проектов, носят организационный характер. К их числу относятся,
например, такие:
\begin{itemize}
\item низкий уровень компьютерной грамотности служащих и, как следствие,
необходимость осуществления в рамках проекта мероприятий по его повышению;
\item реализация проектов осуществляется <<без отрыва от производства>>:
учреждение должно выполнять возложенные на него функции вне зависимости от
этапов выполнения ИТ-проекта; как следствие~--- возникновение сложной задачи
комплексного планирования с участием всех уровней служащих учреждения, в~том
чис\-ле планирование развертывания и поддержки первых этапов эксплуатации с условием
бес\-пре\-рыв\-ности работы учреждения;
\item все выполняемые работы должны учитывать вопросы обеспечения безопасности,
причем не только в целях обеспечения зашиты от не\-санк\-ци\-о\-ни\-ро\-ван\-но\-го доступа
извне, но и в \mbox{целях} блокирования возможностей некорректного использования 
ИТ-ре\-ше\-ния в личных целях служащими самой организации; к этому же кругу проблем 
можно отнести вопросы соблюдения авторских прав и прав соб\-ст\-вен\-ности на 
интеллектуальный продукт.
\end{itemize}

Этот перечень позволяет сформулировать и оценить еще одну проблему, решение
которой, к сожа\-ле\-нию, целиком ложится на, как правило, малокомпетентное в данном
вопросе руководство учреж\-де\-ния~--- выбор исполнителя проекта информатизации.

\subsection{Концепция <<электронного правительства>>} %1.2.

Сравнительно недавно уровень развития ИТ не позволял сформировать сколько-нибудь
обобщенные решения для широкого круга госучреждений. Да, компьютеры и отдельные
ведомственные сис\-те\-мы существуют давно. Взрывное развитие Интернета привело, в
частности, к тому, что использование электронной почты стало необходимостью даже для
самых небольших госучреждений. Однако даже постановки более существенных задач,
например интеграция электронной почты с ведомственными базами данных и средствами
управления контентом, представлялись нереальными задачами. Принципиально на
изменение ситуации повлияло не столько развитие Интернета, сколько результаты
деятельности Интернет-сообщества в области стандартизации форматов, протоколов
взаимодействия и архитектур информационных систем для world-wide web. Результаты
этой деятельности, о которых подробнее будет сказано далее, сделали возможным
сформировать понимание того, какими методами можно достичь сформулированной
стратегической перспективы информатизации органов государственного управления,
выразившееся в формировании концепции <<электронного правительства>>.

Концепция <<электронного правительства>> состоит в определении необходимости
прохождения госучреждением ряда этапов с конечной целью интегрировать весь
комплекс услуг учреждения и обеспечить доступ к ним потребителей~--- граждан и
других учреждений.

Этап первый предполагает использование Интернета для создания сайтов отдельных
гос\-уч\-реж\-де\-ний с целью информирования граждан о работе учреждения, правилах и
сроках подачи документов, получения справок и~пр. Как и всем <<статическим>>
web-ресурсам Интернета, таким сайтам свойственна \textit{односторонняя связь}.

Следующий очевидный этап~--- наделение ведом\-ст\-вен\-ных сайтов интерактивными
возможностями, превращающими их в инструмент \textit{двустороннего общения}
граждан и сотрудников учреждения. Этот этап начинается с появления на сайтах
госучреждений электронных форм, которые предлагается заполнить пользователю с
целью получения ка\-кой-ли\-бо услуги. Обратим внимание, что для реализации даже
простейшего двустороннего обмена требуется, чтобы web-ре\-сурс предоставлял ряд
сервисов: аутентификации и авторизации пользователей, персонализации пред\-став\-ле\-ний,
обработки пользовательских запросов на стороне сервера web-сис\-те\-мы (портлеты). Строго
говоря, наличие таких сервисов является основанием для признания такой системы
web-пор\-та\-лом.

Большинство государств в настоящее время находится либо на первом, либо на втором
этапе <<электронного правительства>>. Следующий качественный уровень предполагает
возможность использования web-ресурсов государства с целью \textit{осуществления
<<финансовых>> операций}, подразумевая под этим широкий спектр услуг, результат
которых носит официальный статус. В качестве примеров можно было бы привести
оплату всевозможных пошлин, сборов и штрафов, получение и/или продление различных
лицензий, подачу заявлений для устройства на работу, подачу конкурсных материалов,
получение субсидий и~пр. Примеров в бизнесе структур, поддерживающих такой
уровень информатизации, множество. Так, большинство банков, включенных в систему
электронных платежей, имеют упомянутые web-ресурсы.

Последний, наиболее сложный этап состоит в превращении web-ресурса госучреждения в
полноценный \textit{web-портал}. При этом именно в данной концепции возможности
портальных технологий востребованы в наиболее полном объеме. От портала органа
госвласти потребуется:
\begin{itemize}
\item интегрированное представление всего комплекса услуг госучреждения;
\item возможность доступа к услугам граждан, исходя из их потребностей, а не из
структуры подразделений учреждения;
\item непротиворечивое представление в рамках едино\-го пользовательского
интерфейса множества разнородных данных и услуг, под\-дер\-жи\-ва\-е\-мых/по\-став\-ля\-е\-мых
разнородными внутриведомственными системами автоматизации.
\end{itemize}

Интегрированный характер последнего этапа концепции <<электронного правительства>>,
который многие исследователи называют также этапом <<\textit{информационной
экономики}>>~\cite{2bos}, конечно, не исчерпывается объединением внутренних ресурсов
одного учреждения, что было бы не таким и сложным. Более значимым ожиданием на
этом этапе является не только полная информатизация операций внутри учреждения, но и
возможность выполнения любых межведомственных операций, требующихся
гражданину, в электронной форме, т.\,е.\ интеграция не внутриведомственная, а уже
межведомственная.

В одном из выступлений Билл Гейтс~\cite{2bos} заявил: <<Пока ни в одной стране мира
не удалось собрать все кусочки этой головоломки (<<электронного правительства>>)
вместе. Тем не менее системы, развертываемые в настоящее время практически
повсеместно, отлично работают, предоставляя гражданам дополнительные возможности.
Они расширяют доступ к информации, что необходимо для совершенствования
демократии, повышают эффективность государственного управления, избавляют от
лишнего ожидания и обеспечивают информационную прозрачность>>. Пока ситуация в
целом сохраняется: законченных решений такого уровня в мире нет, но на данный момент
можно констатировать, что достижение этого этапа концепции уже не является
фантастическим, и, по крайней мере, весь необходимый инструментарий и все нужные
для этого технологии ИТ-сообществом разработаны и апробированы.

\subsection{Проблемы реализации концепции} %1.3.

Как и любая инновация, реализация <<электронного правительства>> сталкивается с
рядом проблем. К главным из них можно отнести:
\begin{itemize}
\item непонимание целей и задач концепции со стороны государственных чиновников,
вплоть до сопротивления реализации конкретных проектов;
\item необходимость консолидированного проявления политической воли со стороны
высших государственных органов, в~том числе для решения вопросов финансирования;\\[-9pt]
\item организационная сложность контроля качества выполнения проектов, в
частности трудности определения критериев оценки их успешности;\\[-9pt]
\item необходимость вовлечения в проекты широкого круга граждан;\\[-9pt]
\item технические вопросы.\\[-9pt]
\end{itemize}

Ответить на вопрос, для чего нужно создавать <<электронное правительство>>, не так
просто. Для большинства чиновников решение этого вопроса представляется во
внедрении компьютеров и ИТ в деятельность гос\-уч\-реж\-де\-ния, в
частности с \mbox{целью} экономии затрат. Место же этой де\-я\-тель\-ности в общем контексте
государственных реформ не рассматривается. На самом деле главный акцент следует
делать на максимальную полноту информации в государственных структурах и
максимально полное удовлетворение потребностей граждан ИТ-сред\-ст\-ва\-ми. Иными
словами, более важным в концепции является тезис о доступности информации и услуг, а
уж за ним~--- электронные формы взаимодействия. В~целом в мире сформировалось
понимание того, что <<электронное правительство>>~--- это, прежде всего, применение
ИТ как инструмента для изменения характера деятельности
государственных служб~\cite{3bos}, и  признается, что ключ к успеху <<электронного
правительства>>~--- ориентированный на граждан подход. Государство должно
рассматривать граждан и частные компании как своих клиентов и предоставлять им
услуги в соответствии с их запросами, а не на основе задач, стоящих перед самим
государством. Одним из главных факторов, определяющих категорию, к которой может
отнести ту или иную страну по показателю развития <<электронного правительства>>,
является степень использования государством данного подхода.

В этой связи высока вероятность противления ИТ-ин\-но\-ва\-ци\-ям со стороны чиновников,
поскольку такие преобразования, во-первых, потребуют новых методов работы от самих
чиновников, во-вторых, сделают работу госучреждений более прозрачной. И то, и другое
никогда не приветствовалось.

Для преодоления подобных препон в первую очередь следует прибегать к разного рода
разъяснительным методам, но добиться полного успеха можно только за счет
формирования консолидированной политической воли высших органов управ\-ле\-ния.
Мировая практика показывает, что за большинством успешных проектов в этой области
стоит сильный государственный лидер, готовый связать с ним свою репутацию и тратить
на него свое время, авторитет и ресурсы~\cite{2bos}. Любопытно отметить, что понимание
этой ситуации давно сложилось и в России~\cite{4bos, 5bos}, и в настоящее время
действует серия государственных программ, поддерживающих концепцию
<<электронного правительства>> на самом высоком уровне.

Но даже при наличии необходимой политической и финансовой поддержки реализация
конкретных проектов испытывает значительные трудности. Прежде всего, речь идет о
том, что очень сложно точно сформулировать критерии ус\-пеш\-ности проектов в этой
области: непосредственных и быстрых экономических выгод нет, оценки степени
прозрачности и числа вовлеченных пользователей весьма туманны. Таким образом,
команде, реализующей проект на самых начальных стадиях, необходимо озаботиться и
сформулировать по возможности более четкие критерии оценки реализации проекта.
Здесь не последнее внимание необходимо уделять целевой аудитории, ради которой,
собственно, и выполняется проект. Привлечь максимальное число заинтересованных
граж\-дан, информировать их о новых возможностях, организовать обратную связь~--- вот
задачи, обязательные к постановке и решению в любом проекте из рас\-смат\-ри\-ва\-емой
области.

На ранних стадиях формирования концепции <<электронного правительства>> к
важнейшим относились и технические вопросы, включая как организацию доступа к
новым ресурсам, т.\,е.\ развитие и обеспечение доступности средств коммуникаций, так и
возможности технологий, позволяющих создавать нужные решения. На данный момент
можно констатировать, что острота этих вопросов в целом снята. Технологические
проблемы, как и упоминалось выше, за последние годы практически решены~--- весь
необходимый инструментарий имеется. И главное~--- имеются необходимые стандарты.
Проблему доступа в Интернет тоже нельзя считать острой, по крайне мере, в нашей
стране, чему немало способствуют и действующие национальные программы в области
обучения.

Преодолев главные проблемы, госучреждение приступает к реализации концепции,
выполняя первый, довольно простой шаг~--- выведение организации в Интернет. Решение
этой задачи, т.\,е.\ создание собственного компетентного представительства в Интернете,
действительно, довольно простая задача, но при ее решении с учетом перспективы
постановки и последующих задач возникает еще одна проблема, хотя и не отнесенная к
главным, но, несомненно, очень значительная. Это проблема выбора команды для
реализации проекта информатизации. Грамотный подход к решению этой проблемы
существенно влияет на общий успех проекта информатизации учреждения в целом, ведь
начиная с малого~--- создания сайта~--- выбранной команде, реализующей концепцию
<<электронного правительства>>, предстоит практически сразу перейти к следующему
этапу~--- налаживанию онлайновых отношений сотрудников учреждения и граж\-дан. В
целом проблема выбора команды обширна и выходит за рамки данной статьи.
Перечислим только те задачи, которые должны решать разработчики, организуя
электронную службу в Интернете для госучреждения:
\begin{itemize}
\item обеспечение компетентного представления учреждения в сети Интернет в
кратчайшие \mbox{сроки};
\item обеспечение информирования граждан о наличии и возможностях
Интернет-пред\-ста\-ви\-тель\-ства учреждения;
\item разработка и внедрение web-форм и системы обработки обращений граждан в
среду созданного Интернет-представительства (скорее всего, с использованием
портальных технологий), включив при этом в состав решения и средства
персонификации;
\item обеспечение высокого уровня конфиденциальности и безопасности при
интеграции необходимых web-служб в состав созданного Интернет-представительства;
\item интеграция сервисов, предоставляемых порталом учреждения, с имеющимися в
учреждении средствами автоматизации;
\item разработка предложений по совершенствованию бизнес-процессов, протекающих
в уч\-реж\-де\-нии, на основе новых сервисов, поддерживаемых созданным порталом, и
предложений по снижению расходов на сопровождение ИТ
и рисков, связанных с технической поддержкой, за счет использования современных
архитектур и открытых стандартов;
\item предвидение и прогнозирование будущих потребностей сотрудников
госучреждения и граж\-дан и учет их в текущих решениях.
\end{itemize}

Таким образом, проблема определения исполнителя, реализующего проект в области
<<электронного правительства>>, требует значительного внимания и с полным
основанием может быть отнесена к главным проблемам реализации концепции.

\section{Порталы и <<электронное правительство>>} %2

\subsection{Задачи, решаемые порталами} %2.1.

Основной задачей при реализации концепции <<электронного правительства>> является
обеспечение открытости деятельности органов госвласти, т.\,е.\ возможность публичного
доступа к информации государственного учреждения. В любой стране именно
государство является крупнейшим поставщиком услуг населению и предприятиям.
Государство также является и крупнейшим потребителем информации и информационных
услуг предприятий и граждан.

На данный момент общепризнано, что оптимальной формой организации публичного
доступа к информации являются web-порталы. В большинстве развитых стран,
поддерживающих концепцию <<электронного правительства>>, доступ в Интернет имеет
подавляющее большинство граждан. В РФ доступность необходимых средств
коммуникаций по разным причинам не столь масштабна, однако более 10\% населения эту
возможность имеет, причем это~--- наиболее активная и деятельная часть населения.

Первая задача портала органа государственного управления~--- обеспечивать
возможность доступа к информации заинтересованных лиц. Эта задача решается путем
публикации информации на
 сайте портала, а необходимые для этого ин\-стру\-менты
предоставляются любым портальным решением. Вторая задача, также легко решаемая
стандартными средствами портала,~--- это активное
 инфор\-ми\-ро\-ва\-ние 
пользователей портала (рассылки). Соответствующие инструменты обеспечивают возможность
рассылки новостей, нормативных актов, сообщений и прочих видов информационного
содержания, публикуемого на портале, пользователям, подписавшимся на рассылку.

Продвижение по этапам реализации концепции <<электронного правительства>> ставит
следующую задачу~--- обеспечение возможности работы с обращениями граждан и
организаций, т.\,е.\ речь идет об организации интерактивного взаимодействия.
Портальные инструменты, применяемые для решения этой задачи, должны обеспечивать
возможность обращения через сайт портала к сотрудникам госучреждения с целью
решения своих вопросов. Кроме того, данный инструментарий должен давать
возможность отслеживания хода выполнения запросов, формирования поручений и
отчетов, т.\,е.\ реализовывать элементы делопроизводства.

Обеспечивая решение этой задачи, портал позволяет радикально изменить практику
работы сотрудников государственного учреждения, повысить эффективность и
уменьшить затраты, в~том числе и затраты времени граждан, которое в противном случае
тратится на личное посещение учреждения.

Далее естественным образом возникает очередная задача~--- централизация и интеграция
услуг, предоставляемых как порталом отдельного уч\-реж\-де\-ния, так и множеством
порталов различных  \mbox{ведомств}. Практика реализации концепции <<электронного
правительства>> показывает, что большинством стран-ли\-де\-ров в этой области данная
задача формулируется как задача создания единых центров доступа~\cite{6bos}. Такие
центры доступа, будучи, очевидно, очередным шагом по реализации концепции, 
во-пер\-вых, также оформляются в виде порталов (т.\,е.\ инструментом для решения данной
задачи продолжает оставаться web-портал), во-вто\-рых, являются наиболее ощутимым
результатом реализации концепции. Кроме того, центры доступа оказываются мощным
инструментом интеграции государственных услуг в целом, существенно способствуя
преодолению межведомственных барьеров, а также~--- толчком к ускорению развития
сетевых услуг в стране. Из сказанного следует, что интеграция услуг
государственных уч\-реж\-де\-ний с помощью порталов означает непременно и
реформирование деятельности государства, т.\,е.\ высказанный выше тезис о
необходимости <<политической воли>> сильно проявляется именно на этой стадии
реализации концепции <<электронного правительства>>.

На текущий момент общемировую практику реализации решения задачи создания
портальных центров доступа (так называемых правительственных порталов)
характеризуют следующие факторы~\cite{6bos}:
\begin{itemize}
\item большинство правительственных порталов ограничивается переадресацией
пользователя на другие web-узлы;
\item отмечается общая тенденция движения в сторону предоставления информации и
услуг в виде, не требующем от пользователя знания деталей устройства
бюрократического аппарата;
\item успешными оказываются портальные проекты, которые, во-первых,
ориентированы на клиента, во-вторых, решают задачи интеграции информации,
систем, процессов и услуг, предоставляемых государственной инфраструктурой в
целом.
\end{itemize}

Решение задачи организации правительственного портала~--- центра доступа, или
правительственного шлюза,~--- наиболее полно задействует возможности современных
портальных технологий, так как создание <<единой точки входа>>, обеспечивающей
функции <<единого окна>> при доступе к различным данным и услугам, и есть задача
web-портала.

Создание правительственного центра доступа обеспечивает решение и ряда
сопутствующих задач, определенных концепцией <<электронного правительства>>, а
именно:
\begin{itemize}
\item полное и объективное информирование граж\-дан о всей совокупности услуг,
предоставляемых государственным аппаратом в целом;
\item вовлечение граждан и организаций, обеспечение социальной поддержки
инициатив в об\-ласти <<электронного правительства>>;
\item обеспечение постоянства и оперативности взаимодействия государственных
учреждений с гражданами и хозяйствующими субъектами, в~том числе достижение
прозрачности и повышение эффективности этого взаимодействия и оптимизация
взаимодействия государственных структур.
\end{itemize}

\subsection{О видах порталов} %2.2.

Термин <<портал>>, первоначально появившийся в ИТ-об\-ласти безотносительно к сети и
технологиям Интернета, означает единые ворота или единую точку доступа ко множеству
разнообразных данных и услуг. В принципе, для реализации портала вовсе нет
необходимости использовать механизмы world-wide web. Однако именно лавинообразное
развитие Интернета и связанных с глобальной сетью технологий стало реальным толчком
к созданию действующих портальных решений. Самая очевидная задача, решаемая как
первыми web-пор\-та\-ла\-ми, так и наиболее известными порталами, действующими в
настоящее время,~--- это создание <<единой точки входа>> в Интернет для всех
пользователей. Такие общедоступные порталы, называемые еще по признаку охвата
пользовательской аудитории <<горизонтальными>>, непременно имеют в своем составе
глобальную поисковую машину и простейшие средства сбора, систематизации и
представления в удобном для пользователя
 виде как можно большего числа полезных
информационных ресурсов. Как правило, такие порталы~--- уникальные штучные
решения, поддерживаемые большими ИТ-ком\-па\-ния\-ми ({\sf www.google.com}, {\sf
www.msn.com}, {\sf www.rambler.ru}, {\sf www.yandex.ru} и~пр.).

В рамках концепции <<электронного правительства>> речь идет, конечно, не о таких
порталах. Уместнее говорить о <<вертикальных>> порталах, ограниченных по сравнению
с общедоступными порталами как целевой аудиторией, так и специальным набором
поддерживаемых ресурсов и сервисов. Если общедоступные порталы, подчеркивая их
<<горизонтальный>> статус, называют еще Интернет-порталами, то для
<<вертикальных>> порталов уместно использовать название интранет-порталов,
подчеркивая тем самым, что у такого портала есть существенная часть, предназначенная
не для общего доступа, а для поддержки деятельности ограниченного круга сотрудников,
выполняющих обязан\-ности в рамках некоторого предприятия.

Внутренняя, корпоративная часть интранет-портала обеспечивает создание
персонализированного рабочего места сотрудника учреждения, позволяющего ему
получать доступ ко всем необходимым внутренним и внешним ресурсам и сервисам в
рамках своих полномочий. Внешняя часть интранет-портала, по сути аналогичная
Интернет-порталу, позволяет широкому кругу внешних пользователей получать доступ
(естественно, ограниченный) к той части корпоративной информации и услуг, что
предназначена для этого.

На текущий момент практически все крупные производители программного обеспечения
имеют <<коробочные>> продукты, позволяющие учреждению реализовывать собственное
портальное решение. Конечно, порталы от IBM, Oracle, Microsoft, Hummingbird, Plumtree,
Lotus, Vignette и~пр.\ отличаются по своим функциональным воз\-мож\-ностям, так что
выдача каких-либо рекомендаций по выбору портальной платформы без детального
анализа функций назначения и информационно-сер\-вис\-но\-го обеспечения портала органа
госвласти не представляется возможной. Отметить в связи с многообразием возможных
решений следует только одно: все они не являются законченными web-порталами
предприятия, которые достаточно развернуть и сразу начать использовать, а представляют
собой только инструментарий, который надо изучить и грамотно применить в конкретном
проекте.

\section{Подходы к разработке портала <<электронного правительства>>} %3

\subsection{Состав портального решения и~требования к компонентам} %3.1.

Определение детальных требований к порталу органа госвласти должно основываться на
определениях: состава целевой аудитории; состава информации, размещаемой на
страницах сайта портала; состава информационных услуг, предоставляемых службами
портала; возможностей и уровня подготовки сотрудников, обеспечивающих актуализацию
и поддержку портального контента (ресурсов и служб). Однако можно выделить
обязательный минимальный набор компонентов портала, необходимый для любого
решения такого уровня. В состав таких компонентов непременно должны входить:
\begin{itemize}
\item сайт портала или компонент представления (обеспечивает формирование
конечного пользовательского web-интерфейса);
\item собственное хранилище портала (обеспечивает размещение собственного
информационного содержания портала);
\item система управления содержанием или блок управления контентом (обеспечивает
решение задачи формирования информационного содержания портала как в
собственном хранилище, так и посредством выполнения запросов в ресурсах портала,
например корпоративных хранилищах);
\item компонент поиска (обеспечивает формирование и исполнение разнообразных
поисковых запросов пользователей в собственном хранилище портала и во
<<внешних>> ресурсах);
\item система безопасности и компонент персонализации (решают задачи
аутентификации и авторизации пользователей (возможно также и \mbox{аудит}) и
формирования собственных персональных настроек).
\end{itemize}

В дополнение к данному базовому набору, как правило, присутствуют и еще несколько
стандартных компонентов, таких как компоненты обмена сообщениями (форумы, чаты,
электронная почта), компоненты коллективной работы (делопроизводство), а для
порталов, готовых к работе с большим числом разнообразных <<внешних>>
источников,~--- компонент интеграции, обеспечивающий единообразное представление
разнородных ресурсов в унифицированном портальном интерфейсе. Наличие
интеграционного сервиса непременно означает и наличие в портале ряда адаптеров~---
компонентов, обеспечивающих преобразование запросов и форматов данных между
ядром портала и <<внешними>> ресурсами.

Общим для всех портальных компонентов требованием является возможность управления
ими, т.\,е.\ использования всех поддерживаемых ими функций, с помощью <<тонкого
клиента>>~--- посредством программы web-обозревателя (браузера). Из других важных
требований остановимся на тех, что предъявляются к компоненту представления и
сис\-те\-ме управления содержанием.

Для интегрированной архитектуры портала <<электронного правительства>>,
предполагающей представление на его сайте большого многообразия ресурсов и услуг,
непременным условием востребованности является обеспечение 
максимального удобства
представления информации, исходя исклю\-чи\-тель\-но из интересов пользователей.
Компонент представления должен, во-пер\-вых, обеспечивать гибкие возможности по
размещению информа\-ци\-он\-но\-го содержания на сайте, как минимум позволяя группировать
информацию по тематическим областям. И во-вто\-рых, компонент пред\-став\-ле\-ния должен
поддерживать многообразие инди\-ви\-ду\-аль\-ных параметров, позволяющих авторизованному
пользователю настраивать интерфейсные элементы <<под себя>>, в частности иметь
возможность формировать состав представляемых ресурсов и услуг согласно
собственным потребностям и вкусам.

Служба управления содержанием (Content Management System, CMS) является, видимо,
самой значимой и сложной частью портала, хотя конечному пользователю, как правило,
она не видна вовсе. Требования, предъявляемые к портальной CMS:
{\looseness=1

}
\begin{itemize}
\item возможность использования шаблонов представления, обеспечивающих
разделение данных и кода, формирующего конечное оформление информации;
\item возможность обновления информационного содержания, состава
информационных услуг портала и состава <<внешних>> ресурсов в реальном
масштабе времени;
\item возможность динамической сборки содержания web-страниц из разных ресурсов,
подключенных к порталу.
\end{itemize}

Еще несколько слов следует сказать по \mbox{поводу} дополнительных сервисов, получающих
рас\-про\-стра\-не\-ние в настоящее время в связи с развитием коммуникационных средств.
Именно
 \mbox{широкому} кругу потенциальных пользователей госу\-дар\-ст\-вен\-ных порталов
сегодня становятся доступны средства коммуникаций, позволяющие актив\-но
использовать видеоизображения и телевидение (WebTV), подключаться к порталу с
использованием мобильного телефона или карманного PDA-ком\-пью\-те\-ра. Таким образом,
граждане имеют возмож\-ность сами определять, как, когда, каким образом и какие услуги
они хотят получать, причем эти потребности уже не исчерпываются возмож\-но\-стя\-ми
домашнего компьютера и Dialup-со\-еди\-не\-ния. Задача правительственного портала~---
обеспечивать удовлетворение этих потребностей, т.\,е.\ поддерживать богатый
мультимедийный контент, средства Ин\-тер\-нет-ве\-ща\-ния и специальные версии сайта,
предназначенные для просмотра с мобильных устройств.

Не столь очевидными, как перечисленные, но не менее важными являются и требования,
которые следует предъявлять к архитектуре портального решения. На текущем этапе
реализации концепции <<электронного правительства>> в разных странах выполняются
серии проектов по созданию порта\-лов для отдельных органов федеральной, региональной
и местной власти. В~не\-кото\-рых странах вы\-пол\-няют\-ся, в других~--- пред\-пола\-га\-ет\-ся, что \mbox{будут}
выпол\-нять\-ся, проекты по созда\-нию центральных правитель\-ст\-вен\-ных порталов, которые
должны объединить часть услуг  порталов отдель\-ных\linebreak
 ведомств. Таким образом,
формируется существенное требование к реализации инфраструктурной со\-став\-ля\-ющей
портала~--- она должна позволять разработчикам использовать механизмы интеграции
услуг различных провайдеров с учетом существующей иерархии ответственности. Как
удовлетворить этому требованию, обсуждается дальше, а здесь отметим, что прежде всего
это требование влияет на центральный правительственный портал, который должен
рассматриваться не только как информационная витрина правительства, но и как
инфраструктурный компонент, отвечающий за обеспечение единых интерфейсов для
разработчиков государственных систем.

\subsection{Технологии и стандарты Интернета} %3.2.

Базой для реализации любого портального реше\-ния, естественно, являются web-сай\-ты~---
\mbox{сама} основа Интернета, т.\,е.\ распределенная IP-сре\-да, в которой присутствуют
web-сер\-ве\-ры, под\-держи\-ва\-ющие стандартный протокол HTTP, и web-кли\-ен\-ты~---
про\-грам\-мы-обо\-зре\-ва\-те\-ли (web-брау\-зе\-ры), обеспечивающие представление полученных по
протоколу HTTP страниц, размеченных на языке HTML (сейчас актуальнее говорить о
разметке XHTML). Однако считать, что именно возможности этих технологий являются
необходимым минимумом для создания портала, было бы серьезной ошибкой.

Первая проблема, с которой сталкивались разработчики порталов,~--- отсутствие
общепринятого стандарта для обмена информацией между разными системами, а ведь в
связи с решением именно этой задачи~--- обеспечить <<единую точку входа>> к данным
разных источников~--- и появилась тер\-ми\-но\-логия порталов. Стереть ограничения и
барь\-е\-ры непо\-ни\-ма\-ния между множеством обязанных взаимодействовать систем, научить
их <<понимать>> данные друг друга мог только общепринятый язык, оформленный как
открытый стандарт, достаточно простой и в то же время выразительный, чтобы выступать
в качестве базового стандарта для интеграции любых данных. Таким языком <<лингва
франка>> для Интернета стал XML (eXtensible Mark-up Language, расширяемый язык
разметки)~\cite{7bos}, утверж\-денный  World-Wide Web Consortium (W3C) и
предостав\-ля\-ющий механизм для обеспечения информационной совместимости,
признаваемый на данный момент всеми без исключения членами ИТ-со\-об\-ще\-ст\-ва.

Следующая проблема~--- обеспечение функциональной совместимости. Попытки ее
решения осуществлялись давно и небезуспешно (можно сослаться на такие технологии,
как Corba и DCOM), но единого общепринятого стандарта не образовывалось вплоть до
утверждения W3C стандарта протокола доступа к простым объектам SOAP 
(Simple Object Access Protocol)~\cite{8bos}. 
Протокол SOAP описывает способ
использования языка XML и протокола HTTP для создания механизмов доставки
информации и вызова удаленных процедур. Протокол SOAP является основой технологии 
web-сервисов, сделавшей реальным применение в промышленных масштабах архитектур SOA
(Service-Oriented Architecture, ориентированная на сервисы архитектура). Идея
SOA~\cite{9bos} создавать системы, формируя их из наборов отдельных сервисов,
предо\-став\-ля\-ющих возможности доступа к простым услугам посредством стандартных
интерфейсов, поначалу находила применение только в системах распределенных
вы\-чис\-ле\-ний, а с недавнего времени~--- в сетях GRID. Но настоящую путевку в жизнь этой
архитектуре дали именно web-сервисы~\cite{10bos}, основой которых является протокол
SOAP.

Другие важные компоненты технологии web-сервисов оформлены в стандартах UDDI и
WSDL. Спецификации UDDI (universal description, discovery and integration specification,
спецификация универсального описания, обнаружения и интеграции 
web-сервисов)~\cite{11bos} задают способ публикации и поиска информации о web-сервисах
путем описания конкретных функций, предоставляемых этим сервисом. Описание
функций, в свою очередь, оформляется на языке WSDL (web-service definition language,
язык описания web-сервисов)~\cite{12bos}.

Эта тройка, собственно, и сделала возможным создание реальных информационных
порталов, обеспечивающих возможности доступа посредством унифицированного
web-интерфейса к множеству разнородных данных и услуг, формируемых в федеративной
среде портала множеством <<внешних>> ресурсов. В результате применения этих
технологий в портальных решениях получается открытая система, позволяющая другим
информационным системам находить и использовать конкретную услугу посредством
стандартного механизма web-сервисов.

Наконец, в связи с web-сервисами обязательно следует отметить, а разработчикам~---
использовать, еще ряд сопутствующих стандартов, разработанных с целью дальнейшего
повышения интероперабильности систем, основанных на web. Эти стандарты
подготовлены сообществом Web Services Interoperability Organization (WS-I,
{\sf www.ws-i.org}) и касаются вопросов:
\begin{itemize}
\item поддержки транзакционности (WS-Trans\-action) \cite{13bos};
\item безопасности передачи сообщений (WS-Security) \cite{14bos};
\item гарантированной доставки сообщений (WS-Reliable-Messaging)~\cite{15bos}.
\end{itemize}

\section{Некоторые решения} %4

На данный момент трудно сколько-нибудь полно описать все проекты, выполненные в
разных странах в рамках реализации концепции <<электронного правительства>>. Здесь
перечислены только некоторые решения, интересные либо с исторической точки зрения
(первые решения), либо с точки зрения информационного и сервисного содержания.

По-видимому, первыми Ин\-тер\-нет-ре\-сур\-са\-ми, получившими общегосударственный статус,
можно считать MAXI ({\sf http://www.maxi.com.au/}), созданный в австралийском штате
Виктория, и \mbox{eCitizen} Centre ({\sf http://www.ecitizen.gov.sg/}), созданный в Сингапуре.
Австралийцы, видимо, наиболее актив\-но продвигают идеи <<электронного
правительства>> в коммерческую среду (см., например, австралийский государственный
информационный биз\-нес-пор\-тал BEP (business entry point) {\sf
http://www.\linebreak business.gov.au}).

Значительное число успешных проектов реализовано в США. Вот только несколько
примеров: {\sf www.cio.gov}~--- Совет директоров по информационным технологиям
США; {\sf www.gao.gov}~--- ресурс Ад\-ми\-ни\-ст\-ра\-тив\-но-бюд\-жет\-но\-го управления; {\sf
www. nascio.org}~--- Национальная ассоциация ИТ-ди\-рек\-то\-ров.

Представляется, что наиболее комплексно и сис\-тем\-но к реализации проектов
<<электронного правительства>> подходят в Великобритании. Координирующей группой
UK GovTalk выполняется проект с широким числом участников, цель которого~---
разработка согласованных стандартов (схем и структур данных), объединяющих
многочисленные разрозненные системы государственного сектора на основе XML. На
сайте {\sf http://www.govtalk.gov.uk/} содержатся проекты схем данных и согласованные
схемы, соответствующие требованиям и рабочим процедурам государственных органов
Великобритании, рекомендации по их практическому использованию и инструментальные
средства. Практическая реализация этой де\-я\-тель\-ности представлена, например,
правительственным шлюзом {\sf http://www.gateway.gov.uk/}, где в режиме онлайн могут
быть получены услуги государственных департаментов сельского хозяйства и занятости
(DARDNI и DEFRA).

В Европе рядом аналогичных проектов по разработке XML-схем, ориентированных на
об\-служива\-ние граждан, руководит Европейская комиссия (программа eEurope, стандарты
опубликованы на {\sf http://www.e-europestandards.org}). В качестве дейст\-ву\-юще\-го и
востребованного примера можно при\-вес\-ти портал {\sf http://www.eu-careers.com/},
предо\-став\-ля\-ющий услуги при поиске и подготовке к трудо\-устрой\-ст\-ву в европейских
странах.

Имеются и успешно действующие межгосударственные проекты. Любопытный пример в
этой области оформился в результате деятельности правоохранительных сообществ США:
открытый ресурс {\sf http://it.ojp.gov}, созданный по инициативе американского
Департамента юстиции активно используется и, главное, пополняется специалистами
правоохранительной сферы всего мира.

Аналогичных ресурсов в Интернете на данный момент огромное множество. К
странам-лидерам в области реализации концепции следует отнести Австралию,
Германию, Великобританию, Испанию, США, Швецию, Тайвань, Сингапур
и~др.~\cite{16bos}.

Позиции Российской Федерации в области <<электронного правительства>> хотя и нельзя
называть лидирующими, но определенные небезуспешные результаты у нас есть.
Отметим, прежде всего, что основная деятельность в России в этой сфере достаточно
централизованно координируется в рамках государственной программы <<Электронная
Россия>>. Так, на данный момент можно считать реализованным этап программы, в
рамках которого предполагалось создание группы сайтов, публикующих информацию
органов власти~--- федеральных министерств и ведомств. В типовой набор информации,
представленной на таком сайте, входят:
\begin{itemize}
\item основные ведомственные нормативные документы;
\item справочная информация (расположение, расписание работы, персоналии);
\item актуальные материалы информационно-но\-вост\-но\-го характера;
\item проекты ведомственных документов;
\item рекомендации физическим или юридическим лицам по решению их типичных
проблем.
\end{itemize}

О завершении этого первого этапа объявлялось еще в 2000~г.~\cite{4bos, 5bos}, хотя и
на данный момент далеко не все даже федеральные органы имеют свои представительства
в Интернете. Наиболее удачные и информационно полные ресурсы:  официальный сайт
президента РФ {\sf www.kremlin.ru}, сайт правительства РФ {\sf www.government.ru},
сайты Совета Федерации {\sf www.council.gov.ru}, Государственной думы {\sf
www.duma.gov.ru}, Совета безопас\-ности {\sf www.scrf.gov.ru}, Счетной палаты {\sf
www.ach.gov.ru} и~др. Важным результатом <<Электронной России>> является
собственный сайт программы~--- {\sf www.e-rus.ru}, выполняющий начальные функции
шлюза <<электронного правительства>>. На данный момент на этом сайте можно найти
ссылки на сайты практически всех федеральных министерств и ведомств.

Можно также отметить, что имеются и успешные примеры реализации второго этапа
концепции <<электронного правительства>>, а именно предо\-став\-ле\-ние населению
некоторых услуг. Два хорошо известных россиянам примера этого представляют сайт
Федеральной налоговой службы ({\sf www.nalog.ru}) и официальный сайт президента РФ.
Так, использование Интернета налоговой службой существенно облегчило для граждан
прохождение процедуры подачи налоговых деклараций, а услуги президентского сайта
позволяют участвовать в ежегодном интерактивном общении с Президентом РФ. 
В~настоящее время осуществляются и другие попытки создания обратной связи с
гражданами вплоть до создания правительственного портала~\cite{17bos}. При этом
постепенно просматривается тенденция перехода на более высокий уровень сервиса в
сравнении с информационно-справочным. К характерным примерам, иллюстрирующим
уровень интерактивности Интернет-ресурсов государственных органов в РФ, следует
отнести: сайт {\sf www.government.ru} (позволяет пользователю обратиться в
правительство РФ), сайт {\sf www.ach.gov.ru} (обращения в Счетную палату РФ), сайт {\sf
www.mvdinform.ru} (обращения в Министерство внутренних дел). При этом
предоставляемый сервис ограничивается возможностью отправки сообщения, т.\,е.\ о
полноценной обратной связи говорить пока рано.

Важнейшим примером тенденции смены этапа <<электронного правительства>> в РФ
является сайт государственных закупок {\sf www.zakupki.gov.ru}, заработавший в полном
объеме в прошлом году в рамках исполнения Федерального закона от 21~июля 2005~г.\
№\,94-ФЗ <<О размещении заказов на поставки товаров, выполнение работ, оказание
услуг для государственных и муниципальных нужд>>. Этот сайт реализует
пользовательский интерфейс уже полнофункционального портального решения,
предоставляя весь необходимый спектр услуг, обеспечивающий выполнение Закона о
государственных закупках. Его услугами пользуются все имеющие бюджетное
финансирование организации, а само его наличие является существенным инструментом
обеспечения открытости информации об использовании бюджетных средств.

Примерами успешной реализации ведомственных порталов являются Информационный
web-пор\-тал Российской академии наук {\sf www.ras.ru}, ресур\-сы Российского фонда
фундаментальных
исследований {\sf www.rfbr.ru}, портал <<Почта России>> {\sf
www.russianpost.ru}.

В целом, однако, говоря о всей совокупности Интернет-проектов органов государственной
влас\-ти в РФ, приходится констатировать отставание от общемировых тенденций:
перехода от <<информационного>> этапа концепции <<электронного правительства>> к
этапу <<услуг>> в государственном масштабе пока не происходит. Любопытно отметить,
что <<коммерческий>> сектор российского Интернета уже ни в чем не уступает ведущим
мировым лидерам. В этой связи надо сказать, что рассмотренные выше проблемы
реализации концепции в РФ все-таки решаются: имеется необходимый уровень
понимания со стороны государственных чиновников, достаточно высок уровень
финансирования, созданы центры государственной компетенции. Проблемными остаются
два направления:
\begin{itemize}
\item из-за все еще ограниченных возможностей досту\-па широкого круга граждан к
средст\-вам коммуникаций эффективность реализуемых
государством
Ин\-тер\-нет-про\-ек\-тов остается не\-вы\-сокой;
\item затруднения возникают в технической со\-став\-ля\-ющей~--- сложным остается
формирование квалифицированной команды технических специалистов высокого
уровня, нехватка которых имеется в стране в целом.
\end{itemize}

\section*{Заключение}

Успешная реализация поддерживаемой всем мировым сообществом концепции
<<электронного правительства>> является необходимым условием создания
информационного общества или, как час\-то говорят, информационной экономики.
Текущее состояние этой реализации в целом соответствует этапности, определенной для
концепции ИТ-со\-об\-ще\-ст\-вом в конце прошлого века,
подтверждают жизнеспособность
<<электронного правительства>>, соответствие получаемых результатов высказываемым
ранее ожиданиям, хотя и с некоторым экстенсивным оттенком. Сформулированные выше
задачи <<электронного правительства>>, проблемы реализации проектов, требования к
уровням и способам достижения поставленных целей и технологический базис позволяют
эффективно решать практически все задачи, определенные концепцией. Движущей силой
для технической составляющей реализации концепции
 являются стандарты и технологии
{\sf world-wide web}, сформировавшиеся к настоящему времени в законченном виде, и
использующие их портальные технологии. Именно порталы являются ключевым звеном
большинства проектов, реализуемых в рамках <<электронного правительства>>, и от
успешности дальнейшего проникновения портальных решений в проекты <<электронного
правительства>> в значительной степени зависит успех всего направления в целом.

{\small\frenchspacing
{%\baselineskip=10.8pt
\addcontentsline{toc}{section}{Литература}
\begin{thebibliography}{99}
\bibitem{1bos}
Государство в XXI~веке. Электронный документооборот и делопроизводство~//
Информационный бюллетень Microsoft. Специальный выпуск, 2003.

\bibitem{2bos}
Государство в XXI~веке. Материалы конференции для государственных деятелей~//
Информационный бюллетень Microsoft. Выпуск~18. Декабрь 2002~г.

\bibitem{3bos}
Realizing the vision: 2002 Global accenture study on eGovernment~// Бюллетень Microsoft
Insight~--- Government.  2002. {\sf
http://www.microsoft.com/europe/ insight/Government/Analyst\_insights/item117.htm}.

\bibitem{4bos}
Правительство--население: диалог в виртуальном пространстве~// Всероссийская
объединенная конференция <<Интернет и современное общество>>, ноябрь 2000. {\sf
http://ims2000.nw.ru/src/TEXT50.HTML}.

\bibitem{5bos}
Кабинет Касьянова обещает облегчить к себе доступ: Интервью с руководителем
департамента правительственной информации правительства РФ А.~Коротковым~//
<<Новая газета>>, август, 2000.
{\sf http://2000.novayagazeta.ru/nomer/2000/36n/n36n-s21.shtml}.

\bibitem{6bos}
Государство в XXI~веке. Интеграция государственных информационных систем и
организация межведомственного взаимодействия~// Информационный бюллетень
Microsoft. Выпуск~21. Сентябрь, 2003.

\bibitem{7bos}
Extensible Markup Language (XML) 1.0 (4th ed.). {\sf
http://www.w3.org/TR/2006/REC-xml-20060816}.

\bibitem{8bos}
SOAP Version 1.2 Part 1: Messaging Framework (2nd~ed.). {\sf
http://www.w3.org/TR/soap12-part1}.

\bibitem{9bos}
Web Services Glossary. {\sf http://www.w3.org/TR/2004/ NOTE-ws-gloss-20040211}.

\bibitem{10bos}
Web Services Architecture.  {\sf http://www.w3.org/TR/}\linebreak 
{\sf ws-arch}.

\bibitem{11bos}
Universal Description, Discovery and Integration specification.
{\sf www.uddi.org}.

\bibitem{12bos}
Web Services Description Language (WSDL) Version~2.0 Part~1: Core Language. {\sf
http://www.w3.org/TR/wsdl20}.

\bibitem{13bos}
Web Services Transactions specifications.
{\sf http://www-128.ibm.com/developerworks/library/specification/ws-tx}.

\bibitem{14bos}
Web Services Security.
{\sf http://www-128.ibm.com/ developerworks/library/specification/ws-secure}.

\bibitem{15bos}
Web Services Reliable Messaging.
{\sf http://www-128. ibm.com/developerworks/library/specification/ws-rm}.
\bibitem{16bos}
Государство в XXI~веке. Реализация проектов электронного правительства~//
Информационный бюллетень Microsoft. Выпуск~19. Март, 2003.
\bibitem{17bos}
\Au{Шеян И.} Ворота во власть~// Computerworld, 2003. No.\,1. Изд-во <<Открытые
Системы>>. {\sf http:// www.osp.ru/cw/2003/01/025\_2.htm}.
\end{thebibliography}

}
}

\end{multicols}

\label{end\stat}