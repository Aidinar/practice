\def\stat{zahar}

\newcommand{\il}[2]{\int\limits_{#1}^{#2}}%интеграл с пределами #1 и #2
\newcommand{\suml}[2]{\sum\limits_{#1}^{#2}}   %сумма
\newcommand{\maxl}[1]{\max\limits_{#1}}
\newcommand{\minl}[1]{\min\limits_{#1}}
\newcommand{\liml}[1]{\lim\limits_{#1}}
\newcommand{\limi}[1]{\liminf\limits_{#1}}
\newcommand{\lims}[1]{\limsup\limits_{#1}}
\newcommand{\cupl}[2]{\bigcup\limits_{#1}^{#2}}

\newcommand{\p}{\sf P}  % вероятность
\newcommand{\e}{\sf E}  % мат. ожидание
\newcommand{\D}{\sf D}  % дисперсия

\newcommand{\rr}{\mathbb{R}^2}     % поле действительных чисел, плоскость
\def\P{\mathop{\bf P}}
\def\limn{\lim\limits_{n\to\infty}}

\newcommand{\proofbegin}{{\sc Доказательство.}}
\newcommand{\proofend}{{\hfill$\Box$}}
\newcommand{\sqq}{\hbox{\vrule\vbox{\hrule\phantom{o}\hrule}\vrule}}


\def\tit{РАЗМЕЩЕНИЯ СИСТЕМ МАССОВОГО ОБСЛУЖИВАНИЯ, МИНИМИЗИРУЮЩИЕ
СРЕДНЮЮ ДЛИНУ ОЧЕРЕДИ}
\def\titkol{Размещения систем массового обслуживания, минимизирующие
среднюю длину очереди}

\def\autkol{Т.\,В.~Захарова}
\def\aut{Т.\,В.~Захарова$^1$}

\titel{\tit}{\aut}{\autkol}{\titkol}

{\renewcommand{\thefootnote}{\arabic{footnote}}} \footnotetext[1]{Московский 
государственный университет им.~М.\,В. Ломоносова, факультет вычислительной 
математики и кибернетики, кафедра математической статистики, lsa@cs.msu.su}

\Abst{В статье рассмотрен класс систем массового обслуживания с
вызовами, возникающими на плоскости. Решается задача оптимизации
расположения станций обслуживания по критерию средней суммарной
длины очереди. Проводится сравнение оптимальных расположений по
критерию средней суммарной длины очереди и критерию среднего
суммарного времени ожидания.}

\KW{система массового обслуживания; вызовы на
плоскости; длина очереди; время ожидания; оптимизация
расположения}

      \vskip 24pt plus 9pt minus 6pt

      \thispagestyle{headings}

      \begin{multicols}{2}

      \label{st\stat}

\section{Введение}

В статье рассмотрен класс систем массового обслуживания с вызовами, возникающими на 
плос\-кости.
Спецификой изучаемого класса систем является необходимость использования информации о
положении обслуживающих приборов, положении поступающих вызовов, их плотности распределения.
Такие модели систем массового обслуживания служат для изучения реальных систем, где
обслуживание производится территориально расположенными объектами.

Как наилучшим образом расположить эти объекты?

В данной статье рассматривается проблема оптимизации расположения станций обслуживания по
критерию средней суммарной длины очереди. Проводится сравнение оптимальных расположений по
критерию средней суммарной длины очереди и критерию среднего
суммарного времени ожидания.

Станции функционируют как независимые сис\-те\-мы массового обслуживания типа $M\vert G\vert 1$.
Описываются свойства оптимальных размещений и приводятся алгоритмы построения асимптотически
оптимальных размещений.

\section{Постановка задачи}

На плоскости $\rr$ возникают
требования в точках, являющихся реализациями некоторого случайного
вектора $\xi$, распределенного с плотностью распределения $p$. Для
обслуживания этих требований имеется $n$ станций. Моменты
поступления требований образуют пуассоновский поток с параметром~$\lambda$. 
Интенсивность входящего потока $\lambda$ изменяется с
ростом числа станций $n$. В случае, когда нужно подчеркнуть эту
зависимость, параметр входящего потока будем обозначать~$\lambda(n)$.

\textbf{Определение~1.} Размещением $n$ станций обслуживания на плоскости $\rr$ назовем
множество точек плоскости $\{x_1,\ldots,x_n\}$, в которых они расположены.

Обозначать размещение станций будем символом $x$, то есть $x=\{x_1,\ldots,x_n\}$. Станцию
обслуживания и точку плоскости, где она расположена, будем обозначать одним и тем же символом.

\textbf{Определение~2.} Зоной влияния станции $x_i$ назовем множество $C_i$ тех точек
плоскости, для которых эта станция является ближайшей, т.\,е.\
$$
C_i=\{v\in\rr:\vert v-x_i\vert \leqslant\vert v-x_j\vert\,,\quad j=1,2,\ldots,n\}\,.
$$

Расстояние $\vert u-v\vert $ между точками $u$ и $v$ 
плос\-кости~$\rr$ задается евклидовой нормой.

Станции обслуживают заявки только из своих зон влияния. Обслуживание осуществляется прибором,
двигающимся только по прямой и с постоянной скоростью. При поступлении заявки прибор со
станции перемещается в точку вызова, заявка обслуживается прибором некоторое случайное 
время~$\eta$, затем прибор возвращается обратно на станцию. Дисциплина обслуживания вызовов
следующая: если в момент поступления вызова прибор занят, то поступающий вызов ставится в
очередь. После освобождения прибора первая заявка из очереди поступает на обслуживание.

Обозначим через $\lambda_i$ интенсивность потока вызовов, поступающих на станцию $x_i$;
$\beta_{i1}$ и $\beta_{i2}$~--- соответственно первый и второй моменты времени обслуживания на
$x_i$.

Предполагается, что станции функционируют как независимые системы
массового обслуживания типа $M\vert G\vert 1$. Тогда их средняя суммарная
длина очереди $L(x)$ при размещении $x$ 
определяется по формуле
\begin{equation*}
L(x)=\sum\limits_{i=1}^{n}\fr{\lambda_i^2}{2}\, \fr{\beta_{i2}}{1-\lambda_i \beta_{i1}}
\end{equation*}
при условии, что загрузка каждой станции меньше единицы, т.\,е.\
$\maxl{1\leqslant i\leqslant n} \lambda_i \beta_{i1}< 1 $.

Задача заключается в нахождении размещений, минимизирующих введенный критерий оптимальности
$L(x)$.

В работе~\cite{1z} рассматривалась аналогичная постановка задачи,
но для систем с малой загрузкой и с более жесткими условиями на
плотность распределения $p$.

\textbf{Определение~3.} Размещение $x^*$ назовем оптимальным, если
$L(x^*)\leqslant L(x)$ для
любого размещения $x$ такого, что $\vert x\vert =\vert x^*\vert$.
Через $\vert x\vert$ здесь обозначено число элементов
размещения $x$.

\textbf{Определение~4.} Правильной $n$-решеткой назовем покрытие плоскости конгруэнтными
правильными $n$-угольниками, пересекающимися лишь по границе. Фундаментальным регионом~---
правильный $n$-угольник.

Введем следующие обозначения:
\begin{align*}
f(k)&=\il{M_k}{}\vert u\vert\,du\,;\\
g(k)&=\il{M_k}{}\vert u\vert^2\,du\,,
\end{align*}
где $M_k$~--- правильный $k$-угольник единичной площади с центром в нуле;
\begin{align*}
\vert p\vert_m & =\left( \int p^m(u)du\right )^{1/m}\,;\\
\e\eta &= \beta_1\,; \quad \e\eta^2=\beta_2\,;\\
x_A^*& =x^*\cap A;
\end{align*}
$\{x\}$~--- последовательность размещений.

\section{Основные результаты}

В следующей теореме описываются асимптотические свойства оптимальных размещений для исходной
модели.

\smallskip
{\bf{Теоpема 1.} } {\it Если плотность $p$ ограничена и
интегрируема по Лебегу,} $\e\vert \xi\vert^2<\infty$, {\it а
интенсивность входящего потока $\lambda(n)$ изменяется так, что}

\noindent
$$
\limn\fr{\lambda(n)}{n}=\fr{\rho}{\beta_1}\,,\quad \rho\in [0,1)\,,
$$
{\it то для всякой последовательности оптимальных размещений
$\{x^*\}$ справедливы равенства:}
\begin{center}
\begin{tabular}{lcl}
&(1)&$\liml{n\rightarrow \infty}\fr{n}{\lambda^2(n)}\,L(x^*)=0{,}5\beta_2(1-\rho)^{-1}\,;$\\
\\
&(2)&$\liml{n\rightarrow\infty}\fr{\vert x_A^* \vert }{n}=\il{A}{}p(u)\,du\,,$
\end{tabular}
\end{center}
{\it где $A$~--- произвольное измеримое по Лебегу множество.}

%\bigskip
Рассмотрим модель, в которой обслуживание состоит только
лишь в перемещении прибора со станции до точки, где возникло
требование, и обратно. В этом случае оптимальные размещения
обладают другими свойствами.

\smallskip
{\bf{Теоpема 2.} } {\it Если плотность $p$ ограничена,
интегрируема по Лебегу,} ${\e}\vert \xi\vert^2<\infty$ {\it и интенсивность
входящего потока требований изменяется так, что}
$$
\limn\fr{\lambda(n)}{n^{3/2}}=\fr{\rho}{2f(6)\vert p\vert_{2/3}}\,, \quad
\rho\in [0,1)\,,
$$
{\it тогда для всякой последовательности оптимальных размещений} $\{x^*\}$

\begin{center}
\begin{tabular}{lcl}
&(1)&$\limn\fr{n^2}{\lambda^2(n)}L(x^*)=2g(6)\vert p\vert_{2/3}^2(1-\rho)^{-1}\,;$\\
\\
&(2)&$\limn\fr{\vert x_A^*\vert }{n}=\vert p\vert_{2/3}^{-2/3}\il{A}{}p^{2/3}(u)\,du$
\end{tabular}
\end{center}
{\it для любого измеримого по Лебегу множества} $A$.

В работе~\cite{2z} исследовались свойства оптимальных размещений по
критерию среднего суммарного времени ожидания 
$$
W(x)=\fr{1}{2n}\,\suml{i=1}{n}\fr{\lambda_i\beta_{i2}}
{1-\lambda_i\beta_{i1}}
$$
в аналогичной постановке задачи.

В данной статье также рассматривается проблема оптимизации расположения станций обслуживания
по критерию $W(x)$, но с более мягкими условиями на плотность распределения $p$.

\smallskip
{\bf{Теоpема 3.} } {\it Пусть плотность $p$ ограничена}, $p^{2/3}$ {\it интегрируема по
Лебегу}, $\e\vert \xi\vert^2<\infty$ {\it и} $\lambda_n=o(n^{1/2})$,
{\it тогда для всякой
последовательности оптимальных размещений} $\{x^*\}$

\begin{center}
{\tabcolsep=0pt
\begin{tabular}{lcl}
&(1)&$\liml{n\rightarrow
\infty}\sqrt{n}\left(\fr{n}{\lambda_n}\,W(x^*)-0{,}5\beta_2\right)
=2 f(6)~\beta_1 \vert p\vert_{2/3}\,;$\\
\\
&(2)&$\liml{n\rightarrow\infty}\fr{\vert x_A^*\vert }{n}=
\vert p\vert_{2/3}^{-2/3}\il{A}{}p^{2/3}(u)\,du\,,$
\end{tabular}
}
\end{center}
{\it где $A$~--- произвольное измеримое по Лебегу множество}.


В случае, когда $\e \eta=0$, т.\,е.\ обслуживание заключается лишь в перемещении прибора до
вызова и обратно, справедлива следующая теорема.

\smallskip
{\bf{Теоpема 4.} } {\it Если плотность $p$ ограничена}, $p^{1/2}$ {\it интегрируема по
Лебегу}, ${\e}\vert \xi\vert^2<\infty$ и $\lambda_n=o(n^{1{,}5})$, {\it тогда для любой
последовательности оптимальных размещений} $\{x^*\}:$

\begin{center}
\begin{tabular}{lcl}
&(1)&$\liml{n\rightarrow
\infty}\fr{n^2}{\lambda_n}\,W(x^*)=2 g(6) \vert p\vert _{1/2}\,;$\\
\\
&(2)&$\liml{n\rightarrow\infty}\fr{\vert x_A^*\vert }{n}=
\vert p\vert_{1/2}^{-1/2}\il{A}{}p^{1/2}(u)\,du$
\end{tabular}
\end{center}
{\it для любого измеримого по Лебегу множества} $A$.

\bigskip
{\bf{Замечание.}} Размещение, являющееся оптимальным по критерию $W$ (стационарное
среднее суммарное время ожидания начала обслуживания), не минимизирует критерий $L$
(стационарная средняя суммарная длина очереди всех станций).

\section{Доказательство основных результатов}

Приведем только доказательство теоремы 1, доказательства остальных теорем проводятся
аналогичными методами.


Для любой станции $x_i$ с зоной влияния $C_i$ некоторого
размещения $x$ первые два момента времени обслуживания оцениваются
как
\begin{align*}
\beta_{i1} & ={\e} \left ( \fr{2\vert \xi-x_i\vert +\eta}
{\xi}\in C_i\right ) \geqslant
\e\eta =\beta_1\,;\\
\beta_{i2}& = {\e} \left ( \fr{\left (2\vert \xi-x_i\vert +\eta\right )^2}{\xi}\in
C_i\right )\geqslant \e\eta^2=\beta_2\,,
\end{align*}
 а интенсивность потока
поступающих на нее требований есть $\lambda_i=\lambda(n) {\p} (C_i)$.

С учетом этого, а также используя выпуклость функции
$f(u,v)=u^2(1-v)^{-1}$, оценим снизу критерий $L(x)$:
%\noindent
\begin{multline*}
L(x) =\suml{i=1}{n}\fr{\lambda_i^2}{2}\,\fr{\beta_{i2}}
{1-\lambda_i \beta_{i1}}\geqslant {}\\[-12pt]
\end{multline*}
\begin{multline*}
{}\geqslant 0{,}5\lambda^2\beta_2\suml{i=1}{n}
\fr{{\p}^2(C_i)} {1-\lambda \beta_1 {\p}(C_i)}\geqslant{}\\
{}\geqslant 0{,}5\lambda^2\beta_2
\fr{1}{n}\left(\suml{i=1}{n}{\p}(C_i)\right)^2\times \\
\times
 \left(1-\lambda
\beta_1 \fr{1}{n}\,\suml{i=1}{n}{\p}(C_i)\right)^{-1}\,,
\end{multline*}
 т.\,е.
$$
\fr{n}{\lambda^2(n)}\,L(x^*)\geqslant
0{,}5\beta_2\left ( 1-\fr{\lambda \beta_1}{n}\right )^{-1}\,.
$$
Устремляя $n$ к бесконечности, получим
$$
\limi{n \to \infty}\fr{n}{\lambda^2(n)}\,L(x^*)\geqslant
0{,}5\beta_2(1-\rho)^{-1}\,.
$$

Предположим сначала, что плотность $p$~--- прос\-тая функция, т.\,е.\
$p=\suml{j=1}{r}p_j \textbf{1}_{K_j}$,
где $K_j$~--- измеримые по
Лебегу непересекающиеся множества на плоскости.

Для такой плотности построим асимптотически оптимальное размещение
$x$, т.\,е.\ такое, что при $\vert x\vert =\vert x^*\vert$ выполняется равенство
$$
\limn \fr{L(x)}{L(x^*)}=1\,.
$$

Для каждого множества $K_j$ выберем элементарное множество $L_j$
так, что $\mu(K_j \Delta L_j)<\varepsilon$, где $\varepsilon$~---
некоторое произвольное положительное число, затем $L_j$ покрываем
правильной 6-решеткой с площадью фундаментального региона
$\sigma_j=K_j/m_j$, где  $m_j=m (1-\delta) p_jK_j$,
$m$~--- некоторое натуральное число и $0<\delta<1$.

Если $\mu(\sigma_j\cap L_j)>0$, то в центры таких $\sigma_j$
помещаем станцию обслуживания. Число таким образом размещенных
станций обозначим через~$n_j$. Построим последовательность
вложенных расширяющихся квадратов $K$ с центром в нуле, которые
удовлетворяют условию ${\e}\vert \xi\vert^2 \textbf{1}_{K^c}=o(m^{-1})$.
$[m\delta]+1$ станцию разместим равномерно на множествах
$(K_i\backslash L_i)\cap K$. Общее число размещенных станций
обозначим через $n$. Тем самым получено некоторое размещение
$x=\{x_1,\ldots,x_n\}$, для которого
\begin{multline*}
L(x)\leqslant\\
\leqslant\suml{j=1}{r}n_j\fr{\lambda^2 p_j^2 \sigma_j^2}{2}\,
\fr{\beta_2+4\beta_1 f(6)\sigma_j^{1/2}+4g(6)\sigma_j}
{1-\lambda \beta_1 p_j \sigma_j-2\lambda f(6)p_j \sigma_j^{3/2}}={}\\
{} =\fr{\lambda^2}{2}\suml{j=1}{r}\, \fr{\beta_2 p_j K_j
(n(1-\delta))^{-1}+o(n^{-1})}{1-\lambda \beta_1
(n(1-\delta))^{-1}+o(n^{-1})} +{}\\
{}+O\left(\fr{\varepsilon}{n~\delta}\right)\,.
\end{multline*}
 Устремим
$\varepsilon$, а затем и $\delta$ к нулю, тогда
$$
\lims{n \to \infty}\fr{n}{\lambda^2(n)}\,L(x)\leqslant
0{,}5\beta_2(1-\rho)^{-1}\,.
$$
Так как всегда $L(x^*)\leqslant L(x)$,
то с учетом нижней оценки для $L(x^*)$ получаем, что $x$~---
асимптотически оптимальное размещение и
$$
\liml{n \to   \infty}\fr{n}{\lambda^2(n)}\,L(x^*) =
0{,}5\beta_2(1-\rho)^{-1}\,.
$$
Пусть $p$~--- произвольная функция, удовлетворяющая условиям
доказываемой теоремы. Введем простые функции $\bar{p}_k(u)$ по
правилу:
$$
\bar{p}_k(u)=\fr{m+1}{k}\,,
$$ 
если $m/k<p(u)\leqslant (m+4)/k,$ для
$k\in \cal{N}$, $m=0$, 1,~\ldots

Очевидно, что $p(u)\leqslant \bar{p}_k(u)$, а для простых
функций выше была получена предельная оценка сверху, поэтому
справедливо следующее неравенство
$$
\lims{n \to
\infty}\fr{n}{\lambda^2(n)}\,L(x^*)\leqslant 0{,}5\beta_2
\vert \bar{p}_k \vert_1 (1-\rho)^{-1}\,.
$$
При $k \to \infty$
$\vert \bar{p}_k \vert_0 \to \vert p\vert_1=1$.
И с учетом
оценки снизу для $L(x^*)$ получаем, что
$$ \liml{n \to
\infty}\fr{n}{\lambda^2(n)}\,L(x^*) = 0{,}5\beta_2(1-\rho)^{-1}\,.
$$

\smallskip
Докажем второй пункт теоремы~1.

Рассмотрим размещение $x_A=x\bigcap A$. Пусть $k=\vert x_A\vert $~---
число станций, попадающих в множество $A$ при размещении $x$.
Определим $L(x_A)$ как $L(x)$ для $p$\textbf{1}$_A$:
\begin{multline*}
L(x_A)=\suml{i=1}{k}\fr{\lambda_i^2}{2}\, \fr{\beta_{i2}}
{1-\lambda_i \beta_{i1}}\geqslant{}\\
{}\geqslant  0{,}5\lambda^2\beta_2\suml{i=1}{k}
\fr{{\p}^2(C_i\bigcap A)} {1-\lambda \beta_1 {\p}(C_i\bigcap
A)}\geqslant {}\\
{} \geqslant 0{,}5\lambda^2\beta_2
\fr{1}{k}\left(\suml{i=1}{k}{\p}(C_i\bigcap A)\right)^2\times\\
\times 
\left(1-\lambda \beta_1 \fr{1}{k}\suml{i=1}{k}{\p}(C_i\bigcap
A)\right)^{-1}\,,
\end{multline*}
значит,
$$
\fr{k}{\lambda^2(n)}L(x_A)\geqslant
0{,}5\beta_2{\p}^2(A)\left( 1-\fr{\lambda \beta_1
{\p}(A)}{k}\right)^{-1}\,.
$$
Следовательно,
\begin{equation*}
\limi{n \to
\infty}\fr{k}{\lambda^2(n)}\,L(x_A)\geqslant
0{,}5\beta_2{\p}^2(A)\left(1-
{\p}(A)\fr{\rho}{\gamma_1}\right)^{-1}\,,
\end{equation*}
где $\gamma_1$~---
предельная точка последовательности $\left\{\vert x_A\vert n^{-1}
\right\}$.

\smallskip
Пусть теперь $x$~--- асимптотически оптимальное размещение для
критерия $L$. Тогда
$$ 
\lims{n \to
\infty}\fr{n}{\lambda^2(n)}\,L(x_A) \leqslant
0{,}5\beta_2{\p}(A)(1-\rho)^{-1}\,.
$$
Из двух последних неравенств
следует, что
\begin{equation*}
 \gamma_1\geqslant \fr{{\p}^2(A)}{1-
{\p}(A)\rho\gamma_1^{-1} }\, \fr{1-\rho}{{\p}(A)}=
\fr{\gamma_1
{\p}(A)(1-\rho)}{\gamma_1-{\p}(A)\rho}\,,
\end{equation*}
т.\,е.
$$
\gamma_1-{\p}(A)\rho \geqslant {\p}(A)(1-\rho)  \Rightarrow
\gamma_1 \geqslant {\p}(A)\,.
$$

Пусть $\gamma_2$~--- предельная точка последовательности
$\left\{\vert x_B\vert n^{-1} \right\}$, где $B=A^c$.

Для $\gamma_2$ аналогично доказывается соответст\-ву\-ющее неравенство
$$
\gamma_2 \geqslant {\p}(B)\,.
$$
Так как
$$
1=\gamma_1+\gamma_2\geqslant {\p}(A)+{\p}(B)=1\,,
$$
то в неравенстве достигнуто равенство. Это возможно, только если
$$
\gamma_1={\p}(A)\,, \qquad \gamma_2={\p}(B)\,.
$$
Тем самым для
любого асимптотически оптимального размещения
$$
\liml{n\rightarrow\infty}\fr{\vert x_A\vert }{n}=\il{A}{}p(u)\,du\,,
$$
а значит, это равенство верно и для оптимального размещения.
Отсюда следует утверждение второго пункта теоремы~1.

{\small\frenchspacing
{%\baselineskip=10.8pt
\addcontentsline{toc}{section}{Литература}
\begin{thebibliography}{9}

\bibitem{1z}
\Au{Назаров Л.\,В., Смирнов~С.\,Н.}
Обслуживание вызовов, распределенных в пространстве~//
Изв. АН СССР. Сер.\ Техническая кибернетика, 1982. №\,1. С.~95--99.
\bibitem{2z}
\Au{Захарова Т.\,В.}
Оптимизация расположения станций
обслуживания на плоскости~// Изв. АН СССР. Сер. Техническая кибернетика,
1987. №\,6. С.~83--91.

\end{thebibliography}


\label{end\stat}

}
}

\end{multicols}