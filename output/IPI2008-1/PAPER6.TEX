\def\stat{sokolov}

\def\tit{НОВЫЙ ЭТАП ИНФОРМАТИЗАЦИИ ОБЩЕСТВА И АКТУАЛЬНЫЕ ПРОБЛЕМЫ
ОБРАЗОВАНИЯ}
\def\titkol{Новый этап информатизации общества и актуальные проблемы
образования}

\def\autkol{И.\,А. Соколов, К.\,К.~Колин}
\def\aut{И.\,А. Соколов$^1$, К.\,К.~Колин$^2$}

\titel{\tit}{\aut}{\autkol}{\titkol}


\renewcommand{\thefootnote}{\arabic{footnote}}
\footnotetext[1]{Институт проблем информатики Российской академии наук, isokolov@ipiran.ru}
\footnotetext[2]{Институт проблем информатики Российской академии наук, kolinkk@mail.ru}

\Abst{Анализируются основные черты и особенности современного этапа процесса
информатизации общества и связанные с этим актуальные проблемы модернизации образования.
Показано, что новые условия существования человека в глобальном информационном обществе
создают для него принципиально новые возможности и проблемы, которые еще недостаточно
учитываются в системе образования и поэтому требуют ее существенной модернизации.
Рассматривается состояние информационного общества в России и актуальные проблемы
российского образования, связанные с переходом страны к инновационной стратегии развития.}

\KW{глобальная информатизация общества; информационно-коммуникационные технологии;
информационное общество в России; новая парадигма образования; опережающее образование;
стратегия инновационного развития}

\vskip 24pt plus 9pt minus 6pt

      \thispagestyle{headings}

      \begin{multicols}{2}

      \label{st\stat}

\section{Введение}

   Одним из основных направлений развития цивилизации в XXI~в.\ является
глобальная информатизация общества на основе создания новых средств информатики и
информационно-ком\-му\-ни\-ка\-ци\-он\-ных технологий (ИКТ) и их все более широкого
использования во всех сферах социальной практики. Овладение информацией и знаниями,
а также обеспечение доступности социально значимой информации для широких слоев
населения~--- это две важнейшие задачи развития общества на ближайшие десятилетия.
Их решение является необходимым условием для развития и эффективного использования
человеческого потенциала, развития экономики, науки, образования и культуры, а также
для решения многих глобальных проблем современности, которые требуют объединенных
усилий мирового сообщества. К чис\-лу таких проблем можно отнести снижение остроты
существующего в настоящее время социального неравенства, решение многих проблем
занятости населения, повышение общего уровня образованности и культуры людей, а
также качества их жизни.

   Анализ показывает, что новые условия существования человека в глобальном
информационном обществе создают для него принципиально новые возможности и новые
проблемы, которые еще недостаточно учитываются в современной системе образования и
требуют ее существенной модернизации. Именно поэтому проблемы образования в
информационном обществе являются сегодня исключительно актуальными и достаточно
широко обсуждаются на страницах печати как в России, так и за рубежом. В статье
рас\-смат\-ри\-ва\-ет\-ся состояние и ближайшие перспективы развития информационного
общества в России и связанные с этим актуальные проблемы модернизации российского
образования.

\section{Отличительные особенности развития процесса
информатизации общества в~XXI~веке}

   События последнего десятилетия свидетельствуют о том, что процесс информатизации
общества в XXI~в.\ перешел на качественно новый этап своего развития. Он стал 
приобретать системный, скоординированный характер и направлен уже не только на 
распространение средств ИКТ в передовых странах и отдельных регионах мира, но 
также и на формирование глобального информационного общества, в котором 
создаются принципиально новые условия для жизни и деятельности человека, 
развития экономики, образования и науки. В~развитии этого процесса усиливается 
роль международных организаций, таких как ООН, ЮНЕСКО, ЮНЕП, ИФЛА и~др. Начиная 
с 2000~г., проблемы формирования информационного общества регулярно обсуждаются 
на авторитетных международных форумах с высоким уровнем представительства. В их 
работе принимают участие не только ученые и общественно-политические деятели, 
но также и руководители государств, международных организаций, члены 
парламентов и правительств различных стран мира. Активно участвует в этих 
форумах и Россия.

   Обсуждение наиболее актуальных и стратегически значимых проблем становления
информационного общества показало, что такими проблемами на ближайшие десятилетия
являются следующие:
   \begin{itemize}
\item проблема эффективного использования тех новых возможностей, которые
открывают достижения научно-технического прогресса в сфере ИКТ, в интересах
социально-эко\-но\-ми\-че\-ско\-го развития общества и эффективного использования
человеческого потенциала;
\item проблема информационного неравенства между людьми, странами и
отдельными регионами в активно формирующейся новой информационной среде
общества, которая порождает новые формы социального неравенства и во многих
случаях усиливает остроту этой проблемы;
\item проблема обеспечения информационной безопас\-ности человека, общества и
государства в условиях информационного общества;
\item проблема формирования новой информационной культуры общества и
новой парадигмы образования, которые были бы адекватными новым условиям
жизни и деятельности людей в информационном обществе.
\end{itemize}

   Общие подходы к решению этих проблем международным научно-образовательным
сообществом в основном уже выработаны, но каждая страна решает эти проблемы 
по-разному, с учетом своей специфики, национальных традиций, состояния экономики,
науки, образования и культуры. Россия здесь не является исключением. Поэтому процесс
развития информационного общества в нашей стране также имеет свои особенности,
которые в своих основных чертах будут рассмотрены ниже.

\section{Современное состояние процесса формирования
информационного общества в~России}

\vspace*{-3pt}
\paragraph*{Общая оценка.} По экспертным оценкам отечественных и зарубежных
специалистов, информационное развитие России идет достаточно быстрыми темпами.
Активно развивается отечественный рынок информационных и коммуникационных
технологий, продуктов и услуг. Продолжается компьютеризация многих отраслей
хозяйства, а также сфер государственного и регионального управления, создается
современная правовая и нормативная база в информационной сфере. В рамках
Федерального целевого проекта <<Образование>> осуществляется массовая
компьютеризация российских школ, в 2007~г.\ завершено их подключение к сети
Интернет. Функционирует созданный в 2006~г.\ Совет главных конструкторов
информатизации регионов России, который осуществляет координацию работ в этой
области, в том числе ряда проектов по созданию информационно-технологической
инфраструктуры страны.

   В 2007~г.\ общий объем отечественного рынка ИКТ вырос на 25,4\% и, по
предварительным оценкам, составил около 1,5~трлн руб. При этом объем услуг
электросвязи оценивается значением 983~млрд руб.\ (рост на 25,8\%), услуг 
почтовой связи~--- 67~млрд руб.\ (рост на 24,8\%), а объем рынка информационных 
технологий увеличился на 24,5\% и приблизился к значению 450~млрд руб. Наиболее 
быстрыми темпами (40,4\%) растет рынок про\-грам\-мных средств, объем которого 
в 2007~г.\ составил около 80~млрд руб. Рынок аппаратных средств составил 
252~млрд руб.\ (рост на 18,5\%), а рынок информационных услуг населению страны 
достиг значения 117,9~млрд руб.\ (рост на 28,4\%).

   По оценкам специалистов Мининформсвязи, приведенные выше показатели развития
отрасли ИКТ во многом обусловлены привлечением в нее отечественных инвестиций,
которые в 2007~г.\ составили 206~млрд руб., что на 20,8\% больше по сравнению с
инвестициями 2006~г. Тем не менее, в традиционном ежегодном рейтинге готовности
стран мирового сообщества к <<электронному развитию>>, который в апреле 2007~г.\
опубликован международной исследовательской компанией Economist Intelligence Unit
(EIU), входящей в состав Economist Group, Россия сегодня занимает только 57~место из
69~стран, обследованных с точки зрения их го\-тов\-ности внедрять новые информационные
технологии. Год назад Россия занимала в этом списке 52~место, а сегодня она уже
уступает Индии, Филиппинам и Китаю, которые занимают в данном рейтинге
соответственно 54, 55 и 56~места. Лидируют же в данном рейтинге Дания, США и
Швеция, а аутсайдерами являются Азербайджан и Иран~\cite{1ss}.

     В качестве главных причин отставания нашей страны в данной области зарубежные
эксперты называют происходящую в России модернизацию сис\-те\-мы образования и
административную реформу правительства, которая была проведена в 2004~г. С~этой
оценкой трудно не согласиться. Ведь проводимая в настоящее время в стране
государственная политика в сфере образования уже не рассматривает ее информатизацию
в качестве приоритетного направления, как это имело место ранее. В особенности это
касается содержательных аспектов образования, переподготовки педагогических кадров и
формирования новой информационной культуры общества, адекватной современным
требованиям формирующегося информационного общества.

\vspace*{-3pt}
\paragraph*{Новый этап развития информационного общества в России.} В
2007~г.\ руководством нашей страны был принят ряд важных организационных решений, %\linebreak
направ\-лен\-ных на формирование в России основ информационного общества.
Президентом РФ одоб\-ре\-на Стратегия развития информационного общества в России на
период до 2015~г.~\cite{2ss}. Ее реализация должна обеспечить повышение конкурентной
способности страны, эффективности сферы государственного управления, повышение
уровня и качества жизни граждан России, укрепление их конституционных прав, а также
создание равных возможностей для доступа к необходимой информации и ИКТ. По
своему содержанию Стратегия представляет собой политический документ, который
определяет цели, принципы и основные на\-прав\-ле\-ния государственной политики России в
области развития ИКТ, науки, образования и культуры для продвижения страны на пути к
информационному обществу. Кроме того, она содержит контрольные значения
показателей развития информационного общества в России на период %\linebreak 
до~2015~г.

     По предложению Президента России В.\,В.~Путина создана Комиссия по вопросам
развития информационного общества, которая будет функционировать при Совете при
Президенте РФ по науке, технологиям и образованию. В ее состав вошли ученые,
губернаторы, руководители ИКТ-компаний, депутаты Государственной Думы России.
Председателем Комиссии назначен Министр информационных технологий и связи РФ
Леонид Рейман. Основной задачей Комиссии является объединение усилий государства,
бизнеса и гражданского общества по построению информационного общества в России.
Она будет разрабатывать предложения по реализации Стратегии и системе прогнозных
показателей, характеризующих уровень развития информационного общества в России,
проводить по поручению президента страны экспертизу проектов федеральных законов и
нормативных актов, а также осуществлять подготовку ежегодного Национального доклада
о развитии информационного общества.

     В Общественной палате РФ создана Рабочая группа по развитию информационного
общества в России, ближайшей целью которой является общественная экспертиза
процессов информатизации и организация диалога между представителями
общественности, государственных органов власти и бизнеса по проблемам формирования
информационного общества.

     В 2007~г.\ Правительством России была также одобрена Концепция формирования
<<Электронного правительства>> на период до 2010~г. Это новая форма деятельности
органов государственной влас\-ти, основанная на широком применении ИКТ для получения
гражданами и организациями государственных услуг.

     Перечисленные выше политические и организационные решения, которые были
приняты в 2007~г.\ Президентом и Правительством РФ, а также Советом безопасности и
Общественной палатой России, свидетельствуют о том, что в стране начался новый этап
процесса формирования информационного общества. Можно ожидать, что этот этап будет
отличаться большей целенаправленностью, а также более высоким уровнем координации
между основными участниками этого весьма сложного по своей реализации 
социально-тех\-но\-ло\-ги\-че\-ско\-го, научно-технического и культурологического процесса дальнейшего
развития нашей страны и обеспечения ее национальной безопасности.

\vspace*{-3pt}
\paragraph*{Состояние основных направлений формирования
информационного общества в России.} В последние годы были предприняты
значительные усилия по развитию отрасли ИКТ, в результате которых Россия 
сегодня является мировым лидером по темпам развития этой отрасли. Общий объем 
отрасли в экономике России достиг значения в 1,5~трлн руб., а ее доля в 
структуре ВВП увеличилась с 3,2\% в 2000~г.\ до 4,8\% в 2007~г. Это означает, 
что данная отрасль становится локомотивом экономического роста страны. Согласно 
программе социально-эко\-но\-ми\-че\-ско\-го развития Российской Федерации, 
вклад ИКТ в общий рост экономики страны должен составить 0,5\%, что будет 
сопоставимо с вкладом от добычи нефти (0,6\%). Основной рост здесь ожидается в 
области разработки про\-грам\-мно\-го обеспечения и системной интеграции. В 
этих сегментах рынка доля отечественных компаний весьма значительна, и уже 
сегодня они успешно конкурируют на внут\-рен\-нем рынке с зарубежными фирмами. 
Предполагается, что в результате реализации основных мероприятий Стратегии 
развития информационного общества в России доля отечественных товаров и услуг 
на внутреннем рынке ИКТ к 2015~г.\ превысит 50\%~\cite{2ss}.

     Анализ показывает, что по мере перемещения производства комплектующих изделий
в станы Азии и снижения стоимости оборудования основная конкуренция на мировом
рынке ИКТ сосредоточивается в области производства про\-грам\-мных продуктов.
Ожидается, что к 2010~г.\ объем мирового рынка услуг в области разработки
про\-грам\-мно\-го обеспечения достигнет 140~млрд долларов. Име\-ющие\-ся в 
России интеллектуальные ресурсы, а также опыт отечественных разработок дают 
нашей стране реальные шансы занять на этом рынке достойное место. Однако для 
этого необходима дальнейшая государственная поддержка отрасли, ее обеспечение 
необходимыми кадрами и создание благоприятной среды предпринимательства.

     \vspace*{1pt}
     \textit{Макроэкономическая среда предпринимательства.} Стратегия развития
информационного общества в России ставит задачу вхождения нашей страны к 2015~г.\ в
число 20~передовых стран мира в области развития информационного общества. При
этом предполагается, что основные направления этой Стратегии должны быть
реализованы силами преимущественно российских компаний. %\linebreak
Поэтому важнейшей задачей сегодня является формирование инфраструктуры их 
деятельности и повышения конкурентной способности. С этой целью в 2006~г.\ 
постановлением Правительства РФ создан и с 2008~г.\ нач\-нет полноценно 
функционировать Российский инвестиционный фонд 
ин\-фор\-ма\-ци\-он\-но-ком\-му\-ни\-ка\-ци\-он\-ных технологий. Его уставной 
капитал составляет 1,45~млрд руб., причем 100\% акций этого фонда находится в 
федеральной соб\-ст\-вен\-ности. {\looseness=1

}

     Состоялось и учредительное собрание Национальной ассоциации инноваций и
развития информационных технологий (НАИРИТ). Ее основной задачей является
реализация проектов, призванных поднять уровень отечественных ИКТ, а также
привлечение в данную сферу зарубежных инвесторов. В состав этой ассоциации выразили
желание войти более 200~инновационных компаний, связанных с ИКТ. Создана
Ассоциация предприятий отечественных производителей телекоммуникационного
оборудования <<Совет главных конструкторов>>. В настоящее время в России
формируется также и система венчурного финансирования, которая должна обеспечить
доступ ИКТ-предприятий к финансовым ресурсам.

     \vspace*{1pt}
     \textit{Первые отечественные технопарки в сфере ИКТ.} Государственная
программа создания технопарков в сфере ИКТ получила в последние годы свое новое
развитие. Такие технопарки создаются сегодня в семи регионах России: Московской
области, Санкт-Петербурге, Новосибирске, Нижнем Новгороде, Казани, Тюмени и Калуге.
В дальнейшем планируется придать им статус особых экономических зон, что даст
дополнительные экономические стимулы для привлечения инвестиций в их развитие со
стороны бизнес-структур. Основная функция технопарков заключается в формировании
инфраструктуры для развития российских предприятий. Кроме того, по прогнозам
Мининформсвязи, развитие технопарков позволит создать порядка 75~тыс.\ новых
рабочих мест, а годовая стоимость выпускаемой ими продукции и оказываемых услуг
может превысить 100~млрд руб.

     Можно ожидать, что создание технопарков позволит также уменьшить отток из
нашей страны интеллектуального капитала, и в особенности та\-лант\-ли\-вых программистов.
Однако реализация этого проекта требует решения ряда проблем, связанных с
подготовкой кадров. В настоящее время система образования России еще не обеспечивает
подготовки необходимого числа квалифицированных специалистов для новой
информационной отрасли, особенно с учетом ее планируемого расширения. Сегодня в
нашей стране имеется острый дефицит квалифицированных программистов, специалистов
среднего звена, менеджеров и руководителей проектов.

     \vspace*{1pt}
     \textit{Система связи и информационных коммуникаций.} Инфраструктура связи в
России за последние годы продолжает улучшаться. В результате модернизации и
реконструкции электронных междугородных АТС, а также ввода в эксплуатацию новых
станций, кабельных и радиорелейных линий, плотность фиксированной телефонной связи
в стране увеличена до 32~аппаратов на 100~человек. При этом очередь на установку
телефона сократилась в полтора раза~--- с 1,5~млн до 1~млн заявок. В 2007~г.\ было
установлено более 73~тыс.\ таксофонов и организовано 16~тыс.\ пунктов коллективного
доступа в Интернет. В 2008~г.\ планируется обеспечить телефонизацию всех населенных
пунктов страны, среди которых около 8\% еще не имеют фиксированной телефонной
связи, и будет завершен первый этап внедрения универсальных услуг связи в масштабах
всей страны. К 2010~г.\ плотность телефонной сети будет повышена в полтора раза по
сравнению с 2005~г. Появится реальная конкуренция на рынках междугородной и
мобильной связи.

     В 2007~г.\ в России начата модернизация всей сис\-те\-мы почтовой связи с переводом
ее на современную технологическую базу. Восстановлена единая структура почты страны,
создано Федеральное государственное унитарное предприятие <<Почта России>>,
которое приняло на свой баланс все наличное имущество почты, в состав которой сегодня
входит 42~тыс.\ почтовых отделений. Это исключительно важный и социально значимый
федеральный проект, реализация которого позволит не только обеспечить современный
уровень услуг почтовой связи, но также и предоставить населению новые виды услуг, в
том числе возможности доступа к сети Интернет и использования электронной поч\-ты.
В~2007~г.\ в почтовых отделениях создано более 3~тыс.\ пунктов коллективного доступа в
Интернет, а их общее количество достигло 23~тыс.

     Число пользователей сети Интернет в России в 2007~г.\ приблизилось к значению
35~млн чел., что почти на 40\% превышает показатели 2006~г. При этом существенно
увеличивается количество линий широкополосного доступа пользователей к сети
Интернет. В соответствии с контрольными значениями показателей Стратегии развития
информационного общества в России, число таких линий на 100~чел.\ населения
должно составить к 2010~г.\ 15~линий, а к 2015~г.~--- 35~линий. Это уже весьма
неплохие показатели по сравнению с мировым уровнем.

     Что же касается использования персональных компьютеров, то, по оценкам
Мининформсвязи, их число в России в настоящее время составляет 31,2~млн~шт., что
превышает аналогичный показатель 2006~г.\ почти на 36\%.

     \vspace*{1pt}
     \textit{Глобальная навигационная спутниковая система.} В настоящее время в
России проводятся работы по восстановлению и повышению работоспособности
отечественной глобальной навигационной спутниковой системы ГЛОНАСС, в состав
которой входит 24~высокоорбитальных спутника Земли. К концу 2007~г.\ их число было
увеличено до~18, что уже обеспечивает возможность непрерывной навигации на всей
территории России с точностью определения местоположения подвижных объектов
порядка 10~м.

     В 2008~г.\ планируется изготовить 400--500~тыс.\ абонентских навигационных
комплектов этой сис\-те\-мы, а в 2009~г.~--- обеспечить ее работу в полном составе. При
этом точность навигационных определений будет сопоставима с точностью аналогичной
американской системы GPS. В настоящее время в России создана ассоциация
разработчиков, производителей и потребителей оборудования и приложений на основе
глобальных навигационно-спут\-ни\-ко\-вых систем <<ГЛОНАСС/ГНСС-Форум>>. Эта
ассоциация создана в рамках осуществления Комплексного плана мероприятий по
форсированному развитию системы ГЛОНАСС в интересах национальной безопасности, а
также для социально-эко\-но\-ми\-че\-ских, научных и коммерческих целей.

     Необходимо отметить, что ГЛОНАСС и GPS~--- уникальные
навигационные системы глобального масштаба, которые сегодня не имеют
конкурентов и играют важную роль в решении проблем обеспечения национальной
безопасности многих развитых стран, и в первую очередь США и России.

     \vspace*{1pt}
     \textit{Готовность потребителей и бизнеса.} Основными потребителем ИКТ в
России продолжают оставаться государство и крупные российские компании. Около 40\%
спроса приходится на предприятия финансовой и нефтегазовой сферы, торговли и связи.
Металлургия, машиностроение, транспорт и другие отрасли еще отстают в использовании
ИКТ. По оценкам специалистов, все еще недостаточно обеспечены современными
средствами ИКТ силовые министерства и ведомства России, машиностроительные
отрасли, строительная промышленность, сельское хозяйство, а также образование,
медицина и сфера культуры.

     Необходимыми мерами, с помощью которых государство может увеличить спрос на
ИКТ со стороны предприятий всех отраслей экономики, являются: увеличение в стране
числа специалистов по ИКТ, повышение компьютерной грамотности и информационной
культуры занятого населения, а также повышение доступности оборудования за счет
снижения импортных пошлин и сокращения сроков амортизации компьютерной техники.

     \vspace*{1pt}
     \textit{Наука и образование.} Социологические исследования показывают, что
научные работники России, хотя и отстают еще от своих европейских коллег по уровню
использования ИКТ, но в целом лидируют в этой области по сравнению с работниками
других сфер деятельности. Интернет и электронную почту сегодня используют более 80\%
российских научных учреждений и вузов, но веб-сайты имеет лишь треть научных
организаций. Около 40\% ученых используют Интернет для выполнения совместных
проектов с зарубежными партнерами. Однако еще не все работники российской науки
имеют навыки работы с компьютерами. Так, например, компьютерной грамотностью
обладают лишь 87\% столичных гуманитариев, в то время как для специалистов в области
естественных наук этот показатель равен~100\%.

     Достаточно быстро развиваются в России научно-образовательные сети
(RUNNet/RBNet и~др.), однако значительная часть используемого в них оборудования
морально устарела. Сохраняются высокие цены на аренду цифровых каналов.
Инфраструктура российских научно-образовательных сетей имеет значительно меньшую
пропускную способность по сравнению с европейской, и, кроме того, она существенно
неоднородна. Так, например, если скорость передачи данных между Москвой и 
Санкт-Петербургом составляет 2,4~Гбит/с (среднеевропейский уровень), то между
Новосибирском и Хабаровском~--- всего 4~Мбит/с.

     Тем не менее в 2007~г.\ в России начата реализация программы <<СКИФ
Университеты>>, которая имеет целью объединение в суперкомпьютерную сеть вузов
России и Белоруссии. Планируется оснастить эти вузы суперкомпьютерами СКИФ, а в
дальнейшем объединить их в единую вычислительную сеть <<СКИФ Полигон>>.
Предполагается, что в состав этой сети войдут суперкомпьютеры, установленные в МГУ
им.~М.\,В.~Ломоносова, Томском, Южно-Уральском, Владимирском, Белгородском и
     Санкт-Петербургском государственных университетах, а также в Объединенном
институте проблем информатики НАН Белоруссии. Ожидается, что уже к середине
2008~г.\ суммарная вычислительная мощность этой сети составит более 100~терафлопсов.
Обучение и переподготовку необходимых специалистов будет обеспечивать НИВЦ МГУ.

     Важным этапом в развитии информационного общества в России является
завершение в 2007~г.\ процесса подключения к сети Интернет всех российских школ,
общее число которых составляет 62~тыс. В рамках приоритетного национального проекта
<<Образование>> в течение 2006--2007~гг.\ к сети Интернет было подключено более
53~тыс.\ школ в 89~областях РФ. Более 60\% из них находится в сельской местности, а
7,5~тыс.~--- в труднодоступных районах страны, где пришлось использовать спутниковые
технологии. Уже к началу 2007/2008~учебного года выход в Интернет по выделенным
каналам имели более 51~тыс.\ школ (98\%). При этом минимальная скорость передачи
данных составляла 128~кбит/с. Необходимо отметить, что к сети Интернет сегодня
подключены не только средние образовательные учреждения, но также вечерние и
кадетские школы, школы-интернаты, коррекционные и специальные школы, а также
школы, находящиеся в составе исправительных учреждений.

     В результате реализации данного проекта учащиеся и преподаватели как городских,
так и сельских школ смогут пользоваться электронными биб\-лио\-те\-ка\-ми и электронными
образовательными %\linebreak 
ресур\-са\-ми, что создает новые возможности для существенного
повышения доступности качественного образования. Однако для этого необходимо будет
осуществить массовую переподготовку преподавателей школ и работников школьных
биб\-лио\-тек, которые сегодня в своем большинстве еще не обладают необходимым уровнем
информационной компетентности.

     \vspace*{1pt}
     \textit{Здравоохранение и медицина.} Информатизация сферы здравоохранения
является сегодня в России исключительно актуальной и важной социальной проблемой. В
приоритетном национальном проекте <<Здоровье>> значительное внимание уделяется
внедрению методов дистанционного обслуживания пациентов с использованием ИКТ,
которые получили название <<телемедицины>>. Для России с ее огромной территорией
значение этих методов трудно переоценить. В последнее время в решении данной
проблемы наметились определенные сдвиги. Так, например, в Ханты-Мансийском
автономном округе уже функционирует 6 телемедицинских центров и
52~телемедицинских пункта, которые обеспечивают возможность оперативного
дистанционного консультирования пациентов. Однако это лишь одно из направлений
информатизации сис\-те\-мы здравоохранения. Не менее актуальным является перевод в
электронный формат всей сис\-те\-мы обслуживания пациентов, информатизация рабочих
мест врачей и других работников поликлиник, внедрение <<электронных медицинских
карт>>, создание электронных баз данных пациентов и~т.\,п. Все это позволило бы резко
ускорить оформление медицинских документов и высвободить время врачей для их
непосредственной работы с больными. Сегодня же это направление в России пока еще
только начинает развиваться.

     \vspace*{3pt}
     \textit{Социальная и культурная среда общества.} Со стороны населения России
спрос на информационную технику и технологии продолжает расти, хотя все еще остается
недостаточным, особенно в регионах Сибири, Севера и Дальнего Востока. По экспертным
оценкам, более 30\% российских семей имеют дома персональный компьютер. Число
абонентов сотовой связи в 2007~г.\ достигло 180~млн, что существенно превышает
общую численность населения страны, а в крупных городах~---  Москве и
     Санкт-Петербурге~--- численность их населения. Мобильная связь является важным
фактором информатизации общества, так как она обеспечивает гражданам России
колоссальную экономию общественно полезного времени. Важный социальный аспект
реализации приоритетного национального проекта <<Образование>> состоит в том, что
он стимулирует развитие телекоммуникационной инфраструктуры в регионах нашей
страны, в результате чего во многих населенных пунктах России впервые появилась
возможность доступа в Интернет по приемлемым для населения ценам.

     Исключительно важной и актуальной задачей формирования информационного
общества в России является обеспечение свободного доступа населения к социально
значимым информационным ресурсам. Решение этой задачи осуществляется сегодня по
трем основным направлениям: обеспечение доступа населения к информации органов
власти и государственным услугам (<<Электронное правительство>>), создание центров
доступа к правовой информации и создание общедоступных пуб\-лич\-ных информационных
ресурсов (электронные библиотеки, архивы, музеи и~т.\,п.).

     Реализацию Концепции <<Электронного правительства>> в России планируется
осуществить в период до 2010~г.\ в два этапа. На первом этапе (2008~г.) разрабатываются
нормативные и правовые документы, завершается проектирование и создание опытных
образцов межведомственных систем <<Электронного правительства>>. На втором этапе
(2009--2010~гг.) должно быть обеспечено внедрение типовых технологических и
организационных решений. Реализация данной концепции будет происходить в рамках
ФЦП <<Электронная Россия>>. При этом ожидается, что трудозатраты органов
государственной власти на обмен информацией между ведомствами снизятся примерно на
50\%, уменьшится административная нагрузка на граждан и организации, что позволит
стране ежегодно экономить до 10~млрд руб.\ на административных расходах. Необходимо
отметить, что некоторые регионы РФ уже сегодня имеют и реализуют собственные
концепции <<Электронного правительства>>. Так, например, Концепция <<Электронного
правительства>> Нижегородской области была утверждена еще в середине 2004~г.

     Важной составной частью будущего <<Электронного правительства>> России
должна стать единая информационная система в сфере здравоохранения и социального
развития страны. По оценкам специалистов, имеющаяся в настоящее время информация
для населения о наборе предоставляемых государством услуг является явно
недостаточной. Необходима комплексная информатизация деятельности всей социальной
сферы. Одной из задач здесь является создание автоматизированной информационной
системы <<Электронный социальный регистр населения>>, которая 
внедряется в Саратовской области, а с 2008~г.\ будет распространена и на все другие
регионы России.

     В области же создания системы общедоступных публичных информационных
ресурсов имеется ряд серьезных проблем, прежде всего правового характера. По мнению
специалистов, входящих в состав Российской ассоциации электронных библиотек,
сегодня необходима разработка Стратегии развития публичных информационных
ресурсов России, а также принятие ряда новых и уточнение действующих
законодательных и нормативных актов.

     \vspace*{1pt}
     \textit{Правовое обеспечение электронного развития.} По оценкам специалистов,
существующая в России нормативная правовая база в области информационной сферы
общества еще не соответствует совре\-мен\-ным требованиям и нуждается в существенном
дальнейшем развитии. В частности, еще %\linebreak 
недостаточно разработаны механизмы защиты
прав на интеллектуальную собственность и обеспечения патентного права, что приводит к
потерям доходов российскими экспортерами и мешает привлечению в Россию крупных
международных компаний.

     Институтом государства и права РАН еще в 2006~г.\ разработана Концепция
развития информационного законодательства в Российской Федерации, которая
предусматривает системное развитие информационного законодательства в нашей стране
на основе принятия ряда новых федеральных законов. В Концепции правового
регулирования в сфере информационных технологий, разработанной Мининформсвязи,
также предусмотрено внесение изменений в действующие законы и принятие ряда
новых законов (<<Об электронном
 документообороте>>, <<Об информации
персонального характера>>, <<Об электронной торговле>>, <<О~праве на
информацию>>, <<Об участии в международном информационном обмене>> и~др.).
Российской ассоциацией электронных библиотек подготовлены проекты Федеральных
законов <<Об обязательном электронном экземпляре документа>> и <<О~праве граждан
на доступ к информации о деятельности органов государственной власти и местного
самоуправления>>. Однако многие из этих проектов еще находятся на разных стадиях
своего прохождения в правительстве и Государственной Думе.
{\looseness=-1

}

     Принятие Стратегии развития информационного общества в России, создание
Комиссии по вопросам развития информационного общества при Совете при Президенте
РФ по науке, технологиям и образованию, а также Рабочей группы по информационному
обществу в Общественной палате РФ должны дать новый импульс развитию
отечественного информационного законодательства. Ведь отста\-ва\-ние России от мирового
уровня в об\-ласти %\linebreak 
развития информационного общества все еще остается весьма
существенным и представляет реальную угрозу для ее национальной безопасности.

\section{Актуальные проблемы дальнейшего развития
информационного общества в~России}

     Активизация процессов развития информационного общества в России обостряет
многие проблемы информационной безопасности государства, человека и общества и
требует адекватных мер противодействия не только на федеральном, но и на
региональном уровне, а также на уровне корпораций, предприятий и общественных
организаций практически во всех сферах социальной активности общества~\cite{3ss}. С этой целью
необходима организация соответствующей просветительской работы в обществе, а также
решение задач кадрового обеспечения информационного развития страны.

   \vspace*{1pt}
   \textit{Развитие информационной инфраструктуры.} Анализ показывает, что
современный уровень развития российской информационной инфраструктуры,
использования информационно-ком\-му\-ни\-ка\-ци\-он\-ных технологий в общественном
производстве и государственном управлении еще не соответствует задачам
инновационного развития страны и повышения ее конкурентоспособности,
диверсификации экономики, повышения благосостояния и качества жизни граждан,
укрепления обороноспособности и безопасности. По основным показателям он
существенно уступает развитым странам мира. Поэтому в ближайшие годы необходимо
сосредоточить усилия на устранении инфраструктурных диспропорций в области связи и
телекоммуникаций, что позволит снизить остроту проблемы <<цифрового неравенства>>
между отдельными регионами России.

   В нашей стране сегодня практически отсутствует производство собственной
конкурентоспособной продукции микроэлектронной промышленности,
телекоммуникационного оборудования и средств вычислительной техники. В результате
этого зависимость развития российской информационной инфраструктуры от поставок
зарубежных ИКТ, по оценкам специалистов, в настоящее время превышает критический
уровень и представляет собой одну из актуальных проблем.

   \vspace*{1pt}
   \textit{В сфере науки и образования.} Система образования России сегодня еще не
обеспечивает в необходимом объеме качественное воспроизводство трудовых ресурсов,
требуемое для повышения конкурентоспособности страны в условиях
пост\-ин\-дуст\-ри\-аль\-но\-го развития и становления информационного общества, основанного
на знаниях. %\linebreak 
Уровень компьютерной грамотности и информационной компетентности
преподавателей об\-ра\-зова\-тель\-ных учреждений, и в особенности об\-ще\-об\-ра\-зо\-ва\-тель\-ных
школ в сельской местности, %\linebreak 
яв\-ля\-ется сегодня чрезвычайно низким и не обеспечивает
эффективного использования современных ИКТ и электронных образовательных
информационных ресурсов общества. Школьные библиотеки еще не стали современными
центрами информационной поддержки образовательного процесса, а их работники не
имеют педагогического статуса и соответствующего уровня оплаты труда.

   Несмотря на принимаемые государством меры по повышению уровня оплаты труда
научных работников и преподавателей, проблема преемственности поколений в науке и
образовании остается достаточно острой. По оценкам ряда специалистов, уровень общей
образованности российского общества снижается. Качество государственных
образовательных стандартов и учебников во многом еще не отвечает современным
требованиям, особенно в области базовых дисциплин (математики, физики, филологии,
истории, информатики). Содержание образования в целом еще не ориентировано на новые
условия существования человека в информационном обществе, а социальные аспекты
информатизации общества, в том числе проблемы информационной безопасности, в
системе образования практически не изучаются. Не развернуты также и работы по
формированию новой информационной культуры личности и общества, адекватной
современным требованиям.

   \vspace*{1pt}
   \textit{В социальной сфере общества.} В современной России наблюдается
значительное неравенство в доступе к информации и современным ИКТ различных групп
населения и регионов, что является важным фактором, тормозящим 
социально-эко\-но\-ми\-че\-ское развитие общества и повышение качества жизни населения. Кроме того,
<<цифровое неравенство>> усиливает социальное расслоение российского общества и
поэтому является одной из угроз для национальной безопасности страны.

   \vspace*{1pt}
     \textit{В сфере культуры.} Работы по использованию ИКТ для сохранения
культурного наследия России, формированию электронных ресурсов библиотек и
использования их в режиме удаленного доступа сегодня еще недостаточно развернуты.
Одной из причин здесь является нерешенная ситуация в об\-ласти авторского права, так как
электронные копии документов должны распространяться бесплатно. Другая причина~---
все еще низкий уровень информатизации учреждений культуры, который не позволяет
обеспечивать потребности населения в социально значимой информации. Так, например,
только 15\% российских библиотек используют современные ИКТ и лишь 9\% из них
подключено сегодня к сети Интернет. Материально-техническая база большинства
библиотек безнадежно устарела и нуждается в качественном обновлении. В стране
практически не строятся новые здания и специализированные помещения для публичных
библиотек.

\vspace*{-12pt}
\paragraph*{Первоочередные задачи.} Формирование информационного общества в
России должно стать одним из важнейших приоритетов ее государственной политики на
ближайшие годы. Необходимо также организовать проведение широкомасштабной
разъяснительной кампании о значении, целях и задачах формирования в России
информационного общества.

     В числе первоочередных задач нужно обеспечить безусловное выполнение планов
Мининформсвязи РФ по подключению в 2008~г.\ всех населенных пунктов страны к
системе стационарной телефонной связи, а также по созданию точек коллективного
доступа в Интернет во всех населенных пунктах страны. При этом к июлю 2008~г.\
минимальная скорость передачи данных должна достигнуть значения 256~кбит/с. Кроме
того, количество таксофонов должно возрасти с~73 до~140~тыс.

   \vspace*{1pt}
     \textit{В сфере науки и образования.} Необходимо сформировать программу массовой
переподготовки преподавателей школ и работников школьных биб\-лио\-тек с целью
обеспечения необходимого уровня их компьютерной грамотности и информационной
компетентности для эффективного использования ИКТ и электронных информационных
образовательных ресурсов общества и приступить к практической реализации этой
программы. Нужно также внести необходимые коррективы в Программу модернизации
российского образования с учетом основных положений Концепции формирования
информационного общества в России. Информатизация сферы образования вновь должна
стать приоритетным направлением этой программы.

   \vspace*{1pt}
   \textit{В области правового обеспечения информационного развития страны.}
Существующая законодательная и нормативная правовая база в информационной сфере
России еще не соответствует современным требованиям, а работы по ее формированию
ведутся очень медленно. В первую очередь здесь необходимо рассмотреть и принять уже
подготовленные проекты законов. Кроме того, необходимо ввес\-ти в текст Федерального
закона <<Об обязательном экземпляре документов>> ряд поправок, регламентирующих
поставку обязательных экземпляров документов в электронной форме, с целью правового
обеспечения работ по формированию национального электронного 
информационно-биб\-лио\-теч\-но\-го фонда социально значимой информации~\cite{6ss}.
{\looseness=-2

}

\vspace*{-10pt}

\section{Инновационное развитие России и новая парадигма
образования}

     Выступая на расширенном заседании Государственного Совета 8~февраля 2008~г.,
Президент России В.\,В.~Путин поставил стратегическую задачу радикального изменения
курса дальнейшего развития страны и реализации крупных преобразований практически
во всех основных сферах нашего общества уже в период до 2020~г. Понятно, что для
решения задач такого масштаба и зна\-чи\-мости %\linebreak 
потре\-бу\-ет\-ся принципиально новый подход
к проблеме развития и использования человеческого потенциала, подготовки кадров для
инновационной России. Какой же должна стать отечественная %\linebreak 
сис\-те\-ма образования для
того, чтобы обеспечить решение этих задач? Анализ процессов развития глобальных
проблем современности и тех основных качеств, которыми должны обладать люди для
своей успешной социальной адаптации в новых условиях стремительно формирующегося
информационного общества, показывает, что перспективная система образования должна
обладать рядом принципиально новых качеств~\cite{4ss, 5ss}.

     Эти качества настолько существенны, что их совокупность может рассматриваться
как \textit{новая образовательная парадигма}, ориентированная на принципиально новые
условия существования человека и общества в XXI~в. В числе этих качеств наиболее
важными представляются следующие:\\[-14.1pt]
     \begin{enumerate}[1.]
\item \textit{Опережающий характер} образования, его нацеленность на решение
проблем формирования информационной цивилизации, развитие творческих
способностей человека, его умения самостоятельно принимать ответственные решения
в условиях все более динамично развивающегося общества.\\[-14.1pt]
\item Существенное расширение и качественное \textit{развитие высшей школы},
которая должна обеспечить необходимое для условий XXI~в.\ число специалистов с
высшим образованием.\\[-14.1pt]
\item \textit{Фундаментализация образования} за счет его все большей ориентации на
изучение новейших достижений науки в области познания фундаментальных
закономерностей развития природы, человека и общества, а также роли информации в
реализации этих закономерностей.\\[-14.1pt]
\item Существенно большая \textit{доступность системы образования} для широких
масс населения, которая необходима не только для повышения общего уровня
интеллектуального и духовного развития общества, но и для достижения большей
социальной стабильности в обществе, уменьшения социального неравенства.\\[-14.1pt]
\item \textit{Непрерывность образования}, переход к практической реализации
концепции ЮНЕСКО <<Образование через всю жизнь>>. Развитие системы
открытого образования и повышения квалификации дипломированных специалистов.
Создание системы образования и самообразования для взрослого населения, а также
для инвалидов и людей с ограниченной мобильностью.
   \end{enumerate}

   Таким образом, речь идет о необходимости перехода к новой философской концепции
образования, целью которого должна быть признана, преж\-де всего, высокая
образованность человека, а не подготовка специалиста узкого профиля, как это в
большинстве случаев имеет место сегодня. Суть этой концепции заключается в 
следующем:
перестроить содержание и методологию учебного процесса во всех звеньях системы
образования таким образом, чтобы она оказалась способной своевременно готовить людей
к новым условиям существования, давать им такие знания, умения и навыки, которые
позволили бы им успешно адаптироваться, жить и работать в новой социальной и
информационной среде общества.

\vspace*{-8.5pt}
\section{Заключение}

\vspace*{-3pt}

   Процесс информатизации общества в XXI~в.\ перешел на качественно новый этап
своего развития~--- он стал системным и направлен на формирование глобального
информационного общества, в котором создаются принципиально новые условия для
жизни и деятельности человека. Развитие средств и методов информатики, их
использование практически во всех сферах социальной практики осуществляется
стремительно и в последние годы принимает масштабы одного из важнейших
направлений развития не только научно-технического прогресса, но и всей современной
цивилизации. Однако эти процессы еще мало учитываются в сис\-те\-ме образования,
которая сегодня не обеспечивает подготовки людей к новым условиям существования в
информационном обществе.

   Проблема формирования информационного общества в России ставит перед
   научно-об\-ра\-зо\-ва\-тель\-ным сообществом, и прежде всего перед сис\-темой образования,
новые стратегические задачи, которые должны в значительной степени изменить
существующую образовательную политику, в первую очередь в области развития высшей
школы. Суть этих изменений заключается в том, чтобы привести структуру и содержание
образования в соответствие с теми новыми требованиями, которые выдвигает на повестку
дня информационное общество. При этом недостаточно будет обеспечить лишь
подготовку необходимого числа специалистов в области ИКТ, информационной
экономики и бизнеса, хотя, конечно же, это нужно сделать в первую очередь. Принятая в
2007~г.\ Стратегия развития информационного общества в России на период до 2015~г.\
требует соответствующей модернизации всей системы образования. При этом прежде
всего необходимо обеспечить переподготовку педагогических кадров средней и
высшей школы, а также существенно переориентировать направленность и содержание
диссертационных исследований, выполняемых в аспирантуре и магистратуре.

   Стратегически важной и существенно более сложной задачей является формирование
новой информационной культуры российского общества. Этой культурой должна будет,
прежде всего, овладеть элитарная часть общества: ученые, преподаватели и аспиранты
вузов, а также молодые специалисты, уже получившие высшее образование. Большую
работу предстоит провести и в области переподготовки государственных служащих,
причем не только чиновников, но также и медицинских работников, специалистов
социальной сферы, правоохранительных органов, библиотечного сообщества, сферы
культуры.

   Для эффективного использования возможностей, которые создает развитие
информационного общества, в стране должны быть соответствующим образом
подготовлены люди. Миллионы российских граждан должны непрерывно получать новые
знания, умения и навыки, необходимые им для жизни и деятельности в быстро
изменяющихся условиях новой среды обитания, характерной для информационного
общества. В значительной степени должна будет измениться и философия образования, в
которой процессы глобальной информатизации общества должны будут получить свое
необходимое отражение.
%
   Информатизация общества в России, имеющей огромную территорию, может и
должна стать приоритетным направлением ее инновационного развития, так как она
открывает новые возможности для использования высокого интеллектуального
потенциала нашей страны.

\vspace*{-7pt}

{\small\frenchspacing
{%\baselineskip=10.8pt
\addcontentsline{toc}{section}{Литература}


\begin{thebibliography}{9}
%\vspace*{-3pt}

  \bibitem{1ss}
  Рейтинг <<электронной готовности>> стран.  {\sf http://}\linebreak 
{\sf e-commerce.ru/cgi/print.asp.}

\bibitem{2ss}
Стратегия развития информационного общества в России~// 
Открытое образование,2006. №\,4(63). С.~4. %--8.

\bibitem{3ss}
Доктрина информационной безопасности Российской Федерации~// Научные и
методологические проблемы информационной безопасности (сборник статей)~/ Под
ред. В.\,П.~Шерстюка.~--- М.: МЦНМО, 2004. С.~149--197.

\bibitem{6ss}
\Au{Антапольский~А.\,Б.}
О стратегии развития электронных публичных информационных ресурсов России~//
Открытое образование, 2007. №\,6(65). С.~58--63.

\bibitem{4ss}
\Au{Кинелев~В.\,Г.}
Образование для информационного общества~// Открытое образование, 2007. №\,5(64). С.~46. %--57.

\label{end\stat}

\bibitem{5ss}
\Au{Колин~К.\,К.}
 Человек в информационном обществе: новые задачи образования, науки и культуры~//
Открытое образование, 2007. №\,5(64). С.~40--46.
\end{thebibliography}

}
}
\end{multicols}