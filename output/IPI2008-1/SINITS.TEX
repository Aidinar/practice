%\def\ss{\textstyle}
%\def\kk{\kappa}
%\def\tr{\,,\,\ldots\,,\,}
%\def\rv{\right\vert\,}
%\def\rrv{\right\vert}
%\def\lv{\,\left\vert}

%\def\rk{\right]}
%\def\lk{\left[}

%\def\rk{\,\right]}
%\def\lk{\left[\,}
%\def\rf{\right\}}
%\def\lf{\left\{}
%\def\prl{\,\parallel}
%\def\prr{\parallel\,}
%\def\paar{\parallel}
%\def\sbs{\subset}
%\def\sps{\supset}
\def\eps{\varepsilon}
\def\si{\sigma}
\def\la{\lambda}
\def\alp{\alpha}
\def\w{\omega}
%\def\W{\Omega}
%\def\sssd{\mathop{\sum\limits^2\sum\limits^2}}
%\def\sssn{\mathop{\sum\limits^n\sum\limits^n}}
%\def\liminf{\mathop{\cup\,inf}} \def\limsup{\mathop{\cup\,sup}}
%\def\iint{\int\limits_{-\infty}^{\infty}}
\def\iii{\int\limits}
%\def\sss{\sum\limits}
\def\prt{\partial}
\def\mm{{\rm M}}

%----------------------------------------------------------

\def\stat{sinits}

\def\tit{КВАЗИЛИНЕЙНЫЕ МЕТОДЫ ПОСТРОЕНИЯ ИНФОРМАЦИОННЫХ МОДЕЛЕЙ ФЛУКТУАЦИЙ
 НЕРАВНОМЕРНОСТИ ВРАЩЕНИЯ ЗЕМЛИ$^*$}

\def\titkol{Квазилинейные методы построения информационных моделей флуктуаций
 неравномерности вращения Земли}

\def\autkol{И.\,Н.~Синицын}
\def\aut{И.\,Н.~Синицын$^1$}

\titel{\tit}{\aut}{\autkol}{\titkol}

{\renewcommand{\thefootnote}{\fnsymbol{footnote}}\footnotetext[1]{Работа
выполнена при финансовой поддержке РФФИ
(проект №\,07-07-00031) и программы ОИТВС РАН <<Фундаментальные
основы информационных технологий и систем>> (проект 1.5).}
\renewcommand{\thefootnote}{\arabic{footnote}}}

\footnotetext[1]{Институт проблем информатики Российской академии наук, sinitsin@dol.ru}

\index{Синицын И.\,Н.}

\label{st\stat}

\Abst{В основе многих современных стохастических информационных технологий
научных исследований лежат методы построения информационных моделей по
априорным и апостериорным данным. В статье рассматривается применение
корреляционных методов построения информационных моделей флуктуаций
неравномерности вращения Земли на внутригодовых интервалах времени,
основанных на эквивалентной линеаризации нелинейных стохастических
уравнений. Приводятся примеры приложения методов и программного
обеспечения из состава информационных ресурсов <<Статистическая
динамика вращения Земли>>.}

\KW{флуктуации неравномерности вращения Земли; информационная модель;
корреляционные характеристики; корреляционные методы; эквивалентная
линеаризация; стохастические дифференциальные уравнения}

\vskip 24pt plus 9pt minus 6pt

\thispagestyle{headings}

\begin{multicols}{2}

\section{Введение}  %1

В [1--10] на основе априорных данных о динамической гравитационно-приливной
структуре Земли построены корреляционные и кинетические стохастические
модели флуктуаций угловой ско\-рости собственного вращения в составе совокупной
модели вращательного движения Земли.

В~[11] предложена линейная модель
возму\-щенных регулярных собственных вращательных движений Земли на
внутригодовых интервалах времени.

Влияние аддитивных и линейных мультипликативных (параметрических)
гармонических и широкополосных гауссовских случайных возмущений на
одно- и двумерные корреляционные характеристики флуктуаций приливной
неравномерности вращения Земли изучено в~[12].

Рассмотрим методы
построения квазилинейных информационных моделей флуктуаций приливной
неравномерности вращения Земли в случае нелинейного механизма диссипации
на основе как априорной, так и апостериорной информации о неравномерности
вращения Земли. При этом учтем также аддитивные и параметрические
гармонические и гауссовские широкополосные случайные
гравитационно-приливные диссипативные возмущения.


\section{Стохастические дифференциальные уравнения флуктуаций
неравномерности вращения Земли}  %2

Обобщая математические модели~[1--12] флуктуаций угловой скорости
собственного вращения Земли на случай воздействия как аддитивных,
так и параметрических гармонических и широкополосных случайных
гравитационно-приливных и флуктуационно-диссипативных возмущений на
внутригодовых интервалах времени, представим дифференциальное
уравнение изменения угла собственного вращения  Земли $\delta \varphi$
в следующем виде:

\noindent
\begin{multline*}
\!\!\!\!\delta \ddot\varphi = M_{10}^S \cos \left(2\pi f_\Gamma t +\chi_1^S\right) +  M_{20}^S \cos \left(4\pi f_\Gamma t +\chi_2^S\right)+{}\\
{}+M_{m0}^L \cos \left(2\pi \nu_m t +\chi_m^L\right)+ M_{f0}^L \cos \left(2\pi \nu_f t +\chi_f^L\right) +{} \\
 {}+X^S (t) + X^L (t)- \mu_1\left[ 1+ \pi^{\mu S}_{11}\cos \left(2\pi f_\Gamma t +\chi_{11}^{\mu S}\right)+{}\right.\\
 {}+\pi^{\mu S}_{12} \cos \left(4\pi f_\Gamma t +\chi_{12}^{\mu S}\right) + \pi^{\mu L}_{1m} \cos \left(2\pi \nu_m t +\chi_{1m}^{\mu L}\right)+{}\\
\left. {}+\pi^{\mu L}_{1f} \cos \left(2\pi \nu_f t +\chi_{1f}^{\mu L}\right)+ X^{\mu S}_1 (t) + X_1^{\mu L}(t)\right] \delta \dot\varphi+{}\\
{} +\mu_n \biggl[ 1 + \pi_{n1}^{\mu S}\cos \left( 2\pi f_\Gamma t + \chi_{n1}^{\mu S}\right)+{}\\
{}+ \pi_{n2}^{\mu S} \cos \left( 4 \pi f_\Gamma t + \chi_{n2}^{\mu S}\right) +{}
\end{multline*}

\noindent
\begin{multline}
 \!\!\!\!{}+ \pi_{nm}^{\mu L} \cos \left( 2\pi \nu_m t + \chi_{nm}^{\mu L}\right)+ \pi_{nf}^{\mu L} \cos \left(2\pi \nu_f t +\chi_{nf}^{\mu L}\right)+{}\\
 {}+X_n^{\mu S} (t) + X_n^{\mu L} (t)\biggr]  F_n\left(\delta\varphi,\dot\varphi\right)\,.\label{e1}
\end{multline}
Здесь введены следующие обозначения:
$\delta\varphi =\varphi\;-$ $-\;r_* \w_*^{-1} t$~--- изменение угла
собственного вращения Земли;
$\varphi$~--- угол собственного вращения Земли, значения которого
определены на дату $t$ (он является параметром, характеризующим
вращение земной системы координат по отношению к небесной);
$r_* = 7{,}292115 \cdot 10^{-5}$~рад/с~--- постоянная средняя составляющая
угловой скорости собственного вращения Земли;
$\w_*$~--- угловая скорость обращения Земли по орбите;
$t$~--- безразмерное время, измеряемое стандартными годами;
$f_\Gamma = 1$, $ 2 f_\Gamma$,
$\nu_m= 13{,}28$, $\nu_f =26{,}28$, $M_{10}^S$, $M_{20}^S$, $M_{m0}^L$,
$M_{f0}^L$ и $\chi_1^S$, $\chi_2^S$, $\chi_m^L$, $\chi_f^L$~---
частоты, амплитуды и начальные фазы аддитивных гармонических возмущений
от Солнца $(S)$ и Луны $(L)$, соответствующие годовому, полугодовому,
месячному и двухнедельному циклам;
$\mu_h$ $(h=1,n)$~--- коэффициенты диссипативных моментов сил, обусловленных
разнообразием геофизических процессов (приливное трение океанических
и земных приливов, атмосферные воздействия, океанические течения,
перераспределение водных масс и~т.\,п.); $ \pi_{h1}^{\mu S}$,
$\pi_{h2}^{\mu S}$, $\pi_{hm}^{\mu L}$, $ \pi_{hf}^{\mu L}$ и
$\chi_{h1}^{\mu S}$, $\chi_{h2}^{\mu S}$, $\chi_{hm}^{\mu L}$,
$\chi_{hf}^{\mu_L}$ $(h=1,n)$~--- амплитуды и начальные фазы
параметрических гармонических диссипативных моментов сил на частотах
$f_\Gamma$, $2f_\Gamma$, $\nu_m$, $\nu_f$; $F_n=F_n \left(\delta\varphi,
\delta\dot\varphi\right)$~--- нелинейная со\-став\-ля\-ющая диссипации (в случае
рэлеевского механизма диссипации $F_n \left(\delta \dot\varphi\right)
=\delta\dot\varphi^3$); $X^S (t)$, $ X^L (t)$ и $X_h^{\mu S} (t)$,
$X_h^{\mu L} (t)$~--- нормальные (гауссовские) широкополосные
аддитивные и параметрические случайные возмущения с известными
математическими ожиданиями и ковариационными характеристиками.


\section{Квазилинейные методы построения информационных корреляционных
моделей по~априорным данным}  %3

Составим приближенные уравнения для математических ожиданий, дисперсий,
ковариаций и ковариационных функций переменных $\delta \varphi = X_1$,
$\delta \dot\varphi = X_2$ при следующих трех основных допуще\-ниях.

\begin{description}
\item $1^0$ Аддитивный возмущающий момент допускает представление
$X^S (t) + X^L (t) = m_0^{SL}+V_1$, где $V_1 $ является скалярным
нормальным белым шумом интенсивности $\nu_1 = \nu_1(t)$.

\item $2^0$ Возмущающие моменты $X_3 = X^{\mu S}_1 (t) + X_1^{\mu L} (t)$
и $X_4 = X_n^{\mu S} (t) + X_n^{\mu L}(t)$ происходят от одного
источника нормального белого шума $V_2$ единичной интенсивности
$(\nu_2 =1)$. Они имеют конечные математические ожидания $m_3$ и
$m_4$ и дисперсии $\si_3^2$ и $\si_4^2$ и удовлетворяют следующим
скалярным уравнениям формирующего фильтра~[13]:
\begin{align}
 \dot X_l^0 &=-\alp_l X_l^0 + \si_l \sqrt{2\alp_l} V_2\,;\notag\\[-6pt]
&\label{e2}\\[-6pt]
 X_l^0 &= X_l- m_l^*,\enskip l=3,4,\enskip \alp_l>0\,,\notag
\end{align}
где $m_l^*$~--- постоянные значения $m_l$.

\item $3^0$ Нелинейные функции $X_2 X_3$, $F_n = F_n (X_1, X_2)$, $ F_n'=
F_n' ( X_1, X_2, X_4)= F_n (X_1, X_2) X_4$, входящие в (1),
допускают статистическую линеаризацию нелинейностей по формулам~[13]:
 \begin{align*}
X_2 X_3 &\approx m_2 m_3 + k_{23}+ m_2 X_3^0 + m_3 X_2^0\,;\\ %\label{e3}
 F_n &= F_n (X_1,X_2)\approx F_{n0}+F_{n1} X_1^0+ F_{n2} X_2^0\,;\\ %\label{e4}
  F_2' &= F_1 (X_1, X_2) X_4 \approx \\
&\hspace*{25pt} \approx F_{n0}' +F_{n1}' X_1^0+ F_{n2}' X_2^0 + F_{n4}' X_4^0\,. %\label{e5}
\end{align*}
Здесь $m_i$ и $k_{ij}$~--- математические ожидания и ковариационные
моменты переменных $X_i$, $ X_i^0=X_i - m_i$,
 \begin{multline*}
 F_{n0} = F_{n0} (m_1, m_2, k_{11}, k_{12}, k_{22}) ={} \\
{} =\mm_N^{(1,2)} \left[ F_n (X_1, X_2)\right]\,;
 \end{multline*}
$$
 F_{n1} =\fr{\partial F_{n0}}{\partial m_1}\,; \enskip\enskip
 F_{n2} =\fr{\partial F_{n0}}{\partial m_2}\,; %\label{e6}
 $$
 \begin{multline*}
  F_{n0}' ={} \\
{} =F_{n0}' (m_1, m_2, m_4, k_{11}, k_{12}, k_{22}, k_{14}, k_{24}, k_{44})={}\\
{}= \mm_N^{(1,2,4)} \left[ F_n' (X_1, X_2, X_4)\right]\,;
 \end{multline*}
 \begin{equation*}
 F_{n1}' =\fr{\partial F_{n0}'}{\partial m_1}\,;\enskip
 F_{n2}' =\fr{\partial F_{n0}'}{\partial m_2}\,;\enskip
 F_{n4}' =\fr{\partial F_{n0}'}{\partial m_4}\,, %\label{e7}
 \end{equation*}
где $\mm_N^{(1,2)} \left[\, \cdot\, \right]$ и $\mm_N^{(1,2,4)} \left[\, 
\cdot\, \right]$~--- символы вероятностного осреднения для двух- и трехмерного 
нормального распределения переменных $X_1, X_2$ и $X_1,X_2, X_4$.
\end{description}

В результате стохастическое нелинейное дифференциальное уравнение
второго порядка~(1) будет эквивалентно нелинейной системе для
математических ожиданий $m_i = \mm X_i$ $\left(i=\overline{1,4}\right)$:
 \begin{align}
 \dot m_1 &= m_2\,;\notag\\
 \dot m_2 &= M_0^{\mathrm{Э}}\,;\label{e8}\\
 \dot m_l &= -\alp_l \left(m_l - m_l^*\right)\enskip (l=3,4)\,,\notag
 \end{align}
где введены обозначения:
 \begin{multline*}
 M_0^{\mathrm{Э}}= \tilde M_{0t} + m_0^{SL} - \left(\mu_1 +\tilde \mu_{1t}\right) m_2 -{}\\
{}-\mu_1 \left(m_2 m_3 + k_{23}\right)
+\left(\mu_n +\tilde \mu_{nt}\right) F_{n0} + \mu_n F_{n0}'\,;\\[-21pt] %\label{e9}
 \end{multline*}
 \begin{multline*}
 \tilde M_{0t} = M_{10}^S \cos\left(2\pi f_\Gamma t +\chi_1^S\right)+{}\\
{}+M_{20}^S \cos\left(4\pi f_\Gamma t +\chi_2^S\right)+
 M_{m0}^L \cos\left(2\pi \nu_m t +\chi_m^L\right)+ {}\\
 {}+ M_{f0}^L  \cos\left(2\pi \nu_f t+\chi_f^L\right)\,;\\[-21pt] %\label{e10}
 \end{multline*}
 \begin{multline}
 \tilde\mu_{lt} = \mu_l\left[ \pi_{l1}^{\mu S} \cos\left(2\pi f_\Gamma t +
\chi_{l1}^{\mu S}\right)+{}\right.\\
{}+\pi_{l2}^{\mu S} \cos\left(4\pi f_\Gamma t +\chi_{l2}^{\mu S}\right)+
 \pi_{lm}^{\mu L} \cos\left(2\pi \nu_m t +\chi_{lm}^{\mu L}\right)+{}\\
\left. {}+\pi_{lf}^{\mu L} \cos\left(2\pi \nu_{lf} t +\chi_{lf}^{\mu 
L}\right)\right]
 \enskip (l=1,3)\,,\label{e11}
 \end{multline}
и линейной системе для центрированных составляющих:
\begin{align*}
\dot X_1^0 &= X_2^0\,;\\
 \dot X_2^0 &= V_1 + \left[ \left(\mu_n +\tilde\mu_{nt}\right) F_{n1} + \mu_n F_{n1}'\right] X_1^0-{}\\
&\hspace*{25pt}{}- \left[ \left(\mu_1 +\tilde\mu_{1t}\right)+\mu_1m_3-\!\left(\mu_n +\tilde\mu_{nt}\right)F_{n2} -{}\right.\\
&\hspace*{40pt}\left. {}-\mu_n F_{n2}'\right] X_2^0- \mu_1 m_2 X_3^0 + \mu_n F_{n4}' X_4^0\,;\\ %\label{e12}
\dot X_l^0 &=\alp_l X_l^0 +\si_l \sqrt{2\alp_l} V_2\,.
\end{align*}
Обозначая через $\eps$ и $\beta$ матрицы
\begin{align}
 \eps = \eps(t) &=
\begin{bmatrix}
 0&1&0&0\\
 \eps_{21}&\eps_{22}&\eps_{23}&\eps_{24}\\
 0&0&-\alp_3&0\\
 0&0&0&-\alp_4
\end{bmatrix}\,;\notag\\[-6pt]
&\label{e13}\\[-6pt]
 \beta &=\begin{bmatrix}
 0&0\\
 1&0\\
 0&\si_3\sqrt{2\alp_3}\\
 0&\si_34\sqrt{2\alp_4}
\end{bmatrix}\,,\notag
\end{align}
где
\begin{align}
 \eps_{21} &= \left(\mu_n +\tilde\mu_{nt}\right) F_{n1} +\mu_n F_{n1}'\,;\notag \\
 \eps_{22} &= -\left(\mu_1 +\tilde\mu_{1t}\right)-\mu_1 m_3+{}\notag \\
 &\hspace*{50pt}{}+ \left(\mu_n +\tilde\mu_{nt}\right) F_{n2} +\mu_n F_{n2}'\,; \label{e14}\\
 \eps_{23} &=-\mu_1m_2\,;\notag\\
 \eps_{24} &= \mu_n F_{n1}'\,,\notag
\end{align}
представим уравнения для ковариационной матрицы $K(t)=\left[ k_{ij}(t)\right]$, 
$k_{ij}(t) =\mm \left[ X_i^0 (t) X_j^0 (t)\right]$ и мат\-ри\-цы для 
ковариационных функций $K(t_1, t_2) = $\linebreak $=\left[ K_{ij} (t_1, 
t_2)\right]$, $ K_{ij} (t_1, t_2) =\mm \left[ X_i^0 (t_1) X_j^0 (t_2)\right]$ 
$(i,j=$\linebreak $=\overline{1,4})$ в следующем виде~[13]:
\begin{align}
\dot K(t) &= \eps (t) K(t) + K(t) \eps^T(t) +\beta \nu \beta^T\,;\notag\\[-6pt]
  \label{e15}\\[-6pt]
 k_{ij} (0) &= k_{ij0}\,;\notag\\
 \fr{\partial K(t_1, t_2)}{\partial t_2} &= K(t_1, t_2) \eps^T(t_2)\,;\notag\\[-6pt]
  \label{e16}\\[-6pt]
 K_{ij} (t, t') &= k_{ij} (t)\,,\notag
\end{align}
где $\nu =\left[ \nu_{ij}\right]$ $( \nu_{11} = \nu_1 (t)$, $\nu_{12} =\nu_{12} 
(t)$, $ \nu_{22}=1)$~--- матрица интенсивностей белых шумов $V_1$ и $V_2$.

Уравнения (7) и (8) допускают следующую развернутую запись:
\begin{align}
 \dot k_{11} &= 2 k_{12}\,;\notag\\
 \dot k_{12} &=\eps_{21} k_{11} +\eps_{22} k_{12} +\eps_{23} k_{13} +\eps_{24} k_{14} + k_{22}\,;\notag\\
 \dot k_{22}&= 2 \left( \eps_{21} k_{12}+ \eps_{22} k_{22} + \eps_{23} k_{23} +\eps_{24} k_{24}\right)+\nu_1\,;\notag\\
  \dot k_{13} &= k_{23} -\alp_3 k_{13}\,;\notag\\
 \dot k_{23}&= \eps_{21} k_{13} +\eps_{22} k_{23} + \eps_{23} k_{33} + \eps_{24} k_{34} -{}\notag\\
 &\hspace*{90pt}{}-\alp_3 k_{23} +\nu_{12} \si_3 \sqrt{2\alp_3}\,;\notag\\[-6pt]
 &\label{e17}\\[-6pt]
 \dot k_{33} &=-2\alp_3 \left(k_{33} - \si_3^2\right)\,;\notag\\
   \dot k_{14} &= k_{24} -\alp_4 k_{14}\,;\notag\\
 \dot k_{24} &=\eps_{21} k_{14} +\eps_{22} k_{24}+ \eps_{23} k_{34} +\eps_{24} k_{44} -{}\notag\\
 &\hspace*{90pt}{}-\alp_4 k_{24} +\nu_{12} \si_4 \sqrt{2\alp_4}\,;\notag\\
 \dot k_{34} &=- \left(\alp_3+\alp_4\right) k_{34} + 2 \si_3 \si_4\sqrt{ \alp_3 \alp_4}\notag\\
 \dot k_{44} &=- 2\alp_4 \left(k_{44} -\si_4^2\right)\notag
\end{align}
при начальных условиях $k_{ij} (0) = k_{ij0}$ и
\begin{align}
 \fr{\prt K_{11} (t_1, t_2)}{\prt t_2} &= K_{12} (t_1, t_2)\,;\notag\\
 \fr{\prt K_{21} (t_1, t_2)}{\prt t_2} &= K_{22} (t_1, t_2)\,;\notag\\
 \fr{\prt K_{31} (t_1, t_2)}{\prt t_2} &=K_{32}(t_1, t_2)\,;\notag\\
 \fr{\prt K_{41} (t_1, t_2)}{ \prt t_2} &=K_{42}(t_2, t_2)\,;\notag\\
 \fr{\prt K_{12} (t_1, t_2)}{\prt t_2} &=\eps_{21, t_2} K_{11} (t_1, t_2) +\eps_{22, t_2} K_{12} (t_1, t_2)+{}\notag\\
  & {}+\eps_{23, t_2} K_{13} (t_1, t_2) + \eps_{24, t_2} K_{14} (t_1, t_2)\,;\notag\\
 \fr{\prt K_{22} (t_1, t_2)}{ \prt t_2} &=\eps_{21,t_2} K_{21} (t_1, t_2) +\eps_{22, t_2} K_{22} (t_1, t_2) +{}\notag\\
  & {}+\eps_{23, t_2} K_{23} (t_1, t_2) + \eps_{24, t_2} K_{24} (t_1, t_2)\,;\notag\\
 \fr{\prt K_{32} (t_1, t_2)}{ \prt t_2} &=\eps_{21,t_2} K_{31} (t_1, t_2) +\eps_{22, t_2} K_{32} (t_1, t_2) +{}\notag\\
  & \hspace*{-10pt}{}+\eps_{23, t_2} K_{33} (t_1, t_2) + \eps_{24, t_2} K_{34} (t_1, t_2)\,;\notag\\[-6pt]
&\label{e18}\\[-6pt]
 \fr{\prt K_{42} (t_1, t_2)}{ \prt t_2} &=\eps_{21,t_2} K_{41} (t_1, t_2) +\eps_{22, t_2} K_{42} (t_1, t_2) +{}\notag\\
  & {}+\eps_{23, t_2} K_{43} (t_1, t_2) + \eps_{24, t_2} K_{44} (t_1, t_2)\,;\notag\\
 \fr{\prt K_{13} (t_1, t_2)}{ \prt t_2} &=-\alp_3 K_{13} (t_1, t_2)\,;\notag\\
 \fr{\prt K_{23} (t_1, t_2)}{ \prt t_2} &=-\alp_3 K_{23} (t_1, t_2)\,;\notag\\
 \fr{\prt K_{33} (t_1, t_2)}{ \prt t_2} &=-\alp_3 K_{33} (t_1, t_2)\,;\notag
\end{align}
\begin{align}
 \fr{\prt K_{43} (t_1, t_2)}{ \prt t_2} &=-\alp_3 K_{33} (t_1, t_2)\,;\notag\\
 \fr{\prt K_{14} (t_1, t_2)}{ \prt t_2} &=-\alp_4 K_{14} (t_1, t_2)\,;\notag\\
 \fr{\prt K_{24} (t_1, t_2)}{ \prt t_2} &=-\alp_4 K_{24} (t_1, t_2)\,;\notag\\
 \fr{\prt K_{34} (t_1, t_2)}{ \prt t_2} &=-\alp_4 K_{34} (t_1, t_2)\,;\notag\\
 \fr{\prt K_{44} (t_1, t_2)}{ \prt t_2} &=-\alp_4 K_{44} (t_1, t_2)\notag
\end{align}
при начальных условиях $K_{ij}(t,t) = k_{ij}(t)$ $(i,j=$\linebreak
$=\overline{1,4})$.

Совокупность уравнений (3), (9) и (10) определяет аналитическую
нелинейную корреляционную модель флуктуаций приливной внутригодовой
неравномерности вращения Земли.

{\small
\medskip
\textbf{ Замечание 1.} Усредняя по периодам основных
гармоник, отвечающих частотам $f_\Gamma, 2f_\Gamma, \nu_m$ и $\nu_f$
для стационарных регулярных и нерегулярных колебаний и постоянных
$m_h = m_h^*$ $(h= 2,3,4)$, из уравнений (3), (9) и (10) находим
соответствующие уравнения для математических ожиданий,
ковариационной матрицы, матрицы ковариационных функций и матрицы
спектральных плотностей.

\medskip
\textbf{ Замечание 2.} Уравнения (3), (9) и (10) применимы для
нелинейных функций $F_n = F_n (X_1, X_2)$, допускающих угловые
точки и даже разрывы. В случае гладких функций $F_n$ рассмотренный
квазилинейный метод переходит в метод непосредственной линеаризации
в окрестностях математических ожиданий $m_1$ и $m_2$.

\medskip
\textbf{ Замечание 3.} Если в исходном уравнении (1) возмущения
$X^S(t), X^L (t), X_1^{\mu S} (t), X_1^{\mu L}(t),  X_n^{\mu S}(t)$ и
$ X_n^{\mu L}(t)$ заданы совместными каноническими
представлениями, то целесообразно использовать метод нормализации
посредством канонических представлений~[13].

\medskip
\textbf{ Замечание 4.} Для негауссовских возмущений в (1), как
показано в~[13--16], можно воспользоваться методом\linebreak эквивалентной
линеаризации, взяв в качестве осред\-няющего вероятностного
распределения отрезок пара\-мет\-ри\-зо\-ванного (моментами, квазимоментами,
семиинвариантами и др.) разложения одно- и двумерных\linebreak плотностей.
 }

\section{Линейная корреляционная модель флуктуаций неравномерности
вращения Земли (тестовые примеры 1 и 2)} %4


 \textit{П р и м е р~~1\/}. Уравнения (3), (9) и (10) при $\mu_1= \mu$ и $\mu_n =0$
(отсутствуют параметрические и нелинейные возмущения) имеют
следующий вид:

\noindent
 \begin{align}
 \dot m_1 &= m_2\,;\notag\\
 \dot m_2 &= \tilde M_{0t} - \left(\mu +\tilde\mu_t\right) m_2 -{}\notag\\[-6pt]
& \label{e19}\\[-6pt]
 &\hspace*{50pt}{}-\mu \left(m_2 m_3 + k_{23}\right)+m_0^{SL}\,;\notag\\
  \dot m_3 &=-\alp_3 \left(m_3-m_3^*\right)\,;\notag\\[3pt]
 \dot k_{11} &= 2 k_{12}\,;\notag\\
 \dot k_{12} &= k_{22} -\left[\mu\left(1+m_3 \right)+\tilde\mu_t\right] k_{12} -\mu m_2 k_{13}\,;\notag\\
 \dot k_{22} &= - 2\left[\mu\left(1+m_3\right) +\tilde\mu_t\right] k_{22} - 2 \mu m_2 k_{23} +\nu_{11}\,;\notag\\
 \dot k_{13} &= -\alp_3 k_{13} + k_{23}\,;\label{e20}\\
 \dot k_{23}&= -\left[\alp_3 +\mu\left(1+m_3 \right)+\tilde\mu_t\right] k_{23} -{}\notag\\
 &\hspace*{70pt}{}- \mu m_2 k_{33}+ \nu_{12}\si_3\sqrt{2\alp_3}\,;\notag\\
 \dot k_{33} &=-2\alp_3 \left(k_{33} -\si_3^2\right)\notag
 \end{align}
при начальных условиях $K_{ij} (0)= k_{ij,0}$ ($i,j= 1$, 2, 3) и
 \begin{align}
 \fr{\prt K_{11} (t_1, t_2)}{ \prt t_2} &= K_{12} (t_1, t_2)\,;\notag\\
 \fr{\prt K_{21} (t_1, t_2)}{ \prt t_2} &= K_{22} (t_1, t_2)\,;\notag\\
 \fr{\prt K_{31} \left(t_1, t_2\right)}{\prt t_2} &=K_{32}\left(t_1, t_2\right)\,;\notag\\
 \fr{\prt K_{12} (t_1, t_2)}{\prt t_2} &=-\left[\mu(1+m_{3t_2} )+\tilde\mu_{t_2}\right] K_{12}(t_1, t_2) -{}\notag\\
 &\hspace*{60pt}{}-\mu m_{2t_2} K_{13} (t_1, t_2)\,;\notag\\
 \fr{\prt K_{22} (t_1, t_2)}{ \prt t_2} &=-\left[\mu\left(1+m_{3t_2} \right)+\tilde\mu_{t_2}\right] K_{22} \left(t_1, t_2\right) -{}\notag\\[-6pt]
& \label{e21}\\[-6pt]
 &\hspace*{60pt}{}-\mu m_{2t_2} K_{23} (t_1, t_2)\,;\notag\\
 \fr{\prt K_{32} \left(t_1, t_2\right)}{ \prt t_2} &= -\left[\mu\left(1+m_{3t_2} \right)+\tilde\mu_{t_2}\right] K_{32} \left(t_1, t_2\right) -{} \notag\\
 &\hspace*{60pt}{}-\mu m_{2t_2} K_{33} \left(t_1, t_2\right)\,;\notag\\
 \fr{\prt K_{13} (t_1, t_2)}{ \prt t_2} &=-\alp_3 K_{13}(t_1, t_2)\,;\notag\\
 \fr{\prt K_{23} (t_1, t_2)}{ \prt t_2} &=-\alp_3 K_{23} (t_1, t_2)\,;\notag\\
 \fr{\prt K_{33} \left(t_1, t_2\right)}{\prt t_2} &=-\alp_3 K_{33} \left(t_1, t_2\right)\notag
 \end{align}
при начальных условиях $K_{ij}(t,t) = k_{ij}(t)$ ($i,j=1$, 2, 3).
Здесь $\tilde \mu_t =\tilde\mu_{lt}$ определены в (11), а $\nu_1 =
\nu_{11}$~--- интенсивность белого шума $V_1$; $\nu_{12}$~---
взаимная интенсивность шумов $V_1$ и $V_2$.

Рассмотрим основные свойства модели (11)--(13).
В условиях аддитивных гармонических и случайных моментов $(\tilde
M_{0t} \ne 0$, $ X^S (t)\ne 0$, $X^L (t)\ne 0)$ при отсутствии
диссипативных моментов $(\mu=0)$ имеют место следующие явные
выражения для моментов первого и второго порядков:

\noindent
 \begin{align}
 m_1 &= m_{10} + m_{20} t +m_0^{SL} \fr{t^2}{2} +{}\notag\\
&\hspace*{60pt}{}+ \iii_0^t\iii_0^{t_1} \tilde M_{0t_1}\,dt_1\,dt\,;\label{e22}\\
 m_2 &= m_{20} +m_0^{SL} t+\iii_0^t \tilde M_{0t_1}dt_1,\notag\\[3pt]
 k_{11} &=\fr{1}{3} \nu_{11} t^3 + k_{22,0} t^2 + 2 k_{12,0} t + k_{11, 0}\,;\notag\\
 k_{12} &= \fr{\nu_{11} t^2}{2}+ k_{22,0}t + k_{12,0}\,;\label{e23}\\
 k_{22} &= \nu_{11} t + k_{22,0}\,;\notag\\[3pt]
  K_{11}(t_1, t_2) &= k_{11} (t_1) + k_{12}(t_1) (t_2-t_1)\,;\notag\\
 K_{12}(t_1, t_2) &= k_{12} (t_1) \,; \notag\\
 K_{22} (t_1, t_2) &= k_{22} (t_1)\,.\notag %\label{e24}
 \end{align}

Из формул (14) и (15) следует, что при нулевых начальных условиях,
во-первых, постоянный момент $m_0^{SL}$ вызывает систематические
временные дрейфы по переменным $\delta\varphi$ и $\delta\dot\varphi$;
во-вторых, гармонический момент $\tilde M_{0t}$ может приводить к
накопленной интегральной ошибке по переменной $\delta\varphi$;
в-третьих, аддитивный белый шум $V_1$ вызывает флуктуационный временной
дрейф по переменным $\delta\varphi$ и $\delta\dot\varphi$. Ненулевые
начальные условия по переменной $\delta\dot\varphi$ вызывают флуктуационный
дрейф по переменной $\delta\varphi$.

 \textit{ П р и м е р~~2\/}. При линейной диссипации, когда $\mu\ne 0$ для $t>\!\!>\mu^{-1}$ и начальных условиях
$m_{10}=0$, $m_{20}=0$, $m_{30} = m_3^*$ и $m_0^{SL} =0$ усредненные
на интервалах времени $2\pi / f_\Gamma,$ $2\pi /\nu_m$ и $2\pi/\nu_f$
математические ожидания, дисперсии и ковариационные моменты имеют вид:
 \begin{align*}
 \langle m_1 \rangle &=-\fr{\la t^2}{2}\,;\notag\\[-2pt]
 \langle m_2 \rangle &=-\la t\,;\notag\\
 \langle m_3 \rangle &=m_3^*\,,\notag\\ %\label{e25}
% \end{align*}
\intertext{где $ \la = \langle  \tilde\mu_t \iii_0^t \tilde M_{0t_1} dt_1 \rangle$;}
% \begin{align*}
 \langle k_{11} \rangle &= \fr{\nu_{11}}{\mu^2} t\,;\notag\\[-5pt]
 \langle k_{12} \rangle &= \fr{\nu_{11}}{ 2\mu^2} \,;\notag\\[-5pt]
 \langle k_{22} \rangle &= \fr{\nu_{11}}{ 2\mu}\,;\notag\\[-5pt]
 \langle k_{13} \rangle &= \si_3\sqrt{2\alp_3}\fr{\nu_{12}}{ \alp_3\left(\mu +\alp_3\right)}\,;\notag\\[-5pt]
 \langle k_{23} \rangle &= \si_3\sqrt{2\alp_3}\fr{\nu_{12}}{\mu +\alp_3}\,. %\label{e26}
 \end{align*}
Здесь $\langle\ldots\rangle$~--- символ усреднения. Отсюда видно, что
имеют место усредненные систематические дрейфы $\langle m_1
\rangle$ и $\langle m_2 \rangle$ и флуктуационный дрейф $\langle
k_{11} \rangle$. При этом $\langle k_{ij} \rangle$ $(i,j=2,3)$
постоянны и зависят от интенсивности $\nu_{11}$ аддитивного шума
$V_1$ и взаимной интенсивности аддитивного и параметрического шумов
$V_1$ и $V_2$. Для $t\ll \mu^{-1}$ соответствующие результаты
следуют из (14) и (15).


\section{Квазилинейная корреляционная модель флуктуаций при аддитивных
возмущениях и нелинейной рэлеевской диссипации (тестовый пример~3)} %5

 \textit{ П р и м е р~~3\/}. В случае рэлеевского механизма диссипации, когда $F_n
=\delta\dot\varphi^3$, имеем следующие выражения для коэффициентов
статистической линеаризации функций $F_n$ и $F_n'$:
\begin{align}
 F_n &= X_2^3 \approx F_{0n} + F_{n2} X_2^0 ={}\notag \\
 &\hspace*{15pt}{}=m_2^3 + 3 k_{22} m_2 + 3\left( m_2^2 + k_{22}\right) X_2^0\,;\label{e27}\\
 F_n' &= X_2^3 X_4 \approx F_{0n}' + F_{n2}' X_2^0 + F_{n4}' X_4^0 ={}\notag\\
 &{}= m_2^3 m_4 + 3 m_2 m_4 k_{22} + 3 m_2^2 k_{24} + 3 k_{22} k_{24} +{}\notag\\
 &\hspace*{30pt}{}+ 3\left(m_2^2 m_4 + m_4 k_{22} + 2 m_2 k_{24}\right) X_2^0 +{}\notag\\
&\hspace*{60pt}{}+ m_2 \left(m_2^2 + 3 k_{22}\right) X_4^0\,.\label{e28}
\end{align}

Тогда при аддитивных гармонических и случайных возмущениях уравнения~(3)
и~(9) для $\delta \varphi = X_1$ и $\delta \dot\varphi = X_2$
имеют следующий вид:
 \begin{align}
 \dot m_1 &= m_2\,;\notag\\[-6pt]
& \label{e29}\\[-6pt]
 \dot m_2 &=\tilde M_{0t}+ m_0^{SL} -\mu_1 m_2 +\mu_3 \left(m_2^3 + 3 m_2 k_{22}\right)\,;\notag\\[3pt]
 \dot k_{11} &= 2 k_{12}\,;\notag\\
 \dot k_{12} &= k_{22} -\mu^{\mathrm{Э}} k_{12}\,;\label{e30}\\
 \dot k_{22} &=- 2 \mu^{\mathrm{Э}} k_{22} +2\mu_3 m_2\left(m_2^2+ k_{22}\right)+\nu_{11}\,,\notag
 \end{align}
где $\mu^{\mathrm{Э}} =\left[ \mu_1 - 3\mu_3 \left(m_2^2 + k_{22}\right) 
\right]$. Отсюда при $m_0^{SL}=0$ для $t\ll 1/\mu^{\mathrm{Э}}$ после 
усреднения стационарные решения $m_2^*$ и $k_{22}^*$ будут определяться 
уравнениями:
 \begin{align}
 -\mu_1 m_2^* + \mu_3 \left(m_{2}^{*2} + 3k_{22}^*\right)m_2^* &=0\,;\label{e31}\\[6pt]
 -2\left[ \mu_1 - 3\mu_3 \left(m_2^{*2} + k_{22}^*\right)\right] k_{22}^* + {}\hspace{30pt}&\notag\\
 {}+2 \mu_3 m_2^* \left(m_2^{*2} + k_{22}\right) +\nu_{11}&=0\,.\label{e32}
\end{align}
При условии
 \begin{equation*}
 \mu^{\mathrm{Э}} = \fr{1}{ 2} \left( \mu_1 \pm
\sqrt{\mu_1^2 - 6\mu_3 \nu_{11}}\right) >0\,,\label{e33}
 \end{equation*}
во-первых, имеют место стационарные решения:
 \begin{equation*}
 m_2^* =0,\enskip
k_{22}^* = \fr{\mu_1 \pm \sqrt{\mu_1^2 - 6 \mu_3\nu_{11}}}{6\mu_3}\,, %\label{e34}
 \end{equation*}
отвечающие двум режимам нерегулярных колебаний, соответственно, при
малых и больших $\mu_3$. Во-вторых, уравнения (20) и (21) допускают
стационарные решения $m_2^*$ и $k_{22}^*$, определяемые из уравнения:
 \begin{equation*}
 -\mu_1 +\mu_3 \left( m_2^{*2} + 3 k_{22}^*\right)=0\,. %\label{e35}
 \end{equation*}
%и уравнения (21). 
Они отвечают смещенным на величину $m_2^*$
нерегулярным колебаниям.

{\small

\medskip

\textbf{ Замечание~5.} По переменной $\delta \varphi = X_1$ стационарные
регулярные и нерегулярные колебания сопровождаются систематическим и
флуктуационным дрейфами, определяемыми в силу~(9) уравнениями:
 \begin{align*}
 \dot m_1 &= m_2^*\,;\notag\\
 \dot k_{11} &= 2 k_{12}\,;\notag\\
 \dot k_{12} &= k_{22} -\langle\mu^{\mathrm{Э}}\rangle
 k_{12} - \mu_1 m_2^* k_{13} +\mu_3 m_2^* \left( m_2^{*2} + k_{22}^*\right) k_{14}\,;\notag\\
 \dot k_{13} &= k_{23}^* -\alp_3 k_{13}^*\,;\notag\\
 \dot k_{14} &= k_{24}^* -\alp_4 k_{14} %\label{e37}
 \end{align*}
при начальных условиях: $m_1=m_{10}$, $ k_{11}= k_{110}$, $ k_{12}=
k_{120}$, $ k_{13}= k_{130},$ $k_{14}=k_{140}$.

\medskip

\textbf{ Замечание~6.} В силу известного линейного соотношения
между вариациями скорости вращения Земли $\delta\dot\varphi (t)$ и
длительностью суток $(l.o. \, d(t))$~[14]
 \begin{equation*}
 l. o. \, d(t)=-\fr{86400}{r_*}\delta \dot\varphi(t)\label{e38}
 \end{equation*}
 уравнения (3), (9), (10), а также (18)--(21) позволяют получить
соответствующие уравнения для корреляционных характеристик $l. o. \,d(t)$.
 }

\section{Квазилинейные информационные модели флуктуаций неравномерности
вращения Земли по~апостериорным данным} %6

Примем за информационные переменные $Z_1$ и~$Z_2$, допускающие
измерения $X_1$ и $X_2$. Положим
 \begin{equation}
 Z_1 = X_1+V_3\,;\enskip Z_2 = X_2 + V_4\,,\label{e39}
 \end{equation}
где $X_1 =\delta\varphi$, $ X_2 =\delta \dot\varphi$; $V_3$ и $V_4$~---
независимые нормальные белые шумы с интенсивностями $\nu_3$ и $\nu_4$
соответственно. Тогда совокупность уравнений (1), (2) и (22) будет
представлять собой исходную систему уравнений для синтеза квазилинейного
фильтра для обработки информации о флуктуациях неравномерности вращения
Земли по апостеорным данным, т.\,е.\ по результатам измерения $Z_1$ и $Z_2$.

Для построения квазилинейного нормального фильтра согласно~[15, 16]
перепишем уравнения (1), (2) и (22) в следующем стандартном виде:
 \begin{equation}
 \dot X = a (X,t) + b(t) \bar V_1\,;\enskip Z= a_1 (X, t) +\bar V_2\,.\label{e40}
 \end{equation}
Здесь
\begin{align*}
 X&=\left[ X_1 X_2 X_3 X_4\right]^T\,;\\% \quad
 Z&= \left[ Z_1 Z_2\right]^T\,;\notag\\
 \bar V_1 &= \left[ V_1 V_2\right]^T\,;\\%\quad
 \bar V_2& =\left[ V_3 V_4\right]^T\,;\notag\\
 a&=a(X,t) ={} \\
{}&=\begin{bmatrix}
 X_2\\
 \tilde M_{0t} m_{0}^{SL} - \left[ \left(\mu_1 +\tilde\mu_{1t}\right) +\mu_1 X_3 \right] X_2 +{}\\
 {}+\left[\left(\mu_n+\tilde\mu_{nt}\right) +\mu_n X_4\right] F_n(X_1, X_2)\\
 -\alp_3 (X_3 - m_3^*)\\
 -\alp_4(X_4-m_4^*)
 \end{bmatrix}\!\,;\\
 a_1&=a_1(X,t) =
 \begin{bmatrix}
 X_1\\
 X_2\\ \end{bmatrix}\,;\notag\\
 b&=b(t) =
 \begin{bmatrix}
 0&0\\
 1&0\\
 0&\si_3\sqrt{2\alp_3}\\
 0&\si_4 \sqrt{2\alp_4}\\
 \end{bmatrix}\,;\notag\\
 \bar\nu_1 &=\begin{bmatrix}
 \nu_1&0\\
 0&\nu_2\\
 \end{bmatrix}\,;\\% \quad
\bar\nu_2&=\begin{bmatrix}
 \nu_3&0\\
 0&\nu_4\\
 \end{bmatrix}\,. %\label{e41}
%\end{gather*}
\end{align*}

Заменим (23) статистически линеаризованной системой уравнений, нелинейной 
относительно математических ожиданий $m^x =\left[ m_1^x m_2^x m_3^x 
m_4^x\right]^T$, $m^z=\left[ m_1^z m_2^z\right]$ и линейной относительно 
центрированных составляющих $X^0 = X-m^x$:
\begin{align}
 \dot m^x &= a_{00},&\enskip m^z &= a_{10},\label{e42}\\
 \dot X^0 &= a_{01} X^0 +\psi (t) \bar V_1,&\enskip Z^0 &= a_{11} X^0 + V_2\,.\label{e43}
\end{align}
Здесь $a_{ij} = a_{ij} \left(m^x, K^x,t\right)$ $(i,j=0,1)$~--- коэффициенты
статистической линеаризации функций $a=a(X,t)$ и $a_1= a_1 (X,t)$.
При этом ковариационная матрица $K_x$ определяется уравнением вида~(7):
 \begin{equation}
 \dot K^x = a_{01} K^x + K^x a_{01}^T + b\bar\nu_1 b^T\,.\label{e44}
 \end{equation}
Применяя к (24) и (25) уравнения фильтра Кал\-ма\-на--Бью\-си~[15, 16],
получим
\begin{align}
 {\dot{\hat X}} &= a_{00} - a_{01} m^x + a_{01} \hat X +{}\notag\\
 &\hspace*{10pt}{}+ R a_{11} \bar\nu_2^{-1}\left(Z-a_{11} \hat X - a_{10} + a_{11} m^x\right)\,;\label{e45}\\
\hat X_0 &=\mm X(t_0)\,;\notag\\[3pt]
 \dot R &= a_{01} R + R a_{01}^T - R a_{11}^T \bar \nu_2^{-1} a_{11} R + b\bar \nu_1 b^T\,;\notag\\[-6pt]
&\label{e46}\\[-6pt]
 R_0 &=\mm \left[ \left(X_0 -\hat X_0\right)\left(X_0 -\hat X_0\right)^T\right]\,.\notag
\end{align}
Совокупность фильтрационных уравнений (27) и (28) при условиях (24) и~(26)
определяет искомый квазилинейный нормальный фильтр для обработки информации
о флуктуациях неравномерности вращения Земли по апостериорным данным,
в том числе в темпе получения результатов наблюдения. Коэффициенты
статистической линеаризации $a_{00}$ и $a_{01}$ приведены в разд.~2,
а $a_{10}=0$, $ a_{11} = I_2$ (в силу линейности второго уравнения (23)).

Фильтрационные уравнения (27) и (28) для произвольного линейного
наблюдения, когда $a_1(X,t) = b_1 (t) X + b_0$, $ a_{11} = b_1 (t)$,
принимают следующий вид:
\begin{align}
 {\dot{\hat X}} &= a_{00} - a_{01} m^x + a_{01} \hat X +{}\notag \\
 &\hspace*{10pt}{}+\beta \left[ Z- b_1(t) \hat X -b_0(t) + b_1(t) m^x\right] \,;\label{e47}\\
 \hat X_0& =\mm X_0\,;\notag\\[3pt]
 \dot R &= a_{01} R + R a_{01}^T - \beta b_1(t) R + b(t) \bar\nu_1 (t) b(t)^T\,;\notag\\[-3pt]
 \label{e48}\\[-6pt]
 R_0 &=\mm \left[ \left(X_0 -\hat X_0\right)\left(X_0 -\hat X_0\right)^T\right]\,.\notag
\end{align}
если через $\beta$ обозначить коэффициент усиления фильтра
$\beta = R b_1 (t)^T \bar\nu_2^{-1}$.

Коэффициенты статистической линеаризации $a_{00}, a_{01}$ и
вспомогательная инструментальная мат\-ри\-ца ошибки фильтрации $R$ не
содержат результатов наблюдений и могут быть определены отдельно (до
получения результатов наблюдений).\linebreak Таким образом, возможна априорная
оценка точ\-нос\-ти квазилинейного фильтра.

{\small
\medskip

 \textbf{ Замечание~7.} В случае, когда известны канонические
 представления возмущений в (1), используются соответствующие версии
 квазилинейного метода~[15, 16].

\medskip

\textbf{ Замечание~8.} Для негауссовских возмущений в (1) можно
воспользоваться методом эквивалентной линеаризации, взяв в качестве
осредняющего распределения отрезок параметризованного распределения,
например плотности, а затем использовать уравнения фильтра
Кал\-ма\-на--Бьюси или Пугачева~[15, 16].
 }

\section{Линейные фильтры для~обработки информации о~неравномерности
скорости вращения Земли (тестовые примеры 4--6)} %7

 \textit{ П\ р\ и\ м\ е\ р~~4\/}. Пусть отсутствуют диссипативные силы
 $\left(\mu_1 =0\right)$. Тогда уравнения (23), (27) и (28) можно предствить
 в виде:
\begin{align}
 \dot X_2 &= \tilde M_{0t} + m_0^{SL} + V_1,\enskip Z_2 = X_2 + V_2\,; \notag \\ %\label{e49}\\
 {\dot{\hat X}}_2 &= \tilde M_{0t} + m_0^{SL} +R \nu_2^{-1} \left(Z_2 -\hat X_2\right),\enskip
 \hat X_{2, t_0} =0\,; \notag \\ %\label{e50}\\
\dot R &= - R^2 \nu_2^{-1} + \nu_1\,;\enskip R_{t_0} = R_0\,.\label{e51}
\end{align}
Решение~(31) допускает аналитическую запись~[16]:
 \begin{equation*}
 R= \sqrt{\nu_1\nu_2} \,\,
\fr{1+\gamma \exp \left(-2\gamma_0 t\right)}
{1 -\gamma \exp \left(-2\gamma_0 t\right)}\,, %\label{e52}
 \end{equation*}
где\\[-6pt]
 \begin{equation*}
 \gamma= \fr{R_0-R_{\infty} }{R_0+R_{\infty}}\,;\enskip
\gamma_0^2 =\nu_1 \nu_2^{-1},\enskip R_\infty =\sqrt{\nu_1\nu_2}\,.
 \end{equation*}
При $t\to \infty$\\[-6pt] 
 \begin{equation*}
 R\to R_\infty \mbox{ и } {\dot{\hat X}}_2 =\tilde M_{0t} +m_0^{SL} +
 \sqrt{\fr{\nu_1}{\nu_2}} \left( Z_2 -\hat X_2\right)  \,. %\label{e53}
 \end{equation*}
 
 
 \textit{ П\ р\ и\ м\ е\ р~~5\/}. В случае чисто линейных диссипативных
 сил $(\mu_1 \ne 0$, $ \tilde\mu_{1t} \equiv 0$, $\mu_n=0$,
 $\mu_{nt}=0)$ уравнения (23), (29) и (30) можно представить в виде:
 \begin{align}
 \dot X_2 &=-\mu_1 X_2 +\tilde M_{0t} +m_0^{SL} + V_1\,;\enskip
 Z_2 = X_2 +V_2\,; \notag \\ %\label{e54}\\[6pt]
 {\dot{\hat X}}_2 &=-\mu_1 \hat X_2+ \tilde M_{0t} +m_0^{SL} +\beta \left(Z_2 -\hat X_2\right)\,;\notag\\[-6pt]
&\label{e55}\\[-6pt]
 \hat X_{2, t_0} &= m_0\,;\notag\\[6pt]
 \beta&= R\nu_2^{-1}\,;\label{e56}\\
 \dot R &=- 2\mu_1 R - R^2 \nu_2^{-1} +\nu_1\,;\enskip R_{t_0} = R_0\,.\label{e57}
 \end{align}
Решение (34) имеет вид~[16]:
 \begin{align}
 R&= R_\infty +\rho\,;\enskip R_\infty = R_\infty^+\,;\notag\\[-6pt]
 &\label{e58}\\[-6pt]
 R_\infty^\pm &=\nu_2 \left(\sqrt{\mu_1^2 +\nu_1\nu_2^{-1}} \pm \mu_1\right)\,,\notag
 \end{align}
где\\[-6pt]
 $$\rho=\frac{R_\infty^+ + R_\infty^-}{\left(R_0 +R_\infty^-\right)\left(R_0 +R_\infty^+\right)^{-1}
e^{2\sqrt{\mu_1^2 +\nu_1\nu_2^{-1}} t}-1}\,,$$
причем $ R\to R_\infty$ при $t\to\infty$.

 \textit{ П\ р\ и\ м\ е\ р~~6\/}. В условиях примера~5, если нормальные
 шумы $V_1$ и $V_2$ коррелированы $(\nu_{12}\ne 0)$, уравнения (32)--(35)
 имеют вид:
  \begin{align*}
 \dot X_2 &=-\mu_1 \hat X_2 +\tilde M_{0t} +m_0^{SL} +{} \\
&\hspace*{50pt}  {}+\left(R+\nu_{12}\right) \nu_2^{-1} \left(Z_2 -\hat X_2\right)\,;\\ %\label{e59}\\
 \dot R &=- 2\left(\mu_1 +\nu_{12} \nu_2^{-1}\right)R+R^2 +\nu_2^{-1} +{}\\
 &\hspace*{110pt}{}+\nu_1- \nu_{12}^2 \nu_2^{-1}\,;\\ %\label{e60}\\
 R&= R_\infty +\rho\,;\notag
 \end{align*}
где
\begin{align*}
 \rho &=\fr{ \left(R_\infty^+ + R_\infty^-\right) \left(R_0 + R_\infty^+\right)}{\left(R_0 +R_\infty^-\right)
 \exp \left(\gamma t\right) - \left(R_0+R_\infty^+\right)}\,;\notag\\
 \gamma^2 &= \left(\mu_1 +\nu_{12} \nu_2^{-1}\right) + \nu_1 \nu_2^{-1}\,;\notag\\
 R_\infty^\pm &=\nu_2 \left(\gamma_\pm \mu_1 - \nu_{12}\nu_2^{-1}\right)\,. %\label{e61}
 \end{align*}

\section{Линейные фильтры для~обработки информации о~неравномерности угла
поворота Земли (тестовые примеры 7 и 8)}  %8

 \textit{ П\ р\ и\ м\ е\ р~~7\/}. При наличии чисто линейной диссипации
 $(\mu_3 =0)$ уравнения (23), (27) и (28) имеют следующий вид:
$$
 \dot X_1 = X_2\,;\enskip \dot X_2 =- 2 \mu_1 X_1 +\tilde M_{0t} + m_0^{SL} + V_1\,;$$ %\notag\\
$$ Z_1= X_1+ V_2\,;\enskip Z_2 = X_2 + V_3, %\label{e62}\\
$$
 \begin{align*}
{\dot{\hat X}}_1 &=\hat X_2 + R_{11} \nu_2^{-1} \left(Z_1-\hat X_1\right) +{}\\
& \hspace*{70pt}{}+ R_{12} \nu_3^{-1} \left(Z_2 -\hat X_2\right)\,;\notag\\
 {\dot{\hat X}}_2 &=-2\mu_1\hat X_2 +\tilde M_{0t} + m_0^{SL} +{}\\
&{}+R_{12} \nu_2^{-1} \left(Z_1 -\hat X_1\right) + R_{22} \nu_3^{-1} \left(Z_2 -\hat X_2\right)\,, %\label{e63}
\end{align*}
где
\begin{align*}
 \dot R_{11} &= 2 R_{12} + R_{11}^2 \nu_2^{-1} + R_{12}^2 \nu_3^{-1}\,,\notag\\
 \dot R_{12} &= R_{22} - 2 \mu_1 R_{12} + R_{11} R_{12} \nu_2^{-1} + R_{12} R_{22} \nu_3^{-1}\,,\notag\\
 \dot R_{22} &= \nu_1 - 2 \mu_1 R_{22} + R_{12}^2 \nu_2^{-1} + R_{22}^2 \nu_3^{-1}\,. %\label{e64}
 \end{align*}

 \textit{ П\ р\ и\ м\ е\ р~~8\/}. В условиях примера~7, если измеряются
 только флуктуации угла $\delta \varphi$, следует пренебречь членами,
 содержащими $\nu_3^{-1}$, а в случае измерения только флуктуаций
 скорости $\delta \dot\varphi$~--- пренебречь членами, содержащими
 $\nu_2^{-1}$. В~[16] содержится ряд других примеров для случая
 автокоррелированных шумов в уравнениях (23).

\section{Квазилинейные фильтры для~обработки информации о~неравномерности
вращения Земли в случае нелинейного рэлеевского механизма диссипации
(тестовые примеры~9 и~10)} %9

 \textit{ П\ р\ и\ м\ е\ р~~9\/}. С учетом формул (16) и (17) в условиях
примера~2, когда $X_3 =0$,
 $X_4=0$ одновременно измеряются $\delta \varphi$ и $\delta
 \dot\varphi$, в фильтрационных уравнениях (27) и (28) или (29) и (30)
 следует взять коэффициенты статистической линеаризации $a_{00}$ и
$a_{01}$ в виде:
\begin{align*}
 a_{00} &= a_{00} \left( m^x, K^x, t\right) ={}\\
 &\hspace*{50pt}{}=\begin{bmatrix}
 m_2\\
-\mu_1 m_2 +\mu_3 \left(m_2^3 + 3 m_2 K_{22}\right)\\
\end{bmatrix}\,;\\
 a_{01} &= a_{01} \left( m^x, K^x, t\right)=\fr{\partial a_{00}}{ \partial m^x} ={}\\
&\hspace*{50pt}{}=\begin{bmatrix}
 0&1\\
 0&-\mu_1 + 3\mu_3 m_2^2 \left(m_2^2+K_{22}^2\right)\\
\end{bmatrix}\,. %\label{e65}
\end{align*}

 \textit{ П\ р\ и\ м\ е\ р~~10\/}. В условиях примера~9, если $X_3$ и
 $X_4$ удовлетворяют уравнениям (2), то коэффициенты $a_{00}$ и
 $a_{01}$ определяются согласно (5), (6), (16) и (17). В~[16]
 приведен ряд примеров, которые можно использовать в качестве тестирующих
 для автокоррелированных шумов в уравнениях состояния и наблюдения.

\section{Заключение}  %10

Разработанные квазилинейные методы построения информационных
моделей флуктуаций неравномерности вращения Земли по априорным
и апостериорным данным реализованы в виде экспериментального
программного обеспечения в среде \mbox{MATLAB}. Проведено
тестирование программного обеспечения на примерах 1--10.

Квазилинейные методы, как показали вычислительные эксперименты и
сравнение с результатами статистического моделирования, обеспечивают
высокую точность фильтрации скорости $\delta \dot\varphi$ (порядка
2\%--3\% для априорной стационарной информации и 0,5\%--1\% при
апостериорной информации). Из-за отсутствия возвращающей силы по
$\delta \varphi$, как видно из уравнения (1), появляются дрейфы и
накапливающиеся ошибки. Поэтому так необходимы точные измерения
$\delta \varphi$.

Методы, алгоритмы, программное обеспечение и тестовые примеры
включены в состав информационных ресурсов по фундаментальной
проблеме РАН <<Статистическая динамика вращения Земли>>.

Среди направлений дальнейших исследований следует выделить
следующие:
\begin{enumerate}[(1)]
\item  учет влияния автокоррелированности различных возмущений в
уравнениях (1);
\item %2)
оценка негауссовости распределений возмущений для оценок больших
уклонений по $\delta \varphi$;
\item %3)
оценивание потенциальной точности. Экстраполяция и интерполяция
апостериорных данных измерений;
\item %4)
оценивание и распознавание возмущений в уравнениях (1) на основе
апостериорной информации от нескольких нелинейных измерительных
систем различной точности;
\item %5)
разработка комплексных статистических моделей вращения Земли,
учитывающих флуктуации полюса и неравномерности вращения Земли,
обобщающие результаты~[1--10, 16, 17].
\end{enumerate}

\bigskip
Автор благодарен Н.\,Н.~Семендяеву за выполненные вычислительные
эксперименты.

{\small\frenchspacing
{%\baselineskip=10.8pt
\addcontentsline{toc}{section}{Литература}
\begin{thebibliography}{99}

\bibitem{1}
\Au{Марков~Ю.\,Г., Синицын~И.\,Н.}
Стохастическая модель движения полюса деформируемой Земли //
ДАН, 2002. Т.~385. №\,2. С.~189--192.

\bibitem{2}
\Au{Марков~Ю.\,Г., Синицын~И.\,Н.}
Флуктуационно-дис\-си\-па\-тивная модель движения полюса деформируемой Земли //
ДАН, 2002. Т.~387. №\,4. С.~482--486.

\bibitem{3}
\Au{Марков~Ю.\,Г., Синицын~И.\,Н.}
Нелинейные стохастические корреляционные модели движения полюса деформируемой
Земли // Астрон. журн., 2003. Т.~80. №\,2. С.~186--192.

\bibitem{4}
\Au{Марков~Ю.\,Г., Синицын~И.\,Н.}
Влияние параметрических флуктуационно-диссипативных сил на движение
полюса Земли // ДАН, 2003. Т.~390. №\,3. С.~343--346.

\bibitem{5}
\Au{Марков~Ю.\,Г., Синицын~И.\,Н.}
Многомерные распределения флуктуаций полюса Земли //
ДАН, 2003. Т.~391. №\,2. С.~194--198.

\bibitem{6}
\Au{Марков~Ю.\,Г., Синицын~И.\,Н.}
Спектрально-кор\-ре\-ля\-ци\-он\-ные модели флуктуаций вращательного движения Земли //
ДАН, 2003. Т.~393. №\,5. С.~618--623.

\bibitem{7}
\Au{Марков~Ю.\,Г., Синицын~И.\,Н.}
Спектрально-кор\-ре\-ля\-ци\-он\-ные и кинетические модели движения Земли //
Астрон. журн., 2004. Т.~81. №\,2. С.~184--192.

\bibitem{8}
\Au{Марков~Ю.\,Г., Синицын~И.\,Н.}
Стохастические корреляционные модели движения полюса деформируемой Земли //
Космические исследования, 2004. Т.~42. №\,1. С.~76--82.

\bibitem{9}
\Au{Синицын~И.\,Н.}
Стохастические модели флуктуаций движения Земли в условиях пуассоновских
возмущений~// Системы и средства информатики.
Спец. вып. Геоинформационные технологии.~--- М.: ИПИ РАН, 2004. С.~39--55.

\bibitem{10}
\Au{Марков Ю.\,Г., Дасаев~Р.\,Р., Перепелкин~В.\,В., Синицын~И.\,Н.,
Синицын~В.\,И.}
Стохастические модели вращения Земли с учетом влияния Луны и планет~//
Космические исследования, 2005. Т.~43. №\,1. С.~54--66.

\bibitem{11}
\Au{Акуленко~Л.\,Д., Марков~Ю.\,Г., Перепелкин~В.\,В.}
Неравномерности вращения Земли //
ДАН, 2007. Т.~417. №\,4. С.~483--488.

\bibitem{12}
\Au{Марков~Ю.\,Г., Синицын~И.\,Н.}
Корреляционная модель приливной неравномерности вращения Земли~//
ДАН  (в печати).

\bibitem{13}
\Au{Пугачев В.\,С., Синицын~И.\,Н.}
Теория стохастических систем. 2-е изд.~--- М.: Логос, 2004.

\bibitem{14}
\Au{Одуан~К., Гино~Б.}
Измерение времени. Основы GPS.~--- М.: Техносфера, 2002.

\bibitem{15}
\Au{Синицын~И.\,Н.}
Развитие теории фильтров Пугачева для оперативной обработки информации
в стохастических системах //
Информатика и её применения, 2007. Т.~1. Вып.~1. С.~3--13.

\bibitem{16}
\Au{Синицын~И.\,Н.}
Фильтры Калмана и Пугачева. 2-е. изд.~--- М.: Логос, 2007.

\bibitem{17}
\Au{Синицын~И.\,Н.}
Корреляционные методы построения аналитических информационных моделей
флуктуаций полюса Земли по априорным данным //
Информатика и её применения, 2007. Т.\,1. Вып.~2. С.~2--14.

\label{end\stat}
\end{thebibliography}
 }
}
\end{multicols}