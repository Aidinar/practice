\def\stat{zatsman}

\def\tit{ПРЕДПОСЫЛКИ И ФАКТОРЫ КОНВЕРГЕНЦИИ ИНФОРМАЦИОННОЙ И~КОМПЬЮТЕРНОЙ НАУК$^*$}
\def\titkol{Предпосылки и факторы конвергенции информационной и~компьютерной наук}
\def\autkol{И.\,М.~Зацман, О.\,С.~Кожунова}
\def\aut{И.\,М.~Зацман$^1$, О.\,С.~Кожунова$^2$}

\titel{\tit}{\aut}{\autkol}{\titkol}

{\renewcommand{\thefootnote}{\fnsymbol{footnote}}\footnotetext[1]{Работа выполнена при
частичной поддержке РФФИ, грант №\,06-07-07001ано.}

\renewcommand{\thefootnote}{\arabic{footnote}}}
\footnotetext[1]{Институт проблем информатики Российской академии наук, im@a170.ipi.ac.ru}
\footnotetext[2]{Институт проблем информатики Российской академии наук, okozhunova@ipiran.ru}

\Abst{Аналитический обзор посвящен проблеме конвергенции информационной науки
(information science) и компьютерной науки (computer science), а также взаимосвязям этих наук
с информационно-ком\-му\-ни\-ка\-ци\-он\-ны\-ми технологиями (ИКТ). Интерес к проблеме
конвергенции возник более сорока лет назад, и одна из формулировок этой проблемы~--- The
computer and information sciences: A new basic discipline~--- стала названием работы С.~Горна,
опубликованной в 1963~г.~\cite{1za}. В наше время актуальность проблемы конвергенции
существенно возросла, о чем свидетельствуют рассматриваемые в обзоре отдельные
приоритетные направления исследований и разработок по ИКТ 7-й Рамочной программы
Европейского Союза, принятой на период 2007--2013~гг.~\cite{2za}. Приоритетные
направления ИКТ позиционируются в обзоре как внешние факторы конвергенции. Кроме
внешних факторов рассматриваются исторические предпосылки конвергенции
информационной и компьютерной наук. Факторы и предпосылки конвергенции
рассматриваются в контексте разработки научных основ создания новых поколений ИКТ.}

\KW{информационная наука; компьютерная наука; информационно-компьютерная нау\-ка;
информационно-коммуникационные технологии; взаимосвязи направлений 
информационно-ком\-му\-ни\-ка\-ционных технологий и компьютерной науки.}

      \vskip 18pt plus 9pt minus 6pt  %24pt

      \thispagestyle{headings}

      \begin{multicols}{2}


      \label{st\stat}

\section{Введение}

      В соответствии с концепцией А. Соломоника научная парадигма любой <<зрелой>>
науки состоит из следующих четырех составляющих, которые могут разрабатываться отдельно,
но объединяются в единую и цельную конструкцию: философские основания, аксиоматика,
классификация объектов предметной области исследования и система терминов~\cite{3za}.
При этом сам термин <<научная парадигма>> трактуется А.~Соломоником в соответствии с
теорией Т.~Куна~\cite{4za}.

      В рамках концепции Соломоника суть проблемы конвергенции можно выразить
следующей фразой: описание парадигмы единой области знаний, охватывающей предметные
области компьютерной и информационной наук, с общими философскими основаниями,
включая позиционирование этой области знаний в системе современного научного
мировоззрения, с общей аксиоматикой, единой классификацией объектов, процессов и явлений
этой области знаний, общей и конвенциональной системой терминов.

      Что касается последней составляющей научной парадигмы (системы терминов), то здесь
необходимо учитывать объективно существующие трудности перевода с одного естественного
языка на другой. Например, в английском языке существует как минимум четыре устойчивых
словосочетания~--- information science, computer science, computer and information science,
computational science~--- которые нередко переводятся на русский язык одним словом
<<информатика>>. Приведенные англоязычные словосочетания обозначают разные научные
дисциплины и их приложения. Однако при переводе на русский язык единым словом
<<информатика>> содержательные отличия этих наук, различие их предметных областей и
приложений не отражаются лексически.

      Кроме того, в английском и русском языках разные по смыслу базовые понятия
информационной науки (information science), компьютерной науки (computer science),
информационно-компью\-тер\-ной науки (computer and information science) и вычислительной
науки (computational science) также часто лексически выражены одним словом, что затрудняет
сравнительное описание предметных областей и приложений этих дисциплин. Например, слово
``information''~--- <<информация>>~--- при его использовании в перечисленных предметных областях
может трактоваться по-разному. 

Различия в трактовках слова <<информация>>, 
суще\-ст\-вен\-ные для этого обзора, 
рассматриваются \mbox{далее}.

      Для описания парадигмы единой области знаний, охватывающей предметные об\-ласти
компьютерной и информационной наук, необходимо явно определить значения используемых
терминов. Ключевой характер явного определения системы терминов и смысла базовых
понятий в любой об\-ласти знаний отметил В.\,Ф.~Турчин в своей книге <<Феномен науки>>. Он
писал: <<Собственно говоря, ввести основные понятия~--- это и значит уже определить данную
науку, ибо остается только добавить: описание мира с помощью этой вот системы понятий и
есть данная конкретная наука>>~[5, с.~17].

      Применяя это положение к предметной об\-ласти обзора, можно сказать, что явно
зафиксировать базовые понятия и определить систему терминов для описания парадигмы
единой области знаний, охватывающей предметные области компьютерной и информационной
наук,~--- это и значит уже во многом определить ее научную парадигму. Это и является причиной
того, что в обзоре акцентируется внимание на терминологии.

      Здесь возникает закономерный вопрос: а являются ли в настоящее время сами
компьютерная и информационная науки по отдельности <<зрелыми>> науками? Этот вопрос
выходит за пределы настоящего обзора и заслуживает самостоятельного изучения. Однако
отметим, что в 1980~г.\ один из классиков информационной науки писал: <<Едва ли
теоретическая информационная наука уже существует. Я различаю рассеянные обрывки
теории, некоторые из которых имеют более или менее стройный вид, но все же они не
складываются в единую область знаний. Поэтому общих допущений (явных или неявных),
которые могли бы претендовать на теоретические основания, попросту не существует.
Информационная наука рас\-смат\-ри\-ва\-ет большое число приложений, которые все больше
требуют участия компьютера. Ни о каких основаниях она не может заявлять, если опирается
только на здравый смысл в области языковых исследований, коммуникаций, отношений знаний
и информации, на приложения компьютерных и телекоммуникационных технологий. Однако
состояние компьютерной науки не намного лучше. В~философском отношении
информационная наука пребывает в подвешенном состоянии, так как у нее нет теоретических
оснований>>~[6, с.~125].

      Что касается проблемы конвергенции компьютерной и информационной наук, то в
обзоре предпринята попытка показать, с одной стороны, что научная парадигма единой области
знаний еще не сформировалась, с другой стороны, что по отдельным позициям уже обозначился
ряд предпосылок и факторов, стимулирующих процессы конвергенции. Здесь важно
подчеркнуть, что кроме внут\-рен\-них предпосылок существует важный внешний фактор,
существенно влияющий на процессы конвергенции~--- достаточно четкие и явно
эксплицированные потребности в разработке научных основ создания новых поколений
ИКТ.

      Одним из примеров явно обозначенных потребностей может служить 7-я Рамочная
программа Европейского Союза, принятая на период 2007--2013~гг. В документах этой
программы сформулировано восемь приоритетных направлений исследований и разработок,
включая в качестве отдельного направления <<Перспективные ИКТ>>~\cite{2za, 7za}.

      Цели проектов, финансируемых в рамках приоритетного направления <<Перспективные
ИКТ>>, в программе на 2007--2008~гг.\ сформулированы следующим образом:
<<Своевременная идентификация и обоснование новых тематических направлений
исследований и разработок, которые имеют большой научно-технический потенциал и могут
стать основой для разработки ИКТ следующих поколений\footnote{В цитируемом документе
подчеркивается, что для разработки ИКТ следующих поколений особенно важны инновационные
исследования и разработки, а не поэтапная модернизация существующих и используемых сегодня
теоретических, прикладных, методологических и технологических принципов, подходов и понятий в
сфере ИКТ.}. Эти проекты должны включать меж\-дис\-цип\-ли\-нар\-ные исследования новых и
альтернативных подходов к разработке ИКТ будущего и быть нацеленными на
фундаментальное переосмысление системы теоретических, прикладных, методологических и
технологических принципов, подходов и понятий, используемых сегодня в сфере
ИКТ>>~[8, с.~54].

      Важно отметить, что в цитируемом документе можно найти идентификацию и
обоснование целого ряда конкурсных тем в рамках приоритетного направления
<<Перспективные ИКТ>>, включая тему <<ИКТ долговременного 
применения>>~[8, с.~57--63].

      Сфера применения результатов исследований и раз\-ра\-бо\-ток по конкурсной теме <<ИКТ
дол\-го\-временно\-го применения>>, ее цели и задачи сфор\-мули\-ро\-ва\-ны следующим образом:
<<Широкое распро\-стра\-не\-ние и применение ИКТ, ин\-фор\-ма\-ци\-он\-но-ком\-пью\-тер\-ных и других
цифровых систем в со\-ци\-аль\-но-зна\-чи\-мых сферах деятельности кардинально увеличивает
требования к их надежности, безопасности и долговечности. Это требует новых решений для
того, чтобы обеспечить доверие пользователей в процессе их использования, защитить от
несанкционированного к ним доступа и сохранить их функциональность в течение долгого
периода времени в условиях крайне децентрализованной и быстрой смены поколений ИКТ,
ин\-фор\-ма\-ци\-он\-но-ком\-пью\-тер\-ных и других цифровых систем>>. Далее формулируются
актуальные исследовательские проблемы в рамках темы <<ИКТ долговременного
применения>>. Приведем формулировки только двух проблем~[8, с.~62--63]:
      \begin{enumerate}[(1)]
      \item \textit{Разработать теоретические и прикладные основы создания долговечных
систем}, обеспечивающих их эволюцию при минимизации затрат на их развитие в условиях
многократной смены поколений программно-аппаратных и сетевых средств и/или форматов
данных. Другими словами, долговечные системы должны быть способны к сохранению своей
первоначальной социально-значимой функциональности в течение долгого периода времени и
изменять ее в случае необходимости. Методы сохранения и изменения функциональных
возможностей должны быть машинно-независимыми и должны обеспечивать устойчивую
эволюцию долговечных систем.
      \item \textit{Разработать новые подходы к представлению и сохранению знаний},
ориентированные на долговременный и безотказный к ним доступ в условиях \textit{локальной
генерации отдельных <<квантов>> знаний}, их интеграции, а также глобального
использования систем представления и сохранения знаний с учетом контекста и временной
эволюции систем. При этом должна быть обеспечена долговременная устойчивость систем
представления и сохранения знаний в условиях многообразия их использования и
\textit{эволюции семантики во времени}.
      \end{enumerate}

      Таким образом, в документах 7-й Рамочной программы Европейского Союза четко
обозначена потребность в разработке научных основ создания ИКТ следующих поколений.
Потребность в конкретных теоретических основаниях иногда формулируется в косвенной
форме и нередко является импликатурой (следствием) приведенных формулировок актуальных
исследовательских проблем.

      Например, фраза <<локальная генерация отдельных <<квантов>> знаний\ldots>> во
второй проблеме подра\-зу\-ме\-ва\-ет возможность членения знаний на <<кванты>> некоторым
способом, их представления в 
 циф\-ро\-вой среде\footnote{Цифровая среда~--- сочетание элементов
цифровой вычислительной техники, средств телекоммуникации, информационно-компьютерных систем,
иных цифровых средств ввода, хранения, поиска, передачи и других процессов обработки данных.} и
обеспечение доступа к сохраненным представлениям <<квантов>> знаний и отношениям между
ними. Эти вопросы рассматриваются далее в обзоре, так как они являются ключевыми для
определения предметной области, относящейся одновременно к компьютерной и
информационной наукам. Здесь отметим только три вопроса, на которые необходимо ответить
при описании любого варианта парадигмы единой области знаний, охватывающей предметные
области компьютерной и информационной наук, если ставится цель предложить вариант
парадигмы как теоретическую основу создания новых поколений ИКТ.

      Во-первых, допускает ли предлагаемый вариант парадигмы возможность членения
системы знаний на <<кванты>>, и если допускает, то является ли положение о возможности
членения аксиомой или следствием других аксиом? 

Во-вторых, допускается только
единственный способ членения системы знаний или предполагается существование множества
разных способов? 

В-третьих, учитывается ли эволюция системы знаний человека во времени, и
если учитывается, то как этапы (стадии) эволюции отражаются в способе (способах) членения,
или аксиоматически предполагается, что в предлагаемом варианте парадигмы знания человека
в разные моменты времени являются самотождественными?

      Отметим, что фраза <<эволюция семантики во времени>> в формулировке второй
проблемы имеет непосредственное отношение к третьему вопросу. Все три перечисленных
вопроса являются ключевыми (но далеко не единственными). На них необходимо дать ответы
при описании любого варианта парадигмы, если она предлагается в качестве теоретической
основы создания новых поколений ИКТ в трактовке новизны ИКТ в рамках 7-й Рамочной
программы Европейского Союза.

      Ответы на поставленные вопросы затрагивают все четыре составляющих научной
парадигмы. Например, к философским основаниям парадигмы относится вопрос: <<Являются
ли знания человека в разные моменты времени самотождественными?>>~\cite{9za}. Вопрос
о возможности членения знаний на <<кванты>> скорее всего относится к аксиоматике этой
области знаний. В классификации объектов, процессов и явлений единой области знаний
необходимо описать отношения между процессами генерации отдельных <<квантов>> знаний,
их интеграции и использования. В системе терминов необходимо дать определение <<кванта>>
знаний и назвать этот <<квант>>, например, концептом, понятием, значением или другим
словом, которое и использовать далее как термин только в этом значении в рамках
предлагаемого варианта парадигмы. Отметим, что в обзоре <<квант>> системы знаний
человека, понятие и концепт рассматриваются как синонимы.

      Далее будет показано, что сформулированные проблемы рассматриваются и в
компьютерной, и в информационной науках. Таким образом, в настоящее время существуют
внешние факторы конвергенции в виде приоритетных направлений ИКТ и актуальных
теоретических проблем, воз\-ни\-ка\-ющих в процессе создания новых поколений ИКТ и
относящихся к предметным областям обеих наук. Как видно из приведенных примеров, иногда
в самих формулировках проблем содержатся явно эксплицированные или косвенно
сформулированные потреб\-но\-сти в конкретных теоретических основаниях. Кроме внешних
факторов в обзоре далее
 рассматриваются исторические предпосылки конвергенции
информационной и компьютерной наук.

      Прежде чем завершить введение к обзору, остановимся на еще одном документе,
появление которого привлекло внимание лиц, принимающих решения в сфере 
научно-технической политики, к необходимости создания новых поколений ИКТ в интересах
обеспечения конкурентоспособности национальной экономики, в том числе к разработке
теоретических основ создания новых поколений ИКТ как ключевой составляющей общества,
основанного на знаниях (knowledge-based society). Речь идет об аналитическом отчете по
вопросам обеспечения конкурентоспособности США в XXI~в., подготовленным
Консультативным комитетом по информационным технологиям при Президенте
США~\cite{10za}. Информацию об этом отчете можно найти в работе~\cite{11za}.

      В аналитическом отчете информационные техно\-ло\-гии позиционируются как важная
со\-став\-ля\-ющая триады <<научная тео\-рия\,--\,на\-уч\-ный экс\-пе\-ри\-мент\,--\,ин\-фор\-ма\-ци\-он\-ные
технологии, обеспечивающие проведение эксперимента>>, являющаяся основой процессов
научного познания практически во всех областях знаний. Чтобы выразить в явной форме сферу
применения, роль и функции информационных технологий в триаде научного познания, авторы
отчета определяют междисциплинарную область исследований и разработок, которую
называют ``computational science'', что в этом обзоре переводится буквально как
<<вычислительная наука>>.

      Определение этой области исследований и разработок, предлагаемое в отчете, имеет
следующий вид~[10, с.~10]: <<Вычислительная наука~--- это быст\-ро растущая
мультидисциплинарная предметная область, в которой используются возможности передового
компьютинга (advanced computing) для понимания и решения сложных проблем.
Вы\-чис\-ли\-тель\-ная наука интегрирует три компонента:
      \begin{enumerate}[(1)]
\item \textit{алгоритмы (численные и нечисленные), про\-грам\-мное обеспечение} для
моделирования и имити\-ро\-ва\-ния, разработанные для решения проблем естественных,
гуманитарных и инженерных \mbox{наук};
\item \textit{информационно-компьютерная наука}, которая разрабатывает и оптимизирует
современные аппаратные, программные и сетевые средства, а также компоненты
управления данными, которые необходимы для решения вычислительно сложных
проблем;
\item \textit{вычислительная инфраструктура}, которая поддерживает решение научных и
инженерных проблем, а также развитие информационно-компью\-тер\-ной
науки>>\footnote{Computational science is a rapidly growing multidisciplinary field that uses advanced
computing capabilities to understand and solve complex problems. Computational science fuses three distinct
elements:
\begin{enumerate}[(1)]
\item Algorithms (numerical and non-numerical) and modeling and simulation software developed to solve
science (e.g., biological, physical, and social), engineering, and humanities problems;
\item Computer and information science that develops and optimizes the advanced system hardware,
software, networking, and data management components needed to solve computationally demanding
problems;
\item The computing infrastructure that supports both the science and engineering problem solving and the
developmental computer and information science~[10, с.~10].
\end{enumerate}}.
\end{enumerate}

В этом определении используется словосочетание \textit{ин\-фор\-ма\-ци\-он\-но-ком\-пью\-тер\-ная
наука}. Это словосочетание одним из первых использовал американский ученый С.~Горн в
1963~г., с той лишь
 разни\-цей, что тогда оно употреблялось им во множественном числе
(computer and information sciences)~\cite{1za}. С 1983~г.\ Горн начал использовать этот
термин в единственном числе~\cite{12za}.

      Таким образом, идея конвергенции была отражена С.~Горном в самом названии научной
дисциплины~--- информационно-компьютерная наука (computer and information science)~---
которая в аналитическом отчете по вопросам обеспечения конкурентоспособности США в
XXI~в.\ позиционируется как одна из трех составляющих вычислительной науки. Отметим, что
трактовки информационно-компьютерной науки в этом отчете и в работах Горна отличаются.
Описание и анализ трактовки Горна является одной из задач обзора.

      Структура предлагаемого обзора имеет сле\-ду\-ющий вид. Разделы~2 и~3 посвящены
отдельным вопросам становления информационной и компьютерной наук соответственно, а
также формированию предпосылок их конвергенции. Краткий обзор работ Горна и Шрейдера
включен в разд.~4, за которым следует заключение.

\section{Информационная наука}

\subsection{Становление информационной науки} %2.1.

      По мнению скандинавского ученого, специалиста в области информационной науки
Петера Ингверсена, изучаемая им наука~--- дисциплина молодая. В своей работе ``Information
and information science'' он подчеркивает, что самое раннее использование термина
<<информационная\linebreak
 наука>> в научных кругах пришлось на 1958~г., когда был сформирован
Institute of Information Scientists (IIS)
в Великобритании. По планам его основателя, Джейсона
Фаррадейна, предполагалось, что <<использование термина <<информационный ученый>>
поможет различать ученых,\linebreak 
занимающихся информационной наукой, и 
уче\-ных-есте\-ст\-во\-ис\-пы\-та\-те\-лей, %\linebreak
 поскольку сотрудники института имели дело в основном со сбором,
хранением и обработкой научно-технической информации>>~[13, с.~137].

      Сотрудники вышеупомянутого института специализировались в разных областях
знаний, за\-час\-тую очень сильно отличающихся друг от друга. В~круг их основных обязанностей
\mbox{входили} организация информационного обслуживания и предо\-став\-ле\-ние научной информации
исследователям из других институтов и промышленных лабораторий. Пионерами
информационной науки были Б.\,К.~Брукс, С.~Клевердон, Р.~Фейтхорн, Е.~Гарфильд,
М.~Кочен, И.~Ланкастер, Дж.~Солтон, Д.~де~Солла Прайс и 
Б.~Викери~[13, с.~137].

      В своей работе Ингверсен подробно объяс\-няет, почему сотрудники IIS
называли себя информационными учеными: <<Называя себя
информационными учеными, они, очевидно, хотели подчеркнуть важность научного подхода к
изучению информации и процессов научных коммуникаций. Их работа являлась продолжением
предыдущих теоретических и эмпирических попыток исследовать проблемы организации,
роста и распространения информации, которая была накоплена перед второй мировой
войной>>~[13, с.~137].

      Чтобы проследить процесс становления информационной науки, Ингверсен обращается
к ее истории и к предпосылкам, которые привели к формированию предметной области
информационной науки. По его мнению, традиционно профессионалы, занимающиеся
хранением документальных и издательских форм представления научных результатов, были
известны как документалисты (позже~--- информационные ученые) и библиотекари: <<Первые
из упомянутых обычным образом обуча\-лись какой-либо научной дисциплине и занимались
прикладными аспектами передачи на\-уч\-но-тех\-ни\-че\-ской информации применительно к своей
дисциплине. Библиотекари обучались библиотечному делу (т.\,е.\ работе в библиотеках). Для
них передача информации в таких учреждениях, как библиотеки, часто символизирует и
социальные, и значимые культурные аспекты. Несмотря на идентичные способы обработки
документов и информации и похожее использование информационных технологий, разделение
на две группы продолжало существовать и в послевоенное время во многих странах, например
в Скандинавии и Соединенных Штатах. Результат этого разделения можно увидеть в другом
названии этой области знания: библиотечно-информационная наука. В дополнение к этому, для
библиотечного сообщества сама по себе библиотечная наука иногда рассматривалась в качестве
научной альтернативы информационной науке. Однако социокультурные коммуникации с
помощью библиотек не могут существовать без процессов передачи
информации>>~[13, с.~138].

      Поэтому Петер Ингверсен отдельно оговаривает, что <<библиотечная наука~--- это
информационная наука и исследовательские методы, примененные к конкретному учреждению
под названием <<библиотека>>. Разделение на библиотечную и информационную науки
является непродуктивным и носит искусственный характер. Наука не может быть целиком
посвящена некоторому учреждению; например, медицина не может быть представлена на
концептуальном уровне как больничная наука>>~[13, с.~138].

      С одной стороны, такое разделение внесло свой вклад в кризис идентификации этой
науки и фрагментировало стройное здание ее теории, построение которого потребовало
нескольких десятилетий. С другой стороны, настаивая на большем числе социальных и
гуманитарных аспектов, ассо\-ци\-иро\-ван\-ных с передачей информации, библиотечное сообщество
в 90-е~гг.\ прошлого века способствовало эволюции в направлении консенсуса в
информационной науке. В конечном счете, фокусирование только лишь на научных
коммуникациях и передаче информации является слишком ограниченной основой для научной
дисциплины. Влияние сообщества информационных ученых расширило концепцию
предметной области за счет информационных процессов в тех сферах человеческой
деятельности, в которых \textit{знания и информация играли жизненно важную роль}, таких как
коммерческая деятельность и социокультурные коммуникации>>~[13, с.~138; 14, 15].

      В своем исследовании информационной науки и ее предмета Ингверсен обращается к
истокам и основаниям этой науки. Он задается вопросами: что же было движущей силой
попыток основания науки, занимающейся, главным образом, обработкой документов, и почему
было так интересно и даже необходимо серьезно изучать вопрос обеспечения эффективной
передачи желаемой и доступной информации от человека-генератора человеку-пользователю?
Один из возможных ответов заключается в том, что информационная наука как область знания
возникла вследствие осознания проблем как физического, так и интеллектуального доступа к
чрезвычайно быстро растущему объему научных знаний (послевоенный информационный
взрыв)~[13, с.~138].

      Ингверсен полагает, что подобного рода ответы получили широкое распространение, но
при этом они являются частично поверхностными. В приведенном объяснении предполагается,
что информацию можно приравнивать к документам, таким как публикации и другие
физические сущности, содержащие какие-либо сообщения. Однако информация в
информационной науке не обозначает физические сущности вроде документов. Если что и
произошло в течение XX~столетия, так, в первую очередь, это впечатляющий
<<документальный взрыв>> как в науке, так и в обществе, что повлекло за собой усложнение
доступа~--- и физического к документам, и интеллектуального~--- для получения адекватной
информации>>~[13, с.~139].

      Главный вывод, который делает Ингверсен о том, каким должно быть понятие
информации в информационной науке, заключается в сле\-ду\-ющем. В пределах предметной
области информационной науки понятие информации должно удов\-ле\-тво\-рять двум
требованиям. С одной стороны, информация является результатом преобразования в
\textit{знаковую форму когнитивных структур че\-ло\-ве\-ка-ге\-не\-ра\-то\-ра}. При этом учитывается
модель представления знаний, имеющихся у получателя этих знаковых форм. С~другой
стороны, это нечто такое, при восприятии и осознании чего имеющиеся знания получателя
информации \textit{подвергаются влиянию и транс\-фор\-ми\-ру\-ют\-ся}. В~результате,
словосочетание <<ин\-фор\-ма\-ци\-он\-ное общество>> также подразумевает, что общество зависит от
того, как оно использует информацию, а не только от того, как оно ее
производит~[13, с.~139].

      Здесь необходимо отметить, что в обзоре, в целях различения и сопоставления точек
зрения разных ученых с использованием лексически отличающихся выражений разных
концептов, информацию как результат преобразования в знаковую форму когнитивных
структур человека будем называть \textit{знаковой информацией}.

      Далее Ингверсен пишет: << \ldots по существу, никому не требуется наука для
обеспечения доступа к документам. Если что-то и необходимо, то это улучшенные методы,
позволяющие людям поспевать за ростом документов. Эта практическая работа уже почти пять
тысяч лет выполняется архивариусами (хранителями архивов), библиотекарями и
документалистами. Они постоянно извлекали пользу из информационных технологий,
доступных им в каждый исторический период времени, начиная с глиняных табличек и
заканчивая пергаментом, бумагой и компьютерными методами>>~[13, с.~139].
Отметим, что первое использование компьютерных технологий для поиска научных
документов относится еще к 1960-м гг.\ прошлого века.

      Главной движущей силой использования технологических инноваций была потребность
в быст\-ром получении документов, релевантных ка\-кой-ли\-бо цели или проблеме. Неудивительно,
что
 мето\-ды решения проблем доступа к документам и, что более существенно, к потенциально
значимой информации \textit{все больше определялись используемыми
технологиями}~[13, с.~139]. Здесь важно отметить идею зависимости методов
решения проблем доступа к инфор\-ма\-ции от степени развития используемых технологий. Эта
идея, имеющая прямое отношение к проблеме конвергенции, более подробно будет
рассмотрена далее.

      С момента создания в 1958~г.\ IIS неоднократно
предпринимались попытки установить основные направления исследований в информационной
науке и определить ее границы с другими областями знания. Основная проблема заключалась в
недостатке \textit{базовых философских подходов к описанию информационных процессов},
кроме подразумеваемых рационалистических взглядов, унаследованных от физических наук.
Причиной этому была неопределенность ее положения в системе научного познания\footnote{С
точки зрения концепции А.~Соломоника здесь речь идет о первой составляющей научной парадигмы
информационной науки~--- философских основаниях и позиционировании этой области знаний в
системе современного научного мировоззрения.}. Ряд специалистов настаивали на том, что
информационную науку необходимо рассматривать как составляющую естествознания.
Поэтому они стремились сформулировать и действительно сформулировали фундаментальные
<<законы>> информации, которые вследствие особенностей человеческого подхода к
использованию информации для познавательных целей можно рассматривать лишь как
индикаторы наличия информационных процессов. Однако важно отметить и эти попытки
<<форсированной научной эволюции>> (основанной на желании совершенствоваться),
поскольку без них данная область знания была бы поглощена близкими когнитивными
дисциплинами еще в 1960-х гг.~[13, с.~141].

      В своем исследовании информационной науки Ингверсен, говоря об истории развития
этой области, резюмирует, что за период ее существования предпринимались попытки слияния
с другими областями знаний, с тем чтобы утвердить более весомую научную позицию в
системе научного познания. Прослеживаются два основных направления: (1)~движение в
направлении теории коммуникаций и (2)~попытка слияния с компьютерной наукой. В то же
время ряд ученых приложили значительные усилия для сохранения независимости
информационной науки с ее собственной инди\-ви\-ду\-аль\-ностью~[13, с.~141;
16--18].

\subsection{Фаррадейн: предметная область информационной науки} %2.2.

      Одним из главных сторонников формирования и развития информационной науки как
независимой области знаний являлся Дж. Фаррадейн, основатель IIS.

      Предметную область этой науки он описал следующим образом: <<информационная
наука по большей части когнитивная наука, т.\,е.\ имеет дело с мыс\-ли\-тель\-ны\-ми процессами,
одной из самых сложных областей исследований. Конечно, она является частью более
обширной области коммуникаций, преподавания и обучения. Но даже такие практические
аспекты этой науки, как хранение и поиск информации, постоянно остаются в тени понятия
релевантности, ментальной оценки, индивидуальной для каждого отдельного получателя
информации и зависящей от его первоначального уровня знаний>>~[16, с.~75].

      Такой подход к вопросу определения информационной науки приводит Фаррадейна к
следующему выводу: <<Чем больше мы изучаем то, что принадлежит когнитивным границам в
информационной науке, т.\,е.\ ментальные процессы, которые генерируют информацию, и
когнитивные процессы, происходящие при получении информации, тем больше у нас
возможностей улучшать и контролировать процессы хранения и поиска информации для
получения желаемых результатов>>~[16, с.~75].

      В описании предметной области информационной науки Фаррадейн сопоставляет знания
и информацию, в тот числе ее трансформацию в знания, как центральные понятия и процессы
обсуждаемой науки. Картина событий при передаче информации, которую описывает ученый,
не содержит какого-либо лингвистического анализа, кроме описания ряда ограничений,
накладываемых естественным языком. Зато она содержит описание процессов мышления
человека-генератора информации и получателя, ее преобразования в знания. Фаррадейн
определяет <<знания>> как отпечаток процессов понимания и осознания, происходящих в
памяти человека, как нечто, доступное лишь в пределах памяти человека. При этом он
отмечает, что сами процессы понимания и осознания в настоящее время остаются
невыясненными.

      Информация определяется им как сущность, заменяющая знания, например язык, и
используемая для коммуникаций. Важно отметить, что определение информации по
Фаррадейну, которую будем называть <<языковой информацией>>, во многом совпадает со
<<знаковой информацией>> по Ингверсену, но принципиально отличается от <<ментальной
информации>> Брукса, которая будет рассмотрена далее.

      О свойствах информации и ее связях со знаниями Фаррадейн говорит следующее:
<<\ldots она [информация] нейтральна в том смысле, что она не обязательно должна быть новой
для вос\-при\-ни\-ма\-юще\-го ее субъекта. Нам известно много практических примеров обработки
информации в системах поиска и хранения, но ее отношение к знаниям не менее важно для
развития информационной науки как науки об информации>>~[16, с.~77].

      Ученый придает большое значение моделированию поисковых процессов и связанных с
ними понятий: <<Поисковые процессы (в частности, в памяти человека), если их можно было
бы обнаружить и моделировать, могли бы предоставить лучшие методики поиска, чем методы
поиска по образцу, которые используются в поисковых системах в настоящее время. Как
правило, человек испытывает более серьезные затруднения при переводе своих потребностей в
новых знаниях в лингвистическую форму запроса, чем при представлении уже имеющихся у
него знаний в форме информации. В~этом нет сомнений, поскольку пробел, породивший
потребность в знаниях, действительно пуст, без каких-либо очевидных связей (отношений) с
существующими у человека когнитивными структурами его знания.

      Таким образом, подразумевается, что структура неполных отношений какой-либо
области знания и порождает информационные потребности пользователя. Если бы
существующие знания проблемной области могли бы быть полностью отображены в системе
ментальных, связанных между собой, понятий, то, возможно, получилось бы выразить эту
<<потребность>> в более точной форме, чем в той, в которой был сформулирован вопрос
пользователя. В комбинации с информацией, хранимой в структурированном виде, мог бы быть
произведен более точный и полный поиск средствами автоматизированной
системы>>~[16, с.~79]. Отметим, что, по мнению Фаррадейна, информационный
поиск относится к предметной области информационной науки.

\subsection{Брукс: основания информационной науки} %2.3.

      Бертрам Брукс (его главная работа в данной области относится к 1981~г.) также был
сотрудником IIS в Великобритании. Он активно \mbox{изучал} вопросы оснований информационной
науки, поскольку в ней к 1980-м~гг.\ накопились нерешенные теоретические вопросы,
касающиеся ее оснований и системы терминов.

      Одним из центральных вопросов при рас\-смот\-ре\-нии философских оснований
информационной науки Брукс считает вопрос соотношения субъективного и объективного
знания в научной деятельности. По его мнению, понятие информации, безусловно, является
главным в системе терминов информационной науки, но оно же предполагает определенные
сложности для ученого-теоретика. Даже на уровне здравого смысла, т.\,е.\ того, как мы ее себе
представляем, информация является сущностью, которая проникает во все сферы человеческой
деятельности. Поэтому особенно сложно отслеживать информационные явления в изоляции,
обособленности, которой обычно требует научный вопрос. Даже процесс описания результатов
наблюдений каких-то явлений сам по себе уже является информационной деятельностью, в
рамках которой не так просто отделить объективное знание от субъективного.

      Все социальные науки сталкиваются с подобной проблемой, но ни одна из них, согласно
Бруксу, не предназначена для ее решения. Информационная наука наиболее близко связана со
взаимодействиями ментальных и физических процессов, субъективных и объективных
способов мышления. Поэтому, с точки зрения Брукса, на информационную науку возлагается
особая ответственность. Она должна, насколько возможно, прояснить эти моменты, что
является одной из основных ее задач.

\subsubsection{Миры Карла Поппера: мир~3 и~информационная наука} %2.3.1.

      В попытке сформулировать основания информационной науки Брукс исследовал
философские труды. Наиболее значимой работой, связанной с проблемами и основаниями
информационной науки, стала для него книга Карла Поппера <<Объективное знание>>. По
мнению Брукса, в чем действительно нуждается информационная наука как в основе, так это в
объективной, а не в субъективной теории знания~[6, с.~127].

      По мере углубления в вопрос соотношения объективного и субъективного знания в
науке Поппер формирует следующие онтологии, так называемые миры Поппера:
      \begin{itemize}
\item мир 1~--- физический мир (мир материальных сущностей);
\item мир 2~--- мир субъективных человеческих знаний;
\item мир 3~--- мир объективных знаний, нашедших выражение в текстах на естественных
языках, искусствах, науках, технологиях.
\end{itemize}

      По мнению Брукса, именно мир~3 Поппера должен быть рекомендован библиотечным и
ин\-формационным ученым, поскольку он впервые предлагает логическое обоснование их
профессиональной деятельности, которое выражается в терминах, отличных от терминов
практических приложений. А практическая деятельность библиотечных и информационных
ученых складывается из собрания и хранения записей мира~3, в то время как теоретической
проблемой является изучение взаимодействий между мирами~2 и~3, их описанием и
объяснением способности или неспособности сис\-те\-ма\-ти\-зи\-ро\-вать знания (скорее их, чем
документы) для более эффективного использования. События мира~2~--- наших
индивидуальных ментальностей~--- происходят в наших индивидуальных частных
пространствах и потому являются субъективными. Для того чтобы объективизировать наши
индивидуальные мысли, нам необходимо каким-то образом выразить их и занести эти записи в
мир 3. Там они становятся доступными другим людям~[6, с.~129--130].

      Но такое разъяснение объективного знания и субъективного предполагает проблемы,
которые, по мнению Брукса, Поппер не учел: <<Не слишком верно полагать, что любое
выражение мысли (или чувства), сохраненное в мире~3, является непосредственно доступным
всякому, кто его ищет, как некая объективность, объективная 
реальность>>~[6, с.~130].

      После этого заключения Брукс переходит к рассмотрению двух вопросов, которые
являются ключевыми при описании теоретических оснований информационной и
компьютерной наук:
      \begin{itemize}
\item допускается ли возможность членения знаний на <<кванты>>;
\item допускается ли эволюция системы знаний человека во времени или знания человека
в разные моменты времени являются самотождественными?
\end{itemize}

\subsubsection{Фундаментальное уравнение информационной науки} %2.3.2.

      Основополагающими понятиями информационной науки Брукс, как и остальные ученые,
занимающиеся информационной наукой, считает информацию и знания. Именно поэтому
вопрос их соотношения для него является центральным при обсуждении оснований этой науки.
Брукс рас\-смат\-ри\-ва\-ет знания как структуру взаимосвязанных понятий, а информацию~--- как
небольшую часть этой структуры, т.\,е.\ ученый относит информацию к ментальной
сфере~[6, с.~131].

      Он выразил это соотношение знаний и информации в виде следующего выражения,
названного им <<фундаментальным уравнением>>:
      $$
      \mathbf{K [S]} + \Delta \mathbf{I} = \mathbf{K [S}+\Delta\mathbf{S]}\,.
      $$
      Это уравнение выражает в абстрактной символьной форме идею, что структура знаний
\textbf{K[S]} под влиянием информации $\Delta \mathbf{I}$ меняется на новую
модифицированную структуру $\mathbf{K[S} + \Delta \mathbf{S]}$, где $\Delta \mathbf{S}$
обозначает эффект модификации структуры знаний.

      Брукс акцентирует внимание на следующем свойстве своего уравнения:
<<Фундаментальное уравнение информационной науки также подчеркивает, что определенная
таким образом информация не является идентичной с философскими данными-ощущениями.
Конечно, информация может зависеть от сенсорно воспринимаемых результатов наблюдений,
но данные-ощущения, полученные таким образом, должны быть субъективно
интерпретированы структурой знаний, чтобы стать информацией. $\langle\ldots\rangle$
Уравнение также имеет своей целью вывести факт, что рост знаний не просто увеличивается.
Поглощение информации структурой знаний может повлечь за собой не прос\-то добавление, а
даже такое уточнение структуры, как изменение системы отношений, связывающих понятия. В
науках прирост информации иногда приводил к серьезной перестройке структуры
знаний>>~[6, с.~131].

      Отсюда следует, что Брукс допускает возможность членения знаний на <<кванты>>,
которые он называет <<информацией>>. Кроме того, он допускает эволюцию знаний во
времени и не считает знания человека в разные моменты времени самотождественными.
Однако у Брукса остается открытым вопрос о том, как этапы эволюции системы знаний связаны
со способом (способами) членения системы знаний.

      Ответ на вопрос о способе (способах) членения системы знаний играет существенную
роль при разработке ИКТ. Покажем это на примере формулировки второй проблемы,
упомянутой в разд.~1 обзора, а именно: <<разработка новых подходов к представлению и
сохранению знаний, ориентированных на долговременный и безотказный к ним доступ в
условиях \textit{локальной генерации отдельных <<квантов>> знаний, их интеграции, а
также глобального использования систем представления и сохранения знаний} с учетом
контекста и временной эволюции систем при условии обеспечения долговременной
устойчивости систем представления и сохранения знаний в ситуации многообразия их
использования и эволюции семантики во времени>>.

      Если парадигма единой области знаний допускает только один способ членения знаний,
например с помощью одного естественного языка, то на этапе интеграции результатов любых
локальных процессов генерации <<квантов>> знаний разработчик ИКТ будет иметь дело, в
терминах лингвистики и семиотики, только с одним планом выражения и одним,
соответствующим ему, планом содержания. Однако такая парадигма может быть использована
при разработке только моноязычных ИКТ.

      Если допускается использование нескольких способов членения знаний, например с
помощью разных естественных языков, то на этапе интеграции результатов любых локальных
процессов генерации <<квантов>> знаний разработчик ИКТ будет иметь дело одновременно с
несколькими планами выражения и содержания. Важно отметить, что разные способы членения
знаний объединяет только сама их возможность устанавливать соответствие между множеством
<<квантов>> знаний и множеством информационных объектов ИКТ, например слов
естественного языка.

      В остальном каждый способ <<определяет>> свои правила членения знаний и
соответствия между этими двумя множествами. Поэтому <<объемы значений>> практически
любых соотносимых в двуязычных словарях пар слов не совпадают~\cite{19za}. При этом
правила членения и соответствия могут изменяться во времени, и со временем могут изменяться
<<объемы значений>> слов в каждом естественном языке.
\begin{figure*} %fig1
%\begin{verbatim}
%\begin{center}
\sf{
\hspace*{2.5mm}0~--- Общий отдел. Наука и знание. Информация. Документация. Библиотечное дело. Организации.\\
\hspace*{2.5mm}\hphantom{0~---~}Публикации в целом.\\
\hspace*{2.5mm}\hspace*{10pt}00~--- Общие вопросы науки и культуры. Пропедевтика.\\
\hspace*{2.5mm}\hspace*{20pt}004~--- Информационные технологии. Вычислительная техника. Теория. Технология и применение\\
\hspace*{2.5mm}\hphantom{0~---~00}\hspace*{20pt}вычислительных машин и систем.\\
\hspace*{2.5mm}\hspace*{30pt}004.8 --- Искусственный интеллект.\\
\hspace*{2.5mm}\hspace*{40pt}004.82 --- Представление знаний.\\
\hspace*{2.5mm}\hspace*{50pt}004.822 --- Сети знаний. Семантические сети.\\
\hspace*{2.5mm}\hspace*{50pt}004.823 --- Фреймовые системы. Фреймы. Схемы. Сценарии.\\
\hspace*{2.5mm}\hspace*{50pt}004.824 --- Множественные миры.\\
\hspace*{2.5mm}\hspace*{50pt}004.825 --- Порождающие системы. Системы правил вывода.\\
\hspace*{2.5mm}\hspace*{50pt}004.826 --- Модель черной доски.\\
\hspace*{2.5mm}\hspace*{50pt}004.827 --- Представление неоднозначности. Неопределенность. Пробелы знаний.
}
%\end{center}
%\end{verbatim}
\vspace*{3pt}
\Caption{ Отношения концепта <<Представление знаний>> с другими концептами
УДК
\label{f1za}}
\vspace*{3pt}
\end{figure*}

      Одновременный учет нескольких способов членения знаний необходим при разработке
мульти\-языч\-ных ИКТ и интеллектуальных систем, так как это позволяет соотносить <<объемы
значений>> слов разных естественных языков, включая описание тех случаев, когда в одних
языках отсутствуют слова для отдельных <<квантов>> знаний, выражаемых одним словом в
других языках. Например, в английском языке есть пять основных значений существительного
\textit{office}~\cite{20za}:
      \begin{enumerate}[(1)]
\item место выполнения профессиональных обязанностей;\\[-7pt]
\item функция или должностные обязанности (it is my office to open the mail~--- в мои
обязанности входит вскрывать почту);\\[-7pt]
\item должность в организации или позиция в иерархии институциональной системы;\\[-7pt]
\item ведомство, министерство, управление, организация или административная
единица (Foreign Office~--- Министерство иностранных дел);\\[-7pt]
\item сотрудники организации или ее подразделения (the whole office was late the
morning of the blizzard~--- из-за снежной бури утром целый офис опоздал на работу).
      \end{enumerate}

      В этом примере английскому слову office соответствуют пять <<квантов>> знаний как
основных значений этого слова в английском языке. Однако в других языках для представления
некоторых <<квантов>> знаний необходимые слова могут отсутствовать. Например, как
отмечается в материалах проекта EuroWordNet~\cite{20za}, в испанском и итальянском
языках нет слов для выражения пятого значения в приведенном примере для слова office.

      Таким образом, вопрос использования одного или нескольких способов членения знаний
является ключевым при описании любого варианта парадигмы единой области знаний,
охватывающей предметные области компьютерной и информационной наук.

      В примерах, иллюстрирующих актуальность вопроса использования одного или
нескольких способов членения знаний, в качестве конкретных способов упоминались только
естественные языки.
Говоря словами Р.~Барта: <<язык есть область артикуляции, а смысл в
первую очередь есть результат членения>>~[21, с.~277]. Однако при разработке
ИКТ и интеллектуальных систем могут использоваться и другие способы членения знаний,
основанные на той или иной системе классификации, а также такие искусственные способы
членения знаний, как семантические словари, тезаурусы и онтологии электронных библиотек.


      Для иллюстрации членения знаний на основе системы классификации рассмотрим
фрагмент Универсальной Десятичной Классификации (УДК) на рис.~\ref{f1za} и отношения
концепта <<Представление знаний>> с другими концептами этого фрагмента. В~УДК концепт
<<Представление знаний>> располагается на пятом уровне десятичной иерархии, имеет номер
004.82 и включает в себя 6~концептов шестого уровня с номерами от 004.822 до 004.827.
Отметим, что все эти 6~концептов в УДК являются терминальными, т.\,е.\ дальнейшее их
членение в УДК отсутствует.

      Концепт <<Представление знаний>> включен в концепт <<Искусственный интеллект>>
четвертого уровня, который, в свою очередь, включен в концепт <<Информационные
технологии. Вычислительная техника. Теория. Технология и применение вычислительных
машин и систем>> третьего уровня и~т.\,д.\ до концепта <<Общий отдел. Наука и знание.
Информация. Документация. Библиотечное дело. Организации. Публикации в целом>> самого
верхнего уровня.

      В заключение отметим важный терминологический аспект. Далее определение
информации по Бруксу будем называть <<ментальной информацией>>. При употреблении этого
словосочетания будем считать, что в информационной науке допускается членение знаний на
<<кванты>> и учитывается воз-\linebreak\vspace*{-12pt}
\pagebreak

\noindent
можность эволюции системы знаний человека во времени.

\subsection{Хьорланд: информационные процессы} %2.4.

      Для того чтобы очертить предметную область информационной науки, Биргер Хьорланд
при\-водит определение, которое сформулировало Американское общество информационной
науки
(American Society for Information Science~--- ASIS): <<Предметом исследований в
информационной науке являются процессы генерации, накопления, организации,
интерпретации, хранения, поиска, распространения, преобразования и использования
информации, а также применение современных технологий в перечисленных процессах. В
качестве дисциплины она стремится создать и структурировать научные и технологические
знания, имеющие отношение к передаче информации. Информационная наука включает
теоретические направления исследований безотносительно к конкретным приложениям и
прикладным исследованиям, ориентированным на создание продуктов и оказание
услуг>>~[22, с.~509].

      Анализируя это определение, Хьорланд отмечает явление, которое он называет
<<понятийным хаосом>> в информационной науке, проявляющийся, в частности, в том, что в
научной литературе насчитывается несколько сотен определений этой науки. Что касается
определения, данного ASIS, то, по мнению Хьорланда, в этом определении в явном виде не
выражена \textit{соотнесенность информации со знаниями}. Он отмечает, что в
информационной науке гораздо более распространенным является исследование информации
посредством рассмотрения процесса модификации знаний ее получателя (ментальная
информация Брукса) или в качестве снятия неопределенности (статистическая информация,
которая рассматривается в следующем разделе обзора).

      С точки зрения проблемы конвергенции в этом определении важно отметить
изо\-ли\-ро\-ванность инфор\-ма\-ци\-он\-ных процессов от применяемых технологий. Десять
перечисленных процессов~--- от генерации до передачи информации~--- являются предметом
исследований в информационной науке, а по отношению к технологиям, в том числе
компьютерным, предметом исследований является только их применение в этих процессах.

      Это определение отражает достаточно длительный исторический этап использования тех
методов обработки информации, которые не зависели от циф\-ро\-вой среды по причине ее
полного отсутствия. С появлением цифровой среды стал актуальным вопрос применения ИКТ в
перечисленных процессах. Со временем ИКТ стали оказывать влияние на создание новых
методов обработки информации, возможности которых стали во многом определяться
характеристиками и свойствами цифровой среды и, наоборот, на ее развитие оказывают
влияние новые методы обработки информации.

      В наше время подобное взаимное влияние становится \textit{одной из ключевых
предпосылок конвергенции информационной и компьютерной наук}. Например, современные
методы представления знаний в медицине и процессы обработки медицинских данных,
генерируемых компьютерными томографами, достаточно сильно зависят от возможностей ИКТ
и часто неразрывно интегрированы с ними~\cite{23za}.

\section{Компьютерная наука} %3

      В начале обзора уже говорилось, что в анг\-лийском языке существует как минимум
четыре устойчивых словосочетания~--- information science, computer science, computer and
information science,
computational science,~--- которые нередко переводятся на русский язык
одним словом <<информатика>>. Кроме того, необходимо учитывать, что в английском языке
смысл рассматриваемого в этом разделе термина ``computer science'' --- <<компьютерная наука>>~---
за последние 40~лет существенно из\-ме\-нился.


      Например, проект Computing Curricula, в рамках которого были подготовлены
<<Рекомендации по преподаванию программной инженерии и компьютерной науки в
университетах>>, ведет свой отсчет с 1968~г., когда была опубликована первая версия
рекомендаций. С тех пор эти рекомендации обновлялись примерно раз в десять лет совместным
комитетом по образованию под эгидой профессиональных ассоциаций Association for
Computing Machinery (ACM) и IEEE Computer Society. В конце 1990-х~гг.\ стало ясно, что
область знаний, связанная с ИКТ, очень сильно разрослась и ее трудно, если вообще возможно,
полностью осветить в рамках одного университетского курса. В связи с этим было принято
решение о его разделении на четыре основные специальности~--- computer science
(компьютерная наука), software engineering (програм\-мная инженерия), hardware engineering
(проектирование аппаратных платформ) и information systems (информационные
системы)~[24, с.~5].

      После этого разделения предметная область компьютерной науки в проекте Computing
Curricula стала включать следующие 14~разделов: дискретные структуры, основы
программирования, алгоритмы и теория сложности, архитектура и организация ЭВМ,
операционные системы, распределенные вычисления, языки программирования,
взаимодействие человека и машины, графика и визуализация, интеллектуальные системы,
управление информацией, социальные и профессиональные вопросы программирования,
программная инженерия, методы вычислений~[24, с.~193].

      Естественно, что эти четыре специальности тематически частично пересекаются.
Например, тему <<Дискретные структуры>> изучают в рамках специальности <<Программная
инженерия>>, а ряд тем <<Программной инженерии>> преподается для студентов
специальности <<Компьютерная наука>>.

      Значительное число основополагающих работ, составляющих теоретические основы
компьютерной науки, относится к первой половине XX~в. Среди ученых, которые глубоко
исследовали теоретические вопросы, связанные с алгоритмами и их возможностями, были
Клини, Черч, Тьюринг и Пост. Неформальное понятие алгоритма для решения некоторого
класса задач подразумевает некоторый набор правил, с помощью которых решение любой
указанной задачи этого класса может быть найдено в случае выполнения этого набора правил.
Так подходит к определению алгоритма Г.~Эббинхаус в своей статье <<Машины Тьюринга и
вычислимые функции~I. Уточнение понятия алгоритма>>~[25, с.~9--11].

      Вплоть до 30-х гг.\ прошлого столетия понятие алгоритма оставалось интуитивно
понятным, имевшим скорее методологическое описание, нежели математическое определение.
В истории науки известно много ярких примеров алгоритмов. Среди них алгоритм Евклида
нахождения наибольшего общего делителя двух натуральных чисел или двух целочисленных
многочленов, алгоритм Гаусса решения системы линейных уравнений, алгоритм разложения
многочлена одной переменной на неприводимые множители. Перечисленные алгоритмы
позволяли решать задачи путем указания и выполнения конкретных процедур. Для решения
подобных задач было достаточно интуитивного понимания алгоритма.

      Однако в начале ХХ~в.\ был сформулирован ряд алгоритмических проблем, решение
которых потребовало разработки и применения новых логических средств. Это связано с тем,
что доказательство существования и разработку разрешающего алгоритма можно осуществить
и с помощью интуитивного понимания алгоритма. Если же требуется доказать, что для решения
задачи не существует алгоритма, то в этом случае необходимо точное определение того, что
такое алгоритм.

      Определение алгоритма было предложено в первой половине XX~в.\ в двух формах: на
основе понятия рекурсивной функции и на основе описания процесса, осуществимого на
абстрактной машине. Был сформулирован тезис (<<тезис Тьюринга>>), утверждающий, что
любой алгоритм может быть реализован на соответствующей машине Тьюринга. Оба подхода, а
также другие подходы (Маркова и Поста) привели к одному и тому же классу алгоритмически
вычислимых функций и подтвердили целесообразность использования тезиса Тьюринга для
решения алгоритмических проблем.

      В настоящее время теория алгоритмов является краеугольным камнем фундамента
компьютерной науки. С ее помощью были уточнены такие понятия, как доказуемость,
эффективность, разрешимость, перечислимость и другие.

      В этом разделе обзора сначала будут рассмотрены отдельные вопросы становления
компьютерной науки, а затем в ее предметной области будут обозначены те тематические
направления, которые являются ключевыми для конвергенции компьютерной и
информационной наук.

\subsection{Машина Тьюринга и универсальные вычислительные машины} %3.1.

      С момента создания первой универсальной вычислительной машины ENIAC (Electronical
Numerical Integrator and Computer), которую в момент ее создания называли математическим
роботом\footnote{Словосочетание <<математический робот>> использовалось в первом пресс-релизе о
компьютере ENIAC от 16~февраля 1946~г.\ (см.\ 
{\sf http://www.americanhistory.si.edu/csr/comphist/pr1.pdf}). Разработка этого компьютера была начата 
в июле 1943~г.\ и завершена осенью
1945~г.}, компьютерная наука была тесно связана с проектами разработки компьютеров,
которые в первую очередь были ориентированы на решение вычислительных задач.

      Вычислительную машину фон Нейман определял как <<устройство, которое может
выполнять команды для вычислений значительной слож\-ности>>. Центральным моментом в
работе фон Неймана была формулировка требований к структуре вычислительной машины. В
них фон Нейман фактически описал структурную схему аналитической машины Бэббиджа,
состоящей (в современной терминологии) из арифметического устройства (АУ), устройства
управления (УУ), оперативного запоминающего устройства (ОЗУ), внешнего запоминающего
устройства (ВЗУ), устройств для ввода и вывода информации (УВВ). Однако в архитектуре фон
Неймана были существенные отличия от схемы аналитической машины Бэббиджа. В частности,
фон Нейман рекомендовал использовать в устройстве не механическую, а электронную
элементную базу и не десятичную, а двоичную систему ис\-чис\-ле\-ния. Бэббидж искал аналогии
между блоками вычислительной машины и структурными производственными единицами
(мельница, склад), в то время как фон Нейман находил аналоги в живом организме (точнее, в
нейронных сетях)~\cite{26za}.

      В процитированной работе~\cite{26za} проводится сравнение универсальной
вычислительной машины с <<фон-неймановской архитектурой>> и <<универсальной машины
Тьюринга>> (Universal Turing Machine~--- UTM). Английский математик Алан Тьюринг
предложил использовать UTM для исследования <<проблемы разрешимости>> (the Hilbert
Entscheidungsproblem), которую сформулировал в 1900~г.\ немецкий математик
Гильберт~\cite{27za}.

      В 1936~г.\ Тьюринг доказал, что эта проблема не имеет решения, и опубликовал
полученные результаты в статье <<О вычислимых числах применительно к проблеме
разрешимости>>~\cite{28za}. Важно то, что для этого доказательства Тьюринг использовал
предложенную им гипотетическую машину UTM.

      Краткое описание одного из вариантов UTM приводит Эббинхаус в своей статье:
<<Машина Тьюринга Т состоит из операционного исполнительного устройства, которое может
находиться в одном из дискретных состояний $q_0, \ldots , q_s$, принадлежащих некоторой
конечной совокупности, комбинированной читающей и пишущей головки, счетной ленты
$\langle\ldots\rangle$ и лентопротяжного механизма. При этом $q_0$ называется начальным
состоянием [машины]~Т. Ячейки ленты пронумерованы, начиная с крайней левой, числами 0, 1,
2,\ldots Читающая и пишущая головка находится в каждый данный момент времени над
некоторой ячейкой ленты~--- текущей рабочей ячейкой. С помощью лентопротяжного
механизма одна из ячеек, соседняя с рабочей ячейкой, может быть помещена под читающей и
пишущей головкой; в таком случае мы будем говорить, что рабочая ячейка сдвинулась на одну
ячейку вправо или влево. Читающая и пишущая головка может читать буквы алфавита
$\mathbf{A} = \{ a_1, \ldots , a_t\}$ и букву $a_0$ (у Тьюринга этот символ означает <<пусто>>,
то есть отсутствие информации в ячейке), стирать их и печатать; \textbf{А} называется рабочим
алфавитом [машины]~Т\ldots >>~[25, с.~24--25].

      Тьюринг показал, что его машина <<за данный большой, но конечный промежуток
времени способна справиться с любым вычислением, которое сможет выполнить всякая сколь
угодно мощная вычислительная машина>>~\cite{26za}. Далее в цитируемой работе
сравнивается концепция <<фон-неймановской архитектуры>> со свойствами UTM:
      \begin{itemize}
\item Тьюринг фактически впервые выдвинул концепцию вычислительной машины с
хранимыми в памяти командами (программой);
\item поскольку операции UTM на каждом такте зависят, в частности, от результата
последнего действия, можно говорить, что машина выполняет команду условного
перехода;
\item Тьюринг показал, что результатом работы машины может быть группа символов,
которые, будучи введены в другую UTM, заставят ее действовать так же, как первую;
иными словами, машина может <<генерировать>> или видоизменять программу, и
Тьюринг понимал, что это ее свойство должно быть перенесено на реальную ЭВМ;
\item любую универсальную вычислительную машину можно запрограммировать так,
что она будет моделировать работу некой специализированной машины (создатели
ENIAC делали то же самое, когда настраивали свою универсальную машину на решение
конкретной задачи)~\cite{26za}.
\end{itemize}

      В описании UTM содержится существенное ограничение: все вычисления выполняются
на одномерной ленте, то есть допускается только линейная конкатенация символов алфавита
UTM при построении символьных выражений. При этом в работе Тьюринга есть замечание о
том, что в математике используются двумерные символьные выражения, которые всегда могут
быть преобразованы в одномерные, что позволяет обрабатывать их с помощью UTM. Говоря о
размещении символов алфавита UTM в ячейках одномерной ленты, Тьюринг отмечает, что в
одной ячейке может располагаться линейная последовательность символов, трактуемая как
единый сложный символ, и проводит аналогию между сложными символами и словами
европейский языков. Другими словами, при обработке с помощью UTM эти \textit{слова
предлагается трактовать как сложные символы, размещаемые в ячейках одномерной
ленты}~\cite{28za}.

\subsection{Статистическая информация} %3.2.

      Параллельно с развитием понятия алгоритма и созданием универсальной
вычислительной машины формировалось еще одно направление исследований, связанное с
именами Норберта Винера и Клода Шеннона. В построении теоретического фундамента этого
направления использовались статистические и вероятностные методы.

      Норберт Винер в гл.~3 своей книги~[29, с.~119] о предметной области
статистической науки говорит следующим образом: <<Существует широкий класс явлений, в
которых объектом наблюдения служит какая-либо числовая величина или последовательность
числовых величин, распределенных во времени. Температура, непрерывно записываемая
самопишущим термометром; курс акций на бирже в конце каждого дня; сводка
метеорологических данных, ежедневно публикуемая бюро погоды,~--- все это временные ряды,
непрерывные или дискретные, одномерные или многомерные. Эти временные ряды меняются
сравнительно медленно, и их вполне можно обрабатывать посредством вычислений вручную
или при помощи обыкновенных вычислительных приборов, таких как счетные линейки и
арифмометры. Их изучение относится к обычным разделам статистической
науки>>.

      О статистическом характере исследуемых объектов и явлений Винер пишет далее: <<Все
эти временные ряды и все устройства, работающие с ними, будь то в вычислительном бюро или
в телефонной схеме, связаны с записью, хранением, передачей и использованием информации.
Что же представляет собой эта информация, и как она измеряется? Одной из простейших,
наиболее элементарных форм информации является запись выбора между двумя
равновероятными, простыми альтернативами, например между орлом и решкой при бросании
монеты. Мы будем называть решением однократный выбор такого рода. Чтобы оценить теперь
количество информации, получаемое при совершенно точном измерении величины, которая
заключена между известными пределами $A$ и $B$ и может находиться с равномерной
априорной вероятностью где угодно в этом интервале, положим $A = 0$, $B = 1$ и представим
нашу величину в двоичной системе бесконечной двоичной дробью $0{,} a_1a_2a_3\ldots
a_n\ldots$, где каждое $a_1$, $a_2$, $a_3$, \ldots , $a_n$  имеет значение 0 
или~1>>~[29, с.~120].

      Для Винера центральным было понятие информации как термина статистической науки.
В~ста\-тистической науке, которая является смежной с компьютерной наукой, слово
<<информация>> включено в единую систему терминов со словами <<шум>>, <<помеха>>,
<<вероятность>>, <<энтропия>> и~т.\,д., но не со словом <<знания>>, как это сделано в
информационной науке. Поэтому далее информацию статистической науки будем называть
статистической информацией, которая понимается совсем не в том смысле, который
вкладывается в словосочетания <<языковая информация>> по Фаррадейну, <<знаковая
информация>> по Ингверсену и <<ментальная информация>> по Бруксу в разд.~2 данного
обзора.

      В качестве меры статистической информации в совокупности сообщений Шеннон
предлагает использовать логарифмическую функцию: <<Если ряд сообщений из множества
сообщений конечен, то этот ряд или его произвольную монотонную функцию можно
рассматривать как меру информации, произведенной на тот момент, когда выбирается одно
сообщение из множества сообщений, причем все выборы равнозначны. Как заметил Хартли,
наиболее естественным выбором является логарифмическая функция. Хотя это определение
должно быть подвергнуто значительному обобщению, когда мы рассматриваем влияние
статистики сообщения и когда мы располагаем постоянным диапазоном сообщений, в любом
случае мы будем использовать, по сути, логарифмическую меру>>~\cite{30za}.

      Выбор основания логарифма соответствует выбору единицы измерения информации.
Если ис\-поль\-зу\-ет\-ся основание~2, то результирующие единицы можно назвать двоичными
числами или \mbox{более} кратко битами, термином, который предложил Дж.~Теки. Устройство с
двумя устойчивыми состояниями, такое как реле или триггер, может хранить один бит
информации; $N$ таких устройств могут хранить $n$~бит, суммарное число возможных
состояний равно $2^n$, а $\log_2 2^n = n$~\cite{30za}.

      Шеннон подчеркивает, что, говоря о статистической информации, ее передаче и
соответ\-ст\-ву\-ющих системах передачи, он имеет в виду именно физическую модель передачи
сообщений, а содержательная сторона передаваемого сообщения при этом не рассматривается.
Иначе говоря, по смыс\-лу <<информация>> статистической науки принципиально отличается от
<<информации>> информационной науки. Таким образом, наблюдается только
\textit{совпадение последовательности 10~литер: <<информация>>, но отличаются
<<кванты>> знаний, соответствующие этим литерам в статистической и информационной
науках}.

      Отметим, что теоретические исследования Шеннона и Винера были необходимы для
практических приложений и результаты этих исследований широко использовались. Например,
Шеннон исследовал природу передаваемой по каналу связи статистической информации и
вопросы оптимизации этого процесса. Теория связи была востребована во время второй
мировой войны, а затем получила развитие в многочисленных приложениях в мирное время.
Винер, в свою очередь, полагал, что методы обработки статистической информации могут
коренным образом изменить взгляд на использование вычислительных машин. Он считал, что
<<при применении этих машин становится все более очевидным, что они требуют специальных
математических методов, совершенно отличных от тех, к которым прибегали в ручных расчетах
или на малых машинах>>~[29, с.~206].

      Отметим, что дальнейшее развитие методов статистической науки, востребованное в
кибернетике, компьютерной науке и теории автоматического управления, нашло свое
отражение в работах Р.\,Е.~Калмана, В.\,С.~Пугачева и ряда других
      исследователей~[31--33]. Разработка специальных
математических методов, ориентированных на применение ЭВМ и совершенно отличных от
тех, к которым прибегали в ручных расчетах, является примером того, что ИКТ оказывают
влияние на создание новых методов в широком спектре областей знаний.

\subsection{Представление знаний и~управление информацией
в~компьютерной науке} %3.3.

      Рассмотренные ранее в этом разделе вопросы, относящиеся к предметной области
компьютерной науки, характеризуют исторические аспекты ее становления, но не дают
представление о тематическом содержании компьютерной науки в ее современном понимании
и о влиянии этой науки на создание новых поколений ИКТ.

      Значительно более полное представление о ее предметной области дает перечень из
14~позиций, приведенный в начале этого раздела, от дискретных структур (1-я позиция) до
методов вычислений (14-я позиция). Каждая из 14~позиций достаточно подробно описана в
Рекомендациях по преподаванию компьютерной науки в университетах~\cite{24za}.
При их описании используются два уровня детализации для каждой позиции.

      Выделим в детализированных описаниях формулировки вопросов представления знаний
и управ\-ле\-ния информацией, которые являются ключевыми одновременно для компьютерной и
информационной наук. Начнем с позиции <<Языки программирования>>, которая на первом
уровне детализации включает 11~пунктов, в том числе семантику языков
программирования~[24, с.~197]. Для этого пункта рассмотрим второй уровень
детализации, который включает 5~следующих тем: неформальная семантика, обзор
формальной семантики, денотационная семантика, аксиоматическая семантика и операционная
семантика~[24, с.~329].

      Таким образом, в программу изучения компьютерной науки включена тема
<<Неформальная семан\-ти\-ка>>, относящаяся и к информационной
науке. При этом важно
отметить, что это не единственный пример подобного тематического пересечения. Отметим
еще две позиции в исходном списке, в которых наблюдается аналогичное явление
тематического пересечения~--- <<Интеллектуальные системы>> и <<Управление
информацией>>.

      Позиция <<Интеллектуальные системы>> на первом уровне детализации включает
10~пунктов: основные вопросы (связанные с интеллектуальными системами), поиск решений,
методы пред\-став\-ле\-ния знаний, углубленное изучение методов поиска, углубленное изучение
методов представления знаний, агенты, обработка текстов на естественном языке, методы
обучения компьютеров и нейронные сети, системы искусственного интеллекта с планируемым
поведением, робототехника~[24, с.~342--347].

      С точки зрения процесса конвергенции привлекает внимание пункт с методами
представления знаний, включающий на втором уровне детализации тему
<<\textit{Структурное представление знаний}>>, а также пункт обработки текстов на
естественном языке, включающий следующие две темы: <<\textit{Методы, основанные на
совокупности текстов}>> и <<\textit{Информационный поиск}>>~[24, с.~342--347], 
которые также являются примерами тематического пересечения
информационной и компьютерной наук.

      Позиция <<Управление информацией>> на первом уровне детализации включает
14~пунктов: информационные модели и системы, системы баз данных, моделирование данных,
реляционные базы данных, языки запросов к базам данных, про\-ек\-тиро\-ва\-ние реляционных баз
данных, обработка
 транз\-ак\-ций, распределенные базы данных, проектирование физической
структуры базы данных, извлечение информации, хранение и поиск информации, гипертекст и
гипермедиа, муль\-ти\-ме\-дий\-ная информация и системы мультимедиа, цифровые библиотеки. 
С~точки зрения процесса
 конвергенции привлекают внимание вопросы хранения, извлечения и
поиска информации, а также цифровые библиотеки. Наиболее явно тематическое пересечение
информационной и компьютерной наук проявляется на втором уровне детализации пункта
<<Хранение и поиск информации>>, который включает сле\-ду\-ющие темы~[24,
с.~342--347]:
      \begin{itemize}
\item \textit{информационные потребности пользователя}, релевантность и оценка
эффективности поиска;
\item тезаурус, онтология, классификация и категоризация;
\item \textit{библиографическая информация и библиометрия};
\item \textit{резюмирование} и визуализация информации;
\item \textit{интеграция ключевых слов и схем классификации}.
\end{itemize}

      Следовательно, в тематических разделах, характеризующих предметную область
компьютерной науки в ее современном понимании, наблюдается пересечение с
информационной наукой в первую очередь по вопросам представления знаний и управления
информацией, которые являются ключевыми одновременно для компьютерной и
информационной наук.

  \end{multicols}

\begin{figure*} %fig2
\vspace*{1pt}
\begin{center}
\mbox{%
%\epsfxsize=158.49mm
%\epsfbox{zac-2.eps}
\epsfxsize=158mm
\epsfbox{zac-2t.eps}
}
\end{center}
\vspace*{-9pt}
\Caption{Количественные взаимосвязи направлений научных исследований с ИТ для
европейских патентов~(\textit{а}) и для патентов США~(\textit{б})
\label{f2za}}
%\vspace*{-.5mm}
\end{figure*}

 \begin{multicols}{2}


      С одной стороны, наблюдаемое тематическое пересечение, естественное для любой
области знаний, в процентном отношении от общего числа тем второго уровня детализации не
превышает нескольких процентов. С другой стороны, если необходимо анализировать
наблюдаемое тематическое пересечение с позиций потребностей в разработке научных основ
создания новых поколений ИКТ, то нельзя ограничиваться только процентным отношением от
общего числа тем.

      Для подобного анализа необходимо использовать количественные меры (индикаторы)
связей областей технологического развития и тех на\-прав\-ле\-ний научных исследований,
результаты которых используются в процессе создания новых технологий~\cite{34za}.
Знание значений этих индикаторов для направлений компьютерной науки позволило бы
оценить их влияние на развитие ИКТ.

      Оценка влияния отдельных направлений компьютерной науки требует учета как
минимум первого уровня детализации для каждой из 14~позиций в тематическом делении этой
науки. Однако в настоящее время отсутствуют индикаторные оценки влияния отдельных
направлений компьютерной науки как для первого уровня детализации ее предметной области,
так и для каждой из 14~позиций в целом.
{\looseness=-1

}

      Имеются сведения о взаимосвязях информационных технологий (ИТ) только со всей
компьютерной наукой в целом и рядом других областей знаний. На рис.~\ref{f2za} 
приведены диаграммы, которые иллюстрируют индикаторы связей ИТ с теоретическими и
прикладными дисциплинами, включая компьютерную науку. Первая диаграмма
(рис.~\ref{f2za},\,\textit{а}) построена на основе данных Европейского патентного ведомства, вторая
диаграмма (рис.~\ref{f2za},\,\textit{б})~--- на основе данных Патентного ведомства США\footnote{Эти
диаграммы взяты из научно-технического отчета `Linking Science to Technology~--- Bibliographic
References in Patents', подготовленного на основе результатов проектов, выполненных по контракту
ERBHPV2-CT-1993-03 в рамках 5-й и 6-й Рамочных программ ЕС.}.

      На каждой из диаграмм приведены процентные отношения для 10~теоретических и
прикладных дисциплин за период 1992--1996~гг. На рис.~\ref{f2za},\,\textit{а} для компьютерной науки
указано процентное отношение 4,55\%. Это число было определено следующим образом: из
европейских патентов за период 1992--1996~гг., относящихся к сфере ИТ, были извлечены все
ссылки на научные публикации, которые затем были распределены по направлениям
теоретических и прикладных исследований. Число~4,55\%~--- это доля научных публикаций по
компьютерной науке от общего числа публикаций по всем теоретическим и прикладным
дисциплинам, на которые есть ссылки в европейских патентах по ИТ. На обеих диаграммах
указаны только первые 10~дисциплин с наибольшим числом публикаций. Отметим, что
информационная наука не попала в первую десятку научных дисциплин на этих диаграммах.

      Приведенные диаграммы иллюстрируют существенные отличия в цитировании научных
публикаций для разных теоретических и прикладных дисциплин. Существуют отличия и в
региональном разрезе, например доля научных публикаций по компьютерной науке, на
которые есть ссылки в патентах США по ИТ, равна 2,35\%, что на 2,2\% меньше, чем доля
научных публикаций по компьютерной науке, на которые есть ссылки в европейских патентах
по ИТ (см.\ рис.~\ref{f2za}).

Существуют аналогичные диаграммы и для телекоммуникационных технологий. Однако
данные имеющихся диаграмм не позволяют количественно оценить взаимосвязи ИКТ с
тематическим пересечением предметных областей информационной и компьютерной наук с
позиций потребностей в разработке научных основ создания ИКТ, так как в этих данных
компьютерная наука рассматривается как единое целое, без деления на составляющие ее
направления.

      В настоящее время можно говорить только о \textit{существенных отличиях в доле
цитирования в патентах научных публикаций} из разных теоретических и прикладных
дисциплин. Поэтому, скорее всего, нельзя ограничиваться только процентным отношением
тематического пересечения информационной и компьютерной наук к общему числу тем
последней в задачах оценки роли этого пересечения для создания новых поколений ИКТ.

\section{Информационно-компьютерная~наука} %4

      Материал разд.~2 и~3 позволяет предположить, что одновременно с формированием
и институционализацией информационной и компьютерной наук как самостоятельных научных
дисциплин и областей применения их результатов наблюдалось развитие отдельных
предпосылок их конвергенции. Отметим, что многоаспектное исследование проблемы
конвергенции началось более 40~лет назад.

Термин <<информационно-компьютерная наука>>, который вынесен в название этого
раздела, одним из первых использовал американский ученый С.~Горн в 1963~г., 
с той разницей, что тогда этот термин употреблялся во множественном чис\-ле. 
Единственное число использовалось ученым начиная с 1983~г.~\cite{12za}. Однако уже в 1963~г.
потенциальный результат конвергенции информационной и компьютерной наук
позиционировался С.~Горном как новая фундаментальная область знаний, что нашло
отражение в самом названии его работы~--- ``a~new basic discipline''~\cite{1za}.

      В течение двадцати лет им была опубликована серия статей о предметной области и
методологии информационно-компьютерной науки~[1, 12, 35--37]. 
В качестве смежных дисциплин С.~Горн называет библиотековедение, теорию
информационного поиска, информационную науку, кибернетику, когнитивную психологию,
искусственный интеллект, семиотику, лингвистику и математику. Среди сфер применения
результатов этой науки он выделяет разработку компьютеров, менеджмент и сферу
образования~[12, с.~121].

      В 1963~г.\ Горн предпринял попытку перечислить вопросы, изучаемые
ин\-фор\-ма\-ци\-он\-но-компью\-тер\-ной наукой: <<Примерами основных вопросов исследования в этой
области могут быть системы программирования, проектирование компьютерных систем,
искусственный интеллект, информационный поиск и~т.\,д. Вероятностная информационная
теория Шеннона определенно принадлежит к этой области знания, но помимо нее существует
еще теория информации искусственных языков и ее обработки, которую также необходимо
включить в предметную область этой науки. Одним из центральных вопросов этой новой
дисциплины, скорее всего, станет синтез и анализ искусственных языков и их
процессоров>>~[1, с.~150].

      В отличие от авторов аналитического доклада~\cite{10za}, Горн в явном виде
включает в перечень направ\-ле\-ний, изучаемых ин\-фор\-ма\-ци\-он\-но-ком\-пью\-тер\-ной наукой,
искусственный интеллект,  информационный поиск, синтез и анализ искусственных языков.
После перечисления этих вопросов, информационно-компьютерная наука \mbox{далее}
рассматривается им уже как учебная дис\-цип\-ли\-на и говорится о необходимости описать
способы различения новой области знаний от соседних с ней областей в учебном процессе. К
примеру, каким образом абитуриент может узнать, относится ли сфера его интересов именно к
этой новой об\-ласти знаний, а не к одной из уже устоявшихся дис\-цип\-лин? Какое ему
необходимо образование для того, чтобы углубиться в эту новую область? 
И~в чем результат
его обучения существенным образом будет отличаться от того образования, которое
потребовалось бы ему в другой области?

      В 60-х гг.\ прошлого века ощущалась потребность в оценке перспектив развития этой
новой дисциплины, в ее позиционировании среди существовавших уже тогда областей знаний и
учебных дисциплин. Горн рассматривает эти вопросы, отталкиваясь в своих рассуждениях от
профессиональных интересов ученых в этой области еще на стадии получения ими
образования:

      <<Информационно-компьютерная наука рас\-смат\-ри\-ва\-ет прагматические аспекты
использования символов их пользователями и интерпретаторами в качестве еще одного
центрального вопроса таким же образом, как эти аспекты должны исследоваться специалистами
в области лингвистики, психологии, философии и инженерных наук.

      Таким образом, студент, изучающий численный анализ, в процессе разработки или
анализа какого-либо алгоритма мыслит себя как математик, если его единственный интерес
заключается в доказательстве существования алгоритма или определения его точности. Но он
является специалистом в области информационно-компьютерной науки, если рассматривает
этот алгоритм прагматически, например с точки зрения его реализации (обработки
процессором), и интересуется эффективностью его работы, временными затратами,
распределением памяти и~т.\,д.

      Аналогично студент, изучающий процедуру адап\-тив\-но\-го управления, описывающую
поведение животного в некоторой ситуации, позиционирует себя как психолог, если его
главной задачей является выяснение того, обладает ли он хорошей моделью поведения этого
животного. Если его интересует проблема искусственного интеллекта как одного из
направлений ин\-фор\-ма\-ци\-он\-но-ком\-пью\-тер\-ной науки, то он интересуется применимостью этой
процедуры независимо от того, является ли она моделью поведения животного или не яв\-ля\-ется.

      Студент, занимающийся порождающей грамматикой, мыслит себя как лингвист, если его
больше всего интересует, действительно ли естественный язык работает так, а не иначе. Однако
он думает как ученый в области информационно-компьютерной науки, если его занимает
вопрос, каким образом можно использовать эту грамматику в информационной системе.
Лингвист может рассматривать механизм стековой памяти, но с глубиной не более семи из-за
ограниченных возможностей локальной памяти человека, но для решения 
информационно-компьютерных задач такой глубины явно недостаточно>>~[1, с.~154].
%z
      Рассмотрев в статье 1963~г.\ эти примеры, Горн предлагает перечень тех дисциплин,
которые должны преподаваться студентам, изучающим ин\-фор\-ма\-ци\-он\-но-компьютерную науку,
включая математику, физику, философию, лингвистику, психологию, вычислительную технику
и компьютерное программирование.

      Предложенный подход к изучению ин\-фор\-ма\-ци\-он\-но-компью\-тер\-ной науки уже тогда
начал реализовываться в Пенсильванском университете. Через двадцать лет, когда уже
накопился большой опыт ее преподавания, Горн пишет, что его понимание концепции
ин\-фор\-ма\-ци\-он\-но-компью\-тер\-ной науки заключается в том, что \textit{эта область знаний не
является ветвью математики, так как она должна соотносить себя с прагматическими
вопросами, от которых математика не должна зависеть}~[12, с.~137].

      Следует отметить, что процитированная статья начинается со следующей фразы:
<<Позвольте мне, прежде всего, выбрать более короткое название, чем
      \textit{информационно-компьютерная наука}. Я выбираю термин
<<\textit{информатика}>>, созвучный французскому Informatique и немецкому Informatik. Он
несет в себе идею информации, а оканчивается так же, как и математика, подразумевая
формализованную теорию. Плохо то, что при использовании слова <<информатика>> теряется
компьютерная составляющая в названии и, кроме того, оно не вызывает ассоциаций с 
какой-либо экспериментальной основой>>~[12, с.~121].

      Следовательно, Горн, используя в 1983~г.\ термин <<информатика>>, подразумевает
под ним именно информационно-компьютерную науку. Ученый обращается к истокам этой
дисциплины, чтобы дать четкое определение информатике: <<Все, что я до сих пор говорил о
вычислениях, ориентировано на практическую деятельность и связано с компьютером. Но сама
теория вычислений уже сформировалась и существовала к тому времени, когда появились
цифровые компьютеры. $\langle\ldots\rangle$ Специалисты в области символьной логики уже
исследовали логические пределы вычислений; была описана универсальная машина Тьюринга
и доказана неразрешимость проблемы остановки; Гедель продемонстрировал пределы
формализма при помощи своих теорем о неразрешимости; Черч, Клини и Карри
проанализировали вычисления в теории рекурсивных функций и комбинаторной логике; Туэ и
Пост, а в более позднее время Марков, рас\-смот\-ре\-ли  вычисления с синтаксической точки
зрения. $\langle\ldots\rangle$ Поэтому, когда появились компьютеры, обсуждение лингвистики
естественных языков Ноамом Хомским происходило в ракурсе вычислений
$\langle\ldots\rangle$ В результате этих новых разработок появились лингвистические описания
процессов программирования, математическая теория автоматов и формальные языки. Эти
результаты, в свою очередь, повлияли на разработки языков программирования и
программируемых вычислительных машин.  $\langle\ldots\rangle$ Теперь под информатикой мы
понимаем нечто, связанное с синтезом и анализом символьных выражений, а также синтез и
анализ процессоров, которые интерпретируют, транслируют и обрабатывают такие выражения.
Если говорить более прозаично, то информатика занимается изучением, проектированием и
использованием структур данных и их обработкой\ldots >>~[12, с.~131].

      Главный вывод Горна о составе и статусе новой области знания, которым он завершает
\mbox{статью}, состоит в следующем: <<$\langle\ldots\rangle$\,не следует отделять компьютерную науку от информационной
науки, а следует пытаться отстаивать единую область знаний~--- информатику. Любая попытка
поощрить такое разделение $\langle\ldots\rangle$ повлечет за собой отделение практической
деятельности от знаний, как это произошло с математикой Пифагора, риторикой софистов,
метафизикой и органоном Аристотеля, грамматикой стоиков, логикой и грамматикой
логических позитивистов. Такое разделение будет причиной прекращения деятельного
кипения, которое поддерживается сплавом знаний и практической
деятельности>>~[12, с.~139--140].

      Проиллюстрировать последствия подобного от\-де\-ле\-ния практической деятельности от
знаний можно было бы количественно с помощью диаграмм, подобных рис.~\ref{f2za}, 
на которых доля ком\-пьютер\-ной науки равна 4,55\% для европейских патен\-тов и
2,35\% для патентов США за период 1992--1996~гг. Однако для полноты картины явно не
хватает исходных данных для выявления тренда изменений этих долей во времени.

      Вернемся к проблеме конвергенции. Важным этапом в развитии идеи Горна,
позиционирующей информатику как единую область знаний и охватывающую предметные
области компьютерной и информационной наук, были работы Ю.\,А.~Шрейдера~\cite{38za, 39za}. 
В статье <<Информация и знание>> говорится, что не существует двух
информатик (информационной науки и компьютерной науки), а есть два облика информатики.
Первый из них (информационная наука) дополнительно нагружен представлениями о
традиционном информационном обслуживании специалистов-ученых и инженеров в области
их профессиональных интересов. Второй облик (компьютерная наука) неправомерно искажен
чисто программистскими проблемами, не специфичными для информатики. Специфические же
проблемы информатики оказываются там, где возникают задачи \textit{информационного
представления знаний в форме, удобной для обработки, передачи и творческого
реконструирования знаний в результате усилий пользователя}~[39, с.~51].

      В этой же работе Ю.\,А.~Шрейдер формулирует ряд положений научной парадигмы
информатики: <<информация есть общественное достояние, она в принципе социальна, в то
время как знание, вообще говоря, соотнесено с конкретной личностью, с тем, кто им владеет и
непосредственно пользуется. $\langle\ldots\rangle$ Информация должна пройти через
<<когнитивный экран>> тех, для кого она пред\-став\-ля\-ет ценность. Так возникает необходимость
считаться не только с суще\-ст\-во\-ва\-ни\-ем мира объективированного социализированного знания,
т.\,е.\ информации как превращенной формы знания, но и с феноменом личностного знания.
$\langle\ldots\rangle$ Тож\-де\-ст\-вен\-ность информации и знания при этом исключается, но
информация как превращенная форма знания сохраняет следы своего происхождения.
$\langle\ldots\rangle$\,наиболее принципиальные вопросы информатики всегда возникали на
стыке информации и знания, там, где речь шла о превращении одного в другое>>. Далее
Шрейдер пишет о пропасти, разделяющей \textit{информацию и знания как сущности разной
природы}~[39, с.~50--51].

\section{Заключение} %5

      Обзор многолетней истории проблемы конвергенции информационной и компьютерной
наук позволяет сделать следующие выводы.

      Во-первых, приведенные положения из работ Горна и Шрейдера являются ключевыми
для описания научной парадигмы информатики как информационно-компьютерной науки, но
не включают всех необходимых ее составляющих, в том числе \textit{аксиоматику,
классификацию объектов, процессов и явлений этой области знаний, а также систему
терминов}.

      Во-вторых, процессы понимания, осознания и экспликации знаний в настоящее время
по-преж\-не\-му остаются во многом невыясненными. В~информационной науке они исследуются
как когнитивные и креативные процессы, с которыми неразрывно связаны процессы генерации
информации, социальных коммуникаций и понимания информации. В этой науке знания
человека (в том числе ментальная информация Брукса) и информация (знаковая информация
Ингверсена и языковая информация Фаррадейна) соотносятся между собой как
\textit{сущности разной природы}~\cite{6za, 13za, 16za}.

      В-третьих, в компьютерной науке в качестве \mbox{базовых} понятий используются, как
правило, <<символы абстрактного алфавита>>, в явном виде не соотнесенные со знаниями
человека и ментальной информацией Брукса, а также со знаковой информацией Ингверсена и
языковой информацией Фаррадейна. Например, в классической работе Тьюринга слова
<<знания>> и <<информация>> не используются, а рассматриваются лишь линейные
символьные выражения. Однако в этой работе отмечается, что в одной ячейке может
располагаться линейная последовательность символов, трактуемая как единый сложный
символ, и проводится аналогия между сложными символами и словами европейских
языков~\cite{28za}.

      Таким образом, имеется \textit{непустое пересечение множества символьных
выражений <<языка>> компьютерной науки и множества слов естественных языков
информационной науки}, являющихся
знаковой информацией, т.\,е.\ знаковыми формами представления знаний и главной сущностью
социокультурных коммуникаций. Это объектное пересечение (т.\,е.\ пересечение объектов
\textit{исследования в компьютерной и информационной науках}) относится одновременно к
предметным областям обеих наук. Однако эти объекты в компьютерной науке трактуются и
обрабатываются как абстрактное множество символьных выражений, а в информационной
науке эти же объекты трактуются и обрабатываются как множества конкретных слов
естественных языков с их собственными планами выражения и содержания. Для интеграции
двух подходов к трактовке и обработке этого объектного пересечения необходима новая
научная парадигма информационно-компьютерной науки.

      Заключительная фраза статьи С.~Горна говорит о том, что не следует отделять
компьютерную науку от информационной науки, а следует пытаться отстаивать единую
область знаний. В настоящее время остаются открытыми главные вопросы проблемы
конвергенции:
\begin{itemize}
\item[$\diamondsuit$] На каких теоретических основаниях должна строиться информационно-компьютерная наука как единая область знаний?
\item[$\diamondsuit$] В какой системе аксиом и с использованием каких терминов следует строить
информационно-компьютерную науку как единую область знаний?
\end{itemize}

      Научная парадигма информацион\-но-компью\-тер\-ной науки, которая будет предлагать
ответы на поставленные вопросы, должна включать описание системы аксиом и теоретических
оснований этой науки. При этом с единых концептуальных позиций должны быть описаны и
классифицированы множество абстрактных символов компьютерной науки и множество
конкретных знаковых систем (в~том числе естественные языки) информационной науки.

      Должны быть описаны отношения между этими множествами символов и знаков на
стыке абстрактного и конкретного, включая процессы превращения одного в другое, а также
\textit{процессы локальной генерации отдельных <<квантов>> знаний, их интеграции в
цифровой среде и глобального использования в среде социальных коммуникаций на
      символьно-знаковой основе}.

      \bigskip
      В завершение необходимо отметить, что настоящий обзор был подготовлен при
поддержке \mbox{РФФИ} и в соответствии с правилами Фонда должен включать также описание
результатов проектов по тематике обзора, финансируемых по грантам \mbox{РФФИ}. Однако авторам
не удалось найти публикаций с изложением результатов проектов \mbox{РФФИ}, посвященных
проблеме конвергенции информационной и компьютерной наук, а также вопросам определения
количественных индикаторов взаимных связей информационной и компьютерной наук с ИКТ.

{\small\frenchspacing
{\baselineskip=10.8pt
\addcontentsline{toc}{section}{Литература}
\begin{thebibliography}{99}

     \bibitem{1za}
     \Au{Gorn S.} The computer and information sciences: A new basic discipline~// SIAM Review,
April, 1963. Vol.~5. No.\,2. P.~150--155.

     \bibitem{2za}
     Decision No\,1982/2006/EC of the European Parliament and of the Council of 18 December
2006 concerning the Seventh Framework Programme of the European Community for research,
technological development and demonstration activities (2007--2013)~// Official\ 
J.\ of the European
Union L412 30.12.2006. P.~1--41.

     \bibitem{3za}
     \Au{Соломоник А.} Парадигма семиотики.~--- Минск: МЕТ, 2006.

     \bibitem{4za}
     \Au{Кун Т.}
     Структура научных революций.~--- М.: АСТ, 2001.

     \bibitem{5za}
     \Au{Турчин В.\,Ф.}
     Феномен науки: кибернетический подход к эволюции.~--- М.: Наука, 1993.

     \bibitem{6za}
     \Au{Brookes~B.\,C.}
      The foundations of information science. Part~I. Philosophical aspects~// J.\  Information
Science, 1980. No.\,2. P.~125--133.

     \bibitem{7za}
     CORDIS ICT Programme Home~--- {\sf http://cordis.}\linebreak {\sf europa.eu/ fp7/ict/programme/home\_en.html} (состояние страницы на 27.07.2007).

     \bibitem{8za}
     ICT FP7 Work Programme~---
     {\sf ftp://ftp.cordis.europa.eu/ pub/fp7/ict/docs/ict-wp-2007-08\_en.pdf} (состояние файла на
27.07.2007).

     \bibitem{9za}
     \Au{Мамардашвили М.}
     Классический и неклассический идеалы рациональности.~--- М.: Логос, 2004. 240~с.

     \bibitem{10za}
     Computational science: Ensuring America's competitiveness. Report to the President.
Arlington, VA: National Coordination Office for Information Technology Research and Development,
2005.

     \bibitem{11za}
     \Au{Колин К.\,К.}
     Новая стратегическая компьютерная инициатива США и задачи России в области
развития фундаментальной информатики~// Информационные технологии, 2006. №\,7. С.~2--5.

     \bibitem{12za}
     \Au{Gorn S.}
     Informatics (computer and information science): Its ideology, methodology, and sociology~//
The studies of information: Interdisciplinary messages~/ Ed. by F.~Machlup and U.~Mansfield.~---
New York: Wiley, 1983. P.~121--140.

     \bibitem{13za}
     \Au{Ingwersen P.}
     Information and information science //~Encyclopaedia of library and information science~/
Ed.\ by A.~Kent.~---
New York: Marcel Dekker Inc., 1995. Vol.~56, sup.~19. P.~137--174.

     \bibitem{15za}
     \Au{Nonaka I., Takeuchi H.}
     The knowledge-creating company.~--- N.Y.: Oxford University Press, 1995. (Перевод: 
     \Au{Нонака И., Такеучи~Х.}
     Компания~--- создатель знания.~--- М.: ЗАО <<Олимп-Бизнес>>, 2003. 384~с.)

     \bibitem{14za}
     \Au{Колин К.\,К.}
     Фундаментальные основы информатики: социальная информатика.~--- М.:
Академический проект, 2000. 350~с.

     \bibitem{16za}
     \Au{Farradane~J.}
     Knowledge, information, and information science~// J.\ Information Science, 1980, No.\,2.
     P.~75--80.

     \bibitem{17za}
     \Au{Арский Ю.\,М., Гиляревский~Р.\,С., Туров~И.\,С., Черный~А.\,И.}
     Инфосфера: информационные структуры, системы и процессы в науке и обществе.~--- М.:
\mbox{ВИНИТИ}, 1996. 489~с.

     \bibitem{18za}
     Информатика как наука об информации: ин\-фор\-ма\-ци\-он\-ный, документальный,
технологический, эконо\-ми\-че\-ский, социальный и организационный
аспекты~/ Под ред.
Р.\,С.~Гиляревского.~--- М.: ФАИР-ПРЕСС, 2006.

     \bibitem{19za}
     \Au{Кибрик А.\,Е.}
     Язык~// Языкознание: Большой эн\-цик\-ло\-пе\-ди\-че\-ский словарь.~--- М.: Большая Российская
энциклопедия, 1998. С.~604--606.

     \bibitem{20za}
     Vossen~P., ed.
     EuroWordNet General Document (Version~3) (URL: {\sf
http://www.illc.uva.nl/EuroWordNet/ docs/GeneralDoc}).

     \bibitem{21za}
     \Au{Барт Р.}
     Основы семиологии~// Французская семиотика: От структурализма к
постструктурализму.~--- М.: Прогресс, 2000. С.~247--310.

     \bibitem{22za}
     \Au{Hjorland B.}
     Library and information science: Practice, theory, and philosophical basis~// Information
Processing and Management, 2000. No.\,36. P.~ 501--531.

     \bibitem{23za}
     Digital imaging and communications in medicine (\mbox{DICOM}). 
Part~16: Content mapping
resource.~--- Rosslyn, Virginia: National Electrical Manufacturers Association, 2004.

     \bibitem{24za}
     Рекомендации по преподаванию программной инженерии и computer science в
университетах (Software Engineering 2004: Curriculum Guidelines for 
Un\-der\-grad\-u\-ate Degree Programs in Software Engineering; Com\-puting 
Curricula 2001: Computer Science)~/ Пер. с англ.~--- М.: 
Ин\-тер\-нет-Уни\-вер\-си\-тет Информационных Технологий, 2007. 462~с.

     \bibitem{25za}
     \Au{Эббинхаус Г.\,Д.}
     Машины Тьюринга и вычислимые функции~I. Уточнение понятия алгоритма~// Машины
Тьюринга и рекурсивные функции.~--- М.: Мир, 1972.

     \bibitem{26za}
     \Au{Полунов Ю.} Автора!!!~// PC Week/RE, 2006. №\,20--21.

     \bibitem{27za}
     \Au{Hilbert D., Ackermann~W.}
     Grundz$\ddot{\mbox{u}}$ge der Theoretischen Logik.~--- Berlin, 1931.

     \bibitem{28za}
     \Au{Turing A.\,M.}
     On computable numbers, with an application to the Entscheidungsproblem~// November,
No.\,12, 1936  ({\sf http://www.abelard.org/turpap2/tp2-ie.asp}).

     \bibitem{29za}
     \Au{Винер Н.}
     Кибернетика, или управление и связь в животном мире.~--- М.: Наука, 1983.

     \bibitem{30za}
     \Au{Shannon C.\,E.}
     A mathematical theory of communication~// The Bell System Technical J., July, October, 1948.
Vol.~27. P.~379--423, 623--656.

     \bibitem{31za}
     \Au{Калман Р.\,Е.}
     Об общей теории систем управления~// Труды ИФАК, 1961. Т.~2. ~--- М.: Изд-во АН
СССР.  С.~521--547.

     \bibitem{32za}
     \Au{Пугачев В.\,С.}
     Основы автоматического управления.~--- М.: Наука, 1974.

     \bibitem{33za}
     \Au{Пугачев В.\,С.}
     Теория вероятности и математическая статистика.~--- М.: Наука, 2002.

     \bibitem{34za}
     \Au{Зацман И.\,М., Шубников~С.\,К.}
     Принципы обработки информационных ресурсов для оценки ин\-но\-ваци\-он\-но\-го потенциала
направлений научных исследований~// Тр.\ 9-й Всероссийской научной конференции
<<Электронные библиотеки: перспективные методы и технологии, электронные
     коллекции>>~--- RCDL'2007 (Переславль, 15--18~октября 2007~г.).~--- Переславль:
     Изд-во <<Университет города Пе\-ре\-слав\-ля>>, 2007. С.~35--44.

     \bibitem{35za}
     \Au{Gorn S.}
     The individual and political life of information systems~// Proc. Symposium on Education for
Information Science.~--- New York: Spartan Books, 1965. P.~33--40.

     \bibitem{36za}
     \Au{Gorn S.}
     Computer and information sciences and the community of disciplines~// Behavioral Science,
November, 1967. Vol.~12. No.\,6. P.~433--452.

     \bibitem{37za}
     \Au{Gorn S.}
     The identification of the computer and information sciences: Their fundamental semiotic concepts
and relationships~// Foundations of Language, November, 1968. Vol.~4. No.\,4. P.~339--372.

     \bibitem{38za}
     \Au{Шрейдер~Ю.\,А.}
     ЭВМ как средство представления знаний~// Природа, 1986. №\,10. С.~14--22.
\label{end\stat}

     \bibitem{39za}
     \Au{Шрейдер Ю.\,А.}
     Информация и знание~// Системная концепция информационных процессов.~--- М.:
\mbox{ВНИИСИ}, 1988. С.~47--52.

\end{thebibliography}

}
}

\end{multicols}