\def\yhxt{({\hat X}_t,Y_t, t)}
\def\hx{\hat X}
\def\yutt{(Y_t, {\hat X}_t,t)}
\def\yxtt{(X_t,Y_t, t)}
\renewcommand\mm{{\sf M}}


\def\stat{SIN-2}

\def\tit{НОРМАЛЬНЫE  УСЛОВНО-ОПТИМАЛЬНЫЕ
ФИЛЬТРЫ И~ЭКСТРАПОЛЯТОРЫ ПУГАЧЁВА
 ДЛЯ~СТОХАСТИЧЕСКИХ СИСТЕМ, 
 ЛИНЕЙНЫХ ОТНОСИТЕЛЬНО СОСТОЯНИЯ}

\def\titkol{Нормальныe  условно-оптимальные
фильтры и~экстраполяторы Пугачёва
 для стохастических систем} %,  линейных относительно состояния}

\def\aut{И.\,Н.~Синицын$^1$, Э.\,Р.~Корепанов$^2$}

\def\autkol{И.\,Н.~Синицын, Э.\,Р.~Корепанов}

\titel{\tit}{\aut}{\autkol}{\titkol}

\index{Синицын И.\,Н.}
\index{Корепанов Э.\,Р.}
\index{Sinitsyn I.\,N.}
\index{Korepanov E.\,R.}

%{\renewcommand{\thefootnote}{\fnsymbol{footnote}} \footnotetext[1]
%{Исследование частично поддержано РФФИ (проекты 16-07-00677 
%и~15-37-20611-мол\_а\_вед).}}


\renewcommand{\thefootnote}{\arabic{footnote}}
\footnotetext[1]{Институт проблем информатики Федерального исследовательского
центра <<Информатика и~управление>> Российской академии наук, sinitsin@dol.ru}
\footnotetext[2]{Институт проблем информатики Федерального исследовательского
центра <<Информатика и~управление>> Российской академии наук, ekorepanov@ipiran.ru}


\Abst{Рассматривается теория аналитического синтеза непрерывных (дифференциальных) 
и~дискретных (разностных) суб-  и~услов\-но-оп\-ти\-маль\-ных фильтров и~экстраполяторов 
Пугачёва для обработки процессов в~гауссовских и~негауссовских стохастических
 системах (СтС), 
линейных относительно вектора состояния. Первые работы по фильтрации и~экстраполяции 
для таких гауссовских систем были выполнены Липцером и~Ширяевым, а для негауссовских~--- 
Пугачёвым и~Синицыным. Приведены алгоритмы нормальных суб- 
и~услов\-но-оп\-ти\-маль\-ных фильтров для непрерывных
и~дискретных систем. Пред\-став\-ле\-ны алгоритмы нормальных суб- и~услов\-но-оп\-ти\-маль\-ных 
экстраполяторов. Разработанные алгоритмы положены в~основу программного обеспечения 
(StS-Filter, 2016). Результаты допускают развитие на случай автокоррелированных шумов 
в~наблюдениях, а~так\-же систем с~мультипликативными шумами.}

\KW{дискретные СтС; дифференциальная СтС;
метод нормальной аппроксимации (МНА) апостериорной плотности;
метод статистической линеаризации (МСЛ);
нормальный услов\-но-оп\-ти\-маль\-ный фильтр  Пугачёва (НФП);
нормальный услов\-но-оп\-ти\-маль\-ный экстраполятор Пугачёва (НЭП);
стохастическая система (СтС); СтС, линейная относительно состояния;
условия Лип\-це\-ра--Ширяева; фильтр Лип\-це\-ра--Ши\-ряева (ФЛШ)}

\DOI{10.14357/19922264160202} 

\vspace*{-6pt}

\vskip 10pt plus 9pt minus 6pt

\thispagestyle{headings}

\begin{multicols}{2}

\label{st\stat}

\section{Введение}

Многие практические задачи обработки информации в~статистических научных 
исследованиях основаны на использовании теории фильтрации процессов в~СтС, 
линейных относительно состояния~[1--5]. 

Первые работы в~этом 
направлении для гауссовских СтС выполнены Липцером и~Ширяевым~\cite{2-s2}, 
а~для негауссовских СтС на основе субоптимальной фильтрации~--- 
Пугачёвым и~Синицыным~\cite{3-s2, 4-s2}.
В~\cite{1-s2} рассмотрены вопросы синтеза алгоритмов нормальных 
услов\-но-оп\-ти\-маль\-ных 
фильтров  Пугачёва (НФП) для обработки информации в~дифференциальных 
гауссовских и~негауссовских СтС, линейных относительно состояния 
(условия Лип\-це\-ра--Ши\-ря\-ева). Особое внимание уделено синтезу НФП для СтС 
при условиях Лип\-це\-ра--Ши\-ря\-ева
 на основе аппроксимации апостериорного распределения нормальным субоптимальным 
 квазилинейным НФП, основанным на статистической линеаризации нелинейных функций, 
 зависящих от наблюдений. Для СтС высокой размерности  путем выбора структурных 
 функций, отражающих аналитическую природу наблюдаемой системы, синтезированы НФП, 
 прос\-тые в~компьютерной реализации и~ способные работать в~режиме реального времени. 
 Алгоритмы положены 
 в~основу модуля инструментального программного обеспечения (StS-Filter, 2016).

Настоящая статья  посвящена вопросам синтеза непрерывных и~дискретных нормальных 
фильтров и~экстраполяторов Пугачёва (НФП и~НЭП) для непрерывных и~дискретных СтС, 
линейных относительно состояния. 
В~разд.~2 приведены уравнения,\linebreak описывающие 
процессы дифференциальных и~разностных СтС. Раздел~3 для гауссовских СтС посвящен 
нормальным фильтрам (НФ) на основе мето-\linebreak дов нормальной 
аппроксимации (МНА) и~метода\linebreak 
статистической линеаризации (МСЛ). В~разд.~4 приводятся алгоритмы НФП на основе 
тео\-рии услов\-но-оп\-ти\-маль\-ной фильтрации Пугачёва. Для дифференциальных 
негауссовских СтС алгоритмы НФП даны в~разд.~5. Обобщение НФ для дискретных СтС 
приведено в~разд.~6. Алгоритмы нормальных услов\-но-оп\-ти\-маль\-ных 
экстраполяторов Пугачёва представлены в~разд.~7. В~заключении приведены основные 
выводы и~возможные обобщения.

\section{Непрерывные и~дискретные стохастические системы, линейные относительно состояния}

Рассмотрим сначала нелинейную дифференциальную СтС~\cite{1-s2}:
  \begin{alignat}{2}
  \dot X_t &=\vrp \left(X_t, Y_t, t\right) + \psi \left(X_t, Y_t, t\right) V\,,
  &\enskip 
  X_{t_0} & = X_0\,;\label{e2.1-s2}\\
  \dot Y_t &=\vrp_1 \left(X_t, Y_t, t\right) + \psi_1 \left(X_t, Y_t, t\right) V
  \,,&\enskip 
  Y_{t_0}& = Y_0\,.\label{e2.2-s2}
  \end{alignat}
 Здесь $X_t$ и~$Y_t$~--- векторы состояния и~наблюдения размерности~$n_x$ и~$n_y$; 
 $V\hm= \dot W$, $W$~--- векторный стохастический процесс (СтП) с~независимыми 
 приращениями, состоящий из винеровской $W_0(t)$ и~пуассоновской частей:
\begin{equation}
\left.
\begin{array}{rl}
W&= \displaystyle W_0 (t) +\iii_{R_0^q} c(\rho) P^0 (t, d\rho)\,;\\[6pt]
\nu^W &=\displaystyle \nu^{W_0} + \iii_{R_0^q} c(\rho) c(\rho)^{\mathrm{T}} 
\nu_P (t, \rho) \,d\rho\,, 
\end{array}
\right\}
\label{e2.3-s2}
\end{equation}
где $c=c(\rho)$~--- векторная функция (той же размерности~$q$, что и~$W$) 
аргумента~$\rho$, а~интеграл при\linebreak любом $t\hm\ge t_0$ представляет собой 
стохастический интеграл по центрированной пуассоновской мере  $P^0 (t, A)$, 
независимой от~$W_0$  и~име\-ющей независимые значения на попарно 
непересе\-ка\-ющих\-ся множествах;  $A$~--- борелевское множество\linebreak
 пространства~$R_0^q$ 
с~выколотым началом~0; $\nu^W$, $\nu^{W_0}$ и~$\nu_P$~--- интенсивности~$W$,
$W_0$ и~$P^0$;
$\vrp\hm=\vrp (X_t, Y_t, t)$, $\psi\hm=\psi (X_t, Y_t, t)$, 
$\vrp_1\hm=\vrp_1 (X_t, Y_t, t)$ и~$\psi_1\hm=\psi_1 (X_t, Y_t, t)$~--- 
известные функции раз\-мер\-ности $(n_x\times 1)$, $(n_x\times n_v)$, $(n_y \times 1)$
и~$(n_y\times n_v)$, удовлетворяющие следующим условиям Лип\-це\-ра--Ши\-ря\-ева~\cite{2-s2}:
\begin{itemize}
\item функции $\vrp$ и~$\vrp_1$ линейны относительно состояния~$X_t$:
        \begin{equation}
        \left.
        \begin{array}{rl}
        \vrp \left(X_t, Y_t, t\right) &= a_1 \left(Y_t, t\right) X_t + a_0 \left(Y_t, t\right) \,;\\[6pt]
        \vrp_1 \left(X_t, Y_t, t\right) &= b_1 \left(Y_t, t\right) X_t +
         b_0 \left(Y_t, t\right) \,;
         \end{array}
         \right\}
         \label{e2.4-s2}
         \end{equation}

\item функции $\psi$ и~$\psi_1$  не зависят от состояния~$X_t$:
\begin{multline}
\hspace*{-1pt}\psi \left(X_t, Y_t, t\right) = \bar\psi \left(Y_t, t\right) ;\  
\psi_1 \left(X_t, Y_t, t\right) ={}\\
{}= \bar\psi_1  \left(Y_t, t\right).
\label{e2.5-s2}
\end{multline}
    \end{itemize}

Предполагается, что уравнения СтС~(\ref{e2.1-s2}) и~(\ref{e2.2-s2}) 
понимаются в~смысле Ито и~имеют решение в~среднем квадратическом (с.к.)~\cite{3-s2}.

Систему~(\ref{e2.1-s2})--(\ref{e2.5-s2}) будем называть гауссовской, 
если  $V\hm=\dot W_0$, а~$X_0$ и~$Y_0$~--- гауссовские случайные величины (с.в.).

Важный частный случай~(\ref{e2.1-s2})--(\ref{e2.5-s2}) 
составляют уравнения с~аддитивными шумами, когда
 \begin{equation}
 \bar\psi\left(Y_t, t\right) = \psi_0 (t)\,;\enskip 
 \bar\psi_1 \left(Y_t, t\right) = \psi_{10} (t)\,.
 \label{e2.6-s2}
 \end{equation}

Для случая, когда в~уравнения~(\ref{e2.1-s2}), (\ref{e2.2-s2}) 
входят независимые белые шумы~$V_1$ и~$V_2$, следует принять:
 \begin{alignat*}{2}
% \left.
% \begin{array}{c}
 V&= \displaystyle\lk V_1^{\mathrm{T}} V_2^{\mathrm{T}}\rk^{\mathrm{T}}\,,&\enskip 
 \nu&=\begin{bmatrix}
    \nu_1&0\\
    0&\nu_2\end{bmatrix}\,;\\[6pt] 
    \psi V &=\psi' V_1\,,&\enskip \psi_1 V &= \psi_1' V_2\,.
    %\end{array}
%    \right\}
%    \label{e2.7-s2}
    \end{alignat*}

Для дискретных СтС, линейных относительно состояния, уравнения состояния и~наблюдения 
имеют следующий вид~\cite{3-s2}:
\begin{align}
X_{k+1}&=\vrp_k \left(X_k, Y_k\right)+ \psi_k' \left(X_k, Y_k\right) V_{k}^d= {}\notag\\
&{}=a_k \left(Y_k\right)X_k + a_{0k} \left(Y_k\right) + c_{0k}\left(Y_k\right)  
V_{k}^d\,,\notag\\ 
&\hspace*{40mm}X_1 = X_{10}\,;\label{e2.8-s2}\\
Y_{k}&=\vrp_{1k} \left(X_k, Y_k\right)+ \psi_{1k} \left(X_k, Y_k\right) V_{k}^d={}\notag\\ 
&{}=b_k \left(Y_k\right)X_k + b_{0k} \left(Y_k\right) + l_{0k} \left(Y_k\right) 
 V_{k}^d\,,\notag\\
 & \hspace*{40mm}Y_1 = Y_{10}\,.\label{e2.9-s2}
 \end{align}
Здесь $n=n_x\hm=n_y$; $V_k^d$~--- дискретный белый шум с~известным 
в~общем случае негауссовским распределением.

Для дискретных СтС с~аддитивными шумами следует в~(\ref{e2.8-s2}), (\ref{e2.9-s2}) принять:
\begin{equation*}
c_{0k} \left(Y_k\right) = c_{0k0}\,; \enskip l_{0k} \left(Y_k\right) = l_{0k0}\,.
%\label{e2.10-s2}
\end{equation*}

\section{Дифференциальные гауссовские стохастические системы. 
Фильтры Липцера--Ширяева и~их~субоптимальная аппроксимация}

Для гауссовской СтС~(\ref{e2.1-s2})--(\ref{e2.5-s2})  
известны следу\-ющие точные уравнения нелинейной фильтрации по критерию минимума с.к.\ 
ошибки~\cite{2-s2, 3-s2}:
\begin{multline}
{\dot{\hat X}}_t= \left[a_1 \left(Y_t,t\right) \hat X_t + a_0 \left(Y_t,t\right)\right] +{}\\
{}+ \left[R_tb_1 \left(Y_t,t\right)^{\mathrm{T}} +
\left(\bar \psi\nu_0\bar \psi_1^{\mathrm{T}}\right)
    \left(Y_t,t\right)\right]\times{}\\
{}\times \left(\bar \psi_1\nu_0\bar \psi_1^{\mathrm{T}}\right)^{-1} \left(Y_t,t\right) 
\!\left\{ \dot Y_t -\left[ b_1\left(Y_t,t\right) \hat X_t +b_0\left(Y_t,t\right)\right] 
\!\right\},\hspace*{-0.84221pt}\\ 
    \hat X_{t_0} = \hat X_0\,;\label{e3.1-s2}
    \end{multline}
    
    
    \noindent
\begin{multline}
\dot R_t =  a_1\left(Y_t,t\right) R_t + R_t a_1 
\left(Y_t,t\right)^{\mathrm{T}} +{}\\
{}+\left(\bar \psi\nu_0\bar \psi^{\mathrm{T}}\right) 
\left(Y_t,t\right) -
    \left[ R_t b_1 \left(Y_t,t\right)^{\mathrm{T}}+{}\right.\\
\left.{}+\left(\bar \psi\nu_0\bar \psi_1^{\mathrm{T}}\right) \left(Y_t,t\right)
\vphantom{\left(Y_t,t\right)^{\mathrm{T}}}
\right] 
\left(\bar\psi_1\nu_0\bar\psi_1^{\mathrm{T}}\right)^{-1} \left(Y_t,t\right) \times{}\\
\!\!\!\!{}\times
\left[ b_1 \left(Y_t,t\right) R_t+ \left(\bar \psi_1 \nu_0 \bar \psi^{\mathrm{T}}\right)
    \left(Y_t,t\right)\right]\,,\enskip R_{t_0} = R_0\,,
\!    \label{e3.2-s2}
    \end{multline}
где $\hx_t$~--- с. к.\ оценка; $R_t$~--- ковариационная матрица ошибки 
фильтрации $X_t\hm- \hx_t$.

Таким образом, имеет место утверждение.

\smallskip

\noindent
\textbf{Теорема~3.1.}\
\textit{Пусть в~гауссовской системе}~(\ref{e2.1-s2})--(\ref{e2.5-s2}) 
\textit{диффузионная матрица  $\sigma_1\hm= \sigma_1 (Y_t, t) \hm=
\bar \psi_1\nu_0\bar \psi_1^{\mathrm{T}} (Y_t, t)$ не вырождена. Тогда алгоритм с.к.\ 
оптимального фильтра определяется уравнением}~(\ref{e3.1-s2}), 
\textit{а~его точность оценивается согласно}~(\ref{e3.2-s2}).

\smallskip

Как и~в случае линейной фильтрации при аддитивных шумах~\cite{3-s2}, 
уравнения дифференциального фильтра Лип\-це\-ра--Ши\-ря\-ева (ФЛШ)  
пред\-став\-ля\-ют собой замкнутую систему
уравнений,\linebreak опре\-де\-ля\-ющую~$\hat X_t$ и~$R_t$. Поэтому с.к.\  
оптимальную
оценку~$\hat X_t$ вектора состояния системы~$X_t$ и~его\linebreak
апостериорную ковариационную матрицу~$R_t$, ха\-рак\-те\-ри\-зу\-ющую
точность с.к.\ оптимальной оценки~$\hat X_t$, можно вычислять по
мере получения результатов наблюдений совместным интегрированием\linebreak
уравнений~(\ref{e3.1-s2}) и~(\ref{e3.2-s2}). Однако в~противоположность 
линейной фильт\-ра\-ции для ФЛШ нельзя вычислить~$R_t$ заранее, до
получения результатов наблюде\-ний, так как от последних зависят
коэффициенты уравнения~(\ref{e3.2-s2}). Поэтому ФЛШ 
в~данном случае должен выполнять интегрирование обоих уравнений~(\ref{e3.1-s2}) 
и~(\ref{e3.2-s2}). Это приводит к~существенному повышению
порядка оптимального фильтра. Если линейный фильтр  всегда
описывается уравнениями  порядка~$n_x$, то в~рассматриваемом более общем случае с.к.\
оптимальный фильтр описывается уравнениями порядка
\begin{equation}
Q_{\mathrm{ФЛШ}}=n_x+\fr{n_x(n_x+1)}{2}= \fr{n_x(n_x+3)}{2}\,.
\label{e3.3-s2}
\end{equation}

Очевидно, что ФЛШ будет совпадать с~обобщенным (приближенным) 
фильт\-ром Кал\-ма\-на--Бью\-си, 
фильтрами второго порядка, гауссовыми фильтрами~\cite{3-s2, 6-s2}.

Так как гауссовское (нормальное) распреде\-ление, аппроксимирующее
апостериорное рас\-пре\-деление вектора~$X_t$, полностью определяется
апостериорными математическим ожиданием~$\hat X_t$ 
и~ковариацион\-ной матрицей~$R_t$ вектора~$X_t$, то согласно 
тео\-рии нелинейной приближенной субоптимальной фильтрации при аппроксимации
апостериорного распределения вектора~$X_t$ нормальным
распределением будут иметь место следующие 
стохастические дифференциальные уравнения,
определяющие~$\hat X_t$ и~$R_t$~\cite{3-s2, 1-s2}:

\noindent
\begin{multline}
\dot{\hat X}_t = f \left(\hat X_t, Y_t,R_t,t\right)+{}\\
{}+
    h\left(\hat X_t,Y_t, R_t,t\right)\lk \dot Y_t - f^{(1)} \left(\hat X_t,Y_t,
    R_t,t\right)\rk\,;\label{e3.4-s2}
    \end{multline}
    
\vspace*{-12pt}

    \noindent
\begin{multline}
\dot R_t=\left\{ 
\vphantom{\left({\hat X}_t, Y_t,R_t,t\right)^{\mathrm{T}}}
f^{(2)}\left(\hat X_t, Y_t,R_t,t\right)-h\left(\hat
    X_t, Y_t,R_t,t\right)\times{}\right.\\
\left.{}{}\times\bar \psi_1\nu_0\bar \psi_1^{\mathrm{T}} 
    \left(Y_t,t\right)  
    h \left({\hat X}_t, Y_t,R_t,t\right)^{\mathrm{T}}\right\} +{}\\
   \hspace*{-3.6mm} {}+\!
\sss_{r=1}^{n_y} \rho_r \left({\hat X}_t,Y_t, R_t,t\right)\!\lk
    \dot Y_r -f_r^{(1)}\left({\hat X}_t,Y_t, R_t,t\right) \!\rk\!,\!\!\!\!
    \label{e3.5-s2}
    \end{multline}
где
   \begin{multline*}
    f\left(\hat X_t, Y_t,R_t,t\right)= \mm^N\lk \vrp \left(Y_t, X,t\right)\rk= {}\\
    {}=
    a_1 \left(Y_t, t\right) \hat X_t + a_0 \left(Y_t, t\right)\,;
  \end{multline*}
    
    
    \vspace*{-12pt}
    
    \noindent
    \begin{multline*}
    f^{(1)}\left(\hat X_t, Y_t,R_t,t\right)=\left\{ f_r^{(1)} 
    \left( \hat X_t, Y_t, R_t, t\right)\right\}={}\\
    {}=
    \mm^N\lk \vrp_1\left(Y_t, X,t\right)\rk=b_1 \left(Y_t, t\right) \hat X_t+ 
    b_0 \left(Y_t, t\right)\,;
    \end{multline*}
    
    \vspace*{-12pt}
    
    \noindent
    \begin{multline*}
h\left(\hat X_t, Y_t,R_t,t\right)={}\\
{}=\mm^N \left\{
\vphantom{\hat X_t f^{(1)}\left(\hat X_t, Y_t,R_t,t\right)^{\mathrm{T}}}
\left[ X\varphi_1\left(
Y_t,X,t\right)^{\mathrm{T}} + \bar \psi\nu_0\bar \psi_1^{\mathrm{T}} 
\left(Y_t,X,t\right)\right]-{}\right.\\
\left.{}-
    \hat X_t f^{(1)}\left(\hat X_t, Y_t,R_t,t\right)^{\mathrm{T}}\right\}
\left(\bar \psi_1\nu_0\bar \psi_1^{\mathrm{T}}\right)^{-1} \left(Y_t,t\right)= {}\\
{}=
\lk R_t b_1 \left(Y_t, t\right)^{\mathrm{T}} + 
\left(\bar \psi\nu_0\bar \psi_1^{\mathrm{T}}\right)\left(Y_t, t\right)
\rk \times{}\\
{}\times\left(\bar \psi\nu_0\bar \psi_1^{\mathrm{T}}\right)^{-1} \left(Y_t, t\right)\,;
\end{multline*}

\vspace*{-12pt}
    
    \noindent
    \begin{multline*}
\hspace*{-2.52pt}f^{(2)}\left(\hat X_t, Y_t,R_t,t\right)=
\mm^N\biggl\{  \left(X-\hat X_t\right)\varphi\left(Y_t,X,t\right)^{\mathrm{T}} + {}\\
{}+
\varphi \left(Y_t,X,t\right) \left(X^{\mathrm{T}}-\hat X_t^{\mathrm{T}}\right) +
\bar\psi\nu_0\bar\psi^{\mathrm{T}} \left(Y_t,X,t\right)\biggr\}={}\\
{}=\lk 
R_t b_1 \left(Y_t, t\right)^{\mathrm{T}} + 
\left(\bar \psi\nu\bar \psi_1^{\mathrm{T}}\right)
\left(Y_t, t\right)\rk \times{}\\
{}\times
\left(\bar \psi_1\nu_0\bar \psi_1^{\mathrm{T}}\right)^{-1}\!\! \left(Y_t, t\right)
 \lk 
b_1 \left(Y_t, t\right) R_t + 
\left(\bar \psi_1\nu_0\bar \psi^{\mathrm{T}}\right)\left(Y_t, t\right)
\rk;\hspace*{-8.054pt}
\end{multline*}

\vspace*{-12pt}
    
    \noindent
    \begin{multline*}
    \rho_r\left(\hat X_t,Y_t, R_t,t\right)={}\\
    {}=
    \mm^N \biggl\{  
    \left(X-\hat X_t\right)\left(X^{\mathrm{T}}-\hat X_t^{\mathrm{T}}\right) 
    a_r \left(Y_t,X,t\right)+{}\\
{}+ \left(X-\hat X_t\right) b_r\left(Y_t,X,t\right)^{\mathrm{T}} 
\left(X^{\mathrm{T}}-\hat X_t^{\mathrm{T}}\right)+ {}\\
{}+
b_r \left(Y_t,X,t\right) \left(X^{\mathrm{T}}-\hat X_t^{\mathrm{T}}\right)\biggr\} =0 
\enskip (r=1\tr n_y)\,.
%\label{e3.6a-s2}
\end{multline*}
Здесь функции $\vrp$, $\vrp_1$, $\bar \psi$ и~$\bar \psi_1$ 
определены~(\ref{e2.4-s2}) и~(\ref{e2.5-s2}); 
$a_r$~--- $r$-й элемент мат\-ри\-цы-стро\-ки
    $A_{\vrp_1} \hm= (\vrp_1^{\mathrm{T}} \hm- \hat\vrp_n^{\mathrm{T}})(\bar \psi_1 \nu_0\bar \psi_1^{\mathrm{T}})^{-1};$
$B_{kr}^{\bar \psi\bar \psi_1}$~--- элемент $k$-й строки и~$r$-го столб\-ца матрицы
    $B^{\bar \psi\bar \psi_1} \hm= (\bar \psi\nu_0\bar \psi_1^{\mathrm{T}})
    (\bar \psi_1\nu_0\bar \psi_1^{\mathrm{T}})^{-1};$
$b_r$~--- $r$-й столбец матрицы $B^{\bar \psi\bar \psi_1}$,
    $b_r^{\bar \psi\bar \psi_1} \hm= \lk b_{1r}^{\bar \psi\bar \psi_1}\cdots 
    b_{n_x r}^{\bar \psi\bar \psi_1}\rk^{\mathrm{T}},$
$\mm^N$~--- символ математического ожидания по $X\hm=X_t$.

Число уравнений МНА одномерного апостериорного распределения
определяется по формуле~(\ref{e3.3-s2}).


За начальные значения $\hat X_t$ и~$R_t$  при интегрировании
уравнений~(\ref{e3.4-s2}) и~(\ref{e3.5-s2}), естественно, следует принять
условные математическое ожидание $\hat X_0$ и~ковариационную матрицу~$R_0$ с.~в.~$X_0$ 
относительно~$Y_0$:
\begin{align*}
%\left.
%\begin{array}{rl}
\hat X_0 &= \mm^N\lk X_0 \mid Y_0\rk\,;\\
R_0 &= \mm^N \lk \left(X_0 -\hat X_0\right) 
\left(X_0^{\mathrm{T}} -\hat X_0^{\mathrm{T}}\right)\mid
Y_0\rk\,.
%\end{array}
%\right\}
%\label{e3.6-s2}
\end{align*}
Если нет
информации об условном распределении~$X_0$ относительно~$Y_0$, то
начальные условия можно взять в~виде:
\begin{align*}
%\left.
%\begin{array}{rl}
\hat X_0 &= \mm^N  X_0\,; \\ %[6pt]
    R_0&= \mm^N  \left(X_0 - \mm^N X_0\right) 
    \left(X_0^{\mathrm{T}} - \mm^N X_0^{\mathrm{T}}\right)\,.
    %\end{array}
%    \right\}
%    \label{e3.7-s2}
    \end{align*}
Если же и~об этих величинах нет никакой информации, то начальные
значения $\hat X_t$ и~$R_t$ приходится задавать произвольно.

Сравнивая  уравнения~(\ref{e3.4-s2}), (\ref{e3.5-s2}) с~уравнениями ФЛШ 
(теорема~3.1), 
имеем следующий результат~\cite{1-s2}.

\smallskip

\noindent
\textbf{Теорема~3.2.}\
\textit{Для гауссовской СтС}~(\ref{e2.1-s2})--(\ref{e2.5-s2}) 
\textit{субоптимальные НФ на основе МНА для одномерной апостериорной плотности 
и~ФЛШ совпадают.}


\section{Дифференциальные гауссовские стохастические системы. Нормальные фильтры Пугачёва}

Следуя~\cite{3-s2, 1-s2}, будем искать фильтр для оценки~$\hat X_t$ в~виде 
следующего уравнения:
 \begin{multline}
 d\hx_t =\alp_t \xi \yhxt\, dt + {}\\
 {}+\beta_t\eta \yhxt\, dY_t +
    \gamma_t \,dt\,.
    \label{e4.1-s2}
    \end{multline}
Здесь  $\xi =\xi \yhxt$ и~$\eta\hm=\eta\yhxt$~--- некоторые известные 
структурные функции
текущих значений оценки наблюдаемого процесса~$Y_t$, $\hx_t$ и~времени~$t$; 
$\alp_t$, $\beta_t$ и~$\gamma_t$~--- неизвестные функции времени.
Если бы коэффициенты  $\alp_t$, $\beta_t$ и~$\gamma_t$ в~(\ref{e4.1-s2})
были известными функциями времени, то уравнение~(\ref{e4.1-s2}) определило
бы фильтр того же  порядка~${n_x}$, что и~ уравнение~(\ref{e2.1-s2}), 
описывающее поведение системы. Поэтому, естественно,
возникает мысль попытаться непосредственно определить коэффициенты~$\alp_t$, 
$\beta_t$ и~$\gamma_t$ в~уравнении~(\ref{e4.1-s2}) как функции
времени из условия минимума с.к.\ ошибки  при всех  $t\hm>t_0$. Это приводит к~теории
услов\-но-оп\-ти\-маль\-но\-го фильтра Пугачева (ФП), когда в~уравнения
ФП задаются заранее и~оптимизируются только коэффициенты этого уравнения.
Итак, приходим к~идее на\-хож\-де\-ния оптимального фильтра
в~некотором классе допустимых фильтров,
определяемом условием, что поведение фильтра описывается
дифференциальным уравнением заданного порядка и~заданной формы.
Таким образом, мы отказываемся от абсолютной оптимизации 
и~ограничиваемся условной оптимизацией в~заданном ограниченном классе фильтров.

Определив класс допустимых фильтров, следует решить вопрос о том,
какой фильтр в~этом классе считается оптимальным. Следуя~\cite{3-s2}, 
будем считать оптимальным такой фильтр, который дает 
в~известном смысле наилучшую оценку при всех $t\hm>t_0$. Иными словами, 
задача оптимизации фильтра при всех
$t\hm>t_0$ является  задачей многокритериальной оптимизации. Такие
задачи, как правило, не имеют решения. Фильтр Кал\-ма\-на--Бью\-си, да\-ющий
оптимальную линейную оценку состояния линейной системы в~каждый
момент $t\hm>t_0$, является исключением~\cite{3-s2}. Значит, надо
определить такую оптимальность фильтра, при которой возможно решение
задачи. Будем считать оптимальным
такой допустимый фильтр, который на каждом бесконечно малом
интервале времени совершает оптимальный переход из того состояния,
в котором он был в~начале этого интервала, в~новое состояние.
Такой допустимый фильтр будем называть услов\-но-оп\-ти\-маль\-ным. 
Тогда задачи фильтрации сведутся к~нахождению оптимальных значений~$\alp_t$, 
$\beta_t$ и~ $\gamma_t$, в~любой момент  $t\ge t_0$ обеспечивающих минимум
с.к.\ ошибки фильтрации в~бесконечно близкий будущий момент  $s\hm> t$, $s\hm\to t$.

Отметим, что ФП обладает тем свойством, что в~данном
классе допустимых фильт\-ров не существует фильт\-ра, который при
данном начальном распределении~$Y_t$, $X_t$ и~$\hx_t$ в~момент~$t_0$ 
был бы лучше услов\-но-оп\-ти\-маль\-но\-го при всех  $t\hm>t_0$. Это
значит, по терминологии теории многокритериальной оптимизации, что
ФП пред\-став\-ля\-ет собой один из мно\-жества
допустимых фильт\-ров~---  оптимальный по Парето~\cite{3-s2}.
Общая теория ФП по с.к.\ критерию развита для урав\-не\-ний~(\ref{e2.1-s2}), (\ref{e2.2-s2})
и подробно изложена в~\cite{3-s2}.
Теория ФП обладает двумя несомненными преимуществами по сравнению с~методами
субоптимальной фильт\-ра\-ции. Во-пер\-вых, она позволяет
получать фильт\-ры более низкого порядка и,~следовательно, более
простые в~реализации. Во-вто\-рых, она дает возможность получать
фильт\-ры не меньшей, а~при желании даже большей точности, чем
фильт\-ры, даваемые методами субоптимальной нелинейной фильт\-рации.

Применяя теорию ФП~\cite{3-s2} 
к~нормальным процессам в~гауссовской СтС~(\ref{e2.1-s2})--(\ref{e2.5-s2}) 
при  $V\hm=V_0$, придем к~нормальному ФП  вида~(\ref{e4.1-s2}). 
Входящие в~(\ref{e4.1-s2}) коэффициенты~$\alp_t$, $\beta_t$ и~$\gamma_t$ 
определяются следующими уравнениями:
    \begin{equation}
    \alp_t m_1 + \beta_t m_2 + \gamma_t=m_0\,;
    \label{e4.2-s2}
    \end{equation}
        \begin{equation}
m_0 = \mm^N  [\varphi]\,,\enskip m_1 = \mm^N  [\xi]\,,\enskip m_2 = 
    \mm^N [\eta]\,;
    \label{e4.3-s2}
    \end{equation}
    \begin{equation}
    \beta_t =\kappa_{02} \kappa_{22}^{-1};\label{e4.4-s2}
    \end{equation}
    
    \vspace*{-12pt}
    
    \noindent
    \begin{multline}
    \kappa_{02} = \mm^N \left[\left(X_t - \hx_t\right) 
    \left(a_1 X_t + a_0\right)^{\mathrm{T}} \eta^{\mathrm{T}}\right]+{}\\
    {}+ \mm^N 
    \left[\bar \psi  \nu_0\bar \psi_1^{\mathrm{T}} \eta^{\mathrm{T}} \right];
    \label{e4.5-s2}
    \end{multline}
    \begin{equation}
    \kappa_{22} = \mm^N \lk\eta\bar\psi_1\nu_0 \bar \psi_1^{\mathrm{T}} 
    \eta^{\mathrm{T}}\rk;\label{e4.6-s2}
    \end{equation}
    
    
    \vspace*{-9pt}
    
    \noindent
    \begin{multline}
\alp_t\kappa_{11} + \mm^N \lk \left(\hx_t-X_t\right)\left(\xi^{\mathrm{T}} 
\alp_t^{\mathrm{T}} +\gamma_t^{\mathrm{T}}\right)
    \fr{\partial \xi^{\mathrm{T}}}{\partial \hat X_t}\rk={}\\
    {}=
    \kappa_{01}'-\beta_t\kappa_{21}';\label{e4.7-s2}
    \end{multline}
    \begin{equation}
    \kappa_{21}' = \mm^N \left\{\lk \eta\left(b_1X_t+b_0\right)-m_2\rk 
    \xi\right\}^{\mathrm{T}};\label{e4.8-s2}
    \end{equation}
    
    \vspace*{-9pt}
    
    \noindent
    \begin{multline}
\kappa_{01}' =\kappa_{01} + \mm^N \left[ \left(X_t-\hx_t\right)  
\fr{\partial \xi^{\mathrm{T}}}{\partial t} \right]+{}\\
{}+
     \mm^N \Bigl\{ \left(X_t-\hx_t\right) \left(b_1 X_t+b_0\right)^{\mathrm{T}} +{}\\
     {}+
    \bar \psi\nu_0\bar \psi_1^{\mathrm{T}} - \beta_t\eta\bar \psi_1\nu_0\bar 
    \psi_1^{\mathrm{T}}\Bigr\} \left(
    \fr{\partial}{\partial Y_t}+\eta^{\mathrm{T}}\beta_t^{\mathrm{T}} 
    \fr{\partial}{\partial \hat X_t}\right)\xi^{\mathrm{T}}+{}\\
{}+\fr{1}{2}\, \mm^N\left[ \left(X_t-\hx_t\right)\times{}\right.\\
{}\times \biggl\{ \mathrm{tr} 
\biggl[ \bar \psi_1\nu_0\bar \psi_1^{\mathrm{T}}
    \left( \fr{\partial}{\partial Y_t}+2 \eta^{\mathrm{T}}\beta_t^{\mathrm{T}} 
    \fr{\partial}{\partial \hat X_t}\right)
    \fr{\partial^{\mathrm{T}}}{\partial Y_t}\biggr]+{}\\
{}+\left.\mathrm{tr}\left[ \beta_t\eta\bar \psi_1\nu_0\bar \psi_1^{\mathrm{T}}
\eta^{\mathrm{T}}\beta_t^{\mathrm{T}} \fr{\partial }{\partial \hat X_t}
\,\fr{\partial^{\mathrm{T}}}{\partial \hat X_t}\right]\biggr\} 
\xi^{\mathrm{T}}\right] \,,\label{e4.9-s2}
\end{multline}

\vspace*{-8pt}

\begin{equation}
\left.
\begin{array}{rl}
\kappa_{11} &= \mm^N \lf \lk \xi - m_1\rk \xi^{\mathrm{T}}\rf;\\[6pt]
    \kappa_{21} &= \mm^N \lf\lk b_1 X_t + b_0  - m_2\rk \xi^{\mathrm{T}} \rf;\\[6pt]
\kappa_{01} &= \mm^N\lf \lk a_1 X_t + a_0  - m_0\rk \xi^{\mathrm{T}} \rf.
\end{array}
\right\}
\label{e4.10-s2}
\end{equation}

Точность НФП определяется уравнением:
\begin{multline}
\dot R_t = \mm^N \left[\left( X_t -\hx_t\right) 
\left(a_1 X_t + a_0\right)^{\mathrm{T}} +{}\right.\\
{}+\left(a_1 X_t + a_0\right)  
\left(X_t^{\mathrm{T}} -\hx_t^{\mathrm{T}}\right)-{}\\
{}-\beta_t \eta \yutt \bar\psi_1 \nu_0 \bar\psi_1^{\mathrm{T}} 
\eta \yutt^{\mathrm{T}} \beta_t^{\mathrm{T}} +{}\\
\left.{}+\bar\psi \nu_0 \bar\psi^{\mathrm{T}}
\vphantom{\left(a_1 X_t + a_0\right)^{\mathrm{T}}}
\right],\enskip R_{t_0} = R_0\,.
\label{e4.11-s2}
\end{multline}

Таким образом, справедливо следующее утверж\-де\-ние~\cite{1-s2}.

\smallskip

\noindent
\textbf{Теорема~4.1.}\
\textit{Пусть для гауссовской системы}~(\ref{e2.1-s2}), (\ref{e2.2-s2}) 
\textit{при условиях Лип\-це\-ра--Ши\-ря\-ева выполнены условия невырожденности 
матрицы  $\kappa_{22}$}~(\ref{e4.6-s2}) 
\textit{и конечности величин~$\kappa_{ij}$ $(i,j\hm=0,1,2)$. Тогда алгоритм НФП 
определяется уравнением}~(\ref{e4.1-s2}),  
\textit{а~коэффициенты~$\alp_t$, $\beta_t$ и~$\gamma_t$}~--- 
(\ref{e4.2-s2})--(\ref{e4.11-s2}).

\smallskip


Отметим, что в~качестве услов\-но-оп\-ти\-маль\-ных НФП могут служить 
субоптимальные фильтры, определяемые теоремой~3.2.


\section{Дифференциальные негауссовские стохастические системы. 
Нормальный  фильтр Пугачёва}

Для негауссовской СтС в~уравнениях теоремы~4.1 следует заменить~$\nu_0$ 
на~$\nu$ согласно~(\ref{e2.3-s2}), а~в~выражении~(\ref{e4.9-s2}) 
учесть два дополнительных интегральных члена:
\begin{multline}
\kappa_{01}'  =\kappa_{01} + \mm^N  \left[ \left(X_t - \hat X_t\right) 
\fr{\partial\xi^{\mathrm{T}}}{\partial t} \right] +{}\\
{}+
\mm^N  \biggl\{ 
\left(X_t -\hat X_t\right) 
\bigg[ \vrp_1^{\mathrm{T}} -\iii_{R_0^q} c(\rho)^{\mathrm{T}} \nu_P (t,\rho)\, d\rho 
\psi_1^{\mathrm{T}} \bigg] +{}\\
{}+ \bar\psi \nu \bar\psi_1^{\mathrm{T}} - \beta_t \eta \bar \psi_1 
\nu\bar\psi_1^{\mathrm{T}} \biggr\} \left( 
\fr{\partial}{\partial Y_t} + \eta^{\mathrm{T}} \beta_t^{\mathrm{T}} 
\fr{\partial}{\partial \hat X_t}\right) \xi^{\mathrm{T}}+ {}\\
{}+
\fr{1}{2}\, \mm^N \biggl\{ \lk \left(X_t -\hat X_t\right)\rk \times{}\\
{}\times\biggl\{ 
\mathrm{tr}\, \lk \bar\psi_1 \nu \bar\psi_1^{\mathrm{T}} \left( 
\fr{\partial}{\partial Y_t} + 2 \eta^{\mathrm{T}} \beta_t 
\fr{\partial}{\partial \hat X_t} \right) 
\fr{\partial^{\mathrm{T}}}{\partial Y_t} \rk+{}\\
{}+
   \mathrm{tr}\, \lk 
   \beta_t \eta \bar\psi_1 \nu\bar\psi_1^{\mathrm{T}} \eta^{\mathrm{T}} 
   \beta_t^{\mathrm{T}} \fr{\partial}{\partial \hat X_t}\,
   \fr{\partial^{\mathrm{T}}}{\partial \hat X_t} \rk 
   \biggr\} \xi^{\mathrm{T}} \biggr\}+{}\\
{}+ \mm^N \biggl\{ \iii_{R_0^q}  \big[ 
X_t -\hat X_t + 
\left(\bar\psi -\beta_t \eta \bar\psi_1\right) c (\rho)\times{}\\
{}\times \xi 
\left(Y_t +\bar\psi c(\rho), \hat X_t + \beta_t \eta \bar\psi_1 c(\rho), t\right) -{}\\
{}-
\xi^{\mathrm{T}}
\big]^{\mathrm{T}} \nu_P (t, d\rho) d\rho\biggr\}\,.
\label{e5.1-s2}
\end{multline}
Здесь функции $\vrp$, $\vrp_1$, $\bar\psi$ и~$\bar\psi_1$ удовлетворяют 
условиям~(\ref{e2.4-s2})--(\ref{e2.5-s2}). В~результате имеем следующее утверждение.

\smallskip

\noindent
\textbf{Теорема~5.1.} 
\textit{Пусть для негауссовской СтС}~(\ref{e2.1-s2})--(\ref{e2.5-s2}) 
\textit{матрица  $\kappa_{22}$}~(\ref{e4.5-s2}) \textit{не вырождена, 
а~интегралы}~(\ref{e4.2-s2}), (\ref{e4.3-s2}), (\ref{e4.6-s2}), (\ref{e4.8-s2}), 
(\ref{e4.10-s2}) \textit{и}~(\ref{e5.1-s2}) \textit{конечны.
Тогда алгоритм НФП определяется уравнением}~(\ref{e4.1-s2}), 
\textit{а коэффициенты}~$\alp_t$, $\beta_t$ и~$\gamma_t$~--- 
(\ref{e4.2-s2})--(\ref{e4.10-s2}).

\smallskip

Теория НФП  не позволяет получить нормальные с.к.\ оптимальные
фильтры. Можно получить только ФП,
которые в~общем случае хуже с.к.\ оптимальных, но зато легко реализуемы.
Однако если с.к.\ оптимальная оценка~$\hx_t$ вектора~$X_t$
удовлетворяет уравнению допустимого фильтра~(\ref{e4.1-s2}) 
при ка\-ких-ли\-бо коэффициентах времени~$\alp_t$,
$\beta_t$ и~$\gamma_t$, то уравнения теорем~4.1 и~5.1, 
конечно, определяют именно эти~$\alp_t$, $\beta_t$
и~$\gamma_t$ и~НФП будет  с.к.\ оптимальным
(последний в~данном случае будет допустимым и,~следовательно,
оптимальным в~классе допустимых фильтров).

Как известно~\cite{3-s2}, теория НФП дает возможность оценивать не все
компоненты вектора состояния системы (в общем случае расширенного),
а~только некоторые из них. Для этого достаточно взять структурные
функции~$\xi$ и~$\eta$ в~(\ref{e4.1-s2}) зависящими лишь от соответствующих
компонент вектора~$\hx_t$. К~примеру, взяв~$\xi$ и~$\eta$ в~(\ref{e4.1-s2})
зависящими лишь от~$Y_t$, $t$ и~оценок неизвестных параметров
системы, можно оценивать только параметры системы, не оценивая ее
со\-сто\-яния. В~таких случаях будут получаться НФП, порядок которых
меньше размерности~$n_x$ расширенного вектора со\-сто\-яния.

Особое практическое значение имеет случай~(\ref{e2.1-s2})--(\ref{e2.6-s2}) 
с~аддитивными (в~общем случае негауссовскими) шумами.
Следуя~\cite{3-s2}, проведем статистическую линеаризацию нелинейных функций:
\begin{align*}
a_1 \left(Y_t, t\right) X_t &\approx 
\left(k_{0x}^{a_1 x} - k_{1x}^{a_1 x}\right) m_t^x +{}\\
&{}+ \left( k_{0y}^{a_1 x} - k_{1y}^{a_1 x}\right) m_t^y + 
k_x^{a_1 y} X_t + k_{0y}^{a_1 x} Y_t\,;\\
b_1 \left(Y_t, t\right) X_t &\approx 
\left(k_{0x}^{b_1 x} - k_{1x}^{b_1 x}\right) m_t^x +{}\\
&{}+ \left( k_{0y}^{b_1 x} - k_{1y}^{b_1 x}\right) m_t^y + 
k_x^{b_1 y} X_t + k_{0y}^{b_1 x} Y_t\,;\\
a_0 \left(Y_t , t\right) &\approx
\left( k_{0y}^{a_0} - k_{1y}^{a_0}\right) m_t^y +
 k_{0y}^{a_0} Y_t\,; \\
b_0 \left(Y_t , t\right) &\approx \left( k_{0y}^{b_0} - k_{1y}^{b_0}\right) m_t^y + 
k_{0y}^{b_0} Y_t)\,.
\end{align*}
Тогда~(\ref{e2.1-s2})--(\ref{e2.6-s2}) приводятся к~эквивалентной 
гауссовской системе,
линейной относительно~$X_t^0, Y_t^0$ и~нелинейной относительно
$m_t^x, m_t^y$:
    \begin{align*}
    \dot X_t &= \bar a Y_t + \bar a_1 X_t + \bar a_0 +\bar\psi V\,;
    %\label{e5.2-s2}
    \\
\dot Y_t &= \bar b Y_t + \bar b_1 X_t + \bar b_0 +\bar\psi_1 V\,.
%\label{e5.3-s2}
\end{align*}
Здесь введены обозначения:
    \begin{equation}
    \left.
    \begin{array}{c}
    \bar a = k_y^{a_1x} + k_y^{a_0}\,; \quad 
    \bar a_1 = k_x^{a_1 x}\,;\\[4pt]
    a_0 = (k_{0y}^{a_0} - k_{1y}^{a_0}) m_t^y\,;\\[4pt]
\bar b = k_y^{b_1x}\,; \ \ \ \bar b_1 = k_x^{b_1 x}\,;\ \ \ \bar b_0 = (k_{0y}^{b_0} - k_{1y}^{b_0}) m_t^y\,.
\end{array}
\right\}
\label{e5.4-s2}
\end{equation}
Правые части уравнений~(\ref{e5.4-s2}) 
зависят от вероятностных моментов первого и~второго порядка и~определяются 
из следующей линейной дифференциальной системы для составного вектора  
$Z_t \hm=\lk X_t^{\mathrm{T}} Y_t^{\mathrm{T}}\rk^{\mathrm{T}}$:
\begin{equation}
\dot{\bar m}_t^z = cm_t^z + c_0\,;\enskip
%\eqno(5.5)$$
 \dot  K_t^{\bar z} = c K_t^z +K_t^z c^{\mathrm{T}} + l\nu l^{\mathrm{T}}\,,
 \label{e5.6-s2}
 \end{equation}
где
    \begin{equation*}
    c=\begin{bmatrix}
    \bar a_1&\bar a\\
    \bar b_1&\bar b\end{bmatrix}\,; \enskip
    c_0=\begin{bmatrix}
    \bar a_o\\
    \bar b_0\end{bmatrix}\,;\enskip
    l=\begin{bmatrix}
    \bar \psi_t\\
    \bar\psi_{1t}\end{bmatrix}\,.
%    \label{e5.7-s2}
    \end{equation*}

Применяя теорию квазилинейной фильтрации~\cite{3-s2} 
к~уравнениям~(\ref{e5.6-s2}), 
получим сле\-ду\-ющие уравнения субоптимального квазилинейного НФП:
    \begin{multline}
    \dot{\hat X}_t = \bar a Y_t +\bar a_1 \hat X_t + \bar a_0 + {}\\
    {}+
    \beta_t \lk \dot Y_t - \left( \bar b Y_t +\bar b_1 \hx_t + \bar b_0\right) \rk\,;
    \label{e5.8-s2}
    \end{multline}
\begin{equation}
\beta_t = R_t \bar b_1^{\mathrm{T}} +
\left(\bar\psi\nu\bar\psi_1^{\mathrm{T}}\right) 
\left(\bar \psi_1 \nu \bar \psi_1^{\mathrm{T}}\right)^{-1}\,;
\label{e5.9-s2}
\end{equation}

\vspace*{-12pt}

\noindent
\begin{multline}
 \dot R_t = \bar a_1 R_t + R_t \bar a_1^{\mathrm{T}} +
 \bar\psi\nu\bar\psi^{\mathrm{T}} -{}\\
 \hspace*{-3mm}{}- 
 \left(R_t \bar b_1^{\mathrm{T}} +\bar\psi\nu\bar\psi_1\right)
 \left(\bar\psi_1 \nu\bar\psi_1^{\mathrm{T}}\right)^{-1} 
 \left( \bar b R_t +\bar\psi_1\nu\bar\psi_1^{\mathrm{T}}\right).
 \label{e5.10-s2}
 \end{multline}

Таким образом, имеем следующий результат~\cite{1-s2}.

\smallskip

\noindent
\textbf{Теорема~5.2.}\
\textit{Пусть уравнения негауссовской СтС}~(\ref{e2.1-s2})--(\ref{e2.6-s2}) 
\textit{с~аддитивными шумами  допускают применение МСЛ. Тогда уравнения алгоритма 
квазилинейного НФП имеют вид}~(\ref{e5.8-s2})--(\ref{e5.10-s2}).

%\smallskip

\section{Обобщения нормальных фильтров на случай дискретных стохастических систем}


Теоремы разд.~3--5 допускают обобщение на случай 
дискретных СтС~(\ref{e2.8-s2}), (\ref{e2.9-s2}). Для
 дискретных гауссовских СтС~(\ref{e2.8-s2}) и~(\ref{e2.9-s2}) 
 при $V_k^d\hm= V_{0k}^d$ алгоритм НФ на основе МНА определяется уравнениями 
 (\textbf{теорема~6.1}):
\begin{multline*}
\hx_{k+1} = f_k \left(\hx_{k+1\mid k}, Y_k, R_{k+1\mid k}\right) +{}\\
   {}+ h_k \left(\hx_{k+1\mid k}, Y_k, R_{k+1\mid k}\right)\times{}\\
   {}\times
\lk Y_{k+1} - f_k^{(1)} \left(\hx_{k+1\mid k}, Y_k, R_{k+1\mid k}\right)\rk;
%\label{e6.1-s2}
\end{multline*}

\vspace*{-12pt}

\noindent
\begin{multline*}
R_{k+1}=\biggl\{ f_k^{(2)} \left(\hx _{k+1\mid k}, Y_k, R_{k+1\mid k}\right) -{}\\
{}- h_k \left(\hx_{k+1\mid k}, Y_k, R_{k+1\mid k}\right)
    \left(\psi_{1,k}\nu_{0k}\psi_{1,k}^{\mathrm{T}}\right)\times{}\\
    {}\times h \left(\hx_{k+1\mid k}, Y_k, R_{k+1\mid k}\right)^{\mathrm{T}}\biggr\} +{}\\[-20pt]
\end{multline*}

%\pagebreak

\noindent
\begin{multline*}
{}+\sum\limits_{r=1}^{n_y}\rho_r \left(\hat X_{k+1\mid k} , Y_k, R_{k+1\mid k}\right)\times{}\\
{}\times
\lk Y_{r, k+1} - f_{r,k}^{(1)}
    \left(\hx_{k+1\mid k}, Y_k,R_{k+1\mid k}\right)\rk.
%    \label{e6.2-s2}
    \end{multline*}
Здесь введены следующие обозначения:
    $$
    f_k=f_k \left(\hat X_{k+1\mid k}, Y_k, R_{k+1\mid k}\right) =
    \mm^N \lk\vrp_k\rk\,;\\
    $$
    $$
    f_k^1 = f_k^{(1)} \left(\hx_{k+1\mid k}, Y_k, R_{k+1\mid k}\right) = 
    \mm^N \lk \vrp_{1k}\rk\,;
    $$
    
    \vspace*{-12pt}
    
    \noindent
    \begin{multline*}
    h_k = h_k \left(\hx_{k+1\mid k}, Y_k, R_{k+1\mid k}\right) ={}\\
    {}=
     \mm^N \lk X\vrp_{1k} (Y_k,X)+ \bar\psi_k\nu_{0k}\bar\psi_{1k}^{\mathrm{T}} 
     \left(Y_k,X\right)\rk\,;
    \end{multline*}
    
    \vspace*{-12pt}

\noindent
\begin{multline*}
    f_k^{(2)}= f_k^{(2)} \left(\hx_{k+1\mid k}, Y_k, R_{k+1\mid k}\right) ={}\\
    {}=
    \mm^N \biggl[ \left(X- \hx_{k+1\mid k}\right) \vrp_k \left(Y_k, X\right)^{\mathrm{T}}+{}\\
{}+\vrp_k \left(Y_k,X\right) 
\left(X^{\mathrm{T}} - \hx_{k+1\mid k}\right)+
\bar\psi_k\nu_{0k}\bar\psi_{1,k}^{\mathrm{T}} \left(Y_k,X\right)\biggr];
\end{multline*}

\vspace*{-12pt}

\noindent
\begin{multline*}
\rho_r \left(\hx_{k+1\mid k}, Y_k, R_{k+1\mid k}\right)={}\\
{}=
\mm^N \biggl[ \left(X- \hx_{k+1\mid k}\right) 
\left(X^{\mathrm{T}}-\hx_{k+1\mid k}^{\mathrm{T}}\right) a_r (Y_k, X)+{}\\
{}+ \left(X-\hx_{k+1\mid k}\right) b_r \left(Y_k,X\right)^{\mathrm{T}} 
\left(X^{\mathrm{T}} -\hx_{k+1\mid k}^{\mathrm{T}}\right)+{}\\
{}+
b_r \left(Y_k,X\right) 
\left(X^{\mathrm{T}}-\hx_{k+1\mid k}^{\mathrm{T}}\right)\biggr],
%\label{e6.3-s2}
\end{multline*}
где $a_r$~--- $r$-й элемент мат\-ри\-цы-стро\-ки $(\vrp_{1k}^{\mathrm{T}}
\hm-\hat\vrp_{1k}^{\mathrm{T}})(\bar\psi_{1k}\nu_{0k}\bar\psi_{1k}^{\mathrm{T}})^{-1}$; 
$b_r \hm= \lk b_{1r}\cdots b_{n_x r}\rk^{\mathrm{T}}$; $ b_{lr}$~--- элемент $l$-й
строки и~$r$-го столб\-ца мат\-ри\-цы $(\bar\psi_k
\nu_{0k}\bar\psi_{1k}^{\mathrm{T}}) 
(\bar\psi_{1k}\nu_k\bar\psi_{1k}^{\mathrm{T}})^{-1}$.

В качестве начальных условий принимаются
    \begin{align*}
%    \left.
%    \begin{array}{rl}
    \hx_{1\mid 1} &= \hx_1 =\mm^N\lk X_1\mid Y_1\rk ;
    \\ 
    R_{1\mid 1} &= R_1 = \mm^N \lk \left(X_1 -X_1^0\right) 
    \left(X_1-{X_1^0}^{\mathrm{T}}\right)\vert Y_1\rk, 
%    \end{array}
%    \right\}
%    \label{e6.4-s2}
    \end{align*}
определяющие начальное нормальное распределение.


Алгоритм дискретного НФП для дискретных СтС~(\ref{e2.8-s2}), (\ref{e2.9-s2}), 
основываясь на~\cite{3-s2}, представим в~следующем виде (\textbf{теорема~6.2}):
    \begin{equation*}
    \hx_{k+1} = \alp_k \xi_k \left(\hx_k\right) + 
    \beta_k\eta_k\left(\hx_k\right) Y_k +\gamma_k.
%    \label{e6.5-s2}
    \end{equation*}
Здесь $\xi_k =\xi_k(\hx_k)$ и~$\eta_k=\eta_k (\hx_k)$~--- известные
структурные функции НФП; неизвестные коэффициенты фильтра~$\alp_k$, 
$\beta_k$ и~$\gamma_k$ определяются из уравнений:
\begin{gather*}
 \alp_k \kappa_{11}^{(k)} +\beta_k \kappa_{21}^{(k)} =\kappa_{01}^{(k)};\quad 
 \alp_k \kappa_{12}^{(k)} +
 \beta_k \kappa_{22}^{(k)} = \kappa_{02}^{(k)};
\\ 
 \gamma_k =\rho_0^{(k)} - \alp_k \rho_1^{(k)} - \beta_k \rho_2^{(k)};\\[-20pt]
\end{gather*}

\noindent
 $$ 
 \rho_k=\lk \rho_1^{(k)T}\rho_2^{(k)T}\rk^{\mathrm{T}};$$
 $$
 \rho_1^{(k)} = \mm^N \lk \xi_k\rk ;\enskip 
 \rho_2^{(k)}=\mm^N\lk \eta_k  \varphi_{1k}\rk;
 $$
 $$
 B_k=\begin{bmatrix}
 \kappa_{11}^{(k)}&\kappa_{12}^{(k)}\\
 \kappa_{21}^{(k)}&\kappa_{22}^{(k)}\end{bmatrix}\,,
 \enskip \det \lv B_k\rv \ne 0;
 $$
 $$
 \kappa_{11}^{(k)} =\mm^N \left\{\lk \xi_k  -\rho_1^{(k)}\rk 
 \xi_k^{\mathrm{T}}\right\};
 $$
 $$
 \kappa_{12}^{(k)} =\kappa_{21}^{(k)T}=\mm^N \left\{\lk \xi_k  -\rho_1^{(k)}\rk \varphi_{1k}^{\mathrm{T}} \eta_k^{\mathrm{T}} \right\};
 $$
 
 \vspace*{-12pt}

\noindent
\begin{multline*}
 \kappa_{22}^{(k)} = \mm^N \lf\lk \eta_k  \varphi_{1k}  -\rho_2^{(k)} \rk 
 \varphi_{1k}^{\mathrm{T}} \eta_k^{\mathrm{T}} \rf+{}\\
 {}+
 \mm^N \lf \eta_k \bar\psi_{1k} \nu_k \bar\psi_{1k}^{\mathrm{T}} \eta_k^{\mathrm{T}}\rf;
\end{multline*}
 $$
  D_k =\lk \kappa_{01}^{(k)} \kappa_{02}^{(k)}\rk;\enskip  
  \kappa_{01}^{(k)} =\mm^N \lf\lk \varphi_k - m_{k+1}\rk \xi_k^{\mathrm{T}}\rf;
  $$
  
  \vspace*{-12pt}

\noindent
\begin{multline*}
 \kappa_{02}^{(k)} = \mm^N \lf\lk \varphi_{k}  - 
 m_{k+1}\rk \varphi_{1k}^{\mathrm{T}} \eta_k^{\mathrm{T}}\rf+{}\\
 {}+\mm^N\lk 
 \bar\psi_k  \nu_k \bar\psi_{1k}^{\mathrm{T}} \eta_k^{\mathrm{T}}\rk;
\end{multline*}

\vspace*{-6pt}

\noindent
 \begin{equation*}
 m_{k+1}=\rho_0^{(k)}\,,\enskip \rho_0^{(k)}= \mm^N \left[\varphi_k\right],
% \label{e6.6-s2}
 \end{equation*}
где функции $\vrp_k$, $\vrp_{1k}$, $\bar\psi_k$ и~$\bar\psi_{1k}$ 
определены~(\ref{e2.8-s2}), (\ref{e2.9-s2}); 
$\mm^N V_k^d \hm=0$, $\mm^N V_k^d V^{dT}=\nu_k$~--- ковариация белого шума~$V_k^d$. 

\vspace*{-5pt}

\section{Нормальные экстраполяторы Пугачёва}

\vspace*{-1pt}

Будем считать услов\-но-оп\-ти\-маль\-ным (по Пугачёву~\cite{3-s2}) такой
экстраполятор из класса до\-пус\-тимых, который при любом совместном
распределении величин $X_t$, $\hat X_t$ и~$Y_t$ в~момент $t\hm\ge t_0$ 
в~дифференциаль\-ной СтС~(\ref{e2.1-s2})--(\ref{e2.5-s2}) дает наилучшую оценку вектора
$X_{s+\Delta}$, в~бесконечно близкий момент $s\hm>t$, $s\hm\to t$,
реализующую минимум с.к.\ ошибки. Тогда задача услов\-но-оп\-ти\-маль\-ной
экстраполяции сведется к~нахождению оптимальных значений~$\alp_t$,
$\beta_t$ и~$\gamma_t$ в~(\ref{e4.1-s2}) в~любой момент времени $t\hm\ge t_0$,
обеспечивающих минимум с.к.\ ошибки экстраполяции в~бесконечно близкий
будущий момент $s\hm>t$, $s\hm\to t$.

Для решения задачи экстраполяции необходимо ограничиться случаем~(\ref{e2.1-s2})--(\ref{e2.5-s2}),
когда функции~$\varphi$ и~$\psi$ не зависят от наблюдаемого
вектора~$Y_t$, процесс $W(t)$ состоит из двух независимых блоков
$W_1 (t)$ и~$W_2(t)$ и~соответственно матрицы~$\psi$ и~$\psi_1$
имеют блочную структуру  $\psi\hm = \lk \psi'\, 0\rk$; $ \psi_1\hm = \lk
0\,\psi_1'\rk$, так что $\psi \,dW \hm= \psi' \,dW_1$ и~$\psi_1 \,dW\hm=
\psi_1'\, dW_2$. Теперь, отбросив штрихи у функций~$\psi'$ и~$\psi_1'$, 
запишем уравнения~(\ref{e2.1-s2}) и~(\ref{e2.2-s2}) в~следующем виде:
\begin{equation}
\left.
\begin{array}{rl}
    \dot X_t &= \varphi \left(X_t,Y_t,t\right) dt + \psi \left(X_t,t\right)V_1\,;\\[5pt]
    & V_1 = \dot W_1\,;\quad X_{t_0} = X_0\,;\\[5pt]
    \dot Y_t &= \varphi \left(X_t,Y_t,t\right) dt + \psi_1 \left(X_t,Y_t,t\right) V_2\,;\\[5pt]
    &V_2 =\dot W_2\,;\quad Y_{t_0} = Y_0\,,
    \end{array}
    \right\}
    \label{e7.1-s2}
\end{equation}
где $W_1 = W_1 (t)$ и~$W_2 \hm= W_2 (t)$~--- независимые процессы 
с~независимыми приращениями и~нулевыми математическими функциями:
\begin{align*}
%\left.
%\begin{array}{rl}
    K_{w_i} (t_1, t_2)  &= k_i (\min (t_1, t_2))\,;\\ %[6pt]
    k_i(t) &= \displaystyle k_i(t_0) + \int\limits_{t_0}^t \nu_i (\tau)
   \, d\tau.
 %   \end{array}
%    \right\}
%    \label{e7.3-s2}
    \end{align*}

Совершенно так же, как в~случае услов\-но-оп\-ти\-маль\-ной нелинейной фильтрации, 
решается задача услов\-но-оп\-ти\-маль\-ной экстраполяции. Разница будет
лишь в~том, что в~случае экстраполяции~$\hx_t$ представляет собой
оценку будущего состояния системы  $X_{t+\Delta}$, $\Delta \hm>0$, которое
определяется стохастическим дифференциальным
уравнением следующего вида:
\begin{multline}
\dot X_{t+\Delta}=\varphi \left(X_{t+\Delta}, t+\Delta\right) +{}\\
{}+\psi \left(X_{t+\Delta}, t+\Delta \right) V(t+\Delta)\,,\enskip 
V=\dot V\,.\label{e7.4-s2}
\end{multline}
Заменив этим уравнением второе уравнение~(\ref{e2.9-s2}) и~повторив все
выкладки ФП, получим уравнения, определяющие коэффициенты 
в~уравнениях нормального услов\-но-оп\-ти\-маль\-но\-го экстраполятора Пугачёва
(НЭП). Приведем окончательные  результаты:
\begin{equation}
\left.
\begin{array}{l}
\kappa_{02} ={}\\[6pt]
{}= \mm^N\left[ \left(X_{t+\Delta} - \hat X_t\right) \varphi_1 
    \yxtt^{\mathrm{T}} \times{}\right.\\
   \left. \hspace*{35mm}{}\times \eta\yutt^{\mathrm{T}}\right]\,;
\\[6pt]
    \kappa_{22} = \mm^N
    \left[\eta\yutt\psi_1\yxtt\nu_1(t)\times{}\right.\\[6pt]
    \left.{}\hspace*{15mm}\times \psi_1\yxtt^{\mathrm{T}} 
    \eta\yutt^{\mathrm{T}}\right]\,;
\\[6pt]
\mm^N \lk \hx\rk = \mm^N \lk X_{t+\Delta}\rk;\\[6pt] 
\mm^N\lk\left(\hx - X_{t+\Delta}\right)\xi_t^{\mathrm{T}} \rk=0;
\end{array}
\right\}
\label{e7.5-s2}
\end{equation}
\begin{equation}
\left.
\begin{array}{rl}
m_0 &= \mm^N\lk \varphi \left(X_{t+\Delta}, t+\Delta\right)\rk \,;\\[6pt]
\kappa_{01} &= \mm^N\left\{ 
\vphantom{\xi\yutt^{\mathrm{T}}}
\left[ \varphi \left(X_{t+\Delta}, t+\Delta\right)-{}\right.\right.\\[6pt]
&\hspace*{25mm}\left.\left.{}- m_0\right] \xi\yutt^{\mathrm{T}}\right\};
\end{array}
\right\}
\label{e7.6-s2}
\end{equation}

\vspace*{-6pt}

\noindent
\begin{multline*}
\kappa_{01}' = \kappa_{01} + \mm^N\left[ \left(X_{t+\Delta}- \hat X_t\right) 
\fr{\partial \xi^{\mathrm{T}}}{\partial t}\right] +{}\\
{}+ \mm^N\!\bigg\{\! \!\left(X_{t+\Delta}-\hat X_t\right)\!\! \bigg[ 
\varphi_1^{\mathrm{T}} -\!\iii_{R_0^q}\! c_2(\rho)^{\mathrm{T}}
    \nu_{sP} (t,\rho) d\rho \bar\psi_1^{\mathrm{T}}\bigg] - {}\\
    {}-
    \beta_t\eta\bar\psi_1\nu_{20}\bar\psi_1^{\mathrm{T}}\bigg\}
\left(\fr{\partial}{\partial Y_t} + \eta^{\mathrm{T}} \beta_t^{\mathrm{T}} 
\fr{\partial}{\partial \hat X_t}\right)\xi^{\mathrm{T}}+{}\\
{}+
    \fr{1}{2}\, \mm^N \bigg\{ 
    \left(X_{t+\Delta} - \hat X_t\right) \times{}
    \end{multline*}


\noindent
\begin{multline}
 {}\times
    \biggl\{ 
    \mathrm{tr}\biggl[ 
    \bar\psi_1 \nu_{20} \bar\psi_1^{\mathrm{T}} \left(
\fr{\partial }{\partial Y_t} + 2\eta^{\mathrm{T}} \beta_t^{\mathrm{T}} 
\fr{\partial}{\partial\hat X_t}\right)
    \fr{\partial^{\mathrm{T}}}{\partial Y_t}\biggr] +{}\\
       {}+
    \mathrm{tr}\left[ \beta_t \eta \bar\psi_1 \nu_{20} \bar\psi_1^{\mathrm{T}} 
    \eta^{\mathrm{T}}\beta_t^{\mathrm{T}}
    \fr{\partial}{\partial \hat X_t} \,
    \fr{\partial^{\mathrm{T}}}{\partial \hat X_t}\right]\biggr\}
    \xi^{\mathrm{T}}\biggr\}+{}\\
{}+\iii_{R_0^q} \mm^N \bigg\{
\vphantom{\left[ \xi \left(X_t+\bar\psi_1 c_2(\rho), \hat X_t +
\beta_t \eta \bar\psi_1 c_2 (\rho),t\right)^{\mathrm{T}} -
    \xi^{\mathrm{T}}\right]} 
\left[ X_{t+\Delta} - \hat X_t-\beta_t\eta\bar\psi_1 c_2(\rho)\right]\times{}\\
{}\times\left[ \xi \left(X_t+\bar\psi_1 c_2(\rho), \hat X_t +
\beta_t \eta \bar\psi_1 c_2 (\rho),t\right)^{\mathrm{T}} -{}\right.\\
\left.{}-
    \xi^{\mathrm{T}}\right]\bigg\} \nu_{2P}(t,\rho)\, d\rho\,.
    \label{e7.7-s2}
    \end{multline}
Здесь  $c_2 (\rho)$, $\nu_{20}$ и~ $\nu_{2P}$~--- соответствующие
величины в~представлении интенсивности~$\nu_2$ процесса  $W_2(t)$
формулой вида~(\ref{e2.3-s2}).

В случае винеровского процесса  $W_2(t)$ имеем $c_2(\rho)\hm =0$,
$\nu_{20}\hm=\nu_2$ и~формула~(\ref{e7.7-s2}) принимает вид:
 \begin{multline*}
 \kappa_{01}' = \kappa_{01} + \mm^N\left[ \left(X_{t+\Delta}- \hat X_t\right) 
 \fr{\partial \xi^{\mathrm{T}}}{\partial t} \right]+{}\\
    {}+ \mm^N\bigg\{\left[
    \vphantom{\fr{\partial^{\mathrm{T}}} {\partial \hat X_t}}
     \left(X_{t+\Delta}-\hat X_t\right)  \varphi_1^{\mathrm{T}} - 
    \beta_t\eta\bar\psi_1\nu_{2}\bar\psi_1^{\mathrm{T}}\right]\times{}\\
    {}\times
    \left(\fr{\partial}{\partial Y_t} + \eta^{\mathrm{T}} \beta_t^{\mathrm{T}} 
    \fr{\partial}{\partial \hat X_t}\right)\xi^{\mathrm{T}}\bigg\}+{}\\
{} + \fr{1}{2}\, \mm^N\bigg\{ \left(X_{t+\Delta} - \hat X_t\right) \times{}\\
{}\times
\biggl\{ \mathrm{tr}\biggl[ 
\bar\psi_1 \nu_{2} \bar\psi_1^{\mathrm{T}}
    \left(\fr{\partial }{\partial Y_t} + 
    2\eta^{\mathrm{T}} \beta_t^{\mathrm{T}} \fr{\partial}{\partial \hat X_t}\right)
    \fr{\partial^{\mathrm{T}}}{\partial Y_t}\biggr]+{}\\
{}+\mathrm{tr}\left[ \beta_t\eta \bar\psi_1 \nu_{2} \bar\psi_1^{\mathrm{T}}
\eta^{\mathrm{T}}\beta^{\mathrm{T}}
    \fr{\partial}{\partial \hat X_t}\,  \fr{\partial^{\mathrm{T}}}
    {\partial \hat X_t}\right]\biggr\}\xi^{\mathrm{T}}\biggr\},
%    \label{e7.8-s2}
    \end{multline*}
при этом точность экстраполятора Пугачёва определяется, согласно~\cite{3-s2}, формулой:
 \begin{multline}
 \dot R_t = \mm^N
 \left[\left( X_{t+\Delta} -\hx\right) \varphi 
 \left(X_{t+\Delta}, t+\Delta\right)^{\mathrm{T}} +{}\right.\\
 {}+\varphi\left(X_{t+\Delta}, t+\Delta\right) 
 \left(X_{t+\Delta}^{\mathrm{T}} -\hx^{\mathrm{T}}\right)-{}\\
{}-\beta_t\eta \yutt \psi_1 \yxtt \nu_2(t) \psi_1\yxtt^{\mathrm{T}}\times{}\\ 
{}\times\eta \yutt^{\mathrm{T}} \beta_t^{\mathrm{T}} +{}\\
\hspace*{-5mm}\left.{}+\psi \left(X_{t+\Delta}, t+\Delta\right) 
 \nu_1(t+\Delta) \psi \left(X_{t+\Delta}, t+\Delta\right)^{\mathrm{T}}
 \right].\!
 \label{e7.9-s2}
 \end{multline}

Для вычисления математических ожиданий в~(\ref{e7.5-s2})--(\ref{e7.9-s2}) 
недостаточно знать
одномерное нормальное распределение составного случайного процесса  $\lk
X_t^{\mathrm{T}} Y_t^{\mathrm{T}}  \hat X_t^{\mathrm{T}}\rk^{\mathrm{T}}$, необходимо также знать совместное нормальное
распределение величин~$ X_t$,  $X_{t+\Delta}$, $Y_t$ и~$\hat X_t$ при каждом~$t$.


Таким образом, приходим к~следующим результатам.

\pagebreak

%\smallskip

\noindent
\textbf{Теорема~7.1.}\
\textit{Пусть векторный стохастический процесс  
$\lk X_t^{\mathrm{T}} Y_t^{\mathrm{T}}\rk^{\mathrm{T}}$ определяется 
негауссовскими уравнениями}~(\ref{e7.1-s2}) \textit{и~обладает 
конечными  двумерными вероятностными моментами. Тогда алгоритм НЭП определяется 
формулами}~(\ref{e7.4-s2})--(\ref{e7.7-s2}), (\ref{e7.9-s2}).

\smallskip

\noindent
\textbf{Теорема~7.2.}\ \textit{Для гауссовской СтС}~(\ref{e7.1-s2}) 
\textit{алгоритм НЭП определяется}~(\ref{e7.4-s2})--(\ref{e7.7-s2}), (\ref{e7.9-s2}).

\smallskip


Аналогично получаются уравнения алгоритмов НЭП для дискретных СтС.

Таким образом, теория условно-оптимальной экстраполяции стохастических процессов дает
возможность строить экстраполяторы Пугачёва для одновременного
оценивания состояния и~параметров системы и~экстраполяции ее
состояния на несколько различных интервалов времени в~реальном
масштабе времени. Все сложные расчеты, необходимые для
проектирования таких экстраполяторов, не опираются на результаты
наблюдения и~могут быть выполнены по априорным данным в~процессе
проектирования. Практическое применение таких экстраполяторов
сводится к~одновременному интегрированию уравнений~(\ref{e4.1-s2}) 
для оценивания текущего и~будущих состояний системы.


\section{Заключение}

Разработана теория аналитического синтеза непрерывных и~дискретных нормальных 
суб- и~услов\-но-оп\-ти\-маль\-ных фильтров и~экстраполяторов\linebreak Пугачёва для обработки 
процессов в~гауссовских и~негаус\-совских СтС, линейных относительно 
состояния.  Результаты допускают обобщение на случай автокоррелированных шумов 
в~наблюдениях. Алгоритмы положены в~основу программного обеспечения StS-Filter  
(version 2016) для стохастических информационных технологий~\cite{7-s2}.


{\small\frenchspacing
 {%\baselineskip=10.8pt
 \addcontentsline{toc}{section}{References}
 \begin{thebibliography}{9}



\bibitem{2-s2}
\Au{Липцер Р.\,Ш., Ширяев~А.\,Н.} 
Статистика случайных процессов.~--- М.: Наука, 1974. 476~с.

\bibitem{3-s2}
\Au{Синицын И.\,Н.}
Фильтры Калмана и~Пугачёва.~--- 2-е изд.~--- М.: Логос, 2007. 776~с.

\bibitem{4-s2}
\Au{Синицын И.\,Н., Синицын~В.\,И.} 
Лекции по нормальной и~эллипсоидальной аппроксимации в~стохастических системах.~--- 
М.: ТОРУС ПРЕСС, 2013. 476~с.

\bibitem{5-s2}
\Au{Синицын И.\,Н.}  
Параметрическое статистическое и~аналитическое моделирование распределений 
в~нелинейных стохастических системах на многообразиях~// 
Информатика и~её применения, 2013. Т.~7. Вып.~2. С.~4--16.

\bibitem{1-s2} %5
\Au{Синицын И.\,Н., Корепанов~Э.\,Р.}
Нормальные услов\-но-оп\-ти\-маль\-ные фильтры Пугачёва для дифференциальных 
стохастических систем, линейных относительно состояния~// Информатика и~её 
применения, 2015. Т.~9. Вып.~2. С.~30--38.

\bibitem{6-s2}
\Au{Ройтенберг Я.\,Н.} Автоматическое управление.~---
 3-е изд., перераб. и~доп.~--- М.: Наука, 1992. 576~с.

\bibitem{7-s2}
\Au{Корепанов Э.\,Р.} 
Стохастические информационные технологии на основе фильтров Пугачёва~// 
Информатика и~её применения, 2011. Т.~5. Вып.~2. С.~36--57.
\end{thebibliography}

 }
 }

\end{multicols}

\vspace*{-3pt}

\hfill{\small\textit{Поступила в~редакцию 02.02.16}}

\vspace*{8pt}

%\newpage

%\vspace*{-24pt}

\hrule

\vspace*{2pt}

\hrule

%\vspace*{8pt}



\def\tit{NORMAL PUGACHEV CONDITIONALLY-OPTIMAL
FILTERS  AND~EXTRAPOLATORS FOR~STATE
LINEAR STOCHASTIC SYSTEMS}

\def\titkol{Normal Pugachev conditionally-optimal
filters  and~extrapolators for~state
linear stochastic systems}

\def\aut{I.\,N.~Sinitsyn and E.\,R.~Korepanov}

\def\autkol{I.\,N.~Sinitsyn and E.\,R.~Korepanov}

\titel{\tit}{\aut}{\autkol}{\titkol}

\vspace*{-9pt}

\noindent
Institute of Informatics Problems, Federal Research Center 
``Computer Science and Control'' of the Russian Academy of Sciences,
44-2~Vavilov Str., Moscow 119333, Russian Federation

\def\leftfootline{\small{\textbf{\thepage}
\hfill INFORMATIKA I EE PRIMENENIYA~--- INFORMATICS AND
APPLICATIONS\ \ \ 2016\ \ \ volume~10\ \ \ issue\ 2}
}%
 \def\rightfootline{\small{INFORMATIKA I EE PRIMENENIYA~---
INFORMATICS AND APPLICATIONS\ \ \ 2016\ \ \ volume~10\ \ \ issue\ 2
\hfill \textbf{\thepage}}}

\vspace*{3pt}



\Abste{The analytical synthesis theory of continuous and discrete sub- and Pugachev 
conditionally optimal filters and extrapolators for information processing in linear 
state stochastic systems (StS) is presented. For Gaussian  StS,
 Liptzer and Shiraev 
performed the first works for filters and extrapolators synthesis. For 
non-Gaussian StS, the first works belong to Pugachev and Sinitsyn. 
Stochastic equatuins 
for state and observation of continuous and discrete StS are given. 
Algorithms for continuous normal sub- and conditionally optimal filters 
and extrapolators are presented. The corresponding algorithms for discrete StS 
are also given. The developed algorithms are the basis of the
software tool ``StS-Filter, 2016.''
The results may be developed for autocorrelated noises and multiplicative noises.}

\KWE{Liptser--Shiraev filter (LSF);
Liptser--Shiraev conditions;
normal approximation method (NAM) for \textit{a~posteriori} density;
normal conditionally optimal Pugachev filter (NPF);
stochastic systems (StS); state linear StS; statistical linearization method (SLM)}

\DOI{10.14357/19922264160202}

%\vspace*{-12pt}

%\Ack
%\noindent



%\vspace*{3pt}

  \begin{multicols}{2}

\renewcommand{\bibname}{\protect\rmfamily References}
%\renewcommand{\bibname}{\large\protect\rm References}

{\small\frenchspacing
 {%\baselineskip=10.8pt
 \addcontentsline{toc}{section}{References}
 \begin{thebibliography}{9}




\bibitem{2-s2-1}
\Aue{Liptser, R.\,Sh., and A.\,N.~Shiryaev}. 1974. 
\textit{Statistika sluchaynykh protsessov} [Statistics of stochastic proesses].~--- 
Moscow: Nauka. 476~p.

\bibitem{3-s2-1}
\Aue{Sinitsyn, I.\,N.} 2007.
\textit{Fil'try Kalmana i~Pugacheva} [Kalman and Pugachev filters]. 2nd ed. Moscow: Logos. 776~p.

\bibitem{4-s2-1}
\Aue{Sinitsyn, I.\,N., and V.\,I.~Sinitsyn}. 2013. 
\textit{Lektsii po normal'noy i ellipsoidal'noy approksimatsii 
v~stokhasticheskikh sistemakh}  [Lectures on normal and ellipsoidal approximation 
of distributions in stochastic systems].  Moscow: TORUS PRESS. 476~p.

\bibitem{5-s2-1}
\Aue{Sinitsyn, I.\,N.} 2013. 
Parametricheskoe statisticheskoe i~analiticheskoe modelirovanie raspredeleniy 
v~nelineynykh stokhasticheskikh sistemakh na mnogoobraziyakh [Parametric statistical 
and analytical modeling of distributions in stochastic systems on manifolds].
\textit{Informatika i~ee Primeneniya}~--- \textit{Inform Appl.} 7(2):4--16.

\bibitem{1-s2-1} %5
\Aue{Sinitsyn, I.\,N., and E.\,R.~Korepanov}. 2015.
Nor\-mal'\-nye uslovno optimal'nye fil'try Pugacheva dlya dif\-fe\-ren\-tsi\-al'\-nykh 
stokhasticheskikh sistem, lineynykh otnositel'no so\-sto\-yaniya [Normal Pugachev 
filters and extrapolators for state linear stochastic systems].
\textit{Informatika i~ee Primeneniya}~--- \textit{Inform. Appl}  9(2):30--38.

\bibitem{6-s2-1}
\Aue{Roytenberg, Ya.\,N.} 1992. 
\textit{Avtomaticheskoe upravlenie} 
[Automatic control]. 3rd ed. Moscow: Nauka. 576~p.

\bibitem{7-s2-1}
\Aue{Korepanov, E.\,R.} 2011. 
Stokhasticheskie informatsionnye tekhnologii na osnove fil'trov Pugacheva 
[Stochastic informational technologies based on Pugachev filters].
\textit{Informatika i~ee Primeneniya}~--- \textit{Inform. Appl}. 5(2):36--57.
\end{thebibliography}

 }
 }

\end{multicols}

\vspace*{-3pt}

\hfill{\small\textit{Received February 2, 2016}}

\Contr

\noindent
\textbf{Sinitsyn Igor N.} (b.\ 1940)~---
Doctor of Science in technology, professor,
Honored scientist of RF, Head of Department, Institute of Informatics Problems, Federal Research Center ``Computer Science and
Control'' of the Russian Academy of Sciences, 44-2~Vavilov Str.,
Moscow 119333, Russian Federation; sinitsin@dol.ru

\vspace*{3pt}

\noindent
\textbf{Korepanov Eduard R.} (b.\ 1966)~---
Candidate of Science (PhD) in technology, 
Head of Laboratory, Institute of Informatics Problems, Federal Research Center 
``Computer Science and Control'' of the Russian Academy of Sciences, 
44-2~Vavilov Str., Moscow 119333, Russian Federation; ekorepanov@ipiran.ru 

\label{end\stat}


\renewcommand{\bibname}{\protect\rm Литература}