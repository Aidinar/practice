\documentclass[10pt]{book}
\usepackage[utf8]{inputenc}

\usepackage{latexsym,amssymb,amsfonts,amsmath,indentfirst,shapepar,%fleqn,%
picinpar,shadow,floatflt,enumerate,multicol,colortbl,moreverb,ipi}

\usepackage{rotating}
\usepackage{mathrsfs}
\usepackage[noend]{algorithmic}
\usepackage{ulem}
%\usepackage{algorithm2e}

\input{epsf}

%\nofiles

%\includeonly{avtor} %+pdf
%\includeonly{obchak,avtor}
%\includeonly{pred}      %
%\includeonly{podgot-rus,podgot-eng}  %+pdf
%\includeonly{ocherk} %+
%\includeonly{nekrol} %+



%\includeonly{bosov}                   %1pdf
%\includeonly{sinits-2}                %2pdf
%\includeonly{sinit-1}                 %3+pdf
%\includeonly{goncharov}                %4pdf
%\includeonly{isach}                   %5pdf
%\includeonly{tyrsin}                  %6DOI+pdf
%\includeonly{shest}                   %7DOI+pdf
%\includeonly{agalarov}                %8DOI+pdf
%\includeonly{kudr}                    %9DOI+pdf
%\includeonly{ulianov}                 %10DOI+pdf
%\includeonly{ushakov}                 %DOI11+pdf
%\includeonly{khokhlov}                %12DOI+pdf
%\includeonly{minin}                   %13DOI+
%\includeonly{meih}                     %14DOIpdf????



%\includeonly{toc-rus, toc-en}
%\includeonly{obchak} %,toc-en}

%\includeonly{rekl}
%\includeonly{rekl-1}
%\includeonly{reshal}  %
%\includeonly{eng-index}
%\includeonly{cover3}

\usepackage{acad}
%\usepackage{courier}
\usepackage{decor}
\usepackage{newton}
\usepackage{pragmatica}
\usepackage{zapfchan}
\usepackage{petrotex}
\usepackage{bm}                     % полужирные греческие буквы
\usepackage{upgreek}                % прямые греческие буквы
\usepackage{eufrak}
\usepackage{verbatim}

\renewcommand{\bottomfraction}{0.99}
\renewcommand{\topfraction}{0.99}
\renewcommand{\textfraction}{0.01}

\setcounter{secnumdepth}{1} %здесь - 3 + chapter = 4

\arraycolsep=1.5pt

%\usepackage[pdftex]{graphicx}

%\usepackage{oz}

%NEW COMMANDS


\renewcommand*{\hm}[1]{#1\nobreak\discretionary{}%
            {\hbox{$\mathsurround=0pt #1$}}{}} %% Дублирует знаки операций
                               %при переносе в формуле (перед знаком, который
                               %надо продублировать ставится команда \hm)

%\newcommand{\endproof}{\hfill$\Box$}
%\renewcommand{\r}{\mathbb{R}}
\newcommand{\I}{{\rm I\hspace{-0.7mm}I}}
%\newcommand{\Ikl}{{\tt{1}}\hspace*{-1.44mm}\mathtt{1}}
\newcommand{\Ik}{\mbox{{\small \tt {1}}\hspace{-1.3mm}{\tt 1}}}
\newcommand{\argmin}{\mathop{\mathrm{arg}\,\mathrm{min}}}
\newcommand{\argmax}{\mathop{\mathrm{arg}\,\mathrm{max}}}
%\newcommand{\capr}{\mathop{\cap\,}}
%\newcommand{\cupr}{\mathop{\cup\,}}
%\def\argmin{\mathop{arg\,min}}

\def\vrp{\varphi}
\def\prt{\partial}
\def\mm{{\sf M}}
\def\modnop#1{\mathop{#1}\limits_{n}}
\def\eam{\mathbin{{\mathop{=}\limits^{\mathrm{def}}}}}
\def\dey#1#2{#1 (#2)}
\def\deyc#1#2{#1 \cdot  #2}
\def\ra#1{\;\mathop{\to}\limits^{#1}\;}
\def\raz#1{\;\mathop{\longrightarrow}\limits^{\!\!\!#1}\;}
\def\ral#1{\;\mathop{\longrightarrow}\limits^{#1}\;}

\newcommand{\Nor}{\mathcal{N}}
\newcommand{\T}{\mathbb{T}}
\newcommand{\Z}{\mathbb{Z}}



\newcommand{\il}[2]{\int\limits_{#1}^{#2}}%интеграл с пределами #1 и #2

\def\sm2{\mathop {\sum\limits^{n^\Theta}\sum\limits^{n^\Theta}}}
\def\sss{\sum\limits}
\def\tr{,\,\ldots\,,\,}
\def\rk{\right]}
\def\lk{\left[}
\def\rf{\right\}}
\def\lf{\left\{}
\def\lv{\,\left\vert}
\def\rv{\right\vert\,}
\def\iii{\int\limits}
\def\iin{\int\limits_{-\infty}^\infty}
\def\rrv{\right\vert}


\def\ee{{\cal E}}
\def\ww{{\cal W}}
\def\yy{{\cal Y}}
\def\vv{{\cal V}}

\newcommand{\R}{\mathbb R}
\newcommand{\E}{\mathbb E}
\newcommand{\N}{\mathbb N}

\renewcommand{\P}{\mathbb{P}}

\newcommand{\h}{{\bf H}}
\newcommand{\p}{{\sf P}}  % вероятность

\newcommand{\e}{{\sf E}}  % мат. ожидание
\newcommand{\D}{{\sf D}}  % дисперсия
\newcommand{\eps}{\varepsilon}
\newcommand{\vp}{{\mathbf p}}
\newcommand{\vz}{{\mathbf z}}
\newcommand{\vx}{{\mathbf x}}
\newcommand{\vf}{{\mathbf f}}
\newcommand{\F}{{\mathcal F}}
\def\ap{{\mathrm{ЭР}}}
\newcommand{\ud}{\Delta_n} %uniform ditance
\newcommand{\nud}{\Delta_n(x)}
\renewcommand{\Re}{\mathrm{Re}\,}

\newcommand{\abs}[1]{\left\vert#1\right\vert}

\newcommand{\norm}[1]{\left\Vert#1\right\Vert}
\def\da{(\Delta_t,A)}

\newcommand{\corr}{\mathrm{corr}}

\newcommand{\cov}{\mathrm{cov}}
\newcommand{\Expect}{\mathbb{E}}

\def\w{\omega}
\def\W{\Omega}

\def\inh{\int\limits_{nh}^{(n+1)h}}

\def\sumin{\sum_{i=1}^N}


\def\bxt{(Y,t)}
\def\xt{(y,t)}

\def\ovth{{\fr{\tau-nh}{h}}}
\def\ov{\overline}
\def\tm{\tilde m}
\def\tl{\tilde\lambda}
\def\tB{\widetilde B}
\def\tb{\tilde b}
\def\ld{\ldots}
\def\cd{\cdots}


\DeclareMathOperator{\sign}{sign}

%\newcommand{\gr}{{\geqslant}}


\newcommand{\g}{\mbox{\textit{g}}}

\renewcommand{\la}{\lambda}
\newcommand{\si}{\sigma}
\newcommand{\alp}{\alpha}

%\newcommand{\pto}{\stackrel{P}{\longrightarrow}} % сходимость по веpоятности

\newcommand{\eqd}{\stackrel{\mathrm{d}}{=}} % равенство по pаспpеделению
\newcommand{\eqdelta}{\stackrel{\Delta}{=}} % равенство по pаспpеделению

\def\be#1{\begin{equation}\label{#1}}
\def\ee{\end{equation}}
\def\re#1{(\ref{#1})}

\def\bn{\begin{enumerate}}
\def\en{\end{enumerate}}
\def\bi{\begin{itemize}}
\def\ei{\end{itemize}}
%\def\i{\item}

%\newcommand{\kp}{\kappa}
%\def\Q{{\cal Q}} \def\H{{\cal H}}
%\newcommand{\bet}{\beta_{2+\delta}}


%\newtheorem{definition}{Определение}
%\renewcommand{\thedefinition}{\arabic{definition}.}
%END NEW COMMANDS

%\renewcommand{\baselinestretch}{1.2}

%\pagestyle{myheadings}

\setlength{\textwidth}{167mm}      % 122mm
\setlength{\textheight}{658pt}
%\setlength{\textheight}{635.6pt}
\setlength{\columnsep}{4.5mm}

\setcounter{secnumdepth}{4}

%\addtolength{\headheight}{2pt}
%\addtolength{\headsep}{-2mm}

%\addtolength{\topmargin}{-20mm}  % for printing


%\hoffset=-30mm  % From Yap
\hoffset=-23mm  % From Acrobat

%\voffset=0mm % From Yap
%\voffset=-15mm   % From Acrobat

\addtolength{\evensidemargin}{-9.5mm} % for printing
\addtolength{\oddsidemargin}{9.5mm}  % for printing

%\renewcommand{\thefootnote}{\fnsymbol{footnote}}
%\renewcommand{\thefootnote}{\arabic{footnote}}
\renewcommand{\figurename}{\protect\bf Рис.}
\renewcommand{\tablename}{\protect\bf Таблица}

\newcommand{\Caption}[1]{\caption{\protect\small %\baselineskip=2.5ex
#1}}

\renewcommand{\thefigure}{\arabic{figure}}
\renewcommand{\thetable}{\arabic{table}}
\renewcommand{\theequation}{\arabic{equation}}
\renewcommand{\thesection}{\arabic{section}}

\renewcommand{\contentsname}{СОДЕРЖАНИЕ}
\newcommand{\fr}[2]{\displaystyle\frac{\displaystyle #1\mathstrut}{\displaystyle #2\mathstrut}}

%\renewcommand{\thefootnote}{\fnsymbol{footnote}}
%\newcommand{\g}{\mbox{\textit{g}}}

%\newcommand{\Caption}[1]{\caption{\protect\small\baselineskip=2ex #1}}
\newcounter{razdel}
\setcounter{razdel}{0}


\newcommand{\titel}[4]{%
\

\vspace*{5pt}

\ifodd\therazdel {\raggedright\noindent\Large\textrm\textbf
 \lineskip .75em
  \baselineskip=3.2ex #1 \par}
\vskip 1em {\noindent\large\textrm\textbf #2 \par}
\addcontentsline{toc}{subsection}{{\textrm\textbf #3}\protect\newline #1}
\def\rightheadline{\underline{\noindent\hbox to \textwidth{\hfill\small\textrm{#4}
%\hfill \large\bf\thepage
}}}
\def\leftheadline{\underline{\noindent\parbox{\textwidth}{
%\raggedleft\large\bf\thepage \hfill
\small\textit{#3}\hfill}}}
\def\leftfootline{\small{\textbf{\thepage}
\hfill ИНФОРМАТИКА И ЕЁ ПРИМЕНЕНИЯ\ \ \ том~10\ \ \ выпуск 2\ \ \ 2016}
}%
 \def\rightfootline{\small{ИНФОРМАТИКА И ЕЁ ПРИМЕНЕНИЯ\ \ \ том~10\ \ \ выпуск~2\ \ \ 2016
\hfill \textbf{\thepage}}}
\vskip 2em \setcounter{figure}{0}
\setcounter{table}{0}
\setcounter{equation}{0}
\setcounter{section}{0}
\setcounter{subsection}{0}
\setcounter{subsubsection}{0}
\setcounter{footnote}{0}
\setcounter{razdel}{0}
%\end{flushleft}
\else {
 \raggedright\noindent\Large\textrm\textbf
 \lineskip .75em
\baselineskip=3.2ex #1 \par} \vskip 1em
%\begin{flushleft}
{\noindent\large\textrm\textbf #2 \par}
\addcontentsline{toc}{subsection}{{\textrm\textbf #3}\protect\newline #1}
\def\rightheadline{\underline{\noindent\hbox to \textwidth{\hfill\small\textrm{#4}
%\hfill \large\bf\thepage
}}}
\def\leftheadline{\underline{\noindent\parbox{\textwidth}{%\raggedleft\large\bf\thepage \hfill
\small\textit{#3}\hfill}}}
\def\leftfootline{\small{\textbf{\thepage}
\hfill ИНФОРМАТИКА И ЕЁ ПРИМЕНЕНИЯ\ \ \ том~10\ \ \ выпуск~2\ \ \ 2016}
}%
 \def\rightfootline{\small{ИНФОРМАТИКА И ЕЁ ПРИМЕНЕНИЯ\ \ \ том~10\ \ \ выпуск~2\ \ \ 2016
\hfill \textbf{\thepage}}} \vskip 2em \setcounter{figure}{0}
\setcounter{table}{0} \setcounter{equation}{0} \setcounter{section}{0}
\setcounter{subsection}{0} \setcounter{subsubsection}{0}
\setcounter{footnote}{0}
%\end{flushleft}
\fi}

\newcommand{\titelr}[2]{%
\

\vspace*{5pt}

\ifodd\therazdel {\raggedright\noindent%\Large\textrm\textbf
 \lineskip .75em
  \baselineskip=3.2ex #1 \par}
\vskip 1em {\noindent\normalsize\textrm\textbf #2 \par}
\else {
 \raggedright\noindent\Large\textrm\textbf
 \lineskip .75em
\baselineskip=3.2ex #1 \par} \vskip 1em
%\begin{flushleft}
{\noindent\large\textrm\textbf #2 \par
%\noindent\normalsize\textrm\textbf #2 \par
} \fi}

\newcommand{\titele}[5]{%
\

%\vspace*{5pt}

\ifodd\therazdel {\raggedright\noindent\large
\textrm\textbf
 \lineskip .75em
%  \baselineskip=3.2ex
#1 \par}
\vskip .5em {\noindent\large\textrm\textbf #2 \par}
\vskip .5em
 {\noindent\textrm #3 \par}
\addcontentsline{toc}{subsection}{{\textrm\textbf #1}\protect\newline #2}
\def\rightheadline{\underline{\noindent\hbox to \textwidth{\hfill\small\textrm{#4}
%\hfill \large\bf\thepage
}}}
\def\leftheadline{\underline{\noindent\parbox{\textwidth}{
%\raggedleft\large\bf\thepage \hfill
\small\textrm{#5}\hfill}}}
\def\leftfootline{\small{\textbf{\thepage}
\hfill ИНФОРМАТИКА И ЕЁ ПРИМЕНЕНИЯ\ \ \ том~10\ \ \ выпуск~2\ \ \ 2016}
}%
 \def\rightfootline{\small{ИНФОРМАТИКА И ЕЁ ПРИМЕНЕНИЯ\ \ \ том~10\ \ \ выпуск~2\ \ \ 2016
\hfill \textbf{\thepage}}} \vskip 1em \setcounter{figure}{0}
\setcounter{table}{0} \setcounter{equation}{0} \setcounter{section}{0}
\setcounter{subsection}{0} \setcounter{subsubsection}{0}
\setcounter{footnote}{0} \setcounter{razdel}{0}
%\end{flushleft}
\else {
 \raggedright\noindent\large
 \textrm\textbf
 \lineskip .75em
%\baselineskip=3.2ex
#1 \par} \vskip .5em
%\begin{flushleft}
{\noindent\large\textrm\textbf #2 \par} \vskip .5em
 {\noindent\textrm #3 \par}
\addcontentsline{toc}{subsection}{{\textrm\textbf #1}\protect\newline #2}
\def\rightheadline{\underline{\noindent\hbox to \textwidth{\hfill\small\textrm{#4}
%\hfill \large\bf\thepage
}}}
\def\leftheadline{\underline{\noindent\parbox{\textwidth}{%\raggedleft\large\bf\thepage \hfill
\small\textrm{#5}\hfill}}}
\def\leftfootline{\small{\textbf{\thepage}
\hfill ИНФОРМАТИКА И ЕЁ ПРИМЕНЕНИЯ\ \ \ том~10\ \ \ выпуск~2\ \ \ 2016}
}%
 \def\rightfootline{\small{ИНФОРМАТИКА И ЕЁ ПРИМЕНЕНИЯ\ \ \ том~10\ \ \ выпуск~2\ \ \ 2016
\hfill \textbf{\thepage}}} \vskip 1em \setcounter{figure}{0}
\setcounter{table}{0} \setcounter{equation}{0} \setcounter{section}{0}
\setcounter{subsection}{0} \setcounter{subsubsection}{0}
\setcounter{footnote}{0}
%\end{flushleft}
\fi}

\def\Abst#1{
\begin{center}\small\nwt
\parbox{150mm}{%\baselineskip=2.5ex
\textbf{Аннотация:}\ \
%\hspace*{\parindent}
#1}
\end{center}}
\def\Abste#1{
\begin{center}\small\nwt
\parbox{150mm}{%\baselineskip=2.5ex
\textbf{Abstract:}\ \
%\hspace*{\parindent}
#1}
\end{center}}

\def\DOI#1{
\begin{center}\small\nwt
\parbox{150mm}{%\baselineskip=2.5ex
\textbf{DOI:}\ \
%\hspace*{\parindent}
#1}
\end{center}}

\def\Abstend#1{
\begin{center}\small\nwt
\parbox{150mm}{%\baselineskip=2.5ex
%\hspace*{\parindent}
#1}
\end{center}}


\def\KW#1{
\begin{center}\small\nwt
\parbox{150mm}{%\baselineskip=2.5ex
\textbf{Ключевые слова:}\ \ #1}
\end{center}}

\def\KWE#1{
\begin{center}\small\nwt
\parbox{150mm}{%\baselineskip=2.5ex
\textbf{Keywords:}\ \ #1}
\end{center}}


\def\KWN#1{
%\begin{center}
%\small
%\parbox{150mm}\end{center}
}

\renewcommand{\thesubsection}{\thesection.\arabic{subsection}\hspace*{-5pt}}
\renewcommand{\thesubsubsection}{\thesubsection\hspace*{5pt}.\arabic{subsubsection}\hspace*{-3pt}}

\newcommand{\Ack}{\section*{\protect\rmfamily Acknowledgments}\noindent}
\newcommand{\Contr}{\section*{\protect\rmfamily Contributors}\noindent}
\newcommand{\Contrl}{\section*{\protect\rmfamily Contributor}\noindent}

\makeindex


\begin{document}
\Rus

\nwt
%\ptb


%\renewcommand{\contentsname}{\protect\Large\bf Содержание}

\setcounter{tocdepth}{2}

%\tableofcontents

\renewcommand{\bibname}{\protect\rmfamily Литература}
  \def\Au#1{{\it #1}}
    \def\Aue#1{{#1}}

%\newcommand{\No}{№}
  \newcommand{\tg}{\,\mathrm{tg}\,}
    \newcommand{\ctg}{\,\mathrm{ctg}\,}
  \newcommand{\arctg}{\,\mathrm{arctg}\,}

\def\forallb{\mathop{\forall}}
\def\cupb{\mathop{\cup}}
\def\existsb{\mathop{\exists}}


\newpage
\addtocounter{razdel}{1}
%\def\razd{РЕГУЛИРУЕМЫЙ ЭЛЕКТРОПРИВОД ДЛЯ ЭЛЕКТРОЭНЕРГЕТИКИ}


\setcounter{page}{2}

%   { %\Large  
   { %\baselineskip=16.6pt
   
   \vspace*{-48pt}
   \begin{center}\LARGE
   \textit{Предисловие}
   \end{center}
   
   %\vspace*{2.5mm}
   
   \vspace*{25mm}
   
   \thispagestyle{empty}
   
   { %\small 

    
Вниманию читателей журнала <<Информатика и её применения>> предлагается 
очередной тематический выпуск <<Вероятностно-статистические методы и 
задачи информатики и информационных технологий>>. Предыдущие тематические 
выпуски журнала по данному направлению вышли в 2008~г.\ (т.~2, вып.~2), 
в 2009~г.\ (т.~3, вып.~3) и в 2010~г.\ (т.~4, вып.~2). 

Статьи, собранные в данном журнале, посвящены разработке новых вероятностно-статистических 
методов, ориентированных на применение к решению конкретных задач информатики и информационных 
технологий, а также~--- в ряде случаев~--- и других прикладных задач. Проблематика, охватываемая 
публикуемыми работами, развивается в рамках научного сотрудничества между Институтом проблем 
информатики Российской академии наук (ИПИ РАН) и Факультетом вычислительной математики и 
кибернетики Московского государственного университета им.\ М.\,В.~Ломоносова в ходе работ 
над совместными научными проектами (в том числе в рамках функционирования 
Научно-образовательного центра <<Вероятностно-статистические методы анализа рисков>>). 
Многие из авторов статей, включенных в данный номер журнала, являются активными участниками 
традиционного международного семинара по проблемам устойчивости стохастических моделей, 
руководимого В.\,М.~Золотаревым и В.\,Ю.~Королевым; регулярные сессии этого семинара 
проводятся под эгидой МГУ и ИПИ РАН (в 2011~г.\ указанный семинар проводится в октябре 
в Калининградской области РФ). 

Наряду с представителями ИПИ РАН и МГУ в число авторов данного выпуска журнала входят 
ученые из Научно-исследовательского института системных исследований РАН, Института 
проблем технологии микроэлектроники и особочистых материалов РАН, Института 
прикладных математических исследований Карельского НЦ РАН, Московского 
авиационного института, Вологодского государственного педагогического университета, 
НИИММ им.\ Н.\,Г.~Чеботарева, Казанского государственного университета, Дебреценского 
университета (Венгрия).

Несколько статей выпуска посвящено разработке и применению стохастических методов и 
информационных технологий для решения различных прикладных задач. В~работе В.\,Г.~Ушакова 
и О.\,В.~Шестакова рассмотрена задача определения вероятностных характеристик случайных 
функций по распределениям интегральных преобразований, возникающих в задачах эмиссионной 
томографии. В~статье Д.\,О.~Яковенко и М.\,А.~Целищева рассмотрены некоторые вопросы 
математической теории риска и предложен новый подход к диверсификации инвестиционных 
портфелей. Работа И.\,А.~Кудрявцевой и А.\,В.~Пантелеева посвящена построению и 
исследованию математической модели, описывающей динамику сильноионизованной плазмы. 
В~статье П.\,П.~Кольцова изучается качество работы ряда алгоритмов сегментации изображений. 
Статья А.\,Н.~Чупрунова и И.~Фазекаша посвящена вероятностному анализу числа без\-оши\-бочных 
блоков при помехоустойчивом кодировании; получены усиленные законы больших чисел для указанных 
величин.

В данном выпуске традиционно присутствует тематика, весьма активно разрабатываемая в течение 
многих лет специалистами ИПИ РАН и МГУ,~--- методы моделирования и управления для 
информационно-телекоммуникационных и вычислительных систем, в частности методы 
теории массового обслуживания. В~статье А.\,И.~Зейфмана с соавторами рассматриваются 
модели обслуживания, описываемые марковскими цепями с непрерывным временем в случае 
наличия катастроф. В~работе М.\,М.~Лери и И.\,А.~Чеплюковой рассматриваются случайные 
графы Интернет-типа, т.\,е.\ графы, степени вершин которых имеют степенные распределения; 
такие задачи находят применение при исследовании глобальных сетей передачи данных. 
Работа Р.\,В.~Разумчика посвящена исследованию систем массового обслуживания специального 
вида~--- с отрицательными заявками и хранением вытесненных заявок.

Ряд статей посвящен развитию перспективных теоретических 
вероятностно-статистических методов, которые находят широкое применение в различных 
задачах информатики и информационных технологий. В~работе В.\,Е.~Бенинга, А.\,К.~Горшенина 
и В.\,Ю.~Королева рассмотрена задача статистической проверки гипотез о числе компонент 
смеси вероятностных распределений, приводится конструкция асимптотически наиболее мощного 
критерия. Результаты этой работы найдут применение в ряде прикладных задач, использующих 
математическую модель смеси вероятностных распределений (в информатике, моделировании 
финансовых рынков, физике турбулентной плазмы и~т.\,д.). В~статье В.\,Ю.~Королева, 
И.\,Г.~Шевцовой и С.\,Я.~Шоргина строится новая, улучшенная оценка точности нормальной 
аппроксимации для пуассоновских случайных сумм; как известно, указанные случайные суммы 
широко используются в качестве моделей многих реальных объектов, в том числе в информатике, 
физике и других прикладных областях. Работа В.\,Г.~Ушакова и Н.\,Г.~Ушакова посвящена 
исследованию ядерной оценки плотности распределения; эти результаты могут применяться, 
в част\-ности, при анализе трафика в телекоммуникационных системах. Серьезные приложения 
в статистике могут получить результаты работы О.\,В.~Шестакова, в которой доказаны оценки 
скорости сходимости распределения выборочного абсолютного медианного отклонения к нормальному 
закону. 

\smallskip

Редакционная коллегия журнала выражает надежду, что данный тематический  выпуск 
будет интересен специалистам в области теории вероятностей и математической статистики 
и их применения к решению задач информатики и информационных технологий.
     
     %\vfill 
     \vspace*{20mm}
     \noindent
     Заместитель главного редактора журнала <<Информатика и её 
применения>>,\\
     директор ИПИ РАН, академик  \hfill
     \textit{И.\,А.~Соколов}\\
     
     \noindent
     Редактор-составитель тематического выпуска,\\
     профессор кафедры математической статистики факультета\\
      вычислительной математики и кибернетики МГУ им.\ М.\,В.~Ломоносова,\\
     ведущий научный сотрудник ИПИ РАН,\\ 
доктор физико-математических наук \hfill
      \textit{В.\,Ю.~Королев}
     
     } }
     }



\def\stat{bosov+stef}

\def\tit{УПРАВЛЕНИЕ ВЫХОДОМ СТОХАСТИЧЕСКОЙ ДИФФЕРЕНЦИАЛЬНОЙ СИСТЕМЫ 
ПО~КВАДРАТИЧНОМУ КРИТЕРИЮ. I.~ОПТИМАЛЬНОЕ РЕШЕНИЕ МЕТОДОМ 
ДИНАМИЧЕСКОГО ПРОГРАММИРОВАНИЯ$^*$}

\def\titkol{Управление выходом стохастической дифференциальной системы 
по~квадратичному критерию. I}
%.~Оптимальное решение методом 
%динамического программирования}

\def\aut{А.\,В.~Босов$^1$, А.\,И.~Стефанович$^2$}

\def\autkol{А.\,В.~Босов, А.\,И.~Стефанович}

\titel{\tit}{\aut}{\autkol}{\titkol}

\index{Босов А.\,В.}
\index{Стефанович А.\,И.}
\index{Bosov A.\,V.}
\index{Stefanovich A.\,I.}




{\renewcommand{\thefootnote}{\fnsymbol{footnote}} \footnotetext[1]
{Работа выполнена при частичной поддержке РФФИ (проект 16-07-00677).}}


\renewcommand{\thefootnote}{\arabic{footnote}}
\footnotetext[1]{Институт проблем информатики Федерального исследовательского центра <<Информатика 
и~управление>> Российской академии наук, \mbox{AVBosov@ipiran.ru}}
\footnotetext[2]{Институт проблем информатики Федерального исследовательского центра <<Информатика 
и~управление>> Российской академии наук, \mbox{AStefanovich@frccsc.ru}}

%\vspace*{8pt}



  
  \Abst{Решается задача оптимального управления для диффузионного процесса 
Ито и~линейного управ\-ля\-емо\-го выхода. Рассматриваемая постановка близка 
к~классической ли\-ней\-но-квад\-ра\-тич\-ной гауссовской задаче управления 
(linear-quadratic Gaussian (LQG) control). Отличия состоят в~том, что состояние описывается нелинейным 
дифференциальным уравнение Ито $dy_t\hm= A_t(y_t) \,dt\hm+ \Sigma_t(y_t)\,dv_t$ 
и~не зависит от управ\-ле\-ния~$u_t$, оптимизации подлежит управ\-ля\-емый 
линейный выход $dz_t\hm= a_t y_t\,dt\hm+ b_t z_t \,dt\hm+ c_t u_t \,dt\hm+ \sigma_t\, 
dw_t$. Дополнительные обобщения внесены в~квад\-ра\-тич\-ный критерий качества 
с~целью воз\-мож\-ности постановки таких задач, как отслеживание выходом 
состояния или управ\-ле\-ни\-ем~--- линейной комбинации состояния и~выхода. Для 
решения используется метод динамического программирования. Функцию 
Беллмана позволяет найти предположение о~ее структуре вида $V_t(y,z)\hm= 
\alpha_t z^2\hm+ \beta_t(y)z \hm+\gamma_t(y)$. Решение дают три 
дифференциальных уравнения для коэффициентов~$\alpha_t$, $\beta_t(y)$ 
и~$\gamma_t(y)$. Эти уравнения со\-став\-ля\-ют оптимальное решение 
рас\-смат\-ри\-ва\-емой задачи.}
  
  \KW{стохастическое дифференциальное уравнение; оптимальное управ\-ле\-ние; 
динамическое программирование; функция Беллмана; уравнение Риккати; 
линейные уравнения параболического типа}

\DOI{10.14357/19922264180314}
  
%\vspace*{4pt}


\vskip 10pt plus 9pt minus 6pt

\thispagestyle{headings}

\begin{multicols}{2}

\label{st\stat}

\section{Введение}

     Ключевые результаты в~области оптимизации стохастических 
динамических систем, со\-став\-ля\-ющие классическую теорию управления, 
получены более~40~лет назад (такова работа~[1] в~отношении задачи 
управ\-ле\-ния ли\-ней\-но-гаус\-сов\-ски\-ми стохастическими сис\-те\-ма\-ми по 
квад\-ра\-тич\-но\-му критерию). К~классической тео\-рии следует относить 
линейные модели стохастических сис\-тем и~квадратичный критерий качества. 
Это исходный базис, на котором основано множество успешно 
исследованных и~решенных задач стохастического управ\-ле\-ния 
и~оптимизации. 

Дальнейшее развитие~--- это новые модели и~критерии, но 
прежде всего это новые методы: от тео\-рии линейных регуляторов, метода 
динамического программирования и~принципа максимума к~адаптивному 
и~минимаксному подходу, импульсному управ\-ле\-нию и~т.\,д. Множество 
инноваций как в~час\-ти моделей, так и~в~час\-ти математического аппарата, 
имевших мес\-то в~по\-сле\-ду\-ющие годы, существенно обогатили тео\-рию 
управ\-ле\-ния. Но и~до настоящего времени линейные модели и~квадратичный 
критерий, несмотря на всю справедливую критику в~отношении их 
аде\-кват\-ности и~гиб\-кости, сохраняют исследовательский интерес и~находят 
современные области приложения.
     
     Не претендуя на сколь\-ко-ни\-будь полное обосно\-ва\-ние последнего 
тезиса, приведем несколько примеров, показавшихся наиболее ин\-те\-рес\-ными. 

Так, в~[2] решается ли\-ней\-но-квад\-ра\-тич\-ная за\-да\-ча в~игровой 
постановке с~запаздыванием. В~близ\-кой по модели работе~[3] задача 
управ\-ле\-ния ставится в~терминах $H_\infty$-ро\-баст\-ности. Точнее \mbox{называть} 
эту тематику $H_2/H_\infty$-управ\-ле\-ни\-ем, и~работ по этой теме очень 
много. Аккуратности ради следует уточнить, что под линейными 
понимаются модели с~мультипликативными по состоянию воз\-му\-ще\-ниями. 

Совсем другой класс моделей, особо популярных в~по\-след\-ние годы, 
составляют скачкообразные процессы. Например, линейные уравнения 
в~сочетании с~пуассоновскими скачками в~[4] используются в~моделях, 
описывающих различные показатели функционирования сетевых протоколов 
передачи данных транспортного уровня. Телекоммуникации представляют 
в~последние годы самый популярный прикладной материал для 
исследований, работ по этой проб\-ле\-ма\-ти\-ке множество, математические 
техники привлекаются самые разные и~самые современные, но и~линейным 
моделям место находится. Еще один любопытный пример исследования 
скачкообразного процесса и~оптимизации на основе квад\-ра\-тич\-но\-го критерия 
можно найти в~[5] применительно к~задаче инвестирования на финансовом 
рынке. Наконец, упомянем еще работу~[6], подводящую итог исследований 
в~отношении классической детерминированной  
ли\-ней\-но-квад\-ра\-тич\-ной задачи с~использованием техники матричных 
неравенств.
     
     В данной работе также эксплуатируются привлекательные свойства 
линейных моделей и~квад\-ра\-тич\-но\-го критерия, причем в~стохастической 
постановке. На\-прав\-ле\-ни\-ем для обобщения \mbox{выбрана} модель динамики 
сис\-те\-мы: основные усилия на\-прав\-ле\-ны на то, чтобы сделать ее нелинейной. 
Кроме того, пред\-став\-лен\-ная постановка может рас\-смат\-ри\-вать\-ся и~как 
обобщение ранее решенной задачи в~дискретном времени~[7, 8] на время 
непрерывное. В~упомянутых работах помимо собственно модельной 
постановки важна еще и~привлекаемая прикладная об\-ласть~--- 
функционирование сложных программных сис\-тем. Результатов, 
ориентированных непосредственно на такие приложения, к~настоящему 
времени пренебрежимо мало, поэтому~[7, 8]~--- это еще и~прикладное 
обоснование рас\-смат\-ри\-ва\-емой далее задачи.
     
     Оптимизируемая динамическая сис\-те\-ма описывается двумя 
уравнениями. Состояние задается нелинейным стохастическим 
дифференциальным уравнением Ито, не содержащим управ\-ля\-емой 
переменной. Возмущение здесь описывается стандартным винеровским 
процессом, накладываются простые условия существования 
и~един\-ст\-вен\-ности решения. Поскольку состояние не управ\-ля\-ет\-ся, то уместно 
его интерпретировать как слож\-ное внешнее возмущение. Вторая 
переменная~--- управ\-ля\-емый выход~--- задается линейным стохастическим 
дифференциальным уравнением. Цель оптимизации выхода формируется 
квадратичным критерием общего вида. Формальная постановка задачи 
приведена в~сле\-ду\-ющем разделе.
     
     Для решения задачи используется метод динамического 
программирования, решается уравнение Беллмана~[9]. Соответственно, 
в~результате получаются аналитические выражения и~для оптимального 
управ\-ле\-ния, и~для значения функционала качества. Технически 
традиционный, стандартный подход к~задаче обременен, пожалуй, 
единственной проблемой~--- поиском верного пред\-став\-ле\-ния структуры 
функции Беллмана. Справиться с~этой проблемой в~большей степени удается 
за счет результата, полученного при решении дискретного по времени 
аналога рассматриваемой постановки~\cite{8-bos}. Конечные соотношения 
для оптимального решения, как и~во всех подобных задачах, включая 
классическую ли\-ней\-но-квад\-ра\-тич\-ную, содержат решения 
определенных дифференциальных уравнений (обыкновенных и~в~частных 
производных). Вывод этих уравнений и~со\-став\-ля\-ет содержание первой час\-ти 
данной работы. Во второй части будет обсуждаться их приближенное 
чис\-лен\-ное решение и~компьютерные эксперименты.
     
     Кратко обозначим основные положения, при\-вле\-ка\-емые далее 
к~решению задачи, следуя в~основном обозначениям 
и~терминологии~\cite{9-bos}, а~именно: будем рассматривать задачу 
оптимального управления в~стохастической динамической сис\-те\-ме по полной 
информации, применяя метод динамического программирования. В~качестве 
целевого функционала, опре\-де\-ля\-юще\-го качество управ\-ле\-ния $U_0^T\hm= \{ 
u_t,\ 0\leq t\leq T\}$, выступает
     \begin{equation}
     J\left(U_0^T\right)={\sf E}\left\{ \int\limits_0^T L_t \left(x_t, u_t\right)\,dt+ 
l\left(x_T\right)\right\}\,.
     \label{e1-bos}
     \end{equation}
Здесь ${\sf E}\{\cdot\}$~--- оператор математического ожидания; $x_t$~--- 
случайный процесс, описываемый стохастическим дифференциальным 
уравнением Ито
     \begin{equation}
     dx_t=m_t\left( x_t, u_t\right) dt+ \sigma_t\left( x_t\right)dW_t\,,\enskip 
x_0=X\,,
     \label{e2-bos}
     \end{equation}
где $W_t$~--- стандартный винеровский процесс подходящей раз\-мер\-ности; 
$X$~--- случайный вектор.

     $U_0^T$ будем выбирать из класса допустимых неупреждающих (по 
отношению к~$W_t$) управлений~\cite{9-bos}. Соответственно, 
относительно функций сноса и~диффузии~$m_t$ и~$\sigma_t$  
в~(\ref{e2-bos}) будем предполагать выполненными ка\-кие-ли\-бо условия 
существования сильного решения для заданного до\-пус\-ти\-мо\-го управ\-ле\-ния. 
Например, для управ\-ле\-ния с~обратной связью $u_t\hm= u_t(x_t)$ будем 
считать, что $m_t(x,u_t(x))$ и~$\sigma_t(x)$ удовлетворяют условию 
линейного рос\-та и~локальному условию Липшица по~$x$ равномерно 
по~$t$ (т.\,е.\ условиям Ито).
     
     Для поиска оптимального управления, минимизирующего $J(U_0^T)$, 
рас\-смат\-ри\-ва\-ет\-ся функция Беллмана
     \begin{equation}
     V_t(x)=\left.\mathop{\mathrm{inf}}\limits_{U_t^T} {\sf E} \left\{ \int\limits_t^T 
L_t \left( x_t, u_t\right)\,dt+l\left( x_T\right) \right\vert \mathcal{F}_t^x\right\}\,,
     \label{e3-bos}
     \end{equation}
где $\mathcal{F}_t^x$~--- $\sigma$-ал\-геб\-ра, по\-рож\-ден\-ная~$x_\tau$, 
$0\hm\leq \tau\hm\leq t$, ${\sf E}\{\cdot\vert \mathcal{F}\}$~--- оператор условного 
математического ожидания относительно~$\mathcal{F}$. Соответственно, 
в~качестве достаточного условия оп\-ти\-маль\-ности воспользуемся уравнением 
динамического программирования
\begin{multline}
\fr{\partial V_t(x)}{\partial t} +\fr{1}{2}\sum\limits^n_{i,j=1} \sigma^2_{t_{ij}}
\fr{\partial^2 V_t(x)}{\partial x_i \partial x_j}+{}\\
{}+\min\limits_u\left[  
\sum\limits^n_{i=1} m_{t_i} \fr{\partial V_t(x)}{\partial x_i} + L_t(x,u)\right] 
=0\,,\\
V_T(x)=l(x)\,,
\label{e4-bos}
\end{multline}
где $m_{t_i}$~--- $i$-й элемент век\-тор-функ\-ции~$m_t(x,u)$; 
$\sigma^2_{t_{ij}} \hm= \sum\nolimits^m_{k=1} 
\sigma_{t_{ik}}\sigma_{t_{ki}}$, $\sigma_{t_{ij}}$~--- $i$-й по строке, $j$-й 
по столб\-цу элемент мат\-рич\-ной функции~$\sigma_t(x)$; $n$ и~$m$~--- 
размерности~$x_t$ и~$W_t$ соответственно.

     Традиционно в~рамках применения метода динамического 
программирования будем предполагать, что функции~$L_t$, $l$, $m_t$ 
и~$\sigma_t$ обеспечивают существование хотя бы одного решения 
уравнения~(\ref{e4-bos}), а~следовательно, и~оптимального 
управления~$u_t^*$, $0\hm\leq t\hm\leq T$, до\-став\-ля\-юще\-го минимум 
целевому функционалу~(\ref{e1-bos}). Задача оптимизации далее получается 
путем указания конкретных выражений для~$L_t$, $l$, $m_t$ и~$\sigma_t$.

\section{Постановка задачи управления выходом}

     Рассматриваемые далее случайные функции будут предполагаться 
скалярными. Такое упрощение позволит разгрузить выкладки и~итоговые 
выражения от не самых существенных деталей.
     
     Рассмотрим стохастическую дифференциальную сис\-те\-му, со\-сто\-яние 
которой представляет диффузи\-он\-ный процесс~$y_t$, описываемый 
нелинейным стохастическим дифференциальным уравнением Ито
     \begin{equation}
     dy_t=A_t\left( y_t\right) dt +\Sigma_t \left( y_t\right) dv_t\,,\enskip 
y_0=Y\,,
     \label{e5-bos}
     \end{equation}
где $v_t$~--- стандартный (одномерный) винеровский процесс; $Y$~--- 
случайная величина с~конечным вторым моментом; функции~$A_t$ 
и~$\Sigma_t$ удовлетворяют условиям Ито:
\begin{equation*}
\left\vert A_t(y)\right\vert +\left\vert \Sigma_t(y)\right\vert \leq C(1+\vert y\vert )\ 
\mbox{для\ всех } 0\leq t\leq T\,;
\end{equation*}

\vspace*{-12pt}

\noindent
\begin{multline*}
\hspace*{-2.10051pt}\left\vert A_t\left(y_1\right) -A_t \left( y_2\right) \right\vert +\left\vert 
\Sigma_t\left( y_1\right) -\Sigma_t \left(y_2\right)\right\vert \leq
C\left\vert y_1-y_2\right\vert\\
 \mbox{для\ всех\ } 0\leq t\leq T\ \mbox{и } 
y_1,y_2\in \mathbb{R}^1\,,
\end{multline*}
обеспечивающим существование единственного сильного (потраекторного) 
решения уравнения.
     
     Будем считать, что~$y_t$ описывает состояние некоторой 
динамической системы. Соответственно, поведение этой сис\-те\-мы опишем 
выходом, линейно связанным с~со\-сто\-янием:
     \begin{equation}
     dz_t=a_t y_t \,dt+ b_t z_t \,dt+ c_t u_t \,dt+\sigma_t \,dw_t\,,\enskip
     z_0=Z\,.
     \label{e6-bos}
     \end{equation}
Здесь $w_t$~--- не зависящий от~$v_t$, $Y$ и~$Z$ стандартный (одномерный) 
винеровский процесс; $Z$~--- случайная величина с~конечным вторым 
моментом; $u_t$~--- допустимое неупреждающее управ\-ле\-ние, качество 
которого определяется целевым функционалом следующего вида:
\begin{multline}
\!\hspace*{-3.98538pt}J\left( U_0^T\right) ={\sf E}\left\{ \int\limits_0^T \!\left( S_t\left( s_ty_t-g_t z_t -h_t 
u_t\right)^2 +G_t z_t^2+{}\right.\right.\\
\left.\left.{}+ H_t u_t^2
\vphantom{S_t\left( s_ty_t-g_t z_t -h_t 
u_t\right)^2}
\right) dt+S_T\left( s_T y_T -g_T 
z_T\right)^2+G_T z_T^2
\vphantom{\int\limits_0^T}\right\}\,,
\label{e7-bos}
\end{multline}
где $S_t$, $G_t$ и~$H_t$~--- неотрицательные функции\linebreak
$0\hm\leq t\hm\leq T$. 
Такой критерий отражает физический смысл задачи распределения ресурсов 
со\-глас\-но аналогичной~(\ref{e5-bos})--(\ref{e7-bos}) задаче для дис\-крет\-но\-го 
времени, рас\-смот\-рен\-ной в~\cite{7-bos}. В~част\-ности,  
функци\-онал~(\ref{e7-bos}) поз\-во\-ля\-ет ставить задачи отслеживания
 выходом 
со\-сто\-яния сис\-те\-мы, используя сла\-га\-емое $(y_t\hm- z_t)^2$, или 
управлением~--- линейной комбинации со\-сто\-яния и~выхода, сла\-га\-емое типа\linebreak 
$(y_t\hm+ z_t\hm- u_t)^2$. Поскольку задача формулируется 
в~предположении наличия пол\-ной информации о~со\-сто\-янии~$y_t$ 
и~выходе~$z_t$ (соответствующую $\sigma$-ал\-геб\-ру 
обозначим~$\mathcal{F}_t^{y,z}$), то допустимое управ\-ле\-ние ищется 
в~классе~$\mathcal{F}_t^{y,z}$-из\-ме\-ри\-мых неупреждающих функций 
(и,~как будет показано далее, оказывается управ\-ле\-ни\-ем с~обратной связью).

     Функции~$a_t$, $b_t$, $c_t$ и~$\sigma_t$ будем предполагать 
ограниченными: $\vert a_t\vert \hm+ \vert b_t\vert \hm+\vert c_t\vert \hm+ \vert 
\sigma_t \vert \hm\leq C$ для всех $0\hm\leq t\hm\leq T$, процесс  
управления~--- допустимым не\-упреж\-да\-ющим~\cite{9-bos}, обеспечивая, 
таким образом, существование сильного решения урав\-не\-ния~(\ref{e6-bos}) 
для любого допустимого управ\-ления.
     
     Задачу составляет поиск~$u_t^*$~--- допустимого управ\-ле\-ния, 
доставляющего минимум квад\-ра\-тич\-но\-му функционалу~$J(U_0^T)$.
      
     Поставленная задача очевидным образом формулируется в~терминах 
введенных выше в~(\ref{e1-bos})--(\ref{e3-bos}) обозначений, а~именно: 
     требуется обозначить
     \begin{gather*}
      x_t=\begin{pmatrix}
     y_t\\ z_t\end{pmatrix};\quad  m_t(x_t, u_t)=\begin{pmatrix}
     A_t(y_t)\\ a_t y_t +b_t z_t +c_t u_t\end{pmatrix};\\
     \sigma_t(x_t)= \begin{pmatrix}
     \Sigma_t(y_t)& 0\\
     0& \sigma_t\end{pmatrix};\quad W_t=\begin{pmatrix}
     v_t \\ w_t\end{pmatrix}
     %     \label{e8-bos}
     \end{gather*}
для записи уравнения со\-сто\-яния типа~(\ref{e2-bos}) и
\begin{align*}
L_t(x,u)&= L_t(y,z,u) ={}\\
&\hspace*{3mm}{}=S_t\left( s_t y-g_t z -h_t u\right)^2 +G_t z^2 +H_t  u^2\,;\\
l(x)&= l(y,z) =S_T \left( S_T y-g_T z\right)^2 +G_T z^2
%\label{e9-bos}
\end{align*}
для записи целевого функционала в~виде~(\ref{e1-bos}).

     Функция Беллмана~(\ref{e3-bos}) принимает вид 
     $V_t(x)\hm= V_t(y,z)$. Для записи со\-от\-вет\-ст\-ву\-юще\-го~(\ref{e4-bos}) 
уравнения Беллмана для~$V_t(y,z)$ заметим, что
     $$
     \left( \sigma^2_{t_{ij}}\right)_{i,j=1,2}= \begin{pmatrix}
     \Sigma_t^2(y) & 0\\
     0 & \sigma_t^2\end{pmatrix}\,.
     $$
     
     С~учетом перечисленных обозначений урав\-не\-ние динамического 
программирования~(\ref{e4-bos}) принимает вид:
     \begin{multline}
     \fr{\partial V_t(y,z)}{\partial t} +\fr{1}{2}\left( \Sigma_t^2(y) \fr{\partial^2 
V_t(y,z)} {\partial y^2}+\sigma_t^2\fr{\partial^2 V_t(y,z)} {\partial 
z^2}\right)+{}\\
    {}+\min\limits_u\! \left[ A_t(y) \fr{\partial V_t(y,z)}{\partial y}+\left( a_t 
y+b_t z+c_t u\right) \fr{\partial V_t(y,z)}{\partial z} +{}\right.\hspace*{-3pt}\\
\left.{}+ S_t\left( s_t y-g_t z-h_t 
u\right)^2+G_t z^2+H_t u^2
     \vphantom{\fr{\partial V_t(y,z)}{\partial y}}\right] =0\,,\\
     V_T(y,z)=S_T\left( s_T y-g_T z\right)^2+G_T z^2\,.
     \label{e10-bos}
     \end{multline}
     Это и~есть то самое уравнение, которое требуется решить: 
существование решения данного урав\-не\-ния суть достаточное условие 
оптимальности; оптимальное управ\-ле\-ние при этом~--- точ\-ка минимума 
со\-от\-вет\-ст\-ву\-юще\-го сла\-га\-емого.
     
\section{Динамическое программирование и~оптимальное 
управление}

     В рассматриваемой постановке линейность\linebreak выхода и~квадратичность 
критерия дают те же преимущества, что и~в~классической  
ли\-ней\-но-квад\-ра\-тич\-ной задаче управ\-ле\-ния~\cite{1-bos}, а~именно: 
позволяют сразу определить вид оптимального управ\-ле\-ния и~фактические 
условия его существования. Действительно, со\-хра\-няя в~(\ref{e10-bos}) под 
знаком $\min\nolimits_u$ только члены, зависящие от~$u$, получаем
     \begin{multline*}
     \fr{\partial V_t(y,z)}{\partial t} +\fr{1}{2}\left( \Sigma_t^2(y) \fr{\partial^2 
V_t(y,z)} {\partial y^2}+\sigma_t^2\fr{\partial^2 V_t(y,z)} {\partial 
z^2}\right)+{}\\
     {}+A_t(y)\fr{\partial V_t(y,z)}{\partial y}+\left( a_t y+b_t z\right) 
\fr{\partial V_t(y,z)}{\partial z}+{}\\
{}+S_t\left( s_t y-g_t z\right)^2 +G_t z^2+{}
\end{multline*}

\noindent
\begin{multline*}
     {}+\min\limits_u \left[ \left( c_t \fr{\partial V_t(y,z)}{\partial z}-2S_t \left( 
s_t y-g_t z\right) h_t\right)u +{}\right.\\
\left.{}+\left( S_t h_t^2+H_t\right) u^2
\vphantom{\fr{\partial V_t(y,z)}{\partial z}}
\right]=0\,,
     %\label{e11-bos}
     \end{multline*}
откуда в~предположении $S_t h_t^2\hm+ H_t\hm>0$ следует, что существует 
оптимальное управ\-ле\-ние, которое определяется равенством
\begin{multline}
u_t^* = u_t^*(y,z)=-\fr{1}{2}\left( S_t h_t^2 +H_t\right)^{-1} \left( c_t 
\fr{\partial V_t(y,z)}{\partial z}-{}\right.\\
\left.{}-2S_t\left( s_t y-g_t z\right) h_t
\vphantom{\fr{\partial V_t(y,z)}{\partial z}}
\right)
\label{e12-bos}
\end{multline}
и доставляет минимум соответствующему сла\-га\-емо\-му в~урав\-не\-нии Беллмана, 
равный
$-\left( S_t h_t^2\hm+\right.$\linebreak
$\left.{}+H_t\right)^{-1} \left( c_t 
{\partial V_t(y,z)}/{\partial 
z}\hm-2S_t\left( s_t y \hm-g_t z\right) h_t \right)^2/4.
$ 
     
     Отметим, что, как и~в~классической ли\-ней\-но-квад\-ра\-тич\-ной 
задаче, управ\-ле\-ние из класса до\-пус\-ти\-мых не\-упреж\-да\-ющих получилось 
управ\-ле\-ни\-ем с~обратной связью.
     
     Таким образом, функция Беллмана описывается сле\-ду\-ющим 
дифференциальным уравнением:
     \begin{multline}
     \fr{\partial V_t(y,z)}{\partial t} +\fr{1}{2}\left( \Sigma_t^2(y) \fr{\partial^2 
V_t(y,z)} {\partial y^2}+\sigma_t^2\fr{\partial^2 V_t(y,z)} {\partial 
z^2}\right)+{}\\
     {}+ A_t(y) \fr{\partial V_t(y,z)}{\partial y}+\left( a_t y+b_t z\right) 
\fr{\partial V_t(y,z)}{\partial z}+{}\\
{}+ S_t \left( s_t y- g_t z\right)^2 +G_t z^2-
 \fr{1}{4}\left( S_t h_t^2+H_t\right)^{-1}\times{}\\
 {}\times \left( c_t \fr{\partial V_t(y,z)} 
{\partial z}-2S_t\left( s_t y -g_t z\right) h_t \right)^2=0\,.
     \label{e13-bos}
     \end{multline}
     
     Возводя в~квадрат по\-след\-нее сла\-га\-емое в~(\ref{e13-bos}), перепишем 
его в~виде:
     \begin{multline}
     \fr{\partial V_t(y,z)}{\partial t} +\fr{1}{2}\left( \Sigma_t^2(y) \fr{\partial^2 
V_t(y,z)} {\partial y^2}+\sigma_t^2\fr{\partial^2 V_t(y,z)} {\partial 
z^2}\!\right)+{}\\
{}+A_t(y) \fr{\partial V_t(y,z)}{\partial y}
+ \left( 
\vphantom{\left( S_t h_t^2 +H_t\right)^{-1}}
a_t y+b_t z+{}\right.\\
\left.{}+\left( S_t h_t^2 +H_t\right)^{-1}
 c_t S_t \left( s_t y-g_t z\right) h_t
\right) 
     \fr{\partial V_t(y,z)}{\partial z}+{}\\
     {}+\left( S_t-\left( S_t h_t^2 +H_t\right)^{-1} S_t^2 h_t^2\right)\left( s_t y -
g_t z\right)^2+{}\\
     \!\!{}+
     G_t z^2 -\fr{1}{4}\left( S_t h_t^2+H_t\right)^{-1}\! c_t^2
     \left(\fr{\partial V_t(y,z)}{\partial z}\right)^{\!2}=0\,.\!\!
     \label{e14-bos}
     \end{multline}
     
     Рассматривая полученное уравнение, заметим, что его решение может 
быть пред\-став\-ле\-но в~виде:
   \begin{equation}
     V_t(y,z)= \alpha_t z^2+\beta_t(y) z +\gamma_t(y)\,,
     \label{e15-bos}
     \end{equation}
т.\,е.\ будем искать решение при дополнительном предположении 
о~квад\-ра\-тич\-ности функции Белл\-ма\-на по переменной~$z$, и~сведем, таким 
образом, поиск оптимального решения к~уравнениям относительно функций 
$\alpha_t$, $\beta_t(y)$ и~$\gamma_t(y)$. Отметим сразу, что явный вид 
функции~$\gamma_t(y)$ для реализации оптимального управ\-ле\-ния не 
требуется, однако далее будет предложен вариант вы\-чис\-ле\-ния и~этой 
функции, что пред\-став\-ля\-ет\-ся небесполезным, поскольку позволит выполнять 
расчет минимума целевого функционала. Источником для 
предложения~(\ref{e15-bos}) является уже упоминавшаяся аналогичная 
задача для случая дис\-крет\-но\-го времени~\cite{7-bos, 8-bos}. В~той задаче 
выражение для функции Беллмана получается формально без 
дополнительных усилий. При этом форма~(\ref{e15-bos}) обнаруживается 
как свойство оптимального решения. В~рассматриваемом случае 
непрерывного времени~(\ref{e15-bos}) постулируется, а~пра\-виль\-ность 
постулата под\-тверж\-да\-ет\-ся далее ре\-зуль\-ти\-ру\-ющи\-ми уравнениями 
для~$\alpha_t$, $\beta_t(y)$ и~$\gamma_t(y)$ Кроме того, данное 
предположение пред\-став\-ля\-ет\-ся вы\-те\-ка\-ющим из линейной структуры задачи 
в~отношении переменной~$z$, в~част\-ности, тем фактом, что такой вид 
функции Беллмана обеспечивает линейность оптимального 
управ\-ле\-ния~(\ref{e12-bos}) по~$z$.

     Граничное условие при выбранном предположении~(\ref{e15-bos}) 
принимает вид:

\noindent
     \begin{multline*}
     V_T(y,z)= S_T\left( s_T y- g_T z\right)^2+G_T z^2 ={}\\[-0.5pt]
     {}=\alpha_T z^2 
+\beta_T(y) z +\gamma_T(y)\,,
    \end{multline*}
т.\,е.

\noindent
\begin{align*}
\alpha_T&= S_T g_T^2 +G_T\,;\\[-0.5pt]
\beta_T(y)&=-2S_T s_T g_T y\,;\\[-0.5pt]
\gamma_T(y)&=S_T s_T^2 y^2\,.
%\label{e16-bos}
\end{align*}
          При этом само оптимальное управ\-ле\-ние, определенное 
выражением~(\ref{e12-bos}), оказывается управ\-ле\-ни\-ем с~обратной связью 
по~$y_t$ и~$z_t$:

\noindent
     \begin{multline}
     u_t^*=u_t^*(y,z) ={}\\[-0.5pt]
     {}=
     -\fr{1}{2}\left( S_t h_t^2 +H_t\right)^{-1}
     \left( c_t \left( 2\alpha_t z +\beta_t(y)\right) +{}\right.\\[-0.5pt]
    \left. {}+2S_t\left( s_t y-g_t z\right) 
h_t\right)\,.
     \label{e17-bos}
     \end{multline}
          Подставляем $V_t(y,z)\hm= \alpha_t z^2 \hm+ \beta_t(y) 
z\hm+\gamma_t(y)$ в~(\ref{e14-bos}):

\noindent
     \begin{multline*}
     \fr{\partial \alpha_t}{\partial t}\, z^2 +
     \fr{\partial \beta_t(y)}{\partial t}\,z +
     \fr{\partial \gamma_t(y)}{\partial t}+{}\\[-0.5pt]
     {}+\fr{1}{2}\left( \Sigma_t^2(y) \left( 
\fr{\partial^2\beta_t(y)}{\partial y^2}\,z +\fr{\partial^2 \gamma_t(y)}{\partial 
y^2}\right) +2\sigma_t^2\alpha_t\right)+{}\\[-0.5pt]
 {}+A_t(y)\left(\fr{\partial \beta_t(y)}{\partial y}\,z + \fr{\partial 
\gamma_t(y)}{\partial y}\right) +{}\\[-0.5pt]
\hspace*{-0.22987pt}{}+\left( a_t y+b_t z+\left( S_t h_t^2 +H_t\right)^{-1} c_t S_t \left( s_t y-
g_t z\right) h_t\right)\times{}
\end{multline*}

\noindent
\begin{multline*}
         {}\times \left( 2\alpha_t z+\beta_t(y)\right)+{}\\
     {}+\left( S_t-\left( S_t h_t^2 +H_t\right)^{-1} S_t^2 h_t^2\right)\left( s_t y-
g_t z\right)^2+{}\\
     {}+ G_t z^2 -\fr{1}{4}\left( S_t h_t^2 +H_t\right)^{-1} c_t^2 \left( 
2\alpha_t z+\beta_t(y)\right)^2=0\,.
     \end{multline*}
          Далее выделяем слагаемые при~$z^2$, $z$ и~$z^0$
          
          \noindent
     \begin{multline*}
     \fr{\partial \alpha_t}{\partial t}\, z^2 +\fr{\partial \beta_t(y)}{\partial t}\,z +
     \fr{\partial \gamma_t(y)}{\partial 
t}+\fr{1}{2}\,\Sigma_t^2(y)\fr{\partial^2\beta_t(y)}{\partial y^2}\,z+ {}\\
{}+
\fr{1}{2}\,\Sigma_t^2(y)\fr{\partial^2\gamma_t(y)}{\partial 
y^2}+\sigma_t^2\alpha_t+A_t(y)\fr{\partial \beta_t(y)}{\partial y}\,z +{}\\
{}+A_t(y) \fr{\partial 
\gamma_t(y)}{\partial y}+{}\\
{}+ 2\alpha_t \left( b_t -\left( S_t h_t^2+H_t\right)^{-1} c_t 
S_t h_t g_t \right)z^2+{}\\
     {}+
     \left( 2\alpha_t\left( \alpha_t+\left( S_t h_t^2+H_t\right)^{-1} c_t S_t h_t 
s_t\right)y +{}\right.\\
\left.{}+\beta_t(y) \left( b_t-\left( S_t h_t^2+H_t\right)^{-1} c_t S_t h_t 
g_t\right) \right) z+{}\\
     {}+\beta_t(y)\left( a_t +\left( S_t h_t^2+H_t\right)^{-1} c_t S_t h_t s_t\right) 
y+{}\\
{}+ \left( S_t -\left( S_t h_t^2+H_t\right)^{-1} S_t^2 h_t^2\right) g_t^2 z^2-{}\\
     {}- 2\left( S_t -\left( S_t h_t^2+H_t\right)^{-1} S_t^2 h_t^2\right) s_t g_t yz 
+{}\\
{}+
     \left( S_t-\left( S_t h_t^2+H_t\right)^{-1} S_t^2 h_t^2\right) s_t^2 y^2+{}\\
     {}+G_t z^2 -\left( S_t h_t^2 +H_t\right)^{-1} c_t^2 \alpha_t^2 z^2 -{}\\
     {}-\left( 
S_t h_t^2+H_t\right)^{-1} c_t^2 \alpha_t \beta_t(y) z-{}\\
{}-
\fr{1}{4}\left( S_t h_t^2+H_t\right)^{-1}  c_t^2 \beta_t^2(y)=0\,,
     \end{multline*}
группируем их и~получаем сле\-ду\-ющие уравнения:
\begin{itemize}
\item  для~$\alpha_t$:

\noindent
\begin{multline}
\fr{\partial\alpha_t}{\partial t}+2\alpha_t\left( b_t-\left( S_t h_t^2+H_t\right)^{-1} c_t 
S_t h_t g_t\right)+{}\\
{}+ \left( S_t- \left( S_t h_t^2+H_t\right)^{-1} S_t^2 h_t^2\right) 
g_t^2+G_t-{}\\
\hspace*{-8mm}{}-\left( S_t h_t^2+H_t\right)^{-1} c_t^2 \alpha_t^2 =0\,,\enskip \alpha_T=S_T 
g_t^2+G_T\,;\!\!
\label{e18-bos}
\end{multline}
\item для $\beta_t$:

\noindent
\begin{multline}
\fr{\partial\beta_t(y)}{\partial 
t}+\fr{1}{2}\,\Sigma_t^2(y)\fr{\partial^2\beta_t(y)}{\partial y^2} 
+A_t(y)\fr{\partial \beta_t(y)}{\partial y}+{}\\
{}+ 2\alpha_t\left( a_t +\left( S_t h_t^2+H_t\right)^{-1} c_t S_t h_t s_t\right) y+{}\\
{}+
\beta_t(y)\left( b_t -\left( S_t h_t^2 +H_t\right)^{-1} c_t S_t h_t g_t\right)-{}\\
{}-2\left( S_t-\left( S_t h_t^2+H_t\right)^{-1} S_t^2 h_t^2\right) s_t g_t y-{}
\\
{}-
\left( S_t h_t^2+H_t\right)^{-1} c_t^2 \alpha_t \beta_t(y)=0\,,\\
\beta_T(y)=-2S_T s_T g_T y\,;
\label{e19-bos}
\end{multline}
\item  для $\gamma_t$:
\begin{multline}
\hspace*{-0.8pt}\fr{\partial \gamma_t(y)}{\partial t}+\fr{1}{2}\,\Sigma_t^2(y)
\fr{\partial^2 \gamma_t(y)}{\partial y^2} +\sigma_t^2 \alpha_t +A_t(y)
\fr{\partial \gamma_t(y)}{\partial y}+{}\\
{}+ \beta_t(y)\left( a_t +\left( S_t h_t^2+H_t\right)^{-1} c_t S_t h_t s_t\right) y+{}\\
{}+
\left( S_t-\left( S_t h_t^2+H_t\right)^{-1} S_t^2 h_t^2\right)  s_t^2 y^2-{}\\
{}-\fr{1}{4}\left( S_t h_t^2+H_t\right)^{-1} c_t^2 \beta_t^2(y) =0\,,\\
\gamma_T(y)=S_T s_T^2 y^2\,.
\label{e20-bos}
\end{multline}
\end{itemize}
     
     Уравнение~(\ref{e18-bos}), легко заметить, является уравнением 
Риккати, которое в~силу сформулированного выше условия   
имеет единственное неотрицательное решение для всех $0\hm\leq t\hm\leq T$. 
Этот факт требует дополнительного комментария. Для получения 
уравнения~(\ref{e18-bos}) рас\-смот\-рим исходную задачу при дополнительных 
условиях $a_t\hm=0$ и~$s_t\hm=0$ для всех $0\hm\leq t\hm\leq T$. Нетрудно 
видеть, что эти условия рассматриваемую по\-ста\-нов\-ку сводят фактически 
к~классической ли\-ней\-но-квад\-ра\-тич\-ной задаче. Имеющуюся 
в~рассматриваемой формулировке чуть более общую форму целевой 
функции (принципиального значения это обобщение, конечно, не имеет) 
сведем к~классической еще одним предположением: $S_t\hm=0$ для всех 
$0\hm\leq t\hm\leq T$. Теперь уравнение для~$\alpha_t$ принимает хорошо 
известный вид:
     \begin{equation}
     \fr{\partial \alpha_t}{\partial t}+2\alpha_t b_t +G_t- H_t^{-1} c_t^2 
\alpha_t^2=0\,,\enskip \alpha_T=G_T\,.
     \label{e21-bos}
     \end{equation}

     В таком случае, как известно~\cite{10-bos}, существует единственное 
оптимальное управление~--- линейное с~обратной связью по выходу~$z_t$, 
с~коэффициентом усиления, опи\-сы\-ва\-емым уравнением  
Риккати~(\ref{e21-bos}). Именно этот результат дают  
уравнения~(\ref{e18-bos})--(\ref{e20-bos}) и~описываемая ими функция 
Беллмана~(\ref{e15-bos}), так как из $a_t\hm=0$ и~$s_t\hm=0$ немедленно 
следует, что $\beta_t(y)\hm=0$, откуда, в~свою очередь, с~учетом 
не\-за\-ви\-си\-мости решения от~$y_t$ следует, что $\gamma_t(y)\hm=\gamma_t$, 
т.\,е.\ не зависит от~$y$ и~задается уравнением: 
     $$
     \fr{\partial \gamma_t(y)}{\partial t} +\sigma^2_t \alpha_t=0\,,\enskip 
\gamma_T=0\,.
     $$ 
     Оптимальное управ\-ле\-ние при этом 
     $$
     u_t^*= -H_t^{-1} c_t \alpha_t z_t\,,
     $$
      т.\,е.\ все полностью совпадает с~известным классическим решением.
     
     С уравнениями~(\ref{e19-bos}) и~(\ref{e20-bos}) ситуация, естественно, 
обстоит сложнее. Это линейные уравнения второго порядка параболического 
типа, поскольку\linebreak
 $\Sigma_t^2(y)\hm>0$. Фактически отсутствуют 
конструктивные условия, гарантирующие существование их\linebreak
 решений 
(требовать, чтобы все фигурирующие в~уравнениях коэффициенты были 
представлены аналитическими функциями на всем пространстве значений, 
вряд ли целесообразно), поэтому далее будем предполагать, что данные 
уравнения имеют на рас\-смат\-ри\-ва\-емом интервале $0\hm\leq t\hm\leq T$ хотя 
бы одно ограниченное решение и~именно эти условия будем рас\-смат\-ри\-вать 
как достаточные условия существования оптимального решения 
рассматриваемой задачи.
     
     Таким образом, доказана следующая тео\-рема.
     
     \smallskip
     
     \noindent
     \textbf{Теорема.}\ \textit{Пусть для диффузионного 
процесса}~(\ref{e5-bos}) \textit{выполнены условия Ито, для 
     процесса}~(\ref{e6-bos})~--- \textit{ограничены коэффициенты, 
уравнения}~(\ref{e18-bos})--(\ref{e20-bos}) \textit{имеют ограниченные 
решения для $0\hm\leq t\hm\leq T$. Тогда минимум  
функционалу}~(\ref{e7-bos}) \textit{доставляет оптимальное 
управ\-ле\-ние}~(\ref{e17-bos}), \textit{где} $y\hm= y_t$; $z\hm=z_t$.
     
\section{Заключение}

     Рассмотренная задача оптимизации в~целом близка и~по модели, и~по 
критерию к~классической ли\-ней\-но-квад\-ра\-тич\-ной постановке. 
Принципиальным отличием является нелинейная модель для описания 
со\-сто\-яния динамической сис\-те\-мы, в~которой отсутствует управ\-ля\-ющее 
воздействие.\linebreak
 Такую модель наряду с~традиционной интер\-пре\-тацией  
<<со\-сто\-яние--вы\-ход>> мож\-но понимать как\linebreak модель неконтролируемого 
слож\-но\-го внешнего воздействия. Небольшое дополнительное отличие дает 
предложенная форма квад\-ра\-тич\-но\-го критерия, поз\-во\-ля\-ющая, в~част\-ности, 
ставить такие задачи, как отслеживание выходом или управ\-ле\-ни\-ем со\-сто\-яния 
сис\-те\-мы или ее выхода.
     
     Поскольку обсуждать возможности точного решения уравнений, 
определяющих оптимальное управ\-ле\-ние, не имеет смыс\-ла, наиболее 
актуальной далее является задача их приближенного чис\-лен\-но\-го решения 
и~анализа воз\-мож\-ности практической реализации. Этому посвящена вторая 
часть данной работы, пла\-ни\-ру\-емая к~выходу в~ближайшее время.

{\small\frenchspacing
 {%\baselineskip=10.8pt
 \addcontentsline{toc}{section}{References}
 \begin{thebibliography}{99}
\bibitem{1-bos}
\Au{Athans M.} Editorial on the LQG problem~// IEEE~T. Automat. Contr., 1971. Vol.~16. 
No.\,6. P.~528--552. doi: 10.1109/TAC.1971.1099845.
\bibitem{2-bos}
\Au{Wu Z.} Forward-backward stochastic differential equations, linear quadratic stochastic 
optimal control and nonzero sum differential games~// J.~Syst. Sci. Complex., 2005. Vol.~18. 
No.\,2. P.~179--192.
\bibitem{3-bos}
\Au{Chen B.\,S., Zhang~W.} Stochastic H2/H1 control with state-dependent noise~// IEEE 
T.~Automat. Contr., 2004. Vol.~49. No.\,1. P.~45--56. doi: 10.1109/TAC.2003.821400.
\bibitem{4-bos}
\Au{Bohacek S.} A~stochastic model of TCP and fair video transmission~// IEEE 
INFOCOM, 2003. Vol.~2. P.~1134--1144. doi: 10.1109/INFCOM.2003.1208950.
\bibitem{5-bos}
\Au{Домбровский В.\,В., Объедко~Т.\,Ю.} Управление с~прогнозированием системами 
с~марковскими скачками при ограничениях и~применение к~оптимизации 
инвестиционного портфеля~// Автомат. телемех., 2011. №\,5. С.~96--112. doi: 
10.1134/S0005117911050079.
\bibitem{6-bos}
\Au{Баландин Д.\,В., Коган~М.\,М.} Оптимальное линейно-квад\-ра\-тич\-ное управление: от 
матричных уравнений к~линейным матричным неравенствам~// Автомат. телемех., 2011. 
№\,11. С.~60--69. doi: 10.1134/ S0005117911110038.
\bibitem{7-bos}
\Au{Босов А.\,В.} Обобщенная задача распределения ресурсов программной системы~// 
Информатика и~её применения, 2014. Т.~8. Вып.~2. С.~39--47. doi: 
10.14357/19922264140204.
\bibitem{8-bos}
\Au{Босов А.\,В.} Управление линейным выходом дискретной стохастической системы по 
квадратичному критерию~// Изв. РАН. Теория и~системы управления, 2016. №\,3.  
С.~19--35. doi: 10.1134/S1064230716030060.
\bibitem{9-bos}
\Au{Флеминг У., Ришел~Р.} Оптимальное управление детерминированными 
и~стохастическими системами~/ Пер. с~англ.~--- М.: Мир, 1978. 316~с. 
(\Au{Fleming~W.\,H., Rishel~R.\,W.} Deterministic and stochastic optimal control.~--- New 
York, NY, USA: Springer-Verlag, 1975. 222~p.)
\bibitem{10-bos}
\Au{Девис М.\,Х.\,А.} Линейное оценивание и~стохастическое управление~/ Пер. с~англ.~--- 
М.: Наука, 1984. 206~с. (\Au{Davis~M.\,H.\,A.} Linear estimation and stochastic control.~--- 
London: Chapman and Hall, 1977. 224~p.)

 \end{thebibliography}

 }
 }

\end{multicols}

\vspace*{-6pt}

\hfill{\small\textit{Поступила в~редакцию 30.03.18}}

\vspace*{4pt}

%\newpage

%\vspace*{-24pt}

\hrule

\vspace*{2pt}

\hrule

\vspace*{-2pt}


\def\tit{STOCHASTIC DIFFERENTIAL SYSTEM OUTPUT CONTROL 
BY~THE~QUADRATIC CRITERION.~I.~DYNAMIC\\ PROGRAMMING 
OPTIMAL SOLUTION}


\def\titkol{Stochastic differential system output control 
by~the~quadratic criterion. I.~Dynamic programming 
optimal solution}

\def\aut{A.\,V.~Bosov and~A.\,I.~Stefanovich}

\def\autkol{A.\,V.~Bosov and~A.\,I.~Stefanovich}

\titel{\tit}{\aut}{\autkol}{\titkol}

\vspace*{-11pt}


\noindent
Institute of Informatics Problems, Federal Research Center ``Computer Science 
and Control'' of the Russian Academy of Sciences, 44-2~Vavilov Str., Moscow 
119333, Russian Federation


\def\leftfootline{\small{\textbf{\thepage}
\hfill INFORMATIKA I EE PRIMENENIYA~--- INFORMATICS AND
APPLICATIONS\ \ \ 2018\ \ \ volume~12\ \ \ issue\ 3}
}%
 \def\rightfootline{\small{INFORMATIKA I EE PRIMENENIYA~---
INFORMATICS AND APPLICATIONS\ \ \ 2018\ \ \ volume~12\ \ \ issue\ 3
\hfill \textbf{\thepage}}}

\vspace*{3pt}



\Abste{The problem of optimal control for the Ito diffusion 
process and a~controlled linear output is solved. The considered 
statement is close to the classical linear-quadratic Gaussian 
control  (LQG control) problem. Differences consist in the fact 
that the state is described by the nonlinear differential Ito equation  $dy_y = A_t(y_t) 
\,dt+\Sigma_t(y_t)\,dv_t$ and does not depend on the control~$u_t$, 
optimization subject is controlled linear output 
 $dz_t=a_ty_t\,dt +b_tz_t\,dt +c_t u_t\,dt +\sigma_t \,dw_t$. 
Additional generalizations are included in the quadratic 
quality criterion for the purpose of statement such problems 
as state tracking by output or a linear combination of state 
and output tracking by control. The method of dynamic programming 
is used for the solution. 
The assumption about Bellman function in the form  $V_t(y,z)= \alpha_t 
z^2+\beta_t(y) z+\gamma_t(y)$ allows one to find it. 
Three differential equations for the coefficients $\alpha_t$,  $\beta_t(y)$,
and $\gamma_t(y)$ give the solution. 
These equations constitute the optimal solution of the problem under consideration.}

\KWE{stochastic differential equation; optimal control; dynamic programming; 
Bellman function; Riccati equation; linear differential equations of parabolic type}


\DOI{10.14357/19922264180314}

\vspace*{-12pt}

\Ack
\noindent
This work was partially supported by the Russian Science Foundation (grant  
16-07-00677).



%\vspace*{6pt}

  \begin{multicols}{2}

\renewcommand{\bibname}{\protect\rmfamily References}
%\renewcommand{\bibname}{\large\protect\rm References}

{\small\frenchspacing
 {%\baselineskip=10.8pt
 \addcontentsline{toc}{section}{References}
 \begin{thebibliography}{99}
\bibitem{1-bos-1}
\Aue{Athans, M.} 1971. Editorial on the LQG problem. \textit{IEEE~T. 
Automat. Contr.} 16(6):528--552. doi: 10.1109/ TAC.1971.1099845.
\bibitem{2-bos-1}
\Aue{Wu, Z.} 2005. Forward-backward stochastic differential equations, linear 
quadratic stochastic optimal control and\linebreak\vspace*{-12pt}

\columnbreak

\noindent
 nonzero sum differential games. 
\textit{J.~Syst. Sci. Complex.} 18(2):179--192.
\bibitem{3-bos-1}
\Aue{Chen, B.\,S. and W.~Zhang.} 2004. Stochastic H2/H1 control with  
state-dependent noise. \textit{IEEE~T. Automat. Contr.} 49(1):45--56.
doi: 10.1109/TAC.2003.821400.
\bibitem{4-bos-1}
\Aue{Bohacek, S.} 2003. A~stochastic model of TCP and fair video 
transmission. \textit{IEEE INFOCOM}. 2:1134--1144.
doi: 10.1109/INFCOM.2003.1208950.
\bibitem{5-bos-1}
\Aue{Dombrovskii, V.\,V., and T.\,Yu.~Ob''edko.} 2011. Predictive control of 
systems with Markovian jumps under constraints and its application to the 
investment portfolio optimization. \textit{Automat. Rem. Contr.}  
72(5):989--1003.
\bibitem{6-bos-1}
\Aue{Balandin, D.\,V., and M.\,M.~Kogan.} 2011. Optimal linear-quadratic 
control: From matrix equations to linear matrix inequalities. \textit{Automat. 
Rem. Contr.} 72(11):2276--2284.
\bibitem{7-bos-1}
\Aue{Bosov, A.\,V.} 2014. Obobshchennaya zadacha raspredeleniya resursov 
programmnoy sistemy [The generalized problem of software system resources 
distribution]. \textit{Informatika i~ee Primeneniya~--- Inform. Appl.}  
8(2):39--47. doi: 
10.14357/19922264140204.
\bibitem{8-bos-1}
\Aue{Bosov, A.\,V.} 2016. Discrete stochastic system linear output control 
with respect to a quadratic criterion. \textit{J.~Comput. Syst. Sc. 
Int.} 55(3):349--364.
\bibitem{9-bos-1}
\Aue{Fleming, W.\,H., and R.\,W.~Rishel.} 1975. \textit{Deterministic and 
stochastic optimal control.} New York, NY: Springer-Verlag. 222~p.
\bibitem{10-bos-1}
\Aue{Davis, M.\,H.\,A.} 1977. \textit{Linear estimation and stochastic 
control.} London: Chapman and Hall. 224~p.
\end{thebibliography}

 }
 }

\end{multicols}

\vspace*{-6pt}

\hfill{\small\textit{Received March 30, 2018}}

%\pagebreak

%\vspace*{-18pt}
     
     \Contr
     
       \noindent
       \textbf{Bosov Alexey V.} (b.\ 1969)~--- Doctor of Science in technology, 
principal scientist, Institute of Informatics Problems, Federal Research 
Center ``Computer Science and Control'' of the Russian Academy of Sciences, 
44-2~Vavilov Str., Moscow 119333, Russian Federation; 
\mbox{AVBosov@ipiran.ru}
       
       \vspace*{3pt}
       
       \noindent
       \textbf{Stefanovich Alexey I.} (b.\ 1983)~--- principal specialist, 
Institute of Informatics Problems, Federal Research Center ``Computer Science 
and Control'' of the Russian Academy of Sciences, 44-2~Vavilov Str., Moscow 
119333, Russian Federation; \mbox{AStefanovich@frccsc.ru}
\label{end\stat}

\renewcommand{\bibname}{\protect\rm Литература}       

       %1
\def\yhxt{({\hat X}_t,Y_t, t)}
\def\hx{\hat X}
\def\yutt{(Y_t, {\hat X}_t,t)}
\def\yxtt{(X_t,Y_t, t)}
\renewcommand\mm{{\sf M}}


\def\stat{SIN-2}

\def\tit{НОРМАЛЬНЫE  УСЛОВНО-ОПТИМАЛЬНЫЕ
ФИЛЬТРЫ И~ЭКСТРАПОЛЯТОРЫ ПУГАЧЁВА
 ДЛЯ~СТОХАСТИЧЕСКИХ СИСТЕМ, 
 ЛИНЕЙНЫХ ОТНОСИТЕЛЬНО СОСТОЯНИЯ}

\def\titkol{Нормальныe  условно-оптимальные
фильтры и~экстраполяторы Пугачёва
 для стохастических систем} %,  линейных относительно состояния}

\def\aut{И.\,Н.~Синицын$^1$, Э.\,Р.~Корепанов$^2$}

\def\autkol{И.\,Н.~Синицын, Э.\,Р.~Корепанов}

\titel{\tit}{\aut}{\autkol}{\titkol}

\index{Синицын И.\,Н.}
\index{Корепанов Э.\,Р.}
\index{Sinitsyn I.\,N.}
\index{Korepanov E.\,R.}

%{\renewcommand{\thefootnote}{\fnsymbol{footnote}} \footnotetext[1]
%{Исследование частично поддержано РФФИ (проекты 16-07-00677 
%и~15-37-20611-мол\_а\_вед).}}


\renewcommand{\thefootnote}{\arabic{footnote}}
\footnotetext[1]{Институт проблем информатики Федерального исследовательского
центра <<Информатика и~управление>> Российской академии наук, sinitsin@dol.ru}
\footnotetext[2]{Институт проблем информатики Федерального исследовательского
центра <<Информатика и~управление>> Российской академии наук, ekorepanov@ipiran.ru}


\Abst{Рассматривается теория аналитического синтеза непрерывных (дифференциальных) 
и~дискретных (разностных) суб-  и~услов\-но-оп\-ти\-маль\-ных фильтров и~экстраполяторов 
Пугачёва для обработки процессов в~гауссовских и~негауссовских стохастических
 системах (СтС), 
линейных относительно вектора состояния. Первые работы по фильтрации и~экстраполяции 
для таких гауссовских систем были выполнены Липцером и~Ширяевым, а для негауссовских~--- 
Пугачёвым и~Синицыным. Приведены алгоритмы нормальных суб- 
и~услов\-но-оп\-ти\-маль\-ных фильтров для непрерывных
и~дискретных систем. Пред\-став\-ле\-ны алгоритмы нормальных суб- и~услов\-но-оп\-ти\-маль\-ных 
экстраполяторов. Разработанные алгоритмы положены в~основу программного обеспечения 
(StS-Filter, 2016). Результаты допускают развитие на случай автокоррелированных шумов 
в~наблюдениях, а~так\-же систем с~мультипликативными шумами.}

\KW{дискретные СтС; дифференциальная СтС;
метод нормальной аппроксимации (МНА) апостериорной плотности;
метод статистической линеаризации (МСЛ);
нормальный услов\-но-оп\-ти\-маль\-ный фильтр  Пугачёва (НФП);
нормальный услов\-но-оп\-ти\-маль\-ный экстраполятор Пугачёва (НЭП);
стохастическая система (СтС); СтС, линейная относительно состояния;
условия Лип\-це\-ра--Ширяева; фильтр Лип\-це\-ра--Ши\-ряева (ФЛШ)}

\DOI{10.14357/19922264160202} 

\vspace*{-6pt}

\vskip 10pt plus 9pt minus 6pt

\thispagestyle{headings}

\begin{multicols}{2}

\label{st\stat}

\section{Введение}

Многие практические задачи обработки информации в~статистических научных 
исследованиях основаны на использовании теории фильтрации процессов в~СтС, 
линейных относительно состояния~[1--5]. 

Первые работы в~этом 
направлении для гауссовских СтС выполнены Липцером и~Ширяевым~\cite{2-s2}, 
а~для негауссовских СтС на основе субоптимальной фильтрации~--- 
Пугачёвым и~Синицыным~\cite{3-s2, 4-s2}.
В~\cite{1-s2} рассмотрены вопросы синтеза алгоритмов нормальных 
услов\-но-оп\-ти\-маль\-ных 
фильтров  Пугачёва (НФП) для обработки информации в~дифференциальных 
гауссовских и~негауссовских СтС, линейных относительно состояния 
(условия Лип\-це\-ра--Ши\-ря\-ева). Особое внимание уделено синтезу НФП для СтС 
при условиях Лип\-це\-ра--Ши\-ря\-ева
 на основе аппроксимации апостериорного распределения нормальным субоптимальным 
 квазилинейным НФП, основанным на статистической линеаризации нелинейных функций, 
 зависящих от наблюдений. Для СтС высокой размерности  путем выбора структурных 
 функций, отражающих аналитическую природу наблюдаемой системы, синтезированы НФП, 
 прос\-тые в~компьютерной реализации и~ способные работать в~режиме реального времени. 
 Алгоритмы положены 
 в~основу модуля инструментального программного обеспечения (StS-Filter, 2016).

Настоящая статья  посвящена вопросам синтеза непрерывных и~дискретных нормальных 
фильтров и~экстраполяторов Пугачёва (НФП и~НЭП) для непрерывных и~дискретных СтС, 
линейных относительно состояния. 
В~разд.~2 приведены уравнения,\linebreak описывающие 
процессы дифференциальных и~разностных СтС. Раздел~3 для гауссовских СтС посвящен 
нормальным фильтрам (НФ) на основе мето-\linebreak дов нормальной 
аппроксимации (МНА) и~метода\linebreak 
статистической линеаризации (МСЛ). В~разд.~4 приводятся алгоритмы НФП на основе 
тео\-рии услов\-но-оп\-ти\-маль\-ной фильтрации Пугачёва. Для дифференциальных 
негауссовских СтС алгоритмы НФП даны в~разд.~5. Обобщение НФ для дискретных СтС 
приведено в~разд.~6. Алгоритмы нормальных услов\-но-оп\-ти\-маль\-ных 
экстраполяторов Пугачёва представлены в~разд.~7. В~заключении приведены основные 
выводы и~возможные обобщения.

\section{Непрерывные и~дискретные стохастические системы, линейные относительно состояния}

Рассмотрим сначала нелинейную дифференциальную СтС~\cite{1-s2}:
  \begin{alignat}{2}
  \dot X_t &=\vrp \left(X_t, Y_t, t\right) + \psi \left(X_t, Y_t, t\right) V\,,
  &\enskip 
  X_{t_0} & = X_0\,;\label{e2.1-s2}\\
  \dot Y_t &=\vrp_1 \left(X_t, Y_t, t\right) + \psi_1 \left(X_t, Y_t, t\right) V
  \,,&\enskip 
  Y_{t_0}& = Y_0\,.\label{e2.2-s2}
  \end{alignat}
 Здесь $X_t$ и~$Y_t$~--- векторы состояния и~наблюдения размерности~$n_x$ и~$n_y$; 
 $V\hm= \dot W$, $W$~--- векторный стохастический процесс (СтП) с~независимыми 
 приращениями, состоящий из винеровской $W_0(t)$ и~пуассоновской частей:
\begin{equation}
\left.
\begin{array}{rl}
W&= \displaystyle W_0 (t) +\iii_{R_0^q} c(\rho) P^0 (t, d\rho)\,;\\[6pt]
\nu^W &=\displaystyle \nu^{W_0} + \iii_{R_0^q} c(\rho) c(\rho)^{\mathrm{T}} 
\nu_P (t, \rho) \,d\rho\,, 
\end{array}
\right\}
\label{e2.3-s2}
\end{equation}
где $c=c(\rho)$~--- векторная функция (той же размерности~$q$, что и~$W$) 
аргумента~$\rho$, а~интеграл при\linebreak любом $t\hm\ge t_0$ представляет собой 
стохастический интеграл по центрированной пуассоновской мере  $P^0 (t, A)$, 
независимой от~$W_0$  и~име\-ющей независимые значения на попарно 
непересе\-ка\-ющих\-ся множествах;  $A$~--- борелевское множество\linebreak
 пространства~$R_0^q$ 
с~выколотым началом~0; $\nu^W$, $\nu^{W_0}$ и~$\nu_P$~--- интенсивности~$W$,
$W_0$ и~$P^0$;
$\vrp\hm=\vrp (X_t, Y_t, t)$, $\psi\hm=\psi (X_t, Y_t, t)$, 
$\vrp_1\hm=\vrp_1 (X_t, Y_t, t)$ и~$\psi_1\hm=\psi_1 (X_t, Y_t, t)$~--- 
известные функции раз\-мер\-ности $(n_x\times 1)$, $(n_x\times n_v)$, $(n_y \times 1)$
и~$(n_y\times n_v)$, удовлетворяющие следующим условиям Лип\-це\-ра--Ши\-ря\-ева~\cite{2-s2}:
\begin{itemize}
\item функции $\vrp$ и~$\vrp_1$ линейны относительно состояния~$X_t$:
        \begin{equation}
        \left.
        \begin{array}{rl}
        \vrp \left(X_t, Y_t, t\right) &= a_1 \left(Y_t, t\right) X_t + a_0 \left(Y_t, t\right) \,;\\[6pt]
        \vrp_1 \left(X_t, Y_t, t\right) &= b_1 \left(Y_t, t\right) X_t +
         b_0 \left(Y_t, t\right) \,;
         \end{array}
         \right\}
         \label{e2.4-s2}
         \end{equation}

\item функции $\psi$ и~$\psi_1$  не зависят от состояния~$X_t$:
\begin{multline}
\hspace*{-1pt}\psi \left(X_t, Y_t, t\right) = \bar\psi \left(Y_t, t\right) ;\  
\psi_1 \left(X_t, Y_t, t\right) ={}\\
{}= \bar\psi_1  \left(Y_t, t\right).
\label{e2.5-s2}
\end{multline}
    \end{itemize}

Предполагается, что уравнения СтС~(\ref{e2.1-s2}) и~(\ref{e2.2-s2}) 
понимаются в~смысле Ито и~имеют решение в~среднем квадратическом (с.к.)~\cite{3-s2}.

Систему~(\ref{e2.1-s2})--(\ref{e2.5-s2}) будем называть гауссовской, 
если  $V\hm=\dot W_0$, а~$X_0$ и~$Y_0$~--- гауссовские случайные величины (с.в.).

Важный частный случай~(\ref{e2.1-s2})--(\ref{e2.5-s2}) 
составляют уравнения с~аддитивными шумами, когда
 \begin{equation}
 \bar\psi\left(Y_t, t\right) = \psi_0 (t)\,;\enskip 
 \bar\psi_1 \left(Y_t, t\right) = \psi_{10} (t)\,.
 \label{e2.6-s2}
 \end{equation}

Для случая, когда в~уравнения~(\ref{e2.1-s2}), (\ref{e2.2-s2}) 
входят независимые белые шумы~$V_1$ и~$V_2$, следует принять:
 \begin{alignat*}{2}
% \left.
% \begin{array}{c}
 V&= \displaystyle\lk V_1^{\mathrm{T}} V_2^{\mathrm{T}}\rk^{\mathrm{T}}\,,&\enskip 
 \nu&=\begin{bmatrix}
    \nu_1&0\\
    0&\nu_2\end{bmatrix}\,;\\[6pt] 
    \psi V &=\psi' V_1\,,&\enskip \psi_1 V &= \psi_1' V_2\,.
    %\end{array}
%    \right\}
%    \label{e2.7-s2}
    \end{alignat*}

Для дискретных СтС, линейных относительно состояния, уравнения состояния и~наблюдения 
имеют следующий вид~\cite{3-s2}:
\begin{align}
X_{k+1}&=\vrp_k \left(X_k, Y_k\right)+ \psi_k' \left(X_k, Y_k\right) V_{k}^d= {}\notag\\
&{}=a_k \left(Y_k\right)X_k + a_{0k} \left(Y_k\right) + c_{0k}\left(Y_k\right)  
V_{k}^d\,,\notag\\ 
&\hspace*{40mm}X_1 = X_{10}\,;\label{e2.8-s2}\\
Y_{k}&=\vrp_{1k} \left(X_k, Y_k\right)+ \psi_{1k} \left(X_k, Y_k\right) V_{k}^d={}\notag\\ 
&{}=b_k \left(Y_k\right)X_k + b_{0k} \left(Y_k\right) + l_{0k} \left(Y_k\right) 
 V_{k}^d\,,\notag\\
 & \hspace*{40mm}Y_1 = Y_{10}\,.\label{e2.9-s2}
 \end{align}
Здесь $n=n_x\hm=n_y$; $V_k^d$~--- дискретный белый шум с~известным 
в~общем случае негауссовским распределением.

Для дискретных СтС с~аддитивными шумами следует в~(\ref{e2.8-s2}), (\ref{e2.9-s2}) принять:
\begin{equation*}
c_{0k} \left(Y_k\right) = c_{0k0}\,; \enskip l_{0k} \left(Y_k\right) = l_{0k0}\,.
%\label{e2.10-s2}
\end{equation*}

\section{Дифференциальные гауссовские стохастические системы. 
Фильтры Липцера--Ширяева и~их~субоптимальная аппроксимация}

Для гауссовской СтС~(\ref{e2.1-s2})--(\ref{e2.5-s2})  
известны следу\-ющие точные уравнения нелинейной фильтрации по критерию минимума с.к.\ 
ошибки~\cite{2-s2, 3-s2}:
\begin{multline}
{\dot{\hat X}}_t= \left[a_1 \left(Y_t,t\right) \hat X_t + a_0 \left(Y_t,t\right)\right] +{}\\
{}+ \left[R_tb_1 \left(Y_t,t\right)^{\mathrm{T}} +
\left(\bar \psi\nu_0\bar \psi_1^{\mathrm{T}}\right)
    \left(Y_t,t\right)\right]\times{}\\
{}\times \left(\bar \psi_1\nu_0\bar \psi_1^{\mathrm{T}}\right)^{-1} \left(Y_t,t\right) 
\!\left\{ \dot Y_t -\left[ b_1\left(Y_t,t\right) \hat X_t +b_0\left(Y_t,t\right)\right] 
\!\right\},\hspace*{-0.84221pt}\\ 
    \hat X_{t_0} = \hat X_0\,;\label{e3.1-s2}
    \end{multline}
    
    
    \noindent
\begin{multline}
\dot R_t =  a_1\left(Y_t,t\right) R_t + R_t a_1 
\left(Y_t,t\right)^{\mathrm{T}} +{}\\
{}+\left(\bar \psi\nu_0\bar \psi^{\mathrm{T}}\right) 
\left(Y_t,t\right) -
    \left[ R_t b_1 \left(Y_t,t\right)^{\mathrm{T}}+{}\right.\\
\left.{}+\left(\bar \psi\nu_0\bar \psi_1^{\mathrm{T}}\right) \left(Y_t,t\right)
\vphantom{\left(Y_t,t\right)^{\mathrm{T}}}
\right] 
\left(\bar\psi_1\nu_0\bar\psi_1^{\mathrm{T}}\right)^{-1} \left(Y_t,t\right) \times{}\\
\!\!\!\!{}\times
\left[ b_1 \left(Y_t,t\right) R_t+ \left(\bar \psi_1 \nu_0 \bar \psi^{\mathrm{T}}\right)
    \left(Y_t,t\right)\right]\,,\enskip R_{t_0} = R_0\,,
\!    \label{e3.2-s2}
    \end{multline}
где $\hx_t$~--- с. к.\ оценка; $R_t$~--- ковариационная матрица ошибки 
фильтрации $X_t\hm- \hx_t$.

Таким образом, имеет место утверждение.

\smallskip

\noindent
\textbf{Теорема~3.1.}\
\textit{Пусть в~гауссовской системе}~(\ref{e2.1-s2})--(\ref{e2.5-s2}) 
\textit{диффузионная матрица  $\sigma_1\hm= \sigma_1 (Y_t, t) \hm=
\bar \psi_1\nu_0\bar \psi_1^{\mathrm{T}} (Y_t, t)$ не вырождена. Тогда алгоритм с.к.\ 
оптимального фильтра определяется уравнением}~(\ref{e3.1-s2}), 
\textit{а~его точность оценивается согласно}~(\ref{e3.2-s2}).

\smallskip

Как и~в случае линейной фильтрации при аддитивных шумах~\cite{3-s2}, 
уравнения дифференциального фильтра Лип\-це\-ра--Ши\-ря\-ева (ФЛШ)  
пред\-став\-ля\-ют собой замкнутую систему
уравнений,\linebreak опре\-де\-ля\-ющую~$\hat X_t$ и~$R_t$. Поэтому с.к.\  
оптимальную
оценку~$\hat X_t$ вектора состояния системы~$X_t$ и~его\linebreak
апостериорную ковариационную матрицу~$R_t$, ха\-рак\-те\-ри\-зу\-ющую
точность с.к.\ оптимальной оценки~$\hat X_t$, можно вычислять по
мере получения результатов наблюдений совместным интегрированием\linebreak
уравнений~(\ref{e3.1-s2}) и~(\ref{e3.2-s2}). Однако в~противоположность 
линейной фильт\-ра\-ции для ФЛШ нельзя вычислить~$R_t$ заранее, до
получения результатов наблюде\-ний, так как от последних зависят
коэффициенты уравнения~(\ref{e3.2-s2}). Поэтому ФЛШ 
в~данном случае должен выполнять интегрирование обоих уравнений~(\ref{e3.1-s2}) 
и~(\ref{e3.2-s2}). Это приводит к~существенному повышению
порядка оптимального фильтра. Если линейный фильтр  всегда
описывается уравнениями  порядка~$n_x$, то в~рассматриваемом более общем случае с.к.\
оптимальный фильтр описывается уравнениями порядка
\begin{equation}
Q_{\mathrm{ФЛШ}}=n_x+\fr{n_x(n_x+1)}{2}= \fr{n_x(n_x+3)}{2}\,.
\label{e3.3-s2}
\end{equation}

Очевидно, что ФЛШ будет совпадать с~обобщенным (приближенным) 
фильт\-ром Кал\-ма\-на--Бью\-си, 
фильтрами второго порядка, гауссовыми фильтрами~\cite{3-s2, 6-s2}.

Так как гауссовское (нормальное) распреде\-ление, аппроксимирующее
апостериорное рас\-пре\-деление вектора~$X_t$, полностью определяется
апостериорными математическим ожиданием~$\hat X_t$ 
и~ковариацион\-ной матрицей~$R_t$ вектора~$X_t$, то согласно 
тео\-рии нелинейной приближенной субоптимальной фильтрации при аппроксимации
апостериорного распределения вектора~$X_t$ нормальным
распределением будут иметь место следующие 
стохастические дифференциальные уравнения,
определяющие~$\hat X_t$ и~$R_t$~\cite{3-s2, 1-s2}:

\noindent
\begin{multline}
\dot{\hat X}_t = f \left(\hat X_t, Y_t,R_t,t\right)+{}\\
{}+
    h\left(\hat X_t,Y_t, R_t,t\right)\lk \dot Y_t - f^{(1)} \left(\hat X_t,Y_t,
    R_t,t\right)\rk\,;\label{e3.4-s2}
    \end{multline}
    
\vspace*{-12pt}

    \noindent
\begin{multline}
\dot R_t=\left\{ 
\vphantom{\left({\hat X}_t, Y_t,R_t,t\right)^{\mathrm{T}}}
f^{(2)}\left(\hat X_t, Y_t,R_t,t\right)-h\left(\hat
    X_t, Y_t,R_t,t\right)\times{}\right.\\
\left.{}{}\times\bar \psi_1\nu_0\bar \psi_1^{\mathrm{T}} 
    \left(Y_t,t\right)  
    h \left({\hat X}_t, Y_t,R_t,t\right)^{\mathrm{T}}\right\} +{}\\
   \hspace*{-3.6mm} {}+\!
\sss_{r=1}^{n_y} \rho_r \left({\hat X}_t,Y_t, R_t,t\right)\!\lk
    \dot Y_r -f_r^{(1)}\left({\hat X}_t,Y_t, R_t,t\right) \!\rk\!,\!\!\!\!
    \label{e3.5-s2}
    \end{multline}
где
   \begin{multline*}
    f\left(\hat X_t, Y_t,R_t,t\right)= \mm^N\lk \vrp \left(Y_t, X,t\right)\rk= {}\\
    {}=
    a_1 \left(Y_t, t\right) \hat X_t + a_0 \left(Y_t, t\right)\,;
  \end{multline*}
    
    
    \vspace*{-12pt}
    
    \noindent
    \begin{multline*}
    f^{(1)}\left(\hat X_t, Y_t,R_t,t\right)=\left\{ f_r^{(1)} 
    \left( \hat X_t, Y_t, R_t, t\right)\right\}={}\\
    {}=
    \mm^N\lk \vrp_1\left(Y_t, X,t\right)\rk=b_1 \left(Y_t, t\right) \hat X_t+ 
    b_0 \left(Y_t, t\right)\,;
    \end{multline*}
    
    \vspace*{-12pt}
    
    \noindent
    \begin{multline*}
h\left(\hat X_t, Y_t,R_t,t\right)={}\\
{}=\mm^N \left\{
\vphantom{\hat X_t f^{(1)}\left(\hat X_t, Y_t,R_t,t\right)^{\mathrm{T}}}
\left[ X\varphi_1\left(
Y_t,X,t\right)^{\mathrm{T}} + \bar \psi\nu_0\bar \psi_1^{\mathrm{T}} 
\left(Y_t,X,t\right)\right]-{}\right.\\
\left.{}-
    \hat X_t f^{(1)}\left(\hat X_t, Y_t,R_t,t\right)^{\mathrm{T}}\right\}
\left(\bar \psi_1\nu_0\bar \psi_1^{\mathrm{T}}\right)^{-1} \left(Y_t,t\right)= {}\\
{}=
\lk R_t b_1 \left(Y_t, t\right)^{\mathrm{T}} + 
\left(\bar \psi\nu_0\bar \psi_1^{\mathrm{T}}\right)\left(Y_t, t\right)
\rk \times{}\\
{}\times\left(\bar \psi\nu_0\bar \psi_1^{\mathrm{T}}\right)^{-1} \left(Y_t, t\right)\,;
\end{multline*}

\vspace*{-12pt}
    
    \noindent
    \begin{multline*}
\hspace*{-2.52pt}f^{(2)}\left(\hat X_t, Y_t,R_t,t\right)=
\mm^N\biggl\{  \left(X-\hat X_t\right)\varphi\left(Y_t,X,t\right)^{\mathrm{T}} + {}\\
{}+
\varphi \left(Y_t,X,t\right) \left(X^{\mathrm{T}}-\hat X_t^{\mathrm{T}}\right) +
\bar\psi\nu_0\bar\psi^{\mathrm{T}} \left(Y_t,X,t\right)\biggr\}={}\\
{}=\lk 
R_t b_1 \left(Y_t, t\right)^{\mathrm{T}} + 
\left(\bar \psi\nu\bar \psi_1^{\mathrm{T}}\right)
\left(Y_t, t\right)\rk \times{}\\
{}\times
\left(\bar \psi_1\nu_0\bar \psi_1^{\mathrm{T}}\right)^{-1}\!\! \left(Y_t, t\right)
 \lk 
b_1 \left(Y_t, t\right) R_t + 
\left(\bar \psi_1\nu_0\bar \psi^{\mathrm{T}}\right)\left(Y_t, t\right)
\rk;\hspace*{-8.054pt}
\end{multline*}

\vspace*{-12pt}
    
    \noindent
    \begin{multline*}
    \rho_r\left(\hat X_t,Y_t, R_t,t\right)={}\\
    {}=
    \mm^N \biggl\{  
    \left(X-\hat X_t\right)\left(X^{\mathrm{T}}-\hat X_t^{\mathrm{T}}\right) 
    a_r \left(Y_t,X,t\right)+{}\\
{}+ \left(X-\hat X_t\right) b_r\left(Y_t,X,t\right)^{\mathrm{T}} 
\left(X^{\mathrm{T}}-\hat X_t^{\mathrm{T}}\right)+ {}\\
{}+
b_r \left(Y_t,X,t\right) \left(X^{\mathrm{T}}-\hat X_t^{\mathrm{T}}\right)\biggr\} =0 
\enskip (r=1\tr n_y)\,.
%\label{e3.6a-s2}
\end{multline*}
Здесь функции $\vrp$, $\vrp_1$, $\bar \psi$ и~$\bar \psi_1$ 
определены~(\ref{e2.4-s2}) и~(\ref{e2.5-s2}); 
$a_r$~--- $r$-й элемент мат\-ри\-цы-стро\-ки
    $A_{\vrp_1} \hm= (\vrp_1^{\mathrm{T}} \hm- \hat\vrp_n^{\mathrm{T}})(\bar \psi_1 \nu_0\bar \psi_1^{\mathrm{T}})^{-1};$
$B_{kr}^{\bar \psi\bar \psi_1}$~--- элемент $k$-й строки и~$r$-го столб\-ца матрицы
    $B^{\bar \psi\bar \psi_1} \hm= (\bar \psi\nu_0\bar \psi_1^{\mathrm{T}})
    (\bar \psi_1\nu_0\bar \psi_1^{\mathrm{T}})^{-1};$
$b_r$~--- $r$-й столбец матрицы $B^{\bar \psi\bar \psi_1}$,
    $b_r^{\bar \psi\bar \psi_1} \hm= \lk b_{1r}^{\bar \psi\bar \psi_1}\cdots 
    b_{n_x r}^{\bar \psi\bar \psi_1}\rk^{\mathrm{T}},$
$\mm^N$~--- символ математического ожидания по $X\hm=X_t$.

Число уравнений МНА одномерного апостериорного распределения
определяется по формуле~(\ref{e3.3-s2}).


За начальные значения $\hat X_t$ и~$R_t$  при интегрировании
уравнений~(\ref{e3.4-s2}) и~(\ref{e3.5-s2}), естественно, следует принять
условные математическое ожидание $\hat X_0$ и~ковариационную матрицу~$R_0$ с.~в.~$X_0$ 
относительно~$Y_0$:
\begin{align*}
%\left.
%\begin{array}{rl}
\hat X_0 &= \mm^N\lk X_0 \mid Y_0\rk\,;\\
R_0 &= \mm^N \lk \left(X_0 -\hat X_0\right) 
\left(X_0^{\mathrm{T}} -\hat X_0^{\mathrm{T}}\right)\mid
Y_0\rk\,.
%\end{array}
%\right\}
%\label{e3.6-s2}
\end{align*}
Если нет
информации об условном распределении~$X_0$ относительно~$Y_0$, то
начальные условия можно взять в~виде:
\begin{align*}
%\left.
%\begin{array}{rl}
\hat X_0 &= \mm^N  X_0\,; \\ %[6pt]
    R_0&= \mm^N  \left(X_0 - \mm^N X_0\right) 
    \left(X_0^{\mathrm{T}} - \mm^N X_0^{\mathrm{T}}\right)\,.
    %\end{array}
%    \right\}
%    \label{e3.7-s2}
    \end{align*}
Если же и~об этих величинах нет никакой информации, то начальные
значения $\hat X_t$ и~$R_t$ приходится задавать произвольно.

Сравнивая  уравнения~(\ref{e3.4-s2}), (\ref{e3.5-s2}) с~уравнениями ФЛШ 
(теорема~3.1), 
имеем следующий результат~\cite{1-s2}.

\smallskip

\noindent
\textbf{Теорема~3.2.}\
\textit{Для гауссовской СтС}~(\ref{e2.1-s2})--(\ref{e2.5-s2}) 
\textit{субоптимальные НФ на основе МНА для одномерной апостериорной плотности 
и~ФЛШ совпадают.}


\section{Дифференциальные гауссовские стохастические системы. Нормальные фильтры Пугачёва}

Следуя~\cite{3-s2, 1-s2}, будем искать фильтр для оценки~$\hat X_t$ в~виде 
следующего уравнения:
 \begin{multline}
 d\hx_t =\alp_t \xi \yhxt\, dt + {}\\
 {}+\beta_t\eta \yhxt\, dY_t +
    \gamma_t \,dt\,.
    \label{e4.1-s2}
    \end{multline}
Здесь  $\xi =\xi \yhxt$ и~$\eta\hm=\eta\yhxt$~--- некоторые известные 
структурные функции
текущих значений оценки наблюдаемого процесса~$Y_t$, $\hx_t$ и~времени~$t$; 
$\alp_t$, $\beta_t$ и~$\gamma_t$~--- неизвестные функции времени.
Если бы коэффициенты  $\alp_t$, $\beta_t$ и~$\gamma_t$ в~(\ref{e4.1-s2})
были известными функциями времени, то уравнение~(\ref{e4.1-s2}) определило
бы фильтр того же  порядка~${n_x}$, что и~ уравнение~(\ref{e2.1-s2}), 
описывающее поведение системы. Поэтому, естественно,
возникает мысль попытаться непосредственно определить коэффициенты~$\alp_t$, 
$\beta_t$ и~$\gamma_t$ в~уравнении~(\ref{e4.1-s2}) как функции
времени из условия минимума с.к.\ ошибки  при всех  $t\hm>t_0$. Это приводит к~теории
услов\-но-оп\-ти\-маль\-но\-го фильтра Пугачева (ФП), когда в~уравнения
ФП задаются заранее и~оптимизируются только коэффициенты этого уравнения.
Итак, приходим к~идее на\-хож\-де\-ния оптимального фильтра
в~некотором классе допустимых фильтров,
определяемом условием, что поведение фильтра описывается
дифференциальным уравнением заданного порядка и~заданной формы.
Таким образом, мы отказываемся от абсолютной оптимизации 
и~ограничиваемся условной оптимизацией в~заданном ограниченном классе фильтров.

Определив класс допустимых фильтров, следует решить вопрос о том,
какой фильтр в~этом классе считается оптимальным. Следуя~\cite{3-s2}, 
будем считать оптимальным такой фильтр, который дает 
в~известном смысле наилучшую оценку при всех $t\hm>t_0$. Иными словами, 
задача оптимизации фильтра при всех
$t\hm>t_0$ является  задачей многокритериальной оптимизации. Такие
задачи, как правило, не имеют решения. Фильтр Кал\-ма\-на--Бью\-си, да\-ющий
оптимальную линейную оценку состояния линейной системы в~каждый
момент $t\hm>t_0$, является исключением~\cite{3-s2}. Значит, надо
определить такую оптимальность фильтра, при которой возможно решение
задачи. Будем считать оптимальным
такой допустимый фильтр, который на каждом бесконечно малом
интервале времени совершает оптимальный переход из того состояния,
в котором он был в~начале этого интервала, в~новое состояние.
Такой допустимый фильтр будем называть услов\-но-оп\-ти\-маль\-ным. 
Тогда задачи фильтрации сведутся к~нахождению оптимальных значений~$\alp_t$, 
$\beta_t$ и~ $\gamma_t$, в~любой момент  $t\ge t_0$ обеспечивающих минимум
с.к.\ ошибки фильтрации в~бесконечно близкий будущий момент  $s\hm> t$, $s\hm\to t$.

Отметим, что ФП обладает тем свойством, что в~данном
классе допустимых фильт\-ров не существует фильт\-ра, который при
данном начальном распределении~$Y_t$, $X_t$ и~$\hx_t$ в~момент~$t_0$ 
был бы лучше услов\-но-оп\-ти\-маль\-но\-го при всех  $t\hm>t_0$. Это
значит, по терминологии теории многокритериальной оптимизации, что
ФП пред\-став\-ля\-ет собой один из мно\-жества
допустимых фильт\-ров~---  оптимальный по Парето~\cite{3-s2}.
Общая теория ФП по с.к.\ критерию развита для урав\-не\-ний~(\ref{e2.1-s2}), (\ref{e2.2-s2})
и подробно изложена в~\cite{3-s2}.
Теория ФП обладает двумя несомненными преимуществами по сравнению с~методами
субоптимальной фильт\-ра\-ции. Во-пер\-вых, она позволяет
получать фильт\-ры более низкого порядка и,~следовательно, более
простые в~реализации. Во-вто\-рых, она дает возможность получать
фильт\-ры не меньшей, а~при желании даже большей точности, чем
фильт\-ры, даваемые методами субоптимальной нелинейной фильт\-рации.

Применяя теорию ФП~\cite{3-s2} 
к~нормальным процессам в~гауссовской СтС~(\ref{e2.1-s2})--(\ref{e2.5-s2}) 
при  $V\hm=V_0$, придем к~нормальному ФП  вида~(\ref{e4.1-s2}). 
Входящие в~(\ref{e4.1-s2}) коэффициенты~$\alp_t$, $\beta_t$ и~$\gamma_t$ 
определяются следующими уравнениями:
    \begin{equation}
    \alp_t m_1 + \beta_t m_2 + \gamma_t=m_0\,;
    \label{e4.2-s2}
    \end{equation}
        \begin{equation}
m_0 = \mm^N  [\varphi]\,,\enskip m_1 = \mm^N  [\xi]\,,\enskip m_2 = 
    \mm^N [\eta]\,;
    \label{e4.3-s2}
    \end{equation}
    \begin{equation}
    \beta_t =\kappa_{02} \kappa_{22}^{-1};\label{e4.4-s2}
    \end{equation}
    
    \vspace*{-12pt}
    
    \noindent
    \begin{multline}
    \kappa_{02} = \mm^N \left[\left(X_t - \hx_t\right) 
    \left(a_1 X_t + a_0\right)^{\mathrm{T}} \eta^{\mathrm{T}}\right]+{}\\
    {}+ \mm^N 
    \left[\bar \psi  \nu_0\bar \psi_1^{\mathrm{T}} \eta^{\mathrm{T}} \right];
    \label{e4.5-s2}
    \end{multline}
    \begin{equation}
    \kappa_{22} = \mm^N \lk\eta\bar\psi_1\nu_0 \bar \psi_1^{\mathrm{T}} 
    \eta^{\mathrm{T}}\rk;\label{e4.6-s2}
    \end{equation}
    
    
    \vspace*{-9pt}
    
    \noindent
    \begin{multline}
\alp_t\kappa_{11} + \mm^N \lk \left(\hx_t-X_t\right)\left(\xi^{\mathrm{T}} 
\alp_t^{\mathrm{T}} +\gamma_t^{\mathrm{T}}\right)
    \fr{\partial \xi^{\mathrm{T}}}{\partial \hat X_t}\rk={}\\
    {}=
    \kappa_{01}'-\beta_t\kappa_{21}';\label{e4.7-s2}
    \end{multline}
    \begin{equation}
    \kappa_{21}' = \mm^N \left\{\lk \eta\left(b_1X_t+b_0\right)-m_2\rk 
    \xi\right\}^{\mathrm{T}};\label{e4.8-s2}
    \end{equation}
    
    \vspace*{-9pt}
    
    \noindent
    \begin{multline}
\kappa_{01}' =\kappa_{01} + \mm^N \left[ \left(X_t-\hx_t\right)  
\fr{\partial \xi^{\mathrm{T}}}{\partial t} \right]+{}\\
{}+
     \mm^N \Bigl\{ \left(X_t-\hx_t\right) \left(b_1 X_t+b_0\right)^{\mathrm{T}} +{}\\
     {}+
    \bar \psi\nu_0\bar \psi_1^{\mathrm{T}} - \beta_t\eta\bar \psi_1\nu_0\bar 
    \psi_1^{\mathrm{T}}\Bigr\} \left(
    \fr{\partial}{\partial Y_t}+\eta^{\mathrm{T}}\beta_t^{\mathrm{T}} 
    \fr{\partial}{\partial \hat X_t}\right)\xi^{\mathrm{T}}+{}\\
{}+\fr{1}{2}\, \mm^N\left[ \left(X_t-\hx_t\right)\times{}\right.\\
{}\times \biggl\{ \mathrm{tr} 
\biggl[ \bar \psi_1\nu_0\bar \psi_1^{\mathrm{T}}
    \left( \fr{\partial}{\partial Y_t}+2 \eta^{\mathrm{T}}\beta_t^{\mathrm{T}} 
    \fr{\partial}{\partial \hat X_t}\right)
    \fr{\partial^{\mathrm{T}}}{\partial Y_t}\biggr]+{}\\
{}+\left.\mathrm{tr}\left[ \beta_t\eta\bar \psi_1\nu_0\bar \psi_1^{\mathrm{T}}
\eta^{\mathrm{T}}\beta_t^{\mathrm{T}} \fr{\partial }{\partial \hat X_t}
\,\fr{\partial^{\mathrm{T}}}{\partial \hat X_t}\right]\biggr\} 
\xi^{\mathrm{T}}\right] \,,\label{e4.9-s2}
\end{multline}

\vspace*{-8pt}

\begin{equation}
\left.
\begin{array}{rl}
\kappa_{11} &= \mm^N \lf \lk \xi - m_1\rk \xi^{\mathrm{T}}\rf;\\[6pt]
    \kappa_{21} &= \mm^N \lf\lk b_1 X_t + b_0  - m_2\rk \xi^{\mathrm{T}} \rf;\\[6pt]
\kappa_{01} &= \mm^N\lf \lk a_1 X_t + a_0  - m_0\rk \xi^{\mathrm{T}} \rf.
\end{array}
\right\}
\label{e4.10-s2}
\end{equation}

Точность НФП определяется уравнением:
\begin{multline}
\dot R_t = \mm^N \left[\left( X_t -\hx_t\right) 
\left(a_1 X_t + a_0\right)^{\mathrm{T}} +{}\right.\\
{}+\left(a_1 X_t + a_0\right)  
\left(X_t^{\mathrm{T}} -\hx_t^{\mathrm{T}}\right)-{}\\
{}-\beta_t \eta \yutt \bar\psi_1 \nu_0 \bar\psi_1^{\mathrm{T}} 
\eta \yutt^{\mathrm{T}} \beta_t^{\mathrm{T}} +{}\\
\left.{}+\bar\psi \nu_0 \bar\psi^{\mathrm{T}}
\vphantom{\left(a_1 X_t + a_0\right)^{\mathrm{T}}}
\right],\enskip R_{t_0} = R_0\,.
\label{e4.11-s2}
\end{multline}

Таким образом, справедливо следующее утверж\-де\-ние~\cite{1-s2}.

\smallskip

\noindent
\textbf{Теорема~4.1.}\
\textit{Пусть для гауссовской системы}~(\ref{e2.1-s2}), (\ref{e2.2-s2}) 
\textit{при условиях Лип\-це\-ра--Ши\-ря\-ева выполнены условия невырожденности 
матрицы  $\kappa_{22}$}~(\ref{e4.6-s2}) 
\textit{и конечности величин~$\kappa_{ij}$ $(i,j\hm=0,1,2)$. Тогда алгоритм НФП 
определяется уравнением}~(\ref{e4.1-s2}),  
\textit{а~коэффициенты~$\alp_t$, $\beta_t$ и~$\gamma_t$}~--- 
(\ref{e4.2-s2})--(\ref{e4.11-s2}).

\smallskip


Отметим, что в~качестве услов\-но-оп\-ти\-маль\-ных НФП могут служить 
субоптимальные фильтры, определяемые теоремой~3.2.


\section{Дифференциальные негауссовские стохастические системы. 
Нормальный  фильтр Пугачёва}

Для негауссовской СтС в~уравнениях теоремы~4.1 следует заменить~$\nu_0$ 
на~$\nu$ согласно~(\ref{e2.3-s2}), а~в~выражении~(\ref{e4.9-s2}) 
учесть два дополнительных интегральных члена:
\begin{multline}
\kappa_{01}'  =\kappa_{01} + \mm^N  \left[ \left(X_t - \hat X_t\right) 
\fr{\partial\xi^{\mathrm{T}}}{\partial t} \right] +{}\\
{}+
\mm^N  \biggl\{ 
\left(X_t -\hat X_t\right) 
\bigg[ \vrp_1^{\mathrm{T}} -\iii_{R_0^q} c(\rho)^{\mathrm{T}} \nu_P (t,\rho)\, d\rho 
\psi_1^{\mathrm{T}} \bigg] +{}\\
{}+ \bar\psi \nu \bar\psi_1^{\mathrm{T}} - \beta_t \eta \bar \psi_1 
\nu\bar\psi_1^{\mathrm{T}} \biggr\} \left( 
\fr{\partial}{\partial Y_t} + \eta^{\mathrm{T}} \beta_t^{\mathrm{T}} 
\fr{\partial}{\partial \hat X_t}\right) \xi^{\mathrm{T}}+ {}\\
{}+
\fr{1}{2}\, \mm^N \biggl\{ \lk \left(X_t -\hat X_t\right)\rk \times{}\\
{}\times\biggl\{ 
\mathrm{tr}\, \lk \bar\psi_1 \nu \bar\psi_1^{\mathrm{T}} \left( 
\fr{\partial}{\partial Y_t} + 2 \eta^{\mathrm{T}} \beta_t 
\fr{\partial}{\partial \hat X_t} \right) 
\fr{\partial^{\mathrm{T}}}{\partial Y_t} \rk+{}\\
{}+
   \mathrm{tr}\, \lk 
   \beta_t \eta \bar\psi_1 \nu\bar\psi_1^{\mathrm{T}} \eta^{\mathrm{T}} 
   \beta_t^{\mathrm{T}} \fr{\partial}{\partial \hat X_t}\,
   \fr{\partial^{\mathrm{T}}}{\partial \hat X_t} \rk 
   \biggr\} \xi^{\mathrm{T}} \biggr\}+{}\\
{}+ \mm^N \biggl\{ \iii_{R_0^q}  \big[ 
X_t -\hat X_t + 
\left(\bar\psi -\beta_t \eta \bar\psi_1\right) c (\rho)\times{}\\
{}\times \xi 
\left(Y_t +\bar\psi c(\rho), \hat X_t + \beta_t \eta \bar\psi_1 c(\rho), t\right) -{}\\
{}-
\xi^{\mathrm{T}}
\big]^{\mathrm{T}} \nu_P (t, d\rho) d\rho\biggr\}\,.
\label{e5.1-s2}
\end{multline}
Здесь функции $\vrp$, $\vrp_1$, $\bar\psi$ и~$\bar\psi_1$ удовлетворяют 
условиям~(\ref{e2.4-s2})--(\ref{e2.5-s2}). В~результате имеем следующее утверждение.

\smallskip

\noindent
\textbf{Теорема~5.1.} 
\textit{Пусть для негауссовской СтС}~(\ref{e2.1-s2})--(\ref{e2.5-s2}) 
\textit{матрица  $\kappa_{22}$}~(\ref{e4.5-s2}) \textit{не вырождена, 
а~интегралы}~(\ref{e4.2-s2}), (\ref{e4.3-s2}), (\ref{e4.6-s2}), (\ref{e4.8-s2}), 
(\ref{e4.10-s2}) \textit{и}~(\ref{e5.1-s2}) \textit{конечны.
Тогда алгоритм НФП определяется уравнением}~(\ref{e4.1-s2}), 
\textit{а коэффициенты}~$\alp_t$, $\beta_t$ и~$\gamma_t$~--- 
(\ref{e4.2-s2})--(\ref{e4.10-s2}).

\smallskip

Теория НФП  не позволяет получить нормальные с.к.\ оптимальные
фильтры. Можно получить только ФП,
которые в~общем случае хуже с.к.\ оптимальных, но зато легко реализуемы.
Однако если с.к.\ оптимальная оценка~$\hx_t$ вектора~$X_t$
удовлетворяет уравнению допустимого фильтра~(\ref{e4.1-s2}) 
при ка\-ких-ли\-бо коэффициентах времени~$\alp_t$,
$\beta_t$ и~$\gamma_t$, то уравнения теорем~4.1 и~5.1, 
конечно, определяют именно эти~$\alp_t$, $\beta_t$
и~$\gamma_t$ и~НФП будет  с.к.\ оптимальным
(последний в~данном случае будет допустимым и,~следовательно,
оптимальным в~классе допустимых фильтров).

Как известно~\cite{3-s2}, теория НФП дает возможность оценивать не все
компоненты вектора состояния системы (в общем случае расширенного),
а~только некоторые из них. Для этого достаточно взять структурные
функции~$\xi$ и~$\eta$ в~(\ref{e4.1-s2}) зависящими лишь от соответствующих
компонент вектора~$\hx_t$. К~примеру, взяв~$\xi$ и~$\eta$ в~(\ref{e4.1-s2})
зависящими лишь от~$Y_t$, $t$ и~оценок неизвестных параметров
системы, можно оценивать только параметры системы, не оценивая ее
со\-сто\-яния. В~таких случаях будут получаться НФП, порядок которых
меньше размерности~$n_x$ расширенного вектора со\-сто\-яния.

Особое практическое значение имеет случай~(\ref{e2.1-s2})--(\ref{e2.6-s2}) 
с~аддитивными (в~общем случае негауссовскими) шумами.
Следуя~\cite{3-s2}, проведем статистическую линеаризацию нелинейных функций:
\begin{align*}
a_1 \left(Y_t, t\right) X_t &\approx 
\left(k_{0x}^{a_1 x} - k_{1x}^{a_1 x}\right) m_t^x +{}\\
&{}+ \left( k_{0y}^{a_1 x} - k_{1y}^{a_1 x}\right) m_t^y + 
k_x^{a_1 y} X_t + k_{0y}^{a_1 x} Y_t\,;\\
b_1 \left(Y_t, t\right) X_t &\approx 
\left(k_{0x}^{b_1 x} - k_{1x}^{b_1 x}\right) m_t^x +{}\\
&{}+ \left( k_{0y}^{b_1 x} - k_{1y}^{b_1 x}\right) m_t^y + 
k_x^{b_1 y} X_t + k_{0y}^{b_1 x} Y_t\,;\\
a_0 \left(Y_t , t\right) &\approx
\left( k_{0y}^{a_0} - k_{1y}^{a_0}\right) m_t^y +
 k_{0y}^{a_0} Y_t\,; \\
b_0 \left(Y_t , t\right) &\approx \left( k_{0y}^{b_0} - k_{1y}^{b_0}\right) m_t^y + 
k_{0y}^{b_0} Y_t)\,.
\end{align*}
Тогда~(\ref{e2.1-s2})--(\ref{e2.6-s2}) приводятся к~эквивалентной 
гауссовской системе,
линейной относительно~$X_t^0, Y_t^0$ и~нелинейной относительно
$m_t^x, m_t^y$:
    \begin{align*}
    \dot X_t &= \bar a Y_t + \bar a_1 X_t + \bar a_0 +\bar\psi V\,;
    %\label{e5.2-s2}
    \\
\dot Y_t &= \bar b Y_t + \bar b_1 X_t + \bar b_0 +\bar\psi_1 V\,.
%\label{e5.3-s2}
\end{align*}
Здесь введены обозначения:
    \begin{equation}
    \left.
    \begin{array}{c}
    \bar a = k_y^{a_1x} + k_y^{a_0}\,; \quad 
    \bar a_1 = k_x^{a_1 x}\,;\\[4pt]
    a_0 = (k_{0y}^{a_0} - k_{1y}^{a_0}) m_t^y\,;\\[4pt]
\bar b = k_y^{b_1x}\,; \ \ \ \bar b_1 = k_x^{b_1 x}\,;\ \ \ \bar b_0 = (k_{0y}^{b_0} - k_{1y}^{b_0}) m_t^y\,.
\end{array}
\right\}
\label{e5.4-s2}
\end{equation}
Правые части уравнений~(\ref{e5.4-s2}) 
зависят от вероятностных моментов первого и~второго порядка и~определяются 
из следующей линейной дифференциальной системы для составного вектора  
$Z_t \hm=\lk X_t^{\mathrm{T}} Y_t^{\mathrm{T}}\rk^{\mathrm{T}}$:
\begin{equation}
\dot{\bar m}_t^z = cm_t^z + c_0\,;\enskip
%\eqno(5.5)$$
 \dot  K_t^{\bar z} = c K_t^z +K_t^z c^{\mathrm{T}} + l\nu l^{\mathrm{T}}\,,
 \label{e5.6-s2}
 \end{equation}
где
    \begin{equation*}
    c=\begin{bmatrix}
    \bar a_1&\bar a\\
    \bar b_1&\bar b\end{bmatrix}\,; \enskip
    c_0=\begin{bmatrix}
    \bar a_o\\
    \bar b_0\end{bmatrix}\,;\enskip
    l=\begin{bmatrix}
    \bar \psi_t\\
    \bar\psi_{1t}\end{bmatrix}\,.
%    \label{e5.7-s2}
    \end{equation*}

Применяя теорию квазилинейной фильтрации~\cite{3-s2} 
к~уравнениям~(\ref{e5.6-s2}), 
получим сле\-ду\-ющие уравнения субоптимального квазилинейного НФП:
    \begin{multline}
    \dot{\hat X}_t = \bar a Y_t +\bar a_1 \hat X_t + \bar a_0 + {}\\
    {}+
    \beta_t \lk \dot Y_t - \left( \bar b Y_t +\bar b_1 \hx_t + \bar b_0\right) \rk\,;
    \label{e5.8-s2}
    \end{multline}
\begin{equation}
\beta_t = R_t \bar b_1^{\mathrm{T}} +
\left(\bar\psi\nu\bar\psi_1^{\mathrm{T}}\right) 
\left(\bar \psi_1 \nu \bar \psi_1^{\mathrm{T}}\right)^{-1}\,;
\label{e5.9-s2}
\end{equation}

\vspace*{-12pt}

\noindent
\begin{multline}
 \dot R_t = \bar a_1 R_t + R_t \bar a_1^{\mathrm{T}} +
 \bar\psi\nu\bar\psi^{\mathrm{T}} -{}\\
 \hspace*{-3mm}{}- 
 \left(R_t \bar b_1^{\mathrm{T}} +\bar\psi\nu\bar\psi_1\right)
 \left(\bar\psi_1 \nu\bar\psi_1^{\mathrm{T}}\right)^{-1} 
 \left( \bar b R_t +\bar\psi_1\nu\bar\psi_1^{\mathrm{T}}\right).
 \label{e5.10-s2}
 \end{multline}

Таким образом, имеем следующий результат~\cite{1-s2}.

\smallskip

\noindent
\textbf{Теорема~5.2.}\
\textit{Пусть уравнения негауссовской СтС}~(\ref{e2.1-s2})--(\ref{e2.6-s2}) 
\textit{с~аддитивными шумами  допускают применение МСЛ. Тогда уравнения алгоритма 
квазилинейного НФП имеют вид}~(\ref{e5.8-s2})--(\ref{e5.10-s2}).

%\smallskip

\section{Обобщения нормальных фильтров на случай дискретных стохастических систем}


Теоремы разд.~3--5 допускают обобщение на случай 
дискретных СтС~(\ref{e2.8-s2}), (\ref{e2.9-s2}). Для
 дискретных гауссовских СтС~(\ref{e2.8-s2}) и~(\ref{e2.9-s2}) 
 при $V_k^d\hm= V_{0k}^d$ алгоритм НФ на основе МНА определяется уравнениями 
 (\textbf{теорема~6.1}):
\begin{multline*}
\hx_{k+1} = f_k \left(\hx_{k+1\mid k}, Y_k, R_{k+1\mid k}\right) +{}\\
   {}+ h_k \left(\hx_{k+1\mid k}, Y_k, R_{k+1\mid k}\right)\times{}\\
   {}\times
\lk Y_{k+1} - f_k^{(1)} \left(\hx_{k+1\mid k}, Y_k, R_{k+1\mid k}\right)\rk;
%\label{e6.1-s2}
\end{multline*}

\vspace*{-12pt}

\noindent
\begin{multline*}
R_{k+1}=\biggl\{ f_k^{(2)} \left(\hx _{k+1\mid k}, Y_k, R_{k+1\mid k}\right) -{}\\
{}- h_k \left(\hx_{k+1\mid k}, Y_k, R_{k+1\mid k}\right)
    \left(\psi_{1,k}\nu_{0k}\psi_{1,k}^{\mathrm{T}}\right)\times{}\\
    {}\times h \left(\hx_{k+1\mid k}, Y_k, R_{k+1\mid k}\right)^{\mathrm{T}}\biggr\} +{}\\[-20pt]
\end{multline*}

%\pagebreak

\noindent
\begin{multline*}
{}+\sum\limits_{r=1}^{n_y}\rho_r \left(\hat X_{k+1\mid k} , Y_k, R_{k+1\mid k}\right)\times{}\\
{}\times
\lk Y_{r, k+1} - f_{r,k}^{(1)}
    \left(\hx_{k+1\mid k}, Y_k,R_{k+1\mid k}\right)\rk.
%    \label{e6.2-s2}
    \end{multline*}
Здесь введены следующие обозначения:
    $$
    f_k=f_k \left(\hat X_{k+1\mid k}, Y_k, R_{k+1\mid k}\right) =
    \mm^N \lk\vrp_k\rk\,;\\
    $$
    $$
    f_k^1 = f_k^{(1)} \left(\hx_{k+1\mid k}, Y_k, R_{k+1\mid k}\right) = 
    \mm^N \lk \vrp_{1k}\rk\,;
    $$
    
    \vspace*{-12pt}
    
    \noindent
    \begin{multline*}
    h_k = h_k \left(\hx_{k+1\mid k}, Y_k, R_{k+1\mid k}\right) ={}\\
    {}=
     \mm^N \lk X\vrp_{1k} (Y_k,X)+ \bar\psi_k\nu_{0k}\bar\psi_{1k}^{\mathrm{T}} 
     \left(Y_k,X\right)\rk\,;
    \end{multline*}
    
    \vspace*{-12pt}

\noindent
\begin{multline*}
    f_k^{(2)}= f_k^{(2)} \left(\hx_{k+1\mid k}, Y_k, R_{k+1\mid k}\right) ={}\\
    {}=
    \mm^N \biggl[ \left(X- \hx_{k+1\mid k}\right) \vrp_k \left(Y_k, X\right)^{\mathrm{T}}+{}\\
{}+\vrp_k \left(Y_k,X\right) 
\left(X^{\mathrm{T}} - \hx_{k+1\mid k}\right)+
\bar\psi_k\nu_{0k}\bar\psi_{1,k}^{\mathrm{T}} \left(Y_k,X\right)\biggr];
\end{multline*}

\vspace*{-12pt}

\noindent
\begin{multline*}
\rho_r \left(\hx_{k+1\mid k}, Y_k, R_{k+1\mid k}\right)={}\\
{}=
\mm^N \biggl[ \left(X- \hx_{k+1\mid k}\right) 
\left(X^{\mathrm{T}}-\hx_{k+1\mid k}^{\mathrm{T}}\right) a_r (Y_k, X)+{}\\
{}+ \left(X-\hx_{k+1\mid k}\right) b_r \left(Y_k,X\right)^{\mathrm{T}} 
\left(X^{\mathrm{T}} -\hx_{k+1\mid k}^{\mathrm{T}}\right)+{}\\
{}+
b_r \left(Y_k,X\right) 
\left(X^{\mathrm{T}}-\hx_{k+1\mid k}^{\mathrm{T}}\right)\biggr],
%\label{e6.3-s2}
\end{multline*}
где $a_r$~--- $r$-й элемент мат\-ри\-цы-стро\-ки $(\vrp_{1k}^{\mathrm{T}}
\hm-\hat\vrp_{1k}^{\mathrm{T}})(\bar\psi_{1k}\nu_{0k}\bar\psi_{1k}^{\mathrm{T}})^{-1}$; 
$b_r \hm= \lk b_{1r}\cdots b_{n_x r}\rk^{\mathrm{T}}$; $ b_{lr}$~--- элемент $l$-й
строки и~$r$-го столб\-ца мат\-ри\-цы $(\bar\psi_k
\nu_{0k}\bar\psi_{1k}^{\mathrm{T}}) 
(\bar\psi_{1k}\nu_k\bar\psi_{1k}^{\mathrm{T}})^{-1}$.

В качестве начальных условий принимаются
    \begin{align*}
%    \left.
%    \begin{array}{rl}
    \hx_{1\mid 1} &= \hx_1 =\mm^N\lk X_1\mid Y_1\rk ;
    \\ 
    R_{1\mid 1} &= R_1 = \mm^N \lk \left(X_1 -X_1^0\right) 
    \left(X_1-{X_1^0}^{\mathrm{T}}\right)\vert Y_1\rk, 
%    \end{array}
%    \right\}
%    \label{e6.4-s2}
    \end{align*}
определяющие начальное нормальное распределение.


Алгоритм дискретного НФП для дискретных СтС~(\ref{e2.8-s2}), (\ref{e2.9-s2}), 
основываясь на~\cite{3-s2}, представим в~следующем виде (\textbf{теорема~6.2}):
    \begin{equation*}
    \hx_{k+1} = \alp_k \xi_k \left(\hx_k\right) + 
    \beta_k\eta_k\left(\hx_k\right) Y_k +\gamma_k.
%    \label{e6.5-s2}
    \end{equation*}
Здесь $\xi_k =\xi_k(\hx_k)$ и~$\eta_k=\eta_k (\hx_k)$~--- известные
структурные функции НФП; неизвестные коэффициенты фильтра~$\alp_k$, 
$\beta_k$ и~$\gamma_k$ определяются из уравнений:
\begin{gather*}
 \alp_k \kappa_{11}^{(k)} +\beta_k \kappa_{21}^{(k)} =\kappa_{01}^{(k)};\quad 
 \alp_k \kappa_{12}^{(k)} +
 \beta_k \kappa_{22}^{(k)} = \kappa_{02}^{(k)};
\\ 
 \gamma_k =\rho_0^{(k)} - \alp_k \rho_1^{(k)} - \beta_k \rho_2^{(k)};\\[-20pt]
\end{gather*}

\noindent
 $$ 
 \rho_k=\lk \rho_1^{(k)T}\rho_2^{(k)T}\rk^{\mathrm{T}};$$
 $$
 \rho_1^{(k)} = \mm^N \lk \xi_k\rk ;\enskip 
 \rho_2^{(k)}=\mm^N\lk \eta_k  \varphi_{1k}\rk;
 $$
 $$
 B_k=\begin{bmatrix}
 \kappa_{11}^{(k)}&\kappa_{12}^{(k)}\\
 \kappa_{21}^{(k)}&\kappa_{22}^{(k)}\end{bmatrix}\,,
 \enskip \det \lv B_k\rv \ne 0;
 $$
 $$
 \kappa_{11}^{(k)} =\mm^N \left\{\lk \xi_k  -\rho_1^{(k)}\rk 
 \xi_k^{\mathrm{T}}\right\};
 $$
 $$
 \kappa_{12}^{(k)} =\kappa_{21}^{(k)T}=\mm^N \left\{\lk \xi_k  -\rho_1^{(k)}\rk \varphi_{1k}^{\mathrm{T}} \eta_k^{\mathrm{T}} \right\};
 $$
 
 \vspace*{-12pt}

\noindent
\begin{multline*}
 \kappa_{22}^{(k)} = \mm^N \lf\lk \eta_k  \varphi_{1k}  -\rho_2^{(k)} \rk 
 \varphi_{1k}^{\mathrm{T}} \eta_k^{\mathrm{T}} \rf+{}\\
 {}+
 \mm^N \lf \eta_k \bar\psi_{1k} \nu_k \bar\psi_{1k}^{\mathrm{T}} \eta_k^{\mathrm{T}}\rf;
\end{multline*}
 $$
  D_k =\lk \kappa_{01}^{(k)} \kappa_{02}^{(k)}\rk;\enskip  
  \kappa_{01}^{(k)} =\mm^N \lf\lk \varphi_k - m_{k+1}\rk \xi_k^{\mathrm{T}}\rf;
  $$
  
  \vspace*{-12pt}

\noindent
\begin{multline*}
 \kappa_{02}^{(k)} = \mm^N \lf\lk \varphi_{k}  - 
 m_{k+1}\rk \varphi_{1k}^{\mathrm{T}} \eta_k^{\mathrm{T}}\rf+{}\\
 {}+\mm^N\lk 
 \bar\psi_k  \nu_k \bar\psi_{1k}^{\mathrm{T}} \eta_k^{\mathrm{T}}\rk;
\end{multline*}

\vspace*{-6pt}

\noindent
 \begin{equation*}
 m_{k+1}=\rho_0^{(k)}\,,\enskip \rho_0^{(k)}= \mm^N \left[\varphi_k\right],
% \label{e6.6-s2}
 \end{equation*}
где функции $\vrp_k$, $\vrp_{1k}$, $\bar\psi_k$ и~$\bar\psi_{1k}$ 
определены~(\ref{e2.8-s2}), (\ref{e2.9-s2}); 
$\mm^N V_k^d \hm=0$, $\mm^N V_k^d V^{dT}=\nu_k$~--- ковариация белого шума~$V_k^d$. 

\vspace*{-5pt}

\section{Нормальные экстраполяторы Пугачёва}

\vspace*{-1pt}

Будем считать услов\-но-оп\-ти\-маль\-ным (по Пугачёву~\cite{3-s2}) такой
экстраполятор из класса до\-пус\-тимых, который при любом совместном
распределении величин $X_t$, $\hat X_t$ и~$Y_t$ в~момент $t\hm\ge t_0$ 
в~дифференциаль\-ной СтС~(\ref{e2.1-s2})--(\ref{e2.5-s2}) дает наилучшую оценку вектора
$X_{s+\Delta}$, в~бесконечно близкий момент $s\hm>t$, $s\hm\to t$,
реализующую минимум с.к.\ ошибки. Тогда задача услов\-но-оп\-ти\-маль\-ной
экстраполяции сведется к~нахождению оптимальных значений~$\alp_t$,
$\beta_t$ и~$\gamma_t$ в~(\ref{e4.1-s2}) в~любой момент времени $t\hm\ge t_0$,
обеспечивающих минимум с.к.\ ошибки экстраполяции в~бесконечно близкий
будущий момент $s\hm>t$, $s\hm\to t$.

Для решения задачи экстраполяции необходимо ограничиться случаем~(\ref{e2.1-s2})--(\ref{e2.5-s2}),
когда функции~$\varphi$ и~$\psi$ не зависят от наблюдаемого
вектора~$Y_t$, процесс $W(t)$ состоит из двух независимых блоков
$W_1 (t)$ и~$W_2(t)$ и~соответственно матрицы~$\psi$ и~$\psi_1$
имеют блочную структуру  $\psi\hm = \lk \psi'\, 0\rk$; $ \psi_1\hm = \lk
0\,\psi_1'\rk$, так что $\psi \,dW \hm= \psi' \,dW_1$ и~$\psi_1 \,dW\hm=
\psi_1'\, dW_2$. Теперь, отбросив штрихи у функций~$\psi'$ и~$\psi_1'$, 
запишем уравнения~(\ref{e2.1-s2}) и~(\ref{e2.2-s2}) в~следующем виде:
\begin{equation}
\left.
\begin{array}{rl}
    \dot X_t &= \varphi \left(X_t,Y_t,t\right) dt + \psi \left(X_t,t\right)V_1\,;\\[5pt]
    & V_1 = \dot W_1\,;\quad X_{t_0} = X_0\,;\\[5pt]
    \dot Y_t &= \varphi \left(X_t,Y_t,t\right) dt + \psi_1 \left(X_t,Y_t,t\right) V_2\,;\\[5pt]
    &V_2 =\dot W_2\,;\quad Y_{t_0} = Y_0\,,
    \end{array}
    \right\}
    \label{e7.1-s2}
\end{equation}
где $W_1 = W_1 (t)$ и~$W_2 \hm= W_2 (t)$~--- независимые процессы 
с~независимыми приращениями и~нулевыми математическими функциями:
\begin{align*}
%\left.
%\begin{array}{rl}
    K_{w_i} (t_1, t_2)  &= k_i (\min (t_1, t_2))\,;\\ %[6pt]
    k_i(t) &= \displaystyle k_i(t_0) + \int\limits_{t_0}^t \nu_i (\tau)
   \, d\tau.
 %   \end{array}
%    \right\}
%    \label{e7.3-s2}
    \end{align*}

Совершенно так же, как в~случае услов\-но-оп\-ти\-маль\-ной нелинейной фильтрации, 
решается задача услов\-но-оп\-ти\-маль\-ной экстраполяции. Разница будет
лишь в~том, что в~случае экстраполяции~$\hx_t$ представляет собой
оценку будущего состояния системы  $X_{t+\Delta}$, $\Delta \hm>0$, которое
определяется стохастическим дифференциальным
уравнением следующего вида:
\begin{multline}
\dot X_{t+\Delta}=\varphi \left(X_{t+\Delta}, t+\Delta\right) +{}\\
{}+\psi \left(X_{t+\Delta}, t+\Delta \right) V(t+\Delta)\,,\enskip 
V=\dot V\,.\label{e7.4-s2}
\end{multline}
Заменив этим уравнением второе уравнение~(\ref{e2.9-s2}) и~повторив все
выкладки ФП, получим уравнения, определяющие коэффициенты 
в~уравнениях нормального услов\-но-оп\-ти\-маль\-но\-го экстраполятора Пугачёва
(НЭП). Приведем окончательные  результаты:
\begin{equation}
\left.
\begin{array}{l}
\kappa_{02} ={}\\[6pt]
{}= \mm^N\left[ \left(X_{t+\Delta} - \hat X_t\right) \varphi_1 
    \yxtt^{\mathrm{T}} \times{}\right.\\
   \left. \hspace*{35mm}{}\times \eta\yutt^{\mathrm{T}}\right]\,;
\\[6pt]
    \kappa_{22} = \mm^N
    \left[\eta\yutt\psi_1\yxtt\nu_1(t)\times{}\right.\\[6pt]
    \left.{}\hspace*{15mm}\times \psi_1\yxtt^{\mathrm{T}} 
    \eta\yutt^{\mathrm{T}}\right]\,;
\\[6pt]
\mm^N \lk \hx\rk = \mm^N \lk X_{t+\Delta}\rk;\\[6pt] 
\mm^N\lk\left(\hx - X_{t+\Delta}\right)\xi_t^{\mathrm{T}} \rk=0;
\end{array}
\right\}
\label{e7.5-s2}
\end{equation}
\begin{equation}
\left.
\begin{array}{rl}
m_0 &= \mm^N\lk \varphi \left(X_{t+\Delta}, t+\Delta\right)\rk \,;\\[6pt]
\kappa_{01} &= \mm^N\left\{ 
\vphantom{\xi\yutt^{\mathrm{T}}}
\left[ \varphi \left(X_{t+\Delta}, t+\Delta\right)-{}\right.\right.\\[6pt]
&\hspace*{25mm}\left.\left.{}- m_0\right] \xi\yutt^{\mathrm{T}}\right\};
\end{array}
\right\}
\label{e7.6-s2}
\end{equation}

\vspace*{-6pt}

\noindent
\begin{multline*}
\kappa_{01}' = \kappa_{01} + \mm^N\left[ \left(X_{t+\Delta}- \hat X_t\right) 
\fr{\partial \xi^{\mathrm{T}}}{\partial t}\right] +{}\\
{}+ \mm^N\!\bigg\{\! \!\left(X_{t+\Delta}-\hat X_t\right)\!\! \bigg[ 
\varphi_1^{\mathrm{T}} -\!\iii_{R_0^q}\! c_2(\rho)^{\mathrm{T}}
    \nu_{sP} (t,\rho) d\rho \bar\psi_1^{\mathrm{T}}\bigg] - {}\\
    {}-
    \beta_t\eta\bar\psi_1\nu_{20}\bar\psi_1^{\mathrm{T}}\bigg\}
\left(\fr{\partial}{\partial Y_t} + \eta^{\mathrm{T}} \beta_t^{\mathrm{T}} 
\fr{\partial}{\partial \hat X_t}\right)\xi^{\mathrm{T}}+{}\\
{}+
    \fr{1}{2}\, \mm^N \bigg\{ 
    \left(X_{t+\Delta} - \hat X_t\right) \times{}
    \end{multline*}


\noindent
\begin{multline}
 {}\times
    \biggl\{ 
    \mathrm{tr}\biggl[ 
    \bar\psi_1 \nu_{20} \bar\psi_1^{\mathrm{T}} \left(
\fr{\partial }{\partial Y_t} + 2\eta^{\mathrm{T}} \beta_t^{\mathrm{T}} 
\fr{\partial}{\partial\hat X_t}\right)
    \fr{\partial^{\mathrm{T}}}{\partial Y_t}\biggr] +{}\\
       {}+
    \mathrm{tr}\left[ \beta_t \eta \bar\psi_1 \nu_{20} \bar\psi_1^{\mathrm{T}} 
    \eta^{\mathrm{T}}\beta_t^{\mathrm{T}}
    \fr{\partial}{\partial \hat X_t} \,
    \fr{\partial^{\mathrm{T}}}{\partial \hat X_t}\right]\biggr\}
    \xi^{\mathrm{T}}\biggr\}+{}\\
{}+\iii_{R_0^q} \mm^N \bigg\{
\vphantom{\left[ \xi \left(X_t+\bar\psi_1 c_2(\rho), \hat X_t +
\beta_t \eta \bar\psi_1 c_2 (\rho),t\right)^{\mathrm{T}} -
    \xi^{\mathrm{T}}\right]} 
\left[ X_{t+\Delta} - \hat X_t-\beta_t\eta\bar\psi_1 c_2(\rho)\right]\times{}\\
{}\times\left[ \xi \left(X_t+\bar\psi_1 c_2(\rho), \hat X_t +
\beta_t \eta \bar\psi_1 c_2 (\rho),t\right)^{\mathrm{T}} -{}\right.\\
\left.{}-
    \xi^{\mathrm{T}}\right]\bigg\} \nu_{2P}(t,\rho)\, d\rho\,.
    \label{e7.7-s2}
    \end{multline}
Здесь  $c_2 (\rho)$, $\nu_{20}$ и~ $\nu_{2P}$~--- соответствующие
величины в~представлении интенсивности~$\nu_2$ процесса  $W_2(t)$
формулой вида~(\ref{e2.3-s2}).

В случае винеровского процесса  $W_2(t)$ имеем $c_2(\rho)\hm =0$,
$\nu_{20}\hm=\nu_2$ и~формула~(\ref{e7.7-s2}) принимает вид:
 \begin{multline*}
 \kappa_{01}' = \kappa_{01} + \mm^N\left[ \left(X_{t+\Delta}- \hat X_t\right) 
 \fr{\partial \xi^{\mathrm{T}}}{\partial t} \right]+{}\\
    {}+ \mm^N\bigg\{\left[
    \vphantom{\fr{\partial^{\mathrm{T}}} {\partial \hat X_t}}
     \left(X_{t+\Delta}-\hat X_t\right)  \varphi_1^{\mathrm{T}} - 
    \beta_t\eta\bar\psi_1\nu_{2}\bar\psi_1^{\mathrm{T}}\right]\times{}\\
    {}\times
    \left(\fr{\partial}{\partial Y_t} + \eta^{\mathrm{T}} \beta_t^{\mathrm{T}} 
    \fr{\partial}{\partial \hat X_t}\right)\xi^{\mathrm{T}}\bigg\}+{}\\
{} + \fr{1}{2}\, \mm^N\bigg\{ \left(X_{t+\Delta} - \hat X_t\right) \times{}\\
{}\times
\biggl\{ \mathrm{tr}\biggl[ 
\bar\psi_1 \nu_{2} \bar\psi_1^{\mathrm{T}}
    \left(\fr{\partial }{\partial Y_t} + 
    2\eta^{\mathrm{T}} \beta_t^{\mathrm{T}} \fr{\partial}{\partial \hat X_t}\right)
    \fr{\partial^{\mathrm{T}}}{\partial Y_t}\biggr]+{}\\
{}+\mathrm{tr}\left[ \beta_t\eta \bar\psi_1 \nu_{2} \bar\psi_1^{\mathrm{T}}
\eta^{\mathrm{T}}\beta^{\mathrm{T}}
    \fr{\partial}{\partial \hat X_t}\,  \fr{\partial^{\mathrm{T}}}
    {\partial \hat X_t}\right]\biggr\}\xi^{\mathrm{T}}\biggr\},
%    \label{e7.8-s2}
    \end{multline*}
при этом точность экстраполятора Пугачёва определяется, согласно~\cite{3-s2}, формулой:
 \begin{multline}
 \dot R_t = \mm^N
 \left[\left( X_{t+\Delta} -\hx\right) \varphi 
 \left(X_{t+\Delta}, t+\Delta\right)^{\mathrm{T}} +{}\right.\\
 {}+\varphi\left(X_{t+\Delta}, t+\Delta\right) 
 \left(X_{t+\Delta}^{\mathrm{T}} -\hx^{\mathrm{T}}\right)-{}\\
{}-\beta_t\eta \yutt \psi_1 \yxtt \nu_2(t) \psi_1\yxtt^{\mathrm{T}}\times{}\\ 
{}\times\eta \yutt^{\mathrm{T}} \beta_t^{\mathrm{T}} +{}\\
\hspace*{-5mm}\left.{}+\psi \left(X_{t+\Delta}, t+\Delta\right) 
 \nu_1(t+\Delta) \psi \left(X_{t+\Delta}, t+\Delta\right)^{\mathrm{T}}
 \right].\!
 \label{e7.9-s2}
 \end{multline}

Для вычисления математических ожиданий в~(\ref{e7.5-s2})--(\ref{e7.9-s2}) 
недостаточно знать
одномерное нормальное распределение составного случайного процесса  $\lk
X_t^{\mathrm{T}} Y_t^{\mathrm{T}}  \hat X_t^{\mathrm{T}}\rk^{\mathrm{T}}$, необходимо также знать совместное нормальное
распределение величин~$ X_t$,  $X_{t+\Delta}$, $Y_t$ и~$\hat X_t$ при каждом~$t$.


Таким образом, приходим к~следующим результатам.

\pagebreak

%\smallskip

\noindent
\textbf{Теорема~7.1.}\
\textit{Пусть векторный стохастический процесс  
$\lk X_t^{\mathrm{T}} Y_t^{\mathrm{T}}\rk^{\mathrm{T}}$ определяется 
негауссовскими уравнениями}~(\ref{e7.1-s2}) \textit{и~обладает 
конечными  двумерными вероятностными моментами. Тогда алгоритм НЭП определяется 
формулами}~(\ref{e7.4-s2})--(\ref{e7.7-s2}), (\ref{e7.9-s2}).

\smallskip

\noindent
\textbf{Теорема~7.2.}\ \textit{Для гауссовской СтС}~(\ref{e7.1-s2}) 
\textit{алгоритм НЭП определяется}~(\ref{e7.4-s2})--(\ref{e7.7-s2}), (\ref{e7.9-s2}).

\smallskip


Аналогично получаются уравнения алгоритмов НЭП для дискретных СтС.

Таким образом, теория условно-оптимальной экстраполяции стохастических процессов дает
возможность строить экстраполяторы Пугачёва для одновременного
оценивания состояния и~параметров системы и~экстраполяции ее
состояния на несколько различных интервалов времени в~реальном
масштабе времени. Все сложные расчеты, необходимые для
проектирования таких экстраполяторов, не опираются на результаты
наблюдения и~могут быть выполнены по априорным данным в~процессе
проектирования. Практическое применение таких экстраполяторов
сводится к~одновременному интегрированию уравнений~(\ref{e4.1-s2}) 
для оценивания текущего и~будущих состояний системы.


\section{Заключение}

Разработана теория аналитического синтеза непрерывных и~дискретных нормальных 
суб- и~услов\-но-оп\-ти\-маль\-ных фильтров и~экстраполяторов\linebreak Пугачёва для обработки 
процессов в~гауссовских и~негаус\-совских СтС, линейных относительно 
состояния.  Результаты допускают обобщение на случай автокоррелированных шумов 
в~наблюдениях. Алгоритмы положены в~основу программного обеспечения StS-Filter  
(version 2016) для стохастических информационных технологий~\cite{7-s2}.


{\small\frenchspacing
 {%\baselineskip=10.8pt
 \addcontentsline{toc}{section}{References}
 \begin{thebibliography}{9}



\bibitem{2-s2}
\Au{Липцер Р.\,Ш., Ширяев~А.\,Н.} 
Статистика случайных процессов.~--- М.: Наука, 1974. 476~с.

\bibitem{3-s2}
\Au{Синицын И.\,Н.}
Фильтры Калмана и~Пугачёва.~--- 2-е изд.~--- М.: Логос, 2007. 776~с.

\bibitem{4-s2}
\Au{Синицын И.\,Н., Синицын~В.\,И.} 
Лекции по нормальной и~эллипсоидальной аппроксимации в~стохастических системах.~--- 
М.: ТОРУС ПРЕСС, 2013. 476~с.

\bibitem{5-s2}
\Au{Синицын И.\,Н.}  
Параметрическое статистическое и~аналитическое моделирование распределений 
в~нелинейных стохастических системах на многообразиях~// 
Информатика и~её применения, 2013. Т.~7. Вып.~2. С.~4--16.

\bibitem{1-s2} %5
\Au{Синицын И.\,Н., Корепанов~Э.\,Р.}
Нормальные услов\-но-оп\-ти\-маль\-ные фильтры Пугачёва для дифференциальных 
стохастических систем, линейных относительно состояния~// Информатика и~её 
применения, 2015. Т.~9. Вып.~2. С.~30--38.

\bibitem{6-s2}
\Au{Ройтенберг Я.\,Н.} Автоматическое управление.~---
 3-е изд., перераб. и~доп.~--- М.: Наука, 1992. 576~с.

\bibitem{7-s2}
\Au{Корепанов Э.\,Р.} 
Стохастические информационные технологии на основе фильтров Пугачёва~// 
Информатика и~её применения, 2011. Т.~5. Вып.~2. С.~36--57.
\end{thebibliography}

 }
 }

\end{multicols}

\vspace*{-3pt}

\hfill{\small\textit{Поступила в~редакцию 02.02.16}}

\vspace*{8pt}

%\newpage

%\vspace*{-24pt}

\hrule

\vspace*{2pt}

\hrule

%\vspace*{8pt}



\def\tit{NORMAL PUGACHEV CONDITIONALLY-OPTIMAL
FILTERS  AND~EXTRAPOLATORS FOR~STATE
LINEAR STOCHASTIC SYSTEMS}

\def\titkol{Normal Pugachev conditionally-optimal
filters  and~extrapolators for~state
linear stochastic systems}

\def\aut{I.\,N.~Sinitsyn and E.\,R.~Korepanov}

\def\autkol{I.\,N.~Sinitsyn and E.\,R.~Korepanov}

\titel{\tit}{\aut}{\autkol}{\titkol}

\vspace*{-9pt}

\noindent
Institute of Informatics Problems, Federal Research Center 
``Computer Science and Control'' of the Russian Academy of Sciences,
44-2~Vavilov Str., Moscow 119333, Russian Federation

\def\leftfootline{\small{\textbf{\thepage}
\hfill INFORMATIKA I EE PRIMENENIYA~--- INFORMATICS AND
APPLICATIONS\ \ \ 2016\ \ \ volume~10\ \ \ issue\ 2}
}%
 \def\rightfootline{\small{INFORMATIKA I EE PRIMENENIYA~---
INFORMATICS AND APPLICATIONS\ \ \ 2016\ \ \ volume~10\ \ \ issue\ 2
\hfill \textbf{\thepage}}}

\vspace*{3pt}



\Abste{The analytical synthesis theory of continuous and discrete sub- and Pugachev 
conditionally optimal filters and extrapolators for information processing in linear 
state stochastic systems (StS) is presented. For Gaussian  StS,
 Liptzer and Shiraev 
performed the first works for filters and extrapolators synthesis. For 
non-Gaussian StS, the first works belong to Pugachev and Sinitsyn. 
Stochastic equatuins 
for state and observation of continuous and discrete StS are given. 
Algorithms for continuous normal sub- and conditionally optimal filters 
and extrapolators are presented. The corresponding algorithms for discrete StS 
are also given. The developed algorithms are the basis of the
software tool ``StS-Filter, 2016.''
The results may be developed for autocorrelated noises and multiplicative noises.}

\KWE{Liptser--Shiraev filter (LSF);
Liptser--Shiraev conditions;
normal approximation method (NAM) for \textit{a~posteriori} density;
normal conditionally optimal Pugachev filter (NPF);
stochastic systems (StS); state linear StS; statistical linearization method (SLM)}

\DOI{10.14357/19922264160202}

%\vspace*{-12pt}

%\Ack
%\noindent



%\vspace*{3pt}

  \begin{multicols}{2}

\renewcommand{\bibname}{\protect\rmfamily References}
%\renewcommand{\bibname}{\large\protect\rm References}

{\small\frenchspacing
 {%\baselineskip=10.8pt
 \addcontentsline{toc}{section}{References}
 \begin{thebibliography}{9}




\bibitem{2-s2-1}
\Aue{Liptser, R.\,Sh., and A.\,N.~Shiryaev}. 1974. 
\textit{Statistika sluchaynykh protsessov} [Statistics of stochastic proesses].~--- 
Moscow: Nauka. 476~p.

\bibitem{3-s2-1}
\Aue{Sinitsyn, I.\,N.} 2007.
\textit{Fil'try Kalmana i~Pugacheva} [Kalman and Pugachev filters]. 2nd ed. Moscow: Logos. 776~p.

\bibitem{4-s2-1}
\Aue{Sinitsyn, I.\,N., and V.\,I.~Sinitsyn}. 2013. 
\textit{Lektsii po normal'noy i ellipsoidal'noy approksimatsii 
v~stokhasticheskikh sistemakh}  [Lectures on normal and ellipsoidal approximation 
of distributions in stochastic systems].  Moscow: TORUS PRESS. 476~p.

\bibitem{5-s2-1}
\Aue{Sinitsyn, I.\,N.} 2013. 
Parametricheskoe statisticheskoe i~analiticheskoe modelirovanie raspredeleniy 
v~nelineynykh stokhasticheskikh sistemakh na mnogoobraziyakh [Parametric statistical 
and analytical modeling of distributions in stochastic systems on manifolds].
\textit{Informatika i~ee Primeneniya}~--- \textit{Inform Appl.} 7(2):4--16.

\bibitem{1-s2-1} %5
\Aue{Sinitsyn, I.\,N., and E.\,R.~Korepanov}. 2015.
Nor\-mal'\-nye uslovno optimal'nye fil'try Pugacheva dlya dif\-fe\-ren\-tsi\-al'\-nykh 
stokhasticheskikh sistem, lineynykh otnositel'no so\-sto\-yaniya [Normal Pugachev 
filters and extrapolators for state linear stochastic systems].
\textit{Informatika i~ee Primeneniya}~--- \textit{Inform. Appl}  9(2):30--38.

\bibitem{6-s2-1}
\Aue{Roytenberg, Ya.\,N.} 1992. 
\textit{Avtomaticheskoe upravlenie} 
[Automatic control]. 3rd ed. Moscow: Nauka. 576~p.

\bibitem{7-s2-1}
\Aue{Korepanov, E.\,R.} 2011. 
Stokhasticheskie informatsionnye tekhnologii na osnove fil'trov Pugacheva 
[Stochastic informational technologies based on Pugachev filters].
\textit{Informatika i~ee Primeneniya}~--- \textit{Inform. Appl}. 5(2):36--57.
\end{thebibliography}

 }
 }

\end{multicols}

\vspace*{-3pt}

\hfill{\small\textit{Received February 2, 2016}}

\Contr

\noindent
\textbf{Sinitsyn Igor N.} (b.\ 1940)~---
Doctor of Science in technology, professor,
Honored scientist of RF, Head of Department, Institute of Informatics Problems, Federal Research Center ``Computer Science and
Control'' of the Russian Academy of Sciences, 44-2~Vavilov Str.,
Moscow 119333, Russian Federation; sinitsin@dol.ru

\vspace*{3pt}

\noindent
\textbf{Korepanov Eduard R.} (b.\ 1966)~---
Candidate of Science (PhD) in technology, 
Head of Laboratory, Institute of Informatics Problems, Federal Research Center 
``Computer Science and Control'' of the Russian Academy of Sciences, 
44-2~Vavilov Str., Moscow 119333, Russian Federation; ekorepanov@ipiran.ru 

\label{end\stat}


\renewcommand{\bibname}{\protect\rm Литература} %2
\def\crn{c_{r,\nu}}
\def\prn{p_{r,\nu}}
\def\qrm{q_{r,\mu}}
\def\kkk{\kappa}
\def\hx{{\hat X}}
\def\pss{(\psi_1\nu_0\psi_1^{\mathrm{T}})}
\def\srn{S_{r,\nu}}





\def\stat{sinits-1}

\def\tit{ЭЛЛИПСОИДАЛЬНЫЕ СУБОПТИМАЛЬНЫЕ ФИЛЬТРЫ
ДЛЯ~НЕЛИНЕЙНЫХ СТОХАСТИЧЕСКИХ
СИСТЕМ НА~МНОГООБРАЗИЯХ$^*$}

\def\titkol{Эллипсоидальные субоптимальные фильтры
для~нелинейных стохастических
систем на~многообразиях}

\def\aut{И.\,Н.~Синицын$^1$, В.\,И.~Синицын$^2$,  Э.\,Р.~Корепанов$^3$}

\def\autkol{И.\,Н.~Синицын, В.\,И.~Синицын,  Э.\,Р.~Корепанов}

\titel{\tit}{\aut}{\autkol}{\titkol}

\index{Синицын И.\,Н.}
\index{Синицын В.\,И.}
\index{Корепанов Э.\,Р.}
\index{Sinitsyn I.\,N.}
\index{Sinitsyn V.\,I.}
\index{Korepanov E.\,R.}

{\renewcommand{\thefootnote}{\fnsymbol{footnote}} \footnotetext[1]
{Работа выполнена при поддержке РФФИ (проект 15-07-02244).}}


\renewcommand{\thefootnote}{\arabic{footnote}}
\footnotetext[1]{Институт проблем информатики Федерального исследовательского
центра <<Информатика и~управление>> Российской академии наук, sinitsin@dol.ru}
\footnotetext[2]{Институт проблем информатики Федерального исследовательского
центра <<Информатика и~управление>> Российской академии наук, vsinitsin@ipiran.ru}
\footnotetext[3]{Институт проблем информатики Федерального исследовательского
центра <<Информатика и~управление>> Российской академии наук, ekorepanov@ipiran.ru}


\Abst{Разработана теория аналитического синтеза эллипсоидальных  субоптимальных 
фильтров (ЭСОФ) для нелинейных дифференциальных стохастических систем (СтС) 
на многообразиях (МСтС). Рассмотрены случаи гауссовских и~негауссовских СтС. Алгоритмы 
положены в~основу модуля экспериментального программного обеспечения  StS-Filter 
(version 2016). Результаты допускают развитие на случай дискретных СтС. 
Теоретический и~практический интерес представляет теория  ЭСОФ на основе 
ненормированных распределений.}

\KW{апостериорное одномерное распределение;
винеровский шум; метод эллипсоидальной аппроксимации (МЭА);
метод эллипсоидальной линеаризации (МЭЛ);
пуассоновский шум; стохастическая система на многообразиях (МСтС);
субоптимальный фильтр (СОФ); уравнения точности МЭА и~МЭЛ;
уравнения чувствительности МЭА и~МЭЛ}

\DOI{10.14357/19922264160203} 

%\vspace*{-4pt}

\vskip 10pt plus 9pt minus 6pt

\thispagestyle{headings}

\begin{multicols}{2}

\label{st\stat}

\section{Введение}

В~[1, 2] метод ортогональных разложений (МОР)  был развит для аналитического 
моделирования одно- и~многомерных распределений в МСтС 
и~дано его применение для задач надежности и~без\-опас\-ности технических сис\-тем.
В~[3] представлена теория субоптимальных фильтров (СОФ) на базе методов нормальной 
аппроксимации (МНА) и~статистической линеаризации (МСЛ), а~также МОР для 
МСтС с~винеровскими шумами в~уравнениях наблюдения и~винеровскими и~пуассоновскими 
шумами в~уравнениях состояния.
В~основу СОФ были положены точные нелинейные уравнения для апостериорного одномерного 
распределения.

Рассмотрим развитие~[3] на случай, когда апостериорное одномерное распределение 
ошибки фильтрации допускает эллипсоидальную аппроксимацию (ЭА)~[4--7]. В~разд.~2 и~3 
приведены точ-\linebreak ные фильтрационные уравнения, а~также уравнения точности и~чувствительности 
на основе МОР. Элементы эллипсоидального анализа распределений даются в~разд.~4. 
Раздел~5 содержит уравнения ЭСОФ на основе методов ЭА
(МЭА) и~эллипсоидальной линеаризации (МЭЛ). Заключение содержит выводы 
и~некоторые обобщения.

\section{Точные фильтрационные уравнения}

\vspace*{-18pt}

На практике часто возникают задачи непрерывного
определения состояния системы по результатам непрерывных наблюдений.
Так как наблюдения всегда сопровождаются случайными ошибками, то
следует говорить не об определении состояния сис\-те\-мы, а~о~его
оценивании (фильтрации, экстраполяции, интерполяции и~т.\,д.) путем
статистической обработки результатов наблюдений. Будем рас\-смат\-ри\-вать
задачи фильтрации состояния сис\-тем, моделями которых могут служить
стохастические дифференциальные  уравнения с~винеровскими и~пуассоновскими шумами.

Часто стохастические дифференциальные уравнения модели изучаемой системы могут
иметь неизвестные параметры и,~как правило, всегда содержат
параметры, известные с~ограниченной точностью. Поэтому возникает
задача непрерывного оценивания неизвестных параметров системы
(точнее, ее модели) по результатам непрерывных наблюдений.
Предположим, что правые части уравнений зависят от конечного множества 
неизвестных параметров, которые
будем рассматривать как компоненты век-\linebreak\vspace*{-12pt}

\pagebreak

\noindent
тора параметров~$\theta$.
Одним из возможных подходов в~таких случаях является следующий
прием: неизвестный векторный параметр~$\theta$ считают стохастическим
процессом  (СтП) $\Theta \hm=\Theta_t$, который определяется
дифференциальным уравнением $\dot\Theta_t \hm=0$, и~включают
компоненты этого векторного процесса в~вектор состояния системы
(<<расширяют>> вектор состояния путем включения в~него неизвестных
параметров в~качестве дополнительных компонент).
Таким образом, задача непрерывного оценивания неизвестных
параметров модели системы сводится к~задаче непрерывного
оценивания состояния системы с~расширенным вектором состояния.
От неизвестных параметров могут зависеть и~уравнения наблюдения. 
Эти параметры следует включить 
в~вектор~$\theta$ и,~следовательно, в~расширенный вектор состояния.

Пусть векторный СтП $\lk X_t^{\mathrm{T}} Y_t^{\mathrm{T}} \rk^{\mathrm{T}}$
определяется системой векторных стохастических дифференциальных
уравнений Ито:
   \begin{multline}
    dX_t =\varphi \left(X_t,Y_t,\Theta, t\right) dt + \psi' \left(X_t,Y_t,\Theta, t\right) 
    dW_0 +{}\\ 
{}+\iii_{R_0^q} \psi''
    \left(X_t,Y_t,\Theta, t,v\right) P^0 (dt, dv)\,,\\ X\left(t_0\right) = X_0\,;
    \label{e2.1-s1}
    \end{multline}
    
    \vspace*{-12pt}
    
    \noindent
    \begin{multline}
dY_t =\varphi_1 \left(X_t,Y_t,\Theta, t\right) dt +
    \psi_1' \left(X_t,Y_t,\Theta, t\right) dW_0 + {}\\
{}+\int\limits_{R_0^q} \psi_1'' \left(X_t,Y_t,\Theta, t,v\right) P^0
    (dt,dv)\,,\\
     Y\left(t_0\right) = Y_0\,.
    \label{e2.2-s1}
    \end{multline}
Здесь $Y_t=Y(t)$~--- $n_y$-мер\-ный наблюдаемый
СтП, $Y_t \hm\in \Delta^y$ ($\Delta^y$~--- гладкое многообразие наблюдений); 
$X_t \hm=X(t)$~--- $n_x$-мер\-ный ненаблюдаемый
СтП (вектор состояния), $X_t \hm\in \Delta^x$ ($\Delta^x$~--- 
гладкое многообразие состояний); $W_0\hm =W_0(t)$~--- $n_w$-мер\-ный
винеровский СтП $(n_w\hm\ge n_y)$ интенсивности  $\nu_0 \hm=\nu_0 (\Theta, t)$; 
$P^0(\Delta,A)\hm=P(\Delta,A)-\mu_P (\Delta,A)$, $P(\Delta,A)$  представляет 
собой для любого множества~$A$ прос\-той пуассоновский СтП, а~$\mu_P (\Delta,A)$~--- 
его математическое ожидание, причем
    $$
    \mu_P (\Delta,A)=\mm P (\Delta,A)=\iii_\Delta \nu_P(\tau, A)\, d\tau\,;
    $$
$\nu_P (\Delta, A)$~--- интенсивность соответствующего пуассоновского
потока событий, $\Delta \hm=(t_1,t_2]$; интегрирование по~$v$
распространяется на все пространство~$R^q$ с~выколотым началом
координат; $\Theta$~--- вектор случайных параметров размерности~$n_\Theta$; 
$\varphi\hm=\varphi(X_t,Y_t,\Theta, t)$,
$\varphi_1\hm=\varphi_1(X_t,Y_t,\Theta, t)$,
 $\psi'\hm=\psi'(X_t,Y_t,\Theta, t)$ и~$\psi_1'\hm=\psi_1'(X_t,Y_t,\Theta, t)$~--- 
 известные функции, отображающие
$R^{n_x}\times R^{n_y}\times  R$ соответственно в~$R^{n_x}$,
$R^{n_y}$, $R^{n_xn_w}$ и~$R^{n_yn_w}$; $\psi''\hm=\psi''(X_t,Y_t,\Theta, t,v)$ 
и~$\psi_1''(X_t,Y_t,\Theta, t,v)$~--- известные
функции, отображающие $R^{n_x}\times R^{n_y}\times R^q$ в~$R^{n_x}$ и~$R^{n_y}$. 
Требуется найти оценку~$\hat X_t$ СтП~$X_t$ 
в~каждый момент времени~$t$ по результатам наблюдения
СтП $Y(\tau)$ до момента~$t$, $Y_{t_0}^t \hm=  \{ Y(\tau) \,: t_0 \hm\le \tau\hm< t\}$.

Следуя~\cite{8-s1}, предположим, что
\begin{itemize}
\item уравнение состояния имеет вид~(\ref{e2.1-s1});

\item уравнение наблюдения~(\ref{e2.2-s1}), во-пер\-вых, не содержит
пуассоновского шума $(\psi_1'' \hm\equiv 0)$, 
а~во-вто\-рых, коэффициент при винеровском шуме~$\psi_1'$  
в~уравнениях наблюдения не зависит от состояния $(\psi_1' (X_t, Y_t,\Theta, t)\hm=\psi_1'
(Y_t,\Theta, t))$.
\end{itemize}

В этом случае уравнения задачи нелинейной фильтрации имеют следующий вид:
\begin{multline}
    dX_t =\varphi \left(X_t, Y_t,\Theta, t\right)\,dt+\psi'
    \left(X_t, Y_t,\Theta, t\right)\,dW_0+{}\\
\!\!{}+\!\int\limits_{R_0^q}\! \psi''\left(X_t, Y_t,\Theta, t,v\right) P^0 (dt, dv)\,,\enskip 
X\left(t_0\right) = X_0;\!\!\label{e2.3-s1}
\end{multline}


\vspace*{-12pt}

\noindent
\begin{multline}
dY_t =\varphi_1 \left(X_t, Y_t,\Theta, t\right)\,dt +
\psi_1 \left(Y_t,\Theta, t\right)\, dW_0\,,\\ Y\left(t_0\right) = Y_0\,.
\label{e2.4-s1}
\end{multline}


Будем считать, что выполнены условия существования и~единственности СтП  
$\lk X_t^{\mathrm{T}}\ Y_t^{\mathrm{T}}\rk^{\mathrm{T}}$, определяемого~(\ref{e2.3-s1}) 
и~(\ref{e2.4-s1}) при соответствующих начальных условиях.

В дальнейшем для стохастического уравнения
\begin{equation}
dZ=a \,dt + b\, dW_0 + \iii_{R_0^q} c P^0 (dt, dv)\label{e2.5-s1}
\end{equation}
потребуется обобщенная формула Ито~\cite{5-s1} 
для дифференциала нелинейной функции $U\hm=U(Z,t)$:
 \begin{multline}
 dU =\lf U_t + U_z^{\mathrm{T}} a + \fr{1}{2}\,\mathrm{tr}\, \lk U_{zz} b\nu b^{\mathrm{T}}\rk\rf dt +{}\\
 {}+ \iii_{R_0^q} \lk U(Z+c,t)^{\mathrm{T}} - U(Z,t)^{\mathrm{T}} - U_z^{\mathrm{T}} c\rk 
 \mu_P (dt, dv)+{}\\
   {}+U_Z^{\mathrm{T}} b \,dW_0 +{}\\
   {}+ \iii_{R_0^q} \lk U(Z+c,t) - U(Z,t) \rk P^0 (dt, dv)\,.
   \label{e2.6-s1}
   \end{multline}
Здесь $a$, $b$ и~$c$~--- известные функции~$Z$ и~$t$.

Как известно~\cite{4-s1, 5-s1}, для любых СтП~$X_t$ и~$Y_t$ оптимальная 
оценка~$\hat X^t$, минимизирующая средний квадрат ошибки в~каждый момент времени~$t$, 
представляет собой апостериорное математическое ожидание СтП~$X_t$: 
$\hat X_t \hm= \mm \lk X_t \mid Y_{t_0}^t\rk$. Чтобы найти это условное 
математическое ожидание, необходимо знать $p_t \hm= p_t (x)$~--- 
апостериорное одномерное распределение СтП~$X_t$.

В основе уравнений оптимальной (в~смыс\-ле минимума средней квадратической ошибки) 
фильт\-ра\-ции для уравнений~(\ref{e2.3-s1}) и~(\ref{e2.4-s1}) 
в~силу~(\ref{e2.6-s1}) лежит следу\-ющая формула для стохастического дифференциала 
апостериорного математического ожидания скалярной функции  $f\hm=f(X,t)$ вектора 
состояния~\cite{8-s1}:
\begin{multline}
d \hat f = d \mm_{\Delta^x}^{p_t} \left[ f_t 
(X,t) + f_x (X,t)^{\mathrm{T}} \vrp (X,Y,t) +{}\right.\\[1pt]
{}+ \fr{1}{2}\,\mathrm{tr}\, 
\left\{ f_{xx} (X,t) (\psi' \nu_0 {\psi'}^{\mathrm{T}}) (X,Y,t)\right\}+{}\\[1pt]
{}+ \iii_{R_0^q}  \left\{ 
\vphantom{f_x (X,t)^{\mathrm{T}}}
f \left(X+ \psi'' , t\right) - f(X,t) -{}\right.\\[1pt]
\left.\left.{}-
f_x (X,t)^{\mathrm{T}} \psi''(X,Y,t)\right\} \nu_P (t, dv)\mid Y_{t_0}^t \right] dt+{}\\[1pt]
  {}+\mm_{\Delta^x}^{p_t} \left\{ f(X,t) \left[ 
  \vrp_1 (X,Y,t)^{\mathrm{T}} -
  \hat \vrp_1^{\mathrm{T}}\right] +{}\right.\\[1pt]
\left.  {}+ f_x (X,t)^{\mathrm{T}} 
  \left(\psi\nu_0\psi_1^{\mathrm{T}}\right) (X,Y,t) \mid Y_{t_0}^t\right\} \times{}\\[1pt]
    {}\times \psi_1 \nu_0 \psi_1^{\mathrm{T}})^{-1} (Y,t) \left(dY-\hat\vrp_1 \,dt\right).
    \label{e2.7-s1}
    \end{multline}
Здесь для краткости аргумент~$\Theta$ опущен; $X\hm=X_t$, $Y\hm=Y_t$, $\nu\hm=\nu_0$ 
и~$\nu_P$~--- интенсивности~$W_0$ и~$P^0$;
$\hat\vrp_1$~--- апостериорное математическое ожидание~$\vrp_1$ при заданной условной плотности  
$p_t\hm=p_t (x, \Theta)$:
\begin{equation*}
\hat\vrp_1 = \mm_{\Delta^x}^{p_t} \lk\vrp_1(X,Y,t)\rk\,.
%\label{e2.8-s1}
\end{equation*}

Полагая в~(\ref{e2.5-s1}) $f(X,t) \hm\equiv g_t (\la,\Theta) \hm=
\mm_{\Delta^x}^{p_t}\lk\exp (i\la^{\mathrm{T}} X)\rk$, получим
точное нелинейное фильтрационное уравнение для  характеристической 
функции $g_t (\la,\Theta)$:
\begin{multline}
\!dg_t (\la,\Theta) = {\mm}_{\Delta^x}^{p_t} \left[ 
\left\{
\vphantom{\fr{1}{2}}
 i\la^{\mathrm{T}} \varphi (X_t,Y_t,\Theta, t) - {}\right.\right.
{}-\fr{1}{2}\, \la^{\mathrm{T}}\times{}\\[1pt]
\left.{}\times \left(\psi\nu_0\psi^{\mathrm{T}}\right) 
\left(X_t,Y_t,\Theta, t\right) \la+\gamma \left(\la, X_t, Y_t, \Theta, t\right) 
\vphantom{\fr{1}{2}}\right\}\times{}\\[1pt]
\!\left.{}\times e^{i\la^{\mathrm{T}} X_t} \mid Y_{t_0}^t 
\vphantom{\fr{1}{2}}
\right] dt+ 
{\mm}_{\Delta^x}^{p_t} \left[ 
\left\{ 
\varphi_1 \left(X_t,Y_t,\Theta, t\right)^{\mathrm{T}} -
\hat\varphi_1^{\mathrm{T}} +{}\right.\right.\hspace*{-0.70326pt}\\[1pt]
{}+
    i\la^{\mathrm{T}} \left(\psi\nu_0\psi_1^{\mathrm{T}}\right) 
    \left(X_t,Y_t, \Theta,t\right) \left.  
    e^{i\la^{\mathrm{T}} X_t} \mid Y_{t_0}^t \right\} \times{}\\[1pt]
    \left.{}\times
  \left(\psi_1\nu_0\psi_1^{\mathrm{T}}\right)^{-1} \left(Y_t,\Theta,t\right) 
  \left(dY_t-\hat\varphi_1 \,dt\right)\right], 
  \label{e2.9-s1}
  \end{multline}
где
\begin{multline}
\gamma=\gamma \left(\la, X_t, Y_t, \Theta, t\right)=
\int\limits_{R_0^q}\left[ e^{i\la^{\mathrm{T}} 
\psi''\left(X_t,Y_t,\Theta, t,v\right)} - 
 {}\right.\\
\left.{}-1- i\la^{\mathrm{T}} \psi''\left(X_t,Y_t,\Theta, t,v\right)
\vphantom{e^{i\la^{\mathrm{T}}}}
\right] 
\nu_P (\Theta, t,v) \,dv\,.\label{e2.10-s1}
\end{multline}

Функции $g_t (\la,\Theta)$ и~$p_t(x,\Theta)$ связаны между собой преобразованием 
Фурье~\cite{8-s1}.

Отсюда для гауссовской МСтС~(\ref{e2.3-s1}), (\ref{e2.4-s1}) $(\psi''\hm\equiv 0)$ 
уравнение~(\ref{e2.7-s1}) при  $\gamma\hm=0$ упрощается и~приобретает вид:

\noindent
\begin{multline}
dg_t (\la,\Theta) = {\mm}_{\Delta^x}^{p_t} \left[
\left\{ \vphantom{\fr{1}{2}}
i\la^{\mathrm{T}} \varphi (X_t,Y_t,\Theta, t) -{}\right.\right.\\
\left.\left.{}- \fr{1}{2}\, \la^{\mathrm{T}} 
\left(\psi\nu_0\psi^{\mathrm{T}}\right) 
\left(X_t,Y_t,\Theta, t\right) \la\right\} 
e^{i\la^{\mathrm{T}} X_t} \mid Y_{t_0}^t\right] dt+{}\\
{}+ {\mm}_{\Delta^x}^{p_t} \biggl[ 
\biggl\{ 
\varphi_1 \left(X_t,Y_t,\Theta, t\right)^{\mathrm{T}} -
\hat\varphi_1^{\mathrm{T}} +{}\\
{}+
    i\la^{\mathrm{T}} \left(\psi\nu_0\psi_1^{\mathrm{T}}\right) 
    \left(X_t,Y_t, \Theta,t\right) e^{i\la^{\mathrm{T}} X_t} \mid Y_{t_0}^t
   \biggr\} \times{}\\
   {}\times\left(\psi_1\nu_0\psi_1^{\mathrm{T}}\right)^{-1} \left(Y_t,\Theta,t\right) 
   \left(dY_t-\hat\varphi_1\, dt\right)
    \biggr]. \label{e2.11-s1}
    \end{multline}

Если функция $\psi''$ в~(\ref{e2.3-s1}) допускает представление~\cite{8-s1, 9-s1}
\begin{equation}
\psi'' = \psi' \omega (\Theta, v)\,,\label{e2.12-s1}
\end{equation}
где  $P^0 (\Delta, A) = P^0 ((0,t], dv)$, то уравнения~(\ref{e2.3-s1}), 
(\ref{e2.4-s1}) примут следующий вид:

\noindent
\begin{multline}
\dot X_t =\vrp \left(X_t, Y_t, \Theta, t\right)+\psi' 
\left(X_t, Y_t, \Theta, t\right)V(\Theta, t)\,,\\
 X\left(t_0\right)=X_0\,;
\label{e2.13-s1}
\end{multline}

\vspace*{-12pt}

\noindent
\begin{multline}
\dot Y_t = \vrp\left(X_t, Y_t, \Theta, t\right)+\psi_1 
\left(Y_t, \Theta, t\right) V_0 (\Theta, t)\,,\\
 Y\left(t_0\right) = Y_0\,.
\label{e2.14-s1}
\end{multline}
Здесь
$V_0 (\Theta, t)\hm =\dot W_0 (\Theta, t)$; $V (\Theta, t)\hm=\dot{\bar W}  (\Theta, t)$,
\begin{equation*}
\bar W (\Theta, t) = W_0  (\Theta, t)+ \iii_{R_0^q} \omega (\Theta, v) 
P^0 ((0,t],dv)\,,
%\label{e2.15-s1}
\end{equation*}
где $\nu_P  (\Theta, t,v) dv\hm= \lk \prt \mu  (\Theta, t,v)/\prt t\rk dv$~--- 
интенсивность пуассоновского потока скачков, равных $\omega (\Theta, t)$.
%
При этом логарифмические производные от одномерных характеристических функций 
определяются известными формулами:

\noindent
\begin{align}
\chi^{W_0} (\rho; t) &= -\fr{1}{2}\, \rho^{\mathrm{T}} \nu_0  (\Theta, t) \rho\,;\notag
\\
\chi^{\bar W} (\rho;t) &= - \fr{1}{2}\,\rho^{\mathrm{T}}  (\Theta, t) 
\rho^{\mathrm{T}}+ {}\notag\\
&\hspace*{-20mm}{}+\!\iii_{R_0^q} \!\lk e^{i\rho^{\mathrm{T}} \omega (\Theta, v)} 
-1- i\rho^{\mathrm{T}} \omega (\Theta, v)\rk \!\nu_P  (\Theta, t,v)\,dv.
\label{e2.16-s1}\!\!\!\!\!
\end{align}
В таком случае уравнение для апостериорной одномерной характеристической 
функции имеет вид~(\ref{e2.9-s1}), где функция~(\ref{e2.10-s1}) допускает 
следующую запись:

\noindent
   \begin{multline*}
    \gamma = \iii_{R_0^q} \left[ e^{ i\la^{\mathrm{T}} \psi' 
    \left(X_t, Y_t, \Theta, t\right)
    \omega(\Theta, v)} - 1- {}\right.\\
\left.    {}-i\la^{\mathrm{T}} 
    \psi'\left(X_t, Y_t, \Theta, t\right)
    \omega (\Theta, v)
    \vphantom{e^{ i\la^{\mathrm{T}}}}
    \right]
    \nu_P  (\Theta, t,v)\,dv\,.
   \end{multline*}
   
   \pagebreak

Таким образом, можно сформулировать следующие утверждения разд.~2.

\smallskip

\noindent
\textbf{Теорема~2.1.}\ \textit{Пусть для МСтС}~(\ref{e2.3-s1}), 
(\ref{e2.4-s1}) \textit{выполнены условия существования и~единственности решения, 
а~мат\-ри\-ца  $\si_1 \hm= \psi_1 \nu_0 \psi_1^{\mathrm{T}}$ не вырождена. 
Тогда при условии ограниченности соответствующих математических ожиданий точное 
фильтрационное уравнение для условной одномерной нормированной характеристической 
функций имеет вид}~(\ref{e2.9-s1}).

\smallskip

\noindent
\textbf{Теорема 2.2.}\ \textit{В~условиях теоремы~$2.1$ при отсутствии пуассоновских 
шумов точное фильтрационное уравнение для условной одномерной нормированной 
характеристической функции имеет вид}~(\ref{e2.11-s1}).


\smallskip

\noindent
\textbf{Теорема~2.3}~\cite{8-s1, 9-s1}. \textit{Пусть для МСтС}~(\ref{e2.13-s1}), 
(\ref{e2.14-s1}) \textit{выполнены условия существования и~единственности решения, 
имеет место представление}~(\ref{e2.12-s1}), 
\textit{а~мат\-ри\-ца  $\si_1 \hm= \psi_1 \nu_0 \psi_1^{\mathrm{T}}$ не вырождена. 
Тогда при условии ограниченности соответствующих математических ожиданий точное 
фильтрационное уравнение  имеет вид}~(\ref{e2.9-s1}) 
\textit{при условии}~(\ref{e2.10-s1}).

\smallskip

Как известно~\cite{8-s1}, точное решение фильтрационных уравнений  возможно
только в~случаях, когда уравнения гауссовской дифференциальной МСтС
линейны или линейны лишь относительно вектора состояния~$X_t$ при
независимой от состояния функции~$\psi$. Эти уравнения
 дают точное решение задачи оптимальной нелинейной фильтрации.  Однако это решение
  не может быть реализовано практически. Для
 нахождения оптимальной оценки вектора состояния необходимо решить
 фильтрационное уравнение  для апостериорной характеристической функции
 (или  фильтрационное уравнение  для апостериорной плотности   вектора
 состояния~$X_t$) после получения результатов наблюдений, затем вы\-чис\-лить оптимальную оценку вектора~$X_t$.
Но методов точного решения этих
 уравнений  в~общем случае пока еще не существует.

 В задачах реального времени численное решение фильтрационных уравнений 
 (или он\-лайн-оце\-ни\-ва\-ния) тоже
 невозможно, так как для этого требуется много времени, а~решать их
 необходимо каждый раз после получения результатов наблюдений.
 Кроме того, практическое применение точной теории оптимальной нелинейной фильтрации
 имеет смысл только в~тех случаях, когда оценки можно вычислять 
 в~реальном масштабе времени по мере получения результатов
 наблюдений. Точная теория дает оптимальные
 оценки в~каждый момент~$t$ по результатам наблюдений, полученным
 к~этому моменту, без использования последующих результатов
 наблюдений. Если эти оценки не могут быть вычислены в~тот же
 момент~$t$ или хотя бы с~фиксированным приемлемым запаздыванием
 и~их вычисление приходится откладывать на будущее, то нет
 никакого смысла отказываться от использования наблюдений,
 получаемых после момента~$t$, для оценивания состояния системы 
 в~момент~$t$. Поэтому для статистической обработки результатов
 после окончания наблюдений, т.\,е.\ для оф\-лайн-оце\-ни\-ва\-ния,
 целесообразно применять известные из математической статистики методы
 постобработки информации~\cite{8-s1}.

 Необходимость обработки результатов наблюдений в~реальном
 масштабе времени непосредст\-вен\-но в~процессе эксперимента
 привела  к~появлению ряда приближенных методов оптимальной\linebreak нелиней\-ной  фильтрации,
 на\-зы\-ва\-емых обычно методами  субоптимальной фильтрации~\cite{8-s1}. Одни
 приближенные методы основаны на  приближенном решении фильтрационных
 уравнений, а~другие~---  на превращении формул
 для стохастических дифференциалов оптимальной
 оценки~$\hat X_t$ и~апостериорной ковариационной  матрицы ошибки~$R_t$ 
 в~стохастические дифференциальные уравнения  
 для~$\hat X_t$ и~$R_t$ путем разложения функций~$\varphi$, $\varphi_1$, 
 $\psi_1$ или $\varphi$, $\varphi_1$, $\psi'$, $\psi''$, $\psi$, $\psi_1$ 
 в~степенные ряды и~отбрасывания остаточных членов.

 Для приближенного решения уравнения  для апостериорной
одномерной характеристической функции  $g_1(\la, \Theta)$ вектора~$X_t$ можно
использовать\linebreak
 методы аналитического моделирования, основанные на 
параметризации одномерных
 распределений СтП, определяемого стохастическим
 дифференциальным уравнением~\cite{8-s1}.  Эти методы
 позволя\-ют изучить
 стохастические дифференциальные уравнения для параметров
 апостериорного распределения. Простейшим таким методом является
 МНА апостериорного распределения.
Исключительно важное практическое значение имеют квазилинейные
фильтры, получаемые с~помощью методов эквивалентной линеаризации~\cite{8-s1}.

\section{Субоптимальные фильтры на~основе метода ортогональных разложений}

При аппроксимации апостериорной одномерной плотности отрезком ее ортогонального 
разложения~\cite{1-s1, 2-s1}
\begin{multline}
p_t (x, \Theta)\approx p^* (x; \Theta, \vartheta) = {}\\
{}=
w (x; \Theta) \lk 1+ \sss_{l=3}^N \sss_{|\nu | =l} c_\kappa p_\kappa (x)\rk
\label{e3.1-s1}
\end{multline}
естественно принять за параметры,  образующие вектор~$\vartheta$, 
апостериорные математическое
 ожидание~$\hat X_t$, ковариационную матрицу~$R_t$ вектора~$X_t$, а~также
 коэффициенты ортогонального разложения (КОР) $c_\kappa$ $(\lv \nu\rv \hm= 3\tr N)$.
 Здесь КОР определяется формулой:
\begin{equation}
c_\kappa = \lk q_\kappa \left(\fr{\partial }{i\partial \la}\right) 
g_t (\la,\Theta)\rk_{\la=0}\,.\label{e3.2-s1}
\end{equation}
Заметим, что полином~$q_\kappa$ зависит от~$\hat X_t$ и~$R_t$.

На основе~(\ref{e2.7-s1}) и~(\ref{e2.9-s1}) для  гауссовской 
МСтС~(\ref{e2.3-s1}), (\ref{e2.4-s1}) при  $\psi''\hm=0$ 
получим, что ортогональный СОФ (ОСОФ) определяется следующими уравнениями:
\begin{align}
d\hat X_t&=f\,dt + h \left(dY_t- f^{(1)}\, dt\right)\,;\label{e3.3-s1}\\
dR_t&= \left(f^{(2)} -h\psi_1\nu_0\psi_1^{\mathrm{T}} h^{\mathrm{T}}\right) dt 
+{}\notag\\
&\hspace*{20mm}{}+\sss_{r=1}^{n_y} \rho_r \left(dY_r -f_r^{(1)}\, dt\right)\,.\label{e3.4-s1}
\end{align}
Здесь введены обозначения:
    \begin{equation}
    \left.
    \begin{array}{l}
    f= f\left(Y_t,\vartheta,\Theta, t\right)=\mm_{\Delta^x}^{p^*} 
    \left[\varphi\left(Y_t,X,\Theta,t\right)\right] \,;\\[6pt]
f^{(1)}= \left\{ f_r^{(1)}\right\}= f^{(1)}\left(Y_t,\vartheta,\Theta, t\right)={}\\[6pt]
\hspace*{20mm}{}=
    \mm_{\Delta^x}^{p^*}  \lk\varphi_1\left(Y_t,X,\Theta,t\right)\rk\,;\\[6pt]
f^{(2)}= f^{(2)}\left(Y_t,\vartheta,\Theta,t\right)={}\\
\hspace*{5mm}{}=    \mm_{\Delta^x}^{p^*} \left[ \left(X-\hat X_t\right)\varphi
    \left(Y_t,X,\Theta,t\right)^{\mathrm{T}}+{}\right.\\[6pt]
\hspace*{10mm}{}+\varphi\left(Y_t,X,\Theta,t\right) \left(X^{\mathrm{T}}-
\hat X_t^{\mathrm{T}}\right) 
+{}\\[6pt]
\left.\hspace*{20mm}{}+ \left(\psi\nu_0\psi^{\mathrm{T}}\right) 
\left(Y_t,X,\Theta,t\right)
\vphantom{\left(Y_t,X,\Theta,t\right)^{\mathrm{T}}}
\right]\,;\\[6pt]
   h= h\left(Y_t,\vartheta,\Theta,t\right)={}\\[6pt]
\hspace*{15mm}   {}=\biggl\{
    \mm_{\Delta^x}^{p^*} \left[ X\varphi_1
    \left(Y_t,X,\Theta,t\right)^{\mathrm{T}}+{}\right.\\[6pt]
\left.\hspace*{15mm}{}+ (\psi\nu_0\psi_1^{\mathrm{T}}) \left(Y_t,X,\Theta,t\right)
\vphantom{    \left(Y_t,X,\Theta,t\right)^{\mathrm{T}}}
\right]-{}\\[6pt]
\hspace*{8mm}{}-
    \hat X_t f^{(1)T}\biggr\} \pss^{-1} \left(Y_t,\Theta,t\right)\,;\\[6pt]
\rho_r= \rho_r\left(Y_t,\vartheta,\Theta,,t\right)={}\\[6pt]
\hspace*{1mm}{}=
   \mm_{\Delta^x}^{p^*}  
   \left[ \left(X-\hat X_t\right) 
   \left(X^{\mathrm{T}}-\hat X_t^{\mathrm{T}}\right)\times{}\right.\\
\hspace*{-3.2mm}{}\times a_r\left(Y_t,X,\Theta,t\right)+ \left(
\!X-\hat X_t\!\right) b_r\left(Y_t,X,\Theta,t\right)^{\mathrm{T}}\!
+{}\\
\left.\hspace*{-4.7mm}{}+ b_r\left(Y_t,X,\Theta,t\right)\left(
\!X^{\mathrm{T}}-\hat X_t^{\mathrm{T}}\!\right)
\right]\ \left(r=1\tr n_y\right).\!\!
\end{array}\!\!
\right\}\!\!\!
    \label{e3.5-s1}
    \end{equation}

Далее перепишем~(\ref{e3.3-s1}), (\ref{e3.4-s1}) покоординатно:
\begin{align}
d\hat X_s &= f_s \,dt + h_s \left(dY_t- f^{(1)}\, dt\right) = {}\notag\\
&
A^{\hat X_s}\, dt + B^{\hat X_s}\, dY_t\enskip (s=1\tr
    n_x)\,;\label{e3.6-s1}\\
 dR_{sq} &=\left(f_{sq}^{(2)} - h_s\psi_1\nu_0\psi_1^{\mathrm{T}}
    h_q^{\mathrm{T}} \right) dt +{}\notag\\
    &\hspace*{-10mm}{}+\eta_{sq} 
    \left(dY_t - f^{(1)}\, dt\right)=A^{R_{sq}}\, dt + B^{R_{sq}}\, dY_t\,,
    \label{e3.7-s1}
    \end{align}
где $\hat X_s (t_0) = X_{s0}$; $R_{sq} (t_0)\hm = R_{sq0}$;  
$s,q\hm=1\tr n_x$; $\eta_{sq}$~--- мат\-ри\-ца-стро\-ка, 
элементами которой служат
соответствующие элементы матрицы  $\rho_1\tr \rho_{n_1}$:
\begin{equation*}
\eta_{sq} =\eta_{e_s+e_q} = \lk \rho_{1sq}\cdots \rho_{msq}\rk
    \enskip (s,q,=1\tr n_x)\,.
    %\label{e3.8-s1}
    \end{equation*}
Здесь и~далее для краткости индекс~$t$ сохраним только у~$Y_t$. 
По формуле дифференцирования Ито для винеровского СтП, учитывая~(\ref{e3.6-s1}) 
и~(\ref{e3.7-s1}), находим в~силу~(\ref{e3.2-s1}) стохастический дифференциал:
\begin{multline*}
dc_\kappa =\lk d\lf  q_\kappa \left(\fr{\partial}{i\partial \la}\right) g_t
    (\la,\Theta)\rf\rk_{\la=0}={}\\
{}=\sss_{s=1}^{n_x} \lk\partial q_\kappa
    \left(\fr{\partial }{i\partial \la}\right) \partial \hat X_s g_t
    (\la,\Theta)\rk_{\la=0}d \hat X_s+{}\\
{}+\sss_{s,u=1}^{n_x} \lk\partial
    q_\kappa \left(\fr{\partial}{i\partial \la}\right) \partial R_{su} g_t
    (\la,\Theta)\rk_{\la=0}dR_{su} +{}\\
    {}+ \lk q_\kappa \left(
    \fr{\partial }{i\partial \la}\right) d g_t (\la,\Theta)\rk_{\la=0}+{}\\
{}+ \biggl\{ \fr{1}{2}\!\! \sss_{s,u=1}^{n_x}\! \lk \fr{\partial^2 q_\kappa
    \left({\partial}/(i\partial \la)\right)
    g_t (\la,\Theta)}{\partial \hat X_s
    \partial \hat X_u }\rk_{\la=0}\!\! h_s \psi_1\nu_0\psi_1^{\mathrm{T}} h_u^{\mathrm{T}} +{}\\
    {}+
    \fr{1}{2}\! \sss_{s,u,k,l=1}^{n_x} \left[ \fr{\partial^2 q_\kappa (\partial
    /(i\partial \la))
    g_t (\la,\Theta)}{\partial R_{su} \partial R_{kl}}
    \right]_{\la=0} \times{}\\
    {}\times \eta_{su} \psi_1\nu_0\psi_1^{\mathrm{T}} \eta_{kl}^{\mathrm{T}}+{}\\
{}+\!\sss_{s,k,l=1}^{n_x}\! \!\lk \fr{\partial^2 q_\kappa (\partial /(i\partial
    \la)) g_t (\la,\Theta)}
    {\partial \hat X_s \partial R_{kl}}\rk_{\la=0}
    \!\!\!h_s \psi_1\nu_0\psi_1^{\mathrm{T}}
     \eta_{kl}^{\mathrm{T}}\biggr\}\, dt.\hspace*{-6.3374pt}
%    \label{e3.9-s1}
    \end{multline*}
Подставив сюда выражения~(\ref{e3.6-s1}), (\ref{e3.7-s1}) и~(\ref{e2.7-s1})
дифференциалов $d\hat X_s$, $dR_{sq}$  и~ $dg_t (\la,\Theta)$ и~вспомнив,
что для любого полинома  $P(x)$ $P\lk (\partial /(i\partial
\la)) g_t(\la)\rk_{\la=0}\hm=P(\alp)$, получаем стохастические
дифференциальные уравнения:
\begin{multline*}
dc_\kappa =\biggl\{ F_\kappa +\sss_{s=1}^{n_x}\fr{\partial q_\kappa
    (\alp)}{\partial \hat X_s} \,f_s +{}\\
    {}+\sss_{s,u=1}^{n_x} \fr{\partial
    q_\kappa (\alp)}{\partial R_{su}}\left(
     f_{su}^{(2)} - h_s \psi_1\nu_0\psi_1^{\mathrm{T}} h_u^{\mathrm{T}}\right)+{}\\
{}+\fr{1}{2}\sss_{s,u=1}^{n_x} \fr{\partial^2 q_\kappa (\alp)}
{\partial \hat X_s \partial \hat X_u}  h_s \psi_1\nu_0\psi_1^{\mathrm{T}} 
h_u^{\mathrm{T}} +{}\\
{}+ \fr{1}{2} \sss_{s,u,k,l=1}^{n_x} 
\fr{\partial^2 q_\kappa (\alp)}{\partial R_{su}\partial R_{kl}}\,
    \eta_{su} \psi_1\nu_0\psi_1^{\mathrm{T}} \eta_{kl}^{\mathrm{T}}+{}\\
    {}+\sss_{s,k,l=1}^{n_x} \fr{\partial^2 q_\kappa (\alp)}
{\partial \hat X_s \partial R_{kl}}\, h_s \psi_1\nu_0\psi_1^{\mathrm{T}} \eta_{kl}^{\mathrm{T}}\biggr\}\, dt +{}\\
{}+\biggl\{ H_\kappa +
    \sss_{s=1}^{n_x} \fr{\partial q_\kappa (\alp)}{\partial \hat X_s}\,h_s +
\sss_{s,u=1}^{n_x} \fr{\partial q_\kappa (\alp)}{\partial R_{su}} \,\eta_{su}
\biggr\}\times{}
\end{multline*}

\noindent
\begin{multline}
{}\times \left(dY_t - f^{(1)}\, dt\right)= 
A^{c_\kappa} dt + B^{c_\kappa}\, dY_t\,,\\ 
c_\kappa \left(t_0\right) = c_{\kappa0} \enskip (\lv\kappa\rv= 3\tr N)\,.
\label{e3.10-s1}
\end{multline}
Здесь в~дополнение к~прежним обозначениям принято:
\begin{equation}
\left.
\begin{array}{rl}
F_\kappa &=\displaystyle F_\kappa \left(Y_t,\Theta,\vartheta,t\right) 
={}\\[6pt]
&\displaystyle{}=\sss_{s=1}^{n_x}
   \mm_{\Delta^x}^{p^*} \lk \varphi_s\left (Y_t,X,\Theta,t\right)
   \fr{\partial q_\kappa (X)}{\partial X_s}\rk+{}\\[6pt]
&\displaystyle\hspace*{-5mm}{}+\fr{1}{2} \sss_{s,u=1}^{n_x}
     \mm_{\Delta^x}^{p^*}\lk \si_{su} \left(Y_t,X,\Theta,t\right)\fr{\partial^2 q_\kappa
    (X)}{\partial X_s\partial X_u}\rk\,;\\[6pt]
H_\kappa &= \displaystyle H_\kappa \left(Y_t,\vartheta,\Theta,t\right) ={}\\[6pt]
&\displaystyle{}=
\biggl\{  \mm_{\Delta^x}^{p^*}
   \lk \varphi_1 \left(Y_t,X,\Theta,t\right)^{\mathrm{T}} q_\kappa (X)\rk+{}\\[6pt]
&\hspace*{-9mm}{}\displaystyle+ \sss_{s=1}^{n_x}
     \mm_{\Delta^x}^{p^*}\!\lk \left(\psi\nu_0\psi_1^{\mathrm{T}}\right)_s 
     \left(Y_t,X,\Theta,t\right)\fr{\partial
    q_\kappa (X)}{\partial X_s} \rk- {}\\[6pt]
    &{}-c_\kappa
    f^{(1)T} \biggr\} \pss^{-1} (Y_t,\Theta,t)\,,
    \end{array}\!
    \right\}\!
    \label{e3.11-s1}
    \end{equation}
где через $(\psi\nu_0\psi_1^{\mathrm{T}})_s$ обозначена  $s$-я строка матрицы
$\psi\nu_0\psi_1^{\mathrm{T}}$; $\si\hm=\psi\nu_0\psi_1^{\mathrm{T}} 
\hm=\lf \si_{su}\rf$.

Функции  $f_s$,
$f^{(1)}$, $f^{(2)}_{su}$, $h_s$, $\eta_{su}$, $F_\kappa$ 
и~$H_\kappa$ в~уравнениях~(\ref{e3.6-s1}), (\ref{e3.7-s1}) 
и~(\ref{e3.10-s1}) представляют
собой линейные комбинации величин  $c_\nu$ $(\lv\nu\rv \hm= 3\tr N)$
с~коэффициентами, зависящими от~$\hat X_t$ и~$R_t$. Величины
$\partial q_\kappa (\alp)/\partial \hat X_s$, $\partial q_\kappa
(\alp)/\partial R_{su}$, $\partial^2 q_\kappa (\alp)/(\partial \hat
X_s \partial \hat X_u)$, $\partial^2 q_\kappa (\alp)/(\partial
R_{su} \partial R_{kl})$ и~$\partial^2 q_\kappa (\alp)/(\partial \hat
X_s \partial R_{kl})$ после замены моментов их выражениями
через~$c_\nu$ тоже будут линейными комбинациями величин~$c_\nu$ 
с~коэффициентами, зависящими от~$\hat X_t$ и~$R_t$.

Таким образом, имеем следующие утверждения.

\smallskip

\noindent
\textbf{Теорема~3.1.}\
\textit{Пусть МСтС}~(\ref{e2.3-s1}), (\ref{e2.4-s1})~--- 
\textit{гауссовская $(\psi'' \hm=0)$, выполнены условия существования 
и~единственности решения, а~мат\-ри\-ца $\si_1 \hm= 
\psi_1 \nu_0 \psi_1^{\mathrm{T}}$ не вырождена. Тогда в~основе алгоритма 
ОСОФ по МОР лежат уравнения}~(\ref{e3.1-s1}), 
(\ref{e3.6-s1})--(\ref{e3.10-s1}) \textit{при условии ограниченности 
функций}~(\ref{e3.11-s1}).

\smallskip

\textbf{Теорема~3.2.}\
\textit{Пусть для МСтС}~(\ref{e2.3-s1}), (\ref{e2.4-s1}) 
\textit{выполнены условия существования и~единственности решения, 
а~мат\-ри\-ца $\si_1\hm=\psi_1\nu_0\psi_1^{\mathrm{T}}$ не вырождена. 
Тогда алгоритм ОСОФ согласно МОР задается уравнениями}~(\ref{e3.1-s1}), 
(\ref{e3.6-s1})--(\ref{e3.10-s1}) 
\textit{при условии ограниченности функций $f$, $f^{(1)}$, $\bar f^{(2)}$, 
$h$, $\rho_r$, $F_\kappa$ и~$H_\kappa$, 
определяемых}~(\ref{e3.5-s1}) и~(\ref{e3.11-s1}).


\smallskip

Применяя методы теории чувствительности~\cite{9-s1, 10-s1} 
для приближенного анализа фильтрационных уравнений и~учитывая 
случайность параметров~$\Theta$, придем к~следующим уравнениям 
для функций чувствительности первого порядка:

\noindent
\begin{align*}
d\nabla^\Theta \hat X_s &= \nabla^\Theta A^{\hat X_s} \,dt + \nabla^\Theta B^{\hat X_s}\,dY_t\,,\\ 
&\hspace*{40mm}\nabla^\Theta B^{\hat X_s}(t_0) =0\,;\\[2pt]
d\nabla^\Theta R_{sq} &= \nabla^\Theta A^{R_{sq}}\, dt + \nabla^\Theta B^{R_{sq}}\,dY_t\,, \\
&\hspace*{40mm}\nabla^\Theta R_{sq}(t_0) =0\,;\\[2pt]
d\nabla^\Theta c_{\kappa} &= \nabla^\Theta A^{c_\kappa}\, dt + \nabla^\Theta 
B^{c_\kappa}\,dY_t,\enskip  \nabla^\Theta c_\kappa(t_0) =0.
%\eqno(3.12)
\end{align*}
Здесь процедура взятия производных осущест\-вляется по всем входящим переменным, 
а~коэффициенты чувствительности вычисляются при\linebreak  $\Theta\hm=m^\Theta$. При 
этом предполагается малость дисперсий по сравнению с~их математическими 
ожиданиями. Очевидно, что при дифференцировании по~$\Theta$ 
$(\nabla^\Theta \hm= \prt /\prt\Theta)$
порядок уравнений возрастает пропорционально числу производных. 

Аналогично 
составляются уравнения для элементов матриц вторых функций чувствитель-\linebreak ности.

Для оценки качества ОСОФ, следуя~\cite{1-s1, 2-s1}, при гауссовских~$\Theta$ 
с~математическим ожиданием~$m^\Theta$ и~ковариационной мат\-ри\-цей~$K^\Theta$  
введем условную функцию потерь, допускающую квадратичную аппроксимацию:
    \begin{multline*}
    \rho^{\hat X_s}=\rho^{\hat X_s}(\Theta) =\rho \left(m^\Theta\right) +
    \sss_{ii=1}^{n^\Theta} \rho_i' \left(m^\Theta\right)\Theta_s^0+ {}\\
    {}+
    \sm2\limits_{i,j=1} \rho_{ij}'' \left(m^\Theta\right)\Theta_i^0 
    \Theta_j^0\,,
%    \label{e3.13-s1}
    \end{multline*}
а также показатель~$\eps$:
\begin{equation*}
\eps =\eps_2^{1/4}.
%\label{e3.14-s1}
\end{equation*}
Здесь введены обозначения:
    $$\eps_2 = \mm^N \lk \rho (\Theta)^2\rk -\rho (m^\Theta)^2\,;$$
    
    \vspace*{-9pt}
    
    \noindent
    \begin{multline*}
\mm^N \lk \rho(\Theta)^2\rk = \rho \left(m^\Theta\right)^2 +
\rho' \left(m^\Theta\right)^{\mathrm{T}} K^\Theta \rho'\left(m^\Theta\right)+ {}\\[3pt]
{}+
2\rho \left(m^\Theta\right) \mathrm{tr}\, \lk \rho''\left(m^\Theta\right)K^\Theta\rk+{}\\[3pt]
{}+\lf \mathrm{tr}\, \lk \rho'' \left(m^\Theta\right) K^\Theta\rk \rf^2+2 
\mathrm{tr}\, \lk \rho''\left(m^\Theta\right) K^\Theta\rk^2\,,
\end{multline*}
а функции $\rho'$ и~$\rho''$ по известным формулам~\cite{9-s1, 10-s1} 
определяются на основе первых и~вторых функций чувствительности.

Изложенные выше методы синтеза ОСОФ дают
принципиальную возможность получить фильтр, близкий к~оптимальному по
оценке с~любой сте\-пенью точности.
Чем выше максимальный порядок учитываемых моментов, КОР, тем выше будет точность
приближения к~оптимальной оценке. Однако число уравнений,
определяющих параметры апостериорного одномерного распределения, быст\-ро растет
с~увеличением числа учитываемых па\-ра\-мет\-ров.
Соответствующие оценки можно найти\linebreak в~[4--6].

\section{Эллипсоидальная аппроксимация распределений}

Для конечномерных МСтС часто оказывается полезной
структурная аппроксимация распределений посредством эллипсоидальных
распределений. Следуя~\cite{7-s1, 11-s1},  для структурной аппроксимации
плотностей вероятности случайных векторов будем использовать
плотности, имеющие эллипсоидальную структуру, т.\,е.\ плотности, 
у~которых поверхностями уровней равной вероятности являются подобные
концентрические эллипсоиды (эллипсы\linebreak
 для двумерных векторов,
эллипсоиды для трехмерных векторов, гиперэллипсоиды для векторов\linebreak
размерности больше трех). В~частности, эллипсо\-и\-даль\-ную структуру
имеет нормальное распределение в~любом конечномерном пространстве.
Харак\-терная особенность таких распределений состоит в~том, что их
плот\-ности вероятности являются функциями  положительно определенной квадратичной
формы $u\hm=u(y)\hm=(y^{\mathrm{T}}-m^{\mathrm{T}})C(y\hm-m)$, 
где~$m$~--- математическое ожидание
случайного вектора~$Y$; $C$~--- некоторая положительно определенная матрица.

Для нахождения ЭА плотности вероятности\linebreak
$r$-мер\-но\-го случайного вектора будем пользоваться конечным
отрезком разложения по биортонормальной системе полиномов
$\{p_{r,\nu}(u(y)),q_{r,\nu}(u(y))\}$, которые зависят только от
квадратичной формы $u\hm=u(y)$ и~функцией веса для которых служит
некоторая плотность вероятности эллипсоидальной структуры
$w(u(y))$:
\begin{equation}
\mm_{\Delta^y}^w \lk w(u(Y))p_{r,\nu} (Y)q_{r,\mu}(u(Y))\rk=
    \delta_{\nu\mu}\,.
    \label{e4.1-s1}
    \end{equation}

Индексы $\nu$ и~$\mu$ у~полиномов означают их степень относительно
переменной~$u$. Конкретный вид и~свойства полиномов определены
ниже. Однако без потери общности можно принять, что
$q_{r,0}(u)\hm=p_{r,0}(u)\hm=1$. Тогда плотность вероятности вектора~$Y$
может быть приближенно представлена в~виде:
\begin{equation}
f(y)\approx  f^*(u)=w(u)\sum\limits_{\nu=0}^N
    \crn \prn(u)\,,\label{e4.2-s1}
    \end{equation}
где коэффициенты~$\crn$ определяются по формуле:
\begin{equation}
\crn=\mm_{\Delta^y}^{\mathrm{ЭА}}\left[q_{r,\nu}(U)\right]\enskip 
(\nu=1,\ldots,N)\,.
    \label{e4.3-s1}
    \end{equation}
Учитывая, что $p_{r,0}(u)$ и~$q_{r,0}(u)$~--- взаимно обратные
постоянные (полиномы нулевой степени), то всегда $c_{r,0}p_{r,0}\hm=1$.
Поэтому из формулы~(\ref{e4.2-s1}) следует, что
\begin{equation*}
f(y)\approx f^*(u)=w(u)\left[\,1+\sum\limits_{\nu=2}^N \crn\prn(u)
    \,\right]\,.
    %\label{e4.4-s1}
    \end{equation*}

Для приложений большое значение имеет случай, когда за
распределение $w(u)$ выбирается нормальное (гауссовское) распределение:
\begin{equation*}
w(u)=w(y^{\mathrm{T}}Cy)=\fr{1}{\sqrt{(2\pi)^r\vert
    K\vert}}\exp\left(-y^{\mathrm{T}}K^{-1}\fr{y}{2}\right)\,.
%    \label{e4.5-s1}
    \end{equation*}
Учитывая, что $C\hm=K^{-1}$, приведем условие биортонормальности~(\ref{e4.1-s1}) 
к~виду:
\begin{equation*}
\fr{1}{2^{r/2}\Gamma(r/2)}\int\limits_0^{\infty}
    \prn(u)\qrm(u)u^{r/2-1}e^{-u/2}\,du=\delta_{\nu\mu}\,.
%    \label{e4.6-s1}
    \end{equation*}
Задача выбора системы полиномов $\{\prn(u),\qrm(u)\}$,
используемой при ЭА~(\ref{e4.2-s1})
и~(\ref{e4.3-s1}), сводится к~нахождению биортонормальной системы
полиномов, для которой весом служит $\chi^2$-рас\-пре\-де\-ле\-ние 
с~$r$~степенями свободы, при этом используются
следующие формулы:
\begin{multline*}
p_{r,\nu}(u)=S_{r,\nu}(u)\,,\enskip q_{r,\nu}(u)={}\\
{}=
\fr{(r-2)!!}{(r+2\nu-2)!!(2\nu)!!}\,S_{r,\nu}(u),
    \quad r\ge 2\,;
    %\label{e4.7-s1}
    \end{multline*}
    
    \vspace*{-12pt}
    
    \noindent
\begin{multline*}
    S_{r,\nu}(u)=S_{\nu}^{r/2-1}(u)={}\\
    {}=\sum\limits_{\mu=0}^{\nu}(-1)^{\nu+\mu}
    C_{\nu}^{\mu}\fr{(r+2\nu-2)!!}{(r+2\mu-2)!!}\,u^{\mu}\,.
   % \label{e4.8-s1}
\end{multline*}

При разложении по полиномам $\srn(u)$ плот\-ности вероятности
случайного вектора~$Y$ и~всех его возможных проекций согласованы.

\vspace*{-3pt}

\section{Субоптимальные фильтры на~основе метода эллипсоидальной
аппроксимации (линеаризации)}

Применим МЭА для приближенного решения задачи оптимальной нелинейной 
фильтрации в~дифференциальной гауссовской СтС. Для этого аппроксимируем
апостериорную плотность $p_t(x)$ формулой:
\begin{equation}
p_t(x) = p^* (x;\Theta) = w_1 (u_t) \lk 1+\sss_{\nu=2}^N 
c_\nu p_\nu (u_t)\rk\,,
    \label{e5.1-s1}
    \end{equation}
    где
    $$
    u_t=\left(x^{\mathrm{T}} - \hat X_t^{\mathrm{T}}\right) 
    C_t\left(x-\hat X_t\right)\,;
$$
$C_t$~--- матрица, обратная по отношению к~ковариационной
матрице ошибки фильтрации~$R_t$, $C_t\hm=R_t^{-1}$;

\noindent
\begin{multline}
c_\kkk =\mm_{\Delta^x}^{f_1} \lk q_\kkk (U_t)\rk =
\left[ q_\kkk \left(\fr{\partial^{\mathrm{T}}}{i\partial \la }- 
m_t^{\mathrm{T}}\right)\times{}\right.\\
\left.{}\times C_t \left(\fr{\partial^{\mathrm{T}} }{i\partial \la} - 
m_t\right) g_1(\la;t)\right]_{\la=0}.
\label{e5.2-s1}
\end{multline}
Чтобы найти стохастический дифференциал~$c_\kkk$, применим формулу
Ито~\cite{5-s1}, учитывая, что~$c_\kkk$ представляет собой функцию
трех случайных процессов~$\hx_t$, $R_t$ и~$g_t(\la)$,
стохастические дифференциалы Ито которых определяются 
формулами~(\ref{e3.6-s1}) и~(\ref{e3.7-s1}). В~результате получим уравнение~(\ref{e3.10-s1}), 
которое на основании равенства  $q_\kkk (\alp) \hm = 
\mm_{\Delta^x}^{f_1} q_\kkk (U_t)$ имеет вид:
    \begin{multline}
    dc_\kkk=\biggl\{ F_\kkk
    +\sum\limits_{s=1}^n \mm_{\Delta^x}^{f_1}\lk\fr{\partial q_\kkk (U_t)}
{\partial \hx_s}\rk
    +{}\\
    {}+\sum\limits_{s=1}^n \mm_{\Delta^x}^{f_1}\lk 
    \fr{\partial q_\kkk (U_t)}{\partial R_{su}}\left(
    f_{su}^{(2)} - h_s \psi_1\nu\psi_1^{\mathrm{T}} h_u^{\mathrm{T}}\right)\rk+{}\\
{}+\fr{1}{2}\sss_{s,u=1}^n \mm_{\Delta^x}^{f_1}\lk
\fr{\partial^2 q_\kkk (U_t)}{\partial \hx_s \partial \hx_u}\,h_s
    \psi_1\nu\psi_1^{\mathrm{T}}  h_u^{\mathrm{T}}\rk
    +{}\\
    {}+\fr{1}{2}\sss_{s,u,k,l=1}^n \mm_{\Delta^x}^{f_1}\lk 
    \fr{\partial^2 q_\kkk (U_t)}{\partial R_{su}\partial R_{kl}}
\,\eta_{su}  \psi_1\nu\psi_1^{\mathrm{T}}  \eta_{kl}^{\mathrm{T}} \rk+{}\\
{}+\sss_{s,k,l=1}^n \mm_{\Delta^x}^{f_1}\lk 
\fr{\partial^2 q_\kkk (U_t)}{ \partial \hx_s \partial R_{kl}}\, h_s
    \psi_1\nu\psi_1^{\mathrm{T}}  \eta_{kl}^{\mathrm{T}}\rk \biggr\} \,dt+{}\\
{}+ \left\{ H_\kkk +\sss_{s=1}^n \mm_{\Delta^x}^{f_1}\lk
\fr{\partial q_\kkk (U_t)}{\partial \hx_s }\,h_s\rk
    +{}\right.\\
   \left. {}+\sss_{s,u=1}^n \mm_{\Delta^x}^{f_1}\lk 
    \fr{\partial q_\kkk (U_t)}{\partial R_{su} }\,\eta_{su}\rk\right\} 
    \left(dY_t - f^{(1)} \,dt\right).
    \label{e5.3-s1}
    \end{multline}

Вычислим входящие в~(\ref{e5.3-s1})
математические ожидания производных полинома $q_\kkk(U_t)$. Имеем:
\begin{multline}
\mm_{\Delta^x}^{f_1} \lk \fr{\partial q_\kkk (U_t)}{\partial \hx_s }\rk= 
\mm_{\Delta^x}^{f_1}\lk
    q_\kkk'\fr{\partial U}{\partial \hx_s } \rk= {}\\
    \!\!{}=
    \mm_{\Delta^y}^{f_1}\lk q_\kkk' (U_t) \left( - 2
    \sss_{j=1}^n  c_{sj} (X_j - \hx_j)\right)\rk =0;\!\label{e5.4-s1}
    \end{multline}
    
    \vspace*{-12pt}
    
    \noindent
    \begin{multline}
\mm_{\Delta^x}^{f_1} \lk\fr{\partial^2 q_\kkk (U_t)}{\partial \hx_s \partial
    \hx_u}\rk={}\\
{}  =\mm_{\Delta^x}^{f_1} \left[  4 q_\kkk'' (U_t) \left( 
\sss_{j=1}^n  c_{uj} \left(X_j - \hx_j\right)\right)\times{}\right.\\
\left.{}\times \left( \sss_{j=1}^n  c_{sj}
    \left(X_j - \hx_j\right)\right)+ 2 q_\kkk' \left(U_t\right) 
    c_{su}\right]\,;
    \label{e5.5-s1}
    \end{multline}
%     \vspace*{-12pt}
    
    \noindent
    \begin{multline*}
\mm_{\Delta^y}^{f_1}\lk q_\kkk''\left(U_t\right) C_t\left(X_t-\hx_t\right) 
\left(X_t^{\mathrm{T}} - \hx_t^{\mathrm{T}}\right) C_t\rk ={}\\
{}= \left\{ \vphantom{\sss_{\nu=2}^N}
\fr{a}{n}\,\mm_{\Delta^U}^{w_1} \lk q_\kkk''(U) 
U^{n/2}\rk + {}\right.\\
\left.{}+\sss_{\nu=2}^N c_\nu \fr{a}{n}
\,\mm_{\Delta^U}^{w_1 p_\nu}\lk q_\kkk''(U) U^{n/2}\rk\right\} C_t\,,
%\label{e5.6-s1}
\end{multline*}
где нормирующий множитель~$a$ определяется соотношением:
     $$
     a^{-1} =\mm_{\Delta^U}^{w_1 p_\nu}\lk  U^{n/2-1}\rk\,.
     %\eqno(5.7)
     $$
Введя обозначения
\begin{align*}
    \xi_{\kkk 0} &= \fr{a}{n}\,\mm_{\Delta^U}^{w_1 }\lk 
    q_\kkk''(U) U^{n/2}\rk\,;\\[2pt]
    \xi_{\kkk\nu} &=
 \fr{a}{n}\,\mm_{\Delta^U}^{w_1 p_\nu}\lk q_\kkk''(U) U^{n/2}\rk
%\eqno(5.8)
 \end{align*}
и заметив, что вследствие ортогональности  $p_\nu (u)$ ко всем
функциям~$u^\la$ при  $\la\hm<\nu'$ величина~$\xi_{\kkk \nu}$
обращается в~нуль при  $\nu\hm>\kkk-1$, получим:
\begin{multline}
\mm_{\Delta^U}^{w_1 }\lk q_\kkk'' \left(U_t\right) C_t\left(X_t-\hx_t\right) 
\left(X_t^{\mathrm{T}}-\hx_t^{\mathrm{T}}\right) C_t \rk={}\\[1pt]
{}= \left( \xi_{\kkk 0}
    +\sss_{\nu=2}^{\kkk-1} c_\nu \xi_{\kkk \nu}\right) C_t.
    \label{e5.9-s1}
    \end{multline}
На основании~(\ref{e5.9-s1}) имеем:
\begin{multline}
4\mm_{\Delta^U}^{w_1 }\left[ q_\kkk'' \left(U_t\right) 
\left( \sss_{j=1}^n c_{uj} \left(X_j - \hx_j\right)\right)\times{}\right.\\[1pt]
\left.{}\times
    \left( \sss_{j=1}^n c_{sj} (X_j - \hx_j)\right)\right]= {}\\[1pt]
    {}=
    4 \left( \xi_{\kkk 0} +\sss_{\nu=2}^{\kkk-1} c_\nu \xi_{\kkk \nu}\right) c_{su}\,.
\label{e5.10-s1}
\end{multline}
Математическое ожидание во втором слагаемом в~(\ref{e5.5-s1}) определяется
формулой:
\begin{multline*}
\mm_{\Delta^U}^{w_1 }\lk q_\kkk' \left(U_t\right) \rk = 
\mm_{\Delta^U}^{w_1 }\lk q_\kkk' (U) U^{n/2-1}\rk +{}\\[1pt]
{}+
\sss_{\nu=2}^N c_\nu \mm_{\Delta^U}^{w_1 p_\nu}\lk q_\kkk'(U)\rk.
%\label{e5.11-s1}
\end{multline*}
Введя обозначения
\begin{align*}
\zeta_{\kkk 0}' &= a \mm_{\Delta^U}^{w_1}\lk q_\kkk' (U) U^{n/2-1}\rk\,;\\[6pt]
\zeta_{\kkk \nu}' &= a \mm_{\Delta^U}^{w_1 p_\nu}\lk q_\kkk' (U) U^{n/2-1}\rk
%\label{e5.12-s1}
\end{align*}
и заметив, что вследствие ортогональности  $p_\nu (u)$ ко всем
функциям~$u^\la$ при  $\la\hm<\nu$ величина  $\zeta_{\kkk\nu}$
обращается в~нуль при  $\nu\hm> \kkk\hm-1$, получим:
\begin{equation}
\mm_{\Delta^U}^{w_1 } \lk q_\kkk' (U_t)\rk =
\zeta_{\kkk 0}' +\sss_{\nu=2}^{\kkk-1} c_\nu \zeta_{\kkk \nu}'\,.
\label{e5.13-s1}
\end{equation}
На основании~(\ref{e5.10-s1}) и~(\ref{e5.13-s1}) формула~(\ref{e5.5-s1}) принимает вид:
\begin{equation*}
\mm_{\Delta^U}^{w_1 }\lk \fr{\partial^2 q_\kkk (U_t)}
{\partial \hx_s \partial \hx_u }\rk =2 \left(\zeta_{\kkk 0} +
    \sss_{\nu=2}^{\kkk-1}  c_{\nu}\zeta_{\kkk \nu}\right) c_{su} \,,
%    \label{e5.14-s1}
    \end{equation*}
где для краткости положено:
    $$
    \zeta_{\kkk 0} = 2 \xi_{\kkk 0} + \zeta_{\kkk 0}'\,;\quad
     \zeta_{\kkk \nu} = 2 \xi_{\kkk \nu} + \zeta_{\kkk \nu}'\,.
    $$

Вычислив производные  $\partial q_\kkk (U_t)/ \partial R_{su}$,
получим:
\begin{align*}
\mm_{\Delta^U}^{w_1 }\lk \fr{\partial q_\kkk (U_t)}{\partial R_{ss} }\rk &=
- c_{ss} \left( \gamma_{\kkk 0}
    +\sss_{\nu=2}^\kkk c_\nu \gamma_{\kkk\nu}\right)\,;%\label{e5.15-s1}
    \\
\mm_{\Delta^U}^{w_1 }\lk \fr{\partial q_\kkk (U_t)}{\partial R_{su} }\rk&=
- 2c_{su} \left( \gamma_{\kkk 0}
    +\sss_{\nu=2}^\kkk c_\nu \gamma_{\kkk\nu}\right)\,,\\ 
    &\hspace*{40mm}    s\ne u\,,
%    \label{e5.16-s1}
    \end{align*}
где $\gamma_{\kkk 0}$, $\gamma_{\kkk 2}\tr \gamma_{\kkk\kkk}$
определяются формулами:
\begin{align*}
\gamma_{\kkk 0} &= \fr{a}{n}\,\mm_{\Delta^U}^{w_1 }\lk q_\kkk'(U) U^{n/2}\rk\,;%\label{e5.17-s1}
\\
\gamma_{\kkk \nu} &= \fr{a}{n}\,\mm_{\Delta^U}^{w_1 p_\nu} \lk q_\kkk'(U) 
U^{n/2}\rk. %\label{e5.18-s1}
\end{align*}

Дифференцируя эти формулы по компонентам вектора  $\hx_t$ и
элементам матрицы $R_t$ и~имея в~виду, что
    \begin{equation}
\fr{\partial c_{ij}}{\partial R_{rr} } =- c_{ri} c_{rj};\enskip
\fr{\partial c_{ij}}{\partial R_{rs} } =- 
\left(c_{ri} c_{sj}+ c_{si} c_{rj}\right),\label{e5.19-s1}
\end{equation}
получаем:
\begin{align}
\mm_{\Delta^U}^{w_1 }\lk \fr{\partial^2 q_\kkk (U_t)}
{\partial \hx_k\partial R_{su} }\rk&=0 \enskip (k,s,u=1\tr n)\,;\label{e5.20-s1}\\
\mm_{\Delta^U}^{w_1 }\lk  \fr{\partial^2 q_\kkk (U_t)}
{\partial R_{ss}\partial R_{rr} }\rk &= c_{rs}^2
    \left( \gamma_{\kkk 0} +\sss_{\nu=2}^\kkk c_\nu \gamma_{\kkk \nu}\right)\,;
    \label{e5.21-s1}\\
\mm_{\Delta^U}^{w_1 }\lk\fr{\partial^2 q_\kkk (U_t)}
{\partial R_{ss}\partial R_{kl} }\rk&=
    2  c_{ls} c_{ks}\left( \gamma_{\kkk 0} +\sss_{\nu=2}^\kkk c_\nu 
    \gamma_{\kkk \nu}\right),\notag\\
    &\hspace*{30mm}k\ne l\,;\label{e5.22-s1}\\
\mm_{\Delta^U}^{w_1 }\lk \fr{\partial q_\kkk (U_t)}
{\partial R_{su}\partial R_{kl} }\rk &={}\notag\\
&\hspace*{-20mm}{}=
    2\left(c_{ks}c_{lu} + c_{ls} c_{ku}\right)     
    \left( \gamma_{\kkk 0} +\sss_{\nu=2}^\kkk c_\nu 
    \gamma_{\kkk \nu}\right)\,,\notag\\ 
    &\hspace*{20mm}s\ne u\,,\enskip k\ne l\,.\label{e5.23-s1}
    \end{align}

Подставив выражения~(\ref{e5.4-s1}), (\ref{e5.19-s1})--(\ref{e5.23-s1}) 
в~уравнение для стохастического дифференциала величины~$c_\kkk$, 
приведем его к~виду:
\begin{multline}
dc_\kkk = \left\{ 
\vphantom{\fr{1}{2} 
\left( \gamma_{\kkk 0} +\sss_{\nu=2}^\kkk c_\nu \gamma_{\kkk \nu}
\right) \sss_{s,u,k,l=1}^n A_{sukl} \eta_{su} \psi_1\nu\psi_1^{\mathrm{T}} 
\eta_{kl}^{\mathrm{T}}}
F_\kkk - \left( \gamma_{\kkk 0} +
\sss_{\nu=2}^\kkk c_\nu \gamma_{\kkk \nu}\right)\times{}\right.\\
{}\times    \,\mathrm{tr}\, \lk \bar C_t \left( f^{(2)} - 
    h \psi_1\nu\psi_1^{\mathrm{T}} h^{\mathrm{T}}\right)\rk+{}\\
{}+  \left( \zeta_{\kkk 0} +\sss_{\nu=2}^{\kkk-1} c_\nu \zeta_{\kkk \nu}\right)
    \,\mathrm{tr}\, \lk C_t  h \psi_1\nu\psi_1^{\mathrm{T}} h^{\mathrm{T}}\rk+{}\\
\left.{}+\fr{1}{2} \!
\left( \!\gamma_{\kkk 0} +\sss_{\nu=2}^\kkk c_\nu \gamma_{\kkk \nu}\!
\right) \!\!\sss_{s,u,k,l=1}^n \!\!\!\!\!\!A_{sukl} \eta_{su} \psi_1\nu\psi_1^{\mathrm{T}} 
\eta_{kl}^{\mathrm{T}}\!
\right\} dt+{}\\
{}+\left\{ H_\kkk -\left( \gamma_{\kkk 0} +\sss_{\nu=2}^n c_\nu \gamma_{\kkk \nu}
\right) \sss_{s,u,=1}^n \bar c_{su} \eta_{su} \right\}\times{}\\
{}\times 
\left(dY_t - f^{(1)} dt\right)\enskip
(\kkk = 2\tr N),
\label{e5.24-s1}
\end{multline}
где
\begin{gather*}
A_{ssrr} = c_{sr}^2\,,\enskip A_{sskl} = 2c_{ks}c_{ls}\,,\enskip k\ne l\,;\\
A_{sukl} = 2(c_{ks}c_{lu} + c_{ls} c_{ku})\,,\enskip s\ne u\,,\ 
k\ne l\,;
%\label{e5.25-s1}
\end{gather*}
$\bar c_{su}$~--- элементы матрицы~$\bar C_t$, определяемой
формулой:
\begin{equation*}
\mm_{\Delta^U}^{w_1 }\lk q_\kkk^K \left(U_t\right) \rk =
-  \left( \gamma_{\kkk 0} +\sss_{\nu=2}^\kkk c_\nu \gamma_{\kkk \nu}\right)C_t.
%\label{e5.26-s1}
\end{equation*}

Формулы~(\ref{e3.5-s1}) для~$f$, $f^{(1)}$, $f^{(2)}$, 
$h$ и~$\rho_r$ при аппроксимации~(\ref{e5.1-s1}) апостериорной плот\-ности
принимают вид:
\begin{multline}
f= f(Y_t,\Theta,t) = \mm_{\Delta^x}^{w_1} \lk 
\varphi\left(Y_t, X,\Theta, t\right)\rk+ {}\\
{}+
\sss_{\nu=2}^N c_\nu \mm_{\Delta^x}^{w_1 p_\nu}\lk 
\vrp\left(Y_t, X,\Theta,t\right)\rk\,;
\label{e5.27-s1}
\end{multline}

\vspace*{-12pt}

\noindent
\begin{multline}
f^{(1)}= f^{(1)}\left(Y_t,\Theta,t\right) = 
\mm_{\Delta^x}^{w_1}\lk \varphi_1\left(Y_t, X, \Theta,t\right)\rk + {}\\
{}+
\sss_{\nu=2}^N c_\nu \mm_{\Delta^x}^{w_1p_\nu }\lk \vrp_1  
\left(Y_t, X,\Theta,t\right)\rk\,;
\label{e5.28-s1}
\end{multline}

\vspace*{-12pt}

\noindent
\begin{multline*}
f^{(2)}= f^{(2)}\left(Y_t,\Theta,t\right) ={}\\
{}= \mm_{\Delta^x}^{w_1}\left[ \left(X-\hx\right)\varphi
\left(Y_t, X,\Theta, t\right)^{\mathrm{T}}+{}\right.\\
{}+
    \varphi\left(Y_t,X,\Theta,t\right) \left(X^{\mathrm{T}}-\hx_t^{\mathrm{T}}
    \right)+{}\\
      \left.{}+
    \left(\psi\nu\psi^{\mathrm{T}}\right) \left(Y_t,X,\Theta,t\right)
    \vphantom{\left(Y_t, X,\Theta, t\right)^{\mathrm{T}}}
    \right]+{}\\
{}+\sss_{\nu=2}^N c_\nu \mm_{\Delta^x}^{w_1p_\nu}\left[ 
\vphantom{\left(X-\hx\right)^{\mathrm{T}}}
\left(X-\hx\right)\vrp\left(Y_t, X,\Theta,t\right)^{\mathrm{T}}+ {}\right.
    \end{multline*}
    
    \noindent
\begin{multline}
  {}+
\vrp\left(Y_t, X,\Theta, t\right) \left(X-\hx\right)^{\mathrm{T}} + {}\\
\left.{}+
\left(\psi\nu\psi^{\mathrm{T}}\right) \left(Y_t, X,\Theta, t\right) 
\vphantom{\left(X-\hx\right)^{\mathrm{T}}}
\right]\,;
\label{e5.29-s1}
\end{multline}

\vspace*{-12pt}

\noindent
\begin{multline}
h= h(Y_t,\Theta,t) ={}\\
{}=\mm_{\Delta^x}^{w_1}\left[ X\varphi_1
\left(Y_t, X,\Theta, t\right)^{\mathrm{T}} + 
\left(\psi\nu\psi^{\mathrm{T}}\right)\left(Y_t,X,\Theta,t\right)\right]+{}\\
{}+\sss_{\nu=2}^N c_\nu \mm_{\Delta^x}^{w_1p_\nu} \left[ 
X\vrp\left(Y_t, X,\Theta, t\right)^{\mathrm{T}} + {}\right.\\
\left.{}+
\left(\psi\nu\psi^{\mathrm{T}}\right) \left(Y_t, X, \Theta,t\right)
\vphantom{\left(Y_t, X,\Theta, t\right)^{\mathrm{T}}}
\right]
    - {}\\
    {}-\hx_t {f^{(1)}}^{\mathrm{T}} \left(\psi_1\nu\psi_1^{\mathrm{T}}\right)^{-1} 
    \left(Y_t,X,\Theta,t\right)\,;
    \label{e5.30-s1}
    \end{multline}

\vspace*{-12pt}

\noindent
\begin{multline}
\rho_r = \rho_r \left(Y_t,\Theta, t\right) ={}\\
{}= \mm_{\Delta^x}^{w_1} \left[ \left(X- \hx_t\right) 
\left(X^{\mathrm{T}}-\hx_t^{\mathrm{T}}\right) a_r \left(Y_t,X,\Theta,t\right) +{}\right.\\
{}+ \left(X-\hx_t\right) b_r\left(Y_t,X,\Theta,t\right)^{\mathrm{T}} + {}\\
\left.{}+
b_r \left(Y_t,X,\Theta,t\right) \left(X^{\mathrm{T}}-\hx_t^{\mathrm{T}}\right)
 \right] +{}\\
 {}+\sss_{\nu=2}^N c_\nu \mm_{\Delta^x}^{w_1p_\nu}\left[ 
 \left(X- \hx_t\right) \left(X^{\mathrm{T}}-\hx_t^{\mathrm{T}}\right) \times{}\right.\\
{}\times a_r \left(Y_t,X,\Theta,t\right) +{}\\
\left.{}+ \left(X-\hx_t\right) b_r\left(Y_t,X,\Theta,t\right)^{\mathrm{T}} + {}\right.\\
\left.{}+
b_r \left(Y_t,X,\Theta,t\right) \left(X^{\mathrm{T}}-\hx_t^{\mathrm{T}}\right) 
\right].
\label{e5.31-s1}
\end{multline}

Формулы~(\ref{e3.11-s1}) для~$F_\kkk$ и~$H_\kkk$ преобразуются к~виду:
\begin{multline}
F_\kkk= F_\kkk\left(Y_t,\Theta,t\right) = {}\\
{}=
\mm_{\Delta^x}^{w_1} \big[ q_\kkk' (U) 2\varphi\left(Y_t, X,\Theta, t\right)^{\mathrm{T}} 
C_t(X-\hx_t)+{}\\
{}+ \mathrm{tr}\, \lk C_t\si \left(Y_t,X,\Theta,t\right)\rk \big]  + {}\\
{}+
\sss_{\nu=2}^N c_\nu \mm_{\Delta^x}^{w_1p_\nu} \bigg[ 
2\varphi\left(Y_t, X,\Theta, t\right)^{\mathrm{T}} C_t\left(X-\hx_t\right)+{}\\
{}+ \mathrm{tr}\, \lk C_t\si \left(Y_t,X,\Theta,t\right)\rk\bigg]\,;
\label{e5.32-s1}
\end{multline}

\vspace*{-12pt}

\noindent
\begin{multline}
H_\kkk= H_\kkk\left(Y_t,\Theta,t\right) ={}\\
{}=
\left\{ \mm_{\Delta^x}^{w_1} \lk\varphi_1\left(Y_t, X, \Theta,t\right)^{\mathrm{T}} 
q_\kkk (U) \rk +{}\right.\\
{}+\sss_{\nu=2}^N c_\nu \mm_{\Delta^x}^{w_1p_\nu}\lk
\varphi_1\left(Y_t, X, t\right)^{\mathrm{T}} q_\kkk (U)\rk  +{}\\
{}+2\mm_{\Delta^x}^{w_1}\lk q'(U) \!\left(X^{\mathrm{T}}-\hx_t^{\mathrm{T}}\right) 
C_t \left(\psi\nu\psi_1^{\mathrm{T}}\right) \!\left(Y_t,X,\Theta,t\right) \rk +{}\\
{}+2\sss_{\nu=2}^N c_\nu \mm_{\Delta^x}^{w_1p_\nu}\times{}\\
{}\times \lk q' (U) 
\left(X^{\mathrm{T}}-\hx^{\mathrm{T}}\right) 
C_t \left(\psi\nu\psi_1^{\mathrm{T}}\right) \left(Y_t,X,\Theta,t\right)\rk-{}\\
\left. {}- {f^{(1)}}^{\mathrm{T}} c_\kkk\right\} 
\left(\psi_1\nu\psi_1^{\mathrm{T}}\right)^{-1} \left(Y_t,\Theta,t\right).
\label{e5.33-s1}
\end{multline}

Таким образом, получен следующий результат.

\smallskip

\noindent
\textbf{Теорема~5.1.}\ \textit{Пусть уравнения
дифференциальной гауссовской СтС}~(\ref{e2.3-s1}), (\ref{e2.4-s1}) 
\textit{допускают применение МЭА.
 Тогда фильтрационные уравнения ЭСОФ 
 имеют  вид}~(\ref{e5.2-s1}), (\ref{e5.3-s1}), (\ref{e5.24-s1})
\textit{при условиях}~(\ref{e5.27-s1})--(\ref{e5.33-s1}).

\smallskip

Аналогично выводятся уравнения ЭСОФ для дифференциальной СтС 
с~винеровскими и~пуассоновскими шумами вида~(\ref{e2.3-s1}) и~(\ref{e2.4-s1}).

Далее, следуя~\cite{7-s1, 6-s1}, построим квазилинейный СОФ на основе МЭЛ для
МСтС~(\ref{e2.1-s1}), (\ref{e2.2-s1}) при $\psi'\hm=\psi'(\Theta,t)$,
$\psi''\hm=\psi''(\Theta,t,v)$, $\psi_1'\hm=\psi_1'(\Theta,t)$
и~$\psi_1''\hm=\psi_1''(\Theta,t,v)$ (т.\,е.\ с~аддитивными винеровскими 
и~пуассоновскими шумами). Уравнения НСОФ проще получаются, если
нелинейные функции~$\vrp$ и~$\vrp_1$ на основе эллипсоидального
распределения с~известным~$c_\nu$ заменить на статистически
линеаризованные~[6--8]:
\begin{equation}
\left.
\begin{array}{rl}
\vrp &=\vrp \left( X_t, Y_t, \Theta, t\right) \approx{}\\[6pt]
&\hspace*{-2mm}{}\approx \vrp_0^{\mathrm{э}} + 
k_x^{\mathrm{э}\vrp} \left(X_t - m_t^x\right) + k_y^{\mathrm{э}\vrp} 
\left(Y_t - m_t^y\right)\,;\\[6pt]
\vrp_1 &= \vrp_1\left( X_t, Y_t, \Theta, t\right) \approx{}\\[6pt]
&\hspace*{-5mm}{}\approx \vrp_{10}^{\mathrm{э}} 
 + k_x^{\mathrm{э}\vrp_1}  \left(X_t - m_t^x\right) + 
 k_y^{\mathrm{э}\vrp_1} \left(Y_t - m_t^y\right)\,,
 \end{array}
 \right\}
 \label{e5.34-s1}
 \end{equation}
а затем использовать уравнения линейной фильтрации~[6--8]. 
Входящие в~(\ref{e5.34-s1}) коэффициенты с~МЭЛ зависят от математических 
ожиданий, дисперсий и~ковариаций:
    $$
    Z_t = \begin{bmatrix}
    X_t\\  Y_t\end{bmatrix}\,; 
    \enskip m_t^z =\begin{bmatrix}
     m_t^x\\ m_t^y\end{bmatrix}\,;
     \enskip K_t^z=\begin{bmatrix}
      K_t^x&K_t^{xy}\\[6pt]
      K_t^{xy}&K_t^y
      \end{bmatrix}\,.
      $$
Они определяются из уравнений:
\begin{alignat*}{2}
\dot Z_t &= A^z Z_t + A_0^z + B_0^z V \,,&\enskip V&= \dot W\,; %\eqno(5.35)
\\
\dot m_t^z &= A^z m_t^z + A_0^z \,,&\enskip m_{t_0}^Z &= m_0^z\,;%\eqno(5.36)$$
\\
\dot K_t^z &= B^z K_t^z + K_t^z (B^z)^{\mathrm{T}} + B_0^z \nu^m (B_0^z)^{\mathrm{T}},&\enskip K_{t_0}^z &= K_0^z\,. %\eqno(5.37)$$
\end{alignat*}
Здесь введены следующие обозначения:
    $$
    A_0^z = \begin{bmatrix}
     a_0\\ b_0\end{bmatrix}\,;\enskip 
     A^z =\begin{bmatrix}
     a_1&a\\ 
     b_1&b\end{bmatrix}\,;\enskip 
     B_0^z =\begin{bmatrix}
      \bar \psi\\ \bar\psi_1\end{bmatrix}\,;
      $$
    $$
    a= k_y^{\mathrm{э}\vrp}\,;\enskip 
    a_1 = k_x^{\mathrm{э}\vrp} \,;\enskip 
    a_0 =\vrp_0^{\mathrm{э}}  - k_x^{\mathrm{э}\vrp}  
    m_t^x - k_y^{\mathrm{э}\vrp}  m_t^y\,;
    $$
    \begin{equation*}
    b= k_y^{\mathrm{э}\vrp_1} \,; \enskip 
    b_1=k_x^{\mathrm{э}\vrp_1} \,;\enskip 
    b_0=\vrp_0^{\mathrm{э}}  -k_x^{\mathrm{э}\vrp_1}  
    m_t^x -k_y^{\mathrm{э}\vrp_1} m_t^y\,;
%    \label{e5.38-s1}
    \end{equation*}
    \begin{equation}
    \left.
    \begin{array}{rl}
\psi \,dW_0 + \displaystyle\iii_{R_0^q} \psi'' P^0 (dt, dv) &=\bar \psi\,  dW\,;\\[6pt]
\psi_1'\, dW_0 + \displaystyle\iii_{R_0^q} \psi_1'' P^0 (dt, dv)&= \bar \psi_1 \,dW\,,
\end{array}
\right\}
\label{e5.39-s1}
\end{equation}
где $\nu^W$~--- интенсивность СтП с~независимыми приращениями, состоящего 
из винеровской и~пуассоновской частей~(\ref{e5.39-s1}). 
Тогда уравнения квазилинейного НСОФ будут иметь вид:
\begin{multline}
\dot{\hat X}_t = a Y_t + a_1\hat X_t + a_0 +{}\\
{}+ \beta_t \lk Z_t - 
\left(bY_t + b_1 \hat X_t + b_0\right)\rk\,;
\label{e5.40-s1}
\end{multline}

\vspace*{-6pt}

\noindent
\begin{equation}
\beta_t = \left(R_t b_1^{\mathrm{T}} + \bar\psi \nu^W \bar\psi_1^{\mathrm{T}}\right) 
\left(\bar\psi_1\nu^W\bar\psi_1^{\mathrm{T}}\right)^{-1}\,;
\label{e5.41-s1}
\end{equation}

\vspace*{-12pt}

\noindent
\begin{multline}
\dot R_t = a_1 R_t + R_t a_1^{\mathrm{T}} + \bar\psi \nu^W \bar\psi^{\mathrm{T}}
- {}\\
{}-\left(R_t b_1^{\mathrm{T}} +\bar\psi \nu^W\bar\psi_1^{\mathrm{T}}\right)
\left(\bar\psi_1 \nu^W\bar\psi_1^{\mathrm{T}}\right)^{-1} \times{}\\
{}\times
\left(b_1 R_t + \bar\psi_1 \nu^W\bar\psi^{\mathrm{T}}\right)\,.
\label{e5.42-s1}
\end{multline}


\noindent
\textbf{Теорема~5.2.}\ \textit{Пусть МСтС}~(\ref{e2.1-s1}), (\ref{e2.2-s1}) 
\textit{содержит только аддитивные винеровские и~пуассоновские шумы и~допускает 
замену статистически линеаризованной, а~матрица  $\si_1 \hm=\bar\psi_1 
\nu^W \bar\psi_1^{\mathrm{T}}$ не вырождена. Тогда в~основе алгоритма 
квазилинейного НСОФ лежат уравнения}~(\ref{e5.40-s1})--(\ref{e5.42-s1}) 
\textit{при соответствующих начальных условиях.}

\smallskip

Аналогично разд.~3 (в~условиях теорем~5.1 и~5.2) составляются уравнения 
для оценки точности и~чувствительности к~параметрам~$\Theta$.

\section{Заключение}

Разработана теория аналитического синтеза ЭСОФ для нелинейных дифференциальных 
МСтС). Рассмотрены случаи гауссовских 
и~негауссовских СтС. Алгоритмы ЭСОФ положены в~основу модуля экспериментального 
программного обеспечения  StS-Filter (version 2016).

Результаты допускают развитие на случай дискретных СтС.

Теоретический и~практический интерес представляет теория  ЭСОФ 
на основе ненормированных апостериорных распределений~\cite{10-s1}.


{\small\frenchspacing
 {%\baselineskip=10.8pt
 \addcontentsline{toc}{section}{References}
 \begin{thebibliography}{99}

\bibitem{1-s1}
\Au{Синицын И.\,Н.}
Аналитическое моделирование распределений на основе ортогональных
разложений в~нелинейных стохастических системах на многообразиях~//
Информатика и~её применения, 2015. Т.~9. Вып.~3. C.~17--24.


\bibitem{2-s1}
\Au{Синицын И.\,Н.}
Применение ортогональных разложений для аналитического моделирования
многомерных распределений в~нелинейных стохастических системах на
многообразиях~// Системы и~средства информатики, 2015. Т.~25. №\,3.
С.~3--22.

\bibitem{3-s1}
\Au{Синицын И.\,Н.}
Ортогональные субоптимальные фильтры для нелинейных стохастических
систем на многообразиях~// Информатика и~её применения, 2016. Т.~10.
Вып.~1. С.~34--44.

\bibitem{7-s1} %4
\Au{Ватанабэ С., Икэда Н.} Стохастические дифференциальные уравнения 
и~диффузионные процессы~/ Пер. с~англ.~--- М.: Наука, 1986. 448~с.
(\Au{Watanabe~S, Ikeda~N.} 
Stochastic differential equations and diffusion processes.~--- 
Amsterdam\,--\,Oxford\,--\,New York: North-Holland Publishing Co.; 
Tokyo: Kodansha Ltd., 1981. 476~p.)

\bibitem{6-s1} %5
Справочник по теории вероятностей и~математической статистике~/ Под
ред. В.\,С.~Королюка, Н.\,И.~Портенко, А.\,В.~Скорохода, А.\,Ф.~Турбина.~--- 
М.: Наука, 1985. 640~с.


\bibitem{4-s1} %6
\Au{Пугачев В.\,С., Синицын И.\,Н.}
 \Au{Пугачёв В.\,С., Синицын~И.\,Н.}
Стохастические дифференциальные системы. Анализ и~фильтрация.~--- М.:
Наука,  1990.  632~с. 

\bibitem{5-s1} %7
\Au{Пугачёв В.\,С., Синицын И.\,Н.}
Теория стохастических систем.~--- М.: Логос, 2000; 2004. 1000~с.
%[Англ. пер. Stochastic Systems. Theory and  Applications. --
%Singapore: World Scientific, 2001. 908~p.].


\bibitem{8-s1} %8
\Au{Синицын И.\,Н.}
Фильтры Калмана и~Пугачева.~--- 2-е изд.~--- М.: Логос, 2007. 776~с.

\bibitem{9-s1}
\Au{Wonham W.\,M.}
Some applications of stochastic differential equations to optimal
nonlinear filtering~// J.~Soc. Ind. Appl. Math. A, 1964.
Vol.~2. No.\,3. P.~347--369.

\bibitem{10-s1}
\Au{Zakai M.}
On the optimal filtering of diffusion processes~// Z.
Wahrscheinlichkeit., 1969. Bd.~11. S.~230--243.


\bibitem{11-s1}
\Au{Синицын И.\,Н., Синицын~В.\,И.}
Лекции по теории нормальной и~эллипсоидальной аппроксимации
распределений в~стохастических системах.~--- М.: ТОРУС ПРЕСС, 2013.
488~с.
\end{thebibliography}

 }
 }

\end{multicols}

\vspace*{-3pt}

\hfill{\small\textit{Поступила в~редакцию 29.02.16}}

\vspace*{8pt}

%\newpage

%\vspace*{-24pt}

\hrule

\vspace*{2pt}

\hrule

%\vspace*{8pt}



\def\tit{ELLIPSOIDAL SUBOPTIMAL
FILTERS FOR~NONLINEAR STOCHASTIC
SYSTEMS ON~MANIFOLDS}

\def\titkol{Ellipsoidal suboptimal
filters for~nonlinear stochastic
systems on~manifolds}

\def\aut{I.\,N.~Sinitsyn, V.\,I.~Sinitsyn, and E.\,R.~Korepanov}

\def\autkol{I.\,N.~Sinitsyn, V.\,I.~Sinitsyn, and E.\,R.~Korepanov}

\titel{\tit}{\aut}{\autkol}{\titkol}

\vspace*{-9pt}

\noindent
Institute of Informatics Problems, Federal Research Center 
``Computer Science and Control'' of the Russian Academy of Sciences,
44-2~Vavilov Str., Moscow 119333, Russian Federation

\def\leftfootline{\small{\textbf{\thepage}
\hfill INFORMATIKA I EE PRIMENENIYA~--- INFORMATICS AND
APPLICATIONS\ \ \ 2016\ \ \ volume~10\ \ \ issue\ 2}
}%
 \def\rightfootline{\small{INFORMATIKA I EE PRIMENENIYA~---
INFORMATICS AND APPLICATIONS\ \ \ 2016\ \ \ volume~10\ \ \ issue\ 2
\hfill \textbf{\thepage}}}

\vspace*{3pt}


\Abste{For nonlinear differential stochastic systems on manifolds (MStS) 
with Wiener and Poisson noises, the synthesis theory of ellipsoidal suboptimal filters 
based on ellipsoidal approximation and ellipsoidal linearization\linebreak\vspace*{-12pt}}

\Abstend{methods is developed. Special attention is paid to MStS with additive non-Gaussian 
noises based on ellipsoidal linearization method. The algorithms are the basis of the
experimental software tool <<StS-Filter>> (version 2016). Accuracy and sensitivity 
equations are presented. Some generalizations are mentioned.}

\KWE{ellipsoidal approximation method (EAM);
ellipsoidal linearization method (ELM);
orthogonal expansions method (OEM);
Poisson  noise;
stochastic system on manifolds (MStS);
suboptimal filter (SOF);
Wiener noise}

\DOI{10.14357/19922264160203}

\vspace*{-12pt}

\Ack
\noindent
The research was supported by the Russian Foundation for Basic Research 
(project 15-07-002244).


%\vspace*{3pt}

  \begin{multicols}{2}

\renewcommand{\bibname}{\protect\rmfamily References}
%\renewcommand{\bibname}{\large\protect\rm References}

{\small\frenchspacing
 {%\baselineskip=10.8pt
 \addcontentsline{toc}{section}{References}
 \begin{thebibliography}{99}


\bibitem{1-s1-1}
\Aue{Sinitsyn, I.\,N.} 2015.
Analiticheskoe modelirovanie raspredeleniy na osnove ortogonal'nykh 
razlozheniy v~ne\-li\-ney\-nykh stokhasticheskikh sistemakh na mnogoobraziyakh 
[Analytical modeling in stochastic systems on manifolds based on orthogonal 
expansions]. \textit{Informatika i ee Primeneniya}~--- \textit{Inform Appl.} 
9(2):17--24.


\bibitem{2-s1-1}
\Aue{Sinitsyn, I.\,N.} 2015.
Primenenie ortogonal'nykh raz\-lo\-zhe\-niy dlya analiticheskogo modelirovaniya 
mno\-go\-mer\-nykh raspredeleniy v~nelineynykh stokhasticheskikh sistemakh na 
mnogoobraziyakh [Applications of orthogonal expansions for analytical modeling 
of multidimensional distributions in stochastic systems on manifolds].
\textit{Sistemy i~Sredstva Informatiki}~--- \textit{Systems and Means of Informatics}
 25(3):3--22.

\bibitem{3-s1-1}
\Aue{Sinitsyn, I.\,N.} 2016.
Ortogonal'nye suboptimal'nye fil'try dlya nelineynykh stokhasticheskikh 
sistem na mno\-go\-ob\-ra\-zi\-yakh [Orthogonal suboptimal filters for nonlinear 
stochastic systems on manifolds].
 \textit{Informatika i ee Primeneniya}~--- \textit{Inform Appl.}  
 10(1):34--44.
 
 \bibitem{7-s1-1} %4
\Aue{Watanabe,~S., and N. Ikeda}. 1981. 
\textit{Stochastic differential equations and diffusion processes}. 
Amsterdam\,--\,Oxford\,--\,New York: North-Holland Publishing Co.; 
Tokyo: Kodansha Ltd. 476~p.

\bibitem{6-s1-1} %5
Korolyuk, V.\,S., N.\,I.~Portenko, A.\,V.~Skorokhod, and
A.\,F.~Turbin, eds. 1985.
\textit{Spravochnik po teorii veroyatnosti i~matematicheskoy statistike}
[Handbook: Probability theory and mathematical statistics].
 Moscow: Nauka. 640~p.

\bibitem{4-s1-1} %6
 \Aue{Pugachev, V.\,S., and I.\,N.~Sinitsyn.} 
1987. \textit{Stochastic differential systems.
Analysis and filtering}. Chichester\,--\,New York, NY: Jonh Wiley.
549~p.


\bibitem{5-s1-1} %7
 \Aue{Pugachev, V.\,S., and I.\,N.~Sinitsyn.} 
 2001.  \textit{Stochastic systems. Theory and  applications}.
Singapore: World Scientific. 908~p.


\bibitem{8-s1-1}
\Aue{Sinitsyn, I.\,N.} 2007.
\textit{Fil'try Kalmana i~Pugacheva} [Kalman and Pugachev filters]. 
2nd ed. Moscow: Logos. 776~p.

\bibitem{9-s1-1}
\Aue{Wonham, W.\,M.} 1964.
Some applications of stochastic differential equations to optimal nonlinear 
filtering. \textit{J.~Soc. Ind. Appl. Math. A} 2(3):347--369.

\bibitem{10-s1-1}
\Aue{Zakai, M.} 1969.
On the optimal filtering of diffusion processes.
\textit{Z. Wahrscheinlichkeit.} 11:230--243.


\bibitem{11-s1-1}
\Aue{Sinitsyn, I.\,N., and V.\,I.~Sinitsyn.} 2013.
\textit{Lektsii po teorii normal'noy i~ellipsoidal'noy approkskimatsii raspredeleniy 
v~stokhasticheskikh sistemakh} [Lectures on normal and ellipsoidal approximation 
of distributions in stochastic systems].  Moscow: TORUS PRESS. 488~p.

\end{thebibliography}

 }
 }

\end{multicols}

\vspace*{-3pt}

\hfill{\small\textit{Received February 29, 2016}}

\Contr


\noindent
\textbf{Sinitsyn Igor N.} (b.\ 1940)~---
Doctor of Science in technology, professor,
Honored scientist of RF, Head of Department, Institute of Informatics Problems, Federal Research Center ``Computer Science and
Control'' of the Russian Academy of Sciences, 44-2 Vavilov Str.,
Moscow 119333, Russian Federation; sinitsin@dol.ru

\vspace*{3pt}

\noindent
\textbf{Sinitsyn Vladimir I.} (b.\ 1968)~---
 Doctor of Science in physics and mathematics,
associate professor, Head of Department, Institute of Informatics Problems, Federal Research Center ``Computer Science and
Control'' of the Russian Academy of Sciences, 44-2 Vavilov Str.,
Moscow 119333, Russian Federation; vsinitsin@ipiran.ru

\vspace*{3pt}

\noindent
\textbf{Korepanov Eduard R.} (b.\ 1966)~---
Candidate of Science (PhD) in technology, 
Head of Laboratory, Institute of Informatics Problems, Federal Research Center 
``Computer Science and Control'' of the Russian Academy of Sciences, 
44-2~Vavilov Str., Moscow 119333, Russian Federation; ekorepanov@ipiran.ru 

\label{end\stat}


\renewcommand{\bibname}{\protect\rm Литература}   %3     
\def\stat{goncharov}

\def\tit{ВЫРАВНИВАНИЕ ДЕКАРТОВЫХ ПРОИЗВЕДЕНИЙ УПОРЯДОЧЕННЫХ МНОЖЕСТВ$^*$}

\def\titkol{Выравнивание декартовых произведений упорядоченных множеств}

\def\aut{А.\,В.~Гончаров$^1$, В.\,В.~Стрижов$^2$}

\def\autkol{А.\,В.~Гончаров, В.\,В.~Стрижов}

\titel{\tit}{\aut}{\autkol}{\titkol}

\index{Гончаров А.\,В.}
\index{Стрижов В.\,В.}
\index{Goncharov A.\,V.}
\index{Strijov V.\,V.}


{\renewcommand{\thefootnote}{\fnsymbol{footnote}} \footnotetext[1]
{Работа выполнена при частичной финансовой поддержке РФФИ 
(проекты 19-07-1155 и~19-07-00885). Настоящая статья содержит 
результаты проекта <<Статистические методы машинного обучения>>, 
выполняемого в~рамках реализации Программы Центра компетенций 
Национальной технологической инициативы <<Центр хранения 
и~анализа больших данных>>, поддерживаемого Министерством науки 
и~высшего образования Российской Федерации по договору МГУ им.\ 
М.\,В.~Ломоносова  с~Фондом поддержки проектов Национальной 
технологической инициативы от 11.12.2018 №\,13/1251/2018.}}


\renewcommand{\thefootnote}{\arabic{footnote}}
\footnotetext[1]{Московский физико-технический институт, alex.goncharov@phystech.edu}
\footnotetext[2]{Вычислительный центр им.\ А.\,А.~Дородницына Федерального исследовательского 
центра <<Информатика и~управ\-ле\-ние>> Российской академии наук; 
Московский фи\-зи\-ко-тех\-ни\-че\-ский институт, \mbox{strijov@ccas.ru}}

%\vspace*{-12pt}



\Abst{Работа посвящена исследованию метрических методов анализа 
объектов сложной структуры. Предлагается обобщить метод динамического 
выравнивания двух временных рядов на случай объектов, определенных на 
двух и~более осях времени. В~дискретном представлении такие объекты 
являются матрицами. Метод динамического выравнивания временных рядов 
обобщается как метод динамического выравнивания матриц. Предложена 
функция расстояния, устойчивая к~монотонным нелинейным деформациям 
декартова произведения двух и~более временных шкал. Определен выравнивающий 
путь между объектами. В~дальнейшем объектом называется матрица, 
в~которой строки и~столбцы соответствуют осям времени. Исследованы 
свойства предложенной функции расстояния. Для иллюстрации метода 
решаются задачи метрической классификации объектов на модельных 
данных и~данных из набора MNIST.}

\KW{функция расстояния; динамическое выравнивание; расстояние между матрицами; 
нелинейные деформации времени; про\-стран\-ст\-вен\-но-вре\-мен\-ные ряды}

\DOI{10.14357/19922264200105} 
  
\vspace*{-3pt}


\vskip 10pt plus 9pt minus 6pt

\thispagestyle{headings}

\begin{multicols}{2}

\label{st\stat}


\section{Введение}

Временн$\acute{\mbox{ы}}$е ряды представляют собой набор измерений, упорядоченных 
по оси времени. Анализ временн$\acute{\mbox{ы}}$х рядов производится при решении задач, 
связанных с~классификацией активности человека по измерениям акселерометра 
телефона, поиском паттернов в~EEG-сиг\-на\-лах (электроэнцефалограмма), 
кластеризации набора ECoG (электрокортикограмма) данных и~во многих других 
задачах~\cite{0}. Рассматриваются объекты, для которых время между измерениями 
фиксированно. В~данной работе для построения адекватной функции 
расстояния между объектами требуется учесть нелинейные деформации 
относительно оси времени: глобальные и~локальные сдвиги, растяжения 
и~сжатия~\cite{1}.

В~\cite{2} приводятся различные методы решения задач анализа 
временн$\acute{\mbox{ы}}$х рядов: классификации, детектирования паттернов, 
кластеризации и~др. В~\cite{3} описание временных рядов 
строится с~по\-мощью анализа параметров моделей, в~\cite{4} 
используется их признаковое описание, в~\cite{5} анализируется их форма. 
Комбинации этих подходов описаны в~\cite{2}.

Метрические методы находят схожие объекты в~наборе. Используются 
функции расстояния над временн$\acute{\mbox{ы}}$ми рядами: расстояние Хаусдорфа~\cite{10}, 
MODH~\cite{11}, расстояние, основанное на HMM
(hiden Markov model)~\cite{6}, евклидово расстояние 
в~исходном пространстве или в~пространстве сниженной размерности~\cite{5}, 
\mbox{LCSS} (longest common\linebreak subsequence)~\cite{7}. Показано~\cite{8}, что в~случае локальных или глобальных 
деформаций времени при решении задач, требующих анализа исходной формы 
временн$\acute{\mbox{о}}$го ряда, метод динамического выравнивания оси времени 
DTW (Dynamic Time Warping) 
превосходит другие функции расстояния~\cite{9} по качеству итогового 
решения задачи, так как при наличии смещений двух объектов относительно 
друг друга требуется выравнивать их оптимальным образом для вычисления 
расстояния между ними.

В данной работе предлагается перейти от рас\-смот\-ре\-ния объекта~$\textbf{s}(t)$, 
временн$\acute{\mbox{о}}$го ряда, к~более общему случаю $\textbf{s}(\textbf{t})$, 
в~котором компоненты вектора~$\textbf{t}$~--- оси времени. Из-за 
существенного рос\-та вы\-чис\-ли\-тель\-ной слож\-ности при увеличении чис\-ла 
осей времени предлагается рас\-смот\-реть объекты $\textbf{s}(t_1, t_2)$, 
определенные на двух осях времени. Оси времени считаются независимыми. 
В~случае единственной дискретной и~ограниченной сверху шкалы времени 
объект представим вектором фиксированной размерности. 
Аналогично объект настоящего исследования представим мат\-ри\-цей.

Вводятся ограничения на зависимости осей времени в~декартовом 
произведении для таких объектов. Определена гипотеза порождения данных: 
объекты одного класса эквивалентности получены при помощи допустимых 
преобразований, а~именно: локальных деформаций (растяжений и~сжатий) 
каждой из осей времени по отдельности. В~дискретном случае преобразование 
представимо дуп\-ли\-ци\-ро\-ва\-ни\-ем строк и~столбцов матриц. 
В~число допустимых преобразований попадают и~глобальные деформации: 
сдвиги по осям времени, представимые добавлением и~удалением крайних 
строк и~столбцов исходных матриц. Для каждой из осей времени выполняются 
свойства времени: монотонность и~непрерывность. Похожими на описанные 
свойствами обладает, например, частотный спектр сигнала, где одна ось 
определяет время, а другая~--- частоту, величину, обратную времени.


Между двумя объектами, матрицами, в~случае допустимых преобразований 
требуется определить инвариантную к~преобразованиям осей времени функцию 
расстояния, которая сможет выделить классы эквивалентности множества 
преобразованных объектов. Работа посвящена определению такой функции 
расстояния, как обобщения метода динамического выравнивания временных рядов 
DTW для матриц.

Цель данной работы~--- построение метода, основанного на динамическом 
выравнивании осей времени для матриц. Метод динамического выравнивания 
временн$\acute{\mbox{ы}}$х рядов~\cite{33} определен только для объектов с~одной осью времени, 
что делает его неприменимым для описанного случая. Однако концепции, 
используемые на каждой стадии вы\-чис\-ле\-ния оптимального выравнивания, обобщены 
на рассматриваемый случай. Работа исследует свойства предложенного 
метода и~сравнивает результаты применения метода к~задачам классификации 
изображений~\cite{12} с~результатами функции расстояния~$L_2$.

Для иллюстрации и~анализа результатов решается задача метрической 
классификации объектов (матриц низкой размерности). Используются наборы данных: 
модельные данные, которые согласуются с~выдвинутой гипотезой порождения 
данных для временн$\acute{\mbox{ы}}$х рядов, подмножество набора MNIST сниженной 
размерности и~частотный спектр сигнала.

\vspace*{-10pt}

\section{Постановка задачи построения функции расстояния}

\vspace*{-2pt}

Рассмотрим задачу построения функции расстояния между объектами. 
Функция расстояния инвариантна к~допустимым преобразованиям осей времени: 
глобальным и~локальным линейным и~нелинейным деформациям временн$\acute{\mbox{о}}$й шкалы. 
Ниже приведены две постановки задачи, с~помощью которых определены свойства 
предложенной функции расстояния, оценено ее качество и~проведено сравнение 
нескольких функций расстояния: предложенной и~$L_2$.

Первая постановка задачи использует общее свойство функций расстояния: 
объединение схожих объектов и~разделение непохожих объектов. 
Вводится определение свойства инвариантности функции расстояния к~допустимым 
преобразованиям осей времени.
Вторая постановка задачи уточняет первую и~заключается в~проведении метрической 
классификации методом ближайшего соседа.

\textbf{Постановка задачи выбора функции расстояния между двумя объектами.}
На двух временн$\acute{\mbox{ы}}$х осях заданы объекты вида 
$\textbf{A}(t_1,t_2)\hm \in \mathbb{R}^{n \times n}$. 
Функция $G_w(\textbf{A}):\mathbb{R}^{n \times n} \hm\rightarrow 
\mathbb{R}^{\hat{n} \times \hat{n}}$ задает допустимые преобразования 
исходного объекта~$\textbf{A}$: глобальные сдвиги, локальные линейные 
и~нелинейные деформации, а~именно: растяжения и~сжатия оси времени, 
сдвиги значений по оси времени. Скалярный параметр $w \hm\in \mathbb{R}^+$
 функции~$G$ фиксирует набор этих преобразований.

Допустимым элементарным преобразованием матрицы~$\textbf{A}$ назовем 
дуплицирование случайных строк и~столбцов исходной матрицы, добавление 
или удаление крайних строк и~столбцов. Допустимым преобразованием 
примем некоторую последовательность допустимых элементарных 
преобразований матрицы~$\textbf{A}$ и~обозначим как~$G_w(\textbf{A})$.

Будем называть объект~$\textbf{B} \hm\in \mathbb{R}^{\hat{n} \times \hat{n}}$ 
полученным из объекта~$\textbf{A}$ при помощи допустимых 
преобразований~$G_{\hat{w}}$, если существует $\hat{w}\hm\in \mathbb{R}^+ : 
\textbf{B} \hm= G_{\hat{w}}(\textbf{A})$.

Функцию расстояния между двумя объектами $\rho: 
\mathbb{R}^{{n} \times {n}} \times \mathbb{R}^{\hat{n} \times \hat{n}} 
\hm\rightarrow  \mathbb{R}^+$ оценим на выборке $\mathfrak{D } \hm= 
\{ \textbf{A}_i \}_{i=1}^m$ объектов вида $\textbf{A}_i \hm\in 
\mathbb{R}^{n \times n}$.

Для каждого объекта выборки~$\textbf{A}_i$ и~объекта~$\textbf{B}_j$ его 
класса эквивалентности $\{\textbf{B}_j\}_i \hm= \{  \textbf{B} 
\hm\in \mathfrak{D} | \exists w_i,w_j: G_{w_i}(\textbf{A}_i) \hm= G_{w_j}
(\textbf{B}_j)   \}$ заданы допустимые трансформации с~параметрами~$w_i$ 
и~$w_j$, такие что $G_{w_i}(\textbf{A}_i)\hm = G_{w_j}(\textbf{B}_j)$. 
Для каждого объекта выборки~$\textbf{A}_i$ и~объекта~$\textbf{C}_j$ 
из других классов эквивалентности $\{ \textbf{C}_k\}_i \hm= 
\{  \textbf{C} \hm\in \mathfrak{D} | \nexists w_i,w_k: G_{w_i}(\textbf{A}_i)
\hm = G_{w_k}(\textbf{C})   \}$ не существует таких $ w_i, w_k : G_{w_i}
(\textbf{A}_i) \hm= G_{w_k}(\textbf{C}_k)$.

Решается задача поиска функции расстояния~$\rho$, значение
 которой на паре объектов одного класса эквивалентности меньше, 
 чем на любой паре объектов из разных: для любых $i,j,k \hm\in 
 \{1,\dots,m\}$ $\quad \rho(\textbf{A}_i,\textbf{B}_j) \hm< 
 \rho(\textbf{A},\textbf{C}_k)$. Функцию расстояния, обладающую 
 таким свойством, назовем инвариантной на классах эквивалентности.

Критерием качества для функции расстояния~$\rho$ на выборке~$\mathfrak{D}$ 
примем долю объектов, для которых указанное неравенство выполняется:
$$
S_{\rho}(\mathfrak{D}) = \fr{1}{m} \sum\limits_{i=1}^m 
\prod\limits_{\{ \textbf{B}_j\}_i} 
\prod\limits_{\{ \textbf{C}_k\}_i}  
\left[  \rho(\textbf{A}_i,\textbf{B}_j) < \rho(\textbf{A}_i,\textbf{C}_k)  
 \right].
 $$
Постановка задачи выбора функции расстояния~$\rho$ 
сводится к~задаче максимизации критерия качества.

\textbf{Прикладное использование функции расстояния.}
Задана выборка $\mathfrak{D}\hm = \{(\textbf{A}_i,y_i)\}^m_{i=1}$, 
состоящая из пар объ\-ект--от\-вет. Объектами служат объекты сложной 
структуры: $\textbf{A}_i\hm \in \mathbb{R}^{n\times n}$, 
а~ответами выступают метки класса~---~$y_i\hm \in Y \hm= \{1,\ldots,E\}$, 
где $E \hm\ll m$. Выборка разделена на обучение $\mathfrak{D}_l \hm= 
\{(\textbf{A}_i,y_i)\}^{m_1}_{i=1}$ и~контроль $\mathfrak{D}_t \hm= 
\{(\textbf{A}_i,y_i)\}_{m_1}^{m_1+m_2}$.

Модель классификации~$f$ принадлежит множеству моделей метрической 
классификации 1NN, которые классифицируемому объекту ставят 
в~соответствие метку класса ближайшего объекта из обучающей 
выборки по заданной функции расстояния~$\rho$:
$$ 
\hat{y} = f(\textbf{B} | \rho) = y \argmin\limits_{i = 1,\dots, m_1} 
\rho\left(B,A_i\right)\,.
$$
Критерий качества $S$ модели~$f$ для задачи классификации~--- 
доля правильно проставленного класса на контрольной выборке:
 $$ 
 S(f | \rho) = \fr{1}{m_2}\sum\limits_{i=m_1}^{m_1+m_2} 
 \left[f(\textbf{A}_i | \rho) = y_i\right].
 $$

Требуется выбрать функцию расстояния~$\rho$ для модели 
классификации~$f:~\mathbb{R}^{n\times n} \hm\rightarrow~Y$, 
максимизируюшую критерий качества~$S$ на контрольной выборке:
\begin{equation*}
f =  \argmax\limits_{\rho \in \{\mathrm{mDTW}, L_2\}}\left(S(f | \rho)\right).
\end{equation*}

\section{Вычисление матричного расстояния mDTW}

Предлагается использовать функцию расстояния DTW, 
модифицированную для случая выравнивания двойной шкалы времени.

\smallskip

\noindent
\textbf{Определение~1.} {Даны два объекта~$\textbf{A},\textbf{B}\hm \in 
\mathbb{R}^{n\times n}$. Тензор 
невязок~$\boldsymbol{\Omega}^{n \times n \times n \times n}$~--- 
такой тензор, что его элемент~$\boldsymbol{\Omega}(i,j,k,l)$ 
равен квадрату разности между элементами~$\textbf{A}(i,j)$ и~$\textbf{B}(k,l)$:}
\begin{equation*}
\boldsymbol{\Omega}(i,j,k,l)=(\textbf{A}(i,j) - \textbf{B}(k,l))^2.
\end{equation*}

\noindent
\textbf{Определение 2.} {Путем~$\boldsymbol{\pi}$ между двумя 
объектами $\textbf{A},\textbf{B} \hm\in \mathbb{R}^{n\times n}$ 
назовем множество индексов тензора~$\boldsymbol{\Omega}$: }
$$
\boldsymbol{\pi} = \{(i,j,k,l)\},\quad i,j,k,l \in \{1,\ldots,n\} ,
$$
\textit{удовлетворяющее следующим условиям:}

{\bfseries\textit{Частичный порядок.}}
Для элементов пути~$\boldsymbol{\pi}$ с~фиксированными значениями~$i,k$ 
задан порядок: выравнивающий путь для фиксированных строк двух 
матриц упорядочен~--- $\{(i,j_r,k,l_r))\}_{r=1}^{R} \hm\subset 
\boldsymbol{\pi}$ мощностью~$R$. Аналогично для фиксированных столбцов 
с~индексами~$j,l$.

{\bfseries\textit{Граничные условия.}}
 Пусть $(i,j,k,l) \in \boldsymbol{\pi}$, тогда $(1,j,1,l) \hm\in 
 \boldsymbol{\pi}$ и~$(i,1,k,1) \hm\in \boldsymbol{\pi}$.
Путь $\boldsymbol{\pi}$ содержит элементы тензора~$\boldsymbol{\Omega}$: 
$(1,1,1,1) \hm\in \boldsymbol{\pi}$ и~$(n,n,n,n) \hm\in \boldsymbol{\pi}$.

{\bfseries\textit{Непрерывность по направлению.}}
Для упорядоченного подмножества пути $\{(i,j_r,k,l_r)\}_{r=1}^{R}
\hm\subset\boldsymbol{\pi}$ выполняется условие непрерывности:
$$
j_{r}-j_{r-1}\leq1\,,\quad l_r-l_{r-1}\leq1\,, \quad r = 2,\ldots,R\,.
$$
На~шаге пути~$\boldsymbol{\pi}$ по фиксированному направлению времени~$i,k$ 
встречаются только соседние элементы матрицы (включая соседние по диагонали). 
Аналогично для фиксированных~$j,l$.

{\bfseries\textit{Монотонность по направлению.}}
Для упорядоченного подмножества пути  $\{(i,j_r,k,l_r)\}_{r=1}^{R}
\hm\subset\boldsymbol{\pi}$ выполняется хотя бы одно из условий 
монотонности функции выравнивания времени: 
$$
j_{r}-j_{r-1}\geq1\,,\quad l_r-l_{r-1}\geq1\,, \quad r = 2,\ldots,R\,.
$$

Свойства пути между матрицами обобщают свойства пути между двумя 
временными рядами.

\smallskip

\noindent
\textbf{Определение~3.}\ {Стоимость 
$\mathrm{Cost}\,(\textbf{A},\textbf{B},{\boldsymbol{\pi}})$ пути $\boldsymbol{\pi}$ 
между объектами $\textbf{A}, \textbf{B}$:
\begin{equation*}
\mathrm{Cost}\,(\textbf{A},\textbf{B},{\boldsymbol{\pi}}) = 
\sum\limits_{(i,j,k,l) \in \boldsymbol{\pi}}{\boldsymbol{\Omega}}(i,j,k,l).
\end{equation*}}

\noindent
\textbf{Определение~4.}\ 
{Выравнивающий путь~$\hat{\boldsymbol{\pi}}$ между 
объектами $\textbf{A},\textbf{B}$~--- путь наименьшей стоимости 
среди всех возможных путей между объектами:
\begin{equation*}
\hat{\boldsymbol{\pi}} = 
\argmin\limits_{{\boldsymbol{\pi}}} \mathrm{Cost}
\left(\textbf{A},\textbf{B},{\boldsymbol{\pi}}\right).
\end{equation*}}
Функция расстояния~$\rho (\textbf{A},\textbf{B})\hm = \mathrm{mDTW}\,
(\textbf{A},\textbf{B})$ между объектами~$\textbf{A}$ и~$\textbf{B}$ 
рассчитывается как стоимость выравнивающего пути~$\hat{\boldsymbol{\pi}}$:
\begin{equation}
\mathrm{mDTW}(\textbf{A},\textbf{B}) = \mathrm{Cost}\left(\textbf{A},
\textbf{B},\hat{\boldsymbol{\pi}}\right).
\end{equation}

\setcounter{figure}{1}
\begin{figure*}[b] %fig2
{\small 
\begin{center}
\begin{tabular}{l}
\hline
DTW(\textbf{s},\textbf{c}):\\
\hspace*{3mm}$\boldsymbol{D}$(1:n+1,1:m+1) = inf;\\
\hspace*{3mm}$\boldsymbol{D}$(1,1) = 0;\\
\hspace*{3mm}for $i = 2$: $n+1$\\
\hspace*{6mm}for $j = 2$ : $m+1$\\
\hspace*{9mm}$d = (\textbf{s}(i-1)-\textbf{c}(j-1))^2$;\\
\hspace*{9mm}$\boldsymbol{D}(i,j) = d + \min( 
[ \boldsymbol{D}(i-1,j), \boldsymbol{D}(i,j-1), \boldsymbol{D}(i-1,j-1) ])$;\\
return\ sqrt$(\boldsymbol{D}(n+1,m+1))$\\
\hline
\end{tabular}
\end{center}}
\vspace*{-9pt}

\Caption{Алгоритм вычисления DTW для временных рядов
\label{ris:dtwts}}
%\end{figure*}
%\begin{figure*} %fig3
\vspace*{6pt}
{\small 
\begin{center}
\begin{tabular}{l}
\hline
\\[-9pt]
Correction $(\overline{i,j,k,l}, \boldsymbol{\pi}(\overline{i,j,k,l})):$\\
\hspace*{3mm}if $\overline{i,j,k,l} \in \{ (i-1, j, k,l)  ;  
(i, j, k-1, l)  ;  (i-1, j, k-1, l) \}$:\\
\hspace*{6mm}$ \widehat{\pi} = \{ (\overline{i}, r, \overline{k}, f) \in 
\boldsymbol{\pi}(\overline{i, j, k, l}) \vert r, f \in \mathbb{N} \}$\\
\hspace*{3mm}elif $\overline{i,j,k,l}\in \{  
(i, j-1, k, l); (i, j, k, l-1); (i, j-1, k, l-1) \}$:\\
\hspace*{6mm}$\widehat{\pi} = \{ (r, \overline{j}, f, \overline{l}) 
\in \boldsymbol{\pi}(\overline{i, j, k, l}) \vert r, f \in \mathbb{N} \}$\\
\hspace*{3mm}elif $\overline{i,j,k,l} =  i-1,j-1,k-1,l-1:$\\
\hspace*{6mm}$\widehat{\pi} = \{ (\overline{i}, r, \overline{k}, f) 
\in \boldsymbol{\pi}(\overline{i, j, k, l}) \vert r,f \in \mathbb{N} \} \cup$\\
\hspace*{6mm}$\cup \{ (r, \overline{j}, f, \overline{l}) \in \boldsymbol{\pi}
(\overline{i, j, k, l}) \vert r,f \in \mathbb{N} \}$\\
\hspace*{3mm}$\boldsymbol{d\pi} = \{ \mathrm{element} \in \widehat{\pi}: 
\mbox{произведены\ замены\ индексов } 
\overline{i} = i,\ \overline{j} = j,\ \overline{k} = k,\ \overline{l} = l \}$\\
return $\boldsymbol{d\pi}$\\
\hline
\end{tabular}
\end{center}
}
\vspace*{-9pt}

\Caption{Алгоритм вычисления поправки $\boldsymbol{d\pi}$ 
пути $\boldsymbol{\pi}$
\label{ris:codedpi}}
\end{figure*}


\textbf{Алгоритм вычисления значения расстояния~(4).}
Построение алгоритма вычисления значения функции расстояния 
между матрицами основан на алгоритме расчета функции расстояния 
между временн$\acute{\mbox{ы}}$ми рядами. В~случае выравнивания одной\linebreak\vspace*{-12pt}

{ \begin{center}  %fig1
 \vspace*{-3pt}
    \mbox{%
 \epsfxsize=79mm 
 \epsfbox{gon-1.eps}
 }


\end{center}


\noindent
{{\figurename~1}\ \ \small{Матрица стоимости оптимального выравнивания, по обеим 
осям отложены временные отсчеты}}
}

\vspace*{12pt}


\noindent 
временн$\acute{\mbox{о}}$й шкалы
 итоговая матрица расстояний~$\boldsymbol{D}$ (рис.~1) в~каждом 
 элементе~$\boldsymbol{D}(i,j)$ содержит рас\-сто\-яние между подрядом 
 первого временн$\acute{\mbox{о}}$го ряда и~подрядом второго временн$\acute{\mbox{о}}$го ряда. 
 Рас\-смот\-рим алгоритм динамического выравнивания двух временн$\acute{\mbox{ы}}$х 
 рядов $\textbf{s} \hm\in R^n$ и~$\textbf{c} \hm\in R^m$ на рис.~2.
 
 

Элемент $\boldsymbol{D}(i,j)$ матрицы~$\boldsymbol{D}$ соответствует 
стоимости выравнивающего пути между подпоследовательностями 
исходных временн$\acute{\mbox{ы}}$х рядов: $\textbf{s}(1:i) \hm= \textbf{s}(t)$, 
$t \hm= 1,\ldots,i,$ и~$\textbf{c}(1:j) \hm= \textbf{c}(t)$, $t \hm= 1,\ldots,j$. 
Алгоритм построения наилучшего выравнивания времени 
подразумевает, что выравнивающий путь между этими 
подпоследовательностями получен одним из трех способов~--- 
если стоимость выравнивающего пути между 
подпоследовательностями~$\textbf{s}(1:\overline{i}) $ 
и~$\textbf{c}(1:\overline{j})$ минимальна для~$\overline{i,j}$ из множества
$$
\overline{i,j} \in \left\{ \{i-1,j\},\{i,j-1\},\{i-1,j-1\} \right\},$$
тогда выравнивающий путь между $\textbf{s}(1:i)$ и~$\textbf{c}(1:j)$ получен добавлением пары~$(i,j)$ к~выбранному 
выравнивающему пути с~минимальной стоимостью из трех.



Предложенный алгоритм переносит эти рас\-суж\-де\-ния на случай 
выравнивания двух матриц~$\textbf{A}$ и~$\textbf{B}$. 
Элемент~$\boldsymbol{D}(i,j,k,l)$ четырехиндексного
 тензора расстояний~$\boldsymbol{D}$ соответствует стоимости выравнивающего 
 пути между $\textbf{A}(1:i,1:j) \hm= \textbf{A}(t_1,t_2)$, 
 $t_1 \hm= 1,\ldots, i$, $t_2 \hm= 1,\ldots, j,$ 
 и~$\textbf{B}(1:k,1:l) \hm= \textbf{B}(t_1,t_2)$, $t_1 \hm= 1,\ldots, k$,
 $t_2 \hm= 1,\ldots, l$. Выравнивающий путь между этими 
 подматрицами получен одним из семи способов~--- 
 если стоимость выравнивающего пути между 
 подматрицами $\textbf{A}(1:\overline{i},1:\overline{j})$ 
 и~$\textbf{B}(1:\overline{k},1:\overline{l})$ 
 минимальна для~$\overline{i,j,k,l}$ из множества
\begin{multline*} 
\overline{i,j,k,l} \in 
\left\{ \{i-1,j,k,l\},\{i,j-1,k,l\},\right.\\
\{i,j,k-1,l\},
\{i,j,k,l-1\}, \{i-1,j,k-1,l\},\\
\left.
\{i,j-1,k,l-1\},\{i-1,j-1,k-1,l-1\}\right\},
\end{multline*}

\setcounter{figure}{3}
\begin{figure*} %fig4
{\small 
\begin{center}
\begin{tabular}{l}
\hline
$\mathrm{mDTW}\left(\textbf{A},\textbf{B}\right):$\\
\hspace*{3mm}$\textbf{D}(1:n+1,1:n+1, 1:n+1, 1:n+1) = inf$;\\
\hspace*{3mm}$\textbf{D}(1,1,1,1) = 0;$\\
\hspace*{3mm}$\boldsymbol{\pi}(1,1,1,1) = ((1,1),(1,1))$\\
\hspace*{3mm}$for\ i,j,k,l  \in \mathbb{N}^{2 : n+1} \times 
\mathbb{N}^{2 : n+1} \times \mathbb{N}^{2 : n+1} \times \mathbb{N}^{2 : n+1}:$\\
\hspace*{6mm}$\overline{i,j,k,l} = \argmin($ [ \textbf{D}(i-1, j, k, l), 
\textbf{D}(i, j-1, k, l), \textbf{D}(i, j, k-1, l), 
\textbf{D}(i, j, k, l-1),    \\
\hspace*{9mm}$\textbf{D}(i-1, j, k-1, l), \textbf{D}(i, j-1, k, l-1), 
\textbf{D}(i-1, j-1, k-1, l-1) ])$;\\
\hspace*{3mm}$\boldsymbol{d \pi} = \mathrm{Correction}\,(\overline{i,j,k,l}, 
\boldsymbol{\pi}(\overline{i,j,k,l}))$\\
\hspace*{3mm}$\boldsymbol{\pi}(i, j, k, l) = \boldsymbol{d \pi} \cup 
\{(\overline{i,j,k,l})\}$\\
\hspace*{3mm}$\mathrm{cost} = (\textbf{A}(i, j)-\textbf{B}(k, l))^2 + 
\sum\nolimits_{(r,f,t,g) \in \boldsymbol{d \pi}}
(\textbf{A}(r, f)-\textbf{B}(t, g))^2$;\\
\hspace*{3mm}$\textbf{D}(i,j,k,l) = \mathrm{cost} + \textbf{D}
(\overline{i,j,k,l})$\\
return  sqrt$(\textbf{D}(n+1,n+1,n+1,n+1))$\\
\hline
\end{tabular}
\end{center}
}
\vspace*{-9pt}

\Caption{Алгоритм вычисления расстояния между матрицами
\label{ris:matrixdtw}}
\end{figure*}

\begin{table*}[b]\small
\begin{center}
\begin{tabular}{|l|c|c|c|c|}
\multicolumn{5}{c}{Снижение расстояний при выполнении преобразований 
для различных наборов данных}\\
\multicolumn{5}{c}{\ }\\[-6pt]
\hline
 &\multicolumn{4}{c|}{Метод}\\
 \cline{2-5}
\multicolumn{1}{|c|}{Данные}  & \multicolumn{2}{c|}{$L_2$} & \multicolumn{2}{c|}{MatrixDTW} \\
\cline{2-5}
& $S(f|p)$  &  $S_{\rho}(\mathfrak{D})$ &  $S(f|p)$ & $S_{\rho}(\mathfrak{D})$ \\
\hline
Модельные данные без преобразований& 92\% & 78\% & 100\%\hphantom{9} & 85\% \\
Модельные данные с~преобразованиями & 86\% & 65\% &  100\%\hphantom{9} & 82\% \\
Модельные данные с~преобразованиями и~шумом& 69\% & 61\% &  92\% & 78\% \\
MNIST без преобразований& 95\% & --- & 95\% & --- \\
MNIST с~преобразованиями & 53\% & --- & 92\% & --- \\
Спектр сигнала& 83\% & --- & 96\% & --- \\
\hline
\end{tabular}
\end{center}
\end{table*}

\noindent
то к~выравнивающему пути между этими под\-мат\-ри\-ца\-ми 
добавляется элемент пути $(i,j,k,l)$ и~поправка~$\boldsymbol{d\pi} $ 
пути~$\boldsymbol{\pi}$, алгоритм вычисления которой приведен ниже.

Обозначим выравнивающий путь между $\textbf{A}(1:i,\linebreak 1:j)$
 и~$\textbf{B}(1:k,1:l)$ как~$\boldsymbol{\pi}(i,j,k,l)$, тогда 
 поправка~$\boldsymbol{d\pi} $ пути~$\boldsymbol{\pi}(i,j,k,l)$ 
 при фиксированных~$\overline{i,j,k,l}$ вычисляется приведенным на рис.~3 
 образом.





Алгоритм динамического выравнивания двух матриц и~вычисления 
расстояния $\mathrm{mDTW}$ между ними с~учетом приведенного выше 
алгоритма примет вид, представленный на рис.~4.





\begin{figure*} %fig5
\vspace*{1pt}
    \begin{center}  
  \mbox{%
 \epsfxsize=161.412mm 
 \epsfbox{gon-5.eps}
 }
\end{center}
\vspace*{-12.5pt}
\Caption{Выравнивание модельных данных: (\textit{а})~один класс без шума; 
(\textit{б})~разные классы без шума; 
(\textit{в})~один класс с~шумом; (\textit{г})~разные классы с~шумом
\label{ris:random}}
%\end{figure*}
%\begin{figure*} %fig6
\vspace*{1pt}
    \begin{center}  
  \mbox{%
 \epsfxsize=163mm 
 \epsfbox{gon-6.eps}
 }
\end{center}
\vspace*{-12.5pt}
\Caption{Выравнивание данных MNIST: левый столбец~--- один класс; 
правый столбец~--- разные 
классы;
(\textit{а})~$\mathrm{mDTW}\hm=720{,}1$; 
(\textit{б})~948,6;
(\textit{в})~2017,0;
(\textit{г})~$\mathrm{mDTW}\hm=2071{,}4$
\label{ris:mnist}}
\end{figure*}


Следует отметить, что алгоритм~\cite{15} имеет\linebreak высокую сложность 
вычисления~--- $O(n^4)$. Предполагается ускорение метода 
с~использованием ограниче\-ния Sakoe-Chiba band, что сократит 
вычислительную сложность алгоритма до $O(n^2k^2)$, где~$k$~--- 
параметр ограничения.


\section{Вычислительный эксперимент}

Вычислительный эксперимент проведен на модельных данных с~допустимыми 
преобразованиями и~на реальных данных: объектах коллекции MNIST с~допустимыми 
преобразованиями и~на спектрограммах зашумленных сигналов.





Решается задача метрической классификации методом ближайшего соседа. В~таблице 
приведены значения критерия качества функции расстояния 
$S_{\rho}(\mathfrak{D})$ и~критерия качества метрической классификации $S(f|p)$ 
при использовании двух функций расстояния: предложенной в~работе $\mathrm{mDTW}$ 
и~$L_2$.

Модельные данные~--- это нулевые матрицы со случайными ненулевыми 
строками, столбцами, подпрямоугольниками с~наложенным шумом. 
К~ним применены допустимые преобразования, согласованные с~гипотезой 
наличия локальных и~глобальных искажений. На рис.~\ref{ris:random} 
показан пример оптимального выравнивания двух объектов. 
Линиями показаны элементы пути~$\boldsymbol{\pi}$.

Подготовлена подвыборка набора данных MNIST. Она 
состоит из~100 объектов классов 0 и~1 сниженной размерности
 с~допустимыми преобразованиями. На рис.~\ref{ris:mnist} 
 показан пример оптимального выравнивания объектов.


Аналогичный эксперимент проведен для решения задачи метрической 
классификации спектров различных сигналов, пример которых приведен на 
рис.~\ref{ris:spectr}. На рисунке показаны примеры Фурье-спект\-ров 
этих сигналов. Спектр получен путем применения быстрого преобразования 
Фурье к~исходному сигналу для различных окон с~фиксированным размером и~сдвигом. 
Исходные временн$\acute{\mbox{ы}}$е ряды обладали свойством периодичности, период выбирался 
случайным образом.



Тестирование проведено на разного рода данных: исходных 
модельных данных без наложения\linebreak\vspace*{-12pt}

\pagebreak

\end{multicols}

\begin{figure*} %fig7
\vspace*{1pt}
    \begin{center}  
  \mbox{%
 \epsfxsize=149.062mm 
 \epsfbox{gon-7.eps}
 }
\end{center}
\vspace*{-8pt}
\Caption{Данные спектров сигнала: (\textit{а})~класс~1; (\textit{б})~спектр 
класса~1; (\textit{в})~класс~2; (\textit{г})~спектр класса~2; 
(\textit{д})~класс~3; (\textit{е})~спектр класса~3
\label{ris:spectr}}
\vspace*{9pt}
\end{figure*}

\begin{multicols}{2}

\noindent допустимых преобразований, с~ними, а~также 
на модельных данных с~наложенным поверх объектов случайным шумом.



В каждом из проведенных экспериментов была продемонстрирована 
устойчивость предложенного подхода к~допустимым преобразованиям. 
Наилучшее значение критерия качества задачи классификации было 
достигнуто при использовании предложенной функции расстояния.

\vspace*{-5pt}

\section{Заключение}

В работе предложено обобщение метода динамического выравнивания
 временн$\acute{\mbox{ы}}$х рядов для случая объектов, определенных на двух осях времени. 
 Существует теоретическое обобщение предлагаемых методов на случай 
 конечного множества осей времени. Вычислительный эксперимент позволил 
 проанализировать свойства подхода: устойчивость к~допустимым 
 преобразованиям и~разделяющая способность функции расстояния как 
 на реальных, так и~на модельных данных. Качество решения задачи 
 метрической классификации выше решения, основанного на евклидовом 
 расстоянии. Вычислительная сложность метода высокая, что ограничивает 
 его применимость на объектах высокой размерности.

\vspace*{-2pt}

{\small\frenchspacing
 {%\baselineskip=10.8pt
 \addcontentsline{toc}{section}{References}
 \begin{thebibliography}{99}
%\bibitem{Karasikov2016}
%\Au{Карасиков~М.\,Е., Стрижов~В.\,В.} Классификация временных рядов 
%в~пространстве параметров по\-рож\-да\-ющих моделей~// Информатика и~её 
%применения,~2016. T.~10. Вып.~4. С.~121--131.

\bibitem{0}
\Au{Hill~N.\,J., Lal~T.\,N., Schroder~M., Hinterberger~T., 
Wilhelm~B., Nijboer~F., Mochty~U., Widman~G., Elger~C., 
Scholkopf~B., Kubler~A., Birbaumer~N.} Classifying EEG and 
ECoG signals without subject training for fast BCI implementation: 
Comparison of nonparalyzed and completely paralyzed subjects~//  
IEEE~T. Neur. Sys. Reh., 2006. Vol.~14. 
Iss.~2. P.~183--186.

\bibitem{1}
\Au{Sakoe~H., Chiba~S.} 
A~dynamic programming approach to continuous speech recognition~// 
7th  Congress (International) on Acoustics Proceedings, 1971. Vol.~3. P.~65--69.

\bibitem{2} %3
\Au{Aghabozorgi~S., Ali~S.\,S., Wah~T.\,Y.} 
Time-series clustering~--- a~decade review~// Inform. Syst., 
2015. Vol.~53. P.~16--38.

\bibitem{3} %4
\Au{Warrenliao~T.} Clustering of time series data~--- a~survey~// 
Pattern Recogn., 2005. Vol.~38. Iss.~11. P.~1857--1874.



\bibitem{4} %5
\Au{Hautamaki~V., Nykanen~P., Franti~P.} 
Time-series clustering by approximate prototypes~// 
19th  Conference (International) on Pattern Recognition Proceedings, 2008. No.\,D. 
P.~1--4.

\bibitem{5} %6
\Au{Faloutsos~C., Ranganathan~M., Manolopoulos~Y.} 
Fast subsequence matching in time-series databases~// \mbox{SIGMOD} Rec., 1994. 
Vol.~23. Iss.~2. P.~419--429.

\bibitem{10} %7
\Au{Basalto~N., Bellotti~R., Carlo~F.\,D., Facchi~P., 
Pascazio~S.} Hausdorff clustering of financial time series~// 
Physica~A, 2007. Vol.~379. Iss.~2. P.~635--644.

\bibitem{11} %8
\Au{Gorelick~L., Blank~M., Shechtman~E., Irani~M., Basri~R.} 
Actions as space-time shapes~// IEEE~T. Pattern Anal., 
2007. Vol.~29. Iss.~12. P.~2247--2253.

\bibitem{6} %9
\Au{Smyth~P.} Clustering sequences with hidden Markov models~// 
Adv. Neural In., 1997. Vol.~9. P.~648--654.

\bibitem{7} %10
\Au{Banerjee~A., Ghosh~J.} Clickstream clustering using weighted 
longest common subsequences~// 
Workshop on Web Mining, SIAM Conference on Data Mining
Proceedings, 2001. P.~33--40.

\bibitem{8} %11
\Au{Aach~J., Church~G.M.} Aligning gene expression time series
 with time warping algorithms~// Bioinformatics, 2001. Vol.~17. Iss.~6. P.~495--508.

\bibitem{9} %12
\Au{Yi~B.\,K., Faloutsos~C.} Fast time sequence indexing 
for arbitrary $\mathcal{L}_p$ norms~// 
26th  Conference (International) on Very Large Data Bases Proceedings, 2000. P.~385--394.

\bibitem{33} %13
\Au{Goncharov~A.\,V., Strijov~V.\,V.} 
Analysis of dissimilarity set between time series~// Computational 
Mathematics Modeling, 2018. Vol.~29. Iss.~3. P.~359--366.

\bibitem{12} %14
\Au{Alon~J., Athitsos~V., Sclaroff~S.}
 Online and offline character recognition using alignment to prototypes~// 
 8th  Conference (International) on Document Analysis and Recognition, 2005. 
 Vol.~2. P.~839--843.

\bibitem{15} %15
\Au{Гончаров~А.\,В.} 
Выравнивания декартовых произведений упорядоченных множеств mDTW. 
Про\-грам\-мная реализация алгоритма, 2019. 
{\sf https://github.
com/Intelligent-Systems-Phystech/PhDThesis/tree/\linebreak  master/Goncharov2019/MatrixDTW/code}.
 \end{thebibliography}

 }
 }

\end{multicols}

\vspace*{-9pt}

\hfill{\small\textit{Поступила в~редакцию 24.04.19}}

\vspace*{6pt}

%\pagebreak

%\newpage

%\vspace*{-28pt}

\hrule

\vspace*{2pt}

\hrule

\vspace*{-4pt}

\def\tit{ALIGNMENT OF~ORDERED SET CARTESIAN PRODUCT\\[-5pt]}


\def\titkol{Alignment of~ordered set cartesian product}

\def\aut{A.\,V.~Goncharov$^1$ and~V.\,V.~Strijov$^{1,2}$}

\def\autkol{A.\,V.~Goncharov and~V.\,V.~Strijov}

\titel{\tit}{\aut}{\autkol}{\titkol}

\vspace*{-13pt}


\noindent
$^1$ Moscow Institute of Physics and Technology, 
9~Institutskiy Per., Dolgoprudny, Moscow Region 141700, Russian\linebreak
$\hphantom{^1}$Federation


\noindent
$^2$A.\,A.~Dorodnicyn Computing Center, Federal Research Center 
``Computer Science and Control'' of the Russian\linebreak
$\hphantom{^1}$Academy of Sciences, 
40~Vavilov Str., Moscow 119333, Russian Federation

\def\leftfootline{\small{\textbf{\thepage}
\hfill INFORMATIKA I EE PRIMENENIYA~--- INFORMATICS AND
APPLICATIONS\ \ \ 2020\ \ \ volume~14\ \ \ issue\ 1}
}%
 \def\rightfootline{\small{INFORMATIKA I EE PRIMENENIYA~---
INFORMATICS AND APPLICATIONS\ \ \ 2020\ \ \ volume~14\ \ \ issue\ 1
\hfill \textbf{\thepage}}}

\vspace*{2pt} 



\Abste{The work is devoted to the study of metric methods for analyzing 
objects with complex structure. It proposes to generalize the dynamic 
time warping method of two time series for the case of objects defined 
on two or more time axes. Such objects are matrices in the discrete 
representation. The DTW (Dynamic Time Warping) method of time series is generalized as 
a~method of matrices dynamic alignment. The paper proposes 
a~distance function resistant to monotonic nonlinear deformations of the 
Cartesian product of two time scales. The alignment path between objects is 
defined. An object is called a~matrix in which the rows and columns correspond 
to the axes of time. The properties of the proposed distance function 
are investigated. To illustrate the method, the problems of metric 
classification of objects are solved on model data and data from the 
MNIST dataset.}

\KWE{distance function; dynamic alignment; distance between matrices; 
nonlinear time warping; space--time series}



\DOI{10.14357/19922264200105} 

%\vspace*{-14pt}

\Ack
\noindent
This work was supported by the Russian Foundation for Basic
Research (projects 19-07-1155 and 19-07-00885). 
The paper contains results of the project Statistical 
methods of machine learning, which is carried out within the 
framework of the Program ``Center of Big Data Storage and Analysis'' 
of the National Technology Initiative Competence Center. 
It is supported by the Ministry of Science and Higher Education 
of the Russian Federation according to the agreement between the
 M.\,V.~Lomonosov Moscow State University and the Foundation 
 of project support of the National Technology Initiative from 11.12.2018, 
 No.\,13/1251/2018.
 


%\vspace*{6pt}

  \begin{multicols}{2}

\renewcommand{\bibname}{\protect\rmfamily References}
%\renewcommand{\bibname}{\large\protect\rm References}

{\small\frenchspacing
 {%\baselineskip=10.8pt
 \addcontentsline{toc}{section}{References}
 \begin{thebibliography}{99}

 \bibitem{0-1}   
\Aue{Hill, N.\,J., T.\,N.~Lal, M.~Schroder, T.~Hinterberger, B.~Wilhelm, 
F.~Nijboer, U.~Mochty, G.~Widman, C.~Elger, B.~Scholkopf, A.~Kubler, and 
N.~Birbaumer.} 2006. Classifying EEG and ECoG signals without subject 
training for fast BCI implementation: Comparison of nonparalyzed and completely 
paralyzed subjects. \textit{IEEE~T. Neur. Sys. 
Reh.} 14(2):183--186.

\bibitem{1-1}   
\Aue{Sakoe, H., and S.~Chiba.} 1971. A~dynamic programming approach 
to continuous speech recognition. \textit{7th 
 Congress (International) on Acoustics Proceedings}. 3:65--69.

\bibitem{2-1}    %2
\Aue{Aghabozorgi,~S., S.\,S.~Ali, and T.\,Y.~Wah.} 2015. 
Time-series clustering~--- a~decade review.  \textit{Inform. Syst.} 
53:16--38.

\bibitem{3-1}   %4 
\Aue{Warrenliao,~T.} 2005. Clustering of time series data~--- a~survey. 
\textit{Pattern Recogn.}
38(11):1857--1874.



\bibitem{4-1}    %5
\Aue{Hautamaki,~V., P.~Nykanen, and P.~Franti.} 2008. 
Time-series clustering by approximate prototypes. 
 \textit{19th  Conference (International) on Pattern Recognition Proceedings}. 
 D:1--4.

\bibitem{5-1}    %6
\Aue{Faloutsos,~C., M.~Ranganathan, and Y.~Manolopoulos.} 1994. 
Fast subsequence matching in time-series databases.  \textit{SIGMOD Rec}. 
23(2):419--429.

\bibitem{10-1}    %7
\Aue{Basalto, N., R.~Bellotti, F.\,D.~Carlo, P.~Facchi, and S.~Pascazio.} 
2007. Hausdorff clustering of financial time series. 
\textit{Physica~A} 379(2):635--644.

\bibitem{11-1}   %8
\Aue{Gorelick, L., M.~Blank, E.~Shechtman, M.~Irani, and R.~Basri.} 
2007. Actions as space-time shapes.
\textit{IEEE~T. Pattern Anal.} 29(12):2247--2253.

\bibitem{6-1}    %9
\Aue{Smyth, P.} 1997. 
Clustering sequences with hidden Markov models. \textit{Adv. Neural In.} 9:648--654.

\bibitem{7-1}    %10
\Aue{Banerjee,~A., and J.~Ghosh.} 2001. 
Clickstream clustering using weighted longest common subsequences.  
\textit{Workshop on Web Mining, SIAM Conference 
on Data Mining Proceedings.} 33--40.

\bibitem{8-1}    %11
\Aue{Aach, J., and G.\,M.~Church.} 2001. 
Aligning gene expression time series with time warping algorithms. 
\textit{Bioinformatics} 17(6):495--508.

\bibitem{9-1}   %12
\Aue{Yi, B.\,K., and C.~Faloutsos.} 2000. 
Fast time sequence indexing for arbitrary $\mathcal{L}_p$ norms. 
\textit{26th  Conference (International) 
on Very Large Data Bases Proceedings}. 385--394.

\bibitem{33-1}   %13 
\Aue{Goncharov,~A.\,V., and V.\,V.~Strijov.} 2018. 
Analysis of dissimilarity set between time series. 
\textit{Computational Mathematics Modeling } 29(3):359--366.



\bibitem{12-1}    %14
\Aue{Alon, J., V.~Athitsos, and S.~Sclaroff.} 2005.
 Online and offline character recognition using alignment to prototypes. 
 \textit{8th  Conference (International) on Document Analysis and Recognition}. 
 2:839--843.

\bibitem{15-1}    %15
\Aue{Goncharov, A.\,V.} Alignment of 
Ordered Set Cartesian Product mDTW. Software implementation of the algorithm. 
Available at: {\sf https://github.com/Intelligent-\linebreak 
Systems-Phystech/PhDThesis/tree/master/Goncharov\linebreak 2019/MatrixDTW/code} 
(accessed December~27, 2019).
\end{thebibliography}

 }
 }

\end{multicols}

%\vspace*{-7pt}

\hfill{\small\textit{Received April 24, 2019}}

%\pagebreak

%\vspace*{-22pt}



\Contr

\noindent
\textbf{Goncharov Alexey V.} (b.\ 1995)~--- 
PhD student, Moscow Institute of Physics and Technology, 
9~Institutskiy Per., Dolgoprudny, Moscow Region 141701, 
Russian Federation; \mbox{alex.goncharov@phystech.edu}

\vspace*{3pt}

\noindent
\textbf{Strijov Vadim V.} (b.\ 1967)~--- 
Doctor of Science in physics and mathematics, leading scientist, 
A.\,A.~Dorodnicyn Computing Centre, Federal Research Center 
``Computer Science and Control'' of the Russian Academy of Sciences, 
40~Vavilov Str., Moscow 119333, Russian Federation;
 professor, Moscow Institute of Physics and Technology, 
 9~Institutskiy Per., Dolgoprudny, Moscow Region 141701, Russian Federation; 
 \mbox{strijov@ccas.ru}
\label{end\stat}

\renewcommand{\bibname}{\protect\rm Литература}  %4


\def\stat{isachenko}

\def\tit{МЕТРИЧЕСКОЕ ОБУЧЕНИЕ В ЗАДАЧАХ МУЛЬТИКЛАССОВОЙ КЛАССИФИКАЦИИ 
ВРЕМЕННЫХ РЯДОВ$^*$}

\def\titkol{Метрическое обучение в~задачах мультиклассовой классификации 
временных рядов}

\def\aut{Р.\,В.~Исаченко$^1$, В.\,В.~Стрижов$^2$}

\def\autkol{Р.\,В.~Исаченко, В.\,В.~Стрижов}

\titel{\tit}{\aut}{\autkol}{\titkol}

\index{Исаченко Р.\,В.}
\index{Стрижов В.\,В.}
\index{Isachenko R.\,V.}
\index{Strijov V.\,V.}

{\renewcommand{\thefootnote}{\fnsymbol{footnote}} \footnotetext[1]
{Работа выполнена при финансовой поддержке РФФИ (проект 16-07-01158).}}


\renewcommand{\thefootnote}{\arabic{footnote}}
\footnotetext[1]{Московский физико-технический институт, isa-ro@yandex.ru}
\footnotetext[2]{Вычислительный центр им.\ А.\,А.~Дородницына 
Федерального исследовательского
центра <<Информатика и~управление>> Российской академии наук, strijov@ccas.ru}


\Abst{Работа посвящена построению модели многоклассовой классификации временн$\acute{\mbox{ы}}$х рядов.
    Предлагается выравнивать временн$\acute{\mbox{ы}}$е ряды относительно центроидов классов.
    Процедура нахождения центроидов и~выравнивания временн$\acute{\mbox{ы}}$х рядов осуществляется с~помощью алгоритма динамической трансформации 
    времени.
    Для повышения качества классификации в~данной работе используются методы метрического обучения.
    Метрическое обучение позволяет модифицировать расстояния между временн$\acute{\mbox{ы}}$ми рядами, сближая 
    временн$\acute{\mbox{ы}}$е ряды из одного класса и~отдаляя временн$\acute{\mbox{ы}}$е ряды из разных классов.
    Расстояние между временн$\acute{\mbox{ы}}$ми рядами измеряется с~помощью метрики Махаланобиса.
    Процедура метрического обучения состоит в~определении оптимальной матрицы трансформаций 
    в~метрике Махаланобиса.
    Для анализа качества построенного алгоритма проведен вычислительный эксперимент на синтетических и~реальных данных показаний с~акселерометра мобильного телефона.}


\KWE{классификация временных рядов; выравнивание; метрическое обучение; алгоритм LMNN}

\DOI{10.14357/19922264160205} 

%\vspace*{-4pt}

\vskip 10pt plus 9pt minus 6pt

\thispagestyle{headings}

\begin{multicols}{2}

\label{st\stat}

\section{Введение}

Решается задача мультиклассовой классификации временн$\acute{\mbox{ы}}$х 
рядов~\cite{popova2015multiclass, ignatov2015multiclass}.
Для решения этой задачи ранее использовались: метод опорных 
векторов~\cite{guler2007mccsvm, ubeyli2007mccsvm2}, нейронные 
сети~\cite{anand1995mccnn}, байесовский подход~\cite{kafai2012mccbn}.
В данной работе для классификации временн$\acute{\mbox{ы}}$х рядов используется идея 
ближайших соседей~\cite{chaovalitwongse2007knn}.

Для повышения качества классификации предлагается использовать методы 
метрического обучения~[8--10].
%\cite{bellet2013mlsurvey, yang2006mlsurvey2, wang2015mlsurvey3}.
Метрическое обучение позволяет модифицировать расстояния между временн$\acute{\mbox{ы}}$ми 
рядами с~помощью линейного преобразования признакового пространства объектов.
В~результате преобразования временн$\acute{\mbox{ы}}$е ряды одного класса оказываются ближе 
друг к~другу по выбранной метрике, а~временн$\acute{\mbox{ы}}$е ряды, принадлежащие разным классам, 
отдаляются друг от друга.
Методы метрического обучения применяются при ранжировании поисковой 
выдачи~\cite{mcfee2010mlranking}, идентификации лиц~\cite{guillaumin2009mlface}, 
распознавании рукописных цифр~\cite{weinberger2008mldigits}.
В~данной работе в~качестве алгоритма метрического обучения был выбран 
алгоритм LMNN (Large Margin Nearest Neighbor)~\cite{weinberger2005lmnn}.
Данный алгоритм основан на идеях метода $k$ ближайших соседей.
Алгоритм для каж\-до\-го объекта минимизирует расстояния до~$k$~ближайших соседей, 
принадлежащих тому же классу, и~штрафует объекты из других классов, попавшие 
на расстояние порядка расстояния до~$k$-го ближайшего соседа.

Алгоритм LMNN позволяет произвести отбор признаков.
С~помощью линейного преобразования алгоритм помещает объекты в~новое пространство.
Если размерность нового пространства меньше размерности исходного пространства, 
то происходит снижение размерности, т.\,е.\ отбор признаков.

Для вычисления расстояний между временн$\acute{\mbox{ы}}$ми рядами в~данной работе 
производится 
их выравнивание относительно центроидов классов 
методом динамической трансформации времени DTW 
(Dynamic Time Warping)~\cite{berndt1994dtw}.
Задача поиска оптимального центроида класса решается с~помощью метода 
выравненного взвешенного усреднения DBA (DTW Barycenter Averaging)~\cite{petitjean2011dba}.
Классификация, основанная на идее ближайших соседей, чувствительна 
к~изменению масштабов признаков.
Для повышения устойчивости классификации выравненные временн$\acute{\mbox{ы}}$е ряды были 
отнормированы.

Таким образом, полученная модель классификации представляет собой 
суперпозицию алгоритмов построения центроидов, выравнивания временн$\acute{\mbox{ы}}$х 
рядов относительно центроидов классов, метрического обучения и~классификации.

\renewcommand{\figurename}{\protect\bf Алгоритм}

\begin{figure*}[b]
\vspace*{9pt}
\hrule

\vspace*{-4pt}

\noindent
\Caption{Нахождение центроида $\mathrm{DBA}(\mathbf{X}_e, \mathrm{n}\_\mathrm{iter})$}
\label{DBA_pseudo}

\vspace*{6pt}

\hrule

\vspace*{6pt}

\noindent
\textbf{Вход:}\ $\mathbf{X}_e$~--- множество временн$\acute{\mbox{ы}}$х рядов, принадлежащих 
одному и~тому же классу, n\_iter~--- количество

\ \ \ \ \ \  итераций алгоритма.

\noindent
\textbf{Выход:}\ $\mathbf{c}$~--- центроид множества $\mathbf{X}_e$.

\ 1:\ \ задать начальное приближение приближение центроида $\mathbf{c}$;

\ 2:\ \ \textbf{для}\ $i = 1, \dots, \text{n\_iter}$

\ 3:\ \ \ \ \textbf{для} $\mathbf{x} \in \mathbf{X}_e$

\ 4:\ \ \ \ \ \ вычислить выравнивающий путь между $\mathbf{c}$ и~$\mathbf{x}$\newline
\hphantom{\ 4:}\ \ \ \ \ \ $\mathrm{alignment}(\mathbf{x}) := \mathrm{DTWalignment}(\mathbf{c}, \mathbf{x})$;

\ 5:\ \ \ \ объединить поэлементно множества индексов для каждого отсчета времени\newline
\hphantom{\ 4:}\ \ \ \ $\mathrm{alignment} := \bigcup_{\mathbf{x} \in \mathbf{X}_e} \text{alignment}(\mathbf{x})$;

\ 6:\ \ \ \ $\mathbf{c} = \text{mean}(\text{alignment})$

\


ПРОЦЕДУРА $\mathrm{DTWalignment}(\mathbf{c}, \mathbf{x})$

\noindent
\textbf{Вход:}\ $\mathbf{c}, \mathbf{x}$~--- временн$\acute{\mbox{ы}}$е ряды.

\noindent
\textbf{Выход:}\ alignment~--- выравнивающий путь.~// {каждый индекс 
временного ряда~$\mathbf{x}$ поставлен в~однозначное\newline
\hphantom{\ 1:}\ \ соответствие индексу временного ряда~$\mathbf{c}$}

\ 1:\ \  {построить $n \times n$-матрицу деформаций DTW}\newline
\hphantom{\ 1:}\ \ $\mathrm{cost} := \mathrm{DTW}(\mathbf{c}, \mathbf{x})$;

\ 2:\ \ {вычислить выравнивающий путь по матрице деформаций}\newline
\hphantom{\ 1:}\ \ $\mathrm{alignment} := \mathrm{DTWpath}(\mathrm{cost})$;

\vspace*{6pt}

\hrule
\end{figure*}

В данной работе вычислительный эксперимент проводился на синтетических временн$\acute{\mbox{ы}}$х 
рядах, представляющих аналитические функции, и~реальных данных показаний акселерометра 
мобильного телефона.
Цель эксперимента~--- определить вид активности человека по форме сигнала акселерометра.
Получена оценка качества работы построенного алгоритма и~проведен анализ его свойств.

\section{Постановка задачи}

Пусть объект $\mathbf{x}_i \hm\in \mathbb{R}^n$~--- временн$\acute{\mbox{о}}$й ряд, последовательность 
измерений некоторой исследуемой величины в~различные моменты времени.
Пусть $\mathbf{X}$~--- множество всех временн$\acute{\mbox{ы}}$х рядов фиксированной длины~$n$; 
$Y \hm= \{1, \dots, K\}$~--- множество меток классов.
Пусть задана выборка $\mathfrak{D} \hm= \{(\mathbf{x}_i, y_i)\}_{i=1}^\ell$~--- 
множество объектов с~известными метками классов $y_i \hm\in Y$.

Требуется построить точную, простую, устойчивую модель классификации
$$
    a: \mathbf{X} \to Y\,.
$$
Данную модель представим в~виде суперпозиции
\begin{equation*}
%\label{eq:classifiers}
a(\mathbf{x}) = b \circ \mathbf{f} \circ G(\mathbf{x}, \{\mathbf{c}_e\}_{e = 1} ^ K)\,,
\end{equation*}
 где $G$~--- процедура выравнивания временн$\acute{\mbox{ы}}$х рядов относительно 
 центроидов классов~$\{\mathbf{c}_e\}_{e = 1} ^ K$; $\mathbf{f}$~--- 
 алгоритм метрического обучения; $b$~--- алгоритм многоклассовой классификации.

\subsection{Выравнивание временн$\acute{\mbox{ы}}$х рядов}

Для повышения качества и~устойчивости алгоритма классификации предлагается 
провести выравнивание временн$\acute{\mbox{ы}}$х рядов каждого класса относительно центроида.

Пусть $\mathbf{X}_e$ --- множество объектов обучающей выборки $\mathfrak{D}$, 
принадлежащих одному классу $e \hm\in \{1, \dots, K\}$.
Центроидом множества объектов $\mathbf{X}_e \hm= \{\mathbf{x}_i|y_i=e\}_{i=1}^\ell$ 
по расстоянию~$\rho$ назовем вектор $\mathbf{c}_e \hm\in \mathbb{R}^n$ такой, что
\begin{equation}
\label{centroid_task}
    \mathbf{c}_e = \mathop{\mathrm{argmin}}\limits_{{\mathbf{c} 
    \in \mathbb{R}^n}}\sum\limits_{\mathbf{x}_i \in \mathbf{X}_e}
    {\rho(\mathbf{x}_i ,\mathbf{c})}\,.
\end{equation}

Для нахождения центроида предлагается в~качестве расстояния между 
временн$\acute{\mbox{ы}}$ми рядами использовать путь наименьшей стоимости~\cite{goncharov2015cost}, 
найден\-ный методом динамической трансформации времени.
Псевдокод решения оптимизационной задачи~(\ref{centroid_task}) приведен 
в~алгоритме~1.



Общая процедура выравнивания имеет сле\-ду\-ющий вид:
\begin{enumerate}[(1)]
    \item
    построить множество центроидов классов $\{\mathbf{c}_e\}_{e = 1}^K$;
    \item
    по множеству центроидов найти пути наименьшей стоимости между каждым
    временн$\acute{\mbox{ы}}$м рядом~$\mathbf{x}_i$ и~центроидом его класса~$\mathbf{c}_{y_i}$;
    \item
    по каждому пути восстановить выравненный временной ряд;
    \item
        привести множества выравненных временн$\acute{\mbox{ы}}$х рядов к~нулевому среднему и~нормировать на дисперсию.
\end{enumerate}

Результатом выравнивания должно стать множество выравненных временн$\acute{\mbox{ы}}$х рядов.

\subsection{Метрическое обучение}

Введем на множестве выравненных временн$\acute{\mbox{ы}}$х рядов расстояние Махаланобиса
$$
    d_\mathbf{A} (\mathbf{x}_i, \mathbf{x}_j) = \sqrt{(\mathbf{x}_i - \mathbf{x}_j)^\mathsf{T} \mathbf{A} (\mathbf{x}_i - \mathbf{x}_j)}\,,
$$
где матрица трансформаций $\mathbf{A} \hm\in \mathbb{R}^{n \times n}$ 
является симметричной и~неотрицательно определенной ($\mathbf{A}^\mathsf{T} \hm= 
\mathbf{A}$, $\mathbf{A} \succeq 0$).
Представим матрицу~$\mathbf{A}$ в~виде разложения 
$\mathbf{A} \hm= \mathbf{L}^\mathsf{T}  \mathbf{L}$.
Матрица $\mathbf{L}\hm \in \mathbb{R}^{p \times n}$~--- мат\-ри\-ца линейного преобразования, 
где~$p$ задает раз\-мер\-ность преобразованного пространства. Если па\-ра\-метр $p\hm < n$, 
то происходит снижение размерности признакового пространства.

Расстояние $d_\mathbf{A} (\mathbf{x}_i, \mathbf{x}_j)$ есть евклидово 
рас\-сто\-яние между $\mathbf{Lx}_i$ и~$\mathbf{Lx}_j$:
\begin{multline*}
    d_\mathbf{A} (\mathbf{x}_i, \mathbf{x}_j) = \sqrt{(\mathbf{x}_i - 
    \mathbf{x}_j)^\mathsf{T} \mathbf{L}^\mathsf{T} \mathbf{L} (\mathbf{x}_i - 
    \mathbf{x}_j)} = {}\\
    {}=\sqrt{(\mathbf{L} (\mathbf{x}_i - \mathbf{x}_j))^\mathsf{T} 
    (\mathbf{L} (\mathbf{x}_i - \mathbf{x}_j))} =
     \|\mathbf{L} (\mathbf{x}_i - \mathbf{x}_j)\|_2\,.
\end{multline*}

В качестве алгоритма метрического обучения в~данной работе был выбран 
алгоритм LMNN. \mbox{Данный} алгоритм сочетает в~себе идеи метода~$k$~ближайших соседей. 
Первая идея заключается в~минимизации расстояний между~$k$~ближайшими объектами, 
находящимися в~одном клас\-се. Запишем функционал качества в~виде
$$
    Q_1(\mathbf{L}) = \sum\limits_{j \rightsquigarrow i} \|\mathbf{L}(\mathbf{x}_i - 
    \mathbf{x}_j)\|^2 \rightarrow \min\limits_{\mathbf{L}}\,,
$$
где $j \rightsquigarrow i$ означает, что $\mathbf{x}_j$ является одним из~$k$~ближайших соседей для $\mathbf{x}_i$.
Вторая идея состоит в~максимизации расстояния между каждым объектом и~его 
объ\-ек\-та\-ми-на\-ру\-ши\-те\-ля\-ми. 
Объек\-том-на\-ру\-ши\-те\-лем для $\mathbf{x}_i$ назовем объект~$\mathbf{x}_l$ 
такой, что
\begin{equation}
\label{impostor}
    \|\mathbf{L}(\mathbf{x}_i - \mathbf{x}_l)\|^2 \leq 
    \|\mathbf{L}(\mathbf{x}_i - \mathbf{x}_j)\|^2 + 1,  
    \mbox{ где } j \rightsquigarrow i.
\end{equation}
Таким образом, необходимо минимизировать следующий функционал:
\begin{multline*}
    Q_2(\mathbf{L}) = \sum\limits_{j \rightsquigarrow i} 
    \sum\limits_l(1 - y_{il})
    \bigl[1 + \|\mathbf{L}(\mathbf{x}_i - \mathbf{x}_j)\|^2 -{}\\
    {}- \|\mathbf{L}
    (\mathbf{x}_i - \mathbf{x}_l)\|^2\bigr]_+ 
    \rightarrow \min\limits_{\mathbf{L}}\,,
\end{multline*}
где $y_{il} = 1$, если $y_i \hm= y_l$, и~$y_{il} \hm= 0$ в~противном случае.
Положительная срезка позволяет штрафовать только те объекты, которые 
удовлетворяют условию~(\ref{impostor}).

Задача метрического обучения состоит в~на\-хож\-де\-нии линейного 
преобразования $\mathbf{f}(\mathbf{x}) \hm= \mathbf{Lx}$, т.\,е.\ 
нахождении матрицы~$\mathbf{L}$ в~виде решения оптимизационной задачи
\begin{equation}
\label{Qmin}
    Q(\mathbf{L}) = \mu Q_1(\mathbf{L}) + (1 - \mu) Q_2(\mathbf{L}) 
    \rightarrow \min\limits_{\mathbf{L}}\,,
\end{equation}
где $\mu \in (0, 1)$~--- весовой параметр, определяющий вклад каждого из функционалов.
Задача~(\ref{Qmin}) представляет собой задачу полуопределенного 
программирования~\cite{vandenberghe1996semidefinite} и~может быть решена 
сущест\-ву\-ющи\-ми оптимизационными пакетами.

\subsection{Классификация временн$\acute{\mbox{ы}}$х рядов}

Пусть $\mathbf{x} \in \mathbf{X}$~--- неразмеченный временной ряд. 
Выравниваем временной ряд~$\mathbf{x}$ относительно всех центроидов классов
$    \mathbf{\hat{x}}_e = G(\mathbf{x}, \mathbf{c}_e)\,,  \mbox{ где } 
    e \hm= \{1, \dots, K\}.$

Отнесем временной ряд к~классу, для которого минимально расстояние 
до соответствующего центроида. В~качестве расстояния используем обучен\-ную метрику 
Махаланобиса с~фиксированной мат\-ри\-цей~$\mathbf{A}$:
$$
    \hat{y} = \mathop{\mathrm{argmin}}\limits_{e \in \{1, \dots, K\}}
    d_\mathbf{A}\left(\mathbf{\hat{x}}_e, \mathbf{c}_e\right)\,.
$$
После нахождения оптимальных центроидов классов и~нахождения 
оптимальной матрицы трансформаций процедура классификации заключается 
в~измерении расстояния между найденными центроидами и~новыми неразмеченными объектами.

Для оценки качества работы алгоритма будем вычислять ошибку классификации 
как долю неправильно классифицированных объектов тестовой выборки~$\mathfrak{U}$:
$$
    \mathrm{error} = \fr{1}{|\mathfrak{U}|} 
    \sum\limits_{i = 1} ^ {|\mathfrak{U}|}\left[a\left(\mathbf{x}_i\right) \ne y_i\right].
$$

\section{Вычислительный эксперимент}

Цель вычислительного эксперимента~--- проверить работоспособность предложенного 
подхода.
Предполагается, что построенный алгоритм мультиклассовой классификации 
способен определить тип активности человека по форме сигнала акселерометра 
мобильного телефона.

\renewcommand{\figurename}{\protect\bf Рис.}
\setcounter{figure}{0}


\begin{figure*} %fig1
\vspace*{1pt}
 \begin{center}  
\mbox{%
 \epsfxsize=79.311mm
 \epsfbox{isa-1.eps}
 }
\end{center} 
\vspace*{-15pt}
\Caption{Центроиды синтетических временн$\acute{\mbox{ы}}$х рядов двух
    классов: \textit{1}~--- $\sin (x\hm+b)$; \textit{2}~--- пилообразные функции
    с~различными сдвигами по временн$\acute{\mbox{о}}$й шкале}
%\end{figure*}
%\begin{figure*} %fig2
\vspace*{1pt}
 \begin{center}
 \mbox{%
 \epsfxsize=163.735mm
 \epsfbox{isa-2.eps}
 }
 \end{center}
 \vspace*{-15pt}
    \Caption{Центроиды временн$\acute{\mbox{ы}}$х рядов акселерометра: (\textit{а})~ходьба;
(\textit{б})~подъем; (\textit{в})~спуск; (\textit{г})~сидение; 
(\textit{д})~стояние; (\textit{е})~лежание}
\label{centroids_real}
\vspace*{-0.15556pt}
\end{figure*}

%\end{multicols}


%\begin{multicols}{2}

Для проведения базового вычислительного эксперимента были подготовлены 
синтетические временн$\acute{\mbox{ы}}$е ряды, принадлежащие двум классам~\cite{Isachenko2015code}.

Первый класс~--- синусы вида $\sin(x \hm+ b)$, где параметр~$b$ определяет сдвиг 
каждого временн$\acute{\mbox{о}}$го ряда.

Второй класс~--- пилообразные функции с~различными сдвигами по временн$\acute{\mbox{о}}$й шкале.
На каж\-дый временной ряд был наложен нормальный шум.
Число временн$\acute{\mbox{ы}}$х рядов каждого клас\-са\;=\;60.
Длина каждого временн$\acute{\mbox{о}}$го ряда $n \hm= 50$.
%
Построенные центроиды классов проиллюстрированы на рис.~1.
Видно, что процедура корректно определяет сдвиги временн$\acute{\mbox{ы}}$х 
рядов.


Чтобы убедиться в~целесообразности применения метрического обучения, данные
временн$\acute{\mbox{ы}}$е\linebreak
 ряды классифицировались в~пространстве с~евклидовой метрикой 
и~в~пространстве с~мет\-ри\-кой Махаланобиса.
Число ближайших соседей $k \hm= 5$; размерность преобразованного пространства 
$p \hm= 40$. Полученные ошибки классификации составили:
\begin{itemize}
\item евклидова метрика --- $27\%$;
\item
метрика Махаланобиса --- $6\%$.
\end{itemize}

\end{multicols}

\begin{figure*}[b] %fig3
\vspace*{-12pt}
 \begin{center}
 \mbox{%
 \epsfxsize=164.279mm
 \epsfbox{isa-3.eps}
 }
 \end{center}
 \vspace*{-9pt}
    \Caption{Временн$\acute{\mbox{ы}}$е ряды акселерометра:
    (\textit{а})~ходьба;
(\textit{б})~подъем; (\textit{в})~спуск; (\textit{г})~сидение; 
(\textit{д})~стояние; (\textit{е})~лежание}
\label{raw_ts}
%\vspace*{-12pt}
\end{figure*}

\begin{multicols}{2}




Реальные данные~\cite{UCI_HarDataset} представляли собой вре\-мен\-н$\acute{\mbox{ы}}$е 
ряды  акселерометра мобильного телефона.
Каж\-дый из шести классов соответствовал определенной физической активности испытуемых.
Для проведения вычислительного эксперимента было выбрано по~200~объектов каж\-до\-го класса.
Длина каж\-до\-го временн$\acute{\mbox{о}}$го ряда равнялась $n \hm= 128$ отсчетам времени.



Построенные центроиды классов изображены на рис.~\ref{centroids_real}.
Найденные центроиды обладают периодичностью, свойственной временн$\acute{\mbox{ы}}$м 
рядам 
по-\linebreak %\vspace*{-12pt}

\columnbreak

\noindent
казаний активности человека.
На рис.~\ref{raw_ts} показаны примеры временн$\acute{\mbox{ы}}$х рядов каждого класса. 
Эти же временн$\acute{\mbox{ы}}$е ряды после процедуры выравнивания относительно построенных 
центроидов изображены на рис.~\ref{aligned_ts}.




Ошибка классификации без использования мет\-ри\-ческого обучения составила~37,5\%.
Алгоритм LMNN позволяет настроить параметры: число ближайших соседей~$k$,
размерность преобразованного евклидова пространства~$p$.
Для выбора оптимальных параметров воспользуемся процедурой кросс-\linebreak %\vspace*{-12pt}

\end{multicols}

\begin{figure*}[b] %fig4
\vspace*{-24pt}
 \begin{center}
 \mbox{%
 \epsfxsize=164.306mm
 \epsfbox{isa-4.eps}
 }
 \end{center}
 \vspace*{-9pt}
    \Caption{Выравненные временн$\acute{\mbox{ы}}$е ряды акселерометра:
    (\textit{а})~ходьба;
(\textit{б})~подъем; (\textit{в})~спуск; (\textit{г})~сидение; 
(\textit{д})~стояние; (\textit{е})~лежание}
\label{aligned_ts}
%\vspace*{-12pt}
\end{figure*}

\pagebreak



\begin{figure*} %fig5
\vspace*{1pt}
 \begin{center}
 \mbox{%
 \epsfxsize=109.202mm
 \epsfbox{isa-5.eps}
 }
 \end{center}
 \vspace*{-9pt}
    \Caption{Ошибка классификации в~зависимости от параметров}
\label{heat_map}
\vspace*{9pt}
\end{figure*}

%\pagebreak

\begin{multicols}{2}

 

\noindent
 про\-вер\-ки.
На рис.~\ref{heat_map} показана ошибка классификации алгоритма 
в~зависимости от его параметров.
На данной выборке алгоритм LMNN оказывается слабо чувствителен 
к~числу ближайших соседей,
и~при уменьшении размерности пространства объектов ошибка классификации растет.


Настроим алгоритм LMNN со следующими параметрами: число 
ближайших соседей $k \hm= 30$, размерность
выходного пространства $p \hm= 128$.
Ошибка
 классификации составила~17,25\%, что вдвое меньше ошибки классификации 
с~использованием евклидовой метрики.


В табл.~1 представлены матрицы несоответствий 
результатов классификации при использовании
евклидовой метрики и~метрики Махаланобиса.
Столбцы соответствуют истинным меткам классов объектов, 
строки~--- предсказанным меткам.
Диагональное преобладание матрицы несоответствий указывает на 
высокую предсказательную способность алгоритма.

В табл.~\ref{improvement} продемонстрировано увеличение точности классификации 
при использовании в~качестве меры расстояния метрики Махаланобиса.
Пересече\-ние $i$-го столбца и~$j$-й строки отвечает\linebreak изменению доли объектов класса~$i$, 
отнесенных к~классу~$j$. Положительное суммарное значе\-ние диагональных элементов 
таблицы соответствует увеличению качества классификации. Значительное улучшение 
предсказания происходит при классификации первых трех классов.
Данные клас-\linebreak

\noindent
 \begin{center}  %tabl
 \vspace*{-12pt}
{{\tablename~1}\ \ \small{Матрицы несоответствий}}

{\small 
\tabcolsep=4pt
\begin{tabular}{|c|c|c|c|c|c|c|}
\multicolumn{7}{c}{\ }\\[-6pt]
    \hline
 Предсказанные& \multicolumn{6}{c|}{Истинные метки классов}\\
  \cline{2-7}
   метки& 1  & 2  & 3  & 4  & 5  & 6  \\ 
   \hline
   \multicolumn{7}{|c|}{Евклидова метрика}\\
   \hline
1  & 80\hphantom{9} & \hphantom{9}0  & \hphantom{9}5  & \hphantom{9}0  & \hphantom{9}0  & \hphantom{9}0   \\
%\hline
2  & 4  & 56 & 33 & \hphantom{9}0  & \hphantom{9}0  & \hphantom{9}0  \\
%\hline
3  & 5  & \hphantom{9}5  & 86 & \hphantom{9}0  & \hphantom{9}0  & \hphantom{9}0\\
%\hline
4  & 7  & \hphantom{9}8  & \hphantom{9}5  & 168\hphantom{9}& \hphantom{9}4  & 21 \\
%\hline
5  & 51\hphantom{9} & 61 & 57 & 12 & 192\hphantom{9}& 11 \\
%\hline
6  & 53\hphantom{9} & 70 & 14 & 20 & \hphantom{9}2  & 168\hphantom{9}\\
\hline
\multicolumn{7}{|c|}{Метрика Махаланобиса}\\
\hline
1& 151\hphantom{9} & 12 & 13    & \hphantom{9}0     & \hphantom{9}0    & \hphantom{9}0 \\ 
%\hline
  2  & 10  & 142\hphantom{9}& 14    & \hphantom{9}0     & \hphantom{9}0    & \hphantom{9}0 \\ 
%  \hline
  3  & \hphantom{9}9   & 10 & 171\hphantom{9}   & \hphantom{9}0     & \hphantom{9}0    & \hphantom{9}0 \\
%  \hline
   4  & 10  & \hphantom{9}7  & \hphantom{9}0     & 173\hphantom{9}   & \hphantom{9}9    & 21\\  
%   \hline
     5  & \hphantom{9}2   & 11 & \hphantom{9}0     & 12    & 186\hphantom{9}  & \hphantom{9}9 \\ 
%     \hline
       6  & 18  & 18 & \hphantom{9}2     & 15    & \hphantom{9}5    & 170\hphantom{9}\\ 
\hline
\end{tabular}}
\end{center} 

 \vspace*{9pt}
 
 \noindent
 сы соответствуют следующим видам физической активности: ходьба, 
подъем, спуск.




\addtocounter{table}{1}


\begin{table*}\small
\begin{center}
\parbox{348pt}{\Caption{Увеличение точности классификации при использовании адекватной оценки матрицы трансформаций}
\label{improvement}

}

\begin{tabular}{|c|r|r|c|c|c|c|}
\multicolumn{7}{c}{\ }\\[-4pt]
\hline
\multicolumn{1}{|c|}{Предсказанные}& \multicolumn{6}{c|}{Истинные метки классов}       \\ 
\cline{2-7}
\multicolumn{1}{|c|}{метки}    & \multicolumn{1}{c|}{1} & \multicolumn{1}{c|}{2} & 3 & 4 & 5 & 6\\ 
    \hline
1 & $\mathbf{0{,}355}$& 0,06\hphantom{9,} & \hphantom{$-$}0,04\hphantom{9} & 0 & 0\hphantom{,999}& 0\hphantom{9}     \\ 
%\hline
2 & 0,03\hphantom{9,}   & $\mathbf{0{,}43}$\hphantom{9}   & $-0{,}095$ & 0& 0\hphantom{,999} & 0\hphantom{9} \\ 
%\hline
3 & 0,02\hphantom{9,}   & 0,025\hphantom{,}  & \hphantom{9,}$\mathbf{0{,}425}$  & 0  & 0\hphantom{,999} & 0\hphantom{9}\\ 
%\hline
4 & 0,015\hphantom{,}  & $-0{,}005$\hphantom{,} & $-0{,}025$ & \hphantom{9.9.9}$\mathbf{0{,}025}$ & 0,025 & 0\hphantom{9}\\ 
%\hline
5 & $-0{,}245$\hphantom{,} & $-0{,}25$\hphantom{9,}  & $-0{,}28$\hphantom{9}  & 0 & $\mathbf{-0{,}03}$\hphantom{99} & $-0{,}01$ \\ 
%\hline
6 & $-0{,}175$\hphantom{,} & $-0{,}26$\hphantom{9,}  & $-0{,}06$\hphantom{9}  & \hphantom{$-$,}$-0{,}025$ & 0,005 & $\mathbf{-0{,}01}$ \\ 
\hline
\end{tabular}
\end{center}
\end{table*}

\vspace*{-6pt}

\section{Заключение}

В данной работе предложен новый подход к~решению задачи многоклассовой 
классификации временн$\acute{\mbox{ы}}$х рядов.
Сравнивались результаты классификации множества временн$\acute{\mbox{ы}}$х рядов, 
основанных на измерении расстояний
с помощью евклидовой метрики и~обученной метрики Махаланобиса.
Проведен вычислительный эксперимент на реальных данных показаний 
акселерометра мобильного телефона.
Построенная модель классификации показала высокое качество распознавания 
активности человека по форме сигнала акселерометра.


{\small\frenchspacing
 {%\baselineskip=10.8pt
 \addcontentsline{toc}{section}{References}
 \begin{thebibliography}{99}

\bibitem{popova2015multiclass}
\Au{Попова~М.\,С., Стрижов~В.\,В.}
Выбор оптимальной модели классификации физической активности по
измерениям акселерометра~// Информатика и~её применения,
2015. Т.~9. Вып.~1. С.~79--89.

\bibitem{ignatov2015multiclass}
\Au{Ignatov~A.\,D., Strijov~V.\,V.}
Human activity recognition using quasiperiodic time series collected
  from a~single tri-axial accelerometer~// Multimed. Tools  Appl., 2015. 14~p.
  doi: 10.1007/s11042-015-26430.

\bibitem{guler2007mccsvm}
\Au{G$\ddot{\mbox{u}}$ler~I., $\ddot{\mbox{\!U}}$beyli~E.~D.}
Multiclass support vector machines for eeg-signals classification~//
IEEE Trans. Inf. Technol. Biomedicine, 2007. Vol.~11. No.\,2. P.~117--126.

\bibitem{ubeyli2007mccsvm2}
\Au{$\ddot{\mbox{U}}$beyli~E.\,D.}
Ecg beats classification using multiclass support vector machines
  with error correcting output codes~// Digit. Signal Process., 2007. Vol.~17. No.\,3. P.~675--684.

\bibitem{anand1995mccnn}
\Au{Anand~R., Mehrotra~K., Mohan~C.~K., Ranka~S.}
Efficient classification for multiclass problems using modular neural
  networks~// IEEE Trans. Neural Networ., 1995. Vol.~6. No.\,1. P.~117--124.

\bibitem{kafai2012mccbn}
\Au{Kafai~M., Bhanu~B.}
Dynamic bayesian networks for vehicle classification in video~// IEEE Trans. 
Ind. Inform., 2012. Vol.~8. No.\,1. P.~100--109.

\bibitem{chaovalitwongse2007knn}
\Au{Chaovalitwongse~W.\,A., Fan~Y.-J., Sachdeo~R.\,C.}
On the time series $k$-nearest neighbor classification of abnormal
  brain activity~// IEEE Trans. Syst. Man  Cy. A,
2007. Vol.~37. No.\,6. P.~1005--1016.

\bibitem{yang2006mlsurvey2} %8
\Au{Yang~L., Jin~R.}
Distance metric learning: A~comprehensive survey.~--- 
Michigan State University, 2006. Vol.~2. 51~p.

\bibitem{bellet2013mlsurvey} %9
\Au{Bellet~A., Habrard~A., Sebban~M.}
A~survey on metric learning for feature vectors and structured data.
arXiv preprint arXiv:1306.6709, 2013.



\bibitem{wang2015mlsurvey3}
\Au{Wang~F., Sun~J.}
Survey on distance metric learning and dimensionality reduction in data mining~//
Data Min. Knowl. Disc., 2015. Vol.~29. No.\,2. P.~534--564.

\bibitem{mcfee2010mlranking}
\Au{McFee~B., Lanckriet~G.\,R.}
Metric learning to rank~// 27th  Conference (International) on Machine
  Learning Proceedings, 2010. P.~775--782.

\bibitem{guillaumin2009mlface}
\Au{Guillaumin~M., Verbeek~J., Schmid~C.}
Is that you? Metric learning approaches for face identification~// 
IEEE 12th  Conference (International) on Computer Vision Proceedings.~--- IEEE, 2009. P.~498--505.

\bibitem{weinberger2008mldigits}
\Au{Weinberger~K.\,Q., Saul~L.\,K.}
Fast solvers and efficient implementations for distance metric
  learning~// 25th  Conference (International) on Machine
Learning Proceedings, 2008. P.~1160--1167.

\bibitem{weinberger2005lmnn}
\Au{Weinberger~K.\,Q., Blitzer~J., Saul~L.\,K.}
Distance metric learning for large margin nearest neighbor
  classification~// Advances in
  Neural Information Processing Systems.~--- Cambridge, MA,
  USA: MIT Press, 2006. P.~1473--1480.

\bibitem{berndt1994dtw}
\Au{Berndt~D.\,J., Clifford~J.}
Using dynamic time warping to find patterns in time series~// KDD Workshop, 
1994. Vol.~10. No.\,16. P.~359--370.

\bibitem{petitjean2011dba}
\Au{Petitjean~F., Ketterlin~A., \mbox{Gan\!{\!\ptb{\c{c}}}arski}~P.}
A~global averaging method for dynamic time warping, with applications
  to clustering~// Pattern Recogn., 2011. Vol.~44. No.\,3. P.~678--693.

\bibitem{goncharov2015cost}
\Au{Гончаров~А.\,В., Попова~М.\,С., Стрижов~В.\,В.}
Метрическая классификация временных рядов с~выравниванием
  относительно центроидов классов~// Системы 
  и~средства информатики, 2015. Т.~25. №\,4. С.~52--64.

\bibitem{vandenberghe1996semidefinite}
\Au{Vandenberghe~L., Boyd~S.}
Semidefinite programming // SIAM Rev., 1996. Vol.~38. No.\,1. P.~49--95.

\bibitem{Isachenko2015code}
\Au{Исаченко~Р.\,В.}
Реализация алгоритма классификации временных рядов~// Sourceforge.net, 2015.
{\sf http://sourceforge.net/p/mlalgorithms/code/HEAD/\linebreak tree/Group274/Isachenko2015TimeSeries/code}.


\bibitem{UCI_HarDataset}
UCI repository. Human activity recognition using smartphones dataset.
{\sf https://archive.ics.uci.edu/\linebreak ml/datasets/Human+Activity+Recognition+Using+\linebreak Smartphones}.
\end{thebibliography}

 }
 }

\end{multicols}

\vspace*{-3pt}

\hfill{\small\textit{Поступила в~редакцию 18.03.16}}

%\vspace*{8pt}

\newpage

\vspace*{-24pt}

%\hrule

%\vspace*{2pt}

%\hrule

%\vspace*{8pt}



\def\tit{METRIC LEARNING IN~MULTICLASS TIME SERIES CLASSIFICATION PROBLEM}

\def\titkol{Metric learning in multiclass time series classification problem}

\def\aut{R.\,V.~Isachenko$^1$ and V.\,V.~Strijov$^2$}

\def\autkol{R.\,V.~Isachenko and V.\,V.~Strijov}

\titel{\tit}{\aut}{\autkol}{\titkol}

\vspace*{-9pt}

\noindent
$^1$Moscow Institute of Physics and Technology, 
 9~Institutskiy Institutskiy Per., Dolgoprudny, Moscow Region\linebreak
 $\hphantom{^1}$141700,  Russian Federation

\noindent
$^2$A.\,A.~Dorodnicyn Computing Centre, Federal Research Center 
``Computer Science and Control'' of the Russian\linebreak
  $\hphantom{^1}$Academy of Sciences, 
40~Vavilov Str., Moscow 119333, Russian Federation
\def\leftfootline{\small{\textbf{\thepage}
\hfill INFORMATIKA I EE PRIMENENIYA~--- INFORMATICS AND
APPLICATIONS\ \ \ 2016\ \ \ volume~10\ \ \ issue\ 2}
}%
 \def\rightfootline{\small{INFORMATIKA I EE PRIMENENIYA~---
INFORMATICS AND APPLICATIONS\ \ \ 2016\ \ \ volume~10\ \ \ issue\ 2
\hfill \textbf{\thepage}}}

\vspace*{3pt}




\Abste{This paper is devoted to the problem of multiclass time series classification.
    It is proposed to align time series in relation to class centroids.
    Building of centroids and alignment of time series is carried out by 
    the dynamic time warping algorithm.
    The accuracy of classification depends significantly on the metric used to compute distances between time series.
   The distance metric learning approach is used to improve classification accuracy.
    The metric learning procedure modifies distances between objects to make objects from the same cluster closer
    and from the different clusters more distant.
    The distance between time series is measured by the Mahalanobis metric.
    The distance metric learning procedure finds the optimal transformation matrix for 
    the Mahalanobis metric.
    To calculate quality of classification,
    a~computational experiment on synthetic data and
    real data of human activity recognition was carried out.}

\KWE{time series classification; time series alignment; distance metric learning; LMNN algorithm}

\DOI{10.14357/19922264160205}

%\vspace*{-12pt}

\Ack
\noindent
The work was financially supported by the Russian Foundation
for Basic Research (project 16-07-01158).


%\vspace*{3pt}

  \begin{multicols}{2}

\renewcommand{\bibname}{\protect\rmfamily References}
%\renewcommand{\bibname}{\large\protect\rm References}

{\small\frenchspacing
 {%\baselineskip=10.8pt
 \addcontentsline{toc}{section}{References}
 \begin{thebibliography}{99}

\bibitem{1-is}
\Aue{Popova, M.\,S., and V.\,V.~Strijov}.
2015.
Vybor optimal'noy modeli klassifikatsii fizicheskoy aktivnosti po izmereniyam
  akselerometra [Selection of optimal physical activity
  classification model using measurements of accelerometer].
\textit{Informatika i ee primeneniya}~--- \textit{Inform. Appl}.
9(1):79--89.

\bibitem{2-is}
\Aue{Ignatov, A.\,D., and V.\,V.~Strijov}.
2015.
Human activity recognition using quasiperiodic time series collected from 
a~single tri-axial accelerometer.
\textit{Multimed. Tools  Appl}. 14~p.
  doi: 10.1007/s11042-015-26430.

\bibitem{3-is}
\Aue{G$\ddot{\mbox{u}}$ler, I., and E.\,D.~$\ddot{\mbox{U}}$beyli}.
2007.
Multiclass support vector machines for eeg-signals classification.
\textit{IEEE Trans. Inf. Technol. Biomedicine}
11(2):117--126.

\bibitem{4-is}
\Aue{$\ddot{\mbox{U}}$beyli, E.\,D.}
2007.
Ecg beats classification using multiclass support vector machines with error
  correcting output codes.
\textit{Digit. Signal Process.} 17(3):675--684.

\bibitem{5-is}
\Aue{Anand, R., K.~Mehrotra, C.~K. Mohan,  and S.~Ranka}.
1995.
Efficient classification for multiclass problems using modular neural networks.
\textit{IEEE Trans. Neural Networ.} 6(1):117--124.

\bibitem{6-is}
\Aue{Kafai, M., and B.~Bhanu.}
2012. Dynamic bayesian networks for vehicle classification in video.
\textit{IEEE Trans. Ind. Inform.} 8(1):100--109.

\bibitem{7-is}
\Aue{Chaovalitwongse, W.\,A., Y.-J.~Fan,  and R.\,C.~Sachdeo.}
2007. On the time series $k$-nearest neighbor classification of abnormal brain
  activity.
\textit{IEEE Trans. Syst. Man Cy. A}
37(6):1005--1016.



\bibitem{9-is} %8
\Aue{Yang, L., and R.~Jin.}
2006.
\textit{Distance metric learning: A~comprehensive survey}.
{Michigan State University}. Vol.~2. 51~p.

\bibitem{8-is} %9
\Aue{Bellet, A., A.~Habrard,  and M.~Sebban.}
2013.
A survey on metric learning for feature vectors and structured data.
{arXiv preprint arXiv:1306.6709}.

\bibitem{10-is}
\Aue{Wang, F., and J.~Sun.}
2015.
Survey on distance metric learning and dimensionality reduction in data mining.
\textit{Data Min. Knowl. Disc.} 29(2):534--564.

\bibitem{11-is}
\Aue{McFee, B., and G.\,R.~Lanckriet.}
2010.
Metric learning to rank.
\textit{27th  Conference (International) on Machine Learning Proceedings}.
775--782.

\bibitem{12-is}
\Aue{Guillaumin, M., J.~Verbeek,  and C.~Schmid.}
2009.
Is that you? Metric learning approaches for face identification.
\textit{IEEE 12th Conference (International) on Computer Vision}.
498--505.

\bibitem{13-is}
\Aue{Weinberger, K.\,Q., and L.\,K.~Saul.}
2008.
Fast solvers and efficient implementations for distance metric learning.
\textit{25th  Conference (International) on Machine Learning Proceedings}.
1160--1167.

\bibitem{14-is}
\Aue{Weinberger, K.\,Q., J.~Blitzer,  and L.\,K.~Saul}.
2005.
Distance metric learning for large margin nearest neighbor classification.
\textit{Advances in neural information processing systems}.
Cambridge, MA: MIT Press. 1473--1480.

\bibitem{15-is}
\Aue{Berndt, D.\,J., and J.~Clifford}.
1994.
Using dynamic time warping to find patterns in time series.
\textit{KDD Workshop}. 10(16):359--370.

\bibitem{16-is}
\Aue{Petitjean, F., A.~Ketterlin,  and P.~\mbox{Gan{\!\ptb{\c{c}}}arski}}.
2011.
A~global averaging method for dynamic time warping, with applications to
  clustering.
\textit{Pattern Recogn.} 44(3):678--693.

\bibitem{17-is}
\Aue{Goncharov, A.\,B., M.\,S.~Popova, and V.\,V.~Strijov}.
2015.
Metricheskaya klassifikatsiya vremennykh ryadov s~vyravnivaniem otnositel'no
  tsentroidov klassov [Metric time series classification using
  dynamic warping relative to centroids of classes].
\textit{Sistemy i~Sredstva Informatiki}~--- \textit{Systems and Means of Informatics}
25(4):52--64.

\bibitem{18-is}
\Aue{Vandenberghe, L.,  and S.~Boyd}.
1996.
Semidefinite programming.
\textit{SIAM Rev.} 38(1):49--95.

\bibitem{19-is}
\Aue{Isachenko,~R.\,V.} 2015.
Project code. \textit{Sourceforge.net.} 
Available at:
{\sf https://sourceforge.net/p/mlalgorithms/\linebreak 
code/HEAD/tree/Group274/Isachenko2015TimeSeries\linebreak /code/}
(accessed March~18, 2016).

\bibitem{20-is}
UCI repository. Human activity recognition using smartphones dataset.
Available at:
{\sf https://archive.ics. uci.edu/ml/datasets/Human+Activity+Recognition+\linebreak Using+Smartphones}
(accessed March~18, 2016).
\end{thebibliography}

 }
 }

\end{multicols}

\vspace*{-3pt}

\hfill{\small\textit{Received March 18, 2016}}


\Contr

\noindent
\textbf{Isachenko Roman V.} (b.\ 1994)~---
 student, Moscow Institute of Physics and Technology, 
 9~Institutskiy Institutskiy Per., Dolgoprudny, Moscow Region 141700, 
 Russian Federation;   \mbox{isa-ro@yandex.ru}




\vspace*{3pt}

\noindent
\textbf{Strijov Vadim V.} (b.\ 1967)~---
Doctor of Science in physics and mathematics, leading scientist, 
A.\,A.~Dorodnicyn Computing Centre, Federal Research Center 
``Computer Science and Control'' of the Russian Academy of Sciences, 
40~Vavilov Str., Moscow 119333, Russian Federation; \mbox{strijov@ccas.ru}


\label{end\stat}


\renewcommand{\bibname}{\protect\rm Литература} %5
\def\stat{tirsin}

\def\tit{МОДЕЛИ УПРАВЛЕНИЯ РИСКОМ В~ГАУССОВСКИХ СТОХАСТИЧЕСКИХ 
СИСТЕМАХ$^*$}

\def\titkol{Модели управления риском в~гауссовских стохастических 
системах}

\def\aut{А.\,Н.~Тырсин$^1$, А.\,А.~Сурина$^2$}

\def\autkol{А.\,Н.~Тырсин, А.\,А.~Сурина}

\titel{\tit}{\aut}{\autkol}{\titkol}

\index{Тырсин А.\,Н.}
\index{Сурина А.\,А.}
\index{Tyrsin A.\,N.}
\index{Surina A.\,A.}




{\renewcommand{\thefootnote}{\fnsymbol{footnote}} \footnotetext[1]
{ Работа выполнена при финансовой поддержке РФФИ (проект 17-01-00315а).}}


\renewcommand{\thefootnote}{\arabic{footnote}}
\footnotetext[1]{Уральский федеральный университет имени первого Президента России Б.\,Н. Ельцина; Институт 
экономики Уральского отделения Российской академии наук, \mbox{at2001@yandex.ru}}
\footnotetext[2]{Южно-Уральский государственный университет (национальный 
исследовательский университет),  \mbox{dallila87@mail.ru}}

\vspace*{-2pt}

 
  
  \Abst{Описан новый подход к~исследованию риска многомерных стохастических 
сис\-тем. Он основан на гипотезе о~том, что рис\-ком можно управ\-лять за счет изменения 
вероятностных свойств компонент многомерной стохастической сис\-те\-мы, в~качестве 
которых используют факторы рис\-ка. Исследован случай гауссовских стохастических сис\-тем, 
опи\-сы\-ва\-емых случайными векторами, име\-ющи\-ми многомерное нормальное распределение. 
Как показало моделирование, не учтенные в~яв\-ном виде многомерность сис\-те\-мы и~взаимная 
коррелированность ее компонент могут привести к~существенному занижению фактического 
риска. Приведены результаты расчета ве\-ро\-ят\-ности опасного исхода в~зависимости от 
чис\-ло\-вых характеристик многомерной гауссовской случайной величины~--- ковариационной 
мат\-ри\-цы и~вектора математических ожиданий. Выполнена апро\-ба\-ция предложенной модели 
на примере анализа популяционного рис\-ка сер\-деч\-но-со\-су\-ди\-стых заболеваний. Описаны 
модели управ\-ле\-ния рис\-ком в~виде задач его минимизации или достижения заданного уров\-ня. 
Управляющими переменными являются чис\-ло\-вые характеристики случайного вектора~--- 
ковариационная мат\-ри\-ца и~век\-тор математических ожиданий. Проведена апро\-ба\-ция метода 
управ\-ле\-ния рис\-ком с~по\-мощью статистического моделирования методом Мон\-те Карло.}
  
  \KW{риск; модель; стохастическая сис\-те\-ма; случайный вектор; управ\-ле\-ние; нормальное 
рас\-пре\-де\-ление}

\DOI{10.14357/19922264180208}
  
%\vspace*{-6pt}


\vskip 10pt plus 9pt minus 6pt

\thispagestyle{headings}

\begin{multicols}{2}

\label{st\stat}
  
\section{Введение}

\vspace*{-4pt}

  Уже не вызывает сомнений наличие общемировой тенденции быст\-ро\-го рос\-та 
ущерба от природных катаклизмов, техногенных катастроф, террористических 
актов и~экономических потрясений. Многие авторы отмечают, что тем\-пы рос\-та 
ущерба значительно превосходят темпы рос\-та экономики~[1--3]. Это можно 
объяснить по\-сто\-ян\-ным возрастанием рис\-ка в~условиях на\-уч\-но-тех\-ни\-че\-ской 
революции и~форсированного развития техносферы~[4]. Очевидно, что для 
снижения ущер\-ба от природных катаклизмов, техногенных катастроф, 
террористических актов и~экономических потрясений необходимо повысить 
без\-опас\-ность функционирования со\-от\-вет\-ст\-ву\-ющих сис\-тем, а~значит, снизить 
риск. Для этого необходимы адекватные модели и~эффективные методы 
управ\-ле\-ния риском сис\-тем.
  
  Реальные системы, как правило, являются многомерными, их 
функционирование во многом носит стохастический характер, у~них час\-то 
мож\-но выделить десятки различных факторов риска~\cite{1-t}. При решении 
задачи управления рис\-ком необходимо опираться на модель рис\-ка. 

Обычно 
моделирование рис\-ка сводится к~выделению опасных исходов, 
количественному заданию по\-след\-ст\-вий от их наступления и~оцениванию 
вероятностей этих исходов~\cite[с.~37--43]{5-t}. При этом вклад компонент 
многомерной сис\-те\-мы объединяют и~рас\-смат\-ри\-ва\-ют уже одномерную сис\-те\-му 
как случайную величину~[5, с.~148--156; 6, с.~82--87]. 

Но вопрос взаимного 
влияния опас\-ных ситуаций, вызванных разными элементами многомерной 
сис\-те\-мы, мало исследован, чаще всего им пренебрегают или существенно 
упрощают, считая разные опас\-ные исходы взаимно независимыми, 
и~пренебрегают ве\-ро\-ят\-ностью их одновременного наступления. 

Для 
относительно прос\-тых объектов, когда можно априори указать все опасные 
исходы, при наличии статистической информации или экспертных оценок 
о~шан\-сах их по\-яв\-ле\-ния в~целом данный подход дает приемлемые на практике 
результаты. 
%
Обычно здесь удается накопить достаточную статистику для 
оценивания вероятностей на\-ступ\-ле\-ния опас\-ных исходов, а~форма взаимосвязи 
между элементами сис\-те\-мы является до\-ста\-точ\-но прос\-той и~может быть 
описана, например, с~по\-мощью ло-\linebreak\vspace*{-12pt}

\pagebreak

\noindent
ги\-ко-ве\-ро\-ят\-ност\-ных моделей 
риска~\cite{7-t} в~рам\-ках тео\-рии струк\-тур\-но-слож\-ных сис\-тем~\cite{8-t}.
  
  Однако у сложных систем структуру взаимодействия между элементами 
обычно не удается описать с~по\-мощью ло\-ги\-ко-ве\-ро\-ят\-ност\-ных моделей~--- 
стохастические связи между элементами не позволяют их адекватно 
моделировать с~по\-мощью алгебры логики (AND, OR, NOT), а~изменения 
со\-сто\-яния\linebreak элементов и~самой сис\-те\-мы носят непрерывный харак\-тер. Понятия 
опас\-ных исходов также могут размываться, делая невозможным их конкретное 
выделение. К~таким сис\-те\-мам мож\-но отнес\-ти,\linebreak например, 
со\-ци\-аль\-но-эко\-но\-ми\-че\-ские сис\-те\-мы, 
вклю\-чая территориальные и~региональные сис\-те\-мы, 
живые сис\-те\-мы, например человека с~точ\-ки зрения со\-сто\-яния здоровья.
  
  Таким образом, несмотря на большое число исследований, взаимному 
влиянию элементов и~различных фак\-то\-ров рис\-ка на без\-опас\-ность слож\-ных 
многомерных сис\-тем уделяется недостаточно внимания. Во многих случаях, 
когда нет воз\-мож\-ности явно связать разные факторы рис\-ка в~виде  
ло\-ги\-ко-ве\-ро\-ят\-ност\-ной модели, их корреляция при расчете рис\-ка не 
учитывается, поэтому проб\-ле\-ма\-ти\-ка исследований в~об\-ласти анализа рис\-ка, 
особенно в~час\-ти со\-зда\-ния эффективных моделей описания и~управ\-ле\-ния 
рис\-ком слож\-ных многомерных сис\-тем, в~на\-сто\-ящее время становится одной из 
актуальных.
  
  В~\cite{9-t, 10-t} предложен подход к~моделированию риска, со\-глас\-но 
которому стохастическую сис\-те\-му представляют в~виде случайного вектора со 
взаимно коррелированными компонентами, а~в~качестве управ\-ля\-ющих 
переменных используют его чис\-ло\-вые характеристики. Целью \mbox{статьи} является 
описание моделей управ\-ле\-ния рис\-ком на основе данного подхода.

\section{Модель риска в~гауссовских стохастических системах}

  Пусть $S$~--- некоторая многомерная стохастическая сис\-те\-ма. Выделим 
в~этой сис\-те\-ме фак\-то\-ры рис\-ка $X_1, X_2, \ldots, X_m$. В~результате получим 
пред\-став\-ле\-ние сис\-те\-мы в~виде случайного вектора $\mathbf{X}\hm= (X_1, X_2, 
\ldots, X_m)$ с~некоторой плот\-ностью 
ве\-ро\-ят\-ности~$p_{\mathbf{X}}(\mathbf{x})$.
  
  Вместо общепринятого выделения конкретных опасных ситуаций будем 
задавать гео\-мет\-ри\-че\-ские области неблагоприятных исходов. Они могут 
выглядеть произвольным образом в~за\-ви\-си\-мости от конкретной задачи 
и~определяются на основе име\-ющей\-ся априорной информации. Для 
опре\-де\-лен\-ности опишем пред\-ла\-га\-емый под\-ход на примере распространенной 
концепции нежелательных событий как больших и~маловероятных отклонений 
случайной величины относительно ее математического ожидания. Тогда 
опасными ситуациями будем считать большие и~маловероятные отклонения 
вы\-бо\-роч\-ных значений~$x_{ij}$ любой из компонент~$X_j$ относительно 
математических ожиданий $\mu_j\hm=Х{\sf M}[X_j]$, $j\hm=1, 2,\ldots ,m$. 
Вероятность неблагоприятного исхода для каж\-дой из компонент~$X_j$ 
зададим как
 \begin{multline*}
  {\sf P}\left(D_j\right)={\sf P}\left(X_j\in D_j\right)=
  {\sf P}\left( X_j\notin \overline{D}_j\right)\,,\\
  \overline{D}_j=\left\{ x:\ \mu_j-A_{1j}\sigma_j<x<\mu_j+A_{2j}\sigma_j\right\}\,,
\end{multline*}
где $\sigma_j$~--- среднее квад\-ра\-ти\-че\-ское отклонение случайной 
величины~$X_j$; $A_{1j}$ и~$A_{2j}$~--- заданные ниж\-ний и~верх\-ний 
пороговые уров\-ни (в~единицах~$\sigma_j$), т.\,е.\ об\-ласть благоприятных 
исходов ограничена диапазоном $(\mu_j\hm-A_{1j}\sigma_j; 
\mu_j+A_{2j}\sigma_j)$.

  Теперь необходимо задать многомерную область опас\-ных ситуаций~$D$, 
учтя взаимное вли\-яние компонент на по\-яв\-ле\-ние неблагоприятных исходов. Она 
равна $D\hm= \mathbf{R}^m\backslash \overline{D}$, где $\overline{D}$~--- 
об\-ласть допустимых значений фак\-то\-ров рис\-ка. Опишем 
об\-ласть~$\overline{D}$. Это можно сделать различными способами. Наиболее 
оправданным с~гео\-мет\-ри\-че\-ской точки зрения пред\-став\-ля\-ет\-ся задать ее в~виде 
внут\-рен\-ней об\-ласти $m$-ос\-но\-го эллипсоида
  $$
  \overline{D}= \left\{ \mathbf{x}=\left( x_1, x_2, \ldots , x_m\right): 
\sum\limits_{j=1}^m \fr{(x_j-\mu_j^\prime)^2}{A^2_j \sigma_j^2}<1\right\}
  $$
с~центром в~точке $\boldsymbol{\mu}^\prime \hm= (\mu_1^\prime, \mu_2^\prime, 
\ldots , \mu_m^\prime)$, $\mu_j^\prime\hm= \mu_j\hm+A_j\sigma_j$, 
$A_j\hm=(A_{1j}\hm+ A_{2j})/2$, $j\hm=1, 2,\ldots, m$. Тогда для случайного 
вектора~$\mathbf{X}$ ве\-ро\-ят\-ность неблагоприятного исхода будет равна
\begin{multline}
{\sf P}(D) ={\sf P}(\mathbf{X}\in D)\,,\quad
D={}\\
\!\!{}=\left\{ \mathbf{x}=\left( x_1, x_2, \ldots ,x_m\right): \sum\limits^m_{j=1} 
\fr{(x_j-\mu_j)^2}{A_j^2\sigma_j^2}\geq 1\right\}.\!\!
\label{e1-t}
\end{multline}

\begin{figure*}[b] %fig1
     \vspace*{1pt}
 \begin{center}
 \mbox{%
 \epsfxsize=162.957mm 
 \epsfbox{tyr-1.eps}
 }
 \end{center}
\vspace*{-9pt}
\Caption{Реализации стандартного нормального случайного вектора: 
(\textit{а})~$\rho\hm= 0$; (\textit{б})~$\rho\hm = 0{,}9$}
\end{figure*}
  
  Заметим, что в~(\ref{e1-t}) об\-ласть~$D$ неблагоприятных исходов 
пред\-став\-ля\-ет собой внешнюю об\-ласть \mbox{$m$-ос}\-но\-го эл\-лип\-со\-ида, у~которого 
полуоси по каж\-дой из координат равны~$A_j\sigma_j$ соответственно, т.\,е.\ по 
каж\-дой $j$-й оси эта об\-ласть соответствует одномерному случаю~$D_j$. 
Очевидно, когда исход не лежит на одной из осей, событие~$D$ может 
реализоваться и~при отсутствии рис\-ко\-вых отклонений по всем компонентам 
(воз\-мож\-ны ситуации $\mathbf{X}\hm\in D$ и~$\forall j\ X_j\notin D_j$).
  
  Задав функцию по\-след\-ст\-вий от опасных си\-ту\-аций в~виде $g(\mathbf{x})$, 
получим модель для количественной оцен\-ки риска:
  \begin{equation*}
  r(\mathbf{X})=\idotsint\limits_{\mathbf{R}^m} g(\mathbf{x}) 
p_{\mathbf{X}}(\mathbf{x})\,d\mathbf{x}\,.
 % \label{e2-t}
  \end{equation*}
  Если принять
  \begin{equation}
  g(\mathbf{x})=\begin{cases}
  1\,, &\ \mathbf{x}\in D\,;\\
  0\,, &\ \mathbf{x}\notin D\,,
  \end{cases}
  \label{e3-t}
  \end{equation}
то $r(\mathbf{X})={\sf P}(\mathbf{X}\in D)$, т.\,е.\ риск оцениваем как ве\-ро\-ят\-ность 
неблагоприятного исхода. Если на ранней стадии исследования сис\-те\-мы 
слож\-но до\-ста\-точ\-но точ\-но описать функцию $g(\mathbf{x})$, то 
формула~(\ref{e3-t}) становится оценкой~${\sf P}(D)$ и~является удоб\-ным 
начальным приб\-ли\-же\-ни\-ем модели риска.

  Рассмотрим далее наиболее рас\-про\-стра\-нен\-ный част\-ный случай, 
когда~$\mathbf{X}$ имеет совместное нормальное рас\-пре\-де\-ле\-ние с~плот\-ностью 
ве\-ро\-ят\-ности
\begin{multline*}
  p_{\mathbf{X}}(\mathbf{x})={}\\
  {}=\fr{1}{\sqrt{(2\pi)^m\vert\boldsymbol{\Sigma}\vert}}\,\exp 
\left\{ -\fr{1}{2}\left( \mathbf{x}-\mathbf{a}\right)^{\mathrm{T}} 
\boldsymbol{\Sigma}^{-1}(\mathbf{x}-\mathbf{a})\right\}\,,
  \end{multline*}
где $\mathbf{a}=(a_1, a_2, \ldots, a_m)^{\mathrm{T}}$~--- век\-тор 
математических ожиданий; $\boldsymbol{\Sigma}\hm= \{ \sigma_{ij}\}_{m\times 
m}$~--- ковариационная мат\-рица.
  
  Использование гауссовского случайного век\-то\-ра опирается на цент\-раль\-ную 
предельную тео\-ре\-му~\cite{11-t}. Как показала апробация на ряде примеров, 
такая идеализация не столь критична, и~если есть ка\-кие-ли\-бо основания 
считать, что плот\-но\-сти вероятностей компонент вектора~$\mathbf{X}$ име\-ют 
более вытянутые хвос\-ты, то это можно скорректировать за счет 
со\-от\-вет\-ст\-ву\-юще\-го задания функции~$g(\mathbf{x})$.
  
  Исследуем влияние многомерности и~коррелированности факторов риска на 
ве\-ро\-ят\-ность по\-яв\-ле\-ния неблагоприятных исходов.
  
  \smallskip
  
  \noindent
  \textbf{Пример~1.}\ Для наглядности рас\-смот\-рим двумерный гауссовский 
случайный вектор ($X_1, X_2$) с~плот\-ностью ве\-ро\-ят\-ности
  \begin{equation}
   p_{X_1, X_2}(x_1, x_2) =\fr{e^{-Q(x_1-a_1, x_2-a_2)/2}}{2\pi 
\sigma_1\sigma_2 \sqrt{1-\rho^2}}\,.
 \label{e4-t}
\end{equation}
Здесь
$$
  Q\left(y_1,y_2\right)=\fr{1}{1-\rho^2}\left( \fr{y_1^2}{\sigma_1^2} -\fr{2\rho 
y_1y_2}{\sigma_1\sigma_2}+\fr{y_2^2}{\sigma_2^2}\right)\,,
$$
где $y_i=x_i-a_i$, $i=1, 2$; $\rho\hm= \sigma_{12}/(\sigma_1\sigma_2)$~--- 
коэффициент корреляции между~$X_1$ и~$X_2$.
  
  На рис.~1 показаны примеры реализаций стандартного нормального 
случайного вектора ($X_1, X_2$) для некоррелированных ($\rho\hm=0$) 
и~коррелированных ($\rho\hm = 0{,}9$) компонент. Видим, что увеличение 
тес\-но\-ты корреляционной связи между компонентами приводит к~вытягиванию 
диа\-грам\-мы рас\-се\-яния и~увеличению ве\-ро\-ят\-ности по\-яв\-ле\-ния больших 
укло\-не\-ний случайного век\-тора.
  
  
 % \smallskip
 
  { \begin{center}  %fig2
 \vspace*{-2pt}
  \mbox{%
 \epsfxsize=78.141mm 
 \epsfbox{tyr-2.eps}
 }


\end{center}


\noindent
{{\figurename~2}\ \ \small{Зависимости $\lg P(D)$ от порогового уров\-ня~$A$: 
(\textit{а})~$D_e(\mathbf{X})\hm = 0$; (\textit{б})~$D_e(\mathbf{X}) \hm= 0{,}5$; 
(\textit{в})~$D_e(\mathbf{X}) \hm=1$; \textit{1}~--- $m \hm= 1$; 
\textit{2}~---2; 
\textit{3}~--- 3; \textit{4}~--- 4; \textit{5}~--- $m\hm= 5$}}
}

\vspace*{18pt}

\setcounter{figure}{2}

  
  \noindent
  \textbf{Пример~2.}\ Зададим для определенности раз\-мер\-ность 
вектора~$\mathbf{X}$ от~1 до~5. В~[12] введен коэффициент тес\-но\-ты 
со\-вмест\-ной линейной корреляционной связи компонент случайного 
век\-то\-ра~$\mathbf{X}$, равный $D_e(\mathbf{X})\hm=1\hm- \vert 
\mathbf{R}_{\mathbf{X}}\vert^{1/m}$, где $\mathbf{R}_{\mathbf{X}}$~--- 
корреляционная мат\-ри\-ца случайного век\-то\-ра~$\mathbf{X}$. Очевидно, что 
$0\hm\leq D_e(\mathbf{X})\hm\leq 1$. Случай $D_e(\mathbf{X})\hm=0$ 
соответствует не\-за\-ви\-си\-мости компонент~$X_1, X_2, \ldots , X_m$, а~при 
$D_e(\mathbf{X})\hm=1$ имеем строгую линейную за\-ви\-си\-мость компонент.
  
  Рассмотрим три случая: $D_e(\mathbf{X}) \hm= 0$, $D_e(\mathbf{X}) \hm= 
0{,}5$ и~$D_e(\mathbf{X}) \hm= 1$. Результаты рас\-че\-та ве\-ро\-ят\-ности 
неблагоприятного исхода~(\ref{e1-t}) приведены на рис.~2. Для большей на\-гляд\-ности 
примем  $A_1\hm=A_2= \cdots =A_m\hm=A$.
  
 
  
  Анализ графиков на рис.~2 говорит о~сле\-ду\-ющем. Увеличение 
раз\-мер\-ности~$m$ и~тес\-но\-ты кор\-ре\-ля\-ционной связи меж\-ду компонентами 
случайного вектора~$\mathbf{X}$ приводит к~резкому рос\-ту ве\-ро\-ят\-ности 
не\-бла\-го\-при\-ят\-но\-го исхода.
  
  Особенно важным оказалось то, что даже относительно малая тес\-но\-та 
корреляционной связи ($D_e(\mathbf{X}) \hm= 0{,}5$), которая почти всегда 
наблюдается на прак\-ти\-ке, уже приводит к~значительному рос\-ту 
ве\-ро\-ят\-ности~${\sf P}(D)$. Эффект усиливается с~увеличением значений~$A_j$, что 
соответствует менее вероятным, но более опас\-ным неблагоприятным исходам. 
Например, при $A\hm = 6$ ве\-ро\-ят\-ность неблагоприятного исхода более чем 
в~7000~раз выше у~коррелированной сис\-те\-мы ($D_e(\mathbf{X}) \hm= 1$) по 
срав\-не\-нию с~некоррелированной ($D_e(\mathbf{X}) \hm= 0$). Поэтому при 
моделировании рис\-ка необходимо учитывать как фактор мно\-го\-мер\-ности, так 
и~тес\-но\-ту корреляционных связей.

  

\section{Апробация модели риска на~примере анализа 
популяционного риска сердечно-сосудистых заболеваний}

  Одной из малоизученных проблем в~медицине является комплексная оценка 
популяционного здоровья одновременно по нескольким факторам риска в~их 
взаимосвязи. Это объясняется тем, что не\-яс\-но, как учитывать вклад каждого 
фак\-то\-ра рис\-ка в~общую оценку со\-сто\-яния здоровья. Обычно в~таких случаях 
используются экспертные оценки, которые нельзя считать в~полной мере 
объективными~[13, 14].
  
  Исследуем динамику изменения с~возрастом популяционного риска 
сер\-деч\-но-со\-су\-ди\-стых за\-бо\-леваний по основным биологическим факторам риска, 
к~которым относят артериальную ги\-пер\-тен\-зию, дис\-ли\-пи\-де\-мию, повышенный 
уро\-вень глюкозы в~крови и~избыточную массу тела~[15]. 

В~качестве 
биологических па\-ра\-мет\-ров, ха\-рак\-те\-ри\-зу\-ющих эти факторы рис\-ка, используют 
уровень общего холестерина (ОХС), сис\-то\-ли\-че\-ское артериальное дав\-ле\-ние 
(САД), индекс массы тела (ИМТ), уровень глюкозы (УГ). 

Статистический 
материал получен в~результате комплексного сплош\-но\-го углуб\-лен\-но\-го  
кли\-ни\-ко-эпи\-де\-мио\-ло\-ги\-че\-ско\-го обследования муж\-ской сельской 
популяции с~гнез\-до\-вой выборкой. Всего было обследовано~1402~мужчины 
одного из сел Челябинской об\-ласти, что со\-ста\-ви\-ло~93\% от списочного со\-ста\-ва 
села. Для всех пациентов был проведен необходимый комплекс клинических, 
лабораторных и~инструментальных методов обследования для 
квалифицированного заключения о~со\-сто\-янии здоровья. Работу проводила 
бригада специалистов, со\-сто\-ящая из со\-труд\-ни\-ков ка\-фед\-ры госпитальной 
терапии и~семейной медицины Челябинской государственной медицинской 
академии и~врачей Челябинской об\-ласт\-ной клинической больницы №\,1~[16].
  
  Пороговые значения биологических па\-ра\-мет\-ров, характеризующих основ\-ные 
биологические фак\-то\-ры рис\-ка, при превышении которых риск 
сер\-деч\-но-со\-су\-ди\-стых 
ослож\-не\-ний резко воз\-рас\-та\-ет (ниже этого значения~--- норма), в~соответствии 
с~[15] рав\-ны: САД~--- 140~мм\ рт.\ ст.; ИМТ~--- 25~кг/м$^2$; ОХС~--- 
5~ммоль/л; УГ~--- 5,5~ммоль/л. 
Исследование проводилось сле\-ду\-ющим 
образом. 

Было сформировано четыре группы по воз-\linebreak рас\-там: 18--24~года,  
25--34~года, 35--44~года и~45--54~года. Проверка по критерию со\-гла\-сия 
\mbox{$\chi^2$-Пир}\-со\-на статистической гипотезы о~соответствии\linebreak каж\-дой группы 
наблюдений для всех фак\-то\-ров рис\-ка нормальному распределению на уровне 
зна\-чи\-мости~0,05 не была отклонена. Поэтому считаем, что имеем гауссовскую 
стохастическую сис\-те\-му раз\-мер\-ности $m\hm = 4$.
  
  Затем для каждой группы были определены средние значения 
и~ковариационные мат\-ри\-цы. Вы\-чис\-ле\-ние вероятности ${\sf P}(D)$ можно 
выполнять двумя способами~--- с~по\-мощью чис\-лен\-но\-го интегрирования для 
малых размерностей ($m\hm\leq 4$) или методом статистических испытаний  
Мон\-те Кар\-ло~[17] при раз\-мер\-ности $m\hm>4$. Результаты расчета 
приведены в~таб\-лице.



  
  Видим, что наблюдается тенденция рос\-та риска возникновения 
сердечно-сосудистых ослож\-не\-ний.\linebreak\vspace*{-12pt}

\vspace*{6pt}

%\begin{table*}
{\small
  \begin{center}
  \begin{tabular}{|c|c|}
\multicolumn{2}{p{43mm}}{Значения вероятностей риска возникновения 
сер\-деч\-но-со\-су\-ди\-стых ослож\-не\-ний}\\[-6pt]
\multicolumn{2}{c}{\ }\\
\hline
Возраст, лет&Вероятность\\
\hline
18--24&0,70\\
25--34&0,78\\
35--44&0,95\\
45--54&0,98\\
\hline
\end{tabular}
\vspace*{2pt}
\end{center}
}
%\end{table*}

\columnbreak
  
  
  \noindent
   Полученные в~целом высокие значения 
вероятностей ${\sf P}(D)$ соответствуют фактическому со\-сто\-янию
   здоровья. 
   
   Как 
показали результаты комплексного сплош\-но\-го углуб\-лен\-но\-го  
кли\-ни\-ко-эпи\-де\-мио\-ло\-ги\-че\-ско\-го обследования, в~обследованной 
популяции здоровых лиц в~воз\-рас\-те старше~34~лет практически не оказалось.

 
\section{Модели управления риском}

  Введенная модель риска позволяет на практике осуществлять управ\-ле\-ние 
сто\-ха\-сти\-че\-ской сис\-те\-мой с~целью его снижения.
  
  \bigskip
  
  \noindent
  \textbf{Пример~3.}\ Проиллюстрируем данный подход на прос\-тей\-шем 
примере гауссовской стохастической сис\-те\-мы с~раз\-мер\-ностью $m\hm=2$. На 
рис.~3 показана воз\-мож\-ность уменьшения ве\-ро\-ят\-ности неблагоприятного 
исхода~${\sf P}(D)$, а~значит, и~рис\-ка за счет варьирования па\-ра\-мет\-ров плот\-ности 
$p_{\mathbf{X}}(\mathbf{x})$. Об\-ласть неблагоприятного исхода~$D$ 
рас\-по\-ло\-же\-на выше линии в~правом верх\-нем углу.
  
  Видим, что возможные варианты изменения па\-ра\-мет\-ров случайного вектора 
($X_1, X_2$): уменьшение ковариации (или коэффициента корреляции), 
изменение математических ожиданий случайных величин, уменьшение 
дис\-пер\-сий~$\sigma_1^2$ или~$\sigma_1^2$~--- могут привести к~снижению 
ве\-ро\-ят\-ности~${\sf P}(D)$.
  

  
  Суть управ\-ле\-ния риском гауссовской стохастической сис\-те\-мы со\-сто\-ит 
в~сле\-ду\-ющем. Задав функцию по\-след\-ст\-вий от опасных ситуаций 
$g(\mathbf{x})$ и~введя ограничения на допустимые значения элементов 
ковариационной мат\-ри\-цы $G(\boldsymbol{\Sigma})$ и~сред\-них значений 
компонент сис\-те\-мы $H(\mathbf{a})$, сформулируем задачу минимизации рис\-ка 
с~переменными~$\boldsymbol{\Sigma}$ и~$\mathbf{a}$:
  \begin{multline}
    r(\boldsymbol{\Sigma}, \mathbf{a})=\displaystyle \idotsint\limits_{\mathbf{R}^m} 
g(\mathbf{x}) p_{\mathbf{X}}(\mathbf{x})\,d\mathbf{x} \to 
\min\limits_{\boldsymbol{\Sigma}, \mathbf{a}}\,,\\
\boldsymbol{\Sigma} \in G(\boldsymbol{\Sigma})\,,\enskip \mathbf{a}\in 
H(\mathbf{a})\,.
    \label{e5-t}
  \end{multline}
  
  Задача~(\ref{e5-t}) является задачей нелинейного про\-грам\-ми\-ро\-ва\-ния. Ее 
мож\-но решить разными методами. Одним из них является метод барьерных 
функ\-ций (внут\-рен\-них штраф\-ных функ\-ций)~[18]. Его основная идея со\-сто\-ит 
в~приведении задачи поиска услов\-но\-го экстремума к~по\-сле\-до\-ва\-тель\-ности задач 
на\-хож\-де\-ния без\-услов\-но\-го экстремума вспомогательной функции:
  $$
  F(\mathbf{X}, b_k) =r(\boldsymbol{\Sigma}, \mathbf{a}) 
+{\sf P}\left(\boldsymbol{\Sigma}, \mathbf{a}, b_k\right)\,,
  $$
где ${\sf P}(\boldsymbol{\Sigma}, \mathbf{a}, b_k)$~--- штраф\-ная функ\-ция; $b_k$~--- 
па\-ра\-метр штрафа.

\pagebreak
  
  
 % \smallskip
 \end{multicols}
 
  \begin{figure*} %fig3
  \vspace*{1pt}
 \begin{center}
 \mbox{%
 \epsfxsize=164.954mm 
 \epsfbox{tyr-3.eps}
 }
 \end{center}
\vspace*{-4pt}
\Caption{Снижение риска: (\textit{а})~исходное со\-сто\-яние; (\textit{б})~за счет уменьшения 
корреляции; (\textit{в}) и~(\textit{г})~за счет изменения 
математических ожиданий случайных величин~$X_1$ или~$X_2$
соответственно; 
(\textit{д}) и~(\textit{е})~за счет уменьшения дис\-пер\-сии~$\sigma_1^2$ 
или~$\sigma_2^2$ соответственно}
\vspace*{12pt}
\end{figure*}
  
  
  \begin{multicols}{2}
  
  \noindent
  \textbf{Пример~4.}\ Рас\-смот\-рим двумерный гауссовский случайный век\-тор 
с~плот\-ностью ве\-ро\-ят\-ности~(\ref{e4-t}). Задача минимизации будет выглядеть 
как
\begin{multline*}
  r(\boldsymbol{\Sigma}, \mathbf{a})={}\\
  {}=\iint\limits_{\mathbf{R}^2} 
   \fr{g(x_1, x_2)}{2\pi \sigma_1\sigma_2 \sqrt{1-\rho^2}} %\times{}\\
%{}\times 
e^{-Q(x_1-a_1, x_2-
a_2)/2} \,dx_1 dx_2 \to {}\\
{}\to \min\limits_{\boldsymbol{\Sigma}, \mathbf{a}}
\end{multline*}
с ограничениями
\begin{equation*}
\begin{array}{l}
\sigma_1^2\sigma_2^2> \sigma_{12}^2\,;\\[6pt]
a_i^- <a_i< a_i^+\,,\enskip i=1,2\,;\\[6pt]
\sigma_{ij}^-<\sigma_{ij}<\sigma_{ij}^+\,,\enskip j=1,2\,.
\end{array}
\end{equation*}
  
  Зададим конкретные значения: $a_1^-\hm=a_2^- \hm=-3$; $a_1^+\hm= a_2^+ 
\hm=3$; $\sigma_{12}^- \hm= 0{,}1$; $\sigma_{12}^+\hm=3$. Для 
опре\-де\-лен\-ности считаем, что $\forall\ \mathbf{x}\hm\notin D$ 
$g(\mathbf{x})\hm=0$. Получаем задачу

\noindent
  \begin{multline*}
  r(\boldsymbol{\Sigma}, \mathbf{a})= \iint\limits_{\mathbf{R}^2} 
\fr{  g(x_1, x_2)}{1{,}2\pi}\times{}\\
\!{}\times e^{-\left((x_1-2)^2-1{,}6x_1 x_2+(x_2-2)^2\right)/\left(2\cdot 0{,}6^2\right)}\, 
dx_1 dx_2\to \min\limits_{\boldsymbol{\Sigma}, \mathbf{a}}\hspace*{-5.74925pt}
\end{multline*}
с ограничениями
\begin{equation*}
\begin{array}{l}
\sigma_1^2\sigma_2^2>\sigma_{12}^2\,;\\[6pt]
-3<a_1, a_2<3\,;\\[6pt]
0{,}1<\sigma_1, \sigma_2<3\,;\\[6pt]
0{,}1< \sigma_2<3\,.
\end{array}
\end{equation*}
  
    
    Выберем в~качестве штрафной функ\-ции обратную: 
    $$
    {\sf P}\left(\boldsymbol{\Sigma},\mathbf{a}, b^k\right) =-b^k \sum\limits^m_{j=1} 
\fr{1}{t_j(\Sigma, a)}\,.
$$
    Тогда с~учетом всех ограничений вспомогательная функ\-ция примет вид:
    
    \noindent
    \begin{multline*}
    F\left( \mathbf{X}, b_k\right) ={}\\
    {}=\iint\limits_{\mathbf{R}^2} \!\fr{1}{1{,}2\pi} 
\,e^{- \left((x_1-2)^2 -1{,}6x_2x_2+(x_2-2)^2\right)/\left(2\cdot 0{,}6^2\right)} \,d\mathbf{x} -{}\\
{}-
b^k\left( \fr{1}{\sigma_1} +\fr{1}{\sigma_2}+\fr{1}{\sigma_1^2 \sigma_2^2-
\sigma_{12}^2} +\fr{1}{a_1+3}+ {}\right.\\
{}+\fr{1}{3-a_1}+ \fr{1}{a_2+3} +\fr{1}{3-a_1} 
+\fr{1}{\sigma_1-0{,}1} +{}\\
\left.{}+\fr{1}{3-\sigma_1} +\fr{1}{\sigma_2 -0{,}1} +\fr{1}{3-
\sigma_2}\right) \to \min\limits_{\boldsymbol{\Sigma}, \mathbf{a}}\,.
    \end{multline*}
  
  Поиск минимума вспомогательной функ\-ции находим с~по\-мощью 
покоординатного спус\-ка. Начальные значения па\-ра\-мет\-ров:
  $$
  \boldsymbol{\Sigma}^0=\begin{pmatrix}
  1 & 0{,}8\\
  0{,}8 & 1
  \end{pmatrix}\,;\enskip \mathrm{a}^0=\begin{pmatrix}
  2\\ 2\end{pmatrix}\,;\enskip b^0=10\,.
  $$
  
  Задача минимизации была решена при значениях па\-ра\-мет\-ров: 
$a_1\hm=2{,}6$; $a_2\hm=0{,}6$; $\sigma_1\hm=2{,}9$; $\sigma_2\hm=2{,}9$; 
$\sigma_{12}\hm= 1{,}39\cdot 10^{-16}$. При этом минимум целевой функ\-ции 
с~точ\-ностью до~0,001:  $r(\boldsymbol{\Sigma}^*, \mathbf{a}^*)\hm= 
0{,}041$.
  
  Задача~(\ref{e5-t}) полезна лишь в~качестве первого приб\-ли\-же\-ния модели 
управ\-ле\-ния рис\-ком, так как не учитывает ограничений, связанных с~за\-тра\-та\-ми 
на изменения варь\-и\-ру\-емых па\-ра\-мет\-ров относительно своих начальных 
значений.
  
  Если ввести ограничения на за\-тра\-ты, связанные с~изменением 
переменных~$\boldsymbol{\Sigma}$ и~$\mathbf{a}$, то получим задачу:
  \begin{equation}
  \left.
  \begin{array}{c}
 \displaystyle r(\boldsymbol{\Sigma}, \mathbf{a}) =\idotsint\limits_{\mathbf{R}^m} 
g(\mathbf{x}) p_{\mathbf{X}}(\mathbf{x})\,d\mathbf{x}\to 
\min\limits_{\boldsymbol{\Sigma}, \mathbf{a}}\,,\\[6pt]
  %\hspace*{27mm}
  \boldsymbol{\Sigma}\in G(\boldsymbol{\Sigma})\,,\enskip \mathbf{a}\in 
H(\mathbf{a})\,,\\[6pt]
  a_i=a_i^0+\delta_i\,,\enskip v_i(\delta_i)\leq V_i\,,\enskip i=1,\ldots ,m\,,\\[6pt]
  \sigma_{ij} =\sigma_{ij}^0 +\Delta_{ij}\,,\enskip w_{ij}(\Delta_{ij})\leq 
W_{ij}\,,\\[6pt]
\hspace*{40mm} i, j=1,\ldots ,m\,,
  \end{array}
  \right\}
  \label{e6-t}
  \end{equation}
где $v_i(\delta_i)$~--- функ\-ция за\-трат на изменение среднего значения $i$-й 
компоненты, име\-ющей начальное значение~$a_i^0$; $V_i$~--- предельная 
величина за\-трат на это изменение; $w_{ij}(\Delta_{ij})$~--- функция за\-трат на 
изменение ковариации между $i$-й и~$j$-й компонентами; $\sigma_{ij}^0$~--- 
начальное значение ковариации; $W_{ij}$~--- предельная величина за\-трат на 
это изменение.
  
  Минимизация риска не всегда может быть приемлемым управ\-ле\-ни\-ем. 
Альтернативой является достижение приемлемого рис\-ка~$r^*$ при 
минимальных изменениях чис\-ло\-вых характеристик гауссовской 
сис\-те\-мы~$\mathbf{X}$. Здесь возможны два варианта по\-ста\-нов\-ки задачи.
  
  Во-первых, на основе~(\ref{e6-t}) получаем альтернативный вариант:
  
  \noindent
  \begin{equation*}
  \begin{array}{c}
  \displaystyle \sum\limits_{i=1}^m v_i(\delta_i) +\sum\limits^m_{i-1} 
\sum\limits^m_{j=i} w_{ij}(\Delta_{ij}) \to \min\limits_{\boldsymbol{\Sigma}, 
\mathbf{a}}\,,\\[6pt]
%  \hspace*{27mm}
\boldsymbol{\Sigma} \in G(\boldsymbol{\Sigma})\,,\enskip \mathbf{a}\in 
H(\mathbf{a})\,,\\[6pt]
  \delta_i =a_i-a_i^0\,,\enskip \Delta_{ij}=\sigma_{ij}-\sigma_{ij}^0\,,\enskip 
i,j=1,\ldots ,m\,,\\[6pt]
  r(\boldsymbol{\Sigma}, \mathbf{a}) =r^*\,.
  \end{array}
  \end{equation*}
  
  
  Во-вторых, если сложно задать функции за\-трат~$v_i(\cdot)$ 
и~$w_{ij}(\cdot)$, то мож\-но минимизировать суммарное 
изменение~$\boldsymbol{\Sigma}$ и~$\mathbf{a}$, перейдя к~задаче
  \begin{equation*}
  \begin{array}{c}
  \displaystyle \sum\limits_{i=1}^m \alpha_i \delta_i^2 +\sum\limits^m_{i-1} 
\sum\limits^m_{j=i} \beta_{ij} \Delta_{ij}^2 \to 
\min\limits_{\boldsymbol{\Sigma}, \mathbf{a}}\,,\\[6pt]
 % \hspace*{27mm}
 \boldsymbol{\Sigma} \in G(\boldsymbol{\Sigma})\,,\enskip \mathbf{a}\in 
H(\mathbf{a})\,,\\[6pt]
  \delta_i =a_i-a_i^0\,,\enskip \Delta_{ij}=\sigma_{ij}-\sigma_{ij}^0\,,\enskip 
i,j=1,\ldots ,m\,,\\[6pt]
  r(\boldsymbol{\Sigma}, \mathbf{a}) =r^*\,,
  \end{array}
  \end{equation*}
где $\alpha_i$ и~$\beta_{ij}$~--- весовые коэффициенты.

\section{Заключение}

\noindent
  \begin{enumerate}[1.]
  \item  Предложен новый подход к~исследованию рис\-ка сложных сис\-тем. 
В~его основе лежит моделирование сис\-те\-мы в~виде мно\-го\-мер\-ной случайной 
величины, компоненты которой являются факторами риска.
  \item  Для гауссовских стохастических сис\-тем предложены модели 
управления рис\-ком на основе его минимизации или до\-сти\-же\-ния заданного 
уров\-ня, используя в~качестве управ\-ля\-ющих переменных чис\-ло\-вые 
характеристики случайного век\-то\-ра~--- вектор математических ожиданий 
и~ковариационную мат\-рицу.
  \item  В настоящее время обычно при исследовании риска слож\-ных 
многомерных сис\-тем не выделяют в~явном виде их компоненты и~их 
коррелированность. Как показало моделирование, неучет в~явном виде 
многомерности сис\-те\-мы и~взаимной коррелированности ее компонент может 
привести к~существенному занижению фактического рис\-ка. Усиление тес\-но\-ты 
корреляционной связи между факторами риска приводит к~значительному 
рос\-ту ве\-ро\-ят\-ности одновременного принятия ими опас\-ных значений.
  \item Предложенная гипотеза об управ\-ле\-нии рис\-ком слож\-ной сис\-те\-мы на 
основе изменения чис\-ло\-вых характеристик ее математической модели в~форме 
случайного век\-то\-ра носит предварительный характер. Необходимо выполнить 
апро\-ба\-цию данного подхода на ряде задач.
  \end{enumerate}
  \vspace*{-8pt}
  
{\small\frenchspacing
 {%\baselineskip=10.8pt
 \addcontentsline{toc}{section}{References}
 \begin{thebibliography}{99}
 
 \bibitem{3-t} %1
\Au{Гор А.} Земля на чаше весов. В~по\-ис\-ках новой об\-щей цели~// Новая 
пост\-ин\-ду\-стри\-аль\-ная волна на Западе: Антология~/ Пер. с~англ.~--- М.: 
Academia, 1999. С.~557--571. (\Au{Gore~A.} Earth in the balance. Forging a~new 
common purpose.~--- London: Earthscan Publications Ltd., 1992.)
\bibitem{1-t} %2
\Au{Воробьев Ю.\,Л., Малинецкий~Г.\,Г., Махутов~Н.\,А.} Управ\-ле\-ние рис\-ком 
и~устойчивое развитие: Человеческое измерение~// Известия вузов. 
Прикладная нелинейная динамика, 2000. Т.~8. №\,6. С.~12--26.
\bibitem{2-t} %3
\Au{Порфирьев Б.\,Н.} Снижение природных рис\-ков экономического развития 
России: роль государства~// Актуальные проб\-ле\-мы гражданской защиты:  
Мат-лы 11-й Междунар. науч.-практич. конф. по проб\-ле\-мам защиты 
населения и~территорий от чрезвычайных ситуаций.~--- 
Н.~Новгород: Вектор-ТиС, 2006. С.~44--50. {\sf 
gov.mari.ru/debzn/omgo/46.djvu}.

\bibitem{4-t}
\Au{Порфирьев Б.\,Н.} Управ\-ле\-ние в~чрез\-вы\-чай\-ных ситуациях. Итоги науки 
и~техники. Проб\-ле\-мы без\-опас\-ности: чрез\-вы\-чай\-ные ситуации. Т.~1.~--- М.: 
ВИНИТИ, 1991. 204~с.
\bibitem{5-t}
\Au{Вишняков Я.\,Д., Радаев~Н.\,Н.} Общая теория рисков.~--- 
2-е изд., испр.~--- М.: Академия, 2008. 368~с.
\bibitem{6-t}
\Au{Акимов В.\,А., Лесных~В.\,В., Радаев~Н.\,Н.} Риски в~природе, техносфере, 
обществе и~экономике.~--- М.: Деловой экспресс, 2004. 352~с.
\bibitem{7-t}
\Au{Соложенцев Е.\,Д.} Сценарное ло\-ги\-ко-ве\-ро\-ят\-ност\-ное управ\-ле\-ние 
рис\-ком в~бизнесе и~технике.~--- 2-е изд.~--- СПб.: Биз\-нес-прес\-са, 2006. 560~с.
\bibitem{8-t}
\Au{Рябинин И.\,А.} На\-деж\-ность и~без\-опас\-ность струк\-тур\-но-слож\-ных  
сис\-тем.~--- СПб.: Политехника, 2000. 248~с.
\bibitem{9-t}
\Au{Тырсин А.\,Н.} О~моделировании рис\-ка в~сис\-те\-мах критичных 
инфра\-струк\-тур~// Экономические и~технические аспекты без\-опас\-ности 
строительных критичных инфраструктур: Тезисы Междунар. конф.~--- 
Екатеринбург: УрФУ, 2015. С.~205--208. {\sf 
http://elar.urfu. ru/bitstream/10995/33468/1/safety\_2015.pdf.}
\bibitem{10-t}
\Au{Тырсин А.\,Н., Сурина~А.\,А.} Моделирование рис\-ка в~многомерных 
сто\-ха\-сти\-че\-ских сис\-те\-мах~// Вестн. Томского государственного университета. 
Управ\-ле\-ние, вы\-чис\-ли\-тель\-ная техника и~информатика, 2017. №\,2(39).  
С.~65--72.
\bibitem{11-t}
\Au{Гнеденко Б.\,В.} Курс тео\-рии вероятностей.~--- 8-е изд., испр. и~доп.~--- М.: 
Едиториал УРСС, 2005. 448~с.
\bibitem{12-t}
\Au{Pena D., Rodriguez~J.} Descriptive measures of multivariate scatter and linear 
dependence~// J.~Multivariate Anal., 2003. Vol.~85. P.~361--374.
\bibitem{13-t}
\Au{Кирьянов Б.\,Ф., Токмачёв~М.\,С.} Математические модели 
в~здравоохранении.~--- Великий Новгород: НовГУ им.\ Ярослава Мудрого, 
2009. 279~с.
\bibitem{14-t}
\Au{Цинкер М.\,Ю., Кирьяков~Д.\,А., Камалтдинов~М.\,Р.} Применение 
комплексного ин\-дек\-са нарушения здо\-ровья населения для оцен\-ки 
популяционного здо\-ровья в~Пермском крае~// Из\-вес\-тия Самарского научного 
цент\-ра РАН, 2013. Т.~15. №\,3(6). С.~1988--1992.
\bibitem{15-t}
Профилактика хронических неинфекционных заболеваний. Рекомендации.~--- 
М., 2013. 128~с. {\sf http:// www.webmed.irkutsk.ru/doc/pdf/prevent.pdf}.
\bibitem{16-t}
\Au{Тырсин А.\,Н., Калев~О.\,Ф., Яшин~Д.\,А., Лебедева~О.\,В.} Оцен\-ка 
со\-сто\-яния здо\-ровья популяции на основе энтропийного моделирования~// 
Математическая био\-ло\-гия и~био\-ин\-фор\-ма\-ти\-ка, 2015. Т.~10. Вып.~1.  
С.~206--219. doi: 10.17537/2015.10.206.
\bibitem{17-t}
\Au{Михайлов Г.\,А., Войтишек~А.\,В.} Чис\-лен\-ное ста\-ти\-сти\-че\-ское 
моделирование. Методы Мон\-те-Кар\-ло.~--- М.: Академия, 2006. 368~с.
\bibitem{18-t}
\Au{Пантелеев А.\,В., Летова~Т.\,А.} Методы оптимизации в~примерах 
и~задачах.~--- 3-е изд., стер.~---М.: Выс\-шая школа, 2008. 544~с.

 \end{thebibliography}

 }
 }

\end{multicols}

\vspace*{-6pt}

\hfill{\small\textit{Поступила в~редакцию 21.08.17}}

\vspace*{6pt}

%\newpage

%\vspace*{-24pt}

\hrule

\vspace*{2pt}

\hrule

%\vspace*{8pt}


\def\tit{A~MODEL OF RISK MANAGEMENT IN~GAUSSIAN\\ STOCHASTIC SYSTEMS}

\def\titkol{A~model of risk management in Gaussian stochastic systems}

\def\aut{A.\,N.~Tyrsin$^{1,2}$ and~A.\,A.~Surina$^3$}

\def\autkol{A.\,N.~Tyrsin and~A.\,A.~Surina}

\titel{\tit}{\aut}{\autkol}{\titkol}

\vspace*{-9pt}


\noindent
$^1$Ural Federal University named after first President of Russia B.\,N.~Yeltsin, 
19~Mira Str., Ekaterinburg 620002,\linebreak
$\hphantom{^1}$Russian Federation 

\noindent
$^2$Institute of Economics, Ural Branch of the Russian Academy of Sciences, 
29~Moskovskaya Str., Yekaterinburg\linebreak
$\hphantom{^1}$620014, Russian Federation

\noindent
$^3$Institute of Natural Sciences, South Ural State University, 87~Lenin Ave., 
Chelyabinsk 454080, Russian Federation


\def\leftfootline{\small{\textbf{\thepage}
\hfill INFORMATIKA I EE PRIMENENIYA~--- INFORMATICS AND
APPLICATIONS\ \ \ 2018\ \ \ volume~12\ \ \ issue\ 2}
}%
 \def\rightfootline{\small{INFORMATIKA I EE PRIMENENIYA~---
INFORMATICS AND APPLICATIONS\ \ \ 2018\ \ \ volume~12\ \ \ issue\ 2
\hfill \textbf{\thepage}}}

\vspace*{3pt}




\Abste{A new approach to research of risk of multidimensional 
stochastic systems is described. It is based on a~hypothesis that 
the risk can be managed by changing probabilistic properties of a~component of 
a~multidimensional stochastic system. The case of Gaussian stochastic systems 
described by random vectors having
the multidimensional normal distribution 
is investigated. Modeling has shown that multidimensionality of a~system
and relative\linebreak\vspace*{-12pt}}

\Abstend{
 correlation of components unaccounted in an explicit form, 
can lead to essential understating of risk factors. Results of calculation 
of the probability of a~dangerous outcome depending on numerical characteristics of 
a~multidimensional Gaussian random variable (a~covariance matrix and 
a~vector of mathematical expectations) are given. Approbation of the suggested model 
is executed by the example of the analysis of the risk of cardiovascular 
diseases in population. Models of risk management in the form of 
a~minimization problem or achievement of the given level are described. 
Control variables are the numerical characteristics of a~random vector covariance 
matrix and a~vector of mathematical expectations. Approbation of the method of 
risk management was carried 
out by means of statistical model operation by the Monte-Carlo method.}

\KWE{risk; model; stochastic system;  random vector; control; normal distribution}


\DOI{10.14357/19922264180208} %

%\vspace*{-14pt}

  \Ack
   \noindent
   The work was supported by the Russian Foundation
   for Basic Research (project 17-01-00315а).
   



%\vspace*{-3pt}

  \begin{multicols}{2}

\renewcommand{\bibname}{\protect\rmfamily References}
%\renewcommand{\bibname}{\large\protect\rm References}

{\small\frenchspacing
 {%\baselineskip=10.8pt
 \addcontentsline{toc}{section}{References}
 \begin{thebibliography}{99}
 
 \bibitem{3-t-1} %1
\Aue{Gore, A.} 1992. \textit{Earth in the balance. Forging a~new common purpose.} 
London: Earthscan Publications Ltd. 440~p.
\bibitem{1-t-1} %2
\Aue{Vorob'ev, Yu.\,L., G.\,G.~Malinetskiy, and N.\,A.~Makhutov}. 2000. Uprav\-le\-nie 
ris\-kom i~ustoy\-chi\-voe raz\-vi\-tie: che\-lo\-ve\-che\-skoe iz\-m\-ere\-nie [Management of risk and 
sustainable development: Human measurement]. \textit{Izvestiya vuzov. Pri\-klad\-naya 
nelineynaya dinamika}  [Proceedings of the Universities. Applied Nonlinear 
Dynamics] 8(6):12--26.
\bibitem{2-t-1} %3
\Aue{Porfir'ev, B.\,N.} 2006. Sni\-zhe\-nie pri\-rod\-nykh ris\-kov eko\-no\-mi\-che\-sko\-go 
raz\-vi\-tiya Ros\-sii: rol' go\-su\-dar\-st\-va
 [The reduction of natural risks of economic 
development of Russia: The role of the state]. \textit{Aktu\-al'\-nye prob\-le\-my 
grazh\-dan\-skoy za\-shchi\-ty: Mat-ly 11-y Mezhdunar.  
nauch.-praktich. konf. po problemam zashchity naseleniya i~territoriy ot\linebreak 
chrezvychaynykh situatsiy} [Actual Problems of Civil\linebreak Protection:  11th Scientific and 
Practical Conference (International) on Problems of Protection of the Population and 
Territories from Emergency Situations Proceedings]. 
N.~Novgorod: Vector-TiS. 44--50. Available at: {\sf 
http://gov.mari.ru/debzn/omgo/46.djvu} (accessed  August~7, 2017).

\bibitem{4-t-1}
\Aue{Porfir'ev,  B.\,N.} 1991. \textit{Up\-rav\-le\-nie v~chrez\-vy\-chay\-nykh si\-tu\-a\-tsi\-yakh. 
T.~1. Ito\-gi nau\-ki i~tekh\-ni\-ki. Prob\-le\-my bez\-opas\-nosti: 
chrezvychaynye situatsii} 
[Management in emergency situations. Vol.~1. The results of science and 
technology. Security concerns: Emergency situations]. Moscow: VINITI. 204~p.
\bibitem{5-t-1}
\Aue{Vishnyakov, Ya.\,D., and N.\,N.~Radaev.} 2008. \textit{Ob\-shhaya teo\-riya 
ris\-kov} [Common theory of risks]. 2nd ed. Moscow: Academy. 368~p.
\bibitem{6-t-1}
\Aue{Akimov, V.\,A.,  V.\,V.~Lesnykh, and N.\,N.~Radaev.} 2004. \textit{Riski 
v~pri\-ro\-de, tekh\-no\-sfe\-re, ob\-shchest\-ve i~eko\-no\-mi\-ke} 
[Risks in the nature, technosphere, 
society, and the economy]. Moscow: Business Express. 352~p.
\bibitem{7-t-1}
\Aue{Solozhentsev, E.\,D.} 2006. \textit{Stse\-nar\-noe 
lo\-gi\-ko-ve\-ro\-yat\-nost\-noe 
uprav\-le\-nie ris\-kom v~biz\-ne\-se i~tekh\-ni\-ke} [Scenario logic and probabilistic 
management of risk in business and engineering]. 2nd ed. St.\ Petersburg: Biznes 
pressa. 560~p.
\bibitem{8-t-1}
\Aue{Ryabinin, I.\,A.} 2000. \textit{Na\-dezh\-nost' i~bezopas\-nost'  
struk\-tur\-no-slozh\-nykh sis\-tem} [Reliability and safety of the structural and composite 
systems]. St.\ Petersburg: Polytechnique. 248~p.
\bibitem{9-t-1}
\Aue{Tyrsin, A.\,N.} 2015. O~mo\-de\-li\-ro\-va\-nii ris\-ka v~sis\-te\-makh kri\-tich\-nykh 
infra\-struk\-tur [About model operation of risk in the systems of critical infrastructures]. 
\textit{Economic and Technical Aspects of
Safety of Civil Engineering Critical Infrastructures
Conference (International) Abstracts}.
Ekaterinburg: Ural Federal University. 205--208.
 Available at: {\sf 
http://elar.urfu.ru/bitstream/10995/33468/1/safety\_\linebreak 2015.pdf}  (accessed August~7, 
2017).
\bibitem{10-t-1}
\Aue{Tyrsin, A.\,N., and A.\,A.~Surina.} 2017. Modelirovanie ris\-ka 
v~mno\-go\-mer\-nykh sto\-kha\-sti\-che\-skikh sis\-te\-makh [Modeling of risk in 
multidimensional stochastic systems]. \textit{Vestn. Tomskogo gosudarstvennogo 
universiteta. Upravlenie, vy\-chis\-li\-tel'\-naya tekh\-ni\-ka 
i~in\-for\-ma\-ti\-ka} [Bull. Tomsk State 
University. Management, Computer Facilities, and Informatics] 2(39):65--72.
\bibitem{11-t-1}
\Aue{Gnedenko, B.\,V.} 2005. \textit{Kurs teo\-rii ve\-ro\-yat\-no\-stey} [Course of 
probability theory]. 8th ed. Moscow: Editorial URSS. 448~p.
\bibitem{12-t-1}
\Aue{Pena, D., and J.~Rodriguez.} 2003. Descriptive measures of multivariate 
scatter and linear dependence.  \textit{J.~Multivariate Anal.} 85:361--374.
\bibitem{13-t-1}
\Aue{Kir'yanov, B.\,F., and M.\,S.~Tokmachev.} 2009. \textit{Ma\-te\-ma\-ti\-che\-skie 
modeli v~zdra\-vo\-okh\-ra\-ne\-nii} [Mathematical models in health care]. Veliky 
Novgorod: NovSU. 279~p.
\bibitem{14-t-1}
\Aue{Tsinker, M.\,Yu., D.\,A.~Kir'yakov, and M.\,R.~Kamaltdinov.} 2013. Pri\-me\-ne\-nie 
komp\-leks\-no\-go indek\-sa na\-ru\-she\-niya zdo\-rov'ya na\-se\-le\-niya dlya otsen\-ki 
po\-pu\-lya\-tsi\-on\-no\-go zdo\-rov'ya v~Permskom krae [The integrated index of health 
situation of the population to assess population health in the Perm region.
\textit{Izvestiya Samarskogo nauchnogo tsent\-ra RAN} [Proceedings of the Samara 
Scientific Center of RAS] 15(3(6)):1988--1992.
\bibitem{15-t-1}
Profilaktika khro\-ni\-che\-skikh ne\-in\-fek\-tsi\-on\-nykh za\-bo\-le\-va\-niy. 
Re\-ko\-men\-da\-tsii 
[Prevention of pre-existing conditions. Recommendations]. Moscow. 128~p.
Available at: {\sf 
http://www.webmed.irkutsk.ru/doc/pdf/prevent.pdf} (accessed August~7, 2017).
\bibitem{16-t-1}
\Aue{Tyrsin, A.\,N., O.\,F.~Kalev, D.\,A.~Yashin, and O.\,V.~Lebedeva.} 2015. 
Otsen\-ka so\-sto\-yaniya zdo\-rov'ya po\-pu\-lya\-tsii 
na osnove entropiynogo mo\-de\-li\-ro\-va\-niya 
[Assessment of health status of a~population on the basis of entropy modeling]. 
\textit{Math. Biol. Bioinf.} 10(1):206--219. doi: 10.17537/2015.10.206.
\bibitem{17-t-1}
\Aue{Mihaylov, G.\,A., and A.\,V.~Voytishek}. 2006. \textit{Chis\-len\-noe 
sta\-ti\-sti\-che\-skoe mo\-de\-li\-ro\-va\-nie. Metody Monte-Karlo} [Numerical statistical model 
operation. Monte-Carlo methods]. Moscow: Akademy. 368~p.
\bibitem{18-t-1}
\Aue{Panteleev, A.\,V., and T.\,A.~Letova.} 2008. \textit{Metody op\-ti\-mi\-za\-tsii 
v~pri\-me\-rakh i~za\-da\-chakh} [Optimization methods in examples and tasks]. 3rd ed. 
Moscow: Higher School. 544~p.
\end{thebibliography}

 }
 }

\end{multicols}

\vspace*{-3pt}

\hfill{\small\textit{Received August 21, 2017}}

%\vspace*{-24pt}

\Contr

\noindent
\textbf{Tyrsin Alexander N.} (b.\ 1961)~-- Doctor of Science in technology, Head of 
Department of Applied Mathematics, Ural Federal University named after first 
President of Russia B.\,N.~Yeltsin, 19~Mira Str., Ekaterinburg 620002, Russian 
Federation; senior scientist, Institute of Economics, Ural Branch of the Russian 
Academy of Sciences, 29~Moskovskaya Str., Yekaterinburg 620014, Russian 
Federation; \mbox{at2001@yandex.ru} 

\vspace*{3pt}

\noindent
\textbf{Surina Alfiya A.} (b.\ 1987)~--- PhD student, Institute of Natural Sciences, 
South Ural State University, 87~Lenin Ave., Chelyabinsk 454080, Russian 
Federation; \mbox{dallila87@mail.ru} 



\label{end\stat}


\renewcommand{\bibname}{\protect\rm Литература}   %6 
\def\stat{shestakov}

\def\tit{ОБРАЩЕНИЕ ОДНОРОДНЫХ ОПЕРАТОРОВ С~ПОМОЩЬЮ
СТАБИЛИЗИРОВАННОЙ ЖЕСТКОЙ ПОРОГОВОЙ ОБРАБОТКИ
ПРИ~НЕИЗВЕСТНОЙ ДИСПЕРСИИ ШУМА$^*$}

\def\titkol{Обращение однородных операторов с~помощью
стабилизированной жесткой пороговой обработки}
%при~неизвестной дисперсии шума}

\def\aut{О.\,В.~Шестаков$^1$}

\def\autkol{О.\,В.~Шестаков}

\titel{\tit}{\aut}{\autkol}{\titkol}

\index{Шестаков О.\,В.}
\index{Shestakov O.\,V.}


{\renewcommand{\thefootnote}{\fnsymbol{footnote}} \footnotetext[1]
{Работа выполнена при частичной финансовой поддержке РФФИ (проект 19-07-00352).}}


\renewcommand{\thefootnote}{\arabic{footnote}}
\footnotetext[1]{Московский государственный университет им.\ М.\,В.~Ломоносова, 
кафедра математической статистики факультета вычислительной математики и~кибернетики; 
Институт проб\-лем информатики Федерального исследовательского центра 
<<Информатика и~управ\-ле\-ние>> Российской академии наук, \mbox{oshestakov@cs.msu.su}}


\vspace*{-6pt}


\Abst{При обращении линейных однородных операторов обычно необходимо использовать 
методы регуляризации, поскольку наблюдаемые данные, как правило, зашумлены. 
Для подавления шума часто используется пороговая обработка 
вейвлет-ко\-эф\-фи\-ци\-ен\-тов функции наблюдаемого сигнала. 
Пороговая обработка стала популярным инструментом подавления 
шума благодаря своей простоте, вы\-чис\-ли\-тель\-ной эффективности и~воз\-мож\-ности 
адаптации к~функциям, имеющим на разных участках разную степень регулярности. 
Рассматривается предложенный недавно стабилизированный метод жесткой 
пороговой обработки, в~котором устранены основные недостатки мягкой и~жесткой 
пороговой обработки, и~исследуются статистические свойства этого метода. 
В~модели данных с~аддитивным гауссовским шумом с~неизвестной дисперсией 
проведен анализ несмещенной оценки среднеквадратичного риска и~показано, 
что при определенных условиях данная оценка является асимптотически нормальной, 
при этом дисперсия предельного распределения зависит от способа оценивания 
дисперсии шума.}

\KW{вейвлеты; пороговая обработка; несмещенная оценка риска; 
асимптотическая нормальность; сильная состоятельность}

\DOI{10.14357/19922264190107}
  
%\vspace*{4pt}


\vskip 10pt plus 9pt minus 6pt

\thispagestyle{headings}

\begin{multicols}{2}

\label{st\stat}

\section{Введение}

В медицинских, физических, астрономических и~других научных проблемах часто 
возникает задача получить представление об объекте, который описывается 
некоторой функцией~$f$, имея возможность наблюдать только функцию~$Kf$, где~$K$~--- 
некоторый линейный оператор. При этом часто нельзя просто применить 
к~наблюдаемым данным обратный оператор~$K^{-1}$, поскольку эти данные, как правило, 
содержат шум и~задача обращения оператора~$K$ некорректно поставлена. 
К~тому же обычно дис\-пер\-сия шума неизвестна и~ее необходимо оценивать 
по наблюдаемым данным. 

Одним из популярных инструментов при регуляризации 
процедуры обращения служит вейв\-лет-раз\-ло\-же\-ние с~последующей 
пороговой обработкой вейв\-лет-ко\-эф\-фи\-ци\-ен\-тов. Наиболее распростра\-нен\-ные 
виды пороговой обработки~--- жесткая и~мягкая. В~работе~\cite{HL10} 
был предложен метод стабилизированной жесткой пороговой обработки, который 
объединяет в~себе преимущества этих двух видов. 
В~ситуации, когда дисперсия шума предполагается известной, в~работе~\cite{SH18} 
доказана асимптотическая нормальность оценки среднеквадратичного риска пороговой 
обработки. 

В~данной работе исследуется влияние способов оценивания дисперсии шума 
на характеристики предельного распределения оценки среднеквадратичного риска. 
Для метода мягкой пороговой обработки подобные исследования проводились 
в~работах~\cite{KS11-1, KS11-2}.

\section{Обращение линейных однородных операторов с~помощью вейглет-вейвлет-разложения}

В данной работе рассматривается метод обращения линейных однородных операторов, 
основанный на вейг\-лет-вейв\-лет-раз\-ло\-же\-нии~\cite{AS98}. Линейный оператор~$K$ 
называется однородным, если
$$
K\left[f\left(a\left(x-x_0\right)\right)\right]=a^{-\alpha}(Kf)\left[a\left(x-x_0\right)\right]
$$
для любого $x_0$ и~любого $a\hm>0$. Параметр~$\alpha$ называется показателем 
однородности. Примерами линейных однородных операторов служат оператор 
интегрирования, преобразование Гильберта и~преобразование Абеля.

Относительно наблюдаемой функции~$Kf$ будем предполагать, что она определена на 
конечном отрезке и~равномерно регулярна по Липшицу с~некоторым показателем $\gamma\hm>0$. 
Вейв\-лет-разложение~$Kf$ представляет собой ряд по ортонормированному базису
\begin{equation}
\label{wavelet_decomp}
Kf = \sum\limits_{j,k \in Z} \langle Kf,\psi_{j,k} \rangle \psi_{j,k}\,,
\end{equation}
где $\psi(t)$~--- некоторая материнская вейв\-лет-функ\-ция, 
а~$\psi_{j,k}(t) \hm= 2^{j/2}\psi(2^jt \hm- k)$. Индекс~$j$ в~(\ref{wavelet_decomp}) 
называется масштабом, а~индекс~$k$~--- сдвигом. Если вейв\-лет-функ\-ция 
обладает определенными свойствами регулярности~\cite{Mal99}, 
то для коэффициентов разложения в~(\ref{wavelet_decomp}) справедливо
\begin{equation}
\label{wavelet_decay}
\abs{\langle Kf, \psi_{j,k} \rangle} \leqslant \fr{C_f}
{2^{j \left( \gamma + 1/2 \right)}}\,,
\end{equation}
где $C_f$~--- некоторая положительная константа.

Поскольку оператор~$K$ линеен и~однороден, существуют такие функции~$u_{j,k}$, 
что $\langle f,u_{j,k}\rangle\hm=\langle Kf,\psi_{j,k}\rangle$. При этом функция~$f$ 
представляется в~виде ряда
\begin{equation}
\label{VWD}
f = \sum\limits_{j,k \in Z}\beta_{j,k}\langle Kf,\psi_{j,k}\rangle u_{j,k},
\end{equation}
где $u_{j,k} = K^{-1}\psi_{j,k}/\beta_{j,k}$, $\beta_{j,k}\hm=2^{\alpha j}\beta_{00}$, 
$\beta_{00} \hm= \norm{K^{-1}\psi}$ (функции~$u_{j,k}$, как и~$\psi_{j,k}$, 
представляют собой сдвиги и~растяжения одной материнской функции~$u$ и~называются 
вейглетами). При соответствующем выборе~$\psi(t)$ последовательность~$\{u_{j,k}\}$ 
образует устойчивый базис~\cite{L97}. Формула~(\ref{VWD}) и~есть основа метода 
вейг\-лет-вейв\-лет-раз\-ло\-же\-ния.

\section*{Пороговая обработка эмпирических коэффициентов}

При фактических измерениях значения функции сигнала регистрируются 
в~дискретных отсчетах, при этом такие значения, как правило, зашумлены. 
Рассмотрим сле\-ду\-ющую модель данных \mbox{с~шумом}:
\begin{equation*}
%\label{Data_Model}
X_i = (Kf)_i + \epsilon_i\,, \enskip i = 1, \dots, 2^J\,, %\notag
\end{equation*}
где $2^J$~--- число отсчетов; $(Kf)_i$~--- незашумленные значения функции сигнала; 
$\epsilon_i$~--- независимые нормально распределенные случайные величины с~нулевым 
средним и~дисперсией~$\sigma^2$.
После применения дискретного вейв\-лет-пре\-об\-ра\-зо\-ва\-ния 
получается следующая модель зашумленных вейв\-лет-ко\-эф\-фи\-ци\-ен\-тов:
\begin{equation*}
Y_{j,k}=\mu_{j,k}+\epsilon^W_{j,k},\enskip 
j=0,\ldots,J-1,\ k=0,\ldots,2^{j}-1\,,
\end{equation*}
где $\epsilon^W_{j,k}$ независимы и~распределены так же, как и~$\epsilon_i$, 
а~$\mu_{j,k}\hm= 2^{J/2}\langle Kf,\psi_{j,k}\rangle$~\cite{Mal99}.

Для подавления шума и~построения оценки функции сигнала к~коэффициентам~$Y_{j,k}$ 
обычно применяется функция жесткой пороговой обработки 
$\rho_{H}(y,T)\hm=x\textbf{I}(\abs{y}>T)$ или мягкой пороговой 
обработки $\rho_{S}(y,T)\hm=\textbf{sgn}(x)\left(\abs{y}-T\right)_{+}$ 
с~порогом~$T$. При таком подходе обнуляются коэффициенты, абсолютная величина 
которых ниже порога, так как в~силу~(\ref{wavelet_decay}) основная часть
 полезного сигнала содержится в~относительно небольшом числе больших по 
 модулю коэффициентов.

Каждому из этих видов пороговой обработки присущи свои недостатки. 
Жесткая пороговая функция разрывна, и~это приводит к~отсутствию устойчивости 
при выборе порога~\cite{B96} и~невозможности построения несмещенной оценки 
среднеквадратичного риска~\cite{J01}. При мягкой пороговой обработке в~оценке 
функции появляется дополнительное смещение. Чтобы частично избежать этих недостатков, 
в~работе~\cite{HL10} был предложен новый вид пороговой обработки, представляющий 
собой сглаженный (стабилизированный) аналог жесткой пороговой обработки. 
В~этом методе оценки~$\mu_{j,k}$ вычисляются по формулам:
\begin{equation*}
\widehat{\mu}_{j,k}=\Expect 
\left[\rho_{H}(Y_{j,k}+\lambda\xi_{j,k},T_j)|Y_{j,k}\right], %\notag
\end{equation*}
где случайные величины~$\xi_{j,k}$ имеют стандартное нормальное распределение и~не 
зависят от~$Y_{j,k}$, а~$\lambda\hm>0$~--- 
параметр стабилизации, отвечающий за степень сглаживания. Вычисляя математическое 
ожидание, получаем:
\begin{multline*}
\hspace*{-8.37947pt}\widehat{\mu}_{j,k}=Y_{j,k}\left[\Phi\!\left(-\fr{T_j+Y_{j,k}}
{\lambda}\right)+1-\Phi\left(\fr{T_j-Y_{j,k}}{\lambda}\right)\!\right]+{}\\
{}+
\lambda\left[\phi\left(\fr{T_j-Y_{j,k}}{\lambda}\right)-
\phi\left(\fr{T_j+Y_{j,k}}{\lambda}\right)\right]. %\notag
\end{multline*}
Достоинством такого метода является бесконечная дифференцируемость~$\widehat{\mu}_{j,k}$ 
по~$Y_{j,k}$, что приводит к~более робастным оценкам~\cite{HL10}. Заметим также, 
что при $\lambda\hm\to0$ получается обычный метод жесткой пороговой обработки. 
В~данной работе параметр~$\lambda$ предполагается фиксированным, а~в~качестве~$T_j$ 
для каждого масштаба~$j$ выбирается порог $T_j\hm=\sigma\sqrt{2\ln 2^j}$. 
Такой порог получил название <<универсальный>>, так как он не зависит 
от наблюдаемых данных. И~при жесткой, и~при мягкой пороговой обработке этот 
порог обеспечивает близость среднеквадратичного риска к~минимальному~\cite{Mal99}.

\section{Несмещенная оценка среднеквадратичного риска}

Среднеквадратичный риск метода пороговой обработки определяется по формуле:
\begin{equation}
\label{Risk}
R_J(\sigma)=\sum\limits_{j=0}^{J-1}\sum\limits_{k=0}^{2^j-1}\beta^2_{j,k}
\Expect\left(\widehat{\mu}_{j,k}(\sigma)-\mu_{j,k}\right)^2.
\end{equation}
В~\cite{HL10} показано, что при стабилизированной жесткой пороговой обработке
\begin{multline*}
\Expect\left(\widehat{\mu}_{j,k}(\sigma)-\mu_{j,k}\right)^2={}\\
{}=
\Expect\left[(Y_{j,k}-\widehat{\mu}_{j,k}(\sigma))^2+
2\sigma^2\fr{\partial}{\partial Y_{j,k}}\,\widehat{\mu}_{j,k}(\sigma)\right]-
\sigma^2, %\notag
\end{multline*}
где
\begin{multline*}
\fr{\partial}{\partial Y_{j,k}}\widehat{\mu}_{j,k}(\sigma)={}\\
{}=\Phi\left(-\fr{T_j+Y_{j,k}}{\lambda}\right)+1-
\Phi\left(\fr{T_j-Y_{j,k}}{\lambda}\right)+{}\\
{}+
\fr{T_j}{\lambda}\left[\phi\left(\fr{T_j-Y_{j,k}}{\lambda}\right)+
\phi\left(\fr{T_j+Y_{j,k}}{\lambda}\right)\right]. %\notag
\end{multline*}
Таким образом, величина
\begin{multline}
\label{Risk_Estimate}
\widehat{R}_J(\sigma)=\sum\limits_{j=0}^{J-1}\sum\limits_{k=0}^{2^j-1}
\beta^2_{j,k}
\Bigg[
\left(
Y_{j,k}-
\widehat{\mu}_{j,k}(\sigma)\right)^2+{}\\
{}+2\sigma^2\fr{\partial}{\partial Y_{j,k}}\,\widehat{\mu}_{j,k}(\sigma)-
\sigma^2
\Bigg]
\end{multline}
является несмещенной оценкой~$R_J$, не зависящей от ненаблюдаемых значений~$\mu_{j,k}$.

В работе~\cite{SH18} доказано следующее утверждение, устанавливающее 
асимптотическую нормальность оценки~(\ref{Risk_Estimate}) и~позволяющее строить 
асимптотические доверительные интервалы для риска~(\ref{Risk}).

\smallskip

\noindent
\textbf{Теорема 1.} 
\textit{Пусть $K$~--- линейный однородный оператор с~показателем 
однородности $\alpha\hm>0$, а~$Kf$ задана на конечном отрезке и~равномерно 
регулярна по Липшицу с~показателем $\gamma\hm>0$. Тогда}
\begin{equation*}
%\label{Normality}
{\sf P}\left(\fr{\widehat{R}_J(\sigma)-
R_J(\sigma)}{D_J}<x\right)\Rightarrow\Phi(x)\,, %\notag
\end{equation*}
\textit{где}
$$
D^2_J=\fr{2\sigma^4\beta_{0,0}^4}{2^{4\alpha+1}-1}2^{(4\alpha+1)J}\,.
$$

\section{Виды оценок дисперсии шума}

Как правило, дисперсия~$\sigma^2$ неизвестна и~вместо ее точного значения 
необходимо использовать некоторую оценку~$\hat{\sigma}^2$, которая обычно 
строится по половине всех вейв\-лет-ко\-эф\-фи\-ци\-ен\-тов для $j\hm=J\hm-1$, 
так как в~силу~(\ref{wavelet_decay}) эти коэффициенты фактически содержат только шум. 
При этом порог вычисляется по формуле $\hat{T}_j\hm=\hat{\sigma}\sqrt{2\ln 2^j}$.

В качестве оценки~$\sigma^2$ (или $\sigma$) в~данной работе 
рассматривается выборочная дисперсия
\begin{equation}
\label{SampleVarianceDef}
\widehat{\sigma}_S^2=\fr{1}{2^{J-1}}
\sum\limits_{k=0}^{2^{J-1}-1}Y_{J-1,k}^2-\overline{Y}^2,
\end{equation}
где
\begin{equation*}
\overline{Y}=\fr{1}{2^{J-1}}\sum\limits_{k=0}^{2^{J-1}-1}Y_{J-1,k}\,,
\end{equation*}
а также соответствующим образом нормированный выборочный интерквартильный 
размах~$\widehat{\sigma}_{R}$ и~выборочное абсолютное медианное 
отклонение~$\widehat{\sigma}_{M}$, которые определяются сле\-ду\-ющим образом:
\begin{align}
\widehat{\sigma}_{R}&=\fr{Y_{(J-1,3/4)}-Y_{(J-1,1/4)}}{2\xi_{3/4}}\,;
\label{IQR_Definition}
\\
\widehat{\sigma}_{M}&=\fr{\mathop{\mbox{med}}\limits_{0\leqslant k\leqslant 2^{J-1}-1}|Y_{J-1,k}-\mathop{\mbox{med}}\limits_{0\leqslant l\leqslant 2^{J-1}-1} Y_{J-1,l}|}{\xi_{3/4}}\,.
\label{MAD_Definition}
\end{align}
Здесь $Y_{(J-1,1/4)}$ и~$Y_{(J-1,3/4)}$~--- выборочные квантили порядка~$1/4$ и~$3/4$, 
построенные по выборке из половины всех вейв\-лет-ко\-эф\-фи\-ци\-ен\-тов при 
$j\hm=J\hm-1$; $\xi_{3/4}$~--- теоретическая квантиль порядка~$3/4$ 
стандартного нормального распределения ($\xi_{3/4}\hm\approx0,6745$); $\mbox{med}$ 
обозначает выборочную медиану.

Выборочная дисперсия служит самой популярной оценкой величины~$\sigma^2$, и~в~случае 
отсутствия выбросов она наиболее предпочтительна. Однако в~случае, когда 
оценка дисперсии строится по выборке сигнала, естественно ожидать, 
что выборка не будет однородной. Преимущество использования последних 
двух оценок заключается в~их ро\-баст\-ности, т.\,е.\ нечувствительности к~выбросам.

\section{Предельная дисперсия оценки среднеквадратичного риска}

Способ оценивания дисперсии шума влияет на вид предельной дисперсии 
оценки среднеквадратичного риска. Подобный эффект наблюдается и~при 
мягкой пороговой обработке~[4].

\noindent
\textbf{Теорема~2.}\ \textit{Пусть $Kf$ задана на конечном отрезке и~равномерно 
регулярна по Липшицу с~показателем $\gamma\hm>1/4$, а оценка дисперсии 
шума задана соотношением}~\eqref{SampleVarianceDef}. \textit{Тогда}
\begin{equation}
\label{CLT_Operator_SampleVar_Sigma}
\mathsf{P}\left(\frac{\widehat{R}_J(\widehat{\sigma}_S)-R_J(\sigma)}{D_J}<x\right)
\Rightarrow \Phi_{\Upsilon_1}(x),\notag
\end{equation}
\textit{где $\Phi_{\Upsilon_1}(x)$~--- функция распределения нормального 
закона с~нулевым средним и~дисперсией}
$$
\Upsilon_1^2=\fr{1}{2^{4\alpha+1}}+
\fr{2^{4\alpha+1}-1}{2^{4\alpha+1}\left(2^{2\alpha+1}-1\right)^2}\,.
$$

\noindent
Д\,о\,к\,а\,з\,а\,т\,е\,л\,ь\,с\,т\,в\,о\,.\ \ Обозначим
\begin{multline*}
\widehat{U}_J(\sigma)=\sum\limits_{j=0}^{J-1}\sum\limits_{k=0}^{2^j-1}
\beta^2_{j,k}\Bigg[
\left(Y_{j,k}-\widehat{\mu}_{j,k}(\sigma)\right)^2+{}\\
{}+2\sigma^2\fr{\partial}{\partial Y_{j,k}}\widehat{\mu}_{j,k}(\sigma)\Bigg] %\notag
\end{multline*}
и запишем $\widehat{R}_J(\hat{\sigma}_S)-R_J(\sigma)$ в~виде
\begin{multline*}
%\label{Three_Sums}
\widehat{R}_J(\hat{\sigma}_S)-R_J(\sigma)={}\\
{}=\left[\widehat{U}_J(\hat{\sigma}_S)-\widehat{U}_J(\sigma)\right]+
\left[\widehat{R}_J(\sigma)-R_J(\sigma)\right]+{}\\
{}+
\fr{2^{(2\alpha+1)J}-1}{2^{2\alpha+1}-1}(\sigma^2-\hat{\sigma}^2_S)
\equiv S_1+S_2+S_3\,.
\end{multline*}

Повторяя рассуждения из работ~\cite{KS11-1, KS11-2} и~учитывая, что если $\gamma\hm>1/4$, 
то выполнено $2^{J/2}\overline{Y}^2\stackrel{{\sf P}}{\to} 0$ при 
$J\hm\rightarrow\infty$~\cite{KS11-2}, можно показать, что
\begin{equation*}
{\sf P}\left(\fr{S_2+S_3}{D_J}<x\right)\Rightarrow\Phi_{\Upsilon_1}(x)\,.%\notag
\end{equation*}
% на самом деле с~условием Линдеберга чуть по-другому (без ограниченности слагаемых). Но дисперсия равномерно ограничена -- значит выполнено.

Докажем, что $D_J^{-1}S_1\stackrel{{\sf P}}{\to}0$ при $J\hm\rightarrow\infty$. 
Пусть $C_\delta\hm>0$~--- некоторая константа, а $\delta_J\hm=C_\delta J^{1/2}2^{-J/2}$. 
Запишем
\begin{multline*}
S_1=\mathbf{1}\left(\abs{\sigma^2-\hat{\sigma}^2_S}>\delta_J\right)S_1+{}\\
{}+
\mathbf{1}\left(\abs{\sigma^2-\hat{\sigma}^2_S}\leqslant\delta_J\right)
S_1\equiv S'_1+S''_1. %\notag
\end{multline*}
Для произвольного $\varepsilon\hm>0$
\begin{equation*}
{\sf P}\left(S'_1>\varepsilon\right)\leqslant{\sf P}
\left(\abs{\sigma^2-\hat{\sigma}^2_S}>\delta_J\right). %\notag
\end{equation*}
При выполнении условий теоремы, если константа~$C_\delta$ достаточно велика, 
то найдется константа~$\tilde{C}_\delta>0$ такая, что~\cite{KS11-2}
\begin{equation*}
{\sf P}\left(\abs{\sigma^2-\hat{\sigma}^2_S}>\delta_J\right)
\leqslant\tilde{C}_\delta2^{-J/2}. %\notag
\end{equation*}
%% комментарии по поводу этого неравенства и~загрязнения выборки есть в~диссертации
Следовательно, $S'_1\stackrel{P}{\to}0$ при $J\hm\rightarrow\infty$.

Обозначим слагаемые в~сумме~$S''_1$ через~$F_{j,k}(\hat{\sigma}_S)$. Пусть 
$A_j\hm=\sqrt{A\ln 2^j}$, где $0\hm<A\hm<2(\sigma^2\hm-\delta_J)$. Имеем:

\noindent
\begin{multline*}
\hspace*{-9.9pt}\sum\limits_{j=0}^{J-1}\sum\limits_{k=0}^{2^j-1}F_{j,k}\left(\hat{\sigma}_S\right)=
\sum\limits_{j=0}^{J-1}\sum\limits_{k=0}^{2^j-1}
\mathbf{1}(\abs{Y_{j,k}}\leqslant A_j)F_{j,k}(\hat{\sigma}_S)+{}\\
{}+
\sum\limits_{j=0}^{J-1}\sum\limits_{k=0}^{2^j-1}
\mathbf{1}\left(\abs{Y_{j,k}}>A_j\right)F_{j,k}(\hat{\sigma}_S)
\equiv  W_1+W_2. %\notag
\end{multline*}
Рассмотрим $W_1$. Учитывая определения $\widehat{\mu}_{j,k}(\sigma)$, 
$({\partial}/{\partial Y_{j,k}})\widehat{\mu}_{j,k}(\sigma)$ и~$A_j$, 
можно убедиться, что найдут\-ся константы $C_1\hm>0$ и~$\theta\hm>0$ такие, что
\begin{equation*}
\abs{\mathbf{1}\left(\abs{Y_{j,k}}\leqslant A_J\right)
F_{j,k}(\hat{\sigma}_S)}\leqslant C_1 
J^{5/2}2^{(2\alpha-\theta)j-J/2}\;\;\mbox{п.в.} %\notag
\end{equation*}
% поскольку выполнено \mathbf{1}(\abs{\sigma^2-\hat{\sigma}^2_S}\leqslant\delta_J). В логарифме степень: от Y идет 1, от T идет 1, от \delta_J идет 1/2 но для J, а не для j, поэтому берем для всех J^{5/2}. В степени 2: 2\alpha от \beta{j,k}, \theta из-за выбора A, J/2 от \delta_J
Следовательно, $D_J^{-1}W_1\hm\rightarrow 0$ п.в.\ при $J\hm\rightarrow\infty$.

Далее для слагаемых~$W_2$ имеем:
\begin{multline*}
\left\vert \mathbf{1}\left(
\left\vert Y_{j,k}\right\vert
> A_J\right)F_{j,k}
\left(\hat{\sigma}_S\right)\right\vert
\leqslant{}\\
{}\leqslant C_2 J^{3/2}2^{2\alpha j-J/2} 
\mathbf{1}\left( \left\vert Y_{j,k}\right\vert > A_J\right) 
\left\vert Y_{j,k}\right\vert^2\;\;\mbox{п.в.},
%\notag
\end{multline*}
% поскольку выполнено \mathbf{1}(\abs{\sigma^2-\hat{\sigma}^2_S}\leqslant\delta_J). В логарифме от T идет 1, от \delta_J идет 1/2.
где $C_2>0$~--- некоторая константа. Учитывая распределение~$Y_{j,k}$, 
нетрудно убедиться, что
\begin{equation*}
\Expect\frac{1}{D_J} \sum\limits_{j=0}^{J-1}
\sum\limits_{k=0}^{2^j-1} J^{3/2}2^{2\alpha j-J/2} 
\mathbf{1}\left(\abs{Y_{j,k}}> A_j\right)
\abs{Y_{j,k}}^2\to 0
\end{equation*}
при $J\rightarrow\infty$. %\notag
Следовательно, используя неравенство Маркова, получаем, что
\begin{equation*}
D_J^{-1}W_2\stackrel{{\sf P}}{\to}0\;\;\mbox{при}\;J\rightarrow\infty\,. %\notag
\end{equation*}
Таким образом, $D_J^{-1}S_1\stackrel{{\sf P}}{\to}0$ при $J\hm\rightarrow\infty$.

Теорема доказана.

\smallskip

Рассмотрим теперь ситуацию, когда в~качестве оценки~$\sigma$ используется 
величина~$\widehat{\sigma}_{R}$ или~$\widehat{\sigma}_{M}$. 
В~этом случае повышаются требования к~гладкости функции сигнала.

\smallskip

\noindent
\textbf{Теорема~3.}\
\textit{Пусть~$Kf$ задана на конечном отрезке и~равномерно регулярна по 
Липшицу с~показателем $\gamma\hm>1/2$, а оценка дисперсии шума~$\hat{\sigma}$ 
задана соотношением}~\eqref{IQR_Definition} 
\textit{или соотношением}~\eqref{MAD_Definition}. \textit{Тогда}
\begin{equation*}
\label{CLT_Operator_RobVar_Sigma}
\mathsf{P}\left(\fr{\widehat{R}_J(\widehat{\sigma})-R_J(\sigma)}{D_J}<x\right)
\Rightarrow \Phi_{\Upsilon_2}(x)\,, %\notag
\end{equation*}
где $\Phi_{\Upsilon_2}(x)$~--- функция распределения нормального закона 
с~нулевым средним и~дисперсией
\begin{multline*}
\Upsilon_2^2=1+\fr{2^{4\alpha+1}-1}{4(2^{2\alpha+1}-1)^2
\xi_{3/4}^2(\phi(\xi_{3/4}))^2}-{}\\
{}-
\fr{2^{4\alpha+1}-1 }{2^{2\alpha-1}(2^{2\alpha+1}-1)}\,.
\end{multline*}

\noindent
Д\,о\,к\,а\,з\,а\,т\,е\,л\,ь\,с\,т\,в\,о\,.\ \
Как и~в~предыдущей теореме, запишем
$\widehat{R}_J(\hat{\sigma})\hm-R_J(\sigma)\hm=S_1\hm+S_2\hm+S_3.$
Учитывая,\linebreak\vspace*{-12pt}

\pagebreak

\noindent
 что $\gamma\hm>1/2$, и~поступая, как в~работах~\cite{SH18, KS11-2, SH12}, 
с~использованием разложения Бахадура для выборочных квантилей~\cite{S80} и~выборочного 
абсолютного медианного отклонения~\cite{SM09}, можно показать, что
\begin{equation*}
{\sf P}\left(\fr{S_2+S_3}{D_J}<x\right)\Rightarrow\Phi_{\Upsilon_2}(x)\,. %\notag
\end{equation*}
% на самом деле с~условием Линдеберга чуть по-другому (без ограниченности слагаемых). Но дисперсия равномерно ограничена -- значит выполнено.

Используя экспоненциальные неравенства для выборочных квантилей~\cite{S80} 
и~выборочного абсолютного медианного отклонения~\cite{SM09}, получаем, что при 
выполнении условий теоремы найдется такая константа $C_\delta\hm>0$, что при 
$\delta_J\hm=C_\delta J^{1/2}2^{-J/2}$ для некоторой константы~$\widetilde{C}_\delta>0$ 
выполнено:
\begin{align*}
\mathsf{P}\left(\abs{\widehat{\sigma}_{R}-\sigma}>\delta_J\right)
&\leqslant\widetilde{C}_\delta2^{-J/2}\,;
\\
\mathsf{P}\left(\abs{\widehat{\sigma}_{M}-\sigma}>\delta_J\right)
&\leqslant\widetilde{C}_\delta2^{-J/2}\,. %\notag
\end{align*}
%% комментарии по поводу этого неравенства и~загрязнения выборки есть в~диссертации
Далее, повторяя рассуждения предыдущей теоремы, заключаем, что 
$D_J^{-1}S_1\stackrel{{\sf P}}{\to}0$ при $J\hm\rightarrow\infty$.


Теорема доказана.



{\small\frenchspacing
 {%\baselineskip=10.8pt
 \addcontentsline{toc}{section}{References}
 \begin{thebibliography}{99}

\bibitem{HL10}
\Au{Huang H.-C., Lee~T.\,C.\,M.} 
Stabilized thresholding with generalized sure for image denoising~// 
IEEE 17th  Conference (International) on Image Processing
Proceedings.~--- IEEE, 2010. P.~1881--1884.

\bibitem{SH18}
\Au{Shestakov O.\,V.} 
Nonlinear regularization of inverse problems for linear homogeneous transforms 
by the stabilized hard thresholding~// J.~Math. Sci., 2018. Vol.~234. No.\,6. P.~780--785.

\bibitem{KS11-1}
\Au{Кудрявцев А.\,А., Шестаков~О.\,В.} 
Асимптотика оценки риска при вейг\-лет-вейв\-лет разложении наблюдаемого сигнала~// 
T-Comm~--- телекоммуникации и~транспорт, 2011. №\,2. С.~54--57.

\bibitem{KS11-2}
\Au{Кудрявцев А.\,А., Шестаков~О.\,В.} 
Асимптотическое распределение оценки риска пороговой обработки 
вейг\-лет-ко\-эф\-фи\-ци\-ен\-тов сигнала при неизвестном уровне шума~// 
T-Comm~--- телекоммуникации и~транспорт, 2011. №\,5. С.~24--30.

\bibitem{AS98}
\Au{Abramovich F., Silverman~B.\,W.} 
Wavelet decomposition approaches to statistical inverse problems~// 
Biometrika, 1998. Vol.~85. No.\,1. P. 115--129.

\bibitem{Mal99}
\Au{Mallat S.} A~Wavelet tour of signal processing.~--- 
New York, NY, USA: Academic Press, 1999. 857~p.

\bibitem{L97}
\Au{Lee N.} Wavelet-vaguelette decompositions and homogenous equations.~--- 
West Lafayette, IN, USA: Purdue University, 1997.  PhD Thesis. 103~p.

\bibitem{B96}
\Au{Breiman L.} Heuristics of instability and stabilization in model selection~// 
Ann. Stat., 1996. Vol.~24. No.\,6. P.~2350--2383.

\bibitem{J01}
\Au{Jansen M.} Noise reduction by wavelet thresholding.~--- 
Lecture notes in statistics ser.~--- New York, NY, USA: Springer Verlag,
2001. Vol.~161. 196~p.

\bibitem{SH12}
\Au{Шестаков О.\,В.} О~скорости сходимости оценки риска пороговой обработки 
вейв\-лет-ко\-эф\-фи\-ци\-ен\-тов к~нормальному закону при использовании 
робастных оценок дисперсии~// Информатика и~её применения, 2012. Т.~6. Вып.~2. 
С.~122--128.

\bibitem{S80}
\Au{Serfling R.} Approximation theorems of mathematical statistics.~--- 
New York, NY, USA: John Wiley \& Sons, 1980. 371~p.

\bibitem{SM09}
\Au{Serfling R., Mazumder~S.} 
Exponential probability inequality and convergence results for the median 
absolute deviation and its modifications~// Stat. Probabil. Lett., 2009. 
Vol.~79. No.\,16. P.~1767--1773.
 \end{thebibliography}

 }
 }

\end{multicols}

\vspace*{-3pt}

\hfill{\small\textit{Поступила в~редакцию 14.12.18}}

\vspace*{8pt}

%\pagebreak

%\newpage

%\vspace*{-28pt}

\hrule

\vspace*{2pt}

\hrule

%\vspace*{-2pt}

\def\tit{INVERSION OF~HOMOGENEOUS OPERATORS USING~STABILIZED HARD THRESHOLDING 
WITH~UNKNOWN NOISE VARIANCE}

\def\titkol{Inversion of~homogeneous operators using~stabilized hard thresholding 
with~unknown noise variance}

\def\aut{O.\,V.~Shestakov}

\def\autkol{O.\,V.~Shestakov}

\titel{\tit}{\aut}{\autkol}{\titkol}

\vspace*{-11pt}


\noindent
Department of Mathematical Statistics, Faculty of Computational Mathematics and Cybernetics, M.V. Lomonosov Moscow State University, 1-52 Leninskiye Gory, GSP-1, Moscow 119991, Russian Federation
Institute of Informatics Problems, Federal Research Center 
``Computer Science and Control'' of the Russian Academy of Sciences, 44-2~Vavilov Str., 
Moscow 119333, Russian Federation

\def\leftfootline{\small{\textbf{\thepage}
\hfill INFORMATIKA I EE PRIMENENIYA~--- INFORMATICS AND
APPLICATIONS\ \ \ 2019\ \ \ volume~13\ \ \ issue\ 1}
}%
 \def\rightfootline{\small{INFORMATIKA I EE PRIMENENIYA~---
INFORMATICS AND APPLICATIONS\ \ \ 2019\ \ \ volume~13\ \ \ issue\ 1
\hfill \textbf{\thepage}}}

\vspace*{6pt}



\Abste{When inverting linear homogeneous operators, it is necessary to use 
regularization methods, since observed data are usually noisy. For noise suppression, 
threshold processing of  wavelet coefficients of the observed signal function 
is often used. Threshold processing has become a~popular noise suppression tool 
due to its simplicity, computational efficiency, and ability to adapt to functions 
that have different degrees of regularity at different domains. The paper 
discusses the recently proposed stabilized hard thresholding method that eliminates 
the main
drawbacks of soft and hard thresholding methods and studies statistical 
properties of this method. In the data model\linebreak\vspace*{-12pt}}

\Abstend{with an additive Gaussian noise with 
unknown variance, an unbiased estimate of the mean square risk is analyzed and it 
is shown that under certain conditions, this estimate is asymptotically normal and 
the variance of the limit distribution depends on the type of estimate of noise variance.}


\KWE{wavelets; threshold processing; unbiased risk estimate; asymptotic normality;
strong consistency}




\DOI{10.14357/19922264190107}

%\vspace*{-14pt}

\Ack
\noindent
This research was partly supported by the Russian  
Foundation for Basic Research (project No.\,19-07-00352).




%\vspace*{6pt}

  \begin{multicols}{2}

\renewcommand{\bibname}{\protect\rmfamily References}
%\renewcommand{\bibname}{\large\protect\rm References}

{\small\frenchspacing
 {%\baselineskip=10.8pt
 \addcontentsline{toc}{section}{References}
 \begin{thebibliography}{99}
\bibitem{1-sh-1}
\Aue{Huang, H.-C., and T.\,C.\,M.~Lee.} 2010. 
Stabilized thresholding with generalized sure for image denoising. 
\textit{IEEE 17th Conference (International) on Image Processing}. IEEE. 1881--1884.

 

\bibitem{2-sh-1}
\Aue{Shestakov, O.\,V.} 2018. 
Nonlinear regularization of inverse problems for linear homogeneous transforms 
by the stabilized hard thresholding. 
\textit{J.~Math. Sci.} 234(6):780--785.

\bibitem{3-sh-1}
\Aue{Kudryavtsev, A.\,A., and O.\,V.~Shestakov.} 2011. Аsimptotika otsenki riska pri 
veyglet-veyvlet razlozhenii nablyuda\-emo\-go signala [The average risk assessment 
of the wavelet decomposition of the signal].
\textit{T-Comm~--- Telecommunications and Their Application in
Transport Industry} 2:54--57.

\bibitem{4-sh-1}
\Aue{Kudryavtsev, A.\,A., and O.\,V.~Shestakov.} 2011. Аsimptoticheskoe raspredelenie 
otsenki riska porogovoy ob\-ra\-bot\-ki veyglet-koeffitsientov signala pri 
neizvestnom urovne shuma [Asymptotic distribution of the risk estimate of 
the signal vaguelette coefficients thresholding at the unknown noise level]. 
\textit{T-Comm~--- Telecommunications and Their Application in
Transport Industry} 5:24--30.

\bibitem{5-sh-1}
\Aue{Abramovich, F., and B.\,W.~Silverman.} 1998. Wavelet 
decomposition approaches to statistical inverse problems. 
\textit{Biometrika} 85(1):115--129.

\bibitem{6-sh-1}
\Aue{Mallat, S.} 1999. \textit{A~wavelet tour of signal processing.} New York, NY: 
Academic Press. 857 p.

\bibitem{7-sh-1}
\Aue{Lee, N.} 1997. Wavelet-vaguelette decompositions and homogenous equations. 
 West Lafayette, IN: Purdue University. PhD Thesis. 103~p.

\bibitem{8-sh-1}
\Aue{Breiman, L.} 1996. 
Heuristics of instability and stabilization in model selection. 
\textit{Ann. Stat.} 24(6):2350--2383.

\bibitem{9-sh-1}
\Aue{Jansen, M.} 2001. \textit{Noise reduction by wavelet thresholding.} 
Lecture notes in statistics ser.
New York, NY: Springer Verlag.  Vol.~161. 196~p.

\bibitem{10-sh-1}
\Aue{Shestakov, O.\,V.} 2012. O~skorosti skhodimosti otsenki riska porogovoy 
obrabotki veyvlet-koeffitsientov k~nor\-mal'\-no\-mu zakonu pri ispol'zovanii robastnykh 
otsenok dispersii [On the rate of convergence to the normal law of risk estimate for 
wavelet coefficients thresholding when using robust variance estimates]. 
\textit{Informatika i~ee Primeneniya~--- Inform. Appl.}  6(2):122--128.

\bibitem{11-sh-1}
\Aue{Serfling, R.} 1980. \textit{Approximation theorems of mathematical statistics}.
New York, NY: John Wiley \& Sons. 371~p.

\bibitem{12-sh-1}
\Aue{Serfling, R., and S.~Mazumder.} 2009. Exponential probability inequality 
and convergence results for the median absolute deviation and its modifications. 
\textit{Stat. Probabil. Lett.} 79(16):1767--1773.
\end{thebibliography}

 }
 }

\end{multicols}

\vspace*{-6pt}

\hfill{\small\textit{Received December 14, 2018}}

%\pagebreak

%\vspace*{-18pt}  

\Contrl

\noindent
\textbf{Shestakov Oleg V.} (b.\ 1976)~--- 
Doctor of Science in physics and mathematics, professor, Department of 
Mathematical Statistics, Faculty of Computational Mathematics and Cybernetics, 
M.\,V.~Lomonosov Moscow State University, 1-52~Leninskiye Gory, GSP-1, Moscow 119991, 
Russian Federation; senior scientist, Institute of Informatics Problems, 
Federal Research Center ``Computer Science and Control'' 
of the Russian Academy of Sciences, 44-2~Vavilov Str., Moscow 119333, 
Russian Federation; \mbox{oshestakov@cs.msu.su}
\label{end\stat}

\renewcommand{\bibname}{\protect\rm Литература} 
     %7  
\def\stat{agalarov}


\def\tit{ПРИБЛИЖЕННЫЙ МЕТОД ВЫЧИСЛЕНИЯ ХАРАКТЕРИСТИК УЗЛА 
ТЕЛЕКОММУНИКАЦИОННОЙ СЕТИ С~ПОВТОРНЫМИ ПЕРЕДАЧАМИ}
\def\titkol{Приближенный метод вычисления характеристик узла 
телекоммуникационной сети с~повторными передачами} 

\def\autkol{Я.\,М.~Агаларов}
\def\aut{Я.\,М.~Агаларов$^1$}

\titel{\tit}{\aut}{\autkol}{\titkol}

%{\renewcommand{\thefootnote}{\fnsymbol{footnote}}\footnotetext[1]
%{Работа выполнена при поддержке РФФИ, проекты 08--07--00152 и 08--01--00567.}}

\renewcommand{\thefootnote}{\arabic{footnote}}
\footnotetext[1]{Институт проблем
информатики Российской академии наук, agglar@yandex.ru}

%\vspace*{-6pt}


\Abst{Рассмотрена модель узла коммутации пакетов c повторными передачами для двух 
схем распределения буферной памяти: полнодоступной и полного разделения. Предложен 
приближенный метод вычисления интенсивностей потоков и вероятностей блокировок узла. 
Получены необходимые и достаточные условия существования и единственности решения 
уравнения для потоков в узле при установившемся режиме работы и доказана сходимость 
итерационного метода решения указанного уравнения.}

\KW{узел коммутации пакетов; буферная память; повторные передачи; вероятности 
блокировок; итерационный метод}

      \vskip 18pt plus 9pt minus 6pt

      \thispagestyle{headings}

      \begin{multicols}{2}

      \label{st\stat}


\section{Введение}

    Одной из основных задач предварительного анализа 
телекоммуникационных сетей коммутации пакетов с ограниченной буферной 
памятью является расчет характеристик потоков и вероятностей блокировок в 
узлах связи. Важность указанных характеристик определяется тем, что от их 
значений существенным образом зависят другие основные показатели сети 
(пропускная способность, задержки пакетов и~др.). 

    Существует множество различных моделей узлов коммутации пакетов и 
методов их расчета (см., например,~[1--6]). Для моделей, рассматривающих 
узел с ограниченной буферной памятью как систему массового обслуживания 
(CMO) типа 
$
\begin{matrix}
M \\ \lambda
\end{matrix}
\left |
\begin{matrix}
M \\ \lambda
\end{matrix}
\right |
\overline{m} \vert N
$ или  $\vert PH\vert PH\vert 1\vert r$, в предположении отсутствия повторных 
передач пакетов получены точные методы вычисления характеристик 
узлов~[1, 3, 4, 6]. Приближенные методы расчета узлов, учитывающие повторные 
попытки передачи, используют модели типа $\vert PH\vert PH\vert 1\vert r$ или 
$
\begin{matrix}
M \\ \lambda
\end{matrix}
\left |
\begin{matrix}
M \\ \lambda
\end{matrix}
\right |
1 \vert N
$ и являются 
итерационными~[2, 3, 5, 7]. Для моделей типа 
$BM\!AP\vert PH\vert 1$, $M\vert G\vert 1\vert r$ и $M\!AP\vert 
(PH,PH)\vert 1$ с повторными заявками получены точные методы вычисления 
характеристик (например, в работах~[8--10]), которые также могут быть 
использованы при расчете узлов.

    Ниже будут рассмотрены модели узла коммутации пакетов с повторными 
передачами для двух схем распределения буферной памяти: с 
полнодоступными буферами и с полным разделением буферной памяти. 
Предлагается приближенный метод расчета характеристик, который в качестве 
модели узла использует СМО типа $
\begin{matrix}
M \\ \lambda
\end{matrix}
\left |
\begin{matrix}
M \\ \lambda
\end{matrix}
\right |
\overline{m} \vert N
$ с повторными заявками. Доказаны утверждения о 
достаточных и необходимых условиях существования и единственности 
решения уравнения для вероятности блокировки в установившемся режиме 
работы и сходимости предлагаемого итерационного метода. 

\section{Модель узла}

    Математическая модель узла представляется в виде СМО с ограниченной 
буферной памятью и различными потоками заявок, каждая из которых требует 
обслуживания только на одной из многоканальных линий связи. 

    Пусть $0<N<\infty$~--- число мест хранения в буферной памяти, $u$~--- 
узел связи, $v$~--- линия связи, $\Omega_u^+$~--- множество исходящих из 
узла~$u$ линий, $c_v$~--- канальная емкость линии~$v$. Поток заявок, 
тре\-бу\-ющих обслуживания на линии~$v$, назовем $v$-по\-то\-ком, заявки этого 
потока~--- $v$-за\-яв\-ка\-ми.


    Пусть выполняются следующие предположения: 
\begin{enumerate}[1.]
\item Места в буферной памяти распределяются согласно одной из двух 
схем:
\begin{enumerate}[($i$)]
\item полнодоступная схема~--- каждое свободное место хранения доступно 
любой заявке;
\item схема полного разделения памяти~--- $v$-за\-яв\-кам доступны всего 
$N_v$ мест, где $\sum\limits_{v\in\Omega_u^+} N_v=N$.
\end{enumerate}
\item Если в момент поступления $v$-заявки в буферной памяти есть 
доступное свободное место, то она сразу занимает это место. Если в момент 
поступления $v$-заявки в системе нет свободного доступного места 
хранения, то поступившая заявка через некоторое время повторно поступает 
на систему, оставаясь $v$-заявкой. 
\item Интенсивности первичных потоков $v$-заявок~--- заданные величины 
$0<\Lambda_v<\infty$, $v\in \Omega_u^+$. Суммарные потоки первичных и 
повторных $v$-заявок являются независимыми в совокупности 
пуассоновскими потоками. Для обслуживания $v$-заявки требуется 
одновременно одно место хранения и один канал типа~$v$, $v\in 
\Omega_u^+$.
\item Первичные нагрузки~--- реализуемые, т.\,е.\ в данном случае 
интенсивности входных первичных потоков равны интенсивностям 
выходных потоков выполненных заявок. 
\item Принятые в СМО $v$-заявки обслуживаются линией~$v$ в порядке 
поступления. 
\item Время занятия канала $v$-заявкой~--- экспоненциально 
распределенная случайная величина с параметром $0<\mu_v<\infty$, 
$v\in\Omega_u^+$, независимая от других случайных событий в узле.
\item Выполненная $v$-заявка с вероятностью~$B_v$ повторяется через 
заданное время~$\tau_v$ (тайм-аут) и с вероятностью $1-B_v$ покидает 
систему через время~$t_v$ навсегда, сразу освободив занятый канал и место 
буферной памяти.
\end{enumerate}

   Будем говорить, что узел блокирован для $v$-за\-яв\-ки, если в буферной 
памяти отсутствует доступное место хранения. Ставится задача вычисления 
вероятностей блокировок и интенсивностей потоков в узле.

\section{Вычисление вероятности блокировки и~интенсивностей~потоков} 

   Пусть $\Lambda_v^*$~--- интенсивность суммарного потока внешних 
заявок, требующих передачи по линии~$v$, $\pi_v$~--- вероятность блокировки 
узла для заявок, требующих передачи по исходящей из узла линии~$v$. 

    Пусть в узле используется полнодоступная схема распределения 
буферной памяти. Тогда, как следует из описания модели, $\pi_v 
=\pi_{v^\prime},\,v,\,v^\prime\in \Omega_u^+$, и для 
интенсивностей~$\Lambda_v^*$, $v\in\Omega_u^+$, справедливы соотношения:
\begin{equation*}
\Lambda_v^* = \fr{\Lambda_v}{1-\pi}\,,
%\label{e1aga}
\end{equation*}
    где
    $\pi =\pi_v$, $v\in\Omega_u^+$.

    Пусть 
    $\overline{k} = \{\overline{k}_v$, $v\in\Omega_u^+\}$~--- состояние 
буферной памяти узла, $\overline{k}_v =\left ( k_v,\,k_v^\prime,\,k_v^{\prime\prime}\right )$; 
$k_v$~--- число $v$-заявок в буферной 
памяти, ожидающих выполнения линией~$v$; $k^\prime_v$~--- число 
$v$-заявок в буферной памяти, ожидающих тайм-аут и неуспешно переданных 
в последующий узел; $k_v^{\prime\prime}$~--- число $v$-за\-явок в буферной 
памяти, успешно переданных в последующий узел и ожидающих 
потверждения; 
$A_m = \left \{ \overline{k}:\ \sum\limits_{v\in\Omega_u^+} \left ( 
k_v+k_v^\prime + k_v^{\prime\prime}\right ) =m \right \}$~--- множество различных 
состояний, при которых в памяти узла занято ровно $m$~буферов. Тогда с 
учетом введенных выше обозначений и предположений для ве\-ро\-ят\-ности 
блокировки узла можно написать формулу~\cite{1aga, 2aga}:
\begin{equation}
\pi = \fr{1}{G_N}\sum\limits_{\overline{k}\in A_N} 
p\left (\overline{k},\overline{\rho}^*\right )\,,
\label{e2aga}
\end{equation}
где  
\begin{gather}
p(\overline{k},\overline{\rho}^*) = \prod\limits_{v\in\Omega_u^+} z_v (\pi, 
\rho_v , k_v , k_v^\prime , k_v^{\prime\prime})\,;\\
z_v (\pi, \rho_v , k_v , k_v^\prime , k_v^{\prime\prime}) ={}\notag\\
\!\!{}=
\begin{cases}
 \fr{\rho_v^{\prime *k_v^\prime}}{k_v^{\prime}!}\,
\fr{\rho_v^{\prime\prime * k_v^{\prime\prime}}}{ k_v^{\prime\prime}!}  \,
\fr{\rho_v^{*k_v}}{ k_{v}!} 
&\mbox{при}\ k_v<c_v\,,\\
 \fr{\rho_v^{\prime * k_v^\prime}}{k_v^{\prime}!} \,
\fr{\rho_v^{\prime\prime * k_v^{\prime\prime}}} { k_v^{\prime\prime}!} 
\fr{\rho_v^{*k_v}}{ c_{v}!c_v^{k_v- c_v}} 
& \mbox{при}\ k_v\geq c_v\,;
\end{cases}\\
G_N = \sum\limits_{m=0}^N\sum\limits_{\overline{k}\in A_m}
p(\overline{k},\overline{\rho}^*)\,;\\ 
\overline{\rho}^*=\{\rho_v^*,\,v\in\Omega_u^+\}\,;\\
\rho_v^* = \fr{\rho_v}{1-\pi}\,;\quad \rho_v =\fr{\Lambda_v}{\mu_v(1- B_v)}\,;\\
\rho_v^{\prime *} =\rho_v^*\mu_v\tau_vB_v\,;\quad \rho_v^{\prime\prime *}=
p_v^* \mu_vt_v,\,\quad  v\in \Omega_u^+\,.\label{e3aga}
\end{gather}

Переобозначив $1-\pi$ через $y$, выражение в правой части равенства~(2)~--- через 
$p_{\overline{k}}(\overline{\rho},y)$, выражение в правой части равенства~(4)~--- 
через $g_N(\overline{\rho},y)$, а выражение в правой 
части равенства~(1)~--- через $1-q_N (\overline{\rho},y)$, 
где $\overline{\rho} = (\rho_v,\,v\in \Omega_u^+)$, $\rho_v = \rho_v^*y\;=$\linebreak 
$=\;\Lambda_v/(\mu_v(1-B_v))$, $v\in\Omega_u^+$, получим нелинейное уравнение 
относительно неизвестной переменной~$y$:
\begin{equation}
y=q_N(\overline{\rho},y)\,.
\label{e4aga}
\end{equation}

    Решим уравнение~(8). Как следует из~(2)--(7), верно 
равенство
\begin{equation}
q_N(\overline{\rho},y) = \fr{g_{N-1}(\overline{\rho},y )}{g_N(\overline{\rho},y)}\,.
\label{e5aga}
\end{equation}
Введем функцию  $d_n(\overline{\rho} ,y)$ среднего числа заявок в узле с 
буферной памятью емкости $n\geq 0$:
$$
d_n(\overline{\rho} ,y) = 
\fr{1}{g_n(\overline{\rho},y)}\,\sum\limits_{m=0}^n m\sum\limits_{\overline{k}\in 
A_m} p_{\overline{k}}(\overline{\rho},y)\,.
$$
Заметим, что $g_n$, $d_n$ и $q_n$, 
$n\geq 0$,~--- непрерывно-дифференцируемые функции по $y\in (0,\,1]$. Взяв 
производную функции~$g_n$ по~$y$, из~(2)--(7) получим
\begin{multline}
\fr{\partial g_n(\overline{\rho},y)}{\partial y} ={}\\
{}= -\sum\limits_{m=0}^n m 
\sum\limits_{\overline{k}\in A_m}\fr{\prod\limits_{v\in\Omega_u^+} z_n 
(0,\rho_v, k_v, k_v^\prime , k_v^{\prime\prime})}{y^{m+1}}={}\\
{}= -\fr{1}{y}\,g_n (\overline{\rho},y)d_n(\overline{\rho},y)\,.
\label{e6aga}
\end{multline}
Взяв производную функции $q_N$ по $y$, из~(\ref{e5aga}) и~(\ref{e6aga}) 
получим
\begin{equation}
\fr{\partial q_N(\overline{\rho},y)}{\partial y} = \fr{q_N(\overline{\rho},y)}{y}\left 
[ d_N (\overline{\rho},y)-d_{N-1}(\overline{\rho},y)\right ]\,.
\label{e7aga}
\end{equation}
    Докажем несколько утверждений о свойствах 
функции~$q_N(\overline{\rho},y)$.
\medskip

\noindent
\textbf{Утверждение 1.} \textit{Справедливы неравенства}
\begin{multline}
0<d_{n+1}(\overline{\rho},y)-d_n(\overline{\rho},y) <1\,,\\
\ \ \ \ \ \ \ \ \ \ \ \ \ \ \ \ \ \ \ \ y\in (0,\,1]\,, \ n\geq 0\,.
\label{e8aga}
\end{multline}


\noindent

Д\,о\,к\,а\,з\,а\,т\,е\,л\,ь\,с\,т\,в\,о\,.\ Подставив выражение для функции 
$d_n(\overline{\rho},y)$ и проведя преобразования, получим
\begin{multline*}
d_{n+1}(\overline{\rho},y) -d_n(\overline{\rho},y) = 
\fr{\sum\limits_{m=0}^{n+1}m\sum\limits_{\overline{k}\in A_m} 
p_{\overline{k}}(\overline{\rho},y)}
{\sum\limits_{m=0}^{n+1}
\sum\limits_{\overline{k}\in A_m} p_{\overline{k}}(\overline{\rho},y)} - {}\\
{}-
\fr{\sum\limits_{m=0}^n m \sum\limits_{\overline{k}\in A_m} p_{\overline{k}} 
(\overline{\rho},y)}{\sum\limits_{m=0}^n
\sum\limits_{\overline{k}\in A_m}p_{\overline{k}}(\overline{\rho},y)}={}\\
{}=\fr{\sum\limits_{m=1}^n m \sum\limits_{\overline{k}\in 
A_m}p_{\overline{k}}(\overline{\rho},y)+(n+1)\sum\limits_{\overline{k}\in 
A_{n+1}}  p_{\overline{k}}(\overline{\rho},y)}{\sum\limits_{m=0}^n\sum\limits_{\overline{k
}\in A_m}p_{\overline{k}}(\overline{\rho},y)+\sum\limits_{\overline{k}\in 
A_{n+1}}p_{\overline{k}}(\overline{\rho},y)} -{}
\end{multline*}
\begin{multline}
{}-
\fr{\sum\limits_{m=0}^n m 
\sum\limits_{\overline{k}\in A_m}p_{\overline{k}}(\overline{\rho},y)}
{\sum\limits_{m=0}^n\sum\limits_{\overline{k}\in A_m} 
p_{\overline{k}}(\overline{\rho},y)}={}\\
{}=\fr{(n+1)\sum\limits_{\overline{k}\in 
A_{n+1}}p_{\overline{k}}(\overline{\rho},y)g_n(\overline{\rho},y)}{g_{n+1}(\overline{\rho},y) g_n(\overline{\rho},y)} -{}\\
{}-
\fr{\sum\limits_{\overline{k}\in 
A_{n+1}}p_{\overline{k}}(\overline{\rho},y)\sum\limits_{m=0}^n  m 
\sum\limits_{\overline{k}\in A_m} p_{\overline{k}}(\overline{\rho},y) }
{g_{n+1}(\overline{\rho},y) g_n(\overline{\rho},y)}
={}\\
{}=\left [ 1-q_{n+1}(\overline{\rho},y)\right ] \left [n+1-d_n(\overline{\rho},y)\right ]\,.
\label{e9aga}
\end{multline}


    Докажем утверждение~1 методом индукции. При $n = 0$, как следует 
из~(\ref{e9aga}), имеем
$$
d_2(\overline{\rho},y) - d_1 (\overline{\rho},y) =1-q_1(\overline{\rho},y)\,,
$$
    т.\,е.\ утверждение~1 при $n = 0$ справедливо. 

    Пусть неравенства~(\ref{e8aga}) справедливы для некоторого $n > 0$. 
Докажем, что они справедливы и для $n + 1$. Из~(\ref{e9aga}) получаем
\begin{multline*}
d_{n+1}(\overline{\rho},y)- d_n(\overline{\rho},y)={}\\
{}=\left [ 1-
q_{n+1}(\overline{\rho},y)\right ] \left [n+1-d_n(\overline{\rho},y)\right ] ={}\\
{}= \left [ 1-
1-q_{n+1}(\overline{\rho},y)\right ] \left [ n-{}\right.\\
{}-\left. d_{n-1}(\overline{\rho},y)+d_{n-1}(\overline{\rho},y)-
d_n(\overline{\rho},y)+1\right ] ={}\\
{}=\left [ 1-q_{n+1}(\overline{\rho},y)\right ] 
\left [ n-d_{n-1}(\overline{\rho},y)-{}\right.\\
{}-\left. \left ( d_n(\overline{\rho},y)-d_{n-1}(\overline{\rho},y)\right )+1\right] = {}\\
{}=
\left [ 1-q_{n+1}(\overline{\rho},y)\right ]
\left [ 
\fr{d_n(\overline{\rho},y) -d_{n-1}(\overline{\rho},y)}{1-
q_n(\overline{\rho},y)}\right.-{}\\
{}-\left.
\left ( d_n(\overline{\rho},y)-d_{n-1}(\overline{\rho},y)\right )+1
\vphantom{\fr{d_n(\overline{\rho})}{(q_n)}}
\right ]={}\\
{}=
\left [ 1-q_{n+1}(\overline{\rho},y)\right ]
\left [ 
\vphantom{\fr{d_n(\overline{\rho})}{(q_n)}}
\left ( d_n(\overline{\rho},y\right)\right. -{}\\
 {}-\left.
d_{n-1}\left(\overline{\rho},y)\right )\fr{q_n(\overline{\rho},y)}{1-
q_n(\overline{\rho},y)}+1\right ]\,.
\end{multline*}
Так как по предположению $d_n (\overline{\rho},y) -d_{n-1}(\overline{\rho},y) 
>0$, то правая часть последнего равенства больше нуля; следовательно, 
$d_{n+1}(\overline{\rho},y)-d_n(\overline{\rho},y)>0$. 

    Продолжив преобразование правой части последнего равенства и 
учитывая предположение $d_n(\overline{\rho},y) -d_{n-1}(\overline{\rho},y)<1$, 
получим
\begin{multline*}
d_{n+1}((\overline{\rho},y) -d_n(\overline{\rho},y)<{}\\
{}< \left [ 1-
q_{n+1}(\overline{\rho},y)\right ]
\left ( \fr{q_n(\overline{\rho},y)}{1-q_n(\overline{\rho},y)}+1\right )={}\\
{}=
\fr{1-q_{n+1}(\overline{\rho},y)}{1-q_n(\overline{\rho},y)}<1\,,
\end{multline*}
так как $0< q_n(\overline{\rho},y)<q_{n+1}(\overline{\rho},y)<1$, $n>0$, $y\in 
(0,\,1]$.

Следовательно, утверждение~1 доказано.

\medskip

\noindent
\textbf{Утверждение 2.} $q_N(\overline{\rho},y)$~--- \textit{монотонно-воз\-рас\-та\-ющая 
функция по $y\in (0,\,1]$. При этом $0< q_N(\overline{\rho},y)\;\leq $\linebreak 
$\leq\;q_N(\overline{\rho},1) <1$, $y\in (0,\,1]$,  и $\underset{y\rightarrow 
0}{\mathrm{lim}}\,q_N(\overline{\rho},y) =0$}.

\medskip

\noindent
Д\,о\,к\,а\,з\,а\,т\,е\,л\,ь\,с\,т\,в\,о\,.\  Возрастание функции 
$q_N(\overline{\rho},y)$ следует непосредственно из~(\ref{e7aga}) и 
утверж\-де\-ния~1. Доказательство неравенств в условии утверждения очевидно 
следует из~(\ref{e5aga}) и вида функции $g_n (\overline{\rho},y)$, $n\geq 0$. 
Для предела имеем:
\begin{multline*}
\underset{y\rightarrow 0}{\mathrm{lim}}\,q_N(\overline{\rho},y) 
=\underset{y\rightarrow 0}{\mathrm{lim}}\,\fr{g_{N- 1}(\overline{\rho},y)}{g_N(\overline{\rho},y)} = {}\\
{}= \underset{y\rightarrow 0}{\mathrm{lim}}\,\left (
g_{N-1}(\overline{\rho},y)\Bigg / \left ( 
\vphantom{\prod\limits_{v\in\Omega_u^+}}
g_{N-1}(\overline{\rho},y)\right.\right.+{}\\
{}+\left.\left.\sum\limits_{\overline{k}\in A_N}\prod\limits_{v\in\Omega_u^+} 
\fr{z_v(0,\rho_v,k_v,k^\prime_v,k^{\prime\prime}_v)}{y^N}\right )\right ) = {}\\
{}= \underset{y\rightarrow 0}{\mathrm{lim}}\,\left (
y^N g_{N-1}(\overline{\rho},y)\Bigg / 
\left ( 
\vphantom{\prod\limits_{v\in\Omega_u^+}}
y^N g_{N-1}(\overline{\rho},y)+{}\right.\right.\\
{}+\left.\left.\sum\limits_{\overline{k}\in A_N}
\prod\limits_{v\in\Omega_u^+} z_v(0,\rho_v,k_v,k_v^\prime , k_v^{\prime\prime}) 
\right ) \right )=0\,.
\end{multline*}
    
\medskip

\noindent
\textbf{Утверждение 3.} \textit{Производная функции~$q_N (\overline{\rho},y)$ по 
$y\in (0,\,1]$ удовлетворяет следующим соотношениям}:
\begin{align}
\underset{y\rightarrow 0}{\mathrm{lim}}\fr{\partial q_N(\overline{p},y)}
{\partial  y} &= \fr{\sum\limits_{\overline{k}\in A_{N-1}} 
p_{\overline{k}}(\overline{\rho},1)}{\sum\limits_{\overline{k}\in 
A_N}p_{\overline{k}}(\overline{\rho},1)}\,;\label{e10aga}\\
\fr{\partial q_N(\overline{\rho},y)}{\partial y}\Big |_{y=1}&<1\,.\label{e11aga}
\end{align}

\medskip

\noindent
Д\,о\,к\,а\,з\,а\,т\,е\,л\,ь\,с\,т\,в\,о\,.\ Проведя преобразования 
функции~$q_N(\overline{\rho},y)$, получим:
\begin{multline*}
\underset{y\rightarrow 0}{\mathrm{lim}}\fr{q_N(\overline{\rho},y)}{y} = {}\\
\!\!{}=
\underset{y\rightarrow 0}{\mathrm{lim}}
\fr{\sum\limits_{m=0}^{N-1}\sum\limits_{\overline{k}\in A_m}
\!\!\left (\prod\limits_{v\in\Omega_u^+}\!\! 
z_v(0,\rho_v,k_v,k_v^\prime , k_v^{\prime\prime})\right )\!\!\Bigg /\!\! y^m}
{y\sum\limits_{m=0}^{N}\sum\limits_{\overline{k}\in A_m}
\!\!\left(\prod\limits_{v\in\Omega_u^+}\!\! z_v\left (0,\rho_v,k_v,k_v^\prime , 
k_v^{\prime\prime}\right )\right )\!\!\Bigg /\!\!y^m} = \!\!\!
\end{multline*}
\begin{multline*}
\!\!\!\!\!\!{}=\underset{y\rightarrow 0}{\mathrm{lim}}\,
\fr{\sum\limits_{m=0}^{N-1}\sum\limits_{\overline{k}\in A_m}
y^{N-1-m}\prod\limits_{v\in\Omega_u^+} z_v(0,\rho_v,k_v,k_v^\prime , 
k_v^{\prime\prime})}{\sum\limits_{m=0}^{N}\sum\limits_{\overline{k}
\in A_m} y^{N-m}
\prod\limits_{v\in\Omega_u^+} z_v(0,\rho_v,k_v,k_v^\prime , 
k_v^{\prime\prime})}={}\!\\
{}=\fr{\sum\limits_{\overline{k}\in A_{N-1}} p_{\overline{k}}(\overline{\rho},1)}{ 
\sum\limits_{\overline{k}\in A_{N}} p_{\overline{k}}(\overline{\rho},1)}\,.
\end{multline*}
Очевидно, $\underset{y\rightarrow 0}{\mathrm{lim}} \,[d_N (\overline{\rho},y) -
d_{N-1} (\overline{\rho},y)]=1$, так как $\underset{y\rightarrow 
0}{\mathrm{lim}}\,d_n (\overline{\rho},y)=n$, $n>0$.

Следовательно, учитывая~(\ref{e7aga}), получаем~(\ref{e10aga}). 
Справедливость~(\ref{e11aga}) непосредственно следует из~(\ref{e7aga}) и 
утверждения~1.

\medskip

\noindent
\textbf{Утверждение 4.} \textit{Пусть $y^*\in (0,\,1]$~--- решение 
уравнения}~(\ref{e4aga}). \textit{Тогда}
\begin{equation*}
\fr{\partial q_N(\overline{\rho},y)}{\partial y}\Big |_{y=y^*}<1\,.
%\label{e12aga}
\end{equation*}

\medskip

\noindent
Д\,о\,к\,а\,з\,а\,т\,е\,л\,ь\,с\,т\,в\,о\,.\ \ Доказательство следует из~(\ref{e7aga}), 
так как $q_N(\overline{\rho},y^*)/y^* =1$.
\medskip

\noindent
\textbf{Утверждение 5.} \textit{Уравнение}~(\ref{e4aga}) \textit{имеет решение $y^*\in 
(0,\,1)$ тогда и только тогда, когда} 
\begin{equation}
\fr{\sum\limits_{\overline{k}\in A_{N-1}} p_{\overline{k}}(\overline{\rho},1)}{ 
\sum\limits_{\overline{k}\in A_{N}} p_{\overline{k}}(\overline{\rho},1)} >1\,.
\label{e13aga}
\end{equation}
\textit{Если уравнение}~(\ref{e4aga}) \textit{имеет решение $y^*\in (0,\,1)$, то оно 
единственное положительное решение}.
\medskip

\noindent
Д\,о\,к\,а\,з\,а\,т\,е\,л\,ь\,с\,т\,в\,о\,.\ Пусть выполняется 
неравенство~(\ref{e13aga}). Тогда, как следует из утверждения~3, 
$\underset{y\rightarrow 0}{\mathrm{lim}} (\partial q_N(\overline{\rho},y)/\partial y) 
>1$. Кроме того, как следует из утверждения~2, 
$\underset{y\rightarrow 0}{\mathrm{lim}} q_N(\overline{\rho},y)=0$. Тогда, так 
как $q_N(\overline{\rho},y)$~--- непрерывно-дифференцируемая функция по 
$y\in (0,\,1]$, существует значение $y^\prime \in (0,\,1)$ такое, что 
$q_N(\overline{\rho},y)>y$ для всех $y\in (0,\,y^\prime]$ (следует из теоремы о 
конечном приращении~\cite{11aga}). В то же время, согласно утверждению~2, 
$q_N(\overline{\rho},y)<y$ в окрестности точки $y=1$ (рис.~\ref{f1aga},\,\textit{а}). 
Следовательно, кривая $x=q_N(\overline{\rho},y)$ пересекает прямую $x=y$ 
хотя бы в одной точке $y=y^*\in (0,\,1)$, т.\,е.\ уравнение~(\ref{e4aga}) имеет 
хотя бы одно решение $y^*\in (0,\,1)$.

\begin{figure*}
\vspace*{1pt}
\begin{center}
\vspace*{1pt}
\mbox{%
\epsfxsize=158mm
\epsfbox{aga-1.eps}
}
\end{center}
\vspace*{-9pt}
\Caption{Примеры кривых $x=q_N(\overline{\rho},y)$ и $x=y$ (\textit{а})~при существовании решения 
уравнения~(\ref{e5aga}) и (\textit{б})~при выполнении условий~(17)
\label{f1aga}}
\vspace*{6pt}
\end{figure*}

Пусть уравнение~(\ref{e4aga}) имеет решение $y^*\in (0,\,1)$ и 
\begin{equation}
\fr{\sum\limits_{\overline{k}\in A_{N-1}}p_{\overline{k}}(\overline{\rho},1)}{ 
\sum\limits_{\overline{k}\in A_{N}}p_{\overline{k}}(\overline{\rho},1)}\leq 
1\,.\label{e14aga}
\end{equation}
Тогда из условий утверждений~2 и~3 следует, что 
уравнение~(\ref{e4aga}) в интервале $(0,\,1)$ имеет более одного решения, что 
может быть только при существовании решения $y^\prime \in (0,\,1)$ такого, 
что в окрестности точки $y=y^\prime$ выполняются неравенства: 
$q_N(\overline{\rho},y)<y$ при $y<y^\prime$ и $q_N(\overline{\rho},y)>y$ при 
$y>y^\prime$, где $y$ принадлежит указанной окрест\-ности точки~$y^\prime$ 
(рис.~\ref{f1aga},\,\textit{б}). Тогда в точке $y=y^\prime$ производная функции 
$q_N(\overline{\rho},y)$ по $y$ больше~1, что противоречит утверждению~4. 
Следовательно, неравенство~(\ref{e13aga}) справедливо.


Пусть уравнение~(\ref{e4aga}) имеет более одного положительного 
решения. Рассуждая точно так же, как и выше (в случае выполнения 
условий~(\ref{e14aga})), получим противоречие утверждению~4. 
Следовательно, утверждение~5 справедливо.
\medskip

\noindent
\textbf{Следствие.} \textit{Неравенства}
\begin{gather*}
\fr{\mu_v c_v (1-B_v)}{\Lambda_v}>1\,,\quad \fr{1-B_v}{\Lambda_v \tau_v B_v}>1\,,\\ 
\fr{1-B_v}{\Lambda_v t_v}>1\,,\ v\in\Omega_u^+\,,
\end{gather*}
\textit{являются необходимым условием существования решения 
уравнения}~(\ref{e4aga}).

\medskip
\noindent
Д\,о\,к\,а\,з\,а\,т\,е\,л\,ь\,с\,т\,в\,о\,.\ Пусть $\overline{k}_v$~--- это 
набор~$\overline{k}$, у которого $k_v=0$. Преобразовав левую 
часть~(\ref{e13aga}), получим

\noindent
\begin{multline*}
\fr{\sum\limits_{\overline{k}\in A_{N-1}} p_{\overline{k}} (\overline{\rho},1)}
{ \sum\limits_{\overline{k}\in A_{N}} 
 p_{\overline{k}}(\overline{\rho},1)} 
={}
\\
{}=
\fr{\sum\limits_{\overline{k}\in A_{N-1}}\prod\limits_{v\in \Omega_u^+} 
z_v\left(0,\rho_v,k_v,k_v^\prime , k_v^{\prime\prime}\right)}
{\sum\limits_{\overline{k}\in A_{N}}
\prod\limits_{v\in \Omega_u^+} z_v\left (0,\rho_v,k_v,k_v^\prime , k_v^{\prime\prime}\right )} \leq{}
\\
{}\leq
\left ( 
\vphantom{\prod\limits_{v^\prime\in\Omega_u^+\backslash v}}
\fr{\mu_v c_v(1-B_v)}{\Lambda_v}\right. \times{}\\
{}\times \sum\limits_{k_v=0}^{N-1}\sum\limits_{\overline{k}_v\in A_{N-1-k_v}} z_v\left(0,\rho_v,k_v+1,k_v^\prime , 
k_v^{\prime\prime}\right )\times{}\\
{}\times \left.\prod\limits_{v^\prime\in\Omega_u^+\backslash v} z_v^\prime 
\left(0,\rho_v,k_v,k_v^\prime , k_v^{\prime\prime}\right) \right)
\Bigg /{}\\
\Bigg / \left ( 
\vphantom{\prod\limits_{v^\prime\in\Omega_u^+\backslash v}}
\sum\limits_{k_v=0}^{N-1} \sum\limits_{\overline{k}_v\in A_{N-1-k_v}} z_v 
\left (0,\rho_v,k_v+1,k_v^\prime , 
k_v^{\prime\prime}\right )\right. \times{}\\
{}\times \prod\limits_{v^\prime\in\Omega_u^+\backslash v} 
z_{v^\prime}\left(0,\rho_v,k_v,k^\prime , k_v^{\prime\prime}\right)+{}\\
{}+
\sum\limits_{\overline{k}_v\in A_N} z_v\left (0,\rho_v, 0,k_v^\prime , 
k_v^{\prime\prime}\right)\times{}\\
\left.{}\times \prod\limits_{v^\prime\in\Omega_u^+\backslash v}z_{v^\prime} 
\left(0,\rho_v,k_v,k_v^\prime , k_v^{\prime\prime}\right )\right )\,.
\end{multline*}
Как следует из правой части последнего неравенства, если 
$\mu_v c_v (1-B_v)/\Lambda_v \leq 1$, то она меньше~1. Поэтому, чтобы 
выполнилось условие~(\ref{e13aga}), необходимо выполнение первого 
неравенства в условии следствия для каждого $v\in\Omega_u^+$. Точно так же 
доказывается необходимость выполнения второго и третьего неравенств в 
условии следствия.

    Пусть $y[n]$, $n\geq 0$, последовательность, полученная по формуле 
$y[n+1]=q_N(\overline{\rho},y[n])$, $y[0]=1$.

\medskip

\noindent
\textbf{Утверждение 6.} \textit{Пусть $y^*\in (0,\,1)$~--- решение 
уравнения}~(8). \textit{Тогда последовательность $y[n]$, $n\geq 0$, сходится 
к решению~$y^*$}.

\medskip

\noindent
Д\,о\,к\,а\,з\,а\,т\,е\,л\,ь\,с\,т\,в\,о\,.\ Отметим, что $y[1]<y[0]$ (это следует из 
утверждения~2, так как $y[0]=1$). Пусть для некоторого $n>1$ выполняется 
условие $y[n]<y[n-1]$. Тогда, как следует из утверждения~2, указанное условие 
выполняется и для $n+1$, т.\,е.\ по индукции следует, что последовательность 
$y[n]$, $n\geq 0$, монотонно убывает. 

    Пусть для некоторого $n>0$ $y[n]>y^*$ (существование такого $n$ 
следует из равенства $y[0]=1$). Тогда, как следует из утверждения~2, 
$y[n+1]\;=$\linebreak $=\;q_N(\overline{\rho},y[n])>q_N(\overline{\rho},y^*) =y^*$, т.\,е.\ 
последовательность ограничена снизу величиной~$y^*$. Значит, существует 
$\underset{n\rightarrow \infty}{\mathrm{lim}}\,y[n]=y^0\geq y^*$. Так как 
$q_n(\overline{\rho},y)$~--- непрерывная по~$y$ функция, то можно написать 
$\underset{n\rightarrow 
\infty}{\mathrm{lim}}\,q_N(\overline{\rho},y[n])=q_N(\overline{\rho},y^0)=y^0$, 
т.\,е.\ $y^0$~--- решение уравнения~(\ref{e4aga}). Из единственности 
положительного решения уравнения~(\ref{e4aga}) получаем $y^0=y^*$.

    Пусть в узле используется схема полного разделения буферной памяти. 
Тогда для интенсив\-ностей~$\Lambda_v^*$, $v\in\Omega_u^+$, справедливы 
соотношения:
$$
\Lambda_v^* = \fr{\Lambda_v}{1-\pi_v}\,,
$$
где $v\in\Omega_u^+$.


Фиксируем произвольную линию сети~$v$. Пусть $\overline{k}_v = (k_v, 
k_v^\prime, k_v^{\prime\prime})$~--- состояние буферной памяти линии~$v$; 
$k_v$, $k_v^\prime$, $k_v^{\prime\prime}$ определены выше. Тогда с 
учетом введенных ранее предположений и обозначений для вероятности 
блокировки линии справедлива формула~\cite{4aga}:
\begin{equation}
\pi_v = \fr{1}{G_{N_v}}\sum\limits_{k_v=N_v} 
z_v(\pi_v,\rho_v,\overline{k}_v)\,,
\label{e15aga}
\end{equation}
где 
\begin{multline*}
z_v(\pi_v, \rho_v, \overline{k}_v)={}\\
{}=
\begin{cases}
\fr{\rho_v^{\prime * k_v^\prime}}{k_v^\prime !}\,
 \fr{\rho_v^{\prime\prime * k_v^{\prime\prime}}}{k_v^{\prime\prime}!}\,
 \fr{\rho_v^{*k_v}}{k_v !} & \mbox{при}\ k_v<c_v\,,\\
 \fr{\rho_v^{\prime *k_v^\prime}}{k_v^{\prime }! }
 \fr{\rho_v^{\prime\prime * k_v^{\prime\prime}}}{k_v^{\prime\prime}!}
\fr{\rho_v^{*k_v}}{c_v !c_v^{k_v-c_v}} & \mbox{при}\ k_v\geq c_v\,;
\end{cases}
\end{multline*}
\begin{align*}
G_{N_v} &= \sum\limits_{m=0}^{N_v} z_v (\pi_v ,\rho_v , \overline{k}_v)\,;\\ 
\rho_v^*&=\fr{\rho_v}{1-\pi_v}\,;
\end{align*}
$\rho_v$, $\rho_v^{\prime *}$, 
$\rho_v^{\prime\prime *}$, $v\in\Omega_u^+$ определены выше.
    
Пусть $y_v=1-\pi_v$, а $q_{N_v} (\rho_v, y_v)$~--- выражение в правой 
части~(\ref{e15aga}). Тогда из равенств~(\ref{e15aga}), взяв~$y_v$ в качестве 
неизвестной переменной, получим систему независимых уравнений:
\begin{equation}
y_v = q_{N_v}(\rho_v, y_v)\,, \quad v\in \Omega_u^+\,.
\label{e16aga}
\end{equation}
    
    Заметим, что для фиксированной $v$ и заданных параметров $\Lambda_v$, 
$\mu_v$, $\tau_v$, $t_v$, $N_v$, $v\in\Omega_u^+$, уравнение в~(\ref{e16aga}) 
является частным случаем уравнения~(\ref{e4aga}) и совпадает с последним, 
когда число исходящих линий из узла равно~1. Следовательно, для схемы 
полного разделения памяти также справедливы все приведенные выше 
утверждения~1--6 и следствие. Заметим, что неравенство~(\ref{e13aga}) в 
условии утверждения~5 при $B_v=0$ и $t_v=0$ преобразуется в неравенство 
$\Lambda_v / (\mu_v c_v) >1$, $v\in\Omega_u^+$. Последовательность 
$\overline{y}[n]$, $n\geq 0$, в утверждении~6 определяется по формуле:
    \begin{gather*}
    \overline{y}[n] =\{y_v[n],\ v\in\Omega_u^+\}\,,\
    y_v[n+1]=q_{N_v} (\rho_v,\,y_v[n])\,,\\
    y_v[0] =1\,,\quad n\geq 0\,,\quad v\in \Omega_u^+\,.
    \end{gather*}


\section{Алгоритм расчета} %4

    Для вычисления интенсивностей потоков и вероятностей блокировок в 
узле предлагается следующий алгоритм, описывающий изложенную выше 
итерационную процедуру. Введем обозначения:
$y_u^v$~--- вероятность блокировки узла для заявок, направляемых на 
линию~$v$,
\begin{gather*}
y_u^v  = 
\begin{cases}
y_u & \mbox{для}\ v\in\Omega_u^+\ \mbox{при}\\
&\mbox{полнодоступной схеме},\\
y_v & \mbox{при схеме полного распределения}\\
&\mbox{памяти};
\end{cases}
\\
q_N^v(\overline{\rho}_u^{-v}, y_u^v)  = 
\begin{cases}
q_N(\overline{\rho},y) & \mbox{для}\ v\in\Omega_u^+\ \mbox{при пол-}\\ 
&\mbox{нодоступной схеме},\\
q_{N_v}(\rho_v, y_v) & \mbox{при схеме полного}\\
&\mbox{распределения}\\ 
&\mbox{памяти},  v\in\Omega_u^+\,.
\end{cases}
\end{gather*}



Тогда уравнения~(\ref{e4aga}) и~(\ref{e16aga}) записываются в виде:
$$
y_u^v = q_N^v (\overline{\rho}^v_u, y^v_u)\,,\quad v\in \Omega_u^+\,.
$$
Для значений, вычисляемых на $k$-м шаге алгоритма, к 
обозначениям соответствующих параметров приписывается знак~$[k]$.
\pagebreak

\textbf{Шаг 0.} 
\begin{enumerate}[1.]
\item  \textit{Инициализация}. Вычисление начальных значений 
параметров~$\rho_v$, $v\in\Omega_u^+$: $\Lambda_v[0]=\Lambda_v$, 
$\rho_v[0]=\Lambda_v[0]/(\mu_v(1-B_v))$, $y_u^v[0]=1$.
\item \textit{Проверка условий существования решения}. Если для некоторой 
линии $v\in\Omega_u^+$ выполняется хотя бы одно неравенство $(c_v\mu_v(1-
B_v))/\Lambda_v[0]\;\leq$\linebreak $\leq\;1$, или $(1-B_v)/(\Lambda_v\tau_v B_v) \leq 1$, или 
$(t_v(1\;-$\linebreak $-\;B_v))/\Lambda_v[0] \leq 1$, то алгоритм заканчивает работу с 
результатом <<нагрузка не реализуема>>. Если в узле используется 
полнодоступная схема и $(c_v\mu_v(1-B_v))/\Lambda_v[0] > 1$, $(1-
B_v)/(\Lambda_v\tau_v B_v)\;>$\linebreak $>\;1$, $(t_v(1-B_v))/\Lambda_v[0] > 1$ для всех 
$v\in\Omega_u^+$, то проверяется условие~(\ref{e13aga}) для $\Lambda_v =
\Lambda_v[0]$, $v\in\Omega_u^+$, и при невыполнении этого условия алгоритм 
заканчивает работу с результатом <<нагрузка не реализуема>>.
\end{enumerate}

    При вычислении левой части неравенства~(\ref{e13aga}) рекомендуется 
использовать метод свертки Базена (см.~\cite{12aga}), позволяющий 
производить рекуррентные вычисления (подробно этот метод описан также 
в~[1, 3--6]).



\medskip
\textbf{Шаг~$k$} ($k > 0$):
\begin{enumerate}[1.]
\item \textit{Вычисление вероятностей блокировок}. Используя значения 
параметров $\overline{\rho}_u^v[k-1]$, $y_u^v[k-1]$, $v\in\Omega_u^+$, 
вычисление с помощью формул~(1)--(7) значений 
вероятностей $y[k]=1- \pi [k]$~--- в случае полнодоступной памяти, или 
$y_v[k]=1- \pi_v[k]$, $v\in\Omega_u^+$, с помощью формул~(\ref{e15aga})~--- в 
случае полного разделения памяти. При вычислении этих значений 
рекомендуется использовать метод свертки Базена.
    \item \textit{Проверка условий останова алгоритма}. Если хотя бы для 
одной $v\in\Omega_u^+$ для заданного значения точности   выполняется 
условие
$$
\fr{\vert \Lambda_v^*[k]-\Lambda_v^*[k-1]\vert}{\Lambda_v^*[k]}> \varepsilon\,,
$$
то вычисление параметров $\overline{\rho}_u^v[k]$, $v\in\Omega_u^+$, и 
переход к шагу~$k$, положив $k$ равным $k+1$, иначе алгоритм завершает 
работу. 
\end{enumerate}

    По завершении алгоритма либо выявится, что нагрузка в системе не 
реализуема, либо будут вычислены интенсивности потоков, поступающих на 
линии узла, и стационарные вероятности блокировок для заявок каждого типа. 
    
\section{Примеры расчета}

    Для проверки точности вычисления результатов с помощью 
предложенного выше алгоритма и приемлемости введенных предположений 
были проведены вычислительные эксперименты с использованием 
аналитических и имитационных моделей. Во всех рассмотренных ниже 
примерах потоки внешних заявок считаются пуассоновскими. 
В~табл.~1 приведены значения вероятности блокировок вновь 
поступивших извне заявок, полученные на основании точной формулы, 
приведенной в~\cite{4aga} для СМО типа $M\vert M\vert 1\vert 0$ с повторными 
заявками при экспоненциальном распределении интервала времени между 
повторными попытками (первая строка таблицы), алгоритма из подраздела~5 
настоящей статьи (вторая строка) и имитационной модели при постоянном 
интервале времени между повторными попытками, равном~10 (третья строка). 
Расчет табл.~1 проведен для узла с одной исходящей одноканальной 
линией при интенсивности первичного потока $\Lambda =1$ и емкости 
накопителя $N_v=1$. Таблицы~2 и~3 вычислены с помощью 
алгоритма из подраздела~5 и имитационной модели соответственно при одной 
исходящей линии с числом каналов~10.


    В табл.~\ref{t4aga} и~\ref{t5aga} приведены значения вероятности 
блокировки узла с тремя исходящими линиями канальной емкости~10 каждая 
при $\mu_v =0{,}2$, $v\in\Omega_u^+$,  вычисленные с помощью алгоритма из 
подраздела~5 и имитационной модели с интервалом повторной попытки, 
равным~10, соответственно. В табл.~\ref{t4aga} и~\ref{t5aga} знак <<--->> в 
ячейках означает, что предложенная нагрузка $\Lambda_v$, $v\in\Omega_u^+$, 
не реализуема.



В табл.~\ref{t6aga} отражены вероятности блокировки такого же узла с 
накопителем $N = 35$ при экспоненциальном распределении интервала 
времени между повторными попытками со средним значением~$\tau$. 


Результаты вычислительного эксперимента показывают, что с  увеличением 
длины интервала между повторными попытками  вероятность блокировки 
увеличивается и приближается к значению,\linebreak
вычисленному с помощью 
алгоритма из подраздела~5 (см.\ табл.~\ref{t4aga} и~\ref{t6aga}), т.\,е.\ при 
пуассоновском внешнем потоке заявок предположение, что суммарный 
входной поток заявок  является пуассоновским, вполне приемлемо для 
предварительного анализа характеристик узла (например, при  $\tau c_v\mu_v > 
10$). Как показывают табл.~1--3, вероятность блокировки 
узла существенно зависит от\linebreak 

\vspace*{6pt}
\noindent
%\begin{table*}\small %tabl1
{\small
{{\tablename~1}\ \ \small{Вероятности блокировок при одной исходящей одноканальной линии}}
%\label{t1aga}}
\vspace*{-3pt}

\begin{center}
{\tabcolsep=7.3pt
\begin{tabular}{|c|c|c|c|c|c|}
\hline
&\multicolumn{5}{c|}{$\mu$}\\
\cline{2-6}
\multicolumn{1}{|c|}{\raisebox{4pt}[0pt][0pt]{№}}&1{,}1&1{,}2&2&3&4\\
\hline
1&0,9091&0,8333&0,5000&0,3333&0,2500\\
2&0,9091&0,8333&0,5000&0,3333&0,2500\\
3&0,8867&0,8452&0,4944&0,3167&0,2396\\
\hline
\end{tabular}}
\end{center}
%\vspace*{-6pt}
%\end{table*}
}
%\bigskip
%\medskip
\addtocounter{table}{1}
\pagebreak

\end{multicols}

\renewcommand{\figurename}{\protect\bf Таблица}
%\renewcommand{\tablename}{\protect\bf Рис.}
\begin{figure*}
{\small
\begin{minipage}[t]{76mm}
%\begin{table*}\small %tabl2
\begin{center}
\Caption{Вероятности блокировок при одной исходящей многоканальной линии ($\varepsilon 
=0{,}0001$)
\label{t2aga}}
\vspace*{2ex}

\tabcolsep=6.5pt
\begin{tabular}{|c|c|c|c|c|c|}
\hline
&\multicolumn{5}{c|}{$\mu$}\\
\cline{2-6}
\multicolumn{1}{|c|}{\raisebox{4pt}[0pt][0pt]{$N$}}&0{,}11&0{,}12&0{,}2&0{,}3&0{,}4\\
\hline
10&0,4845&0,2935&0,0204&0,0017&0,0002\\
15&0,1181&0,0545&0,0006&0,0000&0,0000\\
20&0,0489&0,0167&0,0000&0,0000&0,0000\\
\hline
\end{tabular}
\end{center}
%\end{table*}
\end{minipage}
\hfill
\begin{minipage}[t]{76mm}
%\begin{table*}\small %tabl3
\begin{center}
\Caption{Вероятности блокировок при одной исходящей линии
\label{t3aga}}
\vspace*{2ex}

\tabcolsep=6.5pt
\begin{tabular}{|c|c|c|c|c|c|}
\hline
&\multicolumn{5}{c|}{$\mu_v$}\\
\cline{2-6}
\multicolumn{1}{|c|}{\raisebox{4pt}[0pt][0pt]{$N$}}&0{,}11&0{,}12&0{,}2&0{,}3&0{,}4\\
\hline
10&0,5247&0,3238&0,0219&0,0019&0,0001\\
15&0,1726&0,0912&0,0004&0,0001&0,0000\\
20&0,1180&0,0371&0,0000&0,0000&0,0000\\
\hline
\end{tabular}
\end{center}
%\end{table*}
\end{minipage}
}
\vspace*{6pt}
\end{figure*}

\renewcommand{\figurename}{\protect\bf Рис.}
\renewcommand{\tablename}{\protect\bf Таблица}
\addtocounter{table}{2}

\begin{table}\small %tabl4
\begin{center}
\parbox{400pt}{\Caption{Вероятности блокировок при трех исходящих линиях, вычисленные алгоритмом из 
подраздела~5 ($\varepsilon =0{,}0001$)
\label{t4aga}}
}

\vspace*{2ex}

\tabcolsep=8pt
\begin{tabular}{|c|c|c|c|c|c|c|c|c|c|}
\hline
&\multicolumn{9}{c|}{$\Lambda_v$}\\
\cline{2-10}
\multicolumn{1}{|c|}{\raisebox{4pt}[0pt][0pt]{$N$}}&1&1{,}1&1{,}2&1{,}3&1{,}4&1{,}5&1{,}6&1{,}7&1{,}8\\
\hline
20&0,0677&0,1423&0,2975&0,7653&---&---&---&---&---\\
25&0,0065&0,0173&0,0394&0,0827&0.1690&0.3827&---&---&---\\
30&0,0005&0,0019&0,0059&0,0155&0.0361&0.0790&0.1792&0,7259&---\\
35&0,0000&0,0002&0,0008&0,0030&0,0089&0,0234&0,0574&0,1505&---\\
40&0,0000&0,0000&0,0001&0,0005&0,0022&0,0075&0,0220&0,0617&0,2449\\
\hline
\end{tabular}
\end{center}
%\end{table}
\vspace*{6pt}
%\begin{table}\small %tabl5
\begin{center}
\parbox{400pt}{\Caption{Вероятности блокировок при трех исходящих линиях, вычисленные с помощью 
имитационной модели
\label{t5aga}}
}

\vspace*{2ex}

\tabcolsep=8pt
\begin{tabular}{|c|c|c|c|c|c|c|c|c|c|}
\hline
&\multicolumn{9}{c|}{$\Lambda_v$}\\
\cline{2-10}
\multicolumn{1}{|c|}{\raisebox{4pt}[0pt][0pt]{$N$}}&1&1{,}1&1{,}2&1{,}3&1{,}4&1{,}5&1{,}6&1{,}7&1{,}8\\
\hline
20&0,0786&0,1695&0,3549&0,7056&---&---&---&---&---\\
25&0,0069&0,0190&0,0441&0,0998&0,2266&0,4583&---&---&---\\
30&0,0007&0,0024&0,0075&0,0184&0,0462&0,1025&0,2380&0,6931&---\\
35&0,0000&0,0003&0,0007&0,0040&0,0129&0,0307&0,0890&0,2981&---\\
40&0,0000&0,0000&0,0000&0,0011&0,0041&0,0095&0,0346&0,0790&0,3179\\
\hline
\end{tabular}
\end{center}
%\end{table}
\vspace*{6pt}
%\begin{table}\small %tabl6
\begin{center}
\parbox{356pt}{\Caption{Зависимость вероятности блокировки при трех исходящих линиях, вы\-чис\-лен\-ные с 
помощью имитационной модели со случайным интервалом между повторными попытками
\label{t6aga}}
}

\vspace*{2ex}

\tabcolsep=8pt
\begin{tabular}{|c|c|c|c|c|c|c|c|c|}
\hline
&\multicolumn{8}{c|}{$\Lambda_v$}\\
\cline{2-9}
\multicolumn{1}{|c|}{\raisebox{6pt}[0pt][0pt]{$\tau$}}&1&1{,}1&1{,}2&1{,}3&1{,}4&1{,}5&1{,}6&1{,}7\\
\hline
\hphantom{9}1&0.0001&0,0001&0,0017&0,0063&0,0210&0,0733&0,1996&0,4222\\
\hphantom{9}5&0.0000&0,0002&0,0016&0,0036&0,0446&0,0159&0,1360&0,3273\\
10&0.0000&0,0002&0,0011&0,0036&0,0101&0,0430&0,0818&0,2774\\
20&0.0000&0,0003&0,0007&0,0029&0,0089&0,0257&0,0863&0,2045\\
     \hline
\end{tabular}
\end{center}
\end{table}


\begin{multicols}{2}


\noindent
числа каналов в линии при равной суммарной 
производительности. Кроме того, как видно из табл.~\ref{t5aga} и~\ref{t6aga}, 
вероятность блокировки в большей степени зависит от среднего значения 
длины интервала между повторными попытками передачи, чем от закона 
распределения длины интервала. Таким образом, предложенный в работе 
алгоритм позволяет вы\-чис\-лить с достаточной точностью вероятность 
блокировки узла, интенсивности повторных передач и предельную величину 
реализуемой нагрузки. Отметим, что полученные в данной статье результаты 
могут быть использованы для расчета нагрузок в телекоммуникационной сети с 
повторами заявок в предыдущем узле или из источника. 


{\small\frenchspacing
{%\baselineskip=10.8pt
\addcontentsline{toc}{section}{Литература}
\begin{thebibliography}{99}    
\bibitem{1aga}
\Au{Kamoun~F., Kleinrock~L.}
Analysis of shared finite storage in a computer networks node environment under 
general traffic conditions~// IEEE Trans. on Commun., 1980. Vol.~28. No.\,7. 
P.~992--1003.

\bibitem{6aga} %2
\Au{Агаларов~Я.\,М., Шоргин~С.\,Я.}
Рекуррентный метод вычисления параметров сетей связи~// Техника средств 
связи, 1986. Сер. <<Системы связи>>. Вып.~6. С.~42--46.

\bibitem{3aga}
\Au{Башарин Г.\,П., Бочаров~П.\,П., Коган~Я.\,А.}
Анализ очередей в вычислительных сетях.~--- М.: Наука, 1989. 

\bibitem{4aga}
\Au{Бочаров~П.\,П., Печинкин~А.\,В.}
Теория массового обслуживания.~--- М.: Изд-во РУДН, 1995. 

\bibitem{5aga}
\Au{Вишневский~В.\,М.} 
Теоретические основы проектирования компьютерных сетей.~--- М.: 
Техносфера, 2003. 

\bibitem{2aga} %6
\Au{Башарин Г.\,П.}
Лекции по математической теории телетрафика.~--- М.: Изд-во РУДН, 2007. 

\bibitem{7aga}
\Au{Таранцев~А.\,А.}
Инженерные методы теории массового обслуживания.~--- М.: Наука, 2007.

\bibitem{9aga} %8
\Au{D'Apice~C., De~Simone~T., Manzo~R., Rizelian~G.}
$M\vert G\vert 1\vert r$ retrial queueing system with priority service of primary 
customers and a customers-searching server~// Distributed Computer and 
Communication Networks. Stochastic Modelling and Optimization.~--- М.: 
Техносфера, 2003. P.~106--117.

\bibitem{8aga} %9
\Au{Klimenok~V.\,I., Kim~C.\,S.}
$BM\!AP$/$PH$/1 retrial system operating in random environment~// Proceedings of 
the 5th St.-Petersburg Workshop on Simulation, St.-Petersburg, June~26\,--\,July~2, 
2005.~--- St.-Petersburg: NII Chemistry St.-Petersburg University Publs., 
2005. P.~367--372.   

\bibitem{10aga}
\Au{Krishnamoorthy~A., Babu~S.}
$M\!AP\vert (PH,PH)/c$ retrial queue with selegeneration of priorities 
and non-preemptive service~// Proceedings of the 14th International Conference on 
Analytical and Stochastic Modeling Techniques and Applications, June~4--6, 
2007. Prague, Czech Republic.~--- Sbr.-Dudweiler: Digitaldruck Pirrot GmbH, 
2007. P.~70--74.

\bibitem{11aga}
\Au{Корн~Г., Корн~Т.}
Справочник по математике.~--- М.: Наука, 1974.

\label{end\stat}


\bibitem{12aga}
\Au{Buzen~J.\,P.}
Computational algorithm for closed queuing networks with exponential servers~// 
Communications ACM, 1973. Vol.~16. No.\,9. P.~527--531.
 \end{thebibliography}
}
}
\end{multicols}
 
 
  %8
\def\stat{kudr}

\def\tit{ПРИБЛИЖЕННЫЕ МЕТОДЫ РЕШЕНИЯ ЗАДАЧИ ДИАГНОСТИКИ ПЛОСКИМ 
ЗОНДОМ СИЛЬНОИОНИЗОВАННОЙ ПЛАЗМЫ С~УЧЕТОМ КУЛОНОВСКИХ 
СТОЛКНОВЕНИЙ}

\def\titkol{Приближенные методы решения задачи диагностики плоским 
зондом сильноионизованной плазмы} %с~учетом Кулоновских  столкновений}

\def\autkol{И.\,А.~Кудрявцева, А.\,В.~Пантелеев}
\def\aut{И.\,А.~Кудрявцева$^1$, А.\,В.~Пантелеев$^2$}

\titel{\tit}{\aut}{\autkol}{\titkol}

%{\renewcommand{\thefootnote}{\fnsymbol{footnote}}\footnotetext[1]
%{Работа поддержана Российским фондом фундаментальных исследований
%(проекты 11-01-00515а и 11-07-00112а), а также Министерством
%образования и науки РФ в рамках ФЦП <<Научные и
%научно-педагогические кадры инновационной России на 2009--2013~годы>>.}}


\renewcommand{\thefootnote}{\arabic{footnote}}
\footnotetext[1]{Московский авиационный институт, irina.home.mail@mail.ru}
\footnotetext[2]{Московский авиационный институт, avpanteleev@inbox.ru}

\vspace*{-2pt}

\Abst{Сформирована математическая модель, описывающая динамику сильноионизованной 
плазмы с учетом столкновений заряженных частиц вблизи плоского зонда. Модель включает уравнение 
Фоккера--Планка и уравнение Пуассона. Предложено два подхода к решению задачи: на основе метода 
статистических испытаний Мон\-те-Кар\-ло и на основе композиции метода крупных частиц и метода 
расщепления.} 

\vspace*{-2pt}

\KW{телекоммуникационные системы; метод Монте-Карло; метод крупных частиц; метод 
расщепления; зонд; уравнение Фоккера--Планка; уравнение Пуассона} 

\vspace*{-4pt}

 \vskip 8pt plus 9pt minus 6pt

      \thispagestyle{headings}

      \begin{multicols}{2}
      
            \label{st\stat}

\section{Введение}

В настоящее время в области телекоммуникаций все более востребованными становятся 
информационные технологии, основанные на использовании математических моделей и численных 
методов физики плазмы. Поэтому особенно актуальным является решение разнообразных задач анализа 
поведения плазмы, включающих в себя формирование новых моделей и методов их исследования. 
Помимо этого, в разработке телекоммуникационного оборудования эффективно используются 
собственно физические свойства плазмы. В~частности, изготовлена антенна, работа которой основана 
на газовом разряде низкотемпературной плазмы~[1], интенсивно ведутся разработки по созданию и 
усовершенствованию источников бесперебойного питания на основе плазменных элементов~[2, 3]. 
      
      Одним из наиболее перспективных направлений для построения систем оптической 
беспроводной связи является использование лазеров~\cite{4-k, 5-k}. В~этой связи большое внимание 
уделяется использованию плазмы при разработке импульсных сильноточных коммутаторов~\cite{6-k}, 
так как практическое применение подобных разработок требует повышения уровня надежности и 
быстродействия лазерных систем.
      
      Исследования низкотемпературной плазмы также связаны с разработками в области дальней 
космической связи, так как моделирование процессов взаимодействия заряженного тела с верхними 
слоями атмосферы позволяет предлагать способы улучшения существующих систем радиосвязи с 
космическими летательными аппаратами~\cite{7-k}. 
      
      Наряду с этим актуальными также являются задачи диагностики плазмы, поскольку перспективы 
ее использования в области телекоммуникаций после более полного изучения физических свойств 
могут значительно расшириться. 

Для диагностики плазмы применяют зондовые методы исследования~[8--11]. Эти методы относятся к 
классу контактных методов; как следствие, возникает сложность в исследовании пристеночной области 
вблизи зонда, которая характеризуется достаточно сложным распределением потенциала и функциями 
распределения, отличными от максвелловских. 

Данная работа посвящена исследованию переходного режима обтекания заряженного тела плазмой. Для 
переходного режима выполняется следующее условие: длина свободного пробега иона до столкновения 
с нейтральным атомом или другим ионом невелика по сравнению с характерными размерами тела. 
В~этом случае возникает необходимость учета столкновений заряженных частиц с нейтральными 
атомами и кулоновских столкновений. В~работах~\cite{10-k, 11-k} подробно рассмотрена модель с 
учетом столкновений заряженных частиц с нейтральными атомами. В~настоящей статье представлена 
теоретическая модель, описывающая влияния ион-ионных и ион-элек\-т\-рон\-ных столкновений на 
измеряемые характеристики плазмы, что ранее детально не исследовалось.
      
      В~рамках данной работы предлагается модель, описывающая динамику сильноионизованной 
плазмы с учетом кулоновских столкновений. Эта модель учитывает такие процессы взаимодействия, 
как перенос частиц и столкновения между заряженными частицами типа <<ион--ион>> и 
      <<ион--электрон>> под влиянием макроскопического электрического поля. Перечисленные 
процессы описываются самосогласованной системой уравнений, включающей уравнение 
      Фок\-ке\-ра--План\-ка и уравнение Пуассона~[12].
      
      Вычислительная модель задачи строится на основе двух методов: метода статистических 
испытаний Мон\-те-Кар\-ло и композиции метода крупных частиц и метода расщепления. Приведены 
результаты численного моделирования, полученные с использованием вышеперечисленных методов.

\vspace*{-4pt}

\section{Постановка задачи}

\vspace*{-2pt}

Рассматривается следующая физическая постановка зондовой задачи~[11]. В~невозмущенную 
бесконечно протяженную плазму, состоящую из электронов и однозарядных ионов, внесена большая\linebreak 
заряженная до потенциала $\varphi_p$ плоскость. Плоскость, расположенная поперек потока плазмы, 
является идеально поглощающей для электронов. Ионы при ударе о плоскость нейтрализуются. 
Предполагается, что частицы в плазме движутся под действием внешнего электрического поля, 
магнитное поле отсутствует. Концентрации ионов $n_{i\infty}$ и электронов $n_{e\infty}$, а также 
температуры данных час\-тиц~$T_{i\infty}$ 
и~$T_{e\infty}$ в невозмущенной плазме заданы. За начальные 
функции распределения обоих типов час\-тиц принимаются функции распределения Максвелла. 
      
      Требуется с учетом столкновений между заряженными частицами найти напряженность 
самосогласованного электрического поля $\vec{E}(\vec{r},t)$, функции распределения однозарядных 
ионов $f_i(\vec{r}, \vec{v}, t)$ и электронов $f_e(\vec{r}, \vec{v}, t)$, 
а также их моменты (плотности 
токов ионов и электронов  $j_i(\vec{r},t)\hm
=q\int f_i(\vec{r}, \vec{v}, t)\vec{v}\,d\vec{v}$, $j_e(\vec{r},t) 
\hm={\sf e}\int f_e(\vec{r},\vec{v},t)\vec{v}\,d\vec{v}$, где $q=Z_i{\sf e}$, $Z_i=1$~--- заряд иона, ${\sf 
e}$~--- заряд электрона; концентрации ионов и электронов $n_i(\vec{r},t)\hm=\int 
f_i(\vec{r},\vec{v},t)\,d\vec{v}$, $n_e(\vec{r},t)\hm=\int f_e(\vec{r},\vec{v}, t)\,d\vec{v}$). 
Поведение частиц во 
времени~$t$ характеризуется ра\-ди\-ус-век\-то\-ром~$\vec{r}$ и вектором скорости~$\vec{v}$.
      
      Математическая модель, соответствующая данной физической постановке задачи, имеет 
вид~\cite{11-k, 13-k}:

\noindent
      \begin{equation}
      \left.
      \begin{array}{c}
      \fr{\partial f_\alpha (\vec{r},\vec{v},t)}{\partial t}+
      \vec{v}\fr{\partial f_\alpha (\vec{r},\vec{v},t)}{ 
\partial \vec{r}}+
\fr{\vec{F}_\alpha(\vec{r},t)}{m_\alpha}\times{}\\[4pt]
{}\times\fr{\partial f_\alpha(\vec{r},\vec{v},t)}{ \partial 
\vec{v}}=
\left(\fr{\partial f_\alpha(\vec{r},\vec{v},t)}{ \partial t}\right)_{\mathrm{с}}+S_\alpha 
(\vec{r},\vec{v},t)\,;\\[6pt]
      \Delta\varphi(\vec{r},t)=-\fr{{\sf e}}{\varepsilon_0}\left( n_i(\vec{r},t)-n_e(\vec{r},t)\right)\,;\\[6pt]
      \vec{E}(\vec{r},t)=-\nabla \varphi(\vec{r},t)\,.
      \end{array}\!\!
      \right\}\!\!
      \label{e1-k}
      \end{equation}
Здесь первое уравнение~--- уравнение Фок\-ке\-ра--План\-ка для частиц сорта~$\alpha$ ($\alpha=i,e$), 
второе~--- уравнение Пуассона для самосогласованного электрического поля; 
$f_\alpha(\vec{r},\vec{v},t)$~--- функция\linebreak
распределения час\-тиц сорта~$\alpha$; $(\partial 
f_\alpha(\vec{r},\vec{v},t)/\partial t)_{\mathrm{с}}$~--- 
оператор столкновений Фок\-ке\-ра--План\-ка; 
функция~$S_\alpha(\vec{r},\vec{v},t)$ описывает источники или стоки\linebreak
 час\-тиц; 
$\vec{F}_\alpha(\vec{r},t)=q_\alpha\vec{E}(\vec{r},t)$, где $\vec{E}(\vec{r},t)$~--- напряженность 
самосогласованного электрического поля, 
$$
q_\alpha =
\begin{cases}
-{\sf e}\,, & \alpha=e\,,\\
{\sf e}\,, & \alpha=i\,;
\end{cases}
$$
$\varphi(\vec{r},t)$~--- потенциал самосогласованного электрического поля; $n_\alpha(\vec{r},t)$ ($\alpha 
\hm=i,e$)~--- концентрация частиц сорта~$\alpha$; $m_\alpha$~--- масса частицы сорта~$\alpha$; 
$\varepsilon_0$~--- электрическая постоянная. 

Оператор столкновений Фок\-ке\-ра--План\-ка имеет вид~\cite{13-k, 14-k}
\begin{multline*}
\fr{1}{\Gamma_\alpha}\left( \fr{\partial f_\alpha}{\partial t}\right)_{\mathrm{с}} 
=\fr{1}{2}\,\nabla_v\nabla_v:\left(f_\alpha\nabla_v\nabla_vg_\alpha(\vec{r},\vec{v},t)\right)-{}\\
{}-
\nabla_v\cdot\left(f_\alpha\nabla_v h_\alpha\right)\,,
\end{multline*}
где $\nabla_v\nabla_v g_\alpha(\vec{r},\vec{v},t)$~--- ковариантная тензорная производная второго ранга, 
знак двоеточия ($:$) обозначает операцию двойного суммирования:
\begin{gather*}
\Gamma_\alpha=\fr{Z_\alpha^4 {\sf e}^4}{4\pi \varepsilon_0^2 m^2_\alpha}\,\ln D_\alpha\,;
\\
D_\alpha =\fr{12\pi\varepsilon_0 kT_{\alpha\infty}}{Z_\alpha^2 {\sf e}^2}\left( \fr{\varepsilon_0 k 
T_{e\infty}}{n_{e\infty} {\sf e}^2}\right)^{1/2}\,;\\
g_\alpha (\vec{r},\vec{v},t)=\sum\limits_{b=i,e}\left( \fr{Z_b}{Z_\alpha}\right) \int f_b 
(\vec{r},{\vec{v}}^{\,\prime},t)\left\vert \vec{v}-{\vec{v}}^{\,\prime}\right\vert\,d\vec{v}^{\,\prime}\,;\\
h_\alpha (\vec{r},\vec{v},t)=\sum\limits_{b=i,e} \fr{m_\alpha+m_b}{m_b} 
\left(\fr{Z_b}{Z_\alpha}\right)
\int
\fr{f_b(\vec{r},{\vec{v}}^{\,\prime}, t)}{\vert \vec{v}-{\vec{v}}^{\,\prime}\vert}
\,d{\vec{v}}^{\,\prime}\,;\\
Z_\alpha =1\,, \quad \alpha=i,e\,.
\end{gather*}
 
К системе уравнений~(\ref{e1-k}) необходимо добавить начальные и краевые условия:
\begin{equation}
\!\left.
\begin{array}{rrl}
t=0:\ & f_\alpha(\vec{r},\vec{v},0)&=f_\alpha^{\mathrm{maksv}}\,,\enskip \alpha=i,e;\\[9pt]
\vec{r}\in \Omega_p:\ & f_\alpha(\vec{r},\vec{v},t)\big\vert_{\vec{r}\in\Omega_p}&=0\,,\enskip \alpha=i,e\,;\\[9pt]
&\varphi(\vec{r},t)\big\vert_{\vec{r}\in\Omega_p}&=\varphi_p\,;\\[9pt]
\vec{r}\in\Omega_\infty:\ & 
f_\alpha(\vec{r},\vec{v},t)\big\vert_{\vec{r}\in\Omega_\infty}&= %{}\\[9pt]
f_\alpha^{\mathrm{maksv}}\,,\enskip \alpha=i,e\,;\\[9pt]
&\varphi(\vec{r},t)\big\vert_{\vec{r}\in\Omega_\infty}&=0\,,
\end{array}\!\!
\right\}\!\!\!\!
\label{e2-k}
\end{equation}
    где 
    
    \noindent
    \begin{multline*}
    f_\alpha^{\mathrm{maksv}}=n_{\alpha\infty}\left(\fr{m_\alpha}{2k\pi T_{\alpha\infty}}\right)^{3/2}\times{}\\
    {}\times
    \exp\left( -
\fr{m_\alpha}{2kT_{\alpha\infty}}\left\vert\vec{v}-\vec{v}_\infty\right\vert^2\right)\,,
\enskip \alpha=i, e\,;
\end{multline*} 
$\Omega_p$ и $\Omega_\infty$~--- множество радиус-векторов час\-тиц, концы которых принадлежат плоскости зонда и 
границе возмущенной зоны соответственно.

Для решения поставленной задачи введем декартову систему координат таким образом, чтобы 
заряженная плоскость совпала с плоскостью~$0xz$. Тогда положение частицы в пространстве будет 
определяться координатами $x,y,z$, а скорость~--- координатами $v_x, v_y, v_z$. В~силу того что 
плоскость является бесконечно большой в сравнении с характерным размером задачи, функции 
распределения частиц будут зависеть только от переменных $y, v_y, t$.

Поставленную задачу предлагается решать независимо двумя методами. Первый метод основывается на 
методе статистических испытаний Мон\-те-Кар\-ло, второй метод является композицией метода 
расщепления и метода крупных частиц.

\section{Применение метода Монте-Карло}

Запишем самосогласованную систему уравнений~(\ref{e1-k}) и~(\ref{e2-k}) в декартовой системе 
координат с учетом сделанных предположений:
\begin{equation}
\left.
\begin{array}{l}
\fr{\partial f_\alpha}{\partial t}+
v_y\fr{\partial f_\alpha}{\partial y}+\fr{F_y^\alpha}{m_\alpha}\,\fr{\partial 
f_\alpha}{\partial v_y}=\fr{1}{2}\,\fr{\partial^2 }{\partial [v_y]^2}\times{}\\
{}\times \left( 
f_\alpha\fr{\partial^2 g_\alpha  }{\partial [v_y]^2}\right) -
\fr{\partial}{\partial v_y}\left( f_\alpha\fr{\partial h_\alpha}{\partial v_y}\right)\,,
\enskip \alpha=i,e\,;\\[6pt]
    \fr{\partial^2\varphi}{\partial y^2} =-\fr{{\sf e}}{\varepsilon_0}\left(n_i-n_e\right)\,;
    \enskip E_y=-
\fr{\partial\varphi}{\partial y}\,;\\[6pt]
\hspace*{3.1mm}    t=0:\  \hspace*{2.6mm}f_\alpha(y,v_y,0)=f_\alpha^{\mathrm{maksv}}\,,\ \alpha=i,e\,;\\[9pt]
\hspace*{2.9mm} y=0:\ \hspace*{2.8mm}f_\alpha(0,v_y,t)=0\,,\ \alpha=i,e\,;\\[9pt]
\hspace*{24.3mm}\varphi(0,t)=\varphi_p\,;\\[9pt]
y=y_\infty:\ f_\alpha(y_\infty, v_y, t)=f_\alpha^{\mathrm{maksv}}\,,\ \alpha=i,e\,;\\[9pt]
\hspace*{21.5mm}\varphi(y_\infty, t)=0\,.
\end{array}
\right \}
\label{e3-k}
\end{equation}

В полученной системе уравнений~(\ref{e3-k}) перейдем к безразмерным величинам, применив 
соотношение $X=M_X \hat{X}$, где $M_X$~--- масштаб размерной величины~$X$, $\hat{X}$~--- 
безразмерная величина~$X$. В~качестве используемых масштабов были взяты следующие: радиус 
Дебая, скорость теплового движения частиц, концентрация частиц в невозмущенной плазме, потенциал, 
возникающий при разделении зарядов в дебаевской сфере, и производные от них величины.

Система безразмерных уравнений имеет следующий вид:
%\noindent
\begin{equation}
\left.
\begin{array}{l}
\fr{\partial 
\hat{f}_\alpha}{\partial\hat{t}}+A_\alpha\fr{\partial\hat{f}_\alpha}{\partial\hat{y}}+
B_\alpha\hat{E}_y\fr{\partial\hat{f}_\alpha}{\partial \hat{v}_y}={}\\
\!{}=
\fr{\partial^2}{\partial[\hat{v}_y]^2}\left(D_\alpha 
\hat{f}_\alpha\right)-\fr{\partial}{\partial\hat{v}_y}\left(K_\alpha \hat{f}_\alpha\right),\enskip 
\alpha=i,e;\\[9pt]
\fr{\partial^2\hat{\varphi}}{\partial\hat{y}^2}=-\left(\hat{n}_i-\hat{n}_e\right)\,;\enskip \hat{e}_y=-
\fr{\partial\hat\varphi}{\partial\hat{y}}\,;\\[9pt]
\hspace*{3.1mm}\hat{t}=0:\ \hspace*{2.6mm}\hat{f}_\alpha(\hat{y},\hat{v}_y,0)=\hat{f}_\alpha^{\mathrm{maksv}}\,,\enskip \alpha-i,e\,;\\[9pt]
\hspace*{2.9mm}\hat{y}=0:\ \hspace*{2.8mm}\hat{f}_\alpha(0,\hat{v}_y,\hat{t})=0\,,\enskip \alpha=i,e\,;\\[9pt]
\hspace*{24.3mm}\hat\varphi(0,\hat{t})=\hat{\varphi}_p\,;\\[9pt]
\hat{y}=\hat{y}_\infty:\ \hat{f}_\alpha(\hat{y}_\infty, \hat{v}_y, \hat{t})=\hat{f}^{\mathrm{maksv}}_\alpha\,,\enskip 
\alpha=i,e\,;\\[9pt]
\hspace*{21.5mm}\hat\varphi(\hat{y}_\infty,\hat{t})=0\,.
\end{array}
\right\}
\label{e4-k}
\end{equation}
Здесь 

\vspace*{-2pt}

\noindent
\begin{gather*}
A_\alpha=\sqrt{\delta_\alpha }\,\hat{v}_y\,;\enskip 
B_\alpha=\sqrt{\delta_\alpha}\,\fr{z_\alpha}{2\varepsilon_\alpha}\,;\\
\delta_\alpha=\fr{\varepsilon_\alpha}{\mu_\alpha}\,;\enskip 
\varepsilon_\alpha=\fr{T_{\alpha\infty}}{T_{i\infty}}\,;\\
\mu_\alpha=\fr{m_\alpha}{m_i}\,;\enskip 
D_\alpha=A_g^\alpha\fr{\partial^2\hat{g}_\alpha}{\partial  [\hat{v}_y]^2}\,;\\
K_\alpha=A_h^\alpha \fr{\partial \hat{h}_\alpha}{\partial \hat{v}_y}\,,\enskip \alpha=i,e\,,
\end{gather*}
где $A_g^\alpha$ и $A_h^\alpha$~--- коэффициенты, определяемые характерными параметрами 
задачи~\cite{15-k}.

Поиск решения самосогласованной системы уравнений~(\ref{e4-k}) осуществляется по следующей 
схе-\linebreak ме. Вначале находятся значения напряженности\linebreak
 электрического поля по значениям потенциала, 
полученным из граничной задачи для уравнения Пуассона. Далее, используя найденные значения 
напряженности, решается уравнение Фок\-ке\-ра--План\-ка путем перехода к стохастическому 
дифференциальному уравнению (СДУ) Ито:

\noindent
\begin{multline*}
d\Theta_\alpha(\hat{t}) = a_\alpha \left(\hat{t},\Theta_\alpha(\hat{t})\right)+{}\\
{}+\sigma\left(
\hat{t},\Theta_\alpha(\hat{t})\right)\,dW(\hat{t})\,,\quad \alpha=i,e\,,
%\label{e5-k}
\end{multline*}
где 

\noindent
\begin{align*}
\Theta_\alpha(\hat{t})&=\begin{bmatrix}
\hat{y}(\hat{t})\\ \hat{v}_y(\hat{t})
\end{bmatrix}\,;\\
a_\alpha\left(\hat{t},\Theta_\alpha(\hat{t})\right)&=\begin{bmatrix}
-A_\alpha\\ -K_\alpha -B_\alpha \hat{E}_y
\end{bmatrix}\,;\\
\sigma_\alpha\left(\hat{t},\Theta_\alpha(\hat{t})\right)\sigma_\alpha^{\mathrm{T}}\left( 
\hat{t},\Theta_\alpha(\hat{t})\right)&=D_\alpha\,,\enskip \alpha=i,e\,;
\end{align*} 
$W(\hat{t})$~--- стандартный винеровский случайный процесс.
\pagebreak

Для нахождения значений вектора состояния~$\Theta_\alpha(\hat{t})$ применим явную разностную 
схему стохастического метода Эйлера~\cite{16-k}:
\begin{multline*}
\Theta_\alpha^{n+1}=\Theta_\alpha^n +h_\tau a_\alpha \left( \hat{t}_n, \Theta_\alpha^n\right)+\sigma_\alpha 
\left( \hat{t}_n, \Theta_\alpha^n\right)\Delta W_n\,,\\ 
n=0,\ldots , N\,,\ \alpha=i,e\,,
%\label{e6-k}
\end{multline*}
где $\Theta_\alpha^n$, $n=0,\ldots , N$,~--- приближенное значение вектора 
состояния~$\Theta_\alpha(\hat{t})$, $\alpha=i,e$, в момент времени $\hat{t}\hm=\hat{t}_n$, 
$\hat{t}_n\hm=n h_\tau$, $n=0,\ldots , N$; $h_\tau$~--- достаточно малый шаг интегрирования; $\Delta 
W_n$, $n=0,\ldots ,N$,~--- величина приращения винеровского процесса~$W(\hat{t})$ на отрезке $\left[ 
\hat{t}_n,\,\hat{t}_{n+1}\right]$, по определению независимая от~$\Theta_\alpha^0$, 
$\Delta W_0,\ldots , 
\Delta W_{n-1}$: $\Delta W_n\hm=W(\hat{t}_{n-1})\hm-W(\hat{t}_n)$; $\Delta W_n\hm\sim N(0,\,h_\tau)$, 
т.\,е.\ $\Delta W_n$ представляют собой гауссовские случайные величины с нулевыми математическими 
ожиданиями и дисперсиями, равными шагу интегрирования; $\Theta_\alpha^0$~--- значение вектора 
состояния $\Theta_\alpha(\hat{t})$, $\alpha\hm=i,e$, в момент времени $\hat{t}=0$, 
$\Theta_\alpha^0\hm\sim \hat{f}_\alpha^{\mathrm{maksv}}$. 

Частные производные $\partial^2\hat{g}_\alpha/\partial[\hat{v}_y]^2$ и $\partial \hat{h}_\alpha/\partial 
\hat{v}_y$, являющиеся составляющими матрицы $\sigma_\alpha (\hat{t}_n, 
\Theta_\alpha^n)\sigma_\alpha^{\mathrm{T}}(\hat{t}_n,\Theta_\alpha^n)$ и вектора $a_\alpha(\hat{t}_n, 
\Theta_\alpha^n)$ соответственно, аппроксимируются со вторым порядком точности на трехточечном 
шаблоне на основе значений~$\hat{g}_\alpha$ и~$\hat{h}_\alpha$~\cite{17-k}.
      
      В выражения для функций~$\hat{g}_\alpha$ и~$\hat{h}_\alpha$ входят интегралы, которые 
вычисляются методом Мон\-те-Кар\-ло с использованием набора значений скоростной компоненты 
вектора состояния~$\hat{v}_y$, полученных из решения СДУ Ито:
      \begin{equation*}
      \int \hat{f}_\alpha \left\vert \hat{v}_y-
\hat{v}_y^\prime\right\vert\,dv_y^\prime=M\left(\zeta\left(\hat{V}_y\right)\right)\,,
\end{equation*}
где
$$
      \zeta\left(\hat{V}_y\right)=\left\vert \hat{v}_y-\hat{V}_y\right\vert\,,\enskip \hat{V}_y\sim 
\hat{f}_\alpha\,.
  $$
      
      Для вычисления напряженности самосогласованного электрического поля $\hat{E}_y=-
\partial\hat{\varphi}/\partial\hat{y}$, входящей в вектор $a_\alpha(\hat{t}_n, \Theta_\alpha^n)$, необходимо 
аналогично аппроксимировать со вторым порядком точности производную 
$\partial\hat{\varphi}/\partial\hat{y}$ на трехточечном шаблоне с использованием значений 
потенциала~$\hat{\varphi}$~\cite{17-k}. Значения потенциала~$\hat\varphi$ находятся из решения 
уравнения Пуассона. 
      
      Граничную задачу для уравнения Пуассона 
      \begin{align*}
      \fr{\partial^2 \hat\varphi}{\partial \hat{y}^2} & = -\left(\hat{n}_i-\hat{n}_e\right)\,;\\
      \hat{\varphi}\big|_{\hat{y}=0} &=\hat{\varphi}_p\,;\\
      \hat{\varphi}\big|_{\hat{y}_\infty=0} &=0
      \end{align*}
    предлагается решать путем перехода к конечно-разностной системе с последующим ее решением 
методом прогонки~\cite{17-k}:

\noindent
\begin{gather*}
\hat{\varphi}^n_{l-1}+2\hat{\varphi}_l^n+\hat{\varphi}^n_{l+1}=
h_y\hat{\delta}_l^n\,,\enskip l=1,\ldots , 
N_y\,;\\
\hat{\delta}_l^n=-\left( \hat{n}^n_{i,l}-\hat{n}^n_{e,l}\right)\,;\enskip 
\hat{\varphi}_0=\hat{\varphi}_p\,;\enskip \hat{\varphi}_{N_y}=0\,,
\end{gather*}
где $N_y$~--- число шагов по переменной~$\hat{y}$, $h_y$~--- величина шагов разбиения по~$\hat{y}$. 
      
      Концентрации $\hat{n}_\alpha$, $\alpha=i,e$, и плотности токов частиц на зонд~$\hat{f}_\alpha$, 
$\alpha=i,e$, вычисляются согласно описанному выше методу Мон\-те-Карло.

\section{Применение метода расщепления и~метода крупных~частиц}

Решение задачи в данном случае предлагается начать с записи правой части уравнения 
Фок\-ке\-ра--План\-ка в декартовой системе координат в виде:
$$
\mathbf{Q} f_\alpha = \fr{1}{2}\,\fr{\partial^2 f_\alpha}{\partial [v_y]^2}\,\fr{\partial^2 g_\alpha}{\partial 
[v_y]^2}+\fr{\partial f_\alpha}{\partial v_y}\,\fr{\partial C_\alpha}{\partial v_y}+H_\alpha\,,\enskip 
\alpha=i,e\,,
$$  
где 
\begin{align*}
C_\alpha(\vec{r},\vec{v},t)&=
\begin{cases}
\fr{1-\gamma}{Z_i^2}\int\fr{f_e(\vec{r},{\vec{v}}^{\,\prime},t)}{|\vec{v}-{\vec{v}}^{\,\prime} |}\,d{\vec{v}}^{\,\prime}\,, 
&\alpha=i\,;\\[9pt]
\fr{Z_i^2(\gamma-1)}{\gamma}\int \fr{f_i(\vec{r},{\vec{v}}^{\,\prime}, t)}
{|\vec{v}-{\vec{v}}^{\,\prime} 
|}\,d{\vec{v}}^{\,\prime}\,, &\alpha=e\,;
\end{cases} 
\\
H_\alpha&=
\begin{cases}
4\pi \left( \fr{\gamma f_e}{Z_i^2}+f_i\right)f_i\,, & \alpha=i\,;\\[9pt]
4\pi\left(\fr{Z_i^2 f_i}{\gamma}+f_e\right)f_e\,, &\alpha=e\,.
\end{cases}
\end{align*}
Тогда при переходе к безразмерным величинам (см.\ разд.~3) система~(\ref{e1-k}) запишется 
следующим образом:
      \begin{equation}
      \left.
\!\!\begin{array}{l}
      \fr{\partial 
\hat{f}_\alpha}{\partial\hat{t}}+A_\alpha\fr{\partial\hat{f}_\alpha}{\partial\hat{y}}+
B_\alpha  \hat{E}_y
\fr{\partial\hat{f}_\alpha}{\partial\hat{v}_\alpha}=\tilde{\mathbf{Q}}\hat{f}_\alpha\,,\enskip 
\alpha=i,e;\\[9pt]
      \fr{\partial^2\hat{\varphi}}{\partial\hat{y}^2}=-\left( \hat{n}_i-\hat{n}_e\right)\,,\enskip \hat{E}_y=-
\fr{\partial\hat\varphi}{\partial\hat{y}}\,,\\[9pt]
\hspace*{3.1mm}\hat{t}=0:\ \hspace*{2.6mm}\hat{f}_\alpha(\hat{y},\hat{v}_y, 0)=\hat{f}_\alpha^{\mathrm{maksv}}\,,\enskip \alpha=i,e\,,\\[9pt]
\hspace*{2.9mm} \hat{y}=0:\ \hspace*{2.8mm}\hat{f}_\alpha(0,\hat{v}_y,\hat{t})=0\,,\enskip \alpha=i,e\,;\\[9pt]
\hspace*{24.3mm}\hat\varphi(0,\hat{t})=\hat{\varphi}_p\,;\\[9pt]
      \hat{y}=\hat{y}_\infty:\ \hat{f}_\alpha(\hat{y}_\infty, 
\hat{v}_y,\hat{t})=\hat{f}_\alpha^{\mathrm{maksv}}\,,\enskip \alpha=i,e\,;\\[9pt]
\hspace*{21.5mm}\hat{\varphi}(\hat{y}_\infty,\hat{t})=0\,,\\[9pt]
    \end{array}
\right\}\!\!
\label{e7-k}
\end{equation}
где 
\begin{gather*}
\tilde{\mathbf{Q}} \hat{f}_\alpha=D_\alpha\fr{\partial^2\hat{f}_\alpha}{\partial 
[\hat{v}_y]^2}+K_\alpha\fr{\partial\hat{f}_\alpha}{\partial\hat{v}_y}+H_\alpha\,;\\
D_\alpha=A_g^\alpha\fr{\partial^2\hat{g}_\alpha}{\partial [\hat{v}_y]^2}\,;\enskip 
K_\alpha=A_h^\alpha \fr{\partial \hat{h}_\alpha}{\partial\hat{v}_y}\,,\ \alpha=i,e\,.
\end{gather*}

Для решения системы уравнений~(\ref{e7-k}) применяется модификация метода 
расщепления~\cite{17-k}, согласно которой исходная задача разбивается на две вспомогательные. Такое 
разбиение можно осуществить, переписав уравнение Фок\-ке\-ра--План\-ка в следующем виде:
$$
\fr{\partial\hat{f}_\alpha}{\partial\hat{t}} =
\tilde{\mathbf{Q}}_1\hat{f}_\alpha+\tilde{\mathbf{Q}}_2\hat{f}_\alpha\,,
$$
где 
\begin{align*}
\tilde{\mathbf{Q}}_1\hat{f}_\alpha &=-
\left(A_\alpha\fr{\partial\hat{f}_\alpha}{\partial\hat{y}}+
B_\alpha\fr{\partial\hat{f}_\alpha}{\partial\hat{y}}
\right)\,;\\
\tilde{\mathbf{Q}}_2\hat{f}_\alpha 
&=\left(D_\alpha\fr{\partial^2\hat{f}_\alpha}{\partial[\hat{v}_y]^2}+K_\alpha\fr{\partial 
\hat{f}_\alpha}{\partial\hat{v}_y}+H_\alpha\right)\,.
\end{align*}

      Правая часть уравнения Фок\-ке\-ра--План\-ка представляет собой сумму двух операторов, 
первый из которых отвечает за перенос частиц, второй~--- за столкновения заряженных частиц. 
В~результате образуются следующие задачи, которые решаются последовательно:
      \begin{itemize}
\item первая задача:
\begin{align*}
&\fr{\partial w_\alpha(\hat{y},\hat{v}_y,\hat{t})}{\partial\hat{t}} =\mathbf{Q}_1 
w_\alpha(\hat{y},\hat{v}_y,\hat{t})\,,\enskip \alpha=i,e\,;\\[9pt]
&\fr{\partial^2\hat\varphi}{\partial\hat{y}^2}=-\left(\hat{n}_i-\hat{n}_e\right)\,;\enskip
\hat{E}_y=-
\fr{\partial\hat\varphi}{\partial\hat{y}}\,;\\[9pt]
&w_\alpha(\hat{y},\hat{v}_y,\hat{t}^n)=\hat{f}_\alpha(\hat{y},\hat{v}_y,\hat{t}^n)\,,\enskip n=0,\ldots ,N-
1\,;\\[9pt]
&\hspace{2.9mm}\hat{y}=0:\ \hspace*{2.9mm}w_\alpha(0,\hat{v}_y,\hat{t})=0\,,\enskip \alpha=i,e\,;\\[9pt]
&\hspace*{25.1mm}\hat\varphi(0,\hat{t})=\hat{\varphi}_p\,;\\[9pt]
&\hat{y}=\hat{y}_\infty:\ w_\alpha(\hat{y}_\infty, \hat{v}_y, \hat{t})=
\hat{f}_\alpha^{\mathrm{maksv}}\,,\enskip 
\alpha=i,e\,;\\[9pt]
&\hspace*{22.5mm}\hat\varphi(\hat{y}_\infty,\hat{t})=0\,;
\end{align*}
\item вторая задача:
\begin{align*}
\!\!\!\!\!\!\!\fr{\partial s_\alpha(\hat{y},\hat{v}_y,\hat{t})}{\partial \hat{t}} &=\mathbf{Q}_2 
s_\alpha(\hat{y},\hat{v}_y,\hat{t})\,, & \alpha&=i,e\,;\\
\!\!\!\!\!\!\!s_\alpha (\hat{y},\hat{v}_y,\hat{t}^n) &=w_\alpha (\hat{y},\hat{v}_y, \hat{t}^{n+1}),& n&=0,\ldots ,N-
1.
\end{align*}
\end{itemize}

Первая задача представляет собой систему безразмерных уравнений Вла\-со\-ва--Пуас\-со\-на. Для ее 
решения применяется метод крупных частиц~\cite{18-k}. Согласно этому методу решение задачи 
осуществляется путем расщепления на два этапа: на первом этапе не учитываются конвективные члены 
и решение получается обычным интегрированием на неподвижной эйлеровой сетке, а на втором этапе 
рассматривается система, которая описывает перенос частиц в лагранжевой системе координат. Кроме 
того, на первом этапе необходимо решить уравнение Пуассона для получения значений потенциала 
самосогласованного электрического поля. Для этого применяется метод, описанный в разд.~3. 

Вторая задача решается путем перехода к ко\-неч\-но-раз\-ност\-ной сис\-те\-ме. При этом частные 
производные $\partial^2\hat{g}_\alpha/\partial[\hat{v}_y]^2$ и $\partial\hat{h}_\alpha/\partial\hat{v}_y$ 
аппроксимируются со вторым порядком точности с использованием трехточечного шаблона, а 
производная $\partial s_\alpha/\partial\hat{t}$ аппроксимируется на двухточечном шаблоне с первым 
порядком точности~\cite{16-k}. К~полученной системе разностных уравнений предлагается применить 
один из классических методов решения систем линейных уравнений, например метод 
Гаусса~\cite{19-k}.
      
      Решением первой задачи является функция $w_\alpha(\hat{y}, \hat{v}_y, \hat{t}^n)$, 
$n\hm=0,\ldots ,N$, , которая дает начальное условие для второй задачи. Решая вторую задачу, находим 
функцию $s_\alpha(\hat{y},\hat{v}_y,\hat{t}^n)\hm=\hat{f}_\alpha(\hat{y},\hat{v}_y,\hat{t}^n)$, 
$n=1,\ldots ,N$, $\alpha=i,e$, которая определяет решение $\hat{f}_\alpha(\hat{y},\hat{v}_y,\hat{t}^n)$, 
$\alpha=i,e$, исходной системы~(\ref{e7-k}) для рассматриваемых моментов времени $n=1,\ldots ,N$.

Моменты функций распределения $\hat{f}_\alpha$, $\alpha=i,e$, находятся с помощью методов 
численного интегрирования, например метода трапеций~\cite{19-k}.

\section{Результаты численного моделирования}

Для двух описанных выше методов реализованы две отдельные программы в среде {Matlab~7.0}. 
Эти программы позволяют по заданным значениям концентраций и температур частиц $n_{i\infty}$, 
$n_{e\infty}$, $T_{i\infty}$ и~$T_{e\infty}$ в невозмущенной плазме, а также потенциала~$\varphi_p$, 
подаваемого на зонд, изучить эволюцию во времени плотностей тока частиц~$j_i$ и~$j_e$, концентраций 
частиц~$n_i$  и~$n_e$ в произвольной точке пространства в возмущенной зоне, а также динамику 
изменения напряженности~$E_y$ самосогласованного электрического поля во времени и пространстве.

С использованием разработанных программ проведены серии расчетных экспериментов, в которых 
значение концентраций варьировалось в пределах $n_{i\infty} \hm = n_{e\infty}\hm =10^{18}\div 
10^{22}$~м$^{-3}$. Значение температур было выбрано неизменным и равным $T_{i\infty}\hm = 
T_{e\infty}\hm=3000$~K, а значения потенциала, подаваемого на зонд, изменялись в пределах 
$\varphi_p\hm=0\div 2{,}6$~В.

На рис.~1  и~2 приведены графики изменения напряженности самосогласованного электрического
 поля (см.\ рис.~1) и плотности токов ионов (см.\linebreak\vspace*{-12pt}

\pagebreak

\end{multicols}

\begin{figure} %fig1
\vspace*{1pt}
\begin{center}
\mbox{%
\epsfxsize=162.594mm
\epsfbox{kud-1.eps}
}
\end{center}
\vspace*{-9pt}
\Caption{Динамика изменения плотности тока ионов во времени в фиксированной точке возмущенной 
зоны для значений потенциала: \textit{1}~--- $\varphi_p=-6$; 
\textit{2}~--- $\varphi_p=-16$; \textit{3}~--- $\varphi_p=- 30$ 
в случае применения методов Монте-Карло~(\textit{а}) 
и крупных частиц~(\textit{б})}
\end{figure}

\begin{figure} %fig2
\vspace*{1pt}
\begin{center}
\mbox{%
\epsfxsize=162.713mm
\epsfbox{kud-2.eps}
}
\end{center}
\vspace*{-9pt}
\Caption{Динамика изменения напряженности электрического поля во времени в фиксированной точке 
возмущенной зоны для значений потенциала: 
\textit{1}~--- $\varphi_p=-6$; \textit{2}~--- $\varphi_p=-16$; 
\textit{3}~--- $\varphi_p=-30$ в случае применения методов Монте-Карло~(\textit{а}) и
крупных частиц~(\textit{б})
}
\end{figure}

\begin{multicols}{2}

\noindent
 рис.~2) во времени в фиксированной точке пространства 
возмущенной зоны в случае применения обоих разработанных алгоритмов.


На основании полученных результатов можно отметить похожее поведение зависимостей 
напряженности электрического поля и плотности тока от времени в двух рассматриваемых случаях. 
Графики кривых сначала убывают, затем начинают возрастать, выходя в некоторый момент 
времени~$t^\prime$ (момент установления) на стационарные значения. 

Одинаковое поведение 
напряженности и плот\-ности тока можно объяснить из следующих соображений: плотность тока ионов в 
данной области пространства равна произведению концентрации ионов на их направленную скорость и 
на заряд иона. Скорость ионов, в свою очередь, зависит от заряда, массы и напряженности 
электрического поля. 
%\columnbreak

При внесении в плазму отрицательно заряженного зонда возникает электрическое поле, которое 
нарушает квазинейтральность плазмы. Для того чтобы компенсировать действие внешнего 
электрического поля, ионы устремляются к зонду, а электроны~--- от зонда. Это приводит к дисбалансу 
концентраций вблизи зонда и, как следствие, к увеличению разности потенциалов; график 
напряженности электрического поля убывает. Вскоре разделение зарядов компенсирует внешнее 
электрическое поле; график выходит на стационарное значение. 

Также можно отметить, что значения 
напряженности электрического поля и плотности тока частиц на зонд в момент установления для двух 
методов совпадают. 

Момент установления~$t^\prime$ зависит от при\-ме\-ня\-емо\-го метода решения. В~случае метода 
Мон\-те-Кар\-ло $t^\prime=3{,}5\div 4$~ед., а для метода крупных частиц совместно с методом 
расщепления $t^\prime\hm=5\div 5{,}5$~ед. Используя ко\-неч\-но-раз\-ност\-ный метод, можно 
получить динамику изменения функций распределения частиц~$f_\alpha$, $\alpha=i,e$, во времени и 
пространстве. Функции распределения позволяют наглядно представить влияние на картину 
распределения частиц вблизи зонда самой поверхности зонда и электрического поля.

\section{Заключение}
      
      В работе найдено решение задачи диагностики плоским зондом сильноионизованной плазмы с 
учетом столкновений заряженных частиц. Разработана математическая модель исследуемого явления, 
описываемая уравнениями Фок\-ке\-ра--План\-ка и Пуассона. Решение получено двумя методами:\linebreak 
статистическим и ко\-неч\-но-раз\-ност\-ным на основе\linebreak сформированных алгоритмов. Приведены 
резуль-\linebreak таты численного моделирования при различных\linebreak характерных параметрах задачи.
 Из  проведенных 
вычислительных экспериментов вытекает, что искомые величины: напряженность 
электрического поля, плотности токов частиц на зонд, концентрации частиц вблизи зонда~--- как по 
характеру зависимости, так и по числовым значениям совпадают. При применении метода 
      Мон\-те-Кар\-ло момент установления наступает быстрее по сравнению с конечно-разностным 
методом, однако конечно-разностный метод позволяет получить более наглядные результаты.

{\small\frenchspacing
{%\baselineskip=10.8pt
\addcontentsline{toc}{section}{Литература}
\begin{thebibliography}{99}

\bibitem{1-k}
\Au{Alexeff I., Anderson T.}
Experimental and theoretical results with plasma antenna~// IEEE Trans. Plasma Sci., 2006. Vol.~34. 
No.\,2. P.~166--172.

\bibitem{2-k}
\Au{Сысун В.\,И.}
Сильноионизованная низкотемпературная плазма в приборах электронной техники: Методы 
исследования, свойства, применение. Дисс. \ldots д-ра физ.-мат. наук в форме науч. докл.: 
01.04.08.~--- Пет\-ро\-за\-водск, 1996.

\bibitem{3-k}
\Au{Тухас В.\,А.}
Методология создания средств измерений и испытаний на устойчивость к кондуктивным помехам~// 
Мат-лы VI Междунар. симп. по электромагнитной совместимости и 
электромагнитной экологии.~--- СПб., 2005. С.~231--234.

\bibitem{4-k}
\Au{Гудзенко Л.\,И., Яковленко С.\,И.}
Плазменные лазеры.~--- М.: Атомиздат, 1978.  256~с.

\bibitem{5-k}
\Au{Звелто О.}
Принципы лазеров.~--- М.: Мир, 1990.  560~с.

\bibitem{6-k}
\Au{Сысун В.\,И., Хромой Ю.\,Д.}
Расширение канала мощного импульсного разряда в парах ртути~// Электронная техника, 1974. 
Сер.~4. Вып.~10. С.~80--85. 

\bibitem{7-k}
\Au{Винклер Дж.\,Р.}
Искусственные пучки частиц в космической плазме.~--- М.: Мир, 1985.  451~с.

\bibitem{8-k}
\Au{Bernstein I.\,B., Rabinowitz I.\,N.}
Theory of electrostatic probes in low-density plasma~// Phys. Fluids, 1959. Vol.~2. No.\,2. P.~112--121. 

\bibitem{9-k}
\Au{Альперт Я.\,Л., Гуревич А.\,В., Питаевский~Л.\,П.}
Искусственные спутники в разреженной плазме.~--- М.: Наука, 1964.  282~с.

\bibitem{10-k}
\Au{Чан П., Тэлбот Л., Турян~К.}
Электрические зонды в неподвижной и движущейся плазме.~--- М.: Мир, 1978.  202~с.

\bibitem{11-k}
\Au{Алексеев Б.\,В., Котельников В.\,А.}
Зондовый метод диагностики плазмы.~--- М.: Энергоатомиздат, 1989.  240~с.

\bibitem{12-k}
\Au{Пантелеев А.\,В., Кудрявцева И.\,А.}
Формирование математической модели двухкомпонентной плазмы с учетом столкновений 
заряженных частиц в случае плоского зонда~// Теоретические вопросы вычислительной техники и 
программного обеспечения: Межвузовский сб. научн. тр.~--- М.: МИРЭА, 2006. С.~11--21.

\bibitem{13-k}
\Au{Олдер Б.}
Вычислительные методы в физике плазмы.~--- М.: Мир, 1974.  111~с.

\bibitem{14-k}
\Au{Montgomery D.\,C., Tidman D.\,A.}
Plasma kinetic theory.~--- New York, 1964. 

\bibitem{15-k}
\Au{Кудрявцева И.\,А., Пантелеев А.\,В.}
Применение метода Мон\-те-Кар\-ло для анализа поведения двухкомпонентной плазмы с учетом 
столкновений между заряженными частицами~// Теоретические вопросы\linebreak
вычислительной техники и 
программного обеспечения: Межвузовский сб. научн. тр.~--- М.: МИРЭА, 2008. С.~122--128. 

\bibitem{16-k}
\Au{Семенов В.\,В., Пантелеев А.\,В., Руденко~Е.\,А., Бор\-та\-ков\-ский~А.\,С.}
Методы описания, анализа и синтеза нелинейных систем управления.~--- М.: МАИ, 1993.  312~с.

\bibitem{17-k}
\Au{Киреев В.\,И., Пантелеев А.\,В.}
Численные методы в примерах и задачах.~--- М.: Высшая школа, 2006.  480~с.

\bibitem{18-k}
\Au{Белоцерковский О.\,М., Давыдов~Ю.\,М.}
Метод крупных частиц в газовой динамике. Вычислительный эксперимент.~--- М.: Наука, 
Физматгиз, 1982.

\label{end\stat}

\bibitem{19-k}
\Au{Вержбицкий В.\,М.}
Основы численных методов.~--- М.: Высшая школа, 2002.  840~с.
 \end{thebibliography}
}
}


\end{multicols}        %9
\def\stat{ulianov}

\def\tit{РАЗЛОЖЕНИЯ ТИПА КОРНИША--ФИШЕРА ДЛЯ~РАСПРЕДЕЛЕНИЙ СТАТИСТИК, ПОСТРОЕННЫХ 
ПО~ВЫБОРКАМ~СЛУЧАЙНОГО РАЗМЕРА$^*$}

\def\titkol{Разложения типа Корниша--Фишера для распределений статистик, построенных по 
выборкам случайного размера}

\def\aut{А.\,С.~Марков$^1$,  М.\,М.~Монахов$^2$, В.\,В.~Ульянов$^3$}

\def\autkol{А.\,С.~Марков,  М.\,М.~Монахов, В.\,В.~Ульянов}

\titel{\tit}{\aut}{\autkol}{\titkol}

\index{Марков А.\,С.}
\index{Монахов М.\,М.}
\index{Ульянов В.\,В.}
\index{Markov A.\,S.}
\index{Monakhov M.\,M.}
\index{Ulyanov V.\,V.}

{\renewcommand{\thefootnote}{\fnsymbol{footnote}} \footnotetext[1]
{Исследование выполнено при поддержке Российского научного фонда (проект 14-11-00364).}}


\renewcommand{\thefootnote}{\arabic{footnote}}
\footnotetext[1]{Московский государственный университет им.\ М.\,В.~Ломоносова, 
 факультет вычислительной математики
 и~кибернетики, \mbox{markov.cmc@yandex.ru}}
\footnotetext[2]{Московский государственный университет им.\ М.\,В.~Ломоносова, 
факультет вычислительной математики
 и~кибернетики,  \mbox{mih\_monah@mail.ru}}
\footnotetext[3]{Московский государственный университет им.\ М.\,В.~Ломоносова, 
факультет вычислительной математики
 и~кибернетики, \mbox{vulyanov@cs.msu.ru}}

\Abst{Для квантилей выборочного среднего по выборке случайного объема  
построены  обобщенные разложения Кор\-ни\-ша--Фи\-ше\-ра на базе квантилей 
распределений Лапласа и~Стьюдента. В~последние годы интерес к~разложениям  
Кор\-ни\-ша--Фи\-ше\-ра значительно вырос в~связи с~исследованиями по управлению рисками. 
Широко распространенная мера риска Value at Risk (VaR) является квантилью 
функции потерь. Используется общая теорема переноса, позволяющая 
получать асимптотические разложения для функций распределения  статистик по 
выборкам случайного объема  из асимптотических разложений для функции распределения  
случайного объема выборки и~асимптотических разложений для функций распределения  
статистик по выборкам неслучайного объема. Проведен вычислительный эксперимент, 
иллюстрирующий полученные разложения Кор\-ни\-ша--Фи\-шера.}

\KW{обобщенные разложения Корниша--Фишера; выборка случайного объема; 
распределение Лапласа; распределение Стьюдента}

  
\DOI{10.14357/19922264160210}

%\vspace*{6pt}

\vskip 10pt plus 9pt minus 6pt

\thispagestyle{headings}

\begin{multicols}{2}

\label{st\stat}


\section{Введение}

В классических задачах математической статистики  объем выборки
традиционно считается детерминированным и~играет роль
\textit{известного} параметра, как правило, неограниченно
возрастающего. Однако на практике очень часто    возникают ситуации,
при которых размер выборки не определен заранее и~может считаться
случайным. Обычно это возникает ввиду того, что данные накапливаются
в~течение некоторого промежутка  времени, длительность которого по
разным причинам нельзя считать фиксированной. В~этом случае встает
вопрос об аппроксимирующих разложениях для различных характеристик
статистик, например функций распределения или квантилей, основанных
на выборках случайного объема. 

Предельные распределения для случайно
индексированных последовательностей и~их применения см., например, 
в~монографии \cite{Example3}. Ранее Б.\,В.~Гнеденко в~работе
\cite{Example1}  продемонстрировал, что в~математической статистике
при замене неслучайного объема выборки случайной величиной
асимп\-то\-ти\-ческие свойства статистик могут радикально изменяться. 

В~\cite{BenKorGal} доказана общая теорема переноса, позволяющая
получать асимптотические раз\-ло\-же\-ния для функций распределения
статистик, основанных на выборках случайного объема, из
асимптотических разложений для функции распределения случайного
объема выборки и~асимптотических разложений для функций
распределения  статистик, построенных по выборкам неслучайного
объема. 

Настоящая работа раз\-вивает результаты ра\-боты~\cite{BenKorGal}: 
на основе раз\-ложения функции распределения
статистики, основанной на выборке случайного объема специального
вида, получены обобщенные разложения Кор\-ни\-ша--Фи\-ше\-ра на базе
квантилей распределений Лапласа и~Стью\-дента. 
{ %\looseness=1

}

Классические разложения
Кор\-ни\-ша--Фи\-ше\-ра на базе квантилей нормального  распределения введены
в~\cite{CornFish}, их обобщение было предложено в~\cite{HillDavis}.
В~последние годы интерес к~разложениям Кор\-ни\-ша--Фи\-ше\-ра значительно
вырос в~связи с~исследованиями по управ\-ле\-нию рисками. Широко
распространенная мера риска VaR является, по
существу, квантилью функции потерь, связанной с~описанием
инвестиционного портфеля из  финансовых инструментов (см., например,~\cite{Jashke}).


Рассмотрим случайные величины (с.в.)\ $N_{1}, N_{2}, \ldots$ и~$X_1, X_2, \ldots$, 
заданные на одном и~том же вероятностном пространстве 
$\left(\Omega,\mathbb{A},\mathbb{P}\right)$. В~статистике 
с.в.~$X_1, X_2, \ldots, X_n$ имеют смысл наблюдений, $n$~--- 
неслучайный объем выборки, а~с.в.~$N_n$~--- случайный объем выборки, зависящий 
от натурального параметра $n \hm\in \mathbb{N}$. Предположим, что при каждом 
$n \hm\ge 1$ с.в.~$N_n$ принимает только натуральные значения (т.\,е.\
$N_n \hm\in \mathbb{N}$) и~независима от последовательности с.в. $X_1, X_2, \ldots$

Обозначим через $T_n \hm\equiv T_n \left(X_1, \ldots, X_n\right)$ некоторую статистику.
 Для каждого $n \hm\ge 1$ определим с.в.~$T_{N_n}$, полагая
$$
T_{N_n} (\omega) \equiv T_{N_n(\omega)} \left(X_1 (\omega), \ldots, X_{N_n (\omega)} 
(\omega)\right),\omega \in \Omega\,,
$$
т.\,е.\ $T_{N_n}$~--- это статистика, построенная на основе статистики~$T_n$ 
по выборке случайного объема~$N_n$.

\smallskip

\noindent
\textbf{Определение~1.}\
Квантилью порядка $\alpha$ ($\alpha$-кван\-тилью) случайной величины~$X$ 
с~функцией распределения $F\left(x\right)\hm=\mathbb{P}\left(X<x\right)$ 
называется число~$x_{\alpha}$, такое что
$$ 
x_{\alpha} = \inf \left\{ x; F(x)>\alpha \right\}\,.
 $$

\section{Основные результаты}

Всюду ниже $X_1, X_2, \ldots $~--- независимые одинаково распределенные 
с.в.\ с~$\mathbb{E}\left(X_1\right)\hm=\mu$, 
$ 0\hm<\mathbb{D}\left(X_1\right)\hm=\sigma^{-2}$, 
$\mathbb{E}\left|X_1\right|^{3+2\delta}\hm < \infty$ для $\delta\hm \in 
\left(0,1/2\right)$ и~$\mathbb{E}\left(X_1\hm-\mu\right)^{3}\hm=\mu_3$. 
Для натурального~$n$ обозначим $T_n\hm=\left(X_1+\cdots+X_n\right)/n$.

В дальнейшем предполагаем, что с.в.~$X_1$ удовле\-тво\-ря\-ет условию Крамера:
$$
\limsup\limits_{|t|\rightarrow \infty}\left|\mathbb{E} e^{itX_1}\right|<1\,.
$$

\subsection{Распределение Стьюдента как~предельное}

Предположим, что с.в.~$N_n$ имеет отрицательное биномиальное распределение 
с~параметрами $p\hm=1/n$ и~$r\hm>0$, т.\,е.\ для $k \hm\in \mathbb{N}$
$$
\mathbb{P}(N_n=k)=\fr{\left(k+r-2\right)\cdots r}{(k-1)!}\,
\fr{1}{n^{r}}\left(1-\fr{1}{n}\right)^{k-1}.
$$

Пусть $G_f (x)$~--- функция распределения   Стьюдента с~параметром~$f$, 
соответствующая плотности вида:
$$
g_f (x)=\fr{\Gamma((f+1)/2)}{\sqrt{\pi f} \Gamma\left(f/2\right)} 
\left(1+\fr{x^{2}}{f}\right)^{-(f+1)/2}\,,\enskip x \in \mathbb{R}\,,
$$
где $\Gamma(\cdot)$~--- эйлерова гам\-ма-функ\-ция, а~$f\hm>0$~--- параметр
формы. Если параметр~$f$ натурален, то он называется числом степеней свободы.

\smallskip

\noindent
\textbf{Теорема~1.}\ 
\textit{Пусть $x\hm=x_\alpha$~--- $\alpha$-кван\-тиль 
нормированной статистики $\sigma \sqrt{r(n-1)+1}\left(T_{N_n}\hm-\mu\right)$,
$u\hm=u_\alpha$~--- $\alpha$-кван\-тиль распределения Стьюдента с~параметром~$2r$.
Тогда в~обозначениях, введенных выше, справедливо следующее асимптотическое разложение}:
$$
x = u-\fr{\mu_3\, \sigma^{3}\, \Gamma(r)}
{6\sqrt{n}}\, \fr{1+ru^{2}}{r-1/2} \left[
\fr{1+u^{2}/2r}{1+u^{2}/2}\right]^{r+1/2}+R\,,
$$
где 
$$
R =  \begin{cases}     
\mathcal{O}\left(\fr{1}{n}\right)\,, & r\ge \fr{2}{1+2\delta}\,; \\
   o\left(\fr{1}{\sqrt{n}}\right) \,, & \fr{1}{1+2\delta}<r<\fr{2}{1+2\delta}\,.
            \end{cases}
            $$

\noindent
\textbf{Замечание~1.}\
При $r\hm=1$ объем выборки~$N_n$ имеет геометрическое распределение. 
Асимптотическое разложение для квантилей в~этом случае принимает вид:
$$
x  = u-\fr{\mu_3 \sigma^{3}}{3}\left( 
1+u^{2}\right)\fr{1}{\sqrt{n}}+o\left(\fr{1}{\sqrt{n}}\right)\,.
$$


\subsection{Распределение Лапласа как~предельное}

Предположим, что с.в.\ $N_n \hm= N_n(s)$, где $s\hm\in\mathbb{N}$~--- 
фиксированный параметр, имеет распределение вида:
$$
\mathbb{P}(N_n(s)=k)=\left(\fr{k}{s+k}\right)^{n} - 
\left(\fr{k-1}{s+k-1}\right)^{n},\enskip k \in \mathbb{N}\,.
$$

Пусть  $\Lambda_{\theta}(x)$~--- функция распределения Лапласа 
с~параметром $\theta\hm>0$, соответствующая плотности вида:
$$ 
\lambda_{\theta}(x)=\fr{1}{\theta\sqrt{2}}e^{-{\sqrt{2}|x|}/{\theta}},\enskip 
x\in \mathbb{R}\,. 
$$


\noindent
\textbf{Теорема~2.}\ 
\textit{Пусть $x\hm=x_\alpha$ есть $\alpha$-кван\-тиль 
нормированной статистики $\sigma\sqrt{n}\left(T_{N_{n}(s)}\hm-\mu  \right)$, 
$u\hm=u_\alpha$~---\linebreak $\alpha$-кван\-тиль распределения Лапласа с~па\-ра\-мет\-ром~$1/s$.
Тогда в~обозначениях, введенных выше, справедливо следующее асимптотическое 
разложение}:
\begin{multline*}
%\label{eq:}
 x=u-\fr{\mu_{3}\sigma^{3}}{6}\left(\fr{|u|}{\sqrt{2s}} +\fr{1}{2s}-u^{2}\right)
 \fr{\lambda_{1/\sqrt{s}}(u)}{\lambda_{1/s}(u)} \fr{1}{\sqrt{n}} +{}\\
 {}+
 o\left(\fr{1}{\sqrt{n}}\right)\,.
 \end{multline*}

\noindent
\textbf{Замечание~2.}\
При $s\hm=1$ основной результат значительно упрощается:
$$x = u-\fr{\mu_{3}\sigma^{3}}{6\sqrt{n}}\left(
\fr{|u|}{\sqrt{2}} +\frac{1}{2}-u^{2}\right)  + o\left(\fr{1}{\sqrt{n}}\right).
$$

\noindent
\textbf{Замечание~3.}\
В~теоремах~1 и~2 построены приближения квантилей~$x$~нормированных 
статистик некоторыми функциями от квантилей~$u$ распределений 
Стьюдента и~Лапласа соответственно. При этом точность приближения  дана 
в~виде порядка по~$n$, в~частности как $o(1/\sqrt{n}).$ 
В~\cite{Arx2016} в~общей ситуации показано, что для ошибок приближения квантилей 
можно получать более информативные вы\-чис\-ли\-мые оценки при наличии таковых для 
ошибок приближения распределений  нормированных статистик.

\section{Вывод основных результатов}

\subsection{Вспомогательная лемма}

Пусть $F_{n}(x)$ есть последовательность функций распределения, каждая 
из которых допускает разложения типа Эдж\-вор\-да--Че\-бы\-шё\-ва 
по степеням $\varepsilon\hm=n^{-1/2}$ или $\varepsilon\hm=n^{-1}$ 
(см., например,~\cite{EncStat}):
\begin{align*}
F_n(x)&=G_{k,n}(x)+O(\varepsilon^k)\,; \\
G_{k,n}(x)&={}\\
&\hspace*{-5mm}{}= G(x)+\left\{\varepsilon a_{1}(x)+\cdots+\varepsilon^{k-1}a_{k-1}(x)\right\}
g(x)\,, 
\end{align*}
где $g(x)$~--- плотность распределения $G(x)$.

 В частности, для $k\hm=2$ имеем: 
\begin{equation}
\label{f1}
F_n(x)=G(x)+\varepsilon a_{1}(x)g(x) +\mathcal{O}\left(\varepsilon^{2}\right)\,.
\end{equation}

\noindent
\textbf{Лемма~1} (см., например, \cite{EncStat}).
\textit{В~сформулированных выше обозначениях имеет место следующее разложение}:  
\begin{equation}
\label{e3-u}
 x(u)=u+\varepsilon b_{1}(u) + \mathcal{O}(\varepsilon^{2}),
\end{equation}
\textit{где $x(u)$ и~$u$ суть квантили соответственно распределений $F_{n}$
и~$G$ одинакового порядка, т.\,е.\ $ F_n(x(u))\hm=G(u)$ и}
$b_{1}(u)=-a_{1}(u)$.\\

\subsection{Доказательство теоремы для~распределения Стьюдента как~предельного}

Для вывода необходимого асимптотического разложения квантилей 
распределения статистики, построенной по выборке случайного объема, 
воспользуемся результатом, полученным в~работе~\cite{BenKorGal}.

Введем следующее обозначение:
$$
f_r(x) \equiv \int\limits_{0}^{\infty} \phi\left(x\sqrt{y}\right) \fr{1-x^{2}y}{\sqrt{y}} \,dH_r(y)\,,
$$
где $H_r(x)$~--- функция гам\-ма-рас\-пре\-де\-ле\-ния с~параметром $r\hm>0$:
$$
H_r(x)=\fr{r^{r}}{\Gamma(r)} \int\limits_{0}^{x} e^{-ry} y^{r-1}\, dy, x \ge 0\,.
$$

В работе~\cite{BenKorGal} показано, что в~условиях теоремы для функции 
распределения нормированной статистики~$T_{N_n}$ справедлив следующий результат:
\begin{multline*}
\sup\limits_{x} \left| 
\vphantom{\fr{\mu_3 \sigma^{3}}{6\sqrt{r(n-1)+1}}}
\mathbb{P}(\sigma\sqrt{r(n-1)+1}(T_{N_n}-\mu)<x) -{} \right. \\
 \left. {}-G_{2r}(x)-\fr{\mu_3 \sigma^{3}}{6\sqrt{r(n-1)+1}}f_r(x)\right| =R\,,
\end{multline*}
где 
$$R = \begin{cases}
\mathcal{O}\left(\left(\fr{\log{n}}{n}\right)^{{1}/{2}+\delta}\right) , & r=1; \\[12pt]
\mathcal{O}\left(
\fr{1}{n^{\min(1, r({1}/{2}+\delta))}}\right), & r>1;\\[12pt]
 \mathcal{O}\left(\fr{1}{n^{r(1+2\delta)}}\right), & \fr{1}{1+2\delta}<r<1.
        \end{cases}
        $$
Отсюда при любом $x \hm\in \mathbb{R}$ имеем асимптотическое разложение вида:
 \begin{multline}
P\left(\sigma\sqrt{r(n-1)+1}(T_{N_n}-\mu)<x\right) ={}\\
{} = G_{2r}(x) -\fr{\mu_3 \sigma^{3}}{6\sqrt{r(n-1)+1}}f_r(x)+R_1\,,
\label{e4-u}
\end{multline}
где
$$ 
R_1 = \begin{cases}      \mathcal{O}\left(\fr{1}{n}\right)          , & r\ge \fr{2}{1+2\delta} ; \\
                            o\left(\fr{1}{\sqrt{n}}\right)             , & \fr{1}{1+2\delta}<r<\frac{2}{1+2\delta}.
            \end{cases}$$
Введем дополнительно следующие обозначения:
\begin{equation}
\left.
\begin{array}{c}
 F_n(x)                     \equiv \mathbb{P}(\sigma\sqrt{r(n-1)+1}(T_{N_n}-\mu)<x);    \\[6pt]
 G(x)                       \equiv G_{2r}(x);                                           \\[6pt]
 \varepsilon                \equiv n^{-{1}/{2}};                                    \\[6pt]
 \varepsilon a_1(x) g(x)    \equiv \fr{\mu_3 \sigma^{3}}{6\sqrt{r(n-1)+1}}\,f_r(x),
\end{array}
\right\}
\label{e5-u}
\end{equation}
где $g(x)$ есть плотность распределения~$G$.
Заметим теперь, что в~новых обозначениях разложение~(\ref{e4-u}) 
является схожим с~разложением типа Эджворта--Че\-бы\-шё\-ва из леммы~1. Поэтому, 
подставляя~(\ref{e3-u}) в~разложение~(\ref{e4-u}) 
согласно обозначениям~(\ref{e5-u}),  находим выражение для~$b_1$ 
и~приходим  к~искомому разложению типа Корниша--Фишера. 

Для упрощения вывода введем дополнительное обозначение, 
полагая $\theta \hm\equiv \mu_3 \sigma^{3}/6$. Имеем (ср.~(\ref{f1})):
\begin{multline*}
F_n(x)
   = G_{2r}(u)+g_{2r}(u) \fr{b_1}{\sqrt{n}}+
   \fr{\theta}{\sqrt{r(n-1)+1}} +{}\\
   {}+ f_r(u) + \fr{\theta}{\sqrt{r(n-1)+1}} \fr{1}{\sqrt{n}}\, f_{r}'(u) b_1(u) + 
\mathcal{O}\left(\fr{1}{n}\right) .
\end{multline*}
Заметим, что последнее слагаемое этого равенства является функцией 
порядка  $\mathcal{O}\left(1/n\right)$, а следовательно, 
и~$o\left(1/\sqrt{n}\right)$.

Таким образом, учитывая условие $F_n(x)\hm=G_{2r}(u)$, т.\,е.~$x$ и~$u$ 
суть квантили одного порядка, получаем следующее выражение для коэффициента~$b_1(u)$:
\begin{multline*}
 b_1(u)  =-\fr{\theta \sqrt{n}}{\sqrt{r(n-1)+1}}\,\fr{f_r(u)}{g_{2r}(u)}
     ={}\\
     {}= -\fr{\theta}{\sqrt{r}}\,\fr{f_r(u)}{g_{2r}(u)}+
     \mathcal{O}\left(\fr{1}{n}\right).
\end{multline*}
Для упрощения получившегося выражения воспользуемся следующей леммой.

\smallskip

\noindent
\textbf{Лемма~2.}\
\textit{Для функции $f_r(x)$, введенной выше, имеем}:
$$ 
f_r(x)=\fr{1}{\sqrt{2\pi}}\Gamma\!\left(r-\fr{1}{2}\right) \left[ 
1+rx^{2}\right] \left(1+\fr{x^{2}}{2} \right)^{-\left( r+1/2 \right)}\! .
$$



Для д\,о\,к\,а\,з\,а\,т\,е\,л\,ь\,с\,т\,в\,а\ леммы подставим функцию 
гам\-ма-рас\-пре\-\-\-деления $H_r (y)$ и~плотность стандартного 
нормального распределения  $\phi\left(x\sqrt{y}\right)$ 
в~выражение для функции $f_r(x)$:
\begin{multline*}
f_r(x)=\int\limits_{0}^{\infty} \phi\left(x\sqrt{y}\right) 
\fr{1-x^{2}y}{\sqrt{y}}\, dH_r(y) ={}\\
     {}=\fr{1}{\sqrt{2\pi}}\int\limits_{0}^{\infty} e^{-({x^{2}y})/{2}} 
     \fr{1-x^{2}y}{\sqrt{y}}\, y^{r-1} e^{-y}\, dy = {}\\
     {}=\fr{1}{\sqrt{2\pi}}\left(\int\limits_{0}^{\infty} e^{-y\left(1+
     {x^{2}}/{2}\right)} y^{\left(r-1/2\right)-1}dy \right. +{}\\
     \left. {}+ x^{2}\int\limits_{0}^{\infty} e^{-y\left(1+{x^{2}}/{2}\right)} 
     y^{\left(r+1/2\right)-1}\, dy \right)\,.
\end{multline*}
Рассмотрим вспомогательный интеграл:
\begin{equation*}
h(\alpha, p) \equiv\int\limits_{0}^{\infty}t^{\alpha-1}e^{-pt}\,dt=
%\fr{1}{p^{\alpha}}\int\limits_{0}^{\infty}t^{\alpha-1}e^{-pt}dpt
      \Gamma(\alpha)p^{-\alpha}\,.
\end{equation*}
Тогда $f_r(x)$ запишется в~виде:
\begin{multline*}
f_r(x)  =\fr{1}{\sqrt{2\pi}}  \left(h\left(r-\fr{1}{2}, 1+
\fr{x^{2}}{2}\right)+{}\right.\\
\left.{} +x^{2}h\left(r+\fr{1}{2}, 1+\fr{x^{2}}{2}\right)\right) =     {}\\
{}=\fr{1}{\sqrt{2\pi}} \left(1+\fr{x^{2}}{2}\right)^{-\left(r+{1}/{2}\right)}
  \!\left(\Gamma\left(r-\fr{1}{2}\right)\left(1+\fr{x^{2}}{2}\right)+{}\right.\\
  \left.{}+x^{2}\Gamma\left(
  r+\fr{1}{2}\right)\right) = {}\\
{}=\fr{1}{\sqrt{2\pi}}\left(1+\fr{x^{2}}{2}\right)^{-\left(r+{1}/{2}\right)}\Gamma
\left(r-\fr{1}{2}\right)\left(1+rx^{2}\right)\,.
\end{multline*}
Воспользуемся результатом леммы~2 и~получим
 утверждение теоремы.


Рассмотрим случай $r=1$. Тогда объем выборки~$N_n$ имеет
геометрическое распределение. Имеем:  $\delta \hm\in \left(0,1/2
\right)$; значит,   $2/(1\hm+2\delta) \hm\in \left(1,2\right)$.
Следовательно, остаточный член в~случае $r\hm=1$ имеет порядок
$o\left(1/\sqrt{n}\right)$. Найдем асимптотическое разложение для
квантилей в~этом случае, подставив значение параметра в~итоговое
выражение:
\begin{multline*}
x_{r=1}(u)={}\\
{}=u-\fr{\mu_3\, \sigma^{3}\, \Gamma(1)}{6\sqrt{n}}\,
\fr{1+1 u^{2}}{1-1/2}   \left[{\fr{1+u^{2}/2}{1+u^{2}/2}}\right]^{3/2}+{}\\
{}+o\left(\fr{1}{\sqrt{n}}\right) =
u-\fr{\mu_3 \sigma^{3}}{3}\left( 1+u^{2}\right)\fr{1}{\sqrt{n}}+
o\left(\fr{1}{\sqrt{n}}\right)\,.
\end{multline*}


\subsection{Доказательство теоремы для~распределения Лапласа как~предельного}

В работе~\cite{BenKorGal} показано, что в~условиях данной теоремы 
для функции распределения нормированной статистики $T_{N_{n}(s)}$ справедливо 
асимптотическое разложение вида:
\begin{multline}
\label{e6-u}
\max\limits_{x} \left|
\vphantom{\fr{\mu_{3}\sigma^{3}l_{s}(x)}{6\sqrt{n}}}
\mathbb{P}\left( \sigma\sqrt{n}(T_{N_{n}(s)}-\mu)<x\right)
-\Lambda_{1/s}(x) - {} \right.\\
\left.- \fr{\mu_{3}\sigma^{3}l_{s}(x)}{6\sqrt{n}}\right|= 
O\left(\fr{1}{n^{1/2+\delta}}\right)\,,\enskip n \rightarrow\infty\,,
\end{multline}
где функции $\Lambda_{1/s}(x)$ и~$l_{s}(x)$ определены следующим образом: 

\noindent 
\begin{align*}
 \Lambda_{1/s}(x) &= \int\limits^{\infty}\limits_{0}{\Phi\left(x\sqrt{y}\right)\,de^{-s/y}}\,, \\
 l_{s}(x)&=\int\limits^{\infty}\limits_{0}{\phi(x\sqrt{y})\fr{1-x^{2}y}{\sqrt{y}}\,de^{-s/y}}\,, 
 \end{align*}
а $\Phi(x)$ и~$\phi(x)$~--- функция распределения и~плотность стандартного нормального 
распределения соответственно.

Из формулы~(\ref{e6-u}) вытекает следующее соотношение:  
\begin{multline}
%\label{eq:}
\label{e7-u}
\mathbb{P}\left( \sigma\sqrt{n}(T_{N_{n}(s)}-\mu)<x\right)= 
\Lambda_{1/s}(x) + \fr{\mu_{3}\sigma^{3}l_{s}(x)}{6\sqrt{n}} + {}\\
{}+O\left(\fr{1}{n^{1/2+\delta}}\right),\enskip n \rightarrow\infty\,.
\end{multline}
Формула~(\ref{e7-u}) похожа на разложение типа Эджворда (формула~(\ref{f1})), 
однако  не соответствует ему в~точ\-ности.

Введем следующие обозначения: 
\begin{equation}
\left.
\begin{array}{rl}
 F_{n}(x)&\equiv\mathbb{P}\left( \sigma\sqrt{n}(T_{N_{n}(s)}-\mu)<x\right)\,;\\[6pt]
 G(x) &\equiv \Lambda_{1/s}(x)\,; \\[6pt]
 \gamma &\equiv \displaystyle\fr{\mu_{3}\sigma^{3}l_{s}(x)}{6}\,. 
\end{array}
\right\}
%\label{oboznacheniya}
\label{e8-u}
\end{equation}

Используем представление из~\cite{Prudnikov}  
\begin{multline}
\label{prudnikov}
 \int\limits^{\infty}_{0}x^{-n-{1}/{2}}e^{-px-{q}/{x}}\,dx = 
 \left(-1\right)^n \sqrt{\fr{\pi}{p}}\,\fr{\partial^n}{\partial q^n}
 \,e^{-2\sqrt{pq}}\,,\\
p>0\,,\enskip q>0\,,
\end{multline}
и рассмотрим $l_{s}(x)$: \\
\begin{multline*}
 l_{s}(x)= \int\limits^{\infty}_{0}\phi(x\sqrt{y})
 \fr{1-x^{2}y}{\sqrt{y}}\,de^{-s/y} ={}\\
{}= \fr{s}{\sqrt{2\pi}}\left( \int\limits^{\infty}_{0}y^{-{5}/{2}}
e^{-({x^{2}y})/{2}-{s}/{y}}\,dy - {}\right.\\
\left.{}-
x^2\int\limits^{\infty}_{0}y^{-{3}/{2}}e^{-({x^{2}y})/{2}-{s}/{y}}\,dy   \right)
=l_{s}^{*}(x)\,.
\end{multline*}

Возникают два случая.
\begin{enumerate}[1.]
\item Случай, когда $ x \hm\neq 0:$ 
\begin{multline*}
\hspace*{-1.14159pt} l_{s}^{*}(x)=\fr{2\pi s}{\sqrt{2\pi x^2}}  
 \left( \fr{x^{2}e^{-\sqrt{2sx^2}}}{2s}+\fr{x^{4}e^{-\sqrt{2sx^2}}}{2\sqrt{2}
 (sx^2)^{{3}/{2}}}-{}\right.\\
\hspace*{-16pt}\left. {}-x^{2}\fr{x^{2}e^{-\sqrt{2sx^2}}}{\sqrt{2sx^2}}\right) =
\lambda_{1/\sqrt{s}}(x)\left( \fr{\left|x\right|}{\sqrt{2s}}+\fr{1}{2s}-x^2\right).\hspace*{-1.23863pt} 
\end{multline*}
\item Случай, когда $x \hm= 0:$
\begin{multline*}
l_{s}^{*}(x)= \fr{s}{\sqrt{2\pi}} \int\limits^{\infty}_{0}
y^{-{5}/{2}}e^{-{s}/{y}}\,dy = {}\\
{}=\fr{s}{\sqrt{2\pi}}\,
\fr{\sqrt{\pi}}{2s^{{3}/{2}}} = \fr{1}{2\sqrt{2s}}\,.
\end{multline*}
\end{enumerate}
Поскольку тождество~(\ref{prudnikov}) применимо, только когда $x\hm>0$, проверим 
на непрерывность полученный результат:

\noindent
\begin{multline*}
 \lim\limits_{x\rightarrow 0-0}\lambda_{1/\sqrt{s}}(x)\left( \fr{\left|x\right|}
 {\sqrt{2s}}+\fr{1}{2s}-x^2\right) ={}\\
 {}= \fr{1}{2\sqrt{2s}}
  = l_s(0-0)\,;
  \end{multline*}
  
  \vspace*{-12pt}
  
  \noindent
  \begin{multline*}
\lim\limits_{x\rightarrow 0+0}\lambda_{1/\sqrt{s}}(x)\left( 
\fr{\left|x\right|}{\sqrt{2s}}+\fr{1}{2s}-x^2\right) = {}\\
{}=\fr{1}{2\sqrt{2s}}
  = l_s(0+0)\,. 
  \end{multline*}
Таким образом, в~силу~(\ref{e8-u}) исходное приближение принимает следующий вид: 

\begin{table*}[b]\small
\begin{center}

\begin{tabular}{|c|c|c|c|c|}
\multicolumn{5}{p{102mm}}{Значения метрик для приближения Лапласа 
и~Стьюдента на разных интервалах}  \\
\multicolumn{5}{c}{\ }\\[-6pt]
\hline
\multicolumn{1}{|c|}{\raisebox{-6pt}[0pt][0pt]{$n$}} & \multicolumn{2}{c|}{Приближение Лапласа} &  \multicolumn{2}{c|}{Приближение Стьюдента} \\
\cline{2-5}
&$M_{\mathrm{dif}}$, 10\,,000 & $L_1$, 10\,000 & $M_{\mathrm{dif}}$, 1000 & $L_1$, 1000\\
\hline
\hphantom{,99}0\%--99,99\% &  1,93 & 0,15& 3,00& 0,06\\
0\%--1\%\hphantom{9} &  1,51     & 0,65& 2,42& 0,83\\
0\%--5\%\hphantom{9} &  1,51     & 0,28& 2,42& 0,29\\
1\%--99\% & 0,26& 0,14& 1,16& 0,05\\
 5\%--95\%& 0,17& 0,14& 0,21& 0,04\\
\hline
\end{tabular}
\end{center}
%\begin{figure*} %fig1
\vspace*{6pt}
 \begin{center}
 \mbox{%
 \epsfxsize=162.484mm
 \epsfbox{ula-1.eps}
 }
\vspace*{3pt}

\noindent
{\small Графики квантилей для приближений Лапласа~(\textit{а}) и~Стьюдента~(\textit{б})}
 \end{center}
%\end{figure*}
\end{table*}

\noindent
\begin{multline}
%\label{perekhodOo}
\label{e10-u}
F_{n}(x)=G(x)+\fr{\gamma}{\sqrt{n}}\left( 
\fr{\left|x\right|}{\sqrt{2s}}+\fr{1}{2s}-x^2\right)
\lambda_{1/\sqrt{s}}(x)+{}\\
{}+O\left(\fr{1}{n^{1/2+\delta}}\right)\,,\enskip
 n\rightarrow\infty\,.
\end{multline}
 Поскольку
 
 \noindent
$$
O\left(\fr{1}{n^{1/2+\delta}}\right) = o\left(\fr{1}{\sqrt{n}}\right)\,,\enskip
  n\rightarrow\infty\,,
  $$
формула~(\ref{e10-u}) перепишется в~виде: 

\noindent
\begin{multline*}
F_{n}(x)=G(x)+\fr{\gamma}{\sqrt{n}}\left( 
\fr{\left|x\right|}{\sqrt{2s}}+\fr{1}{2s}-x^2\right)
\lambda_{1/\sqrt{s}}(x)+{}\\
{}+o\left(\fr{1}{\sqrt{n}}\right),\enskip
n\rightarrow\infty\,. 
\end{multline*}
Так как $s \hm\in \mathbb{N}$, то $\lambda_{1/\sqrt{s}}(x)\hm>0$ 
для любого $x\hm\in\mathbb{R}$. Тогда представим $\lambda_{1/\sqrt{s}}(x)\hm>0$ 
в~виде:  
$$ 
\lambda_{1/\sqrt{s}}(x) = 
\fr{\lambda_{1/\sqrt{s}}(x)}{\lambda_{1/s}(x)}\lambda_{1/s}(x)  
$$
и подставим в~предыдущую формулу: 

\noindent
\begin{multline*}
\!\!F_{n}(x)=G(x)+\fr{\gamma}{\sqrt{n}}\left( 
\fr{\left|x\right|}{\sqrt{2s}}+\fr{1}{2s}-x^2\right)
\fr{\lambda_{1/\sqrt{s}}(x)}{\lambda_{1/s}(x)} \times {}\\
{} \times\lambda_{1/s}(x)+o\left(\fr{1}{\sqrt{n}}\right),\enskip n\rightarrow\infty\,. 
\end{multline*}

\vspace*{-12pt}

\pagebreak

\noindent
Очевидно, что $O(\varepsilon^2)\hm=o(\varepsilon)$ при $\varepsilon\hm\rightarrow 0$, 
поэтому полученная формула в~точности совпадает 
с~формулой~(\ref{f1}), в~которой 

\noindent
\begin{align*}
a(x) &\equiv \gamma\left( \fr{\left|x\right|}{\sqrt{2s}}+
\fr{1}{2s}-x^2\right)\fr{\lambda_{1/\sqrt{s}}(x)}{\lambda_{1/s}(x)}\,;\\
 g(x) &\equiv \lambda_{1/s}(x)\,; \\
\varepsilon &\equiv \fr{1}{\sqrt{n}}\,. 
\end{align*}
Следовательно, формула~(\ref{e3-u}) перепишется в~виде:  

\noindent
\begin{multline*}
x(u)={}\\
{}=u-\fr{\mu_{3}\sigma^{3}}{6}\left(\fr{|u|}{\sqrt{2s}} +
\fr{1}{2s}-u^{2}\right)  
\fr{\lambda_{1/\sqrt{s}}(u)}{\lambda_{1/s}(u)}\, \fr{1}{\sqrt{n}} + {}\\
{}+
o\left(\fr{1}{\sqrt{n}}\right)\,.
\end{multline*}

\vspace*{-6pt}

\noindent
Теорема доказана.

\vspace*{-6pt}

\section{Вычислительный эксперимент}

%\vspace*{-6pt}

План эксперимента.
\begin{enumerate}[1.]
    \item   Задаются следующие параметры (сначала для случая, 
    когда распределение Стьюдента является предельным, затем в~скобках 
    для случая предельного распределения Лапласа): 

        $r =1$ $(s=1)$~--- характеризует распределение объема выборки; 

        $\chi^{2}_{4}$~--- задает распределение случайных величин $X_1,X_2,\ldots$; 

        $n = 1000 \:(n = 10\,000)$~--- параметр распределения объема выборки;

        $k = 10\,000$~--- число точек, в~которых рассчитываются эмпирическая 
        и~аппроксимирующая функции квантилей.
    \item   Производится расчет в~следующей последовательности: для каждой точки 
    эмпирической функции моделируется случайный объем~$N_n$, далее моделируется 
    вектор случайных величин $X_1,\ldots,X_{N_n}$, затем рассчитывается значение 
    статистики~$T_{N_n}$, после чего она нормируется;\linebreak отдельно для каж\-дой точ\-ки 
    ап\-прок\-си\-ми\-ру\-ющей функции моделируется значение соответствующей квантили 
    распределения Стью\-ден\-та (Лапласа), значение самой функции рассчитыва\-ется 
    согласно полученным выражениям в~теоремах~1 и~2.
    \item   Результаты даются в~виде таблицы и~графиков на рисунке. 
    В~таблице используются следу\-ющие расстояния  между эмпирической и~аппроксимирующей 
    функциями:
    
    \noindent
            $$
        M_{\mathrm{dif}}: \rho\left(f,g\right)=\max\limits_{1\leq i\leq k}
        \left|f(x_i)-g(x_i)\right|\,; 
        $$
        $$
        L_{1}:  \rho\left(f,g\right)=
        \fr{1}{k}\sum\limits_{i=1}^k\left|f(x_i)-g(x_i)\right|\,.
        $$
\end{enumerate}


Отметим, что для расчета даже одного эмпирического значения статистики  
необходимо для каж\-дой из $k\hm=10\,000$ точек моделировать~$N_n$~случайных 
величин $X_1,\ldots, X_{N_n}$, затем выполнить расчет статистики, что занимает 
несколько минут уже для $n\hm=1000$.

Таблица  иллюстрирует известный факт, что разложения Кор\-ни\-ша--Фи\-ше\-ра
дают хорошее приближение в~центральной зоне, т.\,е.\ для вероятностей
от~0,05 до~0,95. Качество приближения ухудшается для значений
вероятности около~0 и~1. О~том же говорит и~рисунок.

На рисунке приведены графики квантилей для приближения Лапласа 
с~параметрами $n\hm=10\,000$, $s\hm=1$~(\textit{а}) и~Стьюдента с~параметрами
$n\hm=1000$, $r\hm=1$~(\textit{б}). По оси~$OX$ откладываются значения
эмпирических квантилей, по оси $OY$~--- значения, даваемые
разложениями Кор\-ни\-ша--Фи\-шера.

\vspace*{-6pt}

{\small\frenchspacing
 {%\baselineskip=10.8pt
 \addcontentsline{toc}{section}{References}
 \begin{thebibliography}{9}
\bibitem{Example3}
\Au{Королев В.\,Ю.} Предельные распределения для случайно
индексированных последовательностей и~их применения.~--- М.: МГУ,
1993. 269~с.

\columnbreak

\bibitem{Example1}
\Au{Гнеденко Б.\,В.} Об оценке неизвестных параметров распределения
при случайном числе независимых наблюдений~// Тр. Тбилисского
мат. ин-та, 1989. Т.~92. С.~146--150.

\bibitem{BenKorGal}
\Au{Бенинг В.\,Е., Королев В.\,Ю., Галиева~Н.\,К.}   Асимптотические
разложения для функций распределения статистик, построенных по
выборкам случайного объема~// Информатика и~её применения, 2013.
Т.~7. Вып.~2. С.~75--83.

\bibitem{CornFish}
\Au{Cornish E.\,A., Fisher R.\,A.}  Moments and cumulants in the
specification of distributions~// Rev. Inst. Int. Statist.,
1937. Vol.~4. P.~307--320.

\bibitem{HillDavis}
\Au{Hill G.\,W., Davis A.\,W.} Generalized asymptotic expansions of
Cornish--Fisher type~// Ann. Math. Stat., 1968.
Vol.~39. P.~1264--1273.
\bibitem{Jashke}
\Au{Jaschke S.} The Cornish--Fisher expansion in the context of
delta-gamma-normal approximations~// J.~Risk, 2002. Vol.~4.
No.\,4. P.~33--52.

\bibitem{Arx2016}
\Au{Ulyanov V.\,V., Aoshima M., Fujikoshi~Y.}  Non-asymptotic
results for Cornish--Fisher expansions. Technical Report Hiroshima
Statistical Research Group. No.~16-03.~--- Hiroshima: Hiroshima
University, 2016. 8~p. 

\bibitem{EncStat}
\Au{Ulyanov V.\,V.}  Cornish--Fisher expansions~// International
encyclopedia of statistical science~/ Ed. M.~Lovric.~--- Berlin:
Springer, 2011. P.~312--315.

\bibitem{Prudnikov}
\Au{Прудников А.\,П., Брычков Ю.\,А., Маричев~О.\,И.}  Интегралы 
и~ряды. Элементарные функции.~--- М.: Наука, 1981. 344~с.
\end{thebibliography}

 }
 }

\end{multicols}

\vspace*{-6pt}

\hfill{\small\textit{Поступила в~редакцию 02.12.15}}

\vspace*{4pt}

%\newpage

%\vspace*{-24pt}

\hrule

\vspace*{2pt}

\hrule

\vspace*{-2pt}



\def\tit{GENERALIZED CORNISH--FISHER EXPANSIONS FOR~DISTRIBUTIONS 
OF~STATISTICS BASED~ON~SAMPLES~OF~RANDOM~SIZE}

\def\titkol{Generalized Cornish--Fisher expansions for~distributions 
of~statistics based on~samples of~random size}

\def\aut{A.\,S.~Markov, M.\,M.~Monakhov, and V.\,V.~Ulyanov}

\def\autkol{A.\,S.~Markov, M.\,M.~Monakhov, and V.\,V.~Ulyanov}

\titel{\tit}{\aut}{\autkol}{\titkol}

\vspace*{-9pt}

\noindent
Faculty of Computational Mathematics and Cybernetics,  
M.\,V.~Lomonosov Moscow State University, 1-52~Leninskiye Gory, GSP-1, Moscow 119991, 
Russian Federation

\def\leftfootline{\small{\textbf{\thepage}
\hfill INFORMATIKA I EE PRIMENENIYA~--- INFORMATICS AND
APPLICATIONS\ \ \ 2016\ \ \ volume~10\ \ \ issue\ 2}
}%
 \def\rightfootline{\small{INFORMATIKA I EE PRIMENENIYA~---
INFORMATICS AND APPLICATIONS\ \ \ 2016\ \ \ volume~10\ \ \ issue\ 2
\hfill \textbf{\thepage}}}

\vspace*{3pt}



\Abste{Generalized Cornish--Fisher expansions are constructed for 
quantiles of sample mean for a sample of random size in terms of 
quantiles for the Laplace distribution and Student's $t$-test. 
In recent years, the interest in Cornish--Fisher expansions grew significantly in 
the context of research on risk management. The widespread risk measure 
Value at Risk, or VaR, is, in fact, the quantile of the loss function. 
The authors use the general transfer theorem that makes it possible to obtain 
asymptotic expansions for the distribution functions of statistics based 
on samples of random size by asymptotic expansions for the distribution 
function of the random sample size and asymptotic expansions for the 
distribution functions of statistics based on nonrandom samples. 
A~computational experiment was performed to illustrate the obtained Cornish--Fisher 
expansions.}

\KWE{quantiles; generalized Cornish-Fisher expansions; random size sample; 
Laplace distribution}

\DOI{10.14357/19922264160210}

\vspace*{-12pt}

\Ack
\noindent
The work was supported by the Russian Science Foundation
 (project 14-11-00364).


%\vspace*{3pt}

  \begin{multicols}{2}

\renewcommand{\bibname}{\protect\rmfamily References}
%\renewcommand{\bibname}{\large\protect\rm References}

{\small\frenchspacing
 {%\baselineskip=10.8pt
 \addcontentsline{toc}{section}{References}
 \begin{thebibliography}{9}

\bibitem{1-ul}
\Aue{Korolev, V.\,Yu.} 1993. \textit{Predel'nye raspredeleniya 
dlya sluchayno indeksirovannykh posledovatel'nostey i~ikh primeneniya}
[Limit theorems for randomly indexed sequences 
and its applications]. Moscow: MSU. 269~p.

\bibitem{2-ul}
\Aue{Gnedenko, B.\,V.} 1989. Ob otsenke neizvestnykh parametrov 
raspredeleniya pri sluchaynom chisle nezavisimykh nablyudeniy 
[On estimation of unknown parameters of distributions from a~random number 
of independent observations]. \textit{Tr. Tbilisskogo mat. in-ta}
[Proceedings of Tbilisi Mathematical Institute] 92:146--150.

\bibitem{3-ul}
\Aue{Bening, V.\,E., V.\,Yu.~Korolev, and N.\,K.~Galieva}. 
2013. Asimptoticheskie razlozheniya dlya funktsiy raspredeleniya statistik, 
postroennykh po vyborkam sluchaynogo ob"ema 
[Asymptotic expansions for the distribution functions of statistics 
constructed from samples with random sizes]. 
\textit{Informatika i~ee Primeneniya}~--- \textit{Inform. Appl.} 7(2):75--83. 

\bibitem{4-ul}
\Aue{Cornish, E.\,A., and R.\,A.~Fisher}. 1937. Moments and cumulants 
in the specification of distributions. \textit{Rev. Inst. Int. Stat.} 4:307--320.

\bibitem{5-ul}
\Aue{Hill, G.\,W., and A.\,W.~Davis}. 1968. Generalized asymptotic expansions 
of Cornish--Fisher type. \textit{Ann. Math. Stat.} 39:1264--1273.

\bibitem{6-ul}
\Aue{Jaschke, S.} 2002. The Cornish--Fisher expansion in the context of 
delta-gamma-normal approximations. \textit{J.~Risk} 4(4):33--52.

\bibitem{7-ul}
\Aue{Ulyanov, V.\,V., M.~Aoshima, and Y.~Fujikoshi}. 2016. 
Non-asymptotic results for Cornish--Fisher expansions. 
Technical Report Hiroshima Statistical Research Group. No.\,16-03. 
Hiroshima: Hiroshima University. 8~p. 

\bibitem{8-ul}
\Aue{Ulyanov, V.\,V.} 2011. Cornish--Fisher expansions.
\textit{International encyclopedia of statistical science}.
Ed. M.~Lovric. Berlin: Springer. 312--315.

\bibitem{9-ul}
\Aue{Prudnikov, A.\,P., Yu.\,A.~Brychkov, and O.\,I.~Marichev}. 1981. 
\textit{Integraly i~ryady. Elementarnye funktsii} [Integrals and series. Elementary functions]. 
Moscow: Nauka. 344~p.

\end{thebibliography}

 }
 }

\end{multicols}

\vspace*{-3pt}

\hfill{\small\textit{Received December 2, 2015}}



\Contr

\noindent
\textbf{Markov Alexander S.} (b.\ 1993)~--- 
PhD student, Faculty of Computational Mathematics and Cybernetics,  
M.\,V.~Lomonosov Moscow State University, 1-52~Leninskiye Gory, GSP-1, Moscow 119991, 
Russian Federation; \mbox{markov.cmc@yandex.ru}

\vspace*{3pt}

\noindent
\textbf{Monakhov Mikhail M.} (b.\ 1993)~---
PhD student, Faculty of Computational Mathematics and Cybernetics,  
M.\,V.~Lomonosov Moscow State University, 1-52~Leninskiye Gory, GSP-1, Moscow 119991, 
Russian Federation; \mbox{mih\_monah@mail.ru}

\vspace*{3pt}

\noindent
\textbf{Ulyanov Vladimir V.} (b.\ 1953)~---
Doctor of Science in physics and mathematics, professor, Faculty 
of Computational Mathematics and Cybernetics,  M.\,V.~Lomonosov Moscow State University, 
1-52~Leninskiye Gory, GSP-1, Moscow 119991, Russian Federation; 
\mbox{vulyanov@cs.msu.ru}

\label{end\stat}


\renewcommand{\bibname}{\protect\rm Литература} %10
\def\stat{kondranin+ushakov}

\def\tit{СИСТЕМА ОБСЛУЖИВАНИЯ С~ОТНОСИТЕЛЬНЫМ ПРИОРИТЕТОМ  И~ПРОФИЛАКТИКАМИ ПРИБОРА$^*$}

\def\titkol{Система обслуживания с~относительным приоритетом  и~профилактиками прибора}

\def\aut{Е.\,С.~Кондранин$^1$,  В.\,Г.~Ушаков$^2$}

\def\autkol{Е.\,С.~Кондранин,  В.\,Г.~Ушаков}

\titel{\tit}{\aut}{\autkol}{\titkol}

\index{Кондранин Е.\,С.}
\index{Ушаков В.\,Г.}
\index{Kondranin E.\,S.}
\index{Ushakov V.\,G.}




{\renewcommand{\thefootnote}{\fnsymbol{footnote}} \footnotetext[1]
{Работа выполнена при финансовой поддержке РФФИ (проект 18-07-00678).}}


\renewcommand{\thefootnote}{\arabic{footnote}}
\footnotetext[1]{Факультет вычислительной математики и~кибернетики Московского государственного 
университета им.\ М.\,В.~Ломоносова, \mbox{ekondranin@yandex.ru}}
\footnotetext[2]{Факультет вычислительной математики и~кибернетики
Московского государственного университета им.\ М.\,В.~Ломоносова;
Институт проб\-лем информатики Федерального исследовательского
центра <<Информатика и~управ\-ле\-ние>> Российской академии наук,
\mbox{vgushakov@mail.ru}}

\vspace*{-10pt}




\Abst{Изучена одноканальная система
массового обслуживания с~двумя типами требований, бесконечным
числом мест для ожидания, гиперэкспоненциальным входящим потоком 
и~профилактиками обслуживающего прибора при освобождении системы.
Тип  требования определяется случайно с~заданными вероятностями 
в~момент его поступления в~систему обслуживания. Требования первого
типа имеют относительный приоритет перед требованиями второго
типа. Найдено нестационарное совместное распределение числа
требований каждого типа в~системе. Профилактики прибора
заключаются в~том, что в~момент освобождения системы от требований
прибор на случайное время с~заданным распределением становится
недоступным для обслуживания. Если за время профилактики поступает
хотя бы одно требование, то начинается нормальное функционирование
системы. Если требования не поступают, то прибор отправляется на
новую профилактику. Такие системы хорошо описывают
функционирование большого числа реальных вычислительных и~информационных систем.}

\KW{гиперэкспоненциальный поток; профилактики
обслуживающего прибора; одноканальная система; относительный
приоритет; длина очереди}

\DOI{10.14357/19922264180405}
  
%\vspace*{4pt}


\vskip 10pt plus 9pt minus 6pt

\thispagestyle{headings}

\begin{multicols}{2}

\label{st\stat}

\section{Введение}

В классической системе массового обслуживания ожидание требований
в очереди связано только с~занятостью обслуживающего прибора. В~то
же время в~реальных системах сам  прибор может пребывать как 
в~активном, так и~в~неактивном состоянии. Такое неактивное
состояние прибора (в~литературе на английском языке используется
термин vacation, а~на русском~--- профилактика или прогулка) может
быть связано со многими причинами. В~част\-ности, сис\-те\-мы
обслуживания с~профилактиками прибора хорошо описывают
функционирование  реальных вычислительных и~информационных систем,
в которых наряду с~основными требованиями имеются второстепенные.
Второстепенные требования всегда присутствуют в~сис\-те\-ме, а~их
обслуживание может проводиться только тогда, когда нет основных,
т.\,е.\ в~фоновом режиме.

С точки зрения самого процесса профилактики прибора существует
несколько ее разновидностей. Во-пер\-вых, могут быть разными
правила, задающие условия начала профилактики: прибор может брать
перерыв только при  полном исчерпании требований в~очереди
(exhaustive service) либо при наличии определенного их числа
(nonexhaustive service). Во-вто\-рых, могут быть разными правила
возвращения прибора в~работу. С~этой точки зрения различают случаи
однократного (single vacation) и~многократного (multiple vacation)
перерыва в~работе. В~первом случае ушедший на профилактику прибор
после ее окончания находится в~рабочем состоянии независимо от
наличия требований в~системе. Во втором случае прибор, не
обнаружив новых требований в~очереди, уходит на новую
профилактику.


В работах~[1--4] можно найти обзор известных результатов, большое
число постановок задач, описание различных приложений и~обширную
библиографию по анализу систем с~профилактиками обслуживающего
прибора.


В настоящей работе исследуется совместное распределение длин
очередей в~нестационарном режиме в~однолинейной системе 
с~ожиданием, гиперэкспоненциальным входящим потоком, двумя типами
требований и~относительным приоритетом. Аналогичная неприоритетная
система обслуживания исследована в~[5].

\vspace*{-6pt}

\section{Описание модели}

Рассматривается однолинейная система массового обслуживания 
с~двумя приоритетными классами требований. Входящий поток~---
гиперэкспоненциальный с~функцией распределения интервалов между
поступлениями требований вида:
\begin{multline*}
A(t)=\sum\limits_{i=1}^kc_i\left(1-e^{-a_it}\right),\enskip t>0,\enskip
a_i>0,\enskip c_i>0,\\
a_i\ne a_j\,,\enskip i\ne j\,,\enskip  \sum\limits_{i=1}^k c_i=1\,.
\end{multline*}

Каждое поступившее требование направляется в~первый класс 
с~вероятностью~$p$ и~во второй класс с~вероятностью $1\hm-p$
независимо от остальных требований. Требования первого класса
обладают относительным приоритетом перед требованиями второго
класса. Длительности обслуживания требований $i$-го приоритетного
класса~--- независимые в~совокупности и~не зависящие от входящего
потока случайные величины с~функцией распределения~$B_i(x)$,
$i\hm=1,2.$
 Если в~некоторый момент времени система освободилась от требований, 
 то обслуживающий прибор
 отправляется на профилактику, которая длится случайное время с~функцией 
 распределения~$C(x).$
 Не ограничивая общности, будем считать, что $B_i(x)\hm<1$
 и~$C(x)\hm<1$  для любого~$x$ 
 и~существуют плотности
 распределения~$b_i(x)$ и~$c(x).$
  Обозначим:
$$
 \beta_i(s)=\int\limits_0^{\infty}e^{-sx}b_i(x)\,dx\,;\enskip 
  \gamma(s)=\int\limits_0^{\infty}e^{-sx}c(x)\,dx\,.
$$
Пока прибор находится на профилактике, он не доступен для
обслуживания. Если за время профилактики поступают требования,
после ее завершения начинается их обслуживание. Если ни одно
требование не поступает, то прибор отправляется на новую
профилактику. Длительности различных профилактик являются
независимыми случайными величинами 
и~не зависят от входящего потока и~времен обслуживания.

\section{Вспомогательные результаты}

  Рассмотрим многочлен по $\mu$ степени $k$ вида:
\begin{multline}
\label{1}
\prod\limits_{i=1}^k\left(\mu+a_i\right)-{}\\
{}-
\left(pz_1+(1-p)z_2\right)\sum\limits_{j=1}^kc_ja_j\prod\limits_{i\ne
j}\left(\mu+a_i\right)\,.
\end{multline}
Занумеруем его корни $\mu_1(z_1,z_2),\ldots,\mu_k(z_1,z_2)$ таким образом,
чтобы они были непрерывными функциями и~$\mu_1(1,1)\hm=0.$ Тогда
$\mathrm{Re}\, \mu_j\left(z_1,z_2\right)\hm<0$, $|z_1|\hm<1$, 
$|z_2|\hm<1,$ $\mu_i(z_1,z_2)\hm\ne \mu_j(z_1,z_2),$ $ i\hm\ne j$,
$j\hm=1,\ldots,k.$ Обозначим:
$$
\alpha_m(z_1,z_2)=\prod\limits_{j\ne m}\left(\mu_m\left(z_1,z_2\right)-
\mu_j\left(z_1,z_2\right)\right)\,.
$$
Справедливы следующие леммы.

\smallskip

\noindent
\textbf{Лемма~1.}\
\textit{Для любого $l=1,\ldots,\:k$ система уравнений}
$$
z_j=\beta_j(s-\mu_l(z_1,z_2)),\ \ j=1,2,
$$
\textit{имеет единственное решение $z_i=z_{il}(s)$ такое, 
что $|z_{il}(s)|\hm<1$ при $l\hm=2,\ldots, k,$ $\mathrm{Re}\, s\hm\geqslant 0,$ 
а~$z_{i1}(0)\hm=1$, $|z_{i1}(s)|\hm<1$ при} $\mathrm{Re}\, s\hm> 0$, $i\hm=1,2.$

\smallskip

\noindent
\textbf{Лемма~2.}\
\textit{При каждом $l\hm=1,\ldots,k$ уравнение}
$$
z_1=\beta_1\left(s-\mu_l(z_1,z_2)\right)
$$
\textit{имеет единственное решение $z_1\hm=z_{1l}(z_2,s),$ 
аналитическое в~области $\mathrm{Re}\, s\hm>0$, $|z_2|\hm<1.$
}

\smallskip

Положим
$$
\lambda_l(s)=\mu_l\left(z_{1l}(s),z_{2l}(s)\right)\,.
$$




\section{Распределение длины очереди}

  Гиперэкспоненциальный поток можно рас\-смат\-ри\-вать как
пуассоновский поток со случайной интен\-сив\-ностью~$a,$ которая
принимает $k$ различных значений $a_1,\ldots,a_k$  с~вероятностями
$c_1,\ldots,c_k.$ Текущее значение~$a$ разыгрывается в~момент
поступления требования и~не меняется между двумя соседними
поступлениями. Введем случайный процесс~$j(t)$ такой, что если
$a\hm=a_j$ в~момент времени $t,$ то $j(t)\hm=j.$

Целью работы является нахождение распределения случайного процесса
$\left(L_1(t),L_2(t)\right),$ где $L_i(t)$~--- число требований из
$i$-го приоритетного класса, находящихся в~системе в~момент
времени~$t.$

При сделанных предположениях относительно параметров изучаемой
системы обслуживания\linebreak процесс $\left(L_1(t),L_2(t)\right)$ не
является, вообще говоря, марковским. Пусть $i(t)=i$, $i\hm=1,2,$ если
в~момент времени~$t$ обслуживается требование из $i$-го
приоритетного класса, и~$i(t)\hm=0,$ если в~момент времени~$t$ прибор
находится на профилактике. Случайный процесс~$x(t)$ определим
следующим образом. Если $i(t)\hm\ne 0,$ то $x(t)$ есть
время, прошедшее с~начала обслуживания требования, находящегося на
приборе, до момента~$t.$ Если $i(t)\hm=0,$ то $x(t)$ есть время,
прошедшее с~начала профилактики прибора до момента~$t.$ Случайный
процесс $\left(L_1(t),L_2(t),i(t),j(t),x(t)\right)$ является
однородным марковским процессом. Положим
\begin{multline*}
P_{ij}(n_1,n_2,x,t)=\fr{\partial}{\partial x}
\mathbf{P}\left(L_1(t)=n_1,L_2(t)=n_2,\right.\\
\left. i(t)=i,j(t)=j,x(t)<x
\vphantom{L_1}\right)\,,\enskip 
 x\geqslant 0,\\ 
 j=1,\ldots,k,\enskip i=0,1,2;
\end{multline*}
\begin{gather*}
\eta_i(x)=\fr{b_i(x)}{1-B_i(x)},\ i=1,2;\enskip 
\eta_0(x)=\fr{c(x)}{1-C(x)}\,;\\
\delta_{i,j}=\begin{cases}
1,&\ i=j;\\ 
0,&\ i\ne j\,.
\end{cases}
\end{gather*}
Функции $P_{ij}(n_1,n_2,x,t)$  удовлетворяют при $x\hm>0$
системам дифференциальных уравнений:
\begin{multline}
\label{3}
\fr{\partial P_{ij}(n_1,n_2,x,t)}{\partial t}+\fr{\partial
P_{ij}(n_1,n_2,x,t)}{\partial
x}={}\\
{}=-(a_j+\eta_i(x))P_{ij}(n_1,n_2,x,t)+ {}\\
{}+
c_j\sum\limits_{l=1}^ka_l\left(p\:P_{il}(n_1-1,n_2,x,t)+{}\right.\\
\left.{}+
(1-p)P_{il}(n_1,n_2-1,x,t)\right)
\end{multline}
и краевым условиям при $x\hm=0$:
\begin{multline}
\label{5}
P_{0j}(n_1,n_2,0,t)=0,\ n_1+n_2>0;\\
P_{0j}(0,0,0,t)=\int\limits_0^{\infty}P_{0j}(0,0,x,t)\eta_0(x)\,dx+{}\\
 {}+\int\limits_0^{\infty}P_{1j}(1,0,x,t)\eta_1(x)dx+{}\\
 {}+
\int\limits_0^{\infty}P_{2j}(0,1,x,t)\eta_2(x)\,dx\,;
\end{multline}

\vspace*{-12pt}

\noindent
\begin{multline}
\label{6}
P_{1j}(n_1,n_2,0,t)+P_{2j}(n_1,n_2,0,t)={}\\
{}=\int\limits_0^{\infty}P_{1j}(n_1+1,n_2,x,t)\eta_1(x)\,dx+{}\\
{}+
\int\limits_0^{\infty}P_{2j}(n_1,n_2+1,x,t)\eta_2(x)\,dx+{}\\
{}+\int\limits_0^{\infty}P_{0j}(n_1,n_2,0,t)\eta_0(x)\,dx\,.
\end{multline}

Будем предполагать, что в~начальный момент времени $t\hm=0$ система
свободна от требований, а~с~начала профилактики прибора прошло
случайное время с~заданным распределением с~плотностью $d(x).$
Таким образом,
\begin{align*}
P_{ij}\left(n_1,n_2,x,0\right)&=0,\ i=1,2;
\\
P_{0j}\left(n_1,n_2,x,0\right)&=c_jd(x)\delta_{n_1+n_2,0},\ \
j=1,\ldots,k\,.
\end{align*}
Положим
\begin{multline*}
p_{ij}\left(z_1,z_2,x,s\right)={}\\
{}=\sum\limits_{n_1=0}^{\infty}
\sum\limits_{n_2=0}^{\infty}z_1^{n_1}z_2^{n_2}\!
\int\limits_0^{\infty}e^{-st}P_{ij}(n_1,n_2,x,t)\,dt\,;
\end{multline*}
$$
  \psi(s)=\int\limits_0^{\infty}e^{-sx}\,dx
  \int\limits_0^{\infty}\fr{c(u+x)d(u)}{1-C(u)}\,du\,.
$$
Тогда, учитывая начальные условия,  из \eqref{3}
получаем:
\begin{multline}
\label{7} 
\fr{\partial p_{ij}(z_1,z_2,x,s)}{\partial x}={}\\
{}=-\left(s+a_j+\eta_i(x)\right)p_{ij}
\left(z_1,z_2,x,s\right)+{}\\
{}+c_j\left(pz_1+(1-p)z_2\right)
\sum\limits_{l=1}^ka_lp_{il}\left(z_1,z_2,x,s\right),\\ 
i=1,2;
\end{multline}

\vspace*{-12pt}

\noindent
\begin{multline}
\label{8} 
\fr{\partial p_{0j}(z_1,z_2,x,s)}{\partial x}={}\\
{}=-\left(s+a_j+\eta_0(x)\right)p_{0j}\left(z_1,z_2,x,s\right)+{}\\
{}+c_j\left(pz_1+(1-p)z_2\right)\sum\limits_{l=1}^ka_lp_{0l}\left(z_1,z_2,x,s\right)+{}\\
{}+ c_jd(x).
\end{multline}
Решения \eqref{7} и~\eqref{8} имеют вид:
\begin{multline}
\label{9}
p_{ij}\left(z_1,z_2,x,s\right)=\left(1-B_i(x)\right)c_j\times{}\\
{}\times \sum\limits_{m=1}^k\fr{\gamma_i^{(m)}(z_1,z_2,s)}{\mu_m(z_1,z_2)+a_j}\,
e^{-(s-\mu_m(z_1,z_2))x}\,,\\
 i=1,2\,,
\end{multline}
\vspace*{-12pt}

\noindent
\begin{multline}
\label{10}
p_{0j}\left(z_1,z_2,x,s\right)={}\\
{}=\left(1-C(x)\right)
c_j\!\!\sum\limits_{m=1}^k\!\! e^{-(s-\mu_m(z_1,z_2))x}\!
\!\left(\!
\vphantom{\int\limits_{l=1}^k}
\delta^{(m)}\left(z_1,z_2,s\right)+{}\right.\\
%\left.
{}+\alpha_m^{-1}\left(z_1,z_2\right)
\prod\limits_{l=1}^k
\left(\mu_m\left(z_1,z_2\right)+a_l\right)\times{}\\
\left.{}\times \int\limits_0^x\!
e^{(s-\mu_m(z_1,z_2))u}
\fr{d(u)}{1-C(u)}\,du
\right)
\!\Bigg/ \!\left(\mu_m\left(z_1,z_2\right)+{}\right.\\
\left.{}+a_j\right)\,,
\end{multline}
где функции $\gamma_i^{(m)}(z_1,z_2,s)$  и~$\delta^{(m)}(z_1,z_2,s)$ являются
произвольными функциями указанных переменных и~определяются из
краевых условий. Из~\eqref{5} и~\eqref{6} получаем:
\begin{multline}
\label{11}
p_{1j}\left(z_1,z_2,0,s\right)+p_{2j}\left(z_1,z_2,0,s\right)={}\\
{}=z_1^{-1}\int\limits_0^{\infty}p_{1j}\left(z_1,z_2,x,s\right)\eta_1(x)\,dx+{}
\\
+z_2^{-1}\int\limits_0^{\infty}p_{2j}\left(z_1,z_2,x,s\right)\eta_2(x)\,dx+{}\\
{}+
\int\limits_0^{\infty}p_{0j}\left(z_1,z_2,x,s\right)\eta_0(x)\,dx
-p_{0j}\left(z_1,z_2,0,s\right)\,.
\end{multline}
Заметим, что $p_{0j}(z_1,z_2,0,s)$ не зависит от $z_1$ и~$z_2,$ т.\,е.\
$p_{0j}(z_1,z_2,0,s)\hm=q_j(s).$ 
Подставляя~\eqref{9} и~\eqref{10} в~\eqref{11}, получаем:
\begin{multline}
\label{12}
\gamma_1^{(m)}\left(z_1,z_2,s\right)\left(1-z_1^{-1}\beta_1(s-\mu_m(z_1,z_2))\right)+{}\\
{}+
\gamma_2^{(m)}(z_1,z_2,s)\left(1-z_2^{-1}\beta_2(s-\mu_m(z_1,z_2))\right)={}\\
{} =
\delta^{(m)}\left(z_1,z_2,s\right)\left(\gamma\left(s-\mu_m\left(z_1,z_2\right)\right)-1\right)+{}\\
{}+
\alpha_m^{-1}\left(z_1,z_2\right)\prod\limits_{l=1}^k
\left(\mu_m\left(z_1,z_2\right)+a_l\right)\psi\left(s-\mu_m(z_1,z_2)\right),\\
j=1,\ldots,k.
\end{multline}
В силу леммы~1 левая часть~\eqref{12} обращается в~0 при
$z_1\hm=z_{1m}(s)$ и~$z_2\hm=z_{2m}(s)$, $m\hm=1,\ldots,k.$ Следовательно,
\begin{multline}
\label{13}
\delta^{(m)}\left(z_{1m}(s),z_{2m}(s),s\right)={}\\
{}=\fr{\psi(s-\lambda_m(s))}{\alpha_m(z_{1m}(s),z_{2m}(s))
(1-\gamma(s-\lambda_m(s)))}\times{}\\
{}\times \prod\limits_{l=1}^k\left(\lambda_m(s)+a_l\right).
\end{multline}
Из \eqref{10} следует, что
$$
q_j(s)=c_j\sum\limits_{m=1}^k\fr{\delta^{(m)}(z_1,z_2,s)}{\mu_m(z_1,z_2)+a_j},\
j=1,\ldots,k .
$$
Решая эту систему уравнений относительно
$\delta^{(m)}(z_1,z_2,s),$ получаем:
\begin{multline}
\label{n1}
\delta^{(m)}(z_1,z_2,s)=\left(pz_1+(1-p)z_2\right)\times{}\\
{}\times
\fr{\prod\nolimits_{j=1}^k(\mu_m(z_1,z_2)+a_j)}
{\alpha_m(z_1,z_2)}\sum\limits_{l=1}^k\frac{a_lq_l(s)}{\mu_m(z_1,z_2)+a_l}.
\end{multline}
Подставляя в~\eqref{n1} $z_1\hm=z_{1m}(s)$ и~$z_2\hm=z_{2m}(s),$ имеем:
\begin{multline}
\label{14}
\delta^{(m)}\left(z_{1m}(s),z_{1m}(s),s\right)={}\\
{}=
\left(pz_{1m}(s)+(1-p)z_{2m}(s)\right)\times{}\\
{}\times
\fr{\prod\nolimits_{j=1}^k
(\lambda_m(s)+a_j)}{\alpha_m(z_{1m}(s),z_{1m}(s))}
\sum\limits_{l=1}^k\fr{a_lq_l(s)}{\lambda_m(s)+a_l}\,.
\end{multline}
Сравнивая два представления~\eqref{13} в~\eqref{14} для
$\delta^{(m)}(z_m(s),s),$ получаем систему уравнений для~$q_l(s)$:
\begin{multline*}
\sum\limits_{l=1}^k\fr{a_lq_l(s)}{\lambda_m(s)+a_l}={}\\
{}=\fr{\psi(s-\lambda_m(s))}{(pz_{1m}(s)+(1-p)z_{2m}(s))
(1-\gamma(s-\lambda_m(s)))},\\
m=1,\ldots,k\,,
\end{multline*}
из которой находим
\begin{multline}
\hspace*{-3pt}q_l(s)=c_l\prod\limits_{j=1}^k
\left(\lambda_l(s)+a_j\right) 
\sum\limits_{m=1}^k
%\fr
\psi(s-\lambda_m(s))\!\Bigg/ \!
\Bigg(\left(1-{}\right.\\
\left.
{}-\gamma\left(s-\lambda_m(s)\right)\right)(\lambda_m(s)+a_l)\times{}\\
{}\times \prod\limits_{n\ne m}(\lambda_m(s)-\lambda_n(s))\!\Bigg).
\label{15}
\end{multline}
Подставляя \eqref{15} в~\eqref{n1} и~учитывая~\eqref{1}, получаем:
\begin{multline*}
\delta^{(m)}(z_1,z_2,s)=\fr{(pz_1+(1-p)z_2)}{\alpha_m(z_1,z_2)}\times
\\
\times\sum\limits_{j=1}^k
\fr{\psi(s-\lambda_j(s))\prod\nolimits_{l=1}^k(\lambda_j(s)+a_l)}
{(pz_{1j}(s)+(1-p)z_{2j}(s))(1-\gamma(s-\lambda_j(s)))}\times{}\\
{}\times\prod\limits_{\nu\ne j}
\fr{\mu_m(z_1,z_2)-\lambda_{\nu}(s)}{\lambda_j(s)-\lambda_{\nu}(s)}\,.
\end{multline*}
Положим
$$
\lambda_m(z_2,s)=\mu_m\left(z_{1m}(z_2,s),z_2\right),\enskip m=1,\ldots,k\,.
$$
Подставляя в~\eqref{12} $z_1\hm=z_{1m}(z_2,s)$, имеем:
\begin{multline}
\label{1q}
\gamma_2^{(m)}\left(z_{1m}(z_2,s),z_2,s\right)={}\\
{}=\fr{\delta^{(m)}(z_{1m}(z_2,s),z_2,s)(\gamma_m(s-\lambda_m(z_2,s))-1)}
{1-z_2^{-1}\beta_2(s-\lambda_m(z_2,s))}+{}
\\
{}+\alpha_m^{-1}(z_{1m}(z_2,s),z_2)\psi(s-\lambda_m(z_2,s))
\prod\limits_{l=1}^k\left(\lambda_m(z_2,s)+{}\right.\\
\left.{}+a_l\right)\!\Bigg/\!
\left(
1-z_2^{-1}\beta_2(s-\lambda_m(z_2,s))\right).
\end{multline}
Далее, из~\eqref{9} следует:
$$
p_{2j}(z_1,z_2,0,s)=c_j\sum\limits_{m=1}^k
\fr{\gamma_2^{(m)}(z_1,z_2,s)}{\mu_m(z_1,z_2)+a_j}\,.
$$
Отсюда
\begin{multline}
\label{2q}
\gamma_2^{(m)}(z_1,z_2,s)=\fr{pz_1+(1-p)z_2}{\alpha_m(z_1,z_2)}\times{}\\
{}\times
\prod\limits_{j=1}^k(\mu_m(z_1,z_2)+a_j)
\sum\limits_{l=1}^k\fr{a_lp_{2l}(z_1,z_2,0,s)}{\mu_m(z_1,z_2)+a_l}\,.
\end{multline}
Так как $p_{2j}(z_1,z_2,0,s)$ не зависит от $z_1$, то
\begin{multline}
\label{3q}
p_{2j}\left(z_1,z_2,0,s\right)={}\\
{}=c_j
\sum\limits_{m=1}^k\fr{\gamma_2^{(m)}\left(z_{1m}(z_2,s),z_2,s\right)}{\lambda_m(z_2,s)+a_j}\,.
\end{multline}
Таким образом, соотношения~\eqref{1q}--\eqref{3q} полностью
определяют $\gamma_2^{(m)}(z_1,z_2,s)$ при любых $z_1$ и~$z_2$.
Теперь из~\eqref{12} можно найти $\gamma_2^{(m)}(z_1,z_2,s)$.

Все функции, необходимые для вычисления $p_{ij}(z_1,z_2,x,s)$,
$i\hm=0,1,2$, $j\hm=1,\ldots,k,$ найде-\linebreak\vspace*{-12pt}

\columnbreak

\noindent
ны. Искомая производящая функция
процесса $(L_1(t),L_2(t))$ равна:

\noindent
\begin{multline*}
\int\limits_0^{\infty}e^{-st}\mathbf{E}
z_1^{L_1(t)} z_2^{L_2(t)}\,dt={}\\
{}=
\sum\limits_{i=0}^2\sum\limits_{j=1}^k\int\limits_0^{\infty}p_{ij}
\left(z_1,z_2,x,s\right)\,dx\,.
\end{multline*}

\vspace*{-18pt}

{\small\frenchspacing
 {%\baselineskip=10.8pt
 \addcontentsline{toc}{section}{References}
 \begin{thebibliography}{9}
\bibitem{1-u}
\Au{Doshi B.\,T.} Queueing systems with vacations~--- a~survey~// 
Queueing Syst., 1986. Vol.~1.  P.~29--66.
\bibitem{2-u}
\Au{Takagi H.} Time-dependent analysis of $M\vert G\vert 1$ vacation models 
with exhaustive service~// Queueing Syst.,
1990. Vol.~6.  P.~369--390.
\bibitem{3-u}
\Au{Li J., Tian N., Zhang~Z.\,G. , Luh~H.\,P.} 
Analysis of the $M\vert G\vert 1$ queue with exponentially working vacations~--- 
a~matrix analytic approach~// Queueing Syst., 2009. Vol.~61.
P.~139--166.
\bibitem{4-u}
\Au{Bouman N., Borst S.\,C., Boxma~O.\,J., Leeuwaarden~J.\,S.\,H.} 
Queues with random back-offs~// Queueing Syst.,
2014. Vol.~77. P.~33--74.
\bibitem{5-u}
\Au{Ушаков~В.\,Г.} Система обслуживания с~гиперэкспоненциальным входящим потоком 
и~профилактиками прибора~// Информатика и~её применения, 2016. Т.~10. 
Вып.~2. С.~93--98.
 \end{thebibliography}

 }
 }

\end{multicols}

\vspace*{-9pt}

\hfill{\small\textit{Поступила в~редакцию 11.05.18}}

\vspace*{6pt}

%\pagebreak

%\newpage

%\vspace*{-28pt}

\hrule

\vspace*{2pt}

\hrule

%\vspace*{-2pt}

\def\tit{A~HEAD OF~THE~LINE PRIORITY QUEUE\\ WITH~WORKING VACATIONS}

\def\titkol{A head of the line priority queue with working vacations}

\def\aut{E.\,S.~Kondranin$^1$ and~V.\,G.~Ushakov$^{1,2}$}

\def\autkol{E.\,S.~Kondranin and~V.\,G.~Ushakov}

\titel{\tit}{\aut}{\autkol}{\titkol}

\vspace*{-11pt}


\noindent
$^1$Department of 
Mathematical Statistics, Faculty of Computational Mathematics and Cybernetics, 
M.\,V.~Lo\-mo-\linebreak
$\hphantom{^1}$no\-sov Moscow State University, 1-52~Leninskiye Gory, 
Moscow 119991, GSP-1, Russian Federation

\noindent
$^2$Institute of Informatics Problems, Federal Research Center 
``Computer Science and Control'' of the Russian\linebreak
$\hphantom{^1}$Academy of Sciences,  44-2~Vavilov Str., Moscow 119333, Russian Federation

\def\leftfootline{\small{\textbf{\thepage}
\hfill INFORMATIKA I EE PRIMENENIYA~--- INFORMATICS AND
APPLICATIONS\ \ \ 2018\ \ \ volume~12\ \ \ issue\ 4}
}%
 \def\rightfootline{\small{INFORMATIKA I EE PRIMENENIYA~---
INFORMATICS AND APPLICATIONS\ \ \ 2018\ \ \ volume~12\ \ \ issue\ 4
\hfill \textbf{\thepage}}}

\vspace*{3pt}



\Abste{The authors analyze the single-server queueing system with 
two types of customers, head of the line priority, hyperexponential 
input stream, and working vacations. The authors obtain the Laplace 
transform (with respect to an arbitrary point in time) of the joint 
distribution of server state, queue size, and elapsed time in that state. 
The authors restrict themselves to a~system with exhaustive service (the 
queue must be empty when the server starts a vacation) and multiple vacations. 
The queueing systems with vacations have been well studied because of their 
applications in modeling computer networks, communication, and manufacturing 
systems. For example, in many digital systems, the processor is multiplexed 
among a~number of jobs and, hence, is not available all the time to handle one job type. 
Besides such an application, theoretical interest in vacation models 
has been aroused with respect to their relationship with polling models.}

\KWE{hyperexponential input stream; working vacations; single server; 
head of the line priority; queue length}



\DOI{10.14357/19922264180405}

\vspace*{-14pt}

\Ack
\noindent
This work was supported by the Russian Foundation for Basic Research 
(project 18-07-00678).


%\vspace*{6pt}

  \begin{multicols}{2}

\renewcommand{\bibname}{\protect\rmfamily References}
%\renewcommand{\bibname}{\large\protect\rm References}

{\small\frenchspacing
 {%\baselineskip=10.8pt
 \addcontentsline{toc}{section}{References}
 \begin{thebibliography}{9}
\bibitem{1-u-1}
\Aue{Doshi, B.\,T.} 1986. Queueing systems with vacations~--- a~survey. 
\textit{Queueing Syst.} 1:29--66.
\bibitem{2-u-1}
\Aue{Takagi, H.} 1990. Time-dependent analysis of $M\vert G\vert M\vert 1$ 
vacation models with exhaustive service. \textit{Queueing Syst.} 6:369--390.
\bibitem{3-u-1}
\Aue{Li, J., N. Tian, Z.\,G.~Zhang,  and H.\,P.~Luh.} 2009. Analysis of the 
$M\vert G\vert 1$ queue with exponentially working vacations~--- 
a~matrix analytic approach. \textit{Queueing Syst.} 61:139--166.
{\looseness=1

}
\bibitem{4-u-1}
\Aue{Bouman, N., S.\,C.~Borst, O.\,J.~Boxma, and J.\,S.\,H.~Leeuwaarden.} 
2014. Queues with random back-offs. \textit{Queueing Syst.} 77:33--74.
\bibitem{5-u-1}
\Aue{Ushakov, V.\,G.} 2016. Sistema obsluzhivaniya s~gipereksponentsialnym 
vkhodyashchim potokom i~profilaktikami\linebreak pribora [Queueing system with working 
vacations and hyperexponential input stream]. 
\textit{Informatika i~ee Primeneniya~--- Inform. Appl.} 10(2):93--98.
\end{thebibliography}

 }
 }

\end{multicols}

\vspace*{-6pt}

\hfill{\small\textit{Received May 11, 2018}}

%\pagebreak

%\vspace*{-18pt}

\Contr

\noindent
\textbf{Kondranin Egor S.} (b.\ 1995)~---  MSc student, Department of 
Mathematical Statistics, Faculty of Computational Mathematics and Cybernetics, 
M.\,V.~Lomonosov Moscow State University, 1-52~Leninskiye Gory, 
Moscow 119991, GSP-1, Russian Federation; \mbox{ekondranin@yandex.ru}

\vspace*{6pt}

\noindent
\textbf{Ushakov Vladimir G.} (b.\ 1952)~--- 
Doctor of Science in physics and mathematics, professor, Department of Mathematical 
Statistics, Faculty of Computational Mathematics and Cybernetics, 
M.\,V.~Lomonosov Moscow State University, 1-52~Leninskiye Gory, Moscow 119991, 
GSP-1, Russian Federation; 
senior scientist, Institute of Informatics Problems, Federal Research Center 
``Computer Science and Control'' of the Russian Academy of Sciences, 
44-2~Vavilov Str., Moscow 119333, Russian Federation; \mbox{vgushakov@mail.ru}
\label{end\stat}

\renewcommand{\bibname}{\protect\rm Литература}        %11
\def\stat{khokhlov}

\def\tit{МНОГОМЕРНОЕ ДРОБНОЕ ДВИЖЕНИЕ ЛЕВИ И~ЕГО~ПРИЛОЖЕНИЯ$^*$}

\def\titkol{Многомерное дробное движение Леви и~его приложения}

\def\aut{Ю.\,С.~Хохлов$^1$}

\def\autkol{Ю.\,С.~Хохлов}

\titel{\tit}{\aut}{\autkol}{\titkol}

\index{Хохлов Ю.\,С.}
\index{Khokhlov Yu.\,S.}

{\renewcommand{\thefootnote}{\fnsymbol{footnote}} \footnotetext[1]
{Работа поддержана РНФ (проект 14-11-00364).}}


\renewcommand{\thefootnote}{\arabic{footnote}}
\footnotetext[1]{Московский государственный университет им.~М.\,В.~Ломоносова, \mbox{yskhokhlov@yandex.ru} }                             


\Abst{С начала 1990-х гг.\ было проведено большое число эмпирических исследований трафика реальных
телекоммуникационных систем. Было обнаружено, что он обладает рядом специфических 
свойств, отличающих его от обычного
голосового трафика, а~именно: он обладает свойствами самоподобия и~долговременной 
зависимости и~распределение
величины нагрузки, поступающей от одного источника, имеет тяжелые хвосты. Были 
построены новые модели трафика, которые
обладали указанными свойствами. Наиболее известные из них~--- дробное броуновское 
движение и
$\alpha$-устой\-чи\-вое движение Леви. Но каж\-дая из этих моделей обладает 
только частью из перечисленных выше свойств.
Были предприняты попытки построить более сложные модели, являющиеся комбинацией 
этих двух, в~частности, предложен
некоторый вариант одномерного дробного движения Леви.
%
В настоящей работе рассматривается многомерный аналог дробного движения Леви. Этот процесс представляет собой
    многомерное дробное броуновское движение со случайной заменой времени, в~качестве которой рассматривается
$\alpha$-устой\-чи\-вое движение Леви с~односторонними устойчивыми
распределениями. Изучены свойства этого процесса, показано, что он
является самоподобным и~имеет стационарные приращения. Показано
также, что координаты одномерных сечений этого процесса имеют
распределения, отличные от устойчивых. Но асимптотика хвостов этих
распределений в~точности такая же, как и~у~устойчивых распределений.
Далее эта модель использована для анализа неоднородного трафика, 
и~получена нижняя асимптотическая оценка для вероятности переполнения
хотя бы одного буфера при условии, что все буферы большие. Возможны
и~другие приложения.}


\KW{многомерное дробное броуновское движение; $\alpha$-устой\-чивый субординатор; 
самоподобные процессы; вероятность переполнения буфера}


\DOI{10.14357/19922264160212} 

%\vspace*{-4pt}

\vskip 10pt plus 9pt minus 6pt

\thispagestyle{headings}

\begin{multicols}{2}

\label{st\stat}

\section{Введение}

Во многих прикладных задачах часто встречается ситуация, когда процесс риска является
принципиально многомерным. Например, нагрузка на сервер может поступать от многих источников
по несколь\-ким каналам. При этом нагрузка по каж\-до\-му каналу формируется таким образом, что эти
потоки оказываются зависимыми. Страховые компании ведут свою деятельность по многим 
на\-прав\-ле\-ни\-ям,
каждое из которых формирует свой поток рисков, но они взаимосвязаны.
При формировании инвестиционного портфеля используются различные типы ценных бумаг, доходности
которых меняются со временем, при этом они зависят от некоторых общих факторов, определяющих
состояние рынка. Все эти примеры ведут к~многомерным процессам риска с~зависимыми компонентами.
Существует и~множество других примеров, возника\-ющих, например, в~физических задачах.

Многочисленные эмпирические исследования показали, что процессы риска в~описанных выше областях обладают
двумя важными свойствами, а~именно:  свойством долговременной за\-ви\-си\-мости 
и~тяжелыми хвостами распределений
вероятностей. Обычно при управлении процессами риска пользуются усредненными 
характеристиками процессов,\linebreak
которые при сильном агрегировании быстро ста\-билизируются. Это сильно упрощает задачи 
управ\-ле\-ния,
так как позволяет работать с~практически неслучай\-ными величинами. 
Но эмпирические исследования последних 20 лет
показали, что современные процессы риска в~широком диапазоне интервалов 
агрегирования не стабилизируются,
а~остаются (после некоторой нормировки) практически такими же. 
Подобное поведение случайного процесса
называют его самоподобием.

Сказанное выше приводит к~задаче построения многомерных моделей
риска, которые обладают тремя важными свойствами: самоподобием,
долговременной зависимостью и~тяжелыми хвостами распределений.

В одномерном случае этой проблематике посвящено большое число работ.
Обычно рассматривают самоподобные процессы со стационарными
приращениями. 

Типичными примерами таких процессов служат дробное
броуновское движение и~устойчивое движение Леви. Первый процесс
обладает свойством долговременной зависимости, но, как известно,
гауссовское распределение имеет быстро убывающий хвост. Напротив,
устойчивое движение Леви имеет независимые приращения, но устойчивые
распределения (с~показателем устойчивости меньше~2) есть типичный
пример распределений с~тяжелыми хвостами. В~определенном смысле эти
два процесса являются антиподами. В~то же время и~тот и~другой класс
процессов в~качестве частного случая содержит обычное броуновское
движение. 

Хотелось бы найти такую модель, которая объединяла бы
свойства этих двух.  Было предложено несколько вариантов решения
этой задачи. Одним из них является модель FARIMA, использующая
понятие дробного сдвига. Похожий вариант для случая непрерывного
времени используется в~так называемой модели линейного дробного
устойчивого движения. Существует также модель
ло\-га\-риф\-ми\-че\-ски-дроб\-но\-го устойчивого движения и~некоторые другие.
Наиболее близка к~настоящему исследованию работа~\cite{Las02},
которая использует подход Мандельброта при построении дробного
броуновского движения с~помощью некоторого стохастического интеграла
по обычному броуновскому движению. Построенный авторами процесс
является самоподобным и~имеет устойчивые распределения. Но его
использование в~задачах моделирования вызывает некоторые затруднения
в силу его сложной структуры. В~работе~\cite{Ni12} предложен новый
вариант одномерного дробного движения Леви, который представляет
собой подчиненный  процесс, где управляемый процесс есть дробное
броуновское движение, а~управляющий процесс есть одностороннее
устойчивое движение Леви с~показателем устойчивости меньше~1. Эта
модель была использована для получения нижней оценки вероятности
переполнения большого буфера.

В настоящей работе рассматривается аналогичная модель в~многомерном
случае. В~качестве исходного объекта берется многомерное дробное
броуновское движение. Подробное исследование этого процесса
проведено в~работах~\cite{Sto06, Amb12}. Далее
рассматривается подчиненный процесс, для которого управляемый
процесс есть многомерное дробное броуновское движение, а~управляющий
процесс (случайное время), как и~раньше, есть одностороннее
устойчивое движение Леви. Исследованы свойства такого многомерного
процесса. Далее эта модель применяется в~некоторой задаче из теории
телетрафика.

\section{Устойчивые распределения и~процессы Леви}

В предлагаемой модели важную роль играют устойчивые распределения 
и~процессы Леви. Далее приводятся некоторые известные
определения и~рассматриваются необходимые свойства таких распределений и~процессов.

\smallskip

\noindent
\textbf{Определение~1.} 
Случайный процесс $Y \hm= (Y(t), t\hm\geq 0)$ со значениями в~$R^d$ 
называется процессом Леви, если
\begin{enumerate}[(1)]
\item $Y(0) = 0$ п.~н.;
\item $Y$ имеет независимые приращения;
\item $Y$ имеет стационарные приращения, т.\,е.\ для любых $t\hm\geq 0$, $h\hm>0$ 
случайный вектор
$Y(t+h) \hm- Y(t)$ имеет распределение, не зависящее от~$t$.
\end{enumerate}

Очень часто из соображений регулярности требуют выполнения следующего свойства:
с вероятностью единица все траектории~$Y$ должны быть непрерывными справа и~иметь конечные пределы слева.
Это не является дополнительным ограничением, так как всегда можно построить реализацию процесса Леви с~таким
свойством.

Хорошо известно, что все конечномерные распределения процесса~$Y$ 
однозначно определяются по распределению
случайного вектора $Y(1)$, которое является безгранично делимым.

Одним из наиболее известных одномерных примеров процессов Леви служит процесс 
броуновского движения (или винеровский процесс).


\smallskip

\noindent
\textbf{Определение~2.} 
Процесс Леви $B \hm= (B(t), t\geq 0)$ со значениями в~$R^1$ 
называется процессом броуновского движения (BM~--- Brownian motion), если
для любых $t\hm\geq 0$, $h\hm>0$ приращение $B(t+h)\hm -B(t)$ 
имеет гауссовское распределение с~нулевым средним и~дисперсией $\sigma^2 h$.


\smallskip

Если $\sigma^2 =1$,  то имеет место стандартное броуновское движение. 
Нетрудно показать, что
$$
K(t,s) = \mathrm{Cov}\, (Y(t), Y(s)) = \sigma^2 \min (t,s)  \,.
$$


По определению броуновское движение имеет гауссовские распределения. 
В~силу центральной предельной теоремы
такие распределения получаются асимптотически для нормированных сумм 
независимых и~одинаково распределенных случайных величин с~конечной дисперсией.
В~случае бесконечных дисперсий приходим к~понятию устойчивого распределения.


\smallskip

\noindent
\textbf{Определение~3.} 
Говорят, что случайная величина~$Y$ имеет $\alpha$-устой\-чи\-вое распределение, 
если ее характеристическая функция имеет следующий вид:
\begin{multline*}
\varphi (\omega ) := E\left[ e^{i \omega X} \right] ={}\\
{}= \exp \{
i\mu\omega - \sigma |\omega |^{\alpha} [1 - i \beta \,\mathrm{sign}\, (\omega )
\theta (\omega , \alpha )] \},
\end{multline*}
где $0< \alpha \hm\leq 2$; $\sigma\hm\geq 0$;
$-1 \hm\leq \beta \hm\leq 1$; $\mu\hm\in R^1$;
$$
\theta (\omega , \alpha ) =
\begin{cases}
\tan \left( \fr{\alpha \pi }{2} \right)& \alpha \not= 1\,; \\
-\fr{2}{\pi} \ln |\omega |& \alpha =1\,.
\end{cases}
$$


%\smallskip


Параметр $\alpha$ называется {\it характеристическим показателем} 
и~определяет скорость убывания хвостов распределения;
$\sigma$ и~$\mu$ являются параметрами {\it масштаба } и~{\it сдвига} соответственно; 
$\beta$
называется {\it параметром асимметрии}. Если $\beta \hm=0$, то $X$ имеет 
симметричное относительно~$\mu$ распределение.
Если $0\hm<\alpha \hm<1$, $\mu \hm=0$ и~$\beta\hm =1$, то случайная величина~$X$ 
положительна с~вероятностью~1. В~дальнейшем будем говорить, что случайная
величина~$Y$ имеет стандартное $\alpha$-устой\-чи\-вое распределение, если 
$\mu \hm=0$ и~$\sigma \hm= 1$.

$\alpha$-устойчивое распределение является безгранично делимым. Оно порождает 
некоторый процесс Леви.

\smallskip

\noindent
\textbf{Определение~4.} 
Случайный процесс $L_{\alpha} \hm= (L_{\alpha} (t)$, $t\hm\geq 0)$ со значениями 
в~$R^1$ называется $\alpha$-устой\-чи\-вым движением Леви,
если это процесс Леви, для которого $L_{\alpha} (1)$ имеет заданное 
устойчивое распределение.


\smallskip

Если у распределения $L_{\alpha} (1)$ $0\hm<\alpha \hm<1$, $\beta \hm=1$, $\mu \hm=0$, 
то траектории процесса~$L_{\alpha}$ являются положительными и~неубывающими.
Такой процесс называется {\it $\alpha$-устой\-чи\-вым субординатором}.

Если $\alpha =2$, $\mu \hm=0$, то мы вновь возвращаемся к~процессу 
броуновского движения~$B$.

Между $\alpha$-устойчивыми движениями Леви с~различными~$\alpha$ справедливо 
следующее соотношение.

\smallskip

\noindent
\textbf{Теорема~1.}\
\textit{Если $(L_{\alpha_1 } (t), t\geq 0)$, $0\hm< \alpha_1 \hm\leq 2$, есть 
$\alpha_1$-устой\-чи\-вое движение Леви с~симметричными распределениями и
$(L_{\alpha_2 } (t), t\hm\geq 0)$, $0\hm<\alpha_2 \hm<1$, есть $\alpha_2$-устой\-чи\-вый 
субординатор, то случайный процесс
$Y \hm= (Y(t):= L_{\alpha_1 } (L_{\alpha_2} (t)), t\hm\geq 0)$
есть $\alpha_1 \alpha_2$-устой\-чи\-вое движение Леви с~симметричными распределениями}.


\smallskip

Эта теорема есть прямое следствие следующего результата 
В.\,М.~Золотарева~\cite[теорема~3.3.1.]{Zol83}:

\smallskip

\noindent
\textbf{Теорема~2.}\
\textit{Если $Y_1$ имеет симметричное $\alpha_1$-устой\-чи\-вое распределение, 
$0\hm<\alpha_1 \hm\leq 2$, $Y_2$ имеет одностороннее
$\alpha_2$-устой\-чи\-вое распределение, $0\hm<\alpha_2 \hm<1$, то случайная величина
$Y \hm= Y_1 Y_2^{1/\alpha_1}$ имеет симметричное $\alpha_1\alpha_2$-устой\-чи\-вое 
распределение}.


\smallskip

В частности, для $\alpha_1 \hm=2$ и~$0\hm<\alpha_2 \hm=\alpha/2 \hm<1$
получается следующий результат.

\smallskip

\noindent
\textbf{Теорема~3.}\
\textit{Если $B= (B(t), t\geq 0)$ есть броуновское движение,
$L_{\alpha/2} \hm= (L_{\alpha/2} (t) , t\hm\geq 0)$ есть $\alpha/2$-устой\-чи\-вый 
субординатор,
то $L_{\alpha} \hm= (L_{\alpha} (t) :=$\linebreak $:=\;B(L_{\alpha/2} (t)), t\hm\geq 0)$, 
$0\hm<\alpha \hm<2$, есть
$\alpha$-устой\-чи\-вое движение Леви с~симметричными распределениями}.


\smallskip

Рассмотрим теперь многомерные аналоги приведенных выше определений и~результатов.

\smallskip

\noindent
\textbf{Определение~5.}\
Процесс Леви $B = (B(t), t\geq 0)$ со значениями в~$R^d$ 
называется многомерным процессом броуновского движения (MBM~--- multivariate BM), если
для любых $t\hm\geq 0$, $h\hm>0$ приращение $B(t+h)\hm -B(t)$ имеет 
гауссовское распределение с~нулевым средним 
и~матрицей ковариаций $\Sigma h$, где $\Sigma$~--- некоторая 
положительно определенная мат\-рица.


\smallskip

Пусть $Y = (Y_1 , \ldots , Y_d )$ есть случайный вектор.

\smallskip

\noindent
\textbf{Определение~6.}\ Случайный вектор~$Y$ имеет многомерное $\alpha$-устой\-чи\-вое 
распределение
с параметром $\alpha\hm\in (0,2]$, если его характеристическая функция имеет следующий вид: для любого $\omega\in R^d$

\noindent
$$
\varphi_Y (\omega) = E(\exp (i (Y,\omega ))) = \exp (i (a,\omega ) - I(\omega ) )  \,,
$$
где
\begin{enumerate}[(1)]
\item если $\alpha =2$, то $a\hm\in R^d$~--- 
вектор средних, $I(\omega ) \hm= ({1}/{2}) (\Sigma\omega , \omega )$,
$\Sigma$~--- матрица ковариаций случайного вектора~$Y$;

\item если $0\hm<\alpha \hm<2$, то $a\hm\in R^d$ и
$$
I(\omega ) =    \int\limits_{S^{d-1}} |(\omega ,u)|^{\alpha} 
\theta_{\alpha} (\omega ,u) \Gamma (du)  \,,
$$
$\Gamma$~--- конечная мера на сфере $S^{d-1}$ и

\noindent
$$
\hspace*{-5mm}\theta_{\alpha} (\omega  , u ) =
 \begin{cases}
1 - i  \tan \left( \fr{\alpha \pi }{2} \right) \mathrm{sign}\, (\omega ,u),  &\! \alpha \not= 1 , \\
1 + \fr{2}{\pi} \ln |(\omega ,u) |  \mathrm{sign}\, (\omega ,u), &\! \alpha =1 .
\end{cases}
$$
\end{enumerate}

Такое распределение является безгранично
делимым и~порождает некоторый процесс Леви $Y\hm=(Y(t) , t\hm\geq 0)$ со значениями в~$R^d$. 
Будем называть его многомерным
$\alpha$-устой\-чи\-вым движением Леви. Для $\alpha \hm=2$ и~$a\hm=0$ получаем многомерное броуновское движение.

Важным частным случаем являются так называемые многомерные эллиптически 
контурированные устойчивые распределения, которые
кратко будем называть эллиптическими устойчивыми распределениями.
Их характеристическая функция имеет вид:
\begin{multline*}
\varphi_Y (\omega ) :={}\\
{}:=\; E(\exp (i (Y,\omega )) = \exp (i (a, \omega ) - 
(\Sigma\omega , \omega )^{\alpha /2} )  ,
\end{multline*}

\noindent
где $a\in R^d$, $\Sigma$~--- некоторая положительно определенная матрица (см., 
например,~\cite{No13}).



Справедлив следующий аналог теоремы~3.

\smallskip

\noindent
\textbf{Теорема~4.}\
\textit{Если $B= (B(t), t\geq 0)$ есть многомерное броуновское движение с~матрицей ковариаций $\Sigma$,
$L_{\alpha/2} \hm= (L_{\alpha/2} (t) , t\hm\geq 0)$ есть $\alpha/2$-устой\-чи\-вый субординатор,
то $L_{\alpha} \hm= (L_{\alpha} (t) := B(L_{\alpha/2} (t)), t\hm\geq 0)$, 
$0\hm<\alpha \hm<2$, есть
многомерное $\alpha$-устой\-чи\-вое движение Леви с~эллиптически 
контурированными распределениями.}

\section{Самоподобные процессы}


\smallskip

\noindent
\textbf{Определение~7.}\
Вещественный случайный процесс $Y \hm= (Y(t), t\hm\geq 0)$ называется самоподобным 
с~параметром Херста $H\hm\geq 0$, если он удовлетворяет
следующему условию:
$$
Y(t) \stackrel{d}{=} c^{-H} Y(ct)\,, \ \forall\, t\geq 0, \ \forall\, c>0,
$$
где $\stackrel{d}{=}$ обозначает равенство конечномерных распределений.


\smallskip

Двумя наиболее популярными примерами самоподобных процессов являются 
дробное броуновское движение (fBM~--- fractional BM) и~$\alpha$-устой\-чи\-вое движение \mbox{Леви}.

\smallskip

\noindent
\textbf{Определение~8.}\
Дробное броуновское движение с~параметром Херста $H$ есть гауссовский процесс
$(B_H (t), t\hm\geq 0)$ с~нулевыми средними и~ковариационной функцией
$$
K_H (t,s) = \fr{1}{2} \left[ |t|^{2H} + |s|^{2H} - |t-s|^{2H} \right] \, .
$$


\smallskip

При $H=1/2$ возвращаемся к~обычному броуновскому движению.

Определение $\alpha$-устой\-чи\-во\-го движения Леви было дано выше. Для него 
$H\hm=1/\alpha$.

Можно определить многомерный аналог самоподобного процесса.

\smallskip

\noindent
\textbf{Определение~9.}\
Случайный процесс $Y = (Y(t)\hm=(Y_1 (t), \ldots, Y_d (t) ) , t\hm\in R^1 )$ со значениями 
в~$R^d$ называется самоподобным
с параметром Херста $H \hm= (H_1 , \ldots , H_d )\hm\in (0,\infty )^d$, если он 
удовлетворяет следующему условию:
\begin{multline*}
Y(t) \stackrel{d}{=} c^{-H} Y(ct) = (c^{- H_1} Y(c t_1 ) , \ldots , 
c^{- H_d} Y(c t_d )), \\ \forall t\geq 0, \ \forall c>0\,.
\end{multline*}
%где $\stackrel{d}{=}$ обозначает равенство конечномерных распределений.

\smallskip

Возможны и~другие более общие определения. Дополнительную информацию об устойчивых и~самоподобных процессах
можно найти в~книгах~\cite{Sam94, Emb02}.

\section{Многомерное дробное броуновское движение}

Основой модели, которая предлагается в~данной статье, является
многомерное дробное броуновское движение. Далее приводится только
его определение. Более подробную информацию можно найти в~[3,~4].

\smallskip

\noindent
\textbf{Определение~10.}\
Многомерным дробным броуновским движением (MFBM~--- multivariate fBM) с~параметром Херста $H\hm\in (0,1)^d$ 
называется $d$-мер\-ный случайный процесс
$Y\hm = (Y_1 (t) , \ldots Y_d (t) , t\hm\in R^1 )$, который обладает следующими свойствами:
\begin{enumerate}[(1)]
\item  $Y$ есть гауссовский процесс;

\item  $Y$ является самоподобным с~параметром Херс\-та~$H$;

\item  $Y$ имеет однородные приращения.
\end{enumerate}


В данной работе наиболее интересен случай, когда $1/2 \hm< H_p \hm<1$ для 
всех $p: 1\hm\leq p \hm\leq d$. В~этой ситуации при дополнительном ограничении, что 
процесс~$Y$ является обратимым по времени, он имеет нулевые средние и~следующие ковариационные функции:
\begin{multline*}
E(Y_p (s) Y_q (t)) ={}\\
{}= \fr{\sigma_{pq}}{2}
\left[ |s|^{H_p + H_q} + |t|^{H_p + H_q} - |t-s|^{H_p + H_q} \right]  \,,
\end{multline*}
где $\Sigma = (\sigma_{pq} )$ есть некоторая положительно определенная матрица. 
В~частности,
\begin{equation*}
D(Y_p (t)) = E\left(|Y_p (t)|^2 \right) = \sigma_p^2 |t|^{2H_p}  \,.
\end{equation*}

Далее многомерное дробное броуновское движение с~параметром~$H$ будем обозначать, 
как и~в~одномерном случае,~$B_H$.

\section{Многомерное дробное движение Леви}

В этом разделе предлагается  некоторый вариант многомерного дробного движения 
Леви. Одномерный аналог этого процесса был определен
в~работе~\cite{Ni12}.

Пусть $(B_H (t), t\hm\in R^1 )$ есть многомерное дробное броуновское
движение с~параметром Херс\-та~$H$ и~мат\-ри\-цей ковариаций~$\Sigma$;
$(L_{\alpha}^1 (t), t\hm\geq 0)$ и~$(L_{\alpha}^2 (t), t\hm\geq 0)$~---
стандартные $\alpha$-устой\-чи\-вые субординаторы, $0\hm<\alpha \hm< 1$;
процессы~$B_H$, $L_{\alpha}^1$ и~$L_{\alpha}^2$ независимы.

\smallskip

\noindent
\textbf{Определение~11.}\
Многомерным дробным движением Леви называется случайный процесс $X\hm = (X(t) ,
 t\hm\in R^1 )$ со значениями в~$R^d$ такой, что
$$
X(t) := \begin{cases}
B_H (L_{\alpha}^1 (t)), & t\geq 0 \,; \\
B_H (L_{\alpha}^2 (-t)), &  t < 0  \,, \\
\end{cases}
$$


Покажем, что предложенный процесс обладает свойством самоподобия, а~именно имеет место следующий результат.

\smallskip

\noindent
\textbf{Теорема 5.}\
\textit{Построенный выше случайный процесс является самоподобным 
с~параметром Херста~$H/\alpha$}.

\smallskip

\noindent
Д\,о\,к\,а\,з\,а\,т\,е\,л\,ь\,с\,т\,в\,о\,.\ \
Процессы $(L_{\alpha}^k (t), t\hm\geq 0)$, $k\hm=1,2$, являются $\alpha$-устой\-чи\-вы\-ми 
и~самоподобными с~параметром Херста $1/\alpha$.
В~силу этого для любого $c\hm>0$  имеем:
$$
(L_{\alpha}^k (ct), t\geq 0) \stackrel{d}{=}
(c^{1/\alpha} L_{\alpha}^k (t), t\geq 0) .
$$
Тогда
\begin{multline*}
(X(ct) , t\in R^1 ) = B_H (\pm L_{\alpha}^k (c|t|), t\in R^1 ) \stackrel{d}{=}{}\\
{}\stackrel{d}{=}
(B_H (\pm c^{1/\alpha} L_{\alpha}^k (|t|) , t\in R^1 )  \,.
\end{multline*}
Используя самоподобие процесса~$B_H$ для фиксированного
$\tau \hm= L_{\alpha}^k (|t|)$, имеем для любого $a\hm>0$
$$
(B_H (a\tau ), \tau\geq 0) \stackrel{d}{=} (a^H  B_H (\tau),
\tau\geq 0)\,,
$$
или
$$
(B_H (\pm c^{1/\alpha} \tau ), \tau\geq 0) \stackrel{d}{=}
(c^{H/\alpha} B_H (\pm \tau), \tau\geq 0)  \,.
$$
Применяя формулу полной вероятности, получаем нужный результат.

Используя последнюю теорему, можно получить следующий полезный 
для дальнейших приложений результат.

\smallskip

\noindent
\textbf{Следствие~1.}\
Для любого $t\hm>0$
$$
X(t) \stackrel{d}{=} ((L_{\alpha}^1 (t))^{H_1}  Y_1 , \ldots , (L_{\alpha}^1 (t))^{H_d}  Y_d ) ,
$$
где случайный вектор $Y \hm= (Y_1 , \ldots , Y_d )$ имеет многомерное 
нормальное распределение со средним ноль и~матрицей ковариаций~$\Sigma$,
причем $L_{\alpha}^1 (t)$ и~$Y$ независимы.


\smallskip


Этот результат легко следует из определения многомерного дробного 
броуновского движения и~независимости~$B_H$ и~$L_{\alpha}$.

\smallskip

\noindent
\textbf{Следствие~2.}\
 Если $B$ есть многомерное дробное броуновское движение с~матрицей ковариаций~$\Sigma$, 
 то
$H_1 \hm= \cdots \hm= H_d \hm= 1/2$ и~для любого $t\hm>0$ случайный вектор
$$
X(t) \stackrel{d}{=} (L_{\alpha}^1 (t))^{1/2} (Y_1 , \ldots , Y_d )
$$
имеет многомерное $\alpha$-устой\-чи\-вое эллиптически контурированное распределение 
с~матрицей ковариаций~$\Sigma$.


\smallskip

Этот результат есть прямое следствие теоремы~4.

В случае дробного броуновского движения случайный вектор $X(t)$ имеет распределение, отличное от устойчивого, если все $H_p >1/2$. Для доказательства
достаточно рассмотреть одну из компонент этого вектора. Как отмечалось выше, случайная величина $(L_{\alpha} (t) )^{1/2} B(1)$ имеет симметричное
устойчивое распределение с~характеристическим показателем~$\alpha/2$. Но случайные величины $(L_{\alpha} (t) )^{1/2} B(1)$ и~$(L_{\alpha} (t) )^{H_1 } B(1)$
имеют разные распределения, если $H_1 \not= 1/2$.


\smallskip

\noindent
\textbf{Замечание~1.}\ 
Параметр Херста $H/\alpha$ определенного выше процесса~$X$ может быть любым положительным числом. Будем предполагать,
что $1/2 \hm< H_k /\alpha \hm<1$, $k\hm= 1,\ldots ,d$. 
В~этом случае рассматриваемый процесс имеет конечные математические ожидания 
и~обладает свойством долговременной зависимости.


\smallskip




\noindent
\textbf{Теорема 6.}\
\textit{Определенный выше случайный процесс~$X$ имеет однородные по времени приращения.}


\smallskip

\noindent
Д\,о\,к\,а\,з\,а\,т\,е\,л\,ь\,с\,т\,в\,о\,.\ \ 
Хорошо известно, что дробное броуновское движение~$B_H$ имеет однородные приращения. 
Более того, для любых $0\hm < t_1 \hm< t_2 $
$$
B_H \left(t_2 \right) - B_H \left(t_1 \right) \stackrel{d}{=} B_H \left(t_2 - t_1 \right)\,. 
$$
Тогда для любых $t\hm\geq 0$, $h\hm>0$ и~фиксированных $L_{\alpha}^k (t+h) \hm= 
t_2$, $L_{\alpha}^k (t) \hm= t_1$ имеем:
$$
B_H (L_{\alpha}^k (t+h)) - B_H (L_{\alpha}^k (t)) \stackrel{d}{=}
B_H (L_{\alpha}^k (t+h) -  L_{\alpha}^k (t))\,.
$$
В силу формулы полной вероятности имеем то же самое и~для случайных моментов времени.
Процесс $L_{\alpha}^k (t)$ также имеет однородные приращения. В~итоге получаем:
\begin{multline*}
B_H \left(L_{\alpha}^k (t+h)\right) - B_H \left(L_{\alpha}^k (t)\right) \stackrel{d}{=}{}\\
{}\stackrel{d}{=}
B_H \left(L_{\alpha}^k (t+h) -  L_{\alpha}^k (t)\right)
\stackrel{d}{=}
B_H \left(L_{\alpha}^k (h) \right)  \,.
\end{multline*}


\section{Асимптотика хвостов распределений многомерного дробного движения Леви}

В этом разделе изучается поведение хвостов одномерных сечений 
дробного движения Леви, построенного выше.


В силу свойства самоподобия достаточно рассмотреть только распределение 
случайного вектора~$X(1)$. Рассмотрим случайную величину
$Z_{\alpha} :=$\linebreak $:=\;L_{\alpha}^1$. Без ограничения общности можно считать, 
что она имеет стандартное одностороннее
$\alpha$-устой\-чи\-вое распределение. В~силу следствия~2 имеем:
$$
X(1) \stackrel{d}{=} ((Z_{\alpha})^{H_1}  Y_1 , \ldots , (Z_{\alpha})^{H_d}  Y_d ) \, ,
$$
где случайный вектор $Y \hm= (Y_1 , \ldots , Y_d )$ имеет многомерное нормальное распределение со средним ноль и~матрицей ковариаций $\Sigma$,
причем~$Z_{\alpha}$ и~$Y$ независимы.
Покажем, что этот вектор имеет распределения, отличные от устойчивых. Для этого достаточно проверить, что это верно для
отдельно взятой координаты.

Предположим противное: первая координата $V:= (Z_{\alpha} )^{H_1} Y_1$ имеет симметричное устойчивое распределение
с~показателем $0\hm<\alpha_1 \hm<2$. Тогда, как показано выше, существует такая независимая от $Y_1$ случайная величина $S$, которая
имеет одностороннее устойчивое распределение с~показателем $\alpha_1/2$, что
$$
V \stackrel{d}{=} S^{1/2} Y_1 \, .
$$
Отсюда получаем:
$$
(Z_{\alpha} )^{H_1} Y_1 \stackrel{d}{=} S^{1/2} Y_1\,  .
$$
Нормальное распределение обладает свойством идентифицируемости для масштабных смесей. 
В~силу этого получаем
$$
(Z_{\alpha} )^{H_1} \stackrel{d}{=} S^{1/2},
$$
или
$$
(Z_{\alpha} )^{2 H_1} \stackrel{d}{=} S  \,.
$$
Но, как хорошо известно, если случайная величина~$Z_{\alpha}$ имеет устойчивое распределение, то случайная 
величина~$S$ имеет распределение, отличное от устойчивого. Это противоречие 
доказывает нужное утверждение.

Тем не менее можно показать, что хвосты распределений координат
вектора $X(1)$ ведут себя в~точности так же, как хвосты устойчивых
распределений. Для доказательства этого свойства потребуется
результат, известный как теорема Бреймана~\cite{Br65}.

\smallskip

\noindent
\textbf{Теорема 7.}\
\textit{Пусть $X$ и~$Y$ есть независимые неотрицательные случайные величины~и
$$
\bar{F} (x) := P(X>x) = x^{-\alpha}  L(x) \,, \enskip x\to\infty \,,
$$
где $\alpha\hm >0$, $L$~--- медленно меняющаяся функция, 
$E(Y^{\alpha + \varepsilon} ) \hm< \infty$ для некоторого
$\varepsilon \hm> 0$. Тогда при больших} $x\hm>0$
$$
\bar{H} (x) := P(XY >x) \sim E(Y^{\alpha} ) \bar{F} (x)  \,.
$$


Применим этот результат к~распределению $k$-й координаты вектора $X(1)$.
Если $Z_{\alpha}$ имеет стандартное $\alpha$-устой\-чи\-вое распределение, то для больших 
$x\hm>0$
$$
 P(Z_{\alpha} > x ) \sim C(\alpha ) x^{-\alpha}  \,,
$$
где
$$
C(\alpha ) = \fr{\sin (\pi\alpha )}{\pi} \Gamma (\alpha )
$$
(см.~\cite[теорема~2.4.1]{IL65}). Тогда, используя теорему
Бреймана, для больших $x\hm>0$ получаем:
\begin{multline*}
P((Z_{\alpha} )^{H_k} Y_k >x) = \fr{1}{2} P\left( (Z_{\alpha} )^{H_k} |Y_k | >x \right) =
{}\\
{}=
\fr{1}{2} P\left(Z_{\alpha}  |Y_k |^{1/H_k} >x^{1/H_k} \right)\sim{}\\
{}
\sim \fr{1}{2} E\left(|Y_k |^{\alpha/H_k} \right)   
P\left( Z_{\alpha}  >x^{1/H_k} \right) \sim{}\\
{}\sim
\fr{1}{2} C(\alpha ) E\left(|Y_k |^{\alpha/H_k} \right)  x^{-\alpha/H_k } ={}\\
{}=
\fr{1}{2} C(\alpha ) E\left(|Y_k |^{1/H_k^1} \right)  x^{- 1/H_k^1 }  \,.
\end{multline*}


\section{Приложение к~моделированию телетрафика}

В этом разделе построенный выше процесс применяется для
моделирования динамики нагрузки сервера, поступающей по нескольким
каналам, и~делается  нижняя оценка для вероятности переполнения хотя
бы одного из буферов при больших размерах всех буферов.

Пусть имеется система массового обслуживания с~одним сервером, на которую подается нагрузка по нескольким каналам, которые, вообще говоря, зависимы.
Величина нагрузки, поступившей по $k$-му каналу, определяется по правилу:
$$
A_k (t) := m_k t + \left(\sigma_k m_k \right)^{1/\beta_k}  X_k (t), \enskip
k=1,\ldots , d\,.
$$
Векторный случайный процесс $X(t) \hm= (X_1 (t) , \ldots , X_d (t))$
есть дробное движение Леви с~параметром Херста $H^1 \hm= (H_1^1 ,
\ldots , H_d^1 ) := H/\alpha \hm= (H_1 /\alpha , \ldots , H_d /\alpha
)$ и~матрицей $\Sigma \hm= (\sigma_{pq} )$, определенное выше,
$\sigma_k^2 \hm=1$, $\beta_k \hm=  \alpha/H_k $, $k\hm=1, \ldots , d$. Если
обслуживание пакетов осуществляется путем случайного выбора канала 
с~некоторой ве\-ро\-ят\-ностью, то естественно предположить, что скорость
обслуживания нагрузки из $k$-го канала постоянна и~равна~$r_k$.
Чтобы обеспечить устойчивость функционирования системы, предположим,
как обычно, что $r_k\hm > m_k$, $k\hm=1, \ldots , d$. Тогда величина
загрузки $k$-го буфера в~момент времени $t\hm\in R^1$ можно записать в~виде:
$$
Q_k (t,r_k ) = \sup\limits_{s\leq t} (A_k (t) - A_k (s) - r_k 
(t-s))_{+}\,.
$$



В силу теоремы~6 процессы $Q_k \hm= (Q_k (t,r), t\hm\in R^1)$ являются стационарными. 
Пусть~$b_k$ есть размер $k$-го буфера.

Обозначим $b=(b_1 , \ldots , b_d )$. Найдем оценку для следующей вероятности 
переполнения хотя бы одного из буферов:
\begin{multline*}
\varepsilon (b) = P\left(\bigcup\limits_{k=1}^d \left(Q_k (0, r) > b_k \right)\right) ={}\\
{}=
P\left(\bigcup\limits_{k=1}^d \left(\sup\limits_{\tau \geq 0} (A_k (\tau ) - 
r_k  \tau ) >b_k \right)\right)\, .
\end{multline*}
Далее получим нижнюю границу для этой вероятности при больших~$b_k$,
используя технику, развитую в~работах~[1, 10]. Аналогичный результат
в одномерном случае был получен ранее в~работе~\cite{Ni12}.


Используя определение входящего потока и~свойство самоподобия, получаем:
\begin{multline*}
\varepsilon (b) = P\left(\bigcup\limits_{k=1}^d
\left(\sup\limits_{\tau \geq 0} \left(A_k (\tau ) - r_k  \tau \right) >b_k
\right)\right) \geq{}
\\
{}\geq \sup\limits_{\tau \geq 0} P\left(\bigcup\limits_{k=1}^d \left(
\left(A_k (\tau ) - r_k  \tau \right) >b_k \right)\right)= {}
\\
{}= \sup\limits_{\tau\geq 0} P\left(\bigcup\limits_{k=1}^d \left( m_k
\tau +\left(\sigma_k m_k \right)^{1/\beta_k } X_k (\tau ) - r_k \tau  >{}\right.\right.\\
\left.\left.{}> b_k
\vphantom{\left(\sigma_k m_k \right)^{1/\beta_k }}
\right) 
\vphantom{\bigcup\limits_{k=1}^d}
\right) =
\\
{}= \sup\limits_{\tau\geq 0} P\left(\bigcup\limits_{k=1}^d \left( m_k
\tau +\left(\sigma_k m_k \tau \right)^{1/\beta_k } X_k (1) - r_k \tau  >{}\right.\right.\\ 
\hspace*{-7pt}\left.\left.{}>b_k
\right)\right) 
= \sup\limits_{\tau\geq 0} P\left(\bigcup\limits_{k=1}^d \left(  X_k
(1) > \fr{b_k + (r_k - m_k )\tau}{(\sigma_k m_k \tau )^{1/\beta_k
} } \right)\!\right).\hspace*{-7.08pt}
\end{multline*}
Обозначим
$$
f_k (\tau ) =  \fr{b_k + (r_k - m_k )\tau}{(\sigma_k m_k \tau )^{1/\beta_k } } \,.
$$
Используя следствие~1, получаем:
\begin{multline*}
\varepsilon (b) \geq \sup\limits_{\tau\geq 0} P\left(
\bigcup\limits_{k=1}^d \left(  
\left( L_{\alpha}^1 (1) \right)^{H_k}  Y_k  > f_k (\tau ) \right)\right) \geq{}
\\
\hspace*{-4pt}{}\geq \sup\limits_{\tau\geq 0} P\left(\bigcup\limits_{k=1}^d \left(  
\vphantom{\bigcap\limits_{k=1}^d}
\left( L_{\alpha}^1 (1) \right)^{H_k}  Y_k  > f_k (\tau ) , \right.\right.
\end{multline*}

\noindent
\begin{multline*}
\left.\left.\bigcap\limits_{k=1}^d (Y_k > 1) \right)\right) \geq {}\\
{}\geq
\sup\limits_{\tau\geq 0} P\left(\bigcup\limits_{k=1}^d \left(  
\left( L_{\alpha}^1 (1) \right)^{H_k}  > f_k (\tau ) , \right.\right.\\
\left.\left.\bigcap\limits_{k=1}^d \left(Y_k > 1\right) \right)\right) \geq{}
\\
{}\geq \sup\limits_{\tau\geq 0} P\left(\bigcup\limits_{k=1}^d \left( 
\left ( L_{\alpha}^1 (1) \right)^{H_k}  > f_k (\tau ) \right)\right)\times{}
\\
{}\times
P\left( \bigcap\limits_{k=1}^d (Y_k > 1)  \right) ={}
\\
{}= C(\Sigma )  \sup\limits_{\tau\geq 0} P\left(\bigcup\limits_{k=1}^d \left(   
L_{\alpha}^1 (1)   > [f_k (\tau ) ]^{1/H_k }  \right)\right) ={}\\
{}= 
C(\Sigma ) \sup\limits_{\tau\geq 0} P\left( L_{\alpha}^1 (1)
> \min\limits_k [f_k (\tau ) ]^{1/H_k }  \right).
\end{multline*}
Последняя вероятность есть невозрастающая функция от $f_k (\tau )$,
которая в~точке
$$
\tau_k = \fr{b_k H_k^1}{(1-H_k^1 )(r_k - m_k )}
$$
принимает наименьшее значение, равное
$$
d_k := f_k \left(\tau_k \right) 
\fr{(r_k - m_k )^{H_k^1} (1-H_k^1 )^{-(1-H_k^1 )} }
{(\sigma_k m_k H_k^1 )^{H_k^1} } \, b_k^{1-H_k^1} \,.
$$
Так как
\begin{multline*}
\sup\limits_{\tau\geq 0} \min\limits_k \left[f_k (\tau )\right]^{1/H_k} \leq
\min\limits_k \sup\limits_{\tau\geq 0} 
\left [f_k (\tau )\right]^{1/H_k} ={}\\
{}= \min\limits_k \left[d_k \right]^{1/H_k } \,,
\end{multline*}
то получаем
$$
\varepsilon (b) \geq C(\Sigma )  P\left( L_{\alpha}^1 (1)   > \min\limits_k [d_k ]^{1/H_k }  \right)  .
$$



Если $L_{\alpha}^1 (1)$ имеет стандартное $\alpha$-устой\-чи\-вое распределение, 
то для больших $x\hm>0$
$$
 P(L_{\alpha}^1 (1) > x ) \sim C(\alpha ) x^{-\alpha} \,,
$$
где
$$
C(\alpha ) = \fr{\sin (\pi\alpha )}{\pi} \Gamma (\alpha ) 
$$
(см.\ \cite[теорема~2.4.1]{IL65}.) Отсюда для больших~$b_k$
$$
\varepsilon (b) \geq C(\alpha , H, \Sigma )  \max\limits_k \left[ \sigma_k 
\fr{m_k}{r_k - m_k} b_k^{-(1-H_k^1 )/H_k^1 } \right] .
$$
Сформулируем окончательный результат в~виде следующей теоремы.

\smallskip

\noindent
\textbf{Теорема~8.}\
\textit{В рамках описанной выше модели асимптотическая нижняя граница для вероятности переполнения хотя бы одного из буферов
имеет следующий вид}:
$$
\varepsilon (b) \geq C(\alpha , H, \Sigma ) 
\max\limits_k \left[ \sigma_k \frac{m_k}{r_k - m_k} 
b_k^{-(1-H_k^1 )/H_k^1 } \right] ,
$$
\textit{если} $b_k\hm\to\infty$, $1\hm\leq k\hm\leq d$.

{\small\frenchspacing
 {%\baselineskip=10.8pt
 \addcontentsline{toc}{section}{References}
 \begin{thebibliography}{99}
\bibitem{Las02}
\Au{Laskin N., Lambadaris~I., Harmantzis~F.\,C., Devetsikiotis~M.}
Fractional Levy motion and its application to network traffic
modeling~// Computor Networks, 2002. Vol.~40. P.~363--375.

\bibitem{Ni12}
\Au{De Nikola C., Khokhlov~Yu.\,S., Pagano~M., Sidorova~O.\,I.}
Fractional Levy motion with dependent increments and its application
to network traffic modeling~// Информатика и~её применения, 2012.
Т.~6. Вып.~3. С.~58--63.


\bibitem{Sto06} %3
\Au{Stoev S., Taqqu M.} How rich is the class of multifractional
brownian motions?~// Stochastic Processes Their Applications,
2006. Vol.~116. P.~200--221.


\bibitem{Amb12} %4
\Au{Amblard P.\,O., Coeurjolly~J.\,F., Lavancier~F., Philippe~A.}
Basic properties of the multivariate fractional Brownian motion~//
Bull. Society Mathematique de France, Seminaires et Congres,
2012. Vol.~28. P.~65--87.



\bibitem{Zol83} %5
\Au{Золотарев B.\,М.} Одномерные устойчивые распределения.~--- М.:
Наука, 1983. 304~с.

\bibitem{No13}
\Au{Nolan J.\,P.}  Multivariate elliptically contoured stable
distributions: Theory and estimation~//
 Comput. Stat., 2013. Vol.~28. No.\,5. P.~2067--2089.

\bibitem{Sam94}
\Au{Samorodnitsky G., Taqqu~M.\,S.} Stable non-Gaussian random
processes.~--- London: Chapman \& Hall, 1994. 632~p.

\bibitem{Emb02}
\Au{Embrechts~P., Maejima~M.} Selfsimilar process.~--- Princeton,
NJ, USA: Princeton University Press, 2002. 111~p.

\bibitem{Br65}
\Au{Breiman L.} On some limit theorems similar to the arc-sin law~// 
Теория вероятн. и~ее примен., 1965. Т.~10. Вып.~2. С.~351--359.

\bibitem{Nor94} %10
\Au{Norros I.} A~storage model with self-similar input~// Queuing
Syst., 1994. Vol.~16. P.~387--396.

\bibitem{IL65}
\Au{Ибрагимов И.\,А., Линник~Ю.\,В.} Независимые и~стационарно
связанные величины.~--- М.: Наука, 1965.  524~с.
\end{thebibliography}

 }
 }

\end{multicols}

\vspace*{-3pt}

\hfill{\small\textit{Поступила в~редакцию 01.12.15}}

\vspace*{8pt}

%\newpage

%\vspace*{-24pt}

\hrule

\vspace*{2pt}

\hrule

%\vspace*{8pt}



\def\tit{MULTIVARIATE FRACTIONAL LEVY MOTION AND~ITS~APPLICATIONS}

\def\titkol{Multivariate fractional Levy motion and its applications}

\def\aut{Yu.\,S.~Khokhlov}

\def\autkol{Yu.\,S.~Khokhlov}

\titel{\tit}{\aut}{\autkol}{\titkol}

\vspace*{-9pt}

\noindent
Department of Mathematical 
Statistics, Faculty of Computational Mathematics and Cybernetics, 
M.\,V.~Lomonosov Moscow State University, 1-52~Leninskiye Gory, GSP-1, Moscow 119991, 
Russian Federation


\def\leftfootline{\small{\textbf{\thepage}
\hfill INFORMATIKA I EE PRIMENENIYA~--- INFORMATICS AND
APPLICATIONS\ \ \ 2016\ \ \ volume~10\ \ \ issue\ 2}
}%
 \def\rightfootline{\small{INFORMATIKA I EE PRIMENENIYA~---
INFORMATICS AND APPLICATIONS\ \ \ 2016\ \ \ volume~10\ \ \ issue\ 2
\hfill \textbf{\thepage}}}

\vspace*{3pt}



\Abste{Since the beginning of the 1990s, many empirical studies of real telecoomunication systems traffic have been conducted. 
It was found that traffic has some specific properties, which are different from 
common voice traffic, namely, it has the properties of self-similarity and long-range dependence and the distribution of loading size from 
one source has heavy tails. Some new models have been constructed,
 where these features were captured. Brownian fractional 
motion and $\alpha$-stable Levy motion are the well-known examples. 
But both of these models do not have all of the above properties. 
More complicated models have been proposed using some combination of these ones. 
In particular, the authors have proposed 
a~variant of univariate fractional Levy motion. 
%
This paper considers a~multivariate analog of fractional Levy motion. This process is multivariate fractional 
Brownian motion with random change of time, where random change of time is Levy motion with one-sided stable 
distributions. The properties of this process are investigated 
and it is proven that it is self-similar and has stationary 
increments. Next, it is shown that the coordinates of one-dimensional sections of this process have the 
distributions which 
are not stable. But asymptotic of tails for these distributions is the same 
as for the stable ones. 
This model is applied 
to analyze heterogeneous traffic and to get a~lower asymptotic bound of the probability 
of overflow of at least one 
buffer. There are other possible applications.}


\KWE{fractional Brownian motion; $\alpha$-stable subordinator; self-similar processes; buffer overflow probability}

\DOI{10.14357/19922264160212}

\vspace*{-12pt}

\Ack
\noindent
The work was supported by the Russian Science Foundation (project 14-11-00364).


%\vspace*{3pt}

  \begin{multicols}{2}

\renewcommand{\bibname}{\protect\rmfamily References}
%\renewcommand{\bibname}{\large\protect\rm References}

{\small\frenchspacing
 {%\baselineskip=10.8pt
 \addcontentsline{toc}{section}{References}
 \begin{thebibliography}{99}
\bibitem{Las02-1}
\Aue{Laskin, N., I.~Lambadaris, F.\,C.~Harmantzis, and M.~Devetsikiotis}. 
2002. Fractional Levy motion and its application 
to network traffic modeling.  \textit{Computor Networks} 40:363--375. 

\bibitem{Ni12-1}
\Aue{De Nikola, C.,  Y.\,S.~Khokhlov, M.~Pagano, and O.\,I.~Sidorova}.  
2012. Fractional Levy 
motion with dependent increments and its application to network traffic modeling.
\textit{Informatika i~ee Primeneniya}~--- \textit{Inform. Appl.} 6(3):58--63.  


\bibitem{Sto06-1}
\Aue{Stoev, S., and M.~Taqqu}. 2006. How rich is the class of multifractional brownian
motions?  \textit{Stochastic Processes Their Applications}  116:200--221.

\bibitem{Amb12-1}
\Aue{Amblard, P.\,O., J.\,F.~Coeurjolly, F.~Lavancier, and A.~Philippe}.  2012. 
Basic properties of the multivariate fractional Brownian motion. 
\textit{Bull. Society Mathematique de France, Seminaires et Congres}. 28:65--87. 




\bibitem{Zol83-1} %%
\Aue{Zolotarev, V.\,M.} 1986. \textit{One-dimensional stable distributions}.  
Translations of Mathematical Monographs. AMS. Vol.~65. 284~p. 


\bibitem{No13-1}
\Aue{Nolan, J.\,P.} 2013.  Multivariate elliptically contoured stable distributions: 
Theory and estimation.   
\textit{Comput. Stat.} 28(5):2067--2089. 

\bibitem{Sam94-1}
\Aue{Samorodnitsky, G., and M.\,S.~Taqqu}. 1994. 
\textit{Stable non-Gaussian random processes}. 
Chapman \& Hall. 632~p.  

\bibitem{Emb02-1}
\Aue{Embrechts,~P., and M.~Maejima}. 2002.  \textit{Selfsimilar process}. 
Prinston University Press. 111~p. 

\bibitem{Br65-1}
\Aue{Breiman, L.} 1965.
On some limit theorems similar to the arc-sin law. 
\textit{Teoriya Veroyatnostei i Primeneniya} 
[Theory of Probability and its Applications] 10(2):351--359. 

\bibitem{Nor94-1} %10
\Aue{Norros, I.}  1994. 
A~storage model with self-similar input. \textit{Queuing Syst.} 16:387--396. 

\bibitem{IL71-1}
\Aue{Ibragimov, I.\,A., and Yu.\,V.~Linnik}. 1971.  
\textit{Independent and stationary sequences of 
random variables}.  Wolters-Noordhoff, Gronengen. 443~p. 
  \end{thebibliography}

 }
 }

\end{multicols}

\vspace*{-3pt}

\hfill{\small\textit{Received December 1, 2015}}


\Contrl

\noindent
\textbf{Khokhlov Yury S.} (b.\ 1952)~---
Doctor of Science in physics and mathematics, professor, Department of Mathematical 
Statistics, Faculty of Computational Mathematics and Cybernetics, 
M.\,V.~Lomonosov Moscow State University, 1-52~Leninskiye Gory, GSP-1, Moscow 119991, 
Russian Federation; \mbox{yskhokhlov@yandex.ru}


\label{end\stat}


\renewcommand{\bibname}{\protect\rm Литература} %12
\def\stat{minin}

\def\tit{ИНТЕНСИВНОСТЬ ЦИТИРОВАНИЯ НАУЧНЫХ ПУБЛИКАЦИЙ В~ИЗОБРЕТЕНИЯХ 
ПО~ИНФОРМАЦИОННО-КОМПЬЮТЕРНЫМ ТЕХНОЛОГИЯМ, 
ПАТЕНТУЕМЫХ В~РОССИИ ОТЕЧЕСТВЕННЫМИ И~ЗАРУБЕЖНЫМИ ЗАЯВИТЕЛЯМИ$^*$}

\def\titkol{Интенсивность цитирования научных публикаций в~изобретениях по ИКТ, 
патентуемых в~России} %РФ отечественными и~зарубежными заявителями}

\def\aut{В.\,А.~Минин$^1$, И.\,М.~Зацман$^2$, В.\,А.~Хавансков$^3$, 
С.\,К.~Шубников$^4$}

\def\autkol{В.\,А.~Минин, И.\,М.~Зацман, В.\,А.~Хавансков, 
С.\,К.~Шубников}

\titel{\tit}{\aut}{\autkol}{\titkol}

\index{Минин В.\,А.}
\index{Зацман И.\,М.}
\index{Хавансков В.\,А.}
\index{Шубников С.\,К.}
\index{Minin V.\,A.}
\index{Zatsman I.\,M.}
\index{Havanskov V.\,A.}
\index{Shubnikov S.\,K.}

{\renewcommand{\thefootnote}{\fnsymbol{footnote}} \footnotetext[1]
{Работа выполнена при финансовой поддержке РФФИ (проект 16-07-00075).}}


\renewcommand{\thefootnote}{\arabic{footnote}}
\footnotetext[1]{Институт проблем информатики Федерального исследовательского
центра <<Информатика и~управление>> Российской академии наук, aleksisss@ya.ru}
\footnotetext[2]{Институт проблем информатики Федерального исследовательского
центра <<Информатика и~управление>> Российской академии наук, iz\_ipi@a170.ipi.ac.ru}
\footnotetext[3]{Институт проблем информатики Федерального исследовательского
центра <<Информатика и~управление>> Российской академии наук, havanskov@a170.ipi.ac.ru}
\footnotetext[4]{Институт проблем информатики Федерального исследовательского
центра <<Информатика и~управление>> Российской академии наук, sergeysh50@yandex.ru}

     
     
      
      \Abst{Рассматриваются информационные взаимосвязи науки и~технологий, а~также 
методы индикаторного оценивания процессов переноса (трансфера) знаний из разных областей 
исследований в~сферу технологического развития. Предлагаемые методы предназначены для 
определения значений индикатора интенсивности цитирования научных работ в~описаниях 
изобретений, патентуемых в~России отечественными и~зарубежными заявителями. Подобный 
подход может использоваться для получения косвенных оценок инновационного потенциала 
направлений научных исследований (ННИ). Значения индикатора интенсивности вычислялись как 
в~целом, так и~с распределением по странам заявителей. Представлены результаты определения 
значений индикатора, для чего в~качестве исходной информации использовались 
полнотекстовые описания изобретений по классу G06 Международной патентной 
классификации  (МПК,
англ.\  
\textit{International Patent Classification}~--- IPC) 
(Обработка данных; вычисления; счет), опуб\-ли\-ко\-ван\-ные Роспатентом 
      в~2000--2012~гг. Использование информационных ресурсов Роспатента было 
обусловлено тем, что они представлены в~электронном виде, т.\,е.\ доступны для 
автоматизированной обработки. В~результате получены значения индикатора интенсивности 
цитирования (ИЦ) научных работ с~разделением по отечественным, зарубежным и~совместным 
изобретениям, запатентованным в~РФ. Такая детализация позволила оценить активность 
международного технологического сотрудничества и~совместного патентования изобретений 
по ин\-фор\-ма\-ци\-он\-но-компью\-тер\-ным технологиям (ИКТ) в~России, а~так\-же определить тематику 
сотрудничества в~этой области.}
      
      \KW{цитирование научных работ; интенсивность цитирования; взаимосвязи науки 
и~технологий; информационные технологии; Международная патентная классификация; расчет 
значений индикатора интенсивности цитирования}

 \DOI{10.14357/19922264160213} 

%\vspace*{-4pt}

\vskip 10pt plus 9pt minus 6pt

\thispagestyle{headings}

\begin{multicols}{2}

\label{st\stat}

\section{Введение}
     
  Среди различных видов источников знаний, стимулирующих появление новых 
инженерных идей, научные пуб\-ли\-ка\-ции выделяются тем, что они открыты 
и~доступны для использования~[1]. Доля изобретений, в~которых цитируются 
именно научные пуб\-ли\-ка\-ции, зависит от вида технологий. Тематически каждый 
вид технологий описывается, как правило, в~виде списка рубрик 
МПК. Международная патентная классификация~--- иерархическая система 
патентной классификации, которая является средством для рубрицирования 
патентных документов (описаний изобретений, промышленных образцов, 
полезных моделей) единообразно в~международном масштабе. 

\begin{table*}\small
\begin{center}
\Caption{Коды МПК технологий и~значения индикатора ИИЦ}
      \vspace*{2ex}
      
      \begin{tabular}{|l|c|c|}
      \hline
\multicolumn{1}{|c|}{Вид технологий}&Коды МПК вида 
технологий&ИИЦ\\
     \hline
Биотехнологии&C07G; C12M, N, P, Q, R, S&138,43\hphantom{9}\\
Фармацевтические&A61K&83,71\\
Полупроводниковые&H01L&56,44\\
Оптические&G02; G03B, C, D, F, G, H; H01S&21,89\\
Информационные&G06; G11C; G10L&20,39\\
  \hline
  \end{tabular}
  \end{center}
  \vspace*{-3pt}
  \end{table*}
  
  Наиболее часто для такого описания исполь\-зуются списки рубрик МПК из 
номенклатуры,\linebreak раз\-работанной Фраунгоферовским институтом сис\-темотехники 
и~инновационных исследований\linebreak  (Fraunhofer Gesellschaft-Institute 
f$\ddot{\mbox{u}}$r Systemtechnik und Innovationsforschung~--- FhG-ISI). 

Примеры 
списков рубрик МПК из номенклатуры FhG-ISI приведены в~табл.~1 для пяти 
видов технологий. Последний столбец содержит значения индикатора интегральной 
ИЦ (ИИЦ) результатов научных исследований, которые 
связаны с~развитием технологий, указанных в~первом столбце. Значения 
индикатора ИИЦ определены в~табл.~1 как число цитируемых научных 
пуб\-ли\-ка\-ций на~100~описаний изобретений~[2].
  

  
  Заметим, что индикатор ИИЦ не зависит от того или иного деления всей 
системы знаний на отрасли науки и~научные направления, так как учитываются 
все цитируемые в~изобретениях научные пуб\-ли\-ка\-ции по всем отраслям науки. 
Одна из наиболее актуальных задач в~широком спектре исследований 
информационных взаимосвязей науки и~технологий состоит в~вычислении 
распределения интенсивности цитирования с~учетом классификации пуб\-ли\-ка\-ций 
по конкретным научным направлениям или дисциплинам.
  
  Иначе говоря, кроме индикатора ИИЦ необходимо вычислить значения 
индикатора ИЦ для конкретных областей знаний, 
научных направлений или дисциплин. Значения индикатора ИЦ будут зависеть от 
выбора конкретной классификации отраслей знаний (Государственный рубрикатор 
на\-уч\-но-тех\-ни\-че\-ской 
информации (\mbox{ГРНТИ}), рубрикатор Российского фонда фундаментальных 
исследований (РФФИ) и~др.).
  
  Исследования по индикаторному оцениванию информационных взаимосвязей 
науки и~технологий проводятся за рубежом с~конца прошлого века~[3--12]. 
В~России аналогичные работы появились в~начале этого века и~были выполнены 
в основном в~Институте проблем информатики Российской
академии наук (ИПИ РАН)~[13--19]. 
Разработка методов вычисления значений индикатора ИЦ по конкретным 
научным направлениям выполнялась в~более общем контексте проблематики 
информационного мониторинга в~сфере науки~[20--26].
  
  В качестве индикатора ИЦ традиционно используется число цитирований 
научных пуб\-ли\-ка\-ций, упоминаемых в~описаниях изобретений.\linebreak Данный индикатор 
характеризует уровень (частоту) использования научных результатов и~тем самым 
степень воздействия фундаментальной науки на развитие технологической сферы.
  
  Результаты оценивания процессов трансфера знаний, полученные в~ИПИ РАН, 
дали возмож\-ность впервые в~РФ вычислить значения индикатора ИЦ как 
косвенного показателя интенсив\-ности взаимосвязи фундаментальной науки 
и~ИКТ,\linebreak \mbox{к~которым}, в~частности, 
относится класс G06 (Обработка данных; вычисления; счет). При этом для 
описания тематики научных статей, цитируемых в~запатентованных 
изобретениях, использовался ГРНТИ. Вычисленные значения индикатора показали, что наиболее 
часто в~изобретениях по ИКТ цитируются научные статьи по автоматике, 
вычислительной технике, кибернетике, электронике, радиотехнике, 
электротехнике и~информатике. Таким образом, применение рубрик ГРНТИ 
отражает в~основном прикладной аспект результатов, изложенных в~научных 
пуб\-ли\-ка\-ци\-ях, цитируемых в~изобретениях по ИКТ. Было показано также, что для 
получения многоаспектной картины взаимосвязей отраслей науки с~технологиями 
целесообразно также использовать фундаментальную рубрикацию отраслей 
знаний, например классификатор РФФИ~\cite{19-min}.
  
  Предметом настоящего исследования является описание и~предварительный 
анализ результатов индикаторного оценивания процессов трансфера знаний из 
разных областей науки в~сферу ИКТ. В качестве исходной информации 
использовались полнотекстовые описания изобретений по классу G06 МПК, 
опуб\-ли\-ко\-ван\-ные Роспатентом в~2000--2012~гг. В~результате их 
автоматизированной обработки получены значения индикатора ИЦ отдельно для 
отечественных, зарубежных и~совместных изо\-бре\-тений, запатентованных в~РФ.



  
  Чтобы определить интенсивность цитирования научных работ, были 
сопоставлены информационные ресурсы двух категорий~--- патентной  
и~на\-уч\-но-тех\-ни\-че\-ской:
  \begin{itemize}
\item массив полнотекстовых описаний изобретений за определенный 
период времени, которые относятся к~ИКТ;
\item библиографические описания научных работ, которые цитируются 
в~описаниях изобретений, входящих в~первый массив.
\end{itemize}

  Массив полнотекстовых описаний изобретений, на которые выданы патенты, 
является пуб\-лич\-ной информацией, размещенной на сайте Роспатента. Массив 
цитируемых научных работ был сформирован авторами статьи путем выделения 
библиографических описаний этих работ из полнотекстовых описаний 
запатентованных изобретений. В~результате обработки и~анализа второго 
массива были определены (по ГРНТИ) рубрики ННИ, к~которым относятся научные результаты, 
излагаемые в~пуб\-ли\-ка\-циях.
  
  Последующее сопоставление рубрик ННИ и~индексов МПК позволило 
определить частотность взаимосвязей между ними, характеризующую 
интенсивность цитирования в~изобретениях по ИКТ пуб\-ли\-ка\-ций, относящихся 
к~некоторой научной дисциплине, и,~как следствие, косвенно оценить 
интенсивность переноса научных знаний в~сферу изобретений и~технологий.
  
  Целью статьи является анализ экспериментальных данных, характеризующих 
интенсивность цитирования научных работ в~описаниях изобретений 
с~детализацией по отечественным и~зарубежным изобретениям, запатентованным 
в~РФ.

\vspace*{-9pt}
  
\section{Сопоставление патентной активности отечественных 
и~зарубежных заявителей}
  
  Использованная в~работе патентная информация позволяет сравнить патентную 
активность в~России отечественных и~зарубежных заявителей. По данным, 
которые пуб\-ли\-ку\-ют\-ся в~ежегодных отчетах Роспатента, в~2000~г.\ доля патентов 
РФ, выданных иностранным заявителям по всем рубрикам МПК, 
составляла~18\% от общего числа выданных патентов (всего было 
выдано~17\,592~патента РФ).\linebreak
 К~2014~г.\ эта доля возросла до~32\% (всего было\linebreak 
выдано~33\,950~патентов)~\cite{27-min, 28-min}. Индексы МПК опуб\-ли\-ко\-ван\-ных 
патентов на изобретения дают возможность определить эту долю по любому виду 
технологий, патентуемых в~России.
  
  В рамках проекта РФФИ №\,16-07-00075, первым результатам выполнения 
которого посвящена данная статья, анализировались технологии, относящиеся 
к~классу\footnote{Индексы МПК представляют собой многоуровневую иерархическую 
структуру и~соответственно уровням разделены на разделы, классы, подклассы, группы 
и~подгруппы.} G06 МПК <<Обработка данных; вычисление; счет>>. С~этой целью 
с~серверов Роспатента была отобрана информация об изобретениях данного 
класса МПК, в~том числе поля данных о~странах заявителей 
и~патентообладателей. Это дало возможность изучить распределение патентов, 
полученных в~РФ, по странам заявителей.
  
  Используемые подходы и~методы позволяют кроме индикаторов ИИЦ и~ИЦ 
расширить спектр\linebreak применяемых индикаторов и~вычислять их значения, исходя из 
потребностей наукометрических исследований в~рамках перечня 
информационных\linebreak полей, публикуемых Роспатентом, в~том числе определять 
патентную активность в~РФ изобретателей различных стран.
  
  Всего в~РФ за период с~2000 по~2012~гг.\ Роспатент опубликовал сведения 
о~6665~патентах РФ на изобретения по классу G06 (учитывались основные 
и~дополнительные индексы МПК, относящиеся к~этому классу). Как видно из 
табл.~2, права на более чем~45\%~изобретений по G06 принадлежат 
патентообладателям только из России, более~54\%~--- зарубежным 
патентообладателям только из одной страны, кроме РФ, и~менее~1\%~--- 
патентообладателям из двух и~более стран. Учитывая такое распределение, 
в~дальнейшем будем рассматривать только патенты с~правообладателями 
исключительно из одной страны.


  На рис.~1--3 представлено распределение патентов РФ на изобретения класса 
G06, опубликованных в~РФ за период 2000--2012~гг., по годам с~указанием прав 
на изобретение: РФ или другие страны. 

    
     Как следует из рис.~1, общее количество патентов РФ по классу G06 
выросло за этот период более чем в~три раза. Как видно из рис.~2, в~последние 
годы растет не только число патентов класса G06 (см.\ рис.~1), но и~доля их в~общем 
числе патентов.
     
      
  Как уже отмечалось, доля патентов РФ, выданных иностранным заявителям по 
всем рубрикам МПК, в~2014~г.\ составляла~32\%~\cite{28-min}, причем доля 
патентов РФ на изобретения по ИКТ, принадлежа- %\linebreak\vspace*{-12pt}

\end{multicols}

\begin{table*}[h]\small %tabl2
\begin{center}
\Caption{Принадлежность прав на изобретения класса G06, опубликованные в~РФ за период 
2000--2012~гг.}
     \vspace*{2ex}
     
     \begin{tabular}{|l|c|c|}
     \hline
\multicolumn{1}{|c|}{Принадлежность прав на изобретение}&Количество
патентов&Доля от общего числа патентов\\
\hline
Патентообладатели только из РФ&3012&45,19\%\\
Патентообладатели из РФ и~других стран (совместно)&\hphantom{99}28&\hphantom{9}0,42\%\\
Патентообладатели только из одной страны (не РФ)&3611&54,18\%\\
Патентообладатели из нескольких стран (не РФ)&\hphantom{99}14&\hphantom{9}0,21\%\\
\hline
\end{tabular}
\end{center}
\vspace*{-12pt}
\end{table*}

\pagebreak

\begin{figure} %fig1
      \vspace*{1pt}
 \begin{center}
 \mbox{%
 \epsfxsize=114.91mm
 \epsfbox{zac-1.eps}
 }
 \end{center}
 \vspace*{-9pt}
      \Caption{Распределение по годам патентов РФ на изобретения класса G06, 
опубликованных за период 2000--2012~гг.: \textit{1}~--- РФ; \textit{2}~--- другие страны}
%     \end{figure}
%\begin{figure} %fig2
 \vspace*{16pt}
 \begin{center}
 \mbox{%
 \epsfxsize=113.489mm
 \epsfbox{zac-2.eps}
 }
 \end{center}
 \vspace*{-9pt}
\Caption{Доля патентов класса G06 в~общем числе опубликованных патентов РФ 
по годам: \textit{1}~--- РФ; \textit{2}~--- другие страны}
\vspace*{14pt}
     \end{figure}


     

\begin{multicols}{2}

\noindent 
щих зарубежным 
патентообладателям, существенно выше. 

Как видно из рис.~3, эта доля выросла 
с~41\% в~2000~г.\ до~57\% в~2012~г.\ (максимум, почти в~70\%, приходится на 
2009~г.). В~целом это говорит о~значительном интересе, который проявляют 
иностранные компании и~фирмы, получающие патенты РФ, к~российскому рынку 
ИКТ. Этот интерес сохраняется, несмотря на то что начиная с~2010~г.\ доля 
патентов РФ на изобретения по ИКТ, принадлежащих зарубежным 
патентообладателям, имеет тенденцию к~снижению.


  
  Наибольший интерес к~патентованию в~РФ проявляют США (1500~патентов, 
т.\,е.\ половина от~3000~патентов российских патентообладателей), Южная 
Корея, Германия, Япония, Франция и~Нидерланды. 
%
На рис.~4 пред\-став\-ле\-но 
распределение патентов РФ на изобретения класса G06 по\linebreak

\end{multicols}

\begin{figure*} %fig3
 \vspace*{12pt}
 \begin{center}
 \mbox{%
 \epsfxsize=123.972mm
 \epsfbox{zac-3.eps}
 }
 \end{center}
 \vspace*{-11pt}
\Caption{Распределение патентов РФ на изобретения класса G06  по 
отечественным и~зарубежным патентообладателям, а~также по годам: 
\textit{1}~--- РФ; \textit{2}~--- другие страны}
%     \end{figure*}
%\begin{figure*} %fig4
 \vspace*{6pt}
 \begin{center}
 \mbox{%
 \epsfxsize=115.612mm
 \epsfbox{zac-4.eps}
 }
 \end{center}
 \vspace*{-11pt}
\Caption{Распределение патентов РФ на изобретения класса G06 по странам (для стран, 
у~которых число патентов не меньше~50; без патентов с~патентообладателями из разных стран)}
\vspace*{-41pt}
     \end{figure*}
     
\begin{multicols}{2}


\noindent
 странам. Для 
наглядности на рисунке пред\-став\-ле\-ны только те
 страны,
 которые имеют в~РФ не 
менее~50~патентов.

\begin{figure*} %fig5
 \vspace*{1pt}
 \begin{center}
 \mbox{%
 \epsfxsize=115.412mm
 \epsfbox{zac-5.eps}
 }
 \end{center}
 \vspace*{-9pt}
\Caption{Распределение патентов РФ на изобретения класса G06 по подклассам основного 
индекса МПК (расшифровка подклассов дана в~табл.~3): \textit{1}~--- РФ; \textit{2}~--- другие 
страны}
\vspace*{12pt}
     \end{figure*}
     
      \begin{table*}[b]\small %tabl3
%      \vspace*{12pt}
     \begin{center}
     \Caption{Названия приведенных подклассов МПК}
     \vspace*{2ex}
     
     \begin{tabular}{|l|p{140mm}|}
     \hline
\multicolumn{1}{|c|}{\tabcolsep=0pt\begin{tabular}{c}Подкласс\\ МПК\end{tabular}}&\multicolumn{1}{c|}{Название}\\
\hline
\hspace*{2mm}A61B&Диагностика; хирургия; опознание личности\\
\hline
\hspace*{2mm}B42D&Книги; книжные обложки; несброшюрованные листы\\
\hline
\hspace*{2mm}G06C&Механические цифровые вычислительные машины\\
\hline
\hspace*{2mm}G06D&Гидравлические и~пневматические цифровые вычислительные устройства\\
\hline
\hspace*{2mm}G06E&Оптические вычислительные устройства\\
\hline
\hspace*{2mm}G06F&Обработка цифровых данных с~помощью электрических устройств\\
\hline
\hspace*{2mm}G06G&Аналоговые вычислительные машины\\
\hline
\hspace*{2mm}G06J&Гибридные вычислительные устройства\\
\hline
\multicolumn{1}{|l|}{\raisebox{-6pt}[0pt][0pt]{\hspace*{2mm}G06K}}&Распознавание, представление и~воспроизведение данных; манипулирование 
носителями информации; носители информации\\
\hline
\hspace*{2mm}G06M&Счетчики; способы и~устройства для подсчета предметов, не отнесенные к~другим 
подклассам\\
\hline
\hspace*{2mm}G06N&Компьютерные системы, основанные на специфических вычислительных моделях\\
\hline
\multicolumn{1}{|l|}{\raisebox{-6pt}[0pt][0pt]{\hspace*{2mm}G06Q}}&Системы обработки данных или способы, специально предназначенные для 
административных, коммерческих, финансовых, управленческих, надзорных или 
прогностических целей\\
\hline
\hspace*{2mm}G06T&Обработка или генерация данных изображения\\
\hline
\hspace*{2mm}G07F&Монетные или подобные им автоматы\\
\hline
\hspace*{2mm}H04N&Передача изображений\\
\hline
\end{tabular}
\end{center}
\end{table*}

  \vspace*{-3pt}
       
     На рис.~5 представлено распределение по подклассам МПК патентов, 
в~описаниях которых упоминается класс G06.

%\columnbreak

 Подклассы A61B, B42D и~H04N, 
отображенные на рис.~5 (см.\ их названия в~табл.~3) не относятся к~классу G06, 
однако в~описаниях обработанных изобретений, где эти три индекса~--- основные, 
присутствовали также дополнительные индексы, относящиеся к~классу~G06.
     
     Интересно отметить, что российские патентообладатели указывают для 
изобретений по ИКТ в~качестве основных индексы, относящиеся, как правило, 
к~классу G06, в~то время как у зарубежных патентообладателей доля индексов, 
относящихся к~другим классам, выше (согласно проведенным расчетам~---~16\% 
и~25\% соответственно). Это говорит, скорее всего, о~том, что зарубежные 
заявители чаще обозначают сферы прикладного использования патентуемых ими 
ИКТ с~помощью индексов МПК.
     

     

    
     Распределение патентов РФ на изобретения подкласса G06F по группам 
МПК представлено на рис.~6. Названия групп даны в~табл.~4.

\end{multicols}

\begin{figure*} %fig6
 \vspace*{1pt}
 \begin{center}
 \mbox{%
 \epsfxsize=114.412mm
 \epsfbox{zac-6.eps}
 }
 \end{center}
 \vspace*{-9pt}
\Caption{Распределение патентов РФ на изобретения подкласса G06F по группам 
МПК (приведены группы подкласса G06F, содержащие более~20~патентов): 
\textit{1}~--- РФ; \textit{2}~--- другие страны}
     \end{figure*}




\begin{table*}\small %tabl4
\begin{center}
\Caption{Названия групп МПК, приведенных на рис.~6}
      \vspace*{2ex}
      
      \begin{tabular}{|l|p{137mm}|}
      \hline
\multicolumn{1}{|c|}{Группа МПК}&\multicolumn{1}{c|}{Название}\\
\hline
G06F 1/00&Конструктивные элементы вычислительных машин и~устройств для обработки 
данных\\
\hline
G06F 11/00&Обнаружение ошибок, исправление ошибок; контроль\\
\hline
G06F 12/00&Выборка, адресация или распределение данных в~системах или архитектурах 
памяти\\
\hline
\multicolumn{1}{|l|}{\raisebox{-6pt}[0pt][0pt]{G06F 13/00}}&Соединение запоминающих устройств, устройств ввода-вывода или устройств 
центрального процессора\\
\hline
G06F 15/00&Цифровые компьютеры\\
\hline
\multicolumn{1}{|l|}{\raisebox{-6pt}[0pt][0pt]{G06F 17/00}}&Устройства или методы цифровых вычислений или обработки данных, специально 
предназначенные для специфических функций\\
\hline
\multicolumn{1}{|l|}{\raisebox{-6pt}[0pt][0pt]{G06F 19/00}}&Устройства или способы цифровых вычислений или обработки данных для 
специальных применений\\
\hline
\multicolumn{1}{|l|}{\raisebox{-6pt}[0pt][0pt]{G06F 21/00}}&Устройства защиты компьютеров или компьютерных систем от 
несанкционированной деятельности\\
\hline
\multicolumn{1}{|l|}{\raisebox{-12pt}[0pt][0pt]{G06F 3/00}}&Вводные устройства для передачи данных, подлежащих преобразованию в~форму, 
пригодную для обработки в~вычислительной машине; выводные устройства для передачи 
данных из устройств обработки в~устройства вывода\\
\hline
\multicolumn{1}{|l|}{\raisebox{-6pt}[0pt][0pt]{G06F 7/00}}&Способы и~устройства для обработки данных с~воздействием на порядок их 
расположения или на содержание обрабатываемых данных\\
\hline
G06F 9/00&Устройства для программного управления\\
\hline
\end{tabular}
\end{center}
\end{table*}

\begin{multicols}{2}

\begin{table*}\small  %tabl5
%\vspace*{-6pt}
\begin{center}
\Caption{Число цитирований научных работ для изобретений с~патентообладателями из РФ}
      \vspace*{2ex}
      
      \begin{tabular}{|c|c|c|c|c|c|c|c|c|}
      \hline
Подкласс&\multicolumn{8}{c|}{Код рубрики ГРНТИ}\\
\cline{2-9}
МПК&13.00.00&20.00.00&27.00.00&28.00.00&45.00.00&47.00.00&50.00.00&76.00.00\\
\hline
G06E&0&\hphantom{9}0&0&\hphantom{9}0&\hphantom{9}0&16&\hphantom{9}0&0\\
G06F&2&423\hphantom{9}&59\hphantom{9}&758\hphantom{9}&469\hphantom{9}&517\hphantom{9}&915\hphantom{9}&0\\
G06G&0&\hphantom{9}0&0&\hphantom{9}2&15&27&14&0\\
G06K&0&539\hphantom{9}&4&587\hphantom{9}&550\hphantom{9}&602\hphantom{9}&655\hphantom{9}&2\\
G06N&0&19&1&14&30&26&19&1\\
G06Q&0&\hphantom{9}0&0&\hphantom{9}8&\hphantom{9}0&\hphantom{9}0&\hphantom{9}1&5\\
G06T&0&\hphantom{9}8&51\hphantom{9}&51&\hphantom{9}8&17&80&0\\
\hline
     \end{tabular}
     \end{center}
%\end{table*}
%     \begin{table*}\small %tabl6
\begin{center}
\Caption{Число цитирований научных работ для изобретений патентообладателей из 
зарубежных стран}
     \vspace*{2ex}
     
     \tabcolsep=7.5pt
     \begin{tabular}{|c|c|c|c|c|c|c|c|c|c|c}
     \hline
Подкласс &\multicolumn{8}{c|}{Код рубрики ГРНТИ}\\
\cline{2-9}
МПК&13.00.00&20.00.00&27.00.00&28.00.00&45.00.00&47.00.00&50.00.00&76.00.00\\
\hline
G06E&0&0&0&0&0&0&0&0\\
G06F&0&0&45\hphantom{9}&46\hphantom{9}&9&9&54\hphantom{9}&0\\
G06G&0&0&0&0&0&0&0&0\\
G06K&0&0&31\hphantom{9}&13\hphantom{9}&28\hphantom{9}&36\hphantom{9}&65\hphantom{9}&0\\
G06N&0&1&1&1&0&0&1&0\\
G06Q&0&0&0&0&0&0&0&0\\
G06T&0&0&76\hphantom{9}&78\hphantom{9}&32\hphantom{9}&48\hphantom{9}&137\hphantom{99}&2\\
\hline
\end{tabular}
\end{center}
\end{table*}

\section{Интенсивность цитирования научных публикаций}
 
     В работе~\cite{19-min} были приведены значения индикатора ИЦ 
(в~описаниях изобретений по классу G06) для научных работ, тематика 
первоисточников которых (журналов и~трудов конференций) была описана 
с~помощью рубрик ГРНТИ. Определение значений этого индикатора ранее было 
выполнено без детализации по странам патентообладателей. Учет страны 
позволяет выявить различия в~интенсивности цитирования научных работ 
в~российских и~иностранных изобретениях по ИКТ, на которые были выданы 
патенты РФ. 

Таблицы~5 и~6 содержат данные о числе процитированных научных 
публикаций, соответствующих как рубрике ГРНТИ, так и~подклассу МПК, что 
позволяет охарактеризовать интенсивность взаимосвязи конкретного научного 
направления (по ГРНТИ) и~технологии (по подклассу МПК). По уровню 
использования научных результатов, что косвенно отображается чис\-лом 
цитирований научных работ, для технологий подклассов G06F и~G06K важны 
практически все научные направления, указанные в~табл.~5 и~6, причем по чис\-лу 
цитирований выделяются результаты по научным направлениям ГРНТИ: 
<<Кибернетика>> (28.00.00) <<Электроника. Радиотехника>> (47.00.00), 
<<Автоматика. Вычислительная техника>> (50.00.00). Первые строки этих таб\-лиц 
содержат коды восьми рубрик ГРНТИ, которые использовались для описания 
тематики цитируемых научных работ. Названия рубрик даны в~табл.~7.
     

     
     Как видно из табл.~5, самой высокой цитируемостью научных работ 
характеризуются изобретения российских патентообладателей для технологий 
подкласса G06F (Обработка цифровых\linebreak данных с~помощью электрических 
устройств) и~G06K (Распознавание, представление и~воспроизведение данных; 
манипулирование носителями информации; носители информации). 
Показательно, что на долю именно этих технологий приходится наибольшее 
число выданных патентов РФ, что говорит о~более интенсивном развитии данных 
технологий в~РФ.
     


\begin{table*}\small %tabl7
     \begin{center}
     \Caption{Рубрики ГРНТИ источников, публикации которых
     цитируются в~изобретениях подкласса МПК G06F}
     \vspace*{2ex}
     
     \tabcolsep=7pt
     \begin{tabular}{|c|p{110mm}|r|c|}
     \hline
Код ГРНТИ&\multicolumn{1}{c|}{Название рубрики ГРНТИ}&\multicolumn{1}{c|}{РФ}&
\tabcolsep=0pt\begin{tabular}{c}Другие\\ страны\end{tabular}\\
\hline
13.00.00&КУЛЬТУРА. КУЛЬТУРОЛОГИЯ&2&\\
20.00.00&ИНФОРМАТИКА&423&\\
27.00.00&МАТЕМАТИКА&44&45\\
27.03.00&Математическая логика и~основания математики&2&\\
27.35.00&Математические модели естественных наук и~технических наук. Уравнения математической 
физики&2&\\
27.41.00&Вычислительная математика&11&\\
28.00.00&КИБЕРНЕТИКА&702&46\\
28.17.00&Теория моделирования&28&\\
28.19.00&Теория кибернетических систем управления&2&\\
28.23.00&Искусственный интеллект&26&\\
45.00.00&ЭЛЕКТРОТЕХНИКА&468&\hphantom{9}9\\
45.53.00&Электротехническое оборудование специального назначения&1&\\
47.00.00&ЭЛЕКТРОНИКА. РАДИОТЕХНИКА&459&\hphantom{9}9\\
47.01.00&Общие вопросы электроники и~радиотехники&4&\\
47.03.00&Теоретические основы электронной техники&14&\\
47.05.00&Теоретическая радиотехника&2&\\
47.14.00&Проектирование и~конструирование электронных приборов&2&\\
47.29.00&Электровакуумные и~газоразрядные приборы и~устройства&2&\\
47.33.00&Твердотельные приборы&2&\\
47.37.00&Голография&8&\\
47.41.00&Радиоэлектронные схемы&2&\\
47.43.00&Распространение радиоволн&2&\\
47.45.00&Антенны. Волноводы. Элементы СВЧ-техники&2&\\
47.47.00&Радиопередающие и~радиоприемные устройства&2&\\
47.49.00&Радиотехнические системы зондирования, локации и~навигации&2&\\
47.51.00&Телевизионная техника&2&\\
47.53.00&Запись и~воспроизведение сигналов&2&\\
47.55.00&Электроакустика, ультразвуковая и~инфразвуковая техника&2&\\
47.57.00&Инфракрасная техника&2&\\
47.59.00&Узлы, детали и~элементы радиоэлектронной аппаратуры&2&\\
47.61.00&Приборы для радиотехнических измерений&2&\\
47.63.00&Системы и~устройства отображения информации&2&\\
50.00.00&АВТОМАТИКА. ВЫЧИСЛИТЕЛЬНАЯ ТЕХНИКА&891&54\\
50.07.00&Теоретические основы вычислительной техники&4&\\
50.09.00&Элементы, узлы и~устройства автоматики и~вычислительной техники&2&\\
50.11.00&Запоминающие устройства&4&\\
50.41.00&Программное обеспечение вычислительных машин, комплексов и~сетей&2&\\
50.45.00&Системы телеуправления и~телеизмерения&2&\\
50.51.00&Автоматизация проектирования&4&\\
50.53.00&Автоматизация научных исследований&6&\\
50.00.00&ПРИБОРОСТРОЕНИЕ&78&\\
59.01.00&Общие вопросы приборостроения&2&\\
59.14.00&Проектирование и~конструирование приборов&4&\\
59.41.00&Приборы для измерения оптических и~светотехнических величин и~характеристик&6&\\
59.45.00&Приборы неразрушающего контроля изделий и~материалов&2&\\
\hline
\end{tabular}
\end{center}
\end{table*}
     
     В то же время, как следует из табл.~6, для изобретений зарубежных 
патентообладателей по интенсивности цитирования научных работ выделяется 
технология G06T (Обработка или генерация данных изображения), хотя число 
патентов в~подклассе G06T (см.\ рис.~5) меньше, чем в~G06F и~G06K 
(около~10\%).
     
     В табл.~7 даны названия ряда рубрик ГРНТИ, для которых приведены 
данные о~чис\-ле процитированных научных публикаций для подкласса МПК G06F. 
Как видно из таблицы, для большинства научных работ, процитированных 
в~описаниях изобретений, указаны рубрики ГРНТИ самого верхнего уровня 
иерархии, что, конечно, снижает информативность выявленных связей и~говорит 
о~необходимости уточнения рубрик публикаций, цитируемых авторами 
изобретений.
     
    \begin{figure*} %fig7
\vspace*{1pt}
 \begin{center}
 \mbox{%
 \epsfxsize=147.489mm
 \epsfbox{zac-7.eps}
 }
 \end{center}
 \vspace*{-9pt}
\Caption{Распределение <<периодов патентного отклика>> для класса G06 в~качестве 
основного индекса: \textit{1}~--- РФ; \textit{2}~--- другие страны}
     \end{figure*} 

\section{Период патентного отклика на~научные 
публикации}
     
     Временн$\acute{\mbox{ы}}$е аспекты информационных взаимосвязей 
науки и~технологий характеризуются <<периодом патентного отклика>>, т.\,е.\ 
промежутком времени, прошедшим с~момента выхода научных публи\-каций до 
опубликования патентов, в~описании изобретений которых цитируются данные 
пуб\-ли\-ка\-ции. Значения показателя <<период патентного отклика>> дают 
возможность оценить динамику востребованности результатов научных 
исследований в~сфере разработки технологий.
     
     В расчетах, представленных в~работе~\cite{19-min}, были определены 
значения этого показателя для научных публикаций, цитируемых в~изобретениях 
по классу G06. Расчеты, проведенные по полному массиву патентов класса G06 
(в~качестве основного индекса МПК), показывают, что максимум распределения 
значений равен~3~годам для патентообладателей из России и~9~годам~--- для 
зарубежных патентообладателей.
     
     Рисунок~7 демонстрирует распределение <<периодов патентного 
отклика>> для класса G06 для массивов как российских, так и~зарубежных 
патентообладателей (по оси абсцисс указана продолжительность <<периодов 
патентного отклика>> в~годах; по оси ординат для каждого массива~--- доля  
от общего числа цитированных научных публикаций в~данном массиве).


      
     
     Были также получены распределения <<периодов патентного отклика>> для 
отдельных наиболее многочисленных по числу патентов подклассов G06, 
а~именно: G06F, G06K, G06T (рис.~8 и~9).



     
     Для подкласса G06F <<Обработка цифровых данных с~помощью 
электрических устройств>> (см.\ рис.~8,\,\textit{а}), распределения имеют максимумы 
в~6 и~9~лет соответственно.


     Для подкласса G06K <<Распознавание, пред\-став\-ле\-ние и~воспроизведение 
данных; манипулирование носителями информации; носители 
информации>> (см.\ рис.~8,\,\textit{б}) распределения имеют по два максимума в~3 и~22~года 
и~в~5 и~9~лет.



     
     Для подкласса G06T <<Обработка или генерация данных 
изображения$\ldots$>> (см.\ рис.~9), распределения имеют по два максимума в~5 
и~9~лет и~в~6 и~7~лет.
     
     Отметим существенно разный характер полученных распределений для 
различных подклассов. Для выяснения причин различий в~распределениях 
значений данного показателя требуется содержательный анализ изобретений 
и~цитируемых в~них научных публикаций, что выходит за рамки настоящей 
статьи.
     
\section{Заключение}

%\vspace*{-3pt}
     
     В ходе работ по оцениванию процессов трансфера знаний разработаны 
методы и~технологии анализа активности патентования в~РФ с~использованием 
индикаторов, которые учитывают страну патентообладателя. Вычислены 
значения индикаторов ИЦ, которые демонстрируют значительные\linebreak различия 
в~интенсивности цитирования научных публикаций российскими и~зарубежными 
авторами. Экспериментально показано, что интенсивность цитирования зависит 
не только от вида\linebreak технологий (см.\ табл.~1), но и~от страны патентообладателя 
(см.\ табл.~5 и~6).
     
     Например, самая высокая цитируемость научных работ отмечена 
в~описаниях изобретений российских патентообладателей для технологий 
подкласса G06F (Обработка цифровых данных \mbox{с~помощью} электрических 
устройств) и~G06K (Рас-\linebreak\vspace*{-12pt}

\pagebreak

\end{multicols}

\begin{figure*} %fig8
\vspace*{1pt}
 \begin{center}
 \mbox{%
 \epsfxsize=147.489mm
 \epsfbox{zac-8.eps}
 }
 \end{center}
 \vspace*{-9pt}
\Caption{Распределение <<периодов патентного отклика>> для подклассов G06F~(\textit{а})
и~G06K~(\textit{б}) (основной 
индекс): \textit{1}~--- РФ; \textit{2}~--- другие страны}
\vspace*{2pt}
     \end{figure*}

\begin{multicols}{2}


\noindent
познавание, представление и~воспроизведение данных; 
манипулирование носителями информации; носители информации).

\begin{figure*} %fig9
\vspace*{1pt}
 \begin{center}
 \mbox{%
 \epsfxsize=147.503mm
 \epsfbox{zac-10.eps}
 }
 \end{center}
 \vspace*{-9pt}
\Caption{Распределение времени между публикацией статьи и~патента подкласса G06T 
(основной индекс) для патентообладателей из разных стран (в процентах к~общему числу 
цитируемых изобретателями публикаций в~каждой группе); по горизонтальной оси отложена 
разница в~годах: \textit{1}~--- РФ; \textit{2}~--- другие страны}
%\vspace*{-7pt}
     \end{figure*}
 

В~то же 
время в~описаниях изобретений зарубежных патентообладателей по 
интенсивности цитирования научных работ выделяется технология G06T 
(Обработка или генерация данных изображения). При этом число патентов 
в~подклассе G06T существенно меньше, чем в~G06F и~G06K.
     
     Впервые были получены распределения значений <<периода патентного 
отклика>>, которые дают возможность оценить динамику применения 
результатов научных исследований в~сфере разработки технологий. Расчеты, 
проведенные в~данной работе, показывают, что для изобретений по ИКТ 
максимум распределения значений этого индикатора составляет~3~года для 
патентообладателей из России и~9~лет для зарубежных патентообладателей. 
Однако необходимо принимать во внимание и~период времени от момента подачи 
заявки до публикации запатентованного изобретения. Если построить 
аналогичные распределения с~учетом момента подачи заявки, то максимумы этих 
распределений сместятся влево.


\vspace*{-9pt}
  
{\small\frenchspacing
 {%\baselineskip=10.8pt
 \addcontentsline{toc}{section}{References}
 \begin{thebibliography}{99}
\bibitem{1-min}
\Au{Giuri~P., Mariani~M., Brusoni~S., Crespi~G., Francoz~D., Gambardella~A., 
Garcia-Fontes~W., Geuna~A., Gonzales~R., Harhoff~D., Hoisl~K., Le Bas~C., Luzzi~A.,
 Magazzini~L., 
Nesta~L., Nomaler~$\ddot{\mbox{O}}$., Palomares~N., P.~Patel,
Romanelli~M., Verspagen~B.} Inventors 
and invention processes in Europe: Results from the PatVal-EU survey~// Res. Policy, 2007. 
Vol.~36. No.\,8. P.~1107--1127.
\bibitem{2-min}
\Au{Van~Looy~B., Zimmermann~E., Veugelers~R., Verbeek~A., Mello~J., Debackere~K.} Do 
science-technology interactions pay on when developing technology? An exploratory investigation 
of~10~science-intensive technology domains~// Scientometrics, 2003. Vol.~57. No.\,3.  
P.~355--367.
\bibitem{3-min}
\Au{Narin~F., Noma~E.} Is technology becoming science?~// Scientometrics, 1985. Vol.~7.  
No.\,3--6. P.~369--381.
\bibitem{7-min} %4
\Au{Mansfield E.} Academic research and innovation~// Res. Policy, 1991. Vol.~20. No.\,1. 
P.~1--12.
\bibitem{4-min} %5
\Au{Schmoch~U.} Tracing the knowledge transfer from science to technology as reflected in patent 
indicators~// Scientometrics, 1993. Vol.~26. No.\,1. P.~193--211.
\bibitem{8-min} %6
\Au{Mansfield~E.} Academic research underlying industrial innovations: Sources, characteristics 
and financing~// Rev. Econ. Stat., 1995. Vol.~77. No.\,1. P.~55--62.

\bibitem{6-min} %7
\Au{Narin~F., Olivastro~D.} Linkage between patents and papers: An interim EPO/US 
comparison~// Scientometrics, 1998. Vol.~41. No.\,1--2. P.~51--59.


\bibitem{9-min} %8
\Au{Mansfield~E.} Academic research and industrial innovation: An update of empirical findings~// 
Res. Policy, 1998. Vol.~26. No.\,7--8. P.~773--776.

\bibitem{5-min} %9
\Au{Tijssen~R.\,J.\,W., Buter~R.\,K., Van Leeuwen~Th.\,N.} 
Technological relevance of science: An 
assessment of citation linkages between patents and research papers~// Scientometrics, 2000. 
Vol.~47. No.\,2. P.~389--412.

\bibitem{11-min} %10
\Au{Verbeek~А., Debackere~K., Luwel~M., Andries~P., Zimmermann~E., Deleus~D.} Linking 
science to technology: Using bibliographic references in patents to build linkage schemes~// 
Scientometrics, 2002. Vol.~54. No.\,3. P.~399--420.

\bibitem{10-min} %11
European Commission. Third European Report on Science \& Technology Indicators.~--- 
Luxembourg: Office for Official Publications of the European Communities, 2003. 451~p.

\bibitem{12-min} %12
\Au{Van~Looy~B., Hansen~W., Hollanders~H., Tijssen~R.} Using concordance tables to 
disentangle performance dynamics of HT manufacturing industries:
An empirical assessment of 
national innovation systems~//  10th Conference (International) on Science and Technology 
Indicators Proceedings: Book of abstracts.~--- Vienna: ARC GmbH, 2008.  
P.~196--200.
\bibitem{13-min}
\Au{Зацман~И.\,М., Шубников С.\,К.} Принципы обработки информационных ресурсов для 
оценки инновационного потенциала направлений научных исследований~// Электронные 
библиотеки: перспективные методы и~технологии, электронные коллекции: Тр. IX Всеросс. 
науч. конф. RCDL'2007.~--- Переславль: Университет города Переславля, 2007. С.~35--44.
\bibitem{14-min}
\Au{Минин~В.\,А., Зацман~И.\,М., Кружков~М.\,Г., Норекян~Т.\,П.} Методологические 
основы создания информационных систем для вычисления индикаторов тематиче\-ских 
взаимосвязей науки и~технологий~// Информатика и~её применения, 2013. Т.~7. Вып.~1. 
С.~70--81.
\bibitem{15-min}
\Au{Минин~В.\,А., Зацман~И.\,М., Хавансков~В.\,А., Шубников~С.\,К.} Архитектурные 
решения для систем вы\-чис\-ле\-ния индикаторов тематических взаимосвязей науки 
и~технологий~// Системы и~средства информатики, 2013. Т.~23. №\,2. C.~260--283.
\bibitem{16-min}
\Au{Зацман~И.\,М., Хавансков~В.\,А., Шубников~С.\,К.} Метод извлечения 
библиографической информации из полнотекстовых описаний изобретений~// Информатика 
и её применения, 2013. Т.~7. Вып.~4. С.~52--65.
\bibitem{17-min}
\Au{Хавансков~В.\,А., Шубников~С.\,К.} Поиск и~рубрицирование ссылок на цитируемые 
публикации в~электронных библиотеках полнотекстовых описаний\linebreak изобретений~// 
Электронные библиотеки: перспективные методы и~технологии, электронные коллекции: 
Тр. XVI Всеросс. науч. конф. RCDL-2014.~---  Дубна: \mbox{ОИЯИ}, 2014. С.~165--173.
\bibitem{18-min}
\Au{Минин~В.\,А., Зацман~И.\,М., Хавансков~В.\,А., Шубников~С.\,К.} Индикаторы 
тематических взаимосвязей науки и~технологий: от текста к~числам~// Информатика и~её 
применения, 2014. Т.~8. Вып.~3. С.~114--125.
\bibitem{19-min}
\Au{Минин~В.\,А., Зацман~И.\,М., Хавансков~В.\,А., Шубников~С.\,К.} Индикаторы 
тематических взаимосвязей науки и~информационно-компьютерных технологий в~начале 
XXI~века~// Информатика и~её применения, 2015. Т.~9. Вып.~2. С.~111--120.
\bibitem{20-min}
\Au{Зацман~И.\,М., Веревкин~Г.\,Ф.} Информационный мониторинг сферы науки в~задачах 
программно-це\-ле\-во\-го управления~// Системы и~средства информатики, 2006. 
Вып.~16. С.~164--189.
\bibitem{21-min}
\Au{Зацман И.\,М., Кожунова~О.\,С.} Семантический словарь системы информационного 
мониторинга в~сфере науки: задачи и~функции~// Системы и~средства информатики, 2007. Вып.~17. С.~124--141.

\bibitem{26-min} %22
\Au{Zatsman~I., Kozhunova~O.} Evaluating for institutional academic activities: Classification 
scheme for R\&D indicators~// 10th Conference (International) on Science and Technology 
Indicators: Book of abstracts.~--- Vienna: ARC GmbH, 2008. P.~428--431.
\bibitem{25-min} %23
\Au{Zatsman~I., Kozhunova~O.} Evaluation system for the Russian Academy of Sciences: 
Objectives-resources-results approach and R\&D indicators~// Atlanta Conference on Science 
and Innovation Policy Proceedings~/ Eds. S.\,E.~Cozzens, P.~Catalаn. 2009. {\sf 
http://smartech. gatech.edu/bitstream/1853/32300/1/104-674-1-PB.pdf}.

\bibitem{22-min} %24
\Au{Архипова~М.\,Ю., Зацман~И.\,М., Шульга~С.\,Ю.} Индикаторы патентной активности 
в~сфере ин\-фор\-ма\-ци\-он\-но-ком\-му\-ни\-ка\-ци\-он\-ных технологий и~методика их вычисления~// 
Экономика, статистика и~информатика. Вестник УМО, 2010. №\,4. С.~93--104.
\bibitem{24-min} %25
\Au{Zatsman~I., Durnovo~A.} Incompleteness problem of indicators system of research 
programme~// 11th Conference (International) on Science and Technology Indicators: 
Book of abstracts.~--- Leiden: CWTS, 2010. P.~309--311.
\bibitem{23-min} %26
\Au{Зацман~И.\,М., Дурново~А.\,А.} Моделирование процессов формирования экспертных 
знаний для мониторинга программно-целевой деятельности~// Информатика и~её 
применения, 2011. Т.~5. Вып.~4. С.~84--98.
\bibitem{27-min}
Российское агентство по патентам и~товарным знакам (Роспатент): Годовой отчет 2000. 
{\sf http://www1. fips.ru/wps/wcm/connect/content\_ru/ru/otchety/\linebreak otchet\_2000\_r6}.
\bibitem{28-min}
Федеральная служба по интеллектуальной собственности (Роспатент): Годовой отчет 2014. 
{\sf http://www. rupto.ru/about/reports/2014\_1\#1.2}.

\end{thebibliography}

 }
 }

\end{multicols}

\vspace*{-3pt}

\hfill{\small\textit{Поступила в~редакцию 19.04.16}}

%\vspace*{8pt}

\newpage

\vspace*{-24pt}

%\hrule

%\vspace*{2pt}

%\hrule

%\vspace*{8pt}



\def\tit{INTENSITY OF~CITATION OF~SCIENTIFIC PUBLICATIONS IN~INVENTIONS 
ON INFORMATION AND~COMPUTER TECHNOLOGIES PATENTED 
IN~RUSSIA BY~DOMESTIC AND~FOREIGN APPLICANTS}

\def\titkol{Intensity of~citation of~scientific publications in~inventions 
on ICT patented 
in~Russia by~domestic and~foreign applicants}

\def\aut{V.\,A.~Minin, I.\,M.~Zatsman, V.\,A.~Havanskov, and S.\,K.~Shubnikov}

\def\autkol{V.\,A.~Minin, I.\,M.~Zatsman, V.\,A.~Havanskov, and S.\,K.~Shubnikov}

\titel{\tit}{\aut}{\autkol}{\titkol}

\vspace*{-9pt}

\noindent
Institute of Informatics Problems, Federal Research Center 
``Computer Science and Control'' of the Russian Academy of Sciences,
44-2~Vavilov Str., Moscow 119333, Russian Federation


\def\leftfootline{\small{\textbf{\thepage}
\hfill INFORMATIKA I EE PRIMENENIYA~--- INFORMATICS AND
APPLICATIONS\ \ \ 2016\ \ \ volume~10\ \ \ issue\ 2}
}%
 \def\rightfootline{\small{INFORMATIKA I EE PRIMENENIYA~---
INFORMATICS AND APPLICATIONS\ \ \ 2016\ \ \ volume~10\ \ \ issue\ 2
\hfill \textbf{\thepage}}}

\vspace*{12pt}


 


\Abste{The paper discusses the information relationship between science and technology 
and the methods 
for indicator assessment of transfer processes (transfer) of knowledge from different fields of research 
in the area of technological development. The proposed methods are designed to determine the values 
of the indicator of intensity of citation of scientific papers in the descriptions of the inventions patented 
in Russia by domestic and foreign applicants. A~similar approach can be used to obtain indirect 
estimates of innovation potential of scientific research. The indicator values of intensity were 
calculated both in general and with the distribution by country of applicants. The paper presents the 
results of determining the values of the indicator. Full-text descriptions of inventions on class G06 of 
the International Patent Classification (Data Processing; Computing; Score) published by Rospatent in 
2000--2012 were used as the source of information. The use of information resources of 
Rospatent was due to the fact that they are in the electronic form, i.\,e., available for automated 
processing. The result is the values of the indicator of intensity of citation of scientific works patented 
in the Russian Federation, divided into groups of domestic, foreign, 
and joint inventions. This 
specification allowed to estimate the activity of international technological cooperation and joint 
patenting in information and computer technologies (ICT) in Russia, as well as to determine the 
themes of cooperation in this area.}

\KWE{citation of scientific papers; intensity of citation linkages between science and technologies; 
information technology; international patent classification; calculation of values of the indicator of 
intensity of citation}


\DOI{10.14357/19922264160213}

%\vspace*{-12pt}

\Ack
\noindent
The work was financially supported 
by the Russian Foundation for Basic Research (project 16-07-00075).


\vspace*{9pt}

  \begin{multicols}{2}

\renewcommand{\bibname}{\protect\rmfamily References}
%\renewcommand{\bibname}{\large\protect\rm References}

{\small\frenchspacing
 {%\baselineskip=10.8pt
 \addcontentsline{toc}{section}{References}
 \begin{thebibliography}{99}
\bibitem{1-min-1}
\Aue{Giuri,~P., M.~Mariani, S.~Brusoni, G.~Crespi, D.~Francoz, A.~Gambardella, 
W.~Garcia-Fontes, 
A.~Geuna, R.~Gonzales, D.~Harhoff, K.~Hoisl, C.~Le Bas, A.~Luzzi, L.~Magazzini, L.~Nesta, 
$\ddot{\mbox{O}}$.~Nomaler, N.~Palomares, P.~Patel,
M.~Romanelli, and B.~Verspagen}. 2007. Inventors 
and invention processes in Europe: Results from the PatVal-EU survey. \textit{Res. Policy} 
36(8):1107--1127.
\bibitem{2-min-1}
\Aue{Van~Looy,~B., E.~Zimmermann, R.~Veugelers, A.~Verbeek, J.~Mello, and K.~Debackere}. 
2003. Do science-technology interactions pay on when developing technology? An exploratory 
investigation of 10 science-intensive technology domains. \textit{Scientometrics} 57(3):355--367.
\bibitem{3-min-1}
\Aue{Narin, F., and E.~Noma.} 1985. Is technology becoming science? \textit{Scientometrics}  
7(3--6):369--381.
\bibitem{7-min-1} %4
\Aue{Mansfield,~E.} 1991. Academic research and innovation. \textit{Res. Policy} 20(1):1--12.
\bibitem{4-min-1} %5
\Aue{Schmoch,~U.} 1993. Tracing the knowledge transfer from science to technology as reflected in 
patent indicators. \textit{Scientometrics} 26(1):193--211.

\bibitem{8-min-1} %6
\Aue{Mansfield,~E.} 1995. Academic research underlying industrial innovations: Sources, 
characteristics and financing. \textit{Rev. Econ. Stat.} 77(1):55--62.

\bibitem{6-min-1} %7
\Aue{Narin, F., and D.~Olivastro}. 1998. Linkage between patents and papers: An interim EPO/US 
comparison. \textit{Scientometrics} 41(1--2):51--59.


\bibitem{9-min-1} %8
\Aue{Mansfield, E.} 1998. Academic research and industrial innovation: An update of empirical 
findings. \textit{Res. Policy} 26(7--8):773--776.
\bibitem{5-min-1} %9
\Aue{Tijssen,~R.\,J.\,W., R.\,K.~Buter, and Th.\,N.~Van Leeuwen}. 2000. Technological relevance of 
science: An assessment of citation linkages between patents and research papers. 
\textit{Scientometrics} 47(2):389--412.

\bibitem{11-min-1} %10
\Aue{Verbeek,~А., K.~Debackere, M.~Luwel, P.~Andries, E.~Zimmermann, and D.~Deleus}. 2002. 
Linking science to technology: Using bibliographic references in patents to build linkage schemes. 
\textit{Scientometrics} 54(3):399--420.
\bibitem{10-min-1} %11
European Commission. 2003. Third European Report on Science \& Technology Indicators. 
Luxembourg: Office for Official Publications of the European Communities. 451~p.
\bibitem{12-min-1}
\Aue{Van~Looy,~B., W.~Hansen, H.~Hollanders, and R.~Tijssen}. 2008. Using concordance tables to 
disentangle performance dynamics of HT manufacturing industries: An empirical assessment of 
national innovation systems. \textit{10th Conference (International) on Science and Technology 
Indicators Proceedings: Book of abstracts}. Vienna: ARC GmbH. 196--200.
\bibitem{13-min-1}
\Aue{Zatsman,~I.\,M., and S.\,K.~Shubnikov}. 2007. Printsipy obrabotki informatsionnykh resursov 
dlya otsenki in\-no\-va\-tsi\-on\-no\-go potentsiala napravleniy nauchnykh issledovaniy [Principles of 
processing of information resources for assessment of innovative potential of fields of scientific 
research]. \textit{Elektronnye Biblioteki: Perspektivnye Metody i~Tekhnologii, Elektronnye Kollektsii: 
Tr. 9-y Vseross. nauch. konf. RCDL'2007} [Digital Libraries: Perspective Methods and Technologies, 
Electronic collections: 9th All-Russia Scientific Conference RCDL'2007 Proceedings]. Pereslavl'.  
35--44.
\bibitem{14-min-1}
\Aue{Minin,~V.\,A., I.\,M.~Zatsman, M.\,G.~Kruzhkov, and T.\,P.~No\-re\-kyan}. 2013. 
Metodologicheskie osnovy so\-zda\-niya informatsionnykh sistem dlya vychisleniya indikatorov 
tematicheskikh vzaimosvyazey nauki i~tekhnologiy [Methodological basis for the creation of 
information systems for the calculation of indicators of thematic science-technology linkagies]. 
\textit{Informatika i~ee Primeneniya}~--- \textit{Inform. Appl.} 7(1):70--81.
\bibitem{15-min-1}
\Aue{Minin,~V.\,A., I.\,M.~Zatsman, V.\,A.~Havanskov, and S.\,K.~Shubnikov}. 2013. 
Arkhitekturnye resheniya dlya sis\-tem vychisleniya indikatorov tematicheskikh 
vza\-imo\-svya\-zey nauki 
i~tekh\-no\-lo\-giy [Architectural decisions for systems of calculation of indicators of thematic science-
technology linkagies]. \textit{Sistemy i~Sredstva Informatiki}~--- \textit{Systems and Means of 
Informatics} 23(2):260--283.
\bibitem{16-min-1}
\Aue{Zatsman,~I.\,M., V.\,A.~Havanskov, and S.\,K.~Shubnikov}. 2013. Metod izvlecheniya 
bibliograficheskoy in\-for\-ma\-tsii iz polnotekstovykh opisaniy izobreteniy [Method of extraction of 
bibliographic information from full-text descriptions of inventions]. \textit{Informatika i~ee 
Primeneniya}~--- \textit{Inform. Appl.} 7(4):52--65.
\bibitem{17-min-1}
\Aue{Havanskov,~V.\,A., and S.\,K.~Shubnikov}. 2014. Poisk i~rub\-ri\-tsi\-ro\-vanie ssylok na tsitiruemye 
publikatsii v~elektronnykh bibliotekakh polnotekstovykh opisaniy izobreteniy [Search and classifying 
of cited publications in digital libraries of full-text descriptions of inventions]. \textit{Elektronnye 
Biblioteki: Perspektivnye Metody i~Tekhnologii, Elektronnye Kollektsii: Tr. 16-y 
Vseross. nauch. 
konf. RCDL'2014} [Digital Libraries: Perspective Methods and Technologies, Electronic Collections: 
16th All-Russia Scientific Conference RCDL'2014 Proceedings]. Dubna. 165--173.

\columnbreak 

\bibitem{18-min-1}
\Au{Minin,~V.\,A., I.\,M.~Zatsman, V.\,A.~Havanskov, and S.\,K.~Shubnikov}. 2014. Indikatory 
tematicheskikh vzaimosvyazey nauki i~tekhnologiy: Ot teksta k~chislam [Indicators of thematic 
science-technology linkagies: From text to numbers]. \textit{Informatika i~ee Primeneniya}~--- 
\textit{Inform. Appl.} 8(3):114--125.

%\columnbreak

\bibitem{19-min-1}
\Aue{Minin,~V.\,A., I.\,M.~Zatsman, V.\,A.~Havanskov, and S.\,K.~Shubnikov}. 2015. Indikatory 
tematicheskikh vza\-imo\-svya\-zey nauki i~informatsionno-komp'yuternykh tekh\-no\-lo\-giy 
v~nachale XXI~veka 
[Indicators for thematic linkages between science and information and computer technologies at the 
beginning of the XXI century]. \textit{Informatika i~ee Primeneniya}~--- \textit{Inform. Appl.} 
9(2):111--120.
\bibitem{20-min-1}
\Aue{Zatsman,~I.\,M., and G.\,F.~Verevkin}. 2006. Informatsionnyy monitoring sfery nauki v 
zadachakh programmno-tselevogo upravleniya [Information monitoring in the field of science and 
problems of program-oriented management]. \textit{Sistemy i~Sredstva Informatiki}~--- 
\textit{Systems and Means of Informatics} 16:164--189.
\bibitem{21-min-1}
\Aue{Zatsman,~I.\,M., and O.\,S.~Kozhunova}. 2007. Semanti\-che\-skiy slovar' sistemy 
informatsionnogo monitoringa v sfere nauki: Zadachi i funktsii [The semantic dictionary of system for 
information monitoring in science: Tasks and functions]. \textit{Sistemy i~Sredstva 
Informatiki}~--- \textit{Systems and Means of Informatics} 17:124--141.
\bibitem{26-min-1} %22
\Aue{Zatsman,~I., and O.~Kozhunova}. 2008. Evaluating for institutional academic activities: 
Classification scheme for R\&D indicators. \textit{10th Conference (International) on Science and 
Technology Indicators: Book of abstracts}. Vienna: ARC GmbH. 428--431.
\bibitem{25-min-1} %23
\Aue{Zatsman,~I., and O.~Kozhunova}. 2009. Evaluation system for the Russian Academy of 
Sciences: Objectives-resources-results approach and R\&D indicators. \textit{2009 Atlanta Conference 
on Science and Innovation Policy Proceedings}. Eds. S.\,E.~Cozzens and P.~Catalаn. Available at: 
{\sf http://smartech.gatech.edu/bitstream/1853/32300/1/\linebreak 104-674-1-PB.pdf} (accessed January~24, 
2016).
\bibitem{22-min-1} %24
\Aue{Arkhipova,~M.\,Yu., I.\,M.~Zatsman, and S.\,Yu.~Shul'ga}. 2010. Indikatory patentnoy aktivnosti 
v~sfere informatsionno-kommunikatsionnykh tekhnologiy i~metodika ikh vychisleniya [Indicators of 
patent activity in the sphere of information and communication technologies and technique of their 
calculation]. \textit{Ekonomika, statistika i~informatika. Vestnik UMO} [Economy, statistics and 
informatics. Herald of the UMO] 4:93--104.
\bibitem{24-min-1} %25
\Aue{Zatsman, I., and A.~Durnovo}. 2010. Incompleteness problem of indicators system of research 
programme. \textit{11th Conference (International) on Science and 
Technology Indicators: 
Book of abstracts}. Leiden: CWTS. 309--311.
\bibitem{23-min-1} %26
\Aue{Zatsman,~I.\,M., and A.\,A.~Durnovo}. 2011. Modelirovanie protsessov formirovaniya 
ekspertnykh znaniy dlya mo\-ni\-to\-rin\-ga programmno-tselevoy deyatel'nosti [Modeling of creation 
processes of expert knowledge for monitoring program-oriented activities]. \textit{Informatika i~ee 
Primeneniya}~--- \textit{Inform. Appl.} 5(4):84--98.

\pagebreak

\bibitem{27-min-1}
Rossiyskoe agentstvo po patentam i~tovarnym znakam (Rospatent) 
[The Russian Agency for Patents and Trademarks (Rospatent)]. 2000.
 Annual Report. Available at: {\sf 
http://www1.fips.ru/wps/wcm/connect/content\_ru/\linebreak ru/otchety/otchet\_2000\_r6} 
(accessed June~20, 2016).
\bibitem{28-min-1}
Federal'naya sluzhba po intellektual'noy sobstvennosti (Rospatent) [The Federal 
service for intellectual property (Rospatent)]. 2014. Available at: {\sf 
http:// www.rupto.ru/about/reports/2014\_1\#1} (accessed January~24, 2016).
  \end{thebibliography}

 }
 }

\end{multicols}

\vspace*{-3pt}

\hfill{\small\textit{Received April 19, 2016}}

\Contr

\noindent
\textbf{Minin Vladimir A.} (b.\ 1941)~--- Doctor of Science in physics and mathematics, consultant, 
Institute of Informatics Problems, Federal Research Center ``Computer Science and Control'' of the 
Russian Academy of Sciences, 44-2~Vavilov Str., Moscow 119333, Russian Federation; 
\mbox{aleksisss@ya.ru}

      \vspace*{3pt}
      
      \noindent
      \textbf{Zatsman Igor M.} (b.\ 1952)~--- Doctor of Science in technology, Head of 
Department, Institute of Informatics Problems, Federal Research Center ``Computer Science and 
Control'' of the Russian Academy of Sciences, 44-2~Vavilov Str., Moscow 119333, Russian 
Federation; \mbox{iz\_ipi@a170.ipi.ac.ru} 

      \vspace*{3pt}
      
      \noindent
      \textbf{Havanskov Valerij A.} (b.\ 1950)~--- scientist, Institute of Informatics Problems, 
Federal Research Center ``Computer Science and Control'' of the Russian Academy of Sciences, 44-
2~Vavilov Str., Moscow 119333, Russian Federation; \mbox{havanskov@a170.ipi.ac.ru} 
     
      \vspace*{3pt}
      
      \noindent
      \textbf{Shubnikov Sergej K.} (b.\ 1955)~--- senior scientist, Institute of Informatics Problems, 
Federal Research Center ``Computer Science and Control'' of the Russian Academy of Sciences, 44-
2~Vavilov Str., Moscow 119333, Russian Federation; \mbox{sergeysh50@yandex.ru} 
      
\label{end\stat}


\renewcommand{\bibname}{\protect\rm Литература}      %13
\def\stat{meih}

\def\tit{СТАЦИОНАРНЫЕ ВЕРОЯТНОСТИ СОСТОЯНИЙ В~СИСТЕМЕ ОБСЛУЖИВАНИЯ КОНЕЧНОЙ ЕМКОСТИ 
С~ИНВЕРСИОННЫМ ПОРЯДКОМ
ОБСЛУЖИВАНИЯ И~ОБОБЩЕННЫМ ВЕРОЯТНОСТНЫМ
ПРИОРИТЕТОМ$^*$}

\def\titkol{Стационарные вероятности состояний в~системе обслуживания конечной емкости} 
%с~инверсионным порядком обслуживания и~обобщенным вероятностным приоритетом}

\def\aut{Л.\,А.~Мейханаджян$^1$}

\def\autkol{Л.\,А.~Мейханаджян}

\titel{\tit}{\aut}{\autkol}{\titkol}

\index{Мейханаджян Л.\,А.}
\index{Meykhanadzhyan L.\,A.}

{\renewcommand{\thefootnote}{\fnsymbol{footnote}} \footnotetext[1]
{Работа выполнена при поддержке РФФИ (проект 15-07-03007).}}


\renewcommand{\thefootnote}{\arabic{footnote}}
\footnotetext[1]{Российский университет дружбы народов, lameykhanadzhyan@gmail.com}


\Abst{Рассматривается система
$M/G/1/(r-1)$ с~дисциплиной инверсионного порядка обслуживания
и обобщенного вероятностного приоритета.
Предполагается, что в~момент поступления новой заявки в~систему
становится известной ее длина и,~кроме того, в~любой момент времени
известна остаточная длина каждой заявки в~системе.
В~момент поступления очередной заявки в~непустую систему ее
исходная длина сравнивается с~остаточной длиной заявки на приборе, и~в зависимости
от результатов сравнения наступает одно из следующих событий:
обе заявки покидают систему; только одна из заявок
покидает систему (другая остается на приборе);
обе заявки остаются в~системе (одна попадает на прибор, другая~--- в~очередь).
Заявки, оставшиеся в~системе, приобретают новую
(случайную) длину в~соответствии с~заданным распределением, зависящим в~общем случае от
исходных длин заявок.
Заявки, застающие систему полностью заполненной, теряются
и не оказывают на нее никакого воздействия.
В~статье предложены математические соотношения
для вычисления совместного стационарного распределения
числа заявок в~системе и~остаточного времени обслуживания заявки на приборе,
периода занятости системы, стационарного распределения
времени ожидания и~пребывания заявки длины~$x$ (в~терминах преобразования 
Лап\-ла\-са--Стил\-тье\-са (ПЛС)).}

\KW{система массового обслуживания; специальные
дисциплины; инверсионный порядок
обслуживания; вероятностный приоритет}

\DOI{10.14357/19922264160214} 

\vspace*{6pt}

\vskip 12pt plus 9pt minus 6pt

\thispagestyle{headings}

\begin{multicols}{2}

\label{st\stat}

\section{Введение}

В этой работе, являющейся продолжением работ~\cite{n1, n2}, будет
рассматриваться та же однолинейная система
массового обслуживания (СМО), что и~в~\cite{n1}, но ограниченной емкости.
Основной результат работ~\cite{n1, n2} состоит в~нахождении
совместного стационарного распределения вероятностей состояний %\linebreak 
системы
$M/G/1$ с~дисциплиной инверсионного %\linebreak 
порядка
обслуживания и~обобщенного вероятностного приоритета, а~также основных
стационарных вероятностных характеристик в~терминах ПЛС. %\linebreak
Сейчас же задача заключается в~исследовании стационарных 
ве\-ро\-ят\-ност\-но-вре\-мен\-н$\acute{\mbox{ы}}$х характеристик указанной системы
в~случае, когда присутствует ограничение на размер очереди.

\vspace*{-4pt}

\section{Описание системы}

Рассмотрим СМО
с~одним прибором,
одной очередью для ожидающих заявок емкости $(r\hm-1)\hm<\infty$, $r \hm\ge 2$,
и~входящим потоком заявок, который для простоты будем называть здесь
потоком пуассоновского типа. Отличие этого потока от пуассоновского
заключается в~следующем: интенсивность поступления заявок равна~$\lambda$,
если на приборе имеется заявка, и~$\tl$, если система пуста.


Если в~момент поступления заявки в~систему
на приборе имеется заявка, то исходное распределение времени обслуживания поступающей
заявки является произвольным с~функцией распределения (ФР) $B(x)$.
Если же заявка поступает в~систему в~тот момент, когда система пуста, то исходное
распределение времени обслуживания поступающей
заявки является произвольным с~ФР~$\tB(x)$.

Далее для простоты изложения будем считать, что ФР $B(x)$ и~$\tB(x)$ имеют непрерывные
ограниченные плотности распределения $b(x)\hm=B'(x)$ и~$\tb(x)\hm=\tB'(x)$,
причем $\tb \hm= \int_0^\infty x \tb(x)\,dx \hm< \infty$
и~$b \hm= \int_0^\infty x b(x)\,dx \hm< \infty$.

Обобщенный инверсионный порядок обслуживания с~вероятностным приоритетом (LCFS BPP)
заключается в~следующем.
Предполагается, что в~любой момент времени известна остаточная длина (далее будем говорить
просто длина) каждой заявки в~системе.
В~момент поступления в~систему новой заявки ее
исходная длина~$u$ сравнивается с~(остаточной) длиной~$v$ заявки на приборе.
С~вероятностью~$D(x,y|u,v)$,
зависящей только от~$u$ и~$v$, обслуживавшаяся ранее заявка продолжает обслуживаться, причем
ее длина становится меньше~$y$, а~вновь
поступившая становится на первое место в~очереди и~ее длина становится меньше~$x$.
Кроме того, с~вероятностью~$D^*(x,y|u,v)$,
зависящей только от~$u$ и~$v$, вновь поступившая заявка занимает прибор, вытесняя обслуживавшуюся
ранее на первое место в~очереди, причем длина заявки, бывшей ранее на приборе, становится
меньше~$y$, а~вновь поступившей~--- меньше~$x$.

Если на приборе находится заявка остаточной длины~$v$ и~в~систему поступает заявка
длины~$u$, то с~вероятностью $D_0(x|u,v)$ заявка, находящаяся на приборе, покидает
систему, а~поступившая заявка становится на
прибор, причем ее длина становится меньше~$x$.
Кроме того, с~вероятностью
$D_0^*(y|u,v)$ поступившая заявка сразу же покидает систему, а~заявка, находящаяся на
приборе, продолжает обслуживаться, причем ее длина становится меньше~$y$.
Введем также обозначение:
\begin{equation*}
%\label{(2.1)}
D(x|u,v) = D_0(x|u,v) + D_0^*(x|u,v)\,.
\end{equation*}
Здесь $D(x|u,v)$~--- вероятность того, что одна из двух заявок покинет систему, а~вторая встанет
на прибор и~примет длину меньше~$x$.

Наконец, предполагается, что с~вероят\-ностью~$d_0(u,v)$ обе заявки покидают
систему, а~на прибор становится первая заявка из очереди.

Будем считать для удобства изложения, что все ФР 
$D(x,y|u,v)$, $D^*(x,y|u,v)$, $D_0(x|u,v)$,
$D_0^*(y|u,v)$, $D(y|u,v)$ и~$D_0(u,v)$
имеют непрерывные ограниченные плотности
$d(x,y|u,v)\hm=\partial^2 D(x,y|u,v)/(\partial x \partial y)$,
$d^*(x,y|u,v)\hm=\partial^2 D^*(x,y|u,v)/(\partial x \partial y)$,
$d_0(x|u,v)\hm= \partial D_0(x|u,v)/\partial x$,
$d_0^*(y|u,v)=\partial D_0^*(y|u,v)/\partial y$
и~$d(x|u,v)\hm=\partial D(x|u,v)/\partial x$.


Естественно, для любых~$u$ и~$v$ выполнено условие:
\begin{multline*}
%\label{e2.1-m}
\int\limits_0^\infty \int\limits_0^\infty
\left[d(x,y|u,v) + d^*(x,y|u,v)\right]\,dxdy+{}\\
{}+ \int\limits_0^\infty d(x|u,v) \,dx
+ d_0(u,v) =1\,.
\end{multline*}

Если длина заявки на приборе становится
равной нулю, то она мгновенно покидает систему и~на прибор переходит первая
заявка из очереди. Остальная очередь сдвигается на единицу.


Для конечного накопителя необходимо также задать
дисциплину принятия заявок в~систему при отсутствии в~нем свободных мест.
Здесь для простоты изложения будет рассмотрен только
тот случай, когда поступающая в~заполненную систему заявка теряется.
Заметим, что в~этом случае принятая в~систему заявка
будет обязательно обслужена полностью.
Для всех СМО с~такой дисциплиной принятия заявок в~систему
при отсутствии в~накопителе свободных мест стационарные
вероятности\linebreak $p_n(x_1,\ldots,x_n)$ при $n\hm<r$ совпадают
с~точностью до\linebreak постоянной с~аналогичными вероятностями
для системы с~бесконечным накопителем, различие заключается
только в~вероятностях $p_{r}(x_1,\ldots,x_{r})$.
Однако несколько более сложно вычисляются стационарные
распределения, связанные с~временем пребывания заявки 
в~системе, поскольку даже заявки, принятые в~систему, могут покидать
ее недообслуженными.

Далее будем предполагать, что система
функционирует в~стационарном режиме
и~$\tb \hm= \int_0^\infty x\tb(x)\,dx \hm< \infty$
и~$b \hm= \int_0^\infty x b(x)\,dx \hm< \infty$.
Отметим, что параметр $\rho \hm= \lambda b$ для данной системы не
является загрузкой в~традиционном смысле и~может существенно от нее отличаться.


\section{Стационарные вероятностные характеристики}

Обозначим через $\nu(t)$ число заявок в~системе
в~момент~$t$, а~через $\vec\xi(t)\hm =(\xi_{1}(t),\ldots,\xi_{\nu(t)}(t))$~---
вектор, координатой $\xi_{1}(t)$ которого
является (остаточное) время обслуживания
заявки, находящейся в~этот момент на приборе,
$\xi_{2}(t)$~--- первой заявки в~очереди$,\ldots,$ $\xi_{\nu(t)-1}(t)$~---
последней, \mbox{$(\nu(t)-1)$-й} заявки в~очереди.
При $\nu(t)\hm=0$ вектор $\vec\xi(t)$ не определяется.
Тогда $\eta(t)\hm=(\nu(t),\vec\xi(t))$ представляет
собой марковский процесс, описывающий поведение числа заявок в~рассматриваемой системе.

Положим 
\begin{align*}
p_{0}(t)&= \mathbf{P}\{\nu(t)=0\}\,;
\\
P_{n}\left(t;x_1,\ldots,x_{n}\right) &=
\mathbf{P}\{\nu(t)=n\,,\\
&\hspace*{-20pt}\xi_{1}(t)<x_{1},\ldots,\xi_{n}(t)<x_{n}\}
\,,\enskip 1 \le n \le r\,.
\end{align*}
Обозначим через
\begin{equation*}
%\label{(2.1)}
p_{0} = \lim\limits_{t\to\infty}
p_{0}(t) \,;
\end{equation*}
\begin{equation*}
%\label{(2.1)}
P_{n}(x_1,\ldots,x_{n}) = \lim\limits_{t\to\infty}
P_{n}(t;x_1,\ldots,x_{n}) \,,\enskip 1 \le n \le r\,,
\end{equation*}
стационарное распределение процесса $\eta(t)$.
В~силу сделанных в~предыдущем пункте
предположений относительно параметров системы,
можно показать (см., например,~[3; 4, с.~273]), что существуют
непрерывные и~ограниченные плотности 

\noindent
\begin{multline*}
p_n(x_1,\ldots,x_{n}) = \fr{\partial^n }{\partial x_1\cdots \partial x_n}
P_n(x_1,\ldots,x_{n}) \,,\\[-1pt]
1 \le n \le r\,.
\end{multline*}

Выпишем систему интегродифференциальных
уравнений, которой удовлетворяют стационарные
плотности $p_n(x_1,\ldots,x_{n})$ и~которую
для краткости по аналогии с~простейшими СМО будем называть системой уравнений
равновесия (СУР). Для этого рассмотрим вспомогательную систему
с~$(n\hm-1)$ мес\-та\-ми ожидания, отличающуюся от исходной
сис\-те\-мы только тем, что если в~очереди
находится $(n\hm-1)$ заявок, заявка на приборе имеет
остаточную длину~$v$ и~поступает новая заявка
длины~$u$, то с~ве\-ро\-ят\-ностью $d(x,y|u,v)$ на
приборе остается вновь поступившая заявка,
длина которой становится равной~$x$, а~обслуживавшаяся ранее заявка покидает
сис\-те\-му, и~наоборот: с~вероятностью $d^*(y,x|u,v)$ систему покидает вновь
поступившая заявка, а~находившаяся ранее на приборе заявка продолжает обслуживаться, но
ее длина становится равной~$x$.

В силу метода исключения состояний~\cite{ppav}
стационарные вероятности состояний в~исходной
и~вспомогательной системах отличаются лишь на
постоянный множитель (за исключением вероятности $p_{r}(x_1,\ldots,x_{r})$).
Это дает возможность при составлении
уравнений для $p_n(x_1,\ldots,x_{n})$, $n\hm\ge 1$,
воспользоваться вспомогательной системой и~получить следующие соотношения:

\noindent
\begin{multline}
\label{e3.1-mei}
-p'_1(x) = \tl \tb(x) p_0 - \lambda p_1(x)
+ {}\\[-1pt]
{}+\lambda \Bigg( \int\limits_0^\infty \int\limits_0^\infty
d(x|u,v) b(u) p_1(v) \,dudv +{}
\\[-1pt]
{}+ \int\limits_0^\infty \int\limits_0^\infty
\int\limits_0^\infty \left[d(x,y|u,v) b(u) p_1(v)  +{}\right.\\[-1pt]
\left.{}+
d^*(y,x|u,v) b(u) p_1(v)\right] \,dydudv
\Bigg)\,;
\end{multline}

\vspace*{-16pt}

\noindent
\begin{multline*}
-p'_{n}\left(x_1,\ld,x_n\right) ={}
\\
{}=
\lambda \Bigg(
\int\limits_0^\infty \int\limits_0^\infty
\left[d\left(x_2,x_1|u,v\right) b(u) p_{n-1}\left(v,x_3\ld,x_n\right)
+ {}\right.\\
\left.{}+
d^*\left(x_1,x_2|u,v\right) b(u) p_{n-1}\left(v,x_3,\ld,x_n\right)\right]
\,dudv \Bigg)
-{}\\
{}-
\lambda p_{n}\left(x_1,\ld,x_n\right)
+{}\\
{}+ \lambda \Bigg(
\int\limits_0^\infty \int\limits_0^\infty
d\left(x_1|u,v\right) b(u) p_{n}\left(v,x_2,\ld,x_n\right)
\,dudv +{}
\end{multline*}

\noindent
\begin{multline}
{}+
\int\limits_0^\infty \int\limits_0^\infty
\int\limits_0^\infty \left[d\left(x_1,y|u,v\right) b(u) p_{n}\left(v,x_2,\ld,x_n\right)
+{}\right.
\\
\left.{}+
d^*\left(y,x_1|u,v\right) b(u) p_{n}\left(v,x_2,\ld,x_n\right)\right]
\,dy du dv \Bigg)\,,
\\
 1 \le n \le r-1\,;
 \label{e3.2-mei}
\end{multline}

\vspace*{-12pt}

\noindent
\begin{multline}
\label{e3.3-mei}
-p'_{r}\left(x_1,\ld,x_n\right) ={}\\
{}=
\lambda \Bigg(
\int\limits_0^\infty \int\limits_0^\infty
\left[d\left(x_2,x_1|u,v\right) b(u) p_{n-1}\left(v,x_3\ld,x_n\right)
+{}\right.
\\
\!\!\!\!\left.{}+
d^*\left(x_1,x_2|u,v\right) b(u) p_{n-1}\left(v,x_3,\ld,x_n\right)\right]
\,du dv \!\Bigg).\!\!
\end{multline}

Остановимся подробнее на выводе уравнения
для плотности $p_{r}(x_1,\ldots,x_{r})$ (остальные уравнения
получаются так же, как и~в случае накопителя бесконечной емкости~\cite{n1}).
Рассмотрим моменты времени~$t$ и~$(t\hm+\Delta)$.
Тогда для того, чтобы в~момент времени
$(t\hm+\Delta)$ в~системе находилось~$r$~заявок, причем
на приборе заявка длины~$x_1$, а~в~очереди
заявки длин $x_2,\ldots,x_r$, нужно, чтобы произошло одно из следующих событий:
\begin{itemize}
\item в~момент~$t$ в~системе находилось $(r-1)$
заявок, причем заявка на приборе имела
длину~$v$, первая заявка в~очереди имела
длину $x_3,\ldots,$ последняя заявка в~очереди
имела\linebreak
 длину~$x_n$ (с~плотностью вероятностей $p_{r-1}(t;v,x_3,\ldots,x_r)$),
и~за время~$\Delta$ поступила заявка (с~вероятностью $\lambda\Delta$) длины~$u$
(с~плотностью вероятностей $b(u)$).
Заявка на приборе продолжает обслуживаться,
но ее длина становится равной~$x_1$, а~вновь
поступившая заявка занимает первое мес\-то в~очереди и~ее длина становится равной~$x_2$
(с~плот\-ностью вероятностей $d(x_2,x_1|u,v)$);
\item
в момент~$t$ в~системе находилось $(r\hm-1)$
заявок, причем заявка на приборе имела
длину~$v$, первая заявка в~очереди имела
длину $x_3,\ldots,$ последняя заявка в~очереди имела\linebreak
 длину~$x_n$ (с~плот\-ностью
вероятностей $p_{r-1}(t;v,x_3,\ldots,x_r)$),
и~за время~$\Delta$ поступила заявка (с~вероятностью $\lambda\Delta$) длины~$u$
(с~плот\-ностью вероятностей $b(u)$).
Поступившая заявка занимает прибор и~ее длина
становится равной~$x_1$, а~заявка,
обслуживавшаяся до поступления новой заявки,
занимает первое мес\-то в~очереди и~ее длина
становится равной~$x_2$ (с~плотностью вероятностей $d^*(x_1,x_2|u,v)$);
\item
в момент~$t$ в~системе находилось~$r$~заявок,
причем заявка на приборе имела длину $x_1\hm+\Delta$, первая заявка в~очереди
имела дли-\linebreak\vspace*{-12pt}

\pagebreak

\noindent
ну $x_2,\ldots,$ последняя заявка в~очереди имела длину~$x_r$ (с плотностью
вероятностей $p_r(t;x_1\hm+\Delta,x_2,\ldots,x_r)$).
\end{itemize}


Вероятности других событий равны $o(\Delta)$.
Применяя формулу полной вероятности, имеем:
\begin{multline*}
p_{r}\left(t+\Delta;x_1,\ld,x_r\right) ={}\\
\!{}=\!
\lambda\Delta \Bigg(\!
\int\limits_0^\infty \!\int\limits_0^\infty\!
\left[d\left(x_2,x_1|u,v\right) b(u) p_{r-1}\left(t;v,x_3,\ld,x_r\right)
+{}\right.\hspace*{-3.62766pt}
\\
\left.{}+
d^*\left(x_1,x_2|u,v\right) b(u) p_{r-1}\left(t;v,x_3,\ld,x_r\right)\right]
\,du dv \Bigg)
+ {}\\
{}+p_{r}\left(t;x_1+\Delta,x_2,\ld,x_r\right)\,,
\end{multline*}
откуда, перенося слагаемое
$p_r(t;x_1+\Delta,x_2,\ldots,x_{r})$ в~левую часть равенства, деля на~$\Delta$,
устремляя~$\Delta$ к~нулю и~учитывая стационарный режим функционирования системы,
получаем уравнение~\eqref{e3.3-mei}.


К системе уравнений~\eqref{e3.1-mei}--\eqref{e3.3-mei} 
нужно добавить начальные условия, которые удобно записать\linebreak \mbox{в~виде}:
\begin{align}
p_{1}(\infty) &= \lim\limits_{X\to \infty} p_{1}(X)
= 0\,; \label{e3.33-mei}
\\
p_{n}(\infty,x_2,\ld,x_r)
&= {}\notag\\
&\hspace*{-20mm}{}=\lim\limits_{X\to \infty} p_{n}\left(X,x_2,\ld,x_r\right)
= 0\,,\enskip
1 \le n \le r\,.
\label{e3.4-mei}
\end{align}
Как получаются соотношения~\eqref{e3.33-mei} и~\eqref{e3.4-mei},
показано в~\cite{n1}.
Оставшаяся неизвестной стационарная вероятность~$p_0$ отсутствия заявок в~системе
находится, как обычно, из условия нормировки:
\begin{equation}
\label{e3.6-mei}
\sum\limits_{n=0}^r p_n = 1\,, 
\end{equation}
где
$p_n=P_n(\infty,\ld,\infty)$, $1 \hm\le n\hm \le r$,~---
стационарная вероятность наличия в~системе~$n$~заявок.

Как и~в случае системы бесконечной емкости,
полученные соотношения~\eqref{e3.1-mei}--\eqref{e3.6-mei} позволяют
теоретически последовательно по~$n$
находить стационарные плотности вероятностей $p_n(x_1,\ldots,x_{n})$.
Однако на практике такие расчеты связаны с~серьезными вычислительными сложностями.


Как показано, например, в~\cite{n3}, в~практических случаях
бывает достаточно знать только маргинальные плотности
\begin{multline*}
%\label{(2.1)}
p_{n}(x) = \mathop{\int\cd\int}\limits_{x_2,\ld,x_n>0}
p_{n}\left(x,x_2\ld,x_n\right)\,dx_2\cdots dx_n\,,
\\ 
2 \le n \le r\,.
\end{multline*}

Интегрируя~\eqref{e3.2-mei} и~\eqref{e3.3-mei} по
$x_2,\ldots ,x_r$ в~пределах от нуля до бесконечности и~вспоминая равенство~\eqref{e3.1-mei}, 
получаем следующую систему интегродифференциальных уравнений
для $p_{n}(x)$, $1 \hm\le n \hm\le r$:
\begin{align}
-p'_{n}(x) &= a_n(x) - \lambda p_{n}(x) +
\int\limits_0^\infty K_n(x,v) p_{n}(v)\,dv \,,\notag\\ 
&\hspace*{35mm}1  \le n \le r-1\,; \label{e3.7-mei}\\
-p'_{r}(x) &= a_r(x)\,,  \label{e3.7-1-mei}
\end{align}
где $a_1(x)=\tl \tb(x) p_0$ и
\begin{multline*}
a_{n}(x) = \lambda \Bigg( \!\int\limits_0^\infty\!
p_{n-1}(v)\,dv \! \int\limits_0^\infty\!
b(u)\,du \!\int\limits_0^\infty\!
\left[d(y,x|u,v) +{}\right.\\
\left.{}+ d^*(x,y|u,v)\right] \,dy
\Bigg)\,,\enskip 1 \le n \le r\,;
\end{multline*}

\vspace*{-12pt}

\noindent
\begin{multline*}
%\label{(2.1)}
K_n(x,v) = \lambda \int\limits_0^\infty b(u)\,du
\Bigg( d(x|u,v) +{}\\
{}+
\int\limits_0^\infty \!\left[d(x,y|u,v) + d^*(y,x|u,v)\right]
\,dy\! \Bigg),
\enskip 1 \le n \le r-1.
\end{multline*}
Начальные условия для уравнений~\eqref{e3.7-mei} и~\eqref{e3.7-1-mei}
по аналогии с~\eqref{e3.33-mei} запишем в~виде:
\begin{equation}
\label{e3.8-mei}
p_{n}(\infty) = \lim\limits_{X\to \infty} p_{n}(X)
= 0 \,,\enskip 1 \le n \le r\,. 
\end{equation}


Решать систему~\eqref{e3.7-mei} и~\eqref{e3.7-1-mei}
с~начальными условиями~\eqref{e3.8-mei} можно различными способами.
Воспользуемся методом, предложенным в~\cite{n1}.
Прежде всего заметим, что из~\eqref{e3.7-1-mei} немедленно следует, что
\begin{equation*}
%\label{(3.7-1)}
p_r(x) = \int\limits_x^\infty a_r(u) \,du\,.
\end{equation*}
Решение уравнений~\eqref{e3.7-mei} будем искать в~виде:
\begin{equation}
\label{e4.1-mei}
p_n(x) = e^{\lambda x} q_n(x)\,,\enskip 1 \le n \le r-1\,.
\end{equation}
Подставляя в~\eqref{e3.7-mei} вместо $p_n(x)$ ее выражение
по формуле~\eqref{e4.1-mei}, получаем новое интегродифференциальное уравнение:
\begin{multline*}
- q'_n(x) = e^{-\lambda x} a_n(x) +
\int\limits_0^\infty e^{\lambda v} e^{-\lambda x} K_n(x,v) q_n(v)\, dv\,, \\
1 \le n \le r-1\,.
\end{multline*}
Интегрируя последнее равенство по~$x$ в~пределах от~$y$ до~$\infty$ и~учитывая
начальное условие~\eqref{e3.8-mei}, получаем
интегральное уравнение Фредгольма 2-го рода:
\begin{multline}
\label{e2.1n-mei}
q_n(y)= b_n(y) + \int\limits_0^\infty
G_n(y,v) q_n(v)\, dv \,, \\ 
1 \le n \le r-1\,,
\end{multline}
где
\begin{align*}
b_n(y) &= \int\limits_y^\infty e^{-\lambda x} a_n(x)\, dx\,; \\
G_n(y,v) &= \int\limits_y^\infty e^{\lambda (v-x)} K_n(x,v)\, dx\,.                              %       (4.2)
\end{align*}
Отметим, что свободный член $b_n(y)$ и~ядро
$G_n(y,v)$ интегрального уравнения являются неотрицательными функциями.
Далее для расчета $q_n(y)$ можно применить
подходящий метод решения интегральных уравнений Фредгольма 2-го рода
(см., например,~[6--8]).

В~некоторых частных случаях решения уравнений~\eqref{e2.1n-mei}  могут быть выписаны в~явном виде.
 Например, это возможно в~случае, когда  известны сепарабельные аппроксимации для функций
 $d(x,y|u,v)$, $d^*(x,y|u,v)$, $d_0(x|u,v)$ и~$d_0^*(x|u,v)$,
 т.\,е.\ разложения вида:
\begin{align*}
d(x,y|u,v)&=\sum\limits_{i=1}^{N_1} \alpha_{1i}(x)\beta_{1i}(y)\gamma_{1i}(u)\delta_{1i}(v)\,;
\\
d^*(x,y|u,v)&=\sum\limits_{i=1}^{N_2} \alpha_{2i}(x)\beta_{2i}(y)\gamma_{2i}(u)\delta_{2i}(v)\,;
\\
d_0(x|u,v)&=\sum\limits_{i=1}^{N_3} \alpha_{3i}(x)\gamma_{3i}(u)\delta_{3i}(v)\,;
\\
d_0^*(x|u,v)&=\sum\limits_{i=1}^{N_4} \alpha_{4i}(x)\gamma_{4i}(u)\delta_{4i}(v)\,,
\end{align*}
где $N_1$, $N_2$, $N_3$ и~$N_4$~--- некоторые натуральные чис\-ла,
а $\alpha_{ij}(x)$, $\beta_{ij}(x)$, $\gamma_{ij}(x)$ и~$\delta_{ij}(x)$~--- некоторые
известные функции.
Тогда решение уравнения~\eqref{e2.1n-mei} при фиксированном~$n$
сводится к~решению системы линейных уравнений относительно
$(N_1\hm+N_2\hm+N_3\hm+N_4)$ неизвестных.


\section{Стационарные временные характеристики}

\subsection{Стационарное распределение времени ожидания начала обслуживания}

Для того чтобы найти показатели функционирования СМО,
связанные с~временем пребывания в~системе, нужно прежде
всего найти ПЛС периода занятости (ПЗ) системы.

Обозначим через $u_n(s;x)$, $1 \hm\le n \hm\le r$,  ПЛС
времени до того момента, когда в~системе впервые останется $(n\hm-1)$ заявок,
при условии что на приборе начала обслуживаться заявка
(остаточной) длины~$x$ и~в~системе находится~$n$~заявок.

Учитывая, что по принятому соглашению поступающая в~заполненную
систему заявка сразу теряется, ПЛС  $u_r(s;x)$ удовлетворяет уравнению:
\begin{equation}
\label{t1-mei}
u_r(s;x)=e^{-s x}\,.
\end{equation}
Воспользовавшись свойствами ПЛС, получаем, что $u_{n}(s;x)$ равно:
\begin{itemize}
\item $e^{-s x}$, если до момента времени~$x$
окончания обслуживания заявки на приборе
новая заявка не поступила (с~вероятностью $e^{-\lambda x}$);

\item  $e^{-s t}$, если в~момент $0<t<x$
поступила новая заявка и~обе заявки покинули
систему (с плотностью
вероятностей
$ \lambda e^{-\lambda t}
\int\nolimits_0^\infty d_0(y,x\hm-t)b(y)\,dy$);

\item  $e^{-s t} u_{n}(s;v)$, если в~момент времени
$0\hm<t\hm<x$ поступила новая заявка длины~$y$, одна из двух
заявок (поступившая заявка или
заявка на приборе) покинула систему, а~оставшаяся приняла длину~$v$ и,~значит,
время до того момента, как в~системе останется $(n\hm-1)$ заявок,
равно $u_{n}(s;v)$
(плотность вероятности данного события равна
$ \lambda e^{-\lambda t}
\int\nolimits_0^\infty d(v|y,x-t) b(y)\, dy$);

\item  $e^{-s t} u_{n+1}(s;w) u_{n}(s;v)$, если в~момент
времени $0\hm<t\hm<x$ поступила новая заявка длины~$y$,
обе заявки остаются в~системе (новая встает в~очередь),
причем длина новой заявки становится равной~$v$,
а~на приборе --- $w$ (с~плот\-ностью вероятностей
$
\lambda e^{-\lambda t} \int\nolimits_0^\infty
d(v,w|y,x-t) b(y)\, dy$);

\item  $e^{-s t} u_{n+1}(s;v) u_{n}(s;w)$, если в~момент
времени $0\hm<t\hm<x$ поступила новая заявка длины~$y$,
обе заявки остаются в~системе (новая встает в~очередь),
причем длина новой заявки становится равной~$w$,
а~на приборе --- $v$ (с~плот\-ностью вероятностей
$\lambda e^{-\lambda t} \int\nolimits_0^\infty d^*(v,w|y,x-t)\, b(y)\, dy$).
\end{itemize}


По формуле полной вероятности окончательно получаем:
\begin{multline*}
u_{n}(s;x)=e^{-(s+\lambda) x} + {}\\
{}+\int\limits_0^x \lambda e^{-(\lambda+s) t}\,dt
\int\limits_0^\infty d_0(y,x-t)b(y)\,dy+{}
\\
{}+
\int\limits_0^x \lambda e^{-(\lambda+s) t} \,dt \int\limits_0^\infty
 u_{n}(s;v) \, dv \int\limits_0^\infty d(v|y,x-t)\, b(y)\, dy
+{}\\
{}+
\int\limits_0^x \lambda e^{-(\lambda+s) t} \, dt
\int\limits_0^\infty u_{n+1}(s;w)\,dw
\int\limits_0^\infty u_{n}(s;v) \,dv\times{}
\end{multline*}

\noindent
\begin{multline}
{}\times{}
\int\limits_0^\infty d(v,w|y,x-t) b(y)\, dy
+ 
\int\limits_0^x \lambda e^{-(\lambda+s) t} \,dt\times{}\\
{}\times
\int\limits_0^\infty u_{n+1}(s;v) \,dv
\int\limits_0^\infty  u_{n}(s;w)\, dw\times{}\\
{}\times
\int\limits_0^\infty d^*(v,w|y,x-t) b(y)\, dy\,,\\
 1 \le n \le r-1\,.
 \label{t2-mei}
\end{multline}

Система уравнений~\eqref{t1-mei}--\eqref{t2-mei} решается
рекуррентно, начиная с~$n\hm=r\hm-1$.

Зная значения $u_{n}(s;x)$, можно найти основные стационарные
временн$\acute{\mbox{ы}}$е характеристики заявок.
Пусть в~начальный момент в~системе находится~$n$~заявок, $1 \hm\le n \hm\le r-1$,
на приборе обслуживается заявка длины~$y$ и~в~этот момент
в~систему поступает заявка длины~$x$.
Обозначим через $w_n(s;x,y)$ ПЛС времени ожидания
начала обслуживания этой заявки. В~соответствии 
с~дисциплиной обслуживания имеет место равенство:
\begin{multline*}
w_n(s;x,y)=\int\limits_0^\infty \int\limits_0^\infty d^*(v,w|x,y)\, dv dw
+{}\\
{}+\int\limits_0^\infty\! d_0(v|x,y) \,dv
+ \int\limits_0^\infty \!\int\limits_0^\infty \!u_{n+1}(s;w) d(v,w|x,y)\, dv dw.
\end{multline*}
Заметим, что вероятность того, что поступающая заявка 
длины~$x$ будет потеряна при поступлении в~систему, равна:
\begin{multline*}
\pi(x)= {}\\
{}=\int\limits_0^\infty  \sum\limits_{n=1}^{r-1} p_n(y)
\left( d_0(x,y)+\int\limits_0^\infty d_0^*(w|x,y)\, dw\right)\,dy
+{}\\
{}+ \int\limits_0^\infty p_r(y)\, dy\,.
\end{multline*}
Тогда ПЛС $w(s)$ стационарного распределения времени ожидания начала обслуживания принятой
в~систему заявки определяется формулой:
\begin{multline*}
w(s) =\fr{1}{1-\pi} \biggl (
p_0 + {}\\
{}+\int\limits_0^\infty  \sum\limits_{n=1}^{r-1} p_n(y)\, dy
\int\limits_0^\infty b(x)  w_n(s;x,y)\, dx
\biggl )\,, 
\end{multline*}
где $\pi=\int_0^\infty \pi(x) b(x) \,dx$~--- безусловная вероятность потери заявки.


\subsection{Стационарное распределение времени пребывания заявки в~системе}


Распределение полного времени пребывания заявки в~системе вычисляется
несколько сложнее из-за того, что заявка, попавшая на прибор,
может покидать его и~возвращаться на него обратно,
менять свою длину, а~также уйти из системы недообслуженной.

Остановимся на нахождении следующих характеристик, которые
понадобятся в~дальнейшем:
\begin{itemize}
\item стационарное распределение времени пребывания на приборе заявки,
которая была обслужена до конца (с~учетом возможных смен длин
и~прерываний), при условии что в~момент поступления на прибор
ее длина равнялась~$x$, а~в~очереди было~$n$, $0\hm\le n\hm\le r\hm-1$, других заявок.
Через $V_{1,n}(s;x)$ будем обозначать ПЛС этого распределения;

\item стационарное распределение времени пребывания на приборе заявки,
которая могла быть и~не обслужена до конца (с~учетом возможных смен длин
и~прерываний), при условии что в~момент поступления на прибор
ее длина равнялась~$x$, а~в~очереди было~$n$, $0\hm\le n\hm\le r-1$, других заявок.
Через $V_{2,n}(s;x)$ будем обозначать ПЛС этого распределения.
\end{itemize}

Отметим, что здесь подразумевается, что время пребывания поступившей
на прибор заявки включает все времена,
на которые ее обслуживание было прервано.

Ввиду того что поступающая в~заполненную сис\-те\-му заявка теряется, выпишем
$ V_{1,r-1}(s;x)\hm=e^{-s x}$.
Далее, воспользовавшись свойством ПЛС, находим, что $V_{1,n}(s;x)$  равно:
\begin{itemize}
\item  $e^{-s x}$, если до момента времени~$x$
окончания обслуживания заявки на приборе
новая заявка не поступила (с~вероятностью~$e^{-\lambda x}$);
\item  $e^{-s t}V_{1,n}(s;w)$, если в~момент времени
$0\hm<t\hm<x$ поступила новая заявка длины~$y$,
изменила длину заявки на приборе на~$w$, а~сама\linebreak покинула систему 
(с~плотностью вероятностей~$\lambda e^{-\lambda t}
\int\nolimits_0^\infty d^*_0(w|y,x-t) b(y)\, dy$);

\item $e^{-s t}V_{1,n+1}(s;w)$, если в~момент
времени $0\hm<t\hm<x$ поступила новая заявка длины~$y$, которая
встала на первое место в~очереди, причем новая заявка
получила новую длину~$v$, а~заявка на приборе новую
длину~$w$ (с~плотностью вероятностей
$\lambda e^{-\lambda t} \int\nolimits_0^\infty d(v,w|y,x-t) b(y)\, dy$);

\item  $e^{-s t}u_{n+2}(s;v)V_{1,n}(s;w)$, если в~момент
времени $0\hm<t\hm<x$ поступила новая заявка длины~$y$, которая встала на 
прибор, получив \mbox{новую} длину~$v$, а~заявка с~прибора вытеснена на первое место 
в~очереди и~получила новую длину~$w$ (с~плотностью вероятностей
$\lambda e^{-\lambda t} \int\nolimits_0^\infty
d^*(v,w|y,x-t) b(y)\, dy $).
\end{itemize}

Воспользовавшись снова формулой полной вероятности, получаем, что
уравнение для определения ПЛС $V_{1,n}(s;x)$ имеет следующий вид:
\begin{multline*}
\!\!V_{1,n}(s;x)= e^{-(\lambda+s)x} +\int\limits_0^\infty 
V_{1,n+1}(s;w) f(s;x,w) \, dw
+{}\\
{}+\int\limits_0^\infty V_{1,n}(s;w) g_{n+2}(s;x,w) \, dw\,, \enskip 0\le n\le r-2\,,
\end{multline*}
где
\begin{multline*}
f(s;x,w)= {}\\
{}=\int\limits_0^x \lambda e^{-(\lambda+s) t}\,dt \int\limits_0^\infty  \, dv
\int\limits_0^\infty d(v,w|y,x-t)\, b(y)\, dy\,; 
\end{multline*}

\vspace*{-12pt}

\noindent
\begin{multline*}
g_{n+2}(s;x,w) = \int\limits_0^x \lambda e^{-(\lambda+s) t}\,dt
\int\limits_0^\infty u_{n+2}(s;v)  \, dv\times{}\\
{}\times
\int\limits_0^\infty d^*(v,w|y,x-t)\, b(y)\, dy
+{}\\
{}+
\int\limits_0^x \lambda e^{-(\lambda+s) t}\,dt \int\limits_0^\infty d^*_0(w|y,x-t)\, b(y)\, dy\,.
\end{multline*}

Уравнение для определения $V_{2,n}(s;x)$ получается  аналогичным образом.
Действительно, $V_{2,r-1}(s;x)\hm=e^{-s x}$.
Далее, ПЛС $V_{2,n}(s;x)$ равно:
\begin{itemize}
\item $e^{-s x}$, если до момента времени~$x$
окончания обслуживания заявки на приборе
новая заявка не поступила (с~вероятностью~$e^{-\lambda x}$);

\item  $e^{-s t}$, если в~момент $0\hm<t\hm<x$
поступила новая заявка и~она вместе с~заявкой на приборе покинула
систему (с~плотностью вероятностей~$\lambda e^{-\lambda t}
\int\nolimits_0^\infty d_0(y,x-t)b(y)\,dy$);

\item  $e^{-s t}$, если в~момент времени
$0\hm<t\hm<x$ поступила новая заявка длины~$y$, 
сама встала на прибор, а~заявка с~прибора покинула систему 
(с~плотностью вероятностей~$\lambda e^{-\lambda t}
\int\nolimits_0^\infty d_0(v|y,x-t) b(y)\, dy$);

\item $e^{-s t}V_{2,n}(s;w)$, если в~момент времени
$0\hm<t\hm<x$ поступила новая заявка длины~$y$,
изменила длину заявки на приборе на~$w$, 
а~сама\linebreak покинула систему (с~плотностью вероятностей~$\lambda e^{-\lambda t}
\int\nolimits_0^\infty d^*_0(w|y,x-t) b(y)\, dy$);

\item  $e^{-s t}V_{2,n+1}(s;w)$, если в~момент
времени $0\hm<t\hm<x$ поступила новая заявка длины~$y$, которая
встала на первое место в~очереди, причем новая заявка
получила новую длину~$v$, а~заявка на приборе новую
длину~$w$ (с~плотностью вероятностей~$\lambda e^{-\lambda t}
\int\nolimits_0^\infty d(v,w|y,x-t) b(y)\, dy$);

\item $e^{-s t}u_{n+2}(s;v)V_{2,n}(s;w)$, если в~момент
времени $0\hm<t\hm<x$ поступила новая заявка длины~$y$, которая встала на прибор, 
получив новую длину~$v$, а~заявка с~прибора вытеснена на первое место 
в~очереди и~получила новую длину~$w$ (с~плотностью вероятностей~$\lambda e^{-\lambda t}
\int\nolimits_0^\infty d^*(v,w|y,x-t) b(y)\, dy$).
\end{itemize}

Воспользовавшись снова формулой полной вероятности, получаем, что
уравнение для определения ПЛС $V_{2,n}(s;x)$ имеет следующий вид:

\noindent
\begin{multline*}
\!\!V_{2,n}(s;x)=h(s,x)+ \int\limits_0^\infty V_{2,n+1}(s;w) f(s;x,w) \, dw
+{}\\
{}+ \int\limits_0^\infty V_{2,n}(s;w) g_{n+2}(s;x,w) \, dw\,,\enskip 0\le n\le r-2\,,
\end{multline*}
где

\noindent
\begin{multline*}
h(s,x)= e^{-(\lambda+s) x}+{}\\
{}+\int\limits_0^x \lambda e^{-(\lambda+s) t}\,dt
\int\limits_0^\infty d_0(y,x-t)b(y)\,dy
+ {}\\
{}+
\int\limits_0^x \lambda e^{-(\lambda+s) t}\,dt
\int\limits_0^\infty \, dv
\int\limits_0^\infty d_0(v|y,x-t) b(y)\, dy\,.
\end{multline*}

Решение полученных уравнений осуществляется рекуррентным образом,
начиная с~$n\hm=r\hm-1$.
Естественно, ПЛС безусловных распределений получаются усреднением
 $V_{1,n}(s;x)$ и~$V_{2,n}(s;x)$ по распределению длины заявки $B(x)$.

Наконец, перейдем к~нахождению полного времени пребывания заявки 
в~системе. Будем различать два случая: первый~--- когда заявка не может
уходить из системы недообслуженной; второй~--- когда заявка на
приборе может покинуть систему не обслуженной до конца. В~обоих
случаях, как обычно, полное время пребывания заявки в~системе
складывается из времени ожидания заявкой начала обслуживания 
и~времени пребывания заявки на приборе (которое включает времена
прерываний обслуживания).

В первом случае ПЛС стационарного распределения полного времени
пребывания в~системе поступающей заявки длины~$x$ обозначим через
$V_1(s;x)$, во втором~--- через $V_2(s;x)$.

\pagebreak

Рассмотрим первый случай.

Во-первых, заявка длины~$x$ может с~вероят\-ностью~$p_0$ поступить
в~свободную систему, и~тогда время ее пребывания в~системе будет совпадать с~временем 
ее пребывания на приборе (с~учетом прерываний).

Во-вторых, с~плотностью вероятностей $p_n(y)$ поступающая заявка
длины~$x$ может застать в~сис\-те\-ме $1 \hm\le n \hm\le r-1$ заявок,
причем на приборе будет находиться заявка длины~$y$. 
В~этом случае возможны следующие варианты:
\begin{itemize}
\item либо с~вероятностью $d_0(v|x,y)\hm+d^*(v,w|x,y)$ поступающая заявка
встанет на прибор, причем ее длина станет равной~$v$ и~тогда полное время
ее пребывания в~системе будет совпадать 
с~временем ее пребывания на приборе (с~учетом прерываний);

\item либо с~вероятностью $d(v,w|x,y)$ поступающая заявка станет на первое место 
в~очереди, получит новую длину~$v$, а~заявка на приборе~--- новую длину~$w$; 
при этом время пребывания в~системе поступившей заявки будет равно сумме двух 
времен: времени до того момента, когда в~системе снова станет~$n$~заявок, 
и~времени пребывания на приборе (с~учетом прерываний) заявки длины~$v$.
\end{itemize}

Применяя формулу полной вероятности, приходим к~следующему выражению для ПЛС $V_1(s;x)$
стационарного распределения полного времени пребывания принятой заявки в~систему, в~которой
не допускается уход заявок недообслуженными:
\begin{multline}
\label{eq4-mei}
V_1(s;x)= \fr{1}{1-\pi}
\Biggl (
p_0 V_{1,0}(s;x) +{}
\\ 
{}+
\int\limits_0^\infty  \sum\limits_{n=1}^{r-1} p_n(y)
\Biggl [
\int\limits_0^\infty V_{1,n-1}(s;v)
\Biggl ( d_0(v|x,y)+{}\\
{}+\int\limits_0^\infty d^*(v,w|x,y)\,dw \Biggr ) \,dv
\Biggr ]\, dy
+{}\\
\int\limits_0^\infty  \sum\limits_{n=1}^{r-1} p_n(y)
\Biggl [
\int\limits_0^\infty 
\int\limits_0^\infty u_{n+1}(s;w) \times{}\\
{}\times V_{1,n-1}(s;v) d(v,w|x,y) \,dv  dw
\Biggr ]\,dy
\Biggr )\,.
\end{multline}

Наконец, ПЛС $V_1(s)$ стационарного распределения полного
времени пребывания в~системе заявки произвольной длины
получается усреднением $V_1(s;x)$ по распределению длины заявки $B(x)$ и~равно
$V_1(s)\hm=\int_0^\infty V_1(s;x)b(x)\,dx$.
Выражение для $V_2(s;x)$ получается путем замены
в соответству\-ющих местах формулы~\eqref{eq4-mei} $V_{1,n}(s;x)$
на $V_{2,n}(s;x)$.

\section{Заключение}

В заключение скажем несколько слов об условии существования
стационарного режима.
Для рассмотренной системы общего необходимого и~достаточного
условия его существования выписать не удается.
Оно зависит от конкретных параметров
системы и~в~каждом отдельном случае нуждается в~специальном исследовании.
Конечность среднего времени обслуживания
является только необходимым условием
и,~даже несмотря на присутствие ограничения на размер очереди,
не является достаточным\footnote{Например,
если положить $d(x,y|u,v) = e^{-v} b(x)b(y e^{-v})$,
$d^*(x,y|u,v)\hm=0$, $d(x|u,v)\hm=0$, $d_0(u,v)\hm=0$, $u, v\hm>0$,
то среднее время до того момента, когда в~системе останется
$(r\hm-2)$ заявки, при условии что в~начальный момент в~системе
было $(r\hm-1)$ заявок, без дополнительных
ограничений на функцию $b(x)$ может быть равно бесконечности. При этом,
учитывая пуассоновость входящего потока,
с~ненулевой вероятностью система
переходит в~состояние $(r\hm-2)$
и,~вообще говоря, с~ненулевой вероятностью может успеть выполнить
до прихода очередной заявки
любую находящуюся в~ней работу (при условии ее конечности), т.\,е.\
полностью опустошиться.}.


{\small\frenchspacing
 {%\baselineskip=10.8pt
 \addcontentsline{toc}{section}{References}
 \begin{thebibliography}{9}


\bibitem{n1} 
\Au{Мейханаджян Л.\,А., Милованова~Т.\,А., Печинкин~А.\,В., Разумчик~Р.\,В.}
Стационарные вероятности состояний в~системе обслуживания 
с~инверсионным порядком обслуживания и~обобщенным вероятностным
приоритетом~// Информатика и~её применения, 2014. Т.~8. Вып.~3.
С.~16--26.

\bibitem{n2} %2
\Au{Мейханаджян Л.\,А., Милованова~Т.\,А., Разумчик~Р.\,В.}
Время ожидания в~системе обслуживания с~инверсионным порядком
обслуживания и~обобщенным вероятностным приоритетом~// Информатика 
и~её применения, 2015. Т.~9. Вып.~2. С.~14--22.


%\bibitem{shrage} {\it Schrage L.} A proof of the
%optimality of the shortest remaining processing
%time discipline //
%Oper.\ Res., 1968. Vol.~16. P.~687--690.
%


\bibitem{bsev} %3
\Au{Севастьянов Б.\,А.}
Эргодическая теорема для марковских процессов и~ее приложение 
к~телефонным системам с~отказами~// ТВП, 1957. Т.~2. Вып.~1. С.~106--116.

\bibitem{ppav}  %4
\Au{Бочаров  П.\,П., Печинкин~А.\,В.}
Теория массового обслуживания.~--- М.: РУДН, 1995. 529~с.

\bibitem{n3} %5
\Au{Meykhanadzhyan L., Razumchik~R.}
New scheduling policy for estimation of stationary performance
characteristics in single server queues with inaccurate job size
information~// 30th European Conference on Modelling and Simulation Proceedings.~--- 
Dudweiler, Germany: Digitaldruck Pirrot GmbHP, 2016. P.~710--716.



%\bibitem{aaa1} {\it Нагоненко В.\ А.}
%О характеристиках одной нестандартной системы
%массового обслуживания.~I, II //
%Изв.\ АН СССР. Технич.\ кибернет., 1981.
%№~1. С.~187--195; №~3. С.~91--99.
%
%\bibitem{aaa2} {\it Печинкин А.\ В.} Об одной
%инвариантной системе массового обслуживания //
%Math.\ Operationsforsch.\ und Statist.
%Ser.\ Optimization, 1983. Vol.~14. №~3. S.~433--444.
%
%\bibitem{aaa3} {\it Нагоненко В.\ А., Печинкин А.\ В.}
%О большой загрузке в~системе с~инверсионным
%обслуживанием и~вероятностным приоритетом //
%Изв.\ АН СССР. Технич.\ кибернет., 1982. №~1. С.~86--94.
%
%\bibitem{aaa4} {\it Нагоненко В.\ А., Печинкин А.\ В.}
%О малой загрузке в~системе с~инверсионным порядком
%обслуживания и~вероятностным приоритетом //
%Изв.\ АН СССР. Технич.\ кибернет., 1984. №~6. С.~82--89.
%
%\bibitem{av1}{\it Печинкин А.\ В., Стальченко И.\ В.}
%Система $MAP/G/1/\infty$ с~инверсионным порядком
%обслуживания и~вероятностным приоритетом,
%функционирующая в~дискретном времени //
%Вестник Российского университета дружбы народов.
%Сер.\ Математика. Информатика. Физика, 2010.
%№~2. С.~26--36.
%
%\bibitem{av2}{\it Касконе А., Манзо Р.,
%Печинкин А.\ В., Салерно С.}
%Система $MAP/G/1/\infty$ в~дискретном
%времени с~инверсионной вероятностной дисциплиной
%обслуживания //
%Автоматика и~телемеханика, 2010. №~12. С.~57--69.
%
%\bibitem{av3}{\it Милованова Т.\ А., Печинкин А.\ В.}
%Стационарные характеристики системы обслуживания с
%инверсионным порядком обслуживания, вероятностным
%приоритетом и~гистерезисной политикой //
%Информатика и~ее применения, 2013. Т.~7. Вып.~1. С.~22--36.
%
%
\bibitem{jerri} %6
\Au{Jerri A.}
Introduction to integral equations with
applications.~--- New York, NY, USA: John Wiley \& Sons, 1999. 272~p.
%P. 433.

\bibitem{wh} %7
\Au{Press W.\,H., Teukolsky~S.\,A.,
Vetterling~W.\,T., Flannery~B.\,P.}
Numerical recipes:
The art of scientific computing.~--- 3rd ed.~---
 2007. 1256~p.

\bibitem{adav} %8
\Au{Полянин А.\,Д., Манжиров~А.\,В.}
Справочник по интегральным уравнениям.~---
Бока-Ратон\,--\,Лондон: Chapman \& Hall, CRC Press, 2008. 1108~p.


\end{thebibliography}

 }
 }

\end{multicols}

\vspace*{-6pt}

\hfill{\small\textit{Поступила в~редакцию 19.04.16}}

\vspace*{4pt}

%\newpage

%\vspace*{-24pt}

\hrule

\vspace*{2pt}

\hrule

\vspace*{-2pt}



\def\tit{STATIONARY CHARACTERISTICS OF~THE~FINITE CAPACITY QUEUEING SYSTEM
WITH~INVERSE SERVICE ORDER AND~GENERALIZED
PROBABILISTIC PRIORITY}

\def\titkol{Stationary characteristics of~the~finite capacity queueing system
with~inverse service order and~generalized
probabilistic priority}

\def\aut{L.\,A.~Meykhanadzhyan}

\def\autkol{L.\,A.~Meykhanadzhyan}

\titel{\tit}{\aut}{\autkol}{\titkol}

\vspace*{-9pt}

\noindent
Peoples' Friendship University of Russia,
6~Miklukho-Maklaya Str., 
Moscow 117198, Russian Federation

\def\leftfootline{\small{\textbf{\thepage}
\hfill INFORMATIKA I EE PRIMENENIYA~--- INFORMATICS AND
APPLICATIONS\ \ \ 2016\ \ \ volume~10\ \ \ issue\ 2}
}%
 \def\rightfootline{\small{INFORMATIKA I EE PRIMENENIYA~---
INFORMATICS AND APPLICATIONS\ \ \ 2016\ \ \ volume~10\ \ \ issue\ 2
\hfill \textbf{\thepage}}}

%\vspace*{3pt}


\Abste{Consideration is given to the $M/G/1/(r-1)$
queueing system with LIFO (last in, first out) preemptive
generalized probabilistic priority policy.
It is assumed that customer's service time becomes known
upon its arrival at the system
and at any time instant remaining service times
of all customers present in the system
are available. On arrival of a~customer at a~nonempty system,
its service time is compared to the (remaining) service time of the customer in
service and one of the following events occurs:
both customers leave the system at once,
one of the customers leaves the system (the other
occupies the server), or both customers stay in the system (one occupies the server,
the other~--- one place in the queue). Those customers which stay in the system
acquire new service time according to a~known
distribution, which can depend on their initial service times.
Arriving customers which find the queue full, leave the system and have no influence on it.
Analytical expressions for the computation of the
joint stationary distribution of the number of customers
in the system and the remaining service time of the customer
in the server, of the  busy period and the stationary sojourn time
(in terms of Laplace--Stieltjes transform) are proposed.}


\KWE{queueing system; special discipline; LIFO; probabilistic priority}



\DOI{10.14357/19922264160214}

\vspace*{-16pt}

\Ack

\vspace*{-2pt}

\noindent
The work is supported by the
Russian Foundation for Basic Research (project 15-07-03007).


  \vspace*{-1pt}

  \begin{multicols}{2}
  

  

\renewcommand{\bibname}{\protect\rmfamily References}
%\renewcommand{\bibname}{\large\protect\rm References}



{\small\frenchspacing
 {%\baselineskip=10.8pt
 \addcontentsline{toc}{section}{References}
 \begin{thebibliography}{9}

\vspace*{-2pt}
\bibitem{n1-1} 
\Aue{Meykhanadzhyan, L.\,A., T.\,A.~Milovanova, A.\,V.~Pechinkin, 
and R.\,V.~Ra\-zum\-chik}. 2014.
Statsionarnye veroyatnosti so\-sto\-yaniy v~sisteme obslu\-zhi\-va\-niya 
s~inversionnym po\-ryad\-kom
ob\-slu\-zhi\-va\-niya i~obob\-shchen\-nym veroyatnostnym pri\-o\-ri\-te\-tom
[Stationary distribution in a~queueing system with inverse service order and
generalized probabilistic priority].
\textit{Informatika i~ee Primeneniya}~--- \textit{Inform.Appl.}
8(3):16--26.

\bibitem{n2-1} 
\Aue{Meykhanadzhyan, L.\,A., T.\,A.~Milovanova, and R.\,V.~Ra\-zum\-chik}. 2015.
Vremya ozhidaniya v~sis\-te\-me ob\-slu\-zhi\-va\-niya s~inversionnym poryadkom obsluzhivaniya 
i~obobshchennym veroyatnostnym prioritetom
[Stationary\linebreak waiting time in a~queueing system with inverse service order and
generalized probabilistic priority].
\textit{Informatika i~ee Primeneniya}~--- \textit{Inform.Appl.}
9(2):14--22.

\bibitem{bsev-1} %3
\Aue{Sevastyanov, B.\,A.} 1957.
Ergodicheskaya teorema dlya markovskikh protsessov i~ee prilozhenie k~telefonnym 
sistemam s~otkazami
[An ergodic theorem for markov processes and its application to telephone systems with refusals].
\textit{Teor. Veroyatnost. i Primenen.} 
[Probability Theory and Its Applications] 2(1):106--116.



\bibitem{ppav-1} %4
\Aue{Bocharov,  P.\,P., and A.\,V.~Pechinkin}. 1995.
\textit{Teoriya massovogo obsluzhivaniya} [Queueing theory].
Moscow: RUDN. 529~p.

\bibitem{n3-1} %5
\Aue{Meykhanadzhyan, L., and R.~Razumchik}. 2016.
New scheduling policy for estimation of stationary performance
characteristics in single server queues with inaccurate job size
information. \textit{30th European Conference on Modelling and Simulation Proceedings}.
Dudweiler, Germany: Digitaldruck Pirrot GmbHP. 710--716.



\bibitem{jerri-1} %6
\Aue{Jerri, A.} 1999.
\textit{Introduction to integral equations with applications}.
New York, NY: John Wiley \& Sons. 272~p.



\bibitem{wh-1} %7
\Aue{Press, W.\,H., S.\,A.~Teukolsky, W.\,T.~Vetterling,
and B.\,P.~Flannery}. 2007.
\textit{Numerical recipes:  
The art of Scientific computing}. 3rd ed. 1256~p.

\bibitem{adav-1} %8
\Aue{Polyanin, A.\,D., and A.\,V.~Manzhirov}. 2008.
\textit{Handbook of integral equations}.
Boca Raton\,--\,London: Chapman \& Hall,
CRC Press. 1108~p.
\end{thebibliography}

 }
 }

\end{multicols}

\vspace*{-7pt}

\hfill{\small\textit{Received April 19, 2016}}

\vspace*{-17pt}
   

\Contrl

\vspace*{-2pt}

\noindent
\textbf{Meykhanadzhyan Lusine A.} (b.\ 1990)~---
PhD student, Peoples' Friendship University of Russia, 6~Miklukho-Maklaya Str., 
Moscow 117198, Russian Federation; lameykhanadzhyan@gmail.com

 
\label{end\stat}


\renewcommand{\bibname}{\protect\rm Литература}





%%%%%%%%%%%%%%%%%%%%%%%%%%%%%%%%%%%%%%%%%%%%%%%

%\def\stat{rez}
{%\hrule\par
%\vskip 7pt % 7pt
\raggedleft\Large \bf%\baselineskip=3.2ex
Р\,Е\,Ц\,Е\,Н\,З\,И\,И \vskip 17pt
    \hrule
    \par
\vskip 6pt plus 6pt minus 3pt }

%\thispagestyle{headings} %с верхним колонтитулом
%\thispagestyle{myheadings} %с нижним колонтитулом, но в верхнем РЕЦЕНЗИИ

\def\tit{НОВАЯ КНИГА И.\,Н.~СИНИЦЫНА, А.\,С.~ШАЛАМОВА <<ЛЕКЦИИ ПО ТЕОРИИ 
ИНТЕГРИРОВАННОЙ ЛОГИСТИЧЕСКОЙ ПОДДЕРЖКИ>> (М.: ТОРУС ПРЕСС, 2012. 624~с.)}

%1
\def\aut{Д.ф.-м.н., профессор С.\,Я.~Шоргин}

\def\auf{\ }

\def\leftkol{\ % РЕЦЕНЗИИ
}

\def\rightkol{ \ } 

%\def\leftkol{\ } % ENGLISH ABSTRACTS}

%\def\rightkol{\ } %ENGLISH ABSTRACTS}

%\def\leftkol{РЕЦЕНЗИИ}

%\def\rightkol{РЕЦЕНЗИИ}

\titele{\tit}{\aut}{\auf}{\leftkol}{\rightkol}
\vspace*{-18pt}


     \label{st\stat}

     \begin{multicols}{2}
     {\small
     {\baselineskip=10.1pt
     

      В книге представлено системное изложение теоретических основ одного из новейших 
направлений в \mbox{об\-ласти} экономики послепродажного обслуживания изделий наукоемкой 
продукции (ИНП) длительного пользования~--- интегрированной логистической поддержки
(ИЛП). 
{\looseness=1

}

Приведены также результаты новых работ, выполненных в Институте проблем информатики 
Российской академии наук в рамках научного направления <<Информационные технологии и 
анализ сложных сис\-тем>>.
 {%\looseness=1

}
     
      Излагаемые в книге научные подходы позво\-ляют карди\-наль\-но реформировать 
существующие системы производства и эксплуатации ИНП путем создания и внед\-ре\-ния 
методов рационального и оптимального управ\-ле\-ния процессами расходования 
вре\-мен\-н$\acute{\mbox{ы}}$х, 
мате\-ри\-аль\-ных, трудовых и других ресурсов на всех стадиях жизненного цикла изделий (ЖЦИ) по 
критериям экономической целесообразности и эф\-фек\-тив\-ности.
  {\looseness=1

}
    
      В книге приведен краткий обзор причин возник\-новения и
      развития CALS-методологии как основы 
современных международных стандартов по созданию и функционированию глобальных 
ин\-фор\-ма\-ци\-он\-но-ком\-му\-ни\-ка\-ци\-он\-ных систем, ее ключевых возможностей и эффективности 
результатов ее использования. 
Авторы %\linebreak 
предлагают ряд научных обоснований для разработки 
единой теории проектирования и управления систем ИЛП для полноценного использования 
преимуществ %\linebreak
 суще\-ст\-ву\-ющей методологии, определяют \mbox{общую} структурную схему 
комплексной системы <<ИНП-СППО>> и необходимость разработки для ее описания 
гибридных стохастических моделей.
{%\looseness=1

}

%\columnbreak
      
      Книга состоит из пяти частей, где последовательно излагается материал по каждой из 
следующих тем: <<Интегрированная логистическая поддержка>>, <<Теория гибридных 
стохастических систем и компьютерная поддержка исследований и разработок>>, <<Основы 
математического моделирования, анализа и синтеза систем послепродажного обслуживания>>, 
<<Определение и анализ показателей экспортного потенциала ИНП при проектировании>>, 
<<Задачи управления поддержкой послепродажного обслуживания>>, а также 
<<Моделирование инвестиционных процессов ИЛП в условиях неравновесных финансовых 
рынков>>. 
   
      В конце каждой главы приведены выводы и даны вопросы и задания для 
самоконтроля. В~приложениях содержатся основные определения по программам работ по 
анализу ИЛП, логистическим базам данных и компьютерным решениям, эквивалентной статистической 
линеаризации нелинейных преобразований ИЛП, справочный материал, а также развернутые 
уравнения для вероятностных характеристик.


      \def\leftkol{РЕЦЕНЗИИ}

\def\rightkol{РЕЦЕНЗИИ} 

      
      Книга заинтересует широкий круг специалистов и может быть использована научными 
проектными организациями в сфере промышленного производства ИНП. Большое количество 
иллюстраций, примеров и вопросов, обращенных к читателю, позволяет использовать книгу 
также в качестве учебного пособия для студентов и аспирантов машиностроительных, 
транспортных и~других специальностей, а также для самостоятельного изучения. 
{%\looseness=-1

}

Книга 
представляет несомненный интерес для специалистов и студентов в области прикладной 
математики и информатики.
    

}

}
\end{multicols}

%\newpage

\def\stat{authorsrus}
{%\hrule\par
%\vskip 7pt % 7pt
\raggedleft\Large \bf%\baselineskip=3.2ex
О\,Б\ \ А\,В\,Т\,О\,Р\,А\,Х \vskip 17pt
    \hrule
    \par
\vskip 21pt plus 8pt minus 4pt }


\def\tit{\ }

\def\aut{\ }

\def\auf{\ }

\def\leftkol{\ } % ENGLISH ABSTRACTS}

\def\rightkol{ОБ АВТОРАХ} %ENGLISH ABSTRACTS}

\titele{\tit}{\aut}{\auf}{\leftkol}{\rightkol}
      
            \label{st\stat}



\vspace*{24pt}

\begin{multicols}{2}




\noindent
\textbf{Архипов Олег Петрович} (р.\ 1948)~---
кандидат технических наук, директор Орловского филиала Института проб\-лем информатики
Российской академии наук
%302025, г.Орел, Московское шоссе, д.137

\vspace*{3pt}

\noindent
\textbf{Бирюкова Татьяна Константиновна} (р.\ 1968)~---
кандидат фи\-зи\-ко-ма\-те\-ма\-ти\-че\-ских наук, старший научный сотрудник Института проб\-лем информатики
Российской академии наук

\vspace*{3pt}

\noindent 
\textbf{Бобков  Сергей Геннадьевич} (р.\ 1955)~---
доктор технических наук,  заведующий отделением На\-уч\-но-ис\-сле\-до\-ва\-тель\-ско\-го 
института системных исследований Российской академии наук
%117218, Москва, Нахимовский просп., 36, к.1 

\vspace*{3pt}

\noindent \textbf{Васильев Николай Семенович} (р.\ 1952)~--- доктор 
фи\-зи\-ко-ма\-те\-ма\-ти\-че\-ских наук, профессор, 
МГТУ им.\ Н.\,Э.~Баумана 
%, Москва 105005, 2-я Бауманская ул., д.~5,

\vspace*{3pt}

\noindent
\textbf{Гершкович Максим Михайлович} (р.\ 1968)~---
старший научный сотрудник Института проб\-лем информатики
Российской академии наук

\vspace*{3pt}

\noindent 
\textbf{Дьяченко Юрий Георгиевич} (р.\ 1958)~--- кандидат технических наук, 
старший научный сотрудник Института проб\-лем информатики
Российской академии наук

\vspace*{3pt}

\noindent 
\textbf{Ерошенко Александр Андреевич} (р.\ 1989)~--- аспирант кафедры 
математической статистики факультета вычисли\-тельной математики и кибернетики 
Московского государственного университета им.\ М.\,В.~Ломоносова
%119991, Москва ГСП-1, Ленинские горы, д.\ 1, стр. 52

\vspace*{3pt}
 
\noindent 
\textbf{Захаров Виктор Николаевич} (р.\ 1948)~--- 
доктор технических наук, доцент, ученый секретарь Института проб\-лем информатики
Российской академии наук

\vspace*{3pt}

\noindent
\textbf{Зейфман Александр Израилевич} (р.\ 1954)~---
доктор фи\-зи\-ко-ма\-те\-ма\-ти\-че\-ских наук, профессор, 
заведующий кафедрой Вологодского государственного университета; 
старший научный сотрудник Института проб\-лем информатики
Российской академии наук; главный научный сотрудник ИСЭРТ Российской академии наук

\vspace*{3pt}

\noindent
\textbf{Зыкин Сергей Владимирович} (р.\ 1959)~--- 
доктор технических наук, профессор, заведующий лабораторией Института математики 
им.\ С.\,Л.~Соболева Сибирского отделения Российской академии наук, Новосибирск 
%630090, пр.\ ак.\ Коптюга, 4 

\vspace*{4pt}

\noindent
\textbf{Киреев Владимир Иванович} (р.\ 1938)~---
доктор фи\-зи\-ко-ма\-те\-ма\-ти\-че\-ских наук, профессор Московского 
государственного горного университета
%Адрес: Россия, 119991, г. Москва, Ленинский проспект, д. 6

%\columnbreak

\vspace*{4pt}

\noindent
\textbf{Козеренко Елена Борисовна} (р.\ 1959)~---
кандидат филологических наук, заведующая лабораторией Института проб\-лем информатики
Российской академии наук

\vspace*{4pt}

\noindent
\textbf{Королев Виктор Юрьевич} (р.\ 1954)~--- доктор
фи\-зи\-ко-ма\-те\-ма\-ти\-че\-ских наук, профессор кафедры математической 
статистики факультета вычисли\-тельной математики и кибернетики 
Московского государственного университета; 
ведущий научный сотрудник Института проб\-лем информатики
Российской академии наук

\vspace*{4pt}

\noindent
\textbf{Коротышева Анна Владимировна} (р.\ 1988)~---
старший преподаватель Вологодского государственного университета

\vspace*{4pt}

\noindent 
\textbf{Кун Де Турк} (р.\ 1981)~--- научный сотрудник 
исследовательской группы SMACS факультета телекоммуникаций и обработки информации
Университета Гента, Бельгия
%В-9000 Гент, Бельгия

\vspace*{4pt}

\noindent
\textbf{Лупенцов Олег Сергеевич} (р.\ 1986)~---
аспирант Омского государственного института сервиса
%Омск 644043, ул.\ Певцова 13

\vspace*{4pt}

\noindent
\textbf{Лучко Олег Николаевич} (р.\ 1961)~---
кандидат педагогических наук, профессор, заведующий кафедрой 
Омского государственного института сервиса
%Омск 644043, ул.\ Певцова 13

\vspace*{4pt}

\noindent
\textbf{Малашенко Юрий Евгеньевич} (р.\ 1946)~---
доктор фи\-зи\-ко-ма\-те\-ма\-ти\-че\-ских наук, заведующий сектором 
Вычислительного центра им.\ А.\,А.~Дородницына Российской академии наук
%Адрес: 119333, Москва, ул. Вавилова, 40,

\vspace*{4pt}

\noindent
\textbf{Маньяков Юрий Анатольевич} (р.\ 1984)~---
кандидат технических наук, научный сотрудник Орловского филиала Института проб\-лем информатики
Российской академии наук
%302025, г.Орел, Московское шоссе, д.137

\vspace*{4pt}

\noindent
\textbf{Маренко Валентина Афанасьевна} (р.\ 1951)~---
кандидат технических наук, доцент, старший научный сотрудник 
Института математики им.\ С.\,Л.~Соболева Сибирского отделения Российской академии наук
%Новосибирск 630090, пр. ак. Коптюга, 4 

\vspace*{3pt}

\noindent 
\textbf{Морозов Евсей Викторович} (р.\ 1947)~--- доктор 
фи\-зи\-ко-ма\-те\-ма\-ти\-че\-ских, профессор, ведущий научный сотрудник 
Института прикладных математических исследований Карельского научного центра Российской
академии наук; 
%%185910 Россия, Республика Карелия, г.\ Петрозаводск, ул.\ Пушкинская, 11
профессор Петрозаводского государственного университета, Петрозаводск
%185910 Россия, Республика Карелия, г.\ Петрозаводск, пр.\ Ленина, 33

%\pagebreak

\vspace*{3pt}

\noindent
\textbf{Назарова Ирина Александровна} (р.\ 1966)~---
кандидат фи\-зи\-ко-ма\-те\-ма\-ти\-че\-ских наук, 
научный сотрудник Вычислительного центра им.\ А.\,А.~Дородницына Российской академии наук 
%Адрес: 119333, Москва, ул. Вавилова, 40

\vspace*{3pt}

\noindent
\textbf{Павлов Игорь Валерианович} (р.\ 1945)~--- 
доктор фи\-зи\-ко-ма\-те\-ма\-ти\-че\-ских наук, профессор МГТУ им.\ Н.\,Э.~Баумана 
%Москва 105005, 2-я Бауманская ул., д.~5 

%\pagebreak

\vspace*{3pt}

\noindent 
\textbf{Потахина Любовь Викторовна} (р.\ 1989)~--- аспирантка
Института прикладных математических исследований Карельского научного центра
Российской академии наук; 
%%185910 Россия, Республика Карелия, г.\ Петрозаводск, ул.\ Пушкинская, 11
инженер Петрозаводского государственного университета, Петрозаводск
%185910 Россия, Республика Карелия, г.\ Петрозаводск, пр.\ Ленина, 33

\vspace*{3pt}

\noindent 
\textbf{Рождественский Юрий Владимирович} (р.\ 1952)~--- 
кандидат технических наук, заведующий сектором Института проб\-лем информатики
Российской академии наук

\vspace*{3pt}

\noindent 
\textbf{Синицын Игорь Николаевич} (р.\ 1940)~--- доктор технических наук,
профессор, заслуженный деятель\linebreak\vspace*{-12pt}

\columnbreak

\noindent
 науки РФ, заведующий отделом Института проб\-лем информатики
Российской академии наук

\vspace*{7pt}


\noindent
\textbf{Сиротинин Денис Олегович} (р.\ 1984)~---
кандидат технических наук, научный сотрудник Орловского филиала Института проб\-лем информатики
Российской академии наук
%302025, г.Орел, Московское шоссе, д.137

\vspace*{7pt}

%\columnbreak

\noindent 
\textbf{Соколов  Игорь Анатольевич} (р.\ 1954)~--- академик (действительный член) Российской 
академии наук, доктор технических наук, директор Института проб\-лем информатики
Российской академии наук

\vspace*{7pt}

\noindent
\textbf{Степченков Юрий Афанасьевич} (р.\ 1951)~---
кандидат технических наук, заведующий отделом Института проб\-лем информатики
Российской академии наук

\vspace*{7pt}

\noindent
\textbf{Сурков Алексей Викторович} (р.\ 1978)~--- 
старший научный сотрудник На\-уч\-но-ис\-сле\-до\-ва\-тель\-ско\-го 
института системных исследований Российской академии наук
%117218, Москва, Нахимовский просп., 36, к.1 

\vspace*{7pt}

\noindent 
\textbf{Шестаков Олег Владимирович} (р.\ 1976)~--- доктор 
фи\-зи\-ко-ма\-те\-ма\-ти\-че\-ских, доцент кафедры математической статистики 
факультета вычисли\-тельной математики и кибернетики Московского 
государственного университета им.\ М.\,В.~Ломоносова; 
%119991, Москва ГСП-1, Ленинские горы, д.\ 1, стр. 52
старший научный сотрудник Института проб\-лем информатики
Российской академии наук
%, Москва 119333, ул. Вавилова, д.~44, корп.~2

\vspace*{7pt}

\noindent 
\textbf{Шоргин Сергей Яковлевич} (р.\ 1952.)~--- доктор
фи\-зи\-ко-ма\-те\-ма\-ти\-че\-ских наук, профессор, заместитель директора Института 
проб\-лем информатики Российской академии наук





%%%%%%%%%%%%%%%%%%%%%%%%%%%%%%%%%%%%%%%%%%%%%%%%%%%%%%%%%%%%%%%%%%%%%%%%%%%%%%%




%\def\rightkol{ОБ АВТОРАХ}
%\def\leftkol{ОБ АВТОРАХ}

 \label{end\stat}





%\def\leftfootline{\small{\textbf{\thepage}
%\hfill ИНФОРМАТИКА И ЕЁ ПРИМЕНЕНИЯ\ \ \ том~7\ \ \ выпуск~1\ \ \ 2013}
%}%
% \def\rightfootline{\small{ИНФОРМАТИКА И ЕЁ ПРИМЕНЕНИЯ\ \ \ том~7\ \ \ выпуск~1\ \ \ 2013
%\hfill \textbf{\thepage}}}


%\thispagestyle{myheadings}



\end{multicols}

\newpage

%\end{document}

%
\def\stat{rekl}
%\label{preobr}

%\def\tit{АКАДЕМИК ПУГАЧЁВ  ВЛАДИМИР СЕМЁНОВИЧ\\
%25.03.1911--25.03.1998}


%   \vspace*{-48pt}
%   \begin{center}\LARGE
%Академик Пугачёв  Владимир Семёнович\\ (25.03.1911--25.03.1998)
%   \end{center}

   %\vspace*{2.5mm}

   \begin{center}

{\prgsh\LARGE
ЮБИЛЕИ}

\end{center}
%\hrule

\vspace*{6pt}


   \vspace*{8mm}

   \thispagestyle{empty}


%\def\stat{emel}


\section*{К 70-летию заместителя директора ИПИ РАН,\\ члена редколлегии журнала
<<Информатика и её применения>>\\ доктора технических наук В.\,И.~Будзко}

\vspace*{18pt}




          \begin{multicols}{2}

%            \label{st\stat}

\begin{center}
\vspace*{1pt}
\mbox{%
\epsfxsize=78mm
\epsfbox{bud-1.eps}
}
\end{center}

\vspace*{12pt}

      14 августа 2014~г.\ исполнилось 70~лет за\-мес\-ти\-те\-лю директора ИПИ РАН по
научной работе доктору технических наук Владимиру Игоревичу Будзко.

      Владимир Игоревич Будзко родился в г.~Москве. Высшее образование получил на факультете
элект\-рон\-но-вы\-чис\-ли\-тель\-ных устройств в Московском
ин\-же\-нер\-но-фи\-зи\-че\-ском институте
(МИФИ), который он окончил в 1968~г., после чего был на\-прав\-лен для прохождения
службы в одну из войс\-ко\-вых частей, где прошел путь от инженера до первого заместителя
командира войсковой части.

      С приходом В.\,И.~Будзко в ИПИ РАН (2001~г.)\ в институте
сформировалось новое научное на\-прав\-ле\-ние теоретических исследований~--- <<Постро\-ение
ин\-фор\-ма\-ци\-он\-но-те\-ле\-ком\-му\-ни\-ка\-ци\-он\-ных\linebreak сис\-тем
высокой до\-ступ\-ности>>. В~рамках этого
направления выполнен широкий круг фундаментальных исследований по поиску подходов и
определению принципов построения средств обеспечения доступности, конфиденциальности
и целостности современных крупномасштабных
ин\-фор\-ма\-ци\-он\-но-те\-ле\-ком\-му\-ни\-ка\-ци\-он\-ных
сис\-тем (ИТС). Разработаны основные сис\-тем\-но-тех\-ни\-че\-ские принципы и базовые
архитектурные решения построения перспективных для условий России ИТС с
централизованной обработкой и хранением информации, сочетающих в себе свойства
высокой доступности, отказо- и катастрофоустойчивости, информационной защищенности.
Определены принципы, методы и математические основы рационального построения и
оптимизации средств восстановления функционирования центров обработки данных (ЦОД)
после возникновения отказов и катастроф, передачи и хранения данных, обеспечения
информационной безопасности при достижении минимальной совокупной стоимости
владения такими системами. Результаты нашли практическое воплощение при реализации
проектов в интересах ряда отечественных государственных и негосударственных
организаций, таких как Банк России (БР), Внешторгбанк, ОАО <<ГМК <<Норильский Никель>>,
<<Газпром>>, Минэкономразвития России, Правительство Москвы, а также ряд силовых
ведомств.

      Под руководством В.\,И.~Будзко начиная с 2001~г.\ выполнен комплекс
      на\-уч\-но-ис\-сле\-до\-ва\-тель\-ских и
      опыт\-но-кон\-ст\-рук\-тор\-ских работ (свыше 100~проектов),
направленных на развитие электронной информационной технологии БР.
Разработаны концепции развития ИТС БР сначала до 2008~г., а затем до 2013~г., которые
были приняты в качестве основы проведения технической политики. За реализацию проекта
<<Катастрофоустойчивая тер\-ри\-то\-ри\-аль\-но-рас\-пре\-де\-лен\-ная
      ин\-фор\-ма\-ци\-он\-но-те\-ле\-ком\-му\-ни\-ка\-ци\-он\-ная сис\-те\-ма централизованной
обработки банковской информации>> В.\,И.~Будзко удостоен Премии Правительства РФ в
области науки и техники за 2010~г.

      В.\,И.~Будзко возглавлял и возглавляет работы по ряду других прикладных проектов,
связанных с созданием, совершенствованием и развитием крупномасштабных ИТС.

      В.\,И.~Будзко~--- генерал-майор, доктор технических наук, член-кор\-рес\-пон\-дент
Академии криптографии РФ, известный ученый в области информатики и применения
информационных технологий при построении территориально распределенных ИТС
различного назначения. Является автором свыше 250~научных работ, опубликованных в
на\-уч\-но-тех\-ни\-че\-ских и специальных изданиях.

    \thispagestyle{empty}

      В.\,И.~Будзко уделяет большое внимание подготовке научных кадров. Под его
руководством защищено 6~диссертаций на соискание ученой степени кандидата
технических наук. Свыше 30~лет он читает лекции в ИКСИ Академии ФСБ, профессор
кафедры НИЯУ МИФИ. Является членом двух диссертационных советов, главным
редактором журнала <<Системы высокой доступности>> и членом редколлегии журнала
<<Информатика и её применения>>.

      \bigskip

      Редакционный совет и Редакционная коллегия журнала <<Информатика и её
применения>> сердечно поздравляют Владимира Игоревича Будзко с 70-ле\-ти\-ем и желают
крепкого здоровья и новых научных достижений.

\end{multicols}

\def\stat{cont}
{%\hrule\par
%\vskip 7pt % 7pt
\raggedleft\Large \bf%\baselineskip=3.2ex
А\,В\,Т\,О\,Р\,С\,К\,И\,Й\ \ У\,К\,А\,З\,А\,Т\,Е\,Л\,Ь\ \ З\,А\ \ 2\,0\,1\,0 г. \vskip 17pt
    \hrule
    \par
\vskip 21pt plus 6pt minus 3pt }

\label{st\stat}

\def\tit{\ }

\def\aut{\ }
\def\auf{\ }

\def\leftkol{\ } % ENGLISH ABSTRACTS}

\def\rightkol{\ } %АВТОРСКИЙ УКАЗАТЕЛЬ ЗА 2010 г.} %ENGLISH ABSTRACTS}

\titele{\tit}{\aut}{\auf}{\leftkol}{\rightkol}

\vspace*{-12pt}

{\tabcolsep=3pt
\begin{tabular}{p{388pt}rr}
&\textbf{Выпуск} & \textbf{Стр.}\\[6pt]
\hangindent=23pt\noindent\textbf{Арутюнян~А.\,Р.} Моделирование влияния деформаций отпечатков пальцев на 
точность\linebreak
\vspace*{-12pt}\\
\hspace*{23pt}дактилоскопической идентификации$\dotfill$&1&51\\
\hangindent=23pt\noindent\textbf{Архипов~О.\,П., Зыкова~З.\,П.} Интеграция гетерогенной информации о цветных 
пикселях\linebreak
\vspace*{-12pt}\\
\hspace*{23pt}и их цветовосприятии$\dotfill$&4&15\\
\hangindent=23pt\noindent\textbf{Баранов~С.\,И., Френкель~С.\,Л., Захаров~В.\,Н.} Полуформальная верификация 
цифрового устройства с конвейером, основанная на использовании алгоритмических машин\linebreak
\vspace*{-12pt}\\
\hspace*{23pt}состояния$\dotfill$&4&49\\
\textbf{Бекетова~И.\,В.} см.~Каратеев~С.\,Л.&&\\
\textbf{Белоусов~В.\,В.} см.~Синицын~И.\,Н.&&\\
\hangindent=23pt\noindent\textbf{Бенинг~В.\,Е., Королев~Р.\,А.} О предельном поведении мощностей критериев в 
случае\linebreak
\vspace*{-12pt}\\
\hspace*{23pt}распределения Лапласа$\dotfill$&2&63\\
\hangindent=23pt\noindent\textbf{Бенинг~В.\,Е., Сипина~А.\,В.} Асимптотическое разложение для мощности 
критерия,\linebreak
\vspace*{-12pt}\\
\hspace*{23pt}основанного на выборочной медиане, в случае распределения Лапласа$\dotfill$&1&18\\
\textbf{Бондаренко~А.\,В.} см.~Каратеев~С.\,Л.&&\\
\hangindent=23pt\noindent\textbf{Бородина~А.\,В., Морозов~Е.\,В.} Об оценивании асимптотики вероятности 
большого\linebreak
\vspace*{-12pt}\\
\hspace*{23pt}уклонения стационарной регенеративной очереди с одним прибором$\dotfill$&3&29\\
\hangindent=23pt\noindent\textbf{Бунтман~Н.\,В., Минель~Ж.-Л., Ле~Пезан~Д., Зацман~И.\,М.} Типология и 
компьютерное\linebreak
\vspace*{-12pt}\\
\hspace*{23pt}моделирование трудностей перевода$\dotfill$&3&77\\
\textbf{Визильтер~Ю.\,В.} см.~Каратеев~С.\,Л.&&\\
\hangindent=23pt\noindent\textbf{Гавриленко~С.\,В.} Оценки скорости сходимости распределений случайных сумм с 
безгранично делимыми индексами к нормальному закону$\dotfill$&4&81\\
\hangindent=23pt\noindent\textbf{Григорьева~М.\,Е., Шевцова~И.\,Г.} Уточнение неравенства 
Каца--Берри--Эссеена$\dotfill$&2&75\\
\hangindent=23pt\noindent\textbf{Грушо~А.\,А., Грушо~Н.\,А., Тимонина~Е.\,Е.} Поиск конфликтов в политиках 
безопасности: модель случайных графов$\dotfill$&3&38\\
\textbf{Грушо~Н.\,А.} см.~Грушо~А.\,А.&&\\
\hangindent=23pt\noindent\textbf{Гудков~В.\,Ю.} Математические модели изображения отпечатка пальца на основе 
описания линий$\dotfill$&1&58\\
\textbf{Гуртов~А.\,В.} см.~Лукьяненко~А.\,С.&&\\
\textbf{Желтов~С.\,Ю.} см.~Каратеев~С.\,Л.&&\\
\hangindent=23pt\noindent\textbf{Захаров~А.\,А., Серебряков~В.\,А.} Система управления электронной библиотекой 
LibMeta$\dotfill$&4&2\\
\textbf{Захаров~В.\,Н.} см.~Баранов~С.\,И.&&\\
\textbf{Захарова~Т.\,В.} см.~Матвеева~С.\,С.&&\\
\hangindent=23pt\noindent\textbf{Зацаринный~А.\,А., Чупраков~К.\,Г.} Некоторые аспекты выбора технологии для 
постро-\linebreak
\vspace*{-12pt}\\
\hspace*{23pt}ения систем отображения информации ситуационного центра$\dotfill$&3&59\\
\textbf{Зацман~И.\,М.} см.~Бунтман~Н.\,В.&&\\
\hangindent=23pt\noindent\textbf{Зейфман~А.\,И., Коротышева~А.\,В., Сатин~Я.\,А., Шоргин~С.\,Я.} Об 
устойчивости нестаци-\linebreak
\vspace*{-12pt}\\
\hspace*{23pt}онарных систем обслуживания с катастрофами$\dotfill$&3&9\\
\textbf{Зыкова~З.\,П.} см.~Архипов~О.\,П.&&\\
\hangindent=23pt\noindent\textbf{Илюшин~Г.\,Я., Соколов~И.\,А.} Организация управляемого доступа пользователей 
к\linebreak
\vspace*{-12pt}\\
\hspace*{23pt}разнородным ведомственным информационным ресурсам$\dotfill$&1&24\\
\hangindent=23pt\noindent\textbf{Кавагучи~Ю., Ульянов~В.\,В., Фуджикоши~Я.} Приближения для статистик, 
описывающих\linebreak
\vspace*{-12pt}\\
\hspace*{23pt}геометрические свойства данных большой размерности, с оценками 
ошибок$\dotfill$&1&12\\
\hangindent=23pt\noindent\textbf{Каратеев~С.\,Л., Бекетова~И.\,В., Ососков~М.\,В., Князь~В.\,А., 
Визильтер~Ю.\,В., Бондаренко~А.\,В., Желтов~С.\,Ю.} Автоматизированный контроль 
качества цифровых\linebreak
\vspace*{-12pt}\\
\hspace*{23pt}изображений для персональных документов$\dotfill$&1&65\\
\end{tabular}
}

\pagebreak

\def\leftkol{АВТОРСКИЙ УКАЗАТЕЛЬ ЗА 2010 г.} % ENGLISH ABSTRACTS}

\def\rightkol{АВТОРСКИЙ УКАЗАТЕЛЬ ЗА 2010 г.} %ENGLISH ABSTRACTS}

{\tabcolsep=3pt
\begin{tabular}{p{388pt}rr}
&\textbf{Выпуск} & \textbf{Стр.}\\[3pt]
\hangindent=23pt\noindent\textbf{Козеренко~Е.\,Б.} Лингвистические фильтры в статистических моделях машинного\linebreak
\vspace*{-12pt}\\
\hspace*{23pt}перевода$\dotfill$&2&83\\
\hangindent=23pt\noindent\textbf{Козеренко~Е.\,Б., Кузнецов~И.\,П.} Когнитивно-лингвистические представления в 
систе-\linebreak
\vspace*{-12pt}\\
\hspace*{23pt}мах обработки текстов$\dotfill$&3&69\\
\textbf{Князь~В.\,А.} см.~Каратеев~С.\,Л.&&\\
\hangindent=23pt\noindent\textbf{Колесников~А.\,В., Солдатов~С.\,А.} Алгоритм координации для гибридной 
интеллектуальной системы решения сложной задачи оперативно-производственного\linebreak
\vspace*{-12pt}\\
\hspace*{23pt}планирования$\dotfill$&4&61\\
\hangindent=23pt\noindent\textbf{Коновалов~М.\,Г.} О планировании потоков в системах вычислительных 
ресурсов$\dotfill$&2&3\\
\textbf{Конушин~А.\,С.} см.~Конушин~В.\,С.&&\\
\hangindent=23pt\noindent\textbf{Конушин~В.\,С., Кривовязь~Г.\,Р., Конушин~А.\,С.} Алгоритм распознавания людей 
в видео-\linebreak
\vspace*{-12pt}\\
\hspace*{23pt}последовательности по одежде$\dotfill$&1&74\\
\textbf{Корепанов~Э.\, Р.} см.~Синицын~И.\,Н.&&\\
\textbf{Королев~В.\,Ю.} см.~Соколов~И.\,А.&&\\
\textbf{Королев~Р.\,А.} см.~Бенинг~В.\,Е.&&\\
\textbf{Коротышева~А.\,В.} см.~Зейфман~А.\,И.&&\\
\hangindent=23pt\noindent\textbf{Кривенко~М.\,П.} Непараметрическое оценивание элементов байесовского 
клас\-си-\linebreak
\vspace*{-12pt}\\
\hspace*{23pt}фикатора$\dotfill$&2&13\\
\textbf{Кривовязь~Г.\,Р.} см.~Конушин~В.\,С.&&\\
\textbf{Крылов~А.\,С.} см.~Павельева~Е.\,А.&&\\
\hangindent=23pt\noindent\textbf{Крылов~В.\,А.} Моделирование и классификация многоканальных дистанционных\linebreak
\vspace*{-12pt}\\
\hspace*{23pt}изображений с использованием копул$\dotfill$&4&34\\
\hangindent=23pt\noindent\textbf{Крючин~О.\,В.} Разработка параллельных эвристических алгоритмов подбора 
весовых\linebreak
\vspace*{-12pt}\\
\hspace*{23pt}коэффициентов искусственной нейтронной сети$\dotfill$&2&53\\
\hangindent=23pt\noindent\textbf{Кудрявцев~А.\,А., Шоргин~С.\,Я.} Байесовские модели массового обслуживания и 
надеж-\linebreak
\vspace*{-12pt}\\
\hspace*{23pt}ности: характеристики среднего числа заявок в системе $M\vert M \vert 1\vert 
\infty$$\dotfill$&3&16\\
\hangindent=23pt\noindent\textbf{Кузнецов~А.\,А.} Связь между временными и структурно-топологическими 
характери-\linebreak
\vspace*{-12pt}\\
\hspace*{23pt}стиками диаграмм ритма сердца здоровых людей$\dotfill$&4&39\\
\textbf{Кузнецов~И.\,П.} см.~Козеренко~Е.\,Б.&&\\
\textbf{Ле~Пезан~Д.} см.~Бунтман~Н.\,В.&&\\
\hangindent=23pt\noindent\textbf{Лукьяненко~А.\,С., Морозов~Е.\,В., Гуртов~А.\,В.} Анализ сетевого протокола с общей 
функ-\linebreak
\vspace*{-12pt}\\
\hspace*{23pt}цией расширения окна передачи сообщения при конфликтах$\dotfill$&2&46\\
\hangindent=23pt\noindent\textbf{Лямин~О.\,О.} О предельном поведении мощностей критериев в случае обобщенного\linebreak
\vspace*{-12pt}\\
\hspace*{23pt}распределения Лапласа$\dotfill$&3&47\\
\hangindent=23pt\noindent\textbf{Маркин~А.\,В., Шестаков~О.\,В.} Асимптотики оценки риска при пороговой 
обработке\linebreak
\vspace*{-12pt}\\
\hspace*{23pt}вейвлет-вейглет коэффициентов в задаче томографии$\dotfill$&2&36\\
\hangindent=23pt\noindent\textbf{Матвеева~С.\,С., Захарова~Т.\,В.} Сети массового обслуживания с наименьшей 
длиной\linebreak
\vspace*{-12pt}\\
\hspace*{23pt}очереди$\dotfill$&3&22\\
\hangindent=23pt\noindent\textbf{Матюшенко~С.\,И.} Стационарные характеристики двухканальной системы 
обслужива-\linebreak
\vspace*{-12pt}\\
\hspace*{23pt}ния с переупорядочиванием заявок и распределениями фазового типа$\dotfill$&4&68\\
\textbf{Минель~Ж.-Л.} см.~Бунтман~Н.\,В.&&\\
\textbf{Морозов~Е.\,В.} см.~Бородина~А.\,В.&&\\
\textbf{Морозов~Е.\,В.} см.~Лукьяненко~А.\,С.&&\\
\textbf{Ососков~М.\,В.} см.~Каратеев~С.\,Л.&&\\
\hangindent=23pt\noindent\textbf{Павельева~Е.\,А., Крылов~А.\,С.} Поиск и анализ ключевых точек радужной 
оболочки\linebreak
\vspace*{-12pt}\\
\hspace*{23pt}глаза методом преобразования Эрмита$\dotfill$&1&79\\
\textbf{Печинкин~А.\,В.} см.~Френкель~С.\,Л.,&&\\
\hangindent=23pt\noindent\textbf{Протасов~В.\,И.} Составление субъективного портрета с использованием 
эволюционно-\linebreak
\vspace*{-12pt}\\
\hspace*{23pt}го морфинга и квалиметрия метода$\dotfill$&1&83\\
\hangindent=23pt\noindent\textbf{Рудаков~К.\,В., Торшин~И.\,Ю.} Вопросы разрешимости задачи распознавания 
вторичной\linebreak
\vspace*{-12pt}\\
\hspace*{23pt}структуры белка$\dotfill$&2&25\\
\textbf{Сатин~Я.\,А.} см.~Зейфман~А.\,И.&&\\
\hangindent=23pt\noindent\textbf{Сейфуль-Мулюков~Р.\,Б.} Нефть как носитель информации о своем 
происхождении,\linebreak
\vspace*{-12pt}\\
\hspace*{23pt}структуре и эволюции$\dotfill$&1&41\\
\end{tabular}
}

{\tabcolsep=3pt
\begin{tabular}{p{388pt}rr}
&\textbf{Выпуск} & \textbf{Стр.}\\[6pt]
\textbf{Семендяев~Н.\,Н.} см.~Синицын~И.\,Н.&&\\
\textbf{Серебряков~В.\,А.} см.~Захаров~А.\,А.&&\\
\textbf{Синицын~В.\,И.} см.~Синицын~И.\,Н.&&\\
\hangindent=23pt\noindent\textbf{Синицын~И.\,Н., Синицын~В.\,И., Корепанов~Э.\, Р., Белоусов~В.\,В., 
Семендяев~Н.\,Н.} Оперативное построение информационных моделей движения полюса 
Земли\linebreak
\vspace*{-12pt}\\
\hspace*{23pt}методами линейных и линеаризованных фильтров$\dotfill$&1&2\\
\textbf{Сипина~А.\,В.} см.~Бенинг~В.\,Е.&&\\
\hangindent=23pt\noindent\textbf{Соколов~И.\,А.} О работах заслуженного деятеля науки Российской Федерации 
И.\,Н.~Синицына в области информационных технологий и автоматизации (к 70-летию\linebreak
\vspace*{-12pt}\\
\hspace*{23pt}со дня рождения)$\dotfill$&3&84\\
\textbf{Соколов~И.\,А.} см.~Илюшин~Г.\,Я.&&\\
\hangindent=23pt\noindent\textbf{Соколов~И.\,А., Королев~В.\,Ю.} Предисловие$\dotfill$&2&2\\
\textbf{Солдатов~С.\,А.} см.~Колесников~А.\,В.&&\\
\hangindent=23pt\noindent\textbf{Степанов~С.\,Ю.} Использование координатного метода фрагментации 
коммутаторной\linebreak
\vspace*{-12pt}\\
\hspace*{23pt}нейронной сети для сокращения трафика$\dotfill$&2&57\\
\textbf{Тимонина~Е.\,Е.} см.~Грушо~А.\,А.&&\\
\textbf{Торшин~И.\,Ю.} см.~Рудаков~К.\,В.&&\\
\textbf{Ульянов~В.\,В.} см.~Кавагучи~Ю.&&\\
\textbf{Фазекаш~И.} см.~Чупрунов~А.\,Н.&&\\
\textbf{Френкель~С.\,Л.} см.~Баранов~С.\,И.&&\\
\hangindent=23pt\noindent\textbf{Френкель~С.\,Л., Печинкин~А.\,В.} Оценка времени самовосстановления в 
цифровых\linebreak
\vspace*{-12pt}\\
\hspace*{23pt}системах после сбоев, вызываемых переходными помехами$\dotfill$&3&2\\
\textbf{Фуджикоши~Я.} см.~Кавагучи~Ю.&&\\
\hangindent=23pt\noindent\textbf{Цискаридзе~А.\,К.} Математическая модель и метод восстановления позы человека 
по\linebreak
\vspace*{-12pt}\\
\hspace*{23pt}стереопаре силуэтных изображений$\dotfill$&4&27\\
\hangindent=23pt\noindent\textbf{Чупраков~К.\,Г.} К вопросу о размещении коллективных средств отображения в 
ситуа-\linebreak
\vspace*{-12pt}\\
\hspace*{23pt}ционном зале с заданными параметрами$\dotfill$&4&89\\
\textbf{Чупраков~К.\,Г.} см.~Зацаринный~А.\,А.&&\\
\hangindent=23pt\noindent\textbf{Чупрунов~А.\,Н., Фазекаш~И.} Законы повторного логарифма для числа 
безошибочных\linebreak
\vspace*{-12pt}\\
\hspace*{23pt}блоков при помехоустойчивом кодировании$\dotfill$&3&42\\
\textbf{Шевцова~И.\,Г.} см.~Григорьева~М.\,Е.&&\\
\hangindent=23pt\noindent\textbf{Шестаков~О.\,В.} Аппроксимация распределения оценки риска пороговой 
обработки вейвлет-коэффициентов нормальным распределением при использовании 
выбо-\linebreak
\vspace*{-12pt}\\
\hspace*{23pt}рочной дисперсии$\dotfill$&4&73\\
\textbf{Шестаков~О.\,В.} см.~Маркин~А.\,В.&&\\
\textbf{Шоргин~С.\,Я.} см.~Зейфман~А.\,И.&&\\
\textbf{Шоргин~С.\,Я.} см.~Кудрявцев~А.\,А.&&\\
\end{tabular}
}

%\thispagestyle{myheadings}
\def\leftfootline{\small{\textbf{\thepage}
\hfill ИНФОРМАТИКА И ЕЁ ПРИМЕНЕНИЯ\ \ \ том~4\ \ \ выпуск~4\ \ \ 2010}
}%
 \def\rightfootline{\small{ИНФОРМАТИКА И ЕЁ ПРИМЕНЕНИЯ\ \ \ том~4\ \ \ выпуск~4\ \ \ 2010
 \hfill \textbf{\thepage}}}
 \label{end\stat}


%Том 10 Выпуск 1-4 Год 2016

\def\stat{cont-e}
{%\hrule\par
%\vskip 7pt % 7pt
\raggedleft\Large \bf%\baselineskip=3.2ex
2\,0\,1\,6\ \ A\,U\,T\,H\,O\,R\ \ I\,N\,D\,E\,X \vskip 17pt
 \hrule
 \par
\vskip 21pt plus 6pt minus 3pt }

\label{st\stat}

\def\tit{\ }

\def\aut{\ }
\def\auf{\ }

\def\leftkol{\ } %2016 AUTHOR INDEX} % ENGLISH ABSTRACTS}

\def\rightkol{\ } %2016 AUTHOR INDEX} %ENGLISH ABSTRACTS}

\titele{\tit}{\aut}{\auf}{\leftkol}{\rightkol}

\def\leftfootline{\small{\textbf{\thepage}
\hfill INFORMATIKA I EE PRIMENENIYA~--- INFORMATICS AND APPLICATIONS\ \ \ 2016\
\ \ volume~10\ \ \ issue\ 4}
}%
 \def\rightfootline{\small{INFORMATIKA I EE PRIMENENIYA~--- INFORMATICS AND APPLICATIONS\ \ \ 2016\ \ \ volume~10\ \ \ issue\ 4
\hfill \textbf{\thepage}}}

\vspace*{-12pt}
\vspace*{-18pt}

{\tabcolsep=2.8pt
\begin{tabular}{p{382pt}cc}
&\textbf{Issue} & \textbf{Page}\\[6pt]
\Avtors{Agalarov~M.\,Ya.} see~Agalarov~Ya.\,M.&&\\
\Avtors{Agalarov~Ya.\,M., Agalarov~M.\,Ya., and
Shorgin~V.\,S.} About the optimal threshold of queue\linebreak
\\[-12pt]
\hspace*{23pt}length in a~particular problem of profit maximization
in the $M/G/1$ queuing system&2&70--79\\
\Avtors{Alexeyevsky~D.\,A.} BioNLP ontology extraction from 
a~restricted language corpus with\linebreak
\\[-12pt]
\hspace*{23pt}context-free grammars&1&119--128\\
\Avtors{Andreev~S.\,D.} see~Gaidamaka~Yu.\,V.&&\\
\Avtors{Andreev~S.\,D.} see~Ometov~A.\,Ya.&&\\
\Avtors{Arkhipov~O.\,P., Arkhipov~P.\,O., and Sidorkin~I.\,I.} The
option to create a~local coordinate\linebreak
\\[-12pt]
\hspace*{23pt}system for synchronization of selected images&3&91--97\\
\Avtors{Arkhipov~P.\,O.} see~Arkhipov~O.\,P.&&\\
\Avtors{Belousov~V.\,V.} see~Shnurkov~P.\,V.&&\\
\Avtors{Belousov~V.\,V.} see~Shnurkov~P.\,V.&&\\
\Avtors{Bening~V.\,E.} Calculation of~the~asymptotic deficiency
of~some statistical procedures based\linebreak
\\[-12pt]
\hspace*{23pt}on~samples with~random sizes&4&34--45\\
\Avtors{Borisov~A.\,V., Bosov~A.\,V., and Miller~G.\,B.} Modeling and
monitoring of VoIP connection&2&\hphantom{1}2--13\\
\Avtors{Bosov~A.\,V.} see~Borisov~A.\,V.&&\\
\Avtors{Briukhov~D.\,O.} see~Stupnikov~S.\,A.&&\\
\Avtors{Callaos~N.\,K.\ and Seyful-Mulyukov~R.\,B.} Complexity and
its information content&1&129--139\\
\Avtors{Chertok~A.\,V., Kadaner~A.\,I., Khazeeva~G.\,T., and
Sokolov~I.\,A.} Regime switching detection\linebreak
\\[-12pt]
\hspace*{23pt}for~the~Levy driven
Ornstein--Uhlenbeck process using CUSUM methods&4&46--56\\
\Avtors{Chichagov~V.\,V.} Asymptotic expansions of mean absolute
error of uniformly minimum variance unbiased and maximum likelihood
estimators on the one-parameter exponential\linebreak
\\[-12pt]
\hspace*{23pt}family model of lattice distributions&3&66--76\\
\Avtors{Danishevsky~V.\,I.} see~Kolesnikov A.\,V.&&\\
\Avtors{Fazliev~A.\,Z.} see~Kalinichenko~L.\,A.&&\\
\Avtors{Fedoseev~A.\,A.} What is behind the concept of ``knowledge in
small packages''&3&105--110\\
\Avtors{Gaidamaka~Yu.\,V., Andreev~S.\,D., Sopin~E.\,S.,
Samouylov~K.\,E., and Shorgin~S.\,Ya.} Interference analysis
of~the~device-to-device communications model with~regard to~a~signal\linebreak
\\[-12pt]
\hspace*{23pt}propagation environment&4&\hphantom{1}2--10\\
\Avtors{Gasilov~A.\,V.} see~Yakovlev~O.\,A.&&\\
\Avtors{Goncharov~A.\,V.\ and Strijov~V.\,V.} Metric time series
classification using weighted dynamic\linebreak
\\[-12pt]
\hspace*{23pt}warping relative to centroids of classes&2&36--47\\
\Avtors{Gordov~E.\,P.} see~Kalinichenko~L.\,A.&&\\
\Avtors{Gorshenin~A.\,K.} Concept of online service for stochastic
modeling of real processes&1&72--81\\
\Avtors{Gorshenin~A.\,K.} see~Shnurkov~P.\,V.&&\\
\Avtors{Gorshenin~A.\,K.} see~Shnurkov~P.\,V.&&\\
\Avtors{Grusho~A.\,A., Grusho~N.\,A., Zabezhailo~M.\,I., and
Timonina~E.\,E.} Integration of statistical and\linebreak
\\[-12pt]
\hspace*{23pt}deterministic methods for
analysis of information security&3&2--8\\
\Avtors{Grusho~A.\,A., Zabezhailo~M.\,I., and Zatsarinny~A.\,A.} On
the advanced procedure to reduce\linebreak
\\[-12pt]
\hspace*{23pt}calculation of Galois closures&4&\hphantom{1}96--104\\
\Avtors{Grusho~N.\,A.} see~Grusho~A.\,A.&&\\
\Avtors{Havanskov~V.\,A.} see~Minin~V.\,A.&&\\
\Avtors{Inkova~O.\,Yu.} see~Zatsman~I.\,M.&&\\
\Avtors{Isachenko~R.\,V.\ and Strijov~V.\,V.} Metric learning in
multiclass time series classification\linebreak
\\[-12pt]
\hspace*{23pt}problem&2&48--57\\
\end{tabular}
}
\pagebreak

\def\leftfootline{\small{\textbf{\thepage}
\hfill INFORMATIKA I EE PRIMENENIYA~--- INFORMATICS AND APPLICATIONS\ \ \ 2016\
\ \ volume~10\ \ \ issue\ 4}
}%
 \def\rightfootline{\small{INFORMATIKA I EE PRIMENENIYA~---
INFORMATICS AND APPLICATIONS\ \ \ 2016\ \ \ volume~10\ \ \ issue\ 4
\hfill \textbf{\thepage}}}

\def\leftkol{2016 AUTHOR INDEX} % ENGLISH ABSTRACTS}

\def\rightkol{2016 AUTHOR INDEX} %ENGLISH ABSTRACTS}


{\tabcolsep=2.83pt
\begin{tabular}{p{382pt}cc}
&\textbf{Issue} & \textbf{Page}\\[6pt]
\Avtors{Kadaner~A.\,I.} see~Chertok~A.\,V.&&\\[.255pt]
\Avtors{Kalinichenko~L.\,A., Volnova~A.\,A., Gordov~E.\,P.,
Kiselyova~N.\,N., Kovaleva~D.\,A., Malkov~O.\,Yu., Okladnikov~I.\,G.,
Podkolodnyy~N.\,L., Pozanenko~A.\,S., Ponomareva~N.\,V.,
Stupnikov~S.\,A.,} \textbf{and Fazliev~A.\,Z.} Data access challenges for data
intensive\linebreak
\\[-12pt]
\hspace*{23pt}research in Russia&1& 2--22\\[.255pt]
\Avtors{Karasikov~M.\,E.\ and Strijov~V.\,V.} Feature-based
time-series classification&4&121--131\\[.255pt]
\Avtors{Khazeeva~G.\,T.} see~Chertok~A.\,V.&&\\[.255pt]
\Avtors{Khokhlov~Yu.\,S.} Multivariate fractional Levy motion and its
applications&2&\hphantom{1}98--106\\[.255pt]
\Avtors{Kirikov~I.\,A., Kolesnikov~A.\,V., Listopad~S.\,V., and
Rumovskaya~S.\,B.} Fine-grained hybrid\linebreak
\\[-12pt]
\hspace*{23pt}intelligent systems. Part 2:
Bidirectional hybridization&1&\hphantom{1}96--105\\[.255pt]
\Avtors{Kirikov~I.\,A., Kolesnikov~A.\,V., Listopad~S.\,V., and
Rumovskaya~S.\,B.} ``Virtual council''~---\linebreak
\\[-12pt]
\hspace*{23pt}source environment
supporting complex diagnostic decision making&3&81--90\\[.255pt]
\Avtors{Kiselyova~N.\,N.} see~Kalinichenko~L.\,A.&&\\[.255pt]
\Avtors{Kolesnikov A.\,V., Listopad~S.\,V., Rumovskaya~S.\,B., and
Danishevsky~V.\,I.} Informal axiomatic\linebreak
\\[-12pt]
\hspace*{23pt}theory of~the~role visual models&4&114--120\\[.255pt]
\Avtors{Kolesnikov~A.\,V.} see~Kirikov~I.\,A.&&\\[.255pt]
\Avtors{Kolesnikov~A.\,V.} see~Kirikov~I.\,A.&&\\[.255pt]
\Avtors{Kolin~K.\,K.} Humanitarian aspects of information
security&3&111--121\\[.255pt]
\Avtors{Konovalov~M.\,G.\ and Razumchik~R.\,V.} Dispatching
to~two parallel nonobservable queues using\linebreak
\\[-12pt]
\hspace*{23pt}only static
information&4&57--67\\[.255pt]
\Avtors{Korchagin~A.\,Yu.} see~Korolev~V.\,Yu.&&\\[.255pt]
\Avtors{Korchagin~A.\,Yu.} see~Korolev~V.\,Yu.&&\\[.255pt]
\Avtors{Korepanov~E.\,R.} see~Sinitsyn~I.\,N.&&\\[.255pt]
\Avtors{Korepanov~E.\,R.} see~Sinitsyn~I.\,N.&&\\[.255pt]
\Avtors{Korolev~V.\,Yu., Korchagin~A.\,Yu., and Zeifman~A.\,I.} The
Poisson theorem for Bernoulli trials\linebreak
\\[-12pt]
\hspace*{23pt}with~a~random probability
of~success and~a~discrete analog of~the~Weibull distribution&4&11--20\\[.255pt]
\Avtors{Korolev~V.\,Yu., Zeifman~A.\,I., and Korchagin~A.\,Yu.}
Asymmetric Linnik distributions as~limit\linebreak
\\[-12pt]
\hspace*{23pt}laws for~random sums
of~independent random variables with~finite variances&4&21--33\\[.255pt]
\Avtors{Koucheryavy~E.\,A.} see~Ometov~A.\,Ya.&&\\[.255pt]
\Avtors{Kovaleva~D.\,A.} see~Kalinichenko~L.\,A.&&\\[.255pt]
\Avtors{Kovalyov~S.\,P.} Metaprogramming to increase
manufacturability of large-scale software-\linebreak
\\[-12pt]
\hspace*{23pt}intensive systems&1&56--66\\[.255pt]
\Avtors{Krivenko~M.\,P.} Significance tests of feature selection for
classification&3&32--40\\[.255pt]
\Avtors{Kruzhkov~M.\,G.} see~Zalizniak~Anna~A.&&\\[.255pt]
\Avtors{Kruzhkov~M.\,G.} see~Zatsman~I.\,M.&&\\[.255pt]
\Avtors{Kudryavtsev~A.\,A.} Bayesian queueing and reliability models:
\textit{A~priori} distributions with\linebreak
\\[-12pt]
\hspace*{23pt}compact support&1&67--71\\[.255pt]
\Avtors{Kudryavtsev~A.\,A.} Characteristics dependent on the balance
coefficient in Bayesian models\linebreak
\\[-12pt]
\hspace*{23pt}with compact support of \textit{a priori}
distributions&3&77--80\\[.255pt]
\Avtors{Kudryavtsev~A.\,A.\ and Palionnaia~S.\,I.} Bayesian recurrent
model of reliability growth:\linebreak
\\[-12pt]
\hspace*{23pt}Parabolic distribution of parameters&2&80--83\\[.255pt]
\Avtors{Kudryavtsev~A.\,A.\ and Titova~A.\,I.} Bayesian queuing
and~reliability models: Degenerate-\linebreak
\\[-12pt]
\hspace*{23pt}Weibull case&4&68--71\\[.255pt]
\Avtors{Leontyev~N.\,D.\ and Ushakov~V.\,G.} Analysis of a queueing
system with autoregressive arrivals\linebreak
\\[-12pt]
\hspace*{23pt}and nonpreemptive priority&3&15--22\\[.255pt]
\Avtors{Listopad~S.\,V.} see~Kirikov~I.\,A.&&\\[.255pt]
\Avtors{Listopad~S.\,V.} see~Kirikov~I.\,A.&&\\[.255pt]
\Avtors{Listopad~S.\,V.} see~Kolesnikov A.\,V.&&\\[.255pt]
\Avtors{Malkov~O.\,Yu.} see~Kalinichenko~L.\,A.&&\\[.255pt]
\Avtors{Markov~A.\,S., Monakhov~M.\,M., and
Ulyanov~V.\,V.} Generalized Cornish--Fisher expansions\linebreak
\\[-12pt]
\hspace*{23pt}for distributions of statistics based on samples
of random size&2&84--91\\[.255pt]
\Avtors{Melnikov~A.\,K.\ and Ronzhin~A.\,F.} Generalized statistical
method of~text analysis based\linebreak
\\[-12pt]
\hspace*{23pt}on~calculation of~probability distributions
of~statistical values&4&89--95\\
\end{tabular}
}
\pagebreak

\def\leftfootline{\small{\textbf{\thepage}
\hfill INFORMATIKA I EE PRIMENENIYA~--- INFORMATICS AND APPLICATIONS\ \ \ 2016\
\ \ volume~10\ \ \ issue\ 4}
}%
 \def\rightfootline{\small{INFORMATIKA I EE PRIMENENIYA~---
INFORMATICS AND APPLICATIONS\ \ \ 2016\ \ \ volume~10\ \ \ issue\ 4
\hfill \textbf{\thepage}}}

\def\leftkol{2016 AUTHOR INDEX} % ENGLISH ABSTRACTS}

\def\rightkol{2016 AUTHOR INDEX} %ENGLISH ABSTRACTS}


{\tabcolsep=3pt
\begin{tabular}{p{381pt}cc}
&\textbf{Issue} & \textbf{Page}\\[6pt]
\Avtors{Meykhanadzhyan~L.\,A.} Stationary characteristics of the finite
capacity queueing system with\linebreak
\\[-12pt]
\hspace*{23pt}inverse service order and generalized
probabilistic priority&2&123--131\\[.23pt]
\Avtors{Miller~G.\,B.} see~Borisov~A.\,V.&&\\[.23pt]
\Avtors{Minin~V.\,A., Zatsman~I.\,M., Havanskov~V.\,A., and
Shubnikov~S.\,K.} Intensity of citation of scientific publications in
inventions on information and computer technologies patented\linebreak
\\[-12pt]
\hspace*{23pt}in Russia by domestic and foreign applicants&2&107--122\\[.23pt]
\Avtors{Monakhov~M.\,M.} see~Markov~A.\,S.&&\\[.23pt]
\Avtors{Naumov~V.\,A.\ and Samouylov~K.\,E.} On relationship
between queuing systems with resources\linebreak
\\[-12pt]
\hspace*{23pt}and Erlang networks&3&\hphantom{1}9--14\\[.23pt]
\Avtors{Okladnikov~I.\,G.} see~Kalinichenko~L.\,A.&&\\[.23pt]
\Avtors{Ometov~A.\,Ya., Andreev~S.\,D., Turlikov~A.\,M., and
Koucheryavy~E.\,A.} Performance analysis of\linebreak
\\[-12pt]
\hspace*{23pt}a wireless data
aggregation system with contention for contemporary sensor
networks&3&23--31\\[.23pt]
\Avtors{Palionnaia~S.\,I.} see~Kudryavtsev~A.\,A.&&\\[.23pt]
\Avtors{Podkolodnyy~N.\,L.} see~Kalinichenko~L.\,A.&&\\[.23pt]
\Avtors{Ponomareva~N.\,V.} see~Kalinichenko~L.\,A.&&\\[.23pt]
\Avtors{Popkova~N.\,A.} see~Zatsman~I.\,M.&&\\[.23pt]
\Avtors{Pozanenko~A.\,S.} see~Kalinichenko~L.\,A.&&\\[.23pt]
\Avtors{Razumchik~R.\,V.} see~Konovalov~M.\,G.&&\\[.23pt]
\Avtors{Ronzhin~A.\,F.} see~Melnikov~A.\,K.&&\\[.23pt]
\Avtors{Rumovskaya~S.\,B.} see~Kirikov~I.\,A.&&\\[.23pt]
\Avtors{Rumovskaya~S.\,B.} see~Kirikov~I.\,A.&&\\[.23pt]
\Avtors{Rumovskaya~S.\,B.} see~Kolesnikov A.\,V.&&\\[.23pt]
\Avtors{Samouylov~K.\,E.} see~Gaidamaka~Yu.\,V.&&\\[.23pt]
\Avtors{Samouylov~K.\,E.} see~Naumov~V.\,A.&&\\[.23pt]
\Avtors{Serebryanskii~S.\,M.} see~Tyrsin~A.\,N.&&\\[.23pt]
\Avtors{Seyful-Mulyukov~R.\,B.} see~Callaos~N.\,K.&&\\[.23pt]
\Avtors{Shestakov~O.\,V.} Statistical properties of the denoising method
based on the stabilized hard\linebreak
\\[-12pt]
\hspace*{23pt}thresholding&2&65--69\\[.23pt]
\Avtors{Shestakov~O.\,V.} The strong law of large numbers for the risk
estimate in the problem of\linebreak
\\[-12pt]
\hspace*{23pt}tomographic image reconstruction from
projections with a correlated noise&3&41--45\\[.23pt]
\Avtors{Shestakov~O.\,V.} see~Zakharova~T.\,V.&&\\[.23pt]
\Avtors{Shnurkov~P.\,V., Gorshenin~A.\,K., and Belousov~V.\,V.}
Analytical solution of~the~optimal control\linebreak
\\[-12pt]
\hspace*{23pt}task of~a~semi-Markov
process with~finite set of~states&4&72--88\\[.23pt]
\Avtors{Shnurkov~P.\,V., Zasypko~V.\,V., Belousov~V.\,V., and
Gorshenin~A.\,K.} Development of the algorithm of numerical solution
of the optimal investment control problem\linebreak
\\[-12pt]
\hspace*{23pt}in the closed dynamical model of three-sector economy&1&82--95\\[.23pt]
\Avtors{Shorgin~S.\,Ya.} see~Gaidamaka~Yu.\,V.&&\\[.23pt]
\Avtors{Shorgin~V.\,S.} see~Agalarov~Ya.\,M.&&\\[.23pt]
\Avtors{Shubnikov~S.\,K.} see~Minin~V.\,A.&&\\[.23pt]
\Avtors{Sidorkin~I.\,I.} see~Arkhipov~O.\,P.&&\\[.23pt]
\Avtors{Sinitsyn~I.\,N.} Analytical modeling of processes in stochastic
systems with complex fractional\linebreak
\\[-12pt]
\hspace*{23pt}order Bessel nonlinearities&3&55--65\\[.23pt]
\Avtors{Sinitsyn~I.\,N.} Orthogonal supoptimal filters for nonlinear
stochastic systems on manifolds&1&34--44\\[.23pt]
\Avtors{Sinitsyn~I.\,N.\ and Korepanov~E.\,R.} Normal Pugachev
conditionally-optimal filters and extra-\linebreak
\\[-12pt]
\hspace*{23pt}polators for state linear stochastic systems&2&14--23\\[.23pt]
\Avtors{Sinitsyn~I.\,N.\ and Sinitsyn~V.\,I.} Analytical modeling of
distributions in stochastic systems on\linebreak
\\[-12pt]
\hspace*{23pt}manifolds based on ellipsoidal approximation&1&45--55\\[.23pt]
\Avtors{Sinitsyn~I.\,N., Sinitsyn~V.\,I., and
Korepanov~E.\,R.} Ellipsoidal suboptimal filters for nonlinear\linebreak
\\[-12pt]
\hspace*{23pt}stochastic systems on manifolds&2&24--35\\[.23pt]
\Avtors{Sinitsyn~V.\,I.} see~Sinitsyn~I.\,N.&&\\[.23pt]
\Avtors{Sinitsyn~V.\,I.} see~Sinitsyn~I.\,N.&&\\[.23pt]
\Avtors{Skvortsov~N.\,A.} see~Stupnikov~S.\,A.&&\\[.23pt]
\Avtors{Sokolov~I.\,A.} see~Chertok~A.\,V.&&\\
\end{tabular}
}
\pagebreak

\def\leftfootline{\small{\textbf{\thepage}
\hfill INFORMATIKA I EE PRIMENENIYA~--- INFORMATICS AND APPLICATIONS\ \ \ 2016\
\ \ volume~10\ \ \ issue\ 4}
}%
 \def\rightfootline{\small{INFORMATIKA I EE PRIMENENIYA~---
INFORMATICS AND APPLICATIONS\ \ \ 2016\ \ \ volume~10\ \ \ issue\ 4
\hfill \textbf{\thepage}}}

\def\leftkol{2016 AUTHOR INDEX} % ENGLISH ABSTRACTS}

\def\rightkol{2016 AUTHOR INDEX} %ENGLISH ABSTRACTS}


{\tabcolsep=3pt
\begin{tabular}{p{382pt}cc}
&\textbf{Issue} & \textbf{Page}\\[6pt]
\Avtors{Sopin~E.\,S.} see~Gaidamaka~Yu.\,V.&&\\
\Avtors{Strijov~V.\,V.} see~Goncharov~A.\,V.&&\\
\Avtors{Strijov~V.\,V.} see~Isachenko~R.\,V.&&\\
\Avtors{Strijov~V.\,V.} see~Karasikov~M.\,E.&&\\
\Avtors{Stupnikov~S.\,A., Briukhov~D.\,O., and Skvortsov~N.\,A.}
Co-lending systemic risk analysis over\linebreak
\\[-12pt]
\hspace*{23pt}heterogeneous data collections&1&23--33\\
\Avtors{Stupnikov~S.\,A.} see~Kalinichenko~L.\,A.&&\\
\Avtors{Suchkov~A.\,P.} see~Zatsarinny~A.\,A.&&\\
\Avtors{Timonina~E.\,E.} see~Grusho~A.\,A.&&\\
\Avtors{Titova~A.\,I.} see~Kudryavtsev~A.\,A.&&\\
\Avtors{Turlikov~A.\,M.} see~Ometov~A.\,Ya.&&\\
\Avtors{Tyrsin~A.\,N.\ and Serebryanskii~S.\,M.} Recognition of
dependences on the basis of inverse\linebreak
\\[-12pt]
\hspace*{23pt}mapping&2&58--64\\
\Avtors{Ulyanov~V.\,V.} see~Markov~A.\,S.&&\\
\Avtors{Ushakov~V.\,G.} Queueing system with working vacations and
hyperexponential input stream&2&92--97\\
\Avtors{Ushakov~V.\,G.} see~Leontyev~N.\,D.&&\\
\Avtors{Volnova~A.\,A.} see~Kalinichenko~L.\,A.&&\\
\Avtors{Yakovlev~O.\,A.\ and Gasilov~A.\,V.} Speeded-up stereo
matching using geodesic support weights&3&\hphantom{1}98--104\\
\Avtors{Zabezhailo~M.\,I.} see~Grusho~A.\,A.&&\\
\Avtors{Zabezhailo~M.\,I.} see~Grusho~A.\,A.&&\\
\Avtors{Zakharova~T.\,V.\ and Shestakov~O.\,V.} Precision analysis of
wavelet processing of aerodynamic\linebreak
\\[-12pt]
\hspace*{23pt}flow patterns&3&46--54\\
\Avtors{Zalizniak~Anna~A.\ and Kruzhkov~M.\,G.} Database
of~Russian impersonal verbal constructions&4&132--141\\
\Avtors{Zasypko~V.\,V.} see~Shnurkov~P.\,V.&&\\
\Avtors{Zatsarinny~A.\,A.\ and Suchkov~A.\,P.} Systems engineering
approaches to~the~establishment of\linebreak
\\[-12pt]
\hspace*{23pt}a~system for~decision support based
on~situational analysis&4&105--113\\
\Avtors{Zatsarinny~A.\,A.} see~Grusho~A.\,A.&&\\
\Avtors{Zatsman~I.\,M., Inkova~O.\,Yu., Kruzhkov~M.\,G., and
Popkova~N.\,A.} Representation of cross-\linebreak
\\[-12pt]
\hspace*{23pt}lingual knowledge about
connectors in supracorpora databases&1&106--118\\
\Avtors{Zatsman~I.\,M.} see~Minin~V.\,A.&&\\
\Avtors{Zeifman~A.\,I.} see~Korolev~V.\,Yu.&&\\
\Avtors{Zeifman~A.\,I.} see~Korolev~V.\,Yu.&&\\
\end{tabular}
}

%\thispagestyle{myheadings}
\def\leftfootline{\small{\textbf{\thepage}
\hfill INFORMATIKA I EE PRIMENENIYA~--- INFORMATICS AND APPLICATIONS\ \ \ 2016\
\ \ volume~10\ \ \ issue\ 4}
}%
 \def\rightfootline{\small{INFORMATIKA I EE PRIMENENIYA~---
INFORMATICS AND APPLICATIONS\ \ \ 2016\ \ \ volume~10\ \ \ issue\ 4
\hfill \textbf{\thepage}}}

 \label{end\stat}

\newpage

%\def\stat{rekl}
%\label{preobr}

%\def\tit{АКАДЕМИК ПУГАЧЁВ  ВЛАДИМИР СЕМЁНОВИЧ\\
%25.03.1911--25.03.1998}


%   \vspace*{-48pt}
%   \begin{center}\LARGE
%Академик Пугачёв  Владимир Семёнович\\ (25.03.1911--25.03.1998)
%   \end{center}
   
   %\vspace*{2.5mm}
   
   \begin{center}

{\prgsh\LARGE
ОБЪЯВЛЕНИЯ О КОНФЕРЕНЦИЯХ}

\end{center}
%\hrule

\vspace*{6pt}

   
   \vspace*{10mm}
   
   \thispagestyle{empty}

\noindent
\begin{tabular}{cc}
%\begin{center}
\multicolumn{1}{c}{\raisebox{-40pt}[0pt][0pt]{\mbox{%
\epsfxsize=33mm
\epsfbox{vspu.eps}
}}}
%\end{center}
&
\tabcolsep=0pt\begin{tabular}{c}
{\prg{\Large\textbf{XII Всероссийское совещание}}}\\[6pt]
{\prg{\Large\textbf{по проблемам управления}}}\\[12pt]
{\prg{\large 16--19 июня 2014~г.}}\\[6pt] 
{\prg{\large Институт проблем управления имени В.\,А.~Трапезникова РАН}}\\[6pt]
{\prg{\large Москва, Россия}}
\end{tabular}
\end{tabular}

\vspace*{60pt}

     
 { %\large    
 XII Всероссийское совещание по проблемам управления (ВСПУ XII), посвященное 75-летию 
Института проблем управления (ИПУ) имени В.\,А.~Трапезникова РАН, проводится 16--19~июня 
2014~г.\ 
в ИПУ РАН (г.~Москва, Россия). ВСПУ XII организуется ИПУ РАН при поддержке РФФИ, Отделения 
энергетики, машиностроения, механики и процессов управления Российской академии наук, 
Российского 
национального комитета по автоматическому управлению, Академии навигации и управ\-ле\-ния 
движением, 
Научного совета РАН по комплексным проблемам управления и автоматизации, Совета по 
мехатронике и робототехнике РАН. Официальный язык Совещания~--- русский.

\vspace*{24pt}
     
     \textbf{Направления работы}
     \begin{enumerate}[1.]
\item Теория систем управления
\item Управление подвижными объектами и навигация
\item Интеллектуальные системы управления
\item Управление в промышленности, транспортом и логистикой
\item Управление системами междисциплинарной природы
\item Средства измерения, вычислений и контроля в управлении
\item Системный анализ и принятие решений в задачах управления
\item Информационные технологии в управлении
\item Проблемы образования в области управления: современное содержание и технологии обучения
\end{enumerate}

\vspace*{24pt}

     Подробная информация о Совещании находится на сайте {\sf http://vspu2014.ipu.ru}. Срок 
окончательной подачи докладов через систему подачи докладов на сайте~--- \textbf{30~ноября} 
2013~г.
}

%\include{rekl-1}

%\end{document}

%   \vspace*{-48pt}

\begin{center}
\vspace*{6pt}
\mbox{%
\epsfxsize=53.502mm
\epsfbox{foto-1.eps}
}
\end{center}

\vspace*{6pt} %Академик


   \begin{center}
\fbox{\Large\textbf{Профессор Игорь Алексеевич Ушаков}}\\[12pt]
\textbf{\large 22.01.1935--27.02.2015}
   \end{center}


   %\vspace*{2.5mm}

   \vspace*{5mm}

   \thispagestyle{empty}

%\

%\vspace*{-12pt}


Редакционный совет и редакционная коллегия журнала <<Информатика и~её применения>> с~глубоким прискорбием извещают, что 27~февраля 2015~г.\ после тяжелой
и~продолжительной болезни скончался Игорь Алексеевич Ушаков~--- доктор технических наук, профессор, член редколлегии журнала <<Информатика и ее применения>>.

Игорь Алексеевич Ушаков окончил Московский авиационный институт, в~1963~г.\ защитил кандидатскую, а~в~1968~г.~--- докторскую диссертацию. С~1958 по 1989~гг.\ работал в~ряде научно-исследовательских организаций СССР, в~том числе руководил отделами в~НИИ АА и~ВЦ АН СССР; с 1969 по 1989 гг. преподавал в~МФТИ (был профессором, а~затем заведующим кафедрой) и~в~МЭИ. С~1989~г.~---- в~США: являлся профессором университета Дж.\ Вашингтона, университета Дж.\ Мэйсона и~Калифорнийского университета, сотрудником компаний MCI, Qualcomm и Hughes.

И.\,А.~Ушаков с момента основания журнала <<Надежность и~контроль качества>> был заместителем ответственного редактора, а~затем на протяжении многих лет членом редколлегии. В~2006~г.\ основал электронный международный журнал ``Reliability: Theory \& Application'', главным редактором которого оставался до конца жизни.

Учебниками и справочниками по теории надежности, написанными И.\,А.~Ушаковым, пользовались и~пользуются несколько поколений ученых и~специалистов в~разных странах мира.

Игорь Алексеевич всегда уделял огромное внимание работе с~молодежью; более~50 его учеников защитили докторские и~кандидатские диссертации.

И.\,А.~Ушаков вел активную научно-про\-све\-ти\-тель\-скую деятельность. В~частности, он был одним из организаторов и~руководителей Московского кабинета качества и~надежности при Политехническом музее (целью этого Кабинета было оказание консультаций работникам промышленных предприятий и~чтение курсов лекций для инженеров, занимающихся проблемой надежности). Находясь в~США, И.\,А.~Ушаков создал международный ин\-тер\-нет-фо\-рум им.\ Б.\,В.~Гнеденко, объединивший около~400~видных специалистов по приложениям теории вероятностей и~математической статистики, преимущественно в~об\-ласти теории надежности и~анализа риска, из десятков стран мира; коллективным членов этого Форума является и~наш журнал. Цели Форума~--- содействие контактам между специалистами из разных стран, организация обмена профессиональными 
новостями и~информацией (новые публикации, предстоящие события и~др.). Также необходимо отметить большое число на\-уч\-но-по\-пу\-ляр\-ных работ, опубликованных И.\,А.~Ушаковым.

И.\,А.~Ушаков обладал большим личным обаянием, имел широкий круг интересов. Все знавшие И.\,А.~Ушакова всегда будут помнить его как замечательного ученого и~прекрасного человека.

\bigskip

Редакционный совет и редакционная коллегия журнала <<Информатика и~её применения>> 
выражают глубокие соболезнования родным и близким покойного, всем, кто его знал и~работал с~ним.



%\end{document}

%\include{IPPM-25}

\def\stat{cont-rus}
{%\hrule\par
%\vskip 7pt % 7pt
\vspace*{-24pt}
\raggedleft\Large \bf%\baselineskip=3.2ex
Правила подготовки рукописей  для публикации в журнале
<<Информатика~и~её~применения>> \vskip 8pt
    \hrule
    \par
\vskip 14pt plus 6pt minus 3pt }

\label{st\stat}

\def\tit{\ }

\def\aut{\ }
\def\auf{\ }

\def\leftkol{\ }
% Правила подготовки рукописей  для публикации в журнале
%<<Информатика и её применения>>

\def\rightkol{\ }
%Правила подготовки рукописей  для публикации в журнале
%<<Информатика и её применения>>}


\titele{\tit}{\aut}{\auf}{\leftkol}{\rightkol}


\vspace*{-60pt}
{ %\small

Журнал <<Информатика и её применения>>
публикует теоретические, обзорные и дискуссионные статьи,
посвященные научным исследованиям и разработкам в области
информатики и ее приложений.

Журнал издается на русском языке. По специальному решению
редколлегии отдельные статьи могут печататься на английском языке.

Тематика журнала охватывает следующие направления:
\begin{itemize}
\item теоретические основы информатики;\\[-15pt]
      \item
математические методы исследования сложных систем и процессов;\\[-15pt]
           \item
информационные системы и сети;\\[-15pt]
                \item
информационные технологии;\\[-15pt]
                     \item
архитектура и программное обеспечение вычислительных комплексов и сетей.\\[-15pt]
\end{itemize}


\noindent
\begin{enumerate}[1.]
\item В журнале печатаются статьи, содержащие результаты, ранее не опубликованные и
не предназначенные к одновременной публикации в других изданиях.

%Публикация не должна нарушать закон об авторских правах.
Публикация предоставленной автором(ами) рукописи не должна нарушать 
положений глав~69, 70 раздела~VII части~IV Гражданского кодекса, 
которые определяют права на результаты интеллектуальной деятельности 
и~средства индивидуализации, в~том числе авторские права, в~РФ.

Ответственность за нарушение авторских прав, в~случае предъявления претензий к~редакции журнала,  
несут авторы статей.



Направляя рукопись в редакцию, авторы сохраняют свои права на данную
рукопись и при этом передают учредителям и редколлегии журнала неисключительные права на
издание статьи на русском языке 
(или на языке статьи, если он отличен от рус\-ско\-го) и~на перевод ее на английский
язык, а~также на
ее распространение в России и за рубежом. 
Каждый автор должен представить в~редакцию подписанный 
с~его стороны <<Лицензионный договор о~передаче неисключительных прав 
на использование произведения>>, текст которого размещен по адресу 
{\sf http://www.ipiran.ru/publications/licence.doc}. 
Этот договор может быть пред\-став\-лен в~бумажном (в~2-х экз.)\ 
или в~электронном виде (отсканированная копия заполненного и~подписанного документа).




Редколлегия вправе запросить у авторов экспертное заключение о возможности
пуб\-ли\-ка\-ции пред\-став\-лен\-ной статьи в открытой печати.\\[-13.5pt]

\item К статье прилагаются данные автора (авторов) (см.\ п.~8). При наличии нескольких
авторов указывается фамилия автора, ответственного за переписку с редакцией.\\[-13.5pt]

\item Редакция журнала осуществляет экспертизу присланных статей в соответствии с
принятой в журнале процедурой рецензирования.

Возвращение рукописи на доработку не означает ее принятия к печати.

Доработанный вариант с ответом на замечания рецензента необходимо прислать в
редакцию.\\[-13.5pt]

\item Решение редколлегии о публикации статьи или ее отклонении сообщается авторам.

Редколлегия может также направить авторам текст рецензии на их статью. Дискуссия по
поводу отклоненных статей не ведется.\\[-13.5pt]

%\pagebreak

\item Редактура статей высылается авторам для просмотра. Замечания к редактуре должны
быть присланы авторами в кратчайшие сроки.\\[-13.5pt]

\item Рукопись предоставляется в электронном виде в форматах MS WORD (.doc или
.docx) или \LaTeX\  (.tex), дополнительно~--- в формате .pdf, на дискете, лазерном диске
или электронной почтой. Предоставление бумажной рукописи необязательно.\\[-13.5pt]

\item При подготовке рукописи в MS Word рекомендуется использовать следующие
настройки.

Параметры страницы:
формат~--- А4; ориентация~--- книжная; поля (см): внутри~--- 2,5, снаружи~--- 1,5,
сверху~--- 2, снизу~--- 2, от края до нижнего колонтитула~--- 1,3.

Основной текст: стиль~--- <<Обычный>>, шрифт~--- Times New Roman, размер~---
14~пунк\-тов, абзацный отступ~--- 0,5~см, 1,5~интервала, выравнивание~--- по ширине.

\pagebreak

\def\leftkol{Правила подготовки рукописей  для публикации в журнале
<<Информатика и её применения>>}

\def\rightkol{Правила подготовки рукописей  для публикации в журнале
<<Информатика и её применения>>}



Рекомендуемый объем рукописи~--- не свыше 10~страниц указанного формата.
При превышении указанного объема редколлегия вправе потребовать от 
автора сокращения объема рукописи.


Сокращения слов, помимо стандартных, не допускаются. Допускается минимальное
количество аббревиатур.


Все страницы рукописи нумеруются.

Шаблоны оформления представлены в интернете:

\noindent
 {\sf
http://www.ipiran.ru/journal/template\_iiep\_ssi\_2024.zip}\\[-14pt]

\item Статья должна содержать следующую информацию на {\bfseries\textit{русском и
английском языках}}:\\[-16pt]

\begin{itemize}
\item название статьи;\\[-15pt]
\item Ф.И.О.\ авторов, на английском можно только имя и фамилию;\\[-15pt]
\item место работы, с указанием почтового адреса организации и электронного адреса каждого
автора;\\[-15pt]
\item сведения об авторах, в соответствии с форматом, образцы которого
представлены на страницах:



\def\leftfootline{\small{\textbf{\thepage}
\hfill ИНФОРМАТИКА И ЕЁ ПРИМЕНЕНИЯ\ \ \ том\ 18\ \ \ выпуск\ 3\ \ \ 2024}
}%
 \def\rightfootline{\small{ИНФОРМАТИКА И ЕЁ ПРИМЕНЕНИЯ\ \ \ том\ 18\ \ \ выпуск\ 3\ \ \ 2024
\hfill \textbf{\thepage}}}



{\sf http://www.ipiran.ru/journal/issues/2013\_07\_01/authors.asp} и

{\sf http://www.ipiran.ru/journal/issues/2013\_07\_01\_eng/authors.asp};
\item аннотация (не менее 100~слов на каждом из языков). Аннотация~--- это краткое
резюме работы, которое может публиковаться отдельно. Она является основным
источником информации в~ин\-фор\-ма\-ци\-он\-ных системах и базах данных. Английская
аннотация должна быть оригинальной, может не быть дословным переводом русского
текста и должна быть написана хорошим английским языком. В~аннотации не должно
быть ссылок на литературу и, по возможности, формул;\\[-15pt]
\item ключевые слова~--- желательно из принятых в мировой
на\-уч\-но-тех\-ни\-че\-ской литературе тематических тезаурусов. Предложения не
могут быть ключевыми словами;\\[-15pt]
\item источники финансирования работы (ссылки на гранты, проекты,
поддерживающие организации и~т.\,п.).
\end{itemize}



%\pagebreak

\item  Требования к спискам литературы.\\[-14pt]

Ссылки на литературу в тексте статьи нумеруются (в квадратных скобках) и
располагаются в каждом из списков литературы в порядке  первых упоминаний. Если источник имеет DOI и/или EDN,
то их необходимо указывать.

Списки литературы представляются в двух вариантах:\\[-14pt]


\noindent
\begin{enumerate}[(1)]
\item \textbf{Список литературы к русскоязычной части}. Русские и английские
работы~---  на языке и в алфавите оригинала;\\[-14.5pt]
\item  \textbf{References}. Русские работы и работы на других языках~--- в латинской
транслитерации с переводом на английский язык; английские работы и работы на других
языках~--- на языке оригинала.
\end{enumerate}

Необходимо для составления списка ``References'' пользоваться размещенной на сайте
{\sf http://www. translit.net/ru/bgn/} бесплатной программой транслитерации русского
 текста в~латиницу. %, при этом в~за\-клад\-ке <<варианты\ldots>> следует выбратьопцию BGN.

Список литературы ``References'' приводится полностью отдельным блоком, повторяя все
позиции из списка литературы к русскоязычной части, независимо от того, имеются или
нет в нем иностранные источники. Если в списке литературы к русскоязычной части есть
ссылки на иностранные публикации, набранные латиницей, они полностью повторяются в
списке ``References''.

Ниже приведены примеры ссылок на различные виды публикаций в списке ``References''.

\def\leftfootline{\small{\textbf{\thepage}
\hfill ИНФОРМАТИКА И ЕЁ ПРИМЕНЕНИЯ\ \ \ том\ 18\ \ \ выпуск\ 3\ \ \ 2024}
}%
 \def\rightfootline{\small{ИНФОРМАТИКА И ЕЁ ПРИМЕНЕНИЯ\ \ \ том\ 18\ \ \ выпуск\ 3\ \ \ 2024
\hfill \textbf{\thepage}}}

{\small

\noindent
\textbf{Описание статьи из журнала:}

\Aue{Zagurenko, A.\,G., V.\,A.~Korotovskikh, A.\,A.~Kolesnikov, A.\,V.~Timonov, and D.\,V.~Kardymon}. 2008.
Tekhniko-ekonomicheskaya optimizatsiya dizayna gidrorazryva plasta [Technical and
economic optimization of the design
of hydraulic fracturing]. \textit{Neftyanoe hozyaystvo} [\textit{Oil Industry}] 11:54--57.

\Aue{Zhang, Z., and D.~Zhu}. 2008. Experimental research on the localized
electrochemical micromachining. \textit{Russ. J.~Electrochem.}  44(8):926--930.
{\sf doi:10.1134/S1023193508080077}.

\noindent
\textbf{Описание статьи из электронного журнала:}

\Aue{Swaminathan, V., E.~Lepkoswka-White, and B.\,P.~Rao}. 1999. Browsers or buyers in cyberspace? An
investigation of electronic factors influencing electronic exchange. \textit{JCMC}
5(2). Available at: {\sf http://www.ascusc.org/jcmc/vol5/issue2/} (accessed April~28, 2011).

\def\leftkol{Правила подготовки рукописей  для публикации в журнале
<<Информатика и её применения>>}

\def\rightkol{Правила подготовки рукописей  для публикации в журнале
<<Информатика и её применения>>}


\noindent
\textbf{Описание статьи из продолжающегося издания (сборника трудов):}

\Aue{Astakhov, M.\,V., and T.\,V.~Tagantsev}. 2006. Eksperimental'noe
issledovanie prochnosti soedineniy ``stal'--kompozit'' [Experimental study of
the strength of joints ``steel--composite'']. \textit{Trudy MGTU
``Matematicheskoe modelirovanie slozhnykh tekh\-ni\-che\-skikh sistem''}
[\textit{Bauman MSTU ``Mathematical Modeling of Complex Technical
Systems'' Proceedings}]. 593:125--130.


\pagebreak



\noindent
\textbf{Описание материалов конференций:}

\Aue{Usmanov, T.\,S., A.\,A.~Gusmanov, I.\,Z.~Mullagalin, R.\,Ju.~Muhametshina, A.\,N.~Chervyakova, and
A.\,V.~Sveshnikov}. 2007. Osobennosti proektirovaniya razrabotki mestorozhdeniy
s primeneniem gidrorazryva
plasta [Features of the design of field development with the use of hydraulic fracturing].
\textit{Trudy 6-go
Mezhdu\-na\-rod\-no\-go Simpoziuma ``Novye resursosberegayushchie tekhnologii nedropol'zovaniya i povysheniya
neftegazootdachi''} [\textit{6th  Symposium (International) ``New Energy Saving Subsoil Technologies and
the Increasing of the Oil and Gas Impact'' Proceedings}]. Moscow. 267--272.



\def\leftfootline{\small{\textbf{\thepage}
\hfill ИНФОРМАТИКА И ЕЁ ПРИМЕНЕНИЯ\ \ \ том\ 18\ \ \ выпуск\ 3\ \ \ 2024}
}%
 \def\rightfootline{\small{ИНФОРМАТИКА И ЕЁ ПРИМЕНЕНИЯ\ \ \ том\ 18\ \ \ выпуск\ 3\ \ \ 2024
\hfill \textbf{\thepage}}}



\noindent
\textbf{Описание книги (монографии, сборники):}



Lindorf, L.\,S., and L.\,G.~Mamikoniants, eds. 1972.
\textit{Ekspluatatsiya turbogeneratorov s neposredstvennym
okhlazhdeniem} [\textit{Operation of turbine generators with direct cooling}].
Moscow: Energy Publs. 352~p.


\Aue{Latyshev, V.\,N.} 2009. \textit{Tribologiya rezaniya. Kn.~1: Friktsionnye protsessy
pri rezanii metallov}
[\textit{Tribology of cutting. Vol.~1: Frictional processes in metal cutting}]. Ivanovo: Ivanovskii
State Univ. 108~p.

\def\leftkol{Правила подготовки рукописей  для публикации в журнале
<<Информатика и её применения>>}

\def\rightkol{Правила подготовки рукописей  для публикации в журнале
<<Информатика и её применения>>}

\noindent
\textbf{Описание переводной книги}
(в списке литературы к русскоязычной части необходимо указать:~/ Пер.\ с англ.~---
после названия книги, а в конце ссылки указать оригинал книги в круглых скобках):
\begin{enumerate}[1.]
\item  В русскоязычной части:

\def\leftfootline{\small{\textbf{\thepage}
\hfill ИНФОРМАТИКА И ЕЁ ПРИМЕНЕНИЯ\ \ \ том\ 18\ \ \ выпуск\ 3\ \ \ 2024}
}%
 \def\rightfootline{\small{ИНФОРМАТИКА И ЕЁ ПРИМЕНЕНИЯ\ \ \ том\ 18\ \ \ выпуск\ 3\ \ \ 2024
\hfill \textbf{\thepage}}}

\Au{Тимошенко С.\,П., Янг Д.\,Х., Уивер~У.}
Колебания в инженерном деле~/ Пер.\ с англ.~--- М.: Машиностроение, 1985. 472~с.
(\Au{Timoshenko~S.\,P., Young~D.\,H., Weaver~W.}
Vibration problems in engineering.~--- 4th ed.~--- New York, NY, USA: Wiley, 1974. 521~p.)\\[-13.5pt]
\item  В англоязычной части:

\Aue{Timoshenko, S.\,P., D.\,H.~Young, and W.~Weaver}.
1974. \textit{Vibration problems in engineering}. 4th ed. New York: 
Wiley. 521~p.
\end{enumerate}

\vspace*{-3pt}


\noindent
\textbf{Описание неопубликованного документа:}


\Aue{Latypov, A.\,R., M.\,M.~Khasanov, and V.\,A.~Baikov}.
2004 (unpubl.). Geologiya i~dobycha (NGT GiD) [Geology and production (NGT GiD)]. Certificate on official registration of the computer program
No.\,2004611198. 

\noindent
\textbf{Описание интернет-ресурса:}


Pravila tsitirovaniya istochnikov [Rules for the citing of sources]. Available at: {\sf
http://www.scribd.com/doc/1034528/} (accessed February~7, 2011).

%\pagebreak

\noindent
\textbf{Описание диссертации или автореферата диссертации:}

\Aue{Semenov, V.\,I.}
2003. Matematicheskoe modelirovanie plazmy v sisteme kompaktnyy tor [Mathematical
modeling of the plasma in the compact torus].  Moscow.  D.Sc.\ Diss. 272~p.

\Aue{Kozhunova, O.\,S.} 2009. Tekhnologiya razrabotki semanticheskogo
slovarya informatsionnogo monitoringa [Technology of development of
semantic dictionary of information monitoring system].  Moscow: IPI RAN. PhD Thesis. 23~p.


\noindent
\textbf{Описание ГОСТа:}

GOST 8.586.5-2005. 2007. Metodika vypolneniya izmereniy. Izmerenie raskhoda i~kolichestva zhidkostey i~gazov
s~pomoshch'yu standartnykh suzhayushchikh ustroystv [Method of measurement.
Measurement of flow rate and volume of liquids and gases by means of orifice devices]. Moscow:
Standardinform  Publs. 10~p.

\noindent
\textbf{Описание патента:}

\Aue{Bolshakov, M.\,V., A.\,V.~Kulakov, A.\,N.~Lavrenov, and M.\,V.~Palkin}.
2006. Sposob orientirovaniya po krenu letatel'nogo
apparata s opti\-che\-skoy golovkoy
samonavedeniya [The way to orient on the roll of aircraft with optical homing head].
Patent RF No.\,2280590.
}

\item Присланные в редакцию материалы авторам не возвращаются.\\[-13.5pt]

\item При отправке файлов по электронной почте просим придерживаться следующих
правил:
\begin{itemize}
\item указывать в поле subject (тема) название журнала и фамилию автора;\\[-13.5pt]
\item указывать в тексте письма название статьи, авторов и~журнал, в~который направляется статья;\\[-13.5pt]
\item использовать attach (присоединение);\\[-13.5pt]
\item в состав электронной версии статьи должны входить: файл, содержащий текст
статьи, и файл(ы), содержащий(е) иллюстрации.\\[-13.5pt]
\end{itemize}

\item Журнал <<Информатика и её применения>> является некоммерческим изданием.
Плата за публикацию не взимается, гонорар авторам не выплачивается.
\end{enumerate}



\def\leftfootline{\small{\textbf{\thepage}
\hfill ИНФОРМАТИКА И ЕЁ ПРИМЕНЕНИЯ\ \ \ том\ 18\ \ \ выпуск\ 3\ \ \ 2024}
}%
 \def\rightfootline{\small{ИНФОРМАТИКА И ЕЁ ПРИМЕНЕНИЯ\ \ \ том\ 18\ \ \ выпуск\ 3\ \ \ 2024
\hfill \textbf{\thepage}}}


\vspace*{-1mm}

\begin{center}

\textbf{Адрес редакции журнала <<Информатика и её применения>>:} \\




Москва 119333, ул.~Вавилова, д.~44, корп.~2, ФИЦ ИУ РАН\\[-10pt]

\

Тел.: +7\,(499)\,135-86-92\ \ Факс:  +7\,(495)\,930-45-05\\[-10pt]

 \

e-mail:   {\sf iiep@frccsc.ru} (Стригина Светлана Николаевна)\\[-10pt]

\

{\sf http://www.ipiran.ru/journal/issues/}
\end{center}
}


\def\leftkol{Правила подготовки рукописей  для публикации в журнале
<<Информатика и её применения>>}

\def\rightkol{Правила подготовки рукописей  для публикации в журнале
<<Информатика и её применения>>}


\def\leftfootline{\small{\textbf{\thepage}
\hfill ИНФОРМАТИКА И ЕЁ ПРИМЕНЕНИЯ\ \ \ том\ 18\ \ \ выпуск\ 3\ \ \ 2024}
}%
 \def\rightfootline{\small{ИНФОРМАТИКА И ЕЁ ПРИМЕНЕНИЯ\ \ \ том\ 18\ \ \ выпуск\ 3\ \ \ 2024
\hfill \textbf{\thepage}}} 
\def\stat{podg-e}
{%\hrule\par
%\vskip 7pt % 7pt
\vspace*{-24pt}
\raggedleft\Large \bf%\baselineskip=3.2ex
Requirements for manuscripts submitted to Journal
``Informatics~and~Applications'' \vskip 8pt
    \hrule
    \par
\vskip 21pt plus 6pt minus 3pt }

\label{st\stat}

\def\tit{\ }

\def\aut{\ }
\def\auf{\ }

\def\leftkol{\ }

\def\rightkol{\ }
%Requirements for manuscripts submitted to Journal
%``Informatics~and~Applications''}

\titele{\tit}{\aut}{\auf}{\leftkol}{\rightkol}

\def\leftfootline{\small{\textbf{\thepage}
\hfill INFORMATIKA I EE PRIMENENIYA~--- INFORMATICS AND APPLICATIONS\ \ \ 2019\
\ \ volume~13\ \ \ issue\ 4}
}%
 \def\rightfootline{\small{INFORMATIKA I EE PRIMENENIYA~--- INFORMATICS AND APPLICATIONS\ \ \ 2019\ \ \ volume~13\ \ \ issue\ 4
\hfill \textbf{\thepage}}}

\vspace*{-60pt}

{\small

\noindent
Journal ``Informatics and Applications'' (Inform.\ Appl.)
publishes theoretical, review, and discussion
articles on the research and development in the
field of informatics and its applications.

The journal is published in Russian.
By a special decision of the editorial
board, some articles can be published in English.


The topics covered include the following areas:
\begin{itemize}
               \item
     theoretical fundamentals of informatics; \\[-14pt]
\item
mathematical methods for studying complex systems and processes; \\[-14pt]
\item
information systems and networks;\\[-14pt]
\item
information technologies; and \\[-14pt]
\item
architecture and software of computational complexes and networks. \\[-14pt]
\end{itemize}

\noindent
\begin{enumerate}[1.]
\item The Journal publishes original articles which have not been published before and are not
intended for simultaneous publication in other editions. An article submitted to the Journal must not violate the
Copyright law. Sending the manuscript to the Editorial Board, the authors retain all rights of the
owners of the manuscript and transfer the nonexclusive rights to publish the article in Russian
(or the language of the article, if not Russian) and its distribution in Russia and abroad to the
Founders and the Editorial Board. Authors should submit a letter to the Editorial Board in the
following form:

{\bfseries\textit{Agreement on the transfer of rights to publish:}}

``\textit{We, the undersigned authors of the manuscript ``\ldots'', pass to the
Founder and the Editorial Board of the Journal ``Informatics and Applications''
the nonexclusive right to publish the manuscript of the article in Russian (or
in English) in both print and electronic versions of the Journal. We affirm
that this publication does not violate the Copyright of other persons or
organizations.}

\textit{Author(s) signature(s): (name(s), address(es), date).}

This agreement should be submitted in paper form or in the form of a scanned copy (signed by
the authors).


%The Editorial Board has the right to request from the authors an official expert conclusion that
%the submitted article has no secret data prohibited for publication. \\[-13.5pt]
\item
A submitted article should be attached with \textbf{the data on the author(s)} (see item~8). If
there are several authors, the contact person should be indicated who is responsible for
correspondence with the Editorial Board and other authors about revisions and final approval
of the proofs.\\[-13.5pt]

\item The Editorial Board of the Journal examines the article according to the established
reviewing procedure. If the authors receive their article for correction after reviewing, it does not
mean that the article is approved for publication. The corrected article should be sent to the
Editorial Board for the subsequent review and approval.\\[-13.5pt]

\item The decision on the article publication or its rejection is communicated to the authors. The
Editorial Board may also send the reviews on the submitted articles to the authors. Any
discussion upon the rejected articles is not possible.\\[-13.5pt]

\item The edited articles will be sent to the authors for proofread. The comments of the authors
to the edited text of the article should be sent to the Editorial Board as soon as possible.\\[-13.5pt]

\item The manuscript of the article should be presented electronically in the MS WORD (.doc or
.docx) or \LaTeX\ (.tex) formats, and additionally in the .pdf format. All documents
 may be sent
by e-mail or provided on a CD or diskette. A~hard copy submission is not necessary.\\[-13.5pt]

\item The recommended typesetting instructions for manuscript.

Pages parameters: format A4, portrait orientation, document margins (cm): left~--- 2.5, right~---
1.5, above~--- 2.0, below~--- 2.0, footer 1.3.

Text: font~---Times New Roman, font size~--- 14, paragraph indent~--- 0.5, line spacing~--- 1.5,
justified alignment.

The recommended manuscript size: not more than 15~pages of the specified format.
If the specified size exceeded, the editorial board is entitled to require the author
to reduce the manuscript.

Use only standard abbreviations. Avoid  abbreviations in the title and
abstract. The full term for which an abbreviation stands should precede
its first use in the text unless it is a standard unit of measurement.

All pages of the manuscript should be numbered.

The templates for the manuscript typesetting are presented on site: {\sf
http://www.ipiran.ru/journal/template.doc}.\\[-13.5pt]


%\def\leftkol{Requirements for manuscripts submitted to Journal
%``Informatics~and~Applications''}

\item The articles should enclose data both in \textbf{Russian and English}:
\begin{itemize}
\item title;\\[-13.5pt]
\item author's name and surname;\\[-13.5pt]
\item affiliation~--- organization, its address with ZIP code, city, country, and
official e-mail address;\\[-13.5pt]
\item data on authors according to the format: (see site)

{\sf http://www.ipiran.ru/journal/issues/2013\_07\_01/authors.asp}  and

{\sf  http://www.ipiran.ru/journal/issues/2013\_07\_01\_eng/authors.asp};\\[-13.5pt]

\pagebreak

\def\leftfootline{\small{\textbf{\thepage}
\hfill INFORMATIKA I EE PRIMENENIYA~--- INFORMATICS AND APPLICATIONS\ \ \ 2019\
\ \ volume~13\ \ \ issue\ 4}
}%
 \def\rightfootline{\small{INFORMATIKA I EE PRIMENENIYA~--- INFORMATICS AND APPLICATIONS\ \ \ 2019\ \ \ volume~13\ \ \ issue\ 4
\hfill \textbf{\thepage}}}


%\def\leftkol{Requirements for manuscripts submitted to Journal
%``Informatics~and~Applications''}

%\def\rightkol{Requirements for manuscripts submitted to Journal
%``Informatics~and~Applications''}



\item abstract (not less than 100 words) both in Russian and in English. Abstract is a short
summary of the article that can be published separately. The abstract is the
main source of information on the article and it could be included in leading information
systems and data bases. The abstract in English has to be an original text and should
not be an exact translation of the Russian one. Good English is required.
In abstracts, avoid references and formulae;\\[-13.5pt]
\item indexing is performed on the basis of keywords. The use of keywords from the
internationally accepted thematic Thesauri is recommended.

%\def\leftkol{Requirements for manuscripts submitted to Journal
%``Informatics~and~Applications''}

%\def\rightkol{Requirements for manuscripts submitted to Journal
%``Informatics~and~Applications''}

Important! Keywords must not be sentences;
\item Acknowledgments.
\end{itemize}

\item References. Russian references have to be presented both in English translation and Latin
transliteration (refer {\sf http://www.translit.net/ru/bgn/}).

Please take into account the following examples of Russian references appearance:

\noindent
\textbf{Article in journal:}

\Aue{Zhang, Z., and D.~Zhu}. 2008. Experimental research on the localized electrochemical
micromachining.
\textit{Rus. J.~Electrochem.}  44(8):926--930. {\sf doi:10.1134/S1023193508080077}.


\noindent
\textbf{Journal article in electronic format:}

\Aue{Swaminathan, V., E.~Lepkoswka-White, and B.\,P.~Rao}. 1999. Browsers or buyers in
cyberspace? An
investigation of electronic factors influencing electronic exchange. \textit{JCMC}
5(2). Available at: {\sf http://www.ascusc.org/jcmc/vol5/issue2/} (accessed April~28, 2011).




\noindent
\textbf{Article from the continuing publication (collection of works, proceedings):}

\Aue{Astakhov, M.\,V., and T.\,V.~Tagantsev}. 2006. Eksperimental'noe
issledovanie prochnosti soedineniy ``stal'--kompozit'' [Experimental study of
the strength of joints ``steel--composite'']. \textit{Trudy MGTU
``Matematicheskoe modelirovanie slozhnykh tekh\-ni\-che\-skikh sistem''}
[\textit{Bauman MSTU ``Mathematical Modeling of Complex Technical
Systems'' Proceedings}]. 593:125--130.

\def\leftfootline{\small{\textbf{\thepage}
\hfill INFORMATIKA I EE PRIMENENIYA~--- INFORMATICS AND APPLICATIONS\ \ \ 2019\
\ \ volume~13\ \ \ issue\ 4}
}%
 \def\rightfootline{\small{INFORMATIKA I EE PRIMENENIYA~--- INFORMATICS AND APPLICATIONS\ \ \ 2019\ \ \ volume~13\ \ \ issue\ 4
\hfill \textbf{\thepage}}}

\def\leftkol{Requirements for manuscripts submitted to Journal
``Informatics~and~Applications''}

\def\rightkol{Requirements for manuscripts submitted to Journal
``Informatics~and~Applications''}

\noindent
\textbf{Conference proceedings:}

\Aue{Usmanov, T.\,S., A.\,A.~Gusmanov, I.\,Z.~Mullagalin, R.\,Ju.~Muhametshina,
A.\,N.~Chervyakova, and
A.\,V.~Sveshnikov}. 2007. Osobennosti proektirovaniya razrabotki mestorozhdeniy
s primeneniem gidrorazryva
plasta [Features of the design of field development with the use of hydraulic fracturing].
\textit{Trudy 6-go
Mezhdu\-na\-rod\-no\-go Simpoziuma ``Novye resursosberegayushchie tekhnologii
nedropol'zovaniya i povysheniya
neftegazootdachi''} [\textit{6th  Symposium (International) ``New Energy Saving Subsoil
Technologies and
the Increasing of the Oil and Gas Impact'' Proceedings}]. Moscow. 267--272.


\noindent
\textbf{Books and other monographs:}




Lindorf, L.\,S., and L.\,G.~Mamikoniants, eds. 1972.
\textit{Ekspluatatsiya turbogeneratorov s neposredstvennym
okhlazhdeniem} [\textit{Operation of turbine generators with direct cooling}].
Moscow: Energy Publs. 352~p.


%\Aue{Latyshev, V.\,N.} 2009. \textit{Tribologiya rezaniya. Kn.~1: Frikcionnye prosessy
%pri rezanii metallov}
%[\textit{Tribology of cutting. Vol.~1: Frictional processes in metal cutting}]. Ivanovo: Ivanovskii
%State Univ. 108~p.


%\noindent
%\textbf{Unpublished material:}

%\Aue{Latypov, A.\,R., M.\,M.~Khasanov, and V.\,A.~Baikov}.
%2004. Geology and production (NGT GiD). Certificate on official registration of the computer
%program
%No.\,2004611198. (In Russian, unpubl.)

%\noindent
%\textbf{Internet-source:}

%APA Style. 2011. Available at: {\sf http://www.apastyle.org/apa-style-help.aspx} (accessed
%February~5, 2011).

%Pravila citirovaniya istochnikov [Rules for the citing of sources]. Available at: {\sf
%http://www.scribd.com/doc/1034528/} (accessed February~7, 2011).


\noindent
\textbf{Dissertation and Thesis:}

%\Aue{Semenov, V.\,I.}
%2003. Matematicheskoe modelirovanie plazmy v sisteme kompaktnyy tor. [Mathematical
%modeling of the plasma in the compact torus]. D.Sc.\ Diss. Moscow. 272~p.

\Aue{Kozhunova, O.\,S.} 2009. Tekhnologiya razrabotki semanticheskogo
slovarya informatsionnogo monitoringa [Technology of development of
semantic dictionary of information monitoring system]. PhD Thesis. Moscow: IPI RAN. 23~p.


\noindent
\textbf{State standards and patents:}

GOST 8.586.5-2005. 2007. Metodika vypolneniya izmereniy. Izmerenie raskhoda i~kolichestva
zhidkostey i gazov 
s~pomoshch'yu standartnykh suzhayushchikh ustroystv [Method of measurement.
Measurement of flow rate and volume of liquids and gases by means of orifice devices]. M.:
Standardinform
Publs. 10~p.

%\noindent
%\textbf{Patent:}

\Aue{Bolshakov, M.\,V., A.\,V.~Kulakov, A.\,N.~Lavrenov, and M.\,V.~Palkin}.
2006. Sposob orientirovaniya po krenu letatel'nogo
apparata s opti\-che\-skoy golovkoy
samonavedeniya [The way to orient on the roll of aircraft with optical homing head].
Patent RF No.\,2280590.

References in Latin transcription are presented in the original language.

References in the text are numbered according to the order of their
first appearance; the number is
placed in square brackets. All items from the reference list should be
cited.\\[-13.5pt]

\item Manuscripts and additional materials are not returned to Authors by the Editorial Board.\\[-13.5pt]

\item Submissions of files by e-mail must include:\\[-13.5pt]
\begin{itemize}
\item   the journal title and author's name in the ``Subject'' field; \\[-13.5pt]
\item   an article and additional materials have to be attached using the ``attach'' function;\\[-13.5pt]
\item   an electronic version of the article should contain the file with the text and a separate file
with figures.\\[-13.5pt]
\end{itemize}

\item ``Informatics and Applications'' journal is not a profit publication. There are no
charges for the authors as well as there are no royalties.\\[-13.5pt]
\end{enumerate}

\def\leftfootline{\small{\textbf{\thepage}
\hfill INFORMATIKA I EE PRIMENENIYA~--- INFORMATICS AND APPLICATIONS\ \ \ 2019\
\ \ volume~13\ \ \ issue\ 4}
}%
 \def\rightfootline{\small{INFORMATIKA I EE PRIMENENIYA~--- INFORMATICS AND APPLICATIONS\ \ \ 2019\ \ \ volume~13\ \ \ issue\ 4
\hfill \textbf{\thepage}}}

\def\leftkol{Requirements for manuscripts submitted to Journal
``Informatics~and~Applications''}

\def\rightkol{Requirements for manuscripts submitted to Journal
``Informatics~and~Applications''}


%\vspace*{5mm}


\begin{center}
\textbf{Editorial Board address:} \\

%ABOUT AUTHORS



FRC CSC RAS, 44, block~2, Vavilov Str., Moscow 119333, Russia\\[-10pt]

\

Ph.: +7\,(499)\,135\,86\,92,\ \ Fax: +7\,(495)\,930\,45\,05\\[-10pt]

\

 e-mail: {\sf rust@ipiran.ru} (to Prof.\ Rustem Seyful-Mulyukov)\\[-10pt]

\

 {\sf http://www.ipiran.ru/english/journal.asp}
\end{center}
 }
%\thispagestyle{myheadings}

\def\leftkol{Requirements for manuscripts submitted to Journal
``Informatics~and~Applications''}

\def\rightkol{Requirements for manuscripts submitted to Journal
``Informatics~and~Applications''}

\def\leftfootline{\small{\textbf{\thepage}
\hfill INFORMATIKA I EE PRIMENENIYA~--- INFORMATICS AND APPLICATIONS\ \ \ 2019\
\ \ volume~13\ \ \ issue\ 4}
}%
 \def\rightfootline{\small{INFORMATIKA I EE PRIMENENIYA~--- INFORMATICS AND APPLICATIONS\ \ \ 2019\ \ \ volume~13\ \ \ issue\ 4
\hfill \textbf{\thepage}}}

 \label{end\stat}

\newpage


%\vspace*{-60pt} {\small
{\baselineskip=9.1pt
\section*{Правила подготовки рукописей статей для публикации в журнале
<<Информатика и её применения>>}

\thispagestyle{empty}

 Журнал <<Информатика и её применения>> публикует
теоретические, обзорные и дискуссионные статьи, посвященные научным
исследованиям и разработкам в области информатики и ее приложений. Журнал
издается на русском языке. По специальному решению редколлегии отдельные статьи,
в виде исключения, могут печататься на английском языке.
Тематика журнала охватывает следующие направления:
\begin{itemize}
\item теоретические основы информатики; %\\[-13.5pt]
\item математические методы исследования сложных систем и процессов; %\\[-13.5pt]
\item информационные системы и сети; %\\[-13.5pt]
\item информационные технологии; %\\[-13.5pt]
\item архитектура и программное
обеспечение вычислительных комплексов и сетей.
\end{itemize}
\begin{enumerate}
\item В журнале печатаются результаты, ранее не
опубликованные и не предназначенные к одновременной публикации в других
изданиях. Публикация не должна нарушать закон об авторских правах. Направляя
свою рукопись в редакцию, авторы автоматически передают учредителям и
редколлегии неисключительные права на издание данной статьи на русском языке и
на ее распространение в России и за рубежом. При этом за авторами сохраняются
все права как собственников данной рукописи. В связи с этим авторами должно
быть представлено в редакцию письмо в следующей форме:
Соглашение о передаче права на публикацию:

\textit{<<Мы, нижеподписавшиеся, авторы рукописи <<$\qquad\qquad$>>, передаем
учредителям и редколлегии журнала <<Информатика и её применения>>
неисключительное право опубликовать данную рукопись статьи на русском языке как
в печатной, так и в электронной версиях журнала. Мы подтверждаем, что данная
публикация не нарушает авторского права других лиц или организаций. Подписи
авторов: (ф.\,и.\,о., дата, адрес)>>.}

Указанное соглашение может быть представлено 
как в бумажном виде, так и в виде отсканированной копии (с подписями авторов).


Редколлегия вправе запросить у авторов экспертное заключение о возможности
опубликования представленной статьи в открытой печати. %\\[-13.5pt]
\item Статья
подписывается всеми авторами. На отдельном листе представляются данные автора
(или всех авторов): фамилия, полные имя и отчество, телефон, факс, e-mail,
почтовый адрес. Если работа выполнена несколькими авторами, указывается фамилия
одного из них, ответственного за переписку с редакцией. %\\[-13.5pt]
\item Редакция журнала
осуществляет самостоятельную экспертизу присланных статей. Возвращение рукописи
на доработку не означает, что статья уже принята к печати. Доработанный вариант
с ответом на замечания рецензента необходимо прислать в редакцию. %\\[-13.5pt]
\item Решение
редакционной коллегии о принятии статьи к печати или ее отклонении сообщается
авторам. Редколлегия не обязуется направлять рецензию авторам отклоненной
статьи. %\\[-13.5pt]
\item Корректура статей высылается авторам для просмотра. Редакция
просит авторов присылать свои замечания в кратчайшие сроки. %\\[-13.5pt]
\item При
подготовке рукописи в MS Word рекомендуется использовать следующие настройки.
Параметры страницы: формат~--- А4; ориентация~--- книжная; поля (см): внутри~---
2,5, снаружи~--- 1,5, сверху~--- 2, снизу~--- 2, от края до нижнего
колонтитула~--- 1,3. Основной текст: стиль~--- <<Обычный>>: шрифт Times New
Roman, размер 14~пунктов, абзацный отступ~--- 0,5~см, 1,5 интервала,
выравнивание~--- по ширине. Рекомендуемый объем рукописи~--- не свыше
25~страниц указанного формата. Ознакомиться с шаблонами, содержащими примеры
оформления, можно по адресу в Интернете:
\textsf{http://www.ipiran.ru/journal/template.doc}.
\item К рукописи, предоставляемой в 2-х
экземплярах, обязательно прилагается электронная версия статьи (как правило, в
форматах MS WORD (.doc) или \LaTeX\ (.tex), а также~--- дополнительно~--- в
формате .pdf) на дискете, лазерном диске или по электронной почте. Сокращения
слов, кроме стандартных, не применяются. Все страницы рукописи должны быть
пронумерованы. %\\[-13.5pt]
\item Статья должна содержать следующую информацию на русском и
английском языках: название, Ф.И.О. авторов, места работы авторов и их
электронные адреса, подробные сведения об авторах, оформленные в соответствии с форматом, 
определяемым файлами {\sf http://www.ipiran.ru/journal/issues/2011\_05\_01/authors.asp} и 
{\sf http://www.ipiran.ru/journal/issues/2011\_01\_eng/authors.asp},
аннотация (не более 100~слов), ключевые слова. Ссылки на
литературу в тексте статьи нумеруются (в квадратных скобках) и располагаются в
порядке их первого упоминания. В~списке литературы не должно быть позиций, на которые нет ссылки в тексте статьи.
Все фамилии авторов, заглавия статей, названия
книг, конференций и~т.\,п.\ даются на языке оригинала, если этот язык
использует кириллический или латинский алфавит. %\\[-13.5pt]
\item Присланные в редакцию материалы авторам не возвращаются.
\item При отправке файлов по электронной
почте просим придерживаться следующих правил:
\begin{itemize}
\item указывать в поле subject (тема) название журнала и фамилию автора; %\\[-13.5pt]
\item использовать attach (присоединение); %\\[-13.5pt]
\item в случае больших объемов информации возможно
использование общеизвестных архиваторов (ZIP, RAR); %\\[-13.5pt]
\item в состав электронной версии статьи должны входить: файл, содержащий текст статьи, и файл(ы),
содержащий(е) иллюстрации. %\\[-13.5pt]
\end{itemize}
\item Журнал <<Информатика и её применения>> является некоммерческим изданием. 
Плата за публикацию с авторов не взимается, гонорар авторам не выплачивается.
\end{enumerate}
\thispagestyle{empty}
\textbf{Адрес редакции:} Москва 119333,
ул.~Вавилова, д.~44, корп.~2, ИПИ РАН\\
\hphantom{\textbf{Адрес редакции:} }Тел.: +7 (499) 135-86-92\ \
Факс:  +7 (495) 930-45-05\ \  E-mail:   rust@ipiran.ru }
}

\end{document}


%\tableofcontents

%\end{document}





%\def\stat{cont}
{%\hrule\par
%\vskip 7pt % 7pt
\raggedleft\Large \bf%\baselineskip=3.2ex
А\,В\,Т\,О\,Р\,С\,К\,И\,Й\ \ У\,К\,А\,З\,А\,Т\,Е\,Л\,Ь\ \ З\,А\ \ 2\,0\,0\,7 г. \vskip 17pt
    \hrule
    \par
\vskip 21pt plus 6pt minus 3pt }

\label{st\stat}

\def\tit{\ }

\def\aut{\ }
\def\auf{\ }

\def\leftkol{\ } % ENGLISH ABSTRACTS}

\def\rightkol{\ } %ENGLISH ABSTRACTS}

\titele{\tit}{\aut}{\auf}{\leftkol}{\rightkol}


\contentsline {chapter}{\ }{Выпуск \quad Стр.} 
\contentsline {section}{\textbf{Батракова Д.\,А., Королев В.\,Ю., Шоргин С.\,Я.}\ \ Новый метод вероятностно-ста\-ти\-сти\-че\-ско\-го анализа информационных потоков в\nobreakspace {}телекоммуникационных сетях}{\qquad 1 \qquad 40} 
\contentsline {section}{\textbf{Борисов А.\,В.}\ \ Байесовское оценивание в системах наблюдения с\nobreakspace {}марковскими скачкообразными процессами: игровой подход}{\qquad 2 \qquad 65}
\contentsline {section}{\textbf{Босов А.\,В., Иванов А.\,В.}\ \ Программная инфраструктура информационного Web-пор\-тала}{\qquad 2 \qquad 50}
\contentsline {section}{\textbf{Захаров В.\,Н., Калиниченко Л.\,А., Соколов И.\,А., Ступников С.\,А.}\ \ Конструирование канонических информационных моделей для интегрированных информационных систем}{\qquad 2 \qquad 15}
\contentsline {section}{\textbf{Захаров В.\,Н., Козмидиади В.\,А.}\ \ Средства обеспечения отказоустойчивости при\-ло\-жений}{\qquad 1 \qquad 14} 
\contentsline {section}{\textbf{Иванов А.\,В.}\ \ см. Босов А.\,В.\hfill\hfill\hfill\hfill\hfill\hfill\hfill\hfill\hfill\hfill\hfill\hfill\hfill\hfill\hfill\hfill\hfill\hfill\hfill\hfill\hfill\hfill\hfill\hfill\hfill\hfill\hfill\hfill\hfill\hfill\hfill\hfill\hfill\hfill\hfill}{\ }
\contentsline {section}{\textbf{Ильин В.\,Д., Соколов И.\,А.}\ \ Символьная модель системы знаний информатики в\nobreakspace {}че\-ло\-ве\-ко-автоматной среде}{\qquad 1 \qquad 66} 
\contentsline {section}{\textbf{Калиниченко Л.\,А.}\ \ см. Захаров В.\,Н.\hfill\hfill\hfill\hfill\hfill\hfill\hfill\hfill\hfill\hfill\hfill\hfill\hfill\hfill\hfill\hfill\hfill\hfill\hfill\hfill\hfill\hfill\hfill\hfill\hfill\hfill\hfill\hfill\hfill\hfill\hfill\hfill\hfill\hfill\hfill}{\ }
\contentsline {section}{\textbf{Козеренко Е.\,Б.}\ \ Лингвистическое моделирование для систем машинного перевода и обработки знаний}{\qquad 1 \qquad 54} 
\contentsline {section}{\textbf{Козмидиади В.\,А.}\ \ см. Захаров В.\,Н.\hfill\hfill\hfill\hfill\hfill\hfill\hfill\hfill\hfill\hfill\hfill\hfill\hfill\hfill\hfill\hfill\hfill\hfill\hfill\hfill\hfill\hfill\hfill\hfill\hfill\hfill\hfill\hfill\hfill\hfill\hfill\hfill\hfill\hfill\hfill }{\ } 
\contentsline {section}{\textbf{Королев В.\,Ю.}\ \ см. Батракова Д.\,А.\hfill\hfill\hfill\hfill\hfill\hfill\hfill\hfill\hfill\hfill\hfill\hfill\hfill\hfill\hfill\hfill\hfill\hfill\hfill\hfill\hfill\hfill\hfill\hfill\hfill\hfill\hfill\hfill\hfill\hfill\hfill\hfill\hfill\hfill\hfill}{\ } 
\contentsline {section}{\textbf{Кудрявцев А.\,А., Шоргин С.\,Я.}\ \ Байесовский подход к\nobreakspace {}анализу систем массового обслуживания и\nobreakspace {}показателей надежности}{\qquad 2 \qquad 76}
\contentsline {section}{\textbf{Печинкин А.\,В., Соколов И.\,А., Чаплыгин В.\,В.}\ \ Многолинейная система массового обслуживания с конечным накопителем и ненадежными приборами}{\qquad 1 \qquad 27} 
\contentsline {section}{\textbf{Печинкин А.\,В., Соколов И.\,А., Чаплыгин В.\,В.}\ \ Стационарные характеристики многолинейной\nobreakspace {}системы массового обслуживания с\nobreakspace {}одновременными отказами приборов}{\qquad 2 \qquad 39}
\contentsline {section}{\textbf{Синицын И.\,Н.}\ \ Корреляционные методы построения аналитических информационных моделей флуктуаций полюса Земли по априорным данным}{\qquad 2 \qquad \hphantom{9}2}
\contentsline {section}{\textbf{Синицын И.\,Н.}\ \ Развитие теории фильтров Пугачева для оперативной обработки информации в стохастических системах}{{\qquad 1 \qquad \hphantom{9}3}} 
\contentsline {section}{\textbf{Соколов И.\,А.}\ \ см. Захаров В.\,Н.\hfill\hfill\hfill\hfill\hfill\hfill\hfill\hfill\hfill\hfill\hfill\hfill\hfill\hfill\hfill\hfill\hfill\hfill\hfill\hfill\hfill\hfill\hfill\hfill\hfill\hfill\hfill\hfill\hfill\hfill\hfill\hfill\hfill\hfill\hfill}{\ }
\contentsline {section}{\textbf{Соколов И.\,А.}\ \ см. Ильин В.\,Д.\hfill\hfill\hfill\hfill\hfill\hfill\hfill\hfill\hfill\hfill\hfill\hfill\hfill\hfill\hfill\hfill\hfill\hfill\hfill\hfill\hfill\hfill\hfill\hfill\hfill\hfill\hfill\hfill\hfill\hfill\hfill\hfill\hfill\hfill\hfill}{\ } 
\contentsline {section}{\textbf{Соколов И.\,А.}\ \ см. Печинкин А.\,В.\hfill\hfill\hfill\hfill\hfill\hfill\hfill\hfill\hfill\hfill\hfill\hfill\hfill\hfill\hfill\hfill\hfill\hfill\hfill\hfill\hfill\hfill\hfill\hfill\hfill\hfill\hfill\hfill\hfill\hfill\hfill\hfill\hfill\hfill\hfill}{\ } 
\contentsline {section}{\textbf{Соколов И.\,А.}\ \ см. Печинкин А.\,В.\hfill\hfill\hfill\hfill\hfill\hfill\hfill\hfill\hfill\hfill\hfill\hfill\hfill\hfill\hfill\hfill\hfill\hfill\hfill\hfill\hfill\hfill\hfill\hfill\hfill\hfill\hfill\hfill\hfill\hfill\hfill\hfill\hfill\hfill\hfill}{\ }
\contentsline {section}{\textbf{Ступников С.\,А.}\ \ см. Захаров В.\,Н.\hfill\hfill\hfill\hfill\hfill\hfill\hfill\hfill\hfill\hfill\hfill\hfill\hfill\hfill\hfill\hfill\hfill\hfill\hfill\hfill\hfill\hfill\hfill\hfill\hfill\hfill\hfill\hfill\hfill\hfill\hfill\hfill\hfill\hfill\hfill}{\ }
\contentsline {section}{\textbf{Чаплыгин В.\,В.}\ \ см. Печинкин А.\,В.\hfill\hfill\hfill\hfill\hfill\hfill\hfill\hfill\hfill\hfill\hfill\hfill\hfill\hfill\hfill\hfill\hfill\hfill\hfill\hfill\hfill\hfill\hfill\hfill\hfill\hfill\hfill\hfill\hfill\hfill\hfill\hfill\hfill\hfill\hfill}{\ } 
\contentsline {section}{\textbf{Чаплыгин В.\,В.}\ \ см. Печинкин А.\,В.\hfill\hfill\hfill\hfill\hfill\hfill\hfill\hfill\hfill\hfill\hfill\hfill\hfill\hfill\hfill\hfill\hfill\hfill\hfill\hfill\hfill\hfill\hfill\hfill\hfill\hfill\hfill\hfill\hfill\hfill\hfill\hfill\hfill\hfill\hfill}{\ }
\contentsline {section}{\textbf{Шоргин С.\,Я.}\ \ см. Батракова Д.\,А.\hfill\hfill\hfill\hfill\hfill\hfill\hfill\hfill\hfill\hfill\hfill\hfill\hfill\hfill\hfill\hfill\hfill\hfill\hfill\hfill\hfill\hfill\hfill\hfill\hfill\hfill\hfill\hfill\hfill\hfill\hfill\hfill\hfill\hfill\hfill}{\ } 
\contentsline {section}{\textbf{Шоргин С.\,Я.}\ \ см. Кудрявцев А.\,А.\hfill\hfill\hfill\hfill\hfill\hfill\hfill\hfill\hfill\hfill\hfill\hfill\hfill\hfill\hfill\hfill\hfill\hfill\hfill\hfill\hfill\hfill\hfill\hfill\hfill\hfill\hfill\hfill\hfill\hfill\hfill\hfill\hfill\hfill\hfill}{\ }
%\thispagestyle{myheadings}
\def\leftfootline{\small{\textbf{\thepage}
\hfill ИНФОРМАТИКА И ЕЁ ПРИМЕНЕНИЯ\ \ \ том~1\ \ \ выпуск~2\ \ \ 2007}
}%
 \def\rightfootline{\small{ИНФОРМАТИКА И ЕЁ ПРИМЕНЕНИЯ\ \ \ том~1\ \ \ выпуск~2\ \ \ 2007
 \hfill \textbf{\thepage}}}
 \label{end\stat}

%\def\stat{cont-e}
{%\hrule\par
%\vskip 7pt % 7pt
\raggedleft\Large \bf%\baselineskip=3.2ex
2\,0\,0\,7\ \ A\,U\,T\,H\,O\,R\ \ I\,N\,D\,E\,X \vskip 17pt
    \hrule
    \par
\vskip 21pt plus 6pt minus 3pt }

\label{st\stat}

\def\tit{\ }

\def\aut{\ }
\def\auf{\ }

\def\leftkol{\ } % ENGLISH ABSTRACTS}

\def\rightkol{\ } %ENGLISH ABSTRACTS}

\titele{\tit}{\aut}{\auf}{\leftkol}{\rightkol}


\contentsline {chapter}{\ }{Issue \quad Page} 
\contentsline {subsection}{\textbf{Batrakova D.\,A., Korolev V.\,Yu., Shorgin S.\,Ya.}\ \ A New Method for the Probabilistic and Statistical Analysis of Information Flows in Telecommunication Networks}{\qquad 1 \qquad 40} 
\contentsline {subsection}{\textbf{Borisov A.\,V.}\ \ Bayesian Estimation in\nobreakspace {}Observation Systems with\nobreakspace {}Markov Jump Processes: Game-Theoretic Approach}{\qquad 2 \qquad 65} 
\contentsline {subsection}{\textbf{Bosov A.\,V., Ivanov A.\,V.}\ \ Linguistic Simulation for Machine Translation and Knowledge Management Systems}{\qquad 2 \qquad 50} 
\contentsline {subsection}{\textbf{Chaplygin V.\,V.} see Pechinkin A.\,V.\hfill\hfill\hfill\hfill\hfill\hfill\hfill\hfill\hfill\hfill\hfill\hfill\hfill\hfill\hfill\hfill\hfill\hfill\hfill\hfill\hfill\hfill\hfill\hfill\hfill\hfill\hfill\hfill\hfill\hfill\hfill\hfill\hfill\hfill\hfill}{\ }
\contentsline {subsection}{\textbf{Chaplygin V.\,V.} see Pechinkin A.\,V.\hfill\hfill\hfill\hfill\hfill\hfill\hfill\hfill\hfill\hfill\hfill\hfill\hfill\hfill\hfill\hfill\hfill\hfill\hfill\hfill\hfill\hfill\hfill\hfill\hfill\hfill\hfill\hfill\hfill\hfill\hfill\hfill\hfill\hfill\hfill}{\ }
\contentsline {subsection}{\textbf{Ilyin V.\,D., Sokolov I.\,A.}\ \ The Symbol Model of Informatics Knowledge System in Human-Automaton Environment}{\qquad 1 \qquad 66} 
\contentsline {subsection}{\textbf{Ivanov A.\,V.} see Bosov A.\,V.\hfill\hfill\hfill\hfill\hfill\hfill\hfill\hfill\hfill\hfill\hfill\hfill\hfill\hfill\hfill\hfill\hfill\hfill\hfill\hfill\hfill\hfill\hfill\hfill\hfill\hfill\hfill\hfill\hfill\hfill\hfill\hfill\hfill\hfill\hfill}{\ }
\contentsline {subsection}{\textbf{Kalinichenko L.\,A.} see Zakharov V.\,N.\hfill\hfill\hfill\hfill\hfill\hfill\hfill\hfill\hfill\hfill\hfill\hfill\hfill\hfill\hfill\hfill\hfill\hfill\hfill\hfill\hfill\hfill\hfill\hfill\hfill\hfill\hfill\hfill\hfill\hfill\hfill\hfill\hfill\hfill\hfill}{\ }
\contentsline {subsection}{\textbf{Korolev V.\,Yu.} see Batrakova D.\,A.\hfill\hfill\hfill\hfill\hfill\hfill\hfill\hfill\hfill\hfill\hfill\hfill\hfill\hfill\hfill\hfill\hfill\hfill\hfill\hfill\hfill\hfill\hfill\hfill\hfill\hfill\hfill\hfill\hfill\hfill\hfill\hfill\hfill\hfill\hfill}{\ }
\contentsline {subsection}{\textbf{Kozerenko E.\,B.}\ \ Linguistic Simulation for Machine Translation and Knowledge Management Systems}{\qquad 1 \qquad 54} 
\contentsline {subsection}{\textbf{Kozmidiady V.\,A.} see Zakharov V.\,N.\hfill\hfill\hfill\hfill\hfill\hfill\hfill\hfill\hfill\hfill\hfill\hfill\hfill\hfill\hfill\hfill\hfill\hfill\hfill\hfill\hfill\hfill\hfill\hfill\hfill\hfill\hfill\hfill\hfill\hfill\hfill\hfill\hfill\hfill\hfill}{\ }
\contentsline {subsection}{\textbf{Kudryavtsev A.\,A., Shorgin S.\,Ya.}\ \ Bayesian Approach to Queueing Systems and Reliability Characteristics}{\qquad 2 \qquad 76} 
\contentsline {subsection}{\textbf{Pechinkin A.\,V., Sokolov I.\,A., Chaplygin V.\,V.}\ \ Multichannel Queuing System with Finite Buffer and Unreliable Servers}{\qquad 1 \qquad 27} 
\contentsline {subsection}{\textbf{Pechinkin A.\,V., Sokolov I.\,A., Chaplygin V.\,V.}\ \ Stationary Characteristics of a Multichannel Queueing System with\nobreakspace {}Simultaneous Refusals of Servers}{\qquad 2 \qquad 39} 
\contentsline {subsection}{\textbf{Shorgin S.\,Ya.} see Batrakova D.\,A.\hfill\hfill\hfill\hfill\hfill\hfill\hfill\hfill\hfill\hfill\hfill\hfill\hfill\hfill\hfill\hfill\hfill\hfill\hfill\hfill\hfill\hfill\hfill\hfill\hfill\hfill\hfill\hfill\hfill\hfill\hfill\hfill\hfill\hfill\hfill}{\ }
\contentsline {subsection}{\textbf{Shorgin S.\,Ya.} see Kudryavtsev A.\,A.\hfill\hfill\hfill\hfill\hfill\hfill\hfill\hfill\hfill\hfill\hfill\hfill\hfill\hfill\hfill\hfill\hfill\hfill\hfill\hfill\hfill\hfill\hfill\hfill\hfill\hfill\hfill\hfill\hfill\hfill\hfill\hfill\hfill\hfill\hfill}{\ }
\contentsline {subsection}{\textbf{Sinitsyn I.\,N.}\ \ Correlational Methods for Analytical Informational Models of the Earth Pole Fluctuations Design Based on a priori Data}{\qquad 2 \qquad \hphantom{9}2}
\contentsline {subsection}{\textbf{Sinitsyn I.\,N.}\ \ Development of Pugachev Filtering for Stochastic Systems}{\qquad 1 \qquad \hphantom{9}3}
\contentsline {subsection}{\textbf{Sokolov I.\,A.} see Ilyin V.\,D.\hfill\hfill\hfill\hfill\hfill\hfill\hfill\hfill\hfill\hfill\hfill\hfill\hfill\hfill\hfill\hfill\hfill\hfill\hfill\hfill\hfill\hfill\hfill\hfill\hfill\hfill\hfill\hfill\hfill\hfill\hfill\hfill\hfill\hfill\hfill}{\ }
\contentsline {subsection}{\textbf{Sokolov I.\,A.} see Pechinkin A.\,V.\hfill\hfill\hfill\hfill\hfill\hfill\hfill\hfill\hfill\hfill\hfill\hfill\hfill\hfill\hfill\hfill\hfill\hfill\hfill\hfill\hfill\hfill\hfill\hfill\hfill\hfill\hfill\hfill\hfill\hfill\hfill\hfill\hfill\hfill\hfill}{\ }
\contentsline {subsection}{\textbf{Sokolov I.\,A.} see Pechinkin A.\,V.\hfill\hfill\hfill\hfill\hfill\hfill\hfill\hfill\hfill\hfill\hfill\hfill\hfill\hfill\hfill\hfill\hfill\hfill\hfill\hfill\hfill\hfill\hfill\hfill\hfill\hfill\hfill\hfill\hfill\hfill\hfill\hfill\hfill\hfill\hfill}{\ }
\contentsline {subsection}{\textbf{Sokolov I.\,A.} see Zakharov V.\,N.\hfill\hfill\hfill\hfill\hfill\hfill\hfill\hfill\hfill\hfill\hfill\hfill\hfill\hfill\hfill\hfill\hfill\hfill\hfill\hfill\hfill\hfill\hfill\hfill\hfill\hfill\hfill\hfill\hfill\hfill\hfill\hfill\hfill\hfill\hfill}{\ }
\contentsline {subsection}{\textbf{Stupnikov S.\,A.} see Zakharov V.\,N.\hfill\hfill\hfill\hfill\hfill\hfill\hfill\hfill\hfill\hfill\hfill\hfill\hfill\hfill\hfill\hfill\hfill\hfill\hfill\hfill\hfill\hfill\hfill\hfill\hfill\hfill\hfill\hfill\hfill\hfill\hfill\hfill\hfill\hfill\hfill}{\ }
\contentsline {subsection}{\textbf{Zakharov V.\,N., Kalinichenko L.\,A., Sokolov I.\,A., Stupnikov S.\,A.}\ \ Development of Canonical Information Models for Integrated Information Systems}{\qquad 2 \qquad 15} 
\contentsline {subsection}{\textbf{Zakharov V.\,N., Kozmidiady V.\,A.}\ \ Means Providing Applications Fault Tolerance}{\qquad 1 \qquad 14} 
\def\leftfootline{\small{\textbf{\thepage}
\hfill ИНФОРМАТИКА И ЕЁ ПРИМЕНЕНИЯ\ \ \ том~1\ \ \ выпуск~2\ \ \ 2007}
}%
 \def\rightfootline{\small{ИНФОРМАТИКА И ЕЁ ПРИМЕНЕНИЯ\ \ \ том~1\ \ \ выпуск~2\ \ \ 2007
 \hfill \textbf{\thepage}}}
 \label{end\stat}


%\tableofcontents


\end{document}

\newcommand{\Ack}{\subsection*{\protect\large\bf Acknowledgments}}