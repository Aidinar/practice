\newcommand {\ebdr}{\triangleq}
\newcommand {\pc}{{\mathbf P}}
\newcommand{\me}[2]{\mathbf{E}_{ #1 }\left\{ \mathop{#2} \right\} }


\def\stat{bosov}

\def\tit{МОДЕЛИРОВАНИЕ И МОНИТОРИНГ СОСТОЯНИЯ VoIP-СОЕДИНЕНИЯ$^*$}

\def\titkol{Моделирование и~мониторинг состояния VoIP-соединения}

\def\aut{А.\,В.~Борисов$^1$, А.\,В.~Босов$^2$, Г.\,Б.~Миллер$^3$}

\def\autkol{А.\,В.~Борисов, А.\,В.~Босов, Г.\,Б.~Миллер}

\titel{\tit}{\aut}{\autkol}{\titkol}

\index{Борисов А.\,В.}
\index{Босов А.\,В.}
\index{Миллер Г.\,Б.}
\index{Borisov A.\,V.}
\index{Bosov A.\,V.}
\index{Miller G.\,B.}

{\renewcommand{\thefootnote}{\fnsymbol{footnote}} \footnotetext[1]
{Исследование частично поддержано РФФИ (проекты 16-07-00677 
и~15-37-20611-мол\_а\_вед).}}


\renewcommand{\thefootnote}{\arabic{footnote}}
\footnotetext[1]{Институт проблем информатики Федерального исследовательского
центра <<Информатика и~управление>> Российской академии наук, aborisov@frccsc.ru}
\footnotetext[2]{Институт проблем информатики Федерального исследовательского
центра <<Информатика и~управление>> Российской академии наук, abosov@frccsc.ru}
\footnotetext[3]{Институт проблем информатики Федерального исследовательского
центра <<Информатика и~управление>> Российской академии наук, gmiller@frccsc.ru}


\Abst{В результате анализа транспортного протокола Real Time
Transfer Protocol (RTP), исполь\-зу\-емо\-го технологиями Voice over IP
(VoIP) для передачи аудио- и~видеоинформации, предложена
математическая модель описания состояния VoIP-со\-еди\-не\-ния. Она
учитывает как формальные правила, лежащие в~основе технологии VoIP,
так и~особенности сетевого взаимодействия, характерные для RTP:
задержки, потери пакетов и~пр. Математическое описание
исследуемого процесса сетевого взаимодействия основано на аппарате
скрытых марковских моделей. Предполагается, что ненаблюдаемое
состояние соединения описывается марковским процессом с~конечным
множеством состояний, а~наблюдению доступен поток принимаемых
кадров, являющийся немарковским мультивариантным точечным процессом.
%
Для предложенной системы наблюдения решена задача фильтрации
состояния VoIP-со\-еди\-не\-ния по име\-ющим\-ся наблюдениям.
%
Достоверность предложенной модели и~работоспособность алгоритма
фильт\-ра\-ции иллюстрируются примером обработки реального видеопотока,
формируемого сервисом Linphone VoIP и~передаваемого по сотовой сети
стандарта~3G.}

\KW{технологии VoIP; прокотол RTP; сетевое соединение;
скрытая марковская модель; мультивариантный точечный процесс;
оптимальная фильтрация}

\DOI{10.14357/19922264160201} 

\vspace*{-4pt}

\vskip 10pt plus 9pt minus 6pt

\thispagestyle{headings}

\begin{multicols}{2}

\label{st\stat}


\section{Введение} %\label{sec:intro}

Разработка и~верификация математических моделей функционирования
телекоммуникационных систем является одним из интенсивно
разви\-ва\-ющих\-ся направлений научных исследований в~последние полвека.
Формальное описание случайных процессов, возникающих при работе
телекоммуникационных и~информационно-вы\-чис\-ли\-тель\-ных устройств,
обеспечивающих сетевой обмен по каналам связи под управлением
различных сетевых протоколов, в~настоящее время достаточно
востребовано. Такие модели, с~одной стороны, являются фундаментом
для дальнейших исследований, более детального анализа и,~в~конечном
итоге, реализации оптимизированных алгоритмов передачи и~обработки
информации в~составе аппаратного\linebreak или программного обеспечения
телекоммуникационных средств. С другой стороны, разработка 
и~совершенствова\-ние математических моделей стимулируются появлением
новых средств связи и~транспортных протоколов.

Качество моделей традиционно оценивается их способностью отражать
природу описываемого канала связи, циркулирующих в~нем потоков
данных, в~частности нестационарный характер процессов
при\-е\-ма/пе\-ре\-да\-чи информационных пакетов, их задержки, перемешивание,
потери и~т. п.

По-видимому, пионерскими в~данной об\-ласти следует считать ставшие
классическими модели~\cite{G_60, E_63}. Изначально они были
созданы для описания битовых потерь в~телефонных сетях, но
впоследствии достаточно успешно применялись для описания потерь
пакетов в~компьютерных сетях, управляемых протоколами стека TCP/IP.
Будучи предельно простыми, они, тем не менее, хорошо описывали ряд
принципиальных особенностей, свойственных сетям TCP/IP.

Для учета тонкой специфики сетевого взаимодействия предпринимались
попытки детально описывать структуру и~протоколы реальных систем 
и~сетей пакетной передачи данных. Аппаратом, применяемым в~таких
исследованиях, как правило, выступали цепи Маркова с~конечным
множеством состояний~[3--7]. %\cite{AAB_00, masm_05, BCG_08, HH_XX, MASH_XX}.

С~ростом числа возможных состояний, а также при исследовании
процессов в~условиях предельных сетевых нагрузок обоснованное
применение нашли жидкостные и~диффузионные 
аппроксимации~[8--11]. %\cite{W_02, B_03, DDCK_14, LG_03}. 
Попытки частично или полностью отказаться от
марковского свойства моделей привели к~использованию самоподобных
процессов~[12--14] %\cite{LTWW_94, CB_97, TG_00} 
и~скрытых марковских моделей (СММ)~[15--18].
%\cite{YM_02, A_08, EPKP_12, BM_04}.

Накопленный к~настоящему времени опыт моделирования подкрепляет
очевидный тезис о~том, что успех в~разработке математической модели
дости\-га\-ет\-ся при компромиссе между полнотой\linebreak учета технических
деталей, разумными упрощениями, возможностями применяемого
математического аппарата и~устойчивостью к~неточностям параметров
модели.

В настоящее время существуют несколько стандартов, описывающих
наборы протоколов, на базе которых разрабатываются и~функционируют
сервисы IP-телефонии. К наиболее распространенным можно отнести
протокол установления сеанса\linebreak SIP (Session Initiation Protocol),
посредством которого осуществляется обмен служебной информацией,
связанной с~регистрацией и~установкой сеансов связи между клиентами.
Согласование медиаформатов и~других параметров сессии  после
уста\-нов\-ления соединения между клиентами происходит с~использованием
протокола описания сессии SDP (Session Description Protocol). 

Другой
популярный VoIP-стан\-дарт~H.232 описывает отличный от SIP набор
служебных протоколов, тем не менее наиболее часто используемым
протоколом для непосредственной передачи аудио- и~видеопотоков 
в~IP-те\-ле\-фо\-нии обоих стандартов является RTP (Real-Time Transport
Protocol). В~силу того что наибольший объем данных передается по
этому протоколу, именно его естественно использовать для оперативной
оценки состояния канала связи между клиентами на прикладном уровне
сетевой модели.

Таким образом, объектом исследования в~данной статье является RTP~---
протокол прикладного уровня, используемый для передачи аудио- 
и~видеоданных между VoIP-кли\-ен\-та\-ми. Для данного протокола предлагается
новая математическая модель, учитывающая ряд его специфических черт,
характерных именно для технологий VoIP.

Во-первых, RTP является протоколом, не гаран\-ти\-ру\-ющим доставку: в~нем
отсутствует механизм квитирования, т.\,е.\ поток передаваемых пакетов
предполагает случайные потери. Кроме того, RTP допускает временное
взрывное нарастание принимаемых пакетов (так называемое bursting reception) 
и~нарушение порядка их приема. При этом если нарушение порядка пакетов
выявляется непосредственно путем простого сравнения номеров пакетов,
то определение факта их потери возможно только через определенный
временной лаг, необходимый для того, чтобы убедиться в~том, что
пакет потерян вовсе, а не задерживается.

Во-вторых, на принимающей стороне существует возможность регистрации
моментов приема пакетов. Эта временн$\acute{\mbox{а}}$я последовательность является
важным источником информации для определения характеристик задержек
пакетов и~последующей\linebreak оценки состояния VoIP-со\-еди\-не\-ния. При этом
\mbox{ясно,} что это состояние недоступно прямым наблюдениям, поскольку
определяется нагрузкой на программное обеспечение и~средства
телекоммуникаций, формирующие и~обслуживающие канал связи,
допустимыми размерами буферов в~каналообразующей аппаратуре и~т.\,п.
Характеристики задержек пакетов позволяют судить о~состоянии
соединения лишь косвенно. Наконец, анализ реальных данных
показывает, что время ожидания очередного RTP-пакета не является
экспоненциально распределенной случайной величиной вне зависимости
от состояния соединения, т.\,е.\ поток пакетов принимаемых RTP-па\-ке\-тов
не является марковским.

Перечисленные особенности предлагается вклю\-чить в~формальное
описание с~помощью ап\-па\-ра\-та СММ. Предполагается, что состояние\linebreak
VoIP-со\-еди\-не\-ния описывается ненаблюдаемым (<<скрытым>>) марковским
процессом с~конечным множеством состояний, а~наблюдению доступна
последовательность пар <<момент получения па\-ке\-та\,--\,за\-го\-ло\-вок
RTP-па\-ке\-та>>. Эти заголовки позволяют правильно сформировать кадр,
требующийся клиенту сервиса VoIP. Однако непосредственное
использование такой информационной модели для решения последующих
задач оценивания и~оптимизации состояния соединения оказывается
неэффективным в~силу того, что эти <<сырые>> данные избыточны для
решения вышеупомянутых задач. Тем не менее к~ним удается применить
некоторую процедуру предобработки с~фиксированным малым
запаздыванием для извлечения значимой информации о~состоянии
соединения. Эта процедура опирается на наличие в~видеопотоке
неделимого элемента~--- отдельного кадра. Данное обстоятельство
делает естественной агрегацию данных о~моментах получения отдельных
пакетов в~последовательность моментов получения кадров.
Дополнительно каждый кадр сопровождается информацией о~его статусе,
характеризующей качество кадра. Эта информация также является
результатом агрегации информации из заголовков пакетов кадра. Таким
образом, в~качестве наблюдений в~предлагаемой модели используется
результирующая последовательность пар <<момент получения кад\-ра\,--\,ста\-тус кадра>>. Для математического описания этой последовательности
предлагается использовать мультивариантные точечные процессы (МТП)~\cite{LS_89}.

Таким образом, предлагаемая модель сетевого взаимодействия относится
к классу стохастических динамических систем наблюдения, для решения
задач оценивания и~оптимизации в~которых может быть применен хорошо
развитый аппарат стохастического анализа.

Статья организована следующим образом. Раздел~2
содержит неформальное описание процесса\linebreak получения потока RTP-па\-ке\-тов
и~аргументацию в~пользу упомянутой процедуры предобработки. 
В~результате предлагается  математическая модель соединения  в~форме
стохастической дифференциальной системы наблюдений. 

В~разд.~3 практическая задача он\-лайн-мо\-ни\-то\-рин\-га
состояния VoIP-со\-еди\-не\-ния сформулирована в~терминах оптимальной
фильтрации состояния системы наблюдения. Ее решение может оказаться
полезным, например, в~качестве вспомогательной процедуры в~задаче
идентификации параметров модели (например, в~EM-ал\-го\-рит\-ме
(expectation--maximization algorithm)~\cite{EAM_94}) 
или для целей управ\-ле\-ния по неполной информации.
Общее решение\linebreak задачи фильтрации, включая подробное доказательство,
приведено в~\cite{B_14}, обсуждение модели и~применение алгоритма
фильтрации в~сетях массового обслуживания~--- в~\cite{B_15}.

Раздел~4 иллюстрирует представленные результаты.
Качество предложенной модели и~алгоритма оценивания
продемонстрировано на примере обработки реальных статистических
данных, полученных в~результате работы сервиса Linphone VoIP по сети
сотовой связи 3G. Выводы и~заключительные замечания приведены в~разд.~5.

\section{Неформальное описание и~математическая модель} %\label{sec:model}

Процесс получения кадров, передаваемых с~помощью потока RTP-па\-ке\-тов,
предлагается описывать следующей стохастической дифференциальной
системой наблюдения. Ее первый компонент\linebreak определяет ненаблюдаемое
состояние VoIP-со\-едине\-ния. Его значение зависит от множества
случай\-ных факторов: сформировавшейся топологии обеспечива\-ющей
физической сети, состояния и~характеристик телекоммуникационного
оборудования, состояния встроенного программного обеспечения 
и~свойств реализуемых им алгоритмов передачи данных. Наиболее важным
представляется фактор разделения сетевых ресурсов.
Телекоммуникационное оборудование обслуживает одновременно большое
число пользователей, поэтому общая текущая нагрузка на канал
существенно влияет на состояние конкретного соединения и~служит
дополнительным источником неопределенности.

Очевидно, что все упомянутые параметры и~факторы, определяющие
состояние соединения, недоступны прямым наблюдениям, поэтому его
можно считать ненаблюдаемым (<<скрытым>>) процессом. В~то же время
значительное число независимых пользователей, постоянно запрашивая
сетевые ресурсы, обеспечивают отсутствие последействия для этого
процесса, так что предположение о наличии у состояния соединения
марковского свойства представляется вполне реалистичным. 
В~рассматриваемой модели будем предполагать, что состояние соединения
описывается однородным марковским процессом~$X_t$ с~начальным
распределением~$p_0$, матрицей интенсивностей~$\Lambda$ и~тремя
возможными состояниями $\{e_1, e_2, e_3\} \ebdr \mathbb{S}^3$.
Обозначение~$e_j$ здесь и~далее применяется для $j$-го единичного
вектора в~евклидовом пространстве~$\mathbb{R}^3$. Значение~$e_1$
используется для обозначения состояния соединения {\it<<свободно>>},
$e_2$~--- для состояния {\it<<умеренная нагрузка>>} и~$e_3$~--- для
состояния {\it<<перегрузка и/или сбой соединения>>}. Использование
единичных векторов для формального обозначения значений марковского
процесса обеспечивает возможность представить~$X_t$ в~виде решения
стохастического уравнения:

\noindent
\begin{equation}
X_t = \xi + \int\limits_0^t \Lambda^{\top}X_s\,ds+M^X_t, 
\label{eq:state}
\end{equation}
где $M^X_t$ --- мартингал; $\xi$~--- случайный вектор 
с~распределением~$p_0$~\cite{EAM_94}. Параметры $(p_0,\Lambda)$ далее
предполагаются известными или доступными идентификации по имеющимся
статистическим данным (как это сделано, например, в~разд.~4).

На принимающей стороне наблюдению доступны пары вида <<момент
получения па\-ке\-та\,--\,за\-го\-ло\-вок RTP-па\-ке\-та>>. Моменты получения
пакетов\linebreak могут фиксироваться клиентским про\-граммным обеспечением VoIP
или сетевыми мониторами (см., например,~\cite{MMA, Wireshark}).
Формат заголовков пакетов (см., например,~\cite{RFC_3550}) накладывает ряд
важных ограничений на процесс передачи данных. Во-пер\-вых, факты
потерь RTP-па\-ке\-тов на приемнике не могут быть зарегистрированы
непосредственно. Обнаружение потери пакета обеспечивается наличием 
в~его заголовке информации, позволяющей определить, к~какому кадру
относятся полученные в~пакете данные и~какой по порядку пакет из
кадра получен. Во-вто\-рых, поток RTP-па\-ке\-тов не является ординарным:
большинство VoIP-сер\-ве\-ров присваивают общую временн$\acute{\mbox{у}}$ю метку
(timestamp) всем пакетам, образующим кадр, и~отправляют их
одновременно. На приемнике это свойство в~целом также сохраняется:
обычно пакеты, отправленные одновременно, регистрируются на стороне
клиента также одновременно. В-третьих, отправитель использует
некоторые низкоуровневые аппаратные протоколы для получения
дополнительной информации о~качестве физического соединения или
наличии сбоев для последующей адаптации процедуры отправки. При
отсутствии связи или ее неудовлетворительном качестве RTP-па\-ке\-ты
аккумулируются в~серверном буфере, с~тем чтобы быть отправленными,
как только качество соединения улучшится. В~таких ситуациях целые
серии пакетов могут появляться на принимающей стороне одновременно 
и~при этом может нарушаться их порядок.

Сказанное позволяет сделать вывод о~том, что характер сетевого
обмена, задающий в~конечном итоге качество воспроизведения,
определяется равномерностью доставки кадров и~их качеством. Таким
образом, изначально данные заголовков принимаемых пакетов
оказываются избыточными для последующего оценивания состояния
соедине-\linebreak ния, и~требуется их предобработка для извлечения информации 
о~событиях, связанных с~кадрами. Результатом этой процедуры является
МТП, т.\,е.\ последовательность пар $\{(\tau(n),Y(n))\}_{n \in
\mathbb{N}}$, в~которой~$\tau(n)$ является случайным временем $n$-го
наблюдаемого события, $Y(n)$~--- типом события. <<Алфавит>>
(множество возможных событий) задается множеством $\{f_1, f_2, f_3\}
\ebdr \mathbb{S}^3$. Обозначение~$f_k$ здесь и~далее используется для
$k$-го единичного вектора евклидова пространства~$\mathbb{R}^3$.
Значение~$f_1$ присвоено событию {\it <<получен кадр большого
размера>>}, $f_2$~--- событию {\it <<получен кадр малого размера>>},
$f_3$~---  событию {\it <<потери в~составе  кадра>>}.

События первых двух типов регистрируются естественным образом 
в~реальном масштабе времени при поступлении последнего пакета кадра.
Наличие потерь в~конкретном кадре фиксируется в~том случае, если
хотя бы один пакет, со\-став\-ля\-ющий кадр, еще не получен, но уже
получен па\-кет-мар\-кер следующего кадра (в~RTP для каждого кадра
заголовок его последнего пакета маркируется битовым признаком). Тем
не менее время регистрации события третьего типа определяется как
момент получения последнего пакета, составляющего кадр с~потерянными
пакетами.

Наблюдения в~форме МТП $\{(\tau(n),Y(n))\}_{n \in \mathbb{N}}$
эквивалентны следующему случайному процессу с~непрерывным временем:
\begin{equation*}
Y_t \ebdr \sum_{n \in \mathbb{N}}Y(n)\mathbf{I}(t-\tau(n))\,.
%\label{eq:obs_1}
\end{equation*}
Здесь $\mathbf{I}(\cdot)$~--- функция Хевисайда, непрерывная справа.
Компоненты трехмерного процесса $Y_t$~--- считающие: каждый
компонент определяет общее число событий соответствующего типа,
произошедших на интервале $[0,t]$.

Предположим, что известна условная плотность вероятности
$\pi_{jk\ell}(\cdot)$:
\begin{multline}
\label{eq:condition_pdfs}
\pc \left\{\tau(n) \leqslant t\,,\ Y(n) = f_j \vert 
\tau(n-1)=s\,,\right.\\
\left.  Y(n-1)=f_k\,,\ 
 X_u \equiv e_{\ell}\,,\  u \in [s,t]\right\}= {}\\
{}=\mathbf{I}(t-s)\int\limits_s^t \pi_{jk\ell}(u-s)\,du\,,
\end{multline}
в противном случае ее предполагается предварительно идентифицировать
(см.\ разд.~4).

Можно показать, что процесс~$Y_t$ допускает следующее представление:
\begin{equation}
Y_t = \int\limits_0^t \phi(\omega,u)X_{u-}\,du+M^Y_t\,, 
\label{eq:obs_2}
\end{equation}
где $M^Y_t$~--- мартингал, а $\phi(\omega,u):\Omega \times
\mathbb{R}_+ \hm\to \mathbb{R}^{3 \times 3}$~--- предсказуемая
матричнозначная функция мгновенных интенсивностей. Элементы
$\phi_{j\ell}(\cdot)$ обладают ясной вероятностной интерпретацией:
для любого $n \in \mathbb{N}$ 
\begin{multline*}
  \phi_{j\ell}(\omega,t)\,dt =
  \pc\left\{
  \vphantom{\left\{\left(\tau(i),Y(i)\right)\right\}_{i=1}^{n-1}} 
  \tau(n) \in [t,t+dt),Y(n)= f_j
  \bigl|\right.\\
  {}\left.
  \left\{\left(\tau(i),Y(i)\right)\right\}_{i=1}^{n-1}\vee 
  \{X_{t-}=e_{\ell}\} \vee \{\tau(n)\geqslant s\}\right\}.
 \end{multline*}
Матрица интенсивностей может быть вычислена с~помощью условной
плотности вероятности $\pi_{jk\ell}(\cdot)$ (здесь и~далее
предполагается, что $\tau_0 \ebdr 0$):
 \begin{equation*}
 \phi(\omega,t) = \sum\limits_{n \in \mathbb{N}}
 \mathbf{I}_{(\tau(n-1),\tau(n)]}(t)\varphi(\tau(n-1),Y(n-1),t),
%  \label{eq:int_1}
 \end{equation*}
где $\varphi(\cdot)=\|\varphi_{j\ell}(\cdot)\|$~--- матрично-значная
функция соответствующей размерности с~элементами
 \begin{multline*}
 \varphi_{j\ell}(\tau(n-1),Y(n-1),t) \ebdr {}\\ 
 {}\ebdr
 \sum\limits_{k=1}^3
  f_k^{\top}Y(n-1)\pi_{jk\ell}(t-\tau(n-1))\times{}\\
  {}\times \left(
 \sum\limits_{i=1}^3
  \int\limits_{t}^{+\infty}\pi_{ik\ell}(u-\tau(n-1))\,du\right)^+.
%  \label{eq:int_2}
 \end{multline*}

Таким образом, уравнения~(\ref{eq:state}) и~(\ref{eq:obs_2})
образуют искомую математическую модель функционирования
VoIP-со\-еди\-не\-ния. Подтверждению ее адекватности для описания
функционирования сервиса Linphone VoIP посвящен
разд.~4. В следующем разделе приведены постановка 
и~решение задачи он\-лайн-оце\-ни\-ва\-ния (фильтрации) состояния
рассматриваемой системы наблюдения.

\section{Задача онлайн-оценивания состояния соединения} %\label{sec:filter}

Предложенная математическая модель позволяет решить одну из
важнейших практических задач мониторинга состояния VoIP-со\-еди\-не\-ния
по доступным наблюдениям. В~терминах теории стохастических
динамических систем эта задача эквивалентна построению условного
распределения состоя\-ния относительно всех доступных на данный момент
наблюдений. Помимо собственной практической и~теоретической
значимости представление условного распределения играет важную
вспомогательную роль при решении различных задач управления по
неполной информации.

Для рассматриваемой модели задача фильт\-рации состоит в~нахождении
условного ма\-те\-ма\-ти\-ческого ожидания (УМО) состояния VoIP-со\-еди\-нения
по наблюдениям, доступным к~текущему\linebreak моменту времени~$t$:
$\widehat{X}_t \ebdr \me{}{X_t|Y_{[0,t]}}$. Решение этой задачи
определяет следующая теорема.

\smallskip

\noindent
\textbf{Теорема~1.}
\textit{Условное математическое ожидание~$\widehat{X}_t$ определяется как единственное решение следующей
стохастической системы уравнений, имеющей единственное сильное
решение}:
\begin{multline}
\widehat{X}_t = p_0 + \int_0^t \Lambda^{\top}\widehat{X}_{s-}\,ds +{}\\
{}+
\int\limits_0^t\left[ \mathrm{diag}\, \widehat{X}_{s-} - \widehat{X}_{s-}
\widehat{X}^{\top}_{s-}
 \right]\phi^{\top}(\omega,s-) \times{} \\ 
 {}\times
 \left[\mathrm{diag}\left(
 \phi(\omega,s-) \widehat{X}_{s-}
 \right) \right]^{-1}\times{}\\
 {}\times\left(
 dY_s-\phi(\omega,s-)\widehat{X}_{s-}\,ds
 \right).
\label{eq:filt_1}
\end{multline}


Задача фильтрации семимартингала по наблюдению семимартингала 
в~наиболее общей постановке была решена в~\cite{LS_89}, формулы
частного\linebreak случая фильтрации МСП по наблюдениям МТП с~линейным
компенсатором представлены в~\cite{B_14}, а~их обсуждение
применительно к~сетям массового обслуживания имеется в~\cite{B_15}.

\section{Численные эксперименты} %\label{sec:example}

Для проверки адекватности представленной модели сетевого
взаимодействия и~иллюстрации качества  оценок фильтрации был
проведен численный эксперимент по обработке реальных данных,
полученных на клиенте видеоконференции, организованной с~по\-мощью
VoIP-сер\-ви\-са Linphone. Циркулирующие в~рамках этой конференции
пакеты регистрировались на выбранном узле-получателе. Накопленные
таким образом статистические данные использовались для одновременной
идентификации параметров модели наблюдения~(\ref{eq:state}),
(\ref{eq:obs_2}) и~он\-лайн-вы\-чис\-ле\-ния оценок фильтрации.

Поскольку состояние сетевого соединения вычисляет\-ся по
статистическим данным после неко\-торой предварительной их обработки,
идентификация модели (определение матрицы ин\-тен\-сив\-ности и~условных
плотностей $\pi_{jkl}(\cdot)$) также выполняется с~некоторым
запаздыванием по отношению к~поступающему потоку данных. Оценки
фильт\-ра\-ции, хотя и~вычисляются в~онлайн режиме по всему накопленному
объему наблюдений, строятся на основе модели, параметры которой
получены с~использованием лишь части наблюдений. Степень
запаздывания, которая определяется разницей между количеством
наблюдений (пакетов), используемых для идентификации модели 
и~процесса фильтрации, соответствует размеру окна сглаживания. Оценки
фильтрации сравниваются с~<<идеальной моделью>>, состояние которой
вычисляется как результат сглаживания. Полученные качественные
характеристики оценок фильтрации служат подтверждением как
адекватности модели~(\ref{eq:state}), (\ref{eq:obs_2}), так 
и~высокой точности самого алгоритма фильтрации~(\ref{eq:filt_1}).


Для проведения расчетов был зарегистрирован отрезок
последовательности пакетов, переданных в~рамках сессии обмена
потоковым видео. Видеопоследовательность передавалась 
с~использованием канала мобильной связи 3G с~мобильного устройства на
персональный компьютер, подключенный к~сети Интернет. На этом
компьютере перехватывался весь входящий трафик, из него извлекались
пакеты, касающиеся конкретной видеосессии. В~качестве средства
проведения видеоконференции использовалось приложение Linphone VoIP.
Этот широко распространенный кроссплатформенный программный клиент
IP-те\-ле\-фо\-нии имеет открытый исходный код и~поддерживает версии как
для мобильных, так и~для настольных платформ. Наиболее важным
обстоятельством для проводимых исследований является возможность
отключения в~данном приложении шифрования RTP-тра\-фи\-ка, %\linebreak 
что
обеспечивает прозрачность процессов его обработки. В~част\-ности,
благодаря отсутствию шиф\-ро\-ва\-ния удается легко разделять отдельные
\mbox{RTP-сес}\-сии и~выделять аудио- и~видеопоследовательности. В~качестве
сетевого монитора использовалось приложение Wireshark, выбор
которого определялся простотой применения и~совместимостью с~операционной системой.
\begin{figure*} %fig1
\vspace*{1pt}
 \begin{center}
 \mbox{%
 \epsfxsize=144.076mm
 \epsfbox{mil-1.eps}
 }
 \end{center}
 \vspace*{-9pt}
\Caption{Данные сессии RTP:
тонкая тем\-но-се\-рая линия~--- размер кадра $F_t$;
тонкие серая 
и~черная линии~--- скорость получения пакетов~$S_t$ и~ее медиана $\mathrm{med}\left(S_t\right)$;
точки~---
событие <<потери в~составе  кадра>>;
сплошные свет\-ло-се\-рая, 
серая и~тем\-но-се\-рая линии~--- состояние соединения~$e_1$, $e_2$ или~$e_3$;
пунктирные черные линии~--- пороговые значения~0,04 и~0,08 
 \label{pic:1_state_model}} 
\end{figure*}



На рис.~\ref{pic:1_state_model} приведены данные, полученные 
в~те\-чение примерно двадцатиминутной видеоконференции. В~качестве
индикатора состояния VoIP-со\-еди\-не\-ния использовалась скорость
получения\linebreak пакетов~$S_t$, т.\,е.\ среднее время между приемом пакетов,
относящихся к~одному кадру. Отметим, что, несмотря на осреднение,
этот процесс демонстрирует быстрые колебания, вызванные
высоко\-час\-тот\-ной природой цифровых коммуникационных каналов связи.
Для извлечения систематической со\-став\-ля\-ющей, определяющей состояние
сетевого соединения, траектория этого процесса подверглась
сглаживанию путем вычисления скользящей двусторонней медианы
$\mathrm{med}\left(S_t\right)$. Состояния соединения (<<идеальная модель>>)
определялись сле\-ду\-ющим образом:
\begin{itemize}
\item $e_1$~--- состояние {\it<<свободно>>}, если $\mathrm{med}\left(S_t\right)\hm\in [0;\, 0{,}04)$;
\item $e_2$~--- состояние {\it<<умеренная нагрузка>>}, если $\mathrm{med}\left(S_t\right)\hm\in [0{,}04;\, 0{,}08)$;
\item $e_3$~--- состояние {\it<<перегрузка и/или сбой соединения>>}, если $\mathrm{med}\left(S_t\right)\hm\in [0{,}08;\, +\infty)$.
\end{itemize}

Формируемые в~результате предобработки наблюдения зависят от размера
кадра~$F_t$, который равен числу пакетов, составляющих кадр.
Значение наблюдения в~момент~$t$ определяется в~соответствии 
с~предложенной моделью, а~именно:
\begin{itemize}
\item $f_1$~--- событие {\it<<получен кадр большого размера>>}, если $F_t \hm\geq 6$;\\[-8pt]
\item $f_2$~--- событие {\it<<получен кадр малого размера>>}, если $F_t \hm< 6$;\\[-8pt]
\item $f_3$~--- событие {\it<<потери в~составе  кадра>>}.
\end{itemize}

%\columnbreak

При обработке данных было отмечено соответствие, заключающееся 
в~том, что наблюдения~$f_i$ появляются чаще, когда соединение
находится в~состоянии~$e_i$. Дело в~том, что применяемая версия
программного обеспечения поддержки видеоконференции использует на
стороне отправителя низкоуровневый протокол, позволяющий определять
временные проблемы с~мобильной связью. За счет этого программа при
отправке предпринимает определенные действия, призванные
предотвратить возможные потери: адаптирует число пакетов в~кадре,
ухудшая качество, накапливает пакеты в~буфере, с~тем чтобы отправить
их незамедлительно при восстановлении канала связи, и~пр.


Далее приведена оценка матрицы интенсивностей, построенная по
полному объему собранных данных:
\begin{gather*}
\Lambda = \begin{pmatrix}
-2{,}171\ &\ \hphantom{-}1{,}964\ &\ \hphantom{-}0{,}207\\
\hphantom{-}2{,}442\ &\ -3{,}161\ &\ \hphantom{-}0{,}719\\
\hphantom{-}1{,}665\ &\ \hphantom{-}0{,}666\ &\ -2{,}331\\
              \end{pmatrix} \cdot 10^{-3}.
\end{gather*}
Гистограммы времен пребывания в~каждом состоянии показаны на
рис.~2. На этом же рисунке для
сравнения построены соответствующие экспоненциальные аппроксимации.
Визуально экспоненциальная плотность вполне согласуется с~реальными
данными для первого и~второго состояния. Значительная погрешность
для третьего состояния может быть объяснена малым объемом выборки.
Небольшое несоответствие первой гистограммы и~ее экспоненциальной
аппроксимации не является неожиданностью: небольшие отклонения на
<<хвос\-тах>> часто встречаются в~задачах оценки плотностей.

%\end{multicols}

 \begin{figure*} %fig2
\vspace*{1pt}
\begin{center}
\mbox{%
 \epsfxsize=147.9mm
 \epsfbox{mil-2.eps}
 }
\end{center} 
\vspace*{-12.1pt}
\Caption{Гистограммы для времени пребывания в~состоянии~$e_i$~(\textit{1}) 
и~соответствующие им экспоненциальные плотности вероятности~(\textit{2}): 
(\textit{а})~$i\hm=1$; (\textit{б})~2; (\textit{в})~$i\hm=3$}
%\end{figure*}
% \begin{figure} %fig3
\vspace*{1pt}
 \begin{center}
 \mbox{%
 \epsfxsize=145.918mm
 \epsfbox{mil-3.eps}
 }
 \end{center}
 \vspace*{-12.1pt}
\Caption{Оценка условной плотности вероятности 
$\pi_{jk\ell}(\cdot)$~(\ref{eq:condition_pdfs}):
(\textit{а})~$Y_n\hm=f_1$, $Y_{n-1}=f_1$; 
(\textit{б})~$Y_n\hm=f_1$, $Y_{n-1}=f_2$; 
(\textit{в})~$Y_n\hm=f_2$, $Y_{n-1} \hm= f_1$; 
(\textit{г})~$Y_n\hm=f_2$, $Y_{n-1}=f_2$ для разных значений
$X_{n-1}=e_i$;
$X_{n-1} = e_1$~--- цвет гистограммы тем\-но-серый, 
оценка условной плотности~--- пунктирная линия;
%\item 
$X_{n-1} = e_2$~--- цвет гистограммы серый, оценка условной плотности~--- 
тонкая линия;
%\item 
$X_{n-1} = e_3$~--- цвет гистограммы свет\-ло-серый, оценка условной плотности~--- 
толстая линия} \label{pic:3_hists}
\vspace*{-0.15036pt}
\end{figure*}

%\end{multicols}

\pagebreak


%\begin{multicols}{2}


Плотности распределения вероятностей аппроксимировались
гам\-ма-рас\-пре\-де\-ле\-ни\-ями с~параметрами, вычисленными по методу
максимального\linebreak правдоподобия. На рис.~\ref{pic:3_hists} приведены
некоторые гис\-то\-грам\-мы и~соответствующие им оценки плотностей
гамма-распределения, построенные по полному объему наблюдений. Все
графики сгруппированы по следующим условиям: $Y_n\hm=f_j$ при
$Y_{n-1}\hm=f_k$ и~$X_{n-1}\hm=e_{\ell}$ (т.\,е.\ $j$ и~$k$ фиксированы 
и~$\ell \hm=1,2,3$). Приведенные графики позволяют подтвердить хорошую
согласованность оценок гам\-ма-плот\-но\-стей и~соответствующих гистограмм
и~проиллюстрировать их различие при разных состояниях
$X_{n-1}\hm=e_{\ell}$.


Компоненты оценок фильтрации, являющи\-еся условными вероятностями
событий $X_t\hm=e_{\ell}$, представлены на
рис.~\ref{pic:4_estimates_model}. Для сравнения там же приведены
соответствующие компоненты <<идеальной модели>>. Визуальное
сравнение подтверждает высокое качество оценок фильтрации, а~расчеты\linebreak
в~целом~--- возможность их выполнения в~режиме реального времени,
что необходимо для целей он\-лайн-мо\-ни\-то\-рин\-га состояния
VoIP-со\-еди\-нения.

\vspace*{12pt}

\columnbreak 

Качество оценивания, обеспечиваемое пред\-став\-лен\-ным алгоритмом
фильтрации, сравнивалось с~известным аналогом. Если условные
плот\-ности~$\pi_{jk\ell}(\cdot)$ положить экспоненциальными, то\linebreak
наблюдаемый процесс~$Y_t$ будет представлять собой хорошо известный
<<маркированный>> процесс Кокса, а~пара $(X_t,Y_t)$ будет обладать
марковским свойством. Решение задачи фильтрации со\-сто\-яний час\-тич\-но
наблюдаемых марковских процессов такого вида представлено 
в~\cite{EAM_94}. 

Класс систем наблюдения, рас\-смат\-ри\-ва\-емых 
в~предлагаемой  \mbox{статье}, является более широким и~содержит процессы, не
обла\-да\-ющие марковским свойством. 
Алгоритм оптимальной фильт\-ра\-ции 
из~\cite{EAM_94} является частным случаем формул~(\ref{eq:filt_1}). 

Для
сравнения по имеющимся статистическим данным были вычислены оценки
фильт\-ра\-ции со\-сто\-яния в~предположении, что наблю\-да\-емый процесс $Y_t$
является <<маркированным>> процессом Кокса. Необходимые параметры
экспоненциальных распределений для реализации алгоритма фильт\-ра\-ции
вычислялись также с~по\-мощью метода максимального правдоподобия
по нарастающей
 выборке\linebreak %\vspace*{-12pt}

 \end{multicols}
 
 \begin{figure}[h] %fig4
\vspace*{-4pt}
 \begin{center}
 \mbox{%
 \epsfxsize=155.87mm
 \epsfbox{mil-4.eps}
 }
 \end{center}
 \vspace*{-15pt}
\Caption{Оценки фильтрации $\pc \{X_t=e_j|Y_{[0,t]}\}$ для $j\hm=1$~(\textit{а}), 
2~(\textit{б}) и~3~(\textit{в}) (тонкие
черные линии) в~сравнении с~<<идеальной моделью>> (различные оттенки \label{pic:4_estimates_model}
серого) } 
\vspace*{-12pt}
\end{figure}


 
 \pagebreak
 
%\end{multicols}

\begin{figure*} %fig5
\vspace*{1pt}
 \begin{center}
 \mbox{%
 \epsfxsize=156.928mm
 \epsfbox{mil-5.eps}
 }
 \end{center}
 \vspace*{-8pt}
\Caption{Ошибки оценивания процесса $X_t$ для $j\hm=1$~(\textit{а}),
2~(\textit{б}) и~3~(\textit{в}):
ошибка предложенного алгоритма фильтрации (черная линия);
ошибка алгоритма фильтрации~\cite{EAM_94} (серая линия)
} 
\vspace*{3pt}
\label{pic:5_errs}
\end{figure*}

\begin{multicols}{2}

\noindent
 поступающих наблюдений.
   Результаты  оценивания
сравнивались с~упомянутой выше <<идеальной моделью>>. На
рис.~\ref{pic:5_errs} пред\-став\-ле\-ны ошибки оценок сравниваемых
алгоритмов фильтрации. Из рисунка видно, что на значительных
промежутках времени алгоритм из~\cite{EAM_94} неверно идентифицирует
истинное состояние канала: в~районе 5-й, 11-й и~16-й минут.


По оценкам фильтрации для обоих алгоритмов были определены наиболее
вероятные состояния, которые трактовались как точечные оценки
текущего состояния соединения. Эти оценки
 далее сравнивались с~<<идеальной моделью>>. Оценка вероятности ошибки 
 точечной оценки, построенной по предлагаемому фильтру, составила~0,069, 
 в~то время как соответствующее значение для фильтра  из~\cite{EAM_94} равно~0,206.


Статистический анализ проведенного численного эксперимента 
с~привлечением реальных телекоммуникационных данных демонстрирует
хорошее согласование предлагаемой математической модели
функционирования RTP-со\-еди\-не\-ния с~полученной измерительной
информацией, а~также высокую точность алгоритма он\-лайн-оце\-ни\-ва\-ния
состояния этого соединения.

%\vspace*{-3pt}

\section{Заключение} %\label{sec:ourto}

%\vspace*{-2pt}

Основные результаты, представленные в~работе, видятся в~следующем.
\begin{enumerate}[1.]
\item
Предложено новое феноменологическое описание процесса
сетевого взаимодействия, реализуемого по технологии VoIP на базе
протокола RTP. Предложена и~подробно рассмотрена модель,
базирующаяся на концепции СММ. Ненаблюдаемое (<<скрытое>>) состояние
соединения представлено конечномерным марковским скачкообразным
процессом, в~то время\linebreak как наблюдаемое время получения кадров~---
немарковский МТП. Предложенная модель \mbox{об\-ладает} очевидными
достоинствами, одно из
 которых~--- возможность применения к~ним
\mbox{аппарата} стохастического анализа. 
\item В~составе модели
предложена процедура пред\-об\-работки, преобразующая <<сырой>> поток
пакетов в~<<очищенный>> поток кадров, оказавшийся весьма удобным для
он\-лайн-про\-це\-ду\-ры совместного решения задач идентификации параметров
модели и~оценивания состояния VoIP-со\-еди\-не\-ния. 
\item Предложенные для описания как состояния, так и~наблюдений
<<алфавиты>> демонстрируют хорошее соответствие модели реальным
данным: соответствующие гистограммы оказываются унимодальными 
и~хорошо аппроксимируются гам\-ма-рас\-пре\-де\-ле\-нием.
\end{enumerate}

Тем не менее, по  мнению авторов, исследование нельзя считать
законченным. Предложенная модель нуждается во всесторонней
верификации. Далее модель должна быть развита c~целью учета
доступной информации, касающейся сетевой топологии, характеристик
алгоритмов, реали\-зу\-емых программным и~техническим обеспечением,
формирующим канал связи и~реализующим протоколы информационного
обмена, например данными, передаваемыми в~рамках служебного
протокола RTCP, который также используется на прикладном уровне
VoIP. Представляет самостоятельный интерес возможность реализации
робастной версии алгоритма фильтрации состояния. Наконец, финальной
точкой могла бы стать постановка и~решение задачи оптимального и/или
робастного управления в~предложенной модели по неполной информации.

\vspace*{-6pt}

{\small\frenchspacing
 {%\baselineskip=10.8pt
 \addcontentsline{toc}{section}{References}
 \begin{thebibliography}{99}
 
 \vspace*{-2pt}

\bibitem{G_60}
\Au{Gilbert E.\,N.} Capacity of a~burst-noise channel~// {Bell
Syst. Tech. J., 1960.} Vol.~39. P.~1253--1265.

\bibitem{E_63}
\Au{Elliott E.\,O.} Estimates of error rates for codes on
burst-noise channels~// {Bell Syst. Tech. J., 1963.}
Vol.~42. P.~1977--1997.

\bibitem{AAB_00} %3
\Au{Altman E., Avrachenkov~K., Barakat~C.} TCP in presence of
bursty losses~// {Perform. Evaluation, 2000.} Vol.~42.
P.~129--147.

\bibitem{masm_05} %4
\Au{Миллер Б.\,М., Авраченков~К.\,Е., Степанян~К.\,В., Миллер~Г.\,Б.}
Задача оптимального стохастического управ\-ле\-ния потоком данных по
неполной информации~//
{Пробл. передачи информ., 2005.} Т.~41. №\,2. С.~89-110. 
doi: 10.1007/s11122-005-0020-8.

\bibitem{BCG_08} %5
\Au{Bruno R., Conti M., Gregori~E.} Throughput analysis and
measurements in IEEE 802.11 WLANs with TCP and UDP traffic flows~//
{IEEE Trans. Mobile Comput., 2008.} Vol.~7. No.\,2. P.~171--186.

\bibitem{HH_XX} %6
\Au{\mbox{Ha{\!\!\ptb{\ss}}\,linger}~G., Hohlfeld~O.} The Gilbert--Elliott model for
packet loss in real time services on the Internet~// {14th GI/ITG
Conference on Measurement, Modelling and Evaluation of Computer and
Communication Systems (MMB) Proceedings.} ---~Dortmund, Germany,
2008. P.~269--283.

\bibitem{MASH_XX} %7
\Au{Malik M., Aydin M., Shah~Z., Hussain~S.} Stochastic model of
TCP and UDP traffic in IEEE 802.11b/g.~// {IEEE 9th Conference on
Industrial Electronics and Applications (ICIEA) Proceedings.}--- 
IEEE, 2014. P.~2170--2175.







\bibitem{W_02}
\Au{Whitt W.} {Stochastic-process limits. An introduction to
stochastic-process limits and their application to queues.}~---~New York, NY, USA: 
Springer, 2002. 602~p.

\bibitem{B_03} %9
\Au{Bohacek S.} A stochastic model of TCP and fair video
transmission~// {22nd Annual Joint Conference of the IEEE Computer
and Communications (INFOCOM) Proceedings.}~--- IEEE, 2003. Vol.~2.
P.~1134--1144.


\bibitem{LG_03} %10
\Au{Liu Y., Gong W.} On fluid queueing systems with strict
priority~// {IEEE Trans. Automat. Contr.}, 2003. Vol.~48. No.\,12.
P.~2079--2088.

\bibitem{DDCK_14} %11
\Au{Doma$\acute{\mbox{n}}$ska J., Doma$\acute{\mbox{n}}$ski~A., 
Czach$\acute{\mbox{o}}$rski~T., Klamka~J.}
Fluid flow approximation of time-limited TCP/UDP/XCP streams~//
{Bull. Pol. Acad. Sci}. Tech. Sci.,
2014. Vol.~62. No.~2. P.~217--225.

\bibitem{LTWW_94} %12
\Au{Leland W.\,E., Taqqu M.\,S., Willinger~W., Wilson~D.\,V.} On the
self-similar nature of Ethernet traffic~// {IEEE ACM Trans.
Network., 1994.} Vol.~2. No.\,1. P.~1--15.

\bibitem{CB_97}
\Au{Crovella M.\,E., Bestavros~A.} Self-similarity in World Wide Web
traffic: Evidence and possible causes~// {IEEE ACM Trans.
Network., 1997.} Vol.~5. No.\,6. P.~835--846.

\bibitem{TG_00}
\Au{Tsybakov B., Georganas~N.} Overflow and losses in a network
queue with a self-similar input~// {Queueing Syst., 2000.}
Vol.~35. No.\,1--4. P.~201--235.

\bibitem{YM_02} %15
\Au{Yariv E., Merhav N.} Hidden Markov processes~// {IEEE Trans.
Inform. Theory, 2002.} Vol.~48. No.\,6. P.~1518--1569.

\bibitem{BM_04} %16
\Au{Борисов А.\,В., Миллер~Г.\,Б.} Анализ и~фильтрация специальных
марковских процессов в~дискретном времени. II.~Оптимальная
фильтрация~// {Автоматика и~телемеханика, 2005.} №\,7. С.~112--125.
doi: 10.1007/s10513-005-0153-7.

\bibitem{A_08} %17
\Au{Anisimov V.} {Switching processes in queueing models.}~--- New York, NY, USA: 
Wiley, 2008. 352~p.

\bibitem{EPKP_12} %18
\Au{Ellis M., Pezaros D.\,P., Kypraios~T., Perkins~C.} Modelling
packet loss in RTP-based streaming video for residential users~//
{37th Conference on Local Computer Networks Proceedings.} --- New York, NY, USA:
IEEE Press, 2012. P.~220--223.



\bibitem{LS_89} %19
\Au{Липцер Р.\,Ш., Ширяев А.\,Н.} Теория мартингалов.~--- М.: Наука,
1986. 512~с.

\bibitem{EAM_94}
\Au{Elliott R.\,J., Aggoun~L., Moore~J.\,B.} {Hidden Markov models:
Estimation and control.}~--- New York, NY, USA: Springer, 2008. 382~p.

\bibitem{B_14}
\Au{Борисов А.\,В.} Применение алгоритмов оптимальной фильтрации для
решения задачи мониторинга доступности удаленного сервера~//
{Информатика и~её применения, 2014.} Т.~8. Вып.~3. С.~53--69.
doi: 10.14375/19922264140307.

\bibitem{B_15}
\Au{Borisov A.} Partially observable multivariate point processes
with linear random compensators: Analysis and filtering with
applications to queueing networks~// {1st IFAC Conference on
Modelling, Identification and Control of Nonlinear Systems (MICNON)
Proceedings.}~--- Saint Petersburg, 2015. P.~1119--1124.

\bibitem{MMA} 
Microsoft Message Analyzer. 
{\sf www.microsoft.com/en-us/ download/details.aspx?id=44226}.

\bibitem{Wireshark} 
Wireshark. {\sf www.wireshark.org/\#learnWS}.

\bibitem{RFC_3550}
\Au{Schulzrinne H., Casner~S., Frederick~R., Jacobson~V.} RTP: 
A~transport protocol for real-time applications~// {RFC} 3550, 2003.
{\sf tools.ietf.org/html/rfc3550}.
\end{thebibliography}

 }
 }

\end{multicols}

\vspace*{-3pt}

\hfill{\small\textit{Поступила в~редакцию 24.02.16}}

\vspace*{8pt}

%\newpage

%\vspace*{-24pt}

\hrule

\vspace*{2pt}

\hrule

%\vspace*{8pt}



\def\tit{MODELING AND MONITORING OF~VoIP CONNECTION}

\def\titkol{Modeling and Monitoring of VoIP connection}

\def\aut{A.\,V.~Borisov, A.\,V.~Bosov, and~G.\,B.~Miller}

\def\autkol{A.\,V.~Borisov, A.\,V.~Bosov, and~G.\,B.~Miller}

\titel{\tit}{\aut}{\autkol}{\titkol}

\vspace*{-9pt}

\noindent
Institute of Informatics Problems, Federal Research Center 
``Computer Science and Control'' of the Russian Academy of Sciences,
44-2~Vavilov Str., Moscow 119333, Russian Federation

\def\leftfootline{\small{\textbf{\thepage}
\hfill INFORMATIKA I EE PRIMENENIYA~--- INFORMATICS AND
APPLICATIONS\ \ \ 2016\ \ \ volume~10\ \ \ issue\ 2}
}%
 \def\rightfootline{\small{INFORMATIKA I EE PRIMENENIYA~---
INFORMATICS AND APPLICATIONS\ \ \ 2016\ \ \ volume~10\ \ \ issue\ 2
\hfill \textbf{\thepage}}}

\vspace*{3pt}


\Abste{The Real Time Transfer Protocol (RTP), widely used in Voice over IP 
(VoIP) technologies for audio and video data transmission, is analyzed  
to design a~mathematical model of VoIP connection. The model attempts to meet basic 
features of VoIP technologies as well as key features of the real link functioning 
like the frame delays, losses, etc.
The proposed approach is based on the finite-state unobservable hidden Markov model. 
The unobservable state is assumed to be a finite-dimensional Markov process, whereas 
the observation is assumed to be a non-Markovian multivariate point process that 
indicates the heterogeneous frames reception.
%
For the proposed model, the hidden link state optimal filtering problem given 
the packet/losses stream observations is formulated and its solution is provided.
%
Proposed link model validity and filtering algorithm performance are 
illustrated by processing of captured real video streams delivered via 
3G mobile network by Linphone VoIP services.}


\KWE{VoIP technologies; RTP; network link; hidden Markov model; 
multivariate point process; optimal state filtering}

\DOI{10.14357/19922264160201}

\vspace*{-12pt}

\Ack
\noindent
The research is partly supported by the Russian Foundation for Basic Research 
(grants Nos.\,16-07-00677 and 15-37-20611).


%\vspace*{3pt}

  \begin{multicols}{2}

\renewcommand{\bibname}{\protect\rmfamily References}
%\renewcommand{\bibname}{\large\protect\rm References}

{\small\frenchspacing
 {%\baselineskip=10.8pt
 \addcontentsline{toc}{section}{References}
 \begin{thebibliography}{99}

\bibitem{G_60-1}
\Aue{Gilbert, E.\,N.} 1960. Capacity of a~burst-noise channel. 
\textit{Bell Syst. Tech.~J.} 39:1253--1265.

\bibitem{E_63-1}
\Aue{Elliott, E.\,O.} 1963. Estimates of error rates for codes on burst-noise channels. 
\textit{Bell Syst. Tech.~J.} 42:1977--1997.

\bibitem{AAB_00-1}
\Aue{Altman, E., K. Avrachenkov, and C.~Barakat}. 2000. 
TCP in presence of bursty losses. \textit{Perform. Evaluation} 42:129--147.

\bibitem{masm_05-1} 
\Aue{Miller, B.\,M., K.\,E.~Avrachenkov, K.\,V.~Stepanyan, and G.\,B.~Miller}. 2005. 
Flow control as stochastic optimal control
problem with incomplete information. \textit{Probl. Inf. Transm.} 
41(2):150--170. {doi:10.1007/s11122-005-0020-8.}

\bibitem{BCG_08-1} 
\Aue{Bruno, R., M. Conti, and E.~Gregori}. 2008. 
Throughput analysis and measurements
in IEEE 802.11 WLANs with TCP and UDP Traffic Flows. 
\textit{IEEE Trans. Mobile Comput.} 7(2):171--186.

\bibitem{HH_XX-1} 
\Aue{\mbox{Ha{\!\ptb{{\ss}}}linger}, G., and O.~Hohlfeld}. 2008. 
The Gilbert--Elliott model for packet loss in
real time services on the internet. 
\textit{14th GI/ITG Conference on Measurement, Modelling and Evaluation of 
Computer and Communication Systems (MMB) Proceedings.} Dortmund, Germany. 269--283.

\bibitem{MASH_XX-1} Malik, M., M. Aydin, Z. Shah, and S. Hussain. 2014. 
Stochastic model of TCP and UDP traffic in IEEE 802.11b/g.
\textit{IEEE 9th Conference on Industrial Electronics and Applications (ICIEA) 
Proceedings.} 2170--2175.

\bibitem{W_02-1} 
\Aue{Whitt, W.} 2002. \textit{Stochastic-process limits. An 
introduction to stochastic-process limits and their application to queues.} 
New York, NY: Springer. 602~p.

\bibitem{B_03-1}  %9
\Aue{Bohacek, S.} 2003. A~stochastic model of TCP and fair video transmission. 
\textit{22nd Annual Joint Conference of the IEEE Computer and Communications 
(INFOCOM) Proceedings.} 2:1134--1144.



\bibitem{LG_03-1-1}  %10
\Aue{Liu, Y., and W.~Gong}. 2003.
On fluid queueing systems with strict priority. 
\textit{IEEE Trans. Automat. Contr.} 48(12):2079--2088.

\bibitem{DDCK_14-1} %11
\Aue{Doma$\acute{\mbox{n}}$ska, J., A. Doma$\acute{\mbox{n}}$ski, 
T.~Czach$\acute{\mbox{o}}$rski, and J.~Klamka}. 2014.
Fluid flow approximation of time-limited TCP/UDP/XCP streams. 
\textit{Bull. Pol. Acad. Sci. Tech. Sci.} 62(2):217--225.

\bibitem{LTWW_94-1} %12
\Aue{Leland, W.\,E., M.\,S. Taqqu, W.~Willinger, and D.\,V.~Wilson}. 1994. 
On the self-similar nature of Ethernet traffic. 
\textit{IEEE ACM Trans. Network.} 2(1):1--15.

\bibitem{CB_97-1} 
\Aue{Crovella, M.\,E., and A.~Bestavros}. 1997. 
Self-similarity in World Wide Web traffic: Evidence and possible causes. 
\textit{IEEE ACM Trans. Network.} 5(6):835--846.

\pagebreak

\bibitem{TG_00-1}
\Aue{Tsybakov, B., and N.~Georganas}. 2000. 
Overflow and losses in a~network queue with a self-similar input. 
\textit{Queueing Syst.} 35(1-4):201--235.

\bibitem{YM_02-1} %15
\Aue{Yariv, E., and N.~Merhav}. 2002. Hidden Markov processes. 
\textit{IEEE Trans. Inform. Theory} 48(6):1518--1569.

\bibitem{BM_04-1} %16
\Aue{Borisov, A.\,V., and G.\,B.~Miller}. 2005. 
Analysis and filtration of special discrete-time markov processes. 
II.~Optimal filtration. \textit{Automat. Rem. Contr.} 66(7):1125--1136.
{doi:10.1007/s10513-005-0153-7.}

\bibitem{A_08-1} %17
\Aue{Anisimov, V.} 2008. 
\textit{Switching processes in queueing models.} New York, NY: Wiley. 352~p.

\bibitem{EPKP_12-1-1} %18
\Aue{Ellis, M., D.\,P. Pezaros, T.~Kypraios, and C.~Perkins}. 2012. 
Modelling packet loss in RTP-based streaming video for residential users. 
\textit{37th Conference on Local Computer Networks Proceedings.} 
New York, NY: IEEE Press. 220--223.



\bibitem{LS_89-1}
\Aue{Liptser, R.\,Sh., and A.\,N.~Shiryayev}. 1989. \textit{Theory of martingales.}
New York, NY: Springer-Verlag. 812 p.

\bibitem{EAM_94-1} 
\Aue{Elliott, R.\,J., L. Aggoun, and J.\,B.~Moore}. 2008. \textit{Hidden Markov
models: Estimation and control.} New York, NY: Springer. 382~p.

\bibitem{B_14-1}
\Aue{Borisov, A.\,V.} 2014. Primenenie algoritmov optimal'noy fil'tratsii dlya 
resheniya zadachi monitoringa do\-stup\-nosti udalennogo servera [Monitoring remote 
server accessibility: The optimal filtering approach]. 
\textit{Informatika i~ee Primeneniya}~--- \textit{Inform. Appl.} 8(3):53--69. 
doi: 10.14375/19922264140307.

\bibitem{B_15-1}
\Aue{Borisov, A.} 2015. Partially observable multivariate point processes with 
linear random compensators: Analysis and filtering with applications to queueing 
networks. \textit{1st IFAC Conference on Modelling, Identification and Control 
of Nonlinear Systems (MICNON 2015)}. St.\ Petersburg. 1119--1124.

\bibitem{MMA-1} 
Microsoft Message Analyzer. Available at: 
{\sf www. microsoft.com/en-us/download/details.aspx?id=44226} (accessed March~30, 2016).

\bibitem{Wireshark-1} 
Wireshark. Available at: {\sf www.wireshark.org/\#learnWS} (accessed March~30, 2016).

\bibitem{RFC_3550-1} 
\Aue{Schulzrinne, H., S.~Casner., R.~Frederick, and V.~Jacobson}. 2003. 
RTP: A~transport protocol for real-time applications. 
\textit{RFC} 3550. Available at: {\sf 
tools.ietf.org/html/ rfc3550} (accessed March~30, 2016).
\end{thebibliography}

 }
 }

\end{multicols}

\vspace*{-3pt}

\hfill{\small\textit{Received February 24, 2016}}


\Contr

\noindent
\textbf{Borisov Andrey V.} (b.\ 1965)~---
Doctor of Science in physics and mathematics, leading scientist, 
Institute of Informatics Problems, Federal Research Center 
``Computer Science and Control'' of the Russian Academy of Sciences,
44-2~Vavilov Str., Moscow 119333, Russian Federation; \mbox{aborisov@frccsc.ru}

\vspace*{3pt}

\noindent
\textbf{Bosov Alexey V.} (b.\ 1969)~---
Doctor of Science in technology, 
Head of Laboratory, Institute of Informatics Problems, Federal Research Center 
``Computer Science and Control'' 
of the Russian Academy of Sciences, 44-2~Vavilov Str., Moscow 119333, Russian Federation; 
\mbox{abosov@frccsc.ru}

\vspace*{3pt}

\noindent
\textbf{Miller Grigoriy B.} (b.\ 1980)~---
Candidate of Science (PhD) in physics and mathematics, scientist, 
Institute of Informatics Problems, Federal Research Center 
``Computer Science and Control'' of the Russian Academy of Sciences,
44-2~Vavilov Str., Moscow 119333, Russian Federation; \mbox{gmiller@frccsc.ru}
\label{end\stat}


\renewcommand{\bibname}{\protect\rm Литература}