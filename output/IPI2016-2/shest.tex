
\def\stat{shestakov}

\def\tit{СТАТИСТИЧЕСКИЕ СВОЙСТВА МЕТОДА ПОДАВЛЕНИЯ ШУМА,
ОСНОВАННОГО НА СТАБИЛИЗИРОВАННОЙ ЖЕСТКОЙ ПОРОГОВОЙ
ОБРАБОТКЕ$^*$}

\def\titkol{Статистические свойства метода подавления шума,
основанного на стабилизированной жесткой пороговой
обработке}

\def\aut{О.\,В.~Шестаков$^1$}

\def\autkol{О.\,В.~Шестаков}

\titel{\tit}{\aut}{\autkol}{\titkol}

\index{Шестаков О.\,В.}
\index{Shestakov O.\,V.}

{\renewcommand{\thefootnote}{\fnsymbol{footnote}} \footnotetext[1]
{Работа выполнена при частичной финансовой поддержке РФФИ (проект 16-07-00736).}}


\renewcommand{\thefootnote}{\arabic{footnote}}
\footnotetext[1]{Московский государственный университет им.\ М.\,В.~Ломоносова, кафедра 
математической статистики факультета вычислительной математики и~кибернетики; 
Институт проблем информатики Федерального исследовательского
центра <<Информатика и~управление>> Российской академии наук, oshestakov@cs.msu.su}

\vspace*{-12pt}

\Abst{Методы пороговой обработки коэффициентов вейв\-лет-раз\-ло\-же\-ния 
функций сигналов и~изоб\-ра\-жений стали популярным аппаратом для подавления 
шума благодаря своей простоте, вычислительной эффективности и~возможности 
адаптации к~функциям, имеющим на разных участках различную степень регулярности. 
Рассматривается предложенный недавно стабилизированный метод жесткой 
пороговой обработки, в~котором устранены основные недостатки мягкой и~жесткой 
пороговой обработки, и~исследуются статистические свойства этого метода. 
В~модели данных с~аддитивным гауссовским шумом проводится анализ несмещенной оценки 
среднеквадратичного риска и~показывается, что при определенных условиях данная оценка 
является сильно состоятельной и~асимптотически нормальной. Данные свойства позволяют 
строить асимптотические доверительные интервалы для теоретического средне\-квад\-ратичного 
риска метода.}

\KW{вейвлеты; пороговая обработка; несмещенная оценка риска; асимптотическая 
нормальность; сильная состоятельность}

\DOI{10.14357/19922264160207} 

%\vspace*{-4pt}

\vskip 10pt plus 9pt minus 6pt

\thispagestyle{headings}

\begin{multicols}{2}

\label{st\stat}

\section{Введение}

При применении методов подавления шума в~сигналах или изображениях важным 
условием является <<экономное>> представление функций сигналов в~некотором 
специальном базисе. Данное условие позволяет применять популярные процедуры 
мягкой и~жесткой пороговой обработки~[1] к~коэффициентам разложения функций сигналов. 
Эти процедуры просты, не затратны и~часто позволяют достичь удовлетворительных 
результатов. Однако они имеют свои недостатки. Жесткая по\-роговая обработка использует 
разрывную пороговую функцию, что приводит к~появлению дополнительных артефактов, 
отсутствию устойчивости при выборе порога~[2] и~невозможности построения несмещенной 
оценки среднеквадратичной погрешности~[3]. При мягкой пороговой обработке все 
коэффициенты подвергаются изменению, вследствие чего в~оценке функции сигнала 
появляется дополнительное смещение. 

В~работе~[4] предложен стабилизированный 
вариант жесткой пороговой обработки, позволяющий обойти указанные недостатки. 
Для выбора па\-ра\-мет\-ров данного метода и~анализа погрешности можно использовать 
несмещенную оценку среднеквадратичного риска~[5]. В~данной работе исследуются 
статистические свойства этой оценки при разложении функции сигнала по вейв\-лет-ба\-зи\-су 
и~выборе <<универсального>> порога. Для метода мягкой пороговой обработки подобные 
исследования проводились в~работах~[6--9].


\section{Метод пороговой обработки}

Для функции сигнала $f\hm\in L^2(\mathbb{R})$ разложение по вейв\-лет-ба\-зи\-су имеет вид:
\begin{equation}
f=\sum\limits_{j,k\in Z}\langle f,\psi_{jk}\rangle\psi_{jk}\,,
\label{Wavelet_Decomp}
\end{equation}
где $\psi_{jk}(t)=2^{j/2}\psi(2^jt-k)$, а $\psi(t)$~--- некоторая 
вейв\-лет-функ\-ция (семейство $\{\psi_{jk}\}_{jk\in Z}$ образует ортонормированный 
базис в~$L^2(\mathbb{R})$). Индекс~$j$ в~\eqref{Wavelet_Decomp} называется 
масштабом, а индекс~$k$~--- сдвигом.\linebreak
В~данной работе рассматриваются функции сигнала на конечном отрезке $[a,b]$, 
равномерно регулярные по Липшицу с~некоторым показателем $\gamma\hm>0$. 
Кроме того, предполагается, что вейв\-лет-функ\-ция~$\psi$ имеет~$M$~непрерывных 
производных ($M\hm\geqslant\gamma$), $M$~нулевых моментов и~удовлетворяет условию:
$$
\int\limits_{-\infty}^{\infty}\abs{t^{\gamma}\psi(t)}dt<\infty\,.
$$
В этом случае найдется такая константа $C_f\hm>0$, что~[10]
\begin{equation}
\label{Wavelet_CoeffDecacy}
\abs{\langle f,\psi_{j,k}\rangle}\leq\fr{C_f}{2^{j\left(\gamma+1/2\right)}}\,.
\end{equation}

Модель данных, рассматриваемая в~данной работе, предполагает, что функция сигнала задана в~дискретных отсчетах и~наблюдения содержат шум:
\begin{equation*}
X_i = f_i + \epsilon_i\,, \qquad i = 1, \dots, 2^J\,,
\label{Data_Model}
\end{equation*}
где $2^J$~--- количество отсчетов функции сигнала;
$f_i$~--- незашумленные значения функции сигнала; $\epsilon_i$~--- 
независимые нормально распределенные случайные величины с~нулевым средним 
и~дисперсией~$\sigma^2$.
После применения дискретного вейв\-лет-пре\-об\-ра\-зо\-ва\-ния получается следующая 
модель зашумленных вейв\-лет-ко\-эф\-фи\-ци\-ентов:
\begin{equation*}
Y_{jk}=\mu_{jk}+\epsilon^W_{jk}\,,\enskip j=0,\ldots,J-1,\;k=0,\ldots,2^{j}-1\,,
\end{equation*}
где $\epsilon^W_{jk}$ независимы и~имеют такое же распределение, 
как и~$\epsilon_i$, а $\mu_{jk}\hm\approx 2^{J/2}\langle f,\psi_{jk}\rangle$~[10].

Для построения оценки функции сигнала к~коэффициентам~$Y_{jk}$ 
обычно применяется функция жесткой пороговой обработки 
$\rho_{H}(y,T)\hm=x\Ik(\abs{y}\hm>T)$ или функция мягкой пороговой 
обработки $\rho_{S}(y,T)\hm=\mathrm{sgn}\left(x\right)
\left(\abs{y}-T\right)_{+}$ с~порогом~$T$.

Как уже отмечалось, функция~$\rho_{H}$ разрывна, что приводит 
к~отсутствию устойчивости, а функция~$\rho_{S}$ приводит к~появлению 
дополнительного смещения в~оценке функции сигнала. 
В~работе~[4] предложен новый метод пороговой обработки, являющийся 
сглаженным аналогом жесткой пороговой обработки. В~этом методе оценки~$\mu_{jk}$ 
вычисляются по формулам:
\begin{equation*}
\widehat{\mu}_{jk}=\Expect \left[\rho_{H}(Y_{jk}+\lambda\xi_{jk},T)|Y_{jk}\right]\,,
\end{equation*}
где случайные величины~$\xi_{jk}$ имеют стандартное нормальное распределение 
и~не зависят от~$Y_{jk}$, а~$\lambda\hm>0$~--- 
параметр стабилизации, отвечающий за степень сглаживания. Вычисляя математическое 
ожидание, получаем:
\begin{multline*}
\widehat{\mu}_{jk}=Y_{jk}\left[\Phi\left(-\fr{T+Y_{jk}}{\lambda}\right)+
1-\Phi\left(\fr{T-Y_{jk}}{\lambda}\right)\right]+{}\\
{}+
\lambda\left[\phi\left(\fr{T-Y_{jk}}{\lambda}\right)-
\phi\left(\fr{T+Y_{jk}}{\lambda}\right)\right]\,.
\end{multline*}
Достоинством такого метода является бесконечная 
дифференцируемость~$\widehat{\mu}_{jk}$ по~$Y_{jk}$, что приводит к~более 
устойчивым оценкам~[4]. Заметим также, что при $\lambda\hm\to0$ 
получается обычный метод жесткой пороговой обработки.

\section{Несмещенная оценка среднеквадратичного риска и~ее~свойства}

Среднеквадратичная погрешность (риск) описанного выше метода определяется по формуле:
\begin{equation}
\label{Risk}
R_J(T)=\sum\limits_{j=0}^{J-1}\sum\limits_{k=0}^{2^j-1}\Expect
\left(\widehat{\mu}_{jk}-\mu_{jk}\right)^2.
\end{equation}
В~[4] показано, что
\begin{multline*}
\Expect\left(\widehat{\mu}_{jk}-\mu_{jk}\right)^2={}\\
{}=
\Expect\left[(Y_{jk}-\widehat{\mu}_{jk})^2+2\sigma^2
\fr{\partial}{\partial Y_{jk}}\,\widehat{\mu}_{jk}\right]-\sigma^2\!,
\end{multline*}
где
\begin{multline*}
\!\!\!\hspace*{-1.55032pt}\fr{\partial}{\partial Y_{jk}}\,\widehat{\mu}_{jk}=
\left[\Phi\left(-\fr{T+Y_{jk}}{\lambda}\right)+1-
\Phi\left(\fr{T-Y_{jk}}{\lambda}\right)\right]+{}\\
{}+
\fr{T}{\lambda}\left[\phi\left(\fr{T-Y_{jk}}{\lambda}\right)+
\phi\left(\fr{T+Y_{jk}}{\lambda}\right)\right].
\end{multline*}
Таким образом, величина
\begin{multline}
\label{Risk_Estimate}
\widehat{R}_J(T)={}\\
{}=\sum\limits_{j=0}^{J-1}\sum\limits_{k=0}^{2^j-1}
\left[(Y_{jk}-\widehat{\mu}_{jk})^2+2\sigma^2\fr{\partial}{\partial Y_{jk}}\,
\widehat{\mu}_{jk}-\sigma^2\right]
\end{multline}
является несмещенной оценкой~$R_J(T)$, не зависящей от ненаблюдаемых 
<<чистых>> значений~$\mu_{jk}$.

В данной работе параметр~$\lambda$ будем полагать фиксированным, 
а~в~качестве~$T$ рассмотрим так называемый <<универсальный>> порог 
$T_U\hm=\sigma\sqrt{2\ln 2^J}$, который позволяет достичь хороших результатов 
при подавлении шума и~обеспечивает близость среднеквадратичного риска 
к~минимальному при жесткой и~мягкой пороговых обработках~[10].

Покажем, что оценка~(\ref{Risk_Estimate}) является асимптотически нормальной. 
Это свойство позволяет строить асимптотические доверительные интервалы для 
риска~(\ref{Risk}).

\smallskip

\noindent
\textbf{Теорема~1.}\
\textit{Пусть~$f$ задана на конечном отрезке и~является равномерно регулярной по 
Липшицу с~параметром $\gamma\hm>0$. Тогда}
\begin{equation}
\label{Normality}
\mbox{P}\left(\fr{\widehat{R}_J(T_U)-R_J(T_U)}{\sigma^2\sqrt{2^{J+1}}}<x\right)
\Rightarrow\Phi(x)\,.
\end{equation}

\smallskip

\noindent
Д\,о\,к\,а\,з\,а\,т\,е\,л\,ь\,с\,т\,в\,о\,.\ \  
Выберем~$p$ такое, что $(2\gamma\hm+1)^{-1}\hm<p\hm<1$. Обозначим слагаемые 
в~(\ref{Risk_Estimate}) через $F_{jk}(T)$ и~запишем 
разность $\widehat{R}_J(T_U)\hm-R_J(T_U)$ в~виде:
\begin{multline}
\label{Two_Sums}
\hspace*{-5.65323pt}\!\widehat{R}_J(T_U)-R_J(T_U)=\sum\limits_{j=0}^{[pJ]}\sum\limits_{k=0}^{2^j-1}
\left[F_{jk}(T_U)-\Expect F_{jk}(T_U)\right]\;+{}\hspace*{-1.11108pt}\\
{}+
\sum\limits_{j=[pJ]+1}^{J-1}\sum\limits_{k=0}^{2^j-1}
\left[F_{jk}(T_U)-\Expect F_{jk}(T_U)\right].
\end{multline}
Число слагаемых в~первой сумме не превосходит $2^{[pJ]+1}$. Кроме того, 
существует такая константа $C_F\hm>0$, что
\begin{equation}
\label{Bounds}
\abs{F_{jk}(T_U)-\Expect F_{jk}(T_U)}\leqslant C_FT_U\;\;\mbox{п.в.}
\end{equation}
Применяя неравенство Хеффдинга и~учитывая вид~$T_U$, 
получаем, что для любого $\delta\hm>0$ найдется константа $C_\delta\hm>0$ такая, что
\begin{multline*}
\hspace*{-7.24762pt}\mathrm{P}\left(\fr{\abs{\sum\nolimits_{j=0}^{[pJ]}\sum\nolimits_{k=0}^{2^j-1}
\left[F_{jk}(T_U)-\Expect F_{jk}(T_U)\right]}}{\sigma^2\sqrt{2^{J+1}}}>\delta\right)
\leqslant{}\\
{}\leqslant \exp\left\{-C_\delta\fr{2^{J-pJ}}{J}\right\},
\end{multline*}
т.\,е.
\begin{multline}
\label{First_Sum}
\fr{\sum\nolimits_{j=0}^{[pJ]}\sum\nolimits_{k=0}^{2^j-1}
\left[F_{jk}(T_U)-\Expect F_{jk}(T_U)\right]}
{\sigma^2\sqrt{2^{J+1}}}\stackrel{\mathrm{P}}{\to}0\\ 
\mbox{при}\enskip J\to\infty.
\end{multline}
Для всех слагаемых во второй сумме в~(\ref{Two_Sums}) в~силу~(\ref{Wavelet_CoeffDecacy})
выполнено $\mu_{jk}\hm\to0$ при $J\hm\to\infty$.
Рас\-смот\-рим дисперсии слагаемых в~этой сумме. Имеем 
$\D F_{jk}(T_U)\hm=\D [Y^2_{jk}-\sigma^2+U_{jk}(T_U)]$, где
\begin{multline*}
U_{jk}(T_U)=F_{jk}(T_U)-Y^2_{jk}+\sigma^2={}\\
{}=
\left(\widehat{\mu}_{jk}-2Y_{jk}\right)\widehat{\mu}_{jk}+2\sigma^2
\fr{\partial}{\partial Y_{jk}}\widehat{\mu}_{jk}.
\end{multline*}
Пусть $A_J=\sigma\sqrt{A\ln 2^J}$, где $0<A<2$. Тогда
\begin{multline*}
\D U_{jk}(T_U)\leqslant \Expect U^2_{jk}(T_U)={}\\
{}=\Expect 
\left[\Ik(\abs{Y_{jk}}>A_J)U^2_{jk}(T_U)\right]+{}\\
{}+
\Expect\left[\Ik(\abs{Y_{jk}}\leqslant A_J)U^2_{jk}(T_U)\right].
\end{multline*}
Поскольку $T_U-A_J\to\infty$ при $J\hm\to\infty$, учитывая 
определение~$\widehat{\mu}_{jk}$ и~$({\partial}/{\partial Y_{jk}})\widehat{\mu}_{jk}$, 
заключаем, что найдется такое $\varepsilon_J\hm>0$, что 
$\abs{\Ik(\abs{Y_{jk}}\hm\leqslant A_J)U^2_{jk}(T_U)}\hm\leqslant\varepsilon_J$ п.в.\ 
и~$\varepsilon_J\hm\to0$ при $J\hm\to\infty$. Следовательно, 
$\Expect[\Ik(\abs{Y_{jk}}\hm\leqslant A_J)U^2_{jk}(T_U)]\hm\to 0$ при $J\hm\to\infty$.

Далее найдется такая константа $C\hm>0$, что 
$\Expect [\Ik(\abs{Y_{jk}}\hm>A_J)U^2_{jk}(T_U)]\hm\leqslant 
C\Expect[\Ik(\abs{Y_{jk}}\hm>A_J)Y^4_{jk}]\hm\to0$ при $J\hm\to\infty$. Таким образом, 
учитывая очевидное соотношение для дисперсии суммы случайных величин $X+Y$: 
$(\sqrt{\D X}\hm-\sqrt{\D Y})^2\hm\leqslant\D(X+Y)
\hm\leqslant(\sqrt{\D X}\hm+\sqrt{\D Y})^2$, получаем:
\begin{multline}
\label{Var_Lim1}
\hspace*{-9.58pt}\lim\limits_{J\to\infty}\fr{\D\sum\nolimits_{j=[pJ]+1}^{J-1}
\sum\nolimits_{k=0}^{2^j-1}\left[F_{jk}(T_U)-\Expect F_{jk}(T_U)\right]}
{\D\sum\nolimits_{j=[pJ]+1}^{J-1}\sum\nolimits_{k=0}^{2^j-1}Y^2_{jk}}={}\\
{}=1\,.
\end{multline}
Кроме того, поскольку~$Y_{jk}$ независимы, $\D Y^2_{jk}\hm=
2\sigma^4\hm+4\sigma^2\mu^2_{jk}$ и~в~(\ref{Var_Lim1}) $\mu_{jk}\hm\to0$ при 
$J\hm\to\infty$, получаем:
\begin{equation}
\label{Var_Lim2}
\lim\limits_{J\to\infty}\fr{\D\sum\nolimits_{j=[pJ]+1}^{J-1}
\sum\nolimits_{k=0}^{2^j-1}Y^2_{jk}}{\sigma^42^{J+1}}=1\,.
\end{equation}

Наконец, выполнено условие Линдеберга: для любого $\epsilon\hm>0$ при $J\hm\to\infty$
\begin{multline}
\label{Norm_Cond}
\fr{1}{D^2_J}\sum\limits_{j=[pJ]+1}^{J-1}\sum\limits_{k=0}^{2^j-1} \Expect  
\left[
\vphantom{\left((T_U)\right)^2}\left( F_{jk}(T_U)-{}\right.\right.\\
\left.\left.{}-\Expect F_{jk}(T_U)\right)^2\Ik\left( 
\left\vert F_{jk}(T_U)-\Expect F_{jk}(T_U)\right\vert >\epsilon D_J\right)
\right]\rightarrow{}\\
{}\rightarrow 0\,,
\end{multline}
где 
$$
D^2_J=\D\sum\limits_{j=[pJ]+1}^{J-1}\sum\limits_{k=0}^{2^j-1}
[F_{jk}(T_U)-\Expect F_{jk}(T_U)]\,.
$$

Действительно, в~силу~(\ref{Bounds}), (\ref{Var_Lim1}) и~(\ref{Var_Lim2}) начиная 
с~некоторого~$J$ все индикаторы в~(\ref{Norm_Cond}) обращаются 
в~ноль. Объединяя~(\ref{First_Sum})--(\ref{Norm_Cond}), получаем~(\ref{Normality}). 
Теорема доказана.

\smallskip

Докажем теперь свойство сильной состоятельности оценки~(\ref{Risk_Estimate}), 
справедливое при более слабых ограничениях на функцию сигнала.

\smallskip

\noindent
\textbf{Теорема~2.}\ 
\textit{Пусть $f\in  L^2(\mathbb{R})$ и~задана на конечном отрезке, тогда при 
любом $\alpha>1/2$ имеет место сходимость}:
\begin{equation}
\label{Strong_Con}
\fr{\widehat{R}_J(T_U)-R_J(T_U)}{2^{\alpha J}}\rightarrow 0 \enskip \mbox{п.в.\ при}\enskip 
J\rightarrow\infty.
\end{equation}

\noindent
Д\,о\,к\,а\,з\,а\,т\,е\,л\,ь\,с\,т\,в\,о\,.\ \  
Используя неравенство Хеффдинга с~учетом~(\ref{Bounds}) и~вида~$T_U$, получаем, 
что для любого $\delta\hm>0$ найдется константа $C_\delta\hm>0$ такая, что
\begin{multline}
\label{Prob_Conv}
p_J=\p\left(\fr{\abs{\widehat{R}_J(T_U)-R_J(T_U)}}{2^{\alpha J}}>\delta\right)
\leqslant{}\\
{}\leqslant \exp\left\{-C_\delta\fr{2^{2\alpha J-J}}{J}\right\}.
\end{multline}
Далее
\begin{equation*}
\sum\limits_{J=1}^{\infty}p_J<\infty
\end{equation*}
и в~силу леммы Бо\-ре\-ля--Кан\-тел\-ли осуществляется лишь конечное число событий 
под вероятностью в~(\ref{Prob_Conv}). Следовательно, выполнено~(\ref{Strong_Con}). 
Теорема доказана.

{\small\frenchspacing
 {%\baselineskip=10.8pt
 \addcontentsline{toc}{section}{References}
 \begin{thebibliography}{99}
\bibitem{1-sh}
\Au{Donoho D., Johnstone I.\,M.} Ideal spatial adaptation via wavelet shrinkage~// 
Biometrika, 1994. Vol.~81. No.\,3. P.~425--455.

\bibitem{2-sh}
\Au{Breiman L.} Heuristics of instability and stabilization in model selection~// 
Ann. Statist., 1996. Vol.~24. No.\,6. P.~2350--2383.

\bibitem{3-sh}
\Au{Jansen M.} Noise reduction by wavelet thresholding.~--- 
Lecture notes in statistics ser.~--- Springer Verlag, 2001. 
 Vol.~161. 218~p.

\bibitem{4-sh}
\Au{Huang H.-C., Lee T.\,C.\,M.} Stabilized thresholding with
generalized sure for image denoising~// IEEE
17th  Conference (International) on Image Processing Proceedings.~---
IEEE, 2010. P.~1881--1884.

\bibitem{5-sh}
\Au{Stein C.} Estimation of the mean of a~multivariate normal distribution~// 
Ann. Stat., 1981. Vol.~9. No.\,6. P.~1135--1151.

\bibitem{6-sh}
\Au{Маркин А.\,В.} Предельное распределение оценки риска при пороговой обработке 
вейв\-лет-ко\-эф\-фи\-ци\-ен\-тов~// Информатика и~её применения, 2009. Т.~3. 
Вып.~4. С.~57--63.

\bibitem{7-sh}
\Au{Маркин А.\,В., Шестаков~О.\,В.} 
О~состоятельности оценки риска при пороговой обработке вейв\-лет-ко\-эф\-фи\-ци\-ен\-тов~// 
Вестн. Моск. ун-та. Сер.~15. Вычисл. матем. и~киберн., 2010. №\,1. C.~26--34.

\bibitem{8-sh}
\Au{Шестаков О.\,В.}  Асимптотическая нормальность оценки риска пороговой обработки 
вейв\-лет-ко\-эф\-фи\-ци\-ен\-тов при выборе адаптивного порога~// 
Докл. РАН, 2012. Т.~445. №\,5. С.~513--515.

\bibitem{9-sh}
\Au{Шестаков О.\,В.} О~точности приближения распределения оценки риска 
пороговой обработки вейв\-лет-ко\-эф\-фи\-ци\-ен\-тов 
сигнала нормальным законом при неизвестном уровне шума~// Системы и~средства 
информатики, 2012. Т.~22. №\,1. С.~142--152.

\bibitem{10-sh}
\Au{Mallat S.} A~wavelet tour of signal processing.~--- New York, NY, USA: 
Academic Press, 1999. 857~p.

\end{thebibliography}

 }
 }

\end{multicols}

\vspace*{-3pt}

\hfill{\small\textit{Поступила в~редакцию 22.01.16}}

\vspace*{8pt}

%\newpage

%\vspace*{-24pt}

\hrule

\vspace*{2pt}

\hrule

%\vspace*{8pt}



\def\tit{STATISTICAL PROPERTIES OF~THE~DENOISING METHOD BASED~ON~THE~STABILIZED 
HARD THRESHOLDING}

\def\titkol{Statistical properties of the denoising method based on~the~stabilized 
hard thresholding}

\def\aut{O.\,V.~Shestakov$^{1,2}$}

\def\autkol{O.\,V.~Shestakov}

\titel{\tit}{\aut}{\autkol}{\titkol}

\vspace*{-9pt}

\noindent
$^1$Department of Mathematical Statistics, Faculty of Computational Mathematics 
and Cybernetics,\linebreak
$\hphantom{^1}$M.\,V.~Lomonosov Moscow State University, 1-52~Leninskiye Gory, 
GSP-1, Moscow 119991, Russian\linebreak
$\hphantom{^1}$Federation

\noindent
$^2$Institute of Informatics Problems, Federal Research Center 
``Computer Science and Control''
of the Russian\linebreak
$\hphantom{^1}$Academy of Sciences, 44-2~Vavilov Str., Moscow 119333,  Russian Federation

\def\leftfootline{\small{\textbf{\thepage}
\hfill INFORMATIKA I EE PRIMENENIYA~--- INFORMATICS AND
APPLICATIONS\ \ \ 2016\ \ \ volume~10\ \ \ issue\ 2}
}%
 \def\rightfootline{\small{INFORMATIKA I EE PRIMENENIYA~---
INFORMATICS AND APPLICATIONS\ \ \ 2016\ \ \ volume~10\ \ \ issue\ 2
\hfill \textbf{\thepage}}}

\vspace*{3pt}


\Abstend{The thresholding techniques for the wavelet coefficients of the signal 
and image functions have become a popular denoising tool because of their simplicity, 
computational efficiency, and possibility to adapt to the functions with different 
amounts of smoothness in different locations. The paper considers the recently proposed 
stabilized hard thresholding method which avoids the main disadvantages of the popular 
soft and hard thresholding techniques. The statistical properties of this method are 
studied. The unbiased risk estimate is analyzed in the model with an additive Gaussian 
noise. Wavelet thresholding risk analysis is an important practical task, 
because it allows determining the quality of the techniques themselves and 
the equipment which is being used. The paper proves that under certain conditions, 
the unbiased risk estimate is strongly consistent and asymptotically normal. 
These properties allow constructing the asymptotic confidence intervals for the 
theoretical mean squared risk of the method.}

\KWE{wavelets; thresholding; unbiased risk estimate; asymptotic normality; 
strong consistency}


\DOI{10.14357/19922264160207}

%\vspace*{-12pt}

\Ack
\noindent
The work was partly supported by the Russian Foundation for Basic
Research (project 16-07-00736).


%\vspace*{3pt}

  \begin{multicols}{2}

\renewcommand{\bibname}{\protect\rmfamily References}
%\renewcommand{\bibname}{\large\protect\rm References}

{\small\frenchspacing
 {%\baselineskip=10.8pt
 \addcontentsline{toc}{section}{References}
 \begin{thebibliography}{99}
 \bibitem{1-sh-1}
\Aue{Donoho, D., and I.\,M.~Johnstone}. 1994. 
Ideal spatial adaptation via wavelet shrinkage. \textit{Biometrika} 81(3):425--455.

\bibitem{2-sh-1}
\Aue{Breiman, L.} 1996. Heuristics of instability and stabilization in model selection. 
\textit{Ann. Stat.} 24(6):2350--2383.

\bibitem{3-sh-1}
\Aue{Jansen, M.} 2001. \textit{Noise reduction by wavelet thresholding}. 
Lecture notes in statistics ser. Springer Verlag.  Vol.~161. 218~p.

\bibitem{4-sh-1}
\Aue{Huang, H.-C., and T.\,C.\,M.~Lee}. 2010. 
Stabilized thresholding with generalized sure for image denoising. 
\textit{IEEE
17th  Conference (International) on Image Processing Proceedings}.
IEEE. 1881--1884.

\bibitem{5-sh-1}
\Aue{Stein, C.} 1981. Estimation of the mean of a~multivariate normal distribution. 
\textit{Ann. Stat.} 9(6):1135--1151.

\bibitem{6-sh-1}
\Aue{Markin, A.\,V.} 2009. Predel'noe raspredelenie otsenki riska pri porogovoy 
obrabotke veyvlet-koeffitsientov [Limit distribution of risk estimate of wavelet 
coefficient thresholding].
\textit{Informatika i~ee Primeneniya}~--- \textit{Inform. Appl.} 3(4):57--63.

\bibitem{7-sh-1}
\Aue{Markin, A.\,V., and O.\,V.~Shestakov}. 2010. 
Consistency of risk estimation with 
thresholding of wavelet coefficients. 
\textit{Moscow Univ. Comput. Math. Cybern}. 34(1):22--30.

\bibitem{8-sh-1}
\Aue{Shestakov, O.\,V.} 2012. 
Asymptotic normality of adaptive wavelet thresholding risk estimation.
\textit{Dokl. Math.} 86(1):556--558.

\bibitem{9-sh-1}
\Aue{Shestakov, O.\,V.} 2012. O~tochnosti priblizheniya raspredeleniya otsenki 
riska porogovoy obrabotki veyvlet-ko\-ef\-fi\-tsi\-en\-tov signala normal'nym zakonom pri 
neizvestnom urovne shuma [On the accuracy of normal approximation for risk estimate 
distribution when thresholding signal wavelet coefficients in case of unknown 
noise level]. \textit{Sistemy i~Sredstva Informatiki}~--- \textit{Systems and Means of 
Informatics} 22(1):142--152.

\bibitem{10-sh-1}
\Aue{Mallat, S.} 1999. \textit{A~wavelet tour of signal processing}. 
New York, NY:  Academic Press. 857~p.

\end{thebibliography}

 }
 }

\end{multicols}

\vspace*{-3pt}

\hfill{\small\textit{Received January 22, 2016}}

\Contrl

\noindent
\textbf{Shestakov Oleg V.} (b.\ 1976)~--- 
Doctor of Science in physics and mathematics, associate professor, 
Department of Mathematical Statistics, Faculty of Computational Mathematics 
and Cybernetics, M.\,V.~Lomonosov Moscow State University, 1-52~Leninskiye Gory, 
GSP-1, Moscow 119991, Russian Federation; senior scientist, 
Institute of Informatics Problems, Federal Research Center 
``Computer Science and Control''
of the Russian Academy of Sciences, 44-2~Vavilov Str., Moscow 119333, 
Russian Federation; oshestakov@cs.msu.su


\label{end\stat}


\renewcommand{\bibname}{\protect\rm Литература}