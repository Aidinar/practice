\newcommand{\al}{a_\lambda}
\newcommand{\am}{a_\mu}
\newcommand{\fl}{f_\lambda}
\newcommand{\fm}{f_\mu}

\def\stat{kudr}

\def\tit{БАЙЕСОВСКАЯ РЕКУРРЕНТНАЯ МОДЕЛЬ РОСТА НАДЕЖНОСТИ: ПАРАБОЛИЧЕСКОЕ РАСПРЕДЕЛЕНИЕ ПАРАМЕТРОВ}

\def\titkol{Байесовская рекуррентная модель роста надежности: параболическое распределение параметров}

\def\aut{А.\,А.~Кудрявцев$^1$,  С.\,И.~Палионная$^2$}

\def\autkol{А.\,А.~Кудрявцев,  С.\,И.~Палионная}

\titel{\tit}{\aut}{\autkol}{\titkol}

\index{Кудрявцев А.\,А.}
\index{Палионная С.\,И.}
\index{Kudryavtsev A.\,A.}
\index{Palionnaia S.\,I.}

%{\renewcommand{\thefootnote}{\fnsymbol{footnote}} \footnotetext[1]
%{Работа выполнена при частичной финансовой поддержке РФФИ (проект 16-07-00736).}}


\renewcommand{\thefootnote}{\arabic{footnote}}
\footnotetext[1]{Московский государственный университет им.~М.\,В.~Ломоносова, 
факультет вычислительной математики и~кибернетики, \mbox{nubigena@mail.ru}}
\footnotetext[2]{Московский государственный университет им.~М.\,В.~Ломоносова, 
факультет вычислительной математики и~кибернетики, \mbox{sofiapalionnaya@gmail.com}}

\Abst{Данная работа посвящена рассмотрению случая параболического распределения 
параметров в~байесовской рекуррентной модели роста надежности сложных модифицируемых 
информационных сис\-тем. Надежность системы зависит от соотношения параметров, 
интерпретируемых в~теории надежности как показатели <<дефективности>> 
и~<<эффективности>> средства, исправляющего ошибки в~системе. При использовании 
байесовских моделей применительно к задачам теории надежности предполагается, что 
основные параметры системы не являются заданными, а известны только их априорные 
распределения. Приводятся формулы для вычисления средней предельной 
надежности системы для параболического априорного распределения параметров. 
Также приводятся численные результаты для модельных примеров.}

\KW{модифицируемые информационные системы; теория надежности; байесовский подход; 
параболическое распределение}

\DOI{10.14357/19922264160209} 

\vspace*{6pt}

\vskip 12pt plus 9pt minus 6pt

\thispagestyle{headings}

\begin{multicols}{2}

\label{st\stat}

\section{Введение}

Как правило, любая впервые созданная сложная информационная система 
не обладает необходимой надежностью. Вследствие этого она подвер\-гается различным 
модификациям, целью которых является устранение дефектов, препятствующих правильному 
функционированию системы. Изменения в~систему могут вноситься на стадиях разработки, 
испытаний и~опытной эксплуатации, а~так\-же во время штатного функционирования.

Для формализации понятия надежности системы будем характеризовать ее в~каждый момент 
времени~$t$ параметром $p(t)$, при этом время считаем непрерывным. Так как система 
подвергается модификациям в~случайные моменты времени $0\hm=Y_0\hm\le Y_1 \hm\le Y_2\hm
\le\cdots$, параметр $p(t)$ изменяется, принимая соответственно значения 
$p(t) \hm= p(Y_j) \hm= p_j$ при $Y_j\hm \le\hm t \hm<  Y_{j+1}$ (предполагается, 
что траектории процесса $p(t)$ непрерывны справа, а модификации происходят мгновенно). 
Причем при модификациях параметр надежности может как увеличиваться, так и~уменьшаться 
из-за некачественных модификаций.

К числу математических моделей, описыва\-ющих изменение надежности модифицируемых 
информационных систем, относятся рекуррентные модели роста надежности. 
Рассмотрим, в~частности, дискретную экспоненциальную модель, которая 
определяется следующим образом:
$$
p_{j+1} = \eta_{j+1}p_j + \theta_{j+1}(1-p_j)\,.
$$
Здесь $\{(\theta_j, \eta_j)\}$, $j\hm\ge1$,~--- последовательность независимых 
одинаково распределенных двумерных случайных векторов таких, что $0 \hm< \eta_1 \hm< 1$; 
$0 \hm< \theta_1  \hm< 1$ почти наверное. Начальная надежность~$p_0$ считается 
заданной, случайные величины~$\eta_j$ (параметры <<дефективности>>) и~$\theta_j$  
(параметры <<эффективности>>)  описывают соответственно возможное уменьшение 
и~увеличение надежности.

Обозначим $\lambda = 1 \hm- \e\theta_j$, $\mu \hm= \e\eta_j$.
В~книге~\cite{KS2006} доказано, что при условии $\lambda \hm+\mu\hm\neq1$
$$
p=\lim\limits_{j\to\infty}\e p_j = \fr{\mu}{\lambda+\mu}\,.
$$
Величина $p$ характеризует асимптотическое значение надежности системы в~рамках 
некоторой рекуррентной модели, задаваемой набором $\{(\theta_j, \eta_j)\}$.

В рамках байесовского подхода к постановке задач теории надежности 
можно рассмотреть более сложную ситуацию, где основные параметры сис\-те\-мы~$\lambda$ 
и~$\mu$ предполагаются случайными. В~таком случае наиболее естественной и~удобной 
для изучения характеристикой является усредненное значение предельной надежности, 
т.\,е.

\noindent
$$
p_{\mathrm{сред}} = \e p = \e \fr{\mu}{\lambda+\mu}\,,
$$
где усреднение ведется по совместному распределению случайных величин~$\lambda$ и~$\mu$.

Так как случайные величины~$\eta_1$ и~$\theta_1$ удовлетворяют ограничениям 
$0\hm < \eta_1 \hm< 1$, $0 \hm< \theta_1  \hm< 1$, средние значения~$\lambda$ и~$\mu$ 
величин $1 \hm- \e\theta_j$ и~$\e\eta_j$ соответственно также находятся на 
отрезке $[0,1]$. Поэтому в~качестве априорных распределений параметров~$\lambda$ 
и~$\mu$ следует выбирать только распределения, сосредоточенные на $[0,1]$.

Далее будут рассмотрены независимые случайные параметры~$\lambda$ и~$\mu$, 
имеющие параболические распределения на некоторых (вообще говоря, разных) отрезках, 
являющихся подмножествами отрезка $[0,1]$.

\section{Основные результаты}

Пусть средний параметр <<эффективности>>~$\lambda$ и~средний параметр 
<<дефективности>>~$\mu$ независимы и~имеют параболическое 
распределение $P(a_\lambda, b_\lambda)$, $0\hm\le a_\lambda\hm<b_\lambda\hm\le1$, 
и~$P(a_\mu, b_\mu)$, $0\hm\le a_\mu\hm<b_\mu\hm\le1$, соответственно.

Для плотности $f_\xi(x)$ некоторой случайной величины~$\xi$, имеющей 
параболическое распределение с~параметрами $(a_\xi, b_\xi)$, справедливо
\begin{equation}
\label{Density Parabolic}
f_\xi(x) = \fr{6(x-a_\xi)(b_\xi-x)}{(b_\xi-a_\xi)^3}\,,\ x\in[a_\xi, b_\xi]\,.
\end{equation}

Заметим, что плотность параболического распределения может быть представлена в~виде 
полинома:
$$
f_\xi(x)=\sum\limits_{i=0}^{2}c_{\xi,i}\, x^i\cdot\Ik(x\in[a_\xi,b_\xi])\,,
$$
где 
\begin{gather*}
c_{\xi,0} = - \fr{6a_\xi b_\xi}{(b_\xi-a_\xi)^3}\,;\quad
c_{\xi,1} = \fr{6(a_\xi + b_\xi)}{(b_\xi-a_\xi)^3}\,;\\ 
c_{\xi,2} =- \fr{6}{(b_\xi-a_\xi)^3}\,.
\end{gather*}

Очевидно, что при байесовском подходе вы\-чис\-ление вероятностных характеристик 
на\-деж\-ности $p \hm= \mu/{(\lambda+\mu)}$ удобно производить, базируясь на 
известном распределении величины $\rho\hm=\lambda/\mu$. 
В~работе~\cite{K2016} были получены формулы для плотности распределения 
случайной величины~$\rho$ в~предположении, что плотности параметров~$\lambda$ 
и~$\mu$ имеют полиномиальный вид.
Основываясь на характеристиках распределения параметра~$\rho$ и~учитывая, что 
$p \hm=1/(1\hm+\rho),$
имеем для функции распределения и~плот\-ности~$p$ следующие соотношения:
$$
F_p(x)=1-F_\rho\left(\fr{1-x}{x}\right)\,;\enskip 
f_p=\fr{1}{x^2}\,f_\rho\left(\fr{1-x}{x}\right).
$$


Введем обозначение:
\begin{multline}
L(a,b,x)=\fr{1}{x^2} \il{a}{b}{y}\fl\Bigg(\fr{1-x}{x} y\Bigg)\fm(y)\, dy ={}\\
{}=\sum\limits_{i=0}^{2}\sum\limits_{j=0}^{2}c_{\lambda,i}c_{\mu,j}
\fr{b^{i+j+2}-a^{i+j+2}}{i+j+2}\, \fr{(1-x)^i}{x^{i+2}},
\label{Integral_General}
\end{multline}
где $a$ и~$b$ одновременно принадлежат отрезкам $[a_\mu,b_\mu]$ 
и~$[a_\lambda/x,b_\lambda/x]$.

Рассмотрев всевозможные комбинации взаимного расположения точек~$a_\mu$, 
$b_\mu$, $a_\lambda/x$ и~$b_\lambda/x$ на отрезке $[0,1]$, убеждаемся 
в~справедливости следующего утверждения.

\smallskip

\noindent
\textbf{Теорема~1.}\
\textit{Пусть независимые случайные величины~$\lambda$ и~$\mu$ имеют 
параболическое распределение, а их плотности $\fl(x)$ и~$\fm(x)$ 
определяются соотношением}~(\ref{Density Parabolic}) 
\textit{с~соответствующими параметрами.
Тогда случайная величина $p \hm= {{\mu}/{(\lambda+\mu)}}$ имеет плотность}:
\begin{multline*}
f_p(x)={}\\
{}=\Ik\left(\fr{\am}{b_\lambda+\am}<x\le\min\left\{
\fr{\am}{\al+\am},\fr{b_\mu}{b_\lambda+b_\mu}\right\}\right)\times{}\\
{}\times L
\Bigg(\am,\fr{b_\lambda x}{1-x},x\Bigg)+{}\\
{}+\Ik\left(\fr{b_\mu}{b_\lambda+b_\mu}<x\le\fr{\am}{\al+\am}\right)L(\am,b_\mu,x)+{}\\
{}+\Ik\left(\fr{\am}{\al+\am}<x\le\fr{b_\mu}{b_\lambda+b_\mu}\right)L\Bigg(
\fr{\al x}{1-x},\fr{b_\lambda x}{1-x},x\Bigg)+{}\\
{}+\Ik\left(\max\left\{\fr{\am}{\al+\am},\fr{b_\mu}{b_\lambda+b_\mu}\right\}<x\le
\fr{b_\mu}{\al+b_\mu}\right)\times{}\\
{}\times L\left(\fr{\al x}{1-x},b_\mu,x\right),
\end{multline*}
\textit{где величины $L(a,b,x)$ определены соотношением}~(\ref{Integral_General}).

\smallskip

Для $\xi\in\{\lambda,\mu\}$ введем дополнительное обозначение:
\begin{multline}
\label{Integral_J}
J_\xi(d,b)=\sum\limits_{i=0}^{2}\sum\limits_{j=0}^{2}c_{\lambda,i}
c_{\mu,j}\fr{d^{i+j+2}}{i+j+2}\times{}\\
\hspace*{-1.8mm}{}\times\!\sum\limits_{k=0}^{l_\xi}\!C_{l_\xi}^k(-1)^k \!
\Bigg[\!
\Ik\left(k\neq l_\xi\right)\fr{b^{k-l_\xi}}{k-l_\xi}+\Ik\left(
k=l_\xi\right)\ln \,b\Bigg]\!,\!\!\!
\end{multline}
где $l_\xi\hm=i \Ik(\xi=\lambda)+(j+1)\Ik(\xi=\mu)$.

\end{multicols}

\begin{table}\small
\begin{center}
\Caption{Частные значения $p_{\mathrm{сред}}$ ($\lambda \hm\sim P(a_{\lambda}, 
b_{\lambda})$, $\mu \hm\sim P(a_{\mu}, b_{\mu})$)}
\vspace*{2ex}

%\tabcolsep=5pt
\begin{tabular}{|c|c|c|c|c|c|c|c|c|c|c|}
\hline
&\multicolumn{10}{c|}{$a_{\mu}$; $b_{\mu}$}\\
\cline{2-11}
\multicolumn{1}{|c|}{\raisebox{6pt}[0pt][0pt]{$a_{\lambda}$; $b_{\lambda}$}}
& 0,0;  1& 0,1;  1& 0,2;  1& 0,3;  1& 0,4;  1& 0,5;  1& 0,6;  1& 0,7;  1& 0,8;  1& 0,9;  1\\
\hline
 0,0;  1& 0,50& 0,53& 0,56& 0,58& 0,60& 0,62& 0,63& 0,65& 0,66& 0,67\\
 0,1;  1& 0,47& 0,50& 0,53& 0,55& 0,57& 0,59& 0,60& 0,62& 0,63& 0,65\\
 0,2;  1& 0,44& 0,47& 0,50& 0,52& 0,54& 0,56& 0,58& 0,59& 0,61& 0,62\\
 0,3;  1& 0,42& 0,45& 0,48& 0,50& 0,52& 0,54& 0,56& 0,57& 0,59& 0,60\\
 0,4;  1& 0,40& 0,43& 0,46& 0,48& 0,50& 0,52& 0,54& 0,55& 0,57& 0,58\\
 0,5;  1& 0,38& 0,41& 0,44& 0,46& 0,48& 0,50& 0,52& 0,53& 0,55& 0,56\\
 0,6;  1& 0,37& 0,40& 0,42& 0,44& 0,46& 0,48& 0,50& 0,52& 0,53& 0,54\\
 0,7;  1& 0,35& 0,38& 0,41& 0,43& 0,45& 0,47& 0,48& 0,50& 0,51& 0,53\\
 0,8;  1& 0,34& 0,37& 0,39& 0,41& 0,43& 0,45& 0,47& 0,49& 0,50& 0,51\\
 0,9;  1& 0,33& 0,35& 0,38& 0,40& 0,42& 0,44& 0,46& 0,47& 0,49& 0,50\\
\hline
\end{tabular}
\end{center}
\end{table}

\begin{table}\small
\begin{center}
\Caption{Частные значения $p_{\mathrm{сред}}$ ($\lambda \hm\sim P(a_{\lambda}, 
b_{\lambda})$, $\mu \hm\sim P(a_{\mu}, b_{\mu})$)}
\vspace*{2ex}

%\tabcolsep=5pt
\begin{tabular}{|c|c|c|c|c|c|c|c|c|c|c|}
\hline
&\multicolumn{10}{c|}{$a_{\mu}$; $b_{\mu}$}\\
\cline{2-11}
\multicolumn{1}{|c|}{\raisebox{6pt}[0pt][0pt]{$a_{\lambda}$; $b_{\lambda}$}}
& 0;  0,1& 0;  0,2& 0;  0,3& 0;  0,4& 0;  0,5& 0;  0,6& 0;  0,7& 0;  0,8& 0;  0,9& 0;  1,0\\
\hline
 0;  0,1& 0,50& 0,65& 0,72& 0,77& 0,81& 0,83& 0,85& 0,86& 0,88& 0,89\\
 0;  0,2& 0,35& 0,50& 0,59& 0,65& 0,69& 0,72& 0,75& 0,77& 0,79& 0,81\\
 0;  0,3& 0,28& 0,41& 0,50& 0,56& 0,61& 0,65& 0,68& 0,70& 0,72& 0,74\\
 0;  0,4& 0,23& 0,35& 0,44& 0,50& 0,55& 0,59& 0,62& 0,65& 0,67& 0,69\\
 0;  0,5& 0,19& 0,31& 0,39& 0,45& 0,50& 0,54& 0,57& 0,60& 0,63& 0,65\\
 0;  0,6& 0,17& 0,28& 0,35& 0,41& 0,46& 0,50& 0,53& 0,56& 0,59& 0,61\\
 0;  0,7& 0,15& 0,25& 0,32& 0,38& 0,43& 0,47& 0,50& 0,53& 0,55& 0,58\\
 0;  0,8& 0,14& 0,23& 0,30& 0,35& 0,40& 0,44& 0,47& 0,50& 0,53& 0,55\\
 0;  0,9& 0,12& 0,21& 0,28& 0,33& 0,37& 0,41& 0,45& 0,47& 0,50& 0,52\\
 0;  1,0& 0,11& 0,19& 0,26& 0,31& 0,35& 0,39& 0,42& 0,45& 0,48& 0,50\\
\hline
\end{tabular}
\end{center}
\end{table}

\begin{multicols}{2}


Воспользовавшись теоремой~1 для вычисления средней надежности системы
$$
p_{\mathrm{сред}} = \e p =\int x f_p(x)\,dx\,,
$$
убеждаемся в~справедливости следующего утверждения.

\smallskip

\noindent
\textbf{Теорема~2.}\
\textit{Пусть средний параметр <<эф\-фек\-тив\-ности>>~$\lambda$ и~средний 
параметр <<дефективности>>~$\mu$ удовле\-тво\-ря\-ют условиям теоремы~$1$. 
Тогда средняя предельная надежность системы имеет вид}:
\begin{multline}
p_{\mathrm{сред}}=J_\mu\left(
b_\lambda,\fr{b_\lambda}{b_\lambda+a_\mu}\right) + 
J_\mu\left(a_\lambda, \fr{a_\lambda}{a_\lambda+b_\mu}\right)+ {}\\
{}+
J_\lambda\left(a_\mu,\fr{a_\mu}{b_\lambda+a_\mu}\right)+
J_\lambda\left(b_\mu,\fr{b_\mu}{a_\lambda+b_\mu}\right)-{}\\
{}-
J_\lambda\left(a_\mu,\fr{a_\mu}{a_\lambda+a_\mu}\right) -
J_\lambda\left(b_\mu,\fr{b_\mu}{b_\lambda+b_\mu}\right)- {}\\
{}-
J_\mu\left(b_\lambda, \fr{b_\lambda}{b_\lambda+b_\mu}\right) - 
J_\mu\left(a_\lambda, \fr{a_\lambda}{a_\lambda+a_\mu}\right),
\label{Mean}
\end{multline}
\textit{где величины $J_\lambda(d,b)$ и~$J_\mu(d,b)$ определены соотношением}~(\ref{Integral_J}).


\section{Численные результаты}


Основываясь на формуле~(\ref{Mean}), приведем табл.~1 и~2, содержащие частные 
значения (с~точностью до сотых) средней предельной надежности системы для 
некоторых наборов параметров.




{\small\frenchspacing
 {%\baselineskip=10.8pt
 \addcontentsline{toc}{section}{References}
 \begin{thebibliography}{9}


\bibitem{KS2006}
\Au{Королев В.\,Ю., Соколов И.\,А.}
Основы математической теории надежности модифицируемых систем.~--- 
М.: ИПИ РАН, 2006. 102~с.

\bibitem{K2016}
\Au{Кудрявцев А.\,А.}
Байесовские модели массового обслуживания и~надежности: априорные распределения 
с~компактным носителем~//
Информатика и~её применения, 2016. Т.~10. Вып.~1. С.~67--71.

\end{thebibliography}

 }
 }

\end{multicols}

\vspace*{-3pt}

\hfill{\small\textit{Поступила в~редакцию 29.03.16}}

\vspace*{8pt}

\newpage

\vspace*{-24pt}

%\hrule

%\vspace*{2pt}

%\hrule

%\vspace*{8pt}


\def\tit{BAYESIAN RECURRENT MODEL OF~RELIABILITY GROWTH: PARABOLIC DISTRIBUTION OF~PARAMETERS}


\def\titkol{Bayesian recurrent model of~reliability growth: Parabolic distribution of~parameters}

\def\aut{A.\,A.~Kudryavtsev and S.\,I.~Palionnaia}

\def\autkol{A.\,A.~Kudryavtsev and S.\,I.~Palionnaia}

\titel{\tit}{\aut}{\autkol}{\titkol}

\vspace*{-9pt}

\noindent
Department of Mathematical Statistics, Faculty of Computational Mathematics and 
Cybernetics, M.\,V.~Lomonosov Moscow State University, 1-52~Leninskiye Gory, GSP-1, 
Moscow 119991, Russian Federation

\def\leftfootline{\small{\textbf{\thepage}
\hfill INFORMATIKA I EE PRIMENENIYA~--- INFORMATICS AND
APPLICATIONS\ \ \ 2016\ \ \ volume~10\ \ \ issue\ 2}
}%
 \def\rightfootline{\small{INFORMATIKA I EE PRIMENENIYA~---
INFORMATICS AND APPLICATIONS\ \ \ 2016\ \ \ volume~10\ \ \ issue\ 2
\hfill \textbf{\thepage}}}

\vspace*{3pt}



\Abste{This work is devoted to the study of the parabolic distribution 
of parameters in the Bayesian recurrent model of reliability growth of 
complex modifiable information systems. In the reliability theory, the 
reliability of the system depends on the ratio of parameters which are 
interpreted as indexes of ``defectiveness'' and ``efficiency'' 
of the tool correcting the deficiencies in the system. In the framework of 
Bayesian models, it is assumed that only the information about the 
\textit{a~priori} distributions of the system's parameters is given. 
In this work, the average marginal system reliability is calculated for 
the \textit{a~priori} parabolic distribution 
of the parameters. The numerical results for the model examples are obtained.}

\KWE{modifiable information system; reliability theory; 
Bayesian approach; parabolic distribution}


\DOI{10.14357/19922264160209}

%\vspace*{-12pt}

%\Ack
%\noindent


%\vspace*{3pt}

  \begin{multicols}{2}

\renewcommand{\bibname}{\protect\rmfamily References}
%\renewcommand{\bibname}{\large\protect\rm References}

{\small\frenchspacing
 {%\baselineskip=10.8pt
 \addcontentsline{toc}{section}{References}
 \begin{thebibliography}{9}
\bibitem{1-kudr-1}
\Aue{Korolev, V.\,Y., and I.\,A.~Sokolov}. 2006. 
\textit{Osnovy ma\-te\-ma\-ti\-che\-skoy teorii nadezhnosti modifitsiruemykh sistem} 
[The fundamentals of mathematical reliability theory of modifiable systems]. Moscow: IPI RAN. 
102~p.
\bibitem{2-kudr-1}
\Aue{Kudryavtsev, A.\,A.} 2016. Bayesovskie modeli v~teorii massovogo obsluzhivaniya 
i~nadezhnosti: Apriornie raspredeleniya s~kompaktnym nositelem 
[Bayesian queueing and reliability models: \textit{A~priori} 
distributions with compact support]. 
\textit{Informatika i~ee Primeneniya}~--- \textit{Inform. Appl.} 10(1):67--71.

\end{thebibliography}

 }
 }

\end{multicols}

\vspace*{-3pt}

\hfill{\small\textit{Received March 29, 2016}}

\Contr

\noindent
\textbf{Kudryavtsev Alexey A.}\ (b.\ 1978)~---
Candidate of Science (PhD) in physics and mathematics, associate professor, 
Department of Mathematical Statistics, Faculty of Computational Mathematics 
and Cybernetics, M.\,V.~Lomonosov Moscow State University, 1-52~Leninskiye Gory, 
GSP-1, Moscow 119991, Russian Federation; nubigena@mail.ru

\vspace*{3pt}

 \noindent
 \textbf{Palionnaia Sofia I.}\ (b.\ 1995)~--- 
 student, Department of Mathematical Statistics, Faculty of Computational 
 Mathematics and Cybernetics, M.\,V.~Lomonosov Moscow State University, 1-52~Leninskiye 
 Gory, GSP-1, Moscow 119991, Russian Federation; sofiapalionnaya@gmail.com

 
\label{end\stat}


\renewcommand{\bibname}{\protect\rm Литература}