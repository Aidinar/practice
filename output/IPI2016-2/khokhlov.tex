\def\stat{khokhlov}

\def\tit{МНОГОМЕРНОЕ ДРОБНОЕ ДВИЖЕНИЕ ЛЕВИ И~ЕГО~ПРИЛОЖЕНИЯ$^*$}

\def\titkol{Многомерное дробное движение Леви и~его приложения}

\def\aut{Ю.\,С.~Хохлов$^1$}

\def\autkol{Ю.\,С.~Хохлов}

\titel{\tit}{\aut}{\autkol}{\titkol}

\index{Хохлов Ю.\,С.}
\index{Khokhlov Yu.\,S.}

{\renewcommand{\thefootnote}{\fnsymbol{footnote}} \footnotetext[1]
{Работа поддержана РНФ (проект 14-11-00364).}}


\renewcommand{\thefootnote}{\arabic{footnote}}
\footnotetext[1]{Московский государственный университет им.~М.\,В.~Ломоносова, \mbox{yskhokhlov@yandex.ru} }                             


\Abst{С начала 1990-х гг.\ было проведено большое число эмпирических исследований трафика реальных
телекоммуникационных систем. Было обнаружено, что он обладает рядом специфических 
свойств, отличающих его от обычного
голосового трафика, а~именно: он обладает свойствами самоподобия и~долговременной 
зависимости и~распределение
величины нагрузки, поступающей от одного источника, имеет тяжелые хвосты. Были 
построены новые модели трафика, которые
обладали указанными свойствами. Наиболее известные из них~--- дробное броуновское 
движение и
$\alpha$-устой\-чи\-вое движение Леви. Но каж\-дая из этих моделей обладает 
только частью из перечисленных выше свойств.
Были предприняты попытки построить более сложные модели, являющиеся комбинацией 
этих двух, в~частности, предложен
некоторый вариант одномерного дробного движения Леви.
%
В настоящей работе рассматривается многомерный аналог дробного движения Леви. Этот процесс представляет собой
    многомерное дробное броуновское движение со случайной заменой времени, в~качестве которой рассматривается
$\alpha$-устой\-чи\-вое движение Леви с~односторонними устойчивыми
распределениями. Изучены свойства этого процесса, показано, что он
является самоподобным и~имеет стационарные приращения. Показано
также, что координаты одномерных сечений этого процесса имеют
распределения, отличные от устойчивых. Но асимптотика хвостов этих
распределений в~точности такая же, как и~у~устойчивых распределений.
Далее эта модель использована для анализа неоднородного трафика, 
и~получена нижняя асимптотическая оценка для вероятности переполнения
хотя бы одного буфера при условии, что все буферы большие. Возможны
и~другие приложения.}


\KW{многомерное дробное броуновское движение; $\alpha$-устой\-чивый субординатор; 
самоподобные процессы; вероятность переполнения буфера}


\DOI{10.14357/19922264160212} 

%\vspace*{-4pt}

\vskip 10pt plus 9pt minus 6pt

\thispagestyle{headings}

\begin{multicols}{2}

\label{st\stat}

\section{Введение}

Во многих прикладных задачах часто встречается ситуация, когда процесс риска является
принципиально многомерным. Например, нагрузка на сервер может поступать от многих источников
по несколь\-ким каналам. При этом нагрузка по каж\-до\-му каналу формируется таким образом, что эти
потоки оказываются зависимыми. Страховые компании ведут свою деятельность по многим 
на\-прав\-ле\-ни\-ям,
каждое из которых формирует свой поток рисков, но они взаимосвязаны.
При формировании инвестиционного портфеля используются различные типы ценных бумаг, доходности
которых меняются со временем, при этом они зависят от некоторых общих факторов, определяющих
состояние рынка. Все эти примеры ведут к~многомерным процессам риска с~зависимыми компонентами.
Существует и~множество других примеров, возника\-ющих, например, в~физических задачах.

Многочисленные эмпирические исследования показали, что процессы риска в~описанных выше областях обладают
двумя важными свойствами, а~именно:  свойством долговременной за\-ви\-си\-мости 
и~тяжелыми хвостами распределений
вероятностей. Обычно при управлении процессами риска пользуются усредненными 
характеристиками процессов,\linebreak
которые при сильном агрегировании быстро ста\-билизируются. Это сильно упрощает задачи 
управ\-ле\-ния,
так как позволяет работать с~практически неслучай\-ными величинами. 
Но эмпирические исследования последних 20 лет
показали, что современные процессы риска в~широком диапазоне интервалов 
агрегирования не стабилизируются,
а~остаются (после некоторой нормировки) практически такими же. 
Подобное поведение случайного процесса
называют его самоподобием.

Сказанное выше приводит к~задаче построения многомерных моделей
риска, которые обладают тремя важными свойствами: самоподобием,
долговременной зависимостью и~тяжелыми хвостами распределений.

В одномерном случае этой проблематике посвящено большое число работ.
Обычно рассматривают самоподобные процессы со стационарными
приращениями. 

Типичными примерами таких процессов служат дробное
броуновское движение и~устойчивое движение Леви. Первый процесс
обладает свойством долговременной зависимости, но, как известно,
гауссовское распределение имеет быстро убывающий хвост. Напротив,
устойчивое движение Леви имеет независимые приращения, но устойчивые
распределения (с~показателем устойчивости меньше~2) есть типичный
пример распределений с~тяжелыми хвостами. В~определенном смысле эти
два процесса являются антиподами. В~то же время и~тот и~другой класс
процессов в~качестве частного случая содержит обычное броуновское
движение. 

Хотелось бы найти такую модель, которая объединяла бы
свойства этих двух.  Было предложено несколько вариантов решения
этой задачи. Одним из них является модель FARIMA, использующая
понятие дробного сдвига. Похожий вариант для случая непрерывного
времени используется в~так называемой модели линейного дробного
устойчивого движения. Существует также модель
ло\-га\-риф\-ми\-че\-ски-дроб\-но\-го устойчивого движения и~некоторые другие.
Наиболее близка к~настоящему исследованию работа~\cite{Las02},
которая использует подход Мандельброта при построении дробного
броуновского движения с~помощью некоторого стохастического интеграла
по обычному броуновскому движению. Построенный авторами процесс
является самоподобным и~имеет устойчивые распределения. Но его
использование в~задачах моделирования вызывает некоторые затруднения
в силу его сложной структуры. В~работе~\cite{Ni12} предложен новый
вариант одномерного дробного движения Леви, который представляет
собой подчиненный  процесс, где управляемый процесс есть дробное
броуновское движение, а~управляющий процесс есть одностороннее
устойчивое движение Леви с~показателем устойчивости меньше~1. Эта
модель была использована для получения нижней оценки вероятности
переполнения большого буфера.

В настоящей работе рассматривается аналогичная модель в~многомерном
случае. В~качестве исходного объекта берется многомерное дробное
броуновское движение. Подробное исследование этого процесса
проведено в~работах~\cite{Sto06, Amb12}. Далее
рассматривается подчиненный процесс, для которого управляемый
процесс есть многомерное дробное броуновское движение, а~управляющий
процесс (случайное время), как и~раньше, есть одностороннее
устойчивое движение Леви. Исследованы свойства такого многомерного
процесса. Далее эта модель применяется в~некоторой задаче из теории
телетрафика.

\section{Устойчивые распределения и~процессы Леви}

В предлагаемой модели важную роль играют устойчивые распределения 
и~процессы Леви. Далее приводятся некоторые известные
определения и~рассматриваются необходимые свойства таких распределений и~процессов.

\smallskip

\noindent
\textbf{Определение~1.} 
Случайный процесс $Y \hm= (Y(t), t\hm\geq 0)$ со значениями в~$R^d$ 
называется процессом Леви, если
\begin{enumerate}[(1)]
\item $Y(0) = 0$ п.~н.;
\item $Y$ имеет независимые приращения;
\item $Y$ имеет стационарные приращения, т.\,е.\ для любых $t\hm\geq 0$, $h\hm>0$ 
случайный вектор
$Y(t+h) \hm- Y(t)$ имеет распределение, не зависящее от~$t$.
\end{enumerate}

Очень часто из соображений регулярности требуют выполнения следующего свойства:
с вероятностью единица все траектории~$Y$ должны быть непрерывными справа и~иметь конечные пределы слева.
Это не является дополнительным ограничением, так как всегда можно построить реализацию процесса Леви с~таким
свойством.

Хорошо известно, что все конечномерные распределения процесса~$Y$ 
однозначно определяются по распределению
случайного вектора $Y(1)$, которое является безгранично делимым.

Одним из наиболее известных одномерных примеров процессов Леви служит процесс 
броуновского движения (или винеровский процесс).


\smallskip

\noindent
\textbf{Определение~2.} 
Процесс Леви $B \hm= (B(t), t\geq 0)$ со значениями в~$R^1$ 
называется процессом броуновского движения (BM~--- Brownian motion), если
для любых $t\hm\geq 0$, $h\hm>0$ приращение $B(t+h)\hm -B(t)$ 
имеет гауссовское распределение с~нулевым средним и~дисперсией $\sigma^2 h$.


\smallskip

Если $\sigma^2 =1$,  то имеет место стандартное броуновское движение. 
Нетрудно показать, что
$$
K(t,s) = \mathrm{Cov}\, (Y(t), Y(s)) = \sigma^2 \min (t,s)  \,.
$$


По определению броуновское движение имеет гауссовские распределения. 
В~силу центральной предельной теоремы
такие распределения получаются асимптотически для нормированных сумм 
независимых и~одинаково распределенных случайных величин с~конечной дисперсией.
В~случае бесконечных дисперсий приходим к~понятию устойчивого распределения.


\smallskip

\noindent
\textbf{Определение~3.} 
Говорят, что случайная величина~$Y$ имеет $\alpha$-устой\-чи\-вое распределение, 
если ее характеристическая функция имеет следующий вид:
\begin{multline*}
\varphi (\omega ) := E\left[ e^{i \omega X} \right] ={}\\
{}= \exp \{
i\mu\omega - \sigma |\omega |^{\alpha} [1 - i \beta \,\mathrm{sign}\, (\omega )
\theta (\omega , \alpha )] \},
\end{multline*}
где $0< \alpha \hm\leq 2$; $\sigma\hm\geq 0$;
$-1 \hm\leq \beta \hm\leq 1$; $\mu\hm\in R^1$;
$$
\theta (\omega , \alpha ) =
\begin{cases}
\tan \left( \fr{\alpha \pi }{2} \right)& \alpha \not= 1\,; \\
-\fr{2}{\pi} \ln |\omega |& \alpha =1\,.
\end{cases}
$$


%\smallskip


Параметр $\alpha$ называется {\it характеристическим показателем} 
и~определяет скорость убывания хвостов распределения;
$\sigma$ и~$\mu$ являются параметрами {\it масштаба } и~{\it сдвига} соответственно; 
$\beta$
называется {\it параметром асимметрии}. Если $\beta \hm=0$, то $X$ имеет 
симметричное относительно~$\mu$ распределение.
Если $0\hm<\alpha \hm<1$, $\mu \hm=0$ и~$\beta\hm =1$, то случайная величина~$X$ 
положительна с~вероятностью~1. В~дальнейшем будем говорить, что случайная
величина~$Y$ имеет стандартное $\alpha$-устой\-чи\-вое распределение, если 
$\mu \hm=0$ и~$\sigma \hm= 1$.

$\alpha$-устойчивое распределение является безгранично делимым. Оно порождает 
некоторый процесс Леви.

\smallskip

\noindent
\textbf{Определение~4.} 
Случайный процесс $L_{\alpha} \hm= (L_{\alpha} (t)$, $t\hm\geq 0)$ со значениями 
в~$R^1$ называется $\alpha$-устой\-чи\-вым движением Леви,
если это процесс Леви, для которого $L_{\alpha} (1)$ имеет заданное 
устойчивое распределение.


\smallskip

Если у распределения $L_{\alpha} (1)$ $0\hm<\alpha \hm<1$, $\beta \hm=1$, $\mu \hm=0$, 
то траектории процесса~$L_{\alpha}$ являются положительными и~неубывающими.
Такой процесс называется {\it $\alpha$-устой\-чи\-вым субординатором}.

Если $\alpha =2$, $\mu \hm=0$, то мы вновь возвращаемся к~процессу 
броуновского движения~$B$.

Между $\alpha$-устойчивыми движениями Леви с~различными~$\alpha$ справедливо 
следующее соотношение.

\smallskip

\noindent
\textbf{Теорема~1.}\
\textit{Если $(L_{\alpha_1 } (t), t\geq 0)$, $0\hm< \alpha_1 \hm\leq 2$, есть 
$\alpha_1$-устой\-чи\-вое движение Леви с~симметричными распределениями и
$(L_{\alpha_2 } (t), t\hm\geq 0)$, $0\hm<\alpha_2 \hm<1$, есть $\alpha_2$-устой\-чи\-вый 
субординатор, то случайный процесс
$Y \hm= (Y(t):= L_{\alpha_1 } (L_{\alpha_2} (t)), t\hm\geq 0)$
есть $\alpha_1 \alpha_2$-устой\-чи\-вое движение Леви с~симметричными распределениями}.


\smallskip

Эта теорема есть прямое следствие следующего результата 
В.\,М.~Золотарева~\cite[теорема~3.3.1.]{Zol83}:

\smallskip

\noindent
\textbf{Теорема~2.}\
\textit{Если $Y_1$ имеет симметричное $\alpha_1$-устой\-чи\-вое распределение, 
$0\hm<\alpha_1 \hm\leq 2$, $Y_2$ имеет одностороннее
$\alpha_2$-устой\-чи\-вое распределение, $0\hm<\alpha_2 \hm<1$, то случайная величина
$Y \hm= Y_1 Y_2^{1/\alpha_1}$ имеет симметричное $\alpha_1\alpha_2$-устой\-чи\-вое 
распределение}.


\smallskip

В частности, для $\alpha_1 \hm=2$ и~$0\hm<\alpha_2 \hm=\alpha/2 \hm<1$
получается следующий результат.

\smallskip

\noindent
\textbf{Теорема~3.}\
\textit{Если $B= (B(t), t\geq 0)$ есть броуновское движение,
$L_{\alpha/2} \hm= (L_{\alpha/2} (t) , t\hm\geq 0)$ есть $\alpha/2$-устой\-чи\-вый 
субординатор,
то $L_{\alpha} \hm= (L_{\alpha} (t) :=$\linebreak $:=\;B(L_{\alpha/2} (t)), t\hm\geq 0)$, 
$0\hm<\alpha \hm<2$, есть
$\alpha$-устой\-чи\-вое движение Леви с~симметричными распределениями}.


\smallskip

Рассмотрим теперь многомерные аналоги приведенных выше определений и~результатов.

\smallskip

\noindent
\textbf{Определение~5.}\
Процесс Леви $B = (B(t), t\geq 0)$ со значениями в~$R^d$ 
называется многомерным процессом броуновского движения (MBM~--- multivariate BM), если
для любых $t\hm\geq 0$, $h\hm>0$ приращение $B(t+h)\hm -B(t)$ имеет 
гауссовское распределение с~нулевым средним 
и~матрицей ковариаций $\Sigma h$, где $\Sigma$~--- некоторая 
положительно определенная мат\-рица.


\smallskip

Пусть $Y = (Y_1 , \ldots , Y_d )$ есть случайный вектор.

\smallskip

\noindent
\textbf{Определение~6.}\ Случайный вектор~$Y$ имеет многомерное $\alpha$-устой\-чи\-вое 
распределение
с параметром $\alpha\hm\in (0,2]$, если его характеристическая функция имеет следующий вид: для любого $\omega\in R^d$

\noindent
$$
\varphi_Y (\omega) = E(\exp (i (Y,\omega ))) = \exp (i (a,\omega ) - I(\omega ) )  \,,
$$
где
\begin{enumerate}[(1)]
\item если $\alpha =2$, то $a\hm\in R^d$~--- 
вектор средних, $I(\omega ) \hm= ({1}/{2}) (\Sigma\omega , \omega )$,
$\Sigma$~--- матрица ковариаций случайного вектора~$Y$;

\item если $0\hm<\alpha \hm<2$, то $a\hm\in R^d$ и
$$
I(\omega ) =    \int\limits_{S^{d-1}} |(\omega ,u)|^{\alpha} 
\theta_{\alpha} (\omega ,u) \Gamma (du)  \,,
$$
$\Gamma$~--- конечная мера на сфере $S^{d-1}$ и

\noindent
$$
\hspace*{-5mm}\theta_{\alpha} (\omega  , u ) =
 \begin{cases}
1 - i  \tan \left( \fr{\alpha \pi }{2} \right) \mathrm{sign}\, (\omega ,u),  &\! \alpha \not= 1 , \\
1 + \fr{2}{\pi} \ln |(\omega ,u) |  \mathrm{sign}\, (\omega ,u), &\! \alpha =1 .
\end{cases}
$$
\end{enumerate}

Такое распределение является безгранично
делимым и~порождает некоторый процесс Леви $Y\hm=(Y(t) , t\hm\geq 0)$ со значениями в~$R^d$. 
Будем называть его многомерным
$\alpha$-устой\-чи\-вым движением Леви. Для $\alpha \hm=2$ и~$a\hm=0$ получаем многомерное броуновское движение.

Важным частным случаем являются так называемые многомерные эллиптически 
контурированные устойчивые распределения, которые
кратко будем называть эллиптическими устойчивыми распределениями.
Их характеристическая функция имеет вид:
\begin{multline*}
\varphi_Y (\omega ) :={}\\
{}:=\; E(\exp (i (Y,\omega )) = \exp (i (a, \omega ) - 
(\Sigma\omega , \omega )^{\alpha /2} )  ,
\end{multline*}

\noindent
где $a\in R^d$, $\Sigma$~--- некоторая положительно определенная матрица (см., 
например,~\cite{No13}).



Справедлив следующий аналог теоремы~3.

\smallskip

\noindent
\textbf{Теорема~4.}\
\textit{Если $B= (B(t), t\geq 0)$ есть многомерное броуновское движение с~матрицей ковариаций $\Sigma$,
$L_{\alpha/2} \hm= (L_{\alpha/2} (t) , t\hm\geq 0)$ есть $\alpha/2$-устой\-чи\-вый субординатор,
то $L_{\alpha} \hm= (L_{\alpha} (t) := B(L_{\alpha/2} (t)), t\hm\geq 0)$, 
$0\hm<\alpha \hm<2$, есть
многомерное $\alpha$-устой\-чи\-вое движение Леви с~эллиптически 
контурированными распределениями.}

\section{Самоподобные процессы}


\smallskip

\noindent
\textbf{Определение~7.}\
Вещественный случайный процесс $Y \hm= (Y(t), t\hm\geq 0)$ называется самоподобным 
с~параметром Херста $H\hm\geq 0$, если он удовлетворяет
следующему условию:
$$
Y(t) \stackrel{d}{=} c^{-H} Y(ct)\,, \ \forall\, t\geq 0, \ \forall\, c>0,
$$
где $\stackrel{d}{=}$ обозначает равенство конечномерных распределений.


\smallskip

Двумя наиболее популярными примерами самоподобных процессов являются 
дробное броуновское движение (fBM~--- fractional BM) и~$\alpha$-устой\-чи\-вое движение \mbox{Леви}.

\smallskip

\noindent
\textbf{Определение~8.}\
Дробное броуновское движение с~параметром Херста $H$ есть гауссовский процесс
$(B_H (t), t\hm\geq 0)$ с~нулевыми средними и~ковариационной функцией
$$
K_H (t,s) = \fr{1}{2} \left[ |t|^{2H} + |s|^{2H} - |t-s|^{2H} \right] \, .
$$


\smallskip

При $H=1/2$ возвращаемся к~обычному броуновскому движению.

Определение $\alpha$-устой\-чи\-во\-го движения Леви было дано выше. Для него 
$H\hm=1/\alpha$.

Можно определить многомерный аналог самоподобного процесса.

\smallskip

\noindent
\textbf{Определение~9.}\
Случайный процесс $Y = (Y(t)\hm=(Y_1 (t), \ldots, Y_d (t) ) , t\hm\in R^1 )$ со значениями 
в~$R^d$ называется самоподобным
с параметром Херста $H \hm= (H_1 , \ldots , H_d )\hm\in (0,\infty )^d$, если он 
удовлетворяет следующему условию:
\begin{multline*}
Y(t) \stackrel{d}{=} c^{-H} Y(ct) = (c^{- H_1} Y(c t_1 ) , \ldots , 
c^{- H_d} Y(c t_d )), \\ \forall t\geq 0, \ \forall c>0\,.
\end{multline*}
%где $\stackrel{d}{=}$ обозначает равенство конечномерных распределений.

\smallskip

Возможны и~другие более общие определения. Дополнительную информацию об устойчивых и~самоподобных процессах
можно найти в~книгах~\cite{Sam94, Emb02}.

\section{Многомерное дробное броуновское движение}

Основой модели, которая предлагается в~данной статье, является
многомерное дробное броуновское движение. Далее приводится только
его определение. Более подробную информацию можно найти в~[3,~4].

\smallskip

\noindent
\textbf{Определение~10.}\
Многомерным дробным броуновским движением (MFBM~--- multivariate fBM) с~параметром Херста $H\hm\in (0,1)^d$ 
называется $d$-мер\-ный случайный процесс
$Y\hm = (Y_1 (t) , \ldots Y_d (t) , t\hm\in R^1 )$, который обладает следующими свойствами:
\begin{enumerate}[(1)]
\item  $Y$ есть гауссовский процесс;

\item  $Y$ является самоподобным с~параметром Херс\-та~$H$;

\item  $Y$ имеет однородные приращения.
\end{enumerate}


В данной работе наиболее интересен случай, когда $1/2 \hm< H_p \hm<1$ для 
всех $p: 1\hm\leq p \hm\leq d$. В~этой ситуации при дополнительном ограничении, что 
процесс~$Y$ является обратимым по времени, он имеет нулевые средние и~следующие ковариационные функции:
\begin{multline*}
E(Y_p (s) Y_q (t)) ={}\\
{}= \fr{\sigma_{pq}}{2}
\left[ |s|^{H_p + H_q} + |t|^{H_p + H_q} - |t-s|^{H_p + H_q} \right]  \,,
\end{multline*}
где $\Sigma = (\sigma_{pq} )$ есть некоторая положительно определенная матрица. 
В~частности,
\begin{equation*}
D(Y_p (t)) = E\left(|Y_p (t)|^2 \right) = \sigma_p^2 |t|^{2H_p}  \,.
\end{equation*}

Далее многомерное дробное броуновское движение с~параметром~$H$ будем обозначать, 
как и~в~одномерном случае,~$B_H$.

\section{Многомерное дробное движение Леви}

В этом разделе предлагается  некоторый вариант многомерного дробного движения 
Леви. Одномерный аналог этого процесса был определен
в~работе~\cite{Ni12}.

Пусть $(B_H (t), t\hm\in R^1 )$ есть многомерное дробное броуновское
движение с~параметром Херс\-та~$H$ и~мат\-ри\-цей ковариаций~$\Sigma$;
$(L_{\alpha}^1 (t), t\hm\geq 0)$ и~$(L_{\alpha}^2 (t), t\hm\geq 0)$~---
стандартные $\alpha$-устой\-чи\-вые субординаторы, $0\hm<\alpha \hm< 1$;
процессы~$B_H$, $L_{\alpha}^1$ и~$L_{\alpha}^2$ независимы.

\smallskip

\noindent
\textbf{Определение~11.}\
Многомерным дробным движением Леви называется случайный процесс $X\hm = (X(t) ,
 t\hm\in R^1 )$ со значениями в~$R^d$ такой, что
$$
X(t) := \begin{cases}
B_H (L_{\alpha}^1 (t)), & t\geq 0 \,; \\
B_H (L_{\alpha}^2 (-t)), &  t < 0  \,, \\
\end{cases}
$$


Покажем, что предложенный процесс обладает свойством самоподобия, а~именно имеет место следующий результат.

\smallskip

\noindent
\textbf{Теорема 5.}\
\textit{Построенный выше случайный процесс является самоподобным 
с~параметром Херста~$H/\alpha$}.

\smallskip

\noindent
Д\,о\,к\,а\,з\,а\,т\,е\,л\,ь\,с\,т\,в\,о\,.\ \
Процессы $(L_{\alpha}^k (t), t\hm\geq 0)$, $k\hm=1,2$, являются $\alpha$-устой\-чи\-вы\-ми 
и~самоподобными с~параметром Херста $1/\alpha$.
В~силу этого для любого $c\hm>0$  имеем:
$$
(L_{\alpha}^k (ct), t\geq 0) \stackrel{d}{=}
(c^{1/\alpha} L_{\alpha}^k (t), t\geq 0) .
$$
Тогда
\begin{multline*}
(X(ct) , t\in R^1 ) = B_H (\pm L_{\alpha}^k (c|t|), t\in R^1 ) \stackrel{d}{=}{}\\
{}\stackrel{d}{=}
(B_H (\pm c^{1/\alpha} L_{\alpha}^k (|t|) , t\in R^1 )  \,.
\end{multline*}
Используя самоподобие процесса~$B_H$ для фиксированного
$\tau \hm= L_{\alpha}^k (|t|)$, имеем для любого $a\hm>0$
$$
(B_H (a\tau ), \tau\geq 0) \stackrel{d}{=} (a^H  B_H (\tau),
\tau\geq 0)\,,
$$
или
$$
(B_H (\pm c^{1/\alpha} \tau ), \tau\geq 0) \stackrel{d}{=}
(c^{H/\alpha} B_H (\pm \tau), \tau\geq 0)  \,.
$$
Применяя формулу полной вероятности, получаем нужный результат.

Используя последнюю теорему, можно получить следующий полезный 
для дальнейших приложений результат.

\smallskip

\noindent
\textbf{Следствие~1.}\
Для любого $t\hm>0$
$$
X(t) \stackrel{d}{=} ((L_{\alpha}^1 (t))^{H_1}  Y_1 , \ldots , (L_{\alpha}^1 (t))^{H_d}  Y_d ) ,
$$
где случайный вектор $Y \hm= (Y_1 , \ldots , Y_d )$ имеет многомерное 
нормальное распределение со средним ноль и~матрицей ковариаций~$\Sigma$,
причем $L_{\alpha}^1 (t)$ и~$Y$ независимы.


\smallskip


Этот результат легко следует из определения многомерного дробного 
броуновского движения и~независимости~$B_H$ и~$L_{\alpha}$.

\smallskip

\noindent
\textbf{Следствие~2.}\
 Если $B$ есть многомерное дробное броуновское движение с~матрицей ковариаций~$\Sigma$, 
 то
$H_1 \hm= \cdots \hm= H_d \hm= 1/2$ и~для любого $t\hm>0$ случайный вектор
$$
X(t) \stackrel{d}{=} (L_{\alpha}^1 (t))^{1/2} (Y_1 , \ldots , Y_d )
$$
имеет многомерное $\alpha$-устой\-чи\-вое эллиптически контурированное распределение 
с~матрицей ковариаций~$\Sigma$.


\smallskip

Этот результат есть прямое следствие теоремы~4.

В случае дробного броуновского движения случайный вектор $X(t)$ имеет распределение, отличное от устойчивого, если все $H_p >1/2$. Для доказательства
достаточно рассмотреть одну из компонент этого вектора. Как отмечалось выше, случайная величина $(L_{\alpha} (t) )^{1/2} B(1)$ имеет симметричное
устойчивое распределение с~характеристическим показателем~$\alpha/2$. Но случайные величины $(L_{\alpha} (t) )^{1/2} B(1)$ и~$(L_{\alpha} (t) )^{H_1 } B(1)$
имеют разные распределения, если $H_1 \not= 1/2$.


\smallskip

\noindent
\textbf{Замечание~1.}\ 
Параметр Херста $H/\alpha$ определенного выше процесса~$X$ может быть любым положительным числом. Будем предполагать,
что $1/2 \hm< H_k /\alpha \hm<1$, $k\hm= 1,\ldots ,d$. 
В~этом случае рассматриваемый процесс имеет конечные математические ожидания 
и~обладает свойством долговременной зависимости.


\smallskip




\noindent
\textbf{Теорема 6.}\
\textit{Определенный выше случайный процесс~$X$ имеет однородные по времени приращения.}


\smallskip

\noindent
Д\,о\,к\,а\,з\,а\,т\,е\,л\,ь\,с\,т\,в\,о\,.\ \ 
Хорошо известно, что дробное броуновское движение~$B_H$ имеет однородные приращения. 
Более того, для любых $0\hm < t_1 \hm< t_2 $
$$
B_H \left(t_2 \right) - B_H \left(t_1 \right) \stackrel{d}{=} B_H \left(t_2 - t_1 \right)\,. 
$$
Тогда для любых $t\hm\geq 0$, $h\hm>0$ и~фиксированных $L_{\alpha}^k (t+h) \hm= 
t_2$, $L_{\alpha}^k (t) \hm= t_1$ имеем:
$$
B_H (L_{\alpha}^k (t+h)) - B_H (L_{\alpha}^k (t)) \stackrel{d}{=}
B_H (L_{\alpha}^k (t+h) -  L_{\alpha}^k (t))\,.
$$
В силу формулы полной вероятности имеем то же самое и~для случайных моментов времени.
Процесс $L_{\alpha}^k (t)$ также имеет однородные приращения. В~итоге получаем:
\begin{multline*}
B_H \left(L_{\alpha}^k (t+h)\right) - B_H \left(L_{\alpha}^k (t)\right) \stackrel{d}{=}{}\\
{}\stackrel{d}{=}
B_H \left(L_{\alpha}^k (t+h) -  L_{\alpha}^k (t)\right)
\stackrel{d}{=}
B_H \left(L_{\alpha}^k (h) \right)  \,.
\end{multline*}


\section{Асимптотика хвостов распределений многомерного дробного движения Леви}

В этом разделе изучается поведение хвостов одномерных сечений 
дробного движения Леви, построенного выше.


В силу свойства самоподобия достаточно рассмотреть только распределение 
случайного вектора~$X(1)$. Рассмотрим случайную величину
$Z_{\alpha} :=$\linebreak $:=\;L_{\alpha}^1$. Без ограничения общности можно считать, 
что она имеет стандартное одностороннее
$\alpha$-устой\-чи\-вое распределение. В~силу следствия~2 имеем:
$$
X(1) \stackrel{d}{=} ((Z_{\alpha})^{H_1}  Y_1 , \ldots , (Z_{\alpha})^{H_d}  Y_d ) \, ,
$$
где случайный вектор $Y \hm= (Y_1 , \ldots , Y_d )$ имеет многомерное нормальное распределение со средним ноль и~матрицей ковариаций $\Sigma$,
причем~$Z_{\alpha}$ и~$Y$ независимы.
Покажем, что этот вектор имеет распределения, отличные от устойчивых. Для этого достаточно проверить, что это верно для
отдельно взятой координаты.

Предположим противное: первая координата $V:= (Z_{\alpha} )^{H_1} Y_1$ имеет симметричное устойчивое распределение
с~показателем $0\hm<\alpha_1 \hm<2$. Тогда, как показано выше, существует такая независимая от $Y_1$ случайная величина $S$, которая
имеет одностороннее устойчивое распределение с~показателем $\alpha_1/2$, что
$$
V \stackrel{d}{=} S^{1/2} Y_1 \, .
$$
Отсюда получаем:
$$
(Z_{\alpha} )^{H_1} Y_1 \stackrel{d}{=} S^{1/2} Y_1\,  .
$$
Нормальное распределение обладает свойством идентифицируемости для масштабных смесей. 
В~силу этого получаем
$$
(Z_{\alpha} )^{H_1} \stackrel{d}{=} S^{1/2},
$$
или
$$
(Z_{\alpha} )^{2 H_1} \stackrel{d}{=} S  \,.
$$
Но, как хорошо известно, если случайная величина~$Z_{\alpha}$ имеет устойчивое распределение, то случайная 
величина~$S$ имеет распределение, отличное от устойчивого. Это противоречие 
доказывает нужное утверждение.

Тем не менее можно показать, что хвосты распределений координат
вектора $X(1)$ ведут себя в~точности так же, как хвосты устойчивых
распределений. Для доказательства этого свойства потребуется
результат, известный как теорема Бреймана~\cite{Br65}.

\smallskip

\noindent
\textbf{Теорема 7.}\
\textit{Пусть $X$ и~$Y$ есть независимые неотрицательные случайные величины~и
$$
\bar{F} (x) := P(X>x) = x^{-\alpha}  L(x) \,, \enskip x\to\infty \,,
$$
где $\alpha\hm >0$, $L$~--- медленно меняющаяся функция, 
$E(Y^{\alpha + \varepsilon} ) \hm< \infty$ для некоторого
$\varepsilon \hm> 0$. Тогда при больших} $x\hm>0$
$$
\bar{H} (x) := P(XY >x) \sim E(Y^{\alpha} ) \bar{F} (x)  \,.
$$


Применим этот результат к~распределению $k$-й координаты вектора $X(1)$.
Если $Z_{\alpha}$ имеет стандартное $\alpha$-устой\-чи\-вое распределение, то для больших 
$x\hm>0$
$$
 P(Z_{\alpha} > x ) \sim C(\alpha ) x^{-\alpha}  \,,
$$
где
$$
C(\alpha ) = \fr{\sin (\pi\alpha )}{\pi} \Gamma (\alpha )
$$
(см.~\cite[теорема~2.4.1]{IL65}). Тогда, используя теорему
Бреймана, для больших $x\hm>0$ получаем:
\begin{multline*}
P((Z_{\alpha} )^{H_k} Y_k >x) = \fr{1}{2} P\left( (Z_{\alpha} )^{H_k} |Y_k | >x \right) =
{}\\
{}=
\fr{1}{2} P\left(Z_{\alpha}  |Y_k |^{1/H_k} >x^{1/H_k} \right)\sim{}\\
{}
\sim \fr{1}{2} E\left(|Y_k |^{\alpha/H_k} \right)   
P\left( Z_{\alpha}  >x^{1/H_k} \right) \sim{}\\
{}\sim
\fr{1}{2} C(\alpha ) E\left(|Y_k |^{\alpha/H_k} \right)  x^{-\alpha/H_k } ={}\\
{}=
\fr{1}{2} C(\alpha ) E\left(|Y_k |^{1/H_k^1} \right)  x^{- 1/H_k^1 }  \,.
\end{multline*}


\section{Приложение к~моделированию телетрафика}

В этом разделе построенный выше процесс применяется для
моделирования динамики нагрузки сервера, поступающей по нескольким
каналам, и~делается  нижняя оценка для вероятности переполнения хотя
бы одного из буферов при больших размерах всех буферов.

Пусть имеется система массового обслуживания с~одним сервером, на которую подается нагрузка по нескольким каналам, которые, вообще говоря, зависимы.
Величина нагрузки, поступившей по $k$-му каналу, определяется по правилу:
$$
A_k (t) := m_k t + \left(\sigma_k m_k \right)^{1/\beta_k}  X_k (t), \enskip
k=1,\ldots , d\,.
$$
Векторный случайный процесс $X(t) \hm= (X_1 (t) , \ldots , X_d (t))$
есть дробное движение Леви с~параметром Херста $H^1 \hm= (H_1^1 ,
\ldots , H_d^1 ) := H/\alpha \hm= (H_1 /\alpha , \ldots , H_d /\alpha
)$ и~матрицей $\Sigma \hm= (\sigma_{pq} )$, определенное выше,
$\sigma_k^2 \hm=1$, $\beta_k \hm=  \alpha/H_k $, $k\hm=1, \ldots , d$. Если
обслуживание пакетов осуществляется путем случайного выбора канала 
с~некоторой ве\-ро\-ят\-ностью, то естественно предположить, что скорость
обслуживания нагрузки из $k$-го канала постоянна и~равна~$r_k$.
Чтобы обеспечить устойчивость функционирования системы, предположим,
как обычно, что $r_k\hm > m_k$, $k\hm=1, \ldots , d$. Тогда величина
загрузки $k$-го буфера в~момент времени $t\hm\in R^1$ можно записать в~виде:
$$
Q_k (t,r_k ) = \sup\limits_{s\leq t} (A_k (t) - A_k (s) - r_k 
(t-s))_{+}\,.
$$



В силу теоремы~6 процессы $Q_k \hm= (Q_k (t,r), t\hm\in R^1)$ являются стационарными. 
Пусть~$b_k$ есть размер $k$-го буфера.

Обозначим $b=(b_1 , \ldots , b_d )$. Найдем оценку для следующей вероятности 
переполнения хотя бы одного из буферов:
\begin{multline*}
\varepsilon (b) = P\left(\bigcup\limits_{k=1}^d \left(Q_k (0, r) > b_k \right)\right) ={}\\
{}=
P\left(\bigcup\limits_{k=1}^d \left(\sup\limits_{\tau \geq 0} (A_k (\tau ) - 
r_k  \tau ) >b_k \right)\right)\, .
\end{multline*}
Далее получим нижнюю границу для этой вероятности при больших~$b_k$,
используя технику, развитую в~работах~[1, 10]. Аналогичный результат
в одномерном случае был получен ранее в~работе~\cite{Ni12}.


Используя определение входящего потока и~свойство самоподобия, получаем:
\begin{multline*}
\varepsilon (b) = P\left(\bigcup\limits_{k=1}^d
\left(\sup\limits_{\tau \geq 0} \left(A_k (\tau ) - r_k  \tau \right) >b_k
\right)\right) \geq{}
\\
{}\geq \sup\limits_{\tau \geq 0} P\left(\bigcup\limits_{k=1}^d \left(
\left(A_k (\tau ) - r_k  \tau \right) >b_k \right)\right)= {}
\\
{}= \sup\limits_{\tau\geq 0} P\left(\bigcup\limits_{k=1}^d \left( m_k
\tau +\left(\sigma_k m_k \right)^{1/\beta_k } X_k (\tau ) - r_k \tau  >{}\right.\right.\\
\left.\left.{}> b_k
\vphantom{\left(\sigma_k m_k \right)^{1/\beta_k }}
\right) 
\vphantom{\bigcup\limits_{k=1}^d}
\right) =
\\
{}= \sup\limits_{\tau\geq 0} P\left(\bigcup\limits_{k=1}^d \left( m_k
\tau +\left(\sigma_k m_k \tau \right)^{1/\beta_k } X_k (1) - r_k \tau  >{}\right.\right.\\ 
\hspace*{-7pt}\left.\left.{}>b_k
\right)\right) 
= \sup\limits_{\tau\geq 0} P\left(\bigcup\limits_{k=1}^d \left(  X_k
(1) > \fr{b_k + (r_k - m_k )\tau}{(\sigma_k m_k \tau )^{1/\beta_k
} } \right)\!\right).\hspace*{-7.08pt}
\end{multline*}
Обозначим
$$
f_k (\tau ) =  \fr{b_k + (r_k - m_k )\tau}{(\sigma_k m_k \tau )^{1/\beta_k } } \,.
$$
Используя следствие~1, получаем:
\begin{multline*}
\varepsilon (b) \geq \sup\limits_{\tau\geq 0} P\left(
\bigcup\limits_{k=1}^d \left(  
\left( L_{\alpha}^1 (1) \right)^{H_k}  Y_k  > f_k (\tau ) \right)\right) \geq{}
\\
\hspace*{-4pt}{}\geq \sup\limits_{\tau\geq 0} P\left(\bigcup\limits_{k=1}^d \left(  
\vphantom{\bigcap\limits_{k=1}^d}
\left( L_{\alpha}^1 (1) \right)^{H_k}  Y_k  > f_k (\tau ) , \right.\right.
\end{multline*}

\noindent
\begin{multline*}
\left.\left.\bigcap\limits_{k=1}^d (Y_k > 1) \right)\right) \geq {}\\
{}\geq
\sup\limits_{\tau\geq 0} P\left(\bigcup\limits_{k=1}^d \left(  
\left( L_{\alpha}^1 (1) \right)^{H_k}  > f_k (\tau ) , \right.\right.\\
\left.\left.\bigcap\limits_{k=1}^d \left(Y_k > 1\right) \right)\right) \geq{}
\\
{}\geq \sup\limits_{\tau\geq 0} P\left(\bigcup\limits_{k=1}^d \left( 
\left ( L_{\alpha}^1 (1) \right)^{H_k}  > f_k (\tau ) \right)\right)\times{}
\\
{}\times
P\left( \bigcap\limits_{k=1}^d (Y_k > 1)  \right) ={}
\\
{}= C(\Sigma )  \sup\limits_{\tau\geq 0} P\left(\bigcup\limits_{k=1}^d \left(   
L_{\alpha}^1 (1)   > [f_k (\tau ) ]^{1/H_k }  \right)\right) ={}\\
{}= 
C(\Sigma ) \sup\limits_{\tau\geq 0} P\left( L_{\alpha}^1 (1)
> \min\limits_k [f_k (\tau ) ]^{1/H_k }  \right).
\end{multline*}
Последняя вероятность есть невозрастающая функция от $f_k (\tau )$,
которая в~точке
$$
\tau_k = \fr{b_k H_k^1}{(1-H_k^1 )(r_k - m_k )}
$$
принимает наименьшее значение, равное
$$
d_k := f_k \left(\tau_k \right) 
\fr{(r_k - m_k )^{H_k^1} (1-H_k^1 )^{-(1-H_k^1 )} }
{(\sigma_k m_k H_k^1 )^{H_k^1} } \, b_k^{1-H_k^1} \,.
$$
Так как
\begin{multline*}
\sup\limits_{\tau\geq 0} \min\limits_k \left[f_k (\tau )\right]^{1/H_k} \leq
\min\limits_k \sup\limits_{\tau\geq 0} 
\left [f_k (\tau )\right]^{1/H_k} ={}\\
{}= \min\limits_k \left[d_k \right]^{1/H_k } \,,
\end{multline*}
то получаем
$$
\varepsilon (b) \geq C(\Sigma )  P\left( L_{\alpha}^1 (1)   > \min\limits_k [d_k ]^{1/H_k }  \right)  .
$$



Если $L_{\alpha}^1 (1)$ имеет стандартное $\alpha$-устой\-чи\-вое распределение, 
то для больших $x\hm>0$
$$
 P(L_{\alpha}^1 (1) > x ) \sim C(\alpha ) x^{-\alpha} \,,
$$
где
$$
C(\alpha ) = \fr{\sin (\pi\alpha )}{\pi} \Gamma (\alpha ) 
$$
(см.\ \cite[теорема~2.4.1]{IL65}.) Отсюда для больших~$b_k$
$$
\varepsilon (b) \geq C(\alpha , H, \Sigma )  \max\limits_k \left[ \sigma_k 
\fr{m_k}{r_k - m_k} b_k^{-(1-H_k^1 )/H_k^1 } \right] .
$$
Сформулируем окончательный результат в~виде следующей теоремы.

\smallskip

\noindent
\textbf{Теорема~8.}\
\textit{В рамках описанной выше модели асимптотическая нижняя граница для вероятности переполнения хотя бы одного из буферов
имеет следующий вид}:
$$
\varepsilon (b) \geq C(\alpha , H, \Sigma ) 
\max\limits_k \left[ \sigma_k \frac{m_k}{r_k - m_k} 
b_k^{-(1-H_k^1 )/H_k^1 } \right] ,
$$
\textit{если} $b_k\hm\to\infty$, $1\hm\leq k\hm\leq d$.

{\small\frenchspacing
 {%\baselineskip=10.8pt
 \addcontentsline{toc}{section}{References}
 \begin{thebibliography}{99}
\bibitem{Las02}
\Au{Laskin N., Lambadaris~I., Harmantzis~F.\,C., Devetsikiotis~M.}
Fractional Levy motion and its application to network traffic
modeling~// Computor Networks, 2002. Vol.~40. P.~363--375.

\bibitem{Ni12}
\Au{De Nikola C., Khokhlov~Yu.\,S., Pagano~M., Sidorova~O.\,I.}
Fractional Levy motion with dependent increments and its application
to network traffic modeling~// Информатика и~её применения, 2012.
Т.~6. Вып.~3. С.~58--63.


\bibitem{Sto06} %3
\Au{Stoev S., Taqqu M.} How rich is the class of multifractional
brownian motions?~// Stochastic Processes Their Applications,
2006. Vol.~116. P.~200--221.


\bibitem{Amb12} %4
\Au{Amblard P.\,O., Coeurjolly~J.\,F., Lavancier~F., Philippe~A.}
Basic properties of the multivariate fractional Brownian motion~//
Bull. Society Mathematique de France, Seminaires et Congres,
2012. Vol.~28. P.~65--87.



\bibitem{Zol83} %5
\Au{Золотарев B.\,М.} Одномерные устойчивые распределения.~--- М.:
Наука, 1983. 304~с.

\bibitem{No13}
\Au{Nolan J.\,P.}  Multivariate elliptically contoured stable
distributions: Theory and estimation~//
 Comput. Stat., 2013. Vol.~28. No.\,5. P.~2067--2089.

\bibitem{Sam94}
\Au{Samorodnitsky G., Taqqu~M.\,S.} Stable non-Gaussian random
processes.~--- London: Chapman \& Hall, 1994. 632~p.

\bibitem{Emb02}
\Au{Embrechts~P., Maejima~M.} Selfsimilar process.~--- Princeton,
NJ, USA: Princeton University Press, 2002. 111~p.

\bibitem{Br65}
\Au{Breiman L.} On some limit theorems similar to the arc-sin law~// 
Теория вероятн. и~ее примен., 1965. Т.~10. Вып.~2. С.~351--359.

\bibitem{Nor94} %10
\Au{Norros I.} A~storage model with self-similar input~// Queuing
Syst., 1994. Vol.~16. P.~387--396.

\bibitem{IL65}
\Au{Ибрагимов И.\,А., Линник~Ю.\,В.} Независимые и~стационарно
связанные величины.~--- М.: Наука, 1965.  524~с.
\end{thebibliography}

 }
 }

\end{multicols}

\vspace*{-3pt}

\hfill{\small\textit{Поступила в~редакцию 01.12.15}}

\vspace*{8pt}

%\newpage

%\vspace*{-24pt}

\hrule

\vspace*{2pt}

\hrule

%\vspace*{8pt}



\def\tit{MULTIVARIATE FRACTIONAL LEVY MOTION AND~ITS~APPLICATIONS}

\def\titkol{Multivariate fractional Levy motion and its applications}

\def\aut{Yu.\,S.~Khokhlov}

\def\autkol{Yu.\,S.~Khokhlov}

\titel{\tit}{\aut}{\autkol}{\titkol}

\vspace*{-9pt}

\noindent
Department of Mathematical 
Statistics, Faculty of Computational Mathematics and Cybernetics, 
M.\,V.~Lomonosov Moscow State University, 1-52~Leninskiye Gory, GSP-1, Moscow 119991, 
Russian Federation


\def\leftfootline{\small{\textbf{\thepage}
\hfill INFORMATIKA I EE PRIMENENIYA~--- INFORMATICS AND
APPLICATIONS\ \ \ 2016\ \ \ volume~10\ \ \ issue\ 2}
}%
 \def\rightfootline{\small{INFORMATIKA I EE PRIMENENIYA~---
INFORMATICS AND APPLICATIONS\ \ \ 2016\ \ \ volume~10\ \ \ issue\ 2
\hfill \textbf{\thepage}}}

\vspace*{3pt}



\Abste{Since the beginning of the 1990s, many empirical studies of real telecoomunication systems traffic have been conducted. 
It was found that traffic has some specific properties, which are different from 
common voice traffic, namely, it has the properties of self-similarity and long-range dependence and the distribution of loading size from 
one source has heavy tails. Some new models have been constructed,
 where these features were captured. Brownian fractional 
motion and $\alpha$-stable Levy motion are the well-known examples. 
But both of these models do not have all of the above properties. 
More complicated models have been proposed using some combination of these ones. 
In particular, the authors have proposed 
a~variant of univariate fractional Levy motion. 
%
This paper considers a~multivariate analog of fractional Levy motion. This process is multivariate fractional 
Brownian motion with random change of time, where random change of time is Levy motion with one-sided stable 
distributions. The properties of this process are investigated 
and it is proven that it is self-similar and has stationary 
increments. Next, it is shown that the coordinates of one-dimensional sections of this process have the 
distributions which 
are not stable. But asymptotic of tails for these distributions is the same 
as for the stable ones. 
This model is applied 
to analyze heterogeneous traffic and to get a~lower asymptotic bound of the probability 
of overflow of at least one 
buffer. There are other possible applications.}


\KWE{fractional Brownian motion; $\alpha$-stable subordinator; self-similar processes; buffer overflow probability}

\DOI{10.14357/19922264160212}

\vspace*{-12pt}

\Ack
\noindent
The work was supported by the Russian Science Foundation (project 14-11-00364).


%\vspace*{3pt}

  \begin{multicols}{2}

\renewcommand{\bibname}{\protect\rmfamily References}
%\renewcommand{\bibname}{\large\protect\rm References}

{\small\frenchspacing
 {%\baselineskip=10.8pt
 \addcontentsline{toc}{section}{References}
 \begin{thebibliography}{99}
\bibitem{Las02-1}
\Aue{Laskin, N., I.~Lambadaris, F.\,C.~Harmantzis, and M.~Devetsikiotis}. 
2002. Fractional Levy motion and its application 
to network traffic modeling.  \textit{Computor Networks} 40:363--375. 

\bibitem{Ni12-1}
\Aue{De Nikola, C.,  Y.\,S.~Khokhlov, M.~Pagano, and O.\,I.~Sidorova}.  
2012. Fractional Levy 
motion with dependent increments and its application to network traffic modeling.
\textit{Informatika i~ee Primeneniya}~--- \textit{Inform. Appl.} 6(3):58--63.  


\bibitem{Sto06-1}
\Aue{Stoev, S., and M.~Taqqu}. 2006. How rich is the class of multifractional brownian
motions?  \textit{Stochastic Processes Their Applications}  116:200--221.

\bibitem{Amb12-1}
\Aue{Amblard, P.\,O., J.\,F.~Coeurjolly, F.~Lavancier, and A.~Philippe}.  2012. 
Basic properties of the multivariate fractional Brownian motion. 
\textit{Bull. Society Mathematique de France, Seminaires et Congres}. 28:65--87. 




\bibitem{Zol83-1} %%
\Aue{Zolotarev, V.\,M.} 1986. \textit{One-dimensional stable distributions}.  
Translations of Mathematical Monographs. AMS. Vol.~65. 284~p. 


\bibitem{No13-1}
\Aue{Nolan, J.\,P.} 2013.  Multivariate elliptically contoured stable distributions: 
Theory and estimation.   
\textit{Comput. Stat.} 28(5):2067--2089. 

\bibitem{Sam94-1}
\Aue{Samorodnitsky, G., and M.\,S.~Taqqu}. 1994. 
\textit{Stable non-Gaussian random processes}. 
Chapman \& Hall. 632~p.  

\bibitem{Emb02-1}
\Aue{Embrechts,~P., and M.~Maejima}. 2002.  \textit{Selfsimilar process}. 
Prinston University Press. 111~p. 

\bibitem{Br65-1}
\Aue{Breiman, L.} 1965.
On some limit theorems similar to the arc-sin law. 
\textit{Teoriya Veroyatnostei i Primeneniya} 
[Theory of Probability and its Applications] 10(2):351--359. 

\bibitem{Nor94-1} %10
\Aue{Norros, I.}  1994. 
A~storage model with self-similar input. \textit{Queuing Syst.} 16:387--396. 

\bibitem{IL71-1}
\Aue{Ibragimov, I.\,A., and Yu.\,V.~Linnik}. 1971.  
\textit{Independent and stationary sequences of 
random variables}.  Wolters-Noordhoff, Gronengen. 443~p. 
  \end{thebibliography}

 }
 }

\end{multicols}

\vspace*{-3pt}

\hfill{\small\textit{Received December 1, 2015}}


\Contrl

\noindent
\textbf{Khokhlov Yury S.} (b.\ 1952)~---
Doctor of Science in physics and mathematics, professor, Department of Mathematical 
Statistics, Faculty of Computational Mathematics and Cybernetics, 
M.\,V.~Lomonosov Moscow State University, 1-52~Leninskiye Gory, GSP-1, Moscow 119991, 
Russian Federation; \mbox{yskhokhlov@yandex.ru}


\label{end\stat}


\renewcommand{\bibname}{\protect\rm Литература}