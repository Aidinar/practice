\def\stat{agalarov}

\def\tit{ОБ ОПТИМАЛЬНОМ ПОРОГОВОМ ЗНАЧЕНИИ ДЛИНЫ ОЧЕРЕДИ В~ОДНОЙ ЗАДАЧЕ 
МАКСИМИЗАЦИИ ДОХОДА СИСТЕМЫ МАССОВОГО ОБСЛУЖИВАНИЯ ТИПА $M/G/1$$^*$}

\def\titkol{Об оптимальном пороговом значении длины очереди в~одной задаче 
максимизации дохода СМО типа $M/G/1$}

\def\aut{Я.\,М.~Агаларов$^1$, М.\,Я.~Агаларов$^2$, В.\,С.~Шоргин$^3$}

\def\autkol{Я.\,М.~Агаларов, М.\,Я.~Агаларов, В.\,С.~Шоргин}

\titel{\tit}{\aut}{\autkol}{\titkol}

\index{Агаларов Я.\,М.}
\index{Агаларов М.\,Я.}
\index{Шоргин В.\,С.}
\index{Agalarov Ya.\,M.}
\index{Agalarov M.\,Ya.}
\index{Shorgin V.\,S.}

{\renewcommand{\thefootnote}{\fnsymbol{footnote}} \footnotetext[1]
{Работа выполнена при частичной финансовой поддержке РФФИ (проект 15-07-03406).}}


\renewcommand{\thefootnote}{\arabic{footnote}}
\footnotetext[1]{Институт проблем информатики Федерального исследовательского
центра <<Информатика и~управление>> Российской академии наук, agglar@yandex.ru}
\footnotetext[2]{ПАО Промсвязьбанк, murad-agalarov@yandex.ru}
\footnotetext[3]{Институт проблем информатики Федерального исследовательского
центра <<Информатика и~управление>> Российской академии наук, VShorgin@ipiran.ru}

\vspace*{-12pt}
 
  
  \Abst{Рассматривается задача максимизации среднего дохода системы $M/G/1$ в~единицу 
времени на множестве стационарных пороговых стратегий ограничения доступа с~одной 
<<точкой переключения>>. Доход определяется следующими параметрами, измеряемыми 
в~стоимостных единицах: плата, по\-лу\-ча\-емая за обслуживание заявок; затраты на 
техническое обслуживание прибора; вычет из дохода за задержку заявок в~очереди; штраф за 
необслуженные заявки; штраф за простой системы. Получены условия существования 
конечного оптимального порогового значения, предложены метод и~алгоритм расчета оценок 
снизу для оптимального порога и~соответствующего значения максимального дохода 
в~единицу времени. Решена вспомогательная задача максимизации дохода системы, 
усредненного по числу обслуженных заявок, на множестве рассматриваемых пороговых 
стратегий. Получены необходимые и~достаточные условия существования решения 
вспомогательной задачи, предложен метод и~алгоритм ее решения.}
  
  \KW{система массового обслуживания; пороговая стратегия; оптимизация}
  
  \DOI{10.14357/19922264160208} 

\vspace*{-8pt}

\vskip 10pt plus 9pt minus 6pt

\thispagestyle{headings}

\begin{multicols}{2}

\label{st\stat}
  
\section{Введение}

  Для повышения эффективности работы современных систем и~сетей 
передачи данных используют алгоритмы управления потоками (ограничения 
нагрузки), наиболее применяемыми из которых стали различные модификации 
пороговых алгоритмов (пороговых стратегий)~[1]. Одним из основных методов 
исследования эффективности пороговых стратегий является математическое 
моделирование с~использованием аппарата теории массового обслуживания, 
предметом исследования которой выступают системы массового обслуживания 
(СМО) различного типа. 
  
  Большинство работ, в~которых рассмотрены СМО с~пороговой стратегией 
управления потоками, посвящено методам расчета характеристик системы 
(средней длины очереди, среднего времени пребывания, вероятности 
отклонения заявки, времени простоя приборов и~т.\,д.)\ при заданной пороговой 
стратегии (краткий обзор некоторых из них проведен в~[2]). 
%
В~ряде работ, 
посвященных данной тематике, ставится задача оптимизации пороговой 
стратегии в~смысле максимизации дохода системы, представленного в~виде 
стоимостного функционала (см., например,~[3--7]). Хотя практический интерес 
к~такой постановке задачи в~смысле повышения эффективности систем, как 
представляется, выше, чем к~задачам расчета характеристик систем при 
фиксированной пороговой стратегии, вопрос существования эффективных 
методов и~алгоритмов поиска оптимальных пороговых стратегий для СМО 
остается открытым, за исключением самых простых СМО ($M/M/1$, 
$M/M/n$~\cite{3-ag}) и~простых целевых функций (допустимой средней 
задержки заявок в~системе, допустимой интенсивности потерь). 
{\looseness=-1

}

Для более 
сложных СМО (например, $G/M/1$, $M/G/1$) с~более сложными целевыми 
функциями результаты исследований ограничиваются математической 
постановкой задач и~эвристическими алгоритмами их решения. 
%
Отметим 
работу~\cite{4-ag}, где для системы $G/M/n$ сформулирована математическая 
постановка максимизации дохода системы на множестве пороговых стратегий 
с~одним переключением (одной гистерезисной петлей), фиксированными 
платой за своевременное обслуживание и~штрафом за невыполнение этого 
условия для допущенной в~систему заявки. В~работе предложен эвристический 
алгоритм поиска оптимальной стратегии и~выдвинута гипотеза о~том, что для 
систем $G/G/n$ целевая функция унимодальна. 
%
Аналогичная задача 
рассмотрена в~работе~\cite{5-ag} для системы\linebreak\vspace*{-12pt}

\pagebreak

\noindent
 $M/D/1$, где также 
сформулирована математическая постановка задачи и~приведена нижняя 
оценка для оптимального порогового значения. 
%
В~работе~\cite{6-ag} 
рассматривается задача управления запасами, для решения которой в~качестве 
модели используется управляемая на множестве пороговых стратегий СМО 
типа $M/G/1$. 
%
Постановка задачи, наиболее близкая к~рас\-смат\-ри\-ва\-емой 
в~настоящей статье, представлена в~работе~\cite{7-ag}, где решение задачи 
максимизации среднего дохода, получаемого системой $M/G/1$ в~единицу 
времени, ищется на множестве смешанных пороговых стратегий. Решение 
ищется при условии, что поступающая на вход нагрузка меньше единицы. 
Доход системы складывается из следующих составляющих (констант):
  \begin{itemize}
\item плата, получаемая системой за единицу времени обслуживания одной 
заявки прибором;
\item вычет из дохода системы в~единицу времени технического 
обслуживания прибора;
\item вычет из дохода системы в~единицу времени ожидания одной заявки 
в~очереди;
\item штраф за потерю одной заявки;
\item вычет из дохода в~единицу времени простоя системы. 
\end{itemize}

  В работе~\cite{7-ag} доказано, что если решение задачи существует, то оно 
принадлежит множеству чистых стратегий. Для поиска решения задачи 
предлагается использовать численный метод.
  
  В данной статье рассматривается аналогичная задача максимизации среднего 
дохода системы $M/G/1$ на множестве чистых пороговых стратегий для двух 
вариантов формирования дохода: в~одном варианте плату за обслуживание 
система получает в~момент приема заявки и~плата не зависит от длительности 
обслуживания прибором, в~другом плату за обслуживание система получает 
в~момент окончания обслуживания заявки и~величина платы прямо 
пропорциональна длительности обслуживания этой заявки прибором (случай, 
рассмотренный в~\cite{7-ag}). Проведены исследования, касающиеся вопросов 
существования решения задачи максимизации дохода и~метода поиска 
оптимальной стратегии.
{\looseness=-1

}

\vspace*{-6pt}
  
\section{Постановка задачи}

  Рассматривается СМО типа $M/G/1$ с~накопителем бесконечной емкости 
и~одним прибором обслуживания, на которую поступает пуассоновский поток 
заявок с~интенсивностью $\lambda\hm>0$ и~время обслуживания каждой заявки 
распределено по произвольному закону $H(t)$. Поступившая заявка допускается 
в накопитель системы (занимает любое свободное место в~накопителе), если 
в~момент ее поступления число занятых мест в~накопителе меньше~$k$, 
$k\hm>0$~--- некоторое заданное значение (тривиальный случай $k\hm=0$ не 
рассматривается). Такую процедуру доступа заявок в~систему называют 
пороговой стратегией управления доступом (далее для краткости~--- 
стратегией). Обозначим стратегию соответствующим пороговым 
значением~$k$. Если заявка допущена в~накопитель, она занимает любое 
свободное место в~накопителе и~обслуживается на приборе в~порядке 
поступления. Заявка покидает систему только при завершении обслуживания, 
освободив одновременно прибор и~накопитель, а~на освободившийся прибор 
поступает очередная заявка из накопителя (если таковая есть). Система 
получает доход, который определяется следующими составляющими:
  \begin{description}
\item[\,]  $C_0\geq 0$~--- плата, получаемая системой, если поступившая заявка будет 
обслужена системой (допущена в~накопитель); 
\item[\,]  
  $C_1\geq 0$~--- величина штрафа, который платит сис\-те\-ма, если 
поступившая заявка отклонена;
  \item[\,]
  $C_2\geq 0$~--- вычет из дохода сис\-те\-мы за единицу времени ожидания 
заявки в~сис\-теме;
  \item[\,]
  $C_3\geq 0$~--- вычет из дохода сис\-те\-мы за единицу времени простоя 
прибора (за время отсутствия заявок в~сис\-теме);
  \item[\,]
  $C_4\geq 0$~--- затраты сис\-те\-мы в~единицу времени на техническое 
обслуживание сис\-темы. 
  \end{description}
  
  Всюду ниже под доходом системы будем понимать суммарный доход 
с~учетом всех указанных выше составляющих. 
  
  Отметим, что процесс обслуживания заявок в~данной системе описывается 
цепью Mаркова, где переходы цепи определяются моментами окончания 
обслуживания и~состояние системы есть число заявок, остающихся в~ней 
в~момент ухода с~прибора обслуженной заявки (см., например,~[8, 9]). 
Отметим также, что при заданной стратегии~$k$ указанная цепь Маркова имеет 
один положительный возвратный класс состояний $i\hm= 0,\ldots ,k-1$.
  
  Введем обозначения:
  \begin{description}
  \item[\,] $\pi_i^k$, $0\leq i\leq k-1$,~---  стационарное распределение 
вероятностей цепи при стратегии~$k$ ($\pi_i^k$~---  вероятность того, что цепь 
находится в~состоянии~$i$);
  \item[\,] $g^k$~--- среднее значение суммарного предельного дохода, 
усредненного по числу обслуженных заявок;
  \item[\,] $Q^k$~--- средний предельный доход, получаемый сис\-те\-мой в~единицу 
времени при стратегии~$k$;
  \item[\,] $q_i^k$~--- средний доход, получаемый сис\-те\-мой в~состоянии~$i$ при 
стратегии~$k$, $i\hm\geq0$;
  \item[\,] $\overline{v} =\int\nolimits_0^\infty t\, dH(t)$~--- среднее время 
пребывания сис\-те\-мы в~состоянии~$i$, $0\hm< \overline{v}\hm< \infty$;
  \item[\,] $B_{v,m}$~--- событие, состоящее в~том, что за время~$v$ поступило 
ровно~$m$~заявок, $m\hm\geq0$.
  \end{description}
  
  Среднее значение предельного дохода, усредненного по числу обслуженных 
заявок, при стратегии~$k$ равно
  $$
  g^k=\lim\limits_{T\to\infty} \sum\limits_{n=1}^{N_{\mathrm{вх}}} 
\fr{d_n^k}{N_{\mathrm{вых}}(T)}\,,
  $$
где $d_n^k$~--- доход, полученный системой при стратегии~$k$ за 
обслуживание $n$-й заявки, $N_{\mathrm{вч}}(T)$~--- чис\-ло поступивших 
заявок за отрезок времени $[0,T]$;  $N_{\mathrm{вых}}(T)$~--- число 
обслуженных заявок за отрезок времени $[0,T]$. Из определения вложенной 
цепи Маркова следует:
\begin{equation}
g^k=\sum\limits_{i=0}^{k-1} \pi_i^k q_i^k\,.
\label{e1-ag}
\end{equation}
  
  Среднее значение предельного дохода системы в~единицу времени равно: 
  \begin{multline}
  Q^k = \lim\limits_{T\to\infty} \sum\limits_{n=1}^{N_{\mathrm{вх}}(T)}  
\fr{d_n^k}{T} ={}\\
{}=\lim\limits_{T\to\infty} \fr{N_{\mathrm{вых}}(T)}{T} 
\sum\limits_{n=1}^{N_{\mathrm{вх}}(T)} \fr{d_n^k}{N_{\mathrm{вых}}(T)} 
={}\\
  {}= \lim\limits_{T\to\infty} \fr{N_{\mathrm{вых}}(T)}{T} 
\lim\limits_{T\to\infty} \sum\limits_{n=1}^{N_{\mathrm{вх}}(T)} \fr{d_n^k} 
{N_{\mathrm{вых}}(T)} ={}\\
{}=\lambda \left(1- \theta_k^k\right) g^k\,,
  \label{e2-ag}
  \end{multline}
где $\theta_k^k$~--- вероятность того, что поступившая заявка будет допущена 
в систему. 

  Ставится задача: найти оптимальные стационарные стратегии $k^0\hm>0$ и~
$k^*\hm>0$ такие, что
  \begin{equation}
  \mathop{\max}\limits_{k>0} g^k =g^{k^0}\,;\quad \mathop{\max}\limits_{k>0} 
Q^k = Q^{k^*}\,.
  \label{e3-ag}
  \end{equation}
  
\section{Метод решения}

  Перейдем к~рассмотрению параметров $q_i^k$ и~$\pi_j^k$, $j\hm= 0,\ldots , k-
1$, $k\hm\geq 1$. 
  
  Для вероятностей переходов вложенной цепи Маркова~$p^k_{ij}$ 
справедливы формулы:

\noindent
  $$
  p_{ij}^k = \begin{cases}
  r_{j-i+1}\,, & i\leq j\leq k-2\,;\\[3pt]
  1-\sum\limits_{l=0}^{k-i-1} r_l\,, & j=k-1\,,
  \end{cases}
  $$
  если $1\leq i\leq k-1$, $p^k_{0j} \hm= p^k_{1j}$, $p^k_{ij}\hm=0$ при 
$i\hm>k\hm-1$ или $j\hm< i\hm-1$, где
  \begin{equation}
  r_l =\int\limits_0^\infty \fr{(\lambda v)^l}{l!}\, e^{-\lambda v}\,dH(v)\ 
\mbox{ при } 0\leq l\leq k-2\,.
  \label{e4-ag}
  \end{equation}
  
  Для рассматриваемой цепи Маркова при стратегии~$k$ стационарное 
распределение вероятностей является единственным решением системы 
уравнений~\cite{9-ag}:

\noindent
  \begin{equation}
  \left.
  \begin{array}{rl}
  \pi_j^k &= \displaystyle \sum\limits_{i=0}^{j+1} \pi_i^k p^k_{ij}\,,\enskip 
j=0,\ldots, k-2\,;\\[6pt]
  \pi^k_{k-1} &=\displaystyle  \sum\limits_{i=0}^{k-1} \pi_i^k p^k_{ik-1}\,;\\[6pt]
  \displaystyle  \sum\limits_{j=0}^{k-1} \pi_j^k &= 1\,.
  \end{array}
  \right\}
  \label{e5-ag}
  \end{equation}
  
  Из~(\ref{e4-ag}) и~(\ref{e5-ag}) следуют рекуррентные формулы для расчета 
стационарного распределения при стратегии~$k$:

\noindent
  \begin{equation}
  \pi_j^k = \pi_0^k R_j\,,\enskip j=1,\ldots, k-1\,,
  \label{e6-ag}
  \end{equation}
где

\noindent
\begin{gather*}
   \pi_0^k =\left( \sum\limits_{i=0}^{k-1} R_i\right)^{-1}\,,\enskip R_0=1\,,\enskip R_1= \fr{1-r_0}{r_0}\,;\\
   R_{i+1} =\fr{1}{r_0}\left( R_i-r_i -\sum\limits_{j=0}^i R_j r_{i-j+1}\right)\,,\\
    \hspace*{35mm}i=1,\ldots, k-2\,.
   \end{gather*}
  
  
  Величина платы, получаемой системой при пороге $k\hm>1$ за время 
нахождения в~состоянии $1\hm\leq i \hm\leq k-1$ (т.\,е.\ за один шаг 
соответствующей цепи Маркова), вычисляется как сумма плат, получа\-емых за 
обслуживание допущенных в~систему за это время новых заявок, и~выражается 
формулой:

\noindent
  \begin{multline}
  \int\limits_0^\infty \sum\limits_{m=1}^{k-i} \fr{(\lambda v)^m}{m!}\,
  e^{-\lambda v} C_0 m\,dH(v) + {}\\
{}+\int\limits_0^\infty \sum\limits_{m=k-i+1}^\infty 
\fr{(\lambda v)^m}{m!}\, e^{-\lambda v} C_0 (k-i)\,dH(v)={}\\
  {}=C_0 \left[ \sum\limits_{m=1}^{k-i} mr_m +(k-i) 
  \sum\limits^\infty_{m=k-i+1} r_m\right]\,.
  \label{e7-ag}
  \end{multline}
  
  Величина штрафа, которую система платит за не допущенные в~состоянии 
$1\hm\leq i\hm\leq k\hm-1$ заявки, равна:
  \begin{equation}
  C_1 \sum\limits_{m=k-i+1}^\infty \hspace*{-1mm}(m-k+i) r_m\,.
  \label{e8-ag}
  \end{equation}
  
  \vspace*{-12pt}
  
  \pagebreak
  
  Величина вычета из дохода системы за задержку заявок в~системе, 
накопленная за время пребывания системы в~состоянии $1\hm\leq i \hm\leq 
k\hm-1$, равна
  \begin{multline}
  \fr{C_2}{\lambda} \left[ \fr{1}{2}\sum\limits_{m=2}^{k-i+1} \!\!(m-1)mr_m +(k-i) 
\sum\limits_{m=k-i+2}^\infty mr_m-{}\right.\\
  \left.{}- \fr{1}{2} (k-i) (k-i+1) \sum\limits^\infty_{m=k-i+2}\hspace*{-2mm} r_m \right] + C_2(i-
1) \overline{v}.\!
  \label{e9-ag}
  \end{multline}
  
  Формула~(\ref{e9-ag}) получается из следующих рассуждений. Фиксируем 
состояние системы~$i$, $1\hm\leq i\hm\leq k\hm-1$. Пусть $v$~--- длительность 
нахождения системы в~состоянии~$i$; $T_l$~--- последовательные моменты 
поступления заявок в~промежутке времени $[0,v)$; $m$~--- число поступивших 
за время~$v$ заявок; $U_l^i(m,v)$~--- длительность времени из интервала 
$[0,v)$, в~течение которого заявка, поступившая в~момент~$T_l$, 
$l\hm+i\hm\leq k$, ожидала в~системе. 
  
  Так как входящий поток пуассоновский, то совместное распределение 
величин~$T_l$ совпадает с~распределением порядковых статистик из 
выборки~$m$, взятой из равномерного распределения на $(0,v]$ (см., 
например,~\cite{8-ag}) и~маргинальным распределением случайной величины~$T_l/v$ является 
бе\-та-рас\-пре\-де\-ле\-ние с~плотностью
 \begin{multline*}
  f(x) ={}\\
  {}=\begin{cases}
  \fr{m!}{(l-1)! (m-l)!}\, x^{l-1} (1-x)^{m-l}\,, & 0<x<1\,;\\
  0 &\hspace*{-16mm} \mbox{в\ противном\ случае}\,.
  \end{cases}\hspace*{-0.12288pt}
  \end{multline*}
  
  Тогда поскольку $U_l^i(m,v) \hm= v-T_l$, $l\hm+ i\hm\leq k$, то условное 
среднее значение $M[U_l^i(m,v)]\hm= v-lv/(m+1)$ (при условии, что 
длительность нахождения системы в~состоянии~$i$ равна~$v$ и~за это время 
поступило ровно~$m$~заявок). При $m\hm \leq k\hm-1$ получаем, что среднее 
значение $M[\sum\nolimits_{l=1}^m (v\hm- T_l)]\hm= mv\hm- mv/2 \hm= mv/2$, 
а~при $m\hm > k\hm- i$, так как поступившие в~моменты~$T_l$, $l\hm> k\hm- 
i$, отклоняются, условное среднее значение суммарного времени ожидания 
в~системе поступивших в~состоянии~$i$~заявок равно: 
  \begin{multline*}
  M\left[ \sum\limits_{l=1}^{k-i} \left(v-T_l\right)\right] = (k-i) v -v 
\sum\limits_{l=1}^{k-i} \fr{l}{m+1} ={}\\
  {}= (k-i) v- \fr{(k-i)(k-i+1)v}{2(m+1)}\,.
  \end{multline*}
  
  Суммарное время ожидания в~системе заявок, находящихся в~системе 
в~момент перехода в~состояние~$i$, равно $(i\hm- 1)v$. Отсюда получаем, что 
на интервале времени $(0,v]$ (во время пребывания системы в~состоянии~$i$) 
при условии, что в~этом интервале поступило~$m$~заявок, суммарное время 
ожидания находящихся в~системе заявок равно:
  \begin{multline*}
  \fr{mv}{2}\,\chi(k-i-m) +{}\\
  {}+\fr{(k-i)(k-i+1)v}{2(m+1)}\left[ 1-\chi(k-i-m)\right] +(i-
1)v\,,
  \end{multline*}
где $\chi(k-i-m)$~--- функция Хевисайда. Использовав формулу полной 
вероятности и~взяв среднее по распределению $H(v)$ (аналогично 
формуле~(\ref{e7-ag})), из последнего выражения получим  
формулу~(\ref{e9-ag}).

  Использовав формулы~(\ref{e7-ag})--(\ref{e9-ag}) и~то, что величина 
среднего значения вычета из дохода за техническое обслуживание системы 
в~состоянии $1\hm\leq i\hm\leq k\hm-1$ равно $C_4\overline{v}$, получим:
  \begin{multline}
  q_i^k = C_0 \left[ \sum\limits_{m=1}^{k-i} mr_m +(k-i)\sum\limits^\infty_{m=k-
i+1} r_m \right] - {}\\
{}-C_1 \sum\limits^\infty_{m=k-i+1} (m-k+i) r_m-{}\\
  {}-\fr{C_2}{\lambda} \left[ \fr{1}{2}\sum\limits_{m=1}^{k-i+1} (m-1) mr_m 
+(k-i) \sum\limits_{m=k-i+2}^\infty \!mr_m -{}\right.\\
  \left.{}- \fr{1}{2}(k-i) (k-i+1) \sum\limits^\infty_{m=k-i+2} r_m\right] - {}\\
  {}-C_2(i-1) 
\overline{v} -C_4 \overline{v}\,,\enskip 1\leq i\leq k-1\,.
  \label{e10-ag}
  \end{multline}
  
  Состояние $i=0$ отличается от состояния $i\hm=1$ в~смысле дохода только 
тем, что в~состоянии $i\hm=0$ возможен простой длительности~$1/\lambda$, 
из-за которого доход уменьшается на величину $(C_3\hm+C_4)/\lambda$, 
и~завершает обслуживание в~состоянии $i\hm=0$ заявка, которая поступила, 
когда система находилась в~этом же состоянии. Поэтому справедливо равенство:
  \begin{equation*}
  q_0^k = q_1^k +C_0 - \fr{C_3+C_4}{\lambda}\,.
%  \label{e11-ag}
  \end{equation*}
  
  Докажем следующее вспомогательное утверждение.
  
  \smallskip
  
  \noindent
  \textbf{Лемма~1.}\  \textit{Справедливы равенства}
  \begin{multline}
  q_j^{k+1} =q_j^k +\left(C_0+C_1\right) \sum\limits_{m=k-i+1}^\infty r_m -{}\\
\hspace*{-3mm}{}- \fr{C_2}{\lambda} \left[ \sum\limits_{m=k-i+1}^\infty \hspace*{-2mm}mr_m - (k-i+1) 
\sum\limits_{m=k-i+1}^\infty\hspace*{-2mm} r_m \right]\,;
  \label{e12-ag}
  \end{multline}
  \begin{equation}
  \pi_j^{k+1} =  \left( 1-\pi_k^{k+1}\right) \pi_j^k\,\enskip j=0,\ldots, k-1\,.
  \label{e13-ag}
  \end{equation}
  
  \noindent
  Д\,о\,к\,а\,з\,а\,т\,е\,л\,ь\,с\,т\,в\,о\,.\ \ Из~(\ref{e9-ag}) имеем:
  \begin{multline*}
  q_i^{k+1} -q_i^k= C_0\left[
  \sum\limits_{m=1}^{k-i} mr_m +(k-i+1) r_{k-i+1}+{}\right.\\
\left.  {}+ \sum\limits^\infty_{m=k-i+2}\hspace*{-2mm} r_m +(k-i) \sum\limits^\infty_{m=k-i+2}
\hspace*{-2mm} r_m\right] -{}\\
{}-C_1
\sum\limits^\infty_{m=k-i+2}\hspace*{-1mm} (m-k+i) r_m+ 
C_1 \sum\limits^\infty_{m=k-i+2} \hspace*{-2mm} r_m -{}\\
{}-\fr{C_2}{\lambda}
\left[
\fr{1}{2}\sum\limits_{m=2}^{k-i+1} (m-1)mr_m+{}\right.\\
\left.{}+\fr{1}{2}\left( k-i+1\right)
(k-i+2) r_{k-i+2}+{}\right.\\
{}+(k+1-i) \sum\limits^\infty_{m=k-i+3}\hspace*{-2mm} mr_m-{}\\
\left.{}-
\fr{1}{2}\left( k-i+1\right) (k-i+2) 
\sum\limits^\infty_{m=k-i+3}\hspace*{-2mm} r_m\right]-{}\\
{}-C_0\left[ \sum\limits_{m=1}^{k-i} mr_m+(k-i) r_{k-i+1} +(k-i)\!\hspace*{-2.9pt} \sum\limits^\infty_{m=k-i+2}
\hspace*{-3mm} r_m\right]+{}\\
{}+C_1 r_{k-i+1} +C_1\sum\limits^\infty_{m=k-i+2}\hspace*{-2mm} (m-k+i)r_m+{}\\
{}+ \fr{C_2}{\lambda} \left[ \fr{1}{2} \sum\limits_{m=2}^{k-i+1}\hspace*{-1mm}
(m-1)mr_m +{}\right.\\
{}+(k-i) (k-i+2)r_{k-i+2}+
(k-i) \sum\limits^\infty_{m=k-i+3}\hspace*{-3mm} mr_m -{}\\
{}-\fr{1}{2}\left(k-i\right)
(k-i+1) r_{k-i+2}+{}\\
\left.{}+\fr{1}{2}\left( k-i\right) (k-i+1) \sum\limits^\infty_{m=k-i+3}\hspace*{-2mm} r_m\right].
\end{multline*}
    
  
  Проведя преобразования, получим:
  \begin{multline*}
  q_i^{k+1} - q_i^k = \left( C_0+ C_1\right) \sum\limits_{m=k-i+1}^\infty r_m-{}\\
  {}- \fr{C_2}{\lambda}\! \left[ \sum\limits_{m=k-i+3}^\infty \hspace*{-3mm}mr_m\! -\!(k-i+1) 
\hspace*{-3mm}\sum\limits_{m=k-i+3}^\infty\hspace*{-2mm} r_m +r_{k-i+2}\right]\!={}\\
  {}= \left( C_0+C_1\right) \sum\limits_{m=k-i+1}^\infty r_m- {}\\
  {}- \fr{C_2}{\lambda} \left[ \sum\limits_{m=k-i+1}^\infty mr_m -(k-i+1) 
\sum\limits_{m=k-i+1}^\infty r_m\right]\,.
  \end{multline*}
  
  Второе равенство утверждения следует непосредственно из~(\ref{e6-ag}):
  
  \noindent
  \begin{multline*}
  \pi_j^k -\pi_j^{k+1} =\fr{R_j}{R_0+\cdots +R_{k-1}} - 
\fr{R_j}{R_0+\cdots+R_k}={}\\
  {}= \fr{R_j R_k}{(R_0+\cdots +R_{k-1}) (R_0+\cdots + R_k)} = \pi_j^k 
\pi_k^{k+1}\,.
  \end{multline*}
  
  Получим вспомогательное выражение для разности $g^k\hm- g^{k+1}$, 
$0\hm< k\hm<\infty$. Из формулы~(\ref{e1-ag}) следует:
  \begin{multline*}
  g^k-g^{k+1} =\sum\limits_{j=0}^{k-1} \pi_j^k q_j^k - \sum\limits_{j=0}^k 
\pi_j^{k+1} q_j^{k+1} ={}\\
  {}= \sum\limits_{j=0}^{k-1} \pi_j^k q_j^k - \sum\limits_{j=0}^{k-1} \pi_j^{k+1} 
q_j^{k+1}- \pi_k^{k+1} q_k^{k+1}\,.
  \end{multline*}
  
  Подставив вместо $q_i^{k+1}$ правую часть равенства~(\ref{e12-ag}), 
получим:
  \begin{multline*}
  g^k-g^{k+1} =\sum\limits_{j=0}^{k-1} \pi_j^k q_j^k - \sum\limits_{j=0}^{k-1} 
\pi_j^{k+1} q_j^k-{}\\
  {}- \sum\limits_{j=0}^{k-1} \pi_j^{k+1}\left\{ \left( C_0+C_1\right) 
\sum\limits^\infty_{m=k-i+1} r_m -{}\right.\\
\left.{}-\fr{C_2}{\lambda} \sum\limits_{m=k-
i+1}^\infty [m-(k-i+1)]r_m\right\}- \pi_k^{k+1} q_k^{k+1}\,.
  \end{multline*}
  
  Заменив здесь $\pi_j^{k+1}$ на правую часть равенства~(\ref{e13-ag}), имеем:
  \begin{multline}
  g^k - g^{k+1} =\sum\limits_{j=0}^{k-1} \pi_j^k q_j^k - \left(1-\pi_k^{k+1}\right) 
\sum\limits_{j=0}^{k-1} \pi_j^k q_j^k-{}\\
  {}-\left( 1-\pi_k^{k+1}\right) \sum\limits_{j=0}^{k-1} \pi_j^k \left\{ \left( 
C_0+C_1\right) \sum\limits_{m=k-i+1}^\infty r_m-{}\right.\\
  \left.{}- \fr{C_2}{\lambda} \sum\limits^\infty_{m=k-i+1}\hspace*{-2mm} [m-(k-i+1)]r_m\right\} 
-\pi_k^{k+1} q_k^{k+1}={}\\
  {}=\pi_k^{k+1} \!\left\{ \!g^k-\fr{1-\pi_k^{k+1}}{\pi_k^{k+1}} 
\sum\limits_{j=0}^{k-1} \!\pi_j^k \!\left\{ \!\left( C_0+C_1\right)\hspace*{-3mm}
 \sum\limits_{m=k-i+1}^\infty\hspace*{-3mm} r_m -{}\right.\right.\hspace*{-0.3pt}\\
  \left.\left.\hspace*{-5mm}{}- \fr{C_2}{\lambda} \sum\limits_{m=k-i+1}^\infty
  \hspace*{-2mm} [m-(k-i+1)]r_m\right\} -q_k^{k+1}
  \vphantom{\fr{1-\pi_k^{k+1}}{\pi_k^{k+1}} \left\{\sum\limits_{j=0}^{k-1}\right\}}
  \right\}.
  \label{e14-ag}
  \end{multline}
    Заменив в~правой части последнего неравенства $\pi_k^{k+1}$ и~$\pi_j^k$, 
$j\hm= 0,\ldots, k$, на их выражения в~(\ref{e6-ag}), перепишем~(\ref{e14-ag}) 
в~виде:

\noindent
  \begin{multline}
  g^k-g^{k+1} ={}\\
  {}=\pi_k^{k+1} \left\{ g^k -\fr{1}{R_k} \sum\limits_{j=0}^{k-1} R_j 
\left\{ \left( C_0+C_1\right) \hspace*{-1mm}\sum\limits_{m=k-i+1}^\infty \hspace*{-2mm}r_m -{}\right.\right.\\
  \left.\left.\hspace*{-5mm}{}- \fr{C_2}{\lambda} \sum\limits_{m=k-i+1}^\infty 
  \hspace*{-1mm}\left[ m-(k-
i+1)\right] r_m \right\} -q_k^{k+1}\right\}.
  \label{e15-ag}
  \end{multline}
  
  Из предпоследнего уравнения в~(\ref{e5-ag}), подставив~(\ref{e4-ag}) и~(\ref{e6-ag}),
  получим:
  \begin{equation}
  R_k = \fr{\sum\nolimits_{j=0}^{k-1} R_j \sum\nolimits_{i=k-j+1}^\infty 
r_i}{r_0}\,.
  \label{e16-ag}
  \end{equation}
  
  Заменив в~(\ref{e15-ag}) $R_k$ на правую часть~(\ref{e16-ag}) 
и~$q_{i+1}^{k+1}$ на $q_i^k\hm- C_2\overline{v}$ (см.\  
формулу~(\ref{e10-ag})), получим:
  \begin{multline}
  g^k-g^{k+1} =\pi_k^{k+1} \left\{
  \vphantom{\fr{C_1}{\lambda}}
   g^k -\left( C_0+C_1\right) r_0 +{}\right.\\
{}+\fr{C_2}{\lambda} r_0 \fr{\sum\nolimits_{j=0}^{k-1} R_j  
\sum\nolimits_{m=k-j+1}^\infty [m-(k-j)]r_m}{\sum\nolimits_{j=0}^{k-1} R_j 
\sum\nolimits_{i=k-j+1}^\infty r_i} - {}\\
\left.{}-\fr{C_2}{\lambda}\,r_0 -q_1^2 +C_2 (k-1) 
\overline{v}\right\}\,.
  \label{e17-ag}
  \end{multline} 
  
  Обозначим 
  \begin{multline}
  F(k) = {}\\
  {}=\fr{\sum\nolimits_{j=0}^{k-1} R_j  \sum\nolimits_{m=k-j+1}^\infty [m-
(k-j)]r_m}{\sum\nolimits_{j=0}^{k-1} R_j \sum\nolimits_{i=k-j+1}^\infty r_i}\,.
  \label{e18-ag}
  \end{multline}
  
  Заметим, что $F(k)$~--- среднее число заявок, отклоненных при 
стратегии~$k$ в~стационарном режиме работы системы за время обслуживания 
одной заявки, при условии, что поступило больше заявок, чем число свободных 
мест в~накопителе.
  
  Использовав обозначение~(\ref{e18-ag}), перепишем~(\ref{e17-ag}) в~виде
  \begin{multline}
  g^k-g^{k+1}=\pi_k^{k+1}\left\{ g^k -\left( C_0+C_1+\fr{C_2}{\lambda}\right) 
r_0 +{}\right.\\
  \left.{}+ \fr{C_2}{\lambda}\,r_0 F(k) -q_1^2 +C_2(k-1)\overline{v}\right\}\,.
  \label{e19-ag}
  \end{multline}
  
  Как видно из~(\ref{e19-ag}), неравенство $g^k\hm- g^{k+1} \hm<0$ 
эквивалентно неравенству: 
  \begin{multline}
  g^k < \left( C_0 +C_1 +\fr{C_2}{\lambda}\right) r_0 -\fr{C_2}{\lambda}\,r_0 
F(k) +{}\\
  {}+ q_1^2 -C_2(k-1)\overline{v}\,.
  \label{e20-ag}
  \end{multline}
  
  Правую часть~(\ref{e20-ag}) обозначим через $G(k)$:
  \begin{multline}
  G(k) = \left( C_0 +C_1 +\fr{C_2}{\lambda}\right) r_0 -\fr{C_2}{\lambda}\,r_0 
F(k) +{}\\
  {}+ q_1^2 -C_2(k-1)\overline{v}\,.
  \label{e21-ag}
  \end{multline}
  
  Докажем следующее утверждение. 
  
  \smallskip
  
  \noindent
  \textbf{Теорема~1.}\ \textit{Пусть $F(k) \hm- F(k+1) \hm< \lambda 
\overline{v}$, $0\hm< \lambda \hm< \infty$, $0\hm< \overline{v}\hm< \infty$, 
$k\hm>0$. Тогда при любых значениях параметров $0\hm\leq C_i \hm<\infty$, 
$i\hm= 0,\ldots, 4$, $0\hm< C_2 \hm<\infty$ cуществует стационарная стратегия 
$0\hm< k^0 \hm< \infty$.   При этом если $g^1\hm\geq G(1)$, то $k^0\hm=1$, 
а~если $C_2\hm=0$ и~$g^1\hm < G(1)$, то} $k^0\hm=\infty$.
  
  \smallskip
  
  \noindent
  Д\,о\,к\,а\,з\,а\,т\,е\,л\,ь\,с\,т\,в\,о\,.\ \ Из~(\ref{e21-ag}) и~условия 
теоремы следует:
  \begin{multline}
  \hspace*{-6pt}G(k) -G(k+1) = -\fr{C_2}{\lambda}\,r_0 \left[ F(k) -
F(k+1)\right]+C_2\overline{v} >{}\\
{}> C_2\overline{v} \left( 1-r_0\right) >0\,.
  \label{e22-ag}
  \end{multline}
  
  Рассмотрим поведение функции~$g^k$ при последовательном увеличении 
значения $k\hm>0$. Как следует из~(\ref{e19-ag}), неравенства $g^k\hm\leq 
g^{k+1}$ и~$g^k\hm\leq G(k)$ эквивалентны. Отсюда вытекает, что если при 
текущем значении $k\hm>0$ выполняется неравенство $g^k\hm< G(k)$, то при 
следующем значении $k\hm+1$ выполняется неравенство $g^{k+1}\hm> g^k$,
 и~наоборот: если при текущем значении $k\hm>0$ выполняется неравенство 
$g^k\hm> G(k)$, то при следующем значении $k\hm+1$ выполняется 
неравенство $g^{k+1}\hm< g^k$, а~при $g^k\hm= G(k)$ выполняется равенство 
$g^{k+1}\hm= g^k$. 
  
  Рассмотрим сначала случай $C_2\hm>0$. Пусть~$k_1$~--- произвольное 
число, $k_1\hm\geq 1$. Рассмотрим два альтернативных случая: (1)~$g^{k_1} 
\hm\geq G(k_1)$ и~(2)~$g^{k_1}\hm< G(k_1)$. Всюду ниже под~$k^0$ будем 
понимать одну из стратегий, удовлетворяющих условию: $\max\limits_{k>0} 
g^k\hm= g^{k^0}$ (см.~(\ref{e3-ag})).
  
  Пусть $g^{k_1}\geq G(k_1)$. Тогда если $g^{k_1}\hm> G(k_1)$, то 
согласно~(\ref{e19-ag}) (так как $0\hm< \pi_k^{k+1}\hm<1$) выполняется 
$g^{k_1+1}\hm> G(k_1)$, а~так как $G(k)$ убывает по $k\hm>0$ (следует  
из~(\ref{e22-ag})), выполняется $g^{k_1+1}\hm> G(k_1+1)$. Следовательно, 
$g^k\hm> G(k)$, $k\hm\geq k_1$, и~согласно~(\ref{e19-ag}) и~(\ref{e20-ag}) 
последовательность $\{ g^k, k=k_1, k_1+1,\ldots\}$ является убывающей, т.\,е.\ 
$g^k\hm< g^{k_1}$, $k\hm> k_1$, откуда следует, что существует $k^0\hm\leq 
k_1$. Если $g^{k_1}\hm= G(k_1)$, то из~(\ref{e14-ag}) 
  и~неравенства~(\ref{e22-ag}) следует, что $g^{k_1+1} \hm= g^{k_1}$, 
$g^{k_1+1}\hm> G(k_1+1)$. Тогда, как уже показано, существует $k^0\hm\leq 
k_1+1$. Так как $g^{k_1+1} \hm= g^{k_1}$, то существует также $k^0\hm\leq 
k_1$. При этом если $k_1\hm=1$, то получаем $k^0\hm=1$. 
  
  Пусть $g^{k_1}< G(k_1)$. Тогда согласно неравенству~(\ref{e20-ag}) из 
$g^{k_1} \hm< G(k_1)$ следует $g^{k_1+1} \hm> g^{k_1}$ и~неравенство 
$g^{k_1+1} \hm< G(k_1)$, так как $g^{k_1+1} \hm- g^{k_1}\hm< G(k_1) \hm- 
g^{k_1}$ (это следует из равенства~(\ref{e19-ag}) и~неравенства $0\hm < 
\pi_k^{k+1} \hm<1$). Следовательно, пока выполняется неравенство $G(k) \hm> 
g^k$, $k\hm= k_1, k_1+1, \ldots$, будет выполняться неравенство $G(k) \hm> 
g^{k+1}\hm> g^k$. Так как согласно~(\ref{e22-ag}) $G(k)$ убывает по $k\hm>0$ 
и~$G(k)\underset{k}{\to} -\infty$, то существует~$k_2$, $k_1\hm< k_2\hm< 
\infty$, такое, что выполняется система неравенств $g^{k_2+1} \hm\geq 
G(k_2+1)$, $G(k_1)\hm> G(k) \hm> g^{k+1}\hm> g^k$, $k_1 +1\hm\leq k\hm\leq 
k_2$. Тогда поскольку при $k_2\hm+1$ наступает случай~1, то, как уже 
доказано, существует $k^0\hm\leq k_2\hm+1$, т.\,е.\ утверждение теоремы~1 
для случая~2 верно.
  
  Пусть теперь $C_2=0$. В~этом случае из~(\ref{e21-ag}) следует $G(k)\hm= 
const$, $k\hm>0$. Рассуждая точно так же, как и~в случае $C_2\hm>0$, 
получаем последовательность $\{ g^k, k\hm\geq 1\}$, которая удовлетворяет 
неравенствам $g^1\hm\geq g^2\geq \cdots \geq g^n\geq \cdots \geq const$ при 
условии $g^1\hm\geq const$ и~неравенствам $g^1\hm< g^2<\cdots < g^n< \cdots < 
const$ при условии $g^1\hm< const$. Как видим, $k\hm=1$ при $g^1\hm\geq 
const$ и~$k\hm=\infty$ при $g^1\hm< const$ удовлетворяют определению~$k^0$ 
в~условии теоремы. 
  
  \smallskip
  
  \noindent
  \textbf{Следствие~1.}\ \textit{Пусть $F(k)$ удовлетворяет условию 
теоремы. Стратегия~$k^0$ удовлетворяет условию $\max\limits_{k>0} g^k\hm= 
g^{k^0}$ тогда и~только тогда, когда~$k^0$ удовле\-тво\-ря\-ет одному из трех 
условий}:
  \begin{enumerate}[(1)]
  \item $G(1) \leq g^1$, $k^0=1$;
  \item $G(1)>g^1$, $k^0=\min\{ k: G(k)\leq g^k\}$;
  \item $g^{k^0-1} <g^{k^0}$, $g^{k^0+1} >g^{k^0}$, $1\hm< k^0\hm< \infty$.
  \end{enumerate}
  
  \noindent
  Д\,о\,к\,а\,з\,а\,т\,е\,л\,ь\,с\,т\,в\,о\,.\ \ Доказательство следствия~1 
следует непосредственно из доказательства теоремы~1. 
  
  \smallskip
  
  \noindent
  \textbf{Следствие~2.}\ \textit{Пусть $F(k)$ удовлетворяет условию 
тео\-ре\-мы и~$g^{k^0}\hm>0$. Тогда решение (одно из решений)~$k^*$ 
задачи}~(\ref{e3-ag}) \textit{удовлетворяет условиям} $k^*\hm\geq k^0$, $\lambda(1-
\theta_{k^0}^{k^0})g^{k^0}\hm\leq Q^{k^*}\hm \leq \lambda g^{k^0}$. 
  
  \smallskip
  
  \noindent
  Д\,о\,к\,а\,з\,а\,т\,е\,л\,ь\,с\,т\,в\,о\,.\ \  Так как~$\theta_k^k$ убывает 
по~$k$, для любой стратегии $0\hm< k\hm\leq k^0$ выполняются условия 
$Q^k\hm\leq \lambda (1\hm-\theta_{k^0}^{k^0}) g^{k^0} \hm= Q^{k^0} \hm\leq 
Q^{k^*}$, если $g^{k^0} \hm\geq 0$ (см.~(\ref{e2-ag})). Кроме того, $0\hm< 
Q^{k^*} \hm= \lambda (1\hm- \theta_{k^*}^{k^*})g^{k^*} \hm< \lambda g^{k^*} \hm 
< \lambda g^{k^0}$.
  
  На основании следствия~1 стратегию~$k^0$ можно найти с~помощью 
следующего простого алгоритма.
  \begin{enumerate}[1.]
\item Положить $k=1$.
\item Если $G(k)>g^k$, увеличить~$k$ на единицу, иначе перейти 
к~п.~5.
\item Вычислить $G(k)$ и~$g^k$.
\item Если выполняется неравенство $G(k)\hm> g^k$, перейти к~п.~2, 
иначе перейти к~п.~5.
\item Положить $k^0=k$.
\item Конец алгоритма.
  \end{enumerate}
  
\section{Пример}

  В качестве примера рассмотрим СМО $M/H_n/1$ с~функцией распределения 
времени обслуживания $H_n(t) \hm= \sum\nolimits_{i=1}^n f_i (1\hm- e^{-\mu_i 
t})$. Справедливы соотношения:
\begin{align*}
r_{k-j+1}&=\sum\limits_{i=1}^n f_i \fr{\mu_i}{\lambda +\mu_i}\left( 
\fr{\lambda}{\lambda+\mu}\right)^{k-j+1}\,;\\
\sum\limits^\infty_{m=k-j+1} \hspace*{-2mm} r_m&= \sum\limits^\infty_{m=k-j+1}
\sum\limits^n_{i=1} f_i \fr{\mu_i}{\lambda+\mu_i} \left(
\fr{\lambda}{\lambda+\mu_i}\right)^m={}\\
&\hspace*{20mm}{}=\sum\limits^n_{i=1} f_i
\left( \fr{\lambda}{\lambda+\mu_i}\right)^{k-j+1}\,;\\
\sum\limits^\infty_{m=k-j+1}\hspace*{-2mm} mr_m &={}\\
&\hspace*{-12mm}{}=
\sum\limits_{i=1}^n f_i \left( \fr{\lambda}{\lambda+\mu_i}\right)^{k-j+1}
\left( \fr{\lambda+\mu_i}{\mu_i}+k-j\right)\,.
\end{align*}
    
  
  Функции $F(k)$ (см.~(\ref{e18-ag})) и~$G(k)$ принимают вид:
  \begin{multline*}
  F(k) = {}\\
  {}=
  \fr{\sum\nolimits_{j=0}^{k-1} R_j \sum\nolimits^\infty_{m=k-j+1} [m-(k-
j+1)]r_m}{\sum\nolimits_{j=0}^{k-1} R_j  \sum\nolimits^\infty_{i=k-j+1} 
r_i}+1={}\\
  {}= \lambda \fr{\sum\nolimits^n_{i=1} f_i (1/\mu_i) \sum\nolimits_{j=0}^{k-1} 
R_j (\lambda/(\lambda+\mu_i))^{k-j+1}}{\sum\nolimits^n_{i=1} f_i 
\sum\nolimits_{j=0}^{k-1} R_j (\lambda/(\lambda+\mu_i))^{k-j+1}}+1;\hspace*{-4.41524pt}
  \end{multline*}
  
  \vspace*{-12pt}
  
  \noindent
  \begin{multline*}
  G(k) = r_o\left( C_0+C_1\right) -{}\\
  {}- C_2 r_0 \fr{\sum\nolimits_{i=1}^n f_i (1/\mu_i) \sum\nolimits_{j=0}^{k-1} 
R_j (\lambda/(\lambda+\mu_i))^{k-j+1} }{\sum\nolimits_{i=1}^n f_i  
\sum\nolimits_{j=0}^{k-1} R_j (\lambda/(\lambda+\mu_i))^{k-j+1}} +{}\\
{}+q_1^2 -C_2 
\sum\limits_{i=1}^n \fr{f_i}{\mu_i}\left(k-1\right)\,.
  \end{multline*}
  


  На рисунке приведены зависимости функций~$g^k$ и~$G(k)$ от порогового 
значения и~проиллюстрирована зависимость выполнения условия 
$g^{k+1}\hm> g^k$ от условия $g^k\hm< G(k)$. Графики~\textit{1} и~\textit{2} 
на рисунке соответствуют~$g^k$ и~$G(k)$ при значениях параметров 
$C_0\hm= 20$, $C_1\hm= 10$, $C_2\hm= 0{,}5$, $C_3\hm= 0{,}01$, $C_4\hm= 
0{,}01$, $\lambda\hm= 1$, $n\hm=2$, $f_1\hm= 0{,}3$, $f_2\hm= 0{,}7$  
и~$\mu_1\hm=\mu_2\hm=1$; графики~\textit{3} и~\textit{4} изображают 
функции~$g^k$ и~$G(k)$ при $\mu_1\hm=2$ и~$\mu_2\hm=1$ и~тех же значениях 
остальных параметров. Оптимальный порог на рисунке обозначен 
через~$k_{\mathrm{опт}}$. В~случае $M/M/1$ имеем:
  $$
  r_{k-j+1} = \fr{\mu}{\lambda+\mu} \left( 
\fr{\lambda}{\lambda+\mu}\right)^{k-j+1}\,;
  $$
 

  
\noindent
 \begin{center}
 \vspace*{1pt}
 \mbox{%
 \epsfxsize=77.829mm
 \epsfbox{aga-1.eps}
 }

 \end{center}
 
 \noindent
{\small Зависимость предельного дохода и~функции $G(k)$ от порогового значения}

\vspace*{6pt}

  
  \noindent
   \begin{multline*}
   \sum\limits^\infty_{m=k-j+1}\hspace*{-1mm} r_m = \sum\limits_{m=k-j+1}^\infty \fr{\mu}{\lambda+\mu}\left( 
\fr{\lambda}{\lambda+\mu}\right)^m ={}\\
{}= \left( \fr{\lambda}{\lambda+\mu}\right)^{k-j+1}\,;
\end{multline*}

\vspace*{-6pt}

\noindent
   $$
   \sum\limits_{m=k-j+1}^\infty \hspace*{-1mm}mr_m =\left( \fr{\lambda}{\lambda+\mu}\right)^{k-j+1} \left( 
\fr{\lambda+\mu}{\mu}+k-j\right)\,;
   $$
   $$
   F(k) =\lambda\overline{v}+1\,;
   $$
   $$
   G(k) =r_0\left( C_0 +C_1-\fr{C_2}{\mu}\right) +q_1^2 -\fr{C_2}{\mu}\left(k-1\right)\,.
   $$
  
  Как видим, $F(k) \hm- F(k\hm+1) \hm= 0\hm< \lambda \overline{v}$ и~$G(k)$ 
убывает линейно по $k\hm>0$; следовательно, согласно следствию~1 
предложенный выше алгоритм дает оптимальный порог.
  
\section{Заключение}

  В отличие от постановки задачи, рассмотренной в~\cite{7-ag}, выше 
предполагалось, что плата за обслуживание система получает в~момент приема 
задачи в~накопитель системы и~величина платы не зависит от длительности 
обслуживания на приборе. Покажем, что и~в случае, рассмотренном  
в~\cite{7-ag}, приведенные выше утверждения также справедливы. 
  
  Если плату за обслуживание система получает в~момент окончания 
обслуживания заявки и~величина платы прямо пропорциональна длительности 
обслуживания на приборе, средняя величина платы, получаемой системой за 
обслуживание заявки во время нахождения в~данном состоянии, 
равна~$C_0\overline{v}$ (так как в~каждом состоянии на приборе 
обслуживается одна заявка). Заметим, что эта величина не зависит от стратегии 
$k\hm>0$ и~остальные составляющие дохода системы в~этом случае и~в~случае, 
рассмотренном выше, совпадают. После замены в~(\ref{e10-ag}) правой части 
формулы~(\ref{e7-ag}) на $C_0\overline{v}$ формула~(\ref{e12-ag}) принимает 
вид:
  \begin{multline*}
  q_j^{k+1} =q_j^k +C_1\sum\limits_{m=k-i+1}^\infty \hspace*{-1mm}r_m-{}\\
  {}- \fr{C_2}{\lambda} \left [ \sum\limits_{m=k-i+1}^\infty \hspace*{-1mm}mr_m -(k-i+1) 
\sum\limits_{m=k-i+1}^\infty \hspace*{-1mm}r_m\right]\,.
  \end{multline*}
  
  Как видим, в~формуле~(\ref{e12-ag}) произошла только замена константы 
$(C_0\hm+C_1)$ на константу~$C_1$, и,~следовательно, все рассуждения, 
проведенные выше относительно задачи~(\ref{e3-ag}), остаются в~силе. 
Отметим, что выше нигде не требовалось выполнение условия 
$\lambda\overline{v}\hm<1$ (в отличие от~\cite{7-ag}). 
  
  В системах и~сетях передачи данных буферная память узлов связи имеет 
ограниченную емкость, линии связи~--- ограниченную пропускную 
способность. Поэтому в~качестве модели узла связи (или канала связи) больше 
подходит (в смысле адекватности) СМО типа $M/G/1/r$ с~числом мест 
хранения $r\hm<\infty$, чем $M/G/1$ с~бесконечным числом мест ожидания. 
Вложенная цепь Маркова, которая рассматривается как модель процесса 
работы $M/G/1/r$, при любой стационарной смешанной пороговой стратегии 
имеет конечное чис\-ло состояний и~один эргодический класс. Тогда то, что 
задача~(\ref{e3-ag}) имеет решение и~оно принадлежит множеству чис\-тых 
стратегий, следует из~\cite{10-ag}. Поэтому с~практической точки зрения 
в~рассматриваемой выше задаче максимизации~(\ref{e3-ag}) (как 
и~в~аналогичных других задачах) достаточно оптимизировать только чистые 
пороговые стратегии. 
  
  Полученные в~данной работе результаты могут быть использованы для 
поиска оптимальных пороговых стратегий для систем, моделируемых 
с~по\-мощью СМО типа $M/G/1$ ($M/G/1/r$). 
  
  Дальнейшим развитием результатов данной статьи могут стать алгоритмы 
поиска оптимальных пороговых стратегий для СМО других типов, 
построенные с~использованием аналогичных математических приемов, 
например путем построения соотношений, аналогичных~(\ref{e18-ag}). 
     
{\small\frenchspacing
 {%\baselineskip=10.8pt
 \addcontentsline{toc}{section}{References}
 \begin{thebibliography}{99}
\bibitem{1-ag}
\Au{Welzl M.} Network congestion control.~--- New York, NY, USA: Wiley, 2005. 
282~p.
\bibitem{2-ag}
\Au{Печинкин А.\,В., Разумчик Р.\,В.} Время пребывания в~различных 
режимах системы обслуживания с~неординарными пуассоновскими 
входящими потоками, рекуррентным обслуживанием и~гистерезисной 
политикой~// Информационные процессы, 2015. Т.~15. №\,3. C.~324--336. 
\bibitem{6-ag} %3
\Au{Nino-Mora J.} Restless bandit marginal productivity indices, diminishing 
returns, and optimal control of make-to-order/make-to-stock $M/G/1$ queues~// 
Math. Oper. Res., 2006. Vol.~31. No.\,1. P.~50--84.
\bibitem{3-ag} %4
\Au{Жерновый Ю.\,В.} Решение задач оптимального синтеза для некоторых 
марковских моделей обслуживания~// Информационные процессы, 2010. Т.~10. 
№\,3. C.~257--274. 
\bibitem{4-ag} %5
\Au{Коновалов М.\,Г.} Об одной задаче оптимального управ\-ле\-ния нагрузкой на 
сервер~// Информатика и~её применения, 2013. Т.~7. Вып.~4. С.~34--43.
\bibitem{5-ag} %6
\Au{Агаларов Я.\,М.} Пороговая стратегия ограничения доступа к~ресурсам 
в~СМО $M/D/1$ с~функцией штрафов за несвоевременное обслуживание 
заявок~// Информатика и~её применения, 2015. Т.~9. Вып.~3. С.~56--65.

\bibitem{7-ag} %7
\Au{Гришунина Ю.\,Б.} Оптимальное управление очередью в~системе 
$M/G/1/\infty$ с~возможностью ограничения приема заявок~// Автоматика 
и~телемеханика, 2015. №\,3. С.~79--93. 
\bibitem{8-ag}
\Au{Карлин С.} Основы теории случайных процессов~/ Пер. с~англ.~--- М.: 
Мир, 1971. 536~с. (\Au{Karlin S.} A~first course in stochastic processes.~--- New 
York, NY, USA\,--\,London, U.K.: Academic Press, 1968. 502~p.) 
\bibitem{9-ag}
\Au{Бочаров П.\,П., Печинкин А.\,В.} Теория массового обслуживания.~--- М.: 
РУДН, 1995. 529~с.
\bibitem{10-ag}
\Au{Майн Х., Осаки С.} Марковские процессы принятия решений.~--- М.: 
Наука, 1977. 176~с.

\end{thebibliography}

 }
 }

\end{multicols}

\vspace*{-3pt}

\hfill{\small\textit{Поступила в~редакцию 18.02.16}}

\vspace*{8pt}

%\newpage

%\vspace*{-24pt}

\hrule

\vspace*{2pt}

\hrule

%\vspace*{8pt}



\def\tit{ABOUT THE OPTIMAL THRESHOLD OF~QUEUE LENGTH 
IN~A~PARTICULAR PROBLEM OF~PROFIT MAXIMIZATION 
IN~THE~$M/G/1$ QUEUING SYSTEM}

\def\titkol{About the optimal threshold of~queue length 
in~a~particular problem of~profit maximization 
in~the~$M/G/1$ queuing system}

\def\aut{Ya.\,M.~Agalarov$^1$, M.\,Ya.~Agalarov$^2$, and~V.\,S.~Shorgin$^1$}

\def\autkol{Ya.\,M.~Agalarov,  M.\,Ya.~Agalarov, and V.\,S.~Shorgin}

\titel{\tit}{\aut}{\autkol}{\titkol}

\vspace*{-9pt}

\noindent
$^1$Institute of Informatics Problems, Federal Research Center 
``Computer Science and Control'' of the Russian\linebreak
$\hphantom{^1}$Academy of Sciences,
44-2~Vavilov Str., Moscow 119333, Russian Federation

\noindent
$^2$PromsvyazBank OJSC, 10~Smirnovskaya Str., Moscow 109052, Russian
Federation

\def\leftfootline{\small{\textbf{\thepage}
\hfill INFORMATIKA I EE PRIMENENIYA~--- INFORMATICS AND
APPLICATIONS\ \ \ 2016\ \ \ volume~10\ \ \ issue\ 2}
}%
 \def\rightfootline{\small{INFORMATIKA I EE PRIMENENIYA~---
INFORMATICS AND APPLICATIONS\ \ \ 2016\ \ \ volume~10\ \ \ issue\ 2
\hfill \textbf{\thepage}}}

\vspace*{3pt}

    
  
  \Abste{The paper considers the problem of maximizing the average profit per time 
in the $M/G/1$ system on the set of access restriction stationary threshold strategies 
with one ``switch point.'' Profit in the described model is defined as the following measures: 
service fee, hardware maintenance fee, fine for service delay, fine for unhandled 
requests,  and fine for system idle. The conditions of existence of optimal and finite 
threshold values are obtained. The method and the algorithm for calculating the lower 
bound for the optimal threshold and corresponding value of maximal profit per time 
are proposed. The auxiliary problem of maximizing the system profit, averaged by 
number of handled requests on the set of the
considered threshold strategies, is solved. 
The necessary and sufficient conditions of existence of solution of the auxiliary 
problem are found. The method and algorithm for its solution are proposed.}
  
  \KWE{queuing system; threshold strategy; optimization}
  

  
  
\DOI{10.14357/19922264160208}

\vspace*{-9pt}

\Ack
\noindent
The work was partly supported by the
Russian Foundation for Basic Research (project 15-07-03406).


%\vspace*{3pt}

  \begin{multicols}{2}

\renewcommand{\bibname}{\protect\rmfamily References}
%\renewcommand{\bibname}{\large\protect\rm References}

{\small\frenchspacing
 {%\baselineskip=10.8pt
 \addcontentsline{toc}{section}{References}
 \begin{thebibliography}{99}

\bibitem{1-ag-1}
\Aue{Welzl, M.} 2005. \textit{Network congestion control}. New York, NY: 
Wiley. 282~p.
\bibitem{2-ag-1}
\Aue{Pechinkin, A.\,V., and R.\,V.~Razumchik}. 2015. Vremya prebyvaniya 
v~razlichnykh rezhimakh sistemy obsluzhivaniya s~neordinarnymi 
puassonovskimi vkhodyashchimi potokami, rekurrentnym obsluzhivaniem 
i~gisterezisnoy politikoy [First passage times between modes in the queueing 
system with batch Poisson arrivals, general service, and hysteresis policy]. 
\textit{Informatsionnye Protsessy} [Inform. Proc.]  15(3):324--336.

\bibitem{6-ag-1} %3
\Aue{Nino-Mora, J.} 2006. Restless bandit marginal productivity indices, 
diminishing returns, and optimal control of make-to-order/make-to-stock $M/G/1$ 
queues. \textit{Math. Oper. Res.} 31(1):50--84.

\bibitem{3-ag-1} %4
\Aue{Zhernovyy, Yu.\,V.} 2010. Reshenie zadach optimal'nogo sinteza dlya 
nekotorykh markovskikh modeley obsluzhivaniya [Solution of optimum synthesis 
problem for some Markov models of service]. \textit{Informatsionnye Protsessy} 
[Inform. Proc.] 10(3):257--274.
\bibitem{4-ag-1} %5
\Aue{Konovalov, M.\,G.} 2013. Ob odnoy zadache optimal'nogo upravleniya 
nagruzkoy na server [About one task of overload control]. \textit{Informatika i~ee 
Primeneniya}~--- \textit{Inform. Appl.} 7(4):34--43.
\bibitem{5-ag-1} %6
\Aue{Agalarov, Ya.\,M.} 2015. Porogovaya strategiya ogra\-ni\-che\-niya dostupa 
k~resursam v~SMO $M/D/1$ s~funk\-tsi\-ey shtrafov za nesvoevremennoe 
obsluzhivanie zayavok [The threshold strategy for restricting access in the 
$M/D/1$ queueing system with penalty function for late service]. 
\textit{Informatika i~ee Primeneniya}~--- \textit{Inform. Appl.} 9(3):\linebreak 56--65.

\bibitem{7-ag-1}
\Aue{Grishunina, Yu.\,B.} 2015. Optimal'noe upravlenie ochered'yu v~sisteme 
$M/G/1/\infty$ s~vozmozhnost'yu ogranicheniya priema zayavok [Optimal control 
of queue in the $M/G/1/\infty$ system with possibility of customer admission 
restriction]. \textit{Avtomatika i~Telemekhanika} [Automation Remote 
Control] 3:79--93.
\bibitem{8-ag-1}
\Aue{Karlin, S.} 1968. \textit{A~first course in stochastic processes}. New York, 
N.Y.\,--\,London: Academic Press. 502~p.
\bibitem{9-ag-1}
\Aue{Bocharov, P.\,P., and A.\,V.~Pechinkin}. 1995. \textit{Teoriya massovogo 
obsluzhivaniya} [Queueing theory]. Moscow: RUDN. 529~p.
\bibitem{10-ag-1}
\Aue{Mine, H., and S.~Osaki}. 1970. \textit{Markovian decision processes}. New 
York, NY: American Elsevier Publishing Co. 142~p.
  \end{thebibliography}

 }
 }

\end{multicols}

\vspace*{-3pt}

\hfill{\small\textit{Received February 18, 2016}}
   
  \Contr
  
\noindent
\textbf{Agalarov Yaver M.} (b.\ 1952)~--- Candidate of Science (PhD) in technology, associate 
professor; leading scientist, Institute of Informatics Problems, Federal Research Center ``Computer 
Science and Control'' of the Russian Academy of Sciences, 44-2~Vavilov Str., Moscow 119333, 
Russian Federation; \mbox{agglar@yandex.ru}


\vspace*{3pt}

\noindent
\textbf{Agalarov Murad Ya.} (b.\ 1987)~---
Head of System Analysis Department,
PromsvyazBank OJSC, 10~Smirnovskaya Str., Moscow 109052, Russian
Federation; murad-agalarov@yandex.ru

\vspace*{3pt}

\noindent
\textbf{Shorgin Vsevolod S.} (b.\ 1978)~--- 
Candidate of Science (PhD) in technology, senior scientist, Institute of 
Informatics Problems, Federal Research Center ``Computer Science and Control'' 
of the Russian Academy of Sciences, 44-2~Vavilov Str., Moscow 119333, 
Russian Federation; VShorgin@ipiran.ru 


\label{end\stat}


\renewcommand{\bibname}{\protect\rm Литература}
   