%\renewcommand{\le}{\leqslant}
%\renewcommand{\ge}{\geqslant}
\newcommand{\Pbf}{\mathbf P}
\newcommand{\Ebf}{\mathbf E\,}
\newcommand{\Dbf}{\mathbf D\,}
%\renewcommand{\Re}{\mathrm{Re}\,}

\def\stat{ushakov}

\def\tit{СИСТЕМА ОБСЛУЖИВАНИЯ С ГИПЕРЭКСПОНЕНЦИАЛЬНЫМ ВХОДЯЩИМ ПОТОКОМ  
И~ПРОФИЛАКТИКАМИ ПРИБОРА$^*$}

\def\titkol{Система обслуживания с~гиперэкспоненциальным входящим потоком  
и~профилактиками прибора}

\def\aut{В.\,Г.~Ушаков$^1$}

\def\autkol{В.\,Г.~Ушаков}

\titel{\tit}{\aut}{\autkol}{\titkol}

\index{Ушаков В.\,Г.}
\index{Ushakov V.\,G.}

{\renewcommand{\thefootnote}{\fnsymbol{footnote}} \footnotetext[1]
{Работа выполнена при финансовой поддержке Российского научного фонда
(проект 14-11-00397).}}


\renewcommand{\thefootnote}{\arabic{footnote}}
\footnotetext[1]{Факультет вычислительной математики 
и~кибернетики Московского государственного  университета
им.\ М.\,В.~Ломоносова; Институт проблем информатики Федерального исследовательского
центра <<Информатика и~управление>> Российской академии наук, vgushakov@mail.ru}

\vspace*{2pt}


\Abst{Изучена одноканальная система массового обслуживания с~бесконечным 
числом мест для ожидания, гиперэкспоненциальным входящим потоком и~профилактиками 
обслуживающего прибора при освобождении системы. Найдено нестационарное распределение 
числа требований в~системе. Профилактики прибора заключаются в~том, что в~момент 
освобождения системы от требований прибор на случайное время с~заданным распределением 
становится недоступным для обслуживания. Если за время профилактики поступает хотя бы 
одно требование, начинается нормальное функционирование системы. Если требования не 
поступают, то прибор отправляется на новую профилактику. Такие системы хорошо описывают 
функционирование большого числа реальных вычислительных и~информационных систем. 
В~частности, такие модели можно использовать при анализе систем, в~которых наряду 
с~основными требованиями имеются второстепенные. Второстепенные требования всегда 
присутствуют в~системе, а~их обслуживание может проводиться только тогда, когда нет 
основных, т.\,е.\ в~фоновом режиме.}

\KW{гиперэкспоненциальный поток; профилактики обслуживающего прибора; одноканальная 
система; длина очереди}

\DOI{10.14357/19922264160211} 

\vspace*{6pt}

\vskip 12pt plus 9pt minus 6pt

\thispagestyle{headings}

\begin{multicols}{2}

\label{st\stat}

\section{Введение}

Исследованию систем массового обслуживания, в~которых при осуществлении 
определенных событий обслуживающий прибор становится на случайное время 
полностью или частично недоступным, уделяется в~последнее время большое 
внимание. В~работах~[1--4] можно найти обзор известных результатов, большое 
число постановок задач, описание различных приложений и~обширную библиографию.

 Наиболее изученным и~важным для приложений является случай, когда таким событием 
 является освобождение
системы от требований. В~англоязычной литературе для такой особенности работы 
системы принят термин vacation. В~литературе на русском языке общепринятого термина 
нет. Чаще всего используют один из следующих: каникулы, прогулки, профилактики 
обслуживающего прибора. Будем использовать последнее название.

Системы обслуживания с~профилактиками прибора хорошо описывают 
функционирование большого числа реальных вычислительных и~информационных систем. 
В~частности, такие модели можно использовать при анализе систем, в~которых наряду 
с~основными требованиями имеются второстепенные. Второстепенные требования всегда 
присутствуют в~системе, а~их обслуживание может проводиться только тогда, когда 
нет основных, т.\,е.\ в~фоновом режиме.

В настоящей работе исследуется длина очереди в~нестационарном режиме 
в~однолинейной системе с~ожиданием и~гиперэкспоненциальным входящим потоком.
С~пуассоновским потоком аналогичная система обслуживания исследована в~\cite{2-u}.

\section{Описание модели}

В однолинейную систему обслуживания с~неограниченным числом мест для ожидания 
поступает гиперэкспоненциальный поток требований с~функцией распределения 
интервалов между поступлениями вида:
\begin{multline*}
A(t)=\sum\limits_{i=1}^k c_i\left(1-e^{-a_it}\right)\,,\ t>0\,,\ 
a_i>0\,,\\
 c_i>0\,,\ \sum\limits_{i=1}^kc_i=1\,.
\end{multline*}
Гиперэкспоненциальный поток можно рассматривать как пуассоновский поток 
со случайной интенсивностью~$a,$ которая принимает~$k$~различных значений 
$a_1,\ldots,a_k$  с~вероятностями $c_1,\ldots,c_k.$ Текущее значение~$a$ 
разыгрывается в~момент поступления требования и~не меняется между двумя 
соседними поступлениями. Введем случайный процесс $j(t)$ такой, что если $a\hm=a_j$ 
в~момент времени~$t,$ то $j(t)\hm=j.$
{\looseness=1

}

Предполагаем, что длительности обслуживания требований~--- независимые в~совокупности 
и~не зависящие от входящего потока случайные величины с~функцией распределения $B(x).$
 Если в~некоторый момент времени система освободилась от требований, то обслуживающий 
 прибор отправляется на профилактику, которая длится случайное время с~функцией 
 распределения $C(x).$ Не ограничивая общности, будем считать, что $B(x)\hm<1$, $C(x)\hm<1$  
 для любого~$x$ и~существуют плотности распределения $b(x)$ и~$c(x)$  
 (окончательные результаты от этих предположений не зависят, но их 
 доказательства становятся технически более простыми и~менее громоздкими). Обозначим:
\begin{align*}
  \beta(s)&=\int\limits_0^{\infty}e^{-sx}b(x)\,dx\,,\\  
  \gamma(s)&=\int\limits_0^{\infty}e^{-sx}c(x)\,dx\,,\\ 
  \psi(s)&=\int\limits_0^{\infty}e^{-sx}\,dx
  \int\limits_0^{\infty}\fr{c(u+x)d(u)}{1-C(u)}\,du\,.
  \end{align*}
  Пока прибор находится на профилактике, он не доступен для обслуживания. 
  Если за время профилактики поступают требования, после ее завершения 
  начинается их обслуживание. Если ни одного требования не поступает, 
  то прибор отправляется на новую профилактику. Длительности различных 
  профилактик являются независимыми случайными величинами и~не зависят 
  от входящего потока и~времен обслуживания.
  Основным объектом изучения будет случайный процесс $L(t)$~--- число требований 
  в~системе в~момент времени~$t.$ При его исследовании  потребуются свойства ряда 
  функций, определяемых через параметры системы. Рас\-смот\-рим многочлен по~$\mu$ 
  степени~$k$~вида:
\begin{equation}
\label{1-u}
\prod\limits_{i=1}^k\left(\mu+a_i\right)-z\sum\limits_{j=1}^kc_ja_j\prod\limits_{i\ne j}
\left(\mu+a_i\right)\,.
\end{equation}
Занумеруем его корни $\mu_1(z),\ldots,\mu_k(z)$ таким образом, чтобы они были 
непрерывными функциями~$z$ и~$\mu_1(1)\hm=0.$ Тогда  $\mathrm{Re}\, \mu_j(z)\hm<0$, 
$|z|\hm<1$,  $\mu_i(z)\hm\ne \mu_j(z),$ $ i\hm\ne j$, $j=1,\ldots,k.$
Обозначим
$$
\alpha_m(z)=\prod\limits_{j\ne m}\left(\mu_m(z)-\mu_j(z)\right)\,.
$$

В дальнейшем  понадобится следующая лемма.

\smallskip

\noindent
\textbf{Лемма~1.}\
\textit{При каждом $m$, $m\hm=1,\ldots,k,$ уравнение
$$
z=\beta\left(s-\mu_m(z)\right)
$$
имеет в~области $\mathrm{Re}\, s\hm>0$ единственное решение $z\hm=z_m(s)$ такое, что} 
$|z_m(s)|\hm<1.$


\section{Распределение длины очереди}

Для нахождения распределения случайного процесса $L(t)$  потребуются 
и~другие процессы, связанные с~функционированием системы обслуживания.
Пусть $\nu(t)\hm=1,$ если в~момент времени~$t$ прибор занят обслуживанием требования, 
и~$\nu(t)\hm=0,$ если в~момент времени~$t$ прибор находится на профилактике.
Случайный процесс $x(t)$ определим следующим образом. Если $L(t)\hm>0$ 
и~$\nu(t)\hm=1,$ то $x(t)$ есть время, прошедшее с~начала обслуживания 
требования, находящегося на приборе, до момента~$t.$  Если $\nu(t)\hm=0,$ то $x(t)$ 
есть время, прошедшее с~начала профилактики прибора до момента~$t.$

Случайный процесс $(L(t),j(t),x(t),\nu(t))$ является однородным 
марковским процессом. Положим:
\begin{multline*}
P_j(n,x,t)={}\\
{}=\fr{\partial}{\partial x}\,\Pbf(L(t)=n,j(t)=j,\nu(t)=1,x(t)<x)\,,\\ 
n>0\,,\ x\ge 0\,,\ j=1,\ldots,k\,;
\end{multline*}

\vspace*{-12pt}

\noindent
\begin{multline*}
Q_j(n,x,t)={}\\
{}=\fr{\partial}{\partial x}\,\Pbf(L(t)=n,j(t)=j,\nu(t)=0,x(t)<x)\,,\\ 
n\ge 0\,,\ x\ge 0\,,\ j=1,\ldots,k\,;
\end{multline*}
\begin{gather*}
\eta(x)=\fr{b(x)}{1-B(x)}\,;\ \ \varphi(x)=\fr{c(x)}{1-C(x)}\,;\\ 
\delta_{i,j}=\begin{cases}
1\,, &\ i=j\,;\\ 
0\,, &\ i\ne j\,.
\end{cases}
\end{gather*}
Функции $P_j(n,x,t)$ и~$Q_j(n,x,t)$ удовлетворяют при $x\hm>0$ следующим системам 
дифференциальных уравнений:
\begin{multline}
\label{3-u}
\fr{\partial P_j(n,x,t)}{\partial t}+
\fr{\partial P_j(n,x,t)}{\partial x}={}\\
{}=
-\left(a_j+\eta(x)\right)
P_j(n,x,t)+{}\\
{}+\left(1-\delta_{n,1}\right)c_j\sum\limits_{l=1}^ka_lP_l(n-1,x,t)\,;
\end{multline}

\noindent
\begin{multline}
\label{4-u}
\fr{\partial Q_j(n,x,t)}{\partial t}+
\fr{\partial Q_j(n,x,t)}{\partial x}={}\\
{}=
-\left(a_j+\varphi(x)\right)Q_j(n,x,t)+{}\\
{}+
\left(1-\delta_{n,0}\right)c_j\sum\limits_{l=1}^ka_lQ_l(n-1,x,t)
\end{multline}
и краевым условиям при $x\hm=0$:
\begin{multline}
\label{5-u}
Q_j(n,0,t)=0\,,\ \ n>0\,,\ \ 
Q_j(0,0,t)={}\\
{}=\int\limits_0^{\infty}Q_j(0,x,t)\varphi(x)\,dx+
\int\limits_0^{\infty}P_j(1,x,t)\eta(x)\,dx\,;
\end{multline}

\vspace*{-12pt}

\noindent
\begin{multline}
\label{6-u}
P_j(n,0,t)=\int\limits_0^{\infty}P_j(n+1,x,t)\eta(x)\,dx+{}\\
{}+
\int\limits_0^{\infty}Q_j(n,x,t)\varphi(x)\,dx\,.
\end{multline}
Будем предполагать, что в~начальный момент времени $t\hm=0$ 
система свободна от требований, а~с~начала профилактики прибора 
прошло случайное время с~заданным распределением с~плот\-ностью~$d(x).$ Таким образом,
\begin{gather*}
P_j(n,x,0)=0\,,\ n>0\,;\\ 
Q_j(n,x,0)=\delta_{n,0}c_jd(x)\,,\ \ j=1,\ldots,k\,.
\end{gather*}
Положим
\begin{align*}
p_j(z,x,s)&=\sum\limits_{n=1}^{\infty}z^n\int\limits_0^{\infty}e^{-st}P_j(n,x,t)\,dt\,;
\\ 
q_j(z,x,s)&=\sum\limits_{n=0}^{\infty}z^n\int\limits_0^{\infty}e^{-st}Q_j(n,x,t)\,dt\,.
\end{align*}
Тогда, учитывая начальные условия,  из~\eqref{3-u} и~\eqref{4-u} получаем:
\begin{multline}
\label{7-u}
\fr{\partial p_j(z,x,s)}{\partial x}=
-\left(s+a_j+\eta(x)\right)p_j(z,x,s)+{}\\
{}+c_jz\sum\limits_{l=1}^ka_lp_l(z,x,s)\,;
\end{multline}

\vspace*{-12pt}

\noindent
\begin{multline}
\label{8-u}
\fr{\partial q_j(z,x,s)}{\partial x}=
-\left(s+a_j+\varphi(x)\right)q_j(z,x,s)+{}\\
{}+c_jz\sum\limits_{l=1}^ka_lq_l(z,x,s)+c_jd(x)\,.
\end{multline}
Решения \eqref{7-u} и~\eqref{8-u} имеют вид:
\begin{multline}
\label{9-u}
p_j(z,x,s)={}\\
{}= (1-B(x))c_j\sum\limits_{m=1}^k
\fr{\gamma^{(m)}(z,s)}{\mu_m(z)+a_j}\,e^{-(s-\mu_m(z))x}\,;
\end{multline}

\noindent
\begin{multline}
q_j(z,x,s)=\left(1-C(x)\right)c_j\sum\limits_{m=1}^k\ e^{-(s-\mu_m(z))x}\times\\
\times
\Bigg(\delta^{(m)}(z,s)+\alpha_m^{-1}(z)\prod\limits_{l=1}^k
\left(\mu_m(z)+a_l\right)\times{}\\
{}\times \int\limits_0^x
e^{(s-\mu_m(z))u} \fr{d(u)}{1-C(u)}\,{du}\Bigg)\! \Bigg/ \!
({\mu_m(z)+a_j}),\!\! \label{10-u}
\end{multline}
где функции $\gamma^{(m)}(z,s)$  и~$\delta^{(m)}(z,s)$ являются 
произвольными функциями указанных переменных и~определяются из краевых условий. 
Переходя в~\eqref{5-u} и~\eqref{6-u} к преобразованиям Лапласа 
по~$t$ и~производящим функциям, получаем:

\noindent
\begin{multline}
\label{11-u}
p_j(z,0,s)=z^{-1}\int\limits_0^{\infty}p_j(z,x,s)\eta(x)\,dx+{}\\
{}+
\int\limits_0^{\infty}q_j(z,x,s)\varphi(x)dx-q_j(z,0,s)\,.
\end{multline}
Заметим, что $q_j(z,0,s)$ не зависит от $z$, т.\,е.\ $q_j(z,0,s)=q_j(s).$
Подставляя \eqref{9-u} и~\eqref{10-u} в~\eqref{11-u}, получаем:

\noindent
\begin{multline*}
\sum\limits_{m=1}^k
\bigg(\fr{\gamma^{(m)}(z,s)\left(1-z^{-1}\beta(s-\mu_m(z))\right)}
{\mu_m(z)+a_j}+{}\\
{}+
\fr{\delta^{(m)}(z,s)(1-\gamma(s-\mu_m(z)))}{\mu_m(z)+a_j}\bigg)
={}\\
{}=\sum\limits_{m=1}^k\fr
{\alpha_m^{-1}(z)\prod\limits_{l=1}^k(\mu_m(z)+a_l)\psi(s-\mu_m(z))}{\mu_m(z)+a_j}\,,\\ 
j=1,\ldots,k\,.
\end{multline*}
Отсюда

\noindent
\begin{multline}
\label{12-u}
\gamma^{(m)}(z,s)\left(1-z^{-1}\beta(s-\mu_m(z))\right)={}\\
{}=\delta^{(m)}(z,s)\left(\gamma(s-\mu_m(z))-1\right)+{}\\
{}+
\alpha_m^{-1}(z)\prod\limits_{l=1}^k(\mu_m(z)+a_l)\psi\left(s-\mu_m(z)\right)\,.
\end{multline}
В силу леммы~1 левая часть~\eqref{12-u} обращается в~0 при $z\hm=z_m(s).$ 
Следовательно,

\noindent
\begin{multline}
\label{13-u}
\delta^{(m)}(z_m(s),s)=
\fr{\psi(s-\lambda_m(s))}{\alpha_m(z_m(s))(1-\gamma(s-\lambda_m(s)))}\times{}\\
{}\times
\prod\limits_{l=1}^k\left(\lambda_m(s)+a_l\right)\,.
\end{multline}

\vspace*{-12pt}

\pagebreak

\noindent
Здесь $\lambda_m(s)\hm=\mu_m(z_m(s)).$ Из~\eqref{10-u} следует, что
$$
q_j(s)=c_j\sum\limits_{m=1}^k\fr{\delta^{(m)}(z,s)}{\mu_m(z)+a_j}\,,\ j=1,\ldots,k \,.
$$
Решая эту систему уравнений относительно $\delta^{(m)}(z,s),$ получаем:
\begin{equation}
\label{n1}
\delta^{(m)}(z,s)=z\fr{\prod\limits_{j=1}^k(\mu_m(z)+a_j)}{\alpha_m(z)}
\sum\limits_{l=1}^k\fr{a_lq_l(s)}{\mu_m(z)+a_l}\,.
\end{equation}
Подставляя в~\eqref{n1} $z\hm=z_m(s),$ имеем:
\begin{multline}
\label{14-u}
\delta^{(m)}(z_m(s),s)={}\\
{}=z_m(s)\fr{\prod\limits_{j=1}^k(\lambda_m(s)+a_j)}
{\alpha_m(z_m(s))}\sum\limits_{l=1}^k\fr{a_lq_l(s)}{\lambda_m(s)+a_l}\,.
\end{multline}
Сравнивая два представления~\eqref{13-u} и~\eqref{14-u} для $\delta^{(m)}(z_m(s),s),$ 
получаем систему уравнений для~$q_l(s)$:
\begin{multline*}
\sum\limits_{l=1}^k\fr{a_lq_l(s)}{\lambda_m(s)+a_l}=
\fr{\psi(s-\lambda_m(s))}{z_m(s)(1-\gamma(s-\lambda_m(s)))}\,,\\
m=1,\ldots,k\,,
\end{multline*}
из которой находим:
\begin{multline}
\label{15-u}
q_l(s)=c_l\prod\limits_{j=1}^k\left(\lambda_l(s)+a_j\right)\times{}\\
{}\times
\sum\limits_{m=1}^k
\left(
\vphantom{\fr{1}{\prod\limits_{n\ne m}(\lambda_m(s)-\lambda_n(s))}}
 \fr{\psi(s-\lambda_m(s))}{ 
 \left(1-
\gamma(s-\lambda_m(s))\right)\left(\lambda_m(s)+a_l\right)}\times{}\right.\\
\left.{}\times \fr{1}{\prod\limits_{n\ne m}(\lambda_m(s)-\lambda_n(s))}\right)\,.
\end{multline}
Подставляя \eqref{15-u} в~\eqref{n1} и~учитывая~\eqref{1-u}, получаем:
\begin{multline}
\label{16-u}
\delta^{(m)}(z,s)={}\\
{}=\fr{z}{\alpha_m(z)}\sum\limits_{j=1}^k\fr{\psi(s-\lambda_j(s))\prod\limits_{l=1}^k(\lambda_j(s)+a_l)}
{z_j(s)(1-\gamma(s-\lambda_j(s)))}\times{}\\
{}\times
\prod\limits_{\nu\ne j}\fr{\mu_m(z)-\lambda_{\nu}(s)}{\lambda_j(s)-\lambda_{\nu}(s)}\,.
\end{multline}
Таким образом, все величины, определяющие функции $p_j(z,x,s)$ и~$q_j(z,x,s),$ 
найдены.
Заметим, что преобразование Лапласа производящей функции числа требований в~системе 
можно найти по формуле:
\begin{multline*}
p(z,s)=\int\limits_0^{\infty}e^{-st}
\Ebf z^{L(t)}\,dt={}\\
{}=\sum\limits_{j=1}^k\int\limits_0^{\infty}(p_j(z,x,s)+q_j(z,x,s))\,dx\,.
\end{multline*}
Подставляя сюда~\eqref{9-u} и~\eqref{10-u}, имеем:
\begin{multline*}
p(z,s)=\sum\limits_{j=1}^k\sum\limits_{m=1}^k
\fr{c_j}{(\mu_m(z)+a_j)(s-\mu_m(z))}\times{}\\
{}\times
\left(
\vphantom{\prod\limits_{l=1}^k}
\gamma^{(m)}(z,s)(1-\beta(s-\mu_m(z)))+{}\right.\\
{}+\delta^{(m)}(z,s)\left(1-\gamma(s-\mu_m(z))\right)+{}\\
\left.{}+
\prod\limits_{l=1}^k(\mu_m(z)+a_l)\alpha_m^{-1}(z)
\left(1-\psi(s-\mu_m(z))\right)\right).
\end{multline*}
Далее, из~\eqref{12-u} имеем:
\begin{multline*}
\gamma^{(m)}(z,s)=\fr{\delta^{(m)}(z,s)(\gamma(s-\mu_m(z))-1)}
{1-z^{-1}\beta(s-\mu_m(z))}+{}\\
{}+
\fr{\prod\limits_{l=1}^k(\mu_m(z)+a_l)\psi(s-\mu_m(z))}{\alpha_m(z)
\left(1-z^{-1}\beta\left(s-\mu_m(z)\right)\right)}\,.
\end{multline*}
Значит,
\begin{multline*}
p(z,s)=\sum\limits_{j=1}^k\sum\limits_{m=1}^k\fr{c_j}{(\mu_m(z)+a_j)(s-\mu_m(z))}\times{}\\
{}\times
\Bigg(
\fr{1-\beta(s-\mu_m(z))}{1-z^{-1}\beta(s-\mu_m(z))}\times{}\\
{}\times\left(
\vphantom{\fr{\prod\limits_{l=1}^k(\mu_m(z)+a_l)\psi(s-\mu_m(z))}{\alpha_m(z)}}
\delta^{(m)}(z,s)(\gamma(s-\mu_m(z))-1)+{}\right.\\
\left.{}+
\fr{\prod\limits_{l=1}^k(\mu_m(z)+a_l)\psi(s-\mu_m(z))}{\alpha_m(z)}\right)+{}\\
{}+
\delta^{(m)}(z,s)\left(1-\gamma(s-\mu_m(z))\right)+{}
\end{multline*}

\noindent
\begin{multline}
\label{17-u}
{}+
\prod\limits_{l=1}^k(\mu_m(z)+a_l)\alpha_m^{-1}(z)
\left(1-\psi(s-\mu_m(z))\right)\Bigg)={}\\
{}=
\sum\limits_{j=1}^k\sum\limits_{m=1}^k\fr{c_j}{(\mu_m(z)+a_j)(s-\mu_m(z))}
\left(
\vphantom{\fr{(1-z)\beta(s-\mu_m(z))\psi(s-\mu_m(z))}{z-\beta(s-\mu_m(z))}}
\delta^{(m)}(z,s)\times{}\right.\\
{}\times\fr{(z-1)(1-\gamma(s-\mu_m(z)))\beta(s-\mu_m(z))}
{z-\beta(s-\mu_m(z))}+{}\\
\hspace*{-2mm}{}+
\fr{\prod\limits_{l=1}^k(\mu_m(z)+a_l)}{\alpha_m(z)}\times{}\\
\left.{}\times
\left(1+\fr{(1-z)\beta(s-\mu_m(z))\psi(s-\mu_m(z))}{z-\beta(s-\mu_m(z))}\right)\right).\!\!\!
\end{multline}
Обозначим
$$
\tau_m(z)=\sum\limits_{j=1}^k\fr{c_j}{\mu_m(z)+a_j}\,,\ m=1,\ldots,k\,.
$$
Подставляя теперь~\eqref{16-u} в~\eqref{17-u}, окончательно получаем:
\begin{multline*}
p(z,s)=\sum\limits_{m=1}^k\fr{\tau_m(z)}{\alpha_m(z)(s-\mu_m(z))}\times{}\\
{}\times
\Bigg(\fr{z(z-1)(1-\gamma(s-\mu_m(z)))\beta(s-\mu_m(z))}{z-\beta(s-\mu_m(z))}\times{}\\
\end{multline*}

\noindent
\begin{multline*}
{}\times
\sum\limits_{j=1}^k\fr{\psi(s-\lambda_j(s))\prod\limits_{l=1}^k(\lambda_j(s)+a_l)}
{z_j(s)(1-\gamma(s-\lambda_j(s)))}\times{}\\
{}\times \prod\limits_{\nu\ne j}
\fr{\mu_m(z)-\lambda_{\nu}(s)}{\lambda_j(s)-\lambda_{\nu}(s)}+
\prod\limits_{l=1}^k(\mu_m(z)+a_l)\times{}\\
\left.{}\times\left(
1+\fr{(1-z)\beta(s-\mu_m(z))\psi(s-\mu_m(z))}{z-\beta(s-\mu_m(z))}\right)\right)\,.
\end{multline*}

{\small\frenchspacing
 {%\baselineskip=10.8pt
 \addcontentsline{toc}{section}{References}
 \begin{thebibliography}{99}
\bibitem{1-u}
\Au{Doshi B.\,T.} 
Queueing systems with vacations~--- a~survey~// Queueing Syst., 1986. 
Vol.~1.  P.~29--66.
\bibitem{2-u}
\Au{Takagi H.} 
Time-dependent analysis of $M|G|1$ vacation models with exhaustive service~// 
Queueing Syst., 1990. Vol.~6.  P.~369--390.
\bibitem{3-u}
\Au{Li J., Tian N., Zhang~Z.\,G., Luh~H.\,P.} 
Analysis of the $M|G|1$ queue with exponentially working vacations~--- 
a~matrix analytic approach~// Queueing Syst., 2009. Vol.~61. P.~139--166.
\bibitem{4-u}
\Au{Bouman N., Borst~S.\,C., Boxma~O.\,J., Leeuwaarden~J.\,S.\,H.} 
Queues with random back-offs~// Queueing Syst., 2014. Vol.~77. P.~33--74.

\end{thebibliography}

 }
 }

\end{multicols}

\vspace*{-3pt}

\hfill{\small\textit{Поступила в~редакцию 05.03.16}}

\vspace*{8pt}

%\newpage

%\vspace*{-24pt}

\hrule

\vspace*{2pt}

\hrule

%\vspace*{8pt}



\def\tit{QUEUEING SYSTEM WITH WORKING VACATIONS AND~HYPEREXPONENTIAL INPUT STREAM}

\def\titkol{Queueing system with working vacations and hyperexponential input stream}

\def\aut{V.\,G.~Ushakov$^{1,2}$}

\def\autkol{V.\,G.~Ushakov}

\titel{\tit}{\aut}{\autkol}{\titkol}

\vspace*{-9pt}

\noindent
$^1$Department of Mathematical 
Statistics, Faculty of Computational Mathematics and Cybernetics,\linebreak
$\hphantom{^1}$M.\,V.~Lomonosov 
Moscow State University, 1-52~Leninskiye Gory, Moscow 119991, GSP-1, Russian\linebreak 
$\hphantom{^1}$Federation

\noindent
$^2$Institute of Informatics Problems, Federal 
Research Center ``Computer Science and Control'' of the Russian\linebreak
$\hphantom{^1}$Academy of Sciences, 
44-2~Vavilov Str., Moscow 119333, Russian Federation

\def\leftfootline{\small{\textbf{\thepage}
\hfill INFORMATIKA I EE PRIMENENIYA~--- INFORMATICS AND
APPLICATIONS\ \ \ 2016\ \ \ volume~10\ \ \ issue\ 2}
}%
 \def\rightfootline{\small{INFORMATIKA I EE PRIMENENIYA~---
INFORMATICS AND APPLICATIONS\ \ \ 2016\ \ \ volume~10\ \ \ issue\ 2
\hfill \textbf{\thepage}}}

\vspace*{3pt}


\Abste{The time-dependent process in the single server vacation model with 
hyperexponential input stream is analyzed. The Laplace transform
(with respect to an arbitrary point in time)  of the joint distribution of server state, queue size, and elapsed time in that state
is obtained. The author restricts themselves to a~system with
exhaustive service (the queue must be empty when the server starts a vacation). The queueing systems with vacations have been well studied because of their applications in modeling
the computer networks, communication, and manufacturing systems. For example, in many 
digital systems, the processor is multiplexed among a number of jobs and, hence, is not available
all the time to handle one job type. Besides such an application, theoretical 
interest in vacation models has arousen with respect to their relationship with polling models.}

\KWE{hyperexponential input stream; working vacations; single server; queue length}

\DOI{10.14357/19922264160211}

%\vspace*{-12pt}

\Ack
\noindent
This work was supported by the Russian Science Foundation (project 14-11-00397).



%\vspace*{3pt}

  \begin{multicols}{2}

\renewcommand{\bibname}{\protect\rmfamily References}
%\renewcommand{\bibname}{\large\protect\rm References}

{\small\frenchspacing
 {%\baselineskip=10.8pt
 \addcontentsline{toc}{section}{References}
 \begin{thebibliography}{9}




\bibitem{1-u-1}
\Aue{Doshi, B.\,T.} 1986. 
Queueing systems with vacations~--- a~survey. \textit{Queueing Syst.} 1:29--66.
\bibitem{2-u-1}
\Aue{Takagi, H.} 1990. Time-dependent analysis of $M|G|1$ vacation models 
with exhaustive service. \textit{Queueing Syst.} 6:369--390.
\bibitem{3-u-1}
\Aue{Li, J., N.~Tian, Z.\,G.~Zhang, and H.\,P.~Luh}. 2009. 
Analysis of the $M|G|1$ queue with exponentially working vacations~--- 
a~matrix analytic approach. \textit{Queueing Syst.} 61:139--166.
\bibitem{4-u-1}
\Aue{Bouman, N.,  S.\,C.~Borst,  O.\,J.~Boxma, 
and J.\,S.\,H.~Leeuwaarden}. 2014. 
Queues with random back-offs.  \textit{Queueing Syst.} 77:33--74. 
\end{thebibliography}

 }
 }

\end{multicols}

\vspace*{-3pt}

\hfill{\small\textit{Received March 5, 2016}}


\Contrl

\noindent
\textbf{Ushakov Vladimir G.} (b.\ 1952)~---
Doctor of Science in physics and mathematics, professor, Department of Mathematical 
Statistics, Faculty of Computational Mathematics and Cybernetics, M.\,V.~Lomonosov 
Moscow State University, 1-52~Leninskiye Gory, Moscow 119991, GSP-1, Russian 
Federation; senior scientist, Institute of Informatics Problems, Federal 
Research Center ``Computer Science and Control'' of the Russian Academy of Sciences, 
44-2~Vavilov Str., Moscow 119333, Russian Federation; \mbox{vgushakov@mail.ru} 

\label{end\stat}


\renewcommand{\bibname}{\protect\rm Литература}