\newcommand{\fD}{\mathfrak{D}}

\def\stat{goncharov}

\def\tit{МЕТРИЧЕСКАЯ КЛАССИФИКАЦИЯ ВРЕМЕННЫХ РЯДОВ СО~ВЗВЕШЕННЫМ 
ВЫРАВНИВАНИЕМ ОТНОСИТЕЛЬНО ЦЕНТРОИДОВ КЛАССОВ$^*$}

\def\titkol{Метрическая классификация временных рядов со взвешенным 
выравниванием относительно центроидов классов}

\def\aut{А.\,В.~Гончаров$^1$,  В.\,В.~Стрижов$^2$}

\def\autkol{А.\,В.~Гончаров,  В.\,В.~Стрижов}

\titel{\tit}{\aut}{\autkol}{\titkol}

\index{Гончаров А.\,В.}
\index{Стрижов В.\,В.}
\index{Goncharov A.\,V.}
\index{Strijov V.\,V.}

{\renewcommand{\thefootnote}{\fnsymbol{footnote}} \footnotetext[1]
{Работа выполнена при финансовой поддержке РФФИ (проект 16-07-01163) 
и~в~рамках инициативы <<MIT-Сколтех>>.}}


\renewcommand{\thefootnote}{\arabic{footnote}}
\footnotetext[1]{Московский физико-технический институт, alex.goncharov@phystech.edu}
\footnotetext[2]{Вычислительный центр им.\ А.\,А.~Дородницына Федерального 
исследовательского центра 
<<Информатика и~управ\-ле\-ние>> Российской академии наук, strijov@ccas.ru}


\Abst{Рассматривается задача метрического анализа 
и~классификации временных рядов. Мет\-ри\-че\-ские методы используют матрицу попарных 
расстояний, строящуюся при помощи фиксированной функции расстояния. Вычислительная 
сложность алгоритмов, использующих такую матрицу, по меньшей мере квадратична 
относительно числа временных рядов. Проблема снижения вычислительной сложности 
решается путем предварительного выделения эталонных объектов, центроидов классов 
и~последующего их использования для описания классов. В~качестве базовой модели 
классификации выбрана модель, использующая динамическое выравнивание временных 
рядов для построения центроида. В~работе предлагается ввести функцию весов центроида, 
влияющую на вычисление расстояния между объектами. Для анализа алгоритма построения 
центроида использованы как временные ряды элементарных функций, так и~временные ряды 
физической активности человека с~акселерометра мобильного телефона. Свойства 
построенной модели исследуются и~сравниваются со свойствами модели, выбранной 
в~качестве базовой.}

\KWE{взвешенное динамическое выравнивание; классификация временных рядов; 
центроид; функция расстояния}

\DOI{10.14357/19922264160204} 

%\vspace*{-4pt}

\vskip 10pt plus 9pt minus 6pt

\thispagestyle{headings}

\begin{multicols}{2}

\label{st\stat}

\section{Введение}
 Рассматривается задача анализа и~классификации временн$\acute{\mbox{ы}}$х рядов. 
 Существуют различные спосо\-бы ее решения: построение признакового пространства, 
 использование нейронных сетей, аппроксимация параметрическими функциями. Так, 
 в~\cite{Kuznetsov} исследованы методы построения признакового описания временн$\acute{\mbox{ы}}$х 
 рядов, в~частности метод экспертного построения признаков и~метод построения 
 признакового описания на основе гипотезы порождения данных. Результаты~\cite{Kuznetsov} 
 показывают, что построенное признаковое пространство адекватно описывает 
 зависимую переменную. В~\cite{Popova2015} для решения задачи классификации 
 использованы нейронные сети с~небольшим числом связей между нейронами, 
 обладающие свойством устойчивости к~возмущениям данных. 
 В~\cite{Ignatov} для классификации предложен алгоритм разбиения исходных 
 временн$\acute{\mbox{ы}}$х рядов на периоды и~их очистки от шумов. Предложены модификации 
 алгоритма~$k$~ближайших соседей и~нейронной сети для решения по\-став\-лен\-ной задачи. 
 В~вычислительном эксперименте, проведенном на реальных данных, оценена эффективность, 
 а~также проведено сравнение данных алгоритмов между собой. При этом показан высокий 
 процент правильной классификации.

Построение матрицы попарных расстояний между всеми объектами 
в~задаче метрической классификации является вычислительно трудоемкой задачей. 
Для снижения размерности задачи и~вычислительных затрат предлагается решать задачу 
с~предварительным выделением эталонных объектов, или же центроидов классов, 
и~после\-ду\-ющим их использованием для описания множества вре\-мен\-н$\acute{\mbox{ы}}$х рядов.

Метрические методы используют различные функции расстояния для построения 
матрицы попарных расстояний: евклидово расстояние~\cite{Faloutsos1994}, 
метод динамического выравнивания временн$\acute{\mbox{ы}}$х рядов~\cite{Berndt1994,Keogh2005Exact}, 
метод, основанный на нахождении наиболь\-шей общей 
последовательности~\cite{Vlachos2002discovering}, Edit Distance with Real 
Penalty~\cite{Chen2004On}, Edit Distance on Real sequence~\cite{Chen2005Robust}, 
DISSIM~\cite{Frentzos2007Index}, Sequence Weighted Alignment model~\cite{Morse2007An}, 
Spatial Assembling Distance~\cite{Chen2007Spade} и~др. В~качестве базового метода 
для построения функции расстояния в~настоящей работе предлагается использовать 
динамическое выравнивание временн$\acute{\mbox{ы}}$х рядов (Dynamic Time Warping)~\cite{Keogh}. 
Как показано в~\cite{Salvador}, этот метод находит наилучшее соответствие между 
двумя временн$\acute{\mbox{ы}}$ми рядами, если они нелинейно деформированы друг относительно 
друга --- растянуты, сжаты или смещены вдоль оси времени.

Базовой моделью классификации и~анализа временн$\acute{\mbox{ы}}$х рядов принята модель, 
описанная в~\cite{Goncharov}. Там в~качестве центроида выбирается объект выборки, 
являющийся ближайшим ко всем остальным объектам. Применяться же будет метод точного 
его вычисления. Это метод DBA (Dynamic Bandwidth Allocation), 
ре\-ша\-ющий задачу оптимизации для нахождения центроида. 
Алгоритм применения и~доказательство корректности приведены в~\cite{DBA}.

В настоящей работе вводятся понятия вектора весов и~матрицы весов центроида 
и~описываются методы vwDTW и~mwDTW (векторно- и~мат\-рич\-но-взве\-шен\-ный DTW) 
вычисления функции расстояния, основывающиеся на следующем предположении о форме 
временн$\acute{\mbox{ы}}$х рядов: в~одном и~том же классе находятся временн$\acute{\mbox{ы}}$е ряды, имеющие схожую форму 
с~точностью до линейной или нелинейной деформации, локальных или глобальных сдвигов 
по оси времени. Предполагается, что в~центроиде присутствуют характерные для всего 
класса участки, которым соответствуют большие веса вектора весов этого центроида. 
А~функция расстояния, основанная на vwDTW и~использующая вектор весов центроида, 
точнее объединит объекты одного класса и~разделит объекты разных классов, 
чем основанная на DTW. Поэтому в~предлагаемой модели используется метод vwDTW 
как для вычисления центроида по методу DBA, так и~для построения матрицы попарных 
расстояний. Также исследуются свойства, вид матрицы весов центроида и~эффективность 
применения расстояния, вычисленного с~ее помощью~--- mwDTW, в~прикладных задачах.

Для дальнейшей классификации рядов по полученной матрице расстояний сниженной 
раз\-мер\-ности применяется метод~$k$~ближайших соседей, как и~в~базовой модели. Процедура 
классификации выполняется в~три шага:
\begin{enumerate}[(1)]
\item отбор эталонных объектов каждого 
класса;
\item построение матрицы попарных расстояний сниженной размерности 
между временн$\acute{\mbox{ы}}$ми рядами и~эталонными объектами каждого класса; 
\item классификация временн$\acute{\mbox{ы}}$х рядов методом~$k$~ближайших соседей с~помощью матрицы 
попарных расстояний.
\end{enumerate}

Для проверки работоспособности такой модели проведен вычислительный эксперимент 
на реальных и~синтетических данных. Эксперимент включает в~себя анализ и~классификацию 
данных при помощи построенной модели. Полученные результаты сравниваются с~результатами 
применения базовой модели к~тем же исходным данным.



\section{Постановка задачи}

Задана выборка $\fD = \{(\mathbf{s}_i,y_i)\}^m_{i=1}$, 
состоящая из пар объект--от\-вет. Объектами служат временн$\acute{\mbox{ы}}$е 
ряды $\mathbf{s}_i \hm\in \mathbb{R}^n$, а~ответами являются метки класса~---
$y_i\hm \in Y\hm= \{1,\ldots,E\}$, где $E \hm\ll m$. Выборка разбита на 
обучающую~$\fD_l$ и~контрольную~$\fD_t$.

\smallskip

\noindent
\textbf{Определение~1.} 
{Модель классификации~$f$~--- пара\-мет\-ри\-че\-ская функция объектов выборки, 
приближающая целевую зависимость~$y_i$. В~данной работе параметрами модели примем 
множество центроидов $\mathbf{C} \hm= \{\mathbf{c}_e\}_{e=1}^E$ 
и~множество векторов  весов центроидов $\hat{\mathbf{W}} \hm= \{\mathbf{w}_e\}_{e=1}^E$ 
или же матриц весов центроидов $\hat{\mathbf{W}} \hm= \{\mathbf{W}_e\}_{e=1}^E$.}

\smallskip

\noindent
\textbf{Определение~2.}\ 
Функцией ошибки~$S$ модели~$f$ для задачи классификации будем 
считать 
$$
S\left(f,\fD_t\right) = \fr{1}{|\fD_t|}\sum\limits_{i=1}^{|\fD_t|} 
\left[f\left(\mathbf{s}_i\right)\neq y_i\right]\,.
$$

Требуется построить модель 
классификации~$f$: $\mathbb{R}^n \hm\rightarrow Y$, минимизирующую функцию ошибки~$S$ 
на контрольной выборке:
\begin{equation*}
f_{\mathbf{C},\hat{\mathbf{W}}} =  \mathop{\mathrm{argmin}}
\limits_{{\mathbf{C},\hat{\mathbf{W}}}}\left(S\left(f,\fD_t\right)\right).
%\label{e1-g}
\end{equation*}

\section{Вычисление значения функции расстояния}

\subsection{Общие понятия}

В данной работе в~качестве метрического расстояния между объектами предлагается 
использовать стоимость \textit{взвешенного пути наименьшей стоимости} 
между этими объектами.

 Даны два временн$\acute{\mbox{ы}}$х ряда:~$\mathbf{s}_1$ и~$\mathbf{s}_2$. Будем считать, 
 что $\mathbf{s}_1,\mathbf{s}_2 \in \mathbb{R}^{n}$. Пусть 
 $\boldsymbol{\Omega}^{n \times n}$~--- это матрица, такая что ее 
 элемент~$\Omega_{ij}$ равен квадрату разности между $i$-м и~$j$-м элементами 
 последовательностей~$\mathbf{s}_1$ и~$\mathbf{s}_2$:
\begin{equation*}
{\Omega}_{ij}=\left({s}_{1i} - {s}_{2j}\right)^2\,.
\end{equation*}

\noindent
\textbf{Определение~3.} 
Путем $\boldsymbol{\pi}$ между последовательностями~$\mathbf{s}_1$ и~$\mathbf{s}_2$ 
назовем упорядоченное множество пар индексов элементов матрицы~$\boldsymbol{\Omega}$: 
\begin{multline*}
\boldsymbol{\pi} = \{\pi_r\} = \{(i_r,j_r)\}\,,\enskip 
r = 1, \dots, R\,,\\ i,j \in \{1,\ldots,n\} \,,
\end{multline*}
где $R$~--- длина пути, 
зависящая от выбора пути. Он должен удовлетворять следующим условиям.
\begin{description}
\item[\,]
\textbf{Граничные условия.}
  ${\pi}_1 = (1,1)$ и~${\pi}_R \hm= (n,n)$, т.\,е.\
   начало и~конец~$\boldsymbol{\pi}$ находятся на диагонали в~противоположных 
   углах~$\boldsymbol{\Omega}.$

\item[\,] \textbf{Непрерывность.}
 Пусть ${\pi}_r = (p_1,p_2)$ и~${\pi}_{r-1} \hm= (q_1,q_2)$, $r \hm= 2,\ldots,R$. 
 Тогда 
 
 \noindent
 $$
 p_1-q_1\leq1\,,\quad p_2-q_2\leq 1\,.
 $$ 
 Это ограничение нужно, чтобы в~шаге пути~$\boldsymbol{\pi}$ участвовали только 
 соседние элементы матрицы (включая соседние по диагонали).

\item[\,] \textbf{Монотонность.}
 Пусть ${\pi}_r = (p_1,p_2)$ и~${\pi}_{r-1}\hm = (q_1,q_2)$, $r \hm= 2,\ldots,R$. 
 Тогда выполняется хотя бы одно из условий 
 
 \noindent
 $$
 p_1-q_1\geq1\,;\quad p_2-q_2\geq1\,.
 $$
 Это ограничение обусловлено природой рассматриваемых последовательностей 
 и~предназначено для монотонности функции выравнивания времени.

\item[\,] \textbf{Физические ограничения.}
Как уже говорилось во введении, предполагается, что временн$\acute{\mbox{ы}}$е ряды 
одного класса имеют схожую форму и~являются линейно или нелинейно 
деформированными или же смещены друг относительно друга. При этом считается, 
что подобного рода деформации и~смещения являются малыми, локальными. 
В~этом предположении выравнивающий путь в~матрице слабо отклоняется от диагонали,
т.\,е.\

\noindent
$$ \mbox{для каждого} \ \  
\{i_r,j_r\} \in \boldsymbol{\pi} \quad i_r - k\leq j_r\leq i_r + k\,,
$$
где $k$ определяется типом задачи и~ее физическими ограничениями.
\end{description}


\subsection{Векторно-взвешенный путь наименьшей стоимости}

Дадим определение век\-тор\-но-взве\-шен\-но\-го пути наименьшей стоимости, vwDTW. 
Дан вектор весов $\mathbf{w} \in \mathbb{R}^{n}$.

\smallskip

\noindent
\textbf{Определение~4.} 
Стоимостью $\mathrm{Cost}(\mathbf{s}_1,\mathbf{s}_2,\mathbf{w},{\boldsymbol{\pi}})$ 
век\-тор\-но-взве\-шен\-но\-го пути~$\boldsymbol{\pi}$ между 
последовательностями~$\mathbf{s}_1$ 
и~$\mathbf{s}_2$ с~весом~$\mathbf{w}$ назовем
\begin{equation*}
\mathrm{Cost}\left(\mathbf{s}_1,\mathbf{s}_2,\mathbf{w},{\boldsymbol{\pi}}\right) 
= \sum\limits_{(i,j) \in \boldsymbol{\pi}}w_j{\Omega}_{ij}\,.
%\label{e2-g}
\end{equation*}

\noindent
\textbf{Определение~5.}\
Век\-тор\-но-взве\-шен\-ным путем наименьшей стоимости (век\-тор\-но-взве\-шен\-ным 
выравнивающим путем)~$\hat{\boldsymbol{\pi}}$ между последовательностями~$\mathbf{s}_1$ 
и~$\mathbf{s}_2$ назовем взвешенный путь, имеющий наименьшую стоимость среди 
всех возможных век-\linebreak 

\columnbreak

\noindent
тор\-но-взве\-шен\-ных путей между 
последовательностями~$\mathbf{s}_1$ и~$\mathbf{s}_2$:
\begin{equation}
\hat{\boldsymbol{\pi}} = 
\mathop{\mathrm{argmin}}\limits_{{\boldsymbol{\pi}}} \mathrm{Cost}
\left(\mathbf{s}_1,\mathbf{s}_2,\mathbf{w},\boldsymbol{\pi}\right).
\label{e3-g}
\end{equation}

Обозначим стоимость век\-тор\-но-взве\-шен\-но\-го выравнивающего пути между 
последовательностями $\mathbf{s}_1$ и~$\mathbf{s}_2$ через 
$\rho(\mathbf{s}_1,\mathbf{s}_2,\mathbf{w}) \hm= \mathrm{Cost}
\left(\mathbf{s}_1,\mathbf{s}_2,\mathbf{w},\hat{\boldsymbol{\pi}}\right).$

Для вычисления стоимости такого пути в~данной работе используется модифицированный 
метод DTW~--- vwDTW (vector-weighted DTW). Согласно этому методу необходимо 
построить новую матрицу~$\boldsymbol{\Gamma}$, элементы которой определяются 
следующим образом:
\begin{equation*}
\Gamma_{1j} = w_j\Omega_{1j}\,,\quad 
\Gamma_{i1} = w_1\Omega_{i1}\,;\quad 
i,j = 1, \dots , n\,;
\end{equation*}

\vspace*{-12pt}

\noindent
\begin{multline*}
\Gamma_{ij} = w_j\Omega_{ij}+\min\left(\Gamma_{i,j-1},\Gamma_{i-1,j},\Gamma_{i-1,j-1}
\right)\,,\\
  i,j = 2 , \dots , n\,.
\end{multline*}

Элемент $\Gamma_{ij}$ матрицы~$\boldsymbol{\Gamma}$ равен стоимости 
век\-тор\-но-взве\-шен\-но\-го выравнивающего пути между 
последовательностями $\{{s}_{1a}\}_{a=1}^i$ и~$\{{s}_{2a}\}_{a=1}^j$.

В качестве значения функции расстояния между двумя объектами 
выберем стоимость век\-тор\-но-взве\-шен\-но\-го выравнивающего пути между ними~(\ref{e3-g}):
\begin{equation}
\rho(\mathbf{s}_1,\mathbf{s}_2,\mathbf{w})= \Gamma_{nn}\,.
\label{e4-g}
\end{equation}
Заметим, что при единичном векторе весов vwDTW эквивалентен обычному DTW, 
описание которого приведено в~\cite{Goncharov}.

\subsection{Матрично-взвешенный путь наименьшей стоимости}

Дадим определение мат\-рич\-но-взве\-шен\-но\-го пути наименьшей стоимости, mwDTW. 
Дана матрица весов $\mathbf{W} \in \mathbb{R}^{n\times n}$.

\smallskip

\noindent
\textbf{Определение 6.}\ 
Стоимостью $\mathrm{Cost}(\mathbf{s}_1,\mathbf{s}_2,\mathbf{W},{\boldsymbol{\pi}})$ 
мат\-рич\-но-взве\-шен\-но\-го пути~$\boldsymbol{\pi}$ между 
последовательностями~$\mathbf{s}_1$ и~$\mathbf{s}_2$ с~весом~$\mathbf{W}$ назовем
\begin{equation*}
\mathrm{Cost}\left(\mathbf{s}_1,\textbf{s}_2,\textbf{W},{\boldsymbol{\pi}}\right) = 
\sum\limits_{(i,j) \in \boldsymbol{\pi}}W_{ij}{\Omega}_{ij}\,.
%\label{e5-g}
\end{equation*}

\smallskip

\noindent
\textbf{Определение~7.}\
Мат\-рич\-но-взве\-шен\-ным путем наименьшей стоимости 
(мат\-рич\-но-взве\-шен\-ным выравнивающим путем)~$\hat{\boldsymbol{\pi}}$ между 
последовательностями~$\mathbf{s}_1$ и~$\mathbf{s}_2$ назовем взвешенный путь, 
имеющий наименьшую стоимость среди всех возможных мат\-рич\-но-взве\-шен\-ных 
путей между последовательностями~$\mathbf{s}_1$ и~$\mathbf{s}_2$:
\begin{equation}
\hat{\boldsymbol{\pi}} = \mathop{\mathrm{argmin}}\limits_{{\boldsymbol{\pi}}} 
\mathrm{Cost}\left(\mathbf{s}_1,\mathbf{s}_2,\mathbf{W},\boldsymbol{\pi}\right)\,.
\label{e6-g}
\end{equation}


Обозначим стоимость мат\-рич\-но-взве\-шен\-но\-го выравнивающего пути 
между последовательностями $\mathbf{s}_1$ и~$\mathbf{s}_2$ через 
$\rho(\mathbf{s}_1,\mathbf{s}_2,\mathbf{W})\hm = 
\mathrm{Cost}\left(\mathbf{s}_1,\mathbf{s}_2,\mathbf{W},\hat{\boldsymbol{\pi}}\right).$

 Вычисление его стоимости происходит с~помощью еще одного модифицированного 
 метода DTW~--- mwDTW (matrix-weighted DTW). Согласно этому методу необходимо 
 построить новую матрицу~$\boldsymbol{\Gamma}$, элементы которой определяются 
 следующим образом:
\begin{equation*}
\Gamma_{1j} = W_{1j}\Omega_{1j}\,,\quad 
\Gamma_{i1} = W_{i1}\Omega_{i1}\,,\quad 
i,j = 1, \dots , n\,,
\end{equation*}

\vspace*{-12pt}

\noindent
\begin{multline*}
\Gamma_{ij} = W_{ij}\Omega_{ij}+\min\left(
\Gamma_{i,j-1},\Gamma_{i-1,j},\Gamma_{i-1,j-1}\right)\,,\\ 
i,j = 2 , \dots , n\,.
\end{multline*}

Элемент $\Gamma_{ij}$ матрицы~$\boldsymbol{\Gamma}$ равен стоимости 
мат\-рич\-но-взве\-шен\-но\-го выравнивающего пути между 
последовательностями $\{{s}_{1a}\}_{a=1}^i$ и~$\{{s}_{2a}\}_{a=1}^j$.

В качестве значения функции расстояния между двумя объектами 
выберем стоимость мат\-рич\-но-взве\-шен\-но\-го выравнивающего пути между ними~(\ref{e6-g}):
\begin{equation*}
\rho(\mathbf{s}_1,\mathbf{s}_2,\mathbf{W})= \Gamma_{nn}\,.
%\label{e7-g}
\end{equation*}
Заметим, что при использовании матрицы, состоящей из одних единиц, mwDTW 
переходит в~обычный DTW, описание которого приведено в~\cite{Goncharov}.

\section{Вычисление параметров модели классификатора}

\subsection{Построение центроида}

Пусть множество весов~$\hat{\mathbf{W}}$ фиксировано. Построим множество 
центроидов $\mathbf{C}$.

\subsubsection{Постановка задачи построения центроида}

\noindent
\textbf{Определение~8.}\
Пусть $\fD_e$~--- множество элементов из~$\fD$, принадлежащих одному 
классу~$e$ из~$Y$. Центроидом множества векторов $\fD_e \hm= 
\{\mathbf{s}_i|y_i=e\}_{i=1}^{m}$ по расстоянию~$\rho$ назовем вектор
$\mathbf{c}_e \hm\in \mathbb{R}^n$ такой, что
$$
\mathbf{c}_e = \mathop{\text{argmin}}\limits_{{\textbf{c} \in \mathbb{R}^n}}\sum_{\textbf{s}_i \in \fD_e}{\rho(\textbf{s}_i ,\textbf{c})},$$
где $\rho$~--- стоимость \textit{векторно-взвешенного (мат\-рич\-но-взве\-шен\-но\-го) пути наименьшей стоимости} vwDTW (mvDTW).

Центроид найдем как решение оптимизационной задачи для vwDTW:
 \begin{equation}
 \mathbf{c}_e = \mathop{\mathrm{argmin}}\limits_{{\mathbf{c} \in \mathbb{R}^n}}
 \sum\limits_{\mathbf{s}_i \in \fD_e}
\sum\limits_{(t,t') \in \boldsymbol{\hat{\pi}}_i}\mathbf{w}_e(t)
 \left(\mathbf{s}_i(t')-\mathbf{c}(t)\right)^2
 \label{e8-g}
 \end{equation}
или же для mwDTW:
\begin{equation*}
 \mathbf{c}_e = \mathop{\mathrm{argmin}}\limits_{{\mathbf{c} 
 \in \mathbb{R}^n}}\sum\limits_{\mathbf{s}_i \in \fD_e}
 \sum_{(t,t') \in \boldsymbol{\hat{\pi}}_i}
 \mathbf{W}_e(t,t')\left(\mathbf{s}_i(t')-\mathbf{c}(t)\right)^2,
% \label{e9-g}
 \end{equation*}
где $\boldsymbol{\hat{\pi}}_i$~--- век\-тор\-но-взве\-шен\-ный 
(мат\-рич\-но-взве\-шен\-ный) выравнивающий путь между временн$\acute{\mbox{ы}}$ми 
рядами~$\mathbf{s}_i$ и~$\mathbf{c}$.

\subsubsection{Решение задачи нахождения центроида методом DBA}

\noindent
\textbf{Теорема~1}~\cite{DBA}. 
\textit{Пусть дано множество векторов $\fD_e \hm= \{\mathbf{s}_i|y_i=e\}_{i=1}^{m}$ 
одного класса, начальное приближение центроида~$\mathbf{c}_e$ и~множество 
выравнивающих путей между каждым рядом и~начальным приближением 
центроида $\{\boldsymbol{\tilde{\pi}}_i\}_{i=1}^m$. Тогда локальный 
минимум задачи оптимизации}~(\ref{e8-g}) \textit{при единичном векторе весов 
в}~(\ref{e4-g}) \textit{(функция расстояния DTW) достигается при}
\begin{equation*}
\mathbf{c}_e(t) = \fr{1}{N}\sum\limits_{\mathbf{s}_i \in 
\fD_e}\sum\limits_{t' : (t,t') \in \boldsymbol{\tilde{\pi}}_i}\textbf{s}_i(t')\,,
%\label{e10-g}
\end{equation*}
где
$$
N = \sum\limits_{\mathbf{s}_i \in \fD_e}
\sum\limits_{t' : (t,t') \in \boldsymbol{\tilde{\pi}}_i}1\,.
$$


\noindent
Д\,о\,к\,а\,з\,а\,т\,е\,л\,ь\,с\,т\,в\,о.\ \ 
Для поиска центроида и~решения задачи оптимизации~(\ref{e8-g}) 
воспользуемся необходимым условием экстремума. Запишем частные производные 
функционала по $\mathbf{c}_e(t)$, $t \hm= 1,\dots,T$, и~приравняем их к~0:
\begin{equation*}
\fr{\partial F(\mathbf{c}_e,\fD_e)}{\partial \mathbf{c}_e(t)} = 
\sum\limits_{\mathbf{s}_i \in \fD_e}
\sum\limits_{t' : (t,t') \in \boldsymbol{\tilde{\pi}}_i}2
\left(\mathbf{c}_e(t) - \mathbf{s}_i(t')\right)=0\,.
%\label{e11-g}
\end{equation*}
Откуда и~находим значение $\mathbf{c}_e(t)$, $t \hm= 1,\ldots,n$:
\begin{equation*}
\mathbf{c}_e(t) = \fr{1}{N}\sum\limits_{\mathbf{s}_i \in \fD_e}
{\sum\limits_{t' : (t,t') \in \boldsymbol{\tilde{\pi}}_i}\mathbf{s}_i(t')}.
%\label{e12-g}
\end{equation*}


Данный метод вычисления центроида приведен в~\cite{DBA} 
и~называется методом DBA. Там же приведено и~его доказательство. 
При нахождении нового центроида множество выравнивающих рядов меняется, 
данную процедуру нужно проводить несколько раз, пока центроид не стабилизируется. 
При замене единичного вектора весов в~функции расстояния vwDTW на произвольный 
справедливо следующее

\smallskip

\noindent
\textbf{Следствие~1.}\
При использовании произвольного вектора весов центроида $\mathbf{w}$ (замене DTW 
на \mbox{vwDTW} с~вектором весов~$\mathbf{w}$) в~задаче оптимизации~(\ref{e8-g})\linebreak 
алгоритм DBA вычисления центроида находит локальный минимум при замене 
множества путей наименьшей стоимости $\{\boldsymbol{\tilde{\pi}}\}_{i=1}^m$ на 
множество взвешенных путей наименьшей стоимости 
$\{\boldsymbol{\hat{\pi}}_i\}_{i=1}^m$.

\smallskip

Доказательство данного следствия повторяет доказательство теоремы~1 
при замене множества путей наименьшей стоимости $\{\boldsymbol{\hat{\pi}}\}_{i=1}^m$ 
на множество взвешенных путей наименьшей стоимости $\{\boldsymbol{\pi}_i\}_{i=1}^m$.

Для функции расстояния mwDTW следствие сохраняет свою формулировку и~доказательство 
при замене вектора весов центроида на его матрицу весов~$\mathbf{W}$.


\subsection{Оптимизация и~ограничения вектора весов}

Положим теперь множество центроидов~$\mathbf{C}$ фиксированным. Каждому 
центроиду~$\mathbf{c}_e$ из множества~$\mathbf{C}$ поставлен в~соответствие 
вектор неотрицательных весов~$\mathbf{w}_e$, принадлежащий множеству~$\hat{\mathbf{W}}$. 
Значения данного вектора весов выделяют наиболее типичные для класса участки центроида, 
сопоставляя им большие веса. Вычислим этот вектор, решая задачу оптимизации:
\begin{equation}
\mathbf{w}_e = \mathop{\mathrm{argmin}}\limits_{{\mathbf{w} \in 
 \mathbb{R}^n}}\sum\limits_{\mathbf{s}_i \in \fD_e}
 {\sum\limits_{(t,t') \in \boldsymbol{\pi}_i}\mathbf{w}(t)
 \left(\mathbf{s}_i(t')-\mathbf{c}_e(t)\right)^2}.\!\!
 \label{e13-g}
 \end{equation}


При отсутствии ограничений на веса~$\mathbf{w}_e$ минимум~(\ref{e13-g}) 
достигается при $\mathbf{w}_e \hm= \boldsymbol{0}$. Чтобы избежать 
такого тривиального решения, введем ограничения на сумму элементов вектора весов:
 $$
 \sum\limits_{t=1}^T{\mathbf{w}_e(t)} = T\,.
 $$
 Предположим, что при решении задачи~(\ref{e13-g}) нашлось~$t$, 
 для которого выполняется
$$
\sum\limits_{\mathbf{s}_i \in \mathbf{D}}
\sum\limits_{t' : (t,t') \in \boldsymbol{\pi}_i}
\left(\mathbf{s}_i(t')-\mathbf{c}_e(t)\right)^2 = 0\,.
$$
Для таких~$t$ элемент решения задачи оптимизации~(\ref{e13-g}) 
$\mathbf{w}_e(t)$ примет большие значения, которые обеспечат выполнение ограничений 
на сумму элементов. Это приведет к~локальному скоплению больших значений вектора 
весов, что сильно ухудшит дальнейшую интерпретацию вектора весов, а~также сделает 
метод чувствительным к~малым изменениям входных данных. Поэтому введем ограничения 
на элементы вектора весов сверху:
$$
\mathbf{w}_e(t) \leq const\,,\enskip t \in \{1,\ldots ,T\}.
$$

Таким образом, исходная задача~(\ref{e13-g}) примет следующий вид:
\begin{equation}
\left.
\begin{array}{c}
 \mathbf{w}_e = \mathop{\mathrm{argmin}}\limits_{{\mathbf{w} \in \mathbb{R}^n}}
\displaystyle \sum\limits_{\mathbf{s}_i \in \fD_e}
 \sum\limits_{(t,t') \in \boldsymbol{\pi}_i}\hspace*{-1mm}
 \mathbf{w}_e(t)\left(\mathbf{s}_i(t')-{}\right.\\[6pt]
\left.\hspace*{37mm}{}-\mathbf{c}_e(t)\right)^2;
\\[6pt]
 \sum\limits_{t=1}^T{\mathbf{w}_e(t)} = T\,, 
 \enskip 0 \leq \mathbf{w}_e(t) \leq const\,, \\[6pt] 
 t \in \{1,\ldots,T\}\,,
 \end{array}
 \right\}
  \label{e14-g}
\end{equation}
 где $const$~--- некоторая заданная константа.
 
 \vspace*{-2pt}

\subsection{Оптимизация матрицы весов}

% \vspace*{-2pt}

\subsubsection{Ограничения матрицы весов}

Как и~при оптимизации вектора весов, положим множество центроидов~$\mathbf{C}$ 
фиксированным. Каждо-\linebreak\vspace*{-12pt}

\columnbreak

\noindent
му центроиду~$\mathbf{c}_e$ некоторого класса~$e$ 
из множества~$\mathbf{C}$ сопоставлена матрица неотрицательных весов~$\mathbf{W}_e$, 
принадлежащая множеству~$\hat{\mathbf{W}}$.

Используя те же соображения, что и~для 
случая использования расстояния vwDTW, определим задачу нахождения матрицы весов 
центроида как задачу оптимизации с~ограничениями при использовании расстояния mwDTW:
\begin{equation}
\left.
\begin{array}{c}
 \!\!\!\!\!\mathbf{W}_e = \mathop{\mathrm{argmin}}\limits_{{\mathbf{W} \in \mathbb{R}^{n \times n}}}
\displaystyle \sum\limits_{\mathbf{s}_i \in \fD_e}
 \sum\limits_{(t,t') \in \boldsymbol{\pi}_i}\mathbf{W}_e(t,t')
 \left(\mathbf{s}_i(t')-{}\right.\\[6pt]
 \!\!\!\!\! \left.\hspace*{47mm}{}-\mathbf{c}_e(t)\right)^2;
\\[6pt]
 \!\!\!\!\!\hspace*{-35mm}\displaystyle\sum\limits_{t=1}^T\sum\limits_{t'=1}^T{\mathbf{W}_e(t,t')} = T^2\,, \\[6pt] 
\hspace*{10mm}0 \leq \mathbf{W}_e(t) \leq const\,, \quad t \in \{1,\ldots,T\}\,,
\end{array}
\right\}\!\!
 \label{e15-g}
\end{equation}
где $const$~--- некоторая заданная константа.

\vspace*{-6pt}

\subsubsection{Сглаживание полученной матрицы}

Полученная матрица не является устойчивой к~изменению входных данных: 
при использовании других временн$\acute{\mbox{ы}}$х рядов выравнивающие пути будут иметь другой вид, 
а~значит, и~решение задачи оптимизации будет другое. Более устойчивой мат\-ри\-ца 
получится после процедуры сглажи\-вания.
{\looseness=1

}

Будем говорить, что элемент матрицы $\mathbf{W}_e(t,t')$ содержится 
во~множестве~$\Phi$, 
если существует временной ряд $\mathbf{s}_i \hm\in \fD_e$ такой, что путь 
наименьшей мат\-рич\-но-взве\-шен\-ной стоимости проходит через 
элемент $\Omega(t,t')$ 
в~матрице~$\Omega$, построенной для временн$\acute{\mbox{о}}$го ряда~$\mathbf{s}_i$ и~центроида.

При решении задачи оптимизации элементы $\{\mathbf{W}_e(t,t')\} \hm\in \Phi$
будут изменяться в~меньшую сторону, при этом 
$\mathbf{W}_e(t,t') \hm\notin \Phi$ достигнет своей верхней границы для 
выполнения ограничений, накладываемых на сумму элементов матрицы.

Выберем произвольный элемент матрицы весов $\mathbf{W}_e(t,t') \hm \notin \Phi$. 
Вероятность того, что при добавлении нового временн$\acute{\mbox{о}}$го ряда (например, из тестовой 
выборки) выполнится $\mathbf{W}_e(t,t')  \hm\in \Phi$, выше, если среди ближайших 
к~$\mathbf{W}_e(t,t')$ элементов в~строке матрицы весов многие содержатся 
в~множестве~$\Phi$. Тогда значение такого элемента должно быть похожим на 
значения соседних. Добьемся этого, выполнив сглаживание матрицы весов центроида:

\noindent
$$
\widetilde{\mathbf{W}}_e(t,t') = \fr{1}{2\delta} 
\sum\limits_{k=-\delta}^{\delta}{\mathbf{W}_e(t,t'+k)}\,,
$$
где $\delta$~--- величина окна сглаживания, а~$\widetilde{\mathbf{W}}_e$~--- 
искомая матрица весов центроида.

\pagebreak

\subsection{Задача оптимизации параметров модели}

Задача оптимизации параметров модели сведена к~комбинации задач оптимизации~(\ref{e8-g}) 
и~(\ref{e14-g}), (\ref{e15-g}) для vwDTW:

\noindent
\begin{multline*}
 \!\!\hspace*{-3pt}\mathbf{w}_e, \mathbf{c}_e = \mathop{\mathrm{argmin}}\limits_{{\mathbf{c},\mathbf{w} 
 \in \mathbb{R}^n}}\sum\limits_{\mathbf{s}_i \in \fD_e}
 \sum\limits_{(t,t') \in \boldsymbol{\pi}_i}\hspace*{-1mm}\hspace*{-2pt}\left(
 \mathbf{w}(t)\left(\mathbf{s}_i(t')-\mathbf{c}(t)\right)^2\right)\,; \\ 
 e = 1,\ldots,E,
% \label{e16-g}
 \end{multline*}

 \vspace*{-12pt}
 
\noindent
  $$
  \sum\limits_{t=1}^T{\mathbf{w}_e(t)} = T\, , \ 
  0 \leq \mathbf{w}_e(t) \leq const\,, \quad t \in \{1,\ldots,T\},
  $$
или же для mwDTW:
\begin{multline*}
 \hspace*{-2pt}\mathbf{W}_e, \mathbf{c}_e = \mathop{\mathrm{argmin}}\limits_{{\mathbf{W} \in 
 \mathbb{R}^n \times \mathbb{R}^n}}
 \sum\limits_{\mathbf{s}_i \in \fD_e}
 \sum\limits_{\left(t,t'\right) \in \boldsymbol{\pi}_i}\!\!\!\!
 \left(\mathbf{W}_e(t,t')\left(\mathbf{s}_i(t')-{}\right.\right.\\
\left.\left. {}-\mathbf{c}_e(t)\right)^2\right), \quad 
 e = 1,\dots,E\,;
% \label{e17-g}
 \end{multline*}
 
 \vspace*{-14pt}
 
 \noindent
\begin{multline*}
\sum\limits_{t=1}^T\sum\limits_{t'=1}^T
\mathbf{W}_e\left(t,t'\right) = T^2\,, \quad 0 \leq \mathbf{W}_e(t) \leq const\,, \\ 
t \in \{1,\ldots,T\}.
\end{multline*}

Эту задачу будем решать, вычисляя сначала множество центроидов~$\mathbf{C}$ при 
фиксированном начальном приближении множества весов центроидов $\hat{\mathbf{W}}$, 
а~затем вычисляя множество весов центроидов~$\hat{\mathbf{W}}$ при фиксированном 
множестве центроидов~$\mathbf{C}$.
Таким образом, алгоритм вычисления параметров модели будет иметь следующий вид.
\begin{description}
\item[\,] \textbf{Шаг~1.}\ Начальное приближение вектора весов центроида: 

\noindent 
$$
\mathbf{w}_e = \mathbf{1} \,, \quad e=1,\ldots,E\,,
$$
или же начальное приближение матрицы весов центроида: 

\noindent
$$
\mathbf{W}_e = \mathbf{1} \,, \quad e=1,\ldots,E\,.
$$
\item[\,] 
\textbf{Шаг~2.} Начальное приближение центроида класса~--- 
произвольный элемент класса:

\noindent 
$$
\mathbf{c}_e =  \mathbf{s}_j \in \fD_e \,,\quad e=1,\dots,E\,.
$$
\item[\,] 
\textbf{Шаг~3.} Вычисление центроида при фиксированном векторе 
(матрице) весов центроида как решение задачи оптимизации~(\ref{e8-g}).
\item[\,] 
\textbf{Шаг~4.} Вычисление вектора (матрицы) весов центроида при фиксированном 
центроиде как решение задачи оптимизации~(\ref{e14-g}), (\ref{e15-g}).
\end{description}

\vspace*{-6pt}

\section{Вычислительный эксперимент}

\vspace*{-2pt}


Для проверки свойств введенной функции расстояния, а~также выбранной модели был 
проведен

\columnbreak

\noindent
 вычислительный эксперимент на реальных и~синтетических данных. 
Свойства функций расстояния\linebreak vwDTW и~mwDTW для наглядности продемонстрированы 
на синтетических данных, представляющих собой смещенные и~линейно деформированные 
временн$\acute{\mbox{ы}}$е ряды аналитических функций: $\sin{x}$, $\sqrt{x}$, $x^2$~--- длиной~100~точек. 
В~выборке находилось~100~временн$\acute{\mbox{ы}}$х рядов каждого 
класса:~50~в~обуча\-ющей 
и~50~в~контрольной. Обозначим классы как $1 \hm- \sin{x}$, $2 \hm- \sqrt{x}$
и~$3 \hm- x^2$ соответст\-венно.
{\looseness=1

}

Пример такой выборки и~результаты нахождения параметров модели по обучающей выборке 
показаны на рис.~1. По оси абсцисс отложены значения времени, а~по оси ординат~--- 
значения временн$\acute{\mbox{о}}$го ряда. На рис.~1,\,\textit{а} 
приведены примеры аналитических 
функций, используемых в~создании синтетической выборки временн$\acute{\mbox{ы}}$х рядов. На 
рис.~1,\,\textit{б} изображены центроид (нижний временной ряд) и~вектор весов 
(верхний временной ряд) для класса~1. Аналогично на рис.~1,\,\textit{в}
и~1,\,\textit{г} показаны  результаты для классов~2 и~3.


Векторы весов описывают наиболее информативные участки центроида. 
Так, для центроида класса~1 ($\sin{x}$) наиболее информативными оказались минимумы 
и~максимумы, в~отличие от точек перегиба.

 Для сравнения свойств полученной функции расстояния с~функцией расстояния DTW 
 были посчитаны расстояния до центроидов для всех временн$\acute{\mbox{ы}}$х рядов контрольной выборки 
 с~по\-мощью DTW и~vwDTW, после чего производилась классификация. Каждому 
 временн$\acute{\mbox{о}}$му 
 ряду контрольной выборки ставилась в~соответствие метка того класса, расстояние 
 до центроида которого было минимальным. Результат классификации с~помощью функции 
 расстояния vwDTW~--- 97\%, а~для функции расстояния DTW~--- 84\%, что на 15\% меньше.
 Построенная в~работе функция расстояния лучше разделила временн$\acute{\mbox{ы}}$е 
 ряды разных классов, сгруппировала их вокруг соответствующих центроидов.
 
\begin{figure*}%fig1
\vspace*{1pt}
 \begin{center}
 \mbox{%
 \epsfxsize=151.698mm
 \epsfbox{gon-1.eps}
 }
 \end{center}
 \vspace*{-9pt}
\Caption{Примеры синтетических временн$\acute{\mbox{ы}}$х рядов 
аналитических функций (выборка)~(\textit{а}) 
и~результаты построения центроида и~вектора весов для синтетических данных:
(\textit{б})~$\sin x$;
(\textit{в})~$x^2$; (\textit{г})~$\sqrt{x}$
%\label{draw:1}
}
%\end{figure*}
% \begin{figure*}[b]%fig2
\vspace*{1pt}
 \begin{center}
 \mbox{%
 \epsfxsize=152.753mm
 \epsfbox{gon-2.eps}
 }
 \end{center}
 \vspace*{-9pt}
\Caption{Примеры синтетических временн$\acute{\mbox{ы}}$х рядов 
аналитических функций (выборка)~(\textit{а}) 
и~результаты построения центроида класса $\sin x$:
(\textit{б})~множество выравнивающих путей;
(\textit{в})~начальная матрица весов; (\textit{г})~сглаженная матрица весов
%\label{draw:1}
}
\end{figure*}

Также были построены матрицы весов центроида на синтетических данных. 
Вследствие хорошей интерпретируемости в~работе приведено визуальное отображение 
матрицы весов центроида только для класса~1. На рис.~2,\,\textit{а} изображены примеры 
временн$\acute{\mbox{ы}}$х рядов.На рис.~2,\,\textit{б}~--- 
множество выравнивающих путей по 
матрице~$\boldsymbol{\Omega}$. На рис.~2,\,\textit{в}~--- матрица весов центроида 
до сглаживания. На рис.~2,\,\textit{г}~--- после сглаживания. 
Темные цвета соответствуют 
маленьким значениям элементов матрицы, серые~--- большим, более светлые~--- 
промежуточным. Хорошо просматривается периодичность в~матрице, напоминающая 
периодичность вектора\linebreak\vspace*{-12pt}

\begin{table*}[b]\small
\begin{center}
\Caption{Сравнение эффективности предложенной (vwDTW) и~базовой модели классификации 
на данных~\cite{Data}
%\label{tab:2}
}


\vspace*{2ex}

\tabcolsep=14pt
\begin{tabular}{|l|c|c|c|c|c|c|c|}
\hline %\multicolumn{1}{|c|}{\raisebox{-6pt}[0pt][0pt]{
        \multicolumn{1}{|c|}{\raisebox{-6pt}[0pt][0pt]{Модель}} &\multicolumn{7}{c|}{Точность по критерию
        скользящего контроля, \%}\\
        \cline{2-8}
        &{Бег} & {Ходьба} & {Подъем} & {Спуск} & {Сидение} & {Стояние}& {Общее} \\
        \hline
 vwDTW  & 
        97 & 95 & 79 & 75 & 95 & 95 & 89\\
              %  \hline
               DTW~\cite{Goncharov}
         & 
        95 & 92 & 60 & 60 & 85 & 90 & 80\\
        \hline
        \end{tabular}
\end{center}
\end{table*}

\pagebreak

 \noindent
 \begin{center}  %fig3
 \vspace*{1pt}
 \mbox{%
 \epsfxsize=71.691mm
 \epsfbox{gon-3.eps}
 }


\end{center} 
\vspace*{3pt}

\noindent
{{\figurename~3}\ \ \small{Примеры временн$\acute{\mbox{ы}}$х рядов измерений акселерометра для разных видов 
физической активности}}


 \vspace*{9pt}

\addtocounter{figure}{1}


\noindent
 весов. Предполагается, что функция mwDTW будет лучше разделять 
классы, которые сильно различаются между собой, так как матрица весов центроида 
учитывает и~среднее отклонение выравнивающего пути от диагонали 
в~матрице~$\boldsymbol{\Omega}$. При отклонениях пути сильнее типичного 
для данного класса элементам пути будут приписываться большие веса, что видно 
из структуры матрицы весов центроида.



Использование функции mwDTW для классификации синтетических временн$\acute{\mbox{ы}}$х 
рядов улучшает классификацию по сравнению с~DTW. Этот результат аналогичен случаю 
использования vwDTW~--- 97\%.


Для демонстрации работы предложенной модели на реальных данных и~ее сравнения 
с~базовой моделью классификации были использованы данные акселерометра 
мобильного телефона. Вследствие большой вычислительной сложности они 
сравнивались на данных, представляющих 
собой~600~временн$\acute{\mbox{ы}}$х рядов длиной~200~точек, каж\-дый из которых 
представляет собой абсолютные значения ускорения мобильного телефона, 
объединяя три временн$\acute{\mbox{ы}}$х ряда: временной ряд ускорения 
по оси~$X$~(200~измерений), оси~$Y$ (200~измерений) и~оси~$Z$ (200~измерений). 
Выделено шесть типов физической активности: ходьба, бег, сидение, стояние, 
подъем, спуск. Временн$\acute{\mbox{ы}}$е ряды записывались акселерометром, 
который находился в~кармане у~человека, выполняющего один из типов физической 
активности, после чего разделялись на 10-се\-кунд\-ные сегменты. Примеры таких 
временн$\acute{\mbox{ы}}$х рядов приведены на рис.~3.

 
Данные разделялись на обучающую и~контрольную выборку. 
В~обучающую выборку входило\linebreak по~70~временн$\acute{\mbox{ы}}$х рядов каждого вида физической активности, 
а~в~контрольную~--- по~30~вре\-мен\-н$\acute{\mbox{ы}}$х рядов. Производилась 
классификации ме\-то\-дом~$k$~бли-\linebreak жайших 
соседей, использующим построенную\linebreak мат\-ри\-цу попарных расстояний. Осуществлялся контроль 
качества при помощи кросс-ва\-ли\-да\-ции.  

В~табл.~1 приведены результаты 
классификации при применении новой модели, использующей функцию vwDTW, 
и~базовой модели классификации.

Качество классификации базовой модели ухудшилось в~сравнении с~вычислительным 
экспериментом в~работе~\cite{Goncharov}, так как теперь используются абсолютные 
значения ускорения, а~не последовательно соединенные временн$\acute{\mbox{ы}}$е ряды ускорения вдоль 
трех координат. Использование функции расстояния vwDTW улучшает классификацию для 
всех классов физической активности, повышая общий уровень классификации на~9\%.



Полученные параметры модели для реальных данных изображены на рис.~4.
Верхние линии соответствуют центроидам, а~нижние~--- их весам. 
Для класса бег, например, хорошо просматривается периодичность временн$\acute{\mbox{о}}$го ряда как 
для центроида, так и~для вектора весов центроида.

\begin{figure*} %fig4
\vspace*{1pt}
 \begin{center}
 \mbox{%
 \epsfxsize=162.61mm
 \epsfbox{gon-4.eps}
 }
 \end{center}
 \vspace*{-9pt}
\Caption{Примеры временн$\acute{\mbox{ы}}$х рядов измерений акселерометра для разных видов физической 
активности: (\textit{а})~ходьба; (\textit{б})~бег;
(\textit{в})~подъем; (\textit{г})~спуск; (\textit{д})~сидение;
(\textit{е})~стояние
%\label{draw:1}
}
\end{figure*}

\begin{table*}\small %[!ht]
\vspace*{-12pt}
\begin{center}
\Caption{Сравнение эффективности предложенной (mwDTW) и~базовой модели классификации 
и~алгоритма разделяющей классификации на данных~\cite{Data}
%\label{tab:2}
}
\vspace*{2ex}

\tabcolsep=14.7pt
\begin{tabular}{|l|c|c|c|c|c|c|c|}
        \hline
       \multicolumn{1}{|c|}{\raisebox{-6pt}[0pt][0pt]{Модель}} &\multicolumn{7}{c|}{Точность по критерию
        скользящего контроля, \%}\\
                \cline{2-8}
        &{Бег} & {Ходьба} & {Подъем} & {Спуск} & {Сидение} & {Стояние}& {Общее} \\
        \hline
   mwDTW  & 95 & 95 & 78 & 76 & 90 & 90 & 87\\
       % \hline
  DTW \cite{Goncharov} & 88 & 83 & 55 & 60 & 82 & 80 & 74\\
                \hline
\end{tabular}
\end{center}
\end{table*}

 Для сравнения модели, использующей mwDTW, с~базовой моделью классификации были 
 использованы данные, представляющие собой 600~вре\-мен\-н$\acute{\mbox{ы}}$х 
 рядов длиной~100~точек. 
 Временн$\acute{\mbox{ы}}$е ряды такой длины выбраны для разумного 
 ограничения времени работы методов 
 численной оптимизации. В~остальном этот эксперимент повторяет тот, 
 что выполнен для 
 сравнения базовой модели и~модели, использующей vwDTW. 
 %
 Результаты   классификации 
 приведены в~табл.~2. Видно, что длина временн$\acute{\mbox{ы}}$х рядов, а~следовательно, и~количество 
 периодов временн$\acute{\mbox{о}}$го ряда, существенно влияют на качество классификации. Качество 
 базовой модели сильно упало по сравнению с~временн$\acute{\mbox{ы}}$ми рядами длиной~200~точек. 
 При этом использование mwDTW улучшает классификацию на~13\%.



\section{Заключение}

В работе описан новый подход к~работе с~центроидами временн$\acute{\mbox{ы}}$х рядов, построена модель, 
использующая веса и~матрицы центроидов, и~показаны ее преимущества перед моделью, 
описанной в~работе~\cite{Goncharov}. Продемонстрировано влияние структуры, длины 
и~физического смысла временн$\acute{\mbox{ы}}$х рядов на результаты классификации. 
В~последующих работах данный подход будет совершенствоваться. Предполагается ускорить 
данный алгоритм за счет эффективного нахождения пути наименьшей стоимости, его 
аппроксимации. Такой подход применим и~ко всем типам временн$\acute{\mbox{ы}}$х рядов, где возможна 
нелинейная деформация как по оси времени, так и~по оси значений временн$\acute{\mbox{о}}$го ряда.

{\small\frenchspacing
 {%\baselineskip=10.8pt
 \addcontentsline{toc}{section}{References}
 \begin{thebibliography}{99}


\bibitem{Kuznetsov}
\Au{Кузнецов~М.\,П.,  Ивкин~Н.\,П.} 
Алгоритм классификации временных рядов акселерометра по комбинированному 
признаковому описанию~// Машинное обучение и~анализ данных, 2015. T.~1. Вып.~13. C.~1471--1483.
\bibitem{Popova2015}
\Au{Попова~М.\,С., Стрижов~В.\,В.} Выбор оптимальной модели классификации физической 
активности по измерениям акселерометра~// Информатика и~её применения, 2015. Т.~9. 
Вып.~1. С.~79--89.
\bibitem{Ignatov}
    \Au{Ignatov~A.\,D., Strijov~V.\,V.} Human activity types recognition using 
    quasiperiodic sets of time series collected from a single tri-axial accelerometer~// 
    Multimedia tools and applications.~--- Springer, 2015. P.~1--14.
\bibitem{Faloutsos1994}
\Au{Faloutsos~C., Ranganathan~M., Manolopoulos~Y.} Fast subsequence matching 
in time-series databases~// SIGMOD  Conference (International) 
on Management of Data Proceedings.~--- Minneapolis, MN, USA: ACM, 1994. P.~419--429.
\bibitem{Berndt1994}
\Au{Berndt~D.\,J., Clifford~J.} Using dynamic time warping to find patterns 
in time series~// Workshop on Knowledge Discovery in Databases 
at the 12th  Conference (International) on Artificial Intelligence
Proceedings. --- Seattle, WA, 1994. P.~359--370.
\bibitem{Keogh2005Exact}
\Au{Keogh~E.\,J., Ratanamahatana~C.\,A.} Exact indexing of dynamic time warping~// 
Knowl. Inf. Syst., 2005. Vol.~7. No.\,3. P.~358--386.
\bibitem{Vlachos2002discovering}
\Au{Vlachos~M., Gunopulos~D., Kollios~G.} Discovering similar multidimensional 
trajectories~// 18th  Conference (International) on Data Engineering (ICDE'02)
Proceedings.~--- San Jose, CA, USA: IEEE Computer Society, 2002. P.~673--684.
\bibitem{Chen2004On}
\Au{Chen~L., Ng~R.\,T.} On the marriage of lp-norms and edit distance~// 
30th  Conference (International) on Very Large Data Bases 
Proceedings.~--- Toronto: Morgan Kaufmann, 2004. P.~792--803.
\bibitem{Chen2005Robust}
\Au{Chen~L., $\ddot{\mbox{O}}$zsu~M.\,T., Oria~V.} 
Robust and fast similarity search for moving object trajectories~// 
24th ACM  Conference (International) on Management of Data 
Proceedings.~--- Baltimore, MD, USA: ACM, 2005. P.~491--502.
\bibitem{Frentzos2007Index}
\Au{Frentzos~E., Gratsias~K., Theodoridis~Y.} 
Index-based most similar trajectory search~// 
23rd  Conference (International) on Data Engineering 
Proceedings.~--- Istanbul: IEEE Computer Society, 2007. P.~816--825.
\bibitem{Morse2007An}
\Au{Morse~M.\,D., Patel~J.\,M.} An efficient and accurate method for evaluating 
time series similarity~// ACM  Conference (International) on Management of Data 
 Proceedings.~--- Beijing: ACM, 2007. P.~569--580.
\bibitem{Chen2007Spade}
\Au{Chen~Y., Nascimento~M.\,A., Ooi~B.\,C., Tung~A.\,K.\,H.} 
SpADe: On Shape-based pattern detection in streaming time series~// 
23rd  Conference (International) on Data Engineering 
Proceedings.~--- Istanbul: IEEE Computer Society, 2007. P.~786--795.
\bibitem{Keogh}
\Au{Keogh~E.\,J., Pazzani~M.\,J.} Scaling up dynamic time warping to massive datasets~// 
Principles of Data Mining and Knowledge Discovery:  3rd European Conference 
 Proceedings.~--- Berlin--Heidelberg: Springer, 1999. P.~1--11.
\bibitem{Salvador}
\Au{Salvador~S., Chan~P.} Fastdtw: Toward accurate dynamic time warping in 
linear time and space~// 3rd Workshop on Mining Temporal and Sequential Data
Proceedings.~--- Seattle, WA, USA, 2004. P.~70--80.
\bibitem{Goncharov}
\Au{Гончаров~А.\,В., Попова~М.\,С., Стрижов~В.\,В.} 
Метрическая классификация временных рядов с~выравниванием относительно центроидов 
классов~// Системы и~средства информатики, 2015. Т.~25. Вып.~4. С.~52--64.
\bibitem{DBA}
\Au{Petitjean~F., Forestier~G., Webb~G.\,I., Nicholson~A.\,E., Chen~Y., Keogh~E.} 
Dynamic time warping averaging of time series allows faster and more accurate 
classification~// 30th IEEE Conference (International) on Data Engineering
Proceedings.~--- 
Chicago, IL, USA: IEEE Computer Society, 2014. P.~470--479.
\bibitem{Data}
     Data from accelerometer. 
     {\sf https://sourceforge.net/p/\linebreak 
     mlalgorithms/code/HEAD/tree/Group274/Goncharov\linebreak 2015MetricClassification/data/preprocessed\_large.csv}.
     
     \end{thebibliography}

 }
 }

\end{multicols}

\vspace*{-3pt}

\hfill{\small\textit{Поступила в~редакцию 31.12.15}}

%\vspace*{8pt}

\newpage

\vspace*{-24pt}

%\hrule

%\vspace*{2pt}

%\hrule

%\vspace*{8pt}



\def\tit{METRIC TIME SERIES CLASSIFICATION USING~WEIGHTED
DYNAMIC WARPING RELATIVE TO~CENTROIDS OF~CLASSES}


\def\titkol{Metric time series classification using weighted
dynamic warping relative to centroids of classes}

\def\aut{A.\,V.~Goncharov$^1$ and V.\,V.~Strijov$^2$}

\def\autkol{A.\,V.~Goncharov and V.\,V.~Strijov}

\titel{\tit}{\aut}{\autkol}{\titkol}

\vspace*{-9pt}

\noindent
$^1$Moscow Institute of Physics and Technology, 
9~Institutskiy Per., Dolgoprudny, Moscow Region 141700, 
Russian\linebreak
$\hphantom{^1}$Federation


\noindent
$^2$A.\,A.~Dorodnicyn Computing Centre, Federal Research Center 
``Computer Science and Control'' of the Russian\linebreak
$\hphantom{^1}$Academy of Sciences, 
40~Vavilov Str., Moscow 119333, Russian Federation


\def\leftfootline{\small{\textbf{\thepage}
\hfill INFORMATIKA I EE PRIMENENIYA~--- INFORMATICS AND
APPLICATIONS\ \ \ 2016\ \ \ volume~10\ \ \ issue\ 2}
}%
 \def\rightfootline{\small{INFORMATIKA I EE PRIMENENIYA~---
INFORMATICS AND APPLICATIONS\ \ \ 2016\ \ \ volume~10\ \ \ issue\ 2
\hfill \textbf{\thepage}}}

\vspace*{3pt}



\Abste{The paper discusses the problem of metric time series analysis 
and classification. The proposed classification model uses a~matrix of 
distances between time series which is built with a fixed distance function. 
The dimension of this distance matrix is very high and all related calculations 
are time-consuming. The problem of reducing computational complexity is solved 
by selecting reference objects and using them for describing classes. The model 
that uses dynamic time warping for building reference objects or centroids is 
chosen as the basic model. This paper introduces a function of weights for each 
centroid that influences calculation of the distance measure. Time series of different 
analytic functions and time series of human activity from an accelerometer of 
a~mobile phone are used as the objects for classification. The properties and the 
classification result of this model 
are investigated and compared with the properties of the basic model.}

\KWE{metric classification; weighted dynamic time warping; 
time series classification; centroid; distance function}

\DOI{10.14357/19922264160204}

\vspace*{-12pt}

\Ack
\noindent
The work was financially supported by the Russian Foundation for
Basic Research (project 16-07-01163) 
and fulfilled in the frames of the initiative
``MIT-Skoltech.''


%\vspace*{3pt}

  \begin{multicols}{2}

\renewcommand{\bibname}{\protect\rmfamily References}
%\renewcommand{\bibname}{\large\protect\rm References}

{\small\frenchspacing
 {%\baselineskip=10.8pt
 \addcontentsline{toc}{section}{References}
 \begin{thebibliography}{99}
\bibitem{1-g}
\Aue{Kuznecov, M.\,P., and N.\,P.~Ivkin}. 2015. 
Algoritm klassifikatsii vremennykh ryadov akselerometra po kombinirovannomu 
priznakovomu opisaniyu [Time series classification algorithm using combined 
feature description]. 
\textit{Mashinnoe Obuchenie i~Analiz Dannykh} 
[Machine Learning and Data Analysis] 1(13):1471--1483.

\bibitem{2-g}
\Aue{Popova, M.\,S., and V.\,V.~Strijov}. 2015. Vybor optimal'noy modeli 
klassifikatsii fizicheskoy aktivnosti po izmereniyam akselerometra 
[Selection of optimal physical activity classification model using measurements 
of accelerometer]. \textit{Informatika i~ee Primeneniya}~--- \textit{Inform. Appl.} 
9(1):79--89.

\bibitem{3-g}
\Aue{Ignatov, A.\,D. and V.\,V.~Strijov}. 2015. 
Human activity types recognition using quasiperiodic sets of time series. 
\textit{Multimedia tools and applications}. Springer. 1--14.

\bibitem{4-g}
\Aue{Faloutsos, C., M.~Ranganathan, and Y.~Manolopoulos}. 1994. 
Fast subsequence matching in time-series databases. 
\textit{ACM Conference on Management of Data (SIGMOD) International Proceedings}. 
Minneapolis, MN: ACM. 419--429.

\bibitem{5-g}
\Aue{Berndt, D.\,J., and J.~Clifford}. 1994. 
Using dynamic time warping to find patterns in time series. 
\textit{Workshop on Knowledge Discovery in Databases at the 12th
 Conference (International) 
 on Artificial Intelligence Proceedings}. Seattle, WA. 359--370.

\bibitem{6-g}
\Aue{Keogh, E.\,J., and C.\,A.~Ratanamahatana}. 2005. 
Exact indexing of dynamic time warping. \textit{Knowl. Inf. Syst.} 7(3):358--386.

\bibitem{7-g}
\Aue{Vlachos, M., D.~Gunopulos, and G.~Kollios}. 2002. 
Discovering similar multidimensional trajectories. 
\textit{IEEE  Conference (International) on Data Engineering 
Proceedings}.  San Jose, CA: IEEE Computer Society. 673--684.

\bibitem{8-g}
\Aue{Chen, L., and R.\,T.~Ng.} 2004. On the marriage of lp-norms and edit distance. 
\textit{30th Conference (International) on 
Very Large Data Bases Proceedings}. Toronto: Morgan Kaufmann. 792--803.

\bibitem{9-g}
\Aue{Chen, L., M.\,T.~$\ddot{\mbox{O}}$zsu, and V.~Oria}. 2005. 
Robust and fast similarity search for moving object trajectories. 
\textit{24th ACM  Conference (International) on Management of Data 
Proceedings}. Baltimore, MD: ACM. 491--502.

\bibitem{10-g}
\Aue{Frentzos, E., K.~Gratsias, and Y.~Theodoridis}. 2007. 
Index-based most similar trajectory search. 
\textit{23rd IEEE Conference (International) on Data Engineering 
Proceedings}.  Istanbul: IEEE Computer Society. 816-825.

\bibitem{11-g}
\Aue{Morse, M.\,D., and J.\,M.~Patel}. 2007. An efficient and accurate method 
for evaluating time series similarity. 
\textit{ACM  Conference (International) on Management of Data 
Proceedings}. Beijing: ACM. 569--580.

\bibitem{12-g}
\Aue{Chen, Y., M.\,A. Nascimento, B.\,C.~Ooi, and A.\,K.\,H.~Tung}. 2007. 
SpADe: On shape-based pattern detection in streaming time series. 
\textit{23rd  Conference (International) on Data Engineering 
Proceedings}. Istanbul: IEEE Computer Society. 786--795.

\bibitem{13-g}
\Aue{Keogh,  E.\,J., and M.\,J.~Pazzani}. 1999. 
Scaling up dynamic time warping to massive datasets. 
\textit{Principles of Data Mining and Knowledge Discovery: 
3rd European Conference Proceedings}. Berlin--Heidelberg: Springer. 1--11.


\bibitem{14-g}
\Aue{Salvador, S., and P.~Chan}. 2004. Fastdtw: 
Toward accurate dynamic time warping in linear time and space. 
\textit{3rd Workshop on Mining Temporal and Sequential Data Proceedings}. 
Seattle, WA. 70--80.

\bibitem{15-g}
\Au{Goncharov, A.\,V., M.\,S.~Popova, and V.\,V.~Strijov}. 2015. 
Metricheskaya klassifikatsiya vremennykh ryadov s~vyravnivaniem otnositel'no 
tsentroidov klassov [Metric time series classification using dynamic warping relative 
to centroids of classes]. \textit{Sistemy i~Sredstva Informatiki}~---
\textit{Systems and Means of Informatics} 25(4):52--64.

\bibitem{16-g}
\Aue{Petitjean, F., G.~Forestier, G.\,I.~Webb, A.\,E.~Nicholson, Y.~Chen, and E.~Keogh}. 
2014. Dynamic time warping averaging of time series allows faster and more accurate 
classification. \textit{30th IEEE Conference 
(International) on Data Engineering Proceedings}. Chicago, IL: IEEE Computer Society. 470--479.

\bibitem{17-g}
Data from accelerometer. 
Available at:
{\sf https://\linebreak sourceforge.net/p/mlalgorithms/code/HEAD/tree/\linebreak 
Group274/Goncharov2015MetricClassification/data/\linebreak preprocessed\_large.csv}
(accessed April~28, 2016).
\end{thebibliography}

 }
 }

\end{multicols}

\vspace*{-3pt}

\hfill{\small\textit{Received December 31, 2015}}



\Contr

\noindent
\textbf{Goncharov Alexey V.} (b.\ 1995)~---
student, Moscow Institute of Physics and Technology, 
9~Institutskiy Per., Dolgoprudny, Moscow Region 141700, 
Russian Federation; alex.goncharov@phystech.edu

\vspace*{3pt}

\noindent
\textbf{Strijov Vadim V.} (b.\ 1967)~---
Doctor of Science in physics and mathematics, leading scientist, 
A.\,A.~Dorodnicyn Computing Centre, Federal Research Center 
``Computer Science and Control'' of the Russian Academy of Sciences, 
40~Vavilov Str., Moscow 119333, Russian Federation; strijov@ccas.ru


\label{end\stat}


\renewcommand{\bibname}{\protect\rm Литература}