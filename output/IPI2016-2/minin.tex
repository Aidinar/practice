\def\stat{minin}

\def\tit{ИНТЕНСИВНОСТЬ ЦИТИРОВАНИЯ НАУЧНЫХ ПУБЛИКАЦИЙ В~ИЗОБРЕТЕНИЯХ 
ПО~ИНФОРМАЦИОННО-КОМПЬЮТЕРНЫМ ТЕХНОЛОГИЯМ, 
ПАТЕНТУЕМЫХ В~РОССИИ ОТЕЧЕСТВЕННЫМИ И~ЗАРУБЕЖНЫМИ ЗАЯВИТЕЛЯМИ$^*$}

\def\titkol{Интенсивность цитирования научных публикаций в~изобретениях по ИКТ, 
патентуемых в~России} %РФ отечественными и~зарубежными заявителями}

\def\aut{В.\,А.~Минин$^1$, И.\,М.~Зацман$^2$, В.\,А.~Хавансков$^3$, 
С.\,К.~Шубников$^4$}

\def\autkol{В.\,А.~Минин, И.\,М.~Зацман, В.\,А.~Хавансков, 
С.\,К.~Шубников}

\titel{\tit}{\aut}{\autkol}{\titkol}

\index{Минин В.\,А.}
\index{Зацман И.\,М.}
\index{Хавансков В.\,А.}
\index{Шубников С.\,К.}
\index{Minin V.\,A.}
\index{Zatsman I.\,M.}
\index{Havanskov V.\,A.}
\index{Shubnikov S.\,K.}

{\renewcommand{\thefootnote}{\fnsymbol{footnote}} \footnotetext[1]
{Работа выполнена при финансовой поддержке РФФИ (проект 16-07-00075).}}


\renewcommand{\thefootnote}{\arabic{footnote}}
\footnotetext[1]{Институт проблем информатики Федерального исследовательского
центра <<Информатика и~управление>> Российской академии наук, aleksisss@ya.ru}
\footnotetext[2]{Институт проблем информатики Федерального исследовательского
центра <<Информатика и~управление>> Российской академии наук, iz\_ipi@a170.ipi.ac.ru}
\footnotetext[3]{Институт проблем информатики Федерального исследовательского
центра <<Информатика и~управление>> Российской академии наук, havanskov@a170.ipi.ac.ru}
\footnotetext[4]{Институт проблем информатики Федерального исследовательского
центра <<Информатика и~управление>> Российской академии наук, sergeysh50@yandex.ru}

     
     
      
      \Abst{Рассматриваются информационные взаимосвязи науки и~технологий, а~также 
методы индикаторного оценивания процессов переноса (трансфера) знаний из разных областей 
исследований в~сферу технологического развития. Предлагаемые методы предназначены для 
определения значений индикатора интенсивности цитирования научных работ в~описаниях 
изобретений, патентуемых в~России отечественными и~зарубежными заявителями. Подобный 
подход может использоваться для получения косвенных оценок инновационного потенциала 
направлений научных исследований (ННИ). Значения индикатора интенсивности вычислялись как 
в~целом, так и~с распределением по странам заявителей. Представлены результаты определения 
значений индикатора, для чего в~качестве исходной информации использовались 
полнотекстовые описания изобретений по классу G06 Международной патентной 
классификации  (МПК,
англ.\  
\textit{International Patent Classification}~--- IPC) 
(Обработка данных; вычисления; счет), опуб\-ли\-ко\-ван\-ные Роспатентом 
      в~2000--2012~гг. Использование информационных ресурсов Роспатента было 
обусловлено тем, что они представлены в~электронном виде, т.\,е.\ доступны для 
автоматизированной обработки. В~результате получены значения индикатора интенсивности 
цитирования (ИЦ) научных работ с~разделением по отечественным, зарубежным и~совместным 
изобретениям, запатентованным в~РФ. Такая детализация позволила оценить активность 
международного технологического сотрудничества и~совместного патентования изобретений 
по ин\-фор\-ма\-ци\-он\-но-компью\-тер\-ным технологиям (ИКТ) в~России, а~так\-же определить тематику 
сотрудничества в~этой области.}
      
      \KW{цитирование научных работ; интенсивность цитирования; взаимосвязи науки 
и~технологий; информационные технологии; Международная патентная классификация; расчет 
значений индикатора интенсивности цитирования}

 \DOI{10.14357/19922264160213} 

%\vspace*{-4pt}

\vskip 10pt plus 9pt minus 6pt

\thispagestyle{headings}

\begin{multicols}{2}

\label{st\stat}

\section{Введение}
     
  Среди различных видов источников знаний, стимулирующих появление новых 
инженерных идей, научные пуб\-ли\-ка\-ции выделяются тем, что они открыты 
и~доступны для использования~[1]. Доля изобретений, в~которых цитируются 
именно научные пуб\-ли\-ка\-ции, зависит от вида технологий. Тематически каждый 
вид технологий описывается, как правило, в~виде списка рубрик 
МПК. Международная патентная классификация~--- иерархическая система 
патентной классификации, которая является средством для рубрицирования 
патентных документов (описаний изобретений, промышленных образцов, 
полезных моделей) единообразно в~международном масштабе. 

\begin{table*}\small
\begin{center}
\Caption{Коды МПК технологий и~значения индикатора ИИЦ}
      \vspace*{2ex}
      
      \begin{tabular}{|l|c|c|}
      \hline
\multicolumn{1}{|c|}{Вид технологий}&Коды МПК вида 
технологий&ИИЦ\\
     \hline
Биотехнологии&C07G; C12M, N, P, Q, R, S&138,43\hphantom{9}\\
Фармацевтические&A61K&83,71\\
Полупроводниковые&H01L&56,44\\
Оптические&G02; G03B, C, D, F, G, H; H01S&21,89\\
Информационные&G06; G11C; G10L&20,39\\
  \hline
  \end{tabular}
  \end{center}
  \vspace*{-3pt}
  \end{table*}
  
  Наиболее часто для такого описания исполь\-зуются списки рубрик МПК из 
номенклатуры,\linebreak раз\-работанной Фраунгоферовским институтом сис\-темотехники 
и~инновационных исследований\linebreak  (Fraunhofer Gesellschaft-Institute 
f$\ddot{\mbox{u}}$r Systemtechnik und Innovationsforschung~--- FhG-ISI). 

Примеры 
списков рубрик МПК из номенклатуры FhG-ISI приведены в~табл.~1 для пяти 
видов технологий. Последний столбец содержит значения индикатора интегральной 
ИЦ (ИИЦ) результатов научных исследований, которые 
связаны с~развитием технологий, указанных в~первом столбце. Значения 
индикатора ИИЦ определены в~табл.~1 как число цитируемых научных 
пуб\-ли\-ка\-ций на~100~описаний изобретений~[2].
  

  
  Заметим, что индикатор ИИЦ не зависит от того или иного деления всей 
системы знаний на отрасли науки и~научные направления, так как учитываются 
все цитируемые в~изобретениях научные пуб\-ли\-ка\-ции по всем отраслям науки. 
Одна из наиболее актуальных задач в~широком спектре исследований 
информационных взаимосвязей науки и~технологий состоит в~вычислении 
распределения интенсивности цитирования с~учетом классификации пуб\-ли\-ка\-ций 
по конкретным научным направлениям или дисциплинам.
  
  Иначе говоря, кроме индикатора ИИЦ необходимо вычислить значения 
индикатора ИЦ для конкретных областей знаний, 
научных направлений или дисциплин. Значения индикатора ИЦ будут зависеть от 
выбора конкретной классификации отраслей знаний (Государственный рубрикатор 
на\-уч\-но-тех\-ни\-че\-ской 
информации (\mbox{ГРНТИ}), рубрикатор Российского фонда фундаментальных 
исследований (РФФИ) и~др.).
  
  Исследования по индикаторному оцениванию информационных взаимосвязей 
науки и~технологий проводятся за рубежом с~конца прошлого века~[3--12]. 
В~России аналогичные работы появились в~начале этого века и~были выполнены 
в основном в~Институте проблем информатики Российской
академии наук (ИПИ РАН)~[13--19]. 
Разработка методов вычисления значений индикатора ИЦ по конкретным 
научным направлениям выполнялась в~более общем контексте проблематики 
информационного мониторинга в~сфере науки~[20--26].
  
  В качестве индикатора ИЦ традиционно используется число цитирований 
научных пуб\-ли\-ка\-ций, упоминаемых в~описаниях изобретений.\linebreak Данный индикатор 
характеризует уровень (частоту) использования научных результатов и~тем самым 
степень воздействия фундаментальной науки на развитие технологической сферы.
  
  Результаты оценивания процессов трансфера знаний, полученные в~ИПИ РАН, 
дали возмож\-ность впервые в~РФ вычислить значения индикатора ИЦ как 
косвенного показателя интенсив\-ности взаимосвязи фундаментальной науки 
и~ИКТ,\linebreak \mbox{к~которым}, в~частности, 
относится класс G06 (Обработка данных; вычисления; счет). При этом для 
описания тематики научных статей, цитируемых в~запатентованных 
изобретениях, использовался ГРНТИ. Вычисленные значения индикатора показали, что наиболее 
часто в~изобретениях по ИКТ цитируются научные статьи по автоматике, 
вычислительной технике, кибернетике, электронике, радиотехнике, 
электротехнике и~информатике. Таким образом, применение рубрик ГРНТИ 
отражает в~основном прикладной аспект результатов, изложенных в~научных 
пуб\-ли\-ка\-ци\-ях, цитируемых в~изобретениях по ИКТ. Было показано также, что для 
получения многоаспектной картины взаимосвязей отраслей науки с~технологиями 
целесообразно также использовать фундаментальную рубрикацию отраслей 
знаний, например классификатор РФФИ~\cite{19-min}.
  
  Предметом настоящего исследования является описание и~предварительный 
анализ результатов индикаторного оценивания процессов трансфера знаний из 
разных областей науки в~сферу ИКТ. В качестве исходной информации 
использовались полнотекстовые описания изобретений по классу G06 МПК, 
опуб\-ли\-ко\-ван\-ные Роспатентом в~2000--2012~гг. В~результате их 
автоматизированной обработки получены значения индикатора ИЦ отдельно для 
отечественных, зарубежных и~совместных изо\-бре\-тений, запатентованных в~РФ.



  
  Чтобы определить интенсивность цитирования научных работ, были 
сопоставлены информационные ресурсы двух категорий~--- патентной  
и~на\-уч\-но-тех\-ни\-че\-ской:
  \begin{itemize}
\item массив полнотекстовых описаний изобретений за определенный 
период времени, которые относятся к~ИКТ;
\item библиографические описания научных работ, которые цитируются 
в~описаниях изобретений, входящих в~первый массив.
\end{itemize}

  Массив полнотекстовых описаний изобретений, на которые выданы патенты, 
является пуб\-лич\-ной информацией, размещенной на сайте Роспатента. Массив 
цитируемых научных работ был сформирован авторами статьи путем выделения 
библиографических описаний этих работ из полнотекстовых описаний 
запатентованных изобретений. В~результате обработки и~анализа второго 
массива были определены (по ГРНТИ) рубрики ННИ, к~которым относятся научные результаты, 
излагаемые в~пуб\-ли\-ка\-циях.
  
  Последующее сопоставление рубрик ННИ и~индексов МПК позволило 
определить частотность взаимосвязей между ними, характеризующую 
интенсивность цитирования в~изобретениях по ИКТ пуб\-ли\-ка\-ций, относящихся 
к~некоторой научной дисциплине, и,~как следствие, косвенно оценить 
интенсивность переноса научных знаний в~сферу изобретений и~технологий.
  
  Целью статьи является анализ экспериментальных данных, характеризующих 
интенсивность цитирования научных работ в~описаниях изобретений 
с~детализацией по отечественным и~зарубежным изобретениям, запатентованным 
в~РФ.

\vspace*{-9pt}
  
\section{Сопоставление патентной активности отечественных 
и~зарубежных заявителей}
  
  Использованная в~работе патентная информация позволяет сравнить патентную 
активность в~России отечественных и~зарубежных заявителей. По данным, 
которые пуб\-ли\-ку\-ют\-ся в~ежегодных отчетах Роспатента, в~2000~г.\ доля патентов 
РФ, выданных иностранным заявителям по всем рубрикам МПК, 
составляла~18\% от общего числа выданных патентов (всего было 
выдано~17\,592~патента РФ).\linebreak
 К~2014~г.\ эта доля возросла до~32\% (всего было\linebreak 
выдано~33\,950~патентов)~\cite{27-min, 28-min}. Индексы МПК опуб\-ли\-ко\-ван\-ных 
патентов на изобретения дают возможность определить эту долю по любому виду 
технологий, патентуемых в~России.
  
  В рамках проекта РФФИ №\,16-07-00075, первым результатам выполнения 
которого посвящена данная статья, анализировались технологии, относящиеся 
к~классу\footnote{Индексы МПК представляют собой многоуровневую иерархическую 
структуру и~соответственно уровням разделены на разделы, классы, подклассы, группы 
и~подгруппы.} G06 МПК <<Обработка данных; вычисление; счет>>. С~этой целью 
с~серверов Роспатента была отобрана информация об изобретениях данного 
класса МПК, в~том числе поля данных о~странах заявителей 
и~патентообладателей. Это дало возможность изучить распределение патентов, 
полученных в~РФ, по странам заявителей.
  
  Используемые подходы и~методы позволяют кроме индикаторов ИИЦ и~ИЦ 
расширить спектр\linebreak применяемых индикаторов и~вычислять их значения, исходя из 
потребностей наукометрических исследований в~рамках перечня 
информационных\linebreak полей, публикуемых Роспатентом, в~том числе определять 
патентную активность в~РФ изобретателей различных стран.
  
  Всего в~РФ за период с~2000 по~2012~гг.\ Роспатент опубликовал сведения 
о~6665~патентах РФ на изобретения по классу G06 (учитывались основные 
и~дополнительные индексы МПК, относящиеся к~этому классу). Как видно из 
табл.~2, права на более чем~45\%~изобретений по G06 принадлежат 
патентообладателям только из России, более~54\%~--- зарубежным 
патентообладателям только из одной страны, кроме РФ, и~менее~1\%~--- 
патентообладателям из двух и~более стран. Учитывая такое распределение, 
в~дальнейшем будем рассматривать только патенты с~правообладателями 
исключительно из одной страны.


  На рис.~1--3 представлено распределение патентов РФ на изобретения класса 
G06, опубликованных в~РФ за период 2000--2012~гг., по годам с~указанием прав 
на изобретение: РФ или другие страны. 

    
     Как следует из рис.~1, общее количество патентов РФ по классу G06 
выросло за этот период более чем в~три раза. Как видно из рис.~2, в~последние 
годы растет не только число патентов класса G06 (см.\ рис.~1), но и~доля их в~общем 
числе патентов.
     
      
  Как уже отмечалось, доля патентов РФ, выданных иностранным заявителям по 
всем рубрикам МПК, в~2014~г.\ составляла~32\%~\cite{28-min}, причем доля 
патентов РФ на изобретения по ИКТ, принадлежа- %\linebreak\vspace*{-12pt}

\end{multicols}

\begin{table*}[h]\small %tabl2
\begin{center}
\Caption{Принадлежность прав на изобретения класса G06, опубликованные в~РФ за период 
2000--2012~гг.}
     \vspace*{2ex}
     
     \begin{tabular}{|l|c|c|}
     \hline
\multicolumn{1}{|c|}{Принадлежность прав на изобретение}&Количество
патентов&Доля от общего числа патентов\\
\hline
Патентообладатели только из РФ&3012&45,19\%\\
Патентообладатели из РФ и~других стран (совместно)&\hphantom{99}28&\hphantom{9}0,42\%\\
Патентообладатели только из одной страны (не РФ)&3611&54,18\%\\
Патентообладатели из нескольких стран (не РФ)&\hphantom{99}14&\hphantom{9}0,21\%\\
\hline
\end{tabular}
\end{center}
\vspace*{-12pt}
\end{table*}

\pagebreak

\begin{figure} %fig1
      \vspace*{1pt}
 \begin{center}
 \mbox{%
 \epsfxsize=114.91mm
 \epsfbox{zac-1.eps}
 }
 \end{center}
 \vspace*{-9pt}
      \Caption{Распределение по годам патентов РФ на изобретения класса G06, 
опубликованных за период 2000--2012~гг.: \textit{1}~--- РФ; \textit{2}~--- другие страны}
%     \end{figure}
%\begin{figure} %fig2
 \vspace*{16pt}
 \begin{center}
 \mbox{%
 \epsfxsize=113.489mm
 \epsfbox{zac-2.eps}
 }
 \end{center}
 \vspace*{-9pt}
\Caption{Доля патентов класса G06 в~общем числе опубликованных патентов РФ 
по годам: \textit{1}~--- РФ; \textit{2}~--- другие страны}
\vspace*{14pt}
     \end{figure}


     

\begin{multicols}{2}

\noindent 
щих зарубежным 
патентообладателям, существенно выше. 

Как видно из рис.~3, эта доля выросла 
с~41\% в~2000~г.\ до~57\% в~2012~г.\ (максимум, почти в~70\%, приходится на 
2009~г.). В~целом это говорит о~значительном интересе, который проявляют 
иностранные компании и~фирмы, получающие патенты РФ, к~российскому рынку 
ИКТ. Этот интерес сохраняется, несмотря на то что начиная с~2010~г.\ доля 
патентов РФ на изобретения по ИКТ, принадлежащих зарубежным 
патентообладателям, имеет тенденцию к~снижению.


  
  Наибольший интерес к~патентованию в~РФ проявляют США (1500~патентов, 
т.\,е.\ половина от~3000~патентов российских патентообладателей), Южная 
Корея, Германия, Япония, Франция и~Нидерланды. 
%
На рис.~4 пред\-став\-ле\-но 
распределение патентов РФ на изобретения класса G06 по\linebreak

\end{multicols}

\begin{figure*} %fig3
 \vspace*{12pt}
 \begin{center}
 \mbox{%
 \epsfxsize=123.972mm
 \epsfbox{zac-3.eps}
 }
 \end{center}
 \vspace*{-11pt}
\Caption{Распределение патентов РФ на изобретения класса G06  по 
отечественным и~зарубежным патентообладателям, а~также по годам: 
\textit{1}~--- РФ; \textit{2}~--- другие страны}
%     \end{figure*}
%\begin{figure*} %fig4
 \vspace*{6pt}
 \begin{center}
 \mbox{%
 \epsfxsize=115.612mm
 \epsfbox{zac-4.eps}
 }
 \end{center}
 \vspace*{-11pt}
\Caption{Распределение патентов РФ на изобретения класса G06 по странам (для стран, 
у~которых число патентов не меньше~50; без патентов с~патентообладателями из разных стран)}
\vspace*{-41pt}
     \end{figure*}
     
\begin{multicols}{2}


\noindent
 странам. Для 
наглядности на рисунке пред\-став\-ле\-ны только те
 страны,
 которые имеют в~РФ не 
менее~50~патентов.

\begin{figure*} %fig5
 \vspace*{1pt}
 \begin{center}
 \mbox{%
 \epsfxsize=115.412mm
 \epsfbox{zac-5.eps}
 }
 \end{center}
 \vspace*{-9pt}
\Caption{Распределение патентов РФ на изобретения класса G06 по подклассам основного 
индекса МПК (расшифровка подклассов дана в~табл.~3): \textit{1}~--- РФ; \textit{2}~--- другие 
страны}
\vspace*{12pt}
     \end{figure*}
     
      \begin{table*}[b]\small %tabl3
%      \vspace*{12pt}
     \begin{center}
     \Caption{Названия приведенных подклассов МПК}
     \vspace*{2ex}
     
     \begin{tabular}{|l|p{140mm}|}
     \hline
\multicolumn{1}{|c|}{\tabcolsep=0pt\begin{tabular}{c}Подкласс\\ МПК\end{tabular}}&\multicolumn{1}{c|}{Название}\\
\hline
\hspace*{2mm}A61B&Диагностика; хирургия; опознание личности\\
\hline
\hspace*{2mm}B42D&Книги; книжные обложки; несброшюрованные листы\\
\hline
\hspace*{2mm}G06C&Механические цифровые вычислительные машины\\
\hline
\hspace*{2mm}G06D&Гидравлические и~пневматические цифровые вычислительные устройства\\
\hline
\hspace*{2mm}G06E&Оптические вычислительные устройства\\
\hline
\hspace*{2mm}G06F&Обработка цифровых данных с~помощью электрических устройств\\
\hline
\hspace*{2mm}G06G&Аналоговые вычислительные машины\\
\hline
\hspace*{2mm}G06J&Гибридные вычислительные устройства\\
\hline
\multicolumn{1}{|l|}{\raisebox{-6pt}[0pt][0pt]{\hspace*{2mm}G06K}}&Распознавание, представление и~воспроизведение данных; манипулирование 
носителями информации; носители информации\\
\hline
\hspace*{2mm}G06M&Счетчики; способы и~устройства для подсчета предметов, не отнесенные к~другим 
подклассам\\
\hline
\hspace*{2mm}G06N&Компьютерные системы, основанные на специфических вычислительных моделях\\
\hline
\multicolumn{1}{|l|}{\raisebox{-6pt}[0pt][0pt]{\hspace*{2mm}G06Q}}&Системы обработки данных или способы, специально предназначенные для 
административных, коммерческих, финансовых, управленческих, надзорных или 
прогностических целей\\
\hline
\hspace*{2mm}G06T&Обработка или генерация данных изображения\\
\hline
\hspace*{2mm}G07F&Монетные или подобные им автоматы\\
\hline
\hspace*{2mm}H04N&Передача изображений\\
\hline
\end{tabular}
\end{center}
\end{table*}

  \vspace*{-3pt}
       
     На рис.~5 представлено распределение по подклассам МПК патентов, 
в~описаниях которых упоминается класс G06.

%\columnbreak

 Подклассы A61B, B42D и~H04N, 
отображенные на рис.~5 (см.\ их названия в~табл.~3) не относятся к~классу G06, 
однако в~описаниях обработанных изобретений, где эти три индекса~--- основные, 
присутствовали также дополнительные индексы, относящиеся к~классу~G06.
     
     Интересно отметить, что российские патентообладатели указывают для 
изобретений по ИКТ в~качестве основных индексы, относящиеся, как правило, 
к~классу G06, в~то время как у зарубежных патентообладателей доля индексов, 
относящихся к~другим классам, выше (согласно проведенным расчетам~---~16\% 
и~25\% соответственно). Это говорит, скорее всего, о~том, что зарубежные 
заявители чаще обозначают сферы прикладного использования патентуемых ими 
ИКТ с~помощью индексов МПК.
     

     

    
     Распределение патентов РФ на изобретения подкласса G06F по группам 
МПК представлено на рис.~6. Названия групп даны в~табл.~4.

\end{multicols}

\begin{figure*} %fig6
 \vspace*{1pt}
 \begin{center}
 \mbox{%
 \epsfxsize=114.412mm
 \epsfbox{zac-6.eps}
 }
 \end{center}
 \vspace*{-9pt}
\Caption{Распределение патентов РФ на изобретения подкласса G06F по группам 
МПК (приведены группы подкласса G06F, содержащие более~20~патентов): 
\textit{1}~--- РФ; \textit{2}~--- другие страны}
     \end{figure*}




\begin{table*}\small %tabl4
\begin{center}
\Caption{Названия групп МПК, приведенных на рис.~6}
      \vspace*{2ex}
      
      \begin{tabular}{|l|p{137mm}|}
      \hline
\multicolumn{1}{|c|}{Группа МПК}&\multicolumn{1}{c|}{Название}\\
\hline
G06F 1/00&Конструктивные элементы вычислительных машин и~устройств для обработки 
данных\\
\hline
G06F 11/00&Обнаружение ошибок, исправление ошибок; контроль\\
\hline
G06F 12/00&Выборка, адресация или распределение данных в~системах или архитектурах 
памяти\\
\hline
\multicolumn{1}{|l|}{\raisebox{-6pt}[0pt][0pt]{G06F 13/00}}&Соединение запоминающих устройств, устройств ввода-вывода или устройств 
центрального процессора\\
\hline
G06F 15/00&Цифровые компьютеры\\
\hline
\multicolumn{1}{|l|}{\raisebox{-6pt}[0pt][0pt]{G06F 17/00}}&Устройства или методы цифровых вычислений или обработки данных, специально 
предназначенные для специфических функций\\
\hline
\multicolumn{1}{|l|}{\raisebox{-6pt}[0pt][0pt]{G06F 19/00}}&Устройства или способы цифровых вычислений или обработки данных для 
специальных применений\\
\hline
\multicolumn{1}{|l|}{\raisebox{-6pt}[0pt][0pt]{G06F 21/00}}&Устройства защиты компьютеров или компьютерных систем от 
несанкционированной деятельности\\
\hline
\multicolumn{1}{|l|}{\raisebox{-12pt}[0pt][0pt]{G06F 3/00}}&Вводные устройства для передачи данных, подлежащих преобразованию в~форму, 
пригодную для обработки в~вычислительной машине; выводные устройства для передачи 
данных из устройств обработки в~устройства вывода\\
\hline
\multicolumn{1}{|l|}{\raisebox{-6pt}[0pt][0pt]{G06F 7/00}}&Способы и~устройства для обработки данных с~воздействием на порядок их 
расположения или на содержание обрабатываемых данных\\
\hline
G06F 9/00&Устройства для программного управления\\
\hline
\end{tabular}
\end{center}
\end{table*}

\begin{multicols}{2}

\begin{table*}\small  %tabl5
%\vspace*{-6pt}
\begin{center}
\Caption{Число цитирований научных работ для изобретений с~патентообладателями из РФ}
      \vspace*{2ex}
      
      \begin{tabular}{|c|c|c|c|c|c|c|c|c|}
      \hline
Подкласс&\multicolumn{8}{c|}{Код рубрики ГРНТИ}\\
\cline{2-9}
МПК&13.00.00&20.00.00&27.00.00&28.00.00&45.00.00&47.00.00&50.00.00&76.00.00\\
\hline
G06E&0&\hphantom{9}0&0&\hphantom{9}0&\hphantom{9}0&16&\hphantom{9}0&0\\
G06F&2&423\hphantom{9}&59\hphantom{9}&758\hphantom{9}&469\hphantom{9}&517\hphantom{9}&915\hphantom{9}&0\\
G06G&0&\hphantom{9}0&0&\hphantom{9}2&15&27&14&0\\
G06K&0&539\hphantom{9}&4&587\hphantom{9}&550\hphantom{9}&602\hphantom{9}&655\hphantom{9}&2\\
G06N&0&19&1&14&30&26&19&1\\
G06Q&0&\hphantom{9}0&0&\hphantom{9}8&\hphantom{9}0&\hphantom{9}0&\hphantom{9}1&5\\
G06T&0&\hphantom{9}8&51\hphantom{9}&51&\hphantom{9}8&17&80&0\\
\hline
     \end{tabular}
     \end{center}
%\end{table*}
%     \begin{table*}\small %tabl6
\begin{center}
\Caption{Число цитирований научных работ для изобретений патентообладателей из 
зарубежных стран}
     \vspace*{2ex}
     
     \tabcolsep=7.5pt
     \begin{tabular}{|c|c|c|c|c|c|c|c|c|c|c}
     \hline
Подкласс &\multicolumn{8}{c|}{Код рубрики ГРНТИ}\\
\cline{2-9}
МПК&13.00.00&20.00.00&27.00.00&28.00.00&45.00.00&47.00.00&50.00.00&76.00.00\\
\hline
G06E&0&0&0&0&0&0&0&0\\
G06F&0&0&45\hphantom{9}&46\hphantom{9}&9&9&54\hphantom{9}&0\\
G06G&0&0&0&0&0&0&0&0\\
G06K&0&0&31\hphantom{9}&13\hphantom{9}&28\hphantom{9}&36\hphantom{9}&65\hphantom{9}&0\\
G06N&0&1&1&1&0&0&1&0\\
G06Q&0&0&0&0&0&0&0&0\\
G06T&0&0&76\hphantom{9}&78\hphantom{9}&32\hphantom{9}&48\hphantom{9}&137\hphantom{99}&2\\
\hline
\end{tabular}
\end{center}
\end{table*}

\section{Интенсивность цитирования научных публикаций}
 
     В работе~\cite{19-min} были приведены значения индикатора ИЦ 
(в~описаниях изобретений по классу G06) для научных работ, тематика 
первоисточников которых (журналов и~трудов конференций) была описана 
с~помощью рубрик ГРНТИ. Определение значений этого индикатора ранее было 
выполнено без детализации по странам патентообладателей. Учет страны 
позволяет выявить различия в~интенсивности цитирования научных работ 
в~российских и~иностранных изобретениях по ИКТ, на которые были выданы 
патенты РФ. 

Таблицы~5 и~6 содержат данные о числе процитированных научных 
публикаций, соответствующих как рубрике ГРНТИ, так и~подклассу МПК, что 
позволяет охарактеризовать интенсивность взаимосвязи конкретного научного 
направления (по ГРНТИ) и~технологии (по подклассу МПК). По уровню 
использования научных результатов, что косвенно отображается чис\-лом 
цитирований научных работ, для технологий подклассов G06F и~G06K важны 
практически все научные направления, указанные в~табл.~5 и~6, причем по чис\-лу 
цитирований выделяются результаты по научным направлениям ГРНТИ: 
<<Кибернетика>> (28.00.00) <<Электроника. Радиотехника>> (47.00.00), 
<<Автоматика. Вычислительная техника>> (50.00.00). Первые строки этих таб\-лиц 
содержат коды восьми рубрик ГРНТИ, которые использовались для описания 
тематики цитируемых научных работ. Названия рубрик даны в~табл.~7.
     

     
     Как видно из табл.~5, самой высокой цитируемостью научных работ 
характеризуются изобретения российских патентообладателей для технологий 
подкласса G06F (Обработка цифровых\linebreak данных с~помощью электрических 
устройств) и~G06K (Распознавание, представление и~воспроизведение данных; 
манипулирование носителями информации; носители информации). 
Показательно, что на долю именно этих технологий приходится наибольшее 
число выданных патентов РФ, что говорит о~более интенсивном развитии данных 
технологий в~РФ.
     


\begin{table*}\small %tabl7
     \begin{center}
     \Caption{Рубрики ГРНТИ источников, публикации которых
     цитируются в~изобретениях подкласса МПК G06F}
     \vspace*{2ex}
     
     \tabcolsep=7pt
     \begin{tabular}{|c|p{110mm}|r|c|}
     \hline
Код ГРНТИ&\multicolumn{1}{c|}{Название рубрики ГРНТИ}&\multicolumn{1}{c|}{РФ}&
\tabcolsep=0pt\begin{tabular}{c}Другие\\ страны\end{tabular}\\
\hline
13.00.00&КУЛЬТУРА. КУЛЬТУРОЛОГИЯ&2&\\
20.00.00&ИНФОРМАТИКА&423&\\
27.00.00&МАТЕМАТИКА&44&45\\
27.03.00&Математическая логика и~основания математики&2&\\
27.35.00&Математические модели естественных наук и~технических наук. Уравнения математической 
физики&2&\\
27.41.00&Вычислительная математика&11&\\
28.00.00&КИБЕРНЕТИКА&702&46\\
28.17.00&Теория моделирования&28&\\
28.19.00&Теория кибернетических систем управления&2&\\
28.23.00&Искусственный интеллект&26&\\
45.00.00&ЭЛЕКТРОТЕХНИКА&468&\hphantom{9}9\\
45.53.00&Электротехническое оборудование специального назначения&1&\\
47.00.00&ЭЛЕКТРОНИКА. РАДИОТЕХНИКА&459&\hphantom{9}9\\
47.01.00&Общие вопросы электроники и~радиотехники&4&\\
47.03.00&Теоретические основы электронной техники&14&\\
47.05.00&Теоретическая радиотехника&2&\\
47.14.00&Проектирование и~конструирование электронных приборов&2&\\
47.29.00&Электровакуумные и~газоразрядные приборы и~устройства&2&\\
47.33.00&Твердотельные приборы&2&\\
47.37.00&Голография&8&\\
47.41.00&Радиоэлектронные схемы&2&\\
47.43.00&Распространение радиоволн&2&\\
47.45.00&Антенны. Волноводы. Элементы СВЧ-техники&2&\\
47.47.00&Радиопередающие и~радиоприемные устройства&2&\\
47.49.00&Радиотехнические системы зондирования, локации и~навигации&2&\\
47.51.00&Телевизионная техника&2&\\
47.53.00&Запись и~воспроизведение сигналов&2&\\
47.55.00&Электроакустика, ультразвуковая и~инфразвуковая техника&2&\\
47.57.00&Инфракрасная техника&2&\\
47.59.00&Узлы, детали и~элементы радиоэлектронной аппаратуры&2&\\
47.61.00&Приборы для радиотехнических измерений&2&\\
47.63.00&Системы и~устройства отображения информации&2&\\
50.00.00&АВТОМАТИКА. ВЫЧИСЛИТЕЛЬНАЯ ТЕХНИКА&891&54\\
50.07.00&Теоретические основы вычислительной техники&4&\\
50.09.00&Элементы, узлы и~устройства автоматики и~вычислительной техники&2&\\
50.11.00&Запоминающие устройства&4&\\
50.41.00&Программное обеспечение вычислительных машин, комплексов и~сетей&2&\\
50.45.00&Системы телеуправления и~телеизмерения&2&\\
50.51.00&Автоматизация проектирования&4&\\
50.53.00&Автоматизация научных исследований&6&\\
50.00.00&ПРИБОРОСТРОЕНИЕ&78&\\
59.01.00&Общие вопросы приборостроения&2&\\
59.14.00&Проектирование и~конструирование приборов&4&\\
59.41.00&Приборы для измерения оптических и~светотехнических величин и~характеристик&6&\\
59.45.00&Приборы неразрушающего контроля изделий и~материалов&2&\\
\hline
\end{tabular}
\end{center}
\end{table*}
     
     В то же время, как следует из табл.~6, для изобретений зарубежных 
патентообладателей по интенсивности цитирования научных работ выделяется 
технология G06T (Обработка или генерация данных изображения), хотя число 
патентов в~подклассе G06T (см.\ рис.~5) меньше, чем в~G06F и~G06K 
(около~10\%).
     
     В табл.~7 даны названия ряда рубрик ГРНТИ, для которых приведены 
данные о~чис\-ле процитированных научных публикаций для подкласса МПК G06F. 
Как видно из таблицы, для большинства научных работ, процитированных 
в~описаниях изобретений, указаны рубрики ГРНТИ самого верхнего уровня 
иерархии, что, конечно, снижает информативность выявленных связей и~говорит 
о~необходимости уточнения рубрик публикаций, цитируемых авторами 
изобретений.
     
    \begin{figure*} %fig7
\vspace*{1pt}
 \begin{center}
 \mbox{%
 \epsfxsize=147.489mm
 \epsfbox{zac-7.eps}
 }
 \end{center}
 \vspace*{-9pt}
\Caption{Распределение <<периодов патентного отклика>> для класса G06 в~качестве 
основного индекса: \textit{1}~--- РФ; \textit{2}~--- другие страны}
     \end{figure*} 

\section{Период патентного отклика на~научные 
публикации}
     
     Временн$\acute{\mbox{ы}}$е аспекты информационных взаимосвязей 
науки и~технологий характеризуются <<периодом патентного отклика>>, т.\,е.\ 
промежутком времени, прошедшим с~момента выхода научных публи\-каций до 
опубликования патентов, в~описании изобретений которых цитируются данные 
пуб\-ли\-ка\-ции. Значения показателя <<период патентного отклика>> дают 
возможность оценить динамику востребованности результатов научных 
исследований в~сфере разработки технологий.
     
     В расчетах, представленных в~работе~\cite{19-min}, были определены 
значения этого показателя для научных публикаций, цитируемых в~изобретениях 
по классу G06. Расчеты, проведенные по полному массиву патентов класса G06 
(в~качестве основного индекса МПК), показывают, что максимум распределения 
значений равен~3~годам для патентообладателей из России и~9~годам~--- для 
зарубежных патентообладателей.
     
     Рисунок~7 демонстрирует распределение <<периодов патентного 
отклика>> для класса G06 для массивов как российских, так и~зарубежных 
патентообладателей (по оси абсцисс указана продолжительность <<периодов 
патентного отклика>> в~годах; по оси ординат для каждого массива~--- доля  
от общего числа цитированных научных публикаций в~данном массиве).


      
     
     Были также получены распределения <<периодов патентного отклика>> для 
отдельных наиболее многочисленных по числу патентов подклассов G06, 
а~именно: G06F, G06K, G06T (рис.~8 и~9).



     
     Для подкласса G06F <<Обработка цифровых данных с~помощью 
электрических устройств>> (см.\ рис.~8,\,\textit{а}), распределения имеют максимумы 
в~6 и~9~лет соответственно.


     Для подкласса G06K <<Распознавание, пред\-став\-ле\-ние и~воспроизведение 
данных; манипулирование носителями информации; носители 
информации>> (см.\ рис.~8,\,\textit{б}) распределения имеют по два максимума в~3 и~22~года 
и~в~5 и~9~лет.



     
     Для подкласса G06T <<Обработка или генерация данных 
изображения$\ldots$>> (см.\ рис.~9), распределения имеют по два максимума в~5 
и~9~лет и~в~6 и~7~лет.
     
     Отметим существенно разный характер полученных распределений для 
различных подклассов. Для выяснения причин различий в~распределениях 
значений данного показателя требуется содержательный анализ изобретений 
и~цитируемых в~них научных публикаций, что выходит за рамки настоящей 
статьи.
     
\section{Заключение}

%\vspace*{-3pt}
     
     В ходе работ по оцениванию процессов трансфера знаний разработаны 
методы и~технологии анализа активности патентования в~РФ с~использованием 
индикаторов, которые учитывают страну патентообладателя. Вычислены 
значения индикаторов ИЦ, которые демонстрируют значительные\linebreak различия 
в~интенсивности цитирования научных публикаций российскими и~зарубежными 
авторами. Экспериментально показано, что интенсивность цитирования зависит 
не только от вида\linebreak технологий (см.\ табл.~1), но и~от страны патентообладателя 
(см.\ табл.~5 и~6).
     
     Например, самая высокая цитируемость научных работ отмечена 
в~описаниях изобретений российских патентообладателей для технологий 
подкласса G06F (Обработка цифровых данных \mbox{с~помощью} электрических 
устройств) и~G06K (Рас-\linebreak\vspace*{-12pt}

\pagebreak

\end{multicols}

\begin{figure*} %fig8
\vspace*{1pt}
 \begin{center}
 \mbox{%
 \epsfxsize=147.489mm
 \epsfbox{zac-8.eps}
 }
 \end{center}
 \vspace*{-9pt}
\Caption{Распределение <<периодов патентного отклика>> для подклассов G06F~(\textit{а})
и~G06K~(\textit{б}) (основной 
индекс): \textit{1}~--- РФ; \textit{2}~--- другие страны}
\vspace*{2pt}
     \end{figure*}

\begin{multicols}{2}


\noindent
познавание, представление и~воспроизведение данных; 
манипулирование носителями информации; носители информации).

\begin{figure*} %fig9
\vspace*{1pt}
 \begin{center}
 \mbox{%
 \epsfxsize=147.503mm
 \epsfbox{zac-10.eps}
 }
 \end{center}
 \vspace*{-9pt}
\Caption{Распределение времени между публикацией статьи и~патента подкласса G06T 
(основной индекс) для патентообладателей из разных стран (в процентах к~общему числу 
цитируемых изобретателями публикаций в~каждой группе); по горизонтальной оси отложена 
разница в~годах: \textit{1}~--- РФ; \textit{2}~--- другие страны}
%\vspace*{-7pt}
     \end{figure*}
 

В~то же 
время в~описаниях изобретений зарубежных патентообладателей по 
интенсивности цитирования научных работ выделяется технология G06T 
(Обработка или генерация данных изображения). При этом число патентов 
в~подклассе G06T существенно меньше, чем в~G06F и~G06K.
     
     Впервые были получены распределения значений <<периода патентного 
отклика>>, которые дают возможность оценить динамику применения 
результатов научных исследований в~сфере разработки технологий. Расчеты, 
проведенные в~данной работе, показывают, что для изобретений по ИКТ 
максимум распределения значений этого индикатора составляет~3~года для 
патентообладателей из России и~9~лет для зарубежных патентообладателей. 
Однако необходимо принимать во внимание и~период времени от момента подачи 
заявки до публикации запатентованного изобретения. Если построить 
аналогичные распределения с~учетом момента подачи заявки, то максимумы этих 
распределений сместятся влево.


\vspace*{-9pt}
  
{\small\frenchspacing
 {%\baselineskip=10.8pt
 \addcontentsline{toc}{section}{References}
 \begin{thebibliography}{99}
\bibitem{1-min}
\Au{Giuri~P., Mariani~M., Brusoni~S., Crespi~G., Francoz~D., Gambardella~A., 
Garcia-Fontes~W., Geuna~A., Gonzales~R., Harhoff~D., Hoisl~K., Le Bas~C., Luzzi~A.,
 Magazzini~L., 
Nesta~L., Nomaler~$\ddot{\mbox{O}}$., Palomares~N., P.~Patel,
Romanelli~M., Verspagen~B.} Inventors 
and invention processes in Europe: Results from the PatVal-EU survey~// Res. Policy, 2007. 
Vol.~36. No.\,8. P.~1107--1127.
\bibitem{2-min}
\Au{Van~Looy~B., Zimmermann~E., Veugelers~R., Verbeek~A., Mello~J., Debackere~K.} Do 
science-technology interactions pay on when developing technology? An exploratory investigation 
of~10~science-intensive technology domains~// Scientometrics, 2003. Vol.~57. No.\,3.  
P.~355--367.
\bibitem{3-min}
\Au{Narin~F., Noma~E.} Is technology becoming science?~// Scientometrics, 1985. Vol.~7.  
No.\,3--6. P.~369--381.
\bibitem{7-min} %4
\Au{Mansfield E.} Academic research and innovation~// Res. Policy, 1991. Vol.~20. No.\,1. 
P.~1--12.
\bibitem{4-min} %5
\Au{Schmoch~U.} Tracing the knowledge transfer from science to technology as reflected in patent 
indicators~// Scientometrics, 1993. Vol.~26. No.\,1. P.~193--211.
\bibitem{8-min} %6
\Au{Mansfield~E.} Academic research underlying industrial innovations: Sources, characteristics 
and financing~// Rev. Econ. Stat., 1995. Vol.~77. No.\,1. P.~55--62.

\bibitem{6-min} %7
\Au{Narin~F., Olivastro~D.} Linkage between patents and papers: An interim EPO/US 
comparison~// Scientometrics, 1998. Vol.~41. No.\,1--2. P.~51--59.


\bibitem{9-min} %8
\Au{Mansfield~E.} Academic research and industrial innovation: An update of empirical findings~// 
Res. Policy, 1998. Vol.~26. No.\,7--8. P.~773--776.

\bibitem{5-min} %9
\Au{Tijssen~R.\,J.\,W., Buter~R.\,K., Van Leeuwen~Th.\,N.} 
Technological relevance of science: An 
assessment of citation linkages between patents and research papers~// Scientometrics, 2000. 
Vol.~47. No.\,2. P.~389--412.

\bibitem{11-min} %10
\Au{Verbeek~А., Debackere~K., Luwel~M., Andries~P., Zimmermann~E., Deleus~D.} Linking 
science to technology: Using bibliographic references in patents to build linkage schemes~// 
Scientometrics, 2002. Vol.~54. No.\,3. P.~399--420.

\bibitem{10-min} %11
European Commission. Third European Report on Science \& Technology Indicators.~--- 
Luxembourg: Office for Official Publications of the European Communities, 2003. 451~p.

\bibitem{12-min} %12
\Au{Van~Looy~B., Hansen~W., Hollanders~H., Tijssen~R.} Using concordance tables to 
disentangle performance dynamics of HT manufacturing industries:
An empirical assessment of 
national innovation systems~//  10th Conference (International) on Science and Technology 
Indicators Proceedings: Book of abstracts.~--- Vienna: ARC GmbH, 2008.  
P.~196--200.
\bibitem{13-min}
\Au{Зацман~И.\,М., Шубников С.\,К.} Принципы обработки информационных ресурсов для 
оценки инновационного потенциала направлений научных исследований~// Электронные 
библиотеки: перспективные методы и~технологии, электронные коллекции: Тр. IX Всеросс. 
науч. конф. RCDL'2007.~--- Переславль: Университет города Переславля, 2007. С.~35--44.
\bibitem{14-min}
\Au{Минин~В.\,А., Зацман~И.\,М., Кружков~М.\,Г., Норекян~Т.\,П.} Методологические 
основы создания информационных систем для вычисления индикаторов тематиче\-ских 
взаимосвязей науки и~технологий~// Информатика и~её применения, 2013. Т.~7. Вып.~1. 
С.~70--81.
\bibitem{15-min}
\Au{Минин~В.\,А., Зацман~И.\,М., Хавансков~В.\,А., Шубников~С.\,К.} Архитектурные 
решения для систем вы\-чис\-ле\-ния индикаторов тематических взаимосвязей науки 
и~технологий~// Системы и~средства информатики, 2013. Т.~23. №\,2. C.~260--283.
\bibitem{16-min}
\Au{Зацман~И.\,М., Хавансков~В.\,А., Шубников~С.\,К.} Метод извлечения 
библиографической информации из полнотекстовых описаний изобретений~// Информатика 
и её применения, 2013. Т.~7. Вып.~4. С.~52--65.
\bibitem{17-min}
\Au{Хавансков~В.\,А., Шубников~С.\,К.} Поиск и~рубрицирование ссылок на цитируемые 
публикации в~электронных библиотеках полнотекстовых описаний\linebreak изобретений~// 
Электронные библиотеки: перспективные методы и~технологии, электронные коллекции: 
Тр. XVI Всеросс. науч. конф. RCDL-2014.~---  Дубна: \mbox{ОИЯИ}, 2014. С.~165--173.
\bibitem{18-min}
\Au{Минин~В.\,А., Зацман~И.\,М., Хавансков~В.\,А., Шубников~С.\,К.} Индикаторы 
тематических взаимосвязей науки и~технологий: от текста к~числам~// Информатика и~её 
применения, 2014. Т.~8. Вып.~3. С.~114--125.
\bibitem{19-min}
\Au{Минин~В.\,А., Зацман~И.\,М., Хавансков~В.\,А., Шубников~С.\,К.} Индикаторы 
тематических взаимосвязей науки и~информационно-компьютерных технологий в~начале 
XXI~века~// Информатика и~её применения, 2015. Т.~9. Вып.~2. С.~111--120.
\bibitem{20-min}
\Au{Зацман~И.\,М., Веревкин~Г.\,Ф.} Информационный мониторинг сферы науки в~задачах 
программно-це\-ле\-во\-го управления~// Системы и~средства информатики, 2006. 
Вып.~16. С.~164--189.
\bibitem{21-min}
\Au{Зацман И.\,М., Кожунова~О.\,С.} Семантический словарь системы информационного 
мониторинга в~сфере науки: задачи и~функции~// Системы и~средства информатики, 2007. Вып.~17. С.~124--141.

\bibitem{26-min} %22
\Au{Zatsman~I., Kozhunova~O.} Evaluating for institutional academic activities: Classification 
scheme for R\&D indicators~// 10th Conference (International) on Science and Technology 
Indicators: Book of abstracts.~--- Vienna: ARC GmbH, 2008. P.~428--431.
\bibitem{25-min} %23
\Au{Zatsman~I., Kozhunova~O.} Evaluation system for the Russian Academy of Sciences: 
Objectives-resources-results approach and R\&D indicators~// Atlanta Conference on Science 
and Innovation Policy Proceedings~/ Eds. S.\,E.~Cozzens, P.~Catalаn. 2009. {\sf 
http://smartech. gatech.edu/bitstream/1853/32300/1/104-674-1-PB.pdf}.

\bibitem{22-min} %24
\Au{Архипова~М.\,Ю., Зацман~И.\,М., Шульга~С.\,Ю.} Индикаторы патентной активности 
в~сфере ин\-фор\-ма\-ци\-он\-но-ком\-му\-ни\-ка\-ци\-он\-ных технологий и~методика их вычисления~// 
Экономика, статистика и~информатика. Вестник УМО, 2010. №\,4. С.~93--104.
\bibitem{24-min} %25
\Au{Zatsman~I., Durnovo~A.} Incompleteness problem of indicators system of research 
programme~// 11th Conference (International) on Science and Technology Indicators: 
Book of abstracts.~--- Leiden: CWTS, 2010. P.~309--311.
\bibitem{23-min} %26
\Au{Зацман~И.\,М., Дурново~А.\,А.} Моделирование процессов формирования экспертных 
знаний для мониторинга программно-целевой деятельности~// Информатика и~её 
применения, 2011. Т.~5. Вып.~4. С.~84--98.
\bibitem{27-min}
Российское агентство по патентам и~товарным знакам (Роспатент): Годовой отчет 2000. 
{\sf http://www1. fips.ru/wps/wcm/connect/content\_ru/ru/otchety/\linebreak otchet\_2000\_r6}.
\bibitem{28-min}
Федеральная служба по интеллектуальной собственности (Роспатент): Годовой отчет 2014. 
{\sf http://www. rupto.ru/about/reports/2014\_1\#1.2}.

\end{thebibliography}

 }
 }

\end{multicols}

\vspace*{-3pt}

\hfill{\small\textit{Поступила в~редакцию 19.04.16}}

%\vspace*{8pt}

\newpage

\vspace*{-24pt}

%\hrule

%\vspace*{2pt}

%\hrule

%\vspace*{8pt}



\def\tit{INTENSITY OF~CITATION OF~SCIENTIFIC PUBLICATIONS IN~INVENTIONS 
ON INFORMATION AND~COMPUTER TECHNOLOGIES PATENTED 
IN~RUSSIA BY~DOMESTIC AND~FOREIGN APPLICANTS}

\def\titkol{Intensity of~citation of~scientific publications in~inventions 
on ICT patented 
in~Russia by~domestic and~foreign applicants}

\def\aut{V.\,A.~Minin, I.\,M.~Zatsman, V.\,A.~Havanskov, and S.\,K.~Shubnikov}

\def\autkol{V.\,A.~Minin, I.\,M.~Zatsman, V.\,A.~Havanskov, and S.\,K.~Shubnikov}

\titel{\tit}{\aut}{\autkol}{\titkol}

\vspace*{-9pt}

\noindent
Institute of Informatics Problems, Federal Research Center 
``Computer Science and Control'' of the Russian Academy of Sciences,
44-2~Vavilov Str., Moscow 119333, Russian Federation


\def\leftfootline{\small{\textbf{\thepage}
\hfill INFORMATIKA I EE PRIMENENIYA~--- INFORMATICS AND
APPLICATIONS\ \ \ 2016\ \ \ volume~10\ \ \ issue\ 2}
}%
 \def\rightfootline{\small{INFORMATIKA I EE PRIMENENIYA~---
INFORMATICS AND APPLICATIONS\ \ \ 2016\ \ \ volume~10\ \ \ issue\ 2
\hfill \textbf{\thepage}}}

\vspace*{12pt}


 


\Abste{The paper discusses the information relationship between science and technology 
and the methods 
for indicator assessment of transfer processes (transfer) of knowledge from different fields of research 
in the area of technological development. The proposed methods are designed to determine the values 
of the indicator of intensity of citation of scientific papers in the descriptions of the inventions patented 
in Russia by domestic and foreign applicants. A~similar approach can be used to obtain indirect 
estimates of innovation potential of scientific research. The indicator values of intensity were 
calculated both in general and with the distribution by country of applicants. The paper presents the 
results of determining the values of the indicator. Full-text descriptions of inventions on class G06 of 
the International Patent Classification (Data Processing; Computing; Score) published by Rospatent in 
2000--2012 were used as the source of information. The use of information resources of 
Rospatent was due to the fact that they are in the electronic form, i.\,e., available for automated 
processing. The result is the values of the indicator of intensity of citation of scientific works patented 
in the Russian Federation, divided into groups of domestic, foreign, 
and joint inventions. This 
specification allowed to estimate the activity of international technological cooperation and joint 
patenting in information and computer technologies (ICT) in Russia, as well as to determine the 
themes of cooperation in this area.}

\KWE{citation of scientific papers; intensity of citation linkages between science and technologies; 
information technology; international patent classification; calculation of values of the indicator of 
intensity of citation}


\DOI{10.14357/19922264160213}

%\vspace*{-12pt}

\Ack
\noindent
The work was financially supported 
by the Russian Foundation for Basic Research (project 16-07-00075).


\vspace*{9pt}

  \begin{multicols}{2}

\renewcommand{\bibname}{\protect\rmfamily References}
%\renewcommand{\bibname}{\large\protect\rm References}

{\small\frenchspacing
 {%\baselineskip=10.8pt
 \addcontentsline{toc}{section}{References}
 \begin{thebibliography}{99}
\bibitem{1-min-1}
\Aue{Giuri,~P., M.~Mariani, S.~Brusoni, G.~Crespi, D.~Francoz, A.~Gambardella, 
W.~Garcia-Fontes, 
A.~Geuna, R.~Gonzales, D.~Harhoff, K.~Hoisl, C.~Le Bas, A.~Luzzi, L.~Magazzini, L.~Nesta, 
$\ddot{\mbox{O}}$.~Nomaler, N.~Palomares, P.~Patel,
M.~Romanelli, and B.~Verspagen}. 2007. Inventors 
and invention processes in Europe: Results from the PatVal-EU survey. \textit{Res. Policy} 
36(8):1107--1127.
\bibitem{2-min-1}
\Aue{Van~Looy,~B., E.~Zimmermann, R.~Veugelers, A.~Verbeek, J.~Mello, and K.~Debackere}. 
2003. Do science-technology interactions pay on when developing technology? An exploratory 
investigation of 10 science-intensive technology domains. \textit{Scientometrics} 57(3):355--367.
\bibitem{3-min-1}
\Aue{Narin, F., and E.~Noma.} 1985. Is technology becoming science? \textit{Scientometrics}  
7(3--6):369--381.
\bibitem{7-min-1} %4
\Aue{Mansfield,~E.} 1991. Academic research and innovation. \textit{Res. Policy} 20(1):1--12.
\bibitem{4-min-1} %5
\Aue{Schmoch,~U.} 1993. Tracing the knowledge transfer from science to technology as reflected in 
patent indicators. \textit{Scientometrics} 26(1):193--211.

\bibitem{8-min-1} %6
\Aue{Mansfield,~E.} 1995. Academic research underlying industrial innovations: Sources, 
characteristics and financing. \textit{Rev. Econ. Stat.} 77(1):55--62.

\bibitem{6-min-1} %7
\Aue{Narin, F., and D.~Olivastro}. 1998. Linkage between patents and papers: An interim EPO/US 
comparison. \textit{Scientometrics} 41(1--2):51--59.


\bibitem{9-min-1} %8
\Aue{Mansfield, E.} 1998. Academic research and industrial innovation: An update of empirical 
findings. \textit{Res. Policy} 26(7--8):773--776.
\bibitem{5-min-1} %9
\Aue{Tijssen,~R.\,J.\,W., R.\,K.~Buter, and Th.\,N.~Van Leeuwen}. 2000. Technological relevance of 
science: An assessment of citation linkages between patents and research papers. 
\textit{Scientometrics} 47(2):389--412.

\bibitem{11-min-1} %10
\Aue{Verbeek,~А., K.~Debackere, M.~Luwel, P.~Andries, E.~Zimmermann, and D.~Deleus}. 2002. 
Linking science to technology: Using bibliographic references in patents to build linkage schemes. 
\textit{Scientometrics} 54(3):399--420.
\bibitem{10-min-1} %11
European Commission. 2003. Third European Report on Science \& Technology Indicators. 
Luxembourg: Office for Official Publications of the European Communities. 451~p.
\bibitem{12-min-1}
\Aue{Van~Looy,~B., W.~Hansen, H.~Hollanders, and R.~Tijssen}. 2008. Using concordance tables to 
disentangle performance dynamics of HT manufacturing industries: An empirical assessment of 
national innovation systems. \textit{10th Conference (International) on Science and Technology 
Indicators Proceedings: Book of abstracts}. Vienna: ARC GmbH. 196--200.
\bibitem{13-min-1}
\Aue{Zatsman,~I.\,M., and S.\,K.~Shubnikov}. 2007. Printsipy obrabotki informatsionnykh resursov 
dlya otsenki in\-no\-va\-tsi\-on\-no\-go potentsiala napravleniy nauchnykh issledovaniy [Principles of 
processing of information resources for assessment of innovative potential of fields of scientific 
research]. \textit{Elektronnye Biblioteki: Perspektivnye Metody i~Tekhnologii, Elektronnye Kollektsii: 
Tr. 9-y Vseross. nauch. konf. RCDL'2007} [Digital Libraries: Perspective Methods and Technologies, 
Electronic collections: 9th All-Russia Scientific Conference RCDL'2007 Proceedings]. Pereslavl'.  
35--44.
\bibitem{14-min-1}
\Aue{Minin,~V.\,A., I.\,M.~Zatsman, M.\,G.~Kruzhkov, and T.\,P.~No\-re\-kyan}. 2013. 
Metodologicheskie osnovy so\-zda\-niya informatsionnykh sistem dlya vychisleniya indikatorov 
tematicheskikh vzaimosvyazey nauki i~tekhnologiy [Methodological basis for the creation of 
information systems for the calculation of indicators of thematic science-technology linkagies]. 
\textit{Informatika i~ee Primeneniya}~--- \textit{Inform. Appl.} 7(1):70--81.
\bibitem{15-min-1}
\Aue{Minin,~V.\,A., I.\,M.~Zatsman, V.\,A.~Havanskov, and S.\,K.~Shubnikov}. 2013. 
Arkhitekturnye resheniya dlya sis\-tem vychisleniya indikatorov tematicheskikh 
vza\-imo\-svya\-zey nauki 
i~tekh\-no\-lo\-giy [Architectural decisions for systems of calculation of indicators of thematic science-
technology linkagies]. \textit{Sistemy i~Sredstva Informatiki}~--- \textit{Systems and Means of 
Informatics} 23(2):260--283.
\bibitem{16-min-1}
\Aue{Zatsman,~I.\,M., V.\,A.~Havanskov, and S.\,K.~Shubnikov}. 2013. Metod izvlecheniya 
bibliograficheskoy in\-for\-ma\-tsii iz polnotekstovykh opisaniy izobreteniy [Method of extraction of 
bibliographic information from full-text descriptions of inventions]. \textit{Informatika i~ee 
Primeneniya}~--- \textit{Inform. Appl.} 7(4):52--65.
\bibitem{17-min-1}
\Aue{Havanskov,~V.\,A., and S.\,K.~Shubnikov}. 2014. Poisk i~rub\-ri\-tsi\-ro\-vanie ssylok na tsitiruemye 
publikatsii v~elektronnykh bibliotekakh polnotekstovykh opisaniy izobreteniy [Search and classifying 
of cited publications in digital libraries of full-text descriptions of inventions]. \textit{Elektronnye 
Biblioteki: Perspektivnye Metody i~Tekhnologii, Elektronnye Kollektsii: Tr. 16-y 
Vseross. nauch. 
konf. RCDL'2014} [Digital Libraries: Perspective Methods and Technologies, Electronic Collections: 
16th All-Russia Scientific Conference RCDL'2014 Proceedings]. Dubna. 165--173.

\columnbreak 

\bibitem{18-min-1}
\Au{Minin,~V.\,A., I.\,M.~Zatsman, V.\,A.~Havanskov, and S.\,K.~Shubnikov}. 2014. Indikatory 
tematicheskikh vzaimosvyazey nauki i~tekhnologiy: Ot teksta k~chislam [Indicators of thematic 
science-technology linkagies: From text to numbers]. \textit{Informatika i~ee Primeneniya}~--- 
\textit{Inform. Appl.} 8(3):114--125.

%\columnbreak

\bibitem{19-min-1}
\Aue{Minin,~V.\,A., I.\,M.~Zatsman, V.\,A.~Havanskov, and S.\,K.~Shubnikov}. 2015. Indikatory 
tematicheskikh vza\-imo\-svya\-zey nauki i~informatsionno-komp'yuternykh tekh\-no\-lo\-giy 
v~nachale XXI~veka 
[Indicators for thematic linkages between science and information and computer technologies at the 
beginning of the XXI century]. \textit{Informatika i~ee Primeneniya}~--- \textit{Inform. Appl.} 
9(2):111--120.
\bibitem{20-min-1}
\Aue{Zatsman,~I.\,M., and G.\,F.~Verevkin}. 2006. Informatsionnyy monitoring sfery nauki v 
zadachakh programmno-tselevogo upravleniya [Information monitoring in the field of science and 
problems of program-oriented management]. \textit{Sistemy i~Sredstva Informatiki}~--- 
\textit{Systems and Means of Informatics} 16:164--189.
\bibitem{21-min-1}
\Aue{Zatsman,~I.\,M., and O.\,S.~Kozhunova}. 2007. Semanti\-che\-skiy slovar' sistemy 
informatsionnogo monitoringa v sfere nauki: Zadachi i funktsii [The semantic dictionary of system for 
information monitoring in science: Tasks and functions]. \textit{Sistemy i~Sredstva 
Informatiki}~--- \textit{Systems and Means of Informatics} 17:124--141.
\bibitem{26-min-1} %22
\Aue{Zatsman,~I., and O.~Kozhunova}. 2008. Evaluating for institutional academic activities: 
Classification scheme for R\&D indicators. \textit{10th Conference (International) on Science and 
Technology Indicators: Book of abstracts}. Vienna: ARC GmbH. 428--431.
\bibitem{25-min-1} %23
\Aue{Zatsman,~I., and O.~Kozhunova}. 2009. Evaluation system for the Russian Academy of 
Sciences: Objectives-resources-results approach and R\&D indicators. \textit{2009 Atlanta Conference 
on Science and Innovation Policy Proceedings}. Eds. S.\,E.~Cozzens and P.~Catalаn. Available at: 
{\sf http://smartech.gatech.edu/bitstream/1853/32300/1/\linebreak 104-674-1-PB.pdf} (accessed January~24, 
2016).
\bibitem{22-min-1} %24
\Aue{Arkhipova,~M.\,Yu., I.\,M.~Zatsman, and S.\,Yu.~Shul'ga}. 2010. Indikatory patentnoy aktivnosti 
v~sfere informatsionno-kommunikatsionnykh tekhnologiy i~metodika ikh vychisleniya [Indicators of 
patent activity in the sphere of information and communication technologies and technique of their 
calculation]. \textit{Ekonomika, statistika i~informatika. Vestnik UMO} [Economy, statistics and 
informatics. Herald of the UMO] 4:93--104.
\bibitem{24-min-1} %25
\Aue{Zatsman, I., and A.~Durnovo}. 2010. Incompleteness problem of indicators system of research 
programme. \textit{11th Conference (International) on Science and 
Technology Indicators: 
Book of abstracts}. Leiden: CWTS. 309--311.
\bibitem{23-min-1} %26
\Aue{Zatsman,~I.\,M., and A.\,A.~Durnovo}. 2011. Modelirovanie protsessov formirovaniya 
ekspertnykh znaniy dlya mo\-ni\-to\-rin\-ga programmno-tselevoy deyatel'nosti [Modeling of creation 
processes of expert knowledge for monitoring program-oriented activities]. \textit{Informatika i~ee 
Primeneniya}~--- \textit{Inform. Appl.} 5(4):84--98.

\pagebreak

\bibitem{27-min-1}
Rossiyskoe agentstvo po patentam i~tovarnym znakam (Rospatent) 
[The Russian Agency for Patents and Trademarks (Rospatent)]. 2000.
 Annual Report. Available at: {\sf 
http://www1.fips.ru/wps/wcm/connect/content\_ru/\linebreak ru/otchety/otchet\_2000\_r6} 
(accessed June~20, 2016).
\bibitem{28-min-1}
Federal'naya sluzhba po intellektual'noy sobstvennosti (Rospatent) [The Federal 
service for intellectual property (Rospatent)]. 2014. Available at: {\sf 
http:// www.rupto.ru/about/reports/2014\_1\#1} (accessed January~24, 2016).
  \end{thebibliography}

 }
 }

\end{multicols}

\vspace*{-3pt}

\hfill{\small\textit{Received April 19, 2016}}

\Contr

\noindent
\textbf{Minin Vladimir A.} (b.\ 1941)~--- Doctor of Science in physics and mathematics, consultant, 
Institute of Informatics Problems, Federal Research Center ``Computer Science and Control'' of the 
Russian Academy of Sciences, 44-2~Vavilov Str., Moscow 119333, Russian Federation; 
\mbox{aleksisss@ya.ru}

      \vspace*{3pt}
      
      \noindent
      \textbf{Zatsman Igor M.} (b.\ 1952)~--- Doctor of Science in technology, Head of 
Department, Institute of Informatics Problems, Federal Research Center ``Computer Science and 
Control'' of the Russian Academy of Sciences, 44-2~Vavilov Str., Moscow 119333, Russian 
Federation; \mbox{iz\_ipi@a170.ipi.ac.ru} 

      \vspace*{3pt}
      
      \noindent
      \textbf{Havanskov Valerij A.} (b.\ 1950)~--- scientist, Institute of Informatics Problems, 
Federal Research Center ``Computer Science and Control'' of the Russian Academy of Sciences, 44-
2~Vavilov Str., Moscow 119333, Russian Federation; \mbox{havanskov@a170.ipi.ac.ru} 
     
      \vspace*{3pt}
      
      \noindent
      \textbf{Shubnikov Sergej K.} (b.\ 1955)~--- senior scientist, Institute of Informatics Problems, 
Federal Research Center ``Computer Science and Control'' of the Russian Academy of Sciences, 44-
2~Vavilov Str., Moscow 119333, Russian Federation; \mbox{sergeysh50@yandex.ru} 
      
\label{end\stat}


\renewcommand{\bibname}{\protect\rm Литература}     