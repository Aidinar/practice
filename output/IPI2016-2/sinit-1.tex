\def\crn{c_{r,\nu}}
\def\prn{p_{r,\nu}}
\def\qrm{q_{r,\mu}}
\def\kkk{\kappa}
\def\hx{{\hat X}}
\def\pss{(\psi_1\nu_0\psi_1^{\mathrm{T}})}
\def\srn{S_{r,\nu}}





\def\stat{sinits-1}

\def\tit{ЭЛЛИПСОИДАЛЬНЫЕ СУБОПТИМАЛЬНЫЕ ФИЛЬТРЫ
ДЛЯ~НЕЛИНЕЙНЫХ СТОХАСТИЧЕСКИХ
СИСТЕМ НА~МНОГООБРАЗИЯХ$^*$}

\def\titkol{Эллипсоидальные субоптимальные фильтры
для~нелинейных стохастических
систем на~многообразиях}

\def\aut{И.\,Н.~Синицын$^1$, В.\,И.~Синицын$^2$,  Э.\,Р.~Корепанов$^3$}

\def\autkol{И.\,Н.~Синицын, В.\,И.~Синицын,  Э.\,Р.~Корепанов}

\titel{\tit}{\aut}{\autkol}{\titkol}

\index{Синицын И.\,Н.}
\index{Синицын В.\,И.}
\index{Корепанов Э.\,Р.}
\index{Sinitsyn I.\,N.}
\index{Sinitsyn V.\,I.}
\index{Korepanov E.\,R.}

{\renewcommand{\thefootnote}{\fnsymbol{footnote}} \footnotetext[1]
{Работа выполнена при поддержке РФФИ (проект 15-07-02244).}}


\renewcommand{\thefootnote}{\arabic{footnote}}
\footnotetext[1]{Институт проблем информатики Федерального исследовательского
центра <<Информатика и~управление>> Российской академии наук, sinitsin@dol.ru}
\footnotetext[2]{Институт проблем информатики Федерального исследовательского
центра <<Информатика и~управление>> Российской академии наук, vsinitsin@ipiran.ru}
\footnotetext[3]{Институт проблем информатики Федерального исследовательского
центра <<Информатика и~управление>> Российской академии наук, ekorepanov@ipiran.ru}


\Abst{Разработана теория аналитического синтеза эллипсоидальных  субоптимальных 
фильтров (ЭСОФ) для нелинейных дифференциальных стохастических систем (СтС) 
на многообразиях (МСтС). Рассмотрены случаи гауссовских и~негауссовских СтС. Алгоритмы 
положены в~основу модуля экспериментального программного обеспечения  StS-Filter 
(version 2016). Результаты допускают развитие на случай дискретных СтС. 
Теоретический и~практический интерес представляет теория  ЭСОФ на основе 
ненормированных распределений.}

\KW{апостериорное одномерное распределение;
винеровский шум; метод эллипсоидальной аппроксимации (МЭА);
метод эллипсоидальной линеаризации (МЭЛ);
пуассоновский шум; стохастическая система на многообразиях (МСтС);
субоптимальный фильтр (СОФ); уравнения точности МЭА и~МЭЛ;
уравнения чувствительности МЭА и~МЭЛ}

\DOI{10.14357/19922264160203} 

%\vspace*{-4pt}

\vskip 10pt plus 9pt minus 6pt

\thispagestyle{headings}

\begin{multicols}{2}

\label{st\stat}

\section{Введение}

В~[1, 2] метод ортогональных разложений (МОР)  был развит для аналитического 
моделирования одно- и~многомерных распределений в МСтС 
и~дано его применение для задач надежности и~без\-опас\-ности технических сис\-тем.
В~[3] представлена теория субоптимальных фильтров (СОФ) на базе методов нормальной 
аппроксимации (МНА) и~статистической линеаризации (МСЛ), а~также МОР для 
МСтС с~винеровскими шумами в~уравнениях наблюдения и~винеровскими и~пуассоновскими 
шумами в~уравнениях состояния.
В~основу СОФ были положены точные нелинейные уравнения для апостериорного одномерного 
распределения.

Рассмотрим развитие~[3] на случай, когда апостериорное одномерное распределение 
ошибки фильтрации допускает эллипсоидальную аппроксимацию (ЭА)~[4--7]. В~разд.~2 и~3 
приведены точ-\linebreak ные фильтрационные уравнения, а~также уравнения точности и~чувствительности 
на основе МОР. Элементы эллипсоидального анализа распределений даются в~разд.~4. 
Раздел~5 содержит уравнения ЭСОФ на основе методов ЭА
(МЭА) и~эллипсоидальной линеаризации (МЭЛ). Заключение содержит выводы 
и~некоторые обобщения.

\section{Точные фильтрационные уравнения}

\vspace*{-18pt}

На практике часто возникают задачи непрерывного
определения состояния системы по результатам непрерывных наблюдений.
Так как наблюдения всегда сопровождаются случайными ошибками, то
следует говорить не об определении состояния сис\-те\-мы, а~о~его
оценивании (фильтрации, экстраполяции, интерполяции и~т.\,д.) путем
статистической обработки результатов наблюдений. Будем рас\-смат\-ри\-вать
задачи фильтрации состояния сис\-тем, моделями которых могут служить
стохастические дифференциальные  уравнения с~винеровскими и~пуассоновскими шумами.

Часто стохастические дифференциальные уравнения модели изучаемой системы могут
иметь неизвестные параметры и,~как правило, всегда содержат
параметры, известные с~ограниченной точностью. Поэтому возникает
задача непрерывного оценивания неизвестных параметров системы
(точнее, ее модели) по результатам непрерывных наблюдений.
Предположим, что правые части уравнений зависят от конечного множества 
неизвестных параметров, которые
будем рассматривать как компоненты век-\linebreak\vspace*{-12pt}

\pagebreak

\noindent
тора параметров~$\theta$.
Одним из возможных подходов в~таких случаях является следующий
прием: неизвестный векторный параметр~$\theta$ считают стохастическим
процессом  (СтП) $\Theta \hm=\Theta_t$, который определяется
дифференциальным уравнением $\dot\Theta_t \hm=0$, и~включают
компоненты этого векторного процесса в~вектор состояния системы
(<<расширяют>> вектор состояния путем включения в~него неизвестных
параметров в~качестве дополнительных компонент).
Таким образом, задача непрерывного оценивания неизвестных
параметров модели системы сводится к~задаче непрерывного
оценивания состояния системы с~расширенным вектором состояния.
От неизвестных параметров могут зависеть и~уравнения наблюдения. 
Эти параметры следует включить 
в~вектор~$\theta$ и,~следовательно, в~расширенный вектор состояния.

Пусть векторный СтП $\lk X_t^{\mathrm{T}} Y_t^{\mathrm{T}} \rk^{\mathrm{T}}$
определяется системой векторных стохастических дифференциальных
уравнений Ито:
   \begin{multline}
    dX_t =\varphi \left(X_t,Y_t,\Theta, t\right) dt + \psi' \left(X_t,Y_t,\Theta, t\right) 
    dW_0 +{}\\ 
{}+\iii_{R_0^q} \psi''
    \left(X_t,Y_t,\Theta, t,v\right) P^0 (dt, dv)\,,\\ X\left(t_0\right) = X_0\,;
    \label{e2.1-s1}
    \end{multline}
    
    \vspace*{-12pt}
    
    \noindent
    \begin{multline}
dY_t =\varphi_1 \left(X_t,Y_t,\Theta, t\right) dt +
    \psi_1' \left(X_t,Y_t,\Theta, t\right) dW_0 + {}\\
{}+\int\limits_{R_0^q} \psi_1'' \left(X_t,Y_t,\Theta, t,v\right) P^0
    (dt,dv)\,,\\
     Y\left(t_0\right) = Y_0\,.
    \label{e2.2-s1}
    \end{multline}
Здесь $Y_t=Y(t)$~--- $n_y$-мер\-ный наблюдаемый
СтП, $Y_t \hm\in \Delta^y$ ($\Delta^y$~--- гладкое многообразие наблюдений); 
$X_t \hm=X(t)$~--- $n_x$-мер\-ный ненаблюдаемый
СтП (вектор состояния), $X_t \hm\in \Delta^x$ ($\Delta^x$~--- 
гладкое многообразие состояний); $W_0\hm =W_0(t)$~--- $n_w$-мер\-ный
винеровский СтП $(n_w\hm\ge n_y)$ интенсивности  $\nu_0 \hm=\nu_0 (\Theta, t)$; 
$P^0(\Delta,A)\hm=P(\Delta,A)-\mu_P (\Delta,A)$, $P(\Delta,A)$  представляет 
собой для любого множества~$A$ прос\-той пуассоновский СтП, а~$\mu_P (\Delta,A)$~--- 
его математическое ожидание, причем
    $$
    \mu_P (\Delta,A)=\mm P (\Delta,A)=\iii_\Delta \nu_P(\tau, A)\, d\tau\,;
    $$
$\nu_P (\Delta, A)$~--- интенсивность соответствующего пуассоновского
потока событий, $\Delta \hm=(t_1,t_2]$; интегрирование по~$v$
распространяется на все пространство~$R^q$ с~выколотым началом
координат; $\Theta$~--- вектор случайных параметров размерности~$n_\Theta$; 
$\varphi\hm=\varphi(X_t,Y_t,\Theta, t)$,
$\varphi_1\hm=\varphi_1(X_t,Y_t,\Theta, t)$,
 $\psi'\hm=\psi'(X_t,Y_t,\Theta, t)$ и~$\psi_1'\hm=\psi_1'(X_t,Y_t,\Theta, t)$~--- 
 известные функции, отображающие
$R^{n_x}\times R^{n_y}\times  R$ соответственно в~$R^{n_x}$,
$R^{n_y}$, $R^{n_xn_w}$ и~$R^{n_yn_w}$; $\psi''\hm=\psi''(X_t,Y_t,\Theta, t,v)$ 
и~$\psi_1''(X_t,Y_t,\Theta, t,v)$~--- известные
функции, отображающие $R^{n_x}\times R^{n_y}\times R^q$ в~$R^{n_x}$ и~$R^{n_y}$. 
Требуется найти оценку~$\hat X_t$ СтП~$X_t$ 
в~каждый момент времени~$t$ по результатам наблюдения
СтП $Y(\tau)$ до момента~$t$, $Y_{t_0}^t \hm=  \{ Y(\tau) \,: t_0 \hm\le \tau\hm< t\}$.

Следуя~\cite{8-s1}, предположим, что
\begin{itemize}
\item уравнение состояния имеет вид~(\ref{e2.1-s1});

\item уравнение наблюдения~(\ref{e2.2-s1}), во-пер\-вых, не содержит
пуассоновского шума $(\psi_1'' \hm\equiv 0)$, 
а~во-вто\-рых, коэффициент при винеровском шуме~$\psi_1'$  
в~уравнениях наблюдения не зависит от состояния $(\psi_1' (X_t, Y_t,\Theta, t)\hm=\psi_1'
(Y_t,\Theta, t))$.
\end{itemize}

В этом случае уравнения задачи нелинейной фильтрации имеют следующий вид:
\begin{multline}
    dX_t =\varphi \left(X_t, Y_t,\Theta, t\right)\,dt+\psi'
    \left(X_t, Y_t,\Theta, t\right)\,dW_0+{}\\
\!\!{}+\!\int\limits_{R_0^q}\! \psi''\left(X_t, Y_t,\Theta, t,v\right) P^0 (dt, dv)\,,\enskip 
X\left(t_0\right) = X_0;\!\!\label{e2.3-s1}
\end{multline}


\vspace*{-12pt}

\noindent
\begin{multline}
dY_t =\varphi_1 \left(X_t, Y_t,\Theta, t\right)\,dt +
\psi_1 \left(Y_t,\Theta, t\right)\, dW_0\,,\\ Y\left(t_0\right) = Y_0\,.
\label{e2.4-s1}
\end{multline}


Будем считать, что выполнены условия существования и~единственности СтП  
$\lk X_t^{\mathrm{T}}\ Y_t^{\mathrm{T}}\rk^{\mathrm{T}}$, определяемого~(\ref{e2.3-s1}) 
и~(\ref{e2.4-s1}) при соответствующих начальных условиях.

В дальнейшем для стохастического уравнения
\begin{equation}
dZ=a \,dt + b\, dW_0 + \iii_{R_0^q} c P^0 (dt, dv)\label{e2.5-s1}
\end{equation}
потребуется обобщенная формула Ито~\cite{5-s1} 
для дифференциала нелинейной функции $U\hm=U(Z,t)$:
 \begin{multline}
 dU =\lf U_t + U_z^{\mathrm{T}} a + \fr{1}{2}\,\mathrm{tr}\, \lk U_{zz} b\nu b^{\mathrm{T}}\rk\rf dt +{}\\
 {}+ \iii_{R_0^q} \lk U(Z+c,t)^{\mathrm{T}} - U(Z,t)^{\mathrm{T}} - U_z^{\mathrm{T}} c\rk 
 \mu_P (dt, dv)+{}\\
   {}+U_Z^{\mathrm{T}} b \,dW_0 +{}\\
   {}+ \iii_{R_0^q} \lk U(Z+c,t) - U(Z,t) \rk P^0 (dt, dv)\,.
   \label{e2.6-s1}
   \end{multline}
Здесь $a$, $b$ и~$c$~--- известные функции~$Z$ и~$t$.

Как известно~\cite{4-s1, 5-s1}, для любых СтП~$X_t$ и~$Y_t$ оптимальная 
оценка~$\hat X^t$, минимизирующая средний квадрат ошибки в~каждый момент времени~$t$, 
представляет собой апостериорное математическое ожидание СтП~$X_t$: 
$\hat X_t \hm= \mm \lk X_t \mid Y_{t_0}^t\rk$. Чтобы найти это условное 
математическое ожидание, необходимо знать $p_t \hm= p_t (x)$~--- 
апостериорное одномерное распределение СтП~$X_t$.

В основе уравнений оптимальной (в~смыс\-ле минимума средней квадратической ошибки) 
фильт\-ра\-ции для уравнений~(\ref{e2.3-s1}) и~(\ref{e2.4-s1}) 
в~силу~(\ref{e2.6-s1}) лежит следу\-ющая формула для стохастического дифференциала 
апостериорного математического ожидания скалярной функции  $f\hm=f(X,t)$ вектора 
состояния~\cite{8-s1}:
\begin{multline}
d \hat f = d \mm_{\Delta^x}^{p_t} \left[ f_t 
(X,t) + f_x (X,t)^{\mathrm{T}} \vrp (X,Y,t) +{}\right.\\[1pt]
{}+ \fr{1}{2}\,\mathrm{tr}\, 
\left\{ f_{xx} (X,t) (\psi' \nu_0 {\psi'}^{\mathrm{T}}) (X,Y,t)\right\}+{}\\[1pt]
{}+ \iii_{R_0^q}  \left\{ 
\vphantom{f_x (X,t)^{\mathrm{T}}}
f \left(X+ \psi'' , t\right) - f(X,t) -{}\right.\\[1pt]
\left.\left.{}-
f_x (X,t)^{\mathrm{T}} \psi''(X,Y,t)\right\} \nu_P (t, dv)\mid Y_{t_0}^t \right] dt+{}\\[1pt]
  {}+\mm_{\Delta^x}^{p_t} \left\{ f(X,t) \left[ 
  \vrp_1 (X,Y,t)^{\mathrm{T}} -
  \hat \vrp_1^{\mathrm{T}}\right] +{}\right.\\[1pt]
\left.  {}+ f_x (X,t)^{\mathrm{T}} 
  \left(\psi\nu_0\psi_1^{\mathrm{T}}\right) (X,Y,t) \mid Y_{t_0}^t\right\} \times{}\\[1pt]
    {}\times \psi_1 \nu_0 \psi_1^{\mathrm{T}})^{-1} (Y,t) \left(dY-\hat\vrp_1 \,dt\right).
    \label{e2.7-s1}
    \end{multline}
Здесь для краткости аргумент~$\Theta$ опущен; $X\hm=X_t$, $Y\hm=Y_t$, $\nu\hm=\nu_0$ 
и~$\nu_P$~--- интенсивности~$W_0$ и~$P^0$;
$\hat\vrp_1$~--- апостериорное математическое ожидание~$\vrp_1$ при заданной условной плотности  
$p_t\hm=p_t (x, \Theta)$:
\begin{equation*}
\hat\vrp_1 = \mm_{\Delta^x}^{p_t} \lk\vrp_1(X,Y,t)\rk\,.
%\label{e2.8-s1}
\end{equation*}

Полагая в~(\ref{e2.5-s1}) $f(X,t) \hm\equiv g_t (\la,\Theta) \hm=
\mm_{\Delta^x}^{p_t}\lk\exp (i\la^{\mathrm{T}} X)\rk$, получим
точное нелинейное фильтрационное уравнение для  характеристической 
функции $g_t (\la,\Theta)$:
\begin{multline}
\!dg_t (\la,\Theta) = {\mm}_{\Delta^x}^{p_t} \left[ 
\left\{
\vphantom{\fr{1}{2}}
 i\la^{\mathrm{T}} \varphi (X_t,Y_t,\Theta, t) - {}\right.\right.
{}-\fr{1}{2}\, \la^{\mathrm{T}}\times{}\\[1pt]
\left.{}\times \left(\psi\nu_0\psi^{\mathrm{T}}\right) 
\left(X_t,Y_t,\Theta, t\right) \la+\gamma \left(\la, X_t, Y_t, \Theta, t\right) 
\vphantom{\fr{1}{2}}\right\}\times{}\\[1pt]
\!\left.{}\times e^{i\la^{\mathrm{T}} X_t} \mid Y_{t_0}^t 
\vphantom{\fr{1}{2}}
\right] dt+ 
{\mm}_{\Delta^x}^{p_t} \left[ 
\left\{ 
\varphi_1 \left(X_t,Y_t,\Theta, t\right)^{\mathrm{T}} -
\hat\varphi_1^{\mathrm{T}} +{}\right.\right.\hspace*{-0.70326pt}\\[1pt]
{}+
    i\la^{\mathrm{T}} \left(\psi\nu_0\psi_1^{\mathrm{T}}\right) 
    \left(X_t,Y_t, \Theta,t\right) \left.  
    e^{i\la^{\mathrm{T}} X_t} \mid Y_{t_0}^t \right\} \times{}\\[1pt]
    \left.{}\times
  \left(\psi_1\nu_0\psi_1^{\mathrm{T}}\right)^{-1} \left(Y_t,\Theta,t\right) 
  \left(dY_t-\hat\varphi_1 \,dt\right)\right], 
  \label{e2.9-s1}
  \end{multline}
где
\begin{multline}
\gamma=\gamma \left(\la, X_t, Y_t, \Theta, t\right)=
\int\limits_{R_0^q}\left[ e^{i\la^{\mathrm{T}} 
\psi''\left(X_t,Y_t,\Theta, t,v\right)} - 
 {}\right.\\
\left.{}-1- i\la^{\mathrm{T}} \psi''\left(X_t,Y_t,\Theta, t,v\right)
\vphantom{e^{i\la^{\mathrm{T}}}}
\right] 
\nu_P (\Theta, t,v) \,dv\,.\label{e2.10-s1}
\end{multline}

Функции $g_t (\la,\Theta)$ и~$p_t(x,\Theta)$ связаны между собой преобразованием 
Фурье~\cite{8-s1}.

Отсюда для гауссовской МСтС~(\ref{e2.3-s1}), (\ref{e2.4-s1}) $(\psi''\hm\equiv 0)$ 
уравнение~(\ref{e2.7-s1}) при  $\gamma\hm=0$ упрощается и~приобретает вид:

\noindent
\begin{multline}
dg_t (\la,\Theta) = {\mm}_{\Delta^x}^{p_t} \left[
\left\{ \vphantom{\fr{1}{2}}
i\la^{\mathrm{T}} \varphi (X_t,Y_t,\Theta, t) -{}\right.\right.\\
\left.\left.{}- \fr{1}{2}\, \la^{\mathrm{T}} 
\left(\psi\nu_0\psi^{\mathrm{T}}\right) 
\left(X_t,Y_t,\Theta, t\right) \la\right\} 
e^{i\la^{\mathrm{T}} X_t} \mid Y_{t_0}^t\right] dt+{}\\
{}+ {\mm}_{\Delta^x}^{p_t} \biggl[ 
\biggl\{ 
\varphi_1 \left(X_t,Y_t,\Theta, t\right)^{\mathrm{T}} -
\hat\varphi_1^{\mathrm{T}} +{}\\
{}+
    i\la^{\mathrm{T}} \left(\psi\nu_0\psi_1^{\mathrm{T}}\right) 
    \left(X_t,Y_t, \Theta,t\right) e^{i\la^{\mathrm{T}} X_t} \mid Y_{t_0}^t
   \biggr\} \times{}\\
   {}\times\left(\psi_1\nu_0\psi_1^{\mathrm{T}}\right)^{-1} \left(Y_t,\Theta,t\right) 
   \left(dY_t-\hat\varphi_1\, dt\right)
    \biggr]. \label{e2.11-s1}
    \end{multline}

Если функция $\psi''$ в~(\ref{e2.3-s1}) допускает представление~\cite{8-s1, 9-s1}
\begin{equation}
\psi'' = \psi' \omega (\Theta, v)\,,\label{e2.12-s1}
\end{equation}
где  $P^0 (\Delta, A) = P^0 ((0,t], dv)$, то уравнения~(\ref{e2.3-s1}), 
(\ref{e2.4-s1}) примут следующий вид:

\noindent
\begin{multline}
\dot X_t =\vrp \left(X_t, Y_t, \Theta, t\right)+\psi' 
\left(X_t, Y_t, \Theta, t\right)V(\Theta, t)\,,\\
 X\left(t_0\right)=X_0\,;
\label{e2.13-s1}
\end{multline}

\vspace*{-12pt}

\noindent
\begin{multline}
\dot Y_t = \vrp\left(X_t, Y_t, \Theta, t\right)+\psi_1 
\left(Y_t, \Theta, t\right) V_0 (\Theta, t)\,,\\
 Y\left(t_0\right) = Y_0\,.
\label{e2.14-s1}
\end{multline}
Здесь
$V_0 (\Theta, t)\hm =\dot W_0 (\Theta, t)$; $V (\Theta, t)\hm=\dot{\bar W}  (\Theta, t)$,
\begin{equation*}
\bar W (\Theta, t) = W_0  (\Theta, t)+ \iii_{R_0^q} \omega (\Theta, v) 
P^0 ((0,t],dv)\,,
%\label{e2.15-s1}
\end{equation*}
где $\nu_P  (\Theta, t,v) dv\hm= \lk \prt \mu  (\Theta, t,v)/\prt t\rk dv$~--- 
интенсивность пуассоновского потока скачков, равных $\omega (\Theta, t)$.
%
При этом логарифмические производные от одномерных характеристических функций 
определяются известными формулами:

\noindent
\begin{align}
\chi^{W_0} (\rho; t) &= -\fr{1}{2}\, \rho^{\mathrm{T}} \nu_0  (\Theta, t) \rho\,;\notag
\\
\chi^{\bar W} (\rho;t) &= - \fr{1}{2}\,\rho^{\mathrm{T}}  (\Theta, t) 
\rho^{\mathrm{T}}+ {}\notag\\
&\hspace*{-20mm}{}+\!\iii_{R_0^q} \!\lk e^{i\rho^{\mathrm{T}} \omega (\Theta, v)} 
-1- i\rho^{\mathrm{T}} \omega (\Theta, v)\rk \!\nu_P  (\Theta, t,v)\,dv.
\label{e2.16-s1}\!\!\!\!\!
\end{align}
В таком случае уравнение для апостериорной одномерной характеристической 
функции имеет вид~(\ref{e2.9-s1}), где функция~(\ref{e2.10-s1}) допускает 
следующую запись:

\noindent
   \begin{multline*}
    \gamma = \iii_{R_0^q} \left[ e^{ i\la^{\mathrm{T}} \psi' 
    \left(X_t, Y_t, \Theta, t\right)
    \omega(\Theta, v)} - 1- {}\right.\\
\left.    {}-i\la^{\mathrm{T}} 
    \psi'\left(X_t, Y_t, \Theta, t\right)
    \omega (\Theta, v)
    \vphantom{e^{ i\la^{\mathrm{T}}}}
    \right]
    \nu_P  (\Theta, t,v)\,dv\,.
   \end{multline*}
   
   \pagebreak

Таким образом, можно сформулировать следующие утверждения разд.~2.

\smallskip

\noindent
\textbf{Теорема~2.1.}\ \textit{Пусть для МСтС}~(\ref{e2.3-s1}), 
(\ref{e2.4-s1}) \textit{выполнены условия существования и~единственности решения, 
а~мат\-ри\-ца  $\si_1 \hm= \psi_1 \nu_0 \psi_1^{\mathrm{T}}$ не вырождена. 
Тогда при условии ограниченности соответствующих математических ожиданий точное 
фильтрационное уравнение для условной одномерной нормированной характеристической 
функций имеет вид}~(\ref{e2.9-s1}).

\smallskip

\noindent
\textbf{Теорема 2.2.}\ \textit{В~условиях теоремы~$2.1$ при отсутствии пуассоновских 
шумов точное фильтрационное уравнение для условной одномерной нормированной 
характеристической функции имеет вид}~(\ref{e2.11-s1}).


\smallskip

\noindent
\textbf{Теорема~2.3}~\cite{8-s1, 9-s1}. \textit{Пусть для МСтС}~(\ref{e2.13-s1}), 
(\ref{e2.14-s1}) \textit{выполнены условия существования и~единственности решения, 
имеет место представление}~(\ref{e2.12-s1}), 
\textit{а~мат\-ри\-ца  $\si_1 \hm= \psi_1 \nu_0 \psi_1^{\mathrm{T}}$ не вырождена. 
Тогда при условии ограниченности соответствующих математических ожиданий точное 
фильтрационное уравнение  имеет вид}~(\ref{e2.9-s1}) 
\textit{при условии}~(\ref{e2.10-s1}).

\smallskip

Как известно~\cite{8-s1}, точное решение фильтрационных уравнений  возможно
только в~случаях, когда уравнения гауссовской дифференциальной МСтС
линейны или линейны лишь относительно вектора состояния~$X_t$ при
независимой от состояния функции~$\psi$. Эти уравнения
 дают точное решение задачи оптимальной нелинейной фильтрации.  Однако это решение
  не может быть реализовано практически. Для
 нахождения оптимальной оценки вектора состояния необходимо решить
 фильтрационное уравнение  для апостериорной характеристической функции
 (или  фильтрационное уравнение  для апостериорной плотности   вектора
 состояния~$X_t$) после получения результатов наблюдений, затем вы\-чис\-лить оптимальную оценку вектора~$X_t$.
Но методов точного решения этих
 уравнений  в~общем случае пока еще не существует.

 В задачах реального времени численное решение фильтрационных уравнений 
 (или он\-лайн-оце\-ни\-ва\-ния) тоже
 невозможно, так как для этого требуется много времени, а~решать их
 необходимо каждый раз после получения результатов наблюдений.
 Кроме того, практическое применение точной теории оптимальной нелинейной фильтрации
 имеет смысл только в~тех случаях, когда оценки можно вычислять 
 в~реальном масштабе времени по мере получения результатов
 наблюдений. Точная теория дает оптимальные
 оценки в~каждый момент~$t$ по результатам наблюдений, полученным
 к~этому моменту, без использования последующих результатов
 наблюдений. Если эти оценки не могут быть вычислены в~тот же
 момент~$t$ или хотя бы с~фиксированным приемлемым запаздыванием
 и~их вычисление приходится откладывать на будущее, то нет
 никакого смысла отказываться от использования наблюдений,
 получаемых после момента~$t$, для оценивания состояния системы 
 в~момент~$t$. Поэтому для статистической обработки результатов
 после окончания наблюдений, т.\,е.\ для оф\-лайн-оце\-ни\-ва\-ния,
 целесообразно применять известные из математической статистики методы
 постобработки информации~\cite{8-s1}.

 Необходимость обработки результатов наблюдений в~реальном
 масштабе времени непосредст\-вен\-но в~процессе эксперимента
 привела  к~появлению ряда приближенных методов оптимальной\linebreak нелиней\-ной  фильтрации,
 на\-зы\-ва\-емых обычно методами  субоптимальной фильтрации~\cite{8-s1}. Одни
 приближенные методы основаны на  приближенном решении фильтрационных
 уравнений, а~другие~---  на превращении формул
 для стохастических дифференциалов оптимальной
 оценки~$\hat X_t$ и~апостериорной ковариационной  матрицы ошибки~$R_t$ 
 в~стохастические дифференциальные уравнения  
 для~$\hat X_t$ и~$R_t$ путем разложения функций~$\varphi$, $\varphi_1$, 
 $\psi_1$ или $\varphi$, $\varphi_1$, $\psi'$, $\psi''$, $\psi$, $\psi_1$ 
 в~степенные ряды и~отбрасывания остаточных членов.

 Для приближенного решения уравнения  для апостериорной
одномерной характеристической функции  $g_1(\la, \Theta)$ вектора~$X_t$ можно
использовать\linebreak
 методы аналитического моделирования, основанные на 
параметризации одномерных
 распределений СтП, определяемого стохастическим
 дифференциальным уравнением~\cite{8-s1}.  Эти методы
 позволя\-ют изучить
 стохастические дифференциальные уравнения для параметров
 апостериорного распределения. Простейшим таким методом является
 МНА апостериорного распределения.
Исключительно важное практическое значение имеют квазилинейные
фильтры, получаемые с~помощью методов эквивалентной линеаризации~\cite{8-s1}.

\section{Субоптимальные фильтры на~основе метода ортогональных разложений}

При аппроксимации апостериорной одномерной плотности отрезком ее ортогонального 
разложения~\cite{1-s1, 2-s1}
\begin{multline}
p_t (x, \Theta)\approx p^* (x; \Theta, \vartheta) = {}\\
{}=
w (x; \Theta) \lk 1+ \sss_{l=3}^N \sss_{|\nu | =l} c_\kappa p_\kappa (x)\rk
\label{e3.1-s1}
\end{multline}
естественно принять за параметры,  образующие вектор~$\vartheta$, 
апостериорные математическое
 ожидание~$\hat X_t$, ковариационную матрицу~$R_t$ вектора~$X_t$, а~также
 коэффициенты ортогонального разложения (КОР) $c_\kappa$ $(\lv \nu\rv \hm= 3\tr N)$.
 Здесь КОР определяется формулой:
\begin{equation}
c_\kappa = \lk q_\kappa \left(\fr{\partial }{i\partial \la}\right) 
g_t (\la,\Theta)\rk_{\la=0}\,.\label{e3.2-s1}
\end{equation}
Заметим, что полином~$q_\kappa$ зависит от~$\hat X_t$ и~$R_t$.

На основе~(\ref{e2.7-s1}) и~(\ref{e2.9-s1}) для  гауссовской 
МСтС~(\ref{e2.3-s1}), (\ref{e2.4-s1}) при  $\psi''\hm=0$ 
получим, что ортогональный СОФ (ОСОФ) определяется следующими уравнениями:
\begin{align}
d\hat X_t&=f\,dt + h \left(dY_t- f^{(1)}\, dt\right)\,;\label{e3.3-s1}\\
dR_t&= \left(f^{(2)} -h\psi_1\nu_0\psi_1^{\mathrm{T}} h^{\mathrm{T}}\right) dt 
+{}\notag\\
&\hspace*{20mm}{}+\sss_{r=1}^{n_y} \rho_r \left(dY_r -f_r^{(1)}\, dt\right)\,.\label{e3.4-s1}
\end{align}
Здесь введены обозначения:
    \begin{equation}
    \left.
    \begin{array}{l}
    f= f\left(Y_t,\vartheta,\Theta, t\right)=\mm_{\Delta^x}^{p^*} 
    \left[\varphi\left(Y_t,X,\Theta,t\right)\right] \,;\\[6pt]
f^{(1)}= \left\{ f_r^{(1)}\right\}= f^{(1)}\left(Y_t,\vartheta,\Theta, t\right)={}\\[6pt]
\hspace*{20mm}{}=
    \mm_{\Delta^x}^{p^*}  \lk\varphi_1\left(Y_t,X,\Theta,t\right)\rk\,;\\[6pt]
f^{(2)}= f^{(2)}\left(Y_t,\vartheta,\Theta,t\right)={}\\
\hspace*{5mm}{}=    \mm_{\Delta^x}^{p^*} \left[ \left(X-\hat X_t\right)\varphi
    \left(Y_t,X,\Theta,t\right)^{\mathrm{T}}+{}\right.\\[6pt]
\hspace*{10mm}{}+\varphi\left(Y_t,X,\Theta,t\right) \left(X^{\mathrm{T}}-
\hat X_t^{\mathrm{T}}\right) 
+{}\\[6pt]
\left.\hspace*{20mm}{}+ \left(\psi\nu_0\psi^{\mathrm{T}}\right) 
\left(Y_t,X,\Theta,t\right)
\vphantom{\left(Y_t,X,\Theta,t\right)^{\mathrm{T}}}
\right]\,;\\[6pt]
   h= h\left(Y_t,\vartheta,\Theta,t\right)={}\\[6pt]
\hspace*{15mm}   {}=\biggl\{
    \mm_{\Delta^x}^{p^*} \left[ X\varphi_1
    \left(Y_t,X,\Theta,t\right)^{\mathrm{T}}+{}\right.\\[6pt]
\left.\hspace*{15mm}{}+ (\psi\nu_0\psi_1^{\mathrm{T}}) \left(Y_t,X,\Theta,t\right)
\vphantom{    \left(Y_t,X,\Theta,t\right)^{\mathrm{T}}}
\right]-{}\\[6pt]
\hspace*{8mm}{}-
    \hat X_t f^{(1)T}\biggr\} \pss^{-1} \left(Y_t,\Theta,t\right)\,;\\[6pt]
\rho_r= \rho_r\left(Y_t,\vartheta,\Theta,,t\right)={}\\[6pt]
\hspace*{1mm}{}=
   \mm_{\Delta^x}^{p^*}  
   \left[ \left(X-\hat X_t\right) 
   \left(X^{\mathrm{T}}-\hat X_t^{\mathrm{T}}\right)\times{}\right.\\
\hspace*{-3.2mm}{}\times a_r\left(Y_t,X,\Theta,t\right)+ \left(
\!X-\hat X_t\!\right) b_r\left(Y_t,X,\Theta,t\right)^{\mathrm{T}}\!
+{}\\
\left.\hspace*{-4.7mm}{}+ b_r\left(Y_t,X,\Theta,t\right)\left(
\!X^{\mathrm{T}}-\hat X_t^{\mathrm{T}}\!\right)
\right]\ \left(r=1\tr n_y\right).\!\!
\end{array}\!\!
\right\}\!\!\!
    \label{e3.5-s1}
    \end{equation}

Далее перепишем~(\ref{e3.3-s1}), (\ref{e3.4-s1}) покоординатно:
\begin{align}
d\hat X_s &= f_s \,dt + h_s \left(dY_t- f^{(1)}\, dt\right) = {}\notag\\
&
A^{\hat X_s}\, dt + B^{\hat X_s}\, dY_t\enskip (s=1\tr
    n_x)\,;\label{e3.6-s1}\\
 dR_{sq} &=\left(f_{sq}^{(2)} - h_s\psi_1\nu_0\psi_1^{\mathrm{T}}
    h_q^{\mathrm{T}} \right) dt +{}\notag\\
    &\hspace*{-10mm}{}+\eta_{sq} 
    \left(dY_t - f^{(1)}\, dt\right)=A^{R_{sq}}\, dt + B^{R_{sq}}\, dY_t\,,
    \label{e3.7-s1}
    \end{align}
где $\hat X_s (t_0) = X_{s0}$; $R_{sq} (t_0)\hm = R_{sq0}$;  
$s,q\hm=1\tr n_x$; $\eta_{sq}$~--- мат\-ри\-ца-стро\-ка, 
элементами которой служат
соответствующие элементы матрицы  $\rho_1\tr \rho_{n_1}$:
\begin{equation*}
\eta_{sq} =\eta_{e_s+e_q} = \lk \rho_{1sq}\cdots \rho_{msq}\rk
    \enskip (s,q,=1\tr n_x)\,.
    %\label{e3.8-s1}
    \end{equation*}
Здесь и~далее для краткости индекс~$t$ сохраним только у~$Y_t$. 
По формуле дифференцирования Ито для винеровского СтП, учитывая~(\ref{e3.6-s1}) 
и~(\ref{e3.7-s1}), находим в~силу~(\ref{e3.2-s1}) стохастический дифференциал:
\begin{multline*}
dc_\kappa =\lk d\lf  q_\kappa \left(\fr{\partial}{i\partial \la}\right) g_t
    (\la,\Theta)\rf\rk_{\la=0}={}\\
{}=\sss_{s=1}^{n_x} \lk\partial q_\kappa
    \left(\fr{\partial }{i\partial \la}\right) \partial \hat X_s g_t
    (\la,\Theta)\rk_{\la=0}d \hat X_s+{}\\
{}+\sss_{s,u=1}^{n_x} \lk\partial
    q_\kappa \left(\fr{\partial}{i\partial \la}\right) \partial R_{su} g_t
    (\la,\Theta)\rk_{\la=0}dR_{su} +{}\\
    {}+ \lk q_\kappa \left(
    \fr{\partial }{i\partial \la}\right) d g_t (\la,\Theta)\rk_{\la=0}+{}\\
{}+ \biggl\{ \fr{1}{2}\!\! \sss_{s,u=1}^{n_x}\! \lk \fr{\partial^2 q_\kappa
    \left({\partial}/(i\partial \la)\right)
    g_t (\la,\Theta)}{\partial \hat X_s
    \partial \hat X_u }\rk_{\la=0}\!\! h_s \psi_1\nu_0\psi_1^{\mathrm{T}} h_u^{\mathrm{T}} +{}\\
    {}+
    \fr{1}{2}\! \sss_{s,u,k,l=1}^{n_x} \left[ \fr{\partial^2 q_\kappa (\partial
    /(i\partial \la))
    g_t (\la,\Theta)}{\partial R_{su} \partial R_{kl}}
    \right]_{\la=0} \times{}\\
    {}\times \eta_{su} \psi_1\nu_0\psi_1^{\mathrm{T}} \eta_{kl}^{\mathrm{T}}+{}\\
{}+\!\sss_{s,k,l=1}^{n_x}\! \!\lk \fr{\partial^2 q_\kappa (\partial /(i\partial
    \la)) g_t (\la,\Theta)}
    {\partial \hat X_s \partial R_{kl}}\rk_{\la=0}
    \!\!\!h_s \psi_1\nu_0\psi_1^{\mathrm{T}}
     \eta_{kl}^{\mathrm{T}}\biggr\}\, dt.\hspace*{-6.3374pt}
%    \label{e3.9-s1}
    \end{multline*}
Подставив сюда выражения~(\ref{e3.6-s1}), (\ref{e3.7-s1}) и~(\ref{e2.7-s1})
дифференциалов $d\hat X_s$, $dR_{sq}$  и~ $dg_t (\la,\Theta)$ и~вспомнив,
что для любого полинома  $P(x)$ $P\lk (\partial /(i\partial
\la)) g_t(\la)\rk_{\la=0}\hm=P(\alp)$, получаем стохастические
дифференциальные уравнения:
\begin{multline*}
dc_\kappa =\biggl\{ F_\kappa +\sss_{s=1}^{n_x}\fr{\partial q_\kappa
    (\alp)}{\partial \hat X_s} \,f_s +{}\\
    {}+\sss_{s,u=1}^{n_x} \fr{\partial
    q_\kappa (\alp)}{\partial R_{su}}\left(
     f_{su}^{(2)} - h_s \psi_1\nu_0\psi_1^{\mathrm{T}} h_u^{\mathrm{T}}\right)+{}\\
{}+\fr{1}{2}\sss_{s,u=1}^{n_x} \fr{\partial^2 q_\kappa (\alp)}
{\partial \hat X_s \partial \hat X_u}  h_s \psi_1\nu_0\psi_1^{\mathrm{T}} 
h_u^{\mathrm{T}} +{}\\
{}+ \fr{1}{2} \sss_{s,u,k,l=1}^{n_x} 
\fr{\partial^2 q_\kappa (\alp)}{\partial R_{su}\partial R_{kl}}\,
    \eta_{su} \psi_1\nu_0\psi_1^{\mathrm{T}} \eta_{kl}^{\mathrm{T}}+{}\\
    {}+\sss_{s,k,l=1}^{n_x} \fr{\partial^2 q_\kappa (\alp)}
{\partial \hat X_s \partial R_{kl}}\, h_s \psi_1\nu_0\psi_1^{\mathrm{T}} \eta_{kl}^{\mathrm{T}}\biggr\}\, dt +{}\\
{}+\biggl\{ H_\kappa +
    \sss_{s=1}^{n_x} \fr{\partial q_\kappa (\alp)}{\partial \hat X_s}\,h_s +
\sss_{s,u=1}^{n_x} \fr{\partial q_\kappa (\alp)}{\partial R_{su}} \,\eta_{su}
\biggr\}\times{}
\end{multline*}

\noindent
\begin{multline}
{}\times \left(dY_t - f^{(1)}\, dt\right)= 
A^{c_\kappa} dt + B^{c_\kappa}\, dY_t\,,\\ 
c_\kappa \left(t_0\right) = c_{\kappa0} \enskip (\lv\kappa\rv= 3\tr N)\,.
\label{e3.10-s1}
\end{multline}
Здесь в~дополнение к~прежним обозначениям принято:
\begin{equation}
\left.
\begin{array}{rl}
F_\kappa &=\displaystyle F_\kappa \left(Y_t,\Theta,\vartheta,t\right) 
={}\\[6pt]
&\displaystyle{}=\sss_{s=1}^{n_x}
   \mm_{\Delta^x}^{p^*} \lk \varphi_s\left (Y_t,X,\Theta,t\right)
   \fr{\partial q_\kappa (X)}{\partial X_s}\rk+{}\\[6pt]
&\displaystyle\hspace*{-5mm}{}+\fr{1}{2} \sss_{s,u=1}^{n_x}
     \mm_{\Delta^x}^{p^*}\lk \si_{su} \left(Y_t,X,\Theta,t\right)\fr{\partial^2 q_\kappa
    (X)}{\partial X_s\partial X_u}\rk\,;\\[6pt]
H_\kappa &= \displaystyle H_\kappa \left(Y_t,\vartheta,\Theta,t\right) ={}\\[6pt]
&\displaystyle{}=
\biggl\{  \mm_{\Delta^x}^{p^*}
   \lk \varphi_1 \left(Y_t,X,\Theta,t\right)^{\mathrm{T}} q_\kappa (X)\rk+{}\\[6pt]
&\hspace*{-9mm}{}\displaystyle+ \sss_{s=1}^{n_x}
     \mm_{\Delta^x}^{p^*}\!\lk \left(\psi\nu_0\psi_1^{\mathrm{T}}\right)_s 
     \left(Y_t,X,\Theta,t\right)\fr{\partial
    q_\kappa (X)}{\partial X_s} \rk- {}\\[6pt]
    &{}-c_\kappa
    f^{(1)T} \biggr\} \pss^{-1} (Y_t,\Theta,t)\,,
    \end{array}\!
    \right\}\!
    \label{e3.11-s1}
    \end{equation}
где через $(\psi\nu_0\psi_1^{\mathrm{T}})_s$ обозначена  $s$-я строка матрицы
$\psi\nu_0\psi_1^{\mathrm{T}}$; $\si\hm=\psi\nu_0\psi_1^{\mathrm{T}} 
\hm=\lf \si_{su}\rf$.

Функции  $f_s$,
$f^{(1)}$, $f^{(2)}_{su}$, $h_s$, $\eta_{su}$, $F_\kappa$ 
и~$H_\kappa$ в~уравнениях~(\ref{e3.6-s1}), (\ref{e3.7-s1}) 
и~(\ref{e3.10-s1}) представляют
собой линейные комбинации величин  $c_\nu$ $(\lv\nu\rv \hm= 3\tr N)$
с~коэффициентами, зависящими от~$\hat X_t$ и~$R_t$. Величины
$\partial q_\kappa (\alp)/\partial \hat X_s$, $\partial q_\kappa
(\alp)/\partial R_{su}$, $\partial^2 q_\kappa (\alp)/(\partial \hat
X_s \partial \hat X_u)$, $\partial^2 q_\kappa (\alp)/(\partial
R_{su} \partial R_{kl})$ и~$\partial^2 q_\kappa (\alp)/(\partial \hat
X_s \partial R_{kl})$ после замены моментов их выражениями
через~$c_\nu$ тоже будут линейными комбинациями величин~$c_\nu$ 
с~коэффициентами, зависящими от~$\hat X_t$ и~$R_t$.

Таким образом, имеем следующие утверждения.

\smallskip

\noindent
\textbf{Теорема~3.1.}\
\textit{Пусть МСтС}~(\ref{e2.3-s1}), (\ref{e2.4-s1})~--- 
\textit{гауссовская $(\psi'' \hm=0)$, выполнены условия существования 
и~единственности решения, а~мат\-ри\-ца $\si_1 \hm= 
\psi_1 \nu_0 \psi_1^{\mathrm{T}}$ не вырождена. Тогда в~основе алгоритма 
ОСОФ по МОР лежат уравнения}~(\ref{e3.1-s1}), 
(\ref{e3.6-s1})--(\ref{e3.10-s1}) \textit{при условии ограниченности 
функций}~(\ref{e3.11-s1}).

\smallskip

\textbf{Теорема~3.2.}\
\textit{Пусть для МСтС}~(\ref{e2.3-s1}), (\ref{e2.4-s1}) 
\textit{выполнены условия существования и~единственности решения, 
а~мат\-ри\-ца $\si_1\hm=\psi_1\nu_0\psi_1^{\mathrm{T}}$ не вырождена. 
Тогда алгоритм ОСОФ согласно МОР задается уравнениями}~(\ref{e3.1-s1}), 
(\ref{e3.6-s1})--(\ref{e3.10-s1}) 
\textit{при условии ограниченности функций $f$, $f^{(1)}$, $\bar f^{(2)}$, 
$h$, $\rho_r$, $F_\kappa$ и~$H_\kappa$, 
определяемых}~(\ref{e3.5-s1}) и~(\ref{e3.11-s1}).


\smallskip

Применяя методы теории чувствительности~\cite{9-s1, 10-s1} 
для приближенного анализа фильтрационных уравнений и~учитывая 
случайность параметров~$\Theta$, придем к~следующим уравнениям 
для функций чувствительности первого порядка:

\noindent
\begin{align*}
d\nabla^\Theta \hat X_s &= \nabla^\Theta A^{\hat X_s} \,dt + \nabla^\Theta B^{\hat X_s}\,dY_t\,,\\ 
&\hspace*{40mm}\nabla^\Theta B^{\hat X_s}(t_0) =0\,;\\[2pt]
d\nabla^\Theta R_{sq} &= \nabla^\Theta A^{R_{sq}}\, dt + \nabla^\Theta B^{R_{sq}}\,dY_t\,, \\
&\hspace*{40mm}\nabla^\Theta R_{sq}(t_0) =0\,;\\[2pt]
d\nabla^\Theta c_{\kappa} &= \nabla^\Theta A^{c_\kappa}\, dt + \nabla^\Theta 
B^{c_\kappa}\,dY_t,\enskip  \nabla^\Theta c_\kappa(t_0) =0.
%\eqno(3.12)
\end{align*}
Здесь процедура взятия производных осущест\-вляется по всем входящим переменным, 
а~коэффициенты чувствительности вычисляются при\linebreak  $\Theta\hm=m^\Theta$. При 
этом предполагается малость дисперсий по сравнению с~их математическими 
ожиданиями. Очевидно, что при дифференцировании по~$\Theta$ 
$(\nabla^\Theta \hm= \prt /\prt\Theta)$
порядок уравнений возрастает пропорционально числу производных. 

Аналогично 
составляются уравнения для элементов матриц вторых функций чувствитель-\linebreak ности.

Для оценки качества ОСОФ, следуя~\cite{1-s1, 2-s1}, при гауссовских~$\Theta$ 
с~математическим ожиданием~$m^\Theta$ и~ковариационной мат\-ри\-цей~$K^\Theta$  
введем условную функцию потерь, допускающую квадратичную аппроксимацию:
    \begin{multline*}
    \rho^{\hat X_s}=\rho^{\hat X_s}(\Theta) =\rho \left(m^\Theta\right) +
    \sss_{ii=1}^{n^\Theta} \rho_i' \left(m^\Theta\right)\Theta_s^0+ {}\\
    {}+
    \sm2\limits_{i,j=1} \rho_{ij}'' \left(m^\Theta\right)\Theta_i^0 
    \Theta_j^0\,,
%    \label{e3.13-s1}
    \end{multline*}
а также показатель~$\eps$:
\begin{equation*}
\eps =\eps_2^{1/4}.
%\label{e3.14-s1}
\end{equation*}
Здесь введены обозначения:
    $$\eps_2 = \mm^N \lk \rho (\Theta)^2\rk -\rho (m^\Theta)^2\,;$$
    
    \vspace*{-9pt}
    
    \noindent
    \begin{multline*}
\mm^N \lk \rho(\Theta)^2\rk = \rho \left(m^\Theta\right)^2 +
\rho' \left(m^\Theta\right)^{\mathrm{T}} K^\Theta \rho'\left(m^\Theta\right)+ {}\\[3pt]
{}+
2\rho \left(m^\Theta\right) \mathrm{tr}\, \lk \rho''\left(m^\Theta\right)K^\Theta\rk+{}\\[3pt]
{}+\lf \mathrm{tr}\, \lk \rho'' \left(m^\Theta\right) K^\Theta\rk \rf^2+2 
\mathrm{tr}\, \lk \rho''\left(m^\Theta\right) K^\Theta\rk^2\,,
\end{multline*}
а функции $\rho'$ и~$\rho''$ по известным формулам~\cite{9-s1, 10-s1} 
определяются на основе первых и~вторых функций чувствительности.

Изложенные выше методы синтеза ОСОФ дают
принципиальную возможность получить фильтр, близкий к~оптимальному по
оценке с~любой сте\-пенью точности.
Чем выше максимальный порядок учитываемых моментов, КОР, тем выше будет точность
приближения к~оптимальной оценке. Однако число уравнений,
определяющих параметры апостериорного одномерного распределения, быст\-ро растет
с~увеличением числа учитываемых па\-ра\-мет\-ров.
Соответствующие оценки можно найти\linebreak в~[4--6].

\section{Эллипсоидальная аппроксимация распределений}

Для конечномерных МСтС часто оказывается полезной
структурная аппроксимация распределений посредством эллипсоидальных
распределений. Следуя~\cite{7-s1, 11-s1},  для структурной аппроксимации
плотностей вероятности случайных векторов будем использовать
плотности, имеющие эллипсоидальную структуру, т.\,е.\ плотности, 
у~которых поверхностями уровней равной вероятности являются подобные
концентрические эллипсоиды (эллипсы\linebreak
 для двумерных векторов,
эллипсоиды для трехмерных векторов, гиперэллипсоиды для векторов\linebreak
размерности больше трех). В~частности, эллипсо\-и\-даль\-ную структуру
имеет нормальное распределение в~любом конечномерном пространстве.
Харак\-терная особенность таких распределений состоит в~том, что их
плот\-ности вероятности являются функциями  положительно определенной квадратичной
формы $u\hm=u(y)\hm=(y^{\mathrm{T}}-m^{\mathrm{T}})C(y\hm-m)$, 
где~$m$~--- математическое ожидание
случайного вектора~$Y$; $C$~--- некоторая положительно определенная матрица.

Для нахождения ЭА плотности вероятности\linebreak
$r$-мер\-но\-го случайного вектора будем пользоваться конечным
отрезком разложения по биортонормальной системе полиномов
$\{p_{r,\nu}(u(y)),q_{r,\nu}(u(y))\}$, которые зависят только от
квадратичной формы $u\hm=u(y)$ и~функцией веса для которых служит
некоторая плотность вероятности эллипсоидальной структуры
$w(u(y))$:
\begin{equation}
\mm_{\Delta^y}^w \lk w(u(Y))p_{r,\nu} (Y)q_{r,\mu}(u(Y))\rk=
    \delta_{\nu\mu}\,.
    \label{e4.1-s1}
    \end{equation}

Индексы $\nu$ и~$\mu$ у~полиномов означают их степень относительно
переменной~$u$. Конкретный вид и~свойства полиномов определены
ниже. Однако без потери общности можно принять, что
$q_{r,0}(u)\hm=p_{r,0}(u)\hm=1$. Тогда плотность вероятности вектора~$Y$
может быть приближенно представлена в~виде:
\begin{equation}
f(y)\approx  f^*(u)=w(u)\sum\limits_{\nu=0}^N
    \crn \prn(u)\,,\label{e4.2-s1}
    \end{equation}
где коэффициенты~$\crn$ определяются по формуле:
\begin{equation}
\crn=\mm_{\Delta^y}^{\mathrm{ЭА}}\left[q_{r,\nu}(U)\right]\enskip 
(\nu=1,\ldots,N)\,.
    \label{e4.3-s1}
    \end{equation}
Учитывая, что $p_{r,0}(u)$ и~$q_{r,0}(u)$~--- взаимно обратные
постоянные (полиномы нулевой степени), то всегда $c_{r,0}p_{r,0}\hm=1$.
Поэтому из формулы~(\ref{e4.2-s1}) следует, что
\begin{equation*}
f(y)\approx f^*(u)=w(u)\left[\,1+\sum\limits_{\nu=2}^N \crn\prn(u)
    \,\right]\,.
    %\label{e4.4-s1}
    \end{equation*}

Для приложений большое значение имеет случай, когда за
распределение $w(u)$ выбирается нормальное (гауссовское) распределение:
\begin{equation*}
w(u)=w(y^{\mathrm{T}}Cy)=\fr{1}{\sqrt{(2\pi)^r\vert
    K\vert}}\exp\left(-y^{\mathrm{T}}K^{-1}\fr{y}{2}\right)\,.
%    \label{e4.5-s1}
    \end{equation*}
Учитывая, что $C\hm=K^{-1}$, приведем условие биортонормальности~(\ref{e4.1-s1}) 
к~виду:
\begin{equation*}
\fr{1}{2^{r/2}\Gamma(r/2)}\int\limits_0^{\infty}
    \prn(u)\qrm(u)u^{r/2-1}e^{-u/2}\,du=\delta_{\nu\mu}\,.
%    \label{e4.6-s1}
    \end{equation*}
Задача выбора системы полиномов $\{\prn(u),\qrm(u)\}$,
используемой при ЭА~(\ref{e4.2-s1})
и~(\ref{e4.3-s1}), сводится к~нахождению биортонормальной системы
полиномов, для которой весом служит $\chi^2$-рас\-пре\-де\-ле\-ние 
с~$r$~степенями свободы, при этом используются
следующие формулы:
\begin{multline*}
p_{r,\nu}(u)=S_{r,\nu}(u)\,,\enskip q_{r,\nu}(u)={}\\
{}=
\fr{(r-2)!!}{(r+2\nu-2)!!(2\nu)!!}\,S_{r,\nu}(u),
    \quad r\ge 2\,;
    %\label{e4.7-s1}
    \end{multline*}
    
    \vspace*{-12pt}
    
    \noindent
\begin{multline*}
    S_{r,\nu}(u)=S_{\nu}^{r/2-1}(u)={}\\
    {}=\sum\limits_{\mu=0}^{\nu}(-1)^{\nu+\mu}
    C_{\nu}^{\mu}\fr{(r+2\nu-2)!!}{(r+2\mu-2)!!}\,u^{\mu}\,.
   % \label{e4.8-s1}
\end{multline*}

При разложении по полиномам $\srn(u)$ плот\-ности вероятности
случайного вектора~$Y$ и~всех его возможных проекций согласованы.

\vspace*{-3pt}

\section{Субоптимальные фильтры на~основе метода эллипсоидальной
аппроксимации (линеаризации)}

Применим МЭА для приближенного решения задачи оптимальной нелинейной 
фильтрации в~дифференциальной гауссовской СтС. Для этого аппроксимируем
апостериорную плотность $p_t(x)$ формулой:
\begin{equation}
p_t(x) = p^* (x;\Theta) = w_1 (u_t) \lk 1+\sss_{\nu=2}^N 
c_\nu p_\nu (u_t)\rk\,,
    \label{e5.1-s1}
    \end{equation}
    где
    $$
    u_t=\left(x^{\mathrm{T}} - \hat X_t^{\mathrm{T}}\right) 
    C_t\left(x-\hat X_t\right)\,;
$$
$C_t$~--- матрица, обратная по отношению к~ковариационной
матрице ошибки фильтрации~$R_t$, $C_t\hm=R_t^{-1}$;

\noindent
\begin{multline}
c_\kkk =\mm_{\Delta^x}^{f_1} \lk q_\kkk (U_t)\rk =
\left[ q_\kkk \left(\fr{\partial^{\mathrm{T}}}{i\partial \la }- 
m_t^{\mathrm{T}}\right)\times{}\right.\\
\left.{}\times C_t \left(\fr{\partial^{\mathrm{T}} }{i\partial \la} - 
m_t\right) g_1(\la;t)\right]_{\la=0}.
\label{e5.2-s1}
\end{multline}
Чтобы найти стохастический дифференциал~$c_\kkk$, применим формулу
Ито~\cite{5-s1}, учитывая, что~$c_\kkk$ представляет собой функцию
трех случайных процессов~$\hx_t$, $R_t$ и~$g_t(\la)$,
стохастические дифференциалы Ито которых определяются 
формулами~(\ref{e3.6-s1}) и~(\ref{e3.7-s1}). В~результате получим уравнение~(\ref{e3.10-s1}), 
которое на основании равенства  $q_\kkk (\alp) \hm = 
\mm_{\Delta^x}^{f_1} q_\kkk (U_t)$ имеет вид:
    \begin{multline}
    dc_\kkk=\biggl\{ F_\kkk
    +\sum\limits_{s=1}^n \mm_{\Delta^x}^{f_1}\lk\fr{\partial q_\kkk (U_t)}
{\partial \hx_s}\rk
    +{}\\
    {}+\sum\limits_{s=1}^n \mm_{\Delta^x}^{f_1}\lk 
    \fr{\partial q_\kkk (U_t)}{\partial R_{su}}\left(
    f_{su}^{(2)} - h_s \psi_1\nu\psi_1^{\mathrm{T}} h_u^{\mathrm{T}}\right)\rk+{}\\
{}+\fr{1}{2}\sss_{s,u=1}^n \mm_{\Delta^x}^{f_1}\lk
\fr{\partial^2 q_\kkk (U_t)}{\partial \hx_s \partial \hx_u}\,h_s
    \psi_1\nu\psi_1^{\mathrm{T}}  h_u^{\mathrm{T}}\rk
    +{}\\
    {}+\fr{1}{2}\sss_{s,u,k,l=1}^n \mm_{\Delta^x}^{f_1}\lk 
    \fr{\partial^2 q_\kkk (U_t)}{\partial R_{su}\partial R_{kl}}
\,\eta_{su}  \psi_1\nu\psi_1^{\mathrm{T}}  \eta_{kl}^{\mathrm{T}} \rk+{}\\
{}+\sss_{s,k,l=1}^n \mm_{\Delta^x}^{f_1}\lk 
\fr{\partial^2 q_\kkk (U_t)}{ \partial \hx_s \partial R_{kl}}\, h_s
    \psi_1\nu\psi_1^{\mathrm{T}}  \eta_{kl}^{\mathrm{T}}\rk \biggr\} \,dt+{}\\
{}+ \left\{ H_\kkk +\sss_{s=1}^n \mm_{\Delta^x}^{f_1}\lk
\fr{\partial q_\kkk (U_t)}{\partial \hx_s }\,h_s\rk
    +{}\right.\\
   \left. {}+\sss_{s,u=1}^n \mm_{\Delta^x}^{f_1}\lk 
    \fr{\partial q_\kkk (U_t)}{\partial R_{su} }\,\eta_{su}\rk\right\} 
    \left(dY_t - f^{(1)} \,dt\right).
    \label{e5.3-s1}
    \end{multline}

Вычислим входящие в~(\ref{e5.3-s1})
математические ожидания производных полинома $q_\kkk(U_t)$. Имеем:
\begin{multline}
\mm_{\Delta^x}^{f_1} \lk \fr{\partial q_\kkk (U_t)}{\partial \hx_s }\rk= 
\mm_{\Delta^x}^{f_1}\lk
    q_\kkk'\fr{\partial U}{\partial \hx_s } \rk= {}\\
    \!\!{}=
    \mm_{\Delta^y}^{f_1}\lk q_\kkk' (U_t) \left( - 2
    \sss_{j=1}^n  c_{sj} (X_j - \hx_j)\right)\rk =0;\!\label{e5.4-s1}
    \end{multline}
    
    \vspace*{-12pt}
    
    \noindent
    \begin{multline}
\mm_{\Delta^x}^{f_1} \lk\fr{\partial^2 q_\kkk (U_t)}{\partial \hx_s \partial
    \hx_u}\rk={}\\
{}  =\mm_{\Delta^x}^{f_1} \left[  4 q_\kkk'' (U_t) \left( 
\sss_{j=1}^n  c_{uj} \left(X_j - \hx_j\right)\right)\times{}\right.\\
\left.{}\times \left( \sss_{j=1}^n  c_{sj}
    \left(X_j - \hx_j\right)\right)+ 2 q_\kkk' \left(U_t\right) 
    c_{su}\right]\,;
    \label{e5.5-s1}
    \end{multline}
%     \vspace*{-12pt}
    
    \noindent
    \begin{multline*}
\mm_{\Delta^y}^{f_1}\lk q_\kkk''\left(U_t\right) C_t\left(X_t-\hx_t\right) 
\left(X_t^{\mathrm{T}} - \hx_t^{\mathrm{T}}\right) C_t\rk ={}\\
{}= \left\{ \vphantom{\sss_{\nu=2}^N}
\fr{a}{n}\,\mm_{\Delta^U}^{w_1} \lk q_\kkk''(U) 
U^{n/2}\rk + {}\right.\\
\left.{}+\sss_{\nu=2}^N c_\nu \fr{a}{n}
\,\mm_{\Delta^U}^{w_1 p_\nu}\lk q_\kkk''(U) U^{n/2}\rk\right\} C_t\,,
%\label{e5.6-s1}
\end{multline*}
где нормирующий множитель~$a$ определяется соотношением:
     $$
     a^{-1} =\mm_{\Delta^U}^{w_1 p_\nu}\lk  U^{n/2-1}\rk\,.
     %\eqno(5.7)
     $$
Введя обозначения
\begin{align*}
    \xi_{\kkk 0} &= \fr{a}{n}\,\mm_{\Delta^U}^{w_1 }\lk 
    q_\kkk''(U) U^{n/2}\rk\,;\\[2pt]
    \xi_{\kkk\nu} &=
 \fr{a}{n}\,\mm_{\Delta^U}^{w_1 p_\nu}\lk q_\kkk''(U) U^{n/2}\rk
%\eqno(5.8)
 \end{align*}
и заметив, что вследствие ортогональности  $p_\nu (u)$ ко всем
функциям~$u^\la$ при  $\la\hm<\nu'$ величина~$\xi_{\kkk \nu}$
обращается в~нуль при  $\nu\hm>\kkk-1$, получим:
\begin{multline}
\mm_{\Delta^U}^{w_1 }\lk q_\kkk'' \left(U_t\right) C_t\left(X_t-\hx_t\right) 
\left(X_t^{\mathrm{T}}-\hx_t^{\mathrm{T}}\right) C_t \rk={}\\[1pt]
{}= \left( \xi_{\kkk 0}
    +\sss_{\nu=2}^{\kkk-1} c_\nu \xi_{\kkk \nu}\right) C_t.
    \label{e5.9-s1}
    \end{multline}
На основании~(\ref{e5.9-s1}) имеем:
\begin{multline}
4\mm_{\Delta^U}^{w_1 }\left[ q_\kkk'' \left(U_t\right) 
\left( \sss_{j=1}^n c_{uj} \left(X_j - \hx_j\right)\right)\times{}\right.\\[1pt]
\left.{}\times
    \left( \sss_{j=1}^n c_{sj} (X_j - \hx_j)\right)\right]= {}\\[1pt]
    {}=
    4 \left( \xi_{\kkk 0} +\sss_{\nu=2}^{\kkk-1} c_\nu \xi_{\kkk \nu}\right) c_{su}\,.
\label{e5.10-s1}
\end{multline}
Математическое ожидание во втором слагаемом в~(\ref{e5.5-s1}) определяется
формулой:
\begin{multline*}
\mm_{\Delta^U}^{w_1 }\lk q_\kkk' \left(U_t\right) \rk = 
\mm_{\Delta^U}^{w_1 }\lk q_\kkk' (U) U^{n/2-1}\rk +{}\\[1pt]
{}+
\sss_{\nu=2}^N c_\nu \mm_{\Delta^U}^{w_1 p_\nu}\lk q_\kkk'(U)\rk.
%\label{e5.11-s1}
\end{multline*}
Введя обозначения
\begin{align*}
\zeta_{\kkk 0}' &= a \mm_{\Delta^U}^{w_1}\lk q_\kkk' (U) U^{n/2-1}\rk\,;\\[6pt]
\zeta_{\kkk \nu}' &= a \mm_{\Delta^U}^{w_1 p_\nu}\lk q_\kkk' (U) U^{n/2-1}\rk
%\label{e5.12-s1}
\end{align*}
и заметив, что вследствие ортогональности  $p_\nu (u)$ ко всем
функциям~$u^\la$ при  $\la\hm<\nu$ величина  $\zeta_{\kkk\nu}$
обращается в~нуль при  $\nu\hm> \kkk\hm-1$, получим:
\begin{equation}
\mm_{\Delta^U}^{w_1 } \lk q_\kkk' (U_t)\rk =
\zeta_{\kkk 0}' +\sss_{\nu=2}^{\kkk-1} c_\nu \zeta_{\kkk \nu}'\,.
\label{e5.13-s1}
\end{equation}
На основании~(\ref{e5.10-s1}) и~(\ref{e5.13-s1}) формула~(\ref{e5.5-s1}) принимает вид:
\begin{equation*}
\mm_{\Delta^U}^{w_1 }\lk \fr{\partial^2 q_\kkk (U_t)}
{\partial \hx_s \partial \hx_u }\rk =2 \left(\zeta_{\kkk 0} +
    \sss_{\nu=2}^{\kkk-1}  c_{\nu}\zeta_{\kkk \nu}\right) c_{su} \,,
%    \label{e5.14-s1}
    \end{equation*}
где для краткости положено:
    $$
    \zeta_{\kkk 0} = 2 \xi_{\kkk 0} + \zeta_{\kkk 0}'\,;\quad
     \zeta_{\kkk \nu} = 2 \xi_{\kkk \nu} + \zeta_{\kkk \nu}'\,.
    $$

Вычислив производные  $\partial q_\kkk (U_t)/ \partial R_{su}$,
получим:
\begin{align*}
\mm_{\Delta^U}^{w_1 }\lk \fr{\partial q_\kkk (U_t)}{\partial R_{ss} }\rk &=
- c_{ss} \left( \gamma_{\kkk 0}
    +\sss_{\nu=2}^\kkk c_\nu \gamma_{\kkk\nu}\right)\,;%\label{e5.15-s1}
    \\
\mm_{\Delta^U}^{w_1 }\lk \fr{\partial q_\kkk (U_t)}{\partial R_{su} }\rk&=
- 2c_{su} \left( \gamma_{\kkk 0}
    +\sss_{\nu=2}^\kkk c_\nu \gamma_{\kkk\nu}\right)\,,\\ 
    &\hspace*{40mm}    s\ne u\,,
%    \label{e5.16-s1}
    \end{align*}
где $\gamma_{\kkk 0}$, $\gamma_{\kkk 2}\tr \gamma_{\kkk\kkk}$
определяются формулами:
\begin{align*}
\gamma_{\kkk 0} &= \fr{a}{n}\,\mm_{\Delta^U}^{w_1 }\lk q_\kkk'(U) U^{n/2}\rk\,;%\label{e5.17-s1}
\\
\gamma_{\kkk \nu} &= \fr{a}{n}\,\mm_{\Delta^U}^{w_1 p_\nu} \lk q_\kkk'(U) 
U^{n/2}\rk. %\label{e5.18-s1}
\end{align*}

Дифференцируя эти формулы по компонентам вектора  $\hx_t$ и
элементам матрицы $R_t$ и~имея в~виду, что
    \begin{equation}
\fr{\partial c_{ij}}{\partial R_{rr} } =- c_{ri} c_{rj};\enskip
\fr{\partial c_{ij}}{\partial R_{rs} } =- 
\left(c_{ri} c_{sj}+ c_{si} c_{rj}\right),\label{e5.19-s1}
\end{equation}
получаем:
\begin{align}
\mm_{\Delta^U}^{w_1 }\lk \fr{\partial^2 q_\kkk (U_t)}
{\partial \hx_k\partial R_{su} }\rk&=0 \enskip (k,s,u=1\tr n)\,;\label{e5.20-s1}\\
\mm_{\Delta^U}^{w_1 }\lk  \fr{\partial^2 q_\kkk (U_t)}
{\partial R_{ss}\partial R_{rr} }\rk &= c_{rs}^2
    \left( \gamma_{\kkk 0} +\sss_{\nu=2}^\kkk c_\nu \gamma_{\kkk \nu}\right)\,;
    \label{e5.21-s1}\\
\mm_{\Delta^U}^{w_1 }\lk\fr{\partial^2 q_\kkk (U_t)}
{\partial R_{ss}\partial R_{kl} }\rk&=
    2  c_{ls} c_{ks}\left( \gamma_{\kkk 0} +\sss_{\nu=2}^\kkk c_\nu 
    \gamma_{\kkk \nu}\right),\notag\\
    &\hspace*{30mm}k\ne l\,;\label{e5.22-s1}\\
\mm_{\Delta^U}^{w_1 }\lk \fr{\partial q_\kkk (U_t)}
{\partial R_{su}\partial R_{kl} }\rk &={}\notag\\
&\hspace*{-20mm}{}=
    2\left(c_{ks}c_{lu} + c_{ls} c_{ku}\right)     
    \left( \gamma_{\kkk 0} +\sss_{\nu=2}^\kkk c_\nu 
    \gamma_{\kkk \nu}\right)\,,\notag\\ 
    &\hspace*{20mm}s\ne u\,,\enskip k\ne l\,.\label{e5.23-s1}
    \end{align}

Подставив выражения~(\ref{e5.4-s1}), (\ref{e5.19-s1})--(\ref{e5.23-s1}) 
в~уравнение для стохастического дифференциала величины~$c_\kkk$, 
приведем его к~виду:
\begin{multline}
dc_\kkk = \left\{ 
\vphantom{\fr{1}{2} 
\left( \gamma_{\kkk 0} +\sss_{\nu=2}^\kkk c_\nu \gamma_{\kkk \nu}
\right) \sss_{s,u,k,l=1}^n A_{sukl} \eta_{su} \psi_1\nu\psi_1^{\mathrm{T}} 
\eta_{kl}^{\mathrm{T}}}
F_\kkk - \left( \gamma_{\kkk 0} +
\sss_{\nu=2}^\kkk c_\nu \gamma_{\kkk \nu}\right)\times{}\right.\\
{}\times    \,\mathrm{tr}\, \lk \bar C_t \left( f^{(2)} - 
    h \psi_1\nu\psi_1^{\mathrm{T}} h^{\mathrm{T}}\right)\rk+{}\\
{}+  \left( \zeta_{\kkk 0} +\sss_{\nu=2}^{\kkk-1} c_\nu \zeta_{\kkk \nu}\right)
    \,\mathrm{tr}\, \lk C_t  h \psi_1\nu\psi_1^{\mathrm{T}} h^{\mathrm{T}}\rk+{}\\
\left.{}+\fr{1}{2} \!
\left( \!\gamma_{\kkk 0} +\sss_{\nu=2}^\kkk c_\nu \gamma_{\kkk \nu}\!
\right) \!\!\sss_{s,u,k,l=1}^n \!\!\!\!\!\!A_{sukl} \eta_{su} \psi_1\nu\psi_1^{\mathrm{T}} 
\eta_{kl}^{\mathrm{T}}\!
\right\} dt+{}\\
{}+\left\{ H_\kkk -\left( \gamma_{\kkk 0} +\sss_{\nu=2}^n c_\nu \gamma_{\kkk \nu}
\right) \sss_{s,u,=1}^n \bar c_{su} \eta_{su} \right\}\times{}\\
{}\times 
\left(dY_t - f^{(1)} dt\right)\enskip
(\kkk = 2\tr N),
\label{e5.24-s1}
\end{multline}
где
\begin{gather*}
A_{ssrr} = c_{sr}^2\,,\enskip A_{sskl} = 2c_{ks}c_{ls}\,,\enskip k\ne l\,;\\
A_{sukl} = 2(c_{ks}c_{lu} + c_{ls} c_{ku})\,,\enskip s\ne u\,,\ 
k\ne l\,;
%\label{e5.25-s1}
\end{gather*}
$\bar c_{su}$~--- элементы матрицы~$\bar C_t$, определяемой
формулой:
\begin{equation*}
\mm_{\Delta^U}^{w_1 }\lk q_\kkk^K \left(U_t\right) \rk =
-  \left( \gamma_{\kkk 0} +\sss_{\nu=2}^\kkk c_\nu \gamma_{\kkk \nu}\right)C_t.
%\label{e5.26-s1}
\end{equation*}

Формулы~(\ref{e3.5-s1}) для~$f$, $f^{(1)}$, $f^{(2)}$, 
$h$ и~$\rho_r$ при аппроксимации~(\ref{e5.1-s1}) апостериорной плот\-ности
принимают вид:
\begin{multline}
f= f(Y_t,\Theta,t) = \mm_{\Delta^x}^{w_1} \lk 
\varphi\left(Y_t, X,\Theta, t\right)\rk+ {}\\
{}+
\sss_{\nu=2}^N c_\nu \mm_{\Delta^x}^{w_1 p_\nu}\lk 
\vrp\left(Y_t, X,\Theta,t\right)\rk\,;
\label{e5.27-s1}
\end{multline}

\vspace*{-12pt}

\noindent
\begin{multline}
f^{(1)}= f^{(1)}\left(Y_t,\Theta,t\right) = 
\mm_{\Delta^x}^{w_1}\lk \varphi_1\left(Y_t, X, \Theta,t\right)\rk + {}\\
{}+
\sss_{\nu=2}^N c_\nu \mm_{\Delta^x}^{w_1p_\nu }\lk \vrp_1  
\left(Y_t, X,\Theta,t\right)\rk\,;
\label{e5.28-s1}
\end{multline}

\vspace*{-12pt}

\noindent
\begin{multline*}
f^{(2)}= f^{(2)}\left(Y_t,\Theta,t\right) ={}\\
{}= \mm_{\Delta^x}^{w_1}\left[ \left(X-\hx\right)\varphi
\left(Y_t, X,\Theta, t\right)^{\mathrm{T}}+{}\right.\\
{}+
    \varphi\left(Y_t,X,\Theta,t\right) \left(X^{\mathrm{T}}-\hx_t^{\mathrm{T}}
    \right)+{}\\
      \left.{}+
    \left(\psi\nu\psi^{\mathrm{T}}\right) \left(Y_t,X,\Theta,t\right)
    \vphantom{\left(Y_t, X,\Theta, t\right)^{\mathrm{T}}}
    \right]+{}\\
{}+\sss_{\nu=2}^N c_\nu \mm_{\Delta^x}^{w_1p_\nu}\left[ 
\vphantom{\left(X-\hx\right)^{\mathrm{T}}}
\left(X-\hx\right)\vrp\left(Y_t, X,\Theta,t\right)^{\mathrm{T}}+ {}\right.
    \end{multline*}
    
    \noindent
\begin{multline}
  {}+
\vrp\left(Y_t, X,\Theta, t\right) \left(X-\hx\right)^{\mathrm{T}} + {}\\
\left.{}+
\left(\psi\nu\psi^{\mathrm{T}}\right) \left(Y_t, X,\Theta, t\right) 
\vphantom{\left(X-\hx\right)^{\mathrm{T}}}
\right]\,;
\label{e5.29-s1}
\end{multline}

\vspace*{-12pt}

\noindent
\begin{multline}
h= h(Y_t,\Theta,t) ={}\\
{}=\mm_{\Delta^x}^{w_1}\left[ X\varphi_1
\left(Y_t, X,\Theta, t\right)^{\mathrm{T}} + 
\left(\psi\nu\psi^{\mathrm{T}}\right)\left(Y_t,X,\Theta,t\right)\right]+{}\\
{}+\sss_{\nu=2}^N c_\nu \mm_{\Delta^x}^{w_1p_\nu} \left[ 
X\vrp\left(Y_t, X,\Theta, t\right)^{\mathrm{T}} + {}\right.\\
\left.{}+
\left(\psi\nu\psi^{\mathrm{T}}\right) \left(Y_t, X, \Theta,t\right)
\vphantom{\left(Y_t, X,\Theta, t\right)^{\mathrm{T}}}
\right]
    - {}\\
    {}-\hx_t {f^{(1)}}^{\mathrm{T}} \left(\psi_1\nu\psi_1^{\mathrm{T}}\right)^{-1} 
    \left(Y_t,X,\Theta,t\right)\,;
    \label{e5.30-s1}
    \end{multline}

\vspace*{-12pt}

\noindent
\begin{multline}
\rho_r = \rho_r \left(Y_t,\Theta, t\right) ={}\\
{}= \mm_{\Delta^x}^{w_1} \left[ \left(X- \hx_t\right) 
\left(X^{\mathrm{T}}-\hx_t^{\mathrm{T}}\right) a_r \left(Y_t,X,\Theta,t\right) +{}\right.\\
{}+ \left(X-\hx_t\right) b_r\left(Y_t,X,\Theta,t\right)^{\mathrm{T}} + {}\\
\left.{}+
b_r \left(Y_t,X,\Theta,t\right) \left(X^{\mathrm{T}}-\hx_t^{\mathrm{T}}\right)
 \right] +{}\\
 {}+\sss_{\nu=2}^N c_\nu \mm_{\Delta^x}^{w_1p_\nu}\left[ 
 \left(X- \hx_t\right) \left(X^{\mathrm{T}}-\hx_t^{\mathrm{T}}\right) \times{}\right.\\
{}\times a_r \left(Y_t,X,\Theta,t\right) +{}\\
\left.{}+ \left(X-\hx_t\right) b_r\left(Y_t,X,\Theta,t\right)^{\mathrm{T}} + {}\right.\\
\left.{}+
b_r \left(Y_t,X,\Theta,t\right) \left(X^{\mathrm{T}}-\hx_t^{\mathrm{T}}\right) 
\right].
\label{e5.31-s1}
\end{multline}

Формулы~(\ref{e3.11-s1}) для~$F_\kkk$ и~$H_\kkk$ преобразуются к~виду:
\begin{multline}
F_\kkk= F_\kkk\left(Y_t,\Theta,t\right) = {}\\
{}=
\mm_{\Delta^x}^{w_1} \big[ q_\kkk' (U) 2\varphi\left(Y_t, X,\Theta, t\right)^{\mathrm{T}} 
C_t(X-\hx_t)+{}\\
{}+ \mathrm{tr}\, \lk C_t\si \left(Y_t,X,\Theta,t\right)\rk \big]  + {}\\
{}+
\sss_{\nu=2}^N c_\nu \mm_{\Delta^x}^{w_1p_\nu} \bigg[ 
2\varphi\left(Y_t, X,\Theta, t\right)^{\mathrm{T}} C_t\left(X-\hx_t\right)+{}\\
{}+ \mathrm{tr}\, \lk C_t\si \left(Y_t,X,\Theta,t\right)\rk\bigg]\,;
\label{e5.32-s1}
\end{multline}

\vspace*{-12pt}

\noindent
\begin{multline}
H_\kkk= H_\kkk\left(Y_t,\Theta,t\right) ={}\\
{}=
\left\{ \mm_{\Delta^x}^{w_1} \lk\varphi_1\left(Y_t, X, \Theta,t\right)^{\mathrm{T}} 
q_\kkk (U) \rk +{}\right.\\
{}+\sss_{\nu=2}^N c_\nu \mm_{\Delta^x}^{w_1p_\nu}\lk
\varphi_1\left(Y_t, X, t\right)^{\mathrm{T}} q_\kkk (U)\rk  +{}\\
{}+2\mm_{\Delta^x}^{w_1}\lk q'(U) \!\left(X^{\mathrm{T}}-\hx_t^{\mathrm{T}}\right) 
C_t \left(\psi\nu\psi_1^{\mathrm{T}}\right) \!\left(Y_t,X,\Theta,t\right) \rk +{}\\
{}+2\sss_{\nu=2}^N c_\nu \mm_{\Delta^x}^{w_1p_\nu}\times{}\\
{}\times \lk q' (U) 
\left(X^{\mathrm{T}}-\hx^{\mathrm{T}}\right) 
C_t \left(\psi\nu\psi_1^{\mathrm{T}}\right) \left(Y_t,X,\Theta,t\right)\rk-{}\\
\left. {}- {f^{(1)}}^{\mathrm{T}} c_\kkk\right\} 
\left(\psi_1\nu\psi_1^{\mathrm{T}}\right)^{-1} \left(Y_t,\Theta,t\right).
\label{e5.33-s1}
\end{multline}

Таким образом, получен следующий результат.

\smallskip

\noindent
\textbf{Теорема~5.1.}\ \textit{Пусть уравнения
дифференциальной гауссовской СтС}~(\ref{e2.3-s1}), (\ref{e2.4-s1}) 
\textit{допускают применение МЭА.
 Тогда фильтрационные уравнения ЭСОФ 
 имеют  вид}~(\ref{e5.2-s1}), (\ref{e5.3-s1}), (\ref{e5.24-s1})
\textit{при условиях}~(\ref{e5.27-s1})--(\ref{e5.33-s1}).

\smallskip

Аналогично выводятся уравнения ЭСОФ для дифференциальной СтС 
с~винеровскими и~пуассоновскими шумами вида~(\ref{e2.3-s1}) и~(\ref{e2.4-s1}).

Далее, следуя~\cite{7-s1, 6-s1}, построим квазилинейный СОФ на основе МЭЛ для
МСтС~(\ref{e2.1-s1}), (\ref{e2.2-s1}) при $\psi'\hm=\psi'(\Theta,t)$,
$\psi''\hm=\psi''(\Theta,t,v)$, $\psi_1'\hm=\psi_1'(\Theta,t)$
и~$\psi_1''\hm=\psi_1''(\Theta,t,v)$ (т.\,е.\ с~аддитивными винеровскими 
и~пуассоновскими шумами). Уравнения НСОФ проще получаются, если
нелинейные функции~$\vrp$ и~$\vrp_1$ на основе эллипсоидального
распределения с~известным~$c_\nu$ заменить на статистически
линеаризованные~[6--8]:
\begin{equation}
\left.
\begin{array}{rl}
\vrp &=\vrp \left( X_t, Y_t, \Theta, t\right) \approx{}\\[6pt]
&\hspace*{-2mm}{}\approx \vrp_0^{\mathrm{э}} + 
k_x^{\mathrm{э}\vrp} \left(X_t - m_t^x\right) + k_y^{\mathrm{э}\vrp} 
\left(Y_t - m_t^y\right)\,;\\[6pt]
\vrp_1 &= \vrp_1\left( X_t, Y_t, \Theta, t\right) \approx{}\\[6pt]
&\hspace*{-5mm}{}\approx \vrp_{10}^{\mathrm{э}} 
 + k_x^{\mathrm{э}\vrp_1}  \left(X_t - m_t^x\right) + 
 k_y^{\mathrm{э}\vrp_1} \left(Y_t - m_t^y\right)\,,
 \end{array}
 \right\}
 \label{e5.34-s1}
 \end{equation}
а затем использовать уравнения линейной фильтрации~[6--8]. 
Входящие в~(\ref{e5.34-s1}) коэффициенты с~МЭЛ зависят от математических 
ожиданий, дисперсий и~ковариаций:
    $$
    Z_t = \begin{bmatrix}
    X_t\\  Y_t\end{bmatrix}\,; 
    \enskip m_t^z =\begin{bmatrix}
     m_t^x\\ m_t^y\end{bmatrix}\,;
     \enskip K_t^z=\begin{bmatrix}
      K_t^x&K_t^{xy}\\[6pt]
      K_t^{xy}&K_t^y
      \end{bmatrix}\,.
      $$
Они определяются из уравнений:
\begin{alignat*}{2}
\dot Z_t &= A^z Z_t + A_0^z + B_0^z V \,,&\enskip V&= \dot W\,; %\eqno(5.35)
\\
\dot m_t^z &= A^z m_t^z + A_0^z \,,&\enskip m_{t_0}^Z &= m_0^z\,;%\eqno(5.36)$$
\\
\dot K_t^z &= B^z K_t^z + K_t^z (B^z)^{\mathrm{T}} + B_0^z \nu^m (B_0^z)^{\mathrm{T}},&\enskip K_{t_0}^z &= K_0^z\,. %\eqno(5.37)$$
\end{alignat*}
Здесь введены следующие обозначения:
    $$
    A_0^z = \begin{bmatrix}
     a_0\\ b_0\end{bmatrix}\,;\enskip 
     A^z =\begin{bmatrix}
     a_1&a\\ 
     b_1&b\end{bmatrix}\,;\enskip 
     B_0^z =\begin{bmatrix}
      \bar \psi\\ \bar\psi_1\end{bmatrix}\,;
      $$
    $$
    a= k_y^{\mathrm{э}\vrp}\,;\enskip 
    a_1 = k_x^{\mathrm{э}\vrp} \,;\enskip 
    a_0 =\vrp_0^{\mathrm{э}}  - k_x^{\mathrm{э}\vrp}  
    m_t^x - k_y^{\mathrm{э}\vrp}  m_t^y\,;
    $$
    \begin{equation*}
    b= k_y^{\mathrm{э}\vrp_1} \,; \enskip 
    b_1=k_x^{\mathrm{э}\vrp_1} \,;\enskip 
    b_0=\vrp_0^{\mathrm{э}}  -k_x^{\mathrm{э}\vrp_1}  
    m_t^x -k_y^{\mathrm{э}\vrp_1} m_t^y\,;
%    \label{e5.38-s1}
    \end{equation*}
    \begin{equation}
    \left.
    \begin{array}{rl}
\psi \,dW_0 + \displaystyle\iii_{R_0^q} \psi'' P^0 (dt, dv) &=\bar \psi\,  dW\,;\\[6pt]
\psi_1'\, dW_0 + \displaystyle\iii_{R_0^q} \psi_1'' P^0 (dt, dv)&= \bar \psi_1 \,dW\,,
\end{array}
\right\}
\label{e5.39-s1}
\end{equation}
где $\nu^W$~--- интенсивность СтП с~независимыми приращениями, состоящего 
из винеровской и~пуассоновской частей~(\ref{e5.39-s1}). 
Тогда уравнения квазилинейного НСОФ будут иметь вид:
\begin{multline}
\dot{\hat X}_t = a Y_t + a_1\hat X_t + a_0 +{}\\
{}+ \beta_t \lk Z_t - 
\left(bY_t + b_1 \hat X_t + b_0\right)\rk\,;
\label{e5.40-s1}
\end{multline}

\vspace*{-6pt}

\noindent
\begin{equation}
\beta_t = \left(R_t b_1^{\mathrm{T}} + \bar\psi \nu^W \bar\psi_1^{\mathrm{T}}\right) 
\left(\bar\psi_1\nu^W\bar\psi_1^{\mathrm{T}}\right)^{-1}\,;
\label{e5.41-s1}
\end{equation}

\vspace*{-12pt}

\noindent
\begin{multline}
\dot R_t = a_1 R_t + R_t a_1^{\mathrm{T}} + \bar\psi \nu^W \bar\psi^{\mathrm{T}}
- {}\\
{}-\left(R_t b_1^{\mathrm{T}} +\bar\psi \nu^W\bar\psi_1^{\mathrm{T}}\right)
\left(\bar\psi_1 \nu^W\bar\psi_1^{\mathrm{T}}\right)^{-1} \times{}\\
{}\times
\left(b_1 R_t + \bar\psi_1 \nu^W\bar\psi^{\mathrm{T}}\right)\,.
\label{e5.42-s1}
\end{multline}


\noindent
\textbf{Теорема~5.2.}\ \textit{Пусть МСтС}~(\ref{e2.1-s1}), (\ref{e2.2-s1}) 
\textit{содержит только аддитивные винеровские и~пуассоновские шумы и~допускает 
замену статистически линеаризованной, а~матрица  $\si_1 \hm=\bar\psi_1 
\nu^W \bar\psi_1^{\mathrm{T}}$ не вырождена. Тогда в~основе алгоритма 
квазилинейного НСОФ лежат уравнения}~(\ref{e5.40-s1})--(\ref{e5.42-s1}) 
\textit{при соответствующих начальных условиях.}

\smallskip

Аналогично разд.~3 (в~условиях теорем~5.1 и~5.2) составляются уравнения 
для оценки точности и~чувствительности к~параметрам~$\Theta$.

\section{Заключение}

Разработана теория аналитического синтеза ЭСОФ для нелинейных дифференциальных 
МСтС). Рассмотрены случаи гауссовских 
и~негауссовских СтС. Алгоритмы ЭСОФ положены в~основу модуля экспериментального 
программного обеспечения  StS-Filter (version 2016).

Результаты допускают развитие на случай дискретных СтС.

Теоретический и~практический интерес представляет теория  ЭСОФ 
на основе ненормированных апостериорных распределений~\cite{10-s1}.


{\small\frenchspacing
 {%\baselineskip=10.8pt
 \addcontentsline{toc}{section}{References}
 \begin{thebibliography}{99}

\bibitem{1-s1}
\Au{Синицын И.\,Н.}
Аналитическое моделирование распределений на основе ортогональных
разложений в~нелинейных стохастических системах на многообразиях~//
Информатика и~её применения, 2015. Т.~9. Вып.~3. C.~17--24.


\bibitem{2-s1}
\Au{Синицын И.\,Н.}
Применение ортогональных разложений для аналитического моделирования
многомерных распределений в~нелинейных стохастических системах на
многообразиях~// Системы и~средства информатики, 2015. Т.~25. №\,3.
С.~3--22.

\bibitem{3-s1}
\Au{Синицын И.\,Н.}
Ортогональные субоптимальные фильтры для нелинейных стохастических
систем на многообразиях~// Информатика и~её применения, 2016. Т.~10.
Вып.~1. С.~34--44.

\bibitem{7-s1} %4
\Au{Ватанабэ С., Икэда Н.} Стохастические дифференциальные уравнения 
и~диффузионные процессы~/ Пер. с~англ.~--- М.: Наука, 1986. 448~с.
(\Au{Watanabe~S, Ikeda~N.} 
Stochastic differential equations and diffusion processes.~--- 
Amsterdam\,--\,Oxford\,--\,New York: North-Holland Publishing Co.; 
Tokyo: Kodansha Ltd., 1981. 476~p.)

\bibitem{6-s1} %5
Справочник по теории вероятностей и~математической статистике~/ Под
ред. В.\,С.~Королюка, Н.\,И.~Портенко, А.\,В.~Скорохода, А.\,Ф.~Турбина.~--- 
М.: Наука, 1985. 640~с.


\bibitem{4-s1} %6
\Au{Пугачев В.\,С., Синицын И.\,Н.}
 \Au{Пугачёв В.\,С., Синицын~И.\,Н.}
Стохастические дифференциальные системы. Анализ и~фильтрация.~--- М.:
Наука,  1990.  632~с. 

\bibitem{5-s1} %7
\Au{Пугачёв В.\,С., Синицын И.\,Н.}
Теория стохастических систем.~--- М.: Логос, 2000; 2004. 1000~с.
%[Англ. пер. Stochastic Systems. Theory and  Applications. --
%Singapore: World Scientific, 2001. 908~p.].


\bibitem{8-s1} %8
\Au{Синицын И.\,Н.}
Фильтры Калмана и~Пугачева.~--- 2-е изд.~--- М.: Логос, 2007. 776~с.

\bibitem{9-s1}
\Au{Wonham W.\,M.}
Some applications of stochastic differential equations to optimal
nonlinear filtering~// J.~Soc. Ind. Appl. Math. A, 1964.
Vol.~2. No.\,3. P.~347--369.

\bibitem{10-s1}
\Au{Zakai M.}
On the optimal filtering of diffusion processes~// Z.
Wahrscheinlichkeit., 1969. Bd.~11. S.~230--243.


\bibitem{11-s1}
\Au{Синицын И.\,Н., Синицын~В.\,И.}
Лекции по теории нормальной и~эллипсоидальной аппроксимации
распределений в~стохастических системах.~--- М.: ТОРУС ПРЕСС, 2013.
488~с.
\end{thebibliography}

 }
 }

\end{multicols}

\vspace*{-3pt}

\hfill{\small\textit{Поступила в~редакцию 29.02.16}}

\vspace*{8pt}

%\newpage

%\vspace*{-24pt}

\hrule

\vspace*{2pt}

\hrule

%\vspace*{8pt}



\def\tit{ELLIPSOIDAL SUBOPTIMAL
FILTERS FOR~NONLINEAR STOCHASTIC
SYSTEMS ON~MANIFOLDS}

\def\titkol{Ellipsoidal suboptimal
filters for~nonlinear stochastic
systems on~manifolds}

\def\aut{I.\,N.~Sinitsyn, V.\,I.~Sinitsyn, and E.\,R.~Korepanov}

\def\autkol{I.\,N.~Sinitsyn, V.\,I.~Sinitsyn, and E.\,R.~Korepanov}

\titel{\tit}{\aut}{\autkol}{\titkol}

\vspace*{-9pt}

\noindent
Institute of Informatics Problems, Federal Research Center 
``Computer Science and Control'' of the Russian Academy of Sciences,
44-2~Vavilov Str., Moscow 119333, Russian Federation

\def\leftfootline{\small{\textbf{\thepage}
\hfill INFORMATIKA I EE PRIMENENIYA~--- INFORMATICS AND
APPLICATIONS\ \ \ 2016\ \ \ volume~10\ \ \ issue\ 2}
}%
 \def\rightfootline{\small{INFORMATIKA I EE PRIMENENIYA~---
INFORMATICS AND APPLICATIONS\ \ \ 2016\ \ \ volume~10\ \ \ issue\ 2
\hfill \textbf{\thepage}}}

\vspace*{3pt}


\Abste{For nonlinear differential stochastic systems on manifolds (MStS) 
with Wiener and Poisson noises, the synthesis theory of ellipsoidal suboptimal filters 
based on ellipsoidal approximation and ellipsoidal linearization\linebreak\vspace*{-12pt}}

\Abstend{methods is developed. Special attention is paid to MStS with additive non-Gaussian 
noises based on ellipsoidal linearization method. The algorithms are the basis of the
experimental software tool <<StS-Filter>> (version 2016). Accuracy and sensitivity 
equations are presented. Some generalizations are mentioned.}

\KWE{ellipsoidal approximation method (EAM);
ellipsoidal linearization method (ELM);
orthogonal expansions method (OEM);
Poisson  noise;
stochastic system on manifolds (MStS);
suboptimal filter (SOF);
Wiener noise}

\DOI{10.14357/19922264160203}

\vspace*{-12pt}

\Ack
\noindent
The research was supported by the Russian Foundation for Basic Research 
(project 15-07-002244).


%\vspace*{3pt}

  \begin{multicols}{2}

\renewcommand{\bibname}{\protect\rmfamily References}
%\renewcommand{\bibname}{\large\protect\rm References}

{\small\frenchspacing
 {%\baselineskip=10.8pt
 \addcontentsline{toc}{section}{References}
 \begin{thebibliography}{99}


\bibitem{1-s1-1}
\Aue{Sinitsyn, I.\,N.} 2015.
Analiticheskoe modelirovanie raspredeleniy na osnove ortogonal'nykh 
razlozheniy v~ne\-li\-ney\-nykh stokhasticheskikh sistemakh na mnogoobraziyakh 
[Analytical modeling in stochastic systems on manifolds based on orthogonal 
expansions]. \textit{Informatika i ee Primeneniya}~--- \textit{Inform Appl.} 
9(2):17--24.


\bibitem{2-s1-1}
\Aue{Sinitsyn, I.\,N.} 2015.
Primenenie ortogonal'nykh raz\-lo\-zhe\-niy dlya analiticheskogo modelirovaniya 
mno\-go\-mer\-nykh raspredeleniy v~nelineynykh stokhasticheskikh sistemakh na 
mnogoobraziyakh [Applications of orthogonal expansions for analytical modeling 
of multidimensional distributions in stochastic systems on manifolds].
\textit{Sistemy i~Sredstva Informatiki}~--- \textit{Systems and Means of Informatics}
 25(3):3--22.

\bibitem{3-s1-1}
\Aue{Sinitsyn, I.\,N.} 2016.
Ortogonal'nye suboptimal'nye fil'try dlya nelineynykh stokhasticheskikh 
sistem na mno\-go\-ob\-ra\-zi\-yakh [Orthogonal suboptimal filters for nonlinear 
stochastic systems on manifolds].
 \textit{Informatika i ee Primeneniya}~--- \textit{Inform Appl.}  
 10(1):34--44.
 
 \bibitem{7-s1-1} %4
\Aue{Watanabe,~S., and N. Ikeda}. 1981. 
\textit{Stochastic differential equations and diffusion processes}. 
Amsterdam\,--\,Oxford\,--\,New York: North-Holland Publishing Co.; 
Tokyo: Kodansha Ltd. 476~p.

\bibitem{6-s1-1} %5
Korolyuk, V.\,S., N.\,I.~Portenko, A.\,V.~Skorokhod, and
A.\,F.~Turbin, eds. 1985.
\textit{Spravochnik po teorii veroyatnosti i~matematicheskoy statistike}
[Handbook: Probability theory and mathematical statistics].
 Moscow: Nauka. 640~p.

\bibitem{4-s1-1} %6
 \Aue{Pugachev, V.\,S., and I.\,N.~Sinitsyn.} 
1987. \textit{Stochastic differential systems.
Analysis and filtering}. Chichester\,--\,New York, NY: Jonh Wiley.
549~p.


\bibitem{5-s1-1} %7
 \Aue{Pugachev, V.\,S., and I.\,N.~Sinitsyn.} 
 2001.  \textit{Stochastic systems. Theory and  applications}.
Singapore: World Scientific. 908~p.


\bibitem{8-s1-1}
\Aue{Sinitsyn, I.\,N.} 2007.
\textit{Fil'try Kalmana i~Pugacheva} [Kalman and Pugachev filters]. 
2nd ed. Moscow: Logos. 776~p.

\bibitem{9-s1-1}
\Aue{Wonham, W.\,M.} 1964.
Some applications of stochastic differential equations to optimal nonlinear 
filtering. \textit{J.~Soc. Ind. Appl. Math. A} 2(3):347--369.

\bibitem{10-s1-1}
\Aue{Zakai, M.} 1969.
On the optimal filtering of diffusion processes.
\textit{Z. Wahrscheinlichkeit.} 11:230--243.


\bibitem{11-s1-1}
\Aue{Sinitsyn, I.\,N., and V.\,I.~Sinitsyn.} 2013.
\textit{Lektsii po teorii normal'noy i~ellipsoidal'noy approkskimatsii raspredeleniy 
v~stokhasticheskikh sistemakh} [Lectures on normal and ellipsoidal approximation 
of distributions in stochastic systems].  Moscow: TORUS PRESS. 488~p.

\end{thebibliography}

 }
 }

\end{multicols}

\vspace*{-3pt}

\hfill{\small\textit{Received February 29, 2016}}

\Contr


\noindent
\textbf{Sinitsyn Igor N.} (b.\ 1940)~---
Doctor of Science in technology, professor,
Honored scientist of RF, Head of Department, Institute of Informatics Problems, Federal Research Center ``Computer Science and
Control'' of the Russian Academy of Sciences, 44-2 Vavilov Str.,
Moscow 119333, Russian Federation; sinitsin@dol.ru

\vspace*{3pt}

\noindent
\textbf{Sinitsyn Vladimir I.} (b.\ 1968)~---
 Doctor of Science in physics and mathematics,
associate professor, Head of Department, Institute of Informatics Problems, Federal Research Center ``Computer Science and
Control'' of the Russian Academy of Sciences, 44-2 Vavilov Str.,
Moscow 119333, Russian Federation; vsinitsin@ipiran.ru

\vspace*{3pt}

\noindent
\textbf{Korepanov Eduard R.} (b.\ 1966)~---
Candidate of Science (PhD) in technology, 
Head of Laboratory, Institute of Informatics Problems, Federal Research Center 
``Computer Science and Control'' of the Russian Academy of Sciences, 
44-2~Vavilov Str., Moscow 119333, Russian Federation; ekorepanov@ipiran.ru 

\label{end\stat}


\renewcommand{\bibname}{\protect\rm Литература}