\def\stat{ulianov}

\def\tit{РАЗЛОЖЕНИЯ ТИПА КОРНИША--ФИШЕРА ДЛЯ~РАСПРЕДЕЛЕНИЙ СТАТИСТИК, ПОСТРОЕННЫХ 
ПО~ВЫБОРКАМ~СЛУЧАЙНОГО РАЗМЕРА$^*$}

\def\titkol{Разложения типа Корниша--Фишера для распределений статистик, построенных по 
выборкам случайного размера}

\def\aut{А.\,С.~Марков$^1$,  М.\,М.~Монахов$^2$, В.\,В.~Ульянов$^3$}

\def\autkol{А.\,С.~Марков,  М.\,М.~Монахов, В.\,В.~Ульянов}

\titel{\tit}{\aut}{\autkol}{\titkol}

\index{Марков А.\,С.}
\index{Монахов М.\,М.}
\index{Ульянов В.\,В.}
\index{Markov A.\,S.}
\index{Monakhov M.\,M.}
\index{Ulyanov V.\,V.}

{\renewcommand{\thefootnote}{\fnsymbol{footnote}} \footnotetext[1]
{Исследование выполнено при поддержке Российского научного фонда (проект 14-11-00364).}}


\renewcommand{\thefootnote}{\arabic{footnote}}
\footnotetext[1]{Московский государственный университет им.\ М.\,В.~Ломоносова, 
 факультет вычислительной математики
 и~кибернетики, \mbox{markov.cmc@yandex.ru}}
\footnotetext[2]{Московский государственный университет им.\ М.\,В.~Ломоносова, 
факультет вычислительной математики
 и~кибернетики,  \mbox{mih\_monah@mail.ru}}
\footnotetext[3]{Московский государственный университет им.\ М.\,В.~Ломоносова, 
факультет вычислительной математики
 и~кибернетики, \mbox{vulyanov@cs.msu.ru}}

\Abst{Для квантилей выборочного среднего по выборке случайного объема  
построены  обобщенные разложения Кор\-ни\-ша--Фи\-ше\-ра на базе квантилей 
распределений Лапласа и~Стьюдента. В~последние годы интерес к~разложениям  
Кор\-ни\-ша--Фи\-ше\-ра значительно вырос в~связи с~исследованиями по управлению рисками. 
Широко распространенная мера риска Value at Risk (VaR) является квантилью 
функции потерь. Используется общая теорема переноса, позволяющая 
получать асимптотические разложения для функций распределения  статистик по 
выборкам случайного объема  из асимптотических разложений для функции распределения  
случайного объема выборки и~асимптотических разложений для функций распределения  
статистик по выборкам неслучайного объема. Проведен вычислительный эксперимент, 
иллюстрирующий полученные разложения Кор\-ни\-ша--Фи\-шера.}

\KW{обобщенные разложения Корниша--Фишера; выборка случайного объема; 
распределение Лапласа; распределение Стьюдента}

  
\DOI{10.14357/19922264160210}

%\vspace*{6pt}

\vskip 10pt plus 9pt minus 6pt

\thispagestyle{headings}

\begin{multicols}{2}

\label{st\stat}


\section{Введение}

В классических задачах математической статистики  объем выборки
традиционно считается детерминированным и~играет роль
\textit{известного} параметра, как правило, неограниченно
возрастающего. Однако на практике очень часто    возникают ситуации,
при которых размер выборки не определен заранее и~может считаться
случайным. Обычно это возникает ввиду того, что данные накапливаются
в~течение некоторого промежутка  времени, длительность которого по
разным причинам нельзя считать фиксированной. В~этом случае встает
вопрос об аппроксимирующих разложениях для различных характеристик
статистик, например функций распределения или квантилей, основанных
на выборках случайного объема. 

Предельные распределения для случайно
индексированных последовательностей и~их применения см., например, 
в~монографии \cite{Example3}. Ранее Б.\,В.~Гнеденко в~работе
\cite{Example1}  продемонстрировал, что в~математической статистике
при замене неслучайного объема выборки случайной величиной
асимп\-то\-ти\-ческие свойства статистик могут радикально изменяться. 

В~\cite{BenKorGal} доказана общая теорема переноса, позволяющая
получать асимптотические раз\-ло\-же\-ния для функций распределения
статистик, основанных на выборках случайного объема, из
асимптотических разложений для функции распределения случайного
объема выборки и~асимптотических разложений для функций
распределения  статистик, построенных по выборкам неслучайного
объема. 

Настоящая работа раз\-вивает результаты ра\-боты~\cite{BenKorGal}: 
на основе раз\-ложения функции распределения
статистики, основанной на выборке случайного объема специального
вида, получены обобщенные разложения Кор\-ни\-ша--Фи\-ше\-ра на базе
квантилей распределений Лапласа и~Стью\-дента. 
{ %\looseness=1

}

Классические разложения
Кор\-ни\-ша--Фи\-ше\-ра на базе квантилей нормального  распределения введены
в~\cite{CornFish}, их обобщение было предложено в~\cite{HillDavis}.
В~последние годы интерес к~разложениям Кор\-ни\-ша--Фи\-ше\-ра значительно
вырос в~связи с~исследованиями по управ\-ле\-нию рисками. Широко
распространенная мера риска VaR является, по
существу, квантилью функции потерь, связанной с~описанием
инвестиционного портфеля из  финансовых инструментов (см., например,~\cite{Jashke}).


Рассмотрим случайные величины (с.в.)\ $N_{1}, N_{2}, \ldots$ и~$X_1, X_2, \ldots$, 
заданные на одном и~том же вероятностном пространстве 
$\left(\Omega,\mathbb{A},\mathbb{P}\right)$. В~статистике 
с.в.~$X_1, X_2, \ldots, X_n$ имеют смысл наблюдений, $n$~--- 
неслучайный объем выборки, а~с.в.~$N_n$~--- случайный объем выборки, зависящий 
от натурального параметра $n \hm\in \mathbb{N}$. Предположим, что при каждом 
$n \hm\ge 1$ с.в.~$N_n$ принимает только натуральные значения (т.\,е.\
$N_n \hm\in \mathbb{N}$) и~независима от последовательности с.в. $X_1, X_2, \ldots$

Обозначим через $T_n \hm\equiv T_n \left(X_1, \ldots, X_n\right)$ некоторую статистику.
 Для каждого $n \hm\ge 1$ определим с.в.~$T_{N_n}$, полагая
$$
T_{N_n} (\omega) \equiv T_{N_n(\omega)} \left(X_1 (\omega), \ldots, X_{N_n (\omega)} 
(\omega)\right),\omega \in \Omega\,,
$$
т.\,е.\ $T_{N_n}$~--- это статистика, построенная на основе статистики~$T_n$ 
по выборке случайного объема~$N_n$.

\smallskip

\noindent
\textbf{Определение~1.}\
Квантилью порядка $\alpha$ ($\alpha$-кван\-тилью) случайной величины~$X$ 
с~функцией распределения $F\left(x\right)\hm=\mathbb{P}\left(X<x\right)$ 
называется число~$x_{\alpha}$, такое что
$$ 
x_{\alpha} = \inf \left\{ x; F(x)>\alpha \right\}\,.
 $$

\section{Основные результаты}

Всюду ниже $X_1, X_2, \ldots $~--- независимые одинаково распределенные 
с.в.\ с~$\mathbb{E}\left(X_1\right)\hm=\mu$, 
$ 0\hm<\mathbb{D}\left(X_1\right)\hm=\sigma^{-2}$, 
$\mathbb{E}\left|X_1\right|^{3+2\delta}\hm < \infty$ для $\delta\hm \in 
\left(0,1/2\right)$ и~$\mathbb{E}\left(X_1\hm-\mu\right)^{3}\hm=\mu_3$. 
Для натурального~$n$ обозначим $T_n\hm=\left(X_1+\cdots+X_n\right)/n$.

В дальнейшем предполагаем, что с.в.~$X_1$ удовле\-тво\-ря\-ет условию Крамера:
$$
\limsup\limits_{|t|\rightarrow \infty}\left|\mathbb{E} e^{itX_1}\right|<1\,.
$$

\subsection{Распределение Стьюдента как~предельное}

Предположим, что с.в.~$N_n$ имеет отрицательное биномиальное распределение 
с~параметрами $p\hm=1/n$ и~$r\hm>0$, т.\,е.\ для $k \hm\in \mathbb{N}$
$$
\mathbb{P}(N_n=k)=\fr{\left(k+r-2\right)\cdots r}{(k-1)!}\,
\fr{1}{n^{r}}\left(1-\fr{1}{n}\right)^{k-1}.
$$

Пусть $G_f (x)$~--- функция распределения   Стьюдента с~параметром~$f$, 
соответствующая плотности вида:
$$
g_f (x)=\fr{\Gamma((f+1)/2)}{\sqrt{\pi f} \Gamma\left(f/2\right)} 
\left(1+\fr{x^{2}}{f}\right)^{-(f+1)/2}\,,\enskip x \in \mathbb{R}\,,
$$
где $\Gamma(\cdot)$~--- эйлерова гам\-ма-функ\-ция, а~$f\hm>0$~--- параметр
формы. Если параметр~$f$ натурален, то он называется числом степеней свободы.

\smallskip

\noindent
\textbf{Теорема~1.}\ 
\textit{Пусть $x\hm=x_\alpha$~--- $\alpha$-кван\-тиль 
нормированной статистики $\sigma \sqrt{r(n-1)+1}\left(T_{N_n}\hm-\mu\right)$,
$u\hm=u_\alpha$~--- $\alpha$-кван\-тиль распределения Стьюдента с~параметром~$2r$.
Тогда в~обозначениях, введенных выше, справедливо следующее асимптотическое разложение}:
$$
x = u-\fr{\mu_3\, \sigma^{3}\, \Gamma(r)}
{6\sqrt{n}}\, \fr{1+ru^{2}}{r-1/2} \left[
\fr{1+u^{2}/2r}{1+u^{2}/2}\right]^{r+1/2}+R\,,
$$
где 
$$
R =  \begin{cases}     
\mathcal{O}\left(\fr{1}{n}\right)\,, & r\ge \fr{2}{1+2\delta}\,; \\
   o\left(\fr{1}{\sqrt{n}}\right) \,, & \fr{1}{1+2\delta}<r<\fr{2}{1+2\delta}\,.
            \end{cases}
            $$

\noindent
\textbf{Замечание~1.}\
При $r\hm=1$ объем выборки~$N_n$ имеет геометрическое распределение. 
Асимптотическое разложение для квантилей в~этом случае принимает вид:
$$
x  = u-\fr{\mu_3 \sigma^{3}}{3}\left( 
1+u^{2}\right)\fr{1}{\sqrt{n}}+o\left(\fr{1}{\sqrt{n}}\right)\,.
$$


\subsection{Распределение Лапласа как~предельное}

Предположим, что с.в.\ $N_n \hm= N_n(s)$, где $s\hm\in\mathbb{N}$~--- 
фиксированный параметр, имеет распределение вида:
$$
\mathbb{P}(N_n(s)=k)=\left(\fr{k}{s+k}\right)^{n} - 
\left(\fr{k-1}{s+k-1}\right)^{n},\enskip k \in \mathbb{N}\,.
$$

Пусть  $\Lambda_{\theta}(x)$~--- функция распределения Лапласа 
с~параметром $\theta\hm>0$, соответствующая плотности вида:
$$ 
\lambda_{\theta}(x)=\fr{1}{\theta\sqrt{2}}e^{-{\sqrt{2}|x|}/{\theta}},\enskip 
x\in \mathbb{R}\,. 
$$


\noindent
\textbf{Теорема~2.}\ 
\textit{Пусть $x\hm=x_\alpha$ есть $\alpha$-кван\-тиль 
нормированной статистики $\sigma\sqrt{n}\left(T_{N_{n}(s)}\hm-\mu  \right)$, 
$u\hm=u_\alpha$~---\linebreak $\alpha$-кван\-тиль распределения Лапласа с~па\-ра\-мет\-ром~$1/s$.
Тогда в~обозначениях, введенных выше, справедливо следующее асимптотическое 
разложение}:
\begin{multline*}
%\label{eq:}
 x=u-\fr{\mu_{3}\sigma^{3}}{6}\left(\fr{|u|}{\sqrt{2s}} +\fr{1}{2s}-u^{2}\right)
 \fr{\lambda_{1/\sqrt{s}}(u)}{\lambda_{1/s}(u)} \fr{1}{\sqrt{n}} +{}\\
 {}+
 o\left(\fr{1}{\sqrt{n}}\right)\,.
 \end{multline*}

\noindent
\textbf{Замечание~2.}\
При $s\hm=1$ основной результат значительно упрощается:
$$x = u-\fr{\mu_{3}\sigma^{3}}{6\sqrt{n}}\left(
\fr{|u|}{\sqrt{2}} +\frac{1}{2}-u^{2}\right)  + o\left(\fr{1}{\sqrt{n}}\right).
$$

\noindent
\textbf{Замечание~3.}\
В~теоремах~1 и~2 построены приближения квантилей~$x$~нормированных 
статистик некоторыми функциями от квантилей~$u$ распределений 
Стьюдента и~Лапласа соответственно. При этом точность приближения  дана 
в~виде порядка по~$n$, в~частности как $o(1/\sqrt{n}).$ 
В~\cite{Arx2016} в~общей ситуации показано, что для ошибок приближения квантилей 
можно получать более информативные вы\-чис\-ли\-мые оценки при наличии таковых для 
ошибок приближения распределений  нормированных статистик.

\section{Вывод основных результатов}

\subsection{Вспомогательная лемма}

Пусть $F_{n}(x)$ есть последовательность функций распределения, каждая 
из которых допускает разложения типа Эдж\-вор\-да--Че\-бы\-шё\-ва 
по степеням $\varepsilon\hm=n^{-1/2}$ или $\varepsilon\hm=n^{-1}$ 
(см., например,~\cite{EncStat}):
\begin{align*}
F_n(x)&=G_{k,n}(x)+O(\varepsilon^k)\,; \\
G_{k,n}(x)&={}\\
&\hspace*{-5mm}{}= G(x)+\left\{\varepsilon a_{1}(x)+\cdots+\varepsilon^{k-1}a_{k-1}(x)\right\}
g(x)\,, 
\end{align*}
где $g(x)$~--- плотность распределения $G(x)$.

 В частности, для $k\hm=2$ имеем: 
\begin{equation}
\label{f1}
F_n(x)=G(x)+\varepsilon a_{1}(x)g(x) +\mathcal{O}\left(\varepsilon^{2}\right)\,.
\end{equation}

\noindent
\textbf{Лемма~1} (см., например, \cite{EncStat}).
\textit{В~сформулированных выше обозначениях имеет место следующее разложение}:  
\begin{equation}
\label{e3-u}
 x(u)=u+\varepsilon b_{1}(u) + \mathcal{O}(\varepsilon^{2}),
\end{equation}
\textit{где $x(u)$ и~$u$ суть квантили соответственно распределений $F_{n}$
и~$G$ одинакового порядка, т.\,е.\ $ F_n(x(u))\hm=G(u)$ и}
$b_{1}(u)=-a_{1}(u)$.\\

\subsection{Доказательство теоремы для~распределения Стьюдента как~предельного}

Для вывода необходимого асимптотического разложения квантилей 
распределения статистики, построенной по выборке случайного объема, 
воспользуемся результатом, полученным в~работе~\cite{BenKorGal}.

Введем следующее обозначение:
$$
f_r(x) \equiv \int\limits_{0}^{\infty} \phi\left(x\sqrt{y}\right) \fr{1-x^{2}y}{\sqrt{y}} \,dH_r(y)\,,
$$
где $H_r(x)$~--- функция гам\-ма-рас\-пре\-де\-ле\-ния с~параметром $r\hm>0$:
$$
H_r(x)=\fr{r^{r}}{\Gamma(r)} \int\limits_{0}^{x} e^{-ry} y^{r-1}\, dy, x \ge 0\,.
$$

В работе~\cite{BenKorGal} показано, что в~условиях теоремы для функции 
распределения нормированной статистики~$T_{N_n}$ справедлив следующий результат:
\begin{multline*}
\sup\limits_{x} \left| 
\vphantom{\fr{\mu_3 \sigma^{3}}{6\sqrt{r(n-1)+1}}}
\mathbb{P}(\sigma\sqrt{r(n-1)+1}(T_{N_n}-\mu)<x) -{} \right. \\
 \left. {}-G_{2r}(x)-\fr{\mu_3 \sigma^{3}}{6\sqrt{r(n-1)+1}}f_r(x)\right| =R\,,
\end{multline*}
где 
$$R = \begin{cases}
\mathcal{O}\left(\left(\fr{\log{n}}{n}\right)^{{1}/{2}+\delta}\right) , & r=1; \\[12pt]
\mathcal{O}\left(
\fr{1}{n^{\min(1, r({1}/{2}+\delta))}}\right), & r>1;\\[12pt]
 \mathcal{O}\left(\fr{1}{n^{r(1+2\delta)}}\right), & \fr{1}{1+2\delta}<r<1.
        \end{cases}
        $$
Отсюда при любом $x \hm\in \mathbb{R}$ имеем асимптотическое разложение вида:
 \begin{multline}
P\left(\sigma\sqrt{r(n-1)+1}(T_{N_n}-\mu)<x\right) ={}\\
{} = G_{2r}(x) -\fr{\mu_3 \sigma^{3}}{6\sqrt{r(n-1)+1}}f_r(x)+R_1\,,
\label{e4-u}
\end{multline}
где
$$ 
R_1 = \begin{cases}      \mathcal{O}\left(\fr{1}{n}\right)          , & r\ge \fr{2}{1+2\delta} ; \\
                            o\left(\fr{1}{\sqrt{n}}\right)             , & \fr{1}{1+2\delta}<r<\frac{2}{1+2\delta}.
            \end{cases}$$
Введем дополнительно следующие обозначения:
\begin{equation}
\left.
\begin{array}{c}
 F_n(x)                     \equiv \mathbb{P}(\sigma\sqrt{r(n-1)+1}(T_{N_n}-\mu)<x);    \\[6pt]
 G(x)                       \equiv G_{2r}(x);                                           \\[6pt]
 \varepsilon                \equiv n^{-{1}/{2}};                                    \\[6pt]
 \varepsilon a_1(x) g(x)    \equiv \fr{\mu_3 \sigma^{3}}{6\sqrt{r(n-1)+1}}\,f_r(x),
\end{array}
\right\}
\label{e5-u}
\end{equation}
где $g(x)$ есть плотность распределения~$G$.
Заметим теперь, что в~новых обозначениях разложение~(\ref{e4-u}) 
является схожим с~разложением типа Эджворта--Че\-бы\-шё\-ва из леммы~1. Поэтому, 
подставляя~(\ref{e3-u}) в~разложение~(\ref{e4-u}) 
согласно обозначениям~(\ref{e5-u}),  находим выражение для~$b_1$ 
и~приходим  к~искомому разложению типа Корниша--Фишера. 

Для упрощения вывода введем дополнительное обозначение, 
полагая $\theta \hm\equiv \mu_3 \sigma^{3}/6$. Имеем (ср.~(\ref{f1})):
\begin{multline*}
F_n(x)
   = G_{2r}(u)+g_{2r}(u) \fr{b_1}{\sqrt{n}}+
   \fr{\theta}{\sqrt{r(n-1)+1}} +{}\\
   {}+ f_r(u) + \fr{\theta}{\sqrt{r(n-1)+1}} \fr{1}{\sqrt{n}}\, f_{r}'(u) b_1(u) + 
\mathcal{O}\left(\fr{1}{n}\right) .
\end{multline*}
Заметим, что последнее слагаемое этого равенства является функцией 
порядка  $\mathcal{O}\left(1/n\right)$, а следовательно, 
и~$o\left(1/\sqrt{n}\right)$.

Таким образом, учитывая условие $F_n(x)\hm=G_{2r}(u)$, т.\,е.~$x$ и~$u$ 
суть квантили одного порядка, получаем следующее выражение для коэффициента~$b_1(u)$:
\begin{multline*}
 b_1(u)  =-\fr{\theta \sqrt{n}}{\sqrt{r(n-1)+1}}\,\fr{f_r(u)}{g_{2r}(u)}
     ={}\\
     {}= -\fr{\theta}{\sqrt{r}}\,\fr{f_r(u)}{g_{2r}(u)}+
     \mathcal{O}\left(\fr{1}{n}\right).
\end{multline*}
Для упрощения получившегося выражения воспользуемся следующей леммой.

\smallskip

\noindent
\textbf{Лемма~2.}\
\textit{Для функции $f_r(x)$, введенной выше, имеем}:
$$ 
f_r(x)=\fr{1}{\sqrt{2\pi}}\Gamma\!\left(r-\fr{1}{2}\right) \left[ 
1+rx^{2}\right] \left(1+\fr{x^{2}}{2} \right)^{-\left( r+1/2 \right)}\! .
$$



Для д\,о\,к\,а\,з\,а\,т\,е\,л\,ь\,с\,т\,в\,а\ леммы подставим функцию 
гам\-ма-рас\-пре\-\-\-деления $H_r (y)$ и~плотность стандартного 
нормального распределения  $\phi\left(x\sqrt{y}\right)$ 
в~выражение для функции $f_r(x)$:
\begin{multline*}
f_r(x)=\int\limits_{0}^{\infty} \phi\left(x\sqrt{y}\right) 
\fr{1-x^{2}y}{\sqrt{y}}\, dH_r(y) ={}\\
     {}=\fr{1}{\sqrt{2\pi}}\int\limits_{0}^{\infty} e^{-({x^{2}y})/{2}} 
     \fr{1-x^{2}y}{\sqrt{y}}\, y^{r-1} e^{-y}\, dy = {}\\
     {}=\fr{1}{\sqrt{2\pi}}\left(\int\limits_{0}^{\infty} e^{-y\left(1+
     {x^{2}}/{2}\right)} y^{\left(r-1/2\right)-1}dy \right. +{}\\
     \left. {}+ x^{2}\int\limits_{0}^{\infty} e^{-y\left(1+{x^{2}}/{2}\right)} 
     y^{\left(r+1/2\right)-1}\, dy \right)\,.
\end{multline*}
Рассмотрим вспомогательный интеграл:
\begin{equation*}
h(\alpha, p) \equiv\int\limits_{0}^{\infty}t^{\alpha-1}e^{-pt}\,dt=
%\fr{1}{p^{\alpha}}\int\limits_{0}^{\infty}t^{\alpha-1}e^{-pt}dpt
      \Gamma(\alpha)p^{-\alpha}\,.
\end{equation*}
Тогда $f_r(x)$ запишется в~виде:
\begin{multline*}
f_r(x)  =\fr{1}{\sqrt{2\pi}}  \left(h\left(r-\fr{1}{2}, 1+
\fr{x^{2}}{2}\right)+{}\right.\\
\left.{} +x^{2}h\left(r+\fr{1}{2}, 1+\fr{x^{2}}{2}\right)\right) =     {}\\
{}=\fr{1}{\sqrt{2\pi}} \left(1+\fr{x^{2}}{2}\right)^{-\left(r+{1}/{2}\right)}
  \!\left(\Gamma\left(r-\fr{1}{2}\right)\left(1+\fr{x^{2}}{2}\right)+{}\right.\\
  \left.{}+x^{2}\Gamma\left(
  r+\fr{1}{2}\right)\right) = {}\\
{}=\fr{1}{\sqrt{2\pi}}\left(1+\fr{x^{2}}{2}\right)^{-\left(r+{1}/{2}\right)}\Gamma
\left(r-\fr{1}{2}\right)\left(1+rx^{2}\right)\,.
\end{multline*}
Воспользуемся результатом леммы~2 и~получим
 утверждение теоремы.


Рассмотрим случай $r=1$. Тогда объем выборки~$N_n$ имеет
геометрическое распределение. Имеем:  $\delta \hm\in \left(0,1/2
\right)$; значит,   $2/(1\hm+2\delta) \hm\in \left(1,2\right)$.
Следовательно, остаточный член в~случае $r\hm=1$ имеет порядок
$o\left(1/\sqrt{n}\right)$. Найдем асимптотическое разложение для
квантилей в~этом случае, подставив значение параметра в~итоговое
выражение:
\begin{multline*}
x_{r=1}(u)={}\\
{}=u-\fr{\mu_3\, \sigma^{3}\, \Gamma(1)}{6\sqrt{n}}\,
\fr{1+1 u^{2}}{1-1/2}   \left[{\fr{1+u^{2}/2}{1+u^{2}/2}}\right]^{3/2}+{}\\
{}+o\left(\fr{1}{\sqrt{n}}\right) =
u-\fr{\mu_3 \sigma^{3}}{3}\left( 1+u^{2}\right)\fr{1}{\sqrt{n}}+
o\left(\fr{1}{\sqrt{n}}\right)\,.
\end{multline*}


\subsection{Доказательство теоремы для~распределения Лапласа как~предельного}

В работе~\cite{BenKorGal} показано, что в~условиях данной теоремы 
для функции распределения нормированной статистики $T_{N_{n}(s)}$ справедливо 
асимптотическое разложение вида:
\begin{multline}
\label{e6-u}
\max\limits_{x} \left|
\vphantom{\fr{\mu_{3}\sigma^{3}l_{s}(x)}{6\sqrt{n}}}
\mathbb{P}\left( \sigma\sqrt{n}(T_{N_{n}(s)}-\mu)<x\right)
-\Lambda_{1/s}(x) - {} \right.\\
\left.- \fr{\mu_{3}\sigma^{3}l_{s}(x)}{6\sqrt{n}}\right|= 
O\left(\fr{1}{n^{1/2+\delta}}\right)\,,\enskip n \rightarrow\infty\,,
\end{multline}
где функции $\Lambda_{1/s}(x)$ и~$l_{s}(x)$ определены следующим образом: 

\noindent 
\begin{align*}
 \Lambda_{1/s}(x) &= \int\limits^{\infty}\limits_{0}{\Phi\left(x\sqrt{y}\right)\,de^{-s/y}}\,, \\
 l_{s}(x)&=\int\limits^{\infty}\limits_{0}{\phi(x\sqrt{y})\fr{1-x^{2}y}{\sqrt{y}}\,de^{-s/y}}\,, 
 \end{align*}
а $\Phi(x)$ и~$\phi(x)$~--- функция распределения и~плотность стандартного нормального 
распределения соответственно.

Из формулы~(\ref{e6-u}) вытекает следующее соотношение:  
\begin{multline}
%\label{eq:}
\label{e7-u}
\mathbb{P}\left( \sigma\sqrt{n}(T_{N_{n}(s)}-\mu)<x\right)= 
\Lambda_{1/s}(x) + \fr{\mu_{3}\sigma^{3}l_{s}(x)}{6\sqrt{n}} + {}\\
{}+O\left(\fr{1}{n^{1/2+\delta}}\right),\enskip n \rightarrow\infty\,.
\end{multline}
Формула~(\ref{e7-u}) похожа на разложение типа Эджворда (формула~(\ref{f1})), 
однако  не соответствует ему в~точ\-ности.

Введем следующие обозначения: 
\begin{equation}
\left.
\begin{array}{rl}
 F_{n}(x)&\equiv\mathbb{P}\left( \sigma\sqrt{n}(T_{N_{n}(s)}-\mu)<x\right)\,;\\[6pt]
 G(x) &\equiv \Lambda_{1/s}(x)\,; \\[6pt]
 \gamma &\equiv \displaystyle\fr{\mu_{3}\sigma^{3}l_{s}(x)}{6}\,. 
\end{array}
\right\}
%\label{oboznacheniya}
\label{e8-u}
\end{equation}

Используем представление из~\cite{Prudnikov}  
\begin{multline}
\label{prudnikov}
 \int\limits^{\infty}_{0}x^{-n-{1}/{2}}e^{-px-{q}/{x}}\,dx = 
 \left(-1\right)^n \sqrt{\fr{\pi}{p}}\,\fr{\partial^n}{\partial q^n}
 \,e^{-2\sqrt{pq}}\,,\\
p>0\,,\enskip q>0\,,
\end{multline}
и рассмотрим $l_{s}(x)$: \\
\begin{multline*}
 l_{s}(x)= \int\limits^{\infty}_{0}\phi(x\sqrt{y})
 \fr{1-x^{2}y}{\sqrt{y}}\,de^{-s/y} ={}\\
{}= \fr{s}{\sqrt{2\pi}}\left( \int\limits^{\infty}_{0}y^{-{5}/{2}}
e^{-({x^{2}y})/{2}-{s}/{y}}\,dy - {}\right.\\
\left.{}-
x^2\int\limits^{\infty}_{0}y^{-{3}/{2}}e^{-({x^{2}y})/{2}-{s}/{y}}\,dy   \right)
=l_{s}^{*}(x)\,.
\end{multline*}

Возникают два случая.
\begin{enumerate}[1.]
\item Случай, когда $ x \hm\neq 0:$ 
\begin{multline*}
\hspace*{-1.14159pt} l_{s}^{*}(x)=\fr{2\pi s}{\sqrt{2\pi x^2}}  
 \left( \fr{x^{2}e^{-\sqrt{2sx^2}}}{2s}+\fr{x^{4}e^{-\sqrt{2sx^2}}}{2\sqrt{2}
 (sx^2)^{{3}/{2}}}-{}\right.\\
\hspace*{-16pt}\left. {}-x^{2}\fr{x^{2}e^{-\sqrt{2sx^2}}}{\sqrt{2sx^2}}\right) =
\lambda_{1/\sqrt{s}}(x)\left( \fr{\left|x\right|}{\sqrt{2s}}+\fr{1}{2s}-x^2\right).\hspace*{-1.23863pt} 
\end{multline*}
\item Случай, когда $x \hm= 0:$
\begin{multline*}
l_{s}^{*}(x)= \fr{s}{\sqrt{2\pi}} \int\limits^{\infty}_{0}
y^{-{5}/{2}}e^{-{s}/{y}}\,dy = {}\\
{}=\fr{s}{\sqrt{2\pi}}\,
\fr{\sqrt{\pi}}{2s^{{3}/{2}}} = \fr{1}{2\sqrt{2s}}\,.
\end{multline*}
\end{enumerate}
Поскольку тождество~(\ref{prudnikov}) применимо, только когда $x\hm>0$, проверим 
на непрерывность полученный результат:

\noindent
\begin{multline*}
 \lim\limits_{x\rightarrow 0-0}\lambda_{1/\sqrt{s}}(x)\left( \fr{\left|x\right|}
 {\sqrt{2s}}+\fr{1}{2s}-x^2\right) ={}\\
 {}= \fr{1}{2\sqrt{2s}}
  = l_s(0-0)\,;
  \end{multline*}
  
  \vspace*{-12pt}
  
  \noindent
  \begin{multline*}
\lim\limits_{x\rightarrow 0+0}\lambda_{1/\sqrt{s}}(x)\left( 
\fr{\left|x\right|}{\sqrt{2s}}+\fr{1}{2s}-x^2\right) = {}\\
{}=\fr{1}{2\sqrt{2s}}
  = l_s(0+0)\,. 
  \end{multline*}
Таким образом, в~силу~(\ref{e8-u}) исходное приближение принимает следующий вид: 

\begin{table*}[b]\small
\begin{center}

\begin{tabular}{|c|c|c|c|c|}
\multicolumn{5}{p{102mm}}{Значения метрик для приближения Лапласа 
и~Стьюдента на разных интервалах}  \\
\multicolumn{5}{c}{\ }\\[-6pt]
\hline
\multicolumn{1}{|c|}{\raisebox{-6pt}[0pt][0pt]{$n$}} & \multicolumn{2}{c|}{Приближение Лапласа} &  \multicolumn{2}{c|}{Приближение Стьюдента} \\
\cline{2-5}
&$M_{\mathrm{dif}}$, 10\,,000 & $L_1$, 10\,000 & $M_{\mathrm{dif}}$, 1000 & $L_1$, 1000\\
\hline
\hphantom{,99}0\%--99,99\% &  1,93 & 0,15& 3,00& 0,06\\
0\%--1\%\hphantom{9} &  1,51     & 0,65& 2,42& 0,83\\
0\%--5\%\hphantom{9} &  1,51     & 0,28& 2,42& 0,29\\
1\%--99\% & 0,26& 0,14& 1,16& 0,05\\
 5\%--95\%& 0,17& 0,14& 0,21& 0,04\\
\hline
\end{tabular}
\end{center}
%\begin{figure*} %fig1
\vspace*{6pt}
 \begin{center}
 \mbox{%
 \epsfxsize=162.484mm
 \epsfbox{ula-1.eps}
 }
\vspace*{3pt}

\noindent
{\small Графики квантилей для приближений Лапласа~(\textit{а}) и~Стьюдента~(\textit{б})}
 \end{center}
%\end{figure*}
\end{table*}

\noindent
\begin{multline}
%\label{perekhodOo}
\label{e10-u}
F_{n}(x)=G(x)+\fr{\gamma}{\sqrt{n}}\left( 
\fr{\left|x\right|}{\sqrt{2s}}+\fr{1}{2s}-x^2\right)
\lambda_{1/\sqrt{s}}(x)+{}\\
{}+O\left(\fr{1}{n^{1/2+\delta}}\right)\,,\enskip
 n\rightarrow\infty\,.
\end{multline}
 Поскольку
 
 \noindent
$$
O\left(\fr{1}{n^{1/2+\delta}}\right) = o\left(\fr{1}{\sqrt{n}}\right)\,,\enskip
  n\rightarrow\infty\,,
  $$
формула~(\ref{e10-u}) перепишется в~виде: 

\noindent
\begin{multline*}
F_{n}(x)=G(x)+\fr{\gamma}{\sqrt{n}}\left( 
\fr{\left|x\right|}{\sqrt{2s}}+\fr{1}{2s}-x^2\right)
\lambda_{1/\sqrt{s}}(x)+{}\\
{}+o\left(\fr{1}{\sqrt{n}}\right),\enskip
n\rightarrow\infty\,. 
\end{multline*}
Так как $s \hm\in \mathbb{N}$, то $\lambda_{1/\sqrt{s}}(x)\hm>0$ 
для любого $x\hm\in\mathbb{R}$. Тогда представим $\lambda_{1/\sqrt{s}}(x)\hm>0$ 
в~виде:  
$$ 
\lambda_{1/\sqrt{s}}(x) = 
\fr{\lambda_{1/\sqrt{s}}(x)}{\lambda_{1/s}(x)}\lambda_{1/s}(x)  
$$
и подставим в~предыдущую формулу: 

\noindent
\begin{multline*}
\!\!F_{n}(x)=G(x)+\fr{\gamma}{\sqrt{n}}\left( 
\fr{\left|x\right|}{\sqrt{2s}}+\fr{1}{2s}-x^2\right)
\fr{\lambda_{1/\sqrt{s}}(x)}{\lambda_{1/s}(x)} \times {}\\
{} \times\lambda_{1/s}(x)+o\left(\fr{1}{\sqrt{n}}\right),\enskip n\rightarrow\infty\,. 
\end{multline*}

\vspace*{-12pt}

\pagebreak

\noindent
Очевидно, что $O(\varepsilon^2)\hm=o(\varepsilon)$ при $\varepsilon\hm\rightarrow 0$, 
поэтому полученная формула в~точности совпадает 
с~формулой~(\ref{f1}), в~которой 

\noindent
\begin{align*}
a(x) &\equiv \gamma\left( \fr{\left|x\right|}{\sqrt{2s}}+
\fr{1}{2s}-x^2\right)\fr{\lambda_{1/\sqrt{s}}(x)}{\lambda_{1/s}(x)}\,;\\
 g(x) &\equiv \lambda_{1/s}(x)\,; \\
\varepsilon &\equiv \fr{1}{\sqrt{n}}\,. 
\end{align*}
Следовательно, формула~(\ref{e3-u}) перепишется в~виде:  

\noindent
\begin{multline*}
x(u)={}\\
{}=u-\fr{\mu_{3}\sigma^{3}}{6}\left(\fr{|u|}{\sqrt{2s}} +
\fr{1}{2s}-u^{2}\right)  
\fr{\lambda_{1/\sqrt{s}}(u)}{\lambda_{1/s}(u)}\, \fr{1}{\sqrt{n}} + {}\\
{}+
o\left(\fr{1}{\sqrt{n}}\right)\,.
\end{multline*}

\vspace*{-6pt}

\noindent
Теорема доказана.

\vspace*{-6pt}

\section{Вычислительный эксперимент}

%\vspace*{-6pt}

План эксперимента.
\begin{enumerate}[1.]
    \item   Задаются следующие параметры (сначала для случая, 
    когда распределение Стьюдента является предельным, затем в~скобках 
    для случая предельного распределения Лапласа): 

        $r =1$ $(s=1)$~--- характеризует распределение объема выборки; 

        $\chi^{2}_{4}$~--- задает распределение случайных величин $X_1,X_2,\ldots$; 

        $n = 1000 \:(n = 10\,000)$~--- параметр распределения объема выборки;

        $k = 10\,000$~--- число точек, в~которых рассчитываются эмпирическая 
        и~аппроксимирующая функции квантилей.
    \item   Производится расчет в~следующей последовательности: для каждой точки 
    эмпирической функции моделируется случайный объем~$N_n$, далее моделируется 
    вектор случайных величин $X_1,\ldots,X_{N_n}$, затем рассчитывается значение 
    статистики~$T_{N_n}$, после чего она нормируется;\linebreak отдельно для каж\-дой точ\-ки 
    ап\-прок\-си\-ми\-ру\-ющей функции моделируется значение соответствующей квантили 
    распределения Стью\-ден\-та (Лапласа), значение самой функции рассчитыва\-ется 
    согласно полученным выражениям в~теоремах~1 и~2.
    \item   Результаты даются в~виде таблицы и~графиков на рисунке. 
    В~таблице используются следу\-ющие расстояния  между эмпирической и~аппроксимирующей 
    функциями:
    
    \noindent
            $$
        M_{\mathrm{dif}}: \rho\left(f,g\right)=\max\limits_{1\leq i\leq k}
        \left|f(x_i)-g(x_i)\right|\,; 
        $$
        $$
        L_{1}:  \rho\left(f,g\right)=
        \fr{1}{k}\sum\limits_{i=1}^k\left|f(x_i)-g(x_i)\right|\,.
        $$
\end{enumerate}


Отметим, что для расчета даже одного эмпирического значения статистики  
необходимо для каж\-дой из $k\hm=10\,000$ точек моделировать~$N_n$~случайных 
величин $X_1,\ldots, X_{N_n}$, затем выполнить расчет статистики, что занимает 
несколько минут уже для $n\hm=1000$.

Таблица  иллюстрирует известный факт, что разложения Кор\-ни\-ша--Фи\-ше\-ра
дают хорошее приближение в~центральной зоне, т.\,е.\ для вероятностей
от~0,05 до~0,95. Качество приближения ухудшается для значений
вероятности около~0 и~1. О~том же говорит и~рисунок.

На рисунке приведены графики квантилей для приближения Лапласа 
с~параметрами $n\hm=10\,000$, $s\hm=1$~(\textit{а}) и~Стьюдента с~параметрами
$n\hm=1000$, $r\hm=1$~(\textit{б}). По оси~$OX$ откладываются значения
эмпирических квантилей, по оси $OY$~--- значения, даваемые
разложениями Кор\-ни\-ша--Фи\-шера.

\vspace*{-6pt}

{\small\frenchspacing
 {%\baselineskip=10.8pt
 \addcontentsline{toc}{section}{References}
 \begin{thebibliography}{9}
\bibitem{Example3}
\Au{Королев В.\,Ю.} Предельные распределения для случайно
индексированных последовательностей и~их применения.~--- М.: МГУ,
1993. 269~с.

\columnbreak

\bibitem{Example1}
\Au{Гнеденко Б.\,В.} Об оценке неизвестных параметров распределения
при случайном числе независимых наблюдений~// Тр. Тбилисского
мат. ин-та, 1989. Т.~92. С.~146--150.

\bibitem{BenKorGal}
\Au{Бенинг В.\,Е., Королев В.\,Ю., Галиева~Н.\,К.}   Асимптотические
разложения для функций распределения статистик, построенных по
выборкам случайного объема~// Информатика и~её применения, 2013.
Т.~7. Вып.~2. С.~75--83.

\bibitem{CornFish}
\Au{Cornish E.\,A., Fisher R.\,A.}  Moments and cumulants in the
specification of distributions~// Rev. Inst. Int. Statist.,
1937. Vol.~4. P.~307--320.

\bibitem{HillDavis}
\Au{Hill G.\,W., Davis A.\,W.} Generalized asymptotic expansions of
Cornish--Fisher type~// Ann. Math. Stat., 1968.
Vol.~39. P.~1264--1273.
\bibitem{Jashke}
\Au{Jaschke S.} The Cornish--Fisher expansion in the context of
delta-gamma-normal approximations~// J.~Risk, 2002. Vol.~4.
No.\,4. P.~33--52.

\bibitem{Arx2016}
\Au{Ulyanov V.\,V., Aoshima M., Fujikoshi~Y.}  Non-asymptotic
results for Cornish--Fisher expansions. Technical Report Hiroshima
Statistical Research Group. No.~16-03.~--- Hiroshima: Hiroshima
University, 2016. 8~p. 

\bibitem{EncStat}
\Au{Ulyanov V.\,V.}  Cornish--Fisher expansions~// International
encyclopedia of statistical science~/ Ed. M.~Lovric.~--- Berlin:
Springer, 2011. P.~312--315.

\bibitem{Prudnikov}
\Au{Прудников А.\,П., Брычков Ю.\,А., Маричев~О.\,И.}  Интегралы 
и~ряды. Элементарные функции.~--- М.: Наука, 1981. 344~с.
\end{thebibliography}

 }
 }

\end{multicols}

\vspace*{-6pt}

\hfill{\small\textit{Поступила в~редакцию 02.12.15}}

\vspace*{4pt}

%\newpage

%\vspace*{-24pt}

\hrule

\vspace*{2pt}

\hrule

\vspace*{-2pt}



\def\tit{GENERALIZED CORNISH--FISHER EXPANSIONS FOR~DISTRIBUTIONS 
OF~STATISTICS BASED~ON~SAMPLES~OF~RANDOM~SIZE}

\def\titkol{Generalized Cornish--Fisher expansions for~distributions 
of~statistics based on~samples of~random size}

\def\aut{A.\,S.~Markov, M.\,M.~Monakhov, and V.\,V.~Ulyanov}

\def\autkol{A.\,S.~Markov, M.\,M.~Monakhov, and V.\,V.~Ulyanov}

\titel{\tit}{\aut}{\autkol}{\titkol}

\vspace*{-9pt}

\noindent
Faculty of Computational Mathematics and Cybernetics,  
M.\,V.~Lomonosov Moscow State University, 1-52~Leninskiye Gory, GSP-1, Moscow 119991, 
Russian Federation

\def\leftfootline{\small{\textbf{\thepage}
\hfill INFORMATIKA I EE PRIMENENIYA~--- INFORMATICS AND
APPLICATIONS\ \ \ 2016\ \ \ volume~10\ \ \ issue\ 2}
}%
 \def\rightfootline{\small{INFORMATIKA I EE PRIMENENIYA~---
INFORMATICS AND APPLICATIONS\ \ \ 2016\ \ \ volume~10\ \ \ issue\ 2
\hfill \textbf{\thepage}}}

\vspace*{3pt}



\Abste{Generalized Cornish--Fisher expansions are constructed for 
quantiles of sample mean for a sample of random size in terms of 
quantiles for the Laplace distribution and Student's $t$-test. 
In recent years, the interest in Cornish--Fisher expansions grew significantly in 
the context of research on risk management. The widespread risk measure 
Value at Risk, or VaR, is, in fact, the quantile of the loss function. 
The authors use the general transfer theorem that makes it possible to obtain 
asymptotic expansions for the distribution functions of statistics based 
on samples of random size by asymptotic expansions for the distribution 
function of the random sample size and asymptotic expansions for the 
distribution functions of statistics based on nonrandom samples. 
A~computational experiment was performed to illustrate the obtained Cornish--Fisher 
expansions.}

\KWE{quantiles; generalized Cornish-Fisher expansions; random size sample; 
Laplace distribution}

\DOI{10.14357/19922264160210}

\vspace*{-12pt}

\Ack
\noindent
The work was supported by the Russian Science Foundation
 (project 14-11-00364).


%\vspace*{3pt}

  \begin{multicols}{2}

\renewcommand{\bibname}{\protect\rmfamily References}
%\renewcommand{\bibname}{\large\protect\rm References}

{\small\frenchspacing
 {%\baselineskip=10.8pt
 \addcontentsline{toc}{section}{References}
 \begin{thebibliography}{9}

\bibitem{1-ul}
\Aue{Korolev, V.\,Yu.} 1993. \textit{Predel'nye raspredeleniya 
dlya sluchayno indeksirovannykh posledovatel'nostey i~ikh primeneniya}
[Limit theorems for randomly indexed sequences 
and its applications]. Moscow: MSU. 269~p.

\bibitem{2-ul}
\Aue{Gnedenko, B.\,V.} 1989. Ob otsenke neizvestnykh parametrov 
raspredeleniya pri sluchaynom chisle nezavisimykh nablyudeniy 
[On estimation of unknown parameters of distributions from a~random number 
of independent observations]. \textit{Tr. Tbilisskogo mat. in-ta}
[Proceedings of Tbilisi Mathematical Institute] 92:146--150.

\bibitem{3-ul}
\Aue{Bening, V.\,E., V.\,Yu.~Korolev, and N.\,K.~Galieva}. 
2013. Asimptoticheskie razlozheniya dlya funktsiy raspredeleniya statistik, 
postroennykh po vyborkam sluchaynogo ob"ema 
[Asymptotic expansions for the distribution functions of statistics 
constructed from samples with random sizes]. 
\textit{Informatika i~ee Primeneniya}~--- \textit{Inform. Appl.} 7(2):75--83. 

\bibitem{4-ul}
\Aue{Cornish, E.\,A., and R.\,A.~Fisher}. 1937. Moments and cumulants 
in the specification of distributions. \textit{Rev. Inst. Int. Stat.} 4:307--320.

\bibitem{5-ul}
\Aue{Hill, G.\,W., and A.\,W.~Davis}. 1968. Generalized asymptotic expansions 
of Cornish--Fisher type. \textit{Ann. Math. Stat.} 39:1264--1273.

\bibitem{6-ul}
\Aue{Jaschke, S.} 2002. The Cornish--Fisher expansion in the context of 
delta-gamma-normal approximations. \textit{J.~Risk} 4(4):33--52.

\bibitem{7-ul}
\Aue{Ulyanov, V.\,V., M.~Aoshima, and Y.~Fujikoshi}. 2016. 
Non-asymptotic results for Cornish--Fisher expansions. 
Technical Report Hiroshima Statistical Research Group. No.\,16-03. 
Hiroshima: Hiroshima University. 8~p. 

\bibitem{8-ul}
\Aue{Ulyanov, V.\,V.} 2011. Cornish--Fisher expansions.
\textit{International encyclopedia of statistical science}.
Ed. M.~Lovric. Berlin: Springer. 312--315.

\bibitem{9-ul}
\Aue{Prudnikov, A.\,P., Yu.\,A.~Brychkov, and O.\,I.~Marichev}. 1981. 
\textit{Integraly i~ryady. Elementarnye funktsii} [Integrals and series. Elementary functions]. 
Moscow: Nauka. 344~p.

\end{thebibliography}

 }
 }

\end{multicols}

\vspace*{-3pt}

\hfill{\small\textit{Received December 2, 2015}}



\Contr

\noindent
\textbf{Markov Alexander S.} (b.\ 1993)~--- 
PhD student, Faculty of Computational Mathematics and Cybernetics,  
M.\,V.~Lomonosov Moscow State University, 1-52~Leninskiye Gory, GSP-1, Moscow 119991, 
Russian Federation; \mbox{markov.cmc@yandex.ru}

\vspace*{3pt}

\noindent
\textbf{Monakhov Mikhail M.} (b.\ 1993)~---
PhD student, Faculty of Computational Mathematics and Cybernetics,  
M.\,V.~Lomonosov Moscow State University, 1-52~Leninskiye Gory, GSP-1, Moscow 119991, 
Russian Federation; \mbox{mih\_monah@mail.ru}

\vspace*{3pt}

\noindent
\textbf{Ulyanov Vladimir V.} (b.\ 1953)~---
Doctor of Science in physics and mathematics, professor, Faculty 
of Computational Mathematics and Cybernetics,  M.\,V.~Lomonosov Moscow State University, 
1-52~Leninskiye Gory, GSP-1, Moscow 119991, Russian Federation; 
\mbox{vulyanov@cs.msu.ru}

\label{end\stat}


\renewcommand{\bibname}{\protect\rm Литература}