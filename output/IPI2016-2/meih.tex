\def\stat{meih}

\def\tit{СТАЦИОНАРНЫЕ ВЕРОЯТНОСТИ СОСТОЯНИЙ В~СИСТЕМЕ ОБСЛУЖИВАНИЯ КОНЕЧНОЙ ЕМКОСТИ 
С~ИНВЕРСИОННЫМ ПОРЯДКОМ
ОБСЛУЖИВАНИЯ И~ОБОБЩЕННЫМ ВЕРОЯТНОСТНЫМ
ПРИОРИТЕТОМ$^*$}

\def\titkol{Стационарные вероятности состояний в~системе обслуживания конечной емкости} 
%с~инверсионным порядком обслуживания и~обобщенным вероятностным приоритетом}

\def\aut{Л.\,А.~Мейханаджян$^1$}

\def\autkol{Л.\,А.~Мейханаджян}

\titel{\tit}{\aut}{\autkol}{\titkol}

\index{Мейханаджян Л.\,А.}
\index{Meykhanadzhyan L.\,A.}

{\renewcommand{\thefootnote}{\fnsymbol{footnote}} \footnotetext[1]
{Работа выполнена при поддержке РФФИ (проект 15-07-03007).}}


\renewcommand{\thefootnote}{\arabic{footnote}}
\footnotetext[1]{Российский университет дружбы народов, lameykhanadzhyan@gmail.com}


\Abst{Рассматривается система
$M/G/1/(r-1)$ с~дисциплиной инверсионного порядка обслуживания
и обобщенного вероятностного приоритета.
Предполагается, что в~момент поступления новой заявки в~систему
становится известной ее длина и,~кроме того, в~любой момент времени
известна остаточная длина каждой заявки в~системе.
В~момент поступления очередной заявки в~непустую систему ее
исходная длина сравнивается с~остаточной длиной заявки на приборе, и~в зависимости
от результатов сравнения наступает одно из следующих событий:
обе заявки покидают систему; только одна из заявок
покидает систему (другая остается на приборе);
обе заявки остаются в~системе (одна попадает на прибор, другая~--- в~очередь).
Заявки, оставшиеся в~системе, приобретают новую
(случайную) длину в~соответствии с~заданным распределением, зависящим в~общем случае от
исходных длин заявок.
Заявки, застающие систему полностью заполненной, теряются
и не оказывают на нее никакого воздействия.
В~статье предложены математические соотношения
для вычисления совместного стационарного распределения
числа заявок в~системе и~остаточного времени обслуживания заявки на приборе,
периода занятости системы, стационарного распределения
времени ожидания и~пребывания заявки длины~$x$ (в~терминах преобразования 
Лап\-ла\-са--Стил\-тье\-са (ПЛС)).}

\KW{система массового обслуживания; специальные
дисциплины; инверсионный порядок
обслуживания; вероятностный приоритет}

\DOI{10.14357/19922264160214} 

\vspace*{6pt}

\vskip 12pt plus 9pt minus 6pt

\thispagestyle{headings}

\begin{multicols}{2}

\label{st\stat}

\section{Введение}

В этой работе, являющейся продолжением работ~\cite{n1, n2}, будет
рассматриваться та же однолинейная система
массового обслуживания (СМО), что и~в~\cite{n1}, но ограниченной емкости.
Основной результат работ~\cite{n1, n2} состоит в~нахождении
совместного стационарного распределения вероятностей состояний %\linebreak 
системы
$M/G/1$ с~дисциплиной инверсионного %\linebreak 
порядка
обслуживания и~обобщенного вероятностного приоритета, а~также основных
стационарных вероятностных характеристик в~терминах ПЛС. %\linebreak
Сейчас же задача заключается в~исследовании стационарных 
ве\-ро\-ят\-ност\-но-вре\-мен\-н$\acute{\mbox{ы}}$х характеристик указанной системы
в~случае, когда присутствует ограничение на размер очереди.

\vspace*{-4pt}

\section{Описание системы}

Рассмотрим СМО
с~одним прибором,
одной очередью для ожидающих заявок емкости $(r\hm-1)\hm<\infty$, $r \hm\ge 2$,
и~входящим потоком заявок, который для простоты будем называть здесь
потоком пуассоновского типа. Отличие этого потока от пуассоновского
заключается в~следующем: интенсивность поступления заявок равна~$\lambda$,
если на приборе имеется заявка, и~$\tl$, если система пуста.


Если в~момент поступления заявки в~систему
на приборе имеется заявка, то исходное распределение времени обслуживания поступающей
заявки является произвольным с~функцией распределения (ФР) $B(x)$.
Если же заявка поступает в~систему в~тот момент, когда система пуста, то исходное
распределение времени обслуживания поступающей
заявки является произвольным с~ФР~$\tB(x)$.

Далее для простоты изложения будем считать, что ФР $B(x)$ и~$\tB(x)$ имеют непрерывные
ограниченные плотности распределения $b(x)\hm=B'(x)$ и~$\tb(x)\hm=\tB'(x)$,
причем $\tb \hm= \int_0^\infty x \tb(x)\,dx \hm< \infty$
и~$b \hm= \int_0^\infty x b(x)\,dx \hm< \infty$.

Обобщенный инверсионный порядок обслуживания с~вероятностным приоритетом (LCFS BPP)
заключается в~следующем.
Предполагается, что в~любой момент времени известна остаточная длина (далее будем говорить
просто длина) каждой заявки в~системе.
В~момент поступления в~систему новой заявки ее
исходная длина~$u$ сравнивается с~(остаточной) длиной~$v$ заявки на приборе.
С~вероятностью~$D(x,y|u,v)$,
зависящей только от~$u$ и~$v$, обслуживавшаяся ранее заявка продолжает обслуживаться, причем
ее длина становится меньше~$y$, а~вновь
поступившая становится на первое место в~очереди и~ее длина становится меньше~$x$.
Кроме того, с~вероятностью~$D^*(x,y|u,v)$,
зависящей только от~$u$ и~$v$, вновь поступившая заявка занимает прибор, вытесняя обслуживавшуюся
ранее на первое место в~очереди, причем длина заявки, бывшей ранее на приборе, становится
меньше~$y$, а~вновь поступившей~--- меньше~$x$.

Если на приборе находится заявка остаточной длины~$v$ и~в~систему поступает заявка
длины~$u$, то с~вероятностью $D_0(x|u,v)$ заявка, находящаяся на приборе, покидает
систему, а~поступившая заявка становится на
прибор, причем ее длина становится меньше~$x$.
Кроме того, с~вероятностью
$D_0^*(y|u,v)$ поступившая заявка сразу же покидает систему, а~заявка, находящаяся на
приборе, продолжает обслуживаться, причем ее длина становится меньше~$y$.
Введем также обозначение:
\begin{equation*}
%\label{(2.1)}
D(x|u,v) = D_0(x|u,v) + D_0^*(x|u,v)\,.
\end{equation*}
Здесь $D(x|u,v)$~--- вероятность того, что одна из двух заявок покинет систему, а~вторая встанет
на прибор и~примет длину меньше~$x$.

Наконец, предполагается, что с~вероят\-ностью~$d_0(u,v)$ обе заявки покидают
систему, а~на прибор становится первая заявка из очереди.

Будем считать для удобства изложения, что все ФР 
$D(x,y|u,v)$, $D^*(x,y|u,v)$, $D_0(x|u,v)$,
$D_0^*(y|u,v)$, $D(y|u,v)$ и~$D_0(u,v)$
имеют непрерывные ограниченные плотности
$d(x,y|u,v)\hm=\partial^2 D(x,y|u,v)/(\partial x \partial y)$,
$d^*(x,y|u,v)\hm=\partial^2 D^*(x,y|u,v)/(\partial x \partial y)$,
$d_0(x|u,v)\hm= \partial D_0(x|u,v)/\partial x$,
$d_0^*(y|u,v)=\partial D_0^*(y|u,v)/\partial y$
и~$d(x|u,v)\hm=\partial D(x|u,v)/\partial x$.


Естественно, для любых~$u$ и~$v$ выполнено условие:
\begin{multline*}
%\label{e2.1-m}
\int\limits_0^\infty \int\limits_0^\infty
\left[d(x,y|u,v) + d^*(x,y|u,v)\right]\,dxdy+{}\\
{}+ \int\limits_0^\infty d(x|u,v) \,dx
+ d_0(u,v) =1\,.
\end{multline*}

Если длина заявки на приборе становится
равной нулю, то она мгновенно покидает систему и~на прибор переходит первая
заявка из очереди. Остальная очередь сдвигается на единицу.


Для конечного накопителя необходимо также задать
дисциплину принятия заявок в~систему при отсутствии в~нем свободных мест.
Здесь для простоты изложения будет рассмотрен только
тот случай, когда поступающая в~заполненную систему заявка теряется.
Заметим, что в~этом случае принятая в~систему заявка
будет обязательно обслужена полностью.
Для всех СМО с~такой дисциплиной принятия заявок в~систему
при отсутствии в~накопителе свободных мест стационарные
вероятности\linebreak $p_n(x_1,\ldots,x_n)$ при $n\hm<r$ совпадают
с~точностью до\linebreak постоянной с~аналогичными вероятностями
для системы с~бесконечным накопителем, различие заключается
только в~вероятностях $p_{r}(x_1,\ldots,x_{r})$.
Однако несколько более сложно вычисляются стационарные
распределения, связанные с~временем пребывания заявки 
в~системе, поскольку даже заявки, принятые в~систему, могут покидать
ее недообслуженными.

Далее будем предполагать, что система
функционирует в~стационарном режиме
и~$\tb \hm= \int_0^\infty x\tb(x)\,dx \hm< \infty$
и~$b \hm= \int_0^\infty x b(x)\,dx \hm< \infty$.
Отметим, что параметр $\rho \hm= \lambda b$ для данной системы не
является загрузкой в~традиционном смысле и~может существенно от нее отличаться.


\section{Стационарные вероятностные характеристики}

Обозначим через $\nu(t)$ число заявок в~системе
в~момент~$t$, а~через $\vec\xi(t)\hm =(\xi_{1}(t),\ldots,\xi_{\nu(t)}(t))$~---
вектор, координатой $\xi_{1}(t)$ которого
является (остаточное) время обслуживания
заявки, находящейся в~этот момент на приборе,
$\xi_{2}(t)$~--- первой заявки в~очереди$,\ldots,$ $\xi_{\nu(t)-1}(t)$~---
последней, \mbox{$(\nu(t)-1)$-й} заявки в~очереди.
При $\nu(t)\hm=0$ вектор $\vec\xi(t)$ не определяется.
Тогда $\eta(t)\hm=(\nu(t),\vec\xi(t))$ представляет
собой марковский процесс, описывающий поведение числа заявок в~рассматриваемой системе.

Положим 
\begin{align*}
p_{0}(t)&= \mathbf{P}\{\nu(t)=0\}\,;
\\
P_{n}\left(t;x_1,\ldots,x_{n}\right) &=
\mathbf{P}\{\nu(t)=n\,,\\
&\hspace*{-20pt}\xi_{1}(t)<x_{1},\ldots,\xi_{n}(t)<x_{n}\}
\,,\enskip 1 \le n \le r\,.
\end{align*}
Обозначим через
\begin{equation*}
%\label{(2.1)}
p_{0} = \lim\limits_{t\to\infty}
p_{0}(t) \,;
\end{equation*}
\begin{equation*}
%\label{(2.1)}
P_{n}(x_1,\ldots,x_{n}) = \lim\limits_{t\to\infty}
P_{n}(t;x_1,\ldots,x_{n}) \,,\enskip 1 \le n \le r\,,
\end{equation*}
стационарное распределение процесса $\eta(t)$.
В~силу сделанных в~предыдущем пункте
предположений относительно параметров системы,
можно показать (см., например,~[3; 4, с.~273]), что существуют
непрерывные и~ограниченные плотности 

\noindent
\begin{multline*}
p_n(x_1,\ldots,x_{n}) = \fr{\partial^n }{\partial x_1\cdots \partial x_n}
P_n(x_1,\ldots,x_{n}) \,,\\[-1pt]
1 \le n \le r\,.
\end{multline*}

Выпишем систему интегродифференциальных
уравнений, которой удовлетворяют стационарные
плотности $p_n(x_1,\ldots,x_{n})$ и~которую
для краткости по аналогии с~простейшими СМО будем называть системой уравнений
равновесия (СУР). Для этого рассмотрим вспомогательную систему
с~$(n\hm-1)$ мес\-та\-ми ожидания, отличающуюся от исходной
сис\-те\-мы только тем, что если в~очереди
находится $(n\hm-1)$ заявок, заявка на приборе имеет
остаточную длину~$v$ и~поступает новая заявка
длины~$u$, то с~ве\-ро\-ят\-ностью $d(x,y|u,v)$ на
приборе остается вновь поступившая заявка,
длина которой становится равной~$x$, а~обслуживавшаяся ранее заявка покидает
сис\-те\-му, и~наоборот: с~вероятностью $d^*(y,x|u,v)$ систему покидает вновь
поступившая заявка, а~находившаяся ранее на приборе заявка продолжает обслуживаться, но
ее длина становится равной~$x$.

В силу метода исключения состояний~\cite{ppav}
стационарные вероятности состояний в~исходной
и~вспомогательной системах отличаются лишь на
постоянный множитель (за исключением вероятности $p_{r}(x_1,\ldots,x_{r})$).
Это дает возможность при составлении
уравнений для $p_n(x_1,\ldots,x_{n})$, $n\hm\ge 1$,
воспользоваться вспомогательной системой и~получить следующие соотношения:

\noindent
\begin{multline}
\label{e3.1-mei}
-p'_1(x) = \tl \tb(x) p_0 - \lambda p_1(x)
+ {}\\[-1pt]
{}+\lambda \Bigg( \int\limits_0^\infty \int\limits_0^\infty
d(x|u,v) b(u) p_1(v) \,dudv +{}
\\[-1pt]
{}+ \int\limits_0^\infty \int\limits_0^\infty
\int\limits_0^\infty \left[d(x,y|u,v) b(u) p_1(v)  +{}\right.\\[-1pt]
\left.{}+
d^*(y,x|u,v) b(u) p_1(v)\right] \,dydudv
\Bigg)\,;
\end{multline}

\vspace*{-16pt}

\noindent
\begin{multline*}
-p'_{n}\left(x_1,\ld,x_n\right) ={}
\\
{}=
\lambda \Bigg(
\int\limits_0^\infty \int\limits_0^\infty
\left[d\left(x_2,x_1|u,v\right) b(u) p_{n-1}\left(v,x_3\ld,x_n\right)
+ {}\right.\\
\left.{}+
d^*\left(x_1,x_2|u,v\right) b(u) p_{n-1}\left(v,x_3,\ld,x_n\right)\right]
\,dudv \Bigg)
-{}\\
{}-
\lambda p_{n}\left(x_1,\ld,x_n\right)
+{}\\
{}+ \lambda \Bigg(
\int\limits_0^\infty \int\limits_0^\infty
d\left(x_1|u,v\right) b(u) p_{n}\left(v,x_2,\ld,x_n\right)
\,dudv +{}
\end{multline*}

\noindent
\begin{multline}
{}+
\int\limits_0^\infty \int\limits_0^\infty
\int\limits_0^\infty \left[d\left(x_1,y|u,v\right) b(u) p_{n}\left(v,x_2,\ld,x_n\right)
+{}\right.
\\
\left.{}+
d^*\left(y,x_1|u,v\right) b(u) p_{n}\left(v,x_2,\ld,x_n\right)\right]
\,dy du dv \Bigg)\,,
\\
 1 \le n \le r-1\,;
 \label{e3.2-mei}
\end{multline}

\vspace*{-12pt}

\noindent
\begin{multline}
\label{e3.3-mei}
-p'_{r}\left(x_1,\ld,x_n\right) ={}\\
{}=
\lambda \Bigg(
\int\limits_0^\infty \int\limits_0^\infty
\left[d\left(x_2,x_1|u,v\right) b(u) p_{n-1}\left(v,x_3\ld,x_n\right)
+{}\right.
\\
\!\!\!\!\left.{}+
d^*\left(x_1,x_2|u,v\right) b(u) p_{n-1}\left(v,x_3,\ld,x_n\right)\right]
\,du dv \!\Bigg).\!\!
\end{multline}

Остановимся подробнее на выводе уравнения
для плотности $p_{r}(x_1,\ldots,x_{r})$ (остальные уравнения
получаются так же, как и~в случае накопителя бесконечной емкости~\cite{n1}).
Рассмотрим моменты времени~$t$ и~$(t\hm+\Delta)$.
Тогда для того, чтобы в~момент времени
$(t\hm+\Delta)$ в~системе находилось~$r$~заявок, причем
на приборе заявка длины~$x_1$, а~в~очереди
заявки длин $x_2,\ldots,x_r$, нужно, чтобы произошло одно из следующих событий:
\begin{itemize}
\item в~момент~$t$ в~системе находилось $(r-1)$
заявок, причем заявка на приборе имела
длину~$v$, первая заявка в~очереди имела
длину $x_3,\ldots,$ последняя заявка в~очереди
имела\linebreak
 длину~$x_n$ (с~плотностью вероятностей $p_{r-1}(t;v,x_3,\ldots,x_r)$),
и~за время~$\Delta$ поступила заявка (с~вероятностью $\lambda\Delta$) длины~$u$
(с~плотностью вероятностей $b(u)$).
Заявка на приборе продолжает обслуживаться,
но ее длина становится равной~$x_1$, а~вновь
поступившая заявка занимает первое мес\-то в~очереди и~ее длина становится равной~$x_2$
(с~плот\-ностью вероятностей $d(x_2,x_1|u,v)$);
\item
в момент~$t$ в~системе находилось $(r\hm-1)$
заявок, причем заявка на приборе имела
длину~$v$, первая заявка в~очереди имела
длину $x_3,\ldots,$ последняя заявка в~очереди имела\linebreak
 длину~$x_n$ (с~плот\-ностью
вероятностей $p_{r-1}(t;v,x_3,\ldots,x_r)$),
и~за время~$\Delta$ поступила заявка (с~вероятностью $\lambda\Delta$) длины~$u$
(с~плот\-ностью вероятностей $b(u)$).
Поступившая заявка занимает прибор и~ее длина
становится равной~$x_1$, а~заявка,
обслуживавшаяся до поступления новой заявки,
занимает первое мес\-то в~очереди и~ее длина
становится равной~$x_2$ (с~плотностью вероятностей $d^*(x_1,x_2|u,v)$);
\item
в момент~$t$ в~системе находилось~$r$~заявок,
причем заявка на приборе имела длину $x_1\hm+\Delta$, первая заявка в~очереди
имела дли-\linebreak\vspace*{-12pt}

\pagebreak

\noindent
ну $x_2,\ldots,$ последняя заявка в~очереди имела длину~$x_r$ (с плотностью
вероятностей $p_r(t;x_1\hm+\Delta,x_2,\ldots,x_r)$).
\end{itemize}


Вероятности других событий равны $o(\Delta)$.
Применяя формулу полной вероятности, имеем:
\begin{multline*}
p_{r}\left(t+\Delta;x_1,\ld,x_r\right) ={}\\
\!{}=\!
\lambda\Delta \Bigg(\!
\int\limits_0^\infty \!\int\limits_0^\infty\!
\left[d\left(x_2,x_1|u,v\right) b(u) p_{r-1}\left(t;v,x_3,\ld,x_r\right)
+{}\right.\hspace*{-3.62766pt}
\\
\left.{}+
d^*\left(x_1,x_2|u,v\right) b(u) p_{r-1}\left(t;v,x_3,\ld,x_r\right)\right]
\,du dv \Bigg)
+ {}\\
{}+p_{r}\left(t;x_1+\Delta,x_2,\ld,x_r\right)\,,
\end{multline*}
откуда, перенося слагаемое
$p_r(t;x_1+\Delta,x_2,\ldots,x_{r})$ в~левую часть равенства, деля на~$\Delta$,
устремляя~$\Delta$ к~нулю и~учитывая стационарный режим функционирования системы,
получаем уравнение~\eqref{e3.3-mei}.


К системе уравнений~\eqref{e3.1-mei}--\eqref{e3.3-mei} 
нужно добавить начальные условия, которые удобно записать\linebreak \mbox{в~виде}:
\begin{align}
p_{1}(\infty) &= \lim\limits_{X\to \infty} p_{1}(X)
= 0\,; \label{e3.33-mei}
\\
p_{n}(\infty,x_2,\ld,x_r)
&= {}\notag\\
&\hspace*{-20mm}{}=\lim\limits_{X\to \infty} p_{n}\left(X,x_2,\ld,x_r\right)
= 0\,,\enskip
1 \le n \le r\,.
\label{e3.4-mei}
\end{align}
Как получаются соотношения~\eqref{e3.33-mei} и~\eqref{e3.4-mei},
показано в~\cite{n1}.
Оставшаяся неизвестной стационарная вероятность~$p_0$ отсутствия заявок в~системе
находится, как обычно, из условия нормировки:
\begin{equation}
\label{e3.6-mei}
\sum\limits_{n=0}^r p_n = 1\,, 
\end{equation}
где
$p_n=P_n(\infty,\ld,\infty)$, $1 \hm\le n\hm \le r$,~---
стационарная вероятность наличия в~системе~$n$~заявок.

Как и~в случае системы бесконечной емкости,
полученные соотношения~\eqref{e3.1-mei}--\eqref{e3.6-mei} позволяют
теоретически последовательно по~$n$
находить стационарные плотности вероятностей $p_n(x_1,\ldots,x_{n})$.
Однако на практике такие расчеты связаны с~серьезными вычислительными сложностями.


Как показано, например, в~\cite{n3}, в~практических случаях
бывает достаточно знать только маргинальные плотности
\begin{multline*}
%\label{(2.1)}
p_{n}(x) = \mathop{\int\cd\int}\limits_{x_2,\ld,x_n>0}
p_{n}\left(x,x_2\ld,x_n\right)\,dx_2\cdots dx_n\,,
\\ 
2 \le n \le r\,.
\end{multline*}

Интегрируя~\eqref{e3.2-mei} и~\eqref{e3.3-mei} по
$x_2,\ldots ,x_r$ в~пределах от нуля до бесконечности и~вспоминая равенство~\eqref{e3.1-mei}, 
получаем следующую систему интегродифференциальных уравнений
для $p_{n}(x)$, $1 \hm\le n \hm\le r$:
\begin{align}
-p'_{n}(x) &= a_n(x) - \lambda p_{n}(x) +
\int\limits_0^\infty K_n(x,v) p_{n}(v)\,dv \,,\notag\\ 
&\hspace*{35mm}1  \le n \le r-1\,; \label{e3.7-mei}\\
-p'_{r}(x) &= a_r(x)\,,  \label{e3.7-1-mei}
\end{align}
где $a_1(x)=\tl \tb(x) p_0$ и
\begin{multline*}
a_{n}(x) = \lambda \Bigg( \!\int\limits_0^\infty\!
p_{n-1}(v)\,dv \! \int\limits_0^\infty\!
b(u)\,du \!\int\limits_0^\infty\!
\left[d(y,x|u,v) +{}\right.\\
\left.{}+ d^*(x,y|u,v)\right] \,dy
\Bigg)\,,\enskip 1 \le n \le r\,;
\end{multline*}

\vspace*{-12pt}

\noindent
\begin{multline*}
%\label{(2.1)}
K_n(x,v) = \lambda \int\limits_0^\infty b(u)\,du
\Bigg( d(x|u,v) +{}\\
{}+
\int\limits_0^\infty \!\left[d(x,y|u,v) + d^*(y,x|u,v)\right]
\,dy\! \Bigg),
\enskip 1 \le n \le r-1.
\end{multline*}
Начальные условия для уравнений~\eqref{e3.7-mei} и~\eqref{e3.7-1-mei}
по аналогии с~\eqref{e3.33-mei} запишем в~виде:
\begin{equation}
\label{e3.8-mei}
p_{n}(\infty) = \lim\limits_{X\to \infty} p_{n}(X)
= 0 \,,\enskip 1 \le n \le r\,. 
\end{equation}


Решать систему~\eqref{e3.7-mei} и~\eqref{e3.7-1-mei}
с~начальными условиями~\eqref{e3.8-mei} можно различными способами.
Воспользуемся методом, предложенным в~\cite{n1}.
Прежде всего заметим, что из~\eqref{e3.7-1-mei} немедленно следует, что
\begin{equation*}
%\label{(3.7-1)}
p_r(x) = \int\limits_x^\infty a_r(u) \,du\,.
\end{equation*}
Решение уравнений~\eqref{e3.7-mei} будем искать в~виде:
\begin{equation}
\label{e4.1-mei}
p_n(x) = e^{\lambda x} q_n(x)\,,\enskip 1 \le n \le r-1\,.
\end{equation}
Подставляя в~\eqref{e3.7-mei} вместо $p_n(x)$ ее выражение
по формуле~\eqref{e4.1-mei}, получаем новое интегродифференциальное уравнение:
\begin{multline*}
- q'_n(x) = e^{-\lambda x} a_n(x) +
\int\limits_0^\infty e^{\lambda v} e^{-\lambda x} K_n(x,v) q_n(v)\, dv\,, \\
1 \le n \le r-1\,.
\end{multline*}
Интегрируя последнее равенство по~$x$ в~пределах от~$y$ до~$\infty$ и~учитывая
начальное условие~\eqref{e3.8-mei}, получаем
интегральное уравнение Фредгольма 2-го рода:
\begin{multline}
\label{e2.1n-mei}
q_n(y)= b_n(y) + \int\limits_0^\infty
G_n(y,v) q_n(v)\, dv \,, \\ 
1 \le n \le r-1\,,
\end{multline}
где
\begin{align*}
b_n(y) &= \int\limits_y^\infty e^{-\lambda x} a_n(x)\, dx\,; \\
G_n(y,v) &= \int\limits_y^\infty e^{\lambda (v-x)} K_n(x,v)\, dx\,.                              %       (4.2)
\end{align*}
Отметим, что свободный член $b_n(y)$ и~ядро
$G_n(y,v)$ интегрального уравнения являются неотрицательными функциями.
Далее для расчета $q_n(y)$ можно применить
подходящий метод решения интегральных уравнений Фредгольма 2-го рода
(см., например,~[6--8]).

В~некоторых частных случаях решения уравнений~\eqref{e2.1n-mei}  могут быть выписаны в~явном виде.
 Например, это возможно в~случае, когда  известны сепарабельные аппроксимации для функций
 $d(x,y|u,v)$, $d^*(x,y|u,v)$, $d_0(x|u,v)$ и~$d_0^*(x|u,v)$,
 т.\,е.\ разложения вида:
\begin{align*}
d(x,y|u,v)&=\sum\limits_{i=1}^{N_1} \alpha_{1i}(x)\beta_{1i}(y)\gamma_{1i}(u)\delta_{1i}(v)\,;
\\
d^*(x,y|u,v)&=\sum\limits_{i=1}^{N_2} \alpha_{2i}(x)\beta_{2i}(y)\gamma_{2i}(u)\delta_{2i}(v)\,;
\\
d_0(x|u,v)&=\sum\limits_{i=1}^{N_3} \alpha_{3i}(x)\gamma_{3i}(u)\delta_{3i}(v)\,;
\\
d_0^*(x|u,v)&=\sum\limits_{i=1}^{N_4} \alpha_{4i}(x)\gamma_{4i}(u)\delta_{4i}(v)\,,
\end{align*}
где $N_1$, $N_2$, $N_3$ и~$N_4$~--- некоторые натуральные чис\-ла,
а $\alpha_{ij}(x)$, $\beta_{ij}(x)$, $\gamma_{ij}(x)$ и~$\delta_{ij}(x)$~--- некоторые
известные функции.
Тогда решение уравнения~\eqref{e2.1n-mei} при фиксированном~$n$
сводится к~решению системы линейных уравнений относительно
$(N_1\hm+N_2\hm+N_3\hm+N_4)$ неизвестных.


\section{Стационарные временные характеристики}

\subsection{Стационарное распределение времени ожидания начала обслуживания}

Для того чтобы найти показатели функционирования СМО,
связанные с~временем пребывания в~системе, нужно прежде
всего найти ПЛС периода занятости (ПЗ) системы.

Обозначим через $u_n(s;x)$, $1 \hm\le n \hm\le r$,  ПЛС
времени до того момента, когда в~системе впервые останется $(n\hm-1)$ заявок,
при условии что на приборе начала обслуживаться заявка
(остаточной) длины~$x$ и~в~системе находится~$n$~заявок.

Учитывая, что по принятому соглашению поступающая в~заполненную
систему заявка сразу теряется, ПЛС  $u_r(s;x)$ удовлетворяет уравнению:
\begin{equation}
\label{t1-mei}
u_r(s;x)=e^{-s x}\,.
\end{equation}
Воспользовавшись свойствами ПЛС, получаем, что $u_{n}(s;x)$ равно:
\begin{itemize}
\item $e^{-s x}$, если до момента времени~$x$
окончания обслуживания заявки на приборе
новая заявка не поступила (с~вероятностью $e^{-\lambda x}$);

\item  $e^{-s t}$, если в~момент $0<t<x$
поступила новая заявка и~обе заявки покинули
систему (с плотностью
вероятностей
$ \lambda e^{-\lambda t}
\int\nolimits_0^\infty d_0(y,x\hm-t)b(y)\,dy$);

\item  $e^{-s t} u_{n}(s;v)$, если в~момент времени
$0\hm<t\hm<x$ поступила новая заявка длины~$y$, одна из двух
заявок (поступившая заявка или
заявка на приборе) покинула систему, а~оставшаяся приняла длину~$v$ и,~значит,
время до того момента, как в~системе останется $(n\hm-1)$ заявок,
равно $u_{n}(s;v)$
(плотность вероятности данного события равна
$ \lambda e^{-\lambda t}
\int\nolimits_0^\infty d(v|y,x-t) b(y)\, dy$);

\item  $e^{-s t} u_{n+1}(s;w) u_{n}(s;v)$, если в~момент
времени $0\hm<t\hm<x$ поступила новая заявка длины~$y$,
обе заявки остаются в~системе (новая встает в~очередь),
причем длина новой заявки становится равной~$v$,
а~на приборе --- $w$ (с~плот\-ностью вероятностей
$
\lambda e^{-\lambda t} \int\nolimits_0^\infty
d(v,w|y,x-t) b(y)\, dy$);

\item  $e^{-s t} u_{n+1}(s;v) u_{n}(s;w)$, если в~момент
времени $0\hm<t\hm<x$ поступила новая заявка длины~$y$,
обе заявки остаются в~системе (новая встает в~очередь),
причем длина новой заявки становится равной~$w$,
а~на приборе --- $v$ (с~плот\-ностью вероятностей
$\lambda e^{-\lambda t} \int\nolimits_0^\infty d^*(v,w|y,x-t)\, b(y)\, dy$).
\end{itemize}


По формуле полной вероятности окончательно получаем:
\begin{multline*}
u_{n}(s;x)=e^{-(s+\lambda) x} + {}\\
{}+\int\limits_0^x \lambda e^{-(\lambda+s) t}\,dt
\int\limits_0^\infty d_0(y,x-t)b(y)\,dy+{}
\\
{}+
\int\limits_0^x \lambda e^{-(\lambda+s) t} \,dt \int\limits_0^\infty
 u_{n}(s;v) \, dv \int\limits_0^\infty d(v|y,x-t)\, b(y)\, dy
+{}\\
{}+
\int\limits_0^x \lambda e^{-(\lambda+s) t} \, dt
\int\limits_0^\infty u_{n+1}(s;w)\,dw
\int\limits_0^\infty u_{n}(s;v) \,dv\times{}
\end{multline*}

\noindent
\begin{multline}
{}\times{}
\int\limits_0^\infty d(v,w|y,x-t) b(y)\, dy
+ 
\int\limits_0^x \lambda e^{-(\lambda+s) t} \,dt\times{}\\
{}\times
\int\limits_0^\infty u_{n+1}(s;v) \,dv
\int\limits_0^\infty  u_{n}(s;w)\, dw\times{}\\
{}\times
\int\limits_0^\infty d^*(v,w|y,x-t) b(y)\, dy\,,\\
 1 \le n \le r-1\,.
 \label{t2-mei}
\end{multline}

Система уравнений~\eqref{t1-mei}--\eqref{t2-mei} решается
рекуррентно, начиная с~$n\hm=r\hm-1$.

Зная значения $u_{n}(s;x)$, можно найти основные стационарные
временн$\acute{\mbox{ы}}$е характеристики заявок.
Пусть в~начальный момент в~системе находится~$n$~заявок, $1 \hm\le n \hm\le r-1$,
на приборе обслуживается заявка длины~$y$ и~в~этот момент
в~систему поступает заявка длины~$x$.
Обозначим через $w_n(s;x,y)$ ПЛС времени ожидания
начала обслуживания этой заявки. В~соответствии 
с~дисциплиной обслуживания имеет место равенство:
\begin{multline*}
w_n(s;x,y)=\int\limits_0^\infty \int\limits_0^\infty d^*(v,w|x,y)\, dv dw
+{}\\
{}+\int\limits_0^\infty\! d_0(v|x,y) \,dv
+ \int\limits_0^\infty \!\int\limits_0^\infty \!u_{n+1}(s;w) d(v,w|x,y)\, dv dw.
\end{multline*}
Заметим, что вероятность того, что поступающая заявка 
длины~$x$ будет потеряна при поступлении в~систему, равна:
\begin{multline*}
\pi(x)= {}\\
{}=\int\limits_0^\infty  \sum\limits_{n=1}^{r-1} p_n(y)
\left( d_0(x,y)+\int\limits_0^\infty d_0^*(w|x,y)\, dw\right)\,dy
+{}\\
{}+ \int\limits_0^\infty p_r(y)\, dy\,.
\end{multline*}
Тогда ПЛС $w(s)$ стационарного распределения времени ожидания начала обслуживания принятой
в~систему заявки определяется формулой:
\begin{multline*}
w(s) =\fr{1}{1-\pi} \biggl (
p_0 + {}\\
{}+\int\limits_0^\infty  \sum\limits_{n=1}^{r-1} p_n(y)\, dy
\int\limits_0^\infty b(x)  w_n(s;x,y)\, dx
\biggl )\,, 
\end{multline*}
где $\pi=\int_0^\infty \pi(x) b(x) \,dx$~--- безусловная вероятность потери заявки.


\subsection{Стационарное распределение времени пребывания заявки в~системе}


Распределение полного времени пребывания заявки в~системе вычисляется
несколько сложнее из-за того, что заявка, попавшая на прибор,
может покидать его и~возвращаться на него обратно,
менять свою длину, а~также уйти из системы недообслуженной.

Остановимся на нахождении следующих характеристик, которые
понадобятся в~дальнейшем:
\begin{itemize}
\item стационарное распределение времени пребывания на приборе заявки,
которая была обслужена до конца (с~учетом возможных смен длин
и~прерываний), при условии что в~момент поступления на прибор
ее длина равнялась~$x$, а~в~очереди было~$n$, $0\hm\le n\hm\le r\hm-1$, других заявок.
Через $V_{1,n}(s;x)$ будем обозначать ПЛС этого распределения;

\item стационарное распределение времени пребывания на приборе заявки,
которая могла быть и~не обслужена до конца (с~учетом возможных смен длин
и~прерываний), при условии что в~момент поступления на прибор
ее длина равнялась~$x$, а~в~очереди было~$n$, $0\hm\le n\hm\le r-1$, других заявок.
Через $V_{2,n}(s;x)$ будем обозначать ПЛС этого распределения.
\end{itemize}

Отметим, что здесь подразумевается, что время пребывания поступившей
на прибор заявки включает все времена,
на которые ее обслуживание было прервано.

Ввиду того что поступающая в~заполненную сис\-те\-му заявка теряется, выпишем
$ V_{1,r-1}(s;x)\hm=e^{-s x}$.
Далее, воспользовавшись свойством ПЛС, находим, что $V_{1,n}(s;x)$  равно:
\begin{itemize}
\item  $e^{-s x}$, если до момента времени~$x$
окончания обслуживания заявки на приборе
новая заявка не поступила (с~вероятностью~$e^{-\lambda x}$);
\item  $e^{-s t}V_{1,n}(s;w)$, если в~момент времени
$0\hm<t\hm<x$ поступила новая заявка длины~$y$,
изменила длину заявки на приборе на~$w$, а~сама\linebreak покинула систему 
(с~плотностью вероятностей~$\lambda e^{-\lambda t}
\int\nolimits_0^\infty d^*_0(w|y,x-t) b(y)\, dy$);

\item $e^{-s t}V_{1,n+1}(s;w)$, если в~момент
времени $0\hm<t\hm<x$ поступила новая заявка длины~$y$, которая
встала на первое место в~очереди, причем новая заявка
получила новую длину~$v$, а~заявка на приборе новую
длину~$w$ (с~плотностью вероятностей
$\lambda e^{-\lambda t} \int\nolimits_0^\infty d(v,w|y,x-t) b(y)\, dy$);

\item  $e^{-s t}u_{n+2}(s;v)V_{1,n}(s;w)$, если в~момент
времени $0\hm<t\hm<x$ поступила новая заявка длины~$y$, которая встала на 
прибор, получив \mbox{новую} длину~$v$, а~заявка с~прибора вытеснена на первое место 
в~очереди и~получила новую длину~$w$ (с~плотностью вероятностей
$\lambda e^{-\lambda t} \int\nolimits_0^\infty
d^*(v,w|y,x-t) b(y)\, dy $).
\end{itemize}

Воспользовавшись снова формулой полной вероятности, получаем, что
уравнение для определения ПЛС $V_{1,n}(s;x)$ имеет следующий вид:
\begin{multline*}
\!\!V_{1,n}(s;x)= e^{-(\lambda+s)x} +\int\limits_0^\infty 
V_{1,n+1}(s;w) f(s;x,w) \, dw
+{}\\
{}+\int\limits_0^\infty V_{1,n}(s;w) g_{n+2}(s;x,w) \, dw\,, \enskip 0\le n\le r-2\,,
\end{multline*}
где
\begin{multline*}
f(s;x,w)= {}\\
{}=\int\limits_0^x \lambda e^{-(\lambda+s) t}\,dt \int\limits_0^\infty  \, dv
\int\limits_0^\infty d(v,w|y,x-t)\, b(y)\, dy\,; 
\end{multline*}

\vspace*{-12pt}

\noindent
\begin{multline*}
g_{n+2}(s;x,w) = \int\limits_0^x \lambda e^{-(\lambda+s) t}\,dt
\int\limits_0^\infty u_{n+2}(s;v)  \, dv\times{}\\
{}\times
\int\limits_0^\infty d^*(v,w|y,x-t)\, b(y)\, dy
+{}\\
{}+
\int\limits_0^x \lambda e^{-(\lambda+s) t}\,dt \int\limits_0^\infty d^*_0(w|y,x-t)\, b(y)\, dy\,.
\end{multline*}

Уравнение для определения $V_{2,n}(s;x)$ получается  аналогичным образом.
Действительно, $V_{2,r-1}(s;x)\hm=e^{-s x}$.
Далее, ПЛС $V_{2,n}(s;x)$ равно:
\begin{itemize}
\item $e^{-s x}$, если до момента времени~$x$
окончания обслуживания заявки на приборе
новая заявка не поступила (с~вероятностью~$e^{-\lambda x}$);

\item  $e^{-s t}$, если в~момент $0\hm<t\hm<x$
поступила новая заявка и~она вместе с~заявкой на приборе покинула
систему (с~плотностью вероятностей~$\lambda e^{-\lambda t}
\int\nolimits_0^\infty d_0(y,x-t)b(y)\,dy$);

\item  $e^{-s t}$, если в~момент времени
$0\hm<t\hm<x$ поступила новая заявка длины~$y$, 
сама встала на прибор, а~заявка с~прибора покинула систему 
(с~плотностью вероятностей~$\lambda e^{-\lambda t}
\int\nolimits_0^\infty d_0(v|y,x-t) b(y)\, dy$);

\item $e^{-s t}V_{2,n}(s;w)$, если в~момент времени
$0\hm<t\hm<x$ поступила новая заявка длины~$y$,
изменила длину заявки на приборе на~$w$, 
а~сама\linebreak покинула систему (с~плотностью вероятностей~$\lambda e^{-\lambda t}
\int\nolimits_0^\infty d^*_0(w|y,x-t) b(y)\, dy$);

\item  $e^{-s t}V_{2,n+1}(s;w)$, если в~момент
времени $0\hm<t\hm<x$ поступила новая заявка длины~$y$, которая
встала на первое место в~очереди, причем новая заявка
получила новую длину~$v$, а~заявка на приборе новую
длину~$w$ (с~плотностью вероятностей~$\lambda e^{-\lambda t}
\int\nolimits_0^\infty d(v,w|y,x-t) b(y)\, dy$);

\item $e^{-s t}u_{n+2}(s;v)V_{2,n}(s;w)$, если в~момент
времени $0\hm<t\hm<x$ поступила новая заявка длины~$y$, которая встала на прибор, 
получив новую длину~$v$, а~заявка с~прибора вытеснена на первое место 
в~очереди и~получила новую длину~$w$ (с~плотностью вероятностей~$\lambda e^{-\lambda t}
\int\nolimits_0^\infty d^*(v,w|y,x-t) b(y)\, dy$).
\end{itemize}

Воспользовавшись снова формулой полной вероятности, получаем, что
уравнение для определения ПЛС $V_{2,n}(s;x)$ имеет следующий вид:

\noindent
\begin{multline*}
\!\!V_{2,n}(s;x)=h(s,x)+ \int\limits_0^\infty V_{2,n+1}(s;w) f(s;x,w) \, dw
+{}\\
{}+ \int\limits_0^\infty V_{2,n}(s;w) g_{n+2}(s;x,w) \, dw\,,\enskip 0\le n\le r-2\,,
\end{multline*}
где

\noindent
\begin{multline*}
h(s,x)= e^{-(\lambda+s) x}+{}\\
{}+\int\limits_0^x \lambda e^{-(\lambda+s) t}\,dt
\int\limits_0^\infty d_0(y,x-t)b(y)\,dy
+ {}\\
{}+
\int\limits_0^x \lambda e^{-(\lambda+s) t}\,dt
\int\limits_0^\infty \, dv
\int\limits_0^\infty d_0(v|y,x-t) b(y)\, dy\,.
\end{multline*}

Решение полученных уравнений осуществляется рекуррентным образом,
начиная с~$n\hm=r\hm-1$.
Естественно, ПЛС безусловных распределений получаются усреднением
 $V_{1,n}(s;x)$ и~$V_{2,n}(s;x)$ по распределению длины заявки $B(x)$.

Наконец, перейдем к~нахождению полного времени пребывания заявки 
в~системе. Будем различать два случая: первый~--- когда заявка не может
уходить из системы недообслуженной; второй~--- когда заявка на
приборе может покинуть систему не обслуженной до конца. В~обоих
случаях, как обычно, полное время пребывания заявки в~системе
складывается из времени ожидания заявкой начала обслуживания 
и~времени пребывания заявки на приборе (которое включает времена
прерываний обслуживания).

В первом случае ПЛС стационарного распределения полного времени
пребывания в~системе поступающей заявки длины~$x$ обозначим через
$V_1(s;x)$, во втором~--- через $V_2(s;x)$.

\pagebreak

Рассмотрим первый случай.

Во-первых, заявка длины~$x$ может с~вероят\-ностью~$p_0$ поступить
в~свободную систему, и~тогда время ее пребывания в~системе будет совпадать с~временем 
ее пребывания на приборе (с~учетом прерываний).

Во-вторых, с~плотностью вероятностей $p_n(y)$ поступающая заявка
длины~$x$ может застать в~сис\-те\-ме $1 \hm\le n \hm\le r-1$ заявок,
причем на приборе будет находиться заявка длины~$y$. 
В~этом случае возможны следующие варианты:
\begin{itemize}
\item либо с~вероятностью $d_0(v|x,y)\hm+d^*(v,w|x,y)$ поступающая заявка
встанет на прибор, причем ее длина станет равной~$v$ и~тогда полное время
ее пребывания в~системе будет совпадать 
с~временем ее пребывания на приборе (с~учетом прерываний);

\item либо с~вероятностью $d(v,w|x,y)$ поступающая заявка станет на первое место 
в~очереди, получит новую длину~$v$, а~заявка на приборе~--- новую длину~$w$; 
при этом время пребывания в~системе поступившей заявки будет равно сумме двух 
времен: времени до того момента, когда в~системе снова станет~$n$~заявок, 
и~времени пребывания на приборе (с~учетом прерываний) заявки длины~$v$.
\end{itemize}

Применяя формулу полной вероятности, приходим к~следующему выражению для ПЛС $V_1(s;x)$
стационарного распределения полного времени пребывания принятой заявки в~систему, в~которой
не допускается уход заявок недообслуженными:
\begin{multline}
\label{eq4-mei}
V_1(s;x)= \fr{1}{1-\pi}
\Biggl (
p_0 V_{1,0}(s;x) +{}
\\ 
{}+
\int\limits_0^\infty  \sum\limits_{n=1}^{r-1} p_n(y)
\Biggl [
\int\limits_0^\infty V_{1,n-1}(s;v)
\Biggl ( d_0(v|x,y)+{}\\
{}+\int\limits_0^\infty d^*(v,w|x,y)\,dw \Biggr ) \,dv
\Biggr ]\, dy
+{}\\
\int\limits_0^\infty  \sum\limits_{n=1}^{r-1} p_n(y)
\Biggl [
\int\limits_0^\infty 
\int\limits_0^\infty u_{n+1}(s;w) \times{}\\
{}\times V_{1,n-1}(s;v) d(v,w|x,y) \,dv  dw
\Biggr ]\,dy
\Biggr )\,.
\end{multline}

Наконец, ПЛС $V_1(s)$ стационарного распределения полного
времени пребывания в~системе заявки произвольной длины
получается усреднением $V_1(s;x)$ по распределению длины заявки $B(x)$ и~равно
$V_1(s)\hm=\int_0^\infty V_1(s;x)b(x)\,dx$.
Выражение для $V_2(s;x)$ получается путем замены
в соответству\-ющих местах формулы~\eqref{eq4-mei} $V_{1,n}(s;x)$
на $V_{2,n}(s;x)$.

\section{Заключение}

В заключение скажем несколько слов об условии существования
стационарного режима.
Для рассмотренной системы общего необходимого и~достаточного
условия его существования выписать не удается.
Оно зависит от конкретных параметров
системы и~в~каждом отдельном случае нуждается в~специальном исследовании.
Конечность среднего времени обслуживания
является только необходимым условием
и,~даже несмотря на присутствие ограничения на размер очереди,
не является достаточным\footnote{Например,
если положить $d(x,y|u,v) = e^{-v} b(x)b(y e^{-v})$,
$d^*(x,y|u,v)\hm=0$, $d(x|u,v)\hm=0$, $d_0(u,v)\hm=0$, $u, v\hm>0$,
то среднее время до того момента, когда в~системе останется
$(r\hm-2)$ заявки, при условии что в~начальный момент в~системе
было $(r\hm-1)$ заявок, без дополнительных
ограничений на функцию $b(x)$ может быть равно бесконечности. При этом,
учитывая пуассоновость входящего потока,
с~ненулевой вероятностью система
переходит в~состояние $(r\hm-2)$
и,~вообще говоря, с~ненулевой вероятностью может успеть выполнить
до прихода очередной заявки
любую находящуюся в~ней работу (при условии ее конечности), т.\,е.\
полностью опустошиться.}.


{\small\frenchspacing
 {%\baselineskip=10.8pt
 \addcontentsline{toc}{section}{References}
 \begin{thebibliography}{9}


\bibitem{n1} 
\Au{Мейханаджян Л.\,А., Милованова~Т.\,А., Печинкин~А.\,В., Разумчик~Р.\,В.}
Стационарные вероятности состояний в~системе обслуживания 
с~инверсионным порядком обслуживания и~обобщенным вероятностным
приоритетом~// Информатика и~её применения, 2014. Т.~8. Вып.~3.
С.~16--26.

\bibitem{n2} %2
\Au{Мейханаджян Л.\,А., Милованова~Т.\,А., Разумчик~Р.\,В.}
Время ожидания в~системе обслуживания с~инверсионным порядком
обслуживания и~обобщенным вероятностным приоритетом~// Информатика 
и~её применения, 2015. Т.~9. Вып.~2. С.~14--22.


%\bibitem{shrage} {\it Schrage L.} A proof of the
%optimality of the shortest remaining processing
%time discipline //
%Oper.\ Res., 1968. Vol.~16. P.~687--690.
%


\bibitem{bsev} %3
\Au{Севастьянов Б.\,А.}
Эргодическая теорема для марковских процессов и~ее приложение 
к~телефонным системам с~отказами~// ТВП, 1957. Т.~2. Вып.~1. С.~106--116.

\bibitem{ppav}  %4
\Au{Бочаров  П.\,П., Печинкин~А.\,В.}
Теория массового обслуживания.~--- М.: РУДН, 1995. 529~с.

\bibitem{n3} %5
\Au{Meykhanadzhyan L., Razumchik~R.}
New scheduling policy for estimation of stationary performance
characteristics in single server queues with inaccurate job size
information~// 30th European Conference on Modelling and Simulation Proceedings.~--- 
Dudweiler, Germany: Digitaldruck Pirrot GmbHP, 2016. P.~710--716.



%\bibitem{aaa1} {\it Нагоненко В.\ А.}
%О характеристиках одной нестандартной системы
%массового обслуживания.~I, II //
%Изв.\ АН СССР. Технич.\ кибернет., 1981.
%№~1. С.~187--195; №~3. С.~91--99.
%
%\bibitem{aaa2} {\it Печинкин А.\ В.} Об одной
%инвариантной системе массового обслуживания //
%Math.\ Operationsforsch.\ und Statist.
%Ser.\ Optimization, 1983. Vol.~14. №~3. S.~433--444.
%
%\bibitem{aaa3} {\it Нагоненко В.\ А., Печинкин А.\ В.}
%О большой загрузке в~системе с~инверсионным
%обслуживанием и~вероятностным приоритетом //
%Изв.\ АН СССР. Технич.\ кибернет., 1982. №~1. С.~86--94.
%
%\bibitem{aaa4} {\it Нагоненко В.\ А., Печинкин А.\ В.}
%О малой загрузке в~системе с~инверсионным порядком
%обслуживания и~вероятностным приоритетом //
%Изв.\ АН СССР. Технич.\ кибернет., 1984. №~6. С.~82--89.
%
%\bibitem{av1}{\it Печинкин А.\ В., Стальченко И.\ В.}
%Система $MAP/G/1/\infty$ с~инверсионным порядком
%обслуживания и~вероятностным приоритетом,
%функционирующая в~дискретном времени //
%Вестник Российского университета дружбы народов.
%Сер.\ Математика. Информатика. Физика, 2010.
%№~2. С.~26--36.
%
%\bibitem{av2}{\it Касконе А., Манзо Р.,
%Печинкин А.\ В., Салерно С.}
%Система $MAP/G/1/\infty$ в~дискретном
%времени с~инверсионной вероятностной дисциплиной
%обслуживания //
%Автоматика и~телемеханика, 2010. №~12. С.~57--69.
%
%\bibitem{av3}{\it Милованова Т.\ А., Печинкин А.\ В.}
%Стационарные характеристики системы обслуживания с
%инверсионным порядком обслуживания, вероятностным
%приоритетом и~гистерезисной политикой //
%Информатика и~ее применения, 2013. Т.~7. Вып.~1. С.~22--36.
%
%
\bibitem{jerri} %6
\Au{Jerri A.}
Introduction to integral equations with
applications.~--- New York, NY, USA: John Wiley \& Sons, 1999. 272~p.
%P. 433.

\bibitem{wh} %7
\Au{Press W.\,H., Teukolsky~S.\,A.,
Vetterling~W.\,T., Flannery~B.\,P.}
Numerical recipes:
The art of scientific computing.~--- 3rd ed.~---
 2007. 1256~p.

\bibitem{adav} %8
\Au{Полянин А.\,Д., Манжиров~А.\,В.}
Справочник по интегральным уравнениям.~---
Бока-Ратон\,--\,Лондон: Chapman \& Hall, CRC Press, 2008. 1108~p.


\end{thebibliography}

 }
 }

\end{multicols}

\vspace*{-6pt}

\hfill{\small\textit{Поступила в~редакцию 19.04.16}}

\vspace*{4pt}

%\newpage

%\vspace*{-24pt}

\hrule

\vspace*{2pt}

\hrule

\vspace*{-2pt}



\def\tit{STATIONARY CHARACTERISTICS OF~THE~FINITE CAPACITY QUEUEING SYSTEM
WITH~INVERSE SERVICE ORDER AND~GENERALIZED
PROBABILISTIC PRIORITY}

\def\titkol{Stationary characteristics of~the~finite capacity queueing system
with~inverse service order and~generalized
probabilistic priority}

\def\aut{L.\,A.~Meykhanadzhyan}

\def\autkol{L.\,A.~Meykhanadzhyan}

\titel{\tit}{\aut}{\autkol}{\titkol}

\vspace*{-9pt}

\noindent
Peoples' Friendship University of Russia,
6~Miklukho-Maklaya Str., 
Moscow 117198, Russian Federation

\def\leftfootline{\small{\textbf{\thepage}
\hfill INFORMATIKA I EE PRIMENENIYA~--- INFORMATICS AND
APPLICATIONS\ \ \ 2016\ \ \ volume~10\ \ \ issue\ 2}
}%
 \def\rightfootline{\small{INFORMATIKA I EE PRIMENENIYA~---
INFORMATICS AND APPLICATIONS\ \ \ 2016\ \ \ volume~10\ \ \ issue\ 2
\hfill \textbf{\thepage}}}

%\vspace*{3pt}


\Abste{Consideration is given to the $M/G/1/(r-1)$
queueing system with LIFO (last in, first out) preemptive
generalized probabilistic priority policy.
It is assumed that customer's service time becomes known
upon its arrival at the system
and at any time instant remaining service times
of all customers present in the system
are available. On arrival of a~customer at a~nonempty system,
its service time is compared to the (remaining) service time of the customer in
service and one of the following events occurs:
both customers leave the system at once,
one of the customers leaves the system (the other
occupies the server), or both customers stay in the system (one occupies the server,
the other~--- one place in the queue). Those customers which stay in the system
acquire new service time according to a~known
distribution, which can depend on their initial service times.
Arriving customers which find the queue full, leave the system and have no influence on it.
Analytical expressions for the computation of the
joint stationary distribution of the number of customers
in the system and the remaining service time of the customer
in the server, of the  busy period and the stationary sojourn time
(in terms of Laplace--Stieltjes transform) are proposed.}


\KWE{queueing system; special discipline; LIFO; probabilistic priority}



\DOI{10.14357/19922264160214}

\vspace*{-16pt}

\Ack

\vspace*{-2pt}

\noindent
The work is supported by the
Russian Foundation for Basic Research (project 15-07-03007).


  \vspace*{-1pt}

  \begin{multicols}{2}
  

  

\renewcommand{\bibname}{\protect\rmfamily References}
%\renewcommand{\bibname}{\large\protect\rm References}



{\small\frenchspacing
 {%\baselineskip=10.8pt
 \addcontentsline{toc}{section}{References}
 \begin{thebibliography}{9}

\vspace*{-2pt}
\bibitem{n1-1} 
\Aue{Meykhanadzhyan, L.\,A., T.\,A.~Milovanova, A.\,V.~Pechinkin, 
and R.\,V.~Ra\-zum\-chik}. 2014.
Statsionarnye veroyatnosti so\-sto\-yaniy v~sisteme obslu\-zhi\-va\-niya 
s~inversionnym po\-ryad\-kom
ob\-slu\-zhi\-va\-niya i~obob\-shchen\-nym veroyatnostnym pri\-o\-ri\-te\-tom
[Stationary distribution in a~queueing system with inverse service order and
generalized probabilistic priority].
\textit{Informatika i~ee Primeneniya}~--- \textit{Inform.Appl.}
8(3):16--26.

\bibitem{n2-1} 
\Aue{Meykhanadzhyan, L.\,A., T.\,A.~Milovanova, and R.\,V.~Ra\-zum\-chik}. 2015.
Vremya ozhidaniya v~sis\-te\-me ob\-slu\-zhi\-va\-niya s~inversionnym poryadkom obsluzhivaniya 
i~obobshchennym veroyatnostnym prioritetom
[Stationary\linebreak waiting time in a~queueing system with inverse service order and
generalized probabilistic priority].
\textit{Informatika i~ee Primeneniya}~--- \textit{Inform.Appl.}
9(2):14--22.

\bibitem{bsev-1} %3
\Aue{Sevastyanov, B.\,A.} 1957.
Ergodicheskaya teorema dlya markovskikh protsessov i~ee prilozhenie k~telefonnym 
sistemam s~otkazami
[An ergodic theorem for markov processes and its application to telephone systems with refusals].
\textit{Teor. Veroyatnost. i Primenen.} 
[Probability Theory and Its Applications] 2(1):106--116.



\bibitem{ppav-1} %4
\Aue{Bocharov,  P.\,P., and A.\,V.~Pechinkin}. 1995.
\textit{Teoriya massovogo obsluzhivaniya} [Queueing theory].
Moscow: RUDN. 529~p.

\bibitem{n3-1} %5
\Aue{Meykhanadzhyan, L., and R.~Razumchik}. 2016.
New scheduling policy for estimation of stationary performance
characteristics in single server queues with inaccurate job size
information. \textit{30th European Conference on Modelling and Simulation Proceedings}.
Dudweiler, Germany: Digitaldruck Pirrot GmbHP. 710--716.



\bibitem{jerri-1} %6
\Aue{Jerri, A.} 1999.
\textit{Introduction to integral equations with applications}.
New York, NY: John Wiley \& Sons. 272~p.



\bibitem{wh-1} %7
\Aue{Press, W.\,H., S.\,A.~Teukolsky, W.\,T.~Vetterling,
and B.\,P.~Flannery}. 2007.
\textit{Numerical recipes:  
The art of Scientific computing}. 3rd ed. 1256~p.

\bibitem{adav-1} %8
\Aue{Polyanin, A.\,D., and A.\,V.~Manzhirov}. 2008.
\textit{Handbook of integral equations}.
Boca Raton\,--\,London: Chapman \& Hall,
CRC Press. 1108~p.
\end{thebibliography}

 }
 }

\end{multicols}

\vspace*{-7pt}

\hfill{\small\textit{Received April 19, 2016}}

\vspace*{-17pt}
   

\Contrl

\vspace*{-2pt}

\noindent
\textbf{Meykhanadzhyan Lusine A.} (b.\ 1990)~---
PhD student, Peoples' Friendship University of Russia, 6~Miklukho-Maklaya Str., 
Moscow 117198, Russian Federation; lameykhanadzhyan@gmail.com

 
\label{end\stat}


\renewcommand{\bibname}{\protect\rm Литература}