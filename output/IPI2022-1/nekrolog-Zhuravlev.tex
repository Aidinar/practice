
   \vspace*{-48pt}

\begin{center}
\vspace*{6pt}
\mbox{%

\epsfxsize=60mm %112.705
\epsfbox{nzur-1.eps}
}
%\end{center}

\vspace*{14pt} 


%   \begin{center}
\fbox{\large\textbf{Академик Юрий Иванович Журавлев}}\\[12pt]
\textbf{\large 14.01.1935--14.01.2022}
   \end{center}


   %\vspace*{2.5mm}

   \vspace*{8mm}

   \thispagestyle{empty}

%\

%\vspace*{-12pt}
 
      %Академик Юрий Иванович Журавлев (14.01.1935--14.01.2022)
      
      Федеральный исследовательский центр <<Информатика и~управление>> с~глубоким
      прискорбием извещают, что     
      14~января 2022~г.\ %в~возрасте 87~лет 
      скончался выдающийся ученый, академик 
Российской академии наук, председатель Редакционного совета и~член редколлегии 
журнала <<Информатика и~её применения>> Юрий Иванович Журавлев.


      
      Юрий Иванович Журавлев~--- всемирно известный математик, 
выдающийся ученый в~области информатики и~математической кибернетики, создатель 
алгебраической теории алгоритмов и~методов
 принятия решений на основе неполной, 
противоречивой, разнородной информации. Среди его учеников 4~академика РАН, более 
30~докторов и~150~кандидатов наук. 


      
      Родился в~Воронеже в~1935~г. В~1957~г.\ окончил механико-математический 
факультет МГУ имени М.\,В.~Ломоносова и~в~1959~г.~--- аспирантуру того же факультета. 
В~1960--1969~гг.\ работал в~Институте математики Сибирского отделения АН \mbox{СССР} 
в~различных должностях вплоть до заместителя директора по научной работе. 


      
      С 1969~г.\ работал в~Вычислительном центре АН СССР им.\ А.\,А.~Дородницына, 
а~с~2015 г. и~вплоть до настоящего времени~--- в~Федеральном исследовательском центре 
<<Информатика и~управление>> Российской академии наук в~должности главного научного 
сотрудника, научного руководителя отделения математических методов анализа данных 
и~прогнозирования. 
      
      В 1984~г.\ был избран чле\-ном-кор\-рес\-пон\-ден\-том АН СССР, в~1992~г.~--- 
академиком РАН. Руководитель секции <<Прикладная математика и~информатика>> 
Отделения математических наук (ОМН) РАН (до 2017~г.), заместитель академика-секретаря 
ОМН РАН (2002--2017~гг.).
      
      Лауреат Ленинской премии, лауреат премии Совета Министров СССР, лауреат 
премии Правительства РФ в~области образования, лауреат Ломоносовской премии I~степени. 
Награжден орденами Трудового Красного Знамени, <<За заслуги перед Отечеством>> III 
и~IV степеней, Дружбы народов, \mbox{шестью} медалями СССР и~РФ, Кавалерским 
крестом Ордена Почета Республики Польша. Член Европейской академии наук. 
Заслуженный профессор МГУ имени М.\,В.~Ломоносова и~МФТИ.
      
      

\pagebreak

\

\vspace*{-12pt}

%\linebreak\vspace*{-12pt}

\begin{floatingfigure}{50mm} %60mm}
\begin{center} %fig2
\vspace*{6pt}
\mbox{%

\epsfxsize=39.007mm %57mm %46.402 mm
\epsfbox{nzur-2.eps}
}
\end{center}
\vspace*{-6pt}
\end{floatingfigure}

\noindent



     \thispagestyle{empty}
     
     С 1989~г.\ являлся членом Правления Международной ассоциации по распознаванию 
образов (IAPR), председателем Национального комитета РАН по распознаванию образов 
и~анализу изображений~--- коллективного члена IAPR.
     
     С 1991~г.~--- главный редактор журнала <<Pattern Recognition and Image Analysis>> 
(Web of Science, Scopus). Председатель, член редколлегий и~наблюдательных советов 
журналов <<Дискретная математика>>, <<Журнал вычислительной математики 
и~математической физики>> и~многих других.
      
      В 1967~г.\ Ю.\,И.~Журавлев стал одним из организаторов и~первым председателем 
Всесоюзного Совета молодых ученых. В~1997~г.\ организовал на факультете ВМК МГУ 
имени М.\,В.~Ломоносова новую кафедру <<Математические методы прогнозирования>>, 
которой руководил вплоть до настоящего времени. 
      
      \smallskip
      
      Выражаем глубокое соболезнование семье, родственникам, друзьям и~коллегам по работе в~связи с~тяжелой
      невосполнимой утратой.
      %
      Мир потерял выдающегося уче\-но\-го-ма\-те\-ма\-ти\-ка, талантливого педагога 
и~замечательного человека. Светлая память о Юрии Ивановиче навсегда сохранится 
в~сердцах его коллег, учеников и~близких.

\begin{center} %fig3
\vspace*{18pt}
\mbox{%

\epsfxsize=90mm %83.502mm 
\epsfbox{nzur-3.eps}
}
\end{center}
      
      