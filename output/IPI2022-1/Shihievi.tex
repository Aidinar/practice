\def\stat{shihievi}

\def\tit{УПРОЩЕННЫЙ ЯЗЫК ЗРИТЕЛЬНЫХ ОБРАЗОВ}

\def\titkol{Упрощенный язык зрительных образов}

\def\aut{Ш.\,Б.~Шихиев$^1$, Ф.\,Ш.~Шихиев$^2$}

\def\autkol{Ш.\,Б.~Шихиев, Ф.\,Ш.~Шихиев}

\titel{\tit}{\aut}{\autkol}{\titkol}

\index{Шихиев Ш.\,Б.}
\index{Шихиев Ф.\,Ш.}
\index{Shihiev Sh.\,B.}
\index{Shihiev F.\,Sh.}


%{\renewcommand{\thefootnote}{\fnsymbol{footnote}} \footnotetext[1]
%{Работа выполнена при поддержке Министерства науки и~высшего образования Российской Федерации (проект 
%075-15-2020-799).}}


\renewcommand{\thefootnote}{\arabic{footnote}}
\footnotetext[1]{Дагестанский государственный университет, sh\_sh\_b51@mail.ru}
\footnotetext[2]{Дагестанский государственный университет, fuad@mail.ru}

\vspace*{8pt}



     
     
  \Abst{Реализация модели естественного языка (ЕЯ) как информационной сис\-те\-мы (ИС)
нуждается в~определенных сведениях о~том, как функционирует сама языковая 
спо\-соб\-ность человека в~его физиологии и~в~какой по\-сле\-до\-ва\-тель\-ности должны 
взаимодействовать две ее автономные со\-став\-ля\-ющие: грамматика и~семантика как 
программные обеспечения (ПО) языкового явления в~циф\-ро\-вой технике. Изучение этой 
задачи следует начинать с~реализации языка зрительных образов, так как в~этом случае 
экран монитора мож\-но использовать в~качестве <<органа зрения>> носителя языка 
(компьютера), что поз\-во\-ля\-ет вес\-ти параллельную обработку одной и~той же 
графической информации в~двух формах: в~форме текс\-та (синтаксический анализ) 
и~в~форме рисунка (семантический анализ). Глубина мыслительного процесса зависит 
от спо\-соб\-ности ИС трансформировать информацию из одной 
формы (текс\-то\-вой) в~другую (графическую), и~наоборот.}
  
  \KW{естественный язык; синтаксис; семантика; словарь; зрительный образ; 
семантическая сеть; синтаксическая сеть; трансформация; алгоритм; память}

\DOI{10.14357/19922264220110}
  
%\vspace*{-4pt}


\vskip 10pt plus 9pt minus 6pt

\thispagestyle{headings}

\begin{multicols}{2}

\label{st\stat}

\section{Введение}

  Статья завершает цикл работ, представленный в~[1,2]. В~\cite{1-sh} 
можно ознакомиться с~приемами кодирования морфологических функций, 
словоформ и~словосочетаний. В~\cite{2-sh} описаны математические 
модели морфологии, синтаксиса и~семантики компьютерной лингвистики. 
Реализация описанной в~\cite{2-sh} модели ЕЯ как 
ИС нуждается в~демонстрации соответствия 
меж\-ду служ\-ба\-ми (элементами) этих сис\-тем. Для этого придется подробно 
описать компьютерную модель ЕЯ хотя бы для некоторой языковой игры, 
например для языка визуальных объектов.

\section{Язык}

  Язык~--- творение эволюции; в~нем проявлена сущ\-ность природы 
вещей: единство \textit{корпускулярных} структур (слов) 
и~\textit{непрерывной} среды силовых полей нейронов, об\-ра\-зу\-ющих 
память, в~которой хранятся \textit{знания} (значения слов)~--- образы 
\textit{внеш\-них явлений}~[3].Таким образом, каждый \textit{элемент языка} 
(слово и~сочетание слов) пред\-став\-ля\-ет собой ипо\-стась (единство двух 
противоположных про\-яв\-ле\-ний субстанции), и~потому \textit{элементы 
языка} имеют свои и~имена, и~значения; поэтому и~\textit{языкознание} 
также представлено двумя час\-тя\-ми: \textit{грамматикой}, изучающей 
семиотику \textit{элементов языка}, и~\textit{семантикой}, изучающей 
значения тех же \textit{элементов языка}. 
  
  Символ, именующий (обозначающий) некоторое \textit{знание}, сам 
является \textit{знанием} как всякое восприятие, име\-ющее место в~органах 
чувств. Например, слово <<дом>> представлено в~памяти человека двумя 
различными \textit{знаниями}: \textit{знанием} о~семиотике этого слова из 
трех букв и~\textit{знанием} о~стро\-ении определенной конструкции и~предназначении. 
  
  Физиология человека позволяет одновременно воспринимать активность 
(клеток) как в~\textit{органах чувств}, так и~в~\textit{памяти}. Орган, 
следящий за активностью этих двух областей организма, называется 
\textit{со\-зна\-нием}. 
  
  \textit{Информация} генерируется в~органах чувств под воздействием 
внешних сил и~в~памяти, элементы которой могут активизироваться под 
воздействием другого \textit{связанного} (ассоциированного) с~ним знания.
   
  \textit{Память}~--- универсальное образование живых клеток для хранения информации, 
пред\-став\-ля\-ющей собой материальный (физиологический) \textit{образ} внеш\-не\-го явления. 
Под материальным \textit{образом} явления (вещи) будем подразумевать  
фи\-зи\-ко-хи\-ми\-че\-ское со\-сто\-яние ней\-ро\-точ\-ки, которая была образована под воздействием 
данного явления (вещи). В~памяти, в~част\-ности, хранятся материальные образы слов и~их 
значений. 
   
  Область, представляющая собой знание в~органах чувств и~в~памяти, 
назовем \textit{нейроточкой}. Под единством мира слов и~мира 
представлений будем подразумевать \textit{нейронные связи}: (1)~знание 
в~органах чувств и~его копия в~памяти и~знания в~памяти; (2)~знания 
о~вещах; (3)~знание о~слове и~о~его значении; (4)~если знания о вещах 
связаны, то связаны и~знания о~словах, обозначающих их. Структуру из 
\textit{нейроточек} и~\textit{нейронных связей} между ними принято 
называть \textit{нейронной сетью}.
  
  \textit{Нейронную сеть}, в~вершинах (\textit{нейроточках}) которой 
хранятся знания о~словах, назовем \textit{языковой} 
(\textit{синтаксической}) \textit{нейронной сетью} (SinNet), 
происходящие в~ней процессы~--- \textit{языковым явлением} (\textit{языковой 
способностью}), а~его внеш\-нее проявление~--- \textit{языковой 
активностью}.
  
  Ниже предлагается \textit{модель языковой способности} в~трех формах: 
\textit{словесной}, \textit{математической} и~\textit{компьютерной}.
  
  \textbf{Словесная модель языка.} \textit{Нейронную сеть}, в~вершинах 
которой хранятся знания о~вещах, назовем \textit{семантической сетью} 
(SemNet). Из перечисленных выше видов связей в~памяти следует 
\textit{изоморфизм языковых} сетей: SemNet и~SinNet~\cite{4-sh}.
  
  Элемент памяти может находиться в~\textit{активном} со\-сто\-янии 
и~являться (\textit{сознаваемым}) элементом сознания.  
\textit{Сознание}~--- надстройка над памятью, способная реагировать на 
активное со\-сто\-яние элемента памяти. 
  
  Примитивная \textit{языковая способность человека} 
(\textit{мышление})~--- это способность обхода вершин \textit{языковой 
сети} по определенным правилам. От степени ак\-тив\-ности языковой сети 
в~формировании предложений зависит тип мыш\-ле\-ния человека.

\section{Понятие}

  \textit{Понятие}~--- именованное знание, т.\,е.\ \textit{знание}, 
к~которому прикреплено \textit{слово} (имя знания). Понятие~--- пара 
вершин $(v, w)$, где $v$ и~$w$ принадлежат сетям SemNet и~SinNet 
соответственно, а~$v\hm\to w$ задает \textit{изоморфизм} сетей SemNet 
и~SinNet. Ак\-тив\-ность одного из элементов пары~$v$ и~$w$ приводит 
в~активное со\-сто\-яние и~второй ее элемент.
  
  \textbf{Математическая модель языка.} Функционирование языка 
поддерживают сети: 
$$
\mathrm{SemNet}=(\mathrm{IntWorld},R); \enskip \mathrm{SinNet}=(\mathrm{IntWord},Q),
$$
 где 
IntWorld~--- множество \textit{понятий}; $R$~--- \textit{семантические} 
отношения между \textit{понятиями}, т.\,е.~$R$~--- под\-мно\-же\-ст\-во прямого 
произведения $\mathrm{IntWorld}\times \mathrm{IntWorld}$; IntWord~--- 
множество \textit{слов} (имен \textit{понятий}); $Q$~--- 
\textit{синтаксические} отношения между \textit{словами}, т.\,е.~$Q$~--- 
подмножество прямого произведения $\mathrm{IntWord}\hm\times 
\mathrm{IntWord}$. 
  
  Предполагается, что SemNet и~SinNet~--- реальные структуры в~памяти 
человека, \textit{изоморфизм} $f(\mathrm{IntWord})\hm\to\mathrm{IntWorld}$ этих графов, т.\,е.\
 соответствие между знаниями и~их именами, реализован в~памяти 
нейронными связями. Эти две сети образуют ИС, известную как ЕЯ.
  
  Носителю языка для самостоятельного мыш\-ле\-ния не нуж\-ны элементы 
внешнего мира (вви\-ду изоморфизма сетей); язык функционирует как 
явление внут\-рен\-не\-го мира человека. Память человека располагает всем 
тем, что нужно для анализа (распознавания) и~синтеза (по\-стро\-ения) 
предложений языка. Человек идентифицирует себя со своим со\-зна\-нием.
  
  В то же время человек со\-став\-ля\-ет одно целое с~внеш\-ним миром. 
Вершины сети SemNet имеют во внеш\-нем мире свои прообразы (вещи)~---
ExtWorld. Отношения между понятиями в~сети SemNet установлены 
благодаря органам чувств или путем размышлений. 
  
  Вершины сети SinNet имеют свои прообразы~--- мир слов ExtWord во 
внешнем мире. Поэтому сети SemNet и~SinNet~--- нечто со\-тво\-рен\-ное 
благодаря уникальным возможностям физиологии человека. В~ходе 
формирования языка сеть SinNet была дополнена новыми связями для 
поддержания правил грамматики.
  
  О модели грамматики (морфологии и~синтаксиса) можно прочитать 
в~\cite{2-sh, 5-sh}. Для конкретного ЕЯ можно по\-стро\-ить сеть 
SinNet\;=\;(Lex, SR), где Lex~--- лек\-си\-ка данного языка; SR~--- 
синтаксические отношения, определенные на лек\-си\-ке Lex~\cite{6-sh}, так, 
что элементы SinNet по\-рож\-да\-ют все\-воз\-мож\-ные выражения на этом языке. 
  
  \textbf{Компьютерная модель языка.} В~небольшой статье трудно 
описать шаги в~реализации систем SemNet и~SinNet на компьютере. 
Ограничимся только при\-вяз\-кой элементов множеств: ExtWorld, IntWorld, 
IntWord и~ExtWord к~известным служ\-бам компьютера. На рис.~1 показана 
схема отношений между этими множествами.


          
      
  В нижней части рис.~1 расположены элементы внеш\-не\-го мира: ExtWorld и~ExtWord. 
  Для пользователя <<внешним миром>> служит экран 
монитора, разделенный на две час\-ти~--- в левой час\-ти экрана рисуются 
геометрические фигуры, а~в~правой\linebreak\vspace*{-12pt}

{ \begin{center}  %fig1
 \vspace*{12pt}
   \mbox{%
\epsfxsize=79mm
\epsfbox{shi-1.eps}
}

\vspace*{6pt}

\noindent
{{\figurename~1}\ \ \small{
Отношения между составляющими языка
}}
\end{center}
}

     
     
     \noindent
      час\-ти экрана отображаются 
выражения на ЕЯ. Про\-грам\-мное обеспечение, со\-сто\-ящее из \mbox{ПОСем} 
и~\mbox{ПОСин}, должно обеспечить функционирование\linebreak языка зрительных 
образов: сформировать словесное описание гео\-мет\-ри\-че\-ской фигуры из 
левой час\-ти или, наоборот, нарисовать фигуру по ее описанию. На рис.~1 
показаны шаги преобразования \mbox{элементов} ExtWorld в~элементы ExtWord, 
и~наоборот, т.\,е.\ выполняются функции языка. Отображения $f$, $g$, $h$ 
и~$r$ реализуются в~виде программ. Технология пред\-став\-ле\-ния 
графических объектов и~слов (строк) в~оперативной памяти (ОП) хорошо 
известна, и~они же могут быть отображены на экране монитора. 
  
  Задача не нова. Ценность представляет способ ее решения. Здесь 
предлагается алгоритм решения этой задача, ими\-ти\-ру\-ющий те правила, 
которые заложены в~основе языковой спо\-соб\-ности человека. 
  
  Изучать <<физиологию языка>> экспериментально нереально. А~на 
компьютере можно экспериментировать и~оценить жиз\-не\-спо\-соб\-ность 
до\-ступ\-ных наблюдению фрагментов языкового явления. Возможности 
компьютера в~<<зрительном восприятии>> и~в~<<усво\-ении>> 
грамматики ЕЯ~--- шанс для такого эксперимента.
  
  <<Зрительное восприятие>> компьютера достигается определением всей 
терминологии графических объектов (\textit{примитивов}) по\-сред\-ст\-вом 
геометрии, по\-стро\-ен\-ной на битах (пикселях), а~операции над объектами 
пред\-став\-ле\-ны алгоритмами. Будем считать, что со\-от\-вет\-ст\-ву\-ющие 
программы для распознавания и~преобразование фигур уже имеются.
  
  Об <<усвоении>> грамматики ЕЯ компьютером мож\-но посмотреть 
в~\cite{1-sh, 2-sh}. Каждый носитель языка является мыслителем; суть 
мышления заключается в~по\-стро\-ении одного выражения из двух 
име\-ющих\-ся. А~мышление в~об\-ласти зрительных образов заключается 
в~по\-стро\-ении выражения <<синий отрезок>> из двух слов <<синий>> 
и~<<отрезок>>, которые были возвращены программой распознавания. 
И~наоборот: программно <<угадать>>, какое отношение имеют слова 
<<синий>> и~<<отрезок>> из выражения <<синий отрезок>> к~фигуре 
с~названием <<синий отрезок>>.
  %
  Здесь поможет \textit{синтаксическая семантика}, которая указывает на 
явную связь между \textit{синтаксическими отношениями} 
в~словосочетании <<синий отрезок>> и~\textit{семантическими 
отношениями} между явлениями (элементами гео\-мет\-рии) <<синий>> 
и~<<отрезок>>.
  
\section{Пример}

  \textit{Синий экран} размера $1200\times800$ пусть пред\-став\-лен 
в~памяти 3\,840\,000~битами (растровое изображение,\linebreak\vspace*{-12pt}

{ \begin{center}  %fig2
 \vspace*{-1pt}
   \mbox{%
\epsfxsize=79mm
\epsfbox{shi-2.eps}
}

\vspace*{6pt}

\noindent
{{\figurename~2}\ \ \small{
Фрагмент GSem семантической сети SemNet
}}
\end{center}
}

\vspace*{9pt}

\noindent
 глубина цвета~--- 4). 
Программа~$g$ из ПОСем уста\-нав\-ли\-ва\-ет \textit{синий цвет} 
у~\textit{части экрана}, со\-сто\-ящей из \textit{всех} его пикселей (в~этом ее 
назначение); возвращает граф GSem из этих пяти выделенных кур\-си\-вом 
слов (рис.~2) и~передает его программе~$h$. В~вершине графа GSem указан 
термин и~часть речи (ИС~--- имя существительное; ИП~--- имя прилагательное).


     
  Граф GSem является подграфом семантической сети SemNet, 
по\-рож\-ден\-ным пятью вершинами, а~сеть SemNet, разумеется, хранится 
в~ОП.
  
  Программа~$h$ из ПОСем предназначена для формирования выражения 
(русского языка), в~котором будет заключена информация, со\-дер\-жа\-ща\-яся 
в~графе GSem (т.\,е.\ результат бес\-сло\-вес\-но\-го мыш\-ле\-ния по\-сред\-ст\-вом 
программы~$g$).
  
  Работа программы~$h$ описана в~\cite{5-sh}; она выделяет в~GSem 
корневые деревья, удовле\-тво\-ря\-ющие известным условиям. Одно из 
требований, например,~--- кор\-нем дерева должна быть в~этом примере 
вершина <<экран>>. Обходу дерева соответствует определенное 
выражение. Не вдаваясь в~под\-роб\-но\-сти, пе\-ре\-чис\-лим некоторые 
выражения, воз\-вра\-ща\-емые программой~$h$: \textit{экран весь цвета синего; 
экран цвета синего; экран синий} и~т.\,д. Программа~$r$ выводит на экран 
выражения, воз\-вра\-щен\-ные программой~$h$.
  
  Решение обратной задачи построения фигуры, описанной выражением 
на ЕЯ, со\-сто\-ит из тех же шагов (решения прямой задачи), но исполняются 
они в~обрат\-ном порядке.
  
  Этим примером можно было бы и~завершить описание принципа работы 
имитатора языка зрительных объектов, но будет полезно указать на 
некоторые ню\-ан\-сы, которые могут встречаться при реализации 
операции~$hg$.
  
  Во-первых, следует сказать о~хорошо известных <<разночтениях>> 
рисунка. Например, вырез белого цвета внут\-ри тем\-ной прямоугольной 
об\-ласти может восприниматься как темная об\-ласть, на которую наложен 
пред\-мет белого цвета.
  
  Реализация одной (или обеих) ситуаций из двух име\-ющих\-ся зависит от 
кон\-текс\-та, в~котором находится анализатор рисунка. Под 
\textit{контекстом} под\-разуме\-ва\-ет\-ся подграф, который будет выбран из 
семантической сети после про\-смот\-ра экрана. И~далее выбирается вариант 
дерева, оно и~есть выражение языка.
  
  Задача словесного описания геометрических фигур услож\-нит\-ся, если 
они будут в~движении, так как описание движения пред\-ме\-тов требует 
наличия сис\-те\-мы координат, в~которой оно происходит. Только после 
этого мож\-но строить и~оперировать деревьями (в~скобочной 
форме) сле\-ду\-юще\-го содержания: \textbf{отрезок (верхний, 
треугольник(синий))} и~со\-от\-вет\-ст\-ву\-ющим выражением после его 
левостороннего обхода: <<\textbf{отрезок верхний треугольника 
синего}>> или <<\textbf{верхний отрезок синего треугольника}>>.
  
  Завершим статью описанием пошагового алгоритма (схемы) 
транс\-фор\-ма\-ции в~выражение ЕЯ прос\-то\-го рисунка из отрезка крас\-но\-го 
цвета и~ломаной из трех звень\-ев синего цвета. Алгоритм со\-сто\-ит из 
нескольких шагов.
  
  \textbf{Шаг~1.} \textit{Процедура восприятия экрана}. На \textit{экране} 
выделяются пиксели одного цвета, отличного от белого (цвета экрана). По 
результатам этого поиска в~сети SemNet выделяются вершины 
\textit{экран}, \textit{отрезок} и~\textit{красный} и~дуги (\textit{экран}, 
\textit{отрезок}) и~(\textit{отрезок}, \textit{красный}). (Вершина 
\textit{экран} всегда имеется в~SemNet; если вер\-шин \textit{отрезок} 
и~\textit{красный} нет в~сети SemNet, то они и~две упомянутые дуги 
до\-бав\-ля\-ют\-ся SemNet.)
  
  Далее выделяются пик\-се\-ли синего цвета; они образуют \textit{ломаную}. 
В~сети SemNet выделяются дуги: (\textit{экран}, \textit{ломаная}), 
(\textit{ломаная}, \textit{синий}), (\textit{ломаная}, \textit{звено}) 
и~(\textit{звено}, \textit{четыре}).
  
    \textbf{Шаг~2.} По \textit{семантическому признаку} дуги $(v, w)$ определяются 
\textit{синтаксические} формы ее элементов~$v$ и~$w$, а~следовательно, 
и~\textit{словосочетание} <<$v, w$>>. (Каждой дуге сети SemNet 
приписан при\-знак; например, при\-зна\-ком дуги (\textit{звено}, 
\textit{четыре}) служит \textit{количество} (\textit{звеньев ломаной}) и~ей 
соответствует словосочетание <<звеньев четыре>> или <<\textit{четыре 
звена}>>.)
  
    \textbf{Шаг~3.} Вершины и~дуги, выделенные в~SemNet при про\-смот\-ре экрана, 
по\-рож\-да\-ют некоторый подграф GSem\;=\;(DSem, USem), которому в~сети 
SinNet соответствует граф GSin\;=\;(DSin,\ USin).
  
    \textbf{Шаг~4.} Корневому дереву TSem в~GSem соответствует дерево TSin 
  в~GSin. Примеры таких деревьев приводятся в~\cite{5-sh}. Обход остового 
дерева TSin пред\-став\-ля\-ет собой выражение, опи\-сы\-ва\-ющее заданный 
рисунок.

\section{Заключение}

  Разработка синтаксических анализаторов давно вошла в~прак\-ти\-ку 
разработки программ искусственного интеллекта. Разработчики 
анализаторов час\-то умалчивают об этапе проектирования программ. Здесь 
на одном част\-ном примере была продемонстрирована технология 
функционирования ана\-ли\-за\-тора.
  
{\small\frenchspacing
 {%\baselineskip=10.8pt
 %\addcontentsline{toc}{section}{References}
 \begin{thebibliography}{9}
\bibitem{1-sh}
\Au{Мирзабеков Я.\,М., Шихиев~Ш.\,Б.} Дискретный анализ в~синтаксическом 
анализе~// Информатика и~её применения, 2018. Т.~12. Вып.~2. С.~98--104.

\bibitem{2-sh}
\Au{Шихиев Ш.\,Б., Шихиев~Ф.\,Ш.} Инкапсуляция семантических пред\-став\-ле\-ний 
в~элементы грамматики~// Информатика и~её применения, 2020. Т.~14. Вып.~1.  
С.~121--127.
\bibitem{3-sh}
\Au{Ауробиндо Шри}.
Великая психология~/ Пер. с~англ.~--- Введение в~психологическую мысль сер.~--- М.: 
АСТ, 2006. 543~с.
(\Au{Aurobindo~Sri.}  A~greater psychology.~--- An introduction to the psychological thought ser.~---
 Jeremy P.~Tarcher., 2001. 426~p.)
\bibitem{4-sh}
\Au{Харари Ф.} Тео\-рия графов~/ Пер. с~англ.~--- М.: Едиториал УРСС, 2003. 296~с.
(\Au{Harari~F.} {Graph theory}.~--- Reading, MA, USA: Addison-Wesley 
Publishing Co., 1969. 274~p.)
\bibitem{5-sh}
\Au{Шихиев Ф.\,Ш.} Формализация и~сетевая формулировка задачи синтаксического 
анализа: Дис.\ \ldots\ канд. физ.-мат. наук.~--- СПб.: СпбГУ, 2006. 171~с.
\bibitem{6-sh}
Грамматика русского языка: в~2~т.~/ Ред. коллегия: акад. В.\,В.~Виноградов,
Е.\,С.~Истрина, С.\,Г.~Бархударов.~--- 
М.: Изд-во АН СССР, 1960. Т.~1. 719~с.
%; Т.~2: 
%Синтаксис. Ч.~1. 702~с. Т.~2: Синтаксис. Ч.~2.  440~с.

\end{thebibliography}

 }
 }

\end{multicols}

\vspace*{-6pt}

\hfill{\small\textit{Поступила в~редакцию 06.07.20}}

\vspace*{8pt}

%\pagebreak

%\newpage

%\vspace*{-28pt}

\hrule

\vspace*{2pt}

\hrule

%\vspace*{-2pt}

\def\tit{SIMPLIFIED LANGUAGE FOR~VISUAL IMAGES}


\def\titkol{Simplified language for~visual images}


\def\aut{Sh.\,B.~Shihiev and F.\,Sh.~Shihiev}

\def\autkol{Sh.\,B.~Shihiev and F.\,Sh.~Shihiev}

\titel{\tit}{\aut}{\autkol}{\titkol}

\vspace*{-11pt}


\noindent
Department of Discrete Mathematics and Computer Science, Dagestan State University,  
43-a~Gadzhiyev Str., Makhachkala 367000, Republic of Dagestan, Russian Federation

\def\leftfootline{\small{\textbf{\thepage}
\hfill INFORMATIKA I EE PRIMENENIYA~--- INFORMATICS AND
APPLICATIONS\ \ \ 2022\ \ \ volume~16\ \ \ issue\ 1}
}%
 \def\rightfootline{\small{INFORMATIKA I EE PRIMENENIYA~---
INFORMATICS AND APPLICATIONS\ \ \ 2022\ \ \ volume~16\ \ \ issue\ 1
\hfill \textbf{\thepage}}}

\vspace*{3pt} 




\Abste{The implementation of the natural language model as an information system needs 
certain information about how the language ability of a person functions in his/her physiology and 
in what sequence its two autonomous components should interact: grammar and semantics as 
software for the linguistic phenomenon in digital technology.
The study of this problem 
should begin with the implementation of the language of visual images since in this case,\linebreak\vspace{-12pt}}

\Abstend{the 
monitor screen can be used as the ``organ of the visual system'' of a~native speaker 
(computer) which allows parallel processing of the same graphic information in two forms: in 
the form of text (syntactic analysis) and in the form of image (semantic analysis). The depth 
of the thought process depends on the ability of the information system to transform 
information from one form (text) to another (graphic) and vice versa.}

\KWE{natural language; syntax; semantics; dictionary; vision; semantic network; syntactic 
network; transformation; algorithm; memory}

\DOI{10.14357/19922264220110}

%\vspace*{-16pt}

%\Ack
%\noindent




%\vspace*{6pt}

  \begin{multicols}{2}

\renewcommand{\bibname}{\protect\rmfamily References}
%\renewcommand{\bibname}{\large\protect\rm References}

{\small\frenchspacing
 {%\baselineskip=10.8pt
 \addcontentsline{toc}{section}{References}
 \begin{thebibliography}{9}
\bibitem{1-sh-1}
\Aue{Mirzabekov, Ya.\,M., and Sh.\,B.~Shihiev.} 2018. 
Diskretnyy analiz v~sintaksicheskom analize
[Discrete analysis in parsing].
\textit{Informatika i~ee Primeneniya~--- Inform. Appl.} 12(2):98--104.
\bibitem{2-sh-1}
\Aue{Shihiev, Sh.\,B., and F.\,Sh.~Shihiev.} 2020. Inkapsulyatsiya semanticheskikh 
predstavleniy v~elementy grammatiki [Incapsulation of semantic representations into 
elements of a grammar]. \textit{Informatika i~ee Primeneniya~--- Inform. Appl.}  
14(1):121--127.
\bibitem{3-sh-1}
\Aue{Aurobindo, Sri.} 2001. 
\textit{A~greater psychology}. An introduction to the psychological thought ser. Jeremy P.~Tarcher. 426~p.


\bibitem{4-sh-1}
\Aue{Harari, F.} 1969. \textit{Graph theory}. Reading, MA: Addison-Wesley 
Publishing Co. 274~p.
\bibitem{5-sh-1}
\Aue{Shihiev, F.\,Sh.} 2006. Formalizatsiya i setevaya formulirovka zadachi sintaksicheskogo 
analiza [Formalization and net- work interpretation of a~parsing task]. St.\ Petersburg: St.\ 
Petersburg State University. PhD Diss. 171~p.
\bibitem{6-sh-1}
Vinogradov, V.\,V., E.\,S.~Istrina, and S.\,G.~Barkhudarova, eds. 1960. \textit{Grammatika 
russkogo yazyka} [Russian language grammar]. Moscow: AN SSSR. Vol.~1. 719~p.
\end{thebibliography}

 }
 }

\end{multicols}

\vspace*{-6pt}

\hfill{\small\textit{Received July 6, 2020}}

%\pagebreak

%\vspace*{-18pt}

\Contr

\noindent
\textbf{Shihiev Shukur B.} (b.\ 1951)~--- Candidate of Science (PhD) in physics and 
mathematics, associate professor, Department of Discrete Mathematics and Computer 
Science, Dagestan State University, 43-a~Gadzhiyev Str., Makhachkala 367000, Republic of 
Dagestan, Russian Federation; \mbox{sh\_sh\_b51@mail.ru}

\vspace*{3pt}

\noindent
\textbf{Shihiev Fuad Sh.} (b.\ 1980)~--- Candidate of Science (PhD) in physics and 
mathematics, associate professor, Department of Discrete Mathematics and Computer 
Science, Dagestan State University, 43-a~Gadzhiyev Str., Makhachkala 367000, Republic of 
Dagestan, Russian Federation; \mbox{fuad@mail.ru}



\label{end\stat}

\renewcommand{\bibname}{\protect\rm Литература} 