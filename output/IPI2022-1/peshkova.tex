\def\stat{peshkova}

\def\tit{СРАВНЕНИЕ ЭКСТРЕМАЛЬНЫХ ИНДЕКСОВ ВРЕМЕН ОЖИДАНИЯ В СИСТЕМАХ ОБСЛУЖИВАНИЯ 
$M/G/1$$^*$}

\def\titkol{Сравнение экстремальных индексов времен ожидания в~системах обслуживания 
$M/G/1$}

\def\aut{И.\,В.~Пешкова$^1$}

\def\autkol{И.\,В.~Пешкова}

\titel{\tit}{\aut}{\autkol}{\titkol}

\index{Пешкова И.\,В.}
\index{Peshkova I.\,V.}


{\renewcommand{\thefootnote}{\fnsymbol{footnote}} \footnotetext[1]
{Работа выполнена
при финансовой поддержке РНФ (проект 21-71-10135).}}


\renewcommand{\thefootnote}{\arabic{footnote}}
\footnotetext[1]{Петрозаводский государственный университет, 
\mbox{iaminova@petrsu.ru}}

%\vspace*{-6pt}






\Abst{Доказывается теорема, согласно 
которой если исходные стационарные последовательности стохастически упорядочены
 и~существуют предельные распределения для максимумов, а~также упорядочены 
нормализующие последовательности, то их экстремальные индексы также упорядочены. 
Этот результат применен    для сравнения экстремальных индексов  стационарных  
времен ожидания в~двух сис\-те\-мах обслуживания типа $M/G/1$, в~которых входные 
потоки совпадают, а~времена обслуживания стохастически упорядочены. 
Рассмотрены три примера  сис\-тем обслуживания: с~экспоненциальным 
распределением, распределением Парето и~распределением Вейбулла времени 
обслуживания. Для этих распределений получены соотношения между параметрами, 
га\-ран\-ти\-ру\-ющие стохастическую упорядоченность распределений и~нормализующих
последовательностей.}


\KW{распределение экстремальных значений; 
экстремальный индекс; система обслуживания; стохастическая упорядоченность} 

\DOI{10.14357/19922264220109}
  
%\vspace*{-4pt}


\vskip 10pt plus 9pt minus 6pt

\thispagestyle{headings}

\begin{multicols}{2}

\label{st\stat}


\section{Введение}

Теория экстремальных значений  моделирует возникновение редких событий, таких 
как большие (малые) значения данного процесса в~течение некоторого периода 
времени~\cite{embrehts, haan, Leadbetter}. В~финансовом контексте экстремальные 
события проявляются  всякий раз, когда происходит серьезное падение фондового 
рынка. В~кредитном страховании, портфельном инвестировании,  мониторинге 
катастрофических экологических событий, прикладных проблемах обработки 
сигналов и~производительности сетей связи крайне важно оценивать  риск возникновения 
экстремальных событий. Один из эффективных подходов к~принятию решений~--- 
моделирование  событий, связанных с~экстремальными значениями случайных 
процессов.

Как показывают результаты наблюдений, часто экстремумы группируются во времени, 
т.\,е.\ образуют кластеры~\cite{Bertail}.  Такая кластеризация в~тео\-рии 
экстремальных значений характеризуется с~по\-мощью так называемого 
\textit{экстремального индекса}~\cite{Resnick}. Экстремальный индекс служит важным 
показателем, который определяет предельное распределение экстремальных значений 
строго стационарных случайных последовательностей~\cite{Leadbetter}.

Проблема изучения экстремальных значений характеристик производительности сис\-тем 
массового обслуживания неоднократно рассматривалась в~соответствующих работах 
(см., в~част\-ности,~\cite{Asmus, asmus2, iglehart, Rootzen}). Например, алгоритм 
вычисления экстремального индекса стационарного времени ожидания стационарной  
системы $G/G/1$ с~распределением, принадлежащим области притяжения распределения 
Гумбеля, приведен в~статье~\cite{Hooghiemstra}.

Главная идея данного  исследования состоит в~использовании свойства \textit{стохастической 
упорядоченности} времен обслуживания двух систем обслуживания 
типа $M/G/1$ для сравнения экстремальных индексов стационарных времен ожидания 
в~этих сис\-те\-мах.

В  некоторых частных случаях данная  проблема рассматривалась в~пред\-шест\-ву\-ющих 
работах автора~\cite{dccn2021, tomsk2021}. Например, в~работе~\cite{dccn2021} 
исследовалась сис\-те\-ма $M/G/1$ с~Парето-распределением времени обслуживания, а~в~работе~\cite{tomsk2021} 
рас\-смот\-ре\-на сис\-те\-ма $M/G/1$ c распределением  Вейбулла  
с~па\-ра\-мет\-ром формы $0< \beta <1$.
Основной результат  работы \cite{tomsk2021} заключен в~утверж\-де\-нии~1 
(в~разд.~2), которое обобщает предшествующие результаты автора.

В разд.~2 настоящей статьи
приводятся основные сведения из теории экстремальных значений. Кроме того, 
в~этом разделе доказана  теорема~1, которая  утверж\-да\-ет, что если исходные 
стационарные последовательности стохастически упорядочены,  т.\,е.\  $X_n\hm\le_{\!\mathrm{st}}\  
Y_n$ (для всех $n\hm\ge 1)$, и~соответствующие \textit{нор\-ма\-ли\-зу\-ющие 
последовательности} также упорядочены,  то экстремальные индексы этих 
последовательностей  удовле\-тво\-ря\-ют неравенству $\theta_X\hm\ge \theta_Y$.
 Этот результат  был анонсирован
в~работах~\cite{dccn2021, tomsk2021}.
В~разд.~2 также приведены условия для  па\-ра\-мет\-ров экспоненциального  
распределения, а~также распределений Парето и~Вейбулла, при которых выполнено 
указанное  неравенство для экстремальных индексов двух стационарных 
последовательностей.
В~разд.~3 сформулировано утверж\-де\-ние~1, которое   основано на применении 
теоремы~1 для  сравнения экстремальных индексов стационарных времен ожидания 
в~случае  односерверных сис\-тем обслуживания. Далее в~этом разделе  приведены   
примеры  сравнения сис\-тем обслуживания 
с~временами обслуживания, име\-ющи\-ми экспоненциальное распределение, распределение 
Парето и~распределение Вейбулла соответственно.


\section{Сравнение экстремальных индексов двух стохастически упорядоченных 
строго стационарных
последовательностей}
%\label{sec2}

Рассмотрим последовательность независимых одинаково  распределенных (н.о.р.)\ 
случайных величин (с.в.)\  $\{ X_n , \ n\ge 1\}$, заданных   общей функцией 
распределения (ф.р.)~$F$, и~обозначим~$M_n$ максимум  первых~$n$ из этих 
величин, т.\,е.\
\begin{equation*}
   M_n=\max \left(X_1, \dots , X_n\right).
\end{equation*}
Очевидно,   функция распределения~$M_n$ имеет сле\-ду\-ющий вид:
 \begin{multline*}
\mathbb{P} \left(
M_n \le x\right)= \mathbb{P} \left(X_1\le x, X_2\le x, \dots, X_n\le x\right) ={}\\
{}= F^n (x),\enskip 
n\ge1\,.
\end{multline*}
Известно~\cite{Leadbetter}, что   если для некоторых последовательностей 
констант $b_n, a_n\hm>0$,  $n\hm\ge 1$, и~некоторой  невырожденной ф.р.~$G$ 
выполняется сле\-ду\-ющее асимптотическое выражение:
  \begin{equation}
  \label{max1}
       \mathbb{P}\left(\fr{M_n-b_n}{a_n} \le x\right) \to G(x), \enskip \ n\to\infty\,,
  \end{equation}
 то говорят, что ф.р.~$F$ относится к~\textit{об\-ласти притяжения ф.р.}~$G$.

Распределения $G$, удовлетворяющие  соотношению~(\ref{max1}), называются \textit{распределениями 
экстремального типа}  и~имеют сле\-ду\-ющую обобщенную форму~\cite{embrehts}:
\begin{multline*}
%\label{gev}
    \mathbb{P}(X \le x)=: H(x)={}\\
    {}=
    \begin{cases}
    \exp \left (-\left( 1+\eta  \fr{x-\nu}{\sigma}\right)^{-1/{\eta}} \right) & \eta\ne 0\,;\\[12pt]
\exp \left ( - \exp \left(- \fr{x-\nu}{\sigma}\right)  \right) & \eta=0\,,
\end{cases}
\end{multline*}
где  $ 1+ \eta (x-\nu)/\sigma \hm> 0$.
Заметим, что
если   $\eta \hm>0$, то   $H$~--- это распределение Фреше, если $\eta \hm< 0$,  
то~$H$~--- это обратное распределение Вейбулла, и~если $\eta \hm=0$, то  $H$~--- это 
распределение Гумбеля.



Пусть $0\le \tau\hm \le \infty$ и~предположим, что  существует такая 
последовательность вещественных чисел
$\{u_n, \ n\ge 1\}$, что
\begin{equation}
\label{max43}
    n\overline F(u_n) \to \tau \mbox{ при } n\to \infty\,,
\end{equation}
где $\overline F:=1-F$~--- хвост ф.р.~$F$. Тогда \cite{Leadbetter}
\begin{equation}
\label{max2}
\mathbb{P} \left(M_n \le u_n \right) \to e^{-\tau} \ \mbox{при} \ n\to \infty
\end{equation}
и,~обратно, условие~(\ref{max2}) (для некоторого $0\hm\le \tau \hm\le \infty$) влечет 
выполнение~(\ref{max43}).

В случае, когда выполнено соотношение~(\ref{max1}),  сходимость~(\ref{max2}) 
сохраняется  для членов семейства линейных нормализующих последовательностей  
$$
u_n (x)=  a_n x + b_n,\enskip n\ge 1\,,
$$
где $x$ принимает все возможные вещественные значения. В~этом случае~(\ref{max2}) преобразуется
в~$$
\mathbb{P}\left(M_n  \le u_n(x)\right) \to \tau(x),
$$
где вид $\tau(x)$  зависит от  типа предельного распределения, а~именно:  
распределения Гумбеля $\tau(x)\hm=e^{-x}$,  распределения Фреше $\tau(x)\hm=x^{-\eta}$  
и~обратного распределения Вейбулла $\tau(x)\hm=(-x)^{\eta}$.

Приведем примеры линейных нормализующих последовательностей  $\{u_n(x)\}$ для  
некоторых распределений.
Рассмотрим экспоненциальное распределение  с~ф.р.\
$\overline F(x) \hm= e^{-\mu x}$ и~нормализующей последовательностью
 \begin{equation}
 \label{expun}
 u_n(x) = \fr{1} {\mu} \left(x + \log n\right).
 \end{equation}
Тогда можно проверить, что  максимум~$M_n$  при больших~$n$ сходится 
к~распределению Гумбеля. Распределение Парето вида
 \begin{equation}
 \label{pareto}
    F(x) = 1- \left ( \!\fr{x_0}{x_0+x}\! \right )^{\xi}\!\!, \ \xi>0\,, \  x_0 >0\,,\ 
x\ge 0\,,\!
\end{equation}
 относится к~области притяжения распределения
 Фреше, при этом нормализующая последовательность имеет сле\-ду\-ющий вид:
 \begin{equation}
 \label{un-pareto}
     u_n(x) = x_0n^{1/\xi} x - x_0.
 \end{equation}
 Распределение Вейбулла, заданное ф.р.\ вида
 \begin{equation}
\label{weibull}
F(x)=1-e^{-x^{\beta}}, \enskip \beta >0\,, \enskip x\ge  0\,,
\end{equation}
принадлежит  области притяжения распределения Гумбеля. При этом нор\-ма\-ли\-зу\-ющая 
последовательность  имеет вид:%
\begin{equation*}
 \label{weibull22}
 u_n(x) = \fr{x(\log n)^{1/\beta-1}}{\beta}  + (\log n)^{1/\beta}.
 \end{equation*}

В случае \textit{строго стационарной по\-сле\-до\-ва\-тель\-ности}~$\{X_n\}$   для получения 
асимптотического распределения  максимума 
требуются некоторые дополнительные условия на перемешивание с.в.\  (см., 
например,   тео\-ре\-му~3.7.1 в~\cite{Leadbetter}), а~именно: если для некоторого 
$\tau\hm>0$ определена последовательность $\{ u_n (\tau) \}$ и~имеет место 
сходимость  $n\overline F(u_n(\tau))\hm\to \tau$, то соотношение~(\ref{max2}) 
принимает сле\-ду\-ющий вид:
\begin{equation}
\label{max3}
\mathbb{P} \left(M_n \le u_n (\tau)\right) \to e^{-\theta \tau} \ \mbox{при} \ n\to \infty\,,
\end{equation}
где параметр  $\theta \hm\in [0, 1]$ называется \textit{экстремальным индексом} 
последовательности~$\{ X_n \}.$
При этом если~$\{ \hat X_n \}$
есть последовательность н.о.р.\ с.в.\  (называемая \textit{сопровождающей 
последовательностью}) c~тем же (общим) маргинальным   распределением~$F$, что   
и~у~исходной последовательности~$\{ X_n \}$,
то
$$
\lim\limits_{n\to \infty}\mathbb{P} \left(\hat M_n \le u_n (\tau) \right)= e^{-\tau},
$$
где $\hat M_n\hm=\max(\hat X_1,\dots , \hat X_n).$
 Таким образом, максимум~$M_n$ исходной последовательности~$\{ X_n \}$  
 и~максимум~$\hat M _n$ сопровождающей последовательности независимых 
 с.в.~$\{ \hat X_n \}$ имеют предельное распределение одного и~того же типа с~теми же 
нор\-ма\-ли\-зу\-ющи\-ми последовательностями констант $\{a_n\} ,\ \{b_n\}$, но 
различаются па\-ра\-мет\-ром показателя
 экспоненты $\theta\hm > 0$ в~формуле~(\ref{max3}).
Заметим, что если $\mathbb{P} (M_n\hm\le u_n(\tau))$ сходится хотя бы для одного значения 
$\tau \hm> 0$, то~(\ref{max3}) выполнено при всех $\tau \hm> 0$ для некоторого 
фиксированного~$\theta$, $0\hm\le \theta \hm\le 1$.

Предел~(\ref{max3}) и~сходимость  $n\overline F(u_n) \hm\to \tau$ позволяют 
получить следующее  соотношение для вычисления экстремального индекса~$\theta$:
\begin{equation}
\label{theta-1}
    \theta=\lim\limits_{n\to\infty} \fr{\log \mathbb{P} (M_n\le u_n)}{n \log F(u_n)}\,.
\end{equation}

Следующая теорема позволяет сравнить экстремальные индексы двух строго 
стационарных стохастически упорядоченных последовательностей $\{ X_n \}$ 
и~$\{ Y_n \}$, заданных  (маргинальными) ф.р.~$F_X$ и~$F_Y$ соответственно.  
Обозначим максимумы первых~$n$ значений последовательностей через
$$
M_n^X=\max\left(X_1,\ldots, X_n\right);\enskip M_n^Y=\max\left(Y_1,\dots, Y_n\right).
$$
Будем говорить, что с.в.~$X$ \textit{стохастически меньше}, чем  с.в.~$Y$, 
и~писать  $X \hm\le_{\!\mathrm{st}}\ Y$,
если хвосты  ф.р.\ связаны неравенством~\cite{Ross}
\begin{equation}
\label{max-order}
    \overline F_X(x) \le \overline F_Y(x) \ \mbox{для любого} \ x \in (-\infty, 
\infty).
\end{equation}
Для случайных последовательностей $\{ X_n \}$ и~$\{ Y_n \}$ свойство 
стохастической упорядоченности $X_n\hm\le_{st} Y_n$  также определяется  
соотношением~(\ref{max-order}), поскольку они заданы маргинальными  ф.р.~$F_X$ 
и~$F_Y$.

\smallskip

\noindent
\textbf{Теорема~1.}\
\textit{Пусть строго стационарная последовательность  $\{ X_n \}$  
задана (общим) распределением~$F_X$, а строго стационарная последовательность~$\{ Y_n \}$  
задана  распределением~$F_Y$.
Пусть \mbox{существуют} такие нормализующие последовательности}
$\left\{u_n(x)\hm=a_n x\hm+ b_n\right\}$ и $\left\{u_n'(x)\hm=a_n'x\hm+b_n'\right\},$
\textit{что $a_n, a_n' >0$,  $n\hm\ge 1$, $u_n(x), u_n'(x) \hm\to \infty$  для каж\-до\-го~$x$ 
при $n\hm\to\infty$ и}
\begin{equation}
\left.
\begin{array}{rl}
       \mathbb{P}\left(M_n^X \le u_n(x)\right)  &\to  H_X(x); \\[6pt]
        \mathbb{P}\left(M_n^Y \le u_n'(x)\right) &\to H_Y(x), \enskip  n\to\infty,
       \end{array}
       \right\}
\label{teor2-1}
\end{equation}

\vspace*{-12pt}

\noindent
\begin{multline}
n\overline F_X(u_n(x)) \to \tau(x),\enskip n\overline F_Y(u_n'(x)) \to \tau'(x), 
\\  n\to \infty,
\label{teor2-2}
\end{multline}

\noindent
\begin{equation}
u_n(x)\ge u_n'(x) \ \mbox{для всех} \  x, \ n\ge 1\,. 
\label{teor2}
\end{equation}
\textit{Пусть также  последовательности стохастически упорядочены,
т.\,е.}
\begin{equation}
\overline F_X(x) \le \overline F_Y(x)\ \mbox{для всех} \  x\,.
 \label{teor3}
\end{equation}
\textit{Тогда   экстремальные индексы~$\theta_X$ и~$\theta_Y$  последовательностей 
соответственно~$\{ X_n \} $ и~$\{ Y_n \}$ упорядочены сле\-ду\-ющим образом}:
\begin{equation}
    \label{theta_comp}
    \theta_X\ge \theta_Y.
\end{equation}


\noindent
Д\,о\,к\,а\,з\,а\,т\,е\,л\,ь\,с\,т\,в\,о\,.\ \ Напомним, что
$$
\mathbb{P} \left(M_n^X \le x\right)= F_X^n(x);\quad
\mathbb{P} \left(M_n^Y \le x\right)= F_Y^n(x).
$$
Из соотношения~(\ref{teor3}) следует, что $F_Y^n (x) \hm\le F_X^n(x)$ 
для любого~$x$ и~каждого~$n$.
Обозначим 
$$
q_n(x): =\lim\limits_{n\to\infty}\fr{\overline 
F_Y(u_n'(x))}{\overline F_X (u_n(x))}\,.
$$ 
Тогда из условия~(\ref{teor2}) 
вытекает, что $q_n(x)\hm\ge 1$ и~$$
F_Y^n \left(u_n'(x)\right) \le F_X^n\left(u_n(x)\right),\enskip x\ge 0.
$$
Следовательно,
\begin{multline*}
p_n(x) : = \lim\limits_{n\to\infty} \fr{\mathbb{P} (M_n^X \le u_n(x)) }{\mathbb{P} 
\left(M_n^Y \le u_n'(x)\right)}={}\\
{}=\lim\limits_{n\to\infty} 
\fr{F_X^n(u_n(x))}{F_Y^n(u_n'(x))}=\fr{H_X(x)}{H_Y(x)}\ge 1\,.
\end{multline*}
 Заметим, что $\overline F_X(u_n(x)) \hm \to 0$, поскольку $u_n(x)\hm\to \infty$ при 
$n\hm\to\infty$, То же самое  верно в~отношении~$\overline F_Y(u_n'(x))$.  
Используя свойство натурального логарифма $\log (1-x) \hm\sim -x$ при $x\hm\to 0$, 
вычислим отношение экстремальных индексов в~соответствии с~соотношением~(\ref{theta-1}):
\begin{multline*}
\fr{\theta_X}{\theta_Y}  = \lim\limits_{n\to\infty}  \fr{\mathbb{P} (M_n^X \le 
u_n(x)) }{n \log F_X(u_n(x))} \,\fr{n \log F_Y(u_n'(x))}{\mathbb{P} (M_n^Y \le 
u_n'(x))}={}\\
{} =  \lim\limits_{n\to\infty} \fr{F_X^n(u_n(x))}{-\overline F_X(u_n(x))} \,
 \fr{-\overline F_Y(u_n'(x))}{F_Y^n(u_n'(x))}={}\\
{}=q_n(x) p_n(x) \ge 1 \ \mbox{для всех} \ x\ge 0\,.
\end{multline*}

Таким образом, неравенство~(\ref{theta_comp}) доказано.~\hfill$\square$

\smallskip




Заметим, что для выполнения условия $u_n(x)\hm\ge u_n'(x)$ достаточно, чтобы 
выполнялось сле\-ду\-ющее соотношение:
\begin{equation}
\label{gn}
x\ge \fr{b_n'-b_n}{a_n-a_n'}\,:=g_n \ \mbox{для всех}\  n\ge 1\,.
\end{equation}
Также отметим, что неравенство~(\ref{gn}) выполнено для всех $x\hm\ge 0$,  если
$\sup\nolimits_n g_n \le 0.$

\noindent
\textbf{Примеры.}
1. Предположим, что $\{ X_n \}$ и~$\{ Y_n \}$  имеют экспоненциальное 
распределение с~параметрами~$\mu_1$ и~$\mu_2$ соответственно. В~этом случае, 
если $\mu_1 \hm\ge \mu_2$, то $X_n\hm\le_{\!\mathrm{st}}\ Y_n$. Используя соотношение~(\ref{expun}),
 легко найти, что  
 $$
 g_n=-\log n \le0\,, \enskip n\ge 1\,.
$$

2. Теперь  предположим, что  последовательности  имеют распределение Парето вида~(\ref{pareto}) 
с~па\-ра\-мет\-ра\-ми $\xi_1, x_0^1$ и~$\xi_2, x_0^2$ соответственно. 
Если $\xi_1\hm\ge  \xi_2$ и~$x_0^2>x_0^1\hm\ge 1$, то $X_n \hm\le_{\!\mathrm{st}}\ Y_n $ 
и~неравенство~(\ref{gn}) для нор\-ма\-ли\-зу\-ющих последовательностей вида~(\ref{un-pareto}) 
выполняется. В~этом случае из тео\-ре\-мы~1 вытекает требуемое неравенство~(\ref{theta_comp}).

3. Аналогично можно сравнить экстремальные индексы двух последовательностей 
с.в.\ с~распределением Вейбулла~(\ref{weibull}) с~па\-ра\-мет\-ра\-ми~$\beta_1$ и~$\beta_2$
 соответственно. Например, для случая $1 \hm> \beta_1 \hm\ge \beta_2 \hm> 0$ 
свойство упорядоченности  $X_n \hm\le_{\!\mathrm{st}}\ Y_n $ выполнено и~формула~(\ref{gn}) 
принимает вид:
\begin{equation*}
%\label{gn-weibull}
g_n=\fr{\log n ((\log n)^{1/\beta_2-1/\beta_1} -1)}{\beta_1 \beta_2 (\beta_2 - 
\beta_1 (\log n)^{1/\beta_2-1/\beta_1})} \le 0   \ \mbox{для}\ n\ge 3\,.
\end{equation*}
Следовательно, также выполнено неравенство~(\ref{theta_comp}) для экстремальных 
индексов.


\section{Сравнение экстремальных индексов стационарных времен ожидания 
систем обслуживания  $M/G/1$}
%\label{3}

В этом разделе  покажем, как тео\-ре\-му~1 можно применить для сравнения 
экстремальных индексов стационарных времен ожидания односерверных систем 
обслуживания  $M/G/1$. В~част\-ности,  продемонстрируем это на примере  двух 
систем  обслуживания  с~одинаковым входным пуассоновским потоком  заявок  
и~временами обслуживания, име\-ющи\-ми  либо экспоненциальное распределение 
с~па\-ра\-мет\-ра\-ми $\mu_1\hm\ge \mu_2$, либо распределение Парето с~па\-ра\-мет\-ра\-ми  
$\xi_1\hm\ge \xi_2$ и~$x_0^2\hm>x_0^1 \hm\ge 1 $, либо
распределение Вейбулла с~параметрами, удовлетворяющими неравенствам $1 \hm> \beta_1 \hm\ge \beta_2\hm >0$.

Рассмотрим две  системы обслуживания $M/G/1$:   $\Sigma^{(1)}$ и~$\Sigma^{(2)}$. 
(Будем обозначать индексом~$i$ величины, относящиеся к~$i$-й сис\-те\-ме.) В~обеих 
сис\-те\-мах дисциплина обслуживания~--- первым пришел, первым обслужен. Обозначим
  $S^{(i)}$~--- типичное время обслуживания и~$T^{(i)}$~--- типичные интервалы 
между приходами заявок, $\mathbb{E}\,T^{(i)}\hm=1/\lambda,$ $i=\hm1,2$.

  Теперь  сравним  в~системах  $\Sigma^{(1)}$ и~$\Sigma^{(2)}$ экстремальные 
индексы стационарных времен ожидания.
Пусть   $\nu_n^{(i)}$~--- чис\-ло заявок в~сис\-те\-ме,   $Q_n^{(i)}$~---  \textit{размер 
очереди} и~$W_n^{(i)}$~---   \textit{время ожидания в~очереди} в~момент прихода 
в~сис\-те\-му~$\Sigma^{(i)}$
заявки с~номером~$n$, $i\hm=1,2$.
Обозначим (в~случае их существования) пределы по распределению
\begin{multline*}
Q_n^{(i)} \Rightarrow Q^{(i)},\enskip \nu_n^{(i)} \Rightarrow \nu^{(i)}, \enskip 
W_n^{(i)} \Rightarrow W^{(i)} ,\\
 n\to \infty,\  i=1,2\,.
\end{multline*}
Эти  пределы существуют, в~част\-ности,  если времена между приходами заявок~$T^{(i)}$,
$i\hm=1,2$,  являются \textit{нерешетчатыми}  
и~$\rho_i\hm=\lambda_i \mathbb{E} S^{(i)} \hm<1$, $i\hm=1,2$~\cite{Asmus}.
Обозначим максимумы
\begin{align*}
W_n^{(1)*}&=\max\left(W_1^{(1)},\dots, W_n^{(1)}\right),\\
W_n^{(2)*}&=\max\left(W_1^{(2)},\dots, W_n^{(2)}\right),\enskip n\ge 1\,.
\end{align*}

\noindent
\textbf{Утверждение~1.}\
\textit{Предположим, что для сис\-тем $\Sigma^{(1)}$ и~$\Sigma^{(2)}$  
коэффициенты загрузки  $\rho_i\hm<1$, 
$i\hm=1,2$, и~выполнены  сле\-ду\-ющие стохастические соотношения}:
\begin{equation}
\label{ar0:3}
   \nu_1^{(1)}=\nu_1^{(2)}=0;\enskip  T^{(1)}{=_{\mathrm{st}}}
T^{(2)};\enskip S^{(1)}{\le}_{\mathrm{st}}\, S^{(2)}.
\end{equation}
\textit{Тогда если  существуют такие нормализующие последовательности $\{u_n(x)\hm=a_n x\hm+ 
b_n\}$ и~$\{u_n'(x)\hm=a_n'x\hm+b_n'\}$, что $a_n, a_n'\hm >0$, $n\hm\ge 1$, для каждого~$x$ 
$\ u_n(x), u_n'(x) \hm\to \infty$ и~выполнены соотношения}~(\ref{teor2-1})--(\ref{teor2}) 
\textit{для  максимумов стационарных времен ожидания~$W_n^{(1)*}$ и~$W_n^{(2)*}$, то}
\begin{equation}
\label{extr-wait}
  \theta_{W^{(1)}}\ge   \theta_{W^{(2)}},
\end{equation}
\textit{где $\theta_{W^{(i)}}$~--- экстремальные индексы}~$W_n^{(1)*}$,  $i\hm=1,2$.

\smallskip


Поясним коротко схему доказательства утверж\-де\-ния~1. Согласно~\cite{Whitt}, 
соотношения~(\ref{ar0:3}) гарантируют, что
\begin{equation*}
%\label{ar0:6}
 Q_n^{(1)} \le_{\mathrm{st}} Q_n^{(2)},\enskip  W_n^{(1)} \le_{\mathrm{st}} W_n^{(2)},\  n\ge \,1.
\end{equation*}
Следовательно, условие~(\ref{teor3}) тео\-ре\-мы~1  выполнено, 
что влечет истинность~(\ref{theta_comp}) 
для экстремальных индексов стационарных времен ожидания 
рас\-смат\-ри\-ва\-емых сис\-тем.

\smallskip

Рассмотрим практические примеры использования утверж\-де\-ния~1. Пусть времена 
обслуживания имеют экспоненциальное распределение с~па\-ра\-мет\-ра\-ми~$\mu_1$ 
и~$\mu_2$ соответственно. В~этом случае  выражение  для экстремального индекса 
стационарного времени ожидания известно в~явном виде~\cite{Hooghiemstra}:
\begin{equation*}
%\label{theta_exp}
  \theta_{W^{(i)}} = \left(1-\rho_i\right)^2, \enskip i=1,2\,.
\end{equation*}
Легко показать, что предельным распределением максимумов~$W_n^{(1)*}$ является 
распределение Гумбеля.  Если $\mu_1\hm\ge \mu_2$, то   времена обслуживания, 
а~следовательно, и~стационарные времена ожидания  стохастически упорядочены, 
$W^{(1)}\hm{\ge_{\mathrm{st}}} W^{(2)}$. Поскольку $\rho_1\hm\le \rho_2$, то экстремальные 
индексы удовле\-тво\-ря\-ют неравенству~(\ref{extr-wait}).


Теперь  предположим, что  времена обслуживания  имеют распределение Парето вида~(\ref{pareto}) 
с~па\-ра\-мет\-ра\-ми $\xi_1, x_0^1$ и~$\xi_2, x_0^2$ соответственно. 
Если
$\xi_1\hm\ge  \xi_2$ и~$x_0^2\hm>x_0^1\ge 1$,  то   времена обслуживания, 
а~следовательно, и~стационарные времена ожидания  стохастически упорядочены.
В~работе~\cite{dccn2021} показано, что
распределение хвоста стационарного времени ожидания имеет сле\-ду\-ющий  вид:
\begin{equation*}
\mathbb{P}  \left(W^{(i)} > x\right) \sim \fr{\lambda x_0^i}{\xi_i-1-\lambda x_0^i} \! \left( 
\fr{x_0^i} {x_0^i +x} \right )^{\!\xi_i-1}\!, \ i=1,2\,,
\end{equation*}
нормализующие последовательности  $u_n^i(x)\hm=x_0^i n^{1/(\xi_i-1)}x \hm- x_0^i$ 
гарантируют, что при $n\hm\to\infty$
\begin{equation*}
%\label{limitw}
    n\mathbb{P} \left(W^{(i)} > u_n^i (x)\right) \to  \fr{\lambda x_0^i }{\xi_i-1-\lambda 
x_0^i}x^{-\xi_i+1}, \enskip i=1,2\,.
\end{equation*}
Таким образом, асимптотическое распределение максимума стационарного времени 
ожидания $W_n^{(i)*}$ имеет распределение  Фреше с~параметром  $\xi_i\hm-1$ 
и,~более того, при $\xi_1\hm\ge  \xi_2$ и~$x_0^2\hm>x_0^1\hm\ge 1$ выполнено неравенство для 
экстремальных индексов~(\ref{extr-wait}).


 Теперь рассмотрим две сис\-те\-мы, в~которых времена обслуживания имеют 
распределение Вейбулла с~па\-ра\-мет\-ра\-ми~$\beta_1$ и~$\beta_2$ соответственно. Пусть 
$1 \hm> \beta_1 \hm\ge \beta_2 \hm>0$. В~работе~\cite{tomsk2021} показано, что   
распределение максимума стационарного времени ожидания сходится к~распределению 
типа Гумбеля:
\begin{multline*}
%\label{weib5}
 \mathbb{P} \left(W_n^{(i)*} \le u_n^i(x)\right)\to  \exp
 - \fr{\lambda }{\beta_i-\lambda \Gamma
 (1/\beta_i)\beta_i}  e^{-x}\\
  \mbox{при }  n\to\infty\,.
    \end{multline*}
    Здесь
  $
\Gamma(t)=\int\nolimits_0^{\infty} e^{-y}y^{t-1} dy$~--- гам\-ма-функ\-ция, 
а~нормализующая последовательность имеет сле\-ду\-ющий вид:
\begin{equation*}
%\label{weibull-2}
 u_n^i(x) = a_n^i x + b_n^i = \fr{x(\log n)^{1/\beta_i-1}}{\beta_i}  + (\log 
n)^{1/\beta_i}.
 \end{equation*}
При этом соотношения между параметрами\linebreak $1 \hm> \beta_1 \hm\ge \beta_2 \hm>0$ гарантируют 
стохастическую упорядоченность времен обслуживания, а~следовательно, 
и~стационарных времен ожидания, а~также упорядоченность нормализующих 
последовательностей  для $n\hm\ge 3$, вследствие чего  экстремальные индексы 
стационарных времен ожидания удовлетворяют неравенству~(\ref{extr-wait}).



\section{Заключение}

В работе показано, что если исходные стационарные последовательности 
стохастически упорядочены и~существуют предельные распределения для максимумов 
этих последовательностей, а~также упорядочены соответствующие нор\-ма\-ли\-зу\-ющие 
последовательности, то экстремальные индексы исходных стационарных 
последовательностей упорядочены. Этот результат используется  для сравнения 
экстремальных индексов  стационарных  времен ожидания в~двух сис\-те\-мах 
обслуживания $M/G/1$, в~которых входные потоки совпадают, а~времена обслуживания 
стохастически упорядочены.  Рассмотрены примеры сис\-тем обслуживания, в~которых 
параметры времен обслуживания подобраны так, что времена обслуживания 
стохастически упорядочены, нормализующие последовательности их максимумов также 
упорядочены, а~следовательно, упорядочены стационарные времена ожидания и~их 
экстремальные индексы.


{\small\frenchspacing
 {%\baselineskip=10.8pt
 %\addcontentsline{toc}{section}{References}
 \begin{thebibliography}{99}
 \bibitem{Leadbetter}  %1
\Au{Leadbetter M.\,R., Lindgren~G., Rootzin~H.}  Extremes 
and related properties of random sequences
and processes.~--- New York, NY, USA: Springer,  1983. 336~p.


\bibitem{embrehts} %2
\Au{Embrechts P., Kluppelberg~C., Mikosch~T.} Modelling extremal events for 
insurance and finance. Applications of mathematics.~--- Berlin, Heidelberg:   
Springer, 1997. 660~p.

\bibitem{haan} %3
\Au{de Haan L., Ferreira~A.}  Extreme value theory:  An introduction.~---   New 
York, NY, USA:  Springer Science\;+\;Business Media LLC, 2006. 491~p.


\bibitem{Bertail} %4
\Au{Bertail P., Clemencon~S., Tressou~J.}
Extreme values statistics for Markov chains via the
(pseudo-) regenerative method~// Extremes, 2009.  Vol.~12. Iss.~4. P.~327--360.   
doi: 10.1007/s10687-009-0081-y.

\bibitem{Resnick}  %5
\Au{Resnick S.}  Extreme values, regular variation and point 
processes.~--- New York, NY, USA: Springer,  1987. 320~p.




\bibitem{iglehart}  %6
\Au{Iglehart D.\,L.}  Extreme values in GI/$G$/1 queue~// Ann. 
Math. Stat., 1972. Vol.~3. Iss.~2. P.~627--635. doi: 10.1214/aoms/1177692642.




\bibitem{Rootzen}  %7
\Au{Rootzen H.} Maxima and exceedances of stationary Markov 
chains~// Adv. Appl. Probab., 1988. Vol.~20. Iss.~2. P.~371--390. 
doi: 10.2307/1427395.

\bibitem{asmus2} %8
\Au{Asmussen S.} Extreme value theory for queues via cycle maxima~// Extremes, 
1998. Vol.~1. Iss.~2. P.~137--168. doi: 10.1023/A:1009970005784.

\bibitem{Asmus} %9
\Au{Asmussen S.} Applied probability and queues. Stochastic modelling and 
applied probability.~--- New York, NY, USA: Springer-Verlag, 2003. 438~p.



\bibitem{Hooghiemstra} %10
\Au{Hooghiemstra G.,  Meester~L.\,E.} Computing the extremal index of special
Markov chains and queues~// Stoch. Proc. Appl., 1996. 
Vol.~65. Iss.~2. P.~171--185. doi: 10.1016/ S0304-4149(96)00111-1.



\bibitem{dccn2021} %11
\Au{Peshkova I., Morozov~E., Maltseva~M.} On regenerative 
estimation of extremal index in queueing systems~// Distributed computer and 
communication networks: Control, computation, communications~/
Eds. V.\,M.~Vishnevskiy,
K.\,E.~Samouylov, D.\,V.~Kozyrev.~--- 
Lecture notes in computer science ser.~--- Cham, Switzerland: 
Springer, 2021. Vol.~13144. P.~251--264. doi: 10.1007/978-3-030-92507-$9\_21$.

\bibitem{tomsk2021}
\Au{Peshkova I., Morozov~E., Maltseva~M.}  
On comparison of waiting time 
extremal indexes  in queueing systems with Weibull service times~// 
Comm. Com. Inf. Sc., 2022 (in press).






\bibitem{Ross}
\Au{Ross S., Shanthikumar~J., Zhu~Z.}  On increasing-failure-rate random 
variables~// J.~Appl. Probab., 2005. Vol.~42. P.~797--809. doi: 
10.1239/jap/1127322028.


\bibitem{Whitt}
\Au{Whitt W.} Comparing counting processes and queues~// Adv. Appl. Probab., 
1981. Vol.~13. P.~207--220.  doi: 10.2307/1426475.

\end{thebibliography}

 }
 }

\end{multicols}

\vspace*{-7pt}

\hfill{\small\textit{Поступила в~редакцию 09.01.22}}

\vspace*{6pt}

%\pagebreak

%\newpage

%\vspace*{-28pt}

\hrule

\vspace*{2pt}

\hrule

%\vspace*{-2pt}

\def\tit{THE COMPARISON OF WAITING TIME EXTREMAL INDEXES IN~$M/G/1$ QUEUEING SYSTEMS}


\def\titkol{The comparison of waiting time extremal indexes in~$M/G/1$ queueing systems}


\def\aut{I.\,V.~Peshkova}

\def\autkol{I.\,V.~Peshkova}

\titel{\tit}{\aut}{\autkol}{\titkol}

\vspace*{-15pt}


\noindent 
Petrozavodsk State University, 33~Lenina Pr., Petrozavodsk 185910, Russian Federation


\def\leftfootline{\small{\textbf{\thepage}
\hfill INFORMATIKA I EE PRIMENENIYA~--- INFORMATICS AND
APPLICATIONS\ \ \ 2022\ \ \ volume~16\ \ \ issue\ 1}
}%
 \def\rightfootline{\small{INFORMATIKA I EE PRIMENENIYA~---
INFORMATICS AND APPLICATIONS\ \ \ 2022\ \ \ volume~16\ \ \ issue\ 1
\hfill \textbf{\thepage}}}

\vspace*{3pt} 

%Последовательность Времен ожидания задается известной рекурсией Линдли:
%\begin{equation}\label{lindley}
%    W_{n+1}^{(i)}=(W_n^{(i)}+S_n^{(i)}-T_n^{(i)})^+, \quad n\ge 1,\ i=1,2,
%\end{equation}
%здесь предполагается, что $W_1^{(i)}=0$  (рассматривается процесс с~нулевой 
%задержкой), и~$(\cdot)^+=\max(0,\cdot)$. Заметим, что рекурсия \eqref{lindley} 
%определяет оставшуюся работу в~моменты $\{t_n^{-}\}$,  т. е. перед приходом.



\Abste{The theorem  which states that if the initial stationary sequences 
are stochastically ordered, there are limiting distributions for maxima and the normalizing 
sequences are ordered, then their extreme indexes are also ordered is proved. This 
result is applied to compare the extreme indexes of stationary waiting times in two $M/G/1$ 
 systems with the same input flows and stochastically ordered service times. 
 Three examples of queueing systems with exponential distribution, Pareto distribution, 
 and Weibull distribution of service times are considered. For these distributions, 
 the relations between the parameters guaranteeing the stochastic ordering of the 
 distributions and the normalizing sequences are obtained.}


\KWE{extreme value distributions; extremal index; queueing system; stochastic ordering} 





\DOI{10.14357/19922264220109}

\vspace*{-16pt}

\Ack

\vspace*{-2pt}
\noindent
The research has been prepared with the support of the Russian Science Foundation according to
 the research project No.\,21-71-10135. 



%\vspace*{6pt}

  \begin{multicols}{2}

\renewcommand{\bibname}{\protect\rmfamily References}
%\renewcommand{\bibname}{\large\protect\rm References}

{\small\frenchspacing
 {%\baselineskip=10.8pt
 \addcontentsline{toc}{section}{References}
 \begin{thebibliography}{99}
 
 \vspace*{-2pt}
 
 \bibitem{Leadbetter-1}  %1
\Aue{Leadbetter, M.\,R., G.~Lindgren, and H.~Rootzen.}
 1983. \textit{Extremes and related properties of random sequences and processes}. New York, NY: 
 Springer. 336~p.

\bibitem{embrehts-1} %2
\Aue{Embrechts, P., C.~Kluppelberg, and T.~Mikosch.} 1997. 
\textit{Modelling extremal events for insurance and finance}. Berlin, Heidelberg:   Springer. 660~p.

\bibitem{haan-1} %3
\Aue{de Haan, L., and A.~Ferreira.} 2006.  \textit{Extreme value theory:  An introduction}.
New York, NY:  Springer Science\;+\;Business Media LLC. 491~p.



\bibitem{Bertail-1} %4
\Aue{Bertail, P., S.~Clemencon, and J.~Tressou.} 2009.
Extreme values statistics for Markov chains via the (\mbox{pseudo-}) 
regenerative method. \textit{Extremes} 12(4):327--360. doi: 10.1007/s10687-009-0081-y.

\bibitem{Resnick-1} %5
\Aue{Resnick, S.} 1987.
\textit{Extreme values, regular variation and point processes}. 
New York, NY: Springer. 320~p. 




\bibitem{iglehart-1} %6
\Aue{Iglehart, D.\,L.} 1972. 
Extreme values in GI/$G$/1 queue. \textit{Ann. Math. Stat.} 3(2):627--635. doi: 10.1214/aoms/ 1177692642.


\bibitem{Rootzen-1}  %7
\Aue{Rootzen, H.} 1988.
Maxima and exceedances of stationary Markov chains.
{\it Adv. Appl. Probab.} 20(2):371--390. doi: 10.2307/1427395.

\bibitem{asmus2-1} %8
\Aue{Asmussen, S.} 1998. Extreme value theory for queues via cycle maxima. 
\textit{Extremes} 1(2):137--168. doi: 10.1023/ A:1009970005784.

\bibitem{Asmus-1} %9
\Aue{Asmussen, S.} 2003. \textit{Applied probability and queues}. New York, NY: Springer-Verlag. 438~p. 



\bibitem{Hooghiemstra-1} %10
\Aue{Hooghiemstra, G., and  L.\,E.~Meester.} 1996. Computing the extremal index of special
Markov chains and queues. \textit{Stoch. Proc. Appl.} 65(2):171--185. 
doi: 10.1016/S0304-4149(96)00111-1.


\bibitem{dccn2021-1} 
\Aue{Peshkova, I., E. Morozov, and M. Maltseva.} 2021. On regenerative estimation of extremal 
index in queueing systems. 
\textit{Distributed computer and communication networks}. Eds. V.\,M.~Vishnevskiy,
K.\,E.~Samouylov, and D.\,V.~Kozyrev. 
Lecture notes in computer science ser. Springer. 13144:251--264. doi: 10.1007/978-3-030-92507-9\_21.

\bibitem{tomsk2021-1}
\Aue{Peshkova, I., E.~Morozov, and M.~Maltseva.} 2022 (in press).  
On comparison of waiting time extremal indexes  in queueing systems with Weibull service times. 
\textit{Comm. Com. Inf. Sc.} 


\bibitem{Ross-1}
 \Aue{Ross, S., J.~Shanthikumar, and Z.~Zhu.} 2005.  On increasing--failure--rate random variables.
 \textit{J.~Appl. Probab.} 42:797--809. doi: 10.1239/jap/1127322028.


\bibitem{Whitt-1}
\Aue{Whitt, W.} 1981. Comparing counting processes and queues. 
\textit{Adv. Appl. Probab.} 13:207--220.  doi: 10.2307/ 1426475.
\end{thebibliography}

 }
 }

\end{multicols}

\vspace*{-6pt}

\hfill{\small\textit{Received January 9, 2022}}

%\pagebreak

%\vspace*{-18pt}

\Contr

\noindent
\textbf{Peshkova Irina V.} (b.\ 1975)~--- Candidate of Science (PhD) in physics
and mathematics, associate professor,
 Petrozavodsk State University, 33~Lenina Pr., Petrozavodsk 185910, Russian Federation;
\mbox{iaminova@petrsu.ru}

\label{end\stat}

\renewcommand{\bibname}{\protect\rm Литература} 