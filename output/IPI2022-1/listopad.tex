\def\stat{listopad}

\def\tit{РАЗРЕШЕНИЕ КОНФЛИКТОВ В~ГИБРИДНЫХ ИНТЕЛЛЕКТУАЛЬНЫХ  
МНОГОАГЕНТНЫХ СИСТЕМАХ}

\def\titkol{Разрешение конфликтов в~гибридных интеллектуальных  
многоагентных системах}

\def\aut{С.\,В.~Листопад$^1$, И.\,А.~Кириков$^2$}

\def\autkol{С.\,В.~Листопад, И.\,А.~Кириков}

\titel{\tit}{\aut}{\autkol}{\titkol}

\index{Листопад С.\,В.}
\index{Кириков И.\,А.}
\index{Listopad S.\,V.}
\index{Kirikov I.\,A.}


%{\renewcommand{\thefootnote}{\fnsymbol{footnote}} \footnotetext[1]
%{Работа выполнена при поддержке Министерства науки и~высшего образования Российской Федерации (проект 
%075-15-2020-799).}}


\renewcommand{\thefootnote}{\arabic{footnote}}
\footnotetext[1]{Калининградский филиал Федерального исследовательского центра <<Информатика 
  и~управ\-ле\-ние>> Российской 
академии наук, \mbox{ser-list-post@yandex.ru}}
\footnotetext[2]{Калининградский филиал Федерального исследовательского центра 
      <<Информатика и~управ\-ле\-ние>>
       Российской академии наук, \mbox{baltbipiran@mail.ru}}

%\vspace*{-4pt}

       
  
  \Abst{Рассматривается алгоритм снижения интенсивности и~разрешения конфликтов, 
возни\-ка\-ющих в~гибридных интеллектуальных многоагентных сис\-те\-мах (\mbox{ГиИМАС}). Предлагаемый 
алгоритм~--- составная часть метода управления проб\-лем\-но-
и~про\-цес\-сно-ори\-ен\-ти\-ро\-ван\-ны\-ми конфликтами в~таких системах. Данный метод 
позволяет идентифицировать ситуации принятия решений, при необходимости 
стимулировать и~впоследствии разрешать с~помощью предлагаемого алгоритма 
конструктивные формы конфликта, а~так\-же предотвращать его деструктивные формы. 
Стимулирование конфликта совместно с~ге\-те\-ро\-ген\-ностью сис\-те\-мы по моделируемым 
агентами знаниям специалистов и~используемым методам обеспечивает всестороннее 
рассмотрение поставленной проб\-ле\-мы. Предлагаемый алгоритм снижения интенсивности 
и~разрешения конфликтов позволяет агентам автоматически согласовать свои позиции 
и~выработать единое коллективное решение, чтобы избавить пользователя от 
необходимости ручного анализа и~выбора решения из массива альтернатив, предлагаемых 
отдельными агентами.}
  
  \KW{конфликт; гибридная интеллектуальная многоагентная система; коллектив 
специалистов; разрешение конфликта}

\DOI{10.14357/19922264220108}
  
\vspace*{-3pt}


\vskip 10pt plus 9pt minus 6pt

\thispagestyle{headings}

\begin{multicols}{2}

\label{st\stat}

\section{Введение}

\vspace*{-3pt}

  На сегодняшний день быстрое и~точное решение проблем, возникающих при 
управлении сложными со\-ци\-аль\-но-эко\-но\-ми\-че\-ски\-ми сис\-те\-ма\-ми, 
стало одним из важнейших элементов конкурентоспособности компании~[1]. 
При этом последствия принимаемых решений в~условиях неопределенности, 
риска и~динамической среды не могут быть точно спрогнозированы. Поэтому 
все б$\acute{\mbox{о}}$льшую актуальность приобретает разработка 
и~применение интеллектуальных сис\-тем поддержки принятия решений, 
собирающих и~анализирующих данные, предос\-тав\-ля\-ющих пользователю 
альтернативы, релевантные проб\-лем\-ной ситуации, и~их оценки. Для 
по\-стро\-ения таких систем в~[2] предложен подход на основе \mbox{ГиИМАС}, мо\-де\-ли\-ру\-ющих 
коллективное решение проб\-лем специалистами различных профилей <<за 
круглым столом>> путем организации взаимодействия интеллектуальных 
агентов~--- относительно автономных программных сущностей, об\-ла\-да\-ющих 
достаточно развитыми моделями предметной\linebreak об\-ласти и~целеполагания. 
Гибридные интеллектуальные многоагентные сис\-те\-мы 
интегрируют подход гиб\-рид\-ных интеллектуальных сис\-тем 
А.\,В.~Колесникова~[2], обеспечивающих учет \mbox{неоднородности} решаемой 
проблемы, и~аппарат многоагентных сис\-тем (МАС) в~смысле В.\,Б.~Тарасова~[3], 
поз\-во\-ля\-ющий моделировать макроуровневые процессы и~явления 
в~коллективах спе\-ци\-а\-ли\-стов. 
  
  Одна из особенностей \mbox{ГиИМАС}~--- возможность моделирования 
проблемно- и~про\-цес\-сно-ори\-ен\-тиро\-ван\-ных конфликтов между 
агентами, кото-\linebreak рые, как показано в~[4], поз\-во\-ля\-ют всесторонне\linebreak рассмотреть 
проблему и~адаптировать под нее процессы выработки решения системой. 
Причины возникновения конфликтов среди агентов сис\-те\-мы~--- 
противоречия между целями, методами, моделями предметной об\-ласти, 
наборами данных, используемыми отдельными агентами при подготовке 
решений подзадач и~согласовании правил взаимодействия для выработки 
итогового коллективного решения~[1]. Если агенты не могут договориться 
между собой, пользователь сталкивается с~дополнительной задачей ручного 
анализа и~выбора решения из массива альтернатив, пред\-ла\-га\-емых отдельными 
агентами, что обесценивает результаты работы \mbox{ГиИМАС}. Таким 
образом, без автоматического разрешения конфликтов 
агентов \mbox{ГиИМАС} не может рас\-смат\-ри\-вать\-ся как эффективная сис\-те\-ма 
поддержки принятия решений. 

Цель настоящей работы~--- разработка 
алгоритма снижения интенсивности и~разрешения 
конфликтов между агентами в~рамках модели \mbox{ГиИМАС} с~проб\-лем\-но-  
и~про\-цес\-сно-ори\-ен\-ти\-ро\-ван\-ны\-ми конфликтами.

  \vspace*{-6pt}
  
\section{Подходы к~разрешению конфликтов в~системах  
распределенного искусственного интеллекта}

  \vspace*{-2pt}

  Вопросы разрешения конфликтов в~сис\-те\-мах распределенного 
искусственного интеллекта связаны с~преодолением множества проб\-лем, 
таких как выбор агентом стратегии поведения, оценка интересов 
и~прогнозирование поведения других агентов, обеспечение спра\-вед\-ли\-вости 
результатов разрешения конфликта, наличие в~сис\-те\-ме <<эгоистичных>> 
агентов, неоднородность и~взаимозависимость целевых функций агентов, 
согласование противоречивых аргументов, повторяющиеся взаимодействия, 
доверие между агентами и~другие~[5, 6]. В~[7] предложен метод выбора 
стратегии поведения агента при разрешении конфликтов в~МАС 
в~зависимости от типа конфликта (конфликт по целям, планам 
или убеж\-де\-ни\-ям), степени автономности агента и~целевой функции~[8]. В~[9] 
рассматриваются вопросы разрешения конфликтов, связанных с~совместным 
использованием ресурсов и~не\-со\-вмес\-ти\-мостью целей агентов, путем выбора 
одной из шести стратегий в~зависимости от загруженности\linebreak сети обмена 
сообщениями: арбитраж, самомодификация (независимость), централизация, 
переговоры, соглашение о~приоритете и~взаимное согласование. В~[10] 
представлены четыре базовые \mbox{упрощенные} стратегии разрешения конфликтов 
в~МАС: переговоры, арбитраж, самомодификация и~голосование. Выбор 
стратегий зависит от числа агентов, количества требуемых сообщений, 
расчетного процессорного времени, расчетного времени поиска и~принятия 
решения, а~также ожидаемого сред\-не\-го удовлетворения от принятого решения. 
Авторы~[11] предложили метод выбора стратегии разрешения конфликтов 
(ConfRSSM) с~по\-мощью многоагентного моделирования. В~рамках данного 
метода релевантность стратегий (принуждение, подчинение, делегирование, 
переговоры, консенсус) текущей конфликтной ситуации оценивается на основе 
степени уве\-рен\-ности агентов в~своих <<мнениях>> и~интенсивности 
конфликта.
  
  Анализ~\cite{5-l, 6-l, 7-l, 8-l, 9-l, 10-l, 11-l} позволяет выделить следующие 
основные стратегии разрешения конфликтов в~сис\-те\-мах распределенного 
искусственного ин\-тел\-лекта: 
  \begin{itemize}
  \item  переговоры, обеспечивающие обмен знаниями и~информацией о целях 
между двумя или более агентами для достижения взаимоприемлемого 
соглашения по убеждениям, планам действий или целям в~рамках совместной 
работы над поставленными проб\-ле\-ма\-ми~[1]. Работы в~об\-ласти 
организации переговорного процесса агентов охватывают вопросы его 
формализации (создание моделей предметной об\-ласти и~предпочтений, 
необходимых для ведения переговоров, разработка протоколов, ре\-гу\-ли\-ру\-ющих 
взаимодействие между агентами), а~так\-же разработки моделей 
интеллектуальных переговорных агентов, построенных, например, на основе 
стратегии торгов, модели оппонентов или стратегии принятия~\cite{6-l}. 
Реализация стратегии переговоров возможна, если все взаимодействующие 
агенты~--- интеллектуальные, т.\,е.\ обладают развитыми моделями предметной 
области и~под\-сис\-те\-ма\-ми целеполагания, что позволяет им принимать решения 
для достижения своих целей;


  \item делегирование, т.\,е.\ привлечение для разрешения конфликта агентов 
третьей стороны, которая не имеет воз\-мож\-ности напрямую изменять поведение 
агентов~\cite{11-l}. В~рамках стратегий\linebreak данного класса выделяются два 
подкласса: арбитраж и~посредничество. В~первом случае решение арбитра 
обязательно для принятия и~исполнения конфликтующими агентами. \mbox{В~случае} 
посредничества решения третьей стороны принимаются и~выполняются 
с~согласия конфликтующих сторон. В~рамках данной стратегии 
предполагается, что привлекаемый для разрешения конфликта агент имеет 
более развитую базу знаний и~больше возможностей для поиска решений по 
сравнению с~конфликтующими агентами~[12, 13]. Стратегия привлечения 
третьей стороны обеспечивает быстрое принятие решения и~минимальное 
число сообщений между агентами. Эта стратегия наиболее актуальна, когда 
агент не способен взаимодействовать с~другими агентами;
  \item голосование, когда каждый из агентов с~по\-мощью своих рассуждений 
генерирует ре\-ше\-ние-кан\-ди\-дат, после чего агенты голосуют по 
предложенным решениям в~соответствии со своими целевыми функциями, 
чтобы максимизировать об\-ще\-сис\-тем\-ный показатель удовлетворенности~[14];
  \item самомодификация, которая предполагает, что агент, об\-на\-ру\-жив\-ший 
конфликты с~другими агентами, меняет свое поведение вместо того, чтобы 
взаимодействовать с~ними для выработки взаимовыгодного решения. 
Достоинства данной стратегии~--- простота и~высокая скорость разрешения 
конфликтов~\cite{10-l};
  \item  подчинение, которое может применяться, когда агент добровольно 
идет на уступки другому агенту с~более высоким статусом, репутацией, опытом 
или уверенностью в~предложенном решении~\cite{11-l}. В~реальных 
коллективах такое поведение соответствует конформизму;\\[-10pt]
  \item игнорирование, т.\,е.\ отказ от усилий по разрешению конфликта, если 
его интенсивность низкая, а~конфликтующие стороны обладают равным 
статусом, репутацией и~опытом, а также низкой уверенностью в~предлагаемых 
ими решениях~\cite{11-l};
  \item  принуждение, когда агент с~более высоким статусом, репутацией, 
опытом или уверенностью в~предложенном решении принуждает другого 
агента отказаться от своего решения, что помогает снизить слож\-ность принятия 
итогового решения за счет исключения некоторых вариантов в~рамках 
некомпенсаторной стратегии~\cite{14-l}. 
  \end{itemize}
  
  \vspace*{-6pt}
  
\section{Модель гибридных интеллектуальных многоагентных 
систем с~управлением конфликтами}

  \vspace*{-2pt}

  Детально модель \mbox{ГиИМАС}, моделирующей проб\-лем\-но-  
и~про\-цес\-сно-ори\-ен\-ти\-ро\-ван\-ные конфликты, представлена в~\cite{4-l}. 
Рассмотрим основные ее элементы, необходимые для описания алгоритма 
снижения ин\-тен\-сив\-ности и~разрешения конфликтов. Формально 
\mbox{ГиИМАС} определяется выражением~\cite{4-l}:
  \begin{equation}
  \mathrm{himas} = \langle \mathrm{AG}^*, \mathrm{env}, \mathrm{INT}, 
\mathrm{ORG}, \mathrm{MLP}\rangle\,,
  \label{e1-l}
  \end{equation}
где $\mathrm{AG}^*=\{ \mathrm{ag}_1, \ldots , \mathrm{ag}_n, \mathrm{ag}^{\mathrm{dm}}, \mathrm{ag}^{\mathrm{fc}}\}$~--- 
множество 
агентов, включающее~$n$ аген\-тов-спе\-ци\-а\-ли\-стов (АС) $\mathrm{ag}_i$, 
$i\hm\in \mathbb{N}$, $1\hm\leq i\hm\leq n$, агента $\mathrm{ag}^{\mathrm{dm}}$, 
при\-ни\-ма\-юще\-го решения (АПР), и~аген\-та-фа\-си\-ли\-та\-то\-ра (АФ) 
$\mathrm{ag}^{\mathrm{fc}}$, обеспечивающего управ\-ле\-ние конфликтами в~сис\-те\-ме; 
$\mathrm{env}$~--- концептуальная модель внешней среды системы; 
$\mathrm{INT}$~--- множество элементов структурирования 
взаимодействий агентов~\cite{4-l};\linebreak $\mathrm{ORG}$~--- множество 
архитектур \mbox{ГиИМАС}; $\mathrm{MLP}\hm= \{\mathrm{cnfm, gdid}\}$~--- 
множество концептуальных моделей \mbox{макроуровневых} процессов 
в~\mbox{ГиИМАС}, содержащее модель $\mathrm{cnfm}$ процесса управ\-ле\-ния 
\mbox{конфликтами} агентов~\cite{15-l} и~модель $\mathrm{gdid}$ идентификации 
взаимозависимости целей агентов~\cite{15-l}. 

  Агент $\mathrm{ag}_{\mathrm{id}}\hm\in \mathrm{AG}^*$ из формулы~(1) описывается 
выражением

\noindent
  $$
  \mathrm{ag_{id}}=\langle \mathrm{id}_{\mathrm{id}}^{\mathrm{ag}}, 
  \mathrm{gl}_{\mathrm{id}}^{\mathrm{ag}}, \mathrm{ACT}_{\mathrm{id}}^{\mathrm{ag}}\rangle\,,
  $$
где $\mathrm{id}_{\mathrm{id}}^{\mathrm{ag}}$~--- идентификатор агента; 
$\mathrm{gl}_{\mathrm{id}}^{\mathrm{ag}}$~--- нечеткая цель 
агента~\cite{15-l}; $\mathrm{ACT}_{\mathrm{id}}^{\mathrm{ag}}$~--- множество действий агента.
  
  Модель процесса управления конфликтами агентов описывается выражением:
  \begin{multline}
 \mathrm{cnfm} ={}\\
\!\!{}=\left\langle \mathbf{CNF}, \mathrm{cnfcl}, \mathrm{ACT^{afcm}}, 
\mathrm{ACT}^{\mathrm{agcs}}, \mathrm{ACT}^{\mathrm{agcr}}\right\rangle,\!\!\!
  \label{e2-l}
  \end{multline}
где $\mathbf{CNF}$~--- мат\-ри\-ца конфликтов~(\ref{e3-l}) между парами 
агентов; $\mathrm{cnfcl}$~--- это функ\-ция-клас\-си\-фи\-ка\-тор конфликтов~\cite{4-l}, 
необходимая для формирования мат\-ри\-цы~$\mathbf{CNF}$; 
$\mathrm{ACT}^{\mathrm{afcm}}\hm= \{ \mathrm{act}^{\mathrm{cnfm}},\linebreak \mathrm{act}^{\mathrm{cnfi}}, 
\mathrm{act}^{\mathrm{cnfs}}, 
\mathrm{act}^{\mathrm{cnfr}}\}$~--- множество функций АФ по управ\-ле\-нию конфликтами 
в~\mbox{ГиИМАС}, содержащее функцию <<управ\-ле\-ние конфликтом>> 
$\mathrm{act}^{\mathrm{cnfm}}$, обеспечивающую идентификацию $\mathrm{act^{cnfi}}$ конфликтов 
с~по\-мощью классификатора $\mathrm{cnfcl}$ и~инициа\-ли\-за\-цию функции стимуляции 
$\mathrm{act^{cnfs}}$ или разрешения $\mathrm{act^{cnfr}}$ конфликтов; 
$\mathrm{ACT}^{\mathrm{agcs}}$~--- \mbox{множество} действий АС, выполняемых при 
стимулировании противоречий АФ; $\mathrm{ACT}^{\mathrm{agcr}}\hm= \left( \{ 
\mathrm{act}_{\mathrm{ig}}^{\mathrm{agcr}}, 
\mathrm{act}_{\mathrm{sm}}^{\mathrm{agcr}}, \mathrm{act}_{\mathrm{vot}}^{\mathrm{agcr}}, 
\mathrm{act}_{\mathrm{del}}^{\mathrm{agcr}}, 
\mathrm{act}_{\mathrm{neg}}^{\mathrm{agcr}}\}, \overset{\mathrm{prf}}{\prec}\right)$~---
 упорядоченное множество 
допустимых действий (стратегий) АС по разрешению противоречий,\linebreak 
содержащее стратегию переговоров $\mathrm{act}^{\mathrm{agcr}}_{\mathrm{neg}}$, делегирования 
$\mathrm{act}_{\mathrm{del}}^{\mathrm{agcr}}$, голосования 
$\mathrm{act}_{\mathrm{vot}}^{\mathrm{agcr}}$, самомодификации 
$\mathrm{act}_{\mathrm{sm}}^{\mathrm{agcr}}$ и~игнорирования 
$\mathrm{act}_{\mathrm{ig}}^{\mathrm{agcr}}$. Отношение предпочтения 
$\overset{\mathrm{prf}}{\prec}$ задано сле\-ду\-ющим образом: $\mathrm{act}_{\mathrm{ig}}^{\mathrm{agcr}} 
\overset{\mathrm{prf}}{\prec} \mathrm{act}_{\mathrm{sm}}^{\mathrm{agcr}} 
\overset{\mathrm{prf}}{\prec} \mathrm{act}_{\mathrm{vot}}^{\mathrm{agcr}} 
\overset{\mathrm{prf}}{\prec} \mathrm{act}_{\mathrm{del}}^{\mathrm{agcr}}
\overset{\mathrm{prf}}{\prec} \mathrm{act}_{\mathrm{neg}}^{\mathrm{agcr}}$. 
Стратегии подчинения и~принуждения не моделируются в~\mbox{ГиИМАС}, так как 
предполагают наличие иерархии АС и~способствуют возникновению таких 
отрицательных эффектов, как груп\-синк и~конформизм.

\begin{figure*}[b] %fig1
  \vspace*{1pt}
  \begin{center}  
    \mbox{%
\epsfxsize=153.979mm
\epsfbox{Lis-1.eps}
}


\vspace*{6pt}

{\small Алгоритм снижения интенсивности и~разрешения конфликтов в~\mbox{ГиИМАС}}
\end{center}
\end{figure*}
  
  Конфликт между агентами, т.\,е.\ элемент матрицы $\mathbf{CNF}$ из 
выражения~(\ref{e2-l}), представляет собой следующий кортеж:
  \begin{multline}
  \mathrm{cnf}_{i j \,\mathrm{cnft}} ={}\\
  {}=\left\langle  \mathrm{ag}_i, \mathrm{ag}_j, 
  \mathrm{cnfin}, \mathrm{cnft}, \mathrm{ACT}_i^{\mathrm{agcr}}, 
\mathrm{ACT}_j^{\mathrm{agcr}}\right\rangle,
  \label{e3-l}
  \end{multline}
где $\mathrm{ag}_i$ и~$\mathrm{ag}_j$~--- это агенты-субъекты конфликта, $i,j\hm\in 
\mathbb{N}$,  $1\hm\leq i,j\hm\leq n$, $i\not= j$; $\mathrm{cnfin}$~--- напряженность 
конфликта в~виде скалярной величины $\mathrm{cnfin} \hm\in [0,1]$, вычисляемая 
\mbox{классификатором} конфликтов $\mathrm{cnfcl}$ в~соответствии с~мерой напряженности 
конфликта~\cite{4-l}; $\mathrm{cnft}$~--- символьная переменная <<тип конфликта>>, 
определенная на множестве  
$\mathrm{CNFT}=\{\mathrm{cnft}_{\mathrm{prb}}$\;=\;<<проб\-лем\-но-ори\-ен\-ти\-ро\-ван\-ный>>, 
$\mathrm{cnft}_{\mathrm{prc}}$\;=\;<<про\-цес\-сно-ори\-ен\-ти\-ро\-ван\-ный>>$\}$; 
$\mathrm{\mathrm{ACT}}_i^{\mathrm{agcr}}$ и~$\mathrm{ACT}_j^{\mathrm{agcr}}$~--- множество 
допустимых действий агентов $\mathrm{ag}_i$ и~$\mathrm{ag}_j$ соответственно по разрешению 
противоречий, $\mathrm{ACT}_i^{\mathrm{agcr}}\hm\subseteq \mathrm{ACT}_i^{\mathrm{ag}}$, 
$\mathrm{ACT}_j^{\mathrm{agcr}}\hm\subseteq \mathrm{ACT}_j^{\mathrm{ag}}$, 
$\mathrm{ACT}_i^{\mathrm{agcr}}, \mathrm{ACT}_j^{\mathrm{agcr}}\hm\subseteq 
\mathrm{ACT}^{\mathrm{agcr}}$.
  
  Функция <<управление конфликтом>> $\mathrm{act^{cnfm}}$ АФ подробно описана 
в~\cite{4-l}. Она состоит в~первоначальном определении показателя 
взаимоза\-ви\-си\-мости целей агентов, запуска функции <<идентификация 
конфликтов>> $\mathrm{act^{cnfi}}$ при получении очередного со\-об\-ще\-ния-ре\-ше\-ния 
от АС и~в зависимости от ее результатов вызова функции стимуляции 
$\mathrm{act^{cnfs}}$ или разрешения $\mathrm{act^{cnfr}}$ конфликтов. В~случае если одна из 
двух последних функций устанавливает флаг необходимости завершения 
работы, инициализируется процесс завершения работы \mbox{ГиИМАС}. 
Поскольку работа посвящена вопросам снижения интенсивности и~разрешения 
конфликтов в~\mbox{ГиИМАС}, рассмотрим функцию $\mathrm{act^{cnfr}}$ подробнее.
  
\section{Снижение интенсивности и~разрешение конфликтов 
в~гибридных интеллектуальных многоагентных системах}

  Для снижения интенсивности и~разрешения конфликтов в~\mbox{ГиИМАС} 
предлагается алгоритм, представленный на рисунке. Он обеспечивает подбор 
и~смену стратегий, рас\-смот\-рен\-ных в~разд.~2, для каждой пары 
конфликтующих агентов с~учетом характеристик конфликта между ними.
  



  Алгоритм снижения интенсивности и~разрешения конфликтов агентов 
начинается с~запроса у~конфликтующих агентов $\mathrm{ag}_i$ и~$\mathrm{ag}_j$ множеств 
реализуемых ими стратегий разрешения противоречий 
$\mathrm{ACT}_i^{\mathrm{agcr}}$ и~$\mathrm{ACT}_j^{\mathrm{agcr}}$. Агент-фа\-си\-ли\-та\-тор\linebreak
 создает локальные 
копии этих множеств $\mathrm{ACT}_i^{\mathrm{agcr}^*}\hm= \mathrm{ACT}_i^{\mathrm{agcr}}$ 
и~$\mathrm{ACT}_j^{\mathrm{agcr}^*}\hm= \mathrm{ACT}_j^{\mathrm{agcr}}$. В~блоке~2 
выполняется формирование списка (упорядоченного множества) стратегий для 
разрешения конфликта между парой агентов $\mathrm{ag}_i$ и~$\mathrm{ag}_j$ по пра\-вилу:
  \begin{equation}
    \mathrm{ACT}_{ijc}^{\mathrm{agcr}} =\mathrm{ACT}_i^{\mathrm{agcr}^*} \cap  
\mathrm{ACT}_j^{\mathrm{agcr}^*}\cap \mathrm{ACT}_{\mathrm{cnf}}^{\mathrm{agcr}}\,,
  \label{e4-l}
  \end{equation}
  где
  \begin{multline*}
  \mathrm{ACT}_{\mathrm{cnf}}^{\mathrm{agcr}} = {}\\
{}=\begin{cases}
  \{ \mathrm{act}_{\mathrm{neg}}^{\mathrm{agcr}}, \mathrm{act}_{\mathrm{del}}^{\mathrm{agcr}}\}, &
  \!\!\!\!\!\!\!\!\!\!\! \mbox{если } 
\mathrm{cnfin}>\mathrm{cnfin^{htr}}\,;\\
  \mathrm{ACT}^{\mathrm{agcr}}\,, & \hspace*{-18mm} \mbox{если } \mathrm{cnfin^{ltr}}<\mathrm{cnfin}< \mathrm{cnfin^{htr}}\\
   &\hspace*{-18mm}\mbox{и}\ \mathrm{cnft}=\mathrm{cnft_{prb}}\,;\\
  \mathrm{ACT}^{\mathrm{agcr}}\backslash 
  \left\{\mathrm{act}_{\mathrm{ig}}^{\mathrm{agcr}}\right\}, & \!\!\!\!\mbox{если } \\
&\hspace*{-87pt}\mathrm{cnfin^{ltr}}<\mathrm{cnfin} < \mathrm{cnfin^{htr}}\ \mbox{и}\ \mathrm{cnft} =\mathrm{cnft_{prc}}\,;\\
  \{\mathrm{act}_{\mathrm{ig}}^{\mathrm{agcr}}\}, & \hspace*{-17mm}\!\!\!\mbox{если } \mathrm{cnfin}< \mathrm{cnfin^{ltr}}\,.
  \end{cases}
   \end{multline*}
Здесь $\mathrm{cnfin^{ltr}}$ и~$\mathrm{cnfin^{htr}}$~--- 
нижний и~верх\-ний пороги ин\-тен\-сив\-ности конфликта, 
которые первоначально имеют сле\-ду\-ющие значения: 
$\mathrm{cnfin^{ltr}}\hm= 0{,}005$ и~$\mathrm{cnfin^{htr}}\hm= 
0{,}5$~--- и~долж\-ны быть уточнены в~ходе тестирования сис\-те\-мы. Конфликт с~ин\-тен\-сив\-ностью 
$\mathrm{cnfin}\hm< \mathrm{cnfin^{ltr}}$ считается незначительным, а~с~ин\-тен\-сив\-ностью $\mathrm{cnfin} \hm > 
\mathrm{cnfin^{htr}}$~--- сильным.

  Если список $\mathrm{ACT}_{ijc}^{\mathrm{agcr}}$ стратегий для разрешения 
конфликта, сформированный по правилу~(\ref{e4-l}), пуст, то конфликт не 
может быть разрешен, и~функция завершает свою работу. В~противном случае 
инициируется разрешение конфликтов по наиболее предпочтительной 
стратегии в~соответствии с~отношением $\overset{\mathrm{prf}}{\prec}$. Если выбрана 
стратегия <<игнорирование>>, то конфликт считается разрешенным, и~функция 
завершает работу. В~противном случае АФ ожидает  
со\-об\-ще\-ний-ре\-ше\-ний от АС, которые они выработают после применения 
соответствующей стратегии. Получив такие сообщения, АФ вновь 
идентифицирует конфликт между парой агентов $\mathrm{ag}_i$ и~$\mathrm{ag}_j$, оценивая его 
ин\-тен\-сив\-ность и~тип, с~по\-мощью функции $\mathrm{act^{cnfi}}$. Выполняется смена 
стратегии разрешения конфликта, для чего из копий множеств 
$\mathrm{ACT}_i^{\mathrm{agcr}^*}$  и~$\mathrm{ACT}_j^{\mathrm{agcr}^*}$ удаляется текущая 
стратегия и~происходит переход к~блоку~2.
  
  Таким образом, предлагаемый алгоритм дает возможность снизить 
интенсивность сильных и~умеренных конфликтов, которые не позволяют 
агентам \mbox{ГиИМАС} эффективно взаимодействовать при решении 
поставленной проблемы. В~то же время незначительные конфликты могут 
быть проигнорированы из-за того, что временн$\acute{\mbox{ы}}$е затраты на 
их окончательное разрешение превышают затраты, связанные с~нарушениями 
взаимодействия агентов из-за этих конфликтов. Наличие такого алгоритма 
в~арсенале АФ позволяет ему динамически стимулировать возникновение 
противоречий между АС, чтобы всесторонне рас\-смот\-реть по\-став\-лен\-ную  
проб\-ле\-му и~сгенерировать нестандартные решения, не <<опасаясь>>, что 
АС впоследствии не смогут согласовать свои позиции и~выработать совместные 
рекомендации для пользователя.

\section{Заключение}

  В работе показана необходимость механизма автоматического разрешения 
конфликтов между агентами \mbox{ГиИМАС}. Рассмотрены существующие\linebreak 
подходы к~разрешению конфликтов в~сис\-те\-мах\linebreak распределенного 
искусственного интеллекта. Пред\-став\-ле\-на модель \mbox{ГиИМАС} 
с~проблемно- и~про\-цес\-сно-ори\-ен\-ти\-ро\-ван\-ны\-ми конфликтами, 
имитирующая процесс коллективного решения проб\-лем на основе анализа их 
напряженности. Предложен алгоритм снижения ин\-тен\-сив\-ности и~разрешения 
конфликтов между агентами \mbox{ГиИМАС} с~использованием стратегий, 
традиционных для сис\-тем распределенного искусственного интеллекта. 
Снижение интенсивности и~разрешение конфликтов, \mbox{возникших} между 
агентами \mbox{ГиИМАС} на первых стадиях рас\-смот\-ре\-ния по\-став\-лен\-ной 
проблемы, поз\-во\-ля\-ет выработать согласованную позицию с~учетом различных 
точек зрения и~интересов специалистов, мо\-де\-ли\-ру\-емых агентами сис\-те\-мы, 
повышая таким образом ре\-ле\-вант\-ность сис\-те\-мы реальному коллективу 
специалистов. 
  
{\small\frenchspacing
 {%\baselineskip=10.8pt
 %\addcontentsline{toc}{section}{References}
 \begin{thebibliography}{99}
  \bibitem{1-l}
  \Au{Hernes M., Sobieska-Karpi$\acute{\mbox{n}}$ska~J.} A~comparative analysis of conflicts 
resolving methods in multiagent decision support systems~// Cognition Creativity Support 
Systems, 2013. Vol.~153. P.~23--32.
  \bibitem{2-l}
  \Au{Колесников А.\,В., Кириков~И.\,А., Листопад~С.\,В.} Гибридные интеллектуальные 
системы с~самоорганизацией: координация, согласованность, спор.~--- М.: ИПИ РАН, 2014. 
189~с.
  \bibitem{3-l}
  \Au{Тарасов В.\,Б.} От многоагентных сис\-тем к~интеллектуальным организациям: 
философия, психология, информатика.~--- М.: Эдиториал УРСС, 2002. 352~с.
  \bibitem{4-l}
  \Au{Листопад С.\,В., Кириков~И.\,А.} Метод идентификации конфликтов агентов 
в~гибридных интеллектуальных многоагентных сис\-те\-мах~// Сис\-те\-мы и~средства 
информатики, 2020. Т.~30. №\,1. С.~56--65. doi: 10.14357/08696527200105.
  \bibitem{5-l}
  \Au{Baarslag T., Kaisers~M., Gerding~E.\,H., Jonker~C.\,M., Gratch~J.} Computers that 
negotiate on our behalf: Major challenges for self-sufficient, self-directed, and interdependent 
negotiating agents~// {Autonomous agents and multiagent systems}~/ Eds. G.~Sukthankar,  
J.\,A.~Rodriguez-Aguilar.~--- Lecture notes in computer science ser.: Lecture notes in artificial 
intelligence subser.~--- Cham, Switzerland: Springer, 2017. Vol.~10643. P.~143--163.
  \bibitem{6-l}
  \Au{Aydo{\!\ptb{\!\v{g}}}an R., Baarslag~T., Gerding~E.} Artificial intelligence techniques for 
conflict resolution~// Group Decis. Negot., 2021. Vol.~30. P.~879--883.
  \bibitem{7-l}
  \Au{Liu T.\,H., Goel~A., Martin~C.\,E., Barber~K.\,S.} Classification and representation of 
conflict in multi-agent systems.~--- Austin, TX, USA: The University of Texas at Austin, 
1998. Technical Report of the Laboratory for Intelligent Processes and 
Systems TR98-UT-LIPS-AGENTS-01.
13~p. {\sf https://citeseerx.ist.psu.edu/viewdoc/download?\linebreak  doi=10.1.1.35.3694\&rep=rep1\&type=pdf}.
  \bibitem{8-l}
  \Au{Basheer G.\,S., Ahmad~M.\,S., Tang~A.\,Y.\,C.} A~framework\linebreak for conflict resolution in 
multi-agent systems~// Computational collective intelligence: Technologies and applications~/
Eds. C.~B$\grave{\mbox{a}}$dic$\grave{\mbox{a}}$, N.\,T.~Nguyen, 
M.~Brezovan.~--- Lecture notes in computer science ser.: Lecture notes in\linebreak artificial intelligence 
subser.~--- Berlin, Heidelberg: Springer, 2013. Vol.~8083. P.~195--204.
  \bibitem{9-l}
  \Au{Adler M.\,R., Davis~A.\,B., Weihmayer~R., Worrest~R.\,W.} Conflict-resolution strategies 
for nonhierarchical distributed agents~// Distributed artificial intelligence~II.~--- London: Pitman 
Publishing, 1989. P.~139--161. 
  \bibitem{10-l}
  \Au{Barber K.\,S., Liu~T.\,H., Han~D.\,C.} Strategic decision-making for conflict resolution in 
dynamic organized multi-agent systems~// CERA~J., 2000. Special Issue. 18~p. 
{\sf https://citeseerx.ist.psu.edu/viewdoc/\linebreak download?doi=10.1.1.144.6881\&rep=rep1\&type=pdf}.
  \bibitem{11-l}
  \Au{Tang A., Basheer~G.\,A.} Conflict Resolution Strategy Selection Method (ConfRSSM) in 
multi-agent systems~// Int. J.~Advanced Computer Science Applications, 2017. Vol.~8.  
P.~398--404.
  \bibitem{12-l}
  \Au{Ioannidis Y.\,E., Sellis~T.\,K.} Conflict resolution of rules assigning values to virtual 
attributes~// Conference (International) on the Management of Data Proceedings.~--- New York, 
NY, USA: ACM, 1989. P.~205--214.
  \bibitem{13-l}
  \Au{Ephrati~E., Rosenschein~J.\,S.} The Clarke tax as a~consensus mechanism among 
automated agents~// AAAI Proceedings, 1991. Vol.~91. P.~173--178.
  \bibitem{14-l}
  \Au{Helge G.} Decision-making strategies and self-regulated learning: Fostering  
decision-making competence in education for sustainable development: PhD Thesis.~--- 
G$\ddot{\mbox{o}}$ttingen: der Georg-August-Universit$\ddot{\mbox{a}}$t 
G$\ddot{\mbox{o}}$ttingen, 2011. 206~p.
  \bibitem{15-l}
  \Au{Листопад С.\,В., Кириков~И.\,А.} Стимуляция конфликтов агентов в~гибридных 
интеллектуальных многоагентных сис\-те\-мах~// Сис\-те\-мы и~средства информатики, 2021. 
Т.~31. №\,2. С.~47--58. doi: 10.14357/ 08696527210205.
\end{thebibliography}

 }
 }

\end{multicols}

\vspace*{-12pt}

\hfill{\small\textit{Поступила в~редакцию 10.01.22}}

\vspace*{6pt}

%\pagebreak

%\newpage

%\vspace*{-28pt}

\hrule

\vspace*{2pt}

\hrule

%\vspace*{-2pt}

\def\tit{CONFLICT RESOLUTION\\ IN~HYBRID INTELLIGENT MULTIAGENT SYSTEMS}


\def\titkol{Conflict resolution in~hybrid intelligent multiagent systems}


\def\aut{S.\,V.~Listopad and~I.\,A.~Kirikov}

\def\autkol{S.\,V.~Listopad and~I.\,A.~Kirikov}

\titel{\tit}{\aut}{\autkol}{\titkol}

\vspace*{-11pt}


  \noindent
   Kaliningrad Branch of the Federal Research Center ``Computer Science and Control'' of the 
Russian Academy of Sciences, 5~Gostinaya Str., Kaliningrad 236000, Russian Federation

\def\leftfootline{\small{\textbf{\thepage}
\hfill INFORMATIKA I EE PRIMENENIYA~--- INFORMATICS AND
APPLICATIONS\ \ \ 2022\ \ \ volume~16\ \ \ issue\ 1}
}%
 \def\rightfootline{\small{INFORMATIKA I EE PRIMENENIYA~---
INFORMATICS AND APPLICATIONS\ \ \ 2022\ \ \ volume~16\ \ \ issue\ 1
\hfill \textbf{\thepage}}}

\vspace*{3pt} 
  
   
    
   \Abste{The paper discusses an algorithm for reducing the intensity and resolving conflicts that 
arise in hybrid intelligent multiagent systems. The proposed algorithm is an integral part of the 
method for managing problem- and process-oriented conflicts in such systems. This method makes 
it possible to identify decision-making situations, stimulate, if necessary, and subsequently resolve 
constructive forms of conflict, as well as prevent its destructive forms using the proposed algorithm. 
Stimulation of the conflict together with the heterogeneity of the system according to the knowledge 
of specialists simulated by agents and the methods used by them provides an all-sided consideration 
of the problem posed. The proposed algorithm for reducing the intensity and resolving conflicts 
allows agents to reconcile automatically their positions on solving the problem and work out a 
single collective solution in order to save the user from the need for manual analysis and choosing a 
solution from the array of alternatives proposed by individual agents.}
   
   \KWE{conflict; hybrid intelligent multiagent system; team of specialists; conflict resolution}
   
\DOI{10.14357/19922264220108}

%\vspace*{-16pt}

%\Ack
%\noindent




%\vspace*{6pt}

   \begin{multicols}{2}

\renewcommand{\bibname}{\protect\rmfamily References}
%\renewcommand{\bibname}{\large\protect\rm References}

{\small\frenchspacing
 {%\baselineskip=10.8pt
 \addcontentsline{toc}{section}{References}
 \begin{thebibliography}{99}
 
 \vspace*{-2pt}
   
  \bibitem{1-l-1}
   \Aue{Hernes, M., and J.~Sobieska-Karpi$\acute{\mbox{n}}$ska.} 2013. A~comparative 
analysis of conflicts resolving methods in multiagent decision support systems. \textit{Cognition 
Creativity Support Systems} 153:23--32.
  \bibitem{2-l-1}
   \Aue{Kolesnikov, A.\,V., I.\,A.~Kirikov, and S.\,V.~Listopad.} 2014. \textit{Gibridnye 
intellektual'nye sistemy s~samoorganizatsiey: koordinatsiya, soglasovannost', spor} [Hybrid 
intelligent systems with self-organization: Coordination, consistency, and dispute]. Moscow: IPI 
RAN. 189~p.
  \bibitem{3-l-1}
   \Aue{Tarasov, V.\,B.} 2002. \textit{Ot mnogoagentnykh sistem k~intellektual'nym 
organizatsiyam: filosofiya, psikhologiya, informatika} [From multiagent systems to intelligent 
organizations: Philosophy, psychology, and informatics]. Moscow: Editorial URSS. 352~p.
  \bibitem{4-l-1}
   \Aue{Listopad, S.\,V., and I.\,A.~Kirikov.} 2020. Metod identifikatsii konfliktov agentov 
v~gibridnykh intellektual'nykh\linebreak\vspace*{-12pt}

\columnbreak

\noindent
 mnogoagentnykh sistemakh [Agent conflict identification method in 
hybrid intelligent multiagent systems]. \textit{Sistemy i~Sredstva Informatiki~--- Systems and 
Means of Informatics} 30(1):56--65. doi: 10.14357/08696527200105.

%\vspace*{-2pt}

  \bibitem{5-l-1}
   \Aue{Baarslag, T., M.~Kaisers, E.\,H.~Gerding, C.\,M.~Jonker, and J.~Gratch.} 2017. 
Computers that negotiate on our behalf: Major challenges for self-sufficient, self-directed, and 
interdependent negotiating agents. \textit{Autonomous agents and multiagent systems}. Eds. 
G.~{Sukthan\-kar} and\linebreak  J.\,A.~Rodriguez-Aguilar. Lecture notes in computer science ser.:
Lecture notes in artificial 
intelligence subser. Cham, Switzerland: Springer. 
10643:143--163. 

%\vspace*{-2pt}

  \bibitem{6-l-1}
   \Aue{\mbox{Aydo{\!\ptb{\!\v{g}}}an}, R., T.~Baarslag, and E.~Gerding.} 2021. Artificial 
intelligence techniques for conflict resolution. \textit{Group Decis. Negot.}  
30:879--883.

%\pagebreak

  \bibitem{7-l-1}
   \Aue{Liu, T.\,H., A.~Goel, C.\,E.~Martin, and K.\,S.~Barber.} 1998. Classification and 
representation of conflict in multi-agent systems.
Austin, TX, USA: The University of Texas at Austin, 
1998. Technical Report of the Laboratory for Intelligent Processes and 
Systems TR98-UT-LIPS-AGENTS-01. 13~p.  Available at: {\sf 
https://citeseerx.\linebreak  ist.psu.edu/viewdoc/download?doi=10.1.1.35.3694\&\linebreak rep=rep1\&type=pdf} 
(accessed January~17, 2022).
  \bibitem{8-l-1}
   \Aue{Basheer, G.\,S., M.\,S.~Ahmad, and A.\,Y.\,C.~Tang.} 2013. A~framework for conflict 
resolution in multi-agent systems. \textit{Computational Collective Intelligence. Technologies and 
applications}. Eds. C.~B$\grave{\mbox{a}}$dic$\grave{\mbox{a}}$, N.\,T.~Nguyen, and 
M.~Brezovan. Lecture notes in computer science ser..: Lecture notes in artificial intelligence 
subser. Berlin, Heidelberg: Springer. 8083:195--204.
   \bibitem{9-l-1}
   \Aue{Adler, M.\,R., A.\,B.~Davis, R.~Weihmayer, and R.\,W.~Worrest.} 1989.  
Conflict-resolution strategies for nonhierarchical distributed agents. \textit{Distributed artificial 
intelligence~II.} London: Pitman Publishing. 139--161. 
   \bibitem{10-l-1}
   \Aue{Barber, K.\,S., T.\,H.~Liu, and D.\,C.~Han.} 2000. Strategic decision-making for conflict 
resolution in dynamic organized multi-agent systems. \textit{CERA~J.} Special Issue. 18~p.
Available at: {\sf 
https://citeseerx.ist.psu.edu/\linebreak viewdoc/download?doi=10.1.1.144.6881\&rep=rep1\&\linebreak type=pdf} 
(accessed January~17, 2022).
   \bibitem{11-l-1}
   \Aue{Tang, A., and G.~Basheer.} 2017. A~Conflict Resolution Strategy Selection Method 
(ConfRSSM) in multi-agent systems. \textit{Int. J.~Advanced Computer Science Applications} 
8:398--404.
   \bibitem{12-l-1}
   \Aue{Ioannidis, Y.\,E., and T.\,K.~Sellis.} 1989. Conflict resolution of rules assigning values to 
virtual attributes. \textit{Conference (International) on the Management of Data Proceedings}. 
New York, NY: ACM. 205--214.
   \bibitem{13-l-1}
   \Aue{Ephrati, E., and J.\,S.~Rosenschein.} 1991. The Clarke tax as a consensus mechanism among 
automated agents. \textit{AAAI Proceedings} 91:173--178.
   \bibitem{14-l-1}
   \Aue{Helge, G.} 2011. Decision-making strategies and self-regulated learning: Fostering 
decision-making competence in education for sustainable development. 
G$\ddot{\mbox{o}}$ttingen G$\ddot{\mbox{o}}$ttingen: der 
Georg-August-Universit$\ddot{\mbox{a}}$t G$\ddot{\mbox{o}}$ttingen. PhD Thesis. 206~p.
   \bibitem{15-l-1}
   \Aue{Listopad, S.\,V., and I.\,A.~Kirikov.} 2021. Stimulyatsiya konfliktov agentov 
v~gibridnykh intellektual'nykh mnogoagentnykh sistemakh [Stimulation of agent conflicts in hybrid 
intelligent multiagent systems]. \textit{Sistemy i~Sredstva Informatiki~--- Systems and Means of 
Informatics} 31(2):47--58. doi: 10.14357/08696527210205.
   \end{thebibliography}

 }
 }

\end{multicols}

\vspace*{-6pt}

\hfill{\small\textit{Received January 10, 2022}}

%\pagebreak

%\vspace*{-18pt}
   
   \Contr
   
   \noindent
   \textbf{Listopad Sergey V.} (b.\ 1984)~--- Candidate of Science (PhD) in technology, senior 
scientist, Kaliningrad Branch of the Federal Research Center ``Computer Science and Control'' of 
the Russian Academy of Sciences, 5~Gostinaya Str., Kaliningrad 236000, Russian Federation;  
\mbox{ser-list-post@yandex.ru}
   
   \vspace*{3pt}
   
   \noindent
   \textbf{Kirikov Igor A.} (b.\ 1955)~--- Candidate of  Sciences (PhD) in technology, director, 
Kaliningrad Branch of the Federal Research Center ``Computer Science and Control'' of the Russian 
Academy of Sciences, 5~Gostinaya Str., Kaliningrad 236000, Russian Federation; 
\mbox{baltbipiran@mail.ru}

\label{end\stat}

\renewcommand{\bibname}{\protect\rm Литература} 
    