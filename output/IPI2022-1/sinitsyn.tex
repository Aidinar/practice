\def\stat{sinitsyn}

\def\tit{НОРМАЛИЗАЦИЯ СИСТЕМ, СТОХАСТИЧЕСКИ НЕ~РАЗРЕШЕННЫХ ОТНОСИТЕЛЬНО ПРОИЗВОДНЫХ}

\def\titkol{Нормализация систем, стохастически не~разрешенных относительно производных}

\def\aut{И.\,Н.~Синицын$^1$}

\def\autkol{И.\,Н.~Синицын}

\titel{\tit}{\aut}{\autkol}{\titkol}

\index{Синицын И.\,Н.}
\index{Sinitsyn I.\,N.}


%{\renewcommand{\thefootnote}{\fnsymbol{footnote}} \footnotetext[1]
%{Работа выполнена при финансовой поддержке РАН (проект АААА-А19-119091990037-5)
% с~использованием инфраструктуры Центра коллективного 
%пользования <<Высокопроизводительные вычисления и~большие данные>> (ЦКП <<Информатика>>) 
%ФИЦ ИУ РАН (г.~Моск\-ва).}}


\renewcommand{\thefootnote}{\arabic{footnote}}
\footnotetext[1]{Федеральный исследовательский центр <<Информатика 
и~управление>> Российской академии наук, \mbox{sinitsin@dol.ru}}

\vspace*{6pt}



\noindent
\Abst{Для сис\-тем, стохастически не разрешенных относительно производных (СтНРОП), 
разработаны два подхода к~сведению таких сис\-тем к~детерминированным уравнениям, 
не разрешенным относительно математических ожиданий и~ковариационных характеристик (тео\-ре\-ма~1), 
а~так\-же математических ожиданий и~координатных функций канонических разложений (КР) (тео\-ре\-ма~2). 
После сведения таких сис\-тем к~детерминированным используются известные 
результаты автора. %~\cite{1-sin, 2-sin, 5-sin}. 
Приведен пример.
Рас\-смот\-ре\-ны вопросы оценивания (фильтрации, экстраполяции и~др.), 
идентификации и~ка\-либ\-ров\-ки для приведенных моделей.
Тео\-ре\-мы~1 и~2 допускают обобщения на эредитарные стохастические системы (СтС), опи\-сы\-ва\-емые 
ин\-тег\-ро\-диф\-фе\-рен\-ци\-аль\-ны\-ми уравнениями.
Особого внимания заслуживает развитие прямых чис\-лен\-ных методов анализа, моделирования, 
оценивания и~ка\-либ\-ров\-ки как для широкополосных, так и~для узкополосных возмущений.}


\KW{каноническое разложение (КР);
метод аналитического моделирования (МАМ);
нормализация по Пугачеву; сис\-те\-мы, стохастически не разрешенные относительно производных (СтНРОП);
стохастический процесс (СтП); стохастическая сис\-те\-ма (СтС); стохастическая функция}

\DOI{10.14357/19922264220105}
  
\vspace*{6pt}


\vskip 10pt plus 9pt minus 6pt

\thispagestyle{headings}

\begin{multicols}{2}

\label{st\stat}

\section{Введение}


В~\cite{1-sin, 2-sin} развиты приближенные методы аналитического моделирования (МАМ) 
широкополосных стохастических процессов (СтП) в~СтНРОП, основанные на методах 
нормальной аппроксимации (МНА) и~параметризации распределений. 
Для эредитарных ин\-тег\-ро\-диф\-фе\-рен\-ци\-аль\-ных с~гладкими нелинейностями 
\mbox{СтНРОП}, приведенных к~дифференциальным СтС, 
со\-от\-вет\-ст\-ву\-ющие методы даны в~\cite{3-sin, 4-sin}. В~\cite{5-sin} 
пред\-став\-ле\-ны МАМ и~оценивания нормальных СтП в~СтНРОП с~негладкими нелинейностями.

Основываясь на методе нормализации по Пугачеву слож\-ных СтС~\cite{6-sin, 7-sin, 8-sin, 9-sin}, 
рассмотрим вопросы приведения СтНРОП 
к~детерминированным сис\-те\-мам, не разрешенных относительно производных.
В~разд.~2 приведены краткие сведения по нормальным СтС (НСтС). 
Сис\-те\-мы \mbox{СтНРОП} описаны в~разд.~3. В~разд.~4 даются основные результаты\linebreak (теоремы~1 и~2) 
по приведению сис\-тем, стохасти\-чески не разрешенных относительно производных, 
к~со\-от\-вет\-ст\-ву\-ющим детерминированным сис\-те\-мам. В~разд.~5 
приведен пример, ил\-люст\-ри\-ру\-ющий\linebreak основные результаты. Раздел~6 
посвящен вопросам применения к~задачам оценивания, идентификации и~калибровки. 
Заключение содержит выводы и~возможные обобщения.



\section{Нормальные стохастические системы Пугачева}

\vspace*{-6pt}


\subsection{Определения} 

\vspace*{-6pt}


Следуя В.\,С.~Пугачеву~\cite{6-sin, 7-sin, 8-sin, 9-sin}, 
назовем СтС\linebreak в~действительном линейном функциональном пространстве нормальной, 
если при нормальном распределении ее входного {сигнала} в~пространстве~$X$ 
совместное распределение выходного \mbox{сигнала} в~пространстве~$Y$ и~входного сигнала  в~пространстве~$X$ 
нормально. Определим в~пространствах~$X$ и~$Y$ слабые топологии с~помощью множеств линейных
 функционалов~$L_x$ и~$L_y$ соответственно. В~пространствах~$L_x$ и~$L_y$ 
 определим слабые топологии с~по\-мощью множеств линейных функционалов вида~$\la x$ и~$\mu y$, 
 со\-от\-вет\-ст\-ву\-ющих всем возможным  $x\hm\in X$ и~$y \hm\in Y$. Кроме того, 
 будем предполагать, что пространства~$X$ и~$Y$ рефлексивны. Если теперь определить на прямом 
 произведении пространств  $Z\hm=X Y$ множество~$L_z$ линейных функционалов  $\nu z 
 \hm= \la x +\mu y$ для  $\la \hm\in L_x$ и~$ \mu \hm\in L_y$ и~совершенно таким же 
 образом ввести слабые топологии в~$Z$ и~$L_z$, то пространство~$Z$ тоже будет рефлексивным. 
 При этих условиях любой положительно определенный функционал  $g(\la)$ на~$L_x$ (или $L_y$, или~$L_z$), 
 непрерывный во введенной топологии и~удовле\-тво\-ря\-ющий условию  $g(0)\hm =1$, 
 однозначно определяет вероятностную меру на $\sigma$-ал\-геб\-ре, по\-рож\-да\-емой базой в~топологии в~$X$ 
 (соответственно~$Y$, $Z$), и~является характеристическим функционалом, соответствующим 
 этой вероятностной мере.
Поэтому принятые допущения дают воз\-мож\-ность задавать распределения вероятностей 
в~пространствах~$X$, $Y$, $Z$ с~по\-мощью характеристических функционалов (х.ф.).

Введем понятия математического ожидания, ковариационного оператора и~взаимного 
ковариационного оператора посредством сле\-ду\-ющих формул:
\begin{equation*}
    m_x = \mathrm{M} X = \int x \mu_x (dx),
    %\label{e2.1-sin}
\end{equation*}
где $\mu_x$~--- вероятностная мера~$X$;
\begin{equation*}
K_x \la = \mathrm{M} \left(X- m_x\right )\left(\overline{\la X} - \overline{\la m_x}\right),
%    \label{e2.2-sin}
\end{equation*}
где $\la\in L_x$~--- отображение~$L_x$ в~$X$;
\begin{multline*}
    K_{yx} \la =\mathrm{M} (Y-m_y)(\overline{\la X} -\overline{\la m_x}) = {}\\
    {}=\int (y-m_y) 
    (\overline{\la x} - \overline{\la m_x}) \mu_z (dx dy),
    %\label{e2.3-sin}
\end{multline*}
где $\la \hm\in L_x$ (отображение~$L_x$ в~$Y$).
Здесь $\mu_z$~--- вероятностная мера со\-став\-но\-го стохастического элемента (СтЭ) 
$Z\hm= X Y$ с~реализациями в~прямом произведении пространств $Z\hm= X Y$. 
При этом СтЭ~$X$ и~$Y$  коррелированы, если  $K_{yx} \la\hm\ne 0$ хотя бы для одного  $\la\hm\in L_x$, 
и~некоррелированные, если $K_{yx}\la\hm=0$ для всех  $\la\hm\in L_x$. Оператор~$K_x$ 
положителен, если $\la K_x \la \hm\ge 0$ при всех  $\la\hm\in L_x$, и~симметричен, когда $\la' K_x \la
\hm = \la K_x \la'$.

Распределение СтЭ~$X$ в~действительном линейном пространстве называется нормальным, если его х.ф.\ 
определяется формулой:
\begin{equation*}
    g_x (\la) =\mathrm{M}_{\cal N} e^{i\la X} = \exp \!\left[ i\la m_x - \fr{1}{2}\,\la K_x \la\right],\enskip 
    \la\in L_x.
   % \label{e2.4-sin}
\end{equation*}

 Если СтЭ $Z\hm=\lf X,Y\rf$ распределен нормально, то условное распределение СтЭ~$Y$ 
 (или СтЭ~$X$) при любом данном значении~$x$ (соответственного~$y$) СтЭ~$X$ (соответственно~$Y$) 
 регулярно и~нормально, причем условное математическое ожидание СтЭ~$Y$ (соответственно~$X$) 
 зависит линейно от~$x$ (соответственно~$y$), а его услов\-ный ковариационный оператор не зависит от~$x$ 
 (соответственно~$y$).

\subsection{Общий алгоритм нормализации} 

Для получения уравнений НСтС достаточно аппроксимировать:
\begin{enumerate}[(1)]
\item математическое ожидание  $m_{y|x}$ ее выходного сигнала 
от входного сигнала~$x$ приближенной линейной зависимостью $a\hm+Lx$;

\item определить оператор $L$ из уравнения
\begin{equation*}
LK_x = K_{yx};
  %  \label{e2.5-sin}
    \end{equation*}

\item усреднить по возможным реализациям входного сигнала~$x$ ковариационный оператор~$K_{y|x}$ 
выходного сигнала и~заменить переходную плот\-ность вероятности нормальной с~па\-ра\-мет\-рами
\begin{equation*}
    \hspace*{-6mm}a=m_y - Lm_x; \enskip K_{y'} =\mathrm{M}_{\cal N} K_{yx} = \int \! K_{y|x} \mu_x (dx).
  %  \label{e2.6-sin}
\end{equation*}
    \end{enumerate}

В~\cite{6-sin, 7-sin, 8-sin, 9-sin} показано, что любую НСтС можно рассматривать как 
последовательное соединение\linebreak цепочки, со\-сто\-ящей из параллельно соединенных детерминированной 
линейной сис\-те\-мы с~оператором~$L$ и~генератора шума, вырабатывающего нормально распределенный шум~$Y'$, 
независимый от входного сигнала, и~идеального детерминированного сумматора.

Любые соединения НСтС представляют собой НСтС.

Теория НСтС дает возможность применять для стохастического анализа и~моделирования 
НСтС хорошо разработанный аппарат теории линейных\linebreak СтС. 
При этом необходимо только учитывать наряду с~обычными источниками 
помех и~шумов в~детерминированных сис\-те\-мах, внут\-рен\-ние шумы, 
создаваемые генераторами шумов, соединенными с~\mbox{детерминированными} 
линейными системами в~элементах этой сис\-темы.

\subsection{Алгоритм нормализации посредством канонических разложений} 

При использовании КР случайных функций (СФ)
 и~СтП в~основе лежат сле\-ду\-ющие соотношения:
    \begin{multline*}
    X=m_x + \sss_{i=1}^\infty V_i x_i, \enskip K_x =\sss_{i=1}^\infty D_i^V x_i x_i^{\mathrm{T}};
    \\
K_y =\sss_{i=1}^\infty D_i^V y_iy_i^{\mathrm{T}} +\sss_{i=1}^\infty D_j^W z_j z_j^{\mathrm{T}};
%\label{e2.7-sin}
\end{multline*}

\noindent
\begin{equation*}
Y=m_y +\sss_{i=1}^\infty V_i y_i +\sss_{i=1}^\infty W_j z_j;
%\label{e2.8-sin}
\end{equation*}
\begin{equation*}
\sss_{i=1}^\infty V_iy_i = LX^0,\enskip L= \sss_{i=1}^\infty y_i;
%\label{e2.9-sin}
\end{equation*}
\begin{equation*}
\sss_{i=1}^\infty W_j z_j = Y';\enskip K_{y'} =\sss_{i=1}^\infty D_j z_j^W z_J^{\mathrm{T}}.
%\label{e2.10-sin}
\end{equation*}
Здесь $V_i = f_iX^0$ и~$W_j=f_j'Y^0$~--- некоррелированные случайные величины (СВ) с~дисперсиями~$D_i^V$\linebreak\vspace*{-12pt}

\pagebreak

\noindent
и~$D_j^W$; 
$x_i$, $y_i$ и~$z_j$~--- детерминированные координатные функции; $f_i$ и~$f_j'$~--- 
линейные функционалы, определяющие базисы в~соответствующих гильбертовых пространствах,
 порожденных СВ $V_i$ и~$W_j$.

\section{Системы, стохастически не~разрешенные относительно производных}

В задачах статистической идентификации и~калибровки сложных 
СтС встречаются элементы, описываемые СФ вектора 
состояния и~его производных по времени. Примерами могут служить 
элементы с~сухим трением, гистероны, радиолокационные, оптические и~акустические 
устройства со стохастическими пеленгационными характеристиками, а~также
 стохастические ор\-га\-ни\-за\-ци\-он\-но-тех\-ни\-ко-эко\-но\-ми\-че\-ские сис\-те\-мы~\cite{10-sin}. 
 На основе экспериментальных данных для таких сис\-тем труд\-но провести 
 па\-ра\-мет\-ри\-за\-цию соответствующих характеристик.

Типовой системой, стохастически не разрешенной относительно производных, может служить сле\-ду\-ющая:
 \begin{align}
    F\left(t, \dot X_t, X_t, U_t\right) &= \Phi\left(t, \dot X_t, X_t, U_t\right) +{}\notag\\
    &\hspace*{10mm}{}+     \varphi \left(t, X_t, U_t\right)=0\,;
    \label{e3.1-sin}
    \\
    \dot U_t &= a_{0t}^U+ a_t^UU_t+b_t^UV_t.
    \label{e3.2-sin}
\end{align}
Здесь  $X_t$ и~$\dot X_t$~--- $n^x$-мер\-ный вектор со\-сто\-яния\linebreak и~его производная 
(понимаемая в~сред\-не\-квад\-ра\-ти\-че\-ском смыс\-ле); $U_t$~--- $n^U$-мер\-ный СтП; 
$V_t$~---  $n^V$-мер\-ный векторный СтП типа гауссовского (\mbox{нормального}) белого шума 
с~мат\-ри\-цей \mbox{интенсивностей}  $\nu^V\hm=\nu^V(t)$; $\Phi \hm=\Phi(t, \dot X_t, X_t, U_t)$~--- 
векторная СФ отмеченных переменных раз\-мер\-ности~$n^\Phi$; $\varphi \hm=\varphi(t, X_t, U_t)$~--- 
известная векторная функция раз\-мер\-ности~$n^\varphi$; $a_{0t}^U$, $a_t^U$ и~$b_t^U$~--- 
известные функции раз\-мер\-ности~$n^U\times 1$, $n^U\times n^U$ и~$n^U \times n^V$.

При использовании КР для стохастической функции~$\Phi$ и~гауссовского белого шума~$V_t$ 
требуются или априорно заданные экспериментальные данные, или спект\-раль\-но-кор\-ре\-ля\-ци\-он\-ные 
вероятностные характеристики.

\smallskip

\noindent
\textbf{Замечание~1.}\ К~сис\-те\-ме~(\ref{e3.1-sin}) приводятся также уравнения вида:

\noindent
\begin{multline*}
    F\left(t, \Theta,\dot X_t, \dot X_t\tr X_t^{(k)}, U_t, \dot  U_t\tr  U_t^{(r)}\right) ={}\\
{}=\Phi\left(t, \Theta,\dot X_t, \dot X_t\tr X_t^{(k)}, U_t, \dot  U_t\tr  U_t^{(r)}\right) +{}\\
\!{}+\varphi \left(t, \Theta,\dot X_t, \dot X_t\tr X_t^{(k)}, U_t, \dot  U_t\tr  U_t^{(r)}\right)=0\,,
    % \label{e3.3-sin}
\end{multline*}
где
\begin{equation*}
    \dot U_t =a^U\left(t,\Theta, U_t\right)+b^U\left(t,\Theta, U_t\right)V\,, 
    %\label{e3.4-sin}
\end{equation*}
рассмотренные в~\cite{1-sin, 2-sin, 5-sin} соответственно для гладких и~разрывных функций.
Здесь $\Theta$~--- опре\-де\-ля\-ющие инструментальные па\-ра\-мет\-ры сис\-те\-мы;
 $X_t^{(k)}$ и~$U_t^{(r)}$~--- производные от соответствующих функций; $V$~--- 
 нормальный белый шум интенсивности~$\nu$; $a^U \hm=a^U(t,\Theta, U_t)$
 и~$b^U \hm=b^U(t,\Theta, U_t)$~--- известные функции отмеченных переменных.


\section{Основные результаты}

\subsection{Общий ковариационный подход} 

В~условиях существования решения, 
по\-ни\-ма\-емо\-го в~среднеквадратическом смысле, рассмотрим два подхода для приведения~(\ref{e3.1-sin}) 
к~двум сис\-те\-мам детерминированных уравнений для математических ожиданий и~ковариационных характеристик.

Сначала рассмотрим случай, когда уравнения~(\ref{e3.1-sin}) 
допускают нормализацию по Пугачеву относительно переменных $\dot X_t$ 
и~$Z_t \hm=[X_t^{\mathrm{T}} U_t^{\mathrm{T}}]^{\mathrm{T}}$:
\begin{multline}
    \!F\approx \Phi_0\left(t, m_t^{\dot X},m_t^Z\right) +\Delta\Phi_t^0 +k^\Phi\left(t, 
    m_t^{\dot X},m_t^Z\right) \dot X_t^0+{}\\
    {}+k^\Phi Z_t^0+\varphi_0 \left(t, m_t^Z\right) + k^\varphi \left(t, m_t^Z\right) Z_t^0 =0\,.
    \label{e4.1-sin} 
  \end{multline}
Из~(\ref{e4.1-sin}), опуская аргументы, получим сле\-ду\-ющее уравнение:
  \begin{equation}
    \Phi_0+\varphi_0=0.
    \label{e4.2-sin}
\end{equation}
При этом будем считать, что уравнение~(\ref{e4.2-sin}) 
допускает нормализацию по математическому ожиданию~$m_t^{\dot X}$.

Теперь рассмотрим уравнения для центрированных со\-став\-ля\-ющих. Уравнение для~$Z_t^0$ имеет вид:
  \begin{equation}
    k^\Phi \dot X_t^0 +(k^\Phi + k^\varphi) Z_t^0=0.
    \label{e4.3-sin}
\end{equation}
При условии $\mathrm{det}\,k^\Phi  \hm\ne 0$, разрешая~(\ref{e4.3-sin}) относительно~$\dot X_t^0$, придем 
к~сле\-ду\-юще\-му мат\-рич\-но\-му выражению для мат\-ри\-цы~$K_t^{\dot X}$:
  \begin{equation}
    K_t^{\dot X}=-\left(k^\Phi\right)^{-1}\left(k^\Phi+k^\varphi\right) K_t^Z.
    \label{e4.4-sin}
    \end{equation}

Для вычисления $K_t^Z$, $K_t^X$, $K_t^{\Delta\Phi}$ и~$K_t^{X\Delta\Phi}$ с~учетом~(\ref{e3.2-sin}) 
имеем соотношения:
\begin{align}
    \dot Z_t^0 &=AZ_t +BV;
\notag\\ %    \label{e4.5-sin}\\
    \dot K_t^Z&=AK_t^Z+K_t^ZA^{\mathrm{T}}+B\nu^VB^{\mathrm{T}}.
    \label{e4.6-sin}
\end{align}
Здесь введены сле\-ду\-ющие обозначения:
\begin{multline}
    A=\begin{bmatrix}
    \left[ -(k^\Phi)^{-1} \right]& \left[ 0\right];\\[3pt]
    \left[ 0\right] & \left[ a_t^U\right]
    \end{bmatrix}; \enskip
    B=\begin{bmatrix} 
    \left[ 0\right]\\[3pt] 
    \left[ b_t^U\right]
    \end{bmatrix};\\ 
    \nu^V =\left[ \nu_{ij}\right];
    \label{e4.7-sin}
\end{multline}

\noindent
\begin{equation}
K_t^Z = \lk\begin{array}{cc}
    \lk K_t^X\rk& \lk K_t^{X\Delta \Phi}\rk\\[3pt]
    \lk K_t^{X\Delta \Phi}\rk& \lk K_t^{\Delta \Phi}\rk\\\end{array}\rk,
    \label{e4.8-sin}
\end{equation}
где квадратными скобками отмечены блоч\-ные квад\-рат\-ные мат\-ри\-цы со\-от\-вет\-ст\-ву\-ющих 
размеров~$n^X$, $n^\Phi$, $n^{\Delta \Phi}$ и~$n^\varphi$.

\smallskip

\noindent
\textbf{Теорема~1.}\ \textit{Пусть сис\-те\-ма}~(\ref{e3.1-sin}), 
\textit{стохастически не разрешенная относительно производных,
 допускает нормализацию по Пугачеву. Тогда в~основе ковариационного алгоритма лежат 
 уравнения}~(\ref{e4.2-sin}), (\ref{e4.4-sin}) \textit{и}~(\ref{e4.6-sin}) \textit{при 
 условиях}~(\ref{e4.7-sin}) \textit{и}~(\ref{e4.8-sin}) \textit{и}~$\det k^\Phi\hm\ne 0$.
 
 \smallskip


\subsection{Алгоритмы, основанные на~канонических разложениях} 

При использовании КР со СВ~$W_h$ и~$V_j$ в~основе соответствующего алгоритма 
приведения лежат сле\-ду\-ющие соотношения:
\begin{equation}
X_t = m_t^X +\sss_{h=1}^{H_W} W_h x_{ht}^W +\sss_{j=1}^{H_V} V_j x_{jt}^V;
    \label{e4.9-sin}
    \end{equation}
\begin{equation}
\dot X_t =\dot m_t^X + \sss_{h=1}^{H_W} W_h \dot x_{ht}^W + 
    \sss_{j=1}^{H_V} V_j \dot x_{jh}^V;
    \label{e4.10-sin}
    \end{equation}
    
    \vspace*{-12pt}
    
    \noindent
\begin{multline}
Z_t = m_t^Z + \sss_{h=1}^{H_W} W_h z_{ht}^W+ \sss_{j=1}^{H_V} V_jz_{it}^V,\\ 
    V_t=\sss_{j=1}^{H_V} V_j \upsilon_{jt}^V,
    \label{e4.11-sin}
    \end{multline}
    
    \noindent
\begin{equation}
\Phi \approx \Phi_0+\Delta\Phi_t^0;
    \label{e4.12-sin}
    \end{equation}
\begin{equation}
\Phi_0 (t, m_t^{\dot X}, m_t^Z) = \mathrm{M}_{\cal N} \Phi \left(t, \dot X_t, Z_t\right);
    \label{e4.13-sin}
    \end{equation}
    
    \vspace*{-12pt}
    
    \noindent
\begin{multline}
\Delta \Phi_t^0 =\Delta\Phi_t^0 \left(t, m_t^{\dot X}, m_t^Z\right) ={}\\
{}= \sss_{h+1}^{H_\Phi} 
    W_h w_h\left(t, m_t^{\dot X}, m_t^Z\right);
    \label{e4.14-sin}
    \end{multline}
    
    \noindent
\begin{equation}
K_t^Z =\sss_{h=1}^{H_W} D_{jt}^W  x_{ht}^W\overline{x_{ht}^W} + 
    \sss_{j=1}^{H_V} D_j^V x_{jt}^V \overline{x_{jt}^V};
    \label{e4.15-sin}
    \end{equation}
\begin{equation}
\dot z_{ht}^W =Az_{ht}^W,\enskip \dot z_{jt}^V = Az_{jt}^V +Bv_{jt}.
    \label{e4.16-sin}
\end{equation}
Здесь СВ  $W_h$ и~$V_j$ и~соответствующие им координатные функции $x_{t}^V$, $x_{ht}^w$ и~$\upsilon_{jt}^V$ 
находятся экспериментально на основе ковариационных функций~$\Phi$ и~$V$ по известным формулам~\cite{7-sin}.

\smallskip

\noindent
\textbf{Теорема~2.}\ \textit{Пусть сис\-те\-ма}~(\ref{e3.1-sin}), 
\textit{стохастически не разрешенная относительно производных, 
допускает нормализацию посредством КР по Пугачеву. Тогда 
в~основе ковариационного алгоритма лежат уравнения}~(\ref{e4.9-sin})--(\ref{e4.16-sin}) 
\textit{для математических ожиданий и~координатных функций при условиях  $\det k^\Phi \hm\ne 0$
и~$\det k^\varphi \hm\ne 0$}.
 
 \smallskip

Для гладких и~разрывных нелинейностей предложены алгоритмы приведения 
к~стохастическим дифференциальным уравнениям, разрешенным относительно производных, 
основанные на построении стохастических регрессионных моделей согласно~\cite{5-sin}.

\smallskip

\noindent
\textbf{Замечание~2.}\ В~условиях замечания~1 формулируются утверж\-де\-ния, аналогичные тео\-ре\-мам~1 и~2, 
но только для расширенного вектора состояния $\bar X \hm= \lk X^{\mathrm{T}}U^{\mathrm{T}}\rk^{\mathrm{T}}$.

\smallskip

\noindent
\textbf{Замечание~3.}\ Для гладких и~разрывных нелинейностей предложены алгоритмы приведения 
к~стохастическим дифференциальным уравнениям, разрешенным относительно производных, 
основанные на построении стохастических регрессионных моделей согласно~\cite{5-sin}. 
В~этих моделях имеет место зависимость коэффициентов нормализации не только от математических ожиданий, 
но и~от вероятностных моментов $\Phi_0$, $\varphi_0$, $k^\Phi$ и~$k^\varphi$ второго порядка~$K_t^Z$.

\section{Пример}

Сначала рассмотрим скалярную гауссовскую сис\-те\-му вида~(\ref{e3.1-sin})
\begin{equation}
    F\equiv\Phi (\dot X_t) + a_t X_t + u_t =0.
    \label{e5.1-sin}
    \end{equation}
Здесь $X_t=m_t^X+X_t^0$ и~$\dot X_t = m_t^{\dot X} + \dot X_t^0$~--- 
переменная состояния и~ее производная по времени~$t$; $a_t$~--- 
известная скалярная функция времени; $\Phi(\dot X_t)$~--- 
СФ, допускающая нормализацию по Пугачеву, \mbox{сле\-ду\-юще\-го} вида:
\begin{equation}
    \Phi (\dot X_t) \approx \Phi_0 + k^\Phi \dot X_t^0 + \Delta \Phi_t^0.
    \label{e5.2-sin}
        \end{equation}
Здесь $\Delta \Phi_t^0$~--- нормальный СтП, удовлетворяющий линейному уравнению формирующего фильт\-ра
\begin{equation}
    \dot\Delta \Phi_t^0 = a_t^{\Delta \Phi} \Delta\Phi_t^0 + b_t^{\Delta\Phi} V_t,
    \label{e5.3-sin}
        \end{equation}
где $a_t^{\Delta\Phi}$ и~$b_t^{\Delta\Phi}$~--- скалярные известные функции; 
$V_t$~--- нормальный белый шум интенсивности  $\nu^V\hm=\nu^V(t)$. 
Входящие в~(\ref{e5.2-sin}) функции~$\Phi_0$ и~$k^\Phi$ зависят от~$m_t^{\dot X}$ и~$D_t^{\dot X}$.

Уравнение~(\ref{e5.1-sin}) с~учетом~(\ref{e5.2-sin}) 
распадается на два скалярных уравнения: одно для математического ожидания при условии  $k^{\Phi_0}\hm\ne 0$ 
имеет вид:
\begin{equation}
\Phi_0 + a_tm_t^X +u_t =0,\enskip \Phi_0 = k^{\Phi_0} m_t^{\dot X},
    \label{e5.4-sin}
        \end{equation}
а для  $X_t^0$ при условии  $\det k^\Phi\hm\ne 0$ имеем
    $$
    k^\Phi \dot X_t^0 +\Delta\Phi_t^0 + a_t X_t^0 =0
    $$
или

\vspace*{-2pt}

\noindent
\begin{equation}
\dot X_t^0 = a_t (k^\Phi)^{-1} X_t^0 -\left(k^\Phi\right)^{-1} \Delta \Phi_t^0.
\label{e5.5-sin}
\end{equation}
Уравнения~(\ref{e5.5-sin}) и~(\ref{e5.3-sin}) для $Z_t^0\hm= \lk X_t^0\ \Delta\Phi_t^0\rk^{\mathrm{T}}$ 
приводят к~сле\-ду\-юще\-му векторному уравнению для ковариационной мат\-рицы:
\begin{equation}
    \dot K_t^Z = AK_t^Z +K_t^Z A^{\mathrm{T}} +B\nu^V B^{\mathrm{T}},\label{e5.6-sin}
        \end{equation}
        
        \vspace*{-2pt}
        
        \noindent
где

\vspace*{-2pt}

\noindent
  \begin{equation*}
    A=\lk \begin{array}{cc}
    -a_t(k^\Phi)^{-1}&-(k^\Phi)^{-1}\\ 0& a^{\Delta \Phi}\\\end{array}\rk,
    \enskip
  B=\lk \begin{array}{c} 0\\ b^{\Delta\Phi}\\ \end{array}\rk.
 % \label{e5.7-sin}
  \end{equation*}
Уравнения~(\ref{e5.4-sin}) и~(\ref{e5.6-sin}) в~развернутом виде приводят к~сле\-ду\-ющим соотношениям:
\begin{equation}
    m_t^{\dot X} = (a_t m_t^X +u_t) (k^{\Phi_0})^{-1};
    \label{e5.8-sin}
    \end{equation}
    
    \vspace*{-13pt}
    
    \noindent
\begin{multline}
D_t^{\dot X} = a_t^2\left(k^{\Phi}\right)^{-2} D_t^X +\left(k^{\Phi}\right)^{-2}D_t^{\Delta \Phi} -{}\\
{}-
    2a_t \left(k^{\Phi}\right)^{-2} K_t^{X\Delta\Phi};
    \label{e5.9-sin}
    \end{multline}
    
    \vspace*{-13pt}
    
    \noindent
    \begin{gather}
\dot D_t^{X} =-2\left( a_t D_t^X +K_t^{X\Delta \Phi}\right);
\label{e5.10-sin}
\\
\dot D_t^{\Delta\Phi} = 2a_t^{\Delta\Phi} D_t^{\Delta\Phi}+\left(b_t^{\Delta\Phi}\right)^2 \nu^V;
    \label{e5.11-sin}
   \\
\!\!\!\dot K_t^{X\Delta\Phi}\!=\!a_t^{\Delta\Phi}K_t^{\Delta\Phi}-\left(a_tK_t^{X\Delta\Phi}+
    D_t^{\Delta\Phi}\right)\left(k^\Phi\right)^{-1}\!\!.\!
    \label{e5.12-sin}
    \end{gather}

Таким образом, гауссовская стохастическая сис\-те\-ма~(\ref{e5.1-sin})--(\ref{e5.3-sin}) 
сведена к~сис\-те\-ме взаимосвязанных детерминированных конечных уравнений~(\ref{e5.8-sin}) и~(\ref{e5.9-sin}) 
и~дифференциальных уравнений~(\ref{e5.10-sin})--(\ref{e5.12-sin}).

При использовании КР в~случае, когда
 \begin{align*}
    \Phi(\dot X_t)& = \displaystyle \sss_{h=-\infty}^{h=\infty} W_h 
    \exp \left(i\la_h \dot X_t\right);\\[6pt]
     D_h^W &=\displaystyle \int\limits_{\la_h-\Delta}^{\la_h+\Delta} 
    s^\Phi (\la)\, d\la,
    %\label{e5.13-sin}
        \end{align*}
сводится к~совместной детерминированной сис\-те\-ме, не разрешенной относительно 
производных, для математических ожиданий и~координатных функций:

\noindent
\begin{gather*}
F_0 \left(t, m_t^X,\dot m_t^X\right) =\Phi_0 \left(t, \dot m_t^X\right) + u_t +a_tm_t^X=0;
 %   \label{e5.14-sin}
    \\  
\exp\left(i\la_h \dot m_t^X\right) + a_t x_{ht} =0 \enskip (h=\overline{1, H_W});
  %  \label{e5.15-sin}
    \\
    D_t^X =\sss_{h=1}^{H_W} D_h^W x_{ht}\overline{x_{ht}}.
    %\label{e5.16-sin}
    \end{gather*}


\section{ Применение }


\subsection{Фильтрация}

Пусть исходная наблюдаемая СтНРОП 
со\-сто\-яния~$X_t$ и~наблюдения~$Y_t$ имеет вид:
\begin{equation}
F_1\!\left(t,\Theta, \dot X_t, X_t, U_t, Y_t\right)\! +\!\varphi_1 \left(t,\Theta, X_t, U_t, Y_t\right)\! =\!0.\!\!
\label{e6.1-sin}
    \end{equation}
Предположим, что~(\ref{e6.1-sin}) удовлетворяет условиям тео\-ре\-мы~1 для вектора переменных~$\bar X_t$ 
и~может быть приведена к~виду:
    \begin{equation}
    \dot{\bar X}_t =\left(\bar a Y_t +\bar a_1 \bar X_t +\bar a_0\right) +\bar \psi V.
    \label{e6.2-sin}
        \end{equation}
Векторное уравнение наблюдения примем сле\-ду\-ющим:
\begin{equation}
\dot Y_t = \left(bY_t+b_1\bar X_t + b_0\right) + \psi_1 V\,.
\label{e6.3-sin}
    \end{equation}
В~(\ref{e6.2-sin}) и~(\ref{e6.3-sin})~$V$~--- 
гауссовский белый шум интенсивности  $\nu^V$; $\bar\psi$ и~$\psi_1$~--- известные коэффициенты, 
зависящие от времени. В~таком случае, используя\linebreak известные уравнения среднеквадратичной 
оптимальной гауссовской линейной фильтрации~\cite{11-sin}, придем к~сле\-ду\-юще\-му утверж\-де\-нию.

\smallskip

\noindent
\textbf{Теорема~3.}\ \textit{Пусть выполнены условия}:
\begin{enumerate}[(1)]
\item \textit{стохастическая, не разрешенная относительно старшей производной, сис\-те\-ма}~(\ref{e6.1-sin}) 
\textit{может быть разрешена и~приведена к~виду}~(\ref{e6.2-sin}) 
\textit{для наблюдаемого расширенного вектора состояния};
\item 
\textit{наблюдения  $Y_t$ полны и~описываются линейным уравнением}~(\ref{e6.3-sin}), 
\textit{разрешенным относительно производной~$Y_t$}.
\end{enumerate}
\textit{Тогда при условии невырожденности диффузионной мат\-ри\-цы наблюдений
 $\det (\psi_1\nu\psi_1^{\mathrm{T}})\hm\ne 0$ 
уравнения среднеквадратичного несмещенного линейного фильт\-ра для оценки 
расширенного вектора~$\bar X_t$ имеют сле\-ду\-ющий вид}:

\vspace*{-6pt}

\noindent
\begin{multline}
\dot{\hat{\bar X}}_t = \left(a Y_t + a_1 \hat{\bar X}_t + a_0 \right) +{}\\
{}+ 
    \beta_t \lk \dot Y_t - \left(bY_t + b_1\bar X_t + b_0\right)\rk;
    \label{e6.4-sin}
    \end{multline}
    
    \vspace*{-6pt}
    
    \noindent
\begin{equation}
\beta_t = \left( R_t b_1^{\mathrm{T}} +\psi_1\nu\psi_1^{\mathrm{T}}\right) 
    \left(\psi_1\nu\psi_1^{\mathrm{T}}\right)^{-1};
    \label{e6.5-sin}
    \end{equation}
    
    \vspace*{-12pt}
    
    \noindent
\begin{multline}
\dot R_t = a_1  R_t + R_t a_1^{\mathrm{T}} + \psi\nu\psi^{\mathrm{T}}-{}\\
 \!\!\! \!{}-
  \left( R_1 b_1^{\mathrm{T}} +\psi_1\nu\psi_1^{\mathrm{T}}\right)\!
    \left(\psi_1\nu\psi_1^{\mathrm{T}}\right)^{-1}\! 
    \left(b_1 R_t +\psi_1\nu \psi_1^{\mathrm{T}}\right).\!\!\!
    \label{e6.6-sin}
    \end{multline}
\textit{Здесь $\hat{\bar X}_t $~--- среднеквадратичная оценка расширенного вектора состояния; $R_t$~--- 
ковариационная матрица среднеквадратичных ошибок фильт\-ра\-ции расширенного вектора~$\bar X_t$}.

Аналогично формулируется \textbf{теорема~4} для алгоритма, основанного на КР. 
Для этого используются результаты~\cite{6-sin, 7-sin}.

\smallskip

\noindent
\textbf{Замечание~4.}\ В~случае когда результаты наблюдения не влияют на объект наблюдения, полагая 
$a\hm=b\hm=0$, получаем известные уравнения фильтра Кал\-ма\-на--Бью\-си~\cite{11-sin}.

\smallskip

\noindent
\textbf{Замечание~5.}\ Для фильтрации в~условиях, когда помеха в~уравнениях наблюдения не является 
белым шумом, уравнение~(\ref{e6.3-sin}), в~зависимости от гладкости фильт\-ру\-емо\-го 
сигнала и~помехи следует использовать методы Гуль\-ко--Но\-во\-сель\-це\-вой или 
Брай\-со\-на--Ио\-хан\-се\-на~\cite{11-sin}.

\smallskip

\noindent
\textbf{Замечание~6.}\ Случай, когда уравнения~(\ref{e6.2-sin}) и~(\ref{e6.3-sin}) 
содержат параметрические шумы подробно изучен в~\cite{5-sin}.


\subsection{Идентификация и~калибровка}

Если в~со\-став расширенного вектора со\-сто\-яния входят инструментальные па\-ра\-мет\-ры~$\Theta$, 
пред\-став\-ля\-ющие собой СВ, т.\,е.\ когда $\dot \Theta \hm=0$, уравнения~(\ref{e6.4-sin})--(\ref{e6.6-sin}) 
будут пред\-став\-лять собой алгоритм совместной фильт\-ра\-ции и~идентификации для 
$x_t \hm=\lk \bar X_t^{\mathrm{T}}\ \Theta^{\mathrm{T}}\rk^{\mathrm{T}}$. Для решения метрологических задач 
в~технических приложениях широко используются комплексы операций и~алгоритмов для 
сис\-тем калибровки при специальных условиях с~\mbox{целью} под\-тверж\-де\-ния требований стандартов.


\section{ Заключение}


\noindent
\begin{enumerate}[1.]
\item  Для СтНРОП 
разработаны два подхода к~сведению таких сис\-тем к~детерминированным уравнениям, 
не разрешенным относительно математических ожиданий и~ковариационных\linebreak характеристик (тео\-ре\-ма~1), 
а~так\-же математических ожиданий и~координатных функций КР (тео\-ре\-ма~2). 
После сведения таких сис\-тем к~детерминированным используются известные результаты~\cite{1-sin, 2-sin, 5-sin}. 
Приведен пример.

\item  Рассмотрены вопросы фильтрации (теорема~3), идентификации и~калибровки для приведенных моделей.

\item  Теоремы 1 и~2 допускают обобщения на эредитарные СтС, описываемые 
ин\-тег\-ро\-диф\-фе\-рен\-ци\-аль\-ны\-ми 
уравнениями.

\item  Особого внимания заслуживает развитие прямых численных методов анализа, 
моделирования, оценивания и~калибровки как для широкополосных, так и~для узкополосных \mbox{возмущений}.
\end{enumerate}




{\small\frenchspacing
 {%\baselineskip=10.8pt
 %\addcontentsline{toc}{section}{References}
 \begin{thebibliography}{99}

%1
\bibitem{1-sin}
\Au{Синицын И.\,Н.}
Аналитическое моделирование широкополосных процессов в~стохастических сис\-те\-мах, 
не разрешенных относительно производных~// Информатика и~её применения, 2017. Т.~11. Вып.~1. С.~3--10.

%2
\bibitem{2-sin}
\Au{Синицын И.\,Н.}
Параметрическое аналитическое моделирование процессов в~стохастических  сис\-те\-мах,
 не разрешенных относительно производных~// Сис\-те\-мы и~средства информатики, 2017. Т.~27. №\,1. С.~20--45.

%3
\bibitem{3-sin}
\Au{Синицын И.\,Н.}
Нормальные субоптимальные фильт\-ры для дифференциальных стохастических сис\-тем, 
не разрешенных относительно производных~// Информатика и~её применения, 2021. Т.~15. Вып.~1. С.~3--10.

%4
\bibitem{4-sin}
\Au{Синицын И.\,Н.}
Аналитическое моделирование и~фильтрация нормальных процессов в~интегродифференциальных 
стохастических сис\-те\-мах, не разрешенных относительно производных~// Сис\-те\-мы и~средства информатики, 2021.
 Т.~31. №\,1. С.~37--56.

%5
\bibitem{5-sin}
\Au{Sinitsyn I.\,N.}
Analytical modeling and estimation of normal processes defined by stochastic differential 
equations with unsolved derivatives~// J.~Mathematics Statistics Research, 2021. Vol.~3. Iss.~1. Art.~139. 
7~p. doi: 10.36266/\linebreak JMSR/139.

%6
\bibitem{8-sin}
\Au{Пугачев В.\,С., Синицын~И.\,Н.}
Стохастические дифференциальные сис\-те\-мы. Анализ и~фильтрация.~--- М.:
Наука,  1990.  632~с. 

%7
\bibitem{9-sin}
\Au{Пугачев В.\,С., Синицын~И.\,Н.}
Теория стохастических сис\-тем.~--- М.: Логос, 2000; 2004. 1000~с.

%8
\bibitem{7-sin}
\Au{Синицын И.\,Н.}
Канонические представления случайных функций и~их применение в~задачах 
компьютерной поддержки научных исследований.~--- М.: ТОРУС
ПРЕСС, 2009. 768~с.

%9
\bibitem{6-sin} %9
Академик Пугачев Владимир Семенович: к~столетию со дня рождения~/ Под ред. И.\,Н.~Синицына.~--- 
М.: ТОРУС ПРЕСС, 2011. 376~с.






%10
\bibitem{10-sin}
\Au{Синицын И.\,Н., Шаламов~А.\,С.}
Лекции по теории сис\-тем интегрированной логистической поддержки.~--- 
2-е изд.~--- М.: ТОРУС ПРЕСС, 2019. 1072~с.


%11
\bibitem{11-sin}
\Au{Синицын И.\,Н.}
Фильтры Калмана и~Пугачева.~--- 2-е изд.~--- М.: Логос, 2007. 776~с.
\end{thebibliography}

 }
 }

\end{multicols}

\vspace*{-9pt}

\hfill{\small\textit{Поступила в~редакцию 22.03.21}}

%\vspace*{8pt}

%\pagebreak

\newpage

\vspace*{-28pt}

%\hrule

%\vspace*{2pt}

%\hrule

%\vspace*{-2pt}

\def\tit{NORMALIZATION OF SYSTEMS\\ WITH~STOCHASTICALLY UNSOLVED DERIVATIVES}


\def\titkol{Normalization of systems with~stochastically unsolved derivatives}


\def\aut{I.\,N.~Sinitsyn}

\def\autkol{I.\,N.~Sinitsyn}

\titel{\tit}{\aut}{\autkol}{\titkol}

\vspace*{-15pt}


\noindent
Federal Research Center ``Computer Science and Control'' of the 
Russian Academy of Sciences, 44-2~Vavilov Str., Moscow 119333, Russian Federation

\def\leftfootline{\small{\textbf{\thepage}
\hfill INFORMATIKA I EE PRIMENENIYA~--- INFORMATICS AND
APPLICATIONS\ \ \ 2022\ \ \ volume~16\ \ \ issue\ 1}
}%
 \def\rightfootline{\small{INFORMATIKA I EE PRIMENENIYA~---
INFORMATICS AND APPLICATIONS\ \ \ 2022\ \ \ volume~16\ \ \ issue\ 1
\hfill \textbf{\thepage}}}

\vspace*{3pt} 



\Abste{For a system with stochastically unsolved derivatives, two approaches for reduction of 
such systems to deterministic systems are developed. The first approach is based on equations 
for mathematical expectations and covariance characteristics. 
The second approach considers equations for mathematical expectations and coordinate functions 
for canonical expansions. The theory of normal stochastic systems is the basis of the 
developed approaches. An illustrative example is given. Applications to estimation, 
identification, and calibration problems are considered. Some generalizations are mentioned.}
   
\KWE{canonical expansion (CE); methods of analytical modeling (VFV);
 normalization by Pugachev; stochastic processes (StP); 
stochastic systems (StS); stochastic function system with stochastically unsolved derivatives}

\DOI{10.14357/19922264220105}
 
%\vspace*{-18pt}

%\Ack
%\noindent
%The work was financially supported by the Russian Academy of Sciences(project АААА-А19-119091990037-5).




%\vspace*{6pt}

  \begin{multicols}{2}

\renewcommand{\bibname}{\protect\rmfamily References}
%\renewcommand{\bibname}{\large\protect\rm References}

{\small\frenchspacing
 {%\baselineskip=10.8pt
 \addcontentsline{toc}{section}{References}
 \begin{thebibliography}{99}
\bibitem{1-sin-1}
\Aue{Sinitsyn, I.\,N.} 2017. 
Analiticheskoe modelirovanie shirokopolosnykh protsessov v~stokhasticheskikh sistemakh, 
ne razreshennykh otnositel'no proizvodnykh [Analytical modeling of wide band processes in 
stochastic systems with unsolved derivatives]. \textit{Informatika i~ee Primeneniya~--- Inform. Appl.}
 11(1):3--10.
\bibitem{2-sin-1}
\Aue{Sinitsyn, I.\,N.} 2017. Parametricheskoe analiticheskoe modelirovanie protsessov 
v~stokhasticheskikh sistemakh, ne razreshennykh otnositel'no proizvodnykh 
[Parametric analytical modeling of wide band processes in stochastic systems 
with unsolved derivatives]. \textit{Sistemy i~Sredstva Informatiki~--- 
Systems and Means of Informatics} 27(1):20--45.
\bibitem{3-sin-1}
\Aue{Sinitsyn, I.\,N.} 2021. Normal'nye suboptimal'nye fil'try dlya 
differentsial'nykh stokhasticheskikh sistem, ne razreshennykh otnositel'no proizvodnykh 
[Normal suboptimal filters for differential stochastic systems with unsolved derivatives]. 
\textit{Informatika i~ee Primeneniya~--- Inform. Appl.} 15(1):3--10.
\bibitem{4-sin-1}
\Aue{Sinitsyn, I.\,N.} 2021. Ana\-li\-ti\-che\-skoe mo\-de\-li\-ro\-va\-nie i~fil't\-ra\-tsiya
nor\-mal'\-nykh pro\-tses\-sov v~in\-teg\-ro\-dif\-fe\-ren\-tsi\-ro\-van\-nykh 
sto\-kha\-sti\-che\-skikh sis\-te\-makh, 
ne raz\-re\-shen\-nykh ot\-no\-si\-tel'\-no pro\-iz\-vod\-nykh
%Аналитическое моделирование и~фильт\-ра\-ция нормальных процессов 
%в~интегродифференциальных стохастических сис\-те\-мах, не разрешенных относительно производных 
[Analytical modeling and\linebreak filtering in integrodifferential systems with unsolved derivatives]. 
\textit{Sistemy i~Sredstva Informatiki~--- Systems and Means of Informatics} 31(1):37--56.
\bibitem{5-sin-1}
\Aue{Sinitsyn, I.\,N.} 2021. Analytical modeling and estimation of normal processes defined
 by stochastic equations with unsolved derivates. \textit{J.~Mathematics Statistics Research} 3(1):139. 7~p.
 
 \bibitem{8-sin-1} %6
\Aue{Pugachev, V.\,S., and I.\,N.~Sinitsyn.}
 1987. \textit{Stochastic differential systems. Analysis and filtering}. Chichester\,--\,New York: John Wiley \& Sons. 
 549~p.
\bibitem{9-sin-1} %7
\Aue{Pugachev, V.\,S., and I.\,N.~Sinitsyn.}
 2001. \textit{Stochastic systems. Theory and  applications}.
Singapore: World Scientific. 908~p.


\bibitem{7-sin-1} %8
\Aue{Sinitsyn, I.\,N.} 2009. \textit{Kanonicheskie predstavleniya sluchaynykh funktsiy i~ikh primenenie 
v~zadachakh komp'yuternoy podderzhki nauchnykh issledovaniy} [Canonical expansions of random 
functions and their applications in computer-aided support]. Moscow: TORUS PRESS. 768~p.

\bibitem{6-sin-1} %9
Sinitsyn, I.\,N., ed. 2011. \textit{Akademik Pugachev Vladimir Semenovich: k~stoletiyu so dnya rozhdeniya} 
[Academician Pugachev Vladimir Semenovich: To the centenary of his birth]. Moscow: TORUS PRESS. 376~p.

\bibitem{10-sin-1}
\Aue{Sinitsyn, I.\,N.}
 2019. \textit{Lektsii po teorii sistem integrirovannoy logisticheskoy podderzhki} 
 [Lectures on theory of integrated logistic support systems]. Moscow: TORUS PRESS. 1072~p.
\bibitem{11-sin-1}
\Aue{Sinitsyn, I.\,N.}
 2007. \textit{Fil'try Kalmana i~Pugacheva} [Kalman and Pugachev filters]. 2nd. ed. Moscow: Logos. 776~p.
 \end{thebibliography}

 }
 }

\end{multicols}

\vspace*{-9pt}

\hfill{\small\textit{Received March 22, 2021}}

%\pagebreak

\vspace*{-18pt}

\Contrl

\vspace*{-3pt}

\noindent
\textbf{Sinitsyn Igor N.} (b.\ 1940)~--- 
Doctor of Science in technology, professor, Honored scientist of RF, principal scientist, 
Institute of Informatics Problems, Federal Research Center ``Computer Science and Control'' 
of the Russian Academy of Sciences, 44-2~Vavilov Str., Moscow 119333, Russian Federation; \mbox{sinitsin@dol.ru}




\label{end\stat}

\renewcommand{\bibname}{\protect\rm Литература} 