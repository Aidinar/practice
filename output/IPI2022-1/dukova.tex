\def\stat{dukova}

\def\tit{О ПОИСКЕ МАКСИМАЛЬНЫХ ЧАСТЫХ И~МИНИМАЛЬНЫХ НЕЧАСТЫХ НАБОРОВ ПРОИЗВЕДЕНИЯ ЧАСТИЧНЫХ ПОРЯДКОВ}

\def\titkol{О поиске максимальных частых и~минимальных нечастых наборов произведения частичных порядков}

\def\aut{Н.\,А.~Драгунов$^1$, Е.\,В.~Дюкова$^2$}

\def\autkol{Н.\,А.~Драгунов, Е.\,В.~Дюкова}

\titel{\tit}{\aut}{\autkol}{\titkol}

\index{Драгунов Н.\,А.}
\index{Дюкова Е.\,В.}
\index{Dragunov N.\,A.}
\index{Djukova E.\,V.}


%{\renewcommand{\thefootnote}{\fnsymbol{footnote}} \footnotetext[1]
%{Работа выполнена при поддержке Министерства науки и~высшего образования Российской Федерации (проект 
%075-15-2020-799).}}


\renewcommand{\thefootnote}{\arabic{footnote}}
\footnotetext[1]{Федеральный исследовательский центр <<Информатика 
и~управ\-ле\-ние>> Российской академии наук, \mbox{nikitadragunovjob@gmail.com}}
\footnotetext[2]{Федеральный исследовательский центр <<Информатика и~управ\-ле\-ние>> 
Российской академии наук, \mbox{edjukova@mail.ru}}

\vspace*{-3pt}




\Abst{Исследованы актуальные вопросы снижения временных затрат, возникающие при 
логическом анализе данных с~элементами из декартова произведения конечных час\-тич\-но 
упорядоченных множеств. Для задачи поиска по базе транзакций максимальных час\-тых и~минимальных 
нечастых наборов произведения час\-тич\-ных порядков предложен оригинальный метод, 
основанный на решении слож\-ной дискретной задачи, называемой дуализацией 
над произведением час\-тич\-ных порядков. Метод представляет собой синтез двух других 
известных методов, один из которых достаточно очевиден, а~другой использует идею 
инкрементального пе\-ре\-чис\-ле\-ния искомых наборов и~поэтому пред\-став\-ля\-ет 
в~основном тео\-ре\-ти\-че\-ский интерес. Проведено экспериментальное исследование предложенного 
подхода к~решению рас\-смат\-ри\-ва\-емой задачи в~случае произведения конечных цепей,
 выявлены условия его эф\-фек\-тив\-ности и~для проводимого анализа данных показана 
 це\-ле\-со\-об\-раз\-ность применения асимптотически оптимальных алгоритмов дуализации 
 над произведением час\-тич\-ных порядков.}

\KW{максимальные час\-тые наборы; минимальные не\-час\-тые наборы; дуализация над 
произведением час\-тич\-ных порядков; асимп\-то\-ти\-чески оптимальный алгоритм дуализации}

\DOI{10.14357/19922264220112}
  
%\vspace*{-4pt}


\vskip 10pt plus 9pt minus 6pt

\thispagestyle{headings}

\begin{multicols}{2}

\label{st\stat}

    \section{Введение}
    
    Рас\-смат\-ри\-ва\-емая задача анализа данных занимает важ\-ное мес\-то в~об\-ласти 
    информационного поиска и~в~случае бинарных данных ставится сле\-ду\-ющим образом~\cite{4}.
    
    Дано некоторое множество элементов~$V$. Подмножества $X \hm\subseteq V$ называются наборами. Пусть~$D$~--- 
    база данных, содержащая некоторые, не обязательно различные, наборы. Наборы, 
    содержащиеся в~$D$, называются транз\-ак\-ци\-ями. Под частотой набора~$\nu(X)$ понимается доля транз\-ак\-ций в~$D$, 
    содержащих~$X$. Если $\nu(X) \hm\geq s$, $s \hm\in \left[0, 1\right]$, то набор~$X$ называется $s$-час\-тым, 
    иначе он называется $s$-не\-час\-тым. Если набор частый и~он не содержится ни в~каком другом 
    час\-том наборе, то такой набор называется максимальным час\-тым. Если набор не\-час\-тый 
    и~при этом он не содержит в~себе никакого другого не\-час\-то\-го набора, то такой набор 
    называется минимальным нечастым. Требуется найти все максимальные час\-тые и~минимальные не\-час\-тые 
    наборы при заданном~$s$.
    
    Рас\-смат\-ри\-ва\-емая задача имеет много важных приложений, одним из которых является 
    нахождение ассоциативных правил в~базах данных. В~случае бинарных данных ассоциативное правило~---
     это упорядоченная пара $ \left( X, Y \right)$ непересекающихся подмножеств множества~$V$, обо\-зна\-ча\-емая 
     $X \hm\Rightarrow Y$. Поддержкой правила $X \hm\Rightarrow Y$ называется час\-то\-та набора $Z\hm = X \cup Y$.
      Достоверностью правила $X\hm \Rightarrow Y$ называется доля транзакций, со\-дер\-жа\-щих~$Y$, 
      среди всех транзакций, содержащих~$X$. Требуется \mbox{найти} все ассоциативные правила, 
      удовле\-тво\-ря\-ющие заданным минимальной поддержке $s\hm \in [0, 1]$ и~минимальной 
      достоверности $c \hm\in [0, 1]$.  Впервые задача нахождения ассоциативных правил
       была поставлена в~\cite{1}, где она формулировалась как задача анализа по\-тре\-би\-тель\-ской корзины.

    В случае небинарных данных каждый элемент из~$V$ имеет некоторое множество чис\-ло\-вых значений 
    и~вместо наборов элементов рас\-смат\-ри\-ва\-ют\-ся наборы их значений.

    Поиск ассоциативных правил осуществляется в~два этапа. 
    На первом этапе находятся частые наборы, на втором этапе из найденных час\-тых 
    наборов формируются ассоциативные правила. При формировании правил на втором 
    этапе фактически возникает задача поиска $t$-не\-час\-тых наборов, где $t\hm > s/c$.
    
    С ростом размерности современных баз данных находить все час\-тые и~не\-час\-тые 
    наборы становится неэффективно как по времени, так и~по памяти в~силу 
    экспоненциального рос\-та чис\-ла таких наборов. Одно из решений данной проблемы 
    заключается в~поиске только максимальных час\-тых наборов и~только минимальных 
    нечастых наборов, что позволяет компактно хранить информацию о~всех час\-тых и~не\-час\-тых 
    наборах соответственно. 
    
    
    В~\cite{9} рас\-смот\-ре\-на задача поиска множеств максимальных час\-тых наборов~$X_{\max}$ 
    и~минимальных не\-час\-тых наборов~$Y_{\min}$ в~данных, пред\-став\-лен\-ных в~виде декартова 
    произведения час\-тич\-но упорядоченных множеств. Показано, что в~этом случае 
    при построении тре\-бу\-емых наборов возникают соответственно задача поиска 
    максимальных независимых элементов час\-тич\-ных порядков и~задача поиска минимальных 
    независимых элементов час\-тич\-ных порядков.  Каж\-дая из этих задач называется 
    дуализацией над произведением час\-тич\-ных порядков~\cite{8}. Обе задачи относятся к~одним 
    из цент\-раль\-ных труд\-но\-ре\-ша\-емых пе\-ре\-чис\-ли\-тель\-ных задач дис\-крет\-ной математики.
    
    Существует достаточно очевидный способ поиска максимальных час\-тых и~минимальных
     не\-час\-тых наборов произведения час\-тич\-ных порядков, основанный на по\-сле\-до\-ва\-тель\-ном 
     по\-стро\-ении указанных множеств. Одно из множеств ищется, например, алгоритмом Apriori~\cite{2},
      второе множество получается путем дуализации первого. 
      В~настоящей работе показано, что метод эффективен только в~случае, когда чис\-ло час\-тых 
      наборов существенно меньше или, наоборот, существенно больше чис\-ла не\-час\-тых наборов. 
      В~\cite{9} предложена идея со\-вмест\-но\-го пе\-ре\-чис\-ле\-ния~$X_{\max}$ и~$Y_{\min}$ с~использованием
       инкрементального алгоритма дуализации из~\cite{14}, которая автором экспериментально 
       не исследована.
    
    Основной результат настоящей работы~--- разработка нового подхода к~решению 
    поставленной задачи, который является синтезом последовательного и~совместного подходов. 
    
    Экспериментальные исследования, проведенные в~настоящей работе для случая
     произведения цепей, свидетельствуют о~том, что предложенный по\-сле\-до\-ва\-тель\-но-со\-вмест\-ный 
     метод наиболее эффективен в~случае, когда мощ\-ность множества час\-тых наборов примерно 
     равна мощ\-ности множества не\-час\-тых наборов.
     
     \vspace*{-6pt}
     
    
    \section{Постановка задачи поиска максимальных частых 
    и~минимальных нечастых наборов произведения частичных порядков}
    
         \vspace*{-2pt}
    
    Пусть $\mathcal{P} = \mathcal{P}_1 \times \dots \times \mathcal{P}_n$~--- 
    де\-кар\-то\-во произведение час\-тич\-но упорядоченных множеств. Элементы~$\mathcal{P}$ называются наборами. 
    На множестве~$\mathcal{P}$ определяется отношение частичного порядка~$\preceq$ сле\-ду\-ющим образом: 
    если $p \hm= (p_1, \dots, p_n) \hm\in \mathcal{P}$ и~$q \hm= (q_1, \dots, q_n)\hm \in \mathcal{P}$, 
    то $ p \hm\preceq q$ в~$ \mathcal{P}\hm \Leftrightarrow p_1 \hm\preceq q_1$ 
    в~$\mathcal{P}_1, \dots, p_n \hm\preceq q_n$ в~$ \mathcal{P}_n$.
    
    Пусть $\mathcal{D} (\mathcal{P})$~--- некоторая со\-во\-куп\-ность
     наборов из~$\mathcal{P}$, называемая базой данных. Наборы, на\-хо\-дя\-щи\-еся в~базе 
     данных $\mathcal{D} (\mathcal{P})$, необязательно по\-пар\-но раз\-лич\-ны и~называются транзакциями. 
     
    Введем обозначения: 
    $\vert \mathcal{D} (\mathcal{P}) \vert$~--- чис\-ло транз\-ак\-ций в~$\mathcal{D} (\mathcal{P})$; 
    $\mathcal{S}_\mathcal{D}(p)$~--- число транз\-ак\-ций в~$\mathcal{D} (\mathcal{P})$, 
    сле\-ду\-ющих за $p \hm\in \mathcal{P}$; $s \hm\in [0, 1]$. 
    
    \smallskip
    
    \noindent
    \textbf{Определение~1.}\
     Набор $p \in \mathcal{P}$ называется $s$-час\-тым, 
     если $\mathcal{S}_\mathcal{D}(p) / \vert \mathcal{D} (\mathcal{P}) \vert \hm\geq s$. Иначе набор~$p$ 
     называется $s$-не\-час\-тым.
    
    \smallskip
    
    \noindent
    \textbf{Определение~2.}\
    Набор $p \in \mathcal{P}$ называется максимальным $s$-час\-тым, если 
    он $s$-час\-тый и~никакой сле\-ду\-ющий за ним набор~$z$, $z\hm \neq p$, не является $s$-час\-тым.

    
    \smallskip
    
    \noindent
    \textbf{Определение~3.}\
    Набор $p \in \mathcal{P}$ называется минимальным $s$-не\-час\-тым, если он $s$-не\-час\-тый 
    и~никакой пред\-шест\-ву\-ющий ему набор~$z$, $z \hm\neq p$, не является $s$-не\-час\-тым.


\smallskip
    
    Далее вместо $s$-частый ($s$-не\-час\-тый) набор будем писать час\-тый (не\-час\-тый) набор. 
    Множество всех максимальных час\-тых наборов будем обозначать как $X_{\max}$, 
    а~множество всех минимальных не\-час\-тых наборов как $Y_{\min}$.
    
    Пусть $R \subset \mathcal{P}$, $R^+\hm = \{ x \in \mathcal{P} \vert \exists\, a \hm\in R, a \hm\preceq x \}$, 
    $R^- \hm= \{ x \hm\in \mathcal{P} \vert \exists\, a \hm\in R, x \hm\preceq a \}$.


    \noindent
    \textbf{Определение~4.}\
     Множество $I(R^+)$, со\-сто\-ящее из всех максимальных элементов множества~$\mathcal{P} \setminus R^+$, 
     называется максимальным независимым от~$R$.

\smallskip


   \noindent
    \textbf{Определение~5.}\
     Множество $I(R^-)$, со\-сто\-ящее из всех минимальных элементов множества~$\mathcal{P} \setminus R^-$, 
     называется минимальным независимым от~$R$.

\smallskip
    
    Каждая из задач построения $I(R^+)$ и~$I(R^-)$ 
    при заданном множестве~$R$ называется задачей дуализации над произведением час\-тич\-ных порядков.
    
    \smallskip

    \noindent
    \textbf{Утверждение~1.}\
    Если $X \hm\subset X_{\max}$, а~$y \hm\in I(X^-)$~--- не\-час\-тый набор, 
    то~$y$~--- минимальный не\-час\-тый набор.

\smallskip    
    
    \noindent
    Д\,о\,к\,а\,з\,а\,т\,е\,л\,ь\,с\,т\,в\,о\,.\  \ 
    Пусть $y \hm\notin I(X_{\max}^-)$. Так как~$y$~--- 
    нечастый набор, то в~$\mathcal{P} \setminus X^{-}_{\max}$ найдется минимальный не\-час\-тый набор~$x$ 
    такой, что $x\hm \neq y$ и~$x \hm\preceq y$. Из того, что $\mathcal{P} \setminus X^{-}_{\max} 
    \hm\subseteq \mathcal{P} \setminus X^-$, следует, что $x\hm \in \mathcal{P} \setminus X^-$, 
    что противоречит условию $y \hm\in I(X^-)$.

\smallskip

\noindent
\textbf{Утверждение~2.}\
    Пусть $X \hm\subseteq X_{\max}$, $Y\hm \subseteq Y_{\min}$. 
    Тогда $I(X^-) \hm= Y$ в~том и~только в~том случае, когда $X \hm= X_{\max}$ и~$Y \hm= Y_{\min}$.


\smallskip


  \noindent
    Д\,о\,к\,а\,з\,а\,т\,е\,л\,ь\,с\,т\,в\,о\,.\  \
    Пусть $X\! \subset\! X_{\max}, x \hm\in X_{\max}\!\setminus\!X$.
     Так как множество~$X_{\max}$~--- антицепь, то $x \hm\notin X^-$. 
     Следовательно, $x \hm\in \mathcal{P} \setminus X^{-}$.
      Но тогда существует элемент $ q \hm\in I(X^-) : q \preceq x$, 
      который является час\-тым. Однако во множестве~$Y$ частых наборов нет; следовательно, $I(X^-) \hm\neq Y$. 
      Если же $X \hm= X_{\max}$, то $I(X^-) \hm= Y_{\min}$. Таким образом, $I(X^-) \hm= Y$ тогда и~только
       тогда, когда $X \hm= X_{\max}$ и~$Y\hm = Y_{\min}$.


    
    \section{Методы построения множеств~$X_{\max}$ и~$Y_{\min}$}

    \subsection{Последовательное перечисление $X_{\max}$~и~$Y_{\min}$}

    Достаточно очевиден поиск~$X_{\max}$ и~$Y_{\min}$ при заданной $\mathcal{D} (\mathcal{P})$ 
    путем последовательного по\-стро\-ения множеств~$X_{\max}$ и~$Y_{\min}$. 
    Данный поиск осуществляется в~два этапа. На первом этапе находятся все максимальные частые 
    наборы~$X_{\max}$, например алгоритмом Apriori~\cite{2}. На втором этапе  используется свойство 
    двойственности $I \left(X_{\max}^- \right)\hm = Y_{\min}$. 
    Минимальные нечастые наборы~$Y_{\min}$ находятся путем дуализации найденного на первом этапе 
    множества~$X_{\max}$. Аналогично можно сначала искать~$Y_{\min}$ алгоритмом Apriori, а~затем 
    искать~$X_{\max}$ путем дуализации~$Y_{\min}$.

    Очевидно, что данный подход будет проявлять себя наилучшим образом в~случаях, когда 
    алгоритм Apriori или его модификации могут найти одно из искомых множеств существенно
     быст\-рее, чем другое множество, например когда мощ\-ность~$X_{\max}$ 
     существенно меньше (больше) мощ\-ности~$Y_{\min}$.
    
    \subsection{Совместное перечисление $X_{\max}$ и~$Y_{\min}$}

    В~\cite{9} предложена идея совместного перечисления множеств~$X_{\max}$ и~$Y_{\min}$. 
    На первом шаге рас\-смат\-ри\-ва\-ет\-ся некоторый случайный набор $q \hm\in \mathcal{P}$. Если $q$~--- 
    час\-тый набор, то ищется максимальный час\-тый набор, сле\-ду\-ющий за~$q$, 
    который пополняет множество $X \hm\subseteq X_{\max}$. Если $q$~---
     не\-час\-тый набор, то ищется минимальный не\-час\-тый набор, пред\-шест\-ву\-ющий~$q$, 
     который пополняет множество $Y \hm\subseteq Y_{\min}$. Пусть на шаге~$i$ ($i\hm \geq 1$) 
     построены множества $X \hm\subseteq X_{\max}$ и~$Y \hm\subseteq Y_{\min}$. Если $X \hm\neq \varnothing$, 
     $Y \hm= \varnothing$, то ищется набор~$q$ такой, что $q \hm\npreceq x, \forall x \hm\in X$. Если 
     $X \hm= \varnothing$, $Y \hm\neq \varnothing$, то ищется набор~$q$ такой, что 
     $q \hm\nsucceq y, \forall y \hm\in Y$. Если же и~$X \hm\neq \varnothing$, и~$Y \hm\neq \varnothing$, 
     то ищется набор~$q$ такой, что $q \hm\npreceq x, \forall x \hm\in X, q \hm\nsucceq y, \forall y \hm\in Y$.
      Затем, аналогично первому шагу, находится максимальный частый или минимальный нечастый набор. 
      Однако в~\cite{9} идея совместного перечисления искомых множеств экспериментально 
      не исследована и~не предложены конкретные указания по воз\-мож\-ной ее реализации.
    
    Алгоритм, основанный на совместном пе\-ре\-чис\-ле\-нии множеств~$X_{\max}$ и~$Y_{\min}$,
     реализован в~на\-сто\-ящей работе. Алгоритм строит две последовательности: $X_1 \hm\subset X_2 
     \subset \dots \subset X_{\max}$, $Y_1\hm \subset Y_2 \subset \dots \subset Y_{\min}$. 
     На первом шаге $X_1 \hm= \{x\}$, $Y_1 \hm= \{y\}$, где~$x$ и~$y$ ищутся алгоритмом Apriori.
      На шаге $i \hm+ 1$ ($i\hm \geq 1$) строится либо~$I(X^{-}_{i})$, либо~$I(Y^{+}_{i})$. Пусть на 
      шаге $i \hm+ 1$ ($i \hm\geq 1$) построено множество~$I(X^{-}_{i})$. 
      Согласно утверждениям~1 и~2, множество~$I(X^{-}_{i})$ либо не содержит час\-тых наборов 
      и~совпадает с~множеством~$Y_{\min}$ (в~этом случае $X_i \hm= X_{\max}$ 
      и~алгоритм заканчивает работу), либо~$I(X^{-}_{i})$ содержит как час\-тые, так и~не\-час\-тые наборы. 
      Каждый нечастый набор из~$I(X^{-}_{i})$ является минимальным не\-час\-тым и~пополняет множество~$Y_{i}$, 
      формируя в~результате множество~$Y_{i+1}$. Для каждого час\-то\-го набора находится один содержащий 
      его максимальный час\-тый набор путем последовательного увеличения текущего 
      частого набора в~лексикографическом порядке, который пополняет множество~$X_{i}$, 
      формируя в~результате множество~$X_{i+1}$.
      
    В~экспериментальной части работы (см.\ разд.~4) рас\-смот\-рен случай произведения цепей. 
    Задача дуализации решается с~помощью асимптотически оптимального алгоритма дуализации
     цепей \mbox{RUNC-M}+~\cite{7}. Асимптотически оптимальные алгоритмы дуализации 
     являются лидерами по ско\-рости счета~\cite{6}.

    Очевидно, что время работы совместного алгоритма в~основном зависит от чис\-ла
     минимальных не\-час\-тых и~максимальных час\-тых наборов. На\linebreak каж\-дой новой 
     итерации происходит дуализация\linebreak все б$\acute{\mbox{о}}$льших по мощ\-ности множеств~$X$ или~$Y$.\linebreak 
     Если число итераций становится достаточно\linebreak большим, то ско\-рость работы совместного 
     перечисления существенно снижается, что делает его практически неприменимым для 
     задач большой раз\-мер\-ности.
     { %\looseness=1
     
     }

    \subsection{Последовательно-совместное перечисление~$X_{\max}$ и~$Y_{\min}$}

    Предлагается следующий итеративный метод, который синтезирует идеи последовательного
     и~совместного методов, описанных выше. Положим $X_0 \hm= \varnothing$. 
     Строится одна по\-сле\-до\-ва\-тель\-ность $X_1 \hm\subset X_2 \hm\subset \dots \subset X_{\max}$. 
     На первом шаге $X_1\hm = \{x\}$, где $x$ ищется алгоритмом Apriori. На шаге $i \hm+ 1$ ($i \hm\geq 1$) 
     решается задача дуализации множества $X_{i} \setminus X_{i-1}$.

    
    
   \setcounter{figure}{1}
    \begin{figure*}[b] %fig2
  \vspace*{12pt}
  \begin{center}  
    \mbox{%
\epsfxsize=163mm
\epsfbox{duk-2.eps}
}

\end{center}
\vspace*{-9pt}
    \Caption{Зависимость времени работы алгоритмов от суммы мощностей множеств~$X_{\max}$ и~$Y_{\min}$ 
    для случая~1~(\textit{а}) и~2~(\textit{б}):
    \textit{1}~--- по\-сле\-до\-ва\-тель\-но-со\-вмест\-ный;
    \textit{2}~--- последовательный; \textit{3}~--- совместный; \textit{4}~--- Apriori}
    \label{12}
    \end{figure*}
     
    Пусть множество~$D$ есть результат дуализации $X_{i} \hm\setminus X_{i-1}$. Согласно утверждению~1, 
    множество~$D$ содержит частые наборы. Для каждого час\-то\-го набора из~$D$ 
    находится один содержащий его максимальный час\-тый набор путем последовательного 
    увеличения текущего час\-то\-го набора в~лексикографическом порядке. Все найденные максимальные
     частые наборы, которых нет в~множестве~$X_{i}$, до\-бав\-ля\-ют\-ся к~$X_{i}$, 
     и~таким образом формируется~$X_{i+1}$. Если же все найденные частые наборы уже содержатся в~$X_{i}$, 
     то решается задача дуализации множества~$X_{i}$. Если в~$I(X^{-}_{i})$ нет частых наборов, 
     то $I(X^{-}_{i})\hm = Y_{\min}$, $X_i \hm= X_{\max}$ и~алгоритм завершает работу. 
     Иначе для каждого частого набора из~$I(X^{-}_{i})$ находится один содержащий его максимальный 
     час\-тый набор, который пополняет множество~$X_{i}$, формируя в~результате множество~$X_{i+1}$.

    \section{Экспериментальное исследование}
    
    Рас\-смат\-ри\-вал\-ся случай данных, пред\-став\-лен\-ных в~виде произведения цепей мощ\-ности~5. 
    Для\linebreak таких данных проводился поиск максимальных час\-тых и~минимальных нечастых 
    наборов сле\-ду\-ющи\-ми методами: алгоритмом Apriori, модифицированным для случая 
    цепей; последовательным \mbox{методом}; совместным методом; по\-сле\-до\-ва\-тель\-но-со\-вмест\-ным методом.
    
    Все методы реализованы на языке Python~3. 
    Задача дуализации решалась алгоритмом дуализации цепей RUNC-M+~\cite{7}. 
    Эксперименты проведены на случайных базах данных различной раз\-мер\-ности. 
    Можно выделить два сле\-ду\-ющих случая соотношения мощностей множеств всех час\-тых и~не\-час\-тых наборов.
    \begin{description}
    \item[Случай 1:] мощ\-ность множества частых наборов примерно рав\-на мощ\-ности множества нечастых наборов.
    \item[Случай 2:] мощ\-ность множества частых наборов существенно меньше (больше) мощ\-ности множества 
    не\-час\-тых наборов.
    \end{description}
    
    Описанные случаи схематично изображены на рис.~1. 

    Графики зависимости времени работы тестируемых методов 
    от мощ\-ности множеств~$X_{\max}$ и~$Y_{\min}$ приведены на рис.~2.
    
    

    

    Нетрудно видеть, что в~случае~1 лучше работает по\-сле\-до\-ва\-тель\-но-со\-вмест\-ный алгоритм: 
    множества час\-тых и~не\-час\-тых наборов имеют примерно одинаковую мощ\-ность, 
    поэтому быст\-рее будет обрабатывать их по\-сле\-до\-ва\-тель\-но-со\-вмест\-ным методом. В~случае~2 
    быст\-рее работает последовательный алгоритм: быст\-рее найти множество максимальных час\-тых наборов, 
    обработав множество час\-тых наборов, и~дуализировать результат. Время поиска множеств~$X_{\max}$ 
    и~$Y_{\min}$ совместным методом и~модифицированным алгоритмом Apriori рас\-тет существенно 
    быст\-рее времени поиска по\-сле\-до\-ва\-тель\-но-со\-вмест\-ным методом в~обоих случаях.
    
    { \begin{center}  %fig1
 \vspace*{9pt}
    \mbox{%
\epsfxsize=67.963mm
\epsfbox{duk-1.eps}
}

\end{center}

\noindent
{{\figurename~1}\ \ \small{
Два случая соотношения мощностей множеств час\-тых и~не\-час\-тых наборов
}}}

%\vspace*{6pt}


    \section{Заключение}
    
Рас\-смот\-ре\-на задача поиска максимальных час\-тых и~минимальных не\-час\-тых наборов в~данных, 
представленных в~виде декартова произведения час\-тич\-ных порядков. Актуальны вопросы 
снижения временн$\acute{\mbox{ы}}$х затрат, возникающих при реализации методов нахождения искомых наборов.
 Разработан новый подход к~по\-стро\-ению множества максимальных частых наборов~$X_{\max}$ и~множества 
 минимальных не\-час\-тых наборов~$Y_{\min}$, пред\-став\-ля\-ющий собой синтез двух ранее известных 
 подходов: последовательного и~со\-вмест\-но\-го (первый достаточно очевиден, идея второго предложена в~\cite{9}). 
 Сложность последовательного, совместного и~пред\-ла\-га\-емо\-го по\-сле\-до\-ва\-тель\-но-со\-вмест\-но\-го поиска 
 обуслов\-ле\-на, в~том чис\-ле, не\-об\-хо\-ди\-мостью рас\-смат\-ри\-вать в~процессе поиска 
 труд\-но\-ре\-ша\-емую пе\-ре\-чис\-ли\-тель\-ную задачу дис\-крет\-ной математики, на\-зы\-ва\-емую дуализацией 
 над произведением час\-тич\-ных порядков.

Для случая, когда данные пред\-став\-ле\-ны в~виде произведения конечных цепей, 
приведены результаты экспериментального срав\-не\-ния названных подходов, а~так\-же независимого 
способа \mbox{по\-стро\-ения} множеств~$X_{\max}$ и~$Y_{\min}$, не тре\-бу\-юще\-го решения задачи дуализации. 
Эксперименты проводились на модельных задачах с~применением асимптотически оптимального
 алгоритма дуализации над произведением конечных цепей \mbox{RUNC-M}+~\cite{7}. 
 Результаты исследования свидетельствуют о~том, что по\-сле\-до\-ва\-тель\-но-со\-вмест\-ный 
 метод наиболее эффективен (требует меньших временн$\acute{\mbox{ы}}$х затрат по сравнению с~другими рас\-смот\-рен\-ны\-ми 
 методами) в~случае, когда мощ\-ность множества час\-тых наборов примерно равна мощ\-ности множества
  нечастых наборов. Иначе выигрывает последовательный поиск. Наихудшие показатели 
  у~независимого пе\-ре\-чис\-ле\-ния множеств~$X_{\max}$ и~$Y_{\min}$ с~использованием в~качестве
   базового алгоритма Apriori~\cite{2}, точ\-нее его модификации на тес\-ти\-ру\-емый случай. 
   Таким образом, показана це\-ле\-со\-об\-раз\-ность применения алгоритмов дуализации для 
   по\-стро\-ения множеств~$X_{\max}$ и~$Y_{\min}$.

  
  {\small\frenchspacing
 {%\baselineskip=10.8pt
 %\addcontentsline{toc}{section}{References}
 \begin{thebibliography}{9}  
    \bibitem{4}
    \Au{Aggarwal C.} 
    Frequent pattern mining.~--- Heidelberg: Springer, 2014. 467~p.
    
    \bibitem{1}
    \Au{Agrawal~R., Imielinski~T., Swami~A.} Mining association rules 
    between sets of items in large databases~// \mbox{SIGMOD} Conference (International) on Management of Data
    Proceedings.~--- New York, NY, USA: ACM, 1993. P.~207--216.
    
    \bibitem{9}
    \Au{Elbassioni K.} On finding minimal infrequent elements in multi-dimensional 
    data defined over partially ordered sets~// arXiv.org, 2014. 30~p. arXiv:1411.2275 [cs.DB].
    
    \bibitem{8}
    \Au{Elbassioni K.} Algorithms for dualization over products of partially 
    ordered sets~// SIAM J.~Discrete Math., 2009. Vol.~23. Iss.~1. P.~487--510.
    
    \bibitem{2}
    \Au{Agrawal R., Srikant~R.} 
    Fast algorithms for mining association rules in large databases~// 
    20th Conference (International) on Very Large Data Bases Proceedings.~--- San Francisco, CA, USA: 
    Morgan Kaufmann Publs. Inc., 1994. P.~487--499.
    
    \bibitem{14}
    \Au{Хачиян Л.\,Г.} Избранные труды.~--- М.: МЦНМО, 2009. 520~с.
    
    \bibitem{7}
    \Au{Дюкова Е.\,В., Масляков~Г.\,О., Прокофьев~П.\,А.} 
    О~дуализации над произведением частичных порядков~// Машинное обучение и~анализ данных, 2017. Т.~3. №\,4.  
    C.~239--249.
    
    \bibitem{6}
    \Au{Дюкова Е.\,В., Прокофьев~П.\,А.} Об асимптотически оптимальных алгоритмах дуализации~// 
    Ж.~вычисл. матем. и~матем. физ., 2015. Т.~55. №\,5. С.~895--910.
    \end{thebibliography}

 }
 }

\end{multicols}

\vspace*{-6pt}

\hfill{\small\textit{Поступила в~редакцию 15.01.21}}

\vspace*{8pt}

%\pagebreak

%\newpage

%\vspace*{-28pt}

\hrule

\vspace*{2pt}

\hrule

%\vspace*{-2pt}

\def\tit{FINDING MAXIMAL FREQUENT AND~MINIMAL INFREQUENT SETS IN~PARTIALLY ORDERED DATA}


\def\titkol{Finding maximal frequent and~minimal infrequent sets in~partially ordered data}


\def\aut{N.\,A.~Dragunov and E.\,V.~Djukova}

\def\autkol{N.\,A.~Dragunov and E.\,V.~Djukova}

\titel{\tit}{\aut}{\autkol}{\titkol}

\vspace*{-11pt}


\noindent
Federal Research Center ``Computer Science and Control'' 
of the Russian Academy of Sciences, 44-2~Vavilov Str., Moscow 119333, Russian Federation

\def\leftfootline{\small{\textbf{\thepage}
\hfill INFORMATIKA I EE PRIMENENIYA~--- INFORMATICS AND
APPLICATIONS\ \ \ 2022\ \ \ volume~16\ \ \ issue\ 1}
}%
 \def\rightfootline{\small{INFORMATIKA I EE PRIMENENIYA~---
INFORMATICS AND APPLICATIONS\ \ \ 2022\ \ \ volume~16\ \ \ issue\ 1
\hfill \textbf{\thepage}}}

\vspace*{3pt} 


\Abste{Relevant issues of time costs reducing in the logical analysis of data with elements 
from the Cartesian product of finite partially ordered sets are investigated. 
An original method based on solving a complex discrete problem called dualization
 over the product of partial orders is proposed for the problem of finding maximal 
 frequent and minimal infrequent sets in the transaction database. The proposed method 
 is a~synthesis of two other known methods, one of which is quite obvious and the other uses 
 the idea of an incremental enumeration of target\linebreak\vspace*{-12pt}}
 
 \Abstend{sets and is, therefore, mainly 
 of theoretical interest. An experimental study of the considered approaches in
  the case of the product of finite chains is carried out and conditions for
   their effectiveness are revealed. The expediency of applying 
asymptotically optimal dualization algorithms over the product of partial orders is shown.}

\KWE{maximal frequent sets; minimal infrequent sets; dualization over the product of 
partial orders; asymptotically optimal dualization algorithm}

\DOI{10.14357/19922264220112}

%\vspace*{-16pt}

%\Ack
%\noindent




%\vspace*{6pt}

  \begin{multicols}{2}

\renewcommand{\bibname}{\protect\rmfamily References}
%\renewcommand{\bibname}{\large\protect\rm References}

{\small\frenchspacing
 {%\baselineskip=10.8pt
 \addcontentsline{toc}{section}{References}
 \begin{thebibliography}{9}
\bibitem{1-dr}
\Aue{Aggarwal, C.} 2014. \textit{Frequent pattern mining}. Heidelberg: Springer. 467~p.
\bibitem{2-dr}
\Aue{Agrawal, R., T.~Imielinski, and A.~Swami.}
 1993. Mining association rules between sets of items in large databases. 
 \textit{SIGMOD  Conference (International) on Management of Data Proceedings}. New York, NY:
 ACM. 207--216. 
\bibitem{3-dr}
\Aue{Elbassioni, K.}
 2014. On finding minimal infrequent elements in multidimensional data defined over partially ordered sets. 
 arXiv.org. 30~p. Available at: 
 {\sf https://arxiv.org/\linebreak pdf/1411.2275.pdf} (accessed January~25, 2022).
\bibitem{4-dr}
\Aue{Elbassioni, K.} 2009. Algorithms for dualization over products of partially ordered sets. 
\textit{SIAM J.~Discrete Math.} 23(1):487--510.
\bibitem{5-dr}
\Aue{Agrawal, R., and R.~Srikant.}
 1994. Fast algorithms for mining association rules in large databases. 
 \textit{20th Conference (International) on Very Large Data Bases Proceedings}.
 San Francisco, CA: 
    Morgan Kaufmann Publs. Inc.  487--499.
\bibitem{6-dr}
\Aue{Khachiyan, L.\,G.} 2009. \textit{Izbrannye trudy} [Selected works]. Moscow: MCCME. 520~p.
\bibitem{7-dr}
\Aue{Djukova, E.\,V., G.\,O.~Maslyakov, and P.\,A.~Prokofyev.} 
2017. O~dualizatsii nad proizvedeniem chastichnykh poryadkov [On dualization over the product of 
partial orders]. \textit{Mashinnoe obuchenie i~analiz dannykh} [J.~Machine Learning Data Analysis] 
3(4):239--249.
\bibitem{8-dr}
\Aue{Djukova, E.\,V., and P.\,A.~Prokofyev.}
 2015. Asymptotically optimal dualization algorithms. \textit{Comp. Math.
 Math. Phys.} 55(5):891--905. 
 
 \end{thebibliography}

 }
 }

\end{multicols}

\vspace*{-6pt}

\hfill{\small\textit{Received January 15, 2021}}

%\pagebreak

%\vspace*{-18pt}

\Contr

\noindent
\textbf{Dragunov Nikita A.} (b.\ 1997)~--- 
PhD student, Federal Research Center ``Computer Science and Control'' 
of the Russian Academy of Sciences, 44-2~Vavilov Str., Moscow 119333, Russian Federation; 
\mbox{nikitadragunovjob@gmail.com}

\vspace*{3pt}

\noindent
\textbf{Djukova Elena V.} (b.\ 1945)~--- 
Doctor of Science in physics and mathematics, principal scientist, Federal Research Center
``Computer Science and Control'' of the Russian Academy of Sciences, 44-2~Vavilov Str., Moscow 119333, 
Russian Federation; \mbox{edjukova@mail.ru}




\label{end\stat}

\renewcommand{\bibname}{\protect\rm Литература} 