
   \vspace*{-46pt}

\begin{center}
\vspace*{4pt}
\mbox{%

\epsfxsize=55mm %112.705
\epsfbox{zhur-2.eps}
}
%\end{center}

\vspace*{10pt} 


%   \begin{center}
\fbox{\large\textbf{Академик Юрий Иванович Журавлёв}}\\[10pt]
\textbf{\large 14.01.1935--14.01.2022}
   \end{center}


   %\vspace*{2.5mm}

   \vspace*{5mm}

   \thispagestyle{empty}

%\

%\vspace*{-12pt}
       


В январе этого года ушел из жизни главный научный сотрудник Федерального исследовательского 
центра <<Информатика и управление>> РАН, председатель Редакционного совета журнала 
<<Информатика и~её применения>> академик Юрий Иванович Журавлёв. В~его лице мировая 
наука потеряла одного из своих ярчайших представителей~--- выдающегося ученого-исследователя 
и~талантливого ученого-организатора.

Юрий Иванович родился в Воронеже в 1935~г.\ в семье ученого и врача. Среднее образование 
получил в школе №\,6 г.~Фрунзе (ныне Бишкек) Киргизской ССР. В~1952~г.\ поступил на 
ме\-ха\-ни\-ко-ма\-те\-ма\-ти\-че\-ский факультет МГУ им.\ М.\,В.~Ломоносова. В~1957~г.\ Юрий Иванович 
защищает диплом и продолжает обучение в аспирантуре Московского университета на кафедре 
вычислительной математики (возглавляемой тогда академиком С.\,Л.~Соболевым). После 
успешной защиты кандидатской диссертации (к.ф.-м.н., 1959 г., научный руководитель~--- 
А.\,А.~Ляпунов, оппоненты~--- чл.-корр.\ А.\,А.~Марков, к.ф.-м.н.\ О.\,Б.~Лупанов) и~до 
окончательного переезда в Москву в 1969~г.\ работал в Институте математики Сибирского 
отделения АН СССР, занимая в нем последовательно должности младшего научного сотрудника, 
заведующего отделом, заведующего отделением, заместителя директора по научной работе. 
В~этот период (1954--1966~гг.)\ им был опубликован цикл работ по решению задач алгебры и 
математической логики, причем полученные результаты применялись для создания эффективных 
программ для ЭВМ, конструирования схем и сетей для обработки информации. Наиболее значимый 
результат этого периода научной работы~--- обоснование нового направления исследований, 
общей теории локальных алгоритмов. В~ней были окончательно объединены топологические 
принципы и теория алгоритмов. Эта теория и легла в основу докторской диссертации Юрия 
Ивановича (д.ф.-м.н., 1965~г.)\ по еще тогда новой научной специальности <<Математическая 
кибернетика>>. Оппонировали ему как специалисты по кибернетике~--- академик 
В.\,М.~Глушков, член-корреспондент А.\,А.~Ляпунов и О.\,Б.~Лупанов, так и про\-фес\-сор-ал\-геб\-раист А.\,Д.~Тайманов. 

В 1969~г.\ Юрий Иванович переезжает в Москву и возглавляет в Вычислительном центре АН 
СССР лабораторию проблем распознавания. Впоследствии он~--- заместитель директора по 
научной работе. Научные интересы этого периода связаны с проблемами классификации или 
распознавания образов. В~1976--1978~гг.\ Юрий Иванович публикует цикл работ по ставшему 
вскоре знаменитым алгебраическому подходу к проблеме синтеза корректных алгоритмов. Эти 
работы определили современное состояние всей проблематики распознавания и многих смежных 
областей прикладной математики и информатики. В~своих основополагающих работах Юрий 
Иванович показал, что можно в явном виде строить экстремальные по качеству алгоритмы для 
решения очень широких классов плохо формализованных задач. 
{\looseness=-1

}





Научные заслуги Юрия Ивановича получили широкое признание. В~1966~г.\ он совместно с 
О.\,Б.~Лупановым и чле\-ном-кор\-рес\-пон\-ден\-том АН СССР С.\,В.~Яблонским были удостоены 
звания лауреата Ленинской премии в~об\-ласти науки и техники. В~1984~г.\ Юрий Иванович 
был избран членом-корреспондентом АН СССР (по специальности <<Информатика>>), 
а~в~1992~г.~--- академиком РАН (по той же специальности).\linebreak\vspace*{-12pt}

\pagebreak

\

\vspace*{-46pt}

\noindent
\begin{floatingfigure}{48mm}
\begin{center}
%\vspace*{6pt}
\mbox{%

\epsfxsize=46mm %112.705
\epsfbox{zhur-3.eps}
}
\end{center}
\vspace*{6pt}
\end{floatingfigure}

 \thispagestyle{empty}

\noindent
В~1986~г.\ за цикл прикладных 
работ ему и ряду его учеников была при\-суж\-де\-на премия Совета Министров СССР. Он являлся 
членом иностранных академий наук, председателем секции <<Прикладная математика
 и~информатика>> Отделения математических наук РАН, председателем экспертного совета ВАК 
России по управ\-ле\-нию и информатике, заслуженным профессором нескольких университетов, 
председателем Российской ассоциации <<Распознавание образов и обработка изображений>>, 
членом исполкома Международной ассоциации IAPR (распознавание образов и обработка 
изображений). Был награжден 8-ю орденами и медалями СССР и России.

Юрий Иванович проводил большую научно-литературную работу, являясь, в том числе, главным 
редактором международных научных журналов и членом редколлегий ряда рецензируемых 
научных журналов. 


Параллельно с активной научной деятельностью Юрий Иванович вел и преподавательскую 
работу. С~1961 по~1969~гг.~--- в Новосибирском государственном университете на кафедре 
алгебры и математической логики, которую возглавлял в то время академик А.\,И.~Мальцев. 
С~1970~г., будучи уже профессором (1967~г.),~--- в Московском физико-техническом институте 
на кафедре академика Н.\,Н.~Моисеева. В~1997~г.\ по предложению ректора МГУ им.\ 
М.\,В.~Ломоносова академика В.\,А.~Садовничего Юрий Иванович организовал на факультете 
Вычислительной математики и кибернетики новую кафедру <<Математические методы 
прогнозирования>>, которой и руководил до конца жизни. В~2008~г.\ ему была присуждена 
премия Совета Министров РФ в области образования. С~1965~г.\ Юрий Иванович периодически 
читал курсы лекций за рубежом, в университетах США, Франции, Финляндии, Швеции, Австрии, 
Польши, Болгарии, ГДР и других стран. Эта работа в существенной степени обеспечила широкое 
международное признание советской и российской науки в области дискретной математики и~распознавания образов. 

%\begin{floatingfigure}{60mm}
\begin{figure}[b]
\begin{center}
\vspace*{-6pt}
\mbox{%

\epsfxsize=112mm %90mm %112.705
\epsfbox{zhur-1.eps}
}
\end{center}
\end{figure}
%\end{floatingfigure}

Понимая важность вопроса воспитания подрастающего поколения для развития науки в стране, 
Юрий Иванович вскоре после защиты первой диссертации включился в работу по подготовке 
научных кадров. Им создана большая научная школа: под руководством Юрия Ивановича 
защищены более 100~кандидатских диссертаций по всевозможным разделам естествознания 
(математике, информатике, медицине, технике, экономике, геологии), не один десяток докторов 
наук. Он воспитал академиков и членов-корреспондентов РАН и академий государств СНГ. 
С~большим вниманием и участием Юрий Иванович относился к развитию научных школ страны 
в~об\-ласти обработки изображений, распознавания образов и компьютерной оптики. 

Для всех коллег и учеников Юрия Ивановича он останется примером замечательного человека, 
та\-лант\-ли\-во\-го педагога и выдающегося, преданного служению науке ученого. 