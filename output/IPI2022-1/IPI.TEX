\documentclass[10pt]{book}
\usepackage[utf8]{inputenc}

\usepackage{latexsym,amssymb,amsfonts,amsmath,amsxtra,dsfont,
indentfirst,shapepar,%fleqn,%
picinpar,shadow,floatflt,enumerate,multicol,colortbl,moreverb,cite,ipi}

\usepackage{rotating}
\usepackage{mathrsfs}
\usepackage[noend]{algorithmic}
\usepackage{ulem}
\usepackage{graphicx}
%\usepackage{algorithm2e}
\usepackage[linesnumbered,boxed,ruled]{algorithm2e}
%\usepackage{xypic}
\usepackage{oldgerm}
\usepackage{epic}
\usepackage{eepic}

\SetAlgorithmName{Algorithm}{алгоритм}{Список алгоритмов}

%из Дюковой

\newcommand{\algKeyword}[1]{{\bf #1}}
\newcommand{\Proc}[1]{\text{\tt #1}}
\def\CALL{\algKeyword{call}~}

\newenvironment{AlgProcedure}[1]
{
\small
\medskip
%    \hrule
\medskip
\algKeyword{PROCEDURE} #1
\begin{algorithmic}[1]}
{\end{algorithmic}
%    \hrule
\bigskip
}

\def\CALL{\algKeyword{call}~}

%конец для Дюковой

%\RequirePackage[ruled]{algorithm}


\input{epsf}

%\nofiles

%\includeonly{avtor}    %pdf
%\includeonly{podgot-rus-site,podgot-eng-site}  
%\includeonly{podgot-rus,podgot-eng}  
%\includeonly{ipi-ind} 
%\includeonly{index-15i}
%\includeonly{toc-rus, toc-en}
%\includeonly{toc-rus}
%\includeonly{toc-en} 
%\includeonly{popravka}



%\includeonly{andrianova}        %pdf+авт+
%\includeonly{peshkova}          %pdf+авт+
%\includeonly{kovalev}           %pdf+авт+
%\includeonly{malashenko}        %pdf+авт+
%\includeonly{shihievi}          %pdf+авт+
%\includeonly{bitbos}            %pdf+авт+
%\includeonly{torshin}           %pdf+авт+
%\includeonly{sinitsyn}          %pdf+авт+
%\includeonly{smirnov}      %pdf
%\includeonly{listopad}          %pdf+авт+
%\includeonly{agalarov}          %pdf+авт
%\includeonly{dukova}       %pdf
%\includeonly{abgaryan}     %pdf



%\includeonly{nekrolog-new}
%\includeonly{nekrolog-Zhuravlev}  %pdf


%\includeonly{rekl}




\usepackage{acad}
%\usepackage{courier}
\usepackage{decor}
\usepackage{newton}
\usepackage{pragmatica}
\usepackage{zapfchan}
\usepackage{petrotex}
\usepackage{bm}                     % полужирные греческие буквы
\usepackage{upgreek}                % прямые греческие буквы \upalpha
\usepackage{eufrak}
\usepackage{verbatim}

\renewcommand{\bottomfraction}{0.99}
\renewcommand{\topfraction}{0.99}
\renewcommand{\textfraction}{0.01}

\setcounter{secnumdepth}{1} %здесь - 3 + chapter = 4

\arraycolsep=1.5pt

%\usepackage[pdftex]{graphicx}

%\usepackage{oz}

%NEW COMMANDS


\renewcommand*{\hm}[1]{#1\nobreak\discretionary{}%
            {\hbox{$\mathsurround=0pt #1$}}{}} %% Дублирует знаки операций
                               %при переносе в формуле (перед знаком, который
                               %надо продублировать ставится команда \hm)

%\newcommand{\endproof}{\hfill$\Box$}
\renewcommand{\r}{\mathbb{R}}
%\newcommand{\I}{{\rm I\hspace{-0.7mm}I}}
%\newcommand{\Ikl}{{\tt{1}}\hspace*{-1.44mm}\mathtt{1}}
\newcommand{\Ik}{\mbox{{\small \tt {1}}\hspace{-1.3mm}{\tt 1}}}
\newcommand{\argmin}{\mathop{\mathrm{arg}\,\mathrm{min}}}
\newcommand{\argmax}{\mathop{\mathrm{arg}\,\mathrm{max}}}
%\newcommand{\capr}{\mathop{\cap\,}}
%\newcommand{\cupr}{\mathop{\cup\,}}
%\def\argmin{\mathop{arg\,min}}

\def\vrp{\varphi}
\def\prt{\partial}
\def\mm{{\sf M}}
\def\modnop#1{\mathop{#1}\limits_{n}}
\def\eam{\mathbin{{\mathop{=}\limits^{\mathrm{def}}}}}
\def\dey#1#2{#1 (#2)}
\def\deyc#1#2{#1 \cdot  #2}
\def\ra#1{\;\mathop{\to}\limits^{#1}\;}
\def\raz#1{\;\mathop{\longrightarrow}\limits^{\!\!\!#1}\;}
\def\ral#1{\;\mathop{\longrightarrow}\limits^{#1}\;}

\newcommand{\Nor}{\mathcal{N}}
\newcommand{\T}{\mathbb{T}}
\newcommand{\Z}{\mathbb{Z}}



\newcommand{\il}[2]{\int\limits_{#1}^{#2}}%интеграл с пределами #1 и #2

\def\sm2{\mathop {\sum\limits^{n^\Theta}\sum\limits^{n^\Theta}}}
\def\sss{\sum\limits}
\def\tr{,\,\ldots\,,\,}
\def\rk{\right]}
\def\lk{\left[}
\def\rf{\right\}}
\def\lf{\left\{}
\def\lv{\,\left\vert}
\def\rv{\right\vert\,}
\def\iii{\int\limits}
\def\iin{\int\limits_{-\infty}^\infty}
\def\rrv{\right\vert}


\def\ee{{\cal E}}
\def\ww{{\cal W}}
\def\yy{{\cal Y}}
\def\vv{{\cal V}}

\newcommand{\R}{\mathbb R}
\newcommand{\E}{\mathbb E}
\newcommand{\N}{\mathbb N}

\renewcommand{\P}{\mathbb{P}}

\newcommand{\h}{{\bf H}}
\newcommand{\p}{{\sf P}}  % вероятность

\newcommand{\e}{{\sf E}}  % мат. ожидание
\newcommand{\D}{{\sf D}}  % дисперсия
\newcommand{\eps}{\varepsilon}
\newcommand{\vp}{{\mathbf p}}
\newcommand{\vz}{{\mathbf z}}
\newcommand{\vx}{{\mathbf x}}
\newcommand{\vf}{{\mathbf f}}
\newcommand{\F}{{\mathcal F}}
\def\ap{{\mathrm{ЭР}}}
\newcommand{\ud}{\Delta_n} %uniform ditance
\newcommand{\nud}{\Delta_n(x)}
%\renewcommand{\Re}{\mathrm{Re}\,}

\newcommand{\abs}[1]{\left\vert#1\right\vert}

\newcommand{\norm}[1]{\left\Vert#1\right\Vert}
\def\da{(\Delta_t,A)}

\newcommand{\corr}{\mathrm{corr}}

\newcommand{\cov}{\mathrm{cov}}
\newcommand{\Expect}{\mathbb{E}}

\def\w{\omega}
\def\W{\Omega}

\def\inh{\int\limits_{nh}^{(n+1)h}}

\def\sumin{\sum_{i=1}^N}


\def\bxt{(Y,t)}
\def\xt{(y,t)}

\def\ovth{{\fr{\tau-nh}{h}}}
\def\ov{\overline}
\def\tm{\tilde m}
\def\tl{\tilde\lambda}
\def\tB{\widetilde B}
\def\tb{\tilde b}
\def\ld{\ldots}
\def\cd{\cdots}


\DeclareMathOperator{\sign}{sign}

%\newcommand{\gr}{{\geqslant}}


\newcommand{\g}{\mbox{\textit{g}}}

\renewcommand{\la}{\lambda}
\newcommand{\si}{\sigma}
\newcommand{\alp}{\alpha}

\newcommand{\pto}{\stackrel{P}{\longrightarrow}} % сходимость по веpоятности

\newcommand{\eqd}{\stackrel{\mathrm{d}}{=}} % равенство по pаспpеделению
\newcommand{\eqdelta}{\stackrel{\triangle}{=}} % равенство по pаспpеделению

\def\be#1{\begin{equation}\label{#1}}
\def\ee{\end{equation}}
\def\re#1{(\ref{#1})}

\def\bn{\begin{enumerate}}
\def\en{\end{enumerate}}
\def\bi{\begin{itemize}}
\def\ei{\end{itemize}}
%\def\i{\item}

%\newcommand{\kp}{\kappa}
%\def\Q{{\cal Q}} \def\H{{\cal H}}
%\newcommand{\bet}{\beta_{2+\delta}}


%\newtheorem{definition}{Определение}
%\renewcommand{\thedefinition}{\arabic{definition}.}
%END NEW COMMANDS

%\renewcommand{\baselinestretch}{1.2}

%\pagestyle{myheadings}

\setlength{\textwidth}{167mm}      % 122mm
\setlength{\textheight}{658pt}
%\setlength{\textheight}{635.6pt}
\setlength{\columnsep}{4.5mm}

\setcounter{secnumdepth}{4}

%\addtolength{\headheight}{2pt}
%\addtolength{\headsep}{-2mm}

\addtolength{\topmargin}{-7mm}  % for printing


%\hoffset=-30mm  % From Yap
\hoffset=-23mm  % From Acrobat

%\voffset=0mm % From Yap
\voffset=-5mm   % From Acrobat

%\addtolength{\evensidemargin}{-2.5mm} % for printing
%\addtolength{\oddsidemargin}{2.5mm}  % for printing

\addtolength{\evensidemargin}{-12mm} % for printing
\addtolength{\oddsidemargin}{8mm}  % for printing

%\renewcommand{\thefootnote}{\fnsymbol{footnote}}
%\renewcommand{\thefootnote}{\arabic{footnote}}
\renewcommand{\figurename}{\protect\bf Рис.}
\renewcommand{\tablename}{\protect\bf Таблица}

\newcommand{\Caption}[1]{\caption{\protect\small %\baselineskip=2.5ex
#1}}

\renewcommand{\thefigure}{\arabic{figure}}
\renewcommand{\thetable}{\arabic{table}}
\renewcommand{\theequation}{\arabic{equation}}
\renewcommand{\thesection}{\arabic{section}}

\renewcommand{\contentsname}{СОДЕРЖАНИЕ}
\newcommand{\fr}[2]{\displaystyle\frac{\displaystyle #1\mathstrut}{\displaystyle #2\mathstrut}}

%\renewcommand{\thefootnote}{\fnsymbol{footnote}}
%\newcommand{\g}{\mbox{\textit{g}}}

%\newcommand{\Caption}[1]{\caption{\protect\small\baselineskip=2ex #1}}
\newcounter{razdel}
\setcounter{razdel}{0}

\def\god{2022}
\def\tom{16}
\def\vyp{1}


\newcommand{\titel}[4]{%
\

\vspace*{5pt}

\ifodd\therazdel {\raggedright\noindent\Large\textrm\textbf
 \lineskip .75em
  \baselineskip=3.2ex #1 \par}
\vskip 1em {\noindent\large\textrm\textbf #2 \par}
\addcontentsline{toc}{subsection}{{\textrm\textbf #1}\protect\newline #2}
\def\rightheadline{\underline{\noindent\hbox to \textwidth{\hfill\small\textrm{#4}
%\hfill \large\bf\thepage
}}}
\def\leftheadline{\underline{\noindent\parbox{\textwidth}{
%\raggedleft\large\bf\thepage \hfill
\small\textit{#3}\hfill}}}
\def\leftfootline{\small{\textbf{\thepage}
\hfill ИНФОРМАТИКА И ЕЁ ПРИМЕНЕНИЯ\ \ \ том~\tom\ \ \ выпуск~\vyp\ \ \ \god}
}%
 \def\rightfootline{\small{ИНФОРМАТИКА И ЕЁ ПРИМЕНЕНИЯ\ \ \ том~\tom\ \ \ выпуск~\vyp\ \ \ \god
\hfill \textbf{\thepage}}}
\vskip 2em \setcounter{figure}{0}
\setcounter{table}{0}
\setcounter{equation}{0}
\setcounter{section}{0}
\setcounter{subsection}{0}
\setcounter{subsubsection}{0}
\setcounter{footnote}{0}
\setcounter{razdel}{0}
%\end{flushleft}
\else {
 \raggedright\noindent\Large\textrm\textbf
 \lineskip .75em
\baselineskip=3.2ex #1 \par} \vskip 1em
%\begin{flushleft}
{\noindent\large\textrm\textbf #2 \par}
\addcontentsline{toc}{subsection}{{\textrm\textbf #1}\protect\newline #2}
\def\rightheadline{\underline{\noindent\hbox to \textwidth{\hfill\small\textrm{#4}
%\hfill \large\bf\thepage
}}}
\def\leftheadline{\underline{\noindent\parbox{\textwidth}{%\raggedleft\large\bf\thepage \hfill
\small\textit{#3}\hfill}}}
\def\leftfootline{\small{\textbf{\thepage}
\hfill ИНФОРМАТИКА И ЕЁ ПРИМЕНЕНИЯ\ \ \ том~\tom\ \ \ выпуск~\vyp\ \ \ \god}
}%
 \def\rightfootline{\small{ИНФОРМАТИКА И ЕЁ ПРИМЕНЕНИЯ\ \ \ том~16\ \ \ выпуск~\vyp\ \ \ 2022
\hfill \textbf{\thepage}}} \vskip 2em \setcounter{figure}{0}
\setcounter{table}{0} \setcounter{equation}{0} \setcounter{section}{0}
\setcounter{subsection}{0} \setcounter{subsubsection}{0}
\setcounter{footnote}{0}
%\end{flushleft}
\fi}

\newcommand{\titelr}[2]{%
\

\vspace*{5pt}

\ifodd\therazdel {\raggedright\noindent%\Large\textrm\textbf
 \lineskip .75em
  \baselineskip=3.2ex #1 \par}
\vskip 1em {\noindent\normalsize\textrm\textbf #2 \par}
\else {
 \raggedright\noindent\Large\textrm\textbf
 \lineskip .75em
\baselineskip=3.2ex #1 \par} \vskip 1em
%\begin{flushleft}
{\noindent\large\textrm\textbf #2 \par
%\noindent\normalsize\textrm\textbf #2 \par
} \fi}

\newcommand{\titele}[5]{%
\

%\vspace*{5pt}

\ifodd\therazdel {\raggedright\noindent\large
\textrm\textbf
 \lineskip .75em
%  \baselineskip=3.2ex
#1 \par}
\vskip .5em {\noindent\large\textrm\textbf #2 \par}
\vskip .5em
 {\noindent\textrm #3 \par}
\addcontentsline{toc}{subsection}{{\textrm\textbf #1}\protect\newline #2}
\def\rightheadline{\underline{\noindent\hbox to \textwidth{\hfill\small\textrm{#4}
%\hfill \large\bf\thepage
}}}
\def\leftheadline{\underline{\noindent\parbox{\textwidth}{
%\raggedleft\large\bf\thepage \hfill
\small\textrm{#5}\hfill}}}
\def\leftfootline{\small{\textbf{\thepage}
\hfill ИНФОРМАТИКА И ЕЁ ПРИМЕНЕНИЯ\ \ \ том~16\ \ \ выпуск~1\ \ \ 2022}
}%
 \def\rightfootline{\small{ИНФОРМАТИКА И ЕЁ ПРИМЕНЕНИЯ\ \ \ том~16\ \ \ выпуск~1\ \ \ 2022
\hfill \textbf{\thepage}}} \vskip 1em \setcounter{figure}{0}
\setcounter{table}{0} \setcounter{equation}{0} \setcounter{section}{0}
\setcounter{subsection}{0} \setcounter{subsubsection}{0}
\setcounter{footnote}{0} \setcounter{razdel}{0}
%\end{flushleft}
\else {
 \raggedright\noindent\large
 \textrm\textbf
 \lineskip .75em
%\baselineskip=3.2ex
#1 \par} \vskip .5em
%\begin{flushleft}
{\noindent\large\textrm\textbf #2 \par} \vskip .5em
 {\noindent\textrm #3 \par}
\addcontentsline{toc}{subsection}{{\textrm\textbf #1}\protect\newline #2}
\def\rightheadline{\underline{\noindent\hbox to \textwidth{\hfill\small\textrm{#4}
%\hfill \large\bf\thepage
}}}
\def\leftheadline{\underline{\noindent\parbox{\textwidth}{%\raggedleft\large\bf\thepage \hfill
\small\textrm{#5}\hfill}}}
\def\leftfootline{\small{\textbf{\thepage}
\hfill ИНФОРМАТИКА И ЕЁ ПРИМЕНЕНИЯ\ \ \ том~16\ \ \ выпуск~1\ \ \ 2022}
}%
 \def\rightfootline{\small{ИНФОРМАТИКА И ЕЁ ПРИМЕНЕНИЯ\ \ \ том~16\ \ \ выпуск~1\ \ \ 2022
\hfill \textbf{\thepage}}} \vskip 1em \setcounter{figure}{0}
\setcounter{table}{0} \setcounter{equation}{0} \setcounter{section}{0}
\setcounter{subsection}{0} \setcounter{subsubsection}{0}
\setcounter{footnote}{0}
%\end{flushleft}
\fi}

\def\Abst#1{
\begin{center}\small\nwt
\parbox{150mm}{%\baselineskip=2.5ex
\textbf{Аннотация:}\ \
%\hspace*{\parindent}
#1}
\end{center}}
\def\Abste#1{
\begin{center}\small\nwt
\parbox{150mm}{%\baselineskip=2.5ex
\textbf{Abstract:}\ \
%\hspace*{\parindent}
#1}
\end{center}}

\def\DOI#1{
\begin{center}\small\nwt
\parbox{150mm}{%\baselineskip=2.5ex
\textbf{DOI:}\ \
%\hspace*{\parindent}
#1}
\end{center}}

\def\Abstend#1{
\begin{center}\small\nwt
\parbox{150mm}{%\baselineskip=2.5ex
%\hspace*{\parindent}
#1}
\end{center}}


\def\KW#1{
\begin{center}\small\nwt
\parbox{150mm}{%\baselineskip=2.5ex
\textbf{Ключевые слова:}\ \ #1}
\end{center}}

\def\KWE#1{
\begin{center}\small\nwt
\parbox{150mm}{%\baselineskip=2.5ex
\textbf{Keywords:}\ \ #1}
\end{center}}


\def\KWN#1{
%\begin{center}
%\small
%\parbox{150mm}\end{center}
}

\newcommand{\Avtors}[1]{%\smallskip
%\vspace*{.5pt}
\hangindent=23pt\noindent
%\nwt
{\bfseries#1}\
}


\renewcommand{\thesubsection}{\thesection.\arabic{subsection}\hspace*{-5pt}}
\renewcommand{\thesubsubsection}{\thesubsection\hspace*{5pt}.\arabic{subsubsection}\hspace*{-3pt}}

\newcommand{\Ack}{\section*{\protect\rmfamily Acknowledgments}\noindent}
\newcommand{\Contr}{\section*{\protect\rmfamily Contributors}\noindent}
\newcommand{\Contrl}{\section*{\protect\rmfamily Contributor}\noindent}

\makeindex


\begin{document}
\Rus

\nwt
%\ptb


%\renewcommand{\contentsname}{\protect\Large\bf Содержание}

\setcounter{tocdepth}{2}

%\tableofcontents

\renewcommand{\bibname}{\protect\rmfamily Литература}
  \def\Au#1{{\it #1}}
    \def\Aue#1{{#1}}

%\newcommand{\No}{№}
  \newcommand{\tg}{\,\mathrm{tg}\,}
    \newcommand{\ctg}{\,\mathrm{ctg}\,}
  \newcommand{\arctg}{\,\mathrm{arctg}\,}

\def\forallb{\mathop{\forall}}
\def\cupb{\mathop{\cup}}
\def\existsb{\mathop{\exists}}


\newpage
\addtocounter{razdel}{1}
%\def\razd{РЕГУЛИРУЕМЫЙ ЭЛЕКТРОПРИВОД ДЛЯ ЭЛЕКТРОЭНЕРГЕТИКИ}


\setcounter{page}{2}

%   { %\Large  
   { %\baselineskip=16.6pt
   
   \vspace*{-48pt}
   \begin{center}\LARGE
   \textit{Предисловие}
   \end{center}
   
   %\vspace*{2.5mm}
   
   \vspace*{25mm}
   
   \thispagestyle{empty}
   
   { %\small 

    
Вниманию читателей журнала <<Информатика и её применения>> предлагается 
очередной тематический выпуск <<Вероятностно-статистические методы и 
задачи информатики и информационных технологий>>. Предыдущие тематические 
выпуски журнала по данному направлению вышли в 2008~г.\ (т.~2, вып.~2), 
в 2009~г.\ (т.~3, вып.~3) и в 2010~г.\ (т.~4, вып.~2). 

Статьи, собранные в данном журнале, посвящены разработке новых вероятностно-статистических 
методов, ориентированных на применение к решению конкретных задач информатики и информационных 
технологий, а также~--- в ряде случаев~--- и других прикладных задач. Проблематика, охватываемая 
публикуемыми работами, развивается в рамках научного сотрудничества между Институтом проблем 
информатики Российской академии наук (ИПИ РАН) и Факультетом вычислительной математики и 
кибернетики Московского государственного университета им.\ М.\,В.~Ломоносова в ходе работ 
над совместными научными проектами (в том числе в рамках функционирования 
Научно-образовательного центра <<Вероятностно-статистические методы анализа рисков>>). 
Многие из авторов статей, включенных в данный номер журнала, являются активными участниками 
традиционного международного семинара по проблемам устойчивости стохастических моделей, 
руководимого В.\,М.~Золотаревым и В.\,Ю.~Королевым; регулярные сессии этого семинара 
проводятся под эгидой МГУ и ИПИ РАН (в 2011~г.\ указанный семинар проводится в октябре 
в Калининградской области РФ). 

Наряду с представителями ИПИ РАН и МГУ в число авторов данного выпуска журнала входят 
ученые из Научно-исследовательского института системных исследований РАН, Института 
проблем технологии микроэлектроники и особочистых материалов РАН, Института 
прикладных математических исследований Карельского НЦ РАН, Московского 
авиационного института, Вологодского государственного педагогического университета, 
НИИММ им.\ Н.\,Г.~Чеботарева, Казанского государственного университета, Дебреценского 
университета (Венгрия).

Несколько статей выпуска посвящено разработке и применению стохастических методов и 
информационных технологий для решения различных прикладных задач. В~работе В.\,Г.~Ушакова 
и О.\,В.~Шестакова рассмотрена задача определения вероятностных характеристик случайных 
функций по распределениям интегральных преобразований, возникающих в задачах эмиссионной 
томографии. В~статье Д.\,О.~Яковенко и М.\,А.~Целищева рассмотрены некоторые вопросы 
математической теории риска и предложен новый подход к диверсификации инвестиционных 
портфелей. Работа И.\,А.~Кудрявцевой и А.\,В.~Пантелеева посвящена построению и 
исследованию математической модели, описывающей динамику сильноионизованной плазмы. 
В~статье П.\,П.~Кольцова изучается качество работы ряда алгоритмов сегментации изображений. 
Статья А.\,Н.~Чупрунова и И.~Фазекаша посвящена вероятностному анализу числа без\-оши\-бочных 
блоков при помехоустойчивом кодировании; получены усиленные законы больших чисел для указанных 
величин.

В данном выпуске традиционно присутствует тематика, весьма активно разрабатываемая в течение 
многих лет специалистами ИПИ РАН и МГУ,~--- методы моделирования и управления для 
информационно-телекоммуникационных и вычислительных систем, в частности методы 
теории массового обслуживания. В~статье А.\,И.~Зейфмана с соавторами рассматриваются 
модели обслуживания, описываемые марковскими цепями с непрерывным временем в случае 
наличия катастроф. В~работе М.\,М.~Лери и И.\,А.~Чеплюковой рассматриваются случайные 
графы Интернет-типа, т.\,е.\ графы, степени вершин которых имеют степенные распределения; 
такие задачи находят применение при исследовании глобальных сетей передачи данных. 
Работа Р.\,В.~Разумчика посвящена исследованию систем массового обслуживания специального 
вида~--- с отрицательными заявками и хранением вытесненных заявок.

Ряд статей посвящен развитию перспективных теоретических 
вероятностно-статистических методов, которые находят широкое применение в различных 
задачах информатики и информационных технологий. В~работе В.\,Е.~Бенинга, А.\,К.~Горшенина 
и В.\,Ю.~Королева рассмотрена задача статистической проверки гипотез о числе компонент 
смеси вероятностных распределений, приводится конструкция асимптотически наиболее мощного 
критерия. Результаты этой работы найдут применение в ряде прикладных задач, использующих 
математическую модель смеси вероятностных распределений (в информатике, моделировании 
финансовых рынков, физике турбулентной плазмы и~т.\,д.). В~статье В.\,Ю.~Королева, 
И.\,Г.~Шевцовой и С.\,Я.~Шоргина строится новая, улучшенная оценка точности нормальной 
аппроксимации для пуассоновских случайных сумм; как известно, указанные случайные суммы 
широко используются в качестве моделей многих реальных объектов, в том числе в информатике, 
физике и других прикладных областях. Работа В.\,Г.~Ушакова и Н.\,Г.~Ушакова посвящена 
исследованию ядерной оценки плотности распределения; эти результаты могут применяться, 
в част\-ности, при анализе трафика в телекоммуникационных системах. Серьезные приложения 
в статистике могут получить результаты работы О.\,В.~Шестакова, в которой доказаны оценки 
скорости сходимости распределения выборочного абсолютного медианного отклонения к нормальному 
закону. 

\smallskip

Редакционная коллегия журнала выражает надежду, что данный тематический  выпуск 
будет интересен специалистам в области теории вероятностей и математической статистики 
и их применения к решению задач информатики и информационных технологий.
     
     %\vfill 
     \vspace*{20mm}
     \noindent
     Заместитель главного редактора журнала <<Информатика и её 
применения>>,\\
     директор ИПИ РАН, академик  \hfill
     \textit{И.\,А.~Соколов}\\
     
     \noindent
     Редактор-составитель тематического выпуска,\\
     профессор кафедры математической статистики факультета\\
      вычислительной математики и кибернетики МГУ им.\ М.\,В.~Ломоносова,\\
     ведущий научный сотрудник ИПИ РАН,\\ 
доктор физико-математических наук \hfill
      \textit{В.\,Ю.~Королев}
     
     } }
     }


   
\def\stat{kovalev}

\def\tit{МЕТОДЫ ТЕОРИИ КАТЕГОРИЙ В~МОДЕЛЬНО-ОРИЕНТИРОВАННОЙ СИСТЕМНОЙ 
ИНЖЕНЕРИИ}

\def\titkol{Методы теории категорий в~модельно-ориентированной системной 
инженерии}

\def\aut{С.\,П.~Ковалёв$^1$}

\def\autkol{С.\,П.~Ковалёв}

\titel{\tit}{\aut}{\autkol}{\titkol}

\index{Ковалёв С.\,П.}
\index{Kovalyov S.\,P.}


%{\renewcommand{\thefootnote}{\fnsymbol{footnote}} \footnotetext[1]
%{Исследование выполнено при финансовой поддержке Российского научного фонда (проект 16-11-10227).}}


\renewcommand{\thefootnote}{\arabic{footnote}}
\footnotetext[1]{Институт проблем управления им.\ В.\,А.~Трапезникова 
Российской академии наук,  \mbox{kovalyov@nm.ru}}

%\vspace*{-18pt}

\Abst{Предложен математический аппарат на базе теории категорий, который позволяет 
формально описывать и~строго исследовать процедуры применения моделей в~инженерной 
деятельности, составляющие сущность мо\-дель\-но-ори\-ен\-ти\-ро\-ван\-ной системной 
инженерии (Model-Based Systems Engineering, MBSE). В~основе аппарата лежит 
математическое представление сборочных чертежей (мегамоделей сис\-тем) диаграммами 
в~категориях, объектами которых служат модели, а~морфизмы представляют действия по 
сборке моделей сис\-тем из моделей компонентов. Адекватность аппарата обоснована исходя 
из требований стандартов, регламентирующих описание структуры систем, в~том числе 
IEC~81346. Предложены и~исследованы тео\-ре\-ти\-ко-ка\-те\-гор\-ные методы решения ряда 
практических задач сборки систем. Приведены примеры решения таких задач в~категориях, 
представляющих две ключевые области применения MBSE: гео\-мет\-ри\-че\-ское моделирование 
изделий сложной формы и~дис\-крет\-но-со\-бы\-тий\-ное имитационное моделирование 
поведения технических систем.}

\KW{модельно-ориентированная системная инженерия; мегамодель; теория категорий; 
копредел}



\DOI{10.14357/19922264170305} 


\vspace*{6pt}

\vskip 10pt plus 9pt minus 6pt

\thispagestyle{headings}

\begin{multicols}{2}

\label{st\stat}

\section{Введение}

   Модельно-ориентированная системная инженерия состоит в~формализованном применении моделирования в~
поддержке жизненного цикла сис\-тем, включая сбор требований, 
проектирование, проверку и~приемку, другие стадии~[1]. Модели, 
разрабатываемые в~ходе процедур MBSE, пригодны к~автоматической 
обработке на компьютерах. Это позволяет сначала задавать, верифицировать 
и~оптимизировать проектные решения на моделях <<в циф\-ре~и только потом 
воплощать <<в железе>>, снижая затраты на организацию жизненного цикла 
изделий и~сокращая сроки выполнения работ~[2].
   
   И все же внедрение технологий MBSE в~инженерную деятельность 
происходит медленно. Это связано во многом с~нехваткой единой 
концептуальной базы инженерного моделирования: предлагается много 
частных языков и~технологий, слабо совместимых друг с~другом и~плохо 
приспособленных для совместной разработки моделей большими 
мультидисциплинарными коллективами~[3]. Тем самым затрудняется переход 
от набора электронных чертежей к~полноценному электронно-цифровому 
макету (digital mock-up) промышленного изделия.
   
   Естественный, хотя и~<<трудный>>, подход к~получению результатов 
общего характера, унифи\-ци\-ру\-ющих разнородные технологии, состоит в~том, 
чтобы как можно более строго формализовать процедуры моделирования. 
Формализация позволит совершенствовать процедуры MBSE и~передавать их 
на исполнение компьютеру без пробелов и~искажений. Самый высокий уровень 
строгости достигается при привлечении математического аппарата, поскольку 
математика позволяет надежно доказывать или опровергать утверждения, 
ха\-рак\-те\-ри\-зу\-ющие корректность и~эффективность процедур.
   
   В настоящей работе предложен аппарат, основанный на математическом 
представлении сборочных чертежей (<<мегамоделей>> систем) 
ориенти-\linebreak рованными графами (диаграммами). Узлы такого\linebreak графа помечаются 
обозначениями моделей час\-тей, а~реб\-ра помечаются обозначениями действий\linebreak 
(activities), посредством которых части собираются в~систему. Представление 
структуры систем графами регламентируется, в~частности, стандартом 
IEC~81346~[4]. Естественным источником математических методов 
конструирования и~анализа мегамоделей служит теория категорий (см., 
например,~[5, 6]). Модели рассматриваются как объекты подходящих 
категорий, а~действия формально описываются морфизмами. Строятся 
и~исследу-\linebreak ются тео\-ре\-ти\-ко-ка\-те\-гор\-ные конструкции, опи\-сы\-ва\-ющие процедуры 
MBSE на абстрактном кон-\linebreak цептуальном уровне. Определенный опыт такого\linebreak 
исследования был накоплен в~инженерии программного обеспечения~[7] 
и~теперь может быть обобщен для системной инженерии в~целом. Например, 
сборке системы согласно некоторой мегамодели отвечает построение 
копредела диаграммы~--- универсальной конструкции~\cite{5-kov}.
   
   Статья построена следующим образом. В~разд.~2 приведен обзор 
принципов описания структуры сис\-тем согласно стандарту IEC~81346. 
Раздел~3 посвящен практическим проб\-ле\-мам мегамоделирования и~сборке 
сис\-тем. В~разд.~4 вводятся конструкции тео\-рии категорий, позволяющие 
формально решать задачи мегамоделирования. В~заключении приводятся 
выводы и~намечаются направления дальнейших исследований.

\section{Структура систем и~стандарт~IEC~81346}

   Важной проблемой MBSE, отмеченной во введении, является слабая 
совместимость языков и~инструмен\-тов моделирования от разных поставщиков. 
Основным подходом к~достижению совместимости является стандартизация~--- 
принятие обязывающих документов, устанавливающих требования и~принципы 
взаимозаменяемости инструментов. Многие стандарты определяют конкретные 
форматы машиночитаемой записи моделей, нейтральные относительно 
разработчиков инструментов MBSE. Примером служит формат описания 
твердотельных геометрических моделей STEP, стандартизованный семейством 
ISO~10303. Однако для формализации MBSE в~целом интерес представляют 
в~первую очередь стандарты более общего плана, унифицирующие принципы 
и~методы применения моделей в~жизненном цикле систем независимо от 
способа записи моделей. С~этой точки зрения внимания заслуживает 
международный стандарт IEC 81346-1:2009 <<Промышленные системы, 
установки и~обору\-до\-ва\-ние~--- принципы структурирования и~ссылочные 
обозначения~--- часть~1: основные правила>> (<<Industrial Systems, 
Installations and Equipment and Industrial Products~--- Structuring Principles and 
Reference Designations~--- Part~1: Basic Rules>>)~\cite{4-kov}. Стандарт не 
принят в~России, однако ряду его положений в~области структуры систем 
соответствует российский ГОСТ~2.053-2013 <<ЕСКД. Электронная структура 
изделия. Общие положения>>.
   
   В стандарте IEC~81346 рассматривается ряд вопросов моделирования 
структуры систем и~идентификации отдельных единиц в~составе систем. 
Системная единица названа в~стандарте объектом, причем принципиально не 
проводится различие между объектами реального мира, составляющими 
реально существующие системы, и~объектами мыслительной деятельности~--- 
моделями единиц, составляющими модели систем. Таким образом, стандарт 
выходит за рамки MBSE и~рассматривает ряд вопросов системной инженерии 
вообще. Иерар\-хи\-че\-ская структура системы (холархия~\cite{3-kov}) 
изображается деревом, узлы которого помечены обозначениями объектов. 
Важным достижением стандарта является выявление того факта, что одна и~та 
же система задается не одной, а несколькими в~общем случае различными 
иерархическими структурами, возникающими в~результате декомпозиции 
согласно различным принципам (аспектам). В~их числе:
   \begin{itemize}
\item функциональная (function-oriented) структура, отвечающая разделению 
системных единиц по выполняемым ими функциям в~составе сис\-темы;
\item продуктовая (product-oriented), или модульная, структура, отражающая 
сборочную (технологическую) конфигурацию сис\-темы;
\item структура размещения (location-oriented), в~соответствии с~которой 
единицы располагаются в~физическом пространстве.
\end{itemize}

   Ясно, что один и~тот же объект может входить в~несколько структур и~при 
этом находиться на различных уровнях. В~то же время в~некоторых аспектах 
объект может никак не проявлять себя и~вследствие этого отсутствовать 
в~соответствующих структурах. Полное идентифицирующее ссылочное 
обозначение объекта (reference designation) конструируется путем 
последовательного перечисления всех объектов, находящихся на пути от корня 
дерева рассматриваемой структуры до дан\-ного объекта включительно. 
Наименование каж\-до\-го объекта в~этом перечислении составляется из 
символьного обозначения аспекта, буквенного обозначения класса (типа), 
к~которому относится  объект, и~порядкового номера объекта среди 
экземпляров своего класса. Таким путем обеспечивается\linebreak  уникальность 
наименования любой единицы\linebreak
 в~пределах системы. Например, функциональная 
структура обозначается символом <<=>>, а~функциональный класс 
переключателей потоков ресурсов обозначается буквами QA, так что первая по 
порядку единица, выполняющая функцию переключения, называется =QA1, 
а~ее полное ссылочное обозначение может выглядеть как =WP1=WC1=QA1. 
Если объект присутствует в~нескольких структурах, то он может иметь 
несколько ссылочных обозначений, как показано на рис.~1~\cite{4-kov}.

\begin{figure*} %fig1
    \vspace*{1pt}
\begin{center}
\mbox{%
\epsfxsize=165mm
\epsfbox{kov-1.eps}
}
\end{center}
\vspace*{-9pt}
\Caption{Пример ссылочных обозначений структурных единиц системы}
\vspace*{9pt}
\end{figure*}

   С~точки зрения практики системной инженерии большой интерес 
представляет описание эволюции структурного представления системы по ходу 
жизненного цикла, приведенное в~приложении~B к~стандарту IEC~81346. 
<<Строительный материал>> для структур имеет вид (виртуального) 
справочника или каталога объектов, из которого выбираются объекты для 
включения в~структуру. 

В~начале жизненного цикла системы на основе 
исходных требований к~ней конструктор строит ее функциональную структуру. 
Затем определяется пространственное положение функциональных объектов, 
в~результате чего создается структура размещения. На следующей стадии 
формируются закупочные спецификации, образующие продуктовую структуру. 
В~ходе последующих стадий жизненного цикла эти структуры могут 
трансформироваться. На каждой стадии могут происходить замена, слияние 
и~расщепление объектов. Таким образом, объекты разных структур системы 
связаны отношением вида <<многие ко многим>>, вдоль которого 
прослеживаются (трассируются) исходные требования.
   
   В то же время стандарт не предусматривает указа\-ние способов, какими 
объекты собраны в~сис\-те\-мы. Поэтому структуру сис\-те\-мы можно рас\-смат\-ри\-вать 
как эскизный проект, в~котором отражены лишь факты вхождения системных 
единиц более низкого уровня иерархии в~единицы более высокого уровня. 


Проект такого рода поступает на вход технологу, который определяет 
конкретные операции сборки каждой единицы каждого уровня иерархии. При 
необходимости технолог вносит изменения в~конструкцию объектов (такие как 
нарезка резьбы) и~добавляет связующие интерфейсные объекты (такие как 
клей, трансформатор и~др.). В~результате для каждого составного объекта 
формируется сборочный чертеж, на котором указаны все со\-став\-ля\-ющие 
объекты и~действия по их соединению в~целях получения сис\-те\-мы. 
Технологическая проработка требуется на всех стадиях жизненного цикла, на 
которых формируется либо изменяется ка\-кая-ли\-бо из структур системы.

%\vspace*{-6pt}

\section{Мегамоделирование и~сборка~систем}

   В MBSE объекты, образующие 
структуры\linebreak
 сис\-тем, описываются формализованными ком\-пьютерными моделями 
различных видов: геометрическими фигурами и~телами, численными 
аппроксимациями дифференциальных уравнений, оснащенными графами и~
т.\,д. При этом, как свидетель\-ст\-ву\-ют стандарты типа IEC~81346, для анализа 
структуры систем и~организации сборки необходимо знать не столько 
внутреннюю структуру моделей, сколько ассортимент их возможностей 
соединяться с~другими моделями в~целях формирования моделей составных 
объектов. Иными словами, модели рассматриваются как <<черные ящики>> 
с~известным поведением по отношению к~другим моделям. Каталог объектов, 
упоминавшийся в~предыду\-щем разделе, в~условиях применения \mbox{MBSE} 
составляется из моделей и~описаний действий по их соединению.
   
   Структуры систем и~сборочные чертежи представляют собой частные 
случаи мегамоделей (mega\-mod\-el)~--- моделей, состоящих из моделей и~связей 
между ними~\cite{8-kov}. Мегамодель, в~которой связи описывают соединение 
моделей, образующих некоторую сис\-те\-му, называется конфигурацией этой 
сис\-те\-мы~\cite{5-kov}. Существуют и~другие виды мегамоделей, 
предназначенные для описания других процедур \mbox{MBSE}, таких как 
формирование модели согласно заданной метамодели  
(instantiating)~\cite{9-kov}. Но в~настоящей работе сосредоточимся на 
конфигурациях и~сборке систем.
   
   Например, в~моделировании механических сис\-тем, состоящих из твердых 
тел, моделями деталей и~сборочных единиц служат геометрические тела, 
которые могут быть представлены для компьютерной обработки различными 
способами: конструктивным, воксельным, граничным~\cite{10-kov}. Объекты, 
составляющие механические системы, т.\,е.\ представления экземпляров тел, 
получаются из моделей путем аффинных изометрий и~растяжений. Так, из 
набора цилиндров разных размеров составляется модель штанги (спортивного 
снаряда). В~функциональной структуре штанги по IEC~81346 цилиндры 
представлены разными объектами, поскольку они выполняют разные функции, 
хотя порождаются одной и~той же геометрической моделью. Соответственно, 
в~каталоге моделей содержится тело в~форме цилиндра, допускающее 
несколько разных действий по включению в~состав штанги.
   
   В качестве еще одного примера рассмотрим дис\-крет\-но-со\-бы\-тий\-ное 
имитационное моделирование, поддержка которого относится к~числу 
важнейших достижений MBSE~\cite{1-kov}. Здесь модель имеет вид 
сценария~--- фрагмента предполагаемой истории поведения моделируемой 
системы, пред\-став\-лен\-но\-го потоком дискретных событий различных видов. 
Некоторые события могут вызывать либо запрещать возникновение других 
событий. Описания действий по сборке сценариев поведения систем отражают 
вклад сценариев поведения составляющих. Так, сценарий работы цеха 
составляется из сценариев работы станков, связанных друг с~другом согласно 
маршрутным картам~\cite{11-kov}.
   
   Сформулируем задачу мегамоделирования сборки систем в~общем виде 
следующим образом. По мегамодели, представляющей конфигура\-цию 
некоторой системы, требуется сконструировать модель системы как целого 
и~рассчитать для нее моделируемые параметры, в~том числе эмерджентные~--- 
не присущие никакой из со\-став\-ля\-ющих единиц в~отдельности. Принцип 
конструирования модели системы легко усмотреть из организации 
структур-\linebreak\vspace*{-12pt}

\columnbreak

 { \begin{center}  %fig1
 \vspace*{1pt}
\mbox{%
\epsfxsize=57.246mm
\epsfbox{kov-2.eps}
}


\vspace*{12pt}


\noindent
{{\figurename~2}\ \ \small{Схема склеивания}}
\end{center}
}

\vspace*{18pt}

\addtocounter{figure}{1}

\noindent
ного представления: система должна находиться на иерархическом 
уровне, располагающемся непосредственно над уровнем со\-став\-ля\-ющих ее 
объектов. Иными словами, модель системы должна включать в~себя модели 
всех составляющих с~учетом их конфигурационных связей и~в~то же время 
включаться в~любые модели, включающие в~себя модели всех составляющих 
конфигурации.
   
   Поясним этот принцип на простом примере. Предположим, что нужно 
объединить в~систему два объекта~$P$ и~$S$ и~что технолог решил сделать это 
с~по\-мощью клея~--- третьего объекта~$G$, который может быть соединен 
и~с~$P$, и~с~$S$. Действие клея описывается конфигурацией следующего 
вида: объекты~$G$ и~$P$ порождают в~результате соединения известный 
промежуточный комплексный объект~$P_G$, содержащий их, а~объекты~$G$ 
и~$S$ порождают объект~$S_G$. Система~$R$, полученная путем 
склеивания~$P$ с~$S$ при помощи~$G$, отбирается среди объектов, 
содержащих~$P_G$ и~$S_G$, по следующему структурному критерию: 
объект~$R$ должен содержаться в~любом объекте~$T$, содержащем~$P_G$ 
и~$S_G$. Схематически этот критерий изображен на рис.~2.


   Если объект $R$, удовлетворяющий указанному структурному критерию, 
существует, то он действительно отвечает системе, которая собрана из~$S$ 
и~$P$ путем склеивания посредством~$G$ (и~не содержит ничего 
<<лишнего>>). Более того, легко видеть, что такой объект~$R$ определяется, 
по существу, однозначно в~том смысле, что любые два объекта~$R$ 
и~$R^\prime$, удовлетворяющие структурному критерию, содержатся друг 
в~друге. Если же нужного объекта~$R$ не существует, то делается вывод, что 
технолог ошибся: клей~$G$ не способен соединить объекты~$P$ и~$S$.
   
   В структурное представление, выполненное по стандарту IEC~81346 либо по 
ГОСТу 2.053-2013, входят только объекты~$P$, $S$ и~$R$ и~две композитные 
стрелки: $P\hm\to R$, проходящая через~$P_G$, и~$S\hm\to R$, проходящая 
через~$S_G$ (так что мегамодель склеивания~--- это часть схемы, ограниченная 
треугольником~$PSR$). Кроме того, стрелки на схеме склеивания, в~отличие от 
структуры, представляют не просто факты включения объектов друг в~друга, 
а~конкретные действия по их соединению. При этом соблюдается следующее 
естественное условие структурной корректности: если из одного объекта 
можно прийти в~другой разными путями по схеме, то эти пути задают одно и~то 
же композитное действие. Например, клей~$G$ включается в~состав 
системы~$R$ единственным способом, несмотря на наличие двух путей $G 
\hm\to  P_G \hm\to R$ и~$G \hm\to S_G \hm\to R$: в~действительности не имеет 
значения, через какой промежуточный объект <<прослеживается>> включение 
клея в~систему. Таким образом, мегамодель сборки содержит больше 
информации, чем иерархическая структура системы.
   
   Если модели содержат значения тех или иных параметров, а описание 
действий по их соединению позволяет выявить правила преобразования 
значений, то по мегамодели сборки можно вы\-чис\-лить значения параметров для 
системы. Известны примеры вычислений такого рода в~области разработки 
новых композиционных материалов~\cite{12-kov}. Осредненные 
(эффективные) физические характеристики композитов, такие как модуль Юнга и~коэффициент Пуассона, сложным образом зависят от характеристик 
компонентов и~способов изготовления композита из них. При помощи методов 
теории упру\-гости эти зависимости задаются в~форме линеаризованных 
матричных соотношений, которые приписываются к~стрелкам мегамоделей, 
пред\-став\-ля\-ющим включение компонентов в~композиты. Появляется 
возможность рассчитывать на компьютере свойства композитов по базе данных 
компонентов, без проведения дорогостоящих физических экспериментов.
   
   В заключение раздела отметим, что хотя прямой расчет системы по 
конфигурации имеет большое значение, в~MBSE он играет вспомогательную 
роль. Согласно стандарту IEC~81346 и~практикам системной инженерии, 
система обычно проектируется сверху вниз~--- от корня структурной иерархии 
к~составляющим~\cite{13-kov}. Это означает, что технолог в~основном решает 
не прямую, а~обратную задачу: модель системы, которую нужно собрать, 
известна, а~нужно построить (восстановить) конфигурацию, из которой такая 
система может быть получена путем сборки, с~учетом различных ограничений. 
Формальные математические постановки и~методы решения обратных задач 
мегамоделирования представляют собой крупную перспективную тему 
исследований, выходящую за рамки настоящей статьи.

\section{Теория категорий в~мегамоделировании}

   Как указывалось во введении, естественным источни\-ком математических 
методов кон\-стру\-ирова\-ния и~анализа мегамоделей служит теория категорий. 
Категорией называется коллекция абстрактных объектов, попарно связанных 
морфизмами (стрелками). Точное определение занимает буквально несколько 
строк~\cite{14-kov}: категория~$C$ состоит из совокупности 
объектов~$\mathrm{Ob}\,C$ и~совокупности морфизмов~$\mathrm{Mor}\,C$, 
на которых заданы следующие операции:
\begin{enumerate}[(1)]
\item каждому морфизму~$f$ 
сопоставляется два объекта: область $\mathrm{dom}\,f$ и~кообласть 
$\mathrm{codom}\,f$ (соотношения вида $\mathrm{dom}\,f \hm= A$ и~
$\mathrm{codom}\,f \hm= B$ наглядно записываются в~форме стрелки~$f$: 
$A\hm\to B$, а множество всех морфизмов, удовлетворяющих этим 
соотношениям, обозначается через $\mathrm{Mor}(A, B))$;
\item для 
любой пары морфизмов~$f, g$, удовлетворяющей условию 
$\mathrm{codom}\,f\hm = \mathrm{dom}\,g$, определена композиция~--- 
морфизм $g \circ f : \mathrm{dom}\,f \hm\to  \mathrm{codom}\,g$, причем она 
ассоциативна: для любой тройки морфизмов~$f, g, h$, удовлетворяющей 
условиям $\mathrm{codom}\,f \hm= \mathrm{dom}\,g$ и~$\mathrm{codom}\,g 
\hm= \mathrm{dom}\,h$, выполняется соотношение $h \circ (g \circ f) \hm= (h 
\circ g) \circ f$;
\item любой объект~$A$ обладает тождественным 
морфизмом~$1_A : A \to A$ таким, что для любого морфизма~$f : A\hm\to B$ 
выполняется соотношение $f \circ 1_A \hm= 1_B \circ  f \hm= f$.
\end{enumerate}

Классическим 
примером категории служит $\mathbf{Set}$, состоящая из всех множеств и~всех 
их отображений: закон композиции отображений задается стандартной 
подстановкой, а тождественным морфизмом произвольного множества служит 
его тождественное отображение на себя.
   
   Вместе с~категорией вводится понятие функтора~--- отображения категорий, 
сохраняющего структуру. Функтор $\mathrm{fun}\,: C \hm\to D$, действующий из 
категории~$C$ в~$D$,~--- это пара одноименных отображений $\mathrm{fun}\,: 
\mathrm{Ob}\,C \hm\to \mathrm{Ob}\,D$, $\mathrm{fun}\,: \mathrm{Mor}\,C \hm\to 
\mathrm{Mor}\,D$, удовлетворяющая следующим условиям (для произвольных 
$C$-мор\-физ\-мов~$f, g$ и~$C$-объ\-ек\-та~$A$): 
\begin{enumerate}[(1)]
\item $\mathrm{fun}\,(\mathrm{dom}\,f) 
\hm= \mathrm{dom}\,\mathrm{fun}\,(f), \mathrm{fun}\,(\mathrm{codom}\,f)\hm = 
\mathrm{codom}\,\mathrm{fun}\,(f)$;  
\item $\mathrm{fun}\,(g \circ f) = \mathrm{fun}\,(g) \circ \mathrm{fun}\,(f)$, 
если композиция $g \circ f$ определена; 
\item $\mathrm{fun}\,(1_A) \hm= 1_{\mathrm{fun}\,(A)}$.
\end{enumerate}
 Все категории и~все функторы образуют 
(формальную) категорию~$\mathbf{CAT}$. Чтобы исследовать взаимосвязь 
между функторами, вводится следующее понятие: естественным 
преобразованием~$\varepsilon$ функтора $\mathrm{fun}\, : C\hm\to D$ в~$\mathrm{fun}^\prime\, : C 
\hm\to D$ называется любое семейство $D$-мор\-физ\-мов~$\varepsilon_A : 
\mathrm{fun}\,(A) \hm\to \mathrm{fun}^\prime (A)$, $A \hm\in \mathrm{Ob}\,C$, 
такое что для любого 
\mbox{$C$-мор}\-физ\-ма $f : A\hm\to B$ выполняется соотношение $\varepsilon_B \circ 
\mathrm{fun}\,(f) \hm= \mathrm{fun}^\prime(f) \circ \varepsilon_A$:

%\begin{figure*} %рис
\vspace*{1pt}
\begin{center}
\mbox{%
\epsfxsize=54.473mm
\epsfbox{kov-3.eps}
}
\end{center}
%\vspace*{-9pt}
%\end{figure*}

   Эффективность применения теории категорий в~качестве математического 
аппарата \mbox{MBSE} обуслов\-ле\-на тем, что любой каталог моделей представляет 
собой не что иное, как категорию. Действительно, любая цепочка действий по 
соединению моделей порождает композитное действие (процесс) и, кроме того, 
любая модель допускает пустое действие над самой собою, не 
подразумевающее никаких изменений (процедура <<ничегонеделания>>). 
Например, в~твердотельном моделировании механических систем объектами 
категории\linebreak моделей выступают тела~--- подмножества в~$\mathbb{R}^3$, 
которые являются ограниченными, регулярными\linebreak
 (совпадают с~замыканием 
своей внутренности) и~полуаналитическими (допускают представление 
конечными булевыми комбинациями множеств вида $\{(x, y, z) \vert  F_i(x, y, 
z)\hm\leq 0\}$, где~$F_i : \mathbb{R}^3\hm\to \mathbb{R}$ является 
вещественной аналитической функцией для всех~$i$)~\cite{10-kov}. Чтобы 
было возможно задавать процедуры типа склеивания участков поверхности тел, в~категорию геометрических моделей добавляются ограниченные регулярные 
полуаналитические подмножества в~$\mathbb{R}^n$, $0 \hm\leq n \hm\leq 2$, 
при помощи стандартного вложения~$\mathbb{R}^n$ в~$\mathbb{R}^3$. Далее 
выполняется факторизация: отождествляются друг с~другом все множества, 
переходящие друг в~друга под действием аффинных изометрий. Морфизмы 
таких классов эквивалентности, описывающие действия по сборке составных 
механических сис\-тем, порождаются изометрическими вложениями множеств 
и~растяжениями. Получается подкатегория в~\textbf{Set}, которую будем обозначать 
через $\mathbf{MBS}$ (от Multibody Systems).
   
   Для многих известных технологий MBSE формальное описание каталогов 
поддерживаемых моделей приводит к~категориям множеств со структурой~--- 
алгебраических систем, топологических пространств, графов и~т.\,д. 
Морфизмами в~таких категориях служат отображения множеств, со\-вмес\-ти\-мые 
со структурой. На любой такой категории действует канонический функтор 
в~$\mathbf{Set}$, <<забывающий>> структуру. 

В~качестве примера приведем  
дис\-крет\-но-со\-бы\-тий\-ное моделирование, в~котором математической 
моделью сценария служит множество событий, час-\linebreak тич\-но упорядоченное  
при\-чин\-но-след\-ст\-вен\-ны\-ми зависимостями и~размеченное видами 
событий~\cite{15-kov}. Действия по сборке сложных сценариев задаются 
монотонными отображениями, сохраняющими разметку, поскольку ни 
события, ни зависимости, ни метки не могут быть <<потеряны>> при 
соединении сценариев поведения компонентов в~сценарии поведения систем. 
Получается категория~$\mathbf{Pomset}$, состоящая из всех помеченных 
частично упорядоченных множеств и~всех их монотонных отображений, 
сохраняющих разметку. Имеется функтор $\vert \mbox{--} \vert : 
\mathbf{Pomset}\hm\to \mathbf{Set} : S \mapsto \vert S\vert$, <<забывающий>> 
порядок и~разметку.
   
   Зафиксируем произвольную категорию~$C$, представляющую некоторый 
каталог моделей. Как и~для любой алгебраической системы, определена 
конструкция подкатегории в~$C$~--- это пара, состоящая из подкласса 
в~$\mathrm{Ob}\,C$ и~подкласса в~$\mathrm{Mor}\,C$, замкнутых 
относительно унаследованных из~$C$ операций. Подкатегория в~$C$ 
называется полной, если любой \mbox{$C$-мор}\-физм, область и~кообласть которого 
содержатся в~ней, сам содержится в~ней. Например, подкатегориями 
описываются различные аспекты структурного представления систем согласно 
стандарту IEC~81346. Действительно, композиция двух морфизмов, 
представляющих действия по формированию некоторого аспекта структуры, 
также должна входить в~этот аспект, поскольку стандарт предписывает строить 
цепочки для идентификации объектов в~структуре системы. Кроме того, если 
объект присутствует в~аспекте, то его тождественный морфизм формально 
должен быть включен в~этот аспект. В~то же время подкатегории, 
опи\-сы\-ва\-ющие все аспекты, не обязаны образовывать в~совокупности разбиение 
категории~$C$: как показывает рис.~1, возможны как действия, входящие 
в~несколько аспектов одновременно, так и~композитные действия с~переходом 
между структурами, не входящие ни в~один аспект. Требуется лишь, чтобы 
объединение классов объектов всех этих подкатегорий совпадало 
с~$\mathrm{Ob}\,C$, поскольку не имеет смысла вводить модели, не входящие 
ни в~одну структуру.
   
   Категории можно получать из графов: любой ориентированный мультиграф 
порождает категорию, объектами в~которой служат все узлы, а морфизмами~--- 
все пути. Областью и~кообластью морфизма являются соответственно начало 
и~конец пути, композиция морфизмов действует как конкатенация путей, 
а~тождественным морфизмом узла~$a$ является пустой путь из~$a$ в~$a$, не 
содержащий ни одного ребра. Отсюда получается фундаментальное понятие  
$C$-диа\-грам\-мы~--- это функтор вида~$\Delta : X \hm\to C$, где~$X$~--- 
категория, порожденная некоторым графом и~называемая схемой диаграммы. 
Все $C$-диа\-грам\-мы образуют категорию~$\mathbf{D}C$ (ковариантная 
категория <<сверхзапятой>>~\cite{14-kov}), в~которой морфизмом 
диаграммы~$\Delta : X \hm\to C$ в~$\Xi : Y \hm\to C$ служит любая пара 
вида $\langle\gamma, fd\rangle$, состоящая из функтора~$fd : X\hm\to Y$ 
и~естественного преобразования~$\gamma : \Delta\hm\to \Xi \circ fd$; закон 
композиции морфизмов диаграмм имеет вид:
$$
\langle \gamma, fd\rangle \circ 
\langle \varphi, gd\rangle \hm = \langle \gamma_{gd(-)} \circ \varphi, fd \circ 
gd\rangle\,.
$$ 
В~тео\-рии категорий накоплен богатый арсенал алгебраических 
методов конструирования и~анализа диаграмм.
   
   Любая мегамодель задается $C$-диа\-грам\-мой, так что категорное 
представление каталогов моделей позволяет формально решать задачи 
мегамоделирования. Морфизмы диаграмм описывают структурные 
преобразования мегамоделей, выполняемые при помощи инструментов MBSE. 
Покажем, как решаются средствами теории категорий прямые задачи 
мегамоделирования. Здесь применяется одна из основных  
тео\-ре\-ти\-ко-ка\-те\-гор\-ных конструкций~--- копредел  
диаграммы~\cite{5-kov}, который строится следующим образом. Обозначим 
через~$\mathbf{1}$ категорию,\linebreak состоящую из одного объекта~0 и~одного 
морфизма~$1_0$. Из любой категории~$X$ имеется в~точ\-ности один 
функтор~$!_X : X \hm\to \mathbf{1}$, сопоставляющий объект~0  
любому~$X$-объ\-ек\-ту (иными словами, $\mathbf{1}$ является терминальным 
$\mathbf{CAT}$-объ\-ек\-том). Имеется вложение (инъективный функтор) 
$\ulcorner \mbox{--}\urcorner : C \hookrightarrow \mathbf{D}C$, сопоставляющее 
произвольному $C$-объ\-ек\-ту $Q$~точку~--- диаграмму $\ulcorner Q\urcorner : 
\mathbf{1}\hm\to  C : 0 \mapsto Q$. Коконусом (cocone) называется 
$\mathbf{D}C$-мор\-физм, имеющий точку в~качестве кообласти. Можно 
изобразить коконус $\langle \sigma, !_X\rangle : \Delta\hm\to \ulcorner 
Q\urcorner$ над диаграммой $\Delta : X\hm\to C$ в~виде диаграммы, 
<<пририсовав>> к~$\Delta$ дополнительную вершину, помеченную 
объектом~$Q$, и~набор ребер~--- стрелок, по одной для каждого узла $I\hm\in 
\mathrm{Ob}\,X$, направленной из~$I$ в~вершину и~помеченной морфизмом 
$\sigma_I : \Delta (I) \hm\to Q$. Копределом (colimit) диаграммы~$\Delta$ 
называется коконус $\mathrm{colim}\,\Delta : \Delta\hm\to \ulcorner R\urcorner$, 
универсальный в~том смысле, что для любых \mbox{$C$-объ}\-ек\-та~$T$ 
и~коконуса~$\delta : \Delta\hm\to\ulcorner T\urcorner$ существует единственный 
$C$-мор\-физм~$w : R \hm\to T$ такой, что $\delta\hm= \ulcorner w\urcorner \circ  
\mathrm{colim}\,\Delta$. Легко видеть, что это условие универсальности 
представляет собой в~точности структурный критерий из разд.~3. Таким 
образом, конструирование копредела конфигурации~$\Delta$ описывает на 
строгом математическом языке сборку системы, которой отвечает 
вершина~$R$. В~категориях типа $\mathbf{MBS}$ и~$\mathbf{Pomset}$ 
построение копредела сводится к~факторизации раздельных объединений 
объектов, представляющих компоненты системы, по отношениям 
эквивалентности, индуцированным моделями клея и~других средств сборки.
   
   Копредел любой диаграммы, если он сущест\-вует, определяется однозначно 
   с~точностью до изомор\-физма. Более того, можно описать сборку сис\-тем из 
конфигураций в~виде функтора. Пусть $Cd$~--- некоторый класс  
$C$-диа\-грамм, имеющих копределы. Он порождает полную подкатегорию 
в~$\mathbf{D}C$, из которой в~$C$ действует функтор копредела $\mathrm{colim}$, 
сопоставляя каждой диаграмме из~$Cd$~вершину некоторого ее копредела, а 
каждому \mbox{$\mathbf{D}C$-мор}\-физ\-му~$\theta : \Delta\hm\to \Xi$, 
где~$\Delta, \Xi\hm\in Cd$~--- стрелку копредела $\mathrm{colim}\,(\theta)$ такую, что 
$\mathrm{colim}\,\Xi \circ \theta \hm= \ulcorner \mathrm{colim}\,(\theta)\urcorner \circ 
\mathrm{colim}\,\Delta$.

%\begin{figure*}
\vspace*{1pt}
\begin{center}
\mbox{%
\epsfxsize=56.127mm
\epsfbox{kov-4.eps}
}
\end{center}
%\vspace*{-9pt}
%\end{figure*}

   Например, в~категории \textbf{Set} любая диаграмма имеет 
копредел~\cite[упражнение~5.1.8]{14-kov}, поэтому имеется функтор $\mathrm{colim}\, : 
\mathbf{D}(\mathbf{Set})\hm\to \mathbf{Set}$. Примечательно, что этот функтор 
является рефлектором: он сопряжен слева с~вложением $\ulcorner \mbox{--}\urcorner : 
\mathbf{Set}\hookrightarrow \mathbf{D}(\mathbf{Set})$, причем 
единица рефлексии состоит из $\mathbf{D}(\mathbf{Set})$-мор\-физ\-мов 
$\mathrm{colim}\,\Delta : \Delta\hm\to \ulcorner\mathrm{colim}\,(\Delta)\urcorner$, 
$\Delta\hm\in \mathrm{Ob}\ \mathbf{D}(\mathbf{Set})$. Напомним, что единица 
рефлексии~--- это естественное преобразование тождественного функтора 
в~композицию рефлектора и~вложения (в~данном случае, естественное 
преобразование функтора $1_{\mathbf{D}(\mathbf{Set})}$ в~$\ulcorner \mathrm{colim}\,(  
\mbox{--})\urcorner)$, состоящее из универсальных  
стрелок~\cite[разд.~4.3]{14-kov}. И~для произвольного класса~$Cd$, 
содержащего достаточное количество одноточечных диаграмм, функтор 
$\mathrm{colim}$ сопряжен слева с~ограничением  
вложения~$\ulcorner \mbox{--}\urcorner$ на подходящую полную подкатегорию 
в~$C$. А~поскольку сопряженный функтор задается однозначно с~точностью 
до изоморфизма~\cite[разд.~4.1]{14-kov}, можно сделать вывод, что сборка 
систем в~некотором смысле <<зашифрована>> в~процедуре построения 
одноточечных диаграмм~--- моделей систем как целого без раскрытия 
струк\-туры. 

Так наглядно проявляется двойственность прямых и~обратных задач 
мегамоделирования.

\section{Заключение}

   Аппарат теории категорий обладает большим потенциалом в~области 
повышения полезной отдачи от MBSE, в~том числе путем математически 
строгого решения задач мегамоделирования. Так, базовая процедура системной 
инженерии~--- сборка\linebreak
 системы из заданной конфигурации взаимо\-свя\-занных 
компонентов~--- формально описывается тео\-ретико-ка\-те\-гор\-ной 
конструкцией копредела диа\-граммы. Более сложные конструкции отвечают\linebreak 
сложным процедурам сборки, таким как связывание (weaving) общесистемных 
функций, рассеянных по всем компонентам (crosscutting concerns), например 
мониторинговых или защитных~\cite{16-kov}. Математического представления 
требуют и~другие процедуры MBSE, в~частности коллективная модификация 
мегамоделей и~составляющих моделей, восстановление конфигурации заданной 
системы, оценка взаимозаменяемости компонентов. 

Актуальны вопросы 
внедрения аппарата теории категорий в~практику, в~том числе путем развития 
программных инструментов моделирования и~мегамоделирования. Здесь 
открывается широкий спектр направлений для дальнейших исследований.
   
{\small\frenchspacing
 {%\baselineskip=10.8pt
 \addcontentsline{toc}{section}{References}
 \begin{thebibliography}{99}
\bibitem{1-kov}
Modeling and simulation-based systems engineering handbook~/
Eds.\ D.~Gianni,  A.~D'Ambrogio, A.~Tolk.~--- London: CRC Press, 2014. 513~p.
\bibitem{2-kov}
\Au{Ковалёв С.\,П., Толок~А.\,В.} Применение модельно-ори\-ен\-ти\-ро\-ван\-но\-го подхода 
в~управ\-ле\-нии жизненным циклом технических изделий~// Информационные технологии 
в~проектировании и~производстве, 2015. №\,2. С.~3--9.
\bibitem{3-kov}
\Au{Левенчук А.\,И.} Системноинженерное мышление.~--- М.: TechInvestLab, 2015. 305~с.
\bibitem{4-kov}
IEC 81346-1:2009. Industrial Systems, Installations and Equipment and Industrial Products~--- 
Structuring Principles and Reference Designations~--- Part~1: Basic Rules.~--- Geneva: ISO, 2009. 
168~p.
\bibitem{5-kov}
\Au{Ginali S., Goguen~J.} A~categorical approach to general systems~// 
 Conference (International) on Applied General Systems 
Research Proceedings~/
Ed. G.\,J.~Klir.~--- NATO conference series.~--- New York, NY, USA: Plenum 
Press, 1978. Vol.~5. P.~257--270.
\bibitem{6-kov}
\Au{Mabrok M.\,A., Ryan M.\,J.} Category theory as a~formal mathematical foundation for  
model-based systems engineering~// Appl. Math. Inform. Sci., 2017. Vol.~11. No.\,1. P.~43--51.
\bibitem{7-kov}
\Au{Ковалёв С.\,П.} Тео\-ре\-ти\-ко-ка\-те\-гор\-ный подход к~проектированию программных 
сис\-тем~// Фундаментальная и~прикладная математика, 2014. Т.~19. Вып.~3. С.~111--170.
\bibitem{8-kov}
\Au{B$\acute{\mbox{e}}$zivin J., Jouault~F., Rosenthal~P., Valduriez~P.} Modeling in the large 
and modeling in the small~// Model Driven Architecture: European MDA Workshops on 
Foundations and Applications Proceedings~/
Eds.\ U.~A{\!\ptb{\ss}}mann, M.~Aksit,  A.~Rensink.~--- 
Lecture notes in computer science ser.~--- Springer, 2005. Vol.~3599. 
P.~33--46.
\bibitem{9-kov}
\Au{Diskin Z., Kokaly~S., Maibaum~T.} Mapping-aware mega\-mod\-eling: Design patterns and 
laws~// Software Language Engineering: 6th Conference (International) Proceedings~/
Eds.\ M.~Erwig, R.\,F.~Paige, E.~Van Wyk.~--- 
Lecture notes  in computer science ser.~--- Springer, 2013. Vol.~8225. P.~322--343.
\bibitem{10-kov}
\Au{Requicha A.\,G.} Representations for rigid solids: Theory, methods, and systems~// 
ACM  Comput. Surv., 1980. Vol.~12. Iss.~4. P.~437--464.
\bibitem{11-kov}
\Au{K$\acute{\mbox{a}}$d$\acute{\mbox{a}}$r B., Pfeiffer~A., Monostori~L.} Discrete event 
simulation for supporting production planning and scheduling decisions in digital
 factories~//  37th 
CIRP Seminar (International) on Manufacturing Systems Proceedings.~--- Budapest, 2004.  
P.~444--448.
\bibitem{12-kov}
\Au{Giesa T., Spivak D.\,I., Buehler~M.\,J.} Category theory based solution for the building block 
replacement problem in materials design~// Adv. Eng. Mater., 2012. Vol.~14. 
Iss.~9. P.~810--817.
\bibitem{13-kov}
\Au{Косяков А., Свит У., Сеймур~С., Бимер~С.} Системная инженерия. Принципы 
и~практика~/ Пер. с~англ.~--- М.: ДМК-Пресс, 2014. 636~с. (\Au{Kossiakoff~A., Sweet~W.\,N., 
Seymour~S., Biemer~S.\,M.} Systems engineering principles and practice.~--- 2nd ed.~--- New 
York, NY, USA: John Wiley, 2011. 560~p.)
\bibitem{14-kov}
\Au{Маклейн С.} Категории для работающего математика~/ Пер. с~англ.~--- М.: Физматлит, 
2004. 352~с. (\Au{Mac Lane~S.} Categories for the working mathematician.~--- New York, NY, 
USA: Springer, 1978. 317~p.)
\bibitem{15-kov}
\Au{Pratt V.\,R.} Modeling concurrency with partial orders~// Int. J.~Parallel 
Prog., 1986. Vol.~15. No.\,1. P.~33--71.
\bibitem{16-kov}
\Au{Ковалёв С.\,П.} Семантика ас\-пект\-но-ори\-ен\-ти\-ро\-ван\-но\-го моделирования 
данных и~процессов~// Информатика и~её применения, 2013. Т.~7. Вып.~3. С.~70--80.
 \end{thebibliography}

 }
 }

\end{multicols}

\vspace*{-3pt}

\hfill{\small\textit{Поступила в~редакцию 16.01.17}}

%\vspace*{8pt}

\newpage

\vspace*{-30pt}

%\hrule

%\vspace*{2pt}

%\hrule

%\vspace*{8pt}


\def\tit{METHODS OF CATEGORY THEORY IN~MODEL-BASED SYSTEMS ENGINEERING\\[-7pt]}

\def\titkol{Methods of category theory in~model-based systems engineering}

\def\aut{S.\,P.~Kovalyov\\[-12pt]}

\def\autkol{S.\,P.~Kovalyov}

\titel{\tit}{\aut}{\autkol}{\titkol}

\vspace*{-14pt}


\noindent
Institute of Control Sciences, Russian Academy of Sciences, 65~Profsoyuznaya Str., 
Moscow 117997, Russian Federation



\def\leftfootline{\small{\textbf{\thepage}
\hfill INFORMATIKA I EE PRIMENENIYA~--- INFORMATICS AND
APPLICATIONS\ \ \ 2017\ \ \ volume~11\ \ \ issue\ 3}
}%
 \def\rightfootline{\small{INFORMATIKA I EE PRIMENENIYA~---
INFORMATICS AND APPLICATIONS\ \ \ 2017\ \ \ volume~11\ \ \ issue\ 3
\hfill \textbf{\thepage}}}

\vspace*{1pt}

 

\Abste{A mathematical device based on the category theory is proposed to formally describe and 
rigorously explore procedures of employing models in engineering that constitute the contents of 
model-based systems engineering (MBSE). The essence of the device consists in mathematical 
representation of assembly drawings (megamodels of systems) as diagrams in categories whose 
objects are models, and morphisms represent actions associated with assembling system models 
from component models. The soundness of the device is justified on the basis of standards that 
govern description of the systems' structure such as IEC~81346. Category-theoretical methods for 
solving a number of practical problems of assembling systems are proposed and explored. 
Examples of solving such problems are provided in categories that represent two key application 
areas for MBSE: geometric modeling of complex shapes and discrete-event simulation of the 
behavior of industrial systems.}

\KWE{ model-based systems engineering; megamodel; category theory; colimit}

\DOI{10.14357/19922264170305} 

%\vspace*{-18pt}

%\Ack
%\noindent




\vspace*{-7pt}

  \begin{multicols}{2}

\renewcommand{\bibname}{\protect\rmfamily References}
%\renewcommand{\bibname}{\large\protect\rm References}

{\small\frenchspacing
 {%\baselineskip=10.8pt
 \addcontentsline{toc}{section}{References}
 \begin{thebibliography}{99}
\bibitem{1-kov-1}
Gianni, D., A.~D'Ambrogio, and A.~Tolk, eds. 2014. \textit{Modeling and simulation-based 
systems engineering handbook}. London: CRC Press. 513~p.
\bibitem{2-kov-1}
\Aue{Kovalyov, S.\,P., and A.\,V.~Tolok.} 2015. Primenenie model'no-orientirovannogo podkhoda 
v~upravlenii zhiznennym tsiklom tekhnicheskikh izdeliy [Applying model-based approach 
to product lifecycle management].\linebreak \textit{Informatsionnye tekhnologii v~proektirovanii 
i~proizvod\-st\-ve} [Information Technologies in Design and Industry] 2(158):3--9.
\bibitem{3-kov-1}
\Aue{Levenchuk A.\,I.} 2015. 
\textit{Sistemnoinzhenernoe myshlenie} [Systems engineering thinking]. 
Moscow: TechInvestLab. 305~p.
\bibitem{4-kov-1}
IEC 81346-1:2009. 2009. 
Industrial Systems, Installations and Equipment and Industrial 
Products~--- Structuring Principles and Reference Designations~--- 
Part~1: Basic Rules. Geneva:  ISO. 168~p.
\bibitem{5-kov-1}
\Aue{Ginali, S., and J.~Goguen.} 1978. 
A~categorical approach to general systems. \textit{Conference 
(International) on Applied General Systems Research Proceedings}. Ed.\
 G.\,J.~Klir. \mbox{NATO}  conference ser. Plenum Press. 5:257--270.
\bibitem{6-kov-1}
\Aue{Mabrok, M.\,A., and M.\,J.~Ryan}. 
2017. Category theory as a~formal mathematical foundation for 
model-based systems engineering. \textit{Appl. Math.  Inform. Sci.} 11(1):43--51.
\bibitem{7-kov-1}
\Aue{Kovalyov, S.\,P.} 2016. 
Category-theoretic approach to software systems design. \textit{J.~Math. Sci.} 
214(6):814--853.
\bibitem{8-kov-1}
\Aue{B$\acute{\mbox{e}}$zivin, J., F.~Jouault, P.~Rosenthal, and P.~Valduriez.}
 2005. Modeling in 
the large and modeling in the small. 
\textit{Model Driven Architecture: European MDA Workshops on 
Foundations and Applications Proceedings.} 
Eds.\ U.~\mbox{A{\!\ptb{\ss}}mann}, M.~Aksit, and A.~Rensink. 
Lecture notes in computer science ser. Springer. 3599:33--46.
\bibitem{9-kov-1}
\Aue{Diskin, Z., S.~Kokaly, and T.~Maibaum.} 2013. 
Mapping-aware megamodeling: Design patterns 
and laws. \textit{6th Conference (International) on Software Language Engineering 
Proceedings}. Eds.\ M.~Erwig, R.\,F.~Paige, and E.~Van Wyk. 
Lecture notes in computer science ser. Springer. 
8225:322--343.
\bibitem{10-kov-1}
\Aue{Requicha, A.\,G.} 1980. Representations for rigid solids: 
Theory, methods, and systems. \textit{ACM 
Comput. Surv.} 12(4):437--464.
\bibitem{11-kov-1}
\Aue{K$\acute{\mbox{a}}$d$\acute{\mbox{a}}$r,~B., A.~Pfeiffer, and L.~Monostori.}
2004. Discrete 
event simulation for supporting production planning and scheduling decisions in 
digital factories. \textit{37th CIRP Seminar (International) on Manufacturing 
Systems Proceedings}. Budapest.  444--448.
\bibitem{12-kov-1}
\Aue{Giesa, T., D.\,I.~Spivak, and M.\,J.~Buehler.} 2012. 
Category theory based solution for the building 
block replacement problem in materials design. 
\textit{Adv. Eng. Mater.} 14(9):810--817.
\bibitem{13-kov-1}
\Aue{Kossiakoff, A., W.\,N.~Sweet, S.~Seymour, and S.\,M.~Bie\-mer.}
2011. \textit{Systems engineering 
principles and practice}. 2nd ed. New York, NY: John Wiley. 560~p.
\bibitem{14-kov-1}
\Aue{Mac Lane, S.} 1978. \textit{Categories for the working mathematician}. 
New York, NY: Springer. 317~p.
\bibitem{15-kov-1}
\Aue{Pratt, V.\,R.} 1986. Modeling concurrency with partial orders. 
\textit{Int. J.~Parallel Prog.} 15(1):33--71.
\bibitem{16-kov-1}
\Aue{Kovalyov, S.\,P.} 2013. 
Semantika aspektno-ori\-en\-ti\-ro\-van\-no\-go modelirovaniya dannykh 
i~protsessov [Semantics of aspect-oriented modeling of data and processes]. 
\textit{Informatika i~ee  Primeneniya~--- Inform. Appl.} 7(3):70--80.
\end{thebibliography}

 }
 }

\end{multicols}

\vspace*{-9pt}

\hfill{\small\textit{Received January 16, 2017}}

\vspace*{-18pt}

\Contrl

\noindent
\textbf{Kovalyov Sergey P.} (b.\ 1972)~--- Doctor of Science in physics and 
mathematics, leading scientist, Institute of Control Problems, Russian 
Academy of Sciences, 65~Profsoyuznaya Str., Moscow 117997, Russian 
Federation Federation; \mbox{kovalyov@nm.ru} 

\label{end\stat}


\renewcommand{\bibname}{\protect\rm Литература}   %1
\def\stat{bitbos}

\def\tit{О ПОИСКЕ ОПТИМАЛЬНОЙ СХЕМЫ 3D-ПЕЧАТИ КОНСТРУКЦИЙ 
ИЗ~КОМПОЗИЦИОННЫХ МАТЕРИАЛОВ}

\def\titkol{О поиске оптимальной схемы 3D-печати конструкций 
из~композиционных материалов}

\def\aut{А.\,В.~Босов$^1$, Ю.\,И.~Битюков$^2$, Г.\,Ю.~Денискина$^3$}

\def\autkol{А.\,В.~Босов, Ю.\,И.~Битюков, Г.\,Ю.~Денискина}

\titel{\tit}{\aut}{\autkol}{\titkol}

\index{Босов А.\,В.}
\index{Битюков Ю.\,И.}
\index{Денискина Г.\,Ю.}
\index{Bosov A.\,V.}
\index{Bityukov Yu.\,I.}
\index{Deniskina G.\,Yu.}


%{\renewcommand{\thefootnote}{\fnsymbol{footnote}} \footnotetext[1]
%{Работа выполнена при поддержке Министерства науки и~высшего образования Российской Федерации (проект 
%075-15-2020-799).}}


\renewcommand{\thefootnote}{\arabic{footnote}}
\footnotetext[1]{Федеральный исследовательский центр <<Информатика и~управление>> Российской академии наук; 
Московский авиационный институт, \mbox{avbosov@ipiran.ru}}
\footnotetext[2]{Московский авиационный институт, \mbox{yib72@mail.ru}}
\footnotetext[3]{Московский авиационный институт, \mbox{dega17@yandex.ru}}

%\vspace*{-6pt}





\Abst{Статья посвящена задаче поиска оптимальных траекторий укладки волокон при 
изготовлении конструкций, армированных непрерывными волокнами, методом 3D-пе\-ча\-ти. 
Предложена оптимизационная постановка, в которой целевой функцией выступает один из 
критериев разрушения композита. Схемы укладки волокон при печати моделируются  
с~по\-мощью аналитических функций, которые находятся из задачи Неймана для уравнения 
Лапласа. Краевые условия определяются на основе задания углов между волокнами и~границей  
об\-ласти. Задача Неймана решается посредством конформного преобразования области печати 
на круг. Таким образом, критерий разрушения композита становится функцией от углов, 
которые волокна образуют с~границей области. Для минимизации целевой функции 
используется генетический алгоритм поиска глобального минимума функции нескольких 
переменных.}

\KW{композиционные материалы; вейвлеты; 3D-пе\-чать; аналитическая функция}

\DOI{10.14357/19922264220102}
  
%\vspace*{-4pt}


\vskip 10pt plus 9pt minus 6pt

\thispagestyle{headings}

\begin{multicols}{2}

\label{st\stat}

\section{Введение}

  Композиционные материалы (КМ) широко применяются в~разных отраслях 
про\-мыш\-лен\-ности,\linebreak например авиационной, автомобильной. Композиционные материалы состоят из 
ар\-ми\-ру\-юще\-го и~свя\-зу\-юще\-го материалов; в~качестве ар\-ми\-ру\-юще\-го широко 
применяются углеродные волокна, \mbox{об\-ла\-да\-ющие} большой удельной проч\-ностью. 
Механические свойства изделий из КМ зависят от на\-прав\-ле\-ния этого волокна. 
Перспективным на\-прав\-ле\-нием, поз\-во\-ля\-ющим создавать конструкции\linebreak слож\-ной 
формы последовательной укладкой композиционного материала, стала технология  
3D-пе\-ча\-ти (см.\ пример такого решения в~[1]). Применение 3D-пе\-ча\-ти 
и КМ позволяет получать \mbox{конструкции} с~пространственным армированием по 
заданным траекториям. Пол\-ный контроль над расположением волокон во время 
печати позволяет укладывать их именно так, как того требуют условия 
эксплуатации: чтобы~100\% волокон шли в~нужном направлении. Для 
практической реализации, для разработки автоматизированной сис\-те\-мы 
проектирования (CAD/CAE-сис\-те\-мы) и~моделирования процесса 3D-пе\-ча\-ти 
актуальна задача поиска оптимальной схемы печати, дик\-ту\-емой условиями 
эксплуатации изделия.
  
  Как известно~[2], схема укладки волокна заложена в~самих уравнениях 
механики КМ в~виде некоторой (неизвестной) локальной ортогональной сис\-те\-мы 
координат, т.\,е.\ найти эту схему можно только из решений уравнений 
с~разными локальными системами координат. Для задания локальной сис\-те\-мы 
координат в~статье использованы аналитические функции, которые строятся на 
основе известных формул Дини и~Чизотти~[3] всего лишь заданием направления 
укладки волокна на границе изделия. В~качестве критерия оптимизации для 
выбора траекторий уклад\-ки волокон можно взять любой из критериев разрушения 
КМ~[2].
  
\section{Моделирование траекторий укладки волокон  
при~3D-печати}

  Пусть $X\subset \mathbf{R}^2$~---  об\-ласть печати. Касательные векторы 
к~кривым, по которым укладываются волокна при 3D-пе\-ча\-ти, образуют 
векторное поле в~$X$, которое будем характеризовать комплексным чис\-лом 
$r\hm= r_1\hm+ ir_2$, где $r_1\hm= r_1(x_1, x_2)$; $r_2\hm= r_2(x_1, x_2)$. Это 
поле предполагается гармоническим, т.\,е.\ соленоидальным и~потенциальным~[3], 
такое поле не имеет источников и~вих\-рей. Будем считать, что $X$~--- 
подмножество односвязной области~$\tilde{X}$: $X\hm\subset \tilde{X}$. Значит, 
выражение $-r_2dx_1\hm+ r_1dx_2$ есть полный дифференциал некоторой 
функции $v_2\hm= v_2(x_1, x_2)$, определенной на~$\tilde{X}$. Эта функция 
называется функцией тока~[3]. Кроме того, выражение $r_1dx_2\hm+ r_2dx_2$ 
также есть полный дифференциал некоторой функции $v_1\hm= v_1(x_1, x_2)$, 
которая называется потенциалом поля~[3]. Функция тока $v_2(x_1, x_2)$  
и~потенциал поля $v_1(x_1, x_2)$ являются сопряженными гармоническими 
функциями~[3]. Линии тока и~линии равного потенциала образуют ортогональное 
семейство. Аналитическая функция $v_1(x_1, x_2)\hm+ iv_2(x_1,x_2)$, $x_1\hm+ 
ix_2\hm\in \tilde{X}$, называется комплексным потенциалом поля~[3]. Таким 
образом, любая аналитическая функция в~об\-ласти~$\tilde{X}$ дает и~схему 
уклад\-ки волокон, и~локальную криволинейную систему координат в~$X\hm\subset 
\tilde{X}$. Будем далее точ\-ки $x\hm= (x_1, x_2)$ изображать на одной 
комплексной плос\-кости, а~точ\-ки $v\hm= (v_1, v_2)$~--- на другой. Тогда 
преобразование $v_1\hm=v_1(x_2, x_2)$, $v_2\hm= v_2(x_1, x_2)$ и~его обратное 
$x_1\hm= x_1(v_1, v_2)$, $x_2\hm= x_2(v_1, v_2)$ будет пред\-став\-лять собой 
преобразование множества~$\tilde{X}$ на плоскости~$x$ в~множество~$\Omega$ 
на плоскости~$v$. Сеть линий уров\-ня $v_1(x_1, x_2)\hm=const$, $v_2(x_1, 
x_2)\hm= const$ называется изотермической сетью~[3]. Кривые, по которым 
укладываются волокна, определяются па\-ра\-мет\-ри\-че\-ски\-ми пред\-став\-ле\-ни\-ями
  \begin{align*}
  \gamma_{\alpha,1}:\,& r_{\alpha,1} (v_1) 
=x_1(v_1,\alpha)+ix_2(v_1,\alpha)\,,\enskip v_1\in T_{\alpha,1}\,;\\
  \gamma_{\beta,2}:\, & r_{\beta,2} (v_2) =x_1(\beta, v_2)+ix_2(\beta, v_2)\,,\enskip 
v_2\in T_{\beta,2}\,,
  \end{align*}
где $T_{\alpha,1}$ и~$T_{\beta,2}$~--- некоторые промежутки; $\alpha, \beta \hm\in 
\mathbf{R}$; $T_{\alpha,1} \times \{\alpha\}, \{\beta\}\times 
T_{\beta,2}\hm\subset\Omega$.
  
  Поскольку функция $v_1(x_1, x_2)$ является гармонической в~односвязной  
об\-ласти~$\tilde{X}$, то она удовле\-тво\-ря\-ет уравнению Лапласа. 
Пусть~$\theta(x)$~--- угол между внеш\-ней единичной нормалью 
$$
n(x)=n_1(x)+ in_2(x)
$$ 
к~границе об\-ласти~$\tilde{X}$ и~волокном. В~рамках 
рас\-смат\-ри\-ва\-емой модели 3D-пе\-ча\-ти на\-прав\-ле\-ние волокна в~точке $x\hm\in 
\partial \tilde{X}$ задается вектором 
$$
  t(x)= \left.\fr{\partial x_1}{\partial v_1} \right\vert_{v(x)} +
  \left. i \fr{\partial x_2}{\partial v_1} 
  \right\vert_{v(x)}\,.
$$
 Учитывая, что $x_1\equiv x_1(v_1(x_1,x_2), v_2(x_1,x_2))$ 
и~$x_2\hm\equiv x_2(v_1(x_1x_2), v_2(x_1,x_2))$, а~также соотношения  
Ко\-ши--Ри\-ма\-на, получаем 
  $$
  t(x)= \fr{1}{\vert \nabla v_1(x)\vert} \left( \fr{\partial v_1(x)}{\partial x_1} 
+i\fr{\partial v_1(x)}{\partial x_2}\right)\,.
  $$
   Таким образом, на границе об\-ласти должно выполняться 
  $$
  \fr{1}{\vert \nabla v_1(x)\vert} \left( \fr{\partial v_1(x)}{\partial x_1}\,n_1(x)+ 
\fr{\partial v_1(x)}{\partial x_2}\,n_2(x)\right)= \cos\theta(x),
  $$ 
  что можно переписать в~виде 
  $$
  \fr{\partial v_1(x)}{\partial n}= q(x)\cos\theta(x),
  $$
   где 
$q(x)\hm= \vert \nabla v_1(x)\vert$. Известно~[3], что долж\-но выполняться условие
  $$
  \int\limits_{\partial\tilde{X}} \fr{\partial v_1(x)}{\partial n}\,ds 
=\int\limits_{\partial\tilde{X}} \fr{\partial v_2(x)}{\partial s}\,ds 
=\int\limits_{\partial\tilde{X}}dv_2=0\,,
  $$
  где $\partial v_2(x)/\partial s$~--- производная по на\-прав\-ле\-нию касательной 
  к~границе об\-ласти. Поэтому $q(x)$ не может быть произвольной. Итак, функцию 
$v_1(x_1, x_2)$ на~$\tilde{X}$ будем искать из задачи Неймана:
  \begin{equation}
  \left.
  \begin{array}{c}
  \fr{\partial^2 v_1}{\partial x_1^2}+ \fr{\partial^2 v_1}{\partial x_2^2}=0\,;\\[6pt]
  \fr{\partial v_1(x)}{\partial n} =\eta(x)\,,\enskip x=\left( x_1, x_2\right) \in 
\partial\tilde{X}\,,
  \end{array}
  \right\}
  \label{e1-bos}
  \end{equation}
где $\eta(x)\hm= q(x)\cos\theta(x)$. Функция~$q(x)$ в~пред\-став\-лен\-ных ниже 
примерах выбиралась сле\-ду\-ющим образом:
\begin{multline*}
q(x)={}\\
{}= \begin{cases}\!
\left(\,\displaystyle\int\limits_{\cos\theta(x)>0}\!\!\cos\theta(x)\,ds\right)^{-1}\!\!, &\!\! \mbox{если } 
\cos\theta(x)>0;\\[3pt]
-\left(\,\displaystyle\int\limits_{\cos\theta(x)<0}\!\! \cos\theta (x)\,ds\right)^{-1}\!\!, &\!\! \mbox{если 
}\cos\theta(x)<0.
\end{cases}\hspace*{-9pt}
%\label{e2-bos}
\end{multline*}
  
  Пусть $\Gamma_\rho \hm= \{w=w_1\hm+ iw_2: w_1^2\hm+ w_2^2\hm<\rho\}$ 
  и~$x\hm= f(w)\hm= f_1(w_1, w_2)\hm+ if_2(w_1, w_2)$~--- конформное 
преобразование круга~$\Gamma_1$ на некоторую односвязную область~$D$, 
содержащую множество~$X$ (рис.~1). Для $\rho \hm\in (0;1)$  обозначим 
$\tilde{X}_\rho \hm= f(\Gamma_\rho)$. Причем~$\rho$ выбрано так, что 
$X\hm\subset \tilde{X}_\rho$. Если обозначить через~$v(t)$ угол наклона 
касательной к~границе области~$D$ в~точке~$x$, соответствующей точке $w\hm= 
w_1\hm+ iw_2\hm\in \partial \Gamma_1$ при конформном отоб\-ра\-же\-нии $x\hm= 
f(w)$, то конформное преобразование единичного круга на об\-ласть~$D$ может 
быть найдено по формуле Чизотти~[3]:
  \begin{align*}
  x&=f_1(w_1,w_2) +if_2(w_1,w_2)={}\\
 & \hspace*{10mm}{}=i\int\limits_{w_{1,0}+iw_{2,0}}^{w_1+iw_2} 
\fr{e^{i\xi(y)}}{(1-y)^2}\,dy +x_0\,;\\
  \xi(y)&=\fr{1}{2\pi} \int\limits_0^{2\pi} v(t) \fr{e^{it}+y}{e^{it}-y}\,dt+i\xi_0\,,
  \end{align*}
где $\xi_0$~--- некоторая действительная по\-сто\-ян\-ная; $x_0\hm\in D$ 
и~$w_{0,1}\hm+ i w_{0,2}\hm\in \Gamma_1$~--- заданные точки. Пусть теперь 
функция~$v_1(x)$ является решением задачи~(1), где $\tilde{X}\hm= 
\tilde{X}_\rho$. Из соотношений Ко\-ши--Ри\-ма\-на и~гармоничности функций 
$f_1(w_1, w_2)$ и~$f_2(w_1, w_2)$ следует, что функция $z(w)\hm= v_1(f(w))$\linebreak\vspace*{-12pt}

{ \begin{center}  %fig1
 \vspace*{-6pt}
    \mbox{%
\epsfxsize=79mm
\epsfbox{bos-1.eps}
}

\end{center}

\vspace*{-3pt}

\noindent
{{\figurename~1}\ \ \small{
Конформное преобразование единичного круга на многоугольник и~его ограничения 
на круги $\Gamma_{0{,}95}$ и~$\Gamma_{0{,}75}$ 
}}}

\vspace*{18pt}

\setcounter{figure}{1}


\noindent 
удовле\-тво\-ря\-ет уравнению $\partial^2 z/\partial w_1^2\hm+ \partial^2 z/\partial 
w_2^2\hm=0$.
 Определим краевое условие. Пусть $w\hm= w_1\hm+ iw_2\hm\in 
\partial \Gamma_\rho$  и~$m$~--- единичная нормаль к~границе 
круга~$\Gamma_\rho$. Тогда $m\hm= w_1/\rho \hm+ iw_2/\rho$. Следовательно,
\begin{multline*}
\fr{\partial z(w)}{\partial m}=\fr{\partial z(w)}{\partial w_1}\,\fr{w_1}{\rho} 
+\fr{\partial z(w)}{\partial w_2}\fr{w_2}{\rho}={}\\[6pt]
{}=\fr{\partial v_1(f(w))}{\partial x_1}\,\fr{\partial f_1(w)}{\partial m}+\fr{\partial 
v_1(f(w))}{\partial x_2}\,\fr{\partial f_2(w)} {\partial m}\,.
\end{multline*}
  
  Рассмотрим кривую $\partial \Gamma_\rho$  и~кривую с~параметрическим 
представлением~$\gamma_w$: $r_w(t)\hm= w_1\tau\hm+ iw_2\tau$, $\tau\hm\in 
[0;1]$. Они перпендикулярны в~точке $\tau\hm=1$, а~их образы при конформном 
преобразовании $x\hm= f(w)$ пред\-став\-ля\-ют собой кривую 
$\partial \tilde{X}_\rho$ и~кривую $\gamma_{f(w)}$: $r_{f(w)}(\tau)\hm= f_1(w_1\tau, w_2\tau)\hm+ 
if_2(w_1\tau, w_2\tau)$, $\tau\hm\in [0;1]$. Поскольку конформное преобразование 
сохраняет углы между кривыми, то вектор нормали к~границе~$\partial 
\tilde{X}_\rho$ коллинеарен касательному век\-то\-ру кривой $\gamma_{f(w)}$ 
в~точке, со\-от\-вет\-ст\-ву\-ющей $\tau\hm=1$. С~учетом того, что
 \begin{multline*}
 \left\vert r^\prime_{f(w)}(1)\right\vert ={}\\
  {}=\rho \vert \nabla f_1(w)\vert =\rho \sqrt{\left( 
\fr{\partial f_1(w)}{\partial w_1}\right)^2 +\left( \fr{\partial f_1(w)}{\partial 
w_2}\right)^2}\,,
\end{multline*}
получаем
$$
n=\fr{r^\prime_{f(w)} (1)}{\vert r^\prime_{f(w)} (1)\vert }=\fr{\partial f_1(w)/\partial 
m+ i \partial f_2(w)/\partial m} {\left\vert \nabla f_1(w)\right\vert}\,.
$$
  
  Следовательно, функция~$z$ является решением задачи Неймана для 
круга~$\Gamma_\rho$: $\partial^2 z/ \partial w_1^2 \hm+ \partial^2 z/ \partial 
w_2^2\hm=0$, $\partial z(w)/\partial m \hm= \eta(f(w))\vert \nabla f_1(w)\vert$, 
$w\hm\in \Gamma_\rho$. Как известно~[3], решение такой задачи может быть 
найдено по формуле Дини:
  \begin{multline}
  z(w_1,w_2)=z_0-{}\\
  {}-\fr{\rho}{2\pi} \int\limits_0^{2\pi} \eta (f(\rho\cos\tau, \rho\sin\tau)) \vert \nabla 
f_1(\rho\cos\tau, \rho\sin\tau)\vert \times{}\\
{}\times \ln \fr{(w_1-\rho\cos\tau)^2+(w_2-
\rho\sin\tau)^2}{\rho^2}\,d\tau\,,
  \label{e3-bos}
  \end{multline}
где $z_0\in \mathrm{C}$~--- произвольная константа.
  
  Итак, из~(\ref{e3-bos}) можно найти $z(w)$, $w\hm\in \Gamma_\rho$. Зная 
преобразование $x\hm= f(w)$, из соотношения $z(w)\hm= v_1(f(w))$ можно найти 
$v_1(x)\hm= z(f^{-1}(x))$, $x\hm\in \tilde{X}_\rho$. Функция $v_2(x)$ может быть 
найдена аналогично, но для решения уравнений механики КМ она не нужна, 
достаточно $v_1(x)$ и~соотношений Ко\-ши--Ри\-ма\-на. На рис.~1 показано 
конформное преобразование единичного круга на многоугольник и~его 
ограничения на круги~$\Gamma_{0{,}95}$ и~$\Gamma_{0{,}75}$. На рис.~2 
показано векторное поле, образованное касательными к~кривым $v_1(x_1, 
x_2)\hm=const$~(рис.~2,\,\textit{а}) и~$v_2(x_1,x_2)\hm= const$~(рис.~2,\,\textit{б}). 

\setcounter{figure}{1}
\begin{figure*} %fig2
\vspace*{1pt}
  \begin{center}  
    \mbox{%
\epsfxsize=163mm
\epsfbox{bos-2.eps}
}

\end{center}
\vspace*{-9pt}
\Caption{Касательные векторы к~кривым  $v_1(x_1, x_2)\hm=const$~(\textit{а}) и~$v_2(x_1, 
x_2)\hm=const$~(\textit{б})}
\end{figure*}
\begin{figure*}[b] %fig3
\vspace*{1pt}
  \begin{center}  
    \mbox{%
\epsfxsize=161.879mm
\epsfbox{bos-3.eps}
}

\end{center}
\vspace*{-9pt}
\Caption{Оптимальные траектории уклад\-ки волокон при 3D-пе\-ча\-ти}
\end{figure*}

\section{Построение оптимальной схемы 3D-печати}

  Используем теперь описанную модель для поиска оптимальных траекторий 
уклад\-ки волокон при 3D-пе\-ча\-ти. Пусть $\sigma_1^{\pm}$ и~$\sigma_2^{\pm}$~--- 
пределы проч\-ности при растяжении и~сжатии вдоль и~поперек волокон, 
а~$\hat{\tau}_{12}$~--- предел проч\-ности при сдвиге в~плос\-кости слоя~[2]. 
В~качестве целевой функции используем критерии максимальных на\-пря\-же\-ний 
$$
R\left(\sigma_1, \sigma_2, \tau_{12}\right)= \max\left( \fr{\sigma_1}{m_1(\sigma_1)}, 
\fr{\sigma_2}{m_2(\sigma_2)}, \fr{\tau_{12}}{\hat{\tau}_{12}}\right),
$$ 
где
  $$
  m_1(\sigma_1) =\begin{cases}
  \sigma_1^+\,, &\mbox{если }\sigma_1>0\,;\\
  \sigma_1^-\,, &\mbox{если }\sigma_1<0\,;
  \end{cases}
  $$
  $$
  m_2(\sigma_2) =\begin{cases}
  \sigma_2^+\,, &\mbox{если }\sigma_2>0\,;\\
  \sigma_2^-\,, &\mbox{если }\sigma_2<0\,.
  \end{cases}
  $$
  
  Значения $\sigma_1$, $\sigma_2$ и~$\tau_{12}$ могут быть найдены при\-бли\-жен\-но 
из уравнений механики композиционных материалов~[2], если задать 
преобразование $v(x)\hm= (v_1(x), v_2(x))$. Краевые условия к~этим уравнениям 
определяются углами $\theta(x)$, $x\hm\in \partial X$, которые волокна образуют 
с~внеш\-ней нормалью к~границе множества~$X$. Следовательно, целевая функция 
является функцией этих углов:
 \begin{multline*}
  R: \theta\vert_{\partial X} \mapsto v(x)=\left( v_1(x), v_2(x)\right)\mapsto \left( 
\sigma_1, \sigma_2, \tau_{12}\right) \mapsto {}\hspace*{-1.3625pt}\\
{}\mapsto \max \left( 
\fr{\sigma_1}{m_1(\sigma_1)}, \fr{\sigma_2}{m_2(\sigma_2)}, 
\fr{\tau_{12}}{\hat{\tau}_{12}}\right).
\end{multline*}
  
  Соответствие $(v_1(x), v_2(x)) \hm\mapsto (\sigma_1, \sigma_2, \tau_{12})$ дает 
приближенное решение уравнений механики,\linebreak полученное далее методом 
приближенного решения уравнений в~част\-ных производных с~по\-мощью\linebreak 
вейв\-ле\-тов, построенных на основе схем подразделений~[4] и~подъема~[5]. 
Преимущество вейвлетов перед другими базисными функциями состоит\linebreak в~том, что 
вейв\-лет-ко\-эф\-фи\-ци\-ен\-ты убывают быст\-ро, поэтому достаточно небольшого 
числа сла\-га\-емых в~разложениях. Дополнительное преимущество вейвлетов, 
использующих схемы \mbox{подразделений} и~подъема, со\-сто\-ит в~воз\-мож\-ности 
управ\-лять формой базисных функций, например обнулять их на выбранной 
об\-ласти, что еще уменьшает чис\-ло сла\-га\-емых. Эти преимущества оказываются 
важ\-ны, так как при минимизации целевой функции требуется многократно решать 
сис\-те\-му урав\-не\-ний в~част\-ных производных, опи\-сы\-ва\-ющую 
на\-пря\-жен\-но-де\-фор\-ми\-ру\-емое со\-сто\-яние конструкции.
  
  Минимизировать функцию~$R$ предлагается с~по\-мощью метода, 
пред\-став\-лен\-но\-го в~[6] и~реализованного в~биб\-лио\-те\-ке scipy для языка 
программирования Python. Поскольку не требуется искать глобальный минимум, 
а~достаточно обеспечить выполнение условия $R\hm<1$, можно закончить 
итерационный процесс при его выполнении. На рис.~3 пред\-став\-ле\-ны траектории 
уклад\-ки волокон для 3D-пе\-ча\-ти прямоугольной плас\-ти\-ны с~отверстием

\vspace*{-6pt}

\noindent
\begin{multline*}
  X=\left\{ \left( x_1,x_2\right): \left( x_1-x_{1,0}\right)^2 +\left( x_2-
x_{2,0}\right)^2 \leq r^2\,,\right.\\ 
\left.x_2\in [0;a]\,,\ x_2\in [0;b]\right\}
\end{multline*}
после двух итераций генетического алгоритма.

\vspace*{-6pt}

\section{Применение вейвлетов к~приближенному решению задач 
теории упругости}

\vspace*{-2pt}

  В статье~\cite{7-bos} проанализирован общий подход к~по\-стро\-ению 
би\-ор\-то\-го\-наль\-ных вейв\-лет-сис\-тем на основе схем подъема~\cite{5-bos}, в~том 
чис\-ле показано, что эти схемы тесно связаны со схемами  
подразделений~\cite{4-bos} и~поз\-во\-ля\-ют строить вейвлеты с~заданными 
свойствами. На основе таких вейв\-лет-сис\-тем далее представлен алгоритм 
при\-бли\-жен\-но\-го решения уравнений тео\-рии упругости.
  
  Пусть $X\subset \mathbf{R}^n$. Будем рассматривать действительное 
пространство~$L_2(X)$.


\smallskip

\noindent
\textbf{Определение~1}~\cite{5-bos}. Кратномасштабный анализ на~$X$ 
определяется как последовательность подпространств $V_j\hm\subset L_2(X)$, 
$j\hm= 0,1,\ldots$, такая, что $V_j\hm\subset V_{j+1}$; $\cup_j V_j$ плотно 
в~$L_2(X)$ и~для каж\-до\-го~$j$ существуют масштабирующие функции 
$\varphi_{j,k}$, $k\hm\in K_j$, такие что множество $\{\varphi_{j,k}\}_{k\in K_j}$ 
пред\-став\-ля\-ет собой базис Рисса~\cite{8-bos} в~$V_j$, при этом $K_j\hm\subset 
K_{j+1}$.


\smallskip

\noindent
\textbf{Определение~2}~\cite{8-bos}. Пусть $\{V_j\}_{j\geq 0}$ 
и~$\{\tilde{V}_j\}_{j\geq0}$~--- два кратномасштабных анализа на~$X$ 
с~масштабирующими функциями $\varphi_{j,k}$ 
и~$\tilde{\varphi}_{j,k}$, $k\hm\in K_j$, соответственно. Кроме того, пусть 
$V_{j+1}\hm= V_j\hm+ W_j$, $\tilde{V}_{j+1} \hm= \tilde{V}_j\hm+ \tilde{W}_j$ 
и~$\{\psi_{j,k}, k\hm\in M_j\}$, $\{ \tilde{\psi}_{j,k}, k\hm\in M_j\}$~--- базисы Рисса 
в~$W_j$ и~$\tilde{W}_j$ соответственно. Если 
$(\varphi_{j,k},\tilde{\varphi}_{j,k^\prime})\hm= \delta_{k,k^\prime}$, 
$(\tilde{\psi}_{j,m}, \varphi_{j,k})\hm= 0$, $( \tilde{\varphi}_{j,k}, \psi_{j,m})\hm=0$, 
$(\psi_{j,m}, \tilde{\psi}_{j,m^\prime})\hm=\delta_{m,m^\prime} \forall\,j$, $\forall\,m, 
m^\prime\hm\in M_j$, $\forall\,k, k^\prime\hm\in K_j$, то семейства функций 
$\{\psi_{j,k}\}_{j\geq0,\ k\in M_j}$ и~$\{ \tilde{\psi}_{j,k}\}_{j\geq0, k\in M_j}$ 
называются би\-ор\-то\-го\-наль\-ны\-ми вейв\-лет-сис\-те\-мами.
  
  Из определения~2 следует~\cite{5-bos}, что существуют последовательности $\{ 
h_{j,k,l}\}$ и~$\{\tilde{h}_{j,k,l}\}$ такие, что $\varphi_{j,k}\hm= \sum\nolimits_{l\in 
K_{j+1}} h_{j,k,l}\varphi_{j+1,l}$ и~$\tilde{\varphi}_{j,k}\hm= \sum\nolimits_{l\in 
K_{j+1}} \tilde{h}_{j,k,l} \tilde{\varphi}_{j+1,l}$. Поскольку $W_j\hm\subset 
V_{j+1}$ и~$\tilde{W}_j\hm\subset \tilde{V}_{j+1}$, то $\psi_{j,m}\hm= 
\sum\nolimits_{l\in K_{j+1}} g_{j,m,l}\varphi_{j+1,l}$ и~$\tilde{\psi}_{j,m}\hm= 
\sum\nolimits_{l\in K_{j+1}} \tilde{g}_{j,m,l} \tilde{\varphi}_{j+1,l}$. 
Последовательности $h_{j,k,l}$, $\tilde{h}_{j,k,l}$, $g_{j,k,l}$ и~$\tilde{g}_{j,k,l}$ 
называются фильтрами. В~случае би\-ор\-то\-го\-наль\-ных  
вейв\-лет-сис\-тем для $f\hm\in L_2(X)$ имеет мес\-то равенство~\cite{5-bos, 8-bos}:
  $$
  f=\sum\limits_{k\in K_{j_0}} v_{j_0,k}\varphi_{j_0,k}+ \sum\limits_{j\geq j_0} 
\sum\limits_{m\in M_j} \gamma_{j,m}\psi_{j,m}\,,
  $$ 
  где $v_{j,k} = (f,\tilde{\varphi}_{j,k})$; $\gamma_{j,m} \hm= 
(\tilde{\psi}_{j,m},f)$. 
  
  Схема подъема (lifting scheme) позволяет строить би\-ор\-то\-го\-наль\-ные  
вейв\-лет-сис\-те\-мы с~заданными\linebreak свойствами, используя некоторые начальные 
би\-ор\-то\-го\-наль\-ные вейв\-лет-сис\-те\-мы с~фильт\-ра\-ми $h^0_{j,k,l}$, 
$\tilde{h}^0_{j,k,l}$, $g^0_{j,k,l}$ и~$\tilde{g}^0_{j,k,l}$. По схеме подъема новое 
семейство фильт\-ров $h_{j,k,l}$, $\tilde{h}_{j,k,l}$, $g_{j,k,l}$ и~$\tilde{g}_{j,k,l}$, 
опре\-де\-ля\-ющих биортогональные вейв\-лет-сис\-те\-мы, находится по 
формулам~\cite{5-bos}:
  \begin{gather*}
  h_{j,k,l} =h^0_{j,k,l}\,;\quad
  \tilde{h}_{j,k,l} =\tilde{h}^0_{j,k,l} 
+\sum\limits_{m\in M_j} s_{j,k,m}\tilde{g}^0_{j,m,l}\,;\\
  g_{j,m,l}= g^0_{j,m,l} -\sum\limits_{k\in K_j} s_{j,k,m} 
h^0_{j,k,l}\,;\quad
  \tilde{g}_{j,m,l}=\tilde{g}^0_{j,m,l} 
  \end{gather*}
при любом выборе последовательности $\{ s_{j,k,m}\}_{k\in K_j, \ m\in M_j}$. 
Следует заметить, что мас\-шта\-би\-ру\-ющие функции~$\varphi_{j,k}$ одинаковы 
в~исходном и~поднятом крат\-но\-мас\-штаб\-ном анализах. Кроме того, можно не 
менять функцию~$\tilde{\varphi}_{j,k}$, а~поднимать~$\varphi_{j,k}$. Механизм 
этот точ\-но такой же и~называется двойственной схемой подъема. Он позволяет 
улучшить свойства вейв\-ле\-та~$\tilde{\psi}_{j,m}$.
  
  Пусть последовательности $d^i: \mathbf{Z}\hm\to \mathbf{R}$, $i\hm= 0,1,2$, 
и~$a: \mathbf{Z}\hm\to \mathbf{R}$ имеют носитель $\{-1, 0, 1\}$ и~на нем могут 
быть записаны в~виде:
\begin{alignat*}{2}
d^0&\!=\!\begin{pmatrix}
\fr{1}{6} & \fr{4}{6} & \fr{1}{6}\end{pmatrix};&\ 
d^1&\!=\! 2^{j+p-1}(-1\ 0\ 1); \\
d^2&\!=\!
2^{2(j+p)}(1\ -2\ 1);&\ 
a&\!=\! \begin{pmatrix}\!
-\fr{1}{6} & \fr{8}{6} & -\fr{1}{6}\!\end{pmatrix}\!,\ p\!=\!1,2,\ldots
\end{alignat*}
 Для краткости введем обозначение для 
част\-ной производной
  $$
  \partial^{(\alpha_1, \ldots, \alpha_k)} \! f(x_1, \ldots, x_k)\!=\! 
  \fr{\partial^{\alpha_1+\cdots+\alpha_k} f(x_1,\ldots , x_k)}{\partial 
x_1^{\alpha_1}\cdots \partial x_k^{\alpha_k}}
  $$
  порядка $\alpha_1+\cdots + \alpha_k$. На основании результатов, 
пред\-став\-лен\-ных в~\cite{7-bos, 9-bos}, можно сформулировать сле\-ду\-ющую 
тео\-ре\-му, которая пред\-став\-ля\-ет собой алгоритм на\-хож\-де\-ния значений 
масштабирующих функций и~вейвлетов на~$\mathbf{R}^n$, полученных 
с~по\-мощью схемы подъема.

\begin{figure*}[b] %fig4
\vspace*{1pt}
  \begin{center}  
    \mbox{%
\epsfxsize=159.507mm
\epsfbox{bos-4.eps}
}

\end{center}
\vspace*{-9pt}
\Caption{Графики масштабирующих функций}
\end{figure*}

\smallskip

\noindent
\textbf{Теорема.} \textit{Пусть $\Lambda^n$~--- совокупность всех ненулевых 
векторов $e\hm\in \mathbf{Z}^n$, координаты которых равны~$0$ или~$1$. 
Обозначим $K_j\hm= 2^{-j} Z^n$, $M_j\hm= 2^{-j} Z^n\hm+ 2^{-j-1} \Lambda^n$. 
Предположим, что по\-сле\-до\-ва\-тель\-ность $A\hm= \{A_i\}_{i\in Z}$ определяет 
функцию из~$C^2(\mathbf{R})$ с~компактным носителем}~\cite{4-bos}. \textit{Определим 
последовательность $b\hm= \{b_t\}_{t\in \mathbf{Z}^n}$ равенством $b_t\hm= 
\prod\nolimits^n_{i=1} A_{t_i}$, $t\hm= (t_1\cdots t_s)^{\mathrm{T}} \hm\in \mathbf{Z}^n$. Пусть 
оператор~$U$ функции $v: K_j\hm\to \mathbf{R}$ ставит в~соответствие 
функцию $Uv: K_{j+1}\hm\to \mathbf{R}$, определенную равенством $(Uv)_k\hm= 
v_k$, $\forall\,k\hm\in K_j$, и~$(Uv)_m\hm=0$, $\forall\,m\hm\in M_j$. Тогда если 
по\-сле\-до\-ва\-тель\-ность~$v_{j+p}$ получается по схеме подразделений 
\begin{align*}
v_{j+p}&= b* (Uv_{j+p-1}),\enskip p= 1,2,\ldots;
\\
v_{j,\beta} &= \delta_{\alpha,\beta} =
\begin{cases}
1\,, & \alpha=\beta\,;\\
0\,, & \alpha\not= \beta\,,
\end{cases}
\end{align*}
где $*$ обозначает свертку, то значения мас\-шта\-би\-ру\-ющих функций и~их 
част\-ных производных мож\-но найти из при\-бли\-жен\-ных равенств}:

\noindent
\begin{multline*}
\!\varphi_{j,\alpha}\left( 2^{-j-p}\beta\right) \approx v_{j+p,\beta};\enskip \partial^{(l_1,\ldots 
,l_n)} \varphi_{j,\alpha} \left( 2^{-j-p}\beta\right)\approx{}\\[3pt]
{}\approx
\left( \left( d^{l_1}\otimes \cdots \otimes d^{l_n}\right) * \left( \left( a\otimes \cdots 
\otimes a\right) * v_{j+p}\right)\right)_\beta\,,\\[3pt]
l_1,\ldots , l_n=0,1,2\,,\enskip l_1+\cdots+ l_n\leq 2\,.
\end{multline*}

\textit{Вейвлеты могут быть найдены по формуле
$$
\psi_{j,m}= \varphi_{j+1,m}- \sum\limits_{k\in K_j} s_{j,k,m} \varphi_{j,k},
$$
где 
по\-сле\-до\-ва\-тель\-ность~$s_{j,k,m}$ мож\-но выбрать произвольным образом.}

\smallskip

  Выбором последовательностей~$A$ и~$s$ мож\-но задавать свойства 
мас\-шта\-би\-ру\-ющих функций и~вейв\-ле\-тов. Например, можно обнулить их значения 
в~заданной об\-ласти и~на ее границе. На рис.~4 представлен пример, где 
мас\-шта\-би\-ру\-ющие функции, соответствующие узлам $k\hm\in K_j$, 
принадлежащим заданной об\-ласти, обнуляются за пределами этой об\-ласти.

   \begin{figure*}[b] %fig5
\vspace*{6pt}
  \begin{center}  
    \mbox{%
\epsfxsize=143.174mm
\epsfbox{bos-5.eps}
}

\end{center}
\vspace*{-9pt}
\Caption{Триангулируемое пространство $(T,G,X)$~(\textit{а} и~\textit{б}), 
подразделения~$X$~(\textit{в} и~\textit{г}), графики точ\-но\-го решения~$\sigma_1$~(\textit{д}) 
и~его приближения~$\sigma_{1,J}$~(\textit{е})}
\end{figure*}

  Пусть $(T,G,X)$~--- триангулируемое про\-стран\-ст\-во с~конечным множеством 
симплексов. Здесь~$T\hm= \cup^N_{l=1} I_l^n \hm\subset I_0^n\hm\subset 
\mathbf{R}^n$~--- объединение замкнутых $n$-мер\-ных кубов вида $I_l^n\hm= 
\prod^n_{i=1} [ b_{i,l}; b_{i,l}\hm+1]$, где $b_{i,l}\hm\in \mathbf{Z}$, 
$I_0^s$~--- $n$-мер\-ный куб; $G: T\hm\to X\hm\subset \mathbf{R}^n$~--- 
гомеоморфизм, $G: \mathrm{Int}\,(T) \hm \to \mathrm{Int}\,(X)$~--- диффеоморфизм класса~$C^2$, где 
$\mathrm{Int}(X)$~--- внут\-рен\-ность множества~$X$. Пусть $\{\varphi_{j,k}\}_{k\in K_j}$ 
и~$\{\psi_{j,m}\}_{m\in M_j}$~--- масштабирующие функции и~вейвлеты на~$T$. 
Определим мас\-шта\-би\-ру\-ющие функции и~вейвлеты на~$X$ сле\-ду\-ющи\-ми 
равенствами: 
$$
\varphi^X_{j,k} = \varphi_{j,k}\circ G^{-1};\quad
\psi^X_{j,m}=\psi_{j,m}\circ G^{-1},
$$

\columnbreak

\noindent
 где~$\circ$ обозначает композицию функций, т.\,е.\ 
$\varphi_{j,k}\circ G^{-1}(x)\hm= \varphi_{j,k}(G^{-1}(x))$. Тогда если $f: X\hm\to 
\mathbf{R}$ и~$f\circ G\hm\in L_2(T)$, то
$$
f= \sum\limits_{k\in K_{j_0}} 
v_{j_0,k} \varphi^X_{j_0,k}+\sum\limits_{j\geq j_0} \sum\limits_{m\in M_j} 
\gamma_{j,m} \psi^X_{j,m}
$$ 
в~том смыс\-ле, что

\vspace*{-6pt}

\noindent
  \begin{multline*}
  \lim\limits_{J\to +\infty} \int\limits_T \left[ \left(
  \vphantom{\sum\limits^J_{j=j_0}}
   f- \sum\limits_{k\in K_{j_0}} 
v_{j_0,k} \varphi^X_{j_0,k} - \right.\right.\\
{}-\left.\left.\sum\limits^J_{j=j_0} \sum\limits_{m\in M_j} 
\gamma_{j,m} \psi^X_{j,m}\right) \circ G(u)\right]^2 du=0\,.
  \end{multline*}
  
 
\pagebreak
  
  На практике отображение $G$ выписать слож\-но, поэтому используется его 
аппроксимация по точечным соответствиям. Методика по\-стро\-ения этой 
аппроксимации с~по\-мощью ло\-каль\-но-ап\-прок\-си\-ма\-ци\-он\-ных сплайнов 
пред\-став\-ле\-на в~\cite{7-bos}.
  
  Из методов приближенного решения краевых задач математической физики 
в~ре\-ша\-емой задаче наиболее удобен метод наименьших  
квад\-ра\-тов~\cite{10-bos}. Рассмотрим дифференциальное уравнение и~краевые 
условия $Lw\hm= f$ на~$X$ и~$L_i w\hm= f_i$ на $\partial X$, $i\hm=1,2,\ldots , q$, 
в~гильбертовом пространстве~$L_2(X)$, где $L$~--- линейный 
дифференциальный оператор. Пусть $\{V_j\}_{j\geq 0}$~---  
крат\-но\-мас\-штаб\-ный анализ на~$X$. При\-бли\-жен\-ные решения~$w_j$ данной 
кра\-евой задачи будем искать в~виде:
  \begin{multline}
  w_j={}\\
  \!{}= \!\!\sum\limits_{k\in K_0} \!v_{0,k} \varphi^X_{0,k} +\sum\limits_{j=0}^{J-1} 
\sum\limits_{m\in M_j} \!\gamma_{j,m} \psi^X_{j,m} = \!\sum\limits_{k=1}^{M(J)}\! c_k 
\omega_k,\!\!
  \label{e4-bos}
  \end{multline}
где $M(J)$~--- число базисных функций в~$V_J$ и~для удобства базисные 
функции пронумерованы одним индексом и~обозначены~$\omega_k$, 
а~коэффициенты~$v_{0,n}$ и~$\gamma_{j,m}$ обозначены~$c_k$ и~находятся 
методом наименьших квад\-ра\-тов из решения вариационной задач 
$$
w_J=\argmin\limits_{w\in V_J} F_J(w).
$$
 Функционал~$F_J(w)$ определяется 
равенством 
$$
F_J(w)= \| Lw - f\|^2 + \sum\limits^q_{i=1} a_i \| Lw-  f_i\|^2,
$$
где $a_i$~--- положительные весовые коэффициенты. С~учетом того, что при 
построении вейвлетов есть воз\-мож\-ность обнулить часть базисных функций 
в~заданной об\-ласти, мож\-но часть коэффициентов разложения~(\ref{e4-bos}) найти 
из граничных условий. В~этом случае 
$$
w_j= \sum\limits_{k\in \mathrm{Int}\,X} c_k \omega_k + \sum\limits_{t\in \partial X} c_t \omega_t,
$$
 а остальные 
коэффициенты уже находятся из задачи минимизации 
$$
\left\| \sum\limits_{k\in \mathrm{Int}\,X} 
c_k L\omega_k- \left(f- \sum\limits_{t\in \partial X} c_t \omega_t\right)\right\|^2\hm\to 
\min\limits_{c_k}.
$$
  
  
  \smallskip
  
  \noindent
  \textbf{Пример.} Рассмотрим плас\-ти\-ну 
  $$
  X= \left\{ (x_1, x_2): a^2\leq  x_1^2+ x_2^2\leq b^2\right\}
$$ 
в~условиях рас\-тя\-же\-ния/сжа\-тия. Пусть 
модуль Юнга $E\hm= 2{,}1$~МПа; коэффициент Пуассона $\mu\hm= 0{,}3$. 
Размеры пластины: $a\hm= 20$~м; $b\hm=60$~м. Ин\-тен\-сив\-ность нормальной 
к~границе плас\-ти\-ны нагрузки имеет вид: 
$$
F_n= 2\left(1+ \fr{1}{a^2}\right)n;\enskip F_n=2\left(1+ \fr{1}{b^2}\right)n.
$$
 С~использованием биортогональных вейв\-ле\-тов были 
рассчитаны компоненты на\-пря\-же\-ний. Сравним их с~точными: 
\begin{gather*}
\sigma_1=2+ \fr{2}{r^2}\cos(2\theta);\enskip \sigma_2= 2- \fr{2}{r^2}\cos(2\theta);\\ 
\tau_{12}= \fr{2}{r^2}\sin(2\theta).
\end{gather*}
  
  Множества $T$ и~$X$ показаны на рис.~5,\,\textit{а} и~5,\,\textit{б}, 
подразделения~$X$~--- на рис.~5,\,\textit{в} и~5,\,\textit{г}. Погрешности 
$\max_{x_i}\vert \sigma_1(x_i) \hm- \sigma_{1,J}(x_i)\vert$, $\max_{x_i}\vert 
\sigma_2(x_i)\hm- \sigma_{2,J}(x_i)\vert$ и~$\max_{x_i}\vert \tau_{12}(x_i)\hm- 
\tau_{12,J} (x_i)\vert$ не превышают соответственно 0,0009, 0,000955 
и~0,00025~МПа. На рис.~5,\,\textit{д} и~5,\,\textit{е} пред\-став\-ле\-ны графики точ\-но\-го 
решения~$\sigma_1$  и~его при\-бли\-же\-ния~$\sigma_{1,J}$.
  


  
  В заключение отметим, что предложенные методики реализованы  
в~CAD/CAE-сис\-те\-ме по\-стро\-ения таких конструкций, написанной 
с~использованием языка программирования Python.
  
{\small\frenchspacing
 {%\baselineskip=10.8pt
 %\addcontentsline{toc}{section}{References}
 \begin{thebibliography}{99}
\bibitem{1-bos}
Matsuzaki Laboratory. {\sf http://www.rs.tus.ac.jp/\linebreak rmatsuza}.
\bibitem{2-bos}
\Au{Васильев В.\,В.} Механика конструкций из композиционных материалов.~--- 
М.: Машиностроение, 1988. 272~с.
\bibitem{3-bos}
\Au{Лаврентьев М.\,А., Шабат~Б.\,В.} Методы тео\-рии функций комплексного 
переменного.~--- М.: Наука, 1973. 736~с.
\bibitem{4-bos}
\Au{Cavaretta A.\,S., Dahmen~W., Micchelli~C.\,A.} Stationary subdivision 
schemes~// Mem. Am. Math. Soc., 1991. Vol.~93. No.\,453. P.~1--186.
\bibitem{5-bos}
\Au{Schroder P., Sweldens~W.} Spherical wavelets: Efficiently representing functions 
on the sphere~// 22nd Annual Conference on Computer Graphics and Interactive 
Techniques Proceedings~/ Eds. S.\,G.~Mair, R.~Cook.~---
New York, NY, USA: Association for Computing Machinery, 1995. P.~161--172.
\bibitem{6-bos}
\Au{Storn R., Price~K.} Differential evolution~--- a~simple and efficient heuristic for 
global optimization over continuous spaces~// J.~Global Optim., 1997. No.\,11. 
P.~341--359.
\bibitem{7-bos}
\Au{Deniskina G.\,Y., Deniskin~Y.\,I., Bityukov~Y.\,I.} About biortogonal wavelets, 
created on the basis of scheme of increasing of lazy wavelets~// Adv.  Automation~II: 
Rus. Auto Conf. 2020.~--- Lecture notes in electrical engineering ser.~--- Cham: 
Springer, 2021.Vol.~729. P.~173--181. doi: 10.1007/978-3-030-71119-1\_18.
\bibitem{8-bos}
\Au{Новиков~И.\,Я., Протасов~В.\,Ю., Скопина~М.\,А.} Тео\-рия всплес\-ков.~--- М.: 
Физматлит, 2005. 612~c.
\bibitem{9-bos}
\Au{Deniskina G.\,Y., Deniskin~Y.\,I., Bityukov~Y.\,I.} About some computational 
algorithms for locally approximation splines, based on the wavelet transformation and 
convolution~// Advances in automation~II~/
Eds. A.\,A.~Radionov, V.\,R.~Gasiyarov.~--- Lecture notes in 
electrical engineering ser.~--- Cham, Switzerland: Springer, 2021. Vol.~729. P.~182--191. doi: 
10.1007/978-3-030-71119-1\_19.
\bibitem{10-bos}
\Au{Марчук Г.\,И., Акилов~Г.\,П.} Методы вы\-чис\-ли\-тель\-ной математики.~--- М.: Наука, 1989. 744~с.

\end{thebibliography}

 }
 }

\end{multicols}

\vspace*{-6pt}

\hfill{\small\textit{Поступила в~редакцию 27.09.21}}

\vspace*{8pt}

%\pagebreak

%\newpage

%\vspace*{-28pt}

\hrule

\vspace*{2pt}

\hrule

%\vspace*{-2pt}

\def\tit{ABOUT SEARCHING FOR~THE~OPTIMAL 3D~PRINTING SCHEME OF~STRUCTURES 
FROM~COMPOSITE MATERIALS}


\def\titkol{About searching for~the~optimal 3D~printing scheme of~structures 
from~composite materials}


\def\aut{A.\,V.~Bosov$^{1,2}$, Yu.\,I.~Bityukov$^2$, and~G.\,Yu.~Deniskina$^2$}

\def\autkol{A.\,V.~Bosov, Yu.\,I.~Bityukov, and~G.\,Yu.~Deniskina}

\titel{\tit}{\aut}{\autkol}{\titkol}

\vspace*{-11pt}


\noindent
$^1$Federal Research Center ``Computer Science and Control'' of the Russian Academy of Sciences,  
44-2~Vavilov\linebreak
$\hphantom{^1}$Str., Moscow 119333, Russian Federation

\noindent
$^2$Moscow State Aviation Institute (National Research University), 4~Volokolamskoe Shosse, 
Moscow 125933,\linebreak
$\hphantom{^1}$Russian Federation

\def\leftfootline{\small{\textbf{\thepage}
\hfill INFORMATIKA I EE PRIMENENIYA~--- INFORMATICS AND
APPLICATIONS\ \ \ 2022\ \ \ volume~16\ \ \ issue\ 1}
}%
 \def\rightfootline{\small{INFORMATIKA I EE PRIMENENIYA~---
INFORMATICS AND APPLICATIONS\ \ \ 2022\ \ \ volume~16\ \ \ issue\ 1
\hfill \textbf{\thepage}}}

\vspace*{3pt} 



\Abste{The article is devoted to finding optimal fiber laying trajectories in the manufacture of 
structures reinforced with continuous fibers by 3D printing. As the objective function of this 
optimization problem, one of the criteria for the destruction of the composite is selected. Fiber laying 
schemes during printing are modeled using analytical functions that are found from the Neumann 
problem for the Laplace equation. The boundary conditions for this problem are constructed on the 
basis of specifying the angles between the fibers and the boundary of the region. The Neumann 
problem is solved by a~conformal transformation of the print area into a circle. Thus, the criterion for 
the destruction of the composite becomes a~function of the angles that the fibers form with the 
boundary of the region. Minimization is carried out using a~genetic algorithm for searching for the 
global minimum of a~function of several variables.}

\KWE{composite materials; wavelets; 3D printing; analytical function}

\DOI{10.14357/19922264220102}

%\vspace*{-16pt}

%\Ack
%\noindent




%\vspace*{6pt}

  \begin{multicols}{2}

\renewcommand{\bibname}{\protect\rmfamily References}
%\renewcommand{\bibname}{\large\protect\rm References}

{\small\frenchspacing
 {%\baselineskip=10.8pt
 \addcontentsline{toc}{section}{References}
 \begin{thebibliography}{99}
\bibitem{1-bos-1}
 Matsuzaki Laboratory. Available at: {\sf http://www.rs.tus. ac.jp/rmatsuza/} (accessed December~20, 
2021).
\bibitem{2-bos-1}
\Aue{Vasiliev, V.\,V.} 1988. \textit{Mekhanika konstruktsiy iz kom\-po\-zi\-tsi\-on\-nykh materialov} 
[Mechanics of structures made of composite materials]. Moscow: Mashinostroenie. 272~p.
\bibitem{3-bos-1}
\Aue{Lavrent'ev, M.\,A., and B.\,V.~Shabat.} 1973. \textit{Metody teorii funktsii kompleksnogo 
peremennogo} [Methods of the theory of functions of a~complex variable]. Moscow: Nauka. 736~p.
\bibitem{4-bos-1}
\Aue{Cavaretta,~A.\,S., W.~Dahmen, and C.\,A.~Micchelli.} 1991. Stationary subdivision schemes. 
\textit{Mem. Amer. Math. Soc.} 93(453):1--186.
\bibitem{5-bos-1}
\Aue{Schroder, P., and W.~Sweldens.} 1995. Spherical wavelets: Efficiently representing functions on 
the sphere. \textit{22nd Annual Conference on Computer Graphics and Interactive Techniques 
Proceedings}. Eds. S.\,G.~Mair and R.~Cook. New York, NY: Association for Computing Machinery. 161--172.
\bibitem{6-bos-1}
\Aue{Storn, R., and K.~Price.} 1997. Differential evolution~--- a~simple and efficient heuristic for 
global optimization over continuous spaces. \textit{J.~Global Optim.} 11:341--359.
\bibitem{7-bos-1}
\Aue{Deniskina, G.\,Y., Y.\,I.~Deniskin, and Y.\,I.~Bityukov.} 2021. About biortogonal wavelets, 
created on the basis of scheme of increasing of lazy wavelets. \textit{Advances in Automation~II.} 
Lecture notes in electrical engineering ser. Cham: Springer. 729:173--181. 
\bibitem{8-bos-1}
\Aue{Novikov, I.\,Ya., V.\,Yu.~Protasov, and M.\,A.~Skopina.} 2005. \textit{Teoriya vspleskov} [The 
theory of wavelets]. Moscow: Fizmatlit. 612~p.
\bibitem{9-bos-1}
\Aue{Deniskina, G.\,Y., Y.\,I.~Deniskin, and Y.\,I.~Bityukov.} 2021. About some computational 
algorithms for locally approximation splines, based on the wavelet transformation and convolution. 
\textit{Advances in automation~II.}
Eds. A.\,A.~Radionov and V.\,R.~Gasiyarov. Lecture notes in electrical engineering ser. Cham,
Switzerland: Springer. 
729:182--191.  doi: 
10.1007/978-3-030-71119-1\_19.
\bibitem{10-bos-1}
\Aue{Marchuk, G.\,I., and G.\,P.~Akilov.} 1989. \textit{Metody vychislitel'noy matematiki} [Methods 
of computational mathematics]. Moscow: Nauka. 744~p.
\end{thebibliography}

 }
 }

\end{multicols}

\vspace*{-6pt}

\hfill{\small\textit{Received September 27, 2021}}

%\pagebreak

%\vspace*{-18pt}

\Contr


\noindent
\textbf{Bosov Alexey V.} (b.\ 1969)~--- Doctor of Science in technology, principal scientist, Institute 
of Informatics Problems, Federal Research Center ``Computer Science and Control'' of the Russian 
Academy of Sciences, 44-2~Vavilov Str., Moscow 119333, Russian Federation; professor, Moscow 
State Aviation Institute (National Research University), 4~Volokolamskoe Shosse, Moscow 125933, 
Russian Federation; \mbox{AVBosov@ipiran.ru}


\vspace*{3pt}

\noindent
\textbf{Bityukov Yuri I.} (b.\ 1972)~--- Doctor of Science in technology, professor, Moscow Aviation 
Institute (National Research University), 4~Volokolamskoe Shosse, Moscow 125933, Russian 
Federation; \mbox{yib72@mail.ru}


\vspace*{3pt}

\noindent
\textbf{Deniskina Galina Yu.} (b.\ 1993)~--- PhD student, Faculty of Information Technologies and 
Applied Mathematics, Moscow Aviation Institute (National Research University), 4~Volokolamskoe 
Shosse, Moscow 125933, Russian Federation; \mbox{dega17@yandex.ru}

\label{end\stat}

\renewcommand{\bibname}{\protect\rm Литература}    %2
\def\stat{andrianova}

\def\tit{КОНТЕКСТНЫЙ ПОИСК НА ФОТОНАХ С~ИСПОЛЬЗОВАНИЕМ~ТЕСТОВ~БЕЛЛА$^*$}

\def\titkol{Контекстный поиск на фотонах с~использованием тестов Белла}

\def\aut{\fbox{С.\,Н.~Андрианов}$^1$, Н.\,С.~Андрианова$^2$, Ф.\,М.~Аблаев$^3$, 
Ю.\,Ю.~Кочнева$^4$}

\def\autkol{С.\,Н.~Андрианов, Н.\,С.~Андрианова, Ф.\,М.~Аблаев, 
Ю.\,Ю.~Кочнева}

\titel{\tit}{\aut}{\autkol}{\titkol}

\index{Андрианов С.\,Н.}
\index{Андрианова Н.\,С.}
\index{Аблаев Ф.\,М.} 
\index{Кочнева Ю.\,Ю.}
\index{Andrianov S.\,N.}
\index{Andrianova N.\,S.}
\index{Ablaev F.\,M.}
\index{Kochneva Yu.\,Yu.}


{\renewcommand{\thefootnote}{\fnsymbol{footnote}} \footnotetext[1]
{Работа Ф.\,М.~Аблаева выполнена за счет средств субсидии, выделенной Казанскому 
(Приволжскому) федеральному университету для выполнения государственного задания 
в~сфере научной деятельности, проект №\,0671-2020-0065.}}


\renewcommand{\thefootnote}{\arabic{footnote}}
\footnotetext[1]{Институт прикладных исследований, Академия наук Республики Татарстан}
\footnotetext[2]{Казанский (Приволжский) федеральный университет, \mbox{natalia\_an83@mail.ru}}
\footnotetext[3]{Казанский (Приволжский) федеральный университет, \mbox{fablayev@gmail.com}}
\footnotetext[4]{Институт прикладных исследований, Академия наук Республики Татарстан,
\mbox{instpianrt@gmail.com}}

\vspace*{-6pt}




     
     \Abst{Рассмотрена возможность конкретной физической реализации контекстного 
поиска на квантовых состояниях с использованием тестов Белла, который рассматривался 
ранее лишь как абстрактная математическая процедура. Для этого предложено использовать 
контекстную кодировку слов в документах на поляризационных фотонных кубитах. 
Получены конкретизированные аналитические выражения для определения на основе тестов 
Белла параметра контекстного поиска по паре слов, которые могут быть связанными или нет 
в зависимости от значения этого параметра. Наибольшей связанности отвечает состояние 
квантовой перепутанности волновых функций документов по паре выбранных слов, 
которому соответствует определенное значение параметра контекстного поиска. 
Предложенные способы реализации семантического контекстного поиска необходимы для 
определения нелокальной контекстности, которая часто требуется при автоматизированном 
поиске и машинном переводе. При этом второе слово в паре поисковых слов поясняет смысл 
первого через их семантическую связь.}
     
     \KW{гипоним; гипероним; изотопия; родовидовая связь; фотонный кубит; 
перепутанные состояния; белловский тест; голографический процессор}

\DOI{10.14357/19922264220103}
  
\vspace*{-4pt}


\vskip 10pt plus 9pt minus 6pt

\thispagestyle{headings}

\begin{multicols}{2}

\label{st\stat}
     
    В~рамках познавательной дея\-тель\-ности человека информация 
структурируется с точ\-ки зрения ее значения и организации, постепенно 
пре\-вра\-ща\-ясь в знание. Фор\-ма\-ли\-зу\-емые на естественном языке знания 
систематизируются в рамках лексической сис\-те\-мы языка в толковых, 
семантических, идеографических словарях, а~так\-же в~циф\-ро\-вых сетях, 
вклю\-чая квантовые.
    
    Формализация знаний с по\-мощью концептуальной схемы может 
осуществляться в~виде лексических онтологий, например WordNet. Этот 
ресурс является одним из способов пред\-став\-ле\-ния знаний на основе 
установления семантических отношений меж\-ду понятиями с по\-мощью 
составления синонимических рядов (синсетов) со\-от\-вет\-ст\-ву\-ющей час\-ти речи, 
которые, в свою очередь, так\-же связаны меж\-ду собой разнообразными 
семантическими отношениями (гиперонимия, меронимия, антонимия 
и~т.\,д.)~\cite{4-an}.
    
    От логической модели построения баз данных отличается подход, 
представленный методом HAL (Hyperspace Analogue to Language). При 
построении этого языка считается, что любой языковой знак находится 
в~контексте, а также учитывается линейный характер текста и бли\-зость слов 
друг другу в его линейной развертке. Речь идет о построении базы данных 
связей слов с учетом их непосредственного словесного окружения~\cite{5-an}. 
Авторы этой работы вводят так называемое семантическое пространство, 
в~котором смысл словосочетаний отображается при помощи специального 
языка HAL.
    
    В языке HAL слова распределяются не так, как в~словаре обычного языка--- 
последовательно, например просто в алфавитном порядке, а~в~виде\linebreak  
мат\-ри\-цы (таблицы), где слова распределены по вертикали в~самом левом 
столб\-це по алфавиту, как и~в~обычном словаре. Обычный словарь в~этом 
смыс\-ле является аналогом векторного пространства. В~языке HAL в~строках 
каждое слово размещается в пар\-ных сочетаниях с~другими словами, что должно 
отражать ка\-кую-то связь этого отдельного слова с~остальными словами. Будучи 
расположенными в мат\-ри\-це, слова в~языке HAL имеют как бы тензорный 
характер, т.\,е.\ относятся к некоему гиперпространству. При этом слова 
в~мат\-ри\-це располагаются тем ближе, чем они дальше друг от друга  
в~рас\-смат\-ри\-ва\-емом текс\-те. Аналогия с языком в методе HAL заключается 
в~том, что каж\-дой точке гиперпространства можно сопоставить то или иное 
словосочетание языка рас\-смат\-ри\-ва\-емо\-го текста.
    
    Таким путем эмпирический подход языка HAL позволяет выявить 
взаимосвязь слов. Но эта связь может быть и чисто формальной, следствием 
случайных совпадений. Если такая связь является изотопической, то она имеет 
смысловой характер. 
    
    Авторы работы~\cite{6-an} такую связь между парой слов в том или ином 
текс\-те предложили искать целенаправленно, используя квантовые алгоритмы 
при записи слов в текс\-те при помощи языка HAL. При этом квантовые 
алгоритмы реализовывались чис\-то математически с использованием известных 
в~квантовой механике абстрактных формул. В~интерпретации языка HAL 
в~работе~\cite{6-an} квантовое векторное со\-сто\-яние документа определяется 
как $\vert \Psi\rangle \hm= \sum\nolimits_i^N \vert w_{n_i}\rangle$, где $\vert 
w_{ni}\rangle$~--- $i$-е собственное со\-сто\-яние оператора некоторой 
величины~$n$, со\-от\-вет\-ст\-ву\-ющее слову~$i$, т.\,е.\ это вектор, яв\-ля\-ющий\-ся 
суммой векторов отдельных слов.
    
    Рассмотрим возможность физической реализации такого подхода 
с~использованием конкретных фотонных квантовых со\-сто\-яний $\vert 
w_{v_i}\rangle\hm=\sum\nolimits_j^M \vert w_{k_{ij}}\rangle$ как квантовых 
со\-сто\-яний слова~$i$, которые характеризуются час\-то\-той фотона~$v_i$ и его 
вол\-но\-вым вектором~$k_{ij}$. Можно представить эти со\-сто\-яния как сумму 
проекций нормированного со\-сто\-яния фотона $\vert u_{v_{ij}}\rangle \hm= 
a^{(i)}_{\vec{k}_j} \vert u_{\vec{k}_j}\rangle$  на базовые со\-сто\-яния 
на\-прав\-ле\-ний его волнового век\-то\-ра, лежащих в той или иной плос\-кости: 
$$
\vert  u_{v_i}\rangle = \sum\limits_j^M a^{(i)}_{\vec{k}_j} \vert 
u_{\vec{k}_j}\rangle.
$$
    
    Двум словам~$A$ и~$B$ из текста можно сопоставить векторные 
состояния $\vert u_{v_A}\rangle$ и~$\vert u_{v_B}\rangle$  и~об\-щую плос\-кость, 
проходящую через эти векторы. Тогда векторные со\-сто\-яния документа в базисе 
этих слов мож\-но определить как векторную сумму проекций состояния 
документа на со\-сто\-яния этих слов в данной плос\-кости с последующим 
поворотом на~90$^\circ$ относительно ортогональных к векторам слов осей 
и~проекции на оси в плос\-кости, ортогональной векторам слов:
    \begin{equation}
    \left.
    \begin{array}{rl}
    \vert \Psi_A\rangle &= \fr{1}{\sqrt{2}}\left( \alpha_{\sigma_+}\vert 
u_{v_A,\sigma_+} \rangle +\alpha_{\sigma_-}\vert u_{v_A,\sigma_-}\rangle\right);\\[6pt]
%    \label{e1-an}
    \vert \Psi_B\rangle &= \fr{1}{\sqrt{2}}\left( \beta_{\sigma_+}\vert 
u_{v_B,\sigma_+} \rangle +\beta_{\sigma_-}\vert u_{v_B,\sigma_-}\rangle\right).
\end{array}
\right\}
    \label{e2-an}
    \end{equation}
    
    Векторы~(\ref{e2-an})  являются по своей фор\-ме 
поляризационными фотонными кубитами.
    
    Коэффициенты в выражениях~(\ref{e2-an}) мож\-но записать 
как
    \begin{equation}
    \left.
    \begin{array}{rl}
    \alpha_{\sigma_+} &= \fr{\langle u_{v_{A,\sigma_+}}\vert \Psi\rangle} 
{\sqrt{\langle u_{v_{A,\sigma_+}}\vert \Psi\rangle^2}+\langle u_{v_{A,\sigma_-
}}\vert \Psi\rangle^2}\,; %\label{e3-an}
\\[6pt]
    \alpha_{\sigma_-} &= \fr{\langle u_{v_{A,\sigma_-}}\vert \Psi\rangle} 
{\sqrt{\langle u_{v_{A,\sigma_+}}\vert \Psi\rangle^2}+\langle u_{v_{A,\sigma_-
}}\vert \Psi\rangle^2}\,;
\end{array}
\right\}
\label{e4-an}
\end{equation}
\begin{equation}
\left.
\begin{array}{rl}
    \beta_{\sigma_+} &= \fr{\langle u_{v_B,\sigma_+}\vert \Psi\rangle} 
{\sqrt{\langle u_{v_B,\sigma_+}\vert \Psi\rangle^2}+\langle u_{v_B,\sigma_-}\vert 
\Psi\rangle^2}\,;\\[6pt]
%    \label{e5-an}\\
    \beta_{\sigma_-} &= \fr{\langle u_{v_B,\sigma_-}\vert \Psi\rangle} 
{\sqrt{\langle u_{v_B,\sigma_+}\vert \Psi\rangle^2}+\langle u_{v_B,\sigma_-}\vert 
\Psi\rangle^2}\,.
\end{array}
\right\}
    \label{e6-an}
    \end{equation}
    
    Состояния вида~(\ref{e2-an}) позволяют связать каж\-дое 
слово с тем или иным квантовым битом (кубитом) информации. Определим 
теперь операторы запроса контекстного поиска на фотонах с учетом того, что 
поляризации фотонов можно ассоциировать с их спиновыми состояниями. 
Оператор прямого значения слова соответствует оператору $z$-про\-ек\-ции 
спина: 
    \begin{align*}
    \hat{A}\vert \Psi_A\rangle &= \hat{S}_{Az} \vert \Psi_A\rangle 
={}\notag\\
&{}=\fr{1}{\sqrt{2}}\left( \alpha_{\sigma_+} \vert u_{v_A,\sigma_+}\rangle -
\alpha_{\sigma_-} \vert u_{v_A,\sigma_-}\rangle\right);
   % \label{e7-an}
   \\
    \hat{B}\vert \Psi_B\rangle &= \hat{S}_{Bz} \vert \Psi_B\rangle 
={}\notag\\
&{}=\fr{1}{\sqrt{2}}\left( \beta_{\sigma_+} \vert u_{v_B,\sigma_+}\rangle -
\beta_{\sigma_-} \vert u_{v_B,\sigma_-}\rangle\right).
  %  \label{e8-an}
    \end{align*}
    
    Оператор запроса противоположного значения можно определить как
    \begin{multline*}
    \hat{A}_x\vert \Psi_A\rangle = \hat{S}_{Ax} \vert \Psi_A\rangle 
={}\\
{}= \fr{1}{\sqrt{2}}\left( \alpha_{\sigma_-}\vert u_{v_A,\sigma_+}\rangle 
+\alpha_{\sigma_+}\vert u_{v_B,\sigma_-}\rangle\right).
  %  \label{e9-an}
    \end{multline*}
    
    Белловский параметр поиска можно записать через матричные элементы 
операторов запроса по известной из работы~\cite{6-an} формуле:
    \begin{multline*}
    S_{\mathrm{query}}=\left\vert \left \langle \hat{A}\hat{B}_+\right\rangle_\Psi 
+\left\langle \hat{A}_x\hat{B}_+\right\rangle_\Psi\right\vert +{}\\
{}+
    \left\vert \left\langle \hat{A}\hat{B}_-\right\rangle_\Psi -\left\langle 
\hat{A}_x\hat{B}_-\right\rangle_\Psi\right\vert\,,
    %\label{e10-an}
    \end{multline*}
где $\hat{B}_+= -(\hat{B}+\hat{B}_x)$; $\hat{B}_-\hm= \hat{B}\hm- \hat{B}_x$.
    
    Простое вычисление дает
    \begin{multline}
    S_{\mathrm{query}}= \fr{1}{2} \left\{ \left\vert \left( \alpha^2_{\sigma_+} 
+2\alpha_{\sigma_+} \alpha_{\sigma_-}-\alpha^2_{\sigma_-}\right)\right\vert 
+{}\right.\\
    \left.{}+
    \left\vert \left( \alpha^2_{\sigma_+} -2\alpha_{\sigma_+}\alpha_{\sigma_-} - 
\alpha^2_{\sigma_-}\right)\right\vert \right\} 
    \left\vert \left(\beta^2_{\sigma_+} +{}\right.\right.\\
   \left.\left. {}+ 2\beta_{\sigma_+} \beta_{\sigma_-} -
\beta^2_{\sigma_-}\right)\right\vert\,.
    \label{e11-an}
    \end{multline}
                
    
    Вычислив коэффициенты $\alpha_{\sigma_+}$, $\alpha_{\sigma_-}$, 
$\beta_{\sigma_+}$ и~$\beta_{\sigma_-}$ по  
формулам~(\ref{e4-an}) и~(\ref{e6-an}), можно установить значения слов по 
матричным элементам этих операторов. Также можно вычислить белловский 
параметр~$S_{\mathrm{query}}$, величина которого определяет степень пе\-ре\-пу\-тан\-ности 
состояний документа по словам~$A$ и~$B$. Перепутанность состояний 
означает, что со\-сто\-яния связаны между собой путем взаимодействия через 
ка\-кие-ли\-бо другие со\-сто\-яния. Поэтому таким путем мож\-но установить наличие 
смысловой связи между выбранными словами в~этом до\-ку\-менте. 
    
    В работе~\cite{6-an} такие вычисления проведены на обычном 
компьютере. Но можно и по\-стро\-ить автономное вы\-чис\-ли\-тель\-ное устройство, 
работающее, например, на час\-ти\-цах света~--- фотонах. Такое устройство будет 
обладать повышенным быст\-ро\-дей\-ст\-ви\-ем как за счет предельно высокой 
ско\-рости и безынер\-ци\-он\-ности фотонов, так и за счет кван\-то\-вой па\-рал\-лель\-ности 
ис\-поль\-зу\-емых алгоритмов.
    
    В этом устройстве можно определить коэффициенты в состояниях 
документа, вычисляя скалярные произведения  
в~формулах~(\ref{e4-an}) и~(\ref{e6-an}) при помощи классического оптического 
процессора. Особенно удоб\-но использовать голографические 
процессоры~\cite{7-an, 8-an}. Эти процессоры позволяют записывать результат 
скалярного произведения векторов при помощи интерференции фотонов, 
а~затем получать результат с использованием счи\-ты\-ва\-юще\-го поля. После 
определения значения коэффициентов можно вы\-чис\-лить значение па\-ра\-мет\-ра~$S_{\mathrm{query}}$ 
по формуле~(\ref{e11-an}). Так\-же мож\-но методами квантовой 
информатики сгенерировать состояния двух фотонов, со\-от\-вет\-ст\-ву\-ющих 
словам~$A$ и~$B$ и провести измерение па\-ра\-мет\-ра $S_{\mathrm{query}}$ по 
стандартным схемам работ~\cite{9-an, 10-an, 11-an}. 
    
    Итак, в данной работе путем использования фотонных со\-сто\-яний найден 
конкретный способ реализации квантового алгоритма контекстного поиска, 
позволяющий искать изотопию, т.\,е.\ общий\linebreak семантический признак, 
свя\-зы\-ва\-ющий понятия. При поисковом запросе установление связи (выявление 
изотопии) меж\-ду понятиями может осуществляться в текстах различного 
характера (текс\-ты, относящиеся к одной терминологической об\-ласти; текс\-ты 
раз\-ных тематических областей). Подход работы~\cite{6-an} позволяет 
определить, относится ли текст к определенному вопросу, путем введения при 
поиске пары слов. Так, если искать информацию о~политическом скандале  
<<Иран--конт\-рас>>, то, понимая, что в это время Рейган был президентом 
Соединенных Штатов, причастных к скандалу, можно ввести при запросе пару 
слов Рей\-ган--Иран. Если параметр поиска покажет перепутанность со\-сто\-яний, 
соответствующих этим словам, то это будет означать, что рас\-смат\-ри\-ва\-емый 
текст соответствует теме запроса, т.\,е.\ задача найти нуж\-ный текст решена.
    
    Характер текстов, а также характер запроса пользователя (например, 
определение значения тер\-ми\-на-нео\-ло\-гиз\-ма посредством сравнения его 
с~термином, относящимся к той же терминологии, или поиск двух явно не 
связанных друг с~другом понятий) влияет и~на определение изотопии 
(семантической связи) между этими понятиями: гипонимия, гиперонимия 
(родовидовые отношения), метафора, антонимия. Таким образом, можно 
установить семантический признак, свя\-зы\-ва\-ющий понятия. Полученные 
данные могут быть использованы для создания как толковых словарей, так 
и~специализированных словарей (тезаурусов) в~той или иной об\-ласти 
в~зависимости от характера использованных текс\-тов. Они могут применяться 
в~сис\-те\-мах поиска~\cite{12-an, 13-an} и~сис\-те\-мах автоматизированного 
перевода~\cite{14-an, 15-an}.
    
{\small\frenchspacing
 {%\baselineskip=10.8pt
 %\addcontentsline{toc}{section}{References}
 \begin{thebibliography}{99}
%\bibitem{1-an}
%\Au{Greimas A.\,J.} S$\acute{\mbox{e}}$mantique structurale. Recherche de 
%m$\acute{\mbox{e}}$thode.~--- Paris: Larousse, 1966. 262~p.
%\bibitem{2-an}
%\Au{Rastier F.} Le d$\acute{\mbox{e}}$veloppement du concept d'isotopie~// Actes 
%S$\acute{\mbox{e}}$miotiques Documents, 1981. Vol.~3. No.\,29. 48~p.
%\bibitem{3-an}
%\Au{Величковский Б.\,М.} Когнитивная наука: Основы психологии познания: в 2~т.~--- М.: 
%Смысл; Академия, 2006. Т.~2. 432~с.
\bibitem{4-an}
\Au{Усталов Д.} Семантические сети и обработка естественного языка~// Открытые системы. 
СУБД, 2017. №\,2. С.~46--47. {\sf https://www.osp.ru/os/2017/ 02/13052229}.
\bibitem{5-an}
\Au{Lund K., Burgess~C.} Producing high-dimensional semantic spaces from lexical  
co-occurrence~// Behav. Res. Meth. Ins.~C., 1996. Vol.~28.  
P.~203--208.
\bibitem{6-an}
\Au{Barros J., Toffano~Z., Meguebli~Y., Doan B.-L.} Contextual query using bell tests~//  
Quantum interaction~~/
Eds. H.~Atmanspacher, E.~Haven, K.~Kitto, D.~Raine.~--- Lecture notes in computer 
science ser.~--- Springer, 2013. Vol.~8369. P.~110--121.
\bibitem{7-an} %4
\Au{Yariv A.} Phase conjugate optics and real-time holography~// IEEE J.~Quantum Elect., 
1978. Vol.~QE-14. No.\,9. P.~650--660.
\bibitem{8-an}
\Au{Dolev S., Fandina~N., Rosen~J.} Holographic parallel processor for calculating Kronecker 
product~// Nat. Comput., 2015. Vol.~14. P.~433--436.
\bibitem{9-an}
\Au{Clauser J.\,F., Horne~M.\,A., Shimony~A., Holt~R.\,A.} Proposed experiment to test local 
hidden-variable theories~// Phys. Rev. Lett., 1969. Vol.~23. No.\,15. P.~880--884.
\bibitem{10-an}
\Au{Freedman S.\,J., Clauser~J.\,F.} Experimental test of local hidden-variable theories~// Phys. 
Rev. Lett., 1972. Vol.~28. No.\,14. P.~938--941.
\bibitem{11-an}
\Au{Tanji H., Simon~J., Ghosh~S., Vuletic~V.} Simplified measurement of the Bell parameter 
within quantum mechanics~// arXiv.org, 2008. arXiv:0801.4549 [quant-ph].
\bibitem{12-an}
\Au{Beltran L., Geriente~S.} Quantum entanglement in corpuses of documents~// Found. 
Sci., 2019. Vol.~24. P.~227--246.
\bibitem{13-an}
\Au{Бессмертный И.\,А., Васильев~А.\,В., Королева~Ю.\,А., Платонов~А.\,В., 
Полещук~Е.\,А.} Методы квантового формализма в информационном поиске и обработке 
текстов на естественных языках~// Изв. вузов. Приборостроение, 2019. Т.~62. №\,8.  
С.~702--709.
\bibitem{14-an}
\Au{Wang C., Seneff~S.} High-quality speech-to-speech translation for computer-aided language 
learning~// ACM Transactions Speech Language Processing, 2006. Vol.~3. No.\,2. P.~1--21.
\bibitem{15-an}
\Au{Jia Ye., Weiss R.\,J., Biadsy~F., Macherey~W., Johnson~M., Chen~Z., Wu~Y.}  
Direct speech-to-speech translation with a sequence-to-sequence model~// arXiv.org, 2019. 
arXiv:\linebreak 1904.06037v2 [cs.CL].
\end{thebibliography}

 }
 }

\end{multicols}

\vspace*{-9pt}

\hfill{\small\textit{Поступила в~редакцию 12.03.20}}

\vspace*{6pt}

%\pagebreak

%\newpage

%\vspace*{-28pt}

\hrule

\vspace*{2pt}

\hrule

\vspace*{-2pt}

\def\tit{CONTEXT QUERY ON~PHOTONS WITH THE~USE OF~BELL TESTS}


\def\titkol{Context query on~photons with~the~use of~Bell tests}


\def\aut{\fbox{S.\,N.~Andrianov}$^1$, N.\,S.~Andrianova$^2$, F.\,M.~Ablaev$^2$, and~Yu.\,Yu.~Kochneva$^1$}

\def\autkol{S.\,N.~Andrianov, N.\,S.~Andrianova, F.\,M.~Ablaev, and~Yu.\,Yu.~Kochneva}

\titel{\tit}{\aut}{\autkol}{\titkol}

\vspace*{-11pt}


\noindent
$^1$Institute of Applied Research, Tatarstan Academy of Sciences, 36~Levobulachnaya Str., Kazan 420011, 
Russian\linebreak
$\hphantom{^1}$Federation

\noindent
$^2$Kazan Federal University, 18~Kremlyovskaya Str., Kazan 420008, Russian Federation

\def\leftfootline{\small{\textbf{\thepage}
\hfill INFORMATIKA I EE PRIMENENIYA~--- INFORMATICS AND
APPLICATIONS\ \ \ 2022\ \ \ volume~16\ \ \ issue\ 1}
}%
 \def\rightfootline{\small{INFORMATIKA I EE PRIMENENIYA~---
INFORMATICS AND APPLICATIONS\ \ \ 2022\ \ \ volume~16\ \ \ issue\ 1
\hfill \textbf{\thepage}}}

\vspace*{1pt} 




\Abste{The possibilities for physical realization of contextual query on photons 
in an optical processor using Bell tests are considered. To solve this problem, 
context coding of words in documents on quantum states of single photons using the 
well-known method of hyperspace analog language is proposed. Analytical expressions 
for determination of parameters of contextual query by a~pair of words that can be 
bound or not bound depending on the value of this parameter were obtained. 
Most connected is quantum entangled state of document wave functions chosen by 
a~pair of words that corresponds to a~certain value of the contextual query parameter. 
The suggested methods of realization of semantic contextual query are necessary 
for determination of nonlocal context that is demanded for acquiring better 
understanding during automated search and machine translation. 
The second word in the pair of query words clarifies the meaning of the 
first word through their semantic connection.}

\KWE{hyponym; hyperonym; isotopy; genus-species relations; 
photonic qubit; entangled state; Bell test; holographic processor}

\DOI{10.14357/19922264220103}

\vspace*{-24pt}

\Ack

\vspace*{-6pt}

\noindent
The research of F.\,M.~Ablaev was funded by the subsidy allocated to Kazan 
Federal University for the state assignment in the sphere of scientific activities, 
project No.\,0671-2020-0065.



%\vspace*{-6pt}

  \begin{multicols}{2}

\renewcommand{\bibname}{\protect\rmfamily References}
%\renewcommand{\bibname}{\large\protect\rm References}

{\small\frenchspacing
 {%\baselineskip=10.8pt
 \addcontentsline{toc}{section}{References}
 \begin{thebibliography}{99}
 
 \vspace*{-1pt}
 
%\bibitem{1-an-1}
%\Aue{Greimas, A.\,J.} 1966. \textit{S$\acute{\mbox{e}}$mantique structurale. Recherche de 
%m$\acute{\mbox{e}}$thode}. Paris: Larousse. 262~p.
%\bibitem{2-an-1}
%\Aue{Rastier, F.} 1981. Le d$\acute{\mbox{e}}$veloppement du concept d'isotopie. \textit{Actes 
%S$\acute{\mbox{e}}$miotiques Documents} 3(29):1--48.
%\bibitem{3-an-1}
%\Aue{Velichkovskii, B.\,М.} 2006. \textit{Kognitivnaya nauka: Osnovy psikhologii poznaniya} [Cognitive 
%science: Basics of knowledge psychology]. Moscow: Smysl; Akademiya. Vol.~2. 432~p.
\bibitem{4-an-1}
\Aue{Ustalov, D.} 2017. Semanticheskie seti i~obrabotka estestvennogo yazyka [Semantic nets and natural 
language processing].
\textit{Otkrytye sistemy. SUBD} [Open Systems. DBMS] 2:46--47.
 Available at: {\sf https://www.osp.ru/os/ 2017/02/13052229/} (accessed December~22, 
2021).
\bibitem{5-an-1}
\Aue{Lund, K., and C.~Burgess.} 1996. Producing high-dimensional semantic spaces from lexical  
co-occurrence. \textit{Behav. Res. Meth. Ins.~C.} 28:203--208.
\bibitem{6-an-1}
\Aue{Barros, J., Z.~Toffano, Y.~Meguebli, and B.-L.~Doan.} 2013. Contextual query using Bell tests. 
\textit{Quantum interaction}. Eds.\ H.~Atmanspacher, E.~Haven, K.~Kitto, and D.~Raine.
 Lecture notes in computer science ser. 
Springer. 8369:110--121.
\bibitem{7-an-1}
\Aue{Yariv, A.} 1978. Phase conjugate optics and real-time holography. \textit{IEEE J.~Quantum 
Elect.} QE-14(9):650--660.
\bibitem{8-an-1}
\Aue{Dolev, S., N.~Fandina, and J.~Rosen.} 2015. Holographic parallel processor for calculating Kronecker 
product. \textit{Nat. Comput.} 14:433--436.
\bibitem{9-an-1}
\Aue{Clauser, J.\,F., M.\,A.~Horne, A.~Shimony, and R.\,A.~Holt.} 1969. Proposed experiment to test local 
hidden-variable theories. \textit{Phys. Rev. Lett.} 23(15):880--884.
\bibitem{10-an-1}
\Aue{Freedman, S.\,J., and J.\,F.~Clauser.} 1972. Experimental test of local hidden-variable theories. 
\textit{Phys. Rev. Lett.} 28(14):938--941.
\bibitem{11-an-1}
\Aue{Tanji, H., J.~Simon, S.~Ghosh, and V.~Vuletic.} 2008. Simplified measurement of the Bell parameter 
within quantum mechanics. \textit{arXiv.org}. Available at: {\sf https://arxiv.org/abs/0801.4549} (accessed 
December~22, 2021).
\bibitem{12-an-1}
\Aue{Beltran, L., and S.~Geriente.} 2019. Quantum entanglement in corpuses of documents. 
\textit{Found. Sci.} 24:227--246.
\bibitem{13-an-1}
\Aue{Bessmertnyi, I.\,А., А.\,V.~Vasiliev, Yu.\,А.~Koroleva, А.\,V.~Platonov, and Е.\,А.~Poleschuk.} 
2019. Metody kvantovogo formalizma v~informatsionnom poiske i~obrabotke tekstov na estestvennykh 
yazykakh [Quantum formalism methods in information retrieval and processing of texts on natural languages]. 
\textit{Izvestiya vysshikh uchebnykh zavedeniy. Priborostroenie} [J.~Instrument Engineering]  
62(8):702--709.
\bibitem{14-an-1}
\Aue{Wang, C., and S.~Seneff.} 2006. High-quality speech-to-speech translation for computer-aided 
language learning. \textit{ACM Transactions Speech Language Processing} 3(2):1--21.
\bibitem{15-an-1}
\Aue{Jia, Ye., R.\,J.~Weiss, F.~Biadsy, W.~Macherey, M.~Johnson, Z.~Chen, and Y.~Wu.} 2019. Direct 
speech-to-speech translation with a sequence-to-sequence model. \textit{arXiv.org}. Available at: {\sf 
https://arxiv.org/abs/1904.06037} (accessed December~22, 2021).
\end{thebibliography}

 }
 }

\end{multicols}

\vspace*{-12pt}

\hfill{\small\textit{Received March 12, 2020}}

\pagebreak

%\vspace*{-18pt}

\Contr

\noindent
\textbf{Andrianov Sergey N.} (1959--2020)~--- Doctor of Science in physics and mathematics, principal 
scientist, Institute of Applied Research, Tatarstan Academy of Sciences, 36~Levobulachnaya Str., Kazan 
420011, Russian Federation

\vspace*{3pt}

\noindent
\textbf{Andrianova Nataliya S.} (b.\ 1983)~--- Candidate of Science (PhD) in philology, associate professor, 
Department of Theory and Practice of Teaching Foreign Languages, Institute of Philology and Intercultural 
Communication, Kazan Federal University, 18~Kremlyovskaya Str., Kazan 420008, Russian Federation; 
\mbox{natalia\_an83@mail.ru} 

\vspace*{3pt}

\noindent
\textbf{Ablaev Farid M.} (b.\ 1953)~--- Doctor of Science in physics and mathematics, professor, Head of 
Department of Theoretical Cybernetics, Institute of Computational Mathematics and Information 
Technologies, Kazan Federal University, 18~Kremlyovskaya Str., Kazan 420008, Russian Federation; 
\mbox{fablayev@gmail.com}

\vspace*{3pt}

\noindent
\textbf{Kochneva Yulia Yu.} (b.\ 1985)~--- scientist, Institute of Applied Research, Tatarstan Academy of 
Sciences, 36~Levobulachnaya Str., Kazan 420011, Russian Federation; \mbox{instpianrt@gmail.com}




\label{end\stat}

\renewcommand{\bibname}{\protect\rm Литература}  %3
\def\stat{malashenko}

\def\tit{ПОСЛЕДОВАТЕЛЬНЫЙ АНАЛИЗ И~МЕТРИЧЕСКИЕ ОЦЕНКИ ПРЕДЕЛЬНЫХ
РАСПРЕДЕЛЕНИЙ МЕЖУЗЛОВЫХ ПОТОКОВ В~МНОГОПОЛЬЗОВАТЕЛЬСКОЙ СЕТИ}

\def\titkol{Последовательный анализ и~метрические оценки предельных
распределений межузловых потоков в %~многопользовательской 
сети}

\def\aut{Ю.\,Е. Малашенко$^1$}

\def\autkol{Ю.\,Е. Малашенко}

\titel{\tit}{\aut}{\autkol}{\titkol}

\index{Малашенко Ю.\,Е.}
\index{Malashenko Yu.\,E.}


%{\renewcommand{\thefootnote}{\fnsymbol{footnote}} \footnotetext[1]
%{Исследование выполнено при финансовой поддержке Российского научного фонда (проект 
%<<Информатика>> ФИЦ ИУ РАН, Москва).}}


\renewcommand{\thefootnote}{\arabic{footnote}}
\footnotetext[1]{Федеральный исследовательский центр <<Информатика и~управление>> Российской академии 
\mbox{mala-yur@yandex.ru}}


%\vspace*{-6pt}



\Abst{Для оценки функциональных возможностей
многопользовательской сети связи аналилизируется множество векторов межузловых потоков при предельных распределениях ресурсов
сети. В~рамках многопродуктовой модели про\-пуск\-ные спо\-соб\-ности ребер рас\-смат\-ри\-ва\-ют\-ся 
как компоненты вектора ресурсов различных
типов, которые требуются для передачи потоков различных видов.
При проведении вычислительных экспериментов на каждой итерации вычисляются нормы векторов совместно допустимых межузловых
потоков, при передаче которых полностью используется пропускная спо\-соб\-ность всех ребер сети. Полученные последовательности
метрических оценок позволяют анализировать особенности множества до\-сти\-жи\-мости и~эф\-фек\-тив\-ность использования ресурсов сети при
уравнительном распределении про\-пуск\-ной спо\-соб\-ности между корреспондентами.}

\KW{многопродуктовая потоковая сетевая
модель; множество достижимых межузловых потоков; предельные
распределения пропускной способности}

\DOI{10.14357/19922264220306} 
  
%\vspace*{-3pt}


\vskip 10pt plus 9pt minus 6pt

\thispagestyle{headings}

\begin{multicols}{2}

\label{st\stat}

\section{Введение}

Данная работа продолжает исследования функциональных характеристик
сетевых сис\-тем связи~[1]. В~настоящее время математические модели
передачи многопродуктового потока применяются для поиска
распределений потоков и~ресурсов в~многопользовательских
телекоммуникационных\linebreak сетях~[2--4]. Разрабатываются методы анализа
с~учетом вектора требований всех \textit{равноправных} 
и~невзаимозаменяемых корреспондентов~[5]. С~позиций\linebreak методологии
исследования операций изучаются справедливые распределения потоков
и~ресурсов~[6].

Соответствующие \textit{недискриминирующие} правила управления
потоками являются решениями задач на максмин и/или получаются 
в~результате использования процедур последовательной
лексикографически упорядоченной оптимизации~[7].

В~настоящей работе пути соединения корреспондентов прокладываются
через со\-от\-вет\-ст\-ву\-ющие минимальные разрезы. Указанный метод\linebreak \mbox{можно}
рассматривать как возможный вариант решения задачи о~построении
SPLIT-марш\-ру\-тов~[8,~9]. В~рамках вычислительных экспериментов\linebreak на
многопродуктовой модели анализируются распределения межузловых
потоков  и~пропускной способ\-ности сети.  Для оценки функциональных
возможностей многопользовательской сети используется вектор
совместно допустимых межузловых потоков. Под ресурсом, выделяемым
некоторой паре узлов-кор\-рес\-пон\-ден\-тов,  понимается суммарное
значение тре\-бу\-емых пропускных способностей на всех ребрах,
расположенных на всех маршрутах при прохождении межузлового\linebreak потока
данного вида.  Сумма соответствующих реберных потоков трактуется
как полная нагрузка на сеть, возникающая при передаче заданного
межузлового потока. Рас\-смат\-ри\-ва\-ют\-ся распределения пропускной
способности и~межузловых потоков при предельной загрузке сети.
При проведении вычислительных экспериментов на каждой  итерации
вычисляется норма  вектора совместно допустимых межузловых
потоков.   Для оценки величины требуемых ресурсов при соединении
корреспондентов по различным путям для каж\-дой пары узлов
определяется максимальный однопродуктовый поток. Марш\-ру\-ты передачи
всех допустимых межузловых потоков  проходят по ребрам
соответствующих минимальных разрезов. Вычислительные эксперименты
проводились  для получения последовательности  мет\-ри\-че\-ских оценок
векторов межузловых потоков, принадлежащих множеству до\-сти\-жи\-мости
многопользовательской сети.

\section{Математическая модель}

В качестве математической модели для описания
многопользовательской сетевой системы используется следующая
формальная запись условий и~ограничений, которые должны
выполняться при одновременной передаче потоков различных видов
между всеми парами улов-корреспондентов:

Сеть $G(\mathbf{d})$ задается множествами $\langle V,
R,U,P\rangle$:
\begin{itemize}
\item  узлов (вершин) сети 
$$
V=\left \{v_{1}, v_{2},\dots,v_{n},\dots,v_{N}\right\};
$$
\item  неориентированных ребер 
$$
R=\left\{r_{1}, r_{2}, \dots, r_{k}, \dots,
r_{E}\right\}.
$$
\end{itemize}

Ребро $r_{k}$ \textit{соединяет} концевые вершины~$v_{n_k}$ и~$v_{j_k}$. 
Реб\-ру~$r_{k}$ ставятся в~соответствие две
ориентированные дуги $\{u_{k},u_{k+E}\}$ из множества
ориентированных дуг $U\hm=\{u_{1}, u_{2}, \dots, u_{k}, \dots,
u_{2E}\}$. Дуги $\{u_{k}, u_{k+E}\}$ определяют прямое и~обратное
на\-прав\-ле\-ние передачи потока по реб\-ру~$r_{k}$ между концевыми
вершинами $\{v_{n_k}, v_{j_k}\}$.

В многопользовательской сети~$G(\mathbf{d})$ рассматривается
$M\hm=N(N\hm-1)$ независимых, невзаимозаменяемых и~равноправных потоков
различных видов, которые передаются между уз\-ла\-ми-кор\-рес\-пон\-ден\-та\-ми
из множества 
$$
P=\left\{p_{1}, p_{2}, \dots, p_{M}\right\}.
$$

По определению, каждой паре уз\-лов-кор\-рес\-пон\-ден\-тов~$p_{m}$
соответствуют:
\begin{itemize}
\item вершина-ис\-точ\-ник с~номером~$s_{m}$, через которую входной поток
$m$-го вида~$z_{m}$ поступает в~сеть;
\item  вершина-при\-ем\-ник с~номером~$t_{m}$, из которой поток $m$-го
вида~$z_{m}$ покидает сеть.
\end{itemize}

В множестве~$P$ выделяется подмножество $P(R^{+})$ пар
уз\-лов-кор\-рес\-пон\-ден\-тов, расположенных в~концевых вершинах
ребра~$r_{k}$, $k\hm=\overline{1,E}$. Вводятся сле\-ду\-ющие обозначения:
пусть ребро~$r_{k}$  с~номером~$k$ соединяет вершины с~номерами~$n$ и~$j$ такими, что $n\hm< j$. Для со\-от\-вет\-ст\-ву\-ющей пары
уз\-лов-кор\-рес\-пон\-ден\-тов~$p_{k}$, расположенных в~узлах $\{v_{n},
v_{j}\}$, узел~$v_{n}$ считается источником, а узел~$v_{j}$~---
приемником потока $z_{k}$ $k$-го вида, который передается из узла
c номером~$n$ в~узел с~номером~$j$ для пары~$p_{k}$ с~номером~$k$.
Для пары $p^{}_{k+E} \Longleftrightarrow \{v_{j},v_{n}\}$ узел~$v_{j}$ 
считается источником~$s_{k+E}$, а~узел $v_m$~--- приемником~$t_{k+E}$ для пары с~номером~$p_{k+E}$. Формируется
$R^+\hm=\{1,2,3,\dots,E,E+1,\dots,2E\}$~--- список номеров смежных
пар.

Пары $p_{k}$ из подмножества~$P(R^{+})$ называются
\textit{смежными} уз\-ла\-ми-кор\-рес\-пон\-ден\-та\-ми. Все остальные
\textit{несмежные} пары уз\-лов-кор\-рес\-пон\-ден\-тов относятся к~множеству~$P(R^{-})$:
\begin{equation*}
P=P(R^{+})\cup P(R^{-});\quad
P(R^{+}) \cap P(R^{-}) = \emptyset.
\end{equation*}

Введем обозначения:
\begin{description}
\item[\,]
$z_{m}$~--- величина \textit{межузлового} потока $m$-го вида,
который поступает в~сеть из узла с~номером~$s_{m }$ и~покидает из
узла с~номером~$t_{m}$;
\item[\,]
$S(v_{n})$~--- множество номеров исходящих дуг, по которым поток
покидает узел~$v_{n}$;
\item[\,]
$T(v_{n})$~--- множество номеров входящих дуг, по которым поток
поступает в~узел~$v_{n}$.
\end{description}

Во всех узлах $v_{n}\in V$, $n\hm=\overline{1,N}$, для всех видов
потоков должны выполняться условия сохранения потоков:
\begin{multline}
\label{eq1} 
\sum\limits_{i\in S(v_n )} x_{mi}-\sum\limits_{i\in T(v_n )} x_{mi}
={}\\
{}=\begin{cases}
z_m, & \mbox{если } v=v^{}_{S_m}; \\
-z_m,&\mbox{если } v=v_{t_m}; \\
0&\mbox{в остальных случаях}, \\
\end{cases}
\end{multline}
$n=\overline{1,N}$, $m\hm=\overline{1,M}$, $x_{mi}\hm\ge 0$,
$z_{m}\hm\ge0$.

Величина {z}$_{m}$ равна входному потоку $m$-го вида, который
пропускается от источника к~приемнику пары $p_{m}$ при
распределении потоков $x_{mi}$ по дугам сети.

Каждому ребру $r_{k}\hm\in R$ приписывается неотрицательное число~$d_{k}$, 
определяющее суммарный предельно допустимый поток,
который можно передать по реб\-ру~$r_{k}$ в~обоих на\-прав\-ле\-ни\-ях. 
В~исходной сети компоненты вектора про\-пуск\-ных способностей
$\mathbf{d}\hm=(d_{1}, d_{2},\dots, d_{k}, \dots, d_{E})$~--- наперед
заданные положительные числа $d_{k}
\hm> 0$. Вектором $\mathbf{d}$ определяются сле\-ду\-ющие ограничения на сумму
дуговых потоков всех видов, пе\-ре\-да\-ва\-емых по реб\-ру~$r_{k}$:
\begin{multline}
\sum\limits_{m=1}^M (x_{mk}+x_{m(k+E)}) \le d_k,\\
 x_{mk}\ge 0\,,\enskip
 x_{m(k+E)}\ge 0\,, \enskip k=\overline {1,E}\,.
 \label{eq2} 
\end{multline}
В рамках данной модели пропускная спо\-соб\-ность ребер сети~--- вектор~$\mathbf{d}$~--- трактуется как <<\textit{ресурсное ограничение}>>,
а~сумма дуговых
 потоков рас\-смат\-ри\-ва\-ет\-ся как показатель использования
<<\textit{ресурсов}>> сети при передаче межузловых потоков
различных видов.

Для всех $z_{m}$ и~$x_{mi}$, удовлетворяющих
условиям~\eqref{eq1} и~\eqref{eq2}, вычисляются суммарные потоки:
\begin{equation}
 y_{m }=\sum\limits_{i=1}^{2E} {x}_{mi},\quad
m=\overline{1,M}\,.
\label{eq3}
\end{equation}

Суммарный реберный поток~$y_{m}$ характеризует
<<\textit{нагрузку}>> на сеть при передаче межузлового потока
величины $z_{m}$ из уз\-ла-ис\-точ\-ни\-ка~$s_{m}$ в~узел-при\-ем\-ник~$t_{m}$. 
Величина~$y_{m}$ показывает, какой суммарный
\textit{ресурс}~-- пропускная спо\-соб\-ность сети~-- требуется для
передачи межузлового потока~$z_{m}$, а~отношение
$w_{m}\hm={y_m}/{z_m}$,  $m\hm=\overline{1,M},$
показывает, какие \textit{ресурсы} необходимы для передачи
единичного потока $m$-го вида между узлами~$s_{m}$ и~$t_{m}$.

Ограничения~\eqref{eq1}--\eqref{eq3} задают подмножество
допустимых значений компонент вектора межузловых потоков
$\mathbf{z}\hm=\left(z_{1}, z_{2},\dots,z_{m},\dots,z_{M}\right)$:
\begin{equation*}
{Z}(\mathbf{d})=\left\{\mathbf{z} \ge 0 \mid
(\mathbf{z},\mathbf{x},\mathbf{y}) \ \mbox{удовлетворяют~\eqref{eq1}--\eqref{eq3}}
\right\}\!,
\!\!
%\label{eq4} 
\end{equation*}
а все допустимые распределения ресурсов принадлежат подмножеству
\begin{equation*}
{Y}(\mathbf{d})=\left\{\mathbf{y} \ge 0 \mid
(\mathbf{z},\mathbf{x},\mathbf{y}) \ \mbox{удовлетворяют~\eqref{eq1}--\eqref{eq3}}\right\}\!.
%\!\!\!\label{eq5}
\end{equation*}


\section{Метрические оценки предельных распределений}

Для оценки функциональных возможностей сис\-те\-мы рассматриваются
допустимые распределения межузловых потоков при предельных
загрузках ребер сети.

В рамках данного модельного описания монопольным режимом
называется способ управления, при котором все ресурсы сети
используются для передачи потока одной выделенной пары
уз\-лов-кор\-рес\-пон\-ден\-тов $p_{a}\hm\in P(R^-)$, а для всех
остальных потоки полагаются равными нулю.

Предельно допустимый поток, который можно передать между
фиксированной парой уз\-лов-кор\-рес\-пон\-ден\-тов $p_{a}$ в~монопольном
режиме, является решением стандартной, в~данном случае
однопродуктовой, задачи о~максимальном потоке.

\smallskip

\noindent
\textbf{Задача 1.} Найти
$z_a^0\hm=\max\limits_{\langle z,x\rangle \in Z(d)} z_a
$
при условии $z_{i}=0$, $i\hm=\overline{1,M}$, $i\hm\ne a$.

При решении задачи~1 для пары $p_{a}$ вы\-чис\-ля\-ют\-ся: межузловой
поток~$z_a^0$; дуговые потоки $\{x^{0}_{ak};x^{0}_{a(k+E)}\}$,
$k\hm=\overline{1,E}$; суммарное значение реберного
потока~$y_{a}^{0}\hm=\sum\nolimits_{i=1}^{2E} {x}_{ai}^{0}$.

Поток величины $z_a^0$ является \textit{максимальным потоком},
пе\-ре\-да\-ва\-емым в~\textit{монопольном режиме} для пары
уз\-лов-кор\-рес\-пон\-ден\-тов~$p_{a}$, $p_{a}\hm\in P(R^-)$, в~сети~$G(d)$.

Задача~1 решается последовательно для всех $p_{m}\in P(R^-)$,
вы\-чис\-ля\-ют\-ся значения $z_{m}^{0}(t)$.

При проведении вычислительных экспериментов использовалась
итерационная процедура для оценки функциональных возможностей
сис\-те\-мы при передаче межузловых потоков по нескольким маршрутам.
На предварительном этапе шага~$t$ в~сети~$G(t)$ при заданных
значениях пропускной спо\-соб\-ности ребер~$d_k(t)$ для каждой \mbox{пары}
уз\-лов-кор\-рес\-пон\-ден\-тов $p_m\hm\in P(R^-)$ определяется максимально
допустимый однопродуктовый поток~$z^0_m(t)$, со\-от\-вет\-ст\-ву\-ющие
дуговые потоки $(x_{mk}^0(t),x_{m(k+E)}^0(t))$, $p_m\hm\in P(R^-)$, и~коэффициенты нормировки
$\xi_m^0(t)\hm={1}/{z_m^0(t)}$ для всех  $p_m\hm \in P(R^-)$,
таких что $z^0_m(t)\hm>0$, $y_m^0(t)\hm>0$.
Коэффициенты~$\xi_m^0(t)$ используются для поиска текущих
совместно допустимых квот на передачу потоков одновременно между
всеми парами $p_m\in P(R^-)$.

\smallskip

\noindent
\textbf{Задача 2.} Найти $\alpha^*(t)=\max\limits_\alpha \alpha$
при условиях
$$
\alpha\!\!\sum\limits_{m\in R^-}\! \xi_m^0\left(x_{mk}^0(t)+x_{m(k+E)}^0(t)\right)\le d_k(t),\enskip
k=\overline{1,E}\,.
$$

На основании $\alpha^*(t)$ вычисляются совместно допустимые
дуговые потоки:
\begin{multline*}
x_{mk}^*(t)=\alpha^*(t)\xi^0_m(t)x^0_{mk}(t),\\
x^*_{m(k+E)}(t)=\alpha^*(t)\xi^0_m(t)x^0_{m(k+E)}(t),
\\
m=\overline{1,M}\,,\enskip k=\overline{1,E}\,,
\end{multline*}
и остаточная пропускная способность ребер в~сети $G(t+1)$:
\begin{multline*}
d_k(t+1)=d_k(t)-\sum_{m\in R^-} (x_{mk}^*(t)+x_{m(k+E)}(t)),\\
k=\overline{1,E}\,,\enskip p_m\in P(R^-).
\end{multline*}
Формируется вектор допустимых межузловых потоков:
\begin{align*}
z_k^+(t)&=d_k(t+1),\enskip p_k\in P(R^+),\enskip k=\overline{1,E}\,;
\\
z_m^-(t)&=\sum\limits_{\tau=1}^t\alpha^*(\tau)\xi_m^0(\tau) z_m^0(\tau), \enskip p_m\in P(R^-).
\end{align*}

По построению, на шаге~$t$ при передаче вектора межузлового потока
$\mathbf{z}(t)=\{\mathbf{z}^+(t), \mathbf{z}^-(t)\}$ достигается
предельная загрузка, и~пропускная способность всех ребер  сети
используется полностью.

Точка с~координатами $\mathbf{z}(t)$ принадлежит множеству~$Z(d)$.

Расстояние точки от начала координат определяется как норма
соответствующего вектора:
\begin{align*}
\rho^+(t)&=\|\mathbf{z}^+(t)\|=
\left[\,\sum\limits_{k=1}(\mathbf{z}^+(t))^2\right]^{1/2};
\\
\rho^-(t)&=\|\mathbf{z}^-(t)\|= \left[\sum\limits_{p_m\in P(R^-)}(\mathbf{z}_m^-(t))^2\right]^{1/2}.
\end{align*}

Если при выполнении шага $(t+1)$ окажется, что $z_m^0(t+1)=0$ для
всех $p_m\in P(R^-)$, то про\-изойдет останов и~сформируются
массивы финальных данных:
\begin{align*}
z_m^-(T)&=\sum\limits_{\tau=1}^t \alpha^*(\tau)\xi_m^0(\tau) z_m^0(\tau),\enskip 
p_m\in P(R^-),\\
z_k^+(T)&=d_k(t+1),\enskip p_k\in P(R^+),\enskip k=\overline{1,E}\,.
\end{align*}

Вышеописанная вычислительная процедура далее обозначается как
MFPL-про\-це\-ду\-ра (от англ.\ \textit{max-flow-peak-load}).

При проведении второй серии вычислительных экспериментов
MFPL-про\-це\-ду\-ра использовалась для оценки функциональных
характеристик сис\-те\-мы при \textit{уравнительном} поэтапном
распределении пропускной способности между всеми
па\-ра\-ми-кор\-рес\-пон\-ден\-тами.

При реализации MFPL-процедуры выполнение каждого шага разбивается
на несколько этапов. На предварительном этапе шага~$t$ 
в~сети~$G(t)$ при заданных значениях пропускной способности ребер~$d_k(t)$ 
для каждой пары уз\-лов-кор\-рес\-пон\-ден\-тов $p_m\hm\in P(R^-)$
определяется максимально допустимый однопродуктовый
поток~$z_m^0(t)$, соответствующие дуговые потоки
$\left(x_{mk}^0(t),x_{m(k+E)}^0(t)\right)$, $p_m\hm\in P(R^-)$, и~суммарная
реберная нагрузка
$$
y_m^0(t)=\sum\limits_{k=1}^E (x_{mk}^0(t),x_{m(k+E)}^0(t)),\enskip p_m\in P(R^-).
$$

Для каждой пары $p_m\hm\in P(R^-)$ вычисляются коэффициенты
нормировки
$\theta_m^0(t)\hm={1}/{y_m^0(t)}$ для всех  
$p_m\hm\in P(R^-)$, таких что  $z^0_m(t)\hm>0$,
$y_m^0(t)\hm>0$.
Коэффициенты~$\theta_m^0(t)$ используются для поиска совместно
допустимых дуговых потоков для всех $p_m\hm\in P(R^-)$.

\smallskip

\noindent
\textbf{Задача 3.} Найти $\beta^*(t)=\max\nolimits_\beta \beta$ при
условиях
$$
\beta\!\!\!\!\sum\limits_{p_m\in P(R^-)}\!\!
\theta_m^0(x_{mk}^0(t)+x_{m(k+E)}^0(t))\le d_k(t),\enskip
k=\overline{1,E}\,.
$$

 С помощью $\beta^*(t)$ (решения задачи~3) вычисляются текущие допустимые значения дуговых потоков:
\begin{multline*}
x_{mk}^*(t)=\beta^*(t)\theta^0_m(t)x^0_{mk}(t),\\
x^*_{m(k+E)}(t)=\beta^*(t)\theta^0_m(t)x^0_{m(k+E)}(t), \enskip
k=\overline{1,E},
\end{multline*}
и реберных нагрузок при одновременной передаче межузловых потоков:

\noindent
\begin{multline*}
y_m^*(t)=\sum\limits_{i=1}^E
\left[x_{mi}^*(t)+x^*_{m(i+E)}(t)\right]={}\\
{}= \fr{\beta^*(t)}{y_m^0(t)} \sum\limits_{i=1}^E
\left[x_{mi}^0(t)+x^0_{m(i+E)}(t)\right]=\beta^*(t), \\
 p_m\in P(R^-).
\end{multline*}
Таким образом на каждом шаге определенная часть имеющегося ресурса
(пропускной спо\-соб\-ности) делится строго по\-ров\-ну меж\-ду всеми
корреспондентами $p_m\in P(R^-)$, для которых существует путь
передачи в~$G(t)$.

Формируется вектор допустимых межузловых потоков:
\begin{gather*}
\hspace*{-30mm}z_k^{++}(t)=d_k(t+1)={}\hspace*{10mm}\\
{}=d_k(t)-\!\!\! \sum\limits_{p_m\in P(R^-)}\!\!\!
\left(x_{mk}^*(t)+x_{m(k+E)}(t)\right),\\
\hspace*{35mm}k=\overline{1,E}, \enskip
p_k\in P(R^+);\\
z_m^{(=)}(t)\overset{\Delta}{=}\sum\limits_{\tau=1}^t\beta^*(\tau)
\theta_m^0(\tau) z_m^0(\tau), \enskip p_m\in P(R^-).
\end{gather*}

\noindent
Определяются расстояния:
\begin{align*}
\rho^{++}(t)&=\|\mathbf{z}^{++}(t)\|\overset{\Delta}{=}
\left[\sum\limits_{k=1}^E\left(d_k(t+1)\right)^2\right]^{1/2};\\
\rho^{(=)}(t)&=\|\mathbf{z}^{=}(t)\|= \left[\sum\limits_{p_m\in
P(R^-)}\left(z_m^{(=)}(t)\right)^2\right]^{1/2}.
\end{align*}

Если на предварительном этапе на шаге $(t+1)$ окажется, что в~сети~$G(t+1)$ для всех $p_m\hm\in P(R^-)$ все значения
$z_m^0(t+1)\hm=0$, то произойдет останов и~сформируются финальные
массивы:
\begin{align*}
z_k^{(++)}(T)&=d_k(t+1), \enskip
p_k\in P(R^+), \enskip k=\overline{1,E};
\\
z_m^{(=)}(t)&=\sum\limits_{\tau=1}^{t+1}\beta^*(\tau)
\theta_m^0(\tau) z_m^0(\tau), \enskip p_m\in P(R^-).
\end{align*}



\section{Вычислительный эксперимент}

Результаты вычислительных экспериментов, описанные ниже, служат
продолжением исследований, начатых в~[1]. Вычислительные
эксперименты проводились на моделях сетевых сис\-тем, пред\-став\-лен\-ных
на рис.~1 и~2. В~каждой сети~69~узлов. Пропускные спо\-соб\-но\-сти
ребер~-- значения $d_k$~-- выбирались случайным образом из отрезка
$[900,999]$ и~совпадали для ребер, при\-сут\-ст\-ву\-ющих в~обеих сетях.
В~кольцевой сети пропускная спо\-соб\-ность каждого из добавленных
ребер равнялась~900.

\begin{figure*} %fig1
\vspace*{1pt}
\begin{minipage}[t]{80mm}
  \begin{center}  
    \mbox{%
\epsfxsize=69.408mm
\epsfbox{mal-1.eps}
}

\end{center}
\vspace*{-6pt}
\Caption{Базовая сеть}
\end{minipage}
%\end{figure*}
\hfill
%\begin{figure*} %fig2
\vspace*{1pt}
\begin{minipage}[t]{80mm}
  \begin{center}  
    \mbox{%
\epsfxsize=69.408mm
\epsfbox{mal-2.eps}
}

\end{center}
\vspace*{-6pt}
\Caption{Кольцевая сеть}
\end{minipage}
\end{figure*}

\begin{table*}[b]\small %tabl1
\vspace*{-12pt}
\begin{center}

%\renewcommand{\arraystretch}{1.1}
\Caption{Базовая сеть}
\vspace*{2ex}

\begin{tabular}{|c||c|c|c||c|c|c|} 
\hline
&&&&&&\\[-9pt]
$t$  & $\rho^{-}(t)$ & $\rho^{+}(t)$ & $d^{+}(t+1)$ &
$\rho^{=}(t)$ & $\rho^{++}(t)$&  $d^{++}(t+1)$ \\ 
\hline
\hphantom{99}0  & \hphantom{99}0   & 8048&  68256&  \hphantom{9}0   &  8048&   68256\\
1  & \hphantom{9}63  & 4182&  26544&  \hphantom{9}95  &  3881&   24476\\
$\cdots$  & $\cdots$   & $\cdots$   &  $\cdots$    &  $\cdots$   &  $\cdots$   &   $\cdots$\\
11 & \hphantom{9}70  & 3975&  21469&  \hphantom{9}101\hphantom{9} &  3707&   20155\\
$\cdots$& $\cdots$   & $\cdots$   &  $\cdots$    & $\cdots$   &  $\cdots$   &  $\cdots$\\
22 & \hphantom{9}83  & 3861&  19623&  \hphantom{9}122\hphantom{9} &  3586&   18260\\
$\cdots$ & $\cdots$  & $\cdots$   &  $\cdots$   &  $\cdots$   &  $\cdots$  &   $\cdots$\\
33 & \hphantom{9}103\hphantom{9} & 3778&  18827&  \hphantom{9}139\hphantom{9} &  3522&   17601\\
$\cdots$ &$\cdots$  &$\cdots$  & $\cdots$  & $\cdots$   &  $\cdots$  &  $\cdots$\\
44 & \hphantom{9}\bf 190\hphantom{9} & \bf3553&  \bf17503&  \hphantom{9}\bf203\hphantom{9} &  \bf3285&   \bf16201\\
45 & \hphantom{9}\bf1452\hphantom{99}& \bf2166&  \hphantom{9}\bf7069 &  \hphantom{9}\bf1376\hphantom{99}&  \bf2020&   \hphantom{9}\bf6584\\
46 & \hphantom{9}\bf1498\hphantom{99}& \bf2158&  \hphantom{9}\bf6707 &  \hphantom{9}\bf1388\hphantom{99}&  \bf2017&   \hphantom{9}\bf6483\\
$\cdots$ & $\cdots$   & $\cdots$   &  $\cdots$    & $\cdots$   &  $\cdots$   &  $\cdots$\\
52 & \hphantom{9}1535\hphantom{99}& 2155&  \hphantom{9}6413 & \hphantom{9}1442\hphantom{99} &  2011&   \hphantom{9}6059\\
\hline
\end{tabular}
\end{center}
 %\end{table*}
% \begin{table*}\small %tabl2
\begin{center}
\Caption{Кольцевая сеть}
\vspace*{2ex}


\begin{tabular}{|c||c|c|c||c|c|c|} 
\hline
&&&&&&\\[-9pt]
$t$  & $\rho^{-}(t)$ & $\rho^{+}(t)$ & $d^{+}(t+1)$ &
$\rho^{=}(t)$ & $\rho^{++}(t)$&  $d^{++}(t+1)$ \\
 \hline
\hphantom{9}0  &\hphantom{99}0    & 8440  & 75456   &\hphantom{9}0      &8440   &75456\\
\hphantom{9}1  &\hphantom{9}68   & 5317  & 43038   &92     &5045   &40716 \\ 
$\cdots$ &$\cdots$    & $\cdots$     & $\cdots$   &$\cdots$      &$\cdots$      &$\cdots$      \\
11 &\hphantom{9}95   & 3608  & 20459   &124    &3397   &19080  \\
$\cdots$ &$\cdots$   & $\cdots$    & $\cdots$      &$\cdots$     &$\cdots$     &$\cdots$   \\
22 &\hphantom{9}101\hphantom{9}  & 3540  & 19530   &130    &3350   &18338 \\
$\cdots$ &$\cdots$  & $\cdots$   &$\cdots$      &$\cdots$     &$\cdots$   &$\cdots$    \\
33 &\hphantom{9}135\hphantom{9}  & 3346  & 17561   &154    &3220   &17003 \\
$\cdots$  &$\cdots$   & $\cdots$    & $\cdots$      &$\cdots$     &$\cdots$    &$\cdots$    \\
44 &\hphantom{9}234\hphantom{9}  & 3094  & 14881   &269    &2918   &13848 \\
$\cdots$ &$\cdots$   & $\cdots$    &$\cdots$      &$\cdots$     &$\cdots$     &$\cdots$    \\
50 &\hphantom{9}\bf 413\hphantom{9}  & \bf2770  & \bf12901   &\bf329    &\bf2792   &\bf13079 \\
51 &\hphantom{9}\bf1040\hphantom{99} & \bf2299  & \hphantom{9}\bf8801    &\bf334    &\bf2784   &\bf13034 \\
52 &\hphantom{9}\bf1062\hphantom{99} & \bf2297  & \hphantom{9}\bf8672    &\bf974    &\bf2262   &\hphantom{9}\bf8768  \\
$\cdots$ &$\cdots$   &$\cdots$    & $\cdots$      &$\cdots$      &$\cdots$     &$\cdots$    \\
55 &\hphantom{9}1069\hphantom{99} & 2297  & \hphantom{9}8630    &1010\hphantom{9}   &2259   &\hphantom{9}8553  \\
\hline
 \end{tabular}
\end{center}
 \end{table*}




Для базовой сети исходная сумма пропускных способностей:
$D^+(0)\hm=68\,256$, а~для кольцевой сети $D^{++}(0)=75\,456$.
Соответствующие значения $\rho^+(0)$ и~$\rho^{++}(0)$ указаны в~<<нулевой>> строке 
в~табл.~1 и~2, где собраны результаты
вычислительных экспериментов. В~ходе эксперимента при
уравнительном распределении остаточных ресурсов соблюдается
\textit{равномерное} убывание остаточной пропускной спо\-соб\-ности и~<<\textit{длины}>> вектора~$\rho^+(t)$. 
Однако между 44--46
итерациями для базовой и~50--52 для кольцевой сети наблюдается
резкий скачок величин~$\rho^-(t)$, $\rho^{=}(t)$ и~$d^+(t)$,
$d^{++}(t)$.

На указанных шагах полностью используется пропускная способность
ребер в~центральной час\-ти сети. Сеть \textit{распадается} на
несвязные компоненты, и~для $80\%$ корреспондентов пропадают пути
соединения, а~остаточный ресурс распределяется поровну между
оставшимися парами узлов.

Анализ результатов показал, что почти равные значения потоков
достигаются для~80\% корреспондентов и~требуют 60\%--70\%
ресурсов. Однако для~2\% смежных  пар узлов межузловые потоки на
два порядка выше медианных значений, а~затраты пропускной
способности  со\-став\-ля\-ют~20\%--30\%.








\section{Заключение}

Предложенный метод и~проведенные вычислительные эксперименты
показали, что уравнительное поэтапное распределение   приводит 
к~неравномерному  распределению   потоков  для разных групп\linebreak
корреспондентов.    Метрические оценки, полученные  в~ходе
экспериментов, продемонстрировали\linebreak \textit{деформацию} множества
достижимых потоков. В~рамках модели   предполагалось, что  все
корреспонденты  равноправны, а~потоки невзаимозаменяемы,  однако
при уравнительном предельном  распределении  смежные  пары узлов
оказывались в~привилегированном положении при использовании
остаточной пропускной способности. Пропускные способности  ребер
рассматривались  как вектор   ресурсов  различных типов,  которые
распределяются между корреспондентами   при передаче  потоков
различных видов.  По построению, на каж\-дом шаге норма вектора
смежных   межузловых    потоков численно равна   модулю вектора
остаточных  пропускных способностей.   Полученные мет\-ри\-че\-ские
значения  можно использовать  для   оценки функциональных
возможностей сети  в~режиме  предельной загрузки.

{\small\frenchspacing
 {%\baselineskip=10.8pt
 %\addcontentsline{toc}{section}{References}
 \begin{thebibliography}{9}

\bibitem{1-mal}
\Au{Малашенко Ю.\,Е., Назарова И.\,А.} Неоднородность
распределения   потоков при предельной  загрузке
многопользовательской сети~//  Известия РАН. Теория и~сис\-те\-мы
управления,  2022. №\,3. С.~81--96.

\bibitem{4-mal} %2
\Au{Luss H.} Equitable resource allocation: Models,
algorithms, and applications.~--- Hoboken, NJ, USA: John Wiley \& Sons, 2012.
420~p.

\bibitem{2-mal} %3
\Au{Ogryczak W., Luss~H., Pioro~M., Nace~D., Tomaszewski~A.}   Fair
optimization and networks: A~aurvey~// J.~Appl. Math., 2014. Vol.~2014. Art.~ID~612018. 25~p. doi: 10.1155/ 2014/612018.

\bibitem{3-mal} %4
\Au{Salimifard K., Bigharaz~S.} The multicommodity network
flow problem: State of the art classification, applications, and
solution methods~// J.~Oper. Res., 2020. Vol.~18. Iss.~3. P.~1--47.



\bibitem{5-mal}
\Au{Balakrishnan A., Li~G., Mirchandani~P.}  Optimal
network design with end-to-end service requirements~// Oper. Res.,
2017. Vol.~65. Iss.~3. P.~729--750.

\bibitem{6-mal}
\Au{Nace D., Doan~L.\,N., Klopfenstein~O., Bashllari~A.} Max-min
fairness in multicommodity flows~// Comput. Oper. Res., 2008.
Vol.~35. Iss.~2. P.~557--573.

\bibitem{7-mal}
\Au{Ros-Giralt J., Tsai~W.\,K.} A~lexicographic optimization
framework to the flow control problem~// IEEE T.
Inform. Theory, 2010. Vol.~56. Iss.~6. P.~2875--2886.

\bibitem{8-mal}
\Au{Baier G., Kohler~E., Skutella~M.}  The \mbox{k-splittable}
flow problem~//  Algorithmica, 2005. Vol.~42. Iss.~3-4.
P.~231--248.

\bibitem{9-mal}
\Au{Bialon P.\,A.} Randomized rounding approach to 
a~\mbox{k-splittable} multicommodity flow problem with lower path flow
bounds affording solution quality guarantees~// Telecommun. Syst.,
2017. Vol.~64. Iss.~3. P.~525--542.
\end{thebibliography}

 }
 }

\end{multicols}

\vspace*{-6pt}

\hfill{\small\textit{Поступила в~редакцию 10.06.22}}

\vspace*{8pt}

%\pagebreak

%\newpage

%\vspace*{-28pt}

\hrule

\vspace*{2pt}

\hrule

%\vspace*{-2pt}

\def\tit{SEQUENTIAL ANALYSIS AND METRIC ESTIMATES\\ OF~PEAK LOAD FLOWS IN~THE~MULTIUSER NETWORK}


\def\titkol{Sequential analysis and metric estimates of~peak load flows in~the~multiuser network}


\def\aut{Yu.\,E.~Malashenko}

\def\autkol{Yu.\,E.~Malashenko}

\titel{\tit}{\aut}{\autkol}{\titkol}

\vspace*{-8pt}


\noindent
Federal Research Center ``Computer Science and Control'' of the Russian Academy of Sciences, 
44-2~Vavilov Str., Moscow 119333, Russian Federation



\def\leftfootline{\small{\textbf{\thepage}
\hfill INFORMATIKA I EE PRIMENENIYA~--- INFORMATICS AND
APPLICATIONS\ \ \ 2022\ \ \ volume~16\ \ \ issue\ 3}
}%
 \def\rightfootline{\small{INFORMATIKA I EE PRIMENENIYA~---
INFORMATICS AND APPLICATIONS\ \ \ 2022\ \ \ volume~16\ \ \ issue\ 3
\hfill \textbf{\thepage}}}

\vspace*{3pt} 



\Abste{The set of vectors of internodal flows in a~multiuser communication network under peak load is analyzed. Within the framework of
 the multicommodity model, the throughput capacities of edges are considered as the components of a~vector of resources of various types that 
 are required for the transmission of various kinds of
 flows. When conducting computational experiments, at each iteration, the
  norms of vectors of jointly permissible internodal flows are calculated, during the transmission of which the capacity of 
  all network edges is fully used.\linebreak\vspace*{-12pt}}
 
 \Abstend{The proposed method and computational experiments have shown that the equalizing phased 
  distribution leads to an uneven distribution of flows for different groups of correspondents. Metric values obtained during experiments 
  indicate deformation of the sets of accessible flows. Within the framework of the model, all correspondents are tantamount 
  and the flows are noninterchangeable; however, in the case of an equalizing peak load distribution, adjacent pairs 
  of nodes are in a privileged position when using residual capacity. The obtained metric values can be used to 
  evaluate the functional characteristics of the transmission network in the finite capacity loading mode.}

\KWE{multicommodity flow network model; set of achievable internodal flows; peak load distribution}


\DOI{10.14357/19922264220306} 

%\vspace*{-16pt}

%\Ack
%\noindent



%\vspace*{4pt}

  \begin{multicols}{2}

\renewcommand{\bibname}{\protect\rmfamily References}
%\renewcommand{\bibname}{\large\protect\rm References}

{\small\frenchspacing
 {%\baselineskip=10.8pt
 \addcontentsline{toc}{section}{References}
 \begin{thebibliography}{9}
\bibitem{1-mal-1}
\Aue{Malashenko, Yu.\,E., and I.\,A.~Nazarova.}
2022. Heterogeneous flow distribution at the peak load in the multiuser network. \textit{J.~Comput. Sys. Sc. Int.} 61:372--387.

\bibitem{4-mal-1} %2
\Aue{Luss, H.} 2012. \textit{Equitable resource allocation: Models, algorithms, and applications}.
Hoboken, NJ: John Wiley \& Sons. 420~p.

\bibitem{2-mal-1} %3
\Aue{Ogryczak, W., H.~Luss, M.~Pioro, D.~Nace, and A.~Tomaszewski.}
 2014. Fair optimization and networks: A~survey. \textit{J.~Appl. Math.} 2014:612018. 25~p. doi: 10.1155/ 2014/612018.
\bibitem{3-mal-1} %4
\Aue{Salimifard, K., and S.~Bigharaz.}
 2020. The multicommodity network flow problem: State of the art classification, applications, and solution methods. 
 \textit{J.~Oper. Res.} 18(3):\linebreak 1--47.

\bibitem{5-mal-1}
\Aue{Balakrishnan, A., G.~Li, and P.~Mirchandani.} 2017. Optimal network design with end-to-end service requirements. 
\textit{Oper. Res.} 65(3):729--750.
\bibitem{6-mal-1}
\Aue{Nace, D., L.\,N.~Doan, O.~Klopfenstein, and A.~Bashllari.} 2008. Max-min fairness in multicommodity flows. 
\textit{Comput. Oper. Res.} 35(2):557--573.
\bibitem{7-mal-1}
\Aue{Ros-Giralt, J., and W.\,K.~Tsai.} 2010. A~lexicographic optimization framework to the flow control problem. 
\textit{IEEE T.~Inform. Theory} 56(6):2875--2886.
\bibitem{8-mal-1}
\Aue{Baier, G., E.~Kohler, and M.~Skutella.}
 2005. The k-splittable flow problem. \textit{Algorithmica} 42(3-4):231--248.
\bibitem{9-mal-1}
\Aue{Bialon, P.} 2017. A~randomized rounding approach to a~\mbox{k-splittable} multicommodity flow problem with lower path flow bounds affording solution quality guarantees. 
\textit{Telecommun. Syst.} 64(3):525--542.
 \end{thebibliography}

 }
 }

\end{multicols}

\vspace*{-6pt}

\hfill{\small\textit{Received June 10, 2022}}

\Contrl

\noindent
\textbf{Malashenko Yuri E.} (b.\ 1946)~--- 
Doctor of Science in physics and mathematics, principal scientist, Federal Research Center ``Computer Science and Control'' 
of the Russian Academy of Sciences, 44-2~Vavilov Str., Moscow 119333, Russian Federation; \mbox{malash09@ccas.ru} 


\label{end\stat}

\renewcommand{\bibname}{\protect\rm Литература}   %4

%\def\ss2{\mathop {\sum\limits^p\sum\limits^p}}
\def\mna{\mathrm{МНА}}
\def\mop{\mathrm{МОР}}
\def\op{\mathrm{ОР}}

\def\stat{sinits}

\def\tit{АНАЛИТИЧЕСКОЕ МОДЕЛИРОВАНИЕ
РАСПРЕДЕЛЕНИЙ С~ИНВАРИАНТНОЙ МЕРОЙ В~НЕГАУССОВСКИХ ДИФФЕРЕНЦИАЛЬНЫХ
И~ПРИВОДИМЫХ К НИМ ЭРЕДИТАРНЫХ СТОХАСТИЧЕСКИХ СИСТЕМАХ$^*$}

\def\titkol{Аналитическое моделирование
распределений с инвариантной мерой в~негауссовских дифференциальных
%и~приводимых к ним эредитарных       стохастических
системах}

\def\autkol{И.\,Н.~Синицын}

\def\aut{И.\,Н.~Синицын$^1$}

\titel{\tit}{\aut}{\autkol}{\titkol}

{\renewcommand{\thefootnote}{\fnsymbol{footnote}}
\footnotetext[1]{Работа выполнена при финансовой поддержке  программы
       <<Интеллектуальные информационные технологии, системный анализ
       и автоматизация>> (проект~1.7).}}

\renewcommand{\thefootnote}{\arabic{footnote}}
\footnotetext[1]{Институт проблем
информатики Российской академии наук, sinitsin@dol.ru}

\vspace*{-12pt}

\Abst{Представлены методы и алгоритмы аналитического моделирования одно- и
многомерных распределений с инвариантной мерой в дифференциальных и
интегродифференциальных (эредитарных)  стохастических системах (СтС),
описываемых уравнениями Ито в конечномерных пространствах с винеровскими
и пуассоновскими шумами. Сначала в разд.~2 рассматриваются интегродифференциальные
уравнения Пугачева для распределений процессов в дифференциальных СтС (ДСтС).
Применительно к ДСтС с гладкими и разрывными регулярными правыми частями найдены
условия сохранения инвариантной меры для нестационарных и стационарных процессов.
Сформулированы 4~теоремы, определяющие точные алгоритмы аналитического моделирования
распределений с инвариантной мерой в ДСтС общего вида. В~разд.~3 дан краткий обзор
приближенных методов аналитического моделирования в ДСтС,
основанных на параметризации распределений. Особое внимание
уделено методам нормальной аппроксимации и статистической линеаризации
для приближенного определения одно- и двумерных распределений с инвариантной
мерой. Получены условия устойчивости алгоритмов. Сформулированы две теоремы,
определяющие приближенные алгоритмы аналитического моделирования в ДСтС. Раздел~4
посвящен методам и алгоритмам аналитического моделирования распределений с
инвариантной мерой в интегродифференциальных эредитарных СтС (ЭСтС), приводимых
к дифференциальным. Представлены нелинейные стохастические интегродифференциальные
уравнения  Ито с винеровскими и пуассоновскими шумами. Для затухающих физически
возможных эредитарных ядер рассматривается два способа их аппроксимации
(на основе линейных операторных уравнений и вырожденных ядер). Рассмотрены
три теоремы, определяющие точные и приближенные алгоритмы приведения ЭСтС к ДСтС
для гладких и разрывных регулярных правых частей. В~приложении даны тестовые
примеры для разрабатываемого в ИПИ РАН инструментального программного обеспечения
<<ID StS>> в среде MATLAB. Заключение содержит основные выводы и возможные
обобщения. Рассмотрено применение результатов разд.~2--4 к задачам эквивалентности
гауссовских и негауссовских ДСтС и ЭСтС.}

\KW{аналитическое моделирование; гауссовская (нормальная) стохастическая
система; дифференциальная стохастическая система; инструментальное
программное обеспечение <<ID StS>>; метод нормальной аппроксимации;
метод статистической линеаризации; негауссовская (с винеровскими и
пуассоновскими шумами) стохастическая система; распределение с
инвариантной мерой; сингулярное (вырожденное) ядро; стохастическое
дифференциальное уравнение Ито; система, приводимая к
дифференциальной; эредитарное ядро}

\DOI{10.14357/19922264140201}

\vskip 14pt plus 9pt minus 6pt

      \thispagestyle{headings}

      \begin{multicols}{2}

            \label{st\stat}


\section{Введение}

%\vspace*{-4pt}

Вопросам разработки методов, алгоритмов и инструментальных
программных средств для анализа и моделирования  распределений
процессов в гауссовских (нормальных) ДСтС с инвариантной мерой
посвящена обширная литература (см., например,~[1--18]).
Методы анализа и моделирования распределений процессов с
инвариантной мерой в интегродифференциальных ЭСтС
подробно изложены в~[1, 11, 19--22].

Для ДСтС и ЭСтС, приводимых к ДСтС, с винеров\-скими и пуассоновскими
шумами соответствующие точные и приближенные методы аналитического
моделирования распределений с инвариантной мерой не разработаны.
Особое внимание уделяется приближенным методам аналитического
моделирования распределений с инвариантной мерой, основанным на
нормальной аппроксимации одно- и двумерных распределений.
Приводятся тестовые примеры.

\section{Уравнения для распределений процессов с~инвариантной мерой в~дифференциальных
стохастических системах}

Как известно~\cite{1-s, 2-s}, для ДСтС в
конечномерных пространствах используется дифференциальное стохастическое
уравнение Ито следующего \mbox{вида}:
    \begin{multline}
    dY= a (Y,t) \,dt + b(Y,t)\, d W_0 +{}\\
    {}+ \int\limits_{R_0^q} c(Y,t,v)\, dP^0 (t,dv)\,.
    \label{e2.1-s}
    \end{multline}
Здесь $Y$~--- $p$-мер\-ный вектор состояния, $Y\hm\in \Delta^y$
($\Delta^y$~--- многообразие состояний);
$a\hm=a (y,t)$ и $b\hm= b(y,t)$~--- известные $(p\times 1)$-мер\-ная и
$(p\times r)$-мер\-ная функции вектора~$Y$ и времени~$t$; $W_0\hm= W_0(t)$~---\linebreak
$r$-мер\-ный винеровский случайный процесс интенсивности $\nu_0\hm=
\nu_0(t)$; $c(y, t,v)$~--- $(p\times 1)$-мер\-ная функция~$y$, $t$ и
вспомогательного $(q\times 1)$-мер\-но\-го параметра~$v$;
$\int\limits_{\Delta_t}\, d P^0 (t,A)$~---
центрированная пуассоновская мера, удовлетворяющая условию:
 \begin{equation}
 \int\limits_{\Delta_t} d P^0 (t,A)= \int\limits_{\Delta_t} d P (t,A)-
    \int\limits_{\Delta_t} \nu_P (t,A) \,dt\,,
    \label{e2.2-s}
    \end{equation}
где $\int\limits_{\Delta_t} d P (t,A)$~--- число скачков пуассоновского
процесса $P (t,A)$ в интервале времени~$\Delta$; $\nu_P (t,A)$~--- интенсивность
пуассоновского процесса $P(t,A)$; $A$~--- некоторое борелевское
множество пространства~$R^q_0$ с выколотым началом координат.

Интеграл~(\ref{e2.1-s}) в общем случае распространяется на все пространство~$R_0^q$
с выколотым началом координат.
Начальное значение~$Y_0$ вектора~$Y$ пред\-став\-ля\-ет
собой случайную величину, не зависящую от приращений винеровского
процесса $W_0(t)$ и пуассоновского процесса $P(t,A)$ на интервалах
времени $\Delta_t \hm= (t_1, t_2]$, следующих за~$t_0$, $t_0\hm\le t_1\hm\le t_2$,
для любого множества~$A$.

В случае, когда подынтегральная функция $c(y,t,v)$ в уравнении~(\ref{e2.1-s})
допускает представление $c(y,t,v)\hm= b(y,t) c'(v)$,
уравнение~(\ref{e2.1-s}) приводится к  виду:
\begin{equation*}
{\dot Y} = a (Y, t) +b(Y, t)V\,, %\label{e2.3-s}
\end{equation*}
если принять
\begin{equation*}
V=\dot W\,;\enskip W(t)= W_0(t) +\int\limits_{R_0^q} c'(v) P^0 (t, dv)\,.
%\label{e2.4-s}
\end{equation*}

Пусть существуют одно- и $n$-мер\-ные плот\-но\-сти $f_1\hm=f_1(z;t)$ и
$f_n\hm= f_n(z_1\tr z_n; t_1 \tr t_n)$ и характеристические функции
$g_1\hm=g_1(\la;t)$ и $g_n\hm=g_n(\la_1\tr \la_n; t_1\tr t_n)$ $(n\hm\ge 2)$,
удовлетворяющие интегродифференциальным уравнениям Пугачева~\cite{1-s, 2-s}:
    \begin{multline}
    \fr{\prt f_1(z;t)}{\prt t}+\fr{\prt^T}{\prt z}\lk a(z,t)f_1(z;t)\rk ={}\\
    \!\!{}=
    \fr{1}{(2 \pi)^p}
    \iin\iin \!\!\!\chi^f(\la,\zeta,t) e^{i\la^{\mathrm{T}}(\zeta-z)} f_1(z;t) \,d\zeta d\la\,;\!\!
    \label{e2.5-s}
    \end{multline}

    \vspace*{-12pt}

    \noindent
\begin{equation}
f_1(z;t_0)=f_0(z)\,;\label{e2.6-s}
\end{equation}

\vspace*{-12pt}

    \noindent
\begin{multline*}
 \fr{\prt f_n(z_1\tr z_n;t_1\tr t_n)}{\prt t_n}+{}\\
 {}+
    \fr{\prt^{\mathrm{T}}}{\prt z_n}\left[a(z_n, t_n) f_n (z_1\tr z_n; t_1\tr t_n)\right]={}\\
{}= \fr{1}{(2\pi)^{pn}} \iin\iin \chi_n^f (\la_n, \zeta_n, t_n)
\times{}\\
{}\times \exp\!\lf i \sss_{l=1}^n \la_l^{\mathrm{T}} (\zeta_l-z_l)\rf \! f_n
    (\zeta_1\tr \zeta_n; t_1\tr t_n)\times{}\\
    {}\times d\zeta_1\cdots d\zeta_n d\la_1\cdots d\la_n\,;
%    \label{e2.7-s}
    \end{multline*}

    \vspace*{-12pt}

    \noindent
    \begin{multline*}
f_n(z_1\tr z_{n-1},z_n;t_1\tr t_{n-1},t_{n-1})={}\\
{}= f_{n-1} (z_1\tr z_{n-1};t_1\tr t_{n-1})\delta (z_n - z_{n-1})\,,\\
t_1\le t_2 \le \cdots \le t_n\,,\enskip n=2,3,\ldots;
%        \label{e2.8-s}
        \end{multline*}


\vspace*{-12pt}

    \noindent
\begin{multline}
\fr{\prt g_1 (\la;t)}{\prt t} -{}\\
{}-\fr{1}{(2\pi)^p}
    \iin \iin i\la^{\mathrm{T}} a (z,t) e^{i(\la^{\mathrm{T}} -
    \mu^{\mathrm{T}})z} g_1 (\mu;t) \,d\mu dz={}\\
{}=\fr{1}{(2\pi)^k}\! \iin \iin \! \!\chi^g(\la, z,t) e^{i(\la^{\mathrm{T}} -
\mu^{\mathrm{T}})z} \times{}\\
{}\times g_1 (\mu;t)\, d\mu dz\,;
    \label{e2.9-s}
    \end{multline}
\begin{equation}
g_1(\la;t_0) = g_0(\la)\,;
    \label{e2.10-s}
    \end{equation}

    \vspace*{-12pt}

    \noindent
\begin{multline*}
 \fr{\prt g_n (\la_1\tr \la_n; t_1\tr t_n)}{\prt t_n} -
    \fr{1}{(2\pi)^{pn}} \times{}\\
    {}\times \iin\! \cdots \!\iin i\la^{\mathrm{T}} a (z_n,t_n)
    \exp \lk i \sss_{k=1}^n (\la_k^{\mathrm{T}} - \mu_k^{\mathrm{T}}) z_k\rk\times{}\\
    {}\times g_n
    (\mu_1\tr \mu_n; t_1\tr t_n) d\mu_1 \ldots d \mu_n dz_1\ldots dz_n={}\\
{}= \fr{1}{(2\pi)^{pn}} \iin\ldots \iin \chi^n (\la_n, z_n,t_n)\times{}\\
{}\times\exp \lk i \sss_{k=1}^n (\la_k^{\mathrm{T}} - \mu_k^{\mathrm{T}}) z_k\rk \times{}\\
{}\times g_n
     (\mu_1\tr \mu_n; t_1\tr t_n)\, d\mu_1 \cdots d \mu_n dz_1\cdots dz_n\,;
%     \label{e2.11-s}
     \end{multline*}


\vspace*{-12pt}

    \noindent
\begin{multline*}
 g_n (\la_1\tr \la_n; t_1\tr t_{n-1},t_{n-1})={}\\
 {}=
    g_{n-1} (\la_1\tr \la_{n-2},\la_{n-1}+\la_n; t_1\tr t_{n-1})\\
t_1\le t_2 \le\cdots \le t_n\,, \enskip n=2,3,\ldots
%    \label{e2.12-s}
    \end{multline*}
Здесь приняты следующие обозначения:
\begin{multline*}
    \chi^f (\la,\zeta, t) =-\fr{1}{2}\, \la^{\mathrm{T}}
    b(\zeta,t) \nu_0(t) b(\zeta,t)^{\mathrm{T}}\lambda +{}\\
{}+ \iii_{R_0^q}\left\{ \exp \lk i\la^{\mathrm{T}} c(\zeta,t,v) \rk
    -1 -{}\right.\\
\left.    {}-i\la^{\mathrm{T}} c(\zeta, t,v)\right\} \nu_P (t,dv)\,;
%    \label{e2.13-s}
    \end{multline*}

\vspace*{-14pt}

    \noindent
\begin{multline*}
    \chi^f_n (\la_n,\zeta_n, t_n) =-\fr{1}{2}\,
    \la^{\mathrm{T}}_n b(\zeta_n,t) \nu_0(t) b(\zeta_n,t)^{\mathrm{T}}\lambda +{}\\
{}+ \iii_{R_0^q} \left\{ \exp \lk i\la^{\mathrm{T}}_n c(\zeta_n,t_n,v) \rk -
    1 -{}\right.\\
\left.    {}-i\la^{\mathrm{T}}_n c(\zeta_n, t_n,v)\right\} \nu_P (t_n,dv)\,;
%    \label{e2.14-s}
    \end{multline*}

    \vspace*{-14pt}

    \noindent
\begin{multline}
        \chi^g (\la,z, t) =-\fr{1}{2}\, \la^{\mathrm{T}} b(z,t) \nu_0(t) b(z,t)^{\mathrm{T}}\lambda +{}\\
{}+ \iii_{R_0^q} \left\{ \exp \lk i\la^{\mathrm{T}} c(z,t,v) \rk
    -1 -{}\right.\\
\left.    {}-i\la^{\mathrm{T}} c(z, t,v)\right\} \nu_P (t,dv)\,;
    \label{e2.15-s}
    \end{multline}


    \vspace*{-14pt}

    \noindent
\begin{multline*}
\chi^g_n (\la_n,z_n, t_n) =-\fr{1}{2}\, \la^{\mathrm{T}}_n b(z_n,t) \nu_0(t) b(z_n,t)^{\mathrm{T}}\lambda +{}\\
{}+ \iii_{R_0^q} \left\{ \exp \lk i\la^{\mathrm{T}}_n c(z_n,t_n,v) \rk
    -1 -{}\right.\\
\left.    {}-i\la^{\mathrm{T}}_n c(z_n, t_n,v)\right\} \nu_P (t_n,dv)\,.
%    \label{e2.16-s}
    \end{multline*}
При этом одно- и $n$-мер\-ные плотности и характеристические функции связаны
между собой известными соотношениями:
\begin{align*}
f_1(z;t) &=\displaystyle \fr{1}{(2\pi)^{p}} \iin e^{-i\mu^{\mathrm{T}} z} g_1(\mu;t)\, d\mu\,;\\
    g_1(\la;t) &= \iin e^{i\la^{\mathrm{T}} z} f_1(z;t) \,dz\,;
%    \label{e2.17-s}
    \end{align*}

    \vspace*{-14pt}

    \noindent
\begin{multline*}
f_n( z_1\tr z_n; t_1\tr t_n) ={}\\
{}=\fr{1}{(2\pi)^{pn}} \iin\cdots \iin \exp
    \lf - i \sss_{l=1}^n \la_l^{\mathrm{T}} z_l\rf \times{}\\
{}\times g_n (\la_1\tr \la_n; t_1\tr t_n)\, d\la_1\cdots \la_n\,;
\end{multline*}

\vspace*{-12pt}

\noindent
\begin{multline*}
g_n (\la_1\tr \la_n; t_1\tr t_n) ={}\\
{}=\iin\cdots \iin \exp
    \lf i \sss_{l=1}^n \la_l^{\mathrm{T}} z_l\rf \times{}\\
{}\times f_n (z_1\tr z_n; t_1\tr t_n) \,dz_1\cdots dz_n\,.
%    \label{e2.18-s}
    \end{multline*}

Для нахождения одномерных плотностей $f_1(z,t) \hm= f_1^* (z)$ и характеристических
функций $g_1(\la;t) \hm= g_1^* (\la)$ стохастических процессов в стационарных
ДСтС~(\ref{e2.1-s}), когда
    \begin{equation}
    \left.
    \begin{array}{c}
    a(z,t) = a^*(z)\,;\quad b(z,t)=b^*(z)\,;\\[9pt]
    \chi(\mu;t)= \chi^f(\mu,\zeta, t)={\chi*}^f (\mu, \zeta)\,,
    \end{array}
    \right\}
    \label{e2.19-s}
    \end{equation}
    в~(\ref{e2.5-s}) и (\ref{e2.9-s}) следует положить
$\prt f_1/\prt t \hm= 0$ и $\prt g_1/ \prt t\hm =0$.
В~результате получим соответственно сле\-ду\-ющие интегродифференциальные уравнения:
    \begin{multline*}
    \fr{\prt^{\mathrm{T}}}{\prt z}\lk a^* (z) f_1^* (z)\rk ={}\\
    {}=
    \fr{1}{(2\pi)^p} \iin \iin {\chi^*}^f (\la,\zeta) e^{i\la^{\mathrm{T}}(\zeta-z)} f_1^*
    (\zeta)\, d\zeta d\la\,; %\label{e2.20-s}
    \end{multline*}

    \vspace*{-12pt}

    \noindent
\begin{multline*}
-\fr{1}{(2\pi)^p} \iin  \iin i\la^{\mathrm{T}} a^*(z) e^{i(\la^{\mathrm{T}}-\mu^{\mathrm{T}})z} g_1^*(\mu)
   \, d\mu dz={}\\
{}=\fr{1}{(2\pi)^p} \iin  \iin \chi^{*g} (\la, z) e^{i(\la^{\mathrm{T}}-\mu^{\mathrm{T}})z}
    g_1^*(\mu) \,d\mu dz\,. %\label{e2.21-s}
    \end{multline*}

Пусть функция $a$ в ДСтС~(\ref{e2.1-s}) допускает пред\-став\-ле\-ние
    \begin{equation}
    a= a(z,t) = a_1(z,t) +a_2 (z,t)
    \label{e2.22-s}
\end{equation}
такое, что функция  $f_1\hm=f_1(z;t)$ является плот\-ностью инвариантной
меры не возмущенной шумами сис\-те\-мы, описываемой векторным
обыкновенным дифференциальным уравнением вида
\begin{equation}
\dot z = a_1 (z,t)\,,
\label{e2.23-s}
\end{equation}
т.\,е.\ удовлетворяет следующему условию сохранения инвариантной меры:
    \begin{equation}
   \fr{\prt f_1 (z;t)}{\prt t}+ \fr{\prt^{\mathrm{T}}}{\prt z} \lk
   a_1 (z,t) f_1(z;t)\rk =0\,.\label{e2.24-s}
   \end{equation}

Для гладких функций $a_1\hm=a_1(z,t)$ вопросы существования и основные
свойства интегральных инвариантов и инвариантных мер изучены в~\cite{23-s, 24-s}.
При этом  функция $a_2 \hm= a_2(z,t)$ в~(\ref{e2.22-s})
определяется путем решения следующего интегродифференциального уравнения:
    \begin{multline}
    \fr{\prt^{\mathrm{T}}}{\prt z}\lk a_2 (z,t) f_1(z;t) \rk ={}\\
    \hspace*{-4mm}{}=
    \fr{1}{(2\pi)^k} \iin\iin \chi^f
    (\la,\zeta,t) e^{i\la^{\mathrm{T}}(\zeta-z)} f_1(\zeta;t) \,d\zeta d\la\,.\!
    \label{e2.25-s}
    \end{multline}

Для стационарных ДСтС, когда выполнены условия~(\ref{e2.19-s}),
уравнения~(\ref{e2.22-s})--(\ref{e2.24-s}) имеют вид:
    \begin{gather}
    a(z)= a_1(z) + a_2(z)\,;\label{e2.26-s}
\\
    \dot z = a_1(z)\,;\label{e2.27-s}
\\
    \fr{\prt^{\mathrm{T}}}{\prt z}\lk a_2^*(z) f_1^*(z)\rk = 0\,,
    \label{e2.28-s}
\end{gather}

\vspace*{-12pt}

\noindent
\begin{multline}
\fr{\prt^{\mathrm{T}}}{\prt z}\lk a_2^*(z) f_1^*(z)\rk ={}\\
{}= \fr{1}{(2\pi)^p}
    \iin \iin {\chi^*}^f (\la,\zeta) e^{i\la^{\mathrm{T}}(\zeta-z)} f_1^*(\zeta)\, d\zeta d\la\,.
    \label{e2.29-s}
\end{multline}

Условия сохранения инвариантной меры можно представить в следующем
развернутом виде:
    \begin{equation*}
    \fr{\prt f_1 (z;t)}{\prt t} + A_a f_1 (z;t) =0\,;
    \end{equation*}

\vspace*{-12pt}

\noindent
    \begin{align*}
    A_a f_1(z;t)&=\fr{\prt^{\mathrm{T}}}{\prt z} \lk a_1(z,t) f_1(z;t)\rk =
    \mathrm{div}\, \pi(z;t)\,;
%    \label{e2.30-s}
\\
    A_a^* f_1^* (z) &=0\,;\\
    A_a^* f_1(z)&=\fr{\prt^{\mathrm{T}}}{\prt z} \lk a_1^* (z) f_1^* (z)\rk =
    \mathrm{div}\, \pi^* (z)\,;
%    \label{e2.31-s}
    \end{align*}
    \begin{equation}
    \left.
    \begin{array}{rl}
   \displaystyle \fr{\prt g_1 (\la;t)}{\prt t} - B_a g_1(\la;t) &=0\,;\\[9pt]
    B_a g_1(\la;t) &={}\\[9pt]
    &\hspace*{-30mm}{}=   \displaystyle\fr{1}{(2\pi)^p} \iin\iin\! i\la^{\mathrm{T}} a_1(z,t)
    e^{i(\la^{\mathrm{T}}-\mu^{\mathrm{T}})z}\times{}\\[9pt]
    &\hspace*{1mm}{}\times g_1(\mu;t)\, d\mu dz={}\\[9pt]
&\hspace*{-23mm}{}=    \displaystyle\iin i\la^{\mathrm{T}} a(z,t) e^{i\la^{\mathrm{T}}z} f_1(z;t)\, dz={}\\[9pt]
&\hspace*{-12mm}{}=
        \displaystyle\iin e^{i\la^{\mathrm{T}} z} i\la^{\mathrm{T}} \pi(z;t)\, dz\,;
     \end{array}
     \right\}
    \label{e2.32-s}
\end{equation}

    \vspace*{-12pt}

    \noindent
\begin{multline}
B_a^* g_1^* (\la)=0\,;\quad
    B_a^* g_1^* (\la) = {}\\
    {}=\fr{1}{(2\pi)^p} \iin i\la^{\mathrm{T}} a_1^* (z)
    e^{i(\la^{\mathrm{T}} -\mu^{\mathrm{T}})z} g_1^* (\mu) \,d\mu dz={}\\
\!{}=\iin\! i\la^{\mathrm{T}} a_1^*(z) e^{i\la^{\mathrm{T}}z} f_1^* (z)\, dz ={}\\
{}=
    \iin \!e^{i\la^{\mathrm{T}} z} i\la^{\mathrm{T}} \pi^* (z)\, dz\,.\!\!
    \label{e2.33-s}
\end{multline}

Для разрывных функций $a_1 (z,t)$ в терминах характеристических функций
соотношения~(\ref{e2.24-s}) и~(\ref{e2.28-s}) могут быть записаны в
виде~(\ref{e2.32-s}) и~(\ref{e2.33-s}). При этом для
составляющих $a_2(z,t)$ и $a_2^*(z)$ имеют место уравнения:
    \begin{align}
    B_{a_2} g_1(\la;t) &=\notag\\[6pt]
    &\hspace*{-18mm}{}=\fr{1}{(2\pi)^p}\! \!\iin\iin\!\! \chi^g(\la,z,t)
    e^{i(\la^{\mathrm{T}}-\mu^{\mathrm{T}})z} g_1(\mu;t)\, d\mu dz;\!\!\!
    \label{e2.34-s}\\[9pt]
    B_{a_2}^* g_1^*(\la) &={}\notag\\[6pt]
    &\hspace*{-18mm}{}=\fr{1}{(2\pi)^p} \!\!\iin\iin \!\!{\chi^*}^g(\la,z)
    e^{i(\la^{\mathrm{T}}-\mu^{\mathrm{T}})z} g_1^*(\mu)\, d\mu dz.\!\!\!
    \label{e2.35-s}
\end{align}

Отсюда вытекают  точные алгоритмы аналитического
моделирования распределений с инвариантной мерой. В~их основе лежат
следующие тео\-ремы.
{\looseness=1

}

\smallskip

\noindent
\textbf{Теорема 1.}\ \textit{Функция $f_1\hm=f_1(z;t)$ будет решением~$(\ref{e2.5-s})$
и~$(\ref{e2.6-s})$ тогда и только тогда, когда $a\hm=a(z,t)$ допускает
представление~$(\ref{e2.22-s})$ такое, что $f_1\hm=f_1(z;t)$ \mbox{является} плотностью
инвариантной меры обыкновенного дифференциального уравнения~$(\ref{e2.23-s})$,
т.\,е.\ удовле\-тво\-ря\-ет условию~$(\ref{e2.24-s})$. При этом со\-став\-ля\-ющая~$a_2$
определяется из решения интегродифференциального уравнения}~(\ref{e2.25-s}).

\smallskip

\noindent
\textbf{Теорема 2.}\ \textit{Функция $f_1^*\hm=f_1^*(z)$ будет решением~$(\ref{e2.5-s})$
тогда и только тогда, когда $a^*\hm=a^*(z)$ допускает
представление~$(\ref{e2.26-s})$ такое, что $f_1^*\hm=f_1^*(z)$ является плотностью
инвариантной меры~$(\ref{e2.27-s})$. При этом составляющая~$a_2^{*}$
определяется из решения  уравнения}~(\ref{e2.29-s}).

\smallskip

\noindent
\textbf{Теорема 3.} \textit{Функция $g_1\hm=g_1(\la;t)$ будет решением~$(\ref{e2.9-s})$,
$(\ref{e2.10-s})$ тогда и только тогда, когда недифференцируемая функция
$a\hm=a(z,t)$  допускает пред\-став\-ле\-ние~$(\ref{e2.22-s})$ такое, что
$g_1\hm=g_1(\la;t)$ является характеристической функцией инвариантной
меры уравнения~$(\ref{e2.23-s})$, т.\,е.\
удовлетворяет условию~$(\ref{e2.29-s})$. При этом
со\-став\-ля\-ющая~$a_2$ определяется из уравнения}~(\ref{e2.34-s}).

\smallskip

\noindent
\textbf{Теорема 4.} \textit{Функция $g_1^*\hm=g_1^*(\la)$  будет
решением~$(\ref{e2.28-s})$ тогда и только тогда, когда недифференцируемая
функция $a^*\hm=a^*(z)$  допускает представление~$(\ref{e2.26-s})$
такое, что $g_1^*$ является  характеристической функцией инвариантной меры
уравнения~$(\ref{e2.23-s})$. При этом~$a_2^*$ определяется из решения}~(\ref{e2.35-s}).

\smallskip

Теоремы~1--4 легко обобщаются на случай многомерных распределений
с инвариантной мерой.

\section{Приближенные методы и~алгоритмы аналитического моделирования
распределений процессов с~инвариантной мерой в~дифференциальных стохастических
системах}

Пусть нелинейная ДСтС~(\ref{e2.1-s})
допускает применение метода нормальной аппроксимации (МНА)~\cite{1-s, 2-s}.
Тогда одно- и двумерные нормальные плотности $f_1^\mna$,
 $f_2^\mna$ и характеристические функции  $g_1^\mna$,  $g_2^\mna$,
 а также вектор математического ожидания $m_t \hm= {\sf M}^\mna Z(t)$, ковариационная
 мат\-ри\-ца $K_t \hm= {\sf M}^\mna Z^{0T} Z^0 (t)$ $(Z^0 (t) \hm= Z(t) \hm- m_t)$
 и мат\-ри\-ца ковариационных функций
 $K(t_1, t_2) \hm= {\sf M}^\mna Z^{0T} (t_1) Z^0 (t_2)$ $(t_1< t_2)$
 определяются следующими уравнениями:
    \begin{multline}
    f_1^\mna = f_1^\mna (z;t, m_t, K_t) =
    \left[ (2\pi)^p |K_t|\right]^{-1/2}\times{}\\
    {}\times
     \exp \lf -\fr{1}{2} (z^{\mathrm{T}} - m_t^{\mathrm{T}}) K_t^{-1}
    (z-m_t)\rf\,;
    \label{e3.1-s}
    \end{multline}


\vspace*{-12pt}

    \noindent
    \begin{multline*}
    f_2^\mna = {}\\
    {}=f_2^\mna (z_1, z_2;t_1, t_2, m_{t_1}, m_{t_2}, K_{t_1},
    K_{t_2}, K(t_1, t_2))={}\hspace*{-0.62715pt}\\
{}=\left[ \left(2\pi\right)^p |\bar K_2|\right]^{-1/2}  \exp
\left( - \left( \left[ z_1^{\mathrm{T}} z_2^{\mathrm{T}} \right] -
    \bar m_2^{\mathrm{T}}\right)\right.\times{}\\
    \left.{}\times \bar K_2^{-1}\left(\left[
    z_1^{\mathrm{T}} z_2^{\mathrm{T}}\right]^{\mathrm{T}}-\bar m_2\right)\right)\,;
    \label{e3.2-s}
    \end{multline*}
    \begin{equation}
    g_1^\mna (\la;t)=\exp\lf i\la^{\mathrm{T}} m- \fr{1}{2}\, \la^{\mathrm{T}} K_t \la\rf\,;
    \label{e3.3-s}
    \end{equation}

    \vspace*{-12pt}

    \noindent
\begin{multline}
    g_2^\mna (\la_1, \la_2; t_1,t_2) = {}\\
    {}=\exp \lf i \bar \la^{\mathrm{T}} \bar m_2 -
    \fr{1}{2}\, \bar \la^{\mathrm{T}} \bar K_2 \bar \la\rf\,,\enskip
\bar \la =\lk \la_1^{\mathrm{T}} \la_2^{\mathrm{T}}\rk^{\mathrm{T}}\,;
    \label{e3.4-s}
    \end{multline}


\vspace*{-12pt}

\noindent
\begin{multline}
\dot m_t = \Phi_1 (t, m_t, K_t) ={}\\
{}=\iin a(z,t) f_1^\mna (z; t, m_t, K_t) \,dz\,;
    \label{e3.5-s}
    \end{multline}

\vspace*{-12pt}

\noindent
\begin{multline}
\dot K_t = \Phi_2(t, m_t, K_t) = \Phi_{21} + \Phi_{12}+\Phi_{22}={}\\
{}=\left[ \iin a(z,t) (z^{\mathrm{T}}-m_t^{\mathrm{T}}) + (z-m_t)
a^{\mathrm{T}} (z,t) +{}\right.\\
\left.{}+ \vphantom{\iin}
    \bar \si (z,t)\right] f_1^\mna (z;t, m_t, K_t)\, dz\,,\label{e3.6-s}
    \end{multline}

    \vspace*{-12pt}

    \noindent
\begin{multline*}
\fr{\prt K(t_1, t_2)}{\prt t_2} ={}\\
{}= \Phi_3 (t_1, t_2, m_{t_1},m_{t_2},
    K_{t_1}, K_{t_2}, K(t_1,t_2))={}
    \end{multline*}

    \noindent
    \begin{multline}
    {}=
    \lk (2\pi)^{2p} |\bar K_2|\rk^{-1/2}
    \iin\iin \left(z_1-m_{t_1}\right) a\left(z_2, t_2\right)\times{}\\
    {}\times
    \exp\left\{ - \left(\left[z_1^{\mathrm{T}} z_2^{\mathrm{T}}\right]-
    \bar m_2^{\mathrm{T}}\right)\bar K_2^{-1} \times{}\right.\\
\left.    {}\times
    \left(\left[z_1^{\mathrm{T}} z_2^{\mathrm{T}}\right]-
    \bar m_2\right)\right\}\, dz_1 dz_2={}\\
{}= K(t_1, t_2) K(t_2)^{-1} \Phi_{21} (m(t_2), K(t_2), t_2)^{\mathrm{T}}.
    \label{e3.7-s}
    \end{multline}
Здесь введены следующие обозначения:
    \begin{equation}
    \left.
    \begin{array}{c}
    z_1=z_{t_1}\,;\enskip
    z_2=z_{t_2}\,;\enskip \bar m_2 =\lk m_{t_1}^{\mathrm{T}} m_{t_2}^{\mathrm{T}}\rk^{\mathrm{T}}\,;\\[9pt]
    \bar K_2 =\begin{bmatrix}
        K(t_1,t_1)&K(t_1, t_2)\\[3pt]
        K(t_2, t_1)& K(t_2, t_2)\end{bmatrix}\,;
        \end{array}
        \right\}
        \label{e3.8-s}
\end{equation}
    \begin{equation}
    \left.
    \begin{array}{rl}
    \bar \si (z,t) &={}\\
&\hspace*{-10mm}{}=\displaystyle
    \si(z,t) +\iii_{R_0^q} c(z,t,v) c(z,t,v)^{\mathrm{T}} \nu_P (t,dv)\,;
    \\[9pt]
     \si(z,t) &= b(z,t) \nu_0(t) b(z,t)^{\mathrm{T}}\,.
     \end{array}
     \right\}
    \label{e3.9-s}
    \end{equation}

Для стационарных ДСтС  при $\dot m^* \hm=0$, $\dot K^* \hm=0$,
$K(t_1, t_2)\hm= k(\tau)$
$(\tau\hm=t_1-t_2)$  соотношения~(\ref{e3.5-s})--(\ref{e3.9-s}) принимают вид:
    \begin{equation}
    \Phi_1^* (m^*, K^*) =0\,;\label{e3.10-s}
    \end{equation}
\begin{equation}
\Phi_2^*(m^*, K^*) =0\,;
\label{e3.11-s}
\end{equation}
\begin{equation}
\fr{dk(\tau)}{ d\tau} = \Lambda k(\tau)\,;\enskip
    \Lambda =\Phi_{21}(m^*, K^*) K^{*-1} k(\tau)\,;
    \label{e3.12-s}
    \end{equation}
    $$k(\tau) = k(-\tau)^{\mathrm{T}},\enskip k(0)=K\,.$$
Из~(\ref{e3.12-s}) следует, что алгоритм МНА будет устойчивым,
если матрица $\Lambda^* \hm= \Lambda (m^*, K^*)\hm=\Phi_{21}(m^*, K^*) K^{* -1}$
будет асимптотически устойчива.

Уравнения метода статистической линеаризации (МСЛ) в нелинейных ДСтС  при
аддитивных шумах, когда $b(z,t) \hm= b_0(t)$, $b^*(z)\hm=b_0^*$
получаются из~(\ref{e3.5-s})--(\ref{e3.7-s}) и~(\ref{e3.10-s})--(\ref{e3.12-s})
как частный случай.

Условия наличия нормального распределения с инвариантной мерой,
если заменить $a(z,t)$ статистически линеаризованным выражением вида:
    \begin{equation*}
    a(Z,t)\approx a_{10}^\mna (t, m_t, K_t) + a_{11}^\mna (t, m_t, K_t) (Z-m_t)\,,
%    \label{e3.13-s}
    \end{equation*}
где
\begin{equation*}
a_{10}^\mna =a_{10}^\mna (t, m_t, K_t)\,; %\label{e3.14-s}
\end{equation*}

%\vspace*{-12pt}

    \noindent
\begin{multline*}
a_{11}^\mna=a_{11}^\mna (t, m_t, K_t) ={}\\
\hspace*{-3.71278pt}{}=  \!  \left[ \iin\! \!a(z,t) (z^{\mathrm{T}}-m_t^{\mathrm{T}}) f_1^\mna (z; t , m_t, K_t)\, dz\right]\! K_t^{-1} ={}\\
{}=\lk \fr{\prt}{\prt m_t}\left( a_{10}^\mna\right)^{\mathrm{T}}\rk^{\mathrm{T}}\,,
%\label{e3.15-s}
\end{multline*}
приводят к следующим соотношениям:
\begin{multline}
\fr{\prt f_1^\mna (z; t, m_t, K_t)}{\prt t} +{}\\
{}+\fr{\prt^{\mathrm{T}}}{\prt z} \left\{\! \left[ a_{10}^\mna (t, m_t, K_t) +
     a_{11}^\mna (t, m_t, K_t)\times{}\right.\right.\\
\left.\left.     {}\times (z-m_t)
\vphantom{a_{10}^\mna}
\right]
      f_1^\mna ( z; t , m_t, K_t)\right\} =0\,;
     \label{e3.16-s}
     \end{multline}

\vspace*{-12pt}

\noindent
\begin{multline}
\fr{\prt^{\mathrm{T}}}{\prt z} \left\{ \left[ a_{10}^{*\mna}(m^*, K^*) +
    a_{11}^{*\mna}(m^*, K^*)\times{}\right.\right.\\
\left.\left.    {}\times (z-m^*)
\vphantom{a_{10}^\mna}
\right] f_1^{*\mna}(z; m^*, K^*)\right\} =0\,,
    \label{e3.17-s}
    \end{multline}
где

\noindent
\begin{multline*}
    f_1^{*\mna} (z; m^*, K^*) = \lk (2\pi)^p |K^*|\rk^{-1/2}\times{}\\
    {}\times
    \exp \left\{ -\fr{1}{2} \left(z^{\mathrm{T}}-m^{*T}\right)(K^*)^{-1} (z-m^*)\right\}.
    \end{multline*}

Аналогично в развернутом виде выписываются условия~(\ref{e2.32-s})
и~(\ref{e2.33-s}):

\noindent
    \begin{multline}
    \fr{\prt g_1^\mna (\la;t)}{\prt t} -\iin i\la^{\mathrm{T}}
    \left[ a_{10}^\mna (m_t, K_t, t) +{}\right.\\
\left.    {}+ a_{11}^\mna (m_t, K_t, t) (z- m_t) \right]\times{}\\
{}\times e^{i\la^{\mathrm{T}} z} f_1^\mna (z; m_t, K_t, t) dz=0\,,
    \label{e3.18-s}
    \end{multline}

    \vspace*{-14pt}

    \noindent
\begin{multline}
\!\!\!\!\!\!\iin \!\!\!i\la^{\mathrm{T}}\! \left[ a_{10}^{*\mna } (m^*, \!K^*) +  a_{11}^{*\mna } (m^*, K^*)
(z-m^*)\right]\times{}\\
{}\times
e^{i\la^{\mathrm{T}}z} f_1^{*\mna } (z; m^*, K^*) dz = 0\,.
    \label{e3.19-s}
    \end{multline}

Отсюда вытекают следующие теоремы, лежащие в основе приближенных
нелинейных методов.

\smallskip

\noindent
\textbf{Теорема 5.} \textit{Если существуют одно- и двумерные  плотности
стохастического процесса, а  матрица $a_{11}^\mna$ коэффициентов
статистической линеаризации асимптотически устойчива,
то приближенный \mbox{алгоритм} аналитического моделирования МНА
нестационарных стохастических процессов в ДСтС~$(\ref{e2.1-s})$ с инвариантной
мерой определяется выражениями~$(\ref{e3.1-s})$--$(\ref{e3.7-s})$ и}~(\ref{e3.16-s}).

\smallskip

\noindent
\textbf{Теорема 6.} \textit{Если существуют стационарные одно- и
двумерные плотности стохастического процесса, а матрица
$a_{11}^{*\mna}$ коэффициентов статистической линеаризации
асимптотически устойчива, то приближенный алгоритм аналитического
моделирования стационарных стохастических процессов с инвариантной
мерой в стационарной ДСтС~$(\ref{e2.1-s})$ определяется
выражениями}~$(\ref{e3.10-s})$--$(\ref{e3.12-s})$ \textit{и}~(\ref{e3.17-s}).

\smallskip

Как известно~\cite{1-s, 2-s}, одно- и двумерные нормальные распределения
определяют и все  $n$-мер\-ные распределения $(n\hm> 3)$. Поэтому МНА и
МСЛ  при $b(Y,t)\hm=b_0(t)$, $c(Y,t,z) \hm=c_0 (t,v)$ дают приближенные
алгоритмы для любых многомерных плот-\linebreak\vspace*{-12pt}
\columnbreak

\noindent
ностей стохастических процессов,
если они существуют. Аналогично формулируются теоремы~3.3 и~3.4 в
терминах характеристических функций на основе условий~(\ref{e3.18-s}) и~(\ref{e3.19-s}).

 Обобщением МНА являются различные
приближенные методы, основанные на параметризации распределений~\cite{1-s, 2-s}.
Аппроксимируя одномерную характеристическую функцию $g_1 (\la;t)$
и соответствующую плотность $f_1 (z,t)$ известными функциями
 $g_1^* (\la;\theta)$ и $f_1^* (z;\theta)$,  зависящими от
конечномерного векторного параметра~$\theta$, можно свести задачу
приближенного определения одномерного распределения к выводу из
уравнения для характеристических функций обыкновенных
дифференциальных уравнений, определяющих~$\theta$ как функцию
времени. Это относится и к остальным многомерным распределениям.

При аппроксимации многомерных распределений целесообразно выбирать
последовательности функций $\{ f_n^* (z_1,\ldots,z_n;\theta_n)\}$ и
$\{g_n^* (\la_1\tr \la_n;\theta_n)\}$, каждая пара
которых находилась бы в такой  зависимости от векторного параметра~$\theta_n$,
чтобы при любом $n$ множество параметров, образу\-ющих
вектор~$\theta_n$, включало в качестве подмножества множество
параметров, образующих вектор~$\theta_{n-1}$. То-\linebreak гда при
аппроксимации $n$-мер\-но\-го распределения придется определять только
те координаты вектора~$\theta_n$, которые не были определены ранее
при аппроксимации функций $f_1, g_1\tr f_{n-1}$, $g_{n-1}$. %\linebreak
В зависимости от того, что представ\-ляют собой параметры, от
которых зависят функции $f_n^* (z_1\tr z_n;\theta_n)$ и $g_n^*
(\la_1\tr \la_n;\theta_n)$, аппроксими\-ру\-ющие неизвестные
многомерные плотности $f_n (z_1,  \ldots,z_n; t_1 \tr t_n)$ и
характеристические функции $g_n (\la_1\tr \la_n; t_1,\ldots,t_n)$,
используются различные приближенные методы решения
 уравнений, определяющих многомерные
распределения вектора состояния системы~$X_t$, в частности методы
моментов, семиинвариантов, ортогональных разложений, квазимоментов
и~др.~\cite{1-s, 2-s} и в условиях сохранения инвариантной меры.

\vspace*{-20pt}

\section{Анализ и~моделирование распределений с~инвариантной мерой в~эредитарных
стохастических системах,
приводимых к~дифференциальным}

\vspace*{-9pt}
Рассмотрим ЭСтС, описываемую интегродифференциальным уравнением Ито
следующего вида~\cite{21-s}:

\noindent
    \begin{multline}
    d X =\lk a (X,t) +\iii_{t_0}^t a_1 (X(\tau),\tau, t)\, d \tau\rk dt+{}\\
{}+\lk b(X, t) +\iii_{t_0}^t b_1 (X(\tau), \tau, t) \,d \tau\rk dW_0+{}\\
\hspace*{-5mm}{}+\iii_{R_0^q} \lk c(X,t,v) +\iii_{t_0}^t c_1 (X(\tau),  \tau, t,v) \rk
    d P^0 (t, dv)\!\label{e4.1-s}
    \end{multline}
с начальным условием $X(t_0)\hm= X_0$.

В~(\ref{e4.1-s}) приняты следующие обозначения и допущения: $X\hm=X(t)$~---
$p$-мер\-ный вектор состояния;
    $W_0$~--- $r$-мер\-ный винеровский процесс интенсивности $\nu_0 \hm= \nu_0 (t)$;
    $ \iii_{\Delta_t} d P^0 (t, A)$~--- центрированная пуассоновская мера,
    удовлетворяющая условию~(\ref{e2.2-s}).

Функции $a=a(X, t)$, $a_1 \hm= a_1(X (\tau),\tau, t)$, $b\hm=b(X, t)$,
$b_1 \hm= b_1(X (\tau),\tau, t)$, $c\hm=c(X,t,v)$ и
$c_1 \hm= c_1(X (\tau),\tau, t,v)$ имеют соответственно размерности
$p\times 1$, $p\times 1$, $p\times r$, $p\times r$, $p\times 1$ и
$p\times 1$ и допускают представления следующего вида:
    \begin{equation}
    \left.
    \begin{array}{rl}
    a_1&=A(t,\tau) \vrp (X(\tau) , \tau)\,;\\[9pt]
    b_1&=B(t,\tau) \psi (X(\tau) ,  \tau)\,;\\[9pt]
    c_1&=C(t,\tau) \chi (X(\tau) ,  \tau, v)\,.
    \end{array}
    \right\}
    \label{e4.2-s}
    \end{equation}
Здесь эредитарные ядра $A\hm=A(t,\tau)\hm=\lk A_{ij}(t,\tau)\rk$
$(i,j\hm=\overline{1,p})$,
$B\hm=B(t,\tau)\hm=\lk B_{i l}(t,\tau)\rk$ $(i\hm=\overline{1,p}$,
$l\hm=\overline{1,r})$ и $C\hm=C(t,\tau)=\lk C_{ij}(t,\tau)\rk$
$(i,j\hm=\overline{1,p})$ имеют соответственно размерности
$p\times p$, $p\times r$ и $p\times p$. Они удовлетворяют следующим условиям
физической реализуемости и асимптотического затухания:
    \begin{equation}
    \left.
    \begin{array}{c}
    A_{ij}(t,\tau)=0\,;\enskip
    B_{i l}(t,\tau)=0\,;\\[9pt]
    C_{ij}(t,\tau)=0\enskip \forall \tau >t\,;
    \end{array}
    \right\}
    \label{e4.3-s}
    \end{equation}
\begin{equation}
\left.
\begin{array}{rl}
\displaystyle\iin \lv A_{ij} (t,\tau) \rv d\tau &<\infty \,;\\[9pt]
\displaystyle\iin \lv B_{i l} (t,\tau) \rv d\tau &<\infty \,;\\[9pt]
\displaystyle \iin \lv C_{ij} (t,\tau) \rv d\tau &<\infty\,.
 \end{array}
 \right\}
 \label{e4.4-s}
 \end{equation}
При этом нелинейные в общем случае функции
$\vrp\hm=\vrp(X(\tau),\tau)$, $\psi \hm=\psi(X(\tau), \tau)$ и $\chi \hm=\chi
(X(\tau),  \tau, v)$ имеют размерности $p\times 1$, $p\times p$ и
$p\times 1$ соответст\-венно.

В случае если эредитарные ядра $A$, $B$, $C$ удовле\-тво\-ря\-ют условиям

\noindent
    \begin{align*}
    A_{ij} (t,\tau) &=\tilde A_{ij} (u)\,;\\
 B_{i l} (t,\tau) &=\tilde B_{i l} (u)\,;\\
    C_{ij} (t,\tau)& =\tilde C_{ij} (u)\enskip (u=t-\tau)\,,
   %    \label{e4.5-s}
    \end{align*}
то говорят об ЭСтС со стационарным затуханием.

Важный класс ядер представляют собой сингулярные (вырожденные) ядра,
когда имеют место представления:
\begin{equation}
\left.
\begin{array}{rl}
A_{ij} (t,\tau) &= A_{ij}^+(t) A_{ij}^-(\tau)\,;\\[9pt]
    B_{i l} (t,\tau) &= B_{il}^+(t) B_{il}^-(\tau)\,;\\[9pt]
    C_{ij} (t,\tau) &= C_{ij}^+ ( t) C_{ij}^- (\tau)
    \end{array}
    \right\}
    \label{e4.6-s}
    \end{equation}
$(i,l= \overline{1,p}$; $j=\overline{1,r}).$

В случае, когда подынтегральные функции  $c(X, t, v)$ и  $c_1(X(\tau), \tau, v)$
в~(\ref{e4.1-s}) допускают пред\-став\-ле\-ния

\noindent
    \begin{align*}
    c(X,t, v)&=b(X, t)c'(v)\,;\\
    c_1(X(\tau), \tau, v)&=b(X(\tau),\tau)c'(v)\,,
%    \label{e4.7-s}
    \end{align*}
ЭСтС~(\ref{e4.1-s}) приводится к виду:

\noindent
\begin{multline}
\dot X =  a(X, t)+\iii_{t_0}^t a_1 (X(\tau),\tau, t)\,d\tau
    +{}\\
    {}+\lk b(X, t)+ \iii_{t_0}^t b_1 (X(\tau),\tau, t)\,d\tau\rk V\,,
    \label{e4.8-s}
    \end{multline}
если принять

\noindent
    $$
    V=\dot W\,;\enskip W(t) = W_0(t) +\iii_{R_0^q} c' (v) P^0 (t, dv)\,.$$

В~\cite{21-s} решена задача приведения ЭСтС~(\ref{e4.1-s}) при
условиях~(\ref{e4.2-s})--(\ref{e4.4-s})
и~(\ref{e4.2-s})--(\ref{e4.4-s}), (\ref{e4.6-s}) к ДСтС~(\ref{e2.1-s}),
а также установлены следующие утверждения.

Рассмотрим сначала ЭСтС~(\ref{e4.1-s}) при условиях~(\ref{e4.2-s})--(\ref{e4.4-s}).
Будем считать, что эредитарные ядра  $A(t,\tau)$, $B(t,\tau)$, $C(t,\tau)$
удовлетворяют следующим нестационарным линейным операторным уравнениям:

\noindent
    \begin{align*}
    F^{At}A(t,\tau) &= H^{At} \delta (t-\tau)\,;\\
    F^{Bt}B(t,\tau) &= H^{Bt} \delta (t-\tau)\,;\\
    F^{Ct}C(t,\tau) &= H^{Ct} \delta (t-\tau)\,;\\
    A(t,\tau)&= A'(t,\tau)^{\mathrm{T}} (H^{A*\tau})^{\mathrm{T}}\,;
\\
    A'(t,\tau)^{\mathrm{T}} (F^{A*\tau})^{\mathrm{T}}&=    I_h^A\delta(t-\tau)\,;\\
B(t,\tau)&= B'(t,\tau)^{\mathrm{T}} (H^{B*\tau})^{\mathrm{T}}\,;\\
    B'(t,\tau)^{\mathrm{T}} (F^{B*\tau})^{\mathrm{T}}&= I_h^B\delta(t-\tau)\,;
\\
    C(t,\tau)&= C'(t,\tau)^{\mathrm{T}} (H^{C*\tau})^{\mathrm{T}}\,;\\
    C'(t,\tau)^{\mathrm{T}} (F^{C*\tau})^{\mathrm{T}}&= I_h^C\delta(t-\tau)\,.
%    \label{e4.10-s}
    \end{align*}
Здесь $F^A$, $H^A$, $F^B$, $H^B$, $F^C$ и~$H^C$~---
известные матричные дифференциальные операторы размерности
$h_A\times h_A$, $h_B\times h_B$, $h_C\times h_C$ порядка $n_A,
m_A$, $n_B, m_B$, $n_C, m_C$, ($n_A\hm>m_A$, $n_B\hm>m_B$, $n_C\hm>m_C$)
соответственно:
\begin{equation}
\hspace*{-3.8mm}\left.
\begin{array}{rl}
F^A = F^A (t,D)&=\displaystyle\sss_{l=0}^{n_A} \alp_l^A (t) D^l\,;\\[9pt]
H^A=H^A(t,D) &=\displaystyle\sss_{l=0}^{m_A} \beta_l^A (t) D^l\,;
\\[9pt]
F^B = F^B (t,D)&=\displaystyle\sss_{l=0}^{n_B} \alp_l^B (t) D^l\,;
\\[9pt]
    H^B=H^B(t,D) &=\displaystyle\sss_{l=0}^{m_B} \beta_l^B (t) D^l\,;
\\[9pt]
F^C = F^C (t,D)&=\displaystyle\sss_{l=0}^{n_C} \alp_l^C (t) D^l\,;
\\[9pt]
    H^C=H^C(t,D) &=\displaystyle\sss_{l=0}^{m_CA} \beta_l^C (t) D^l\,;
    \end{array}
    \right\}\!
     \label{e4.11-s}
    \end{equation}
индекс~$t$ у операторов означает, что оператор
действует на функцию от~$t$ при фиксированном~$\tau$; звездочкой
обозначен символ сопряжения оператора; $I_h^A$, $I_h^B$, $I_h^C$~---
единичные $(h\times h)$-мат\-ри\-цы. Введем $h^A$-, $h^B$-, $h^C$-мер\-ные
векторы посредством соотношений:
\begin{align*}
Z_1' &= U' =\iii_{t_0}^t A(t,\tau) \vrp(X(\tau), \tau)\,d\tau\,;\\
Z_1''&= U'' =\iii_{t_0}^t B(t,\tau) \psi(X(\tau), \tau)\,d\tau\,;\\
Z_1'''&= U''' =\iii_{t_0}^t C(t,\tau) \chi(X(\tau), \tau,v)\,d\tau\,.
%    \label{e4.12-s}
    \end{align*}
Эти переменные  $Z'$, $Z''$, $Z'''$ будут удовлетворять следующим линейным
дифференциальным уравнениям:
\begin{align*}
F^A(t, D) Z_1' &= H^A (t, D) \vrp (X, t)\,;\\
    F^B(t, D) Z_1'' &= H^B (t, D) \psi (X, t)\,;\\
F^C(t, D) Z_1''' &= H^C (t, D) \chi (X, t,v)\,.
% \label{e4.13-s}
 \end{align*}
Тогда ЭСтС~(\ref{e4.1-s}) приводится к искомой ДСтС для расширенного вектора
состояния $Z\hm= \lk X^{\mathrm{T}} {Z_1'}^{\mathrm{T}}\, {Z_1''}^{\mathrm{T}}\,{Z_1'''}^{\mathrm{T}}\rk^{\mathrm{T}}$:

\noindent
    \begin{multline}
    dZ = a_1^z (Z, t)\, dt + b_1^z (Z, t)\, dW_0+{}\\
    {}+
    \iii_{R_0^q} c_1^z (Z,  t, v) \,d P^0 (t, dv)\,.
    \label{e4.14-s}
    \end{multline}
Для случая  $h_A \hm= h_B \hm= h_C=h$, $n_A\hm= n_B\hm=n_C\hm=n$,
$m_A\hm=m_B\hm=m_C\hm=m$ в подробной записи функции $a^z_1 (Z,t)$,
$b^z_1 (Z,t)$, $c^z_1 (Z, t,v)$ имеют следующий вид:
\begin{equation}
\left.
\begin{array}{c}
a_1^z (Z, t)=\begin{bmatrix}
        a(X, t)+ Z_1'\\
        a'(t)Z_1'\\
        a''(t) Z_1''\\
        a'''(t)Z_1'''\end{bmatrix}\,;\\[20pt]
    b_1^z (Z, t)=\begin{bmatrix}
        b(X, t)+ Z_1''\\
        b''(t)Z_1''\\
        0\\
        0\end{bmatrix}\,;\\[20pt]
c_1^z (Z, t,v)=\begin{bmatrix}
        c(X, t,v)+ Z_1'''\\
        c'''(t)Z_1'''\\
        0\\
        0\end{bmatrix}\,.
        \end{array}
        \right\}
        \label{e4.15-s}
        \end{equation}
При условии существования обратных матриц $(\alp_n^A)^{-1}$,
 $(\alp_n^B)^{-1}$, $(\alp_n^C)^{-1}$ входящие в~(\ref{e4.15-s})
 переменные и коэффициенты допускают  следующую запись:
\begin{equation}
\left.
\begin{array}{rl}
Z_{j+1}' &=\dot Z_j' - q_j' \vrp(X,t)\,;\\[9pt]
    Z_{j+1}'' &=\dot Z_j'' - q_j'' \psi(X,t)\,;\\[9pt]
 Z_{j+1}''' &=\dot Z_j'''- q_j''' \chi(X,t,v)\enskip (j=\overline{1,(n-1)})\,;
\end{array}
\right\}
 \label{e4.16-s}
 \end{equation}

 \vspace*{-13pt}

\noindent
\begin{multline}
a'(t) ={}\\[-3pt]
\hspace*{-2mm}{}=\!\!\begin{bmatrix}
    I_h&0&\cdots&0\\
    0&I_h&\ddots&0\\
    \vdots&\ddots&\ddots&\cdots\\
    0&0&\cdots&I_h\\
    -(\alp_n^A)^{-1}\alp_0^A& -(\alp_n^A)^{-1}\alp_1^A&\cdots&-(\alp_n^A)^{-1} \alp_{n-1}^A
    \end{bmatrix}\!\!;\!\!\!\!
    \end{multline}

 \vspace*{-13pt}

\noindent
\begin{multline}
a''(t) ={}\\[-3pt]
\hspace*{-1.5mm}{}=\!\!\begin{bmatrix}
    I_h&0&\cdots&0\\
    0&I_h&\ddots&0\\
    \vdots&\ddots&\ddots&\cdots\\
    0&0&\cdots&I_h\\
    -(\alp_n^B)^{-1}\alp_0^B& -(\alp_n^B)^{-1}\alp_1^B&\cdots&-(\alp_n^B)^{-1}\alp_{n-1}^B
    \end{bmatrix}\!\!;\!\!\!
    \end{multline}

 \vspace*{-13pt}

\noindent
\begin{multline}
a'''(t) ={}\\[-3pt]
\hspace*{-2mm}{}=\!\!\begin{bmatrix}
    I_h&0&\cdots&0\\
    0&I_h&\ddots&0\\
    \vdots&\ddots&\ddots&\cdots\\
    0&0&\cdots&I_h\\
    -(\alp_n^C)^{-1}\alp_0^C& -(\alp_n^C)^{-1}\alp_1^C&\cdots&-(\alp_n^C)^{-1}\alp_{n-1}^C
    \end{bmatrix}\!\!;\!\!\!
    \label{e4.17-s}
    \end{multline}

 \vspace*{-12pt}

\noindent
\begin{multline}
    q_j' = (\alp_n^A)^{-1} \left[
    \vphantom{    \sss_{l=0}^{j-i}}
    \beta_{n-j}^A -{}\right.\\
\left.    {}-\sss_{i=0}^{j-1}
    \sss_{l=0}^{j-i} {\cal C}_{n-j-l}^{n-j} \alp_{n-j+i+l}^A {q_i'}^{(l)}\right]\,;
\end{multline}

%\vspace*{-6pt}

    \noindent
    \begin{equation}
    q_n' = (\alp_n^A)^{-1} \lk \beta_{0}^A -\sss_{i=0}^{n-1} \sss_{l=0}^{n-i}
    \alp_{i+l}^A {q_i'}^{(l)}\rk\,;
    \end{equation}

     \vspace*{-12pt}

\noindent
\begin{multline}
    q_j'' = (\alp_n^B)^{-1} \left[
        \vphantom{    \sss_{l=0}^{j-i}}
        \beta_{n-j}^B -{}\right.\\[3pt]
\left.    {}-\sss_{i=0}^{j-1}
    \sss_{l=0}^{j-i} {\cal C}_{n-j-l}^{n-j} \alp_{n-j+i+l}^A {q_i''}^{(l)}\right]\,;
    \end{multline}

    \vspace*{-6pt}

    \begin{equation}
    q_n'' = (\alp_n^B)^{-1} \lk \beta_{0}^B -\sss_{i=0}^{n-1}
    \sss_{l=0}^{n-i}  \alp_{i+l}^A {q_i''}^{(l)}\rk\,;
    \end{equation}

    \vspace*{-12pt}

    \noindent
    \begin{multline}
    q_j''' = (\alp_n^C)^{-1} \left[
        \vphantom{    \sss_{l=0}^{j-i}}
         \beta_{n-j}^C -{}\right.\\[3pt]
\left.    {}-\sss_{i=0}^{j-1}
    \sss_{l=0}^{j-i} {\cal C}_{n-j-l}^{n-j} \alp_{n-j+i+l}^C {q_i'''}^{(l)}\right]\,;
    \end{multline}

\noindent
\begin{equation}
    q_j''' = (\alp_n^C)^{-1} \lk \beta_{0}^C -\sss_{i=0}^{n-1} \sss_{l=0}^{n-i}
    \alp_{i+l}^C {q_i'''}^{(l)}\rk\,.
    \label{e4.18-s}
    \end{equation}
Здесь $C_m^n=n!/(m!(n-m)!)$; индекс~$l$ означает, что суммирование
проводится по всем индексам, исключая~$l$.

Таким образом, справедливо следующее утверждение.

\medskip

\noindent
\textbf{Теорема 7.} \textit{Пусть  ядра $A(t,\tau)$, $B(t,\tau)$,
$C(t,\tau)$ в ЭСтС~$(\ref{e2.1-s})$ удовлетворяют
условиям~$(\ref{e4.2-s})$--$(\ref{e4.4-s})$
или~$(\ref{e4.15-s})$, причем матрицы  $\alp_n^A$, $\alp_n^B$, $\alp_n^C$
в~$(\ref{e4.11-s})$ обратимы, а функции  $\vrp$, $\psi$, $\chi$
дифференцируемы по переменным расширенного вектора состояния достаточное число раз.
Тогда ЭСтС~$(\ref{e4.1-s})$ приводится к ДСтС~$(\ref{e4.14-s})$
на основе}~(\ref{e4.15-s})--(\ref{e4.18-s}).

\medskip

\noindent
\textbf{Замечание 1.} Векторное уравнение~(\ref{e4.14-s})
всегда линейно относительно $Z_1'$, $Z_1''$, $Z_1'''$, но в общем
случае нелинейно относительно~$X$.

\medskip

В том случае, когда выполнены условия~(\ref{e4.2-s})--(\ref{e4.4-s}),
а функции $\vrp$, $\psi$, $\chi$ не дифференцируемы по переменным расширенного
вектора состояния, целесообразна аппроксимация вырожденными ядрами~(\ref{e4.6-s}).
В~этом случае имеют место следующие соотношения:
   \begin{multline}
    dZ= a_2^z (Z,t)\, dt+ b_2^z(Z,t)\, dW_0 +{}\\
    {}+
    \iii_{R_0^q} c_2^z (Z,t,v)\,dP^0 (t, dv)\,;
    \label{e4.19-s}
    \end{multline}

\vspace*{-9pt}

\noindent
\begin{equation}
Z=\lk X^{\mathrm{T}} {Y'}^{\mathrm{T}} {Y''}^{\mathrm{T}} {Y'''}^{\mathrm{T}}\rk^{\mathrm{T}}\,;
    \label{e4.20-s}
    \end{equation}
\begin{equation}
\left.
\begin{array}{rl}
\displaystyle\iii_{t_0}^t A(t,\tau) \vrp(X(\tau),  \tau) d\tau &= A^+ Y';\\[9pt]
\displaystyle \iii_{t_0}^t B(t,\tau) \psi(X(\tau),  \tau) d\tau &= B^+ Y'';\\[9pt]
\displaystyle\iii_{t_0}^t C(t,\tau) \chi(X(\tau),  \tau,v) d\tau &= C^+ Y''';
\end{array}
\right\}
    \label{e4.21-s}
    \end{equation}

    \vspace*{-6pt}


        \begin{gather*}
\dot Y'=A^-\vrp\,;\enskip  Y'(t_0)=0\,;\enskip
        \dot Y''=B^-\psi\,;\\[2pt]
        Y''(t_0)=0\,;\enskip
        \dot Y'''=C^-\chi\,;\enskip Y'''(t_0)=0\,;
%        \label{e4.22-s}
        \end{gather*}

        \vspace*{-3pt}

        \begin{equation}
        \left.
        \begin{array}{rl}
        a_2^z (Z,t)&=\displaystyle\begin{bmatrix}
        a(X,t)+A^+\vrp\\
        A^-\vrp\\
        B^- \psi\\
        C^-\chi\end{bmatrix}\,;\\[21pt]
    b_2^z (Z,t)&=\displaystyle\begin{bmatrix}
        b(X,t)+B^+\psi\\
        0\\
        0\\
        0\end{bmatrix}\,; \\[21pt]
      c_2^z (Z,t,v)&\displaystyle=\begin{bmatrix}
        c(X,t,v)+C^+\chi\\
        0\\
        0\\
        0\end{bmatrix}\,.
        \end{array}
        \right\}
        \label{e4.23-s}
        \end{equation}

Таким образом, имеем следующий результат.

\smallskip

\noindent
\textbf{Теорема 8.} \textit{Пусть эредитарные ядра $A(t,\tau)$,
$B(t,\tau)$, $C(t,\tau)$ в ЭСтС~$(\ref{e4.1-s})$ удовлетворяют условиям~$(\ref{e4.3-s})$,
$(\ref{e4.4-s})$ и~$(\ref{e4.6-s})$, а функции $\vrp$, $\psi$, $\chi$
не дифференцируемы по переменным расширенного вектора состояния.
Тогда ЭСтС~$(\ref{e4.1-s})$ приводится к ДСтС~$(\ref{e4.19-s})$
на основе~$(\ref{e4.20-s})$--$(\ref{e4.23-s})$.}

\smallskip

\noindent
\textbf{Замечание 2.}
Векторное уравнение~(\ref{e4.19-s}) для $Y'$, $Y''$, $Y'''$ относится к
числу так называемых приводимых к линейным уравнениям~\cite{2-s}.

Аналогичные теоремы устанавливаются для ЭСтС~(\ref{e4.8-s}).

Следовательно, если выполнены условия теорем~7 и~8, то ЭСтС~(\ref{e4.1-s})
приводится к ДСтС~(\ref{e4.14-s}) или~(\ref{e4.19-s}) и могут быть использованы
точные и приближенные методы анализа и моделирования распределений с
инвариантной мерой (см.\ разд.~2 и~3).

\smallskip

Таким образом, получены следующие утверждения, лежащие в основе точных и
приближенных методов для ЭСтС, приводимых к ДСтС.

\smallskip

\noindent
\textbf{Теорема 9.} \textit{В~условиях теоремы~$7$ для гладких функций
$a$, $a_1$, $b$, $b_1$, $c$, $c_1$ одномерные нестационарные и стационарные
распределения с инвариантной мерой определяются уравнениями теорем}~1 \textit{и}~2.

\smallskip

\noindent
\textbf{Теорема 10.} \textit{В~условиях теоремы~$8$ для разрывных функций~$a$,
$a_1$, $b$, $b_1$, $c$, $c_1$ одномерные нестационарные и стационарные
распределения с инвариантной мерой определяются уравнениями теорем~$3$ и~$4$.}

\smallskip

\noindent
\textbf{Теорема 11.} \textit{В~условиях теорем~$7$ и~$8$
приближенный алгоритм аналитического моделирования нестационарных
процессов с инвариантной мерой по МНА определяется теоремой~$5$,
а стационарных процессов~--- теоремой~$6$.}

\vspace*{-12pt}

\section{Заключение}

Получено обобщение точных и приближенных (основанных на
параметризации распределений) методов и алгоритмов моделирования
стационарных и нестационарных процессов с инвариантной мерой в
негауссовских ДСтС и ЭСтС с винеровскими и пуассоновскими шумами,
приводимых к ДСтС, для случаев гладких и разрывных регулярных правых
частей уравнений.

Особое внимание уделено приближенным
МНА и МСЛ для нахождения
распределений процессов с инвариантной мерой в ДСтС и ЭСтС,
приводимых к ДСтС.

Аналогично~\cite{2-s, 15-s, 25-s}, результаты допускают обобщение на случай
ЭСтС, приводимых к ДСтС, с автокоррелированными шумами.

Разработан комплекс тестовых примеров для инструментального
программного обеспечения в   <<ID StS>> в среде  MATLAB (см.\
приложение).

Аналогично~[2--8] может быть рас\-смот\-ре\-но применение представленных
методов в задачах эквивалентности гауссовских и негауссовских ДСтС и
ЭСтС. В~частности, соотношения~(\ref{e2.15-s}) и~(\ref{e3.9-s})
позволяют заменять в
ДСтС и ЭСтС стационарные и нестационарные негауссовские шумы
гауссовскими. Часто оказывается полезным заменить  $p$-мер\-ную
негауссовскую ДСтС или ЭСтС эквивалентной системой из  $p_1$
независимых ДСтС меньшей размерности  $(p_1\hm<p)$. В~этом случае
следует учесть дополнительные связи на  $K_{ij}(t)$, вытекающие из
аналитической природы рассматриваемой задачи.

\vspace*{-12pt}

{\small

\setcounter{equation}{0}

 \section*{\raggedleft Приложение}

 \vspace*{-6pt}

 \section*{Тестовые примеры}

\renewcommand{\theequation}{П\arabic{equation}}

\noindent
\textbf{Пример 1.}
 В условиях примера~6~\cite{22-s}, когда
\begin{multline*}
\ddot{X}+ \w^2 X -\mu X^3 =
    -\delta \dot X +\gamma + V^{\mathrm{ОР}}-{}\\
  {}-
    \int\limits_{t_0}^{t} \lk \w_1 X(\tau) -\delta_1 \dot X (\tau) +
    \mu_1 X^3 (\tau)\rk e^{-\la | t-\tau|}\, d\tau \,,
\\
  X(t_0) = X_0\,,\enskip \dot X (t_0) =\dot X_0\,,
%      \label{p1.1}
\end{multline*}
для обобщенного пуассоновского белого шума интенсивности $\nu
\hm=\nu^{\mathrm{OP}}$ уравнения для математических ожиданий, дисперсий и
ковариаций в силу теоремы~11 имеют следующий вид:
\begin{equation}
\left.
\begin{array}{rl}
\dot m_1 &=m_2;\\[9pt]
\dot m_2 &=-\w^2_{\mathrm{0э}} m_1 -\delta m_2 -\la^{-1} m_3 +\gamma\,;\\[9pt]
\dot m_3 &=\la (\w_{\mathrm{1э}} m_1 +\delta_1 m_2 -m_3)\,;
\end{array}
\right\}
\label{p1.2}
\end{equation}
\begin{equation}
\left.
\begin{array}{rl}
  \dot K_{11} &= 2 K_{12}\,;\\[9pt]
    \dot K_{12}& = K_{22} -(\w_{\mathrm{0э}}^2 K_{11} + \delta K_{12}
    +\la^{-1} K_{13})\,;\\[9pt]
    \dot K_{13} &= K_{23} +\la \w_{\mathrm{1э}}' K_{11} +\la \delta_1 K_{12} -
    \la K_{13}\,;\\[9pt]
    \dot K_{22} &=-2 (\w_{\mathrm{0э}}^2 K_{12} +\delta K_{22} +
    \la^{-1} K_{23})+\nu^{\mathrm{ОР}}\,;\\[9pt]
    \dot K_{23} &=-(\w_{\mathrm{0э}}^2 K_{13} +\la^{-1} K_{33})+
    \la\w_{\mathrm{1э}}' K_{12} +{}\\[9pt]
    &\hspace*{20mm}{}+\la \delta_1 K_{22} -(\delta+\la) K_{23}\,;\\[9pt]
    \dot K_{33} &= 2 ( \la \w_{\mathrm{1э}}' K_{13} +\la \delta_1 K_{23} -
    \la K_{33} )\,.
    \end{array}
    \right\}
    \label{p1.3}
    \end{equation}
Здесь приняты следующие обозначения:
\begin{gather*}
Z_1 = X\,;\enskip Z_2 = \dot X_3\,;
\\
     Z_3 =\int\limits_{t_0}^t \lk \w_1 Z_1^{(\tau)} +
    \delta_1 Z_2 (\tau) -\mu_1 Z_1^3 (\tau)\rk e^{-\la |t-\tau|} \,d\tau\,;
  \\
    m_i = {\sf M} Z_i \ (i=1,2,3)\,;\enskip K_{ij} = {\sf M} Z_i^0 Z_j^0\
    (i,j=1,2,3)\,;
    \end{gather*}
\begin{align*}
    &\w_{\mathrm{0э}}^2=\w_{\mathrm{0э}}^2(m_1, D_1) =\w^2 \lk 1-
    \mu \fr{(m_1^2 +3 D_1)}{\w^2}\rk\,;\\
    &\w_{\mathrm{1э}}=\w_{\mathrm{1э}}(m_1, D_1) =
    \w_1 \lk 1- \fr{\mu_1 (m_1^2 +3 D_1)}{\w_1}\rk\,;\\
    &\w_{\mathrm{1э}}'=\w_{\mathrm{1э}}'(m_1, D_1) =
    \w_1 \lk 1- \fr{3\mu (m_1^2 +3 D_1)}{\w_1}\rk\,.
    \end{align*}

Приравнивая в~(\ref{p1.2}) и~(\ref{p1.3}) правые части нулю, получим уравнения
для стационарных значений  $m_i^*$ и $K_{ij}^*$. Для устойчивости (в
среднем квадратическом) стационарных колебаний необходима
асимптотическая устойчивость матрицы
    \begin{equation*}
    \Lambda =\begin{bmatrix}
    0&1&0\\
    -\w_{\mathrm{0э}}^2 (m_1, D_1)& -\delta& -\la^{-1}\\
    \la \w_{\mathrm{1э}}'(m_1, D_1)& \la \delta_1 & -\la
    \end{bmatrix}
%\label{p1.4}
\end{equation*}
в~(\ref{e3.12-s}).

\smallskip

\noindent
\textbf{Пример~2.}
Для релейного осциллятора
    \begin{multline*}
    \ddot X +\alpha \,\mathrm{sgn}\, X =
    - \delta \dot X +\gamma +V^{\mathrm{ОР}}-{}\\
    {}-\int\limits_{t_0}^t \lk \alpha_1
    \mathrm{sgn}, X(\tau) +\delta_1 \dot X (\tau)\rk
    e^{-\la |t-\tau|}\,d\tau\,;\\
%\label{p2.1}
X(t_0) = X_0\,;\enskip \dot X (t_0) =\dot X \enskip (\alpha>0)
   \end{multline*}
в случае обобщенного пуассоновского белого шума~$V^{\mathrm{ОР}}$ интенсивности
$\nu\hm=\nu^{\mathrm{ОР}}$ в силу теоремы~11 имеем:

\noindent
\begin{equation}
\left.
\begin{array}{rl}
\dot m_1 &=m_2\,;\\[6pt]
\dot m_2 &=\alpha k_0 m_1 -\delta m_2 -\la^{-1} m_3 +\gamma\,;\\[6pt]
\dot m_3 &=\la (\alpha_1 k_0 m_1 +\delta_1 m_2 -m_3)\,;
\end{array}
\right\}
\label{p2.2}
\end{equation}
\begin{equation}
\left.
\begin{array}{rl}
\dot K_{11} &= 2 K_{12}\,;\\[5pt]
\dot K_{12} &= K_{22} -\alpha k_1 K_{11} - \delta K_{12} -\la^{-1} K_{13}\,;\\[5pt]
\dot K_{13} &= K_{23} +\la \alpha_1 k_1 K_{12} +\la \delta_1 K_{22} -\la K_{13}\,;\\[5pt]
\dot K_{22} &=-2 \alpha k_1 K_{12} -2\delta K_{22} -2\la^{-1} K_{23}+
 \nu^{\mathrm{ОР}}\,;\\[5pt]
\dot K_{23} &=-\alpha k_1 K_{13}  -(\delta+\la) K_{23}-\la^{-1} K_{33}+{}\\[5pt]
&\hspace*{24mm}{}+
    \la\alpha_1 k_1 K_{12} +\la \delta_1 K_{22}\,;\\[5pt]
\dot K_{33} &= 2  \la (\alpha_1 k_1 K_{13} +\delta_1 K_{23} -
    K_{33})\,.
    \end{array}
    \right\}
\label{p2.3}
\end{equation}
Здесь введены следующие обозначения:
    \begin{gather*}
    Z_1 = X\,;\quad Z_2 =\dot X\,;
    \\[-1pt]
    Z_3 =\la \int\limits_{t_0}^t \lk \alpha_1 \mathrm{sgn}\, X(\tau) +
    \delta_1 \dot X(\tau)\rk e^{-\la |t-\tau |} \,d\tau\,;
\\
    k_0 = k_0(m_1, D_1) =2\Phi\left(\fr{m_1}{\sqrt{ D_1}}\right)\,;\\
    k_1 = k_1(m_1, D_1) =\fr{2}{\sqrt{2\pi D_1}} \exp \lk -\fr{1}{2}
    \left(\fr{m_1}{\sqrt{ D_1}}\right)^2\rk\,.
    \end{gather*}

Приравнивая правые части~(\ref{p2.2}) и~(\ref{p2.3})
 нулю, получим уравнения для стационарных значений  $m_i^*$ и~$K_{ij}^*$.
Устойчивость стационарных колебаний определяется асимптотической устойчивостью
матрицы
    \begin{equation*}
    \Lambda =\begin{bmatrix}
    0&1&0\\
    -\alpha k_1 (m_1, D_1)& -\delta& -\la^{-1}\\
    \la \alpha_1 k_1(m_1, D_1)& \la \delta_1 & -\la
    \end{bmatrix}
%    \label{p2.4}
    \end{equation*}
    в~(\ref{e3.12-s}).


}

\vspace*{-8pt}


{\small\frenchspacing
{%\baselineskip=10.8pt
%\addcontentsline{toc}{section}{References}
\begin{thebibliography}{99}

\vspace*{-4pt}

\bibitem{1-s}
\Au{Пугачев В.\,С., Синицын И.\,Н.}
Стохастические дифференциальные системы. Анализ и фильтрация.~--- М.:
Наука,  1990.  632~с. [Англ. пер. Stochastic differential systems.
Analysis and filtering.~--- Chichester, N.Y.: Jonh Wiley, 1987.
549~p.].

\bibitem{3-s} %2
\Au{Moshchuk N.\,K., Sinitsyn I.\,N.}
On stationary distributions in nonlinear stochastic differential systems.~---
Coventry, UK: University of Warwick, Mathematics Institute, 1989. Preprint. 15~p.

\bibitem{4-s} %3
\Au{Moshchuk N.\,K., Sinitsyn I.\,N.} On stochastic nonholonomic systems.~---
Coventry, UK: University of Warwick, Mathematics Institute,
1989. Preprint. 32~p.

\bibitem{5-s} %4
\Au{Мощук Н.\,К., Синицын И.\,Н.}
О~стохастических неголономных системах~// Прикладная механика и математика, 1990.
Т.~54. Вып.~2. С.~213--223.

\bibitem{6-s} %5
\Au{Moshchuk N.\,K., Sinitsyn I.\,N.}
On stationary distributions in nonlinear stochastic differential systems~//
Quart. J.~Mech. Appl. Math., 1991. Vol.~44.  Pt.~4.  P.~571--579.

\bibitem{7-s} %6
\Au{Мощук Н.\,К., Синицын И.\,Н.} О~стационарных и приводимых к стационарным
режимах в нормальных стохастических системах~// Прикладная механика и математика,
1991. Т.~55. Вып.~6. С.~895--903.

\bibitem{8-s} %7
\Au{Мощук Н.\,К., Синицын И.\,Н.}
Распределения с инвариантной мерой в механических стохастических нор-\linebreak\vspace*{-12pt}

\columnbreak

\noindent
мальных
системах~// Докл. АН СССР, 1992. Т.~322. №\,4. С.~662--667.

\bibitem{9-s} %8
\Au{Синицын И.\,Н.} Конечномерные распределения с инвариантной мерой в
стохастических механических системах~// Докл. РАН, 1993. Т.~328. №\,3. С.~308--310.

\bibitem{13-s} %9
\Au{Soize C.} The Fokker--Plank equation for stochastic dynamical
systems and its explicit steady state solutions.~--- Singapore: World Scientific,
1994. 321~p.

\bibitem{10-s} %10
\Au{Синицын И.\,Н.} Конечномерные распределения с инвариантной мерой в
стохастических нелинейных дифференциальных системах.~--- М.:
Диа\-лог--МГУ, 1997. С.~129--140.

\bibitem{2-s} %11
\Au{Пугачев В.\,С., Синицын И.\,Н.}
Теория стохастических систем.~--- М.: Логос, 2000; 2004. 1000~с.
[Англ. пер. Stochastic systems. Theory and  applications.~---
Singapore: World Scientific, 2001. 908~p.].


\bibitem{11-s} %12
\Au{Синицын И.\,Н., Корепанов Э.\,Р., Белоусов~В.\,В.}
Точные методы расчета стационарных режимов с инвариантной мерой в
стохастических системах управления~// Кибернетика и высокие
технологии XXI века: Сб. докл.  II~Междунар.
на\-уч.-тех\-нич. конф.~--- Воронеж: Саквоее, 2002.
С.~124--131.

\bibitem{12-s} %13
\Au{Синицын И.\,Н., Корепанов Э.\,Р., Белоусов~В.\,В.}
Точные аналитические методы в статистической динамике нелинейных
ин\-фор\-ма\-ци\-он\-но-управ\-ля\-ющих сис\-тем~//
Системы и средства информатики.
Спец. вып. Математическое и алгоритмическое обеспечение
ин\-фор\-ма\-ци\-он\-но-те\-ле\-ком\-му\-ни\-ка\-ци\-он\-ных сис\-тем.~---
М.: Наука, 2002. С.~112--121.


\bibitem{14-s}
\Au{Синицын И.\,Н.} Развитие методов аналитического моделирования
распределений с инвариантной мерой в стохастических системах~//
Современные проблемы прикладной математики, информатики, автоматизации,
управления: Мат-лы Междунар. семинара.~--- Севастополь:  СевНТУ, 2012.
С.~24--35.

\bibitem{15-s}
\Au{Синицын И.\,Н.} Аналитическое моделирование распределений с
инвариантной мерой в стохастических системах с автокоррелированными шумами~//
Информатика и её применения, 2012. Т.~6. Вып.~4. С.~4--8.

\bibitem{16-s}
\Au{Синицын И.\,Н. }
Аналитическое моделирование распределений с инвариантной мерой в
стохастических системах с разрывными характеристиками~// Информатика
и её применения, 2013. Т.~7. Вып.~1.  С.~3--11.

\bibitem{17-s}
\Au{Синицын И.\,Н. }
Параметрическое статистическое и аналитическое моделирование распределений
в нелинейных стохастических системах на многообразиях~//
Информатика и её применения, 2013. Т.~7. Вып.~2. С.~4--16.

\bibitem{18-s}
\Au{Синицын И.\,Н.,  Синицын В.\,И. }
Лекции по нормальной и эллипсоидальной аппроксимации распределений в
стохастических сис\-те\-мах.~--- М.: ТОРУС ПРЕСС, 2013. 488~с.

\bibitem{20-s} %19
\Au{Синицын И.\,Н. }
Stochastic hereditary control systems~// Проблемы управления и
теории информации, 1986. Т.~15. №\,4. С.~287--298.

\bibitem{19-s} %20
\Au{Синицын И.\,Н. }
Конечномерные распределения процессов в стохастических интегральных
и интегродифференциальных системах~// Preprints of the 2nd IFAC
Symposium on Stochastic Control. Vol.~1.~--- Zurich: Pergamon Press,
1987. P.~144--153.


\bibitem{22-s} %21
\Au{Синицын И.\,Н., Синицын~В.\,И., Корепанов~Э.\,Р., Белоусов~В.\,В.,
Сергеев~И.\,В., Басилашвили~Д.\,А.}
Опыт моделирования эредитарных стохастических систем~//
Кибернетика и высокие технологии XXI~века: Сб. докл.  XIII~Междунар.
на\-уч.-тех\-нич. конф.~--- Воронеж: Саквоее, 2012. Т.~2. C.~346--357.

%\columnbreak

\bibitem{21-s} %22
\Au{Синицын И.\,Н. }
Анализ и моделирование распределений в эредитарных стохастических
системах~// Информатика и её применения, 2014. Т.~8. Вып.~1.
С.~2--11.


\bibitem{23-s}
\Au{Немыцкий В.\,В., Степанов В.\,В.}
Качественная теория дифференциальных уравнений.~--- М.--Л.: Гостехиздат,
1949. 448~с.

\bibitem{24-s}
\Au{Козлов В.\,В.} О~существовании интегрального инварианта гладких
динамических систем~// ПММ, 1987. №\,1. С.~538--545.

\bibitem{25-s}
\Au{Синицын И.\,Н.} Фильтры Калмана и Пугачева.~--- 2-е изд.~--- М.: Логос,
2007. 776~с.

%\bibitem{26-s}
%\Au{Синицын И.\,Н. }
%Канонические представления случайных функций и их применение в
%задачах компьютерной поддержки научных исследований.~--- М.: ТОРУС
%ПРЕСС, 2009. 768~с.

\end{thebibliography}
} }

\end{multicols}

\vspace*{-9pt}

\hfill{\small\textit{Поступила в редакцию 16.01.14}}

%\newpage


\vspace*{6pt}

\hrule

\vspace*{2pt}

\hrule

\vspace*{-6pt}


\def\tit{ANALYTICAL MODELING OF~DISTRIBUTIONS
 WITH  INVARIANT MEASURE IN~NON-GAUSSIAN
DIFFERENTIAL  AND~REDUCABLE TO~DIFFERENTIAL HEREDITARY
 STOCHASTIC SYSTEMS}

\def\titkol{Analytical modeling of distributions
 with  invariant measure in~non-Gaussian
differential  and~reducable to~differential HStS} %hereditary stochastic systems}

\def\aut{I.\,N.~Sinitsyn}
\def\autkol{I.\,N.~Sinitsyn}


\titel{\tit}{\aut}{\autkol}{\titkol}

\vspace*{-12pt}

\noindent
Institute of Informatics Problems, Russian Academy of Sciences,
44-2 Vavilov Str., Moscow 119333, Russian
Federation



\def\leftfootline{\small{\textbf{\thepage}
\hfill INFORMATIKA I EE PRIMENENIYA~--- INFORMATICS AND APPLICATIONS\ \ \ 2014\ \ \ volume~8\ \ \ issue\ 2}
}%
 \def\rightfootline{\small{INFORMATIKA I EE PRIMENENIYA~--- INFORMATICS AND APPLICATIONS\ \ \ 2014\ \ \ volume~8\ \ \ issue\ 2
\hfill \textbf{\thepage}}}

%\vspace*{6pt}



\Abste{Exact and approximate methods and algorithms of one- and multidimensional
distributions with invariant measure for analytical modeling in differential
non-Gaussian (with Wiener and Poisson noises) stochastic systems (StS) and
hereditary StS (HStS) reducible to differential are presented.
    Four theorems giving exact methods of analysis modeling in differential StS (DStS)
    of general type are proved. Approximate methods based on distributions
    parametrization in DStS are disscused. Special attention is paid to the methods
    of normal approximation (MNA) and statistical linearization (MSL) for
    one-    and dimensional distributions in DStS. Stability conditions are presented.
    three theorems giving exact and approximate analytical modeling in HStS resucible
    to DStS with asymptotically dying kernels are given.
    Some equavalency applications  of DStS and HStS are considered. Test examples
    for software tools ``ID StS'' are given.}

\KWE{analytical modeling; differential stochastic system;
distribution with invariant measure; Gaussian (normal) stochastic system;
hereditary kernel; hereditary stochastic system;
hereditary system reducible to differential;
Ito stochastic differential equation; method of statistical linearization;
non-Gaussian (with Wiener and Poisson noises) stochastic system;
normal approximation method;
singular kernel; software tools  ``ID  StS'' }

\DOI{10.14357/19922264140201}

\vspace*{-24pt}

\Ack
\noindent
The work was financially supported by the Program ``Intelligent information
technology, system analysis, and automation'' (project~1.7).

\vspace*{-2pt}


  \begin{multicols}{2}

\renewcommand{\bibname}{\protect\rmfamily References}
%\renewcommand{\bibname}{\large\protect\rm References}

{\small\frenchspacing
{%\baselineskip=10.8pt
\addcontentsline{toc}{section}{References}
\begin{thebibliography}{99}

\vspace*{-5pt}

\bibitem{1-s-1}
\Aue{Pugachev, V.\,S., and  I.\,N.~Sinitsyn}.  1987.
\textit{Stochastic differential systems. Analysis and filtering.}
Chichester, New York: Jonh Wiley. 549~p.


\bibitem{3-s-1} %2
\Aue{Moshchuk, N.\,K., and I.\,N.~Sinitsyn}. 1989.
\textit{On stationary distributions in nonlinear stochastic differential systems}.
Coventry, UK: University of Warwick, Mathematics Institute. Preprint.
15~p.

\bibitem{4-s-1} %3
\Aue{Moshchuk, N.\,K., and I.\,N.~Sinitsyn}.  1989.
\textit{On stochastic nonholonomic systems}.
Coventry, UK: University of Warwick, Mathematics Institute. Preprint. 32~p.

\bibitem{5-s-1} %4
\Aue{Moshchuk, N.\,K., and I.\,N.~Sinitsyn}. 1990.
 O stokhasti\-che\-skikh negolonomnykh sistemakh
 [About stochastic nonholonomial systems].
 \textit{Prikladnaya Mekhanika i Matematika}
 [\textit{Appl. Mech. Math.}]. 54(2):213--223.

\bibitem{6-s-1} %5
\Aue{Moshchuk, N.\,K., and I.\,N.~Sinitsyn}. 1991.
On stationary distributions in nonlinear stochastic differential systems.
\textit{Quart. J.~Mech. Appl. Math.} 44(4):571--579.

\bibitem{7-s-1} %6
\Aue{Moshchuk, N.\,K., and I.\,N.~Sinitsyn}. 1991.
 O statsi\-o\-nar\-nykh i privodimykh k statsionarnym rezhimakh v
 normal'nykh stokhasticheskikh sistemakh
 [About stationary and reducible to stationary regimes in normal
 stochastic systems].
 \textit{Prikladnaya Mekhanika i Matematika}
 [\textit{Appl. Mech. Math.}]. 55(6):895--903.

\bibitem{8-s-1} %7
\Aue{Moshchuk, N.\,K., and I.\,N.~Sinitsyn}. 1992.
Raspredeleniya s invariantnoy meroy v mekhanicheskikh sto\-kha\-sti\-che\-skikh
normal'nykh sistemakh [Distributions with invariant measure in mechanical
stochastic normal systems]. \textit{Dokl. AN SSSR} 322(4):662--667.

\bibitem{9-s-1} %8
\Aue{Sinitsyn, I.\,N.} 1993.
 Konechnomernye raspredeleniya s invariantnoy meroy v stokhasticheskikh
 mekhanicheskikh sistemakh [Finite dimensional distributions with invariant
 measure in stochastic mechanical systems]. \textit{Dokl. RAN} 328(3):308--310.

 \bibitem{13-s-1} %9
\Aue{Soize, C.}  1994.
 \textit{The Fokker--Plank equation for stochastic dynamical systems and
 its explicit steady state solutions}. Singapore: World Scientific. 321~p.


\bibitem{10-s-1} %10
\Aue{Sinitsyn, I.\,N.} 1997.
 Konechnomernye raspredeleniya s invariantnoy meroy v stokhasticheskikh
 nelineynykh differentsial'nykh sistemakh [Finite dimensional distributions
 with invariant measure in stochastic nonlinear differential systems].
 Moscow: Dialog--MGU. 129--140.

 \bibitem{2-s-1} %11
\Aue{Pugachev, V.\,S., and I.\,N.~Sinitsyn}. 2001.
\textit{Stochastic systems. Theory and  applications.}
 Singapore: World Scientific. 908~p.


\bibitem{11-s-1} %12
\Aue{Sinitsyn, I.\,N., E.\,R.~Korepanov, and  V.\,V.~Belousov}.  2002.
 Tochnye metody rascheta statsionarnykh rezhimov s invariantnoy
 meroy v stokhasticheskikh sistemakh upravleniya [Exact  analysis of
 stationary with invariant\linebreak
  measure regimes in stochastic control systems].
 \textit{Kibernetika i Tekhnologii XXI~veka: Tr. II Mezhdunar. Nauch.-Tekhnich.
 Konf.} [Cybernetics and High Technologies of XXI~Century.
2nd~International Science and\linebreak Technology Conference Proceedings] C\&T'2002.
 Voronezh: Sakvoe. 124--131.

\bibitem{12-s-1} %13
\Aue{Sinitsyn, I.\,N., E.\,R.~Korepanov, and V.\,V.~Belousov}.  2002.
{Tochnye analiticheskie metody v statisticheskoy dinamike
nelineynykh informatsionno-upravlyayushchikh sistem} [Exact analytical
methods in statistical dynamics of nonlinear informational and control issue].
\textit{Sistemy i Sredstva Informatiki}~---
\textit{Systems and Means of Informatics}. Spets. Vyp. Matematicheskoe i
algoritmicheskoe obespechenie informatsionno-telecommunikatsionnykh sistem
[Mathematical sofware for information and telecommunication systems].
Moscow: Nauka. 112--121.


\bibitem{14-s-1}
\Aue{Sinitsyn, I.\,N.} 2012.
 Razvitie metodov analiticheskogo modelirovaniya raspredeleniy s invariantnoy
 meroy v stokhasticheskikh sistemakh [Development of analytical modeling
 methods for distributions with invariant measure in stochastic systems].
 \textit{Sovremennye Problemy Prikladnoy Matematiki, Informatiki, Avtomatizatsii,
 Upravleniya: Materialy Mezhdunar. Seminara}
 [Modern Problems of Applied Mathematics Informatics, Atomization and Control:
 Seminar (International) Proceedings].
 Sevastopol':  SevNTU. 24--35.

\bibitem{15-s-1}
\Aue{Sinitsyn, I.\,N.} 2012.
 Analiticheskoe modelirovanie raspredeleniy s invariantnoy meroy v
 stokhasticheskikh sistemakh s avtokorrelirovannymi shumami
 [Analytical modeling of distributions with invariant measure in stochastic
 systems with autocorrelated noise].
 \textit{Informatika i ee Primeneniya}~--- \textit{Inform. Appl.} 6(4):4--8.

\bibitem{16-s-1}
\Aue{Sinitsyn, I.\,N.}  2013.
Analiticheskoe modelirovanie raspredeleniy s invariantnoy meroy v
stokhasticheskikh sistemakh s razryvnymi kharakteristikami
[Analytical modeling of distributions with invariant measure in stochastic
systems with discontinuous nonlinearities].
\textit{Informatika i ee Primeneniya}~--- \textit{Inform. Appl.} 7(1):3--11.

\bibitem{17-s-1}
\Aue{Sinitsyn, I.\,N.}  2013.
Parametricheskoe statisticheskoe i analiticheskoe modelirovanie raspredeleniy
v nelineynykh stokhasticheskikh sistemakh na mnogoobraziyakh
[Parametric statistical and analytical modeling of distributions in stochastic
systems on manifolds].
\textit{Informatika i ee Primeneniya}~--- \textit{Inform. Appl.} 7(2):4--16.

\bibitem{18-s-1}
\Aue{Sinitsyn, I.\,N., and  V.\,I.~Sinitsyn}.  2013.
\textit{Lektsii po normal'noy i ellipsoidal'noy approksimatsii raspredeleniy
v stokhasticheskikh sistemakh} [Lectures on normal and ellipsoidal
approximation of distributions in stochastic systems].
Moscow: TORUS PRESS. 488~p.

\bibitem{20-s-1} %19
\Aue{Sinitsyn, I.\,N.}  1986.
{Stochastic hereditary control systems}.
\textit{Problems Control Inform. Theory} 15(4):287--298.


\bibitem{19-s-1} %20
\Aue{Sinitsyn, I.\,N.}  1987.
Konechnomernye raspredeleniya protsessov v stokhasticheskikh integral'nykh i
integ\-ro\-dif\-fe\-ren\-tsi\-al'nykh sistemakh [Finite dimensional distributions
of processes in stochastic integral and integrodifferential systems].
\textit{2nd Symposium (International) IFAC on Stochastic Control}.
Preprints. Vilnius, 1986. Pergamon Press. 1:144--153.

\bibitem{22-s-1} %21
\Aue{Sinitsyn, I.\,N., V.\,I.~Sinitsyn, E.\,R.~Korepanov, V.\,V.~Belousov,
I.\,V.~Sergeev, and D.\,A.~Basilashvili}. 2012.
Opyt modelirovaniya ereditarnykh stokhasticheskikh sistem
[Experience of modeling in hereditary stochastic systems].
\textit{Kibernetika i Vysokie Tekhnologii XXI~Veka: Sbornik dokladov
XIII Mezhdunar. Nauch.-Tekhnich. Konf.}
[Cybernatics and High Technologies of XXI~Century.
13th  Scientific and Technological Conference (International) Proceedings].
Voronezh: Sakvoee. 2:346--357.


\bibitem{21-s-1} %22
\Aue{Sinitsyn, I.\,N.}  2014.
Analiz i modelirovanie raspredeleniy v ereditarnykh stokhasticheskikh sistemakh
[Analysis and modeling of distributions in hereditary stochastic systems].
\textit{Informatika i ee Primeneniya}~--- \textit{Inform. Appl.} 8(1):2--11.



\bibitem{23-s-1}
\Aue{Nemytskiy, V.\,V., and V.\,V.~Stepanov}.  1949.
\textit{Ka\-chest\-ven\-naya teoriya differentsial'nykh uravneniy}
[Analytical theory of differential equations].
Moscow--Leningrad: Gostekhizdat. 448~p.

\bibitem{24-s-1}
\Aue{Kozlov, V.\,V.} 1987.
O~sushchestvovanii integral'nogo invarianta gladkikh dinamicheskikh sistem
[Existence of integral invariants in oblique dynamical systems].
\textit{Prikladnaya Mekhanika i Matematika}
[Appl. Mech. Math.] 1:~538--545.

\bibitem{25-s-1}
\Aue{Sinitsyn, I.\,N.} 2007.
\textit{Fil'try Kalmana i Pugacheva} [Kalman and Pugachev filters].
 2nd ed. Moscow: Logos.  776~p.

%\bibitem{26-s-1}
%\Aue{Sinitsyn, I.\,N.} 2009.
%\textit{Kanonicheskie predstavleniya sluchaynykh funktsiy i ikh primenenie v
%zadachakh komp'yuternoy podderzhki nauchnykh issledovaniy}
%[Canonical expansions of random functions and their application to scientific
%computer-aided support]. Moscow: TORUS PRESS. 768~p.

       \end{thebibliography}
} }


\end{multicols}

\vspace*{-12pt}

\hfill{\small\textit{Received January 16, 2014}}

\vspace*{-24pt}


\Contrl

\vspace*{-2pt}

\noindent
\textbf{Sinitsyn Igor N.} (b.\ 1940)~---
Doctor of Science in technology, professor, Honored scientist of RF, Head of Department, Institute of
Informatics Problems, Russian Academy of Sciences, 44-2 Vavilov Str., Moscow 119333, Russian
Federation; sinitsin@dol.ru

\vspace*{-12pt}

 \label{end\stat}

\renewcommand{\bibname}{\protect\rm Литература} %5
\def\stat{torshin}

\def\tit{О ПОРОЖДЕНИИ СИНТЕТИЧЕСКИХ ПРИЗНАКОВ НА~ОСНОВЕ~ОПОРНЫХ ЦЕПЕЙ 
И~ПРОИЗВОЛЬНЫХ МЕТРИК В~РАМКАХ~ТОПОЛОГИЧЕСКОГО ПОДХОДА 
К~АНАЛИЗУ ДАННЫХ.\\ ЧАСТЬ~2.~ЭКСПЕРИМЕНТАЛЬНАЯ АПРОБАЦИЯ\\ НА~ЗАДАЧАХ ФАРМАКОИНФОРМАТИКИ$^*$}

\def\titkol{О порождении синтетических признаков на основе опорных цепей 
и~произвольных метрик} % в~рамках топологического подхода  к~анализу данных. Часть~2. Экспериментальная апробация на  задачах фармакоинформатики}

\def\aut{И.\,Ю.~Торшин$^1$}

\def\autkol{И.\,Ю.~Торшин}

\titel{\tit}{\aut}{\autkol}{\titkol}

\index{Торшин И.\,Ю.}
\index{Torshin I.\,Yu.}


{\renewcommand{\thefootnote}{\fnsymbol{footnote}} \footnotetext[1]
{Работа выполнена при поддержке гранта РНФ (проект №\,23-21-00154) с~использованием инфраструктуры 
Центра коллективного пользования <<Высокопроизводительные вычисления и~большие данные>> (ЦКП 
<<Информатика>>) ФИЦ ИУ РАН (г.~Москва).}}


\renewcommand{\thefootnote}{\arabic{footnote}}
\footnotetext[1]{Федеральный исследовательский центр <<Информатика и~управление>> Российской академии наук, 
\mbox{tiy135@yahoo.com}}

\vspace*{-12pt}


\Abst{Рассмотрение прецедентных отношений между признаками и~таргетной переменной в~виде наборов элементов булевой решетки указывает на возможность порождения 
синтетических признаков с~использованием метрических функций расстояния. 
Сформулированы подходы к~(1)~оценке релевантности (<<информативности>>) метрик 
по отношению к~решаемым задачам, (2)~порождению и~(3)~отбору синтетических 
признаков, более информативных, чем исходные признаковые описания. Представленные 
результаты топологического анализа 2400~выборок данных  
<<мо\-ле\-ку\-ла--свойство>> из ProteomicsDB позволили получить достаточно 
эффективные алгоритмы прогнозирования свойств молекул (ранговая корреляция  
в~кросс-ва\-ли\-да\-ции~--- $0{,}90\pm0{,}23$). На данной выборке задач установлены 
метрики, которые наиболее часто порождают информативные синтетические признаки: 
максимальное уклонение Колмогорова, <<косое>> расстояние, метрики Lp, Реньи, фон 
Мизеса. Для решения изученного комплекса задач показано преимущество полиномных 
корректоров по сравнению с~нейросетевыми и~с~корректорами типа <<случайный 
лес>>.}

\KW{топологический анализ данных; теория решеток; алгебраический подход 
Ю.\,И.~Жу\-рав\-лё\-ва; фармакоинформатика}

\DOI{10.14357/19922264240207}{OTXCUD}
  
\vspace*{-1pt}


\vskip 10pt plus 9pt minus 6pt

\thispagestyle{headings}

\begin{multicols}{2}

\label{st\stat}

\section{Введение}

     В первой части работы~[1] принимается, что задано регулярное 
множество прецедентов 
$$
\mathbf{Q}\hm= \{\mathrm{D}(x_i)\vert x_i\in 
\mathbf{X}\}
$$ 
на решетке $L(T(\mathbf{X}))$, по\-рож\-ден\-ное на основе 
множества исходных описаний объектов $\mathbf{X}\hm= \{ x_1, \ldots , 
x_{N_0}\}$. Для индивидуального объекта\linebreak $x_i\hm\in \mathbf{X}$ 
прецедентному соотношению между значениями признаками 
$\Gamma_k(x_i)$ и~\mbox{$t$-й} таргетной переменной соответствует множество пар 
$\{(\{\Gamma_k^{-1}(\Gamma_k(x_i)),\linebreak k\hm=\overline{1, ,n}\}, \Gamma_t^{-1}(\Gamma_t(x_i))), i\hm=\overline{1,N_0},\
 k\hm=\overline{1,n},\linebreak t\hm=\overline{n+1, n+l}\}$, где $l$~--- 
число таргетных переменных. В~рамках топологической теории 
распознавания прецедентное соотношение между множествами $\{ 
\Gamma_k^{-1}(\Gamma_k(x_i))\}$ и~$\Gamma_t^{-1}(\Gamma_t(x_i))$ 
моделируется как со\-от\-вет\-ст\-ву\-ющие массивы расстояний, по\-рож\-да\-емые той 
или иной мет\-ри\-кой~$\rho_m$: $L(T(\mathbf{X}))^2\hm\to [0\ldots 1]$, 
$m\hm= \overline{1, m_0}$. В~[1] предложены способы <<встра\-и\-ва\-ния>> 
в~формализм полуэмирических рас\-сто\-яний на множествах $a\hm\in 
L(T(\mathbf{X}))$, векторах $\vec{v}_\alpha [a] \hm= ( v_{\alpha_1}[a], 
v_{\alpha_2}[a], \ldots , v_{\alpha_i}[a],\ldots)$ и~функциях 
$\hat{\phi}(x)\bm{\Gamma}_t(u)$. 
     
     Здесь для практического приложения формализма сформулированы 
подходы к~исследованию свойств~$\rho_m$, способы оценки релевантности 
функций~$\rho_m$ по отношению к~решаемым задачам, способы 
порождения и~отбора синтетических признаков, основанных на~$\rho_m$. 
Представлены результаты экспериментальной апробации на задачах 
фармакоинформатики.
     
\section{Об исследовании свойств функций расстояния~$\rho_m$}

    Рабочая гипотеза настоящего исследования со\-сто\-ит в~том, что для 
порождения более <<информативных>> признаков могут использоваться 
полуэмпирические функционалы расстояния на \mbox{множествах}, векторах, 
функциях~[2]. Метрические свойства ис\-поль\-зу\-емых функций 
расстояния~$\rho_m$ могут исследоваться аналитически или комбинаторно 
с~использованием аксиом метрики~[3]. Для анализа свойств этих 
функционалов в~топологической теории распознавания вводится следующее 
понятие.

\smallskip

\noindent
\textbf{Определение~1.} Обобщенной оценочной функцией расстояния 
будем называть конструкцию вида 
$$
\rho(a,b) = f(g ( v[a\vee b]) - g(v[a\wedge b])),
$$
 в~которой~$f$ и~$g$~--- функции, монотонные на 
соответствующих участках действительной оси; $v:\ L\hm\to R^+$~--- 
изотонная оценка, для которой выполнено условие оценки (\textbf{уО}: $\forall_L 
a,b: v[a]\hm+v[b]\hm= v[a \wedge b]\hm+ v[a\vee b]$) и~изотонности 
(\textbf{уИ}:  $\forall_L a,b: a\supseteq b \hm\Rightarrow v[a]\hm\geq v[b]$). 

\smallskip

\noindent
\textbf{Теорема~1.} \textit{Функция расстояния~$\rho$ считается 
обобщенной оценочной функцией расстояния тогда и~только тогда, когда 
$\rho(a,b)\hm= \rho(a\vee b, a\wedge b)$, а~термы от $a$ и~$b$ в~формуле для 
$\rho(a,b)$ представляют собой композицию монотонной функции 
и~изотонной оценки}. 

\smallskip

Необходимость следует из  $a\vee b\hm= (a\vee b)\vee (a\wedge b)$ и~$a\wedge b \hm= (a\vee b) \wedge (a\wedge b)$  при 
подстановке $a\vee b$ и~$a\wedge b$ вместо $a$ и~$b$ в~определение~1. 
Эквивалентность $\rho(a,b)$ и~$\rho(a\vee b, a\wedge b)$ указывает на то, что 
в~выражение для вычисления~$\rho$ входят тер\-мы-функ\-ци\-о\-на\-лы, 
содержащие выражения $a\vee b$ и~$a\wedge b$, взаимозаменяемые с~$a$ 
и~$b$, т.\,е.\ термы вида $g^\prime (a\vee b)$ и~$g^\prime(a\wedge b)$. По 
условию теоремы эти термы включают монотонную функцию от изотонной 
оценки, т.\,е.~$g^\prime$ монотонна. Так как $\rho$~--- функция расстояния, 
то $g^\prime$-тер\-мы не могут входить в~выражение для~$\rho$ в~виде 
произведения, суммы, отношения, степени или суммы, а~только в~виде 
разности, т.\,е.\
$$
\rho(a,b) = f\left(g^\prime(a\vee b) \hm- g^\prime (a\wedge b)\right),
$$ 
из чего следует достаточность. Теорема доказана.

\smallskip

\noindent
\textbf{Следствие~1.} Для обобщенной оценочной~$\rho$ 
\begin{multline*}
\forall \ell \subseteq L(T(\mathbf{X})): \Delta_{\vee\wedge}(\ell)\equiv 0,\\ 
\Delta_{\vee\wedge}(\ell)=  \sum\limits_{a,b\in \ell} \vert\rho(a,b)- 
\rho(a\vee b, a\wedge b)\vert \fr{2}{\vert\ell\vert/(\vert\ell\vert -1)}\,.
\end{multline*}

\smallskip

\noindent
\textbf{Следствие~2.} Выберем <<опорное>> множество $a\hm\in 
L(T(\mathbf{X}))$ и~обобщенную оценочную~$\rho$. При $f(x)\hm= g(x)\hm= x$ 
$v_{a,\rho}[b]\hm= \rho(a,b)\hm= \rho(a\vee b, a\wedge b)$~--- изотонная 
оценка. 

Следует из того, что любая линейная комбинация изотонных оценок~--- 
изотонная оценка при условии положительной определенности (теорема~2 
в~[4]). Также проверяется прямой подстановкой $v_{a,\rho}[b]$ в~уО и~уИ. 

\smallskip

\noindent
\textbf{Следствие~3.} Расстояния Фре\-ше--Ни\-ко\-ди\-ма, Амана,  
Рэн\-да/Ще\-ка\-нов\-ско\-го, Со\-ка\-ла--Сни\-са (варианты~1, 2 и~3),  
Рас\-се\-ла--Рао, Род\-же\-ра--Та\-ни\-мо\-то, Фейта, Тверского и~Юле 
могут служить обобщенными оценочными функциями расстояния. 

\smallskip

\noindent
\textbf{Следствие~4.} Расстояния Симпсона, Бра\-у\-на--Блан\-ке, 
Андерберга и~Говера-2  не входят в~число обобщенных оценочных функций 
расстояния.

\smallskip

     Теорема~1 со следствиями предоставляет аналитический 
и~комбинаторный инструментарий для исследования свойств 
полуэмпирических функций расстояния. Если заданная~$\rho$ служит 
обобщенной оценочной функцией расстояния, то могут быть получены 
соответствующие аналитические выражения для функций~$f$ и~$g$. 
Например, расстояние Со\-ка\-ла--Сни\-са-2
$$
\rho(a,b) = 1- \fr{\vert a\cap 
b\vert }{\vert a\cup b\vert + \vert a\Delta b\vert}
$$ 
выступает 
обобщенным оценочным расстоянием с~$f(x)\hm= (e^x\hm-1)/(0{,}5e^x\hm-1)$ и~$g(x)\hm=\ln (x)$. При невозможности аналитической проверки 
свойства~$\rho$ как обобщенной оценочной могут быть изучены на 
подмножествах~$\ell$ решетки $L(T(\mathbf{X}))$ посредством вычисления 
значений функционала $\Delta_{\vee\wedge}(\ell)$ (следствие~1). 

\section{О способах оценки релевантности метрик~$\rho_m$ по~отношению к~задаче клас\-сификации/прогнозирования}

     Биекция между множеством прецедентов~$\mathbf{Q}$ и~множеством 
исходных описаний объектов~$\mathbf{X}$, существующая при выполнении 
условия регулярности по Журавлёву ($\forall \mathrm{x}\hm\in \mathbf{X}, 
\mathrm{x}\hm= D^{-1}(D(\mathrm{x}))$, гарантирует однозначность 
соответствия описаний~$x_i$ и~$q_i$. Это делает возможным рассматривать 
прецедентные соотношения, заданные на~$\mathbf{Q}$, в~терминах 
множеств $\{ \Gamma_k^{-1}(\Gamma_k(x_i))\}$ и~$\Gamma_t^{-1}( 
\Gamma_t(x_i))$ с~использованием расстояний~$\rho_m$ на подмножествах 
множества~$\mathbf{X}$~[1].
     
     Пусть таргетный класс объектов $\mathbf{c}_{\bm{\alpha}}$ задан 
посредством $\alpha$-го значения $t$-й переменной $\lambda_{t\alpha}\hm\in 
\mathrm{I}_t$, $t\hm= \overline{n+1,  n+l}$, как $\mathbf{c}_{{\bm 
\alpha}} \hm= \Gamma_t^{-1}(\lambda_{t\alpha})$. В~случае числовой 
переменной за $\mathbf{c}_{\bm{\alpha}}$ может приниматься каждый из 
элементов $u(\lambda_{t\alpha})$ цепи~$A_t$. Так как 
$L(T(\mathbf{X}))$ булева, то дополнение множества 
$\mathbf{c}_{\bm{\alpha}}$, $\overline{\mathbf{c}}_{\bm{\alpha}} \hm= 
\mathbf{X}\backslash \Gamma_t^{-1}(\lambda_{t\alpha})$, определено 
однозначно. Таким образом, выделение класса $\mathbf{c}_{\bm{\alpha}}$ 
порождает задачу классификации $\mathbf{c}_{\bm{\alpha}}/ 
\overline{\mathbf{c}}_{\bm{\alpha}}$. Любая задача числового 
прогнозирования может быть сведена к~последовательности корректно 
решаемых задач $\mathbf{c}_{\bm{\alpha}}/ 
\overline{\mathbf{c}}_{\bm{\alpha}}$~\cite{5-tor}.
     
     Пусть задано подмножество признаков~$p \hm\subseteq [1\ldots n]$ 
     и~элемент решетки $c\in L(T(\mathbf{X}))$. Определим функцию 
$$
\bm{\rho}_{\mathbf{mc}} (x_i, c, {p}) \hm= \{ \rho_m(c, \Gamma_k^{-1}(\Gamma_k (x_i)),\ k\hm\in {p})\}.
$$
 При заданных~$\rho_m$, $p$, 
$\mathbf{c}_{\bm{\alpha}}$ и~$\overline{\mathbf{c}}_{\bm{\alpha}}$ 
для~$x_i$ вычислимы множества расстояний $\bm{\rho}_{\mathbf{mc}}(x_i, 
\mathbf{c}_{\bm{\alpha}}, {p})$ и~$\bm{\rho}_{\mathbf{mc}}(x_i, 
\overline{\mathbf{c}}_{\bm{\alpha}}, {p})$. Обозначим 
\begin{align*}
\bm{\rho}_{\mathbf{m}\bm{\alpha}}(x_i) &=  \bm{\rho}_{\mathbf{mc}} 
(x_i, \mathbf{c}_{\bm{\alpha}}, [1\ldots n]); \\
\bm{\rho}_{\mathbf{m}\overline{\bm{\alpha}}} (x_i) &= 
\bm{\rho}_{\mathbf{mc}}(x_i, \overline{\mathbf{c}}_{\bm{\alpha}} , [1\ldots n]).
\end{align*}
 Для $x_i\hm\in \mathbf{X}$ 
определено множество 
\begin{multline*}
\bm{\rho}_{\mathbf{m}}(x_i,{p})=\left \{ \rho_{mk_1k_2}(x_i, {p}) = {}\right.\\
{}\rho_m\left(\Gamma^{-1}_{k_1}\left(\Gamma_{k_1}(x_i), \Gamma^{-1}_{k_2}\left(\Gamma_{k_2}(x_i)\right)\right)\right),\\
\left. k_1, k_2\hm \in {p},\  k_1\not= k_2\right\},\ \bm{\rho}_{\mathbf{m}}(x_i)=  \bm{\rho}_{\mathbf{m}}(x_i, [1\ldots n]).
\end{multline*}
     
     На основе $\bm{\rho}_{\mathbf{m}{\bm{\alpha}}}(x_i)$ 
и~$\bm{\rho}_{\mathbf{m}\overline{\bm{\alpha}}}(x_i)$ вводятся оценки 
релевантности~$\rho_m$. По отношению к~задаче $\mathbf{c}_{\bm{\alpha}}/ 
\overline{\mathbf{c}}_{\bm{\alpha}}$ более релевантна или 
<<информативна>> такая мет\-ри\-ка~$\rho_m$, которая для всех $x\hm\in 
\mathbf{c}_{\bm{\alpha}}$ минимизирует расстояния в~списке 
$\bm{\rho}_{\mathbf{m}{\bm{\alpha}}}(x)$ и~максимизирует расстояния 
в~списке $\bm{\rho}_{\mathbf{m}\overline{\bm{\alpha}}}(x)$ (т.\,е.\ 
<<приближает>> объекты к~их классам). Выделены два взаимосвязанных 
направления дальнейших исследований: 
\begin{enumerate}[(1)]
\item нахождение подмножеств $p$ 
признаков, <<более информативных>> для~$\rho_m$;  
\item на\-строй\-ка/вы\-бор~$\rho_m$ при фиксированном~$p$.
\end{enumerate}
     
     Для $c^\prime\hm\in L(T(\mathbf{X}))$ определим 
$\vartheta_{\mathbf{mc}}$, операцию слияния списков 
$\bm{\rho}_{\mathbf{mc}}$:
$$
\vartheta_{\mathbf{mc}}(c^\prime, c, 
{p})\hm= \bigcup\limits_{y\in c^\prime} \bm{\rho}_{\mathbf{mc}} (y,c, 
{p}).
$$
 Обозначим 
 $$
 \vartheta_{\mathbf{m}\bm{\alpha}}(\mathbf{c}, 
{p}) \!=\! \vartheta_{\mathbf{mc}}(\mathbf{c}, 
\mathbf{c}_{\bm{\alpha}}, {p});\ 
\vartheta_{\mathbf{m}\bm{\alpha}}(\mathbf{c},{p})\!=\! 
\vartheta_{\mathrm{mc}}(\mathbf{c}, \overline{\mathbf{c}}_{\bm \alpha}, 
{p}),
$$
 вычислим множества $\vartheta_{\mathbf{m}{\bm \alpha}} 
(\mathbf{c}_{\bm \alpha}, {p})$ и~$\vartheta_{\mathbf{m}{\bm \alpha}} 
(\overline{\mathbf{c}}_{\bm \alpha},{p})$ и~сформируем 
эмпирические функции распределения (э.ф.р.)\ $\hat{\phi}(x) 
\vartheta_{\mathbf{m}{\bm \alpha}} (\mathbf{c}_{\bm \alpha}, {p})$ 
и~$\hat{\phi}(x) \vartheta_{\mathbf{m}{\bm \alpha}} 
(\overline{\mathbf{c}}_{\bm \alpha}, {p})$. На пространстве 
однородных монотонно возрастающих функций 
\begin{multline*}
\mathbf{M}^+_{0\ldots1} ={}\\
{}= 
\{f: [0\ldots 1]\hm\to [0\ldots 1],\ x\geq y\hm\Rightarrow f(x)\geq f(y)\}
\end{multline*}
введем 
функционал расстояния $d_f$: $\mathbf{M}^+_{0..1}\hm\to [0\ldots 1]$ 
(максимальное уклонение Колмогорова $D(f(x), g(x))\hm= \mathrm{sup}_x 
\vert f(x)\hm- g(x)\vert$, метрики фон Мизеса, Реньи и~др.). Выбор~$d_f$ 
делает возможной постановку ряда задач топологического анализа данных:
     \begin{enumerate}[(1)]
\item количественные оценки релевантности~$\rho_m$ как 
$d_f(\hat{\phi}(x)\vartheta_{\mathbf{m}{\bm \alpha}}(\mathbf{c}_{\bm \alpha}, 
{p}), \hat{\phi}(x)\vartheta_{\mathbf{m}{\bm 
\alpha}}(\overline{\mathbf{c}}_{\bm \alpha}, {p}))$ для 
разных~$\mathbf{c}_{\bm \alpha}$, $\lambda_{t\alpha} \hm\in \mathrm{I}_t$, 
$\alpha \hm= \overline{1, \vert \mathrm{I}_t\vert}$;
\item задачи оптимизации для увеличения разделения классов 
$\mathbf{c}_{\bm \alpha}/\overline{\mathbf{c}}_{\bm \alpha}$ 
($\argmax_{\rho_m,{p}} d_f(\hat{\phi}\vartheta_{\mathbf{m}{\bm \alpha}}(\overline{\mathbf{c}}_{\bm \alpha},{p}), 
\hat{\phi}\vartheta_{\mathbf{m}{\bm \alpha}}(\mathbf{c}_{\bm \alpha}, {p}))$,
$\argmax_{\rho_m,{p}} d_f(\hat{\phi}\vartheta_{\mathbf{m}\overline{\bm{\alpha}}}, (\overline{\mathbf{c}}_{\bm \alpha}, {p}), 
\hat{\phi}\vartheta_{\mathbf{m}\overline{\bm \alpha}}
(\mathbf{c}_{\bm \alpha}, {p}))$  и~др.);
\item определение $\rho_q$-мет\-рик на пространстве объектов~[2, с.~184--199] 
(например, в~виде $d_f (\hat{\phi}\bm{\rho}_{\mathbf{m}{\bm \alpha}}(x, 
{p}), \hat{\phi}\bm{\rho}_{\mathbf{m}{\bm \alpha}}(y, {p})), 
d_f (\hat{\phi}\bm{\rho}_{\mathbf{m}}(x,{p})$, 
$\hat{\phi}\bm{\rho}_{\mathbf{m}}(y, {p}))$); 
\item оценка близости метрик~$\rho_q$ к~метрике разреза по классам 
$\mathbf{c}_{\bm{\alpha}}/ \overline{\mathbf{c}}_{\bm{\alpha}}$; 
\item формулировка критериев раз\-ре\-ши\-мости/ре\-гу\-ляр\-ности задачи 
$\mathbf{c}_{\bm{\alpha}}/ \overline{\mathbf{c}}_{\bm{\alpha}}$~[6]; 
\item оценки компактности классов $\mathbf{c}_{\bm{\alpha}}$  
и~$\overline{\mathbf{c}}_{\bm{\alpha}}$~[3]. 
\end{enumerate}

\section{О способах порождения и~отбора синтетических 
признаков на~основании функций расстояния}

     Множества $\bm{\rho}_{\mathbf{m}{\bm \alpha}}(x_i,{p})$, 
$\bm{\rho}_{\mathbf{m}{\overline{\bm \alpha}}}(x_i, {p})$ 
и~$\bm{\rho}_{\mathbf{m}}(x_i)$ и~отдельные $\rho_m(\mathbf{c}_{\bm 
\alpha}, \Gamma_k^{-1}(\Gamma_k(x_i))$ используются для формирования 
синтетических числовых признаков $\Gamma_{k^\prime}(x_i)$ 
объекта~$x_i$, $k^\prime\hm= \overline{n+ l+1, n+l+n_S}$. 
Значение синтетического признака~$\Gamma_{k^\prime}(x_i)$ зависит от 
выбора~$\rho_m$, классов $\mathbf{c}_{\bm{\alpha}}$ 
и~$\overline{\mathbf{c}}_{\bm{\alpha}}$  и~от способа его вы\-чис\-ле\-ния: 
\begin{enumerate}[(1)]
\item $\rho_m(\mathbf{c}_{\bm \alpha}, \Gamma_k^{-1}(\Gamma_k(x_i))$; 
\item $\rho_m(\overline{\mathbf{c}}_{\bm \alpha}, \Gamma_k^{-1}(\Gamma_k(x_i))$; 
\item $\rho_m(\mathbf{c}_{\bm \alpha}, \ldots ) \hm- \rho_m(\overline{\mathbf{c}}_{\bm \alpha}, \ldots)$;
\item $1\hm- \rho_m(\mathbf{c}_{\bm \alpha}, \ldots)$;
\item значения э.ф.р.\ 
$\hat{\phi}(x)\bm{\rho}_{\mathbf{m}{\bm \alpha}}(x_i,{p})$ при 
разных~$x$ (например, соответствующих процентилям 
$\hat{\phi}\bm{\rho}_{\mathbf{m}{\bm \alpha}}(x_i,{p})$); 
\item значения $\hat{\phi}(x)\bm{\rho}_{\mathbf{m}\overline{\bm{\alpha}}} 
(x_i, {p})$ при разных~$x$;
\item $\hat{\phi}(x\hm+ \Delta x) 
\bm{\rho}_{\mathbf{m}{\bm \alpha}}(x_i,p) \hm- 
\hat{\phi}(x)\bm{\rho}_{\mathbf{m}{\bm \alpha}} (x_i, {p})$ 
и~$\hat{\phi}(x\hm+ \Delta x) \bm{\rho}_{\mathbf{m}{\overline{\bm \alpha}}} 
(x_i,{p}) \hm- \hat{\phi}(x)\bm{\rho}_{\mathbf{m}\overline{\bm 
\alpha}} (x_i, {p})$, где $\Delta x$~--- шаг.
\end{enumerate}
     
     Кроме того, $\mathbf{c}_{\bm{\alpha}}$ может определяться как 
$\Gamma_t^{-1}(\lambda_{t\alpha})$ или как $u(\lambda_{t\alpha})$; если 
$\mathbf{c}_{\bm \alpha} \hm= \Gamma_t^{-1}(\lambda_{t\alpha})$, то 
$\overline{\mathbf{c}}_{\bm{\alpha}}$ может быть равно $\Gamma^{-1}_t 
(\lambda_{t\alpha+1})$; классы $\mathbf{c}_{\bm{\alpha}}/ 
\overline{\mathbf{c}}_{\bm{\alpha}}$  
$t$-й переменной могут определяться с~использованием раз\-би\-ений на 
различные процентили (которые определяются как подвыборка значений 
$\lambda_{t\alpha} \hm\in \mathrm{I}_t$) и~т.\,д. 
     
     Таким образом, предлагаемые схемы порождают значительное число 
синтетических признаков $\Gamma_{k^\prime}(x_i)$ ($10n$ и~более при $n$ 
исходных признаках $\Gamma_k$), что делает необходимым введение 
процедур отбора признаков. Таргетная переменная $\Gamma_t(x_i)$~--- 
чис\-ло\-вая, и~по\-рож\-да\-емые признаки $\Gamma_{k^\prime}(x_i)$~--- также 
чис\-ло\-вые. Для данного случая в~прикладной математике имеется несколько 
различных подходов к~оценке взаимосвязи $\Gamma_t(x_i)$ 
и~$\Gamma_{k^\prime}(x_i)$: корреляционные оценки (для линейных 
закономерностей), полиномная аппроксимация с~оценкой качества (для 
нелинейных закономерностей) и~методы теории  
ве\-ро\-ят\-но\-стей\,/\,ма\-те\-ма\-ти\-че\-ской статистики, не зависящие от 
вида закономерности (в~том числе на основе <<взаимной 
информации>>~[7]).
{\looseness=1

}
     
     Наиболее фундаментальным представляется тес\-ти\-ро\-ва\-ние взаимосвязи 
двух переменных на осно\-ве <<нулевой гипотезы>> об их независимости. 
Пусть заданы пары тестируемых значений, $(x_i, y_i)$,\linebreak $i\hm= \overline{1,\mathbf{n}_{(\mathrm{x,y})}}$, э.ф.р.~$F_{xy}(x,y)$ характеризует 
совместное распределение~$x$ и~$y$, а~э.ф.р.~$F_{{x}}(x)$ 
и~$F_{{y}}(y)$~--- индивидуальные распределения переменных. 
Эмпирическая функция распределения нулевой \mbox{гипотезы} (независимость~$x$ и~$y$) определяется как 
$F_{{x}}(x)F_{{y}}(y)$. 
     
     Для оценки отличий между $F_{{xy}}(x,y)$\linebreak 
и~$F_{{x}}(x) F_{{y}}(y)$ необходимо ввести расстояние 
меж-\linebreak ду такими функциями (так называемую <<статисти-\linebreak ку>>) и~оценить 
достоверность различий посред\-ст\-вом \mbox{того} или иного статистического\linebreak \mbox{тес\-та}. 
В~качестве расстояния можно использовать функции~$d_f$, адап\-ти\-ро\-ван\-ные 
для 2-мер\-но\-го случая (например, макси\-маль\-ное уклонение 
     $D(\mathrm{F}_{{xy}}(x,y), \mathrm{F}_{{x}}(x) 
\mathrm{F}_{{y}}(y)) \hm= \max ( \vert 
\mathrm{F}_{{xy}}(x_i,y_i) \hm- \mathrm{F}_{{x}}(x_i) 
\mathrm{F}_{{y}}(y_i)\vert )$) и~статистический тест  
Кол\-мо\-го\-ро\-ва--Смир\-но\-ва 
$P_{\mathrm{КС}}$ $(D 
(\mathrm{F}_{{xy}}(x,y), \mathrm{F}_{{x}}(x) 
\mathrm{F}_{{y}}(y)), n_{(x,y)})$. Тогда $1\hm- 
P_{\mathrm{КС}}$ характеризует <<информативность>>~$x$ 
относительно~$y$. 
     
     Более универсальным подходом к~оценке достоверности различий 
между $\mathrm{F}_{{xy}}(x,y)$ и~$\mathrm{F}_{{x}}(x) 
\mathrm{F}_{{y}}(y)$ считается прямое вычисление выбранной 
статистики~$d_f$ на множествах пар значений $(x_i, y_i)$, полученных 
датчиком случайных чисел. 
     
     Пусть \textit{оператор $\hat{\zeta}$, семплирующий} 
множество~$\mathbf{X}$, формирует набор семплов 
$$
\hat{\zeta}\mathbf{X}\hm= \{a_1, a_2, \ldots , a_k, \ldots , 
a_{\vert\hat{\zeta}X\vert}\vert a_k\hm\subset \mathbf{X}\},
$$
 а~процедура 
random~--- датчик случайных чисел (в~диапазоне $[0\ldots 1]$). Для каждого 
семпла~$a_k$ принимается, что ${n}_{({x,y})} \hm= \vert 
a_k\vert$, и~вычисляется множество значений~$d_f$ для случайных 
выборок, 

\noindent
\begin{multline*}
\mathrm{rnd}\,(\hat{\zeta}\mathbf{X}, d_f)= \left\{ 
\vphantom{i=\overline{1,\left\vert \hat{\zeta} X\right\vert }}
d_f\left(
\vphantom{\overline{1, \vert a_i\vert }}
\mathrm{F}_{{xy}}(x_{ij}, y_{ij}), 
\mathrm{F}_{{x}}(x_{ij}) \mathrm{F}_{{y}}(y_{ij}),\right.\right.\\
\left.\left. x_{ij}, 
y_{ij}= \mathrm{random},\  j=\overline{1, \vert a_i\vert }\right),\ i=\overline{1,\left\vert \hat{\zeta} X\right\vert }\right\}.
\end{multline*}

 Для $a\hm\in \hat{\zeta} \mathbf{X}$ значение 
${P}(d_f, \hat{\zeta}\mathbf{X}, a, k^\prime, t)\hm= 1\hm-
\hat{\phi}(d_f(\mathrm{F}_{k^\prime t}(\Gamma_{k^\prime}(z), \Gamma_t(z)), F_{k^\prime}(\Gamma_{k^\prime}(z)) 
\mathrm{F}_t(\Gamma_t(z)))\vert z\hm\in a) \mathrm{rnd}\,(\hat{\zeta}\mathbf{X}, d_f)$~--- статистическая достоверность 
<<зависимости>> $\Gamma_t(z)$ и~$\Gamma_{k^\prime}(z)$ по 
статистике~$d_f$ на семпле~$a$, а~$1\hm- P(d_f, 
\hat{\zeta}\mathbf{X}, a, k^\prime, t)$ количественно оценивает зависимость.
    

При заданном способе оценки зависимости $1\hm- P(d_f, 
\hat{\zeta}\mathbf{X}, a, k^\prime, t)$ задача отбора информативных 
признаков решается посредством так называемого\linebreak  
В-ал\-го\-рит\-ма, исходно разработанного для построения оптимальных 
словарей финальных ин\-фор\-маций (чему и~соответствует литера~<<В>>)~[8]. 
\mbox{Данный} алгоритм, основанный на критерии раз\-ре\-ши\-мости по Журавлёву, 
позволяет выбирать множества финальных информаций на основе 
максимального час\-тич\-но\-го покрытия при минимуме\linebreak элементов покрытия. 
Замена мощности пересечения множеств на $1\hm- P(d_f, 
\hat{\zeta}\mathbf{X}, a, k^\prime, t)$ приведет к~тому, что  
В-ал\-го\-ритм будет выбирать минимум признаков с~максимальной 
<<информативностью>>\linebreak (наиболее информативные признаки, см.\ 
теоремы~1, 7  и~8 работы~[8]).

    Таким образом, в~рамках развиваемого формализма синтез более 
информативных синтетических~$\Gamma_{k^\prime}(x_i)$ осуществляется 
в~5~стадий: 
\begin{enumerate}[(1)]
\item определяется набор исходных (как правило, 
<<низкоинформативных>>) признаков~$\Gamma_k(x_i)$ и~таргетная 
переменная~$\Gamma_t(x_i)$;
\item вводится набор метрик~$\rho_m$, 
оценивается их релевантность $d_f(\hat{\phi}(x)\vartheta_{\mathbf{m}{\bm 
\alpha}}(\mathbf{c}_{\bm \alpha},{p})$,\linebreak 
$\hat{\phi}(x)\vartheta_{\mathbf{m}{\bm \alpha}}(\overline{\mathbf{c}}_{\bm 
\alpha}, {p}))$ для каждого класса~$\mathbf{c}_{\bm \alpha}$ 
значений $t$-й переменной и~отбираются наиболее релевантные~$\rho_m$; 
\item посредством каждой из отобранных~$\rho_m$ по\-рож\-да\-ют\-ся 
синтетические признаки~$\Gamma_{k^\prime}(x_i)$;
\item посредством 
вычислений $1\hm- P(d_f, \hat{\zeta}\mathbf{X}, a, k^\prime, t)$  
и~В-ал\-го\-рит\-ма отбирается минимальное чис\-ло признаков максимальной 
<<ин\-фор\-ма\-тив\-ности>>;
\item применяется алгоритм прогнозирования 
таргетной переменной (корректор по Жу\-рав\-лё\-ву--Ру\-да\-кову). 
\end{enumerate}

\begin{table*}\small
\begin{center}
\begin{tabular}{|l|c|c|}
\multicolumn{3}{p{140mm}}{Ранговые корреляции между экспериментальными 
и~расчетными значениями $EC_{50}$ и~других величин хемокиномного анализа: $r$~--- 
коэффициент ранговой корреляции на обучении; $r_c$~---  на контроле. Усреднение~$r$ 
и~$r_c$ проводилось по 2400~выборкам хемокиномных данных}\\
\multicolumn{3}{c}{\ }\\[-6pt]
\hline
\multicolumn{1}{|c|}{{Эксперимент}}&$r$&$r_c$\\
\hline
{\boldmath $f_{\theta_k}$}\textbf{-алгоритмы, корректор~--- нейросеть}&\boldmath{$0{,}88\pm 
0{,}15$}&\boldmath{$0{,}86\pm0{,}20$}\\
Синтетические $\Gamma_{k^\prime}(x_i)$, корректор~--- нейросеть (2~слоя)&$0{,}45\pm 
0{,}22$&$0{,}22\pm 0{,}21$\\
Синтетические $\Gamma_{k^\prime}(x_i)$, корректор~--- нейросеть 
(10~слоев)&$0{,}52\pm 0{,}25$&$0{,}21\pm 0{,}20$\\
Синтетические $\Gamma_{k^\prime}(x_i)$, корректор~--- <<случайный лес>>, 
вариант~1&$0{,}98\pm 0{,}15$&$0{,}67\pm 0{,}31$\\
Синтетические $\Gamma_{k^\prime}(x_i)$, корректор~--- <<случайный лес>>, 
вариант~2&$0{,}99\pm 0{,}14$&$0{,}71\pm 0{,}35$\\
\textbf{Синтетические {\boldmath $\Gamma_{k^\prime}(x_i)$}, полиномные корректоры, 
вариант~1}&\boldmath{$0{,}93\pm 0{,}11$}&\boldmath{$0{,}90\pm 0{,}23$}\\
\textbf{Синтетические {\boldmath $\Gamma_{k^\prime}(x_i)$}, полиномные корректоры, 
вариант~2}&\boldmath{$0{,}95\pm0{,}08$}&\boldmath{$0{,}86\pm 0{,}27$}\\
\hline
\end{tabular}
\end{center}
\end{table*}

\section{Экспериментальная апробация }

    Формализм апробирован на комплексе задач\linebreak фармакоинформатики: 
получение количественных оценок ингибирования киназ протеома 
перспективными лекарствами (хемокиномный анализ)~[9]. Использованы 
2400~выборок данных <<\mbox{мо\-ле\-ку\-ла}--свой\-ст\-во>> из ProteomicsDB; 
свойства молекул включили константы $EC_{50}$ и~активности для 
концентраций~$(E_j(C_i))$.

     Исходные признаки $\Gamma_k(x_i)$ определялись как булевы 
инварианты над множествами $\chi$-це\-пей и~$\chi$-уз\-лов 
хемографов~$x_i$, как и~в~[9]. Таргетная $\Gamma_t(x_i)$ определялась как 
числовое значение прогнозируемого свойства. В~качестве~$\rho_m$ 
использовались функции расстояния на множествах, векторах и~э.ф.р.\ (всего 
65~функций из справочника~[2]). Классы~$\mathbf{c}_{\bm{\alpha}}$ 
определялись как квартили значений~$\Gamma_t$. Векторы элементов 
$L(T(\mathbf{X}))$ формировались из оценок $v^+_\alpha$, $v^-_\alpha$ 
и~$d_\alpha$~\cite{4-tor} для каждого~$\mathbf{c}_{\bm{\alpha}}$. 
Релевантность~$\rho_m$ по $d_f(\hat{\phi}(x),\vartheta_{\mathbf{m}{\bm 
\alpha}}(\mathbf{c}_{\bm{\alpha}},{p}), 
\hat{\phi}(x)\vartheta_{\mathbf{m}{\bm \alpha}} 
(\overline{\mathbf{c}}_{\bm{\alpha}}, {p}))$ оценивалась для 
каждого~$\mathbf{c}_{\bm{\alpha}}$, $d_f$~--- максимальное уклонение. 
Синтетические признаки~$\Gamma_{k^\prime}(x_i)$ по\-рож\-да\-лись всеми 
перечисленными выше способами; их отбор проводился В-ал\-го\-рит\-мом 
с~использованием $1\hm- {P}(d_f, \hat{\zeta}\mathbf{X}, a, 
     k^\prime, t)$. 
     
     В качестве корректоров использовались нейронные сети с~несколькими 
слоями (от~2 до~10) с~функцией активации softmax, полиномы различных 
конструкций (более 20~формул, в~том числе квазиполиномные модели 
с~элементарными функциями) и~<<случайные леса>> решающих деревьев. 
Оператор семплирования~$\hat{\zeta}$ был реализован как десятикратная  
кросс-ва\-ли\-да\-ция с~делением каждой выборки объектов на 80\% 
(обучение) и~20\% (конт\-роль). Результаты экспериментов суммированы 
в~таблице.
     

     
     Наилучший результат применения нового <<топологического>> 
формализма с~полиномным корректором ($r_c\hm=0{,}90\hm\pm0{,}23$) 
немного превзошел наилучший результат применения \mbox{метода} опорных 
функций (композиций вида $f_{\theta_k} \hm= g(f_1(\sum \omega_k^j x_k), 
\ldots\linebreak \ldots , f_l(\sum \omega_k^j x_k))$, см.~[9]), для которого 
$r_c\hm=0{,}86\hm\pm0{,}20$. Полиномными формулами, наиболее часто 
показывавшими наилучший результат, оказались полиномы 1-й или 2-й 
степеней с~произведениями переменных первой степени, полиномы 5-й 
степени, квазиполиномы 5-й степени с~сигмоидами и~Фурье-по\-ли\-но\-мы  
3-й степени.
     
     Нейросетевые корректоры всех использованных конфигураций 
отличались крайне низкими показателями ($r\hm=0{,}45\hm\pm0{,}22$, 
$r_c\hm=0{,}22\hm\pm0{,}21$), а~<<случайный лес>> приводил 
к~существенному переобучению (см.\ таб\-ли\-цу). При этом в~290 
из~2400~выборок данных (12\%) <<случайный лес>> приводил к~улучшению 
результатов по сравнению с~наилучшими полиномными корректорами, 
а~в~1670 из 2400~выборок данных (70\%)~--- к~ухудшению.
     
     
     Анализ синтетических признаков $\Gamma_{k^\prime}(x_i)$, 
вошедших в~наилучшие полиномные модели, показал, что среди более 
информативных (по оценке $1\hm- P(d_f, \hat{\zeta}\mathbf{X},  
a, k^\prime, t)$) признаков чаще всего встречались признаки, порождаемые 
с~использованием э.ф.р.\ на основе опорных цепей (теорема~1 в~1-й части 
работы~[1]), среди наименее информативных~--- исходные признаки 
$\Gamma_k(x_i)$ и~признаки на основе отдельных расстояний 
$\rho_m(\mathbf{c}_{\bm{\alpha}} , \Gamma_k^{-1}(\Gamma_k(x_i))$. 
Функциями~$\rho_m$, наиболее часто порождающими информативные 
$\Gamma_{k^\prime}(x_i)$ на пространстве э.ф.р., оказались максимальное 
уклонение Колмогорова, <<косое>> расстояние, метрики $\mathrm{Lp}$, 
Реньи, $\chi2$, фон Мизеса, инженерная~\cite{2-tor}. В~среднем по всем 
выборкам данных эти~7~разновидностей~$\rho_m$ порождали более 50\% 
самых информативных признаков~$\Gamma_{k^\prime}(x_i)$, отобранных  
В-ал\-го\-рит\-мом.

\vspace*{-6pt}

\section{Заключение}

\vspace*{-2pt}

    Предлагаемый подход к~порождению информативных синтетических 
признаков подразумевает последовательные трансформации описаний 
объекта:\\[-13pt]
\begin{enumerate}[(1)]
\item исходное множество значений признаков;\\[-13.5pt]
\item множество 
соответствующих элементов решетки;\\[-13.5pt] 
\item ~множество расстояний 
(измеряемых посредством~$\rho_m$) между элементами решетки, 
соответствующими классам и~признакам;\\[-13.5pt]
\item множество э.ф.р.\ расстояний, 
измеренных заданными~$\rho_m$;\\[-13.5pt] 
\item множество синтетических признаков 
объ-\linebreak екта.
\end{enumerate}

\noindent
 Использование многочисленных метрик на стадии порождения 
признаков позволяет рассматривать развиваемый формализм как вариант 
развития идеологии АВО (алгоритмы вычисления \mbox{оценок}) научной школы 
Ю.\,И.~Журавлёва. Экспериментальная апробация предлагаемого подхода на 
2400~однородных задачах фармакоинформатики позволила повысить 
аккуратность и~обобщающую способность алгоритмов. 


{\small\frenchspacing
 {\baselineskip=10.6pt
 %\addcontentsline{toc}{section}{References}
 \begin{thebibliography}{99}
  
  \bibitem{1-tor}
\Au{Торшин И.\,Ю.} О~порождении синтетических признаков на основе опорных цепей 
и~произвольных метрик в~рамках топологического подхода к~анализу данных. Часть~1. 
Включение в~формализм эмпирических функций расстояния~// Информатика и~её 
применения, 2024. Т.~18. Вып.~1. С.~71--77. doi: 10.14357/19922264240110. EDN: 
RIVOXR.
  \bibitem{2-tor}
  \Au{Деза Е.\,И., Деза~М.\,М.} Энциклопедический словарь расстояний~/ Пер. с~англ.~--- М.: Наука, 
2008. 444~с. (\Au{Deza~E.\,I., Deza~M.\,M.} {Dictionary of distances}.~--- North-Holland: 
Elsevier, 2006. 412~p. doi: 10.1016/B978-0-444-52087-6.X5000-8.)
  \bibitem{3-tor}
  \Au{Torshin I.\,Y., Rudakov~K.\,V.} Combinatorial analysis of the solvability properties of 
the problems of recognition and completeness of algorithmic models. Part~2: Metric approach 
within the framework of the theory of classification of feature values~// Pattern Recognition Image 
Analysis, 2017. Vol.~27. No.\,2. P.~184--199. doi: 10.1134/S1054661817020110.
  \bibitem{4-tor}
\Au{Торшин И.\,Ю.} О~формировании множеств прецедентов на основе таблиц 
разнородных признаковых описаний методами топологической теории анализа данных~// 
Информатика и~её применения, 2023. Т.~17. Вып.~3. С.~2--7. doi: 
10.14357/19922264230301. EDN: AQEUYO.
  \bibitem{5-tor}
  \Au{Torshin I.\,Yu., Rudakov~K.\,V.} On the procedures of generation of numerical features 
over partitions of sets of objects in the problem of predicting numerical target variables~// 
Pattern Recognition Image Analysis, 2019. Vol.~29. No.\,4. P.~654--667. doi: 
10.1134/S1054661819040175. 
  \bibitem{6-tor}
  \Au{Torshin I.\,Y., Rudakov~K.\,V.} Combinatorial analysis of the solvability properties of 
the problems of recognition and completeness of algorithmic models. Part~1: Factorization 
approach~// Pattern Recognition Image Analysis, 2017. Vol.~27. No.\,1. P.~16--28. doi: 
10.1134/S1054661817010151.
  \bibitem{7-tor}
  \Au{Sosa-Cabrera G., G$\acute{\mbox{o}}$mez-Guerrero~S.,  
Garc$\acute{\iota}$a-Torres~M., Schaerer~C.\,E.} Feature selection: A~perspective on inter-attribute 
cooperation~// Int. J. Data Science Analytics, 2024. Vol.~17. P.~139--151. doi:  
10.1007/s41060-023-00439-z.
  \bibitem{8-tor}
  \Au{Torshin I.\,Y.} Optimal dictionaries of the final information on the basis of the solvability 
criterion and their applications in bioinformatics~// Pattern Recognition Image Analysis, 2013. 
Vol.~23. No.\,2. P.~319--327. doi: 10.1134/S1054661813020156.
  \bibitem{9-tor}
\Au{Торшин И.\,Ю.} О~задачах оптимизации, воз\-ни\-ка\-ющих при применении 
топологического анализа данных к~поиску алгоритмов прогнозирования 
с~фиксированными корректорами~// Информатика и~её применения, 2023. Т.~17. Вып.~2. 
С.~2--10. doi: 10.14357/19922264230201. EDN: IGSPEW.

\end{thebibliography}

 }
 }

\end{multicols}

\vspace*{-8pt}

\hfill{\small\textit{Поступила в~редакцию 09.04.24}}

\vspace*{6pt}

%\pagebreak

%\newpage

%\vspace*{-28pt}

\hrule

\vspace*{2pt}

\hrule



\def\tit{ON THE GENERATION OF~SYNTHETIC FEATURES BASED~ON~SUPPORT~CHAINS 
AND~ARBITRARY METRICS\\ WITHIN THE~FRAMEWORK OF~A~TOPOLOGICAL 
APPROACH\\ TO~DATA ANALYSIS. PART~2. EXPERIMENTAL TESTING 
ON~PHARMACOINFORMATICS PROBLEMS}


\def\titkol{On the generation of~synthetic features based on~support chains 
and~arbitrary metrics} % within the~framework of~a~topological  approach to~data analysis. Part~2. Experimental testing  on~pharmacoinformatics problems}


\def\aut{I.\,Yu.~Torshin}

\def\autkol{I.\,Yu.~Torshin}

\titel{\tit}{\aut}{\autkol}{\titkol}

\vspace*{-15pt}


\noindent
Federal Research Center ``Computer Science and Control'' of the Russian Academy of 
Sciences, 44-2~Vavilov Str., Moscow 119333, Russian Federation

\def\leftfootline{\small{\textbf{\thepage}
\hfill INFORMATIKA I EE PRIMENENIYA~--- INFORMATICS AND
APPLICATIONS\ \ \ 2024\ \ \ volume~18\ \ \ issue\ 2}
}%
 \def\rightfootline{\small{INFORMATIKA I EE PRIMENENIYA~---
INFORMATICS AND APPLICATIONS\ \ \ 2024\ \ \ volume~18\ \ \ issue\ 2
\hfill \textbf{\thepage}}}

\vspace*{3pt}
  
  


\Abste{Consideration of precedent relationships between features and a target variable in the 
form of sets of Boolean lattice elements indicates the possibility of generating synthetic features 
using metric distance functions. Approaches to ($i$)~assessing the relevance (``informativeness'') 
of metrics in relation to the problems being solved; ($ii$)~generating; and ($iii$)~selecting synthetic 
features that are more informative than the original feature descriptions are formulated. The 
results of topological analysis of~2400~samples of ``molecule--property'' data
from 
ProteomicsDB made it possible to obtain fairly effective algorithms for 
predicting the properties of molecules (rank correlation in cross-validation is~$0.90\pm 0.23$). 
Using this sample of problems, metrics have been established\linebreak\vspace*{-12pt}}

\Abstend{that most often generate 
informative synthetic features: maximum Kolmogorov deviation, ``oblique'' distance, and Lp, Renyi, 
and von Mises metrics. To solve the studied set of problems, the advantage of polynomial 
correctors compared to neural network and random forest correctors is shown.}

\KWE{topological data analysis; lattice theory; algebraic approach of Yu.\,I.~Zhuravlev; 
pharmacoinformatics}




\DOI{10.14357/19922264240207}{OTXCUD}

%\vspace*{-12pt}

\Ack

\vspace*{-3pt}


\noindent
The research was funded by the Russian Science Foundation, project No.\,23-21-00154. The 
research was carried out using the infrastructure of the Shared Research Facilities ``High 
Performance Computing and Big Data'' (CKP ``Informatics'') of FRC CSC RAS (Moscow).
 


  \begin{multicols}{2}

\renewcommand{\bibname}{\protect\rmfamily References}
%\renewcommand{\bibname}{\large\protect\rm References}

{\small\frenchspacing
 {%\baselineskip=10.8pt
 \addcontentsline{toc}{section}{References}
 \begin{thebibliography}{9} 
 
 %\vspace*{-3pt}
  \bibitem{1-tor-1}
\Aue{Torshin, I.\,Yu.} 2024. O~porozhdenii sinteticheskikh priznakov na osno\-ve opor\-nykh 
tsepey i~proizvol'nykh metrik v~ram\-kakh topologicheskogo podkhoda k~analizu dannykh. 
Chast'~1. Vklyuchenie v~formalizm empiricheskikh funktsiy rasstoyaniya [On the generation 
of synthetic features based on support chains and arbitrary metrics within a~topological approach 
to data analysis. Part~1. Inclusion of empirical distance functions into the formalism]. 
\textit{Informatika i~ee Primeneniya~--- Inform Appl.} 18(1):71--77. doi: 
10.14357/19922264240110. EDN: RIVOXR.
  \bibitem{2-tor-1}
\Aue{Deza, E.\,I., and M.\,M.~Deza.} 2006. \textit{Dictionary of distances}. North-Holland: 
Elsevier. 412~p. doi: 10.1016/B978-0-444-52087-6.X5000-8.
  \bibitem{3-tor-1}
\Aue{Torshin, I.\,Yu., and K.\,V.~Rudakov.} 2017. Combinatorial analysis of the solvability 
properties of the problems of recognition and completeness of algorithmic models. Part~2: 
Metric approach within the framework of the theory of classification of feature values. 
\textit{Pattern Recognition Image Analysis} 27(2):184--199. doi: 10.1134/S1054661817020110.
  \bibitem{4-tor-1}
\Aue{Torshin, I.\,Yu.} 2023. O~formirovanii mnozhestv pretsedentov na osnove tablits 
raznorodnykh priznakovykh opisaniy metodami topologicheskoy teorii analiza dannykh [On the 
formation of sets of precedents based on tables of heterogeneous feature descriptions by methods 
of topological theory of data analysis]. \textit{Informatika i~ee Primeneniya~--- Inform Appl.} 
17(3):2--7. doi: 10.14357/19922264230301. EDN: AQEUYO.
  \bibitem{5-tor-1}
\Aue{Torshin, I.\,Yu., and K.\,V.~Rudakov.} 2019. On the procedures of generation of 
numerical features over partitions of sets of objects in the problem of predicting numerical target 
variables. \textit{Pattern Recognition Image Analysis} 29(4):654--667. doi: 
10.1134/S1054661819040175.
  \bibitem{6-tor-1}
\Aue{Torshin, I.\,Y., and K.\,V.~Rudakov.} 2017. Combinatorial analysis of the solvability of 
the problems of recognition, completeness of algorithmic models. Part~1: Factorization 
approach. \textit{Pattern Recognition Image Analysis} 27(1):16--28. doi: 
10.1134/S1054661817010151.
  \bibitem{7-tor-1}
\Aue{Sosa-Cabrera, G., S.~Gуmez-Guerrero, \mbox{M.~Garc$\acute{\!\mbox{{\ptb{\i}}}}$a}-Torres, 
and C.\,E.~Schaerer.} 2024. Feature selection: A~perspective on inter-attribute cooperation. \textit{Int. J. 
Data Science Analytics} 17:139--151. doi: 10.1007/s41060-023-00439-z.
  \bibitem{8-tor-1}
\Aue{Torshin, I.\,Y.} 2013. Optimal dictionaries of the final information on the basis of the 
solvability criterion and their applications in bioinformatics. \textit{Pattern Recognition Image 
Analysis}  23(2):319--327. doi: 10.1134/ S1054661813020156.
  \bibitem{9-tor-1}
\Aue{Torshin, I.\,Yu.} 2023. O~zadachakh optimizatsii, voznikayushchikh pri primenenii 
topologicheskogo analiza dannykh k~poisku algoritmov prognozirovaniya s~fiksirovannymi 
korrektorami [On optimization problems arising from the application of topological data analysis 
to the search for forecasting algorithms with fixed correctors]. \textit{Informatika i~ee 
Primeneniya~--- Inform Appl.} 17(2):2--10. doi: 10.14357/19922264230201. EDN: IGSPEW.

\end{thebibliography}

 }
 }

\end{multicols}

\vspace*{-6pt}

\hfill{\small\textit{Received April 9, 2024}} 

\vspace*{-12pt}


\Contrl

\vspace*{-3pt}

\noindent
\textbf{Torshin Ivan Y.} (b.\ 1972)~--- Candidate of Science (PhD) in physics and mathematics, 
Candidate of Science (PhD) in chemistry, leading scientist, Federal Research Center ``Computer 
Science and Control'' of the Russian Academy of Sciences, 44-2~Vavilov Str, Moscow 119333, 
Russian Federation; \mbox{tiy135@yahoo.com}
  
  



\label{end\stat}

\renewcommand{\bibname}{\protect\rm Литература}   %6

\def\stat{smirnov}

\def\tit{ПЕРСОНАЛЬНЫЙ КОГНИТИВНЫЙ АССИСТЕНТ: ПЛАНИРОВАНИЕ ПОВЕДЕНИЯ\\ 
НА~ОСНОВЕ СЦЕНАРИЕВ ДЕЯТЕЛЬНОСТИ$^*$}

\def\titkol{Персональный когнитивный ассистент: планирование поведения 
на~основе сценариев деятельности}

\def\aut{И.\,В.~Смирнов$^1$, А.\,И.~Панов$^2$, А.\,А.~Чуганская$^3$, 
М.\,И.~Суворова$^4$, Г.\,А.~Киселёв$^5$,\\ И.\,А.~Курузов$^6$, 
О.\,Г.~Григорьев$^7$}

\def\autkol{И.\,В.~Смирнов, А.\,И.~Панов, А.\,А.~Чуганская и~др.} 
%М.\,И.~Суворова$^4$, Г.\,А.~Киселёв$^5$, И.\,А.~Курузов$^6$,  О.\,Г.~Григорьев$^7$}

\titel{\tit}{\aut}{\autkol}{\titkol}

\index{Смирнов И.\,В.}
\index{Панов А.\,И.}
\index{Чуганская А.\,А.}
\index{Суворова М.\,И.}
\index{Киселёв Г.\,А.}
\index{Курузов И.\,А.}
\index{Григорьев О.\,Г.}
\index{Smirnov I.\,V.}
\index{Panov A.\,I.}
\index{Chuganskaya A.\,A.}
\index{Suvorova M.\,I.}
\index{Kiselev G.\,A.}
\index{Kuruzov I.\,A.}
\index{Grigoriev O.\,G.}


{\renewcommand{\thefootnote}{\fnsymbol{footnote}} \footnotetext[1]
{Работа выполнена при финансовой поддержке РФФИ (проект 18-29-22027).
}}

\renewcommand{\thefootnote}{\arabic{footnote}}
\footnotetext[1]{Федеральный исследовательский центр <<Информатика и~управление>> Российской академии наук; 
Российский университет дружбы народов, \mbox{ivs@isa.ru}}
\footnotetext[2]{Федеральный исследовательский центр <<Информатика и~управление>> Российской академии наук; 
Московский физико-тех\-ни\-че\-ский институт (национальный исследовательский университет), \mbox{pan@isa.ru}}
\footnotetext[3]{Федеральный исследовательский центр <<Информатика и~управление>> Российской академии наук, 
%Российский университет дружбы народов, 
\mbox{anfisa.makh@gmail.com}}
\footnotetext[4]{Федеральный исследовательский центр <<Информатика и~управление>> Российской академии наук, 
\mbox{suvorova@isa.ru}}
\footnotetext[5]{Федеральный исследовательский центр <<Информатика и~управление>> Российской академии наук; 
Российский университет дружбы народов, \mbox{kiselev@isa.ru}}
\footnotetext[6]{Московский физико-технический институт (национальный исследовательский университет), 
\mbox{kuruzov2014@mail.ru}}
\footnotetext[7]{Федеральный исследовательский центр <<Информатика и~управление>> Российской академии наук, 
\mbox{oleggpolikvart@yandex.ru}}

\vspace*{-6pt}

  \Abst{Представлены процедуры планирования поведения когнитивного ассистента (КА) на 
основе сценариев~--- обобщенных схем решения задач. Когнитивный ассистент является 
виртуальным интеллектуальным агентом, обладающим своей собственной картиной мира 
и~строящим картину мира пользователя, которому он помогает решать различные 
повседневные или профессиональные задачи. Ключевой компонентой целенаправленного 
поведения КА являются сценарии~--- неоднократно используемые 
абстрактные последовательности действий и~ситуаций, на основе которых ассистент 
порождает конкретный план действий для пользователя. Рассмотрено понятие сценария 
в~психологической и~лингвистической интерпретации, рассмотрена возможность извлечения 
сценариев из текстов, выполнена формализация сценария и~плана поведения на основе 
знакового подхода к~представлению знаний, предложены методы синтеза плана поведения. 
Рассмотрен модельный пример синтеза плана поведения для задачи ассистирования при 
покупке автомобиля.}
  
  \KW{когнитивный ассистент; сценарий деятельности; планирование поведения}
  
\DOI{10.14357/19922264220107}
  
\vspace*{-6pt}


\vskip 10pt plus 9pt minus 6pt

\thispagestyle{headings}

\begin{multicols}{2}

\label{st\stat}
  
\section{Введение}

  Настоящая работа развивает концепцию персонального КА, 
  представленную в~статье~\cite{1-sm}. 
  %
  Когнитивный ассистент 
является виртуальным интеллектуальным агентом, обладающим знаковой 
\mbox{картиной} мира и~действующим на основе по\-стро\-ения плана действий 
и~сценария поведения пользователя, которому он помогает решать 
повседневные или профессиональные задачи. 
%
Когнитивный\linebreak ас\-сис\-тент 
действует проактивно, предсказывая поведение пользователя, обладает 
целеполаганием и~инструментами мотивационного характера, 
обес\-пе\-чи\-ва\-ющи\-ми достижение поставленных перед ним целей. 
  
  Ключевой компонентой целенаправленного поведения КА
   выступают сценарии~--- неоднократно используемые абстрактные 
последовательности действий и~ситуаций~\cite{2-sm, 3-sm}, стереотипные 
последовательности поведения в~определенных обстоятельствах. Они 
выступают обобщенными схемами решения задач, на основе которых 
ассистирующая система может предлагать конкретный план для пользователя. 
%
Знания о~возможной последовательности действий для достижения цели 
и~способность формировать из них конкретный план (алгоритм) для 
пользователя отличает КА от существующих аналогов, 
которые, как правило, выполняют только функцию голосового интерфейса 
к~различным сервисам. 
Рассмотрим далее концепцию сценариев деятельности 
и~способы порождения из них планов поведения применительно 
к~КА.

\vspace*{-9pt}

\section{Сценарий и~план поведения}

\vspace*{-2pt}

\subsection{Психологические основы концепций сценария и~плана
поведения}

\vspace*{-1pt}

  В психологических исследованиях понятие <<сценария>> поведения возникло 
для объяснения социально обусловленных нор\-ма\-тив\-но-ро\-ле\-вых 
последовательностей действий в~рамках определенной деятельности. 
Первоначально термин был предложен Э.~Берном. Американский психолог 
обозначил трансакцией единицу общения между людьми, когда 
коммуникативный стимул вызывает соответствующую реакцию~\cite{4-sm}. 
Такие взаимосвязанные последовательности образуют ритуальные, т.\,е.\ 
социально обоснованные повторяющиеся действия. <<Существенной 
особенностью и~процедур, и~ритуалов мы считаем то, что они  
стереотипны>>~\cite[с.~14]{4-sm}. Для построения ри\-ту\-ала-сце\-на\-рия 
значима принципиальная предсказуемость, когда появление одной трансакции 
с~большой вероятностью определяет появление другой. Берн назвал такую 
форму взаимоотношений людей играми, а~набор игр~--- жизненными 
сценариями, которые формируются у~детей и~определяют дальнейшие 
стратегии их поведения. 
  
  В близкой Э. Берну методологической парадигме бихевиоризма развивались 
идеи когнитивных исследований. Росс и~Нисбетт отмечают: <<В~основе 
концепции сценариев лежит представление о~том, что люди вступают 
в~пред\-ска\-зу\-емые, едва ли не ритуальные взаимодействия в~попытке 
удовлетворить свои по\-треб\-но\-сти ценою насколько воз\-мож\-но малого 
социального напряжения и~когнитивных усилий>>~\cite[с.~145]{5-sm}. 
В~таком рассмотрении сценария обозначается еще одна его значимая 
функция~--- экономия когнитивных ресурсов, что позволяет повышать 
адаптивные возможности пси\-хики.
{ %\looseness=1

}
  
  Принципиально иначе выглядит деятельностный подход к~поведению. 
В~концепции А.\,Н.~Ле\-онть\-ева~\cite{6-sm} ключевым становится анализ 
деятельности и~предпосылок ее возникновения. В~таком же методологическом 
ключе рассуждал Д.\,Н.~Узнадзе, утверждая, что поведение невозможно понять 
без мотива, соотнесения с~объективной ситуацией и~<<нуж\-ностью>> предмета 
потребности. Все эти\linebreak функции объединяет установка, которая и~определяет 
путь к~реализации цели. В~содержательном плане это дает начальную точку 
для реализации сценария и~построения плана. Узнадзе \mbox{разделяет} 
установки на два типа: индивидуальные и~<<опосредованные чужой 
объективацией>>~\cite[с.~47]{7-sm}.\linebreak Если первый возникает в~ходе 
собственной дея\-тель\-ности человека и,~как правило, на\-прав\-лен на об\-ласть 
предметных взаимодействий или характерен для ситуаций затруднения 
(встреча с~новыми явлениями для субъекта), то второй тип установок 
социально обуслов\-лен. Этот второй тип перешел <<в~достояние людей в~виде 
\textit{готовых \mbox{формул}}, не тре\-бу\-ющих более непосредственного учас\-тия 
процессов объективации. Источником, откуда черпаются такого рода формулы, 
является воспитание и~обуче\-ние>>~\cite[с.~203]{8-sm}. Таким образом, 
содержательное разграничение плана и~сценария воз\-мож\-но за счет выделения 
актуально дей\-ст\-ву\-ющей уста\-нов\-ки де\-я\-тель\-ности и~уров\-ня ее реа\-ли\-за\-ции.
{\looseness=1

}
  
  В когнитивной науке встречались подходы, которые предлагали фиксировать 
фреймовую структуру сценариев в~виде набора речевых параметров 
(М.~Минский, Ч.~Филмор и~др.). В~этом плане на современном этапе развития 
общества такими источниками могут выступать сетевые тексты, в~которых 
в~словесной форме фиксируется сценарная информация~\cite{9-sm}. 
Индивидуальные сценарии в~рамках коммуникации могут переходить 
в~социальные. В~рамках индивидуального сценария возможно построение 
плана, который реализуется в~пошаговых действиях для достижения целей 
различного \mbox{уровня}. 
  
  Таким образом, при построении данного исследования будем основываться 
на следующем определении: <<Под сценарием предлагается понимать 
вербализованное представление знаний об участниках, ключевых признаках, 
условиях, целях, способах и~этапах их достижения как компонентах типичной 
ситуации взаимодействия субъекта с~реальностью>>~\cite[с.~219]{9-sm}. При 
этом план выступает индивидуализированной формой реализации 
деятельности, ключевым моментом которой является соответствие достигаемой 
цели.
  
  \subsection{Лингвистическое моделирование сценариев и~возможности 
извлечения сценариев из~текстов}
  
   Современные методы анализа и~генерации текс\-та, ис\-поль\-зу\-емые, к~примеру, 
в~чат-бо\-тах, системах поиска и~сравнения текс\-тов, достаточно плохо 
учитывают глобальную структуру текс\-та. Один из распространенных способов 
заложить в~модель информацию о~глобальном содержании документа~--- 
описание тематики документа с~по\-мощью метода <<мешок слов>> или 
тематическое моделирование. Но при таком подходе теряется информация 
о~структуре изложения, о~взаимосвязях упо\-ми\-на\-емых в~текс\-те концептов друг 
с~другом. Методы извлечения сю\-жет\-но-ком\-по\-зи\-ци\-он\-но\-го стро\-ения текс\-та, 
к~которым относится и~сценарный анализ, позволяют преодолеть эту проб\-ле\-му 
и~существенно повысить качество решения многих задач, вклю\-чая синтез 
текс\-та и~рассуждения по текс\-ту~\cite{10-sm}. Сценарный анализ~--- это один из 
этапов анализа структуры текс\-та, к~результатам которого можно применять 
правила или другие методы решения конечной задачи.
  
  В настоящей работе предпринята попытка использования сценарного 
подхода к~анализу текс\-тов инструкций. Такие текс\-ты содержат, как правило, 
прямые наименования основных действий, со\-сто\-яний, признаков ситуации, 
в~которой ин\-струк\-ти\-ру\-емый (коллективный адресат) мыс\-лит\-ся автором текста 
как исполнитель определенной роли~\cite{11-sm}. Инструкции 
характеризуются чет\-ки\-ми и~недвусмысленными формулировками, наличием 
эксплицитной мо\-ти\-ви\-ру\-ющей со\-став\-ля\-ющей, уси\-ли\-ва\-ющей побудительную 
мо\-даль\-ность текс\-та, что важ\-но с~точ\-ки зрения минимизации поведенческой 
ва\-ри\-а\-тив\-ности~\cite{9-sm, 11-sm}.
   
   Понимание сценарного подхода авторами \mbox{статьи} близко по смыслу 
к~скриптам Шенка~--- одному из глав\-ных формализмов, опи\-сы\-ва\-ющих 
типичную последовательность событий в~мире. Однако создание скриптов~--- 
трудоемкий процесс, тре\-бу\-ющий большого руч\-но\-го труда для описания каж\-дой 
отдельной предметной об\-ласти. Подход, предложенный Чем\-бер\-сом  
и~Журафски~\cite{12-sm}, не требует никакой разметки текс\-тов, даже указания 
их тематики. Подход получил название нарративных цепочек событий 
(NarrativeEventChains). Речь идет о~час\-тич\-но упорядоченных наборах событий, 
относящихся к~одному дей\-ст\-ву\-юще\-му лицу. 
   
Пример нарративной цепочки:

\_\textit{обвинил}~$X$

$X$ \textit{утверждал, что}

$X$ \textit{заявил, что}

\_\textit{уволил} $X$
    
    В основе подхода нарративных цепочек лежит кореференция. На первом 
шаге с~помощью дистрибутивных методов определяются нарративные 
отношения между событиями (извлекаются взаимосвязанные события и~их 
участники), связанными одними кореферентными аргументами. На втором 
шаге с~по\-мощью временн$\acute{\mbox{о}}$го классификатора эти события 
час\-тич\-но упорядочиваются. На третьем шаге автономные цепочки из 
про\-стран\-ст\-ва событий отсекаются и~группируются.

\section{Синтез плана поведения когнитивным ассистентом}

  Формализация сценария как элемента картины мира представлена 
в~работе~\cite{13-sm}. Алгоритм синтеза плана поведения (АСПП) 
КА опирается на знаковый подход пред\-став\-ле\-ния 
знаний и~является одним из применений алгоритма MAP, получившего 
название на основе обозначений идентификаторов компонент знака ($M$~--- 
компонента значения; $A$~--- компонента смыс\-ла; $P$~--- компонента 
образа)~\cite{14-sm}. Суть применения алгоритма за\-клю\-ча\-ет\-ся в~синтезе плана 
поведения, рег\-ла\-мен\-ти\-ру\-юще\-го це\-ле\-на\-прав\-лен\-ную де\-я\-тель\-ность пользователя, 
в~том случае\linebreak когда деятельность обладает качествами пред\-мет\-ности 
и~си\-ту\-а\-тив\-ности. С~по\-мощью АСПП когнитивный ассистент осуществляет синтез рекомендуемых 
пользователю шагов, пред\-остав\-ля\-ющих \mbox{воз\-мож\-ность} достижения же\-ла\-емой для 
пользователя ситуации из текущего со\-сто\-яния. Результатом работы алгоритма 
становится план поведения, который пред\-став\-лен по\-сле\-до\-ва\-тель\-ностью 
кортежей
  \begin{equation}
  \left\langle \left( s_0, a_0,s_1\right), \left( s_1, a_1, s_2\right),\ldots , \left( s_{n-1}, 
a_{n-1}, s_n\right)\right\rangle
  \label{e1-sm}
  \end{equation}
таких, что $\{f_a^0\} \hm\subseteq \{ f_{s_0}\}$, $\{f_a^{\mathrm{end}}\hm\subseteq 
\{f_{s_n}\}$, где множества $\{f_a^o\}$ и~$\{f_a^{\mathrm{end}}\}$~--- множества фак\-тов 
из мира пользователя, формирующие его пред\-став\-ле\-ние о~начальной 
и~конечной ситуации;  $\{f_{s_0}\}$ и~$\{f_{s_n}\}$~--- множества фак\-тов 
в~картине мира КА, формирующие необходимые для планировщика системы 
условия синтеза плана поведения. В~формуле~(1) $s_0,\ldots , s_n$~--- знаки 
начальной, промежуточных и~целевой ситуаций, а~$a_0,\ldots , a_n$~--- 
каузальные матрицы на сети смыс\-лов знаков действий, спо\-соб\-ст\-ву\-ющих 
активации знаков ситуаций плана. Множества $\{f_{s_0}\}$ и~$\{f_{s_n}\}$ 
являются дополненными множествами $\{f_a^0\}$ и~$f_a^{\mathrm{end}}\}$, недостающие 
факты выбираются из картины мира КА на основе правил формирования 
ситуации и~требуются для создания условий применения планировщика.
  
  Каждое из действий КА является иерархическим и~имеет свой 
операциональный со\-став, способ извлечения которого опирается 
на три основные категории:
  \begin{enumerate}[(1)]
\item в~картине мира КА присутствует полное описание требуемого действия 
и~операциональный состав извлекается из образной компоненты знака действия;
  \item в процессе синтеза сценария помимо основного сценария дея\-тель\-ности 
может быть получен сценарий выполнения требуемого действия, не 
входившего ранее в~картину мира КА. В~этом случае процесс уточ\-не\-ния 
требует вызова планировщика, которому передается сценарий выполнения для 
сокращения необходимого КА времени на синтез плана с~по\-мощью АСПП;
  \item отсутствие полного описания действия в~картине мира КА требует 
вызова АСПП для удовле\-тво\-ре\-ния требований по детализации плана. Уровень 
детализации формируется по требованию пользователя или на основе работы 
алгоритма рас\-суж\-де\-ний.
  \end{enumerate}
  
  \begin{figure*} %fig1
  \vspace*{1pt}
  \begin{center}  
    \mbox{%
\epsfxsize=121.467mm
\epsfbox{smi-1.eps}
}

\end{center}
\vspace*{-2pt}

  \Caption{Синтез плана поведения на основе сценария деятельности КА}
  \vspace*{-3pt}
  \end{figure*}
  
  Для вышеупомянутых категорий характерен процесс уточнения имеющегося 
операционального со\-ста\-ва деятельности знаниями, полученными на основе 
анализа $\{f_a^0\}$ в~рамках процедуры пополнения знаний КА. В~общем виде 
план поведения на основе сценария де\-ятель\-ности пред\-став\-лен на рис.~1. Серым 
цветом отмечены кортежи ситуаций и~действий, выбранные КА в~процессе 
АСПП из множества кортежей в~рам\-ках сценария. Для каждой возможной 
по\-сле\-до\-ва\-тель\-ности действий сценария строится план. Из множества всех 
по\-стро\-ен\-ных планов выбирается наиболее подходящий для $\{ f_a^0\}$ 
и~$\{f_a^{\mathrm{end}}\}$.
  
  Алгоритм синтеза плана поведения выполнен на основе процедуры MAP\_ITERATION, со\-сто\-ящей из 
четырех основных этапов:
  \begin{enumerate}[(1)]
\item S-этап~--- поиск прецедента деятельности из текущей ситуации, 
который ранее способствовал достижению схожей целевой ситуации. Поиск 
осуществляется в~базе предактивированных прецедентных действий;
\item M-этап~--- поиск применимых действий на множестве воз\-мож\-ных 
значений. Поиск проходит на основе получения множества действий над 
объектами, присутствующими в~ситуации, и~выборе среди них применимых. 
При синтезе плана на основе име\-юще\-го\-ся сценария деятельности этот этап 
отвечает за выбор схемы действий, со\-от\-вет\-ст\-ву\-ющей рас\-смат\-ри\-ва\-емой 
ситуации;
\item A-этап~--- генерация мат\-риц действий на множестве смыс\-лов, 
со\-от\-вет\-ст\-ву\-ющих найденным значениям. Все сгенерированные действия 
эвристически оцениваются и~отбираются те, которые способствуют 
наиболее быст\-рой активации целевой ситуации;
\item P-этап~--- построение новой ситуации по множеству признаков 
условий найденных действий.
\end{enumerate}

\section{Пример модельного сценария и~плана покупки 
автомобиля}

\vspace*{-3pt}

  Рассмотрим модельный пример сценария покупки автомобиля. 
Автомобильная тематика была выбрана по причине наличия \mbox{достаточно} 
однозначного набора действий, обес\-пе\-чи\-ва\-ющих \mbox{достижение} же\-ла\-емой цели. 
Модельный сценарий реконструирован на основе текс\-тов людей, опи\-сы\-ва\-ющих 
свой опыт покупки автомобиля. В~качестве исходного набора данных был 
собран корпус из 100~текс\-тов (159\,698~словоупотреблений). В~корпус во\-шли 
инструкции о~покупке, осмотре и~оформ\-ле\-нии автомобиля на свое имя (как 
нового, так и~подержанного), т.\,е.\ текс\-ты, содержащие сценарии в~более-менее 
явном виде (рекомендации авторов читателям). 
  
  Фрагмент схемы сценария покупки автомобиля пред\-став\-лен на рис.~2. 
Данная схема применима и~для других пред\-мет\-ных областей. Она может быть 
использована на практике для задачи управ\-ле\-ния поведением 
интеллектуального агента. Центральное мес\-то в~этой структуре занимает 
сценарий. Он со\-сто\-ит из сле\-ду\-ющих элементов:
  \begin{itemize}
\item одиночное действие (на рис.~2, например, это элемент <<НоваяИлиБУ: 
ПростоеДействие>>)\\[-13pt]
\item ветвление, которое состоит из множества шагов, порядок которых не 
важен, может выполняться не полностью или вза\-и\-мо\-ис\-клю\-ча\-юще (на рис.~2 
это элемент <<ОпределениеТребований: Ветв\-ле\-ние>>)\\[-13pt]
\item цепочка шагов, т.\,е.\ прос\-тая последовательность, которая говорит, что 
и~в~каком порядке надо сделать (на рис.~2 это элемент <<ПокупкаАвто: 
ЦепочкаШагов>>).
\end{itemize}
  
  Каждый шаг в~сценарии является, по сути, отдельным вложенным сценарием.
  
  \pagebreak
  
  \end{multicols}

\begin{figure*} %fig2
  \vspace*{1pt}
  \begin{center}  
    \mbox{%
\epsfxsize=163mm
\epsfbox{smi-2.eps}
}

\end{center}
\vspace*{-4pt}

\Caption{Фрагмент схемы сценария покупки автомобиля}
\vspace*{-4pt}
\end{figure*}
  
  \begin{multicols}{2}
  
  У каждого шага в~сценарии есть также цели и~предусловия. Конечная цель~--- 
это описание ситуации, к~которой желательно прийти в~результате выполнения 
сценария. Ситуацию можно описать как набор фактов о~мире из рабочей памяти 
агента, т.\,е.\ каким мир должен стать в~результате выполнения этого шага. 
Предусловия~--- это некий ограничитель, который не позволяет нарушать 
последовательность шагов в~сценарии и~определяет тот набор фактов, при 
удовлетворении которых можно приступать к~выполнению каждого 
сле\-ду\-юще\-го шага: какая информация должна быть получена, какие решения 
приняты, какие ресурсы необходимы и~т.\,д. У~каждого шага также есть 
субъект, который его выполняет, и~операнды (автомобиль, лакокрасочное 
покрытие, водительские права и~др.). 
  
  Стрелками на рис.~2 обозначены связи (или отношения) между элементами 
сценария. Связь <<СледШаг>> задает следующий сценарий, который 
необходимо выполнить после текущего. Связь <<Варианты>> задает 
множество других сценариев, которые нужно выполнить после текущего, 
причем порядок выполнения этих сценариев не важен (и~не все они могут быть 
выполнены). Связь <<Шаги>> задает множество сценариев, которые должны 
быть выполнены строго последовательно. 

Все эти связи служат для явного 
указания последовательности действий. Однако на практике возможны 
ситуации, когда порядок действий неизвестен (к~примеру, если сценарий 
собирался КА на основе разных текс\-тов, в~которых 
содержалась неполная информация), но ка\-кие-то элементы сценария все равно 
присутствуют, например есть понимание цели. Эти факты ссылаются (связь 
<<Упоминание>>) на операнды (например, автомобиль, тип трансмиссии 
и~др.), другие цели (зачем нужен автомобиль, кого на нем возить и~куда ездить) 
и~т.\,д. Переходя по таким связям, можно находить другие сценарии 
и~действия, которые нужно выполнить для удовле\-тво\-ре\-ния текущей цели. Это 
позволяет, двигаясь с~других сторон, восстановить по\-сле\-до\-ва\-тель\-ность шагов 
сценария, выполняемого впервые.

\section{Заключение}

  Моделирование целенаправленного поведения и~сценариев де\-я\-тель\-ности~--- 
одно из новых на\-прав\-ле\-ний исследований в~об\-ласти искусственного 
интеллекта. Задача по\-стро\-ения сценариев по текстам задействует верхние 
уровни языка (дискурс, сюжет) и~становится новой актуальной проблемой для 
обработки русского языка. Использование извле\-ка\-емых из текстов сценариев 
де\-я\-тель\-ности повышает степень ра\-зум\-ности и~ав\-то\-ном\-ности интеллектуальных 
систем.
  
  Дальнейшим развитием КА будет проработка 
механизмов взаимодействия ас\-сис\-тен\-та с~пользователем на естественном языке. 
Предполагается, что ас\-сис\-тент будет предлагать пользователю дальнейшие 
шаги по решению задачи в~виде рекомендаций, генерируемых автоматически 
на основе текущего со\-сто\-яния и~предыдущих шагов в~синтезированном 
персональном плане ас\-сис\-ти\-ро\-ва\-ния. Текущее состояние (за\-вер\-шен\-ность 
текущего шага плана) также будет определяться ассистентом на основе диалога с~пользователем.
  
{\small\frenchspacing
 {%\baselineskip=10.8pt
 %\addcontentsline{toc}{section}{References}
 \begin{thebibliography}{99}
  \bibitem{1-sm}
  \Au{Смирнов И.\,В., Панов~А.\,И., Скрынник~А.\,А., Чистова~Е.\,В.} Персональный 
когнитивный ассистент: концепция и~принципы работы~// Информатика и~её применения, 
2019. Т.~13. Вып.~3. С.~105--113.
  \bibitem{2-sm}
  \Au{Осипов Г.\,С., Панов~А.\,И.} Отношения и~операции в~знаковой картине мира 
субъекта поведения~// Искусственный интеллект и~принятие решений, 2017. №\,4. С.~5--22.
  \bibitem{3-sm}
  \Au{Осипов Г.\,С., Панов~А.\,И.} Синтез рационального поведения когнитивного 
семиотического агента в~динамической среде~// Искусственный интеллект и~принятие 
решений, 2020. №\,4. С.~80--97.
  \bibitem{4-sm}
  \Au{Берн Э.} Люди, которые играют в~игры. Психология человеческой судьбы~/ Пер. 
  с~англ. А.~Грузберг.~--- М.: Эксмо, 2008. 576~с.
  (\Au{Berne~E.}  Games people play: The psychology of human relationships.~--- New York, 
NY, USA: Ballantine Books, 1973. 192~p.)
  \bibitem{5-sm}
  \Au{Росс Л., Нисбетт~Р.} Человек и~ситуация. Уроки социальной психологии~/ Пер. 
с~англ. В.\,В.~Румынского.~--- М.: Аспект Пресс, 2000. 429~с. (\Au{Ross~L., Nisbett~R.} The 
person and the situation: Perspectives of social psychology.~--- Philadelphia, PA, USA:  
Temple University Press, 1991.)
  \bibitem{6-sm}
  \Au{Леонтьев А.\,Н.} Деятельность. Сознание. Личность.~--- М.: Политиздат, 1975. 304~с.
  \bibitem{7-sm}
  \Au{Асмолов А.\,Г.} Деятельность и~установка.~--- М.: Изд-во Моск. ун-та, 1974. 150~с.
  \bibitem{8-sm}
  \Au{Узнадзе Д.\,Н.} Экспериментальные основы психологии установки.~--- Тбилиси: 
Акад. наук Груз. ССР, 1961. 210~с.
  \bibitem{9-sm}
  \Au{Кузнецова Ю.\,М., Пенкина~М.\,Ю.}Сценарий отказа от решения проблемы как 
предмет сетевых обсуждений~// Экопсихологические исследования, 2020. Т.~6. С.~218--222.
  \bibitem{10-sm}
  \Au{Суворова М.\,И., Кобозева~М.\,В., Толдова~С.\,Ю., Соколова~Е.\,Г.} Извлечение 
сценарной информации из текстов. Ч.~1: Постановка задачи и~обзор методов~// 
Искусственный интеллект и~принятие решений, 2020. №\,1. С.~17--26.
  \bibitem{11-sm}
  \Au{Мишланов В.\,А., Чуганская~А.\,А., Смирнов~И.\,В., Суворова~М.\,И., Курузов~И.\,А.} 
Разработка методов анализа сценариев поведения (на материале инструктивных  
ин\-тер\-нет-текс\-тов)~// Медиалингвистика, 2020. Т.~7. №\,1. С.~16--28.
  \bibitem{12-sm}
  \Au{Chambers N., Jurafsky~D.} Unsupervised learning of narrative event chains~// 46th Annual Meeting 
  of the Association for Computational Linguistics Proceedings.~---
  Columbus, OH, USA: Association for Computational Linguistics, 2008.
   P.~789--797.
  \bibitem{13-sm}
  \Au{Панов А.\,И.} Формирование образной компоненты знаний когнитивного агента со 
знаковой картиной мира~// Информационные технологии и~вычислительные системы, 2018. 
№\,4. С.~84--96.
  \bibitem{14-sm}
  \Au{Киселев Г.\,А.} Интеллектуальная система планирования поведения коалиции 
робототехнических агентов с~STRL архитектурой~// Информационные технологии 
и~вычислительные системы, 2020. №\,2. С.~21--37.
\end{thebibliography}

 }
 }

\end{multicols}

\vspace*{-6pt}

\hfill{\small\textit{Поступила в~редакцию 05.02.21}}

%\vspace*{8pt}

%\pagebreak

\newpage

\vspace*{-28pt}

%\hrule

%\vspace*{2pt}

%\hrule

%\vspace*{-2pt}

\def\tit{PERSONAL COGNITIVE ASSISTANT:\\ PLANNING ACTIVITY WITH~SCRIPTS}


\def\titkol{Personal cognitive assistant: Planning activity with~scripts}


\def\aut{I.\,V.~Smirnov$^{1,2}$, A.\,I.~Panov$^{1,3}$, A.\,A.~Chuganskaya$^{1}$, M.\,I.~Suvorova$^1$, 
G.\,A.~Kiselev$^{1,2}$, I.\,A.~Kuruzov$^3$, and~O.\,G.~Grigoriev$^1$}

\def\autkol{I.\,V.~Smirnov, A.\,I.~Panov, A.\,A.~Chuganskaya, et al.}
%M.\,I.~Suvorova$^1$,  G.\,A.~Kiselev$^{1,2}$, I.\,A.~Kuruzov$^3$, and~O.\,G.~Grigoriev}

\titel{\tit}{\aut}{\autkol}{\titkol}

\vspace*{-11pt}


 \noindent
    $^1$Federal Research Center ``Computer Science and Control'' of the Russian Academy of 
Sciences, 44-2~Vavilov\linebreak
$\hphantom{^1}$Str., Moscow 119333, Russian Federation
    
    
    \noindent
    $^2$Peoples' Friendship University of Russia (RUDN University), 6~Miklukho-Maklaya Str., 
Moscow 117198, Russian\linebreak
$\hphantom{^1}$Federation
  
  
    \noindent
    $^3$Moscow Institute of Physics and Technology (National Research University), 9~Institutskiy 
Per., Dolgoprudny,\linebreak
$\hphantom{^1}$Moscow Region 141701, Russian Federation

\def\leftfootline{\small{\textbf{\thepage}
\hfill INFORMATIKA I EE PRIMENENIYA~--- INFORMATICS AND
APPLICATIONS\ \ \ 2022\ \ \ volume~16\ \ \ issue\ 1}
}%
 \def\rightfootline{\small{INFORMATIKA I EE PRIMENENIYA~---
INFORMATICS AND APPLICATIONS\ \ \ 2022\ \ \ volume~16\ \ \ issue\ 1
\hfill \textbf{\thepage}}}

\vspace*{3pt} 
      
   
  
  
  \Abste{The paper presents procedures for a~cognitive assistant's behavior planning based on 
scripts~--- generalized schemes of tasks solving. A~cognitive assistant is a~virtual intelligent agent 
that has its own worldview and builds a~worldview of the user, it helps to solve various common or 
specific problems. The key component of assistant's goal-based behavior is scenario~--- a~reusable 
abstract sequence of actions and situations that are used for synthesis of a~concrete plan of actions for 
a user. The concept of a~scenario in psychological and linguistic interpretation  as well 
as the procedure of scenarios extraction from texts are considered. The scenario and the plan of behavior are 
formalized using the sign-based approach. Methods for synthesizing a~plan of behavior are 
proposed. A~test case of a behavior plan synthesis for buying a car is considered.}
  
  \KWE{cognitive assistant; activity script; behavior planning}
  
  
  
\DOI{10.14357/19922264220107}

\vspace*{-16pt}

\Ack
  \noindent
  The reported study was partially funded by the Russian Foundation for Basic Research (project 
No.\,18-29-22027).




%\vspace*{6pt}

  \begin{multicols}{2}

\renewcommand{\bibname}{\protect\rmfamily References}
%\renewcommand{\bibname}{\large\protect\rm References}

{\small\frenchspacing
 {%\baselineskip=10.8pt
 \addcontentsline{toc}{section}{References}
 \begin{thebibliography}{99}
  
  \bibitem{1-sm-1}
  \Aue{Smirnov, I.\,V., A.\,I.~Panov, A.\,A.~Skrynnik, and E.\,V.~Chistova.} 2019. Personal'nyy 
kognitivnyy assistent: kon\-tsep\-tsiya i~printsipy raboty [Personal cognitive assistant:\linebreak Concept and 
key principals]. \textit{Informatika i~ee Pri\-me\-ne\-niya~--- Inform. Appl.} 13(3):105--113.
  \bibitem{2-sm-1}
  \Aue{Osipov, G.\,S., and A.\,I.~Panov.} 2018. Relationships and operations in agent's  
sign-based model of the world. \textit{Scientific Technical Information Processing} 45(5):1--14.
  \bibitem{3-sm-1}
  \Aue{Osipov, G.\,S., and A.\,I.~Panov.} 2021. Planning rational behavior of 
cognitive semiotic agents in a~dynamic environment. \textit{Scientific Technical Information 
Processing} 48(6):502--516.
  \bibitem{4-sm-1}
  \Aue{Berne, E.} 1973. \textit{Games people play: The psychology of human relationships}. New 
York, NY: Ballantine Books. 192~p.
  \bibitem{5-sm-1}
  \Aue{Ross, L., and R.~Nisbett.} 2011. \textit{The person and the situation: Perspectives of 
social psychology}. London: McGraw-Hill. 288~p.
  \bibitem{6-sm-1}
  \Aue{Leont'ev, A.\,N.} 1975. \textit{Deyatel'nost'. Soznanie. Lichnost'} [Activity, 
consciousness, and personality]. Moscow: Po\-lit\-iz\-dat. 304~p.
  \bibitem{7-sm-1}
  \Aue{Asmolov, A.\,G.} 1974. \textit{Deyatel'nost' i~ustanovka} [Activity and attitude]. 
Moscow: Izd-vo Moskovskogo un-ta. 150~p.
  \bibitem{8-sm-1}
  \Aue{Uznadze, D.\,N.} 1961. \textit{Eksperimental'nye osnovy psikhologii ustanovki} 
[Experimental foundations of installation theory]. Tbilisi: AN GSSR. 210~p.
  \bibitem{9-sm-1}
  \Aue{Kuznetsova, Yu.\,M., and M.\,Yu.~Penkina.} 2020. Stsenariy otkaza ot resheniya 
problemy kak predmet setevykh obsuzhdeniy [Scenario of failure to solve a~problem in 
network discussions]. \textit{Ekopsikhologicheskie issledovaniya} 
[Ecopsychological Research] 6:218--222.
  \bibitem{10-sm-1}
  \Aue{Suvorova, M.\,I., M.\,V.~Kobozeva, S.\,Yu.~Toldova, and E.\,G.~Sokolova.} 2021 (in 
press). Extraction of script information from texts. Part~1: Statement of the problem and review of 
methods]. \textit{Scientific Technical Information Processing} 48.
  \bibitem{11-sm-1}
  \Aue{Mishlanov, V.\,A., A.\,A.~Chuganskaya, I.\,V.~Smirnov, M.\,I.~Suvorova, and 
I.\,A.~Kuruzov.} 2020. Razrabotka metodov analiza stsenariev povedeniya (na materiale 
instruktivnykh internet-tekstov) [Developing methods for behavior scenario analysis (on 
the material of instructional texts)]. \textit{Medialingvistika} [Media Linguistics] 7(1):16--28.
  \bibitem{12-sm-1}
  \Aue{Chambers, N., and D.~Jurafsky.} 2008. Unsupervised learning of narrative event chains. 
\textit{46th Annual Meeting of the Association for Computational Linguistics Proceedings}. 
Columbus, OH: Association for Computational Linguistics. 789--797.
  \bibitem{13-sm-1}
  \Aue{Panov, A.\,I.} 2018. Formirovanie obraznoy komponenty znaniy kognitivnogo agenta so 
znakovoy kartinoy mira [Formation of an image component of knowledge of the cognitive agent 
with a~sign-based model of worldview]. \textit{Informatsionnye tekhnologii i~vychislitel'nye 
sistemy} [J.~Information Technologies Computing Systems] 4:84--96.
  \bibitem{14-sm-1}
  \Aue{Kiselev, G.\,A.} 2020. Intellektual'naya sistema pla\-ni\-ro\-va\-niya povedeniya koalitsii 
robototekhnicheskikh agen\-tov s~STRL arkhitekturoy [Intelligent behavior planning system for 
a~coalition of robotic agents with STRL architecture]. \textit{Informatsionnye tekhnologii 
i~vychislitel'nye sistemy} [J.~Information Technologies Computing Systems] 2:21--37.
  \end{thebibliography}

 }
 }

\end{multicols}

\vspace*{-6pt}

\hfill{\small\textit{Received February 5, 2021}}

%\pagebreak

%\vspace*{-18pt}

  
  \Contr
  
  \noindent
  \textbf{Smirnov Ivan V.} (b.\ 1978)~--- Candidate of Science (PhD) in physics and 
mathematics; head of department, Institute of Artificial Intelligence Problems, Federal Research 
Center ``Computer Science and Control'' of the Russian Academy of Sciences, 9,~60-letiya 
Oktyabrya Prosp., Moscow 117312, Russian Federation; associate professor, Peoples' Friendship 
University of Russia (RUDN University), 6~Miklukho-Maklaya Str., Moscow 117198, Russian 
Federation; \mbox{ivs@isa.ru}
  
  \vspace*{3pt}
  
  \noindent
  \textbf{Panov Aleksandr I.} (b.\ 1987)~--- Candidate of Science (PhD) in physics and 
mathematics; head of department, Institute of Artificial Intelligence Problems, Federal Research 
Center ``Computer Science and Control'' of the Russian Academy of Sciences, 9,~60-letiya 
Oktyabrya Prosp., Moscow 117312, Russian Federation; deputy head of laboratory, Moscow 
Institute of Physics and Technology (National Research University), 9~Institutskiy Per., 
Dolgoprudny, Moscow Region 141701, Russian Federation; \mbox{pan@isa.ru}
  
  \vspace*{3pt}
  
  
  \noindent
  \textbf{Chuganskaya Anfisa A.} (b.\ 1985)~--- Candidate of Science (PhD) in psychology, 
researcher, Institute of Artificial Intelligence Problems, Federal Research Center 
``Computer 
Science and Control'' of the Russian Academy of Sciences, 9,~60-letiya Oktyabrya Prosp., Moscow 
117312, Russian Federation; \mbox{anfisa.makh@gmail.com}
  
  \vspace*{3pt}
  
  
  \noindent
  \textbf{Suvorova Margarita I.} (b.\ 1991)~--- researcher, Institute of Artificial Intelligence 
Problems, Federal Research Center ``Computer Science and Control'' of the Russian Academy of 
Sciences, 9,~60-letiya Oktyabrya Prosp., Moscow 117312, Russian Federation; 
\mbox{suvorova@isa.ru}
  
  \vspace*{3pt}
  
  
  \noindent
  \textbf{Kiselev Gleb A.} (b.\ 1992)~--- researcher, Institute of Artificial Intelligence Problems, 
Federal Research Center ``Computer Science and Control'' of the Russian Academy of Sciences, 
9,~60-letiya Oktyabrya Prosp., Moscow 117312, Russian Federation; assistant, Peoples' Friendship 
University of Russia (RUDN University), 6~Miklukho-Maklaya Str., Moscow117198, Russian 
Federation; \mbox{kiselev@isa.ru}
  
  \vspace*{3pt}
  
  \noindent
  \textbf{Kuruzov Ilya A.} (b.\ 1999)~--- PhD student, Moscow Institute of Physics and 
Technology (National Research University), 9~Institutskiy Per., Dolgoprudny, Moscow Region 
141701, Russian Federation; \mbox{kuruzov2014@mail.ru}
  
  \vspace*{3pt}
  
  \noindent
  \textbf{Grigoriev Oleg G.} (b.\ 1957)~--- Doctor of Science (PhD) in technology, head of 
Institute of Artificial Intelligence Research Problems, Federal Research Center ``Computer Science and Control'' 
of the Russian Academy of Sciences, 9,~60-letiya Oktyabrya Prosp., Moscow 117312, Russian 
Federation; \mbox{oleggpolikvart@yandex.ru}
  


\label{end\stat}

\renewcommand{\bibname}{\protect\rm Литература} 
    %7
\def\stat{listopad}

\def\tit{ЖИЗНЕННЫЙ ЦИКЛ МЕТОДОЛОГИИ ПОСТРОЕНИЯ РЕФЛЕКСИВНО-АКТИВНЫХ 
СИСТЕМ ИСКУССТВЕННЫХ ГЕТЕРОГЕННЫХ ИНТЕЛЛЕКТУАЛЬНЫХ АГЕНТОВ$^*$}

\def\titkol{Жизненный цикл методологии построения РАСИГИА} %рефлексивно-активных систем искусственных гетерогенных интеллектуальных агентов}

\def\aut{С.\,В.~Листопад$^1$}

\def\autkol{С.\,В.~Листопад}

\titel{\tit}{\aut}{\autkol}{\titkol}

\index{Листопад С.\,В.}
\index{Listopad S.\,V.}


{\renewcommand{\thefootnote}{\fnsymbol{footnote}} \footnotetext[1]
{Исследование выполнено за счет гранта Российского научного фонда №\,23-21-00218, 
{\sf https://rscf.ru/project/23-21-00218/}.}}


\renewcommand{\thefootnote}{\arabic{footnote}}
\footnotetext[1]{Федеральный исследовательский центр <<Информатика и~управ\-ле\-ние>> Российской академии наук, 
\mbox{ser-list-post@yandex.ru}}

%\vspace*{-12pt}

  
  

  \Abst{Представлена темпоральная структура (жизненный цикл) методологии построения 
рефлексивно-активных систем искусственных гетерогенных интеллектуальных агентов (\mbox{РАСИГИА}). 
Такие системы предназначены для компьютерного моделирования процессов и~эффектов, 
возникающих при решении практических проблем коллективами специалистов под 
руководством лица, принимающего решения. Искусственные гетерогенные 
интеллектуальные агенты реф\-лек\-сив\-но-ак\-тив\-ных сис\-тем~--- активные субъекты, 
способные к~рассуждениям, коммуникации и~рефлексии как умению моделировать 
рассуждения других агентов системы и~себя самих. Моделирование рефлексивных процессов 
обеспечивает выработку агентами согласованного представления об объекте управ\-ле\-ния, 
цели коллективной работы и~нормах взаимодействия, позволяя системе в~ходе 
самоорганизации генерировать заново релевантный гибридный интеллектуальный метод 
решения очередной проб\-лемы.} 
  
  \KW{рефлексия; методология; рефлексивно-активная сис\-те\-ма искусственных 
гетерогенных интеллектуальных агентов; гибридная интеллектуальная многоагентная 
система; коллектив специалистов}

\DOI{10.14357/19922264240112}{GUAMVE}
  
%\vspace*{-6pt}


\vskip 10pt plus 9pt minus 6pt

\thispagestyle{headings}

\begin{multicols}{2}

\label{st\stat}

\section{Введение}

  Компьютерное моделирование процессов и~эффектов, возникающих при 
решении практических проблем коллективами специалистов, каждый из 
которых обладает собственным опытом, знаниями и~пониманием предметной 
области,~--- перспективное на\-прав\-ле\-ние научных разработок, которое 
Д.\,А.~Поспелов выделял как одну из десяти горячих точек в~исследованиях по 
искусственному\linebreak интеллекту~[1]. Для компьютерного моделирования 
рас\-суж\-де\-ний коллективов специалистов предлагается создание \mbox{РАСИГИА} 
в~рамках многоагентного подхода~[2] на основе модели 
\mbox{ги\-брид\-ных} интеллектуальных многоагентных сис\-тем~[3]. Агенты 
\mbox{РАСИГИА}~--- активные программные сущности, способные 
рас\-суж\-дать, взаимодействовать и~рефлексировать. Рефлексивное 
моделирование агентами друг друга обеспечивает выработку согласованного 
пред\-став\-ле\-ния об объекте управ\-ле\-ния, \mbox{цели} коллективной работы и~нормах 
взаимодействия, а~также эволюцию \mbox{РАСИГИА} в~ходе 
самоорганизации в~сильном смыс\-ле. В~на\-сто\-ящей работе рас\-смат\-ри\-ва\-ют\-ся 
вопросы создания методологии разработки сис\-тем такого класса, которая 
понимается как учение об организации продуктивной де\-я\-тель\-ности 
в~це\-лост\-ную сис\-те\-му с~чет\-ко определенными характеристиками, логической 
структурой и~процессом ее осуществления (темпоральной структурой)~[4]. 
Характеристики (особенности и~принципы) и~логическая структура (субъект, 
объект, предмет, методы, средства, результат) методологии разработки 
\mbox{РАСИГИА} рас\-смот\-ре\-ны в~[5]. Данная работа по\-свя\-ще\-на разработке 
жизненного цик\-ла (темпоральной структуры) предлагаемой методологии.

\begin{figure*} %fig1
\vspace*{1pt}
      \begin{center}
     \mbox{%
\epsfxsize=148.855mm 
\epsfbox{lis-1.eps}
}
\end{center}
%\vspace*{-9pt}

{\small Темпоральная структура методологии построения РАСИГИА: \textit{1}~--- этап методологии; \textit{2}~--- стадия методологии;
\textit{3}~--- граница фазы методологии; \textit{4}~--- отношение следования при нормальном завершении этапа;  
\textit{5}~--- возврат к~предыдущим этапам при выявлении допущенных на них недочетов}
\end{figure*}

\vspace*{-6pt}
  
\section{Темпоральная структура методологии}

\vspace*{-6pt}

  Укрупненно в~жизненном цикле методологии построения \mbox{РАСИГИА}, 
показанном на рисунке, могут быть выделены проектная, технологическая 
и~рефлексивная фазы, которые со\-сто\-ят из стадий и~этапов. Как видно, 
последовательное выполнение этапов методологии приводит к~же\-ла\-емо\-му 
результату лишь в~идеальном случае, когда проектировщик сразу получает всю 
необходимую достоверную информацию, имеет необходимый арсенал методов, 
не совершает ошибок ни на одном из этапов и,~по сути, заранее знает, какой 
должна быть разрабатываемая \mbox{РАСИГИА}. В~реальности на каждом из 
этапов могут обнаруживаться ранее допущенные недочеты, требующие 
возврата к~соответствующему этапу, их исправления и~повторного выполнения 
проделанной работы с~новыми исходными данными. В~определенном смыс\-ле 
такой подход представляет собой метод проб и~ошибок, и~чем слож\-нее 
проблема, для которой проектируется \mbox{РАСИГИА}, с~точ\-ки зрения 
конкретного коллектива разработчиков, тем больше будет возвратов в~ходе 
проектирования системы~[6]. Рас\-смот\-рим по\-дроб\-нее каждую из фаз 
методологии.



\section{Проектная фаза}

  Проектная фаза включает в~себя стадии концептуального описания проб\-ле\-мы и~моделирования, выполняемые сис\-тем\-ны\-ми аналитиками из коллектива 
разработчиков. В~рамках первой стадии фазы на доформальном, 
содержательном уровне рас\-смат\-ри\-ва\-ет\-ся проб\-ле\-ма как отрицательное 
отношение субъекта к~реальности~[6] и~проблемная ситуация как объективное 
стечение обстоятельств, обуслов\-ли\-ва\-ющее проб\-ле\-му. Данная стадия со\-сто\-ит из 
сле\-ду\-ющих этапов:
  \begin{itemize}
\item формулирование проб\-ле\-мы, ее предварительное описание в~ходе 
интервьюирования лица, при\-ни\-ма\-юще\-го решение, его советников и~активных 
групп на естественном языке с~использованием привычных для них 
определений и~формулировок~[7];
  \item определение проб\-ле\-ма\-ти\-ки, т.\,е.\ комплекса проб\-лем, связанных 
с~рас\-смат\-ри\-ва\-емой~[4], чтобы учесть создаваемые ее решением последствия 
для каж\-дой из них. Необходимо охватить весь круг участников проб\-лем\-ной 
ситуации (стейкхолдеров, заинтересованных лиц): непосредственных 
участников ситуации, пред\-ста\-ви\-те\-лей проб\-ле\-мо\-раз\-ре\-ша\-ющих 
и~проб\-ле\-мо\-со\-дер\-жа\-щих сис\-тем, же\-ла\-емых помощников или союзников, 
субъектов, связанных с~ситуацией юридически, лиц с~возможным негативным 
отношением к~решению проб\-ле\-мы~[6]. Для по\-стро\-ения проб\-ле\-ма\-ти\-ки может 
быть использована, например, технология Дж.~Уор\-фил\-да, подходы 
с~использованием метафор организации, взгляда на проблему стейкхолдером 
с~раз\-ных точек зрения, рас\-смот\-ре\-ния проб\-ле\-мы в~рамках различных парадигм 
(функциональной, объяснительной, освободительной, пост\-мо\-дер\-нист\-ской)~[4, 6]. 
Формируется древовидная или сетевая структура в~виде диаграммы связей, 
концептуальной кар\-ты или аналогичных инструментов;
  \item определение целей проектирования \mbox{РАСИГИА}, 
пред\-по\-ла\-га\-ющее проведение собеседований с~каж\-дым стейк\-хол\-де\-ром, 
выяснение их целей и~пожеланий, формирование и~структурирование 
множества целей в~виде дерева или сетевидной структуры и~его 
визуализация~[4, 6]. Выделяются следующие уровни целей: ожи\-да\-емые 
в~плановом периоде результаты; задачи, которые не будут решены 
в~рас\-смат\-ри\-ва\-емом периоде, но будет достигнут существенный прогресс на 
пути к~ним; не\-до\-сти\-жи\-мые идеалы, к~которым следует стремиться~[8];
  \item выбор критериев, т.\,е.\ до\-ступ\-ных для наблюдения и~измерения 
характеристик, опи\-сы\-ва\-ющих важ\-ные особенности объектов или процессов 
и~поз\-во\-ля\-ющих сравнивать \mbox{пред\-ла\-га\-емые} альтернативы, контролировать 
процесс решения~[6]. Со\-во\-куп\-ность критериев долж\-на быть релевантной 
количественной моделью выделенных ранее качественных целей. Отдельно 
выделяются ограничения, фик\-си\-ру\-ющие условия, которые не могут нарушаться 
при до\-сти\-же\-нии цели;
  \item оценка концептуального описания проб\-ле\-мы в~ходе специально 
спланированного эксперимента. Если существует коллектив специалистов, 
решающий на практике по\-став\-ле\-нную или схожие проб\-ле\-мы, он выступает 
образцом, прототипом создаваемой сис\-те\-мы агентов. В~этом случае 
выполняется наблюдение за работой такого коллектива в~рамках решения 
реальных или тренировочных проб\-лем и~оценка релевантности 
зафиксированной информации сведениям, полученным в~ходе предыду\-щих 
этапов. Если выявлено существенное рас\-хож\-де\-ние, выполняется возврат 
к~этапу, в~рамках которого были получены некорректные сведения. Сведения 
о~составе участников коллектива, вы\-де\-ля\-емых ими под\-проб\-ле\-мах, методах их 
решения используются на по\-сле\-ду\-ющих этапах проектирования 
\mbox{РАСИГИА} при по\-стро\-ении со\-от\-вет\-ст\-ву\-ющих моделей проб\-ле\-мы 
и~сис\-те\-мы <<как есть сейчас>>. Данные о~качестве принятых решений 
и~дли\-тель\-ности их выработки используются в~дальнейшем как показатель 
эффекта от разработки и~внед\-ре\-ния \mbox{РАСИГИА}. Если подобных 
коллективов нет или не\-воз\-мож\-но реализовать со\-от\-вет\-ст\-ву\-ющий эксперимент, 
данный этап отсутствует.
  \end{itemize}
  
  Стадия моделирования предполагает разработку формализованного описания 
проб\-ле\-мы, коллектива специалистов, ре\-ша\-юще\-го проб\-ле\-му на момент 
разработки \mbox{РАСИГИА}, если он существует, и~самой 
\mbox{РАСИГИА}. Модели строятся с~использованием визуального 
метаязыка~[9], что позволяет наглядно их изобразить, а~так\-же поз\-во\-ля\-ет 
с~использованием заранее заданных соответствий однозначно отоб\-ра\-зить 
графическое пред\-став\-ле\-ние моделей в~формальное символьное пред\-став\-ле\-ние, 
пригодное для компьютерной интерпретации. Данная стадия со\-сто\-ит из 
сле\-ду\-ющих этапов:
  \begin{itemize}
  \item моделирование проб\-ле\-мы, которое обеспечивает ее формальное 
пред\-став\-ле\-ние на макро- и~мик\-ро\-уров\-не. Мак\-ро\-уров\-не\-вая модель описывает 
проб\-ле\-му как <<чер\-ный ящик>>, отражая ее место в~ме\-та\-проб\-ле\-ме (проб\-ле\-ме 
более высокого уровня), свойства как целого и~связи с~другими проб\-ле\-ма\-ми 
ме\-та\-проб\-ле\-мы. Атрибуты проблемы на макроуровне~--- цели, критерии 
(включая ограничения), исходные данные и~идентификатор. Мик\-ро\-уров\-не\-вая 
модель раскрывает со\-став и~структуру проб\-ле\-мы, описывает ее под\-проб\-ле\-мы 
и~связи между ними. Для каждой под\-проб\-ле\-мы специфицируются цели, 
критерии, исходные данные и~идентификатор, выполняется поиск релевантных 
методов решения. Если такие методы найдены, дальнейшая декомпозиция 
под\-проб\-ле\-мы не требуется, иначе выполняется по\-стро\-ение ее мик\-ро\-уров\-не\-вой 
модели, т.\,е.\ модели более глубокого уров\-ня иерархии. Таким образом, 
формируется многоуровневая иерархическая структура по\-став\-лен\-ной 
проб\-лемы;
  \item моделирование коллектива, которое отражает ситуацию решения 
проб\-ле\-мы <<как есть сейчас>> со всеми ее преимуществами и~недостатками. 
Модель коллектива~--- основа, образец для проектирования \mbox{РАСИГИА} и~оценки эф\-фек\-тив\-ности 
альтернативных конфигураций \mbox{РАСИГИА}. 
При моделировании коллектива специалистов фиксируется его со\-став в~виде 
множества ролей участников, час\-ти проб\-ле\-мы, ре\-ша\-емые каж\-дым из 
участников с~определенной ролью, знания и~методы, ис\-поль\-зу\-емые 
участниками для решения своей части проб\-ле\-мы, а~так\-же порядок и~нормы 
взаимодействия участников коллектива; 
  \item моделирование \mbox{РАСИГИА}, фор\-ми\-ру\-ющее идеализированное 
пред\-став\-ле\-ние <<как должно стать>> о~коллективе интеллектуальных агентов, 
ре\-ша\-ющих по\-став\-лен\-ную проб\-ле\-му. В~ходе моделирования \mbox{РАСИГИА} 
должны быть специфицированы со\-став и~иерархия ролей агентов, множество 
агентов, ис\-поль\-зу\-емые протоколы взаимодействия, под\-дер\-жи\-ва\-емые языки 
передачи сообщений, базовая онтология как осно\-ва для интерпретации 
семантики пе\-ре\-да\-ва\-емых сообщений, модель окру\-жа\-ющей среды, содержащая 
в~том чис\-ле пул, из которого агенты могут привлекаться сис\-те\-мой по мере 
не\-об\-хо\-ди\-мости и~в~который попадают ис\-клю\-ча\-емые из нее агенты, множество 
моделей архитектур \mbox{РАСИГИА}, множество необходимых моделей 
мак\-ро\-уров\-не\-вых эффектов. В~множестве агентов должны присутствовать 
агенты, пред\-став\-ля\-ющие стейк\-хол\-де\-ров с~их целями, критериями достижения 
цели и~ограничениями. Если на предыду\-щем этапе была по\-стро\-ена модель 
коллектива, то одна из архитектур \mbox{РАСИГИА} долж\-на соответствовать 
данной модели. 
  \end{itemize}
  
\section{Технологическая фаза}

  Технологическая фаза включает в~себя разработку эскизного проекта 
\mbox{РАСИГИА}, ее технического проекта и~программной реализации. 
Стадия разработки эскизного проекта \mbox{РАСИГИА} обеспечивает 
пред\-став\-ле\-ние создаваемой сис\-те\-мы и~ее внеш\-ней среды в~виде 
взаимосвязанных мо\-ду\-лей-бло\-ков в~соответствии с~моделью 
\mbox{РАСИГИА}, по\-стро\-ен\-ной на стадии проектирования. Данная стадия 
со\-сто\-ит из сле\-ду\-ющих этапов:
  \begin{itemize}
  \item разработка функциональной структуры, в~ходе которой строится 
множество взаимосвязанных схем-диа\-грамм, определяющих под\-сис\-те\-мы 
РАСИГИА, распределение агентов по ним, функционал агентов, до\-пус\-ти\-мые 
языки передачи сообщений и~протоколы взаимодействия для каж\-дой пары или 
группы ролей агентов, технологические элементы сис\-те\-мы, потоки 
информации и~управ\-ле\-ния, а~также отношения, воз\-ни\-ка\-ющие между агентами 
в~процессе решения проб\-лем. Для каждой роли указывается множество 
релевантных ей уже существующих (разработанных ранее для других сис\-тем) 
агентов, если таковые имеются. В~случае отсутствия релевантных агентов они 
должны быть разработаны на сле\-ду\-ющих этапах. Кроме того, 
специфицируются функциональные мо\-ду\-ли-бло\-ки, от\-ве\-ча\-ющие за организацию 
макроуровневых эффектов в~\mbox{РАСИГИА};
  \item разработка структуры внешней среды по аналогии с~разработкой 
функциональной структуры \mbox{РАСИГИА} предполагает построение схем-диа\-грамм, описывающих виртуальную внеш\-нюю среду, ее под\-сис\-те\-мы, роли 
агентов и~способы взаимодействия \mbox{РАСИГИА} с~ними, т.\,е.\ языки 
передачи сообщений и~протоколы взаимодействия, отношения, потоки 
информации и~управ\-ле\-ния. Для каж\-дой роли указываются су\-щест\-ву\-ющие 
релевантные ей агенты, если они имеются;
  \item разработка архитектур агентов выполняется для тех ролей 
в~функциональной структуре и~структуре внеш\-ней среды, для которых не 
найдено релевантных реализованных агентов. Архитектура агента~--- схема, 
описывающая со\-став, структуру и~взаимосвязь функ\-ций-бло\-ков, 
ре\-а\-ли\-зу\-емых агентом, обеспечивающая выполнение им своего предназначения. 
Для каждой функ\-ции-бло\-ка указывается метод или алгоритм, с~по\-мощью 
которого она реализуется, в~случае если таковые отсутствуют, они долж\-ны 
быть разработаны в~рамках сле\-ду\-ющей стадии.
  \end{itemize}
  
  Стадия разработки технического проекта \mbox{РАСИГИА} обеспечивает 
создание недостающих блоков для ее агентов или технологических элементов. 
При этом может по\-тре\-бо\-вать\-ся разработка методов решения под\-проб\-лем, 
алгоритмов на основе метода, баз данных, онтологий и~др. Порядок их 
разработки не регламентируется на\-сто\-ящей методологией в~связи 
с~существенным разнообразием и~не\-воз\-мож\-ностью совместного рас\-смот\-ре\-ния. 
На данной стадии должен быть сформирован технический проект, 
опи\-сы\-ва\-ющий для каждого блока со\-став, структуру и~форму пред\-став\-ле\-ния 
входных и~выходных данных, алгоритм его функционирования, спецификацию 
необходимых технических средств~[10].
  
  Стадия программной реализации и~отладки предполагает разработку 
программного кода \mbox{РАСИГИА} и~его тестирование на предмет 
корректной работы с~\mbox{целью} формирования полноценного программного 
продукта, а~так\-же разработку программной документации. Данная стадия 
со\-сто\-ит из сле\-ду\-ющих этапов:
  \begin{itemize}
  \item программная реализация и~разработка документации выполняется 
с~использованием платформы JaCaMo~[11], объединяющей технологию Jason 
для программирования автономных агентов, Cartago для программирования 
элементов внеш\-ней среды и~Moise для программирования многоагентных 
организаций. Кроме того, применяется язык Java для программирования 
отдельных элементов сис\-те\-мы и~тонкой настройки механизмов 
платформы~[12];
  \item тестирование и~отладка обеспечивают выявление и~устранение 
основных дефектов в~сис\-те\-ме. Ввиду того что полное тестирование  
сколь\-ко-ни\-будь слож\-ной программы не\-воз\-мож\-но~[13], выполняется 
выборочное тестирование в~сле\-ду\-ющем порядке: отдельные функ\-ции и~блоки 
из состава аген\-тов и~технологических элементов, межмодульные связи, агенты 
и~технологические элементы в~целом, протоколы взаимодействия агентов, 
\mbox{РАСИГИА} в~целом. В~тес\-ти\-ро\-ва\-нии принимают участие 
представители всех ролей команды разработчиков, так как каж\-дый из них 
выполняет поиск ошибок разного рода~[14]. При этом выделяется отдельная 
роль тестировщика, опре\-де\-ля\-юще\-го стратегию тес\-ти\-ро\-ва\-ния,  
тест-тре\-бо\-ва\-ния и~тест-пла\-ны для каждой из фаз проекта; он выполняет 
тестирование сис\-те\-мы, собирает и~анализирует отчеты о~про\-хож\-де\-нии 
тестирования. 
\end{itemize}

\section{Рефлексивная фаза}

  Рефлексивная фаза предназначена для оценки показателей реализованной 
\mbox{РАСИГИА} и~процесса ее разработки, выявления ее недостатков и~при 
не\-об\-хо\-ди\-мости до\-ра\-бот\-ки как сис\-те\-мы, так и~методологии ее построения. 
Стадия оценки эф\-фек\-тив\-ности \mbox{РАСИГИА} предполагает сбор 
показателей работы сис\-те\-мы и~их сравнение с~целевыми значениями. Если 
выявляется их несоответствие, выполняется анализ причин отклонений, 
переход к~этапу методологии, вызвавшему их, и~повторное выполнение данного и~по\-сле\-ду\-ющих этапов с~учетом тре\-бу\-емых корректировок. Кроме того, на этой 
стадии продолжается отладка сис\-те\-мы. Данная стадия выполняется в~три этапа:
  \begin{enumerate}[(1)]
  \item оценка в~лабораторных условиях командой разработчиков, когда 
система работает в~виртуальной внешней среде, решая тестовые проб\-ле\-мы. На 
данной стадии оценка сис\-те\-мы выполняется вычислительными моделями 
стейкхолдеров, реализованными со\-от\-вет\-ст\-ву\-ющи\-ми агентами виртуальной 
внеш\-ней среды; 
  \item оценка по результатам тестовой эксплуатации, когда \mbox{РАСИГИА} 
функционирует в~реальной внеш\-ней среде параллельно с~традиционным 
методом решения проб\-ле\-мы и~выполняется сравнение их эф\-фек\-тив\-ности 
пользователями и~реальными стейк\-хол\-де\-ра\-ми. Первоначально 
у~\mbox{РАСИГИА} должна быть отключена воз\-мож\-ность оказывать ка\-кое-ли\-бо воздействие на реальную внеш\-нюю среду, а~результатом ее\linebreak
 работы 
долж\-ны стать рекомендации по оказанию таких воздействий. После 
удовлетворительной оценки пользователей и~стейк\-хол\-де\-ров \mbox{РАСИГИА} 
может быть переведена в~\mbox{автоматический} режим взаимодействия со средой, 
а~традиционный метод решения проб\-ле\-мы используется в~качестве резервного 
для проверки ее работы еще в~течение некоторого времени. Длительности 
каждого из этих периодов долж\-ны определяться заказчиками сис\-те\-мы для 
решения конкретной проб\-ле\-мы совместно с~коллективом разработчиков; 
  \item сопровождение после внед\-ре\-ния поз\-во\-ля\-ет собирать жалобы, замечания и~предложения в~процессе эксплуатации \mbox{РАСИГИА}, в~том чис\-ле от 
людей, которые ошибочно не были включены в~со\-став стейкхолдеров.\\[-13pt] 
  \end{enumerate}
  
  Стадия оценки и~корректировки методологии в~определенном смысле длится 
на протяжении всего проекта, так как для ее реализации долж\-ны вес\-тись 
протоколы де\-я\-тель\-ности разработчиков,\linebreak в~которых отмечается дли\-тель\-ность 
реализации каж\-до\-го этапа, выполненные возвраты и~их причины. Однако 
именно по завершении проекта выполняется рефлексия проделанной работы, 
когда разработчики долж\-ны проанализировать удачные и~провальные решения, 
причины рас\-хож\-де\-ния результатов с~планами, возвратов к~предыду\-щим этапам 
и~фазам разработки \mbox{РАСИГИА}, затягивания отдельных этапов 
разработки, из\-бы\-точ\-ность или, наоборот, не\-ин\-фор\-ма\-тив\-ность 
по\-стро\-ений~\cite{4-lis}. По результатам анализа в~методологию вносятся 
изменения в~статусе <<предложение>>, которые после под\-тверж\-де\-ния 
эф\-фек\-тив\-ности в~новых проектах закрепляются в~новой версии методологии.

\vspace*{-9pt}

\section{Заключение}

\vspace*{-3pt}

  В работе представлена темпоральная структура (жизненный цикл) 
разработки \mbox{РАСИГИА}, опи\-сы\-ва\-ющая процессы сис\-тем\-но\-го анализа 
проб\-ле-\linebreak мы, моделирования, эскизного и~технического \mbox{проектирования} сис\-те\-мы, 
ее программной реализации, отладки и~тестирования. 
Основной результат 
организации работ в~соответствии с~предложенной методологией~--- 
программная реализация \mbox{РАСИГИА}, релевантно моделирующая 
коллектив специалистов, со\-вмест\-но ре\-ша\-ющих по\-став\-лен\-ную проб\-ле\-му 
с~учетом ее слабой формализации, не\-од\-но\-род\-ности, сетевого характера условий 
и~целей, не\-опре\-де\-лен\-ности и~ди\-на\-мич\-ности~\cite{5-lis}. Кроме того, в~результате 
рефлексивной стадии методологии формируется ее новая версия или 
подтверждается эф\-фек\-тив\-ность су\-щест\-ву\-ющей, что представляется\linebreak 
дополнительным результатом работ. Таким образом, методология предполагает 
свое развитие, потенциально обеспечивающее ее ре\-ле\-вант\-ность \mbox{актуальным}
подходам к~проектированию и~реализации интеллектуальных информационных 
сис\-тем.

\vspace*{-9pt}
  
{\small\frenchspacing
 { %\baselineskip=10.6pt
 %\addcontentsline{toc}{section}{References}
 \begin{thebibliography}{99}
 
 \vspace*{-3pt}
 
  \bibitem{1-lis}
   \Au{Поспелов Д.\,А.} Десять <<горячих точек>> в~исследованиях по искусственному 
интеллекту~// Искусственный\linebreak\vspace*{-12pt}

\columnbreak

\noindent
 интеллект и~принятие решений, 2019. №\,4. С.~3--9. doi: 
10.14357/20718594190401. EDN: BAUHFV.
  
  \bibitem{2-lis}
\Au{Тарасов В.\,Б.} От многоагентных сис\-тем к~интеллектуальным организациям: 
философия, психология, информатика.~--- М.: Эдиториал УРСС, 2002. 348~с.
  \bibitem{3-lis}
  \Au{Колесников А.\,В., Кириков~И.\,А., Листопад~С.\,В.} Ги\-брид\-ные интеллектуальные 
сис\-те\-мы с~самоорганизацией: координация, со\-гла\-со\-ван\-ность, спор.~--- М.: ИПИ РАН, 2014. 
189~с.
  \bibitem{4-lis}
  \Au{Новиков А.\,М., Новиков~Д.\,А.} Методология.~--- М.: Синтег, 2007. 668~с.
  \bibitem{5-lis}
  \Au{Листопад С.\,В.} Характеристики и~логическая структура методологии по\-стро\-ения  
реф\-лек\-сив\-но-ак\-тив\-ных сис\-тем искусственных гетерогенных интеллектуальных 
агентов~// Сис\-те\-мы и~средства \mbox{информатики}, 2023. Т.~33. №\,4. С.~16--27. doi: 
10.14357/ 08696527230402. EDN: TRTHEI.
  \bibitem{6-lis}
  \Au{Тарасенко Ф.\,П.} Прикладной сис\-те\-мный анализ.~--- М.: 
КНОРУС, 2010. 224~с.
  \bibitem{7-lis}
  \Au{Ларичев О.\,И.} Вербальный анализ решений.~--- М.: Наука, 2006. 181~с.
  \bibitem{8-lis}
  \Au{Акофф Р.} Акофф о менеджменте~/ Пер.\ с~англ.~--- СПб.: Питер, 2002. 448~с.
  (\Au{Akoff~R.\,L.} Ackoff's best: His classic writings on management.~--- New 
York, NY, USA: Wiley, 1999. 368~p.)
  \bibitem{9-lis}
  \Au{Колесников А.\,В., Листопад~С.\,В., Румовская~С.\,Б., Данишевский~В.\,И.} 
Неформальная аксиоматическая тео\-рия ролевых визуальных моделей~// Информатика и~её 
применения, 2016. Т.~10. Вып.~4. С.~114--120.  doi: 10.14357/19922264160412. EDN: XGSIVN.
  \bibitem{10-lis}
  \Au{Черушева Т.\,В.} Проектирование программного обеспечения.~--- Пенза: ПГУ, 2014. 
172~с.
  \bibitem{11-lis}
  \Au{Boissier O., Bordini~R.\,H., Hubnerand~J., Ricci~A.} Multi-agent oriented programming: 
Programming multi-agent systems using JaCaMo.~--- Intelligent robotics and autonomous agents 
series.~--- Cambridge: The MIT Press, 2020. 264~p.
  \bibitem{12-lis}
  \Au{Смирнов С.\,С., Смольянинова~В.\,А.} Введение в~разработку многоагентных сис\-тем 
в~среде Jason. Основы программирования на языке AgentSpeak.~--- М.: \mbox{МИРЭА}, 2009. 136~с.
  \bibitem{13-lis}
  \Au{Канер~С., Фолк~Д., Нгуен~Е.\,К.} Тестирование про\-грам\-мно\-го обеспечения. 
Фундаментальные концепции менеджмента биз\-нес-при\-ло\-же\-ний~/
Пер. с~англ.~--- Киев: ДиаСофт, 
2001. 544~с. (\Au{Kaner~С., Falk~J., Nguyen~H.\,Q.} {Testing computer software}.~--- 
International Thomson Computer Press,  1999. 496~p.)
  \bibitem{14-lis}
  \Au{Романькова Т.\,Л.} Тестирование программного обеспечения. {\sf 
https://elib.gstu.by/bitstream/handle/220612/ 9860/416.pdf}.

\end{thebibliography}

 }
 }

\end{multicols}

\vspace*{-6pt}

\hfill{\small\textit{Поступила в~редакцию 25.11.23}}

%\vspace*{8pt}

%\pagebreak

\newpage

\vspace*{-28pt}

%\hrule

%\vspace*{2pt}

%\hrule



\def\tit{LIFE CYCLE OF METHODOLOGY FOR~CONSTRUCTING REFLEXIVE-ACTIVE SYSTEMS OF~ARTIFICIAL HETEROGENEOUS INTELLIGENT AGENTS}


\def\titkol{Life cycle of methodology for~constructing reflexive-active systems of~artificial heterogeneous intelligent agents}


\def\aut{S.\,V.~Listopad}

\def\autkol{S.\,V.~Listopad}

\titel{\tit}{\aut}{\autkol}{\titkol}

\vspace*{-8pt}


\noindent
Federal Research Center ``Computer Science and Control'' of the Russian Academy of 
Sciences, 44-2~Vavilov Str., Moscow 119333, Russian Federation

\def\leftfootline{\small{\textbf{\thepage}
\hfill INFORMATIKA I EE PRIMENENIYA~--- INFORMATICS AND
APPLICATIONS\ \ \ 2024\ \ \ volume~18\ \ \ issue\ 1}
}%
 \def\rightfootline{\small{INFORMATIKA I EE PRIMENENIYA~---
INFORMATICS AND APPLICATIONS\ \ \ 2024\ \ \ volume~18\ \ \ issue\ 1
\hfill \textbf{\thepage}}}

\vspace*{4pt}
  
  
   
   \Abste{The paper presents the temporal structure (life cycle) of the methodology for 
constructing reflexive-active systems of artificial heterogeneous intelligent agents. These systems 
are designed for computer modeling of processes and effects that arise when solving practical 
problems by teams of specialists under the guidance of a~decision maker. Artificial heterogeneous 
intelligent agents of reflexive-active systems are active subjects capable of reasoning, 
communication, and reflection as the ability to model the reasoning of other agents of the system 
and themselves. Modeling of reflexive processes ensures the development by agents of a~consistent 
understanding of the control object, the purpose of collective work, and the norms of interaction 
allowing the system to self-organize and re-develop a relevant hybrid intelligent method for solving 
the next problem.}
   
   \KWE{reflection; methodology; reflexive-active system of artificial heterogeneous intelligent 
agents; hybrid intelligent multiagent system; team of specialists}
   
 
   
\DOI{10.14357/19922264240112}{GUAMVE}

\vspace*{-8pt}

\Ack

\vspace*{-1pt}


     \noindent
     This work was supported by the Russian Science Foundation, project No.\,23-21-00218.


\vspace*{6pt}

  \begin{multicols}{2}

\renewcommand{\bibname}{\protect\rmfamily References}
%\renewcommand{\bibname}{\large\protect\rm References}

{\small\frenchspacing
 {\baselineskip=11.5pt
 \addcontentsline{toc}{section}{References}
 \begin{thebibliography}{99} 
  \bibitem{1-lis-1}
   \Aue{Pospelov, D.\,A.} 2019. Desyat' ``goryachikh tochek'' v~issledovaniyakh po 
iskusstvennomu intellektu [Ten hot topics in AI research]. \textit{Is\-kus\-stven\-nyy in\-tel\-lekt 
i~pri\-nya\-tie re\-she\-niy} [Artificial Intelligence and Decision Making] 4:3--9. doi: 
10.14357/20718594190401. EDN: BAUHFV.
  \bibitem{2-lis-1}
   \Aue{Tarasov, V.\,B.} 2002. \textit{Ot mnogoagentnykh sis\-tem k~in\-tel\-lek\-tu\-al'\-nym 
or\-ga\-ni\-za\-tsi\-yam: fi\-lo\-so\-fiya, psi\-kho\-lo\-giya, in\-for\-ma\-ti\-ka} [From multiagent systems to intelligent 
organizations: Philosophy, psychology, and computer science]. Moscow: Editorial URSS. 348~p.
  \bibitem{3-lis-1}
   \Aue{Kolesnikov, A.\,V., I.\,A.~Kirikov, and S.\,V.~Listopad.} 2014. \textit{Gib\-rid\-nye 
in\-tel\-lek\-tu\-al'\-nye sis\-te\-my s~sa\-mo\-or\-ga\-ni\-za\-tsiey: ko\-or\-di\-na\-tsiya, so\-gla\-so\-van\-nost', spor} [Hybrid 
intelligent systems with self-organization: Coordination, consistency, and dispute]. Moscow: IPI 
RAN. 189~p.
  \bibitem{4-lis-1}
   \Aue{Novikov, A.\,M., and D.\,A.~Novikov.} 2007. \textit{Me\-to\-do\-lo\-giya} [Methodology]. 
Moscow: SINTEG. 668~p.
  \bibitem{5-lis-1}
   \Aue{Listopad, S.\,V.} 2023. Kharakteristiki i~logicheskaya struk\-tu\-ra me\-to\-do\-lo\-gii po\-stro\-eniya 
refleksivno-aktivnykh sis\-tem is\-kus\-stven\-nykh ge\-te\-ro\-gen\-nykh in\-tel\-lek\-tu\-al'\-nykh agen\-tov 
[Characteristics and logical structure of the methodology for constructing reflexive-active systems 
of artificial heterogeneous intelligent agents]. \textit{Sistemy i~Sredstva Informatiki~--- Systems 
and Means of Informatics} 33(4):16--27. doi: 10.14357/08696527230402. EDN: TRTHEI.
  \bibitem{6-lis-1}
   \Aue{Tarasenko, F.\,P.} 2010. \textit{Pri\-klad\-noy sis\-tem\-nyy ana\-liz} 
[Applied systems analysis]. Moscow: KNORUS. 224~p.
  \bibitem{7-lis-1}
   \Aue{Larichev, O.\,I.} 2006. \textit{Ver\-bal'\-nyy ana\-liz re\-she\-niy} [Verbal analysis of decisions]. 
Moscow: Nauka. 181~p.
  \bibitem{8-lis-1}
   \Aue{Akoff, R.\,L.} 1999. \textit{Ackoff's best: His classic writings on management}. New 
York, NY: Wiley. 368~p.
  \bibitem{9-lis-1}
   \Aue{Kolesnikov, A.\,V., S.\,V.~Listopad, S.\,B.~Rumovskaya, and V.\,I.~Danishevskiy.} 
2016. Ne\-for\-mal'\-naya ak\-sio\-ma\-ti\-che\-skaya teo\-riya ro\-le\-vykh vi\-zu\-al'\-nykh mo\-de\-ley [Informal axiomatic 
theory of the role visual models]. \textit{Informatika i~ee Primeneniya~--- Inform. Appl.} 
10(4):114--120. doi: 10.14357/19922264160412. EDN: XGSIVN.
  \bibitem{10-lis-1}
   \Aue{Cherusheva, T.\,V.} 2014. \textit{Pro\-ek\-ti\-ro\-va\-nie pro\-gram\-mno\-go obes\-pe\-che\-niya} 
[Software design]. Penza: PGU. 172~p.
  \bibitem{11-lis-1}
   \Aue{Boissier, O., R.\,H.~Bordini, J.~Hubnerand, and A.~Ricci}. 2020. \textit{Multi-agent 
oriented programming: Programming multi-agent systems using JaCaMo}. Intelligent robotics and 
autonomous agents ser. Cambridge: The MIT Press. 264~p.
  \bibitem{12-lis-1}
   \Aue{Smirnov, S.\,S., and V.\,A.~Smol'yaninova}. 2009. \textit{Vve\-de\-nie v~raz\-ra\-bot\-ku 
mno\-go\-agent\-nykh sis\-tem v~sre\-de Jason. Osno\-vy pro\-gram\-mi\-ro\-va\-niya na yazy\-ke AgentSpeak} 
[Introduction to the development of multiagent systems in the Jason environment. Fundamentals of 
programming in the AgentSpeak language]. Moscow: MIREA. 136~p.
  \bibitem{13-lis-1}
   \Aue{Kaner, С., J.~Falk, and H.\,Q.~Nguyen}. 1999. \textit{Testing computer software}. 
International Thomson Computer Press. 496~p.
  \bibitem{14-lis-1}
   \Aue{Romankova, T.\,L.} 2014. Tes\-ti\-ro\-va\-nie pro\-gram\-mno\-go obes\-pe\-che\-niya [Software testing]. 
Available at: {\sf https://}\linebreak\vspace*{-12pt}

\columnbreak

\noindent
 {\sf elib.gstu.by/bitstream/handle/220612/9860/416.pdf} (accessed January~16, 
2024).
   
  \end{thebibliography}

 }
 }

\end{multicols}

\vspace*{-6pt}

\hfill{\small\textit{Received November 25, 2023}} 

%\vspace*{-18pt}
     
     \Contrl
     
 %    \vspace*{-3pt}
   
   \noindent
   \textbf{Listopad Sergey V.} (b.\ 1984)~--- Candidate of Science (PhD) in technology, senior 
scientist, Federal Research Center ``Computer Science and Control'' of the Russian Academy of 
Sciences, 44-2~Vavilov Str., Moscow 119133, Russian Federation;  
\mbox{ser-list-post@yandex.ru}
   
    
\label{end\stat}

\renewcommand{\bibname}{\protect\rm Литература}  %8

%\def\F{{\rm I\!F}}
\def\P{{\rm I\!P}}

\def\stat{peshkova}

\def\tit{ГРАНИЦЫ НЕЗАВЕРШЕННОЙ РАБОТЫ В~СИСТЕМЕ С~ПОВТОРНЫМИ ВЫЗОВАМИ РАЗНЫХ КЛАССОВ 
И~ПОКАЗАТЕЛЬНЫМ ВРЕМЕНЕМ ОБСЛУЖИВАНИЯ$^*$}

\def\titkol{Границы незавершенной работы в~системе с~повторными вызовами разных классов 
и~показательным временем} % обслуживания}

\def\aut{И.\,В.~Пешкова$^1$}

\def\autkol{И.\,В.~Пешкова}

\titel{\tit}{\aut}{\autkol}{\titkol}

\index{Пешкова И.\,В.}
\index{Peshkova I.\,V.}


{\renewcommand{\thefootnote}{\fnsymbol{footnote}} \footnotetext[1]
{Работа выполнена при финансовой поддержке РНФ (проект 21-71-10135).}}


\renewcommand{\thefootnote}{\arabic{footnote}}
\footnotetext[1]{Петрозаводский государственный университет; 
Институт прикладных математических исследований Карельского 
научного центра Российской академии наук, \mbox{iaminova@petrsu.ru}}


%\vspace*{-12pt}



\Abst{Исследуется односерверная система 
с~повторными вызовами и~пуассоновским входным потоком, в~которую поступает~$M$ 
классов заявок.
Для системы с~временами обслуживания, имеющими показательное распределение, 
получены верхняя и~нижняя границы для стационарной незавершенной работы. 
В~качестве нижней границы выступает   стационарная незавершенная работа 
в~классической сис\-те\-ме  $M/H_M/1$ с~гиперэкспоненциальным временем обслуживания. 
Верхней границей служит незавершенная работа в~сис\-те\-ме, в~которой к~времени 
обслуживания добавляется дополнительное время, равное интервалу между попытками 
попасть на сервер с~самой <<медленной орбиты>>. Полученные результаты численного 
моделирования подтверждают теоретические выводы.}


\KW{система с~повторными вызовами; незавершенная 
работа; стохастическая упо\-ря\-до\-чен\-ность} 

\DOI{10.14357/19922264230408}{UOKQRD}
  
%\vspace*{-6pt}


\vskip 10pt plus 9pt minus 6pt

\thispagestyle{headings}

\begin{multicols}{2}

\label{st\stat}

\section{Введение}

Модели систем с~повторными вызовами широко используются для моделирования  
телефонных станций, кол-цент\-ров, сис\-тем связи, телекоммуникационных сетей. 
В~работах~\cite{Ar1, Ar3} \mbox{изложены}  приложения и~математические методы анализа 
сис\-тем c~повторными вызовами, включая сис\-те\-мы с~постоянной интенсивностью 
повторов. В~работе~\cite{F86}   телефонная станция была смоделирована  с~по\-мощью 
такой сис\-те\-мы. Аналогичная модель используется для описания широкого класса 
протоколов множественного доступа~\cite {CSA92, CRP93}. В~част\-ности, в~работе~\cite{BG92} 
показано, что постоянная интенсивность повторных вызовов  снижает 
интенсивность попыток (при незапланированном множественном доступе) обратно 
пропорционально числу задержанных пакетов. В~результате общая ско\-рость повторной 
обработки становится нечувствительной к~виртуальному <<размеру орбиты>> (числу 
отложенных пакетов). Более того, сис\-те\-мы с~повторными вызовами с~постоянной 
интенсивностью вызовов использовались для описания TCP-тра\-фи\-ка с~короткими HTTP-со\-еди\-не\-ни\-ями~\cite{AY08,AY10} 
и~оп\-ти\-ко-элект\-ри\-че\-ской гиб\-рид\-ной схемой разрешения 
конфликтов~\cite{Wongetal09,Yaoetal02}.

 Большинство же современных моделей повторных вызовов имеют сложную природу или 
конфигурацию, и~поэтому для их исследования применяются численные методы или 
имитационное моделирование.

Ранее в~работе~\cite{mathematics2022} была доказана тео\-ре\-ма о~верх\-ней и~ниж\-ней 
границах стационарной незавершенной работы  для исходной сис\-те\-мы с~повторными 
вызовами с~постоянной интенсивностью вызовов  (см.\ тео\-ре\-му~1 ниже). Эта тео\-ре\-ма 
стала основой анализа, развитого в~данной \mbox{статье}.

В данном исследовании рассматривается частный случай  односерверной сис\-те\-мы  
с~повторными вызовами с~пуассоновским входным потоком и~показательным 
распределением времени обслуживания, при этом  время обслуживания  и~время 
нахождения на орбите зависят от класса заявки~$k$.
%
Для такой системы предлагается строить две классические сис\-те\-мы с~неограниченной 
очередью (с~ожиданием) типа $M/G/1$: в~первой (минорантной) сис\-те\-ме время 
обслуживания пред\-став\-ля\-ет собой конечную смесь времен обслуживания заявок всех 
классов (т.\,е.\ имеет гиперэкспоненциальное распределение), во второй 
(мажорантной) сис\-те\-ме ко времени обслуживания первой сис\-те\-мы добавляется 
дополнительное время, равное  интервалу между вызовами  с~самой <<медленной 
орбиты>>.   Более того, для минорантной сис\-те\-мы получено распределение 
стационарного времени ожидания в~явном виде для трех классов ($M\hm=3$).
Сис\-те\-мы, в~которых  время обслуживания задано в~виде конечной смеси 
распределений, обсуждались ранее  в~работах~\cite{pesh-mor2022, pesh2022}.
 
Структура статьи следующая. В~разд.~2 приведено описание модели сис\-те\-мы 
с~повторными вызовами, минорантной и~мажорантной сис\-тем, а~также основная тео\-ре\-ма, 
полученная авторами в~работе~\cite{mathematics2022}.
В~разд.~3 получены коэффициенты загрузки, математические ожидания 
незавершенной нагрузки, а~также преобразования Лап\-ла\-са--Стилть\-еса для 
незавершенной нагрузки в~минорантной и~мажорантной сис\-те\-мах с~показательным 
распределением времени обслуживания. В~разд.~4 приведены результаты численного 
эксперимента для случая трех классов. При этом параметры для минорантной системы 
были использованы такие же, как в~работе~\cite{rego}, в~которой получена  
функция распределения  стационарного времени пребывания. Отметим, что  в~работе~\cite{rego} 
неверно указано, что полученное распределение~--- это распределение 
стационарного времени ожидания. В~статье получена функция распределения  
стационарного времени ожидания для минорантной сис\-те\-мы  в~явном  виде. Для 
исходной сис\-те\-мы с~повторными вызовами и~мажорантной  сис\-те\-мы проведены 
численные эксперименты и~построены эмпирические функции распределения. 
Полученные результаты численного эксперимента иллюстрируют доказанную 
стохастическую упо\-ря\-до\-чен\-ность стационарной незавершенной работы в~рассмотренных 
сис\-те\-мах.

\section{Описание модели}

Рассмотрим односерверную систему с~повторными вызовами~$\Sigma$, в~которой 
обслуживаются~$M$~классов заявок. Заявки $k$-го класса поступают в~сис\-те\-му в~соответствии 
с~пуассоновским потоком с~па\-ра\-мет\-ром~$\lambda_k$, $k\hm=1,\ldots,M$. 
Если заявка застает сервер пустым, то она немедленно обслуживается, в~противном 
случае, если сервер занят,  заявка уходит на $k$-ю орбиту бесконечного объема, 
образуя очередь в~порядке поступления на орбиту, $k\hm=1,\ldots,M$. Первая 
в~очереди на $k$-й орбите заявка производит независимые попытки присоединиться 
к~обслуживанию на сервере через экспоненциальное  время~$\eta_k$.
Интенсивность вызовов с~орбиты не зависит от размера орбиты (т.\,е.\ от числа 
заявок на орбите), но может зависеть от класса орбиты~$k$. Такие сис\-те\-мы 
называют сис\-те\-ма\-ми  с~постоянной ин\-тен\-сив\-ностью вызовов.

Обозначим через $t_n$ момент прихода $n$-й заявки в~общий пуассоновский входной  
поток (образованный суперпозицией~$M$~входных потоков, $t_1\hm=0$),   $S_n{(k)}$~--- 
время обслуживания  $n$-й заявки  $k$-го класса,  $k\hm=1,\ldots,M$, $n\hm\ge1$. Пусть 
последовательность независимых одинаково распределенных (н.\,о.\,р.)\ интервалов 
между приходами заявок  $\{T_n:=t_{n+1}\hm-t_n,\ n\hm\ge 1\}$ и~последовательность 
времен обслуживания  $\{S_n{(k)},\ n\hm\ge1\}$ независимы для каждого  $k$-го 
класса.
Предположим, что интервалы между приходами заявок с~(непустой)  $k$-й орбиты~$\eta_k$ 
распределены показательно и~не зависят от размера орбиты  (числа заявок 
на $k$-й орбите). Время обслуживания заявок $k$-го класса  $S(k)$ имеет 
произвольное распределение  с~функцией распределения (ф.~р.)\ $F_{S(k)}$, 
$k\hm=1,\ldots, M$. (Далее в~обозначениях  опускаем индекс номера заявки.)  Обозначим
\begin{equation*}
%\label{rates}
\lambda=\sum\limits_{k=1}^M\lambda_k ;\ \ \ \rho_k=\lambda_k\mathbb{E} S{(k)}; \enskip 
\rho=\sum\limits_{k=1}^M \rho_k.
\end{equation*}
Пусть $W(t)$ есть незавершенная работа в~момент времени~$t^-$, и~предположим, 
что система в~начальный момент времени пуста: $W(0)\hm=0$. Обозначим 
$W_n=W(t_n)$, $n\hm\ge1$.
Известно~\cite{Morozov2019}, что неравенство
 \begin{equation}
 \label{stability}
 \rho + \max\limits_{k=1,\ldots, M} \fr{\lambda}{\lambda+\eta_k} < 1
 \end{equation}
служит достаточным условием стационарности сис\-те\-мы.  При  таком условии 
существует  предел
$$
W_n \Rightarrow W\,,\quad n\to\infty
$$
(где обозначим $\Rightarrow$ схо\-ди\-мость по распределению). Функция распределения~$F_W$ 
стационарной незавершенной работы~$W$ для исходной сис\-те\-мы~$\Sigma$ неизвестна. 
С~другой стороны, $W$ служит важной характеристикой качества обслуживания 
сис\-те\-мы. Ниже мы построим верхнюю и~нижнюю границы~$F_W$, используя 
классические  $M/G/1$ сис\-те\-мы с~неограниченной оче\-редью, в~которых время 
обслуживания пред\-став\-ля\-ет\-ся конечной смесью распределений.


Рассмотрим две новые системы: \textit{минорантную сис\-те\-му}~$\Sigma^{(1)}$ 
и~\textit{мажорантную сис\-те\-му}~$\Sigma^{(2)}$. (Далее индекс~$(i)$ означает номер 
сис\-те\-мы.)
Входной поток во все три сис\-те\-мы~--- пуассоновский  с~параметром~$\lambda$ (это 
суперпозиция входных потоков, образованных заявками разных классов).

Пусть в~минорантной сис\-те\-ме~$\Sigma^{(1)}$ время обслуживания $S^{(1)}\hm=S$ задано 
конечной смесью распределений вида
\begin{equation}
\label{mixture}
S=\sum\limits_{k=1}^M I(k) S(k), \enskip n\ge1\,,
\end{equation}
где  $I(k)$~--- индикатор, такой что  $\mathbb{E} I(k)\hm=p_k=\lambda_k/\lambda$; $S(k)$~--- время  обслуживания заявки $k$-го класса.
Заметим, что сис\-те\-ма~$\Sigma^{(1)}$ пред\-став\-ля\-ет собой классическую 
(с~дисциплиной обслуживания первым при\-шел\,--\,пер\-вым обслужен) односерверную сис\-те\-му 
типа $M/G/1$ с~неограниченной очередью, в~которой время обслуживания~(\ref{mixture})
является конечной смесью времен обслуживания заявок всех классов исходной 
сис\-те\-мы.

В мажорантной системе~$\Sigma^{(2)}$~--- классической односерверной сис\-те\-ме 
типа  $M/G/1$ с~неограниченной оче\-редью~--- каждая заявка обслуживается на 
сервере в~течение времени~$S$, заданного соотношением~\eqref{mixture},  плюс 
время~$\xi_0$, имеющее показательное распределение с~па\-ра\-мет\-ром  
$$
\mu_0=\min\limits_{1\le k\le M} (\lambda\hm+\eta_k),
$$
 т.\,е.
%\begin{equation*}
%\label{sums2}
$S^{(2)} \hm= S\hm +\xi_0$.
%\end{equation*}
Таким образом, случайная величина (с.\,в.)~$\xi_0$ соответствует самой 
<<медленной>> орбите (с наибольшими интервалами между попытками). Заметим, что 
мажорантная сис\-те\-ма~$\Sigma^{(2)}$ имеет другой коэффициент загрузки,
\begin{equation*}
%\label{rho2def}
 \rho^{(2)}=\lambda \mathbb{E} S+\fr{\lambda}{\mu_0}=\rho+\fr{\lambda}{\mu_0}\,,
\end{equation*}
и условие стационарности~\eqref{stability} для нее принимает вид:
$$
\rho^{(2)}<1.
$$

В работе~\cite{mathematics2022}  доказана сле\-ду\-ющая тео\-ре\-ма, в~которой даны 
верхняя и~нижняя границы незавершенной работы~$W$ в~исходной сис\-те\-ме 
с~повторными вызовами~$\Sigma$.

\smallskip

\noindent
\textbf{Теорема~1.}
\textit{Пусть сис\-те\-мы  $\Sigma^{(1)}$, $\Sigma^{(2)}$ и~$\Sigma$ в~начальный момент времени 
пустые, т.\,е.}
 \begin{equation*}
W_1^{(1)}=W_1=W_1^{(2)}=0\,.
 \end{equation*}
\textit{Тогда при  выполнении условия}~\eqref{stability} \textit{стационарные времена 
незавершенной работы стохастически упорядочены}:
 \begin{equation}
 \label{theor1-1}
 W^{(1)}\underset{\mathrm{st}}\le W \underset{\mathrm{st}}\le W^{(2)},
 \end{equation}
\textit{где $W^{(1)}\le_{\mathrm{st}} W$ означает $\overline F_{W^{(1)}} (x) \hm\le 
\overline F_{W} (x) $ для любого $x\hm\ge 0$, $\overline F_{W^{(1)}}  (x)\hm= 1\hm-  
F_{W^{(1)}} (x)$}.


\smallskip

В следующем разделе применим данный результат для сис\-те\-мы, в~которой~$M$~классов 
заявок, име\-ющих  показательное распределение времени обслуживания.

\section{Границы незавершенной работы~$W$ в~системе с~показательным 
обслуживанием разных классов }

Пусть в~исходной сис\-те\-ме с~повторными вызовами~$\Sigma$ времена обслуживания 
$k$-го класса~$S(k)$ имеют показательное распределение с~ф.\,р.
\begin{equation}
\label{hyperexp}
F_{S(k)}(x)= 1- e^{-\mu_k x}, \enskip x\ge 0\,, \ \mu_k >0\,.
\end{equation}

В качестве минорантной сис\-те\-мы~$\Sigma^{(1)}$ рассмотрим сис\-те\-му 
с~неограниченной  очередью  $M/H_M/1$, в~которой времена обслуживания $S^{(1)}\hm=S$ 
имеют гиперэкспоненциальное распределение (пред\-став\-ля\-ют\-ся $M$-ком\-по\-нент\-ной 
смесью показательно распределенных с.\,в.~$S(k)$) с~ф.\,р.
\begin{multline*}
F_{S^{(1)}}(x) = 1 -  \sum\limits_{k=1}^M p_k e^{-\mu_k x}, \enskip \mu_k > 0\,, \\ 
p_k\ge 0\,,\enskip k=1,\ldots,M, \enskip \sum\limits_{i=k}^M p_k=1\,.
\end{multline*}

Обозначим коэффициент загрузки  $\rho^{(1)} \hm=\sum\nolimits_{k=1}^M \lambda 
p_k/\mu_k$ в~сис\-те\-ме~$\Sigma^{(1)}$. Поскольку
$$
\rho^{(1)} \le \rho + \fr{\lambda}{\mu_0 }\,,
$$
то, если условие стационарности~\eqref{stability} выполнено,    сис\-те\-ма~$\Sigma^{(1)}$ также стационарна.

Рассмотрим преобразование Лап\-ла\-са--Стилть\-еса:
\begin{equation*}
%\label{lstdef}
 \psi_{S_e}(z)=\int\limits_0^\infty e^{-zt} \,dF_{S_e}(t),
\end{equation*}
где $F_{S_e}$ -- так называемый \textit{интегрированный хвост
распределения} с~плот\-ностью
\begin{equation*}
%\label{fequilibr}
f_{S_e}(x)=\fr{1}{\mathbb{E} S}\, \overline F_S(x),\enskip x\ge0\,.
% f_{S_e}(x)=\mu \overline{F}_S(x),\enskip x\ge0\,.
\end{equation*}

Распределение $F_{S_e}$ соответствует распределению стационарного перескока 
процесса вос\-ста\-нов\-ле\-ния, фор\-ми\-ру\-емо\-го по\-сле\-до\-ва\-тель\-ностью н.\,о.\,р.\ времен 
обслуживания~$\{S_n\}$~\cite{Asmus}.

В работе~\cite{mathematics2022} доказано, что преобразование\linebreak Лап\-ла\-са--Стилть\-еса 
стационарной незавершенной работы~$W^{(1)}$ выражается через преобразования 
Лап\-ла\-са--Стилть\-еса компонент смеси времен обслуживания в~сле\-ду\-ющем виде:
\begin{multline}
\label{lstformixture}
\psi_{W^{(1)}}(z)=\fr{1-\rho}{z\left(1-\rho\sum\nolimits_{k=1}^M 
(\rho_k/{\rho}) \psi_{S_e(k)}(z)\right)}={}\\
{}=\fr{1-\rho}{z\left(1-
\sum\nolimits_{k=1}^M \rho_k \psi_{S_e(k)}(z)\right)}.
\end{multline}

Преобразование Лапласа--Стилть\-еса для показательного распределения хорошо 
известно: 
$$
\psi_{S_e(k)}(z) = \fr{\mu_k}{\mu_k +z}\,.
$$ 
Подставляя его в~соотношение~\eqref{lstformixture},  получим
$$
\psi_{W^{(1)}}(z)=\fr{1-\sum\nolimits_{k=1}^M \lambda_k/\mu_k}{z\left(1-
\sum\nolimits_{k=1}^M  {\lambda_k}/(\mu_k +z)   
\right)}\,.
$$

Применяя формулу По\-ла\-чи\-ка--Хин\-чи\-на, получим среднюю стационарную незавершенную 
работу  в~сис\-те\-ме~$\Sigma^{(1)}$ в~виде:
\begin{equation}
\label{ew5}
\mathbb{E}  W^{(1)} = \fr{\lambda \mathbb{E} ( S^{(1)})^2}{2(1-\rho^{(1)})} = 
\fr{\sum\nolimits_{k=1}^M \rho_k^2 +\rho^2}{2\lambda (1-\rho)}\,.
\end{equation}


Рассмотрим теперь мажорантную сис\-те\-му~$\Sigma^{(2)}$,  время обслуживания 
в~которой равно сумме с.\,в.~$S$ с~гиперэкспоненциальным распределением~\eqref{hyperexp} и~с.~в.~$\xi_0$, т.\,е.\
 $\ S^{(2)}\hm=S \hm+ \xi_0$. Обозначим для простоты 
$\mu_0\hm=\min\nolimits_{1\le k\le M} (\lambda\hm+\eta_k)$, и~пусть с.\,в.~$\xi_0$ имеет 
показательное распределение  с~па\-ра\-мет\-ром~$\mu_0$. Условие стационарности 
в~такой сис\-те\-ме совпадает с~\eqref{stability}.

Известно \cite{mathematics2022}, что в~сис\-те\-ме~$\Sigma^{(2)}$ преобразование 
Лап\-ла\-са--Стилть\-еса для стационарной незавершенной работы~$W^{(2)}$ имеет 
сле\-ду\-ющий вид:
\begin{multline}
\label{reslemma2}
 \psi_{W^{(2)}} (z)=
 \left(1-\rho- \fr{\lambda}{\mu_0}\right) \Bigg/
\left(  z\left(
\vphantom{\left(\sum\limits_{k=1}^M\right)}
1-{}\right.\right.\\
\left.\left.{}-\fr{\mu_0}{\mu_0+z}\left(\sum\limits_{k=1}^M \rho_k \psi_{S_e(k)}(z) +
\fr{\lambda}{\mu_0}\right)\right)\right)\,.
\end{multline}
Подставив $\psi_{S_e(k)}(z) = \mu_k/(\mu_k \hm+z)$ в~\eqref{reslemma2}, получим
\begin{multline*}
\psi_{W^{(2)}} (z)=
\left(1-\sum\limits_{k=1}^M \fr{\lambda_k}{\mu_k}-\fr{\lambda}{\mu_0}\right) \Bigg/
\left(z\left(
\vphantom{\left(\sum\limits_{k=1}^M\right)}
1-{}\right.\right.\\
\left.\left.{}-\fr{\mu_0}{\mu_0+z}\left(\sum\limits_{k=1}^M \fr{\lambda_k}{\mu_k+z}  +\fr{\lambda}{\mu_0}\right)\right)\right)\,.
\end{multline*}
Аналогично формуле~\eqref{ew5} можно получить среднюю стационарную незавершенную 
работу
\begin{multline}
\label{ew6}
\mathbb{E}  W^{(2)} = \fr{\lambda \mathbb{E} \left( S^{(2)}\right)^2}{2\left(1-\rho^{(2)}\right)} = {}\\
{}=
\fr{\mu_0^2\left(\rho^2 + \sum\nolimits_{k=1}^M \rho_k^2 \right) +2 \lambda (\lambda +\rho \mu_0)}{2\lambda \mu_0 (\mu_0 -\rho \mu_0 -\lambda)}.
\end{multline}



Из теоремы~1 следует, что
стационарные времена незавершенной работы стохастически упорядочены:
 $$
 W^{(1)}\underset{\mathrm{st}}\le W \underset{\mathrm{st}}\le W^{(2)},
 $$
а следовательно,  их математические ожидания также упорядочены~\cite{Ross}:
$$
 \mathbb{E} W^{(1)}\le \mathbb{E} W \le \mathbb{E} W^{(2)}.
$$

Действительно, легко проверить, что
\begin{multline*}
\mathbb{E}  W^{(2)} = \fr{\lambda \mathbb{E} \left( S^{(2)}\right)^2}{2(1-\rho^{(2)})} = {}\\
{}=
\fr{\mu_0^2 \lambda \mathbb{E} ( S^{(1)})^2 + 2 \lambda (\lambda +\rho \mu_0)} {\mu_0^2 2(1-\rho^{(1)}) - 2\lambda^2 \mu_0 } 
\ge \mathbb{E}  W^{(1)}.
\end{multline*}


\section{Численный эксперимент}

В качестве примера рассмотрим систему с~повторными вызовами~$\Sigma$ с~тремя 
классами заявок ($M\hm=3$), в~которую поступает пуассоновский поток 
с~ин\-тен\-сив\-ностью $\lambda\hm=10$.
Пусть
$p_1\hm=1/2$, $p_2\hm=1/3$, $p_3\hm=1/6$, $\mu_1\hm=10$, $\mu_2\hm=30$ и~$\mu_3\hm=60$.
Будем полагать, что~$\eta_k$ принимают значения
$\eta_1\hm=50$, $\eta_2\hm=100$ и~$\eta_3\hm=150.$
В~этом случае $\mu_0\hm=\lambda\hm+\eta_1\hm=60$,  коэффициенты загрузки \mbox{равны}
$$
\rho^{(1)}=\rho= \sum\limits_{k=1}^3 \fr{\lambda p_k}{\mu_k} = \fr{23}{36}\,; \enskip  
\rho^{(2)} = \rho+\fr{\lambda}{\mu_0} = \fr{29}{36} < 1
$$
и условие стационарности~\eqref{stability} выполнено. По 
формулам~\eqref{ew5}--\eqref{ew6} находим математические ожидания $\mathbb{E} W^{(1)} \hm\approx 0{,}093$ 
и~$\mathbb{E} W^{(2)} \hm\approx 0{,}106$.

Рассмотрим минорантную сис\-те\-му~$\Sigma^{(1)}$, в~которой время обслуживания 
имеет гиперэкспоненциальное распределение
$$
\overline F_S(x)=\sum\limits_{k=1}^3 p_k e^{-\mu_k x}.
$$
%
Для такой классической системы $M/G/1$ в~работе~\cite{rego} получено 
распределение числа клиентов в~сис\-те\-ме в~стационарном режиме~$N^{(1)}$ в~сле\-ду\-ющем виде:
$$
\pi_n^\ast=\mathbb{P}\left\{N^{(1)}=n\right\}=\sum\limits_{k=1}^3 \beta_k (r_k)^n,
$$
где параметры  $\beta_k$ и~$r_k$ вычислены и~равны
\begin{alignat*}{3}
    \beta_1&=0{,}040;&\enskip \beta_2&=0{,}075;&\enskip \beta_3&=0{,}245; \\
    r_1&=0{,}146;&\enskip r_2&=0{,}268;&\enskip r_3&=0{,}711.
\end{alignat*}
В работе~\cite{rego} с~помощью результата~\cite{haji}  получено стационарное 
распределение  \textit{времени пребывания} клиента в~сис\-те\-ме~$V^{(1)}$ с~хвостом 
ф.\,р.\ вида
$$
\overline F_{V^{(1)}}(x)=\sum\limits_{k=1}^3 \gamma_k e^{-\theta_k x},
$$
где коэффициенты $\gamma_k$  и~па\-ра\-мет\-ры~$\theta_k$  для исходных параметров 
\mbox{равны}
\begin{alignat*}{3}
\gamma_1&=0{,}047;&\quad \gamma_2&=0{,}103; &\quad \gamma_3&=0{,}850; \\
\theta_1&=58{,}633; &\quad \theta_2&=27{,}307;&\quad \theta_3&= 4{,}059.
\end{alignat*}
При этом в~утверждении теоремы~3 из работы ~\cite{rego} было  ошибочно указано, 
что $\sum\nolimits_{k=1}^3 \gamma_k\hm=1\hm-\rho^{(1)}$ (что было бы верно, если бы 
распределение~$F_V^{(1)}$ соответствовало  \textit{времени ожидания}). На самом 
деле легко проверить, что $\sum\nolimits_{k=1}^3 \gamma_k\hm=1$. Для исправления этой 
неточности повторим вывод, получив выражение для стационарного времени 
\textit{ожидания} в~сис\-те\-ме (что  соответствует незавершенной работе в~сис\-те\-ме 
в~момент прихода клиента). Заметим, что~$\pi_{n+1}^\ast$ есть стационарная 
вероятность наблюдать~$n$~клиентов в~очереди, т.\,е.
$$
\mathbb{P}\{Q^{(1)}=n\}=\pi_{n+1}^\ast,
$$
где $Q^{(1)}$ есть число клиентов \textit{в очереди} в~стационарном режиме.

Вычислив производящую функцию вероятностей~$\pi (z)$ для~$Q^{(1)}$, получим
\begin{multline*}
    \pi (z)=\sum\limits_{n=0}^\infty z^n \pi_{n+1}^\ast = \sum\limits_{k=1}^3
    \fr{\beta_k }{z}\sum\limits_{n=1}^\infty (r_k z)^n={}\\
    {}=\sum\limits_{k=1}^3 \fr{\beta_k }{z}\left(\fr{1}{1-r_k z}-1\right)=\sum\limits_{k=1}^3 \fr{\beta_k r_k}{1-r_k z}\,.
\end{multline*}

C другой стороны,  производящая функция стационарной очереди~$\pi(z)$ и~преобразование Лап\-ла\-са--Стилть\-еса для стационарного времени 
ожидания~$\psi_{W^{(1)}}(z)$ связаны формулой:
\begin{multline*}
\pi(z)= \sum\limits_{n=0}^{\infty}  \int\limits_{0^-}^{\infty} z^n e^{-\lambda x} \fr{(\lambda x )^n}{n!}\, dF_{W^{(1)}} (x) ={}\\
{}=
\int\limits_{0^-}^{\infty} e^{-(\lambda-\lambda z) x} \, d F_{W^{(1)}} (x) ={}\\
{}=\psi_{W^{(1)}} (\lambda - \lambda z) + \left(1-\rho^{(1)}\right),
\end{multline*}
 где $F_{W^{(1)}}(0) = (1-\rho^{(1)})$~--- скачок  ф.~р.\ в~нуле. Сделав замену 
переменной $s\hm=\lambda\hm-\lambda z$,  получим
$$
\psi_{W^{(1)}} (s)=\sum_{k=1}^3 \fr{\beta_k r_k}{1-r_k(1-
s/\lambda)}=\sum\limits_{k=1}^3 \fr{\beta_k r_k}{1-r_k}\,\fr{\theta_k}{\theta_k+ s}\,,
$$
где, как и~в~работе~\cite{rego},
$$
\theta_k=\fr{\lambda(1-r_k)}{r_k}\,.
$$

{ \begin{center}  %fig1
 \vspace*{-1pt}
     \mbox{%
\epsfxsize=79mm 
\epsfbox{pes-1.eps}
}

\end{center}



\noindent
{\small{Функции распределения в~нижней~$\Sigma^{(1)}$~(\textit{1}), исходной $\Sigma$~(\textit{2}) 
и~верх\-ней~$\Sigma^{(2)}$~(\textit{3}) сис\-те\-мах при $\lambda \hm= 10$, $p_1\hm= 1/2$, $p_2\hm= 1/3$, $p_3\hm=1/6$, 
$\mu_1\hm=10$, $\mu_2\hm=30$, $\mu_3\hm= 60$, $\eta_1\hm=50$, $\eta_2\hm=100$ и~$\eta_3\hm=150$}}}

\vspace*{12pt}

\noindent
Таким образом,~$\psi_{W^{(1)}}$ соответствует взвешенной  сумме показательных 
распределений. Отметим при этом, что, в~отличие от~\cite{rego}, коэффициенты 
смеси имеют вид:
$$
\hat\gamma_k=\fr{\beta_k r_k}{1-r_k}\,.
$$
Таким образом,
\begin{equation}
\label{fwlower}
\overline F_{W^{(1)}}(x)=\sum\limits_{k=1}^3 \hat\gamma_k e^{-\theta_k x} + \left(1- \rho^{(1)}\right),
\end{equation}
где
$\hat\gamma_1=0{,}007$, $\hat\gamma_2\hm=0{,}027$ и~$\hat\gamma_3\hm=0{,}604$. Заметим, что 
$\sum\nolimits_{k=1}^3 \hat\gamma_k\hm=\rho^{(1)}\hm\approx 0{,}638\hm <1.$

Воспользуемся выражением~\eqref{fwlower} для по\-стро\-ения ф.\,р.\ 
в~сис\-те\-ме~$\Sigma^{(1)}$, а~для по\-стро\-ения оценок в~исходной~$\Sigma$ и~верхней~$\Sigma^{(2)}$ 
сис\-те\-мах воспользуемся имитационным моделированием.
Построим графики (эмпирических) ф.~р.\ для незавершенной 
работы в~трех сис\-те\-мах. Как видно на рисунке, стохастический 
порядок~\eqref{theor1-1} для стационарных времен ожидания выполнен, что 
и~следовало ожидать.






\section{Заключение}

В работе показано, что для исходной системы с~повторными вызовами можно 
построить минорантную и~мажорантную системы так, что стационарная незавершенная 
нагрузка во всех трех сис\-те\-мах будет стохастически упорядочена. Численный 
эксперимент для сис\-те\-мы с~показательными временами обслуживания подтверждает 
теоретические выводы. При этом в~качестве примера рас\-смот\-ре\-ны такие па\-ра\-мет\-ры 
(как в~работе~\cite{rego}), для которых получена ф.\,р.\ 
стационарного времени ожидания в~явном виде в~минорантной сис\-теме.


{\small\frenchspacing
 { %\baselineskip=10.6pt
 %\addcontentsline{toc}{section}{References}
 \begin{thebibliography}{99}

\bibitem{Ar1}
\Au{Artalejo J.\,R.} {Accessible bibliography on retrial queues}~// Math. 
Comput. Model., 1999. Vol.~30. Iss.~3-4. P.~1--6. doi: 10.1016/S0895-7177(99)00128-4.


\bibitem{Ar3}
\Au{Artalejo J.,   Gomez-Corral~A.}
{Retrial queueing systems: A~computational approach}.~---   Springer, 2008. 318~p.
doi: 10.1007/978-3-540-78725-9.


\bibitem{F86} 
\Au{Fayolle G.}
A~simple telephone exchange with delayed feedbacks~// 
Seminar (International) on Teletraffic Analysis and Computer Performance 
Evaluation Proceedings.~--- Elseiver Science, 1986. P.~245--253.

\bibitem{CSA92}
\Au{Choi~B.\,D.,  Shin~Y.\,W.,  Ahn~W.\,C.}
Retrial queues with collision arising from unslotted {CSMA/CD} protocol~//
Queueing Syst., 1992.  Vol.~11. P.~335--356. doi: 10.1007/ BF01163860.

\bibitem{CRP93}
\Au{Choi B.\,D., Rhee~K.\,H., Park~K.\,K.} {The $M/G/1$ retrial queue with
retrial rate control policy}~//
Probab.  Eng. Inform. Sc., 1993.  Vol.~7. P.~29--46. doi: 10.1017/ S0269964800002771.

\bibitem{BG92}
\Au{Bertsekas D., Gallager~R.}
{Data networks}.~--- Athena Scientific, 2021.  570~p.

\bibitem{AY08}
\Au{Avrachenkov K., Yechiali~U.}
Retrial networks with finite buffers and their application to Internet data 
traffic~//  Probab. Eng. Inform. Sc., 2008. 
Vol.~22. P.~519--536. doi: 10.1017/S0269964808000314.

\bibitem{AY10} %8
\Au{Avrachenkov K., Yechiali~U.}
{On tandem blocking queues with a~common retrial queue}~// Comput.  
Oper. Res., 2010. Vol.~37. Iss.~7. P.~1174--1180. doi: 10.1016/j.cor.2009. 10.004.



\bibitem{Yaoetal02} %9
\Au{Yao S., Xue~F.,  Mukherjee~B.,  Yoo~S.\,J.\,B., Dixit~S.}
{Electrical ingress buffering and traffic aggregation for optical packet 
switching and their effect on TCP-level performance in optical mesh networks}~//
IEEE Commun. Mag., 2002.
Vol.~40. Iss.~9. P.~66--72. doi: 10.1109/MCOM. 2002.1031831.

\bibitem{Wongetal09} %10
\Au{Wong E.\,W.\,M.,  Andrew L.\,L.\,H.,  Cui~T.,  Moran~B.,  Zalesky~A., Tucker~R.\,S., Zukerman~M.}
{Towards a~bufferless optical internet}~//
J.~Lightwave Technol., 2009. Vol.~27. P.~2817--2833. doi: 10.1109/JLT.2009.2017211.

\bibitem{mathematics2022} %11
\Au{Morozov E.\,V., Peshkova~I.\,V., Rumyantsev~A.\,S.} Bounds and maxima for the 
workload in a~multiclass orbit queue~// Mathematics, 2023. Vol.~11. Iss.~3. 
Art.~564. doi: 10.3390/math11030564.

\bibitem{pesh-mor2022}  %12
\Au{Peshkova I., Morozov~E.} On comparison of 
multiserver systems with multicomponent mixture distributions~// J.~Math. Sci., 2022. Vol.~267. No.\,2. P.~260--272.
doi: 10.1007/ s10958-022-06132-z. 

\bibitem{pesh2022} %13
\Au{Пешкова И.\,В.} 
Границы экстремального индекса времени ожидания в~системе
$M/G/1$ с~распределением времени обслуживания в~виде конечной
смеси~// Информатика и~её применения, 2022.
Т.~16. Вып.~2. С.~26--33. doi: 10.14357/19922264220405. EDN: VFKRKT.



\bibitem{rego} %14
\Au{Rego V.}
Some explicit formulas for mixed exponential service systems~//
Computers Operations Research, 1988. Vol.~15. Iss.~6. P.~509--520. doi: 
{10.1016/0305-0548(88)90047-0}.

\bibitem{Morozov2019}  %15
\Au{Morozov E.\,V.,   Rumyantsev~A.\,S., Dey~S.,  Deepak~T.\,G.}
Performance analysis and stability of multiclass orbit queue with constant 
retrial rates and balking~//
 Perform. Evaluation, 2019.  Vol.~134. Art.~102005. doi: 
10.1016/ J.PEVA.2019.102005.

\bibitem{Asmus} %16
\Au{Asmussen S.} Applied probability and queues. Stochastic modelling and 
applied probability.~--- New York, NY, USA: Springer-Verlag, 2003. 438~p.

\bibitem{Ross} %17
\Au{Ross S., Shanthikumar~J., Zhu~Z.}  On increasing-failure-rate random 
variables~// J.~Appl. Probab., 2005. Vol.~42. P.~797--809. doi: 
10.1239/jap/1127322028.

\bibitem{haji} %18
\Au{Haji R.,  Newell~G.\,F.}  A~relation between stationary queue and waiting 
time distributions~// J.~Appl. Probab., 1971. Vol.~8. P.~617--620. doi: 10.2307/3212186.




\end{thebibliography}

 }
 }

\end{multicols}

\vspace*{-10pt}

\hfill{\small\textit{Поступила в~редакцию 26.08.23}}

\vspace*{8pt}

%\pagebreak

%\newpage

%\vspace*{-28pt}

\hrule

\vspace*{2pt}

\hrule



\def\tit{BOUNDS OF THE WORKLOAD IN~A~MULTICLASS RETRIAL QUEUE WITH~EXPONENTIAL SERVICES}


\def\titkol{Bounds of the workload in~a~multiclass retrial queue with~exponential services}


\def\aut{I.\,V.~Peshkova$^{1,2}$}

\def\autkol{I.\,V.~Peshkova}

\titel{\tit}{\aut}{\autkol}{\titkol}

\vspace*{-10pt}


\noindent 
$^1$Petrozavodsk State University, 33~Lenina Pr., Petrozavodsk 185910, Russian Federation

\noindent 
$^2$Karelian Research Center of
the Russian Academy of Sciences, 11~Pushkinskaya Str., Petrozavodsk 185910,\linebreak
$\hphantom{^1}$Russian Federation 

\def\leftfootline{\small{\textbf{\thepage}
\hfill INFORMATIKA I EE PRIMENENIYA~--- INFORMATICS AND
APPLICATIONS\ \ \ 2023\ \ \ volume~17\ \ \ issue\ 4}
}%
 \def\rightfootline{\small{INFORMATIKA I EE PRIMENENIYA~---
INFORMATICS AND APPLICATIONS\ \ \ 2023\ \ \ volume~17\ \ \ issue\ 4
\hfill \textbf{\thepage}}}

\vspace*{3pt}

 


\Abste{A~multiclass retrial queue with Poisson input and $M$ classes of customers is investigated. 
For the given retrial system with exponential service times, the lower and upper bounds of the workload are derived. 
It is shown that the workload in the classical system $M/H_M/1$ with hyperexponential service times is the lower bound for the workload of the given retrial system. 
The upper bound is the workload in the classical $M/G/1$ system where each customer occupies the server for the given service time and additional
 time corresponding to the inter-retrial time from the ``slowest'' orbit. 
The presented simulation results confirm the theoretical conclusions.}


\KWE{retrial queue; workload; stochastic ordering}  




\DOI{10.14357/19922264230408}{UOKQRD}

\vspace*{-12pt}

\Ack

\vspace*{-4pt}

\noindent
The research has been prepared with the support of the Russian Science Foundation according to
the research project No.\,21-71-10135. 



  \begin{multicols}{2}

\renewcommand{\bibname}{\protect\rmfamily References}
%\renewcommand{\bibname}{\large\protect\rm References}

{\small\frenchspacing
 {%\baselineskip=10.8pt
 \addcontentsline{toc}{section}{References}
 \begin{thebibliography}{99} 
%1
\bibitem{Ar1-1}
\Aue{Artalejo, J.\,R.} 1999. Accessible bibliography on retrial queues. \textit{Math.
Comput. Model.} 30(3-4):1--6. doi: 10.1016/S0895-7177(99)00128-4.
%2
\bibitem{Ar3-1}
\Aue{Artalejo, J., and A.~Gomez-Corral.} 2008. 
\textit{Retrial queueing systems: A computational approach}. Springer. 318~p.
doi: 10.1007/978-3-540-78725-9.
%3
\bibitem{F86-1} 
\Aue{Fayolle, G.} 1986. 
A simple telephone exchange with delayed feedbacks. \textit{Seminar (International) on Teletraffic Analysis and Computer Performance Evaluation Proceedings}.
Elseiver Science. 245--253.
%4
\bibitem{CSA92-1}
\Aue{Choi, B.\,D., Y.\,W.~Shin, and W.\,C.~Ahn.} 1992. 
Retrial queues with collision arising from unslotted \mbox{CSMA}/CD protocol. \textit{Queueing Syst.} 11:335--356.  
doi: 10.1007/ BF01163860.
%5
\bibitem{CRP93-1}
\Aue{Choi, B.\,D., K.\,H.~Rhee, and K.\,K.~Park.} 1993. 
The $M/G/1$ retrial queue with retrial rate control policy.
\textit{Probab. Eng. Inform. Sc.} 7(1):29--46.
doi: 10.1017/ S0269964800002771.
%6
\bibitem{BG92-1}
\Aue{Bertsekas, D., and R.~Gallager.} 2021.
\textit{Data networks}. Athena Scientific. 570~p.
%7
\bibitem{AY08-1}
\Aue{Avrachenkov, K., and U.~Yechiali.} 2008.
Retrial networks with finite buffers and their application to Internet data traffic. \textit{Probab. Eng. Inform. Sc.} 22(4):519--536.
doi: 10.1017/S0269964808000314.
%8
\bibitem{AY10-1} 
\Aue{Avrachenkov, K., and U.~Yechiali.} 2010.
On tandem blocking queues with a~common retrial queue. \textit{Comput. Oper. Res.} 37(7):1174--1180.
doi: 10.1016/j.cor.2009.10.004.

%9
\bibitem{Yaoetal02-1}
\Aue{Yao, S., F.~Xue, B.~Mukherjee, S.\,J.\,B.~Yoo, and S.~Dixit.} 2002.
Electrical ingress buffering and traffic aggregation for optical packet switching and their
effect on TCP-level performance in optical mesh networks.
\textit{IEEE Commun. Mag.} 40(9):66--72. doi: 10.1109/MCOM.2002.1031831.

%10
\bibitem{Wongetal09-1}
\Aue{Wong, E.\,W.\,M., L.\,L.\,H.~Andrew, T.~Cui, B.~Moran, A.~Zalesky, R.\,S.~Tucker, and M.~Zukerman.} 2009.
Towards a~bufferless optical internet.
\textit{J.~Lightwave Technol.} 27(14):2817--2833. doi: 10.1109/JLT.2009.2017211.

%11
\bibitem{mathematics2022-1}
\Aue{Morozov, E.\,V., I.\,V.~Peshkova, and A.\,S.~Rumyantsev.}
 2023. Bounds and maxima for the workload in a~multiclass orbit queue. \textit{Mathematics} 11(3):564. doi: 10.3390/ math11030564.

%12
\bibitem{pesh-mor2022-1} 
\Aue{Peshkova, I., and E.~Morozov.} 2022. On comparison of multiserver systems with multicomponent mixture distributions. 
\textit{J.~Math. Sci.} 267(2):260--272. doi: 10.1007/ s10958-022-06132-z.
%13
\bibitem{pesh2022-1}
\Aue{Peshkova, I.\,V.} 2022. Granitsy ekstremal'nogo in\-dek\-sa vre\-me\-ni ozhi\-da\-niya v~sis\-te\-me $M/G/1$ 
s~raspredeleniem vremeni obsluzhivaniya v~vide konechnoy
smesi [On bounds of the stationary waiting time extremal index in $M/G/1$
system with mixture service times]. \textit{Informatika i~ee Primeneniya~--- Inform. Appl.} 16(4):26--33. doi: 10.14357/19922264220405. EDN: VFKRKT.

%14
\bibitem{rego-1}
\Aue{Rego, V.} 1988. 
Some explicit formulas for mixed exponential service systems. 
\textit{Comput. Oper. Res.} 15(6):509--520. doi: 10.1016/0305-0548(88)90047-0.
%15
\bibitem{Morozov2019-1} 
\Aue{Morozov, E.\,V., A.\,S.~Rumyantsev, S.~Dey, and T.\,G.~Deepak.} 2019.
Performance analysis and stability of multiclass orbit queue with constant retrial rates and balking.
\textit{Perform. Evaluation} 134:102005. doi: 10.1016/ J.PEVA.2019.102005.
%16
\bibitem{Asmus-1}
\Aue{Asmussen, S.} 2003. \textit{Applied probability and queues. Stochastic modelling and 
applied probability.} New York, NY: Springer. 438~p.

%17
\bibitem{Ross-1}
\Aue{Ross, S., J.~Shanthikumar, and Z.~Zhu.}
 2005. On increasing-failure-rate random variables. \textit{J.~Appl. Probab.} 42(3):797--809. doi: 10.1239/jap/1127322028.
 
 %18
\bibitem{haji-1}
\Aue{Haji, R., and G.\,F.~Newell.} 1971. A~relation between stationary queue and waiting time distributions. \textit{J. Appl. Probab.} 8(3):617--620.
doi: 10.2307/3212186.

\end{thebibliography}

 }
 }

\end{multicols}

\vspace*{-6pt}

\hfill{\small\textit{Received August 26, 2023}} 

%\vspace*{-18pt}

\Contrl

\vspace*{-4pt}

\noindent
\textbf{Peshkova Irina V.} (b.\ 1975)~--- 
Candidate of Science (PhD) in physics and mathematics, associate professor, Petrozavodsk State University, 33~Lenina Pr., Petrozavodsk 185910, 
Russian Federation; senior scientist, Karelian Research Center of the Russian Academy of Sciences, 
11~Pushkinskaya Str., Petrozavodsk 185910, Russian Federation; \mbox{iaminova@petrsu.ru}


\label{end\stat}

\renewcommand{\bibname}{\protect\rm Литература}   %9
\include{shihievi} %10
\def\stat{agalarov}


\def\tit{ПРИБЛИЖЕННЫЙ МЕТОД ВЫЧИСЛЕНИЯ ХАРАКТЕРИСТИК УЗЛА 
ТЕЛЕКОММУНИКАЦИОННОЙ СЕТИ С~ПОВТОРНЫМИ ПЕРЕДАЧАМИ}
\def\titkol{Приближенный метод вычисления характеристик узла 
телекоммуникационной сети с~повторными передачами} 

\def\autkol{Я.\,М.~Агаларов}
\def\aut{Я.\,М.~Агаларов$^1$}

\titel{\tit}{\aut}{\autkol}{\titkol}

%{\renewcommand{\thefootnote}{\fnsymbol{footnote}}\footnotetext[1]
%{Работа выполнена при поддержке РФФИ, проекты 08--07--00152 и 08--01--00567.}}

\renewcommand{\thefootnote}{\arabic{footnote}}
\footnotetext[1]{Институт проблем
информатики Российской академии наук, agglar@yandex.ru}

%\vspace*{-6pt}


\Abst{Рассмотрена модель узла коммутации пакетов c повторными передачами для двух 
схем распределения буферной памяти: полнодоступной и полного разделения. Предложен 
приближенный метод вычисления интенсивностей потоков и вероятностей блокировок узла. 
Получены необходимые и достаточные условия существования и единственности решения 
уравнения для потоков в узле при установившемся режиме работы и доказана сходимость 
итерационного метода решения указанного уравнения.}

\KW{узел коммутации пакетов; буферная память; повторные передачи; вероятности 
блокировок; итерационный метод}

      \vskip 18pt plus 9pt minus 6pt

      \thispagestyle{headings}

      \begin{multicols}{2}

      \label{st\stat}


\section{Введение}

    Одной из основных задач предварительного анализа 
телекоммуникационных сетей коммутации пакетов с ограниченной буферной 
памятью является расчет характеристик потоков и вероятностей блокировок в 
узлах связи. Важность указанных характеристик определяется тем, что от их 
значений существенным образом зависят другие основные показатели сети 
(пропускная способность, задержки пакетов и~др.). 

    Существует множество различных моделей узлов коммутации пакетов и 
методов их расчета (см., например,~[1--6]). Для моделей, рассматривающих 
узел с ограниченной буферной памятью как систему массового обслуживания 
(CMO) типа 
$
\begin{matrix}
M \\ \lambda
\end{matrix}
\left |
\begin{matrix}
M \\ \lambda
\end{matrix}
\right |
\overline{m} \vert N
$ или  $\vert PH\vert PH\vert 1\vert r$, в предположении отсутствия повторных 
передач пакетов получены точные методы вычисления характеристик 
узлов~[1, 3, 4, 6]. Приближенные методы расчета узлов, учитывающие повторные 
попытки передачи, используют модели типа $\vert PH\vert PH\vert 1\vert r$ или 
$
\begin{matrix}
M \\ \lambda
\end{matrix}
\left |
\begin{matrix}
M \\ \lambda
\end{matrix}
\right |
1 \vert N
$ и являются 
итерационными~[2, 3, 5, 7]. Для моделей типа 
$BM\!AP\vert PH\vert 1$, $M\vert G\vert 1\vert r$ и $M\!AP\vert 
(PH,PH)\vert 1$ с повторными заявками получены точные методы вычисления 
характеристик (например, в работах~[8--10]), которые также могут быть 
использованы при расчете узлов.

    Ниже будут рассмотрены модели узла коммутации пакетов с повторными 
передачами для двух схем распределения буферной памяти: с 
полнодоступными буферами и с полным разделением буферной памяти. 
Предлагается приближенный метод расчета характеристик, который в качестве 
модели узла использует СМО типа $
\begin{matrix}
M \\ \lambda
\end{matrix}
\left |
\begin{matrix}
M \\ \lambda
\end{matrix}
\right |
\overline{m} \vert N
$ с повторными заявками. Доказаны утверждения о 
достаточных и необходимых условиях существования и единственности 
решения уравнения для вероятности блокировки в установившемся режиме 
работы и сходимости предлагаемого итерационного метода. 

\section{Модель узла}

    Математическая модель узла представляется в виде СМО с ограниченной 
буферной памятью и различными потоками заявок, каждая из которых требует 
обслуживания только на одной из многоканальных линий связи. 

    Пусть $0<N<\infty$~--- число мест хранения в буферной памяти, $u$~--- 
узел связи, $v$~--- линия связи, $\Omega_u^+$~--- множество исходящих из 
узла~$u$ линий, $c_v$~--- канальная емкость линии~$v$. Поток заявок, 
тре\-бу\-ющих обслуживания на линии~$v$, назовем $v$-по\-то\-ком, заявки этого 
потока~--- $v$-за\-яв\-ка\-ми.


    Пусть выполняются следующие предположения: 
\begin{enumerate}[1.]
\item Места в буферной памяти распределяются согласно одной из двух 
схем:
\begin{enumerate}[($i$)]
\item полнодоступная схема~--- каждое свободное место хранения доступно 
любой заявке;
\item схема полного разделения памяти~--- $v$-за\-яв\-кам доступны всего 
$N_v$ мест, где $\sum\limits_{v\in\Omega_u^+} N_v=N$.
\end{enumerate}
\item Если в момент поступления $v$-заявки в буферной памяти есть 
доступное свободное место, то она сразу занимает это место. Если в момент 
поступления $v$-заявки в системе нет свободного доступного места 
хранения, то поступившая заявка через некоторое время повторно поступает 
на систему, оставаясь $v$-заявкой. 
\item Интенсивности первичных потоков $v$-заявок~--- заданные величины 
$0<\Lambda_v<\infty$, $v\in \Omega_u^+$. Суммарные потоки первичных и 
повторных $v$-заявок являются независимыми в совокупности 
пуассоновскими потоками. Для обслуживания $v$-заявки требуется 
одновременно одно место хранения и один канал типа~$v$, $v\in 
\Omega_u^+$.
\item Первичные нагрузки~--- реализуемые, т.\,е.\ в данном случае 
интенсивности входных первичных потоков равны интенсивностям 
выходных потоков выполненных заявок. 
\item Принятые в СМО $v$-заявки обслуживаются линией~$v$ в порядке 
поступления. 
\item Время занятия канала $v$-заявкой~--- экспоненциально 
распределенная случайная величина с параметром $0<\mu_v<\infty$, 
$v\in\Omega_u^+$, независимая от других случайных событий в узле.
\item Выполненная $v$-заявка с вероятностью~$B_v$ повторяется через 
заданное время~$\tau_v$ (тайм-аут) и с вероятностью $1-B_v$ покидает 
систему через время~$t_v$ навсегда, сразу освободив занятый канал и место 
буферной памяти.
\end{enumerate}

   Будем говорить, что узел блокирован для $v$-за\-яв\-ки, если в буферной 
памяти отсутствует доступное место хранения. Ставится задача вычисления 
вероятностей блокировок и интенсивностей потоков в узле.

\section{Вычисление вероятности блокировки и~интенсивностей~потоков} 

   Пусть $\Lambda_v^*$~--- интенсивность суммарного потока внешних 
заявок, требующих передачи по линии~$v$, $\pi_v$~--- вероятность блокировки 
узла для заявок, требующих передачи по исходящей из узла линии~$v$. 

    Пусть в узле используется полнодоступная схема распределения 
буферной памяти. Тогда, как следует из описания модели, $\pi_v 
=\pi_{v^\prime},\,v,\,v^\prime\in \Omega_u^+$, и для 
интенсивностей~$\Lambda_v^*$, $v\in\Omega_u^+$, справедливы соотношения:
\begin{equation*}
\Lambda_v^* = \fr{\Lambda_v}{1-\pi}\,,
%\label{e1aga}
\end{equation*}
    где
    $\pi =\pi_v$, $v\in\Omega_u^+$.

    Пусть 
    $\overline{k} = \{\overline{k}_v$, $v\in\Omega_u^+\}$~--- состояние 
буферной памяти узла, $\overline{k}_v =\left ( k_v,\,k_v^\prime,\,k_v^{\prime\prime}\right )$; 
$k_v$~--- число $v$-заявок в буферной 
памяти, ожидающих выполнения линией~$v$; $k^\prime_v$~--- число 
$v$-заявок в буферной памяти, ожидающих тайм-аут и неуспешно переданных 
в последующий узел; $k_v^{\prime\prime}$~--- число $v$-за\-явок в буферной 
памяти, успешно переданных в последующий узел и ожидающих 
потверждения; 
$A_m = \left \{ \overline{k}:\ \sum\limits_{v\in\Omega_u^+} \left ( 
k_v+k_v^\prime + k_v^{\prime\prime}\right ) =m \right \}$~--- множество различных 
состояний, при которых в памяти узла занято ровно $m$~буферов. Тогда с 
учетом введенных выше обозначений и предположений для ве\-ро\-ят\-ности 
блокировки узла можно написать формулу~\cite{1aga, 2aga}:
\begin{equation}
\pi = \fr{1}{G_N}\sum\limits_{\overline{k}\in A_N} 
p\left (\overline{k},\overline{\rho}^*\right )\,,
\label{e2aga}
\end{equation}
где  
\begin{gather}
p(\overline{k},\overline{\rho}^*) = \prod\limits_{v\in\Omega_u^+} z_v (\pi, 
\rho_v , k_v , k_v^\prime , k_v^{\prime\prime})\,;\\
z_v (\pi, \rho_v , k_v , k_v^\prime , k_v^{\prime\prime}) ={}\notag\\
\!\!{}=
\begin{cases}
 \fr{\rho_v^{\prime *k_v^\prime}}{k_v^{\prime}!}\,
\fr{\rho_v^{\prime\prime * k_v^{\prime\prime}}}{ k_v^{\prime\prime}!}  \,
\fr{\rho_v^{*k_v}}{ k_{v}!} 
&\mbox{при}\ k_v<c_v\,,\\
 \fr{\rho_v^{\prime * k_v^\prime}}{k_v^{\prime}!} \,
\fr{\rho_v^{\prime\prime * k_v^{\prime\prime}}} { k_v^{\prime\prime}!} 
\fr{\rho_v^{*k_v}}{ c_{v}!c_v^{k_v- c_v}} 
& \mbox{при}\ k_v\geq c_v\,;
\end{cases}\\
G_N = \sum\limits_{m=0}^N\sum\limits_{\overline{k}\in A_m}
p(\overline{k},\overline{\rho}^*)\,;\\ 
\overline{\rho}^*=\{\rho_v^*,\,v\in\Omega_u^+\}\,;\\
\rho_v^* = \fr{\rho_v}{1-\pi}\,;\quad \rho_v =\fr{\Lambda_v}{\mu_v(1- B_v)}\,;\\
\rho_v^{\prime *} =\rho_v^*\mu_v\tau_vB_v\,;\quad \rho_v^{\prime\prime *}=
p_v^* \mu_vt_v,\,\quad  v\in \Omega_u^+\,.\label{e3aga}
\end{gather}

Переобозначив $1-\pi$ через $y$, выражение в правой части равенства~(2)~--- через 
$p_{\overline{k}}(\overline{\rho},y)$, выражение в правой части равенства~(4)~--- 
через $g_N(\overline{\rho},y)$, а выражение в правой 
части равенства~(1)~--- через $1-q_N (\overline{\rho},y)$, 
где $\overline{\rho} = (\rho_v,\,v\in \Omega_u^+)$, $\rho_v = \rho_v^*y\;=$\linebreak 
$=\;\Lambda_v/(\mu_v(1-B_v))$, $v\in\Omega_u^+$, получим нелинейное уравнение 
относительно неизвестной переменной~$y$:
\begin{equation}
y=q_N(\overline{\rho},y)\,.
\label{e4aga}
\end{equation}

    Решим уравнение~(8). Как следует из~(2)--(7), верно 
равенство
\begin{equation}
q_N(\overline{\rho},y) = \fr{g_{N-1}(\overline{\rho},y )}{g_N(\overline{\rho},y)}\,.
\label{e5aga}
\end{equation}
Введем функцию  $d_n(\overline{\rho} ,y)$ среднего числа заявок в узле с 
буферной памятью емкости $n\geq 0$:
$$
d_n(\overline{\rho} ,y) = 
\fr{1}{g_n(\overline{\rho},y)}\,\sum\limits_{m=0}^n m\sum\limits_{\overline{k}\in 
A_m} p_{\overline{k}}(\overline{\rho},y)\,.
$$
Заметим, что $g_n$, $d_n$ и $q_n$, 
$n\geq 0$,~--- непрерывно-дифференцируемые функции по $y\in (0,\,1]$. Взяв 
производную функции~$g_n$ по~$y$, из~(2)--(7) получим
\begin{multline}
\fr{\partial g_n(\overline{\rho},y)}{\partial y} ={}\\
{}= -\sum\limits_{m=0}^n m 
\sum\limits_{\overline{k}\in A_m}\fr{\prod\limits_{v\in\Omega_u^+} z_n 
(0,\rho_v, k_v, k_v^\prime , k_v^{\prime\prime})}{y^{m+1}}={}\\
{}= -\fr{1}{y}\,g_n (\overline{\rho},y)d_n(\overline{\rho},y)\,.
\label{e6aga}
\end{multline}
Взяв производную функции $q_N$ по $y$, из~(\ref{e5aga}) и~(\ref{e6aga}) 
получим
\begin{equation}
\fr{\partial q_N(\overline{\rho},y)}{\partial y} = \fr{q_N(\overline{\rho},y)}{y}\left 
[ d_N (\overline{\rho},y)-d_{N-1}(\overline{\rho},y)\right ]\,.
\label{e7aga}
\end{equation}
    Докажем несколько утверждений о свойствах 
функции~$q_N(\overline{\rho},y)$.
\medskip

\noindent
\textbf{Утверждение 1.} \textit{Справедливы неравенства}
\begin{multline}
0<d_{n+1}(\overline{\rho},y)-d_n(\overline{\rho},y) <1\,,\\
\ \ \ \ \ \ \ \ \ \ \ \ \ \ \ \ \ \ \ \ y\in (0,\,1]\,, \ n\geq 0\,.
\label{e8aga}
\end{multline}


\noindent

Д\,о\,к\,а\,з\,а\,т\,е\,л\,ь\,с\,т\,в\,о\,.\ Подставив выражение для функции 
$d_n(\overline{\rho},y)$ и проведя преобразования, получим
\begin{multline*}
d_{n+1}(\overline{\rho},y) -d_n(\overline{\rho},y) = 
\fr{\sum\limits_{m=0}^{n+1}m\sum\limits_{\overline{k}\in A_m} 
p_{\overline{k}}(\overline{\rho},y)}
{\sum\limits_{m=0}^{n+1}
\sum\limits_{\overline{k}\in A_m} p_{\overline{k}}(\overline{\rho},y)} - {}\\
{}-
\fr{\sum\limits_{m=0}^n m \sum\limits_{\overline{k}\in A_m} p_{\overline{k}} 
(\overline{\rho},y)}{\sum\limits_{m=0}^n
\sum\limits_{\overline{k}\in A_m}p_{\overline{k}}(\overline{\rho},y)}={}\\
{}=\fr{\sum\limits_{m=1}^n m \sum\limits_{\overline{k}\in 
A_m}p_{\overline{k}}(\overline{\rho},y)+(n+1)\sum\limits_{\overline{k}\in 
A_{n+1}}  p_{\overline{k}}(\overline{\rho},y)}{\sum\limits_{m=0}^n\sum\limits_{\overline{k
}\in A_m}p_{\overline{k}}(\overline{\rho},y)+\sum\limits_{\overline{k}\in 
A_{n+1}}p_{\overline{k}}(\overline{\rho},y)} -{}
\end{multline*}
\begin{multline}
{}-
\fr{\sum\limits_{m=0}^n m 
\sum\limits_{\overline{k}\in A_m}p_{\overline{k}}(\overline{\rho},y)}
{\sum\limits_{m=0}^n\sum\limits_{\overline{k}\in A_m} 
p_{\overline{k}}(\overline{\rho},y)}={}\\
{}=\fr{(n+1)\sum\limits_{\overline{k}\in 
A_{n+1}}p_{\overline{k}}(\overline{\rho},y)g_n(\overline{\rho},y)}{g_{n+1}(\overline{\rho},y) g_n(\overline{\rho},y)} -{}\\
{}-
\fr{\sum\limits_{\overline{k}\in 
A_{n+1}}p_{\overline{k}}(\overline{\rho},y)\sum\limits_{m=0}^n  m 
\sum\limits_{\overline{k}\in A_m} p_{\overline{k}}(\overline{\rho},y) }
{g_{n+1}(\overline{\rho},y) g_n(\overline{\rho},y)}
={}\\
{}=\left [ 1-q_{n+1}(\overline{\rho},y)\right ] \left [n+1-d_n(\overline{\rho},y)\right ]\,.
\label{e9aga}
\end{multline}


    Докажем утверждение~1 методом индукции. При $n = 0$, как следует 
из~(\ref{e9aga}), имеем
$$
d_2(\overline{\rho},y) - d_1 (\overline{\rho},y) =1-q_1(\overline{\rho},y)\,,
$$
    т.\,е.\ утверждение~1 при $n = 0$ справедливо. 

    Пусть неравенства~(\ref{e8aga}) справедливы для некоторого $n > 0$. 
Докажем, что они справедливы и для $n + 1$. Из~(\ref{e9aga}) получаем
\begin{multline*}
d_{n+1}(\overline{\rho},y)- d_n(\overline{\rho},y)={}\\
{}=\left [ 1-
q_{n+1}(\overline{\rho},y)\right ] \left [n+1-d_n(\overline{\rho},y)\right ] ={}\\
{}= \left [ 1-
1-q_{n+1}(\overline{\rho},y)\right ] \left [ n-{}\right.\\
{}-\left. d_{n-1}(\overline{\rho},y)+d_{n-1}(\overline{\rho},y)-
d_n(\overline{\rho},y)+1\right ] ={}\\
{}=\left [ 1-q_{n+1}(\overline{\rho},y)\right ] 
\left [ n-d_{n-1}(\overline{\rho},y)-{}\right.\\
{}-\left. \left ( d_n(\overline{\rho},y)-d_{n-1}(\overline{\rho},y)\right )+1\right] = {}\\
{}=
\left [ 1-q_{n+1}(\overline{\rho},y)\right ]
\left [ 
\fr{d_n(\overline{\rho},y) -d_{n-1}(\overline{\rho},y)}{1-
q_n(\overline{\rho},y)}\right.-{}\\
{}-\left.
\left ( d_n(\overline{\rho},y)-d_{n-1}(\overline{\rho},y)\right )+1
\vphantom{\fr{d_n(\overline{\rho})}{(q_n)}}
\right ]={}\\
{}=
\left [ 1-q_{n+1}(\overline{\rho},y)\right ]
\left [ 
\vphantom{\fr{d_n(\overline{\rho})}{(q_n)}}
\left ( d_n(\overline{\rho},y\right)\right. -{}\\
 {}-\left.
d_{n-1}\left(\overline{\rho},y)\right )\fr{q_n(\overline{\rho},y)}{1-
q_n(\overline{\rho},y)}+1\right ]\,.
\end{multline*}
Так как по предположению $d_n (\overline{\rho},y) -d_{n-1}(\overline{\rho},y) 
>0$, то правая часть последнего равенства больше нуля; следовательно, 
$d_{n+1}(\overline{\rho},y)-d_n(\overline{\rho},y)>0$. 

    Продолжив преобразование правой части последнего равенства и 
учитывая предположение $d_n(\overline{\rho},y) -d_{n-1}(\overline{\rho},y)<1$, 
получим
\begin{multline*}
d_{n+1}((\overline{\rho},y) -d_n(\overline{\rho},y)<{}\\
{}< \left [ 1-
q_{n+1}(\overline{\rho},y)\right ]
\left ( \fr{q_n(\overline{\rho},y)}{1-q_n(\overline{\rho},y)}+1\right )={}\\
{}=
\fr{1-q_{n+1}(\overline{\rho},y)}{1-q_n(\overline{\rho},y)}<1\,,
\end{multline*}
так как $0< q_n(\overline{\rho},y)<q_{n+1}(\overline{\rho},y)<1$, $n>0$, $y\in 
(0,\,1]$.

Следовательно, утверждение~1 доказано.

\medskip

\noindent
\textbf{Утверждение 2.} $q_N(\overline{\rho},y)$~--- \textit{монотонно-воз\-рас\-та\-ющая 
функция по $y\in (0,\,1]$. При этом $0< q_N(\overline{\rho},y)\;\leq $\linebreak 
$\leq\;q_N(\overline{\rho},1) <1$, $y\in (0,\,1]$,  и $\underset{y\rightarrow 
0}{\mathrm{lim}}\,q_N(\overline{\rho},y) =0$}.

\medskip

\noindent
Д\,о\,к\,а\,з\,а\,т\,е\,л\,ь\,с\,т\,в\,о\,.\  Возрастание функции 
$q_N(\overline{\rho},y)$ следует непосредственно из~(\ref{e7aga}) и 
утверж\-де\-ния~1. Доказательство неравенств в условии утверждения очевидно 
следует из~(\ref{e5aga}) и вида функции $g_n (\overline{\rho},y)$, $n\geq 0$. 
Для предела имеем:
\begin{multline*}
\underset{y\rightarrow 0}{\mathrm{lim}}\,q_N(\overline{\rho},y) 
=\underset{y\rightarrow 0}{\mathrm{lim}}\,\fr{g_{N- 1}(\overline{\rho},y)}{g_N(\overline{\rho},y)} = {}\\
{}= \underset{y\rightarrow 0}{\mathrm{lim}}\,\left (
g_{N-1}(\overline{\rho},y)\Bigg / \left ( 
\vphantom{\prod\limits_{v\in\Omega_u^+}}
g_{N-1}(\overline{\rho},y)\right.\right.+{}\\
{}+\left.\left.\sum\limits_{\overline{k}\in A_N}\prod\limits_{v\in\Omega_u^+} 
\fr{z_v(0,\rho_v,k_v,k^\prime_v,k^{\prime\prime}_v)}{y^N}\right )\right ) = {}\\
{}= \underset{y\rightarrow 0}{\mathrm{lim}}\,\left (
y^N g_{N-1}(\overline{\rho},y)\Bigg / 
\left ( 
\vphantom{\prod\limits_{v\in\Omega_u^+}}
y^N g_{N-1}(\overline{\rho},y)+{}\right.\right.\\
{}+\left.\left.\sum\limits_{\overline{k}\in A_N}
\prod\limits_{v\in\Omega_u^+} z_v(0,\rho_v,k_v,k_v^\prime , k_v^{\prime\prime}) 
\right ) \right )=0\,.
\end{multline*}
    
\medskip

\noindent
\textbf{Утверждение 3.} \textit{Производная функции~$q_N (\overline{\rho},y)$ по 
$y\in (0,\,1]$ удовлетворяет следующим соотношениям}:
\begin{align}
\underset{y\rightarrow 0}{\mathrm{lim}}\fr{\partial q_N(\overline{p},y)}
{\partial  y} &= \fr{\sum\limits_{\overline{k}\in A_{N-1}} 
p_{\overline{k}}(\overline{\rho},1)}{\sum\limits_{\overline{k}\in 
A_N}p_{\overline{k}}(\overline{\rho},1)}\,;\label{e10aga}\\
\fr{\partial q_N(\overline{\rho},y)}{\partial y}\Big |_{y=1}&<1\,.\label{e11aga}
\end{align}

\medskip

\noindent
Д\,о\,к\,а\,з\,а\,т\,е\,л\,ь\,с\,т\,в\,о\,.\ Проведя преобразования 
функции~$q_N(\overline{\rho},y)$, получим:
\begin{multline*}
\underset{y\rightarrow 0}{\mathrm{lim}}\fr{q_N(\overline{\rho},y)}{y} = {}\\
\!\!{}=
\underset{y\rightarrow 0}{\mathrm{lim}}
\fr{\sum\limits_{m=0}^{N-1}\sum\limits_{\overline{k}\in A_m}
\!\!\left (\prod\limits_{v\in\Omega_u^+}\!\! 
z_v(0,\rho_v,k_v,k_v^\prime , k_v^{\prime\prime})\right )\!\!\Bigg /\!\! y^m}
{y\sum\limits_{m=0}^{N}\sum\limits_{\overline{k}\in A_m}
\!\!\left(\prod\limits_{v\in\Omega_u^+}\!\! z_v\left (0,\rho_v,k_v,k_v^\prime , 
k_v^{\prime\prime}\right )\right )\!\!\Bigg /\!\!y^m} = \!\!\!
\end{multline*}
\begin{multline*}
\!\!\!\!\!\!{}=\underset{y\rightarrow 0}{\mathrm{lim}}\,
\fr{\sum\limits_{m=0}^{N-1}\sum\limits_{\overline{k}\in A_m}
y^{N-1-m}\prod\limits_{v\in\Omega_u^+} z_v(0,\rho_v,k_v,k_v^\prime , 
k_v^{\prime\prime})}{\sum\limits_{m=0}^{N}\sum\limits_{\overline{k}
\in A_m} y^{N-m}
\prod\limits_{v\in\Omega_u^+} z_v(0,\rho_v,k_v,k_v^\prime , 
k_v^{\prime\prime})}={}\!\\
{}=\fr{\sum\limits_{\overline{k}\in A_{N-1}} p_{\overline{k}}(\overline{\rho},1)}{ 
\sum\limits_{\overline{k}\in A_{N}} p_{\overline{k}}(\overline{\rho},1)}\,.
\end{multline*}
Очевидно, $\underset{y\rightarrow 0}{\mathrm{lim}} \,[d_N (\overline{\rho},y) -
d_{N-1} (\overline{\rho},y)]=1$, так как $\underset{y\rightarrow 
0}{\mathrm{lim}}\,d_n (\overline{\rho},y)=n$, $n>0$.

Следовательно, учитывая~(\ref{e7aga}), получаем~(\ref{e10aga}). 
Справедливость~(\ref{e11aga}) непосредственно следует из~(\ref{e7aga}) и 
утверждения~1.

\medskip

\noindent
\textbf{Утверждение 4.} \textit{Пусть $y^*\in (0,\,1]$~--- решение 
уравнения}~(\ref{e4aga}). \textit{Тогда}
\begin{equation*}
\fr{\partial q_N(\overline{\rho},y)}{\partial y}\Big |_{y=y^*}<1\,.
%\label{e12aga}
\end{equation*}

\medskip

\noindent
Д\,о\,к\,а\,з\,а\,т\,е\,л\,ь\,с\,т\,в\,о\,.\ \ Доказательство следует из~(\ref{e7aga}), 
так как $q_N(\overline{\rho},y^*)/y^* =1$.
\medskip

\noindent
\textbf{Утверждение 5.} \textit{Уравнение}~(\ref{e4aga}) \textit{имеет решение $y^*\in 
(0,\,1)$ тогда и только тогда, когда} 
\begin{equation}
\fr{\sum\limits_{\overline{k}\in A_{N-1}} p_{\overline{k}}(\overline{\rho},1)}{ 
\sum\limits_{\overline{k}\in A_{N}} p_{\overline{k}}(\overline{\rho},1)} >1\,.
\label{e13aga}
\end{equation}
\textit{Если уравнение}~(\ref{e4aga}) \textit{имеет решение $y^*\in (0,\,1)$, то оно 
единственное положительное решение}.
\medskip

\noindent
Д\,о\,к\,а\,з\,а\,т\,е\,л\,ь\,с\,т\,в\,о\,.\ Пусть выполняется 
неравенство~(\ref{e13aga}). Тогда, как следует из утверждения~3, 
$\underset{y\rightarrow 0}{\mathrm{lim}} (\partial q_N(\overline{\rho},y)/\partial y) 
>1$. Кроме того, как следует из утверждения~2, 
$\underset{y\rightarrow 0}{\mathrm{lim}} q_N(\overline{\rho},y)=0$. Тогда, так 
как $q_N(\overline{\rho},y)$~--- непрерывно-дифференцируемая функция по 
$y\in (0,\,1]$, существует значение $y^\prime \in (0,\,1)$ такое, что 
$q_N(\overline{\rho},y)>y$ для всех $y\in (0,\,y^\prime]$ (следует из теоремы о 
конечном приращении~\cite{11aga}). В то же время, согласно утверждению~2, 
$q_N(\overline{\rho},y)<y$ в окрестности точки $y=1$ (рис.~\ref{f1aga},\,\textit{а}). 
Следовательно, кривая $x=q_N(\overline{\rho},y)$ пересекает прямую $x=y$ 
хотя бы в одной точке $y=y^*\in (0,\,1)$, т.\,е.\ уравнение~(\ref{e4aga}) имеет 
хотя бы одно решение $y^*\in (0,\,1)$.

\begin{figure*}
\vspace*{1pt}
\begin{center}
\vspace*{1pt}
\mbox{%
\epsfxsize=158mm
\epsfbox{aga-1.eps}
}
\end{center}
\vspace*{-9pt}
\Caption{Примеры кривых $x=q_N(\overline{\rho},y)$ и $x=y$ (\textit{а})~при существовании решения 
уравнения~(\ref{e5aga}) и (\textit{б})~при выполнении условий~(17)
\label{f1aga}}
\vspace*{6pt}
\end{figure*}

Пусть уравнение~(\ref{e4aga}) имеет решение $y^*\in (0,\,1)$ и 
\begin{equation}
\fr{\sum\limits_{\overline{k}\in A_{N-1}}p_{\overline{k}}(\overline{\rho},1)}{ 
\sum\limits_{\overline{k}\in A_{N}}p_{\overline{k}}(\overline{\rho},1)}\leq 
1\,.\label{e14aga}
\end{equation}
Тогда из условий утверждений~2 и~3 следует, что 
уравнение~(\ref{e4aga}) в интервале $(0,\,1)$ имеет более одного решения, что 
может быть только при существовании решения $y^\prime \in (0,\,1)$ такого, 
что в окрестности точки $y=y^\prime$ выполняются неравенства: 
$q_N(\overline{\rho},y)<y$ при $y<y^\prime$ и $q_N(\overline{\rho},y)>y$ при 
$y>y^\prime$, где $y$ принадлежит указанной окрест\-ности точки~$y^\prime$ 
(рис.~\ref{f1aga},\,\textit{б}). Тогда в точке $y=y^\prime$ производная функции 
$q_N(\overline{\rho},y)$ по $y$ больше~1, что противоречит утверждению~4. 
Следовательно, неравенство~(\ref{e13aga}) справедливо.


Пусть уравнение~(\ref{e4aga}) имеет более одного положительного 
решения. Рассуждая точно так же, как и выше (в случае выполнения 
условий~(\ref{e14aga})), получим противоречие утверждению~4. 
Следовательно, утверждение~5 справедливо.
\medskip

\noindent
\textbf{Следствие.} \textit{Неравенства}
\begin{gather*}
\fr{\mu_v c_v (1-B_v)}{\Lambda_v}>1\,,\quad \fr{1-B_v}{\Lambda_v \tau_v B_v}>1\,,\\ 
\fr{1-B_v}{\Lambda_v t_v}>1\,,\ v\in\Omega_u^+\,,
\end{gather*}
\textit{являются необходимым условием существования решения 
уравнения}~(\ref{e4aga}).

\medskip
\noindent
Д\,о\,к\,а\,з\,а\,т\,е\,л\,ь\,с\,т\,в\,о\,.\ Пусть $\overline{k}_v$~--- это 
набор~$\overline{k}$, у которого $k_v=0$. Преобразовав левую 
часть~(\ref{e13aga}), получим

\noindent
\begin{multline*}
\fr{\sum\limits_{\overline{k}\in A_{N-1}} p_{\overline{k}} (\overline{\rho},1)}
{ \sum\limits_{\overline{k}\in A_{N}} 
 p_{\overline{k}}(\overline{\rho},1)} 
={}
\\
{}=
\fr{\sum\limits_{\overline{k}\in A_{N-1}}\prod\limits_{v\in \Omega_u^+} 
z_v\left(0,\rho_v,k_v,k_v^\prime , k_v^{\prime\prime}\right)}
{\sum\limits_{\overline{k}\in A_{N}}
\prod\limits_{v\in \Omega_u^+} z_v\left (0,\rho_v,k_v,k_v^\prime , k_v^{\prime\prime}\right )} \leq{}
\\
{}\leq
\left ( 
\vphantom{\prod\limits_{v^\prime\in\Omega_u^+\backslash v}}
\fr{\mu_v c_v(1-B_v)}{\Lambda_v}\right. \times{}\\
{}\times \sum\limits_{k_v=0}^{N-1}\sum\limits_{\overline{k}_v\in A_{N-1-k_v}} z_v\left(0,\rho_v,k_v+1,k_v^\prime , 
k_v^{\prime\prime}\right )\times{}\\
{}\times \left.\prod\limits_{v^\prime\in\Omega_u^+\backslash v} z_v^\prime 
\left(0,\rho_v,k_v,k_v^\prime , k_v^{\prime\prime}\right) \right)
\Bigg /{}\\
\Bigg / \left ( 
\vphantom{\prod\limits_{v^\prime\in\Omega_u^+\backslash v}}
\sum\limits_{k_v=0}^{N-1} \sum\limits_{\overline{k}_v\in A_{N-1-k_v}} z_v 
\left (0,\rho_v,k_v+1,k_v^\prime , 
k_v^{\prime\prime}\right )\right. \times{}\\
{}\times \prod\limits_{v^\prime\in\Omega_u^+\backslash v} 
z_{v^\prime}\left(0,\rho_v,k_v,k^\prime , k_v^{\prime\prime}\right)+{}\\
{}+
\sum\limits_{\overline{k}_v\in A_N} z_v\left (0,\rho_v, 0,k_v^\prime , 
k_v^{\prime\prime}\right)\times{}\\
\left.{}\times \prod\limits_{v^\prime\in\Omega_u^+\backslash v}z_{v^\prime} 
\left(0,\rho_v,k_v,k_v^\prime , k_v^{\prime\prime}\right )\right )\,.
\end{multline*}
Как следует из правой части последнего неравенства, если 
$\mu_v c_v (1-B_v)/\Lambda_v \leq 1$, то она меньше~1. Поэтому, чтобы 
выполнилось условие~(\ref{e13aga}), необходимо выполнение первого 
неравенства в условии следствия для каждого $v\in\Omega_u^+$. Точно так же 
доказывается необходимость выполнения второго и третьего неравенств в 
условии следствия.

    Пусть $y[n]$, $n\geq 0$, последовательность, полученная по формуле 
$y[n+1]=q_N(\overline{\rho},y[n])$, $y[0]=1$.

\medskip

\noindent
\textbf{Утверждение 6.} \textit{Пусть $y^*\in (0,\,1)$~--- решение 
уравнения}~(8). \textit{Тогда последовательность $y[n]$, $n\geq 0$, сходится 
к решению~$y^*$}.

\medskip

\noindent
Д\,о\,к\,а\,з\,а\,т\,е\,л\,ь\,с\,т\,в\,о\,.\ Отметим, что $y[1]<y[0]$ (это следует из 
утверждения~2, так как $y[0]=1$). Пусть для некоторого $n>1$ выполняется 
условие $y[n]<y[n-1]$. Тогда, как следует из утверждения~2, указанное условие 
выполняется и для $n+1$, т.\,е.\ по индукции следует, что последовательность 
$y[n]$, $n\geq 0$, монотонно убывает. 

    Пусть для некоторого $n>0$ $y[n]>y^*$ (существование такого $n$ 
следует из равенства $y[0]=1$). Тогда, как следует из утверждения~2, 
$y[n+1]\;=$\linebreak $=\;q_N(\overline{\rho},y[n])>q_N(\overline{\rho},y^*) =y^*$, т.\,е.\ 
последовательность ограничена снизу величиной~$y^*$. Значит, существует 
$\underset{n\rightarrow \infty}{\mathrm{lim}}\,y[n]=y^0\geq y^*$. Так как 
$q_n(\overline{\rho},y)$~--- непрерывная по~$y$ функция, то можно написать 
$\underset{n\rightarrow 
\infty}{\mathrm{lim}}\,q_N(\overline{\rho},y[n])=q_N(\overline{\rho},y^0)=y^0$, 
т.\,е.\ $y^0$~--- решение уравнения~(\ref{e4aga}). Из единственности 
положительного решения уравнения~(\ref{e4aga}) получаем $y^0=y^*$.

    Пусть в узле используется схема полного разделения буферной памяти. 
Тогда для интенсив\-ностей~$\Lambda_v^*$, $v\in\Omega_u^+$, справедливы 
соотношения:
$$
\Lambda_v^* = \fr{\Lambda_v}{1-\pi_v}\,,
$$
где $v\in\Omega_u^+$.


Фиксируем произвольную линию сети~$v$. Пусть $\overline{k}_v = (k_v, 
k_v^\prime, k_v^{\prime\prime})$~--- состояние буферной памяти линии~$v$; 
$k_v$, $k_v^\prime$, $k_v^{\prime\prime}$ определены выше. Тогда с 
учетом введенных ранее предположений и обозначений для вероятности 
блокировки линии справедлива формула~\cite{4aga}:
\begin{equation}
\pi_v = \fr{1}{G_{N_v}}\sum\limits_{k_v=N_v} 
z_v(\pi_v,\rho_v,\overline{k}_v)\,,
\label{e15aga}
\end{equation}
где 
\begin{multline*}
z_v(\pi_v, \rho_v, \overline{k}_v)={}\\
{}=
\begin{cases}
\fr{\rho_v^{\prime * k_v^\prime}}{k_v^\prime !}\,
 \fr{\rho_v^{\prime\prime * k_v^{\prime\prime}}}{k_v^{\prime\prime}!}\,
 \fr{\rho_v^{*k_v}}{k_v !} & \mbox{при}\ k_v<c_v\,,\\
 \fr{\rho_v^{\prime *k_v^\prime}}{k_v^{\prime }! }
 \fr{\rho_v^{\prime\prime * k_v^{\prime\prime}}}{k_v^{\prime\prime}!}
\fr{\rho_v^{*k_v}}{c_v !c_v^{k_v-c_v}} & \mbox{при}\ k_v\geq c_v\,;
\end{cases}
\end{multline*}
\begin{align*}
G_{N_v} &= \sum\limits_{m=0}^{N_v} z_v (\pi_v ,\rho_v , \overline{k}_v)\,;\\ 
\rho_v^*&=\fr{\rho_v}{1-\pi_v}\,;
\end{align*}
$\rho_v$, $\rho_v^{\prime *}$, 
$\rho_v^{\prime\prime *}$, $v\in\Omega_u^+$ определены выше.
    
Пусть $y_v=1-\pi_v$, а $q_{N_v} (\rho_v, y_v)$~--- выражение в правой 
части~(\ref{e15aga}). Тогда из равенств~(\ref{e15aga}), взяв~$y_v$ в качестве 
неизвестной переменной, получим систему независимых уравнений:
\begin{equation}
y_v = q_{N_v}(\rho_v, y_v)\,, \quad v\in \Omega_u^+\,.
\label{e16aga}
\end{equation}
    
    Заметим, что для фиксированной $v$ и заданных параметров $\Lambda_v$, 
$\mu_v$, $\tau_v$, $t_v$, $N_v$, $v\in\Omega_u^+$, уравнение в~(\ref{e16aga}) 
является частным случаем уравнения~(\ref{e4aga}) и совпадает с последним, 
когда число исходящих линий из узла равно~1. Следовательно, для схемы 
полного разделения памяти также справедливы все приведенные выше 
утверждения~1--6 и следствие. Заметим, что неравенство~(\ref{e13aga}) в 
условии утверждения~5 при $B_v=0$ и $t_v=0$ преобразуется в неравенство 
$\Lambda_v / (\mu_v c_v) >1$, $v\in\Omega_u^+$. Последовательность 
$\overline{y}[n]$, $n\geq 0$, в утверждении~6 определяется по формуле:
    \begin{gather*}
    \overline{y}[n] =\{y_v[n],\ v\in\Omega_u^+\}\,,\
    y_v[n+1]=q_{N_v} (\rho_v,\,y_v[n])\,,\\
    y_v[0] =1\,,\quad n\geq 0\,,\quad v\in \Omega_u^+\,.
    \end{gather*}


\section{Алгоритм расчета} %4

    Для вычисления интенсивностей потоков и вероятностей блокировок в 
узле предлагается следующий алгоритм, описывающий изложенную выше 
итерационную процедуру. Введем обозначения:
$y_u^v$~--- вероятность блокировки узла для заявок, направляемых на 
линию~$v$,
\begin{gather*}
y_u^v  = 
\begin{cases}
y_u & \mbox{для}\ v\in\Omega_u^+\ \mbox{при}\\
&\mbox{полнодоступной схеме},\\
y_v & \mbox{при схеме полного распределения}\\
&\mbox{памяти};
\end{cases}
\\
q_N^v(\overline{\rho}_u^{-v}, y_u^v)  = 
\begin{cases}
q_N(\overline{\rho},y) & \mbox{для}\ v\in\Omega_u^+\ \mbox{при пол-}\\ 
&\mbox{нодоступной схеме},\\
q_{N_v}(\rho_v, y_v) & \mbox{при схеме полного}\\
&\mbox{распределения}\\ 
&\mbox{памяти},  v\in\Omega_u^+\,.
\end{cases}
\end{gather*}



Тогда уравнения~(\ref{e4aga}) и~(\ref{e16aga}) записываются в виде:
$$
y_u^v = q_N^v (\overline{\rho}^v_u, y^v_u)\,,\quad v\in \Omega_u^+\,.
$$
Для значений, вычисляемых на $k$-м шаге алгоритма, к 
обозначениям соответствующих параметров приписывается знак~$[k]$.
\pagebreak

\textbf{Шаг 0.} 
\begin{enumerate}[1.]
\item  \textit{Инициализация}. Вычисление начальных значений 
параметров~$\rho_v$, $v\in\Omega_u^+$: $\Lambda_v[0]=\Lambda_v$, 
$\rho_v[0]=\Lambda_v[0]/(\mu_v(1-B_v))$, $y_u^v[0]=1$.
\item \textit{Проверка условий существования решения}. Если для некоторой 
линии $v\in\Omega_u^+$ выполняется хотя бы одно неравенство $(c_v\mu_v(1-
B_v))/\Lambda_v[0]\;\leq$\linebreak $\leq\;1$, или $(1-B_v)/(\Lambda_v\tau_v B_v) \leq 1$, или 
$(t_v(1\;-$\linebreak $-\;B_v))/\Lambda_v[0] \leq 1$, то алгоритм заканчивает работу с 
результатом <<нагрузка не реализуема>>. Если в узле используется 
полнодоступная схема и $(c_v\mu_v(1-B_v))/\Lambda_v[0] > 1$, $(1-
B_v)/(\Lambda_v\tau_v B_v)\;>$\linebreak $>\;1$, $(t_v(1-B_v))/\Lambda_v[0] > 1$ для всех 
$v\in\Omega_u^+$, то проверяется условие~(\ref{e13aga}) для $\Lambda_v =
\Lambda_v[0]$, $v\in\Omega_u^+$, и при невыполнении этого условия алгоритм 
заканчивает работу с результатом <<нагрузка не реализуема>>.
\end{enumerate}

    При вычислении левой части неравенства~(\ref{e13aga}) рекомендуется 
использовать метод свертки Базена (см.~\cite{12aga}), позволяющий 
производить рекуррентные вычисления (подробно этот метод описан также 
в~[1, 3--6]).



\medskip
\textbf{Шаг~$k$} ($k > 0$):
\begin{enumerate}[1.]
\item \textit{Вычисление вероятностей блокировок}. Используя значения 
параметров $\overline{\rho}_u^v[k-1]$, $y_u^v[k-1]$, $v\in\Omega_u^+$, 
вычисление с помощью формул~(1)--(7) значений 
вероятностей $y[k]=1- \pi [k]$~--- в случае полнодоступной памяти, или 
$y_v[k]=1- \pi_v[k]$, $v\in\Omega_u^+$, с помощью формул~(\ref{e15aga})~--- в 
случае полного разделения памяти. При вычислении этих значений 
рекомендуется использовать метод свертки Базена.
    \item \textit{Проверка условий останова алгоритма}. Если хотя бы для 
одной $v\in\Omega_u^+$ для заданного значения точности   выполняется 
условие
$$
\fr{\vert \Lambda_v^*[k]-\Lambda_v^*[k-1]\vert}{\Lambda_v^*[k]}> \varepsilon\,,
$$
то вычисление параметров $\overline{\rho}_u^v[k]$, $v\in\Omega_u^+$, и 
переход к шагу~$k$, положив $k$ равным $k+1$, иначе алгоритм завершает 
работу. 
\end{enumerate}

    По завершении алгоритма либо выявится, что нагрузка в системе не 
реализуема, либо будут вычислены интенсивности потоков, поступающих на 
линии узла, и стационарные вероятности блокировок для заявок каждого типа. 
    
\section{Примеры расчета}

    Для проверки точности вычисления результатов с помощью 
предложенного выше алгоритма и приемлемости введенных предположений 
были проведены вычислительные эксперименты с использованием 
аналитических и имитационных моделей. Во всех рассмотренных ниже 
примерах потоки внешних заявок считаются пуассоновскими. 
В~табл.~1 приведены значения вероятности блокировок вновь 
поступивших извне заявок, полученные на основании точной формулы, 
приведенной в~\cite{4aga} для СМО типа $M\vert M\vert 1\vert 0$ с повторными 
заявками при экспоненциальном распределении интервала времени между 
повторными попытками (первая строка таблицы), алгоритма из подраздела~5 
настоящей статьи (вторая строка) и имитационной модели при постоянном 
интервале времени между повторными попытками, равном~10 (третья строка). 
Расчет табл.~1 проведен для узла с одной исходящей одноканальной 
линией при интенсивности первичного потока $\Lambda =1$ и емкости 
накопителя $N_v=1$. Таблицы~2 и~3 вычислены с помощью 
алгоритма из подраздела~5 и имитационной модели соответственно при одной 
исходящей линии с числом каналов~10.


    В табл.~\ref{t4aga} и~\ref{t5aga} приведены значения вероятности 
блокировки узла с тремя исходящими линиями канальной емкости~10 каждая 
при $\mu_v =0{,}2$, $v\in\Omega_u^+$,  вычисленные с помощью алгоритма из 
подраздела~5 и имитационной модели с интервалом повторной попытки, 
равным~10, соответственно. В табл.~\ref{t4aga} и~\ref{t5aga} знак <<--->> в 
ячейках означает, что предложенная нагрузка $\Lambda_v$, $v\in\Omega_u^+$, 
не реализуема.



В табл.~\ref{t6aga} отражены вероятности блокировки такого же узла с 
накопителем $N = 35$ при экспоненциальном распределении интервала 
времени между повторными попытками со средним значением~$\tau$. 


Результаты вычислительного эксперимента показывают, что с  увеличением 
длины интервала между повторными попытками  вероятность блокировки 
увеличивается и приближается к значению,\linebreak
вычисленному с помощью 
алгоритма из подраздела~5 (см.\ табл.~\ref{t4aga} и~\ref{t6aga}), т.\,е.\ при 
пуассоновском внешнем потоке заявок предположение, что суммарный 
входной поток заявок  является пуассоновским, вполне приемлемо для 
предварительного анализа характеристик узла (например, при  $\tau c_v\mu_v > 
10$). Как показывают табл.~1--3, вероятность блокировки 
узла существенно зависит от\linebreak 

\vspace*{6pt}
\noindent
%\begin{table*}\small %tabl1
{\small
{{\tablename~1}\ \ \small{Вероятности блокировок при одной исходящей одноканальной линии}}
%\label{t1aga}}
\vspace*{-3pt}

\begin{center}
{\tabcolsep=7.3pt
\begin{tabular}{|c|c|c|c|c|c|}
\hline
&\multicolumn{5}{c|}{$\mu$}\\
\cline{2-6}
\multicolumn{1}{|c|}{\raisebox{4pt}[0pt][0pt]{№}}&1{,}1&1{,}2&2&3&4\\
\hline
1&0,9091&0,8333&0,5000&0,3333&0,2500\\
2&0,9091&0,8333&0,5000&0,3333&0,2500\\
3&0,8867&0,8452&0,4944&0,3167&0,2396\\
\hline
\end{tabular}}
\end{center}
%\vspace*{-6pt}
%\end{table*}
}
%\bigskip
%\medskip
\addtocounter{table}{1}
\pagebreak

\end{multicols}

\renewcommand{\figurename}{\protect\bf Таблица}
%\renewcommand{\tablename}{\protect\bf Рис.}
\begin{figure*}
{\small
\begin{minipage}[t]{76mm}
%\begin{table*}\small %tabl2
\begin{center}
\Caption{Вероятности блокировок при одной исходящей многоканальной линии ($\varepsilon 
=0{,}0001$)
\label{t2aga}}
\vspace*{2ex}

\tabcolsep=6.5pt
\begin{tabular}{|c|c|c|c|c|c|}
\hline
&\multicolumn{5}{c|}{$\mu$}\\
\cline{2-6}
\multicolumn{1}{|c|}{\raisebox{4pt}[0pt][0pt]{$N$}}&0{,}11&0{,}12&0{,}2&0{,}3&0{,}4\\
\hline
10&0,4845&0,2935&0,0204&0,0017&0,0002\\
15&0,1181&0,0545&0,0006&0,0000&0,0000\\
20&0,0489&0,0167&0,0000&0,0000&0,0000\\
\hline
\end{tabular}
\end{center}
%\end{table*}
\end{minipage}
\hfill
\begin{minipage}[t]{76mm}
%\begin{table*}\small %tabl3
\begin{center}
\Caption{Вероятности блокировок при одной исходящей линии
\label{t3aga}}
\vspace*{2ex}

\tabcolsep=6.5pt
\begin{tabular}{|c|c|c|c|c|c|}
\hline
&\multicolumn{5}{c|}{$\mu_v$}\\
\cline{2-6}
\multicolumn{1}{|c|}{\raisebox{4pt}[0pt][0pt]{$N$}}&0{,}11&0{,}12&0{,}2&0{,}3&0{,}4\\
\hline
10&0,5247&0,3238&0,0219&0,0019&0,0001\\
15&0,1726&0,0912&0,0004&0,0001&0,0000\\
20&0,1180&0,0371&0,0000&0,0000&0,0000\\
\hline
\end{tabular}
\end{center}
%\end{table*}
\end{minipage}
}
\vspace*{6pt}
\end{figure*}

\renewcommand{\figurename}{\protect\bf Рис.}
\renewcommand{\tablename}{\protect\bf Таблица}
\addtocounter{table}{2}

\begin{table}\small %tabl4
\begin{center}
\parbox{400pt}{\Caption{Вероятности блокировок при трех исходящих линиях, вычисленные алгоритмом из 
подраздела~5 ($\varepsilon =0{,}0001$)
\label{t4aga}}
}

\vspace*{2ex}

\tabcolsep=8pt
\begin{tabular}{|c|c|c|c|c|c|c|c|c|c|}
\hline
&\multicolumn{9}{c|}{$\Lambda_v$}\\
\cline{2-10}
\multicolumn{1}{|c|}{\raisebox{4pt}[0pt][0pt]{$N$}}&1&1{,}1&1{,}2&1{,}3&1{,}4&1{,}5&1{,}6&1{,}7&1{,}8\\
\hline
20&0,0677&0,1423&0,2975&0,7653&---&---&---&---&---\\
25&0,0065&0,0173&0,0394&0,0827&0.1690&0.3827&---&---&---\\
30&0,0005&0,0019&0,0059&0,0155&0.0361&0.0790&0.1792&0,7259&---\\
35&0,0000&0,0002&0,0008&0,0030&0,0089&0,0234&0,0574&0,1505&---\\
40&0,0000&0,0000&0,0001&0,0005&0,0022&0,0075&0,0220&0,0617&0,2449\\
\hline
\end{tabular}
\end{center}
%\end{table}
\vspace*{6pt}
%\begin{table}\small %tabl5
\begin{center}
\parbox{400pt}{\Caption{Вероятности блокировок при трех исходящих линиях, вычисленные с помощью 
имитационной модели
\label{t5aga}}
}

\vspace*{2ex}

\tabcolsep=8pt
\begin{tabular}{|c|c|c|c|c|c|c|c|c|c|}
\hline
&\multicolumn{9}{c|}{$\Lambda_v$}\\
\cline{2-10}
\multicolumn{1}{|c|}{\raisebox{4pt}[0pt][0pt]{$N$}}&1&1{,}1&1{,}2&1{,}3&1{,}4&1{,}5&1{,}6&1{,}7&1{,}8\\
\hline
20&0,0786&0,1695&0,3549&0,7056&---&---&---&---&---\\
25&0,0069&0,0190&0,0441&0,0998&0,2266&0,4583&---&---&---\\
30&0,0007&0,0024&0,0075&0,0184&0,0462&0,1025&0,2380&0,6931&---\\
35&0,0000&0,0003&0,0007&0,0040&0,0129&0,0307&0,0890&0,2981&---\\
40&0,0000&0,0000&0,0000&0,0011&0,0041&0,0095&0,0346&0,0790&0,3179\\
\hline
\end{tabular}
\end{center}
%\end{table}
\vspace*{6pt}
%\begin{table}\small %tabl6
\begin{center}
\parbox{356pt}{\Caption{Зависимость вероятности блокировки при трех исходящих линиях, вы\-чис\-лен\-ные с 
помощью имитационной модели со случайным интервалом между повторными попытками
\label{t6aga}}
}

\vspace*{2ex}

\tabcolsep=8pt
\begin{tabular}{|c|c|c|c|c|c|c|c|c|}
\hline
&\multicolumn{8}{c|}{$\Lambda_v$}\\
\cline{2-9}
\multicolumn{1}{|c|}{\raisebox{6pt}[0pt][0pt]{$\tau$}}&1&1{,}1&1{,}2&1{,}3&1{,}4&1{,}5&1{,}6&1{,}7\\
\hline
\hphantom{9}1&0.0001&0,0001&0,0017&0,0063&0,0210&0,0733&0,1996&0,4222\\
\hphantom{9}5&0.0000&0,0002&0,0016&0,0036&0,0446&0,0159&0,1360&0,3273\\
10&0.0000&0,0002&0,0011&0,0036&0,0101&0,0430&0,0818&0,2774\\
20&0.0000&0,0003&0,0007&0,0029&0,0089&0,0257&0,0863&0,2045\\
     \hline
\end{tabular}
\end{center}
\end{table}


\begin{multicols}{2}


\noindent
числа каналов в линии при равной суммарной 
производительности. Кроме того, как видно из табл.~\ref{t5aga} и~\ref{t6aga}, 
вероятность блокировки в большей степени зависит от среднего значения 
длины интервала между повторными попытками передачи, чем от закона 
распределения длины интервала. Таким образом, предложенный в работе 
алгоритм позволяет вы\-чис\-лить с достаточной точностью вероятность 
блокировки узла, интенсивности повторных передач и предельную величину 
реализуемой нагрузки. Отметим, что полученные в данной статье результаты 
могут быть использованы для расчета нагрузок в телекоммуникационной сети с 
повторами заявок в предыдущем узле или из источника. 


{\small\frenchspacing
{%\baselineskip=10.8pt
\addcontentsline{toc}{section}{Литература}
\begin{thebibliography}{99}    
\bibitem{1aga}
\Au{Kamoun~F., Kleinrock~L.}
Analysis of shared finite storage in a computer networks node environment under 
general traffic conditions~// IEEE Trans. on Commun., 1980. Vol.~28. No.\,7. 
P.~992--1003.

\bibitem{6aga} %2
\Au{Агаларов~Я.\,М., Шоргин~С.\,Я.}
Рекуррентный метод вычисления параметров сетей связи~// Техника средств 
связи, 1986. Сер. <<Системы связи>>. Вып.~6. С.~42--46.

\bibitem{3aga}
\Au{Башарин Г.\,П., Бочаров~П.\,П., Коган~Я.\,А.}
Анализ очередей в вычислительных сетях.~--- М.: Наука, 1989. 

\bibitem{4aga}
\Au{Бочаров~П.\,П., Печинкин~А.\,В.}
Теория массового обслуживания.~--- М.: Изд-во РУДН, 1995. 

\bibitem{5aga}
\Au{Вишневский~В.\,М.} 
Теоретические основы проектирования компьютерных сетей.~--- М.: 
Техносфера, 2003. 

\bibitem{2aga} %6
\Au{Башарин Г.\,П.}
Лекции по математической теории телетрафика.~--- М.: Изд-во РУДН, 2007. 

\bibitem{7aga}
\Au{Таранцев~А.\,А.}
Инженерные методы теории массового обслуживания.~--- М.: Наука, 2007.

\bibitem{9aga} %8
\Au{D'Apice~C., De~Simone~T., Manzo~R., Rizelian~G.}
$M\vert G\vert 1\vert r$ retrial queueing system with priority service of primary 
customers and a customers-searching server~// Distributed Computer and 
Communication Networks. Stochastic Modelling and Optimization.~--- М.: 
Техносфера, 2003. P.~106--117.

\bibitem{8aga} %9
\Au{Klimenok~V.\,I., Kim~C.\,S.}
$BM\!AP$/$PH$/1 retrial system operating in random environment~// Proceedings of 
the 5th St.-Petersburg Workshop on Simulation, St.-Petersburg, June~26\,--\,July~2, 
2005.~--- St.-Petersburg: NII Chemistry St.-Petersburg University Publs., 
2005. P.~367--372.   

\bibitem{10aga}
\Au{Krishnamoorthy~A., Babu~S.}
$M\!AP\vert (PH,PH)/c$ retrial queue with selegeneration of priorities 
and non-preemptive service~// Proceedings of the 14th International Conference on 
Analytical and Stochastic Modeling Techniques and Applications, June~4--6, 
2007. Prague, Czech Republic.~--- Sbr.-Dudweiler: Digitaldruck Pirrot GmbH, 
2007. P.~70--74.

\bibitem{11aga}
\Au{Корн~Г., Корн~Т.}
Справочник по математике.~--- М.: Наука, 1974.

\label{end\stat}


\bibitem{12aga}
\Au{Buzen~J.\,P.}
Computational algorithm for closed queuing networks with exponential servers~// 
Communications ACM, 1973. Vol.~16. No.\,9. P.~527--531.
 \end{thebibliography}
}
}
\end{multicols}
 
 
  %11
\def\stat{dukova}

\def\tit{О ПОИСКЕ МАКСИМАЛЬНЫХ ЧАСТЫХ И~МИНИМАЛЬНЫХ НЕЧАСТЫХ НАБОРОВ ПРОИЗВЕДЕНИЯ ЧАСТИЧНЫХ ПОРЯДКОВ}

\def\titkol{О поиске максимальных частых и~минимальных нечастых наборов произведения частичных порядков}

\def\aut{Н.\,А.~Драгунов$^1$, Е.\,В.~Дюкова$^2$}

\def\autkol{Н.\,А.~Драгунов, Е.\,В.~Дюкова}

\titel{\tit}{\aut}{\autkol}{\titkol}

\index{Драгунов Н.\,А.}
\index{Дюкова Е.\,В.}
\index{Dragunov N.\,A.}
\index{Djukova E.\,V.}


%{\renewcommand{\thefootnote}{\fnsymbol{footnote}} \footnotetext[1]
%{Работа выполнена при поддержке Министерства науки и~высшего образования Российской Федерации (проект 
%075-15-2020-799).}}


\renewcommand{\thefootnote}{\arabic{footnote}}
\footnotetext[1]{Федеральный исследовательский центр <<Информатика 
и~управ\-ле\-ние>> Российской академии наук, \mbox{nikitadragunovjob@gmail.com}}
\footnotetext[2]{Федеральный исследовательский центр <<Информатика и~управ\-ле\-ние>> 
Российской академии наук, \mbox{edjukova@mail.ru}}

\vspace*{-3pt}




\Abst{Исследованы актуальные вопросы снижения временных затрат, возникающие при 
логическом анализе данных с~элементами из декартова произведения конечных час\-тич\-но 
упорядоченных множеств. Для задачи поиска по базе транзакций максимальных час\-тых и~минимальных 
нечастых наборов произведения час\-тич\-ных порядков предложен оригинальный метод, 
основанный на решении слож\-ной дискретной задачи, называемой дуализацией 
над произведением час\-тич\-ных порядков. Метод представляет собой синтез двух других 
известных методов, один из которых достаточно очевиден, а~другой использует идею 
инкрементального пе\-ре\-чис\-ле\-ния искомых наборов и~поэтому пред\-став\-ля\-ет 
в~основном тео\-ре\-ти\-че\-ский интерес. Проведено экспериментальное исследование предложенного 
подхода к~решению рас\-смат\-ри\-ва\-емой задачи в~случае произведения конечных цепей,
 выявлены условия его эф\-фек\-тив\-ности и~для проводимого анализа данных показана 
 це\-ле\-со\-об\-раз\-ность применения асимптотически оптимальных алгоритмов дуализации 
 над произведением час\-тич\-ных порядков.}

\KW{максимальные час\-тые наборы; минимальные не\-час\-тые наборы; дуализация над 
произведением час\-тич\-ных порядков; асимп\-то\-ти\-чески оптимальный алгоритм дуализации}

\DOI{10.14357/19922264220112}
  
%\vspace*{-4pt}


\vskip 10pt plus 9pt minus 6pt

\thispagestyle{headings}

\begin{multicols}{2}

\label{st\stat}

    \section{Введение}
    
    Рас\-смат\-ри\-ва\-емая задача анализа данных занимает важ\-ное мес\-то в~об\-ласти 
    информационного поиска и~в~случае бинарных данных ставится сле\-ду\-ющим образом~\cite{4}.
    
    Дано некоторое множество элементов~$V$. Подмножества $X \hm\subseteq V$ называются наборами. Пусть~$D$~--- 
    база данных, содержащая некоторые, не обязательно различные, наборы. Наборы, 
    содержащиеся в~$D$, называются транз\-ак\-ци\-ями. Под частотой набора~$\nu(X)$ понимается доля транз\-ак\-ций в~$D$, 
    содержащих~$X$. Если $\nu(X) \hm\geq s$, $s \hm\in \left[0, 1\right]$, то набор~$X$ называется $s$-час\-тым, 
    иначе он называется $s$-не\-час\-тым. Если набор частый и~он не содержится ни в~каком другом 
    час\-том наборе, то такой набор называется максимальным час\-тым. Если набор не\-час\-тый 
    и~при этом он не содержит в~себе никакого другого не\-час\-то\-го набора, то такой набор 
    называется минимальным нечастым. Требуется найти все максимальные час\-тые и~минимальные не\-час\-тые 
    наборы при заданном~$s$.
    
    Рас\-смат\-ри\-ва\-емая задача имеет много важных приложений, одним из которых является 
    нахождение ассоциативных правил в~базах данных. В~случае бинарных данных ассоциативное правило~---
     это упорядоченная пара $ \left( X, Y \right)$ непересекающихся подмножеств множества~$V$, обо\-зна\-ча\-емая 
     $X \hm\Rightarrow Y$. Поддержкой правила $X \hm\Rightarrow Y$ называется час\-то\-та набора $Z\hm = X \cup Y$.
      Достоверностью правила $X\hm \Rightarrow Y$ называется доля транзакций, со\-дер\-жа\-щих~$Y$, 
      среди всех транзакций, содержащих~$X$. Требуется \mbox{найти} все ассоциативные правила, 
      удовле\-тво\-ря\-ющие заданным минимальной поддержке $s\hm \in [0, 1]$ и~минимальной 
      достоверности $c \hm\in [0, 1]$.  Впервые задача нахождения ассоциативных правил
       была поставлена в~\cite{1}, где она формулировалась как задача анализа по\-тре\-би\-тель\-ской корзины.

    В случае небинарных данных каждый элемент из~$V$ имеет некоторое множество чис\-ло\-вых значений 
    и~вместо наборов элементов рас\-смат\-ри\-ва\-ют\-ся наборы их значений.

    Поиск ассоциативных правил осуществляется в~два этапа. 
    На первом этапе находятся частые наборы, на втором этапе из найденных час\-тых 
    наборов формируются ассоциативные правила. При формировании правил на втором 
    этапе фактически возникает задача поиска $t$-не\-час\-тых наборов, где $t\hm > s/c$.
    
    С ростом размерности современных баз данных находить все час\-тые и~не\-час\-тые 
    наборы становится неэффективно как по времени, так и~по памяти в~силу 
    экспоненциального рос\-та чис\-ла таких наборов. Одно из решений данной проблемы 
    заключается в~поиске только максимальных час\-тых наборов и~только минимальных 
    нечастых наборов, что позволяет компактно хранить информацию о~всех час\-тых и~не\-час\-тых 
    наборах соответственно. 
    
    
    В~\cite{9} рас\-смот\-ре\-на задача поиска множеств максимальных час\-тых наборов~$X_{\max}$ 
    и~минимальных не\-час\-тых наборов~$Y_{\min}$ в~данных, пред\-став\-лен\-ных в~виде декартова 
    произведения час\-тич\-но упорядоченных множеств. Показано, что в~этом случае 
    при построении тре\-бу\-емых наборов возникают соответственно задача поиска 
    максимальных независимых элементов час\-тич\-ных порядков и~задача поиска минимальных 
    независимых элементов час\-тич\-ных порядков.  Каж\-дая из этих задач называется 
    дуализацией над произведением час\-тич\-ных порядков~\cite{8}. Обе задачи относятся к~одним 
    из цент\-раль\-ных труд\-но\-ре\-ша\-емых пе\-ре\-чис\-ли\-тель\-ных задач дис\-крет\-ной математики.
    
    Существует достаточно очевидный способ поиска максимальных час\-тых и~минимальных
     не\-час\-тых наборов произведения час\-тич\-ных порядков, основанный на по\-сле\-до\-ва\-тель\-ном 
     по\-стро\-ении указанных множеств. Одно из множеств ищется, например, алгоритмом Apriori~\cite{2},
      второе множество получается путем дуализации первого. 
      В~настоящей работе показано, что метод эффективен только в~случае, когда чис\-ло час\-тых 
      наборов существенно меньше или, наоборот, существенно больше чис\-ла не\-час\-тых наборов. 
      В~\cite{9} предложена идея со\-вмест\-но\-го пе\-ре\-чис\-ле\-ния~$X_{\max}$ и~$Y_{\min}$ с~использованием
       инкрементального алгоритма дуализации из~\cite{14}, которая автором экспериментально 
       не исследована.
    
    Основной результат настоящей работы~--- разработка нового подхода к~решению 
    поставленной задачи, который является синтезом последовательного и~совместного подходов. 
    
    Экспериментальные исследования, проведенные в~настоящей работе для случая
     произведения цепей, свидетельствуют о~том, что предложенный по\-сле\-до\-ва\-тель\-но-со\-вмест\-ный 
     метод наиболее эффективен в~случае, когда мощ\-ность множества час\-тых наборов примерно 
     равна мощ\-ности множества не\-час\-тых наборов.
     
     \vspace*{-6pt}
     
    
    \section{Постановка задачи поиска максимальных частых 
    и~минимальных нечастых наборов произведения частичных порядков}
    
         \vspace*{-2pt}
    
    Пусть $\mathcal{P} = \mathcal{P}_1 \times \dots \times \mathcal{P}_n$~--- 
    де\-кар\-то\-во произведение час\-тич\-но упорядоченных множеств. Элементы~$\mathcal{P}$ называются наборами. 
    На множестве~$\mathcal{P}$ определяется отношение частичного порядка~$\preceq$ сле\-ду\-ющим образом: 
    если $p \hm= (p_1, \dots, p_n) \hm\in \mathcal{P}$ и~$q \hm= (q_1, \dots, q_n)\hm \in \mathcal{P}$, 
    то $ p \hm\preceq q$ в~$ \mathcal{P}\hm \Leftrightarrow p_1 \hm\preceq q_1$ 
    в~$\mathcal{P}_1, \dots, p_n \hm\preceq q_n$ в~$ \mathcal{P}_n$.
    
    Пусть $\mathcal{D} (\mathcal{P})$~--- некоторая со\-во\-куп\-ность
     наборов из~$\mathcal{P}$, называемая базой данных. Наборы, на\-хо\-дя\-щи\-еся в~базе 
     данных $\mathcal{D} (\mathcal{P})$, необязательно по\-пар\-но раз\-лич\-ны и~называются транзакциями. 
     
    Введем обозначения: 
    $\vert \mathcal{D} (\mathcal{P}) \vert$~--- чис\-ло транз\-ак\-ций в~$\mathcal{D} (\mathcal{P})$; 
    $\mathcal{S}_\mathcal{D}(p)$~--- число транз\-ак\-ций в~$\mathcal{D} (\mathcal{P})$, 
    сле\-ду\-ющих за $p \hm\in \mathcal{P}$; $s \hm\in [0, 1]$. 
    
    \smallskip
    
    \noindent
    \textbf{Определение~1.}\
     Набор $p \in \mathcal{P}$ называется $s$-час\-тым, 
     если $\mathcal{S}_\mathcal{D}(p) / \vert \mathcal{D} (\mathcal{P}) \vert \hm\geq s$. Иначе набор~$p$ 
     называется $s$-не\-час\-тым.
    
    \smallskip
    
    \noindent
    \textbf{Определение~2.}\
    Набор $p \in \mathcal{P}$ называется максимальным $s$-час\-тым, если 
    он $s$-час\-тый и~никакой сле\-ду\-ющий за ним набор~$z$, $z\hm \neq p$, не является $s$-час\-тым.

    
    \smallskip
    
    \noindent
    \textbf{Определение~3.}\
    Набор $p \in \mathcal{P}$ называется минимальным $s$-не\-час\-тым, если он $s$-не\-час\-тый 
    и~никакой пред\-шест\-ву\-ющий ему набор~$z$, $z \hm\neq p$, не является $s$-не\-час\-тым.


\smallskip
    
    Далее вместо $s$-частый ($s$-не\-час\-тый) набор будем писать час\-тый (не\-час\-тый) набор. 
    Множество всех максимальных час\-тых наборов будем обозначать как $X_{\max}$, 
    а~множество всех минимальных не\-час\-тых наборов как $Y_{\min}$.
    
    Пусть $R \subset \mathcal{P}$, $R^+\hm = \{ x \in \mathcal{P} \vert \exists\, a \hm\in R, a \hm\preceq x \}$, 
    $R^- \hm= \{ x \hm\in \mathcal{P} \vert \exists\, a \hm\in R, x \hm\preceq a \}$.


    \noindent
    \textbf{Определение~4.}\
     Множество $I(R^+)$, со\-сто\-ящее из всех максимальных элементов множества~$\mathcal{P} \setminus R^+$, 
     называется максимальным независимым от~$R$.

\smallskip


   \noindent
    \textbf{Определение~5.}\
     Множество $I(R^-)$, со\-сто\-ящее из всех минимальных элементов множества~$\mathcal{P} \setminus R^-$, 
     называется минимальным независимым от~$R$.

\smallskip
    
    Каждая из задач построения $I(R^+)$ и~$I(R^-)$ 
    при заданном множестве~$R$ называется задачей дуализации над произведением час\-тич\-ных порядков.
    
    \smallskip

    \noindent
    \textbf{Утверждение~1.}\
    Если $X \hm\subset X_{\max}$, а~$y \hm\in I(X^-)$~--- не\-час\-тый набор, 
    то~$y$~--- минимальный не\-час\-тый набор.

\smallskip    
    
    \noindent
    Д\,о\,к\,а\,з\,а\,т\,е\,л\,ь\,с\,т\,в\,о\,.\  \ 
    Пусть $y \hm\notin I(X_{\max}^-)$. Так как~$y$~--- 
    нечастый набор, то в~$\mathcal{P} \setminus X^{-}_{\max}$ найдется минимальный не\-час\-тый набор~$x$ 
    такой, что $x\hm \neq y$ и~$x \hm\preceq y$. Из того, что $\mathcal{P} \setminus X^{-}_{\max} 
    \hm\subseteq \mathcal{P} \setminus X^-$, следует, что $x\hm \in \mathcal{P} \setminus X^-$, 
    что противоречит условию $y \hm\in I(X^-)$.

\smallskip

\noindent
\textbf{Утверждение~2.}\
    Пусть $X \hm\subseteq X_{\max}$, $Y\hm \subseteq Y_{\min}$. 
    Тогда $I(X^-) \hm= Y$ в~том и~только в~том случае, когда $X \hm= X_{\max}$ и~$Y \hm= Y_{\min}$.


\smallskip


  \noindent
    Д\,о\,к\,а\,з\,а\,т\,е\,л\,ь\,с\,т\,в\,о\,.\  \
    Пусть $X\! \subset\! X_{\max}, x \hm\in X_{\max}\!\setminus\!X$.
     Так как множество~$X_{\max}$~--- антицепь, то $x \hm\notin X^-$. 
     Следовательно, $x \hm\in \mathcal{P} \setminus X^{-}$.
      Но тогда существует элемент $ q \hm\in I(X^-) : q \preceq x$, 
      который является час\-тым. Однако во множестве~$Y$ частых наборов нет; следовательно, $I(X^-) \hm\neq Y$. 
      Если же $X \hm= X_{\max}$, то $I(X^-) \hm= Y_{\min}$. Таким образом, $I(X^-) \hm= Y$ тогда и~только
       тогда, когда $X \hm= X_{\max}$ и~$Y\hm = Y_{\min}$.


    
    \section{Методы построения множеств~$X_{\max}$ и~$Y_{\min}$}

    \subsection{Последовательное перечисление $X_{\max}$~и~$Y_{\min}$}

    Достаточно очевиден поиск~$X_{\max}$ и~$Y_{\min}$ при заданной $\mathcal{D} (\mathcal{P})$ 
    путем последовательного по\-стро\-ения множеств~$X_{\max}$ и~$Y_{\min}$. 
    Данный поиск осуществляется в~два этапа. На первом этапе находятся все максимальные частые 
    наборы~$X_{\max}$, например алгоритмом Apriori~\cite{2}. На втором этапе  используется свойство 
    двойственности $I \left(X_{\max}^- \right)\hm = Y_{\min}$. 
    Минимальные нечастые наборы~$Y_{\min}$ находятся путем дуализации найденного на первом этапе 
    множества~$X_{\max}$. Аналогично можно сначала искать~$Y_{\min}$ алгоритмом Apriori, а~затем 
    искать~$X_{\max}$ путем дуализации~$Y_{\min}$.

    Очевидно, что данный подход будет проявлять себя наилучшим образом в~случаях, когда 
    алгоритм Apriori или его модификации могут найти одно из искомых множеств существенно
     быст\-рее, чем другое множество, например когда мощ\-ность~$X_{\max}$ 
     существенно меньше (больше) мощ\-ности~$Y_{\min}$.
    
    \subsection{Совместное перечисление $X_{\max}$ и~$Y_{\min}$}

    В~\cite{9} предложена идея совместного перечисления множеств~$X_{\max}$ и~$Y_{\min}$. 
    На первом шаге рас\-смат\-ри\-ва\-ет\-ся некоторый случайный набор $q \hm\in \mathcal{P}$. Если $q$~--- 
    час\-тый набор, то ищется максимальный час\-тый набор, сле\-ду\-ющий за~$q$, 
    который пополняет множество $X \hm\subseteq X_{\max}$. Если $q$~---
     не\-час\-тый набор, то ищется минимальный не\-час\-тый набор, пред\-шест\-ву\-ющий~$q$, 
     который пополняет множество $Y \hm\subseteq Y_{\min}$. Пусть на шаге~$i$ ($i\hm \geq 1$) 
     построены множества $X \hm\subseteq X_{\max}$ и~$Y \hm\subseteq Y_{\min}$. Если $X \hm\neq \varnothing$, 
     $Y \hm= \varnothing$, то ищется набор~$q$ такой, что $q \hm\npreceq x, \forall x \hm\in X$. Если 
     $X \hm= \varnothing$, $Y \hm\neq \varnothing$, то ищется набор~$q$ такой, что 
     $q \hm\nsucceq y, \forall y \hm\in Y$. Если же и~$X \hm\neq \varnothing$, и~$Y \hm\neq \varnothing$, 
     то ищется набор~$q$ такой, что $q \hm\npreceq x, \forall x \hm\in X, q \hm\nsucceq y, \forall y \hm\in Y$.
      Затем, аналогично первому шагу, находится максимальный частый или минимальный нечастый набор. 
      Однако в~\cite{9} идея совместного перечисления искомых множеств экспериментально 
      не исследована и~не предложены конкретные указания по воз\-мож\-ной ее реализации.
    
    Алгоритм, основанный на совместном пе\-ре\-чис\-ле\-нии множеств~$X_{\max}$ и~$Y_{\min}$,
     реализован в~на\-сто\-ящей работе. Алгоритм строит две последовательности: $X_1 \hm\subset X_2 
     \subset \dots \subset X_{\max}$, $Y_1\hm \subset Y_2 \subset \dots \subset Y_{\min}$. 
     На первом шаге $X_1 \hm= \{x\}$, $Y_1 \hm= \{y\}$, где~$x$ и~$y$ ищутся алгоритмом Apriori.
      На шаге $i \hm+ 1$ ($i\hm \geq 1$) строится либо~$I(X^{-}_{i})$, либо~$I(Y^{+}_{i})$. Пусть на 
      шаге $i \hm+ 1$ ($i \hm\geq 1$) построено множество~$I(X^{-}_{i})$. 
      Согласно утверждениям~1 и~2, множество~$I(X^{-}_{i})$ либо не содержит час\-тых наборов 
      и~совпадает с~множеством~$Y_{\min}$ (в~этом случае $X_i \hm= X_{\max}$ 
      и~алгоритм заканчивает работу), либо~$I(X^{-}_{i})$ содержит как час\-тые, так и~не\-час\-тые наборы. 
      Каждый нечастый набор из~$I(X^{-}_{i})$ является минимальным не\-час\-тым и~пополняет множество~$Y_{i}$, 
      формируя в~результате множество~$Y_{i+1}$. Для каждого час\-то\-го набора находится один содержащий 
      его максимальный час\-тый набор путем последовательного увеличения текущего 
      частого набора в~лексикографическом порядке, который пополняет множество~$X_{i}$, 
      формируя в~результате множество~$X_{i+1}$.
      
    В~экспериментальной части работы (см.\ разд.~4) рас\-смот\-рен случай произведения цепей. 
    Задача дуализации решается с~помощью асимптотически оптимального алгоритма дуализации
     цепей \mbox{RUNC-M}+~\cite{7}. Асимптотически оптимальные алгоритмы дуализации 
     являются лидерами по ско\-рости счета~\cite{6}.

    Очевидно, что время работы совместного алгоритма в~основном зависит от чис\-ла
     минимальных не\-час\-тых и~максимальных час\-тых наборов. На\linebreak каж\-дой новой 
     итерации происходит дуализация\linebreak все б$\acute{\mbox{о}}$льших по мощ\-ности множеств~$X$ или~$Y$.\linebreak 
     Если число итераций становится достаточно\linebreak большим, то ско\-рость работы совместного 
     перечисления существенно снижается, что делает его практически неприменимым для 
     задач большой раз\-мер\-ности.
     { %\looseness=1
     
     }

    \subsection{Последовательно-совместное перечисление~$X_{\max}$ и~$Y_{\min}$}

    Предлагается следующий итеративный метод, который синтезирует идеи последовательного
     и~совместного методов, описанных выше. Положим $X_0 \hm= \varnothing$. 
     Строится одна по\-сле\-до\-ва\-тель\-ность $X_1 \hm\subset X_2 \hm\subset \dots \subset X_{\max}$. 
     На первом шаге $X_1\hm = \{x\}$, где $x$ ищется алгоритмом Apriori. На шаге $i \hm+ 1$ ($i \hm\geq 1$) 
     решается задача дуализации множества $X_{i} \setminus X_{i-1}$.

    
    
   \setcounter{figure}{1}
    \begin{figure*}[b] %fig2
  \vspace*{12pt}
  \begin{center}  
    \mbox{%
\epsfxsize=163mm
\epsfbox{duk-2.eps}
}

\end{center}
\vspace*{-9pt}
    \Caption{Зависимость времени работы алгоритмов от суммы мощностей множеств~$X_{\max}$ и~$Y_{\min}$ 
    для случая~1~(\textit{а}) и~2~(\textit{б}):
    \textit{1}~--- по\-сле\-до\-ва\-тель\-но-со\-вмест\-ный;
    \textit{2}~--- последовательный; \textit{3}~--- совместный; \textit{4}~--- Apriori}
    \label{12}
    \end{figure*}
     
    Пусть множество~$D$ есть результат дуализации $X_{i} \hm\setminus X_{i-1}$. Согласно утверждению~1, 
    множество~$D$ содержит частые наборы. Для каждого час\-то\-го набора из~$D$ 
    находится один содержащий его максимальный час\-тый набор путем последовательного 
    увеличения текущего час\-то\-го набора в~лексикографическом порядке. Все найденные максимальные
     частые наборы, которых нет в~множестве~$X_{i}$, до\-бав\-ля\-ют\-ся к~$X_{i}$, 
     и~таким образом формируется~$X_{i+1}$. Если же все найденные частые наборы уже содержатся в~$X_{i}$, 
     то решается задача дуализации множества~$X_{i}$. Если в~$I(X^{-}_{i})$ нет частых наборов, 
     то $I(X^{-}_{i})\hm = Y_{\min}$, $X_i \hm= X_{\max}$ и~алгоритм завершает работу. 
     Иначе для каждого частого набора из~$I(X^{-}_{i})$ находится один содержащий его максимальный 
     час\-тый набор, который пополняет множество~$X_{i}$, формируя в~результате множество~$X_{i+1}$.

    \section{Экспериментальное исследование}
    
    Рас\-смат\-ри\-вал\-ся случай данных, пред\-став\-лен\-ных в~виде произведения цепей мощ\-ности~5. 
    Для\linebreak таких данных проводился поиск максимальных час\-тых и~минимальных нечастых 
    наборов сле\-ду\-ющи\-ми методами: алгоритмом Apriori, модифицированным для случая 
    цепей; последовательным \mbox{методом}; совместным методом; по\-сле\-до\-ва\-тель\-но-со\-вмест\-ным методом.
    
    Все методы реализованы на языке Python~3. 
    Задача дуализации решалась алгоритмом дуализации цепей RUNC-M+~\cite{7}. 
    Эксперименты проведены на случайных базах данных различной раз\-мер\-ности. 
    Можно выделить два сле\-ду\-ющих случая соотношения мощностей множеств всех час\-тых и~не\-час\-тых наборов.
    \begin{description}
    \item[Случай 1:] мощ\-ность множества частых наборов примерно рав\-на мощ\-ности множества нечастых наборов.
    \item[Случай 2:] мощ\-ность множества частых наборов существенно меньше (больше) мощ\-ности множества 
    не\-час\-тых наборов.
    \end{description}
    
    Описанные случаи схематично изображены на рис.~1. 

    Графики зависимости времени работы тестируемых методов 
    от мощ\-ности множеств~$X_{\max}$ и~$Y_{\min}$ приведены на рис.~2.
    
    

    

    Нетрудно видеть, что в~случае~1 лучше работает по\-сле\-до\-ва\-тель\-но-со\-вмест\-ный алгоритм: 
    множества час\-тых и~не\-час\-тых наборов имеют примерно одинаковую мощ\-ность, 
    поэтому быст\-рее будет обрабатывать их по\-сле\-до\-ва\-тель\-но-со\-вмест\-ным методом. В~случае~2 
    быст\-рее работает последовательный алгоритм: быст\-рее найти множество максимальных час\-тых наборов, 
    обработав множество час\-тых наборов, и~дуализировать результат. Время поиска множеств~$X_{\max}$ 
    и~$Y_{\min}$ совместным методом и~модифицированным алгоритмом Apriori рас\-тет существенно 
    быст\-рее времени поиска по\-сле\-до\-ва\-тель\-но-со\-вмест\-ным методом в~обоих случаях.
    
    { \begin{center}  %fig1
 \vspace*{9pt}
    \mbox{%
\epsfxsize=67.963mm
\epsfbox{duk-1.eps}
}

\end{center}

\noindent
{{\figurename~1}\ \ \small{
Два случая соотношения мощностей множеств час\-тых и~не\-час\-тых наборов
}}}

%\vspace*{6pt}


    \section{Заключение}
    
Рас\-смот\-ре\-на задача поиска максимальных час\-тых и~минимальных не\-час\-тых наборов в~данных, 
представленных в~виде декартова произведения час\-тич\-ных порядков. Актуальны вопросы 
снижения временн$\acute{\mbox{ы}}$х затрат, возникающих при реализации методов нахождения искомых наборов.
 Разработан новый подход к~по\-стро\-ению множества максимальных частых наборов~$X_{\max}$ и~множества 
 минимальных не\-час\-тых наборов~$Y_{\min}$, пред\-став\-ля\-ющий собой синтез двух ранее известных 
 подходов: последовательного и~со\-вмест\-но\-го (первый достаточно очевиден, идея второго предложена в~\cite{9}). 
 Сложность последовательного, совместного и~пред\-ла\-га\-емо\-го по\-сле\-до\-ва\-тель\-но-со\-вмест\-но\-го поиска 
 обуслов\-ле\-на, в~том чис\-ле, не\-об\-хо\-ди\-мостью рас\-смат\-ри\-вать в~процессе поиска 
 труд\-но\-ре\-ша\-емую пе\-ре\-чис\-ли\-тель\-ную задачу дис\-крет\-ной математики, на\-зы\-ва\-емую дуализацией 
 над произведением час\-тич\-ных порядков.

Для случая, когда данные пред\-став\-ле\-ны в~виде произведения конечных цепей, 
приведены результаты экспериментального срав\-не\-ния названных подходов, а~так\-же независимого 
способа \mbox{по\-стро\-ения} множеств~$X_{\max}$ и~$Y_{\min}$, не тре\-бу\-юще\-го решения задачи дуализации. 
Эксперименты проводились на модельных задачах с~применением асимптотически оптимального
 алгоритма дуализации над произведением конечных цепей \mbox{RUNC-M}+~\cite{7}. 
 Результаты исследования свидетельствуют о~том, что по\-сле\-до\-ва\-тель\-но-со\-вмест\-ный 
 метод наиболее эффективен (требует меньших временн$\acute{\mbox{ы}}$х затрат по сравнению с~другими рас\-смот\-рен\-ны\-ми 
 методами) в~случае, когда мощ\-ность множества час\-тых наборов примерно равна мощ\-ности множества
  нечастых наборов. Иначе выигрывает последовательный поиск. Наихудшие показатели 
  у~независимого пе\-ре\-чис\-ле\-ния множеств~$X_{\max}$ и~$Y_{\min}$ с~использованием в~качестве
   базового алгоритма Apriori~\cite{2}, точ\-нее его модификации на тес\-ти\-ру\-емый случай. 
   Таким образом, показана це\-ле\-со\-об\-раз\-ность применения алгоритмов дуализации для 
   по\-стро\-ения множеств~$X_{\max}$ и~$Y_{\min}$.

  
  {\small\frenchspacing
 {%\baselineskip=10.8pt
 %\addcontentsline{toc}{section}{References}
 \begin{thebibliography}{9}  
    \bibitem{4}
    \Au{Aggarwal C.} 
    Frequent pattern mining.~--- Heidelberg: Springer, 2014. 467~p.
    
    \bibitem{1}
    \Au{Agrawal~R., Imielinski~T., Swami~A.} Mining association rules 
    between sets of items in large databases~// \mbox{SIGMOD} Conference (International) on Management of Data
    Proceedings.~--- New York, NY, USA: ACM, 1993. P.~207--216.
    
    \bibitem{9}
    \Au{Elbassioni K.} On finding minimal infrequent elements in multi-dimensional 
    data defined over partially ordered sets~// arXiv.org, 2014. 30~p. arXiv:1411.2275 [cs.DB].
    
    \bibitem{8}
    \Au{Elbassioni K.} Algorithms for dualization over products of partially 
    ordered sets~// SIAM J.~Discrete Math., 2009. Vol.~23. Iss.~1. P.~487--510.
    
    \bibitem{2}
    \Au{Agrawal R., Srikant~R.} 
    Fast algorithms for mining association rules in large databases~// 
    20th Conference (International) on Very Large Data Bases Proceedings.~--- San Francisco, CA, USA: 
    Morgan Kaufmann Publs. Inc., 1994. P.~487--499.
    
    \bibitem{14}
    \Au{Хачиян Л.\,Г.} Избранные труды.~--- М.: МЦНМО, 2009. 520~с.
    
    \bibitem{7}
    \Au{Дюкова Е.\,В., Масляков~Г.\,О., Прокофьев~П.\,А.} 
    О~дуализации над произведением частичных порядков~// Машинное обучение и~анализ данных, 2017. Т.~3. №\,4.  
    C.~239--249.
    
    \bibitem{6}
    \Au{Дюкова Е.\,В., Прокофьев~П.\,А.} Об асимптотически оптимальных алгоритмах дуализации~// 
    Ж.~вычисл. матем. и~матем. физ., 2015. Т.~55. №\,5. С.~895--910.
    \end{thebibliography}

 }
 }

\end{multicols}

\vspace*{-6pt}

\hfill{\small\textit{Поступила в~редакцию 15.01.21}}

\vspace*{8pt}

%\pagebreak

%\newpage

%\vspace*{-28pt}

\hrule

\vspace*{2pt}

\hrule

%\vspace*{-2pt}

\def\tit{FINDING MAXIMAL FREQUENT AND~MINIMAL INFREQUENT SETS IN~PARTIALLY ORDERED DATA}


\def\titkol{Finding maximal frequent and~minimal infrequent sets in~partially ordered data}


\def\aut{N.\,A.~Dragunov and E.\,V.~Djukova}

\def\autkol{N.\,A.~Dragunov and E.\,V.~Djukova}

\titel{\tit}{\aut}{\autkol}{\titkol}

\vspace*{-11pt}


\noindent
Federal Research Center ``Computer Science and Control'' 
of the Russian Academy of Sciences, 44-2~Vavilov Str., Moscow 119333, Russian Federation

\def\leftfootline{\small{\textbf{\thepage}
\hfill INFORMATIKA I EE PRIMENENIYA~--- INFORMATICS AND
APPLICATIONS\ \ \ 2022\ \ \ volume~16\ \ \ issue\ 1}
}%
 \def\rightfootline{\small{INFORMATIKA I EE PRIMENENIYA~---
INFORMATICS AND APPLICATIONS\ \ \ 2022\ \ \ volume~16\ \ \ issue\ 1
\hfill \textbf{\thepage}}}

\vspace*{3pt} 


\Abste{Relevant issues of time costs reducing in the logical analysis of data with elements 
from the Cartesian product of finite partially ordered sets are investigated. 
An original method based on solving a complex discrete problem called dualization
 over the product of partial orders is proposed for the problem of finding maximal 
 frequent and minimal infrequent sets in the transaction database. The proposed method 
 is a~synthesis of two other known methods, one of which is quite obvious and the other uses 
 the idea of an incremental enumeration of target\linebreak\vspace*{-12pt}}
 
 \Abstend{sets and is, therefore, mainly 
 of theoretical interest. An experimental study of the considered approaches in
  the case of the product of finite chains is carried out and conditions for
   their effectiveness are revealed. The expediency of applying 
asymptotically optimal dualization algorithms over the product of partial orders is shown.}

\KWE{maximal frequent sets; minimal infrequent sets; dualization over the product of 
partial orders; asymptotically optimal dualization algorithm}

\DOI{10.14357/19922264220112}

%\vspace*{-16pt}

%\Ack
%\noindent




%\vspace*{6pt}

  \begin{multicols}{2}

\renewcommand{\bibname}{\protect\rmfamily References}
%\renewcommand{\bibname}{\large\protect\rm References}

{\small\frenchspacing
 {%\baselineskip=10.8pt
 \addcontentsline{toc}{section}{References}
 \begin{thebibliography}{9}
\bibitem{1-dr}
\Aue{Aggarwal, C.} 2014. \textit{Frequent pattern mining}. Heidelberg: Springer. 467~p.
\bibitem{2-dr}
\Aue{Agrawal, R., T.~Imielinski, and A.~Swami.}
 1993. Mining association rules between sets of items in large databases. 
 \textit{SIGMOD  Conference (International) on Management of Data Proceedings}. New York, NY:
 ACM. 207--216. 
\bibitem{3-dr}
\Aue{Elbassioni, K.}
 2014. On finding minimal infrequent elements in multidimensional data defined over partially ordered sets. 
 arXiv.org. 30~p. Available at: 
 {\sf https://arxiv.org/\linebreak pdf/1411.2275.pdf} (accessed January~25, 2022).
\bibitem{4-dr}
\Aue{Elbassioni, K.} 2009. Algorithms for dualization over products of partially ordered sets. 
\textit{SIAM J.~Discrete Math.} 23(1):487--510.
\bibitem{5-dr}
\Aue{Agrawal, R., and R.~Srikant.}
 1994. Fast algorithms for mining association rules in large databases. 
 \textit{20th Conference (International) on Very Large Data Bases Proceedings}.
 San Francisco, CA: 
    Morgan Kaufmann Publs. Inc.  487--499.
\bibitem{6-dr}
\Aue{Khachiyan, L.\,G.} 2009. \textit{Izbrannye trudy} [Selected works]. Moscow: MCCME. 520~p.
\bibitem{7-dr}
\Aue{Djukova, E.\,V., G.\,O.~Maslyakov, and P.\,A.~Prokofyev.} 
2017. O~dualizatsii nad proizvedeniem chastichnykh poryadkov [On dualization over the product of 
partial orders]. \textit{Mashinnoe obuchenie i~analiz dannykh} [J.~Machine Learning Data Analysis] 
3(4):239--249.
\bibitem{8-dr}
\Aue{Djukova, E.\,V., and P.\,A.~Prokofyev.}
 2015. Asymptotically optimal dualization algorithms. \textit{Comp. Math.
 Math. Phys.} 55(5):891--905. 
 
 \end{thebibliography}

 }
 }

\end{multicols}

\vspace*{-6pt}

\hfill{\small\textit{Received January 15, 2021}}

%\pagebreak

%\vspace*{-18pt}

\Contr

\noindent
\textbf{Dragunov Nikita A.} (b.\ 1997)~--- 
PhD student, Federal Research Center ``Computer Science and Control'' 
of the Russian Academy of Sciences, 44-2~Vavilov Str., Moscow 119333, Russian Federation; 
\mbox{nikitadragunovjob@gmail.com}

\vspace*{3pt}

\noindent
\textbf{Djukova Elena V.} (b.\ 1945)~--- 
Doctor of Science in physics and mathematics, principal scientist, Federal Research Center
``Computer Science and Control'' of the Russian Academy of Sciences, 44-2~Vavilov Str., Moscow 119333, 
Russian Federation; \mbox{edjukova@mail.ru}




\label{end\stat}

\renewcommand{\bibname}{\protect\rm Литература}  %12
\def\stat{abgaryan}

\def\tit{ПРОГРАММНЫЙ КОМПЛЕКС ДЛЯ~МНОГОМАСШТАБНОГО МОДЕЛИРОВАНИЯ 
СТРУКТУРНЫХ СВОЙСТВ КОМПОЗИЦИОННЫХ МАТЕРИАЛОВ$^*$}

\def\titkol{Программный комплекс для многомасштабного моделирования 
структурных свойств композиционных материалов}

\def\aut{К.\,К.~Абгарян~$^1$, Е.\,С.~Гаврилов$^2$}

\def\autkol{К.\,К.~Абгарян, Е.\,С.~Гаврилов}

\titel{\tit}{\aut}{\autkol}{\titkol}

\index{Абгарян К.\,К.}
\index{Гаврилов Е.\,С.}
\index{Abgaryan K.\,K.}
\index{Gavrilov E.\,S.}


{\renewcommand{\thefootnote}{\fnsymbol{footnote}} \footnotetext[1]
{Работа выполнена при поддержке Министерства науки и~высшего образования Российской Федерации (проект 
075-15-2020-799).}}


\renewcommand{\thefootnote}{\arabic{footnote}}
\footnotetext[1]{Федеральный исследовательский центр <<Информатика и~управление>> Российской академии наук, 
\mbox{kristal83@mail.ru}}
\footnotetext[2]{Федеральный исследовательский центр <<Информатика и~управление>> Российской академии наук; 
Московский авиационный институт (национальный исследовательский университет), \mbox{eugavrilov@gmail.com}}

%\vspace*{-6pt}
    
      
         
      
      \Abst{Создание новых композиционных материалов (КМ) с~прогнозируемыми свойствами 
      и~разработка способов их конструирования на сегодня стали одними из актуальных и~важнейших 
задач, связанных с~модернизацией промышленного производства в~нашей стране. Для их 
решения активно развиваются технологии многомасштабного компьютерного 
моделирования. Они стали связующим звеном между фундаментальной физикой (химией) 
и~инженерным материаловедением. В~работе представлен программный комплекс по 
моделированию структурных свойств КМ, поз\-во\-ля\-ющий решать ряд 
задач данного класса. Он ориентирован на высокопроизводительные вы\-чис\-ле\-ния. В~основе 
комплекса лежит оригинальная многомасштабная технология, которая позволяет оперативно 
проводить многовариантный анализ различных классов КМ 
и~проводить исследования по проектированию новых с~прогнозируемыми свойствами. 
Разработанные подходы в~сочетании с~экспериментальными данными могут быть использованы 
для лучшего понимания физических основ изменения свойств в~за\-ви\-си\-мости от структуры и,~как 
следствие, для удешевления и~ускорения поиска новых КМ
с~заданными свойствами.}
      
      \KW{многомасштабное моделирование; композиционные материалы; интеграционная 
платформа; программный комплекс; распределенная сис\-тема}

\DOI{10.14357/19922264220113}
  
%\vspace*{-3pt}


\vskip 10pt plus 9pt minus 6pt

\thispagestyle{headings}

\begin{multicols}{2}

\label{st\stat}

\section{Введение}

\vspace*{-3pt}

     Создание новых КМ с~прогнозируемыми 
свойствами и~разработка способов их конструирования на сегодня стали одними 
из актуальных и~важнейших задач по модернизации промышленного 
производства в~нашей стране. Особенно важны такие материалы в~областях, где 
соотношение между проч\-ностью и~массой конструкции определяет ее 
эф\-фек\-тив\-ность. На сегодня процессы создания КМ
непосредственно связаны с~этапом моделирования, включая применение наиболее 
эффективных методов многомасштабного компьютерного моделирования и~анализа данных. 
     
     Для решения данного класса задач разработан\linebreak программный комплекс по 
моделированию структурных свойств КМ. Он 
ориентирован на высокопроизводительные вы\-чис\-ле\-ния. В~осно\-ве комплекса 
лежит оригинальная многомасштабная \mbox{технология}, пред\-став\-лен\-ная в~[1, 2], 
которая позволяет оперативно проводить многовариантный анализ различных 
классов КМ. На базе разработанной технологии была 
создана распределенная информационная сис\-те\-ма для проведения 
многоуровневых исследований в~об\-ласти моделирования~КМ. 

Согласно разработанным подходам в~за\-ви\-си\-мости от типа 
мо\-де\-ли\-ру\-емо\-го КМ строится многомасштабная 
композиция и~ее схематическое представление. На ее основе в~программной среде 
формируется сценарий расчета структурных характеристик и~отдельных свойств 
рас\-смат\-ри\-ва\-емо\-го материала. Созданный программный комплекс позволяет 
автоматизировать уни\-фи\-ци\-ру\-емые этапы моделирования и~помогает 
сформировать на основе анализа полученных результатов более глубокое 
понимание физических процессов. Комплекс построен с~применением 
современных программных средств и~решений и~не уступает международному 
уровню на\-уч\-но-тех\-ни\-че\-ских разработок в~об\-ласти информационной 
поддержки для многомасштабного моделирования новых материалов. 
     
     Разработка такого средства информационной поддержки поз\-во\-ля\-ет 
обеспечить формирование информации для многопараметрического анализа 
структуры и~физических свойств различных классов су\-ще\-ст\-ву\-ющих 
КМ, рассмотреть большое чис\-ло вариантов 
в~на\-прав\-ле\-нии поиска новых материалов и,~таким образом, ускорить и~удешевить 
процесс подбора па\-ра\-мет\-ров получения материалов.  Ис-\linebreak\vspace*{-12pt}

\pagebreak

\noindent
пользование данного 
комплекса позволяет за ограниченное время строить гиб\-рид\-ные модели для 
обоснованного выбора КМ с~заданными свойствами для  
авиа\-ци\-он\-но-кос\-ми\-че\-ской и~других областей промышленности. 
     
     В связи с~тем что традиционные материалы (преимущественно металлы)
      не в~полной мере отвечают высоким фи\-зи\-ко-ме\-ха\-ни\-че\-ским, 
технологическим и~эксплуатационными свойствам, развитие производства 
современных надежных и~экономичных конструкций в~машиностроении 
основано на применении новых КМ. Под 
композиционными понимаются материалы, со\-сто\-ящие из двух или более 
физически различных компонент (фаз), возможные комбинации которых 
приводят к~появлению уникальных свойств, отличных от тех, которыми обладала 
каж\-дая из них отдельно. На сегодня для развития авиа\-ци\-он\-но-кос\-ми\-че\-ской 
отрасли, включая самолетостроение, вертолетостроение, ракетостроение, 
требуется постоянное увеличение доли полимерных КМ
с~набором заданных свойств. Современные летательные аппараты обладают 
слож\-ной конструкцией, со\-сто\-ящей из металлов и~неметаллических материалов. 
Применяются детали из алю\-ми\-ни\-евых и~сталь\-ных сплавов, коррозионностойких 
сталей, титановых сплавов и~полимерных КМ (стек\-ло-, 
угле-, органопластики и~др.). Для снижения веса и~продления срока службы 
летательных аппаратов при производстве деталей все шире применяют 
полимерные~КМ.
     
     Сегодня наиболее востребованные САЕ- (Computer-Aided Engineering) 
сис\-те\-мы, такие как ABAQUS ({\sf https://simulia.com}), \mbox{ANSYS} ({\sf 
https://\linebreak Ansys.com}), LMS Engineering innovation ({\sf https://\linebreak trademarks.justia.com}), 
Femap ({\sf https://www.cad-is.ru/femap}), MSC Software ({\sf 
http://www.mscsoftware.\linebreak ru}) включают в~себя базы данных со свойствами 
материалов. Для КМ мож\-но выбрать тип композита со 
стандартными свойствами (угле-, стекло-, органопластики на основе 
эпоксифенолформальдегидных, кремнийорганических смол, эпоксидные 
боропластики и~т.\,д.). Имеется возможность коррекции данных свойств 
и~внесения материала с~новыми свойствами в~базу данных. Следует также отметить 
российские разработки в~об\-ласти моделирования КМ, 
такие как пакет CAE-Fidesys ({\sf https://cae-fidesys.com}), программный пакет для 
моделирования полимерных материалов Multicomp ({\sf 
https://www.kintechlab.com/products}), Российский исследовательский 
и~ин\-же\-нер\-но-тех\-но\-ло\-ги\-че\-ский проект N1 Composites ({\sf 
http://n1composites.com}) и~др.
{\looseness=-1

}
     
     Программные комплексы позволяют задать\linebreak свойства материалов, из 
которых состоит КМ, такие как изотропность, 
ортотропность, анизотропность. Важная часть проектирования композиционных 
конструкций~--- преобразование модели,\linebreak созданной с~применением CAD 
(Computer-aided design, сис\-те\-мы автоматизированного проектирования) 
в~модель, пригодную для CAE-ана\-ли\-за (нетривиальная задача, тре\-бу\-ющая 
за\-час\-тую создания экспертной сис\-те\-мы). Следует отметить, что функционал всех 
мировых лидеров в~CAE-сег\-мен\-те схож. 
     %
     Так, функционал MSC позволяет встраивать разработанные пользователем 
модули в~программный комплекс (например, можно включить метод имитации 
процесса производства КМ).
     
     Помимо используемых ведущими CAE-сис\-те\-ма\-ми модулями существуют 
коммерческие сис\-те\-мы, позволяющие генерировать КМ на микроуровне, а~затем 
проводить чис\-лен\-ные эксперименты на макроуровне. К~таким сис\-те\-мам 
относятся модуль генерации и~моделирования механических характеристик 
КМ GeoDict ({\sf www.math2market.com}) с~различными типами КМ, 
ге\-не\-ри\-ру\-емы\-ми модулем GeoDict, и~программный комплекс COMSOL ({\sf 
www.comsol.ru}).
     
     В современных ведущих CAE-сис\-те\-мах учет мик\-ро\-струк\-ту\-ры 
КМ проводится после гомогенизации свойств материала 
или определения мак\-ро\-мас\-штаб\-ных свойств КМ. При этом, однако, теряются 
индивидуальные детали микроструктуры КМ~\cite{3-ab}. При определении макромасштабных свойств КМ обычно 
исходят из идеальных условий: оптимального формирования граничной 
поверхности, идеального распределения(отсутствия взаимодействия час\-тиц 
между собой) и~отсутствия влияния компонента на мат\-рицу.
     
     Однако результаты, которые на сегодня могут быть получены 
     с~использованием САЕ-систем для\linebreak воспроизведения характеристик известных 
структур, зачастую могут расходиться с~данными экспериментов~--- например, 
когда речь идет о~полимерных КМ с~на\-но\-вклю\-че\-ни\-ями 
(\mbox{нанотрубками}). \mbox{Известно} влияние до\-бав\-ле\-ния на\-но\-раз\-мер\-ных\linebreak час\-тиц 
наполнителя на изменение механических свойств КМ. 
В~литературе широко описано изменение коэффициента теп\-ло\-про\-вод\-ности 
полимерных\linebreak мат\-риц в~несколько раз при их наполнении 
нанотрубками, пред\-став\-ле\-ны тео\-ре\-ти\-че\-ские исследования с~аналогичными 
результатами~\cite{1-ab}. Использование CAE-сис\-те\-м не позволяет в~полной 
мере \mbox{оценить} фактор влияния на\-но\-час\-тиц на данные свойства. Кроме 
того, применение CAE-сис\-тем в~контексте многомасштабного моделирования 
затруднено жесткими ограничениями пакетных решений. В~настоящее время 
развиваются \mbox{системы} c~программным обеспечением для многомасштабного 
моделирования, такие как Computational Soft Materials (Comsoft) Workbench, 
поз\-во\-ля\-ющий моделировать КМ с~<<мягкой>> 
структурой (полимеры, полимерные композиты), программный пакет LAMMPS 
({\sf https://www.lammps.org}), ис\-поль\-зу\-емый для моделирования в~рамках 
классической молекулярной динамики на атомистическом и~мезомасштабном 
уровнях полимерных, металлических, биологических сис\-тем и~др. Каждый из 
разрабатываемых программных продуктов обладает своими достоинствами 
и~областями применения. В~связи с~большим разнообразием типов 
КМ и~все воз\-рас\-та\-ющи\-ми требованиями к~наборам 
свойств, которыми они должны обладать, пред\-став\-ля\-ет\-ся важ\-ным\linebreak создание 
программных средств, поз\-во\-ля\-ющих оперативно вы\-стра\-и\-вать сис\-тем\-ные 
решения в~об\-ласти\linebreak многомасштабного моделирования с~применением 
высокопроизводительных вычислений, поз\-во\-ля\-ющих проводить моделирование от  
атом\-но-крис\-тал\-ли\-че\-ско\-го до мак\-ро\-уров\-ня. Такие системы \mbox{позволят} 
генерировать и~выполнять в~автоматическом режиме сценарии проведения 
расчетов под конкретную задачу, включать в~вычислительную схему расчеты на 
всех необходимых мас\-штаб\-ных уровнях. Для предсказательного моделирования 
структурных свойств различных классов КМ такой 
подход поз\-во\-ля\-ет создавать вы\-чис\-ли\-тель\-ную среду, в~которой задействованы 
возможности CАE-сис\-тем для верх\-не\-уров\-не\-во\-го (мак\-ро-) моделирования, 
методы анализа экспериментальных и~аналитических данных, а также 
собственные разработки и~пакетные приложения для расчетов на атом\-но-крис\-тал\-ли\-че\-ском и~наноуровне.

\vspace*{-9pt}

\section{Многомасштабная модель для~расчета структурных 
свойств композиционных материалов}

     В работе~\cite{2-ab} представлена общая схема многомасштабной модели 
для расчета структурных характеристик КМ. Для ее 
описания используется тео\-ре\-ти\-ко-мно\-жест\-вен\-ный аппарат, изложенный 
в~\cite{1-ab, 2-ab}. На ее основе формируются схемы для расчета разных классов 
КМ: нанокомпозитов на основе полимерной мат\-ри\-цы, 
КМ с~металлической мат\-ри\-цей, полимерных 
КМ с~углеволокном и~др.

\vspace*{-9pt}
     
     \subsection*{Основные уровни моделирования}
     
     \vspace*{-2pt}
     
     
     \textbf{Квантово-механический}. Рассматриваются отдельные молекулы. 
Решается уравнение Шредингера, определяется атомарная струк\-ту\-ра молекул 
полимера и~наполнителя, строится электронная струк\-ту\-ра и~рас\-счи\-ты\-ва\-ет\-ся 
когезионная энергия, рас\-счи\-ты\-ва\-ют\-ся меж\-атом\-ные и~меж\-мо\-ле\-ку\-ляр\-ные силы, 
определяются отдельные фи\-зи\-ко-хи\-ми\-че\-ские свойства.
     
     \textbf{Молекулярно-динамический}. Изучаются ан\-самб\-ли из молекул. 
Решаются уравнения молекулярной динамики с~использованием потенциалов 
межатомного взаимодействия, рас\-счи\-ты\-ва\-ют\-ся структурные характеристики 
мат\-ри\-цы (полимерной, металлической и~др.), наполнителя (нанотрубки, 
волокна и~др.), физические свойства. 
     
     \textbf{Мезоскопический}. Рас\-смат\-ри\-ва\-ют\-ся крупнозернистые модели. 
Используется упрощенное строение молекул. Цель моделирования на 
мезоуровне~--- получение распределения час\-тиц \mbox{наполнителя} в~мат\-ри\-це 
(полимерной, металлической и~др.)\ с~по\-сле\-ду\-ющим расчетом инженерных 
свойств полученных сис\-тем. 

\begin{figure*}[b] %fig1
\vspace*{8pt}
  \begin{center}  
    \mbox{%
\epsfxsize=133.618mm
\epsfbox{abg-1.eps}
}

\end{center}
\vspace*{-2pt}
\Caption{Схема многомасштабной композиции $\mathbf{MK}_{0,1,2,3,4}^{(\mathrm{Ti/Mo})}$ 
для расчета структурных свойств МКМ}
\end{figure*}
     
     \textbf{Континуальный} (\textbf{макроскопический}). Проводится расчет 
инженерных свойств (механические свойства, теп\-ло\-про\-вод\-ность и~др.). Задачи 
решаются с~применением механики сплош\-ных сред, гид\-ро\-ди\-на\-ми\-ки, тео\-рии 
упру\-гости. Применяются метод конечных элементов, методы решения краевых 
задач для моделирования различных процессов. 
     
     Рассмотрим пример построения многомасштабной композиции для 
тес\-то\-во\-го рас\-че\-та структурных свойств металлического 
КМ (МКМ) на основе Ti (титана), армированного волокнами Mo 
(молибдена). На сегодня Ti и~титановые сплавы стали очень привлекательными 
материалами для перспективных сфер применения благодаря таким свойствам, 
как низкая плот\-ность, высокие механические свойства и~коррозионная стой\-кость. 
Использование данных материалов в~конструкциях самолетов (реактивный 
двигатель и~фюзеляж) и~применение в~автомобильной про\-мыш\-лен\-ности рас\-тут 
быст\-ры\-ми темпами. Одним из способов совершенствования\linebreak титановых сплавов 
стало их применение в~качестве мат\-ри\-цы для КМ, 
армированных волокнами, например из Mo, которые обладают очень \mbox{хорошими} 
механическими свойствами ({\sf http://\linebreak viam-works.ru/ru/articles?art\_id=1103}). 
     
     Задействуем четыре перечисленных выше масштабных уров\-ня (не считая 
нулевого). Используя обозначения из~\cite{1-ab, 2-ab}, для построения 
многомасштабной композиции 
$$
\mathbf{MK}_{0,1,2,3,4}^{(\mathrm{Mo}, \mathrm{Ti}; 
1{,}1; 1{,}2; 2{,}1; 2{,}2; 3{,}1; 4{,}1)}= \mathbf{MK}_{0,1,2,3,4}^{(\mathrm{Ti/Mo})}
$$ 

\vspace*{-3pt}

\noindent
приведем экземпляры базовых мо\-де\-лей-ком\-по\-зи\-ций: 

\vspace*{-9pt}

\noindent
     \begin{align*}
     \mathbf{El}_{01}^{\mathrm{Ti}}:& \left\{ V_{01}^{\mathrm{Ti}}, 
X_{01}^{\mathrm{Ti}}, \mathrm{MA}_{01}^{\mathrm{Ti}}\right\};\\[-3pt]
     \mathbf{El}_{01}^{\mathrm{Mo}}:& \left\{ V_{01}^{\mathrm{Mo}}, 
X_{01}^{\mathrm{Mo}}, \mathrm{MA}_{01}^{\mathrm{Mo}}\right\};\\[-3pt]
\mathbf{MC}_{11}^{\mathrm{Ti}}:& \left\{ V_{11}^{\mathrm{Ti}}, 
X_{11}^{\mathrm{Ti}}, \mathrm{MA}_{11}^{\mathrm{Ti}}\right\};\\[-3pt]
\mathbf{MC}_{11}^{\mathrm{Mo}}:& \left\{ V_{11}^{\mathrm{Mo}}, 
X_{11}^{\mathrm{Mo}}, \mathrm{MA}_{11}^{\mathrm{Mo}}\right\};
\end{align*}

\noindent
\begin{align*}
               \mathbf{MC}_{12}^{\mathrm{Ti}}:& \left\{ V_{12}^{\mathrm{Ti}}, 
X_{12}^{\mathrm{Ti}}, \mathrm{MA}_{12}^{\mathrm{Ti}}\right\};\\
     \mathbf{MC}_{12}^{\mathrm{Mo}}:& \left\{ V_{12}^{\mathrm{Mo}}, 
X_{12}^{\mathrm{Mo}}, \mathrm{MA}_{12}^{\mathrm{Mo}}\right\};\\
     \mathbf{MC}_{21}^{\mathrm{Ti}}:& \left\{ V_{21}^{\mathrm{Ti}}, 
X_{21}^{\mathrm{Ti}}, \mathrm{MA}_{21}^{\mathrm{Ti}}\right\};\\
     \mathbf{MC}_{21}^{\mathrm{Mo}}:& \left\{ V_{21}^{\mathrm{Mo}}, 
X_{21}^{\mathrm{Mo}}, \mathrm{MA}_{21}^{\mathrm{Mo}}\right\};\\
     \mathbf{MC}_{22}^{\mathrm{Ti}}:& \left\{ V_{22}^{\mathrm{Ti}}, 
X_{22}^{\mathrm{Ti}}, \mathrm{MA}_{22}^{\mathrm{Ti}}\right\};\\
     \mathbf{MC}_{22}^{\mathrm{Mo}}:& \left\{ V_{22}^{\mathrm{Mo}}, 
X_{22}^{\mathrm{Mo}}, \mathrm{MA}_{22}^{\mathrm{Mo}}\right\};\\
     \mathbf{MC}_{31}^{\mathrm{Ti}/\mathrm{Mo}}:& \left\{
     V_{31}^{\mathrm{Ti}/\mathrm{Mo}}, X_{31}^{\mathrm{Ti}/\mathrm{Mo}}, 
\mathrm{MA}_{31}^{\mathrm{Ti}/\mathrm{Mo}}\right\};\\
     \mathbf{MC}_{41}^{\mathrm{Ti}/\mathrm{Mo}}:& \left\{
     V_{41}^{\mathrm{Ti}/\mathrm{Mo}}, X_{41}^{\mathrm{Ti}/\mathrm{Mo}}, 
\mathrm{MA}_{41}^{\mathrm{Ti}/\mathrm{Mo}}\right\}.
     \end{align*}
     
     Согласно схематическому пред\-став\-ле\-нию (рис.~1) многомасштабная 
композиция $\mathbf{MK}_{0,1,2,3,4}^{(\mathrm{Ti/Mo})}$ со\-сто\-ит из связанных между 
собой экземпляров базовых моделей композиций, размещенных на 
со\-от\-вет\-ст\-ву\-ющих мас\-штаб\-ных уровнях. На наноуровне проводится  
мо\-ле\-ку\-ляр\-но-ди\-на\-ми\-че\-ское моделирование структурных свойств 
титановой мат\-ри\-цы и~молибденовых волокон. На мезоуровне рас\-смат\-ри\-ва\-ет\-ся 
распределение час\-тиц в~МКМ, на мак\-ро\-уров\-не проводится расчет механических 
свойств МКМ.

\setcounter{figure}{2}
\begin{figure*}[b] %fig3
\vspace*{-6pt}
  \begin{center}  
    \mbox{%
\epsfxsize=120.383mm
\epsfbox{abg-3.eps}
}

\end{center}
\vspace*{-9pt}
\Caption{Пример сценария с~цик\-лом}
\end{figure*}
%\pagebreak
     
\vspace*{-10pt}

\section{Программный комплекс}

\vspace*{-2pt}

   Программный комплекс, интегрированный с~расчетными пакетами 
и~модулями, размещается на высокопроизводительных многоядерных сис\-те\-мах, 
оснащенных мощными графическими процессорами. Это связано с~тем, что 
исполнение вычислительных экспериментов, а~так\-же обработка 
и~анализ результатов вы\-чис\-ли\-тель\-ных  экспериментов
 ориентированы на 
распределенные сис\-те\-мы сбора, хранения и~обработки больших данных. В~основе 
программного комплекса лежит интеграционная платформа для 
многомасштабного моделирования, которая объединяет информационные потоки 
на разных мас\-штаб\-ных уровнях. При решении конкретной задачи, такой как 
расчет структурных особенностей, механических или иных свойств 
КМ, при изучении процессов их де\-гра\-да\-ции 
и~разрушения и~др.\ выделяются конкретные уров\-ни моделирования, которые 
необходимо задействовать. Первоначально строится многомасштабная 
композиция~--- информационный аналог\linebreak мно\-го\-мас\-штаб\-ной  
фи\-зи\-ко-ма\-те\-ма\-ти\-че\-ской модели. Для программной реализации на базе 
интеграционной платформы~\cite{4-ab} из име\-ющих\-ся программных модулей 
формируется вы\-чис\-ли\-тель\-ный \mbox{комплекс}~\cite{5-ab, 6-ab}.
   
   Перечислим пользовательские роли в~интеграционной плат\-фор\-ме 
мно\-го\-мас\-штаб\-но\-го моделирования:
   \begin{itemize}
\item разработчик вычислительных модулей реализует расчетный модуль или 
осуществляет конфигурирование при\-клад\-но\-го па\-кета;\\[-15pt]
\item системный разработчик создает веб-сер\-ви\-сы для вы\-чис\-ли\-тель\-но\-го модуля 
и~интегрирует его в~плат\-форму;\\[-15pt]
\item разработчик расчетных сценариев создает сценарии в~среде моделирования;\\[-15pt]
\item ученый-исследователь прикладной об\-ласти запускает расчетные сценарии 
с~различными па\-ра\-мет\-ра\-ми и~анализирует ре\-зуль\-таты.
\end{itemize}
    
    Как отмечалось в~\cite{5-ab, 6-ab}, программный комплекс предназначен для 
создания и~исполнения сценариев многомасштабных расчетов для моделирования 
структурных свойств композитных материалов.
    
    Сценарий~--- программная реализация мно\-го\-мас\-штаб\-ной композиции~--- 
пред\-став\-ля\-ет собой алгоритм последовательного выполнения расчетов отдельных 
физических характеристик материалов, входящих в~со\-став композита, 
посредством интегрированных с~программным комплексом вы\-чис\-ли\-тель\-ных 
модулей. Среда моделирования сценариев поз\-во\-ля\-ет создавать или 
модифицировать сценарии, учитывая особенности конкретного 
КМ и~тре\-бу\-емые свойства.



 
    
    Среда исполнения сценариев дает возможность осуществить его запуск 
    с~заданными входными па\-ра\-мет\-ра\-ми, отслеживать его выполнение в~целом и~по 
со\-став\-ным задачам, про\-смат\-ри\-вать входные и~выходные данные (результаты 
расчетов). Интеграционная роль среды исполнения заключается\linebreak в~формировании 
входных данных для вычислительных модулей в~со\-от\-вет\-ст\-ву\-ющем формате 
и~единицах измерения, отслеживании работы модулей,\linebreak получении конечного 
результата расчета и~преобразовании его в~формат и~единицы измерения, 
до\-ступ\-ные для других модулей сценария. Таким образом, среда исполнения 
обеспечивает соответствие потока исполнения вы\-чис\-ли\-тель\-ных модулей 
заданному алгоритму в~сценарии и~це\-лост\-ность потока данных между блоками 
сценария. Кроме того,\linebreak среда исполнения предостав\-ля\-ет общие словари для\linebreak 
согласования вход\-ных-вы\-ход\-ных данных вы\-чис\-ли\-тель\-ных экспериментов, 
такие как справочник\linebreak химических элементов и~их свойств, химических формул 
веществ, ис\-поль\-зу\-емых в~композитных материалах, типы крис\-тал\-ли\-че\-ских 
сис\-тем, типы атомных радиусов, пространственные группы.

\setcounter{figure}{3}
\begin{figure*}[b] %fig4
\vspace*{-9pt}
  \begin{center}  
    \mbox{%
\epsfxsize=163mm
\epsfbox{abg-4.eps}
}

\end{center}
\vspace*{-9pt}
\Caption{Сценарий для расчета МКМ}
\end{figure*}

\vspace*{-10pt}
   
    \subsection*{Алгоритм программы}
    
    \vspace*{-2pt}
    
    Алгоритм исполнения сценария основан на стандарте BPMN~2.0 и~со\-сто\-ит из 
сле\-ду\-ющих ключевых элементов (рис.~2).

{ \begin{center}  %fig2
 \vspace*{6pt}
    \mbox{%
\epsfxsize=70.82mm
\epsfbox{abg-2.eps}
}

\vspace*{6pt}

\noindent
{{\figurename~2}\ \ \small{
Пример простого сценария
}}
\end{center}
}

%\vspace*{6pt} 

\noindent
\begin{description}
\item[Э1.]  Точка начала выполнения сценария. В~свойствах этого элемента 
указывается список кодов физических величин, которые пользователь дол\-жен 
будет ввес\-ти перед запуском сценария.

\item[Э2.] Сплошная стрелка определяет строгую по\-сле\-до\-ва\-тель\-ность 
выполнения шагов сценария.

 \begin{figure*}[b] %fig5
  \vspace*{1pt}
  \begin{center}  
    \mbox{%
\epsfxsize=131mm %.834mm
\epsfbox{abg-5.eps}
}

\end{center}
\vspace*{-9pt}
  \Caption{Сценарий для расчета механических свойств полимерного нанокомпозита}
  \end{figure*}

\item[Э3.] Вычислительная задача пред\-став\-ля\-ет\-ся в~BPMN как <<внеш\-няя 
сервисная задача>> (External Service Task). В~поле topic в~настройках 
задачи вводится название очереди задач со\-от\-вет\-ст\-ву\-юще\-го 
вы\-чис\-ли\-тель\-но\-го модуля. Например, для  
кван\-то\-во-ме\-ха\-ни\-че\-ско\-го расчета на пакете VASP вводится 
<<vasp\_topic>>. Список до\-ступ\-ных вы\-чис\-ли\-тель\-ных модулей 
с~названиями очередей хранится в~базе данных в~таб\-ли\-це <<Module>>.\\[-15pt]

\item[Э4.] Точка завершения выполнения сценария. Если в~сценарии существует 
ветвление, точек завершения может быть несколько.



    \item[Э5.] Шаг сценария, в~рамках которого выполняется скрипт, заданный 
пользователем. В~па\-ра\-мет\-рах задачи может быть указан язык скрип\-та и~сам 
скрипт. Доступны языки Groovy и~Jython (реализация языка Python на Java). 
Скрип\-ты могут использоваться для изменения входных и~выходных па\-ра\-мет\-ров, 
небольших вы\-чис\-ле\-ний на основе текущих до\-ступ\-ных данных сценария. 
В~примере на рис.~3 в~цик\-ле определяется список векторов 
крис\-тал\-ли\-че\-ской решетки, по которым будет проводиться кван\-то\-во-ме\-ха\-ни\-че\-ский 
рас\-чет деформированной решетки.\\[-19.5pt]

\begin{figure*}[b] %fig6
\vspace*{1pt}
  \begin{center}  
    \mbox{%
\epsfxsize=163mm
\epsfbox{abg-6.eps}
}

\end{center}
\vspace*{-9pt}
\Caption{Сценарий для расчета КМ с~полимерной мат\-ри\-цей 
и~наполнителем из углеволокна}
\end{figure*}
    
    \item[Э6.] Подпроцесс сценария <<цикл с~параллельным запуском>>  
(Parallel multi-instance) позволяет параллельно запустить выполнение час\-ти 
сцена- %\linebreak\vspace*{-12pt}

\columnbreak

\noindent
рия несколько раз. В~свойствах подпроцесса требуется указать коллекцию 
(Collection), по элементам которой будет проводиться ите\-ри\-ро\-ва\-ние, и~название 
переменной цикла (Element Variable). Весь элемент считается выполненным, когда 
все параллельно выполняющиеся подпроцессы завершат свою работу. Например, 
если требуется запустить кван\-то\-во-ме\-ха\-ни\-че\-ский расчет для некоторого 
множества деформированных решеток (для определения в~дальнейшем констант 
упру\-гости), предварительно в~скрип\-те перед цик\-лом формируется список 
деформированных векторов решетки и~сохраняется в~переменную процесса. 
Далее для каж\-дой деформации параллельно вызывается\linebreak\vspace*{-12pt}

\pagebreak

\noindent  
кван\-то\-во-ме\-ха\-ни\-че\-ский модуль VASP для расчета энергии и~объема 
решетки. Получившаяся таб\-ли\-ца с~данными может использоваться для расчета 
констант элас\-тич\-ности, модуля упру\-гости и~других свойств материала.
\end{description}

\vspace*{-9pt}
  
  \subsection*{Примеры тестовых сценариев для~расчета~структурных~характеристик 
  и~отдельных~свойств различных классов 
композиционных материалов}


     
     \textbf{Пример~1.} Тестовый сценарий для расчета структурных свойств 
КМ с~металлической мат\-ри\-цей (рис.~4).

\smallskip

     
     \textbf{Пример~2.} Тестовый сценарий для расчета механических свойств 
полимерного нанокомпозита (полифениленсульфид с~углеродными нанотрубками). 
На сле\-ду\-ющем этапе проекта планируется расширить сценарий для 
оценки влияния процентного содержания углеродных нанотрубок на изменение 
коэффициента теп\-ло\-про\-вод\-ности полимерного нанокомпозита (рис.~5).
  
 
     
     \textbf{Пример~3.} Тестовый сценарий для расчета механических свойств 
КМ с~полимерной мат\-ри\-цей (эпоксидная смола) и~углеволокном
(рис.~6).

\vspace*{-6pt}

\section{Выводы}

\vspace*{-2pt}

     В работе представлен программный комплекс для расчета структурных 
характеристик КМ с~тре\-бу\-емы\-ми свойствами. В~его 
основе лежит интеграционная плат\-фор\-ма для многомасштабного моделирования, 
которая объединяет информационные потоки на разных мас\-штаб\-ных уровнях. На 
ее основе формируются схемы для рас\-че\-та структурных характеристик разных 
клас\-сов КМ: нанокомпозитов на основе полимерной 
мат\-ри\-цы, КМ с~металлической мат\-ри\-цей, полимерных 
КМ с~углеволокном и~другие. Разработанные подходы 
поз\-во\-ля\-ют моделировать свойства КМ (механические, 
теп\-ло\-вые и~др.), а~так\-же многомасштабные процессы, связанные с~усталостным 
разрушением при случайных по\-вреж\-де\-ни\-ях в~ходе эксплуатации, и~другие 
динамические процессы. Программный комплекс со\-сто\-ит из программных 
модулей и~базируется на типовых сер\-ви\-сах вы\-чис\-ли\-тель\-ных модулей, общей 
интеграционной оболочки и~модулей сценариев. Про\-грам\-мные решения 
сертифицированы. В~дальнейшем планируется раз\-ра\-бо\-тать 
полнофункциональную про\-грам\-мную сис\-те\-му с~целью решения различных 
классов обратных задач в~об\-ласти наук о~материалах. Разработанные подходы 
в~сочетании с~экспериментальными данными могут быть использованы для 
лучшего понимания физических основ изменения свойств в~за\-ви\-си\-мости от 
струк\-ту\-ры и,~как след\-ст\-вие, для уде\-шев\-ле\-ния и~уско\-ре\-ния поиска новых 
КМ с~заданными свойствами.

\vspace*{-6pt}
   
{\small\frenchspacing
 {%\baselineskip=10.8pt
 %\addcontentsline{toc}{section}{References}
 \begin{thebibliography}{9}
 
 \vspace*{-2pt}
   
   \bibitem{1-ab}
   \Au{Абгарян К.\,К.} Многомасштабное моделирование в~задачах структурного 
материаловедения.~--- М.: МАКСПресс, 2017. 284~с.
\bibitem{2-ab}
\Au{Абгарян~К.\,К.} Информационная технология по\-стро\-ения многомасштабных моделей 
в~задачах вы\-чис\-ли\-тель\-но\-го материаловедения~// Сис\-те\-мы высокой до\-ступ\-ности, 2018. Т.~14. 
№\,2. С.~9--15.
\bibitem{3-ab}
\Au{Naffakh M., D$\acute{\!\mbox{{\!\ptb{\i}}}}$ez-Pascuala~A.\,M., Marcoa~C., Ellisa~G.} Morphology and thermal properties of novel poly (phenylene sulfide) 
hybrid nanocomposites based on single-walled carbon nanotubes and 8 inorganic fullerene-like WS~2 
nanoparticles~// J.~Mater. Chem., 2012. Vol.~22. No.\,4. P.~1418--1425.
\bibitem{4-ab}
\Au{Абгарян К.\,К., Гаврилов~Е.\,С.} Распределенная информационная сис\-те\-ма для расчета 
структурных свойств композиционных материалов~// Информатика и~её применения, 2021. 
Т.~15. Вып.~4. С.~50--58. doi: 10.14357/ 19922264210407.
\bibitem{5-ab}
\Au{Гаврилов Е.\,С.} Интегрированный интерфейс к~модулю сплош\-но\-сред\-но\-го взаимодействия. 
Свидетельство о~регистрации программ для ЭВМ №\,2021681058, 2021.
\bibitem{6-ab}
\Au{Гаврилов Е.\,С.} Программные средства для хранения и~обмена данными в~задачах 
моделирования композитных материалов. Свидетельство о~регистрации программ для ЭВМ 
№\,2021681762, 2021.

\end{thebibliography}

 }
 }

\end{multicols}

\vspace*{-8pt}

\hfill{\small\textit{Поступила в~редакцию 22.01.22}}

\vspace*{8pt}

%\pagebreak

%\newpage

%\vspace*{-28pt}

\hrule

\vspace*{2pt}

\hrule

%\vspace*{-2pt}

\def\tit{SOFTWARE PACKAGE FOR MULTISCALE MODELING OF~STRUCTURAL PROPERTIES 
OF~COMPOSITE MATERIALS}


\def\titkol{Software package for multiscale modeling of~structural properties 
of~composite materials}


\def\aut{K.\,K.~Abgaryan$^1$ and~E.\,S.~Gavrilov$^{1,2}$}

\def\autkol{K.\,K.~Abgaryan and~E.\,S.~Gavrilov}

\titel{\tit}{\aut}{\autkol}{\titkol}

\vspace*{-18pt}


\noindent
$^1$Federal Research Center ``Computer Science and Control'' of the Russian Academy of Sciences, 
44-2~Vavilov\linebreak
$\hphantom{^1}$Str., Moscow 119333, Russian Federation

\noindent
$^2$Moscow Aviation Institute (National Research University), 4~Volokolamskoe Shosse, Moscow 
125080, Russian\linebreak
$\hphantom{^1}$Federation

\def\leftfootline{\small{\textbf{\thepage}
\hfill INFORMATIKA I EE PRIMENENIYA~--- INFORMATICS AND
APPLICATIONS\ \ \ 2022\ \ \ volume~16\ \ \ issue\ 1}
}%
 \def\rightfootline{\small{INFORMATIKA I EE PRIMENENIYA~---
INFORMATICS AND APPLICATIONS\ \ \ 2022\ \ \ volume~16\ \ \ issue\ 1
\hfill \textbf{\thepage}}}

\vspace*{3pt} 
      
      
  
\Abste{Today, creation of new composite materials and methods of their construction with predictable 
properties is one of the urgent and most important tasks connected with modernization of 
industrial production in our country. For their solution, technologies of multiscale computer modeling 
are actively developed. They have become a~link between fundamental physics (chemistry) and 
engineering materials science. The paper presents a~software package for modeling structural 
properties of composite materials which allows solving a~number of problems of this class. It is 
focused on high-performance computations. The complex is based on an original multiscale 
technology which allows one to promptly conduct multivariate analysis of different classes of 
composite materials and conduct research on designing the new ones with predictable properties. The 
developed approaches in combination with experimental data can be used for a~better understanding of 
the physical foundations of the change of properties depending on the structure and, as a~consequence, 
for cheaper and faster search of new composite materials with predetermined properties.}

\KWE{multiscale modeling; composite materials; integration platform; software package; distributed 
system}



\DOI{10.14357/19922264220113}

\vspace*{-16pt}

\Ack
\noindent
The research was supported by the Ministry of Science and Higher Education of the Russian 
Federation (project No.\,075-15-2020-799).




%\vspace*{4pt}

  \begin{multicols}{2}

\renewcommand{\bibname}{\protect\rmfamily References}
%\renewcommand{\bibname}{\large\protect\rm References}

{\small\frenchspacing
 {%\baselineskip=10.8pt
 \addcontentsline{toc}{section}{References}
 \begin{thebibliography}{9}
\bibitem{1-ab-1}
\Aue{Abgaryan, K.\,K.} 2017. \textit{Mnogomasshtabnoe modelirovanie v~zadachakh strukturnogo 
materialovedeniya} [Multiscale modeling for structural materials science applications]. Moscow: 
MAKS Press. 284~p.

\vspace*{-2pt}

\bibitem{2-ab-1}
\Aue{Abgaryan, K.\,K.} 2018. In\-for\-ma\-tsi\-on\-naya tekh\-no\-lo\-giya po\-stro\-eniya mno\-go\-mas\-shtab\-nykh 
mo\-de\-ley v~za\-da\-chakh vy\-chis\-li\-tel'\-no\-go ma\-te\-ri\-a\-lo\-ve\-de\-niya 
[Information technology is the construction 
of multi-scale models in problems of computational materials science]. \textit{Sistemy vysokoy 
dostupnosti} [Highly Available Systems] 14(2):9--15.
\bibitem{3-ab-1}
\Aue{Naffakh, M., A.\,M.~D$\acute{\mbox{{\!\ptb{\i}}}}$ez-Pascuala, C.~Marcoa, and G.~Ellisa.} 
2012. Morphology and thermal properties of novel poly (phenylene sulfide) hybrid nanocomposites 
based on single-walled carbon nanotubes and~8~inorganic fullerene-like WS~2 nanoparticles. 
\textit{J.~Mater. Chem.}  
22(4):1418--1425.
{\looseness=1

}
\bibitem{4-ab-1}
  \Aue{Abgaryan, K.\,K., and E.\,S.~Gavrilov.} 2021. 
  Ras\-pre\-de\-len\-naya in\-for\-ma\-tsi\-on\-naya sis\-te\-ma   dlya 
ras\-che\-ta struk\-tur\-nykh svoystv kom\-po\-zi\-tsi\-on\-nykh ma\-te\-ri\-alov 
[Distributed information system for 
calculating the structural properties of composite materials]. \textit{Informatika i~ee Primeneniya~--- 
Inform. Appl.} 15(4):50--58. doi: 10.14357/19922264210407.
\bibitem{5-ab-1}
  \Aue{Gavrilov, E.\,S.} 2021. In\-teg\-ri\-ro\-van\-nyy in\-ter\-feys k~mo\-du\-lyu 
  splosh\-no\-sred\-no\-go 
vza\-imo\-dey\-stviya [Integrated interface to the solid-medium interaction module]. Certificate on official 
registration of the computer program No.\,2021681058.
\bibitem{6-ab-1}
  \Aue{Gavrilov, E.\,S.} 2021. Pro\-gram\-mnye sred\-st\-va dlya khra\-ne\-niya 
  i~ob\-me\-na dan\-ny\-mi  v~za\-da\-chakh mo\-de\-li\-ro\-va\-niya kom\-po\-zit\-nykh ma\-te\-ri\-a\-lov 
  [Software tools for data persistence and data flow in 
composite materials modeling tasks]. Certificate on official registration of the computer program 
No.\,2021681762.
\end{thebibliography}

 }
 }

\end{multicols}

\vspace*{-6pt}

\hfill{\small\textit{Received January 22, 2022}}


\Contr

\noindent
\textbf{Abgaryan Karine K.} (b.\ 1963)~--- Doctor of Science in physics and mathematics, principal 
scientist, A.\,A.~Dorodnicyn Computing Center, Federal Research Center ``Computer Science and 
Control'' of the Russian Academy of Sciences, 40~Vavilov Str., Moscow 119333, Russian Federation; 
head of department, Moscow Aviation Institute (National Research University), 4~Volokolamskoe 
Shosse, Moscow 125080, Russian Federation; \mbox{kristal83@mail.ru}

\vspace*{3pt}

\noindent
\textbf{Gavrilov Evgeny S.} (b.\ 1982)~--- scientist, A.\,A.~Dorodnicyn Computing Center, Federal 
Research Center ``Computer Science and Control'' of the Russian Academy of Sciences, 40~Vavilov 
Str., Moscow 119333, Russian Federation; senior lecturer, Moscow Aviation Institute (National 
Research University), 4~Volokolamskoe Shosse, Moscow 125080, Russian Federation; 
\mbox{eugavrilov@gmail.com}
       

\label{end\stat}

\renewcommand{\bibname}{\protect\rm Литература}  %13


%%%%%%%%%%%%%%%%%%%%%%%%%%%%%%%%%%%%%%%%

%\def\stat{rez}
{%\hrule\par
%\vskip 7pt % 7pt
\raggedleft\Large \bf%\baselineskip=3.2ex
Р\,Е\,Ц\,Е\,Н\,З\,И\,И \vskip 17pt
    \hrule
    \par
\vskip 6pt plus 6pt minus 3pt }

%\thispagestyle{headings} %с верхним колонтитулом
%\thispagestyle{myheadings} %с нижним колонтитулом, но в верхнем РЕЦЕНЗИИ

\def\tit{НОВАЯ КНИГА И.\,Н.~СИНИЦЫНА, А.\,С.~ШАЛАМОВА <<ЛЕКЦИИ ПО ТЕОРИИ 
ИНТЕГРИРОВАННОЙ ЛОГИСТИЧЕСКОЙ ПОДДЕРЖКИ>> (М.: ТОРУС ПРЕСС, 2012. 624~с.)}

%1
\def\aut{Д.ф.-м.н., профессор С.\,Я.~Шоргин}

\def\auf{\ }

\def\leftkol{\ % РЕЦЕНЗИИ
}

\def\rightkol{ \ } 

%\def\leftkol{\ } % ENGLISH ABSTRACTS}

%\def\rightkol{\ } %ENGLISH ABSTRACTS}

%\def\leftkol{РЕЦЕНЗИИ}

%\def\rightkol{РЕЦЕНЗИИ}

\titele{\tit}{\aut}{\auf}{\leftkol}{\rightkol}
\vspace*{-18pt}


     \label{st\stat}

     \begin{multicols}{2}
     {\small
     {\baselineskip=10.1pt
     

      В книге представлено системное изложение теоретических основ одного из новейших 
направлений в \mbox{об\-ласти} экономики послепродажного обслуживания изделий наукоемкой 
продукции (ИНП) длительного пользования~--- интегрированной логистической поддержки
(ИЛП). 
{\looseness=1

}

Приведены также результаты новых работ, выполненных в Институте проблем информатики 
Российской академии наук в рамках научного направления <<Информационные технологии и 
анализ сложных сис\-тем>>.
 {%\looseness=1

}
     
      Излагаемые в книге научные подходы позво\-ляют карди\-наль\-но реформировать 
существующие системы производства и эксплуатации ИНП путем создания и внед\-ре\-ния 
методов рационального и оптимального управ\-ле\-ния процессами расходования 
вре\-мен\-н$\acute{\mbox{ы}}$х, 
мате\-ри\-аль\-ных, трудовых и других ресурсов на всех стадиях жизненного цикла изделий (ЖЦИ) по 
критериям экономической целесообразности и эф\-фек\-тив\-ности.
  {\looseness=1

}
    
      В книге приведен краткий обзор причин возник\-новения и
      развития CALS-методологии как основы 
современных международных стандартов по созданию и функционированию глобальных 
ин\-фор\-ма\-ци\-он\-но-ком\-му\-ни\-ка\-ци\-он\-ных систем, ее ключевых возможностей и эффективности 
результатов ее использования. 
Авторы %\linebreak 
предлагают ряд научных обоснований для разработки 
единой теории проектирования и управления систем ИЛП для полноценного использования 
преимуществ %\linebreak
 суще\-ст\-ву\-ющей методологии, определяют \mbox{общую} структурную схему 
комплексной системы <<ИНП-СППО>> и необходимость разработки для ее описания 
гибридных стохастических моделей.
{%\looseness=1

}

%\columnbreak
      
      Книга состоит из пяти частей, где последовательно излагается материал по каждой из 
следующих тем: <<Интегрированная логистическая поддержка>>, <<Теория гибридных 
стохастических систем и компьютерная поддержка исследований и разработок>>, <<Основы 
математического моделирования, анализа и синтеза систем послепродажного обслуживания>>, 
<<Определение и анализ показателей экспортного потенциала ИНП при проектировании>>, 
<<Задачи управления поддержкой послепродажного обслуживания>>, а также 
<<Моделирование инвестиционных процессов ИЛП в условиях неравновесных финансовых 
рынков>>. 
   
      В конце каждой главы приведены выводы и даны вопросы и задания для 
самоконтроля. В~приложениях содержатся основные определения по программам работ по 
анализу ИЛП, логистическим базам данных и компьютерным решениям, эквивалентной статистической 
линеаризации нелинейных преобразований ИЛП, справочный материал, а также развернутые 
уравнения для вероятностных характеристик.


      \def\leftkol{РЕЦЕНЗИИ}

\def\rightkol{РЕЦЕНЗИИ} 

      
      Книга заинтересует широкий круг специалистов и может быть использована научными 
проектными организациями в сфере промышленного производства ИНП. Большое количество 
иллюстраций, примеров и вопросов, обращенных к читателю, позволяет использовать книгу 
также в качестве учебного пособия для студентов и аспирантов машиностроительных, 
транспортных и~других специальностей, а также для самостоятельного изучения. 
{%\looseness=-1

}

Книга 
представляет несомненный интерес для специалистов и студентов в области прикладной 
математики и информатики.
    

}

}
\end{multicols}

%\newpage

%\def\stat{popravka}



\def\tit{ПОПРАВКА К СТАТЬЕ О.\,В.~ШЕСТАКОВА 
<<ПОРОГОВЫЕ ФУНКЦИИ В~МЕТОДАХ ПОДАВЛЕНИЯ ШУМА, ОСНОВАННЫХ~НА~ВЕЙВЛЕТ-РАЗЛОЖЕНИИ СИГНАЛА>>\\
(Информатика и её применения, 2021. Т.\ 15.  Вып.\,3. C.\ 51--56)}



\def\titkol{Поправка к статье О.\,В.~Шестакова\\
<<Пороговые функции в~методах подавления шума, основанных
на~вейвлет-разложении сигнала>>}



  \def\aut{\ }

  \def\autkol{\ } 

\titel{\tit}{\aut}{\autkol}{\titkol}

\def\leftfootline{\small{\textbf{\thepage}
\hfill INFORMATIKA I EE PRIMENENIYA~--- INFORMATICS AND
APPLICATIONS\ \ \ 2021\ \ \ volume~15\ \ \ issue\ 4}
}%
 \def\rightfootline{\small{INFORMATIKA I EE PRIMENENIYA~---
INFORMATICS AND APPLICATIONS\ \ \ 2021\ \ \ volume~15\ \ \ issue\ 4
\hfill \textbf{\thepage}}}


 \label{st\stat}

 \thispagestyle{headings}
 
 \vspace*{-24pt}  

\noindent
{\textbf{DOI:} 10.14357/19922264210307}

\vspace*{20pt}

\def\leftfootline{\small{\textbf{\thepage}
\hfill INFORMATIKA I EE PRIMENENIYA~--- INFORMATICS AND
APPLICATIONS\ \ \ 2021\ \ \ volume~15\ \ \ issue\ 4}
}%
 \def\rightfootline{\small{INFORMATIKA I EE PRIMENENIYA~---
INFORMATICS AND APPLICATIONS\ \ \ 2021\ \ \ volume~15\ \ \ issue\ 4
\hfill \textbf{\thepage}}}


%%%%%%%%%

\medskip

\noindent
С.~55, вместо 

\bigskip

\noindent
{\large ANALYSIS OF THE UNBIASED MEAN-SQUARE RISK ESTIMATE\\[6pt]
 OF~THE~BLOCK THRESHOLDING METHOD}

 



\bigskip

\noindent
должно быть

\bigskip

\noindent
{\large THRESHOLDING FUNCTIONS IN~THE~NOISE SUPPRESSION METHODS\\[6pt] 
BASED ON~THE~WAVELET EXPANSION OF~THE~SIGNAL}

 



 
\vskip 10pt plus 9pt minus 6pt

 \thispagestyle{headings}
 
 %\vspace*{-22pt}
  

\label{end\stat}

\renewcommand{\bibname}{\protect\rm Литература} 


\vspace*{8pt}

\hrule

\vspace*{2pt}

\hrule 

\vspace*{12pt}


\def\stat{popravka-1}



\def\tit{ПОПРАВКА К СТАТЬЕ А.\,А.~КУДРЯВЦЕВА, О.\,В.~ШЕСТАКОВА, С.\,Я.~ШОРГИНА
<<МЕТОД ОЦЕНИВАНИЯ ПАРАМЕТРОВ ИЗГИБА, ФОРМЫ И~МАСШТАБА
ГАММА-ЭКСПОНЕНЦИАЛЬНОГО РАСПРЕДЕЛЕНИЯ>>\\
(Информатика и её применения, 2021. Т.\ 15.  Вып.\,3. C.\ 57--62)}



\def\titkol{Поправка к статье А.\,А.~Кудрявцева, О.\,В.~Шестакова, С.\,Я.~Шоргина
<<Метод оценивания параметров изгиба, формы и масштаба
гамма-экспоненциального распределения>>}



  \def\aut{\ }

  \def\autkol{\ } 

\titel{\tit}{\aut}{\autkol}{\titkol}


 \label{st\stat}

 \thispagestyle{headings}
 
 \vspace*{-24pt}  

\noindent
{\textbf{DOI:} 10.14357/19922264210308}

\vspace*{20pt}




%%%%%%%%%

\medskip

\noindent
С.~61, вместо 

\bigskip

\noindent
{\large PROBABILISTIC CHARACTERISTICS OF~BALANCE INDEX
OF~FACTORS\\[6pt] 
WITH~GENERALIZED GAMMA DISTRIBUTION}



 



\bigskip

\noindent
должно быть

\bigskip

\noindent
{\large A METHOD FOR~ESTIMATING BENT, SHAPE AND~SCALE PARAMETERS\\[6pt] 
OF~THE~GAMMA-EXPONENTIAL DISTRIBUTION} 



 



 
\vskip 10pt plus 9pt minus 6pt

 \thispagestyle{headings}
 
 %\vspace*{-22pt}
  

\label{end\stat}

\renewcommand{\bibname}{\protect\rm Литература}  
%\include{popravka-1}

\def\stat{authorsrus}
{%\hrule\par
%\vskip 7pt % 7pt
\raggedleft\Large \bf%\baselineskip=3.2ex
О\,Б\ \ А\,В\,Т\,О\,Р\,А\,Х \vskip 17pt
    \hrule
    \par
\vskip 21pt plus 8pt minus 4pt }


\def\tit{\ }

\def\aut{\ }

\def\auf{\ }

\def\leftkol{\ } % ENGLISH ABSTRACTS}

\def\rightkol{ОБ АВТОРАХ} %ENGLISH ABSTRACTS}

\titele{\tit}{\aut}{\auf}{\leftkol}{\rightkol}
      
            \label{st\stat}



\vspace*{24pt}

\begin{multicols}{2}




\noindent
\textbf{Архипов Олег Петрович} (р.\ 1948)~---
кандидат технических наук, директор Орловского филиала Института проб\-лем информатики
Российской академии наук
%302025, г.Орел, Московское шоссе, д.137

\vspace*{3pt}

\noindent
\textbf{Бирюкова Татьяна Константиновна} (р.\ 1968)~---
кандидат фи\-зи\-ко-ма\-те\-ма\-ти\-че\-ских наук, старший научный сотрудник Института проб\-лем информатики
Российской академии наук

\vspace*{3pt}

\noindent 
\textbf{Бобков  Сергей Геннадьевич} (р.\ 1955)~---
доктор технических наук,  заведующий отделением На\-уч\-но-ис\-сле\-до\-ва\-тель\-ско\-го 
института системных исследований Российской академии наук
%117218, Москва, Нахимовский просп., 36, к.1 

\vspace*{3pt}

\noindent \textbf{Васильев Николай Семенович} (р.\ 1952)~--- доктор 
фи\-зи\-ко-ма\-те\-ма\-ти\-че\-ских наук, профессор, 
МГТУ им.\ Н.\,Э.~Баумана 
%, Москва 105005, 2-я Бауманская ул., д.~5,

\vspace*{3pt}

\noindent
\textbf{Гершкович Максим Михайлович} (р.\ 1968)~---
старший научный сотрудник Института проб\-лем информатики
Российской академии наук

\vspace*{3pt}

\noindent 
\textbf{Дьяченко Юрий Георгиевич} (р.\ 1958)~--- кандидат технических наук, 
старший научный сотрудник Института проб\-лем информатики
Российской академии наук

\vspace*{3pt}

\noindent 
\textbf{Ерошенко Александр Андреевич} (р.\ 1989)~--- аспирант кафедры 
математической статистики факультета вычисли\-тельной математики и кибернетики 
Московского государственного университета им.\ М.\,В.~Ломоносова
%119991, Москва ГСП-1, Ленинские горы, д.\ 1, стр. 52

\vspace*{3pt}
 
\noindent 
\textbf{Захаров Виктор Николаевич} (р.\ 1948)~--- 
доктор технических наук, доцент, ученый секретарь Института проб\-лем информатики
Российской академии наук

\vspace*{3pt}

\noindent
\textbf{Зейфман Александр Израилевич} (р.\ 1954)~---
доктор фи\-зи\-ко-ма\-те\-ма\-ти\-че\-ских наук, профессор, 
заведующий кафедрой Вологодского государственного университета; 
старший научный сотрудник Института проб\-лем информатики
Российской академии наук; главный научный сотрудник ИСЭРТ Российской академии наук

\vspace*{3pt}

\noindent
\textbf{Зыкин Сергей Владимирович} (р.\ 1959)~--- 
доктор технических наук, профессор, заведующий лабораторией Института математики 
им.\ С.\,Л.~Соболева Сибирского отделения Российской академии наук, Новосибирск 
%630090, пр.\ ак.\ Коптюга, 4 

\vspace*{4pt}

\noindent
\textbf{Киреев Владимир Иванович} (р.\ 1938)~---
доктор фи\-зи\-ко-ма\-те\-ма\-ти\-че\-ских наук, профессор Московского 
государственного горного университета
%Адрес: Россия, 119991, г. Москва, Ленинский проспект, д. 6

%\columnbreak

\vspace*{4pt}

\noindent
\textbf{Козеренко Елена Борисовна} (р.\ 1959)~---
кандидат филологических наук, заведующая лабораторией Института проб\-лем информатики
Российской академии наук

\vspace*{4pt}

\noindent
\textbf{Королев Виктор Юрьевич} (р.\ 1954)~--- доктор
фи\-зи\-ко-ма\-те\-ма\-ти\-че\-ских наук, профессор кафедры математической 
статистики факультета вычисли\-тельной математики и кибернетики 
Московского государственного университета; 
ведущий научный сотрудник Института проб\-лем информатики
Российской академии наук

\vspace*{4pt}

\noindent
\textbf{Коротышева Анна Владимировна} (р.\ 1988)~---
старший преподаватель Вологодского государственного университета

\vspace*{4pt}

\noindent 
\textbf{Кун Де Турк} (р.\ 1981)~--- научный сотрудник 
исследовательской группы SMACS факультета телекоммуникаций и обработки информации
Университета Гента, Бельгия
%В-9000 Гент, Бельгия

\vspace*{4pt}

\noindent
\textbf{Лупенцов Олег Сергеевич} (р.\ 1986)~---
аспирант Омского государственного института сервиса
%Омск 644043, ул.\ Певцова 13

\vspace*{4pt}

\noindent
\textbf{Лучко Олег Николаевич} (р.\ 1961)~---
кандидат педагогических наук, профессор, заведующий кафедрой 
Омского государственного института сервиса
%Омск 644043, ул.\ Певцова 13

\vspace*{4pt}

\noindent
\textbf{Малашенко Юрий Евгеньевич} (р.\ 1946)~---
доктор фи\-зи\-ко-ма\-те\-ма\-ти\-че\-ских наук, заведующий сектором 
Вычислительного центра им.\ А.\,А.~Дородницына Российской академии наук
%Адрес: 119333, Москва, ул. Вавилова, 40,

\vspace*{4pt}

\noindent
\textbf{Маньяков Юрий Анатольевич} (р.\ 1984)~---
кандидат технических наук, научный сотрудник Орловского филиала Института проб\-лем информатики
Российской академии наук
%302025, г.Орел, Московское шоссе, д.137

\vspace*{4pt}

\noindent
\textbf{Маренко Валентина Афанасьевна} (р.\ 1951)~---
кандидат технических наук, доцент, старший научный сотрудник 
Института математики им.\ С.\,Л.~Соболева Сибирского отделения Российской академии наук
%Новосибирск 630090, пр. ак. Коптюга, 4 

\vspace*{3pt}

\noindent 
\textbf{Морозов Евсей Викторович} (р.\ 1947)~--- доктор 
фи\-зи\-ко-ма\-те\-ма\-ти\-че\-ских, профессор, ведущий научный сотрудник 
Института прикладных математических исследований Карельского научного центра Российской
академии наук; 
%%185910 Россия, Республика Карелия, г.\ Петрозаводск, ул.\ Пушкинская, 11
профессор Петрозаводского государственного университета, Петрозаводск
%185910 Россия, Республика Карелия, г.\ Петрозаводск, пр.\ Ленина, 33

%\pagebreak

\vspace*{3pt}

\noindent
\textbf{Назарова Ирина Александровна} (р.\ 1966)~---
кандидат фи\-зи\-ко-ма\-те\-ма\-ти\-че\-ских наук, 
научный сотрудник Вычислительного центра им.\ А.\,А.~Дородницына Российской академии наук 
%Адрес: 119333, Москва, ул. Вавилова, 40

\vspace*{3pt}

\noindent
\textbf{Павлов Игорь Валерианович} (р.\ 1945)~--- 
доктор фи\-зи\-ко-ма\-те\-ма\-ти\-че\-ских наук, профессор МГТУ им.\ Н.\,Э.~Баумана 
%Москва 105005, 2-я Бауманская ул., д.~5 

%\pagebreak

\vspace*{3pt}

\noindent 
\textbf{Потахина Любовь Викторовна} (р.\ 1989)~--- аспирантка
Института прикладных математических исследований Карельского научного центра
Российской академии наук; 
%%185910 Россия, Республика Карелия, г.\ Петрозаводск, ул.\ Пушкинская, 11
инженер Петрозаводского государственного университета, Петрозаводск
%185910 Россия, Республика Карелия, г.\ Петрозаводск, пр.\ Ленина, 33

\vspace*{3pt}

\noindent 
\textbf{Рождественский Юрий Владимирович} (р.\ 1952)~--- 
кандидат технических наук, заведующий сектором Института проб\-лем информатики
Российской академии наук

\vspace*{3pt}

\noindent 
\textbf{Синицын Игорь Николаевич} (р.\ 1940)~--- доктор технических наук,
профессор, заслуженный деятель\linebreak\vspace*{-12pt}

\columnbreak

\noindent
 науки РФ, заведующий отделом Института проб\-лем информатики
Российской академии наук

\vspace*{7pt}


\noindent
\textbf{Сиротинин Денис Олегович} (р.\ 1984)~---
кандидат технических наук, научный сотрудник Орловского филиала Института проб\-лем информатики
Российской академии наук
%302025, г.Орел, Московское шоссе, д.137

\vspace*{7pt}

%\columnbreak

\noindent 
\textbf{Соколов  Игорь Анатольевич} (р.\ 1954)~--- академик (действительный член) Российской 
академии наук, доктор технических наук, директор Института проб\-лем информатики
Российской академии наук

\vspace*{7pt}

\noindent
\textbf{Степченков Юрий Афанасьевич} (р.\ 1951)~---
кандидат технических наук, заведующий отделом Института проб\-лем информатики
Российской академии наук

\vspace*{7pt}

\noindent
\textbf{Сурков Алексей Викторович} (р.\ 1978)~--- 
старший научный сотрудник На\-уч\-но-ис\-сле\-до\-ва\-тель\-ско\-го 
института системных исследований Российской академии наук
%117218, Москва, Нахимовский просп., 36, к.1 

\vspace*{7pt}

\noindent 
\textbf{Шестаков Олег Владимирович} (р.\ 1976)~--- доктор 
фи\-зи\-ко-ма\-те\-ма\-ти\-че\-ских, доцент кафедры математической статистики 
факультета вычисли\-тельной математики и кибернетики Московского 
государственного университета им.\ М.\,В.~Ломоносова; 
%119991, Москва ГСП-1, Ленинские горы, д.\ 1, стр. 52
старший научный сотрудник Института проб\-лем информатики
Российской академии наук
%, Москва 119333, ул. Вавилова, д.~44, корп.~2

\vspace*{7pt}

\noindent 
\textbf{Шоргин Сергей Яковлевич} (р.\ 1952.)~--- доктор
фи\-зи\-ко-ма\-те\-ма\-ти\-че\-ских наук, профессор, заместитель директора Института 
проб\-лем информатики Российской академии наук





%%%%%%%%%%%%%%%%%%%%%%%%%%%%%%%%%%%%%%%%%%%%%%%%%%%%%%%%%%%%%%%%%%%%%%%%%%%%%%%




%\def\rightkol{ОБ АВТОРАХ}
%\def\leftkol{ОБ АВТОРАХ}

 \label{end\stat}





%\def\leftfootline{\small{\textbf{\thepage}
%\hfill ИНФОРМАТИКА И ЕЁ ПРИМЕНЕНИЯ\ \ \ том~7\ \ \ выпуск~1\ \ \ 2013}
%}%
% \def\rightfootline{\small{ИНФОРМАТИКА И ЕЁ ПРИМЕНЕНИЯ\ \ \ том~7\ \ \ выпуск~1\ \ \ 2013
%\hfill \textbf{\thepage}}}


%\thispagestyle{myheadings}



\end{multicols}

\newpage  

%\def\stat{cont}
{%\hrule\par
%\vskip 7pt % 7pt
\raggedleft\Large \bf%\baselineskip=3.2ex
А\,В\,Т\,О\,Р\,С\,К\,И\,Й\ \ У\,К\,А\,З\,А\,Т\,Е\,Л\,Ь\ \ З\,А\ \ 2\,0\,0\,7 г. \vskip 17pt
    \hrule
    \par
\vskip 21pt plus 6pt minus 3pt }

\label{st\stat}

\def\tit{\ }

\def\aut{\ }
\def\auf{\ }

\def\leftkol{\ } % ENGLISH ABSTRACTS}

\def\rightkol{\ } %ENGLISH ABSTRACTS}

\titele{\tit}{\aut}{\auf}{\leftkol}{\rightkol}


\contentsline {chapter}{\ }{Выпуск \quad Стр.} 
\contentsline {section}{\textbf{Батракова Д.\,А., Королев В.\,Ю., Шоргин С.\,Я.}\ \ Новый метод вероятностно-ста\-ти\-сти\-че\-ско\-го анализа информационных потоков в\nobreakspace {}телекоммуникационных сетях}{\qquad 1 \qquad 40} 
\contentsline {section}{\textbf{Борисов А.\,В.}\ \ Байесовское оценивание в системах наблюдения с\nobreakspace {}марковскими скачкообразными процессами: игровой подход}{\qquad 2 \qquad 65}
\contentsline {section}{\textbf{Босов А.\,В., Иванов А.\,В.}\ \ Программная инфраструктура информационного Web-пор\-тала}{\qquad 2 \qquad 50}
\contentsline {section}{\textbf{Захаров В.\,Н., Калиниченко Л.\,А., Соколов И.\,А., Ступников С.\,А.}\ \ Конструирование канонических информационных моделей для интегрированных информационных систем}{\qquad 2 \qquad 15}
\contentsline {section}{\textbf{Захаров В.\,Н., Козмидиади В.\,А.}\ \ Средства обеспечения отказоустойчивости при\-ло\-жений}{\qquad 1 \qquad 14} 
\contentsline {section}{\textbf{Иванов А.\,В.}\ \ см. Босов А.\,В.\hfill\hfill\hfill\hfill\hfill\hfill\hfill\hfill\hfill\hfill\hfill\hfill\hfill\hfill\hfill\hfill\hfill\hfill\hfill\hfill\hfill\hfill\hfill\hfill\hfill\hfill\hfill\hfill\hfill\hfill\hfill\hfill\hfill\hfill\hfill}{\ }
\contentsline {section}{\textbf{Ильин В.\,Д., Соколов И.\,А.}\ \ Символьная модель системы знаний информатики в\nobreakspace {}че\-ло\-ве\-ко-автоматной среде}{\qquad 1 \qquad 66} 
\contentsline {section}{\textbf{Калиниченко Л.\,А.}\ \ см. Захаров В.\,Н.\hfill\hfill\hfill\hfill\hfill\hfill\hfill\hfill\hfill\hfill\hfill\hfill\hfill\hfill\hfill\hfill\hfill\hfill\hfill\hfill\hfill\hfill\hfill\hfill\hfill\hfill\hfill\hfill\hfill\hfill\hfill\hfill\hfill\hfill\hfill}{\ }
\contentsline {section}{\textbf{Козеренко Е.\,Б.}\ \ Лингвистическое моделирование для систем машинного перевода и обработки знаний}{\qquad 1 \qquad 54} 
\contentsline {section}{\textbf{Козмидиади В.\,А.}\ \ см. Захаров В.\,Н.\hfill\hfill\hfill\hfill\hfill\hfill\hfill\hfill\hfill\hfill\hfill\hfill\hfill\hfill\hfill\hfill\hfill\hfill\hfill\hfill\hfill\hfill\hfill\hfill\hfill\hfill\hfill\hfill\hfill\hfill\hfill\hfill\hfill\hfill\hfill }{\ } 
\contentsline {section}{\textbf{Королев В.\,Ю.}\ \ см. Батракова Д.\,А.\hfill\hfill\hfill\hfill\hfill\hfill\hfill\hfill\hfill\hfill\hfill\hfill\hfill\hfill\hfill\hfill\hfill\hfill\hfill\hfill\hfill\hfill\hfill\hfill\hfill\hfill\hfill\hfill\hfill\hfill\hfill\hfill\hfill\hfill\hfill}{\ } 
\contentsline {section}{\textbf{Кудрявцев А.\,А., Шоргин С.\,Я.}\ \ Байесовский подход к\nobreakspace {}анализу систем массового обслуживания и\nobreakspace {}показателей надежности}{\qquad 2 \qquad 76}
\contentsline {section}{\textbf{Печинкин А.\,В., Соколов И.\,А., Чаплыгин В.\,В.}\ \ Многолинейная система массового обслуживания с конечным накопителем и ненадежными приборами}{\qquad 1 \qquad 27} 
\contentsline {section}{\textbf{Печинкин А.\,В., Соколов И.\,А., Чаплыгин В.\,В.}\ \ Стационарные характеристики многолинейной\nobreakspace {}системы массового обслуживания с\nobreakspace {}одновременными отказами приборов}{\qquad 2 \qquad 39}
\contentsline {section}{\textbf{Синицын И.\,Н.}\ \ Корреляционные методы построения аналитических информационных моделей флуктуаций полюса Земли по априорным данным}{\qquad 2 \qquad \hphantom{9}2}
\contentsline {section}{\textbf{Синицын И.\,Н.}\ \ Развитие теории фильтров Пугачева для оперативной обработки информации в стохастических системах}{{\qquad 1 \qquad \hphantom{9}3}} 
\contentsline {section}{\textbf{Соколов И.\,А.}\ \ см. Захаров В.\,Н.\hfill\hfill\hfill\hfill\hfill\hfill\hfill\hfill\hfill\hfill\hfill\hfill\hfill\hfill\hfill\hfill\hfill\hfill\hfill\hfill\hfill\hfill\hfill\hfill\hfill\hfill\hfill\hfill\hfill\hfill\hfill\hfill\hfill\hfill\hfill}{\ }
\contentsline {section}{\textbf{Соколов И.\,А.}\ \ см. Ильин В.\,Д.\hfill\hfill\hfill\hfill\hfill\hfill\hfill\hfill\hfill\hfill\hfill\hfill\hfill\hfill\hfill\hfill\hfill\hfill\hfill\hfill\hfill\hfill\hfill\hfill\hfill\hfill\hfill\hfill\hfill\hfill\hfill\hfill\hfill\hfill\hfill}{\ } 
\contentsline {section}{\textbf{Соколов И.\,А.}\ \ см. Печинкин А.\,В.\hfill\hfill\hfill\hfill\hfill\hfill\hfill\hfill\hfill\hfill\hfill\hfill\hfill\hfill\hfill\hfill\hfill\hfill\hfill\hfill\hfill\hfill\hfill\hfill\hfill\hfill\hfill\hfill\hfill\hfill\hfill\hfill\hfill\hfill\hfill}{\ } 
\contentsline {section}{\textbf{Соколов И.\,А.}\ \ см. Печинкин А.\,В.\hfill\hfill\hfill\hfill\hfill\hfill\hfill\hfill\hfill\hfill\hfill\hfill\hfill\hfill\hfill\hfill\hfill\hfill\hfill\hfill\hfill\hfill\hfill\hfill\hfill\hfill\hfill\hfill\hfill\hfill\hfill\hfill\hfill\hfill\hfill}{\ }
\contentsline {section}{\textbf{Ступников С.\,А.}\ \ см. Захаров В.\,Н.\hfill\hfill\hfill\hfill\hfill\hfill\hfill\hfill\hfill\hfill\hfill\hfill\hfill\hfill\hfill\hfill\hfill\hfill\hfill\hfill\hfill\hfill\hfill\hfill\hfill\hfill\hfill\hfill\hfill\hfill\hfill\hfill\hfill\hfill\hfill}{\ }
\contentsline {section}{\textbf{Чаплыгин В.\,В.}\ \ см. Печинкин А.\,В.\hfill\hfill\hfill\hfill\hfill\hfill\hfill\hfill\hfill\hfill\hfill\hfill\hfill\hfill\hfill\hfill\hfill\hfill\hfill\hfill\hfill\hfill\hfill\hfill\hfill\hfill\hfill\hfill\hfill\hfill\hfill\hfill\hfill\hfill\hfill}{\ } 
\contentsline {section}{\textbf{Чаплыгин В.\,В.}\ \ см. Печинкин А.\,В.\hfill\hfill\hfill\hfill\hfill\hfill\hfill\hfill\hfill\hfill\hfill\hfill\hfill\hfill\hfill\hfill\hfill\hfill\hfill\hfill\hfill\hfill\hfill\hfill\hfill\hfill\hfill\hfill\hfill\hfill\hfill\hfill\hfill\hfill\hfill}{\ }
\contentsline {section}{\textbf{Шоргин С.\,Я.}\ \ см. Батракова Д.\,А.\hfill\hfill\hfill\hfill\hfill\hfill\hfill\hfill\hfill\hfill\hfill\hfill\hfill\hfill\hfill\hfill\hfill\hfill\hfill\hfill\hfill\hfill\hfill\hfill\hfill\hfill\hfill\hfill\hfill\hfill\hfill\hfill\hfill\hfill\hfill}{\ } 
\contentsline {section}{\textbf{Шоргин С.\,Я.}\ \ см. Кудрявцев А.\,А.\hfill\hfill\hfill\hfill\hfill\hfill\hfill\hfill\hfill\hfill\hfill\hfill\hfill\hfill\hfill\hfill\hfill\hfill\hfill\hfill\hfill\hfill\hfill\hfill\hfill\hfill\hfill\hfill\hfill\hfill\hfill\hfill\hfill\hfill\hfill}{\ }
%\thispagestyle{myheadings}
\def\leftfootline{\small{\textbf{\thepage}
\hfill ИНФОРМАТИКА И ЕЁ ПРИМЕНЕНИЯ\ \ \ том~1\ \ \ выпуск~2\ \ \ 2007}
}%
 \def\rightfootline{\small{ИНФОРМАТИКА И ЕЁ ПРИМЕНЕНИЯ\ \ \ том~1\ \ \ выпуск~2\ \ \ 2007
 \hfill \textbf{\thepage}}}
 \label{end\stat} 
                     
%\def\stat{cont-e}
{%\hrule\par
%\vskip 7pt % 7pt
\raggedleft\Large \bf%\baselineskip=3.2ex
2\,0\,0\,7\ \ A\,U\,T\,H\,O\,R\ \ I\,N\,D\,E\,X \vskip 17pt
    \hrule
    \par
\vskip 21pt plus 6pt minus 3pt }

\label{st\stat}

\def\tit{\ }

\def\aut{\ }
\def\auf{\ }

\def\leftkol{\ } % ENGLISH ABSTRACTS}

\def\rightkol{\ } %ENGLISH ABSTRACTS}

\titele{\tit}{\aut}{\auf}{\leftkol}{\rightkol}


\contentsline {chapter}{\ }{Issue \quad Page} 
\contentsline {subsection}{\textbf{Batrakova D.\,A., Korolev V.\,Yu., Shorgin S.\,Ya.}\ \ A New Method for the Probabilistic and Statistical Analysis of Information Flows in Telecommunication Networks}{\qquad 1 \qquad 40} 
\contentsline {subsection}{\textbf{Borisov A.\,V.}\ \ Bayesian Estimation in\nobreakspace {}Observation Systems with\nobreakspace {}Markov Jump Processes: Game-Theoretic Approach}{\qquad 2 \qquad 65} 
\contentsline {subsection}{\textbf{Bosov A.\,V., Ivanov A.\,V.}\ \ Linguistic Simulation for Machine Translation and Knowledge Management Systems}{\qquad 2 \qquad 50} 
\contentsline {subsection}{\textbf{Chaplygin V.\,V.} see Pechinkin A.\,V.\hfill\hfill\hfill\hfill\hfill\hfill\hfill\hfill\hfill\hfill\hfill\hfill\hfill\hfill\hfill\hfill\hfill\hfill\hfill\hfill\hfill\hfill\hfill\hfill\hfill\hfill\hfill\hfill\hfill\hfill\hfill\hfill\hfill\hfill\hfill}{\ }
\contentsline {subsection}{\textbf{Chaplygin V.\,V.} see Pechinkin A.\,V.\hfill\hfill\hfill\hfill\hfill\hfill\hfill\hfill\hfill\hfill\hfill\hfill\hfill\hfill\hfill\hfill\hfill\hfill\hfill\hfill\hfill\hfill\hfill\hfill\hfill\hfill\hfill\hfill\hfill\hfill\hfill\hfill\hfill\hfill\hfill}{\ }
\contentsline {subsection}{\textbf{Ilyin V.\,D., Sokolov I.\,A.}\ \ The Symbol Model of Informatics Knowledge System in Human-Automaton Environment}{\qquad 1 \qquad 66} 
\contentsline {subsection}{\textbf{Ivanov A.\,V.} see Bosov A.\,V.\hfill\hfill\hfill\hfill\hfill\hfill\hfill\hfill\hfill\hfill\hfill\hfill\hfill\hfill\hfill\hfill\hfill\hfill\hfill\hfill\hfill\hfill\hfill\hfill\hfill\hfill\hfill\hfill\hfill\hfill\hfill\hfill\hfill\hfill\hfill}{\ }
\contentsline {subsection}{\textbf{Kalinichenko L.\,A.} see Zakharov V.\,N.\hfill\hfill\hfill\hfill\hfill\hfill\hfill\hfill\hfill\hfill\hfill\hfill\hfill\hfill\hfill\hfill\hfill\hfill\hfill\hfill\hfill\hfill\hfill\hfill\hfill\hfill\hfill\hfill\hfill\hfill\hfill\hfill\hfill\hfill\hfill}{\ }
\contentsline {subsection}{\textbf{Korolev V.\,Yu.} see Batrakova D.\,A.\hfill\hfill\hfill\hfill\hfill\hfill\hfill\hfill\hfill\hfill\hfill\hfill\hfill\hfill\hfill\hfill\hfill\hfill\hfill\hfill\hfill\hfill\hfill\hfill\hfill\hfill\hfill\hfill\hfill\hfill\hfill\hfill\hfill\hfill\hfill}{\ }
\contentsline {subsection}{\textbf{Kozerenko E.\,B.}\ \ Linguistic Simulation for Machine Translation and Knowledge Management Systems}{\qquad 1 \qquad 54} 
\contentsline {subsection}{\textbf{Kozmidiady V.\,A.} see Zakharov V.\,N.\hfill\hfill\hfill\hfill\hfill\hfill\hfill\hfill\hfill\hfill\hfill\hfill\hfill\hfill\hfill\hfill\hfill\hfill\hfill\hfill\hfill\hfill\hfill\hfill\hfill\hfill\hfill\hfill\hfill\hfill\hfill\hfill\hfill\hfill\hfill}{\ }
\contentsline {subsection}{\textbf{Kudryavtsev A.\,A., Shorgin S.\,Ya.}\ \ Bayesian Approach to Queueing Systems and Reliability Characteristics}{\qquad 2 \qquad 76} 
\contentsline {subsection}{\textbf{Pechinkin A.\,V., Sokolov I.\,A., Chaplygin V.\,V.}\ \ Multichannel Queuing System with Finite Buffer and Unreliable Servers}{\qquad 1 \qquad 27} 
\contentsline {subsection}{\textbf{Pechinkin A.\,V., Sokolov I.\,A., Chaplygin V.\,V.}\ \ Stationary Characteristics of a Multichannel Queueing System with\nobreakspace {}Simultaneous Refusals of Servers}{\qquad 2 \qquad 39} 
\contentsline {subsection}{\textbf{Shorgin S.\,Ya.} see Batrakova D.\,A.\hfill\hfill\hfill\hfill\hfill\hfill\hfill\hfill\hfill\hfill\hfill\hfill\hfill\hfill\hfill\hfill\hfill\hfill\hfill\hfill\hfill\hfill\hfill\hfill\hfill\hfill\hfill\hfill\hfill\hfill\hfill\hfill\hfill\hfill\hfill}{\ }
\contentsline {subsection}{\textbf{Shorgin S.\,Ya.} see Kudryavtsev A.\,A.\hfill\hfill\hfill\hfill\hfill\hfill\hfill\hfill\hfill\hfill\hfill\hfill\hfill\hfill\hfill\hfill\hfill\hfill\hfill\hfill\hfill\hfill\hfill\hfill\hfill\hfill\hfill\hfill\hfill\hfill\hfill\hfill\hfill\hfill\hfill}{\ }
\contentsline {subsection}{\textbf{Sinitsyn I.\,N.}\ \ Correlational Methods for Analytical Informational Models of the Earth Pole Fluctuations Design Based on a priori Data}{\qquad 2 \qquad \hphantom{9}2}
\contentsline {subsection}{\textbf{Sinitsyn I.\,N.}\ \ Development of Pugachev Filtering for Stochastic Systems}{\qquad 1 \qquad \hphantom{9}3}
\contentsline {subsection}{\textbf{Sokolov I.\,A.} see Ilyin V.\,D.\hfill\hfill\hfill\hfill\hfill\hfill\hfill\hfill\hfill\hfill\hfill\hfill\hfill\hfill\hfill\hfill\hfill\hfill\hfill\hfill\hfill\hfill\hfill\hfill\hfill\hfill\hfill\hfill\hfill\hfill\hfill\hfill\hfill\hfill\hfill}{\ }
\contentsline {subsection}{\textbf{Sokolov I.\,A.} see Pechinkin A.\,V.\hfill\hfill\hfill\hfill\hfill\hfill\hfill\hfill\hfill\hfill\hfill\hfill\hfill\hfill\hfill\hfill\hfill\hfill\hfill\hfill\hfill\hfill\hfill\hfill\hfill\hfill\hfill\hfill\hfill\hfill\hfill\hfill\hfill\hfill\hfill}{\ }
\contentsline {subsection}{\textbf{Sokolov I.\,A.} see Pechinkin A.\,V.\hfill\hfill\hfill\hfill\hfill\hfill\hfill\hfill\hfill\hfill\hfill\hfill\hfill\hfill\hfill\hfill\hfill\hfill\hfill\hfill\hfill\hfill\hfill\hfill\hfill\hfill\hfill\hfill\hfill\hfill\hfill\hfill\hfill\hfill\hfill}{\ }
\contentsline {subsection}{\textbf{Sokolov I.\,A.} see Zakharov V.\,N.\hfill\hfill\hfill\hfill\hfill\hfill\hfill\hfill\hfill\hfill\hfill\hfill\hfill\hfill\hfill\hfill\hfill\hfill\hfill\hfill\hfill\hfill\hfill\hfill\hfill\hfill\hfill\hfill\hfill\hfill\hfill\hfill\hfill\hfill\hfill}{\ }
\contentsline {subsection}{\textbf{Stupnikov S.\,A.} see Zakharov V.\,N.\hfill\hfill\hfill\hfill\hfill\hfill\hfill\hfill\hfill\hfill\hfill\hfill\hfill\hfill\hfill\hfill\hfill\hfill\hfill\hfill\hfill\hfill\hfill\hfill\hfill\hfill\hfill\hfill\hfill\hfill\hfill\hfill\hfill\hfill\hfill}{\ }
\contentsline {subsection}{\textbf{Zakharov V.\,N., Kalinichenko L.\,A., Sokolov I.\,A., Stupnikov S.\,A.}\ \ Development of Canonical Information Models for Integrated Information Systems}{\qquad 2 \qquad 15} 
\contentsline {subsection}{\textbf{Zakharov V.\,N., Kozmidiady V.\,A.}\ \ Means Providing Applications Fault Tolerance}{\qquad 1 \qquad 14} 
\def\leftfootline{\small{\textbf{\thepage}
\hfill ИНФОРМАТИКА И ЕЁ ПРИМЕНЕНИЯ\ \ \ том~1\ \ \ выпуск~2\ \ \ 2007}
}%
 \def\rightfootline{\small{ИНФОРМАТИКА И ЕЁ ПРИМЕНЕНИЯ\ \ \ том~1\ \ \ выпуск~2\ \ \ 2007
 \hfill \textbf{\thepage}}}
 \label{end\stat} 


%\end{document}

%
\def\stat{rekl}
%\label{preobr}

%\def\tit{АКАДЕМИК ПУГАЧЁВ  ВЛАДИМИР СЕМЁНОВИЧ\\
%25.03.1911--25.03.1998}


%   \vspace*{-48pt}
%   \begin{center}\LARGE
%Академик Пугачёв  Владимир Семёнович\\ (25.03.1911--25.03.1998)
%   \end{center}

   %\vspace*{2.5mm}

   \begin{center}

{\prgsh\LARGE
ЮБИЛЕИ}

\end{center}
%\hrule

\vspace*{6pt}


   \vspace*{8mm}

   \thispagestyle{empty}


%\def\stat{emel}


\section*{К 70-летию заместителя директора ИПИ РАН,\\ члена редколлегии журнала
<<Информатика и её применения>>\\ доктора технических наук В.\,И.~Будзко}

\vspace*{18pt}




          \begin{multicols}{2}

%            \label{st\stat}

\begin{center}
\vspace*{1pt}
\mbox{%
\epsfxsize=78mm
\epsfbox{bud-1.eps}
}
\end{center}

\vspace*{12pt}

      14 августа 2014~г.\ исполнилось 70~лет за\-мес\-ти\-те\-лю директора ИПИ РАН по
научной работе доктору технических наук Владимиру Игоревичу Будзко.

      Владимир Игоревич Будзко родился в г.~Москве. Высшее образование получил на факультете
элект\-рон\-но-вы\-чис\-ли\-тель\-ных устройств в Московском
ин\-же\-нер\-но-фи\-зи\-че\-ском институте
(МИФИ), который он окончил в 1968~г., после чего был на\-прав\-лен для прохождения
службы в одну из войс\-ко\-вых частей, где прошел путь от инженера до первого заместителя
командира войсковой части.

      С приходом В.\,И.~Будзко в ИПИ РАН (2001~г.)\ в институте
сформировалось новое научное на\-прав\-ле\-ние теоретических исследований~--- <<Постро\-ение
ин\-фор\-ма\-ци\-он\-но-те\-ле\-ком\-му\-ни\-ка\-ци\-он\-ных\linebreak сис\-тем
высокой до\-ступ\-ности>>. В~рамках этого
направления выполнен широкий круг фундаментальных исследований по поиску подходов и
определению принципов построения средств обеспечения доступности, конфиденциальности
и целостности современных крупномасштабных
ин\-фор\-ма\-ци\-он\-но-те\-ле\-ком\-му\-ни\-ка\-ци\-он\-ных
сис\-тем (ИТС). Разработаны основные сис\-тем\-но-тех\-ни\-че\-ские принципы и базовые
архитектурные решения построения перспективных для условий России ИТС с
централизованной обработкой и хранением информации, сочетающих в себе свойства
высокой доступности, отказо- и катастрофоустойчивости, информационной защищенности.
Определены принципы, методы и математические основы рационального построения и
оптимизации средств восстановления функционирования центров обработки данных (ЦОД)
после возникновения отказов и катастроф, передачи и хранения данных, обеспечения
информационной безопасности при достижении минимальной совокупной стоимости
владения такими системами. Результаты нашли практическое воплощение при реализации
проектов в интересах ряда отечественных государственных и негосударственных
организаций, таких как Банк России (БР), Внешторгбанк, ОАО <<ГМК <<Норильский Никель>>,
<<Газпром>>, Минэкономразвития России, Правительство Москвы, а также ряд силовых
ведомств.

      Под руководством В.\,И.~Будзко начиная с 2001~г.\ выполнен комплекс
      на\-уч\-но-ис\-сле\-до\-ва\-тель\-ских и
      опыт\-но-кон\-ст\-рук\-тор\-ских работ (свыше 100~проектов),
направленных на развитие электронной информационной технологии БР.
Разработаны концепции развития ИТС БР сначала до 2008~г., а затем до 2013~г., которые
были приняты в качестве основы проведения технической политики. За реализацию проекта
<<Катастрофоустойчивая тер\-ри\-то\-ри\-аль\-но-рас\-пре\-де\-лен\-ная
      ин\-фор\-ма\-ци\-он\-но-те\-ле\-ком\-му\-ни\-ка\-ци\-он\-ная сис\-те\-ма централизованной
обработки банковской информации>> В.\,И.~Будзко удостоен Премии Правительства РФ в
области науки и техники за 2010~г.

      В.\,И.~Будзко возглавлял и возглавляет работы по ряду других прикладных проектов,
связанных с созданием, совершенствованием и развитием крупномасштабных ИТС.

      В.\,И.~Будзко~--- генерал-майор, доктор технических наук, член-кор\-рес\-пон\-дент
Академии криптографии РФ, известный ученый в области информатики и применения
информационных технологий при построении территориально распределенных ИТС
различного назначения. Является автором свыше 250~научных работ, опубликованных в
на\-уч\-но-тех\-ни\-че\-ских и специальных изданиях.

    \thispagestyle{empty}

      В.\,И.~Будзко уделяет большое внимание подготовке научных кадров. Под его
руководством защищено 6~диссертаций на соискание ученой степени кандидата
технических наук. Свыше 30~лет он читает лекции в ИКСИ Академии ФСБ, профессор
кафедры НИЯУ МИФИ. Является членом двух диссертационных советов, главным
редактором журнала <<Системы высокой доступности>> и членом редколлегии журнала
<<Информатика и её применения>>.

      \bigskip

      Редакционный совет и Редакционная коллегия журнала <<Информатика и её
применения>> сердечно поздравляют Владимира Игоревича Будзко с 70-ле\-ти\-ем и желают
крепкого здоровья и новых научных достижений.

\end{multicols}

%%Информатика и её применения
%Том 14 Выпуск 1-4 Год 2020

\def\stat{cont}
{%\hrule\par
%\vskip 7pt % 7pt
\raggedleft\Large \bf%\baselineskip=3.2ex
А\,В\,Т\,О\,Р\,С\,К\,И\,Й\ \ У\,К\,А\,З\,А\,Т\,Е\,Л\,Ь\ \ З\,А\ \ 2\,0\,2\,0 г. \vskip 17pt
 \hrule
 \par
\vskip 21pt plus 6pt minus 3pt }

\label{st\stat}

\def\tit{\ }

\def\aut{\ }
\def\auf{\ }

\def\leftkol{\ } % ENGLISH ABSTRACTS}

\def\rightkol{\ } %АВТОРСКИЙ УКАЗАТЕЛЬ ЗА 2020 г.} %ENGLISH ABSTRACTS}

\titele{\tit}{\aut}{\auf}{\leftkol}{\rightkol}
\addcontentsline{toc}{subsection}{\textrm\textbf Авторский указатель за 2020 г.}

\vspace*{-24pt}

\noindent
{\tabcolsep=3pt
\begin{tabular}{p{397pt}cc}
&\textbf{Вып.} & \textbf{Стр.}\\[6pt]
\Avtors{Абгарян~К.\,К., Гаврилов~Е.\,С.} Интеграционная платформа для многомасштабного моде-\linebreak
\\[-12pt]
\hspace*{23pt}лирования нейроморфных систем&2&104--110\\
\Avtors{Абгарян~К.\,К., Колбин~И.\,С.} Применение многомасштабного подхода и методов анализа\linebreak
\\[-12pt]
\hspace*{23pt}данных для моделирования теплопроводности в слоистых структурах&4&91--99\\
\Avtors{Агаларов~Я.\,М.} Оптимизация емкости основного накопителя в системе массового\linebreak
\\[-12pt]
\hspace*{23pt}обслуживания типа $G/M/1/K$ с дополнительным накопителем&2&72--79\\
\Avtors{Агасандян~Г.\,А.} Вычислительные аспекты применения CC-VaR на совокупности рынков&3&62--70\\
\Avtors{Агеев~К.\,А., Сопин~Э.\,С., Яркина~Н.\,В., Самуйлов~К.\,Е., Шоргин~С.\,Я.} Анализ механизмов\linebreak
\\[-12pt]
\hspace*{23pt}нарезки сети с учетом гарантий для различных типов трафика&3&\hphantom{1}94--100\\
\Avtors{Адамова~К.\,А.} см.\ Шнурков~П.\,В.&&\\
\Avtors{Базилевский~М.\,П.} Многофакторные модели полносвязной линейной регрессии без\linebreak
\\[-12pt]
\hspace*{23pt}ограничений на соотношения дисперсий ошибок переменных&2&92--97\\
\Avtors{Бахтеев~О.\,Ю.} см.\ Грабовой~А.\,В.&&\\
\Avtors{Беленков~В.\,Г.} см.\ Будзко~В.\,И.&&\\
\Avtors{Бетелин~В.\,Б., Кушниренко~А.\,Г., Леонов~А.\,Г.} Основные понятия программирования\linebreak
\\[-12pt]
\hspace*{23pt}в изложении для дошкольников&3&55--61\\
\Avtors{Бетелин~В.\,Б., Кушниренко~А.\,Г., Семенов~А.\,Л., Сопрунов~С.\,Ф.} О цифровой грамотности\linebreak
\\[-12pt]
\hspace*{23pt}и средах ее формирования&4&100--107\\
\Avtors{Борисов~А.\,В.} Численные схемы фильтрации марковских скачкообразных процессов по\linebreak
\\[-12pt]
\hspace*{23pt}дискретизованным наблюдениям II: случай аддитивных шумов&1&17--23\\
\Avtors{Борисов~А.\,В.} Численные схемы фильтрации марковских скачкообразных процессов по\linebreak
\\[-12pt]
\hspace*{23pt}дискретизованным наблюдениям III: случай мультипликативных шумов&2&10--18\\
\Avtors{Босов~А.\,В.} Управление выходом стохастической дифференциальной системы по квад-\linebreak
\\[-12pt]
\hspace*{23pt}ратичному критерию. V. Случай неполной информации о состоянии&2&19--25\\
\Avtors{Босов~А.\,В., Мартюшова~Я.\,Г., Наумов~А.\,В., Сапунова~А.\,П.} Байесовский подход к~по\-стро\-ению индивидуальной траектории пользователя в~системе дистанционного\linebreak
\\[-12pt]
\hspace*{23pt}обучения&3&86--93\\
\Avtors{Босов~А.\,В., Стефанович~А.\,И.} Управление выходом стохастической дифференциальной\linebreak
\\[-12pt]
\hspace*{23pt}системы по квадратичному критерию. IV. Альтернативное численное решение&1&24--30\\
\Avtors{Брюхов~Д.\,О., Ступников~С.\,А., Ковалёв~Д.\,Ю., Шанин~И.\,А.} Нейрофизиология как\linebreak
\\[-12pt]
\hspace*{23pt}предметная область для решения задач с интенсивным использованием данных&1&40--47\\
\Avtors{Будзко~В.\,И., Ядринцев~В.\,В., Соченков~И.\,В., Королёв~В.\,И., Беленков~В.\,Г.} Об одном подходе
 к формированию в условиях высокой неопределенности марке-\linebreak
\\[-12pt]
\hspace*{23pt}ров конфиденциальности в системах интенсивного использования данных&4&69--76\\
\Avtors{Вайсер~К.\,О.} см.\ Потанин~М.\,С.&&\\
\Avtors{Вохминцев~А.\,В., Мельников~А.\,В., Пачганов~C.\,А.} Метод навигации и составления карты в трехмерном пространстве на основе комбинированного решения вариационной\linebreak
\\[-12pt]
\hspace*{23pt}подзадачи точка--точка ICP для аффинных преобразований&1&101--112\\
\Avtors{Гаврилов~Е.\,С.} см.\ Абгарян~К.\,К.&&\\
\Avtors{Гайдамака~Ю.\,В.} см.\  Москалева~Ф.\,А.&&\\
\Avtors{Голембиовский~Д.\,Ю.} см.\ Данилишин~А.\,Р.&&\\
\Avtors{Голембиовский~Д.\,Ю.} см.\ Данилишин~А.\,Р.&&\\
\Avtors{Гончаров~А.\,А., Зацман~И.\,М., Кружков~М.\,Г.} Эволюция классификаций в надкорпусных\linebreak
\\[-12pt]
\hspace*{23pt}базах данных&4&108--116\\
\Avtors{Гончаров~А.\,В., Стрижов~В.\,В.} Выравнивание декартовых произведений упорядоченных\linebreak
\\[-12pt]
\hspace*{23pt}множеств&1&31--39\\
\end{tabular}
}

\pagebreak

\def\leftkol{АВТОРСКИЙ УКАЗАТЕЛЬ ЗА 2020 г.} % ENGLISH ABSTRACTS}

\def\rightkol{АВТОРСКИЙ УКАЗАТЕЛЬ ЗА 2020 г.} %ENGLISH ABSTRACTS}

%\thispagestyle{myheadings}
\def\leftfootline{\small{\textbf{\thepage}
\hfill ИНФОРМАТИКА И ЕЁ ПРИМЕНЕНИЯ\ \ \ том~14\ \ \ выпуск~4\ \ \ 2020}
}%
 \def\rightfootline{\small{ИНФОРМАТИКА И ЕЁ ПРИМЕНЕНИЯ\ \ \ том~14\ \ \ выпуск~4\ \ \ 2020
 \hfill \textbf{\thepage}}}


\noindent
{\tabcolsep=3pt
\begin{tabular}{p{394pt}cc}
&\textbf{Вып.} & \textbf{Стр.}\\[3pt]
\Avtors{Горшенин~А.\,К., Королев~В.\,Ю.} Аппроксимация распределений размеров частиц лунного\linebreak
\\[-12pt]
\hspace*{23pt}реголита на основе метода статистической симуляции выборок&2&50--57\\
\Avtors{Горшенин~А.\,К., Королев~В.\,Ю., Щербинина~А.\,А.} Статистическое оценивание распределений случайных коэффициентов стохастического дифференциального уравнения\linebreak
\\[-12pt]
\hspace*{23pt}Ланжевена&3&\hphantom{1}3--12\\
\Avtors{Горшенин~А.\,К., Кузьмин~В.\,Ю.} Анализ конфигураций LSTM-сетей для построения\linebreak
\\[-12pt]
\hspace*{23pt}среднесрочных векторных прогнозов&1&10--16\\
\Avtors{Грабовой~А.\,В., Бахтеев~О.\,Ю., Стрижов~В.\,В.} Введение отношения порядка на множестве\linebreak
\\[-12pt]
\hspace*{23pt}параметров аппроксимирующих моделей&2&58--65\\
\Avtors{Грушо~А.\,А., Забежайло~М.\,И., Смирнов~Д.\,В., Тимонина~Е.\,Е.} О вероятностных оценках\linebreak
\\[-12pt]
\hspace*{23pt}достоверности эмпирических выводов&4&3--8\\
\Avtors{Грушо~А.\,А., Забежайло~М.\,И., Смирнов~Д.\,В., Тимонина~Е.\,Е., Шоргин~С.\,Я.} Методы\linebreak
\\[-12pt]
\hspace*{23pt}математической статистики в задаче поиска инсайдера&3&71--75\\
\Avtors{Грушо~А.\,А., Забежайло~М.\,И., Тимонина~Е.\,Е.} О каузальной репрезентативности обуча-\linebreak
\\[-12pt]
\hspace*{23pt}ющих выборок прецедентов в задачах диагностического типа&1&80--86\\
\Avtors{Грушо~А.\,А., Тимонина~Е.\,Е., Грушо~Н.\,А., Терехина~И.\,Ю.} Выявление аномалий с по-\linebreak
\\[-12pt]
\hspace*{23pt}мощью метаданных&3&76--80\\
\Avtors{Грушо~А.\,А.} см.\ Грушо~Н.\,А.&&\\
\Avtors{Грушо~Н.\,А., Грушо~А.\,А., Забежайло~М.\,И., Тимонина~Е.\,Е.} Методы нахождения причин\linebreak
\\[-12pt]
\hspace*{23pt}сбоев в информационных технологиях  с помощью метаданных&2&33--39\\
\Avtors{Грушо~Н.\,А.} см.\ Грушо~А.\,А.&&\\
\Avtors{Данилишин~А.\,Р., Голембиовский~Д.\,Ю.} Оценка стоимости опционов на основе моделей\linebreak
\\[-12pt]
\hspace*{23pt}ARIMA--GARCH с ошибками, распределенными по закону $S_u$ Джонсона&4&83--90\\
\Avtors{Данилишин~А.\,Р., Голембиовский~Д.\,Ю.} Риск-нейтральная динамика для модели ARIMA-\linebreak
\\[-12pt]
\hspace*{23pt}GARCH с ошибками, распределенными по закону $S_U$ Джонсона&1&48--55\\
\Avtors{Диментов~А.\,В.} см.\ Краснов~Ф.\,В.&&\\
\Avtors{Донской~В.\,И.} Извлечение оптимизационных моделей из данных&3&109--118\\
\Avtors{Дубнов~Ю.\,А.} см.\ Попков~Ю.\,С.&&\\
\Avtors{Дулин~С.\,К., Дулина~Н.\,Г., Ермаков~П.\,В.} Информационный синтез документов&1&128--135\\
\Avtors{Дулина~Н.\,Г.} см.\ Дулин~С.\,К.&&\\
\Avtors{Дьяченко~Ю.\,Г.} см.\ Соколов~И.\,А.&&\\
\Avtors{Ермаков~П.\,В.} см.\ Дулин~С.\,К.&&\\
\Avtors{Ефросинин~Д.\,В.} см.\ Харин~П.\,А.&&\\
\Avtors{Жолобов~В.\,А.} см.\ Потанин~М.\,С.&&\\
\Avtors{Забежайло~М.\,И.} см.\ Грушо~А.\,А.&&\\
\Avtors{Забежайло~М.\,И.} см.\ Грушо~А.\,А.&&\\
\Avtors{Забежайло~М.\,И.} см.\ Грушо~А.\,А.&&\\
\Avtors{Забежайло~М.\,И.} см.\ Грушо~Н.\,А.&&\\
\Avtors{Захаров В. Н.} см.\ Френкель С. Л.&&\\
\Avtors{Зацман~И.\,М.} Проблемно-ориентированная верификация полноты темпоральных\linebreak
\\[-12pt]
\hspace*{23pt}онтологий и заполнение понятийных лакун&3&119--128\\
\Avtors{Зацман~И.\,М.} см.\ Гончаров~А.\,А.&&\\
\Avtors{Зацман~И.\,М.} см.\ Нуриев~В.\,А.&&\\
\Avtors{Зейфман~А.\,И.} см.\ Сатин~Я.\,А.&&\\
\Avtors{Кириков~И.\,А.} см.\ Румовская~С.\,Б.&&\\
\Avtors{Кирилюк~И.\,Л., Сенько~О.\,В.} Выбор моделей оптимальной сложности методами Монте-Карло (на примере моделей производственных функций регионов Российской\linebreak
\\[-12pt]
\hspace*{23pt}Федерации)&2&111--118\\
\Avtors{Ковалёв~Д.\,Ю.} см.\ Брюхов~Д.\,О.&&\\
\Avtors{Козеренко~Е.\,Б., Михеев~М.\,Ю., Сомин~Н.\,В., Эрлих~Л.\,И., Кузнецов~К.\,И.} Аналити\-че\-ская
текс\-тология в системах интеллектуальной обработки неструктурированных\linebreak
\\[-12pt]
\hspace*{23pt}данных&1&113--120\\
\Avtors{Колбин~И.\,С.} см.\ Абгарян~К.\,К.&&\\
\end{tabular}
}

\pagebreak

\def\leftkol{АВТОРСКИЙ УКАЗАТЕЛЬ ЗА 2020 г.} % ENGLISH ABSTRACTS}

\def\rightkol{АВТОРСКИЙ УКАЗАТЕЛЬ ЗА 2020 г.} %ENGLISH ABSTRACTS}

%\thispagestyle{myheadings}
\def\leftfootline{\small{\textbf{\thepage}
\hfill ИНФОРМАТИКА И ЕЁ ПРИМЕНЕНИЯ\ \ \ том~14\ \ \ выпуск~4\ \ \ 2020}
}%
 \def\rightfootline{\small{ИНФОРМАТИКА И ЕЁ ПРИМЕНЕНИЯ\ \ \ том~14\ \ \ выпуск~4\ \ \ 2020
 \hfill \textbf{\thepage}}}


\noindent
{\tabcolsep=3pt
\begin{tabular}{p{394pt}cc}
&\textbf{Вып.} & \textbf{Стр.}\\[3pt]
\Avtors{Королев~В.\,Ю.} О распределении отношения суммы элементов выборки, превосходящих\linebreak
\\[-12pt]
\hspace*{23pt}некоторый порог, к сумме всех элементов выборки.~I&3&35--43\\
\Avtors{Королев~В.\,Ю.} О распределении отношения суммы элементов выборки, превосходящих\linebreak
\\[-12pt]
\hspace*{23pt}некоторый порог, к сумме всех элементов выборки.~II&4&33--36\\
\Avtors{Королев~В.\,Ю.} см.\ Горшенин~А.\,К&&\\
\Avtors{Королев~В.\,Ю.} см.\ Горшенин~А.\,К.&&\\
\Avtors{Королёв~В.\,И.} см.\ Будзко~В.\,И.&&\\
\Avtors{Костина~А.\,А., Мирин~А.\,Ю., Молдовян~Д.\,Н., Фахрутдинов~Р.\,Ш.} Метод задания конечных некоммутативных ассоциативных алгебр произвольной четной размерности\linebreak
\\[-12pt]
\hspace*{23pt}для построения постквантовых криптосхем&1&\hphantom{1}94--100\\
\Avtors{Кочеткова~И.\,А.} см.\ Харин~П.\,А.&&\\
\Avtors{Краснов~Ф.\,В., Диментов~А.\,В., Шварцман~М.\,Е.} Использование тематических моделей\linebreak
\\[-12pt]
\hspace*{23pt}для парного сравнения  коллекций научных статей&3&129--135\\
\Avtors{Кривенко~М.\,П.} Последовательный анализ серий данных на основе многомерных ре-\linebreak
\\[-12pt]
\hspace*{23pt}фе\-рен\-с\-ных регионов&2&86--91\\
\Avtors{Кружков~М.\,Г.} см.\ Гончаров~А.\,А.&&\\
\Avtors{Кудрявцев~А.\,А., Шестаков~О.\,В.} Метод логарифмических моментов для оценивания\linebreak
\\[-12pt]
\hspace*{23pt}параметров гамма-экспоненциального распределения&3&49--54\\
\Avtors{Кузнецов~К.\,И.} см.\ Козеренко~Е.\,Б.&&\\
\Avtors{Кузьмин~В.\,Ю.} см.\ Горшенин~А.\,К.&&\\
\Avtors{Кушниренко~А.\,Г.} см.\ Бетелин~В.\,Б.&&\\
\Avtors{Кушниренко~А.\,Г.} см.\ Бетелин~В.\,Б.&&\\
\Avtors{Леонов~А.\,Г.} см.\ Бетелин~В.\,Б.&&\\
\Avtors{Макеева~Е.\,Д.} см.\ Харин~П.\,А.&&\\
\Avtors{Малашенко~Ю.\,Е., Назарова~И.\,А.} Аппроксимация множества достижимых потоков\linebreak
\\[-12pt]
\hspace*{23pt}многопользовательской сети&3&81--85\\
\Avtors{Мартюшова~Я.\,Г.} см.\ Босов~А.\,В.&&\\
\Avtors{Матюшенко~С.\,И., Разумчик~Р.\,В.} Стационарные характеристики системы Geo$/G/1/\infty $\linebreak
\\[-12pt]
\hspace*{23pt}с неординарным входящим потоком, управляющим размером очереди&4&25--32\\
\Avtors{Мейханаджян~Л.\,А., Разумчик~Р.\,В.} Стационарные характеристики системы $M/G/2/\infty$ с одним частным случаем дисциплины инверсионного порядка обслуживания\linebreak
\\[-12pt]
\hspace*{23pt}с обобщенным  вероятностным приоритетом&2&66--71\\
\Avtors{Мельников~А.\,В.} см.\ Вохминцев~А.\,В.&&\\
\Avtors{Мельников~С.\,Ю., Самуйлов~К.\,Е.} Статистические свойства двоичных неавтономных\linebreak
\\[-12pt]
\hspace*{23pt}регистров сдвига  с внутренним суммированием&2&80--85\\
\Avtors{Милованова~Т.\,А., Разумчик~Р.\,В.} Однолинейная система массового обслуживания с инверсионным порядком обслуживания с вероятностным приоритетом, групповым\linebreak
\\[-12pt]
\hspace*{23pt}пуассоновским потоком и фоновыми заявками&3&26--34\\
\Avtors{Мирин~А.\,Ю.} см.\ Костина~А.\,А.&&\\
\Avtors{Михеев~М.\,Ю.} см.\ Козеренко~Е.\,Б.&&\\
\Avtors{Молдовян~Д.\,Н.} см.\ Костина~А.\,А.&&\\
\Avtors{Москалева~Ф.\,А., Гайдамака~Ю.\,В., Шоргин~В.\,С.} Влияние параметров изоляции на\linebreak
\\[-12pt]
\hspace*{23pt}разделение ресурсов при нарезке сети&4&\hphantom{1}9--16\\
\Avtors{Назарова~И.\,А.} см.\ Малашенко~Ю.\,Е.&&\\
\Avtors{Наумов~А.\,В.} см.\ Босов~А.\,В.&&\\
\Avtors{Наумов~В.\,А., Самуйлов~К.\,Е.} О марковских и рациональных потоках случайных со-\linebreak
\\[-12pt]
\hspace*{23pt}бытий.~I&3&13--19\\
\Avtors{Наумов~В.\,А., Самуйлов~К.\,Е.} О марковских и рациональных потоках случайных со-\linebreak
\\[-12pt]
\hspace*{23pt}бытий.~II&4&37--46\\
\Avtors{Новиков~Д.\,А.} см.\ Шнурков~П.\,В.&&\\
\Avtors{Нуриев~В.\,А., Зацман~И.\,М.} Редуцирование спектра моделей перевода в надкорпусных\linebreak
\\[-12pt]
\hspace*{23pt}базах данных&2&119--126\\
\Avtors{Пачганов~C.\,А.} см.\ Вохминцев~А.\,В.&&\\
\end{tabular}
}

\pagebreak

\def\leftkol{АВТОРСКИЙ УКАЗАТЕЛЬ ЗА 2020 г.} % ENGLISH ABSTRACTS}

\def\rightkol{АВТОРСКИЙ УКАЗАТЕЛЬ ЗА 2020 г.} %ENGLISH ABSTRACTS}

%\thispagestyle{myheadings}
\def\leftfootline{\small{\textbf{\thepage}
\hfill ИНФОРМАТИКА И ЕЁ ПРИМЕНЕНИЯ\ \ \ том~14\ \ \ выпуск~4\ \ \ 2020}
}%
 \def\rightfootline{\small{ИНФОРМАТИКА И ЕЁ ПРИМЕНЕНИЯ\ \ \ том~14\ \ \ выпуск~4\ \ \ 2020
 \hfill \textbf{\thepage}}}


\noindent
{\tabcolsep=3pt
\begin{tabular}{p{394pt}cc}
&\textbf{Вып.} & \textbf{Стр.}\\[3pt]
\Avtors{Попков~А.\,Ю.} см.\ Попков~Ю.\,С.&&\\
\Avtors{Попков~Ю.\,С., Попков~А.\,Ю., Дубнов~Ю.\,А.} Методы детерминированных и рандомизи-\linebreak
\\[-12pt]
\hspace*{23pt}рованных энтропийных проекций для редукции размерности матрицы данных&4&47--54\\
\Avtors{Попов~Г.\,А., Симаворян~С.\,Ж., Симонян~А.\,Р., Улитина~Е.\,И.} Моделирование процесса мониторинга систем информационной безопасности на основе систем массового\linebreak
\\[-12pt]
\hspace*{23pt}обслуживания&1&71--79\\
\Avtors{Попов~М.\,В., Посыпкин~М.\,А.} Аппроксимация множества решений систем нелинейных\linebreak
\\[-12pt]
\hspace*{23pt}неравенств с использованием графических ускорителей&3&20--25\\
\Avtors{Посыпкин~М.\,А.} см.\ Попов~М.\,В.&&\\
\Avtors{Потанин~М.\,С., Вайсер~К.\,О., Жолобов~В.\,А., Стрижов~В.\,В.} Оптимизация структуры\linebreak
\\[-12pt]
\hspace*{23pt}сетей глубокого обучения&4&55--62\\
\Avtors{Разумчик~Р.\,В.} см.\ Матюшенко~С.\,И.&&\\
\Avtors{Разумчик~Р.\,В.} см.\ Мейханаджян~Л.\,А.&&\\
\Avtors{Разумчик~Р.\,В.} см.\ Милованова~Т.\,А.&&\\
\Avtors{Рождественский~Ю.\,В.} см.\ Соколов~И.\,А.&&\\
\Avtors{Румовская~С.\,Б., Кириков~И.\,А.} Метод визуального представления конфликтов в гибрид-\linebreak
\\[-12pt]
\hspace*{23pt}ных интеллектуальных многоагентных системах&4&77--82\\
\Avtors{Самуйлов~К.\,Е.} см.\ Агеев~К.\,А.&&\\
\Avtors{Самуйлов~К.\,Е.} см.\ Мельников~С.\,Ю.&&\\
\Avtors{Самуйлов~К.\,Е.} см.\ Наумов~В.\,А.&&\\
\Avtors{Самуйлов~К.\,Е.} см.\ Наумов~В.\,А.&&\\
\Avtors{Сапунова~А.\,П.} см.\ Босов~А.\,В.&&\\
\Avtors{Сатин~Я.\,А., Зейфман~А.\,И., Шилова~Г.\,Н.} О подходах к построению предельных режимов\linebreak
\\[-12pt]
\hspace*{23pt}для некоторых моделей массового обслуживания&2&3--9\\
\Avtors{Севастьянов~Л.\,А., Щетинин~Е.\,Ю.} О методах повышения точности многоклассовой\linebreak
\\[-12pt]
\hspace*{23pt}классификации на несбалансированных данных&1&63--70\\
\Avtors{Семенов~А.\,Л.} см.\ Бетелин~В.\,Б.&&\\
\Avtors{Сенько~О.\,В.} см.\ Кирилюк~И.\,Л.&&\\
\Avtors{Серебрянский~С.\,М., Тырсин~А.\,Н.} Повышение точности решения обратных задач за\linebreak
\\[-12pt]
\hspace*{23pt}счет уточнения граничных условий&1&56--62\\
\Avtors{Симаворян~С.\,Ж.} см.\ Попов~Г.\,А.&&\\
\Avtors{Симонян~А.\,Р.} см.\ Попов~Г.\,А.&&\\
\Avtors{Смирнов~Д.\,В.} см.\ Грушо~А.\,А.&&\\
\Avtors{Смирнов~Д.\,В.} см.\ Грушо~А.\,А.&&\\
\Avtors{Соколов~И.\,А., Степченков~Ю.\,А., Дьяченко~Ю.\,Г., Рождественский~Ю.\,В.} Повышение\linebreak
\\[-12pt]
\hspace*{23pt}сбоеустойчивости самосинхронных схем&4&63--68\\
\Avtors{Сомин~Н.\,В.} см.\ Козеренко~Е.\,Б.&&\\
\Avtors{Сопин~Э.\,С.} см.\ Агеев~К.\,А.&&\\
\Avtors{Сопрунов~С.\,Ф.} см.\ Бетелин~В.\,Б.&&\\
\Avtors{Соченков~И.\,В.} см.\ Будзко~В.\,И.&&\\
\Avtors{Степченков~Ю.\,А.} см.\ Соколов~И.\,А.&&\\
\Avtors{Стефанович~А.\,И.} см.\ Босов~А.\,В.&&\\
\Avtors{Стрижов~В.\,В.} см.\ Гончаров~А.\,В.&&\\
\Avtors{Стрижов~В.\,В.} см.\ Грабовой~А.\,В.&&\\
\Avtors{Стрижов~В.\,В.} см.\ Потанин~М.\,С.&&\\
\Avtors{Ступников~С.\,А.} см.\ Брюхов~Д.\,О.&&\\
\Avtors{Терехина~И.\,Ю.} см.\ Грушо~А.\,А.&&\\
\Avtors{Тимонина~Е.\,Е.} см.\  Грушо~А.\,А.&&\\
\Avtors{Тимонина~Е.\,Е.} см.\ Грушо~А.\,А.&&\\
\Avtors{Тимонина~Е.\,Е.} см.\ Грушо~А.\,А.&&\\
\Avtors{Тимонина~Е.\,Е.} см.\ Грушо~А.\,А.&&\\
\Avtors{Тимонина~Е.\,Е.} см.\ Грушо~Н.\,А.&&\\
\Avtors{Тырсин~А.\,Н.} см.\ Серебрянский~С.\,М.&&\\
\Avtors{Улитина~Е.\,И.} см.\ Попов~Г.\,А.&&\\
\end{tabular}
}

\pagebreak

\def\leftkol{АВТОРСКИЙ УКАЗАТЕЛЬ ЗА 2020 г.} % ENGLISH ABSTRACTS}

\def\rightkol{АВТОРСКИЙ УКАЗАТЕЛЬ ЗА 2020 г.} %ENGLISH ABSTRACTS}

%\thispagestyle{myheadings}
\def\leftfootline{\small{\textbf{\thepage}
\hfill ИНФОРМАТИКА И ЕЁ ПРИМЕНЕНИЯ\ \ \ том~14\ \ \ выпуск~4\ \ \ 2020}
}%
 \def\rightfootline{\small{ИНФОРМАТИКА И ЕЁ ПРИМЕНЕНИЯ\ \ \ том~14\ \ \ выпуск~4\ \ \ 2020
 \hfill \textbf{\thepage}}}


\noindent
{\tabcolsep=3pt
\begin{tabular}{p{394pt}cc}
&\textbf{Вып.} & \textbf{Стр.}\\[3pt]
\Avtors{Фахрутдинов~Р.\,Ш.} см.\ Костина~А.\,А.&&\\
\Avtors{Френкель С. Л., Захаров В. Н.} Совместная оценка предсказуемости данных и качества\linebreak
\\[-12pt]
\hspace*{23pt}предикторов&2&40--49\\
\Avtors{Харин~П.\,А., Макеева~Е.\,Д., Кочеткова~И.\,А., Ефросинин~Д.\,В., Шоргин~С.\,Я.} 
Система массового обслуживания с орбитами для анализа совместного обслуживания трафика 
с малыми задержками URLLC и~широкополосного доступа eMBB в~беспроводных\linebreak
\\[-12pt]
\hspace*{23pt}сетях пятого поколения&4&17--24\\
\Avtors{Хусаинов~А.\,А.} Производительность ограниченного конвейера&1&87--93\\
\Avtors{Шанин~И.\,А.} см.\ Брюхов~Д.\,О.&&\\
\Avtors{Шварцман~М.\,Е.} см.\ Краснов~Ф.\,В.&&\\
\Avtors{Шестаков~О.\,В.} Асимптотика оценки среднеквадратичного риска в задаче обращения\linebreak
\\[-12pt]
\hspace*{23pt}преобразования Радона по проекциям, регистрируемым на случайной сетке&2&26--32\\
\Avtors{Шестаков~О.\,В.} Асимптотическая регулярность вейвлет-методов обращения линейных однородных операторов по наблюдениям, регистрируемым в случайные моменты\linebreak
\\[-12pt]
\hspace*{23pt}времени&1&3--9\\
\Avtors{Шестаков~О.\,В.} О статистических свойствах оценки риска в задаче обращения преобра-\linebreak
\\[-12pt]
\hspace*{23pt}зования Радона при случайном объеме проекционных данных&3&44--48\\
\Avtors{Шестаков~О.\,В.} см.\ Кудрявцев~А.\,А.&&\\
\Avtors{Шилова~Г.\,Н.} см.\ Сатин~Я.\,А.&&\\
\Avtors{Шихиев~Ф.\,Ш.} см.\ Шихиев~Ш.\,Б.&&\\
\Avtors{Шихиев~Ш.\,Б., Шихиев~Ф.\,Ш.} Инкапсуляция семантических представлений в элементы\linebreak
\\[-12pt]
\hspace*{23pt}грамматики&1&121--127\\
\Avtors{Шнурков~П.\,В., Адамова~К.\,А.} Решение задачи безусловного экстремума для дробно-\linebreak
\\[-12pt]
\hspace*{23pt}линейного интегрального функционала, зависящего от параметра&2&\hphantom{1}98--103\\
\Avtors{Шнурков~П.\,В., Новиков~Д.\,А.} О концепции стохастической модели с управлением в~моменты выхода процесса на границу заданного подмножества множества\linebreak
\\[-12pt]
\hspace*{23pt}состояний&3&101--108\\
\Avtors{Шоргин~В.\,С.} см.\ Москалева~Ф.\,А.&&\\
\Avtors{Шоргин~С.\,Я.} см.\ Агеев~К.\,А.&&\\
\Avtors{Шоргин~С.\,Я.} см.\ Грушо~А.\,А.&&\\
\Avtors{Шоргин~С.\,Я.} см.\ Харин~П.\,А.&&\\
\Avtors{Щербинина~А.\,А.} см.\ Горшенин~А.\,К.&&\\
\Avtors{Щетинин~Е.\,Ю.} см.\ Севастьянов~Л.\,А.&&\\
\Avtors{Эрлих~Л.\,И.} см.\ Козеренко~Е.\,Б.&&\\
\Avtors{Ядринцев~В.\,В.} см.\ Будзко~В.\,И.&&\\
\Avtors{Яркина~Н.\,В.} см.\ Агеев~К.\,А.&&\\
\end{tabular}
}

%\thispagestyle{myheadings}
\def\leftfootline{\small{\textbf{\thepage}
\hfill ИНФОРМАТИКА И ЕЁ ПРИМЕНЕНИЯ\ \ \ том~14\ \ \ выпуск~4\ \ \ 2020}
}%
 \def\rightfootline{\small{ИНФОРМАТИКА И ЕЁ ПРИМЕНЕНИЯ\ \ \ том~14\ \ \ выпуск~4\ \ \ 2020
 \hfill \textbf{\thepage}}}

 \label{end\stat}

\newpage

\def\stat{cont-e}
{%\hrule\par
%\vskip 7pt % 7pt
\raggedleft\Large \bf%\baselineskip=3.2ex
2\,0\,2\,0\ \ A\,U\,T\,H\,O\,R\ \ I\,N\,D\,E\,X \vskip 17pt
 \hrule
 \par
\vskip 21pt plus 6pt minus 3pt }

\label{st\stat}

\def\tit{\ }

\def\aut{\ }
\def\auf{\ }

\def\leftkol{\ } %2020 AUTHOR INDEX} % ENGLISH ABSTRACTS}

\def\rightkol{\ } %2020 AUTHOR INDEX} %ENGLISH ABSTRACTS}

\titele{\tit}{\aut}{\auf}{\leftkol}{\rightkol}
\addcontentsline{toc}{subsection}{\textrm\textbf 2020 Author Index}

\def\leftfootline{\small{\textbf{\thepage}
\hfill INFORMATIKA I EE PRIMENENIYA~--- INFORMATICS AND APPLICATIONS\ \ \ 2020\
\ \ volume~14\ \ \ issue\ 4}
}%
 \def\rightfootline{\small{INFORMATIKA I EE PRIMENENIYA~--- INFORMATICS AND APPLICATIONS\ \ \ 2020\ \ \ volume~14\ \ \ issue\ 4
\hfill \textbf{\thepage}}}

\vspace*{-24pt}

\noindent
{\tabcolsep=3pt
\begin{tabular}{p{395.89pt}cc}
&\textbf{Issue} & \textbf{Page}\\[6pt]
\Avtors{Abgaryan~K.\,K. and Gavrilov~E.\,S.} Integration platform for multiscale modeling of neuromorphic\linebreak
\\[-12pt]
\hspace*{23pt}systems&2&104--110\\
\Avtors{Abgaryan~K.\,K. and Kolbin~I.\,S.} Application of multiscale approach and data sciences for\linebreak
\\[-12pt]
\hspace*{23pt}modeling thermal conductivity in layered structures&4&91--99\\
\Avtors{Adamova~K.\,A.} see Shnurkov~~P.\,V.&&\\
\Avtors{Agalarov~Ya.\,M.} Optimization of the capacity of the main storage in $G/M/1/K$ queueing system\linebreak
\\[-12pt]
\hspace*{23pt}with an additional storage device&2&72--79\\
\Avtors{Agasandyan~G.\,A.} Computational aspects of optimization on CC-VaR in a complex of markets&3&62--70\\
\Avtors{Ageev~K.\,A., Sopin~E.\,S., Yarkina~N.\,V., Samouylov~K.\,E., and Shorgin~S.\,Ya.} Analysis of the\linebreak
\\[-12pt]
\hspace*{23pt}network slicing mechanisms with guaranteed allocated resources for various traffic types&3&\hphantom{1}94--100\\
\Avtors{Bakhteev~O.\,Yu.} see Grabovoy~A.\,V.&&\\
\Avtors{Bazilevskiy~M.\,P.} Multifactor fully connected linear regression models without constraints to the\linebreak
\\[-12pt]
\hspace*{23pt}ratios of variables errors variances&2&92--97\\
\Avtors{Belenkov~V.\,G.} see Budzko~V.\,I.&&\\
\Avtors{Betelin~V.\,B., Kushnirenko~A.\,G., and Leonov~A.\,G.} Basic concepts of programming expounded\linebreak
\\[-12pt]
\hspace*{23pt}for preschoolers&3&55--61\\
\Avtors{Betelin~V.\,B., Kushnirenko~A.\,G., Semenov~A.\,L., and Soprunov~S.\,F.} About digital literacy and\linebreak
\\[-12pt]
\hspace*{23pt}environments for its development&4&100--107\\
\Avtors{Borisov~A.\,V.} Numerical schemes of Markov jump process filtering given discretized observa-\linebreak
\\[-12pt]
\hspace*{23pt}tions~II: Additive noise case&1&17--23\\
\Avtors{Borisov~A.\,V.} Numerical schemes of Markov jump process filtering given discretized observa-\linebreak
\\[-12pt]
\hspace*{23pt}tions III: Multiplicative noises case&2&10--18\\
\Avtors{Bosov~A.\,V.} Stochastic differential system output control by the quadratic criterion. V. Case of\linebreak
\\[-12pt]
\hspace*{23pt}incomplete state information&2&19--28\\
\Avtors{Bosov~A.\,V., Martyushova~Ya.\,G., Naumov~A.\,V., and Sapunova~A.\,P.} Bayesian approach to the\linebreak
\\[-12pt]
\hspace*{23pt}construction of an individual user trajectory in the system of distance learning&3&86--93\\
\Avtors{Bosov~A.\,V. and Stefanovich~A.\,I.} Stochastic differential system output control by the quadratic\linebreak
\\[-12pt]
\hspace*{23pt}criterion. IV. Alternative numerical decision&1&24--30\\
\Avtors{Briukhov~D.\,O., Stupnikov~S.\,A., Kovalev~D.\,Yu., and Shanin~I.\,A.} Neurophysiology as a subject\linebreak
\\[-12pt]
\hspace*{23pt}domain for~data intensive problem solving&1&40--47\\
\Avtors{Budzko~V.\,I., Yadrintsev~V.\,V., Sochenkov~I.\,V., Korolev~V.\,I., and Belenkov~V.\,G.} Extraction of confidentiality markers from texts under conditions of high uncertainty in systems with\linebreak
\\[-12pt]
\hspace*{23pt}data intensive usage&4&69--76\\
\Avtors{Danilishin~A.\,R. and Golembiovsky~D.\,Yu.} Estimating the fair value of options based on\linebreak
\\[-12pt]
\hspace*{23pt}ARIMA--GARCH models with errors distributed according to the Johnson's $S_u$ law&4&83--90\\
\Avtors{Danilishin~A.\,R. and Golembiovsky~D.\,Yu.} Risk-neutral dynamics for the ARIMA-GARCH\linebreak
\\[-12pt]
\hspace*{23pt}random process with errors distributed according to the Johnson's $S_u$ law&1&48--55\\
\Avtors{Diachenko~Yu.\,G.} see Sokolov~I.\,A.&&\\
\Avtors{Dimentov~A.\,V.} see Krasnov~F.\,V.&&\\
\Avtors{Donskoy~V.\,I.} Optimization models extraction from data&3&109--118\\
\Avtors{Dubnov~Y.\,A.} see Popkov~Y.\,S.&&\\
\Avtors{Dulin~S.\,K., Dulina~N.\,G., and Ermakov~P.\,V.} Information fusion of documents&1&128--135\\
\Avtors{Dulina~N.\,G.} see Dulin~S.\,K.&&\\
\Avtors{Efrosinin~D.\,V.} see Kharin~P.\,A.&&\\
\Avtors{Ehrlich~L.\,I.} see Kozerenko~E.\,B.&&\\
\Avtors{Ermakov~P.\,V.} see Dulin~S.\,K.&&\\
\end{tabular}
}
\pagebreak

\def\leftfootline{\small{\textbf{\thepage}
\hfill INFORMATIKA I EE PRIMENENIYA~--- INFORMATICS AND APPLICATIONS\ \ \ 2020\
\ \ volume~14\ \ \ issue\ 4}
}%
 \def\rightfootline{\small{INFORMATIKA I EE PRIMENENIYA~---
INFORMATICS AND APPLICATIONS\ \ \ 2020\ \ \ volume~14\ \ \ issue\ 4
\hfill \textbf{\thepage}}}

\def\leftkol{2020 AUTHOR INDEX} % ENGLISH ABSTRACTS}

\def\rightkol{2020 AUTHOR INDEX} %ENGLISH ABSTRACTS}


\noindent
{\tabcolsep=3pt
\begin{tabular}{p{395.48108pt}cc}
&\textbf{Issue} & \textbf{Page}\\[6pt]
\Avtors{Fahrutdinov~R.\,Sh.} see Kostina~A.\,A.&&\\
\Avtors{Frenkel~S.\,L. and Zakharov~V.\,N.} Joint assessment of data predictability and quality pre-\linebreak
\\[-12pt]
\hspace*{23pt}dictors&2&40--49\\
\Avtors{Gaidamaka~Yu.\,V.} see Moskaleva~F.\,A.&&\\
\Avtors{Gavrilov~E.\,S.} see Abgaryan~K.\,K.&&\\
\Avtors{Golembiovsky~D.\,Yu.} see Danilishin~A.\,R.&&\\
\Avtors{Golembiovsky~D.\,Yu.} see Danilishin~A.\,R.&&\\
\Avtors{Goncharov~A.\,V. and Strijov~V.\,V.} Alignment of ordered set Cartesian product&1&31--39\\
\Avtors{Goncharov~A.\,A., Zatsman~I.\,M., and Kruzhkov~M.\,G.} Evolution of classifications in supracorpora\linebreak
\\[-12pt]
\hspace*{23pt}databases&4&108--116\\
\Avtors{Gorshenin~A.\,K. and Korolev~V.\,Yu.} Approximation of particle size distributions of lunar regolith\linebreak
\\[-12pt]
\hspace*{23pt}based on the resampling&2&50--57\\
\Avtors{Gorshenin~A.\,K., Korolev~V.\,Yu., and Shcherbinina~A.\,A.} Statistical estimation of distributions\linebreak
\\[-12pt]
\hspace*{23pt}of random coefficients in the Langevin stochastic differential equation&3&\hphantom{1}3--12\\
\Avtors{Gorshenin~A.\,K. and Kuzmin~V.\,Yu.} Analysis of configurations of LSTM networks for medium-\linebreak
\\[-12pt]
\hspace*{23pt}term vector forecasting&1&10--16\\
\Avtors{Grabovoy~A.\,V., Bakhteev~O.\,Yu., and Strijov~V.\,V.} Ordering the set of neural network parameters&2&58--65\\
\Avtors{Grusho~A.\,A., Timonina~E.\,E., Grusho~N.\,A., and Teryokhina~I.\,Yu.} Identifying anomalies using\linebreak
\\[-12pt]
\hspace*{23pt}metadata&3&76--80\\
\Avtors{Grusho~A.\,A., Zabezhailo~M.\,I., Smirnov~D.\,V., and Timonina~E.\,E.} On probabilistic estimates of\linebreak
\\[-12pt]
\hspace*{23pt}the validity of empirical conclusions&4&3--8\\
\Avtors{Grusho~A.\,A., Zabezhailo~M.\,I., and Timonina~E.\,E.} On causal representativeness of training\linebreak
\\[-12pt]
\hspace*{23pt}samples of precedents in diagnostic type tasks&1&80--86\\
\Avtors{Grusho~A.\,A.} see Grusho~N.\,A.&&\\
\Avtors{Grusho~N.\,A., Grusho~A.\,A., Zabezhailo~M.\,I., and Timonina~E.\,E.} Methods of finding the causes\linebreak
\\[-12pt]
\hspace*{23pt}of information technology failures by means of metadata&2&33--39\\
\Avtors{Grusho~N.\,A., Zabezhailo~M.\,I., Smirnov~D.\,V., Timonina~E.\,E., and Shorgin~S.\,Ya.} Mathematical\linebreak
\\[-12pt]
\hspace*{23pt}statistics in the task of identifying hostile insiders&3&71--75\\
\Avtors{Grusho~N.\,A.} see Grusho~A.\,A.&&\\
\Avtors{Kharin~P.\,A., Makeeva~E.\,D., Kochetkova~I.\,A., Efrosinin~D.\,V., and Shorgin~S.\,Ya.} Retrial\linebreak
\\[-12pt]
\hspace*{23pt}queuing model for analyzing joint URLLC and eMBB transmission in 5G networks&4&17--24\\
\Avtors{Khusainov~A.\,A.} Performance of the bounded pipeline&1&87--93\\
\Avtors{Kirikov~I.\,A.} see Rumovskaya~S.\,B.&&\\
\Avtors{Kirilyuk~I.\,L. and Sen'ko~O.\,V.} Selection of optimal complexity models by methods of nonparametric statistics (on the example of production function model of regions of the Russian\linebreak
\\[-12pt]
\hspace*{23pt}Federation)&2&111--118\\
\Avtors{Kochetkova~I.\,A.} see Kharin~P.\,A.&&\\
\Avtors{Kolbin~I.\,S.} see Abgaryan~K.\,K.&&\\
\Avtors{Korolev~V.\,I.} see Budzko~V.\,I.&&\\
\Avtors{Korolev~V.\,Yu.} On the distribution of the ratio of the sum of sample elements exceeding\linebreak
\\[-12pt]
\hspace*{23pt}a threshold to the total sum of sample elements.~I&3&35--43\\
\Avtors{Korolev~V.\,Yu.} On the distribution of the ratio of the sum of sample elements exceeding\linebreak
\\[-12pt]
\hspace*{23pt}a threshold to the total sum of sample elements.~II&4&33--36\\
\Avtors{Korolev~V.\,Yu.} see Gorshenin~A.\,K.&&\\
\Avtors{Korolev~V.\,Yu.} see Gorshenin~A.\,K.&&\\
\Avtors{Kostina~A.\,A., Mirin~A.\,Yu., Moldovyan~D.\,N., and Fahrutdinov~R.\,Sh.} Method for defining finite noncommutative associative algebras of arbitrary even dimension for development of the\linebreak
\\[-12pt]
\hspace*{23pt}postquantum cryptoschemes&1&\hphantom{1}94--100\\
\Avtors{Kovalev~D.\,Yu.} see Briukhov~D.\,O.&&\\
\Avtors{Kozerenko~E.\,B., Mikheev~M.\,Y., Somin~N.\,V., Ehrlich~L.\,I., and Kuznetsov~K.\,I.} Analytical\linebreak
\\[-12pt]
\hspace*{23pt}textology in intelligent processing systems for unstructured data&1&113--120\\
\Avtors{Krasnov~F.\,V., Dimentov~A.\,V., and Shvartsman~M.\,E.} Using topic models for pairwise comparison\linebreak
\\[-12pt]
\hspace*{23pt}of collections of scientific papers&3&129--135\\
\end{tabular}
}
\pagebreak

\def\leftfootline{\small{\textbf{\thepage}
\hfill INFORMATIKA I EE PRIMENENIYA~--- INFORMATICS AND APPLICATIONS\ \ \ 2020\
\ \ volume~14\ \ \ issue\ 4}
}%
 \def\rightfootline{\small{INFORMATIKA I EE PRIMENENIYA~---
INFORMATICS AND APPLICATIONS\ \ \ 2020\ \ \ volume~14\ \ \ issue\ 4
\hfill \textbf{\thepage}}}

\def\leftkol{2020 AUTHOR INDEX} % ENGLISH ABSTRACTS}

\def\rightkol{2020 AUTHOR INDEX} %ENGLISH ABSTRACTS}


\noindent
{\tabcolsep=3pt
\begin{tabular}{p{395.48108pt}cc}
&\textbf{Issue} & \textbf{Page}\\[6pt]
\Avtors{Krivenko~M.\,P.} Sequential analysis of serial measurements based on multivariate reference\linebreak
\\[-12pt]
\hspace*{23pt}regions&2&86--91\\
\Avtors{Kruzhkov~M.\,G.} see Goncharov~A.\,A.&&\\
\Avtors{Kudryavtsev~A.\,A. and Shestakov~O.\,V.} Method of logarithmic moments for estimating the\linebreak
\\[-12pt]
\hspace*{23pt}gamma-exponential distribution parameters&3&49--54\\
\Avtors{Kushnirenko~A.\,G.} see Betelin~V.\,B.&&\\
\Avtors{Kushnirenko~A.\,G.} see Betelin~V.\,B.&&\\
\Avtors{Kuzmin~V.\,Yu.} see Gorshenin~A.\,K.&&\\
\Avtors{Kuznetsov~K.\,I.} see Kozerenko~E.\,B.&&\\
\Avtors{Leonov~A.\,G.} see Betelin~V.\,B.&&\\
\Avtors{Makeeva~E.\,D.} see Kharin~P.\,A.&&\\
\Avtors{Malashenko~Yu.\,E. and Nazarova~I.\,A.} Approximation of the multiuser network feasible\linebreak
\\[-12pt]
\hspace*{23pt}flows set&3&81--85\\
\Avtors{Martyushova~Ya.\,G.} see Bosov~A.\,V.&&\\
\Avtors{Matyushenko~S.\,I. and Razumchik~R.\,V.} Stationary characteristics of discrete-time Geo$/G/1/\infty$\linebreak
\\[-12pt]
\hspace*{23pt}queue with batch arrivals and one queue skipping policy&4&25--32\\
\Avtors{Melnikov~A.\,V.} see Vokhmintcev~A.\,V.&&\\
\Avtors{Melnikov~S.\,Yu. and Samouylov~K.\,E.} Statistical properties of binary nonautonomous shift\linebreak
\\[-12pt]
\hspace*{23pt}registers with internal xor&2&80--85\\
\Avtors{Meykhanadzhyan~L.\,A. and Razumchik~R.\,V.} Stationary characteristics of $M/G/2/\infty$ queue\linebreak
\\[-12pt]
\hspace*{23pt}with identical servers, LIFO service, and resampling policy&2&66--71\\
\Avtors{Mikheev~M.\,Y.} see Kozerenko~E.\,B.&&\\
\Avtors{Milovanova~T.\,A. and Razumchik~R.\,V.} A single-server queueing system with LIFO service,\linebreak
\\[-12pt]
\hspace*{23pt}probabilistic priority, batch Poisson arrivals, and background customers&3&26--34\\
\Avtors{Mirin~A.\,Yu.} see Kostina~A.\,A.&&\\
\Avtors{Moldovyan~D.\,N.} see Kostina~A.\,A.&&\\
\Avtors{Moskaleva~F.\,A., Gaidamaka~Yu.\,V., and Shorgin~V.\,S.} Impact of the isolation parameters on\linebreak
\\[-12pt]
\hspace*{23pt}resource allocation in the network slicing model&4&\hphantom{1}9--16\\
\Avtors{Naumov~A.\,V.} see Bosov~A.\,V.&&\\
\Avtors{Naumov~V.\,A. and Samouylov~К.\,Е.} On Markovian and rational arrival processes.~I&3&13--19\\
\Avtors{Naumov~V.\,A. and Samouylov~K.\,E.} On Markovian and rational arrival processes.~II&4&37--46\\
\Avtors{Nazarova~I.\,A.} see Malashenko~Yu.\,E.&&\\
\Avtors{Novikov~D.\,A.} see Shnurkov~P.\,V.&&\\
\Avtors{Nuriev~V.\,A. and Zatsman~I.\,M.} Reducing the spectrum of translation models in supracorpora\linebreak
\\[-12pt]
\hspace*{23pt}databases&2&119--126\\
\Avtors{Pachganov~S.\,A.} see Vokhmintcev~A.\,V.&&\\
\Avtors{Popkov~A.\,Y.} see Popkov~Y.\,S.&&\\
\Avtors{Popkov~Y.\,S., Popkov~A.\,Y., and Dubnov~Y.\,A.} Deterministic and randomized methods of entropy\linebreak
\\[-12pt]
\hspace*{23pt}projection for dimensionality reduction problems&4&47--54\\
\Avtors{Popov~G.\,A., Simavoryan~S.\,Zh., Simonyan~A.\,R., and Ulitina~E.\,I.} Modeling of monitoring of\linebreak
\\[-12pt]
\hspace*{23pt}information security process on the basis of queuing systems&1&71--79\\
\Avtors{Popov~M.\,V. and Posypkin~M.\,A.} Approximation of the set of solutions of systems of nonlinear\linebreak
\\[-12pt]
\hspace*{23pt}inequalities using graphic accelerators&3&20--25\\
\Avtors{Posypkin~M.\,A.} see Popov~M.\,V.&&\\
\Avtors{Potanin~M.\,S., Vayser~K.\,O., Zholobov~V.\,A., and Strijov~V.\,V.} Deep learning neural network\linebreak
\\[-12pt]
\hspace*{23pt}structure optimization&4&55--62\\
\Avtors{Razumchik~R.\,V.} see Matyushenko~S.\,I.&&\\
\Avtors{Razumchik~R.\,V.} see Meykhanadzhyan~L.\,A.&&\\
\Avtors{Razumchik~R.\,V.} see Milovanova~T.\,A.&&\\
\Avtors{Rogdestvenski~Yu.\,V.} see Sokolov~I.\,A.&&\\
\Avtors{Rumovskaya~S.\,B. and Kirikov~I.\,A.} Conflict visual representation method in hybrid intelligent\linebreak
\\[-12pt]
\hspace*{23pt}multiagent systems&4&77--82\\
\Avtors{Samouylov~K.\,E.} see Ageev~K.\,A.&&\\
\end{tabular}
}
\pagebreak

\def\leftfootline{\small{\textbf{\thepage}
\hfill INFORMATIKA I EE PRIMENENIYA~--- INFORMATICS AND APPLICATIONS\ \ \ 2020\
\ \ volume~14\ \ \ issue\ 4}
}%
 \def\rightfootline{\small{INFORMATIKA I EE PRIMENENIYA~---
INFORMATICS AND APPLICATIONS\ \ \ 2020\ \ \ volume~14\ \ \ issue\ 4
\hfill \textbf{\thepage}}}

\def\leftkol{2020 AUTHOR INDEX} % ENGLISH ABSTRACTS}

\def\rightkol{2020 AUTHOR INDEX} %ENGLISH ABSTRACTS}


\noindent
{\tabcolsep=3pt
\begin{tabular}{p{395.48108pt}cc}
&\textbf{Issue} & \textbf{Page}\\[6pt]
\Avtors{Samouylov~K.\,E.} see Melnikov~S.\,Yu.&&\\
\Avtors{Samouylov~K.\,E.} see Naumov~V.\,A.&&\\
\Avtors{Samouylov~K.\,Е.} see Naumov~V.\,A.&&\\
\Avtors{Sapunova~A.\,P.} see Bosov~A.\,V.&&\\
\Avtors{Satin~Ya.\,A., Zeifman~A.\,I., and Shilova~G.\,N.} On approaches to constructing limiting regimes\linebreak
\\[-12pt]
\hspace*{23pt}for some queuing models&2&3--9\\
\Avtors{Semenov~A.\,L.} see Betelin~V.\,B.&&\\
\Avtors{Sen'ko~O.\,V.} see Kirilyuk~I.\,L.&&\\
\Avtors{Serebryanskii~S.\,M. and Tyrsin~A.\,N.} Improvement of the accuracy of solution of tasks for the\linebreak
\\[-12pt]
\hspace*{23pt}account of the construction of boundary conditions&1&56--62\\
\Avtors{Sevastianov~L.\,A. and Shchetinin~E.\,Yu.} On methods for improving the accuracy of multiclass\linebreak
\\[-12pt]
\hspace*{23pt}classification on imbalanced data&1&63--70\\
\Avtors{Shanin~I.\,A.} see Briukhov~D.\,O.&&\\
\Avtors{Shcherbinina~A.\,A.} see Gorshenin~A.\,K.&&\\
\Avtors{Shchetinin~E.\,Yu.} see Sevastianov~L.\,A.&&\\
\Avtors{Shestakov~O.\,V.} Asymptotic regularity of the wavelet methods of inverting linear homogeneous\linebreak
\\[-12pt]
\hspace*{23pt}operators from observations recorded at random times&1&3--9\\
\Avtors{Shestakov~O.\,V.} Asymptotics of the mean-square risk estimate in the problem of inverting the\linebreak
\\[-12pt]
\hspace*{23pt}Radon transform from projections registered on a random grid&2&29--32\\
\Avtors{Shestakov~O.\,V.} On the statistical properties of risk estimate in the problem of inverting the\linebreak
\\[-12pt]
\hspace*{23pt}Radon transform with a random volume of projection data&3&44--48\\
\Avtors{Shestakov~O.\,V.} see Kudryavtsev~A.\,A.&&\\
\Avtors{Shihiev~F.\,Sh.} see Shihiev~Sh.\,B.&&\\
\Avtors{Shihiev~Sh.\,B. and Shihiev~F.\,Sh.} Incapsulation of semantic representations into elements of\linebreak
\\[-12pt]
\hspace*{23pt}a grammar&1&121--127\\
\Avtors{Shilova~G.\,N.} see Satin~Ya.\,A.&&\\
\Avtors{Shnurkov~~P.\,V. and Adamova~K.\,A.} Solution of the unconditional extremal problem for a~linear-\linebreak
\\[-12pt]
\hspace*{23pt}fractional integral functional dependent on the parameter&2&\hphantom{1}98--103\\
\Avtors{Shnurkov~P.\,V. and Novikov~D.\,A.} On the concept of a stochastic model with control at the\linebreak
\\[-12pt]
\hspace*{23pt}moments of the process at the border of a presented subset of multiple states&3&101--108\\
\Avtors{Shorgin~S.\,Ya.} see Ageev~K.\,A.&&\\
\Avtors{Shorgin~S.\,Ya.} see Grusho~N.\,A.&&\\
\Avtors{Shorgin~S.\,Ya.} see Kharin~P.\,A.&&\\
\Avtors{Shorgin~V.\,S.} see Moskaleva~F.\,A.&&\\
\Avtors{Shvartsman~M.\,E.} see Krasnov~F.\,V.&&\\
\Avtors{Simavoryan~S.\,Zh.} see Popov~G.\,A.&&\\
\Avtors{Simonyan~A.\,R.} see Popov~G.\,A.&&\\
\Avtors{Smirnov~D.\,V.} see Grusho~A.\,A.&&\\
\Avtors{Smirnov~D.\,V.} see Grusho~N.\,A.&&\\
\Avtors{Sochenkov~I.\,V.} see Budzko~V.\,I.&&\\
\Avtors{Sokolov~I.\,A., Stepchenkov~Yu.\,A., Diachenko~Yu.\,G., and Rogdestvenski~Yu.\,V.} Improvement of\linebreak
\\[-12pt]
\hspace*{23pt}self-timed circuit soft error tolerance&4&63--68\\
\Avtors{Somin~N.\,V.} see Kozerenko~E.\,B.&&\\
\Avtors{Sopin~E.\,S.} see Ageev~K.\,A.&&\\
\Avtors{Soprunov~S.\,F.} see Betelin~V.\,B.&&\\
\Avtors{Stefanovich~A.\,I.} see Bosov~A.\,V.&&\\
\Avtors{Stepchenkov~Yu.\,A.} see Sokolov~I.\,A.&&\\
\Avtors{Strijov~V.\,V.} see Goncharov~A.\,V.&&\\
\Avtors{Strijov~V.\,V.} see Grabovoy~A.\,V.&&\\
\Avtors{Strijov~V.\,V.} see Potanin~M.\,S.&&\\
\Avtors{Stupnikov~S.\,A.} see Briukhov~D.\,O.&&\\
\Avtors{Teryokhina~I.\,Yu.} see Grusho~A.\,A.&&\\
\Avtors{Timonina~E.\,E.} see Grusho~A.\,A.&&\\
\end{tabular}
}
\pagebreak

\def\leftfootline{\small{\textbf{\thepage}
\hfill INFORMATIKA I EE PRIMENENIYA~--- INFORMATICS AND APPLICATIONS\ \ \ 2020\
\ \ volume~14\ \ \ issue\ 4}
}%
 \def\rightfootline{\small{INFORMATIKA I EE PRIMENENIYA~---
INFORMATICS AND APPLICATIONS\ \ \ 2020\ \ \ volume~14\ \ \ issue\ 4
\hfill \textbf{\thepage}}}

\def\leftkol{2020 AUTHOR INDEX} % ENGLISH ABSTRACTS}

\def\rightkol{2020 AUTHOR INDEX} %ENGLISH ABSTRACTS}


\noindent
{\tabcolsep=3pt
\begin{tabular}{p{395.48108pt}cc}
&\textbf{Issue} & \textbf{Page}\\[6pt]
\Avtors{Timonina~E.\,E.} see Grusho~A.\,A.&&\\
\Avtors{Timonina~E.\,E.} see Grusho~A.\,A.&&\\
\Avtors{Timonina~E.\,E.} see Grusho~N.\,A.&&\\
\Avtors{Timonina~E.\,E.} see Grusho~N.\,A.&&\\
\Avtors{Tyrsin~A.\,N.} see Serebryanskii~S.\,M.&&\\
\Avtors{Ulitina~E.\,I.} see Popov~G.\,A.&&\\
\Avtors{Vayser~K.\,O.} see Potanin~M.\,S.&&\\
\Avtors{Vokhmintcev~A.\,V., Melnikov~A.\,V., and Pachganov~S.\,A.} Simultaneous localization and mapping method in  three-dimensional space based on the combined solution of the  point--point\linebreak
\\[-12pt]
\hspace*{23pt}variation problem ICP for an affine transformation&1&101--112\\
\Avtors{Yadrintsev~V.\,V.} see Budzko~V.\,I.&&\\
\Avtors{Yarkina~N.\,V.} see Ageev~K.\,A.&&\\
\Avtors{Zabezhailo~M.\,I.} see Grusho~A.\,A.&&\\
\Avtors{Zabezhailo~M.\,I.} see Grusho~A.\,A.&&\\
\Avtors{Zabezhailo~M.\,I.} see Grusho~N.\,A.&&\\
\Avtors{Zabezhailo~M.\,I.} see Grusho~N.\,A.&&\\
\Avtors{Zakharov~V.\,N.} see Frenkel~S.\,L.&&\\
\Avtors{Zatsman~I.\,M.} Problem-oriented verifying the completeness  of~temporal ontologies and\linebreak
\\[-12pt]
\hspace*{23pt}filling~conceptual lacunas&3&119--128\\
\Avtors{Zatsman~I.\,M.} see Goncharov~A.\,A.&&\\
\Avtors{Zatsman~I.\,M.} see Nuriev~V.\,A.&&\\
\Avtors{Zeifman~A.\,I.} see Satin~Ya.\,A.&&\\
\Avtors{Zholobov~V.\,A.} see Potanin~M.\,S.&&\\
\end{tabular}
}

%\thispagestyle{myheadings}
\def\leftfootline{\small{\textbf{\thepage}
\hfill INFORMATIKA I EE PRIMENENIYA~--- INFORMATICS AND APPLICATIONS\ \ \ 2020\
\ \ volume~14\ \ \ issue\ 4}
}%
 \def\rightfootline{\small{INFORMATIKA I EE PRIMENENIYA~---
INFORMATICS AND APPLICATIONS\ \ \ 2020\ \ \ volume~14\ \ \ issue\ 4
\hfill \textbf{\thepage}}}

 \label{end\stat}

\newpage


%\linebreak
%\\[-12pt]
%\hspace*{23pt}
%%Информатика и её применения
%Том 15 Выпуск 1-4 Год 2021

\def\stat{cont}
{%\hrule\par
%\vskip 7pt % 7pt
\raggedleft\Large \bf%\baselineskip=3.2ex
А\,В\,Т\,О\,Р\,С\,К\,И\,Й\ \ У\,К\,А\,З\,А\,Т\,Е\,Л\,Ь\ \ З\,А\ \ 2\,0\,2\,1 г. \vskip 17pt
 \hrule
 \par
\vskip 21pt plus 6pt minus 3pt }

\label{st\stat}

\def\tit{\ }

\def\aut{\ }
\def\auf{\ }

\def\leftkol{\ } % ENGLISH ABSTRACTS}

\def\rightkol{\ } %АВТОРСКИЙ УКАЗАТЕЛЬ ЗА 2021 г.} %ENGLISH ABSTRACTS}

\titele{\tit}{\aut}{\auf}{\leftkol}{\rightkol}
\addcontentsline{toc}{subsection}{\textrm\textbf Авторский указатель за 2021 г.}

\vspace*{-24pt}

\noindent
{\tabcolsep=3pt
\begin{tabular}{p{397pt}cc}
&\textbf{Вып.} & \textbf{Стр.}\\[6pt]
\Avtors{Абгарян~К.\,К., Гаврилов~Е.\,С.} Распределенная информационная система для расчета\linebreak
\\[-12pt]
\hspace*{23pt}структурных свойств композиционных материалов&4&50--58\\
\Avtors{Агаларов~Я.\,М.} Оптимальное пороговое управление доступом в системе $M/M/s$ с не-\linebreak
\\[-12pt]
\hspace*{23pt}однородными приборами и общим накопителем&1&57--64\\
\Avtors{Андрианова~Е.\,Г.} см.\ Сигов~А.\,С.&&\\
\Avtors{Арутюнов~Е.\,Н., Кудрявцев~А.\,А., Недоливко~Ю.\,Н.} Вероятностные характеристики\linebreak
\\[-12pt]
\hspace*{23pt}индекса баланса факторов, имеющих обобщенные гамма-распределения&1&65--71\\
\Avtors{Базилевский~М.\,П.} Метод выпрямления искаженных из-за мультиколлинеарности\linebreak
\\[-12pt]
\hspace*{23pt}коэффициентов в регрессионных моделях&2&60--65\\
\Avtors{Бахтеев~О.\,Ю.} см.\ Гребенькова~О.\,С.&&\\
\Avtors{Бахтеев~О.\,Ю.} см.\ Кузнецова~Р.\,В.&&\\
\Avtors{Борисов~А.\,В., Казанчян~Д.\,Х.} Фильтрация состояний марковских скачкообразных про-\linebreak
\\[-12pt]
\hspace*{23pt}цессов по комплексным наблюдениям I: точное решение задачи&2&12--19\\
\Avtors{Борисов~А.\,В., Казанчян~Д.\,Х.} Фильтрация состояний марковских скачкообразных про-\linebreak
\\[-12pt]
\hspace*{23pt}цессов по комплексным наблюдениям II: численный алгоритм&3&\hphantom{1}9--15\\
\Avtors{Босов~А.\,В.} О некоторых частных случаях в задаче управления выходом стохастической\linebreak
\\[-12pt]
\hspace*{23pt}дифференциальной системы по квадратичному критерию&1&11--17\\
\Avtors{Босов~А.\,В.} Управление линейным выходом марковской цепи по квадратичному кри-\linebreak
\\[-12pt]
\hspace*{23pt}терию&2&\hphantom{1}3--11\\
\Avtors{Босов~А.\,В., Жуков~Д.\,В.} Экспертная система для мониторинга и прогнозирования\linebreak
\\[-12pt]
\hspace*{23pt}процессов распределения ресурсов&3&29--40\\
\Avtors{Босов~А.\,В., Игнатов~А.\,Н., Наумов~А.\,В.} Алгоритмы приближенного решения задачи\linebreak
\\[-12pt]
\hspace*{23pt}назначения <<технологического окна>> на участках железнодорожной сети&4&\hphantom{1}3--11\\
\Avtors{Брюхов~Д.\,О., Ступников~С.\,А., Ковалёв~Д.\,Ю., Шанин~И.\,А.} Архитектура распределенного\linebreak
\\[-12pt]
\hspace*{23pt}решения задач анализа данных в области нейрофизиологии&1&78--85\\
\Avtors{Власкина~А.\,С.} см.\ Кочеткова~И.\,А.&&\\
\Avtors{Ву~Н.\,Н.} см.\ Кочеткова~И.\,А.&&\\
\Avtors{Вышинский~Л.\,Л., Флёров~Ю.\,А.} Информационная модель весового облика летательных\linebreak
\\[-12pt]
\hspace*{23pt}аппаратов&1&50--56\\
\Avtors{Вышинский~Л.\,Л., Флёров~Ю.\,А.} Теоретические основы формирования весового облика\linebreak
\\[-12pt]
\hspace*{23pt}самолета&4&\hphantom{1}93--102\\
\Avtors{Гаврилов~Е.\,С.} см.\ Абгарян~К.\,К.&&\\
\Avtors{Гончаренко~М.\,Б., Захарова~Т.\,В.} Некоторые свойства смесей нормальных распределений\linebreak
\\[-12pt]
\hspace*{23pt}и~их приложения к задачам магнитоэнцефалографии&2&44--51\\
\Avtors{Гончаров~А.\,А., Зацман~И.\,М.} Принципы структуризации статей в электронных словарях&2&89--95\\
\Avtors{Гончаров~А.\,А., Зацман~И.\,М., Кружков~М.\,Г.} Представление новых лексикографических\linebreak
\\[-12pt]
\hspace*{23pt}знаний в динамических классификационных системах&1&86--93\\
\Avtors{Гончаров~А.\,А., Зацман~И.\,М., Кружков~М.\,Г., Лощилова~Е.\,Ю.} Отражение эволюции\linebreak
\\[-12pt]
\hspace*{23pt}лексикографических знаний в~динамических классификационных системах&4&41--49\\
\Avtors{Гончаров~А.\,А., Инькова~О.\,Ю.} Извлечение знаний о средствах выражения логико-\linebreak
\\[-12pt]
\hspace*{23pt}се\-ман\-ти\-че\-ских отношений при помощи надкорпусной базы данных&2&\hphantom{1}96--103\\
\Avtors{Горшенин~А.\,К., Кузьмин~В.\,Ю.} Метод повышения точности нейросетевых прогнозов с использованием смешанных вероятностных моделей и его реализация в виде\linebreak
\\[-12pt]
\hspace*{23pt}цифрового сервиса&3&63--74\\
\Avtors{Гребенькова~О.\,С., Бахтеев~О.\,Ю., Стрижов~В.\,В.} Вариационная оптимизация модели\linebreak
\\[-12pt]
\hspace*{23pt}глубокого обучения с контролем сложности&1&42--49\\
\end{tabular}
}

\pagebreak

\def\leftkol{АВТОРСКИЙ УКАЗАТЕЛЬ ЗА 2021 г.} % ENGLISH ABSTRACTS}

\def\rightkol{АВТОРСКИЙ УКАЗАТЕЛЬ ЗА 2021 г.} %ENGLISH ABSTRACTS}

%\thispagestyle{myheadings}
\def\leftfootline{\small{\textbf{\thepage}
\hfill ИНФОРМАТИКА И ЕЁ ПРИМЕНЕНИЯ\ \ \ том~15\ \ \ выпуск~4\ \ \ 2021}
}%
 \def\rightfootline{\small{ИНФОРМАТИКА И ЕЁ ПРИМЕНЕНИЯ\ \ \ том~15\ \ \ выпуск~4\ \ \ 2021
 \hfill \textbf{\thepage}}}


\noindent
{\tabcolsep=3pt
\begin{tabular}{p{394pt}cc}
&\textbf{Вып.} & \textbf{Стр.}\\[3pt]
\Avtors{Гринченко~С.\,Н.} Антропогенная <<третья>> природа: относительно автономный статус\linebreak
\\[-12pt]
\hspace*{23pt}ее искусственных интеллектуальных субъектов&4&110--114\\
\Avtors{Гринченко~С.\,Н.} О системной иерархии искусственного интеллекта&1&111--115\\
\Avtors{Грушо~А.\,А., Грушо~Н.\,А., Забежайло~М.\,И., Смирнов~Д.\,В., Тимонина~Е.\,Е., Шоргин~С.\,Я.} Статистика и кластеры в~поисках аномальных вкраплений в~условиях больших\linebreak
\\[-12pt]
\hspace*{23pt}данных&4&79--86\\
\Avtors{Грушо~А.\,А., Грушо~Н.\,А., Забежайло~М.\,И., Тимонина~Е.\,Е.} Удаленный мониторинг\linebreak
\\[-12pt]
\hspace*{23pt}рабочих процессов&3&2--8\\
\Avtors{Грушо~А.\,А., Забежайло~М.\,И., Смирнов~Д.\,В., Тимонина~Е.\,Е.} Интеллектуальный анализ\linebreak
\\[-12pt]
\hspace*{23pt}пополняемых коллекций Big Data в режиме процессно-реального времени&2&36--40\\
\Avtors{Грушо~Н.\,А.} см.\ Грушо~А.\,А.&&\\
\Avtors{Грушо~Н.\,А.} см.\ Грушо~А.\,А.&&\\
\Avtors{Дараселия~А.\,В., Сопин~Э.\,С., Молчанов~Д.\,А., Самуйлов~К.\,Е.} Анализ стратегии разгрузки\linebreak
\\[-12pt]
\hspace*{23pt}базовых станций 5G NR с помощью технологии NR-U&3&98--111\\
\Avtors{Дорофеева~А.\,В.} см.\ Королев~В.\,Ю.&&\\
\Avtors{Дьяченко~Ю.\,Г.} см.\ Соколов~И.\,А.&&\\
\Avtors{Дюкова~Е.\,В., Масляков~Г.\,О.} О выборе частичных порядков на множествах значений\linebreak
\\[-12pt]
\hspace*{23pt}признаков в~задаче классификации&4&72--78\\
\Avtors{Егорова~А.\,Ю.} см.\ Нуриев~В.\,А.&&\\
\Avtors{Жуков~Д.\,В.} см.\ Босов~А.\,В.&&\\
\Avtors{Жуков~Д.\,О., Хватова~Т.\,Ю., Зальцман~А.\,Д.} Моделирование стохастической динамики изменения состояний узлов и~перколяционных переходов в~социальных сетях\linebreak
\\[-12pt]
\hspace*{23pt}с~учетом самоорганизации и наличия памяти&1&102--110\\
\Avtors{Забежайло~М.\,И.} см.\ Грушо~А.\,А.&&\\
\Avtors{Забежайло~М.\,И.} см.\ Грушо~А.\,А.&&\\
\Avtors{Забежайло~М.\,И.} см.\ Грушо~А.\,А.&&\\
\Avtors{Зальцман~А.\,Д.} см.\ Жуков~Д.\,О.&&\\
\Avtors{Захарова~Т.\,В.} см.\ Гончаренко~М.\,Б.&&\\
\Avtors{Зацман~И.\,М.} Концепция создания ВОЗ-центра компетенций по пандемиям и~эпиде-\linebreak
\\[-12pt]
\hspace*{23pt}миям: ключевые понятия и~их терминологический анализ&4&103--109\\
\Avtors{Зацман~И.\,М.} Проблемно-ориентированная актуализация словарных статей двуязыч-\linebreak
\\[-12pt]
\hspace*{23pt}ных словарей и медицинской терминологии: сопоставительный анализ&1&\hphantom{1}94--101\\
\Avtors{Зацман~И.\,М.} Формы представления нового знания, извлеченного из текстов&3&83--90\\
\Avtors{Зацман~И.\,М.} см.\ Гончаров~А.\,А.&&\\
\Avtors{Зацман~И.\,М.} см.\ Гончаров~А.\,А.&&\\
\Avtors{Зацман~И.\,М.} см.\ Гончаров~А.\,А.&&\\
\Avtors{Зейфман~А.\,И., Сатин~Я.\,А., Ковалёв~И.\,А.} Об одной нестационарной модели обслужи-\linebreak
\\[-12pt]
\hspace*{23pt}вания с катастрофами и тяжелыми хвостами&2&20--25\\
\Avtors{Игнатов~А.\,Н.} см.\ Босов~А.\,В.&&\\
\Avtors{Инькова~О.\,Ю., Кружков~М.\,Г.} Структурированные определения дискурсивных отно-\linebreak
\\[-12pt]
\hspace*{23pt}шений в~надкорпусной базе данных коннекторов&4&27--32\\
\Avtors{Инькова~О.\,Ю.} см.\ Гончаров~А.\,А.&&\\
\Avtors{Истратов~Л.\,А.} см.\ Сигов~А.\,С.&&\\
\Avtors{Казанчян~Д.\,Х.} см.\ Борисов~А.\,В.&&\\
\Avtors{Казанчян~Д.\,Х.} см.\ Борисов~А.\,В.&&\\
\Avtors{Каменских~А.\,Н.} см.\ Соколов~И.\,А.&&\\
\Avtors{Кириков~И.\,А., Листопад~С.\,В.} Согласование целей агентов сплоченных гибридных\linebreak
\\[-12pt]
\hspace*{23pt}интеллектуальных многоагентных систем&2&66--71\\
\Avtors{Кириков~И.\,А.} см.\ Румовская~С.\,Б.&&\\
\Avtors{Ковалёв~Д.\,Ю.}см.\ Брюхов~Д.\,О.&&\\
\Avtors{Ковалёв~И.\,А.} см.\ Зейфман~А.\,И.&&\\
\Avtors{Ковалёв~С.\,П.} Методы теории категорий в цифровом проектировании гетерогенных\linebreak
\\[-12pt]
\hspace*{23pt}киберфизических систем&1&23--29\\
\end{tabular}
}

\pagebreak

\def\leftkol{АВТОРСКИЙ УКАЗАТЕЛЬ ЗА 2021 г.} % ENGLISH ABSTRACTS}

\def\rightkol{АВТОРСКИЙ УКАЗАТЕЛЬ ЗА 2021 г.} %ENGLISH ABSTRACTS}

%\thispagestyle{myheadings}
\def\leftfootline{\small{\textbf{\thepage}
\hfill ИНФОРМАТИКА И ЕЁ ПРИМЕНЕНИЯ\ \ \ том~15\ \ \ выпуск~4\ \ \ 2021}
}%
 \def\rightfootline{\small{ИНФОРМАТИКА И ЕЁ ПРИМЕНЕНИЯ\ \ \ том~15\ \ \ выпуск~4\ \ \ 2021
 \hfill \textbf{\thepage}}}


\noindent
{\tabcolsep=3pt
\begin{tabular}{p{394pt}cc}
&\textbf{Вып.} & \textbf{Стр.}\\[3pt]
\Avtors{Коновалов~М.\,Г., Разумчик~Р.\,В.} Диспетчеризация в системе с параллельным обслужи-\linebreak
\\[-12pt]
\hspace*{23pt}ванием с помощью распределенного градиентного управления марковской цепью&3&41--50\\
\Avtors{Королев~В.\,Ю., Дорофеева~А.\,В.} О точности нормальной аппроксимации при отсутствии\linebreak
\\[-12pt]
\hspace*{23pt}нормальной сходимости&1&116--121\\
\Avtors{Кочеткова~И.\,А., Власкина~А.\,С., Ву~Н.\,Н., Шоргин~В.\,С.} Система массового обслуживания с управляемым по сигналам перераспределением приборов для анализа\linebreak
\\[-12pt]
\hspace*{23pt}нарезки ресурсов сети 5G&3&91--97\\
\Avtors{Кочеткова~И.\,А., Кущазли~А.\,И., Харин~П.\,А., Шоргин~С.\,Я.} Модель схемы приоритетного доступа 
трафика URLLC и~eMBB в~сети пятого поколения в~виде ресурсной\linebreak
\\[-12pt]
\hspace*{23pt}системы массового обслуживания&4&87--92\\
\Avtors{Кривенко~М.\,П.} Мягкие вычисления в задачах медицинской диагностики&2&52--59\\
\Avtors{Кружков~М.\,Г.} см.\ Гончаров~А.\,А.&&\\
\Avtors{Кружков~М.\,Г.} см.\ Гончаров~А.\,А.&&\\
\Avtors{Кружков~М.\,Г.} см.\ Инькова~О.\,Ю.&&\\
\Avtors{Кудрявцев~А.\,А., Шестаков~О.\,В.} Минимаксные оценки функции потерь, основанной на интегральных вероятностях ошибок при пороговой обработке вейвлет-\linebreak
\\[-12pt]
\hspace*{23pt}коэффициентов&4&12--19\\
\Avtors{Кудрявцев~А.\,А., Шестаков~О.\,В., Шоргин~С.\,Я.} Метод оценивания параметров изгиба,\linebreak
\\[-12pt]
\hspace*{23pt}формы и масштаба гамма-экспоненциального распределения&3&57--62\\
\Avtors{Кудрявцев~А.\,А.} см.\ Арутюнов~Е.\,Н.&&\\
\Avtors{Кузнецова~Р.\,В., Бахтеев~О.\,Ю., Чехович~Ю.\,В.} Методы обнаружения переводных\linebreak
\\[-12pt]
\hspace*{23pt}заимствований в больших текстовых коллекциях&1&30--41\\
\Avtors{Кузьмин~В.\,Ю.} см.\ Горшенин~А.\,К.&&\\
\Avtors{Кущазли~А.\,И.} см.\ Кочеткова~И.\,А.&&\\
\Avtors{Липатьев~А.\,А.} Неасимптотический анализ статистики Бартлетта--Нанда--Пилая для\linebreak
\\[-12pt]
\hspace*{23pt}данных большой размерности&1&72--77\\
\Avtors{Листопад~С.\,В.} см.\ Кириков~И.\,А.&&\\
\Avtors{Лощилова~Е.\,Ю.} см.\ Гончаров~А.\,А&&\\
\Avtors{Малашенко~Ю.\,Е.} Максимальные межузловые потоки при предельной загрузке много-\linebreak
\\[-12pt]
\hspace*{23pt}пользовательской сети&3&24--28\\
\Avtors{Малашенко~Ю.\,Е., Назарова~И.\,А.} Анализ распределения предельных нагрузок в~мно-\linebreak
\\[-12pt]
\hspace*{23pt}гопользовательской сети&4&20--26\\
\Avtors{Масляков~Г.\,О.} см.\ Дюкова~Е.\,В.&&\\
\Avtors{Молчанов~Д.\,А.} см.\ Дараселия~А.\,В.&&\\
\Avtors{Монахов~М.\,М.} Разложения Чебышёва--Эджворта для распределений обобщенных\linebreak
\\[-12pt]
\hspace*{23pt}статистик типа Хотеллинга, построенных по выборкам случайного размера&2&72--81\\
\Avtors{Назарова~И.\,А.} см.\ Малашенко~Ю.\,Е.&&\\
\Avtors{Наумов~А.\,В.} см.\ Босов~А.\,В.&&\\
\Avtors{Недоливко~Ю.\,Н.} см.\ Арутюнов~Е.\,Н.&&\\
\Avtors{Нуриев~В.\,А., Егорова~А.\,Ю.} Методы оценки качества машинного перевода: современное\linebreak
\\[-12pt]
\hspace*{23pt}состояние&2&104--111\\
\Avtors{Павлов~Ю.\,Л.} Связность конфигурационных графов в моделях сложных сетей&1&18--22\\
\Avtors{Разумчик~Р.\,В.} см.\ Коновалов~М.\,Г.&&\\
\Avtors{Рождественский~Ю.\,В.} см.\ Соколов~И.\,А.&&\\
\Avtors{Румовская~С.\,Б., Кириков~И.\,А.} Метод визуализации стимуляции конфликтов в гибрид-\linebreak
\\[-12pt]
\hspace*{23pt}ных интеллектуальных многоагентных системах&3&75--82\\
\Avtors{Самуйлов~К.\,Е.} см.\ Дараселия~А.\,В.&&\\
\Avtors{Сатин~Я.\,А.} см.\ Зейфман~А.\,И.&&\\
\Avtors{Севастьянов~Л.\,А.} см.\ Щетинин~Е.\,Ю.&&\\
\Avtors{Сигов~А.\,С., Андрианова~Е.\,Г., Истратов~Л.\,А.} Стохастическая динамика самоорганизу-\linebreak
\\[-12pt]
\hspace*{23pt}ющихся социальных систем с памятью (электоральные процессы)&2&112--121\\
\Avtors{Синицын~И.\,Н.} Нормальные субоптимальные фильтры для дифференциальных стоха-\linebreak
\\[-12pt]
\hspace*{23pt}стических систем, не разрешенных относительно производных&1&\hphantom{1}3--10\\
\end{tabular}
}

\pagebreak

\def\leftkol{АВТОРСКИЙ УКАЗАТЕЛЬ ЗА 2021 г.} % ENGLISH ABSTRACTS}

\def\rightkol{АВТОРСКИЙ УКАЗАТЕЛЬ ЗА 2021 г.} %ENGLISH ABSTRACTS}

%\thispagestyle{myheadings}
\def\leftfootline{\small{\textbf{\thepage}
\hfill ИНФОРМАТИКА И ЕЁ ПРИМЕНЕНИЯ\ \ \ том~15\ \ \ выпуск~4\ \ \ 2021}
}%
 \def\rightfootline{\small{ИНФОРМАТИКА И ЕЁ ПРИМЕНЕНИЯ\ \ \ том~15\ \ \ выпуск~4\ \ \ 2021
 \hfill \textbf{\thepage}}}


\noindent
{\tabcolsep=3pt
\begin{tabular}{p{394pt}cc}
&\textbf{Вып.} & \textbf{Стр.}\\[3pt]
\Avtors{Смирнов~Д.\,В.} см.\ Грушо~А.\,А.&&\\
\Avtors{Смирнов~Д.\,В.} см.\ Грушо~А.\,А.&&\\
\Avtors{Соколов~И.\,А., Степченков~Ю.\,А., Дьяченко~Ю.\,Г., Рождественский~Ю.\,В., Каменских~А.\,Н.}\linebreak
\\[-12pt]
\hspace*{23pt}Базис реализации сбоеустойчивых электронных схем&4&65--71\\
\Avtors{Сопин~Э.\,С.} см.\ Дараселия~А.\,В.&&\\
\Avtors{Степченков~Ю.\,А.} см.\ Соколов~И.\,А.&&\\
\Avtors{Стрижов~В.\,В.} см.\ Гребенькова~О.\,С.&&\\
\Avtors{Ступников~С.\,А.}см.\ Брюхов~Д.\,О.&&\\
\Avtors{Сушко~Д.\,В.} Алгоритмы сжатия данных массивов силовых кривых I: кодирование\linebreak
\\[-12pt]
\hspace*{23pt}ошибок предсказания&2&82--88\\
\Avtors{Сушко~Д.\,В.} Алгоритмы сжатия данных массивов силовых кривых II: кодирование\linebreak
\\[-12pt]
\hspace*{23pt}компонент вейвлет-преобразования&3&16--23\\
\Avtors{Тимонина~Е.\,Е.} см.\ Грушо~А.\,А.&&\\
\Avtors{Тимонина~Е.\,Е.} см.\ Грушо~А.\,А.&&\\
\Avtors{Тимонина~Е.\,Е.} см.\ Грушо~А.\,А.&&\\
\Avtors{Ушаков~В.\,Г., Ушаков~Н.\,Г.} Многомерные распределения выходящих потоков в системе\linebreak
\\[-12pt]
\hspace*{23pt}с абсолютным приоритетом&2&26--29\\
\Avtors{Ушаков~Н.\,Г.} см.\ Ушаков~В.\,Г.&&\\
\Avtors{Флёров~Ю.\,А.} см.\ Вышинский~Л.\,Л.&&\\
\Avtors{Флёров~Ю.\,А.} см.\ Вышинский~Л.\,Л.&&\\
\Avtors{Харин~П.\,А.} см.\ Кочеткова~И.\,А.&&\\
\Avtors{Хватова~Т.\,Ю.} см.\ Жуков~Д.\,О.&&\\
\Avtors{Чехович~Ю.\,В.} см.\ Кузнецова~Р.\,В.&&\\
\Avtors{Шанин~И.\,А.}см.\ Брюхов~Д.\,О.&&\\
\Avtors{Шестаков~О.\,В.} Анализ несмещенной оценки среднеквадратичного риска метода блоч-\linebreak
\\[-12pt]
\hspace*{23pt}ной пороговой обработки&2&30--35\\
\Avtors{Шестаков~О.\,В.} Пороговые функции в методах подавления шума, основанных на\linebreak
\\[-12pt]
\hspace*{23pt}вейвлет-разложении сигнала&3&51--56\\
\Avtors{Шестаков~О.\,В.} см.\ Кудрявцев~А.\,А.&&\\
\Avtors{Шестаков~О.\,В.} см.\ Кудрявцев~А.\,А.&&\\
\Avtors{Шнурков~П.\,В.} Создание стохастической динамической односекторной экономической модели с~дискретным временем и~анализ соответствующей задачи оптимального\linebreak
\\[-12pt]
\hspace*{23pt}управления&4&33--40\\
\Avtors{Шоргин~В.\,С.} см.\ Кочеткова~И.\,А.&&\\
\Avtors{Шоргин~С.\,Я.} см.\ Грушо~А.\,А.&&\\
\Avtors{Шоргин~С.\,Я.} см.\ Кочеткова~И.\,А.&&\\
\Avtors{Шоргин~С.\,Я.} см.\ Кудрявцев~А.\,А.&&\\
\Avtors{Щетинин~Е.\,Ю., Севастьянов~Л.\,А.} О методах переноса глубокого обучения в~задачах\linebreak
\\[-12pt]
\hspace*{23pt}классификации биомедицинских изображений&4&59--64\\
\end{tabular}
}

%\thispagestyle{myheadings}
\def\leftfootline{\small{\textbf{\thepage}
\hfill ИНФОРМАТИКА И ЕЁ ПРИМЕНЕНИЯ\ \ \ том~15\ \ \ выпуск~4\ \ \ 2021}
}%
 \def\rightfootline{\small{ИНФОРМАТИКА И ЕЁ ПРИМЕНЕНИЯ\ \ \ том~15\ \ \ выпуск~4\ \ \ 2021
 \hfill \textbf{\thepage}}}

 \label{end\stat}

\newpage

\def\stat{cont-e}
{%\hrule\par
%\vskip 7pt % 7pt
\raggedleft\Large \bf%\baselineskip=3.2ex
2\,0\,2\,1\ \ A\,U\,T\,H\,O\,R\ \ I\,N\,D\,E\,X \vskip 17pt
 \hrule
 \par
\vskip 21pt plus 6pt minus 3pt }

\label{st\stat}

\def\tit{\ }

\def\aut{\ }
\def\auf{\ }

\def\leftkol{\ } %2021 AUTHOR INDEX} % ENGLISH ABSTRACTS}

\def\rightkol{\ } %2021 AUTHOR INDEX} %ENGLISH ABSTRACTS}

\titele{\tit}{\aut}{\auf}{\leftkol}{\rightkol}
\addcontentsline{toc}{subsection}{\textrm\textbf 2021 Author Index}

\def\leftfootline{\small{\textbf{\thepage}
\hfill INFORMATIKA I EE PRIMENENIYA~--- INFORMATICS AND APPLICATIONS\ \ \ 2021\
\ \ volume~15\ \ \ issue\ 4}
}%
 \def\rightfootline{\small{INFORMATIKA I EE PRIMENENIYA~--- INFORMATICS AND APPLICATIONS\ \ \ 2021\ \ \ volume~15\ \ \ issue\ 4
\hfill \textbf{\thepage}}}

\vspace*{-24pt}

\noindent
{\tabcolsep=3pt
\begin{tabular}{p{395.89pt}cc}
&\textbf{Issue} & \textbf{Page}\\[6pt]
\Avtors{Abgaryan~K.\,K. and Gavrilov~E.\,S.} Distributed information system for calculating the structural\linebreak
\\[-12pt]
\hspace*{23pt}properties of composite materials&4&50--58\\
\Avtors{Agalarov~Ya.\,M.} Optimal threshold-based admission control in the $M/M/s$ system with hetero-\linebreak
\\[-12pt]
\hspace*{23pt}geneous servers and a common queue&1&57--64\\
\Avtors{Andrianova~E.\,G.} see Sigov~A.\,S.&&\\
\Avtors{Arutyunov~E.\,N., Kudryavtsev~A.\,A., and~Nedolivko~Iu.\,N.} Probabilistic characteristics of balance\linebreak
\\[-12pt]
\hspace*{23pt}index of factors with generalized gamma distribution&1&65--71\\
\Avtors{Bakhteev~O.\,Yu.} see Grebenkova~O.\,S.&&\\
\Avtors{Bakhteev~O.\,Yu.} see Kuznetsova~R.\,V.&&\\
\Avtors{Bazilevskiy~M.\,P.} Method of straightening distorted due to multicollinearity coefficients in\linebreak
\\[-12pt]
\hspace*{23pt}regression models&2&60--65\\
\Avtors{Borisov~A.\,V.\ and~Kazanchyan~D.\,Kh.} Filtering of Markov jump processes given composite\linebreak
\\[-12pt]
\hspace*{23pt}ob\-ser\-va\-tions I: Exact solution&2&12--19\\
\Avtors{Borisov~A.\,V.\ and~Kazanchyan~D.\,Kh.} Filtering of Markov jump processes given composite\linebreak
\\[-12pt]
\hspace*{23pt}observations II: Numerical algorithm&3&\hphantom{1}9--15\\
\Avtors{Bosov~A.\,V.} Linear output control of Markov chains by the quadratic criterion&2&\hphantom{1}3--11\\
\Avtors{Bosov~A.\,V.} On some special cases in the problem of stochastic differential system output control\linebreak
\\[-12pt]
\hspace*{23pt}by the quadratic criterion&1&11--17\\
\Avtors{Bosov~A.\,V., Ignatov~A.\,N., and Naumov~A.\,V.} Algorithms for an approximate solution of the\linebreak
\\[-12pt]
\hspace*{23pt}track possession problem on the railway network segment&4&\hphantom{1}3--11\\
\Avtors{Bosov~A.\,V.\ and~Zhukov~D.\,V.} Expert system for monitoring and forecasting of resource allocation\linebreak
\\[-12pt]
\hspace*{23pt}processes&3&29--40\\
\Avtors{Briukhov~D.\,O., Stupnikov~S.\,A., Kovalev~D.\,Yu., and~Shanin~I.\,A.} An architecture for distributed\linebreak
\\[-12pt]
\hspace*{23pt}data analysis problem solving in neurophysiology&1&78--85\\
\Avtors{Chekhovich~Yu.\,V.} see Kuznetsova~R.\,V.&&\\
\Avtors{Daraseliya~А.\,V., Sopin~E.\,S., Moltchanov~D.\,А., and~Samouylov~K.\,E.} Analysis of 5G NR base\linebreak
\\[-12pt]
\hspace*{23pt}stations offloading by means of NR-U technology&3&98--111\\
\Avtors{Diachenko~Yu.\,G.} see Sokolov~I.\,A.&&\\
\Avtors{Djukova~E.\,V. and Masliakov~G.\,O.} On the choice of partial orders on feature values sets in the\linebreak
\\[-12pt]
\hspace*{23pt}supervised classification problem&4&72--78\\
\Avtors{Dorofeeva~A.\,V.} see Korolev~V.\,Yu.&&\\
\Avtors{Egorova~A.\,Yu.} see Nuriev~V.\,A.&&\\
\Avtors{Flerov~Yu.\,A.} see Vyshinsky~L.\,L.&&\\
\Avtors{Flerov~Yu.\,A.} see Vyshinsky~L.\,L.&&\\
\Avtors{Gavrilov~E.\,S.} see Abgaryan~K.\,K.&&\\
\Avtors{Goncharenko~M.\,B.\ and~Zakharova~T.\,V.} Some properties of Gaussian mixtures and applications\linebreak
\\[-12pt]
\hspace*{23pt}to magnetoencephalography problems&2&44--51\\
\Avtors{Goncharov~A.\,A.\ and~Inkova~O.\,Yu.} Extracting knowledge about means of expression of\linebreak
\\[-12pt]
\hspace*{23pt}logical-semantic relations from the supracorpora database&2&\hphantom{1}96--103\\
\Avtors{Goncharov~A.\,A.\ and~Zatsman~I.\,M.} Structuring principles of electronic dictionary's entries&2&89--95\\
\Avtors{Goncharov~A.\,A., Zatsman~I.\,M., and~Kruzhkov~M.\,G.} Representation of new lexicographical\linebreak
\\[-12pt]
\hspace*{23pt}knowledge in dynamic classification systems&1&86--93\\
\Avtors{Goncharov~A.\,A., Zatsman~I.\,M., Kruzhkov~M.\,G., and Loshchilova~E.\,Yu.} Capturing evolution\linebreak
\\[-12pt]
\hspace*{23pt}of lexicographic knowledge in dynamic classification systems&4&41--49\\

\Avtors{Gorshenin~A.\,K.\ and~Kuzmin~V.\,Yu.} Method for improving accuracy of neural network forecasts\linebreak
\\[-12pt]
\hspace*{23pt}based on probability mixture models and its implementation as a digital service&3&63--74\\
\end{tabular}
}
\pagebreak

\def\leftfootline{\small{\textbf{\thepage}
\hfill INFORMATIKA I EE PRIMENENIYA~--- INFORMATICS AND APPLICATIONS\ \ \ 2021\
\ \ volume~15\ \ \ issue\ 4}
}%
 \def\rightfootline{\small{INFORMATIKA I EE PRIMENENIYA~---
INFORMATICS AND APPLICATIONS\ \ \ 2021\ \ \ volume~15\ \ \ issue\ 4
\hfill \textbf{\thepage}}}

\def\leftkol{2021 AUTHOR INDEX} % ENGLISH ABSTRACTS}

\def\rightkol{2021 AUTHOR INDEX} %ENGLISH ABSTRACTS}


\noindent
{\tabcolsep=3pt
\begin{tabular}{p{395.5pt}cc}
&\textbf{Issue} & \textbf{Page}\\[6pt]
\Avtors{Grebenkova~O.\,S., Bakhteev~O.\,Yu., and~Strijov~V.\,V.} Variational deep learning model optimi-\linebreak
\\[-12pt]
\hspace*{23pt}zation with complexity control&1&42--49\\[-.15pt]
\Avtors{Grinchenko~S.\,N.} Anthropogenic ``third'' nature: The relatively autonomous status of its artificial\linebreak
\\[-12pt]
\hspace*{23pt}intellectual subjects&4&110--114\\[-.15pt]
\Avtors{Grinchenko~S.\,N.} On the system hierarchy of artificial intelligence&1&111--115\\[-.15pt]
\Avtors{Grusho~A.\,A., Grusho~N.\,A., Zabezhailo~M.\,I., Smirnov~D.\,V., Timonina~E.\,E., and Shorgin~S.\,Ya.}\linebreak
\\[-12pt]
\hspace*{23pt}Statistics and clusters for detection of anomalous insertions in Big Data
en\-vi\-ron\-ment&4&79--86\\[-.15pt]
\Avtors{Grusho~A.\,A., Grusho~N.\,A., Zabezhailo~M.\,I., and~Timonina~E.\,E.} Remote monitoring of\linebreak
\\[-12pt]
\hspace*{23pt}workflows&3&2--8\\[-.15pt]
\Avtors{Grusho~A.\,A., Zabezhailo~M.\,I., Smirnov~D.\,V., and~Timonina~E.\,E.} Intelligent analysis of Big\linebreak
\\[-12pt]
\hspace*{23pt}Data extendible collections under the limits of process-real time&2&36--43\\[-.15pt]
\Avtors{Grusho~N.\,A.} see Grusho~A.\,A.&&\\[-.15pt]
\Avtors{Grusho~N.\,A.} see Grusho~A.\,A.&&\\[-.15pt]
\Avtors{Ignatov~A.\,N.} see Bosov~A.\,V.&&\\[-.15pt]
\Avtors{Inkova~O.\,Yu.~and Kruzhkov~M.\,G.} Structured definitions of discourse relations in the\linebreak
\\[-12pt]
\hspace*{23pt}Supracorpora Database of Connectives&4&27--32\\[-.15pt]
\Avtors{Inkova~O.\,Yu.} see Goncharov~A.\,A.&&\\[-.15pt]
\Avtors{Istratov~L.\,A.} see Sigov~A.\,S.&&\\[-.15pt]
\Avtors{Kamenskih~A.\,N.} see Sokolov~I.\,A.&&\\[-.15pt]
\Avtors{Kazanchyan~D.\,Kh.} see Borisov~A.\,V.&&\\[-.15pt]
\Avtors{Kazanchyan~D.\,Kh.} see Borisov~A.\,V.&&\\[-.15pt]
\Avtors{Kharin~P.\,A.} see Kochetkova~I.\,A.&&\\[-.15pt]
\Avtors{Khvatova~T.\,Yu.} see Zhukov~D.\,O.&&\\[-.15pt]
\Avtors{Kirikov~I.\,A.\ and~Listopad~S.\,V.} Coordination of agents' goals in cohesive hybrid intelligent\linebreak
\\[-12pt]
\hspace*{23pt}multiagent systems&2&66--71\\[-.15pt]
\Avtors{Kirikov~I.\,A.} see Rumovskaya~S.\,B.&&\\[-.15pt]
\Avtors{Kochetkova~I.\,A., Kushchazli~A.\,I., Kharin~P.\,A., and Shorgin~S.\,Ya.} Model for analyzing priority 
admission control of URLLC and eMBB communications in 5G networks as a~resource\linebreak
\\[-12pt]
\hspace*{23pt}queuing system&4&87--92\\[-.15pt]
\Avtors{Kochetkova~I.\,A., Vlaskina~A.\,S., Vu~N.\,N., and~Shorgin~V.\,S.} Queuing system with signals for\linebreak
\\[-12pt]
\hspace*{23pt}dynamic resource allocation for analyzing network slicing in 5G networks&3&91--97\\[-.15pt]
\Avtors{Konovalov~M.\,G.\ and~Razumchik~R.\,V.} Routing jobs to heterogeneous parallel queues using\linebreak
\\[-12pt]
\hspace*{23pt}distributed policy grandient algorithm&3&41--50\\[-.15pt]
\Avtors{Korolev~V.\,Yu.\ and~Dorofeeva~A.\,V.} On the accuracy of the normal approximation under the\linebreak
\\[-12pt]
\hspace*{23pt}violation of the normal convergence&1&116--121\\[-.15pt]
\Avtors{Kovalev~D.\,Yu.} see Briukhov~D.\,O.&&\\[-.15pt]
\Avtors{Kovalev~I.\,A.} see Zeifman~A.\,I.&&\\[-.15pt]
\Avtors{Kovalyov~S.\,P.} Methods of the category theory in digital design of heterogeneous cyber-physical\linebreak
\\[-12pt]
\hspace*{23pt}systems&1&23--29\\[-.15pt]
\Avtors{Krivenko~M.\,P.} Soft computing in problems of medical diagnostics&2&52--59\\[-.15pt]
\Avtors{Kruzhkov~M.\,G.} see Goncharov~A.\,A&&\\[-.15pt]
\Avtors{Kruzhkov~M.\,G.} see Goncharov~A.\,A.&&\\[-.15pt]
\Avtors{Kruzhkov~M.\,G.} see Inkova~O.\,Yu.&&\\[-.15pt]
\Avtors{Kudryavtsev~A.\,A. and Shestakov~O.\,V.} Minimax estimates of the loss function based on integral\linebreak
\\[-12pt]
\hspace*{23pt}error probabilities during threshold processing of wavelet coefficients&4&12--19\\[-.15pt]
\Avtors{Kudryavtsev~A.\,A., Shestakov~O.\,V., and~Shorgin~S.\,Ya.} A~method for estimating bent, shape and\linebreak
\\[-12pt]
\hspace*{23pt}scale parameters of the gamma-exponential distribution&3&57--62\\[-.15pt]
\Avtors{Kudryavtsev~A.\,A.} see Arutyunov~E.\,N.&&\\[-.15pt]
\Avtors{Kushchazli~A.\,I.} see Kochetkova~I.\,A.&&\\[-.15pt]
\Avtors{Kuzmin~V.\,Yu.} see Gorshenin~A.\,K.&&\\[-.15pt]
\Avtors{Kuznetsova~R.\,V., Bakhteev~O.\,Yu., and~Chekhovich~Yu.\,V.} Methods of cross-lingual text reuse\linebreak
\\[-12pt]
\hspace*{23pt}detection in large textual collections&1&30--41\\[-.15pt]
\Avtors{Lipatiev~A.\,A.} Nonasymptotic analysis of Bartlett--Nanda--Pillai statistic for high-dimensional\linebreak
\\[-12pt]
\hspace*{23pt}data&1&72--77\\
\end{tabular}
}
\pagebreak

\def\leftfootline{\small{\textbf{\thepage}
\hfill INFORMATIKA I EE PRIMENENIYA~--- INFORMATICS AND APPLICATIONS\ \ \ 2021\
\ \ volume~15\ \ \ issue\ 4}
}%
 \def\rightfootline{\small{INFORMATIKA I EE PRIMENENIYA~---
INFORMATICS AND APPLICATIONS\ \ \ 2021\ \ \ volume~15\ \ \ issue\ 4
\hfill \textbf{\thepage}}}

\def\leftkol{2021 AUTHOR INDEX} % ENGLISH ABSTRACTS}

\def\rightkol{2021 AUTHOR INDEX} %ENGLISH ABSTRACTS}


\noindent
{\tabcolsep=3pt
\begin{tabular}{p{395.5pt}cc}
&\textbf{Issue} & \textbf{Page}\\[6pt]
\Avtors{Listopad~S.\,V.} see Kirikov~I.\,A.&&\\
\Avtors{Loshchilova~E.\,Yu.} see Goncharov~A.\,A.&&\\
\Avtors{Malashenko~Yu.\,E.} Maximum internode flows at peak load of a multiuser network&3&24--28\\
\Avtors{Malashenko~Yu.\,E. and Nazarova~I.\,A.} Analysis of peak load distribution in the multiuser\linebreak
\\[-12pt]
\hspace*{23pt}network&4&20--26\\
\Avtors{Masliakov~G.\,O.} see Djukova~E.\,V.&&\\
\Avtors{Moltchanov~D.\,А.} see Daraseliya~А.\,V.&&\\
\Avtors{Monakhov~M.\,M.} Chebyshev--Edgeworth expansions for distributions of generalised Hotelling-\linebreak
\\[-12pt]
\hspace*{23pt}type statistics based on random size samples&2&72--81\\
\Avtors{Naumov~A.\,V.} see Bosov~A.\,V.&&\\
\Avtors{Nazarova~I.\,A.} see Malashenko~Yu.\,E.&&\\
\Avtors{Nedolivko~Iu.\,N.} see Arutyunov~E.\,N.&&\\
\Avtors{Nuriev~V.\,A.\ and~Egorova~A.\,Yu.} Methods of quality estimation for machine translation:\linebreak
\\[-12pt]
\hspace*{23pt}State-of-the-art&2&104--111\\
\Avtors{Pavlov~Yu.\,L.} Connectivity of configuration graphs in complex network models&1&18--22\\
\Avtors{Razumchik~R.\,V.} see Konovalov~M.\,G.&&\\
\Avtors{Rogdestvenski~Yu.\,V.} see Sokolov~I.\,A.&&\\
\Avtors{Rumovskaya~S.\,B.\ and~Kirikov~I.\,A.} Visual representation method for the conflict stimulation in\linebreak
\\[-12pt]
\hspace*{23pt}hybrid intelligent multiagent systems&3&75--82\\
\Avtors{Samouylov~K.\,E.} see Daraseliya~А.\,V.&&\\
\Avtors{Satin~Ya.\,A.} see Zeifman~A.\,I.&&\\
\Avtors{Sevastianov~L.\,A.} see Shchetinin~E.\,Yu.&&\\
\Avtors{Shanin~I.\,A.} see Briukhov~D.\,O.&&\\
\Avtors{Shchetinin~E.\,Yu. and Sevastianov~L.\,A.} On transfer learning methods in biomedical images\linebreak
\\[-12pt]
\hspace*{23pt}classification tasks&4&59--64\\
\Avtors{Shestakov~O.\,V.} Analysis of the unbiased mean-square risk estimate of the block thresholding\linebreak
\\[-12pt]
\hspace*{23pt}method&2&30--35\\
\Avtors{Shestakov~O.\,V.} Thresholding functions in the noise suppression methods based on the wavelet\linebreak
\\[-12pt]
\hspace*{23pt}expansion of the signal&3&51--56\\
\Avtors{Shestakov~O.\,V.} see Kudryavtsev~A.\,A.&&\\
\Avtors{Shestakov~O.\,V.} see Kudryavtsev~A.\,A.&&\\
\Avtors{Shnurkov~P.\,V.} Creation of a stochastic dynamic one-sector economic model with discrete time\linebreak
\\[-12pt]
\hspace*{23pt}and analysis of the corresponding optimal control problem&4&33--40\\
\Avtors{Shorgin~S.\,Ya.} see Grusho~A.\,A.&&\\
\Avtors{Shorgin~S.\,Ya.} see Kochetkova~I.\,A.&&\\
\Avtors{Shorgin~S.\,Ya.} see Kudryavtsev~A.\,A. &&\\
\Avtors{Shorgin~V.\,S.} see Kochetkova~I.\,A.&&\\
\Avtors{Sigov~A.\,S., Andrianova~E.\,G., and~Istratov~L.\,A.} Stochastic dynamics of self-organizing social\linebreak
\\[-12pt]
\hspace*{23pt}systems with memory (electoral processes)&2&112--121\\
\Avtors{Sinitsyn~I.\,N.} Normal suboptimal filtering for differential stochastic systems with unsolved\linebreak
\\[-12pt]
\hspace*{23pt}derivatives&1&\hphantom{1}3--10\\
\Avtors{Smirnov~D.\,V.} see Grusho~A.\,A.&&\\
\Avtors{Smirnov~D.\,V.} see Grusho~A.\,A.&&\\
\Avtors{Sokolov~I.\,A., Stepchenkov~Yu.\,A., Diachenko~Yu.\,G., Rogdestvenski~Yu.\,V., and Kamenskih~A.\,N.}\linebreak
\\[-12pt]
\hspace*{23pt}The electronic component base of failure resilience digital circuits&4&65--71\\
\Avtors{Sopin~E.\,S.} see Daraseliya~А.\,V.&&\\
\Avtors{Stepchenkov~Yu.\,A.} see Sokolov~I.\,A.&&\\
\Avtors{Strijov~V.\,V.} see Grebenkova~O.\,S.&&\\
\Avtors{Stupnikov~S.\,A.} see Briukhov~D.\,O.&&\\
\Avtors{Sushko~D.\,V.} Compression algorithms for force volume data I: Coding of prediction errors&2&82--88\\
\Avtors{Sushko~D.\,V.} Compression algorithms for force volume data II: Coding of wavelet transform\linebreak
\\[-12pt]
\hspace*{23pt}components&3&16--23\\
\Avtors{Timonina~E.\,E.} see Grusho~A.\,A.&&\\
\end{tabular}
}
\pagebreak

\def\leftfootline{\small{\textbf{\thepage}
\hfill INFORMATIKA I EE PRIMENENIYA~--- INFORMATICS AND APPLICATIONS\ \ \ 2021\
\ \ volume~15\ \ \ issue\ 4}
}%
 \def\rightfootline{\small{INFORMATIKA I EE PRIMENENIYA~---
INFORMATICS AND APPLICATIONS\ \ \ 2021\ \ \ volume~15\ \ \ issue\ 4
\hfill \textbf{\thepage}}}

\def\leftkol{2021 AUTHOR INDEX} % ENGLISH ABSTRACTS}

\def\rightkol{2021 AUTHOR INDEX} %ENGLISH ABSTRACTS}


\noindent
{\tabcolsep=3pt
\begin{tabular}{p{395.5pt}cc}
&\textbf{Issue} & \textbf{Page}\\[6pt]
\Avtors{Timonina~E.\,E.} see Grusho~A.\,A.&&\\
\Avtors{Timonina~E.\,E.} see Grusho~A.\,A.&&\\
\Avtors{Ushakov~N.\,G.} see Ushakov~V.\,G.&&\\
\Avtors{Ushakov~V.\,G.\ and~Ushakov~N.\,G.} The multivariate distributions of output streams in a queueing\linebreak
\\[-12pt]
\hspace*{23pt}system with preemptive repeat priority&2&26--29\\
\Avtors{Vlaskina~A.\,S.} see Kochetkova~I.\,A.&&\\
\Avtors{Vu~N.\,N.} see Kochetkova~I.\,A.&&\\
\Avtors{Vyshinsky~L.\,L.\ and~Flerov~Yu.\,A.} Information model of aircraft weight profile&1&50--56\\
\Avtors{Vyshinsky~L.\,L. and Flerov~Yu.\,A.} Theoretical foundation of formation of aircraft weight\linebreak
\\[-12pt]
\hspace*{23pt}appearance&4&\hphantom{1}93--102\\
\Avtors{Zabezhailo~M.\,I.} see Grusho~A.\,A.&&\\
\Avtors{Zabezhailo~M.\,I.} see Grusho~A.\,A.&&\\
\Avtors{Zabezhailo~M.\,I.} see Grusho~A.\,A.&&\\
\Avtors{Zakharova~T.\,V.} see Goncharenko~M.\,B.&&\\
\Avtors{Zaltcman~A.\,D.} see Zhukov~D.\,O.&&\\
\Avtors{Zatsman~I.\,M.} Forms representing new knowledge discovered in texts&3&83--90\\
\Avtors{Zatsman~I.\,M.} Problem-oriented updating of dictionary entries of bilingual dictionaries and\linebreak
\\[-12pt]
\hspace*{23pt}medical terminology: Comparative analysis&1&\hphantom{1}94--101\\
\Avtors{Zatsman~I.\,M.} The conception of creating WHO Hub for Pandemic and Epidemic Intelligence:\linebreak
\\[-12pt]
\hspace*{23pt}Keywords and their terminological analysis&4&103--109\\
\Avtors{Zatsman~I.\,M.} see Goncharov~A.\,A.&&\\
\Avtors{Zatsman~I.\,M.} see Goncharov~A.\,A.&&\\
\Avtors{Zatsman~I.\,M.} see Goncharov~A.\,A.&&\\
\Avtors{Zeifman~A.\,I., Satin~Ya.\,A., and~Kovalev~I.\,A.} On one nonstationary service model with\linebreak
\\[-12pt]
\hspace*{23pt}catastrophes and heavy tails&2&20--25\\
\Avtors{Zhukov~D.\,O., Khvatova~T.\,Yu., and~Zaltcman~A.\,D.} Modeling of the stochastic dynamics of changes in node states and percolation transitions in social networks with self-organization\linebreak
\\[-12pt]
\hspace*{23pt}and memory&1&102--110\\
\Avtors{Zhukov~D.\,V.} see Bosov~A.\,V.&&\\
\end{tabular}
}

%\thispagestyle{myheadings}
\def\leftfootline{\small{\textbf{\thepage}
\hfill INFORMATIKA I EE PRIMENENIYA~--- INFORMATICS AND APPLICATIONS\ \ \ 2021\
\ \ volume~15\ \ \ issue\ 4}
}%
 \def\rightfootline{\small{INFORMATIKA I EE PRIMENENIYA~---
INFORMATICS AND APPLICATIONS\ \ \ 2021\ \ \ volume~15\ \ \ issue\ 4
\hfill \textbf{\thepage}}}

 \label{end\stat}

\newpage


%\linebreak
%\\[-12pt]
%\hspace*{23pt}


   \vspace*{-46pt}

\begin{center}
\vspace*{4pt}
\mbox{%

\epsfxsize=55mm %112.705
\epsfbox{zhur-2.eps}
}
%\end{center}

\vspace*{10pt} 


%   \begin{center}
\fbox{\large\textbf{Академик Юрий Иванович Журавлёв}}\\[10pt]
\textbf{\large 14.01.1935--14.01.2022}
   \end{center}


   %\vspace*{2.5mm}

   \vspace*{5mm}

   \thispagestyle{empty}

%\

%\vspace*{-12pt}
       


В январе этого года ушел из жизни главный научный сотрудник Федерального исследовательского 
центра <<Информатика и управление>> РАН, председатель Редакционного совета журнала 
<<Информатика и~её применения>> академик Юрий Иванович Журавлёв. В~его лице мировая 
наука потеряла одного из своих ярчайших представителей~--- выдающегося ученого-исследователя 
и~талантливого ученого-организатора.

Юрий Иванович родился в Воронеже в 1935~г.\ в семье ученого и врача. Среднее образование 
получил в школе №\,6 г.~Фрунзе (ныне Бишкек) Киргизской ССР. В~1952~г.\ поступил на 
ме\-ха\-ни\-ко-ма\-те\-ма\-ти\-че\-ский факультет МГУ им.\ М.\,В.~Ломоносова. В~1957~г.\ Юрий Иванович 
защищает диплом и продолжает обучение в аспирантуре Московского университета на кафедре 
вычислительной математики (возглавляемой тогда академиком С.\,Л.~Соболевым). После 
успешной защиты кандидатской диссертации (к.ф.-м.н., 1959 г., научный руководитель~--- 
А.\,А.~Ляпунов, оппоненты~--- чл.-корр.\ А.\,А.~Марков, к.ф.-м.н.\ О.\,Б.~Лупанов) и~до 
окончательного переезда в Москву в 1969~г.\ работал в Институте математики Сибирского 
отделения АН СССР, занимая в нем последовательно должности младшего научного сотрудника, 
заведующего отделом, заведующего отделением, заместителя директора по научной работе. 
В~этот период (1954--1966~гг.)\ им был опубликован цикл работ по решению задач алгебры и 
математической логики, причем полученные результаты применялись для создания эффективных 
программ для ЭВМ, конструирования схем и сетей для обработки информации. Наиболее значимый 
результат этого периода научной работы~--- обоснование нового направления исследований, 
общей теории локальных алгоритмов. В~ней были окончательно объединены топологические 
принципы и теория алгоритмов. Эта теория и легла в основу докторской диссертации Юрия 
Ивановича (д.ф.-м.н., 1965~г.)\ по еще тогда новой научной специальности <<Математическая 
кибернетика>>. Оппонировали ему как специалисты по кибернетике~--- академик 
В.\,М.~Глушков, член-корреспондент А.\,А.~Ляпунов и О.\,Б.~Лупанов, так и про\-фес\-сор-ал\-геб\-раист А.\,Д.~Тайманов. 

В 1969~г.\ Юрий Иванович переезжает в Москву и возглавляет в Вычислительном центре АН 
СССР лабораторию проблем распознавания. Впоследствии он~--- заместитель директора по 
научной работе. Научные интересы этого периода связаны с проблемами классификации или 
распознавания образов. В~1976--1978~гг.\ Юрий Иванович публикует цикл работ по ставшему 
вскоре знаменитым алгебраическому подходу к проблеме синтеза корректных алгоритмов. Эти 
работы определили современное состояние всей проблематики распознавания и многих смежных 
областей прикладной математики и информатики. В~своих основополагающих работах Юрий 
Иванович показал, что можно в явном виде строить экстремальные по качеству алгоритмы для 
решения очень широких классов плохо формализованных задач. 
{\looseness=-1

}





Научные заслуги Юрия Ивановича получили широкое признание. В~1966~г.\ он совместно с 
О.\,Б.~Лупановым и чле\-ном-кор\-рес\-пон\-ден\-том АН СССР С.\,В.~Яблонским были удостоены 
звания лауреата Ленинской премии в~об\-ласти науки и техники. В~1984~г.\ Юрий Иванович 
был избран членом-корреспондентом АН СССР (по специальности <<Информатика>>), 
а~в~1992~г.~--- академиком РАН (по той же специальности).\linebreak\vspace*{-12pt}

\pagebreak

\

\vspace*{-46pt}

\noindent
\begin{floatingfigure}{48mm}
\begin{center}
%\vspace*{6pt}
\mbox{%

\epsfxsize=46mm %112.705
\epsfbox{zhur-3.eps}
}
\end{center}
\vspace*{6pt}
\end{floatingfigure}

 \thispagestyle{empty}

\noindent
В~1986~г.\ за цикл прикладных 
работ ему и ряду его учеников была при\-суж\-де\-на премия Совета Министров СССР. Он являлся 
членом иностранных академий наук, председателем секции <<Прикладная математика
 и~информатика>> Отделения математических наук РАН, председателем экспертного совета ВАК 
России по управ\-ле\-нию и информатике, заслуженным профессором нескольких университетов, 
председателем Российской ассоциации <<Распознавание образов и обработка изображений>>, 
членом исполкома Международной ассоциации IAPR (распознавание образов и обработка 
изображений). Был награжден 8-ю орденами и медалями СССР и России.

Юрий Иванович проводил большую научно-литературную работу, являясь, в том числе, главным 
редактором международных научных журналов и членом редколлегий ряда рецензируемых 
научных журналов. 


Параллельно с активной научной деятельностью Юрий Иванович вел и преподавательскую 
работу. С~1961 по~1969~гг.~--- в Новосибирском государственном университете на кафедре 
алгебры и математической логики, которую возглавлял в то время академик А.\,И.~Мальцев. 
С~1970~г., будучи уже профессором (1967~г.),~--- в Московском физико-техническом институте 
на кафедре академика Н.\,Н.~Моисеева. В~1997~г.\ по предложению ректора МГУ им.\ 
М.\,В.~Ломоносова академика В.\,А.~Садовничего Юрий Иванович организовал на факультете 
Вычислительной математики и кибернетики новую кафедру <<Математические методы 
прогнозирования>>, которой и руководил до конца жизни. В~2008~г.\ ему была присуждена 
премия Совета Министров РФ в области образования. С~1965~г.\ Юрий Иванович периодически 
читал курсы лекций за рубежом, в университетах США, Франции, Финляндии, Швеции, Австрии, 
Польши, Болгарии, ГДР и других стран. Эта работа в существенной степени обеспечила широкое 
международное признание советской и российской науки в области дискретной математики и~распознавания образов. 

%\begin{floatingfigure}{60mm}
\begin{figure}[b]
\begin{center}
\vspace*{-6pt}
\mbox{%

\epsfxsize=112mm %90mm %112.705
\epsfbox{zhur-1.eps}
}
\end{center}
\end{figure}
%\end{floatingfigure}

Понимая важность вопроса воспитания подрастающего поколения для развития науки в стране, 
Юрий Иванович вскоре после защиты первой диссертации включился в работу по подготовке 
научных кадров. Им создана большая научная школа: под руководством Юрия Ивановича 
защищены более 100~кандидатских диссертаций по всевозможным разделам естествознания 
(математике, информатике, медицине, технике, экономике, геологии), не один десяток докторов 
наук. Он воспитал академиков и членов-корреспондентов РАН и академий государств СНГ. 
С~большим вниманием и участием Юрий Иванович относился к развитию научных школ страны 
в~об\-ласти обработки изображений, распознавания образов и компьютерной оптики. 

Для всех коллег и учеников Юрия Ивановича он останется примером замечательного человека, 
та\-лант\-ли\-во\-го педагога и выдающегося, преданного служению науке ученого. 


%\def\stat{cont}
{%\hrule\par
%\vskip 7pt % 7pt
\raggedleft\Large \bf%\baselineskip=3.2ex
А\,В\,Т\,О\,Р\,С\,К\,И\,Й\ \ У\,К\,А\,З\,А\,Т\,Е\,Л\,Ь\ \ З\,А\ \ 2\,0\,1\,0 г. \vskip 17pt
    \hrule
    \par
\vskip 21pt plus 6pt minus 3pt }

\label{st\stat}

\def\tit{\ }

\def\aut{\ }
\def\auf{\ }

\def\leftkol{\ } % ENGLISH ABSTRACTS}

\def\rightkol{\ } %АВТОРСКИЙ УКАЗАТЕЛЬ ЗА 2010 г.} %ENGLISH ABSTRACTS}

\titele{\tit}{\aut}{\auf}{\leftkol}{\rightkol}

\vspace*{-12pt}

{\tabcolsep=3pt
\begin{tabular}{p{388pt}rr}
&\textbf{Выпуск} & \textbf{Стр.}\\[6pt]
\hangindent=23pt\noindent\textbf{Арутюнян~А.\,Р.} Моделирование влияния деформаций отпечатков пальцев на 
точность\linebreak
\vspace*{-12pt}\\
\hspace*{23pt}дактилоскопической идентификации$\dotfill$&1&51\\
\hangindent=23pt\noindent\textbf{Архипов~О.\,П., Зыкова~З.\,П.} Интеграция гетерогенной информации о цветных 
пикселях\linebreak
\vspace*{-12pt}\\
\hspace*{23pt}и их цветовосприятии$\dotfill$&4&15\\
\hangindent=23pt\noindent\textbf{Баранов~С.\,И., Френкель~С.\,Л., Захаров~В.\,Н.} Полуформальная верификация 
цифрового устройства с конвейером, основанная на использовании алгоритмических машин\linebreak
\vspace*{-12pt}\\
\hspace*{23pt}состояния$\dotfill$&4&49\\
\textbf{Бекетова~И.\,В.} см.~Каратеев~С.\,Л.&&\\
\textbf{Белоусов~В.\,В.} см.~Синицын~И.\,Н.&&\\
\hangindent=23pt\noindent\textbf{Бенинг~В.\,Е., Королев~Р.\,А.} О предельном поведении мощностей критериев в 
случае\linebreak
\vspace*{-12pt}\\
\hspace*{23pt}распределения Лапласа$\dotfill$&2&63\\
\hangindent=23pt\noindent\textbf{Бенинг~В.\,Е., Сипина~А.\,В.} Асимптотическое разложение для мощности 
критерия,\linebreak
\vspace*{-12pt}\\
\hspace*{23pt}основанного на выборочной медиане, в случае распределения Лапласа$\dotfill$&1&18\\
\textbf{Бондаренко~А.\,В.} см.~Каратеев~С.\,Л.&&\\
\hangindent=23pt\noindent\textbf{Бородина~А.\,В., Морозов~Е.\,В.} Об оценивании асимптотики вероятности 
большого\linebreak
\vspace*{-12pt}\\
\hspace*{23pt}уклонения стационарной регенеративной очереди с одним прибором$\dotfill$&3&29\\
\hangindent=23pt\noindent\textbf{Бунтман~Н.\,В., Минель~Ж.-Л., Ле~Пезан~Д., Зацман~И.\,М.} Типология и 
компьютерное\linebreak
\vspace*{-12pt}\\
\hspace*{23pt}моделирование трудностей перевода$\dotfill$&3&77\\
\textbf{Визильтер~Ю.\,В.} см.~Каратеев~С.\,Л.&&\\
\hangindent=23pt\noindent\textbf{Гавриленко~С.\,В.} Оценки скорости сходимости распределений случайных сумм с 
безгранично делимыми индексами к нормальному закону$\dotfill$&4&81\\
\hangindent=23pt\noindent\textbf{Григорьева~М.\,Е., Шевцова~И.\,Г.} Уточнение неравенства 
Каца--Берри--Эссеена$\dotfill$&2&75\\
\hangindent=23pt\noindent\textbf{Грушо~А.\,А., Грушо~Н.\,А., Тимонина~Е.\,Е.} Поиск конфликтов в политиках 
безопасности: модель случайных графов$\dotfill$&3&38\\
\textbf{Грушо~Н.\,А.} см.~Грушо~А.\,А.&&\\
\hangindent=23pt\noindent\textbf{Гудков~В.\,Ю.} Математические модели изображения отпечатка пальца на основе 
описания линий$\dotfill$&1&58\\
\textbf{Гуртов~А.\,В.} см.~Лукьяненко~А.\,С.&&\\
\textbf{Желтов~С.\,Ю.} см.~Каратеев~С.\,Л.&&\\
\hangindent=23pt\noindent\textbf{Захаров~А.\,А., Серебряков~В.\,А.} Система управления электронной библиотекой 
LibMeta$\dotfill$&4&2\\
\textbf{Захаров~В.\,Н.} см.~Баранов~С.\,И.&&\\
\textbf{Захарова~Т.\,В.} см.~Матвеева~С.\,С.&&\\
\hangindent=23pt\noindent\textbf{Зацаринный~А.\,А., Чупраков~К.\,Г.} Некоторые аспекты выбора технологии для 
постро-\linebreak
\vspace*{-12pt}\\
\hspace*{23pt}ения систем отображения информации ситуационного центра$\dotfill$&3&59\\
\textbf{Зацман~И.\,М.} см.~Бунтман~Н.\,В.&&\\
\hangindent=23pt\noindent\textbf{Зейфман~А.\,И., Коротышева~А.\,В., Сатин~Я.\,А., Шоргин~С.\,Я.} Об 
устойчивости нестаци-\linebreak
\vspace*{-12pt}\\
\hspace*{23pt}онарных систем обслуживания с катастрофами$\dotfill$&3&9\\
\textbf{Зыкова~З.\,П.} см.~Архипов~О.\,П.&&\\
\hangindent=23pt\noindent\textbf{Илюшин~Г.\,Я., Соколов~И.\,А.} Организация управляемого доступа пользователей 
к\linebreak
\vspace*{-12pt}\\
\hspace*{23pt}разнородным ведомственным информационным ресурсам$\dotfill$&1&24\\
\hangindent=23pt\noindent\textbf{Кавагучи~Ю., Ульянов~В.\,В., Фуджикоши~Я.} Приближения для статистик, 
описывающих\linebreak
\vspace*{-12pt}\\
\hspace*{23pt}геометрические свойства данных большой размерности, с оценками 
ошибок$\dotfill$&1&12\\
\hangindent=23pt\noindent\textbf{Каратеев~С.\,Л., Бекетова~И.\,В., Ососков~М.\,В., Князь~В.\,А., 
Визильтер~Ю.\,В., Бондаренко~А.\,В., Желтов~С.\,Ю.} Автоматизированный контроль 
качества цифровых\linebreak
\vspace*{-12pt}\\
\hspace*{23pt}изображений для персональных документов$\dotfill$&1&65\\
\end{tabular}
}

\pagebreak

\def\leftkol{АВТОРСКИЙ УКАЗАТЕЛЬ ЗА 2010 г.} % ENGLISH ABSTRACTS}

\def\rightkol{АВТОРСКИЙ УКАЗАТЕЛЬ ЗА 2010 г.} %ENGLISH ABSTRACTS}

{\tabcolsep=3pt
\begin{tabular}{p{388pt}rr}
&\textbf{Выпуск} & \textbf{Стр.}\\[3pt]
\hangindent=23pt\noindent\textbf{Козеренко~Е.\,Б.} Лингвистические фильтры в статистических моделях машинного\linebreak
\vspace*{-12pt}\\
\hspace*{23pt}перевода$\dotfill$&2&83\\
\hangindent=23pt\noindent\textbf{Козеренко~Е.\,Б., Кузнецов~И.\,П.} Когнитивно-лингвистические представления в 
систе-\linebreak
\vspace*{-12pt}\\
\hspace*{23pt}мах обработки текстов$\dotfill$&3&69\\
\textbf{Князь~В.\,А.} см.~Каратеев~С.\,Л.&&\\
\hangindent=23pt\noindent\textbf{Колесников~А.\,В., Солдатов~С.\,А.} Алгоритм координации для гибридной 
интеллектуальной системы решения сложной задачи оперативно-производственного\linebreak
\vspace*{-12pt}\\
\hspace*{23pt}планирования$\dotfill$&4&61\\
\hangindent=23pt\noindent\textbf{Коновалов~М.\,Г.} О планировании потоков в системах вычислительных 
ресурсов$\dotfill$&2&3\\
\textbf{Конушин~А.\,С.} см.~Конушин~В.\,С.&&\\
\hangindent=23pt\noindent\textbf{Конушин~В.\,С., Кривовязь~Г.\,Р., Конушин~А.\,С.} Алгоритм распознавания людей 
в видео-\linebreak
\vspace*{-12pt}\\
\hspace*{23pt}последовательности по одежде$\dotfill$&1&74\\
\textbf{Корепанов~Э.\, Р.} см.~Синицын~И.\,Н.&&\\
\textbf{Королев~В.\,Ю.} см.~Соколов~И.\,А.&&\\
\textbf{Королев~Р.\,А.} см.~Бенинг~В.\,Е.&&\\
\textbf{Коротышева~А.\,В.} см.~Зейфман~А.\,И.&&\\
\hangindent=23pt\noindent\textbf{Кривенко~М.\,П.} Непараметрическое оценивание элементов байесовского 
клас\-си-\linebreak
\vspace*{-12pt}\\
\hspace*{23pt}фикатора$\dotfill$&2&13\\
\textbf{Кривовязь~Г.\,Р.} см.~Конушин~В.\,С.&&\\
\textbf{Крылов~А.\,С.} см.~Павельева~Е.\,А.&&\\
\hangindent=23pt\noindent\textbf{Крылов~В.\,А.} Моделирование и классификация многоканальных дистанционных\linebreak
\vspace*{-12pt}\\
\hspace*{23pt}изображений с использованием копул$\dotfill$&4&34\\
\hangindent=23pt\noindent\textbf{Крючин~О.\,В.} Разработка параллельных эвристических алгоритмов подбора 
весовых\linebreak
\vspace*{-12pt}\\
\hspace*{23pt}коэффициентов искусственной нейтронной сети$\dotfill$&2&53\\
\hangindent=23pt\noindent\textbf{Кудрявцев~А.\,А., Шоргин~С.\,Я.} Байесовские модели массового обслуживания и 
надеж-\linebreak
\vspace*{-12pt}\\
\hspace*{23pt}ности: характеристики среднего числа заявок в системе $M\vert M \vert 1\vert 
\infty$$\dotfill$&3&16\\
\hangindent=23pt\noindent\textbf{Кузнецов~А.\,А.} Связь между временными и структурно-топологическими 
характери-\linebreak
\vspace*{-12pt}\\
\hspace*{23pt}стиками диаграмм ритма сердца здоровых людей$\dotfill$&4&39\\
\textbf{Кузнецов~И.\,П.} см.~Козеренко~Е.\,Б.&&\\
\textbf{Ле~Пезан~Д.} см.~Бунтман~Н.\,В.&&\\
\hangindent=23pt\noindent\textbf{Лукьяненко~А.\,С., Морозов~Е.\,В., Гуртов~А.\,В.} Анализ сетевого протокола с общей 
функ-\linebreak
\vspace*{-12pt}\\
\hspace*{23pt}цией расширения окна передачи сообщения при конфликтах$\dotfill$&2&46\\
\hangindent=23pt\noindent\textbf{Лямин~О.\,О.} О предельном поведении мощностей критериев в случае обобщенного\linebreak
\vspace*{-12pt}\\
\hspace*{23pt}распределения Лапласа$\dotfill$&3&47\\
\hangindent=23pt\noindent\textbf{Маркин~А.\,В., Шестаков~О.\,В.} Асимптотики оценки риска при пороговой 
обработке\linebreak
\vspace*{-12pt}\\
\hspace*{23pt}вейвлет-вейглет коэффициентов в задаче томографии$\dotfill$&2&36\\
\hangindent=23pt\noindent\textbf{Матвеева~С.\,С., Захарова~Т.\,В.} Сети массового обслуживания с наименьшей 
длиной\linebreak
\vspace*{-12pt}\\
\hspace*{23pt}очереди$\dotfill$&3&22\\
\hangindent=23pt\noindent\textbf{Матюшенко~С.\,И.} Стационарные характеристики двухканальной системы 
обслужива-\linebreak
\vspace*{-12pt}\\
\hspace*{23pt}ния с переупорядочиванием заявок и распределениями фазового типа$\dotfill$&4&68\\
\textbf{Минель~Ж.-Л.} см.~Бунтман~Н.\,В.&&\\
\textbf{Морозов~Е.\,В.} см.~Бородина~А.\,В.&&\\
\textbf{Морозов~Е.\,В.} см.~Лукьяненко~А.\,С.&&\\
\textbf{Ососков~М.\,В.} см.~Каратеев~С.\,Л.&&\\
\hangindent=23pt\noindent\textbf{Павельева~Е.\,А., Крылов~А.\,С.} Поиск и анализ ключевых точек радужной 
оболочки\linebreak
\vspace*{-12pt}\\
\hspace*{23pt}глаза методом преобразования Эрмита$\dotfill$&1&79\\
\textbf{Печинкин~А.\,В.} см.~Френкель~С.\,Л.,&&\\
\hangindent=23pt\noindent\textbf{Протасов~В.\,И.} Составление субъективного портрета с использованием 
эволюционно-\linebreak
\vspace*{-12pt}\\
\hspace*{23pt}го морфинга и квалиметрия метода$\dotfill$&1&83\\
\hangindent=23pt\noindent\textbf{Рудаков~К.\,В., Торшин~И.\,Ю.} Вопросы разрешимости задачи распознавания 
вторичной\linebreak
\vspace*{-12pt}\\
\hspace*{23pt}структуры белка$\dotfill$&2&25\\
\textbf{Сатин~Я.\,А.} см.~Зейфман~А.\,И.&&\\
\hangindent=23pt\noindent\textbf{Сейфуль-Мулюков~Р.\,Б.} Нефть как носитель информации о своем 
происхождении,\linebreak
\vspace*{-12pt}\\
\hspace*{23pt}структуре и эволюции$\dotfill$&1&41\\
\end{tabular}
}

{\tabcolsep=3pt
\begin{tabular}{p{388pt}rr}
&\textbf{Выпуск} & \textbf{Стр.}\\[6pt]
\textbf{Семендяев~Н.\,Н.} см.~Синицын~И.\,Н.&&\\
\textbf{Серебряков~В.\,А.} см.~Захаров~А.\,А.&&\\
\textbf{Синицын~В.\,И.} см.~Синицын~И.\,Н.&&\\
\hangindent=23pt\noindent\textbf{Синицын~И.\,Н., Синицын~В.\,И., Корепанов~Э.\, Р., Белоусов~В.\,В., 
Семендяев~Н.\,Н.} Оперативное построение информационных моделей движения полюса 
Земли\linebreak
\vspace*{-12pt}\\
\hspace*{23pt}методами линейных и линеаризованных фильтров$\dotfill$&1&2\\
\textbf{Сипина~А.\,В.} см.~Бенинг~В.\,Е.&&\\
\hangindent=23pt\noindent\textbf{Соколов~И.\,А.} О работах заслуженного деятеля науки Российской Федерации 
И.\,Н.~Синицына в области информационных технологий и автоматизации (к 70-летию\linebreak
\vspace*{-12pt}\\
\hspace*{23pt}со дня рождения)$\dotfill$&3&84\\
\textbf{Соколов~И.\,А.} см.~Илюшин~Г.\,Я.&&\\
\hangindent=23pt\noindent\textbf{Соколов~И.\,А., Королев~В.\,Ю.} Предисловие$\dotfill$&2&2\\
\textbf{Солдатов~С.\,А.} см.~Колесников~А.\,В.&&\\
\hangindent=23pt\noindent\textbf{Степанов~С.\,Ю.} Использование координатного метода фрагментации 
коммутаторной\linebreak
\vspace*{-12pt}\\
\hspace*{23pt}нейронной сети для сокращения трафика$\dotfill$&2&57\\
\textbf{Тимонина~Е.\,Е.} см.~Грушо~А.\,А.&&\\
\textbf{Торшин~И.\,Ю.} см.~Рудаков~К.\,В.&&\\
\textbf{Ульянов~В.\,В.} см.~Кавагучи~Ю.&&\\
\textbf{Фазекаш~И.} см.~Чупрунов~А.\,Н.&&\\
\textbf{Френкель~С.\,Л.} см.~Баранов~С.\,И.&&\\
\hangindent=23pt\noindent\textbf{Френкель~С.\,Л., Печинкин~А.\,В.} Оценка времени самовосстановления в 
цифровых\linebreak
\vspace*{-12pt}\\
\hspace*{23pt}системах после сбоев, вызываемых переходными помехами$\dotfill$&3&2\\
\textbf{Фуджикоши~Я.} см.~Кавагучи~Ю.&&\\
\hangindent=23pt\noindent\textbf{Цискаридзе~А.\,К.} Математическая модель и метод восстановления позы человека 
по\linebreak
\vspace*{-12pt}\\
\hspace*{23pt}стереопаре силуэтных изображений$\dotfill$&4&27\\
\hangindent=23pt\noindent\textbf{Чупраков~К.\,Г.} К вопросу о размещении коллективных средств отображения в 
ситуа-\linebreak
\vspace*{-12pt}\\
\hspace*{23pt}ционном зале с заданными параметрами$\dotfill$&4&89\\
\textbf{Чупраков~К.\,Г.} см.~Зацаринный~А.\,А.&&\\
\hangindent=23pt\noindent\textbf{Чупрунов~А.\,Н., Фазекаш~И.} Законы повторного логарифма для числа 
безошибочных\linebreak
\vspace*{-12pt}\\
\hspace*{23pt}блоков при помехоустойчивом кодировании$\dotfill$&3&42\\
\textbf{Шевцова~И.\,Г.} см.~Григорьева~М.\,Е.&&\\
\hangindent=23pt\noindent\textbf{Шестаков~О.\,В.} Аппроксимация распределения оценки риска пороговой 
обработки вейвлет-коэффициентов нормальным распределением при использовании 
выбо-\linebreak
\vspace*{-12pt}\\
\hspace*{23pt}рочной дисперсии$\dotfill$&4&73\\
\textbf{Шестаков~О.\,В.} см.~Маркин~А.\,В.&&\\
\textbf{Шоргин~С.\,Я.} см.~Зейфман~А.\,И.&&\\
\textbf{Шоргин~С.\,Я.} см.~Кудрявцев~А.\,А.&&\\
\end{tabular}
}

%\thispagestyle{myheadings}
\def\leftfootline{\small{\textbf{\thepage}
\hfill ИНФОРМАТИКА И ЕЁ ПРИМЕНЕНИЯ\ \ \ том~4\ \ \ выпуск~4\ \ \ 2010}
}%
 \def\rightfootline{\small{ИНФОРМАТИКА И ЕЁ ПРИМЕНЕНИЯ\ \ \ том~4\ \ \ выпуск~4\ \ \ 2010
 \hfill \textbf{\thepage}}}
 \label{end\stat}
%
%Том 10 Выпуск 1-4 Год 2016

\def\stat{cont-e}
{%\hrule\par
%\vskip 7pt % 7pt
\raggedleft\Large \bf%\baselineskip=3.2ex
2\,0\,1\,6\ \ A\,U\,T\,H\,O\,R\ \ I\,N\,D\,E\,X \vskip 17pt
 \hrule
 \par
\vskip 21pt plus 6pt minus 3pt }

\label{st\stat}

\def\tit{\ }

\def\aut{\ }
\def\auf{\ }

\def\leftkol{\ } %2016 AUTHOR INDEX} % ENGLISH ABSTRACTS}

\def\rightkol{\ } %2016 AUTHOR INDEX} %ENGLISH ABSTRACTS}

\titele{\tit}{\aut}{\auf}{\leftkol}{\rightkol}

\def\leftfootline{\small{\textbf{\thepage}
\hfill INFORMATIKA I EE PRIMENENIYA~--- INFORMATICS AND APPLICATIONS\ \ \ 2016\
\ \ volume~10\ \ \ issue\ 4}
}%
 \def\rightfootline{\small{INFORMATIKA I EE PRIMENENIYA~--- INFORMATICS AND APPLICATIONS\ \ \ 2016\ \ \ volume~10\ \ \ issue\ 4
\hfill \textbf{\thepage}}}

\vspace*{-12pt}
\vspace*{-18pt}

{\tabcolsep=2.8pt
\begin{tabular}{p{382pt}cc}
&\textbf{Issue} & \textbf{Page}\\[6pt]
\Avtors{Agalarov~M.\,Ya.} see~Agalarov~Ya.\,M.&&\\
\Avtors{Agalarov~Ya.\,M., Agalarov~M.\,Ya., and
Shorgin~V.\,S.} About the optimal threshold of queue\linebreak
\\[-12pt]
\hspace*{23pt}length in a~particular problem of profit maximization
in the $M/G/1$ queuing system&2&70--79\\
\Avtors{Alexeyevsky~D.\,A.} BioNLP ontology extraction from 
a~restricted language corpus with\linebreak
\\[-12pt]
\hspace*{23pt}context-free grammars&1&119--128\\
\Avtors{Andreev~S.\,D.} see~Gaidamaka~Yu.\,V.&&\\
\Avtors{Andreev~S.\,D.} see~Ometov~A.\,Ya.&&\\
\Avtors{Arkhipov~O.\,P., Arkhipov~P.\,O., and Sidorkin~I.\,I.} The
option to create a~local coordinate\linebreak
\\[-12pt]
\hspace*{23pt}system for synchronization of selected images&3&91--97\\
\Avtors{Arkhipov~P.\,O.} see~Arkhipov~O.\,P.&&\\
\Avtors{Belousov~V.\,V.} see~Shnurkov~P.\,V.&&\\
\Avtors{Belousov~V.\,V.} see~Shnurkov~P.\,V.&&\\
\Avtors{Bening~V.\,E.} Calculation of~the~asymptotic deficiency
of~some statistical procedures based\linebreak
\\[-12pt]
\hspace*{23pt}on~samples with~random sizes&4&34--45\\
\Avtors{Borisov~A.\,V., Bosov~A.\,V., and Miller~G.\,B.} Modeling and
monitoring of VoIP connection&2&\hphantom{1}2--13\\
\Avtors{Bosov~A.\,V.} see~Borisov~A.\,V.&&\\
\Avtors{Briukhov~D.\,O.} see~Stupnikov~S.\,A.&&\\
\Avtors{Callaos~N.\,K.\ and Seyful-Mulyukov~R.\,B.} Complexity and
its information content&1&129--139\\
\Avtors{Chertok~A.\,V., Kadaner~A.\,I., Khazeeva~G.\,T., and
Sokolov~I.\,A.} Regime switching detection\linebreak
\\[-12pt]
\hspace*{23pt}for~the~Levy driven
Ornstein--Uhlenbeck process using CUSUM methods&4&46--56\\
\Avtors{Chichagov~V.\,V.} Asymptotic expansions of mean absolute
error of uniformly minimum variance unbiased and maximum likelihood
estimators on the one-parameter exponential\linebreak
\\[-12pt]
\hspace*{23pt}family model of lattice distributions&3&66--76\\
\Avtors{Danishevsky~V.\,I.} see~Kolesnikov A.\,V.&&\\
\Avtors{Fazliev~A.\,Z.} see~Kalinichenko~L.\,A.&&\\
\Avtors{Fedoseev~A.\,A.} What is behind the concept of ``knowledge in
small packages''&3&105--110\\
\Avtors{Gaidamaka~Yu.\,V., Andreev~S.\,D., Sopin~E.\,S.,
Samouylov~K.\,E., and Shorgin~S.\,Ya.} Interference analysis
of~the~device-to-device communications model with~regard to~a~signal\linebreak
\\[-12pt]
\hspace*{23pt}propagation environment&4&\hphantom{1}2--10\\
\Avtors{Gasilov~A.\,V.} see~Yakovlev~O.\,A.&&\\
\Avtors{Goncharov~A.\,V.\ and Strijov~V.\,V.} Metric time series
classification using weighted dynamic\linebreak
\\[-12pt]
\hspace*{23pt}warping relative to centroids of classes&2&36--47\\
\Avtors{Gordov~E.\,P.} see~Kalinichenko~L.\,A.&&\\
\Avtors{Gorshenin~A.\,K.} Concept of online service for stochastic
modeling of real processes&1&72--81\\
\Avtors{Gorshenin~A.\,K.} see~Shnurkov~P.\,V.&&\\
\Avtors{Gorshenin~A.\,K.} see~Shnurkov~P.\,V.&&\\
\Avtors{Grusho~A.\,A., Grusho~N.\,A., Zabezhailo~M.\,I., and
Timonina~E.\,E.} Integration of statistical and\linebreak
\\[-12pt]
\hspace*{23pt}deterministic methods for
analysis of information security&3&2--8\\
\Avtors{Grusho~A.\,A., Zabezhailo~M.\,I., and Zatsarinny~A.\,A.} On
the advanced procedure to reduce\linebreak
\\[-12pt]
\hspace*{23pt}calculation of Galois closures&4&\hphantom{1}96--104\\
\Avtors{Grusho~N.\,A.} see~Grusho~A.\,A.&&\\
\Avtors{Havanskov~V.\,A.} see~Minin~V.\,A.&&\\
\Avtors{Inkova~O.\,Yu.} see~Zatsman~I.\,M.&&\\
\Avtors{Isachenko~R.\,V.\ and Strijov~V.\,V.} Metric learning in
multiclass time series classification\linebreak
\\[-12pt]
\hspace*{23pt}problem&2&48--57\\
\end{tabular}
}
\pagebreak

\def\leftfootline{\small{\textbf{\thepage}
\hfill INFORMATIKA I EE PRIMENENIYA~--- INFORMATICS AND APPLICATIONS\ \ \ 2016\
\ \ volume~10\ \ \ issue\ 4}
}%
 \def\rightfootline{\small{INFORMATIKA I EE PRIMENENIYA~---
INFORMATICS AND APPLICATIONS\ \ \ 2016\ \ \ volume~10\ \ \ issue\ 4
\hfill \textbf{\thepage}}}

\def\leftkol{2016 AUTHOR INDEX} % ENGLISH ABSTRACTS}

\def\rightkol{2016 AUTHOR INDEX} %ENGLISH ABSTRACTS}


{\tabcolsep=2.83pt
\begin{tabular}{p{382pt}cc}
&\textbf{Issue} & \textbf{Page}\\[6pt]
\Avtors{Kadaner~A.\,I.} see~Chertok~A.\,V.&&\\[.255pt]
\Avtors{Kalinichenko~L.\,A., Volnova~A.\,A., Gordov~E.\,P.,
Kiselyova~N.\,N., Kovaleva~D.\,A., Malkov~O.\,Yu., Okladnikov~I.\,G.,
Podkolodnyy~N.\,L., Pozanenko~A.\,S., Ponomareva~N.\,V.,
Stupnikov~S.\,A.,} \textbf{and Fazliev~A.\,Z.} Data access challenges for data
intensive\linebreak
\\[-12pt]
\hspace*{23pt}research in Russia&1& 2--22\\[.255pt]
\Avtors{Karasikov~M.\,E.\ and Strijov~V.\,V.} Feature-based
time-series classification&4&121--131\\[.255pt]
\Avtors{Khazeeva~G.\,T.} see~Chertok~A.\,V.&&\\[.255pt]
\Avtors{Khokhlov~Yu.\,S.} Multivariate fractional Levy motion and its
applications&2&\hphantom{1}98--106\\[.255pt]
\Avtors{Kirikov~I.\,A., Kolesnikov~A.\,V., Listopad~S.\,V., and
Rumovskaya~S.\,B.} Fine-grained hybrid\linebreak
\\[-12pt]
\hspace*{23pt}intelligent systems. Part 2:
Bidirectional hybridization&1&\hphantom{1}96--105\\[.255pt]
\Avtors{Kirikov~I.\,A., Kolesnikov~A.\,V., Listopad~S.\,V., and
Rumovskaya~S.\,B.} ``Virtual council''~---\linebreak
\\[-12pt]
\hspace*{23pt}source environment
supporting complex diagnostic decision making&3&81--90\\[.255pt]
\Avtors{Kiselyova~N.\,N.} see~Kalinichenko~L.\,A.&&\\[.255pt]
\Avtors{Kolesnikov A.\,V., Listopad~S.\,V., Rumovskaya~S.\,B., and
Danishevsky~V.\,I.} Informal axiomatic\linebreak
\\[-12pt]
\hspace*{23pt}theory of~the~role visual models&4&114--120\\[.255pt]
\Avtors{Kolesnikov~A.\,V.} see~Kirikov~I.\,A.&&\\[.255pt]
\Avtors{Kolesnikov~A.\,V.} see~Kirikov~I.\,A.&&\\[.255pt]
\Avtors{Kolin~K.\,K.} Humanitarian aspects of information
security&3&111--121\\[.255pt]
\Avtors{Konovalov~M.\,G.\ and Razumchik~R.\,V.} Dispatching
to~two parallel nonobservable queues using\linebreak
\\[-12pt]
\hspace*{23pt}only static
information&4&57--67\\[.255pt]
\Avtors{Korchagin~A.\,Yu.} see~Korolev~V.\,Yu.&&\\[.255pt]
\Avtors{Korchagin~A.\,Yu.} see~Korolev~V.\,Yu.&&\\[.255pt]
\Avtors{Korepanov~E.\,R.} see~Sinitsyn~I.\,N.&&\\[.255pt]
\Avtors{Korepanov~E.\,R.} see~Sinitsyn~I.\,N.&&\\[.255pt]
\Avtors{Korolev~V.\,Yu., Korchagin~A.\,Yu., and Zeifman~A.\,I.} The
Poisson theorem for Bernoulli trials\linebreak
\\[-12pt]
\hspace*{23pt}with~a~random probability
of~success and~a~discrete analog of~the~Weibull distribution&4&11--20\\[.255pt]
\Avtors{Korolev~V.\,Yu., Zeifman~A.\,I., and Korchagin~A.\,Yu.}
Asymmetric Linnik distributions as~limit\linebreak
\\[-12pt]
\hspace*{23pt}laws for~random sums
of~independent random variables with~finite variances&4&21--33\\[.255pt]
\Avtors{Koucheryavy~E.\,A.} see~Ometov~A.\,Ya.&&\\[.255pt]
\Avtors{Kovaleva~D.\,A.} see~Kalinichenko~L.\,A.&&\\[.255pt]
\Avtors{Kovalyov~S.\,P.} Metaprogramming to increase
manufacturability of large-scale software-\linebreak
\\[-12pt]
\hspace*{23pt}intensive systems&1&56--66\\[.255pt]
\Avtors{Krivenko~M.\,P.} Significance tests of feature selection for
classification&3&32--40\\[.255pt]
\Avtors{Kruzhkov~M.\,G.} see~Zalizniak~Anna~A.&&\\[.255pt]
\Avtors{Kruzhkov~M.\,G.} see~Zatsman~I.\,M.&&\\[.255pt]
\Avtors{Kudryavtsev~A.\,A.} Bayesian queueing and reliability models:
\textit{A~priori} distributions with\linebreak
\\[-12pt]
\hspace*{23pt}compact support&1&67--71\\[.255pt]
\Avtors{Kudryavtsev~A.\,A.} Characteristics dependent on the balance
coefficient in Bayesian models\linebreak
\\[-12pt]
\hspace*{23pt}with compact support of \textit{a priori}
distributions&3&77--80\\[.255pt]
\Avtors{Kudryavtsev~A.\,A.\ and Palionnaia~S.\,I.} Bayesian recurrent
model of reliability growth:\linebreak
\\[-12pt]
\hspace*{23pt}Parabolic distribution of parameters&2&80--83\\[.255pt]
\Avtors{Kudryavtsev~A.\,A.\ and Titova~A.\,I.} Bayesian queuing
and~reliability models: Degenerate-\linebreak
\\[-12pt]
\hspace*{23pt}Weibull case&4&68--71\\[.255pt]
\Avtors{Leontyev~N.\,D.\ and Ushakov~V.\,G.} Analysis of a queueing
system with autoregressive arrivals\linebreak
\\[-12pt]
\hspace*{23pt}and nonpreemptive priority&3&15--22\\[.255pt]
\Avtors{Listopad~S.\,V.} see~Kirikov~I.\,A.&&\\[.255pt]
\Avtors{Listopad~S.\,V.} see~Kirikov~I.\,A.&&\\[.255pt]
\Avtors{Listopad~S.\,V.} see~Kolesnikov A.\,V.&&\\[.255pt]
\Avtors{Malkov~O.\,Yu.} see~Kalinichenko~L.\,A.&&\\[.255pt]
\Avtors{Markov~A.\,S., Monakhov~M.\,M., and
Ulyanov~V.\,V.} Generalized Cornish--Fisher expansions\linebreak
\\[-12pt]
\hspace*{23pt}for distributions of statistics based on samples
of random size&2&84--91\\[.255pt]
\Avtors{Melnikov~A.\,K.\ and Ronzhin~A.\,F.} Generalized statistical
method of~text analysis based\linebreak
\\[-12pt]
\hspace*{23pt}on~calculation of~probability distributions
of~statistical values&4&89--95\\
\end{tabular}
}
\pagebreak

\def\leftfootline{\small{\textbf{\thepage}
\hfill INFORMATIKA I EE PRIMENENIYA~--- INFORMATICS AND APPLICATIONS\ \ \ 2016\
\ \ volume~10\ \ \ issue\ 4}
}%
 \def\rightfootline{\small{INFORMATIKA I EE PRIMENENIYA~---
INFORMATICS AND APPLICATIONS\ \ \ 2016\ \ \ volume~10\ \ \ issue\ 4
\hfill \textbf{\thepage}}}

\def\leftkol{2016 AUTHOR INDEX} % ENGLISH ABSTRACTS}

\def\rightkol{2016 AUTHOR INDEX} %ENGLISH ABSTRACTS}


{\tabcolsep=3pt
\begin{tabular}{p{381pt}cc}
&\textbf{Issue} & \textbf{Page}\\[6pt]
\Avtors{Meykhanadzhyan~L.\,A.} Stationary characteristics of the finite
capacity queueing system with\linebreak
\\[-12pt]
\hspace*{23pt}inverse service order and generalized
probabilistic priority&2&123--131\\[.23pt]
\Avtors{Miller~G.\,B.} see~Borisov~A.\,V.&&\\[.23pt]
\Avtors{Minin~V.\,A., Zatsman~I.\,M., Havanskov~V.\,A., and
Shubnikov~S.\,K.} Intensity of citation of scientific publications in
inventions on information and computer technologies patented\linebreak
\\[-12pt]
\hspace*{23pt}in Russia by domestic and foreign applicants&2&107--122\\[.23pt]
\Avtors{Monakhov~M.\,M.} see~Markov~A.\,S.&&\\[.23pt]
\Avtors{Naumov~V.\,A.\ and Samouylov~K.\,E.} On relationship
between queuing systems with resources\linebreak
\\[-12pt]
\hspace*{23pt}and Erlang networks&3&\hphantom{1}9--14\\[.23pt]
\Avtors{Okladnikov~I.\,G.} see~Kalinichenko~L.\,A.&&\\[.23pt]
\Avtors{Ometov~A.\,Ya., Andreev~S.\,D., Turlikov~A.\,M., and
Koucheryavy~E.\,A.} Performance analysis of\linebreak
\\[-12pt]
\hspace*{23pt}a wireless data
aggregation system with contention for contemporary sensor
networks&3&23--31\\[.23pt]
\Avtors{Palionnaia~S.\,I.} see~Kudryavtsev~A.\,A.&&\\[.23pt]
\Avtors{Podkolodnyy~N.\,L.} see~Kalinichenko~L.\,A.&&\\[.23pt]
\Avtors{Ponomareva~N.\,V.} see~Kalinichenko~L.\,A.&&\\[.23pt]
\Avtors{Popkova~N.\,A.} see~Zatsman~I.\,M.&&\\[.23pt]
\Avtors{Pozanenko~A.\,S.} see~Kalinichenko~L.\,A.&&\\[.23pt]
\Avtors{Razumchik~R.\,V.} see~Konovalov~M.\,G.&&\\[.23pt]
\Avtors{Ronzhin~A.\,F.} see~Melnikov~A.\,K.&&\\[.23pt]
\Avtors{Rumovskaya~S.\,B.} see~Kirikov~I.\,A.&&\\[.23pt]
\Avtors{Rumovskaya~S.\,B.} see~Kirikov~I.\,A.&&\\[.23pt]
\Avtors{Rumovskaya~S.\,B.} see~Kolesnikov A.\,V.&&\\[.23pt]
\Avtors{Samouylov~K.\,E.} see~Gaidamaka~Yu.\,V.&&\\[.23pt]
\Avtors{Samouylov~K.\,E.} see~Naumov~V.\,A.&&\\[.23pt]
\Avtors{Serebryanskii~S.\,M.} see~Tyrsin~A.\,N.&&\\[.23pt]
\Avtors{Seyful-Mulyukov~R.\,B.} see~Callaos~N.\,K.&&\\[.23pt]
\Avtors{Shestakov~O.\,V.} Statistical properties of the denoising method
based on the stabilized hard\linebreak
\\[-12pt]
\hspace*{23pt}thresholding&2&65--69\\[.23pt]
\Avtors{Shestakov~O.\,V.} The strong law of large numbers for the risk
estimate in the problem of\linebreak
\\[-12pt]
\hspace*{23pt}tomographic image reconstruction from
projections with a correlated noise&3&41--45\\[.23pt]
\Avtors{Shestakov~O.\,V.} see~Zakharova~T.\,V.&&\\[.23pt]
\Avtors{Shnurkov~P.\,V., Gorshenin~A.\,K., and Belousov~V.\,V.}
Analytical solution of~the~optimal control\linebreak
\\[-12pt]
\hspace*{23pt}task of~a~semi-Markov
process with~finite set of~states&4&72--88\\[.23pt]
\Avtors{Shnurkov~P.\,V., Zasypko~V.\,V., Belousov~V.\,V., and
Gorshenin~A.\,K.} Development of the algorithm of numerical solution
of the optimal investment control problem\linebreak
\\[-12pt]
\hspace*{23pt}in the closed dynamical model of three-sector economy&1&82--95\\[.23pt]
\Avtors{Shorgin~S.\,Ya.} see~Gaidamaka~Yu.\,V.&&\\[.23pt]
\Avtors{Shorgin~V.\,S.} see~Agalarov~Ya.\,M.&&\\[.23pt]
\Avtors{Shubnikov~S.\,K.} see~Minin~V.\,A.&&\\[.23pt]
\Avtors{Sidorkin~I.\,I.} see~Arkhipov~O.\,P.&&\\[.23pt]
\Avtors{Sinitsyn~I.\,N.} Analytical modeling of processes in stochastic
systems with complex fractional\linebreak
\\[-12pt]
\hspace*{23pt}order Bessel nonlinearities&3&55--65\\[.23pt]
\Avtors{Sinitsyn~I.\,N.} Orthogonal supoptimal filters for nonlinear
stochastic systems on manifolds&1&34--44\\[.23pt]
\Avtors{Sinitsyn~I.\,N.\ and Korepanov~E.\,R.} Normal Pugachev
conditionally-optimal filters and extra-\linebreak
\\[-12pt]
\hspace*{23pt}polators for state linear stochastic systems&2&14--23\\[.23pt]
\Avtors{Sinitsyn~I.\,N.\ and Sinitsyn~V.\,I.} Analytical modeling of
distributions in stochastic systems on\linebreak
\\[-12pt]
\hspace*{23pt}manifolds based on ellipsoidal approximation&1&45--55\\[.23pt]
\Avtors{Sinitsyn~I.\,N., Sinitsyn~V.\,I., and
Korepanov~E.\,R.} Ellipsoidal suboptimal filters for nonlinear\linebreak
\\[-12pt]
\hspace*{23pt}stochastic systems on manifolds&2&24--35\\[.23pt]
\Avtors{Sinitsyn~V.\,I.} see~Sinitsyn~I.\,N.&&\\[.23pt]
\Avtors{Sinitsyn~V.\,I.} see~Sinitsyn~I.\,N.&&\\[.23pt]
\Avtors{Skvortsov~N.\,A.} see~Stupnikov~S.\,A.&&\\[.23pt]
\Avtors{Sokolov~I.\,A.} see~Chertok~A.\,V.&&\\
\end{tabular}
}
\pagebreak

\def\leftfootline{\small{\textbf{\thepage}
\hfill INFORMATIKA I EE PRIMENENIYA~--- INFORMATICS AND APPLICATIONS\ \ \ 2016\
\ \ volume~10\ \ \ issue\ 4}
}%
 \def\rightfootline{\small{INFORMATIKA I EE PRIMENENIYA~---
INFORMATICS AND APPLICATIONS\ \ \ 2016\ \ \ volume~10\ \ \ issue\ 4
\hfill \textbf{\thepage}}}

\def\leftkol{2016 AUTHOR INDEX} % ENGLISH ABSTRACTS}

\def\rightkol{2016 AUTHOR INDEX} %ENGLISH ABSTRACTS}


{\tabcolsep=3pt
\begin{tabular}{p{382pt}cc}
&\textbf{Issue} & \textbf{Page}\\[6pt]
\Avtors{Sopin~E.\,S.} see~Gaidamaka~Yu.\,V.&&\\
\Avtors{Strijov~V.\,V.} see~Goncharov~A.\,V.&&\\
\Avtors{Strijov~V.\,V.} see~Isachenko~R.\,V.&&\\
\Avtors{Strijov~V.\,V.} see~Karasikov~M.\,E.&&\\
\Avtors{Stupnikov~S.\,A., Briukhov~D.\,O., and Skvortsov~N.\,A.}
Co-lending systemic risk analysis over\linebreak
\\[-12pt]
\hspace*{23pt}heterogeneous data collections&1&23--33\\
\Avtors{Stupnikov~S.\,A.} see~Kalinichenko~L.\,A.&&\\
\Avtors{Suchkov~A.\,P.} see~Zatsarinny~A.\,A.&&\\
\Avtors{Timonina~E.\,E.} see~Grusho~A.\,A.&&\\
\Avtors{Titova~A.\,I.} see~Kudryavtsev~A.\,A.&&\\
\Avtors{Turlikov~A.\,M.} see~Ometov~A.\,Ya.&&\\
\Avtors{Tyrsin~A.\,N.\ and Serebryanskii~S.\,M.} Recognition of
dependences on the basis of inverse\linebreak
\\[-12pt]
\hspace*{23pt}mapping&2&58--64\\
\Avtors{Ulyanov~V.\,V.} see~Markov~A.\,S.&&\\
\Avtors{Ushakov~V.\,G.} Queueing system with working vacations and
hyperexponential input stream&2&92--97\\
\Avtors{Ushakov~V.\,G.} see~Leontyev~N.\,D.&&\\
\Avtors{Volnova~A.\,A.} see~Kalinichenko~L.\,A.&&\\
\Avtors{Yakovlev~O.\,A.\ and Gasilov~A.\,V.} Speeded-up stereo
matching using geodesic support weights&3&\hphantom{1}98--104\\
\Avtors{Zabezhailo~M.\,I.} see~Grusho~A.\,A.&&\\
\Avtors{Zabezhailo~M.\,I.} see~Grusho~A.\,A.&&\\
\Avtors{Zakharova~T.\,V.\ and Shestakov~O.\,V.} Precision analysis of
wavelet processing of aerodynamic\linebreak
\\[-12pt]
\hspace*{23pt}flow patterns&3&46--54\\
\Avtors{Zalizniak~Anna~A.\ and Kruzhkov~M.\,G.} Database
of~Russian impersonal verbal constructions&4&132--141\\
\Avtors{Zasypko~V.\,V.} see~Shnurkov~P.\,V.&&\\
\Avtors{Zatsarinny~A.\,A.\ and Suchkov~A.\,P.} Systems engineering
approaches to~the~establishment of\linebreak
\\[-12pt]
\hspace*{23pt}a~system for~decision support based
on~situational analysis&4&105--113\\
\Avtors{Zatsarinny~A.\,A.} see~Grusho~A.\,A.&&\\
\Avtors{Zatsman~I.\,M., Inkova~O.\,Yu., Kruzhkov~M.\,G., and
Popkova~N.\,A.} Representation of cross-\linebreak
\\[-12pt]
\hspace*{23pt}lingual knowledge about
connectors in supracorpora databases&1&106--118\\
\Avtors{Zatsman~I.\,M.} see~Minin~V.\,A.&&\\
\Avtors{Zeifman~A.\,I.} see~Korolev~V.\,Yu.&&\\
\Avtors{Zeifman~A.\,I.} see~Korolev~V.\,Yu.&&\\
\end{tabular}
}

%\thispagestyle{myheadings}
\def\leftfootline{\small{\textbf{\thepage}
\hfill INFORMATIKA I EE PRIMENENIYA~--- INFORMATICS AND APPLICATIONS\ \ \ 2016\
\ \ volume~10\ \ \ issue\ 4}
}%
 \def\rightfootline{\small{INFORMATIKA I EE PRIMENENIYA~---
INFORMATICS AND APPLICATIONS\ \ \ 2016\ \ \ volume~10\ \ \ issue\ 4
\hfill \textbf{\thepage}}}

 \label{end\stat}

\newpage

%\def\stat{rekl}
%\label{preobr}

%\def\tit{АКАДЕМИК ПУГАЧЁВ  ВЛАДИМИР СЕМЁНОВИЧ\\
%25.03.1911--25.03.1998}


%   \vspace*{-48pt}
%   \begin{center}\LARGE
%Академик Пугачёв  Владимир Семёнович\\ (25.03.1911--25.03.1998)
%   \end{center}
   
   %\vspace*{2.5mm}
   
   \begin{center}

{\prgsh\LARGE
ОБЪЯВЛЕНИЯ О КОНФЕРЕНЦИЯХ}

\end{center}
%\hrule

\vspace*{6pt}

   
   \vspace*{10mm}
   
   \thispagestyle{empty}

\noindent
\begin{tabular}{cc}
%\begin{center}
\multicolumn{1}{c}{\raisebox{-40pt}[0pt][0pt]{\mbox{%
\epsfxsize=33mm
\epsfbox{vspu.eps}
}}}
%\end{center}
&
\tabcolsep=0pt\begin{tabular}{c}
{\prg{\Large\textbf{XII Всероссийское совещание}}}\\[6pt]
{\prg{\Large\textbf{по проблемам управления}}}\\[12pt]
{\prg{\large 16--19 июня 2014~г.}}\\[6pt] 
{\prg{\large Институт проблем управления имени В.\,А.~Трапезникова РАН}}\\[6pt]
{\prg{\large Москва, Россия}}
\end{tabular}
\end{tabular}

\vspace*{60pt}

     
 { %\large    
 XII Всероссийское совещание по проблемам управления (ВСПУ XII), посвященное 75-летию 
Института проблем управления (ИПУ) имени В.\,А.~Трапезникова РАН, проводится 16--19~июня 
2014~г.\ 
в ИПУ РАН (г.~Москва, Россия). ВСПУ XII организуется ИПУ РАН при поддержке РФФИ, Отделения 
энергетики, машиностроения, механики и процессов управления Российской академии наук, 
Российского 
национального комитета по автоматическому управлению, Академии навигации и управ\-ле\-ния 
движением, 
Научного совета РАН по комплексным проблемам управления и автоматизации, Совета по 
мехатронике и робототехнике РАН. Официальный язык Совещания~--- русский.

\vspace*{24pt}
     
     \textbf{Направления работы}
     \begin{enumerate}[1.]
\item Теория систем управления
\item Управление подвижными объектами и навигация
\item Интеллектуальные системы управления
\item Управление в промышленности, транспортом и логистикой
\item Управление системами междисциплинарной природы
\item Средства измерения, вычислений и контроля в управлении
\item Системный анализ и принятие решений в задачах управления
\item Информационные технологии в управлении
\item Проблемы образования в области управления: современное содержание и технологии обучения
\end{enumerate}

\vspace*{24pt}

     Подробная информация о Совещании находится на сайте {\sf http://vspu2014.ipu.ru}. Срок 
окончательной подачи докладов через систему подачи докладов на сайте~--- \textbf{30~ноября} 
2013~г.
}

%\include{rekl-1}

%\end{document}

%\include{nekrolog-rb}


%\end{document}

%\include{IPPM-25}

\def\stat{cont-rus}
{%\hrule\par
%\vskip 7pt % 7pt
\vspace*{-24pt}
\raggedleft\Large \bf%\baselineskip=3.2ex
Правила подготовки рукописей  для публикации в журнале
<<Информатика~и~её~применения>> \vskip 8pt
    \hrule
    \par
\vskip 14pt plus 6pt minus 3pt }

\label{st\stat}

\def\tit{\ }

\def\aut{\ }
\def\auf{\ }

\def\leftkol{\ }
% Правила подготовки рукописей  для публикации в журнале
%<<Информатика и её применения>>

\def\rightkol{\ }
%Правила подготовки рукописей  для публикации в журнале
%<<Информатика и её применения>>}


\titele{\tit}{\aut}{\auf}{\leftkol}{\rightkol}


\vspace*{-60pt}
{ %\small

Журнал <<Информатика и её применения>>
публикует теоретические, обзорные и дискуссионные статьи,
посвященные научным исследованиям и разработкам в области
информатики и ее приложений.

Журнал издается на русском языке. По специальному решению
редколлегии отдельные статьи могут печататься на английском языке.

Тематика журнала охватывает следующие направления:
\begin{itemize}
\item теоретические основы информатики;\\[-15pt]
      \item
математические методы исследования сложных систем и процессов;\\[-15pt]
           \item
информационные системы и сети;\\[-15pt]
                \item
информационные технологии;\\[-15pt]
                     \item
архитектура и программное обеспечение вычислительных комплексов и сетей.\\[-15pt]
\end{itemize}


\noindent
\begin{enumerate}[1.]
\item В журнале печатаются статьи, содержащие результаты, ранее не опубликованные и
не предназначенные к одновременной публикации в других изданиях.

%Публикация не должна нарушать закон об авторских правах.
Публикация предоставленной автором(ами) рукописи не должна нарушать 
положений глав~69, 70 раздела~VII части~IV Гражданского кодекса, 
которые определяют права на результаты интеллектуальной деятельности 
и~средства индивидуализации, в~том числе авторские права, в~РФ.

Ответственность за нарушение авторских прав, в~случае предъявления претензий к~редакции журнала,  
несут авторы статей.



Направляя рукопись в редакцию, авторы сохраняют свои права на данную
рукопись и при этом передают учредителям и редколлегии журнала неисключительные права на
издание статьи на русском языке 
(или на языке статьи, если он отличен от рус\-ско\-го) и~на перевод ее на английский
язык, а~также на
ее распространение в России и за рубежом. 
Каждый автор должен представить в~редакцию подписанный 
с~его стороны <<Лицензионный договор о~передаче неисключительных прав 
на использование произведения>>, текст которого размещен по адресу 
{\sf http://www.ipiran.ru/publications/licence.doc}. 
Этот договор может быть пред\-став\-лен в~бумажном (в~2-х экз.)\ 
или в~электронном виде (отсканированная копия заполненного и~подписанного документа).




Редколлегия вправе запросить у авторов экспертное заключение о возможности
пуб\-ли\-ка\-ции пред\-став\-лен\-ной статьи в открытой печати.\\[-13.5pt]

\item К статье прилагаются данные автора (авторов) (см.\ п.~8). При наличии нескольких
авторов указывается фамилия автора, ответственного за переписку с редакцией.\\[-13.5pt]

\item Редакция журнала осуществляет экспертизу присланных статей в соответствии с
принятой в журнале процедурой рецензирования.

Возвращение рукописи на доработку не означает ее принятия к печати.

Доработанный вариант с ответом на замечания рецензента необходимо прислать в
редакцию.\\[-13.5pt]

\item Решение редколлегии о публикации статьи или ее отклонении сообщается авторам.

Редколлегия может также направить авторам текст рецензии на их статью. Дискуссия по
поводу отклоненных статей не ведется.\\[-13.5pt]

%\pagebreak

\item Редактура статей высылается авторам для просмотра. Замечания к редактуре должны
быть присланы авторами в кратчайшие сроки.\\[-13.5pt]

\item Рукопись предоставляется в электронном виде в форматах MS WORD (.doc или
.docx) или \LaTeX\  (.tex), дополнительно~--- в формате .pdf, на дискете, лазерном диске
или электронной почтой. Предоставление бумажной рукописи необязательно.\\[-13.5pt]

\item При подготовке рукописи в MS Word рекомендуется использовать следующие
настройки.

Параметры страницы:
формат~--- А4; ориентация~--- книжная; поля (см): внутри~--- 2,5, снаружи~--- 1,5,
сверху~--- 2, снизу~--- 2, от края до нижнего колонтитула~--- 1,3.

Основной текст: стиль~--- <<Обычный>>, шрифт~--- Times New Roman, размер~---
14~пунк\-тов, абзацный отступ~--- 0,5~см, 1,5~интервала, выравнивание~--- по ширине.

\pagebreak

\def\leftkol{Правила подготовки рукописей  для публикации в журнале
<<Информатика и её применения>>}

\def\rightkol{Правила подготовки рукописей  для публикации в журнале
<<Информатика и её применения>>}



Рекомендуемый объем рукописи~--- не свыше 10~страниц указанного формата.
При превышении указанного объема редколлегия вправе потребовать от 
автора сокращения объема рукописи.


Сокращения слов, помимо стандартных, не допускаются. Допускается минимальное
количество аббревиатур.


Все страницы рукописи нумеруются.

Шаблоны оформления представлены в интернете:

\noindent
 {\sf
http://www.ipiran.ru/journal/template\_iiep\_ssi\_2024.zip}\\[-14pt]

\item Статья должна содержать следующую информацию на {\bfseries\textit{русском и
английском языках}}:\\[-16pt]

\begin{itemize}
\item название статьи;\\[-15pt]
\item Ф.И.О.\ авторов, на английском можно только имя и фамилию;\\[-15pt]
\item место работы, с указанием почтового адреса организации и электронного адреса каждого
автора;\\[-15pt]
\item сведения об авторах, в соответствии с форматом, образцы которого
представлены на страницах:



\def\leftfootline{\small{\textbf{\thepage}
\hfill ИНФОРМАТИКА И ЕЁ ПРИМЕНЕНИЯ\ \ \ том\ 18\ \ \ выпуск\ 3\ \ \ 2024}
}%
 \def\rightfootline{\small{ИНФОРМАТИКА И ЕЁ ПРИМЕНЕНИЯ\ \ \ том\ 18\ \ \ выпуск\ 3\ \ \ 2024
\hfill \textbf{\thepage}}}



{\sf http://www.ipiran.ru/journal/issues/2013\_07\_01/authors.asp} и

{\sf http://www.ipiran.ru/journal/issues/2013\_07\_01\_eng/authors.asp};
\item аннотация (не менее 100~слов на каждом из языков). Аннотация~--- это краткое
резюме работы, которое может публиковаться отдельно. Она является основным
источником информации в~ин\-фор\-ма\-ци\-он\-ных системах и базах данных. Английская
аннотация должна быть оригинальной, может не быть дословным переводом русского
текста и должна быть написана хорошим английским языком. В~аннотации не должно
быть ссылок на литературу и, по возможности, формул;\\[-15pt]
\item ключевые слова~--- желательно из принятых в мировой
на\-уч\-но-тех\-ни\-че\-ской литературе тематических тезаурусов. Предложения не
могут быть ключевыми словами;\\[-15pt]
\item источники финансирования работы (ссылки на гранты, проекты,
поддерживающие организации и~т.\,п.).
\end{itemize}



%\pagebreak

\item  Требования к спискам литературы.\\[-14pt]

Ссылки на литературу в тексте статьи нумеруются (в квадратных скобках) и
располагаются в каждом из списков литературы в порядке  первых упоминаний. Если источник имеет DOI и/или EDN,
то их необходимо указывать.

Списки литературы представляются в двух вариантах:\\[-14pt]


\noindent
\begin{enumerate}[(1)]
\item \textbf{Список литературы к русскоязычной части}. Русские и английские
работы~---  на языке и в алфавите оригинала;\\[-14.5pt]
\item  \textbf{References}. Русские работы и работы на других языках~--- в латинской
транслитерации с переводом на английский язык; английские работы и работы на других
языках~--- на языке оригинала.
\end{enumerate}

Необходимо для составления списка ``References'' пользоваться размещенной на сайте
{\sf http://www. translit.net/ru/bgn/} бесплатной программой транслитерации русского
 текста в~латиницу. %, при этом в~за\-клад\-ке <<варианты\ldots>> следует выбратьопцию BGN.

Список литературы ``References'' приводится полностью отдельным блоком, повторяя все
позиции из списка литературы к русскоязычной части, независимо от того, имеются или
нет в нем иностранные источники. Если в списке литературы к русскоязычной части есть
ссылки на иностранные публикации, набранные латиницей, они полностью повторяются в
списке ``References''.

Ниже приведены примеры ссылок на различные виды публикаций в списке ``References''.

\def\leftfootline{\small{\textbf{\thepage}
\hfill ИНФОРМАТИКА И ЕЁ ПРИМЕНЕНИЯ\ \ \ том\ 18\ \ \ выпуск\ 3\ \ \ 2024}
}%
 \def\rightfootline{\small{ИНФОРМАТИКА И ЕЁ ПРИМЕНЕНИЯ\ \ \ том\ 18\ \ \ выпуск\ 3\ \ \ 2024
\hfill \textbf{\thepage}}}

{\small

\noindent
\textbf{Описание статьи из журнала:}

\Aue{Zagurenko, A.\,G., V.\,A.~Korotovskikh, A.\,A.~Kolesnikov, A.\,V.~Timonov, and D.\,V.~Kardymon}. 2008.
Tekhniko-ekonomicheskaya optimizatsiya dizayna gidrorazryva plasta [Technical and
economic optimization of the design
of hydraulic fracturing]. \textit{Neftyanoe hozyaystvo} [\textit{Oil Industry}] 11:54--57.

\Aue{Zhang, Z., and D.~Zhu}. 2008. Experimental research on the localized
electrochemical micromachining. \textit{Russ. J.~Electrochem.}  44(8):926--930.
{\sf doi:10.1134/S1023193508080077}.

\noindent
\textbf{Описание статьи из электронного журнала:}

\Aue{Swaminathan, V., E.~Lepkoswka-White, and B.\,P.~Rao}. 1999. Browsers or buyers in cyberspace? An
investigation of electronic factors influencing electronic exchange. \textit{JCMC}
5(2). Available at: {\sf http://www.ascusc.org/jcmc/vol5/issue2/} (accessed April~28, 2011).

\def\leftkol{Правила подготовки рукописей  для публикации в журнале
<<Информатика и её применения>>}

\def\rightkol{Правила подготовки рукописей  для публикации в журнале
<<Информатика и её применения>>}


\noindent
\textbf{Описание статьи из продолжающегося издания (сборника трудов):}

\Aue{Astakhov, M.\,V., and T.\,V.~Tagantsev}. 2006. Eksperimental'noe
issledovanie prochnosti soedineniy ``stal'--kompozit'' [Experimental study of
the strength of joints ``steel--composite'']. \textit{Trudy MGTU
``Matematicheskoe modelirovanie slozhnykh tekh\-ni\-che\-skikh sistem''}
[\textit{Bauman MSTU ``Mathematical Modeling of Complex Technical
Systems'' Proceedings}]. 593:125--130.


\pagebreak



\noindent
\textbf{Описание материалов конференций:}

\Aue{Usmanov, T.\,S., A.\,A.~Gusmanov, I.\,Z.~Mullagalin, R.\,Ju.~Muhametshina, A.\,N.~Chervyakova, and
A.\,V.~Sveshnikov}. 2007. Osobennosti proektirovaniya razrabotki mestorozhdeniy
s primeneniem gidrorazryva
plasta [Features of the design of field development with the use of hydraulic fracturing].
\textit{Trudy 6-go
Mezhdu\-na\-rod\-no\-go Simpoziuma ``Novye resursosberegayushchie tekhnologii nedropol'zovaniya i povysheniya
neftegazootdachi''} [\textit{6th  Symposium (International) ``New Energy Saving Subsoil Technologies and
the Increasing of the Oil and Gas Impact'' Proceedings}]. Moscow. 267--272.



\def\leftfootline{\small{\textbf{\thepage}
\hfill ИНФОРМАТИКА И ЕЁ ПРИМЕНЕНИЯ\ \ \ том\ 18\ \ \ выпуск\ 3\ \ \ 2024}
}%
 \def\rightfootline{\small{ИНФОРМАТИКА И ЕЁ ПРИМЕНЕНИЯ\ \ \ том\ 18\ \ \ выпуск\ 3\ \ \ 2024
\hfill \textbf{\thepage}}}



\noindent
\textbf{Описание книги (монографии, сборники):}



Lindorf, L.\,S., and L.\,G.~Mamikoniants, eds. 1972.
\textit{Ekspluatatsiya turbogeneratorov s neposredstvennym
okhlazhdeniem} [\textit{Operation of turbine generators with direct cooling}].
Moscow: Energy Publs. 352~p.


\Aue{Latyshev, V.\,N.} 2009. \textit{Tribologiya rezaniya. Kn.~1: Friktsionnye protsessy
pri rezanii metallov}
[\textit{Tribology of cutting. Vol.~1: Frictional processes in metal cutting}]. Ivanovo: Ivanovskii
State Univ. 108~p.

\def\leftkol{Правила подготовки рукописей  для публикации в журнале
<<Информатика и её применения>>}

\def\rightkol{Правила подготовки рукописей  для публикации в журнале
<<Информатика и её применения>>}

\noindent
\textbf{Описание переводной книги}
(в списке литературы к русскоязычной части необходимо указать:~/ Пер.\ с англ.~---
после названия книги, а в конце ссылки указать оригинал книги в круглых скобках):
\begin{enumerate}[1.]
\item  В русскоязычной части:

\def\leftfootline{\small{\textbf{\thepage}
\hfill ИНФОРМАТИКА И ЕЁ ПРИМЕНЕНИЯ\ \ \ том\ 18\ \ \ выпуск\ 3\ \ \ 2024}
}%
 \def\rightfootline{\small{ИНФОРМАТИКА И ЕЁ ПРИМЕНЕНИЯ\ \ \ том\ 18\ \ \ выпуск\ 3\ \ \ 2024
\hfill \textbf{\thepage}}}

\Au{Тимошенко С.\,П., Янг Д.\,Х., Уивер~У.}
Колебания в инженерном деле~/ Пер.\ с англ.~--- М.: Машиностроение, 1985. 472~с.
(\Au{Timoshenko~S.\,P., Young~D.\,H., Weaver~W.}
Vibration problems in engineering.~--- 4th ed.~--- New York, NY, USA: Wiley, 1974. 521~p.)\\[-13.5pt]
\item  В англоязычной части:

\Aue{Timoshenko, S.\,P., D.\,H.~Young, and W.~Weaver}.
1974. \textit{Vibration problems in engineering}. 4th ed. New York: 
Wiley. 521~p.
\end{enumerate}

\vspace*{-3pt}


\noindent
\textbf{Описание неопубликованного документа:}


\Aue{Latypov, A.\,R., M.\,M.~Khasanov, and V.\,A.~Baikov}.
2004 (unpubl.). Geologiya i~dobycha (NGT GiD) [Geology and production (NGT GiD)]. Certificate on official registration of the computer program
No.\,2004611198. 

\noindent
\textbf{Описание интернет-ресурса:}


Pravila tsitirovaniya istochnikov [Rules for the citing of sources]. Available at: {\sf
http://www.scribd.com/doc/1034528/} (accessed February~7, 2011).

%\pagebreak

\noindent
\textbf{Описание диссертации или автореферата диссертации:}

\Aue{Semenov, V.\,I.}
2003. Matematicheskoe modelirovanie plazmy v sisteme kompaktnyy tor [Mathematical
modeling of the plasma in the compact torus].  Moscow.  D.Sc.\ Diss. 272~p.

\Aue{Kozhunova, O.\,S.} 2009. Tekhnologiya razrabotki semanticheskogo
slovarya informatsionnogo monitoringa [Technology of development of
semantic dictionary of information monitoring system].  Moscow: IPI RAN. PhD Thesis. 23~p.


\noindent
\textbf{Описание ГОСТа:}

GOST 8.586.5-2005. 2007. Metodika vypolneniya izmereniy. Izmerenie raskhoda i~kolichestva zhidkostey i~gazov
s~pomoshch'yu standartnykh suzhayushchikh ustroystv [Method of measurement.
Measurement of flow rate and volume of liquids and gases by means of orifice devices]. Moscow:
Standardinform  Publs. 10~p.

\noindent
\textbf{Описание патента:}

\Aue{Bolshakov, M.\,V., A.\,V.~Kulakov, A.\,N.~Lavrenov, and M.\,V.~Palkin}.
2006. Sposob orientirovaniya po krenu letatel'nogo
apparata s opti\-che\-skoy golovkoy
samonavedeniya [The way to orient on the roll of aircraft with optical homing head].
Patent RF No.\,2280590.
}

\item Присланные в редакцию материалы авторам не возвращаются.\\[-13.5pt]

\item При отправке файлов по электронной почте просим придерживаться следующих
правил:
\begin{itemize}
\item указывать в поле subject (тема) название журнала и фамилию автора;\\[-13.5pt]
\item указывать в тексте письма название статьи, авторов и~журнал, в~который направляется статья;\\[-13.5pt]
\item использовать attach (присоединение);\\[-13.5pt]
\item в состав электронной версии статьи должны входить: файл, содержащий текст
статьи, и файл(ы), содержащий(е) иллюстрации.\\[-13.5pt]
\end{itemize}

\item Журнал <<Информатика и её применения>> является некоммерческим изданием.
Плата за публикацию не взимается, гонорар авторам не выплачивается.
\end{enumerate}



\def\leftfootline{\small{\textbf{\thepage}
\hfill ИНФОРМАТИКА И ЕЁ ПРИМЕНЕНИЯ\ \ \ том\ 18\ \ \ выпуск\ 3\ \ \ 2024}
}%
 \def\rightfootline{\small{ИНФОРМАТИКА И ЕЁ ПРИМЕНЕНИЯ\ \ \ том\ 18\ \ \ выпуск\ 3\ \ \ 2024
\hfill \textbf{\thepage}}}


\vspace*{-1mm}

\begin{center}

\textbf{Адрес редакции журнала <<Информатика и её применения>>:} \\




Москва 119333, ул.~Вавилова, д.~44, корп.~2, ФИЦ ИУ РАН\\[-10pt]

\

Тел.: +7\,(499)\,135-86-92\ \ Факс:  +7\,(495)\,930-45-05\\[-10pt]

 \

e-mail:   {\sf iiep@frccsc.ru} (Стригина Светлана Николаевна)\\[-10pt]

\

{\sf http://www.ipiran.ru/journal/issues/}
\end{center}
}


\def\leftkol{Правила подготовки рукописей  для публикации в журнале
<<Информатика и её применения>>}

\def\rightkol{Правила подготовки рукописей  для публикации в журнале
<<Информатика и её применения>>}


\def\leftfootline{\small{\textbf{\thepage}
\hfill ИНФОРМАТИКА И ЕЁ ПРИМЕНЕНИЯ\ \ \ том\ 18\ \ \ выпуск\ 3\ \ \ 2024}
}%
 \def\rightfootline{\small{ИНФОРМАТИКА И ЕЁ ПРИМЕНЕНИЯ\ \ \ том\ 18\ \ \ выпуск\ 3\ \ \ 2024
\hfill \textbf{\thepage}}} 
\def\stat{podg-e}
{%\hrule\par
%\vskip 7pt % 7pt
\vspace*{-24pt}
\raggedleft\Large \bf%\baselineskip=3.2ex
Requirements for manuscripts submitted to Journal
``Informatics~and~Applications'' \vskip 8pt
    \hrule
    \par
\vskip 21pt plus 6pt minus 3pt }

\label{st\stat}

\def\tit{\ }

\def\aut{\ }
\def\auf{\ }

\def\leftkol{\ }

\def\rightkol{\ }
%Requirements for manuscripts submitted to Journal
%``Informatics~and~Applications''}

\titele{\tit}{\aut}{\auf}{\leftkol}{\rightkol}

\def\leftfootline{\small{\textbf{\thepage}
\hfill INFORMATIKA I EE PRIMENENIYA~--- INFORMATICS AND APPLICATIONS\ \ \ 2019\
\ \ volume~13\ \ \ issue\ 4}
}%
 \def\rightfootline{\small{INFORMATIKA I EE PRIMENENIYA~--- INFORMATICS AND APPLICATIONS\ \ \ 2019\ \ \ volume~13\ \ \ issue\ 4
\hfill \textbf{\thepage}}}

\vspace*{-60pt}

{\small

\noindent
Journal ``Informatics and Applications'' (Inform.\ Appl.)
publishes theoretical, review, and discussion
articles on the research and development in the
field of informatics and its applications.

The journal is published in Russian.
By a special decision of the editorial
board, some articles can be published in English.


The topics covered include the following areas:
\begin{itemize}
               \item
     theoretical fundamentals of informatics; \\[-14pt]
\item
mathematical methods for studying complex systems and processes; \\[-14pt]
\item
information systems and networks;\\[-14pt]
\item
information technologies; and \\[-14pt]
\item
architecture and software of computational complexes and networks. \\[-14pt]
\end{itemize}

\noindent
\begin{enumerate}[1.]
\item The Journal publishes original articles which have not been published before and are not
intended for simultaneous publication in other editions. An article submitted to the Journal must not violate the
Copyright law. Sending the manuscript to the Editorial Board, the authors retain all rights of the
owners of the manuscript and transfer the nonexclusive rights to publish the article in Russian
(or the language of the article, if not Russian) and its distribution in Russia and abroad to the
Founders and the Editorial Board. Authors should submit a letter to the Editorial Board in the
following form:

{\bfseries\textit{Agreement on the transfer of rights to publish:}}

``\textit{We, the undersigned authors of the manuscript ``\ldots'', pass to the
Founder and the Editorial Board of the Journal ``Informatics and Applications''
the nonexclusive right to publish the manuscript of the article in Russian (or
in English) in both print and electronic versions of the Journal. We affirm
that this publication does not violate the Copyright of other persons or
organizations.}

\textit{Author(s) signature(s): (name(s), address(es), date).}

This agreement should be submitted in paper form or in the form of a scanned copy (signed by
the authors).


%The Editorial Board has the right to request from the authors an official expert conclusion that
%the submitted article has no secret data prohibited for publication. \\[-13.5pt]
\item
A submitted article should be attached with \textbf{the data on the author(s)} (see item~8). If
there are several authors, the contact person should be indicated who is responsible for
correspondence with the Editorial Board and other authors about revisions and final approval
of the proofs.\\[-13.5pt]

\item The Editorial Board of the Journal examines the article according to the established
reviewing procedure. If the authors receive their article for correction after reviewing, it does not
mean that the article is approved for publication. The corrected article should be sent to the
Editorial Board for the subsequent review and approval.\\[-13.5pt]

\item The decision on the article publication or its rejection is communicated to the authors. The
Editorial Board may also send the reviews on the submitted articles to the authors. Any
discussion upon the rejected articles is not possible.\\[-13.5pt]

\item The edited articles will be sent to the authors for proofread. The comments of the authors
to the edited text of the article should be sent to the Editorial Board as soon as possible.\\[-13.5pt]

\item The manuscript of the article should be presented electronically in the MS WORD (.doc or
.docx) or \LaTeX\ (.tex) formats, and additionally in the .pdf format. All documents
 may be sent
by e-mail or provided on a CD or diskette. A~hard copy submission is not necessary.\\[-13.5pt]

\item The recommended typesetting instructions for manuscript.

Pages parameters: format A4, portrait orientation, document margins (cm): left~--- 2.5, right~---
1.5, above~--- 2.0, below~--- 2.0, footer 1.3.

Text: font~---Times New Roman, font size~--- 14, paragraph indent~--- 0.5, line spacing~--- 1.5,
justified alignment.

The recommended manuscript size: not more than 15~pages of the specified format.
If the specified size exceeded, the editorial board is entitled to require the author
to reduce the manuscript.

Use only standard abbreviations. Avoid  abbreviations in the title and
abstract. The full term for which an abbreviation stands should precede
its first use in the text unless it is a standard unit of measurement.

All pages of the manuscript should be numbered.

The templates for the manuscript typesetting are presented on site: {\sf
http://www.ipiran.ru/journal/template.doc}.\\[-13.5pt]


%\def\leftkol{Requirements for manuscripts submitted to Journal
%``Informatics~and~Applications''}

\item The articles should enclose data both in \textbf{Russian and English}:
\begin{itemize}
\item title;\\[-13.5pt]
\item author's name and surname;\\[-13.5pt]
\item affiliation~--- organization, its address with ZIP code, city, country, and
official e-mail address;\\[-13.5pt]
\item data on authors according to the format: (see site)

{\sf http://www.ipiran.ru/journal/issues/2013\_07\_01/authors.asp}  and

{\sf  http://www.ipiran.ru/journal/issues/2013\_07\_01\_eng/authors.asp};\\[-13.5pt]

\pagebreak

\def\leftfootline{\small{\textbf{\thepage}
\hfill INFORMATIKA I EE PRIMENENIYA~--- INFORMATICS AND APPLICATIONS\ \ \ 2019\
\ \ volume~13\ \ \ issue\ 4}
}%
 \def\rightfootline{\small{INFORMATIKA I EE PRIMENENIYA~--- INFORMATICS AND APPLICATIONS\ \ \ 2019\ \ \ volume~13\ \ \ issue\ 4
\hfill \textbf{\thepage}}}


%\def\leftkol{Requirements for manuscripts submitted to Journal
%``Informatics~and~Applications''}

%\def\rightkol{Requirements for manuscripts submitted to Journal
%``Informatics~and~Applications''}



\item abstract (not less than 100 words) both in Russian and in English. Abstract is a short
summary of the article that can be published separately. The abstract is the
main source of information on the article and it could be included in leading information
systems and data bases. The abstract in English has to be an original text and should
not be an exact translation of the Russian one. Good English is required.
In abstracts, avoid references and formulae;\\[-13.5pt]
\item indexing is performed on the basis of keywords. The use of keywords from the
internationally accepted thematic Thesauri is recommended.

%\def\leftkol{Requirements for manuscripts submitted to Journal
%``Informatics~and~Applications''}

%\def\rightkol{Requirements for manuscripts submitted to Journal
%``Informatics~and~Applications''}

Important! Keywords must not be sentences;
\item Acknowledgments.
\end{itemize}

\item References. Russian references have to be presented both in English translation and Latin
transliteration (refer {\sf http://www.translit.net/ru/bgn/}).

Please take into account the following examples of Russian references appearance:

\noindent
\textbf{Article in journal:}

\Aue{Zhang, Z., and D.~Zhu}. 2008. Experimental research on the localized electrochemical
micromachining.
\textit{Rus. J.~Electrochem.}  44(8):926--930. {\sf doi:10.1134/S1023193508080077}.


\noindent
\textbf{Journal article in electronic format:}

\Aue{Swaminathan, V., E.~Lepkoswka-White, and B.\,P.~Rao}. 1999. Browsers or buyers in
cyberspace? An
investigation of electronic factors influencing electronic exchange. \textit{JCMC}
5(2). Available at: {\sf http://www.ascusc.org/jcmc/vol5/issue2/} (accessed April~28, 2011).




\noindent
\textbf{Article from the continuing publication (collection of works, proceedings):}

\Aue{Astakhov, M.\,V., and T.\,V.~Tagantsev}. 2006. Eksperimental'noe
issledovanie prochnosti soedineniy ``stal'--kompozit'' [Experimental study of
the strength of joints ``steel--composite'']. \textit{Trudy MGTU
``Matematicheskoe modelirovanie slozhnykh tekh\-ni\-che\-skikh sistem''}
[\textit{Bauman MSTU ``Mathematical Modeling of Complex Technical
Systems'' Proceedings}]. 593:125--130.

\def\leftfootline{\small{\textbf{\thepage}
\hfill INFORMATIKA I EE PRIMENENIYA~--- INFORMATICS AND APPLICATIONS\ \ \ 2019\
\ \ volume~13\ \ \ issue\ 4}
}%
 \def\rightfootline{\small{INFORMATIKA I EE PRIMENENIYA~--- INFORMATICS AND APPLICATIONS\ \ \ 2019\ \ \ volume~13\ \ \ issue\ 4
\hfill \textbf{\thepage}}}

\def\leftkol{Requirements for manuscripts submitted to Journal
``Informatics~and~Applications''}

\def\rightkol{Requirements for manuscripts submitted to Journal
``Informatics~and~Applications''}

\noindent
\textbf{Conference proceedings:}

\Aue{Usmanov, T.\,S., A.\,A.~Gusmanov, I.\,Z.~Mullagalin, R.\,Ju.~Muhametshina,
A.\,N.~Chervyakova, and
A.\,V.~Sveshnikov}. 2007. Osobennosti proektirovaniya razrabotki mestorozhdeniy
s primeneniem gidrorazryva
plasta [Features of the design of field development with the use of hydraulic fracturing].
\textit{Trudy 6-go
Mezhdu\-na\-rod\-no\-go Simpoziuma ``Novye resursosberegayushchie tekhnologii
nedropol'zovaniya i povysheniya
neftegazootdachi''} [\textit{6th  Symposium (International) ``New Energy Saving Subsoil
Technologies and
the Increasing of the Oil and Gas Impact'' Proceedings}]. Moscow. 267--272.


\noindent
\textbf{Books and other monographs:}




Lindorf, L.\,S., and L.\,G.~Mamikoniants, eds. 1972.
\textit{Ekspluatatsiya turbogeneratorov s neposredstvennym
okhlazhdeniem} [\textit{Operation of turbine generators with direct cooling}].
Moscow: Energy Publs. 352~p.


%\Aue{Latyshev, V.\,N.} 2009. \textit{Tribologiya rezaniya. Kn.~1: Frikcionnye prosessy
%pri rezanii metallov}
%[\textit{Tribology of cutting. Vol.~1: Frictional processes in metal cutting}]. Ivanovo: Ivanovskii
%State Univ. 108~p.


%\noindent
%\textbf{Unpublished material:}

%\Aue{Latypov, A.\,R., M.\,M.~Khasanov, and V.\,A.~Baikov}.
%2004. Geology and production (NGT GiD). Certificate on official registration of the computer
%program
%No.\,2004611198. (In Russian, unpubl.)

%\noindent
%\textbf{Internet-source:}

%APA Style. 2011. Available at: {\sf http://www.apastyle.org/apa-style-help.aspx} (accessed
%February~5, 2011).

%Pravila citirovaniya istochnikov [Rules for the citing of sources]. Available at: {\sf
%http://www.scribd.com/doc/1034528/} (accessed February~7, 2011).


\noindent
\textbf{Dissertation and Thesis:}

%\Aue{Semenov, V.\,I.}
%2003. Matematicheskoe modelirovanie plazmy v sisteme kompaktnyy tor. [Mathematical
%modeling of the plasma in the compact torus]. D.Sc.\ Diss. Moscow. 272~p.

\Aue{Kozhunova, O.\,S.} 2009. Tekhnologiya razrabotki semanticheskogo
slovarya informatsionnogo monitoringa [Technology of development of
semantic dictionary of information monitoring system]. PhD Thesis. Moscow: IPI RAN. 23~p.


\noindent
\textbf{State standards and patents:}

GOST 8.586.5-2005. 2007. Metodika vypolneniya izmereniy. Izmerenie raskhoda i~kolichestva
zhidkostey i gazov 
s~pomoshch'yu standartnykh suzhayushchikh ustroystv [Method of measurement.
Measurement of flow rate and volume of liquids and gases by means of orifice devices]. M.:
Standardinform
Publs. 10~p.

%\noindent
%\textbf{Patent:}

\Aue{Bolshakov, M.\,V., A.\,V.~Kulakov, A.\,N.~Lavrenov, and M.\,V.~Palkin}.
2006. Sposob orientirovaniya po krenu letatel'nogo
apparata s opti\-che\-skoy golovkoy
samonavedeniya [The way to orient on the roll of aircraft with optical homing head].
Patent RF No.\,2280590.

References in Latin transcription are presented in the original language.

References in the text are numbered according to the order of their
first appearance; the number is
placed in square brackets. All items from the reference list should be
cited.\\[-13.5pt]

\item Manuscripts and additional materials are not returned to Authors by the Editorial Board.\\[-13.5pt]

\item Submissions of files by e-mail must include:\\[-13.5pt]
\begin{itemize}
\item   the journal title and author's name in the ``Subject'' field; \\[-13.5pt]
\item   an article and additional materials have to be attached using the ``attach'' function;\\[-13.5pt]
\item   an electronic version of the article should contain the file with the text and a separate file
with figures.\\[-13.5pt]
\end{itemize}

\item ``Informatics and Applications'' journal is not a profit publication. There are no
charges for the authors as well as there are no royalties.\\[-13.5pt]
\end{enumerate}

\def\leftfootline{\small{\textbf{\thepage}
\hfill INFORMATIKA I EE PRIMENENIYA~--- INFORMATICS AND APPLICATIONS\ \ \ 2019\
\ \ volume~13\ \ \ issue\ 4}
}%
 \def\rightfootline{\small{INFORMATIKA I EE PRIMENENIYA~--- INFORMATICS AND APPLICATIONS\ \ \ 2019\ \ \ volume~13\ \ \ issue\ 4
\hfill \textbf{\thepage}}}

\def\leftkol{Requirements for manuscripts submitted to Journal
``Informatics~and~Applications''}

\def\rightkol{Requirements for manuscripts submitted to Journal
``Informatics~and~Applications''}


%\vspace*{5mm}


\begin{center}
\textbf{Editorial Board address:} \\

%ABOUT AUTHORS



FRC CSC RAS, 44, block~2, Vavilov Str., Moscow 119333, Russia\\[-10pt]

\

Ph.: +7\,(499)\,135\,86\,92,\ \ Fax: +7\,(495)\,930\,45\,05\\[-10pt]

\

 e-mail: {\sf rust@ipiran.ru} (to Prof.\ Rustem Seyful-Mulyukov)\\[-10pt]

\

 {\sf http://www.ipiran.ru/english/journal.asp}
\end{center}
 }
%\thispagestyle{myheadings}

\def\leftkol{Requirements for manuscripts submitted to Journal
``Informatics~and~Applications''}

\def\rightkol{Requirements for manuscripts submitted to Journal
``Informatics~and~Applications''}

\def\leftfootline{\small{\textbf{\thepage}
\hfill INFORMATIKA I EE PRIMENENIYA~--- INFORMATICS AND APPLICATIONS\ \ \ 2019\
\ \ volume~13\ \ \ issue\ 4}
}%
 \def\rightfootline{\small{INFORMATIKA I EE PRIMENENIYA~--- INFORMATICS AND APPLICATIONS\ \ \ 2019\ \ \ volume~13\ \ \ issue\ 4
\hfill \textbf{\thepage}}}

 \label{end\stat}

\newpage

%\vspace*{-60pt} {\small
{\baselineskip=9.1pt
\section*{Правила подготовки рукописей статей для публикации в журнале
<<Информатика и её применения>>}

\thispagestyle{empty}

 Журнал <<Информатика и её применения>> публикует
теоретические, обзорные и дискуссионные статьи, посвященные научным
исследованиям и разработкам в области информатики и ее приложений. Журнал
издается на русском языке. По специальному решению редколлегии отдельные статьи,
в виде исключения, могут печататься на английском языке.
Тематика журнала охватывает следующие направления:
\begin{itemize}
\item теоретические основы информатики; %\\[-13.5pt]
\item математические методы исследования сложных систем и процессов; %\\[-13.5pt]
\item информационные системы и сети; %\\[-13.5pt]
\item информационные технологии; %\\[-13.5pt]
\item архитектура и программное
обеспечение вычислительных комплексов и сетей.
\end{itemize}
\begin{enumerate}
\item В журнале печатаются результаты, ранее не
опубликованные и не предназначенные к одновременной публикации в других
изданиях. Публикация не должна нарушать закон об авторских правах. Направляя
свою рукопись в редакцию, авторы автоматически передают учредителям и
редколлегии неисключительные права на издание данной статьи на русском языке и
на ее распространение в России и за рубежом. При этом за авторами сохраняются
все права как собственников данной рукописи. В связи с этим авторами должно
быть представлено в редакцию письмо в следующей форме:
Соглашение о передаче права на публикацию:

\textit{<<Мы, нижеподписавшиеся, авторы рукописи <<$\qquad\qquad$>>, передаем
учредителям и редколлегии журнала <<Информатика и её применения>>
неисключительное право опубликовать данную рукопись статьи на русском языке как
в печатной, так и в электронной версиях журнала. Мы подтверждаем, что данная
публикация не нарушает авторского права других лиц или организаций. Подписи
авторов: (ф.\,и.\,о., дата, адрес)>>.}

Указанное соглашение может быть представлено 
как в бумажном виде, так и в виде отсканированной копии (с подписями авторов).


Редколлегия вправе запросить у авторов экспертное заключение о возможности
опубликования представленной статьи в открытой печати. %\\[-13.5pt]
\item Статья
подписывается всеми авторами. На отдельном листе представляются данные автора
(или всех авторов): фамилия, полные имя и отчество, телефон, факс, e-mail,
почтовый адрес. Если работа выполнена несколькими авторами, указывается фамилия
одного из них, ответственного за переписку с редакцией. %\\[-13.5pt]
\item Редакция журнала
осуществляет самостоятельную экспертизу присланных статей. Возвращение рукописи
на доработку не означает, что статья уже принята к печати. Доработанный вариант
с ответом на замечания рецензента необходимо прислать в редакцию. %\\[-13.5pt]
\item Решение
редакционной коллегии о принятии статьи к печати или ее отклонении сообщается
авторам. Редколлегия не обязуется направлять рецензию авторам отклоненной
статьи. %\\[-13.5pt]
\item Корректура статей высылается авторам для просмотра. Редакция
просит авторов присылать свои замечания в кратчайшие сроки. %\\[-13.5pt]
\item При
подготовке рукописи в MS Word рекомендуется использовать следующие настройки.
Параметры страницы: формат~--- А4; ориентация~--- книжная; поля (см): внутри~---
2,5, снаружи~--- 1,5, сверху~--- 2, снизу~--- 2, от края до нижнего
колонтитула~--- 1,3. Основной текст: стиль~--- <<Обычный>>: шрифт Times New
Roman, размер 14~пунктов, абзацный отступ~--- 0,5~см, 1,5 интервала,
выравнивание~--- по ширине. Рекомендуемый объем рукописи~--- не свыше
25~страниц указанного формата. Ознакомиться с шаблонами, содержащими примеры
оформления, можно по адресу в Интернете:
\textsf{http://www.ipiran.ru/journal/template.doc}.
\item К рукописи, предоставляемой в 2-х
экземплярах, обязательно прилагается электронная версия статьи (как правило, в
форматах MS WORD (.doc) или \LaTeX\ (.tex), а также~--- дополнительно~--- в
формате .pdf) на дискете, лазерном диске или по электронной почте. Сокращения
слов, кроме стандартных, не применяются. Все страницы рукописи должны быть
пронумерованы. %\\[-13.5pt]
\item Статья должна содержать следующую информацию на русском и
английском языках: название, Ф.И.О. авторов, места работы авторов и их
электронные адреса, подробные сведения об авторах, оформленные в соответствии с форматом, 
определяемым файлами {\sf http://www.ipiran.ru/journal/issues/2011\_05\_01/authors.asp} и 
{\sf http://www.ipiran.ru/journal/issues/2011\_01\_eng/authors.asp},
аннотация (не более 100~слов), ключевые слова. Ссылки на
литературу в тексте статьи нумеруются (в квадратных скобках) и располагаются в
порядке их первого упоминания. В~списке литературы не должно быть позиций, на которые нет ссылки в тексте статьи.
Все фамилии авторов, заглавия статей, названия
книг, конференций и~т.\,п.\ даются на языке оригинала, если этот язык
использует кириллический или латинский алфавит. %\\[-13.5pt]
\item Присланные в редакцию материалы авторам не возвращаются.
\item При отправке файлов по электронной
почте просим придерживаться следующих правил:
\begin{itemize}
\item указывать в поле subject (тема) название журнала и фамилию автора; %\\[-13.5pt]
\item использовать attach (присоединение); %\\[-13.5pt]
\item в случае больших объемов информации возможно
использование общеизвестных архиваторов (ZIP, RAR); %\\[-13.5pt]
\item в состав электронной версии статьи должны входить: файл, содержащий текст статьи, и файл(ы),
содержащий(е) иллюстрации. %\\[-13.5pt]
\end{itemize}
\item Журнал <<Информатика и её применения>> является некоммерческим изданием. 
Плата за публикацию с авторов не взимается, гонорар авторам не выплачивается.
\end{enumerate}
\thispagestyle{empty}
\textbf{Адрес редакции:} Москва 119333,
ул.~Вавилова, д.~44, корп.~2, ИПИ РАН\\
\hphantom{\textbf{Адрес редакции:} }Тел.: +7 (499) 135-86-92\ \
Факс:  +7 (495) 930-45-05\ \  E-mail:   rust@ipiran.ru }
}

%\include{ipi-ind}

%\tableofcontents

\end{document}

%\tableofcontents

%\end{document}

%\tableofcontents


\end{document}

\newcommand{\Ack}{\subsection*{\protect\large\bf Acknowledgments}}

\vphantom*{\int\limits_0^T}

{ \begin{center}  %fig1
 \vspace*{3pt}
    \mbox{%
 \epsfxsize=79mm 
 \epsfbox{gru-1.eps}
 }

\end{center}

\noindent
{{\figurename~1}\ \ \small{
Временные зависимости данные 
}}}

\vspace*{6pt}

\addtocounter{figure}{1}

$\acute{\mbox{о}}$

\linebreak