\def\stat{agalarov}

\def\tit{ОПТИМИЗАЦИЯ ПОРОГОВОГО УПРАВЛЕНИЯ ПЕРЕКЛЮЧЕНИЕМ СКОРОСТИ 
ОБСЛУЖИВАНИЯ В~СИСТЕМЕ~МАССОВОГО ОБСЛУЖИВАНИЯ $G/M/1$$^*$}

\def\titkol{Оптимизация порогового управления переключением скорости 
обслуживания в~СМО %системе массового обслуживания 
$G/M/1$}

\def\aut{Я.\,М.~Агаларов$^1$}

\def\autkol{Я.\,М.~Агаларов}

\titel{\tit}{\aut}{\autkol}{\titkol}

\index{Агаларов Я.\,М.}
\index{Agalarov Ya.\,M.}


{\renewcommand{\thefootnote}{\fnsymbol{footnote}} \footnotetext[1]
{Работа выполнена при поддержке РФФИ (проект~20-07-00804).}}


\renewcommand{\thefootnote}{\arabic{footnote}}
\footnotetext[1]{Федеральный исследовательский центр <<Информатика и~управление>> Российской академии 
наук, \mbox{agglar@yandex.ru}}

%\vspace*{-6pt}


\Abst{Рассматривается задача оптимизации управления переключением скорости обслуживания 
в~сис\-те\-ме массового обслуживания  (СМО) типа $G/M/1$ с~целевой функцией, учитывающей плату за 
обслуживание заявки, штраф за задержку заявки в~очереди, штраф за отклонение заявки, штраф за 
простой прибора обслуживания и~затраты на техническое обслуживание сис\-те\-мы. В~рамках этой 
задачи рассмотрена аналогичная задача для СМО с~ограниченной 
очередью. Она сформулирована как задача нелинейного программирования, в~которой целевой 
функцией служит доход сис\-те\-мы, а~допустимым планом (переменной управления)~--- длина 
очереди. Доказано свойство унимодальности функции дохода, сформулированы условия 
существования оптимального порогового управления переключением скорости обслуживания, 
необходимые и~достаточные условия оптимальности управления.}
 
  \KW{система массового обслуживания; пороговое управление; доход сис\-темы}
  
\DOI{10.14357/19922264220111}
  
%\vspace*{-4pt}


\vskip 10pt plus 9pt minus 6pt

\thispagestyle{headings}

\begin{multicols}{2}

\label{st\stat}
  
  
\section{Введение}

  Существуют много реальных сис\-тем (экономических, транспортных, 
информационных и~т.\,д.), для расчета экономической и~технической 
эффективности которых на стадии проектирования в~качестве математических 
моделей используют СМО. 
С~использованием этих моделей, как правило, решают задачи анализа, синтеза 
и~выбора оптимального управления для проектируемой сис\-те\-мы (см., 
например,~[1--3]). Исследуемой ниже одной из таких задач посвящен ряд 
публикаций (см., например,~[4--8]), в~которых она сформулирована\linebreak как задача 
поиска оптимального порогового значения длины очереди или времени 
пребывания заявки, по достижении которого переключается скорость 
обслуживания с~целью максимизации \mbox{дохода} СМО. В~работе~\cite{4-aga} 
изучена сис\-те\-ма с~простейшим потоком, экспоненциальным обслуживанием, 
произвольным числом симметричных резервных каналов и~гистерезисной 
стратегией. \mbox{Показано}, что оптимальная стратегия включения и~отключения 
резервных каналов является безгистерезисной. Приведены соображения по 
минимизации целевой\linebreak функции, учитывающей штрафы за потери и~задержки 
заявок в~очереди, путем оптимизации стратегии включения и~отключения 
резервных каналов. В~рассмотренной в~\cite{5-aga} однолинейной \mbox{сис\-те\-ме} 
входящий поток~--- простейший, время обслуживания~--- экспоненциальное 
с~ин\-тен\-сив\-ностью обслуживания, зависящей от текущего времени пребывания 
в~сис\-те\-ме заявки. Получено уравнение для оптимальной интенсивности 
обслуживания при целевой функции, учитывающей потери и~амортизацию. 
В~работе~\cite{6-aga} изучена однолинейная СМО с~простейшим входным 
потоком, экспоненциальным обслуживанием и~с~однотипным резервным 
прибором, управляемым по текущему времени ожидания заявки, находящейся 
первой в~очереди. Проведена оптимизация сис\-те\-мы при учете потерь на 
ожидание и~амортизацию. Работа~\cite{7-aga} посвящена анализу СМО 
с~бесконечным накопителем, основным и~резервным прибором, 
с~экспоненциальным обслуживанием, скачкообразной интенсивностью 
входного дважды стохастического пуассоновского потока. 
  
  Ниже приведены результаты исследования задачи максимизации дохода 
СМО типа $G/M/1$ с~переключением скорости обслуживания при достижении 
длиной очереди порогового значения.

\vspace*{-12pt}

\section{Постановка задачи}

\vspace*{-3pt}

  Рассматривается СМО типа $G/M/1$ с~одним прибором обслуживания 
  и~неограниченной оче\-редью, на которую поступает рекуррентный поток заявок 
  с~функцией распределения вероятностей $A(t)$. Прибор может работать в~одном 
из двух скоростных режимов, выбираемом в~момент поступления заявки 
в~зависимости от числа заявок\linebreak\vspace*{-12pt}

\pagebreak

\noindent
 в~сис\-те\-ме $(i)$ и~не меняющемся до 
поступления следующей заявки. Первый режим назначается при $i\hm\geq0$, 
если $i\hm\leq h_1\hm- 1$, $h_1\hm\geq 1$, а~второй~--- если $i\hm\geq h_1$. 
Время обслуживания заявки распределено по экспоненциальному закону 
с~па\-ра\-мет\-ром $\mu_1\hm>0$ при первом режиме функционирования прибора 
и~с~па\-ра\-мет\-ром $\mu_2\hm\geq \mu_1$ при втором режиме. Величину~$h_1$ 
будем называть пороговым значением числа заявок в~сис\-теме.
  
  Заявка обслуживается на приборе в~порядке поступления и~покидает сис\-те\-му 
только при завершении обслуживания, освободив одновременно прибор 
и~накопитель, а~на освободившийся прибор поступает очередная заявка из 
накопителя (если такая есть). 
  
  Система получает доход, зависящий от сле\-ду\-ющих параметров:
  \begin{description}
  \item[\,] $C_0\geq 0$~--- плата, получаемая сис\-те\-мой, если поступившая 
заявка обслужена сис\-те\-мой (принята в~накопитель); 
  \item[\,] $C_1\geq 0$~--- величина штрафа, который платит сис\-те\-ма, если 
поступившая заявка отклонена;
  \item[\,] $C_2\geq 0$~--- штраф за единицу времени ожидания заявки 
в~сис\-теме;
  \item[\,] $C_3\geq 0$~--- штраф за единицу времени простоя единицы 
пропускной способности прибора;
  \item[\,] $C_{1,4}\geq 0$~--- затраты за единицу времени технического 
обслуживания сис\-те\-мы при первом режиме работы;
  \item[\,] $C_{2,4}\geq C_{1,4}$~--- затраты за единицу времени технического 
обслуживания сис\-те\-мы при втором режиме работы. 
  \end{description}
  
  
Под доходом сис\-те\-мы будем понимать суммарный доход с~учетом всех 
вышеуказанных со\-став\-ля\-ющих.
\begin{description}
\item[\,] $D^{h_1}$~--- предельное среднее значение дохода сис\-те\-мы в~единицу 
времени при пороге~$h_1$;
  \item[\,] $g^{h_1}(a)$~--- предельное среднее значение суммарного дохода 
сис\-те\-мы, усред\-нен\-но\-го по чис\-лу по\-сту\-пив\-ших заявок при пороге~$h_1$;
  \item[\,] $q_i^{h_1}$~---  средний доход, получаемый сис\-те\-мой в~со\-сто\-янии~$i$ при пороге~$h_1$.
  \end{description}
  
  Предельное среднее значение суммарного дохода сис\-те\-мы, усред\-нен\-но\-го по 
чис\-лу по\-сту\-пив\-ших заявок при пороге~$h_1$, равно
  $$
  g^{h_1} = \lim\limits_{T\to\infty} \sum\limits_{n=1}^{ N_{\mathrm{вх}}} 
\fr{d_n^{h_1}}{N_{\mathrm{вх}}(T)}\,.
  $$
  
  Предельное среднее значение дохода сис\-те\-мы в~единицу времени при 
пороге~$h_1$ равно

\noindent
  \begin{multline*}
  D^{h_1}=\lim\limits_{T\to\infty} \sum\limits_{n=1}^{N_{\mathrm{вх}}(T)}
  \fr{d_n^{h_1}}{T} ={}\\
  {}= \lim\limits_{T\to\infty} \fr{ N_{\mathrm{вх}}(T)}{T} 
\lim\limits_{T\to\infty} \sum\limits_{n=1}^{N_{\mathrm{вх}}(T)} \fr{d_n^{h_1}}
  {N_{\mathrm{вх}}(T)} =\fr{g^{h_1}}{\overline{v}}\,,
  \end{multline*}
где $d_n^{h_1}$~--- доход, полученный сис\-те\-мой при пороге~$h_1$ за $n$-ю 
поступившую заявку; $N_{\mathrm{вх}}(T)$~--- чис\-ло по\-сту\-пив\-ших за отрезок 
времени $[0,T]$ заявок. 
  
  Ставится задача максимизации функции~$D^{h_1}$ по $h_1\hm>0$, которая 
эквивалентна задаче: найти оптимальный порог $h_1^*\hm>0$ такой, что
  \begin{equation}
  \max\limits_{h_1>0} g^{h_1}=g^{h_1^*}\,.
  \label{e1-aga}
  \end{equation}
  
\section{Метод решения}

  Рассмотрим задачу~(\ref{e1-aga}) для вспомогательной СМО, отличающейся 
от описанной выше в~разд.~2 только тем, что число мест в~накопителе 
ограничено величиной $h_2\hm= h_1\hm+ a$, $h_1\hm\geq 1$, $a\hm= const 
\hm\geq0$. В~этой сис\-те\-ме второй режим включается, если заявка застает 
в~сис\-те\-ме~$i$ заявок, $h_1\hm\leq i\hm\leq h_2$, и~отключается, если 
в~момент поступления $i\hm <h_1$. Заявка, заставшая сис\-те\-му  
в~со\-сто\-янии~$h_2$, теряется. 
  
  Отметим, что процесс обслуживания заявок в~данной сис\-те\-ме описывается 
цепью Маркова с~одним положительным возвратным классом со\-сто\-яний $\{0, 
\ldots , h_2\}$, где переходы цепи определяются моментами поступления 
заявок, а~состояние цепи~--- числом заявок, находящихся в~сис\-те\-ме в~момент 
поступления. 
  
  Введем обозначения:
  \begin{description}
  \item[\,] $\overline{v}$~--- первый момент функции распределения $A(t)$, 
$0\hm<\overline{v}\hm< \infty$;
  \item[\,]  $\{\pi_i^{h_1}, \ 0\hm\leq i\hm\leq h_2\}$~--- стационарное 
распределение вероятностей цепи при пороге~$h_1$ ($\pi_i^{h_1}$~--- 
стационарная вероятность того, что цепь находится в~состоянии~$i$);
  \item[\,]  $g^{h_1}(a)$~--- предельное среднее значение суммарного дохода 
вспомогательной сис\-темы.
  \end{description}
  
Из определения вложенной цепи Маркова следует:
\begin{equation}
g^{h_1}(a)=\sum\limits_{i=0}^{h_2} \pi_i^{h_1} q_i^{h_1}\,.
\label{e2-aga}
\end{equation}
  
  Выпишем выражения для стационарных вероятностей~$\pi_i^{h_1}$, $i\hm= 
0,\ldots , h_2$. Для вероятностей  переходов~$p^{h_1}_{i,j}$ вложенной цепи 
Маркова справедлива формула:

\noindent
  $$
  p_{i,j}^{h_1}= 
  \begin{cases}
  r^{h_1}_{i,i+1-j}, &\!\!0\leq i\leq h_2-1\,,\ 1\leq j\leq i+1\,;\\
  r^{h_1}_{i,h_2-j}\,, &\!\! i=h_2,\ 1\leq j\leq h_2\,;\\
  1-\displaystyle \sum\limits^i_{m=0} \!r^{h_1}_{i,m}\,, &\!\! i\leq h_2-1,\ j=0\,;\\
  1-\displaystyle \sum\limits_{m=0}^{h_2-1}\!r^{h_1}_{i,m}, &\!\! i=h_2,\ j=0\,.
  \end{cases}
  \hspace*{-2.05016pt}
  $$
Здесь $m\geq 0$; $r^{h_1}_{i,m}$~--- вероятность того, что в~состоянии~$i$ будут 
обслужены ровно~$m$~заявок:
\begin{equation*}
r^{h_1}_{i,m} = \int\limits_0^\infty \fr{(\mu_i^{h_1}t)^m}{m!}\,e^{-
\mu_i^{h_1} t}\,dA(t)\,,
\end{equation*}
где
\begin{align*}
\mu_i^{h_1} &= \begin{cases}
\mu_1, &  0\leq i\leq h_1-1\,;\\
\mu_2, & h_2\leq i\leq h_2\,.
\end{cases}
\end{align*}
  
  Для рассматриваемой цепи Маркова при пороге~$h_1$ стационарное 
распределение вероятностей является единственным решением сис\-те\-мы 
уравнений
  \begin{equation}
  \left.
  \begin{array}{rl}
  \pi_0^{h_1} &= \displaystyle \sum\limits^{h_1}_{i=0} \pi_i^{h_1} 
p^{h_1}_{i,0}\,;\\[6pt]
  \pi_j^{h_1} &= \displaystyle \sum\limits^{h_2}_{i=j-1} \pi_i^{h_1} 
p^{h_1}_{i,j}\,, \ 1\leq j\leq h_2\,;\\[6pt]
  \displaystyle \sum\limits_{i=0}^{h_2} \pi_i^{h_1} &= 1\,,\ \pi_i^{h_1}>0\,,\ 0\leq 
i\leq h_2\,.
  \end{array}
  \right\}
  \label{e3-aga}
  \end{equation}
  
  Из~(\ref{e3-aga}), заменив в~каждом уравнения~$\pi_j^{h_1}$ 
на~$R_j^{h_1}\pi^{h_1}_{h_2}$ и~исключив последовательно в~каждом 
уравнении~$R_j^{h_1}$ с~наименьшим индексом, получим рекуррентные 
формулы для вычисления стационарных вероятностей состояний $\pi_j^{h_1}$, 
$0\hm\leq j\hm\leq h_2$:
 \begin{equation}
  \pi_j^{h_1}=\fr{R_j^{h_1}}{\sum\nolimits_{i=0}^{h_2} R_i^{h_1}}\,,\enskip 
0\leq j\leq h_2\,,
   \label{e4-aga}
\end{equation}
  где
  \begin{align*}
    &R_{h_2}^{h_1} =1\,;\quad R^{h_1}_{h_2-1}=\displaystyle
  \fr{1-r_{h_2,0}^{h_1}}{r^{h_1}_{h_2,0}}\,;\\
   &R_{j-1}^{h_1}= \fr{R_j^{h_1}(1-r^{h_1}_{j,1})}{r^{h_1}_{j-1,0}} -{}\\
   &{}-
\displaystyle\fr{\sum\nolimits_{i=j+1}^{h_2-1} R_i^{h_1} r^{h_1}_{j,i+1-j} -R^{h_1}_{h_2} r^{h_1}_{h_2,h_2-j}}
 {r^{h_1}_{j-1,0}}\,,\\
&\hspace*{40mm}  h_1+1\leq j \leq h_2-1\,;
\end{align*}

\noindent
\begin{align*}
  & R^{h_1}_{h_1-1}=\displaystyle \fr{R^{h_1}_{h_1}(1-r_{h_1,1})}{r^{h_1}_{h_1-1,0}} -{}\\
  &\hspace*{10mm} {}- 
   \fr{\sum\nolimits_{i=h_1+1}^{h_2-1} R_i^{h_1} r_{i,i+1-j} - R^{h_1}_{h_2} r_{h_2,h_2-h_1}} 
{r^{h_1}_{h_1-1,0}}\,;\\
  & R^{h_1}_{j-1} = \displaystyle\fr{R_j^{h_1}(1-r_{j,1})}{r^{h_1}_{j-1,0}}-{}\\
   & {}- \displaystyle\fr{\sum\nolimits_{i=j+1}^{h_1-1} R_i^{h_1} r_{i,i+1-j} - 
\sum\nolimits_{i=h_1}^{h_2-1} R_i^{h_1} r_{i,i+1-j} }{r^{h_1}_{j-1,0}} - {}\\
&\hspace*{23mm}{}- \fr{R^{h_1}_{h_2} r_{h_2,h_2-j}} 
{r^{h_1}_{j-1,0}}\,, \enskip 1\leq j\leq h_1-1\,.
     \end{align*}
  
  Докажем несколько лемм. 
  
  \smallskip
  
  \noindent
  \textbf{Лемма~1.}\ \textit{Среднее значение дохода, получаемого сис\-те\-мой 
при пороге~$h_1$ в~состоянии~$i$, равно}
  \begin{equation}
  q_i^{h_1}=\begin{cases}
  d_i^{h_1}+C_0\,, & 0\leq i\leq h_2-1\,;\\
  d^{h_1}_{h_2-1} -C_1\,, & i=h_2\,,
  \end{cases}
  \label{e5-aga}
  \end{equation}
  \textit{где}
  \begin{multline*}
  d_i^{h_1} =\fr{C_2}{\mu_i^{h_1}} \left[ \fr{1}{2}\sum\limits_{m=1}^{i+1} (m-1) 
mr^{h_1}_{i,m} -i\sum\limits_{m=1}^{i+1} mr^{h_1}_{i,m} -{}\right.\\
\left.{}-\fr{1}{2} i(i+1) 
\sum\limits^\infty_{m=i+2} r^{h_1}_{i,m}\right] - C_3 \sum\limits^\infty_{m=i+2} 
(m-i-1) r^{h_1}_{i,m} -{}\\
{}-C_4^{h_1}(i)\overline{v}\,,
\enskip
   C_4^{h_1}(i)= \begin{cases}
   C_{1,4}\,, & 0\leq i\leq h_1-1\,;\\
   C_{2,4}\,, & h_1\leq i\leq h_2-1\,.
   \end{cases}
  \end{multline*}
  
  \noindent
  Д\,о\,к\,а\,з\,а\,т\,е\,л\,ь\,с\,т\,в\,о\,.\ \ Фиксируем со\-сто\-яние~$i$, 
и~пусть время нахождения сис\-те\-мы в~со\-сто\-янии~$i$ рав\-но~$v$. Найдем 
выражения для суммарного сред\-не\-го времени ожидания всех заявок в~очереди 
и~среднего времени простоя прибора в~со\-сто\-янии~$i$, т.\,е.\ в~интервале 
времени $(0,v]$. 
  
  Рассмотрим случайную величину (СВ) вида $W_m\hm= \sum\nolimits^m_{j=1} \tau_j$, $m\hm\geq 
1$, где $\tau_j$~--- независимые экспоненциально распределенные СВ
с~па\-ра\-мет\-ром $\mu(i)\hm>0$. Пусть~$B_n$~--- событие вида ($W_n\hm\leq v$, 
$W_{n+1}\hm>v$), $B_0$~--- событие ($W_1\hm>v$). Обратим внимание, 
что~$B_n$, $n\hm\geq 0$,~--- несовместные события и~в~совокупности 
со\-став\-ля\-ют пол\-ную группу событий. 
  
  Известно (см., например,~\cite{5-aga}), что математическое ожидание СВ
$W_n$ при условии выполнения события~$B_m$ равно 
  $$
  {\sf M}\left[\fr{W_n}{B_m}\right]= \fr{n}{m+1}\,v,\enskip n\leq m\,.
  $$
  
  Заметим, что в~состоянии~$i$ общее время ожидания $n$-й в~очереди заявки 
при $n\hm\geq 1$ является СВ вида~$W_n$, а~выполнение 
события~$B_m$ равносильно завершению обслуживания за время~$v$ ров\-но 
$m$~заявок. 
  
  Обозначим через $\overline{W}_{\mathrm{обсл}/m}$ среднее суммарное 
время ожидания заявок, обслуживание которых завершилось или  началось 
в~со\-сто\-янии~$i$, при условии~$B_m$,  
$\overline{W}_{\mathrm{необсл}/m}$~--- среднее суммарное время ожидания 
в~очереди заявок, обслуживание которых не началось в~со\-сто\-янии~$i$, при 
условии~$B_m$, $\overline{W}_{\mathrm{пр}/m}$~--- сред\-нее время простоя 
прибора в~со\-сто\-янии~$i$ при условии~$B_m$, $k$~--- чис\-ло заявок в~сис\-те\-ме 
в~состоянии~$i$ с~учетом по\-сту\-пив\-шей (т.\,е.\ $k\hm= i\hm+1$, если она 
принимается в~накопитель, и~$k\hm=i$, если она отвергается). Тогда при 
$m\hm\leq k\hm-1$ получим: 
  \begin{equation}
  \overline{W}_{\mathrm{обсл}/m} =\sum\limits^m_{n=1} {\sf M}\left[ 
\fr{W_n}{B_m}\right] = \sum\limits^m_{n=1} \fr{n}{m+1}\,v=\fr{mv}{2}\,; 
\label{e6-aga}\\
  \end{equation}
  \begin{equation}
  \overline{W}_{\mathrm{необсл}/m} = \left[ k-(m+1)\right] v=(k-1-m)v\,.
  \label{e7-aga}
  \end{equation}
  
  Выполнение события~$B_m$ при $m\hm\geq k$ означает завершение 
обслуживания всех заявок в~очереди и~по\-сле\-ду\-ющее про\-ста\-и\-ва\-ние прибора до 
момента выхода сис\-те\-мы из со\-сто\-яния~$i$. Поэтому при $m\hm\geq k\hm\geq 
1$ вер\-но равенство:

\vspace*{-3pt}

\noindent
  \begin{multline}
  \overline{W}_{\mathrm{обсл}/m} =\sum\limits^{k-1}_{n=1} {\sf  M}\left[ 
\fr{W_n}{B_m}\right] ={}\\
{}=\sum\limits_{n=1}^{k-1} \fr{n}{m+1}\,v= \fr{k(k-1)v}{2(m+1)}\,.
  \label{e8-aga}
  \end{multline}
  
  \vspace*{-3pt}
  
  \noindent
При $m\geq k$ и~условии~$B_m$ для времени простоя вер\-но соотношение:

\vspace*{-3pt}

\noindent
\begin{multline}
\overline{W}_{\mathrm{пр}/m} = v-{\sf M}\left[ \fr{W_k}{B_m}\right] ={}\\
{}= v-\fr{k}{m+1}\,v=\fr{(m-k+1)v}{m+1}\,.
\label{e9-aga}
\end{multline}

\vspace*{-3pt}
  
  Обозначим через $d_i^{h_1}(v)$ величину суммарного дохода сис\-те\-мы при 
пороге~$h_1$ в~со\-сто\-янии~$i$ ($0\hm\leq i\hm\leq h_2\hm-1$) без учета платы за 
обслуживание и~штрафа за отклонение по\-сту\-пив\-шей заявки при условии, что 
время пребывания в~со\-сто\-янии~$i$ рав\-но~$v$. Так как вероятность 
события~$B_m$ рав\-на $((\mu_i^{h_1}v)^m/m!)e^{-\mu_i^{h_1}v}$, из формулы 
полной ве\-ро\-ят\-ности и~из~(\ref{e6-aga})--(\ref{e9-aga}) получим для 
$d_i^{h_1}(v)$ выражение вида:

\vspace*{-3pt}

\noindent
  \begin{multline*}
  d_i^{h_1}(v)=-C_4^{h_1}(i) v-{}\\
  {}- C_2\!\sum\limits^i_{m=0}\!
  \fr{\left(\mu_i^{h_1} v\right)^m}{m!}\,e^{-\mu_i^{h_1}v} \left( 
\overline{W}_{\mathrm{обсл}/m} +\overline{W}_{\mathrm{необсл}/m}\right)-
{}\hspace*{-5.52901pt}
\end{multline*}

\noindent
 \begin{multline*}
  {}- \sum\limits^\infty_{m=i+1} \fr{\left(\mu_i^{h_1} v\right)^m}{m!}\, e^{-\mu_i^{h_1}v}\left( 
  C_2 \overline{W}_{\mathrm{обсл}/m} +{}\right.\\
\left.  {}+C_3\mu_i^{h_1}\overline{W}_{\mathrm{пр}/m}\right) =-C_4^{h_1}(i)v-{}\\
  {}- C_2\sum\limits^i_{m=0} \fr{\left(\mu_i^{h_1} v\right)^m}{m!}\,e^{-\mu_i^{h_1}v} 
\left[ \fr{mv}{2}+(i-m)v\right] -{}\\
{}-C_2
  \sum\limits^\infty_{m=i+1} \fr{\left(\mu_i^{h_1}v\right)^m}{m!}\,e^{-\mu_i^{h_1}v} 
\fr{i(i+1)v}{2(m+1)}-{}\\
  {}- C_3\mu_i^{h_1}\sum\limits^\infty_{m=i+1} \fr{\left(\mu_i^{h_1} 
v\right)^m}{m!}\,e^{-\mu_i^{h_1}v} \fr{\left(m-i\right)v}{m+1}={}\\
  {}= -C_4^{h_1}(i)v +\fr{C_2}{2}\sum\limits^i_{m=0} \fr{m\left(\mu_i^{h_1}\right)^m 
v^{m+1}}{m!}\,e^{-\mu_i^{h_1}v}-{}\\
  {}- C_2i \sum\limits^i_{m=0} \fr{\left(\mu_i^{h_1}\right)^m v^{m+1}}{m!}\,e^{-\mu_i^{h_1}v} -{}\\
  {}-C_2 \fr{i(i+1)}{2} \sum\limits^\infty_{m=i+1} 
\fr{\left(\mu_i^{h_1}\right)^m v^{m+1}}{(m+1)!}\,e^{-\mu_i^{h_1}v}-{}\\
  {}- C_3 \mu_i^{h_1} \sum\limits^\infty_{m=i+1} \fr{\left(\mu_i^{h_1}\right)^m 
v^{m+1}}{(m+1)!} \left( m-i\right) e^{-\mu_i^{h_1}v}\,.
  \end{multline*}
  %
  Следовательно,
  
  \vspace*{-3pt}
  
  \noindent 
  \begin{multline*}
  d_i^{h_1}=\int\limits_0^\infty d_i(v)\,dA(v)=-C_4^{h_1}(i)\int\limits_0^\infty 
v\,dA(v)+{}\\
  {}+\fr{C_2}{2\mu_i^{h_1}} \sum\limits^{i+1}_{m=1} m(m-1) 
\int\limits_0^\infty \fr{\left(\mu_i^{h_1}\right)^m v^m}{m!}\,e^{-\mu_i^{h_1}v}\,dA(v)-{}\\
  {}- \fr{C_2 i}{\mu_i^{h_1}} \sum\limits_{m=1}^{i+1} m\int\limits_0^\infty 
\fr{\left(\mu_i^{h_1}\right)^m v^m}{m!}\,e^{-\mu_i^{h_1}v}\, dA(v)-{}\\
  {}-\fr{C_2 i(i+1)}{2\mu_i^{h_1}}\sum\limits_{m=i+2}^\infty 
\fr{\left(\mu_i^{h_1}\right)^m v^m}{m!}\,e^{-\mu_i^{h_1}v}\, dA(v)-{}\\
  {}-C_3\sum\limits^\infty_{m=i+2} (m-i-1)\int\limits_0^\infty
  \fr{\left(\mu_i^{h_1}\right)^m v^m}{m!}\,e^{-\mu_i^{h_1}v}\,dA(v)\,.
  \end{multline*}
  
  \vspace*{-3pt}
  
  Отсюда и~из выражения для $r^{h_1}_{i,m}$ следует~(\ref{e5-aga}) для 
состояния~$i$, так как при $0\hm\leq i\hm\leq h_2\hm-1$ сис\-те\-ма получает 
плату~$C_0$ и~не платит штраф, а~при $i\hm= h_2$ сис\-те\-ма не получает плату 
и~сама платит\linebreak штраф~$C_1$. 
  
  \smallskip
  
  \noindent
  \textbf{Лемма~2.} \textit{Справедливы равенства}:
  \begin{multline}
  d_{i+1}^{h_1+1} = d_i^{h_1} +C_3\sum\limits^\infty_{m=i+2} r^{h_1}_{i,m} -
\fr{C_2}{\nu_i^{h_1}} \sum\limits_{m=1}^{i+1} mr^{h_1}_{i,m} -{}\\
{}-
\fr{C_2(i+1)}{\mu_i^{h_1}} \sum\limits^\infty_{m=i+2} r^{h_1}_{i,m}\,,\enskip 0\leq 
i\leq h_2-1\,;\label{e10-aga}
  \end{multline}

\begin{equation}
\pi^{h_1+1}_{i+1} =\left( 1-\pi_0^{h_1+1}\right) \pi_i^{h_1}\,,\enskip 0\leq i\leq 
h_2\,.
\label{e11-aga}
\end{equation}
  
  \noindent
  Д\,о\,к\,а\,з\,а\,т\,е\,л\,ь\,с\,т\,в\,о\,.\ \ Сделав преобразования, 
из~(\ref{e6-aga}) получим~(\ref{e10-aga}):
  \begin{multline*}
  d_{i+1}^{h_1+1} -d_i^{h_1} = \fr{C_2}{2\mu_{i+1}^{h_1+1}} 
\sum\limits_{m=1}^{i+1} m(m-1) r^{h_1+1}_{i+1,m}+ {}\\
{}+
\fr{C_2}{2\mu_{i+1}^{h_1+1}} (i+2)(i+1) r^{h_1+1}_{i+1,i+2}-{}\\
  {}-  \fr{C_2(i+1)}{\mu_{i+1}^{h_1+1}} \sum\limits_{m=1}^{i+1} 
mr_{i+1,m}^{h_1+1} -{}\\
{}-\fr{C_2}{\mu_{i+1}^{h_1+1}}\left(i+1\right) (i+2) 
r^{h_1+1}_{i+2,i+2}-{}\\
  {}-  \fr{C_2(i+2)(i+1)}{2\mu_{i+1}^{h_1+1}} \sum\limits^\infty_{m=i+3} 
r^{h_1+1}_{i+1,m} -{}\\
{}-C_3\sum\limits^\infty_{m=i+3} (m-i-2) r_{i+1,m}^{h_1+1}-
- C_4^{h_1+1} (i+1) -{}\\
{}-\fr{C_2}{2\mu_i^{h_1}} \sum\limits_{m=1}^{i+1} m(m-1) r^{h_1}_{i,m} +
\fr{C_2(i+1)}{\mu_i^{h_1}} \sum\limits_{m=1}^{i+1} mr^{h_1}_{i,m} -{}\\
  {}- \fr{C_2}{\mu_i^{h_1}} \sum\limits_{m=1}^{i+1} mr^{h_1}_{i,m} +
  \fr{C_2 i (i+1)}{2\mu_i^{h_1}} \sum\limits^\infty_{m=i+3} r^{h_1}_{i,m} + {}\\
  {}+
\fr{C_2}{2\mu_i^{h_1}} \, i(i+1) r^{h_1}_{i,i+2}+ C_3\sum\limits^\infty_{m=i+3} (m-i-2) r^{h_1}_{i,m} 
+{}\\
{}+
C_3\sum\limits^\infty_{m=i+2} r^{h_1}_{i,m} +C_4^{h_1}(i)=
 -\fr{C_2}{\mu_i^{h_1}}\left(i+1\right) r^{h_1}_{i,i+2} -{}\\
 {}-\fr{C_2}{\mu_i^{h_1}}
\sum\limits_{m=1}^{i+1} mr^{h_1}_{i,m} -\fr{C_2(i+1)}{\mu_i^{h_1}} 
\sum\limits^\infty_{m=i+3} r^{h_1}_{i,m}+{}\\
  {}+ C_3\sum\limits^\infty_{m=i+2} r^{h_1}_{i,m} 
=C_3\sum\limits^\infty_{m=i+2} r^{h_1}_{i,m} -\fr{C_2}{\mu_i^{h_1}} 
\sum\limits_{m=1}^{i+1} mr^{h_1}_{i,m} -{}\\
{}-\fr{C_2(i+1)}{\mu_i^{h_1}} 
\sum\limits^\infty_{m=i+2} r^{h_1}_{i,m}\,.
  \end{multline*}
  
  Докажем~(\ref{e11-aga}). Подставив вместо $\pi_j^{h_1}$ 
и~$\pi_{j+1}^{h_1+1}$ соответствующие выражения из~(\ref{e5-aga}), получим 
  \begin{equation}
  \pi_j^{h_1} -\pi_{j+1}^{h_1+1} =\fr{R_j^{h_1}}{\sum\nolimits^{h_1}_{i=0} 
R_i^{h_1}} - \fr{R_{j+1}^{h_1+1}}{\sum\nolimits_{i=0}^{h_2+1} 
R_i^{h_1+1}}\,.
  \label{e12-aga}
  \end{equation}
  
  Из~(\ref{e4-aga}) следует $R_j^{h_1}\hm= R_{j+1}^{h_1+1}$, $0\hm\leq 
j\hm\leq h_2$. Тогда~(\ref{e12-aga}) приводится к~виду:
  \begin{multline*}
  \pi_j^{h_1}-\pi_{j+1}^{h_1+1} = \fr{R_j^{h_1} R_0^{h_1+1}} 
{\left(\sum\nolimits_{i=0}^{h_2} R_i^{h_1}\right) \left(\sum\nolimits^{h_2+1}_{i=0} 
R_i^{h_1+1}\right)} ={}\\
{}=\pi_j^{h_1} \pi_0^{h_1+1}\,.
  \end{multline*}
  
  Введем обозначения $B(h_1,a)$, $F(h_1,a)$ и~$G(h_1,a)$:
  \begin{equation}
\left.
\begin{array}{rl}
    \hspace*{-3.4mm}B(h_1,a) &=\displaystyle \fr{1}{\mu_1} \!\!\sum\limits_{i=0}^{h_1-1} 
\pi_i^{h_1}\!\sum\limits^\infty_{m=i+2} (m-i-1) r^{h_1}_{i,m} +{}\\[6pt]
  &\hspace*{-2mm}{}+\fr{1}{\mu_2} 
 \sum\limits_{i=h_1}^{h_2-1} \pi_i^{h_1} \sum\limits^\infty_{m=i+2} (m-i-1) r_{i,m}^{h_1} 
+{}\\[6pt]
  &\hspace*{6mm}{}+
\fr{1}{\mu_2}\,\pi^{h_1}_{h_2} \displaystyle \sum\limits^\infty_{m=h_2+1} (m-h_2) r^{h_1}_{h_2,m}\,;\\[6pt]
    \hspace*{-3.4mm}F(h_1,a)&=\fr{1-\pi_0^{h_1+1}}{\pi_0^{h_1+1}} \left[ \overline{v}-B(h_1,a)\right]\,;\\[9pt]
   \hspace*{-3.4mm} G(h_1,a)&=C_3 r^{h_1+1}_{0,0} -C_2 F(h_1a) +q_0^{h_1+1}\,,
   \end{array}
\!\!  \right\}\!
  \label{e13-aga}
  \end{equation}
  где
  $$ 
   q_0^{h_1+1} = C_0+C_3\left( 1-r^{h_1}_{0,0} -\mu_1\overline{v}\right)-
C_{1,4}\overline{v}\,.
 $$
  
  \noindent
  \textbf{Лемма~3.} \textit{Для любого $h_1\hm>0$ справедливо 
соотношение}:
  \begin{equation}
  g^{h_1}(a)-g^{h_1+1}(a)=\pi_0^{h_1+1} \left[ g^{h_1}(a)-G(h_1,a)\right]\,.
  \label{e14-aga}
  \end{equation}
  
  \noindent
  Д\,о\,к\,а\,з\,а\,т\,е\,л\,ь\,с\,т\,в\,о\,.\ \ Из формул~(\ref{e2-aga}),  
(\ref{e5-aga}), (\ref{e10-aga}) и~(\ref{e11-aga}) после не\-слож\-ных 
преобразований находим:
  \begin{multline*}
  g^{h_1}(a)-g^{h_1+1}(a)={}\\
  {}=\sum\limits_{i=0}^{h_2-1} \pi_i^{h_1} q_i^{h_1} 
+\pi^{h_1}_{h_2} q^{h_1}_{h_2}-
 \sum\limits_{i=0}^{h_2} \pi_i^{h_1+1} q_i^{h_1+1} - {}\\
 {}- \pi^{h_1+1}_{h_2+1} 
q_{h_2+1}^{h_1+1} = \sum\limits_{i=0}^{h_2-1} \pi_i^{h_1} q_i^{h_1} 
+\pi_{h_2}^{h_1} q_{h_2}^{h_1}-{}\\
  {}- \left( 1-\pi_0^{h_1+1}\right) \left( \sum\limits_{i=0}^{h_2-1} \pi_i^{h_1} 
q_{i+1}^{h_1+1} + \pi_{h_2}^{h_1} q^{h_1+1}_{h_2+1}\right) -{}\\
{}- \pi_0^{h_1+1}  q_0^{h_1+1}= \sum\limits_{i=0}^{h_2-1} \pi_i^{h_1} q_i^{h_1} +\pi_{h_2}^{h_1} 
q_{h_2}^{h_1} -{}\\
{}- \left( 1-\pi_0^{h_1+1}\right)  \left( \sum\limits_{i=0}^{h_2-1} \pi_i^{h_1} \left( q_i^{h_1} 
+d_{i+1}^{h_1+1} -d_i^{h_1}\right) +{}\right.\\
\left.{}+\pi_{h_2}^{h_1} \left( q^{h_1}_{h_2} 
+d_{h_2}^{h_1+1} -d^{h_1}_{h_2-1}\right)\!
\vphantom{\sum\limits_{i=0}^{h_2-1}}
\right)- \pi_0^{h_1+1} q_0^{h_1+1} ={}
\end{multline*}

\noindent
\begin{multline}
\hspace*{-6mm}{}= \pi_0^{h_1+1} \!\left\{ 
  g^{h_1} -\fr{1-\pi_0^{h_1+1}}{\pi_0^{h_1+1}}\!\left[  \sum\limits_{i=0}^{h_2-1} 
\!\pi_i^{h_1} \left( d_{i+1}^{h_1+1} -d_i^{h_1}\right)+{}\right.\right.\\
  \left.\left.{}+ \pi_{h_2}^{h_1} \left( d_{h_2}^{h_1+1}  -
d^{h_1}_{h_2-1}\right)\!
\vphantom{\sum\limits_{i=0}^{h_2-1}}
\right] -q_0^{h_1+1}\right\}={}\\
  {}= \pi_0^{h_1+1} \!\left\{ \!g^{h_1}- \fr{1-\pi_0^{h_1+1}} {\pi_0^{h_1+1}}
  \left[\sum\limits_{i=0}^{h_2-1} \!\pi_i^{h_1} \!\!\left[ C_3\!\sum\limits^\infty_{m=i+2}\!\! r^{h_1}_{i,m} -{}\right.\right.\right.\\
\left.  {}-\fr{C_2}{\mu_i^{h_1}} 
\sum\limits_{m=1}^{i+1} mr^{h_1}_{i,m} -
\fr{C_2(i+1)}{\mu_i^{h_1}}\sum\limits^\infty_{m=i+2} r^{h_1}_{i,m}\right]+{}\\
{}+
  \pi^{h_1}_{h_2} \left[ C_3 \sum\limits^\infty_{m=h_2+1} r^{h_1}_{i,m} -
\fr{C_2}{\mu^{h_1}_{h_2}} \sum\limits^{h_2}_{m=1} mr^{h_1}_{i,m} -{}\right.\\
\left.\left.\left.{}-\fr{C_2h_2}{\mu_{h_2}^{h_1}} \sum\limits^\infty_{m=h_2+1} 
r^{h_1}_{i,m}\right]\right] -q_0^{h_1+1}\right\}.
  \label{e15-aga}
  \end{multline}
  
  \vspace*{-3pt}
  
  Из первого равенства в~(\ref{e3-aga}) следует 
  $$
  \pi_0^{h_1+1} =\sum\limits_{i=0}^{h_2} \pi_i^{h_1+1} 
\sum\limits^\infty_{m=i+1} r_{i,m}^{h_1+1} +\pi_{h_2+1}^{h_1+1} 
\sum\limits^\infty_{m=h_2+1} r^{h_1+1}_{h_2,m}\,.
  $$
  
  Отсюда и~из~(\ref{e11-aga}) имеем:
  
  \vspace*{-3pt}
  
  \noindent
  \begin{multline}
  \sum\limits_{i=0}^{h_2-1} \pi_i^{h_1} \sum\limits^\infty_{m=i+2} 
r^{h_1}_{i,m} +\pi^{h_1}_{h_2} \sum\limits^\infty_{m=h_2+1} r^{h_1}_{h_2,m} 
= {}\\
{}=\left[ r_{0,0}^{h_1+1} \fr{1-\pi_0^{h_1+1}}{\pi_0^{h_1+1}}\right]^{-1}\,.
  \label{e16-aga}
  \end{multline}
  
  \vspace*{-3pt}
  
  \noindent
Заменив в~(\ref{e15-aga}) $(1\hm- \pi_0^{h_1+1})/\pi_0^{h_1+1}$ на правую 
часть равенства~(\ref{e16-aga}), получим

\vspace*{-3pt}

\noindent
\begin{multline*}
g^{h_1}(a) -g^{h_1+1}(a) ={}\\
{}=\pi_0^{h_1+1} \left\{ 
g^{h_1}(a) -\left[ 
\vphantom{\sum\limits^\infty_{m=h_2+1}}
C_3 r^{h_1+1}_{0,0} -C_2 \fr{1-\pi_0^{h_1+1}}{\pi_0^{h_1+1}}\left[ 
\vphantom{\sum\limits^\infty_{m=h_2+1}}
\overline{v} -{}\right.\right.\right.\\
{}-
\sum\limits_{i=0}^{h_2-1} \fr{1}{\mu_i^{h_1}}\,\pi_i^{h_1} 
\sum\limits^\infty_{m=i+2} (m-i-1) r^{h_1}_{i,m}-{} \\
\left.\left.\left.{}- \fr{1}{\mu^{h_1}_{h_2}} \,\pi^{h_1}_{h_2} 
\sum\limits^\infty_{m=h_2+1} (m-h_2) r^{h_1}_{h_2,m} \right]\right] - 
q_0^{h_1+1}\right\}\,.
\end{multline*}

\vspace*{-3pt}

\noindent
Подставив в~последнее выражение со\-от\-вет\-ст\-ву\-ющие обозначения  
из~(\ref{e13-aga}), получим доказательство леммы~3.

\smallskip

\noindent
\textbf{Лемма~4.} \textit{Функция $G(h_1,a)$ не возрастает по} 
$h_1\hm>0$.

\smallskip

\noindent
Д\,о\,к\,а\,з\,а\,т\,е\,л\,ь\,с\,т\,в\,о\,.\ \ Покажем, что $F(h_1,a)$ воз\-рас\-та\-ет 
по~$h_1$. Рассмотрим вспомогательную сис\-те\-му с~ограниченным накопителем 
емкости $h_2\hm+1$, в~которой одно из мест накопителя отмечено мет\-кой 
и~заявка, занявшая это место, также отмечается меткой. Предположим, что 
в~этой сис\-те\-ме, в~отли-\linebreak\vspace*{-12pt}

\columnbreak

\noindent
чие от исходной сис\-те\-мы, используется  
сле\-ду\-ющая дисциплина обслуживания. Поступившая заявка, если 
в~накопителе есть свободное место, в~первую очередь занимает немеченое 
место, а~при отсутствии такого занимает меченое место и~становится меченой 
заявкой. Если в~момент по\-ступ\-ле\-ния заявки на приборе находится меченая 
заявка, то эта заявка возвращается в~накопитель на меченое мес\-то, освободив 
прибор, а~прибор занимает по\-сту\-пив\-шая заявка. Немеченые заявки 
обслуживаются в~порядке по\-ступ\-ле\-ния, а~меченая заявка занимает прибор 
только при отсутствии в~накопителе немеченых заявок. Предполагается, что 
в~остальном вспомогательная сис\-те\-ма пол\-ностью совпадает с~исходной 
СМО с~ограниченным накопителем ем\-кости $h_2\hm+1$. 

  Обратим внимание, что процесс обслуживания немеченых заявок во 
вспомогательной СМО происходит точ\-но так же, как и~в~исходной СМО 
с~накопителем ем\-кости $h_2\hm>0$, и~если со\-сто\-яни\-ем вложенной цепи 
вспомогательной СМО считать чис\-ло немеченых заявок в~сис\-те\-ме в~момент 
по\-ступ\-ле\-ния заявки, то стационарные распределения вероятностей  
со\-сто\-яний для обеих сис\-тем совпадут. Обратим также внимание на то, что, 
не\-смот\-ря на разный порядок обслуживания в~исходной и~вспомогательной 
СМО, если для вложенной цепи вспомогательной СМО со\-сто\-яни\-ем 
является общее чис\-ло заявок в~момент поступления, то распределение 
стационарных вероятностей со\-сто\-яний для вспомогательной сис\-те\-мы 
совпадает со стационарными вероятностями исходной с~накопителем ем\-кости 
$h_2\hm+1$. Это объясняется тем, что в~этих сис\-те\-мах время обслуживания 
заявки прибором (так как распределение вероятностей времени  
обслуживания~--- экспоненциальное), поведение входного потока и~процедура 
приема заявок в~сис\-те\-му не зависят от дисциплины обслуживания 
в~со\-от\-вет\-ст\-ву\-ющих сис\-те\-мах.  
  
  Обратим также внимание на то, что в~сис\-те\-ме в~любой момент времени 
меченых заявок максимум одна и~она может находиться на приборе только при 
отсутствии в~накопителе немеченых заявок. Очевидно, ве\-ро\-ят\-ность 
на\-хож\-де\-ния в~произвольный момент времени на приборе вспомогательной 
сис\-те\-мы меченой 
заявки является положительной величиной. Тогда во вспомогательной сис\-те\-ме 
событие <<в~произвольный момент времени в~сис\-те\-ме немеченых заявок не 
меньше $l\hm>0$>> влечет событие <<в~произвольный момент времени 
в~сис\-те\-ме заявок не меньше $l\hm>0$>> и~при этом ве\-ро\-ят\-ность первого 
события меньше ве\-ро\-ят\-ности второго. Так как вероятность первого события 
рав\-на 

\noindent
$$
Q_l^{h_1}\hm= \sum\limits^{h_2}_{i=l} \pi_i^{h_1},
$$

\noindent
 а~ве\-ро\-ят\-ность второго рав\-на 
$$
Q_l^{h_1+1} = \sum\limits_{i=l}^{h_2+1} \pi_i^{h_1+1},
$$ 
то выполняется неравенство 
$$
Q_l^{h_1+1}> Q_l^{h_1},\enskip l>0\,.
$$
 Тогда 
имеем 
$$
\sum\limits_{i=l}^{h_2+1} Q_i^{h_1+1}> \sum\limits^{h_2}_{i=l} 
Q_i^{h_1},
$$
 т.\,е.\ сред\-нее чис\-ло заявок в~сис\-те\-ме в~момент по\-ступ\-ле\-ния 
новой заявки при пороге $h_1\hm+1$ больше, чем при~$h_1$. Как следует 
  из~(\ref{e9-aga}) и~(\ref{e13-aga}), $B(h_1,a)$~--- сред\-нее время простоя 
прибора. Так как из приведенных выше рассуждений следует, что среднее 
значение СВ времени обслуживания всех заявок (включая 
и~по\-сту\-пив\-шую) за время на\-хож\-де\-ния в~произвольном со\-сто\-янии воз\-рас\-та\-ет 
по~$h_1$, получаем, что сред\-нее время простоя  $B(h_1,a)$ убывает по~$h_1$. 
Так как $\pi_0^{h_1}$ убывает по~$h_1$ (выше показано, что 
$Q_1^{h_1+1}\hm> Q_1^{h_1}$), то $F(h_1,a)$ возрастает по~$h_1$. Тогда из 
выражения для $G(h_1,a)$ в~(\ref{e13-aga}) следует доказательство леммы~4. 
  
  Пусть $h_1^*$~--- решение задачи~(\ref{e1-aga}) для вспомогательной 
сис\-те\-мы. Справедлива сле\-ду\-ющая тео\-рема.
  
  \smallskip
  
  \noindent
  \textbf{Теорема~1.}\ \textit{Справедливы утверж\-де\-ния}: 
  \begin{enumerate}[(1)]
  \item \textit{существует порог}  $h_1^*\hm<\infty$, 
  \textit{если  $\mathop{\mathrm{\inf}}\limits_{h_1>0} G(h_1,a)\hm< 
  \mathop{\mathrm{sup}}\limits_{h_1>0} g^{h_1}(a)$  и~$C_2\hm>0$}; 
 \item  $h_1^*\hm=\infty$, \textit{если} $g^1(a)\hm< G(1,a)$ и~$C_2\hm=0$; 
\item $h_1^*\hm=1$, \textit{если} $g^1(a)\hm\geq G(1,a)$; 
\item \textit{для существования $0\hm< h_1^*\hm< \infty$
  необходимо и~достаточно выполнение условий} 
  $g^{h_1^*-1}(a)\hm< g^{h_1^*}(a)$ и~$g^{h_1^*+1}(a)\hm\geq g^{h_1^*}(a)$.
  \end{enumerate}
  
%  \smallskip
  
  \noindent
  Д\,о\,к\,а\,з\,а\,т\,е\,л\,ь\,с\,т\,в\,о\,.\ \ Функция $g^{h_1}(a)$, 
$h_1\hm>0$, удовле\-тво\-ря\-ет всем условиям тео\-ре\-мы в~работе~\cite{9-aga} 
и~поэтому является унимодальной, что доказывает существование решения 
$h_1^*$ задачи~(\ref{e1-aga}) для вспомогательной сис\-те\-мы и~утверж\-де\-ние~4 
тео\-ре\-мы~1. Утверж\-де\-ние~1 тео\-ре\-мы~1 (существование порога 
$h_1^*\hm<\infty$) следует из леммы~3 и~леммы~4, утверж\-де\-ние~2 следует из 
леммы~4 и~неравенства $G(h_1,a)\hm> g^{h_1}(a)$ для всех $h_1\hm>0$ при 
$C_2\hm=0$. Утверж\-де\-ние~3 доказываемой тео\-ре\-мы вытекает из лемм~3 и~4, 
так как из леммы~3 следует равенство 
\begin{multline*}
  g^{h_1+1}(a) -g^{h_1+2}(a) ={}\\
  {}=\pi_0^{h_1+2}\left[ \left( 1-\pi_0^{h_1+1}\right) 
\left( g^{h_1}(a)-G\left(h_1+1,a\right)\right)+{}\right.\\
\left.{}+ \pi_0^{h_1+1} \left( G(h_1,a)-
G(h_1+1,a)\right)\right]\,,
\end{multline*}

{ \begin{center}  %fig1
 \vspace*{-1pt}
   \mbox{%
\epsfxsize=78.211mm
\epsfbox{Aga-1.eps}
}

\end{center}

\noindent
{\small
Зависимости функций $g^{h_1}(a)/\overline{v}$~(\textit{1}) и~$G(h_1,a)/\overline{v}$~(\textit{2}) от 
порогового значения~$h_1$
}
}

\vspace*{9pt}


\noindent
  а из этого равенства, леммы~4 и~неравенства $g^1(a)\hm\geq G(1,a)$ по 
индукции следует $g^{h_1+1}(a)\hm- g^{h_1+2}(a)\hm>0$ для всех $h_1\hm>0$, 
и,~следовательно, утверж\-де\-ние~3 тео\-ре\-мы~1 справедливо.
  
  Рассмотрим теперь случай $a\hm= \infty$ (СМО с~неограниченной оче\-редью) 
при условии $\mu_2/\overline{v}\hm< 1$ и~$C_2\hm>0$ (при $C_2\hm=0$, 
очевидно, $h_1^*\hm=\infty$, в~случае $\mu_2/\overline{v}\hm\geq 1$, 
$C_2\hm>0$ оче\-редь бесконечна и~$g^{h_1}\hm=-\infty$). Так как для любого 
конечного~$h_1$ по\-сле\-до\-ва\-тель\-ности $g^{h_1}(a)$ и~$G(h_1,a)$ являются 
сходящимися при $a\hm\to \infty$, то, очевидно, тео\-ре\-ма~1 справедлива и~для 
СМО с~неограниченной оче\-редью.

 
  
  На рисунке проиллюстрировано отношение эк\-ви\-ва\-лент\-ности условий 
$g^{h_1+1} (a)\hm> g^{h_1}(a)$ и~$g^{h_1}(a)< G(h_1,a)$, яв\-ля\-юще\-еся 
следствием соотношения~(\ref{e14-aga}), и~поведение функций $g^{h_1}(a)$ 
и~$G(h_1,a)$ при изменении значения порога~$h_1$ 
     для сле\-ду\-ющих исходных 
данных:  $A(t)\hm= f_1(1\hm- e^{-\lambda_1 t})\hm+ f_2(1\hm- e^{-\lambda_2t})$;
$f_i\hm> 0$; $\lambda_i\hm>0$; $i\hm= 1,2$; $f_1\hm+ f_2\hm= 1$; 
$f_1\hm= 0{,}3$; $f_2\hm= 0{,}7$; $\lambda_1\hm= 2$; $\lambda_2\hm= 3$; $a\hm=30$; 
$C_0\hm= 10$; $C_1\hm= 0$; $C_2\hm= 0{,}1$; $C_3\hm= 5$; $C_{1,4}\hm= 10$; $C_{2,4}\hm= 20$;
$\mu_1\hm=2$; $\mu_2\hm= 4$.
 
  
\section{Заключение}

  Из полученных результатов следуют выводы:
  \begin{itemize}
\item целевая функция задачи~(\ref{e1-aga}) и~для вспомогательной, и~для 
основ\-ной СМО (при $\mu_2/\overline{v}\hm<1$) унимодальна по~$h_1$;
\item при $C_2>0$ существует оптимальная точ\-ка переключения ско\-рости 
обслуживания (конечный оптимальный порог длины очереди);
\item при $C_2>0$ для поиска оптимальной точ\-ки переключения $h_1\hm= 
h_1^*\hm<\infty$ достаточно \mbox{найти} $h_1\hm\geq1$, при котором 
выполняются условия $g^{h_1-1}\hm< g^{h_1}$ и~$g^{h_1+1}\hm\geq 
g^{h_1}$.
\end{itemize}

  Результаты данной работы могут быть использованы для оценки технической 
  и~экономической эф\-фек\-тив\-ности реальных сис\-тем, для которых в~качестве 
моделей используют СМО типа $G/M/1$ ($G/M/1/r$) с~управ\-ле\-ни\-ем 
переключением ско\-рости обслуживания. 

{\small\frenchspacing
 {%\baselineskip=10.8pt
 %\addcontentsline{toc}{section}{References}
 \begin{thebibliography}{9}
\bibitem{1-aga}
\Au{Агаларов Я.\,М., Ушаков~В.\,Г.} Об унимодальности функции дохода сис\-те\-мы массового 
обслуживания типа $G/M/s$ с~управ\-ля\-емой оче\-редью~// Информатика и~её применения, 
2019. Т.~13. Вып.~1. С.~55--61.
\bibitem{2-aga}
\Au{Агаларов Я.\,М., Коновалов~М.\,Г.} Доказательство уни\-мо\-даль\-ности целевой функции 
в~задаче порогового управ\-ле\-ния нагрузкой на сервер~// Информатика и~её применения, 2019. 
Т.~13. Вып.~2. С.~2--6.
\bibitem{3-aga}
\Au{Агаларов Я.\,М.} Оптимальное пороговое управ\-ле\-ние доступом в~сис\-те\-ме $M/M/s$ 
с~неоднородными приборами и~общим накопителем~// Информатика и~её применения, 2021. 
Т.~15. Вып.~1. С.~57--64.

\bibitem{8-aga} %4
\Au{Карлин С.} Основы тео\-рии случайных процессов~/ Пер. с~англ.~--- М.: Мир, 1971. 537~с. 
(\Au{Karlin~S.} A~first course in stochastic processes.~--- New York, NY, USA: Academic Press, 
1968. 502~p.)

\bibitem{4-aga} %5
\Au{Горцев А.\,М.} Сис\-те\-ма массового обслуживания с~произвольным чис\-лом резервных 
каналов и~гистерезисным управ\-ле\-ни\-ем включением и~выключением резервных каналов~// 
Автоматика и~телемеханика, 1977. №\,10. С.~30--37.
\bibitem{5-aga} %6
\Au{Зиновьева Л.\,И., Терпугов~А.\,Ф.} Однолинейная сис\-те\-ма массового обслуживания 
с~переменной ин\-тен\-сив\-ностью обслуживания, зависящей от времени ожидания~// 
Автоматика и~телемеханика, 1981. №\,1. С.~27--30. 
\bibitem{6-aga} %7
\Au{Самочернова Е.\,С., Петров~Л.\,И.} Оптимизация сис\-те\-мы массового обслуживания 
с~однотипным ре\-зерв\-ным прибором~// Известия Томского политехнического университета, 
2010. Т.~317. №\,5. С.~28--31.
\bibitem{7-aga} %8
\Au{Крылова Д.\,С., Головко~Н.\,И., Жук~Т.\,А.} Анализ СМО с~ре\-зерв\-ным прибором 
и~скачкообразной ин\-тен\-сив\-ностью входного потока~// Вестник ВГУ. Сер. Физика, математика, 
2017. №\,4. С.~109--123.

\bibitem{9-aga}
\Au{Агаларов Я.\,М.} Признак уни\-мо\-даль\-ности це\-ло\-чис\-лен\-ной функ\-ции одной переменной~// 
Обозрение при\-клад\-ной и~промышленной математики, 2019. Т.~26. Вып.~1. С.~65--66.
\end{thebibliography}

 }
 }

\end{multicols}

\vspace*{-6pt}

\hfill{\small\textit{Поступила в~редакцию 08.11.21}}

\vspace*{8pt}

%\pagebreak

%\newpage

%\vspace*{-28pt}

\hrule

\vspace*{2pt}

\hrule

%\vspace*{-2pt}

\def\tit{OPTIMIZATION OF~THE~THRESHOLD SERVICE SPEED CONTROL IN~THE~$G/M/1$ 
QUEUE}


\def\titkol{Optimization of~the~threshold service speed control in~the~$G/M/1$ 
queue}


\def\aut{Ya.\,M.~Agalarov}

\def\autkol{Ya.\,M.~Agalarov}

\titel{\tit}{\aut}{\autkol}{\titkol}

\vspace*{-11pt}


\noindent
  Federal Research Center ``Computer Science and Control'' of the Russian Academy of Sciences, 
44-2~Vavilov Str., Moscow 119333, Russian Federation

\def\leftfootline{\small{\textbf{\thepage}
\hfill INFORMATIKA I EE PRIMENENIYA~--- INFORMATICS AND
APPLICATIONS\ \ \ 2022\ \ \ volume~16\ \ \ issue\ 1}
}%
 \def\rightfootline{\small{INFORMATIKA I EE PRIMENENIYA~---
INFORMATICS AND APPLICATIONS\ \ \ 2022\ \ \ volume~16\ \ \ issue\ 1
\hfill \textbf{\thepage}}}

\vspace*{3pt} 
  
  
  
  \Abste{Consideration is given to the problem of optimal service speed switching in a $G/M/1$ 
queuing system with an objective function which takes into account the fee for customer's service, 
the penalty for customer's delay in the queue, the penalty for customer's rejection, the penalty for 
server being idle, and the maintenance costs. The case of finite capacity queue is also considered. 
The nonlinear optimization problem is formulated and solved in which the objective function is the 
system's revenue and the control variable is the queue length. The author proves that the objective 
function is unimodal and formulates the conditions for the existence of an optimal threshold service 
speed switching and necessary and sufficient conditions for the optimal control.}
  
  \KWE{queuing system; threshold control; system costs}
  
  
  
\DOI{10.14357/19922264220111}

%\vspace*{-16pt}

\Ack
  \noindent
  The reported study was partly funded by RFBR, project number 20-07-00804.




%\vspace*{6pt}

  \begin{multicols}{2}

\renewcommand{\bibname}{\protect\rmfamily References}
%\renewcommand{\bibname}{\large\protect\rm References}

{\small\frenchspacing
 {%\baselineskip=10.8pt
 \addcontentsline{toc}{section}{References}
 \begin{thebibliography}{9}
\bibitem{1-aga-1}
  \Aue{Agalarov, Ya.\,M., and V.\,G.~Ushakov.} 2019. Ob unimodal'nosti funktsii dokhoda 
  sistemy massovogo obsluzhivaniya 
tipa $G/M/s$ s~upravlyaemoy ochered'yu [On the unimodality of the income function of a type 
$G/M/s$ queueing system with controlled queue]. \textit{Informatika i~ee primeneniya~--- Inform. 
Appl.} 13(1):55--61. 
\bibitem{2-aga-1}
\Aue{Agalarov, Ya.\,M., and M.\,G.~Konovalov.} 2019. Dokazatel'stvo unimodal'nosti tselevoy 
funktsii v~zadache porogovogo upravleniya nagruzkoy na server [Proof of the unimodality of the 
objective function in $M/M/N$ queue with threshold-based congestion control]. \textit{Informatika 
i~ee primeneniya~--- Inform. Appl.} 13(2):2--6.
\bibitem{3-aga-1}
  \Aue{Agalarov, Ya.\,M.} 2021. Optimal'noe porogovoe upravlenie dostupom v~sisteme 
$M/M/s$ s~neodnorodnymi priborami i~obshchim nakopitelem [Optimal threshold-based 
admission control in the $M/M/s$ system with heterogeneous servers and a~common queue]. 
\textit{Informatika i~ee primeneniya~--- Inform. Appl.} 15(1):57--64.

\bibitem{8-aga-1} %4
  \Aue{Karlin, S.} 1968. \textit{A~first course in stochastic processes.} New York, NY:  
Academic Press. 502~p.
\bibitem{4-aga-1} %5
  \Aue{Gortsev, A.\,M.} 1977. Sistema massovogo obsluzhivaniya s~proizvol'nym chislom 
rezervnykh kanalov i~gisterezisnym upravleniem vklyucheniem i~vyklyucheniem rezervnykh 
kanalov [A~queueing system with an arbitrary number of stand-by channels and hysteresis control 
of their connection and disconnection]. \textit{Automat. Rem. 
Contr.} 10:30--37.
\bibitem{5-aga-1} %6
  \Aue{Zinov'eva, L.\,I., and A.\,F.~Terpugov.} 1981. Odnolineynaya sistema massovogo 
obsluzhivaniya s~peremennoy intensivnost'yu, zavisyashchey ot vremeni ozhidaniya [A~single 
flow service system whose throughput depends on the queueing time]. \textit{Automat. Rem. Contr.} 1:27--30. 
\bibitem{6-aga-1} %7
  \Aue{Samochernova, E.\,S., and L.\,I.~Petrov.} 2010. Op\-ti\-mi\-za\-tsiya sistemy massovogo 
obsluzhivaniya s~od\-no\-tip\-nym re\-zerv\-nym priborom [Optimization of the queuing system with the 
same type of backup device]. \textit{Bulletin 
of the Tomsk Polytechnic University} 317(5):28--31.
\bibitem{7-aga-1} %8
  \Aue{Krylova, D.\,S., N.\,I.~Golovko, and T.\,A.~Zhuk.} 2017. Analiz SMO s~rezervnym 
priborom i~skachkoobraznoy intensivnost'yu vkhodnogo potoka [Analysis of the queueing system 
with backup server and abrupt intensity of the input stream]. \textit{Vestnik VGU. Ser. Fizika, 
matematika} [Proceedings of Voronezh State University. Ser. Physics. Mathematics] 4:109--123.

\bibitem{9-aga-1}
  \Aue{Agalarov, Ya.\,M.} 2019. Priznak unimodal'nosti tselochislennoy funktsii odnoy 
peremennoy [A sign of unimodality of an integer function of one variable]. \textit{Obozrenie 
prikladnoy i~promyshlennoy matematiki} [Surveys Applied and Industrial Mathematics]  
26(1):65--66.
\end{thebibliography}

 }
 }

\end{multicols}

\vspace*{-6pt}

\hfill{\small\textit{Received November 8, 2021}}

%\pagebreak

%\vspace*{-18pt}
  
  \Contrl
  
  \noindent
  \textbf{Agalarov Yaver M.} (b.\ 1952)~--- Candidate of Science (PhD) in technology, associate 
professor, leading scientist, Institute of Informatics Problems, Federal Research Center ``Computer 
Science and Control'' of the Russian Academy of Sciences, 44-2~Vavilov Str., Moscow 119333, 
Russian Federation; \mbox{agglar@yandex.ru}
  
  

\label{end\stat}

\renewcommand{\bibname}{\protect\rm Литература} 

   