\def\stat{torshin}

\def\tit{О ПРИМЕНЕНИИ ТОПОЛОГИЧЕСКОГО ПОДХОДА К~АНАЛИЗУ ПЛОХО 
ФОРМАЛИЗУЕМЫХ ЗАДАЧ ДЛЯ~ПОСТРОЕНИЯ АЛГОРИТМОВ ВИРТУАЛЬНОГО 
СКРИНИНГА КВАНТОВО-МЕХАНИЧЕСКИХ СВОЙСТВ\\ ОРГАНИЧЕСКИХ МОЛЕКУЛ~I: ОСНОВЫ
 ПРОБЛЕМНО ОРИЕНТИРОВАННОЙ ТЕОРИИ$^*$}

\def\titkol{О применении топологического подхода к~анализу плохо 
формализуемых задач для~построения алгоритмов~I} %виртуального 
%скрининга}
% квантово-механических свойств органических молекул.  Часть~1: Основы проблемно ориентированной теории}

\def\aut{И.\,Ю.~Торшин$^1$}

\def\autkol{И.\,Ю.~Торшин}

\titel{\tit}{\aut}{\autkol}{\titkol}

\index{Торшин И.\,Ю.}
\index{Torshin I.\,Yu.}


{\renewcommand{\thefootnote}{\fnsymbol{footnote}} \footnotetext[1]
{Работа выполнена при поддержке РФФИ (проекты 19-07-00356, 18-07-00944, 20-07-00537).}}


\renewcommand{\thefootnote}{\arabic{footnote}}
\footnotetext[1]{Федеральный исследовательский центр <<Информатика и~управление>> Российской академии наук, 
\mbox{tiy135@yahoo.com}}

%\vspace*{-6pt}



      \Abst{Топологический подход к~анализу плохо формализованных задач и~теория 
хемографов являются расширениями алгебраического подхода к~распознаванию, 
развиваемого в~научной школе академика РАН Ю.\,И.~Журавлёва. В~первой части статьи 
предложен проблемно ориентированный формализм для разработки алгоритмов 
скрининговых оценок кван\-то\-во-ме\-ха\-ни\-че\-ских (КМ) свойств молекул по их химической 
структуре. Предложены способы введения метрик на множествах молекул и~процедуры 
порождения <<синтетических>> признаковых описаний, основанные на согласовании 
значений <<экспертной>> метрики на множестве значений свойств молекул со значениями 
настраиваемой метрики на множестве структур молекул. }
      
      \KW{алгебраический подход; хемоинформатика; размеченные графы; комбинаторный 
анализ разрешимости}

\DOI{10.14357/19922264220106}
  
\vspace*{12pt}


\vskip 10pt plus 9pt minus 6pt

\thispagestyle{headings}

\begin{multicols}{2}

\label{st\stat}
      
\section{Введение. Проблемная область}

     Алгебраический подход к~анализу проблем распознавания, 
классификации включает формулировку и~исследование критериев 
разрешимости и~регулярности задач~[1--3], корректности и~полноты моделей 
алгоритмов~[4--6]. Исследуются алгоритмы в~виде композиций 
$A\hm= {B}\circ {C}\circ {D}$, где 
${B}$~--- распознающий оператор; ${C}$~--- кор\-рек\-ти\-ру\-ющая 
операция (корректор); ${D}$~--- решающее правило. Исходно в~рамках 
алгебраического подхода наибольшее внимание уделяется именно 
формулировке и~исследованию критериев разрешимости и~регулярности задач 
рас\-по\-зна\-ва\-ния/клас\-си\-фи\-ка\-ции, корректности и~полноты моделей 
алгоритмов~[1]. 
     
     Исследование выполнимости этих критериев и~<<обучение>> алгоритмов 
подразумевает, что задано формальное описание: \textit{матрица информации} 
(набор признаковых описаний объектов) и~\textit{информационная матрица} 
(отнесение объектов к~определенным классам). В~то же время во многих 
прикладных областях (биология, химия, медицина) встречаются задачи, 
формальные описания которых могут быть получены очень многими 
спо\-со\-бами. 
{ %\looseness=1

}

     
     Прикладные задачи, для которых не существует однозначного метода 
выделения объектов, определения признаковых описаний и~классов объектов на 
основе имеющегося <<исходного описания>> считаются \textit{плохо 
формализуемыми}~\cite{6-tr}. Адекватная формализация задач позволяет 
улучшить аккуратность и~обобщающую способность со\-от\-вет\-ст\-ву\-ющих 
алгоритмов распознавания. С~этой целью в~русле алгебраического подхода 
разрабатываются топологическая и~метрическая тео\-рии анализа данных~[\mbox{7--9}].
{\looseness=1

} 
     
     К плохо формализуемым относятся и~рас\-смат\-ри\-ва\-емые в~настоящей 
работе задачи типа <<струк\-ту\-ра--свой\-ст\-во>>, где существенную 
неоднозначность представляют собой способы порождения признаковых 
описаний молекул. 

В~статье пред\-став\-ле\-ны результаты совместного 
применения топологического подхода к~плохо формализованным 
задачам~\cite{6-tr} и~методов теории анализа хемографов~\cite{7-tr, 8-tr} для 
разработки алгоритмов виртуального скрининга  
КМ свойств молекул.

\section{Введение. О~виртуальных скринингах молекул}

     Квантовая механика~--- один из ярких примеров успешного 
применения теории вероятностей, теории операторов, теории групп 
и~функционального анализа в~теоретической физике. Тем не менее точное 
аналитическое решение уравнения Шредингера имеется только для атома 
водорода, а~высокоточные полуэмпирические схемы КМ-рас\-че\-тов 
вычислительно затратны. В~то же время задачи поиска перспективных 
материалов и~лекарств подразумевают проведение скринингов пространства 
всех возможных <<малых>> органических молекул 
(более~10$^{11}$~структур~\cite{10-tr}).
     
     Для осуществления первоначальных стадий таких виртуальных 
скринингов необходимы алгоритмы для оценочных расчетов КМ-свойств 
молекул, позволяющие выделять подмножества \mbox{молекул} с~экстремальными 
значениями КМ-свойств. Практически важны (а)~точ\-ность, (б)~обоб\-ща\-ющая 
способность, (в)~производительность вычислений и~(г)~интерпретируемость 
таких алгоритмов. В~случае оценочных расчетов КМ-свойств молекул 
необходимо особо акцентировать интерпретируемость таких алгоритмов 
в~терминах теории химической связи на уровне, доступном для химиков-прак\-ти\-ков. 
Одно из возможных решений~--- алгоритмы, построенные на 
основании теории хемографов в~рамках топологического подхода к~анализу 
данных.

\vspace*{-6pt}
     
\section{Основы топологической теории анализа данных}

\vspace*{-2pt}

     Алгебраический подход требует определения множества \textit{начальных 
информаций} (I$_i$) и~множества \textit{конечных информаций} 
 (I$_f$)~\cite{1-tr}. Исследуемые алгоритмы ${A}(\theta): 
\mathrm{I}_i\hm\to \mathrm{I}_f$ ($\theta$~--- вектор внут\-рен\-них  
па\-ра\-мет\-ров) <<обучаются>> на основе конечных множеств прецедентов 
$\mathrm{Pr}\subset \mathrm{I}_i \times \mathrm{I}_f$. Топологическая теория 
анализа данных позволяет проводить систематические исследования различных 
способов определения множеств~$\mathrm{I}_i$ и~$\mathrm{I}_f$ и~так 
называемого множества оценок (об\-ласть значений распознающего 
оператора~${B}$).

     
     Пусть $\mathbf{X}\hm= \{ x_1, x_2, \ldots , x_\alpha, \ldots , x_{N_0}\}$~--- 
множество исходных описаний объектов, $\mathbf{X}\hm\subseteq 
{S}$, где ${S}$~--- пространство таких описаний. 
Пусть~${J}_{\mathrm{ob}}$~--- пространство допустимых 
признаковых описаний\linebreak объектов и~определена функция~${D}: 
{S}\hm\to {J}_{\mathrm{ob}}$. Определим 
${J}_{\mathrm{ob}}\hm\subseteq \mathrm{I}_1\times 
\mathrm{I}_2\times \cdots \times \mathrm{I}_k\times \cdots \times 
\mathrm{I}_{n+1}$, $\mathrm{I}_i\hm\subseteq \mathrm{I}_1\times 
\mathrm{I}_2\times\cdots \times \mathrm{I}_k\times \cdots \times \mathrm{I}_n$ 
и~$\mathrm{I}_f\hm\subseteq \mathrm{I}_{n+1} \times 
\mathrm{I}_{n+2}\times\cdots\times \mathrm{I}_{n+l}$ на основе 
$\mathrm{I}_k\hm= \{ \lambda_{k_1},\lambda_{k_2}, \ldots, \lambda_{k_b}, 
\ldots , \lambda_{k_{\vert \mathrm{I}_k\vert -1}}, \Delta\}$~--- множеств 
значений $k$-х компонент признакового описания, $k\hm= 1,\ldots , n\hm+l$, 
где~$\Delta$~--- неопределенность; $n$~--- чис\-ло признаков; $l$~--- число 
таргетных (прогнозируемых) переменных. 
     
     Формализация задачи, т.\,е.\ переход от множе\-ст\-ва 
$\mathbf{X}\hm\subseteq {S}$ к~множеству прецедентов 
$\mathbf{Q}\hm\subseteq {J}_{\mathrm{ob}}$,\linebreak заключается 
в~определении таких функций $\Gamma_k: {S}\hm\to \mathrm{I}_k$, 
$k\hm= 1,\ldots ,n \hm+l$, что ${D}(x_\alpha) \hm= \left( \Gamma_1(x_\alpha)\times 
\cdots \times \Gamma_k(x_\alpha)\times \cdots \times 
\Gamma_{n+l}(x_\alpha)\right)_\Delta$~\cite{5-tr, 6-tr}, $\mathrm{m}_\alpha\hm= \left( 
\Gamma_1(x_\alpha)\times \cdots \times \Gamma_n(x_\alpha)\right)_\Delta$~--- 
\textit{вектор значений признаков}, а~$t_\alpha\hm= \left( \Gamma_{n+1}(x_\alpha) \times 
\cdots\times \Gamma_{n+l}(x_\alpha)\right)_\Delta$~--- \textit{вектор значений}~$l$ 
\textit{таргетных переменных}, обра\-зу\-ющих \textit{матрицу информации} 
$\mathbf{M}_{\mathrm{I}}\hm= (m_\alpha)$, $m_\alpha\hm\in \mathrm{I}_i$, 
и~\textit{информационную матрицу} $\mathbf{M}_{\mathrm{F}}\hm= (t_\alpha)$, 
$t_\alpha\hm\in \mathrm{I}_f$, соответственно. Функции~${D}$ 
соответствует функция $\varphi: 2^{{S}}\hm\to 
2^{{J}_{\mathrm{ob}}}$, $\varphi(\mathbf{X})\hm= \left\{ 
{D}(x_\alpha)\vert x_\alpha\hm\in \mathbf{X}\right\}$, фор\-ми\-ру\-ющая 
множество прецедентов $\mathbf{Q}\hm= \{q_i\vert q_i \hm= (m_i, t_i)\} 
\hm\subseteq \mathrm{I}_i\times \mathrm{I}_f$, $q_i[k]\hm\in 
\mathrm{I}_k$, $i\hm= 1,\ldots , \vert \mathbf{X}\vert$, $\mathbf{Q}\hm= 
\varphi(\mathbf{X})$~\cite{6-tr}. Для~$\Gamma_k$ определена функция полного 
прообраза значения $\lambda_{k_b}\hm\in \mathrm{I}_k$ 
в~множестве~$\mathbf{X}$, $\Gamma^{-1}_k (\lambda_{k_b}) \hm\subseteq 
\mathbf{X}$, а~для функции~${D}$~--- функция полного прообраза 
объекта, ${D}^{-1}(q)\hm= \bigcap\limits_{k=1,n} \Gamma^{-1}_k(q[k])$. 

\vspace*{2pt}

\noindent
\textbf{Определение~1.} \textit{Регулярным будем называть множество 
прецедентов~$\mathbf{Q}$, для которого выполнено} 
$\underset{\mathbf{Q}^2}{\forall} (q_1,q_2): {D}^{-1}(q_1)\not= 
{D}^{-1}(q_2)$. 
%
Если множества~$\mathbf{X}$ и~$\mathbf{Q}$ 
изоморфны, так что $\forall\,x\hm\in \mathbf{X}: x={D}^{-
1}({D}(x))$, то~$\mathbf{X}$ также регулярно~\cite{6-tr}. 
В~дальнейшем рас\-смат\-ри\-ва\-ют\-ся только регулярные множества 
$\mathbf{X}/\mathbf{Q}$.
     
     \vspace*{2pt}
     
     В~подавляющем большинстве реальных задач 
множества~$\mathrm{I}_k$ выбираются произвольно. Для получения более 
информативных <<синтетических признаковых описаний>>~\cite{6-tr} 
множество $\mathbf{U}(\mathbf{X})\hm= \left\{ \Gamma_k^{-1} 
(\lambda_{k_b})\right\}$ рассматривается как предбаза \textit{топологии} 
     $\mathbf{T}(\mathbf{X})\hm= \left\{ \emptyset, \mathrm{I}, a\cup b, a\cap b 
: a,b \hm\in \mathbf{U}(\mathbf{X})\right\}$, где $\mathrm{I}\hm= 
\{\mathbf{X}\}$~--- единичный элемент. Введение отношения порядка на 
элементах $\mathbf{T}(\mathbf{X})$ позволяет частично упорядочить 
элементы топологии $\mathbf{T}(\mathbf{X})$ в~\textit{решетку} 
$\mathbf{L}(\mathbf{T}(\mathbf{X}))\hm=\left\{ a\vee b, a \wedge b: a,b\hm\in 
\mathbf{T}(\mathbf{X}), (a\hm\geq b)\ \mbox{или}\ (a\hm\leq b) \right\}$ так, что 
$a\hm\leq b\hm\equiv a\hm\subseteq b$ и~$a\vee b\hm=  
\mathrm{sup}\,(a,b)$~---  объединение, а $a\wedge b\hm= \mathrm{inf}\,(a,b)$~--- 
пересечение множеств $a$ и~$b$. Показано, что при регулярности 
$\mathbf{X}/\mathbf{Q}$ (определение~1) решетка 
$\mathbf{L}(\mathbf{T}(\mathbf{X}))$~--- булева и~элементы 
$\mathbf{U}(\mathbf{X})$ являются вершинами 
$\mathbf{L}(\mathbf{T}(\mathbf{X}))$~\cite{6-tr}. При этом возникают три 
фундаментальные разновидности признаков: \textit{булевы} (вершина решетки 
$\Gamma_k^{-1}(1)$), \textit{категорные} (антицепи решетки) 
и~\textit{числовые} (цепи решетки)~\cite{6-tr}. 
{\looseness=-1

}
  
     
     Таким образом, $k$-я числовая величина, заданная на $\mathbf{Q}$ 
посредством~$\mathrm{I}_k$, соответствует некоторой цепи 
$A_k(\mathbf{X})$ в~$\mathbf{L}(\mathbf{T}(\mathbf{X}))$, образованной 
множествами $u(\lambda_{k_b}) \hm= \bigcup^b_{\beta=1} \Gamma_k^{-1} 
(\lambda_{k_\beta})$, $\lambda_{k_{b-1}}\hm\leq \lambda_{k_b}\hm\leq 
\lambda_{k_{b+1}}$. Каж\-до\-му значению~$\lambda_{k_b}$ соответствует 
множество объектов $u(\lambda_{k_b})$ и~единственное дополнение $\neg 
u(\lambda_{k_b})$ (так как решетка булева). Совокупность точек 
$\mathrm{cdf}\,(A_k(\mathbf{X}))\hm= \{ ( \lambda_{k_b}, \vert 
u(\lambda_{k_b})\vert / N)\}$ представляет \textit{эмпирическую функцию 
распределения} (э.ф.р.)\ $k$-й переменной.

\vspace*{-6pt}
     
\section{Порождение <<синтетических>> числовых признаков}

\vspace*{-2pt}

     Булевой решетке $\mathbf{L}(\mathbf{T}(\mathbf{X}))$ сопоставляется 
\textit{метрическое пространство значений признаков} 
$\mathbf{M}_{\mathbf{L}} (\mathbf{L}(\mathbf{T}(\mathbf{X})), 
\rho_{\mathbf{L}})$ с~метрикой $\rho_{\mathbf{L}}: \mathbf{L}^2\hm\to 
\mathbf{R}^+$. Над $\mathbf{L}(\mathbf{T}(\mathbf{X}))$ также 
определяется \textit{метрическое пространство объектов} $\mathbf{M}_q 
[\mathbf{L}(\mathbf{T}(\mathbf{X}))](\mathbf{Q},\rho_q)$ с~мет\-ри\-кой 
$\rho_q: \mathbf{Q}^2\hm\to \mathbf{R}^+$. Для определения 
метрик~$\rho_{\mathbf{L}}$ и~$\rho_q$ вводятся понятия \textit{изотонной 
оценки} на решетки и~\textit{окрест\-ности} элемента в~топологии~\cite{6-tr}.
     
     \smallskip
     
     \noindent
     \textbf{Определение~2.}\ \textit{Оценка на решетке~$\mathbf{L}$ есть 
такая функция $v: \mathbf{L}\hm\to \mathbf{R}^+$, что 
$\underset{\mathbf{L}}{\forall}\, a,b: v[a]\hm+v[b]\hm= v[a\wedge b]\hm+ v[a\vee b].
$
 Оценка изотонна, если} 
 $\underset{\mathbf{L}}{\forall}\, a,b: a\hm\subseteq 
b\hm\Rightarrow v[a]\hm\geq v[b]$ 
(например, высота элемента $h[x]$). Функция 
$\rho(x,y)\hm= v[x\vee y]\hm- v[x\wedge y]$, где~$v$ изотонна, является 
метрикой типа~$\rho_{\mathbf{L}}$~\cite{6-tr}.

\smallskip

\noindent
\textbf{Определение~3.} \textit{Окрестность $u(x)$ точки $x\hm\in 
\mathbf{X}$ в~топологии $\mathbf{T}(\mathbf{X})$ есть произвольное $u\hm\in 
\mathbf{T}(\mathbf{X})$, $x\hm\in \mathbf{u}$; $u(x)$ разделяет $x,y\hm\in 
\mathbf{X}$ при} $(x\hm\in {u})\not= (y\hm\in {u})$. 
     
     Любой объект $q\hm\in \mathbf{Q}$ представлен 
в~$\mathbf{L}(\mathbf{T}(\mathbf{X}))$ как набор окрестностей 
$\Gamma_k^{-1}(q[k])$, $k\hm=1, \ldots , n$, что позволяет определить метрику 
$\rho_q(q_1,q_2)$ как функцию $f_q: \mathbf{R}^n\hm\to \mathbf{R}^+$, 
$\rho_q(q_1,q_2)\hm= f_q((\rho_{\mathbf{L}}(\Gamma^{-1}_k (q_1[k]), 
\Gamma_k^{-1}(q_2[k])))$, $k\hm= 1, \ldots , n$. 

\smallskip

\noindent
\textbf{Определение~4.} \textit{Для аддитивных~$f_q$ расстояние 
$\rho_q(q_1,q_2)$ между объектами $q_1$ и~$q_2$ определяется посредством 
линейной комбинации $S\left( \sum\nolimits_{k=\overline{1,n}} \omega_k 
{s}\left(\rho_{\mathbf{L}} \left(\Gamma_k^{-1}(q_1[k]),  
\Gamma_k^{-1}\left(q_2[k]\right)\right)\right)\right)$, где ${S}, {s}: 
\mathbf{R}\hm\to \mathbf{R}^+$~--- произвольные <<сглаживающие>> 
функции, а~$\omega_k$~--- вес $k$-го признакового описания}. 
     
     Если ${S}\hm=1$, ${s}\hm= 1/\vert\mathbf{Q}\vert$, 
$\rho_{\mathbf{L}}(x,y)\hm= (x\not= y)$, $(\omega_k=1)$, а~$\Gamma_k^{-1}$ 
и~$\Gamma_k$ определены для $\mathrm{I}_k\hm= [0,1]$, то $\rho_q$~--- 
метрика Хэмминга. При ${S}\hm= \sqrt[p]{\vphantom{1}}$,  ${s}(x)\hm= x^p$, 
$\rho_{\mathbf{L}}(x,y)\hm= \vert x\Delta y\vert$, $\mathrm{I}_k\hm\subset 
\mathbf{R}$, $\rho_q$~--- взвешенная мет\-ри\-ка Минковского и~т.\,д. 
В~рамках топологического подхода порождение синтетических признаков 
осуществляется посредством метрики~$\rho_{\mathbf{L}}$ либо 
метрики~$\rho_q$ (последнее требует определения~$\rho_{\mathbf{L}}$). 

\vspace*{-6pt}
     
\section{Синтетические признаки на~основании 
метрики~$\rho_{\mathbf{L}}$}

\vspace*{-2pt}

\noindent
\textbf{Определение~5.} \textit{Расстояние $\rho_A: 
\mathbf{A}(\mathbf{X})^2\hm\to \mathbf{R}^+$ между цепями $a\hm= \langle 
a_1, \ldots , a_i, \ldots \mathrm{I}\rangle$ и~$b\hm= \langle b_1, \ldots , b_j, 
\ldots \mathrm{I}\rangle$ есть сумма расстояний между 
соответствующими элементами}: 
\begin{multline*}
\rho_{\mathbf{A}}(a,b)= \min \left( \sum\limits_{i=1, \vert a\vert} \!\rho_{\mathbf{L}} \!\left( a_i, \argmin\limits_{b_j\in b}
\rho_{\mathbf{L}}\left(a_i, b_j\right)\!\right) \!,\right.\\
\left.\sum\limits_{i=1,\vert b\vert} \!\rho_{\mathbf{L}}\left(b_j, \argmin\limits_{a_i\in 
a} \rho_{\mathbf{L}} \left(b_j,a_i\right)\right)\right).
\end{multline*}
     
     
     
     Пусть $\mathbf{A}(\mathbf{X})$~--- множество всех цепей 
$\mathbf{L}(\mathbf{T}(\mathbf{X}))$. Алгоритмы прогнозирования $k$-й 
переменной соответствуют цепям в~$\mathbf{A}(\mathbf{X})$, наиболее 
близким к~цепи $\mathbf{A}_k(\mathbf{X})$. Пусть 
$\mathbf{A}(\mathbf{X})_{1,n}\hm\subset \mathbf{A}(\mathbf{X})$  соответствует 
цепям над элементами $\mathbf{U}(\mathbf{X})$, $k\hm=1,\ldots , n$. Тогда 
искомые алгоритмы соответствуют решению задачи
     \begin{equation}
     aa=\argmin\limits_{a\in \mathbf{A}(\mathbf{X})_{i,n}} \rho_{\mathbf{A}} 
(A_k(\mathbf{X}),a)\,.
     \label{e1-tr}
     \end{equation}
     
     \vspace*{-12pt}

\section{Синтетические признаки на~основании метрики~$\rho_q$}

\vspace*{-2pt}

     В рамках данного подхода согласовываются значения настраиваемой 
<<признаковой>> метрики $\rho_q$ с~вектором весов ($\omega_k$) 
     \begin{multline}
     \argmin\limits_{\{\omega_k\}} \sum\limits_{m=1}^{\vert\mathbf{X}\vert} 
\sum\limits^{\vert\mathbf{X}\vert}_{j\not= m} \mathbf{L}_{\mathrm{F}} \left(\rho_q \left( 
(\omega_k), \mathrm{X}_m, \mathrm{X}_j\right) -{}\right.\\[-2pt]
\left.{}- \rho_e (\mathrm{X}_m, 
\mathrm{X}_j)\right)\,,
     \label{e2-tr}
     \end{multline}
     
   
     
     
     \noindent
где $\mathbf{L}_{\mathrm{F}}$~--- та или иная функция потерь. Например, в~настоящей 
работе в~качестве~$\mathbf{L}_{\mathrm{F}}$ используется функция модуля 
$\mathbf{L}_{\mathrm{F}}(x)\hm= \vert x\vert$. Практически важным случаем экспертной 
метрики~$\rho_e$ является <<скалярная>> метрика (например, модуль 
разности значений).

\vspace*{2pt}

\noindent
\textbf{Теорема~1.} \textit{При использовании скалярной метрики~$\rho_e$ 
условие}~(\ref{e1-tr}) \textit{для метрики Хэмминга~$\rho_q$ соответствует 
аддитивной схеме учета признаков}. 

\vspace*{2pt}

\noindent
     Д\,о\,к\,а\,з\,а\,т\,е\,л\,ь\,с\,т\,в\,о\,.\ \ Для~$\rho_e$ в~виде модуля 
разности выполнены все три аксиомы метрики (так как они выполнены для 
любых трех коллинеарных точек). Пусть нулевой элемент входит во все 
множества~$\mathrm{I}_k$, так что можно определить расстояние от нулевого 
элемента до любого другого элемента множества~$\mathrm{I}_k$ посредством 
экспертной метрики $\rho_e$.  
В~$\rho$-кон\-фи\-гу\-ра\-ции объектов, образованной метрикой~$\rho_q$, за 
нулевую точку с~номером~$m_0$ можно, вообще говоря, принять любой 
объект (например, центральный объект, соответствующий минимальной сумме 
расстояний до всех остальных). Пусть в~$m$-сум\-ме в~выражении~(\ref{e2-tr}) 
одна из $j$-сумм соответствует нулевому элементу~$\mathbf{X}_{m_0}$.
     
     Использование скалярной метрики~$\rho_e$ в~условии~(\ref{e2-tr}) 
соответствует проекции $\vert\mathbf{X}\vert$-мер\-ной  
$\rho_q$-кон\-фи\-гу\-ра\-ции на одномерную числовую ось 
значений~$\mathrm{I}_k$. Любая из $j$-сумм в~(\ref{e1-tr}) включает все 
объекты из~$\mathbf{X}$ и~покрывает все множество 
значений~$\mathrm{I}_k$. Разница между любыми двумя значениями 
из~$\mathrm{I}_k$ (т.\,е.\ $\rho_e(\mathbf{X}_m, \mathbf{X}_j)$) также 
является отрезком на числовой оси $k$-й переменной, соответствующим сдвигу 
значений в~множестве~$\mathrm{I}_k$ на константу~$\lambda_{k_\beta}$. 
     
     Так как каждая из $j$-сумм в~(\ref{e2-tr}) соответствует одной и~той же 
(с~точ\-ностью до константы) задаче согласования метрик, то 
 задача~(\ref{e2-tr}) с~$\mathbf{L}_{\mathrm{F}}(x)\hm= \vert x\vert$ может быть 
переформулирована через расстояния от нулевого элемента, т.\,е.\ произведен 
переход от оценки попарных расстояний к~суммированию по всем объектам:
     \begin{equation}
     \argmin\limits_{\{\omega_k\}} \sum\limits^{\vert \mathbf{X}\vert}_{m\not= 
m_0} \left\vert \rho_q (\{\omega_k\}, \mathbf{X}_{m_0}, \mathbf{X}_m) -
\mathbf{T}_m\right\vert,
     \label{e2.1-tr}
     \end{equation}
     где $\mathbf{T}_m\hm= \rho_e(\mathbf{X}_{m_0}, \mathbf{X}_m)$~--- 
значение прогнозируемой $k$-й числовой переменной для 
объекта~$\mathbf{X}_m$ (так как, по построению, 
точка~$\mathbf{X}_{m_0}$ соответствует нулевому значению). Если~$\rho_q$ 
определена как метрика Хэмминга (см.\ комментарий к~определению~4), то 
задача в~постановке~(\ref{e2.1-tr}) соответствует представлению~$\rho_q$ 
в~виде линейной формы, т.\,е.\ аддитивной схеме учета признаков. Тео\-ре\-ма 
доказана.

\smallskip

    Таким образом, в~случае скалярных~$\rho_e$ расстояния~$\rho_q$ служат 
своего рода <<синтетическими>> числовыми признаками объектов 
в~множествах $\mathbf{X}/\mathbf{Q}$. Для практического 
применения~(\ref{e2-tr}) необходимо определить функции~$\Gamma_k$, а~для 
определения метрических пространств~$\mathbf{M}_{\mathbf{L}}$ 
и~$\mathbf{M}_q$~--- метрики~$\rho_{\mathbf{L}}$ и~$\rho_q$. В~случае 
задач типа <<струк\-ту\-ра--свой\-ст\-во>> молекул для этого применяется 
теория хемографов.

\vspace*{-6pt}

\section{О теории анализа размеченных графов}

\vspace*{-2pt}

     Для порождения признаковых описаний молекулярных структур в~рамках 
теории анализа размеченных графов вводится понятие 
хемогр$\acute{\mbox{а}}$фа~\cite{7-tr}. Признаковые описания порождаются 
над разметками хемографов~\cite{8-tr}.
     
     \smallskip
     
     \noindent
\textbf{Определение~6.} \textit{Граф $G\hm= (\mathbf{V}, \mathbf{E})$~--- совокупность 
множества вершин $\mathbf{V}\hm= \mathbf{V}(G)$ и~множества ребер $\mathbf{E}\hm=\mathbf{E}(G)$, 
$\mathbf{E}\hm\subset \mathbf{V}^2$. Хемограф ($\chi$-граф)~--- конечный, связный, 
неориентированный, размеченный граф без петель, с~кликовым числом не 
более}~3. Множество $\boldsymbol{\Gamma}\hm= \{ (\mathbf{V}, \mathbf{E})\vert \mathbf{V} \hm\subset \mathbf{N}, 
\mathbf{E}\hm\subset \mathbf{N}^2\}$, где $\mathbf{N}$~--- натуральный ряд, есть 
\textit{множество всех графов}.

\smallskip

     Множество вершин $\chi$-гра\-фа~$\mathbf{X}$ соответствует\linebreak 
множеству атомов молекулы, а~множество ребер~--- множеству химических 
связей молекулы. Хемографы строятся на основе <<внеш\-них>> (декартовых) 
координат ядер атомов $\mathbf{\mathbf{R}}(X)\hm= \{ \vec{R}_i(X)\vert \vec{R}_i\hm\in 
\mathbf{R}^3\}$,\linebreak $i\hm= 1,\ldots , \vert \mathbf{V}(\mathrm{X})\vert$, или  
<<внут\-рен\-них>> координат (межатомные расстояния), так что $\chi$-гра\-фу~${X}$ 
со\-по\-став\-лен ряд мат\-риц $\mathbf{M}({X})\hm= 
(m_{ij}({X}))$, $m_{ij}\hm\in \mathbf{R}$, $i,j\hm= 1,\ldots , \vert 
\mathbf{V}({X})\vert$ (матрица межатомных расстояний, матрица смежности 
и~др.). Химические формулы в~различных <<машинных>> форматах (SMILES, 
XYZ и~др.)\ суть упрощенные пред\-став\-ле\-ния мат\-риц 
$\mathbf{M}({X})$ или координат~$\mathbf{R}(X)$.
     
     Признаковые описания $\chi$-гра\-фов вводятся на основании анализа 
специальной разновидности подграфов $\chi$-гра\-фов~--- цепей и~узлов, для 
описания которых в~работах~\cite{7-tr, 8-tr} вводится комплекс понятий 
и~соответствующих обозначений: \textit{множество всех замкнутых 
подграфов $\chi$-гра\-фа}~${X}$, $\boldsymbol{\Pi}({X})$ 
(включает \textit{множество связных подграфов} $\mathbf{S}(X)$, множество 
\textit{цепей} $\mathbf{C}(X)$, \textit{множество  
$\chi$-уз\-лов}~$\mathbf{K}(X)$), \textit{операции  
объеди\-не\-ния/пе\-ре\-се\-че\-ния множества подграфов} 
($\overset{\smile}{\Pi}\hm= \bigcup_{i=1}^{\vert\Pi\vert} \pi_i\hm= ( 
\bigcup_{i=1}^{\vert\Pi\vert} \mathbf{v}_i, \bigcup_{i=1}^{\vert\Pi\vert} 
\mathbf{e}_i)$; $\overset{\frown}{\Pi}\hm= \bigcap_{i=1}^{\vert\Pi\vert} 
\pi_i$), \textit{условие образования} графа $\overset{\smile}{{O}}\hm= 
{X}$ множеством $\mathbf{O}\hm\subseteq 
\boldsymbol{\Pi}({X})$, оператор $\hat{c}^n$ вычисления \textit{цепей 
длины}~$n$, \textit{множество смеж\-ности} (\textit{окружение}) 
вершины~$v$: $\Gamma(v)\hm= \hat{v}\hat{e} v$, \textit{узел} вершины~$v$, 
($\Gamma(v),\hat{e}v$) \textit{и др.}~\cite{7-tr}. Такие подграфы необходимы 
для исследования изоморфизма $\chi$-гра\-фов.
     
     \smallskip
     
     \noindent
     \textbf{Определение~7.} \textit{Графы $G_1$ и~$G_2$ изоморфны 
($G_1\simeq G_2$) если существует взаимно однозначное соответствие 
между их вершинами и~ребрами}. Изоморфизм $G_1\simeq G_2$ соответствует 
существованию функции перенумерации $\mu_{\mathbf{I}}: \mathbf{N}\hm\to 
\mathbf{N}$, так что из $G_2\hm=\mu_{\mathrm{I}}(G_1)$ следует $G_1\simeq 
G_2$. Определяется класс $\mathbf{\mathbf{I}}(G)\hm= \{ g\hm\in 
\boldsymbol{\Gamma}\vert \exists\,\mu_{\mathbf{I}}: G\hm= 
\mu_{\mathbf{I}}(g)\}$.
     
     \smallskip
     
     \noindent
     \textbf{Определение~8.} \textit{Пусть задан алфавит меток $\mathbf{Y}\hm= \{ 
\upsilon_1, \upsilon_2, \ldots, \upsilon_{n(Y)}\}$. Функция разметки $\mu_{\mathbf{V}}: 
\mathbf{V}\hm\to \mathbf{Y}$ сопоставляет метку каждой вершине $\chi$-графа}. 
     
     \smallskip
     
     Для комбинаторного анализа свойств изоморфизма графов и~для 
порождения признаковых описаний $\chi$-гра\-фов в~тео\-рии анализа 
размеченных \mbox{графов} вводятся \textit{множество} $\chi$-\textit{це\-пей} 
$\tilde{\mathbf{Y}}$ над алфавитом~$\mathbf{Y}$ (\textit{в~том числе подмножества всех  
$\chi$-це\-пей длины}~$n$, $\tilde{\mathbf{Y}}^n \hm\subset \tilde{\mathbf{Y}}$), множество 
$\chi$-\textit{уз\-лов} $\hat{\mathbf{Y}}\hm= \bigcup_{k=2} \hat{\mathbf{Y}}(k)$, 
$\hat{\mathbf{Y}}(k)\hm= \{\mathbf{Y}\times 2^{\mathbf{Y}^k}\}$, понятия \textit{инварианта 
графа}  ($\iota: \boldsymbol{\Gamma}\hm\to \mathbf{R}^n, n\hm\in \mathbf{N}: 
\forall\,a \hm\in \boldsymbol{\Gamma}: b\hm\in \mathbf{I}(a)\hm\Rightarrow 
\iota (b)\hm= \iota(a)$), \textit{кор\-теж-ин\-ва\-ри\-ан\-та} $\iota: 
\boldsymbol{\Gamma}\hm\to \mathbf{R}^n$, $n\geq 2$, \textit{полноты 
инварианта} $\forall\,a\hm\in \boldsymbol{\Gamma}: b\hm\in \mathbf{I}(G) 
\hm\Leftrightarrow \iota(a) \hm= \iota(b)$, \textit{изомерных по 
инварианту}~$\iota$ \textit{графов}, \textit{функции разметки $\chi$-це\-пей}  
$\mu_c: \mathbf{C}\hm\to \tilde{\mathbf{Y}}$ и~$\chi$-\textit{узлов} 
$\mu_\kappa: \mathbf{K}\hm\to \hat{\mathbf{Y}}$,  
$\chi$-\textit{фраг\-мен\-тов} как элементов множеств $\hat{\mu}_c^{-1}\alpha$ 
и~$\hat{\mu}_\kappa^{-1}\kappa$, \textit{оператора вхождения множества 
подграфов} $\boldsymbol{\pi}\hm\subset \boldsymbol{\Gamma}$ в~$\chi$-граф 
$\mathbf{X}$ $\hat{\beta}[\mathbf{X}]: 2^{\boldsymbol{\Gamma}}\hm\to \{0,1\}$, 
\textit{оператора числа вхождений множества подграфов} в~$\mathbf{X}$ 
$\hat{\eta}[\mathbf{X}]: 2^{\boldsymbol{\Gamma}}\hm\to \mathbf{N}$ и~др.~\cite{8-tr}. 
     
     Эти понятия позволяют проводить комбинаторный анализ изоформизма 
$\chi$-гра\-фов и~порождать признаковые описания $\chi$-гра\-фов как 
$\hat{\eta}[\mathbf{X}]\hat{\mu}^{-1}_c \alpha$, 
$\hat{\eta}[\mathbf{X}]\hat{\mu}_\kappa^{-1}\kappa$, 
     $\hat{\beta}[\mathbf{X}]\hat{\mu}_c^{-1}\alpha$, 
$\hat{\beta}[\mathbf{X}]\hat{\mu}^{-1}_\kappa \kappa$ (лемма~1 
в~работе~\cite{9-tr}). Результат последовательного применения 
     $\hat{\mu}_c^{-1}$ к~$\boldsymbol{\alpha}\hm= \{\alpha \hm\in 
\tilde{\mathbf{Y}}\}$ обозначим $\hat{\boldsymbol{\mu}}_c^{-1}
\boldsymbol{\alpha} \hm= \{ \hat{\mu}_c^{-1}\alpha, \alpha\hm\in 
\boldsymbol{\alpha}\}$, оператора $\hat{\mu}_\kappa^{-1}$ ко множеству  
$\chi$-уз\-лов $\boldsymbol{\kappa}\hm= \{\kappa\hm\in \hat{\mathbf{Y}}\}$~---
$\hat{\boldsymbol{\mu}}_\kappa^{-1}\boldsymbol{\kappa}\hm= \{  
\hat{\mu}^{-1}_\kappa \kappa, \kappa\hm\in \boldsymbol{\kappa}\}$, 
оператора~$\hat{\beta}$ ко множеству $\tilde{\boldsymbol{\pi}}\hm= \{ 
\boldsymbol{\pi}_1, \boldsymbol{\pi}_2, \ldots , \boldsymbol{\pi}_n\}$, 
$\tilde{\boldsymbol{\pi}}\hm\subset \boldsymbol{\Gamma}$ обозначим 
$\hat{\boldsymbol{\beta}}\tilde{\boldsymbol{\pi}}\hm= \{ 
\hat{\beta}\boldsymbol{\pi}_1, \hat{\beta}\boldsymbol{\pi}_2, \ldots , 
\hat{\beta}\boldsymbol{\pi}_n\}$. 
     
     Комбинаторный анализ разрешимости задачи\linebreak анализа изоморфизма  
$\chi$-гра\-фов сводится к~уста\-нов\-ле\-нию локальной полноты некоторых 
 кор\-теж-ин\-ва\-ри\-ан\-тов~\cite{8-tr}. Пусть $\iota: 
\boldsymbol{\Gamma}\hm\to \mathbf{R}^n$, $n\hm\geq 2$,~---  
кор\-теж-ин\-ва\-ри\-ант, построенный над некоторым \textit{\mbox{множеством} 
из}~$n$~\textit{элементарных инвариантов} $\boldsymbol{\iota}_e\hm\subset 
\mathbf{E}$. Пусть для графа~$G$ выражение $\boldsymbol{\iota}_i(G)\hm= \{ 
\iota_i(G), i\hm= 1,\ldots , n\}$ будет означать множество значений инвариантов 
из~$\boldsymbol{\iota}_e$. Определим \textit{функцию нумерации 
элементарных инвариантов} $\lambda: \boldsymbol{\iota}_e\hm\to \mathbf{N}$ и~\textit{оператор формирования  
кор\-теж-ин\-ва\-ри\-ан\-та}. \textit{Оператор формирования  
кор\-теж-ин\-ва\-ри\-ан\-та} $\hat{\iota}: 2^{\mathbf{E}}\hm\to \mathbf{R}^n$ по 
заданному~$\boldsymbol{\iota}_e$ определим как 
$\hat{\iota}\boldsymbol{\iota}_e\hm= (\iota_j, \iota_k, \ldots , \iota_l)$, $\iota_j, 
\iota_k,\ldots , \iota_l\hm\in \boldsymbol{\iota}_e$, $\lambda(\iota_j)\hm= 1\hm\leq 
\lambda(\iota_k)< \cdots < \lambda(\iota_l)\hm=n$, а~значение $i$-го элемента 
кортежа~$\hat{\iota}\boldsymbol{\iota}_e$ обозначим 
$\hat{\iota}[i]\boldsymbol{\iota}_e(G)\hm=\iota(G)\vert \lambda(\iota)\hm=i$. 
В~теореме~2 показана взаимосвязь между изоморфизмом $\chi$-гра\-фов 
и~полнотой кор\-теж-ин\-ва\-ри\-ан\-тов.
     
     \smallskip
     
     \noindent
     \textbf{Теорема~2.} $\forall\,a,b\in\boldsymbol{\Gamma}: \vert 
\mathbf{I}(a)\cap \mathbf{I}(b)\vert >0 \hm\Leftrightarrow 
\underset{i=\overline{1,n}}{\exists}\, i: \hat{\iota}[i] \boldsymbol{\iota}_e (a)\not= 
\hat{\iota}[i]\boldsymbol{\iota}_e(b)$, \textit{так что 
$\hat{\iota}\boldsymbol{\iota}_e$~--- полный кор\-теж-ин\-ва\-ри\-ант}. 

\smallskip
     
     Д\,о\,к\,а\,з\,а\,т\,е\,л\,ь\,с\,т\,в\,о\ \ приведено в~работе~\cite{9-tr}. 
Следствия теоремы важны для анализа множества прецедентов~Pr.
     
     \smallskip

\noindent
\textbf{Следствие~1.} Полные инварианты могут быть образованы над 
подмножествами множеств~$\tilde{\mathbf{Y}}$ и~$\hat{\mathbf{Y}}$, если 
для каждого $\chi$-гра\-фа~$\mathbf{X}$ множества прецедентов 
$\mathrm{Pr}$ существуют $\boldsymbol{\kappa}^\prime(\mathbf{X}) \hm= 
\{\kappa\hm\in \boldsymbol{\kappa} \vert \hat{\eta}[\mathbf{X}] 
\hat{\mu}_\kappa^{-1} \kappa \hm=1\}$ и~$\boldsymbol{\alpha}^\prime 
(\mathbf{X})\hm= \{\alpha \hm\in \boldsymbol{\alpha}^\prime \vert 
\hat{\eta}[\mathbf{X}] \hat{\mu}^{-1}_{c}\alpha \hm=1\}$ такие, что при 
$\boldsymbol{\pi}^\prime(\mathbf{X})\hm= \hat{\boldsymbol{\mu}}_\kappa^{-1}
\boldsymbol{\kappa}^\prime(\mathbf{X})\cup \hat{\boldsymbol{\mu}}_c^{-1}
\boldsymbol{\alpha}^\prime(\mathbf{X})\ 
\overset{\smile}{\boldsymbol{\pi}}(\mathbf{X})\hm=\mathbf{X}$. 

\smallskip

\noindent
\textbf{Следствие~2.} Пусть в~конечном $\mathrm{Pr}\hm\subset 
\boldsymbol{\Gamma}$ каждый граф~$G$ помечен меткой изоморфности $\mathrm{iso} 
(G): \mathbf{I}(\boldsymbol{\Gamma})\hm\to \mathbf{N}$. Разрешимость 
$$
\underset{a,b\in \mathrm{Pr}}{\forall}\, \mathrm{iso}(a)\not= \mathrm{iso}(b)\hm\Rightarrow 
\iota(a)\hm\not= \iota(b)
$$
эквивалентна полноте 
$$
\underset{a,b\in \mathrm{Pr}}{\forall}\,\mathrm{iso}(a)\not= 
\mathrm{iso}(b)\hm\Rightarrow
\underset{i=\overline{1,\vert\chi\vert}}{\exists}\, i: \hat{\iota}[i] \chi(a) \not= \hat{\iota} [i] 
\chi(b).
$$


%\columnbreak

\noindent
\textbf{Следствие~3.} Задача распознавания изоморфных графов разрешима 
тогда и~только тогда, когда $\sum\nolimits_{a\in\mathrm{Pr}} \vert 
\mathbf{i}\boldsymbol{\mu}(a,\iota,\mathrm{Pr})\backslash 
\mathbf{i}(a,\mathrm{Pr})\vert \hm=0$, где $\mathbf{i}\boldsymbol{\mu}(G,\iota, 
\mathrm{Pr})$~--- множество изомерных $G$ графов; $\mathbf{i}(G, 
\mathrm{Pr})$~--- множество изоморфных $G$ графов. 

\smallskip


\noindent
\textbf{Следствие~4.} Инвариант~$\iota$ полон при $r_\iota (\iota, 
\mathrm{Pr})=1-(1/\vert \mathrm{Pr}\vert^2)\sum\nolimits_{a\in\mathrm{Pr}} 
\mathbf{i}\boldsymbol{\mu}(a,\iota,\mathrm{Pr})\backslash 
\mathbf{i}(a,\mathrm{Pr})\vert \hm=1$. 

\smallskip

\noindent
\textbf{Следствие~5.} Пусть 
$\phi(\hat{\iota}\boldsymbol{\chi},i,\mathrm{Pr})\hm= \{(a,b)\vert a,b\hm\in 
\mathrm{Pr},\ \hat{\iota}[i]\boldsymbol{\chi}(a)\not= 
\hat{\iota}[i]\boldsymbol{\chi}(b)\}$ 
и~$\phi(\hat{\iota}\boldsymbol{\chi},i,\mathrm{Pr})\cap\phi(\hat{\iota}\boldsymbol{\chi},j,\mathrm{Pr})
\hm=\emptyset$. Тогда $r_\iota(\iota, \mathrm{Pr})\hm= 
\sum\nolimits_{i=1}^{\vert\chi\vert} \varphi_\iota 
(\hat{\iota}\boldsymbol{\chi},i,\mathrm{Pr})$,  где
$\varphi_\iota(\hat{\iota}\boldsymbol{\chi},i,\mathrm{Pr})\hm= \vert 
\phi(\hat{\iota}\boldsymbol{\chi}, i,\mathrm{Pr}) \vert\backslash 
\vert\mathrm{Pr}\vert^2$. 

\vspace*{-6pt}

\section{Топологическая теория анализа $\chi$-графов}

\vspace*{-2pt}

     На основании алфавита $\mathbf{Y}$ строятся различные множества $\chi$-це\-пей 
длины~$m$, $\tilde{\mathbf{Y}}^m$, $m\hm=1,\ldots , m_{\max}$, множество всех  
$\chi$-уз\-лов~$\hat{\mathbf{Y}}$ и~затем со\-от\-вет\-ст\-ву\-ющие элементарные  
$\chi$- и~кор\-теж-ин\-ва\-ри\-ан\-ты. Для $\chi$-гра\-фа 
${X}\hm\in \mathbf{X}$ вычисляются множество всех  
$\chi$-це\-пей~${X}$, $\tilde{\mathbf{Y}}({X})$, и~множество 
всех $\chi$-уз\-лов~${X}$, $\hat{\mathbf{Y}}({X})$. Для 
любых $\alpha\hm\in \tilde{\mathbf{Y}}({X})$, $\kappa\hm\in 
\hat{\mathbf{Y}}({X})$  
пусть $\hat{\beta}[{X}]\hat{\mu}^{-1}_c\alpha$, 
$\hat{\beta}[\mathrm{X}]\hat{\nu}^{-1}_\kappa \kappa$~--- булевы инварианты. 
Тогда для выбранных $\boldsymbol{\alpha}\hm\subseteq \tilde{\mathbf{Y}}$ 
и~$\boldsymbol{\kappa}\hm\subseteq \hat{\mathbf{Y}}$ таких, что $\vert 
\boldsymbol{\alpha}\vert \hm+ \vert \boldsymbol{\kappa}\vert \hm= n$, 
вычисляются множества $\hat{\boldsymbol{\mu}}_c^{-1}\boldsymbol{\alpha}$ 
и~$\hat{\boldsymbol{\mu}}_\kappa^{-1}\boldsymbol{\kappa}$. Если 
множества~$\mathbf{I}_k$ определены как~$[0,1]$, то множество 
элементарных инвариантов $\boldsymbol{\iota}_e\hm= 
\hat{\boldsymbol{\beta}}\hat{\boldsymbol{\mu}}_c^{-1}\boldsymbol{\alpha}\cup 
\hat{\boldsymbol{\beta}}\hat{\boldsymbol{\mu}}_\kappa^{-
1}\boldsymbol{\kappa}$, а~если $\mathbf{I}_k \hm\subset \mathbf{N}$, то 
$\boldsymbol{\iota}_e\hm=\hat{\boldsymbol{\eta}}\hat{\boldsymbol{\mu}}_c^{-1}
\boldsymbol{\alpha}\cup 
\hat{\boldsymbol{\eta}}\hat{\boldsymbol{\mu}}_\kappa^{-
1}\boldsymbol{\kappa}$. Определив функцию нумерации инвариантов $\lambda: 
\boldsymbol{\iota}_e\hm\to \mathbf{N}$, определяем функцию $\Gamma_k$ как 
$\Gamma_k({X})\hm= \hat{\iota}[k]\boldsymbol{\iota}_e({X})$, 
а~функцию~$D$~--- как $D({X})\hm= 
\hat{\iota}\boldsymbol{\iota}_e(\mathrm{X})$. 
     
     Таким образом, предбаза $\mathbf{U}(\mathbf{X})$ топологии 
$\mathbf{T}(\mathbf{X})$ состоит из подмножеств $\chi$-гра\-фов, 
соответствующих $\chi$-це\-пям или $\chi$-уз\-лам. Для $D(\mathbf{X})\hm= 
\hat{\iota}\boldsymbol{\iota}_e(\mathbf{X})$ задача~(\ref{e2-tr}) реализуется 
как \textit{хемометрический анализ}~\cite{9-tr}:

\vspace*{-12pt}

\noindent
     \begin{multline*}
     \argmin\limits_{\{\omega_k\}} \sum\limits_{m_0=1}^{\vert 
\mathrm{X}\vert} \sum\limits_{m\not=m_0}^{\vert \mathrm{X}\vert}
     \left\vert
     \sum\limits_{k=1}^{n} 
\omega_k\hat{\iota}[k]\hat{\boldsymbol{\beta}}[X_{m_0}] \boldsymbol{\pi}\oplus{}\right.\\[-4pt]
\left.{}\oplus 
\hat{\iota}[k]\hat{\boldsymbol{\beta}}[X_m]\boldsymbol{\pi} -\vert T_m -
T_{m_0}
\vphantom{\sum\limits_{k=1}^{n}}
\vert \right\vert,
%     \label{e2.2-tr}
\end{multline*}

\vspace*{-12pt}

\noindent
где $\boldsymbol{\pi} \subseteq \hat{\boldsymbol{\mu}}_c^{-1}\boldsymbol{\alpha}
\cup \hat{\boldsymbol{\mu}}_\kappa^{-1}{\boldsymbol{\kappa}}$~--- 
некоторое <<опорное>> множество подграфов; $T_m$~--- оцениваемое 
свойство \mbox{$m$-й} молекулы, а~выражение под знаком суммы по~$k$ 
соответствует $\rho_q$-мет\-ри\-ке Хэмминга. Применяя теорему~1 
и~<<сглаживающую>> функцию $S: \mathbf{R}\hm\to 
\mathbf{R}^+$ (определение~4), получаем:
\begin{equation*}
\argmin\limits_{\{\omega_k\}} \sum\limits_{m=1}^{\vert\mathrm{X}\vert} 
\left\vert S\left( \sum\limits^n_{k=1} \omega_k 
s\left(\hat{\iota}[k]\hat{\boldsymbol{\beta}}[X_m]\boldsymbol{\pi}\right)\right)-T_m\right\vert.
%\label{e2.3-tr}
\end{equation*}

\vspace*{-18pt}

\section{Заключение}

\vspace*{-2pt}

     Для построения проблемно ориентированных метрик и~алгоритмов 
прогнозирования  КМ свойств\linebreak\vspace*{-12pt}

\pagebreak

\noindent
 молекул по их 
химической структуре разработанные признаковые описания хемографов 
анализируются посредством теории топологического анализа данных. 
В~результате анализа топологии $\mathbf{T}(\mathbf{X})$ и~решетки 
$\mathbf{L}(\mathbf{T}(\mathbf{X}))$ с~использованием условий~(\ref{e1-tr}) 
и~(\ref{e2-tr}) могут быть получены различные алгоритмы прог\-но\-зи\-ро\-вания 
произвольной числовой переменной на основе структуры хемографов. Эти 
алгоритмы одновременно являются алгоритмами порождения 
<<синтетических>> числовых признаков, информативных относительно этой 
переменной. Такие <<синтетические>> признаки могут использоваться 
в~алгоритмических конструкциях алгебраического подхода и~в обычных 
методах машинного обучения (регрессия, нейронные сети, метрические методы 
и~др.). Сопоставление полученных моделей алгоритмов с~формализмом 
квантовой механики, равно как и~экспериментальная апробация 
соответствующих алгоритмов, будут представлены во второй части статьи.

\vspace*{-6pt}
     
{\small\frenchspacing
 {%\baselineskip=10.8pt
 %\addcontentsline{toc}{section}{References}
 \begin{thebibliography}{99}
 
 \vspace*{-2pt}
 
\bibitem{1-tr}
\Au{Журавлев Ю.\,И.} Об алгебраическом подходе к~решению задач распознавания или 
классификации~// Проб\-ле\-мы кибернетики, 1978. №\,33. С.~5--68.
\bibitem{2-tr}
\Au{Рудаков К.\,В., Торшин~И.\,Ю.} Вопросы разрешимости задачи распознавания 
вторичной структуры белка~// Информатика и~её применения, 2010. Т.~4. Вып.~2. С.~25--35.
\bibitem{3-tr}
\Au{Рудаков К.\,В., Торшин~И.\,Ю.} Анализ ин\-фор\-ма\-тив\-ности мотивов на основе критерия 
разрешимости в~задаче распознавания вторичной структуры белка~// Информатика и~её 
применения, 2011. Т.~5. Вып.~4. С.~40--50.

\bibitem{6-tr} %4
\Au{Torshin I.\,Yu., Rudakov~K.\,V.} On the theoretical basis of metric analysis of poorly 
formalized problems of recognition and classification~// Pattern Recognition Image Analysis, 2015. 
Vol.~25. No.\,4. P.~577--587.

\bibitem{4-tr} %5
\Au{Torshin I.\,Yu., Rudakov~K.\,V.} Combinatorial analysis of the solvability properties of the 
problems of recognition and completeness of algorithmic models. Part~2: Metric approach within 
the framework of the theory of classification of feature values~// Pattern Recognition Image Analysis, 
2017. Vol.~27. No.\,2. P.~184--199.
\bibitem{5-tr} %6
\Au{Torshin I.\,Yu., Rudakov~K.\,V.} On the procedures of generation of numerical features over 
partitions of sets of objects in the problem of predicting numerical target variables~// Pattern 
Recognition  Image Analysis, 2019. Vol.~29. No.\,4. P.~654--667. doi: 10.1134/S1054661819040175.

\bibitem{9-tr} %7
\Au{Ruddigkeit L., van Deursen~R., Blum~L., Reymond~J.} Enumeration of 166~billion organic 
small molecules in the chemical universe database GDB-17~// J.~Chem. Inf. Model., 2012. Vol.~52. 
No.\,11. P.~2864--2875. doi: 10.1021/ci300415d.

\bibitem{7-tr} %8
\Au{Torshin I.\,Yu., Rudakov~K.\,V.} On the application of the combinatorial theory of solvability 
to the analysis of chemographs. Part~1: Fundamentals of modern chemical bonding theory and the 
concept of the chemograph~// Pattern Recognition Image Analysis, 2014. Vol.~24. No.\,1.  
P.~11--23.
\bibitem{8-tr} % 9
\Au{Torshin I.\,Yu., Rudakov~K.\,V.} On the application of the combinatorial theory of solvability 
to the analysis of chemographs. Part~2: Local completeness of invariants of chemographs in view 
of the combinatorial theory of solvability~// Pattern Recognition Image Analysis, 2014. Vol.~24. No.\,2. 
P.~196--208.

\bibitem{10-tr}
\Au{Torshin I.\,Yu., Rudakov~K.\,V.} Topological data analysis in materials science: The case of 
high-temperature cuprate superconductors~// Pattern Recognition Image Analysis, 2020. Vol.~30. No.\,2. 
P.~262--274. doi: 10.1134/S1054661820020157.
\end{thebibliography}

 }
 }

\end{multicols}

\vspace*{-6pt}

\hfill{\small\textit{Поступила в~редакцию 30.03.21}}

\vspace*{6pt}

%\pagebreak

%\newpage

%\vspace*{-28pt}

\hrule

\vspace*{2pt}

\hrule

%\vspace*{-2pt}

\def\tit{ON THE APPLICATION OF~A~TOPOLOGICAL APPROACH TO~ANALYSIS OF~POORLY 
FORMALIZED PROBLEMS FOR~CONSTRUCTING ALGORITHMS FOR~VIRTUAL 
SCREENING OF~QUANTUM-MECHANICAL PROPERTIES OF~ORGANIC MOLECULES~I:
THE BASICS OF~THE~PROBLEM-ORIENTED~THEORY}


\def\titkol{On the application of~a~topological approach to~analysis of~poorly 
formalized problems for~constructing algorithms~I}
% for %~virtual 
%screening}
% of~quantum-mechanical properties of~organic molecules.  Part~1: The basics of~the~problem-oriented theory}


\def\aut{I.\,Yu.~Torshin}

\def\autkol{I.\,Yu.~Torshin}

\titel{\tit}{\aut}{\autkol}{\titkol}

\vspace*{-15pt}


\noindent
Federal Research Center ``Computer Science and Control'' of the Russian Academy of Sciences, 
44-2~Vavilov Str., Moscow 119333, Russian Federation


\def\leftfootline{\small{\textbf{\thepage}
\hfill INFORMATIKA I EE PRIMENENIYA~--- INFORMATICS AND
APPLICATIONS\ \ \ 2022\ \ \ volume~16\ \ \ issue\ 1}
}%
 \def\rightfootline{\small{INFORMATIKA I EE PRIMENENIYA~---
INFORMATICS AND APPLICATIONS\ \ \ 2022\ \ \ volume~16\ \ \ issue\ 1
\hfill \textbf{\thepage}}}

\vspace*{3pt} 



\Abste{The topological approach to the analysis of poorly formalized problems and the theory of 
chemographs are extensions of Zhuravlev's algebraic approach to recognition. In the first part of the 
article, a~problem-oriented
formalism is proposed aimed at development of algorithms for 
screening assessments of the quantum-mechanical\linebreak\vspace*{-12pt}}

\Abstend{properties of molecules on the basis of their 
chemical structure. Methods for introducing metrics on sets of molecules and procedures for 
generating ``synthetic'' feature descriptions are proposed. The latter are generated by matching the 
values of some ``expert'' metric on the set of molecular properties to a~tunable metric on the set of 
molecular structures.}

\KWE{algebraic approach; chemoinformatics; labeled graphs; combinatorial solvability analysis}



\DOI{10.14357/19922264220106}

%\vspace*{-16pt}

\Ack
\noindent
This work was supported in part by RFBR grants 19-07-00356, 18-07-00944, 20-07-00537.




%\vspace*{6pt}

  \begin{multicols}{2}

\renewcommand{\bibname}{\protect\rmfamily References}
%\renewcommand{\bibname}{\large\protect\rm References}

{\small\frenchspacing
 {%\baselineskip=10.8pt
 \addcontentsline{toc}{section}{References}
 \begin{thebibliography}{99}
\bibitem{1-tr-1}
\Aue{Zhuravlev, Yu.\,I.} 1978. Ob algebraicheskom podkhode k~resheniyu zadach raspoznavaniya 
ili klassifikatsii [On algebraic approach to recognition and classification problems]. 
\textit{Problemy kibernetiki} [Cybernetic Problems] 33:5--68.
\bibitem{2-tr-1}
\Aue{Rudakov, K.\,V., and I.\,Yu.~Torshin}. 2010. Voprosy razreshimosti zadachi raspoznavaniya 
vtorichnoy struktury belka [Questions of solvability of the problem of recognition of the secondary 
structure of a protein]. \textit{Informatika i~ee Primeneniya~--- Inform Appl.} 4(2):25--35.
\bibitem{3-tr-1}
\Aue{Rudakov, K.\,V., and I.\,Yu.~Torshin.} 2011. Analiz informativnosti motivov na osnove 
kriteriya razreshimosti v~zadache raspoznavaniya vtorichnoy struktury belka [Analysis of the 
informativeness of motives based on the criterion of solvability in the problem of recognizing the 
secondary structure of a~protein]. \textit{Informatika i~ee Primeneniya~--- Inform Appl.} 
5(4):40--50.

\bibitem{6-tr-1} %4
\Aue{Torshin, I.\,Yu., and K.\,V.~Rudakov.} 2015. On the theoretical basis of metric analysis of 
poorly formalized problems of recognition and classification. \textit{Pattern Recognition Image 
Analysis} 25(4):577--587.
\bibitem{4-tr-1} %5
\Aue{Torshin, I.\,Yu., and K.\,V.~Rudakov.} 2017. Combinatorial analysis of the solvability 
properties of the problems of recognition and completeness of algorithmic models. Part~2: Metric 
approach within the framework of the theory of classification of feature values. 
\textit{Pattern Recognition Image Analysis} 27(2):184--199. 
\bibitem{5-tr-1} %6
\Aue{Torshin, I.\,Yu., and K.\,V.~Rudakov.} 2019. On the procedures of generation of numerical 
features over partitions of sets of objects in the problem of predicting numerical target variables. 
\textit{Pattern Recognition Image Analysis} 29(4):654--667.  doi: 10.1134/S1054661819040175.

\bibitem{9-tr-1} %7
\Aue{Ruddigkeit, L., R.~van Deursen, L.~Blum, and J.~Reymond.} 2012. Enumeration of 
166~billion organic small molecules in the chemical universe database GDB-17. \textit{J.~Chem. 
Inf. Model.} 52(11):2864--2875.  doi: 10.1021/ci300415d.
\bibitem{7-tr-1} %8
\Aue{Torshin, I.\,Yu., and K.\,V.~Rudakov.} 2014. On the application of the combinatorial theory 
of solvability to the analysis of chemographs. Part~1: Fundamentals of modern chemical bonding 
theory and the concept of the chemograph. \textit{Pattern Recognition Image Analysis} 24(1):11--23. 
\bibitem{8-tr-1} %9
\Aue{Torshin, I.\,Yu., and K.\,V.~Rudakov.} 2014. On the application of the combinatorial theory 
of solvability to the analysis of chemographs. Part~2: Local completeness of invariants of 
chemographs in view of the combinatorial theory of solvability. \textit{Pattern Recognition Image 
Analysis} 24(2):196--208.

\bibitem{10-tr-1}
\Aue{Torshin, I.\,Yu., and K.\,V.~Rudakov.} 2020. Topological data analysis in materials science: 
The case of high-temperature cuprate superconductors. \textit{Pattern Recognition Image Analysis} 
30(2):262--274.  doi: 10.1134/S1054661820020157.
\end{thebibliography}

 }
 }

\end{multicols}

\vspace*{-6pt}

\hfill{\small\textit{Received March 30, 2021}}

%\pagebreak

%\vspace*{-18pt}

  \Contrl
  
  \noindent
  \textbf{Torshin Ivan Yu.} (b.\ 1972)~--- Candidate of Science (PhD) in physics and mathematics, 
Candidate of Science (PhD) in chemistry, senior scientist, A.\,A.~Dorodnicyn Computing Center, 
Federal Research Center ``Computer Science and Control'' of the Russian Academy of Sciences, 
40~Vavilov Str., Moscow 119333, Russian Federation; \mbox{tiy135@yahoo.com}



\label{end\stat}

\renewcommand{\bibname}{\protect\rm Литература} 