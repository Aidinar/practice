\def\stat{andrianova}

\def\tit{КОНТЕКСТНЫЙ ПОИСК НА ФОТОНАХ С~ИСПОЛЬЗОВАНИЕМ~ТЕСТОВ~БЕЛЛА$^*$}

\def\titkol{Контекстный поиск на фотонах с~использованием тестов Белла}

\def\aut{\fbox{С.\,Н.~Андрианов}$^1$, Н.\,С.~Андрианова$^2$, Ф.\,М.~Аблаев$^3$, 
Ю.\,Ю.~Кочнева$^4$}

\def\autkol{С.\,Н.~Андрианов, Н.\,С.~Андрианова, Ф.\,М.~Аблаев, 
Ю.\,Ю.~Кочнева}

\titel{\tit}{\aut}{\autkol}{\titkol}

\index{Андрианов С.\,Н.}
\index{Андрианова Н.\,С.}
\index{Аблаев Ф.\,М.} 
\index{Кочнева Ю.\,Ю.}
\index{Andrianov S.\,N.}
\index{Andrianova N.\,S.}
\index{Ablaev F.\,M.}
\index{Kochneva Yu.\,Yu.}


{\renewcommand{\thefootnote}{\fnsymbol{footnote}} \footnotetext[1]
{Работа Ф.\,М.~Аблаева выполнена за счет средств субсидии, выделенной Казанскому 
(Приволжскому) федеральному университету для выполнения государственного задания 
в~сфере научной деятельности, проект №\,0671-2020-0065.}}


\renewcommand{\thefootnote}{\arabic{footnote}}
\footnotetext[1]{Институт прикладных исследований, Академия наук Республики Татарстан}
\footnotetext[2]{Казанский (Приволжский) федеральный университет, \mbox{natalia\_an83@mail.ru}}
\footnotetext[3]{Казанский (Приволжский) федеральный университет, \mbox{fablayev@gmail.com}}
\footnotetext[4]{Институт прикладных исследований, Академия наук Республики Татарстан,
\mbox{instpianrt@gmail.com}}

\vspace*{-6pt}




     
     \Abst{Рассмотрена возможность конкретной физической реализации контекстного 
поиска на квантовых состояниях с использованием тестов Белла, который рассматривался 
ранее лишь как абстрактная математическая процедура. Для этого предложено использовать 
контекстную кодировку слов в документах на поляризационных фотонных кубитах. 
Получены конкретизированные аналитические выражения для определения на основе тестов 
Белла параметра контекстного поиска по паре слов, которые могут быть связанными или нет 
в зависимости от значения этого параметра. Наибольшей связанности отвечает состояние 
квантовой перепутанности волновых функций документов по паре выбранных слов, 
которому соответствует определенное значение параметра контекстного поиска. 
Предложенные способы реализации семантического контекстного поиска необходимы для 
определения нелокальной контекстности, которая часто требуется при автоматизированном 
поиске и машинном переводе. При этом второе слово в паре поисковых слов поясняет смысл 
первого через их семантическую связь.}
     
     \KW{гипоним; гипероним; изотопия; родовидовая связь; фотонный кубит; 
перепутанные состояния; белловский тест; голографический процессор}

\DOI{10.14357/19922264220103}
  
\vspace*{-4pt}


\vskip 10pt plus 9pt minus 6pt

\thispagestyle{headings}

\begin{multicols}{2}

\label{st\stat}
     
    В~рамках познавательной дея\-тель\-ности человека информация 
структурируется с точ\-ки зрения ее значения и организации, постепенно 
пре\-вра\-ща\-ясь в знание. Фор\-ма\-ли\-зу\-емые на естественном языке знания 
систематизируются в рамках лексической сис\-те\-мы языка в толковых, 
семантических, идеографических словарях, а~так\-же в~циф\-ро\-вых сетях, 
вклю\-чая квантовые.
    
    Формализация знаний с по\-мощью концептуальной схемы может 
осуществляться в~виде лексических онтологий, например WordNet. Этот 
ресурс является одним из способов пред\-став\-ле\-ния знаний на основе 
установления семантических отношений меж\-ду понятиями с по\-мощью 
составления синонимических рядов (синсетов) со\-от\-вет\-ст\-ву\-ющей час\-ти речи, 
которые, в свою очередь, так\-же связаны меж\-ду собой разнообразными 
семантическими отношениями (гиперонимия, меронимия, антонимия 
и~т.\,д.)~\cite{4-an}.
    
    От логической модели построения баз данных отличается подход, 
представленный методом HAL (Hyperspace Analogue to Language). При 
построении этого языка считается, что любой языковой знак находится 
в~контексте, а также учитывается линейный характер текста и бли\-зость слов 
друг другу в его линейной развертке. Речь идет о построении базы данных 
связей слов с учетом их непосредственного словесного окружения~\cite{5-an}. 
Авторы этой работы вводят так называемое семантическое пространство, 
в~котором смысл словосочетаний отображается при помощи специального 
языка HAL.
    
    В языке HAL слова распределяются не так, как в~словаре обычного языка--- 
последовательно, например просто в алфавитном порядке, а~в~виде\linebreak  
мат\-ри\-цы (таблицы), где слова распределены по вертикали в~самом левом 
столб\-це по алфавиту, как и~в~обычном словаре. Обычный словарь в~этом 
смыс\-ле является аналогом векторного пространства. В~языке HAL в~строках 
каждое слово размещается в пар\-ных сочетаниях с~другими словами, что должно 
отражать ка\-кую-то связь этого отдельного слова с~остальными словами. Будучи 
расположенными в мат\-ри\-це, слова в~языке HAL имеют как бы тензорный 
характер, т.\,е.\ относятся к некоему гиперпространству. При этом слова 
в~мат\-ри\-це располагаются тем ближе, чем они дальше друг от друга  
в~рас\-смат\-ри\-ва\-емом текс\-те. Аналогия с языком в методе HAL заключается 
в~том, что каж\-дой точке гиперпространства можно сопоставить то или иное 
словосочетание языка рас\-смат\-ри\-ва\-емо\-го текста.
    
    Таким путем эмпирический подход языка HAL позволяет выявить 
взаимосвязь слов. Но эта связь может быть и чисто формальной, следствием 
случайных совпадений. Если такая связь является изотопической, то она имеет 
смысловой характер. 
    
    Авторы работы~\cite{6-an} такую связь между парой слов в том или ином 
текс\-те предложили искать целенаправленно, используя квантовые алгоритмы 
при записи слов в текс\-те при помощи языка HAL. При этом квантовые 
алгоритмы реализовывались чис\-то математически с использованием известных 
в~квантовой механике абстрактных формул. В~интерпретации языка HAL 
в~работе~\cite{6-an} квантовое векторное со\-сто\-яние документа определяется 
как $\vert \Psi\rangle \hm= \sum\nolimits_i^N \vert w_{n_i}\rangle$, где $\vert 
w_{ni}\rangle$~--- $i$-е собственное со\-сто\-яние оператора некоторой 
величины~$n$, со\-от\-вет\-ст\-ву\-ющее слову~$i$, т.\,е.\ это вектор, яв\-ля\-ющий\-ся 
суммой векторов отдельных слов.
    
    Рассмотрим возможность физической реализации такого подхода 
с~использованием конкретных фотонных квантовых со\-сто\-яний $\vert 
w_{v_i}\rangle\hm=\sum\nolimits_j^M \vert w_{k_{ij}}\rangle$ как квантовых 
со\-сто\-яний слова~$i$, которые характеризуются час\-то\-той фотона~$v_i$ и его 
вол\-но\-вым вектором~$k_{ij}$. Можно представить эти со\-сто\-яния как сумму 
проекций нормированного со\-сто\-яния фотона $\vert u_{v_{ij}}\rangle \hm= 
a^{(i)}_{\vec{k}_j} \vert u_{\vec{k}_j}\rangle$  на базовые со\-сто\-яния 
на\-прав\-ле\-ний его волнового век\-то\-ра, лежащих в той или иной плос\-кости: 
$$
\vert  u_{v_i}\rangle = \sum\limits_j^M a^{(i)}_{\vec{k}_j} \vert 
u_{\vec{k}_j}\rangle.
$$
    
    Двум словам~$A$ и~$B$ из текста можно сопоставить векторные 
состояния $\vert u_{v_A}\rangle$ и~$\vert u_{v_B}\rangle$  и~об\-щую плос\-кость, 
проходящую через эти векторы. Тогда векторные со\-сто\-яния документа в базисе 
этих слов мож\-но определить как векторную сумму проекций состояния 
документа на со\-сто\-яния этих слов в данной плос\-кости с последующим 
поворотом на~90$^\circ$ относительно ортогональных к векторам слов осей 
и~проекции на оси в плос\-кости, ортогональной векторам слов:
    \begin{equation}
    \left.
    \begin{array}{rl}
    \vert \Psi_A\rangle &= \fr{1}{\sqrt{2}}\left( \alpha_{\sigma_+}\vert 
u_{v_A,\sigma_+} \rangle +\alpha_{\sigma_-}\vert u_{v_A,\sigma_-}\rangle\right);\\[6pt]
%    \label{e1-an}
    \vert \Psi_B\rangle &= \fr{1}{\sqrt{2}}\left( \beta_{\sigma_+}\vert 
u_{v_B,\sigma_+} \rangle +\beta_{\sigma_-}\vert u_{v_B,\sigma_-}\rangle\right).
\end{array}
\right\}
    \label{e2-an}
    \end{equation}
    
    Векторы~(\ref{e2-an})  являются по своей фор\-ме 
поляризационными фотонными кубитами.
    
    Коэффициенты в выражениях~(\ref{e2-an}) мож\-но записать 
как
    \begin{equation}
    \left.
    \begin{array}{rl}
    \alpha_{\sigma_+} &= \fr{\langle u_{v_{A,\sigma_+}}\vert \Psi\rangle} 
{\sqrt{\langle u_{v_{A,\sigma_+}}\vert \Psi\rangle^2}+\langle u_{v_{A,\sigma_-
}}\vert \Psi\rangle^2}\,; %\label{e3-an}
\\[6pt]
    \alpha_{\sigma_-} &= \fr{\langle u_{v_{A,\sigma_-}}\vert \Psi\rangle} 
{\sqrt{\langle u_{v_{A,\sigma_+}}\vert \Psi\rangle^2}+\langle u_{v_{A,\sigma_-
}}\vert \Psi\rangle^2}\,;
\end{array}
\right\}
\label{e4-an}
\end{equation}
\begin{equation}
\left.
\begin{array}{rl}
    \beta_{\sigma_+} &= \fr{\langle u_{v_B,\sigma_+}\vert \Psi\rangle} 
{\sqrt{\langle u_{v_B,\sigma_+}\vert \Psi\rangle^2}+\langle u_{v_B,\sigma_-}\vert 
\Psi\rangle^2}\,;\\[6pt]
%    \label{e5-an}\\
    \beta_{\sigma_-} &= \fr{\langle u_{v_B,\sigma_-}\vert \Psi\rangle} 
{\sqrt{\langle u_{v_B,\sigma_+}\vert \Psi\rangle^2}+\langle u_{v_B,\sigma_-}\vert 
\Psi\rangle^2}\,.
\end{array}
\right\}
    \label{e6-an}
    \end{equation}
    
    Состояния вида~(\ref{e2-an}) позволяют связать каж\-дое 
слово с тем или иным квантовым битом (кубитом) информации. Определим 
теперь операторы запроса контекстного поиска на фотонах с учетом того, что 
поляризации фотонов можно ассоциировать с их спиновыми состояниями. 
Оператор прямого значения слова соответствует оператору $z$-про\-ек\-ции 
спина: 
    \begin{align*}
    \hat{A}\vert \Psi_A\rangle &= \hat{S}_{Az} \vert \Psi_A\rangle 
={}\notag\\
&{}=\fr{1}{\sqrt{2}}\left( \alpha_{\sigma_+} \vert u_{v_A,\sigma_+}\rangle -
\alpha_{\sigma_-} \vert u_{v_A,\sigma_-}\rangle\right);
   % \label{e7-an}
   \\
    \hat{B}\vert \Psi_B\rangle &= \hat{S}_{Bz} \vert \Psi_B\rangle 
={}\notag\\
&{}=\fr{1}{\sqrt{2}}\left( \beta_{\sigma_+} \vert u_{v_B,\sigma_+}\rangle -
\beta_{\sigma_-} \vert u_{v_B,\sigma_-}\rangle\right).
  %  \label{e8-an}
    \end{align*}
    
    Оператор запроса противоположного значения можно определить как
    \begin{multline*}
    \hat{A}_x\vert \Psi_A\rangle = \hat{S}_{Ax} \vert \Psi_A\rangle 
={}\\
{}= \fr{1}{\sqrt{2}}\left( \alpha_{\sigma_-}\vert u_{v_A,\sigma_+}\rangle 
+\alpha_{\sigma_+}\vert u_{v_B,\sigma_-}\rangle\right).
  %  \label{e9-an}
    \end{multline*}
    
    Белловский параметр поиска можно записать через матричные элементы 
операторов запроса по известной из работы~\cite{6-an} формуле:
    \begin{multline*}
    S_{\mathrm{query}}=\left\vert \left \langle \hat{A}\hat{B}_+\right\rangle_\Psi 
+\left\langle \hat{A}_x\hat{B}_+\right\rangle_\Psi\right\vert +{}\\
{}+
    \left\vert \left\langle \hat{A}\hat{B}_-\right\rangle_\Psi -\left\langle 
\hat{A}_x\hat{B}_-\right\rangle_\Psi\right\vert\,,
    %\label{e10-an}
    \end{multline*}
где $\hat{B}_+= -(\hat{B}+\hat{B}_x)$; $\hat{B}_-\hm= \hat{B}\hm- \hat{B}_x$.
    
    Простое вычисление дает
    \begin{multline}
    S_{\mathrm{query}}= \fr{1}{2} \left\{ \left\vert \left( \alpha^2_{\sigma_+} 
+2\alpha_{\sigma_+} \alpha_{\sigma_-}-\alpha^2_{\sigma_-}\right)\right\vert 
+{}\right.\\
    \left.{}+
    \left\vert \left( \alpha^2_{\sigma_+} -2\alpha_{\sigma_+}\alpha_{\sigma_-} - 
\alpha^2_{\sigma_-}\right)\right\vert \right\} 
    \left\vert \left(\beta^2_{\sigma_+} +{}\right.\right.\\
   \left.\left. {}+ 2\beta_{\sigma_+} \beta_{\sigma_-} -
\beta^2_{\sigma_-}\right)\right\vert\,.
    \label{e11-an}
    \end{multline}
                
    
    Вычислив коэффициенты $\alpha_{\sigma_+}$, $\alpha_{\sigma_-}$, 
$\beta_{\sigma_+}$ и~$\beta_{\sigma_-}$ по  
формулам~(\ref{e4-an}) и~(\ref{e6-an}), можно установить значения слов по 
матричным элементам этих операторов. Также можно вычислить белловский 
параметр~$S_{\mathrm{query}}$, величина которого определяет степень пе\-ре\-пу\-тан\-ности 
состояний документа по словам~$A$ и~$B$. Перепутанность состояний 
означает, что со\-сто\-яния связаны между собой путем взаимодействия через 
ка\-кие-ли\-бо другие со\-сто\-яния. Поэтому таким путем мож\-но установить наличие 
смысловой связи между выбранными словами в~этом до\-ку\-менте. 
    
    В работе~\cite{6-an} такие вычисления проведены на обычном 
компьютере. Но можно и по\-стро\-ить автономное вы\-чис\-ли\-тель\-ное устройство, 
работающее, например, на час\-ти\-цах света~--- фотонах. Такое устройство будет 
обладать повышенным быст\-ро\-дей\-ст\-ви\-ем как за счет предельно высокой 
ско\-рости и безынер\-ци\-он\-ности фотонов, так и за счет кван\-то\-вой па\-рал\-лель\-ности 
ис\-поль\-зу\-емых алгоритмов.
    
    В этом устройстве можно определить коэффициенты в состояниях 
документа, вычисляя скалярные произведения  
в~формулах~(\ref{e4-an}) и~(\ref{e6-an}) при помощи классического оптического 
процессора. Особенно удоб\-но использовать голографические 
процессоры~\cite{7-an, 8-an}. Эти процессоры позволяют записывать результат 
скалярного произведения векторов при помощи интерференции фотонов, 
а~затем получать результат с использованием счи\-ты\-ва\-юще\-го поля. После 
определения значения коэффициентов можно вы\-чис\-лить значение па\-ра\-мет\-ра~$S_{\mathrm{query}}$ 
по формуле~(\ref{e11-an}). Так\-же мож\-но методами квантовой 
информатики сгенерировать состояния двух фотонов, со\-от\-вет\-ст\-ву\-ющих 
словам~$A$ и~$B$ и провести измерение па\-ра\-мет\-ра $S_{\mathrm{query}}$ по 
стандартным схемам работ~\cite{9-an, 10-an, 11-an}. 
    
    Итак, в данной работе путем использования фотонных со\-сто\-яний найден 
конкретный способ реализации квантового алгоритма контекстного поиска, 
позволяющий искать изотопию, т.\,е.\ общий\linebreak семантический признак, 
свя\-зы\-ва\-ющий понятия. При поисковом запросе установление связи (выявление 
изотопии) меж\-ду понятиями может осуществляться в текстах различного 
характера (текс\-ты, относящиеся к одной терминологической об\-ласти; текс\-ты 
раз\-ных тематических областей). Подход работы~\cite{6-an} позволяет 
определить, относится ли текст к определенному вопросу, путем введения при 
поиске пары слов. Так, если искать информацию о~политическом скандале  
<<Иран--конт\-рас>>, то, понимая, что в это время Рейган был президентом 
Соединенных Штатов, причастных к скандалу, можно ввести при запросе пару 
слов Рей\-ган--Иран. Если параметр поиска покажет перепутанность со\-сто\-яний, 
соответствующих этим словам, то это будет означать, что рас\-смат\-ри\-ва\-емый 
текст соответствует теме запроса, т.\,е.\ задача найти нуж\-ный текст решена.
    
    Характер текстов, а также характер запроса пользователя (например, 
определение значения тер\-ми\-на-нео\-ло\-гиз\-ма посредством сравнения его 
с~термином, относящимся к той же терминологии, или поиск двух явно не 
связанных друг с~другом понятий) влияет и~на определение изотопии 
(семантической связи) между этими понятиями: гипонимия, гиперонимия 
(родовидовые отношения), метафора, антонимия. Таким образом, можно 
установить семантический признак, свя\-зы\-ва\-ющий понятия. Полученные 
данные могут быть использованы для создания как толковых словарей, так 
и~специализированных словарей (тезаурусов) в~той или иной об\-ласти 
в~зависимости от характера использованных текс\-тов. Они могут применяться 
в~сис\-те\-мах поиска~\cite{12-an, 13-an} и~сис\-те\-мах автоматизированного 
перевода~\cite{14-an, 15-an}.
    
{\small\frenchspacing
 {%\baselineskip=10.8pt
 %\addcontentsline{toc}{section}{References}
 \begin{thebibliography}{99}
%\bibitem{1-an}
%\Au{Greimas A.\,J.} S$\acute{\mbox{e}}$mantique structurale. Recherche de 
%m$\acute{\mbox{e}}$thode.~--- Paris: Larousse, 1966. 262~p.
%\bibitem{2-an}
%\Au{Rastier F.} Le d$\acute{\mbox{e}}$veloppement du concept d'isotopie~// Actes 
%S$\acute{\mbox{e}}$miotiques Documents, 1981. Vol.~3. No.\,29. 48~p.
%\bibitem{3-an}
%\Au{Величковский Б.\,М.} Когнитивная наука: Основы психологии познания: в 2~т.~--- М.: 
%Смысл; Академия, 2006. Т.~2. 432~с.
\bibitem{4-an}
\Au{Усталов Д.} Семантические сети и обработка естественного языка~// Открытые системы. 
СУБД, 2017. №\,2. С.~46--47. {\sf https://www.osp.ru/os/2017/ 02/13052229}.
\bibitem{5-an}
\Au{Lund K., Burgess~C.} Producing high-dimensional semantic spaces from lexical  
co-occurrence~// Behav. Res. Meth. Ins.~C., 1996. Vol.~28.  
P.~203--208.
\bibitem{6-an}
\Au{Barros J., Toffano~Z., Meguebli~Y., Doan B.-L.} Contextual query using bell tests~//  
Quantum interaction~~/
Eds. H.~Atmanspacher, E.~Haven, K.~Kitto, D.~Raine.~--- Lecture notes in computer 
science ser.~--- Springer, 2013. Vol.~8369. P.~110--121.
\bibitem{7-an} %4
\Au{Yariv A.} Phase conjugate optics and real-time holography~// IEEE J.~Quantum Elect., 
1978. Vol.~QE-14. No.\,9. P.~650--660.
\bibitem{8-an}
\Au{Dolev S., Fandina~N., Rosen~J.} Holographic parallel processor for calculating Kronecker 
product~// Nat. Comput., 2015. Vol.~14. P.~433--436.
\bibitem{9-an}
\Au{Clauser J.\,F., Horne~M.\,A., Shimony~A., Holt~R.\,A.} Proposed experiment to test local 
hidden-variable theories~// Phys. Rev. Lett., 1969. Vol.~23. No.\,15. P.~880--884.
\bibitem{10-an}
\Au{Freedman S.\,J., Clauser~J.\,F.} Experimental test of local hidden-variable theories~// Phys. 
Rev. Lett., 1972. Vol.~28. No.\,14. P.~938--941.
\bibitem{11-an}
\Au{Tanji H., Simon~J., Ghosh~S., Vuletic~V.} Simplified measurement of the Bell parameter 
within quantum mechanics~// arXiv.org, 2008. arXiv:0801.4549 [quant-ph].
\bibitem{12-an}
\Au{Beltran L., Geriente~S.} Quantum entanglement in corpuses of documents~// Found. 
Sci., 2019. Vol.~24. P.~227--246.
\bibitem{13-an}
\Au{Бессмертный И.\,А., Васильев~А.\,В., Королева~Ю.\,А., Платонов~А.\,В., 
Полещук~Е.\,А.} Методы квантового формализма в информационном поиске и обработке 
текстов на естественных языках~// Изв. вузов. Приборостроение, 2019. Т.~62. №\,8.  
С.~702--709.
\bibitem{14-an}
\Au{Wang C., Seneff~S.} High-quality speech-to-speech translation for computer-aided language 
learning~// ACM Transactions Speech Language Processing, 2006. Vol.~3. No.\,2. P.~1--21.
\bibitem{15-an}
\Au{Jia Ye., Weiss R.\,J., Biadsy~F., Macherey~W., Johnson~M., Chen~Z., Wu~Y.}  
Direct speech-to-speech translation with a sequence-to-sequence model~// arXiv.org, 2019. 
arXiv:\linebreak 1904.06037v2 [cs.CL].
\end{thebibliography}

 }
 }

\end{multicols}

\vspace*{-9pt}

\hfill{\small\textit{Поступила в~редакцию 12.03.20}}

\vspace*{6pt}

%\pagebreak

%\newpage

%\vspace*{-28pt}

\hrule

\vspace*{2pt}

\hrule

\vspace*{-2pt}

\def\tit{CONTEXT QUERY ON~PHOTONS WITH THE~USE OF~BELL TESTS}


\def\titkol{Context query on~photons with~the~use of~Bell tests}


\def\aut{\fbox{S.\,N.~Andrianov}$^1$, N.\,S.~Andrianova$^2$, F.\,M.~Ablaev$^2$, and~Yu.\,Yu.~Kochneva$^1$}

\def\autkol{S.\,N.~Andrianov, N.\,S.~Andrianova, F.\,M.~Ablaev, and~Yu.\,Yu.~Kochneva}

\titel{\tit}{\aut}{\autkol}{\titkol}

\vspace*{-11pt}


\noindent
$^1$Institute of Applied Research, Tatarstan Academy of Sciences, 36~Levobulachnaya Str., Kazan 420011, 
Russian\linebreak
$\hphantom{^1}$Federation

\noindent
$^2$Kazan Federal University, 18~Kremlyovskaya Str., Kazan 420008, Russian Federation

\def\leftfootline{\small{\textbf{\thepage}
\hfill INFORMATIKA I EE PRIMENENIYA~--- INFORMATICS AND
APPLICATIONS\ \ \ 2022\ \ \ volume~16\ \ \ issue\ 1}
}%
 \def\rightfootline{\small{INFORMATIKA I EE PRIMENENIYA~---
INFORMATICS AND APPLICATIONS\ \ \ 2022\ \ \ volume~16\ \ \ issue\ 1
\hfill \textbf{\thepage}}}

\vspace*{1pt} 




\Abste{The possibilities for physical realization of contextual query on photons 
in an optical processor using Bell tests are considered. To solve this problem, 
context coding of words in documents on quantum states of single photons using the 
well-known method of hyperspace analog language is proposed. Analytical expressions 
for determination of parameters of contextual query by a~pair of words that can be 
bound or not bound depending on the value of this parameter were obtained. 
Most connected is quantum entangled state of document wave functions chosen by 
a~pair of words that corresponds to a~certain value of the contextual query parameter. 
The suggested methods of realization of semantic contextual query are necessary 
for determination of nonlocal context that is demanded for acquiring better 
understanding during automated search and machine translation. 
The second word in the pair of query words clarifies the meaning of the 
first word through their semantic connection.}

\KWE{hyponym; hyperonym; isotopy; genus-species relations; 
photonic qubit; entangled state; Bell test; holographic processor}

\DOI{10.14357/19922264220103}

\vspace*{-24pt}

\Ack

\vspace*{-6pt}

\noindent
The research of F.\,M.~Ablaev was funded by the subsidy allocated to Kazan 
Federal University for the state assignment in the sphere of scientific activities, 
project No.\,0671-2020-0065.



%\vspace*{-6pt}

  \begin{multicols}{2}

\renewcommand{\bibname}{\protect\rmfamily References}
%\renewcommand{\bibname}{\large\protect\rm References}

{\small\frenchspacing
 {%\baselineskip=10.8pt
 \addcontentsline{toc}{section}{References}
 \begin{thebibliography}{99}
 
 \vspace*{-1pt}
 
%\bibitem{1-an-1}
%\Aue{Greimas, A.\,J.} 1966. \textit{S$\acute{\mbox{e}}$mantique structurale. Recherche de 
%m$\acute{\mbox{e}}$thode}. Paris: Larousse. 262~p.
%\bibitem{2-an-1}
%\Aue{Rastier, F.} 1981. Le d$\acute{\mbox{e}}$veloppement du concept d'isotopie. \textit{Actes 
%S$\acute{\mbox{e}}$miotiques Documents} 3(29):1--48.
%\bibitem{3-an-1}
%\Aue{Velichkovskii, B.\,М.} 2006. \textit{Kognitivnaya nauka: Osnovy psikhologii poznaniya} [Cognitive 
%science: Basics of knowledge psychology]. Moscow: Smysl; Akademiya. Vol.~2. 432~p.
\bibitem{4-an-1}
\Aue{Ustalov, D.} 2017. Semanticheskie seti i~obrabotka estestvennogo yazyka [Semantic nets and natural 
language processing].
\textit{Otkrytye sistemy. SUBD} [Open Systems. DBMS] 2:46--47.
 Available at: {\sf https://www.osp.ru/os/ 2017/02/13052229/} (accessed December~22, 
2021).
\bibitem{5-an-1}
\Aue{Lund, K., and C.~Burgess.} 1996. Producing high-dimensional semantic spaces from lexical  
co-occurrence. \textit{Behav. Res. Meth. Ins.~C.} 28:203--208.
\bibitem{6-an-1}
\Aue{Barros, J., Z.~Toffano, Y.~Meguebli, and B.-L.~Doan.} 2013. Contextual query using Bell tests. 
\textit{Quantum interaction}. Eds.\ H.~Atmanspacher, E.~Haven, K.~Kitto, and D.~Raine.
 Lecture notes in computer science ser. 
Springer. 8369:110--121.
\bibitem{7-an-1}
\Aue{Yariv, A.} 1978. Phase conjugate optics and real-time holography. \textit{IEEE J.~Quantum 
Elect.} QE-14(9):650--660.
\bibitem{8-an-1}
\Aue{Dolev, S., N.~Fandina, and J.~Rosen.} 2015. Holographic parallel processor for calculating Kronecker 
product. \textit{Nat. Comput.} 14:433--436.
\bibitem{9-an-1}
\Aue{Clauser, J.\,F., M.\,A.~Horne, A.~Shimony, and R.\,A.~Holt.} 1969. Proposed experiment to test local 
hidden-variable theories. \textit{Phys. Rev. Lett.} 23(15):880--884.
\bibitem{10-an-1}
\Aue{Freedman, S.\,J., and J.\,F.~Clauser.} 1972. Experimental test of local hidden-variable theories. 
\textit{Phys. Rev. Lett.} 28(14):938--941.
\bibitem{11-an-1}
\Aue{Tanji, H., J.~Simon, S.~Ghosh, and V.~Vuletic.} 2008. Simplified measurement of the Bell parameter 
within quantum mechanics. \textit{arXiv.org}. Available at: {\sf https://arxiv.org/abs/0801.4549} (accessed 
December~22, 2021).
\bibitem{12-an-1}
\Aue{Beltran, L., and S.~Geriente.} 2019. Quantum entanglement in corpuses of documents. 
\textit{Found. Sci.} 24:227--246.
\bibitem{13-an-1}
\Aue{Bessmertnyi, I.\,А., А.\,V.~Vasiliev, Yu.\,А.~Koroleva, А.\,V.~Platonov, and Е.\,А.~Poleschuk.} 
2019. Metody kvantovogo formalizma v~informatsionnom poiske i~obrabotke tekstov na estestvennykh 
yazykakh [Quantum formalism methods in information retrieval and processing of texts on natural languages]. 
\textit{Izvestiya vysshikh uchebnykh zavedeniy. Priborostroenie} [J.~Instrument Engineering]  
62(8):702--709.
\bibitem{14-an-1}
\Aue{Wang, C., and S.~Seneff.} 2006. High-quality speech-to-speech translation for computer-aided 
language learning. \textit{ACM Transactions Speech Language Processing} 3(2):1--21.
\bibitem{15-an-1}
\Aue{Jia, Ye., R.\,J.~Weiss, F.~Biadsy, W.~Macherey, M.~Johnson, Z.~Chen, and Y.~Wu.} 2019. Direct 
speech-to-speech translation with a sequence-to-sequence model. \textit{arXiv.org}. Available at: {\sf 
https://arxiv.org/abs/1904.06037} (accessed December~22, 2021).
\end{thebibliography}

 }
 }

\end{multicols}

\vspace*{-12pt}

\hfill{\small\textit{Received March 12, 2020}}

\pagebreak

%\vspace*{-18pt}

\Contr

\noindent
\textbf{Andrianov Sergey N.} (1959--2020)~--- Doctor of Science in physics and mathematics, principal 
scientist, Institute of Applied Research, Tatarstan Academy of Sciences, 36~Levobulachnaya Str., Kazan 
420011, Russian Federation

\vspace*{3pt}

\noindent
\textbf{Andrianova Nataliya S.} (b.\ 1983)~--- Candidate of Science (PhD) in philology, associate professor, 
Department of Theory and Practice of Teaching Foreign Languages, Institute of Philology and Intercultural 
Communication, Kazan Federal University, 18~Kremlyovskaya Str., Kazan 420008, Russian Federation; 
\mbox{natalia\_an83@mail.ru} 

\vspace*{3pt}

\noindent
\textbf{Ablaev Farid M.} (b.\ 1953)~--- Doctor of Science in physics and mathematics, professor, Head of 
Department of Theoretical Cybernetics, Institute of Computational Mathematics and Information 
Technologies, Kazan Federal University, 18~Kremlyovskaya Str., Kazan 420008, Russian Federation; 
\mbox{fablayev@gmail.com}

\vspace*{3pt}

\noindent
\textbf{Kochneva Yulia Yu.} (b.\ 1985)~--- scientist, Institute of Applied Research, Tatarstan Academy of 
Sciences, 36~Levobulachnaya Str., Kazan 420011, Russian Federation; \mbox{instpianrt@gmail.com}




\label{end\stat}

\renewcommand{\bibname}{\protect\rm Литература} 