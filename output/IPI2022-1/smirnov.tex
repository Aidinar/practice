
\def\stat{smirnov}

\def\tit{ПЕРСОНАЛЬНЫЙ КОГНИТИВНЫЙ АССИСТЕНТ: ПЛАНИРОВАНИЕ ПОВЕДЕНИЯ\\ 
НА~ОСНОВЕ СЦЕНАРИЕВ ДЕЯТЕЛЬНОСТИ$^*$}

\def\titkol{Персональный когнитивный ассистент: планирование поведения 
на~основе сценариев деятельности}

\def\aut{И.\,В.~Смирнов$^1$, А.\,И.~Панов$^2$, А.\,А.~Чуганская$^3$, 
М.\,И.~Суворова$^4$, Г.\,А.~Киселёв$^5$,\\ И.\,А.~Курузов$^6$, 
О.\,Г.~Григорьев$^7$}

\def\autkol{И.\,В.~Смирнов, А.\,И.~Панов, А.\,А.~Чуганская и~др.} 
%М.\,И.~Суворова$^4$, Г.\,А.~Киселёв$^5$, И.\,А.~Курузов$^6$,  О.\,Г.~Григорьев$^7$}

\titel{\tit}{\aut}{\autkol}{\titkol}

\index{Смирнов И.\,В.}
\index{Панов А.\,И.}
\index{Чуганская А.\,А.}
\index{Суворова М.\,И.}
\index{Киселёв Г.\,А.}
\index{Курузов И.\,А.}
\index{Григорьев О.\,Г.}
\index{Smirnov I.\,V.}
\index{Panov A.\,I.}
\index{Chuganskaya A.\,A.}
\index{Suvorova M.\,I.}
\index{Kiselev G.\,A.}
\index{Kuruzov I.\,A.}
\index{Grigoriev O.\,G.}


{\renewcommand{\thefootnote}{\fnsymbol{footnote}} \footnotetext[1]
{Работа выполнена при финансовой поддержке РФФИ (проект 18-29-22027).
}}

\renewcommand{\thefootnote}{\arabic{footnote}}
\footnotetext[1]{Федеральный исследовательский центр <<Информатика и~управление>> Российской академии наук; 
Российский университет дружбы народов, \mbox{ivs@isa.ru}}
\footnotetext[2]{Федеральный исследовательский центр <<Информатика и~управление>> Российской академии наук; 
Московский физико-тех\-ни\-че\-ский институт (национальный исследовательский университет), \mbox{pan@isa.ru}}
\footnotetext[3]{Федеральный исследовательский центр <<Информатика и~управление>> Российской академии наук, 
%Российский университет дружбы народов, 
\mbox{anfisa.makh@gmail.com}}
\footnotetext[4]{Федеральный исследовательский центр <<Информатика и~управление>> Российской академии наук, 
\mbox{suvorova@isa.ru}}
\footnotetext[5]{Федеральный исследовательский центр <<Информатика и~управление>> Российской академии наук; 
Российский университет дружбы народов, \mbox{kiselev@isa.ru}}
\footnotetext[6]{Московский физико-технический институт (национальный исследовательский университет), 
\mbox{kuruzov2014@mail.ru}}
\footnotetext[7]{Федеральный исследовательский центр <<Информатика и~управление>> Российской академии наук, 
\mbox{oleggpolikvart@yandex.ru}}

\vspace*{-6pt}

  \Abst{Представлены процедуры планирования поведения когнитивного ассистента (КА) на 
основе сценариев~--- обобщенных схем решения задач. Когнитивный ассистент является 
виртуальным интеллектуальным агентом, обладающим своей собственной картиной мира 
и~строящим картину мира пользователя, которому он помогает решать различные 
повседневные или профессиональные задачи. Ключевой компонентой целенаправленного 
поведения КА являются сценарии~--- неоднократно используемые 
абстрактные последовательности действий и~ситуаций, на основе которых ассистент 
порождает конкретный план действий для пользователя. Рассмотрено понятие сценария 
в~психологической и~лингвистической интерпретации, рассмотрена возможность извлечения 
сценариев из текстов, выполнена формализация сценария и~плана поведения на основе 
знакового подхода к~представлению знаний, предложены методы синтеза плана поведения. 
Рассмотрен модельный пример синтеза плана поведения для задачи ассистирования при 
покупке автомобиля.}
  
  \KW{когнитивный ассистент; сценарий деятельности; планирование поведения}
  
\DOI{10.14357/19922264220107}
  
\vspace*{-6pt}


\vskip 10pt plus 9pt minus 6pt

\thispagestyle{headings}

\begin{multicols}{2}

\label{st\stat}
  
\section{Введение}

  Настоящая работа развивает концепцию персонального КА, 
  представленную в~статье~\cite{1-sm}. 
  %
  Когнитивный ассистент 
является виртуальным интеллектуальным агентом, обладающим знаковой 
\mbox{картиной} мира и~действующим на основе по\-стро\-ения плана действий 
и~сценария поведения пользователя, которому он помогает решать 
повседневные или профессиональные задачи. 
%
Когнитивный\linebreak ас\-сис\-тент 
действует проактивно, предсказывая поведение пользователя, обладает 
целеполаганием и~инструментами мотивационного характера, 
обес\-пе\-чи\-ва\-ющи\-ми достижение поставленных перед ним целей. 
  
  Ключевой компонентой целенаправленного поведения КА
   выступают сценарии~--- неоднократно используемые абстрактные 
последовательности действий и~ситуаций~\cite{2-sm, 3-sm}, стереотипные 
последовательности поведения в~определенных обстоятельствах. Они 
выступают обобщенными схемами решения задач, на основе которых 
ассистирующая система может предлагать конкретный план для пользователя. 
%
Знания о~возможной последовательности действий для достижения цели 
и~способность формировать из них конкретный план (алгоритм) для 
пользователя отличает КА от существующих аналогов, 
которые, как правило, выполняют только функцию голосового интерфейса 
к~различным сервисам. 
Рассмотрим далее концепцию сценариев деятельности 
и~способы порождения из них планов поведения применительно 
к~КА.

\vspace*{-9pt}

\section{Сценарий и~план поведения}

\vspace*{-2pt}

\subsection{Психологические основы концепций сценария и~плана
поведения}

\vspace*{-1pt}

  В психологических исследованиях понятие <<сценария>> поведения возникло 
для объяснения социально обусловленных нор\-ма\-тив\-но-ро\-ле\-вых 
последовательностей действий в~рамках определенной деятельности. 
Первоначально термин был предложен Э.~Берном. Американский психолог 
обозначил трансакцией единицу общения между людьми, когда 
коммуникативный стимул вызывает соответствующую реакцию~\cite{4-sm}. 
Такие взаимосвязанные последовательности образуют ритуальные, т.\,е.\ 
социально обоснованные повторяющиеся действия. <<Существенной 
особенностью и~процедур, и~ритуалов мы считаем то, что они  
стереотипны>>~\cite[с.~14]{4-sm}. Для построения ри\-ту\-ала-сце\-на\-рия 
значима принципиальная предсказуемость, когда появление одной трансакции 
с~большой вероятностью определяет появление другой. Берн назвал такую 
форму взаимоотношений людей играми, а~набор игр~--- жизненными 
сценариями, которые формируются у~детей и~определяют дальнейшие 
стратегии их поведения. 
  
  В близкой Э. Берну методологической парадигме бихевиоризма развивались 
идеи когнитивных исследований. Росс и~Нисбетт отмечают: <<В~основе 
концепции сценариев лежит представление о~том, что люди вступают 
в~пред\-ска\-зу\-емые, едва ли не ритуальные взаимодействия в~попытке 
удовлетворить свои по\-треб\-но\-сти ценою насколько воз\-мож\-но малого 
социального напряжения и~когнитивных усилий>>~\cite[с.~145]{5-sm}. 
В~таком рассмотрении сценария обозначается еще одна его значимая 
функция~--- экономия когнитивных ресурсов, что позволяет повышать 
адаптивные возможности пси\-хики.
{ %\looseness=1

}
  
  Принципиально иначе выглядит деятельностный подход к~поведению. 
В~концепции А.\,Н.~Ле\-онть\-ева~\cite{6-sm} ключевым становится анализ 
деятельности и~предпосылок ее возникновения. В~таком же методологическом 
ключе рассуждал Д.\,Н.~Узнадзе, утверждая, что поведение невозможно понять 
без мотива, соотнесения с~объективной ситуацией и~<<нуж\-ностью>> предмета 
потребности. Все эти\linebreak функции объединяет установка, которая и~определяет 
путь к~реализации цели. В~содержательном плане это дает начальную точку 
для реализации сценария и~построения плана. Узнадзе \mbox{разделяет} 
установки на два типа: индивидуальные и~<<опосредованные чужой 
объективацией>>~\cite[с.~47]{7-sm}.\linebreak Если первый возникает в~ходе 
собственной дея\-тель\-ности человека и,~как правило, на\-прав\-лен на об\-ласть 
предметных взаимодействий или характерен для ситуаций затруднения 
(встреча с~новыми явлениями для субъекта), то второй тип установок 
социально обуслов\-лен. Этот второй тип перешел <<в~достояние людей в~виде 
\textit{готовых \mbox{формул}}, не тре\-бу\-ющих более непосредственного учас\-тия 
процессов объективации. Источником, откуда черпаются такого рода формулы, 
является воспитание и~обуче\-ние>>~\cite[с.~203]{8-sm}. Таким образом, 
содержательное разграничение плана и~сценария воз\-мож\-но за счет выделения 
актуально дей\-ст\-ву\-ющей уста\-нов\-ки де\-я\-тель\-ности и~уров\-ня ее реа\-ли\-за\-ции.
{\looseness=1

}
  
  В когнитивной науке встречались подходы, которые предлагали фиксировать 
фреймовую структуру сценариев в~виде набора речевых параметров 
(М.~Минский, Ч.~Филмор и~др.). В~этом плане на современном этапе развития 
общества такими источниками могут выступать сетевые тексты, в~которых 
в~словесной форме фиксируется сценарная информация~\cite{9-sm}. 
Индивидуальные сценарии в~рамках коммуникации могут переходить 
в~социальные. В~рамках индивидуального сценария возможно построение 
плана, который реализуется в~пошаговых действиях для достижения целей 
различного \mbox{уровня}. 
  
  Таким образом, при построении данного исследования будем основываться 
на следующем определении: <<Под сценарием предлагается понимать 
вербализованное представление знаний об участниках, ключевых признаках, 
условиях, целях, способах и~этапах их достижения как компонентах типичной 
ситуации взаимодействия субъекта с~реальностью>>~\cite[с.~219]{9-sm}. При 
этом план выступает индивидуализированной формой реализации 
деятельности, ключевым моментом которой является соответствие достигаемой 
цели.
  
  \subsection{Лингвистическое моделирование сценариев и~возможности 
извлечения сценариев из~текстов}
  
   Современные методы анализа и~генерации текс\-та, ис\-поль\-зу\-емые, к~примеру, 
в~чат-бо\-тах, системах поиска и~сравнения текс\-тов, достаточно плохо 
учитывают глобальную структуру текс\-та. Один из распространенных способов 
заложить в~модель информацию о~глобальном содержании документа~--- 
описание тематики документа с~по\-мощью метода <<мешок слов>> или 
тематическое моделирование. Но при таком подходе теряется информация 
о~структуре изложения, о~взаимосвязях упо\-ми\-на\-емых в~текс\-те концептов друг 
с~другом. Методы извлечения сю\-жет\-но-ком\-по\-зи\-ци\-он\-но\-го стро\-ения текс\-та, 
к~которым относится и~сценарный анализ, позволяют преодолеть эту проб\-ле\-му 
и~существенно повысить качество решения многих задач, вклю\-чая синтез 
текс\-та и~рассуждения по текс\-ту~\cite{10-sm}. Сценарный анализ~--- это один из 
этапов анализа структуры текс\-та, к~результатам которого можно применять 
правила или другие методы решения конечной задачи.
  
  В настоящей работе предпринята попытка использования сценарного 
подхода к~анализу текс\-тов инструкций. Такие текс\-ты содержат, как правило, 
прямые наименования основных действий, со\-сто\-яний, признаков ситуации, 
в~которой ин\-струк\-ти\-ру\-емый (коллективный адресат) мыс\-лит\-ся автором текста 
как исполнитель определенной роли~\cite{11-sm}. Инструкции 
характеризуются чет\-ки\-ми и~недвусмысленными формулировками, наличием 
эксплицитной мо\-ти\-ви\-ру\-ющей со\-став\-ля\-ющей, уси\-ли\-ва\-ющей побудительную 
мо\-даль\-ность текс\-та, что важ\-но с~точ\-ки зрения минимизации поведенческой 
ва\-ри\-а\-тив\-ности~\cite{9-sm, 11-sm}.
   
   Понимание сценарного подхода авторами \mbox{статьи} близко по смыслу 
к~скриптам Шенка~--- одному из глав\-ных формализмов, опи\-сы\-ва\-ющих 
типичную последовательность событий в~мире. Однако создание скриптов~--- 
трудоемкий процесс, тре\-бу\-ющий большого руч\-но\-го труда для описания каж\-дой 
отдельной предметной об\-ласти. Подход, предложенный Чем\-бер\-сом  
и~Журафски~\cite{12-sm}, не требует никакой разметки текс\-тов, даже указания 
их тематики. Подход получил название нарративных цепочек событий 
(NarrativeEventChains). Речь идет о~час\-тич\-но упорядоченных наборах событий, 
относящихся к~одному дей\-ст\-ву\-юще\-му лицу. 
   
Пример нарративной цепочки:

\_\textit{обвинил}~$X$

$X$ \textit{утверждал, что}

$X$ \textit{заявил, что}

\_\textit{уволил} $X$
    
    В основе подхода нарративных цепочек лежит кореференция. На первом 
шаге с~помощью дистрибутивных методов определяются нарративные 
отношения между событиями (извлекаются взаимосвязанные события и~их 
участники), связанными одними кореферентными аргументами. На втором 
шаге с~по\-мощью временн$\acute{\mbox{о}}$го классификатора эти события 
час\-тич\-но упорядочиваются. На третьем шаге автономные цепочки из 
про\-стран\-ст\-ва событий отсекаются и~группируются.

\section{Синтез плана поведения когнитивным ассистентом}

  Формализация сценария как элемента картины мира представлена 
в~работе~\cite{13-sm}. Алгоритм синтеза плана поведения (АСПП) 
КА опирается на знаковый подход пред\-став\-ле\-ния 
знаний и~является одним из применений алгоритма MAP, получившего 
название на основе обозначений идентификаторов компонент знака ($M$~--- 
компонента значения; $A$~--- компонента смыс\-ла; $P$~--- компонента 
образа)~\cite{14-sm}. Суть применения алгоритма за\-клю\-ча\-ет\-ся в~синтезе плана 
поведения, рег\-ла\-мен\-ти\-ру\-юще\-го це\-ле\-на\-прав\-лен\-ную де\-я\-тель\-ность пользователя, 
в~том случае\linebreak когда деятельность обладает качествами пред\-мет\-ности 
и~си\-ту\-а\-тив\-ности. С~по\-мощью АСПП когнитивный ассистент осуществляет синтез рекомендуемых 
пользователю шагов, пред\-остав\-ля\-ющих \mbox{воз\-мож\-ность} достижения же\-ла\-емой для 
пользователя ситуации из текущего со\-сто\-яния. Результатом работы алгоритма 
становится план поведения, который пред\-став\-лен по\-сле\-до\-ва\-тель\-ностью 
кортежей
  \begin{equation}
  \left\langle \left( s_0, a_0,s_1\right), \left( s_1, a_1, s_2\right),\ldots , \left( s_{n-1}, 
a_{n-1}, s_n\right)\right\rangle
  \label{e1-sm}
  \end{equation}
таких, что $\{f_a^0\} \hm\subseteq \{ f_{s_0}\}$, $\{f_a^{\mathrm{end}}\hm\subseteq 
\{f_{s_n}\}$, где множества $\{f_a^o\}$ и~$\{f_a^{\mathrm{end}}\}$~--- множества фак\-тов 
из мира пользователя, формирующие его пред\-став\-ле\-ние о~начальной 
и~конечной ситуации;  $\{f_{s_0}\}$ и~$\{f_{s_n}\}$~--- множества фак\-тов 
в~картине мира КА, формирующие необходимые для планировщика системы 
условия синтеза плана поведения. В~формуле~(1) $s_0,\ldots , s_n$~--- знаки 
начальной, промежуточных и~целевой ситуаций, а~$a_0,\ldots , a_n$~--- 
каузальные матрицы на сети смыс\-лов знаков действий, спо\-соб\-ст\-ву\-ющих 
активации знаков ситуаций плана. Множества $\{f_{s_0}\}$ и~$\{f_{s_n}\}$ 
являются дополненными множествами $\{f_a^0\}$ и~$f_a^{\mathrm{end}}\}$, недостающие 
факты выбираются из картины мира КА на основе правил формирования 
ситуации и~требуются для создания условий применения планировщика.
  
  Каждое из действий КА является иерархическим и~имеет свой 
операциональный со\-став, способ извлечения которого опирается 
на три основные категории:
  \begin{enumerate}[(1)]
\item в~картине мира КА присутствует полное описание требуемого действия 
и~операциональный состав извлекается из образной компоненты знака действия;
  \item в процессе синтеза сценария помимо основного сценария дея\-тель\-ности 
может быть получен сценарий выполнения требуемого действия, не 
входившего ранее в~картину мира КА. В~этом случае процесс уточ\-не\-ния 
требует вызова планировщика, которому передается сценарий выполнения для 
сокращения необходимого КА времени на синтез плана с~по\-мощью АСПП;
  \item отсутствие полного описания действия в~картине мира КА требует 
вызова АСПП для удовле\-тво\-ре\-ния требований по детализации плана. Уровень 
детализации формируется по требованию пользователя или на основе работы 
алгоритма рас\-суж\-де\-ний.
  \end{enumerate}
  
  \begin{figure*} %fig1
  \vspace*{1pt}
  \begin{center}  
    \mbox{%
\epsfxsize=121.467mm
\epsfbox{smi-1.eps}
}

\end{center}
\vspace*{-2pt}

  \Caption{Синтез плана поведения на основе сценария деятельности КА}
  \vspace*{-3pt}
  \end{figure*}
  
  Для вышеупомянутых категорий характерен процесс уточнения имеющегося 
операционального со\-ста\-ва деятельности знаниями, полученными на основе 
анализа $\{f_a^0\}$ в~рамках процедуры пополнения знаний КА. В~общем виде 
план поведения на основе сценария де\-ятель\-ности пред\-став\-лен на рис.~1. Серым 
цветом отмечены кортежи ситуаций и~действий, выбранные КА в~процессе 
АСПП из множества кортежей в~рам\-ках сценария. Для каждой возможной 
по\-сле\-до\-ва\-тель\-ности действий сценария строится план. Из множества всех 
по\-стро\-ен\-ных планов выбирается наиболее подходящий для $\{ f_a^0\}$ 
и~$\{f_a^{\mathrm{end}}\}$.
  
  Алгоритм синтеза плана поведения выполнен на основе процедуры MAP\_ITERATION, со\-сто\-ящей из 
четырех основных этапов:
  \begin{enumerate}[(1)]
\item S-этап~--- поиск прецедента деятельности из текущей ситуации, 
который ранее способствовал достижению схожей целевой ситуации. Поиск 
осуществляется в~базе предактивированных прецедентных действий;
\item M-этап~--- поиск применимых действий на множестве воз\-мож\-ных 
значений. Поиск проходит на основе получения множества действий над 
объектами, присутствующими в~ситуации, и~выборе среди них применимых. 
При синтезе плана на основе име\-юще\-го\-ся сценария деятельности этот этап 
отвечает за выбор схемы действий, со\-от\-вет\-ст\-ву\-ющей рас\-смат\-ри\-ва\-емой 
ситуации;
\item A-этап~--- генерация мат\-риц действий на множестве смыс\-лов, 
со\-от\-вет\-ст\-ву\-ющих найденным значениям. Все сгенерированные действия 
эвристически оцениваются и~отбираются те, которые способствуют 
наиболее быст\-рой активации целевой ситуации;
\item P-этап~--- построение новой ситуации по множеству признаков 
условий найденных действий.
\end{enumerate}

\section{Пример модельного сценария и~плана покупки 
автомобиля}

\vspace*{-3pt}

  Рассмотрим модельный пример сценария покупки автомобиля. 
Автомобильная тематика была выбрана по причине наличия \mbox{достаточно} 
однозначного набора действий, обес\-пе\-чи\-ва\-ющих \mbox{достижение} же\-ла\-емой цели. 
Модельный сценарий реконструирован на основе текс\-тов людей, опи\-сы\-ва\-ющих 
свой опыт покупки автомобиля. В~качестве исходного набора данных был 
собран корпус из 100~текс\-тов (159\,698~словоупотреблений). В~корпус во\-шли 
инструкции о~покупке, осмотре и~оформ\-ле\-нии автомобиля на свое имя (как 
нового, так и~подержанного), т.\,е.\ текс\-ты, содержащие сценарии в~более-менее 
явном виде (рекомендации авторов читателям). 
  
  Фрагмент схемы сценария покупки автомобиля пред\-став\-лен на рис.~2. 
Данная схема применима и~для других пред\-мет\-ных областей. Она может быть 
использована на практике для задачи управ\-ле\-ния поведением 
интеллектуального агента. Центральное мес\-то в~этой структуре занимает 
сценарий. Он со\-сто\-ит из сле\-ду\-ющих элементов:
  \begin{itemize}
\item одиночное действие (на рис.~2, например, это элемент <<НоваяИлиБУ: 
ПростоеДействие>>)\\[-13pt]
\item ветвление, которое состоит из множества шагов, порядок которых не 
важен, может выполняться не полностью или вза\-и\-мо\-ис\-клю\-ча\-юще (на рис.~2 
это элемент <<ОпределениеТребований: Ветв\-ле\-ние>>)\\[-13pt]
\item цепочка шагов, т.\,е.\ прос\-тая последовательность, которая говорит, что 
и~в~каком порядке надо сделать (на рис.~2 это элемент <<ПокупкаАвто: 
ЦепочкаШагов>>).
\end{itemize}
  
  Каждый шаг в~сценарии является, по сути, отдельным вложенным сценарием.
  
  \pagebreak
  
  \end{multicols}

\begin{figure*} %fig2
  \vspace*{1pt}
  \begin{center}  
    \mbox{%
\epsfxsize=163mm
\epsfbox{smi-2.eps}
}

\end{center}
\vspace*{-4pt}

\Caption{Фрагмент схемы сценария покупки автомобиля}
\vspace*{-4pt}
\end{figure*}
  
  \begin{multicols}{2}
  
  У каждого шага в~сценарии есть также цели и~предусловия. Конечная цель~--- 
это описание ситуации, к~которой желательно прийти в~результате выполнения 
сценария. Ситуацию можно описать как набор фактов о~мире из рабочей памяти 
агента, т.\,е.\ каким мир должен стать в~результате выполнения этого шага. 
Предусловия~--- это некий ограничитель, который не позволяет нарушать 
последовательность шагов в~сценарии и~определяет тот набор фактов, при 
удовлетворении которых можно приступать к~выполнению каждого 
сле\-ду\-юще\-го шага: какая информация должна быть получена, какие решения 
приняты, какие ресурсы необходимы и~т.\,д. У~каждого шага также есть 
субъект, который его выполняет, и~операнды (автомобиль, лакокрасочное 
покрытие, водительские права и~др.). 
  
  Стрелками на рис.~2 обозначены связи (или отношения) между элементами 
сценария. Связь <<СледШаг>> задает следующий сценарий, который 
необходимо выполнить после текущего. Связь <<Варианты>> задает 
множество других сценариев, которые нужно выполнить после текущего, 
причем порядок выполнения этих сценариев не важен (и~не все они могут быть 
выполнены). Связь <<Шаги>> задает множество сценариев, которые должны 
быть выполнены строго последовательно. 

Все эти связи служат для явного 
указания последовательности действий. Однако на практике возможны 
ситуации, когда порядок действий неизвестен (к~примеру, если сценарий 
собирался КА на основе разных текс\-тов, в~которых 
содержалась неполная информация), но ка\-кие-то элементы сценария все равно 
присутствуют, например есть понимание цели. Эти факты ссылаются (связь 
<<Упоминание>>) на операнды (например, автомобиль, тип трансмиссии 
и~др.), другие цели (зачем нужен автомобиль, кого на нем возить и~куда ездить) 
и~т.\,д. Переходя по таким связям, можно находить другие сценарии 
и~действия, которые нужно выполнить для удовле\-тво\-ре\-ния текущей цели. Это 
позволяет, двигаясь с~других сторон, восстановить по\-сле\-до\-ва\-тель\-ность шагов 
сценария, выполняемого впервые.

\section{Заключение}

  Моделирование целенаправленного поведения и~сценариев де\-я\-тель\-ности~--- 
одно из новых на\-прав\-ле\-ний исследований в~об\-ласти искусственного 
интеллекта. Задача по\-стро\-ения сценариев по текстам задействует верхние 
уровни языка (дискурс, сюжет) и~становится новой актуальной проблемой для 
обработки русского языка. Использование извле\-ка\-емых из текстов сценариев 
де\-я\-тель\-ности повышает степень ра\-зум\-ности и~ав\-то\-ном\-ности интеллектуальных 
систем.
  
  Дальнейшим развитием КА будет проработка 
механизмов взаимодействия ас\-сис\-тен\-та с~пользователем на естественном языке. 
Предполагается, что ас\-сис\-тент будет предлагать пользователю дальнейшие 
шаги по решению задачи в~виде рекомендаций, генерируемых автоматически 
на основе текущего со\-сто\-яния и~предыдущих шагов в~синтезированном 
персональном плане ас\-сис\-ти\-ро\-ва\-ния. Текущее состояние (за\-вер\-шен\-ность 
текущего шага плана) также будет определяться ассистентом на основе диалога с~пользователем.
  
{\small\frenchspacing
 {%\baselineskip=10.8pt
 %\addcontentsline{toc}{section}{References}
 \begin{thebibliography}{99}
  \bibitem{1-sm}
  \Au{Смирнов И.\,В., Панов~А.\,И., Скрынник~А.\,А., Чистова~Е.\,В.} Персональный 
когнитивный ассистент: концепция и~принципы работы~// Информатика и~её применения, 
2019. Т.~13. Вып.~3. С.~105--113.
  \bibitem{2-sm}
  \Au{Осипов Г.\,С., Панов~А.\,И.} Отношения и~операции в~знаковой картине мира 
субъекта поведения~// Искусственный интеллект и~принятие решений, 2017. №\,4. С.~5--22.
  \bibitem{3-sm}
  \Au{Осипов Г.\,С., Панов~А.\,И.} Синтез рационального поведения когнитивного 
семиотического агента в~динамической среде~// Искусственный интеллект и~принятие 
решений, 2020. №\,4. С.~80--97.
  \bibitem{4-sm}
  \Au{Берн Э.} Люди, которые играют в~игры. Психология человеческой судьбы~/ Пер. 
  с~англ. А.~Грузберг.~--- М.: Эксмо, 2008. 576~с.
  (\Au{Berne~E.}  Games people play: The psychology of human relationships.~--- New York, 
NY, USA: Ballantine Books, 1973. 192~p.)
  \bibitem{5-sm}
  \Au{Росс Л., Нисбетт~Р.} Человек и~ситуация. Уроки социальной психологии~/ Пер. 
с~англ. В.\,В.~Румынского.~--- М.: Аспект Пресс, 2000. 429~с. (\Au{Ross~L., Nisbett~R.} The 
person and the situation: Perspectives of social psychology.~--- Philadelphia, PA, USA:  
Temple University Press, 1991.)
  \bibitem{6-sm}
  \Au{Леонтьев А.\,Н.} Деятельность. Сознание. Личность.~--- М.: Политиздат, 1975. 304~с.
  \bibitem{7-sm}
  \Au{Асмолов А.\,Г.} Деятельность и~установка.~--- М.: Изд-во Моск. ун-та, 1974. 150~с.
  \bibitem{8-sm}
  \Au{Узнадзе Д.\,Н.} Экспериментальные основы психологии установки.~--- Тбилиси: 
Акад. наук Груз. ССР, 1961. 210~с.
  \bibitem{9-sm}
  \Au{Кузнецова Ю.\,М., Пенкина~М.\,Ю.}Сценарий отказа от решения проблемы как 
предмет сетевых обсуждений~// Экопсихологические исследования, 2020. Т.~6. С.~218--222.
  \bibitem{10-sm}
  \Au{Суворова М.\,И., Кобозева~М.\,В., Толдова~С.\,Ю., Соколова~Е.\,Г.} Извлечение 
сценарной информации из текстов. Ч.~1: Постановка задачи и~обзор методов~// 
Искусственный интеллект и~принятие решений, 2020. №\,1. С.~17--26.
  \bibitem{11-sm}
  \Au{Мишланов В.\,А., Чуганская~А.\,А., Смирнов~И.\,В., Суворова~М.\,И., Курузов~И.\,А.} 
Разработка методов анализа сценариев поведения (на материале инструктивных  
ин\-тер\-нет-текс\-тов)~// Медиалингвистика, 2020. Т.~7. №\,1. С.~16--28.
  \bibitem{12-sm}
  \Au{Chambers N., Jurafsky~D.} Unsupervised learning of narrative event chains~// 46th Annual Meeting 
  of the Association for Computational Linguistics Proceedings.~---
  Columbus, OH, USA: Association for Computational Linguistics, 2008.
   P.~789--797.
  \bibitem{13-sm}
  \Au{Панов А.\,И.} Формирование образной компоненты знаний когнитивного агента со 
знаковой картиной мира~// Информационные технологии и~вычислительные системы, 2018. 
№\,4. С.~84--96.
  \bibitem{14-sm}
  \Au{Киселев Г.\,А.} Интеллектуальная система планирования поведения коалиции 
робототехнических агентов с~STRL архитектурой~// Информационные технологии 
и~вычислительные системы, 2020. №\,2. С.~21--37.
\end{thebibliography}

 }
 }

\end{multicols}

\vspace*{-6pt}

\hfill{\small\textit{Поступила в~редакцию 05.02.21}}

%\vspace*{8pt}

%\pagebreak

\newpage

\vspace*{-28pt}

%\hrule

%\vspace*{2pt}

%\hrule

%\vspace*{-2pt}

\def\tit{PERSONAL COGNITIVE ASSISTANT:\\ PLANNING ACTIVITY WITH~SCRIPTS}


\def\titkol{Personal cognitive assistant: Planning activity with~scripts}


\def\aut{I.\,V.~Smirnov$^{1,2}$, A.\,I.~Panov$^{1,3}$, A.\,A.~Chuganskaya$^{1}$, M.\,I.~Suvorova$^1$, 
G.\,A.~Kiselev$^{1,2}$, I.\,A.~Kuruzov$^3$, and~O.\,G.~Grigoriev$^1$}

\def\autkol{I.\,V.~Smirnov, A.\,I.~Panov, A.\,A.~Chuganskaya, et al.}
%M.\,I.~Suvorova$^1$,  G.\,A.~Kiselev$^{1,2}$, I.\,A.~Kuruzov$^3$, and~O.\,G.~Grigoriev}

\titel{\tit}{\aut}{\autkol}{\titkol}

\vspace*{-11pt}


 \noindent
    $^1$Federal Research Center ``Computer Science and Control'' of the Russian Academy of 
Sciences, 44-2~Vavilov\linebreak
$\hphantom{^1}$Str., Moscow 119333, Russian Federation
    
    
    \noindent
    $^2$Peoples' Friendship University of Russia (RUDN University), 6~Miklukho-Maklaya Str., 
Moscow 117198, Russian\linebreak
$\hphantom{^1}$Federation
  
  
    \noindent
    $^3$Moscow Institute of Physics and Technology (National Research University), 9~Institutskiy 
Per., Dolgoprudny,\linebreak
$\hphantom{^1}$Moscow Region 141701, Russian Federation

\def\leftfootline{\small{\textbf{\thepage}
\hfill INFORMATIKA I EE PRIMENENIYA~--- INFORMATICS AND
APPLICATIONS\ \ \ 2022\ \ \ volume~16\ \ \ issue\ 1}
}%
 \def\rightfootline{\small{INFORMATIKA I EE PRIMENENIYA~---
INFORMATICS AND APPLICATIONS\ \ \ 2022\ \ \ volume~16\ \ \ issue\ 1
\hfill \textbf{\thepage}}}

\vspace*{3pt} 
      
   
  
  
  \Abste{The paper presents procedures for a~cognitive assistant's behavior planning based on 
scripts~--- generalized schemes of tasks solving. A~cognitive assistant is a~virtual intelligent agent 
that has its own worldview and builds a~worldview of the user, it helps to solve various common or 
specific problems. The key component of assistant's goal-based behavior is scenario~--- a~reusable 
abstract sequence of actions and situations that are used for synthesis of a~concrete plan of actions for 
a user. The concept of a~scenario in psychological and linguistic interpretation  as well 
as the procedure of scenarios extraction from texts are considered. The scenario and the plan of behavior are 
formalized using the sign-based approach. Methods for synthesizing a~plan of behavior are 
proposed. A~test case of a behavior plan synthesis for buying a car is considered.}
  
  \KWE{cognitive assistant; activity script; behavior planning}
  
  
  
\DOI{10.14357/19922264220107}

\vspace*{-16pt}

\Ack
  \noindent
  The reported study was partially funded by the Russian Foundation for Basic Research (project 
No.\,18-29-22027).




%\vspace*{6pt}

  \begin{multicols}{2}

\renewcommand{\bibname}{\protect\rmfamily References}
%\renewcommand{\bibname}{\large\protect\rm References}

{\small\frenchspacing
 {%\baselineskip=10.8pt
 \addcontentsline{toc}{section}{References}
 \begin{thebibliography}{99}
  
  \bibitem{1-sm-1}
  \Aue{Smirnov, I.\,V., A.\,I.~Panov, A.\,A.~Skrynnik, and E.\,V.~Chistova.} 2019. Personal'nyy 
kognitivnyy assistent: kon\-tsep\-tsiya i~printsipy raboty [Personal cognitive assistant:\linebreak Concept and 
key principals]. \textit{Informatika i~ee Pri\-me\-ne\-niya~--- Inform. Appl.} 13(3):105--113.
  \bibitem{2-sm-1}
  \Aue{Osipov, G.\,S., and A.\,I.~Panov.} 2018. Relationships and operations in agent's  
sign-based model of the world. \textit{Scientific Technical Information Processing} 45(5):1--14.
  \bibitem{3-sm-1}
  \Aue{Osipov, G.\,S., and A.\,I.~Panov.} 2021. Planning rational behavior of 
cognitive semiotic agents in a~dynamic environment. \textit{Scientific Technical Information 
Processing} 48(6):502--516.
  \bibitem{4-sm-1}
  \Aue{Berne, E.} 1973. \textit{Games people play: The psychology of human relationships}. New 
York, NY: Ballantine Books. 192~p.
  \bibitem{5-sm-1}
  \Aue{Ross, L., and R.~Nisbett.} 2011. \textit{The person and the situation: Perspectives of 
social psychology}. London: McGraw-Hill. 288~p.
  \bibitem{6-sm-1}
  \Aue{Leont'ev, A.\,N.} 1975. \textit{Deyatel'nost'. Soznanie. Lichnost'} [Activity, 
consciousness, and personality]. Moscow: Po\-lit\-iz\-dat. 304~p.
  \bibitem{7-sm-1}
  \Aue{Asmolov, A.\,G.} 1974. \textit{Deyatel'nost' i~ustanovka} [Activity and attitude]. 
Moscow: Izd-vo Moskovskogo un-ta. 150~p.
  \bibitem{8-sm-1}
  \Aue{Uznadze, D.\,N.} 1961. \textit{Eksperimental'nye osnovy psikhologii ustanovki} 
[Experimental foundations of installation theory]. Tbilisi: AN GSSR. 210~p.
  \bibitem{9-sm-1}
  \Aue{Kuznetsova, Yu.\,M., and M.\,Yu.~Penkina.} 2020. Stsenariy otkaza ot resheniya 
problemy kak predmet setevykh obsuzhdeniy [Scenario of failure to solve a~problem in 
network discussions]. \textit{Ekopsikhologicheskie issledovaniya} 
[Ecopsychological Research] 6:218--222.
  \bibitem{10-sm-1}
  \Aue{Suvorova, M.\,I., M.\,V.~Kobozeva, S.\,Yu.~Toldova, and E.\,G.~Sokolova.} 2021 (in 
press). Extraction of script information from texts. Part~1: Statement of the problem and review of 
methods]. \textit{Scientific Technical Information Processing} 48.
  \bibitem{11-sm-1}
  \Aue{Mishlanov, V.\,A., A.\,A.~Chuganskaya, I.\,V.~Smirnov, M.\,I.~Suvorova, and 
I.\,A.~Kuruzov.} 2020. Razrabotka metodov analiza stsenariev povedeniya (na materiale 
instruktivnykh internet-tekstov) [Developing methods for behavior scenario analysis (on 
the material of instructional texts)]. \textit{Medialingvistika} [Media Linguistics] 7(1):16--28.
  \bibitem{12-sm-1}
  \Aue{Chambers, N., and D.~Jurafsky.} 2008. Unsupervised learning of narrative event chains. 
\textit{46th Annual Meeting of the Association for Computational Linguistics Proceedings}. 
Columbus, OH: Association for Computational Linguistics. 789--797.
  \bibitem{13-sm-1}
  \Aue{Panov, A.\,I.} 2018. Formirovanie obraznoy komponenty znaniy kognitivnogo agenta so 
znakovoy kartinoy mira [Formation of an image component of knowledge of the cognitive agent 
with a~sign-based model of worldview]. \textit{Informatsionnye tekhnologii i~vychislitel'nye 
sistemy} [J.~Information Technologies Computing Systems] 4:84--96.
  \bibitem{14-sm-1}
  \Aue{Kiselev, G.\,A.} 2020. Intellektual'naya sistema pla\-ni\-ro\-va\-niya povedeniya koalitsii 
robototekhnicheskikh agen\-tov s~STRL arkhitekturoy [Intelligent behavior planning system for 
a~coalition of robotic agents with STRL architecture]. \textit{Informatsionnye tekhnologii 
i~vychislitel'nye sistemy} [J.~Information Technologies Computing Systems] 2:21--37.
  \end{thebibliography}

 }
 }

\end{multicols}

\vspace*{-6pt}

\hfill{\small\textit{Received February 5, 2021}}

%\pagebreak

%\vspace*{-18pt}

  
  \Contr
  
  \noindent
  \textbf{Smirnov Ivan V.} (b.\ 1978)~--- Candidate of Science (PhD) in physics and 
mathematics; head of department, Institute of Artificial Intelligence Problems, Federal Research 
Center ``Computer Science and Control'' of the Russian Academy of Sciences, 9,~60-letiya 
Oktyabrya Prosp., Moscow 117312, Russian Federation; associate professor, Peoples' Friendship 
University of Russia (RUDN University), 6~Miklukho-Maklaya Str., Moscow 117198, Russian 
Federation; \mbox{ivs@isa.ru}
  
  \vspace*{3pt}
  
  \noindent
  \textbf{Panov Aleksandr I.} (b.\ 1987)~--- Candidate of Science (PhD) in physics and 
mathematics; head of department, Institute of Artificial Intelligence Problems, Federal Research 
Center ``Computer Science and Control'' of the Russian Academy of Sciences, 9,~60-letiya 
Oktyabrya Prosp., Moscow 117312, Russian Federation; deputy head of laboratory, Moscow 
Institute of Physics and Technology (National Research University), 9~Institutskiy Per., 
Dolgoprudny, Moscow Region 141701, Russian Federation; \mbox{pan@isa.ru}
  
  \vspace*{3pt}
  
  
  \noindent
  \textbf{Chuganskaya Anfisa A.} (b.\ 1985)~--- Candidate of Science (PhD) in psychology, 
researcher, Institute of Artificial Intelligence Problems, Federal Research Center 
``Computer 
Science and Control'' of the Russian Academy of Sciences, 9,~60-letiya Oktyabrya Prosp., Moscow 
117312, Russian Federation; \mbox{anfisa.makh@gmail.com}
  
  \vspace*{3pt}
  
  
  \noindent
  \textbf{Suvorova Margarita I.} (b.\ 1991)~--- researcher, Institute of Artificial Intelligence 
Problems, Federal Research Center ``Computer Science and Control'' of the Russian Academy of 
Sciences, 9,~60-letiya Oktyabrya Prosp., Moscow 117312, Russian Federation; 
\mbox{suvorova@isa.ru}
  
  \vspace*{3pt}
  
  
  \noindent
  \textbf{Kiselev Gleb A.} (b.\ 1992)~--- researcher, Institute of Artificial Intelligence Problems, 
Federal Research Center ``Computer Science and Control'' of the Russian Academy of Sciences, 
9,~60-letiya Oktyabrya Prosp., Moscow 117312, Russian Federation; assistant, Peoples' Friendship 
University of Russia (RUDN University), 6~Miklukho-Maklaya Str., Moscow117198, Russian 
Federation; \mbox{kiselev@isa.ru}
  
  \vspace*{3pt}
  
  \noindent
  \textbf{Kuruzov Ilya A.} (b.\ 1999)~--- PhD student, Moscow Institute of Physics and 
Technology (National Research University), 9~Institutskiy Per., Dolgoprudny, Moscow Region 
141701, Russian Federation; \mbox{kuruzov2014@mail.ru}
  
  \vspace*{3pt}
  
  \noindent
  \textbf{Grigoriev Oleg G.} (b.\ 1957)~--- Doctor of Science (PhD) in technology, head of 
Institute of Artificial Intelligence Research Problems, Federal Research Center ``Computer Science and Control'' 
of the Russian Academy of Sciences, 9,~60-letiya Oktyabrya Prosp., Moscow 117312, Russian 
Federation; \mbox{oleggpolikvart@yandex.ru}
  


\label{end\stat}

\renewcommand{\bibname}{\protect\rm Литература} 
  