\def\stat{bitbos}

\def\tit{О ПОИСКЕ ОПТИМАЛЬНОЙ СХЕМЫ 3D-ПЕЧАТИ КОНСТРУКЦИЙ 
ИЗ~КОМПОЗИЦИОННЫХ МАТЕРИАЛОВ}

\def\titkol{О поиске оптимальной схемы 3D-печати конструкций 
из~композиционных материалов}

\def\aut{А.\,В.~Босов$^1$, Ю.\,И.~Битюков$^2$, Г.\,Ю.~Денискина$^3$}

\def\autkol{А.\,В.~Босов, Ю.\,И.~Битюков, Г.\,Ю.~Денискина}

\titel{\tit}{\aut}{\autkol}{\titkol}

\index{Босов А.\,В.}
\index{Битюков Ю.\,И.}
\index{Денискина Г.\,Ю.}
\index{Bosov A.\,V.}
\index{Bityukov Yu.\,I.}
\index{Deniskina G.\,Yu.}


%{\renewcommand{\thefootnote}{\fnsymbol{footnote}} \footnotetext[1]
%{Работа выполнена при поддержке Министерства науки и~высшего образования Российской Федерации (проект 
%075-15-2020-799).}}


\renewcommand{\thefootnote}{\arabic{footnote}}
\footnotetext[1]{Федеральный исследовательский центр <<Информатика и~управление>> Российской академии наук; 
Московский авиационный институт, \mbox{avbosov@ipiran.ru}}
\footnotetext[2]{Московский авиационный институт, \mbox{yib72@mail.ru}}
\footnotetext[3]{Московский авиационный институт, \mbox{dega17@yandex.ru}}

%\vspace*{-6pt}





\Abst{Статья посвящена задаче поиска оптимальных траекторий укладки волокон при 
изготовлении конструкций, армированных непрерывными волокнами, методом 3D-пе\-ча\-ти. 
Предложена оптимизационная постановка, в которой целевой функцией выступает один из 
критериев разрушения композита. Схемы укладки волокон при печати моделируются  
с~по\-мощью аналитических функций, которые находятся из задачи Неймана для уравнения 
Лапласа. Краевые условия определяются на основе задания углов между волокнами и~границей  
об\-ласти. Задача Неймана решается посредством конформного преобразования области печати 
на круг. Таким образом, критерий разрушения композита становится функцией от углов, 
которые волокна образуют с~границей области. Для минимизации целевой функции 
используется генетический алгоритм поиска глобального минимума функции нескольких 
переменных.}

\KW{композиционные материалы; вейвлеты; 3D-пе\-чать; аналитическая функция}

\DOI{10.14357/19922264220102}
  
%\vspace*{-4pt}


\vskip 10pt plus 9pt minus 6pt

\thispagestyle{headings}

\begin{multicols}{2}

\label{st\stat}

\section{Введение}

  Композиционные материалы (КМ) широко применяются в~разных отраслях 
про\-мыш\-лен\-ности,\linebreak например авиационной, автомобильной. Композиционные материалы состоят из 
ар\-ми\-ру\-юще\-го и~свя\-зу\-юще\-го материалов; в~качестве ар\-ми\-ру\-юще\-го широко 
применяются углеродные волокна, \mbox{об\-ла\-да\-ющие} большой удельной проч\-ностью. 
Механические свойства изделий из КМ зависят от на\-прав\-ле\-ния этого волокна. 
Перспективным на\-прав\-ле\-нием, поз\-во\-ля\-ющим создавать конструкции\linebreak слож\-ной 
формы последовательной укладкой композиционного материала, стала технология  
3D-пе\-ча\-ти (см.\ пример такого решения в~[1]). Применение 3D-пе\-ча\-ти 
и КМ позволяет получать \mbox{конструкции} с~пространственным армированием по 
заданным траекториям. Пол\-ный контроль над расположением волокон во время 
печати позволяет укладывать их именно так, как того требуют условия 
эксплуатации: чтобы~100\% волокон шли в~нужном направлении. Для 
практической реализации, для разработки автоматизированной сис\-те\-мы 
проектирования (CAD/CAE-сис\-те\-мы) и~моделирования процесса 3D-пе\-ча\-ти 
актуальна задача поиска оптимальной схемы печати, дик\-ту\-емой условиями 
эксплуатации изделия.
  
  Как известно~[2], схема укладки волокна заложена в~самих уравнениях 
механики КМ в~виде некоторой (неизвестной) локальной ортогональной сис\-те\-мы 
координат, т.\,е.\ найти эту схему можно только из решений уравнений 
с~разными локальными системами координат. Для задания локальной сис\-те\-мы 
координат в~статье использованы аналитические функции, которые строятся на 
основе известных формул Дини и~Чизотти~[3] всего лишь заданием направления 
укладки волокна на границе изделия. В~качестве критерия оптимизации для 
выбора траекторий уклад\-ки волокон можно взять любой из критериев разрушения 
КМ~[2].
  
\section{Моделирование траекторий укладки волокон  
при~3D-печати}

  Пусть $X\subset \mathbf{R}^2$~---  об\-ласть печати. Касательные векторы 
к~кривым, по которым укладываются волокна при 3D-пе\-ча\-ти, образуют 
векторное поле в~$X$, которое будем характеризовать комплексным чис\-лом 
$r\hm= r_1\hm+ ir_2$, где $r_1\hm= r_1(x_1, x_2)$; $r_2\hm= r_2(x_1, x_2)$. Это 
поле предполагается гармоническим, т.\,е.\ соленоидальным и~потенциальным~[3], 
такое поле не имеет источников и~вих\-рей. Будем считать, что $X$~--- 
подмножество односвязной области~$\tilde{X}$: $X\hm\subset \tilde{X}$. Значит, 
выражение $-r_2dx_1\hm+ r_1dx_2$ есть полный дифференциал некоторой 
функции $v_2\hm= v_2(x_1, x_2)$, определенной на~$\tilde{X}$. Эта функция 
называется функцией тока~[3]. Кроме того, выражение $r_1dx_2\hm+ r_2dx_2$ 
также есть полный дифференциал некоторой функции $v_1\hm= v_1(x_1, x_2)$, 
которая называется потенциалом поля~[3]. Функция тока $v_2(x_1, x_2)$  
и~потенциал поля $v_1(x_1, x_2)$ являются сопряженными гармоническими 
функциями~[3]. Линии тока и~линии равного потенциала образуют ортогональное 
семейство. Аналитическая функция $v_1(x_1, x_2)\hm+ iv_2(x_1,x_2)$, $x_1\hm+ 
ix_2\hm\in \tilde{X}$, называется комплексным потенциалом поля~[3]. Таким 
образом, любая аналитическая функция в~об\-ласти~$\tilde{X}$ дает и~схему 
уклад\-ки волокон, и~локальную криволинейную систему координат в~$X\hm\subset 
\tilde{X}$. Будем далее точ\-ки $x\hm= (x_1, x_2)$ изображать на одной 
комплексной плос\-кости, а~точ\-ки $v\hm= (v_1, v_2)$~--- на другой. Тогда 
преобразование $v_1\hm=v_1(x_2, x_2)$, $v_2\hm= v_2(x_1, x_2)$ и~его обратное 
$x_1\hm= x_1(v_1, v_2)$, $x_2\hm= x_2(v_1, v_2)$ будет пред\-став\-лять собой 
преобразование множества~$\tilde{X}$ на плоскости~$x$ в~множество~$\Omega$ 
на плоскости~$v$. Сеть линий уров\-ня $v_1(x_1, x_2)\hm=const$, $v_2(x_1, 
x_2)\hm= const$ называется изотермической сетью~[3]. Кривые, по которым 
укладываются волокна, определяются па\-ра\-мет\-ри\-че\-ски\-ми пред\-став\-ле\-ни\-ями
  \begin{align*}
  \gamma_{\alpha,1}:\,& r_{\alpha,1} (v_1) 
=x_1(v_1,\alpha)+ix_2(v_1,\alpha)\,,\enskip v_1\in T_{\alpha,1}\,;\\
  \gamma_{\beta,2}:\, & r_{\beta,2} (v_2) =x_1(\beta, v_2)+ix_2(\beta, v_2)\,,\enskip 
v_2\in T_{\beta,2}\,,
  \end{align*}
где $T_{\alpha,1}$ и~$T_{\beta,2}$~--- некоторые промежутки; $\alpha, \beta \hm\in 
\mathbf{R}$; $T_{\alpha,1} \times \{\alpha\}, \{\beta\}\times 
T_{\beta,2}\hm\subset\Omega$.
  
  Поскольку функция $v_1(x_1, x_2)$ является гармонической в~односвязной  
об\-ласти~$\tilde{X}$, то она удовле\-тво\-ря\-ет уравнению Лапласа. 
Пусть~$\theta(x)$~--- угол между внеш\-ней единичной нормалью 
$$
n(x)=n_1(x)+ in_2(x)
$$ 
к~границе об\-ласти~$\tilde{X}$ и~волокном. В~рамках 
рас\-смат\-ри\-ва\-емой модели 3D-пе\-ча\-ти на\-прав\-ле\-ние волокна в~точке $x\hm\in 
\partial \tilde{X}$ задается вектором 
$$
  t(x)= \left.\fr{\partial x_1}{\partial v_1} \right\vert_{v(x)} +
  \left. i \fr{\partial x_2}{\partial v_1} 
  \right\vert_{v(x)}\,.
$$
 Учитывая, что $x_1\equiv x_1(v_1(x_1,x_2), v_2(x_1,x_2))$ 
и~$x_2\hm\equiv x_2(v_1(x_1x_2), v_2(x_1,x_2))$, а~также соотношения  
Ко\-ши--Ри\-ма\-на, получаем 
  $$
  t(x)= \fr{1}{\vert \nabla v_1(x)\vert} \left( \fr{\partial v_1(x)}{\partial x_1} 
+i\fr{\partial v_1(x)}{\partial x_2}\right)\,.
  $$
   Таким образом, на границе об\-ласти должно выполняться 
  $$
  \fr{1}{\vert \nabla v_1(x)\vert} \left( \fr{\partial v_1(x)}{\partial x_1}\,n_1(x)+ 
\fr{\partial v_1(x)}{\partial x_2}\,n_2(x)\right)= \cos\theta(x),
  $$ 
  что можно переписать в~виде 
  $$
  \fr{\partial v_1(x)}{\partial n}= q(x)\cos\theta(x),
  $$
   где 
$q(x)\hm= \vert \nabla v_1(x)\vert$. Известно~[3], что долж\-но выполняться условие
  $$
  \int\limits_{\partial\tilde{X}} \fr{\partial v_1(x)}{\partial n}\,ds 
=\int\limits_{\partial\tilde{X}} \fr{\partial v_2(x)}{\partial s}\,ds 
=\int\limits_{\partial\tilde{X}}dv_2=0\,,
  $$
  где $\partial v_2(x)/\partial s$~--- производная по на\-прав\-ле\-нию касательной 
  к~границе об\-ласти. Поэтому $q(x)$ не может быть произвольной. Итак, функцию 
$v_1(x_1, x_2)$ на~$\tilde{X}$ будем искать из задачи Неймана:
  \begin{equation}
  \left.
  \begin{array}{c}
  \fr{\partial^2 v_1}{\partial x_1^2}+ \fr{\partial^2 v_1}{\partial x_2^2}=0\,;\\[6pt]
  \fr{\partial v_1(x)}{\partial n} =\eta(x)\,,\enskip x=\left( x_1, x_2\right) \in 
\partial\tilde{X}\,,
  \end{array}
  \right\}
  \label{e1-bos}
  \end{equation}
где $\eta(x)\hm= q(x)\cos\theta(x)$. Функция~$q(x)$ в~пред\-став\-лен\-ных ниже 
примерах выбиралась сле\-ду\-ющим образом:
\begin{multline*}
q(x)={}\\
{}= \begin{cases}\!
\left(\,\displaystyle\int\limits_{\cos\theta(x)>0}\!\!\cos\theta(x)\,ds\right)^{-1}\!\!, &\!\! \mbox{если } 
\cos\theta(x)>0;\\[3pt]
-\left(\,\displaystyle\int\limits_{\cos\theta(x)<0}\!\! \cos\theta (x)\,ds\right)^{-1}\!\!, &\!\! \mbox{если 
}\cos\theta(x)<0.
\end{cases}\hspace*{-9pt}
%\label{e2-bos}
\end{multline*}
  
  Пусть $\Gamma_\rho \hm= \{w=w_1\hm+ iw_2: w_1^2\hm+ w_2^2\hm<\rho\}$ 
  и~$x\hm= f(w)\hm= f_1(w_1, w_2)\hm+ if_2(w_1, w_2)$~--- конформное 
преобразование круга~$\Gamma_1$ на некоторую односвязную область~$D$, 
содержащую множество~$X$ (рис.~1). Для $\rho \hm\in (0;1)$  обозначим 
$\tilde{X}_\rho \hm= f(\Gamma_\rho)$. Причем~$\rho$ выбрано так, что 
$X\hm\subset \tilde{X}_\rho$. Если обозначить через~$v(t)$ угол наклона 
касательной к~границе области~$D$ в~точке~$x$, соответствующей точке $w\hm= 
w_1\hm+ iw_2\hm\in \partial \Gamma_1$ при конформном отоб\-ра\-же\-нии $x\hm= 
f(w)$, то конформное преобразование единичного круга на об\-ласть~$D$ может 
быть найдено по формуле Чизотти~[3]:
  \begin{align*}
  x&=f_1(w_1,w_2) +if_2(w_1,w_2)={}\\
 & \hspace*{10mm}{}=i\int\limits_{w_{1,0}+iw_{2,0}}^{w_1+iw_2} 
\fr{e^{i\xi(y)}}{(1-y)^2}\,dy +x_0\,;\\
  \xi(y)&=\fr{1}{2\pi} \int\limits_0^{2\pi} v(t) \fr{e^{it}+y}{e^{it}-y}\,dt+i\xi_0\,,
  \end{align*}
где $\xi_0$~--- некоторая действительная по\-сто\-ян\-ная; $x_0\hm\in D$ 
и~$w_{0,1}\hm+ i w_{0,2}\hm\in \Gamma_1$~--- заданные точки. Пусть теперь 
функция~$v_1(x)$ является решением задачи~(1), где $\tilde{X}\hm= 
\tilde{X}_\rho$. Из соотношений Ко\-ши--Ри\-ма\-на и~гармоничности функций 
$f_1(w_1, w_2)$ и~$f_2(w_1, w_2)$ следует, что функция $z(w)\hm= v_1(f(w))$\linebreak\vspace*{-12pt}

{ \begin{center}  %fig1
 \vspace*{-6pt}
    \mbox{%
\epsfxsize=79mm
\epsfbox{bos-1.eps}
}

\end{center}

\vspace*{-3pt}

\noindent
{{\figurename~1}\ \ \small{
Конформное преобразование единичного круга на многоугольник и~его ограничения 
на круги $\Gamma_{0{,}95}$ и~$\Gamma_{0{,}75}$ 
}}}

\vspace*{18pt}

\setcounter{figure}{1}


\noindent 
удовле\-тво\-ря\-ет уравнению $\partial^2 z/\partial w_1^2\hm+ \partial^2 z/\partial 
w_2^2\hm=0$.
 Определим краевое условие. Пусть $w\hm= w_1\hm+ iw_2\hm\in 
\partial \Gamma_\rho$  и~$m$~--- единичная нормаль к~границе 
круга~$\Gamma_\rho$. Тогда $m\hm= w_1/\rho \hm+ iw_2/\rho$. Следовательно,
\begin{multline*}
\fr{\partial z(w)}{\partial m}=\fr{\partial z(w)}{\partial w_1}\,\fr{w_1}{\rho} 
+\fr{\partial z(w)}{\partial w_2}\fr{w_2}{\rho}={}\\[6pt]
{}=\fr{\partial v_1(f(w))}{\partial x_1}\,\fr{\partial f_1(w)}{\partial m}+\fr{\partial 
v_1(f(w))}{\partial x_2}\,\fr{\partial f_2(w)} {\partial m}\,.
\end{multline*}
  
  Рассмотрим кривую $\partial \Gamma_\rho$  и~кривую с~параметрическим 
представлением~$\gamma_w$: $r_w(t)\hm= w_1\tau\hm+ iw_2\tau$, $\tau\hm\in 
[0;1]$. Они перпендикулярны в~точке $\tau\hm=1$, а~их образы при конформном 
преобразовании $x\hm= f(w)$ пред\-став\-ля\-ют собой кривую 
$\partial \tilde{X}_\rho$ и~кривую $\gamma_{f(w)}$: $r_{f(w)}(\tau)\hm= f_1(w_1\tau, w_2\tau)\hm+ 
if_2(w_1\tau, w_2\tau)$, $\tau\hm\in [0;1]$. Поскольку конформное преобразование 
сохраняет углы между кривыми, то вектор нормали к~границе~$\partial 
\tilde{X}_\rho$ коллинеарен касательному век\-то\-ру кривой $\gamma_{f(w)}$ 
в~точке, со\-от\-вет\-ст\-ву\-ющей $\tau\hm=1$. С~учетом того, что
 \begin{multline*}
 \left\vert r^\prime_{f(w)}(1)\right\vert ={}\\
  {}=\rho \vert \nabla f_1(w)\vert =\rho \sqrt{\left( 
\fr{\partial f_1(w)}{\partial w_1}\right)^2 +\left( \fr{\partial f_1(w)}{\partial 
w_2}\right)^2}\,,
\end{multline*}
получаем
$$
n=\fr{r^\prime_{f(w)} (1)}{\vert r^\prime_{f(w)} (1)\vert }=\fr{\partial f_1(w)/\partial 
m+ i \partial f_2(w)/\partial m} {\left\vert \nabla f_1(w)\right\vert}\,.
$$
  
  Следовательно, функция~$z$ является решением задачи Неймана для 
круга~$\Gamma_\rho$: $\partial^2 z/ \partial w_1^2 \hm+ \partial^2 z/ \partial 
w_2^2\hm=0$, $\partial z(w)/\partial m \hm= \eta(f(w))\vert \nabla f_1(w)\vert$, 
$w\hm\in \Gamma_\rho$. Как известно~[3], решение такой задачи может быть 
найдено по формуле Дини:
  \begin{multline}
  z(w_1,w_2)=z_0-{}\\
  {}-\fr{\rho}{2\pi} \int\limits_0^{2\pi} \eta (f(\rho\cos\tau, \rho\sin\tau)) \vert \nabla 
f_1(\rho\cos\tau, \rho\sin\tau)\vert \times{}\\
{}\times \ln \fr{(w_1-\rho\cos\tau)^2+(w_2-
\rho\sin\tau)^2}{\rho^2}\,d\tau\,,
  \label{e3-bos}
  \end{multline}
где $z_0\in \mathrm{C}$~--- произвольная константа.
  
  Итак, из~(\ref{e3-bos}) можно найти $z(w)$, $w\hm\in \Gamma_\rho$. Зная 
преобразование $x\hm= f(w)$, из соотношения $z(w)\hm= v_1(f(w))$ можно найти 
$v_1(x)\hm= z(f^{-1}(x))$, $x\hm\in \tilde{X}_\rho$. Функция $v_2(x)$ может быть 
найдена аналогично, но для решения уравнений механики КМ она не нужна, 
достаточно $v_1(x)$ и~соотношений Ко\-ши--Ри\-ма\-на. На рис.~1 показано 
конформное преобразование единичного круга на многоугольник и~его 
ограничения на круги~$\Gamma_{0{,}95}$ и~$\Gamma_{0{,}75}$. На рис.~2 
показано векторное поле, образованное касательными к~кривым $v_1(x_1, 
x_2)\hm=const$~(рис.~2,\,\textit{а}) и~$v_2(x_1,x_2)\hm= const$~(рис.~2,\,\textit{б}). 

\setcounter{figure}{1}
\begin{figure*} %fig2
\vspace*{1pt}
  \begin{center}  
    \mbox{%
\epsfxsize=163mm
\epsfbox{bos-2.eps}
}

\end{center}
\vspace*{-9pt}
\Caption{Касательные векторы к~кривым  $v_1(x_1, x_2)\hm=const$~(\textit{а}) и~$v_2(x_1, 
x_2)\hm=const$~(\textit{б})}
\end{figure*}
\begin{figure*}[b] %fig3
\vspace*{1pt}
  \begin{center}  
    \mbox{%
\epsfxsize=161.879mm
\epsfbox{bos-3.eps}
}

\end{center}
\vspace*{-9pt}
\Caption{Оптимальные траектории уклад\-ки волокон при 3D-пе\-ча\-ти}
\end{figure*}

\section{Построение оптимальной схемы 3D-печати}

  Используем теперь описанную модель для поиска оптимальных траекторий 
уклад\-ки волокон при 3D-пе\-ча\-ти. Пусть $\sigma_1^{\pm}$ и~$\sigma_2^{\pm}$~--- 
пределы проч\-ности при растяжении и~сжатии вдоль и~поперек волокон, 
а~$\hat{\tau}_{12}$~--- предел проч\-ности при сдвиге в~плос\-кости слоя~[2]. 
В~качестве целевой функции используем критерии максимальных на\-пря\-же\-ний 
$$
R\left(\sigma_1, \sigma_2, \tau_{12}\right)= \max\left( \fr{\sigma_1}{m_1(\sigma_1)}, 
\fr{\sigma_2}{m_2(\sigma_2)}, \fr{\tau_{12}}{\hat{\tau}_{12}}\right),
$$ 
где
  $$
  m_1(\sigma_1) =\begin{cases}
  \sigma_1^+\,, &\mbox{если }\sigma_1>0\,;\\
  \sigma_1^-\,, &\mbox{если }\sigma_1<0\,;
  \end{cases}
  $$
  $$
  m_2(\sigma_2) =\begin{cases}
  \sigma_2^+\,, &\mbox{если }\sigma_2>0\,;\\
  \sigma_2^-\,, &\mbox{если }\sigma_2<0\,.
  \end{cases}
  $$
  
  Значения $\sigma_1$, $\sigma_2$ и~$\tau_{12}$ могут быть найдены при\-бли\-жен\-но 
из уравнений механики композиционных материалов~[2], если задать 
преобразование $v(x)\hm= (v_1(x), v_2(x))$. Краевые условия к~этим уравнениям 
определяются углами $\theta(x)$, $x\hm\in \partial X$, которые волокна образуют 
с~внеш\-ней нормалью к~границе множества~$X$. Следовательно, целевая функция 
является функцией этих углов:
 \begin{multline*}
  R: \theta\vert_{\partial X} \mapsto v(x)=\left( v_1(x), v_2(x)\right)\mapsto \left( 
\sigma_1, \sigma_2, \tau_{12}\right) \mapsto {}\hspace*{-1.3625pt}\\
{}\mapsto \max \left( 
\fr{\sigma_1}{m_1(\sigma_1)}, \fr{\sigma_2}{m_2(\sigma_2)}, 
\fr{\tau_{12}}{\hat{\tau}_{12}}\right).
\end{multline*}
  
  Соответствие $(v_1(x), v_2(x)) \hm\mapsto (\sigma_1, \sigma_2, \tau_{12})$ дает 
приближенное решение уравнений механики,\linebreak полученное далее методом 
приближенного решения уравнений в~част\-ных производных с~по\-мощью\linebreak 
вейв\-ле\-тов, построенных на основе схем подразделений~[4] и~подъема~[5]. 
Преимущество вейвлетов перед другими базисными функциями состоит\linebreak в~том, что 
вейв\-лет-ко\-эф\-фи\-ци\-ен\-ты убывают быст\-ро, поэтому достаточно небольшого 
числа сла\-га\-емых в~разложениях. Дополнительное преимущество вейвлетов, 
использующих схемы \mbox{подразделений} и~подъема, со\-сто\-ит в~воз\-мож\-ности 
управ\-лять формой базисных функций, например обнулять их на выбранной 
об\-ласти, что еще уменьшает чис\-ло сла\-га\-емых. Эти преимущества оказываются 
важ\-ны, так как при минимизации целевой функции требуется многократно решать 
сис\-те\-му урав\-не\-ний в~част\-ных производных, опи\-сы\-ва\-ющую 
на\-пря\-жен\-но-де\-фор\-ми\-ру\-емое со\-сто\-яние конструкции.
  
  Минимизировать функцию~$R$ предлагается с~по\-мощью метода, 
пред\-став\-лен\-но\-го в~[6] и~реализованного в~биб\-лио\-те\-ке scipy для языка 
программирования Python. Поскольку не требуется искать глобальный минимум, 
а~достаточно обеспечить выполнение условия $R\hm<1$, можно закончить 
итерационный процесс при его выполнении. На рис.~3 пред\-став\-ле\-ны траектории 
уклад\-ки волокон для 3D-пе\-ча\-ти прямоугольной плас\-ти\-ны с~отверстием

\vspace*{-6pt}

\noindent
\begin{multline*}
  X=\left\{ \left( x_1,x_2\right): \left( x_1-x_{1,0}\right)^2 +\left( x_2-
x_{2,0}\right)^2 \leq r^2\,,\right.\\ 
\left.x_2\in [0;a]\,,\ x_2\in [0;b]\right\}
\end{multline*}
после двух итераций генетического алгоритма.

\vspace*{-6pt}

\section{Применение вейвлетов к~приближенному решению задач 
теории упругости}

\vspace*{-2pt}

  В статье~\cite{7-bos} проанализирован общий подход к~по\-стро\-ению 
би\-ор\-то\-го\-наль\-ных вейв\-лет-сис\-тем на основе схем подъема~\cite{5-bos}, в~том 
чис\-ле показано, что эти схемы тесно связаны со схемами  
подразделений~\cite{4-bos} и~поз\-во\-ля\-ют строить вейвлеты с~заданными 
свойствами. На основе таких вейв\-лет-сис\-тем далее представлен алгоритм 
при\-бли\-жен\-но\-го решения уравнений тео\-рии упругости.
  
  Пусть $X\subset \mathbf{R}^n$. Будем рассматривать действительное 
пространство~$L_2(X)$.


\smallskip

\noindent
\textbf{Определение~1}~\cite{5-bos}. Кратномасштабный анализ на~$X$ 
определяется как последовательность подпространств $V_j\hm\subset L_2(X)$, 
$j\hm= 0,1,\ldots$, такая, что $V_j\hm\subset V_{j+1}$; $\cup_j V_j$ плотно 
в~$L_2(X)$ и~для каж\-до\-го~$j$ существуют масштабирующие функции 
$\varphi_{j,k}$, $k\hm\in K_j$, такие что множество $\{\varphi_{j,k}\}_{k\in K_j}$ 
пред\-став\-ля\-ет собой базис Рисса~\cite{8-bos} в~$V_j$, при этом $K_j\hm\subset 
K_{j+1}$.


\smallskip

\noindent
\textbf{Определение~2}~\cite{8-bos}. Пусть $\{V_j\}_{j\geq 0}$ 
и~$\{\tilde{V}_j\}_{j\geq0}$~--- два кратномасштабных анализа на~$X$ 
с~масштабирующими функциями $\varphi_{j,k}$ 
и~$\tilde{\varphi}_{j,k}$, $k\hm\in K_j$, соответственно. Кроме того, пусть 
$V_{j+1}\hm= V_j\hm+ W_j$, $\tilde{V}_{j+1} \hm= \tilde{V}_j\hm+ \tilde{W}_j$ 
и~$\{\psi_{j,k}, k\hm\in M_j\}$, $\{ \tilde{\psi}_{j,k}, k\hm\in M_j\}$~--- базисы Рисса 
в~$W_j$ и~$\tilde{W}_j$ соответственно. Если 
$(\varphi_{j,k},\tilde{\varphi}_{j,k^\prime})\hm= \delta_{k,k^\prime}$, 
$(\tilde{\psi}_{j,m}, \varphi_{j,k})\hm= 0$, $( \tilde{\varphi}_{j,k}, \psi_{j,m})\hm=0$, 
$(\psi_{j,m}, \tilde{\psi}_{j,m^\prime})\hm=\delta_{m,m^\prime} \forall\,j$, $\forall\,m, 
m^\prime\hm\in M_j$, $\forall\,k, k^\prime\hm\in K_j$, то семейства функций 
$\{\psi_{j,k}\}_{j\geq0,\ k\in M_j}$ и~$\{ \tilde{\psi}_{j,k}\}_{j\geq0, k\in M_j}$ 
называются би\-ор\-то\-го\-наль\-ны\-ми вейв\-лет-сис\-те\-мами.
  
  Из определения~2 следует~\cite{5-bos}, что существуют последовательности $\{ 
h_{j,k,l}\}$ и~$\{\tilde{h}_{j,k,l}\}$ такие, что $\varphi_{j,k}\hm= \sum\nolimits_{l\in 
K_{j+1}} h_{j,k,l}\varphi_{j+1,l}$ и~$\tilde{\varphi}_{j,k}\hm= \sum\nolimits_{l\in 
K_{j+1}} \tilde{h}_{j,k,l} \tilde{\varphi}_{j+1,l}$. Поскольку $W_j\hm\subset 
V_{j+1}$ и~$\tilde{W}_j\hm\subset \tilde{V}_{j+1}$, то $\psi_{j,m}\hm= 
\sum\nolimits_{l\in K_{j+1}} g_{j,m,l}\varphi_{j+1,l}$ и~$\tilde{\psi}_{j,m}\hm= 
\sum\nolimits_{l\in K_{j+1}} \tilde{g}_{j,m,l} \tilde{\varphi}_{j+1,l}$. 
Последовательности $h_{j,k,l}$, $\tilde{h}_{j,k,l}$, $g_{j,k,l}$ и~$\tilde{g}_{j,k,l}$ 
называются фильтрами. В~случае би\-ор\-то\-го\-наль\-ных  
вейв\-лет-сис\-тем для $f\hm\in L_2(X)$ имеет мес\-то равенство~\cite{5-bos, 8-bos}:
  $$
  f=\sum\limits_{k\in K_{j_0}} v_{j_0,k}\varphi_{j_0,k}+ \sum\limits_{j\geq j_0} 
\sum\limits_{m\in M_j} \gamma_{j,m}\psi_{j,m}\,,
  $$ 
  где $v_{j,k} = (f,\tilde{\varphi}_{j,k})$; $\gamma_{j,m} \hm= 
(\tilde{\psi}_{j,m},f)$. 
  
  Схема подъема (lifting scheme) позволяет строить би\-ор\-то\-го\-наль\-ные  
вейв\-лет-сис\-те\-мы с~заданными\linebreak свойствами, используя некоторые начальные 
би\-ор\-то\-го\-наль\-ные вейв\-лет-сис\-те\-мы с~фильт\-ра\-ми $h^0_{j,k,l}$, 
$\tilde{h}^0_{j,k,l}$, $g^0_{j,k,l}$ и~$\tilde{g}^0_{j,k,l}$. По схеме подъема новое 
семейство фильт\-ров $h_{j,k,l}$, $\tilde{h}_{j,k,l}$, $g_{j,k,l}$ и~$\tilde{g}_{j,k,l}$, 
опре\-де\-ля\-ющих биортогональные вейв\-лет-сис\-те\-мы, находится по 
формулам~\cite{5-bos}:
  \begin{gather*}
  h_{j,k,l} =h^0_{j,k,l}\,;\quad
  \tilde{h}_{j,k,l} =\tilde{h}^0_{j,k,l} 
+\sum\limits_{m\in M_j} s_{j,k,m}\tilde{g}^0_{j,m,l}\,;\\
  g_{j,m,l}= g^0_{j,m,l} -\sum\limits_{k\in K_j} s_{j,k,m} 
h^0_{j,k,l}\,;\quad
  \tilde{g}_{j,m,l}=\tilde{g}^0_{j,m,l} 
  \end{gather*}
при любом выборе последовательности $\{ s_{j,k,m}\}_{k\in K_j, \ m\in M_j}$. 
Следует заметить, что мас\-шта\-би\-ру\-ющие функции~$\varphi_{j,k}$ одинаковы 
в~исходном и~поднятом крат\-но\-мас\-штаб\-ном анализах. Кроме того, можно не 
менять функцию~$\tilde{\varphi}_{j,k}$, а~поднимать~$\varphi_{j,k}$. Механизм 
этот точ\-но такой же и~называется двойственной схемой подъема. Он позволяет 
улучшить свойства вейв\-ле\-та~$\tilde{\psi}_{j,m}$.
  
  Пусть последовательности $d^i: \mathbf{Z}\hm\to \mathbf{R}$, $i\hm= 0,1,2$, 
и~$a: \mathbf{Z}\hm\to \mathbf{R}$ имеют носитель $\{-1, 0, 1\}$ и~на нем могут 
быть записаны в~виде:
\begin{alignat*}{2}
d^0&\!=\!\begin{pmatrix}
\fr{1}{6} & \fr{4}{6} & \fr{1}{6}\end{pmatrix};&\ 
d^1&\!=\! 2^{j+p-1}(-1\ 0\ 1); \\
d^2&\!=\!
2^{2(j+p)}(1\ -2\ 1);&\ 
a&\!=\! \begin{pmatrix}\!
-\fr{1}{6} & \fr{8}{6} & -\fr{1}{6}\!\end{pmatrix}\!,\ p\!=\!1,2,\ldots
\end{alignat*}
 Для краткости введем обозначение для 
част\-ной производной
  $$
  \partial^{(\alpha_1, \ldots, \alpha_k)} \! f(x_1, \ldots, x_k)\!=\! 
  \fr{\partial^{\alpha_1+\cdots+\alpha_k} f(x_1,\ldots , x_k)}{\partial 
x_1^{\alpha_1}\cdots \partial x_k^{\alpha_k}}
  $$
  порядка $\alpha_1+\cdots + \alpha_k$. На основании результатов, 
пред\-став\-лен\-ных в~\cite{7-bos, 9-bos}, можно сформулировать сле\-ду\-ющую 
тео\-ре\-му, которая пред\-став\-ля\-ет собой алгоритм на\-хож\-де\-ния значений 
масштабирующих функций и~вейвлетов на~$\mathbf{R}^n$, полученных 
с~по\-мощью схемы подъема.

\begin{figure*}[b] %fig4
\vspace*{1pt}
  \begin{center}  
    \mbox{%
\epsfxsize=159.507mm
\epsfbox{bos-4.eps}
}

\end{center}
\vspace*{-9pt}
\Caption{Графики масштабирующих функций}
\end{figure*}

\smallskip

\noindent
\textbf{Теорема.} \textit{Пусть $\Lambda^n$~--- совокупность всех ненулевых 
векторов $e\hm\in \mathbf{Z}^n$, координаты которых равны~$0$ или~$1$. 
Обозначим $K_j\hm= 2^{-j} Z^n$, $M_j\hm= 2^{-j} Z^n\hm+ 2^{-j-1} \Lambda^n$. 
Предположим, что по\-сле\-до\-ва\-тель\-ность $A\hm= \{A_i\}_{i\in Z}$ определяет 
функцию из~$C^2(\mathbf{R})$ с~компактным носителем}~\cite{4-bos}. \textit{Определим 
последовательность $b\hm= \{b_t\}_{t\in \mathbf{Z}^n}$ равенством $b_t\hm= 
\prod\nolimits^n_{i=1} A_{t_i}$, $t\hm= (t_1\cdots t_s)^{\mathrm{T}} \hm\in \mathbf{Z}^n$. Пусть 
оператор~$U$ функции $v: K_j\hm\to \mathbf{R}$ ставит в~соответствие 
функцию $Uv: K_{j+1}\hm\to \mathbf{R}$, определенную равенством $(Uv)_k\hm= 
v_k$, $\forall\,k\hm\in K_j$, и~$(Uv)_m\hm=0$, $\forall\,m\hm\in M_j$. Тогда если 
по\-сле\-до\-ва\-тель\-ность~$v_{j+p}$ получается по схеме подразделений 
\begin{align*}
v_{j+p}&= b* (Uv_{j+p-1}),\enskip p= 1,2,\ldots;
\\
v_{j,\beta} &= \delta_{\alpha,\beta} =
\begin{cases}
1\,, & \alpha=\beta\,;\\
0\,, & \alpha\not= \beta\,,
\end{cases}
\end{align*}
где $*$ обозначает свертку, то значения мас\-шта\-би\-ру\-ющих функций и~их 
част\-ных производных мож\-но найти из при\-бли\-жен\-ных равенств}:

\noindent
\begin{multline*}
\!\varphi_{j,\alpha}\left( 2^{-j-p}\beta\right) \approx v_{j+p,\beta};\enskip \partial^{(l_1,\ldots 
,l_n)} \varphi_{j,\alpha} \left( 2^{-j-p}\beta\right)\approx{}\\[3pt]
{}\approx
\left( \left( d^{l_1}\otimes \cdots \otimes d^{l_n}\right) * \left( \left( a\otimes \cdots 
\otimes a\right) * v_{j+p}\right)\right)_\beta\,,\\[3pt]
l_1,\ldots , l_n=0,1,2\,,\enskip l_1+\cdots+ l_n\leq 2\,.
\end{multline*}

\textit{Вейвлеты могут быть найдены по формуле
$$
\psi_{j,m}= \varphi_{j+1,m}- \sum\limits_{k\in K_j} s_{j,k,m} \varphi_{j,k},
$$
где 
по\-сле\-до\-ва\-тель\-ность~$s_{j,k,m}$ мож\-но выбрать произвольным образом.}

\smallskip

  Выбором последовательностей~$A$ и~$s$ мож\-но задавать свойства 
мас\-шта\-би\-ру\-ющих функций и~вейв\-ле\-тов. Например, можно обнулить их значения 
в~заданной об\-ласти и~на ее границе. На рис.~4 представлен пример, где 
мас\-шта\-би\-ру\-ющие функции, соответствующие узлам $k\hm\in K_j$, 
принадлежащим заданной об\-ласти, обнуляются за пределами этой об\-ласти.

   \begin{figure*}[b] %fig5
\vspace*{6pt}
  \begin{center}  
    \mbox{%
\epsfxsize=143.174mm
\epsfbox{bos-5.eps}
}

\end{center}
\vspace*{-9pt}
\Caption{Триангулируемое пространство $(T,G,X)$~(\textit{а} и~\textit{б}), 
подразделения~$X$~(\textit{в} и~\textit{г}), графики точ\-но\-го решения~$\sigma_1$~(\textit{д}) 
и~его приближения~$\sigma_{1,J}$~(\textit{е})}
\end{figure*}

  Пусть $(T,G,X)$~--- триангулируемое про\-стран\-ст\-во с~конечным множеством 
симплексов. Здесь~$T\hm= \cup^N_{l=1} I_l^n \hm\subset I_0^n\hm\subset 
\mathbf{R}^n$~--- объединение замкнутых $n$-мер\-ных кубов вида $I_l^n\hm= 
\prod^n_{i=1} [ b_{i,l}; b_{i,l}\hm+1]$, где $b_{i,l}\hm\in \mathbf{Z}$, 
$I_0^s$~--- $n$-мер\-ный куб; $G: T\hm\to X\hm\subset \mathbf{R}^n$~--- 
гомеоморфизм, $G: \mathrm{Int}\,(T) \hm \to \mathrm{Int}\,(X)$~--- диффеоморфизм класса~$C^2$, где 
$\mathrm{Int}(X)$~--- внут\-рен\-ность множества~$X$. Пусть $\{\varphi_{j,k}\}_{k\in K_j}$ 
и~$\{\psi_{j,m}\}_{m\in M_j}$~--- масштабирующие функции и~вейвлеты на~$T$. 
Определим мас\-шта\-би\-ру\-ющие функции и~вейвлеты на~$X$ сле\-ду\-ющи\-ми 
равенствами: 
$$
\varphi^X_{j,k} = \varphi_{j,k}\circ G^{-1};\quad
\psi^X_{j,m}=\psi_{j,m}\circ G^{-1},
$$

\columnbreak

\noindent
 где~$\circ$ обозначает композицию функций, т.\,е.\ 
$\varphi_{j,k}\circ G^{-1}(x)\hm= \varphi_{j,k}(G^{-1}(x))$. Тогда если $f: X\hm\to 
\mathbf{R}$ и~$f\circ G\hm\in L_2(T)$, то
$$
f= \sum\limits_{k\in K_{j_0}} 
v_{j_0,k} \varphi^X_{j_0,k}+\sum\limits_{j\geq j_0} \sum\limits_{m\in M_j} 
\gamma_{j,m} \psi^X_{j,m}
$$ 
в~том смыс\-ле, что

\vspace*{-6pt}

\noindent
  \begin{multline*}
  \lim\limits_{J\to +\infty} \int\limits_T \left[ \left(
  \vphantom{\sum\limits^J_{j=j_0}}
   f- \sum\limits_{k\in K_{j_0}} 
v_{j_0,k} \varphi^X_{j_0,k} - \right.\right.\\
{}-\left.\left.\sum\limits^J_{j=j_0} \sum\limits_{m\in M_j} 
\gamma_{j,m} \psi^X_{j,m}\right) \circ G(u)\right]^2 du=0\,.
  \end{multline*}
  
 
\pagebreak
  
  На практике отображение $G$ выписать слож\-но, поэтому используется его 
аппроксимация по точечным соответствиям. Методика по\-стро\-ения этой 
аппроксимации с~по\-мощью ло\-каль\-но-ап\-прок\-си\-ма\-ци\-он\-ных сплайнов 
пред\-став\-ле\-на в~\cite{7-bos}.
  
  Из методов приближенного решения краевых задач математической физики 
в~ре\-ша\-емой задаче наиболее удобен метод наименьших  
квад\-ра\-тов~\cite{10-bos}. Рассмотрим дифференциальное уравнение и~краевые 
условия $Lw\hm= f$ на~$X$ и~$L_i w\hm= f_i$ на $\partial X$, $i\hm=1,2,\ldots , q$, 
в~гильбертовом пространстве~$L_2(X)$, где $L$~--- линейный 
дифференциальный оператор. Пусть $\{V_j\}_{j\geq 0}$~---  
крат\-но\-мас\-штаб\-ный анализ на~$X$. При\-бли\-жен\-ные решения~$w_j$ данной 
кра\-евой задачи будем искать в~виде:
  \begin{multline}
  w_j={}\\
  \!{}= \!\!\sum\limits_{k\in K_0} \!v_{0,k} \varphi^X_{0,k} +\sum\limits_{j=0}^{J-1} 
\sum\limits_{m\in M_j} \!\gamma_{j,m} \psi^X_{j,m} = \!\sum\limits_{k=1}^{M(J)}\! c_k 
\omega_k,\!\!
  \label{e4-bos}
  \end{multline}
где $M(J)$~--- число базисных функций в~$V_J$ и~для удобства базисные 
функции пронумерованы одним индексом и~обозначены~$\omega_k$, 
а~коэффициенты~$v_{0,n}$ и~$\gamma_{j,m}$ обозначены~$c_k$ и~находятся 
методом наименьших квад\-ра\-тов из решения вариационной задач 
$$
w_J=\argmin\limits_{w\in V_J} F_J(w).
$$
 Функционал~$F_J(w)$ определяется 
равенством 
$$
F_J(w)= \| Lw - f\|^2 + \sum\limits^q_{i=1} a_i \| Lw-  f_i\|^2,
$$
где $a_i$~--- положительные весовые коэффициенты. С~учетом того, что при 
построении вейвлетов есть воз\-мож\-ность обнулить часть базисных функций 
в~заданной об\-ласти, мож\-но часть коэффициентов разложения~(\ref{e4-bos}) найти 
из граничных условий. В~этом случае 
$$
w_j= \sum\limits_{k\in \mathrm{Int}\,X} c_k \omega_k + \sum\limits_{t\in \partial X} c_t \omega_t,
$$
 а остальные 
коэффициенты уже находятся из задачи минимизации 
$$
\left\| \sum\limits_{k\in \mathrm{Int}\,X} 
c_k L\omega_k- \left(f- \sum\limits_{t\in \partial X} c_t \omega_t\right)\right\|^2\hm\to 
\min\limits_{c_k}.
$$
  
  
  \smallskip
  
  \noindent
  \textbf{Пример.} Рассмотрим плас\-ти\-ну 
  $$
  X= \left\{ (x_1, x_2): a^2\leq  x_1^2+ x_2^2\leq b^2\right\}
$$ 
в~условиях рас\-тя\-же\-ния/сжа\-тия. Пусть 
модуль Юнга $E\hm= 2{,}1$~МПа; коэффициент Пуассона $\mu\hm= 0{,}3$. 
Размеры пластины: $a\hm= 20$~м; $b\hm=60$~м. Ин\-тен\-сив\-ность нормальной 
к~границе плас\-ти\-ны нагрузки имеет вид: 
$$
F_n= 2\left(1+ \fr{1}{a^2}\right)n;\enskip F_n=2\left(1+ \fr{1}{b^2}\right)n.
$$
 С~использованием биортогональных вейв\-ле\-тов были 
рассчитаны компоненты на\-пря\-же\-ний. Сравним их с~точными: 
\begin{gather*}
\sigma_1=2+ \fr{2}{r^2}\cos(2\theta);\enskip \sigma_2= 2- \fr{2}{r^2}\cos(2\theta);\\ 
\tau_{12}= \fr{2}{r^2}\sin(2\theta).
\end{gather*}
  
  Множества $T$ и~$X$ показаны на рис.~5,\,\textit{а} и~5,\,\textit{б}, 
подразделения~$X$~--- на рис.~5,\,\textit{в} и~5,\,\textit{г}. Погрешности 
$\max_{x_i}\vert \sigma_1(x_i) \hm- \sigma_{1,J}(x_i)\vert$, $\max_{x_i}\vert 
\sigma_2(x_i)\hm- \sigma_{2,J}(x_i)\vert$ и~$\max_{x_i}\vert \tau_{12}(x_i)\hm- 
\tau_{12,J} (x_i)\vert$ не превышают соответственно 0,0009, 0,000955 
и~0,00025~МПа. На рис.~5,\,\textit{д} и~5,\,\textit{е} пред\-став\-ле\-ны графики точ\-но\-го 
решения~$\sigma_1$  и~его при\-бли\-же\-ния~$\sigma_{1,J}$.
  


  
  В заключение отметим, что предложенные методики реализованы  
в~CAD/CAE-сис\-те\-ме по\-стро\-ения таких конструкций, написанной 
с~использованием языка программирования Python.
  
{\small\frenchspacing
 {%\baselineskip=10.8pt
 %\addcontentsline{toc}{section}{References}
 \begin{thebibliography}{99}
\bibitem{1-bos}
Matsuzaki Laboratory. {\sf http://www.rs.tus.ac.jp/\linebreak rmatsuza}.
\bibitem{2-bos}
\Au{Васильев В.\,В.} Механика конструкций из композиционных материалов.~--- 
М.: Машиностроение, 1988. 272~с.
\bibitem{3-bos}
\Au{Лаврентьев М.\,А., Шабат~Б.\,В.} Методы тео\-рии функций комплексного 
переменного.~--- М.: Наука, 1973. 736~с.
\bibitem{4-bos}
\Au{Cavaretta A.\,S., Dahmen~W., Micchelli~C.\,A.} Stationary subdivision 
schemes~// Mem. Am. Math. Soc., 1991. Vol.~93. No.\,453. P.~1--186.
\bibitem{5-bos}
\Au{Schroder P., Sweldens~W.} Spherical wavelets: Efficiently representing functions 
on the sphere~// 22nd Annual Conference on Computer Graphics and Interactive 
Techniques Proceedings~/ Eds. S.\,G.~Mair, R.~Cook.~---
New York, NY, USA: Association for Computing Machinery, 1995. P.~161--172.
\bibitem{6-bos}
\Au{Storn R., Price~K.} Differential evolution~--- a~simple and efficient heuristic for 
global optimization over continuous spaces~// J.~Global Optim., 1997. No.\,11. 
P.~341--359.
\bibitem{7-bos}
\Au{Deniskina G.\,Y., Deniskin~Y.\,I., Bityukov~Y.\,I.} About biortogonal wavelets, 
created on the basis of scheme of increasing of lazy wavelets~// Adv.  Automation~II: 
Rus. Auto Conf. 2020.~--- Lecture notes in electrical engineering ser.~--- Cham: 
Springer, 2021.Vol.~729. P.~173--181. doi: 10.1007/978-3-030-71119-1\_18.
\bibitem{8-bos}
\Au{Новиков~И.\,Я., Протасов~В.\,Ю., Скопина~М.\,А.} Тео\-рия всплес\-ков.~--- М.: 
Физматлит, 2005. 612~c.
\bibitem{9-bos}
\Au{Deniskina G.\,Y., Deniskin~Y.\,I., Bityukov~Y.\,I.} About some computational 
algorithms for locally approximation splines, based on the wavelet transformation and 
convolution~// Advances in automation~II~/
Eds. A.\,A.~Radionov, V.\,R.~Gasiyarov.~--- Lecture notes in 
electrical engineering ser.~--- Cham, Switzerland: Springer, 2021. Vol.~729. P.~182--191. doi: 
10.1007/978-3-030-71119-1\_19.
\bibitem{10-bos}
\Au{Марчук Г.\,И., Акилов~Г.\,П.} Методы вы\-чис\-ли\-тель\-ной математики.~--- М.: Наука, 1989. 744~с.

\end{thebibliography}

 }
 }

\end{multicols}

\vspace*{-6pt}

\hfill{\small\textit{Поступила в~редакцию 27.09.21}}

\vspace*{8pt}

%\pagebreak

%\newpage

%\vspace*{-28pt}

\hrule

\vspace*{2pt}

\hrule

%\vspace*{-2pt}

\def\tit{ABOUT SEARCHING FOR~THE~OPTIMAL 3D~PRINTING SCHEME OF~STRUCTURES 
FROM~COMPOSITE MATERIALS}


\def\titkol{About searching for~the~optimal 3D~printing scheme of~structures 
from~composite materials}


\def\aut{A.\,V.~Bosov$^{1,2}$, Yu.\,I.~Bityukov$^2$, and~G.\,Yu.~Deniskina$^2$}

\def\autkol{A.\,V.~Bosov, Yu.\,I.~Bityukov, and~G.\,Yu.~Deniskina}

\titel{\tit}{\aut}{\autkol}{\titkol}

\vspace*{-11pt}


\noindent
$^1$Federal Research Center ``Computer Science and Control'' of the Russian Academy of Sciences,  
44-2~Vavilov\linebreak
$\hphantom{^1}$Str., Moscow 119333, Russian Federation

\noindent
$^2$Moscow State Aviation Institute (National Research University), 4~Volokolamskoe Shosse, 
Moscow 125933,\linebreak
$\hphantom{^1}$Russian Federation

\def\leftfootline{\small{\textbf{\thepage}
\hfill INFORMATIKA I EE PRIMENENIYA~--- INFORMATICS AND
APPLICATIONS\ \ \ 2022\ \ \ volume~16\ \ \ issue\ 1}
}%
 \def\rightfootline{\small{INFORMATIKA I EE PRIMENENIYA~---
INFORMATICS AND APPLICATIONS\ \ \ 2022\ \ \ volume~16\ \ \ issue\ 1
\hfill \textbf{\thepage}}}

\vspace*{3pt} 



\Abste{The article is devoted to finding optimal fiber laying trajectories in the manufacture of 
structures reinforced with continuous fibers by 3D printing. As the objective function of this 
optimization problem, one of the criteria for the destruction of the composite is selected. Fiber laying 
schemes during printing are modeled using analytical functions that are found from the Neumann 
problem for the Laplace equation. The boundary conditions for this problem are constructed on the 
basis of specifying the angles between the fibers and the boundary of the region. The Neumann 
problem is solved by a~conformal transformation of the print area into a circle. Thus, the criterion for 
the destruction of the composite becomes a~function of the angles that the fibers form with the 
boundary of the region. Minimization is carried out using a~genetic algorithm for searching for the 
global minimum of a~function of several variables.}

\KWE{composite materials; wavelets; 3D printing; analytical function}

\DOI{10.14357/19922264220102}

%\vspace*{-16pt}

%\Ack
%\noindent




%\vspace*{6pt}

  \begin{multicols}{2}

\renewcommand{\bibname}{\protect\rmfamily References}
%\renewcommand{\bibname}{\large\protect\rm References}

{\small\frenchspacing
 {%\baselineskip=10.8pt
 \addcontentsline{toc}{section}{References}
 \begin{thebibliography}{99}
\bibitem{1-bos-1}
 Matsuzaki Laboratory. Available at: {\sf http://www.rs.tus. ac.jp/rmatsuza/} (accessed December~20, 
2021).
\bibitem{2-bos-1}
\Aue{Vasiliev, V.\,V.} 1988. \textit{Mekhanika konstruktsiy iz kom\-po\-zi\-tsi\-on\-nykh materialov} 
[Mechanics of structures made of composite materials]. Moscow: Mashinostroenie. 272~p.
\bibitem{3-bos-1}
\Aue{Lavrent'ev, M.\,A., and B.\,V.~Shabat.} 1973. \textit{Metody teorii funktsii kompleksnogo 
peremennogo} [Methods of the theory of functions of a~complex variable]. Moscow: Nauka. 736~p.
\bibitem{4-bos-1}
\Aue{Cavaretta,~A.\,S., W.~Dahmen, and C.\,A.~Micchelli.} 1991. Stationary subdivision schemes. 
\textit{Mem. Amer. Math. Soc.} 93(453):1--186.
\bibitem{5-bos-1}
\Aue{Schroder, P., and W.~Sweldens.} 1995. Spherical wavelets: Efficiently representing functions on 
the sphere. \textit{22nd Annual Conference on Computer Graphics and Interactive Techniques 
Proceedings}. Eds. S.\,G.~Mair and R.~Cook. New York, NY: Association for Computing Machinery. 161--172.
\bibitem{6-bos-1}
\Aue{Storn, R., and K.~Price.} 1997. Differential evolution~--- a~simple and efficient heuristic for 
global optimization over continuous spaces. \textit{J.~Global Optim.} 11:341--359.
\bibitem{7-bos-1}
\Aue{Deniskina, G.\,Y., Y.\,I.~Deniskin, and Y.\,I.~Bityukov.} 2021. About biortogonal wavelets, 
created on the basis of scheme of increasing of lazy wavelets. \textit{Advances in Automation~II.} 
Lecture notes in electrical engineering ser. Cham: Springer. 729:173--181. 
\bibitem{8-bos-1}
\Aue{Novikov, I.\,Ya., V.\,Yu.~Protasov, and M.\,A.~Skopina.} 2005. \textit{Teoriya vspleskov} [The 
theory of wavelets]. Moscow: Fizmatlit. 612~p.
\bibitem{9-bos-1}
\Aue{Deniskina, G.\,Y., Y.\,I.~Deniskin, and Y.\,I.~Bityukov.} 2021. About some computational 
algorithms for locally approximation splines, based on the wavelet transformation and convolution. 
\textit{Advances in automation~II.}
Eds. A.\,A.~Radionov and V.\,R.~Gasiyarov. Lecture notes in electrical engineering ser. Cham,
Switzerland: Springer. 
729:182--191.  doi: 
10.1007/978-3-030-71119-1\_19.
\bibitem{10-bos-1}
\Aue{Marchuk, G.\,I., and G.\,P.~Akilov.} 1989. \textit{Metody vychislitel'noy matematiki} [Methods 
of computational mathematics]. Moscow: Nauka. 744~p.
\end{thebibliography}

 }
 }

\end{multicols}

\vspace*{-6pt}

\hfill{\small\textit{Received September 27, 2021}}

%\pagebreak

%\vspace*{-18pt}

\Contr


\noindent
\textbf{Bosov Alexey V.} (b.\ 1969)~--- Doctor of Science in technology, principal scientist, Institute 
of Informatics Problems, Federal Research Center ``Computer Science and Control'' of the Russian 
Academy of Sciences, 44-2~Vavilov Str., Moscow 119333, Russian Federation; professor, Moscow 
State Aviation Institute (National Research University), 4~Volokolamskoe Shosse, Moscow 125933, 
Russian Federation; \mbox{AVBosov@ipiran.ru}


\vspace*{3pt}

\noindent
\textbf{Bityukov Yuri I.} (b.\ 1972)~--- Doctor of Science in technology, professor, Moscow Aviation 
Institute (National Research University), 4~Volokolamskoe Shosse, Moscow 125933, Russian 
Federation; \mbox{yib72@mail.ru}


\vspace*{3pt}

\noindent
\textbf{Deniskina Galina Yu.} (b.\ 1993)~--- PhD student, Faculty of Information Technologies and 
Applied Mathematics, Moscow Aviation Institute (National Research University), 4~Volokolamskoe 
Shosse, Moscow 125933, Russian Federation; \mbox{dega17@yandex.ru}

\label{end\stat}

\renewcommand{\bibname}{\protect\rm Литература} 