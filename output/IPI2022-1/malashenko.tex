
\def\stat{malashenko}

\def\tit{МЕТРИЧЕСКИЕ ОЦЕНКИ УГЛОВЫХ ТОЧЕК МНОЖЕСТВА ДОСТИЖИМЫХ
МЕЖУЗЛОВЫХ ПОТОКОВ МНОГОПОЛЬЗОВАТЕЛЬСКОЙ СЕТИ}

\def\titkol{Метрические оценки угловых точек множества достижимых
межузловых потоков многопользовательской сети}

\def\aut{Ю.\,Е.~Малашенко$^1$}

\def\autkol{Ю.\,Е.~Малашенко}

\titel{\tit}{\aut}{\autkol}{\titkol}

\index{Малашенко Ю.\,Е.}
\index{Malashenko Yu.\,E.}


%{\renewcommand{\thefootnote}{\fnsymbol{footnote}} \footnotetext[1]
%{Работа выполнена при поддержке Министерства науки и~высшего образования Российской Федерации (проект 
%075-15-2020-799).}}


\renewcommand{\thefootnote}{\arabic{footnote}}
\footnotetext[1]{Федеральный
исследовательский центр <<Информатика и~управление>> Российской
академии наук, \mbox{mala-yur@yandex.ru}}

%\vspace*{-6pt}

 

\Abst{Рассматривается модель
многопользовательской сети связи, в~которой между всеми парами
уз\-лов-кор\-рес\-пон\-ден\-тов одновременно передаются информационные
потоки различных видов. Анализируются результаты вы\-чис\-ли\-тель\-ных
экспериментов по оценке мет\-ри\-че\-ских характеристик выпуклого
множества допустимых межузловых потоков и~угловых точек,
расположенных на гранях. {Базовыми} считаются угловые точ\-ки
пересечения внеш\-ней границы множества с~осями координат, каждая из
которых соответствует максимальному межузловому потоку
определенного вида. Для определения координат {опорных}
угловых точек вычисляются   значения допустимых межузловых
потоков, при одновременной передаче которых достигается предельная
загрузка всех ребер сети. Вы\-чис\-ля\-ют\-ся так\-же координаты
{реперной} точки,  в~которой сумма межузловых потоков
достигает максимально воз\-мож\-ной величины на множестве
до\-сти\-жи\-мости. В~ходе вы\-чис\-ли\-тель\-ных экспериментов в~условных
единицах потоков оцениваются нормы век\-то\-ров, со\-от\-вет\-ст\-ву\-ющих
координатам  угловых точек. Приводятся сравнительные диаграммы
услов\-ных {рас\-сто\-яний между реперной и~опорными угловыми}
точками.}

\KW{многопродуктовая  сетевая  модель;
множество  допустимых межузловых потоков; предельная за\-груз\-ка сети}

\DOI{10.14357/19922264220104}
  
%\vspace*{-4pt}


\vskip 10pt plus 9pt minus 6pt

\thispagestyle{headings}

\begin{multicols}{2}

\label{st\stat}

\section{Введение}

Для   многопользовательской сети связи предлагается развитие
методов агрегированного описания множества до\-пус\-ти\-мых  межузловых
потоков различных видов,  которые  могут одновременно передаваться
между  всеми парами  уз\-лов-кор\-рес\-пон\-ден\-тов.  Для  оценки  
и~аппроксимации множества предельно достижимых многопродуктовых
потоков строится внут\-рен\-ний опорный каркас~[1].  В~данной работе
анализируются результаты вы\-чис\-ли\-тель\-ных экспериментов по  поиску
предельных распределений  межузловых потоков  и~оценке
мет\-ри\-че\-ских особенностей  в~расположении   со\-от\-вет\-ст\-ву\-ющих угловых
точек опорного каркаса~[2].  В~ходе выполнения вы\-чис\-ли\-тель\-ной
процедуры  на первом этапе  определяются   точ\-ки  пересечения осей
координат  с~границей множества до\-сти\-жи\-мости. Координаты базовых
угловых точек  находятся как решение  по\-сле\-до\-ва\-тель\-ности
задач поиска максимального однопродуктового      потока   меж\-ду
соответствующей   парой узлов  при фиксированных нулевых
значениях для всех остальных [3,~4]. Координаты \textit{опорной
угловой  точки определяются  значениями  компонент} вектора
межузловых потоков, при которых достигается предельная загрузка
\textit{сети}. При по\-стро\-ении внут\-рен\-не\-го опорного  каркаса
формируется  реперный вектор межузловых  потоков, сумма  компонент
которого  достигает максимальной величины на данном множестве.
Для  оценки допустимых распределений межузловых потоков
вы\-чис\-ля\-ют\-ся метрические характеристики векторов  и~строятся
диаграммы  рас\-сто\-яний между всеми угловыми точками опорного
каркаса. Вы\-чис\-ли\-тель\-ные эксперименты    поз\-во\-ля\-ют формировать  
и~в~явном виде    выписывать компоненты векторов  межузловых
потоков,  вы\-пук\-лая комбинация которых   может быть использована
для гарантированной оценки    функциональных характеристик сети
при предельной за\-груз\-ке всех ребер.


\section{Математическая модель}

В качестве математической модели для описания
многопользовательской сетевой сис\-те\-мы используется сле\-ду\-ющая
формальная запись условий и~ограничений, которые долж\-ны
выполняться при одновременной передаче потоков различных видов
меж\-ду всеми парами уз\-лов-кор\-рес\-пон\-ден\-тов.

Сеть $G(\mathbf{d})$ задается множествами $\langle V,
R,U,P\rangle$:
\begin{itemize}
\item узлов (вершин) сети $V\hm\!=\!\{v_{1}, v_{2},
\dots,v_{n},\dots,v_{N}\}$;

\item неориентированных ребер $R\hm\!=\!\{r_{1}, r_{2}, \dots, r_{k}, \dots\linebreak
\dots,
r_{E}\}$.
\end{itemize}

Ребро $r_{k}$ \textit{соединяет} концевые вершины~$v_{n_k}$ 
и~$v_{j_k}$. Реб\-ру~$r_{k}$ ставятся в~соответствие две
ориентированные дуги $\{u_{k},u_{k+E}\}$ из множества
ориентированных дуг $U\hm=\{u_{1}, u_{2}, \dots, u_{k}, \dots,
u_{2E}\}$. Дуги $\{u_{k}, u_{k+E}\}$ определяют прямое и~обратное
на\-прав\-ле\-ние передачи потока по реб\-ру~$r_{k}$ меж\-ду концевыми
вершинами $\{v_{n_k}, v_{j_k}\}$.

В многопользовательской сети $G(\mathbf{d})$ рас\-смат\-ри\-ва\-ет\-ся
$M\hm=N(N\hm-1)$ независимых, не\-вза\-и\-мо\-за\-ме\-ня\-емых и~рав\-но\-прав\-ных потоков
различных видов, которые передаются между уз\-ла\-ми-кор\-рес\-пон\-ден\-та\-ми
из множества $P\hm=\{p_{1}, p_{2}, \dots, p_{M}\}$.

По определению, каждой паре уз\-лов-кор\-рес\-пон\-ден\-тов $p_{m}$
соответствуют:
\begin{itemize}
\item вершина-источник с~номером~$s_{m}$, через которую вход\-ной поток
$m$-го вида~$z_{m}$ по\-сту\-па\-ет в~сеть;

\item вершина-при\-ем\-ник с~номером~$t_{m}$, из которой поток $m$-го
вида~$z_{m}$ покидает сеть.
\end{itemize}

В множестве~$P$ выделяется подмножество~$P(R^{+})$ пар
уз\-лов-кор\-рес\-пон\-ден\-тов, расположенных в~концевых вершинах
ребра~$r_{k}$, $k\hm=\overline{1,E}$. Вводятся сле\-ду\-ющие обозначения:
пусть реб\-ро~$r_{k}$ со\-еди\-ня\-ет вершины с~номерами~$n$ и~$j$ такими,
что $n\hm< j$. Для соответствующей пары уз\-лов-кор\-рес\-пон\-ден\-тов~$p_{k}$, 
расположенных в~узлах $\{v_{n}, v_{j}\}$, узел~$v_{n}$
считается источником, а~узел~$v_{j}$~--- приемником потока~$z_{k}$
$k$-го вида, который передается из узла 
c~номером~$n$ в~узел с~номером~$j$ для пары~$p_{k}$. Для пары $p_{k+Е}
\hm\Longleftrightarrow \{v_{j},v_{n}\}$ узел~$v_{j}$ считается
источником и~устанавливается соответствие
$s_{k}\hm\Longleftrightarrow j$, а~узел~$v_{n}$~--- приемником 
и~$t_{k} \hm\Longleftrightarrow n$.

Пары $p_{m}$ из подмножества~$P(R^{+})$ называются
\textit{смежными} уз\-ла\-ми-кор\-рес\-пон\-ден\-та\-ми. Все остальные пары
уз\-лов-кор\-рес\-пон\-ден\-тов относятся к~множеству~$P(R^{-})$:
\begin{gather*}
P=P(R^{+})\cup P(R^{-});\\
P(R^{+}) \cap P(R^{-}) = \varnothing.
\end{gather*}

Введем обозначения:
\begin{description}
\item[\,] $z_{m}$~--- величина \textit{межузлового} потока $m$-го вида,
который поступает в~сеть из узла с~номером~$s_{m }$ и~покидает из
узла с~номером~$t_{m}$;
\item[\,]
$S(v_{n})$~--- множество номеров исходящих дуг, по которым поток
покидает узел~$v_{n}$;
\item[\,]
$T(v_{n})$~--- множество номеров входящих дуг, по которым поток
поступает в~узел~$v_{n}$.
\end{description}

Во всех узлах $v_{n}\in V$, $n\hm=\overline{1,N}$, для всех видов
потоков долж\-ны выполняться условия сохранения потоков:

\noindent
\begin{multline}
\label{eq1} 
\sum\limits_{i\in S(v_n )} \!x_{mi}-\sum\limits_{i\in T(v_n )}\! x_{mi}
={}\\
{}=\begin{cases}
z_m, &\mbox{если } v=v^{}_{S_m}; \\
-z_m,&\mbox{если } v=v_{t_m}; \\
0&\mbox{в остальных случаях}, \\
\end{cases}
\end{multline}
$n=\overline{1,N}$, $m\hm=\overline{1,M}$, $x_{mi}\hm\ge 0$,
$z_{m}\hm\ge0$.

Величина $z_{m}$ равна входному потоку $m$-го вида, который
пропускается от источника к~приемнику пары~$p_{m}$ при
распределении потоков~$x_{mi}$ по дугам сети.

Каждому ребру $r_{k}\hm\in R$ приписывается неотрицательное чис\-ло~$d_{k}$, 
опре\-де\-ля\-ющее суммарный предельно допустимый поток,
который мож\-но передать по реб\-ру~$r_{k}$ в~обоих на\-прав\-ле\-ни\-ях. 
В~исходной сети компоненты век\-то\-ра про\-пуск\-ных способностей
$\mathbf{d}=(d_{1}, d_{2},\dots, d_{k}, \dots, d_{E})$~--- наперед
заданные положительные чис\-ла $d_{k}\hm > 0$. 
Вектором~$\mathbf{d}$ определяются сле\-ду\-ющие ограничения на сум\-му
дуговых потоков всех видов, пе\-ре\-да\-ва\-емых по реб\-ру~$r_{k}$:
\begin{multline}
\label{eq2} 
\sum\limits_{m=1}^M \left(x_{mk}+x_{m(k+E)}\right) \le d_k,\\
x_{mk} \ge 0\,,\enskip x_{m(k+E)}\ge 0\,, \enskip  k=\overline {1,E}.
\end{multline}
В рамках данной модели про\-пуск\-ная спо\-соб\-ность ребер сети~--- вектор~$\mathbf{d}$~--- 
трактуется как <<\textit{ресурсное ограничение}>>,
а~сумма дуговых
 потоков рас\-смат\-ри\-ва\-ет\-ся как показатель использования
<<\textit{ресурсов}>> сети при передаче межузловых потоков
различных видов.

Для всех $z_{m}$ и~$x_{mi}$, удовле\-тво\-ря\-ющих
условиям~\eqref{eq1} и~\eqref{eq2}, вы\-чис\-ля\-ют\-ся суммарные потоки:
\begin{equation}
\label{eq3} 
y_{m }=\sum\limits_{i=1}^{2E} {x}_{mi},\enskip
m=\overline{1,M}\,.
\end{equation}

Суммарный реберный поток $y_{m}$ характеризует
<<\textit{нагрузку}>> на сеть при передаче межузлового потока
величины $z_{m}$ из уз\-ла-ис\-точ\-ни\-ка~$s_{m}$ в~узел-при\-ем\-ник~$t_{m}$. 
Величина~$y_{m}$ показывает, какой суммарный
\textit{ресурс}~--- про\-пуск\-ная спо\-соб\-ность сети~--- требуется для
передачи межузлового потока~$z_{m}$, а~отношение
$w_{m}\hm={y_m}/{z_m}$, $m\hm=\overline{1,M},$
показывает, какие \textit{ресурсы} необходимы для передачи
единичного потока $m$-го вида меж\-ду узлами~$s_{m}$ и~$t_{m}$.

Ограничения~\eqref{eq1}--\eqref{eq3} задают подмножество
допустимых значений компонент вектора межузловых потоков
$\mathbf{z}\hm=(z_{1}, z_{2},\dots,z_{m},\dots,z_{M})$:
\begin{equation}
\label{eq4} 
{Z}(\mathbf{d})\!=\!\{\mathbf{z} \ge 0 \mid\!
(\mathbf{z},\mathbf{x},\mathbf{y}) \mbox{ удовлетворяют }
\text{\eqref{eq1}--\eqref{eq3}}\!\},\!\!
\end{equation}
а все допустимые распределения ресурсов принадлежат подмножеству
\begin{equation*}
%\label{eq5} 
{Y}(\mathbf{d})=\{\mathbf{y} \ge 0 \mid
(\mathbf{z},\mathbf{x},\mathbf{y}) \mbox{ удовлетворяют }
\text{\eqref{eq1}--\eqref{eq3}}\}.
\end{equation*}


\section{Базовые угловые точки}

В рамках данного модельного описания, по определению, монопольным
режимом называется способ управ\-ле\-ния, при котором все ресурсы сети
используются для передачи потока одной выделенной пары
уз\-лов-кор\-рес\-пон\-ден\-тов $p_{a}\hm\in P$, а~для всех остальных
потоки полагаются равными нулю.

Предельно допустимый поток, который можно передать между
фиксированной парой уз\-лов-кор\-рес\-пон\-ден\-тов~$p_{a}$ в~монопольном
режиме, является решением стан\-дарт\-ной, в~данном случае
однопродуктовой, задачи о~максимальном потоке.

\smallskip

\noindent
\textbf{Задача 1.} Найти:
$$
z_a^0=\max\limits_{\langle z,x\rangle \in Z(d)} z_a
$$
при условии $z_{i}=0$, $i\hm=\overline{1,M}$, $i\hm\ne a$.

\smallskip

При решении задачи~1 для пары $p_{a}$ вы\-чис\-ля\-ют\-ся: межузловой
поток~$z_a^0$; дуговые потоки $\{x^{0}_{ak};x^{0}_{a(k+E)}\}$,
$k\hm=\overline{1,E}$; суммарное значение реберного
потока~$y_{a}^{0}\hm=\sum\nolimits_{i=1}^{2E} {x}_{ai}^{0}$.

Поток величины~$z_a^0$ называется МРМ-по\-то\-ком и~является
\textit{максимальным потоком}, пе\-ре\-да\-ва\-емым в~\textit{монопольном
режиме} для пары уз\-лов-кор\-рес\-пон\-ден\-тов~$p_{a}$.

Задача~1 решается последовательно для всех $p_{m}\hm\in P$,
вы\-чис\-ля\-ют\-ся значения $
z_{m}^{0}$ для $m=\overline {1,M}.$

Вектор $\mathbf{z}^{0}(m)\hm=\langle 0, 0, \dots,z_m^0,
\dots,0,0\rangle$ определяет координаты угловой точ\-ки множества
до\-сти\-жи\-мости~$Z(\mathbf{d})$, которая лежит на пересечении
границы~$Z(\mathbf{d})$ с~со\-от\-вет\-ст\-ву\-ющей координатной
\mbox{осью}~$z_{m}$.

Множество векторов
\begin{equation*}
%\label{eq6} 
\mathbf{Z}^{(0)}=\langle \mathbf{z}^{0}(1),
\mathbf{z}^{0}(2), \dots, \mathbf{z}^0(M)\rangle
\end{equation*}
определяет угловые точки \textit{базового} МРМ-се\-че\-ния.


\section{Реперная угловая точка}

В множестве $Z(\mathbf{d})$ существует допустимый вектор~$\mathbf{z}(d)$ c~компонентами
\begin{align*}
z_{m}(d)&=0,\ p_{m}\in P(R^{-});\\
z_{k}(d)&=d_{k},\ p_{k}\in P(R^{+}).
\end{align*}

Сумма межузловых потоков
$$
\sigma(d)=\sum\limits_{k=1}^Ez_{k}(d)= \sum\limits_{k=1}^E d_{k}=\mathrm{D}^*
$$
является максимально возможной среди всех до\-пус\-ти\-мых век\-то\-ров
$\mathbf{z}\hm\in Z(\mathbf{d})$.

Для каждой пары $p_{k}\in P(R^{+})$ межузловому потоку~$z_{k}(d)$
соответствует реберный поток~$x^{0}_{k}(d)$, 
а~$y^{0}_{k}(d)\hm=x^{0}_{k}(d)$. Суммарная \textit{нагрузка}
$$
\sum\limits_{k=1}^{2E} y^{0}_{k}(d)= \sum\limits_{k=1}^{2E}
 x^{0}_{k}(d)=\sum\limits_{k=1}^E z_{k}(d)=
\sum\limits_{k=1}^E d_{k}=\mathrm{D}^*.
$$

Вектор $\mathbf{z}(d)$ задает распределение PLD-по\-то\-ков (от \textit{англ}.\
Peak Load Distribution), поскольку при одновременной передаче всех
межузловых потоков $z_{k}(d)$, $p_{k}\hm\in P(R^{+})$, достигается
\textit{предельно до\-пус\-ти\-мая за\-груз\-ка} всех ребер сети.

Вектор определяет координаты \textit{реперной} точ\-ки на границе
множества~$Z(\mathbf{d})$.

\section{Опорный внутренний каркас}

Пусть для некоторой пары $p_{a}\hm\in P$ в~результате решения задачи~1 
найден МРМ-по\-ток~$z_{a}^{0}$ и~дуговые потоки
$\{{x}^{0}_{ak};x^{0}_{a(k+E)}\}$, $k\hm=\overline{1,E}$. 
Значению~$z_{a}^{0}$ ставится в~соответствие вектор
$$
\mathbf{z}^{1}(a)=\left\langle{z}^{1}_{1}(a), z^{1}_{2}(a),
z^{1}_{3}(a),\dots, z_a^1(a),\dots, z^{1}_{M}(a)\right\rangle
$$
с компонентами
\begin{align*}
z_{m}^{1}(a)&=0 \mbox{ для } m\ne a,\ p_{m}\in P(R^{-});\\
z_{a}^{1}(a)&=z_{a}^{0};\\
z_{k}^{1}(a)&=d_{k}- (x^{0}_{ak}+x^{0}_{a(k+E)}),\ p_{k}\in
P(R^{+})\\
& \hspace*{40mm}\mbox{ для\ всех } k=\overline{1,k}\,.
\end{align*}

По построению, для любого вектора $\mathbf{z}^{1}(j)$, $p_{j}\hm\in
P$, суммарная (общая) \textit{за\-грузка}:
\begin{multline*}
\sigma(j)=\sum\limits_{m=1}^M y_{m}^{1}(j)=
\sum\limits_{k=1}^E \left(x^{0}_{jk}+x^{0}_{j(k+E)}\right)+{}\\
{}+\sum\limits_{k=1}^E 
\left[d_{k}-\left(x^{0}_{jk}+x^{0}_{j(k+E)}\right)\right]={}\\
{}=\sum\limits_{k=1}^E d_{k} +\sum\limits_{k=1}^E
\left({x}^{0}_{jk}+x^{0}_{j(k+E)}\right)-{}\\
{}-\sum\limits_{k=1}^E
\left(x^{0}_{jk}+x^{0}_{j(k+E)}\right)=\mathrm{D}^*.
\end{multline*}

\begin{figure*}[b] %fig1
\vspace*{1pt}
  \begin{center} 
   \mbox{%
\epsfxsize=153.408mm
\epsfbox{mal-1.eps}
}

\end{center}
\vspace*{-9pt}
\Caption{Модели сетевых сис\-тем: (\textit{а})~базовая; (\textit{б})~кольцевая
}
\end{figure*}

Множество векторов
$$
\mathbf{Z}^{1}=\left\{\mathbf{z}^{1}(1), \mathbf{z}^{1}(2),
\mathbf{z}^{1}(3),\dots, \mathbf{z}^{1}(m),\dots,
\mathbf{z}^{1}(M)\right\}
$$ 
определяет угловые точ\-ки PLD-се\-че\-ния. На
основе вектора~$\mathbf{z}(d)$ и~угловых век\-то\-ров из множеств~$\mathbf{Z}^{0}$ 
и~$\mathbf{Z}^{1}$ формируется опор\-ный внут\-рен\-ний каркас:
$$
\operatorname{SiF}(1)=\left\{\mathbf{z}(d),
\mathbf{Z}^{0},\mathbf{Z}^{1}\right\}
$$ 
(от \textit{англ.}\ Support internal
Frame~--- опорный внут\-рен\-ний каркас). Любая вы\-пук\-лая комбинация
век\-то\-ров из множества~$\operatorname{SiF}(1)$ задает до\-пус\-ти\-мое
распределение потоков, которые могут одновременно передаваться
меж\-ду всеми парами уз\-лов-кор\-рес\-пон\-ден\-тов.

\section{Вычислительный эксперимент}

Результаты вычислительных экспериментов, описанные ниже, служат
продолжением исследований, начатых в~[1]. Вы\-чис\-ли\-тель\-ные
эксперименты проводились на моделях сетевых сис\-тем, пред\-став\-лен\-ных
на рис.~1. В~каждой
сети~69~узлов. Про\-пуск\-ные способности ребер~-- значения $d_k$~---
выбирались случайным образом из от\-рез\-ка $[900,999]$ и~совпадали
для ребер, при\-сут\-ст\-ву\-ющих в~обеих сетях. В~кольцевой сети
про\-пуск\-ная спо\-соб\-ность каж\-до\-го из до\-бав\-лен\-ных ребер рав\-ня\-лась~900.



В рамках вычислительных экспериментов оценивались
\textit{мет\-ри\-че\-ские рас\-сто\-яния}   меж\-ду угловыми точками
$\operatorname{SiF}(1)$-каркаса, координаты которых задаются
значениями компонент со\-от\-вет\-ст\-ву\-ющих век\-то\-ров. На первом этапе
вы\-чис\-ля\-лись \textit{рас\-сто\-яния} между реперной и~всеми базовыми
угловыми точ\-ками:
\begin{multline*}
\rho_m^0=\rho(\mathbf{z}(d)), \mathbf z^0(m))= \|\mathbf
z(d)-\mathbf{z}^0(m)\|={}\\
{}=
\left[\sum\limits_{j=1}^M\left(z_j(d)-z_j^0(m)\right)^2\right]^{1/2}.
\end{multline*}
для всех $p_m\in P$, $m\hm=1,M$.





\begin{figure*} %fig2
\vspace*{1pt}
  \begin{center}  
    \mbox{%
\epsfxsize=153.085mm
\epsfbox{mal-2.eps}
}

\end{center}
\vspace*{-12pt}
\Caption{Диаграммы распределения расстояний между реперной и базовыми угловыми точками:
(\textit{а})~базовая сеть; (\textit{б})~кольцевая сеть}
\label{fig2}
%\end{figure*}
%\begin{figure*} %fig3
\vspace*{3pt}
  \begin{center}  
    \mbox{%
\epsfxsize=152.768mm
\epsfbox{mal-4.eps}
}

\end{center}
\vspace*{-12pt}
\Caption{Диаграммы распределения расстояний от опорных угловых точек до
начала координат и~реперной точ\-ки: (\textit{а})~базовая сеть; (\textit{б})~кольцевая сеть}
\label{fig4}
\end{figure*}




Диаграммы распределения \textit{расстояний} меж\-ду реперной 
и~базовыми угловыми точ\-ка\-ми пред\-ставле\-ны на рис.~\ref{fig2}. 
Диа\-грам\-мы в~правом столбце\linebreak
относятся к~угловым точ\-кам, со\-от\-вет\-ст\-ву\-ющим смеж\-ным парам
корреспондентов $p_k\hm\in P(R_+)$, а~в~левом столбце~--- для $p_m\hm\in  P(R_-)$.
Значения $\rho^0_{(\,\cdot\,)}$ упорядочены по величине от
большего к~меньшему (по не\-воз\-рас\-та\-нию), а~значения
 $\pi_+(k)\hm={k}/{M_+}$ и~$\pi_-(m)\hm={m}/{M_-}$ по горизонтальным осям определялись 
 в~соответствии с~номерами в~спис\-ках $k\hm\in R_+$ и~$m\hm\in R_-$, 
 где $M_+\hm=\vert P(R_+)\vert $ и~$M_-\hm=\vert P(R_-)\vert$~--- общее чис\-ло элементов в~списке.

Тонкие пунктирные линии на диаграммах указывают
\textit{рас\-сто\-яние} реперной угловой точ\-ки от начала координат~---
значение нор\-мы вектора:
$$
\rho^d=\rho(0,\mathbf{z}(d))=\|\mathbf{z}(d)\|.
$$





Компоненты вектора $\mathbf{z}(d)$ определяют Па\-ре\-то-оп\-ти\-маль\-ное
распределение межузловых потоков. Анализ диа\-грамм на
рис.~\ref{fig2} показывает, что рас\-сто\-яния от реперной
угловой точ\-ки до всех базовых угловых точек $\rho_m^0\hm>\rho^d$,
$p_m \hm\in P(R_-)$,
 для всех несмежных пар. При этом рас\-сто\-яния $\rho_k^0\hm\le\rho^d$, $p_k \hm\in P(R_+)$,
для~80\% базовых угловых точек смеж\-ных пар. Следовательно, 
в~множестве~(\ref{eq4}) до\-пус\-ти\-мых межузловых потоков под\-мно\-же\-ст\-во угловых
базовых точек смеж\-ных пар расположено ближе к~максимальной
<<\textit{линии уровня}>> для суммы до\-пус\-ти\-мых межузловых потоков:
$\sum\nolimits_m^\prime z_m\hm=D^*$. Указанную структурную осо\-бен\-ность следует
учитывать при решении оптимизационных задач с~линейным
функционалом *на-сум\-му-по\-то\-ков* при поиске распределений потоков
различных видов.



\textit{Распределение расстояний} от опорных угловых точек до
начала координат и~реперной точ\-ки пред\-став\-ле\-но на рис.~\ref{fig4} ($m\hm=\overline{1,M}$):
\begin{align*}
\rho_m^1&=\rho(\boldsymbol{0}, \mathbf{z}^1(m))=
\|\mathbf{z}^1(m))\|=
\left[\sum\limits_{j=1}^Mz_j^2(m)\right]^{1/2};\\
\rho_m^\Delta&=\rho(\mathbf{z}(d), \mathbf{z}^1(m))=
\left[\sum\limits_{j=1}^M(z_j(d)-z_j^1(m))^2\right]^{1/2}.
\end{align*}


\noindent
Верхние кривые соответствуют \textit{расстояниям} от начала
координат, а~ниж\-ние~--- рас\-сто\-яни\-ям до реперной угловой \mbox{точки}.
%
Диаграммы в~правом столбце относятся к~смеж\-ным парам
 и~поз\-во\-ля\-ют уточ\-нить результаты анализа кривых на
рис.~\ref{fig2}. Из графиков следует, что более~20\%
опор\-ных угловых точек смеж\-ных пар~$\rho_k^\Delta$, $p_k\hm\in
P(R_+)$, \textit{совпадают} с~реперной угловой точ\-кой $\mathbf{z}(d)$. 
Указанные угловые опор\-ные точ\-ки соответствуют парам,
которые имеют \textit{единственно} воз\-мож\-ный путь со\-еди\-не\-ния по
смеж\-но\-му реб\-ру и~получают привилегированный до\-ступ к~остаточной
про\-пуск\-ной спо\-соб\-ности при любой \textit{уравнительной
недискриминирующей} стратегии распределения меж\-уз\-ло\-вых потоков.
Диаграммы в~левом столбце (см.\ рис.~3) характеризуют
расположение опорных угловых точек для несмежных пар и~рас\-сто\-яния
от со\-от\-вет\-ст\-ву\-ющих базовых граничных точек на координатных осях.



\section{Заключение}

В статье~[1] был предложен способ по\-стро\-ения внут\-рен\-не\-го опор\-но\-го
кар\-ка\-са  для   исследования  предельных функциональных
возможностей многопользовательской  сети~[1,~2]. В~данной работе
рас\-смат\-ри\-ва\-ют\-ся результаты вы\-чис\-ли\-тель\-ных экспериментов по  поиску
край\-них точек опорного кар\-ка\-са, соответствующих до\-пус\-ти\-мым
распределениям межузловых   потоков различных видов,  при
одновременной передаче которых достигается пол\-ная загрузка всех
ребер  сети.    При  формировании   базовых  век\-то\-ров  опорного
каркаса  многократно решается задача  о~максимальном потоке  
и~минимальном раз\-ре\-зе~[3, 4]. Ре\-зуль\-ти\-ру\-ющие вы\-чис\-ли\-тель\-ные за\-тра\-ты
оцениваются полиномиальной функцией от  общего  чис\-ла узлов
сети~[4].


{\small\frenchspacing
 {%\baselineskip=10.8pt
 %\addcontentsline{toc}{section}{References}
 \begin{thebibliography}{9}

\bibitem{1-mal}
\Au{Малашенко~Ю.\,Е.}   Максимальные   межузловые потоки
при предельной за\-груз\-ке  многопользовательской сети~//  Информатика и~её применения, 
2021. Т.~15. Вып.~3.  С.~24--28.

\bibitem{2-mal}
\Au{Лотов~А.\,В., Поспелова~И.\,И.} Многокритериальные задачи
принятия решений.~--- М.: Макс Пресс, 2008. 197~с.

\bibitem{3-mal}
\Au{Йенсен~П., Барнес~Д.} Потоковое программирование~/ Пер.
с~англ.~--- М.: Радио и~связь, 1984. 392~с. (\Au{Jensen~P.\,A.,
Barnes~J.\,W.} Network flow programming.~--- New York, NY, USA:
Wiley, 1980. 408~p.)


\bibitem{4-mal}
\Au{Кормен~Т.\,Х., Лейзерсон~Ч.\,И., Ривест~Р.\,Л., Штайн~К.}
Алгоритмы: по\-стро\-ение и~анализ~/ Пер. с~англ.~--- М.: Вильямс,
2005. 1296~c. (\Au{Cormen~T.\,H., Leiserson~C.\,E.,
Rivest~R.\,L.,  Stein~C.} Introduction to algorithms.~--- New York,
NY, USA: McGraw-Hill, 2001.  1056~p.)
\end{thebibliography}

 }
 }

\end{multicols}

\vspace*{-9pt}

\hfill{\small\textit{Поступила в~редакцию 14.12.21}}

\vspace*{8pt}

%\pagebreak

%\newpage

%\vspace*{-28pt}

\hrule

\vspace*{2pt}

\hrule

%\vspace*{-2pt}

\def\tit{METRIC EVALUATIONS OF~THE~ANGULAR POINTS OF~THE~SET OF~ATTAINABLE INTERNODAL 
FLOWS OF~MULTIUSER NETWORK}


\def\titkol{Metric evaluations of the angular points of~the~set of~attainable internodal 
flows of~multiuser network}


\def\aut{Yu.\,E.~Malashenko}

\def\autkol{Yu.\,E.~Malashenko}

\titel{\tit}{\aut}{\autkol}{\titkol}

\vspace*{-15pt}


\noindent
Federal Research Center ``Computer Science and Control'' of the Russian Academy of Sciences, 
44-2~Vavilov Str., Moscow 119333, Russian Federation

\def\leftfootline{\small{\textbf{\thepage}
\hfill INFORMATIKA I EE PRIMENENIYA~--- INFORMATICS AND
APPLICATIONS\ \ \ 2022\ \ \ volume~16\ \ \ issue\ 1}
}%
 \def\rightfootline{\small{INFORMATIKA I EE PRIMENENIYA~---
INFORMATICS AND APPLICATIONS\ \ \ 2022\ \ \ volume~16\ \ \ issue\ 1
\hfill \textbf{\thepage}}}

\vspace*{3pt} 



\Abste{The paper considers a~model of a multiuser communication network in which information 
flows of various types are simultaneously transmitted between all pairs of correspondent nodes. 
The results of computational experiments for evaluation of the metric characteristics of 
a~convex set of attainable interstitial flows and angular points located on the faces are analyzed. 
The angular points of intersection of the outer boundary of the set with the coordinate axes, 
each of which corresponds to the maximum internodal flow of a~certain type, are considered as basic-point. 
To determine the coordinates of the support-angular-points, the values of permissible internodal 
flows are calculated, with simultaneous transmission of which the maximum load of all network 
edges is achieved. The coordinates of the bench-mark-point at which the sum of the internodal 
flows reaches the maximum possible value on the attainable set are also calculated.
 The norm of vectors corresponding to the coordinates of angular points is estimated in 
 conditional units of flows during the computational experiments. Comparative diagrams
  of conditional distances between the bench-mark and support-angular points are provided. 
  Computational experiments make it possible to write an explicit form of the vectors of 
  internodal flows, the convex combination of which can be used to guarantee the evaluation 
  of the functional characteristics of the network at the maximum load of all edges. 
  When forming the basic vectors of the support frame, the problem of the maximum 
  flow and the minimum cut is repeatedly solved. The resulting computational costs are 
  estimated by a~polynomial function of the total number of network nodes.}

\KWE{multicommodity network model; internodal flows set; network peak-load}

\DOI{10.14357/19922264220104}

%\vspace*{-16pt}

%\Ack
%\noindent

\pagebreak


%\vspace*{6pt}

  \begin{multicols}{2}

\renewcommand{\bibname}{\protect\rmfamily References}
%\renewcommand{\bibname}{\large\protect\rm References}

{\small\frenchspacing
 {%\baselineskip=10.8pt
 \addcontentsline{toc}{section}{References}
 \begin{thebibliography}{9}
\bibitem{1-mal-1}
\Aue{Malashenko, Yu.\,E.}
 2021. Maksimal'nye mezhuzlovye potoki pri predel'noy 
 zagruzke mnogopol'zovatel'skoy seti [Maximum internode flows at peak load of a~multiuser network]. 
 \textit{Informatika i~ee Primeneniya~--- Inform. Appl.} 15(3):24--28.
 
 \columnbreak
 
\bibitem{2-mal-1}
\Aue{Lotov, A.\,V., and I.\,I.~Pospelova.}
 2008. \textit{Mnogokriterial'nye zadachi prinyatiya resheniy} [Multicriteria decision-making problems]. 
 Moscow: Maks Press. 197~p.
 
 \vspace*{-3pt}
 
 
\bibitem{3-mal-1}
\Aue{Jensen, P.\,A., and J.\,W.~Barnes.}
 1980. \textit{Network flow programming}. New York, NY: Wiley. 408~p.
 
  \vspace*{-3pt}
  
\bibitem{4-mal-1}
\Aue{Cormen, T.\,H., C.\,E.~Leiserson, R.\,L.~Rivest, and C.~Stein.}
 2001. \textit{Introduction to algorithms}. New York, NY: McGraw-Hill. 1056~p.
 
 \end{thebibliography}

 }
 }

\end{multicols}

\vspace*{-6pt}

\hfill{\small\textit{Received December 14, 2021}}

%\pagebreak

%\vspace*{-18pt}
 
\Contrl

\noindent
\textbf{Malashenko Yuri E.} (b.\ 1946)~--- 
Doctor of Science in physics and mathematics, principal scientist, Federal Research Center 
``Computer Science and Control'' of the Russian Academy of Sciences, 
44-2~Vavilov Str., Moscow 119333, Russian Federation; \mbox{malash09@ccas.ru}


\label{end\stat}

\renewcommand{\bibname}{\protect\rm Литература} 