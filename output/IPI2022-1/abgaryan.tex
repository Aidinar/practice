\def\stat{abgaryan}

\def\tit{ПРОГРАММНЫЙ КОМПЛЕКС ДЛЯ~МНОГОМАСШТАБНОГО МОДЕЛИРОВАНИЯ 
СТРУКТУРНЫХ СВОЙСТВ КОМПОЗИЦИОННЫХ МАТЕРИАЛОВ$^*$}

\def\titkol{Программный комплекс для многомасштабного моделирования 
структурных свойств композиционных материалов}

\def\aut{К.\,К.~Абгарян~$^1$, Е.\,С.~Гаврилов$^2$}

\def\autkol{К.\,К.~Абгарян, Е.\,С.~Гаврилов}

\titel{\tit}{\aut}{\autkol}{\titkol}

\index{Абгарян К.\,К.}
\index{Гаврилов Е.\,С.}
\index{Abgaryan K.\,K.}
\index{Gavrilov E.\,S.}


{\renewcommand{\thefootnote}{\fnsymbol{footnote}} \footnotetext[1]
{Работа выполнена при поддержке Министерства науки и~высшего образования Российской Федерации (проект 
075-15-2020-799).}}


\renewcommand{\thefootnote}{\arabic{footnote}}
\footnotetext[1]{Федеральный исследовательский центр <<Информатика и~управление>> Российской академии наук, 
\mbox{kristal83@mail.ru}}
\footnotetext[2]{Федеральный исследовательский центр <<Информатика и~управление>> Российской академии наук; 
Московский авиационный институт (национальный исследовательский университет), \mbox{eugavrilov@gmail.com}}

%\vspace*{-6pt}
    
      
         
      
      \Abst{Создание новых композиционных материалов (КМ) с~прогнозируемыми свойствами 
      и~разработка способов их конструирования на сегодня стали одними из актуальных и~важнейших 
задач, связанных с~модернизацией промышленного производства в~нашей стране. Для их 
решения активно развиваются технологии многомасштабного компьютерного 
моделирования. Они стали связующим звеном между фундаментальной физикой (химией) 
и~инженерным материаловедением. В~работе представлен программный комплекс по 
моделированию структурных свойств КМ, поз\-во\-ля\-ющий решать ряд 
задач данного класса. Он ориентирован на высокопроизводительные вы\-чис\-ле\-ния. В~основе 
комплекса лежит оригинальная многомасштабная технология, которая позволяет оперативно 
проводить многовариантный анализ различных классов КМ 
и~проводить исследования по проектированию новых с~прогнозируемыми свойствами. 
Разработанные подходы в~сочетании с~экспериментальными данными могут быть использованы 
для лучшего понимания физических основ изменения свойств в~за\-ви\-си\-мости от структуры и,~как 
следствие, для удешевления и~ускорения поиска новых КМ
с~заданными свойствами.}
      
      \KW{многомасштабное моделирование; композиционные материалы; интеграционная 
платформа; программный комплекс; распределенная сис\-тема}

\DOI{10.14357/19922264220113}
  
%\vspace*{-3pt}


\vskip 10pt plus 9pt minus 6pt

\thispagestyle{headings}

\begin{multicols}{2}

\label{st\stat}

\section{Введение}

\vspace*{-3pt}

     Создание новых КМ с~прогнозируемыми 
свойствами и~разработка способов их конструирования на сегодня стали одними 
из актуальных и~важнейших задач по модернизации промышленного 
производства в~нашей стране. Особенно важны такие материалы в~областях, где 
соотношение между проч\-ностью и~массой конструкции определяет ее 
эф\-фек\-тив\-ность. На сегодня процессы создания КМ
непосредственно связаны с~этапом моделирования, включая применение наиболее 
эффективных методов многомасштабного компьютерного моделирования и~анализа данных. 
     
     Для решения данного класса задач разработан\linebreak программный комплекс по 
моделированию структурных свойств КМ. Он 
ориентирован на высокопроизводительные вы\-чис\-ле\-ния. В~осно\-ве комплекса 
лежит оригинальная многомасштабная \mbox{технология}, пред\-став\-лен\-ная в~[1, 2], 
которая позволяет оперативно проводить многовариантный анализ различных 
классов КМ. На базе разработанной технологии была 
создана распределенная информационная сис\-те\-ма для проведения 
многоуровневых исследований в~об\-ласти моделирования~КМ. 

Согласно разработанным подходам в~за\-ви\-си\-мости от типа 
мо\-де\-ли\-ру\-емо\-го КМ строится многомасштабная 
композиция и~ее схематическое представление. На ее основе в~программной среде 
формируется сценарий расчета структурных характеристик и~отдельных свойств 
рас\-смат\-ри\-ва\-емо\-го материала. Созданный программный комплекс позволяет 
автоматизировать уни\-фи\-ци\-ру\-емые этапы моделирования и~помогает 
сформировать на основе анализа полученных результатов более глубокое 
понимание физических процессов. Комплекс построен с~применением 
современных программных средств и~решений и~не уступает международному 
уровню на\-уч\-но-тех\-ни\-че\-ских разработок в~об\-ласти информационной 
поддержки для многомасштабного моделирования новых материалов. 
     
     Разработка такого средства информационной поддержки поз\-во\-ля\-ет 
обеспечить формирование информации для многопараметрического анализа 
структуры и~физических свойств различных классов су\-ще\-ст\-ву\-ющих 
КМ, рассмотреть большое чис\-ло вариантов 
в~на\-прав\-ле\-нии поиска новых материалов и,~таким образом, ускорить и~удешевить 
процесс подбора па\-ра\-мет\-ров получения материалов.  Ис-\linebreak\vspace*{-12pt}

\pagebreak

\noindent
пользование данного 
комплекса позволяет за ограниченное время строить гиб\-рид\-ные модели для 
обоснованного выбора КМ с~заданными свойствами для  
авиа\-ци\-он\-но-кос\-ми\-че\-ской и~других областей промышленности. 
     
     В связи с~тем что традиционные материалы (преимущественно металлы)
      не в~полной мере отвечают высоким фи\-зи\-ко-ме\-ха\-ни\-че\-ским, 
технологическим и~эксплуатационными свойствам, развитие производства 
современных надежных и~экономичных конструкций в~машиностроении 
основано на применении новых КМ. Под 
композиционными понимаются материалы, со\-сто\-ящие из двух или более 
физически различных компонент (фаз), возможные комбинации которых 
приводят к~появлению уникальных свойств, отличных от тех, которыми обладала 
каж\-дая из них отдельно. На сегодня для развития авиа\-ци\-он\-но-кос\-ми\-че\-ской 
отрасли, включая самолетостроение, вертолетостроение, ракетостроение, 
требуется постоянное увеличение доли полимерных КМ
с~набором заданных свойств. Современные летательные аппараты обладают 
слож\-ной конструкцией, со\-сто\-ящей из металлов и~неметаллических материалов. 
Применяются детали из алю\-ми\-ни\-евых и~сталь\-ных сплавов, коррозионностойких 
сталей, титановых сплавов и~полимерных КМ (стек\-ло-, 
угле-, органопластики и~др.). Для снижения веса и~продления срока службы 
летательных аппаратов при производстве деталей все шире применяют 
полимерные~КМ.
     
     Сегодня наиболее востребованные САЕ- (Computer-Aided Engineering) 
сис\-те\-мы, такие как ABAQUS ({\sf https://simulia.com}), \mbox{ANSYS} ({\sf 
https://\linebreak Ansys.com}), LMS Engineering innovation ({\sf https://\linebreak trademarks.justia.com}), 
Femap ({\sf https://www.cad-is.ru/femap}), MSC Software ({\sf 
http://www.mscsoftware.\linebreak ru}) включают в~себя базы данных со свойствами 
материалов. Для КМ мож\-но выбрать тип композита со 
стандартными свойствами (угле-, стекло-, органопластики на основе 
эпоксифенолформальдегидных, кремнийорганических смол, эпоксидные 
боропластики и~т.\,д.). Имеется возможность коррекции данных свойств 
и~внесения материала с~новыми свойствами в~базу данных. Следует также отметить 
российские разработки в~об\-ласти моделирования КМ, 
такие как пакет CAE-Fidesys ({\sf https://cae-fidesys.com}), программный пакет для 
моделирования полимерных материалов Multicomp ({\sf 
https://www.kintechlab.com/products}), Российский исследовательский 
и~ин\-же\-нер\-но-тех\-но\-ло\-ги\-че\-ский проект N1 Composites ({\sf 
http://n1composites.com}) и~др.
{\looseness=-1

}
     
     Программные комплексы позволяют задать\linebreak свойства материалов, из 
которых состоит КМ, такие как изотропность, 
ортотропность, анизотропность. Важная часть проектирования композиционных 
конструкций~--- преобразование модели,\linebreak созданной с~применением CAD 
(Computer-aided design, сис\-те\-мы автоматизированного проектирования) 
в~модель, пригодную для CAE-ана\-ли\-за (нетривиальная задача, тре\-бу\-ющая 
за\-час\-тую создания экспертной сис\-те\-мы). Следует отметить, что функционал всех 
мировых лидеров в~CAE-сег\-мен\-те схож. 
     %
     Так, функционал MSC позволяет встраивать разработанные пользователем 
модули в~программный комплекс (например, можно включить метод имитации 
процесса производства КМ).
     
     Помимо используемых ведущими CAE-сис\-те\-ма\-ми модулями существуют 
коммерческие сис\-те\-мы, позволяющие генерировать КМ на микроуровне, а~затем 
проводить чис\-лен\-ные эксперименты на макроуровне. К~таким сис\-те\-мам 
относятся модуль генерации и~моделирования механических характеристик 
КМ GeoDict ({\sf www.math2market.com}) с~различными типами КМ, 
ге\-не\-ри\-ру\-емы\-ми модулем GeoDict, и~программный комплекс COMSOL ({\sf 
www.comsol.ru}).
     
     В современных ведущих CAE-сис\-те\-мах учет мик\-ро\-струк\-ту\-ры 
КМ проводится после гомогенизации свойств материала 
или определения мак\-ро\-мас\-штаб\-ных свойств КМ. При этом, однако, теряются 
индивидуальные детали микроструктуры КМ~\cite{3-ab}. При определении макромасштабных свойств КМ обычно 
исходят из идеальных условий: оптимального формирования граничной 
поверхности, идеального распределения(отсутствия взаимодействия час\-тиц 
между собой) и~отсутствия влияния компонента на мат\-рицу.
     
     Однако результаты, которые на сегодня могут быть получены 
     с~использованием САЕ-систем для\linebreak воспроизведения характеристик известных 
структур, зачастую могут расходиться с~данными экспериментов~--- например, 
когда речь идет о~полимерных КМ с~на\-но\-вклю\-че\-ни\-ями 
(\mbox{нанотрубками}). \mbox{Известно} влияние до\-бав\-ле\-ния на\-но\-раз\-мер\-ных\linebreak час\-тиц 
наполнителя на изменение механических свойств КМ. 
В~литературе широко описано изменение коэффициента теп\-ло\-про\-вод\-ности 
полимерных\linebreak мат\-риц в~несколько раз при их наполнении 
нанотрубками, пред\-став\-ле\-ны тео\-ре\-ти\-че\-ские исследования с~аналогичными 
результатами~\cite{1-ab}. Использование CAE-сис\-те\-м не позволяет в~полной 
мере \mbox{оценить} фактор влияния на\-но\-час\-тиц на данные свойства. Кроме 
того, применение CAE-сис\-тем в~контексте многомасштабного моделирования 
затруднено жесткими ограничениями пакетных решений. В~настоящее время 
развиваются \mbox{системы} c~программным обеспечением для многомасштабного 
моделирования, такие как Computational Soft Materials (Comsoft) Workbench, 
поз\-во\-ля\-ющий моделировать КМ с~<<мягкой>> 
структурой (полимеры, полимерные композиты), программный пакет LAMMPS 
({\sf https://www.lammps.org}), ис\-поль\-зу\-емый для моделирования в~рамках 
классической молекулярной динамики на атомистическом и~мезомасштабном 
уровнях полимерных, металлических, биологических сис\-тем и~др. Каждый из 
разрабатываемых программных продуктов обладает своими достоинствами 
и~областями применения. В~связи с~большим разнообразием типов 
КМ и~все воз\-рас\-та\-ющи\-ми требованиями к~наборам 
свойств, которыми они должны обладать, пред\-став\-ля\-ет\-ся важ\-ным\linebreak создание 
программных средств, поз\-во\-ля\-ющих оперативно вы\-стра\-и\-вать сис\-тем\-ные 
решения в~об\-ласти\linebreak многомасштабного моделирования с~применением 
высокопроизводительных вычислений, поз\-во\-ля\-ющих проводить моделирование от  
атом\-но-крис\-тал\-ли\-че\-ско\-го до мак\-ро\-уров\-ня. Такие системы \mbox{позволят} 
генерировать и~выполнять в~автоматическом режиме сценарии проведения 
расчетов под конкретную задачу, включать в~вычислительную схему расчеты на 
всех необходимых мас\-штаб\-ных уровнях. Для предсказательного моделирования 
структурных свойств различных классов КМ такой 
подход поз\-во\-ля\-ет создавать вы\-чис\-ли\-тель\-ную среду, в~которой задействованы 
возможности CАE-сис\-тем для верх\-не\-уров\-не\-во\-го (мак\-ро-) моделирования, 
методы анализа экспериментальных и~аналитических данных, а также 
собственные разработки и~пакетные приложения для расчетов на атом\-но-крис\-тал\-ли\-че\-ском и~наноуровне.

\vspace*{-9pt}

\section{Многомасштабная модель для~расчета структурных 
свойств композиционных материалов}

     В работе~\cite{2-ab} представлена общая схема многомасштабной модели 
для расчета структурных характеристик КМ. Для ее 
описания используется тео\-ре\-ти\-ко-мно\-жест\-вен\-ный аппарат, изложенный 
в~\cite{1-ab, 2-ab}. На ее основе формируются схемы для расчета разных классов 
КМ: нанокомпозитов на основе полимерной мат\-ри\-цы, 
КМ с~металлической мат\-ри\-цей, полимерных 
КМ с~углеволокном и~др.

\vspace*{-9pt}
     
     \subsection*{Основные уровни моделирования}
     
     \vspace*{-2pt}
     
     
     \textbf{Квантово-механический}. Рассматриваются отдельные молекулы. 
Решается уравнение Шредингера, определяется атомарная струк\-ту\-ра молекул 
полимера и~наполнителя, строится электронная струк\-ту\-ра и~рас\-счи\-ты\-ва\-ет\-ся 
когезионная энергия, рас\-счи\-ты\-ва\-ют\-ся меж\-атом\-ные и~меж\-мо\-ле\-ку\-ляр\-ные силы, 
определяются отдельные фи\-зи\-ко-хи\-ми\-че\-ские свойства.
     
     \textbf{Молекулярно-динамический}. Изучаются ан\-самб\-ли из молекул. 
Решаются уравнения молекулярной динамики с~использованием потенциалов 
межатомного взаимодействия, рас\-счи\-ты\-ва\-ют\-ся структурные характеристики 
мат\-ри\-цы (полимерной, металлической и~др.), наполнителя (нанотрубки, 
волокна и~др.), физические свойства. 
     
     \textbf{Мезоскопический}. Рас\-смат\-ри\-ва\-ют\-ся крупнозернистые модели. 
Используется упрощенное строение молекул. Цель моделирования на 
мезоуровне~--- получение распределения час\-тиц \mbox{наполнителя} в~мат\-ри\-це 
(полимерной, металлической и~др.)\ с~по\-сле\-ду\-ющим расчетом инженерных 
свойств полученных сис\-тем. 

\begin{figure*}[b] %fig1
\vspace*{8pt}
  \begin{center}  
    \mbox{%
\epsfxsize=133.618mm
\epsfbox{abg-1.eps}
}

\end{center}
\vspace*{-2pt}
\Caption{Схема многомасштабной композиции $\mathbf{MK}_{0,1,2,3,4}^{(\mathrm{Ti/Mo})}$ 
для расчета структурных свойств МКМ}
\end{figure*}
     
     \textbf{Континуальный} (\textbf{макроскопический}). Проводится расчет 
инженерных свойств (механические свойства, теп\-ло\-про\-вод\-ность и~др.). Задачи 
решаются с~применением механики сплош\-ных сред, гид\-ро\-ди\-на\-ми\-ки, тео\-рии 
упру\-гости. Применяются метод конечных элементов, методы решения краевых 
задач для моделирования различных процессов. 
     
     Рассмотрим пример построения многомасштабной композиции для 
тес\-то\-во\-го рас\-че\-та структурных свойств металлического 
КМ (МКМ) на основе Ti (титана), армированного волокнами Mo 
(молибдена). На сегодня Ti и~титановые сплавы стали очень привлекательными 
материалами для перспективных сфер применения благодаря таким свойствам, 
как низкая плот\-ность, высокие механические свойства и~коррозионная стой\-кость. 
Использование данных материалов в~конструкциях самолетов (реактивный 
двигатель и~фюзеляж) и~применение в~автомобильной про\-мыш\-лен\-ности рас\-тут 
быст\-ры\-ми темпами. Одним из способов совершенствования\linebreak титановых сплавов 
стало их применение в~качестве мат\-ри\-цы для КМ, 
армированных волокнами, например из Mo, которые обладают очень \mbox{хорошими} 
механическими свойствами ({\sf http://\linebreak viam-works.ru/ru/articles?art\_id=1103}). 
     
     Задействуем четыре перечисленных выше масштабных уров\-ня (не считая 
нулевого). Используя обозначения из~\cite{1-ab, 2-ab}, для построения 
многомасштабной композиции 
$$
\mathbf{MK}_{0,1,2,3,4}^{(\mathrm{Mo}, \mathrm{Ti}; 
1{,}1; 1{,}2; 2{,}1; 2{,}2; 3{,}1; 4{,}1)}= \mathbf{MK}_{0,1,2,3,4}^{(\mathrm{Ti/Mo})}
$$ 

\vspace*{-3pt}

\noindent
приведем экземпляры базовых мо\-де\-лей-ком\-по\-зи\-ций: 

\vspace*{-9pt}

\noindent
     \begin{align*}
     \mathbf{El}_{01}^{\mathrm{Ti}}:& \left\{ V_{01}^{\mathrm{Ti}}, 
X_{01}^{\mathrm{Ti}}, \mathrm{MA}_{01}^{\mathrm{Ti}}\right\};\\[-3pt]
     \mathbf{El}_{01}^{\mathrm{Mo}}:& \left\{ V_{01}^{\mathrm{Mo}}, 
X_{01}^{\mathrm{Mo}}, \mathrm{MA}_{01}^{\mathrm{Mo}}\right\};\\[-3pt]
\mathbf{MC}_{11}^{\mathrm{Ti}}:& \left\{ V_{11}^{\mathrm{Ti}}, 
X_{11}^{\mathrm{Ti}}, \mathrm{MA}_{11}^{\mathrm{Ti}}\right\};\\[-3pt]
\mathbf{MC}_{11}^{\mathrm{Mo}}:& \left\{ V_{11}^{\mathrm{Mo}}, 
X_{11}^{\mathrm{Mo}}, \mathrm{MA}_{11}^{\mathrm{Mo}}\right\};
\end{align*}

\noindent
\begin{align*}
               \mathbf{MC}_{12}^{\mathrm{Ti}}:& \left\{ V_{12}^{\mathrm{Ti}}, 
X_{12}^{\mathrm{Ti}}, \mathrm{MA}_{12}^{\mathrm{Ti}}\right\};\\
     \mathbf{MC}_{12}^{\mathrm{Mo}}:& \left\{ V_{12}^{\mathrm{Mo}}, 
X_{12}^{\mathrm{Mo}}, \mathrm{MA}_{12}^{\mathrm{Mo}}\right\};\\
     \mathbf{MC}_{21}^{\mathrm{Ti}}:& \left\{ V_{21}^{\mathrm{Ti}}, 
X_{21}^{\mathrm{Ti}}, \mathrm{MA}_{21}^{\mathrm{Ti}}\right\};\\
     \mathbf{MC}_{21}^{\mathrm{Mo}}:& \left\{ V_{21}^{\mathrm{Mo}}, 
X_{21}^{\mathrm{Mo}}, \mathrm{MA}_{21}^{\mathrm{Mo}}\right\};\\
     \mathbf{MC}_{22}^{\mathrm{Ti}}:& \left\{ V_{22}^{\mathrm{Ti}}, 
X_{22}^{\mathrm{Ti}}, \mathrm{MA}_{22}^{\mathrm{Ti}}\right\};\\
     \mathbf{MC}_{22}^{\mathrm{Mo}}:& \left\{ V_{22}^{\mathrm{Mo}}, 
X_{22}^{\mathrm{Mo}}, \mathrm{MA}_{22}^{\mathrm{Mo}}\right\};\\
     \mathbf{MC}_{31}^{\mathrm{Ti}/\mathrm{Mo}}:& \left\{
     V_{31}^{\mathrm{Ti}/\mathrm{Mo}}, X_{31}^{\mathrm{Ti}/\mathrm{Mo}}, 
\mathrm{MA}_{31}^{\mathrm{Ti}/\mathrm{Mo}}\right\};\\
     \mathbf{MC}_{41}^{\mathrm{Ti}/\mathrm{Mo}}:& \left\{
     V_{41}^{\mathrm{Ti}/\mathrm{Mo}}, X_{41}^{\mathrm{Ti}/\mathrm{Mo}}, 
\mathrm{MA}_{41}^{\mathrm{Ti}/\mathrm{Mo}}\right\}.
     \end{align*}
     
     Согласно схематическому пред\-став\-ле\-нию (рис.~1) многомасштабная 
композиция $\mathbf{MK}_{0,1,2,3,4}^{(\mathrm{Ti/Mo})}$ со\-сто\-ит из связанных между 
собой экземпляров базовых моделей композиций, размещенных на 
со\-от\-вет\-ст\-ву\-ющих мас\-штаб\-ных уровнях. На наноуровне проводится  
мо\-ле\-ку\-ляр\-но-ди\-на\-ми\-че\-ское моделирование структурных свойств 
титановой мат\-ри\-цы и~молибденовых волокон. На мезоуровне рас\-смат\-ри\-ва\-ет\-ся 
распределение час\-тиц в~МКМ, на мак\-ро\-уров\-не проводится расчет механических 
свойств МКМ.

\setcounter{figure}{2}
\begin{figure*}[b] %fig3
\vspace*{-6pt}
  \begin{center}  
    \mbox{%
\epsfxsize=120.383mm
\epsfbox{abg-3.eps}
}

\end{center}
\vspace*{-9pt}
\Caption{Пример сценария с~цик\-лом}
\end{figure*}
%\pagebreak
     
\vspace*{-10pt}

\section{Программный комплекс}

\vspace*{-2pt}

   Программный комплекс, интегрированный с~расчетными пакетами 
и~модулями, размещается на высокопроизводительных многоядерных сис\-те\-мах, 
оснащенных мощными графическими процессорами. Это связано с~тем, что 
исполнение вычислительных экспериментов, а~так\-же обработка 
и~анализ результатов вы\-чис\-ли\-тель\-ных  экспериментов
 ориентированы на 
распределенные сис\-те\-мы сбора, хранения и~обработки больших данных. В~основе 
программного комплекса лежит интеграционная платформа для 
многомасштабного моделирования, которая объединяет информационные потоки 
на разных мас\-штаб\-ных уровнях. При решении конкретной задачи, такой как 
расчет структурных особенностей, механических или иных свойств 
КМ, при изучении процессов их де\-гра\-да\-ции 
и~разрушения и~др.\ выделяются конкретные уров\-ни моделирования, которые 
необходимо задействовать. Первоначально строится многомасштабная 
композиция~--- информационный аналог\linebreak мно\-го\-мас\-штаб\-ной  
фи\-зи\-ко-ма\-те\-ма\-ти\-че\-ской модели. Для программной реализации на базе 
интеграционной платформы~\cite{4-ab} из име\-ющих\-ся программных модулей 
формируется вы\-чис\-ли\-тель\-ный \mbox{комплекс}~\cite{5-ab, 6-ab}.
   
   Перечислим пользовательские роли в~интеграционной плат\-фор\-ме 
мно\-го\-мас\-штаб\-но\-го моделирования:
   \begin{itemize}
\item разработчик вычислительных модулей реализует расчетный модуль или 
осуществляет конфигурирование при\-клад\-но\-го па\-кета;\\[-15pt]
\item системный разработчик создает веб-сер\-ви\-сы для вы\-чис\-ли\-тель\-но\-го модуля 
и~интегрирует его в~плат\-форму;\\[-15pt]
\item разработчик расчетных сценариев создает сценарии в~среде моделирования;\\[-15pt]
\item ученый-исследователь прикладной об\-ласти запускает расчетные сценарии 
с~различными па\-ра\-мет\-ра\-ми и~анализирует ре\-зуль\-таты.
\end{itemize}
    
    Как отмечалось в~\cite{5-ab, 6-ab}, программный комплекс предназначен для 
создания и~исполнения сценариев многомасштабных расчетов для моделирования 
структурных свойств композитных материалов.
    
    Сценарий~--- программная реализация мно\-го\-мас\-штаб\-ной композиции~--- 
пред\-став\-ля\-ет собой алгоритм последовательного выполнения расчетов отдельных 
физических характеристик материалов, входящих в~со\-став композита, 
посредством интегрированных с~программным комплексом вы\-чис\-ли\-тель\-ных 
модулей. Среда моделирования сценариев поз\-во\-ля\-ет создавать или 
модифицировать сценарии, учитывая особенности конкретного 
КМ и~тре\-бу\-емые свойства.



 
    
    Среда исполнения сценариев дает возможность осуществить его запуск 
    с~заданными входными па\-ра\-мет\-ра\-ми, отслеживать его выполнение в~целом и~по 
со\-став\-ным задачам, про\-смат\-ри\-вать входные и~выходные данные (результаты 
расчетов). Интеграционная роль среды исполнения заключается\linebreak в~формировании 
входных данных для вычислительных модулей в~со\-от\-вет\-ст\-ву\-ющем формате 
и~единицах измерения, отслеживании работы модулей,\linebreak получении конечного 
результата расчета и~преобразовании его в~формат и~единицы измерения, 
до\-ступ\-ные для других модулей сценария. Таким образом, среда исполнения 
обеспечивает соответствие потока исполнения вы\-чис\-ли\-тель\-ных модулей 
заданному алгоритму в~сценарии и~це\-лост\-ность потока данных между блоками 
сценария. Кроме того,\linebreak среда исполнения предостав\-ля\-ет общие словари для\linebreak 
согласования вход\-ных-вы\-ход\-ных данных вы\-чис\-ли\-тель\-ных экспериментов, 
такие как справочник\linebreak химических элементов и~их свойств, химических формул 
веществ, ис\-поль\-зу\-емых в~композитных материалах, типы крис\-тал\-ли\-че\-ских 
сис\-тем, типы атомных радиусов, пространственные группы.

\setcounter{figure}{3}
\begin{figure*}[b] %fig4
\vspace*{-9pt}
  \begin{center}  
    \mbox{%
\epsfxsize=163mm
\epsfbox{abg-4.eps}
}

\end{center}
\vspace*{-9pt}
\Caption{Сценарий для расчета МКМ}
\end{figure*}

\vspace*{-10pt}
   
    \subsection*{Алгоритм программы}
    
    \vspace*{-2pt}
    
    Алгоритм исполнения сценария основан на стандарте BPMN~2.0 и~со\-сто\-ит из 
сле\-ду\-ющих ключевых элементов (рис.~2).

{ \begin{center}  %fig2
 \vspace*{6pt}
    \mbox{%
\epsfxsize=70.82mm
\epsfbox{abg-2.eps}
}

\vspace*{6pt}

\noindent
{{\figurename~2}\ \ \small{
Пример простого сценария
}}
\end{center}
}

%\vspace*{6pt} 

\noindent
\begin{description}
\item[Э1.]  Точка начала выполнения сценария. В~свойствах этого элемента 
указывается список кодов физических величин, которые пользователь дол\-жен 
будет ввес\-ти перед запуском сценария.

\item[Э2.] Сплошная стрелка определяет строгую по\-сле\-до\-ва\-тель\-ность 
выполнения шагов сценария.

 \begin{figure*}[b] %fig5
  \vspace*{1pt}
  \begin{center}  
    \mbox{%
\epsfxsize=131mm %.834mm
\epsfbox{abg-5.eps}
}

\end{center}
\vspace*{-9pt}
  \Caption{Сценарий для расчета механических свойств полимерного нанокомпозита}
  \end{figure*}

\item[Э3.] Вычислительная задача пред\-став\-ля\-ет\-ся в~BPMN как <<внеш\-няя 
сервисная задача>> (External Service Task). В~поле topic в~настройках 
задачи вводится название очереди задач со\-от\-вет\-ст\-ву\-юще\-го 
вы\-чис\-ли\-тель\-но\-го модуля. Например, для  
кван\-то\-во-ме\-ха\-ни\-че\-ско\-го расчета на пакете VASP вводится 
<<vasp\_topic>>. Список до\-ступ\-ных вы\-чис\-ли\-тель\-ных модулей 
с~названиями очередей хранится в~базе данных в~таб\-ли\-це <<Module>>.\\[-15pt]

\item[Э4.] Точка завершения выполнения сценария. Если в~сценарии существует 
ветвление, точек завершения может быть несколько.



    \item[Э5.] Шаг сценария, в~рамках которого выполняется скрипт, заданный 
пользователем. В~па\-ра\-мет\-рах задачи может быть указан язык скрип\-та и~сам 
скрипт. Доступны языки Groovy и~Jython (реализация языка Python на Java). 
Скрип\-ты могут использоваться для изменения входных и~выходных па\-ра\-мет\-ров, 
небольших вы\-чис\-ле\-ний на основе текущих до\-ступ\-ных данных сценария. 
В~примере на рис.~3 в~цик\-ле определяется список векторов 
крис\-тал\-ли\-че\-ской решетки, по которым будет проводиться кван\-то\-во-ме\-ха\-ни\-че\-ский 
рас\-чет деформированной решетки.\\[-19.5pt]

\begin{figure*}[b] %fig6
\vspace*{1pt}
  \begin{center}  
    \mbox{%
\epsfxsize=163mm
\epsfbox{abg-6.eps}
}

\end{center}
\vspace*{-9pt}
\Caption{Сценарий для расчета КМ с~полимерной мат\-ри\-цей 
и~наполнителем из углеволокна}
\end{figure*}
    
    \item[Э6.] Подпроцесс сценария <<цикл с~параллельным запуском>>  
(Parallel multi-instance) позволяет параллельно запустить выполнение час\-ти 
сцена- %\linebreak\vspace*{-12pt}

\columnbreak

\noindent
рия несколько раз. В~свойствах подпроцесса требуется указать коллекцию 
(Collection), по элементам которой будет проводиться ите\-ри\-ро\-ва\-ние, и~название 
переменной цикла (Element Variable). Весь элемент считается выполненным, когда 
все параллельно выполняющиеся подпроцессы завершат свою работу. Например, 
если требуется запустить кван\-то\-во-ме\-ха\-ни\-че\-ский расчет для некоторого 
множества деформированных решеток (для определения в~дальнейшем констант 
упру\-гости), предварительно в~скрип\-те перед цик\-лом формируется список 
деформированных векторов решетки и~сохраняется в~переменную процесса. 
Далее для каж\-дой деформации параллельно вызывается\linebreak\vspace*{-12pt}

\pagebreak

\noindent  
кван\-то\-во-ме\-ха\-ни\-че\-ский модуль VASP для расчета энергии и~объема 
решетки. Получившаяся таб\-ли\-ца с~данными может использоваться для расчета 
констант элас\-тич\-ности, модуля упру\-гости и~других свойств материала.
\end{description}

\vspace*{-9pt}
  
  \subsection*{Примеры тестовых сценариев для~расчета~структурных~характеристик 
  и~отдельных~свойств различных классов 
композиционных материалов}


     
     \textbf{Пример~1.} Тестовый сценарий для расчета структурных свойств 
КМ с~металлической мат\-ри\-цей (рис.~4).

\smallskip

     
     \textbf{Пример~2.} Тестовый сценарий для расчета механических свойств 
полимерного нанокомпозита (полифениленсульфид с~углеродными нанотрубками). 
На сле\-ду\-ющем этапе проекта планируется расширить сценарий для 
оценки влияния процентного содержания углеродных нанотрубок на изменение 
коэффициента теп\-ло\-про\-вод\-ности полимерного нанокомпозита (рис.~5).
  
 
     
     \textbf{Пример~3.} Тестовый сценарий для расчета механических свойств 
КМ с~полимерной мат\-ри\-цей (эпоксидная смола) и~углеволокном
(рис.~6).

\vspace*{-6pt}

\section{Выводы}

\vspace*{-2pt}

     В работе представлен программный комплекс для расчета структурных 
характеристик КМ с~тре\-бу\-емы\-ми свойствами. В~его 
основе лежит интеграционная плат\-фор\-ма для многомасштабного моделирования, 
которая объединяет информационные потоки на разных мас\-штаб\-ных уровнях. На 
ее основе формируются схемы для рас\-че\-та структурных характеристик разных 
клас\-сов КМ: нанокомпозитов на основе полимерной 
мат\-ри\-цы, КМ с~металлической мат\-ри\-цей, полимерных 
КМ с~углеволокном и~другие. Разработанные подходы 
поз\-во\-ля\-ют моделировать свойства КМ (механические, 
теп\-ло\-вые и~др.), а~так\-же многомасштабные процессы, связанные с~усталостным 
разрушением при случайных по\-вреж\-де\-ни\-ях в~ходе эксплуатации, и~другие 
динамические процессы. Программный комплекс со\-сто\-ит из программных 
модулей и~базируется на типовых сер\-ви\-сах вы\-чис\-ли\-тель\-ных модулей, общей 
интеграционной оболочки и~модулей сценариев. Про\-грам\-мные решения 
сертифицированы. В~дальнейшем планируется раз\-ра\-бо\-тать 
полнофункциональную про\-грам\-мную сис\-те\-му с~целью решения различных 
классов обратных задач в~об\-ласти наук о~материалах. Разработанные подходы 
в~сочетании с~экспериментальными данными могут быть использованы для 
лучшего понимания физических основ изменения свойств в~за\-ви\-си\-мости от 
струк\-ту\-ры и,~как след\-ст\-вие, для уде\-шев\-ле\-ния и~уско\-ре\-ния поиска новых 
КМ с~заданными свойствами.

\vspace*{-6pt}
   
{\small\frenchspacing
 {%\baselineskip=10.8pt
 %\addcontentsline{toc}{section}{References}
 \begin{thebibliography}{9}
 
 \vspace*{-2pt}
   
   \bibitem{1-ab}
   \Au{Абгарян К.\,К.} Многомасштабное моделирование в~задачах структурного 
материаловедения.~--- М.: МАКСПресс, 2017. 284~с.
\bibitem{2-ab}
\Au{Абгарян~К.\,К.} Информационная технология по\-стро\-ения многомасштабных моделей 
в~задачах вы\-чис\-ли\-тель\-но\-го материаловедения~// Сис\-те\-мы высокой до\-ступ\-ности, 2018. Т.~14. 
№\,2. С.~9--15.
\bibitem{3-ab}
\Au{Naffakh M., D$\acute{\!\mbox{{\!\ptb{\i}}}}$ez-Pascuala~A.\,M., Marcoa~C., Ellisa~G.} Morphology and thermal properties of novel poly (phenylene sulfide) 
hybrid nanocomposites based on single-walled carbon nanotubes and 8 inorganic fullerene-like WS~2 
nanoparticles~// J.~Mater. Chem., 2012. Vol.~22. No.\,4. P.~1418--1425.
\bibitem{4-ab}
\Au{Абгарян К.\,К., Гаврилов~Е.\,С.} Распределенная информационная сис\-те\-ма для расчета 
структурных свойств композиционных материалов~// Информатика и~её применения, 2021. 
Т.~15. Вып.~4. С.~50--58. doi: 10.14357/ 19922264210407.
\bibitem{5-ab}
\Au{Гаврилов Е.\,С.} Интегрированный интерфейс к~модулю сплош\-но\-сред\-но\-го взаимодействия. 
Свидетельство о~регистрации программ для ЭВМ №\,2021681058, 2021.
\bibitem{6-ab}
\Au{Гаврилов Е.\,С.} Программные средства для хранения и~обмена данными в~задачах 
моделирования композитных материалов. Свидетельство о~регистрации программ для ЭВМ 
№\,2021681762, 2021.

\end{thebibliography}

 }
 }

\end{multicols}

\vspace*{-8pt}

\hfill{\small\textit{Поступила в~редакцию 22.01.22}}

\vspace*{8pt}

%\pagebreak

%\newpage

%\vspace*{-28pt}

\hrule

\vspace*{2pt}

\hrule

%\vspace*{-2pt}

\def\tit{SOFTWARE PACKAGE FOR MULTISCALE MODELING OF~STRUCTURAL PROPERTIES 
OF~COMPOSITE MATERIALS}


\def\titkol{Software package for multiscale modeling of~structural properties 
of~composite materials}


\def\aut{K.\,K.~Abgaryan$^1$ and~E.\,S.~Gavrilov$^{1,2}$}

\def\autkol{K.\,K.~Abgaryan and~E.\,S.~Gavrilov}

\titel{\tit}{\aut}{\autkol}{\titkol}

\vspace*{-18pt}


\noindent
$^1$Federal Research Center ``Computer Science and Control'' of the Russian Academy of Sciences, 
44-2~Vavilov\linebreak
$\hphantom{^1}$Str., Moscow 119333, Russian Federation

\noindent
$^2$Moscow Aviation Institute (National Research University), 4~Volokolamskoe Shosse, Moscow 
125080, Russian\linebreak
$\hphantom{^1}$Federation

\def\leftfootline{\small{\textbf{\thepage}
\hfill INFORMATIKA I EE PRIMENENIYA~--- INFORMATICS AND
APPLICATIONS\ \ \ 2022\ \ \ volume~16\ \ \ issue\ 1}
}%
 \def\rightfootline{\small{INFORMATIKA I EE PRIMENENIYA~---
INFORMATICS AND APPLICATIONS\ \ \ 2022\ \ \ volume~16\ \ \ issue\ 1
\hfill \textbf{\thepage}}}

\vspace*{3pt} 
      
      
  
\Abste{Today, creation of new composite materials and methods of their construction with predictable 
properties is one of the urgent and most important tasks connected with modernization of 
industrial production in our country. For their solution, technologies of multiscale computer modeling 
are actively developed. They have become a~link between fundamental physics (chemistry) and 
engineering materials science. The paper presents a~software package for modeling structural 
properties of composite materials which allows solving a~number of problems of this class. It is 
focused on high-performance computations. The complex is based on an original multiscale 
technology which allows one to promptly conduct multivariate analysis of different classes of 
composite materials and conduct research on designing the new ones with predictable properties. The 
developed approaches in combination with experimental data can be used for a~better understanding of 
the physical foundations of the change of properties depending on the structure and, as a~consequence, 
for cheaper and faster search of new composite materials with predetermined properties.}

\KWE{multiscale modeling; composite materials; integration platform; software package; distributed 
system}



\DOI{10.14357/19922264220113}

\vspace*{-16pt}

\Ack
\noindent
The research was supported by the Ministry of Science and Higher Education of the Russian 
Federation (project No.\,075-15-2020-799).




%\vspace*{4pt}

  \begin{multicols}{2}

\renewcommand{\bibname}{\protect\rmfamily References}
%\renewcommand{\bibname}{\large\protect\rm References}

{\small\frenchspacing
 {%\baselineskip=10.8pt
 \addcontentsline{toc}{section}{References}
 \begin{thebibliography}{9}
\bibitem{1-ab-1}
\Aue{Abgaryan, K.\,K.} 2017. \textit{Mnogomasshtabnoe modelirovanie v~zadachakh strukturnogo 
materialovedeniya} [Multiscale modeling for structural materials science applications]. Moscow: 
MAKS Press. 284~p.

\vspace*{-2pt}

\bibitem{2-ab-1}
\Aue{Abgaryan, K.\,K.} 2018. In\-for\-ma\-tsi\-on\-naya tekh\-no\-lo\-giya po\-stro\-eniya mno\-go\-mas\-shtab\-nykh 
mo\-de\-ley v~za\-da\-chakh vy\-chis\-li\-tel'\-no\-go ma\-te\-ri\-a\-lo\-ve\-de\-niya 
[Information technology is the construction 
of multi-scale models in problems of computational materials science]. \textit{Sistemy vysokoy 
dostupnosti} [Highly Available Systems] 14(2):9--15.
\bibitem{3-ab-1}
\Aue{Naffakh, M., A.\,M.~D$\acute{\mbox{{\!\ptb{\i}}}}$ez-Pascuala, C.~Marcoa, and G.~Ellisa.} 
2012. Morphology and thermal properties of novel poly (phenylene sulfide) hybrid nanocomposites 
based on single-walled carbon nanotubes and~8~inorganic fullerene-like WS~2 nanoparticles. 
\textit{J.~Mater. Chem.}  
22(4):1418--1425.
{\looseness=1

}
\bibitem{4-ab-1}
  \Aue{Abgaryan, K.\,K., and E.\,S.~Gavrilov.} 2021. 
  Ras\-pre\-de\-len\-naya in\-for\-ma\-tsi\-on\-naya sis\-te\-ma   dlya 
ras\-che\-ta struk\-tur\-nykh svoystv kom\-po\-zi\-tsi\-on\-nykh ma\-te\-ri\-alov 
[Distributed information system for 
calculating the structural properties of composite materials]. \textit{Informatika i~ee Primeneniya~--- 
Inform. Appl.} 15(4):50--58. doi: 10.14357/19922264210407.
\bibitem{5-ab-1}
  \Aue{Gavrilov, E.\,S.} 2021. In\-teg\-ri\-ro\-van\-nyy in\-ter\-feys k~mo\-du\-lyu 
  splosh\-no\-sred\-no\-go 
vza\-imo\-dey\-stviya [Integrated interface to the solid-medium interaction module]. Certificate on official 
registration of the computer program No.\,2021681058.
\bibitem{6-ab-1}
  \Aue{Gavrilov, E.\,S.} 2021. Pro\-gram\-mnye sred\-st\-va dlya khra\-ne\-niya 
  i~ob\-me\-na dan\-ny\-mi  v~za\-da\-chakh mo\-de\-li\-ro\-va\-niya kom\-po\-zit\-nykh ma\-te\-ri\-a\-lov 
  [Software tools for data persistence and data flow in 
composite materials modeling tasks]. Certificate on official registration of the computer program 
No.\,2021681762.
\end{thebibliography}

 }
 }

\end{multicols}

\vspace*{-6pt}

\hfill{\small\textit{Received January 22, 2022}}


\Contr

\noindent
\textbf{Abgaryan Karine K.} (b.\ 1963)~--- Doctor of Science in physics and mathematics, principal 
scientist, A.\,A.~Dorodnicyn Computing Center, Federal Research Center ``Computer Science and 
Control'' of the Russian Academy of Sciences, 40~Vavilov Str., Moscow 119333, Russian Federation; 
head of department, Moscow Aviation Institute (National Research University), 4~Volokolamskoe 
Shosse, Moscow 125080, Russian Federation; \mbox{kristal83@mail.ru}

\vspace*{3pt}

\noindent
\textbf{Gavrilov Evgeny S.} (b.\ 1982)~--- scientist, A.\,A.~Dorodnicyn Computing Center, Federal 
Research Center ``Computer Science and Control'' of the Russian Academy of Sciences, 40~Vavilov 
Str., Moscow 119333, Russian Federation; senior lecturer, Moscow Aviation Institute (National 
Research University), 4~Volokolamskoe Shosse, Moscow 125080, Russian Federation; 
\mbox{eugavrilov@gmail.com}
       

\label{end\stat}

\renewcommand{\bibname}{\protect\rm Литература} 