\def\stat{kovalev}

\def\tit{АЛГЕБРАИЧЕСКАЯ СПЕЦИФИКАЦИЯ ГРАФОВЫХ ВЫЧИСЛИТЕЛЬНЫХ СТРУКТУР}

\def\titkol{Алгебраическая спецификация графовых вычислительных структур}

\def\aut{С.\,П.~Ковалёв$^1$}

\def\autkol{С.\,П.~Ковалёв}

\titel{\tit}{\aut}{\autkol}{\titkol}

\index{Ковалёв С.\,П.}
\index{Kovalyov S.\,P.}


%{\renewcommand{\thefootnote}{\fnsymbol{footnote}} \footnotetext[1]
%{Работа выполнена при поддержке Министерства науки и~высшего образования Российской Федерации (проект 
%075-15-2020-799).}}


\renewcommand{\thefootnote}{\arabic{footnote}}
\footnotetext[1]{Институт проблем управления им.\ В.\,А.~Трапезникова Российской академии наук, 
\mbox{kovalyov@sibnet.ru}}

%\vspace*{-6pt}

    

\Abst{Рассматриваются проблемы со\-став\-ле\-ния ал\-геб\-ра\-и\-че\-ских спецификаций 
многокомпонентных вы\-чис\-ли\-тель\-ных структур, пред\-став\-ля\-емых графами потоков данных. 
Кратко показано, как при помощи теории категорий алгебраические средства спецификации 
программ развивались от многосортных ал\-гебр через ко\-ал\-геб\-ры к~объ\-ем\-лю\-щей конструкции 
диалгебры, способной описывать интерактивные вычислительные узлы. В~качестве ее 
обобщения предложена новая тео\-ре\-ти\-ко-ка\-те\-гор\-ная конструкция 
графалгебры, позволяющая комбинировать диалгебры в~произвольные ориентированные 
мультиграфы (диаграммы), реб\-ра которых отвечают вы\-чис\-ли\-тель\-ным операциям в~узлах, 
а~вершины описывают пе\-ре\-да\-ва\-емые меж\-ду узлами данные. Приведены примеры 
графалгебраических спецификаций нейронных сетей и~разнообразных многопроцессорных 
вы\-чис\-ли\-тель\-ных сис\-тем. Описан способ по\-стро\-ения категорий графалгебр посредством 
универсальных конструкций. Для вы\-чис\-ли\-тель\-ной структуры вида сис\-те\-мы сис\-тем, 
со\-сто\-ящей из графовых структур, предложены методы иерархического по\-стро\-ения 
алгебраической спецификации из спецификаций со\-став\-ля\-ющих.}

\KW{алгебраическая спецификация; графовая вычислительная структура; система систем; 
теория категорий; диалгебра; графалгебра; декартов квадрат}

\DOI{10.14357/19922264220101}
  
\vspace*{4pt}


\vskip 10pt plus 9pt minus 6pt

\thispagestyle{headings}

\begin{multicols}{2}

\label{st\stat}

\section{Введение}

\vspace*{-4pt}

  Вычислительные структуры, пред\-став\-ля\-емые графами потоков данных, 
(снова) набирают по\-пу\-ляр\-ность: к~ним относятся искусственные нейронные 
сети, облачные среды оркестровки веб-сер\-ви\-сов, аппаратные решения от 
логических мат\-риц до многопроцессорных систем и~др. Закономерно 
оживает и~интерес к~формальным математическим методам спецификации 
и~верификации таких структур. Традиционно здесь применяются методы на 
основе ис\-чис\-ле\-ний процессов, берущие начало от классического исчисления 
CCS (Calculus of Communicating Systems) и~аналогичных~[1]. Они позволяют 
специфицировать многокомпонентные вы\-чис\-ли\-тель\-ные структуры 
выражениями, в~которых переменные отвечают процессам и~действиям, 
а~операции описывают их организацию и~взаимодействие. Можно формально 
проверять такие выражения, например, на предмет отсутствия тупиков 
(deadlocks).
  
  Однако спецификации в~исчислении процессов могут казаться недостаточно 
наглядными, поскольку не содержат средств для графического изображения 
струк\-тур. Это неудобство преодолевается \mbox{путем} привлечения тео\-рии 
категорий~--- раздела высшей ал\-геб\-ры, который <<начинается с~наблюдения, 
что многие свойства математических сис\-тем мож\-но представить просто 
и~единообразно посредством диаграмм>>~[2]. Вводятся симметричные 
моноидальные категории, в~которых основные виды\linebreak операций над процессами, 
а~именно: последовательное и~параллельное исполнение~--- пред\-став\-ля\-ют\-ся 
композицией и~тензорным произведением\linebreak соответственно. Тем самым 
процессы представляются морфизмами, а~пе\-ре\-да\-ва\-емые меж\-ду ними  
данные~--- объектами таких категорий. Графовые вы\-чис\-ли\-тель\-ные структуры 
мож\-но изображать струнными диаграммами в~таких категориях~[3].
  %
  При этом остается нерешенной существенная проб\-ле\-ма исчислений 
и~категорий процессов, со\-сто\-ящая в~сла\-бости средств специфицирования типов 
данных и~содержания вы\-чис\-ли\-тель\-ных операций над данными. Описывается 
только поведение графовой вы\-чис\-ли\-тель\-ной структуры с~точки зрения 
внеш\-не\-го наблюдателя, способного увидеть комбинирование процессов 
и~передачу данных, но не <<заглянуть \mbox{внутрь}>> тех и~других. 


Предпринимаются попытки дополнить ис\-чис\-ле\-ния процессов конструкциями 
обработки данных, заимствованными из императивных языков 
программирования~[4], однако по\-лу\-ча\-ющи\-еся формализмы выглядят чересчур 
громоздкими. 

Концептуально более стройный подход пред\-лагается в~рамках 
современных формализмов проверки на моделях (model checking)~\cite{5-kov}. Они 
поз\-воляют детально описать на формальном языке и~со\-сто\-яния узлов как 
наборы значений переменных, и~изменения значений под действием как 
вычислительных процедур в~узлах, так и~актов передачи данных между узлами. 
Такую спецификацию можно автоматизированным способом проверить на 
соответствие требованиям к~отсутствию тупиков и~отказов, к~порядку 
и~дли\-тель\-ности выполнения операций и~т.\,д. Однако такая спецификация час\-то 
получается чересчур конкретной~--- близ\-кой  
к~про\-грам\-мно-ап\-па\-рат\-ной реализации вы\-чис\-ли\-тель\-ной структуры. 
Со\-став\-ле\-ние такой спецификации может оказаться значительно более 
трудоемким, чем программирование, и~не при\-вес\-ти к~высокоуровневому 
концептуальному пред\-став\-ле\-нию графовой вы\-чис\-ли\-тель\-ной структуры, 
до\-пус\-ка\-юще\-му широкий спектр способов реализации.
  
  В настоящей работе предлагается подойти к~решению вышеуказанной 
проб\-ле\-мы с~противоположной стороны: расширить традиционную 
ал\-геб\-ра\-и\-че\-скую спецификацию средствами описания \mbox{графовых} структур. 
Напомним, что (многосортные) ал\-геб\-ры служат спецификациями абстрактных 
типов данных (АТД), каж\-дый из которых вклю\-ча\-ет множество значений, набор 
операций над ними (в~том чис\-ле с~учас\-ти\-ем значений других АТД) и~ак\-сио\-мы, 
ха\-рак\-те\-ри\-зу\-ющие операции (см., например,~\cite{6-kov}). В~такой спецификации мож\-но 
проверять те или иные свойства АТД путем индукции по структуре термов 
алгебры, а~при помощи аппарата тео\-рии категорий была введена двойственная 
конструкция ко\-ал\-геб\-ры, позволяющая абстрактно специфицировать 
и~верифицировать внеш\-нее поведение интерактивных сис\-тем (реакцию на 
входные данные)~\cite{7-kov}. Естественным сле\-ду\-ющим шагом\linebreak выглядит объединение 
алгебры и~коалгебры в~объ\-ем\-лю\-щую конструкцию, способную описывать 
интерактивные вы\-чис\-ли\-тель\-ные узлы одновременно <<из\-вне>> 
и~<<из\-нут\-ри>>. Такая объ\-ем\-лю\-щая конструкция известна под названием 
ди\-ал\-геб\-ры (dialgebra)~\cite{8-kov}. В~на\-сто\-ящей работе делается еще один шаг: 
ди\-ал\-геб\-ры комбинируются в~графы произвольных многокомпонентных 
вы\-чис\-ли\-тель\-ных структур, по\-рож\-дая новый класс категорных конструкций, 
названных \textit{графалгебрами}. Исследуются способы по\-стро\-ения 
и~некоторые свойства категорий графалгебр.

\vspace*{-6pt}

\section{Алгебраические спецификации с~точки зрения теории 
категорий}

\vspace*{-3pt}

  Пусть $C$~--- произвольная категория, $T, S : C\hm\to C$~--- пара 
эндофункторов. Ди\-ал\-геб\-рой называется произвольная пара $(A, a)$, со\-сто\-ящая 
из \mbox{$C$-объ}\-ек\-та~$A$ (\textit{носителя}) и~$C$-мор\-физ\-ма $a : TA\hm\to 
SA$ (\textit{структуры})~\cite{8-kov}. Гомоморфизм ди\-ал\-геб\-ры $(A, a)$ 
в~$(A^\prime, a^\prime)$~--- это любой $C$-мор\-физм $f : A\hm\to A^\prime$, 
удовле\-тво\-ря\-ющий условию сохранения структуры $Sf\circ a\hm= a^\prime \circ 
Tf$.  Ясно, что при фиксированных $T$, $S$ из всех ди\-ал\-гебр и~всех их 
гомоморфизмов получается категория и~из нее в~$C$ действует 
<<за\-бы\-ва\-ющий>> функ\-тор носителя $(A,a)\mapsto A$.
  
  При $S=1_C$ получается классическая конструкция категории $T$-ал\-гебр 
$T$-\textbf{Alg}~\cite[определение~5.37]{9-kov}. В~уз\-ком смыс\-ле алгебры 
сигнатуры первого порядка~$\Sigma$ строятся над категорией всех\linebreak множеств 
и~отображений \textbf{Set} с~функтором $T_\Sigma : \mathbf{Set} \hm\to 
\mathbf{Set} : A\mapsto \coprod_{\varphi\in \Sigma} A^{n(\varphi)}$, где 
$n(\varphi)$~--- мест-\linebreak ность (чис\-ло аргументных мест) функционального\linebreak символа 
$\varphi\hm\in \Sigma$. В~этом случае отображение $a : T_\Sigma A\hm\to A$ 
собирается из компонентов~$a_\varphi : A^{n(\varphi)} \hm\to A$, $\varphi\hm\in 
\Sigma$, задающих опе\-ра\-ции-ин\-тер\-пре\-та\-ции сим\-волов, а~гомоморфизмы 
сохраняют операции в~обыч\-ном смыс\-ле. Многосортные \mbox{алгебры}, ис\-поль\-зу\-емые 
как классические алгебраические \mbox{спецификации} АТД, строятся аналогичным 
способом над категорией $\mathbf{Set}^N$, где $N$~--- множество имен сортов.
  
  В свою очередь, при $T\hm= 1_C$ получается конструкция  
$S$-ко\-ал\-геб\-ры. Обычно категория коалгебр над~$C$ вводится как 
двойственная к~категории ал\-гебр над~$C^{\mathrm{op}}$, т.\,е.\ путем обращения 
на\-прав\-ле\-ния стрел\-ки у~структуры~\cite{7-kov}. Коалгебры над \textbf{Set} 
применяются\linebreak в~качестве спецификаций поведения интерактивных программ 
с~точки зрения внеш\-не\-го наблюдателя, например в~форме помеченных сис\-тем 
переходов: (недерминированная) сис\-те\-ма переходов \mbox{задается} коалгеброй 
с~функтором $S_Q : A \mapsto (PA)^Q$ для некоторого фиксированного 
множества~$Q$. Здесь~$A$ интерпретируется как множество со\-сто\-яний 
сис\-те\-мы, а $A$~--- как множество наблюдаемых действий (в~част\-ности, 
данных, по\-сту\-па\-ющих на вход сис\-те\-мы извне), так что структура  
$S_Q$-ко\-ал\-геб\-ры $a : A\hm\to S_QA$ задает для каж\-до\-го со\-сто\-яния 
$s\hm\in A$ и~действия $q\hm\in Q$ множество со\-сто\-яний $a(s)q \hm \subseteq 
A$~--- воз\-мож\-ных результатов перехода из~$s$ под действием~$q$.
  
  Можно ожидать, что ди\-ал\-геб\-ра общего вида позволяет описывать 
  и~вычислительные, и~интерактивные аспекты в~одной спецификации, 
а~в~на\-сто\-ящей работе ди\-ал\-геб\-ры комбинируются в~\mbox{произвольные} 
(ориентированные муль\-ти-) графы в~целях специфицирования графовых 
вы\-чис\-ли\-тель\-ных структур. Например, многослойная нейросеть прямого 
рас\-про\-стра\-не\-ния описывается не\-пус\-той конечной цепочкой ди\-ал\-гебр~--- 
графом множеств и~отоб\-ра\-же\-ний вида
  $A^{k_0}\to A^{k_1}\to \cdots \to A^{k_l},$
где $A$~--- множество значений распознаваемого сигнала (например, $A\hm= 
\mathbb{R}$); $l\hm>1$~--- глубина сети; $k_i\hm>0$~--- чис\-ло нейронов в~$i$-м 
слое, $i\hm = 0, \ldots, l \hm- 1$. Функция обработки сигнала в~каж\-дом слое $f_i 
: A^{k_i}\hm\to  A^{k_{i+1}}$ со\-сто\-ит из отоб\-ра\-же\-ний-ком\-по\-нен\-тов, например:

\vspace*{-6pt}

\noindent
\begin{multline*}
     \hspace*{-5.69pt}f_{is} : \mathbb{R}^{k_i}\to  \mathbb{R} : \left(x_1, \ldots, x_{k_i}\right) 
\mapsto \!\sum\limits_{j=1}^{k_i} w_{ijs}\sigma_{ij}\!\left(x_j + b_{ij}\right),\\
      s = 1, \ldots, k_{i+1},
\end{multline*}

\vspace*{-6pt}

\noindent
где $\sigma_{ij} : \mathbb{R}\hm\to \mathbb{R}$~--- функ\-ция активации $j$-го 
нейрона $i$-го слоя; $w_{ijs}$ и~$b_{ij}$~--- веса и~смещения, под\-би\-ра\-емые при 
обуче\-нии ней\-ро\-се\-ти. У~нейросети с~об\-рат\-ной связью появляются 
дополнительные обрат\-ные стрел\-ки, на\-прав\-лен\-ные от по\-сле\-ду\-ющих слоев 
к~предыду\-щим. Разрабатываются и~нейросети, архитектура которых задается 
более слож\-ны\-ми графами~\cite{10-kov}.

  Другой класс примеров со\-став\-ля\-ют многопроцессорные вы\-чис\-ли\-тель\-ные 
сис\-те\-мы, которые описываются разнообразными графами вы\-чис\-ли\-тель\-ных 
узлов, связанных потоками данных. \mbox{Рас\-смат\-ри\-ва\-емое} алгебраическое 
пред\-став\-ле\-ние таких сис\-тем, как и~нейросетей, приводит к~(характерным для  
тео\-ре\-ти\-ко-ка\-те\-гор\-ной семантики вообще) производным графам, 
у~которых ребра отвечают вы\-чис\-ли\-тель\-ным операциям в~узлах, а~вершины 
описывают пе\-ре\-да\-ва\-емые меж\-ду узлами данные. Например, параллельный 
(в~уз\-ком смыс\-ле этого термина) вы\-чис\-ли\-тель пред\-став\-ля\-ет\-ся графом, 
со\-сто\-ящим из не менее двух параллельных цепочек, име\-ющих одно и~то же 
начало (входные данные) и~один и~тот же конец (выходные данные). Граф 
прос\-тей\-ше\-го параллельного вы\-чис\-ли\-те\-ля имеет вид $\bullet \rightrightarrows 
\bullet$. Небезынтересны также графы, имеющие всего одну вершину 
(и~произвольное чис\-ло ре\-бер-пе\-тель): они представляют распределенные 
вы\-чис\-ли\-тель\-ные сис\-те\-мы с~раз\-де\-ля\-емой памятью, которой отвечает вершина. 
Например, если пометить вершину множеством $\coprod_{n=0}^\infty A^n$ 
(<<звезда Клини>> исходного множества данных~$A$), то получится 
спецификация сис\-те\-мы класса <<пространство кортежей>> (tuple space), 
родоначальницей которого была Linda~\cite{11-kov}.
  
  Диалгебрам присуще одно ограничение, сужа\-ющее область их 
применимости: их гомоморфизмы стро\-ят\-ся из того же <<материала>>, что 
и~операции, поскольку сигнатура со\-сто\-ит из эндофункторов. В~приложениях 
в~информатике встречаются противоречия этому ограничению, например, 
когда рас\-смат\-ри\-ва\-ют\-ся час\-тич\-ные алгебры~--- алгебраические спецификации 
с~час\-тич\-но определенными функциями в~качестве сигнатурных операций, 
гомоморфизмы которых, однако, долж\-ны быть определены  
пол\-ностью~\cite{12-kov}. Для учета таких ситуаций вводятся сигнатурные 
функторы, об\-ласть которых отличается от ко\-об\-ласти.

%\vspace*{-12pt}
  
\section{Категория графалгебр}

  Следуя вышеизложенным соображениям, обобщим тео\-ре\-ти\-ко-ка\-те\-гор\-ную 
конструкцию ал\-геб\-ры так, чтобы структура пред\-став\-ля\-лась диаграммой 
произвольной формы в~некоторой категории, а~носители и~гомоморфизмы 
заимствовались из другой категории. Назовем такое обоб\-ще\-ние графалгеброй 
(graphalgebra).
  
  \smallskip
  
  \noindent
  \textbf{Определение~1.}\ Пусть $I$~--- произвольная малая категория 
(схема, форма), $F_i : D \hm\to  C$, $i \hm\in \mathrm{Ob}\,I$,~--- произвольное 
семейство функторов. \textit{Графалгеброй} сигнатуры $\langle I, F\rangle$ 
(кратко~--- $\langle I, F\rangle$-\textit{ал\-геб\-рой}) называется произвольная 
пара $(A, \Delta)$, со\-сто\-ящая из \mbox{$D$-объ}\-екта~$A$ (\textit{носителя}) 
и~диа\-грам\-мы $\Delta : I \hm \to C$ (\textit{структуры}), такая что $\Delta i \hm= 
F_iA$, $i\hm\in \mathrm{Ob}\,I$. \textit{Гомоморфизмом} $\langle I,  
F\rangle$-\textit{ал\-гебр} $(A, \Delta)$ в~$(A^\prime, \Delta^\prime)$ называется 
$D$-мор\-физм $f : A \hm\to A^\prime$, удовле\-тво\-ря\-ющий сле\-ду\-юще\-му \textit{условию 
сохранения структуры}: семейство $C$-мор\-физ\-мов $F_if : \Delta i \hm\to 
\Delta^\prime i$, $i \hm\in \mathrm{Ob}\,I$, образует естественное 
преобразование~$\Delta$ в~$\Delta^\prime$ (т.\,е.\ для любых вершин 
формы~$i, k \hm\in\mathrm{Ob}\, I$ и~стрелки $h : i \hm\to k$ выполняется 
условие $F_kf \circ \Delta h \hm= \Delta^\prime h \circ F_if$). \textit{Областью}
  $\langle I, F\rangle$-ал\-геб\-ры называется категория~$D$, 
а~\textit{ко\-об\-ластью}~--- категория~$C$. \textit{Эндографалгеброй} называется 
$\langle I, F\rangle$-ал\-геб\-ра, об\-ласть которой совпадает 
с~ко\-об\-ластью.~$\square$
  
  Легко видеть, что для любой фиксированной сигнатуры $\langle I, F\rangle$ 
со\-во\-куп\-ность всех $\langle I, F\rangle$-ал\-гебр и~всех их гомоморфизмов 
образует категорию, которую будем обозначать через ${\sf A}_IF$. Покажем, 
что ее можно получить при помощи универсальных конструкций 
в~<<категории всех категорий>> \textbf{CAT}, а~именно: произведения, 
декартова квад\-ра\-та и~экспоненты. Напомним, в~част\-ности, что категория $C^{\vert I\vert}$ 
(где через $\vert I\vert$ обозначается дис\-крет\-ная подкатегория в~$I$, со\-сто\-ящая 
из всех $I$-объ\-ек\-тов и~всех их тож\-дест\-вен\-ных морфизмов) изоморфна 
произведению~$\mathrm{Ob}\,I$ экземпляров категории~$C$ и~снаб\-же\-на 
канонической проекцией на $j$-й экземпляр $\mathrm{Pr}_j : C^{\vert I\vert} \hm\to C$ 
для любого $j\hm\in \mathrm{Ob}\,I$, причем существует единственный 
функтор $\langle F_i\rangle_{i\in \mathrm{Ob}\,I} : D\hm\to C^{\vert I\vert}$, 
удовле\-тво\-ря\-ющий условию $\mathrm{Pr}_j\circ \langle F_i\rangle_{i\in 
\mathrm{Ob}\,I}\hm= F_j$  для всех $j\hm \in \mathrm{Ob}\,I$. Имеет \mbox{место}
  
  \smallskip
  
  \noindent
  \textbf{Теорема~1.}\ \textit{Категория $\langle I, F\rangle$-ал\-гебр 
изоморфна вершине (объекту, находящемуся в~левом верх\-нем углу) 
сле\-ду\-юще\-го декартова квад\-ра\-та в}~\textbf{CAT}:
  %\fig1
  \vspace*{1pt}
  \begin{center}  
    \mbox{%
\epsfxsize=28.493mm
\epsfbox{kov-1.eps}
}

\end{center}
%\vspace*{-9pt}
  
  \noindent
  Д\,о\,к\,а\,з\,а\,т\,е\,л\,ь\,с\,т\,в\,о\,.\ \ Проверяется непосредственно по 
правилам вы\-чис\-ле\-ния пределов и~экспонент в~\textbf{CAT}.~$\square$
  
  Штриховые стрелки декартова квад\-ра\-та из тео\-ре\-мы~1 задают два 
канонических <<за\-бы\-ва\-ющих>> функтора, определенных на любой категории 
$\langle I, F\rangle$-ал\-гебр. Первый функтор задается левой вертикальной 
стрелкой и~обозначается через ${\sf U}_I^F : {\sf A}_IF \hm\to D : (A, \Delta) 
\mapsto A$. По аналогии с~ал\-геб\-ра\-ми\linebreak будем называть его \textit{функтором 
носителя}. Он унивалентен (faithful, т.\,е.\ инъективен на любом множестве 
гомоморфизмов $\langle I, F\rangle$-ал\-гебр, име\-ющих одну и~ту же об\-ласть 
и~одну и~ту же кообласть). \mbox{Второй} функтор, изображенный верх\-ней 
горизонтальной стрел\-кой, будем обозначать через ${\sf S}_I^F : {\sf A}_I F 
\hm\to  C^I : (A, \Delta) \mapsto \Delta$ и~называть \textit{функтором 
структуры}.
  
  Рассмотрим сигнатуру вида $\langle\boldsymbol{2}, F\rangle$, где 
$\boldsymbol{2}$~--- категория, со\-сто\-ящая из двух объектов~0 и~1 и~одного 
нетривиального морфизма $0\hm\to 1$. Эн\-до\-граф\-ал\-геб\-ра такой сигнатуры~--- 
это не что иное, как ди\-ал\-геб\-ра. Из граф\-ал\-гебр такой сигнатуры получается еще 
одна хорошо известная конструкция~--- категория запятой (comma category)~[2, \S~II.6], 
а~именно: для произвольной пары функторов $Q_0 : D_0 \hm\to C 
\hm\leftarrow D_1 : Q_1$ категория запятой $Q_0/Q_1$~--- это\linebreak ${\sf A}_2(Q_0\mathrm{Pr}_0, 
Q_1\mathrm{Pr}_1 : D_0\times  D_1 \hm\to  C$). Известен способ по\-стро\-ения категории 
запятой посредством декартова квад\-ра\-та в~\textbf{CAT}~--- част\-ный случай 
тео\-ре\-мы~1~\cite{13-kov}. Графал\-геб\-ра\-иче\-ская трак\-тов\-ка \mbox{позволяет} 
естественным образом обобщить эту конструкцию на произвольную 
форму~$I$~--- рас\-смот\-реть для произвольного семейства функторов $Q_i : D_i 
\hm\to  C$, $i\hm\in \mathrm{Ob}\,I$, категорию графалгебр сигнатуры $\langle 
I, (Q_i\mathrm{Pr}_i : \prod_{j \in \mathrm{Ob}\,I} D_j \hm\to C,\ i \hm\in \mathrm{Ob}\,I)\rangle$. 
Эта категория была впервые введена и~исследована автором на\-сто\-ящей статьи 
под названием \textit{категории мультизапятой} (multicomma)~\cite{14-kov}. 
Для мультизапятой сформулированы и~доказаны аналоги тео\-рем~1 и~2. 
Мультизапятая имеет прикладное значение как способ формально описать 
в~архитектурном пред\-став\-ле\-нии~$C$ все\-воз\-мож\-ные сис\-те\-мы, со\-би\-ра\-емые со\-глас\-но 
структурной схеме~$I$ из разнородных компонентов, <<каталоги>> которых 
описываются категориями~$D_i$.
  
\section{Иерархическое конструирование графалгебр}

  Специфицирование графовых вычислительных структур затрудняется тем 
обстоятельством, что такие структуры ред\-ко создаются из элементарных 
не\-спе\-ци\-фи\-ци\-ру\-емых единиц. Напротив, на практике часто возникает 
потребность в~иерархической структуре вида сис\-те\-мы сис\-тем (System of 
Systems,\linebreak SoS), единицами которой вы\-сту\-па\-ют цельные графовые структуры, 
требующие полноценной спецификации. Например, актуальные задачи 
искусственного интеллекта требуют собирать из \mbox{нейросетей} целые 
ан\-самб\-ли~\cite{15-kov}. Другим примером служат вы\-чис\-ли\-тель\-ные клас\-те\-ры 
и~облака, со\-сто\-ящие из сложных мик\-ро\-про\-цес\-сор\-ных узлов, каждый из 
которых содержит несколько вза\-имо\-дей\-ст\-ву\-ющих вы\-чис\-ли\-тель\-ных устройств 
и~блоков памяти с~разными функциями.
  
  Покажем, как строить алгебраические спецификации иерархических 
графовых вычислительных структур SoS иерархически из ал\-геб\-ра\-и\-че\-ских 
спецификаций со\-став\-ля\-ющих. Формально структура SoS описывается графом,
 в~вершинах которой находятся структуры, в~свою очередь опи\-сы\-ва\-емые 
подходящими графами (вообще говоря, различными). Такой граф задается 
диаграммой вида $\Xi : I\hm\to \mathbf{Cat}$. Ее можно отрисовать~--- 
преобразовать в~малую категорию, по\-рож\-ден\-ную заменой каж\-дой вершины 
$i\hm\in \mathrm{Ob}\,I$ графом~$\Xi i$, а~каж\-дой стрелки $h : i \hm\to  
l$~--- со\-во\-куп\-ностью стрелок, по одной для каж\-дой вершины~$k$ 
графа~$\Xi i$, на\-прав\-лен\-ной из~$k$ в~вершину $\Xi h(k)$ графа~$\Xi 
l$, с~наложением подходящих условий естест\-вен\-ности. Вместе  
с~\textbf{Cat}-диа\-грам\-ма\-ми можно отрисовывать и~их морфизмы, так что 
имеется функтор отрисовки $\mathbf{K}_1 : \mathbf{D}(\mathbf{Cat})\hm\to 
\mathbf{Cat}$ (это частный случай общей конструкции отрисовки, входящей 
в~монаду диаграмм $\mathbf{D} : \mathbf{CAT}\hm\to  
\mathbf{CAT}$~\cite{16-kov}). Опишем процедуру построения категории 
графалгебр, сигнатура которых имеет форму отрисовки некоторой  
\textbf{Cat}-диа\-грам\-мы~$\Xi$, из категорий графалгебр сигнатур формы 
$\Xi i$ (при условии что все эти графалгебры имеют одну и~ту же область 
и~одну и~ту же ко\-об\-ласть). Для этого понадобится конструкция расслоенного 
произведения (multiple pullback)~--- предела диаграммы, состоящей из 
произвольного семейства морфизмов с~общей ко\-об\-ластью 
(sink)~\cite[упражнение~11L]{9-kov}. Напомним, что для произвольного 
множества~$S$ расслоенным произведением объектов~$X_s$, $s\hm\in S$, 
снабженных морфизмами $f_s : X_s \hm\to  Y$, называется универсальная 
конструкция, состоящая из объекта ${\vphantom{\prod_{s\in S}}}_{Y}\!\!\prod_{s\in S}X_s$
%${}_Y\prod_{s\in S} X_s$ 
и~семейства 
мор\-физ\-мов-про\-ек\-ций $_Y\mathbf{pr}_s$, дей\-ст\-ву\-ющих из него в~$X_s$, таких что значение 
композиции $f_s\circ {}_Y \mathbf{pr}_s$ не зависит от~$s$:

%\fig2
  \vspace*{1pt}
  
  \begin{center}  
    \mbox{%
\epsfxsize=50.166mm
\epsfbox{kov-2.eps}
}

\end{center}
%\vspace*{-9pt}

\begin{figure*} %fig5
  \vspace*{1pt}
  \begin{center}  
    \mbox{%
\epsfxsize=107.786mm
\epsfbox{kov-5.eps}
}


\vspace*{5pt}
{\small К доказательству теоремы~2: диаграмма в \textbf{CAT}, состоящая из трех декартовых квадратов}

\end{center}
\vspace*{-6pt}
\end{figure*}
 
Расслоенным произведением пустой совокупности объектов служит~$Y$. При 
$\vert S\vert \hm=1$ рас\-сло\-ен\-ное произведение изоморфно своему 
единственному сомножителю. При $\vert S\vert \hm=2$ расслоенное 
произведение превращается в~обычный декартов квад\-рат. Как легко проверить 
непосредственно, рас\-сло\-ен\-ное произведение мож\-но вы\-чис\-лить и~для 
произвольного множества~$S$ посредством декартова квад\-ра\-та сле\-ду\-юще\-го 
вида:
  %\fig3
  \vspace*{1pt}
  
  \begin{center}  
    \mbox{%
\epsfxsize=38.23mm
\epsfbox{kov-3.eps}
}

\end{center}
%\vspace*{-9pt}
 
  Имеет место
  
  \smallskip
  
  \noindent
  \textbf{Теорема~2.}\ \textit{Пусть задана произвольная малая категория~$I$, 
диаграмма $\Xi : I\hm\to \mathbf{Cat}$ и~семейство функторов $G^{(i)}_k : 
D \hm\to  C$, $k\hm\in \mathrm{Ob}\,\Xi i$, $i\hm\in \mathrm{Ob}\,I$. Пусть 
%${\vphantom{\prod_{s\in S}}}_{Y}\!\!\prod_{s\in S}X_s$
${\vphantom{\prod_{s\in S}}}_D\!\prod_{i\in \vert I\vert} {\sf A}_{\Xi i} G^{(i)}$~--- расслоенное 
произведение (в}~\textbf{CAT}) \textit{категорий $\langle \Xi i, G^{(i)}\rangle$-ал\-гебр, 
$i\hm\in \mathrm{Ob}\,I$, снаб\-жен\-ных функторами носителя. Категория 
$\langle \mathbf{K}_1\Xi,  
G\rangle$-ал\-гебр изоморфна вершине сле\-ду\-юще\-го декартова квад\-ра\-та 
в}~\textbf{CAT}:
 %\fig4
  \vspace*{1pt}
  
  \begin{center}  
    \mbox{%
\epsfxsize=63.014mm
\epsfbox{kov-4.eps}
}

\end{center}
%\vspace*{-9pt}
 
  
  
  \noindent
  Д\,о\,к\,а\,з\,а\,т\,е\,л\,ь\,с\,т\,в\,о\,.\ \ Рассмотрим сле\-ду\-ющую диаграмму 
в~\textbf{CAT}, со\-сто\-ящую из трех декартовых квад\-ра\-тов (см. рисунок).
 В этой диаграмме правый нижний квад\-рат действительно является декартовым, 
поскольку он получен путем <<почленного>> перемножения декартовых 
квад\-ра\-тов из тео\-ре\-мы~1, от\-ве\-ча\-ющих категориям $\langle \Xi i,\ 
G^{(i)}\rangle$-ал\-гебр, $i\hm\in \mathrm{Ob}\,I$.\linebreak Левый нижний декартов 
квад\-рат задает рас\-сло\-ен\-ное произведение из условия тео\-ре\-мы. Верх\-ний 
прямоугольник~--- это декартов квад\-рат, изображенный в~условии тео\-ре\-мы. 
Ввиду композиционных свойств декартовых 
квад\-ра\-тов~\cite[предложение 11.10(1)]{9-kov}, на\-руж\-ный контур диаграммы 
{пред\-став\-ля\-ет} собой декартов квад\-рат. Следовательно, ввиду тео\-ре\-мы~1 
и~соотношения $\vert \mathbf{K}_1\Xi\vert \hm\cong \coprod_{i\in \mathrm{Ob}\,I} \vert \Xi i\vert$ 
(отрисовка не до\-бав\-ля\-ет и~не удаляет вершины), объект, находящийся в~левом 
верх\-нем углу диаграммы, изоморфен категории $\langle \mathbf{K}_1\Xi, 
G\rangle$-ал\-гебр.~$\square$
  
  Теорема~2 допускает ряд частных случаев. Например, если все вершины 
диаграммы~$\Xi$ помечены одной и~той же категорией~$K$ и~все стрелки 
помечены тождественными функторами~$1_K$, то $\mathbf{K}_1\Xi 
\hm\cong I\times K$ и~категория $\langle I\times K, G\rangle$-ал\-гебр 
изоморфна категории $\langle I, (\mathrm{Pr}_i S_K^{\hat{G}} : {\sf A}_K \hat{G} \hm\to 
C^K,\ i\hm\in \mathrm{Ob}\,I)\rangle$-ал\-гебр, где~$\hat{G}$~--- семейство 
функторов $\langle G^{(i)}_k\rangle_{i\in \mathrm{Ob}\,I} : D \hm\to C^{\vert 
I\vert}$, $k\hm \in \mathrm{Ob}\, K$; 
$\mathrm{Pr}_i: (C^{\vert I\vert})^K \hm \cong (C^K)^{\vert I\vert}\hm\to C^K$~--- проекция. 
Проиллюстрируем этот результат примером из области спецификации 
многопроцессорных вы\-чис\-ли\-тель\-ных сис\-тем. Рассмотрим сис\-те\-му,  
со\-сто\-ящую из процессора~$\boldsymbol{p}$ 
и~сопроцессора~$\boldsymbol{c}$. Взаимодействие процессора 
с~сопроцессором сводится к~двум операциям: конвертация входных данных 
при выдаче процессором вы\-чис\-ли\-тель\-но\-го задания сопроцессору, которую 
обозначим стрелкой $d_c : \boldsymbol{p}\hm\to \boldsymbol{c}$, и~обратная 
конвертация $d_p : \boldsymbol{c}\hm\to \boldsymbol{p}$ при получении 
процессором результата вы\-чис\-ле\-ния задания от сопроцессора. Подчиненное 
положение сопроцессора по отношению к~процессору проявляется в~том, что 
данные процессора включаются в~данные сопроцессора инвариантно 
относительно конвертации, т.\,е.\ имеет место соотношение $d_p\circ d_c\hm= 
1_p$. Получается следующая схема вы\-чис\-ли\-тель\-ной сис\-те\-мы, обозначаемая 
далее через~$I_{\boldsymbol{pc}}$:

\vspace*{-2pt}

  \begin{center}  
    \mbox{%
\epsfxsize=35.684mm
\epsfbox{kov-6.eps}
}

\end{center}
%\vspace*{-9pt}

\vspace*{-2pt}
 
  
  Предположим, что процессор и~сопроцессор\linebreak специфицируются как АТД: 
спецификации процессора берутся из категории $T_{\boldsymbol{p}}\mbox{-}\mathbf{Alg}$, 
а~сопроцессора~--- из $T_{\boldsymbol{c}}\mbox{-}\mathbf{Alg}$ для подходящих 
сигнатурных\linebreak функторов $T_{\boldsymbol{p}}, T_{\boldsymbol{c}} : \mathbf{Set}\hm\to \mathbf{Set}$. Если 
операции конвертации данных при взаимодействии никак не согласуются 
с~собственными вы\-чис\-ли\-тель\-ны\-ми операциями процессора и~сопроцессора, то 
спецификацией сис\-те\-мы может служить <<двухсортная>> графалгебра 
сигнатуры $\langle I_{\boldsymbol{pc}}, (U_i \mathrm{Pr}_i : T_{\boldsymbol{p}}\mbox{-}\mathbf{Alg} 
\hm\times T_{\boldsymbol{c}}\mbox{-}\mathbf{Alg}\hm\to 
\mathbf{Set},\ i\hm\in \{\boldsymbol{p}, \boldsymbol{c}\})\rangle$, где $U_i : 
T_i\mbox{-}\mathbf{Alg} \hm\to \mathbf{Set}$~--- функтор носителя. Однако 
чаще \mbox{требуется}, чтобы конвертация вела себя естественно относительно 
вы\-чис\-лений, и~далее будет продемонстрировано, как удовле\-тво\-рить это 
требование путем специфицирования сис\-те\-мы в~виде иерархической SoS, 
а~именно: у~рас\-смат\-ри\-ва\-емой SoS спецификации обоих узлов имеют 
сигнатуру формы~$\boldsymbol{2}$, поэтому спецификация сис\-те\-мы имеет 
сигнатуру\linebreak формы $I_{\boldsymbol{pc}}\times \boldsymbol{2}$, к~которой можно применить 
вышеприведенный результат. Согласно ему, сначала спецификации процессора
 и~сопроцессора комбинируются в~двухсортную ал\-геб\-ру с~сигнатурным\linebreak 
функтором $T_{\boldsymbol{p}}\times T_{\boldsymbol{c}}: \mathbf{Set}\times \mathbf{Set} \hm\to 
\mathbf{Set}\times \mathbf{Set}$, а~затем в~качестве спецификации сис\-те\-мы 
выбирается граф\-ал\-геб\-ра сигнатуры $\langle I_{\boldsymbol{pc}}, (\mathrm{Pr}_i S_{\boldsymbol{pc}} : 
(T_{\boldsymbol{p}}\hm\times 
T_{\boldsymbol{c}})\mbox{-}\mathbf{Alg} \hm\to \mathbf{Set}^{\boldsymbol{2}},\ i\hm\in 
\{\boldsymbol{p}, \boldsymbol{c}\} )\rangle$, где

\vspace*{-2pt}

\noindent
  \begin{align*}
  S_{\boldsymbol{pc}} &: \left( T_{\boldsymbol{p}} \times T_{\boldsymbol{c}}\right)\mbox{-}\mathbf{Alg} 
  \to \left( \mathbf{Set} 
\times \mathbf{Set}\right)^{\mathbf{2}}\\
&  \hspace*{-13.41975pt}:   \left(\left( A_{\boldsymbol{p}}, A_{\boldsymbol{c}}\right), \left(
   a_{\boldsymbol{p}} : T_{\boldsymbol{p}} A_{\boldsymbol{p}}\to 
   A_{\boldsymbol{p}},\ a_{\boldsymbol{c}} : T_{\boldsymbol{c}}A_{\boldsymbol{c}}\to 
A_{\boldsymbol{c}}\right)\right)\mapsto{}\\
&\hspace*{55mm}{}\mapsto \left(a_{\boldsymbol{p}}, a_{\boldsymbol{c}}\right)\,.
  \end{align*}
  
  \vspace*{-2pt}
  
  \noindent
Искомая согласованность конвертации с~вы\-чис\-ле\-ни\-ями обеспечивается тем, 
что морфизмы, опи\-сы\-ва\-ющие операции конвертации, заимствуются в~этой 
спецификации не из \textbf{Set}, а~из категории стрелок 
\textbf{Set}$^{\boldsymbol{2}}$, т.\,е.\ образуют коммутативные квад\-ра\-ты 
с~вы\-чис\-ли\-тель\-ны\-ми операциями. При помощи тео\-ре\-мы~2 этот пример можно 
обобщить и~на случай, когда процессор и~сопроцессор специфицируются 
произвольными графалгебрами разных форм.

\vspace*{-6pt}

\section{Заключение}
  
  Настоящая работа представляет первый шаг в~исследовании новой 
категорной конструкции графалгебры: введены самые основные понятия 
и~показаны некоторые взаимосвязи между ними. В~ходе дальнейшей работы 
предполагается, с~одной стороны, рас\-смот\-реть воз\-мож\-ность и~це\-ле\-со\-об\-раз\-ность 
обобщения на графалгебры различных приемов алгебраической спецификации: 
конструирования канонической семантики, верификации путем структурной 
индукции, па\-ра\-мет\-ри\-за\-ции, выделения многообразий и~др. С~другой стороны, 
интерес пред\-став\-ля\-ет разработка приемов, специфичных для графовых 
структур~--- в~первую очередь это\linebreak естественные переходы между 
графалгебрами различных сигнатур, в~том чис\-ле вклю\-ча\-ющие трансформации 
форм. Актуальна так\-же разработка программных инструментов, 
под\-дер\-жи\-ва\-ющих \mbox{со\-став\-ле\-ние} и~анализ графалгебраических спецификаций.

{\small\frenchspacing
 {%\baselineskip=10.8pt
 %\addcontentsline{toc}{section}{References}
 \begin{thebibliography}{99}
\bibitem{1-kov}
\Au{Cleaveland R., Smolka~S.\,A.} Strategic directions in concurrency research~// ACM Comput. 
Surv., 1996. Vol.~28. No.\,4. P.~607--625.
\bibitem{2-kov}
\Au{Маклейн С.} Категории для ра\-бо\-та\-юще\-го математика~/ Пер. с~англ.~--- М.: Физматлит, 
2004. 352~с. (\Au{Mac Lane~S.} Categories for the working mathematician.~--- New York, NY, 
USA: Springer, 1978. 317~p.).
\bibitem{3-kov}
\Au{Abramsky S., Gay~S.\,J., Nagarajan~R.} Interaction categories and foundations of typed 
concurrent programming~// Deductive program design~/ Ed. M.~Broy.~--- 
NATO ASI ser.~F.~--- Springer-Verlag, 1996. P.~35--113.
\bibitem{4-kov}
\Au{Bergstra J.\,A., Middelburg~C.\,A.} Using Hoare logic in a~process algebra setting~// 
arXiv.org, 2019. 24~p. \mbox{arXiv}:1906.04491 [cs.LO].
\bibitem{5-kov}
\Au{Stewart R., Berthomieu~B., Garcia~P., Ibrahim~I., Michaelson~G., Wallace~A.} Verifying 
parallel dataflow transformations with model checking and its application to FPGAs~// J.~Syst. 
Architect., 2019. Vol.~101. Art. 101657.
\bibitem{6-kov}
\Au{Ковалёв С.\,П.} Формальный подход к~разработке программных сис\-тем.~--- 
Новосибирск: НГУ, 2004. 180~с.
\bibitem{7-kov}
\Au{Jacobs B., Rutten~J.} A~tutorial on (co)algebras and (co)induction~// EATCS Bulletin, 1997. 
Vol.~62. P.~222--259.
\bibitem{8-kov}
\Au{Poll E., Zwanenburg~J.} From algebras and coalgebras to dialgebras~// Electronic Notes  
Theoretical Computer Science, 2001. Vol.~44. Iss.~1. P.~289--307.
\bibitem{9-kov}
\Au{Ad$\acute{\mbox{a}}$mek J., Herrlich~H., Strecker~G.\,E.} Abstract and concrete 
categories.~--- New York, NY, USA: John Wiley, 1990. 507~p.
\bibitem{10-kov}
\Au{Looks M., Herreshoff~M., Hutchins~D., Norvig~P.} Deep learning with dynamic computation 
graphs~// arXiv.org, 2017. arXiv:1702.02181 [cs.NE].
\bibitem{11-kov}
\Au{Carriero N.\,J., Gelernter~D., Mattson~T.\,G., Sherman~A.\,H.} The Linda alternative to 
message passing systems~// Parallel Comput., 1994. Vol.~20. No.\,4. P.~633--655.
\bibitem{12-kov}
\Au{Burmeister P.} Partial algebras~--- an introductory survey~// Algebras and orders~/ Eds. 
I.\,G.~Rosenberg, G.~Sabidussi.~--- NATO ASI ser.~C.~--- Kluwer Academic Publs., 1993. Vol.~389. P.~1--70.
\bibitem{13-kov}
Comma category. nLab: 2019. {\sf https://ncatlab.org/nlab/\linebreak show/comma+category}.
\bibitem{14-kov}
\Au{Ковалёв С.\,П.} Методы теории категорий в~цифровом проектировании гетерогенных 
киберфизических сис\-тем~// Информатика и~её применения, 2021. Т.~15. Вып.~1. С.~23--29.
\bibitem{15-kov}
\Au{Fawaz H.\,I., Forestier~G., Weber~J., Idoumghar~L., Muller~P.} Deep neural network 
ensembles for time series classification~// Joint Conference (International) on Neural Networks 
Proceedings.~--- Piscataway, NJ, USA: IEEE, 2019. Art.\ 8852316. 6~p. doi: 
10.1109/IJCNN.2019.8852316.
\bibitem{16-kov}
\Au{Ковалёв С.\,П.} Теория категорий как математическая прагматика  
мо\-дель\-но-ори\-ен\-ти\-ро\-ван\-ной сис\-тем\-ной инженерии~// Информатика и~её 
применения, 2018. Т.~12. Вып.~1. С.~95--104.
\end{thebibliography}

 }
 }

\end{multicols}

\vspace*{-6pt}

\hfill{\small\textit{Поступила в~редакцию 03.09.20}}

\vspace*{8pt}

%\pagebreak

%\newpage

%\vspace*{-28pt}

\hrule

\vspace*{2pt}

\hrule

%\vspace*{-2pt}

\def\tit{ALGEBRAIC SPECIFICATION OF~GRAPH~COMPUTATIONAL~STRUCTURES}


\def\titkol{Algebraic specification of~graph computational structures}


\def\aut{S.\,P.~Kovalyov}

\def\autkol{S.\,P.~Kovalyov}

\titel{\tit}{\aut}{\autkol}{\titkol}

\vspace*{-9pt}


\noindent
   V.\,A.~Trapeznikov Institute of Control Sciences of the Russian Academy of Sciences, 
65~Profsoyuznaya Str., Moscow 117997, Russian Federation

\def\leftfootline{\small{\textbf{\thepage}
\hfill INFORMATIKA I EE PRIMENENIYA~--- INFORMATICS AND
APPLICATIONS\ \ \ 2022\ \ \ volume~16\ \ \ issue\ 1}
}%
 \def\rightfootline{\small{INFORMATIKA I EE PRIMENENIYA~---
INFORMATICS AND APPLICATIONS\ \ \ 2022\ \ \ volume~16\ \ \ issue\ 1
\hfill \textbf{\thepage}}}

\vspace*{6pt} 


   
   
   \Abste{Problems of composing algebraic specifications for computational structures 
represented by data flow graphs are considered. The evolution of algebraic program specification 
tools is briefly outlined, from many-sorted algebra via coalgebra to a~category-theoretical 
construction of dialgebra capable of describing interactive computing nodes. As a~next step, 
a~novel category-theoretical construction called graphalgebra is proposed which allows combining 
dialgebras into arbitrary directed multigraphs whose edges represent computational operations at 
nodes and whose vertices describe data exchanged between nodes. Examples of graphalgebraic 
specifications for neural networks and multiprocessor computational systems are given. The method 
of building categories of graphalgebras via universal constructions is described. For 
a~computational structure of the system of systems kind consisting of graph structures, 
methods of hierarchical construction of an algebraic specification from the specifications of 
components are proposed.}
   
   \KWE{algebraic specification; graph computational structure; system of systems; category 
theory; dialebra; graphalgebra; pullback}
   
\DOI{10.14357/19922264220101}

%\vspace*{-16pt}

%\Ack
%\noindent




\vspace*{12pt}

  \begin{multicols}{2}

\renewcommand{\bibname}{\protect\rmfamily References}
%\renewcommand{\bibname}{\large\protect\rm References}

{\small\frenchspacing
 {%\baselineskip=10.8pt
 \addcontentsline{toc}{section}{References}
 \begin{thebibliography}{99}
   
\bibitem{1-kov-1}
   \Aue{Cleaveland, R., and S.\,A.~Smolka.} 1996. Strategic directions in concurrency research. 
\textit{ACM Comput. Surv.} 28(4):607--625.
\bibitem{2-kov-1}
   \Aue{Mac Lane, S.} 1978. \textit{Categories for the working mathematician.} New York, NY: 
Springer. 317~p.
\bibitem{3-kov-1}
   \Aue{Abramsky, S., S.\,J.~Gay, and R.~Nagarajan.} 1995. Interaction categories and 
foundations of typed concurrent programming. \textit{Deductive program design}.
Ed.\ M.~Broy. NATO ASI ser.~F. Springer-Verlag. 35--113.
\bibitem{4-kov-1}
   \Aue{Bergstra, J.\,A., and C.\,A.~Middelburg.} 2019. Using Hoare logic in a process algebra 
setting. \textit{arXiv.org}. 24~p. Available at: {\sf https://arxiv.org/abs/1906.04491} (accessed 
December~17, 2021).
\bibitem{5-kov-1}
   \Aue{Stewart, R., B.~Berthomieu, P.~Garcia, I.~Ibrahim, G.~Michaelson, and A.~Wallace.} 
2019. Verifying parallel dataflow transformations with model checking and its application to 
FPGAs. \textit{J.~Syst. Architect.} 101:101657.

\columnbreak

\bibitem{6-kov-1}
   \Aue{Kovalyov, S.\,P.} 2004. \textit{Formal'nyy podkhod k~razrabotke programmnykh sistem} 
[Formal approach to software systems development]. Novosibirsk: NGU. 180~p.
\bibitem{7-kov-1}
   \Aue{Jacobs, B., and J.~Rutten.} 1997. A~tutorial on (co)algebras and (co)induction. 
\textit{EATCS Bulletin} 62:222--259.
\bibitem{8-kov-1}
   \Aue{Poll, E., and J.~Zwanenburg.} 2001. From algebras and coalgebras to dialgebras. 
\textit{Electronic Notes Theoretical Computer Science} 44(1):289--307.
\bibitem{9-kov-1}
   \Aue{Ad$\acute{\mbox{a}}$mek, J., H.~Herrlich, and G.\,E.~Strecker}. 1990. \textit{Abstract 
and concrete categories}. New York, NY: John Wiley. 507~p.
\bibitem{10-kov-1}
   \Aue{Looks, M., M.~Herreshoff, D.~Hutchins, and P.~Norvig.} 2017. Deep learning with 
dynamic computation graphs. \textit{arXiv.org}. Available at: {\sf 
https://arxiv.org/abs/1702.02181} (accessed December~17, 2021).
\bibitem{11-kov-1}
   \Aue{Carriero, N.\,J., D.~Gelernter, T.\,G.~Mattson, and A.\,H.~Sherman}. 1994. The Linda 
alternative to message passing systems. \textit{Parallel Comput.} 20(4):633--655.
\bibitem{12-kov-1}
   \Aue{Burmeister, P.} 1993. Partial algebras~--- an introductory survey. \textit{Algebras and 
orders}. Eds. I.\,G.~Rosenberg and G.~Sabidussi. NATO ASI ser.~C. 
Kluwer Academic Publs. 389:1--70.
\bibitem{13-kov-1}
   Comma category. Available at: {\sf https://ncatlab.org/nlab/\linebreak show/comma+category} (accessed 
December~17, 2021).
\bibitem{14-kov-1}
   \Aue{Kovalyov, S.\,P.} 2021. Metody teorii kategoriy v~tsifrovom proektirovanii 
geterogennykh kiberfizicheskikh sistem [Methods of category theory in digital design of 
heterogeneous cyber-physical systems]. \textit{Informatika i~ee Primeneniya~--- Inform. Appl.} 
15(1):23--29.
\bibitem{15-kov-1}
   \Aue{Fawaz, H.\,I., G.~Forestier, J.~Weber, L.~Idoumghar, and P.~Muller.} 2019. Deep neural 
network ensembles for time series classification. \textit{Joint Conference (International) on Neural 
Networks Proceedings}. Piscataway, NJ: IEEE. 8852316. 6~p. doi: 10.1109/IJCNN.2019.8852316.
\bibitem{16-kov-1}
   \Aue{Kovalyov, S.\,P.} 2018. Teoriya kategoriy kak ma\-te\-ma\-ti\-che\-skaya pragmatika  
model'no-oriyentirovannoy sistemnoy inzhenerii [Category theory as a mathematical pragmatics of 
model-based systems engineering]. \textit{Informatika i~ee Primeneniya~--- Inform. Appl.} 
12(1):95--104.

\end{thebibliography}

 }
 }

\end{multicols}

\vspace*{-6pt}

\hfill{\small\textit{Received September 3, 2020}}

%\pagebreak

%\vspace*{-18pt}
   
   
\Contrl

\noindent
\textbf{Kovalyov Sergey P.} (b.\ 1972)~--- Doctor of Science in physics and 
mathematics, leading scientist, V.\,A.~Trapeznikov Institute of Control Sciences of the 
Russian Academy of Sciences, 65~Profsoyuznaya Str., Moscow 117997, Russian 
Federation; \mbox{kovalyov@sibnet.ru}
   



\label{end\stat}

\renewcommand{\bibname}{\protect\rm Литература} 
   