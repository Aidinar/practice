\def\stat{seif}


\def\tit{НЕФТЬ КАК НОСИТЕЛЬ ИНФОРМАЦИИ О~СВОЕМ 
ПРОИСХОЖДЕНИИ, СТРУКТУРЕ И ЭВОЛЮЦИИ}
\def\titkol{Нефть как носитель информации о~своем 
происхождении, структуре и эволюции}

\def\autkol{Р.\,Б.~Сейфуль-Мулюков}
\def\aut{Р.\,Б.~Сейфуль-Мулюков$^1$}

\titel{\tit}{\aut}{\autkol}{\titkol}

%{\renewcommand{\thefootnote}{\fnsymbol{footnote}}\footnotetext[1]
%{Работа выполнена
%при финансовой поддержке РФФИ, проекты 08-01-00567 и
%08-07-00152.}}

\renewcommand{\thefootnote}{\arabic{footnote}}
\footnotetext[1]{Институт проблем информатики Российской академии наук, rust@ipiran.ru}     
     
     \Abst{Статья посвящена одному из аспектов применения законов информатики при 
исследовании сложных природных систем. Нефть как сложную систему можно 
рассматривать в качестве носителя информации, необходимой для оценки гипотез ее 
происхождения. Этой информацией является элементный и углеводородный состав нефти, 
распространение по площади, разрезу и в разновозрастных комплексах осадочного чехла. 
Носителем информации в нефти служат атомы углерода и водорода, определяющие 
химические связи и способности формировать углеводородные последовательности. 
Показано, что объем информации в исходном веществе и самой нефти позволяет оценить 
и сравнить существующие модели ее происхождения.}
     
     \KW{нефть; происхождение нефти; нефть как носитель информации; нефть как
сложная система; оценка объема информации; объем информации нефти}

     \vskip 18pt plus 9pt minus 6pt

      \thispagestyle{headings}

      \begin{multicols}{2}

      \label{st\stat}
     
     \section{Введение}
     
     Наряду со многими удивительными свойствами и характеристиками нефть 
обладает уникальной особенностью~--- ее исходное вещество, состав, условия 
залегания и происхождение можно трактовать с самых разных, нередко 
диаметрально противоположных точек зрения, и все они выглядят весьма 
обоснованными. Наглядно это можно видеть на примере гипотез о происхождения 
нефти.
     
     Вначале появились две противоположные гипотезы, причем обе были 
высказаны априори как догадки, поскольку геологическая, геохимическая и 
физическая доказательные базы в то время только закладывались. 
     
     Согласно первой, органической, гипотезе нефть образовалась из органических 
остатков бактерий, животных и растений, т.\,е.\ является детищем биосферы. Эта 
гипотеза активно поддерживается большинством геохимиков-нефтяников, 
несмотря на то, что ее единственным аргументом служат результаты изучения 
геохимии незначительной (не более первых процентов) составляющей осадочных 
пород, керогена. 
     
     Согласно неорганической гипотезе нефть имеет абиогенное, глубинное 
происхождение. Пред\-став\-ле\-ния об абиогенном происхождении нефти получили 
доказательства в связи с установлением углеводородной дегазации Земли, 
наличием клатратов и газогидратов. Эти явления и их масштабы на Земле в целом 
описаны в работах~[1--4].
     
     Анализ альтернативных гипотез показывает, что понятия, принципы и законы 
информатики практически не используются в исследованиях природы нефти. 
Законы информатики о сложной системе и ее развитии, единице информации (бите), 
которой можно выразить любое физическое вещество, информационная энтропия и 
ее изменения, наконец, понятие носитель информации и формы его существования 
и выражения практически не рассматриваются в приложении к изучению нефти. 
     
     Вместе с тем химический и элементный состав нефти, ее физические свойства 
и их воплощение в геологических условиях залегания нефти определяют ее как 
сложную систему~\cite{21s}. Поэтому законы информатики как средство их 
познания применимы к изучению проблемы ее генезиса. В~статье рас\-смат\-ри\-ва\-ет\-ся 
нефть как носитель информации, достаточной для объяснения ее состава, 
распределения в недрах по агрегатному состоянию и причинно-следственных 
факторов образования нефти~--- от исходного до конечного вещества.

\section{Информация о~первооснове нефти}
     
     Считается, что носитель информации~--- это физическое средство выражения 
абстрактного понятия. Носителями информации являются камень, дерево, папирус, 
бумага, холст, флэш-память, диск, другие физические материалы и нефть. 
В~первооснове носителем информации является атом, а точнее~--- электроны его 
орбиталей и элементарные частицы ядра. Биологическим аналогом является ген, в 
котором заложена вся информация о живом организме. 
     
     Изучение проблем нефти с позиций информатики диктует необходимость 
выделения ее первоосновы в каждой из двух гипотез происхождения нефти. 
У~органиков это химически стойкое органическое вещество, а именно липиды 
(соли жирных кислот), созданное живыми организмами биосферы, у 
     неоргаников~--- химические вещества и минералы, слагающие недра Земли, 
взаимодействие которых приводит к образованию углеводородов. 
     
     В органической гипотезе носителем информации о нефти является геном 
живого организма. В~неорганической гипотезе первичная информация о нефти как 
совокупности углеводородных последовательностей формируется на субатомном 
уровне, отражая особенности взаимодействия атомов углерода и водорода между 
собой и с другими элементами. 
     
     Определим информацию, заложенную в первооснове нефти. 
     
     Одной из важнейших является информация об элементном составе. Нефть в 
среднем на 99,9\% состоит из пяти элементов, их усредненный состав: 
углерода~--- 85\%, водорода~--- 13\%, кислорода~--- 0,8\%, азота~--- 0,6\% и серы~--- 
0,6\%~\cite{16s}. Эта информация определяет термодинамические, геологические и 
геохимические условия, обеспечившие концентрацию в одном веществе 85\% 
углерода и 13\% водорода. Глубинные зоны Земли могут обеспечить условия с 
такой термодинамикой, геологией и геохимией, а следовательно, и процесс 
соединения углеводородных последовательностей в нефть.
{\looseness=1

}
     
     Органическая гипотеза образования нефти требует обосновать энергию, 
обеспечившую трансформацию низкоуглеродного вещества живого, 
соответственно 20\% и 8,5\% в бактерии и водоросли, в нефть с 85\% углерода. 
Такое обоснование не всегда возможно. Недра многих нефтеносных территорий, 
особенно платформенных, никогда не погружались на глубины с температурой и 
давлением, необходимыми для катагенеза органических остатков~\cite{19s}. 
Недостаток температуры и давления для катагенеза сторонники органической 
гипотезы компенсируют временем низкотемпературного катагенеза, увеличивая 
его на десятки и даже сотни миллионов лет, или миграцией нефти из далеко 
расположенных зон генерации, в которых такие условия создавались, к зонам 
аккумуляции. Время, даже если оно ис\-чис\-ля\-ет\-ся сотнями миллионов лет, не может 
компенсировать недостаток энергии (температуры и давления), необходимой для 
катагенеза исходного вещества. Исходное вещество органической природы за эти 
миллионы лет в результате обмена веществом и энергией с окружающей геологической 
средой, и прежде всего с водой, неизменно потеряет свои самые ценные 
компоненты и утратит нефтеродные свойства. 
     
     Углеводородный состав нефти несет информацию о ее генезисе. 
Углеводороды нефти разделяются на две неравные части. Большая из них, 
составляющая от 80\% до~90\%,~--- это чистые углеводороды, т.\,е.\ молекулы, 
состоящие только из атомов углерода и водорода. Углеводородных соединений в 
нефти примерно~500, и ни одно из них в чистом виде не встречается ни в одном 
живом организме и не синтезируется непосредственно ни одним из них. 
Простейший углеводород метан может быть газообразным продуктом 
жизнедеятельности либо анаэробного биохимического разложения органических 
остатков на ранних стадиях их разложения, т.\,е.\ образованным из готовых 
органических материалов. Однако количества образующегося таким образом 
метана недостаточно для формирования углеводородов нефти, содержащихся в 
осадочных породах.
     
     Меньшая часть нефти (в среднем~4,5\%) пред\-став\-ле\-на гетероциклическими 
соединениями или гетероциклами. Это углеводородные соединения, часть атомов 
углерода которых замещена на атомы серы, кислорода или азота в более мягких 
термодинамических условиях, что создало условия для невалентных, водородных и 
ковалентных химических связей углеводородов с металлами и неметаллами. Нефть 
включает примерно 250~сернистых, 85~кис\-ло\-род\-ных и 30~зотных 
гетероциклов~\cite{16s}. Ге\-те\-ро\-цик\-лы имеют прямое отношение к биосфере, 
особенно азотсодержащие. Их некоторые характеристики важны для понимания 
генезиса нефти. Например, кероген нефтематеринских пород как основа гипотезы 
органического происхождения нефти~--- это в большей части 
гетероциклы~\cite{34s, 4s}. Как отмечает Пожарский~\cite{18s}, 
гетероциклы участвуют в строении и во многих жизненно важных процессах 
живой клетки, выполняя биохимические, биологические или физиологические 
функции, например такие гетероциклы, как липополисахариды бактерий, витамины 
и ферменты животных и рас\-те\-ний и~др. 
     
     Особенности гетероциклических углеводородов проявляются на сравнительно 
небольших глубинах земной коры, термодинамика которых обеспечивает 
формирование гетероциклов на основе углеводородов и реализацию их 
реакционной способности. 
     
     Таким образом, информация об углеводородном составе нефти дает 
основание считать, что 90\% углеводородов нефти не имеет прямого отношения к 
биосфере, а 4,5\% могут образоваться в термодинамических условиях земной коры, 
успешно взаимодействовать со всеми элементами биосферы, являться частью ее и 
накапливаться с органическими остатками в осадочных породах. 
     
\section{Информация о~способности нефти к~миграции}

     Вязкость нефти~--- это информация о ее миг\-ра\-ци\-он\-ных возможностях. 
Относительная (удельная) вязкость, выражающая отношение абсолютной 
(динамической) вязкости к вязкости воды,\linebreak зависит от температуры. Для различной 
нефти при температуре более 50~$^\circ$C она превосходит вязкость воды 
минимум в 1,5~раза. Эта физическая характеристика вместе с фильтрацией~--- 
показателем характера движения флюида через породу и свойством самой породы 
пропускать флюид, называемым проницаемостью,~--- определяет ее 
миг\-ра\-ци\-он\-ную способность в пористой среде пластов осадочных горных пород. 
     
     Нефть всегда эпигенетична и может попасть в залежь только в результате 
миграции: горизонтальной, вертикальной или их комбинации, которые зависят от 
характеристик флюида и породы.
     
     Горизонтальная миграция~--- это движение нефти по порам и пустотам 
осадочных пород различного размера и формы. При любой их комбинации 
перемещение флюида может осуществляться только при наличии градиента 
давления, а поскольку в горизонтально залегающих толщах он практически равен 
нулю, то в таких толщах флюид как бы стоит, даже если он находится под 
огромным геостатическим давлением. Лейбензон~\cite{12s} установил, что 
даже при значительной разнице начальных и конечных давлений в идеальном 
грунте происходит значительное падение давления мигрирующего флюида за счет 
сил адсорбции частицами пористой среды. В~реальной среде потери на адсорбцию 
при горизонтальной миграции превышают возможности даже самых 
<<продуктивных>> нефтематеринских свит. Многие исследователи считают, что 
нефть не способна к далекой горизонтальной миграции в толщах осадочных пород, 
поэтому она не может служить основным фактором формирования нефтяных 
залежей. 
     
     Для вертикальной миграции газа, жидкости и их смесей в недрах значительно 
больше возможностей. Эта миграция всегда обусловливается градиентом давления 
и температуры от среды с большей энергией к среде с меньшей энергией. 
Вертикальной миграции способствуют ослабленные тектонические зоны, 
которыми являются активные глубинные разломы, границы крупных фрагментов 
земной коры и их частей и каналы, созданные в силу особенностей состава и 
строения литосферы.
     
     Именно вертикальной миграции газообразных углеводородов глубинной 
природы и обязаны метан, газогидраты и клатраты. Валяев~\cite{6s} 
оценивает объем вертикальной миграции глубинных углеводородов (в основном 
метана) в $5\cdot 10^{13}$~г/год или $2{,}5\cdot 10^{16}$~т за 500~млн лет, что 
соответствует времени палеозоя, мезозоя и кайнозоя. Запасы клатратов 
оцениваются в $3\cdot 10^{12}$~т~\cite{17s}, а газогидратов~--- в $3\cdot
10^{12}$~т~\cite{26s}. Эти формы простейшего углеводорода не имеют никакого 
отношения к биосфере, и их количество в земной коре на порядки превосходит 
запасы всех жидких и газообразных углеводородов, содержащихся в рассеянном 
виде в осадочных породах.
     
\section{Информация о~месте образования нефти}

     Информация о нефти~--- в сочетании ее территориального распространения и 
постоянного состава. Нефть в промышленных масштабах или в виде проявлений 
различной интенсивности присутствует в недрах всех континентов, на дне 
прилегающих к ним акваторий мирового океана, в жерлах вулканов, на дне озера 
Байкал~\cite{2s}, в гранитах Скандинавии~\cite{30s}, в породах кристаллического 
фундамента и во многих других местах. К~1996~г.\ было выделено более 
550~нефтегазоносных бассейнов, из которых более~70~--- на шельфах морей и 
океанов. В~226~бассейнах было открыто более 20\,000~нефтяных и нефтегазовых 
месторождений~\cite{15s}, каждое из 50~наиболее крупных из них содержит более 
1~млрд~т~\cite{11s}. Общее представление о распространении нефтеносных 
территорий дают схематические карты, приведенные во многих работах. 
     
     Повсеместное распространение нефти дополняется постоянством ее среднего 
углеводородного состава, независимо от географического места и глубины 
залегания. Нефть может быть немного легче, немного тяжелее, содержать 0,5\% 
или 2\% серы, но это повсюду нефть определенного углеводородного состава. 
В~этом проявляется инвариантность и уникальность нефти как сложной 
системы~\cite{21s}.
     
     Повсеместность распространения и постоянство состава говорят только об 
одном~--- едином, глобальном источнике и едином природном механизме 
генерации углеводородных последовательностей. Неорганическая гипотеза легко 
объясняет эту особенность нефти, поскольку глубины, на которых протекают 
процессы образования углерода и водорода как элементов, и глубины литосферы, 
на которых формируются углеводородные последовательности, имеют 
необходимую энергетику и катализаторы, и эта сфера охватывает весь земной шар. 
     
     Органическая гипотеза в объяснении этого феномена вынуждена оперировать 
осадочными бассейнами прошлых эпох как областями аккумуляции органических 
веществ в виде остатков бактерий, животных и растений. В~геологической 
ретроспективе области осадконакопления были развиты не повсеместно, 
разновозрастные бассейны не совпадали в пространстве, а условия, в которые затем 
попадали осадки, отложенные в любом бассейне и сформированные из них 
комплексы осадочных пород, в том числе и с остатками органических веществ, 
очень сильно различались и нередко целиком размывались~\cite{19s}. Поэтому 
геологический фактор, в принципе весьма динамичный и очень разный в 
проявлениях на платформенных и геосинклинальных областях, не объясняет 
консерватизма состава нефти и повсеместности ее распространения, в том числе и 
вне пределов развития осадочных пород. 
     
\section{Информация о~возрасте нефти}

     Информация о нефти~--- в комплексах осадочных и магматических пород, от 
кристаллического фундамента, с возрастом более 1,5~млрд лет, до третичного, не 
старше 86~млн лет. Этот факт важен для понимания генезиса нефти. 
     
     В органической гипотезе возраст нефти и вмещающей породы может быть 
одинаков. Следовательно, нефть может быть докембрийской, напри\-мер вендской 
(старше 650~млн лет), палеозойской, например девонской (не моложе 364~млн 
лет), или третичной, например майкопской (не моложе 12~млн лет). Признание 
факта накопления материнского вещества нефти в докембрии или палеозое 
неизбежно привязывает исходное вещество к биосфере прошлых периодов. 
Например, биосфера венда была представлена бактериями и многоклеточными, 
беспозвоночными, безраковинными животными типа медуз и слизняков, 
относимой к так называемой эдиакарской фауне~\cite{14s}. Сохранение их 
остатков в морском осадке, равно как и органических остатков бактерий, а тем 
более формирование из них керогена, проблематично, хотя бы потому, что эти 
остатки были частью замкнутых трофических экосистем.
     
     Образование нефти сотни миллионов лет тому назад неизбежно означает 
стабильный, неизменный элементный и углеводородный ее состав, сохраняющийся 
сотни миллионов лет. Это противоречит закону информатики о постоянном развитии и 
изменении сложных систем и закону о самопроизвольном возрастании энтропии 
или беспорядка любой сложной системы. Неоднократное изменение геологической 
структуры, которое испытывала любая нефтеносная территория, также изменяло 
нефть. 
     
     Неорганическая гипотеза в ее существующих вариантах не рассматривает 
зависимость между образованием нефти и ее возрастом, концентрируясь на 
физической, химической и термодинамической сторонах процесса образования 
нефти.
     
\section{Информация об~исходном веществе и~генезисе нефти}

     Нефть несет информацию о себе тем, что молекулы углеводородов состоят из 
атомов углерода и водорода. Поэтому в связи с генезисом нефти, состоящей в 
основном из этих двух элементов, вопрос~--- откуда берутся сами атомы, в силу 
каких физических процессов и химических реакций они соединяются в молекулы 
углеводородных последовательностей, вполне закономерен. В~такой постановке 
проблема происхождения нефти никогда не ставилась. Косвенно 
Жармен~\cite{33s}, анализируя каталитические превращения систем С--Н на 
атомном и молекулярном уровне, практически впервые доказал возможность 
превращения простейших парафинов в олефины и ароматические углеводороды и 
наоборот при температурах, не превышающих $+1150$~$^\circ$C, в присутствии 
катализаторов, т.\,е.\ в условиях земной коры.
     
     Происхождение атомов углерода и водорода как первоосновы нефти~--- это 
один из ключевых вопросов в проблеме ее образования.
     
     Схема появления атомного, а затем формирования молекулярного состояния 
вещества рас\-смот\-ре\-на в публикации Фомина~\cite{25s}. В~этой проблеме 
он опирался на результаты исследований Штерн\-хай\-ме\-ра о 
     физико-химических характеристиках астеносферы и представлениях 
Тхоровской~\cite{23s} и Капустинского~\cite{10s} о появлении 
атомов как факте саморазвития материи Земли. Для идей о развитии материи 
важным явилось представление Амбарцумяна~\cite{1s}, основанное на 
астрофизических наблюдениях <<\ldots развитие материи идет от простого к 
сложному, от более плотного к менее плотному состоянию>>. В~рас\-смат\-ри\-ва\-емом 
контексте плотным является ультрасжатое под давлением 50~ГПа (50~тыс.\ атм) 
гомогенное вещество внутренних час\-тей Земли, находящееся ниже уровня 400~км, 
где проходит верхняя астеносфера. При таком давлении, как считает 
Фомин, вещество не может быть горячим, а атом не может сохранить свою 
ядерно-орби\-таль\-ную конфигурацию.
     
     Переход аномального состояния вещества в <<нормальное>>, как называет 
эти состояния Фомин, сопровождается выделением огромной тепловой 
энергии, расплавлением вещества мантии и его декомпрессией. Дегазация Земли, в 
том числе и углеводородная, появление рас\-плав\-лен\-ной магмы и появление 
<<нормальных>> атомов~--- это и есть последствия декомпрессии. 
     
     С этого рубежа атомы проявляют способность формировать химические 
связи. Появление первых соединений углерода с водородом связано со свойствами 
атома углерода, обладающего уникальной способностью реализовать свои 
валентные возможности. Так возникают первые, простейшие, газообразные 
углеводородные последовательности: СН$_4$--метан, С$_2$Н$_6$--этан и 
С$_2$Н$_2$--ацетилен. 
     
Жермен~\cite{33s} показал, что система углерод--во\-до\-род в присутствии 
катализаторов испытывает превращения, определяемые заданным соотношением 
температуры и давления. Углеводород парафинового ряда при высокой 
температуре может крекироваться до смеси парафинов и олефинов, 
дегидрогенизироваться до олефинов с тем же чис\-лом атомов углерода, 
дегидроциклизироваться до ароматики либо разложиться на изначальный углерод 
и водород. В литосфере усложняется состав и структура системы С--Н с 
формированием углеводородных последовательностей, и эти изменения 
определяются термодинамикой и катализаторами. 
     
     Важно, что все процессы и превращения углеводородных 
последовательностей происходят не локально, не в изолированных 
разновозрастных линзах и толщах осадочных пород, а повсеместно,\linebreak охватывая 
определенную геосферу Земли, начиная с верхней астеносферы, где зарождаются 
атомы, до зоны нефтенакопления, отражая общую эволюцию развития вещества 
планеты. Локализовать точное место образования нефти нельзя, ее образование 
начинается с момента появления атомов углерода и водорода и заканчивается в 
залежи. Там же начинается процесс разрушения нефти. Нефть как сложная система 
не развивается по-другому.
     
     Для сравнения гипотез происхождения и эволюции нефти необходим 
одинаковый критерий, позволяющий оценить динамическое состояние сис\-те\-мы на 
отдельных этапах ее развития согласно существующим точкам зрения и сравнить 
динамику этих состояний. Две гипотезы происхождения нефти такого единого, 
общепризнанного критерия не используют, но такой критерий существует, и 
     это~--- информационное содержание. Информация содержится в атомах, 
составляющих молекулы исходного и промежуточного вещества нефти, а оценка 
объема информации в атомах не зависит от точек зрения на природу этих веществ.
     
     Великие физики, включая Эйнштейна~\cite{29s}, Шеннона~\cite{27s}, 
Бриллюена~\cite{5s}, доказали связь основных категорий мироздания с 
информацией. Поэтому использование информации для оценки состояния системы 
и ее динамики во времени и пространстве отражает и основывается на связи между 
информацией и энергией, информацией и энтропией, информацией и гравитацией, 
ин\-форма\-ци\-ей и массой~\cite{7s}. Рассматривая информационные взаимодействия в 
системах неживой и живой природы~\cite{20s}, мы основывались на этих 
положениях, считая, что информация~--- универсальная категория, позволяющая 
выразить количество и качество материи, энергии, пространства, времени и 
движения, вовлеченных в любой процесс или явление.
     
     Использование информации позволяет измерить первооснову системы~--- 
атомы и вещество, ими составленное, и проследить динамику изменения 
информационного содержания вещества на различных этапах его преобразования. 
     
     Изменение информационной характеристики сложных систем и материи в 
целом согласуется и с общими закономерностями развития, уста\-нов\-лен\-ны\-ми 
Амбарцумяном~--- развитие материи идет от простого к сложному, от более 
плотного к менее плотному состоянию. Процесс перехода от простого состояния к 
более сложному~--- это процесс увеличения порядка, информации и, 
соответственно, уменьшения неопределенности и энтропии. Развитие материи от 
плотного к менее плотному состоянию трудно выразить в процессах образования 
нефти. Однако можно считать, что источник исходного вещества находится в более 
плотном состоянии, нежели горные породы земной коры и тем более рыхлый 
морской осадок, в котором накапливается органическое вещество.
     
     В общем случае, если процесс формирования конечной сложной системы 
идет с поглощением тепла, то этот процесс происходит с возрастанием ее 
энтропии, увеличением неопределенности и уменьшением информации в системе. 
Эндотермический процесс катагенеза керогена как основа\linebreak
образования нефти~--- 
это и есть процесс увеличения энтропии системы. Однако процесс образования 
нефти как сложной системы углеводородных последовательностей означает 
уменьшение\linebreak
 энтропии, увеличение сложности и, соответственно, увеличение 
объема информации. Поэтому образование нефти с термодинамической точки 
зрения есть процесс трансформации исходного вещества, находящегося в нагретом 
состоянии, в вещество более холодное, от материи, находящейся в более плотном 
состоянии, к веществу менее плотному. Используя объем информации как 
критерий, можно проследить, как она изменяется по всей цепочке трансформации 
от исходного вещества до нефти и битума как для органической, так и для 
неорганической гипотезы.
     Для расчетов объема информации используем данные об объеме информации 
в атомах элементов таблицы Менделеева, приведенные в работе~\cite{8s}. 
Значения объема информации в атомах пяти элементов, составляющих 99,9\% 
нефти, равны: 
     С~--- 109,642; Н~--- 10,422; S~--- 317,504; N~--- 138,908 и О~--- 149,33~бит.
     
     Объем информации в единице массы вещества, например нефти или керогена, 
складывается из информации атомов каждого элемента в молекулах, составляющих 
единицу массы. При этом используется брутто-формула вещества либо 
эмпирическая формула, по которой и рассчитывается количество информации в 
атомах. Помимо информации в атомах учитывается объем информации в 
структуре молекулы. В расчетах Гуревича~\cite{8s} он основан на числе 
валентных связей атомов, составляющих структуру молекулы. 
     
     Объем информации, содержащейся в нефти, может быть рассчитан как сумма 
каждой из трех основных ее частей: углеводородов, гетероциклов и примесей~--- либо 
по эмпирической формуле, со\-став\-лен\-ной по процентному содержанию слагающих 
ее основных элементов. 
     
     Углеводородный состав нефти, составляющий 90\% ее массы, включает три 
группы углеводородов: парафины (30\%--35\%), нафтены (25\%--75\%) и 
ароматические (10\%--15\%)~\cite{16s}. Расчет по брутто-фор\-му\-лам для типичных 
жидких представителей этих групп: пентана С$_5$Н$_{10}$, циклогексана 
С$_6$Н$_{12}$ и бензола С$_6$Н$_6$~--- и для трех азотсодержащих гетероциклов: 
нейтральных, кислых и основных аминокислот~--- определяет объем информации в 
условной молекуле нефти, равный \textbf{6171}~бит.
     
     Объем информации для легкой и тяжелой нефти рассчитан также и по 
эмпирической формуле. Учитывалось, что в легкой нефти на 2\% меньше С и 
соответственно больше Н и О, а в тяжелой~--- на 2\% больше С и соответственно 
меньше Н и~О. В~качестве примера приведем расчет количества информации, 
содержащейся в условной молекуле легкой нефти.
     
     Элементный состав легкой нефти в среднем таков: С~--- 83\%, Н~--- 14\%, 
О~--- 2\%, S~--- 0,9\% и N~--- 0,1\%. Количество атомов каждого элемента, исходя из его 
процентного содержания и массы, равно: 
C~--- 83/12\;=\;6,9; H~--- 14; O~--- 2/16\;=\;0,13;
S~--- 0{,}9/32\;=\;0,03; N~--- 0,1/14\;=\;0,01.
     Общее число атомов~--- 6,9\;+\;14\;+\;0,13\;+\;0,03\;+\linebreak +\;0,01\;=\;21,07 атомных 
единиц массы, а число атомов в эмпирической формуле (условной молекуле) 
легкой нефти равно:
     C~--- 32,7; H~--- 66,4; O~--- 0,6; S~--- 0,1; N~--- 0,1, и эмпирическая формула 
имеет вид C$_{32}$H$_{66}$OSN.
     
     Соответственно, объем информации составит:
     C~--- 3507, H~--- 686; O~--- 149; N~--- 138,9; S~--- 317,5. В сумме это 
4798~бит плюс 64~бита на структуру условной молекулы, итого \textbf{4862}~бит. 
     
     Элементный состав тяжелой нефти: С~--- 86\%, Н~--- 12\%, О~--- 0\%, S~--- 0,9\% 
и N~--- 0,1\%.
     
     Аналогичный расчет дает ее эмпирическую формулу C$_{37}$H$_{62}$SN. 
Объем информации в условной молекуле тяжелой нефти равен \textbf{5228}~бит.
     
     Таким образом, разные подсчеты показали небольшую разницу в объеме 
информации в условной молекуле нефти.
     
\section{Информация об~эволюции нефти в~земной коре}

     Первыми углеводородными последователь\-ностями, образующимися на 
начальных этапах генезиса нефти, являются: СН$_4$~--- метан, С$_2$Н$_6$~--- этан и 
С$_2$Н$_2$~--- ацетилен. Самый простой и\linebreak стойкий из них~--- метан. Информация, 
содержащаяся в его молекуле, принята за исходную для всех последующих 
углеводородных последовательностей, а поскольку молекула метана состоит из 
одного атома углерода и четырех атомов водорода, то это дает \textbf{156}~бит, 
включая объем информации, содержащейся в структуре молекулы метана. 
     

     Эволюция углеводородных последовательностей от простейших, 
образующихся в астеносфере, до аккумуляции самой полной 
     по\-сле\-до\-ва\-тель\-ности~--- нефти в зоне нефтенакопления есть\linebreak процесс 
усложнения состава и структуры последовательностей. Существует уровень, ниже 
которого углеводороды могут существовать только в газообразном состоянии, а 
выше~--- в газообразном, жидком и твердом. Газообразным углеводородом этого 
уровня условно принят бутан~--- С$_4$Н$_{10}$, молекула которого состоит из 
четырех атомов углерода и десяти атомов водорода, что соответствует 
\textbf{547}~битам информации.
     

     Конечной стадией эволюции углеводородных последовательностей нефти в 
верхней части земной коры являются битумы битуминозных пород.\linebreak Би\-тумы~--- это 
асфальтово-смолистые вещества, наиболее тяжелые компоненты нефти с наиболее 
сложной структурой молекул. Их усредненный элементный состав, приведенный 
в~\cite{16s}, таков: С~--- 84\%, Н~--- 8\%, S~--- 3\%, O~--- 4\% и N~--- 1\%. 
Соответственно, в условной единице массы битума содержится: С~--- 6,91; Н~--- 8; 
S~--- 0,09; N~--- 0,07; O~--- 0,25 и в сумме 15,41~атомов, а в атомных процентах: 
     С~--- 45,4; Н~--- 51,9; S~--- 0,6; N~--- 0,5; O~--- 1,6. Эмпирическая формула 
битума~--- C$_{45}$H$_{51}$OSN, а объем информации в условной единице массы 
битума составляет \textbf{6365}~бит. 
     
     Таким образом, динамика изменения информации в единице условной массы 
от начальных стадий образования углеводородных последовательностей до 
конечной в виде их совокупности в нефти и, наконец, в битуме битуминозной 
породы по неорганической схеме выгладит следующим образом:
     \begin{itemize}
\item на уровне образования первого простейшего углеводорода~--- \textbf{156}~бит; 
\item на уровне существования газообразных углеводородов~--- \textbf{547}~бит;
\item на уровне главной зоны нефтенакопления (1500--3500~м) полная 
совокупность углеводородных последовательностей (нефть)~--- 
\textbf{6171}~бит; 
\item верхний уровень, близкий к поверхности Земли (нефть, лишенная легких 
компонентов,~--- битум битуминозной породы)~--- \textbf{6365}~бит. 
     \end{itemize}
     
     Динамика изменения веществ от исходного до конечного в органической 
гипотезе выглядит следующим образом. Одной из наиболее полных работ об 
образовании нефти по этой схеме является монография~\cite{34s}. Исходя из 
химического состава биомассы, авторы определили основные компоненты жиров и 
масел, встречающихся в организмах, которые откладываются как органическое 
вещество осадочной породы. Для расчета объема информации условной единицы 
исходного вещества выбраны: триглицерид~--- С$_{18}$Н$_{20}$О$_6$, стеариновая 
С$_{18}$Н$_{36}$О$_2$ и линоленовая С$_{18}$Н$_{30}$О$_2$ 
аминокислоты~\cite{34s, 13s}. Объем их информации соответственно равен 3100, 
2670 и 2590, а условная единица массы исходного вещества для образования нефти 
содержит \textbf{8360}~бит информации. 
     
     В монографии~\cite{4s} приводится эмпирическая формула керогена, 
выведенная из соотношения масс его элементов,~--- С$_{55}$Н$_{40}$NSO$_3$. 
Соответственно, объем информации в условной молекуле керогена равен 
\textbf{7369}~бит.
     
     Таким образом, динамика изменения объема информации от начальных 
стадий накопления остатков липидов, созревания керогена в нефтематеринской 
породе до конечной стадии~--- нефти и битума битуминозной породы~--- по 
органической схеме выглядит следующим образом: 
     \begin{itemize}
\item на уровне биосферы исходное вещество (липид)~--- \textbf{8360}~бит; 
\item на уровнях созревания керогена (глубины 5--15~км)~--- \textbf{7369}~ бит; 
\item на уровне главной зоны нефтенакопления (1500--3500~м) как полная 
совокупность углеводородных последовательностей (нефть)~--- 
\textbf{6171}~бит; 
\item на верхнем уровне, близком к поверхности Земли (битум битуминозной 
породы)~--- \textbf{6365}~бит. 
     \end{itemize}
     
     Таким образом, для сравнения схем генезиса нефти имеется два ряда 
последовательных значений объема информации, содержащейся в исходном и 
конечном веществе, отражающих две различные схемы. Подобная оценка более 
чем нетривиальна. Она произведена на основе современных пред\-став\-ле\-ний о роли 
информации в строении и развитии материи и базируется на массе и заряде атомов 
или структуре молекулы веществ, определяемых на базе общепринятых 
химических и физических констант, что позволяет сравнить результаты, 
полученные для разных схем процесса генезиса, но по одному стандарту 
измерения. 
     
     Полученные показатели объема информации не привязаны к геологическому 
времени, к глубине или месту с заданными физико-географическими, 
геологическими и геохимическими параметрами, которые по отношению к нефти и 
к значениям отдельных стадий преобразования исходного вещества являются 
переменными величинами и допускают различные интерпретации. 
     
     Последовательности объемов информации позволяют сделать следующие 
выводы.
     
     Абиогенная схема генезиса от образования первых углеводородных 
последовательностей и дальнейшего усложнения их состава и структуры в\linebreak 
литосфере и в земной коре подкрепляется соответствующим изменением объема 
информации. Ее увеличение соответствует усложнению структуры и состава 
углеводородных последовательностей от простейших вплоть до нефти и битума во 
всем вертикальном диапазоне их распространения.
     
     Органическая гипотеза по последовательности изменения объема 
информации в исходном, промежуточном и конечном веществе не соответствует 
законам развития материи, требует ответа на некоторые вопросы. Схема 
объективно отражает последовательное снижение объема информации от 
исходного вещества до нефти и затем некоторое его увеличение до стадии битума, 
что трактует синтез нефти из готовых органических материалов как образование 
сложной системы, т.\,е.\ процесс, протекающий от сложного к простому. В~этом 
процессе вещество от менее разогретого состояния, находящегося под меньшим 
давлением, переходит в более разогретое состояние под большим давлением. 
     
     Однако с точки зрения кинетики процесс перестройки структуры исходного 
вещества, его декомпозиция, разрушение существующих химических связей и 
создание нового вещества, в котором существенно увеличивается концентрация 
углерода, не может быть трансформацией от сложного к прос\-тому. 
     
     С геохимической точки зрения кероген есть нерастворимая, 
дебитуминизированная часть ор\-га\-ниче\-ского вещества осадочных пород или 
остаток\linebreak существовавшего ранее. Кероген, нефтеродный потенциал которого 
определяют в настоящее время, а породы, его содержащие, называют 
нефтематеринскими,~--- фактически это породы, бывшие нефтематеринскими и 
уже реализовавшие этот потенциал. Содержащееся в нем органическое вещество и 
объем его информации есть не отражение этого потенциала, а характеристики 
метаморфизованных остатков липидов, смешанных с углеводородными 
последовательностями глубинного про\-ис\-хож\-де\-ния, отложившимися вместе с 
осадком. 
     
     Битум~--- это смолисто-асфальтовые компоненты нефти, состоящие из 
сложных по структуре молекул, а нефть по ее углеводородному составу и 
соотношению углерод--водород не вписывается в схему между исходным и 
конечным состоянием. 
     
     Следовательно, природный процесс: 
    \begin{multline*}
     \mbox{липид}\;\rightarrow\;\mbox{осадок обогащенный}\; 
\mathrm{С}_{\mathrm{орг}}\;\rightarrow\\
\rightarrow\;\mbox{кероген}\;\rightarrow\;\mbox{микронефть}
\;\rightarrow\;\mbox{нефть}\;\rightarrow\,\mbox{битум}\hspace*{-4.55pt}
     \end{multline*}
     с точки зрения закономерностей развития материи, изменения энтропии и 
информационного содержания не реализуем. 
     
     Осадочные толщи с повышенным содержанием углерода и углеводородов 
развиты глобально и гораздо шире, чем собственно нефть. Толщи, считавшиеся 
ранее нефтематеринскими, потому что сейчас содержат первые проценты 
углеводородных веществ, на самом деле таковыми не являлись и не являются. 
Содержащиеся в них углеводороды, принимаемые за C$_{\mathrm{орг}}$, в 
основном представляют собой смесь органического вещества биосферной природы 
и углеводородных последовательностей глубинного происхождения, 
аккумулированные в осадках одновременно с осадконакоплением. 
     
     В условиях термодинамики земной коры эти углеродсодержащие толщи не 
обладали нефтегенерирующим потенциалом, что показало изучение кернов пород 
моложе 150~млн лет и содержащегося в них углерода, в том 
числе~C$_{\mathrm{орг}}$. Керны были отобраны по программе глубоководного 
бурения на шельфах и на дне мирового океана. Сошлемся лишь на два 
отчета~\cite{31s, 32s}. 
     
     Индексы нефтегенерационного потенциала, соответ\-ствующие 10~градациям 
катагенеза~\cite{3s}, на самом деле отражают степень метаморфизма 
углеводо\-родных последовательностей, аккумулированных с осадком. Остатки 
углеводородных гетероциклов биосферы, захороненные с ними углеводородные 
последовательности не имеют прямого отношения к нефти, аккумулированной в 
залежах и месторождениях главной зоны нефтенакопления.
     
     
\section{Заключение}
\
     Итак,      
углеводородный и элементный состав нефти показывает абсолютное 
преобладание в ней углерода, не характерного для живого организма биосферы. 
     Образование нефти в процессе перевода в верхней части земной коры 
низкоуглеродистого органического вещества в высокоуглеродистую нефть в 
результате катагененеза керогена в об\-ластях с осадочным чехлом относительно 
небольшой мощности не обеспечено необходимой для этого процесса 
температурой и давлением. Геологическое время не может быть заменой энергии, 
необходимой для катагенеза, поскольку изменения внутренней структуры и состава 
керогена, происходящие в течение требуемого геологического времени (не менее 
10~млн лет), независимо от внешних условий среды носят деструктивный, 
необратимый для керогена характер. 
     
     Углеводородный состав нефти означает, что преобладающая ее часть в виде 
углеводородов (соединений только атомов углерода и водорода) не имеет 
биосферного происхождения.
     
     Нефть в силу своей вязкости и адсорбции час\-ти\-ца\-ми горной породы не 
способна мигрировать по горизонтально залегающим комплексам осадочных 
пород на большие расстояния, поэтому латеральная миграция не может рассматриваться как фактор 
формирования ее залежей.
     
     Глобальное распространение нефти, ее повсеместность в земной коре 
доказывают независимость ее образования от геологических условий недр. 
Образование нефти не привязано к осадочным седиментационным бассейнам. 
Структура, состав, возраст и взаимоотношение вме\-ща\-ющих пород имеют для 
нефти значение только как каналы миграции, создатели условий для аккумуляции, 
завершающей стадии оформления состава нефти в данной геологической формации 
и кратковременной, в геологическом масштабе времени, консервации нефти в 
залежи.
     
     Нахождение нефти в комплексах пород различного состава и любого возраста 
не доказывает, что они <<ровесники>>. Докембрийской, палеозойской и 
мезозойской нефти в недрах быть не может, а есть залежи нефти, близкой по 
элементному и углеводородному составу, в фундаменте докембрийских, 
палеозойских, мезозойских и кайнозойских отложений. Нефть в любых породах 
как сложная сис\-те\-ма может существовать в недрах в первозданном виде короткое в 
геологическом масштабе время.
     
     Определение объема информации в нефти и веществах, с ней генетически 
связанных, может явиться вполне современным инструментом, дополняющим 
существующие методы изучения нефти и ее генезиса.

{\small\frenchspacing
{%\baselineskip=10.8pt
\addcontentsline{toc}{section}{Литература}
\begin{thebibliography}{99}     

\bibitem{6s} %1
\Au{Валяев Б.\,М.}
Углеводородная дегазация Земли и генезис нефтегазовых месторождений~// 
Геология нефти и газа, 1997. №\,9.

\bibitem{9s} %2
\Au{Дмитриевский А.\,Н., Валяев Б.\,М.}
Углеводородная дегазация через дно океана: локализация, проявления, масштабы, 
значимость~// Дегазация Земли и генезис углеводородных флюидов и 
месторождений.~--- М.: ГЕОС, 2002. С.~7--36.

\bibitem{26s} %4 %3
\Au{Фрадкин В.}
Газ на дне океана как альтернатива энергоносителя. 2004. {\sf http://n-t.ru/tp/ie/gn.htm}.

\bibitem{17s} %3 %4
Новый вид ископаемого топлива, использующийся только в России~// Наука. 
Известия, 2009. {\sf  http:// www.inauka.ru/news/article93252html}.

\bibitem{21s} %5
\Au{Сейфуль-Мулюков Р.\,Б.}
Нефть в квантовом мире~// Системы и средства информатики. Доп. вып.~--- М.: 
ИПИ РАН, 2008. С.~195--213.

\bibitem{16s} %8 %6
Нефть. {\sf http://ru.wikipedia.org/wiki/нефть}.

%\bibitem{28s} %6
%Элементный состав нефти и гетероатомные компоненты. {\sf  http://ru.wikipedia.org}.

\bibitem{19s} %7
\Au{Сейфуль-Мулюков Р.\,Б.}
Палеотектонические факторы нефтеобразования и нефтенакопления.~--- М.: Недра, 
1983.  269~с.

\bibitem{34s} %9 %8
\Au{Tissot B.\,P., Velte~D.\,H.}
Petroleum formation and occurrence. A new approach to oil and gas exploration.~--- 
Berlin: Springer-Verlag, 1978. 
     
\bibitem{4s} %10 %9
\Au{Богородская Л.\,И. Конторович А.\,Э., Ларичева~А.\,И.}
Кероген: методы изучения, геохимическая интерпретация.~--- Новосибирск: Изд-во 
СО РАН, 2005.  254~с.

\bibitem{18s} %11 %10
\Au{Пожарский А.\,Ф.}
Гетероциклические соединения в биологии и медицине~// Статьи Соросовского 
Образовательного журнала, 1996. {\sf  
http://www.pereplet.ru/ obrazovanie/stsoros/112.html}.

\bibitem{12s} %12 %11
\Au{Лейбензон Л.\,С.}
Движение природных жидкостей и газов в пористой среде.~--- М.: 
ОГИЗ--Гостехиздат, 1947.  244~с.

\bibitem{2s} %13 %12
Байкал открывает свои тайны: на дне озера обнару\-жили источник нефти. 2008. {\sf  
http://www.rian.ru/\linebreak elements/20080807/150155919.html}.

\bibitem{30s} %14 %13
\Au{Laherrere J.}
No free lunch, Part~1: A critique of Thomas Gold's claims for abiotic oil.~--- The Wilderness Publications, 2004.
 {\sf   http://www.copvcia.com/ free/ww3/102104\_no\_free\_pt1.shtml}.

\bibitem{15s} %15 %14
Нефтегазоносные бассейны мира. Карта.~--- СПб.: ВСЕГЕИ, 1995. 

\bibitem{11s} %16 %15
Крупнейшие нефтяные месторождения мира. {\sf  http:// ru.wikipedia.org/wiki/крупнейшие\_нефтяные\_место рождения\_мира}.

\bibitem{14s} %17 %16
\Au{Малаховская Я.\,Е., Иванцов А.\,Ю.}
Вендские жители Земли.~--- Архангельск: ПИН РАН, 2003.  48~с.

\bibitem{33s} %18 %17
\Au{Germain J.\,E.}
Catalytic conversion of hydrocarbons.~--- London--New York: Academic Press, 1969.

\bibitem{25s} %19 %18
\Au{Фомин Ю.\,М.}
Верхняя астеносфера~--- переходная зона между веществом мантии и литосферы. {\sf 
http://www.evolbiol.ru/fomin.htm}.

\bibitem{23s} %20 %19
\Au{Тхоровская Н.\,В.}
Аномалия Земли~// Материалы международной конференции памяти акад.\ 
П.\,Н.~Кропоткина, 20--24~мая 2002.~--- М.: ГЕОС, 2002. С.~454--455.

\bibitem{10s} %21 %20
\Au{Капустинский А.\,Ф.}
Геосферы и химические свойства атомов~// Геохимия, 1956. №\,1.  С.~53--61.

\bibitem{1s}  %22 %21
\Au{Амбарцумян В.\,А.}
Научные труды. Т.~2.~--- Ереван: Изд-во АН Армянской ССР, 1960.

\bibitem{29s} %23 %22
\Au{Эйнштейн А.}
Основы общей теории относитель\-ности~// Собр. науч. тр. Т.~1.~--- М.: 
Наука, 1965.

\bibitem{27s} %24 %23
\Au{Шеннон К.}
Математическая теория связи~// Работы по теории информации и кибернетике.~--- 
М.: Изд-во иностранной литературы, 1963.


\bibitem{5s} %25 %24
\Au{Бриллюен Л.}
Наука и теория информации.~--- М.: Физматгиз, 1960.


\bibitem{7s} %26 %25
\Au{Гуревич И.\,М.}
Законы информатики~--- основа строения и познания сложных систем.~--- М.: 
ТОРУС ПРЕСС, 2007.  400~с.

\bibitem{20s} %27 %26
\Au{Сейфуль-Мулюков Р.\,Б.}
Информация и информационные процессы в системах неживой и живой природы~// 
Системы и средства информатики. Спец. вып.~--- М.: ИПИ РАН, 2007. С.~140--156.

\bibitem{8s} %28 %27
\Au{Гуревич И.\,М.} 
Информационные характеристики физических систем.~--- М.: Изд-во 11~формат, 2009. 167~с.


\bibitem{13s} %29 %28
Липиды. {\sf http://lipid.narod.ru/fa.html}.


\bibitem{31s} %30 %29
\Au{McIver R.\,D.}
Hydrocarbons in canned mud from sites 185, 186, 189, and 191~--- Leg~19~//
Initial Reports of the Deep Sea Drilling Project, 1973. Vol.~19. P.~875--877. 
{\sf http:// www.deepseadrilling.org/19/volume/dsdp19\_34.pdf}. 

\bibitem{32s} %31 %30
\Au{Simoneit B.\,R.\,T.}
Organic geochemistry of the shales from the Northwestern Proto-Atlantic, DSDP Leg~43~//
Initial Reports of the Deep Sea Drilling Project, 1979. Vol.~43. P.~643--650. 
{\sf http://www.deepseadrilling.org/43/ volume/dsdp43\_25.pdf}.

\label{end\stat}

\bibitem{3s} %32
\Au{Баженова Т.\,К., Шиманский В.\,К.}
Исследование онтогенеза углеводородных систем как основа раздельного прогноза 
нефте- и газонасыщенности осадочных бассейнов~// Нефтегазовая геология. Теория 
и практика, 2007. №\,2. {\sf http://www.ngtp.ru}.


%\bibitem{22s}
%Справочник по химическому составу и технологическим свойствам водорослей, 
%беспозвоночных и морских млекопитающих~/ Под ред. В.\,П.~Быкова.~--- М.: 
%ВНИРО, 1999.


%\bibitem{24s}
%Физико-химические свойства нефти. {\sf http://\linebreak forum.socionic.ru}.
 \end{thebibliography}}}%
\end{multicols}
     