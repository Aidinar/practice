\def\stat{ilushin}


\def\tit{ОРГАНИЗАЦИЯ УПРАВЛЯЕМОГО ДОСТУПА ПОЛЬЗОВАТЕЛЕЙ 
К~РАЗНОРОДНЫМ ВЕДОМСТВЕННЫМ ИНФОРМАЦИОННЫМ 
РЕСУРСАМ}
\def\titkol{Организация управляемого доступа пользователей 
к~разнородным ведомственным информационным 
ресурсам} 

\def\autkol{Г.\,Я.~Илюшин, И.\,А.~Соколов}
\def\aut{Г.\,Я.~Илюшин$^1$, И.\,А.~Соколов$^2$}

\titel{\tit}{\aut}{\autkol}{\titkol}

%{\renewcommand{\thefootnote}{\fnsymbol{footnote}}\footnotetext[1]
%{Работа выполнена
%при финансовой поддержке РФФИ, проекты 08-01-00567 и
%08-07-00152.}}

\renewcommand{\thefootnote}{\arabic{footnote}}
\footnotetext[1]{Институт проблем информатики Российской академии наук, ilushin@ipiran.ru}
\footnotetext[2]{Институт проблем информатики Российской академии наук, isokolov@ipiran.ru}


\Abst{Статья посвящена вопросам реализации интероперабельности приложений и управления доступом 
пользователей к хранилищам информации в условиях модернизации ин\-фра\-струк\-ту\-ры 
информационных технологий (ИТ) крупных ведомств. Для 
обеспечения направленной эволюции унаследованных автоматизированных
информационных сис\-тем (АИС) и хранилищ данных без остановки 
функционирования существующих систем предложена технология создания программно-технической 
инфраструктуры промежуточного слоя. Промежуточный слой, с использованием технологий 
сер\-вис\-но-ориен\-ти\-ро\-ван\-ной архитектуры и веб-сер\-ви\-сов, 
решает как задачи обеспечения интероперабельности разнородных приложений, так и задачи реализации 
централизованной модели управления доступом пользователей к разнородным хранилищам данных на основе 
формализованных ролей. Помимо задач интероперабельности, инфраструктура промежуточного слоя позволяет 
решить целый спектр задач обеспечения необходимого уровня информационной безопасности.}


\KW{интероперабельность; управление доступом; промежуточный слой; веб-сервисы; унаследованные системы; 
метаданные}

      \vskip 18pt plus 9pt minus 6pt

      \thispagestyle{headings}

      \begin{multicols}{2}

      \label{st\stat}

\section{Введение}

      Усиление роли и расширение финансовых возможностей государства в целом и 
федеральных органов власти в частности, существенно из\-ме\-нив\-шиеся требования 
законодательства, накопление позитивного опыта использования современных технологий 
интеграции данных в крупных пилотных проектах федерального и регионального уровня 
заставили по-новому взглянуть на ИТ-ин\-фра\-струк\-ту\-ру ключевых федеральных ведомств и 
поставить масштабные задачи коренного изменения положения дел в информатизации органов 
государственной власти. Большинство федеральных ведомств прошли непростой путь 
создания автоматизированных систем, располагают отлаженными и достаточно современными 
программно-техническими системами, имеют собственную разветвленную и довольно 
инерционную управленческую структуру. Существующие ведомственные системы 
создавались и развивались в течение десятилетий, от разрозненных функционально и/или 
регионально ориентированных подсистем к интегрированным системам. При этом больше 
внимания уделялось функциональной интеграции систем, чем информационной.
      
       Необходимость максимально возможного сохранения вложенных инвестиций, 
невозможность остановки действующих систем на длительный период, сложности внедрения 
из-за необходимости массового переобучения персонала и необходимость существенной 
модернизации информационного взаимодействия департаментов ведомства создают 
значительные трудности как в принятии\linebreak правильных управленческих решений руководством 
ведомства, так и в технической реализации крупных интеграционных проектов. И все же 
именно коренное изменение ранее существовавших подходов к созданию, модернизации и 
развитию ведомственных систем диктуется целым комплексом объективных причин. 
      \begin{enumerate}[1.]
\item Основа существующих технических решений построения информационных систем 
ведомств создавалась в начальный период централизации государственной власти, когда 
финансирование ИТ-про\-ек\-тов осуществлялось преимущественно из региональных бюджетов.
      
      В регионах-донорах создавалась собственная, не совместимая с другими регионами 
      ИТ-ин\-фра\-струк\-ту\-ра, проводилась собственная техническая политика, закупались и 
внедрялись технические и программные платформы (в том числе базовые операционные сис\-те\-мы, 
сис\-те\-мы управления базами данных (СУБД) и 
инструментальные средства), самостоятельно выбирались методы, стандарты и технологии 
интеграции. Разрабатываемые и на территориальном уровне внед\-ря\-емые собственные типовые 
решения в интересах подразделений ведомств, расположенных в регионе, интегрировались на 
основе инструментария мощных и дорогостоящих СУБД, поскольку региональные власти не 
особенно заботились о лицензионной чис\-то\-те, а следовательно, и стоимости используемого 
программного обеспечения (ПО). Координация множества региональных разработок в интересах ведомства на федеральном 
уровне (руководства ведомства) была затруднена, а зачастую невозможна.
      
      В~дотационных регионах отсутствие до\-ста\-точных средств на внедрение и обучение, 
не\-раз\-витость ИТ-ин\-фра\-струк\-ту\-ры и недостаток спе\-циалистов, владеющих современными 
компьютерными технологиями, привели к консервации технических решений 90-х годов 
прош\-ло\-го века, основанных чаще всего еще на\linebreak технологиях MS DOS.
      
      Значительный разрыв между регионами-до\-но\-ра\-ми и дотационными регионами (в 
      5--10~раз) наблюдается в настоящее время как в показателях развитости 
      ИТ-ин\-фра\-струк\-ту\-ры\linebreak
       (оснащенность учреждений и домохозяйств фиксированной и 
мобильной связью, компьютерами, высокоскоростным Интернетом, средствами 
автоматизации деятельности), так и в показателях уровня накопленного <<человеческого 
капитала>> (знаний и опыта населения в использовании современных ИТ)~[1].
\item За последние несколько лет произошла существенная реорганизация системы 
государственного управления, в то же время существенно изменились требования к качеству 
информации и одновременно к ее защите от несанкционированного доступа (НСД), появились 
законодательные ограничения по защите персональных данных, ужесточились требования к 
лицензированию ПО.
      
      По мере формирования новой управленче-\linebreak ской структуры, расширения возможностей 
централизованного финансирования, каждое ведомст\-во начинало создавать свою 
      ИТ-ин\-фраструкту\-ру и вертикаль информационных ресурсов (ИР), выбирая собственные 
технические и прог\-рам\-мные платформы реализации прикладных задач и мало заботясь о 
со\-вмес\-ти\-мости с техническими решениями других ведомств. В~то же время ужесточились 
требования федеральной власти к самим ведомствам по ответственности должностных лиц за 
актуальность и достоверность информации в их базах данных. 
{ %\looseness=1

}
      
      По мере внедрения новых информационных систем в связи с вводом в действие 
императивных норм и регламентов происходила постепенная техническая изоляция 
информационных систем и баз данных разных ведомств, внедрялись разные технические 
решения по организации разграничения доступа и защите данных от НСД. Интеграционные 
технические решения, созданные в крупных регионах, начали постепенно изживать себя, 
вступая в противоречие с регламентами и технической политикой отдельных ведомств.
      
      В рамках каждого ведомства начали формироваться свои требования по разграничению\linebreak 
доступа сотрудников к информации в соот\-ветствии с их должностными функциями. 
А~поскольку практически каждое ведомство нуж\-далось в информации из других ведомств,
налаживался обмен массивами информации\linebreak  между ними. На практике это привело к 
рас\-со\-гла\-со\-ва\-нию данных об одних и тех же\linebreak информационных объектах из-за временного 
разрыва в об\-нов\-ле\-нии информации, неправильной фильтрации данных при обмене массивами, 
из-за недостаточно продуманных и некачественно реализованных регламентов\linebreak 
взаимодействия, отсутствия надежных межведомственных информационных контуров 
обратной связи. Наиболее трудноразрешимые проблемы (как для сотрудников ведомств, так и 
для граждан) возникают в случаях, когда ошибки, возникшие в недрах одного ведомства, 
выявляются служащими другого ведомства в связи с предоставлением граж\-да\-на\-ми или 
организациями первичных документов. Стало понятно, что межведомственный обмен 
массивами данных в силу своих органических недостатков не может служить основой 
преодоления противоречий в информационных хранилищах ведомств и необходимо внедрение 
более сложных и гибких технологий обеспечения целостности баз данных в различных 
органах государственной власти.
\item Новые законодательно устанавливаемые требования к повышению качества 
обслуживания населения (идеология <<одного окна>>), отмена\linebreak
множества ведомственных 
инструкций по\linebreak различного рода <<откреплениям>> и <<прикреп\-ле\-ниям>> граждан в связи со 
сменой места жительства или пребывания, настоятельные требования высших 
государственных органов\linebreak
 власти по предоставлению гражданам различных сервисов 
<<электронного правительства>> привели к серьезным затруднениям, а в некоторых случаях 
даже к невозможности модер\-ни\-зации технических решений ведомственных сис\-тем без 
изменения основополагающих принципов их построения. 
\end{enumerate}

      Одновременно выявилась сложность модернизации информационных систем в 
условиях разрыва информационных связей между системами автоматизации различных 
ведомств и технологическая их несовместимость при острой не\-об\-хо\-ди\-мости совместного 
использования данных. В~условиях разнородности технических и программных платформ, 
несогласованной политики и не\-со\-вмес\-ти\-мости технических решений по разграничению 
доступа к данным на внутриведомственном, а часто и на межрегиональном уровне реализация 
межведомственного доступа к разнородным базам данных ведомств в реальном масштабе 
времени оказалась практически неосуществимой задачей. В~то же время интеграционные 
решения отдельных регионов пришли в полное противоречие с новыми требованиями 
законодательства.
      
      В настоящее время в большинстве федеральных ведомств начата работа по серьезной 
модернизации своих информационных систем и хранилищ данных. Эта работа ведется с 
учетом перераспределения функций и полномочий ведомств, изменения нормативной базы, 
необходимости межведомственного и международного взаимодействия, а также требований 
по взаимодействию каждого ведомства с гражданами и организациями. Наряду с работой по 
модернизации ведомственных сетей передачи данных, она ведется каждым ведомством в трех 
основных направлениях:
      \begin{enumerate}[(1)]
\item интеграция данных на региональном и федеральном уровне;
\item разработка и внедрение технических и программных средств защиты информации 
от НСД и систем управления доступом пользователей к информационным хранилищам;
\item разработка и внедрение унифицированных прикладных систем для всех уровней 
объектов управления с возможностью программного доступа к информационным 
хранилищам как своего, так и других ведомств (решение задач интероперабельности 
приложений и обеспечения регламентов информационного взаимодействия). 
\end{enumerate}

      Вопросам реализации задач интероперабель\-ности приложений и управления доступом 
пользователей к хранилищам информации посвящена данная статья.
      
       Современные теоретические подходы, практические методы интеграции 
неоднородных ИР и даже стандарты интероперабельности в 
настоящее время широко известны~[2--4]. Все они так или иначе опираются на идеологию 
создания промежуточного слоя прикладного ПО, которое берет на себя функции 
нивелирования технических решений разных автоматизированных систем путем реализации 
новых и поддержки некоторых старых API-ин\-тер\-фей\-сов (Application Programming Interface~--- интерфейс прикладного 
программирования).
      
      Термин <<промежуточный слой>>, исполь\-зу\-емый в настоящей статье, функционально 
более насыщен, чем традиционно используемые виды\linebreak
 промежуточного слоя, как например 
CORBA
 (Common Object Request Broker Architecture~--- архитектура посредника объектного
запроса), Web Services (веб-сер\-ви\-сы), SOA (Service-Oriented Architecture~--- 
сер\-вис\-но-ориен\-ти\-ро\-ван\-ная архитектура), 
Message-oriented middleware (промежуточный слой ПО, ориентированный на сообщение) 
и~пр. Так, наряду со средствами 
обеспечения технической ин\-тер\-опе\-ра\-бель\-ности, в предлагаемом промежуточном слое 
реализуются средства интегрированного доступа к разнородным информационным системам и 
хранилищам данных. При этом заметим, что веб-сер\-вис\-ный вариант промежуточного слоя 
является естественным выбором при нынешнем уровне технологического  
развития и что применение этой архитектуры промежуточного слоя решает проблему  
технической интероперабельности приложений, позволяя ведомству осуществлять 
планомерную и постепенную модернизацию унаследованных приложений наряду с 
разработкой новых.
      
      Выделим две значимые цели ведомства, оправдывающие создание промежуточного 
слоя:
      \begin{enumerate}[(1)]
\item обеспечение интероперабельности информационных систем на уровне приложений 
промежуточного слоя при совместной обработке данных о субъектах, событиях и объектах из\linebreak 
множества информационных хранилищ, включая комплексные и каскадные запросы сразу к 
нескольким разнородным информационным сис\-те\-мам одного или нескольких ведомств;
      \item перенаправление всех запросов сотрудников ведомства (и других ведомств) в 
единую территориально распределенную инфраструктуру аутентификации и управления 
доступом пользователей взамен ранее используемых прямых обращений к АИС и 
ИР, реализация необходимых контрольных (в том числе фискальных) 
функций.
      \end{enumerate}
      \pagebreak
      
      \end{multicols}

\begin{figure} %fig1
\vspace*{1pt}
\begin{center}
\mbox{%
\epsfxsize=95.847mm
\epsfbox{ily-1.eps}
}
\end{center}
\vspace*{-9pt}
\Caption{Типовая последовательность действий по обработке запроса
\label{f1il}}
\vspace*{3pt}
\end{figure}

\begin{multicols}{2}
      
      Основные функции ПО промежуточного слоя:
      \begin{itemize}
\item поддержка специально для этой цели разработанных детальных API-ин\-тер\-фей\-сов 
взаимодействия приложений;\\[-14pt]
\item  осуществление в реальном масштабе времени необходимых преобразований 
семантически однотипных данных, по-разному пред\-став\-лен\-ных в структурах хранения 
интегрируемых систем;\\[-14pt]
\item  централизованное управление доступом пользователей к приложениям, контроль 
взаимодействия приложений между собой с целью обеспечения надежной защиты 
ИР от~НСД.
\end{itemize}


      На рис.~\ref{f1il} показана типовая последовательность действий, выполняемых 
компонентами промежуточного слоя при обработке заявок приложений на поиск информации 
в ведомственных ИР. 
      
      Все крупнейшие производители технического и программного обеспечения 
декларируют полноценную реализацию современных стандартов интеграции, но каждый~--- 
по-своему. Поэтому декларируемая совместимость промышленного ПО на уровне стандартов на 
практике вовсе не означает\linebreak технической совместимости ПО разных производителей. Во 
всяком случае, в процессе разработки программных средств обеспечения 
ин\-тер\-опе\-ра\-бельности систем, построенных на разных\linebreak
программных платформах, требуется не 
только знание многих нюансов реализации стандартов каж\-дым производителем, но порой и 
знание недокументированных особенностей всех программных платформ. 

Создание команды 
разработчиков, обладающих столь обширными знаниями, трудноосуществимо. Вынужденное 
применение метода проб и ошибок приводит к существенным временным задержкам в 
реализации интеграционных проектов и непредусмотренным дополнительным финансовым 
издержкам.
      
       Крупные производители промышленного ПО (операционных систем, СУБД, 
инструментальных средств) оказывают существенную поддержку разработчикам 
ведомственных федеральных проектов во имя грядущей финансовой выгоды, устраняют 
выявленные ошибки и дорабатывают свое ПО с учетом опыта его внедрения в крупных 
ведомствах. В~случае же использования в ведомственных проектах свободно 
распространяемого ПО получить информацию о технических нюансах этого ПО 
затруднительно, а оказать влияние на скорейшее устранение выявленных ошибок невозможно.
      
      Из сказанного можно сделать следующие вы\-воды:
      \begin{enumerate}[(1)]
\item модернизация систем информационного обеспечения ведомства должна осуществляться 
с учетом необходимости взаимодействия с аналогичными системами других ведомств. 
Обеспечение интероперабельности разнородных систем целесообразно с использованием 
программно-технической инфраструктуры промежуточного слоя;
\item правильный выбор программных платформ, качество проектирования 
архитектуры и уровень технических решений программного обеспечения 
промежуточного слоя имеет далеко идущие последствия, поскольку недостатки 
технического решения выявятся только на\linebreak этапе последующей модернизации 
существующих и создания новых информационных сис\-тем и баз данных ведомства. 
\end{enumerate}
      
      Институт проблем информатики Российской академии наук 
      накопил значительный опыт практи\-ческой разработки и внедрения средств 
обеспечения интероперабельности разнородных информационных систем на основе 
инфраструктуры промежуточного слоя. Этот опыт может быть с успехом использован как для 
разработки систем информационной интеграции разнородных (в том числе унаследованных) 
ИР крупных ведомств, так и для обеспечения межведомственного 
информационного взаимодействия.
      
      В последующих разделах рассмотрены следующие практические аспекты 
создания и по\-стро\-ения инфраструктуры промежуточного слоя крупного ведомства:
      \begin{itemize}
\item задачи и этапы построения инфраструктуры интероперабельности, назначение и 
состав метаданных промежуточного слоя, общая схема фильтрации на уровне запросов и 
результатов их выполнения при обращении пользователей к разнородным 
ИР (разд.~2);
\item основные типы базовых (технологических) сервисов промежуточного слоя (разд.~3);
\item методы организации централизованно управ\-ля\-емо\-го доступа пользователей к ИР на 
основе ролей (разд.~4).
\end{itemize}

\section{Выбор инфраструктуры промежуточного слоя}

\subsection{Назначение промежуточного слоя}

      При выборе принципов построения (модернизации) крупной 
      тер\-ри\-то\-ри\-аль\-но рас\-пре\-де\-лен\-ной ведомственной системы особенно важным является решение по степени 
централизации хранения и обработки данных. Наиболее эффективной с точки зрения простоты 
интеграции, эко\-но\-мич\-ности обслуживания и модернизации представляется структура на 
основе центров обработки данных (ЦОД). Однако принятию решения о полной реконструкции 
информационной инфраструктуры и объединении множества ИР 
крупного ведомства в единое хранилище данных препятствует множество объективных 
факторов:
      \begin{itemize}
\item сложившаяся в ведомстве распределенная структура сотен региональных отраслевых 
(интегрированных) и специализированных баз данных, а также тысяч поддерживающих их 
АИС, десятков тысяч 
специализированных автоматизированных рабочих мест (АРМ);
\item накопленный в течение многих лет опыт и нормативная база формирования и 
использования ИР ведомства (включая специализированные базы данных);
\item обоснованное нежелание департаментов (ведомств) передавать функции ведения 
своих ИР от собственных подразделений специально созданному 
подразделению без передачи ответственности за их качество;
\item несовершенная и неоднородная по своему качеству сеть передачи данных, не 
обес\-пе\-чи\-ва\-ющая высокоскоростной доступ всех пользователей к удаленным 
ИР со своих рабочих мест.
\end{itemize}

      Одномоментное существенное изменение сложившейся ИТ-ин\-фра\-струк\-ту\-ры ведомства 
на практике чаще всего оказывается неприемлемым по целому ряду причин:
      \begin{itemize}
\item значительный риск потери на длительный переходный период необходимого уровня 
качества существующих разнородных ИР и их до\-ступ\-ности;
\item замораживание процессов модернизации функций существующих ИР на весь 
переходный период, что может препятствовать вводу в действие новых регламентов и 
сервисов электронного правительства;
\item необходимость крупных инвестиций в на\-уч\-но-ис\-сле\-до\-ва\-тель\-ские и опыт\-но-кон\-ст\-рук\-тор\-ские
работы, апробирование и внедрение новых сис\-тем, переобучение персонала;
\item необходимость пересмотра нормативной и правовой базы, сопротивление 
радикальным изменениям со стороны сотрудников ведомства.
\end{itemize}
      
      По этим и ряду других причин революционный метод модернизации (одномоментная 
перестройка ИТ-ин\-фра\-струк\-ту\-ры, создание абсолютно новых АИС взамен существующих, 
полное переобучение персонала) неприемлем для крупных ведомств.
      
      Вместо революционных методов модернизации, сулящих в отдаленном будущем 
значительные экономические выгоды и более высокую степень адап\-тив\-ности, руководство 
крупных ведомств осознанно делает выбор в пользу более дорогостоящих мето\-дов 
направленной эволюции ИТ-ин\-фра\-струк\-ту\-ры и АИС при максимально возможном 
использовании вложенных инвестиций. Достоинством такого подхода является возможность 
обеспечения непрерывности функционирования множества действующих АИС, постепенная 
их модернизация, разработка новых АИС, опирающихся уже на новую инфраструктуру и 
      API-ин\-тер\-фей\-сы, постепенная реструктуризация баз данных, поэтапное переобучение 
персонала. Внедрение новых систем может осуществляться поэтапно, с корректировкой 
выявленных технических и организационных проблем. Например, в период внедрения новой 
информационной инфраструктуры в одних регионах другие регионы могут пользоваться 
старыми отработанными решениями. Такую направленную эволюцию ИТ-ин\-фра\-струк\-ту\-ры и 
АИС предлагается реализовать на основе концепции промежуточного слоя.
      
      Основная идея создания промежуточного слоя состоит в том, что прямое 
взаимодействие АРМ (клиентских приложений) с существующими АИС (серверными 
приложениями) постепенно исключается. На плановой основе осуществляется модернизация 
унаследованных и разработка новых АИС, которые взаимодействуют с другими 
приложениями только через API-ин\-тер\-фей\-сы промежуточного слоя. Взаимодействие же 
приложений с пользователями разрешается только с использованием новой централизованной 
инфраструктуры идентификации и управления правами пользователей. Ядром этой 
инфраструктуры являются сис\-те\-ма удостоверяющих центров (СУЦ), Active Directory и база данных 
пользователей, содержащая для каждого пользователя все его детализированные права 
доступа к базам данных, управляемых АИС, модернизированными в соответствии с новыми 
требованиями. На рис.~\ref{f2il} схематично представлено изменение информационных 
потоков пользователей в связи с созданием промежуточного слоя и постепенной 
модернизацией существующих клиентских и серверных приложений.
      


      До создания промежуточного слоя (пунктирные линии на рис.~\ref{f2il}) каждый 
сотрудник ведомства, нуждающийся в доступе к нескольким ИР (базам данных 
территориального, регионального и федерального уровня), вынужден был работать с 
несколькими АИС, каждая из которых имела свою программную платформу и свой 
пользовательский интерфейс. Администратор каждой АИС прописывал информацию о 
пользователе и его правах доступа к информационным объектам ИР в соответствующей 
административной базе данных. В~итоге пользователь получал множество Login (по числу ИР) 
и множество интерактивных интерфейсов взаимодействия с ИР (по числу типовых АИС).
      
      В новых условиях сотрудник ведомства должен получить техническое средство 
персональной идентификации и электронное удостоверение, содержащее цифровой 
сертификат. Права доступа всех\linebreak пользователей ко всем информационным объектам всех ИР, в 
соответствии со служебными обязанностями пользователей, сосредоточиваются в единой\linebreak
 базе 
данных прав доступа пользователей (ПДП). В~последующем, по мере разработки и внедрения новых 
типовых АИС, исчезнет необходимость изуче\-ния пользователем интерактивных инструментов 
множества АИС. В~конечном итоге каждый пользователь через свой специализированный 
АРМ (приложение, автоматизирующее процесс выполнения должностных обязанностей) 
сможет получить единый интерактивный интерфейс доступа и обработки данных, получаемых 
из множества федеральных, региональных и территориальных ИР.
      
       Создание промежуточного слоя в условиях ранее сложившейся территориально 
распределенной неоднородной ИТ-ин\-фра\-струк\-ту\-ры ведомства\linebreak
приводит к необходимости 
разработки и внедрения новой вспомогательной инфраструктуры~--- совокупности 
взаимосвязанных территориально\linebreak
 распределенных программно-технических комплексов 
промежуточного слоя. Комплексы будут размещены в подразделениях и информационных\linebreak\vspace*{-12pt}
\pagebreak

\end{multicols}

\begin{figure} %fig2
\vspace*{1pt}
\begin{center}
\mbox{%
\epsfxsize=161.835mm
\epsfbox{ily-2.eps}
}
\end{center}
\vspace*{-3pt}
\Caption{Изменение информационных потоков (запросов и данных) в результате внедрения 
промежуточного слоя: АРМ~--- автоматизированное рабочее место; АИС~--- автоматизированная 
информационная система; ПТК~--- программно-технический комплекс; ИР~--- информационный 
ресурс ведомства; БД~--- база данных в составе информационного ресурса
\label{f2il}}
\vspace*{6pt}
\end{figure}

\begin{multicols}{2}

\noindent
центрах ведомства. Эти комплексы в по\-сле\-ду\-ющем станут основой (стержнем) 
горизонтальной и вертикальной информационной интеграции, модернизации и последующего 
развития всех АИС ведомства на основе функциональной и программной стандартизации. 
Программно-техническая среда промежуточного слоя возьмет на себя функции 
централизованного управления доступом пользователей к хранилищам информации, защиты 
информации от НСД, эффективного противодействия потенциальным угрозам и конкретным 
нарушениям информационной безопасности, мониторинга программно-тех\-ни\-че\-ских 
комплексов АИС, об\-нов\-ле\-ния ПО и некоторые другие важные функции.

\subsection{Задачи и этапы разработки компонентов промежуточного слоя}

      Модернизация ИТ-ин\-фра\-струк\-ту\-ры ведомства в направлении централизации должна 
предусмат\-ри\-вать \textit{модернизацию коммуникационной инфраструктуры}~[5]. 
Модернизация цифровых коммуникаций (транспортной сети ведомства) должна включать как 
создание соответствующих центров управления транспортной сетью ведомства, так и развитие 
этой транспортной сети до уровня первичных подразделений, традиционно использовавших 
открытые сети региональных провайдеров.
      
      Для обеспечения возможности построения надежной системы управления доступом 
пользователей к хранилищам информации и внедрения ведомственной системы электронного 
документооборота (СЭД) необходимо \textit{создание инфраструктуры удостоверяющих 
центров}, сертифицированной соответствующими уполномоченными государственными 
учреждениями. Помимо этого, для надежной аутентификации пользователей необходимо 
тотальное использование средств персональной идентификации.
      
      Фундаментом последующей интеграции неоднородных (представленных в разных 
моделях) ИР на уровне разнообразных приложений является 
\textit{создание единой информационной модели данных ведомства} (на первых порах хотя бы 
ядра такой единой\linebreak\vspace*{-12pt}
\pagebreak
\end{multicols}

\begin{figure} %fig3
\vspace*{1pt}
\begin{center}
\mbox{%
\epsfxsize=147.198mm
\epsfbox{ily-3.eps}
}
\end{center}
\vspace*{-3pt}
\Caption{Схема фильтрации доступа пользователей к запросам, информационным ресурсам, их частям 
и отдельным полям (группам полей)
\label{f3il}}
\vspace*{6pt}
\end{figure}

\begin{multicols}{2}

\noindent
 модели), а также \textit{единой системы классификации и кодирования 
информации (ЕСКК)}. Для обеспечения
 реализации заранее не детерминированных запросов 
необходима также разработка языковой основы средств манипулирования данными в форме 
\textit{базовых конструкций языка манипулирования данными (ЯМД)}. Созданию единой 
модели данных должно сопутствовать создание совокупности отображений этой модели на 
информационные модели всех унас\-ле\-до\-ван\-ных АИС и их хранилищ данных, а если таковые 
существуют, то и на модели данных других ведомств, с которыми необходимо осуществлять 
информационное взаимодействие. 
      
      По мере модернизации унаследованных и разработки новых АИС и баз данных 
координаты приложений, взаимодействующих с инфраструктурой промежуточного слоя, и 
управляемых ими баз\linebreak
 данных должны размещаться в хранилище метаданных~--- 
\textit{центральном репозитории} (реестре) \textit{метаданных (ЦРМ)}, образуя в конечном 
итоге \textit{информационное хранилище ведомства}~--- совокупность ИР и собственно ЦРМ. 
В частности, в ЦРМ ведомства целесообразно хранить единую схему данных или 
совокупность схем данных всех типов ИР, запросы к которым будут осуществляться с 
использованием ПО промежуточного слоя. 
      
      Помимо этого, в ЦРМ могут храниться и специфические структуры, отражающие 
особенности реализации промежуточного слоя ведомства. К~специфическим структурам 
можно отнести пред\-став\-ле\-ния данных при их визуализации, описания учетных документов и 
их структуры во взаимосвязи с особенностями политики управления доступом, описания 
сценариев поиска данных в подмножествах хранилища данных (в ИР и 
их составных частях~--- базах данных). Для последующей организации управления доступом 
пользователей к ИР важно, чтобы в схеме единой модели данных (ЕМД) были выделены классы, подклассы и 
составные классы, представленные во всей совокупности ИР ведомства, для которых могут 
быть установлены ограничения доступа (рис.~\ref{f3il})
      

      
      Разработка корректной полной и всеобъемлющей схемы данных ведомства, 
переписывание тысяч разнообразных запросов и отчетов~--- трудоемкая задача, требующая 
детальной формальной и содержательной проработки, длительного времени и значительного 
финансирования~[6]. Сложность разработки обусловлена тем, что как в самом ведомстве, так 
и в смежных по функциям ведомствах помимо сотен отраслевых баз данных федерального и 
регионального уровня существуют также сотни специализированных АИС федерального, 
регионального и территориального уровня. И~все они имеют собственные модели данных, 
используют разные языки описания и манипулирования данными, опираются на разные 
программные платформы.
      
      Обычно информационные хранилища ведомства содержат преимущественно 
структурированную информацию, и первостепенной утилитарной задачей создания ядра 
модели данных ведомства и ее отображений на модели данных унаследованных систем 
является унификация представления семантически эквивалентных объектных типов данных, 
по-разному представленных в структурах хранения множества разнородных баз данных. 

      
      В таких случаях, в особенности на начальных этапах работы по формированию ядра 
единой модели данных ведомства, целесообразно анализировать лишь те информационные 
объекты и связи между ними, которые присутствуют хотя бы в двух разнородных ИР и 
должны совместно обрабатываться. Поскольку каждая модернизируемая или вновь 
создаваемая АИС (приложение) ведомства формирует возможность доступа к конечному 
подмножеству баз данных в виде вполне определенного конечного перечня запросов, доступ к 
базам данных разнородных АИС из множества АРМ также может быть ограничен конечным 
(пусть и расширяемым) перечнем запросов. 
      
      Требования к \textit{системе централизованного управ\-ле\-ния доступом пользователей} 
к разнородным ИР должны включать не только требования по обеспечению разрешения 
(запрета) на доступ к конкретным базам данных этих ИР для каждой категории пользователей, 
но также ограничения доступа к отдельным информационным разделам и даже отдельным 
полям этих баз данных. С~организационной точки зрения должно быть проведено детальное 
изучение информационных потреб\-ностей\linebreak
основных категорий пользователей. Перечень 
категорий пользователей может соответствовать су\-щест\-ву\-ющей структуре управления 
ведомства\linebreak 
(например, управлениям и департаментам). В~соответствии с должностными 
обязанностями и информационными потребностями сотрудников внутри каж\-дой категории 
пользователей должны быть выделены функциональные \textit{роли}, определяющие права 
доступа каждой группы пользователей к содержимому каждого из существующих типов 
ИР.

      
      Интеграция отраслевых и специализированных АИС на уровне приложений имеет 
целый ряд особенностей. Задача их интеграции особенно усложняется в тех случаях, когда 
ведомство опирается на деятельность не только собственных подразделений, но и 
независимых поставщиков сервисов и услуг. Это характерно для ведомств, ориентированных 
на предоставление услуг населению (здравоохранение, образование, социальная защита, 
электронный бизнес). В~таких случаях общее число объектов информатизации даже на 
региональном уровне достигает десятков тысяч, а по стране~--- сотен тысяч, при этом число 
технических решений\linebreak
АИС и ИР также исчисляется тысячами. Включение такого числа 
разнородных объектов в информационное пространство ведомства~--- нетривиальная 
техническая и управленческая задача.


Некоторые подходы к интеграции множества 
разнородных социально-ориентированных ведомственных хранилищ данных и 
специализированных АИС на основе промежуточного слоя предложены в работах~[7--9]. Но 
эти подходы и методы нацелены на решение лишь части комплекса задач интеграции 
множества независимых баз данных и АИС. 

Кроме того, каждое крупное ведомство имеет 
уникальную исходную ИТ-ин\-фра\-струк\-ту\-ру, и надо весьма осторожно относиться к переносу 
опыта интеграции одних ведомств и регионов на другие ведомства и регионы. 

\bigskip
      
      Выводы:
      \begin{enumerate}[(1)]
\item для крупного ведомства разработка единой информационной модели сводится к 
разработке описания и единой логической модели данных и реализации расширяемого 
перечня запросов и отчетов на основе единого языка (или совокупности языков) 
манипулирования данными;
\item в качестве первого шага в решении этой задачи целесообразно создание ядра 
единой модели (например, в форме базового набора типов объектов и связей между 
ними) схемы данных и совокупности отображений этой схемы данных на множество 
схем данных всех унаследованных и проектируемых ИР. Кроме того, необходимо 
выбрать базовый ЯМД, но на начальных стадиях 
проектирования и ввода определить совокупность базовых конструкций и 
ограничений ЯМД; 
\item разработка требований к централизованной системе управления доступом сводится к 
выделению основных категорий пользователей и определению ролей доступа к содержимому 
всех типов ИР внутри каждой категории.
\end{enumerate}


\section{Специфика приложений ведомства и~состав веб-сервисов 
промежуточного слоя }

\subsection{Основные виды приложений}

      Обычно использование архитектуры, предложенной консорциумом W3C (World Wide Web Consortium)~\cite{2il}, 
ассоциируется с организацией доступа к ИР на основе портальных решений, а основным 
инструментом доступа со стороны пользователя представляется <<тонкий клиент>> (браузер). 
Портальные решения для поиска данных в базах\linebreak
 данных ведомства и предоставления 
различных сервисов~--- естественный компонент создаваемой новой ИТ-ин\-фра\-струк\-ту\-ры и 
комплекса приложений ведомства. Портальные решения удобны для реализации чисто 
информационных запросов к информационным хранилищам, а также являются пока что 
безальтернативным инструментом для предоставления ведомством услуг населению 
(сервисов).\linebreak
 Правда, в отличие от сотрудников ведомства, надежная аутентификация граждан в 
настоящее время невозможна из-за отсутствия единого национального стандарта средств 
персональной\linebreak
 идентификации граждан (электронных удостоверений) и единой 
государственной инфраструктуры поддержки электронных подписей (удосто\-ве\-ря\-ющих 
центров), что ограничивает возможности предос\-тав\-ле\-ния гражданам всего комплекса 
необходимых сервисов со стороны ведомства. 
      
      При всей важности обеспечения чисто информационных запросов пользователей к ИР 
ведомства основной задачей внедрения компьютерных технологий все же является 
автоматизация непосредственной деятельности сотрудников ведомства при решении ими 
служебных задач. В~условиях поэтапного внедрения в рамках ведомства единой схемы 
данных и единой системы классификации и кодирования информации (совокупности 
кодификаторов и справочников) необходимо реализовать как контур доступа к 
ИР через промежуточный слой, так и контур формирования и 
обновления этих ИР.
      
      Значительная часть решаемых задач требует использования специализированных 
устройств и программных инструментов, что исключает применение <<тонкого клиента>> для 
этих целей. Во многих случаях к отдельным типам приложений предъявляются жесткие 
требования в части ре\-ак\-тив\-ности. 
      
      Это характерно для приложений, автоматизирующих процесс предоставления 
сотрудникам ведомства или населению жестко регламентированных услуг. Для повышения 
реактивности таких приложений широко используются локальные классификаторы и 
внутренние базы данных приложений, представление которых в единой схеме данных 
ведомства нецелесообразно.
      
      Существуют также приложения, использующие специфические форматы данных и 
специализированные драйверы (например, при работе с устройствами обработки 
биометрической информации, лабораторным оборудованием и~т.\,п.). 
      
      В связи с этим многие специализированные АИС территориальных подразделений 
ведомства целесообразно реализовывать в виде клиент-сер\-вер\-ных решений, причем основная 
часть необходимой информации, включая все необходимые классификаторы, должна 
располагаться непосредственно на серверах локальной сети подразделения. Получение же 
информации из других (внешних) баз данных должно осуществляться через промежуточный 
слой в интерактивном или пакетном режиме с использованием программных интерфейсов и 
базы данных запросов при обеспечении надежной аутентификации и идентификации как 
пользователей запросов, так и собственно приложений. 
      
      Практически в любом ведомстве существует класс задач, для решения которых удобно 
использовать не интерактивный, а пакетный (отложенный) режим обработки запросов к базам 
данных. Отложенный режим характерен для решения большинства задач электронного 
документооборота и значительной части задач по формированию ИР (ввод новых данных), 
обновления всевозможных классификаторов, формирования и получения различных справок и 
отчетов, при выполнении запросов, допускающих получение информации из базы данных в 
течение нескольких часов или минут, а не секунд. Пакетный режим обработки предпочтителен 
(вне зависимости от типа запросов) в случае использования коммутируемых и 
низкоскоростных линий связи.
      
      С учетом вышесказанного технические решения промежуточного слоя должны 
обеспечить доступ к базам данных со стороны систем автоматизации подразделений 
ведомства на уровне\linebreak
программных интерфейсов, включая поддержку интерфейсов обращения 
к АИС как в режиме реального времени, так и с использованием технологий электронной 
почты.

\begin{table*}\small
\begin{center}
\Caption{Пример набора технологических сервисов промежуточного слоя
\label{t1il}}
\vspace*{2ex}

\begin{tabular}{|p{51mm}|p{51mm}|p{51mm}|}
\hline
\multicolumn{1}{|c|}{
\tabcolsep=0pt\begin{tabular}{c}Сервисы\\ манипулирования данными\\
(для компонентов АИС,\\ взаимодействующих\\ с базами 
данных ИР)\end{tabular}}&
\multicolumn{1}{|c|}{
\tabcolsep=0pt\begin{tabular}{c}Сервисы доступа к базам данных\\ промежуточного слоя\\ (для компонентов АИС,\\ 
взаимодействующих\\ с промежуточным слоем)\end{tabular}}&
\multicolumn{1}{|c|}{
\tabcolsep=0pt\begin{tabular}{c}Административные сервисы\\
(для административных\\ компонентов\\ промежуточного слоя)\end{tabular}} \\
\hline
\multicolumn{1}{|l|}{\raisebox{-4pt}[0pt][0pt]{Реализация запросов к ИР}}&Доступ к схемам данных ИР и 
классификаторам&\multicolumn{1}{|l|}{\raisebox{-4pt}[0pt][0pt]{Корректировка ЕМД, ЕСКК, ЦРМ}}\\
\hline
Обновление данных в базах данных ИР&Доступ к описаниям типовых запросов&Корректировка баз 
данных управления доступом пользователей\\
\hline
Обмен информационными массивами (загрузка/выгрузка)&Получение прав доступа к ИР для 
текущего пользователя&Корректировка прав доступа административного персонала\\
\hline
\end{tabular}
\end{center}
\end{table*}

\subsection{Базовые веб-сервисы промежуточного слоя}

      После развертывания программно-технической инфраструктуры промежуточного слоя 
и модернизации унаследованных АИС новая ИТ-ин\-фра\-струк\-ту\-ра ведомства может быть 
представлена как совокупность приложений, реализующих заданные прикладные функции с 
использованием новых API-ин\-тер\-фей\-сов и удовлетворяющих требованиям информационной 
безопасности ведомства. Между приложениями и промежуточным слоем определяются 
стандартизованные интерфейсы, реализуемые в форме веб-сер\-ви\-сов. Для определения состава 
этих веб-сер\-ви\-сов исходными данными являются решения по архитектуре и структуре 
промежуточного слоя, а также требования ведомства к основным функциям промежуточного 
слоя. 
      
      Поскольку в основу интероперабельности разнородных АИС положена 
      сервис-ориентированная архитектура, все взаимодействие приложений 
специализированных АРМ различного назначения с АИС осуществляется не напрямую, а 
исключительно через специальную инфраструктуру промежуточного слоя.
      
      Детальный анализ требований по взаимодействию со стороны приложений ведомства 
позволяет сформулировать перечень веб-сер\-ви\-сов промежуточного слоя. Для каждого 
ведомства перечень этих сервисов может разниться. Однако можно выделить перечень 
сервисов, которые должны быть реализованы в любом случае, поскольку они носят не 
содержательный, а технологический характер. Пример такого набора технологических 
сервисов приведен в табл.~\ref{t1il}. 
       
       

      
      Взаимодействие узлов промежуточного слоя с АРМ/АИС заключается в приеме 
информации (запросов и данных для обновления БД) от АРМ/АИС и в передаче им 
результатов исполнения запросов и информации о ходе отработки запросов и об\-нов\-ле\-ния 
данных в базах данных. Реализация указанных видов взаимодействия выполняется с помощью 
интерактивных и пакетных сервисов (сервисы взаимодействия с АРМ/АИС по отработке 
запросов и сервисы взаимодействия с АРМ/АИС по вводу данных с \mbox{целью} обновления ИР). 
      
      К основным способам хранения документов, описаний метаданных и приложений в 
ведомственных информационных системах можно отнести:
      \begin{itemize}
\item документы на языках HTML\footnote{HyperText Markup Language~--- язык разметки гипертекста.},
XML\footnote{EXtensible Markup Language~--- расширяемый
язык разметки.}, а также файлы (текстовые, в форматах Word, 
Excel, PowerPoint, Flash, Acrobat, графические, звуковые и видео файлы, файлы 
биометрической информации и различные файлы специализированного формата, такие 
как DICOM\footnote{Digital Imaging and COmmunications in Medicine~--- индустриальный стандарт создания,
хранения, передачи и визуализации медицинских изображений и документов
обследованных пациентов.});
\item базы данных и процедуры SQL\footnote{Structured Query Language~--- язык структурированных
запросов.}, JAVA-аплеты, скрипты и~т.\,п.
\end{itemize}

      Визуализация данных должна осуществляться приложениями на устройства различного 
типа:
      \begin{itemize}
\item мониторы настольных и портативных компьютеров (различной разрешающей 
способности);
\item карманные персональные компьютеры (PDA~--- Personal Digital Assistant));
\item сенсорные мониторы для точек публичного доступа;
\item другие устройства, прежде всего различные мобильные (гибридные) устройства, 
объеди\-ня\-ющие в едином корпусе мобильный телефон, сканер, видеокамеру и PDA.
\end{itemize}
      
      Для каждого типа устройств, принятого на вооружение в ведомстве, необходимо 
обеспечить представление информации в необходимом формате и размере. При этом часто 
возникает проб\-ле\-ма визуализации одного и того же документа в различных форматах, 
необходимость хранения нескольких экземпляров одного и того же документа и даже 
реализация нескольких комплектов одних и тех же запросов для различных типов устройств.
      
      Выводы:
      \begin{enumerate}[(1)]
\item перечень веб-сер\-ви\-сов промежуточного слоя специфичен для каждого ведомства, 
поскольку должен учитывать особенности предметной области и технологий представления и 
обработки данных. Представление данных как в информационных хранилищах, так и на 
оконечных устройствах пользователей для некоторых ведомств могут иметь значительную 
специфику;
\item можно выделить совокупность веб-сер\-ви\-сов, не связанных со спецификой предметной 
об\-ласти. Это сервисы, реализующие техноло\-гические функции, связанные с аутен\-ти\-фикацией 
и определением ПДП, поиском и обнов\-ле\-нием данных, доступа к 
метаданным и классификаторам. В~отдельный блок можно выделить сервисы доступа к 
административным базам данных и сервисы, необходимые для прикладного уровня 
управления функционированием.
\end{enumerate}

 \begin{figure*} %fig4
 \vspace*{1pt}
\begin{center}
\mbox{%
\epsfxsize=85.041mm
\epsfbox{ily-4.eps}
}
\end{center}
\vspace*{-3pt}
 \Caption{Упрощенная схема взаимодействия приложения с базами данных ЕСК, УЦ и ПДП: ЕСК~--- 
Единая система каталогов, содержит учетные записи пользователей; УЦ~--- удостоверяющий центр, 
содержит сертификаты (электронные подписи) пользователей; ПДП~--- права доступа пользователей;
ПСИ~--- персональное средство 
идентификации, съемный носитель, содержащий сертификат пользователя; SID (Security IDentifier)~--- 
идентификатор безопасности
 \label{f4il}}
 \end{figure*}


\section{Доступ пользователей к~информационным ресурсам и~управление доступом 
на~основе ролей}
      
      Методы и технологии реализации цент\-ра\-ли\-зованного доступа пользователей к 
разнородным базам данных ведомства могут различаться. Значительное влияние на 
технологии реализации оказывает выбор базовой программной платформы промежуточного 
слоя, а также программные платформы ключевых унаследованных АИС ведомства. Ниже 
описываются основные структуры хране-\linebreak ния
и методы реализации системы управления\linebreak 
доступом пользователей к разнородным АИС ведомства, реализованные в рамках нескольких 
на\-уч\-но-исследовательских и опытно-кон\-струк\-тор\-ских 
работ, выполненных в ИПИ РАН.

      
      Основными структурами хранения информации о пользователях различных 
приложений ведомства являются следующие:
      \begin{itemize}
\item база данных цифровых сертификатов (открытых и закрытых ключей и их 
владельцев)~--- хранится только в центрах регистрации, не доступных для программного 
доступа со стороны любых приложений на сетевом уровне;
\item базы данных действующих и отозванных циф\-ро\-вых сертификатов (открытых 
ключей) сотрудников ведомства, доступные приложениям, но не включающие 
информации о\linebreak
 владельцах,~--- хранятся в центрах сертификации;
\item единая система каталогов пользователей (учетные записи всех пользователей);
\item база данных прав доступа всех категорий пользователей, включая все категории 
об\-слу\-жи\-ва\-юще\-го персонала;
\item база данных пользователей и принадлежность этих пользователей к категориям 
доступа.
\end{itemize}
      
      Важной вспомогательной структурой, используемой при управлении доступом 
пользователей к ИР, является база данных перечня типовых запросов 
(ПТЗ).
      
      Если первые три структуры хранения информации о пользователях обязаны 
присутствовать в составе ИТ-ин\-фра\-струк\-ту\-ры ведомства независимо от необходимости 
создания централизованной\linebreak
 сис\-те\-мы управления доступом пользователей к множеству 
ИР, то две последние структуры хранения необходимы именно для 
реализации централизованного управления доступом пользователей к базам данных 
разнородных ИР и электронного документооборота. Все 
вышеперечисленные административные базы данных должны быть корректны и увязаны 
между собой, что является одной из важных задач обслу\-жи\-ва\-юще\-го персонала. 
На~рис.~\ref{f4il} приведена упрощенная схема информационного взаимодействия 
приложения АРМ пользователя с базами данных Active Directory (осуществляется системными 
средствами операционной системы) и ПДП к выполнению конкретного запроса или 
доступа к конкретному ИР (осуществляется с использованием 
      веб-сер\-ви\-са системы управления доступом). В~качестве персонального средства 
идентификации (ПСИ) могут использоваться любые съемные носители (магнитные карты, дискеты, 
флеш-носители).
 
      
      Остановимся несколько подробнее на специфике этих структур хранения информации 
о пользователях. 
      
      \textit{Система удостоверяющих центров} реализуется как совокупность 
комплекса прог\-рам\-мно-тех\-ни\-че\-ских средств центров регистрации и цент\-ров сертификации, а 
также орга\-ни\-за\-ци\-он\-но-тех\-ни\-че\-ских мероприятий, необходимых для использования 
криптографических функций в целях защиты информации от НСД, аутентификации 
пользователей единого информационного пространства 
и разграничения их прав доступа, подтверждения авторства и обеспечения 
целостности и подлинности электронных документов.
      
      Корневой удостоверяющий центр является вершиной древовидной структуры иерархии 
доверия в СУЦ и является вышестоящим по отношению к подчиненным УЦ. Подчиненные 
УЦ ведомства целесообразно располагать в информационных центрах ведомства, например в 
субъектах федерации. Основной задачей УЦ является выпуск сертификатов открытого ключа 
для пользователей ведомства. Выпуск сертификатов, все действия по управлению 
сертификатами на основании заявок, публикация сертификатов пользователей и списка 
отозванных сертификатов УЦ осуществляются в автоматизированном режиме. Система удостоверяющих центров ведомства 
может взаимодействовать с СУЦ других органов государственной власти.
      
      \textit{Единая система каталогов} пользователей ЕСК (LDAP\footnote{Lightweight Directory Access Protocol~---
      облегченный протокол доступа к каталогам.} 
      или Active Directory)~--- 
общепринятая для всех программных платформ основа идентификации пользователей на 
прикладном уровне. Детальный анализ назначения, функций и методов\linebreak проектирования и 
использования ЕСК в ведомственных системах выходит за рамки данной публикации. Здесь 
важно только отметить, что учетные записи пользователей должны содержать как минимум 
сис\-тем\-ные идентификаторы безопасности (SID) всех пользователей, а также поля для связи с 
базой данных пользователей.
      
      \textit{База данных ПДП} является основой управления 
доступом пользователей на всех уровнях обработки запросов 
(приложение\;$\rightarrow$\;промежуточный слой\;$\rightarrow$\;АИС). Для каждого 
пользователя в базе данных ПДП определяется персональная запись, связанная с его учетной записью в 
ЕСК, и следующие, связанные с учетной записью пользователя, данные: 
      \begin{itemize}
\item перечень выданных пользователю разрешений на доступ;
\item перечень типовых ролей, в которые включен пользователь.
\end{itemize}

      Также в базе данных ПДП хранится информация о типовых ролях пользователей и информация 
об иерархии типовых ролей.

\begin{figure*} %fig5
\vspace*{1pt}
\begin{center}
\mbox{%
\epsfxsize=110.676mm
\epsfbox{ily-5.eps}
}
\end{center}
\vspace*{-3pt}
      \Caption{Базы метаданных, используемые для управления доступом и обработки запросов: 
ПДП~--- права доступа пользователей; ЕСКК~--- единая система классификации и кодирования; 
ЕМД~--- единая модель данных; ЦРМ~--- центральный репозиторий метаданных; ПТЗ~--- перечень 
типовых запросов
       \label{f5il}}
       \vspace*{6pt}
       \end{figure*} 

      
      \textit{Типовые запросы.} Каждый типовой запрос описывает определенный класс 
реальных (ис\-пол\-ня\-емых) запросов, которые могут различаться\linebreak
 значениями поисковых 
параметров, составом возвращаемых полей данных или составом поисковых фильтров. Для 
описания такого класса запросов предназначен шаблон запроса~--- формальное 
      XML-описание, специфицирующее общую структуру\linebreak
       типового запроса, его семантику, 
а также все возможные вариации запроса. Простейшим при\-емом поддержки варьируемых 
структурных элементов типового запроса может быть их описание с по\-мощью текстовых 
подстановок, включаемых в шаблон запроса. Неварьируемые (фиксированные) структурные 
элементы типового запроса всегда должны описываться в терминах единой модели данных. 
Описание запроса в терминах единой модели является универсальным и не зависит от 
специфики реализации различных приложений, которые исполняют запрос. Рассмотрим эту 
технологию несколько подробнее.
      
      Шаблон запроса описывает целый класс реальных (исполняемых) запросов, поэтому 
каждый раз при формировании реального запроса необходимо конкретизировать его 
структуру и параметры. Для этой цели служит макет запроса, пред\-став\-ля\-ющий собой 
      XML-описание структуры и параметров исполняемого запроса. Подготовка макета 
запроса является задачей приложения, инициирующего поисковый запрос. Макет запроса 
может рассматриваться как формальная спецификация поискового запроса, предназначенного 
для исполнения в АИС.
      
      Каждый типовой запрос связывается с множеством ИР, где он может быть исполнен. 
При конструировании и связывании типового запроса с конкретными приложениями и базами 
данных выражения в шаблоне запроса переписываются из терминов единой схемы данных в 
термины схемы данных ИР (отображение шаблона запроса на ИР). 
      
      Перед исполнением типового запроса в приложении, осуществляющем доступ к базам 
данных ИР, всегда выполняется трансляция запроса. Процесс трансляции осуществляется 
автоматически средствами ПО промежуточного слоя перед передачей его на исполнение в 
приложение. Результатом трансляции служит текст запроса в терминах схемы базы данных 
ИР, готовый к подаче на вход конкретного приложения. Результаты исполнения любого 
типового запроса представляются в XML-фор\-ма\-те в терминах единой схемы данных. Для 
каждого типового запроса задается схема данных, описывающая формат представления 
результатов. 

      
      Каждый типовой запрос связывается с одной или несколькими группами доступа. На 
основании\linebreak
этих связей вычисляются права доступа конкретного пользователя к типовому 
запросу. Фильтрация информации с учетом прав доступа конкретного пользователя согласно 
его роли может\linebreak
осуществляться как на уровне приложений, так и на уровне компонентов 
промежуточного слоя. Типовые запросы могут связываться также с типовыми ролями с целью 
блокирования возможности отдельных групп пользователей обращаться к базам данных ИР с 
определенными запросами. На рис.~\ref{f5il} схематично представлен состав баз метаданных, 
используемых компонентами промежуточного слоя при реализации сервисов обработки 
запросов и управления доступом. 
      

      Завершая описание предложенной модели групповых политик пользователей по 
доступу к ИР ведомства, можно сформулировать основные уровни 
ограничения (или наоборот разрешения)\linebreak доступа групп пользователей к структурам данных и 
операциям. Методически представляется целесообразным выделить два уровня управления 
доступом пользователей~--- уровень запросов и уровень данных. 
      
      Уровень запросов:
      \begin{itemize}
\item доступ к запросам на обновление баз данных (индивидуально для каждого ИР);
\item доступ к запросам на поиск в конкретных базах данных однотипных (например, типовых 
территориальных) ИР и разнородных ИР с семантически однородными данными.
\end{itemize}

Уровень данных (для всех запросов):
\begin{itemize}
\item доступ к отдельным ИР и группам ИР;
\item доступ к отдельным базам данных (разделам ИР) и отдельным учетным документам;
\item доступ к отдельным полям ИР (в терминах единой схемы данных) для всех запросов.
\end{itemize}

      Помимо вышеперечисленных, в рамках ве\-домства могут существовать и 
дополнительные требова\-ния к групповым политикам. Прежде всего, могут накладываться 
ограничения по возможности использования различных масок и логических функций в 
поисковых полях запросов, а\linebreak также ограничения по числу релевантных объектов, 
предос\-тав\-ля\-емых пользователю в качестве результата. Особое значение такие требования 
могут иметь в целях выполнения законодательных ограничений на доступ к персональным 
данным, а также предот\-вра\-ще\-ния несанкционированного копирования данных из баз данных 
ведомства. Могут быть сформулированы требования к фильтрации результатов запросов к 
совокупности ИР (например, с целью устранения дублирующей информации), реализации 
каскадных запросов (результаты выполнения одного запроса являются параметрами 
по\-сле\-ду\-ющих запросов). Поскольку обычно новые законодательные, межведомственные и 
внут\-ри\-ведомственные требования на начальном этапе не подкреплены правоприменительной 
практикой, необходимо поэтапное уточнение требований на примере решения конкретных 
прикладных задач. 
      
      По мере наработки успешных технических решений реализации этих дополнительных 
требований в рамках конкретных АИС, методы и алгоритмы решений могут быть обобщены и\linebreak 
представлены в промежуточном слое. Например, могут быть реализованы дополнительные 
препроцессоры запросов или постпроцессоры обработки результатов запросов. Обращения же 
к соответствующим дополнительным функциям промежуточного слоя будут оформлены в 
виде веб-сер\-висов.
\columnbreak
      
      Выводы: 
      \begin{enumerate}[(1)]
\item практический опыт реализации системы управ\-ле\-ния доступом пользователей к 
разнородным базам данных на основе единой схемы данных и технологий веб-сер\-ви\-сов 
доказал возможность интеграции на\linebreak прикладном уровне разнородных унаследованных АИС, 
постро\-енных на различных прог\-рам\-мных платформах. При этом трудоемкость модернизации 
унаследованных АИС составляла не более 5\% трудоемкости проектирования новых АИС, 
удовлетворяющих требованиям ведомства к интеграции данных и обеспечению 
информационной безопасности;
\item структурной основой хранения информации о пользователях (сотрудников данного 
ведомства и других ведомств) и их правах доступа к ИР ведомства 
являются:
\begin{itemize}
\item СУЦ;
\item единая система каталогов пользователей;
\item база данных ПДП (групповых ролей);
\item база данных пользователей;
\end{itemize}
\item в рамках описанного технического решения групповые политики доступа пользователей 
к АИС и базам данных ИР ведомства осуществляются на следующих 
уровнях:
\begin{itemize}
\item уровень перечня ИР;
\item уровень отдельных баз данных (разделов) каждого ИР;
\item уровень отдельных полей единой схемы данных;
\item уровень возможности взаимодействия с конкретными АИС и доступности запросов к 
базам данных ИР;
\end{itemize}
\item для упрощения процесса проектирования сис\-те\-мы запросов в терминах единой схемы 
данных с использованием базового подмножества ЯМД реализованы методы шаблона\linebreak
 запроса 
и подстановок. Эти методы обеспечивают единообразную интерпретацию компонентами 
промежуточного слоя запросов к множеству АИС на основе единой схемы данных и 
компактного подмножества конструкций единого ЯМД, интерпретируемого всеми АИС. 
С~другой стороны, эти методы позволяют дополнять базовые конструкции ЯМД произвольными 
языковыми конструкциями, в том числе конструкциями ЯМД конкретных СУБД, 
используемых при разработке АИС.
\end{enumerate}

\section{Заключение}

      Модернизация систем информационного обес\-пе\-че\-ния ведомства должна 
осуществляться с\linebreak
учетом необходимости взаимодействия с аналогичными системами других 
ведомств. Интероперабельность разнородных систем можно обеспечить при использовании 
инфраструктуры промежуточного слоя. Инфраструктура промежуточного слоя решает 
следующие задачи:
      \begin{enumerate}[(1)]
\item  обеспечение интероперабельности информационных систем на уровне приложений 
промежуточного слоя при совместной обработке данных из множества информационных 
хранилищ, включая комплексные и каскадные запросы сразу к нескольким разнородным 
информационным системам;
      \item перенаправление всех запросов сотрудников ведомства (и других ведомств) в 
единую территориально распределенную инфраструктуру аутентификации и управления 
доступом пользователей взамен ранее используемых прямых обращений к АИС и~ИР.
      \end{enumerate}
      
      Создание промежуточного слоя в условиях ранее сложившейся территориально 
распределенной неоднородной ИТ-ин\-фра\-струк\-ту\-ры ведомства\linebreak
 сводится к разработке и 
внедрению совокупности взаимосвязанных территориально распределенных 
прог\-рам\-мно-тех\-ни\-че\-ских комплексов промежуточного слоя. Эти комплексы служат 
основой\linebreak (стержнем) горизонтальной и вертикальной информационной интеграции, 
модернизации и последующего развития всех АИС ведомства на\linebreak
 основе функциональной и 
программной стандартизации. Программно-тех\-ни\-че\-ская среда промежуточного слоя берет на 
себя функции централизованного управления доступом пользователей к хранилищам 
информации и обеспечения информационной безопасности.
      
      Разработка единой информационной модели ведомства сводится к разработке единой 
схемы данных и реализации расширяемого перечня запросов и отчетов на основе единого 
языка (или со\-во\-куп\-ности языков) манипулирования данными. В~качестве первого шага в 
решении этой задачи целесообразно создание ядра единой схемы данных и определение 
совокупности базовых конструкций и ограничений выбранного языка манипулирования 
данными.
      
      Перечень веб-сер\-ви\-сов ПО промежуточного слоя специфичен для каждого ведомства, 
однако можно выделить совокупность веб-сер\-ви\-сов, не связанных со спецификой предметной 
области. Это сервисы, связанные с аутентификацией и определением прав доступа 
пользователей, поиском и об\-нов\-ле\-ни\-ем данных, доступом к метаданным и классификаторам, 
доступом к административным базам данных, и сервисы, необходимые для прикладного 
уровня управления функционированием.
      
      Разработка требований к централизованной сис\-те\-ме управления доступом сводится к 
выделению основных категорий пользователей и определению ролей доступа к 
ИР внутри каждой категории. Структурной основой хранения 
информации о пользователях (сотрудниках данного ведомства и других ведомств) и их правах 
доступа к ИР ведомства являются:
      \begin{itemize}
\item СУЦ;
\item единая система каталогов пользователей;
\item база данных ПДП (групповых ролей);
\item база данных пользователей. 
\end{itemize}
      
      Для упрощения процесса проектирования сис\-те\-мы запросов в терминах единой схемы 
данных с использованием базового подмножества ЯМД целесообразно использовать шаблоны 
запросов и подстановок. 
      
      Технические решения, изложенные в данной пуб\-ли\-ка\-ции, прошли практическую 
апробацию~[7--9]. При этом осуществлялась модернизация унаследованных АИС, 
разработанных с использованием разных инструментальных средств и функционирующих на 
комплексах различной технической и программной архитектуры (3~семейства операционных 
систем, 4~семейства СУБД).

\vspace*{6pt}

{\small\frenchspacing
{%\baselineskip=10.8pt
\addcontentsline{toc}{section}{Литература}
\begin{thebibliography}{9}
     
\bibitem{1il}
Анализ развития и использования информационно-ком\-му\-ни\-ка\-ци\-он\-ных технологий в 
регионах России: Аналитический доклад~/ Под ред.\ Ю.\,Е.~Хохлова.~--- М.: Институт 
развития информационного общества, 2008.  240~с.

\bibitem{2il}
Web Services Activity, W3C. {\sf http://www.w3.org/ 2002/ws/}.

\bibitem{3il}
\Au{Захаров В.\,Н., Калиниченко~Л.\,А., Соколов~И.\,А., Ступников~С.\,А.}
Конструирование канонических информационных моделей для интегрированных 
информационных систем~// Информатика и её применения, 2007. Т.~1. Вып.~2. С.~15--38.

\bibitem{4il}
\Au{Босов А\, В.}
Порталы в системах органов государственной власти~// Информатика и её применения, 
2008. Т.~2. Вып.~1. С.~44--54.

\bibitem{5il}
\Au{Зацаринный А.\,А., Ионенков~Ю.\,С., Кондрашев~В.\,А.}
Об одном подходе к выбору системотехничсеких решений построения 
информационно-те\-ле\-ком\-му\-ни\-ка\-ци-\linebreak\vspace*{-12pt}
\pagebreak

\noindent
он\-ных систем~// Системы и средства информатики, 2006. 
Вып.~16. C.~66--71.

\bibitem{6il}
\Au{Брюхов Д.\,О., Вовченко А.\,Е., Захаров~В.\,Н., Желенкова~О.\,П., Калиниченко~Л.\,А., 
Мартынов~Д.\,О., Скворцов~Н.\,А., Ступников~С.\,А.}
Архитектура промежуточного слоя предметных посредников для решения задач над 
множеством интегрируемых неоднородных распределенных информационных ресурсов в 
гибридной грид-инфраструктуре виртуальных обсерваторий~// Информатика и её 
применения, 2008. Т.~2. Вып.~1. С.~2--34. 

\bibitem{7il}
\Au{Поляков С.\,В., Костомарова~Л.\,Г., Щаренская~Т.\,Н., Илюшин~Г.\,Я.}
 Корпоративная автоматизированная сис\-те\-ма здравоохранения города Москвы (КАИС 
<<Мосгорздрав>>)~// Информационное общество, 2008. №\,1. С.~20--25.

\bibitem{8il}
\Au{Илюшин Г.\,Я.}
Информационная архитектура региональных проектов здравоохранения на примере проекта 
<<Удаленная регистратура>>~// Информационное общество, 2008. №\,1. С.~31--40.



\label{end\stat}

\bibitem{9il}
\Au{Соколов И.\,А., Зацаринный А.\,А., Захаров~В.\,Н., Илюшин~Г.\,Я., Кузьмин~А.\,П., 
Цыганков~В.\,С.}
Основные сис\-те\-мо\-тех\-ни\-ческие решения по построению ЕИТКС ОВД~// Системы и 
средства информатики. Спец. вып. На\-уч\-но-тех\-ни\-че\-ские вопросы построения и развития 
ин\-фор\-ма\-ци\-он\-но-те\-ле\-ком\-му\-ни\-ка\-ци\-он\-ной сис\-те\-мы органов внутренних дел.~--- М.: ИПИ 
РАН, 2009. С.~11--33.
       
 \end{thebibliography}
}
}
\end{multicols}