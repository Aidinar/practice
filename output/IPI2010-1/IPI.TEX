\documentclass[10pt]{book}
\usepackage[utf8]{inputenc}

\usepackage{latexsym,amssymb,amsfonts,amsmath,indentfirst,shapepar,%fleqn,%
picinpar,shadow,floatflt,enumerate,multicol,ipi}

\usepackage{rotating}
\input{epsf}

%\nofiles

%\includeonly{avtor} %,avtor-eng}
%\includeonly{avtor-eng}
%\includeonly{pred}

%\includeonly{bening} %+pdf   4
%\includeonly{gudkov}   %+  8pdf
%\includeonly{karateev}  %+  9+pdf
%\includeonly{protasov} % +   12pdf
%\includeonly{konushin}   %+  10pdf
%\includeonly{pavel}   %+pdf   11
%\includeonly{kavaguchi}   %+3pdf
%\includeonly{seiful}   %+ 6pdf
%\includeonly{markin}     %+ 2pdf
%\includeonly{ilushin}     %+ 5pdf
%\includeonly{arutun}     %+ 7pdf
%\includeonly{sinits}     %+ 1pdf
%\includeonly{podgot}



%\includeonly{toc-rus,toc-en}
%\includeonly{toc-en}


%\includeonly{obchak}
%\includeonly{reshal}
%\includeonly{eng-index}
%\includeonly{cover3}

\usepackage{acad}
\usepackage{courier}
\usepackage{decor}
\usepackage{newton}
\usepackage{pragmatica}
\usepackage{zapfchan}
\usepackage{petrotex}
\usepackage{bm}                     % полужирные греческие буквы
\usepackage{upgreek}                % прямые греческие буквы
%\usepackage{verbatim}

\renewcommand{\bottomfraction}{0.99}
\renewcommand{\topfraction}{0.99}
\renewcommand{\textfraction}{0.01}

\setcounter{secnumdepth}{1} %здесь - 3 + chapter = 4

\arraycolsep=1.5pt

%\usepackage[pdftex]{graphicx}

%\usepackage{oz}

%NEW COMMANDS



\renewcommand{\r}{{\rm I\hspace{-0.7mm}\rm R}}
\newcommand{\I}{{\rm I\hspace{-0.7mm}I}}
\newcommand{\Ik}{\mbox{{\small \tt {1}}\hspace{-1.5mm}{\tt 1}}}
%\newcommand{\Ikl}{{\small \tt{1}}\hspace*{-0.4mm}\mathtt{1}}

%\mathrm{I}\hspace*{-0.7mm}\mathrm{R}

\newcommand{\il}[2]{\int\limits_{#1}^{#2}}%интеграл с пределами #1 и #2

\newcommand{\p}{{\sf P}}  % вероятность
\newcommand{\e}{{\sf E}}  % мат. ожидание
\newcommand{\D}{{\sf D}}  % дисперсия
\newcommand{\eps}{\varepsilon}
\newcommand{\vp}{\mathrm{v.p.}}
\newcommand{\F}{{\mathcal F}}
%\def\iint{\int\limits_{-\infty}^{\infty}}

%\newcommand{\gr}{{\geqslant}}

%\renewcommand{\la}{\lambda}
\newcommand{\si}{\sigma}
%\renewcommand{\a}{\alpha}

%\newcommand{\pto}{\stackrel{P}{\longrightarrow}} % сходимость по веpоятности

%\newcommand{\eqd}{\stackrel{d}{=}} % равенство по pаспpеделению

%\newcommand{\kp}{\kappa}
%\def\Q{{\cal Q}} \def\H{{\cal H}}
%\newcommand{\bet}{\beta_{2+\delta}}


%\newtheorem{definition}{Определение}
%\renewcommand{\thedefinition}{\arabic{definition}.}
%END NEW COMMANDS

%\renewcommand{\baselinestretch}{1.2}

%\pagestyle{myheadings}

\setlength{\textwidth}{167mm}      % 122mm
\setlength{\textheight}{658pt}
%\setlength{\textheight}{635.6pt}
\setlength{\columnsep}{4.5mm}

\setcounter{secnumdepth}{4}

%\addtolength{\headheight}{2pt}
%\addtolength{\headsep}{-2mm}

%\addtolength{\topmargin}{-20mm}  % for printing


\hoffset=-30mm  % From Yap
%\hoffset=-20mm  % From Acrobat

%\voffset=0mm % From Yap
%\voffset=-15mm   % From Acrobat

\addtolength{\evensidemargin}{-9.5mm} % for printing
\addtolength{\oddsidemargin}{9.5mm}  % for printing

%\renewcommand{\thefootnote}{\fnsymbol{footnote}}
%\renewcommand{\thefootnote}{\arabic{footnote}}
\renewcommand{\figurename}{\protect\bf Рис.}
\renewcommand{\tablename}{\protect\bf Таблица}

\newcommand{\Caption}[1]{\caption{\protect\small %\baselineskip=2.5ex
#1}}

\renewcommand{\thefigure}{\arabic{figure}}
\renewcommand{\thetable}{\arabic{table}}
\renewcommand{\theequation}{\arabic{equation}}
\renewcommand{\thesection}{\arabic{section}}

\renewcommand{\contentsname}{СОДЕРЖАНИЕ}
\newcommand{\fr}[2]{\displaystyle\frac{\displaystyle #1\mathstrut}{\displaystyle #2\mathstrut}}

%\renewcommand{\thefootnote}{\fnsymbol{footnote}}
%\newcommand{\g}{\mbox{\textit{g}}}

%\newcommand{\Caption}[1]{\caption{\protect\small\baselineskip=2ex #1}}
\newcounter{razdel}
\setcounter{razdel}{0}


\newcommand{\titel}[4]{%
\

\vspace*{5pt}

\ifodd\therazdel {\raggedright\noindent\Large\textrm\textbf
 \lineskip .75em
  \baselineskip=3.2ex #1 \par}
\vskip 1em {\noindent\large\textrm\textbf #2 \par}
\addcontentsline{toc}{subsection}{{\textrm\textbf #3}\protect\newline #1}
\def\rightheadline{\underline{\noindent\hbox to \textwidth{\hfill\small\textrm{#4}
%\hfill \large\bf\thepage
}}}
\def\leftheadline{\underline{\noindent\parbox{\textwidth}{
%\raggedleft\large\bf\thepage \hfill
\small\textit{#3}\hfill}}}
\def\leftfootline{\small{\textbf{\thepage}
\hfill ИНФОРМАТИКА И ЕЁ ПРИМЕНЕНИЯ\ \ \ том~4\ \ \ выпуск 1\ \ \ 2010}
}%
 \def\rightfootline{\small{ИНФОРМАТИКА И ЕЁ ПРИМЕНЕНИЯ\ \ \ том~4\ \ \ выпуск~1\ \ \ 2010
\hfill \textbf{\thepage}}} \vskip 2em \setcounter{figure}{0}
\setcounter{table}{0} \setcounter{equation}{0} \setcounter{section}{0}
\setcounter{subsection}{0} \setcounter{subsubsection}{0}
\setcounter{footnote}{0} \setcounter{razdel}{0}
%\end{flushleft}
\else {
 \raggedright\noindent\Large\textrm\textbf
 \lineskip .75em
\baselineskip=3.2ex #1 \par} \vskip 1em
%\begin{flushleft}
{\noindent\large\textrm\textbf #2 \par}
\addcontentsline{toc}{subsection}{{\textrm\textbf #3}\protect\newline #1}
\def\rightheadline{\underline{\noindent\hbox to \textwidth{\hfill\small\textrm{#4}
%\hfill \large\bf\thepage
}}}
\def\leftheadline{\underline{\noindent\parbox{\textwidth}{%\raggedleft\large\bf\thepage \hfill
\small\textit{#3}\hfill}}}
\def\leftfootline{\small{\textbf{\thepage}
\hfill ИНФОРМАТИКА И ЕЁ ПРИМЕНЕНИЯ\ \ \ том~4\ \ \ выпуск~1\ \ \ 2010}
}%
 \def\rightfootline{\small{ИНФОРМАТИКА И ЕЁ ПРИМЕНЕНИЯ\ \ \ том~4\ \ \ выпуск~1\ \ \ 2010
\hfill \textbf{\thepage}}} \vskip 2em \setcounter{figure}{0}
\setcounter{table}{0} \setcounter{equation}{0} \setcounter{section}{0}
\setcounter{subsection}{0} \setcounter{subsubsection}{0}
\setcounter{footnote}{0}
%\end{flushleft}
\fi}

\newcommand{\titelr}[2]{%
\

\vspace*{5pt}

\ifodd\therazdel {\raggedright\noindent\large\textrm\textbf
 \lineskip .75em
  \baselineskip=3.2ex #1 \par}
\vskip 1em {\noindent\normalsize\textrm\textbf #2 \par}
\else {
 \raggedright\noindent\large\textrm\textbf
 \lineskip .75em
\baselineskip=3.2ex #1 \par} \vskip 1em
%\begin{flushleft}
{\noindent\normalsize\textrm\textbf #2 \par}
\fi}

\newcommand{\titele}[5]{%
\

%\vspace*{5pt}

\ifodd\therazdel {\raggedright\noindent%\large
\textrm\textbf
 \lineskip .75em
%  \baselineskip=3.2ex
#1 \par}
\vskip .5em {\noindent\large\textrm\textbf #2 \par}
\vskip .5em
 {\noindent\textrm #3 \par}
\addcontentsline{toc}{subsection}{{\textrm\textbf #1}\protect\newline #2}
\def\rightheadline{\underline{\noindent\hbox to \textwidth{\hfill\small\textrm{#4}
%\hfill \large\bf\thepage
}}}
\def\leftheadline{\underline{\noindent\parbox{\textwidth}{
%\raggedleft\large\bf\thepage \hfill
\small\textrm{#5}\hfill}}}
\def\leftfootline{\small{\textbf{\thepage}
\hfill ИНФОРМАТИКА И ЕЁ ПРИМЕНЕНИЯ\ \ \ том~4\ \ \ выпуск~1\ \ \ 2010}
}%
 \def\rightfootline{\small{ИНФОРМАТИКА И ЕЁ ПРИМЕНЕНИЯ\ \ \ том~4\ \ \ выпуск~1\ \ \ 2010
\hfill \textbf{\thepage}}} \vskip 1em \setcounter{figure}{0}
\setcounter{table}{0} \setcounter{equation}{0} \setcounter{section}{0}
\setcounter{subsection}{0} \setcounter{subsubsection}{0}
\setcounter{footnote}{0} \setcounter{razdel}{0}
%\end{flushleft}
\else {
 \raggedright\noindent%\large
 \textrm\textbf
 \lineskip .75em
%\baselineskip=3.2ex
#1 \par} \vskip .5em
%\begin{flushleft}
{\noindent\large\textrm\textbf #2 \par} \vskip .5em
 {\noindent\textrm #3 \par}
\addcontentsline{toc}{subsection}{{\textrm\textbf #1}\protect\newline #2}
\def\rightheadline{\underline{\noindent\hbox to \textwidth{\hfill\small\textrm{#4}
%\hfill \large\bf\thepage
}}}
\def\leftheadline{\underline{\noindent\parbox{\textwidth}{%\raggedleft\large\bf\thepage \hfill
\small\textrm{#5}\hfill}}}
\def\leftfootline{\small{\textbf{\thepage}
\hfill ИНФОРМАТИКА И ЕЁ ПРИМЕНЕНИЯ\ \ \ том~4\ \ \ выпуск~1\ \ \ 2010}
}%
 \def\rightfootline{\small{ИНФОРМАТИКА И ЕЁ ПРИМЕНЕНИЯ\ \ \ том~4\ \ \ выпуск~1\ \ \ 2010
\hfill \textbf{\thepage}}} \vskip 1em \setcounter{figure}{0}
\setcounter{table}{0} \setcounter{equation}{0} \setcounter{section}{0}
\setcounter{subsection}{0} \setcounter{subsubsection}{0}
\setcounter{footnote}{0}
%\end{flushleft}
\fi}

\def\Abst#1{
\begin{center}\small\nwt
\parbox{150mm}{%\baselineskip=2.5ex
\textbf{Аннотация:}\ \
%\hspace*{\parindent}
#1}
\end{center}}
\def\Abste#1{
\begin{center}\small\nwt
\parbox{150mm}{%\baselineskip=2.5ex
\textbf{Abstract:}\ \
%\hspace*{\parindent}
#1}
\end{center}}

\def\KW#1{
\begin{center}\small\nwt
\parbox{150mm}{%\baselineskip=2.5ex
\textbf{Ключевые слова:}\ \ #1}
\end{center}}

\def\KWE#1{
\begin{center}\small\nwt
\parbox{150mm}{%\baselineskip=2.5ex
\textbf{Keywords:}\ \ #1}
\end{center}}


\def\KWN#1{
%\begin{center}
%\small
%\parbox{150mm}\end{center}
}

\renewcommand{\thesubsection}{\thesection.\arabic{subsection}\hspace*{-5pt}}
\renewcommand{\thesubsubsection}{\thesubsection\hspace*{5pt}.\arabic{subsubsection}\hspace*{-3pt}}

\begin{document}
\Rus

\nwt
%\ptb

%\renewcommand{\contentsname}{\protect\Large\bf Содержание}

\setcounter{tocdepth}{2}

%\tableofcontents

\renewcommand{\bibname}{\protect\rmfamily Литература}
  \def\Au#1{{\it #1}}

%\newcommand{\No}{№}
  \newcommand{\tg}{\rm  tg}
    \newcommand{\ctg}{\mathrm{  ctg}}
  \newcommand{\arctg}{\rm  arctg}

\setcounter{page}{1}

\newpage
\addtocounter{razdel}{1}
%\def\razd{РЕГУЛИРУЕМЫЙ ЭЛЕКТРОПРИВОД ДЛЯ ЭЛЕКТРОЭНЕРГЕТИКИ}
%\newpage
%\def\stat{zakh}
\def\tit{СРЕДСТВА ОБЕСПЕЧЕНИЯ ОТКАЗОУСТОЙЧИВОСТИ ПРИЛОЖЕНИЙ}
\def\titkol{Средства обеспечения отказоустойчивости приложений}

\def\aut{В.\,Н.~Захаров$^1$, В.\,А.~Козмидиади$^2$}
\titel{\razd}{\tit}{\aut}{\titkol}


\Abst{Рассмотрены проблемы построения отказоустойчивых серверов, возникающие в связи с недетерминированностью поведения приложений. Предложена формальная модель, описывающая поведение приложения, основными объектами которой являются ресурсы и события. Предложены алгоритмы протоколирования работы приложения на резервном узле кластера, а также восстановления и продолжения его работы при отказе основного узла. При этом для клиентов сбой остается незаметным, за исключением некоторого увеличения времени обслуживания.}

\KW{сервер приложений $\bullet$ прозрачная отказоустойчивость $\diamond$
 процесс $\diamond$ ресурс $\diamond$ событие $\diamond$ контрольная точка
$\bullet$ детерминированность}

\vskip 12pt plus 6pt minus 3pt

\begin{multicols}{2}

\section*{ВВЕДЕНИЕ}

Средства вычислительной техники стали использоваться в областях,
требующих безотказной работы систем в течение многих лет (24 часа
в сутки, 365 дней в году).

\label{st\stat}

\footnotetext{$^1$ФГУП Центральный институт авиационного моторостроения
им. П.И. Баранова, Москва, Россия}
\footnotetext{$^2$ФГУП Центральный институт авиационного моторостроения
им. П.И. Баранова, Москва, Россия}

К таким областям относятся, например, центры хранения и обработки данных  в сетях (системы резервирования билетов, биллинговые,  банковские и т.д.), массированные распределенные вычисления (GRID-вычисления) и другие.

\thispagestyle{headings}

Обычно в подобных системах применяются частные решения, ориентированные в основном на обеспечение надежного хранения данных (например, файловые серверы, использующие для хранения RAID-контроллеры) и корректного их состояния при отказах (серверы баз данных с транзакционным выполнением запросов). Однако большинство приложений не гарантируют, что не произойдет потери части данных при отказе системы. Обычно предполагается, что клиентские средства должны повторять запросы после восстановления серверов, для того, чтобы данные не были потеряны, или что можно сделать возврат по времени на некоторое время назад и повторить работу с этого места. Однако далеко не все клиентские средства и условия применения приложений допускают это.

Отказоустойчивые системы для критически важных приложений, корректно решающие проблемы восстановления после сбоев,   предлагаемые ведущими производителями, как правило, дороги. Кроме того, они включают специфические серверные и клиентские приложения, не совместимые со стандартными приложениями, не обеспечивающими отказоустойчивость. Примером такого подхода к решению проблемы отказоустойчивости  хранения данных являются системы NetApp FAS компании Network Appliance, работающие на базе специализированной операционной системы Data ONTAP [1].

Построение отказоустойчивых систем, использующих серверы со стандартными приложениями, в свете вышесказанного, является актуальной проблемой, вызывающей значительный интерес. Рассмотрение методов достижения прозрачной отказоустойчивости таких систем и является предметом статьи.
\begin{figure*} %fig1
\vspace*{1pt}
\begin{center}
\mbox{%
\epsfxsize=1.6in
\epsfxsize=100mm
\epsfbox{BbR-1.eps}
}
\end{center}
\vspace*{-9pt}
\Caption{Базовый вариант трубы с разными выходными устройствами
(цилиндрическое, расширяющееся и сужающееся сопло)
\label{f1bab}}
\vspace*{-3pt}
\end{figure*}


\section{ОСНОВНЫЕ ПОНЯТИЯ И ПОДХОДЫ}

Под сервером в данной работе понимается вычислительный центр
(отдельный компьютер или кластер) в сети, предоставляющий клиентам
(пользователям, клиентским компьютерам) определенные услуги, разделяя
между ними свои ресурсы. Подобные серверы названы серверами приложений.
Широко распространенным примером сервера такого типа является файловый сервер, обеспечивающий удаленный коллективный доступ к файловой системе. Часто используются вычислительные серверы, предоставляющие клиентам возможность выполнять на них свои программы (например, в центрах коллективного пользования).


Обычно приложение представляет собой программу или группу программ, работающих в операционной среде, создаваемой операционной системой (в другой терминологии - один или несколько взаимодействующих процессов или потоков (threads)), которые реализуют функциональность сервера. Для построения отказоустойчивых серверов приложений широко используется кластерная технология. Следуя [2], кластером, названа разновидность параллельной или распределенной системы, которая:
\begin{itemize}
\item состоит из нескольких компьютеров (узлов кластера), связанных как минимум необходимыми коммуникационными каналами;
\item используется как единый, унифицированный компьютерный ресурс.
\end{itemize}

Прозрачная отказоустойчивость (Transparent Fault Tolerance, TFT) сервера приложений - это такое его поведение при возникновении аппаратных или программных отказов либо отказов в сети, при котором:
\begin{itemize}
\item отказ не вызывает потери или искажения данных, находящихся в базе данных сервера;
\item сервер продолжает нормально функционировать, несмотря на имевшие место отказы.
\end{itemize}

Клиенты сервера "не замечают" произошедших отказов. Единственным\footnote{допустимым
отклонением сервера от нормального поведения с точки зрения клиента является
некоторое увеличение времени обслуживания} (на несколько секунд или десятков секунд).

Обычно приложения, работающие на серверах приложений, не ориентированы на прозрачную отказоустойчивость. Они "заботятся" лишь о собственной целостности (например, состояния файловой системы или базы данных). Восстановление работоспособности сервера приводит к разрыву соединений с клиентами и потере их запросов. Это замечают клиенты - запросы следует повторять, на что клиентские приложения далеко не всегда рассчитаны. В данной работе предполагается, что приложения (прикладные программные средства), выполняемые на сервере, являются стандартными, то есть не имеют специальных средств, обеспечивающих отказоустойчивость.
\begin{figure*}[b] %fig1
\vspace*{1pt}
\begin{center}
\mbox{%
\epsfxsize=1.6in
\epsfxsize=100mm
\epsfbox{BbR-1.eps}
}
\end{center}
\vspace*{-9pt}
\Caption{Базовый вариант трубы с разными выходными устройствами
(цилиндрическое, расширяющееся и сужающееся сопло)
\label{f1bab}}
\vspace*{-3pt}
\end{figure*}

Серьезные исследования в области обеспечения отказоустойчивости серверов были развернуты после создания вычислительных серверов, предназначенных для решения задач, требующих больших вычислительных ресурсов. Решение этих задач выполняется на суперкомпьютерах, обеспечивающих массово-параллельные вычисления и представляющих собой кластеры из сотен и тысяч узлов (процессоров). Однако даже на этих "монстрах" решение может требовать десятков или сотен часов, и одиночный сбой, если не предприняты специальные меры, может привести к необходимости начинать работу сначала. Обычно решение вычислительной задачи в таких случаях осуществляется в модели относительно редко взаимодействующих между собой процессов, выполняемых на разных узлах кластера. Эти взаимодействия нужны для координации работы процессов, в частности, для обмена данными и промежуточными результатами. Взаимодействия опираются на специальный протокол, называемый MPI (Message-Passing Interface) и представляющий собой стандарт "de facto" [3].

Для преодоления последствий сбоя достаточно давно была разработана и широко применяется технология, опирающаяся на механизм контрольных точек (checkpoints) [4-6]. По этой технологии система должна иметь стабильную память, которая не меняется при отказах. Соответствующие программные средства периодически сохраняют информацию о состоянии процессов приложения в стабильной памяти. Все процессы также имеют доступ к устройству стабильной памяти.  В случае отказа или сбоя, записанная в стабильную память информация используется для повторения вычисления с момента, когда была записана эта информация, то есть выполняется откат назад по времени. Данные, сохранение которых позволяет выполнить откат, называются контрольной точкой. В качестве устройства стабильной памяти может использоваться дисковый том, энергонезависимая оперативная память, память другого узла или узлов кластера. В последнем случае узел, которому требуется сохранить информацию, пересылает ее через быстрый канал связи на другой узел. Стабильная память после отказа одного из узлов должна быть доступной узлу, на котором делается повтор.

Однако решение, опирающееся только на контрольные точки, не является прозрачным, поскольку не скрывает от клиентов факт отказа системы и требует от них выполнения определенных действий. Так как при работе процессы обмениваются сообщениями, возможны два варианта решения проблемы. Первый - все процессы выполняют записи контрольных точек одновременно, что затруднительно. Второй вариант, при несоблюдении синхронности, - возврат в каждом процессе к такому скоординированному набору контрольных точек, при котором невозможна противоречивая ситуация. Такая ситуация возникает, когда один процесс вернулся к контрольной точке, после которой он должен получить сообщение от другого процесса, а этот другой процесс вернулся к точке, которая следует за выдачей этого сообщения. Однако при повторе ожидаемое первым процессом сообщение не поступит. В этом случае  возможен эффект домино, в результате процессы оказываются отброшены как угодно далеко назад.

В этом состоит первая проблема, которую необходимо преодолеть.

Если нужно, чтобы последствия отказа узла не были видны клиенту,  это означает:
\begin{itemize}
\item клиент не должен терять и потом восстанавливать соединения с сервером;
\item клиент не должен повторять свои запросы;
\item клиент не должен повторно получать сообщения, которые он уже получил.
\end{itemize}

Вторая проблема, которую надо решать, связана с недетерминированностью поведения сервера приложений. Приведем пример.  Пусть имеется система продажи билетов на самолеты. Два клиента одновременно обратились к системе с запросом билета на один и тот же рейс. Клиентам безразлично, какие места им зарезервирует система. Система выполняет запросы клиентов параллельно, поэтому в какой-то момент между процессами, обрабатывающими эти запросы, может возникнуть конкуренция за ресурс - в данном случае, скажем, рейс. Один из процессов захватывает ресурс первым, резервирует место и освобождает ресурс. Потом второй процесс проделывает то же самое.

Порядок, в котором в этом примере процессы захватили ресурс, зависит от многих факторов и, в конечном счете, случаен. Однако  это не мешает правильному функционированию системы, поскольку клиентам важно одно - получить билеты, причем на разные места. Однако отсутствие детерминизма в поведении приложения приводит к тому, что при повторном выполнении могут быть получены другие результаты: например, клиенту уже сообщено, что ему зарезервировано место №5, а при повторе может получиться, что зарезервировано место №6. Система должна устранить это несоответствие и сделать его невидимым для клиента.
\begin{figure*} %fig1
\vspace*{1pt}
\begin{center}
\mbox{%
\epsfxsize=1.6in
\epsfxsize=100mm
\epsfbox{BbR-1.eps}
}
\end{center}
\vspace*{-9pt}
\Caption{Базовый вариант трубы с разными выходными устройствами
(цилиндрическое, расширяющееся и сужающееся сопло)
\label{f1bab}}
\vspace*{-3pt}
\end{figure*}

Недетерминированность поведения системы это следствие, по крайней мере, двух обстоятельств. Во-первых, это присущая системам с разделением времени неопределенность в порядке выполнения процессов. Во-вторых, это конкуренция процессов за общие ресурсы. Перечислим некоторые причины недетерминированного поведения приложений:
\begin{itemize}
\item синхронизация процессов с помощью семафоров или атомарных операций над операндами в общей памяти процессов;
\item зависимость от порядка получения клиентских запросов;
\item время, затраченное процессом на обработку полученного запроса;
\item генераторы случайных чисел;
\item системное управление процессами и потоками;
\item локальные таймеры;
\item доступ к реальному времени.
\end{itemize}

По различным  причинам время, которое тратится на выполнение вычислительной задачи с одними и теми же исходными данными, не является константой, то есть повторное выполнение может дать другое время. Процессы используют общие ресурсы, обращение к которым требует организации очередности выполнения (сериализации) - первым пришел, первым захватил. И, наконец,  результат работы процесса может зависеть от состояния ресурса, а это состояние может изменить другой процесс, ранее захвативший ресурс. Все это создает значительные трудности при попытках воспроизведения поведения процессов с сохраненной контрольной точки.

Прозрачная отказоустойчивость серверов приложений обычно осуществляется переносом приложения на другой узел кластера, идентичный первому по конфигурации аппаратных средств и операционной среды. Это делается методом, называемым snapshot/restore. На основном узле (оригинале)  периодически фиксируется состояние приложения на этом узле кластера (так называемый снимок или snapshot). После отказа оригинала на резервном узле (копии) делается восстановление (restore), то есть восстанавливается последнее зафиксированное состояние приложения. Операционная среда при этом приводится в состояние, которое соответствует моменту изготовления снимка. После этого узел-копия продолжает работу с зафиксированного места. Сравнение метода  snapshot/restore с другими подходами приведено в [7].

Ниже рассматриваются информационные  технологии, позволяющие решить ряд принципиальных вопросов, связанных с реализацией прозрачной отказоустойчивости серверов приложений. Ими являются:
\begin{itemize}
\item виртуализация операционной среды, в которой работает серверное приложение;
\item отказоустойчивая реализация протокола TCP;
\item создание контрольных точек состояния приложения и файловой системы, которые делаются внешним по отношению к приложению образом;
\item восстановление серверного приложения на основании контрольной точки.
\end{itemize}
\begin{figure*} %fig1
\vspace*{1pt}
\begin{center}
\mbox{%
\epsfxsize=1.6in
\epsfxsize=100mm
\epsfbox{BbR-1.eps}
}
\end{center}
\vspace*{-9pt}
\Caption{Базовый вариант трубы с разными выходными устройствами
(цилиндрическое, расширяющееся и сужающееся сопло)
\label{f1bab}}
\vspace*{-3pt}
\end{figure*}

\section{МОДЕЛЬ ОПИСАНИЯ ПОВЕДЕНИЯ ПРИЛОЖЕНИЯ}

Предлагаемый подход опирается на построение модели вычислений, связанной с использованием понятия времени в многопроцессных приложениях. Впервые подобные проблемы были изучены в классической работе Л. Лампорта [8].

Многопроцессными приложения называются потому, что в них параллельно работают несколько процессов. Процесс ведет себя детерминированно, пока в предписанном кодом порядке выполняет процессорные инструкции. Конечно, его работа может быть прервана практически в любой момент и процессор передан другому процессу или ядру. Поэтому абсолютное время, которое затрачивает процесс на выполнение определенной работы, не  константа, а случайная  величина. То же  относится к относительному времени, то есть времени, которое процесс занимал процессор,  поскольку одни и те же обращения к операционной среде могут вызвать работы разной длительности, а значит потребовать разное время на свое выполнение.

Кэшированность инструкций и данных, а также длина хэш-списков влияют на действительное время пребывания в операционной среде. Утрачивает смысл понятие одновременность действий, поскольку  нельзя установить, выполнили ли два разных процесса какие-либо действия одновременно или одно из них предшествовало другому. Таким образом, с процессом можно связать только его локальное время, которое линейно упорядочивает события,  происходившие в этом процессе.  Глобальное время, линейно упорядочивающее действия во всех процессах, отсутствует. Расстояние (в этом качестве используется время) между действиями оказывается случайной величиной.

Эти соображения важны, поскольку процессы в интересующих нас приложениях взаимодействуют и используют общие ресурсы. Для взаимодействия они используют средства синхронизации, предоставляемые операционной средой - например, наборы семафоров SVR4 (System V Release 4), POSIX-семафоры, бинарные семафоры и другие примитивы взаимного исключения (POSIX- mutual exclusion locks) и т.д. Подобные средства операционной среды, которые позволяют процессам синхронизировать свою деятельность друг с другом или сериализовать обращения к совместно используемым объектам,  будут ниже  называться ресурсами.

С каждым ресурсом связано свое локальное время, линейно упорядочивающее события в жизни ресурса. Например, в случае двоичных семафоров это создание семафора, а также его захват и освобождение процессом. Заметим, что событие - это не намерение процесса (например, захватить бинарный семафор), а сам факт захвата семафора процессом (т.е. успешное выполнение намерения). От изъявления намерения до его осуществления может многое произойти. Например, семафор, который хочет захватить рассматриваемый процесс, принадлежал другому процессу, потом тот процесс его освободил, но семафор был сначала передан операционной средой третьему процессу, который также на него претендовал, и т.д. Поведение рассматриваемого процесса в это время нас не интересует - он ресурсом еще не овладел, а только его захват определяет его дальнейшее поведение. По причинам,  изложенным выше, расстояние между двумя событиями - случайная величина. Однако, события замечательны тем, что они одновременно присутствуют и в локальном времени процесса, и в локальном времени ресурса. Поэтому все, что произошло в истории процесса или/и ресурса до этого события, предшествует ему. Далее  будет считаться, что истории и ресурсов и процессов состоят только из событий, причем между двумя последовательными событиями в жизни процесса последний ведет себя детерминированно.

Это означает, что на  поведении процесса сказывается только его предыдущая история, то есть состояние ресурсов, с которыми он взаимодействовал. Это свойство процессов ниже будет называться локальной детерминированностью. Этим свойством не обладают ресурсы, поскольку - следующее событие в истории ресурса не определяется однозначно по его предыдущей истории. Утверждение, касающееся детерминированного поведения процессов, неявно опирается на предположение,  что учтены все ресурсы, которые могут привести к  недетерминированности процессов.

Таким образом, описанное нами очень неформально время в многопроцессном комплексе представляет собой отношение частичного порядка, введенное на множестве событий. Зная полное состояние комплекса в некоторый момент времени,  нельзя однозначно определить, какое событие в истории ресурса наступит следующим. Можно говорить только о вероятности наступления того или иного события. Недетерминированность поведения есть следствие двух обстоятельств. Во-первых, это неопределенность времени, которое тратит процесс на переход от одного события к другому. Во-вторых, конкуренция процессов за общие ресурсы.

Выполнение приложения, на множестве событий которого введена частичная упорядоченность, можно описать направленным ациклическим графом выполнения. Вершинами этого графа являются события, с каждым  из которых связаны две входящие в него дуги. Одна дуга начинается в событии, которое непосредственно предшествует данному событию в истории процесса, другая - в истории ресурса.

Построение средств обеспечения прозрачной отказоустойчивости приложений опирается на следующее утверждение: для восстановления работы приложения после отказа достаточно располагать:
\begin{itemize}
\item контрольной точкой, которая отражает на некоторый момент времени состояния процессов и других ресурсов, образующих приложение;
\item графом выполнения приложения, который описывает работу приложения, начинающуюся с контрольной точки и заканчивающуюся отказом. Данные, которые нужны для построения графа выполнения, далее называются протоколом.
\end{itemize}
\begin{figure*} %fig1
\vspace*{1pt}
\begin{center}
\mbox{%
\epsfxsize=1.6in
\epsfxsize=100mm
\epsfbox{BbR-1.eps}
}
\end{center}
\vspace*{-9pt}
\Caption{Базовый вариант трубы с разными выходными устройствами
(цилиндрическое, расширяющееся и сужающееся сопло)
\label{f1bab}}
\vspace*{-3pt}
\end{figure*}

Вся эта информация должна находиться в стабильной памяти, не разрушающейся при отказе.

Ниже неформально описан алгоритм восстановления работы приложения после отказа, который опирается на наличие контрольной точки и графа выполнения. Будем считать, что в распоряжении имеются средства, позволяющие остановить процесс в тот момент, когда он намерен совершить некоторую операцию над ресурсом. Заметим, что событие в графе выполнения соответствует не изъявлению намерения, а его удовлетворению, то есть завершению выполнения операции.

Предварительно сделаем следующее:
\begin{itemize}
\item используя контрольную точку, приведем приложение в состояние, соответствующее этой контрольной точке;
\item в графе выполнения пометим все вершины (события) как "не наступившие". У некоторых вершин графа отсутствуют им непосредственно предшествующие; соответствующие события наступили сразу же после создания контрольной точки. Для каждой такой вершины включим в граф дополнительную вершину, ей предшествующую в истории процесса, и отметим эту дополнительную вершину как "наступившую";
\item разрешим процессам приложения выполняться.
\end{itemize}

Пусть некоторый процесс проявляет намерение выполнить операцию над каким-либо ресурсом. Отыщем для этого процесса в его истории последнее наступившее событие. Следующее в его истории событие - это то, которое соответствует требуемой операции. Посмотрим, наступило ли событие в истории ресурса, которое ему предшествует. Если нет, переведем процесс в состояния ожидания, отметив в предшествующем событии, что данный процесс ожидает его наступления. Если да, разрешим процессу выполняться, то есть выполнить операцию над ресурсом.

Пусть некоторый процесс объявляет, что он выполнил операцию над каким-либо ресурсом (это соответствует моменту протоколирования при оригинальном выполнении). Отыщем для этого процесса в его истории последнее наступившее событие и перейдем к следующему событию в его истории. Это опять то событие, которое мы рассматриваем. Отметим его как "наступившее". Если наступления этого события ожидал какой-нибудь процесс, выведем этот процесс из состояния ожидания. Наконец, разрешим процессу, выполнившему операцию, продолжаться дальше.

Когда выясняется, что наступили все события графа выполнения, повторное выполнение считается законченным.

Естественным следствием из сказанного является следующее утверждение: для того, чтобы размер протокола не рос неограниченно, нужно периодически создавать контрольные точки, очищая при этом протокол.

\section{ФОРМАЛЬНОЕ ОПИСАНИЕ МОДЕЛИ ПОВЕДЕНИЯ МНОГОПРОЦЕССНОГО ПРИЛОЖЕНИЯ}
\begin{figure*} %fig1
\vspace*{1pt}
\begin{center}
\mbox{%
\epsfxsize=1.6in
\epsfxsize=100mm
\epsfbox{BbR-1.eps}
}
\end{center}
\vspace*{-9pt}
\Caption{Базовый вариант трубы с разными выходными устройствами
(цилиндрическое, расширяющееся и сужающееся сопло)
\label{f1bab}}
\vspace*{-3pt}
\end{figure*}

Опишем формально поведение приложения, неформальное описание которого было приведено выше. Рассматриваются два типа объектов:
\begin{itemize}
\item ресурсы (r), например, наборы семафоров (POSIX- или SVR4-семафоры), бинарные семафоры (POSIX-mutex's), таймер реального времени, сокеты (sockets), то есть двусторонние виртуальные соединения с внешним миром;
\item процессы (p), например, процессы или потоки (threads) пользователя.
\end{itemize}

\end{multicols}

\label{end\stat}

%\def\stat{batr}

\def\tit{НОВЫЙ МЕТОД ВЕРОЯТНОСТНО-СТАТИСТИЧЕСКОГО\newline
АНАЛИЗА ИНФОРМАЦИОННЫХ ПОТОКОВ
В~ТЕЛЕКОММУНИКАЦИОННЫХ СЕТЯХ$^*$}
\def\titkol{Новый метод вероятностно-статистического
анализа информационных потоков
в~телекоммуникационных сетях}
\def\autkol{Д.\,А.~Батракова, В.\,Ю.~Королев, С.\,Я.~Шоргин}
\def\aut{Д.\,А.~Батракова$^1$, В.\,Ю.~Королев$^2$, С.\,Я.~Шоргин$^3$}

\titel{\tit}{\aut}{\autkol}{\titkol}

{\renewcommand{\thefootnote}{\fnsymbol{footnote}}\footnotetext[1]{Работа 
выполнена при поддержке РФФИ, проекты №№\,04-01-00671, 05-07-90103.} 
\renewcommand{\thefootnote}{\arabic{footnote}}}
 \footnotetext[1]{ИПИ РАН, 
daria.batrakova@gmail.com} \footnotetext[2]{Факультет вычислительной математики 
и кибернетики МГУ им.~М.\,В.~Ломоносова, ИПИ РАН, vkorolev@comtv.ru} 
\footnotetext[3]{ИПИ РАН, sshorgin@ipiran.ru}



\Abst{В данной работе предлагается метод исследования стохастической структуры
хаотических информационных потоков в сложных телекоммуникационных
сетях. Предлагаемый метод основан на стохастической модели
телекоммуникационной сети, в рамках которой она представляется в виде
суперпозиции некоторых простых последовательно-параллельных структур.
Эта модель естественно порождает смеси гамма-распределений для времени
выполнения (обработки) запроса сетью. Параметры получаемой смеси
гамма-распределений характеризуют стохастическую структуру
информационных потоков в сети. Для решения задачи статистического
оценивания параметров смесей экспоненциальных и гамма-распределений
(задачи разделения смесей) используется ЕМ-алгоритм. Чтобы проследить
изменение стохастической структуры информационных потоков во времени,
ЕМ-алгоритм применяется в режиме скользящего окна. Описывается
программный инструментарий для применения полученного решения к
реальным статистическим данным. Приводится интерпретация результатов.}

\KW{телекоммуникационные сети; информационные потоки;
разделение смесей  распределений;
метод скользящего окна;  программа для разделения смесей}

\vskip 24pt plus 9pt minus 6pt

\thispagestyle{headings}

\begin{multicols}{2}


\label{st\stat}

\section{Введение}

Развитие телекоммуникационных сетей, их усложнение поставило перед
инженерами важную прикладную задачу исследования характеристик
информационных потоков, возникающих в этих сетях. Здесь под
информационным потоком мы будем понимать упорядоченное движение
любого вида информации по сети.

Если на заре эры телекоммуникаций, в эпоху первых телефонных линий и
телеграфа эта проблема не была столь насущной, то со временем, при
постепенном охвате мирового пространства сетями возникла необходимость в
построении и исследовании адекватных моделей сетей и процессов,
происходящих в них.

\thispagestyle{headings}


Современные сети~--- это \textit{конвергентные} сети, т.е.\ совокупность крайне
разнородных как по топологии, так и по физической архитектуре сетей, которые
предлагают конечному пользователю самые разнообразные сервисы. Это~--- огромное
виртуальное и физическое пространство, состоящее из миллионов процессоров,
операционных платформ, линий передачи данных и стыковочных узлов.
%
Существует множество классификаций телекоммуникационных сетей по различным
признакам:
\begin{itemize}
\item масштабу (локальные сети~--- LAN, масштаба города~---
MAN, широкого масштаба~--- WAN);
\item топологии, или логической организации (<<звезда>>,
<<кольцо>>, <<шина>>);
\item физической организации (оптические сети, радио);
\item предлагаемым услугам (сотовые сети, для доступа в
Интернет);
\item назначению (военные, гражданские) и~др.
\end{itemize}


Конвергентная сеть входит во все эти классы, причем, как правило,
обладает всеми этими признаками. Поэтому построение модели для ее анализа
является и очень важной, и очень сложной задачей.

Существуют достаточно многочисленные математические методы, ориентированные на
моделирование и анализ телекоммуникационных сетей. В~большинстве своем они
основываются на теории массового обслуживания, то есть разделе теории
вероятностей, посвященном описанию функционирования сложных систем обслуживания
(в том чис\-ле телекоммуникационных сетей и систем) с помощью стохастических
процессов особого вида и анализу таких процессов. Указанная теория является
весьма развитой и широко применяется на практике. Тем не менее, ее применимость
ограничена~--- во-первых, все возрастающей сложностью структур и дисциплин
обслуживания в рас\-смат\-ри\-ва\-емых реальных сетях. Эта сложность во многих
случаях принципиально не может найти адекватного отображения в моделях
массового обслуживания, даже несмотря на постоянно растущую сложность самих
этих моделей. В результате даже модели, допускающие точный математический
анализ, дают возможность расчета всего лишь приближенных значений характеристик
реальных сетей, ибо предположения, принимаемые при построении моделей, во
многих случаях не соответствуют практике. Во-вторых, для описания
телекоммуникационной сети в виде сети массового обслуживания исследователь
должен располагать детальным описанием структуры сети, что далеко не всегда
имеет мес\-то на практике. В-третьих, разработано крайне мало моделей массового
обслуживания, в которых используется в качестве входной информация о
наблюдаемых (статистических) показателях функционирования сети; в то же время,
такая информация очень часто доступна исследователю, и ее использование при
анализе сети весьма целесообразно.

В данной работе предлагается в определенной степени альтернативный подход к
моделированию сложных телекоммуникационных сетей. Строится и исследуется
вероятностная модель сложной телекоммуникационной сети как суперпозиции
достаточно простых структур. При этом практически никакая априорная информация
о структуре исследуемой сети не используется~--- наоборот, в результате
исследования модели исследователь получает приближенное представление об этой
структуре. Характеристики типовых простых структур, составляющих в совокупности
модель сети, и сети в целом при этом подходе описываются
гам\-ма-рас\-пре\-де\-ле\-ни\-я\-ми. Ставится задача оценки параметров модели
на основе статистических данных о функционировании сети, а также предлагается
математическое решение этой задачи. В статье описан также созданный на основе
разработанных математических методов программный инструментарий и приведены
результаты расчетов для реального трафика. {\looseness=-1

}

\section{Математическая модель и~постановка задачи}

\subsection{Логическая модель сети}
 %1.1

Рассмотрим абстрактную \textit{конвергентную телекоммуникационную
сеть}. Это может быть как крупномасштабная транспортная сеть (WAN), сеть
отдельного оператора масштаба города (MAN) с различными сервисами, так и
локальная сеть (LAN).

Любой из этих случаев можно описать как ($p,\,q$)-\textit{сеть}.

\medskip
\textbf{Определение 1.} В теории графов и сетей под ($p,\,q)$-сетью понимается
набор вида $S =$\linebreak $=(G,\,V^\prime ,\,V^{\prime\prime})$, где $G$~---
граф, а $V^\prime$ и $V^{\prime\prime}$~--- выборки из множества $V(G)$ (вершин
графа) длины~$p$ и $q$ соответственно. При этом выборка $V^\prime$
($V^{\prime\prime}$) считается \textit{входной} (\textit{выходной}) выборкой, а
ее $i$-я вершина называется $i$-\textit{м} \textit{входным} (\textit{выходным})
\textit{полюсом} или, иначе, $i$-\textit{м} \textit{входом} (\textit{выходом})
сети~$S$. Вершины, не участвующие во входной и выходной выборках сети,
считаются ее внутренними вершинами~\cite{1bat}.

Сеть $S$ (рис.~\ref{f1bat}) имеет $p$ точек входа~--- точек соединения
с внешней средой (это могут быть точки стыковки разнородных сетей, сетей
различных операторов, физические подключения к интерфейсам
маршрутизаторов и~т.п.). Под \textit{внешней средой} будем понимать другие
сети, которые передают данные в сеть~$S$. Отдельные <<единицы>> данных
(кадры, сообщения, датаграммы, пакеты) поступают на входы сети~$S$,
обрабатываются и подаются на каждый из $q$ выходов, которые могут быть
соединены как с конечными пользователями, так и с другими сетями.
\begin{figure*} %fig1
\vspace*{1pt}
\begin{center}
\mbox{%
\epsfxsize=139.7mm \epsfbox{bat-1.eps}
%\epsfxsize=139.698mm
%\epsfbox{bek-3.eps}
}
\end{center}
\vspace*{-9pt} \Caption{Абстрактная телекоммуникационная сеть \label{f1bat}}
\end{figure*}

Следует отметить, что структура сложных телекоммуникационных сетей обладает
свойством некоторого самоподобия, т.е.\ на каком бы уровне сетевой архитектуры
мы ни рассматривали поведение информационных потоков, мы можем выделить
некоторые базовые структуры, субпотоки, суперпозицией которых мы можем получить
модель конкретной сети, какой бы уровень <<детализации>> сегментов сети мы ни
взяли. Так, например, физические подключения к интерфейсам оптического
кросс-коннекта в этом смысле подобны <<виртуальным>> подключениям к портам TCP
на сервере приложений.

Итак, независимо от уровня сетевой архитектуры мы можем
рассматривать некоторую величину, характеризующую количество каких-либо
ресурсов сети~$S$, занимаемых в процессе передачи и обработки данных.  Это
могут быть ресурсы, относящиеся как к <<объему>> (памяти сетевого
устройства, количеству занятых линий, размеру пакета), так и ко <<времени>>
(времени обслуживания заявки, времени простоя). Эта величина случайна, т.к.\
мы не можем абсолютно точно сказать для сложной телекоммуникационной
сети, какое сообщение на какой из входов поступит и какого размера оно будет.
Таким образом, случайный характер данной величины определяется
случайностью поведения внешней среды.

Пусть $R$~--- это описанная выше случайная величина, $R>0$. Далее, не
ограничивая общности, будем подразумевать под ней время, необходимое для
какой-либо операции сети (обработки <<единицы>> данных), предполагая, что
время обработки прямо зависит от объема сообщения.

\subsection{Вероятностная модель сети} %1.2.

Даже не зная реальной топологии сети, мы можем описать
функционирование некоторых ее участков как процесс выполнения операций
(задач сети) в последовательном  порядке (например, если доступен только
один канал данных) или как процесс одновременного выполнения субопераций
(когда доступно более одного пути выполнения). Это значит, что мы можем
представить функционирование сложной телекоммуникационной сети как
\textit{суперпозицию} таких <<последовательных>> и <<параллельных>>
блоков.

Для построения вероятностной модели распределения~$R$ используется
комбинация асимптотического подхода, основанного на предельных теоремах
теории вероятностей, и принципа максимальной неопределенности (энтропии).

Рассмотрим следующую модель. Предположим, что мы можем разделить
сеть~$S$ на несколько сегментов $S_i$. Пусть $T$~--- случайная величина,
время выполнения операции в отдельно взятом блоке $S_i$ (сегменте сети).

Если операции выполняются \textit{параллельно}, то время, необходимое
для их выполнения~--- это максимальное время, затрачиваемое на какую-либо
субоперацию:
$$
T = \underset{i}{\max}\, T_i\,,
$$
где $T_i$~--- случайные величины для со\-от\-вет\-ст\-ву\-ющих субопераций.
Модель такого выполнения пред\-став\-ле\-на на рис.~\ref{f2bat}.

\begin{figure*} %fig2
\vspace*{1pt}
\begin{center}
\mbox{%
\epsfxsize=117.271mm
\epsfbox{bat-2.eps}
}
\end{center}
\vspace*{-9pt}
\Caption{Параллельное выполнение
\label{f2bat}}
\end{figure*}

Известно, что предельное распределение экстремальных значений для
выборок ~--- это экспоненциальное распределение с плотностью~\cite{2bat}
$$
f(x) =
\begin{cases}
\lambda e^{-\lambda x}\,, & x>0\,,\\
0\,, & x\leq 0\,,
\end{cases}
$$
где $\lambda >0$~--- параметр масштаба.

 Учитывая также энтропийный подход, естественно будет считать
распределение $T$ экспоненциальным, т.к.\ экспоненциальное распределение
обладает наибольшей энтропией среди всех распределений с $x>0$.

Если же операции сети выполняются \textit{последовательно}, то величина
$T$~--- это сумма времен $T_i$, необходимых для выполнения каждой
субоперации:
$$
T = \sum\limits_i T_i\,,
$$
где $T_i$~--- случайные величины для со\-от\-вет\-ст\-ву\-ющих субопераций.
%
Такая модель представлена на рис.~\ref{f3bat}.

\begin{figure*} %fig3
\vspace*{1pt}
\begin{center}
\mbox{%
\epsfxsize=139.592mm
\epsfbox{bat-3.eps}
}
\end{center}
\vspace*{-9pt}
\Caption{Последовательное  выполнение
\label{f3bat}}
\end{figure*}

Это значит, что распределение общей длительности $T$ выполнения
блока~--- это свертка распределений <<элементарных>> времен $T_i$
(экспоненциально распределенных).

Известно, что результатом свертки экспоненциальных распределений
является гамма-распределение, определяемое плотностью
$$
\g_{\lambda , \alpha} (x) =
\begin{cases}
\fr{\lambda_0^{\alpha_0}}{\Gamma (\alpha_0 )}\,x^{\alpha_0-1}
e^{\lambda_0 x}\,, & x>0\,,\\
0,\, & x\leq 0\,,
\end{cases}
$$
где $\alpha >0$~--- параметр формы,  $\lambda >0$  параметр масштаба, а
$\Gamma (z)$~--- гамма-функция Эйлера:
$$
\Gamma (z) = \int\limits_0^\infty x^{z-1} e^{-x}\,dx\,.
$$

\begin{figure*} %fig4
\vspace*{1pt}
\begin{center}
\mbox{%
\epsfxsize=120.831mm
\epsfbox{bat-4.eps}
}
\end{center}
\vspace*{-9pt}
\Caption{Модель пути  обработки сообщения сетью~$S$
\label{f4bat}}
\end{figure*}

Известно~\cite{2bat}, что класс гамма-распределений замкнут над операцией
свертки, поэтому ре\-зуль\-ти\-ру\-ющее распределение случайной величины~$R$
будет также гамма-распределением
$$
\g_{\lambda , \alpha} (x) =
\begin{cases}
\fr{\lambda^{\alpha}}{\Gamma (\alpha )}\,x^{\alpha -1} e^{-\lambda x}\,, &
x>0\,,\\
0,\, & x\leq 0\,.
\end{cases}
$$

В силу случайного характера ввода данных в сеть~$S$ из внешней среды маршрут
передачи данных становится случайным, что представлено на рис.~\ref{f4bat}. Это
означает, что параметры ре\-зуль\-ти\-ру\-юще\-го распределения~$R$ тоже
случайны. Отсюда имеем следующую модель \textit{смеси
гам\-ма-рас\-пре\-де\-ле\-ний}, определяемой плотностью

\begin{equation} %1
p(x) = \iint \g_{\lambda , \alpha}(x)\,dH (\lambda ,\,\alpha )\,,
\end{equation}
где $H(\lambda , \alpha )$~--- смешивающая функция, функция распределения
входных параметров.

Поясним понятие \textit{смеси распределений}.

\medskip
\textbf{Определение~2.} Пусть имеется двух\-па\-ра\-мет\-ри\-че\-ское
семейство $n$-мерных плотностей  распределения
\begin{equation}
F = \{ f_\omega (x;\, \theta (\omega ))\}\,,
\end{equation}
где одномерный (целочисленный или непрерывный) параметр $\omega$ в
качестве нижнего индекса функции $f$ определяет специфику общего вида
каж\-до\-го компонента~--- распределения смеси, а в качестве аргумента при
многомерном, вообще говоря, параметре $\theta$ определяет зависимость
значений хотя бы части компонентов этого параметра от того, в каком именно
составляющем распределении $f_\omega$ он присутствует. Кроме того, пусть
$P = \{P(\omega )\}$~--- \textit{семейство смешивающих функций}
распределения.

Функция плотности распределения
\begin{equation}
f(x) = \int f_\omega (x;\,\theta(\omega ))\,dP (\omega )
\end{equation}
называется $P$-\textit{смесью} (или просто \textit{смесью})
\textit{распределений} семейства~$F$,  интеграл в~(3) понимается в смысле
Лебега--Стильтьеса~\cite{3bat}.

\medskip
\textbf{Определение 3.} Семейство смесей~(3) называется
\textit{идентифицируемым} (\textit{различимым}), если из равенства
$$
\int f_\omega (x;\,\theta(\omega ))\,dP (\omega ) =\int f_\omega
(x,\,\theta(\omega )) dP^*(\omega )
$$
следует, что $P(\omega ) \equiv P^*(\omega )$ для всех $P \in P(\omega
)$~\cite{3bat}.

\subsection{Постановка задачи} %1.3.

Перед нами встает задача \textit{разделения} такой смеси. Вообще говоря,
задача разделения смесей распределений со смешивающими функциями
общего вида является \textit{некорректно поставленной}, т.к.\ она допускает
существование нескольких решений. Поэтому будем искать решение в классе
\textit{конечных идентифицируемых смесей распределений}, где смешивающая
функция дискретна.

Для этого сузим данное выше определение и будем рассматривать в дальнейшем лишь 
случай конечного числа $k$ возможных значений па\-ра\-мет\-ра~$\omega$, что 
соответствует конечному числу скачков смешивающих функций $P(\omega )$.  
Величины этих скачков как раз и будут играть роль \textit{удельных весов} 
(\textit{априорных вероятностей}) $p_j$ компонентов смеси. Более того, в нашем 
случае мы постулируем также однотипность компонентов распределений $f_j$, т.е.\ 
принадлежность всех $f_j$ к одному общему па\-ра\-мет\-ри\-че\-ско\-му 
семейству $\{ f(X;\,\theta )\}$, где $\theta$~--- многомерный, вообще говоря, 
параметр. Так что~(3) в этом случае может быть записано в виде
\begin{equation} %4
p(x) = \sum\limits^k_{j=1} p_j f_j (x;\,\theta_j )\,.
\end{equation}

Переформулируем понятие идентифицируемости (различимости) смесей
специально применительно к такому виду смесей.

\medskip
\textbf{Определение 4.} \textit{Конечная смесь}~(3) называется
\textit{идентифицируемой} (\textit{различимой}), если из равенства
$$
\sum\limits_{j=1}^k p_j f_j (x;\,\theta_j ) = \sum\limits_{l=1}^{k^*} p_l^* f_l
(x;\,\theta_l^* )
$$
следует, что $k=k^*$ и для любого $j$ ($1\leq j \leq k$) найдется такое $l$ 
($1\leq l \leq k^*$), что $p_j = p_l^*$ и $f_j (x;\,\theta_j ) = f_l 
(x;\,\theta_l^* )$~\cite{3bat}.

Решить эту задачу в выборочном варианте~--- значит по выборке
классифицируемых наблюдений
$X_1,\ldots , X_n, $ извлеченной из генеральной совокупности, яв\-ля\-ющей\-ся смесью~(3)
генеральных совокупностей типа~(2) (при заданном общем виде составляющих
смесь функций $f_j (x;\,\theta_j )$), построить статистические оценки для числа
компонентов смеси~$k$, их удельных весов $p_j$ и, главное, для каждого из
компонентов %f_j (x;\,\theta_j )$ анализируемой смеси. Далее будет считать, что
функции $f_j$ однозначно определяются своими параметрами $\theta_j$: $f_j
=f(x;\,\theta_j)$.

Однако не следует ставить знак тождества между задачей разделения смеси
и задачей статистического оценивания параметров в модели~(4) по выборке $
X_1,\ldots , X_n$, поскольку задача разделения сохраняет смысл и
применительно к генеральным совокупностям, т.е.\ в теоретическом
варианте~\cite{3bat}.

Итак, для статистического анализа на основе реальных данных мы
аппроксимируем нашу общую модель~(1) следующей:
$$
p(x) \approx \hat{p}(x) = \sum\limits_{j=1}^k p_j \g_{\lambda_j , \alpha_j}
(x)\,,
$$
где $p_j$~--- дискретные смешивающие параметры, $\g_{\lambda_j , \alpha_j}
(x)$~--- плотности гамма-распределений.

Такая аппроксимация не только позволяет решить поставленную статистическую
задачу, но и полу\-чить наглядную визуализацию результатов. Существуют
достаточно эффективные методики разделения смесей распределений, среди них~---
семейство так называемых \textit{ЕМ-алгоритмов}
(\textit{Expectation-Maximization Algorithms}).

Полученные результаты могут быть достаточно просто интерпретированы. Число
компонентов смеси символизирует число типичных параллельных или
последовательных структур. Значения параметров составляющих смесь
гам\-ма-рас\-пре\-де\-ле\-ний показывают <<степень параллелизма>>
соответствующей структуры: чем ближе параметр формы к~1, тем выше эта
<<степень>>. И наоборот, чем дальше значение параметра формы от~1, тем больше
последовательных операций выполняется в соответствующем блоке.

Веса компонентов характеризуют примерную долю использования
ресурсов для сообщений, соответствующих каждому распределению входных
данных.

Итак, предложенный подход позволяет получить представление о
стохастической структуре телекоммуникационной сети.

\section{ЕМ-алгоритм разделения смесей распределений}

\subsection{Описание алгоритма} %2.1.

Определяемый ниже итерационный алгоритм решения поставленной в
предыдущем разделе задачи относится к процедурам, базирующимся на
\textit{методе максимального правдоподобия}.

Этот алгоритм позволяет находить максимум логарифмической функции
правдоподобия по параметрам $p_1,\,p_2,\ldots ,\,p_k$, $\theta_1 ,\,\theta_2,\ldots ,\,
\theta_k$ при фиксированном $k$ по выборке $X_1, \ldots , X_n$, т.е.\ решение
оптимизационной задачи вида

\begin{equation} \sum\limits_{i=1}^n \ln \left ( \sum\limits_{j=1}^k p_j f_j
(X_i;\,\theta_j )\right ) \rightarrow \underset{p_j,\,\theta_j}{\max}\,.
\end{equation}

Конкретные алгоритмы, построенные по этой схеме, часто называют
\textit{алгоритмами типа ЕМ}, поскольку в каждом из них можно выделить два
этапа, находящихся по отношению друг к другу в последовательности
итерационного взаимодействия: \textit{оценивание} (\textit{Estimation}) и
\textit{максимизация} (\textit{Maximization})~\cite{4bat}.

Введем в рассмотрение так называемые апостериорные вероятности
$\g_{ij}$ принадлежности наблюдения $X_i$ к $j$-му классу:
\begin{equation} %6
\g_{ij} = \fr{p_j f(X_i;\,\theta_j )}{\sum\limits_{l=1}^k p_l f(X_i;\,\theta_l 
)} \ (i=1,\ldots , n;\ j=1,\ldots ,k)\,.\!\!\end{equation} 
Очевидно, что для 
всех $i=1,\ldots ,n$ и $j=1,\ldots ,k$
$$
\g_{ij} \geq 0,\quad \sum_{j=1}^k \g_{ij} =1\,.
$$


Далее обозначим $\Theta = (p_1,\ldots p_k,\,\theta_1,\ldots ,\theta_k )$ и
представим анализируемую логарифмическую функцию правдоподобия
$$
\ln L(\Theta ) = \sum\limits_{i=1}^n \ln \left (\sum\limits_{j=1}^k p_j f_j
(X_i;\,\theta_j )\right )
$$
в виде
\begin{multline}
\ln L (\Theta ) = \sum\limits_{j=1}^k\sum\limits_{i=1}^n \g_{ij} \ln p_j+{}\\
{}+\sum\limits_{j=1}^k\sum\limits_{i=1}^n \g_{ij} f(X_i;\,\theta_j)-
\sum\limits_{j=1}^k\sum\limits_{i=1}^n \g_{ij} \ln \g_{ij}\,.
\end{multline}

Справедливость этого тождества легко проверяется с учетом
$$
\sum\limits_{j=1}^k \g_{ij} =1\,.
$$

Далее идея построения итерационного алгоритма вычисления оценок
$\hat{\Theta} = (\hat{p}_1,\ldots , \hat{p}_k,\
\hat{\theta}_1,\ldots , \hat{\theta}_k)$
для параметров $\Theta = (p_1,\ldots , p_k,\ \theta_1,\ldots , \theta_k)$ состоит в
следующем:
\begin{enumerate}[1.]
\item Выбирается некоторое \textit{начальное приближение}~$\hat{\Theta}^0$.
\item \textbf{E-step:} вычисляются по формулам~(6) начальные приближения
$\g_{ij}^0$ для апостериорных вероятностей $\g_{ij}$~--- \textit{этап
оценивания}.
\item \textbf{M-step:} затем, возвращаясь к~(7), при вычисленных значениях
$\g^0_{ij}$ следует определить значения $\hat{\Theta}^1$ из условия
максимизации отдельно каждого из первых двух слагаемых правой
части~(7), поскольку первое слагаемое
$$
\sum_{j=1}^k \sum_{i=1}^n \g_{ij} \ln p_j
$$
зависит только от параметров $p_j$, а второе слагаемое
$$
\sum_{j=1}^k \sum_{i=1}^n \g_{ij} f(X_i;\,\theta_j )
$$
зависит только от параметров $\theta_j$~--- \textit{этап максимизации}.
\item Проверяется \textit{условие останова}:
$$
\parallel \Theta^{(t)} - \Theta^{t-1}\parallel <\varepsilon\,,
$$
где $t$~--- номер итерации, а
$\parallel\bullet\parallel$~--- евклидова норма, для некоторого $\varepsilon
>0$.
\end{enumerate}

Очевидно, решение оптимизационной задачи
$$
\sum\limits_{j=1}^k\sum\limits_{i=1}^n \g_{ij}^{(t)}\ln p_j \rightarrow
\underset{p_j}{\max}
$$
дается выражением (с учетом $\sum_{j=1}^k p_j =1$):
$$
p_{ij}^{(t+1)} =\fr{1}{n}\,\sum\limits_{i=1}^n \g_{ij}^{(t)}\,,
$$
где $t$~--- номер итерации, $t = 0$, 1, 2,\,\ldots

Решение оптимизационной задачи
$$
\sum\limits_{j=1}^k \sum\limits_{i=1}^n \g_{ij}^{(t)} f(X_i;\,\theta_j )
\rightarrow \underset{\theta_j}{\max}
$$
получить намного проще решения задачи~(5): выражение для $\theta_j$
записывается с учетом знания конкретного вида функций
$f(X,\,\theta)$~\cite{3bat}.

\subsection{О сходимости алгоритма} %2.2.

В работе М.\,И.~Шлезингера~\cite{5bat}, где эта схема (позднее названная
ЕМ-схемой) впервые предложена, установлены и основные свойства
реа\-ли\-зу\-ющих ее алгоритмов. В частности, было доказано, что при достаточно
широких предположениях \textit{предельные точки} всякой последовательности,
порожденной итерациями ЕМ-алгоритма, являются стационарными точками
оптимизируемой логарифмической функции правдоподобия $\ln L(\Theta )$ и что
найдется неподвижная точка алгоритма, к которой будет сходиться каждая из таких
последовательностей. Если дополнительно потребовать положительной
определенности информационной мат\-ри\-цы Фишера для $\ln L(\Theta )$ при
истинных зна\-че\-ни\-ях па\-ра\-мет\-ра $\Theta$, то можно показать, что
асимптотически по $n\rightarrow\infty$ (т.е.\ при больших выборках) существует
единственное сходящееся (по веро\-ят\-но\-сти) решение $\hat{\Theta}(n)$
уравнений метода максимального правдоподобия и, кроме того, существует в
пространстве параметров $\Theta$ норма, в которой последовательность
$\Theta^{(t)}(n)$, порожденная ЕМ-ал\-го\-рит\-мом, сходится к $\hat{\Theta}
(n)$, если только начальная аппроксимация $\hat{\Theta}^0$ не была слишком
далека от~$\hat{\Theta} (n)$. {%\looseness=1

}

Таким образом, результаты исследования свойств ЕМ-алгоритмов метода
максимального правдоподобия разделения смеси и их практическое
использование показали, что они являются достаточно работоспособными (при
известном чис\-ле компонентов смеси) даже при большом чис\-ле $k$ компонентов и
при высоких размерностях анализируемого признака~$X$~\cite{3bat}.

\subsection{Уравнения для смеси экспоненциальных распределений}
%2.3.

Применим описанный выше алгоритм к разделению смеси
экспоненциальных распределений:
$$
p(x) = \sum\limits_{j=1}^k p_j \lambda_j e^{-\lambda_j x}\,.
$$
Получаем следующие итерационные уравнения:
\begin{align*}
\g_{ij}^{(t+1)} & = \fr{p_j^{(t)}\lambda_j^{(t)}e^{-
\lambda_j^{(t)}X_i}}{\sum\limits_{l=1}^k p_l^{(t)}\lambda_l^{(t)}
e^{-\lambda_l^{(t)}X_i}}\,,\\
p_j^{(t+1)} & = \fr{1}{n}\,\sum\limits_{i=1}^n \g_{ij}^{(t)}\,.
\end{align*}

Чтобы найти  оценки $\lambda_j$, подсчитаем первую производную функции
$$\sum_{j=1}^k\sum_{i=1}^n \g_{ij}^{(t)} \ln (\lambda_j e^{-\lambda_j X_i}):$$
\vspace*{-8pt}
\begin{multline*}
\left ( \sum\limits_{j=1}^k \sum\limits_{i=1}^n
\g_{ij}^{(t)}\ln \left ( \lambda_j
e^{-\lambda_j X_i} \right ) \right )^\prime \lambda_j =\\[-3pt]
{}= \left (
\sum\limits_{j=1}^k\sum\limits_{i=1}^n \g_{ij}^{(t)}\ln (\lambda_j -\lambda_j X_i )
\right )^\prime \lambda_j =\\[-3pt]
{}= \sum\limits_{i=1}^n \g_{ij}^{(t)}\left (
\fr{1}{\lambda_j} - X_i \right )\,.
\end{multline*}

Разрешая уравнение
$$
\sum\limits_{i=1}^n \g_{ij}^{(t)}\left ( \fr{1}{\lambda_j} -X_i\right ) =0
$$
относительно $\lambda_j$, получаем следующее итерационное уравнение:
$$
\lambda_j^{(t+1)} = \fr{\sum\limits_{i=1}^n
\g_{ij}^{(t)}}{\sum\limits_{i=1}^n \g_{ij}^{(t)} X_i}\,.
$$

\subsection{Уравнения для смеси гамма-распределений } %2.4.

Применим теперь ЕМ-алгоритм к смеси гам\-ма-рас\-пре\-де\-ле\-ний вида
$$
p(x) = \sum\limits_{j=1}^k p_j \fr{\alpha_j^{\alpha_j} x^{\alpha_j -
1}}{\lambda_j^{\alpha_j} \Gamma (\alpha_j )}\,e^{-(\alpha_j / \lambda_j)x}\,.
$$

Такая параметризация удобна для нахождения
оценок~$\alpha_j$~\cite{6bat}.

Аналогичным способом выписываются итерационные уравнения:
\begin{align*}
\g_{ij}^{(t+1)} & =   \fr{p_j^{(t)}\fr{(\alpha_j^{\alpha_j} )^{(t)}
x^{\alpha_j - 1}}{(\lambda_j^{\alpha_j} )^{(t)}\Gamma (\alpha_j)}\,
e^{-(\alpha_j /\gamma_j)^{(t)}x}}{\sum\limits_{l=1}^k
p_l^{(t)}\fr{(\alpha_l^{\alpha_l})^{(t)} x^{\alpha_l -
1}}{(\lambda_l^{\alpha_l})^{(t)}\Gamma (\alpha_l )}\,
e^{-(\alpha_l /\lambda_l)^{(t)} x}}\,,\\
p_j^{(t+1)} & = \fr{1}{n}\,\sum\limits_{i=1}^n \g_{ij}^{(t)}\,.
\end{align*}

Далее найдем оценки $\lambda_j$ для данного случая, приравнивая
производную
\begin{equation} %8
\sum\limits_{j=1}^k \sum\limits_{i=1}^n \g_{ij}^{(t)} \ln \left (
\fr{\alpha_j^{\alpha_j} x^{\alpha_j -1}}{\lambda_j^{\alpha_j}\Gamma
(\alpha_j)}\,e^{-(\alpha_j /\lambda_j) x}\right )
\end{equation}
по $\lambda_j$ к нулю и разрешая относительно~$\lambda_j$ уравнение:
$$
\sum\limits_{i=1}^n \g_{ij}^{(t+1)}\left ( \fr{\alpha_j^{(t)}}{\lambda_j^{(t)}}
- \fr{\alpha_j^{(t)}X_i}{\left ( \lambda_j^{(t)}\right )^2}\right ) =0 \,.
$$
Получаем
$$
\lambda_j^{(t+1)} = \fr{\sum\limits_{i=1}^n \g_{ij}^{(t)}
X_i}{\sum\limits_{i=1}^n \g_{ij}^{(t)}}\,.
$$

Для того чтобы получить итерационные уравнения для $\alpha_j$, найдем
первую производную~(8):
\begin{multline*}
\left ( \sum\limits_{j=1}^k\sum\limits_{i=1}^n \g_{ij}^{(t)}\ln \left (
\fr{\alpha_j^{\alpha_j} x^{\alpha_j -1}}{\lambda_j^{\alpha_j}\Gamma (\alpha_j
)}\,e^{-(\alpha_j /\lambda_j ) x} \right ) \right )^\prime \alpha_j ={}\\[-3pt]
{}=\left ( \sum\limits_{j=1}^k\sum\limits_{i=1}^n \g_{ij}^{(t)}\ln \left (
\fr{\alpha_j^{\alpha_j}}{\lambda_j^{\alpha_j}}\right ) - \ln \Gamma (\alpha_j )+{} \right.\\[-3pt]
{}+\left.
(\alpha_j -1 )\ln X_i - \fr{\alpha_j}{\lambda_j}\,X_i \right )^\prime \alpha_j =\\[-3pt]
{}=\sum\limits_{i=1}^n \g_{ij}^{(t)} \left ( \ln \alpha_j +1-\ln \lambda_j -
\fr{\Gamma^\prime (\alpha_j )}{\Gamma (\alpha_j)}\right.+\\[-3pt]
{}+\left. \ln X_i - \fr{X_i}{\lambda_j}\right )\,;
\end{multline*}
\begin{multline*}
\sum\limits_{i=1}^n \g_{ij}^{(t)} \left(  \ln \alpha_j +1 -\ln \lambda_j -{}\right. \\[-3pt]
\left. {}-\fr{\Gamma^\prime (\alpha_j )}{\Gamma (\alpha_j )}+\ln X_i 
-\fr{X_i}{\lambda_j} \right) =0\,;
\end{multline*}
\begin{multline}
\fr{\Gamma^\prime (\alpha_j )}{\Gamma (\alpha_j )} ={}\\[-3pt]
{}= \fr{\sum\limits_{i=1}^n \g_{ij}^{(t)} \left ( \ln \alpha_j +1-\ln\lambda_j 
+\ln X_i -\fr{X_i}{\lambda_j} \right )}{\sum\limits_{i=1}^n \g_{ij}^{(t)}}\,.
\end{multline}
%
Здесь $\Gamma^\prime (\alpha_j ) / \Gamma (\alpha_j )$~--- это
\textit{логарифмическая производная гамма-функции}. Для нее существует так
называемое \textit{разложение Абрамовитца}--\textit{Стигана}~\cite{4bat}:
$$
\fr{\Gamma^\prime (\alpha ) }{ \Gamma (\alpha )} = \mathrm{log}\,\alpha -
\fr{1}{2\alpha }-\fr{1}{12\alpha^2 }+\ldots
$$

Подставим первые три члена разложения в~(9) и разрешим это уравнение
относительно~$\alpha_j$:
$$
\alpha_{ij}^{(t+1)} = \fr{\sum\limits_{i=1}^n
\g_{ij}^{(t+1)}}{2\sum\limits_{i=1}^n \g_{ij}^{(t +1)}\left ( \fr{X_i}{\lambda_j^{(t)}} -
\ln \fr{X_i}{\lambda_j^{(t)}} -1\right )}\,.
$$
В итоге получаем итерационные уравнения для ~$\alpha_j$.

\section{Описание программного обеспечения (программа~ЕМ)}

\subsection{Назначение программы} %3.1.

Разработанная авторами статьи программа ЕМ предназначена для решения задачи
разделения смесей экспоненциальных и гамма-распределений, поставленной в
разд.~2, с использованием ЕМ-ал\-го\-рит\-ма и формул, описанных в разд.~3.

\subsection{Инструменты разработки} %3.2.

Для создания программы была использована среда разработки Microsoft
Visual Studio .NET 2005 и объектно-ориентированный язык C\#. Для
визуализации результатов была использована свободно распространяемая
графическая библиотека ZedGraph~\cite{7bat}.


\subsection{Возможности  программы} %3.3.

\noindent
\begin{itemize}
\item Загрузка выборочных данных из текстового файла
\item Оценивание по выборке параметров смеси экспоненциальных
распределений
\item Оценивание по выборке параметров смеси гамма-распределений
\item Отслеживание изменений параметров смесей распределений во
времени в режиме <<скользящего окна>>
\item Построение гистограммы по выборке
\end{itemize}

\subsection{Входные и выходные данные. Функционирование
программы} %3.4.

В качестве \textit{входных данных} программа ЕМ получает:
\begin{itemize}
\item выборочные данные из текстового файла;
\item число компонентов смеси;
\item размер <<скользящего окна>>;
\item размер класса гистограммы.
\end{itemize}

На \textit{выходе} мы получаем:
\begin{itemize}
\item точечные оценки параметров смеси экспоненциальных
распределений;
\item точечные оценки параметров смеси гамма-распределений;
\item графическое изображение результирующей смеси распределения;
\item графическое изображение компонентов каж\-дой смеси;
\item графическое изображение того, как меняются параметры смесей
распределений с течением времени в режиме <<скользящего окна>>;
\item гистограмма, построенная по выборке;
\item значение статистического теста.
\end{itemize}

Выборочные данные загружаются из текстового файла в память программы и подаются
на вход двум независимо работающим реализациям ЕМ-алгоритма~--- для
идентификации смеси экспоненциальных распределений и для идентификации смеси
гамма-распределений. Результатом их работы являются наборы значений оцениваемых
параметров модели, предложенной в разд.~2. Кроме того, результирующие
распределения визуализируются в виде графиков. В программе можно запустить
режим <<скользящего окна>>, который для всех подвыборок заданного
размера с помощью ЕМ-алгоритма оценивает параметры смесей распределений этих
подвыборок. Все действия программы документируются в окне информации.

\section{Описание тестовых расчетов}

С использованием разработанной программы были проведены тестовые
расчеты на выборочных данных реального сетевого трафика.

На вход программы EM были поданы выборки трафика:
\begin{enumerate}[I]
\item Между лабораторией Lawrence Berkeley (Berkeley, California) и
внешним миром размера примерно 7000~\cite{8bat}~--- \textit{выборка~1}.
\item
Сети радиодоступа ЗАО <<Синтерра>> размера примерно 1000~\cite{9bat}~---
 \textit{выборка~2}.
\end{enumerate}

\subsection{Выборка 1 ``Berkeley''} %5.1.

При числе компонентов смеси~5 и случайном начальном приближении
были получены результаты, представленные в табл.~\ref{t1bat}.


Данные результаты иллюстрирует рис.~\ref{f5bat}.

Гистограмма  на рис.~\ref{f6bat} показывает статистическую значимость
полученных результатов.

Данная выборка обладает той особенностью, что она собиралась в течение
достаточно длительного времени и в ней агрегирован самый разнородный
трафик. Поэтому в ней присутствует не только большое количество
<<коротких>> сообщений (что обычно для выборок из телетрафика), но и
некоторый массив сообщений средней длины, а также определенный
<<выброс>> больших сообщений. Это свидетельствует о \textit{пиковости}
телетрафика на довольно больших промежутках времени.

Как мы видим, ЕМ-алгоритм удачно справился с задачей идентификации
смеси.

\subsection{Выборка~2 ``Synterra''} %5.2.

Результаты применения ЕМ-алгоритма к выборке ``Synterra''
представлены в табл.~\ref{t2bat}.
\begin{table*}\small
\begin{minipage}[t]{76mm}
\begin{center}
\Caption{Результаты применения ЕМ-алго\-рит\-ма к выборке~1 ``Berkeley'' 
\label{t1bat}} \vspace*{2ex}

\tabcolsep=8.7pt
\begin{tabular}{|c|c|c|}
\hline
№&Начальное приближение&Результат\\
\hline
\multicolumn{3}{|c|}{$P$}\\
\hline
0&0,2&0,1896\\
1&0,2&0,1858\\
2&0,2&0,1830\\
3&0,2&0,2259\\
4&0,2&0,2154\\
\hline
\multicolumn{3}{|c|}{$\alpha$}\\
\hline
0&2,7028&10,9783\hphantom{9}\\
1&3,6273&5,8621 \\
2&5,7598&2,7092\\
3&0,2315&1,0235\\
4&0,9110&0,4772\\
\hline
\multicolumn{3}{|c|}{$\lambda$}\\
\hline
0&85,2066&137,1714  \\
1&23,9592&136,7349\\
2&63,8425&132,6482\\
3&\hphantom{9}1,8026&116,7317\\
4&98,3882&102,5278\\
\hline
\end{tabular}
\end{center}
\end{minipage}\hfill
\begin{minipage}[t]{76mm}
%\end{table*}
%\begin{table*}\small
\begin{center}
\Caption{Результаты применения ЕМ-алго\-рит\-ма к выборке~2 ``Synterra'' 
\label{t2bat}} \vspace*{2ex}

\tabcolsep=8.7pt
\begin{tabular}{|c|c|c|}
\hline
№&Начальное приближение&Результат\\
\hline
\multicolumn{3}{|c|}{$P$}\\
\hline
0&0,2&$0{,}3815\hphantom{{}\cdot 10^{-9}}$\\
1&0,2&$0{,}3594\hphantom{{}\cdot 10^{-9}}$\\
2&0,2&$0{,}2589\hphantom{{}\cdot 10^{-9}}$\\
3&0,2&$0{,}4401\cdot 10^{-9}$\\
4&0,2&$0{,}0\hphantom{{}\cdot 10^{-9}999}$\\
\hline
\multicolumn{3}{|c|}{$\alpha$}\\
\hline
0&6,0804&1,5833\\
1&3,1838&0,8554\\
2&1,4886&0,4557\\
3&4,6407&0,2278\\
4&3,7843&0,1139\\
\hline
\multicolumn{3}{|c|}{$\lambda$}\\
\hline
0&17,3387&15,8682\\
1&47,8294&16,9150\\
2&54,1984&19,2866\\
3&\hphantom{1}8,6254&19,2866\\
4&\hphantom{1}5,7252&19,2866\\
\hline
\end{tabular}
\end{center}
\end{minipage}
\end{table*}


Данные результаты иллюстрируют рис.~\ref{f7bat}.


Эти результаты также отражают действительную картину, как показано на
рис.~\ref{f8bat}.


Этот трафик был снят с базовой станции <<Лукойл-Юго-Запад>> сети
широкополосного радиодоступа ЗАО <<Синтерра>>. Сеть радиодоступа
является реализацией так называемой <<последней мили>>, переносящей два
разных вида трафика: данные (Ethernet пакеты) и голос (IP-телефония, VoIP).
Поэтому здесь присутствуют в качестве основной массы короткие, но
интенсивные сообщения (пакеты SIP и голосовые фреймы), а также длинные
сообщения, содержащие данные.

Как мы видим, программная реализация ЕМ-ал\-го\-рит\-ма успешно справилась с
задачей разделения смесей распределений для этих двух выборок, что делает
данную программу удобным инструментом построения стохастической картины
конкретной сети. По полученным данным, используя метод интерпретации,
предложенный в разд.~2, можно получить представление о количестве
последовательных и параллельных структур вероятностной модели сети.

\subsection{Режим <<скользящего окна>>} %5.3.

Результаты для выборки
``Berkeley'' в режиме <<скользящего окна>>  представлены
на рис.~\ref{f9bat}.


Данные графики показывают изменение параметров распределений подвыборок выборки 
``Berkeley''. Видно, что параметры распределений подвыборок не остаются 
неизменными во времени, наоборот, они имеют внешне случайный характер. На 
рис.~\ref{f9bat},\,\textit{в} видна даже своеобразная пульсация первой 
компоненты.
%
На основании расчетов можно сделать вывод о том, что пиковость трафика
обусловливается как формой, так и интенсивностью сообщений.

\section{Заключение}

В данной работе исследована вероятностная модель  информационных потоков,
возникающих в сложных телекоммуникационных конвергентных сетях, построенная с
помощью асимптотического и энтропийного подходов. Эта модель предполагает, что
функционирование сложной телекоммуникационной сети можно представить в виде
суперпозиции довольно простых стохастических структур~--- последовательных и
параллельных, которые по\-рож\-да\-ют смеси гамма-распределений для случайной
величины времени обработки и передачи сообщений в сети. Предложена простая
интерпретация параметров данной модели.
\begin{figure*} %fig5
\vspace*{1pt}
\begin{center}
\mbox{%
\epsfxsize=130mm %145.109mm 
\epsfbox{bat-5.eps} }
\end{center}
\vspace*{-13pt} \Caption{Компоненты смеси начального приближения~(\textit{а}) и 
результата~(\textit{б}) для выборки~1 ``Berkeley'' \label{f5bat}}
%\end{figure*}
%\begin{figure*} %fig6
\vspace*{12pt}
\begin{center}
\mbox{%
\epsfxsize=130mm %148.256mm 
\epsfbox{bat-7.eps} }
\end{center}
\vspace*{-13pt} \Caption{График смеси распределений~(\textit{1}) и гистограмма 
для выборки~1 ``Berkeley''~(\textit{2}) \label{f6bat}}
\end{figure*}



\begin{figure*} %fig7
\vspace*{1pt}
\begin{center}
\mbox{%
\epsfxsize=130mm %144.283mm 
\epsfbox{bat-8.eps} }
\end{center}
\vspace*{-16pt} \Caption{Компоненты смеси начального приближения~(\textit{а}) и 
результата~(\textit{б}) для выборки~2 ``Synterra'' \label{f7bat}}
%\end{figure*}
%\begin{figure*} %fig8
\vspace*{12pt}
\begin{center}
\mbox{%
\epsfxsize=130mm %148.256mm 
\epsfbox{bat-10.eps} }
\end{center}
\vspace*{-11pt} \Caption{График смеси распределений~(\textit{1}) и гистограмма
для выборки~2 ``Synterra''~(\textit{2}) \label{f8bat}}
\end{figure*}

\begin{figure*} %fig9
\vspace*{1pt}
\begin{center}
\mbox{%
\epsfxsize=119.041mm
\epsfbox{bat-11.eps} }
\end{center}
\vspace*{-9pt} \Caption{Изменение  смешивающих параметров~(\textit{а}), 
параметров формы~(\textit{б}) и параметров масштаба~(\textit{в}) во времени для 
выборки~1 ``Berkeley'' \label{f9bat}}
\end{figure*}

Для решения вытекающей из модели задачи предложен итерационный алгоритм,
базирующийся на методе максимального правдоподобия~--- ЕМ-ал\-го\-ритм, для
которого получены формулы для конкретного вида смесей~--- экспоненциальных и
гамма-распределений.
%
Кроме того, разработан программный инструментарий для оценки параметров 
предложенной модели на выборках из реальных трафиковых данных. Проведены 
исследования, которые подтвердили предположения вероятностной модели. 


Получение информации о стохастической структуре
телекоммуникационных сетей и наличие программных инструментов для
выявления более или менее стабильных структур позволит понять причины
возникновения неожиданных больших нагрузок, предотвратить такие нагрузки,
а также поможет в будущем в проектировании надежных, оптимальных по
стоимости и уровню сервиса телекоммуникационных сетей нового поколения.

%\vspace*{-15pt} 
{\small\frenchspacing
{%\baselineskip=10.8pt
\addcontentsline{toc}{section}{Литература}
\begin{thebibliography}{9}
\bibitem{1bat}
Teletraffic Engeneering Handbook. International Telecommunication Union, 
Geneva, 2005 {\sf http://www.itu.int}. \vspace*{5pt} 
\bibitem{2bat}
\Au{Севастьянов~Б.\,А.} Курс теории вероятностей и математической статистики. 
М., 2004. \vspace*{5pt} 
\bibitem{3bat}
\Au{Айвазян~C.\,А., Бухштабер~В.\,М., Енюков~И.\,С, Мешалкин~Л.\,Д.} Прикладная 
статистика. Классификация и снижение размерности~// Финансы и статистика. М., 
1989. \vspace*{5pt} 
\bibitem{4bat}
\Au{Bilmes~J.\,A.} A gentle tutorial of the EM algorithm and its application to 
parameter estimation for Gaussian mixture and hidden Markov models. Berkeley, 
CA, USA: International Computer Science Institute,  1998. \vspace*{5pt} 
\bibitem{5bat}
\Au{Шлезингер~М.\,И.} О самопроизвольном различении образов~// Шлезингер~М.\,И. 
Читающие. автоматы. Киев: Наукова думка, 1965. С.~38--45. \vspace*{5pt} 
\bibitem{6bat}
\Au{Hsiao~I.-T., Rangarajan~A., Gindi~G.}. Joint-MAP 
reconstruction/segmentation for transmission tomography using mixture-models as 
priors. Yale University, 1998. \vspace*{5pt} 
\bibitem{7bat}
{\sf http://zedgraph.org}. \vspace*{4pt} 
\bibitem{8bat}
{\sf http://ita.ee.lbl.gov/html/contrib/LBL-PKT.html}. \vspace*{5pt} 
\bibitem{9bat}
{\sf http://www.synterra.ru}.
\end{thebibliography}

} } \label{end\stat}
\end{multicols}


%\addtocounter{razdel}{1}
%\def\razd{НЕРЕГУЛИРУЕМЫЙ ЭЛЕКТРОПРИВОД ДЛЯ ЭЛЕКТРОЭНЕРГЕТИКИ}

\setcounter{page}{2}

%   { %\Large  
   { %\baselineskip=16.6pt
   
   \vspace*{-48pt}
   \begin{center}\LARGE
   \textit{Предисловие}
   \end{center}
   
   %\vspace*{2.5mm}
   
   \vspace*{25mm}
   
   \thispagestyle{empty}
   
   { %\small 

    
Вниманию читателей журнала <<Информатика и её применения>> предлагается 
очередной тематический выпуск <<Вероятностно-статистические методы и 
задачи информатики и информационных технологий>>. Предыдущие тематические 
выпуски журнала по данному направлению вышли в 2008~г.\ (т.~2, вып.~2), 
в 2009~г.\ (т.~3, вып.~3) и в 2010~г.\ (т.~4, вып.~2). 

Статьи, собранные в данном журнале, посвящены разработке новых вероятностно-статистических 
методов, ориентированных на применение к решению конкретных задач информатики и информационных 
технологий, а также~--- в ряде случаев~--- и других прикладных задач. Проблематика, охватываемая 
публикуемыми работами, развивается в рамках научного сотрудничества между Институтом проблем 
информатики Российской академии наук (ИПИ РАН) и Факультетом вычислительной математики и 
кибернетики Московского государственного университета им.\ М.\,В.~Ломоносова в ходе работ 
над совместными научными проектами (в том числе в рамках функционирования 
Научно-образовательного центра <<Вероятностно-статистические методы анализа рисков>>). 
Многие из авторов статей, включенных в данный номер журнала, являются активными участниками 
традиционного международного семинара по проблемам устойчивости стохастических моделей, 
руководимого В.\,М.~Золотаревым и В.\,Ю.~Королевым; регулярные сессии этого семинара 
проводятся под эгидой МГУ и ИПИ РАН (в 2011~г.\ указанный семинар проводится в октябре 
в Калининградской области РФ). 

Наряду с представителями ИПИ РАН и МГУ в число авторов данного выпуска журнала входят 
ученые из Научно-исследовательского института системных исследований РАН, Института 
проблем технологии микроэлектроники и особочистых материалов РАН, Института 
прикладных математических исследований Карельского НЦ РАН, Московского 
авиационного института, Вологодского государственного педагогического университета, 
НИИММ им.\ Н.\,Г.~Чеботарева, Казанского государственного университета, Дебреценского 
университета (Венгрия).

Несколько статей выпуска посвящено разработке и применению стохастических методов и 
информационных технологий для решения различных прикладных задач. В~работе В.\,Г.~Ушакова 
и О.\,В.~Шестакова рассмотрена задача определения вероятностных характеристик случайных 
функций по распределениям интегральных преобразований, возникающих в задачах эмиссионной 
томографии. В~статье Д.\,О.~Яковенко и М.\,А.~Целищева рассмотрены некоторые вопросы 
математической теории риска и предложен новый подход к диверсификации инвестиционных 
портфелей. Работа И.\,А.~Кудрявцевой и А.\,В.~Пантелеева посвящена построению и 
исследованию математической модели, описывающей динамику сильноионизованной плазмы. 
В~статье П.\,П.~Кольцова изучается качество работы ряда алгоритмов сегментации изображений. 
Статья А.\,Н.~Чупрунова и И.~Фазекаша посвящена вероятностному анализу числа без\-оши\-бочных 
блоков при помехоустойчивом кодировании; получены усиленные законы больших чисел для указанных 
величин.

В данном выпуске традиционно присутствует тематика, весьма активно разрабатываемая в течение 
многих лет специалистами ИПИ РАН и МГУ,~--- методы моделирования и управления для 
информационно-телекоммуникационных и вычислительных систем, в частности методы 
теории массового обслуживания. В~статье А.\,И.~Зейфмана с соавторами рассматриваются 
модели обслуживания, описываемые марковскими цепями с непрерывным временем в случае 
наличия катастроф. В~работе М.\,М.~Лери и И.\,А.~Чеплюковой рассматриваются случайные 
графы Интернет-типа, т.\,е.\ графы, степени вершин которых имеют степенные распределения; 
такие задачи находят применение при исследовании глобальных сетей передачи данных. 
Работа Р.\,В.~Разумчика посвящена исследованию систем массового обслуживания специального 
вида~--- с отрицательными заявками и хранением вытесненных заявок.

Ряд статей посвящен развитию перспективных теоретических 
вероятностно-статистических методов, которые находят широкое применение в различных 
задачах информатики и информационных технологий. В~работе В.\,Е.~Бенинга, А.\,К.~Горшенина 
и В.\,Ю.~Королева рассмотрена задача статистической проверки гипотез о числе компонент 
смеси вероятностных распределений, приводится конструкция асимптотически наиболее мощного 
критерия. Результаты этой работы найдут применение в ряде прикладных задач, использующих 
математическую модель смеси вероятностных распределений (в информатике, моделировании 
финансовых рынков, физике турбулентной плазмы и~т.\,д.). В~статье В.\,Ю.~Королева, 
И.\,Г.~Шевцовой и С.\,Я.~Шоргина строится новая, улучшенная оценка точности нормальной 
аппроксимации для пуассоновских случайных сумм; как известно, указанные случайные суммы 
широко используются в качестве моделей многих реальных объектов, в том числе в информатике, 
физике и других прикладных областях. Работа В.\,Г.~Ушакова и Н.\,Г.~Ушакова посвящена 
исследованию ядерной оценки плотности распределения; эти результаты могут применяться, 
в част\-ности, при анализе трафика в телекоммуникационных системах. Серьезные приложения 
в статистике могут получить результаты работы О.\,В.~Шестакова, в которой доказаны оценки 
скорости сходимости распределения выборочного абсолютного медианного отклонения к нормальному 
закону. 

\smallskip

Редакционная коллегия журнала выражает надежду, что данный тематический  выпуск 
будет интересен специалистам в области теории вероятностей и математической статистики 
и их применения к решению задач информатики и информационных технологий.
     
     %\vfill 
     \vspace*{20mm}
     \noindent
     Заместитель главного редактора журнала <<Информатика и её 
применения>>,\\
     директор ИПИ РАН, академик  \hfill
     \textit{И.\,А.~Соколов}\\
     
     \noindent
     Редактор-составитель тематического выпуска,\\
     профессор кафедры математической статистики факультета\\
      вычислительной математики и кибернетики МГУ им.\ М.\,В.~Ломоносова,\\
     ведущий научный сотрудник ИПИ РАН,\\ 
доктор физико-математических наук \hfill
      \textit{В.\,Ю.~Королев}
     
     } }
     }



%   { %\Large  
   { %\baselineskip=16.6pt
   
   \vspace*{-48pt}
   \begin{center}\LARGE
   \textit{Предисловие}
   \end{center}
   
   %\vspace*{2.5mm}
   
   \vspace*{25mm}
   
   \thispagestyle{empty}
   
   { %\small 

    
Вниманию читателей журнала <<Информатика и её применения>> предлагается 
очередной тематический выпуск <<Вероятностно-статистические методы и 
задачи информатики и информационных технологий>>. Предыдущие тематические 
выпуски журнала по данному направлению вышли в 2008~г.\ (т.~2, вып.~2), 
в 2009~г.\ (т.~3, вып.~3) и в 2010~г.\ (т.~4, вып.~2). 

Статьи, собранные в данном журнале, посвящены разработке новых вероятностно-статистических 
методов, ориентированных на применение к решению конкретных задач информатики и информационных 
технологий, а также~--- в ряде случаев~--- и других прикладных задач. Проблематика, охватываемая 
публикуемыми работами, развивается в рамках научного сотрудничества между Институтом проблем 
информатики Российской академии наук (ИПИ РАН) и Факультетом вычислительной математики и 
кибернетики Московского государственного университета им.\ М.\,В.~Ломоносова в ходе работ 
над совместными научными проектами (в том числе в рамках функционирования 
Научно-образовательного центра <<Вероятностно-статистические методы анализа рисков>>). 
Многие из авторов статей, включенных в данный номер журнала, являются активными участниками 
традиционного международного семинара по проблемам устойчивости стохастических моделей, 
руководимого В.\,М.~Золотаревым и В.\,Ю.~Королевым; регулярные сессии этого семинара 
проводятся под эгидой МГУ и ИПИ РАН (в 2011~г.\ указанный семинар проводится в октябре 
в Калининградской области РФ). 

Наряду с представителями ИПИ РАН и МГУ в число авторов данного выпуска журнала входят 
ученые из Научно-исследовательского института системных исследований РАН, Института 
проблем технологии микроэлектроники и особочистых материалов РАН, Института 
прикладных математических исследований Карельского НЦ РАН, Московского 
авиационного института, Вологодского государственного педагогического университета, 
НИИММ им.\ Н.\,Г.~Чеботарева, Казанского государственного университета, Дебреценского 
университета (Венгрия).

Несколько статей выпуска посвящено разработке и применению стохастических методов и 
информационных технологий для решения различных прикладных задач. В~работе В.\,Г.~Ушакова 
и О.\,В.~Шестакова рассмотрена задача определения вероятностных характеристик случайных 
функций по распределениям интегральных преобразований, возникающих в задачах эмиссионной 
томографии. В~статье Д.\,О.~Яковенко и М.\,А.~Целищева рассмотрены некоторые вопросы 
математической теории риска и предложен новый подход к диверсификации инвестиционных 
портфелей. Работа И.\,А.~Кудрявцевой и А.\,В.~Пантелеева посвящена построению и 
исследованию математической модели, описывающей динамику сильноионизованной плазмы. 
В~статье П.\,П.~Кольцова изучается качество работы ряда алгоритмов сегментации изображений. 
Статья А.\,Н.~Чупрунова и И.~Фазекаша посвящена вероятностному анализу числа без\-оши\-бочных 
блоков при помехоустойчивом кодировании; получены усиленные законы больших чисел для указанных 
величин.

В данном выпуске традиционно присутствует тематика, весьма активно разрабатываемая в течение 
многих лет специалистами ИПИ РАН и МГУ,~--- методы моделирования и управления для 
информационно-телекоммуникационных и вычислительных систем, в частности методы 
теории массового обслуживания. В~статье А.\,И.~Зейфмана с соавторами рассматриваются 
модели обслуживания, описываемые марковскими цепями с непрерывным временем в случае 
наличия катастроф. В~работе М.\,М.~Лери и И.\,А.~Чеплюковой рассматриваются случайные 
графы Интернет-типа, т.\,е.\ графы, степени вершин которых имеют степенные распределения; 
такие задачи находят применение при исследовании глобальных сетей передачи данных. 
Работа Р.\,В.~Разумчика посвящена исследованию систем массового обслуживания специального 
вида~--- с отрицательными заявками и хранением вытесненных заявок.

Ряд статей посвящен развитию перспективных теоретических 
вероятностно-статистических методов, которые находят широкое применение в различных 
задачах информатики и информационных технологий. В~работе В.\,Е.~Бенинга, А.\,К.~Горшенина 
и В.\,Ю.~Королева рассмотрена задача статистической проверки гипотез о числе компонент 
смеси вероятностных распределений, приводится конструкция асимптотически наиболее мощного 
критерия. Результаты этой работы найдут применение в ряде прикладных задач, использующих 
математическую модель смеси вероятностных распределений (в информатике, моделировании 
финансовых рынков, физике турбулентной плазмы и~т.\,д.). В~статье В.\,Ю.~Королева, 
И.\,Г.~Шевцовой и С.\,Я.~Шоргина строится новая, улучшенная оценка точности нормальной 
аппроксимации для пуассоновских случайных сумм; как известно, указанные случайные суммы 
широко используются в качестве моделей многих реальных объектов, в том числе в информатике, 
физике и других прикладных областях. Работа В.\,Г.~Ушакова и Н.\,Г.~Ушакова посвящена 
исследованию ядерной оценки плотности распределения; эти результаты могут применяться, 
в част\-ности, при анализе трафика в телекоммуникационных системах. Серьезные приложения 
в статистике могут получить результаты работы О.\,В.~Шестакова, в которой доказаны оценки 
скорости сходимости распределения выборочного абсолютного медианного отклонения к нормальному 
закону. 

\smallskip

Редакционная коллегия журнала выражает надежду, что данный тематический  выпуск 
будет интересен специалистам в области теории вероятностей и математической статистики 
и их применения к решению задач информатики и информационных технологий.
     
     %\vfill 
     \vspace*{20mm}
     \noindent
     Заместитель главного редактора журнала <<Информатика и её 
применения>>,\\
     директор ИПИ РАН, академик  \hfill
     \textit{И.\,А.~Соколов}\\
     
     \noindent
     Редактор-составитель тематического выпуска,\\
     профессор кафедры математической статистики факультета\\
      вычислительной математики и кибернетики МГУ им.\ М.\,В.~Ломоносова,\\
     ведущий научный сотрудник ИПИ РАН,\\ 
доктор физико-математических наук \hfill
      \textit{В.\,Ю.~Королев}
     
     } }
     }

\def\stat{sinits}

\def\tit{АНАЛИТИЧЕСКОЕ МОДЕЛИРОВАНИЕ
НОРМАЛЬНЫХ ПРОЦЕССОВ В~СТОХАСТИЧЕСКИХ СИСТЕМАХ СО~СЛОЖНЫМИ~НЕЛИНЕЙНОСТЯМИ}

\def\titkol{Аналитическое моделирование
нормальных процессов в~стохастических системах со~сложными нелинейностями}

\def\aut{И.\,Н.~Синицын$^1$, В.\,И.~Синицын$^2$}

\def\autkol{И.\,Н.~Синицын, В.\,И.~Синицын}

\titel{\tit}{\aut}{\autkol}{\titkol}

\renewcommand{\thefootnote}{\arabic{footnote}}
\footnotetext[1]{Институт проблем
информатики Российской академии наук, sinitsin@dol.ru}
\footnotetext[2]{Институт проблем
информатики Российской академии наук, vsinitsin@ipiran.ru}


\Abst{Рассматриваются конечномерные дифференциальные стохастические системы
(ДСтС) и эредитарные (интегродифференциальные) стохастические системы  (ЭСтС)
с винеровскими и пуассоновскими шумами, приводимые к ДСтС со сложными конечными,
дифференциальными и интегральными нелинейностями. Такие модели функционирования
описывают поведение многих современных нано- и кван\-то\-во-оп\-ти\-че\-ских
технических средств информатики. Приводятся уравнения методов нормальной
аппроксимации (МНА) и статистической линеаризации (МСЛ) для аналитического
моделирования нестационарных и стационарных нормальных (гауссовских) процессов
в нелинейных ДСтС и  нелинейных ЭСтС путем аппроксимации эредитарных ядер
линейными операторными уравнениями для дифференцируемых нелинейностей и
сингулярными ядрами для недифференцируемых нелинейностей. Рассматриваются
методы вычисления типовых интегралов МНА (МСЛ) для сложных (многомерных и
векторного аргумента) конечных и дифференциальных нелинейностей. Особое
внимание уделяется иррациональным и дробно-рациональным нелинейностям
скалярного аргумента. Приводятся примеры вычисления интегралов. Подробно
рассматриваются вопросы вычисления типовых интегралов МНА (МСЛ) для сложных
интегральных нелинейностей.}

\KW{аналитическое моделирование;
дифференциальные стохастические системы с винеровскими и пуассоновскими шумами (ДСтС);
метод нормальной аппроксимации (МНА);
метод статистической линеаризации (МСЛ);
сложные иррациональные нелинейности;
сложные конечные, дифференциальные и интегральные нелинейности;
эредитарные стохастические системы (ЭСтС), приводимые к дифференциальным}

\DOI{10.14357/19922264140302}

\vspace*{9pt}

\vskip 16pt plus 9pt minus 6pt

\thispagestyle{headings}

\begin{multicols}{2}

\label{st\stat}


\section{Введение}


Моделями функционирования многих современных технических сис\-тем информатики
служат стохастические системы (СтС), описываемые дифференциальными, интегральными
и интегродифференциальными уравнениями со сложными дроб\-но-ра\-ци\-о\-наль\-ны\-ми,
иррациональными и интегральными нелинейностями. В~[1] дано систематическое
изложение МНА и МСЛ для ДСтС и ЭСтС, приводимых к дифференциальным.

Обобщая~[2--7], рассмотрим развитие МНА и МСЛ для аналитического моделирования
нормальных стохастических процессов (СтП) на случай СтС со сложными конечными,
дифференциальными и интегральными нелинейностями.

Как показано в~\cite{4-sin}, альтернативным подходом к аналитическому моделированию
СтП в ДСтС и ЭСтС служит подход, основанный на дискретизации стохастических
дифференциальных уравнений на основе использования обобщенной формы Ито и
кратных стохастических интегралов от винеровских и пуассоновских СтП с
последующим применением дискретных версий МНА (МСЛ).

Статья состоит из введения, пяти разделов и заключения.

В~разд.~2 и~3
приводятся уравнения МНА и МСЛ для аналитического моделирования одно- и
двумерных распределений стационарных и нестационарных СтП в ДСтС и ЭСтС,
приводимых к ДСтС.

Типовые интегралы МНА и МСЛ рассматриваются в разд.~4.

Особенности аналитического моделирования в ДСтС со сложными конечными и
дифференциальными нелинейностями обсуждаются в разд.~5.

Раздел~6
посвящен аналитическому моделированию СтП в ДСтС со сложными интегральными
нелинейностями.

Приводятся примеры.


\section{Уравнения методов нормальной~аппроксимации и~статистической
линеаризации для~дифференциальных стохастических систем}

Как известно~\cite{2-sin, 3-sin},  уравнения конечномерных непрерывных нелинейных сис\-тем
со стохастическими возмущениями путем расширения вектора состояния ДСтС
могут быть записаны в виде следующего векторного стохастического
дифференциального уравнения Ито:
    \begin{multline}
    dY_t = a(Y_t, t)\, dt + b (Y_t, t) \,dW_0+{}\\
    {}+ \iii_{R_0} c (Y_t, t, v) P^0
    (dt, dv)\,,\enskip Y(t_0) = Y_0\,.\label{e2.1-sin}
    \end{multline}
Здесь $a=a(Y_t, t)$ и $b\hm=b(y_t, t)$~--- известные
$(p\times 1)$-мер\-ная и  $(p\times m)$-мер\-ная функции~$Y_t$ и~$t$;
$W_0\hm= W_0(t)$~--- $r$-мер\-ный винеровский СтП интенсивности
$\nu_0 \hm= \nu_0(t)$; $c(Y_t, t, v)$~--- $(p\times 1)$-мер\-ная функция  $Y_t, t$
и вспомогательного $(q\times 1)$-мер\-но\-го па\-ра\-мет\-ра~$v$;
$\iii_{\Delta} dP^0 (t, A)$~--- центрированная пуассоновская мера,
определяемая
\begin{equation*}
\iii_{\Delta} dP^0 (t, A) = \iii_{\Delta} dP (t,A) =
\iii_{\Delta} \nu_P (t,A)\, dt\,. %\label{e2.2-sin}
\end{equation*}
В~(\ref{e2.1-sin}) принято: $\iii_{\Delta}$~-- число скачков пуассоновского
СтП в интервале времени  $\Delta \hm= (t_1, t_2]$; $\nu_P (t, A)$~---
интенсивность пуассоновского СтП  $P(t,A)$; $A$~--- некоторое борелевское
множество пространства  $R_0^q$ с выколотым началом.
Начальное значение~$Y_0$ представляет собой случайную величину, не зависящую
от приращений СтП  $W_0(t)$ и $P(t,A)$ на интервалах времени, следующих
за~$t_0$, $t_0 \hm\le t_1\hm\le t_2$ для любого множества~$A$.

В случае аддитивных нормальных (гауссовских) и обобщенных
пуассоновских возмущений уравнение~(\ref{e2.1-sin}) имеет вид:
\begin{equation}
\dot Y_t = a(Y_t,t)+ b_0 (t) V\,, \enskip
V = \dot W\,,\enskip Y(t_0) = Y_0\,.\label{e2.3-sin}
\end{equation}
Здесь $W$~--- СтП с независимыми приращениями, представляющий собой
смесь нормального и обобщенного пуассоновского СтП.

Если предположить существование конечных вероятностных
моментов второго порядка для моментов времени~$t_1$ и~$t_2$, то уравнения
МНА примут следующий вид~\cite{2-sin, 3-sin}:
\begin{itemize}
\item  для характеристических функций
    \begin{equation}
    g_1^N (\la;t) =\exp \lk i\la^{\mathrm{T}} m_t - \fr{1}{2}\, \la^{\mathrm{T}} K_t \la\rk\,;\label{e2.4-sin}
    \end{equation}
\begin{equation}
\hspace*{-7.5mm}g_{t_1, t_2}^N (\la_1, \la_2;t_1, t_2 ) =\exp \lk i\bar \la^{\mathrm{T}} \bar m_2 -
\fr{1}{2}\, \bar \la^{\mathrm{T}} \bar K_2 \la\rk\,,\!\!\label{e2.5-sin}
\end{equation}
где
    \begin{gather*}
    \bar \la =\lk \la_1^{\mathrm{T}}\la_2^{\mathrm{T}}\rk^{\mathrm{T}}\,; \quad
        \bar m_2 = \lk m_{t_1}^{\mathrm{T}} m_{t_2}^{\mathrm{T}}\rk^{\mathrm{T}}\,;\\
        \bar K_2= \begin{bmatrix}
    K(t_1, t_1)& K(t_1, t_2)\\
    K(t_2, t_1)& K(t_2, t_2)
    \end{bmatrix}\,;
    \end{gather*}

\item для математических ожиданий  $m_t$, ковариационной матрицы~$K_t$ и
матрицы ковариационных функций $K(t_1, t_2)$:
    \begin{equation}
    \dot m_t = a_1 (m_t, K_t, t)\,,\enskip m_0 = m(t_0)\,;\label{e2.6-sin}
    \end{equation}
\begin{equation}
\dot K_t = a_2 (m_t, K_t, t)\,,\enskip K_0 = K(t_0)\,;\label{e2.7-sin}
\end{equation}

\vspace*{-12pt}

\noindent
\begin{multline}
\fr{\prt K(t_1, t_2)}{\prt t_2 }= K(t_1, t_2) a_{21} (m_{t_2}, K_{t_2}, t_2)^{\mathrm{T}}\,;\\
K(t_1, t_1) = K_{t_1}\,.
\label{e2.8-sin}
\end{multline}
    \end{itemize}
Здесь приняты следующие обозначения:
\begin{equation}
a_1 = a_1 (m_t, K_t, t) = M_N a (Y_t, t)\,;\label{e2.9-sin}
\end{equation}

\vspace*{-12pt}

\noindent
\begin{multline}
a_2 = a_2 (m_t, K_t, t) = a_{21} (m_t, K_t, t)+{}\\
{}+ a_{21} (m_t, K_t, t)^{\mathrm{T}} +
a_{22}(m_t, K_t, t)\,;\label{e2.10-sin}
\end{multline}

\vspace*{-12pt}

\noindent

\begin{equation}
a_{21} = a_{21}(m_t, K_t, t)=  M_N a(Y_t, t) Y_{t}^{0\mathrm{T}}\,;\label{e2.11-sin}
\end{equation}
\begin{equation*}
a_{22} = a_{22}(m_t, K_t, t)= M_N \sigma (Y_t, t)\,;
%\label{e2.12-sin}
\end{equation*}

\vspace*{-12pt}

\noindent
\begin{multline*}
\sigma (Y_t, t) = b(Y_t, t) \nu_0(t) b(Y_t, t)^{\mathrm{T}} +{}\\
{}+
\iii_{R_0^q} c (Y_t, t, v) c(Y_t, t,v)^{\mathrm{T}}
\nu_P (t, dv)\,; %\label{e2.13-sin}
\end{multline*}

\vspace*{-12pt}

\begin{gather*}
m_t = MY_t\,,\quad Y_t^0 = Y_t - m_t\,,\\
K_t = M_N Y_0^0 Y_t^{0\mathrm{T}}\,,\quad K(t_1, t_2) =
M_N Y_{t_1}^0 Y_{t_2}^0\,; %\label{e2.14-sin}
\end{gather*}
$M_N$~--- символ вычисления математического ожидания для нормальных
распределений~(\ref{e2.4-sin}) и~(\ref{e2.5-sin}).

Для стационарных ДСтС нормальные стационарные СтП~--- если они существуют,
то  $m_t \hm=\bar m$, $ K_t \hm=\bar K$, $K(t_1, t_2) \hm= k(\tau)$
$(\tau \hm= t_1\hm-t_2)$,~--- определяются уравнениями~\cite{2-sin, 3-sin}:
   \begin{equation}
   a_1 (\bar m, \bar K) =0\,;\enskip a_2 (\bar m, \bar K)=0\,;\label{e2.15-sin}
   \end{equation}
   \begin{equation}
   \left.
   \hspace*{-2.8mm}\begin{array}{l}
  \dot k_\tau (\tau) = a_{21} (\bar m, \bar K)\bar K^{-1} k(\tau)\,;\\[9pt]
  k(0) =\bar K \enskip (\forall \tau >0)\,, \
  k(\tau) = k(-\tau)^{\mathrm{T}} \enskip
  (\forall\tau <0)\,.
  \end{array}\!\!
  \right\}\!\!
  \label{e2.16-sin}
  \end{equation}
При этом необходимо, чтобы матрица  $a_{21} (\bar m, \bar K)\hm=\bar a_{21}$
была бы асимптотически устойчивой.

Для ДСтС~(\ref{e2.3-sin}) уравнения МНА переходят в уравнения МСЛ
Казакова~\cite{2-sin, 3-sin}, если принять
\begin{equation}
a(Y_t,t) = a_1 (m_t, K_t) + k_1^a (m_t, K_t) Y_t^0\,;\label{e2.17-sin}
\end{equation}
\begin{equation}\left.
\begin{array}{rl}
b(Y_t,t) &= b_0 (t)\,;\\[9pt]
    \si(Y_t, t)&= b_0(t) \nu(t) b_0(t)^{\mathrm{T}} = \si_0(t)\,,
    \end{array}
    \right\}\label{e2.18-sin}
    \end{equation}
    \begin{equation}
k_1^a (m_t, K_t, t) =\lk \left(\fr{\prt}{\prt m_t} \right)
    a_0 (m_t, K_t, t)^{\mathrm{T}}\rk^{\mathrm{T}}\,;\label{e2.19-sin}
    \end{equation}
    \begin{equation}
\dot m_t = a_1 (m_t, K_t, t) \,,\enskip m_0 = m(t_0)\,,\label{e2.20-sin}
\end{equation}

\vspace*{-12pt}

\noindent
\begin{multline}
\dot K_t = k_1^a (m_t, K_t, t) K_t + K_t k_1^a (m_t, K_t, t)^{\mathrm{T}}
    +\si_0(t)\,;\\
    K_0 = K(t_0)\,;
    \label{e2.21-sin}
    \end{multline}

    \vspace*{-12pt}

    \noindent
\begin{multline}
\fr{\prt K(t_1, t_2)}{\prt t_2} =
    K(t_1, t_2) k_{t_2} k_1^a (m_{t_2}, K_{t_2}, t_2)^{\mathrm{T}}\,;\\
    K(t_1, t_2) = K_{t_1}\,.
    \label{e2.22-sin}
\end{multline}

Для стационарных ДСтС~(\ref{e2.3-sin})
при условии асимптотической устойчивости матрицы $k_1^a (\bar m, \bar K)$
в основе МСЛ лежат уравнения~(\ref{e2.15-sin}), записанные в виде:
    \begin{gather}
    a_1 (\bar m, \bar K) =0\,; \label{e2.23-sin}\\
k_1^a (\bar m, \bar K) \bar K + \bar K k_1^a
(\bar m, \bar K)^{\mathrm{T}} +\bar \si_0 =0\,;\label{e2.24-sin}
\end{gather}

\vspace*{-12pt}

\noindent
\begin{multline}
k_\tau (\tau) = k_1^a (\bar m, \bar K)k(\tau)\,,\enskip
k(0) =\bar K \enskip (\forall \tau >0)\,,\\
k(\tau) = k (-\tau)^{\mathrm{T}} \enskip (\forall \tau <0)\,.
\label{e2.25-sin}
\end{multline}

Уравнения~(\ref{e2.4-sin})--(\ref{e2.8-sin})
лежат в основе МНА для ДСтС~(\ref{e2.1-sin}), а уравнения~(\ref{e2.17-sin})--(\ref{e2.22-sin})~---
в основе МСЛ для ДСтС~(\ref{e2.3-sin}). Для определения стационарных СтП
согласно МНА служат соотношения~(\ref{e2.15-sin}) и~(\ref{e2.16-sin}),
а МСЛ~--- (\ref{e2.17-sin})--(\ref{e2.25-sin}).

\section{Уравнения методов нормальной~аппроксимации и~статистической линеаризации
для~эредитарных стохастических систем, приводимых к~дифференциальным}

Рассмотрим ЭСтС, описываемую интегродифференциальным уравнением Ито
следующего вида~\cite{7-sin}:

\noindent
\begin{multline}
dX_t = \lk a(X_t, t) +\iii_{t_0}^t a_1 (X(\tau) ,\tau, t)\,d\tau\rk dt+{}\\
{}+\lk b(X_t, t) +\iii_{t_0}^t b_1 (X(\tau) ,\tau, t)\,d\tau\rk dW_0+{}\\
\hspace*{-1.5mm}{}+\!\!\iii_{R_0^q}\!\!\lk c(X_t, t,v) +\!\iii_{t_0}^t\! c_1 (X(\tau) ,\tau, t,v)\,d\tau\!\rk\! dP^0 (t, dv)
\!\!\!\!\label{e3.1-sin}
\end{multline}
с начальным условием  $X(t_0) = X_0$. В~(\ref{e3.1-sin})
сохранены обозначения разд.~2.

Функции $a=a(X_t, t)$, $a_1\hm = a_1(X (\tau),\tau, t)$,
$b\hm=b(X_t, t)$, $b_1\hm = b_1(X (\tau),\tau, t)$,
$c\hm=c(X_t,t,v)$ и $c_1\hm = c_1(X (\tau),\tau, t,v)$ имеют
соответственно размерности $p\times 1$, $p\times 1$, $p\times r$,
$p\times r$, $p\times 1$ и $p\times 1$ и допускают представления следующего вида:
\begin{equation}
\left.
\begin{array}{rl}
a_1&=A(t,\tau) \vrp (X(\tau) , \tau)\,;\\[9pt]
b_1&=B(t,\tau) \psi (X(\tau) ,  \tau)\,;\\[9pt]
c_1&=C(t,\tau) \chi (X(\tau) ,  \tau, v)\,.
\end{array}
\right\}
\label{e3.2-sin}
\end{equation}
Здесь эредитарные ядра $A\hm=A(t,\tau)\hm=\lk A_{ij}(t,\tau)\rk$
$(i,j\hm=\overline{1,p})$,
$B\hm=B(t,\tau)=\lk B_{i l}(t,\tau)\rk$ $(i\hm=\overline{1,p}$;
$l\hm=\overline{1,r})$ и $C\hm=C(t,\tau)=\lk C_{ij}(t,\tau)\rk$
$(i,j\hm=\overline{1,p})$ имеют соответственно размерности
$p\times p$, $p\times r$ и $p\times p$. Они удовлетворяют следующим условиям
физической реализуемости и асимптотического затухания:
\begin{multline}
A_{ij}(t,\tau)=0;\enskip B_{i l}(t,\tau)=0;\\[1pt]
C_{ij}(t,\tau)=0\enskip \forall \tau >t;\label{e3.3-sin}
\end{multline}

\vspace*{-12pt}

\begin{equation}
\left.
\hspace*{-3mm}\begin{array}{c}
\displaystyle\iin\! \lv A_{ij} (t,\tau) \rv d\tau <\infty\,;\
\displaystyle\iin\! \lv B_{i l} (t,\tau) \rv d\tau <\infty \,;\\[9pt]
\displaystyle\iin \!\lv C_{ij} (t,\tau) \rv d\tau <\infty\,.
\end{array}\!
\right\}\!
\label{e3.4-sin}
\end{equation}

В дальнейшем ограничимся случаем, когда эредитарные ядра удовлетворяют
линейным операторным уравнениям~\cite{6-sin, 5-sin, 7-sin}.

Нелинейные в общем случае функции $\vrp\hm=\vrp(X(\tau),\tau)$,
$\psi \hm=\psi(X(\tau), \tau)$, $\chi \hm=\chi (X(\tau),  \tau, v)$
отражают нелинейные свойства ЭСтС, зависят от  $X(\tau)$ и имеют размерности
$p\times 1$, $p\times p$, $p\times 1$ соответственно.

Важный класс  эредитарных ядер представляют собой
сингулярные (вырожденные) ядра, когда имеют место представления:
\begin{equation}
\left.
\hspace*{-3mm}\begin{array}{rl}
A_{ij} (t,\tau) &= A_{ij}^+(t) A_{ij}^-(\tau)\,;\\[9pt]
B_{i l} (t,\tau)& = B_{il}^+(t) B_{il}^-(\tau)\,;\\[9pt]
C_{ij} (t,\tau) &= C_{ij}^+ ( t) C_{ij}^- (\tau)\
(i,l= \overline{1,p}, j=\overline{1,r}).
\end{array}\!
\right\}\!\!
\label{e3.5-sin}
\end{equation}

В~\cite{6-sin, 5-sin, 7-sin} показано, что для дифференцируемых нелинейных
функций~$\vrp$, $\psi$, $\chi$ путем расширения вектора состояния за счет
инструментальных переменных, аппроксимируемых линейными операторными уравнениями,
определяющими эредитарные ядра в ЭСтС, (\ref{e3.1-sin})--(\ref{e3.4-sin})
приводятся к ДСтС вида~(\ref{e2.1-sin}) или~(\ref{e2.3-sin}).
В~случае недифференцируемых нелинейных функций~$\vrp$, $\psi$, $\chi$
ЭСтС~(\ref{e3.1-sin})--(\ref{e3.4-sin}) приводятся к~(\ref{e2.1-sin}) или~(\ref{e2.3-sin})
на основе аппроксимации вырожденными (сингулярными) ядрами~\cite{6-sin, 5-sin, 7-sin}.

Таким образом, после приведения ЭСтС~(\ref{e3.1-sin}) к ДСтС~(\ref{e2.1-sin})
или~(\ref{e2.3-sin}) можно воспользоваться уравнениями МНА и МСЛ разд.~2.

\section{Типовые интегралы методов нормальной аппроксимации и~статистической
линеаризации}

Как следует из уравнений~(\ref{e2.9-sin})--(\ref{e2.11-sin}),
для МНА необходимо уметь вычислять следующие интегралы:
\begin{multline}
I_0^a = I_0^a (m_t, K_t, t) = a_1 (m_t, K_t, t)={}\\
{}= M_N a(Y_t, t)\,;
\label{e4.1-sin}
\end{multline}

\vspace*{-12pt}

\noindent
\begin{multline}
I_1^a = I_1^a (m_t, K_t, t)= a_{21}(m_t, K_t, t)= {}\\
{}=M_N a(Y_t , t) Y_t^{0\mathrm{T}}\,;\label{e4.2-sin}
\end{multline}

\vspace*{-12pt}

\noindent
\begin{multline}
I_0^{\bar \si} = I_0^{\bar \si} (m_t, K_t, t) = a_{22}(m_t, K_t, t) ={}\\
{}= M_N \bar \si (Y_t, t)\,.\label{e4.3-sin}
\end{multline}
Для МСЛ достаточно вычислить интеграл~(\ref{e4.1-sin}),
причем интеграл~$I_1^a$ вычисляется по формуле~\cite{2-sin, 3-sin, 4-sin}:
\begin{equation*}
k_1^a = k_1^a (m_t, K_t, t)=\lk \left( \fr{\prt}{\prt m_t}\right)
I_0^a (m_t, K_t, t)^{\mathrm{T}}\rk^{\mathrm{T}}. %\label{e4.4-sin}
\end{equation*}

\medskip

\noindent
\textbf{Пример 1.} В~[1] для типовых степенных, тригоно\-мет\-ри\-че\-ских,
показательных и ку\-соч\-но-по\-сто\-ян\-ных нелинейностей $Z_t \hm=\vrp (Y_t, t)$
скалярного и векторного аргумента приведены формулы для интегралов
$I_0^\vrp \hm= I_0^\vrp (m_t^y, K_t^y, t)$, а также
$k_1^\vrp \hm= k_1^\vrp (m_t^y, K_t^y, t)$.

\medskip

\noindent
\textbf{Замечание.}
 Важно иметь в виду, что уравнения МНА (МСЛ) содержат интегралы
 $I_0^a$, $I_1^a$, $I_0^\si$ в виде соответствующих коэффициентов.
 Поэтому процедура вычисления интегралов должна быть согласована с
 методом численного решения обыкновенных дифференциальных уравнений для
 $m_t$, $K_t$ и $K(t_1, t_2)$. Эти коэффициенты допускают дифференцирование
 по~$m_t$ и~$K_t$, так как под интегралом стоит сглаживающая нормальная плотность.

\section{Сложные конечные и~дифференциальные нелинейности}

Важный класс сложных конечных нелинейностей (многомерных и векторного аргумента)
представляют собой сложные функции вида:
    \begin{equation*}
    \xi =\vrp (X_t, Y_t, t)\,,\enskip X_t =\psi (Y_t, t)\,. %\label{e5.1-sin}
    \end{equation*}
В~этом случае вычисление интегралов (см.\ разд.~4) проводится по совокупности
переменных  $\lk X_t^{\mathrm{T}} Y_t^{\mathrm{T}}\rk^{\mathrm{T}}$.
К таким нелинейностям, например, относятся дроб\-но-ра\-ци\-о\-наль\-ные,
иррациональные  нелинейности, выражаемые специальными функциями, многозначные
нелинейности, зависящие от СтП~$X_t$ и его производных~$\dot X_t$,  $\ddot X_t$
и~др.

\medskip

\noindent
\textbf{Пример 2.}
Рассмотрим вычисление интегралов~(\ref{e4.1-sin}) и~(\ref{e4.2-sin})
для сложных одномерных иррациональных нелинейностей скалярного аргумента
\begin{equation}
\vrp (Y_t, t) =\lv Y_t\rrv^{\alpha-1}\, \mathrm{sgn}\, Y_t
\label{e5.2-sin}
\end{equation}
($\alpha$~--- нецелый показатель).

Пользуясь~(\ref{e2.16-sin}) и~(\ref{e2.19-sin}), представим~(\ref{e5.2-sin}) в виде
\begin{equation*}
\vrp(Y_t, t) = \vrp_0 (m_t, D_t, t) + k_1^\vrp(m_t, D_t, t) Y_t^0. %\label{e5.3-sin}
\end{equation*}
Здесь введены следующие обозначения:
\begin{gather*}
\vrp_0(m_t, D_t, t) =\Gamma(\alpha) D_t^{1/2} e^{-\xi^2/4} D_{-\alpha} (\xi)\,;%\label{e5.4-sin}
\\
k_1^a (m_t, D_t, t) =\fr {\prt \vrp_0(m_t, D_t, t)}{\prt m_t}\,,%\label{e5.5-sin}
\end{gather*}
где  $\Gamma(\alpha)$~--- гамма-функция,  $\xi \hm= m_t/\sqrt{D_t}$~---
отношение <<сиг\-нал--шум>>; $D_{-\alpha} (\xi)$~---
функция параболического цилиндра~\cite{9-sin}.
При вычислении были учтены следующие соотношения~\cite{9-sin, 8-sin}:
\begin{multline}
\iii_0^\infty x^{\alpha-1} e^{-\beta x^2 - \gamma x} \,dx ={}\\
{}=
(2\beta)^{-\alpha/2} \Gamma(\alpha) \exp \left(\fr{\gamma^2}{8\beta}\right)
D_{-\alpha} \left(\fr{\gamma}{\sqrt{2\beta}}\right)\,;\label{e5.6-sin}
\end{multline}

\vspace*{-12pt}

\noindent
\begin{multline}
\fr{dD_\rho(\xi)}{d\xi} =
   -\fr{\xi}{2}\, D_\rho (\xi) -\rho D_{\rho-1} (\xi) =
   \fr{\xi}{2}\, D_\rho (\xi) -{}\\
   {}- D_{\rho+1} (\xi) \enskip
   (\mathrm{Re}\, \beta>0\,,\enskip \mathrm{Re}\,\alpha>0\,,\enskip
   \rho=-\alpha)\,.\label{e5.7-sin}
   \end{multline}

Соотношения~(\ref{e5.6-sin}) и~(\ref{e5.7-sin})
могут быть использованы также для вычисления интегралов~(\ref{e4.3-sin}).

\medskip

\noindent
\textbf{Замечание.}
Для вычисления интегралов $I_0^a$, $I_1^a$ и $I_0^{\bar \si}$
применительно к типовым иррациональным нелинейностям вида
    $\lv Y_t\rrv^{\alp-1} e^{\delta Y_t}$, $\lv Y_t\rrv^{\alp-1}  \cos \w Y_t$,
    $\lv Y_t\rrv^{\alp-1}  \sin \w Y_t$
и более общим нелинейностям \mbox{вида}
    \begin{equation*}
    \vrp (Y_t, t) =\Phi^\vrp \left( \lv Y_t\rrv^{\alpha-1}, t\right) %\label{e5.8-sin}
    \end{equation*}
можно рекомендовать известные численные методы вычисления функций на ЭВМ~\cite{8-sin}.

\smallskip

\noindent
\textbf{Пример 3.}
Для нелинейной дроб\-но-ра\-ци\-о\-наль\-ной функции

\noindent
\begin{equation*}
\vrp (Y_t, t) = \fr{a}{(b+Y_t)^2} %\label{e5.9-sin}
\end{equation*}
имеем

\vspace*{-3pt}

\noindent
\begin{gather*}
\vrp_0 (m_t, D_t, t) =a b^{-2} \lk 1+ \chi (m_t, D_t, t)\rk\,; %\label{e5.10-sin}
\\
k_1^\vrp (m_t, D_t, t) =  a b^{-2}\fr{\prt \chi (m_t, D_t, t)}{\prt m_t}\,. %\label{e5.11-sin}
\end{gather*}
Здесь

\vspace*{-3pt}

\noindent
\begin{multline*}
\chi (m_t, D_t, t) ={}\\
{}=\sss_{n=1}^\infty \sss_{l=0}^{E(n/2)}
\fr{(-1)^n (n+1) n!}{(n-2l)! (2l)!}\, b^{-n} m_t^n \left( \fr{D_t}{ 2 m_t^2}
\right)^l, %\label{e5.12-sin}
\end{multline*}
где  $E(n/2)$~--- целая часть~$n/2$; $a\hm=a(t)$; $b\hm= b(t)$.

\vspace*{-6pt}

\section{Сложные интегральные нелинейности}

\vspace*{-2pt}

Пусть сначала векторно-матричная нелинейность имеет эредитарный характер, т.\,е.\
\begin{equation}
\underline{\vrp} (Y_t, t) =\iii_{t_0}^t A(t,\tau) \vrp (Y(\tau), \tau) \,d\tau\,.
\label{e6.1-sin}
\end{equation}
Тогда, как показано в~\cite{6-sin, 5-sin, 7-sin}, следует соответст\-ву\-ющие
интегродифференциальные соотношения путем введения  инструментальных
переменных привести к дифференциальным соотношениям.  Для
дифференцируемых функций~$\vrp$ и асимптотически устойчивых ядер
$A(t,\tau)$ зависимость~(\ref{e3.5-sin}) имеет следующий дифференциальный вид:
\begin{equation*}
F^A (t, D) \underline{\vrp} (Y_t, t) = H^A (t, D) \vrp (Y_t, t)\,. %\label{e6.2-sin}
\end{equation*}
Здесь $F^A (t, D)$ и  $H^A (t, D)$~--- линейные дифференциальные операторы $(D\hm= d/dt)$.

Для недифференцируемых функций~$\vrp$ и асимптотически устойчивых
сингулярных ядер~(\ref{e3.5-sin}) используются соотношения:
\begin{equation*}
\underline{\vrp} (Y_t, t) = A^+ Z\,,\enskip
\dot Z = A^- \vrp\,,\enskip
Z(t_0)=0\,. %\label{e6.3-sin}
\end{equation*}

Многочисленные примеры аналитического моделирования ЭСтС можно найти
в~[1--3, 5, 7, 10, 11].

Как отмечалось в~\cite{3-sin}, часто наряду с интегральными
нелинейностями~(\ref{e6.1-sin}) рассматривают нелинейности вида:

\columnbreak

\noindent
\begin{equation*}
Z_s =\sss_{\rho=1}^R \mathcal{A}_\rho \vrp_\rho (Y_{t_1}\tr Y_{t_r})\,, %\label{e6.2-sin}
\end{equation*}
где $\mathcal{A}_1 \tr \mathcal{A}_R$~--- произвольные линейные операторы,
действующие над функциями~$r$ переменных  $t_1\tr t_r$; $\vrp_\rho
\hm=\vrp_\rho (Y_{t_1} \tr Y_{t_r})$~--- линейные функции отмеченных
переменных. Такие нелинейности называются приводимыми к линейным.
Важным частным случаем~(\ref{e6.1-sin}) являются интегральные нелинейности вида:

\noindent
\begin{gather}
Z_s =\iii_T \vrp (Y_t, t, s)\, dt\,; \notag%\label{e6.3-sin}
\\
Z_s =\!\iii_T \!\cdots\!\iii_T\! \vrp (Y_{t_1}\tr Y_{t_r}; t_1\tr t_r, s)\,dt_1
\ldots dt_r,\notag %\label{e6.4-sin}
\end{gather}
В этом случае используется МСЛ по совокупности переменных  $Y_{t_1} \tr Y_{t_r}$.

\vspace*{-9pt}

\section{Заключение}

\vspace*{-2pt}

Разработаны методы и алгоритмы МНА и МСЛ для ДСтС и ЭСтС,
приводимых к ДСтС со сложными конечными, дроб\-но-ра\-ци\-о\-наль\-ны\-ми,
иррациональными, а также дифференциальными и интегральными нелинейностями.
Приведены примеры.

Результаты допускают обобщение на случай ДСтС и ЭСтС со
стохастическими нелинейностями, заданными каноническими разложениями и
интегральными каноническими  представлениями~\cite{1-sin, 3-sin, 11-sin}.

\vspace*{-9pt}

{\small\frenchspacing
 {%\baselineskip=10.8pt
 \addcontentsline{toc}{section}{References}
 \begin{thebibliography}{99}

 \vspace*{-2pt}

\bibitem{1-sin}
\Au{Синицын И.\,Н.,  Синицын~В.\,И.}
Лекции по нормальной и эллипсоидальной аппроксимации распределений в
стохастических сис\-те\-мах.~--- М.: ТОРУС ПРЕСС, 2013. 488~с.

\bibitem{6-sin} %2
\Au{Синицын И.\,Н. }
Stochastic hereditary control systems~// Проблемы управления и
теории информации, 1986. Т.~15. №\,4. С.~287--298.

\bibitem{2-sin} %3
\Au{Пугачев В.\,С., Синицын~И.\,Н.}
Стохастические дифференциальные сис\-те\-мы. Анализ и фильтрация.~--- М.:
Наука,  1990.  632~с. [Англ. пер.
 Stochastic differential systems.
Analysis and filtering.~--- Chichester, New York: Jonh Wiley, 1987.
549~p.].

\bibitem{5-sin} %4
\Au{Синицын И.\,Н. }
Конечномерные распределения процессов в стохастических интегральных
и интегродифференциальных системах~// Preprints of the 2nd IFAC
Symposium on Stochastic Control.~--- Vilnius: Pergamon Press,
1987.  Vol.~1. P.~144--153.

\bibitem{3-sin} %5
\Au{Пугачев В.\,С., Синицын~И.\,Н.}
Теория стохастических систем.~--- М.: Логос, 2000; 2004. 1000~с.
[Англ. пер.\linebreak\vspace*{-12pt}

\pagebreak

\noindent Stochastic systems. Theory and  applications.~---
Singapore: World Scientific, 2001. 908~p.].

\bibitem{4-sin} %6
\Au{Синицын И.\,Н.}
Параметрическое статистическое и аналитическое моделирование распределений
в нелинейных стохастических сис\-те\-мах на многообразиях~//
Информатика и её применения, 2013. Т.~7. Вып.~2. С.~4--16.

\bibitem{7-sin} %7
\Au{Синицын И.\,Н. }
Анализ и моделирование распределений в эредитарных стохастических
сис\-те\-мах~// Информатика и её применения, 2014. Т.~8. Вып.~1.\linebreak
С.~2--11.



\bibitem{9-sin} %8
\Au{Градштейн И.\,С., Рыжик~И.\,М.}
Таблицы интегралов, сумм, рядов и произведений.~--- М.: ГИФМЛ, 1963. 1100~с.

\bibitem{8-sin} %9
\Au{Попов Б.\,А., Теслер~Г.\,С. }
Вычисление функций на ЭВМ: Справочник.~--- Киев: Наукова Думка, 1984. 599~с.


\bibitem{11-sin} %10
\Au{Синицын И.\,Н.}
Канонические представления случайных функций и их применение в
задачах компьютерной поддержки научных исследований.~--- М.: ТОРУС
ПРЕСС, 2009. 768~с.

\bibitem{10-sin} %11
\Au{Синицын И.\,Н., Синицын~В.\,И., Корепанов~Э.\,Р., Белоусов~В.\,В.,
Сергеев~И.\,В., Басилашвили~Д.\,А.}
Опыт моделирования эредитарных стохастических сис\-тем~//
Кибернетика и высокие технологии XXI века: Сб. докл.  XIII Междунар.
науч.-технич. конф.~--- Воронеж: Саквоее, 2012. Т.~2. C.~346--357.

 \end{thebibliography}

 }
 }

\end{multicols}

\vspace*{-9pt}

\hfill{\small\textit{Поступила в редакцию 05.05.14}}

%\newpage

\vspace*{12pt}

\hrule

\vspace*{2pt}

\hrule

\vspace*{12pt}

\def\tit{ANALYTICAL MODELING OF NORMAL PROCESSES
 IN~STOCHASTIC SYSTEMS WITH~COMPLEX NONLINEARITIES}

\def\titkol{Analytical modeling of normal processes
 in~stochastic systems with~complex nonlinearities}

\def\aut{I.\,N.~Sinitsyn and V.\,I.~Sinitsyn}

\def\autkol{I.\,N.~Sinitsyn and V.\,I.~Sinitsyn}

\titel{\tit}{\aut}{\autkol}{\titkol}

\vspace*{-9pt}

\noindent
Institute of Informatics Problems, Russian Academy of Sciences,
44-2 Vavilov Str., Moscow 119333, Russian Federation


\def\leftfootline{\small{\textbf{\thepage}
\hfill INFORMATIKA I EE PRIMENENIYA~--- INFORMATICS AND
APPLICATIONS\ \ \ 2014\ \ \ volume~8\ \ \ issue\ 3}
}%
 \def\rightfootline{\small{INFORMATIKA I EE PRIMENENIYA~---
INFORMATICS AND APPLICATIONS\ \ \ 2014\ \ \ volume~8\ \ \ issue\ 3
\hfill \textbf{\thepage}}}

\vspace*{6pt}

\Abste{Differential stochastic systems (DStS) with Wiener and Poisson
noises and complex finite, differential, and  integral nonlinearities and
hereditary StS reducible to DStS are considered. Equations and algorithms
of analytical modeling based on the normal approximation method (NAM) and the
statistical linearization method (SLM) are given. The case of complex
continuous and discontinuous nonlinearities of scalar and vector arguments
is considered. Special attention is paid to NAM (SLM) typical integrals
for finite rational and irrational nonlinear and integral scalar and vector
nonlinear functions. The general case of integral nonlinearities reducible to
linear is considered. Test examples are given.}

\KWE{analytical modeling;
complex finite differential and integral nonlinearities;
complex irrational nonlinerarites
differential stochastic system with Wiener and Poisson noises;
method of normal approximation;
method of statistical linearization;
hereditary stochastic systems reducible to differential}

\DOI{10.14357/19922264140302}

  \begin{multicols}{2}

\renewcommand{\bibname}{\protect\rmfamily References}
%\renewcommand{\bibname}{\large\protect\rm References}

{\small\frenchspacing
 {%\baselineskip=10.8pt
 \addcontentsline{toc}{section}{References}
 \begin{thebibliography}{99}



\bibitem{1-sin-1}
\Aue{Sinitsyn, I.\,N., and  V.\,I.~Sinitsyn}.  2013.
Lektsii po normal'noy i ellipsoidal'noy approksimatsii raspredeleniy
v stokhasticheskikh sistemakh [Lectures on normal and ellipsoidal
approximation of distributions in stochastic systems].
Moscow: TORUS PRESS. 488~p.

\bibitem{6-sin-1} %2
\Aue{Sinitsyn, I.\,N.}  1986.
{Stochastic hereditary control systems}.
\textit{Problems Control Inform. Theory} 15(4):287--298.

\bibitem{2-sin-1} %3
\Aue{Pugachev, V.\,S., and  I.\,N.~Sinitsyn}.  1987.
\textit{Stochastic differential systems. Analysis and filtering.}
Chichester, New York: Jonh Wiley. 549~p.

\bibitem{5-sin-1} %4
\Aue{Sinitsyn, I.\,N.}  1987.
Konechnomernye raspredeleniya protsessov v stokhasticheskikh integral'nykh
i in\-teg\-ro\-dif\-fe\-ren\-tsial'nykh sistemakh [Finite dimensional distributions
of processes in stochastic integral and integrodifferential systems].
\textit{2nd  Symposium (International) IFAC on Stochastic Control
Preprints}. Vilnius: Pergamon Press. 1:144--153.

\bibitem{3-sin-1} %5
\Aue{Pugachev, V.\,S., and I.\,N.~Sinitsyn}. 2001.
\textit{Stochastic systems. Theory and  applications}.
Singapore: World Scientific. 908~p.

\bibitem{4-sin-1} %6
\Aue{Sinitsyn, I.\,N.}  2013.
Parametricheskoe statisticheskoe i analiticheskoe modelirovanie
raspredeleniy v nelineynykh stokhasticheskikh sistemakh na mnogoobraziyakh
[Parametric statistical and analytical modeling of distributions in
stochastic systems on manifolds].
\textit{Informatika i ee Primeneniya}~--- \textit{Inform. Appl.} 7(2):4--16.


\bibitem{7-sin-1} %7
\Aue{Sinitsyn, I.\,N.}  2014.
Analiz i modelirovanie raspredeleniy v ereditarnykh stokhasticheskikh sistemakh
[Analysis and modeling of distributions in hereditary stochastic systems].
\textit{Informatika i ee Primeneniya}~--- \textit{Inform. Appl.} 8(1):2--11.

\bibitem{9-sin-1} %8
\Aue{Gradshteyn, I.\,S., and I.\,M.~Ryzhik}.  1963.
\textit{Tablitsy integralov, summ, ryadov i proizvedeniy}
[Tables of integrals, sums, series, and products]. Moscow:  GIFML.   1100~p.

\pagebreak

\bibitem{8-sin-1} %9
\Aue{Popov, B.\,A., and G.\,S.~Tesler}.  1984.
\textit{Vychislenie funktsiy na EVM}. Spravochnik [Computing of functions].
Kiev: Naukova Dumka.  599~p.


\bibitem{11-sin-1} %10
\Au{Sinitsyn, I.\,N.} 2009.
\textit{Kanonicheskie predstavleniya sluchaynykh funktsiy i ikh primenenie v
zadachakh komp'yuternoy podderzhki nauchnykh issledovaniy}
[Canonical expansions of random functions and its application to
scientific computer-aided support]. Moscow: TORUS PRESS. 768~p.

\bibitem{10-sin-1} %11
\Aue{Sinitsyn, I.\,N., V.\,I.~Sinitsyn, E.\,R.~Korepanov,
V.\,V.~Belousov, I.\,V.~Sergeev, and D.\,A.~Basilashvili}.
2012. Opyt modelirovaniya ereditarnykh stokhasticheskikh sistem
[Experience of modeling in hereditary stochastic systems].
\textit{Kibernetika i Vysokie Tekhnologii XXI~Veka:
Sbornik dokladov  XIII Mezhdunar. nauch.-tekhnich. konf.}
[Cybernatics ans High Technologies of the XXI Century: Materials of
XIII  Scientific and Technological Conference (International)].
Voronezh: Sakvoee. 2:346--357.

\end{thebibliography}

 }
 }

\end{multicols}

\vspace*{-6pt}

\hfill{\small\textit{Received May 05, 2014}}

\vspace*{-18pt}

\Contr

\noindent
\textbf{Sinitsyn Igor N.} (b.\ 1940)~---
Doctor of Science in technology, professor, Honored scientist of RF, Head of Department, Institute of
Informatics Problems, Russian Academy of Sciences,
44-2 Vavilov Str., Moscow 119333, Russian
Federation; sinitsin@dol.ru

\vspace*{3pt}

\noindent
\textbf{Sinitsyn Vladimir I.} (b.\ 1968)~--- Doctor of Science in physics
and mathematics, associate professor, Head of Department, Institute of
Information Problems, Russian Academy of Sciences,
44-2 Vavilov Str., Moscow 119333, Russian Federation; VSinitsin@ipiran.ru




\label{end\stat}

\renewcommand{\bibname}{\protect\rm Литература}     %+ 1

\newcommand{\cov}{\textrm{cov}}
%\newcommand{\indic}{\mathbb{1}}
\newcommand{\Obig}{\textsf{O}}
\newcommand{\osml}{\textsf{o}}
\newcommand{\hsig}{\hat\sigma^2}
\newcommand{\Yljk}{Y_{\lambda;j,\mathbf{k}}}

\newcommand{\Yljks}{Y_{\lambda';j',\mathbf{k'}}}
\newcommand{\muljk}{\mu_{\lambda;j,\mathbf{k}}}
\newcommand{\solj}{\sigma_{\lambda;j}}
\newcommand{\soljs}{\sigma_{\lambda';j'}}
\newcommand{\solz}{\sigma_{\lambda;0}}
\newcommand{\silz}{\sigma_{1;0}}
\newcommand{\siilz}{\sigma_{2;0}}
\newcommand{\siiilz}{\sigma_{3;0}}
\newcommand{\slj}{\solj^2}
\newcommand{\sljs}{\soljs^2}
\newcommand{\hslj}{\hat\sigma^2_{\lambda;j}}
\newcommand{\hsljs}{\hat\sigma^2_{\lambda';j'}}
\newcommand{\hslz}{\hat\sigma_{\lambda;0}^2}
\newcommand{\Tlj}{T_{\lambda;j}}
\newcommand{\hTlj}{\hat T_{\lambda;j}}

\newcommand{\indYjklTj}{\Ik_{\left|\Yljk\right|\leqslant \Tlj}}
\newcommand{\indYjkgTj}{\Ik_{\left|\Yljk\right|>\Tlj}}
\newcommand{\prbYjkgTj}{\p\left( \left|\Yljk\right|>\Tlj \right)}
\newcommand{\indYjklhTj}{\Ik_{\left|\Yljk\right|\leqslant \hTlj}}
\newcommand{\indYjkghTj}{\Ik_{\left|\Yljk\right|>\hTlj}}
\newcommand{\sumljk}{\sum\limits_{\lambda,j,\mathbf{k}}}

\def\stat{markin}

\def\tit{АСИМПТОТИКИ ОЦЕНКИ РИСКА ПРИ ПОРОГОВОЙ ОБРАБОТКЕ ВЕЙВЛЕТ-ВЕЙГЛЕТ КОЭФФИЦИЕНТОВ В ЗАДАЧЕ ТОМОГРАФИИ}

\def\titkol{Асимптотики оценки риска при пороговой обработке вейвлет-вейглет коэффициентов в задаче томографии}

\def\autkol{А.\,В.~Маркин, О.\,В.~Шестаков}
\def\aut{А.\,В.~Маркин$^1$, О.\,В.~Шестаков$^2$}

\titel{\tit}{\aut}{\autkol}{\titkol}

%{\renewcommand{\thefootnote}{\fnsymbol{footnote}}\footnotetext[1]
%{Исследования выполнены при частичной поддержке РФФИ, гранты 08-01-00567, 08-01-91205, 09-01-12180.}}

\renewcommand{\thefootnote}{\arabic{footnote}}
\footnotetext[1]{Московский государственный университет им.\ М.\,В.~Ломоносова, 
факультет вычислительной математики и кибернетики, кафедра математической статистики, artem.v.markin@mail.ru}
\footnotetext[2]{Московский государственный университет им.\ М.\,В.~Ломоносова, 
факультет вычислительной математики и кибернетики, кафедра математической статистики,
oshestakov@cs.msu.su}


\Abst{Рассмотрена задача реконструкции изображения по радоновскому образу с помощью вейв\-лет-вейг\-лет разложения. 
Исследованы свойства оценки риска пороговой обработки вейг\-лет-коэф\-фи\-ци\-ен\-тов, такие как состоятельность и 
асимптотическая нормальность.}

\KW{вейвлеты; томография; пороговая обработка; оценка риска; предельное распределение}

     \vskip 18pt plus 9pt minus 6pt

      \thispagestyle{headings}

      \begin{multicols}{2}

      \label{st\stat}


\section{Введение}

Вейвлет-преобразование является весьма популярным и удобным методом обработки нестационарных сигналов и 
изображений. Одна из основных задач, для которых используются вейв\-ле\-ты,~---
удаление шума и сжатие. Эти операции производятся путем пороговой обработки 
вейв\-лет-коэффициентов. Кроме того, вейвлеты могут быть использованы для обращения 
линейных операторов, таких, например, как преобразование Радона. В этом случае 
пороговая обработка выполняет задачу регуляризации соответствующей формулы обращения.

Пусть на плоскости $(x,\,y)$ задана функция~$f$. Определим образ 
(или проекции) Радона~$\mathcal{R}f$ как набор интегралов от~$f$ по всевозможным прямым плоскости
\begin{equation}
\label{eq_radonTransform}
\mathcal{R}f(s,\theta)=\int\limits_{L_{s,\theta}}f\left(x,\,y\right)\,dl\,,
\end{equation}
где
\begin{equation*}
L_{s,\theta}=\left\{ (x,\,y): x\cos\theta+y\sin\theta-s=0 \right \}\,.
\end{equation*}
Формула обращения преобразования~(\ref{eq_radonTransform}) впервые была получена 
Радоном, ее можно записать в следующем виде~\cite{Natterer}:
\begin{equation}
\label{eq_radonInverse}
f = \fr{1}{2}\mathcal{R}^{\#}\mathcal{I}^{-1}\mathcal{R}f\,,
\end{equation}
где $\mathcal{R^{\#}}$~--- оператор обратного проецирования:
\begin{equation*}
\left(\mathcal{R^{\#}}g\right)(x,y)=\int\limits_0^{2\pi}g(x\cos\theta+y\sin\theta,\theta)\,d\theta\,;
\end{equation*}
$\mathcal{I}$~--- потенциал Рисса:
\begin{equation}
\label{eq_RieszPoten}
\left(\mathcal{F}_1\mathcal{I}^\alpha g\right) (\omega) = |\omega|^{-\alpha}\left(\mathcal{F}_1 g\right)(\omega)\,,
\end{equation}
а $\mathcal{F}_k$~--- $k$-мерное преобразование Фурье.

Для точного восстановления~$f$ требуется точное знание всевозможных проекций~$\mathcal{R}f(s,\,\theta)$. 
На практике же имеют дело с конечным числом проекций, причем в проекциях присутствует шум.
При этом задача томографии является некорректной, т.\,е.\ малые изменения в проекциях могут 
при\-вес\-ти к восстановлению изображения, существенно отличающегося от исходного. Математически
это выражается в наличии множителя~$|\omega|$ в формуле~(\ref{eq_RieszPoten}) (и, следовательно, 
в~(\ref{eq_radonInverse})), который <<подчеркивает>> высокие частоты.

Выход видится в регуляризации~(\ref{eq_radonInverse}) путем умножения~$|\omega|$ на некоторый множитель, 
называемый частотным фильтром (или стабилизирующим множителем)~\cite{TikhonovArsenin}. 
Общая идея регуляризации такова:\linebreak
немного <<испортить>> проекционные данные, подавив влияние 
высоких частот, но при этом обеспечить реконструкцию, близкую к оригиналу. Подроб\-нее о 
регуляризации формулы обращения можно прочитать в монографии~\cite{Herman}.

\section{Вейвлет-вейглет разложение}

Задачу томографии можно решить и с помощью вейвлетов. Пусть~$\phi(t)$ и~$\psi(t)$~--- 
одномерные отцовский и материнский вейвлеты. Определим
\begin{align*}
\phi_{j,k_1,k_2}(x,y) &= 2^{j} \phi\left(2^jx-k_1\right) \phi\left(2^jy-k_2\right)\,;\\
\psi^{[1]}_{j,k_1,k_2}(x,y) &= 2^{j} \phi\left(2^jx-k_1\right) \psi\left(2^jy-k_2\right)\,;
\end{align*}

\noindent
\begin{align*}
\psi^{[2]}_{j,k_1,k_2}(x,y) &= 2^{j} \psi\left(2^jx-k_1\right) \phi\left(2^jy-k_2\right)\,;\\
\psi^{[3]}_{j,k_1,k_2}(x,y) &= 2^{j} \psi\left(2^jx-k_1\right) \psi\left(2^jy-k_2\right)\,.
\end{align*}
Заметим, что параметр масштаба~$j$ контролирует сразу обе функции в произведении. 
Это так называемое тензорное произведение двух одномерных кратномасштабных анализов~\cite{Daub}. 
Тогда набор функций $\left\{ \phi_{j_0,k_1,k_2}, \, \psi^{[\lambda]}_{j,k_1,k_2}, \right\}$, 
где $j$, $k_1$, $k_2\in\mathbb{Z}$, $j\geq j_0$, $\lambda=\overline{1,3}$, 
будет ортонормированным базисом~$\mathbf{L}^2(\mathbb{R}^2)$.

Донохо~\cite{DonohoWVD} решил задачу обращения ряда линейных операторов 
с помощью вейвлетов и родственных им функций специального вида, названных вейглетами (\textit{vaguelettes}). 
Вейглеты для обращения оператора Радона выглядят так:
\begin{multline*}
\xi^{[\lambda]}_{j,k_1,k_2}(s,\,\theta)=
\int\limits_{-\infty}^\infty|\omega|\left(\mathcal{F}_2\psi^{[\lambda]}_{j,k_1,k_2}\right)\times{}\\
{}\times \left( \omega\cos\theta,\,\omega\sin\theta \right)\exp(i2\pi s\omega)\,d\omega\,.
%\label{eq_vagueletteDef}
\end{multline*}
Идея метода реконструкции заключается в том, что вейглет-коэффициенты проекций~$\mathcal{R}f(s,\theta)$ 
равны вейвлет-коэффициентам исходной функции~$f(x,y)$:
\begin{equation*}
\left[\mathcal{R}f,\,\xi^{[\lambda]}_{j,k_1,k_2}\right] = \left\langle f,\,\psi^{[\lambda]}_{j,k_1,k_2}\right\rangle\,,
\end{equation*}
и поэтому
\begin{multline}
f = \sum\limits_{k_1,k_2}\left[\mathcal{R}f,\,\tau_{j_0,k_1,k_2}\right] \phi_{j_0,k_1,k_2} +{}\\
{}+ \sum\limits_{j\geqslant j_0,k_1,k_2,\lambda} \left[\mathcal{R}f,\,\xi^{[\lambda]}_{j,k_1,k_2}\right] \psi^{[\lambda]}_{j,k_1,k_2}\,,
\label{eq_radonInverseWVD}
\end{multline}
где
\begin{multline*}
\tau_{j_0,k_1,k_2}(s,\,\theta)=\int\limits_{-\infty}^\infty|\omega|\left(\mathcal{F}_2\phi_{j_0,k_1,k_2}\right)\times{}\\
{}\times \left( \omega\cos\theta,\,\omega\sin\theta \right)
\exp\left(i2\pi s\omega\right)\,d\omega\,.
\end{multline*}
Регуляризация вейвлет-вейглет формулы~(\ref{eq_radonInverseWVD}) производится с 
помощью мягкой пороговой обработки вейглет-коэффициентов (см.\ разд.~\ref{sect_ThreshholdingTomo}).

\section{Дискретизация и модель шума}

Пусть функция $f(x,y)$ задана на квадрате $[0,\,1]\;\times$\linebreak $\times\;[0,\,1]$. Разбив стороны квадрата на~$N=2^J$ 
равных частей и вычислив значения~$f$ в точках отсчета, получим дискретизованную версию~$f$. Одна\-ко на практике 
нередко бывает удобно нормировать длину отрезка разбиения и рас\-сматривать вместо~$f$ ее <<растянутую>>
версию~--- функцию~$\bar f(Nx,Ny)\;=$\linebreak $={f}(x,y)$. Тогда для вейвлет-коэффициентов функции~$f$ справедливо равенство:
\begin{multline}
\left\langle f,\, \psi^{[\lambda]}_{j,u_1,u_2}\right\rangle ={}\\
{}= \iint f(x,y)\,2^j\overline{\psi^{[\lambda]}\left(2^jx-k_1,\,2^jy-k_2\right)}\,dx\,dy ={}\\
{}=\left(\mathcal{W}^{[\lambda]}f\right)\left(2^{-j},k_1,k_2\right)={}\\
{}=\fr{1}{N}\left(\mathcal{W}^{[\lambda]}\bar f\right)\left(N\,2^{-j},k_1,k_2\right)\,.
\label{eq_contToDiscrCoeff}
\end{multline}
Заметим, что при работе с растянутой функцией растягиваются и вейвлет-функции.
Коэффициенты аппроксимации, получаемые через скалярное произведение~$f$ и~$\phi$, не рассматриваются, 
так как пороговая обработка (см.\ разд.~4) применяется к коэффициентам деталей, которые дают функции~$\psi^{[\lambda]}$. 
Далее везде, кроме разд.~\ref{sect_RegularityTomo}, предполагается, что используются именно коэффициенты 
растянутой версии функции~$f$.

Задача томографии ставится следующим образом. Имеются наблюдения~$X$, состоящие из 
проекций~$\mathcal{R}f$ функции~$f$ и шума~$\epsilon$:
\begin{equation*}%\label{eq_tomoTask}
X=\mathcal{R}f+\epsilon\,, 
\end{equation*}
$\epsilon$~--- независимые нормальные случайные величины с нулевым средним и дисперсией~$\sigma^2$. 
Необходимо восстановить~$f$ по~$X$. При этом при достаточно большом~$N$~\cite{KolaczykArticle}
\begin{equation}
\left.
\begin{array}{rl}
\e \left[X,\,\xi^{[\lambda]}_{j,k_1,k_2}\right] &= \left[\mathcal{R}f,\,\xi^{[\lambda]}_{j,k_1,k_2}\right]\,;\\[9pt]
\D \left[X,\,\xi^{[\lambda]}_{j,k_1,k_2}\right] &= \sigma^2 \left\|\xi^{[\lambda]}_{j,k_1,k_2}\right\|_2^2=\sigma^2_{\lambda;j}\,;\\[9pt]
\left\|\xi^{[\lambda]}_{j,k_1,k_2}\right\|_2^2 &= 2^j \left\|\xi^{[\lambda]}_{0,0,0}\right\|_2^2\,.
\end{array}
\right \}
\label{eq_expctVageuletteCoef}
\end{equation}
Как видим, дисперсия коэффициентов растет вмес\-те с уровнем разложения. Это является следствием 
некорректности задачи томографии. При этом вейглеты не ортогональны, а почти ортогональны. И, 
стало быть, вейг\-лет-коэф\-фи\-ци\-ен\-ты не независимы, а почти независимы. Однако нередко 
этим фактом пренебрегают, так как исследование этой зависимости сопряжено с рядом трудностей. 
И потому порог выбирается исходя из предположения независимости коэффициентов. Как будет видно 
далее, уже только тот факт, что дисперсия растет на каждом уровне, заметно влияет на оценку 
риска пороговой обработки.

\section{Пороговая обработка}\label{sect_ThreshholdingTomo}

Мягкая пороговая функция определяется следующим образом:
\begin{equation*}
\rho(x, T)=
\begin{cases}
x-T & \text{при } x>T\,;\\
x+T & \text{при } x<-T\,;\\
0 & \text{при } |x|\leq T\,.
\end{cases} 
\end{equation*}
Эта функция применяется к вейглет-ко\-эф\-фи\-ци\-ен\-там проекций.

Допустим, что размер изображения равен $N^2\;=$\linebreak $=2^{2J}=L$, разложение идет до уровня~$J-1$. 
В~качестве порога взят порог Колашика~\cite{KolaczykArticle, KolaczykThesis}:
\begin{equation*}
\Tlj = \sqrt{2\ln 2^{2j}} \, 2^{j/2}\sigma  \left\|\xi^{[\lambda]}_{0,0,0}\right\|_2\,.
\end{equation*}
В случае использования оценки дисперсии шума~$\hsig$ порог принимает вид
\begin{equation*}
\hTlj = \sqrt{2\ln 2^{2j}}\,2^{j/2}\hat\sigma \left\|\xi^{[\lambda]}_{0,0,0}\right\|_2\,.
\end{equation*}
Идея выбора такого порога схожа с идеей выбора порога~\textit{VisuShrink} 
$T=\sigma\sqrt{2\ln N}$ (одномерный случай, $N$~--- размер сигнала): при таком пороге 
убирается почти весь шум~\cite{DJideal, DJunkn}. Это следует из того факта, что если $Z_1,\ldots,Z_N$~--- 
независимые стандартные нормальные случайные величины, то
\begin{equation*}
\p\left( \underset{1\leqslant i\leqslant N}{\max}|Z_i| > \sqrt{2\ln N} \right) \rightarrow 0\
\mbox{при}\ N\rightarrow\infty\,.
\end{equation*}

Пороговая обработка идет с уровня~$j_M$, т.\,е.\ в формуле~(\ref{eq_radonInverseWVD}) 
$j_0=j_M$ ($j_M$ определим ниже). Риск~$r(f)$ такой пороговой обработки определяется следующим образом:
\begin{multline}
r(f)=\sum\limits_{j=j_M}^{J-1}\sum_{\lambda=1}^3\sum_{k_1=0}^{2^j-1}
\sum_{k_2=0}^{2^j-1}\e
\left\{ \left\langle f,\,\psi^{[\lambda]}_{j,k_1,k_2}\right\rangle - {}\right.\\
\left.{}-\rho\left(\left[X,\,\xi^{[\lambda]}_{j,k_1,k_2}\right],\,\Tlj\right) \right\}^2\,.
\label{eq_riskEstimDefTomo}
\end{multline}
Так как на практике коэффициенты $\left\langle f,\,\psi^{[\lambda]}_{j,k_1,k_2}\right\rangle$ 
неизвестны, то строят оценку риска. Например,
на основе функции~$\Phi(x,T)$~\cite{Mallat}:
\begin{equation*}
\Phi(x,\Tlj)=
\begin{cases}
x-\slj & \text{при } x\leqslant \Tlj^2\,;\\
\slj+\Tlj^2 & \text{при } x> \Tlj^2\,.
\end{cases} 
\end{equation*}
Оценка риска принимает вид:
\begin{equation*}
\tilde r(f)=\sum\limits_{j=j_M}^{J-1}\sum\limits_{\lambda,k_1,k_2}\Phi\left( 
\left| \left[X,\,\xi^{[\lambda]}_{j,k_1,k_2}\right] \right|^2 ,\,\Tlj\right)\,.
\end{equation*}
Если вместо~$\sigma^2$ используется оценка~$\hsig$, то
\begin{equation*}
\hat r(f)=\sum\limits_{j=j_M}^{J-1}
\sum\limits_{\lambda,k_1,k_2}\hat\Phi\left( 
\left| \left[X,\,\xi^{[\lambda]}_{j,k_1,k_2}\right] \right|^2 ,\,\hTlj\right)\,,
\end{equation*}
где
\begin{equation*}
\hat\Phi(x,\hTlj)=
\begin{cases}
x-\hslj & \text{при } x\leqslant \hTlj^2\,;\\
\hslj+\hTlj^2 & \text{при } x> \hTlj^2\,.
\end{cases} 
\end{equation*}


В работах~\cite{MarkinShestakovConsist, MarkinLimitDistr} рассмотрены асимптотические свойства оценки 
риска пороговой обработки вейв\-лет-коэффициентов в одномерном случае при прямом наблюдении~$f$. 
Показано, что $(\hat r -r)/N^a$ сходится по вероятности к нулю и по распределению к нормальному 
закону при соответствующих~$a$.
Величина~$a$ существенно зависит от свойств оценки~$\hsig$. Однако даже при весьма общих ограничениях 
на моменты~$\hsig$ порядок $a=1$ обеспечивал
сходимость по вероятности к нулю. Ниже будет показано, что в задаче томографии для сходимости 
по вероятности к нулю недостаточно делить на число коэффициентов ($N^2=L$),
т.\,е.\ некоторый аналог закона больших чисел уже не выполнен. Важнейшим фактором 
является то, что~$f$ наблюдается через оператор Радона~$\mathcal{R}$, обратный к 
которому не является непрерывным (т.\,е.\ ограниченным).

\section{Регулярность функции и~вейвлет-коэффициенты}\label{sect_RegularityTomo}

Известно (см., например,~\cite{Mallat}), что если функция~$f(x,y)$ является регулярной по Липшицу 
с параметром $0\leq\alpha\leq 1$, т.\,е.\
\begin{multline*}
\left|f(x_1,y_1)-f(x_2,y_2)\right|\leq{}\\
{}\leq C \left( |x_1-x_2|^2 + |y_1-y_2|^2 \right)^{\alpha/2}
\end{multline*}
для некоторой константы~$C$, не зависящей от $(x_1,y_1)$ и $(x_2,y_2)$, то существует не зависящая от~$J$, 
$j$, $k_1$ и $k_2$ константа~$A$ такая, что
\begin{equation*}
\left(\mathcal{W}^{[\lambda]}f\right)\left(2^{-j},k_1,k_2\right)\leq \fr{A}{2^{j(\alpha+1)}}\,.
\end{equation*}
В отечественной литературе вместо регулярности по Липшицу обычно используется термин <<непрерывность 
по Гёльдеру>>.
С учетом~(\ref{eq_contToDiscrCoeff}) получаем
\begin{equation*}
\left(\mathcal{W}^{[\lambda]}\bar f\right)\left(N\cdot 2^{-j},k_1,k_2\right) \leqslant \frac{A\cdot 2^J}{2^{j(\alpha+1)}}\,.
\end{equation*}

\textit{Предположение о регулярности~$f$: будем полагать, что функция~$f$ является регулярной по Липшицу с 
показателем~$\alpha>0$}. Будем считать, что пороговая обработка ведется с уровня 
$j_M\geq J/(\alpha+1)$. Заметим, что $J-j_M\rightarrow\infty$ при $J\rightarrow\infty$. 
Тогда при определенном выборе вейвлет-базиса~\cite{Mallat} найдется константа~$C_1$ такая, что для 
всех $j\geq j_M$ выполнено
\begin{equation}
\label{eq_WaveletCoeffUpperBoundTomo}
\left(\mathcal{W}^{[\lambda]}\bar f\right)\left(N\cdot 2^{-j},k_1,k_2\right) \leqslant C_1\,,
\end{equation}
причем $C_1$ не зависит от~$N$. Значит, математические ожидания в~(\ref{eq_expctVageuletteCoef}) ограничены.

В работе используется буква~$C$ (с индексом или без индекса) для обозначения констант, причем в 
разных местах~--- вообще говоря, разных.

\section{Асимптотика оценки риска при~известной дисперсии шума}\label{sect_ConsitKnownSTomo}

В работе~\cite{MarkinLimitDistr} показано, что в одномерном случае при известной дисперсии шума 
разность риска и оценки риска при делении на $\sqrt{N}$ сходится по распределению к нормальному 
закону. В задаче томографии уже надо делить не на~$\sqrt{L}$, а на~$L$.

Для краткости введем обозначения:
\begin{align*}
\Yljk &= \left[X,\,\xi^{[\lambda]}_{j,k_1,k_2}\right]\,;\\
\muljk &= \left\langle f,\,\psi^{[\lambda]}_{j,k_1,k_2}\right\rangle\,,
\end{align*}
где $\mathbf{k}=\left(k_1,\,k_2\right)$. Еще раз напомним, что~$\muljk$ рассматриваются 
как коэффициенты растянутой версии дискретизованной функции~$f$. С учетом предположения об 
ортогональности вейглетов получаем
\begin{equation}
\label{eq_YljkNormalDistributed}
\Yljk \sim \mathcal{N}\left(\muljk,\,\slj\right)\,,
\end{equation}
причем $\Yljk$~--- независимые случайные величины.

\medskip

\noindent
\textbf{Теорема 1.}
\textit{Пусть справедливы предположения о регулярности~$f$ из разд.~\ref{sect_RegularityTomo}. 
При известной дисперсии шума в задаче томографии}
\begin{equation*}
\fr{\tilde r(f)-r(f)}{L \sqrt{ b_2 \left( \silz^4 + \siilz^4 + \siiilz^4 \right) }} \Rightarrow \mathcal{N}(0,\,1)
\end{equation*}
\textit{при} $L\rightarrow\infty$, \textit{где} $b_2=2/(2^4-1)=2/15$.

\medskip

\noindent
Д\,о\,к\,а\,з\,а\,т\,е\,л\,ь\,с\,т\,в\,о.\ 
Представим разность оценки риска и самого риска в виде
\begin{multline*}
\tilde r-r=\sumljk\left(\Yljk^2-\slj\right)\indYjklTj +{}\\
\!\!\!\!{}+ \sumljk\left(\slj+\Tlj^2\right)\indYjkgTj - {}
\end{multline*}

\noindent
\begin{multline}
\ \ {}- \sumljk\e\left(\Yljk^2-\slj\right)\indYjklTj -{}\\
{}- \sumljk\e\left(\slj+\Tlj^2\right)\indYjkgTj ={}\\
{}= \sumljk\left(\Yljk^2-\e\Yljk^2\right) -{}\\
{}- \sumljk\left(\Yljk^2-\slj\right)\indYjkgTj + {}\\
{}+ \sumljk\e\left(\Yljk^2-\slj\right)\indYjkgTj +{}\\
{}+ \sumljk\left(\slj+\Tlj^2\right)\indYjkgTj -{}\\
{}- \sumljk\left(\slj+\Tlj^2\right)\prbYjkgTj\,.
\label{eq_diffRiskEstimKnownS}
\end{multline}
Покажем, что при делении на~$L$ первая сумма в~(\ref{eq_diffRiskEstimKnownS}) сходится по распределению
к нормальному закону, а остальные суммы~--- к нулю по вероятности.

Итак, рассмотрим первую сумму в~(\ref{eq_diffRiskEstimKnownS}). Имеем
\begin{multline}
D_L^2 = \D\sumljk\Yljk^2 = \sumljk \D \Yljk^2={}\\
{}=\sum\limits_\lambda \sum_{j=j_M}^{J-1} \sum_{\mathbf{k}}
\left( 2\solj^4 + 4\muljk^2\slj \right)={}\\
{}= \sum\limits_\lambda \sum_{j=j_M}^{J-1}\left\{ 2\cdot 2^{2j}
\solz^4 \cdot 2^{2j} + \sum_{\mathbf{k}}4\muljk^2 2^j \solz^2 \right\} \simeq{}\\
{}\simeq \sum\limits_\lambda \sum_{j=j_M}^{J-1} 2\cdot 2^{4j}
\solz^4 = \sum\limits_\lambda 2\solz^4 \frac{2^{4J} - 2^{4j_M}}{2^4 - 1} \simeq{}\\
{}\simeq \fr{2}{15} 2^{4J} \left( \silz^4 + \siilz^4 + \siiilz^4 \right)\,.
\label{eq_DLknownS}
\end{multline}
Знак~$\simeq$ означает, что при $J\rightarrow\infty$ предел отношения левой и правой частей~(\ref{eq_DLknownS}) 
равен единице. Если выполнено условие Линдеберга, т.\,е.\ для любого~$\delta>0$
\begin{multline}
\fr{1}{D_L^2}\sumljk\e\left\{ \left( \Yljk^2 - \muljk^2 - \slj \right)^2\times{}\right.\\
\left.{}\times \Ik_{\left|\Yljk^2 - \muljk^2 - \slj\right|>\delta D_L} \right\} \rightarrow 0\,,
\label{eq_LindCondTomo}
\end{multline}
то будет иметь место сходимость к нормальному распределению. Так как~$D_L$ имеет порядок~$L$ и чис\-ло слагаемых 
в~(\ref{eq_LindCondTomo}) имеет порядок~$L$, то достаточно показать, что при $L\rightarrow\infty$
\begin{multline*}
\e\left\{ \fr{\left( \Yljk^2 - \muljk^2 - \slj \right)^2}{D_L} \times{}\right.\\
\left.{}\times\Ik_{\left(\Yljk^2 - \muljk^2 - \slj\right)^2/D_L>\delta^2 D_L} \right\} \rightarrow 0\,.
\end{multline*}
А последнее выполнено потому, что у случайных величин вида $\left( \Yljk^2 - \muljk^2 - \slj \right)^2\!/D_L$ 
конечные математические ожидания и $D_L\rightarrow\infty$.

Теперь рассмотрим вторую сумму в~(\ref{eq_diffRiskEstimKnownS}). В силу~(\ref{eq_YljkNormalDistributed}) имеем
\begin{multline*}
\p\left( |\Yljk| > \Tlj \right) < {}\\
{}< \frac{\exp\left( -(\Tlj-\muljk)^2/(2\slj) \right)}{\Tlj} +{}\\
{}+ \frac{\exp\left( -(\Tlj+\muljk)^2/(2\slj) \right)}{\Tlj} \leqslant \fr{C}{2^{5j/2} \sqrt{j} }
\end{multline*}
при $J \rightarrow \infty$ (и, следовательно, $j \rightarrow \infty$). Это можно получить из 
следующей цепочки равенств:
\begin{multline*}
\exp\left( -\fr{(\Tlj-\muljk)^2}{2\slj} \right) = {}\\
{}=\exp\left( -\fr{\Tlj^2}{2\slj} + \fr{2\Tlj\muljk}{2\slj} - \fr{\muljk^2}{2\slj} \right) ={}\\
{}= \exp\left( -\ln 2^{2j} + \fr{\sqrt{2\ln(2^{2j})}\muljk}{2^{j/2}\solz} - 
\fr{\muljk^2}{2\slj} \right) \simeq{}\\
{}\simeq 2^{-2j}\mbox{ при }j \rightarrow \infty\,,
\end{multline*}
так как
\begin{equation*}
\fr{\sqrt{2\ln(2^{2j})}\muljk}{2^{j/2}\solz} \rightarrow 0\quad\text{и}\quad\fr{\muljk^2}{2\slj} \rightarrow 0\,.
\end{equation*}
С помощью неравенств Чебышёва и Коши--Бу\-ня\-ков\-ского получаем для любого $\delta>0$ при $J\rightarrow\infty$
\begin{multline*}
\p\left( \fr{ \left|\sumljk\left(\Yljk^2-\slj\right)\indYjkgTj \right|}{D_L} > \delta \right) \leq{}\\
{}\leq \fr{ \e\left| \sumljk\left(\Yljk^2-\slj\right)\indYjkgTj \right| }{\delta D_L} \leq {}\\
{}\leq \fr{ \sumljk \e\left| \Yljk^2-\slj\right|\indYjkgTj }{\delta D_L} \leq{}\\
{}\leq \fr{ \sumljk \sqrt{ \e\left( \Yljk^2-\slj\right)^2 \p\left( |\Yljk| > \Tlj \right) } }{\delta D_L} 
\leq{}
\end{multline*}

\noindent
\begin{multline*}
{}\leq \fr{1}{\delta D_L}\sumljk 
\left  ( \vphantom{4\cdot 2^j\solz^2\muljk^2  C\cdot 2^{-5j/2} j^{-1/2}}
\left(
\muljk^4 + 2\cdot 2^{2j}\solz^4 + {}\right.\right.\\
\left.\left.{}+4\cdot 2^j\solz^2\muljk^2 
\right) C\cdot 2^{-5j/2} j^{-1/2} 
\right )^{1/2} \rightarrow 0\,.
\end{multline*}
Аналогично проводятся рассуждения для оставшихся сумм в~(\ref{eq_diffRiskEstimKnownS}).~$\square$

\section{Свойства оценки риска при~использовании оценки дисперсии шума}

В работе~\cite{MarkinShestakovConsist} показано, что при достаточно слабых ограничениях 
на моменты оценки дисперсии шума для сходимости разности риска и его оценки к нулю по 
вероятности ее надо нормировать числом вейвлет-коэффициентов, т.\,е.\ порядок знаменателя 
вырастает почти на~1/2. Покажем, что в задаче томографии порядок тоже повышается почти на~1/2, 
но знаменатель уже будет много больше числа коэффициентов.

Введем обозначение
\begin{equation*}
\hslj = 2^j \hsig \left\|\xi^{[\lambda]}_{0,0,0}\right\|_2^2\,.
\end{equation*}

\medskip
\noindent
\textbf{Теорема 2.} \textit{Пусть справедливы предположения о регулярности~$f$. 
Пусть $\hsig$~--- оценка дисперсии, $\e\hsig-\sigma^2=\nu_L$ и 
$\D\hsig=\theta_L=\Obig(L^{-\beta})$, $\nu_L=\osml(1)$, $\beta>0$. 
Тогда при $L\rightarrow\infty$ выполнено}
\begin{equation}
\label{eq_ConsistTomo32}
\fr{\hat r(f)-r(f)}{L^{3/2}} \xrightarrow{\textsf{P}} 0\,.
\end{equation}

\medskip

\noindent
Д\,о\,к\,а\,з\,а\,т\,е\,л\,ь\,с\,т\,в\,о.\
Подобно доказательству теоремы~3 в~\cite{MarkinShestakovConsist} запишем
\begin{equation*}
\hat r-r = S_1 + S_2\,,
\end{equation*}
где
\begin{multline}
S_1 = \sumljk\left(\Yljk^2-\hslj\right) -{}\\
{}- \sumljk\e\left(\Yljk^2-\slj\right)\,; \label{eq_riskSplitSoTomo}
\end{multline}

\vspace*{-6pt}

\noindent
\begin{multline*}
S_2 = - \sumljk\left(\Yljk^2-\hslj\right)\indYjkghTj +{}\\
\!\!{}+ \sumljk\left(\hslj+\hTlj^2\right)\indYjkghTj +{} 
\end{multline*}

\noindent
\begin{multline}
{}+ \sumljk\e\left(\Yljk^2-\slj\right)\indYjkgTj -{}\\
{}- \sumljk\e\left(\slj+\Tlj^2\right)\indYjkgTj\,.
\label{eq_riskSplitStTomo}
\end{multline}
Далее будет показано, что при делении на $L^{3/2}$ и~$S_1$, и~$S_2$ сходятся к нулю по вероятности.

Сначала рассмотрим~$S_1$: по неравенству Чебышёва при любом $\delta>0$
\begin{multline}
\p\left( \fr{|S_1|}{L^{3/2}} > \delta \right) \leq{}\\
{}\leq
\fr{ \e\left( \sumljk \left( \Yljk^2 - \hslj - \e\Yljk^2 + \slj \right) \right)^2 }{\delta^2 L^3} ={}\\
{}= \fr{ \sumljk \e\left( \Yljk^2 - \hslj - \e\Yljk^2 + \slj \right)^2 }{ \delta^2 L^3 } + {}\\
{}+ \fr{1}{\delta^2 L^3}
 \sum \e\left( \Yljk^2 - \hslj - \e\Yljk^2 + \slj \right)\times{}\\
 {}\times \left( \Yljks^2 - \hsljs - \e\Yljks^2 + \sljs \right)\,.
\label{eq_riskSplitUnknSTomo}
\end{multline}
Во второй сумме~(\ref{eq_riskSplitUnknSTomo}) суммирование идет по индексам 
$(\lambda,j,\mathbf{k})\ne(\lambda',j',\mathbf{k}')$. Понятно, что первое слагаемое в~(\ref{eq_riskSplitUnknSTomo}) 
стремится к нулю~--- в сумме всего порядка~$L$ слагаемых, они имеют порядок не выше~$L$ и 
сумма делится на~$L^3$ (напомним, что $L=2^{2J}$).

Рассмотрим одно из слагаемых второй суммы~(\ref{eq_riskSplitUnknSTomo}):
\begin{multline*}
\e\left( \Yljk^2 - \hslj - \e\Yljk^2 + \slj \right) \times{}\\
{}\times\left( \Yljks^2 - \hsljs - \e\Yljks^2 + \sljs \right) = {}\\
{}= \e\Yljk^2\Yljks^2 - \e\Yljk^2\hsljs - \e\Yljk^2 \e\Yljks^2 +{}\\
{}+ \sljs \e\Yljk^2 - 
 \e\hslj\Yljks^2 + \e\hslj\hsljs +{}\\
 {}+ \e\hslj\e\Yljks^2 - \sljs\e\hslj 
- \e\Yljk^2 \e\Yljks^2 +{}\\
{}+ \e\hsljs\e\Yljk^2 + \e\Yljk^2 \e\Yljks^2 - \sljs\e\Yljk^2 
+ {}\\
{}+\slj\e\Yljks^2 - \slj\e\hsljs - \slj\e\Yljks^2 +{}\\
{}+ \slj\sljs = 
 - \cov\left( \hsljs,\,\Yljk^2 \right) -{}\\
 {}- \cov\left( \hslj,\,\Yljks^2 \right) +
 \fr{\slj\sljs}{\sigma^4}\left( \nu_L^2 + \theta_L \right)\,.
\end{multline*}
С учетом того, что $\D\Yljk^2$ имеет порядок~$2^{2j}$, а ковариацию можно оценить по неравенству Коши--Бу\-ня\-ков\-ско\-го, 
получаем, что каждое слагаемое второй суммы~(\ref{eq_riskSplitUnknSTomo}) можно оценить как 
$2^{j+j'}\cdot\osml(1)$. Всего таких слагаемых порядка~$L^2$, а максимальное значение $2^{j+j'}$ 
равно $2^{J-1+J-1}=L/4$. Следовательно, после суммирования получаем, что второе сла\-га\-емое 
в~(\ref{eq_riskSplitUnknSTomo}) оценивается как~$\osml(1)$. Значит, $S_1/L^{3/2}$ сходится к нулю по вероятности.

Для оценки~$S_2$ используем другую модификацию неравенства Чебышёва:
\begin{equation*}
\p\left( \fr{|S_2|}{L^{3/2}} > \delta \right) \leq \fr{\e|S_2|}{\delta L^{3/2}} = 
\fr{\e\left[|S_2|/L^{1/2}\right]}{\delta L}\,.
\end{equation*}
Величину $\e|S_2|$ можно оценить сверху суммой математических ожиданий модулей сумм, входящих в~$S_2$, 
а эти суммы, в свою очередь,~--- суммой математических ожиданий входящих в них слагаемых.

По формуле полной вероятности для некоторого $0<\gamma<1$ получаем
\begin{multline*}
\p\left( |\Yljk| > \hTlj \right)={}\\
{}=\p\left( |\Yljk| > \hTlj \,|\, \hTlj \leqslant (1-\gamma)\solj\sqrt{2\ln 2^{2j}} \right)\times{}\\
{}\times \p\left( \hTlj \leq (1-\gamma)\solj\sqrt{2\ln 2^{2j}} \right) + {}\\
{}+ \p\left( |\Yljk| > \hTlj \,,\, \hTlj > (1-\gamma)\solj\sqrt{2\ln 2^{2j}} \right)\,.
\end{multline*}
В силу свойств~$\hsig$
\begin{multline*}
\p\left( \hTlj \leq (1-\gamma)\solj\sqrt{2\ln 2^{2j}} \right)={}\\
{}=\p\left(\hsig\leqslant (1-\gamma)^2\sigma^2\right)\leq{}\\
{}\leq\p\left(|\hsig-\sigma^2-\nu_L|\geqslant(2\gamma-\gamma^2)\sigma^2+\nu_L\right)\leq{}\\
{}\leq \fr{\D\hsig}{\left((2\gamma-\gamma^2)\sigma^2+\nu_L\right)^2}=\Obig\left(L^{-\beta}\right)
\end{multline*}
для достаточно большого~$L$. Далее
\begin{multline}
\p\left( |\Yljk| > \hTlj \,,\, \hTlj > (1-\gamma)\solj\sqrt{2\ln 2^{2j}} \right) \leq{}\\
{}\leq \p\left( |\Yljk| > (1-\gamma)\solj\sqrt{2\ln 2^{2j}} \right) = {}\\
{}=
\fr{ C }{ 2^{2j(1-\gamma)^2}\cdot 2^{j/2}\sqrt{j} }\,.
\label{eq_prbSplitGammaTomo}
\end{multline}
Теперь оцениваем математические ожидания компонентов сумм из~$S_2$ при делении на~$L^{1/2}$:
\begin{multline*}
\e\left[\fr{\left| \Yljk^2 - \hslj \right|}{L^{1/2}}\indYjkghTj\right] \leq {}\\
{}\leq\sqrt{ \e\left[\fr{ \left(\Yljk^2 - \hslj\right)^2 }{2^{2J}}\right]  \p\left( |\Yljk| > \hTlj \right) } 
\rightarrow 0\,;
\end{multline*}

\vspace*{-6pt}
\noindent
\begin{multline*}
\e\left[\fr{\left| \hslj + \hTlj^2 \right|}{L^{1/2}}\indYjkghTj\right] \leq{}\\
{}\leq 
\left( 2\ln 2^{2j} + 1 \right) 2^{j-J} \times{}\\
{}\times\sqrt{ \e\left(\hslz\right)^2 \p\left( |\Yljk| > \hTlj \right) } \rightarrow 0
\end{multline*}
при $j\geq j_M$ и $J\rightarrow\infty$.
Остальные слагаемые оцениваются аналогично. Итак, $S_2/L^{3/2}$ тоже сходится к нулю по вероятности.~$\square$

\smallskip

Как и в одномерном случае (см.~\cite{MarkinLimitDistr}), порядок знаменателя в~(\ref{eq_ConsistTomo32}) 
можно понизить, введя дополнительные ограничения на~$\nu_L$.

\medskip

\noindent
\textbf{Теорема 3.}
\textit{Пусть справедливы предположения о регулярности~$f$. Пусть $\hsig$~--- 
оценка дисперсии, $\e\hsig-\sigma^2=$\linebreak $=\nu_L=\Obig(L^{-\upsilon})$ и 
$\D\hsig=\theta_L=\Obig(L^{-\beta})$, $\upsilon$, $\beta>0$. Тогда
при любом $a>1/2-c$, $c=\min\left\{1/2, \upsilon, \beta/2\right\}$ и $L\rightarrow\infty$ выполнено
\begin{equation*}
\fr{\hat r(f)-r(f)}{L^{a+1}} \xrightarrow{\textsf{P}} 0\,.
\end{equation*}}
\medskip

\noindent
Д\,о\,к\,а\,з\,а\,т\,е\,л\,ь\,с\,т\,в\,о.
Заметим, что $0<c\leq 1/2$ и, стало быть, $a>0$. Так же, как и в доказательстве теоремы~2, 
разобьем $\hat r - r$ на те же суммы~$S_1$ и~$S_2$ (см.\ формулы~(\ref{eq_riskSplitSoTomo})
и~(\ref{eq_riskSplitStTomo})), только~$S_1$ запишем в виде
\begin{multline*}
S_1 = \sumljk\left(\Yljk^2-\e\Yljk^2\right) - \sumljk \left(\hslj-\slj\right) = {}\\
{}= \sumljk\left(\Yljk^2-\e\Yljk^2\right) - {}\\
{}-\sum\limits_\lambda \sum_j 2^{2j}\,2^j \left(\hslz-\solz^2\right)\,.
%\label{eq_riskSplitSoLimTomo}
\end{multline*}
Первая сумма при делении на~$L$ сходится по распределению к нормальному закону 
(см.\ разд.~\ref{sect_ConsitKnownSTomo}) и, следовательно, сходится по вероятности к нулю при делении 
на~$L^{a+1}$, где $a>0$. Вторая сумма пред\-став\-ля\-ет собой произведение 
$\left(\hsig-\sigma^2\right)$ и множителя, имеющего порядок $2^{3J}=L^{3/2}$. Легко видеть, что
\begin{equation*}
\fr{L^{3/2}\left(\hsig-\sigma^2\right)}{L^{a+1}} \xrightarrow{\textsf{P}} 0
\end{equation*}
при указанных в формулировке теоремы ограничениях на~$a$.

Покажем теперь, что $S_2/L^{a+1}$ сходится к нулю по вероятности. 
Обозначим $\varkappa = a-1/2+c>$\linebreak $>\;0$. В теореме~2 есть оценки для вероятности 
$\p\left( |\Yljk| > \hTlj \right)$:
\begin{multline}
\p\left( |\Yljk| > \hTlj \right) = {}\\
{}=\max\left\{ \fr{C_1}{2^{2J\beta}},\,\fr{C_2}{2^{2j(1-\gamma)^2}\cdot 2^{j/2}\sqrt{j}} \right\} 
\label{eq_ProbYghTOrdersTomo}
\end{multline}
для некоторого $0<\gamma<1$. При $J\rightarrow\infty$ имеем
\begin{multline}
\label{eq_restEstimConsistTomo1}
\fr{\e \left(\Yljk^2\right)^2 C_1/2^{2J\beta}}{L^{2a}} \leq \fr{C_3\cdot 2^{2j}
\cdot 2^{-2j\beta}}{2^{2J(1-2c+2\varkappa)}} ={}\\
{}= \fr{C_3\cdot 2^{2j}\cdot2^{2J\min\left\{1, 2\upsilon, \beta\right\}}}{2^{2J}\cdot 2^{2J\beta}\cdot 2^{4J\varkappa}}  \rightarrow 0,
\end{multline}

\columnbreak 
%\vspace*{-6pt}

\noindent
\begin{multline}
\fr{\e \left(\Yljk^2\right)^2 C_2\cdot 2^{-2j(1-\gamma)^2}\cdot 2^{-j/2}/\sqrt{j}}{L^{2a}} \leq {}\\
{}\leq
\fr{C_4\cdot 2^{2j}\cdot 2^{2J\min\left\{1, 2\upsilon, \beta\right\}}}{2^{2j(1-\gamma)^2+j/2}\cdot 2^{2J}\cdot 2^{4J\varkappa}\sqrt{j}} \rightarrow 0
\label{eq_restEstimConsistTomo2}
\end{multline}
для достаточно малого~$\gamma$. Отсюда имеем для произвольного $\delta>0$
\begin{multline*}
\p\left(\fr{\sumljk \Yljk^2 \indYjkghTj }{L^{a+1}}>\delta\right) \leq{}\\
{}\leq
\fr{\sumljk \e \left[\Yljk^2/L^a\right] \indYjkghTj  }{\delta L} \rightarrow 0
\end{multline*}
при $J\rightarrow\infty$ в силу неравенств Чебышёва и Коши--Бу\-ня\-ков\-ско\-го. Оценки для суммы 
с членами вида $\hslj \indYjkghTj$ получаются аналогично. А для сумм, в которые входят $\indYjkgTj$, 
оценки получены в теореме~1.~$\square$

Можно сформулировать и доказать теорему сходимости по распределению к нетривиальному пределу.

\medskip

\noindent
\textbf{Теорема 4.} 
\textit{Пусть справедливы предположения о регулярности~$f$. 
Пусть $\hsig$~--- оценка дисперсии, 
$\e\hsig-\sigma^2=\nu_L=\Obig(L^{-\upsilon})$ и 
$\D\hsig=\theta_L=\Obig(L^{-\beta})$, $\upsilon>0$, $\beta>1/2$. 
Пусть $\hsig$ не зависит от $\Yljk$ и $\sqrt{L}\left( \hsig - \sigma^2 \right) 
\Rightarrow \mathcal{N}\left(0,\,\Sigma^2\right)$ при $L\rightarrow\infty$, тогда
\begin{multline*}
\fr{\hat r(f)-r(f)}{ L \sqrt{ b_2 \left( \silz^4 + \siilz^4 + \siiilz^4 \right) } } \Rightarrow{}\\
{}\Rightarrow \mathcal{N}\left( 0, 1+\frac{ \left( \silz^2 + \siilz^2 + \siiilz^2 \right)^2 \Sigma^2 }{ d_2 \,\sigma^4 \left( \silz^4 + \siilz^4 + \siiilz^4 \right) } \right)\,,
\end{multline*}
где $b_2=2/(2^4-1)=2/15$, $d_2 = (2(2^3-1)^2)/(2^4-1)=$\linebreak $=98/15$.}

\medskip

\noindent
Д\,о\,к\,а\,з\,а\,т\,е\,л\,ь\,с\,т\,в\,о.
В теореме~3 было существенным наличие~$\varkappa>0$, которое давало сходимость к нулю 
в~(\ref{eq_restEstimConsistTomo1}) (в~(\ref{eq_restEstimConsistTomo2}) это несущественно). 
Сейчас же $\varkappa=0$, поэтому доказательство необходимо изменить.

Оценим $S_2$ более тонко. Имеем
\vspace*{-9pt}

\noindent
\begin{multline*}
\Yljk^2 \indYjkghTj - \e \Yljk^2 \indYjkgTj = {}\\[3pt]
{}= \Yljk^2 \indYjkghTj - \Yljk^2 \indYjkgTj +{}\\[3pt]
{}+ \Yljk^2 \indYjkgTj - \e \Yljk^2 \indYjkgTj\,.
\vspace*{-3pt}
\end{multline*}
\vspace*{-18pt}

\pagebreak

Вопрос о двух последних слагаемых решен в теореме~1. Рассмотрим два первых:
\begin{multline*}
\e \left| \Yljk^2 \indYjkghTj - \Yljk^2 \indYjkgTj \right| = {}\\
{}=\e \Yljk^2 \Ik_{\Tlj<|\Yljk|\leqslant\hTlj} +{}\\
{}+ \e \Yljk^2 \Ik_{\hTlj<|\Yljk|\leqslant\Tlj}\,.
\end{multline*}
При этом
\begin{multline*}
\e \Yljk^2 \Ik_{\Tlj<|\Yljk|\leqslant\hTlj} \leq \e \hTlj^2 \indYjkgTj \leq{}\\
{}\leq \sqrt{\fr{C\cdot j^2\cdot 2^{2j}}{2^{2j+j/2}\sqrt{j}}}\rightarrow 0,\quad J\rightarrow\infty\,,
\end{multline*}

%\vspace*{-3pt}

\noindent
и

%\vspace*{-3pt}
\noindent
\begin{multline}
\e \Yljk^2 \Ik_{\hTlj<|\Yljk|\leq\Tlj} \leq {}\\
{}\leq \Tlj^2 \e \Ik_{\hTlj<|\Yljk|\leq 
 \Tlj} \leq{}\\
{}\leq C j\cdot 2^{j}
\e \indYjkghTj\,.
\label{eq_FineRestEstimTomo}
\end{multline}
С учетом~(\ref{eq_ProbYghTOrdersTomo}) получаем, что
\begin{equation*}
\e \Yljk^2 \Ik_{\hTlj<|\Yljk|\leqslant\Tlj} \rightarrow 0
\end{equation*}
при $J\rightarrow\infty$ и $\beta>1/2$. Отметим, что, в отличие от работы~\cite{MarkinLimitDistr}, 
требование на~$\beta$ повысилось (там требовалось только $\beta>0$). Это является следствием роста дисперсии с 
ростом~$j$, которое выражается в наличии множителя~$2^j$ в~(\ref{eq_FineRestEstimTomo}). 
Аналогично получаем соотношения для~$\hTlj$:
\begin{multline*}
\e \hTlj^2 \Ik_{\Tlj<|\Yljk|\leq\hTlj} \leq \e \hTlj^2 \indYjkgTj \leq{}\\
{}\leq \sqrt{ \fr{ C j^2 \cdot 2^{2j} }{ 2^{2j+j/2}\sqrt{j} } } \rightarrow 0\,;
\end{multline*}

\vspace*{-12pt}

\noindent
\begin{multline*}
\e \hTlj^2 \Ik_{\hTlj<|\Yljk|\leq\Tlj} \leq{}\\
{}\leq \Tlj^2 \e\Ik_{\hTlj<|\Yljk|\leq\Tlj} \rightarrow 0\,.
\end{multline*}
Для $\hslj$ заметим, что $\hslj\leqslant\hTlj^2$. После применения неравенства Чебышёва получим, что 
$S_2/L$ сходится к нулю по вероятности.

В~$S_1$ оба слагаемых сходятся по распределению к нормальному закону и при этом независимы. 
Поэтому их сумма тоже сходится по распределению к нормальному закону. Осталось убедиться в 
правильности параметров. Имеем
\begin{multline*}
\sumljk \left(\hslj-\slj\right) = \sum\limits_\lambda \sum_j 2^{2j}\cdot 2^j \left(\hslz-\solz^2\right) = {}\\
{}= \left( \left\|\xi^{[1]}_{0,0,0}\right\|_2^2 + \left\|\xi^{[2]}_{0,0,0}\right\|_2^2 + 
\left\|\xi^{[3]}_{0,0,0}\right\|_2^2 \right)\times{}\\
{}\times \fr{2^{3J}-2^{3j_M}}{2^3-1} \left(\hsig-\sigma^2\right) = {}
\end{multline*}

\noindent
$$%\begin{multline*}
{}= \fr{  \silz^2 + \siilz^2 + \siiilz^2 }{ \sigma^2 }\, \fr{2^{3J}-2^{3j_M}}{7} \left(\hsig-\sigma^2\right)\,.\hfil\square
$$%\end{multline*}

%\columnbreak
\medskip

\noindent
\textbf{Замечание}. 
Если функция $f$ регулярная с параметром $\alpha\geq 1/4$, а $j_M\geq 4J/5$, то можно ослабить требования 
на~$\hsig$. Достаточно потребовать только состоятельность, асимптотическую нормальность и независимость от~$\Yljk$.

\smallskip

В теоремах~2 и~3 при оценке $\p\left( |\Yljk| > \hTlj \right)$ использовалось число $0<\gamma<1$. 
Можно заменить~$\gamma$ бесконечно малой последовательностью~$\gamma_L$, которая не испортит порядок знаменателя 
в~(\ref{eq_prbSplitGammaTomo}).

По формуле полной вероятности для любого $\delta>0$
\begin{multline}
\label{eq_sumOfIndicTomo}
\p\left( \sumljk\indYjkghTj > \delta \right) ={}\\
{}= \p\left( \hTlj \leqslant \left( 1-\gamma_L \right)\solj\sqrt{2\ln 2^{2j}} \right) \times{} \\
{}\times \p\left( \sumljk \indYjkghTj>\delta\,|\,\hTlj \leqslant{}\right.\\
\left.{}\vphantom{\sumljk\indYjkghTj}\leq \left(1-\gamma_L\right) \solj\sqrt{2\ln 2^{2j}} \right) +{} \\
{}+ \p\left(\sumljk\indYjkghTj>\delta\,,\right. \\
\left.\vphantom{\sumljk\indYjkghTj}\hTlj > \left( 1-\gamma_L \right) \solj\sqrt{2\ln 2^{2j}}\right)\,,
\end{multline}
где
$\gamma_L = 1/J$.
При таком $\gamma_L$ получаем
\begin{multline*}
\p\left( |\Yljk| > (1-\gamma_L)\solj\sqrt{2\ln 2^{2j}} \right) = {}\\
{}=\fr{C}{ 2^{2j(1-\gamma_L)^2}\cdot 2^{j/2}\sqrt{j} } \leq \fr{C_1}{ 2^{2J} \sqrt{j} }
\end{multline*}
в силу выбора $j_M$ и того, что
\begin{equation*}
2^{2j\left(1-\gamma_L\right)^2} = 2^{2j\left( 1-2/J + 1/J^2 \right)} > 2^{2j-4}\,.
\end{equation*}
По неравенству Чебышёва
\begin{multline*}
\p\left(\sumljk\indYjkghTj>\delta\,, \right.\\
\left. \vphantom{\sumljk\indYjkghTj}\hTlj > \left( 1-\gamma_L \right) \solj\sqrt{2\ln 2^{2j}}\right) \leq{}\\
{}\leq \p\left( \sumljk \Ik_{ |\Yljk| > (1-\gamma_L)\solj\sqrt{2\ln 2^{2j}} } > \delta \right) \leq{}
\end{multline*}

\noindent
\begin{multline*}
{}\leq \fr{ \sumljk \p\left( |\Yljk| > (1-\gamma_L)\solj\sqrt{2\ln 2^{2j}} \right) }{\delta} = {}\\
{}=\Obig\left( \fr{1}{\sqrt{j}} \right)\,.
\end{multline*}

Используя свойство асимптотической нормальности~$\hsig$, можно для любого $\delta'>0$ оценить
\begin{equation}
\label{eq_hatTdevProbAsympTomo}
\p\left( \hTlj \leq (1-\gamma_L)\solj\sqrt{2\ln 2^{2j}} \right)< \delta'\,,
\end{equation}
причем отметим, что~$\delta$ здесь фиксировано, а~$\delta'$ можно делать произвольно малым. Имеем
\begin{multline*}
%\label{eq_hatTdevProbAsymp}
\p\left(\hTlj\leqslant(1-\gamma_L)\solj\sqrt{2\ln 2^{2j}}\right) ={}\\
{}= \p\left(\hsig\leqslant(1-\gamma_L)^2\sigma^2\right)={}\\
{}= \p\left( \left(\hsig-\sigma^2\right) \leqslant \sigma^2\left(-2\gamma_L+\gamma_L^2\right) \right) ={}\\
{}= \p\left( \sqrt{L}\left(\hsig-\sigma^2\right) \leqslant -\fr{\sqrt{L}\sigma^2(2J-1)}{J^2} \right)\,.
\end{multline*}
Для произвольного~$\delta'>0$ найдется $J_0$ ($L_0=2^{2J_0}$) такое, что
\begin{equation*}
F_\Sigma\left( -\fr{\sqrt{L_0}\sigma^2(2J_0-1 )}{J_0^2} \right) < \fr{\delta'}{2}\,,
\end{equation*}
где $F_\Sigma$~--- функция распределения нормального закона с нулевым средним и дисперсией~$\Sigma^2$. 
При этом для любого $J\geq J_0$
\begin{multline*}
\p\left( \sqrt{L}\left(\hsig-\sigma^2\right) \leq -\fr{\sqrt{L}\sigma^2(J -1 )}{J^2} \right) \leq{} \\
{}\leq \p\left( \sqrt{L}\left(\hsig-\sigma^2\right) \leq -\fr{\sqrt{L_0}\sigma^2(2J_0 -1 )}{J_0^2}  \right)\,.
\end{multline*}
В силу асимптотической нормальности~$\hsig$ и непрерывности~$F_\Sigma$ для этого же~$\delta'$ 
найдется~$J_1$ $\left(L_1=2^{2J_1}\right)$ такое, что для любого $J\geq J_1$
\begin{equation*}
\left| \p\left( \sqrt{L_1}\left(\hsig-\sigma^2\right) \leq x \right) - F_\Sigma(x)\right| < \fr{\delta'}{2}\,,
\end{equation*}
причем $J_1$ не зависит от~$x$. Возьмем $x_0 =$\linebreak $= -\sqrt{L_0}\sigma^2(2J_0-1)/J_0^2$ и 
$J_2 = \max\{J_0,J_1\}$. Для любого $J\geq J_2$ имеем
\begin{equation*}
\p\left( \sqrt{L}\left(\hsig-\sigma^2\right) \leq x_0 \right) < \delta'\,,
\end{equation*}
а значит, справедливо~(\ref{eq_hatTdevProbAsympTomo}).

Получаем, что сумма индикаторов в~(\ref{eq_sumOfIndicTomo}) сходится к нулю по вероятности:
\begin{equation*}
%\label{eq_sumIndConsisthTTomo}
\p\left( \sumljk\indYjkghTj > \delta \right) \rightarrow 0 \mbox{ при }J\rightarrow\infty\,.
\end{equation*}
Для суммы индикаторов с неслучайным порогом аналогично получаем
\begin{equation*}%\label{eq_sumIndConsistTTomo}
\p\left( \sumljk\indYjkgTj > \delta \right) \rightarrow 0\,.
\end{equation*}
Далее воспользуемся дискретной версией неравенства Коши--Буняковского:
\begin{multline*}
\fr{ \sumljk \Yljk^2\indYjkghTj }{ L } \leq{}\\
{}\leq \sqrt{ \fr{\sumljk \Yljk^4/L}{L} \, \sumljk \indYjkghTj } \,\xrightarrow{\textsf{P}} 0\,,
\end{multline*}
так как $\e\left[ \Yljk^4/L \right]$ ограничено,
\begin{equation*}
\fr{ \sumljk \hTlj^2 \indYjkghTj }{L} \leq \fr{ \sumljk \Yljk^2 \indYjkghTj }{L} \xrightarrow{\mathsf{P}} 0
\end{equation*}
и
\begin{equation*}
\hslj\indYjkghTj \leqslant \hTlj^2\indYjkghTj\,.
\end{equation*}
Оценки для слагаемых с $\indYjkgTj$ получены в теореме~1.

\medskip

\noindent
\textbf{Замечание}. 
Всюду выше в этом разделе предполагалось, что пороговая обработка и суммирование в выражении для 
риска~(\ref{eq_riskEstimDefTomo}) ведутся с уровня~$j_M$, причем $j_M\rightarrow\infty$ при 
$J\rightarrow\infty$. Однако если ввести дополнительные ограничения на регулярность~$f$, 
то можно вести пороговую обработку и суммирование с уровня $j_0\nrightarrow\infty$. Если 
$j_M=J/(\alpha+1)$, то для коэффициентов, соответствующих $j<j_M$, неравенство~(\ref{eq_WaveletCoeffUpperBoundTomo}), 
вообще говоря, не выполнено. Оценим вклад больших коэффициентов в оценку риска:
\begin{multline*}
L^{-1} \sum\limits_{j=j_0}^{j_M-1}\sum\limits_{\lambda,\mathbf{k}} \left\{\left|\Yljk^2-\hslj\right|\indYjklhTj +{}\right.\\
\left.{}+ \left(\hslj+\hTlj^2\right)\indYjkghTj \right\} \leq{}\\
{}\leq L^{-1} \sum\limits_{j=j_0}^{j_M-1}\sum\limits_{\lambda,\mathbf{k}} \left\{ \left(\hslj+\hTlj^2\right) +
 \left(\hslj+\hTlj^2\right) \right\} \xrightarrow{\mathsf{P}}{}\\
\xrightarrow{\mathsf{P}} {} 0
\end{multline*}
в силу состоятельности~$\hsig$ и того, что

\noindent
\begin{multline*}
L^{-1}\left\{\sum\limits_{j=j_0}^{j_M-1}j2^j\cdot2^{2j}\right\} \leq 2^{-2J}
\left\{j_M\sum\limits_{j=j_0}^{j_M-1}2^{3j}\right\} \simeq{}\\
{}\simeq 2^{2J}\cdot j_M\cdot2^{3j_M}\rightarrow 0
\end{multline*}
при $J\rightarrow\infty$, если $3j_M<2J$, т.\,е.\ 
$\alpha>1/2$. Слагаемые риска оцениваются аналогично. Итак, 
при $\alpha>1/2$ суммирование в~(\ref{eq_riskEstimDefTomo}) можно начинать с произвольного~$j_0$.


{\small\frenchspacing
{%\baselineskip=10.8pt
\addcontentsline{toc}{section}{Литература}
\begin{thebibliography}{99}

\bibitem{Natterer} %1
\Au{Наттерер Ф.} 
Математические аспекты компьютерной томографии.~--- М.: Мир, 1990.

\bibitem{TikhonovArsenin}  %2
\Au{Тихонов А.\,Н., Арсенин В.\,Я.} 
Методы решения некорректных задач.~--- М.: Наука, 1979.

\bibitem{Herman}  %3
\Au{Хермен Г.} 
Восстановление изображений по проекциям: основы реконструктивной томографии.~--- М.: Наука, 1983.

\bibitem{Daub}  %4
\Au{Добеши И.} 
Десять лекций по вейвлетам.~--- Ижевск: НИЦ <<Регулярная и хаотическая динамика>>, 2001.

\bibitem{DonohoWVD} 
\Au{Donoho D.\,L.} 
Nonlinear solution of linear inverse problems by wavelet-vaguelette decomposition~// 
Appl. Comput. Harmonic Anal., 1995. Vol.~2. P.~101--126.

\bibitem{KolaczykArticle}  %7
\Au{Kolaczyk E.\,D.} 
A wavelet shrinkage approach to tomographic image reconstruction~// J. Amer. Statistical Association, 1996. 
Vol.~91. No.\,435. P.~1079--1090.

\bibitem{KolaczykThesis} %6
\textit{Kolaczyk E.\,D.} 
Wavelet methods for the inversion of certain homogeneous linear operators in the presence of noisy data.  Ph.D.\ 
Thesis, 1994.

\bibitem{DJideal} 
\textit{Donoho D.\,L., Johnstone I.\,M.} 
Ideal spatial adaptation via wavelet shrinkage~// Biometrika, 1994. Vol.~81. No.\,3. P.~425--455.

\bibitem{DJunkn}  %9
\textit{Donoho D.\,L., Johnstone I.\,M.} 
Adapting to unknown smoothness via wavelet shrinkage~// J. Amer.\ Statistical Association, 1995. Vol.~90. P.~1200--1224.

\bibitem{Mallat} %10
\Au{Mallat S.} 
A wavelet tour of signal processing.~--- Academic Press, 1999.


\bibitem{MarkinLimitDistr}  %11
\Au{Маркин А.\,В.} 
Предельное распределение оценки риска при пороговой обработке вейвлет-ко\-эф\-фи\-ци\-ен\-тов~// 
Информатика и её применения, 2009. Т.~3. Вып.~4. С.~57--63.

\label{end\stat}

\bibitem{MarkinShestakovConsist}  %12
\Au{Маркин А.\,В., Шестаков О.\,В.} 
О состоятельности оценки риска при пороговой обработке вейвлет-ко\-эф\-фи\-ци\-ен\-тов~// Вестник Московского университета. 
Сер.~15. Вычислительная математика и кибернетика, 2010. №\,1. С.~26--33.


 \end{thebibliography}
}
}

\end{multicols}    %+2

\def\E{\mbox{\rm E}}
%\def\erm{\mbox{\rm e}}
\def\O{\mbox{\rm O}}
\def\o{\mbox{\rm o}}

\def\P{\mbox{\rm P}}

\def\vx{\mbox{\boldmath{$x$}}}
\def\vy{\mbox{\boldmath{$y$}}}


\def\IP{\mbox{$\Phi$}}
\def\Ip{\mbox{$\phi$}}

\def\stat{kavag}


\def\tit{ПРИБЛИЖЕНИЯ ДЛЯ СТАТИСТИК, ОПИСЫВАЮЩИХ ГЕОМЕТРИЧЕСКИЕ СВОЙСТВА 
ДАННЫХ БОЛЬШОЙ РАЗМЕРНОСТИ, С~ОЦЕНКАМИ ОШИБОК$^*$}

\def\titkol{Приближения для статистик, описывающих геометрические свойства данных большой размерности} %, с~оценками ошибок}

\def\autkol{Ю.~Кавагучи, В.\,В.~Ульянов, Я.~Фуджикоши}
\def\aut{Ю.~Кавагучи$^1$, В.\,В.~Ульянов$^2$, Я.~Фуджикоши$^3$}

\titel{\tit}{\aut}{\autkol}{\titkol}

{\renewcommand{\thefootnote}{\fnsymbol{footnote}}\footnotetext[1]
{Исследования выполнены при частичной поддержке РФФИ, гранты 08-01-00567, 08-01-91205, 09-01-12180.}}

\renewcommand{\thefootnote}{\arabic{footnote}}
\footnotetext[1]{Высшая научная и инженерная школа, Университет Чуо, Токио, n15007@gug.math.chuo-u.ac.jp}
\footnotetext[2]{Московский государственный университет им.\
М.\,В.~Ломоносова, факультет вычислительной математики и кибернетики,
 vulyan@gmail.com}
\footnotetext[3]{Высшая научная и инженерная школа, 
Университет Чуо, Токио, fuji@math.sci.hiroshima-u.ac.jp}

\vspace*{2pt}

\Abst{При описании геометрических свойств~$n$  наблюдений с~$p$ признаками необходимо 
исследовать асимптотическое поведение трех статистик: длины $p$-мерного вектора наблюдений, 
расстояния между двумя векторами наблюдений и угла между ними. В~[1] найдено асимптотическое 
поведение указанных статистик, когда размерность~$p$ стремится к бесконечности, а объем выборки~$n$ 
остается постоянным. В настоящей работе результаты~[1] уточняются путем построения асимптотических разложений. 
Кроме этого, найдены вычислимые оценки погрешности приближения предельными распределениями для первых двух 
статистик. Эти результаты будут также полезны  в ситуациях, когда размерность данных не является чрезмерно большой.}

\vspace*{1pt}

\KW{данные большой размерности; асимптотические разложения; оценки погрешности; геометрические свойства}

\vspace*{3pt}

     \vskip 18pt plus 9pt minus 6pt

      \thispagestyle{headings}

      \begin{multicols}{2}

      \label{st\stat}

\section{Введение}

Рассмотрим случайную выборку  $\vx_1, \ldots, \vx_n$, взятую из  $p$-мер\-но\-го распределения. 
Набор данных можно рассматривать как  $n$~векторов или точек в $p$-мер\-ном пространстве. При анализе данных 
полезным оказывается исследование следующих трех геометрических характеристик: длины $p$-мер\-но\-го 
вектора наблюдений, расстояния между двумя векторами наблюдений и угла между ними.

В последние годы значительный интерес вызывают исследования для данных большой раз\-мер\-ности. 
Это связано с тем, что такого типа данные встречаются все в большем числе приложений, в частности 
при анализе финансовых рынков и деятельности финансовых организаций, в социологии, генетике, биологии 
и математической физике.
В асимптотической теории для данных большой размерности предполагается, что либо ($i$)~оба параметра~--- 
размерность~$p$ и объем выборки~$n$~--- стремятся к бесконечности, либо ($ii$)~$p$~стремится к бесконечности, а 
объем~$n$ фиксирован.

 Недавний результат в рамках предположения~($i$) представлен, например, в~[2], где получены также вычислимые 
 оценки точности приближений.

 Если~$\vx_i$  взяты из нормального распределения~ $N(0, I_p)$,  в~[1] в предположении~($ii$) 
 показано, что три статистики, описывающие геометрические свойства данных, удовлетворяют следующим 
 предельным соотношениям:
\begin{align*}
\|\vx_i\| &= \sqrt{p} +\O_p(1), \!\!\!& i&=1,\ldots,n\,; \\
\|\vx_i- \vx_j\| &= \sqrt{2p} +\O_p(1), \!\!\!& i,j&=1,\ldots n\,,\enskip i \neq j\,; \\
\mathrm{ang}(\vx_i,\vx_j) &= \fr{1}{2}\pi +\O_p(p^{-1/2}),\!\!\!& i,j&=1,\ldots n,~i \neq j\,,
\end{align*}
где $\| \cdot \|$~--- евклидово расстояние и  $\O_p$ обозначает стохастический порядок малости. 
Из этих  результатов вытекает, что при увеличении размерности данные сходятся к вершинам правильного симплекса.

В настоящей работе  эти результаты сначала уточняются  путем построения асимптотических разложений 
для распределений указанных трех статистик. Затем   строятся вычислимые оценки погрешности для 
предельных распределений первых двух статистик. При этом получены два типа неравенств с использованием 
идей из работы~[3], где найдена оценка порядка~$\O(p^{-1})$  хи-квадрат приближения для преобразованной 
хи-квадрат случайной величины с $p$~степенями свободы. Тем самым получена возможность судить о геометрических 
свойствах статистических данных, размерность которых не является чрезмерно большой.


\section{Асимптотические разложения для распределений статистик}

Пусть
$\vx_i = (x_{i1}, \ldots, x_{ip})^\prime$ $(i =1, \ldots, n)$ есть выборка из распределения~$N(0,I_p)$.

Рассмотрим сначала статистику
$$
\|\vx_i\|= S_{ii}^{1/2}\,,
$$
где  $S_{ii} = \sum\limits^{p}_{k=1} x^2_{ik}$. Поскольку статистика~$S_{ii}$ распределена как~$\chi_p^{2}$~--- 
хи-квадрат случайная величина с $p$~степенями свободы, она не зависит от~$i$.  Положим
\begin{equation*}
 V = \fr{S_{ii} - p}{\sqrt{2p}}\,.
\end{equation*}
Распределение~$V$ сходится к стандартному нормальному при стремлении~$p$ к бесконечности.
Поскольку  $S_{ii} = p(1 + \sqrt{2}\,p^{-1/2}V)$, имеем
\begin{multline*}
\|\vx_i\|= S_{ii}^{1/2}
     =\sqrt{p}\left(1 + \fr{1}{2}\,\sqrt{2}Vp^{-1/2} -{}\right.\\
\left.     {}- \fr{1}{4}\,Vp^{-1} + \fr{1}{8}\,\sqrt{2}V^3 p^{-3/2} + \cdots \right)\,.
\end{multline*}

Рассмотрим асимптотическое разложение для распределения случайной величины
$$
T_1 = \sqrt 2(\|\vx_i\| -\sqrt{p})\,.
$$
Характеристическая функция величины~$T_1$ запишется в виде
\begin{multline*}
C_{T_1}(t) = \E\left[\exp\left\{(it) \sqrt 2 \left(\|\vx_i\| - \sqrt{p}\right)\right\}\right] ={}\\
= C_0(t) + \fr{1}{\sqrt{p}}\,C_1(t)+ \fr{1}{p}\,C_2(t) + \O\left(p^{-3/2}\right)\,,
\end{multline*}
где
\begin{align*}
C_0 (t) &= \E\left[\exp \left \{(it) V \right\}\right]\,; \\
C_1 (t) &= \E\left[- \fr{1}{4}\,\sqrt{2}(it)V^2 \exp \left \{(it) V\right\} \right]\,;\\
C_2 (t) &= \E\left[ \left(\fr{1}{4}\,(it)V^3 + \fr{1}{16}\,(it)^2 V^4 \right) \exp\{(it)V \}\right]\,.
\end{align*}
Для вычисления функций~$C_0(t)$, $C_1(t)$ и~$C_2(t)$ используем асимптотическое разложение для плот\-ности 
случайной величины (с.в.)~$V$.
Характеристическая функция с.в.~$V$ может быть записана в виде
\begin{multline*}
C_V(t) = \exp\left\{\fr{1}{2}\left(it\right)^2\right\}
            \left[1 + \fr{1}{\sqrt{p}}\,\fr{1}{6}(it)^3\kappa_3 +{}\right.\\
\left.{}  + \fr{1}{p}\left\{\fr{1}{72}\,(it)^6\kappa_3^2 + \fr{1}{24}\,(it)^4\kappa_4 \right\} + \O\left(p^{-3/2}\right)\right]\,.
\end{multline*}
Соответственно, плотность с.в.~$V$ раскладывается как
\begin{multline*}
\Ip(x)\left[1 + \fr{1}{\sqrt p}\,\fr{\sqrt{2}}{3}\,h_3(x)
                  + {}\right.\\
\left.                  {}+\fr{1}{p}\left\{\fr{1}{9}\,h_6(x) + \fr{1}{2}\,h_4(x) \right\}\right]+ \O\left(p^{-3/2}\right)\,,
\end{multline*}
где  $h_i(x)$~--- многочлен Эрмита порядка~$i$, а $\Ip(x)$~--- плотность стандартного нормального распределения.
Используя это разложение,   можно  найти $C_0(t)$, $C_1(t)$ и~$C_2(t)$.
Асимптотическое разложение для распределения с.в.~$T_1$ можно получить, обращая ее характеристическую функцию. 
Тем самым доказана следующая теорема:

\medskip

\noindent
\textbf{Теорема 1.} %\label{th2.1}
\textit{Пусть $\vx_i$ есть $p$-мерный случайный вектор с распределением~$N(0,I_p)$.
Тогда функция распределения с.в.\ $T_1 = \sqrt 2(\|\vx_i\| -\sqrt{p})$ может быть представлена в виде}
\begin{equation*}
\IP(x) -\Ip(x)\left[\fr{1}{\sqrt p}\, \ell_1(x) + \fr{1}{p}\,\ell_2(x) \right] + \O(p^{-3/2})\,,
\end{equation*}
\textit{где $\IP(x)$~--- функция распределения стандартного нормального закона, а $\ell_1(x)$ и~$\ell_2(x)$ 
определяются формулами}
\begin{align*}
\ell_1(x) &= \fr{1}{12}\,\sqrt{2}h_2(x) - \fr{1}{4}\,\sqrt{2}h_0(x)\,;\\
\ell_2(x) &=\fr{1}{144}\,\left[  -15 h_5(x) -6 h_3(x) + 16 h_2(x) -{}\right.\\
&\hspace*{10mm}\left.{}- 81 h_1(x) + 72h_0(x)\right]\,.
\end{align*}


Распределение с.в.\
$\| \vx_i - \vx_j \|$ $(i \neq j)$ тесно связано с распределением
$\| \vx_i \|$, так как $(x_{ik} - x_{jk})/\sqrt{2}\sim$\linebreak $ \sim N(0,1)$. Точнее, распределение 
с.в. $\| \vx_i - \vx_j \|/\sqrt{2}$ совпадает с распределением с.в.~$\| \vx_i \|$. 
Следовательно, распределение~$T_1$ совпадает с распределением
\begin{equation*}
T_2 = \sqrt{2}\left\{\fr{\| \vx_i - \vx_j \|}{\sqrt{2}}-\sqrt{p}\right\}
=\| \vx_i - \vx_j \|-\sqrt{2p}
%\label{deft2}
\end{equation*}
и справедливо следствие:

\medskip

\noindent
\textbf{Следствие 1.} %\label{cor2.2}
\textit{Пусть $\vx_i$ и $\vx_j$~--- независимые случайные векторы с распределением~$N(0,I_p)$.
Тогда распределение $T_2 =\| \vx_i - \vx_j \| -\sqrt{2p}$ совпадает с распределением~ $T_1$. 
В~частности, для функции распределения~$T_2$  имеет место то же асимптотическое разложение, что и для~$T_1$.}
\medskip

Далее рассмотрим асимптотическое разложение для распределения угла~$\theta$ между двумя независимыми 
векторами~$\vx_i$ и~$\vx_j$, взятыми из нормального распределения~$N(0,I_p)$.
Поскольку
$$
\|\vx_i - \vx_j\|^2 = \sum\limits^{p}_{k = 1} \left(x_{ik} - x_{jk} \right)^2 = S_{ii} +S_{jj} -2 S_{ij}\,,
$$
где  $S_{ij} = \sum\limits^{p}_{k=1} x_{ik} x_{jk}$, получаем
\begin{multline*}
\cos \theta = \fr{\|\vx_i\|^2+\|\vx_j\|^2 - \|\vx_i -\vx_j\|^2}{2 \|\vx_i\|\|\vx_j\|}={}\\
{} =\fr{S_{ij}}{\sqrt{S_{ii}S_{jj}}}
                = r_{ij} \,,
\end{multline*}
где $r_{ij}$~--- выборочный коэффициент корреляции. Распределение~$r_{ij}$ исследовалось во многих работах. 
В~частности, если  $\rho_{ij} = 0$, для функции распределения~$r_{ij}$ имеем~[4]:
\begin{multline*}
\P\left (\sqrt{p}\cdot r_{ij} < x\right)={}\\
{}= \IP(x) +\fr{1}{p}\, \left \{- \fr{3}{4}\,x
+\fr{1}{4}\,x^3\right\} \Ip(x) 
+ \O\left(p^{-3/2}\right)\,.
\end{multline*}
Поскольку~ $\theta$ может быть разложен в терминах~$r_{ij}$ в виде
\begin{equation*}
\theta = \arccos r_{ij}
       = \fr{1}{2}\,\pi - \left(r_{ij} + \fr{1}{6}\,r_{ij}^3 + \fr{3}{40}\, r_{ij}^5 + \cdots\right)\,,
\end{equation*}
полагая   $y = \sqrt{p}\cdot r_{ij}$, имеем
\begin{equation*}
\sqrt p \left( \fr{1}{2}\,\pi - \theta \right) =  y + \fr{1}{6p}\,y^3 + \o\left(p^{-1}\right)\,.
\end{equation*}
Тем самым характеристическую функцию $T_3 =$\linebreak $= \sqrt {p} \left(\pi/2 - \theta \right)$ можно записать в виде
\begin{equation*}
C_{\mathrm{ang}}(it) = C_0(t) +\fr{1}{6p}\,C_1(t) + \o(p^{-1})\,,
\end{equation*}
где
\begin{align*}
C_0(t) &= \E \left[\exp\left\{(it) y\right\}\right ]\,;\\
C_1(t) &= \E \left[\exp\left\{(it) y\right\}(it)y^3\right ]\,.
\end{align*}
Используя асимптотическое разложение для распределения~$\sqrt{p}\cdot r_{ij}$,
вычисляем~$C_0(t)$ и~$C_1(t)$. Обращая полученную характеристическую функцию, приходим к следующей теореме:

\smallskip

\noindent
\textbf{Теорема 2.} %\label{th2.4}
\textit{Пусть $\vx_i = (x_{i1}, \ldots, x_{ip})^\prime$ есть $p$-мер\-ный случайный вектор, 
взятый из нормального распределения~$N(0,I_p)$, и~$\theta$~--- угол между двумя независимыми 
векторами~$\vx_i$ и~$\vx_j$.
Тогда функция распределения с.в.\  $T_3 = \sqrt{p}\left(\pi/2 - \theta\right)$ может быть представлена в виде}
\begin{equation*}
\IP(x) +\fr{1}{12p}\,
       \left[h_3(u) - 6h_1(x)\right]\Ip(x) + \o(p^{-1})\,.
\end{equation*}


\section{Оценки погрешностей приближений}

В этой части   получены вычислимые выражения~$B(p)$ такие, что
%\vspace*{-1pt}
\begin{equation}
| \P(T_1 \le x) - \IP(x) | \le  B(p) = \O(p^{-1/2})\,,
\label{ebound1}
\end{equation}
где $T_1 = \sqrt 2(\|\vx_i\| -\sqrt{p})$. В силу следствия~1 результат~(\ref{ebound1}) 
справедлив и для $T_2 =\| \vx_i - \vx_j \| -\sqrt{2p}$.

Идея доказательства опирается на работу~[3]. Рассмотрим случайную величину
\begin{equation*}
 V_p = \fr{\chi_p^2 - p}{\sqrt{2p}}\,.
\end{equation*}
Пусть~$h(x)$ есть действительная функция, определяемая формулой
$$
h(x) = \sqrt{2}\left[\left(\sqrt{2p}\cdot x + p \right)^{{1}/{2}} - \sqrt{p}\right]\,.
$$
Тогда
\begin{equation*}
 h(V_p)=\sqrt{2}\left(\sqrt{\chi_p^2}- \sqrt{p}\right)
       =\sqrt{2}\left(\|X\|- \sqrt{p}\right)
       =T_1\,.
\end{equation*}
Если   $x \ge -\sqrt{2p}$,    то

\vspace*{-1pt}

\noindent
\begin{multline*}
\sup\limits_{-\sqrt{2p}\; \le\; x} \left |\P(T_1 \le x) - \IP(x) \right|={}\\
{}  =\sup\limits_{-\sqrt{2p}\; \le\; x} \left| \P(h(V_p) \le x) - \IP(x)\right| ={}\\
{}=\sup\limits_{-\sqrt{2p} \;\le\;
 x} \left| \P\left(V_p \le\, x + \fr{x^2}{2\sqrt{2p}}\right) - \IP(x)\right|\leq{}\\
{} \leq\sup\limits_{-\sqrt{2p}\; \le\; x} \left(\left|I_1\right| +\left|I_2 \right|\right)\,,
\end{multline*}
где
\vspace*{-4pt}

\noindent
\begin{align*}
                 I_1&= \P\left(V_p  \le x + \fr{x^2}{2\sqrt{2p}}\right)
         - \IP \left(x + \fr{x^2}{2\sqrt{2p}}\right)\,;\\
                 I_2&= \IP \left(x + \fr{x^2}{2\sqrt{2p}}\right)
             - \IP(x) \,.
\end{align*}
Для  $I_1$   в следующем разделе (см.\ леммы~1 и~2 ниже) будут получены два типа оценок:

\vspace*{-1pt}

\noindent
\begin{multline*}
|I_1| =\left| \P\left(V_p \le x + \frac{x^2}{2\sqrt{2p}}\right)
         - \IP \left( x + \frac{x^2}{2\sqrt{2p}}\right)\right|
      \leq{}\\
      {}\leq D_i(\lambda,p) \ \mbox{для}\  i=1, 2.
\end{multline*}
  Здесь же   займемся оцениванием~$I_2$:
%  \pagebreak

\vspace*{-3pt}  
  \noindent
\begin{multline*}
|I_2|  =\left |\IP \left(V_p  \le x + \fr{x^2}{2\sqrt{2p}}\right) - \IP(x)\right|={}\\
       =\left |\fr{1}{\sqrt{2 \pi}}\,\int\limits^{x + x^2/(2\sqrt{2p})}_{x} \exp \left\{- \fr{z^2}{2}\right\}\,dz\right|\,.
\end{multline*}
\pagebreak

\noindent
Если     $x \ge 0$, то  для~$I_2$     получаем
\begin{equation*}
I_2  \leq \fr{x^2}{2\sqrt{2p}} \, \fr{1}{\sqrt{2\pi}}\exp \left\{- \fr{x^2}{2}\right\}
 \leq \fr{1}{2 e \sqrt{p \pi}}\,.
\end{equation*}
Если $-\sqrt{2p} \le x \le 0$, то $x +  {x^2}/(2\sqrt{2p}) \le x / 2$. Следовательно,
\begin{multline*}
I_2 \leq \fr{x^2}{2\sqrt{2p}} \, \fr{1}{\sqrt{2\pi}}\,
\exp \left\{-\fr{1}{2}\, \left(x + \fr{x^2}{2\sqrt{2p}} \right )^2\right\}\leq{}\\
\leq \fr{x^2}{2\sqrt{2p}}\, \fr{1}{\sqrt{2\pi}}\,
\exp \left\{-\fr{1}{2} \left( \fr{x}{2}\right )^2\right\}
 \leq \fr{2}{ e\sqrt{p \pi}}\,.
\end{multline*}
Поскольку $1/(2 e \sqrt{p \pi})< 2 /(e\sqrt{p \pi})$,
 при выполнении условия  $x \ge - \sqrt{2p}$ имеем
\begin{equation}
\sup\limits_{-\sqrt{2p} \;\le\; x} |\P(T_1 \le x) - \IP(x)|
  \leq D(\lambda,p) + \fr{2}{e \sqrt{p \pi}}\,, 
  \label{eq12}
\end{equation}
где $D(\lambda,p) = \min\left(D_1(\lambda,p), D_2(\lambda,p)\right)$.

Известно (см., например, 7.1.13 в~[5]), что для любого     $x \ge 0$
\begin{equation}
 e^{x^2} \int\limits^{\infty}_{x} e^{- t^2} \,dt \leq \fr{1}{x + \sqrt{x^2 + 4/\pi}}\,. 
 \label{eq13}
\end{equation}
Заменой переменных из~(\ref{eq13})  для всех $x > 0$ получаем
$$
   1 -\IP(x) \leq  \sqrt{\fr{2}{\pi}}\, \fr{e^{-x^2/2}}{x + \sqrt{x^2 + 8/\pi}}\,.
$$
Тогда
\begin{multline}
\IP(-\sqrt{2p})  = 1 - \IP(\sqrt{2p})\leq{}\\
{} \leq \sqrt{\fr{2}{\pi}}\, \fr{e^{-p}}{\sqrt{2p} + \sqrt{2p + 8/\pi}}\,.
 \label{eq14}
\end{multline}
 Из~(\ref{eq12}) и~(\ref{eq14})  получаем следующую оценку
\begin{multline*}
\sup\limits_{x \in R} |\P(T_1 \leq x) - \IP(x)| \leq{}\\
{}\leq
\max\left(\vphantom{\fr{e^{-p}}{\sqrt{2}}}
\min_{\lambda} D(\lambda,p) + \fr{2}{e\sqrt{p \pi}},\,{}\right.\\
\left.\sqrt{\fr{2}{\pi}}\, \fr{e^{-p}}{\sqrt{2p}+\sqrt{2p + 8/\pi}}\right)\,.
\end{multline*}
Поскольку       
$$
\fr{2}{e\sqrt{p \pi}} > \sqrt{\fr{2}{\pi}}\, \fr{e^{-p}}{\sqrt{2p}+\sqrt{2p + 8/\pi}}
$$
и 
$$\min\limits_{\lambda} \,D(\lambda,p) >0\,,
$$ 
имеем
\begin{equation*}
\sup\limits_{x \in R} |\P(T_1 \le x) - \IP(x)| \leq \underset{\lambda}{\min}\,D\left(\lambda,p\right)
+ \fr{2}{e\sqrt{p \pi}}\,.
\end{equation*}

Таким образом (см.~(\ref{D_2})), доказана следующая теорема:

\smallskip

\noindent
\textbf{Теорема 3.} %{\label{th3.1}}
\textit{Справедлива оценка}
\begin{equation*}
\sup\limits_{x \in R}|\P(T_1 \leq x) - \IP(x) | <B(p)\,,
\end{equation*}
\textit{где}
\begin{equation}
B(p)= \underset{\lambda}{\min}\,D(\lambda,p) + \fr{2}{e\sqrt{p \pi}}\,, 
\label{eq.11}
\end{equation}
\textit{минимум берется по всем $\lambda \in (0,\sqrt{3}-1)$
и}
\begin{multline*}
D(\lambda,p) =\fr{2}{\pi} \left( \sqrt{\fr{\pi}{p}}\,\fr{1}{6}
 +\fr{ 2 (1-\lambda)}{p(2-2\lambda-\lambda^2)^2}+{}\right.\\
\left. {}+ \fr{(1+
\lambda^2)}{\lambda^2 p} \left(1+\lambda^2\right)^{-p/4}
  + \fr{1}{\lambda^2 p} \,\exp \left(-\fr{\lambda^2 p}{4}\right)
  \vphantom{\sqrt{\fr{\pi}{p}}}\right)\,.
\end{multline*}

\medskip

Из следствия~1 и оценки для~$T_1$   также получается

\smallskip

\noindent
\textbf{Следствие 2.}
\textit{Для статистики} $T_2 = \|\vx_i - \vx_j\| -\sqrt{2p}$
\begin{equation*}
\sup_{x \in R}|\P(T_2 \le x) - \IP(x) | <B(p)\,,
\end{equation*}
где $B(p)$ определено в~(\ref{eq.11}).
%\smallskip

\section{Два подхода к оценке $I_1$}

Будем использовать результат из~[3] для на\-хож\-де\-ния величины~$D(\lambda, p)$,
 которая будет выступать оценкой для~$I_1$.
В~[3] доказано, что
\begin{equation*}
\sup\limits_{x \in R} |F_p(x) - \IP_p(x)| \leq D_0(\lambda,p)\,,
\end{equation*}
где
\begin{align*}
F_p(x) &= \P(V_p \le x) =\P\left(\chi^{2}_p -p \le \sqrt{2p}x\right)\,; \\
\IP_p(x) &= \IP(x) + \fr{\sqrt{2}(1-x^2)}{3\sqrt{p}}\,\Ip(x)
\end{align*}
и    
\begin{multline*}
D_0(\lambda,p)= \fr{2}{\pi p}\left(
\vphantom{\fr{(1+\lambda^2)^{1-p/4}}{\lambda^2}}
\fr{4}{9}+\fr{2(1-\lambda)}{(2-2\lambda-\lambda^2)^2}+{}\right.\\
\left.{}+
\fr{(1+\lambda^2)^{1-p/4}}{\lambda^2}+
\fr{3+\lambda}{3\lambda^2}
e^{-\lambda^2(3-\lambda)p/(12+4\lambda)} \right)\,.
\end{multline*}
 Чтобы оценить~$I_1$,  надо найти равномерную оценку для  
$F_p(x) - \IP(x)$.
С этой целью будем использовать два подхода.
Первый подход очень прост. Используя  результат из~[3]
\begin{multline*}
\sup\limits_{x \in R} |F_p(x) - \IP(x)|={}\\
{}  = \sup\limits_{x \in R} \left|F_p(x) - \left(\IP_p(x) -
  \fr{\sqrt{2}(1-x^2)}{3\sqrt{p}}\Ip(x)\right)\right|\leq{}\\
{}  \leq \sup\limits_{x \in R} \left|F_p(x) -\IP_p(x) \right|+
  \fr{\sqrt{2}}{3\sqrt{p}}\,\sup\limits_{x \in R}\left|(1-x^2)\Ip(x)\right|\leq{}\\
{}\le D_0(\lambda, p) + \fr{1}{3\sqrt{p \pi}}\,,
\end{multline*}
получаем лемму:

\smallskip

\noindent
\textbf{Лемма 1.} %\begin{lemma}
\textit{Для всех}
 $\lambda \in \left(0,\sqrt{3}-1\right)$  и целых $p >1$
\begin{equation*}
\sup\limits_{x \in R} |F_p(x) - \IP(x)| \le D_1(\lambda,p)\,,
\end{equation*}
где
\begin{multline*}
D_1(\lambda, p)= \fr{2}{\pi p}\left (\fr{4}{9} +\fr{2(1-\lambda)}{(2-2\lambda-\lambda^2)^2}
+{}\right.{}\\
{}+\fr{(1+\lambda^2)^{1-p/4}}{\lambda^2}+{}\\
\left.{}+\fr{3+\lambda}{3\lambda^2}\,
e^{-\lambda^2(3-\lambda)p/(12+4\lambda)} \right )
+ \fr{1}{3\sqrt{p \pi}}\,.
\end{multline*}

\smallskip

Второй подход основан на модификации результата из~[3].
Пусть~$f_p(t)$ и~$g(t)$~--- характеристические функции распределений~$F_p(x)$ и~$\IP(x)$ соответственно.
Положим $p^*=p/2$. Имеем
\begin{equation*}
f_p(t) = \exp(-it \sqrt{p^*}) \left(1 -\fr{it}{\sqrt{p^*}}\right)^{-p^*}\!\,\!;\enskip
g(t) = e^{-t^2/2}\,.
\end{equation*}
Из формулы обращения для характеристических функций получаем
\begin{equation*}
|F_p(x) - \IP(x)| \le \fr{1}{2\pi} (I_1 + I_2 +I_3)\,,
\end{equation*}
где
\begin{align*}
I_1 &= \int\limits_{|t|\; < \;\lambda \sqrt{p^*}} \fr{1}{|t|}\left| f_p(t) - e^{-t^2/2}\right|\,dt\,;\\
I_2 &= \int\limits_{|t|\; \ge \;\lambda \sqrt{p^*}} \fr{|f_p(t)|}{|t|}\,dt\,;\\
I_3 &= \int\limits_{|t| \;\ge\; \lambda \sqrt{p^*}} \fr{e^{-t^2/2}}{|t|}\,dt\,.
\end{align*}
Используя ту часть комплексной функции $\tau (z) =$\linebreak $= \log (1- z )$,  для которой $\tau(0)=0$,
запишем:
\begin{multline*}
f_p(t) = \exp \left\{-it \sqrt{p^*} -p^* \log\left(1 - \fr{it}{\sqrt{p^*}}\right)\right \} ={}\\
{}=
\exp \left\{- \fr{t^2}{2} - \fr{it^3}{3\sqrt{p^*}} + p^* R_{p^*}(t) \right \}={}\\
{}       = \exp \left\{- \fr{t^2}{2} - \fr{it^3}{3\sqrt{p^*}} \right \}+ S_{p^*}(t)\,,
\end{multline*}
где
\begin{align*}
 R_{p^*}(t) &= - \log \left(1 - \fr{it}{\sqrt{p^*}}\right) -\fr{it}{\sqrt{p^*}}
 - \fr{(it)^2}{2p^*} -\fr{(it)^3}{3{p^*}^{3/2}}\,;\\
 S_{p^*}(t) &= \exp\left\{-\fr{t^2}{2} - \fr{it^3}{3\sqrt{p^*}}\right\}(\exp{p^* R_{p^*}(t)} -1)\,.
\end{align*}
В~[3] для  $|t| < \lambda \sqrt{p^*}$  показано, что
\begin{equation*}
\left\vert p^* R_{p^*}(t)\right\vert \leq \fr{|t|^4}{4p^*\left(1 -|t|/\sqrt{p^*}\right)}
\end{equation*}
и
\begin{multline*}
\left\vert S_{p^*}(t)\right\vert \leq e^{-t^2/2} \left\vert \exp \left\{p^* R_{p^*}(t)\right\} - 1\right\vert \leq{}\\
{}\leq e^{-t^2/2} \fr{|t|^4}{4p^* (1-\lambda)}\,
             \exp \left\{\fr{|t|^2 \lambda^2}{4p^*(1 -\lambda)} \right\}={}\\
{}             =  \fr{|t|^4}{4p^* (1-\lambda)} e^{-(-a|t|^2)/2}
\end{multline*}
с $a = 1 - \lambda^2 /(2(1-\lambda))$.
Следовательно,
\begin{equation*}
I_1 \le I_{11} +I_{12}\,,
\end{equation*}
где
\begin{align*}
I_{11} &= \int\limits_{|t| \;<\; \lambda \sqrt{p^*}} \fr{e^{-t^2/2}}{|t|}
\left| e^{-it^3/(3\sqrt{p^*})} - 1\right|\,dt\,;\\
I_{12} &= \int\limits_{|t|\; <\; \lambda \sqrt{p^*}} \fr{|S_{p^*}|}{|t|}\,dt\,.
\end{align*}
Используя неравенство $|e^{-iz} -1| \le |z|$, получаем
$$
I_{11}\leq \sqrt{\fr{\pi}{2}}\,\fr{2}{3\sqrt{p^*}}
$$
и
\begin{multline*}
I_{12} \le \fr{1}{4p^*(1-\lambda)}\int\limits_{|t|\; <\; \lambda \sqrt{p^*}} |t|^3 e^{-at^2/2} \,dt
 \leq{}\\
 {}\leq \fr{2}{4p^*(1-\lambda)} \int\limits^{\infty}_{0} t^3 e^{-at^2/2}\,dt ={}\\
{}=\fr{4}{4p^*(1-\lambda)a^2} =\fr{2}{p(1-\lambda)a^2}\,,
\end{multline*}
где $a = 1 - \lambda^2/(2-2\lambda)$.
Заметим, что $|f_p (t)| =$\linebreak
$= (1 + t^2/p^*)^{-p^*/2}$. Оценку для~$I_2$ непосредственно заимствуем из~[3]:
\begin{multline*}
I_2 \leq \int\limits_{|t|\; \ge\; \lambda \sqrt{p^*}} \fr{1}{|t|}\, \fr{1}{(1 +t^2/p^*)^{p^*/2}}\,dt \leq{}\\
{} \leq \fr{1+ \lambda^2}{\lambda^2} \int\limits_{u \;\ge\; \lambda^2} (1+u)^{-1-m/2}\,du ={} \\
                   =\fr{4\left(1+ \lambda^2\right)}{\lambda^2 p} \left(1+\lambda^2\right)^{-p/4}\,.
\end{multline*}
Рассуждая аналогично~[6] и используя интегрирование по частям, получаем
\begin{multline*}
I_3 \leq 2 \int\limits_{|t|\; \ge\; \lambda \sqrt{p^*}} \fr{1}{t} e^{-t^2/2}\,dt ={}\\
{}= \fr{2}{\lambda^2 p^*} \exp \left(-\fr{\lambda^2 p^*}{2}\right)
                    -2 \int\limits^{\infty}_{\lambda \sqrt{p^*}} t^{-3} e^{-t^2/2}\,dt \leq{}\\
                    {}\leq \fr{4}{\lambda^2 p} \exp \left(-\fr{\lambda^2 p}{4}\right)\,.
\end{multline*}
Объединяя оценки для~$I_1$, $I_2$ и~$I_3$, завершаем доказательство следующей леммы:

\smallskip

\noindent
\textbf{Лемма 2.} %\begin{lemma}
\textit{Для всех $\lambda \in (0,\sqrt{3}-1)$ и целых $p >1$}
\begin{equation*}
\sup\limits_{x \in R} |F_p(x) - \IP(x)| \le D_2(\lambda,p)\,,
\end{equation*}
\textit{где}
\begin{multline*}
D_2(\lambda, p) =\fr{2}{\pi} \left( \sqrt{\fr{\pi}{p}}\,\fr{1}{6}
 +\fr{ 2 (1-\lambda)}{p(2-2\lambda-\lambda^2)^2}+ {}\right.\\
\left. {}+
 \fr{(1+
\lambda^2)}{\lambda^2 p} \left(1+\lambda^2\right)^{-p/4}
  + \fr{1}{\lambda^2 p} \exp \left(-\fr{\lambda^2 p}{4}\right)
  \vphantom{\sqrt{\fr{\pi}{p}}}
  \right)\,.
\end{multline*}

\smallskip

Покажем, что оценка леммы~2 точнее оценки леммы~1, т.\,е.\
\begin{equation}
D(\lambda,p) = \min\left(D_1(\lambda,p), D_2(\lambda,p)\right) = D_2(\lambda,p)\,.
 \label{D_2}
\end{equation}
Действительно, поскольку 
$$
\fr{3+\lambda}{3} > 1
$$ и 
$$
\fr{3-\lambda}{3+\lambda} < 1
$$ 
для положительных~$\lambda$, равенство~(\ref{D_2}) верно.

В табл.~1 для некоторых значений~$p$ даны величины~$\underset{\lambda}{\min}\, D(\lambda,p)$ 
и соответствующие значения~$\lambda$, при которых достигается минимум.

\bigskip

\begin{center} %tabl1
\noindent
\parbox{59mm}{{\tablename~1}\ \ \small{Минимумы $D(\lambda,p)$ при фиксированном~$p$}}
\end{center}
%\vspace*{2pt}

{\small
\begin{center}
\tabcolsep=16pt
\begin{tabular}{|c|c|c|}
\hline
$p$ & $\lambda$ &$\min\, D$\\
\hline
 10 & 0,551 & 0,460\hphantom{9}\\
 30  &  0,517 & 0,104\hphantom{9}\\
 50  &  0,483 & 0,0550\\
 100\hphantom{9} & 0,416 & 0,0276\\
 500\hphantom{9}  & 0,244 & 0,0094\\
\hline
\end{tabular}
\end{center}
}
%\vspace*{6pt}


%\bigskip
\addtocounter{table}{1}


{\small\frenchspacing
{%\baselineskip=10.8pt
\addcontentsline{toc}{section}{Литература}
\begin{thebibliography}{9}

\bibitem{1kav}
\Au{Hall~ P., Marron~J.\,S.,   Neeman~ A.} 
Geometric representation of high dimension,
low sample size data~// J.~Royal Statistical Soc. Series~B,  2005. Vol.~67. P.~427--444.

\bibitem{2kav}
\Au{Ulyanov~V.\,V., Wakaki~ H., Fujikoshi~Y.} 
Berry--Esseen bound for high dimensional asymptotic approximation of Wilks' lambda distribution~// 
Statist. Probab. Lett.,  2006. Vol.~76.  No.\,12. P.~1191--1200.

\bibitem{3kav}
\Au{Ulyanov~V.\,V., Christoph~G., Fujikoshi~Y.}  
On approximations
of transformed chi-squared distributions in statistical applications~//
 Siber.\ Math.\ J., 2006. Vol.~47. No.\,6. P.~1154--1166.

\bibitem{4kav}
\Au{Konishi S.} 
Asymptotic expansions for the distributions of functions of a correlation matrix~//
J.\ Multivariate Analysis, 1979. Vol.~9. No.\,2. P.~259--266.


\bibitem{5kav}
Справочник по специальным функциям~/ Под ред.\ М.~Абрамовица и И.~Стиган.~--- М.: Наука, 1979.

\label{end\stat}

\bibitem{6kav}
\Au{Dobric~V., Ghosh~B.\,K.} 
Some analogs of the Berry--Esseen bounds for first-order
Chebyshev--Edgeworth expansions~// Statist. Decisions, 1996. Vol.~14. No.\,4. P.~383--404.
 \end{thebibliography}
}
}

\end{multicols}   %+3

\def\stat{bening}


\def\tit{АСИМПТОТИЧЕСКОЕ
РАЗЛОЖЕНИЕ ДЛЯ МОЩНОСТИ КРИТЕРИЯ, ОСНОВАННОГО НА ВЫБОРОЧНОЙ
МЕДИАНЕ, В~СЛУЧАЕ РАСПРЕДЕЛЕНИЯ ЛАПЛАСА$^*$}
\def\titkol{Асимптотическое
разложение для мощности критерия, основанного на выборочной
медиане} %, в случае распределения Лапласа}

\def\autkol{В.\,Е.~Бенинг, А.\,В.~Сипина}
\def\aut{В.\,Е.~Бенинг$^1$, А.\,В.~Сипина$^2$}

\titel{\tit}{\aut}{\autkol}{\titkol}

{\renewcommand{\thefootnote}{\fnsymbol{footnote}}\footnotetext[1]
{Работа выполнена
при финансовой поддержке РФФИ, проекты 08-01-00567 и
08-07-00152.}}

\renewcommand{\thefootnote}{\arabic{footnote}}
\footnotetext[1]{Московский государственный университет им.\
М.\,В.~Ломоносова, факультет вычислительной математики и
кибернетики; Институт проблем информатики Российской академии наук, bening@yandex.ru}
\footnotetext[2]{Московский государственный университет им.\
М.\,В.~Ломоносова, факультет вычислительной математики и
кибернетики, anna@sipin.ru}



\Abst{В работе прямыми методами, использующими асимптотические разложения,
получена формула для предельного отклонения мощности критерия, 
основанного на выборочной медиане, от мощности наилучшего критерия в случае распределения Лапласа.}

\KW{выборочная медиана; асимптотичсекое разложение; функция мощности; распределение Лапласа}

      \vskip 18pt plus 9pt minus 6pt

      \thispagestyle{headings}

      \begin{multicols}{2}

      \label{st\stat}


\section{Введение}

Следуя работе~\cite{3ben}, рассмотрим задачу проверки гипотезы
\begin{equation*}
{\sf H}_0: \theta = 0     
%\label{e1.1b}
\end{equation*}
против последовательности сложных близких альтернатив вида
\begin{equation*}
{\sf H}_{n,1}: \theta = \fr{t}{\sqrt{n}}\,,\quad 0<t<C\,,\quad
 C > 0
% \label{e1.2b}
\end{equation*}
на основе выборки $(X_1, \ldots , X_n)$~--- независимых одинаково распределенных наблюдений, имеющих распределение Лапласа 
с плотностью
\begin{equation}
p(x, \theta) = \fr{1}{2}e^{-|x-\theta|}\,, \quad x,\:
\theta \in{\sf R}^1\,. 
\label{e1.3b}
\end{equation}
Распределение Лапласа широко применяется в прикладной статистике, например
в задачах вы\-де\-ле\-ния полезного сигнала на фоне помех~[2--4].
Естественность возникновения этого распределения обоснована в
работе~\cite{6ben}.

Для каждого фиксированного $t\in (0,C]$
обозначим через~$\beta_n^*(t)$ мощность наилучшего критерия размера
$\alpha\in (0,1)$. По лемме Неймана--Пирсона %\linebreak 
[6, с.~94]
такой критерий всегда существует и  основан на логарифме отношения правдоподобия
\begin{equation}
\Lambda_n(t) = 
\sum_{i=1}^{n}\left( \left|X_i\right|-\left|X_i-tn^{-1/2}\right|\right)\,.
 \label{e1.4b}
\end{equation}
В работах~\cite{3ben, 2ben} рассмотрен критерий, основанный на знаковой статистике,
и получена формула для предельного отклонения мощности данного
критерия от мощности наилучшего критерия, основанного на~$\Lambda_n(t)$.
Поскольку у плотности~$p(x,\theta)$ не существует производной по~$\theta$ в 
точке $\theta = 0$, то это семейство не является регулярным.
Это выражается в нарушении естественного порядка~$n^{-1}$ разности мощностей
этих критериев и приводит к порядку~$n^{-1/2}$.

В  работе рассматривается статистика
\begin{equation*}
T_n = \sqrt{2k}\, \zeta_n\,,\quad k=\left[\fr{n}{2}\right]\,, 
%\label{e1.5b}
\end{equation*}
где $\zeta_n$~--- выборочная медиана:
\begin{equation*}
\zeta_n= 
\begin{cases}
X_{(k+1)}\,, & n=2k+1\,; \\
\fr{X_{(k)}+X_{(k+1)}}{2}\,, &  n=2k\,.
\end{cases}
%\label{e1.6b}
\end{equation*}
Заметим, что в случае распределения Лапласа выборочная медиана
совпадает с оценкой максимального правдоподобия (см.~\cite{1ben}).

Обозначим через~$\beta_n(t)$ мощность критерия размера $\alpha\in (0,1)$,
основанного на статистике~$T_n$. В работе получено асимптотическое
разложение для~$\beta_n(t)$ и вычислен предел разности мощностей~$\beta_n^*(t)$ и~$\beta_n(t)$
$$
r(t)\equiv\lim_{n\to\infty}\sqrt n\left(\beta_n^*(t)-\beta_n(t)\right)
$$
критериев (см.~(\ref{e2.14b})),
основанных соответственно на статистиках~$\Lambda_n$ и~$T_n$.

В работе также приведено полное доказательство  (см.~\cite{5ben})
представления выборочной медианы в виде случайной суммы
независимых экспоненциально распределенных  случайных величин.


\section{Асимптотическое разложение для мощности критерия,
основанного на выборочной медиане}

В этом разделе будет построено  асимптотическое разложение  для мощности~$\beta_n(t)$.
Основой для его получения служит  работа~\cite{1ben} (см.\ теорему~2.1),
в которой получено разложение для функции распределения выборочной медианы.
Члены порядка~$n^{-1/2}$ в разложении для функции распределения выборочной медианы
без доказательства приведены  также в работе~\cite{9ben}.

\medskip
\noindent
\textbf{Теорема 1.} {\it Для мощности~$\beta_n(t)$ равномерно по
$t\in(0,C]$, $C>0$,
справедливо следующее асимптотическое разложение:
\begin{equation*}
\beta_n(t)=
\begin{cases}
\Phi(t-u_\alpha)-\fr{t(2u_\alpha-t)}{2\sqrt{n}}\,\varphi(u_\alpha-t)+{} \\
\hspace*{8mm}{}+o\left(n^{-1/2}\right)\,,  \quad t \le u_\alpha\,,\enskip  \alpha <\fr{1}{2}\,;\\
\Phi(t-u_\alpha)-\fr{2u_\alpha^2+t^2-2u_\alpha t}{2\sqrt{n}}\,\varphi(u_\alpha -t)+{}\\
\hspace*{8mm}{}+o\left(n^{-1/2}\right)\,, \quad t>u_\alpha\,, \enskip \alpha <\fr{1}{2}\,;\\
\Phi(t-u_\alpha)+\fr{t(2u_\alpha-t)}{2\sqrt{n}}\,\varphi(u_\alpha -t)+{}\\
\hspace*{22mm}{}+{} o\left(n^{-1/2}\right)\,, \quad 
\alpha \ge \fr{1}{2}\,,
\end{cases}\hspace*{-6pt}
%\label{e2.1b}
\end{equation*}
где  $\Phi(x)$  и~ $\varphi(x)$~---  функция распределения и
плотность стандартного нормального закона и $\Phi(u_\alpha)=1-\alpha$.}

\medskip

\noindent
Д\,о\,к\,а\,з\,а\,т\,е\,л\,ь\,с\,т\,в\,о\,.\
Для доказательства теоремы воспользуемся асимптотическим разложением
для функции распределения выборочной медианы в случае
распределения Лапласа из работы~\cite{1ben} (см.\ формулу~(1.3)):
\begin{multline}
\p_{n,\theta} \left( \sqrt{2k}(\zeta_n - \theta) < x \right) = 
\Phi(x)-\fr{x|x|}{2\sqrt{2k}}\,\varphi(x)+{}\\
{}+
\fr{x(18+10x^2-3x^4)}{48k}\,\varphi(x)+ o(n^{-1})\,.
\label{e2.2b}
\end{multline}
Подберем критическое значение~$d_n$, исходя из условия
\begin{equation*}
\p_{n,0}(T_n>d_n)=\alpha+ o(n^{-1})\,.
%\label{e2.3b}
\end{equation*}
Будем искать $d_n$ в виде
\begin{equation*}
d_n = u_\alpha +\fr{a}{\sqrt{2k}}+\fr{b}{2k}\,.
%\label{e2.4b}
\end{equation*}
Из формулы~(\ref{e2.2b}) следует, что

\noindent
\begin{multline}
\p_{n, 0} \left( T_n> d_n \right) = 1 -
\Phi(d_n)+\fr{d_n|d_n|}{2\sqrt{2k}}\varphi(d_n)-{}\\
{}-
\fr{d_n(18+10d_n^2-3d_n^4)}{48k}\,\varphi(d_n)+ o(n^{-1})\,.
\label{e2.5b}
\end{multline}
Чтобы раскрыть модуль в выражении~(\ref{e2.5b}),  рас\-смот\-рим два случая:
$\alpha<1/2$ и $\alpha \ge 1/2$.

Рассмотрим случай $\alpha < 1/2$. Это означает, что при достаточно
больших $n$ справедливо неравенство $d_n > 0$.
Подставляя выражение для~$d_n$ в формулу~(\ref{e2.5b}) и применяя следующие разложения:
\begin{multline*}
\Phi(d_n)=\Phi\left(u_\alpha+\fr{a}{\sqrt{2k}}+\fr{b}{2k}\right)=
\Phi(u_\alpha)+{}\\
{}+
\left(\fr{a}{\sqrt{2k}}+\fr{b}{2k}\right)\varphi(u_\alpha)-
\fr{u_\alpha a^2}{4k}\varphi(u_\alpha)+ o(n^{-1})\,;
\end{multline*}
\vspace*{-12pt}

\noindent
\begin{multline*}
\varphi(d_n)=\varphi\left(u_\alpha+\fr{a}{\sqrt{2k}}
+\fr{b}{2k}\right)= {}\\
{}=
\varphi(u_\alpha)-\left(\fr{a}{\sqrt{2k}}+\fr{b}{2k}\right)u_\alpha
\varphi(u_\alpha)+ o(n^{-1})\,,
\end{multline*}
получаем
\begin{multline*}
1-\Phi(u_\alpha)-\left(\fr{a}{\sqrt{2k}}+
\fr{b}{2k}\right)\varphi(u_\alpha)+\fr{u_\alpha a^2}{4k}\,\phi(u_\alpha)
+{}\\
{}+\fr{(u_\alpha+(a/\sqrt{2k})+b/(2k))^2}
{2\sqrt{2k}}\times{}\\
{}\times \left(\varphi(u_\alpha) - \fr{a}{\sqrt{2k}}\,u_\alpha
\varphi(u_\alpha)\right)-{}\\
{}-
\fr{u_\alpha(18+10u_\alpha^2-3u_\alpha^4)}{48k}\,\varphi(u_\alpha)=
\alpha + o(n^{-1})\,.
\end{multline*}
Приравнивая коэффициенты при~$1/\sqrt{2k}$ и~$1/(2k)$ к нулю,
находим выражения для~$a$ и~$b$:
\begin{gather*}
a=\fr{u_\alpha^2}{2}\,;
\\
b=-\fr{3}{4}\,u_\alpha+\fr{1}{12}\,u_\alpha^3\,;
\\
d_n = u_\alpha+\fr{u_\alpha^2}{2\sqrt{2k}}-\fr{3}{8k}\,
u_\alpha+\fr{1}{24k}\,u_\alpha^3\,.
\end{gather*}
Теперь для получения асимптотического разложения мощности критерия используем
разложение
\begin{multline*}
\p_{n,tn^{-1/2}}(T_n<x)= \Phi\left(x-t\sqrt{2k}n^{-1/2}\right) -{}\\
{}-
\fr{\left(x-t\sqrt{2k}n^{-1/2}\right)\left| x\:-\:t\sqrt{2k}\,n^{-1/2}\right|}{2\sqrt{2k}}\,
{}\times{}\\
{}\times\varphi(x-t\sqrt{2k}\,n^{-1/2})+ {}
\end{multline*}
\begin{multline*}
{}+
\fr{ x-t\sqrt{2k}\,n^{-1/2}}{48k}
\left(18+10(x-
t\sqrt{2k}\,n^{-1/2})^2-{}\right.\\
\left.{}-3(x-t\sqrt{2k}\,n^{-1/2})^4\right)\times{}
\\
{}\times\varphi\left(x-t\sqrt{2k}\,n^{-1/2}\right)+ o\left(n^{-1}\right)\,,
%\label{e2.6b}
\end{multline*}
которое  получается при подстановке $\theta=tn^{-1/2}$ в
формулу~(\ref{e2.2b}).

Имеем
\begin{multline*}
\beta_n(t)=\p_{n,tn^{-1/2}}\left(T_n>d_n\right) ={}\\
{}=
1-\Phi\left(d_n-t\right) +
\fr{\left(d_n-t\right)\left|d_n-t\right|}{2\sqrt{2k}}\,\varphi\left(d_n-t\right)-{}
\\\!
{}-\fr{d_n-t}{48k}\left(18+10\left(d_n-t\right)^2
-3(d_n-t)^4\right)\, \varphi\left(d_n-t\right)+{}\\
{}+ o\left(n^{-1}\right)\,.
%\label{e2.7b}
\end{multline*}
Аналогично предыдущему, рассмотрим  два случая: $t\le u_\alpha$ и
$t>u_\alpha$.

Пусть сначала $t \le u_\alpha$.
Используя разложения
\begin{multline*}
\Phi\left(d_n-t\right)={}\\
{}=\Phi\left(u_\alpha-t+
\fr{u_\alpha^2}{2\sqrt{2k}}-\fr{3}{8k}\,u_\alpha+
\fr{1}{24k}\,u_\alpha^3\right)={}\\
{}=\Phi\left(u_\alpha-t\right)+
\left(\fr{u_\alpha^2}{2\sqrt{2k}}-\fr{3}{8k}\,u_\alpha+
\fr{1}{24k}\,u_\alpha^3\right)\times{}\\
{}\times\varphi\left(u_\alpha-t\sqrt{2k}\,n^{-1/2}\right)-{}
\\
{}-
\fr{\left(u_\alpha-t\sqrt{2k}\,n^{-1/2}\right)\varphi\left(u_\alpha-
t\sqrt{2k}\,n^{-1/2}\right)u_\alpha^4}{16k}+{}\\
{}+ o\left(n^{-1}\right)\,; 
%\label{e2.8b}
\end{multline*}

\vspace*{-12pt}

\noindent
\begin{multline*}
\varphi\left(d_n-t\right)={}\\
{}= \varphi\left(u_\alpha-t+
\fr{u_\alpha^2}{2\sqrt{2k}}-\fr{3}{8k}\,u_\alpha+
\fr{1}{24k}\,u_\alpha^3\right)={}\\
{}=
\varphi\left(u_\alpha-t\right)-\left(u_\alpha-t\right)
\varphi\left(u_\alpha-t\right)\fr{u_\alpha^2}{2\sqrt{2k}}+{}\\
{}+
o\left(n^{-1/2}\right)\,,
%\label{e2.9}
\end{multline*}
получаем, что
\begin{multline*}
\beta_n(t)=1-\Phi\left(u_\alpha-t\right)-
\fr{u_\alpha^2}{2\sqrt{2k}}\,\varphi\left(u_\alpha-t\right)+{}\\
{}+\fr{u_\alpha^2}{2\sqrt{2k}}\,\varphi(u_\alpha-t)-
\fr{2u_\alpha t - t^2}{2\sqrt{2k}}\,\varphi(u_\alpha-t)+{}\\
{}+
o\left(n^{-1/2}\right)=
\Phi\left(t-u_\alpha\right)-\fr{t\left(2u_\alpha - t\right)}{2\sqrt{2k}}\,
\varphi\left(u_\alpha - t\right)+{}\\
{}+ o\left(n^{-1/2}\right)\,.
%\label{e2.10b}
\end{multline*}
Во втором случае при $t > u_\alpha$  выражение
для мощности приобретает вид:

\noindent
\begin{multline*}
\beta_n(t)=\Phi\left(t-u_\alpha\right)-{}\\
{}-
\fr{t\left(2u_\alpha^2+t^2 -2u_\alpha t\right)}{2\sqrt{n}}\,
\varphi\left(u_\alpha-t\right)+ o\left(n^{-1/2}\right)\,.
%\label{e2.11b}
\end{multline*}
При $\alpha \ge 1/2$  аналогичным образом имеем
\begin{multline*}
\beta_n(t)={}\\
{}=
 \Phi\left(t-u_\alpha\right)+
\fr{t\left(2u_\alpha - t\right)}{2\sqrt{n}}\,\varphi\left(u_\alpha - t\right)+
o\left(n^{-1/2}\right)\,.
%\label{e2.12b}
\end{multline*}
Из этих формул следует утверждение теоремы.~$\Box$

\medskip

В работе~\cite{2ben} было показано, что для мощ\-ности~$\beta_n^*(t)$ 
критерия размера $\alpha\in (0,1)$, осно\-ван\-но\-го на
логарифме отношения прав\-до\-подобия~$\Lambda_n(t)$~(\ref{e1.4b}),
справедливо  асимптотическое\linebreak разложение
\begin{equation*}
\beta_n^*(t)=\Phi(t-u_\alpha) - \fr{t^2}{6\sqrt{n}}\,
\varphi(t-u_\alpha)+ o(n^{-1/2})\,.
%\label{e2.13b}
\end{equation*}
Используя это разложение и теорему~1, получаем формулу
для предельного отклонения нормированной разности мощностей
рассматриваемых критериев:
\begin{multline}
r(t)= \lim_{n \to \infty}\sqrt{n}(\beta_n^*(t)-\beta_n(t))
={}\\
{}=
\begin{cases}
\left(t u_\alpha-\fr{2t^2}{3}\right)
\varphi(u_\alpha-t)\,,\\
\hspace*{30mm} t \le u_\alpha\,,\enskip \alpha < \fr{1}{2}\,; \\
\left(u_\alpha^2+\fr{t^2}{3}-u_\alpha t \right)
\varphi(u_\alpha - t)\,,\\
\hspace*{30mm}  t>u_\alpha\,,\enskip \alpha<\fr{1}{2}\,; \\
\left(\fr{t^2}{3}-t u_\alpha\right)\varphi(u_\alpha-t)\,, \quad\quad\ \  \alpha \ge \fr{1}{2}\,. 
\end{cases}
\label{e2.14b}
\end{multline}

\section{Представление выборочной медианы в~виде случайной суммы}

В этом разделе докажем лемму о представлении выборочной медианы
в случае распределения Лапласа в виде суммы случайного числа
независимых экспоненциально распределенных случайных величин.
Формулы для представления порядковых статистик в случае распределения
Лапласа в виде подобной суммы приведены в работе~[4, с.~63],
но без строгого доказательства.

\bigskip

\noindent
\textbf{Лемма 1.}
{\it В случае распределения Лапласа выборочную медиану
можно представить в следующем виде (здесь равенства по распределению):
\begin{align}
\zeta_{2k+1} &\stackrel{d}{=}\delta_{2k+1}
\sum\limits_{j=k+1}^{K_{2k+1}}{\fr{W_j}{j}}\,;
\label{e3.1b}\\
\zeta_{2k}&\stackrel{d}{=}\fr{W_1-W_2}{2k}\,\mathbf{1}(B_{2k+1}=k)+{}\notag\\[1pt]
&\!\!\!\!\!\!\!\!\!\!\!\!\!\!{}+
\left(\delta_{2k}\sum\limits_{j=k+1}^{K_{2k+1}}\fr{W_j}{j}+
\delta_{2k}\fr{W_k}{2k}\right)\mathbf{1}\left(B_{2k+1} \ne k\right)\,,
\label{e3.2b}
\end{align}
где
$$
\delta_n=\mathrm{sign}\left(B_n-k-\fr{1}{2}\right)\,,
$$
$W_j$~--- независимые экспоненциально (с параметром~1) распределенные
случайные величины; $B_n$~--- бернуллиевские случайные величины с параметрами
$p=1/2$ и~$n$, независимые от~$W_j$;
\begin{equation*}
K_n = \max\left(B_n, \bar{B_n}\right)\,,\quad
\bar{B_n}= n - B_n\,.
\end{equation*}
}

\smallskip

\noindent
Д\,о\,к\,а\,з\,а\,т\,е\,л\,ь\,с\,т\,в\,о\,.

Вначале докажем две вспомогательные формулы, справедливые для любого
действительного чис\-ла~$s$
и любых натуральных чисел~$a$ и~ $b$:
\begin{gather}
\prod\limits_{j=a}^{a+b}{\fr{1}{j+is}}=
\sum\limits_{j=0}^b \fr{(-1)^j}{(a+j+is)(b-j)!j!}\,;
\label{e3.3b}
\\[3pt]
\!\!\!\!\!\!\!\!\sum\limits_{l=0}^k\fr{k!}{l!} \prod\limits_{j=a}^{a+k-l}\fr{1}{j+is}=
\sum\limits_{l=0}^k \begin{pmatrix}
k\\ l\end{pmatrix}
\fr{(-1)^l 2^{k-l}}{a+l+is}\,.
\label{e3.4b}
\end{gather}
Формулу~(\ref{e3.3b}) докажем методом математической индукции.

При $b=1$ формула верна. Предполагая ее верной при $b\ge1$,
докажем что она  верна и  при~$b+1$:
\begin{multline*}
\prod\limits_{j=a}^{a+b+1}\fr{1}{j+is}=\fr{1}{a+b+1+is}\prod\limits_{j=a}^{a+b}
\fr{1}{j+is}={}\\[2pt]
{}=
\fr{1}{a+b+1+is}\left(\sum\limits_{l=0}^k 
\begin{pmatrix}
k\\ l
\end{pmatrix}
\fr{(-1)^l 2^{k-l}}
{a+l+is}\right)={}\\[2pt]
{}=
\sum\limits_{j=0}^{b}\fr{(-1)^j}{(b-j)!j!} \left(\fr{1}{(b+1-j)(a+j+is)}
- {}\right.\\[2pt]
\left.{}-\fr{1}{(b+1-j)(a+b+1+is)} \right)={}
\end{multline*}
\begin{multline*}
{}=
\sum\limits_{j=0}^{b}\fr{(-1)^j}{(a+j+is)(b+1-j)!j!}-{}\\
{}-
\fr{1}{a+b+1+is}\sum\limits_{j=0}^{b}\fr{(-1)^j}{(b-j+1)!j!}\,.
\end{multline*}
Заметим, что
\begin{multline*}
\!\!\sum\limits_{j=0}^b\fr{(-1)^j}{(b-j+1)!j!}=
\sum\limits_{j=0}^{b+1}\fr{(-1)^j}{(b-j+1)!j!}
-\fr{(-1)^{b+1}}{(b+1)!}={}\\
{}=
\fr{1}{(b+1)!}(1-1)^{b+1}-\fr{(-1)^{b+1}}{(b+1)!}=
-\fr{(-1)^{b+1}}{(b+1)!}\,.
\end{multline*}
И следовательно, формула~(\ref{e3.3b}) доказана.
Формула~(\ref{e3.4b}) следует  из доказанной формулы~(\ref{e3.3b}), по\-скольку
\begin{multline*}
\sum_{l=0}^k{\fr{k!}{l!}}\prod\limits_{j=a}^{a+k-l}\fr{1}{j+is}={}\\
{}=
\sum\limits_{l=0}^{k}\fr{k!}{l!}\sum\limits_{j=0}^{k-l}
\fr{(-1)^j}{(a+j+is)(k-l-j)! j!}={}\\
{}
=\sum\limits_{j=0}^{k}\fr{(-1)^j}{a+j+is}\sum\limits_{l=0}^{k-j}
\begin{pmatrix}
k\\ j
\end{pmatrix}
\begin{pmatrix}
k-j\\  l
\end{pmatrix}={}\\
{}=
\sum\limits_{j=0}^k\fr{(-1)^j}{a+j+is}
\begin{pmatrix}
k\\ j
\end{pmatrix}
2^{k-j}\,.
\end{multline*}
Теперь приступим к доказательству основного утверждения леммы.
Рассмотрим сначала случай $n=2k+1$.
Плотность $(k+1)$-й порядковой статистики, как известно,
выражается формулой (см.~\cite{4ben})
\begin{equation*}
p_{2k+1}(x) = (2k+1)
\begin{pmatrix}
2k\\  k\end{pmatrix}
f(x)(F(x)(1-F(x))^k\,,
%\label{3.5b}
\end{equation*}
где $f(x)$ и  $F(x)$~--- соответственно плотность и
функция распределения исходных случайных величин.

Найдем характеристическую функцию~$\phi_{2k+1}(s)$ выборочной
медианы~$\zeta_{2k+1}$:
\begin{multline*}
\phi_{2k+1}(s)=\e e^{is\zeta_{2k+1}}=
\int\limits_{-\infty}^{\infty}e^{isx}f(x)\,dx={}\\
{}=
(2k+1)
\begin{pmatrix}
2k\\  k\end{pmatrix}
2^{-(k+1)}\times{}\\
{}\times
\sum\limits_{j=0}^k (-1)^j 2^{-j}
\begin{pmatrix}
k\\ e j\end{pmatrix}
\fr{2(k+1+j)}{(k+1+j)^2+s^2}\,.
%\label{e3.6b}
\end{multline*}
Теперь найдем характеристическую функцию~$f_{2k+1}(s)$ случайной величины, определенной\linebreak\vspace*{-12pt}\pagebreak

\noindent
в правой части  формулы~(\ref{e3.1b}).
С учетом того, что
 характеристическая функция стандартной экспоненциальной
случайной величины равна $1/(1-is)$, имеем
\begin{multline*}
f_{2k+1}(s)={}\\
{}=
\sum\limits_{l=0}^{2k+1}\e \exp \left(is\delta_{2k+1}
\sum\limits_{j=k+1}^{K_{2k+1}}\fr{W_j}{j}\right)\mathbf{1}(B_{2k+1}=l)={}
\\
=2^{-(2k+1)}\left(\sum\limits_{l=0}^k \begin{pmatrix}
2k+1\\  l\end{pmatrix}
\prod\limits_{j=k+1}^{2k+1-l}\fr{j}{j+is}+{}\right.\\
\left.{}+
\sum\limits_{l=k+1}^{2k+1}
\begin{pmatrix}
2k+1\\ l\end{pmatrix}
\prod\limits_{j=k+1}^{l}\fr{j}{j-is}
\right)={}\\
{}
=2^{-(2k+1)}(2k+1)
\begin{pmatrix}
2k\\ k\end{pmatrix}
\sum\limits_{l=0}^k\fr{k!}{l!}
\left(\prod\limits_{j=k+1}^{2k+1-l}\fr{1}{j+is} +{}\right.\\
\left.{}+
\prod\limits_{j=k+1}^{2k+1-l}\fr{1}{j-is}\right)\,.
%\label{e3.7b}
\end{multline*}
Применяя формулу~(\ref{e3.4b}), получаем
\begin{multline*}
f_{2k+1}(s)=(2k+1)
\begin{pmatrix}
2k\\  k
\end{pmatrix}
2^{-(k+1)}\times{}\\
{}\times
\sum\limits_{j=0}^k(-1)^j 2^{-j}
\begin{pmatrix}
k\\  j\end{pmatrix}
\fr{2(k+1+j)}{(k+1+j)^2+s^2}\,.
%\label{e3.8b}
\end{multline*}
Значит, $f_{2k+1}(s)\equiv\phi_{2k+1}(s)$ и представление~(\ref{e3.1b}) доказано.
\medskip

Перейдем теперь к рассмотрению случая четного $n=2k$.
Совместная плотность двух порядковых статистик~$X_{(k)}$ и~$X_{(k+1)}$
определяется формулой (см.~\cite{4ben})
\begin{equation*}
p(x,y)=\fr{(2k)!}{((k-1)!)^2}\,(F(x)(1-F(y)))^{k-1}f(x)f(y)\,.
%\label{e3.9b}
\end{equation*}
Из этой формулы нетрудно получить, что плотность случайной величины
$$
\zeta_{2k}=\fr{X_{(k)}+X_{(k+1)}}{2}
$$
равна
\begin{multline*}
p_{2k}(x) = \fr{(2k)!}{2^k ((k-1)!)^2}\times{}\\
{}\times
\left(\sum_{j=0}^{k-2}\fr{(-1)^j
\begin{pmatrix}
k-1\\ j
\end{pmatrix}
2^{-j}}{k-1-j}
e^{-(k+1+j)|x|}\times{}\right.
\end{multline*}
\begin{multline}
\left.{}\times \left(1-e^{-(k-1-j)|x|}\right)- \right.{}\\
{}\left.
- \fr{(-1)^k}{2^{k-1}}|x|e^{-2k|x|} + \fr{1}{k2^k}e^{-2k|x|}
\vphantom{\fr{(-1)^j
\begin{pmatrix}
k-1\\ j
\end{pmatrix}
2^{-j}}{k-1-j}}\right)\,.
\label{e3.10b}
\end{multline}
Подробный вывод этой формулы приведен в работе~\cite{8ben}.
Исходя их формулы~(\ref{e3.10b}), найдем характеристическую функцию~$\phi_{2k}(s)$
выборочной медианы~$\zeta_{2k}$:
\begin{multline*}
\!\phi_{2k}(s)=
\fr{(2k)!}{2^k ((k-1)!)^2}
\left( \sum\limits_{j=0}^{k-2}
\fr{(-1)^j
\begin{pmatrix}
k-1\\ j
\end{pmatrix}
2^{-j}}{k-1-j}\times{}\right.
\\
\left.{}\times
\left(
\fr{2(k+1+j)}{(k+1+j)^2+s^2}  -
 \fr{4k}{4k^2+s^2} \right)-{}\right.\\
\left. {}- 
 \fr{(-1)^k}{2^{k-2}(4k^2+s^2)} + \fr{1}{2^{k-2}(4k^2+s^2)}
 \vphantom{\sum\limits_{j=0}^{k-2}
\fr{(-1)^j
\begin{pmatrix}
k-1\\ j
\end{pmatrix}
2^{-j}}{k-1-j}}
\right)\,.
%\label{e3.11b}
\end{multline*}
Найдем теперь характеристическую функ-\linebreak цию~$f_{2k}(s)$ случайной величины,
определенной
 в правой части формулы~(\ref{e3.2b}). Учитывая формулу~(\ref{e3.4b}), получим
\begin{multline*}
f_{2k}(s)=\sum\limits_{l=0}^{k-1}{\p(B_{2k}=l)
\fr{2k}{2k+is}\prod\limits_{j=k+1}^{2k-l}{\fr{j}{j+is}}}+{}\\
{}+
\sum\limits_{j=k+1}^{2k}{\p(B_{2k}=l)\fr{2k}{2k-is}\prod\limits_{j=k+1}^{2k-l}\fr{j}{j-is}}+{}\\
{}+
\p(B_{2k}=k)\fr{4k^2}{4k^2+s^2}={}\\
{}=
\fr{(2k)!}{2^k ((k-1)!)^2} \left( \fr{1}{2^{k-2}(4k^2+s^2)}
+{}\right.\\
\left.{}+2^{1-k}\sum\limits_{l=0}^{k-1}(-1)^l 2^{k-l-1}
\begin{pmatrix}
k-1\\ l\end{pmatrix}\times\right.{}\\
{}\times
\left( \fr{1}{(2k+is)(k+1+l-is)}+{}\right.\\
\left.\left.{}+ 
\fr{1}{(2k-is)(k+1+l-is)}\right) \right)\,.
\end{multline*}
Применяя при $l \ne k-1$ следующее соотношение:
\begin{multline*}
\fr{1}{(2k+is)(k+1+l+is)}={}\\
{}=
\fr{1}{k-1-l}\left( \fr{1}{k+1+l+is} - \fr{1}{2k+is}\right)\,,
\end{multline*}
получаем равенство

\noindent
\begin{multline*}
f_{2k}(s)=
\fr{(2k)!}{2^k ((k-1)!)^2}
\left( \sum\limits_{j=0}^{k-2}
\fr{(-1)^j 
\begin{pmatrix}
k-1\\ j
\end{pmatrix}
2^{-j}}{k-1-j}\times{}\right.\\
\left.{}\times
\left(
\fr{2\left(k+1+j\right)}{(k+1+j)^2+s^2} 
-  \fr{4k}{4k^2+s^2} \right)
-{}\right. \\
\left.{}- \fr{\left(-1\right)^k}{2^{k-2}\left(4k^2+s^2\right)} + \fr{1}{2^{k-2}(4k^2+s^2)}
\vphantom{\sum_{j=0}^{k-2}
\fr{(-1)^j 
\begin{pmatrix}
k-1\\ j
\end{pmatrix}
2^{-j}}{k-1-j}}
\right)\,.
%\label{e3.12b}
\end{multline*}
Таким образом,  $\phi_{2k}(s)\equiv f_{2k}(s)$ и утверждение~(\ref{e3.2b})
доказано.~$\Box$

{\small\frenchspacing
{%\baselineskip=10.8pt
\addcontentsline{toc}{section}{Литература}
\begin{thebibliography}{9}

\bibitem{3ben} %1
\Au{Королев Р.\,А., Тестова  А.\,В., Бенинг~В.\,Е.} 
О мощ\-ности асимптотически оптимального критерия в случае 
распределения Лапласа~// Вестник Тверского Государственного Университета, 
серия Прикладная математика, 2008. Вып.~8. №\,4(64). С.~5--23.

\bibitem{9ben} %2
\Au{Takeuchi K.} 
Asymptotic theory of statistical estimation.~---  Tokyo, 1974. (In Japanese.)

\bibitem{1ben} %3
\Au{Бурнашев М.\,В.} 
Асимптотические разложения для 
медианной оценки параметра~// Теор. вероятн. и ее
прим., 1996. Т.~41. Вып.~4. С.~738--753.

\bibitem{5ben}  %4
\Au{Kotz S., Kozubowski~T.\,J., Podgorski~K.}
The Laplace distribution and generalizations: 
A revisit with applications to communications, economics, engineering, 
and finance.~--- Birkhauser, 2001.  P.~349.

\bibitem{6ben}  %5
\Au{Бенинг В.\,Е., Королев В.\,Ю.}
Некоторые статистические  задачи, связанные с распределением Лапласа~// 
Информатика и её применения, 2008. Т.~2.  Вып.~2. С.~19--34.

\bibitem{7ben}  %6
\Au{Леман Э.} 
Проверка статистических гипотез.~--- М.: Наука, 1964. 498~с.

\bibitem{2ben} %7
\Au{Королев Р.\,А., Бенинг В.\,Е.}
Асимптотические 
разложения для мощностей критериев в случае распределения Лапласа~//
Вестник Тверского Государственного Университета, серия 
Прикладная математика, 2008. Вып.~3(10). №\,26(86). С.~97--107.

\bibitem{4ben} %8
\Au{David H.\,A., Nagaraja H.\,N.}
Order Statistics.  3rd ed.~--- New Jersey: Wiley, 2003.  P.~458.

\label{end\stat}

\bibitem{8ben} %9
\Au{Asrabadi B.\,R.} 
The exact confidence interval for 
the scale parameter and the MVUE of the Laplace distribution~// 
Communications in statistics. Theory and methods, 1985. Vol.~14. No.\,3. 
P.~713--733.

 \end{thebibliography}
}
}
\end{multicols}  %4

\def\stat{ilushin}


\def\tit{ОРГАНИЗАЦИЯ УПРАВЛЯЕМОГО ДОСТУПА ПОЛЬЗОВАТЕЛЕЙ 
К~РАЗНОРОДНЫМ ВЕДОМСТВЕННЫМ ИНФОРМАЦИОННЫМ 
РЕСУРСАМ}
\def\titkol{Организация управляемого доступа пользователей 
к~разнородным ведомственным информационным 
ресурсам} 

\def\autkol{Г.\,Я.~Илюшин, И.\,А.~Соколов}
\def\aut{Г.\,Я.~Илюшин$^1$, И.\,А.~Соколов$^2$}

\titel{\tit}{\aut}{\autkol}{\titkol}

%{\renewcommand{\thefootnote}{\fnsymbol{footnote}}\footnotetext[1]
%{Работа выполнена
%при финансовой поддержке РФФИ, проекты 08-01-00567 и
%08-07-00152.}}

\renewcommand{\thefootnote}{\arabic{footnote}}
\footnotetext[1]{Институт проблем информатики Российской академии наук, ilushin@ipiran.ru}
\footnotetext[2]{Институт проблем информатики Российской академии наук, isokolov@ipiran.ru}


\Abst{Статья посвящена вопросам реализации интероперабельности приложений и управления доступом 
пользователей к хранилищам информации в условиях модернизации ин\-фра\-струк\-ту\-ры 
информационных технологий (ИТ) крупных ведомств. Для 
обеспечения направленной эволюции унаследованных автоматизированных
информационных сис\-тем (АИС) и хранилищ данных без остановки 
функционирования существующих систем предложена технология создания программно-технической 
инфраструктуры промежуточного слоя. Промежуточный слой, с использованием технологий 
сер\-вис\-но-ориен\-ти\-ро\-ван\-ной архитектуры и веб-сер\-ви\-сов, 
решает как задачи обеспечения интероперабельности разнородных приложений, так и задачи реализации 
централизованной модели управления доступом пользователей к разнородным хранилищам данных на основе 
формализованных ролей. Помимо задач интероперабельности, инфраструктура промежуточного слоя позволяет 
решить целый спектр задач обеспечения необходимого уровня информационной безопасности.}


\KW{интероперабельность; управление доступом; промежуточный слой; веб-сервисы; унаследованные системы; 
метаданные}

      \vskip 18pt plus 9pt minus 6pt

      \thispagestyle{headings}

      \begin{multicols}{2}

      \label{st\stat}

\section{Введение}

      Усиление роли и расширение финансовых возможностей государства в целом и 
федеральных органов власти в частности, существенно из\-ме\-нив\-шиеся требования 
законодательства, накопление позитивного опыта использования современных технологий 
интеграции данных в крупных пилотных проектах федерального и регионального уровня 
заставили по-новому взглянуть на ИТ-ин\-фра\-струк\-ту\-ру ключевых федеральных ведомств и 
поставить масштабные задачи коренного изменения положения дел в информатизации органов 
государственной власти. Большинство федеральных ведомств прошли непростой путь 
создания автоматизированных систем, располагают отлаженными и достаточно современными 
программно-техническими системами, имеют собственную разветвленную и довольно 
инерционную управленческую структуру. Существующие ведомственные системы 
создавались и развивались в течение десятилетий, от разрозненных функционально и/или 
регионально ориентированных подсистем к интегрированным системам. При этом больше 
внимания уделялось функциональной интеграции систем, чем информационной.
      
       Необходимость максимально возможного сохранения вложенных инвестиций, 
невозможность остановки действующих систем на длительный период, сложности внедрения 
из-за необходимости массового переобучения персонала и необходимость существенной 
модернизации информационного взаимодействия департаментов ведомства создают 
значительные трудности как в принятии\linebreak правильных управленческих решений руководством 
ведомства, так и в технической реализации крупных интеграционных проектов. И все же 
именно коренное изменение ранее существовавших подходов к созданию, модернизации и 
развитию ведомственных систем диктуется целым комплексом объективных причин. 
      \begin{enumerate}[1.]
\item Основа существующих технических решений построения информационных систем 
ведомств создавалась в начальный период централизации государственной власти, когда 
финансирование ИТ-про\-ек\-тов осуществлялось преимущественно из региональных бюджетов.
      
      В регионах-донорах создавалась собственная, не совместимая с другими регионами 
      ИТ-ин\-фра\-струк\-ту\-ра, проводилась собственная техническая политика, закупались и 
внедрялись технические и программные платформы (в том числе базовые операционные сис\-те\-мы, 
сис\-те\-мы управления базами данных (СУБД) и 
инструментальные средства), самостоятельно выбирались методы, стандарты и технологии 
интеграции. Разрабатываемые и на территориальном уровне внед\-ря\-емые собственные типовые 
решения в интересах подразделений ведомств, расположенных в регионе, интегрировались на 
основе инструментария мощных и дорогостоящих СУБД, поскольку региональные власти не 
особенно заботились о лицензионной чис\-то\-те, а следовательно, и стоимости используемого 
программного обеспечения (ПО). Координация множества региональных разработок в интересах ведомства на федеральном 
уровне (руководства ведомства) была затруднена, а зачастую невозможна.
      
      В~дотационных регионах отсутствие до\-ста\-точных средств на внедрение и обучение, 
не\-раз\-витость ИТ-ин\-фра\-струк\-ту\-ры и недостаток спе\-циалистов, владеющих современными 
компьютерными технологиями, привели к консервации технических решений 90-х годов 
прош\-ло\-го века, основанных чаще всего еще на\linebreak технологиях MS DOS.
      
      Значительный разрыв между регионами-до\-но\-ра\-ми и дотационными регионами (в 
      5--10~раз) наблюдается в настоящее время как в показателях развитости 
      ИТ-ин\-фра\-струк\-ту\-ры\linebreak
       (оснащенность учреждений и домохозяйств фиксированной и 
мобильной связью, компьютерами, высокоскоростным Интернетом, средствами 
автоматизации деятельности), так и в показателях уровня накопленного <<человеческого 
капитала>> (знаний и опыта населения в использовании современных ИТ)~[1].
\item За последние несколько лет произошла существенная реорганизация системы 
государственного управления, в то же время существенно изменились требования к качеству 
информации и одновременно к ее защите от несанкционированного доступа (НСД), появились 
законодательные ограничения по защите персональных данных, ужесточились требования к 
лицензированию ПО.
      
      По мере формирования новой управленче-\linebreak ской структуры, расширения возможностей 
централизованного финансирования, каждое ведомст\-во начинало создавать свою 
      ИТ-ин\-фраструкту\-ру и вертикаль информационных ресурсов (ИР), выбирая собственные 
технические и прог\-рам\-мные платформы реализации прикладных задач и мало заботясь о 
со\-вмес\-ти\-мости с техническими решениями других ведомств. В~то же время ужесточились 
требования федеральной власти к самим ведомствам по ответственности должностных лиц за 
актуальность и достоверность информации в их базах данных. 
{ %\looseness=1

}
      
      По мере внедрения новых информационных систем в связи с вводом в действие 
императивных норм и регламентов происходила постепенная техническая изоляция 
информационных систем и баз данных разных ведомств, внедрялись разные технические 
решения по организации разграничения доступа и защите данных от НСД. Интеграционные 
технические решения, созданные в крупных регионах, начали постепенно изживать себя, 
вступая в противоречие с регламентами и технической политикой отдельных ведомств.
      
      В рамках каждого ведомства начали формироваться свои требования по разграничению\linebreak 
доступа сотрудников к информации в соот\-ветствии с их должностными функциями. 
А~поскольку практически каждое ведомство нуж\-далось в информации из других ведомств,
налаживался обмен массивами информации\linebreak  между ними. На практике это привело к 
рас\-со\-гла\-со\-ва\-нию данных об одних и тех же\linebreak информационных объектах из-за временного 
разрыва в об\-нов\-ле\-нии информации, неправильной фильтрации данных при обмене массивами, 
из-за недостаточно продуманных и некачественно реализованных регламентов\linebreak 
взаимодействия, отсутствия надежных межведомственных информационных контуров 
обратной связи. Наиболее трудноразрешимые проблемы (как для сотрудников ведомств, так и 
для граждан) возникают в случаях, когда ошибки, возникшие в недрах одного ведомства, 
выявляются служащими другого ведомства в связи с предоставлением граж\-да\-на\-ми или 
организациями первичных документов. Стало понятно, что межведомственный обмен 
массивами данных в силу своих органических недостатков не может служить основой 
преодоления противоречий в информационных хранилищах ведомств и необходимо внедрение 
более сложных и гибких технологий обеспечения целостности баз данных в различных 
органах государственной власти.
\item Новые законодательно устанавливаемые требования к повышению качества 
обслуживания населения (идеология <<одного окна>>), отмена\linebreak
множества ведомственных 
инструкций по\linebreak различного рода <<откреплениям>> и <<прикреп\-ле\-ниям>> граждан в связи со 
сменой места жительства или пребывания, настоятельные требования высших 
государственных органов\linebreak
 власти по предоставлению гражданам различных сервисов 
<<электронного правительства>> привели к серьезным затруднениям, а в некоторых случаях 
даже к невозможности модер\-ни\-зации технических решений ведомственных сис\-тем без 
изменения основополагающих принципов их построения. 
\end{enumerate}

      Одновременно выявилась сложность модернизации информационных систем в 
условиях разрыва информационных связей между системами автоматизации различных 
ведомств и технологическая их несовместимость при острой не\-об\-хо\-ди\-мости совместного 
использования данных. В~условиях разнородности технических и программных платформ, 
несогласованной политики и не\-со\-вмес\-ти\-мости технических решений по разграничению 
доступа к данным на внутриведомственном, а часто и на межрегиональном уровне реализация 
межведомственного доступа к разнородным базам данных ведомств в реальном масштабе 
времени оказалась практически неосуществимой задачей. В~то же время интеграционные 
решения отдельных регионов пришли в полное противоречие с новыми требованиями 
законодательства.
      
      В настоящее время в большинстве федеральных ведомств начата работа по серьезной 
модернизации своих информационных систем и хранилищ данных. Эта работа ведется с 
учетом перераспределения функций и полномочий ведомств, изменения нормативной базы, 
необходимости межведомственного и международного взаимодействия, а также требований 
по взаимодействию каждого ведомства с гражданами и организациями. Наряду с работой по 
модернизации ведомственных сетей передачи данных, она ведется каждым ведомством в трех 
основных направлениях:
      \begin{enumerate}[(1)]
\item интеграция данных на региональном и федеральном уровне;
\item разработка и внедрение технических и программных средств защиты информации 
от НСД и систем управления доступом пользователей к информационным хранилищам;
\item разработка и внедрение унифицированных прикладных систем для всех уровней 
объектов управления с возможностью программного доступа к информационным 
хранилищам как своего, так и других ведомств (решение задач интероперабельности 
приложений и обеспечения регламентов информационного взаимодействия). 
\end{enumerate}

      Вопросам реализации задач интероперабель\-ности приложений и управления доступом 
пользователей к хранилищам информации посвящена данная статья.
      
       Современные теоретические подходы, практические методы интеграции 
неоднородных ИР и даже стандарты интероперабельности в 
настоящее время широко известны~[2--4]. Все они так или иначе опираются на идеологию 
создания промежуточного слоя прикладного ПО, которое берет на себя функции 
нивелирования технических решений разных автоматизированных систем путем реализации 
новых и поддержки некоторых старых API-ин\-тер\-фей\-сов (Application Programming Interface~--- интерфейс прикладного 
программирования).
      
      Термин <<промежуточный слой>>, исполь\-зу\-емый в настоящей статье, функционально 
более насыщен, чем традиционно используемые виды\linebreak
 промежуточного слоя, как например 
CORBA
 (Common Object Request Broker Architecture~--- архитектура посредника объектного
запроса), Web Services (веб-сер\-ви\-сы), SOA (Service-Oriented Architecture~--- 
сер\-вис\-но-ориен\-ти\-ро\-ван\-ная архитектура), 
Message-oriented middleware (промежуточный слой ПО, ориентированный на сообщение) 
и~пр. Так, наряду со средствами 
обеспечения технической ин\-тер\-опе\-ра\-бель\-ности, в предлагаемом промежуточном слое 
реализуются средства интегрированного доступа к разнородным информационным системам и 
хранилищам данных. При этом заметим, что веб-сер\-вис\-ный вариант промежуточного слоя 
является естественным выбором при нынешнем уровне технологического  
развития и что применение этой архитектуры промежуточного слоя решает проблему  
технической интероперабельности приложений, позволяя ведомству осуществлять 
планомерную и постепенную модернизацию унаследованных приложений наряду с 
разработкой новых.
      
      Выделим две значимые цели ведомства, оправдывающие создание промежуточного 
слоя:
      \begin{enumerate}[(1)]
\item обеспечение интероперабельности информационных систем на уровне приложений 
промежуточного слоя при совместной обработке данных о субъектах, событиях и объектах из\linebreak 
множества информационных хранилищ, включая комплексные и каскадные запросы сразу к 
нескольким разнородным информационным сис\-те\-мам одного или нескольких ведомств;
      \item перенаправление всех запросов сотрудников ведомства (и других ведомств) в 
единую территориально распределенную инфраструктуру аутентификации и управления 
доступом пользователей взамен ранее используемых прямых обращений к АИС и 
ИР, реализация необходимых контрольных (в том числе фискальных) 
функций.
      \end{enumerate}
      \pagebreak
      
      \end{multicols}

\begin{figure} %fig1
\vspace*{1pt}
\begin{center}
\mbox{%
\epsfxsize=95.847mm
\epsfbox{ily-1.eps}
}
\end{center}
\vspace*{-9pt}
\Caption{Типовая последовательность действий по обработке запроса
\label{f1il}}
\vspace*{3pt}
\end{figure}

\begin{multicols}{2}
      
      Основные функции ПО промежуточного слоя:
      \begin{itemize}
\item поддержка специально для этой цели разработанных детальных API-ин\-тер\-фей\-сов 
взаимодействия приложений;\\[-14pt]
\item  осуществление в реальном масштабе времени необходимых преобразований 
семантически однотипных данных, по-разному пред\-став\-лен\-ных в структурах хранения 
интегрируемых систем;\\[-14pt]
\item  централизованное управление доступом пользователей к приложениям, контроль 
взаимодействия приложений между собой с целью обеспечения надежной защиты 
ИР от~НСД.
\end{itemize}


      На рис.~\ref{f1il} показана типовая последовательность действий, выполняемых 
компонентами промежуточного слоя при обработке заявок приложений на поиск информации 
в ведомственных ИР. 
      
      Все крупнейшие производители технического и программного обеспечения 
декларируют полноценную реализацию современных стандартов интеграции, но каждый~--- 
по-своему. Поэтому декларируемая совместимость промышленного ПО на уровне стандартов на 
практике вовсе не означает\linebreak технической совместимости ПО разных производителей. Во 
всяком случае, в процессе разработки программных средств обеспечения 
ин\-тер\-опе\-ра\-бельности систем, построенных на разных\linebreak
программных платформах, требуется не 
только знание многих нюансов реализации стандартов каж\-дым производителем, но порой и 
знание недокументированных особенностей всех программных платформ. 

Создание команды 
разработчиков, обладающих столь обширными знаниями, трудноосуществимо. Вынужденное 
применение метода проб и ошибок приводит к существенным временным задержкам в 
реализации интеграционных проектов и непредусмотренным дополнительным финансовым 
издержкам.
      
       Крупные производители промышленного ПО (операционных систем, СУБД, 
инструментальных средств) оказывают существенную поддержку разработчикам 
ведомственных федеральных проектов во имя грядущей финансовой выгоды, устраняют 
выявленные ошибки и дорабатывают свое ПО с учетом опыта его внедрения в крупных 
ведомствах. В~случае же использования в ведомственных проектах свободно 
распространяемого ПО получить информацию о технических нюансах этого ПО 
затруднительно, а оказать влияние на скорейшее устранение выявленных ошибок невозможно.
      
      Из сказанного можно сделать следующие вы\-воды:
      \begin{enumerate}[(1)]
\item модернизация систем информационного обеспечения ведомства должна осуществляться 
с учетом необходимости взаимодействия с аналогичными системами других ведомств. 
Обеспечение интероперабельности разнородных систем целесообразно с использованием 
программно-технической инфраструктуры промежуточного слоя;
\item правильный выбор программных платформ, качество проектирования 
архитектуры и уровень технических решений программного обеспечения 
промежуточного слоя имеет далеко идущие последствия, поскольку недостатки 
технического решения выявятся только на\linebreak этапе последующей модернизации 
существующих и создания новых информационных сис\-тем и баз данных ведомства. 
\end{enumerate}
      
      Институт проблем информатики Российской академии наук 
      накопил значительный опыт практи\-ческой разработки и внедрения средств 
обеспечения интероперабельности разнородных информационных систем на основе 
инфраструктуры промежуточного слоя. Этот опыт может быть с успехом использован как для 
разработки систем информационной интеграции разнородных (в том числе унаследованных) 
ИР крупных ведомств, так и для обеспечения межведомственного 
информационного взаимодействия.
      
      В последующих разделах рассмотрены следующие практические аспекты 
создания и по\-стро\-ения инфраструктуры промежуточного слоя крупного ведомства:
      \begin{itemize}
\item задачи и этапы построения инфраструктуры интероперабельности, назначение и 
состав метаданных промежуточного слоя, общая схема фильтрации на уровне запросов и 
результатов их выполнения при обращении пользователей к разнородным 
ИР (разд.~2);
\item основные типы базовых (технологических) сервисов промежуточного слоя (разд.~3);
\item методы организации централизованно управ\-ля\-емо\-го доступа пользователей к ИР на 
основе ролей (разд.~4).
\end{itemize}

\section{Выбор инфраструктуры промежуточного слоя}

\subsection{Назначение промежуточного слоя}

      При выборе принципов построения (модернизации) крупной 
      тер\-ри\-то\-ри\-аль\-но рас\-пре\-де\-лен\-ной ведомственной системы особенно важным является решение по степени 
централизации хранения и обработки данных. Наиболее эффективной с точки зрения простоты 
интеграции, эко\-но\-мич\-ности обслуживания и модернизации представляется структура на 
основе центров обработки данных (ЦОД). Однако принятию решения о полной реконструкции 
информационной инфраструктуры и объединении множества ИР 
крупного ведомства в единое хранилище данных препятствует множество объективных 
факторов:
      \begin{itemize}
\item сложившаяся в ведомстве распределенная структура сотен региональных отраслевых 
(интегрированных) и специализированных баз данных, а также тысяч поддерживающих их 
АИС, десятков тысяч 
специализированных автоматизированных рабочих мест (АРМ);
\item накопленный в течение многих лет опыт и нормативная база формирования и 
использования ИР ведомства (включая специализированные базы данных);
\item обоснованное нежелание департаментов (ведомств) передавать функции ведения 
своих ИР от собственных подразделений специально созданному 
подразделению без передачи ответственности за их качество;
\item несовершенная и неоднородная по своему качеству сеть передачи данных, не 
обес\-пе\-чи\-ва\-ющая высокоскоростной доступ всех пользователей к удаленным 
ИР со своих рабочих мест.
\end{itemize}

      Одномоментное существенное изменение сложившейся ИТ-ин\-фра\-струк\-ту\-ры ведомства 
на практике чаще всего оказывается неприемлемым по целому ряду причин:
      \begin{itemize}
\item значительный риск потери на длительный переходный период необходимого уровня 
качества существующих разнородных ИР и их до\-ступ\-ности;
\item замораживание процессов модернизации функций существующих ИР на весь 
переходный период, что может препятствовать вводу в действие новых регламентов и 
сервисов электронного правительства;
\item необходимость крупных инвестиций в на\-уч\-но-ис\-сле\-до\-ва\-тель\-ские и опыт\-но-кон\-ст\-рук\-тор\-ские
работы, апробирование и внедрение новых сис\-тем, переобучение персонала;
\item необходимость пересмотра нормативной и правовой базы, сопротивление 
радикальным изменениям со стороны сотрудников ведомства.
\end{itemize}
      
      По этим и ряду других причин революционный метод модернизации (одномоментная 
перестройка ИТ-ин\-фра\-струк\-ту\-ры, создание абсолютно новых АИС взамен существующих, 
полное переобучение персонала) неприемлем для крупных ведомств.
      
      Вместо революционных методов модернизации, сулящих в отдаленном будущем 
значительные экономические выгоды и более высокую степень адап\-тив\-ности, руководство 
крупных ведомств осознанно делает выбор в пользу более дорогостоящих мето\-дов 
направленной эволюции ИТ-ин\-фра\-струк\-ту\-ры и АИС при максимально возможном 
использовании вложенных инвестиций. Достоинством такого подхода является возможность 
обеспечения непрерывности функционирования множества действующих АИС, постепенная 
их модернизация, разработка новых АИС, опирающихся уже на новую инфраструктуру и 
      API-ин\-тер\-фей\-сы, постепенная реструктуризация баз данных, поэтапное переобучение 
персонала. Внедрение новых систем может осуществляться поэтапно, с корректировкой 
выявленных технических и организационных проблем. Например, в период внедрения новой 
информационной инфраструктуры в одних регионах другие регионы могут пользоваться 
старыми отработанными решениями. Такую направленную эволюцию ИТ-ин\-фра\-струк\-ту\-ры и 
АИС предлагается реализовать на основе концепции промежуточного слоя.
      
      Основная идея создания промежуточного слоя состоит в том, что прямое 
взаимодействие АРМ (клиентских приложений) с существующими АИС (серверными 
приложениями) постепенно исключается. На плановой основе осуществляется модернизация 
унаследованных и разработка новых АИС, которые взаимодействуют с другими 
приложениями только через API-ин\-тер\-фей\-сы промежуточного слоя. Взаимодействие же 
приложений с пользователями разрешается только с использованием новой централизованной 
инфраструктуры идентификации и управления правами пользователей. Ядром этой 
инфраструктуры являются сис\-те\-ма удостоверяющих центров (СУЦ), Active Directory и база данных 
пользователей, содержащая для каждого пользователя все его детализированные права 
доступа к базам данных, управляемых АИС, модернизированными в соответствии с новыми 
требованиями. На рис.~\ref{f2il} схематично представлено изменение информационных 
потоков пользователей в связи с созданием промежуточного слоя и постепенной 
модернизацией существующих клиентских и серверных приложений.
      


      До создания промежуточного слоя (пунктирные линии на рис.~\ref{f2il}) каждый 
сотрудник ведомства, нуждающийся в доступе к нескольким ИР (базам данных 
территориального, регионального и федерального уровня), вынужден был работать с 
несколькими АИС, каждая из которых имела свою программную платформу и свой 
пользовательский интерфейс. Администратор каждой АИС прописывал информацию о 
пользователе и его правах доступа к информационным объектам ИР в соответствующей 
административной базе данных. В~итоге пользователь получал множество Login (по числу ИР) 
и множество интерактивных интерфейсов взаимодействия с ИР (по числу типовых АИС).
      
      В новых условиях сотрудник ведомства должен получить техническое средство 
персональной идентификации и электронное удостоверение, содержащее цифровой 
сертификат. Права доступа всех\linebreak пользователей ко всем информационным объектам всех ИР, в 
соответствии со служебными обязанностями пользователей, сосредоточиваются в единой\linebreak
 базе 
данных прав доступа пользователей (ПДП). В~последующем, по мере разработки и внедрения новых 
типовых АИС, исчезнет необходимость изуче\-ния пользователем интерактивных инструментов 
множества АИС. В~конечном итоге каждый пользователь через свой специализированный 
АРМ (приложение, автоматизирующее процесс выполнения должностных обязанностей) 
сможет получить единый интерактивный интерфейс доступа и обработки данных, получаемых 
из множества федеральных, региональных и территориальных ИР.
      
       Создание промежуточного слоя в условиях ранее сложившейся территориально 
распределенной неоднородной ИТ-ин\-фра\-струк\-ту\-ры ведомства\linebreak
приводит к необходимости 
разработки и внедрения новой вспомогательной инфраструктуры~--- совокупности 
взаимосвязанных территориально\linebreak
 распределенных программно-технических комплексов 
промежуточного слоя. Комплексы будут размещены в подразделениях и информационных\linebreak\vspace*{-12pt}
\pagebreak

\end{multicols}

\begin{figure} %fig2
\vspace*{1pt}
\begin{center}
\mbox{%
\epsfxsize=161.835mm
\epsfbox{ily-2.eps}
}
\end{center}
\vspace*{-3pt}
\Caption{Изменение информационных потоков (запросов и данных) в результате внедрения 
промежуточного слоя: АРМ~--- автоматизированное рабочее место; АИС~--- автоматизированная 
информационная система; ПТК~--- программно-технический комплекс; ИР~--- информационный 
ресурс ведомства; БД~--- база данных в составе информационного ресурса
\label{f2il}}
\vspace*{6pt}
\end{figure}

\begin{multicols}{2}

\noindent
центрах ведомства. Эти комплексы в по\-сле\-ду\-ющем станут основой (стержнем) 
горизонтальной и вертикальной информационной интеграции, модернизации и последующего 
развития всех АИС ведомства на основе функциональной и программной стандартизации. 
Программно-техническая среда промежуточного слоя возьмет на себя функции 
централизованного управления доступом пользователей к хранилищам информации, защиты 
информации от НСД, эффективного противодействия потенциальным угрозам и конкретным 
нарушениям информационной безопасности, мониторинга программно-тех\-ни\-че\-ских 
комплексов АИС, об\-нов\-ле\-ния ПО и некоторые другие важные функции.

\subsection{Задачи и этапы разработки компонентов промежуточного слоя}

      Модернизация ИТ-ин\-фра\-струк\-ту\-ры ведомства в направлении централизации должна 
предусмат\-ри\-вать \textit{модернизацию коммуникационной инфраструктуры}~[5]. 
Модернизация цифровых коммуникаций (транспортной сети ведомства) должна включать как 
создание соответствующих центров управления транспортной сетью ведомства, так и развитие 
этой транспортной сети до уровня первичных подразделений, традиционно использовавших 
открытые сети региональных провайдеров.
      
      Для обеспечения возможности построения надежной системы управления доступом 
пользователей к хранилищам информации и внедрения ведомственной системы электронного 
документооборота (СЭД) необходимо \textit{создание инфраструктуры удостоверяющих 
центров}, сертифицированной соответствующими уполномоченными государственными 
учреждениями. Помимо этого, для надежной аутентификации пользователей необходимо 
тотальное использование средств персональной идентификации.
      
      Фундаментом последующей интеграции неоднородных (представленных в разных 
моделях) ИР на уровне разнообразных приложений является 
\textit{создание единой информационной модели данных ведомства} (на первых порах хотя бы 
ядра такой единой\linebreak\vspace*{-12pt}
\pagebreak
\end{multicols}

\begin{figure} %fig3
\vspace*{1pt}
\begin{center}
\mbox{%
\epsfxsize=147.198mm
\epsfbox{ily-3.eps}
}
\end{center}
\vspace*{-3pt}
\Caption{Схема фильтрации доступа пользователей к запросам, информационным ресурсам, их частям 
и отдельным полям (группам полей)
\label{f3il}}
\vspace*{6pt}
\end{figure}

\begin{multicols}{2}

\noindent
 модели), а также \textit{единой системы классификации и кодирования 
информации (ЕСКК)}. Для обеспечения
 реализации заранее не детерминированных запросов 
необходима также разработка языковой основы средств манипулирования данными в форме 
\textit{базовых конструкций языка манипулирования данными (ЯМД)}. Созданию единой 
модели данных должно сопутствовать создание совокупности отображений этой модели на 
информационные модели всех унас\-ле\-до\-ван\-ных АИС и их хранилищ данных, а если таковые 
существуют, то и на модели данных других ведомств, с которыми необходимо осуществлять 
информационное взаимодействие. 
      
      По мере модернизации унаследованных и разработки новых АИС и баз данных 
координаты приложений, взаимодействующих с инфраструктурой промежуточного слоя, и 
управляемых ими баз\linebreak
 данных должны размещаться в хранилище метаданных~--- 
\textit{центральном репозитории} (реестре) \textit{метаданных (ЦРМ)}, образуя в конечном 
итоге \textit{информационное хранилище ведомства}~--- совокупность ИР и собственно ЦРМ. 
В частности, в ЦРМ ведомства целесообразно хранить единую схему данных или 
совокупность схем данных всех типов ИР, запросы к которым будут осуществляться с 
использованием ПО промежуточного слоя. 
      
      Помимо этого, в ЦРМ могут храниться и специфические структуры, отражающие 
особенности реализации промежуточного слоя ведомства. К~специфическим структурам 
можно отнести пред\-став\-ле\-ния данных при их визуализации, описания учетных документов и 
их структуры во взаимосвязи с особенностями политики управления доступом, описания 
сценариев поиска данных в подмножествах хранилища данных (в ИР и 
их составных частях~--- базах данных). Для последующей организации управления доступом 
пользователей к ИР важно, чтобы в схеме единой модели данных (ЕМД) были выделены классы, подклассы и 
составные классы, представленные во всей совокупности ИР ведомства, для которых могут 
быть установлены ограничения доступа (рис.~\ref{f3il})
      

      
      Разработка корректной полной и всеобъемлющей схемы данных ведомства, 
переписывание тысяч разнообразных запросов и отчетов~--- трудоемкая задача, требующая 
детальной формальной и содержательной проработки, длительного времени и значительного 
финансирования~[6]. Сложность разработки обусловлена тем, что как в самом ведомстве, так 
и в смежных по функциям ведомствах помимо сотен отраслевых баз данных федерального и 
регионального уровня существуют также сотни специализированных АИС федерального, 
регионального и территориального уровня. И~все они имеют собственные модели данных, 
используют разные языки описания и манипулирования данными, опираются на разные 
программные платформы.
      
      Обычно информационные хранилища ведомства содержат преимущественно 
структурированную информацию, и первостепенной утилитарной задачей создания ядра 
модели данных ведомства и ее отображений на модели данных унаследованных систем 
является унификация представления семантически эквивалентных объектных типов данных, 
по-разному представленных в структурах хранения множества разнородных баз данных. 

      
      В таких случаях, в особенности на начальных этапах работы по формированию ядра 
единой модели данных ведомства, целесообразно анализировать лишь те информационные 
объекты и связи между ними, которые присутствуют хотя бы в двух разнородных ИР и 
должны совместно обрабатываться. Поскольку каждая модернизируемая или вновь 
создаваемая АИС (приложение) ведомства формирует возможность доступа к конечному 
подмножеству баз данных в виде вполне определенного конечного перечня запросов, доступ к 
базам данных разнородных АИС из множества АРМ также может быть ограничен конечным 
(пусть и расширяемым) перечнем запросов. 
      
      Требования к \textit{системе централизованного управ\-ле\-ния доступом пользователей} 
к разнородным ИР должны включать не только требования по обеспечению разрешения 
(запрета) на доступ к конкретным базам данных этих ИР для каждой категории пользователей, 
но также ограничения доступа к отдельным информационным разделам и даже отдельным 
полям этих баз данных. С~организационной точки зрения должно быть проведено детальное 
изучение информационных потреб\-ностей\linebreak
основных категорий пользователей. Перечень 
категорий пользователей может соответствовать су\-щест\-ву\-ющей структуре управления 
ведомства\linebreak 
(например, управлениям и департаментам). В~соответствии с должностными 
обязанностями и информационными потребностями сотрудников внутри каж\-дой категории 
пользователей должны быть выделены функциональные \textit{роли}, определяющие права 
доступа каждой группы пользователей к содержимому каждого из существующих типов 
ИР.

      
      Интеграция отраслевых и специализированных АИС на уровне приложений имеет 
целый ряд особенностей. Задача их интеграции особенно усложняется в тех случаях, когда 
ведомство опирается на деятельность не только собственных подразделений, но и 
независимых поставщиков сервисов и услуг. Это характерно для ведомств, ориентированных 
на предоставление услуг населению (здравоохранение, образование, социальная защита, 
электронный бизнес). В~таких случаях общее число объектов информатизации даже на 
региональном уровне достигает десятков тысяч, а по стране~--- сотен тысяч, при этом число 
технических решений\linebreak
АИС и ИР также исчисляется тысячами. Включение такого числа 
разнородных объектов в информационное пространство ведомства~--- нетривиальная 
техническая и управленческая задача.


Некоторые подходы к интеграции множества 
разнородных социально-ориентированных ведомственных хранилищ данных и 
специализированных АИС на основе промежуточного слоя предложены в работах~[7--9]. Но 
эти подходы и методы нацелены на решение лишь части комплекса задач интеграции 
множества независимых баз данных и АИС. 

Кроме того, каждое крупное ведомство имеет 
уникальную исходную ИТ-ин\-фра\-струк\-ту\-ру, и надо весьма осторожно относиться к переносу 
опыта интеграции одних ведомств и регионов на другие ведомства и регионы. 

\bigskip
      
      Выводы:
      \begin{enumerate}[(1)]
\item для крупного ведомства разработка единой информационной модели сводится к 
разработке описания и единой логической модели данных и реализации расширяемого 
перечня запросов и отчетов на основе единого языка (или совокупности языков) 
манипулирования данными;
\item в качестве первого шага в решении этой задачи целесообразно создание ядра 
единой модели (например, в форме базового набора типов объектов и связей между 
ними) схемы данных и совокупности отображений этой схемы данных на множество 
схем данных всех унаследованных и проектируемых ИР. Кроме того, необходимо 
выбрать базовый ЯМД, но на начальных стадиях 
проектирования и ввода определить совокупность базовых конструкций и 
ограничений ЯМД; 
\item разработка требований к централизованной системе управления доступом сводится к 
выделению основных категорий пользователей и определению ролей доступа к содержимому 
всех типов ИР внутри каждой категории.
\end{enumerate}


\section{Специфика приложений ведомства и~состав веб-сервисов 
промежуточного слоя }

\subsection{Основные виды приложений}

      Обычно использование архитектуры, предложенной консорциумом W3C (World Wide Web Consortium)~\cite{2il}, 
ассоциируется с организацией доступа к ИР на основе портальных решений, а основным 
инструментом доступа со стороны пользователя представляется <<тонкий клиент>> (браузер). 
Портальные решения для поиска данных в базах\linebreak
 данных ведомства и предоставления 
различных сервисов~--- естественный компонент создаваемой новой ИТ-ин\-фра\-струк\-ту\-ры и 
комплекса приложений ведомства. Портальные решения удобны для реализации чисто 
информационных запросов к информационным хранилищам, а также являются пока что 
безальтернативным инструментом для предоставления ведомством услуг населению 
(сервисов).\linebreak
 Правда, в отличие от сотрудников ведомства, надежная аутентификация граждан в 
настоящее время невозможна из-за отсутствия единого национального стандарта средств 
персональной\linebreak
 идентификации граждан (электронных удостоверений) и единой 
государственной инфраструктуры поддержки электронных подписей (удосто\-ве\-ря\-ющих 
центров), что ограничивает возможности предос\-тав\-ле\-ния гражданам всего комплекса 
необходимых сервисов со стороны ведомства. 
      
      При всей важности обеспечения чисто информационных запросов пользователей к ИР 
ведомства основной задачей внедрения компьютерных технологий все же является 
автоматизация непосредственной деятельности сотрудников ведомства при решении ими 
служебных задач. В~условиях поэтапного внедрения в рамках ведомства единой схемы 
данных и единой системы классификации и кодирования информации (совокупности 
кодификаторов и справочников) необходимо реализовать как контур доступа к 
ИР через промежуточный слой, так и контур формирования и 
обновления этих ИР.
      
      Значительная часть решаемых задач требует использования специализированных 
устройств и программных инструментов, что исключает применение <<тонкого клиента>> для 
этих целей. Во многих случаях к отдельным типам приложений предъявляются жесткие 
требования в части ре\-ак\-тив\-ности. 
      
      Это характерно для приложений, автоматизирующих процесс предоставления 
сотрудникам ведомства или населению жестко регламентированных услуг. Для повышения 
реактивности таких приложений широко используются локальные классификаторы и 
внутренние базы данных приложений, представление которых в единой схеме данных 
ведомства нецелесообразно.
      
      Существуют также приложения, использующие специфические форматы данных и 
специализированные драйверы (например, при работе с устройствами обработки 
биометрической информации, лабораторным оборудованием и~т.\,п.). 
      
      В связи с этим многие специализированные АИС территориальных подразделений 
ведомства целесообразно реализовывать в виде клиент-сер\-вер\-ных решений, причем основная 
часть необходимой информации, включая все необходимые классификаторы, должна 
располагаться непосредственно на серверах локальной сети подразделения. Получение же 
информации из других (внешних) баз данных должно осуществляться через промежуточный 
слой в интерактивном или пакетном режиме с использованием программных интерфейсов и 
базы данных запросов при обеспечении надежной аутентификации и идентификации как 
пользователей запросов, так и собственно приложений. 
      
      Практически в любом ведомстве существует класс задач, для решения которых удобно 
использовать не интерактивный, а пакетный (отложенный) режим обработки запросов к базам 
данных. Отложенный режим характерен для решения большинства задач электронного 
документооборота и значительной части задач по формированию ИР (ввод новых данных), 
обновления всевозможных классификаторов, формирования и получения различных справок и 
отчетов, при выполнении запросов, допускающих получение информации из базы данных в 
течение нескольких часов или минут, а не секунд. Пакетный режим обработки предпочтителен 
(вне зависимости от типа запросов) в случае использования коммутируемых и 
низкоскоростных линий связи.
      
      С учетом вышесказанного технические решения промежуточного слоя должны 
обеспечить доступ к базам данных со стороны систем автоматизации подразделений 
ведомства на уровне\linebreak
программных интерфейсов, включая поддержку интерфейсов обращения 
к АИС как в режиме реального времени, так и с использованием технологий электронной 
почты.

\begin{table*}\small
\begin{center}
\Caption{Пример набора технологических сервисов промежуточного слоя
\label{t1il}}
\vspace*{2ex}

\begin{tabular}{|p{51mm}|p{51mm}|p{51mm}|}
\hline
\multicolumn{1}{|c|}{
\tabcolsep=0pt\begin{tabular}{c}Сервисы\\ манипулирования данными\\
(для компонентов АИС,\\ взаимодействующих\\ с базами 
данных ИР)\end{tabular}}&
\multicolumn{1}{|c|}{
\tabcolsep=0pt\begin{tabular}{c}Сервисы доступа к базам данных\\ промежуточного слоя\\ (для компонентов АИС,\\ 
взаимодействующих\\ с промежуточным слоем)\end{tabular}}&
\multicolumn{1}{|c|}{
\tabcolsep=0pt\begin{tabular}{c}Административные сервисы\\
(для административных\\ компонентов\\ промежуточного слоя)\end{tabular}} \\
\hline
\multicolumn{1}{|l|}{\raisebox{-4pt}[0pt][0pt]{Реализация запросов к ИР}}&Доступ к схемам данных ИР и 
классификаторам&\multicolumn{1}{|l|}{\raisebox{-4pt}[0pt][0pt]{Корректировка ЕМД, ЕСКК, ЦРМ}}\\
\hline
Обновление данных в базах данных ИР&Доступ к описаниям типовых запросов&Корректировка баз 
данных управления доступом пользователей\\
\hline
Обмен информационными массивами (загрузка/выгрузка)&Получение прав доступа к ИР для 
текущего пользователя&Корректировка прав доступа административного персонала\\
\hline
\end{tabular}
\end{center}
\end{table*}

\subsection{Базовые веб-сервисы промежуточного слоя}

      После развертывания программно-технической инфраструктуры промежуточного слоя 
и модернизации унаследованных АИС новая ИТ-ин\-фра\-струк\-ту\-ра ведомства может быть 
представлена как совокупность приложений, реализующих заданные прикладные функции с 
использованием новых API-ин\-тер\-фей\-сов и удовлетворяющих требованиям информационной 
безопасности ведомства. Между приложениями и промежуточным слоем определяются 
стандартизованные интерфейсы, реализуемые в форме веб-сер\-ви\-сов. Для определения состава 
этих веб-сер\-ви\-сов исходными данными являются решения по архитектуре и структуре 
промежуточного слоя, а также требования ведомства к основным функциям промежуточного 
слоя. 
      
      Поскольку в основу интероперабельности разнородных АИС положена 
      сервис-ориентированная архитектура, все взаимодействие приложений 
специализированных АРМ различного назначения с АИС осуществляется не напрямую, а 
исключительно через специальную инфраструктуру промежуточного слоя.
      
      Детальный анализ требований по взаимодействию со стороны приложений ведомства 
позволяет сформулировать перечень веб-сер\-ви\-сов промежуточного слоя. Для каждого 
ведомства перечень этих сервисов может разниться. Однако можно выделить перечень 
сервисов, которые должны быть реализованы в любом случае, поскольку они носят не 
содержательный, а технологический характер. Пример такого набора технологических 
сервисов приведен в табл.~\ref{t1il}. 
       
       

      
      Взаимодействие узлов промежуточного слоя с АРМ/АИС заключается в приеме 
информации (запросов и данных для обновления БД) от АРМ/АИС и в передаче им 
результатов исполнения запросов и информации о ходе отработки запросов и об\-нов\-ле\-ния 
данных в базах данных. Реализация указанных видов взаимодействия выполняется с помощью 
интерактивных и пакетных сервисов (сервисы взаимодействия с АРМ/АИС по отработке 
запросов и сервисы взаимодействия с АРМ/АИС по вводу данных с \mbox{целью} обновления ИР). 
      
      К основным способам хранения документов, описаний метаданных и приложений в 
ведомственных информационных системах можно отнести:
      \begin{itemize}
\item документы на языках HTML\footnote{HyperText Markup Language~--- язык разметки гипертекста.},
XML\footnote{EXtensible Markup Language~--- расширяемый
язык разметки.}, а также файлы (текстовые, в форматах Word, 
Excel, PowerPoint, Flash, Acrobat, графические, звуковые и видео файлы, файлы 
биометрической информации и различные файлы специализированного формата, такие 
как DICOM\footnote{Digital Imaging and COmmunications in Medicine~--- индустриальный стандарт создания,
хранения, передачи и визуализации медицинских изображений и документов
обследованных пациентов.});
\item базы данных и процедуры SQL\footnote{Structured Query Language~--- язык структурированных
запросов.}, JAVA-аплеты, скрипты и~т.\,п.
\end{itemize}

      Визуализация данных должна осуществляться приложениями на устройства различного 
типа:
      \begin{itemize}
\item мониторы настольных и портативных компьютеров (различной разрешающей 
способности);
\item карманные персональные компьютеры (PDA~--- Personal Digital Assistant));
\item сенсорные мониторы для точек публичного доступа;
\item другие устройства, прежде всего различные мобильные (гибридные) устройства, 
объеди\-ня\-ющие в едином корпусе мобильный телефон, сканер, видеокамеру и PDA.
\end{itemize}
      
      Для каждого типа устройств, принятого на вооружение в ведомстве, необходимо 
обеспечить представление информации в необходимом формате и размере. При этом часто 
возникает проб\-ле\-ма визуализации одного и того же документа в различных форматах, 
необходимость хранения нескольких экземпляров одного и того же документа и даже 
реализация нескольких комплектов одних и тех же запросов для различных типов устройств.
      
      Выводы:
      \begin{enumerate}[(1)]
\item перечень веб-сер\-ви\-сов промежуточного слоя специфичен для каждого ведомства, 
поскольку должен учитывать особенности предметной области и технологий представления и 
обработки данных. Представление данных как в информационных хранилищах, так и на 
оконечных устройствах пользователей для некоторых ведомств могут иметь значительную 
специфику;
\item можно выделить совокупность веб-сер\-ви\-сов, не связанных со спецификой предметной 
об\-ласти. Это сервисы, реализующие техноло\-гические функции, связанные с аутен\-ти\-фикацией 
и определением ПДП, поиском и обнов\-ле\-нием данных, доступа к 
метаданным и классификаторам. В~отдельный блок можно выделить сервисы доступа к 
административным базам данных и сервисы, необходимые для прикладного уровня 
управления функционированием.
\end{enumerate}

 \begin{figure*} %fig4
 \vspace*{1pt}
\begin{center}
\mbox{%
\epsfxsize=85.041mm
\epsfbox{ily-4.eps}
}
\end{center}
\vspace*{-3pt}
 \Caption{Упрощенная схема взаимодействия приложения с базами данных ЕСК, УЦ и ПДП: ЕСК~--- 
Единая система каталогов, содержит учетные записи пользователей; УЦ~--- удостоверяющий центр, 
содержит сертификаты (электронные подписи) пользователей; ПДП~--- права доступа пользователей;
ПСИ~--- персональное средство 
идентификации, съемный носитель, содержащий сертификат пользователя; SID (Security IDentifier)~--- 
идентификатор безопасности
 \label{f4il}}
 \end{figure*}


\section{Доступ пользователей к~информационным ресурсам и~управление доступом 
на~основе ролей}
      
      Методы и технологии реализации цент\-ра\-ли\-зованного доступа пользователей к 
разнородным базам данных ведомства могут различаться. Значительное влияние на 
технологии реализации оказывает выбор базовой программной платформы промежуточного 
слоя, а также программные платформы ключевых унаследованных АИС ведомства. Ниже 
описываются основные структуры хране-\linebreak ния
и методы реализации системы управления\linebreak 
доступом пользователей к разнородным АИС ведомства, реализованные в рамках нескольких 
на\-уч\-но-исследовательских и опытно-кон\-струк\-тор\-ских 
работ, выполненных в ИПИ РАН.

      
      Основными структурами хранения информации о пользователях различных 
приложений ведомства являются следующие:
      \begin{itemize}
\item база данных цифровых сертификатов (открытых и закрытых ключей и их 
владельцев)~--- хранится только в центрах регистрации, не доступных для программного 
доступа со стороны любых приложений на сетевом уровне;
\item базы данных действующих и отозванных циф\-ро\-вых сертификатов (открытых 
ключей) сотрудников ведомства, доступные приложениям, но не включающие 
информации о\linebreak
 владельцах,~--- хранятся в центрах сертификации;
\item единая система каталогов пользователей (учетные записи всех пользователей);
\item база данных прав доступа всех категорий пользователей, включая все категории 
об\-слу\-жи\-ва\-юще\-го персонала;
\item база данных пользователей и принадлежность этих пользователей к категориям 
доступа.
\end{itemize}
      
      Важной вспомогательной структурой, используемой при управлении доступом 
пользователей к ИР, является база данных перечня типовых запросов 
(ПТЗ).
      
      Если первые три структуры хранения информации о пользователях обязаны 
присутствовать в составе ИТ-ин\-фра\-струк\-ту\-ры ведомства независимо от необходимости 
создания централизованной\linebreak
 сис\-те\-мы управления доступом пользователей к множеству 
ИР, то две последние структуры хранения необходимы именно для 
реализации централизованного управления доступом пользователей к базам данных 
разнородных ИР и электронного документооборота. Все 
вышеперечисленные административные базы данных должны быть корректны и увязаны 
между собой, что является одной из важных задач обслу\-жи\-ва\-юще\-го персонала. 
На~рис.~\ref{f4il} приведена упрощенная схема информационного взаимодействия 
приложения АРМ пользователя с базами данных Active Directory (осуществляется системными 
средствами операционной системы) и ПДП к выполнению конкретного запроса или 
доступа к конкретному ИР (осуществляется с использованием 
      веб-сер\-ви\-са системы управления доступом). В~качестве персонального средства 
идентификации (ПСИ) могут использоваться любые съемные носители (магнитные карты, дискеты, 
флеш-носители).
 
      
      Остановимся несколько подробнее на специфике этих структур хранения информации 
о пользователях. 
      
      \textit{Система удостоверяющих центров} реализуется как совокупность 
комплекса прог\-рам\-мно-тех\-ни\-че\-ских средств центров регистрации и цент\-ров сертификации, а 
также орга\-ни\-за\-ци\-он\-но-тех\-ни\-че\-ских мероприятий, необходимых для использования 
криптографических функций в целях защиты информации от НСД, аутентификации 
пользователей единого информационного пространства 
и разграничения их прав доступа, подтверждения авторства и обеспечения 
целостности и подлинности электронных документов.
      
      Корневой удостоверяющий центр является вершиной древовидной структуры иерархии 
доверия в СУЦ и является вышестоящим по отношению к подчиненным УЦ. Подчиненные 
УЦ ведомства целесообразно располагать в информационных центрах ведомства, например в 
субъектах федерации. Основной задачей УЦ является выпуск сертификатов открытого ключа 
для пользователей ведомства. Выпуск сертификатов, все действия по управлению 
сертификатами на основании заявок, публикация сертификатов пользователей и списка 
отозванных сертификатов УЦ осуществляются в автоматизированном режиме. Система удостоверяющих центров ведомства 
может взаимодействовать с СУЦ других органов государственной власти.
      
      \textit{Единая система каталогов} пользователей ЕСК (LDAP\footnote{Lightweight Directory Access Protocol~---
      облегченный протокол доступа к каталогам.} 
      или Active Directory)~--- 
общепринятая для всех программных платформ основа идентификации пользователей на 
прикладном уровне. Детальный анализ назначения, функций и методов\linebreak проектирования и 
использования ЕСК в ведомственных системах выходит за рамки данной публикации. Здесь 
важно только отметить, что учетные записи пользователей должны содержать как минимум 
сис\-тем\-ные идентификаторы безопасности (SID) всех пользователей, а также поля для связи с 
базой данных пользователей.
      
      \textit{База данных ПДП} является основой управления 
доступом пользователей на всех уровнях обработки запросов 
(приложение\;$\rightarrow$\;промежуточный слой\;$\rightarrow$\;АИС). Для каждого 
пользователя в базе данных ПДП определяется персональная запись, связанная с его учетной записью в 
ЕСК, и следующие, связанные с учетной записью пользователя, данные: 
      \begin{itemize}
\item перечень выданных пользователю разрешений на доступ;
\item перечень типовых ролей, в которые включен пользователь.
\end{itemize}

      Также в базе данных ПДП хранится информация о типовых ролях пользователей и информация 
об иерархии типовых ролей.

\begin{figure*} %fig5
\vspace*{1pt}
\begin{center}
\mbox{%
\epsfxsize=110.676mm
\epsfbox{ily-5.eps}
}
\end{center}
\vspace*{-3pt}
      \Caption{Базы метаданных, используемые для управления доступом и обработки запросов: 
ПДП~--- права доступа пользователей; ЕСКК~--- единая система классификации и кодирования; 
ЕМД~--- единая модель данных; ЦРМ~--- центральный репозиторий метаданных; ПТЗ~--- перечень 
типовых запросов
       \label{f5il}}
       \vspace*{6pt}
       \end{figure*} 

      
      \textit{Типовые запросы.} Каждый типовой запрос описывает определенный класс 
реальных (ис\-пол\-ня\-емых) запросов, которые могут различаться\linebreak
 значениями поисковых 
параметров, составом возвращаемых полей данных или составом поисковых фильтров. Для 
описания такого класса запросов предназначен шаблон запроса~--- формальное 
      XML-описание, специфицирующее общую структуру\linebreak
       типового запроса, его семантику, 
а также все возможные вариации запроса. Простейшим при\-емом поддержки варьируемых 
структурных элементов типового запроса может быть их описание с по\-мощью текстовых 
подстановок, включаемых в шаблон запроса. Неварьируемые (фиксированные) структурные 
элементы типового запроса всегда должны описываться в терминах единой модели данных. 
Описание запроса в терминах единой модели является универсальным и не зависит от 
специфики реализации различных приложений, которые исполняют запрос. Рассмотрим эту 
технологию несколько подробнее.
      
      Шаблон запроса описывает целый класс реальных (исполняемых) запросов, поэтому 
каждый раз при формировании реального запроса необходимо конкретизировать его 
структуру и параметры. Для этой цели служит макет запроса, пред\-став\-ля\-ющий собой 
      XML-описание структуры и параметров исполняемого запроса. Подготовка макета 
запроса является задачей приложения, инициирующего поисковый запрос. Макет запроса 
может рассматриваться как формальная спецификация поискового запроса, предназначенного 
для исполнения в АИС.
      
      Каждый типовой запрос связывается с множеством ИР, где он может быть исполнен. 
При конструировании и связывании типового запроса с конкретными приложениями и базами 
данных выражения в шаблоне запроса переписываются из терминов единой схемы данных в 
термины схемы данных ИР (отображение шаблона запроса на ИР). 
      
      Перед исполнением типового запроса в приложении, осуществляющем доступ к базам 
данных ИР, всегда выполняется трансляция запроса. Процесс трансляции осуществляется 
автоматически средствами ПО промежуточного слоя перед передачей его на исполнение в 
приложение. Результатом трансляции служит текст запроса в терминах схемы базы данных 
ИР, готовый к подаче на вход конкретного приложения. Результаты исполнения любого 
типового запроса представляются в XML-фор\-ма\-те в терминах единой схемы данных. Для 
каждого типового запроса задается схема данных, описывающая формат представления 
результатов. 

      
      Каждый типовой запрос связывается с одной или несколькими группами доступа. На 
основании\linebreak
этих связей вычисляются права доступа конкретного пользователя к типовому 
запросу. Фильтрация информации с учетом прав доступа конкретного пользователя согласно 
его роли может\linebreak
осуществляться как на уровне приложений, так и на уровне компонентов 
промежуточного слоя. Типовые запросы могут связываться также с типовыми ролями с целью 
блокирования возможности отдельных групп пользователей обращаться к базам данных ИР с 
определенными запросами. На рис.~\ref{f5il} схематично представлен состав баз метаданных, 
используемых компонентами промежуточного слоя при реализации сервисов обработки 
запросов и управления доступом. 
      

      Завершая описание предложенной модели групповых политик пользователей по 
доступу к ИР ведомства, можно сформулировать основные уровни 
ограничения (или наоборот разрешения)\linebreak доступа групп пользователей к структурам данных и 
операциям. Методически представляется целесообразным выделить два уровня управления 
доступом пользователей~--- уровень запросов и уровень данных. 
      
      Уровень запросов:
      \begin{itemize}
\item доступ к запросам на обновление баз данных (индивидуально для каждого ИР);
\item доступ к запросам на поиск в конкретных базах данных однотипных (например, типовых 
территориальных) ИР и разнородных ИР с семантически однородными данными.
\end{itemize}

Уровень данных (для всех запросов):
\begin{itemize}
\item доступ к отдельным ИР и группам ИР;
\item доступ к отдельным базам данных (разделам ИР) и отдельным учетным документам;
\item доступ к отдельным полям ИР (в терминах единой схемы данных) для всех запросов.
\end{itemize}

      Помимо вышеперечисленных, в рамках ве\-домства могут существовать и 
дополнительные требова\-ния к групповым политикам. Прежде всего, могут накладываться 
ограничения по возможности использования различных масок и логических функций в 
поисковых полях запросов, а\linebreak также ограничения по числу релевантных объектов, 
предос\-тав\-ля\-емых пользователю в качестве результата. Особое значение такие требования 
могут иметь в целях выполнения законодательных ограничений на доступ к персональным 
данным, а также предот\-вра\-ще\-ния несанкционированного копирования данных из баз данных 
ведомства. Могут быть сформулированы требования к фильтрации результатов запросов к 
совокупности ИР (например, с целью устранения дублирующей информации), реализации 
каскадных запросов (результаты выполнения одного запроса являются параметрами 
по\-сле\-ду\-ющих запросов). Поскольку обычно новые законодательные, межведомственные и 
внут\-ри\-ведомственные требования на начальном этапе не подкреплены правоприменительной 
практикой, необходимо поэтапное уточнение требований на примере решения конкретных 
прикладных задач. 
      
      По мере наработки успешных технических решений реализации этих дополнительных 
требований в рамках конкретных АИС, методы и алгоритмы решений могут быть обобщены и\linebreak 
представлены в промежуточном слое. Например, могут быть реализованы дополнительные 
препроцессоры запросов или постпроцессоры обработки результатов запросов. Обращения же 
к соответствующим дополнительным функциям промежуточного слоя будут оформлены в 
виде веб-сер\-висов.
\columnbreak
      
      Выводы: 
      \begin{enumerate}[(1)]
\item практический опыт реализации системы управ\-ле\-ния доступом пользователей к 
разнородным базам данных на основе единой схемы данных и технологий веб-сер\-ви\-сов 
доказал возможность интеграции на\linebreak прикладном уровне разнородных унаследованных АИС, 
постро\-енных на различных прог\-рам\-мных платформах. При этом трудоемкость модернизации 
унаследованных АИС составляла не более 5\% трудоемкости проектирования новых АИС, 
удовлетворяющих требованиям ведомства к интеграции данных и обеспечению 
информационной безопасности;
\item структурной основой хранения информации о пользователях (сотрудников данного 
ведомства и других ведомств) и их правах доступа к ИР ведомства 
являются:
\begin{itemize}
\item СУЦ;
\item единая система каталогов пользователей;
\item база данных ПДП (групповых ролей);
\item база данных пользователей;
\end{itemize}
\item в рамках описанного технического решения групповые политики доступа пользователей 
к АИС и базам данных ИР ведомства осуществляются на следующих 
уровнях:
\begin{itemize}
\item уровень перечня ИР;
\item уровень отдельных баз данных (разделов) каждого ИР;
\item уровень отдельных полей единой схемы данных;
\item уровень возможности взаимодействия с конкретными АИС и доступности запросов к 
базам данных ИР;
\end{itemize}
\item для упрощения процесса проектирования сис\-те\-мы запросов в терминах единой схемы 
данных с использованием базового подмножества ЯМД реализованы методы шаблона\linebreak
 запроса 
и подстановок. Эти методы обеспечивают единообразную интерпретацию компонентами 
промежуточного слоя запросов к множеству АИС на основе единой схемы данных и 
компактного подмножества конструкций единого ЯМД, интерпретируемого всеми АИС. 
С~другой стороны, эти методы позволяют дополнять базовые конструкции ЯМД произвольными 
языковыми конструкциями, в том числе конструкциями ЯМД конкретных СУБД, 
используемых при разработке АИС.
\end{enumerate}

\section{Заключение}

      Модернизация систем информационного обес\-пе\-че\-ния ведомства должна 
осуществляться с\linebreak
учетом необходимости взаимодействия с аналогичными системами других 
ведомств. Интероперабельность разнородных систем можно обеспечить при использовании 
инфраструктуры промежуточного слоя. Инфраструктура промежуточного слоя решает 
следующие задачи:
      \begin{enumerate}[(1)]
\item  обеспечение интероперабельности информационных систем на уровне приложений 
промежуточного слоя при совместной обработке данных из множества информационных 
хранилищ, включая комплексные и каскадные запросы сразу к нескольким разнородным 
информационным системам;
      \item перенаправление всех запросов сотрудников ведомства (и других ведомств) в 
единую территориально распределенную инфраструктуру аутентификации и управления 
доступом пользователей взамен ранее используемых прямых обращений к АИС и~ИР.
      \end{enumerate}
      
      Создание промежуточного слоя в условиях ранее сложившейся территориально 
распределенной неоднородной ИТ-ин\-фра\-струк\-ту\-ры ведомства\linebreak
 сводится к разработке и 
внедрению совокупности взаимосвязанных территориально распределенных 
прог\-рам\-мно-тех\-ни\-че\-ских комплексов промежуточного слоя. Эти комплексы служат 
основой\linebreak (стержнем) горизонтальной и вертикальной информационной интеграции, 
модернизации и последующего развития всех АИС ведомства на\linebreak
 основе функциональной и 
программной стандартизации. Программно-тех\-ни\-че\-ская среда промежуточного слоя берет на 
себя функции централизованного управления доступом пользователей к хранилищам 
информации и обеспечения информационной безопасности.
      
      Разработка единой информационной модели ведомства сводится к разработке единой 
схемы данных и реализации расширяемого перечня запросов и отчетов на основе единого 
языка (или со\-во\-куп\-ности языков) манипулирования данными. В~качестве первого шага в 
решении этой задачи целесообразно создание ядра единой схемы данных и определение 
совокупности базовых конструкций и ограничений выбранного языка манипулирования 
данными.
      
      Перечень веб-сер\-ви\-сов ПО промежуточного слоя специфичен для каждого ведомства, 
однако можно выделить совокупность веб-сер\-ви\-сов, не связанных со спецификой предметной 
области. Это сервисы, связанные с аутентификацией и определением прав доступа 
пользователей, поиском и об\-нов\-ле\-ни\-ем данных, доступом к метаданным и классификаторам, 
доступом к административным базам данных, и сервисы, необходимые для прикладного 
уровня управления функционированием.
      
      Разработка требований к централизованной сис\-те\-ме управления доступом сводится к 
выделению основных категорий пользователей и определению ролей доступа к 
ИР внутри каждой категории. Структурной основой хранения 
информации о пользователях (сотрудниках данного ведомства и других ведомств) и их правах 
доступа к ИР ведомства являются:
      \begin{itemize}
\item СУЦ;
\item единая система каталогов пользователей;
\item база данных ПДП (групповых ролей);
\item база данных пользователей. 
\end{itemize}
      
      Для упрощения процесса проектирования сис\-те\-мы запросов в терминах единой схемы 
данных с использованием базового подмножества ЯМД целесообразно использовать шаблоны 
запросов и подстановок. 
      
      Технические решения, изложенные в данной пуб\-ли\-ка\-ции, прошли практическую 
апробацию~[7--9]. При этом осуществлялась модернизация унаследованных АИС, 
разработанных с использованием разных инструментальных средств и функционирующих на 
комплексах различной технической и программной архитектуры (3~семейства операционных 
систем, 4~семейства СУБД).

\vspace*{6pt}

{\small\frenchspacing
{%\baselineskip=10.8pt
\addcontentsline{toc}{section}{Литература}
\begin{thebibliography}{9}
     
\bibitem{1il}
Анализ развития и использования информационно-ком\-му\-ни\-ка\-ци\-он\-ных технологий в 
регионах России: Аналитический доклад~/ Под ред.\ Ю.\,Е.~Хохлова.~--- М.: Институт 
развития информационного общества, 2008.  240~с.

\bibitem{2il}
Web Services Activity, W3C. {\sf http://www.w3.org/ 2002/ws/}.

\bibitem{3il}
\Au{Захаров В.\,Н., Калиниченко~Л.\,А., Соколов~И.\,А., Ступников~С.\,А.}
Конструирование канонических информационных моделей для интегрированных 
информационных систем~// Информатика и её применения, 2007. Т.~1. Вып.~2. С.~15--38.

\bibitem{4il}
\Au{Босов А\, В.}
Порталы в системах органов государственной власти~// Информатика и её применения, 
2008. Т.~2. Вып.~1. С.~44--54.

\bibitem{5il}
\Au{Зацаринный А.\,А., Ионенков~Ю.\,С., Кондрашев~В.\,А.}
Об одном подходе к выбору системотехничсеких решений построения 
информационно-те\-ле\-ком\-му\-ни\-ка\-ци-\linebreak\vspace*{-12pt}
\pagebreak

\noindent
он\-ных систем~// Системы и средства информатики, 2006. 
Вып.~16. C.~66--71.

\bibitem{6il}
\Au{Брюхов Д.\,О., Вовченко А.\,Е., Захаров~В.\,Н., Желенкова~О.\,П., Калиниченко~Л.\,А., 
Мартынов~Д.\,О., Скворцов~Н.\,А., Ступников~С.\,А.}
Архитектура промежуточного слоя предметных посредников для решения задач над 
множеством интегрируемых неоднородных распределенных информационных ресурсов в 
гибридной грид-инфраструктуре виртуальных обсерваторий~// Информатика и её 
применения, 2008. Т.~2. Вып.~1. С.~2--34. 

\bibitem{7il}
\Au{Поляков С.\,В., Костомарова~Л.\,Г., Щаренская~Т.\,Н., Илюшин~Г.\,Я.}
 Корпоративная автоматизированная сис\-те\-ма здравоохранения города Москвы (КАИС 
<<Мосгорздрав>>)~// Информационное общество, 2008. №\,1. С.~20--25.

\bibitem{8il}
\Au{Илюшин Г.\,Я.}
Информационная архитектура региональных проектов здравоохранения на примере проекта 
<<Удаленная регистратура>>~// Информационное общество, 2008. №\,1. С.~31--40.



\label{end\stat}

\bibitem{9il}
\Au{Соколов И.\,А., Зацаринный А.\,А., Захаров~В.\,Н., Илюшин~Г.\,Я., Кузьмин~А.\,П., 
Цыганков~В.\,С.}
Основные сис\-те\-мо\-тех\-ни\-ческие решения по построению ЕИТКС ОВД~// Системы и 
средства информатики. Спец. вып. На\-уч\-но-тех\-ни\-че\-ские вопросы построения и развития 
ин\-фор\-ма\-ци\-он\-но-те\-ле\-ком\-му\-ни\-ка\-ци\-он\-ной сис\-те\-мы органов внутренних дел.~--- М.: ИПИ 
РАН, 2009. С.~11--33.
       
 \end{thebibliography}
}
}
\end{multicols}  %5

\def\stat{seif}


\def\tit{НЕФТЬ КАК НОСИТЕЛЬ ИНФОРМАЦИИ О~СВОЕМ 
ПРОИСХОЖДЕНИИ, СТРУКТУРЕ И ЭВОЛЮЦИИ}
\def\titkol{Нефть как носитель информации о~своем 
происхождении, структуре и эволюции}

\def\autkol{Р.\,Б.~Сейфуль-Мулюков}
\def\aut{Р.\,Б.~Сейфуль-Мулюков$^1$}

\titel{\tit}{\aut}{\autkol}{\titkol}

%{\renewcommand{\thefootnote}{\fnsymbol{footnote}}\footnotetext[1]
%{Работа выполнена
%при финансовой поддержке РФФИ, проекты 08-01-00567 и
%08-07-00152.}}

\renewcommand{\thefootnote}{\arabic{footnote}}
\footnotetext[1]{Институт проблем информатики Российской академии наук, rust@ipiran.ru}     
     
     \Abst{Статья посвящена одному из аспектов применения законов информатики при 
исследовании сложных природных систем. Нефть как сложную систему можно 
рассматривать в качестве носителя информации, необходимой для оценки гипотез ее 
происхождения. Этой информацией является элементный и углеводородный состав нефти, 
распространение по площади, разрезу и в разновозрастных комплексах осадочного чехла. 
Носителем информации в нефти служат атомы углерода и водорода, определяющие 
химические связи и способности формировать углеводородные последовательности. 
Показано, что объем информации в исходном веществе и самой нефти позволяет оценить 
и сравнить существующие модели ее происхождения.}
     
     \KW{нефть; происхождение нефти; нефть как носитель информации; нефть как
сложная система; оценка объема информации; объем информации нефти}

     \vskip 18pt plus 9pt minus 6pt

      \thispagestyle{headings}

      \begin{multicols}{2}

      \label{st\stat}
     
     \section{Введение}
     
     Наряду со многими удивительными свойствами и характеристиками нефть 
обладает уникальной особенностью~--- ее исходное вещество, состав, условия 
залегания и происхождение можно трактовать с самых разных, нередко 
диаметрально противоположных точек зрения, и все они выглядят весьма 
обоснованными. Наглядно это можно видеть на примере гипотез о происхождения 
нефти.
     
     Вначале появились две противоположные гипотезы, причем обе были 
высказаны априори как догадки, поскольку геологическая, геохимическая и 
физическая доказательные базы в то время только закладывались. 
     
     Согласно первой, органической, гипотезе нефть образовалась из органических 
остатков бактерий, животных и растений, т.\,е.\ является детищем биосферы. Эта 
гипотеза активно поддерживается большинством геохимиков-нефтяников, 
несмотря на то, что ее единственным аргументом служат результаты изучения 
геохимии незначительной (не более первых процентов) составляющей осадочных 
пород, керогена. 
     
     Согласно неорганической гипотезе нефть имеет абиогенное, глубинное 
происхождение. Пред\-став\-ле\-ния об абиогенном происхождении нефти получили 
доказательства в связи с установлением углеводородной дегазации Земли, 
наличием клатратов и газогидратов. Эти явления и их масштабы на Земле в целом 
описаны в работах~[1--4].
     
     Анализ альтернативных гипотез показывает, что понятия, принципы и законы 
информатики практически не используются в исследованиях природы нефти. 
Законы информатики о сложной системе и ее развитии, единице информации (бите), 
которой можно выразить любое физическое вещество, информационная энтропия и 
ее изменения, наконец, понятие носитель информации и формы его существования 
и выражения практически не рассматриваются в приложении к изучению нефти. 
     
     Вместе с тем химический и элементный состав нефти, ее физические свойства 
и их воплощение в геологических условиях залегания нефти определяют ее как 
сложную систему~\cite{21s}. Поэтому законы информатики как средство их 
познания применимы к изучению проблемы ее генезиса. В~статье рас\-смат\-ри\-ва\-ет\-ся 
нефть как носитель информации, достаточной для объяснения ее состава, 
распределения в недрах по агрегатному состоянию и причинно-следственных 
факторов образования нефти~--- от исходного до конечного вещества.

\section{Информация о~первооснове нефти}
     
     Считается, что носитель информации~--- это физическое средство выражения 
абстрактного понятия. Носителями информации являются камень, дерево, папирус, 
бумага, холст, флэш-память, диск, другие физические материалы и нефть. 
В~первооснове носителем информации является атом, а точнее~--- электроны его 
орбиталей и элементарные частицы ядра. Биологическим аналогом является ген, в 
котором заложена вся информация о живом организме. 
     
     Изучение проблем нефти с позиций информатики диктует необходимость 
выделения ее первоосновы в каждой из двух гипотез происхождения нефти. 
У~органиков это химически стойкое органическое вещество, а именно липиды 
(соли жирных кислот), созданное живыми организмами биосферы, у 
     неоргаников~--- химические вещества и минералы, слагающие недра Земли, 
взаимодействие которых приводит к образованию углеводородов. 
     
     В органической гипотезе носителем информации о нефти является геном 
живого организма. В~неорганической гипотезе первичная информация о нефти как 
совокупности углеводородных последовательностей формируется на субатомном 
уровне, отражая особенности взаимодействия атомов углерода и водорода между 
собой и с другими элементами. 
     
     Определим информацию, заложенную в первооснове нефти. 
     
     Одной из важнейших является информация об элементном составе. Нефть в 
среднем на 99,9\% состоит из пяти элементов, их усредненный состав: 
углерода~--- 85\%, водорода~--- 13\%, кислорода~--- 0,8\%, азота~--- 0,6\% и серы~--- 
0,6\%~\cite{16s}. Эта информация определяет термодинамические, геологические и 
геохимические условия, обеспечившие концентрацию в одном веществе 85\% 
углерода и 13\% водорода. Глубинные зоны Земли могут обеспечить условия с 
такой термодинамикой, геологией и геохимией, а следовательно, и процесс 
соединения углеводородных последовательностей в нефть.
{\looseness=1

}
     
     Органическая гипотеза образования нефти требует обосновать энергию, 
обеспечившую трансформацию низкоуглеродного вещества живого, 
соответственно 20\% и 8,5\% в бактерии и водоросли, в нефть с 85\% углерода. 
Такое обоснование не всегда возможно. Недра многих нефтеносных территорий, 
особенно платформенных, никогда не погружались на глубины с температурой и 
давлением, необходимыми для катагенеза органических остатков~\cite{19s}. 
Недостаток температуры и давления для катагенеза сторонники органической 
гипотезы компенсируют временем низкотемпературного катагенеза, увеличивая 
его на десятки и даже сотни миллионов лет, или миграцией нефти из далеко 
расположенных зон генерации, в которых такие условия создавались, к зонам 
аккумуляции. Время, даже если оно ис\-чис\-ля\-ет\-ся сотнями миллионов лет, не может 
компенсировать недостаток энергии (температуры и давления), необходимой для 
катагенеза исходного вещества. Исходное вещество органической природы за эти 
миллионы лет в результате обмена веществом и энергией с окружающей геологической 
средой, и прежде всего с водой, неизменно потеряет свои самые ценные 
компоненты и утратит нефтеродные свойства. 
     
     Углеводородный состав нефти несет информацию о ее генезисе. 
Углеводороды нефти разделяются на две неравные части. Большая из них, 
составляющая от 80\% до~90\%,~--- это чистые углеводороды, т.\,е.\ молекулы, 
состоящие только из атомов углерода и водорода. Углеводородных соединений в 
нефти примерно~500, и ни одно из них в чистом виде не встречается ни в одном 
живом организме и не синтезируется непосредственно ни одним из них. 
Простейший углеводород метан может быть газообразным продуктом 
жизнедеятельности либо анаэробного биохимического разложения органических 
остатков на ранних стадиях их разложения, т.\,е.\ образованным из готовых 
органических материалов. Однако количества образующегося таким образом 
метана недостаточно для формирования углеводородов нефти, содержащихся в 
осадочных породах.
     
     Меньшая часть нефти (в среднем~4,5\%) пред\-став\-ле\-на гетероциклическими 
соединениями или гетероциклами. Это углеводородные соединения, часть атомов 
углерода которых замещена на атомы серы, кислорода или азота в более мягких 
термодинамических условиях, что создало условия для невалентных, водородных и 
ковалентных химических связей углеводородов с металлами и неметаллами. Нефть 
включает примерно 250~сернистых, 85~кис\-ло\-род\-ных и 30~зотных 
гетероциклов~\cite{16s}. Ге\-те\-ро\-цик\-лы имеют прямое отношение к биосфере, 
особенно азотсодержащие. Их некоторые характеристики важны для понимания 
генезиса нефти. Например, кероген нефтематеринских пород как основа гипотезы 
органического происхождения нефти~--- это в большей части 
гетероциклы~\cite{34s, 4s}. Как отмечает Пожарский~\cite{18s}, 
гетероциклы участвуют в строении и во многих жизненно важных процессах 
живой клетки, выполняя биохимические, биологические или физиологические 
функции, например такие гетероциклы, как липополисахариды бактерий, витамины 
и ферменты животных и рас\-те\-ний и~др. 
     
     Особенности гетероциклических углеводородов проявляются на сравнительно 
небольших глубинах земной коры, термодинамика которых обеспечивает 
формирование гетероциклов на основе углеводородов и реализацию их 
реакционной способности. 
     
     Таким образом, информация об углеводородном составе нефти дает 
основание считать, что 90\% углеводородов нефти не имеет прямого отношения к 
биосфере, а 4,5\% могут образоваться в термодинамических условиях земной коры, 
успешно взаимодействовать со всеми элементами биосферы, являться частью ее и 
накапливаться с органическими остатками в осадочных породах. 
     
\section{Информация о~способности нефти к~миграции}

     Вязкость нефти~--- это информация о ее миг\-ра\-ци\-он\-ных возможностях. 
Относительная (удельная) вязкость, выражающая отношение абсолютной 
(динамической) вязкости к вязкости воды,\linebreak зависит от температуры. Для различной 
нефти при температуре более 50~$^\circ$C она превосходит вязкость воды 
минимум в 1,5~раза. Эта физическая характеристика вместе с фильтрацией~--- 
показателем характера движения флюида через породу и свойством самой породы 
пропускать флюид, называемым проницаемостью,~--- определяет ее 
миг\-ра\-ци\-он\-ную способность в пористой среде пластов осадочных горных пород. 
     
     Нефть всегда эпигенетична и может попасть в залежь только в результате 
миграции: горизонтальной, вертикальной или их комбинации, которые зависят от 
характеристик флюида и породы.
     
     Горизонтальная миграция~--- это движение нефти по порам и пустотам 
осадочных пород различного размера и формы. При любой их комбинации 
перемещение флюида может осуществляться только при наличии градиента 
давления, а поскольку в горизонтально залегающих толщах он практически равен 
нулю, то в таких толщах флюид как бы стоит, даже если он находится под 
огромным геостатическим давлением. Лейбензон~\cite{12s} установил, что 
даже при значительной разнице начальных и конечных давлений в идеальном 
грунте происходит значительное падение давления мигрирующего флюида за счет 
сил адсорбции частицами пористой среды. В~реальной среде потери на адсорбцию 
при горизонтальной миграции превышают возможности даже самых 
<<продуктивных>> нефтематеринских свит. Многие исследователи считают, что 
нефть не способна к далекой горизонтальной миграции в толщах осадочных пород, 
поэтому она не может служить основным фактором формирования нефтяных 
залежей. 
     
     Для вертикальной миграции газа, жидкости и их смесей в недрах значительно 
больше возможностей. Эта миграция всегда обусловливается градиентом давления 
и температуры от среды с большей энергией к среде с меньшей энергией. 
Вертикальной миграции способствуют ослабленные тектонические зоны, 
которыми являются активные глубинные разломы, границы крупных фрагментов 
земной коры и их частей и каналы, созданные в силу особенностей состава и 
строения литосферы.
     
     Именно вертикальной миграции газообразных углеводородов глубинной 
природы и обязаны метан, газогидраты и клатраты. Валяев~\cite{6s} 
оценивает объем вертикальной миграции глубинных углеводородов (в основном 
метана) в $5\cdot 10^{13}$~г/год или $2{,}5\cdot 10^{16}$~т за 500~млн лет, что 
соответствует времени палеозоя, мезозоя и кайнозоя. Запасы клатратов 
оцениваются в $3\cdot 10^{12}$~т~\cite{17s}, а газогидратов~--- в $3\cdot
10^{12}$~т~\cite{26s}. Эти формы простейшего углеводорода не имеют никакого 
отношения к биосфере, и их количество в земной коре на порядки превосходит 
запасы всех жидких и газообразных углеводородов, содержащихся в рассеянном 
виде в осадочных породах.
     
\section{Информация о~месте образования нефти}

     Информация о нефти~--- в сочетании ее территориального распространения и 
постоянного состава. Нефть в промышленных масштабах или в виде проявлений 
различной интенсивности присутствует в недрах всех континентов, на дне 
прилегающих к ним акваторий мирового океана, в жерлах вулканов, на дне озера 
Байкал~\cite{2s}, в гранитах Скандинавии~\cite{30s}, в породах кристаллического 
фундамента и во многих других местах. К~1996~г.\ было выделено более 
550~нефтегазоносных бассейнов, из которых более~70~--- на шельфах морей и 
океанов. В~226~бассейнах было открыто более 20\,000~нефтяных и нефтегазовых 
месторождений~\cite{15s}, каждое из 50~наиболее крупных из них содержит более 
1~млрд~т~\cite{11s}. Общее представление о распространении нефтеносных 
территорий дают схематические карты, приведенные во многих работах. 
     
     Повсеместное распространение нефти дополняется постоянством ее среднего 
углеводородного состава, независимо от географического места и глубины 
залегания. Нефть может быть немного легче, немного тяжелее, содержать 0,5\% 
или 2\% серы, но это повсюду нефть определенного углеводородного состава. 
В~этом проявляется инвариантность и уникальность нефти как сложной 
системы~\cite{21s}.
     
     Повсеместность распространения и постоянство состава говорят только об 
одном~--- едином, глобальном источнике и едином природном механизме 
генерации углеводородных последовательностей. Неорганическая гипотеза легко 
объясняет эту особенность нефти, поскольку глубины, на которых протекают 
процессы образования углерода и водорода как элементов, и глубины литосферы, 
на которых формируются углеводородные последовательности, имеют 
необходимую энергетику и катализаторы, и эта сфера охватывает весь земной шар. 
     
     Органическая гипотеза в объяснении этого феномена вынуждена оперировать 
осадочными бассейнами прошлых эпох как областями аккумуляции органических 
веществ в виде остатков бактерий, животных и растений. В~геологической 
ретроспективе области осадконакопления были развиты не повсеместно, 
разновозрастные бассейны не совпадали в пространстве, а условия, в которые затем 
попадали осадки, отложенные в любом бассейне и сформированные из них 
комплексы осадочных пород, в том числе и с остатками органических веществ, 
очень сильно различались и нередко целиком размывались~\cite{19s}. Поэтому 
геологический фактор, в принципе весьма динамичный и очень разный в 
проявлениях на платформенных и геосинклинальных областях, не объясняет 
консерватизма состава нефти и повсеместности ее распространения, в том числе и 
вне пределов развития осадочных пород. 
     
\section{Информация о~возрасте нефти}

     Информация о нефти~--- в комплексах осадочных и магматических пород, от 
кристаллического фундамента, с возрастом более 1,5~млрд лет, до третичного, не 
старше 86~млн лет. Этот факт важен для понимания генезиса нефти. 
     
     В органической гипотезе возраст нефти и вмещающей породы может быть 
одинаков. Следовательно, нефть может быть докембрийской, напри\-мер вендской 
(старше 650~млн лет), палеозойской, например девонской (не моложе 364~млн 
лет), или третичной, например майкопской (не моложе 12~млн лет). Признание 
факта накопления материнского вещества нефти в докембрии или палеозое 
неизбежно привязывает исходное вещество к биосфере прошлых периодов. 
Например, биосфера венда была представлена бактериями и многоклеточными, 
беспозвоночными, безраковинными животными типа медуз и слизняков, 
относимой к так называемой эдиакарской фауне~\cite{14s}. Сохранение их 
остатков в морском осадке, равно как и органических остатков бактерий, а тем 
более формирование из них керогена, проблематично, хотя бы потому, что эти 
остатки были частью замкнутых трофических экосистем.
     
     Образование нефти сотни миллионов лет тому назад неизбежно означает 
стабильный, неизменный элементный и углеводородный ее состав, сохраняющийся 
сотни миллионов лет. Это противоречит закону информатики о постоянном развитии и 
изменении сложных систем и закону о самопроизвольном возрастании энтропии 
или беспорядка любой сложной системы. Неоднократное изменение геологической 
структуры, которое испытывала любая нефтеносная территория, также изменяло 
нефть. 
     
     Неорганическая гипотеза в ее существующих вариантах не рассматривает 
зависимость между образованием нефти и ее возрастом, концентрируясь на 
физической, химической и термодинамической сторонах процесса образования 
нефти.
     
\section{Информация об~исходном веществе и~генезисе нефти}

     Нефть несет информацию о себе тем, что молекулы углеводородов состоят из 
атомов углерода и водорода. Поэтому в связи с генезисом нефти, состоящей в 
основном из этих двух элементов, вопрос~--- откуда берутся сами атомы, в силу 
каких физических процессов и химических реакций они соединяются в молекулы 
углеводородных последовательностей, вполне закономерен. В~такой постановке 
проблема происхождения нефти никогда не ставилась. Косвенно 
Жармен~\cite{33s}, анализируя каталитические превращения систем С--Н на 
атомном и молекулярном уровне, практически впервые доказал возможность 
превращения простейших парафинов в олефины и ароматические углеводороды и 
наоборот при температурах, не превышающих $+1150$~$^\circ$C, в присутствии 
катализаторов, т.\,е.\ в условиях земной коры.
     
     Происхождение атомов углерода и водорода как первоосновы нефти~--- это 
один из ключевых вопросов в проблеме ее образования.
     
     Схема появления атомного, а затем формирования молекулярного состояния 
вещества рас\-смот\-ре\-на в публикации Фомина~\cite{25s}. В~этой проблеме 
он опирался на результаты исследований Штерн\-хай\-ме\-ра о 
     физико-химических характеристиках астеносферы и представлениях 
Тхоровской~\cite{23s} и Капустинского~\cite{10s} о появлении 
атомов как факте саморазвития материи Земли. Для идей о развитии материи 
важным явилось представление Амбарцумяна~\cite{1s}, основанное на 
астрофизических наблюдениях <<\ldots развитие материи идет от простого к 
сложному, от более плотного к менее плотному состоянию>>. В~рас\-смат\-ри\-ва\-емом 
контексте плотным является ультрасжатое под давлением 50~ГПа (50~тыс.\ атм) 
гомогенное вещество внутренних час\-тей Земли, находящееся ниже уровня 400~км, 
где проходит верхняя астеносфера. При таком давлении, как считает 
Фомин, вещество не может быть горячим, а атом не может сохранить свою 
ядерно-орби\-таль\-ную конфигурацию.
     
     Переход аномального состояния вещества в <<нормальное>>, как называет 
эти состояния Фомин, сопровождается выделением огромной тепловой 
энергии, расплавлением вещества мантии и его декомпрессией. Дегазация Земли, в 
том числе и углеводородная, появление рас\-плав\-лен\-ной магмы и появление 
<<нормальных>> атомов~--- это и есть последствия декомпрессии. 
     
     С этого рубежа атомы проявляют способность формировать химические 
связи. Появление первых соединений углерода с водородом связано со свойствами 
атома углерода, обладающего уникальной способностью реализовать свои 
валентные возможности. Так возникают первые, простейшие, газообразные 
углеводородные последовательности: СН$_4$--метан, С$_2$Н$_6$--этан и 
С$_2$Н$_2$--ацетилен. 
     
Жермен~\cite{33s} показал, что система углерод--во\-до\-род в присутствии 
катализаторов испытывает превращения, определяемые заданным соотношением 
температуры и давления. Углеводород парафинового ряда при высокой 
температуре может крекироваться до смеси парафинов и олефинов, 
дегидрогенизироваться до олефинов с тем же чис\-лом атомов углерода, 
дегидроциклизироваться до ароматики либо разложиться на изначальный углерод 
и водород. В литосфере усложняется состав и структура системы С--Н с 
формированием углеводородных последовательностей, и эти изменения 
определяются термодинамикой и катализаторами. 
     
     Важно, что все процессы и превращения углеводородных 
последовательностей происходят не локально, не в изолированных 
разновозрастных линзах и толщах осадочных пород, а повсеместно,\linebreak охватывая 
определенную геосферу Земли, начиная с верхней астеносферы, где зарождаются 
атомы, до зоны нефтенакопления, отражая общую эволюцию развития вещества 
планеты. Локализовать точное место образования нефти нельзя, ее образование 
начинается с момента появления атомов углерода и водорода и заканчивается в 
залежи. Там же начинается процесс разрушения нефти. Нефть как сложная система 
не развивается по-другому.
     
     Для сравнения гипотез происхождения и эволюции нефти необходим 
одинаковый критерий, позволяющий оценить динамическое состояние сис\-те\-мы на 
отдельных этапах ее развития согласно существующим точкам зрения и сравнить 
динамику этих состояний. Две гипотезы происхождения нефти такого единого, 
общепризнанного критерия не используют, но такой критерий существует, и 
     это~--- информационное содержание. Информация содержится в атомах, 
составляющих молекулы исходного и промежуточного вещества нефти, а оценка 
объема информации в атомах не зависит от точек зрения на природу этих веществ.
     
     Великие физики, включая Эйнштейна~\cite{29s}, Шеннона~\cite{27s}, 
Бриллюена~\cite{5s}, доказали связь основных категорий мироздания с 
информацией. Поэтому использование информации для оценки состояния системы 
и ее динамики во времени и пространстве отражает и основывается на связи между 
информацией и энергией, информацией и энтропией, информацией и гравитацией, 
ин\-форма\-ци\-ей и массой~\cite{7s}. Рассматривая информационные взаимодействия в 
системах неживой и живой природы~\cite{20s}, мы основывались на этих 
положениях, считая, что информация~--- универсальная категория, позволяющая 
выразить количество и качество материи, энергии, пространства, времени и 
движения, вовлеченных в любой процесс или явление.
     
     Использование информации позволяет измерить первооснову системы~--- 
атомы и вещество, ими составленное, и проследить динамику изменения 
информационного содержания вещества на различных этапах его преобразования. 
     
     Изменение информационной характеристики сложных систем и материи в 
целом согласуется и с общими закономерностями развития, уста\-нов\-лен\-ны\-ми 
Амбарцумяном~--- развитие материи идет от простого к сложному, от более 
плотного к менее плотному состоянию. Процесс перехода от простого состояния к 
более сложному~--- это процесс увеличения порядка, информации и, 
соответственно, уменьшения неопределенности и энтропии. Развитие материи от 
плотного к менее плотному состоянию трудно выразить в процессах образования 
нефти. Однако можно считать, что источник исходного вещества находится в более 
плотном состоянии, нежели горные породы земной коры и тем более рыхлый 
морской осадок, в котором накапливается органическое вещество.
     
     В общем случае, если процесс формирования конечной сложной системы 
идет с поглощением тепла, то этот процесс происходит с возрастанием ее 
энтропии, увеличением неопределенности и уменьшением информации в системе. 
Эндотермический процесс катагенеза керогена как основа\linebreak
образования нефти~--- 
это и есть процесс увеличения энтропии системы. Однако процесс образования 
нефти как сложной системы углеводородных последовательностей означает 
уменьшение\linebreak
 энтропии, увеличение сложности и, соответственно, увеличение 
объема информации. Поэтому образование нефти с термодинамической точки 
зрения есть процесс трансформации исходного вещества, находящегося в нагретом 
состоянии, в вещество более холодное, от материи, находящейся в более плотном 
состоянии, к веществу менее плотному. Используя объем информации как 
критерий, можно проследить, как она изменяется по всей цепочке трансформации 
от исходного вещества до нефти и битума как для органической, так и для 
неорганической гипотезы.
     Для расчетов объема информации используем данные об объеме информации 
в атомах элементов таблицы Менделеева, приведенные в работе~\cite{8s}. 
Значения объема информации в атомах пяти элементов, составляющих 99,9\% 
нефти, равны: 
     С~--- 109,642; Н~--- 10,422; S~--- 317,504; N~--- 138,908 и О~--- 149,33~бит.
     
     Объем информации в единице массы вещества, например нефти или керогена, 
складывается из информации атомов каждого элемента в молекулах, составляющих 
единицу массы. При этом используется брутто-формула вещества либо 
эмпирическая формула, по которой и рассчитывается количество информации в 
атомах. Помимо информации в атомах учитывается объем информации в 
структуре молекулы. В расчетах Гуревича~\cite{8s} он основан на числе 
валентных связей атомов, составляющих структуру молекулы. 
     
     Объем информации, содержащейся в нефти, может быть рассчитан как сумма 
каждой из трех основных ее частей: углеводородов, гетероциклов и примесей~--- либо 
по эмпирической формуле, со\-став\-лен\-ной по процентному содержанию слагающих 
ее основных элементов. 
     
     Углеводородный состав нефти, составляющий 90\% ее массы, включает три 
группы углеводородов: парафины (30\%--35\%), нафтены (25\%--75\%) и 
ароматические (10\%--15\%)~\cite{16s}. Расчет по брутто-фор\-му\-лам для типичных 
жидких представителей этих групп: пентана С$_5$Н$_{10}$, циклогексана 
С$_6$Н$_{12}$ и бензола С$_6$Н$_6$~--- и для трех азотсодержащих гетероциклов: 
нейтральных, кислых и основных аминокислот~--- определяет объем информации в 
условной молекуле нефти, равный \textbf{6171}~бит.
     
     Объем информации для легкой и тяжелой нефти рассчитан также и по 
эмпирической формуле. Учитывалось, что в легкой нефти на 2\% меньше С и 
соответственно больше Н и О, а в тяжелой~--- на 2\% больше С и соответственно 
меньше Н и~О. В~качестве примера приведем расчет количества информации, 
содержащейся в условной молекуле легкой нефти.
     
     Элементный состав легкой нефти в среднем таков: С~--- 83\%, Н~--- 14\%, 
О~--- 2\%, S~--- 0,9\% и N~--- 0,1\%. Количество атомов каждого элемента, исходя из его 
процентного содержания и массы, равно: 
C~--- 83/12\;=\;6,9; H~--- 14; O~--- 2/16\;=\;0,13;
S~--- 0{,}9/32\;=\;0,03; N~--- 0,1/14\;=\;0,01.
     Общее число атомов~--- 6,9\;+\;14\;+\;0,13\;+\;0,03\;+\linebreak +\;0,01\;=\;21,07 атомных 
единиц массы, а число атомов в эмпирической формуле (условной молекуле) 
легкой нефти равно:
     C~--- 32,7; H~--- 66,4; O~--- 0,6; S~--- 0,1; N~--- 0,1, и эмпирическая формула 
имеет вид C$_{32}$H$_{66}$OSN.
     
     Соответственно, объем информации составит:
     C~--- 3507, H~--- 686; O~--- 149; N~--- 138,9; S~--- 317,5. В сумме это 
4798~бит плюс 64~бита на структуру условной молекулы, итого \textbf{4862}~бит. 
     
     Элементный состав тяжелой нефти: С~--- 86\%, Н~--- 12\%, О~--- 0\%, S~--- 0,9\% 
и N~--- 0,1\%.
     
     Аналогичный расчет дает ее эмпирическую формулу C$_{37}$H$_{62}$SN. 
Объем информации в условной молекуле тяжелой нефти равен \textbf{5228}~бит.
     
     Таким образом, разные подсчеты показали небольшую разницу в объеме 
информации в условной молекуле нефти.
     
\section{Информация об~эволюции нефти в~земной коре}

     Первыми углеводородными последователь\-ностями, образующимися на 
начальных этапах генезиса нефти, являются: СН$_4$~--- метан, С$_2$Н$_6$~--- этан и 
С$_2$Н$_2$~--- ацетилен. Самый простой и\linebreak стойкий из них~--- метан. Информация, 
содержащаяся в его молекуле, принята за исходную для всех последующих 
углеводородных последовательностей, а поскольку молекула метана состоит из 
одного атома углерода и четырех атомов водорода, то это дает \textbf{156}~бит, 
включая объем информации, содержащейся в структуре молекулы метана. 
     

     Эволюция углеводородных последовательностей от простейших, 
образующихся в астеносфере, до аккумуляции самой полной 
     по\-сле\-до\-ва\-тель\-ности~--- нефти в зоне нефтенакопления есть\linebreak процесс 
усложнения состава и структуры последовательностей. Существует уровень, ниже 
которого углеводороды могут существовать только в газообразном состоянии, а 
выше~--- в газообразном, жидком и твердом. Газообразным углеводородом этого 
уровня условно принят бутан~--- С$_4$Н$_{10}$, молекула которого состоит из 
четырех атомов углерода и десяти атомов водорода, что соответствует 
\textbf{547}~битам информации.
     

     Конечной стадией эволюции углеводородных последовательностей нефти в 
верхней части земной коры являются битумы битуминозных пород.\linebreak Би\-тумы~--- это 
асфальтово-смолистые вещества, наиболее тяжелые компоненты нефти с наиболее 
сложной структурой молекул. Их усредненный элементный состав, приведенный 
в~\cite{16s}, таков: С~--- 84\%, Н~--- 8\%, S~--- 3\%, O~--- 4\% и N~--- 1\%. 
Соответственно, в условной единице массы битума содержится: С~--- 6,91; Н~--- 8; 
S~--- 0,09; N~--- 0,07; O~--- 0,25 и в сумме 15,41~атомов, а в атомных процентах: 
     С~--- 45,4; Н~--- 51,9; S~--- 0,6; N~--- 0,5; O~--- 1,6. Эмпирическая формула 
битума~--- C$_{45}$H$_{51}$OSN, а объем информации в условной единице массы 
битума составляет \textbf{6365}~бит. 
     
     Таким образом, динамика изменения информации в единице условной массы 
от начальных стадий образования углеводородных последовательностей до 
конечной в виде их совокупности в нефти и, наконец, в битуме битуминозной 
породы по неорганической схеме выгладит следующим образом:
     \begin{itemize}
\item на уровне образования первого простейшего углеводорода~--- \textbf{156}~бит; 
\item на уровне существования газообразных углеводородов~--- \textbf{547}~бит;
\item на уровне главной зоны нефтенакопления (1500--3500~м) полная 
совокупность углеводородных последовательностей (нефть)~--- 
\textbf{6171}~бит; 
\item верхний уровень, близкий к поверхности Земли (нефть, лишенная легких 
компонентов,~--- битум битуминозной породы)~--- \textbf{6365}~бит. 
     \end{itemize}
     
     Динамика изменения веществ от исходного до конечного в органической 
гипотезе выглядит следующим образом. Одной из наиболее полных работ об 
образовании нефти по этой схеме является монография~\cite{34s}. Исходя из 
химического состава биомассы, авторы определили основные компоненты жиров и 
масел, встречающихся в организмах, которые откладываются как органическое 
вещество осадочной породы. Для расчета объема информации условной единицы 
исходного вещества выбраны: триглицерид~--- С$_{18}$Н$_{20}$О$_6$, стеариновая 
С$_{18}$Н$_{36}$О$_2$ и линоленовая С$_{18}$Н$_{30}$О$_2$ 
аминокислоты~\cite{34s, 13s}. Объем их информации соответственно равен 3100, 
2670 и 2590, а условная единица массы исходного вещества для образования нефти 
содержит \textbf{8360}~бит информации. 
     
     В монографии~\cite{4s} приводится эмпирическая формула керогена, 
выведенная из соотношения масс его элементов,~--- С$_{55}$Н$_{40}$NSO$_3$. 
Соответственно, объем информации в условной молекуле керогена равен 
\textbf{7369}~бит.
     
     Таким образом, динамика изменения объема информации от начальных 
стадий накопления остатков липидов, созревания керогена в нефтематеринской 
породе до конечной стадии~--- нефти и битума битуминозной породы~--- по 
органической схеме выглядит следующим образом: 
     \begin{itemize}
\item на уровне биосферы исходное вещество (липид)~--- \textbf{8360}~бит; 
\item на уровнях созревания керогена (глубины 5--15~км)~--- \textbf{7369}~ бит; 
\item на уровне главной зоны нефтенакопления (1500--3500~м) как полная 
совокупность углеводородных последовательностей (нефть)~--- 
\textbf{6171}~бит; 
\item на верхнем уровне, близком к поверхности Земли (битум битуминозной 
породы)~--- \textbf{6365}~бит. 
     \end{itemize}
     
     Таким образом, для сравнения схем генезиса нефти имеется два ряда 
последовательных значений объема информации, содержащейся в исходном и 
конечном веществе, отражающих две различные схемы. Подобная оценка более 
чем нетривиальна. Она произведена на основе современных пред\-став\-ле\-ний о роли 
информации в строении и развитии материи и базируется на массе и заряде атомов 
или структуре молекулы веществ, определяемых на базе общепринятых 
химических и физических констант, что позволяет сравнить результаты, 
полученные для разных схем процесса генезиса, но по одному стандарту 
измерения. 
     
     Полученные показатели объема информации не привязаны к геологическому 
времени, к глубине или месту с заданными физико-географическими, 
геологическими и геохимическими параметрами, которые по отношению к нефти и 
к значениям отдельных стадий преобразования исходного вещества являются 
переменными величинами и допускают различные интерпретации. 
     
     Последовательности объемов информации позволяют сделать следующие 
выводы.
     
     Абиогенная схема генезиса от образования первых углеводородных 
последовательностей и дальнейшего усложнения их состава и структуры в\linebreak 
литосфере и в земной коре подкрепляется соответствующим изменением объема 
информации. Ее увеличение соответствует усложнению структуры и состава 
углеводородных последовательностей от простейших вплоть до нефти и битума во 
всем вертикальном диапазоне их распространения.
     
     Органическая гипотеза по последовательности изменения объема 
информации в исходном, промежуточном и конечном веществе не соответствует 
законам развития материи, требует ответа на некоторые вопросы. Схема 
объективно отражает последовательное снижение объема информации от 
исходного вещества до нефти и затем некоторое его увеличение до стадии битума, 
что трактует синтез нефти из готовых органических материалов как образование 
сложной системы, т.\,е.\ процесс, протекающий от сложного к простому. В~этом 
процессе вещество от менее разогретого состояния, находящегося под меньшим 
давлением, переходит в более разогретое состояние под большим давлением. 
     
     Однако с точки зрения кинетики процесс перестройки структуры исходного 
вещества, его декомпозиция, разрушение существующих химических связей и 
создание нового вещества, в котором существенно увеличивается концентрация 
углерода, не может быть трансформацией от сложного к прос\-тому. 
     
     С геохимической точки зрения кероген есть нерастворимая, 
дебитуминизированная часть ор\-га\-ниче\-ского вещества осадочных пород или 
остаток\linebreak существовавшего ранее. Кероген, нефтеродный потенциал которого 
определяют в настоящее время, а породы, его содержащие, называют 
нефтематеринскими,~--- фактически это породы, бывшие нефтематеринскими и 
уже реализовавшие этот потенциал. Содержащееся в нем органическое вещество и 
объем его информации есть не отражение этого потенциала, а характеристики 
метаморфизованных остатков липидов, смешанных с углеводородными 
последовательностями глубинного про\-ис\-хож\-де\-ния, отложившимися вместе с 
осадком. 
     
     Битум~--- это смолисто-асфальтовые компоненты нефти, состоящие из 
сложных по структуре молекул, а нефть по ее углеводородному составу и 
соотношению углерод--водород не вписывается в схему между исходным и 
конечным состоянием. 
     
     Следовательно, природный процесс: 
    \begin{multline*}
     \mbox{липид}\;\rightarrow\;\mbox{осадок обогащенный}\; 
\mathrm{С}_{\mathrm{орг}}\;\rightarrow\\
\rightarrow\;\mbox{кероген}\;\rightarrow\;\mbox{микронефть}
\;\rightarrow\;\mbox{нефть}\;\rightarrow\,\mbox{битум}\hspace*{-4.55pt}
     \end{multline*}
     с точки зрения закономерностей развития материи, изменения энтропии и 
информационного содержания не реализуем. 
     
     Осадочные толщи с повышенным содержанием углерода и углеводородов 
развиты глобально и гораздо шире, чем собственно нефть. Толщи, считавшиеся 
ранее нефтематеринскими, потому что сейчас содержат первые проценты 
углеводородных веществ, на самом деле таковыми не являлись и не являются. 
Содержащиеся в них углеводороды, принимаемые за C$_{\mathrm{орг}}$, в 
основном представляют собой смесь органического вещества биосферной природы 
и углеводородных последовательностей глубинного происхождения, 
аккумулированные в осадках одновременно с осадконакоплением. 
     
     В условиях термодинамики земной коры эти углеродсодержащие толщи не 
обладали нефтегенерирующим потенциалом, что показало изучение кернов пород 
моложе 150~млн лет и содержащегося в них углерода, в том 
числе~C$_{\mathrm{орг}}$. Керны были отобраны по программе глубоководного 
бурения на шельфах и на дне мирового океана. Сошлемся лишь на два 
отчета~\cite{31s, 32s}. 
     
     Индексы нефтегенерационного потенциала, соответ\-ствующие 10~градациям 
катагенеза~\cite{3s}, на самом деле отражают степень метаморфизма 
углеводо\-родных последовательностей, аккумулированных с осадком. Остатки 
углеводородных гетероциклов биосферы, захороненные с ними углеводородные 
последовательности не имеют прямого отношения к нефти, аккумулированной в 
залежах и месторождениях главной зоны нефтенакопления.
     
     
\section{Заключение}
\
     Итак,      
углеводородный и элементный состав нефти показывает абсолютное 
преобладание в ней углерода, не характерного для живого организма биосферы. 
     Образование нефти в процессе перевода в верхней части земной коры 
низкоуглеродистого органического вещества в высокоуглеродистую нефть в 
результате катагененеза керогена в об\-ластях с осадочным чехлом относительно 
небольшой мощности не обеспечено необходимой для этого процесса 
температурой и давлением. Геологическое время не может быть заменой энергии, 
необходимой для катагенеза, поскольку изменения внутренней структуры и состава 
керогена, происходящие в течение требуемого геологического времени (не менее 
10~млн лет), независимо от внешних условий среды носят деструктивный, 
необратимый для керогена характер. 
     
     Углеводородный состав нефти означает, что преобладающая ее часть в виде 
углеводородов (соединений только атомов углерода и водорода) не имеет 
биосферного происхождения.
     
     Нефть в силу своей вязкости и адсорбции час\-ти\-ца\-ми горной породы не 
способна мигрировать по горизонтально залегающим комплексам осадочных 
пород на большие расстояния, поэтому латеральная миграция не может рассматриваться как фактор 
формирования ее залежей.
     
     Глобальное распространение нефти, ее повсеместность в земной коре 
доказывают независимость ее образования от геологических условий недр. 
Образование нефти не привязано к осадочным седиментационным бассейнам. 
Структура, состав, возраст и взаимоотношение вме\-ща\-ющих пород имеют для 
нефти значение только как каналы миграции, создатели условий для аккумуляции, 
завершающей стадии оформления состава нефти в данной геологической формации 
и кратковременной, в геологическом масштабе времени, консервации нефти в 
залежи.
     
     Нахождение нефти в комплексах пород различного состава и любого возраста 
не доказывает, что они <<ровесники>>. Докембрийской, палеозойской и 
мезозойской нефти в недрах быть не может, а есть залежи нефти, близкой по 
элементному и углеводородному составу, в фундаменте докембрийских, 
палеозойских, мезозойских и кайнозойских отложений. Нефть в любых породах 
как сложная сис\-те\-ма может существовать в недрах в первозданном виде короткое в 
геологическом масштабе время.
     
     Определение объема информации в нефти и веществах, с ней генетически 
связанных, может явиться вполне современным инструментом, дополняющим 
существующие методы изучения нефти и ее генезиса.

{\small\frenchspacing
{%\baselineskip=10.8pt
\addcontentsline{toc}{section}{Литература}
\begin{thebibliography}{99}     

\bibitem{6s} %1
\Au{Валяев Б.\,М.}
Углеводородная дегазация Земли и генезис нефтегазовых месторождений~// 
Геология нефти и газа, 1997. №\,9.

\bibitem{9s} %2
\Au{Дмитриевский А.\,Н., Валяев Б.\,М.}
Углеводородная дегазация через дно океана: локализация, проявления, масштабы, 
значимость~// Дегазация Земли и генезис углеводородных флюидов и 
месторождений.~--- М.: ГЕОС, 2002. С.~7--36.

\bibitem{26s} %4 %3
\Au{Фрадкин В.}
Газ на дне океана как альтернатива энергоносителя. 2004. {\sf http://n-t.ru/tp/ie/gn.htm}.

\bibitem{17s} %3 %4
Новый вид ископаемого топлива, использующийся только в России~// Наука. 
Известия, 2009. {\sf  http:// www.inauka.ru/news/article93252html}.

\bibitem{21s} %5
\Au{Сейфуль-Мулюков Р.\,Б.}
Нефть в квантовом мире~// Системы и средства информатики. Доп. вып.~--- М.: 
ИПИ РАН, 2008. С.~195--213.

\bibitem{16s} %8 %6
Нефть. {\sf http://ru.wikipedia.org/wiki/нефть}.

%\bibitem{28s} %6
%Элементный состав нефти и гетероатомные компоненты. {\sf  http://ru.wikipedia.org}.

\bibitem{19s} %7
\Au{Сейфуль-Мулюков Р.\,Б.}
Палеотектонические факторы нефтеобразования и нефтенакопления.~--- М.: Недра, 
1983.  269~с.

\bibitem{34s} %9 %8
\Au{Tissot B.\,P., Velte~D.\,H.}
Petroleum formation and occurrence. A new approach to oil and gas exploration.~--- 
Berlin: Springer-Verlag, 1978. 
     
\bibitem{4s} %10 %9
\Au{Богородская Л.\,И. Конторович А.\,Э., Ларичева~А.\,И.}
Кероген: методы изучения, геохимическая интерпретация.~--- Новосибирск: Изд-во 
СО РАН, 2005.  254~с.

\bibitem{18s} %11 %10
\Au{Пожарский А.\,Ф.}
Гетероциклические соединения в биологии и медицине~// Статьи Соросовского 
Образовательного журнала, 1996. {\sf  
http://www.pereplet.ru/ obrazovanie/stsoros/112.html}.

\bibitem{12s} %12 %11
\Au{Лейбензон Л.\,С.}
Движение природных жидкостей и газов в пористой среде.~--- М.: 
ОГИЗ--Гостехиздат, 1947.  244~с.

\bibitem{2s} %13 %12
Байкал открывает свои тайны: на дне озера обнару\-жили источник нефти. 2008. {\sf  
http://www.rian.ru/\linebreak elements/20080807/150155919.html}.

\bibitem{30s} %14 %13
\Au{Laherrere J.}
No free lunch, Part~1: A critique of Thomas Gold's claims for abiotic oil.~--- The Wilderness Publications, 2004.
 {\sf   http://www.copvcia.com/ free/ww3/102104\_no\_free\_pt1.shtml}.

\bibitem{15s} %15 %14
Нефтегазоносные бассейны мира. Карта.~--- СПб.: ВСЕГЕИ, 1995. 

\bibitem{11s} %16 %15
Крупнейшие нефтяные месторождения мира. {\sf  http:// ru.wikipedia.org/wiki/крупнейшие\_нефтяные\_место рождения\_мира}.

\bibitem{14s} %17 %16
\Au{Малаховская Я.\,Е., Иванцов А.\,Ю.}
Вендские жители Земли.~--- Архангельск: ПИН РАН, 2003.  48~с.

\bibitem{33s} %18 %17
\Au{Germain J.\,E.}
Catalytic conversion of hydrocarbons.~--- London--New York: Academic Press, 1969.

\bibitem{25s} %19 %18
\Au{Фомин Ю.\,М.}
Верхняя астеносфера~--- переходная зона между веществом мантии и литосферы. {\sf 
http://www.evolbiol.ru/fomin.htm}.

\bibitem{23s} %20 %19
\Au{Тхоровская Н.\,В.}
Аномалия Земли~// Материалы международной конференции памяти акад.\ 
П.\,Н.~Кропоткина, 20--24~мая 2002.~--- М.: ГЕОС, 2002. С.~454--455.

\bibitem{10s} %21 %20
\Au{Капустинский А.\,Ф.}
Геосферы и химические свойства атомов~// Геохимия, 1956. №\,1.  С.~53--61.

\bibitem{1s}  %22 %21
\Au{Амбарцумян В.\,А.}
Научные труды. Т.~2.~--- Ереван: Изд-во АН Армянской ССР, 1960.

\bibitem{29s} %23 %22
\Au{Эйнштейн А.}
Основы общей теории относитель\-ности~// Собр. науч. тр. Т.~1.~--- М.: 
Наука, 1965.

\bibitem{27s} %24 %23
\Au{Шеннон К.}
Математическая теория связи~// Работы по теории информации и кибернетике.~--- 
М.: Изд-во иностранной литературы, 1963.


\bibitem{5s} %25 %24
\Au{Бриллюен Л.}
Наука и теория информации.~--- М.: Физматгиз, 1960.


\bibitem{7s} %26 %25
\Au{Гуревич И.\,М.}
Законы информатики~--- основа строения и познания сложных систем.~--- М.: 
ТОРУС ПРЕСС, 2007.  400~с.

\bibitem{20s} %27 %26
\Au{Сейфуль-Мулюков Р.\,Б.}
Информация и информационные процессы в системах неживой и живой природы~// 
Системы и средства информатики. Спец. вып.~--- М.: ИПИ РАН, 2007. С.~140--156.

\bibitem{8s} %28 %27
\Au{Гуревич И.\,М.} 
Информационные характеристики физических систем.~--- М.: Изд-во 11~формат, 2009. 167~с.


\bibitem{13s} %29 %28
Липиды. {\sf http://lipid.narod.ru/fa.html}.


\bibitem{31s} %30 %29
\Au{McIver R.\,D.}
Hydrocarbons in canned mud from sites 185, 186, 189, and 191~--- Leg~19~//
Initial Reports of the Deep Sea Drilling Project, 1973. Vol.~19. P.~875--877. 
{\sf http:// www.deepseadrilling.org/19/volume/dsdp19\_34.pdf}. 

\bibitem{32s} %31 %30
\Au{Simoneit B.\,R.\,T.}
Organic geochemistry of the shales from the Northwestern Proto-Atlantic, DSDP Leg~43~//
Initial Reports of the Deep Sea Drilling Project, 1979. Vol.~43. P.~643--650. 
{\sf http://www.deepseadrilling.org/43/ volume/dsdp43\_25.pdf}.

\label{end\stat}

\bibitem{3s} %32
\Au{Баженова Т.\,К., Шиманский В.\,К.}
Исследование онтогенеза углеводородных систем как основа раздельного прогноза 
нефте- и газонасыщенности осадочных бассейнов~// Нефтегазовая геология. Теория 
и практика, 2007. №\,2. {\sf http://www.ngtp.ru}.


%\bibitem{22s}
%Справочник по химическому составу и технологическим свойствам водорослей, 
%беспозвоночных и морских млекопитающих~/ Под ред. В.\,П.~Быкова.~--- М.: 
%ВНИРО, 1999.


%\bibitem{24s}
%Физико-химические свойства нефти. {\sf http://\linebreak forum.socionic.ru}.
 \end{thebibliography}}}%
\end{multicols}
        %+6

   { %\Large  
   { %\baselineskip=16.6pt
   
   \vspace*{-48pt}
   \begin{center}\LARGE
   \textit{Предисловие}
   \end{center}
   
   %\vspace*{2.5mm}
   
   \vspace*{25mm}
   
   \thispagestyle{empty}
   
   { %\small 

    
Вниманию читателей журнала <<Информатика и её применения>> предлагается 
очередной тематический выпуск <<Вероятностно-статистические методы и 
задачи информатики и информационных технологий>>. Предыдущие тематические 
выпуски журнала по данному направлению вышли в 2008~г.\ (т.~2, вып.~2), 
в 2009~г.\ (т.~3, вып.~3) и в 2010~г.\ (т.~4, вып.~2). 

Статьи, собранные в данном журнале, посвящены разработке новых вероятностно-статистических 
методов, ориентированных на применение к решению конкретных задач информатики и информационных 
технологий, а также~--- в ряде случаев~--- и других прикладных задач. Проблематика, охватываемая 
публикуемыми работами, развивается в рамках научного сотрудничества между Институтом проблем 
информатики Российской академии наук (ИПИ РАН) и Факультетом вычислительной математики и 
кибернетики Московского государственного университета им.\ М.\,В.~Ломоносова в ходе работ 
над совместными научными проектами (в том числе в рамках функционирования 
Научно-образовательного центра <<Вероятностно-статистические методы анализа рисков>>). 
Многие из авторов статей, включенных в данный номер журнала, являются активными участниками 
традиционного международного семинара по проблемам устойчивости стохастических моделей, 
руководимого В.\,М.~Золотаревым и В.\,Ю.~Королевым; регулярные сессии этого семинара 
проводятся под эгидой МГУ и ИПИ РАН (в 2011~г.\ указанный семинар проводится в октябре 
в Калининградской области РФ). 

Наряду с представителями ИПИ РАН и МГУ в число авторов данного выпуска журнала входят 
ученые из Научно-исследовательского института системных исследований РАН, Института 
проблем технологии микроэлектроники и особочистых материалов РАН, Института 
прикладных математических исследований Карельского НЦ РАН, Московского 
авиационного института, Вологодского государственного педагогического университета, 
НИИММ им.\ Н.\,Г.~Чеботарева, Казанского государственного университета, Дебреценского 
университета (Венгрия).

Несколько статей выпуска посвящено разработке и применению стохастических методов и 
информационных технологий для решения различных прикладных задач. В~работе В.\,Г.~Ушакова 
и О.\,В.~Шестакова рассмотрена задача определения вероятностных характеристик случайных 
функций по распределениям интегральных преобразований, возникающих в задачах эмиссионной 
томографии. В~статье Д.\,О.~Яковенко и М.\,А.~Целищева рассмотрены некоторые вопросы 
математической теории риска и предложен новый подход к диверсификации инвестиционных 
портфелей. Работа И.\,А.~Кудрявцевой и А.\,В.~Пантелеева посвящена построению и 
исследованию математической модели, описывающей динамику сильноионизованной плазмы. 
В~статье П.\,П.~Кольцова изучается качество работы ряда алгоритмов сегментации изображений. 
Статья А.\,Н.~Чупрунова и И.~Фазекаша посвящена вероятностному анализу числа без\-оши\-бочных 
блоков при помехоустойчивом кодировании; получены усиленные законы больших чисел для указанных 
величин.

В данном выпуске традиционно присутствует тематика, весьма активно разрабатываемая в течение 
многих лет специалистами ИПИ РАН и МГУ,~--- методы моделирования и управления для 
информационно-телекоммуникационных и вычислительных систем, в частности методы 
теории массового обслуживания. В~статье А.\,И.~Зейфмана с соавторами рассматриваются 
модели обслуживания, описываемые марковскими цепями с непрерывным временем в случае 
наличия катастроф. В~работе М.\,М.~Лери и И.\,А.~Чеплюковой рассматриваются случайные 
графы Интернет-типа, т.\,е.\ графы, степени вершин которых имеют степенные распределения; 
такие задачи находят применение при исследовании глобальных сетей передачи данных. 
Работа Р.\,В.~Разумчика посвящена исследованию систем массового обслуживания специального 
вида~--- с отрицательными заявками и хранением вытесненных заявок.

Ряд статей посвящен развитию перспективных теоретических 
вероятностно-статистических методов, которые находят широкое применение в различных 
задачах информатики и информационных технологий. В~работе В.\,Е.~Бенинга, А.\,К.~Горшенина 
и В.\,Ю.~Королева рассмотрена задача статистической проверки гипотез о числе компонент 
смеси вероятностных распределений, приводится конструкция асимптотически наиболее мощного 
критерия. Результаты этой работы найдут применение в ряде прикладных задач, использующих 
математическую модель смеси вероятностных распределений (в информатике, моделировании 
финансовых рынков, физике турбулентной плазмы и~т.\,д.). В~статье В.\,Ю.~Королева, 
И.\,Г.~Шевцовой и С.\,Я.~Шоргина строится новая, улучшенная оценка точности нормальной 
аппроксимации для пуассоновских случайных сумм; как известно, указанные случайные суммы 
широко используются в качестве моделей многих реальных объектов, в том числе в информатике, 
физике и других прикладных областях. Работа В.\,Г.~Ушакова и Н.\,Г.~Ушакова посвящена 
исследованию ядерной оценки плотности распределения; эти результаты могут применяться, 
в част\-ности, при анализе трафика в телекоммуникационных системах. Серьезные приложения 
в статистике могут получить результаты работы О.\,В.~Шестакова, в которой доказаны оценки 
скорости сходимости распределения выборочного абсолютного медианного отклонения к нормальному 
закону. 

\smallskip

Редакционная коллегия журнала выражает надежду, что данный тематический  выпуск 
будет интересен специалистам в области теории вероятностей и математической статистики 
и их применения к решению задач информатики и информационных технологий.
     
     %\vfill 
     \vspace*{20mm}
     \noindent
     Заместитель главного редактора журнала <<Информатика и её 
применения>>,\\
     директор ИПИ РАН, академик  \hfill
     \textit{И.\,А.~Соколов}\\
     
     \noindent
     Редактор-составитель тематического выпуска,\\
     профессор кафедры математической статистики факультета\\
      вычислительной математики и кибернетики МГУ им.\ М.\,В.~Ломоносова,\\
     ведущий научный сотрудник ИПИ РАН,\\ 
доктор физико-математических наук \hfill
      \textit{В.\,Ю.~Королев}
     
     } }
     }

%%%%%%%%%%%%%%%%%%%%%%%%%%%%%%%%%%%%%%%%%%%%%%%

\def\stat{arutun}


\def\tit{МОДЕЛИРОВАНИЕ ВЛИЯНИЯ ДЕФОРМАЦИЙ ОТПЕЧАТКОВ ПАЛЬЦЕВ НА~ТОЧНОСТЬ 
ДАКТИЛОСКОПИЧЕСКОЙ ИДЕНТИФИКАЦИИ$^*$}
\def\titkol{Моделирование влияния деформаций отпечатков пальцев на~точность 
дактилоскопической идентификации} 

\def\autkol{А.\,Р.~Арутюнян}
\def\aut{А.\,Р.~Арутюнян$^1$}

\titel{\tit}{\aut}{\autkol}{\titkol}

{\renewcommand{\thefootnote}{\fnsymbol{footnote}}\footnotetext[1]
{Работа выполнена
в рамках исследований Научно-образовательного центра ИПИ РАН\,--\,ВМК МГУ 
<<Биометрическая информатика>>.}}

\renewcommand{\thefootnote}{\arabic{footnote}}
\footnotetext[1]{Институт проблем безопасного развития атомной энергетики Российской академии наук, 
artem@ibrae.ac.ru}

%\vspace*{-6pt}

\Abst{Рассмотрена проблема учета влияния искажающих факторов на точность 
био\-мет\-ри\-ческой идентификационной системы. Предложена модель искажающих факторов, 
основанная на приближении методом моментов условных плотностей распределений меры 
сходства биометрических образцов. Разработан способ их оценки и учета. Проведены 
эксперименты по моделированию влияния эластичных деформаций на точность 
дактилоскопической идентификации.}

%\vspace*{-2pt}

\KW{биометрическая идентификация; операционные испытания; нелинейные деформации 
отпечатков}

\vskip 18pt plus 9pt minus 6pt

%\vspace{6pt}

      \thispagestyle{headings}

      \begin{multicols}{2}

      \label{st\stat}

\section{Введение }

  На сегодняшний день биометрические технологии идентификации личности получили 
широкое распространение в различных областях обеспечения безопасности: от контроля и 
управления доступом в офисные помещения до гражданской идентификации и 
правоохранительных приложений~[1--3]. С~распространением таких технологий актуальной 
становится проблема выбора того или иного метода биометрической идентификации. Основным 
фактором, определяющим результаты выбора, является точность идентификации, выраженная 
ROC (Receiver Operating Characteristics~--- функциональные характеристики приемника)
кривой соотношения вероятностей ошибок 1-го (FRR~--- False Reject Rate, вероятность ложного отказа)
и 2-го (FAR~--- False Acceptance Rate, вероятность ложной идентификации) рода~[4]. При этом 
  ROC-кривая зависит от данных, которые использовались при ее оценке, т.\,е.\ фактически 
зависит от среды и ха\-рак\-те\-ри\-зу\-ющих ее искажающих факторов. На практике эти зависимости 
могут быть очень сильными. На рис.~\ref{f1ar} приведен пример измерения ROC-кривой для 
технологии NIST VTB (National Institute of Standards and
Technology Verification Test Bed)~[5] на массивах отпечатков пальцев, полученных в разное время. Видно, 
что оценки вероятности FRR при одной и той же вероятности FAR могут различаться в 5--6~раз. 
Аналогичные оценки потенциального влияния среды могут быть получены теоретически~[6,~7].
  
  Подобная ситуация типична и для других биометрических характеристик. На рис.~2 
показано изменение точности идентификации за период с 1993 по 2006~гг., измеренное в ходе 
технологических испытаний NIST (США). В~частности, эти цифры указывают на то, что с~2002 
по 2006~гг.\ FRR снизилась в 20~раз, хотя при измерении точности на одном массиве (рис.~3)
наблюдается изменение в 3--4~раза, что больше соответствует практическим наблюдениям. 
Основной причиной таких рас\-хож\-де\-ний является изменение условий испытаний. В~2006~г.\ 
использовались данные, полученные в лучших операционных условиях по сравнению с~2002~г.\ 

Обрат\-ная си\-ту\-а\-ция наблюдалась в испытаниях технологий распознавания по отпечаткам пальцев 
FVC (Fingerprint Verification Competition): данные тестовых массивов FVC2004 оказались значительно хуже массивов FVC2002. 
Поэтому наблюдалось формальное снижение точности идентификации в то время, когда 
технологии распознавания по отпечаткам пальцев с точки зрения практики значительно 
улучшились.

  Такая неточность в оценке вероятности ошибок идентификации может приводить к 
не\-пра\-вильным решениям при проектировании и эксплуатации биометрических систем и даже к 
неправильному выбору используемой биометрической характеристики.
  
  Для решения этой проблемы в статье предложена модель искажающих факторов 
биометрической идентификации и способ оценки степени их влияния на точность распознавания 
и его последующего учета при эксплуатации. Входными параметрами модели являются 
численные оценки искажающих\linebreak\vspace*{-12pt}
\pagebreak

\end{multicols}

\begin{figure} %fig1
\vspace*{1pt}
\begin{center}
\mbox{%
\epsfxsize=154.762mm
\epsfbox{aru-1.eps}
}
\end{center}
\vspace*{-9pt}
\Caption{Кривые ROC идентификации по отпечаткам пальцев для технологии NIST VTB на 
различных массивах: TAR~--- True Acceptance Rate
%\textit{1}~--- `dos-cli.dat'; \textit{2}~--- `dos-cri.dat'; \textit{3}~--- `dhs2-cli.dat'; 
%\textit{4}~--- `dhs2-cri.dat'; \textit{5}~--- `benli.dat';  \textit{6}~--- `ben??.dat'?????; 
%\textit{7}~--- `benri.dat'; \textit{8}~--- `benrt.dat'; \textit{9}~--- `dhs10li.dat'; \textit{10}~--- 
%`dhs10ri.dat'; \textit{11}~--- `dhs10rt.dat'; \textit{12}~--- `txdpsli.dat'; \textit{13}~--- `txdpslt.dat'???;
%\textit{14}~--- `txdpsri??.dat'; \textit{15}~--- `txdpsrt.dat'; \textit{16}~--- `visit\_poe\_bvali.dat';
%\textit{17}~--- `visit\_poe\_bvari.dat'; \textit{18}~--- `visit\_poeli.dat'; 
%\textit{19}~---  `visit\_poeri.dat'
\label{f1ar}}
\end{figure}

\begin{multicols}{2}

\noindent
 факторов и ROC-кривая в эталонных операционных условий. 
Выходными параметрами~--- ROC-кри\-вая в реальных условиях эксплуатации.


  Статья организована следующим образом. В~разд.~2 приведена модель искажающих факторов. 
Раздел~3 посвящен применению модели к учету упругих деформаций в задаче 
дактилоскопической идентификации. Заключение содержит основные выводы по работе.

\noindent
\begin{center} %fig2
\vspace*{3pt}
\mbox{%
\epsfxsize=79.993mm
\epsfbox{aru-3.eps}
}
\end{center}
\vspace*{3pt}
{{\figurename~2}\ \ \small{Номинальный прогресс точности идентификации по форме лица в испытаниях NIST}}
%\end{center}
%\vspace*{6pt}



%\bigskip
\addtocounter{figure}{1}

\section{Модель операционных условий}

  При оценке точности идентификации био\-мет\-ри\-че\-ская система достаточно полно 
характеризуется условными плотностями распределений: меры сходства биометрических 
образцов в <<своих>> и <<чужих>> сравнениях. Обозначим их через~$f^{\mathrm{gen}}(x)$ 
и~$f^{\mathrm{imp}}(x)$ соответственно. Принятие решения осуществляется на основе сравнения меры 
сходства с порогом. Ошибки идентификации при выбранном пороге определяются как интегралы 
от плотностей (рис.~\ref{f4ar}). Ошибки FAR и FRR определяются по следующим формулам:

\vspace*{-3pt}

\noindent
  \begin{align*}
  \mathrm{FAR}(t) & = \int\limits_t^{+\infty} f^{\mathrm{imp}}(x)\,dx\,;\\
  \mathrm{FRR}(t) & = \int\limits_{-\infty}^{t} f^{\mathrm{gen}}(x)\,dx\,,
  \end{align*}
где $t$~--- порог принятия решения.



  Под влиянием различных операционных условий плотности распределений меняются. 
В~\cite{6ar} был предложен способ грубого учета операцион-\linebreak\vspace*{-12pt}
\pagebreak

\end{multicols}

\begin{figure} %fig3
\vspace*{1pt}
\begin{center}
\mbox{%
\epsfxsize=159.966mm
\epsfbox{aru-2.eps}
}
\end{center}
\vspace*{-9pt}
\Caption{Изменение точности идентификации при испытаниях NIST FRVT2002 и NIST 
FRVT2006 на одном массиве: \textit{1}~--- $\mathrm{FAR}=10^{-2}$; \textit{2}~--- 
$\mathrm{FAR}=10^{-3}$; \textit{3}~---  $\mathrm{FAR}=10^{-4}$ 
\label{f2ar}}
\end{figure}

\begin{figure} %fig4
\vspace*{1pt}
\begin{center}
\mbox{%
\epsfxsize=108.872mm
\epsfbox{aru-4.eps}
}
\end{center}
\vspace*{-9pt}
\Caption{Примерный вид плотностей условных распределений меры сходства биометрического 
сравнения
\label{f4ar}}
\end{figure}

\begin{multicols}{2}

\noindent
ных условий на основе 
вероятностных моментов первого и второго порядка. Воспользуемся теперь более точным 
приближением плотностей био\-мет\-ри\-че\-ских тестов~\cite{8ar} методом моментов с базовой 
нормальной плотностью~\cite{9ar}. 
  
  Пусть на основе эмпирических данных оценены вероятностные моменты условных 
распределений. Обозначим их через $\{m^{\mathrm{gen}}, \sigma^{\mathrm{gen}}, \gamma_3^{\mathrm{gen}}, \ldots , 
\gamma_n^{\mathrm{gen}}\}$  и $\{m^{\mathrm{imp}}, \sigma^{\mathrm{imp}}, \gamma_3^{\mathrm{imp}}, \ldots , 
\gamma_n^{\mathrm{imp}}\}$ для <<своих>> и <<чужих>> сравнений соответственно. 
Статистика~$\gamma$ (нормированный вероятностный момент) вычисляется по следующей 
формуле:
  \begin{equation*}
  \gamma_k =M\left [ \left (\fr{x-m}{\sigma}\right )^k\right]\,.
%  \label{e3ar}
  \end{equation*}
  
  Плотности распределений приближаются следующими функциями:
  \begin{equation*}
  f(x) =\sum\limits_{i=1}^n \gamma_i R_i\left (m+\alpha x\right) f^M_{0,1}\left (m+\alpha x\right )\,,
%  \label{e4ar}
  \end{equation*}
где $R_i$~--- базисные полиномы,  $f_{0,1}^N$~--- плотность стандартного нормального 
распределения.

  Приближение достаточно точно описывает реальные биометрические данные с точки зрения 
основной технологической характеристики: ROC-кри\-вой (рис.~5). 

\setcounter{figure}{5}
\begin{figure*}[b] %fig6
\vspace*{1pt}
\begin{center}
\mbox{%
\epsfxsize=162.398mm
\epsfbox{aru-6.eps}
}
\end{center}
\vspace*{-9pt}
\Caption{Среднее изменение ROC при идентификации отпечатков под действием деформаций (\textit{1}~--- без 
деформаций; \textit{2}~--- с деформациями):
(\textit{а})~Biolink MST; (\textit{б})~прямое наложение изображений 
\label{f6ar}}
\end{figure*}
  
  Операционные условия влияют на ROC-кри\-вую. Если воздействие искажающего фактора 
приводит к изменениям в ROC-кри\-вой, то оно также\linebreak\vspace*{-12pt}
\pagebreak

\noindent
\begin{center} %fig5
\vspace*{6pt}
\mbox{%
\epsfxsize=73.258mm
\epsfbox{aru-5.eps}
}
\end{center}
\vspace*{6pt}
{{\figurename~5}\ \ \small{Приближение ROC дактилоскопической идентификации: \textit{1}~--- исходная ROC;
\textit{2}~--- нормальная аппроксимация; \textit{3}~--- аппроксимация методом моментов (база NIST 
BSSR1$\backslash$ri)~[4]}}
%\end{center}
%\vspace*{6pt}



\bigskip
%\addtocounter{figure}{1}

\noindent
приводит к изменениям в векторе 
параметров~$S$, который можно рассматривать как функцию~$S$ от операционных 
условий~$u$. Если есть численная оценка операционных факторов, то изменение распределений 
можно приближенно определить по формуле
  \begin{equation}
  S(u) =S(0) +\sum\limits_{i=1}^w \fr{\partial S}{\partial u_i}\,u_i\,.
  \label{e5ar}
  \end{equation}

\section{Моделирование искажающих факторов на примере деформаций отпечатков пальцев}

  В качестве примера использования моделирования искажающих факторов рассмотрим 
деформации отпечатков пальцев. Выбор деформаций в качестве модельного примера 
определяется тем, что это один из немногих искажающих факторов дактилоскопической 
идентификации, который может быть численно оценен. Как показано в~\cite{10ar, 11ar}, сила 
воздействия деформации может быть определена энергией деформации. 
  
  Для оценки влияния деформаций на параметры распределений были проведены эксперименты 
на базе FVC2002~DB1 с двумя алгоритмами распознавания отпечатков пальцев: Biolink MST и 
прямого наложения изображений. На рис.~\ref{f6ar} представлены примеры изменения ROC под 
воздействием деформаций отпечатков пальцев~\cite{9ar, 10ar} для каждого из этих методов 
распознавания. В табл.~\ref{t1ar} и~\ref{t2ar} приведены изменения вероятностных моментов.


  
\begin{table*}\small
\begin{center}
\parbox{105mm}{\Caption{Изменение вероятностных моментов условных распределений мер сходства под 
влиянием деформаций отпечатков пальцев (Biolink MST)
\label{t1ar}}

}

\vspace*{2ex}

\begin{tabular}{|l|c|c|c|c|c|c|}
\hline
\multicolumn{1}{|c|}{Отпечатки}&$\mu $&$\sigma$&$\gamma_3$&$\gamma_4$&$\gamma_5$&$\gamma_6$\\
\hline
Свои (без деформаций)&730&181&$-0{,}22$&2,84&$-3{,}29$&16,70\\
Свои (с деформациями) &680&198&$-0{,}78$&3,94&$-9{,}75$&36,00\\
Чужие (без деформаций)&100&\hphantom{9}73&\hphantom{9,}0,76&3,01&\hphantom{9,}6,00&18,10\\
Чужие (с деформациями)&\hphantom{9}96&\hphantom{9}73&\hphantom{9,}0,66&3,13&\hphantom{9,}5,56&19,02\\
\hline
\end{tabular}
\end{center}
\end{table*}


\begin{table*}\small
\begin{center}
\parbox{105mm}{\Caption{Изменение вероятностных моментов условных распределений мер сходства под 
влиянием деформаций отпечатков пальцев (прямое наложение изображений)
\label{t2ar}}

}

\vspace*{2ex}

\tabcolsep=6.4pt
\begin{tabular}{|l|c|c|c|c|c|c|}
\hline
\multicolumn{1}{|c|}{Отпечатки}&$\mu$&$\sigma$&$\gamma_3$&$\gamma_4$&$\gamma_5$&$\gamma_6$\\
\hline
Свои (без деформаций)&335&147&0,33&2,81&\hphantom{9}2,83&\hphantom{9}14,96\\
Свои (с деформациями) &264&138&0,83&3,35&\hphantom{9}6,81&\hphantom{9}22,47\\
Чужие (без деформаций)&\hphantom{9}54&\hphantom{9}22&1,31&6,22&23,45&116,00\\
Чужие (с деформациями)&\hphantom{9}55&\hphantom{9}21&1,23&5,79&21,59&104,76\\
\hline
\end{tabular}
\end{center}
\end{table*}


  Для уточнения измерения влияния деформаций согласно уравнению~(\ref{e5ar}) были 
построены зависимости параметров распределений от энергии деформации. На рис.~\ref{f8ar} 
представлены зависимости меры сходства в своих сравнениях от энергии деформации на основе 
эмпирических данных FVC2002DB1. В чужих сравнениях была принята гипотеза о 
независимости распределений от фактора деформаций, так как для этих сравнений моменты 
практически не изменяются (см.\ табл.~\ref{t1ar} и~\ref{t2ar}).
  
\begin{figure*} %fig7
\vspace*{1pt}
\begin{center}
\mbox{%
\epsfxsize=157.606mm
\epsfbox{aru-8.eps}
}
\end{center}
%\vspace*{-9pt}
\Caption{Выборочные зависимости меры сходства от энергии деформации на репрезентативной 
выборке в экспериментах с базой FVC2002DB1: \textit{1}~--- прямое наложение; \textit{2}~--- 
Biolink MST; \textit{3}~--- линейный (прямое наложение); \textit{4}~--- линейный (Biolink MST)
\label{f8ar}}
\end{figure*}

  Из рис.~\ref{f8ar} видно, что имеется ощутимая зависимость меры сходства от энергии 
деформации. На рис.~\ref{f9ar} представлены графики относительных изменений 
параметров распределений в своих сравнениях при росте силы деформации отпечатков. 
В~качестве единичной эталонной энергии деформации взята средняя энергия, измеренная на базе 
FVC2002 DB1.


  Результаты искусственного моделирования точности идентификации в зависимости от силы 
деформации представлены на рис.~\ref{f11ar}.

\end{multicols}

\begin{figure}  %fig8
\vspace*{1pt}
\begin{center}
\mbox{%
\epsfxsize=161.822mm
\epsfbox{aru-9.eps}
}
\end{center}
\vspace*{-9pt}
\Caption{Относительное изменение параметров распределений в зависимости от силы 
деформации (от~1 до~6): (\textit{а})~Biolink MST; (\textit{б})~прямое наложение
\label{f9ar}}
\vspace*{6pt}
\end{figure}
\begin{figure} %fig9
\vspace*{1pt}
\begin{center}
\mbox{%
\epsfxsize=162.166mm
\epsfbox{aru-11.eps}
}
\end{center}
\vspace*{-9pt}
\Caption{Изменение ROC в зависимости от силы деформации (\textit{1}~--- без деформаций (факт.);
\textit{2}~--- без деформаций (модель);
\textit{3}~--- с деформациями\;$\times$\;1 (модель);
\textit{4}~--- с деформациями (факт.);
\textit{5}~--- с деформациями\;$\times$\;0,5 (модель);
\textit{6}~--- с деформациями\;$\times$\;2 (модель)):
(\textit{а})~Biolink MST; 
(\textit{б})~прямое  наложение
\label{f11ar}}
\vspace*{6pt}
\end{figure}

\begin{multicols}{2}

\section{Заключение}

В статье представлена модель операционных условий и метод оценки показателей точности 
биометрической идентификации при их изменении. Эксперименты с дактилоскопической 
идентификацией позволяют сделать предварительное заключение о практической применимости 
модели. 

В дальнейшем модель может применяться к оценке других искажающих факторов 
биометрической идентификации, допускающих непосредственное численное измерение, 
например угла поворота лица.


{\small\frenchspacing
{%\baselineskip=10.8pt
\addcontentsline{toc}{section}{Литература}
\begin{thebibliography}{99}

\bibitem{3ar} %1
\Au{Bolle R.\,M., Connell J.\,H., Pankanti~S., Ratha~N.\,K., Senior~A.\,W.}
Guide to biometrics.~--- New York: Springer-Verlag, 2003.

\bibitem{1ar} %2
\Au{Dessimoz D., Champod~C., Richiadi~J., Drygajlo~A.}
Multimodal biometrics for identity documents. Research Report, PFS 314-08.05. UNIL, June~2006.

\bibitem{2ar} %3
\Au{Sinitsyn I.\,N., Ushmaev~O.\,S.}
Development of metrological and biometric technologies and systems~// 9th Conference (International) 
in Pattern Recognition and Image Analysis: New Information Technologies~--- PRIA-9-2002 
Proceedings.~--- Nizhni Novgorod, 2008. Vol.~2. P.~169--172.

\bibitem{4ar}
\Au{Wayman~J.\,L., Jain~A.\,K., Maltoni~D., Maio~D.}
Biometric systems: Technology, design and performance evaluation.~--- London: 
Springer Verlag, 2005.

\bibitem{5ar}
\Au{Watson C., Wilson~C., Indovina~M., Cochran~B.}
Two finger matching with Vendor SDK matchers. NISTIR 7249. July 2005.

\bibitem{6ar}
\Au{Ушмаев О.\,С.}
Адаптация биометрической системы к искажающим факторам на примере 
дактилоскопической идентификации~// Информатика и её применения, 2009. Т.~3. 
Вып.~2. С.~25--33.
\pagebreak

\bibitem{7ar}
\Au{Ушмаев~О.\,С., Арутюнян~А.\,Р.}
Метод оценки качества биометрической идентификации в операционных условиях 
на примере дактилоскопической идентификации~// Труды конференции 
ГрафиКон'2009: 19-я Международная конференция по компьютерной графике и 
зрению.~--- М.: МАКС ПРЕСС, 2009. С.~232--235.

\bibitem{8ar}
\Au{Ushmaev O., Novikov S.}
Biometric fusion: Robust approach~// 2nd Workshop on Multimodal User 
Authentication~--- MMUA'06 Proceedings.  Toulouse, France, 11--12~May 2006. {\sf  
http://mmua.cs.uchb.edu/ MMUA2006/Papers/127.pdf}.

\bibitem{9ar}
\Au{Pugachev V.\,S., Sinitsyn I.\,N.}
Stochastic systems: Theory and applications.~--- World Scientific, 2001.

\label{end\stat}

\bibitem{10ar}
\Au{Ушмаев О.\,С., Арутюнян А.\,Р.}
Влияние деформаций на качество биометрической идентификации по отпечаткам 
пальцев~// Информатика и её применения, 2009. Т.~3. Вып.~4. С.~12--21.



\bibitem{11ar}
\Au{Novikov S. Ushmaev~O.}
Registration and modelling of elastic deformations of fingerprints~// Biometric 
Authentication: ECCV 2004 International Workshop~--- BioAW2004 Proceedings.~--- 
Berlin--Heidelberg: Springer-Verlag, 2004. 
P.~80--88.
 \end{thebibliography}
}
}
\end{multicols}  %7

\def\stat{gudkov}

\def\tit{БЫСТРАЯ ОБРАБОТКА ИЗОБРАЖЕНИЙ ОТПЕЧАТКОВ ПАЛЬЦЕВ}

\def\titkol{Быстрая обработка изображений отпечатков пальцев}

\def\autkol{В.\,Ю.~Гудков, М.\,В.~Боков}
\def\aut{В.\,Ю.~Гудков$^1$, М.\,В.~Боков$^2$}

\titel{\tit}{\aut}{\autkol}{\titkol}

%{\renewcommand{\thefootnote}{\fnsymbol{footnote}}\footnotetext[1]
%{Работа выполнена при поддержке РФФИ (гранты 09-07-12098, 09-07-00212-а и 
%09-07-00211-а) и Минобрнауки РФ (контракт №\,07.514.11.4001).}}


\renewcommand{\thefootnote}{\arabic{footnote}}
\footnotetext[1]{Челябинский государственный университет, diana@sonda.ru}
\footnotetext[2]{Южно-Уральский государственный университет, guardian@mail.ru}


\Abst{Предложена последовательность методов распознавания частных 
признаков на изображении отпечатка пальца с жесткими ограничениями на время 
обработки. Частные признаки сохраняются в шаблоне изображения. По шаблонам 
выполняется идентификация изображений.}

\KW{отпечаток пальца; обработка изображений; матрица потоков; мат\-ри\-ца периодов; 
част\-ные признаки}

 \vskip 14pt plus 9pt minus 6pt

      \thispagestyle{headings}

      \begin{multicols}{2}
      
            \label{st\stat}

\section{Введение}
  
  Исследования в области биометрии начались более ста лет назад с 
разработки методов сравнения отпечатков пальцев. С~развитием 
вы\-чис\-ли\-тель\-ной техники появилась возможность учета лиц в электронных 
системах. Функционирование таких электронных систем, подобно 
деятельности экс\-пер\-та-кри\-ми\-на\-лис\-та, опирается на модель 
дактилоскопического изображения (ДИ) в виде частных признаков и 
отношений между ними~[1]. Среди электронных систем наиболее известны 
системы криминального и гражданского назначения. Если для первых сис\-тем 
основным показателем эффективности служит величина ошибки 
идентификации  подозреваемого лица, то для вторых наравне с величиной 
ошибки аутентификации пользователя не\linebreak\vspace*{-12pt}
\begin{center} %fig1
\vspace*{12pt}
\mbox{%
  \epsfxsize=75.856mm
 \epsfbox{gud-1.eps}
}
\end{center}
\begin{center}
\vspace*{3pt}
{{\figurename~1}\ \ \small{Изображение отпечатка пальца}}
\end{center}
%\vspace*{11pt}

%\smallskip
\addtocounter{figure}{1}


\noindent
 менее важна и производительность 
сис\-те\-мы~[2]. Это оказывает сильное влияние на выбор методов обработки 
ДИ, например в системах контроля и управления доступом к объекту.
  
  На рис.~1 на узоре левой петли выделены частные признаки в виде 
окончания и разветвления, распознавание которых простыми методами 
неэффективно~\cite{1-g, 3-g}. Поэтому быстрая обработка ДИ реализуется в 
виде последовательности специальных методов измерения, анализа и синтеза 
параметров изображения. Настройка и обучение этих методов минимизируют 
влияние дефектов изображения. Тем не менее жесткие ограничения по времени 
сужают класс ДИ, пригодных для быстрой обработки, преимущественно до 
изображений хорошего и среднего качества.


\section{Постановка задачи}

  Обычно при распознавании частных признаков выполняются этапы 
предварительной обработки и повышения качества ДИ. Для этого изображение 
представляется в прямоугольной области~$G$ мощностью $\vert G\vert 
\hm=x_0y_0$ в виде $F\hm=\{f(x,y)\hm\in 0, \ldots , 2^b-1\vert (x,y)\hm\in X\times Y\}$, где 
$b$~--- глубина изоб\-ра\-же\-ния; $X\hm=0, \ldots , x_0\hm-1$ и $Y\hm=0, \ldots 
, y_0\hm-1$. 
  
  Обработка изображения структурно пред\-став\-ля\-ет\-ся в виде 
пирамиды~$\mathfrak{R}$ взаимосвязанных иерархий~[3--5], в которых 
сегментация изображения производится для любого слоя произвольной 
иерархии. Например, $l$-й слой $k$-й иерархии $F_k^{(l)}$ разбивается на 
$x_h y_h$ квадратных сегментов $S_{hk}^{(l)}(x,y)$ с длиной стороны~$2^{h-
k}$ и вершинами $(x,y)\hm\in X_h\hm\times Y_h$, где $k\hm<h$ и $h$~--- номер иерархии; 
$X_h=0, \ldots , x_h-1$ и $Y_h\hm=0, \ldots , y_h-1$.
  
  Доступ к каждой точке сегмента $S_{hk}(x,y)$ по~\cite{3-g} записывается в 
координатах $(u,v)\in \overline{X}_{hk}\times \overline{Y}_{hk}$:

\noindent
  \begin{equation}
  \left.
  \begin{array}{rl}
  \overline{X}_{hk} &= \left\{ u+x2^{h-k}\vert x\in{}\right.\\[9pt]
  &\hspace*{7mm}\left.{}\in X_h \land u\in 0, \ldots , 2^{h-
k}-1\right\}\,;\\[9pt]
  \overline{Y}_{hk} &= \left\{ v+y2^{h-k}\vert y\in {}\right.\\[9pt]
  &\hspace*{7mm}\left.{}\in Y_h \land v\in 0, \ldots , 2^{h-
k}-1\right\}\,.
  \end{array}
  \right\}
  \label{e1-g}
  \end{equation}
  
  Доступ к центральной точке сегмента $S_{hk}(x,y)$ записывается в 
координатах $(u,v)\in \hat{X}_{hk}\times \hat{Y}_{hk}$:
  \begin{equation}
  \left.
  \begin{array}{rl}
  \hat{X}_{hk} &=\left\{ 2^{h-k-1}+x 2^{h-k}\vert x\in X_h\right\}\,;\\[9pt]
  \hat{Y}_{hk} &=\left\{ 2^{h-k-1}+y 2^{h-k}\vert y\in Y_h\right\}\,.
  \end{array}
  \right\}
  \label{e2-g}
  \end{equation}
  
  Размер области $h$-й иерархии: $x_h=\lceil x_0/2^h \rceil$ и $y_h=\lceil 
y_0/2^h\rceil$, где $\lceil a\rceil$~--- наименьшее целое число, превышающее 
вещественную величину~$a$.
  
  При иерархической сегментации сегменты слоя~$F_k$ отображаются на 
вершины сегментов слоя~$F_h$ пирамиды~$\mathfrak{R}$, где $k\hm<h$. 
Соответственно, вершины сегментов отображаются на сегменты, 
расположенные ближе к основанию пирамиды~\cite{3-g}. Размер сегмента 
заметно влияет на время и качество обработки. Далее положим 
$S_h(x,y)=S_{h0}(x,y)$ и вершины $S_h(x,y)\in F_h$.
  
  Слои пирамиды можно представить множеством действительных чисел, а 
исходное изображение~--- множеством неотрицательных действительных 
чисел~\cite{4-g, 5-g}. Это снимает необходимость утомительного 
целочисленного представления сигнала и упрощает выражения, однако 
дискретизация изображения (слоев пирамиды~$\mathfrak{R}$) в пространстве 
сохраняется.
  
  Для компактной математической формализации методов 
классификационного анализа (КА) широко применяется аппарат апертур. 
Ключевую роль при этом играют прямолинейные щелевые $A_h(x,y,\alpha,w)$ 
и $A_h^-(x,y,\alpha,w)$ и круговая $A_h(x,y,w)$ апертуры, представляемые 
множеством точек слоя данных $h$-й иерархии и связанными с ними углами в 
виде элементов упорядоченных троек $(u,v,\beta)$. Эти апертуры определяются 
по формулам:
  \begin{equation}
  \left.
  \begin{array}{rl}
  \!A_h(x,y,\alpha,w) &= \left\{ (u,v,\beta) 
={}\right.\\[6pt]
&\hspace*{-23mm}\left.{}=(x+]w\cos(\alpha)[,y+]w\sin(\alpha)[,\beta)\vert w\in Z_w\right\}\,;\\[6pt]
  \!A_h^- (x,y,\alpha,w) &=\left\{ 
(u,v,\beta)={}\right.\\[6pt]
&\hspace*{-23mm}\left.{}=(x+]w\cos(\alpha)[,y+]w\sin(\alpha)[,\beta)\vert w\in Z_w^-\right\}\,;
  \end{array}
  \right\}\!
  \label{e3-g}
  \end{equation}

\noindent
\begin{equation}
A_h(x,y,w) =\cupb\limits_{\alpha\in Z^*} A_h(x,y,\alpha,w)\,,
\label{e4-g}
\end{equation}
где $(x,y)\in X_h\times Y_h$~--- центр апертуры; $(u,v)\hm\in X_h\times Y_h$~--- 
точка апертуры; $w$~--- размер апертуры; $Z_w=1, \ldots , w$; $Z_w^- \hm = -
w, \ldots  , -1\cup 1, \ldots , w$; $\alpha$~--- угол на\-прав\-ле\-ния апертуры; $]a[$~--- 
ближайшая целая часть вещественного числа~$a$. Угол, определяющий 
направление из центра апертуры $(x,y)$ в точку $(u,v)$, находится в виде:
$$
\beta =\arctg \left( \fr{v-y}{u-x}\right)+\pi n\enskip \mbox{при}\ n\in 0, \ldots , 1\,.
$$
  
  Для задачи распознавания частных признаков этапы предварительной 
обработки и повышения качества ДИ должны удовлетворять требованию на 
ограничение по времени. При этом алгоритм должен обеспечивать приемлемое 
качество распознавания частных признаков, которое проверяется на тестовой 
базе ДИ. Список частных признаков формируется в виде:
  \begin{equation}
  L_m=\left\{ M_i=\left\{ x_i,y_i,\alpha_i,t_i\right\}\vert i\in 1, \ldots , n\right\}\,,
  \label{e5-g}
  \end{equation}
где $M_i$~---  частный признак, индексированный номером~$i$; $n=\vert 
L_m\vert$~---  мощность списка частных признаков; $(x_i,y_i)$~---  координаты 
частного признака~$M_i$; $\alpha_i\in 0, \ldots , 359$~---  направление~$M_i$ как 
угол; $t_i\in \{0,1\}$~--- тип~$M_i$ (окончание или разветвление).
  
  Компромиссным решением задачи, устра\-ня\-ющим противоречие 
  ка\-че\-ст\-во--ско\-рость, может служить реализация шести этапов обработки 
ДИ: (1)~построение матрицы потоков; (2)~сглаживание вдоль потоков; 
(3)~построение матрицы плотности линий; (4)~сегментация; (5)~бинаризация; 
(6)~скелетизация и распознавание частных признаков.

\section{Быстрая обработка}

  Большинство алгоритмов КА отпечатков пальцев нацелено на распознавание 
частных признаков~\cite{1-g}. Решение задачи быстрой обработки 
продемонстрируем на примере исходного изображения $F_0^{(0)}\hm= \left\{ 
f_0^{(0)}(x,y)\right\}$ (см.\ рис.~1).

\subsection{Интегральное изображение} %3.1
  
  Интегральное изображение~$I$ позволяет вы\-чис\-лить сумму элементов 
прямоугольной области изоб\-ра\-же\-ния с постоянным числом операций 
независимо от размера области. Для вычисления интегрального изображения 
используется следующая формула:
  \begin{multline}
  I(x,y)=f(x,y)+{}\\
  {}+I(x-1,y)+I(x,y-1)-I(x-1,y-1)\,.
  \label{e6-g}
  \end{multline}
  
  Результат вычисления интегрального изображения показан на рис.~2. 
Однажды вычислив интегральное изображение, можно найти сумму элементов 
любой прямоугольной области изображения с\linebreak\vspace*{-12pt}
\begin{center} %fig2
\vspace*{4pt}
\mbox{%
 \epsfxsize=70.657mm
 \epsfbox{gud-2.eps}
}
\end{center}
%\begin{center}
\vspace*{3pt}
{{\figurename~2}\ \ \small{Прямоугольная область изображения~(\textit{а}) и интегральное изображение~(\textit{б})}}
%\end{center}
\vspace*{8pt}

%\smallskip
\addtocounter{figure}{1}

\noindent
 левым верхним углом $(x_1, y_1)$ 
и правым нижним углом $(x_2, y_2)$ за постояное время, используя следующее 
уравнение:
  \begin{multline}
  \sum\limits_{x=x_1}^{x_2}\sum\limits_{y=y_1}^{y_2} f(x,y)=I(x_2,y_2)-I(x_1-
1,y_2)-{}\\
{}-I(x_2,y_1-1)+I(x_1-1,y_1-1)\,.
  \label{e7-g}
  \end{multline}



\subsection{Построение матрицы потоков} %3.2
  
  Это базовый этап обработки, от которого зависит точность распознавания 
частных признаков. Он состоит из двух последовательно выполняемых 
процедур обработки ДИ. Самый простой подход к вы\-чис\-ле\-нию матрицы 
потоков основан на вы\-чис\-ле\-нии градиента.
  
  \textit{Измерение матрицы потоков.} Суть метода заключается в разбиении 
изображения на сегменты с точками $(u,v)\in S_h(x,y)$ при $h\hm=3$ $(8\times 8)$ 
по~(\ref{e1-g}) и вычислении для вершины каждого сегмента величины угла 
$0^\circ\leq \delta_h^{(0)}(x,y)\hm<180^\circ$ как элемента матрицы 
потоков~$\Lambda_h^{(0)}$ по формуле:
  \begin{multline*}
  \Lambda_h^{(0)} =\left\{ \delta_h^{(0)}(x,y)\right\} ={}\\
  {}= \left\{ 
\fr{\pi}{2}+\fr{1}{2}\,\arctg \left( \fr{2J_{12}(x,y)}{J_{22}(x,y)-
J_{11}(x,y)}\right)\right\}\,,
%  \label{e8-g}
  \end{multline*}
где 

\noindent
\begin{align*}
J_{12}(x,y)&=\sum\limits_{(u,v)\in S_h(x,y)}\nabla_x \nabla_y\,;\\
J_{11}(x,y)&=\sum\limits_{(u,v)\in S_h(x,y)}\nabla_x\nabla_x\,;\\
J_{22}(x,y) &= 
\sum\limits_{(u,v)\in S_h(x,y)}\nabla_y\nabla_y\,.
\end{align*}
Компоненты градиента в 
отсчетах $(u,v)\in \overline{X}_{hk}\hm\times \overline{Y}_{hk}$ по~(\ref{e1-g}) 
основания сегмента $S_h(x,y)$ вычисляются в виде 
$\nabla_x=\mathbf{H}_x**f_0^{(0)}(u,v)$,  $\nabla_y\hm=\mathbf{H}_y ** 
f_0^{(0)}(u,v)$, где ядра двумерной свертки как оптимизированные по 
величине ошибки угла ориентации операторы Собела~\cite{5-g} находятся в 
виде:
$$
\mathbf{H}_x=\begin{bmatrix}
-3 & 0 & 3\\
-10 & 0 & 10\\
-3 & 0 & 3
\end{bmatrix}\,;\quad 
\mathbf{H}_y=\begin{bmatrix}
-3 & -10 & -3\\
0&0&0\\
3 &10&3
\end{bmatrix}\,.
$$
  
  Таким образом, предварительно необходимо вычислить интегральные 
изображения $IG_{xy}(x,y)$, $IG_{xx}(x,y)$ и $IG_{yy}(x,y)$ по~(\ref{e6-g}) 
для определения тензоров $J_{12}(x,y)$, $J_{11}(x,y)$ и $J_{22}(x,y)$ 
соответственно и расчета потока независимо от размера апертуры. Здесь
  \begin{multline*}
  IG_{xy}(x,y) =G_{xy}(x,y)+IG_{xy}(x-1,y)+{}\\
  {}+IG_{xy}(x,y-1)-IG_{xy}(x-1,y-1)\,,
  \end{multline*}
где $G_{xy}(x,y)=\nabla_x(x,y)\nabla_y(x,y)$.
  
  Интегральные изображения $IG_{xx}(x,y)$ и $IG_{yy}(x,y)$  вычисляются 
аналогично $IG_{xy}(x,y)$. Затем определяем $J_{12}(x,y)$, $J_{11}(x,y)$ и 
$J_{22}(x,y)$ по следующим формулам:
\begin{align*}
J_{12}(x,y) &=IG_{xy}(x_2,y_2)-IG_{xy}(x_1-1,y_2)-{}\\
&{}-IG_{xy}(x_2,y_1-1)+IG_{xy}(x_1-1,y_1-1)\,;\\
J_{11}(x,y) &=IG_{xx}(x_2,y_2)-IG_{xx}(x_1-1,y_2)-{}\\
&{}-IG_{xx}(x_2,y_1-1)+IG_{xx}(x_1-1,y_1-1)\,;\\ 
J_{22}(x,y) &=IG_{yy}(x_2,y_2)-IG_{yy}(x_1-1,y_2)-{}\\
&{}-IG_{yy}(x_2,y_1-1)+IG_{yy}(x_1-1,y_1-1)\,,
\end{align*}
 где точка $(x_1,y_1)$~--- левый верхний угол заданной прямоугольной области 
изображения, а точка $(x_2,y_2)$~--- правый нижний угол.
  
  Фактически элементы из~$\Lambda_h$ вычисляются сглаживанием в 
сегментах $\{S_h(x,y)\}$ компонент поточечного структурного тензорного 
оператора~\cite{5-g}, записываемого в виде:
  \begin{equation}
  J=\begin{bmatrix}
  J_{11}+J_{22}\\
  J_{22}-J_{11}\\
  2J_{12}
  \end{bmatrix}\,.
  \label{e9-g}
  \end{equation}
  
  \textit{Анализ и коррекция матрицы потоков}. В~иерархии $h\hm=3$ на 
основе~(\ref{e9-g}) для $(x,y)\in X_h\times Y_h$ рассчитывается когерентность 
потоков по фор\-муле:
  \begin{multline}
  M_h^{(0)} =  \left[ \mu_h^{(0)}(x,y) \right] ={}\\
  {}=\left[ 
\fr{\sqrt{(J_{22}(x,y)-
J_{11}(x,y))^2+4J^2_{12}(x,y)}}{J_{11}(x,y)+J_{22}(x,y)}\right]\,.
  \label{e10-g}
  \end{multline}



  Когерентность для идеальных линий равна единице, а для изотропной 
структуры~--- нулю~\cite{5-g}. На\linebreak\vspace*{-12pt}
\pagebreak

\end{multicols}

\begin{figure} %fig3
 \vspace*{1pt}
 \begin{center}
 \mbox{%
 \epsfxsize=161.898mm
 \epsfbox{gud-3.eps}
 }
 \end{center}
 \vspace*{-9pt}
\Caption{Результаты вычислений $\Lambda_h$ (справа) и
соответствующих им $M_h$ (слева):
(\textit{а})~$\Lambda_h^{(0)}$ и $M_h^{(0)}$;
(\textit{б})~$\Lambda_h^{(1)}$ и $M_h^{(1)}$;  
(\textit{в})~$\Lambda_h^{(2)}$ и $M_h^{(2)}$;
(\textit{г})~$\Lambda_h^{(3)}$ и $M_h^{(3)}$}
\end{figure}


\begin{multicols}{2}

\noindent
 основе ранее вычисленных интегральных 
изоб\-ра\-же\-ний находим $\Lambda_h^{(0)}$, $\Lambda_h^{(1)}$,  $\Lambda_h^{(2)}$, 
$\Lambda_h^{(3)}$ и $M_h^{(0)}$, $M_h^{(1)}$, $M_h^{(2)}$, $M_h^{(3)}$ при 
$h\hm=3$ и апертурах $A_h(x,y,8)$, $A_h(x,y,24)$, $A_h(x,y,40)$, $A_h(x,y,56)$ 
соответственно. Результаты вычислений представлены на рис.~3.
  
  Затем матрицы потоков $\Lambda_h^{(0)}$, $\Lambda_h^{(1)}$,  $\Lambda_h^{(2)}$, 
$\Lambda_h^{(3)}$ и матрицы когерентностей $M_h^{(0)}$, $M_h^{(1)}$, $M_h^{(2)}$, 
$M_h^{(3)}$ анализируют. Для этого рассчитывают матрицу производных 
$N_n^{(0)}$ и мат\-ри\-цу потоков $O_h^{(0)}$ в виде:
  \begin{align*}
  N_h^{(0)} &= \left [ v_h^{(0)}(x,y)\right] ={}\\
  &\hspace*{14mm}{}=\left[ \fr{1}{n}\sum\limits_{l=1}^{n-
1} \left( \mu_h^{(l)}(x,y)-\mu_h^{(l-1)}(x,y)\right)\right]\,;\\
  O_h^{(0)} &= \lfloor o_h^{(0)}(x,y)\rfloor =\lfloor \delta_h^{(c)}(x,y)\rfloor\,,
  \end{align*}
где $c=\arg \max\limits_l \left( \mu_h^{(l)}(x,y)\right)$; $n$~--- количество слоев 
(в реализации~4). Матрица~$N_h^{(0)}$ отображает изменение когерентности 
потоков, а $O_h^{(0)}$~--- лучшие потоки.
  
  На основе матриц $N_h^{(0)}$ и $O_h^{(0)}$ вычисляют матрицу потока 
$\Lambda_h^{(4)}$: 
\begin{equation}
 \Lambda_h^{(4)} =\left\{ \delta_h^{(4)} (x,y)\right\}\,.
 \label{e11-g}
 \end{equation}
 Здесь
 \begin{equation*}
  \delta_h^{(4)}(x,y)=
  \begin{cases}
\delta_h^{(3)}(x,y), &\mbox{если}\ v_h^{(0)}(x,y)>T\,;\\
  o_h^{(0)}(x,y) & \mbox{иначе}\,,
  \end{cases}
  \end{equation*}
где $T$~--- пороговое значение (в реализации~0.1).
  
  Элементы из $\Lambda_h^{(4)}$ корректируют по формулам:
  \begin{equation*}
  \Lambda_h^{(5)} = \left\{ \delta_h^{(5)}(x,y)\right\}= \left\{ \fr{1}{2}\arctg\left( 
\fr{\mathrm{Im}_h^{(0)}(x,y)}{\mathrm{Re}_h^{(0)}(x,y)}\right)\right\}\,;
\end{equation*}

\noindent
\begin{equation*}
\mathrm{Re}_h^{(0)}(x,y) = \!\!\!\!\sum\limits_{(u,v)\in A_h(x,y,1)}\!\!\!\!\mu_h^{(0)}(u,v)\cos\left( 
2\delta_h^{(4)}(u,v)\right)\,;
\end{equation*}

%\noindent

\begin{center} %fig4
\vspace*{12pt}
\mbox{%
 \epsfxsize=79mm
 \epsfbox{gud-4.eps}
}
\end{center}
\begin{center}
%\vspace*{3pt}
{{\figurename~4}\ \ \small{Результирующая матрица потоков}}
\end{center}
\vspace*{9pt}

%\smallskip
\addtocounter{figure}{1}


\noindent
\begin{equation*}
\mathrm{Im}_h^{(0)}(x,y) = \!\!\!\!\sum\limits_{(u,v)\in A_h(x,y,1)}\!\!\!\! \mu_h^{(0)}(u,v)\sin\left( 
2\delta_h^{(4)}(u,v)\right)\,,
  \end{equation*}
где $\mu_h^{(0)}(u,v)$ и $\delta_h^{(4)}(u,v)$~--- когерентность и поток 
по~(\ref{e10-g}) и~(\ref{e11-g}) в отсчете $(u,v)$ апертуры $A_h(x,y,1)$ 
по~(\ref{e4-g}). На рис.~4 показана матрица потоков~$\Lambda_h^{(5)}$.



\subsection{Сглаживание} %3.3

  Сглаживающий фильтр устраняет микроразрывы и микрозалипания линий. 
Перед сглаживанием вычисляют матрицу регулярности потока $IR_h^{(0)}$, 
$h\hm=3$, используя следующую формулу~\cite{1-g}:
  \begin{multline*}
  IR_h^{(0)} =\left[ ir_h^{(0)}(x,y)\right] ={}\\
  {}=\left[ 1-\fr{\left \Vert 
  \sum\limits_{m=-1}^1 \sum\limits_{n=-1}^1 d(x+n,y+m)\right\Vert}{\sum\limits_{m=-1}^1 
\sum\limits_{n=-1}^1 \parallel d(x+n,y+m)\parallel}\right]\,,
  \end{multline*}
где $d(x,y)=\lfloor \cos 2\delta_h^{(5)}(x,y),\;\sin2\delta_h^{(5)}(x,y)\rfloor$; 
$\parallel x\parallel$~--- норма~$L_2$.
  
  Затем в основании каждого сегмента $S_h(x,y)$ величины отсчетов 
сглаживают по формуле:
  \begin{equation*}
  F_0^{(1)} =\begin{cases}
  f_0^{(1)}(x,y)=\mathbf{H}_1*\Xi_0^{(\alpha)}(x,y)\,, &\\
  &\hspace*{-30pt}\mbox{если}\ 
ir_h^{(0)}(x,y)>t\,;\\
  f_0^{(1)}(x,y)=f_0^{(0)}(x,y)\,, & \hspace*{-11mm}\mbox{иначе}\,,
  \end{cases}
%  \label{e12-g}
  \end{equation*}
где $\mathbf{H}_1$~---  ядро одномерной свертки; $t$~--- некоторый порог (в 
реализации~0.3); набор $\Xi_0^{(\alpha)}(x,y)\hm=\left\{ 
\xi_0^{(\alpha)}(u,v)\right\}$ состоит из элементов, выбираемых из~$F_0^{(0)}$ 
прямолинейной щелевой апертурой по~(\ref{e3-g}) в виде
\begin{multline*}
\left\{ \xi_0^{(\alpha)}(u,v)\right\} ={}\\
{}=\left\{ f_0^{(0)}(u,v)\vert(u,v)\in A_0^-
(x,y,\alpha,w) \cup (x,y)\right\}\,;
\end{multline*}
$\alpha= \delta_h^{(5)}(x,y)\in \Lambda_h^{(5)}$~---  направление апертуры, 
одинаковое для всех отсчетов основания сегмента $S_h(x,y)$; $w$~---  размер 
апертуры. 

Перенумеруем упорядоченные отсчеты набора 
$\Xi_0^{(\alpha)}(x,y)\hm= \left\{ \xi_0^{(\alpha)}(u,v)\right\}$, 
сгенерированного щелевой {апертурой} по~(\ref{e3-g}), в виде $ k \mapsto 
(u_k,v_k)$, где $k\in 0, \ldots , N$; $N=2w+1$. Тогда ядро свертки~$\mathbf{H}_1$ 
рассчитывают в виде:
$$
\mathbf{H}_1=\exp\left ( -\fr{(w-k)^2}{2\sigma^2}\right)\,,
$$
где $\sigma$~---  среднеквадратичное отклонение, определяющее крутизну 
гауссианы~\cite{1-g} (2--4 в реализации); $k\equiv w$~---  отсчет центра окна, 
здесь равный размеру апертуры~$w$. Сглаживающий фильтр, по сути, является 
выделенным первым сомножителем разделимого фильтра Габора с  
числом отсчетов, меньшим в $\pi w/2$ раз, что позволяет повысить производительность 
обработки. Результат сглаживания представлен на рис.~5.

\begin{center} %fig5
\vspace*{12pt}
\mbox{%
  \epsfxsize=79mm
 \epsfbox{gud-5.eps}
}
\end{center}
\begin{center}
\vspace*{3pt}
{{\figurename~5}\ \ \small{Исходное~(\textit{а}) и сглаженное~(\textit{б}) изображения}}
\end{center}
%\vspace*{11pt}

%\smallskip
\addtocounter{figure}{1}


  

\subsection{Построение матрицы периодов} %3.4

  Это базовый этап обработки, влияющий на точность распознавания частных 
признаков. Он выполняется на той же иерархии $h\hm=3$ и состоит из двух 
последовательно выполняемых процедур обработки ДИ.
  
  \subsubsection*{Измерение матрицы локальных периодов линий}
  
  \noindent
  \textbf{Определение 1.} Под периодом линий понимается величина $t=w/n$, 
обратно пропорциональная среднему числу $n$~линий, умещающихся в 
окрестности размером~$w$ на прямой, проведенной перпендикулярно 
линиям~\cite{1-g}.
  
  Зададим отрезок $C(x,y)=A_0^- (x,y,\alpha,w)\cup(x,y)$, сгенерированный 
щелевой апертурой по~(\ref{e3-g}), и перенумеруем отсчеты $(u,v)\in C(x,y)$ в 
виде $k\hm\mapsto (u_k,v_k)$, где $k\hm\in(0,\ldots  , N$; $N\hm=2w\hm+1$. В~отрезке 
$C(x,y)$ с центром $k\in \{w\}$ в отсчете $(x,y)\in X\times Y$ собираются 
упорядоченные по~$k$ величины $f_0^{(1)}(k)$ яркости изображения. 
Ориентация щелевой апертуры определяется углом~$\alpha$, выбираемым 
перпендикулярно потоку по формуле
  $$
  \alpha =\fr{\pi}{2}+\delta_h^{(5)}(x,y)
  $$
при $(x,y)\in X_h\times Y_h$, а ее длина определяется окрестностью размером 
$w$ для отсчета $(x,y)\in X\times Y$ (16 в~реализации). Для отрезка $C(x,y)$, 
центрированного в отсчете $(x,y)\in \hat{X}_{h0}\times \hat{Y}_{h0}$ 
по~(\ref{e2-g}) на сегменте $S_h(x,y)$, введем автокорреляционную функцию в 
виде:
\begin{equation}
r(i) =\fr{1}{N}\sum\limits_{k=0}^{N-i-1} \overset{\frown}{f}_0^{(1)}(k) 
\overset{\frown}{f}_0^{(1)}(k+i)\,,
\label{e13-g}
\end{equation}
где $\overset{\frown}{f}_0^{(1)}(k)\hm= f_0^{(1)}(k)-\overline{f}$ и 
$\overline{f} \hm= (1/N) \sum\limits_{k=0}^{N-1} f_0^{(1)}(k)$;  $N$~--- чис\-ло 
отсчетов щелевой апертуры. Определим $\Delta r(i) \hm= r(i+1)\hm-
r(i)$. Тогда элементы матрицы локального периода линий $T_h^{(0)}\hm=\left\{ 
t_h^{(0)}(x,y)\right\}$ вычисляются по формуле:
\begin{multline}
t_h^{(0)} (x,y)={}\\
{}=\arg \min\limits_j \left\{ \left( \Delta r(0),\ldots , \Delta r(j)\right) 
\vert \Delta r(j-1)>0 \wedge{}\right.\\
\left.{}\wedge \Delta r(j)\leq 0\right\}\,.
\label{e14-g}
\end{multline}
  
  Фактически для каждого сегмента иерархии $h\hm=3$ выделяют его центр, 
через который проводят отрезок перпендикулярно потоку. На рис.~6 величины 
яркости изображения собираются в отсчетах забеленного отрезка. Для них 
по~(\ref{e14-g}) оценивается локальный период линий на основе 
автокорреляционной функции по~(\ref{e13-g}), график которой показан на 
рис.~6. Выбор иерархии $h\hm=3$ сокращает число оценок в 64~раза.
  
  Отметим, что формула~(\ref{e14-g}) определяет такой локальный период 
линий, который соответствует экстремуму отсчетов для положительных 
величин автокорреляционной функции во второй положительной полуволне. 
На рис.~6 период линий равен~9.
\begin{center} %fig6
\vspace*{3pt}
\mbox{%
 \epsfxsize=72.704mm
 \epsfbox{gud-6.eps}
}
\end{center}
\begin{center}
\vspace*{3pt}
{{\figurename~6}\ \ \small{Определение периода}}
\end{center}
\vspace*{9pt}

%\smallskip
\addtocounter{figure}{1}

\noindent
 Выбор экстремума <<центрирует>> маску 
фильт\-ра, применяемого для фильтрации ДИ. Однако предположение о том, что 
оценка периода линий $t_h^{(0)}(x,y)$ может быть смещена, оставляет 
пространство для маневрирования параметрами фильтрации. На ровном фоне 
изображения элементы мат\-ри\-цы $T_h^{(0)}$ нулевые.


  \subsubsection*{Анализ и коррекция матрицы периодов линий}
  
  Имея в виду то, что 
отношение значения в точке первого максимума $r(t_h^{(0)}(x,y))$ к 
постоянной составляющей автокорреляционной функции~$r(0)$ для функции 
синус равно~1, определим матрицу достоверности периода в виде
  \begin{equation}
  Z_h^{(0)} =\left\{ \zeta_h^{(0)}(x,y)\equiv \fr{r(t_h^{(0)}(x,y)}{r(0)}\right\}\,.
  \label{e15-g}
  \end{equation}
  
  Для ДИ с разрешением 500~dpi $4\leq t_h^{(0)}(x,y)\hm\leq 17$~\cite{1-g}. 
Это позволяет фильтровать ошибки распознавания локального периода, задавая 
$t_h^{(0)}(x,y)\hm=0$.
  
  Суть процедуры сводится к расчету матрицы периодов линий 
$T_h^{(1)}\hm=\left\{ t_h^{(1)}(x,y)\right\}$, которая в начальной итерации 
с номером $j\hm=0$ инициализируется: $T_h^{(1)}\hm=T_h^{(0)}$. Далее 
номер~$j$ итерации инкрементируется. В~первой итерации для 
$t_n^{(1)}(x,y)\not\in \{0\}$ период линий сглаживается по формуле:
  \begin{equation}
  t_{h,j}^{(1)}(x,y) =\fr{\sum\limits_R t_{h,j-
1}^{(1)}(u,v)\zeta_h^{(0)}(u,v)}{\sum\limits_R \zeta_h^{(0)}(u,v)}\,,
  \label{e16-g}
  \end{equation}
где условие суммирования $R=t_{h,j-1}^{(1)}(u,v)\hm>0$ элементов апертуры 
$(u,v)\in A_h(x,y,1)$ определяется  по~(\ref{e4-g}); $n=\sum\limits_R 1$~--- количество 
ненулевых элементов в апертуре. В~последующих итерациях для каждого 
отсчета с кодом пропуска $t_n^{(1)}(x,y)\in \{0\}$ и для смежных с ним 
ненулевых элементов количеством~$n$, если $n\hm>4$, период линий 
прогнозируется по~(\ref{e16-g}). Число итераций ограничивают 
величиной~2--3. Если ограничение снять, то б$\acute{\mbox{о}}$льшая часть элементов 
из~$T_h^{(1)}$ определится.
\pagebreak

\begin{center} %fig7
\vspace*{1pt}
\mbox{%
 \epsfxsize=79mm
 \epsfbox{gud-7.eps}
}
\end{center}
%\begin{center}
\vspace*{3pt}
{{\figurename~7}\ \ \small{Результирующая матрица периодов~(\textit{а}) и матрица их достоверностей~(\textit{б})}}
%\end{center}
\vspace*{14pt}

%\smallskip
\addtocounter{figure}{1}


  Таким образом, ошибки измерений фильтруются, периоды линий 
сглаживаются и в финале прогнозируются. Результат коррекции матрицы 
локальных периодов линий представлен на рис.~7. Нулевые значения периодов 
показаны черным цветом. Большие значения периодов окрашены светлее.



\subsection{Сегментация} %3.5
  
  Сегментация необходима для отделения информативных областей ДИ от 
неинформативных. Она выполняется на той же иерархии $h\hm=3$ и 
заключается в расчете матрицы меток $C_h^{(0)}\hm= \left\{ 
c_h^{(0)}(x,y)\right\}$ по формуле:
  \begin{multline*}
  c_h^{(0)} (x,y) ={}\\
  {}=
  \begin{cases}
  1\,, &\hspace*{-2mm} \mbox{если}\ k_1\zeta_h^{(1)}(x,y)+k_2\mu_h^{(1)}(x,y)>\kappa_0\,;\\
  0 & \hspace*{-2mm} \mbox{иначе}\,.
  \end{cases}
%  \label{e17-g}
  \end{multline*}
где $\zeta_h^{(1)}(x,y)$~--- величина достоверности периода по~(\ref{e15-g}), 
сглаженная с помощью маски размером $3\times 3$; $\mu_h^{(1)}(x,y)$~--- 
когерентность потоков по~(\ref{e10-g}), сглаженная с помощью маски 
размером $3\times 3$; $k_0$, $k_1$ и $k_2$~---  обучаемые коэффициенты.
  
  Фактически при выделении информативных областей опираются на два 
признака: корреляционную функцию и когерентность потоков. Эти признаки 
сами по себе комплексные. Их сочетание позволяет повысить точность 
сегментации.
  
  При сегментации могут образовываться островки <<разнородных>> 
областей. Их можно дополнительно классифицировать операциями 
морфологической обработки изображения~\cite{4-g, 5-g}. Однако из-за 
ограничения по времени это нежелательно.

\subsection{Бинаризация} %3.6
  
  Бинаризация опирается на ранее вычисленные данные, а именно: на матрицу 
периодов~$T_h^{(1)}$ и матрицу меток~$C_h^{(0)}$. Каждый элемент 
сегмента $S_h(x,y)$ изображения $f_0^{(1)}(x,y)$ с меткой $c_h^{(0)}(x,y)\hm\in 
\{1\}$ бинаризуют следующим образом. Если значение элемента 
$f_0^{(1)}(x,y)$ меньше среднего значения $m$ элементов в круговой апертуре 
$A_h(x,y,w)$, умноженного на некоторый коэффициент~$k$ (в реализации 
0.98), то элементу присваивается значение~1, иначе~--- 0, т.\,е. 
  $$
  f_0^{(2)} (x,y) =
  \begin{cases}
  1\,, &\ \mbox{если}\ f_0^{(1)}(x,y)<km\,;\\
  0 & \ \mbox{иначе}\,.
  \end{cases}
  $$


\begin{center} %fig8
\vspace*{9pt}
\mbox{%
 \epsfxsize=60mm
 \epsfbox{gud-8.eps}
}
\end{center}
\begin{center}
\vspace*{1pt}
{{\figurename~8}\ \ \small{Бинаризованное изображение}}
\end{center}
\vspace*{11pt}

%\smallskip
\addtocounter{figure}{1}

%\vspace*{-9pt}

  Среднее значение рассчитывается с помощью интегральных изображений по 
формулам~(\ref{e6-g}) и~(\ref{e7-g}), что позволяет заметно ускорить процесс 
бинаризации. Размер апертуры~$w$ для каждой точки изображения берется из 
матрицы периодов~$T_h^{(1)}$. Результат бинаризации представлен на рис.~8.

\subsection{Скелетизация и распознавание частных признаков} %3.7

\begin{figure*} %fig9
 \vspace*{1pt}
 \begin{center}
 \mbox{%
 \epsfxsize=150mm
 \epsfbox{gud-9.eps}
 }
 \end{center}
 \vspace*{-9pt}
\Caption{Исходное изображение~(\textit{а}), частные признаки~(\textit{б}) 
и скелет изображения~(\textit{в})}
\end{figure*}


  Линии бинарного ДИ утончаются до скелетных (рис.~9). Введем 
некоторые определения.
  
%  \medskip
  
  \noindent
  \textbf{Определение 2.} Скелетом линии называется прос\-тая цепь $\langle 
u,v\rangle$ с вершинами~$u$ и~$v$ в 8-смеж\-ности, проходящая вблизи 
геометрического центра линии, причем для каждой вершины $p_1\in\langle 
u,v\rangle$ существует ровно две смежные с ней вершины~$p_2$ и~$p_3$, при 
этом вершины~$p_2$ и~$p_3$ несмежные.
  
  \smallskip
  
  \noindent
  \textbf{Определение 3.} Окончанием называется такая вершина~$p_1$ 
скелета, что для вершины~$p_1$ существует ровно одна смежная с ней 
вершина~$p_2$.
  
  \smallskip
  
  \noindent
  \textbf{Определение 4.} Разветвлением называется такая вершина~$p_1$ 
скелета, что для вершины~$p_1$ существуют ровно три смежные с ней 
вершины~$p_2$, $p_3$ и~$p_4$, при этом любые две вершины из множества 
$\{p_2, p_3, p_4\}$ попарно несмежные.


  Скелетизация опирается на раскрашивание точек линий $f_0^{(2)}(x,y)\in 
\{0\}$ по правилам $P(\xi(x,y))$, определяемым в специальной табличной 
форме на основе идентификатора окрестности точки в виде
  $$
  \xi(x,y) =\sum\limits_{i\in I} f(i)\cdot 2^i\,,
  $$
где $f(i)$ принимает значение~1 для линии и~0 в противном случае; $i\in 
I=0, \ldots  ,7$~--- \mbox{номер} сектора апер\-ту\-ры $3\times 3$ по~(\ref{e4-g}). Величина 
$\xi(x,y)\hm\in 0,\ldots , 255$ определяет ячейку в табличной форме. 
Согласно~\cite{6-g}, итерационное применение правил из $P(\xi(x,y))$ 
позволяет вычислить скелет линий, показанный на рис.~9. С~вершин скелета 
как графа~\cite{7-g} считываются окончания и разветвления, располагающиеся 
в информативной области изоб\-ра\-же\-ния на достаточном расстоянии от ее 
границы, и помещаются в список~(\ref{e5-g}). Затем применяется структурная 
пост\-обра\-бот\-ка скелета~\cite{1-g}. Частные признаки и их направления 
показаны на рис.~9, причем окончания окрашены черным цветом, а 
разветвления~--- серым.

\vspace*{6pt}

\section{Заключение}
  
  В~статье предложена группа взаимосвязанных методов, обеспечивающая 
приемлемое качество распознавания частных признаков при жестких 
ограничениях на время обработки ДИ. К~ним относятся: измерение и 
коррекция матриц потоков, сглаживание изображения, измерение и коррекция 
матриц периодов линий, сегментация, бинаризация, скелетизация и считывание 
с вершин скелета частных признаков. Построение матриц потоков основано на 
тензорном анализе простых окрестностей, а матриц периодов линий~--- на 
автокорреляционной функции. Общее время обработки составляет 
приблизительно 100~мс (изображение $320 \times 480$~пикселей, процессор 
Intel Pentium~4 CPU 3.0~ГГц). 
  
  Обработка основана на операции свертки, что с учетом временн$\acute{\mbox{ы}}$х 
характеристик позволяет перенести ее на целевые платы TMS или 
процессоры DSP~\cite{8-g} и использовать встроенные в них операции 
свертки. Последнее позволяет реализовать простые портативные 
биометрические системы, удовлетворяющие достаточно жестким требованиям 
к производительности.
  
  Аналогично процедуре сглаживания в качестве фильтра может быть 
использован одномерный фильтр Габора. Это повысит качество обработки, но 
ухудшит производительность. Кроме того, полученное бинарное изображение 
можно сгладить, что обычно повышает качество обработки. Дальнейшее 
направление развития быстрой обработки видится в улучшении метода 
сегментации изображения, т.\,е.\ в более качественном определении 
информативных зон изображения.

{\small\frenchspacing
{%\baselineskip=10.8pt
\addcontentsline{toc}{section}{Литература}
\begin{thebibliography}{9}

\bibitem{1-g}
\Au{Maltoni~D., Maio~D., Jain~A.\,K., Prabhakar~S.} Handbook of fingerprint 
recognition.~--- New York: Springer-Verlag, 2009. 494~p.

\bibitem{2-g}
\Au{Bolle~R.\,M., Connel~J.\,Y., Pankanti~S., Ratha~N.\,K.}
Guide to biometrics.~--- New York: Springer-Verlag, 2004. 368~p.

\bibitem{3-g}
\Au{Гудков~В.\,Ю.} Методы первой обработки дактилоскопических 
изображений.~--- Миасс: Геотур, 2008. 127~с.

\bibitem{4-g}
\Au{Гонсалес~Р., Вудс~Р.} Цифровая обработка изображений~/ Пер. с англ.~--- 
М.: Техносфера, 2006. 1072~c.

\bibitem{5-g}
\Au{Яне~Б.}
Цифровая обработка изображений~/ Пер. с англ. А.\,М.~Измайловой.~--- М.: 
Техносфера, 2007. 584~с.

\bibitem{6-g}
\Au{Гудков~В.\,Ю., Коляда~А.\,А. , Чернявский~А.\,В.}
Новая технология формирования скелетов дактилоскопических изображений~// 
Методы, алгоритмы и программное обеспечение гибких информационных 
технологий для автоматизированных идентификационных систем.~--- Минск: 
БГУ, 1999. С.~71--82.

\bibitem{7-g}
\Au{Новиков~Ф.\,А.}
Дискретная математика для программистов: Учебник для вузов.~--- 3-е изд.~--- 
СПб.: Питер, 2008. 384~с.

\label{end\stat}

\bibitem{8-g}
\Au{Сергиенко~А.\,Б.}
Цифровая обработка сигналов.~--- СПб.: Питер, 2002. 608~с.
 \end{thebibliography}
}
}


\end{multicols}         %8

\def\stat{karateev}


\def\tit{АВТОМАТИЗИРОВАННЫЙ КОНТРОЛЬ КАЧЕСТВА ЦИФРОВЫХ 
ИЗОБРАЖЕНИЙ ДЛЯ ПЕРСОНАЛЬНЫХ ДОКУМЕНТОВ$^*$}
\def\titkol{Автоматизированный контроль качества цифровых 
изображений для персональных документов}

\def\autkol{С.\,Л. Каратеев, И.\,В.~Бекетова, М.\,В.~Ососков, В.\,А.~Князь, Ю.\,В.~Визильтер, 
А.\,В.~Бондаренко, С.\,Ю.~Желтов}
\def\aut{С.\,Л.~Каратеев$^1$, И.\,В.~Бекетова$^2$, М.\,В.~Ососков$^3$, В.\,А.~Князь$^3$, Ю.\,В.~Визильтер$^3$, 
А.\,В.~Бондаренко$^3$, С.\,Ю.~Желтов$^3$}

\titel{\tit}{\aut}{\autkol}{\titkol}

{\renewcommand{\thefootnote}{\fnsymbol{footnote}}\footnotetext[1]
{Работа выполнена при поддержке РФФИ, проект 09-07-13551-офи\_ц.}}

\renewcommand{\thefootnote}{\arabic{footnote}}
\footnotetext[1]{ФГУП <<Государственный научно-исследовательский институт авиационных систем>>, goga@gosniias.ru}
\footnotetext[2]{ФГУП <<Государственный научно-исследовательский институт авиационных систем>>, irus@gosniias.ru}
\footnotetext[3]{ФГУП <<Государственный научно-исследовательский институт авиационных систем>>}

\vspace*{6pt}

\Abst{Приведено описание программно-аппаратного комплекса для автоматизации процесса получения и контроля 
цифровых фотографий для биометрических документов: общая структура комплекса, функции, аппаратное и 
алгоритмическое обеспечение. Алгоритмическое обеспечение комплекса включает: детектор лица, модуль оценки 
яркостных и цветовых характеристик; детектор открытых/закрытых глаз; детектор очков; детектор бликов и теней; 
детектор основных элементов лица (рот, нос, брови), детектор поворотов и наклонов головы. Проведено исследование 
точности используемых алгоритмов косвенной оценки углов поворота лица по монокулярному цифровому изображению. 
Для этого разработана специальная методика, основанная на использовании синтезированных ракурсных изображений 
реальных лиц, реконструированных по результатам трехмерного сканирования.}

\vspace*{2pt}

\KW{биометрия; персональные документы; обнаружение лиц; бустинг; трехмерное сканирование; трехмерное 
моделирование}
\vspace*{11pt}

     \vskip 18pt plus 9pt minus 6pt

      \thispagestyle{headings}

      \begin{multicols}{2}

      \label{st\stat}

      
%      \vspace*{6pt}

\section{Введение}

Введение стандарта на цифровые фотографии лица определяет необходимость автоматизации операций контроля качества 
изображений лиц как непосредственно в процессе получения этих изоб\-ра\-же\-ний, 
так и на любом этапе подготовки паспортных, 
визовых и иных персональных доку\-ментов.
{\looseness=1

}

Для обеспечения согласованности национальных стандартов цифровых фотографий международной организацией по 
стандартизации были выработаны рекомендации ISO/IEC FCD
\mbox{19794-5}. В~России стандартом, определяющим требования к 
изображениям лиц для биометрических документов, является ГОСТ ИСО/МЭК 19794-5-2006.
{\looseness=1

}

Для автоматизации процесса получения циф\-ро\-вых фотографий, удовлетворяющих основным требованиям и рекомендациям 
ГОСТ ИСО/МЭК 19794-5-2006, в ГосНИИАС был разработан описанный в данной статье программно-аппаратный комплекс. 
Комплекс обеспечивает получение цифро\-вых фотографий лица, а также оценку в реальном времени основных характеристик 
изображения и параметров лица, что позволяет оператору с мини-\linebreak\vspace*{-12pt}
\columnbreak

\noindent
мальными усилиями, не превышающими усилия, необходимые 
для получения обычной качественной фотографии лица, получать циф\-ро\-вые фотографии лиц, гарантированно удовле\-тво\-ря\-ющие 
требованиям данного ГОСТа. Кроме того, мобильный комплекс может быть использован для контроля параметров фотографий 
лиц, полученных от других источников изображений,~--- как в цифровом, так и в бумажном виде, предоставляя возможность 
оценки пригодности фотографий для последующей биометрической обра\-ботки.
{ %\looseness=1

}

В данной статье также описаны результаты исследования возможной точности алгоритмов косвенной оценки углов поворота лица 
по монокулярному цифровому изображению. Поскольку в соответ\-ствии с требованиями ГОСТ ИСО/МЭК 19794-5-2006 на 
цифровой фотографии лица в биометрических документах допустимый угол поворота изоб\-ра\-же\-ния лица (головы) вокруг любой 
из осей координатной системы не должен превышать~5$^\circ$,\linebreak задача оценки возможной точности измерения 
пространственной ориентации головы на цифровом изоб\-ра\-же\-нии представляет существенный интерес, равно как и задача выбора 
наиболее надежно работающих косвенных методов оценки углов по\-во\-рота.
{\looseness=1

}

\section{Требования к цифровым фотографиям для биометрических документов}

ГОСТ ИСО/МЭК 19794-5-2006 определяет основные требования и дополняющие их рекомендации к цифровым изображениям 
лица и форматам сохранения данных. Общий вид и основные геометрические характеристики фотографии лица приведены на 
рис.~1.

\begin{center} %fig1
\vspace*{18pt}
\mbox{%
\epsfxsize=68.242mm
\epsfbox{kar-1.eps}
}
\end{center}
\vspace*{6pt}
{{\figurename~1}\ \ \small{Основные геометрические характеристики изображения лица}}
%\end{center}
%\vspace*{6pt}


\bigskip
\addtocounter{figure}{1}

Основные характеристики изображения лица на цифровой фотографии для биометрических документов должны соответствовать 
следующим требованиям:
\begin{itemize}
\item минимальный размер фотографии \mbox{$525\;\times$}\linebreak $\times\;420$~пикселей;
\item изображение лица на фотографии должно быть фронтальным и не иметь отклонения относительно основных осей более чем 
на 5$^\circ$;
\item соотношение ширины фотографии и ширины головы ($A$:$CC$) должно быть не менее 7:5 и не более чем 2:1;
\item расстояние от нижней границы фотографии до горизонтальной линии, проходящей через центры глаз ($BB$), должно 
составлять 50\%--70\% от высоты полного изображения;
\item площадь лица на фотографии должна со\-став\-лять 70\%--80\% от площади фотографии;
\item цвет и яркость фона должны обеспечивать надежное определение контура головы;
\item на изображениях лиц не должно быть световых бликов и сильного затенения;
\item на фоне не должно быть теней от головы или каких-либо предметов;
\item на изображениях лица не должно быть закрытых глаз, предметов, закрывающих глаза и лицо или искажающих черты лица.
\end{itemize}

Как видно из вышеприведенных требований, проверка цифровых изображений лица на соответствие требованиям ГОСТа 
является весьма нетривиальной задачей. Здесь недостаточно визуального анализа фотографии для принятия решения о\linebreak 
ее  пригодности для использования в документах,
 удос\-то\-ве\-ря\-ющих личность. Поэтому при создании
 мобильного комплекса 
биометрической\linebreak регистрации изображений лиц было разработано специ\-ализированное программное обеспече-\linebreak ние, 
автоматизирующее процесс создания и\linebreak контроля фотографий, удовлетворяющих требованиям \mbox{ГОСТа}.

\section{Алгоритмическое обеспечение комплекса}

На основании измеренных и рассчитанных характеристик изображения лица алгоритмическое обеспечение осуществляет 
диагностику наличия и причин отклонений от требований ГОСТа и вывод сообщений об этих отклонениях и их возможных 
причинах.

Для получения оценок основных параметров изображения лица в автоматическом режиме решаются следующие задачи:
\begin{itemize}
\item автоматическое обнаружение лица на изображении;
\item автоматическое определение контура и оценка параметров лица на изображении;
\item автоматическое обнаружение глаз на изображении и оценка координат центров зрачков;
\item обнаружение бликов и областей сильной затененности на изображении лица;
\item формирование фронтальных и условно-фрон\-таль\-ных цветных и монохромных изображений заданного размера для печати 
фотографий;
\item формирование изображений для систем обмена биометрическими данными.
\end{itemize}

Алгоритм автоматического обнаружения об\-ласти лица является вариантом каскадного детектора, обучаемого с помощью метода 
Adaboost~\cite{1kor, 2kor}. В~алгоритме слабые (weak) классификаторы построены на основе фильтров Хаара, однако отклик 
формируется с использованием аппроксимации распределения вероятностей амплитуды откликов.\linebreak
 Аппроксимация распределения 
вероятностей откликов представляется в виде гистограммы, которая строится по взвешенным примерам, в результате чего 
обучение слабых классификаторов\linebreak
 проводится на подвыборках одного и того же фиксированного размера, но имеющих 
различные распределения обучающих изображений. В~процессе обучения при формировании каскадного классификатора для 
каж\-до\-го последующего классификатора признаковое пространство сокращается за счет устранения признаков, на которых 
построены предыдущие классификаторы. Поскольку каждый\linebreak
 следующий классификатор обучается в другом\linebreak
  подпространстве 
признаков, он обладает уточ\-ня\-ющи\-ми свойствами и работает по принципу последовательных приближений. Классификатор 
представляет собой линейную комбинацию слабых классификаторов, число которых варьируется от~5 до~75. Классификаторы 
объединяются в каскадную структуру, число слоев каскада варьируется от~4 до~8 в зависимости от желаемого уровня 
соотношения ошибок первого и второго рода. Математическое моделирование показало~\cite{3kor}, что при работе по случайно 
выбранной совокупности тестовых изоб\-ра\-же\-ний алгоритм автоматического обнаружения изображений лица обеспечивает 
вероятность правильного обнаружения не менее~0,95 при веро\-ят\-ности ложного обнаружения лица не более~0,01. 

Алгоритм автоматического определения контура лица на изображении построен на использовании информации об оттенках кожи 
человека.\linebreak
Об\-ластью интереса алгоритма является фрагмент изоб\-ра\-же\-ния, классифицированный как лицо каскадным 
классификатором Adaboost. Каждому пикселю цветного RGB-фрагмента изображения ставится в соответствие вектор параметров 
цвета (H, S, V) в цветовом пространстве HSV ($H$ue, $S$aturation, $V$alue~--- цветовой тон, насыщенность, яркость)~\cite{6kor}. 
Распределение оттенков кожи представлено бинарной картой, хранящейся в структуре данных типа <<просмотровой таблицы>> 
(Look-Up-Table, LUT). Сегментация фрагмента изображения выполняется путем проверки принадлежности параметров цвета 
пикселей к кластеру модели оттенков кожи с помощью операции поиска по таблице. Область изображения кожи формируется из 
пикселей, векторы параметров которых вошли в один из кластеров. Для формирования одной однородной области пикселей, по 
цвету соответствующих оттенкам кожи, и удаления мелких областей, линий и отдельных пикселей к изображению применяются 
такие операции математической морфологии, как дилатация и эрозия. Линии контура лица формируются с помощью алгоритма 
сплайн-интерполяции к координатам граничных пикселей области кожи лица. Размеры прямоугольника, в который вписаны 
линии контура лица, являются линейными размерами изображения лица. На рис.~2 приведены результаты работы 
алгоритма цветовой сегментации изображения лица. Белым цветом отображены пиксели, классифицированные как кожа.

\begin{center} %fig2
\vspace*{4pt}
\mbox{%
\epsfxsize=78.293mm
\epsfbox{kar-2.eps}
}
\end{center}
\vspace*{2pt}
{{\figurename~2}\ \ \small{Пример работы алгоритма цветовой пиксельной сегментации кожи лица: (\textit{а})~
исходное изображение; (\textit{б})~результат выделения кожи на изображении}}
%\end{center}
%\vspace*{6pt}


\bigskip
\addtocounter{figure}{1}


Задача обнаружения изображений глаз решается как задача поиска и распознавания на цифро\-вом изображении лица локальных 
областей, об\-ладающих специфическими характерными для\linebreak
 изображений глаз параметрами. Областью интереса алгоритма 
является часть изображения лица, пред\-став\-ля\-ющая собой область ожидаемого расположения глаз. На основе статистической 
модели проводится предварительная оценка ожидаемого положения и размеров глаз, благодаря чему существенно возрастает 
вычислительная эффективность алгоритма. Алгоритм автоматического обнаружения области глаз с помощью каскадного 
классификатора Adaboost определяет координаты глаз в пределах радужной оболочки. Для точной локализации изображений глаз 
производится поиск координат центров зрачков с использованием операций морфологической фильтрации. Морфологический 
фильтр выделяет изображение зрачка и радужной оболочки глаза, устраняя при этом шумовые помехи и артефакты изображения, 
например блики. Координаты центров зрачков определяются путем свертки изображения с круговым фильтром, 
подчеркивающим форму зрачка. Проведенное математическое моделирование показало, что при работе по случайно выбранной 
совокупности тестовых изображений разработанный алгоритм автоматического обнаружения и локализации изображений глаз 
обеспечивает вероятность правильного обнаружения и локализации не менее~0,95 при ве\-ро\-ят\-ности ложного обнаружения глаз не 
более~0,01. Пример работы алгоритмов обнаружения изображения лица, выделения контура лица и обнаружения глаз 
представлен на рис.~3.

Определение бликов и областей затенения осуществляется путем анализа пространственных распределений яркостей на 
последовательности изоб\-ра\-же\-ний лица, сглаженных окнами различных\linebreak
 размеров. При этом бликом считается появление 
связанной области площадью более 1\% от площади лица, в которой все компоненты RGB равны и превышают по 
амплитуде~250, а признаком затенения является наличие на изображении лица об\-ласти размером более 10\% от площади лица, 
яркость которой отличается более чем на 10\% от яркости симметричной ей об\-ласти изображения. В~связи с тем, что яркостные 
отличия могут быть вызваны причинами, не связанными с наличием тени, например
 дефектами лица или небритостью, признак 
затенения носит рекомендательный характер и не влияет на вывод о соответствии изображения лица требованиям ГОСТа. 
\begin{center} %fig3
%\vspace*{-2pt}
\mbox{%
\epsfxsize=70mm
\epsfbox{kar-3.eps}
}
\end{center}
\vspace*{6pt}
{{\figurename~3}\ \ \small{Пример работы алгоритмов обнаружения лица и обнаружения глаз}}
%\end{center}
%\vspace*{6pt}


\bigskip
\addtocounter{figure}{1}

После проведения основных операций над изоб\-ра\-же\-нием лица выполняется анализ полученных данных на соответствие ГОСТу. 
Для этого производится расчет оценок характеристик изображения, производных от геометрических параметров лица, и проверка 
наличия артефактов на самом изображении. Выполняются следующие операции:
\begin{itemize}
\item определение оси симметрии лица;
\item определение центровки изображения лица;
\item определение угла поворота лица;
\item определение угла наклона лица;
\item обнаружение очков.
\end{itemize}

На заключительном этапе интерпретации\linebreak
результатов проводится проверка оцененных параметров изображения на соответствие 
требованиям стандарта. В случае несоответствия вычисленных параметров изображения требованиям стандарта выдаются 
рекомендации по изменению условий съемки. В случае формирования условно-фрон\-таль\-ных изображений система выполняет 
необходимые повороты и перемасштабирование изображения.

\section{Конструктивные особенности программно-аппаратного комплекса}
Одним из основных приоритетов при выборе конструкции комплекса было требование создания наиболее простой, мобильной и 
сравнительно дешевой конструкции, состоящей из широкодоступных готовых компонентов.

Пограммно-аппаратный комплекс включает:
\begin{itemize}
\item персональный компьютер;
\item цифровой фотоаппарат;
\item источник освещения;
\item специальный штатив для крепления фотоаппарата и источника освещения;
\item планшетный сканер;
\item специализированное программное обеспечение.
\end{itemize}

Комплекс обеспечивает выполнение сле\-ду\-ющих основных функций:
\begin{itemize}
\item захват (оцифровка) и отображение на мониторе последовательности изображений лица, получаемых от цифрового 
фотоаппарата в реальном времени;
\item сохранение изображений на жестком диске компьютера;
\item загрузка и отображение изображений с жесткого диска компьютера;
\item обнаружение изображений лиц, близких к фронтальному положению;
\item обнаружение глаз, определение контура лица, вычисление осей симметрии;
\item определение центровки изображения лица;
\item определение размеров изображения головы;
\item определение углов наклона и поворота головы;
\item обнаружение очков на изображении;
\item оценка качества изображения~--- наличие теней, бликов, оценка цвета, яркости и текстуры фона;
\item сравнение измеренных и вычисленных па\-ра\-мет\-ров изображения лица с требованиями стандартов;
\item индикация результатов сравнения в виде пиктограмм и текстовых сообщений;
\item выбор изображения, удовлетворяющего требованиям стандартов (автоматически или вручную);
\item вывод изображения на печать в заданном формате.
\end{itemize}


При установке системы предлагается выбор используемого устройства видеоввода. В качестве такого устройства может 
использоваться цифровой фотоаппарат, имеющий программный интерфейс с компьютером, или сканер. Кроме этого, в качестве 
источника данных может использоваться любой внешний носитель информации, содержащий массивы цифровых фотографий в 
форматах BMP или JPEG. На рис.~4 представлен общий вид комплекса.


Программное обеспечение работает под управ\-лением ОС Windows 2000/XP. Интерфейс программы представляет собой 
диалоговое окно, в котором помимо изображения текущей фотографии также отображаются результаты проверки требований к 
изображению лица в виде пиктограмм и текстовой информации. Если полученное изображение имеет отклонения от норм ГОСТа, 
оператор получает визуальное и звуковое оповещение. При этом изображения на пиктограммах и соответствующие текстовые 
сообщения подсказывают ему причину
 ошибки. Каждая из пиктограмм, имеющихся в окне программы, соответствует одному из 
приведенных выше требований ГОСТа по характеристикам изображения лица и фотографии. Результаты 
проверки отображаются 
в виде текстовых сообщений\linebreak\vspace*{-12pt}
%\noindent
\begin{center} %fig4
\vspace*{4pt}
\mbox{%
\epsfxsize=79.8mm
\epsfbox{kar-4.eps}
}
%\end{center}

\vspace*{9pt}
{{\figurename~4}\ \ \small{Программно-аппаратный комплекс в сборе}}
\end{center}
\vspace*{-4pt}


\bigskip
\addtocounter{figure}{1}

\noindent
в специальном окне. Кроме этого, для каждой обработанной фотографии программа сохраняет 
результаты всех проверок, выполненных в процессе обработки.


\section{Примеры работы комплекса}

На рис.~5 показаны примеры нескольких типовых ошибок, возникающих при получении циф\-ро\-вых 
фотографий для биометрических документов, и результаты обработки этих изображений про\-грам\-мным обеспечением комплекса.


На приведенных рисунках хорошо видны особенности графического интерфейса. При обнаружении отклонений от требований 
ГОСТа возле ряда\linebreak пиктограмм, содержащего пиктограмму, соответствующую данному отклонению от требований, появляется 
информирующий тревожный сигнал красного цвета, сопровождаемый звуковым сигналом.\linebreak При этом подсвечивается 
соответствующая пиктограмма и появляется текстовая надпись в информационном окне с указанием ошибки и ее фактического 
значения (если это возможно). В случае\linebreak выявления артефактов, неоднозначно трактуемых или нежелательных, но не 
запрещенных ГОСТом, выводится предупредительная визуальная сигнализация на изображении лица и делается 
соответствующая запись в информационном окне.

\section{Методика оценки точности определения пространственной ориентации головы}

Вообще говоря, построение связанной трехмерной системы координат на основании двухмерного\linebreak\vspace*{-12pt}
\pagebreak

\begin{center} %fig5
\vspace*{2pt}
\mbox{%
\epsfxsize=80mm
\epsfbox{kar-5.eps}
}
\end{center}
\vspace*{6pt}
{{\figurename~5}\ \ \small{Несоответствие требованиям: закрытые глаза~(\textit{а}); 
поворот головы~(\textit{б}); наклон головы~(\textit{в})}}
%\end{center}
%\vspace*{6pt}


%\bigskip
\addtocounter{figure}{1}
\columnbreak

\noindent
 изображения лица является 
некорректной обратной задачей. Поэтому в системах подготовки и контроля цифровых фотографий для биометрических 
документов для определения углов поворота относительно пространственных осей используются косвенные методы оценки углов 
поворота головы по изменению пропорций лица, как показано на рис.~5,\,\textit{б}  и~\textit{в}.
Вследствие этого важной 
проблемой является анализ точности определения углов поворота лица, основанного на косвенных методах измерений, и выбор 
наиболее надежно работающих косвенных методов оценки углов поворота.

Координатная система, рекомендованная в стандарте, представлена на рис.~6. Здесь $XYZ$~--- правая система 
координат с центром в точке, соответствующей кончику носа на изображении лица. Углы поворотов определяются 
относительно неподвижной системы координат~$XYZ$, соответствующей полнофронтальному изображения 
лица с углами поворота~(0, 0, 0). 
При этом точного %\linebreak\vspace*{-12pt}
%\noindent
 определения фронтального лица в ГОСТе не приводится. Для определения истинных углов поворота 
изображения необходимо ввести две системы координат: связанную с головой и опорную. Связанная с головой система 
координат $XsYsZs$ определяется двумя ортогональными плоскостями: франкфуртской (или глазнично-ушной) го\-ри\-зон\-талью~--- 
плос\-костью~$XsZs$, проходящей через верхние края отверстий наружного слухового прохода и нижнюю точку нижнего края 
левой орбиты, и плос\-костью~$YsZs$, перпендикулярной плос\-кости~$XsZs$ и проходящей через ось симметрии лица (центр 
переносицы, центр губ). Ось~$Zs$ совпадает с линией пересечения плоскостей и направлена от поверхности лица. Опорная 
система координат~$XYZ$\linebreak

\noindent 
\begin{center} %fig6
%\vspace*{6pt}
\mbox{%
\epsfxsize=66.117mm
\epsfbox{kar-8.eps}
}
\end{center}
\vspace*{2pt}
{{\figurename~6}\ \ \small{Система координат для определения  углов поворота лица}}
%\end{center}
%\vspace*{-6pt}


\bigskip
\addtocounter{figure}{1}
\pagebreak


\noindent
определяет виртуальную камеру, формирующую двумерное изображение, и представляется следующим 
образом: ось~$Y$ параллельна вертикальной оси плоскости изображения, ось~$X$ параллельна горизонтальной оси плоскости 
изображения, ось~$Z$ дополняет до правой системы координат и на\-прав\-ле\-на от модели лица, центр расположен в середине 
двумерного изображения. Изображение считается фронтальным, когда оси опорной и связанной систем координат коллинеарны.


Для оценки точности определения углов поворота изображения лица была предложена методика, основанная на использовании 
трехмерных моделей лиц, получаемых путем трехмерного сканирования. Первым шагом в оценке точности является получение 
трехмерной модели лица человека со строго вертикальным расположением головы. Эта модель считается базовой, и к ней 
привязывается опорная система координат. На следующих этапах исследования модель поворачивается вокруг осей опорной 
системы координат на заданные углы и с нее строится проекция на фокальную плоскость. Эта проекция и считается 
изображением повернутого лица, по которому осуществляются оценки углов поворота изображения. На рис.~7 
и~8 показаны трехмерные модели и соответствующие им синтетические изображения фронтального и повернутого 
лица, созданные по предлагаемой методике.

\begin{center} %fig7
\vspace*{12pt}
\mbox{%
\epsfxsize=79.88mm
\epsfbox{kar-9.eps}
}
\end{center}
\vspace*{2pt}
{{\figurename~7}\ \ \small{Трехмерная модель и синтетическое изображение фронтального лица}}
%\end{center}
%\vspace*{-6pt}


\bigskip
\addtocounter{figure}{1}

\begin{center} %fig8
\vspace*{6pt}
\mbox{%
\epsfxsize=79.88mm
\epsfbox{kar-10.eps}
}
\end{center}
\vspace*{2pt}
{{\figurename~8}\ \ \small{Трехмерная модель и синтетическое изображение искусственно развернутого лица}}
%\end{center}
%\vspace*{-6pt}


\bigskip
\addtocounter{figure}{1}


Для получения трехмерной модели лица используется специализированный комплекс бесконтактных измерений, построенный на 
фотограмметрических принципах, позволяющих рассчитать \mbox{координаты} заданной точки объекта по двум разноракурсным 
изображениям объекта, наблюдаемого стереосистемой видеокамер.

Для применения в задачах антропометрии сис\-те\-ма бесконтактных измерений должна удовлетворять ряду специфических 
требований, таких как безопасность и комфортность для объекта съемки (человека) и высокая скорость съемки, необходимая для 
устранения ошибок, вызванных невозможностью долгого сохранения неподвижности человеком. Кроме того, система 
бесконтактных антропометрических измерений должна измерять координаты поверхности с высоким разрешением и 
представлять их в форме компьютерной трехмерной модели (предпочтительно текстурированной) для последующего анализа. 
Изображение специализированного комплекса бесконтактных измерений приведено на рис.~9.



\begin{center} %fig9
\vspace*{6pt}
\mbox{%
\epsfxsize=80mm
\epsfbox{kar-11.eps}
}
\end{center}
\vspace*{2pt}
{{\figurename~9}\ \ \small{Специализированный комплекс  бесконтактных измерений}}
%\end{center}
%\vspace*{-6pt}


\bigskip
\addtocounter{figure}{1}


Система бесконтактных измерений включает:
\begin{itemize}
\item две камеры для технического зрения, предназначенные для захвата черно-белых изображений человека в 
структурированном свете и расчета трехмерных координат поверхности лица;
\item цветную фотокамеру высокого разрешения для получения цветного изображения и фотореалистичного текстурирования 
трехмерной модели;
\item портативный DLP-проектор для создания ПК-управляемого подсвета, обеспечивающего автоматизацию решения задачи 
стереоотождествления;
\item персональный компьютер.
\end{itemize}

Предварительным этапом работы с фо\-то\-грам\-мет\-ри\-ческим комплексом бесконтактных измерений является 
калибровка~\cite{4kor, 5kor}, т.\,е.\ оценка па\-ра\-мет\-ров модели камер, учитывающих нелинейные\linebreak
 искажения, возникающие при 
формировании изоб\-ра\-же\-ний камерой. Калибровка системы поз\-во\-ля\-ет обеспечить высокую точность измерений трехмерных 
координат объекта.

Внешнее ориентирование системы выполняется для оценки положения и ориентации камер в системе координат, задаваемой 
специальным тестовым объектом. В результате процедуры ориентирования, выполняемой по снимкам тестового сюжета, 
определяются координаты и углы поворота камер в заданной системе координат. В дальнейшем координаты точек поверхности 
объекта рассчитываются в системе координат, заданной ориентированием системы.

Для расчета трехмерных координат поверхности объекта и построения его трехмерной модели необходимо для каждой видимой 
точки объекта найти ее координаты на левом и правом изображениях (решить задачу стереоотождествления точек левого и 
правого изображений). Тогда с использованием результатов ориентирования стереосистемы (положения камер) рассчитываются 
трехмерные координаты точки.

Для автоматизированного решения задачи стереоотождествления соответственных точек изображения с левой и правой камер в 
системе применяется оригинальный кодированный подсвет объекта, минимизирующий число кадров, использующихся для 
расчета трехмерных координат поверхности объекта, при сохранении высокой плотности измерений. 

Основные технические характеристики сис\-темы:
\begin{itemize}
\item время сканирования: $\sim  0{,}5$~с; 
\item время расчета трехмерной модели: 5~с;
\item плотность измерения координат: 10--25 точек на мм$^2$;
\item точность измерения пространственных координат: 0,5~мм.
\end{itemize}

Система выполняет следующие функции: 
\begin{itemize}
\item сканирование и получение необходимого числа снимков лица для последующего использования при построении трехмерной 
модели лица;
\item построение высокоточной трехмерной модели лица;
\item текстурирование полученной трехмерной модели.
\end{itemize}

Для расчета трехмерных координат по\-верх\-ности\linebreak
 используются последовательности снимков с чер\-но-бе\-лых камер, а в качестве 
текстуры служит цветной снимок высокого разрешения, по\-лу\-ча\-емый с циф\-ро\-во\-го фотоаппарата. Текстурирование трехмерной 
модели выполняется автоматически на основе данных ориентирования циф\-ро\-во\-го фотоаппарата.

С помощью описанной методики непосредственных трехмерных измерений были получены оценки точности косвенного 
оценивания па\-ра\-мет\-ров трехмерного позиционирования головы человека по монокулярному изображению, осуществляемого в 
описанной выше системе контроля качества изображений для персональных документов. Полученные оценки точности 
составляют около~1$^\circ$ для углов поворота относительно осей~$Z$ и~$Y$ и около~5$^\circ$ для поворотов относительно 
оси~$X$. Худшая оценка точности при определении угла поворота относительно оси~$X$ связана с высокой вариабельностью 
вертикальных пропорций лица у разных людей.

\section{Заключение}

В статье представлен программно-аппаратный комплекс, предназначенный для автоматизации процесса получения цифровых 
фотографий, удовле\-тво\-ря\-ющих основным требованиям и рекомендациям ГОСТ ИСО/МЭК 19794-5-2006. Комплекс обеспечивает 
получение как фронтальных, так и условно-фрон\-таль\-ных цифровых фотографий лица, а также оценку в реальном времени 
основных характеристик изображения и параметров лица, подготовку изображений к печати с заданными размерами и 
разрешением, формирование изображений в биометрическом формате обмена данными, сохранение изображений в различных 
графических форматах. 

Проведено исследование точности ис\-поль\-зу\-емых алгоритмов косвенной оценки углов поворота лица относительно 
пространственных осей по монокулярному цифровому изображению. Для этого разработана специальная методика, основанная 
на использовании синтезированных ракурсных изоб\-ра\-же\-ний реальных лиц, реконструированных по результатам трехмерного 
сканирования. Разработано алгоритмическое и программное обеспечение для моделирования и измерений, собран 
специализированный комплекс для бесконтактного трехмерного сканирования лиц. С использованием данного комплекса 
получены трехмерные модели и фотореалистические текстуры тестовых лиц. Полученная в результате оценки точность 
измерений составляет в среднем около~1$^\circ$ для углов поворота относительно осей~$Z$ и~$Y$ и около~5$^\circ$ для 
поворотов относительно оси~$X$. Более низкая оценка точности при определении угла поворота относительно оси~$X$ связана с 
высокой вариабельностью вертикальных пропорций лица у разных людей.

{\small\frenchspacing
{%\baselineskip=10.8pt
\addcontentsline{toc}{section}{Литература}
\begin{thebibliography}{9}

\bibitem{1kor}
\Au{Freund Y., Schapire R.}
A short introduction to boosting~// J.~of Japanese Society for Artificial Intelligence, 1999. Vol.~14. No.\,5. P.~771--780.

\bibitem{2kor}
\Au{Viola P., Jones M.}
Robust real time object detection~// IEEE ICCV Workshop Statistical and Computational Theories of Vision, July 2001.

\bibitem{3kor}
\Au{Бекетова И.\,В., Каратеев С.\,Л., Визильтер~Ю.\,В., Бондаренко~А.\,В., Желтов~С.\,Ю.}
Автоматическое обнаружение лиц на цифровых изображениях на основе метода адаптивной классификации AdaBoost~// Вестник 
компьютерных и информационных технологий, 2007. №\,8. С.~2--6.

\label{end\stat}

\bibitem{6kor} %4
\Au{Albiol A., Torres L., Delp~E.\,J.}
Optimum color spaces for skin detection~// Conference (International) on Image Processing Proceedings, 2001. Vol.~1. P.~122--124.

\bibitem{4kor} %5
\Au{Schenk T.\,F.}
Digital photogrammetry.~--- TerraScience, 1999. 

\bibitem{5kor} %6
\Au{Luhmann T., Robson S., Kyle S., Harley I.}
Close range photogrammetry, principles, methods and applications.~--- Whittles, 2006.  510~p. 


 \end{thebibliography}
}
}
\end{multicols} %9

\def\stat{konushin}


\def\tit{АЛГОРИТМ РАСПОЗНАВАНИЯ ЛЮДЕЙ В~ВИДЕОПОСЛЕДОВАТЕЛЬНОСТИ ПО~ОДЕЖДЕ$^*$}
\def\titkol{Алгоритм распознавания людей в видеопоследовательности по 
одежде}

\def\autkol{В.\,C.~Конушин, Г.\,Р.~Кривовязь, А.\,С.~Конушин}
\def\aut{В.\,C.~Конушин$^1$, Г.\,Р.~Кривовязь$^2$, А.\,С.~Конушин$^3$}

\titel{\tit}{\aut}{\autkol}{\titkol}

{\renewcommand{\thefootnote}{\fnsymbol{footnote}}\footnotetext[1]
{Работа выполнена при поддержке гранта РФФИ 09-01-92474-МНКС\_а.}}

\renewcommand{\thefootnote}{\arabic{footnote}}
\footnotetext[1]{Институт прикладной математики им.\ М.\,В.~Келдыша РАН; Московский государственный университет им.\ 
М.\,В.~Ломоносова, vadim@graphics.cs.msu.ru}
\footnotetext[2]{Московский государственный университет им.\ М.\,В.~Ломоносова, gkrivovyaz@graphics.cs.msu.ru}
\footnotetext[3]{Московский государственный университет им.\ М.\,В.~Ломоносова, ktosh@graphics.cs.msu.ru}

\vspace*{6pt}

\Abst{Предложен новый алгоритм распознавания людей по одежде. Алгоритм основан 
на классификации случайных регионов с помощью метода случайного леса деревьев (random 
forest). Главным достоинством предложенного алгоритма является то, что в нем не используется 
маска человека, поэтому он применим для видеоданных с произвольным сложным фоном. 
Приведены результаты тестирования алгоритма на собственной выборке 
видеопоследовательностей, полученных со стационарной камеры видеонаблюдения.}


\vspace*{2pt}

\KW{распознавание человека по видео; машинное обучение; случайный лес деревьев; вычитание 
фона}


\vspace*{6pt}

   \vskip 18pt plus 9pt minus 6pt

      \thispagestyle{headings}

      \begin{multicols}{2}

      \label{st\stat}



\section{Введение}
Идентификация личности является одной из самых бурно развивающихся областей 
компьютерного зрения, что вызвано широкой практической применимостью систем на ее основе. 
Наиболее надежными методами идентификации личности считаются методы распознавания по 
отпечаткам пальцев или по радужной оболочке глаза. Однако эти методы относятся к 
инвазивным~--- для распознавания человека они требуют его <<сотрудничества>>, например 
прикладывания пальца к устройству по считыванию отпечатков. Поэтому представляет интерес 
разработка алгоритма, способного распознавать людей лишь по фотографии/видео. 

Для идентификации личности по изображению лица было предложено множество 
алгоритмов~\cite{9konu}. Однако в связи с тем, что данный подход за\-час\-тую не обеспечивает 
требуемой надежности идентификации, много работ посвящено исследованию дополнительных 
признаков. В частности, в последнее время все чаще используется информация об одежде, о 
цвете волос и~т.\,д. Эта информация не является инвариантной для человека~--- он может 
перекрасить волосы, сменить одежду. Однако на протяжении небольшого промежутка времени, 
например одного дня, все эти признаки можно считать неизменными. В данной статье 
предлагается новый алгоритм распознавания человека по видео, полученному со стационарной 
камеры (рис.~\ref{f1konu}). 


Статья организована следующим образом. В~разд.~2 приведен обзор существующих методов. 
В~разд.~3 описывается предложенный алгоритм. Раздел~4 представляет результаты 
проведенных экспериментов. Заключение содержит основные результаты статьи.

\section{Существующие подходы}

Распознавание людей, учитывающее одежду, используется в двух задачах: аннотации 
изображений/видео и в системах видеонаблюдения. 


\begin{figure*} %fig1
\vspace*{1pt}
\begin{center}
\mbox{%
\epsfxsize=164.818mm
\epsfbox{kon-1.eps}
}
\end{center}
\vspace*{-9pt}
\Caption{Пример роликов из созданной тестовой выборки
\label{f1konu}}
\end{figure*}

В алгоритмах аннотации изображений/видео~\cite{3konu, 5konu} одна из основных проблем~--- 
сегментация изоб\-ра\-же\-ния человека. Современные алгоритмы сегментации не позволяют 
автоматически получить точную маску объекта на произвольном фоне. Поэтому стандартная 
схема устроена следующим образом: вначале на изображении находится лицо человека, после 
чего в качестве области одежды берется прямоугольник под лицом человека. Координаты и 
размер прямоугольника задаются эвристически. Однако такой прямоугольник содержит лишь 
небольшую часть от всей области одежды. Более того, в некоторых случаях этот прямоугольник 
может час\-тич\-но захватить область фона. 

В~\cite{1konu} используется более сложный алгоритм сегментации, однако ему тоже для первого 
приближения необходимо найти лицо человека. Поскольку современные алгоритмы нахождения 
лица могут относительно надежно находить лишь фронтальные лица, то в случаях, когда человек 
ни разу не посмотрит прямо в камеру (или его лицо будет чем-то загорожено), данный подход не 
сработает.
%\pagebreak

В системах видеонаблюдения~\cite{4konu, 8konu} маска человека находится с помощью 
алгоритмов вычитания фона. Однако современные алгоритмы вычитания фона хорошо работают 
лишь в относительно прос\-тых случаях. Например, при установке камеры в хорошо освещенном 
коридоре. Но, как было обнаружено на нашей тестовой выборке, в более сложных случаях эти 
алгоритмы могут давать неудовлетворительную маску.


\section{Алгоритм}

Предлагаемый алгоритм распознавания людей по одежде основан на методе классификации~--- 
случайном лесе деревьев~\cite{2konu}. В качестве признаков для классификации используются 
цвета пикселей набора квадратных регионов, случайным образом выбираемых из кадров 
видеоролика. В следующих подразделах описываются соответственно процессы обучения 
классификатора и распознавания людей в видеороликах с помощью обученного классификатора.

\subsection{Обучение классификатора}

Для обучения алгоритма из всех видеороликов обучающей выборки извлекается большое чис\-ло 
($N_{\mathrm{Train}}$) произвольных квадратных регионов (рис.~\ref{f2konu},\,\textit{а}). Размер и 
положение региона, а также номер кадра при этом выбираются абсолютно случайно. 

После этого проводится нормализация всех регионов: каждый из них масштабируется до размера 
$r\times r$, переводится в цветовое пространство HSV и представляется вектором длины $3r^2$. 

Так как все видеоролики сняты одной и той же стационарной камерой, становится возможным 
использование в качестве признаков региона еще и его пространственного положения~--- сдвига 
$(x,y)$ и ширины~$w$. Таким образом, получается обучающая выборка из $N_{\mathrm{Train}}$ векторов 
размера $3r^2 + 3$ (рис.~\ref{f2konu},\,\textit{б}). Каждому элементу выборки ставится в 
соответствие метка~--- идентификатор человека с соответствующего видео. 

Затем происходит обучение классификатора, который по произвольному региону (развернутому в 
вектор $3r^2 + 3$) выдает метку человека, присутствующего на видео.

Одним из главных критериев при выборе алгоритма машинного обучения была скорость, так как 
и размер входной выборки, и длина вектора признаков каждого элемента выборки очень велики. 

Поэтому в качестве такого алгоритма был выбран случайный лес деревьев~\cite{2konu}. При его 
обуче\-нии для каждого отдельного дерева выбирается случайная подвыборка из общей 
обуча\-ющей выборки. Используемые в узлах дерева функции~--- сравнение одной из координат 
входного вектора с порогом. При этом для скорости выбор координаты и самого порога при 
обучении происходит абсолютно случайно. Каждое дерево строится до тех пор, пока оно не 
станет правильно классифицировать свою обучающую подвыборку.

\subsection{Распознавание}

Для распознавания из тестового видеоролика также извлекается большое число, $N_{\mathrm{Test}}$, 
случай-\linebreak\vspace*{-12pt}
\pagebreak

\end{multicols}

\begin{figure} %fig2
\vspace*{1pt}
\begin{center}
\mbox{%
\epsfxsize=164.162mm
\epsfbox{kon-2.eps}
}
\end{center}
\vspace*{-9pt}
\Caption{Схема работы алгоритма: (\textit{а})~извлечение случайных квадратных регионов из 
видео; (\textit{б})~нормализация регионов, вытягивание их в векторы; (\textit{в})~классификация 
регионов с помощью случайного леса деревьев; (\textit{г})~получение финального результата 
классификации голосованием
\label{f2konu}}
\end{figure}

\begin{multicols}{2}


\noindent
но расположенных квадратных регионов, которые проходят описанную выше процедуру 
нормализации. После этого все они независимо друг от друга подаются на вход обученному 
классификатору (рис.~\ref{f2konu},\,\textit{в}). На выходе получается $N_{\mathrm{Test}}K$ меток, где 
$K$~--- число деревьев в обученном классифи\-каторе.
{\looseness=1

}

Понятно, что при этом регионы, расположенные полностью в области фона, будут помечены 
произвольными метками. Распределение неправильных меток в списке будет равномерным. Зато 
регионы, перекрывающие область объекта (человека), в целом будут классифицированы 
правильно, поэтому в среднем большинство меток будет соответствовать реальному человеку 
на видео.


В связи с этим в качестве вероятности каждой метки~$L$ можно брать относительное число раз, когда 
регион из тестового видео был классифицирован как~$L$:

\noindent
$$
p(L) =\fr{\sum\limits_{k=1}^K \sum\limits_{x_i\in X} l(T_k(x_i)=L)}{N_{\mathrm{Test}}K}\,,
$$
где $X$~--- множество из $N_{\mathrm{Test}}$ векторов, а $T_i(x)$~--- результат классификации 
вектора~$x$ $i$-м деревом (рис.~\ref{f2konu},\,\textit{г}).

\section{Эксперименты}

\subsection{Тестовые данные}

\begin{figure*}[b] %fig3
\vspace*{1pt}
\begin{center}
\mbox{%
\epsfxsize=161.135mm
\epsfbox{kon-3.eps}
}
\end{center}
\vspace*{-9pt}
\Caption{Результаты работы алгоритма по метрике <<$N$ лучших>>. Штриховой линией 
отмечен результат работы предложенного алгоритма, сплошной линией~--- средний результат 
случайной классификации: (\textit{а})~учет роликов, на которых человек входит в комнату;
(\textit{б})~учет всех роликов
\label{f3konu}}
\end{figure*}


Для экспериментов была собрана собственная выборка видеороликов. Эти видеоролики 
записывались камерой, висящей под потолком и на\-прав\-лен\-ной на входную дверь в лабораторию. 
Запись проводилась в течение 8~дней, всего было собрано 463~видеоролика. Разрешение
 видео~--- $352 \times 240$, средняя длительность ролика~--- около 10~с. 

Примеры кадров из собранных видеороликов показаны на рис.~\ref{f1konu}. Всего на разных 
видеороликах присутствует 25~разных людей.

Каждый ролик был вручную аннотирован на предмет присутствующих на нем людей (их меток), 
а также флагом~--- входит данный человек в комнату или, наоборот, выходит из нее.



Во многих случаях было невозможно разбить общее видео на несколько роликов, в каждом из 
которых присутствовал бы лишь один человек~--- например, когда два--три человека 
одновременно выходят из комнаты. В данной работе такие ролики были удалены из 
рассмотрения. Стоит отметить, что существующие методы тоже пока не способны обрабатывать 
такие случаи (надежно сегментировать людей в кадре).

С учетом исключения из рассмотрения таких роликов, в каждый конкретный день людей, 
появляющихся на двух и более роликах, в среднем было 5--8~человек.

Стоит отметить, что заснятая на видеороликах сцена является очень сложной для алгоритмов 
вычитания фона. В частности, протестированные алгоритмы~\cite{7konu, 6konu} давали 
неудовлетворительные маски объекта. Основными причинами таких результатов были 
нестабильность фона (открывающаяся дверь, полупрозрачное стекло) и нестабильное освещение. 
Таким образом, методы, явно опирающиеся на известную маску человека, здесь неприменимы.

\subsection{Проведенные эксперименты}

При проведении экспериментов использовались следующие параметры: число регионов в 
обуча\-ющей выборке $N_{\mathrm{Train}} =1\,000\,000$; нормализованный размер региона $r=16$; число 
регионов, извлекаемых из тестового видео $N_{\mathrm{Test}}=3000$; число деревьев $K=20$.

Для тестирования использовалось два сценария. В первом сценарии задействованы лишь те 
ролики, на которых человек входит в комнату. В качестве обучающих данных использовались 
первые два видеоролика каждого человека, остальные попадали в тестовую выборку. Во втором 
сценарии уже использовались все видеоролики. Обучающими были первые три ролика каждого 
человека, все остальные~--- тестовые.

В обоих случаях человек из тестовой выборки присутствовал в обучающих данных.

Результаты работы алгоритма представлены на рис.~\ref{f3konu}. Демонстрируемые результаты 
являются суммой результатов за все 8~дней наблюдения.

В качестве метрики качества использовалась мет\-ри\-ка <<$N$ лучших>> (Top$N$), которая для 
каждого конкретного $n$ показывала, для какого процента всех роликов правильная метка была 
среди первых $n$ результатов.

Так как за один день в обучающей выборке могло присутствовать максимум 14~человек, 
значение метрики при $n\geq 14$ равно 100\%.

Для ориентира на графиках приведен результат распознавания в случае, если бы классификация 
осуществлялась случайно (подбрасыванием мо\-нетки). 

Из графиков видно, что полученные результаты еще далеки от того, чтобы их можно было 
надежно использовать в реальной системе. Например, в первом сценарии процент правильно 
распознанных роликов (т.\,е.\ Top$N(1)$) составляет лишь~45\%.

Однако следует учитывать сложность входных данных. В частности, человек в один и тот же 
день может появляться на видео как в куртке, так и без нее. А значит, если, скажем, на первых 
двух (обучающих) роликах он будет в куртке, то распознать его на остальных роликах по 
используемым признакам будет практически невозможно.

Большую проблему составляет освещение. К~кон\-цу дня, когда уже темно, но еще не включили 
свет в комнате, даже человек зачастую не сразу распознает людей на видео. 

Во втором сценарии используется тот факт, что во многих случаях цвет и текстура одежды на 
спине очень похожа на одежду спереди. А значит, есть надежда, что алгоритм, обученный на 
видео, на котором человек входит, сможет распознавать его же, но когда он выходит, и наоборот. 
Как видно, пока, к сожалению, при данном сценарии алгоритм сработал значительно хуже, чем в 
первом.

\section{Заключение}

В данной статье был предложен новый алгоритм распознавания человека по видео. Основным 
его преимуществом является то, что он не опирается на маску переднего плана, полученную с 
по\-мощью алгоритмов вычитания фона, как это делает большинство существующих алгоритмов. 
Благодаря этому для его обучения достаточно предоставить лишь выборку видеороликов с 
меткой-идентификатором присутствующего на видео человека. Алгоритм был протестирован 
на собственной выборке видеопоследовательностей, полученных с камеры видеонаблюдения.

{\small\frenchspacing
{%\baselineskip=10.8pt
\addcontentsline{toc}{section}{Литература}
\begin{thebibliography}{9}

\bibitem{9konu} %1
\Au{Zhao W., Chellappa~R., Phillips~P.\,J., Rosenfeld~A.}
Face recognition: A literature survey~// ACM Computing Surveys, 2003. 
Vol.~35. No.\,4. P.~399--458.

\bibitem{3konu} %2
\Au{Jaffre G., Joly~P.}
Costume: A new feature for automatic video content indexing~// RIAO Proceedings, 2004. P.~314--325. 

\bibitem{5konu} %3
\Au{Song Y., Leung T.}
Context-aided human recognition~--- clustering~// ECCV Proceedings, 2006. P.~382--395. 

\bibitem{1konu} %4
\Au{Gallagher A., Chen~T.}
Clothing cosegmentation for recognizing people~// Proc. of CVPR, 2008. No.\,1. P.~1--8. 

\bibitem{4konu} %5
\Au{Nakajima C., Pontil M., Heisele~B., Poggio~T.}
Full-body person recognition system~// Pattern Recognition, 2003.
Vol.~36. No.\,9. P.~1997--2006.

\bibitem{8konu} %6
\Au{Yoon K., Harwood D., Davis~L.}
Appearance-based person recognition using color/path-length profile~// J.~Visual Communication Image Representation, 2006.
Vol.~17. No.\,3. P.~605--622.

\bibitem{2konu} %7
\Au{Geurts P., Ernst D., Wehenkel~L.}
Extremely randomized trees~// Machine Learning J., 2006. Vol.~63. No.\,1.

\bibitem{7konu} %8
\Au{Wren C., Azarbayejani A., Darrell~T., Pentland~A.}
Pfinder: Real-time tracking of the human body~// IEEE Trans.\ PAMI, 1997. Vol.~19. No.\,7. 
P.~780--785. 


\label{end\stat}

\bibitem{6konu} %9
\Au{Stauffer C., Grimson~W.\,E.\,L.}
Adaptive background mixture modelsfor real-time tracking~// CVPR Proceedings, 1999. P.~246--252. 



 \end{thebibliography}
}
}
\end{multicols}    %10

\def\stat{pavel}


\def\tit{ПОИСК И АНАЛИЗ КЛЮЧЕВЫХ ТОЧЕК РАДУЖНОЙ ОБОЛОЧКИ ГЛАЗА МЕТОДОМ 
ПРЕОБРАЗОВАНИЯ ЭРМИТА$^*$}
\def\titkol{Поиск и анализ ключевых точек радужной оболочки глаза методом 
преобразования Эрмита}

\def\autkol{Е.\,А.~Павельева, А.\,С.~Крылов}
\def\aut{Е.\,А.~Павельева$^1$, А.\,С.~Крылов$^2$}

\titel{\tit}{\aut}{\autkol}{\titkol}

{\renewcommand{\thefootnote}{\fnsymbol{footnote}}\footnotetext[1]
{Работа выполнена при финансовой поддержке РФФИ (проект №\,10-07-00433-а).}}

\renewcommand{\thefootnote}{\arabic{footnote}}
\footnotetext[1]{Московский государственный университет им.\ М.\,В.~Ломоносова,
факультет вычислительной математики и кибернетики, 
 paveljeva@yandex.ru}
\footnotetext[2]{Московский государственный университет 
им.\ М.\,В.~Ломоносова, факультет вычислительной математики и кибернетики,
kryl@cs.msu.ru}

\vspace*{-6pt}

\Abst{Предложен алгоритм поиска ключевых точек для распознавания человека по 
радужной оболочке глаза, основанный на локальном преобразовании Эрмита. 
Использование для распознавания только ключевых точек радужной оболочки позволяет 
хранить небольшой объем информации при достаточно хорошем качестве распознавания.}
\vspace*{1pt}

\KW{биометрия; радужная оболочка глаза; преобразование Эрмита; ключевые точки}

   \vskip 18pt plus 9pt minus 6pt

      \thispagestyle{headings}
      
       \vspace*{6pt}

      \begin{multicols}{2}

      \label{st\stat}
      
  

\section{Введение}

Преобразование Эрмита~[1] является известным методом, применяющимся для решения 
биометрических задач~[2--4]. Этот локальный метод основан на вычислении сверток функции 
интенсивности изображения с функциями преобразования Эрмита в каждой точке изображения. 
При этом в работе~[5] показано, что для задачи распознавания по радужной оболочке глаза 
наиболее информативными являются свертки с двумерной функцией 
пре\-об\-разо\-ва\-ния Эрмита~$\varphi_{1,0}$. Также широко известным методом в обработке сигналов является метод 
моментов Гаусса--Эрмита~\cite{2pav}, эквивалентный преобразованию Эрмита с 
точностью до знаков сверток с нечетными функциями преобразования Эрмита. При этом для 
решения задач распознавания формируются бинарные матрицы, составленные из знаков сверток 
в каждой точке, которые затем сравниваются с матрицами из базы данных.
{ %\looseness=1

}

В данной работе на основе преобразования Эрмита предложен метод нахождения ключевых 
точек текстуры радужной оболочки. Эти точки соответствуют наиболее значимым экстремумам 
свертки функции интенсивности изображения с функцией~$\varphi_{1,0}$. 

В разд.~2 дается описание преобразования Эрмита. В разд.~3 приведены некоторые детали 
использованного метода предобработки изображений радужной оболочки. Раздел~4 описывает 
алгоритм нахождения ключевых точек радужной оболочки, приведены результаты 
экспериментов на базе данных CASIA-IrisV3~\cite{6pav}. 

\section{Преобразование Эрмита}

Функции Эрмита определяются как
$$
\psi_n (x) = \fr{(-1)^n e^{-x^2/2}}{\sqrt{2^n n!\sqrt{\pi}}}\,H_n(x)\,,
$$
где $H_n(x)$~--- полиномы Эрмита:
\begin{gather*}
H_0(x) =1\,,\quad H_1(x) =2x\,,\\
H_n(x) =2x H_{n-1}(x) -2(n-1)H_{n-2}(x)\,.
\end{gather*}
      
Функции Эрмита являются собственными функциями преобразования Фурье и образуют полную 
ортонормированную систему функций в пространстве~$L_2(-\infty,\,\infty)$. 

Функции преобразования Эрмита (рис.~1) связаны с функциями Эрмита соотношением
$$
\varphi_n(x) =\psi_0(x)\psi_n(x) =\fr{(-1)^n e^{-x^2}}{\sqrt{2^n n!\pi}}\,H_n(x)\,.
$$
При вычислениях они являются одновременно локализованными в координатном и частотном 
пространствах. Так как~$\psi_n$~--- ортонормированная сис\-те\-ма функций, то
$$
\int\limits_{-\infty}^\infty \varphi_n(x)\,dx =\int\limits_{-\infty}^\infty \psi_0(x)\psi_n(x)\,dx =0\quad 
\forall n>0\,,
$$ 
т.\,е.\ функции~$\varphi_n(x)$ имеют среднее нулевое значение для номеров $n>0$. 
Это свойство очень важно\linebreak\vspace*{-12pt}
\pagebreak

\noindent
\begin{center} %fig1
\vspace*{3pt}
\mbox{%
\epsfxsize=77.701mm %78.39mm
\epsfbox{pav-1.eps}
}
%\end{center}
%\vspace*{6pt}

{{\figurename~1}\ \ \small{Примеры функций преобразования Эрмита}}
\end{center}
\vspace*{-6pt}


\bigskip
\addtocounter{figure}{1}


\noindent
 для методов, исполь\-зу\-ющих знаки сверток с такими функциями. 

Двумерные функции преобразования Эрмита являются произведением одномерных функций:
%\noindent
$$
\varphi_{n,m}(x,y) =\varphi_n(x)\varphi_m(y)\,.
$$


Преобразование Эрмита для изображения опреде\-ляется в каждой точке~$(x_0,y_0)$ значениями 
сверток функции интенсивности изображения~$I(x,y)$ с функциями преобразования %\linebreak 
Эрмита~$\varphi_{m,n}(x,y)$ для выбранного конечного набора индек\-сов~$(m,n)$~[1,~4]:
\begin{multline*}
M_{m,n}(x_0,y_0) =(I(x,y) * \varphi_{m,n}(x,y))(x_0,y_0)={}\\
{}= \iint\limits_G I(x,y) \varphi_{m,n}(x_0-x,y_0-y)\,dxdy\,,
\end{multline*}
где $G$~--- область сосредоточения функции~$\varphi_{m,n}$.



\section{Предобработка изображений и~контроль наличия века в~области 
параметризации}

Алгоритм нахождения радужной оболочки на изображении глаза описан в~\cite{5pav, 7pav} и 
основывается на поиске максимального скачка средней интенсивности вдоль круговых контуров 
изображения. После локализации радужная оболочка глаза переводится в прямоугольное 
нормализованное изобра-\linebreak\vspace*{-12pt}


\noindent
\begin{center} %fig2
\vspace*{6pt}
\mbox{%
\epsfxsize=78.847mm %78.39mm
\epsfbox{pav-2.eps}
}
%\end{center}
\vspace*{6pt}

{{\figurename~2}\ \ \small{Нормализация радужной оболочки}}
\end{center}
%\vspace*{-6pt}


\bigskip
\addtocounter{figure}{1}


\noindent
жение. Для дальнейшей параметризации в работе используется только 
область, включающая правую верхнюю четверть нормализованного изображения, на которую, 
как правило, не попадают ресницы и веки~\cite{5pav} (рис.~2).


Тем не менее для определения наличия века в этой области параметризации используется 
специальный алгоритм. 
Ищется максимум вертикальной производной яркости изображения 
$$
\underset{y}{\max}\sum\limits_x\left\vert \fr{\partial I(x,y)}{\partial y}\right\vert
$$
в области $[x_p-r/2,\,x_p+r/2][y_p+r,\,y_p+(R+r)/2]$, выделенной на рис.~3. 
Здесь~$(x_p,y_p)$~--- центр зрачка,
 $r$ и $R$~--- радиусы границ радужной оболочки. Если это 
значение больше порогового, то считается, что нижнее веко попало в область параметризации.

\begin{center} %fig3
\vspace*{6pt}
\mbox{%
\epsfxsize=80.281mm 
\epsfbox{pav-3.eps}
}
%\end{center}
%\vspace*{1pt}

{{\figurename~3}\ \ \small{Изображения $I(x,y)$ и $I_y^\prime(x,y)$}}
\end{center}
%\vspace*{-6pt}

%\bigskip
\addtocounter{figure}{1}

 
\section{Метод ключевых точек параметризации радужной оболочки}

В работе~\cite{5pav} показано, что наиболее информативными номерами двумерных функций 
преобразования Эрмита~$\varphi_{m,n}(x,y)$ для задачи идентификации по радужной оболочке 
являются номера~(1,\,0), (1,\,1), (2,\,0) в указанном порядке. Поэтому для параметризации данных 
радужной оболочки в данной работе была выбрана функция пре\-обра\-зо\-ва\-ния Эрмита~$\varphi_{1,0}(x,y)$. 

Рассмотрим в каждой точке области парамет\-ризации величину $F_1=M_{1,0}$. В~качестве кода 
радужной оболочки (ключевых точек) рассматри-\linebreak\vspace*{-12pt}
\pagebreak

\noindent
\begin{center} %fig4
%\vspace*{6pt}
\mbox{%
\epsfxsize=80mm %78.39mm
\epsfbox{pav-4.eps}
}
\end{center}
%\vspace*{-1pt}
{{\figurename~4}\ \ \small{Область параметризации радужной оболочки с кодом радужной оболочки}}

\vspace*{18pt}

 
\begin{center}
%\vspace*{12pt}
\mbox{%
\epsfxsize=80mm %78.39mm
\epsfbox{pav-5.eps}
}
\end{center}
%\vspace*{-1pt}
{{\figurename~5}\ \ \small{Примеры работы алгоритма выделения ключевых точек для изображений с наложением века и бликов на область 
параметризации}}
%\end{center}
%\vspace*{12pt}


\bigskip
\medskip
\addtocounter{figure}{2}

\noindent
ваются~$N$ ($N = 50$, 100, 150, 200) точек, 
разбитых на две группы: $N/2$ точек с максимальными значениями~$F_1$, удаленных друг от 
друга не менее чем на 2~пикселя, и аналогично~$N/2$~--- с минимальными значениями~$F_1$. 
Пример кода радужной оболочки для $N = 150$ ключевых точек приведен на рис.~4. 
Черными точками обозначены ключевые точки с максимальными значениями~$F_1$, белыми~--- 
с минимальными. На рис.~4 также обозначена граница возможных значений точек кода, 
отстоящая от краев области параметризации на полуширину области сосредоточения 
функции~$\varphi_1$.


Отметим, что этот метод эффективен только для изображений без наложения века на область 
параметризации, так как в области века и бликов большинство детектируемых точек не является 
точками радужной оболочки (рис.~5). 


При идентификации по радужной оболочке используются матрицы сравнения ключевых точек.
Для построения матрицы сравнения область пара\-мет\-ризации разбивается на непересекающиеся 
блоки размера $3\times 3$. Если в блок не попадает ни одной точки кода, то ему соответствует 
значение~0. Если в блок попадает хотя бы одна черная (белая) точка кода, то соответствующее 
значение матрицы сравнения равняется~1~($-1$), если и черная, и белая, то значение равняется 2. 
При сравнении двух матриц считается, что соответствующие блоки равны, если в них попали 
точки одного цвета (табл.~1).


Чтобы алгоритм был устойчив к поворотам глаза (поворот глаза соответствует циклическому 
сдвигу всего нормализованного изображения), сравниваются матрицы точек уменьшенного 
размера. Границы такой урезанной матрицы показаны на рис.~6 и сдвигаются у 
исследуемого изображения в обе стороны до границ возможных значений точек ко-\linebreak\vspace*{-12pt}
\columnbreak

%\bigskip

\noindent
\begin{center}
\noindent
\parbox{51mm}{{\tablename~1}\ \ \small{Сопоставление значений матриц сравнения: <<$+$>> означает, что блоки считаются 
равными, <<$-$>>~--- различными}}
\end{center}
%\vspace*{2ex}

\begin{center}
\tabcolsep=9pt
\begin{tabular}{|c|c|c|c|c|}
\hline
&0&1&$-1$&2\\
\hline
0&$+$&$-$&$-$&$-$\\
1&$-$&$+$&$-$&$+$\\
$-1$\hphantom{$-$}&$-$&$-$&$+$&$+$\\
2&$-$&$+$&$+$&$+$\\
\hline
\end{tabular}
\end{center}
\vspace*{12pt}

%\bigskip
\addtocounter{table}{1}

\begin{center} %fig6
\vspace*{6pt}
\mbox{%
\epsfxsize=80mm %78.39mm
\epsfbox{pav-6.eps}
}
\end{center}
%\vspace*{6pt}
{{\figurename~6}\ \ \small{Области сравнения матриц кодов радужных оболочек}}
%\end{center}
%\vspace*{-6pt}


\bigskip
\addtocounter{figure}{1}


\noindent
да. В~данной 
работе учитываются углы поворота от~$-10^\circ$ до ~$10^\circ$.



Алгоритм параметризации радужной оболочки по ключевым точкам протестирован на базе 
данных  CASIA-IrisV3~\cite{6pav}, содержащей 2655~изображений глаз. Результаты работы 
алгоритма приведены в табл.~2 и на рис.~7. Здесь CRR (Correct Recognition Rate)~--- вероятность верного распознавания.

\bigskip
%\vspace*{3pt}

%\begin{center}
\noindent
{{\tablename~2}\ \ \small{Результаты работы алгоритма ключевых точек}}
%\end{center}
%\vspace*{2ex}

{\small 
\begin{center}
\tabcolsep=8pt
\begin{tabular}{|c|c|c|c|}
\hline
\tabcolsep=0pt\begin{tabular}{c}Число\\ ключевых\\ точек $N$\end{tabular}&
\tabcolsep=0pt\begin{tabular}{c}Число\\ неверных\\ ближайших\\ изображений\\ (из 2655)\end{tabular}&
\tabcolsep=0pt\begin{tabular}{c}Неверные\\ из-за\\ наложения\\ века\\ и бликов\end{tabular}&
\tabcolsep=0pt\begin{tabular}{c}CRR,\\ \%\end{tabular}\\
\hline
200&$23 = 18 + 5$&18&99.81\\
150&$36 = 30 + 6$&30&99.77\\
100&\hphantom{9}$58 = 41 + 17$&41&99.35\\
\hphantom{9}50&$135 = 65 + 70$&65&97.36\\
\hline
\end{tabular}
\end{center}
}
\vspace*{12pt}


%\bigskip
\addtocounter{table}{1}


\begin{center} %fig7
\vspace*{6pt}
\mbox{%
\epsfxsize=74.136mm 
\epsfbox{pav-7.eps}
}
\end{center}
\vspace*{6pt}
{{\figurename~7}\ \ \small{График зависимости CRR от числа взятых ключевых точек}}
%\end{center}
%\vspace*{6pt}


%\bigskip
\addtocounter{figure}{1}

%\noindent


\section{Заключение}

В работе предложен алгоритм идентификации человека, использующий ключевые точки 
радужной оболочки глаза, найденные локальным методом преобразования Эрмита. Этот 
алгоритм позволяет получать достаточно хорошие результаты распознавания даже при 
небольшом объеме хранимой информации. Он достаточно перспективен для использования в 
мультибиометрических системах распознавания.

\bigskip
Работа выполнена при поддержке ФЦП <<Научные и научно-педагогические кадры 
инновационной России>> на 2009--2013~гг.

{\small\frenchspacing
{%\baselineskip=10.8pt
\addcontentsline{toc}{section}{Литература}
\begin{thebibliography}{9}

\bibitem{1pav}
\Au{Martens J.\,B.}
The Hermite transform-theory~// IEEE Transactions on Acoustics, Speech, and Signal Processing, 1990. 
Vol.~38. No.\,9. P.~1595--1606.

\bibitem{2pav}
\Au{Ma~L., Tan~T., Zhang~D., Wang~Y.}
Local intensity variation analysis for iris recognition~// Pattern Recognition, 2004. Vol.~37. No.\,6. 
P.~1287--1298.

\bibitem{3pav}
\Au{Wang L., Dai~M.}
Extraction of singular points in fingerprints by the distribution of Gaussian--Hermite moment~// IEEE 
1st Conference (International) DFMA Proceedings, 2005. P.~206--209.

\bibitem{4pav}
\Au{Estudillo-Romero~A., Escalante-Ramirez~B.}
The Hermite transform: An alternative image representation model for iris recognition~// LNCS, 2008. 
No.\,5197. P.~86--93.


\bibitem{5pav}
\Au{Павельева Е.\,А., Крылов~А.\,С., Ушмаев~О.\,С.}
Развитие информационной технологии идентификации человека по радужной оболочке глаза на 
основе преобразования Эрмита~// Системы высокой доступности, 2009. №\,1. С.~36--42.

\bibitem{6pav}
База данных CASIA-IrisV3. {\sf  http://www.cbsr.ia.ac.cn/ IrisDatabase.htm}.

\label{end\stat}

\bibitem{7pav}
\Au{Krylov A.\,S., Pavelyeva~E.\,A.}
Iris data parametrization by Hermite projection method~// GraphiCon'2007 Conference Proceedings, 2007. P.~147--149. 
 \end{thebibliography}
}
}
\end{multicols}


           %+  %11

\def\stat{protasov}


\def\tit{СОСТАВЛЕНИЕ СУБЪЕКТИВНОГО ПОРТРЕТА С~ИСПОЛЬЗОВАНИЕМ 
ЭВОЛЮЦИОННОГО МОРФИНГА И~КВАЛИМЕТРИЯ МЕТОДА$^*$}
\def\titkol{Составление субъективного портрета с использованием 
эволюционного морфинга и квалиметрия метода}

\def\autkol{В.\,И.~Протасов}
\def\aut{В.\,И.~Протасов$^1$}

\titel{\tit}{\aut}{\autkol}{\titkol}

{\renewcommand{\thefootnote}{\fnsymbol{footnote}}\footnotetext[1]
{Работа выполнена при поддержке РФФИ, грант №\,08-07-00447-а.}}

\renewcommand{\thefootnote}{\arabic{footnote}}
\footnotetext[1]{Институт физико-технической информатики, Протвино, 
protonus@yandex.ru}
 

\Abst{Приведено описание нового метода составления субъективного портрета группой 
свидетелей, базирующегося на генетических алгоритмах. На основе модели <<виртуального 
свидетеля>> определяется точность составления субъективного портрета. Образ лица, 
<<вспоминаемого>> виртуальным свидетелем, моделируется набором компонент вектора, 
определяющего трехмерное изображение лица. Модель описывает художественные 
способности человека к рисованию лиц и способности к сравнению разных лиц по степени 
их похожести на оригинал. Введена характеристика, описывающая утомляемость свидетеля 
при длительной работе по распознаванию и сравнению лиц. Полученная таким образом 
модель позволяет производить настройку сетевого метода эволюционного морфинга для 
получения оптимального результата и его квалиметрию.}

\KW{субъективный портрет; эволюционный морфинг; генетические алгоритмы; принятие 
решений; квалиметрия}
   \vskip 18pt plus 9pt minus 6pt

      \thispagestyle{headings}

      \begin{multicols}{2}

      \label{st\stat}
     
\section{Введение}
     
     Одна из наиболее сложных задач в современной криминалистике~--- 
составление субъективного портрета (фоторобота) человека, увиденного и 
запомненного с разной степенью точности одним или несколькими 
свидетелями. Развитие современных информационно-коммуникационных 
технологий привело к принципиально новым возможностям решения этой 
проблемы. 
     
     <<Согласно теории отражения, мысленный образ предмета формируется 
в сознании под воздействием самого предмета. Человеческий мозг обладает 
способностью воспринимать внешнюю информацию о предметах через органы 
чувств (в нашем случае предмет~--- это некий человек, его внешний облик, а 
орган восприятия~--- глаза), и длительное время удерживать в памяти 
представление о нем\ldots\ В~связи с тем, что мысленный образ может 
забываться и он имеет место только в сознании ограниченного числа лиц 
(свидетели, очевидцы или потерпевшие), необходимо как можно быст\-рее 
закрепить его с помощью других средств и методов~--- <<актуализировать>> в 
материальной форме>>~\cite{1pr}. Известно, что существующие методы 
со\-став\-ле\-ния субъективного портрета с использованием компьютерных 
технологий или с помощью полицейских художников не всегда дают 
удовлетворительные результаты. 
     
     В настоящей работе представлена новая информационная технология 
генетического консилиума (ГК)~[2--8], предназначенная для составления 
субъективного портрета коллективом свидетелей или одиночным свидетелем с 
использованием эволюционного морфинга лица. В~основу технологии 
положены генетические алгоритмы. Для составления объемных фотороботов 
используется программа FaceGen Modeller~[9]. 
     
     Технологию ГК можно представить следующим образом. Свидетели, 
основываясь на своих воспоминаниях, составляют в первом приближении свои 
варианты фотороботов и отправляют их на сервер. Сетевая программа 
предъявляет каждому свидетелю по два варианта фотороботов, полученных на 
первой итерации его коллегами. Свидетели с помощью специальной процедуры 
скрещивают эти портреты, получая варианты-потомки. Далее они подвергают 
мутации лучший из них и выбирают из нескольких мутированных вариантов 
один, наиболее похожий на оригинал. Эти варианты вновь отправляются на 
сервер, и цикл итераций повторяется до тех пор, пока в эво\-лю\-цио\-ни\-ру\-ющей 
популяции не окажутся варианты-близнецы, в наибольшей степени похожие на 
оригинал. 
     
     Описанные в~[3--8] эксперименты показали принципиальную 
работоспособность нового метода составления субъективного портрета с 
использованием трехмерного морфинга лица группой свидетелей и одиночным 
свидетелем. Было показано, что итерационные процессы получения 
консоли-\linebreak\vspace*{-12pt}
\pagebreak

\noindent
дированного решения сходятся достаточно быстро и качество 
составления фоторобота на заключительной итерации существенно лучше 
качества вариантов первой итерации. Аналогичные результаты были получены 
независимо от этих работ группой шотландских исследователей~[10]. Они 
предложили и использовали технологию эволюционного морфинга, в 
значительной степени повторяющую технологию ГК. 
     
     Недостатками существующих методов составления субъективных 
портретов является то, что неизвестно, с какой точностью коллектив 
свидетелей или одиночный свидетель могут восстановить исходный портрет. 
Даже в случае, когда свидетели имеют достаточно времени для запоминания 
лица некоего человека, неясно, с какой точностью они смогут сделать это. 
     
      В настоящей работе впервые предпринята попытка осуществить 
квалиметрию процесса со\-ставления субъективного портрета коллективом или 
одиночным свидетелем с использованием технологии эволюционного 
морфинга. Для дости\-жения этой цели было необходимо разработать модель 
виртуального свидетеля. Модель должна %\linebreak 
пол\-ностью замещать реального 
свидетеля, так чтобы результаты деятельности виртуального свидетеля были 
неотличимы от результатов деятельности реального свидетеля с такими же 
параметрами.
     
\section{Модель виртуального свидетеля}
     
     Виртуальный свидетель представляет собой программу, содержащую ряд 
параметров, характеризующих основные свойства реального свидетеля, 
пытающегося восстановить и зафиксировать портрет ранее виденного им 
человека. В первом приближении эти свойства можно описать тремя 
параметрами. К ним относятся величина~$K_a$, характеризующая способности 
свидетеля как художника, и величины~$K_0$ и~$\alpha$, описанные ниже и 
характеризующие способности свидетеля к распознаванию лиц. Образ лица, 
<<вспоминаемого>> виртуальным свидетелем, моделируется набором $n$ 
относительных величин некоторого вектора~$G_i$, $i = 1, 2, \ldots , n$. Эти 
величины однозначно определяют трехмерное изображение лица в виде 
полигональной модели. Так, в программе FaceGen Modeller для описания 
симметричного лица без текстуры $n = 50$. Если ставить перед виртуальным 
свидетелем задачу по хранящемуся у него образу восстановить трехмерный 
портрет лица, характеризующегося вектором~$G_i$, то программа, 
моделирующая виртуального свидетеля, восстановит искаженный 
век\-тор-об\-раз~$U_i$ этого лица

\noindent
     \begin{equation}
     U_i = G_i\left[1+K_a \chi \left(1-2\xi\right)\right]\,,
     \label{e1pr}
     \end{equation}
     где $\chi$ и $\xi$~--- случайные числа от 0 до~1. Случайное число~$\chi$ 
отражает особенность свидетеля каждый раз рисовать разные портреты, 
отличающиеся от оригинала, но с постоянным, характеризующим данного 
свидетеля коэффициентом сходства. 
     
     Эксперименты показали, что такая модель достаточно реалистично 
описывает художественные способности человека. Коэффициент $K_a=0$ 
соответствует идеальному художнику, с уменьшением изобразительных 
способностей величина коэффициента растет. 
     
     Для оценки качества <<нарисованного>> портрета введем два 
коэффициента~$K_R$ и $K_S$ следующим образом:
     \begin{gather}
     K_R = \fr{1}{n(G_{\max}-G_{\min})}\sum\limits_{i=1}^n\vert U_i-
G_i\vert\,; \label{e2pr}\\
     K_S=1-K_R\,.\label{e3pr}
     \end{gather}
     Здесь $K_R$~--- коэффициент различия двух портретов, $G_{\max}$~--- 
максимальное и $G_{\min}$~--- минимальное из возможных значений вектора 
$G_i$, $K_S$~--- коэффициент сходства.
     
     По аналогии с коэффициентом художественных способностей вводится 
коэффициент способности к распознаванию лиц~$K_0$. Этот коэффициент 
определяется из эксперимента по распознаванию свидетелем ряда 
сгенерированных программой лиц, в разной степени похожих на искомое. 
     
     При длительных экспериментах, когда перед свидетелем проходит 
длинная череда рас\-по\-зна\-ва\-емых лиц, он начинает ошибаться и его способность 
к распознаванию снижается. Эту способность свидетеля можно 
охарактеризовать величиной <<утомляемости>>~$\alpha$, определяемой из 
выражения
     \begin{equation}
     \alpha = -\fr{1}{N}\ln\fr{K_{0N}}{K_0}\,,
     \label{e5pr}
     \end{equation}
     где $K_{0N}$~--- значение коэффициента~$K_0$ после предъявления 
$N$ портретов в течение одного сеанса экспериментов. 

Типичные значения величин коэффициентов  $K_a$, $K_0$ и $\alpha$, 
полученные из  экспериментов с разными людьми, приведены в 
табл.~1. 

\bigskip

{\small\begin{center} %tabl1
\noindent
{{\tablename~1}\ \ \small{Значения коэффициентов $K_a$, $K_0$ и  $\alpha$}}
\end{center}
%\vspace*{2pt}

\begin{center}
\tabcolsep=5.6pt
\begin{tabular}{|c|c|c|}
\hline
Коэффициент&\tabcolsep=0pt\begin{tabular}{c}Минимальное\\ значение\end{tabular}&
\tabcolsep=0pt\begin{tabular}{c}Максимальное\\ значение\end{tabular}\\
\hline
$K_a$ &0,02\hphantom{999}&0,3\hphantom{9999}\\
$K_0$ &0,92\hphantom{999}&0,99\hphantom{999}\\
$\alpha$ &0,00003&0,00009\\
\hline
\end{tabular}
\end{center}
%\vspace*{6pt}
}


%\bigskip
\addtocounter{table}{1}


     \begin{figure*}[b] %fig1
     \vspace*{1pt}
\begin{center}
\mbox{%
\epsfxsize=147.006mm
\epsfbox{pro-1.eps}
}
\end{center}
\vspace*{-9pt}
     \Caption{Блок-схема настройки параметров эволюционного морфинга
\label{f1pr}}
\end{figure*}

\section{Метод настройки параметров генетического консилиума}
     
     Наряду с получением аппарата для измерения способностей свидетелей к 
процессу составления субъективного портрета создание модели виртуального 
свидетеля преследует еще одну цель. Это разработка механизма настройки 
параметров ГК,\linebreak реализующего эволюционный морфинг, на достижение 
наилучшего результата при работе конкретной группы свидетелей. 
     
     На первом этапе свидетели тестируются на предмет их способностей к 
составлению субъективного портрета. Им предъявляются тестовые примеры, 
при работе над которыми измеряются значения их художественных и 
распознавательных способностей. Определяются также значения параметров 
утомляемости. Эти цифры вводятся в программу настройки параметров 
эволюционного морфинга, и программа при настройке параметров метода на 
оптимальные значения работает с виртуальными свидетелями, обладающими 
свойствами реальных. На втором этапе, отделенном от первого временем 
отдыха, свидетели приступают к работе по со\-став\-ле\-нию искомого портрета. 
     
     Анализ задачи получения оптимальных па\-ра\-мет\-ров ГК при работе 
группы свидетелей показывает, что оптимизируемая величина (степень 
соответствия составленного экспертами портрета\linebreak исходному) в пространстве 
поиска оптимальных па\-ра\-мет\-ров не может быть выражена через них в явном 
виде. Поэтому для проведения оптимизации были выбраны генетические 
алгоритмы. Параметрами настройки эволюционного морфинга являются 
величины KPM~--- чис\-ло <<прогонов>> программы для получения 
установившихся значений коэффициентов сходства, NC~--- количество 
скрещенных портретов, выдаваемых программой виртуальному свидетелю для 
выбора, NM~--- чис\-ло мутированных портретов из одного выбранного портрета 
после скрещивания, PM~--- параметр мутации, или относительная величина 
изменения генов выбранного портрета. На рис.~\ref{f1pr} приведена блок-схе\-ма 
программы настройки этих параметров.    
     
     Данный подход позволяет существенно сократить время работы 
коллектива свидетелей по со\-став\-ле\-нию субъективного портрета, поскольку 
генетическая процедура настройки параметров\linebreak эволюционного морфинга 
(нахождения оптимальной стратегии генетического консилиума) 
приспосабливает этот метод к конкретным особенностям реальных свидетелей.
     

\section{Экспериментальная часть}


     Для проверки предложенного метода настройки параметров ГК с 
использованием модели вир-\linebreak\vspace*{-12pt}
\pagebreak

\noindent
\begin{center} %fig2
%\vspace*{18pt}
\mbox{%
\epsfxsize=77.822mm
\epsfbox{pro-2.eps}
}
\end{center}
\vspace*{6pt}
{{\figurename~2}\ \ \small{Сходимость метода эволюционного морфинга при пяти различных начальных 
приближениях}}
%\end{center}
%\vspace*{6pt}


\bigskip
\bigskip
\addtocounter{figure}{1}


\noindent
туального свидетеля были проведены три группы 
экспериментов.

     
    В первой группе экспериментов проверялась сходимость результатов 
составления субъективного портрета с использованием ГК независимо от 
начальных условий~--- виртуальным свидетелям были заданы низкие 
коэффициенты художественных способностей. На рис.~2 приведены 
результаты сходимости метода для группы из девяти виртуальных свидетелей 
при пяти различных начальных приближениях. Несмотря на то что коллективы 
виртуальных свидетелей <<стартовали>> исходя из популяций портретов с 
низкими значениями~$K_S$, они неизменно получали лучшие результаты.
     
     Во второй группе экспериментов с виртуальными свидетелями 
осуществлялась проверка влияния параметров настройки ГК на качество 
восстанавливаемого портрета.

     Для коллектива из девяти виртуальных свидетелей с известными 
параметрами~$K_a$, $K_0$ и~$\alpha$ было испытано большое число 
вариантов составления портретов для разных наборов параметров 
эволюционного морфинга KPM, NC, NM и PM. Наблюдалось значительное 
влияние параметров ГК на конечный результат. 
     
     Из анализа этих экспериментов был сделан вывод о том, что настройка 
параметров ГК может существенно повысить качество составления 
субъективного портрета. 


     В третьей группе экспериментов проверялась работоспособность метода 
настройки ГК и качество его работы с реальными свидетелями. В одном из 
экспериментов, результаты которого приведены ниже, были предварительно 
измерены способности пяти свидетелей. Способности свидетелей 
характеризовались коэффициентами $0{,}03 <K_a < 0{,}12$, 
$0{,}93 <K_0< 0{,}98$ и $4\cdot 10^{-5} <\alpha< 7\cdot 10^{-5}$. С~использованием 
полученных значений была проведена настройка метода ГК.
     
     Затем реальным свидетелям на непродолжительное время были 
предъявлены фотографии в фас и в профиль неизвестного им лица 
(рис.~\ref{f3pr},\,\textit{а}). Далее свидетели с использованием ГК, 
настроенного на их способности, составляли субъективный портрет. 
Усредненный результат по пяти свидетелям после первой итерации с 
$K_S = 0{,}77$ приведен на рис.~\ref{f3pr},\,\textit{б}. На 
рис.~\ref{f3pr},\,\textit{в} приведены результаты со\-став\-ле\-ния субъективного 
портрета этой группой. Для составления субъективного портрета с 
$K_S = 0{,}92$ потребовалось всего 6~итераций. 


     Метод эволюционного морфинга, при\-ме\-ня\-емый шотландскими 
исследователями в~[10], отличается тем, что свидетели при проведении 
эволюционного морфинга работают поодиночке. При этом в качестве 
окончательного результата используется усредненный портрет. 
В~предложенном же методе свидетели работают в составе ГК совместно, в 
результате чего возникает синергетический эффект усиления 
интеллекта~\cite{4pr}, а составленные субъективные портреты обладают 
большей схо\-жестью с оригиналом. 
     
     С использованием коллектива виртуальных свидетелей была проведена 
сравнительная проверка метода ГК и шотландского метода. Для приведенного 
выше случая величина коэффициента сходства~$K_S$ составила, как было 
показано, 0,92, в то время как для метода, описанного в~[10], $K_S = 0{,}86$. 
{\looseness=1

}
     
     Эксперименты с различными коллективами виртуальных свидетелей и 
исходными портретами показали, что коэффициенты сходства у шотландского 
метода лучше, как и следовало ожидать, чем усредненные коэффициенты после 
первой итерации, но всегда хуже, чем в ГК после шести итераций. 
     
     Здесь следует отметить, что метод шотландских исследователей принят 
на вооружение английской полицией. Было бы интересно провести 
сравнительный анализ двух методов в реальных условиях криминалистической 
практики.
     
\section{Заключение}
     
     В результате проведенных исследований получены следующие 
результаты:
     \begin{itemize}
\item разработан и исследован новый метод со\-став\-ле\-ния субъективного 
портрета, основанный на применении эволюционного морфинга и модели 
виртуального свидетеля;\end{itemize}
%\pagebreak


\end{multicols}

\begin{figure} %fig3
\vspace*{1pt}
\begin{center}
\mbox{%
\epsfxsize=164.062mm
\epsfbox{pro-3.eps}
}
\end{center}
\vspace*{-9pt}
\Caption{Результаты составления субъективного портрета группой из пяти свидетелей 
методом настраиваемого эволюционного морфинга
\label{f3pr}}
\vspace*{6pt}
\end{figure}
     

\begin{multicols}{2}

\noindent
\begin{itemize}
\item использование модели виртуального свидетеля позволяет проводить 
тестирование способностей свидетелей к составлению субъективного 
портрета, настраивать параметры метода на получение оптимальных 
результатов и осуществлять квалиметрию метода;
\item показано, что настройка этих параметров для конкретного состава 
свидетелей позволяет улучшить качество субъективного портрета и 
определить точность его составления;
\item использование модели виртуальных свидетелей позволяет 
сравнивать методы составления субъективных портретов, применяемые 
разными авторами.
\end{itemize}

     Направлением дальнейших исследований является апробация метода 
эволюционного морфинга в реальных условиях криминалистической практики. 
     
{\small\frenchspacing
{%\baselineskip=10.8pt
\addcontentsline{toc}{section}{Литература}
\begin{thebibliography}{99}

\bibitem{1pr}
\Au{Муравев-Витковский А.\,В.}
Габитоскопия. {\sf http:// www.expert.aaanet.ru/rabota/gabito.htm}.

\bibitem{2pr}
\Au{Протасов В.\,И., Панфилов~Д.\,С., Здоровеющев~Ю.\,Ю.}
Генерация фоторобота с помощью сетевого че\-ло\-ве\-ко-ма\-шин\-но\-го интеллекта~// 
Международная на\-уч\-но-тех\-ни\-че\-ская конференция <<Интеллектуальные многопроцессорные 
системы ИМС-99>>.~--- Таганрог, 1999. С.~106--107.

\bibitem{3pr}
\Au{Протасов В.\,И.}
Генерация новых знаний сетевым че\-ло\-ве\-ко-машинным интеллектом. Постановка 
проб\-ле\-мы~// Нейрокомпьютеры. Разработка и применение, 2001. № \,7--8.

\bibitem{4pr}
\Au{Протасов В.\,И.}
Метасистемный эффект самоорганизации интеллекта более высокого уровня из 
искусственных и естественных компонентов~// Сб.\ научных трудов IV Всероссийской 
на\-уч\-но-тех\-ни\-че\-ской конференции <<Нейроинформатика-2002>>.~--- М., 2002. 
С.~33--40.

\bibitem{5pr}
\Au{Протасов В.\,И.}
Тестирование гибридного человеко-машинного интеллекта на шахматных задачах~//\linebreak 
Материалы международной научно-технической конференции <<Искусственный интеллект 
2002>>.~---\linebreak Кацивели, Крым, 2002. С.~348--353.

\bibitem{6pr}
\Au{Шустов Е.\,В., Протасов~В.\,И., Витиска~Н.\,И.}
Решение задачи формирования инвестиционного портфеля кластером компьютеров с 
использованием метода <<двухступенчатого усиления интеллекта>>~//\linebreak
Российский 
экономический Интернет-журнал, 2003. {\sf http://www.e-rej.ru/Articles/2003/Invest.pdf}.

\bibitem{7pr}
\Au{Протасов В.\,И., Витиска~Н.\,И., Шустов~Е.\,В.}
Решение многокритериальной задачи назначений методом генетического консилиума~// 
Российский экономический Интернет-журнал, 2003.\linebreak 
{\sf http://www.e-rej.ru/Articles/2003/Counsil.pdf}.

\bibitem{8pr}
\Au{Протасов В.\,И., Дружинин А.\,А., Михайлов~Л.\,В.}
Методика восстановления субъективного портрета коллективом свидетелей с 
использованием 3D-мор\-фин\-га~// Программные продукты и системы, 2007.  №\,1(77). 
С.~21--24. 

\bibitem{9pr}
FaceGen Modeller~3.1. {\sf http://www.facegen.com/\linebreak modeller.htm}.

\label{end\stat}

\bibitem{10pr}
\Au{Frowd C.\,D., Hancock~P.\,J.\,B., Carson~D.}
EvoFIT: A holistic, evolutionary facial identification technique for creating composites~// 
Association for Computing Machinery Transactions on Applied Psychology, 2004. Vol.~1. 
P.~1--21. 
 \end{thebibliography}
}
}
\end{multicols}   %12
                        


%\def\stat{rez}
{%\hrule\par
%\vskip 7pt % 7pt
\raggedleft\Large \bf%\baselineskip=3.2ex
Р\,Е\,Ц\,Е\,Н\,З\,И\,И \vskip 17pt
    \hrule
    \par
\vskip 6pt plus 6pt minus 3pt }

%\thispagestyle{headings} %с верхним колонтитулом
%\thispagestyle{myheadings} %с нижним колонтитулом, но в верхнем РЕЦЕНЗИИ

\def\tit{НОВАЯ КНИГА И.\,Н.~СИНИЦЫНА, А.\,С.~ШАЛАМОВА <<ЛЕКЦИИ ПО ТЕОРИИ 
ИНТЕГРИРОВАННОЙ ЛОГИСТИЧЕСКОЙ ПОДДЕРЖКИ>> (М.: ТОРУС ПРЕСС, 2012. 624~с.)}

%1
\def\aut{Д.ф.-м.н., профессор С.\,Я.~Шоргин}

\def\auf{\ }

\def\leftkol{\ % РЕЦЕНЗИИ
}

\def\rightkol{ \ } 

%\def\leftkol{\ } % ENGLISH ABSTRACTS}

%\def\rightkol{\ } %ENGLISH ABSTRACTS}

%\def\leftkol{РЕЦЕНЗИИ}

%\def\rightkol{РЕЦЕНЗИИ}

\titele{\tit}{\aut}{\auf}{\leftkol}{\rightkol}
\vspace*{-18pt}


     \label{st\stat}

     \begin{multicols}{2}
     {\small
     {\baselineskip=10.1pt
     

      В книге представлено системное изложение теоретических основ одного из новейших 
направлений в \mbox{об\-ласти} экономики послепродажного обслуживания изделий наукоемкой 
продукции (ИНП) длительного пользования~--- интегрированной логистической поддержки
(ИЛП). 
{\looseness=1

}

Приведены также результаты новых работ, выполненных в Институте проблем информатики 
Российской академии наук в рамках научного направления <<Информационные технологии и 
анализ сложных сис\-тем>>.
 {%\looseness=1

}
     
      Излагаемые в книге научные подходы позво\-ляют карди\-наль\-но реформировать 
существующие системы производства и эксплуатации ИНП путем создания и внед\-ре\-ния 
методов рационального и оптимального управ\-ле\-ния процессами расходования 
вре\-мен\-н$\acute{\mbox{ы}}$х, 
мате\-ри\-аль\-ных, трудовых и других ресурсов на всех стадиях жизненного цикла изделий (ЖЦИ) по 
критериям экономической целесообразности и эф\-фек\-тив\-ности.
  {\looseness=1

}
    
      В книге приведен краткий обзор причин возник\-новения и
      развития CALS-методологии как основы 
современных международных стандартов по созданию и функционированию глобальных 
ин\-фор\-ма\-ци\-он\-но-ком\-му\-ни\-ка\-ци\-он\-ных систем, ее ключевых возможностей и эффективности 
результатов ее использования. 
Авторы %\linebreak 
предлагают ряд научных обоснований для разработки 
единой теории проектирования и управления систем ИЛП для полноценного использования 
преимуществ %\linebreak
 суще\-ст\-ву\-ющей методологии, определяют \mbox{общую} структурную схему 
комплексной системы <<ИНП-СППО>> и необходимость разработки для ее описания 
гибридных стохастических моделей.
{%\looseness=1

}

%\columnbreak
      
      Книга состоит из пяти частей, где последовательно излагается материал по каждой из 
следующих тем: <<Интегрированная логистическая поддержка>>, <<Теория гибридных 
стохастических систем и компьютерная поддержка исследований и разработок>>, <<Основы 
математического моделирования, анализа и синтеза систем послепродажного обслуживания>>, 
<<Определение и анализ показателей экспортного потенциала ИНП при проектировании>>, 
<<Задачи управления поддержкой послепродажного обслуживания>>, а также 
<<Моделирование инвестиционных процессов ИЛП в условиях неравновесных финансовых 
рынков>>. 
   
      В конце каждой главы приведены выводы и даны вопросы и задания для 
самоконтроля. В~приложениях содержатся основные определения по программам работ по 
анализу ИЛП, логистическим базам данных и компьютерным решениям, эквивалентной статистической 
линеаризации нелинейных преобразований ИЛП, справочный материал, а также развернутые 
уравнения для вероятностных характеристик.


      \def\leftkol{РЕЦЕНЗИИ}

\def\rightkol{РЕЦЕНЗИИ} 

      
      Книга заинтересует широкий круг специалистов и может быть использована научными 
проектными организациями в сфере промышленного производства ИНП. Большое количество 
иллюстраций, примеров и вопросов, обращенных к читателю, позволяет использовать книгу 
также в качестве учебного пособия для студентов и аспирантов машиностроительных, 
транспортных и~других специальностей, а также для самостоятельного изучения. 
{%\looseness=-1

}

Книга 
представляет несомненный интерес для специалистов и студентов в области прикладной 
математики и информатики.
    

}

}
\end{multicols}

%\newpage

\include{obchak}



\def\stat{authorsrus}
{%\hrule\par
%\vskip 7pt % 7pt
\raggedleft\Large \bf%\baselineskip=3.2ex
О\,Б\ \ А\,В\,Т\,О\,Р\,А\,Х \vskip 17pt
    \hrule
    \par
\vskip 21pt plus 8pt minus 4pt }


\def\tit{\ }

\def\aut{\ }

\def\auf{\ }

\def\leftkol{\ } % ENGLISH ABSTRACTS}

\def\rightkol{ОБ АВТОРАХ} %ENGLISH ABSTRACTS}

\titele{\tit}{\aut}{\auf}{\leftkol}{\rightkol}
      
            \label{st\stat}



\vspace*{24pt}

\begin{multicols}{2}




\noindent
\textbf{Архипов Олег Петрович} (р.\ 1948)~---
кандидат технических наук, директор Орловского филиала Института проб\-лем информатики
Российской академии наук
%302025, г.Орел, Московское шоссе, д.137

\vspace*{3pt}

\noindent
\textbf{Бирюкова Татьяна Константиновна} (р.\ 1968)~---
кандидат фи\-зи\-ко-ма\-те\-ма\-ти\-че\-ских наук, старший научный сотрудник Института проб\-лем информатики
Российской академии наук

\vspace*{3pt}

\noindent 
\textbf{Бобков  Сергей Геннадьевич} (р.\ 1955)~---
доктор технических наук,  заведующий отделением На\-уч\-но-ис\-сле\-до\-ва\-тель\-ско\-го 
института системных исследований Российской академии наук
%117218, Москва, Нахимовский просп., 36, к.1 

\vspace*{3pt}

\noindent \textbf{Васильев Николай Семенович} (р.\ 1952)~--- доктор 
фи\-зи\-ко-ма\-те\-ма\-ти\-че\-ских наук, профессор, 
МГТУ им.\ Н.\,Э.~Баумана 
%, Москва 105005, 2-я Бауманская ул., д.~5,

\vspace*{3pt}

\noindent
\textbf{Гершкович Максим Михайлович} (р.\ 1968)~---
старший научный сотрудник Института проб\-лем информатики
Российской академии наук

\vspace*{3pt}

\noindent 
\textbf{Дьяченко Юрий Георгиевич} (р.\ 1958)~--- кандидат технических наук, 
старший научный сотрудник Института проб\-лем информатики
Российской академии наук

\vspace*{3pt}

\noindent 
\textbf{Ерошенко Александр Андреевич} (р.\ 1989)~--- аспирант кафедры 
математической статистики факультета вычисли\-тельной математики и кибернетики 
Московского государственного университета им.\ М.\,В.~Ломоносова
%119991, Москва ГСП-1, Ленинские горы, д.\ 1, стр. 52

\vspace*{3pt}
 
\noindent 
\textbf{Захаров Виктор Николаевич} (р.\ 1948)~--- 
доктор технических наук, доцент, ученый секретарь Института проб\-лем информатики
Российской академии наук

\vspace*{3pt}

\noindent
\textbf{Зейфман Александр Израилевич} (р.\ 1954)~---
доктор фи\-зи\-ко-ма\-те\-ма\-ти\-че\-ских наук, профессор, 
заведующий кафедрой Вологодского государственного университета; 
старший научный сотрудник Института проб\-лем информатики
Российской академии наук; главный научный сотрудник ИСЭРТ Российской академии наук

\vspace*{3pt}

\noindent
\textbf{Зыкин Сергей Владимирович} (р.\ 1959)~--- 
доктор технических наук, профессор, заведующий лабораторией Института математики 
им.\ С.\,Л.~Соболева Сибирского отделения Российской академии наук, Новосибирск 
%630090, пр.\ ак.\ Коптюга, 4 

\vspace*{4pt}

\noindent
\textbf{Киреев Владимир Иванович} (р.\ 1938)~---
доктор фи\-зи\-ко-ма\-те\-ма\-ти\-че\-ских наук, профессор Московского 
государственного горного университета
%Адрес: Россия, 119991, г. Москва, Ленинский проспект, д. 6

%\columnbreak

\vspace*{4pt}

\noindent
\textbf{Козеренко Елена Борисовна} (р.\ 1959)~---
кандидат филологических наук, заведующая лабораторией Института проб\-лем информатики
Российской академии наук

\vspace*{4pt}

\noindent
\textbf{Королев Виктор Юрьевич} (р.\ 1954)~--- доктор
фи\-зи\-ко-ма\-те\-ма\-ти\-че\-ских наук, профессор кафедры математической 
статистики факультета вычисли\-тельной математики и кибернетики 
Московского государственного университета; 
ведущий научный сотрудник Института проб\-лем информатики
Российской академии наук

\vspace*{4pt}

\noindent
\textbf{Коротышева Анна Владимировна} (р.\ 1988)~---
старший преподаватель Вологодского государственного университета

\vspace*{4pt}

\noindent 
\textbf{Кун Де Турк} (р.\ 1981)~--- научный сотрудник 
исследовательской группы SMACS факультета телекоммуникаций и обработки информации
Университета Гента, Бельгия
%В-9000 Гент, Бельгия

\vspace*{4pt}

\noindent
\textbf{Лупенцов Олег Сергеевич} (р.\ 1986)~---
аспирант Омского государственного института сервиса
%Омск 644043, ул.\ Певцова 13

\vspace*{4pt}

\noindent
\textbf{Лучко Олег Николаевич} (р.\ 1961)~---
кандидат педагогических наук, профессор, заведующий кафедрой 
Омского государственного института сервиса
%Омск 644043, ул.\ Певцова 13

\vspace*{4pt}

\noindent
\textbf{Малашенко Юрий Евгеньевич} (р.\ 1946)~---
доктор фи\-зи\-ко-ма\-те\-ма\-ти\-че\-ских наук, заведующий сектором 
Вычислительного центра им.\ А.\,А.~Дородницына Российской академии наук
%Адрес: 119333, Москва, ул. Вавилова, 40,

\vspace*{4pt}

\noindent
\textbf{Маньяков Юрий Анатольевич} (р.\ 1984)~---
кандидат технических наук, научный сотрудник Орловского филиала Института проб\-лем информатики
Российской академии наук
%302025, г.Орел, Московское шоссе, д.137

\vspace*{4pt}

\noindent
\textbf{Маренко Валентина Афанасьевна} (р.\ 1951)~---
кандидат технических наук, доцент, старший научный сотрудник 
Института математики им.\ С.\,Л.~Соболева Сибирского отделения Российской академии наук
%Новосибирск 630090, пр. ак. Коптюга, 4 

\vspace*{3pt}

\noindent 
\textbf{Морозов Евсей Викторович} (р.\ 1947)~--- доктор 
фи\-зи\-ко-ма\-те\-ма\-ти\-че\-ских, профессор, ведущий научный сотрудник 
Института прикладных математических исследований Карельского научного центра Российской
академии наук; 
%%185910 Россия, Республика Карелия, г.\ Петрозаводск, ул.\ Пушкинская, 11
профессор Петрозаводского государственного университета, Петрозаводск
%185910 Россия, Республика Карелия, г.\ Петрозаводск, пр.\ Ленина, 33

%\pagebreak

\vspace*{3pt}

\noindent
\textbf{Назарова Ирина Александровна} (р.\ 1966)~---
кандидат фи\-зи\-ко-ма\-те\-ма\-ти\-че\-ских наук, 
научный сотрудник Вычислительного центра им.\ А.\,А.~Дородницына Российской академии наук 
%Адрес: 119333, Москва, ул. Вавилова, 40

\vspace*{3pt}

\noindent
\textbf{Павлов Игорь Валерианович} (р.\ 1945)~--- 
доктор фи\-зи\-ко-ма\-те\-ма\-ти\-че\-ских наук, профессор МГТУ им.\ Н.\,Э.~Баумана 
%Москва 105005, 2-я Бауманская ул., д.~5 

%\pagebreak

\vspace*{3pt}

\noindent 
\textbf{Потахина Любовь Викторовна} (р.\ 1989)~--- аспирантка
Института прикладных математических исследований Карельского научного центра
Российской академии наук; 
%%185910 Россия, Республика Карелия, г.\ Петрозаводск, ул.\ Пушкинская, 11
инженер Петрозаводского государственного университета, Петрозаводск
%185910 Россия, Республика Карелия, г.\ Петрозаводск, пр.\ Ленина, 33

\vspace*{3pt}

\noindent 
\textbf{Рождественский Юрий Владимирович} (р.\ 1952)~--- 
кандидат технических наук, заведующий сектором Института проб\-лем информатики
Российской академии наук

\vspace*{3pt}

\noindent 
\textbf{Синицын Игорь Николаевич} (р.\ 1940)~--- доктор технических наук,
профессор, заслуженный деятель\linebreak\vspace*{-12pt}

\columnbreak

\noindent
 науки РФ, заведующий отделом Института проб\-лем информатики
Российской академии наук

\vspace*{7pt}


\noindent
\textbf{Сиротинин Денис Олегович} (р.\ 1984)~---
кандидат технических наук, научный сотрудник Орловского филиала Института проб\-лем информатики
Российской академии наук
%302025, г.Орел, Московское шоссе, д.137

\vspace*{7pt}

%\columnbreak

\noindent 
\textbf{Соколов  Игорь Анатольевич} (р.\ 1954)~--- академик (действительный член) Российской 
академии наук, доктор технических наук, директор Института проб\-лем информатики
Российской академии наук

\vspace*{7pt}

\noindent
\textbf{Степченков Юрий Афанасьевич} (р.\ 1951)~---
кандидат технических наук, заведующий отделом Института проб\-лем информатики
Российской академии наук

\vspace*{7pt}

\noindent
\textbf{Сурков Алексей Викторович} (р.\ 1978)~--- 
старший научный сотрудник На\-уч\-но-ис\-сле\-до\-ва\-тель\-ско\-го 
института системных исследований Российской академии наук
%117218, Москва, Нахимовский просп., 36, к.1 

\vspace*{7pt}

\noindent 
\textbf{Шестаков Олег Владимирович} (р.\ 1976)~--- доктор 
фи\-зи\-ко-ма\-те\-ма\-ти\-че\-ских, доцент кафедры математической статистики 
факультета вычисли\-тельной математики и кибернетики Московского 
государственного университета им.\ М.\,В.~Ломоносова; 
%119991, Москва ГСП-1, Ленинские горы, д.\ 1, стр. 52
старший научный сотрудник Института проб\-лем информатики
Российской академии наук
%, Москва 119333, ул. Вавилова, д.~44, корп.~2

\vspace*{7pt}

\noindent 
\textbf{Шоргин Сергей Яковлевич} (р.\ 1952.)~--- доктор
фи\-зи\-ко-ма\-те\-ма\-ти\-че\-ских наук, профессор, заместитель директора Института 
проб\-лем информатики Российской академии наук





%%%%%%%%%%%%%%%%%%%%%%%%%%%%%%%%%%%%%%%%%%%%%%%%%%%%%%%%%%%%%%%%%%%%%%%%%%%%%%%




%\def\rightkol{ОБ АВТОРАХ}
%\def\leftkol{ОБ АВТОРАХ}

 \label{end\stat}





%\def\leftfootline{\small{\textbf{\thepage}
%\hfill ИНФОРМАТИКА И ЕЁ ПРИМЕНЕНИЯ\ \ \ том~7\ \ \ выпуск~1\ \ \ 2013}
%}%
% \def\rightfootline{\small{ИНФОРМАТИКА И ЕЁ ПРИМЕНЕНИЯ\ \ \ том~7\ \ \ выпуск~1\ \ \ 2013
%\hfill \textbf{\thepage}}}


%\thispagestyle{myheadings}



\end{multicols}

\newpage

%\vspace*{-48pt}
\begin{center}\LARGE
\textit{About Authors}
\end{center}

\thispagestyle{empty}
\def\tit{\ }

\def\aut{\ }

\def\auf{\ }


\def\leftkol{ABOUT AUTHORS}

\def\rightkol{ABOUT AUTHORS}

\vspace*{-18pt}

\titele{\tit}{\aut}{\auf}{\leftkol}{\rightkol}

%\vspace*{36pt}

\def\rightmark{{\noindent\hbox to \textwidth{\hfill\small ABOUT AUTHORS
%\hfill \large\bf\thepage
}}}
\def\leftmark{{\noindent\parbox{\textwidth}{
%\raggedleft\large\bf\thepage \hfill
\small\textrm{ABOUT AUTHORS}\hfill}}}


\def\leftfootline{\small{\textbf{\thepage}
\hfill ИНФОРМАТИКА И ЕЁ ПРИМЕНЕНИЯ\ \ \ том~6\ \ \ выпуск~2\ \ \ 2012}
}%
 \def\rightfootline{\small{ИНФОРМАТИКА И ЕЁ ПРИМЕНЕНИЯ\ \ \ том~6\ \ \ выпуск~2\ \ \ 2012
\hfill \textbf{\thepage}}}


\begin{multicols}{2}

\noindent
\textbf{Agalarov Yaver M.} (b.\ 1952)~--- Candidate of Science (PhD)
in technology, 
leading scientist, Institute of Informatics Problems, Russian Academy of Sciences

\vspace*{5pt}


  \noindent
\textbf{Bosov Alexey V.} (b.\ 1969)~--- Doctor of Science in technology, Head of
Laboratory, Institute of Informatics Problems, Russian Academy of Sciences

\vspace*{5pt}


\noindent
\textbf{Dulin Sergey K.} (b.\ 1950)~--- Doctor of Science in technology, 
professor, senior scientist, Institute of Informatics Problems, Russian Academy of Sciences

\vspace*{5pt}

\noindent
\textbf{Gorshenin Andrey K.}~--- (b.\ 1986)~--- Candidate of Science (PhD)
in physics and mathematics,
senior scientist, Institute of Informatics Problems, Russian Academy of Sciences

\vspace*{5pt}

\noindent
\textbf{Kalenov Nikolay E.}  (b.\ 1945)~--- Doctor of Science in technology,
professor, Director, Library for Natural Sciences,  Russian Academy of Sciences 

\vspace*{5pt}

\noindent
\textbf{Kalinichenko Leonid A.} (b.\ 1937)~--- Doctor of Science in physics and mathematics, 
professor, Honored scientist of RF, 
Head of Laboratory, Institute of Informatics Problems, Russian Academy of Sciences 

\vspace*{5pt}

\noindent
\textbf{Karpov Alexey A.} (b.\ 1978)~--- Candidate of Science (PhD) in technology, 
senior scientist, St.\ Petersburg Institute for
Informatics and Automation,  Russian Academy of Sciences

\vspace*{5pt}

\noindent
\textbf{Kuznetsov Igor P.} (b.\ 1938)~--- Doctor of Science in technology, 
professor, principal scientist, Institute of Informatics Problems, Russian Academy of Sciences

\vspace*{5pt}


\noindent
\textbf{Markova Natalia A.} (b.\ 1950)~--- Candidate of Science (PhD) in
physics and mathematics, leading scientist,  
Institute of Informatics Problems, Russian Academy of Sciences

\vspace*{5pt}

\noindent
\textbf{Nikolaev Andrey V.} (b.\ 1985)~--- Candidate of Science (PhD) in technology, 
senior lecturer, Tchaikovsky Technological Institute, Branch of the Izhevsk State Technical 
University

\vspace*{6pt}

\noindent
\textbf{Pavlov Igor V.} (b.\ 1945)~---  Doctor of Science in physics and mathematics,
professor, Bauman Moscow State Technical University

\vspace*{6pt}

%\columnbreak

\noindent
\textbf{Rozenberg Igor N.} (b.\ 1965)~--- Doctor of Science in technology, 
First Deputy Director General, Research \& Design Institute for Information 
Technology, Signalling and Telecommunications on Railway Transport (JSC NIIAS)

\vspace*{6pt}


\noindent
\textbf{Semenov Konstantin K.} (b.\ 1986)~--- MPhil, 
associate professor, St.\ Petersburg State Polytechnical University

\vspace*{6pt}

\noindent
\textbf{Sharnin Mikhail M.} (b.\ 1959)~--- Candidate of Science (PhD) 
in technology, senior scientist, Institute of Informatics Problems, Russian Academy of Sciences

\vspace*{6pt}

\noindent 
\textbf{Shestakov Oleg V.} (b.\ 1976)~--- Candidate of Science (PhD) in physics and mathematics,
associate professor, Department of Mathematical Statistics, Faculty of Computational Mathematics and Cybernetics,
M.\,V.~Lomonosov Moscow State University; senior scientist, Institute of Informatics Problems, 
Russian Academy of Sciences

\vspace*{6pt}

\noindent
\textbf{Stupnikov Sergey A.} (b.\ 1978)~--- Candidate of Science (PhD) in technology, 
senior scientist, Institute of Informatics Problems, Russian Academy of Sciences 

\vspace*{6pt}

\noindent
\textbf{Umansky Vladimir I.} (b.\ 1954)~--- Candidate of Science (PhD) in technology, 
Director General, ``IntechGeoTrans'' Closed Joint Stock Company

\vspace*{6pt}

\noindent
\textbf{Zhevnerchuk Dmitry V.} (b.\ 1978)~--- Candidate of Science (PhD) in technology, 
associate professor, Tchaikovsky Technological Institute, Branch of the Izhevsk State 
Technical University

%\vspace*{6pt}

\def\leftfootline{\small{\textbf{\thepage}
\hfill ИНФОРМАТИКА И ЕЁ ПРИМЕНЕНИЯ\ \ \ том~6\ \ \ выпуск~2\ \ \ 2012}
}%
 \def\rightfootline{\small{ИНФОРМАТИКА И ЕЁ ПРИМЕНЕНИЯ\ \ \ том~6\ \ \ выпуск~2\ \ \ 2012
\hfill \textbf{\thepage}}}



%\thispagestyle{myheadings}

\end{multicols}
\newpage



%\tableofcontents

%\end{document}

%\def\stat{cont}
{%\hrule\par
%\vskip 7pt % 7pt
\raggedleft\Large \bf%\baselineskip=3.2ex
А\,В\,Т\,О\,Р\,С\,К\,И\,Й\ \ У\,К\,А\,З\,А\,Т\,Е\,Л\,Ь\ \ З\,А\ \ 2\,0\,1\,0 г. \vskip 17pt
    \hrule
    \par
\vskip 21pt plus 6pt minus 3pt }

\label{st\stat}

\def\tit{\ }

\def\aut{\ }
\def\auf{\ }

\def\leftkol{\ } % ENGLISH ABSTRACTS}

\def\rightkol{\ } %АВТОРСКИЙ УКАЗАТЕЛЬ ЗА 2010 г.} %ENGLISH ABSTRACTS}

\titele{\tit}{\aut}{\auf}{\leftkol}{\rightkol}

\vspace*{-12pt}

{\tabcolsep=3pt
\begin{tabular}{p{388pt}rr}
&\textbf{Выпуск} & \textbf{Стр.}\\[6pt]
\hangindent=23pt\noindent\textbf{Арутюнян~А.\,Р.} Моделирование влияния деформаций отпечатков пальцев на 
точность\linebreak
\vspace*{-12pt}\\
\hspace*{23pt}дактилоскопической идентификации$\dotfill$&1&51\\
\hangindent=23pt\noindent\textbf{Архипов~О.\,П., Зыкова~З.\,П.} Интеграция гетерогенной информации о цветных 
пикселях\linebreak
\vspace*{-12pt}\\
\hspace*{23pt}и их цветовосприятии$\dotfill$&4&15\\
\hangindent=23pt\noindent\textbf{Баранов~С.\,И., Френкель~С.\,Л., Захаров~В.\,Н.} Полуформальная верификация 
цифрового устройства с конвейером, основанная на использовании алгоритмических машин\linebreak
\vspace*{-12pt}\\
\hspace*{23pt}состояния$\dotfill$&4&49\\
\textbf{Бекетова~И.\,В.} см.~Каратеев~С.\,Л.&&\\
\textbf{Белоусов~В.\,В.} см.~Синицын~И.\,Н.&&\\
\hangindent=23pt\noindent\textbf{Бенинг~В.\,Е., Королев~Р.\,А.} О предельном поведении мощностей критериев в 
случае\linebreak
\vspace*{-12pt}\\
\hspace*{23pt}распределения Лапласа$\dotfill$&2&63\\
\hangindent=23pt\noindent\textbf{Бенинг~В.\,Е., Сипина~А.\,В.} Асимптотическое разложение для мощности 
критерия,\linebreak
\vspace*{-12pt}\\
\hspace*{23pt}основанного на выборочной медиане, в случае распределения Лапласа$\dotfill$&1&18\\
\textbf{Бондаренко~А.\,В.} см.~Каратеев~С.\,Л.&&\\
\hangindent=23pt\noindent\textbf{Бородина~А.\,В., Морозов~Е.\,В.} Об оценивании асимптотики вероятности 
большого\linebreak
\vspace*{-12pt}\\
\hspace*{23pt}уклонения стационарной регенеративной очереди с одним прибором$\dotfill$&3&29\\
\hangindent=23pt\noindent\textbf{Бунтман~Н.\,В., Минель~Ж.-Л., Ле~Пезан~Д., Зацман~И.\,М.} Типология и 
компьютерное\linebreak
\vspace*{-12pt}\\
\hspace*{23pt}моделирование трудностей перевода$\dotfill$&3&77\\
\textbf{Визильтер~Ю.\,В.} см.~Каратеев~С.\,Л.&&\\
\hangindent=23pt\noindent\textbf{Гавриленко~С.\,В.} Оценки скорости сходимости распределений случайных сумм с 
безгранично делимыми индексами к нормальному закону$\dotfill$&4&81\\
\hangindent=23pt\noindent\textbf{Григорьева~М.\,Е., Шевцова~И.\,Г.} Уточнение неравенства 
Каца--Берри--Эссеена$\dotfill$&2&75\\
\hangindent=23pt\noindent\textbf{Грушо~А.\,А., Грушо~Н.\,А., Тимонина~Е.\,Е.} Поиск конфликтов в политиках 
безопасности: модель случайных графов$\dotfill$&3&38\\
\textbf{Грушо~Н.\,А.} см.~Грушо~А.\,А.&&\\
\hangindent=23pt\noindent\textbf{Гудков~В.\,Ю.} Математические модели изображения отпечатка пальца на основе 
описания линий$\dotfill$&1&58\\
\textbf{Гуртов~А.\,В.} см.~Лукьяненко~А.\,С.&&\\
\textbf{Желтов~С.\,Ю.} см.~Каратеев~С.\,Л.&&\\
\hangindent=23pt\noindent\textbf{Захаров~А.\,А., Серебряков~В.\,А.} Система управления электронной библиотекой 
LibMeta$\dotfill$&4&2\\
\textbf{Захаров~В.\,Н.} см.~Баранов~С.\,И.&&\\
\textbf{Захарова~Т.\,В.} см.~Матвеева~С.\,С.&&\\
\hangindent=23pt\noindent\textbf{Зацаринный~А.\,А., Чупраков~К.\,Г.} Некоторые аспекты выбора технологии для 
постро-\linebreak
\vspace*{-12pt}\\
\hspace*{23pt}ения систем отображения информации ситуационного центра$\dotfill$&3&59\\
\textbf{Зацман~И.\,М.} см.~Бунтман~Н.\,В.&&\\
\hangindent=23pt\noindent\textbf{Зейфман~А.\,И., Коротышева~А.\,В., Сатин~Я.\,А., Шоргин~С.\,Я.} Об 
устойчивости нестаци-\linebreak
\vspace*{-12pt}\\
\hspace*{23pt}онарных систем обслуживания с катастрофами$\dotfill$&3&9\\
\textbf{Зыкова~З.\,П.} см.~Архипов~О.\,П.&&\\
\hangindent=23pt\noindent\textbf{Илюшин~Г.\,Я., Соколов~И.\,А.} Организация управляемого доступа пользователей 
к\linebreak
\vspace*{-12pt}\\
\hspace*{23pt}разнородным ведомственным информационным ресурсам$\dotfill$&1&24\\
\hangindent=23pt\noindent\textbf{Кавагучи~Ю., Ульянов~В.\,В., Фуджикоши~Я.} Приближения для статистик, 
описывающих\linebreak
\vspace*{-12pt}\\
\hspace*{23pt}геометрические свойства данных большой размерности, с оценками 
ошибок$\dotfill$&1&12\\
\hangindent=23pt\noindent\textbf{Каратеев~С.\,Л., Бекетова~И.\,В., Ососков~М.\,В., Князь~В.\,А., 
Визильтер~Ю.\,В., Бондаренко~А.\,В., Желтов~С.\,Ю.} Автоматизированный контроль 
качества цифровых\linebreak
\vspace*{-12pt}\\
\hspace*{23pt}изображений для персональных документов$\dotfill$&1&65\\
\end{tabular}
}

\pagebreak

\def\leftkol{АВТОРСКИЙ УКАЗАТЕЛЬ ЗА 2010 г.} % ENGLISH ABSTRACTS}

\def\rightkol{АВТОРСКИЙ УКАЗАТЕЛЬ ЗА 2010 г.} %ENGLISH ABSTRACTS}

{\tabcolsep=3pt
\begin{tabular}{p{388pt}rr}
&\textbf{Выпуск} & \textbf{Стр.}\\[3pt]
\hangindent=23pt\noindent\textbf{Козеренко~Е.\,Б.} Лингвистические фильтры в статистических моделях машинного\linebreak
\vspace*{-12pt}\\
\hspace*{23pt}перевода$\dotfill$&2&83\\
\hangindent=23pt\noindent\textbf{Козеренко~Е.\,Б., Кузнецов~И.\,П.} Когнитивно-лингвистические представления в 
систе-\linebreak
\vspace*{-12pt}\\
\hspace*{23pt}мах обработки текстов$\dotfill$&3&69\\
\textbf{Князь~В.\,А.} см.~Каратеев~С.\,Л.&&\\
\hangindent=23pt\noindent\textbf{Колесников~А.\,В., Солдатов~С.\,А.} Алгоритм координации для гибридной 
интеллектуальной системы решения сложной задачи оперативно-производственного\linebreak
\vspace*{-12pt}\\
\hspace*{23pt}планирования$\dotfill$&4&61\\
\hangindent=23pt\noindent\textbf{Коновалов~М.\,Г.} О планировании потоков в системах вычислительных 
ресурсов$\dotfill$&2&3\\
\textbf{Конушин~А.\,С.} см.~Конушин~В.\,С.&&\\
\hangindent=23pt\noindent\textbf{Конушин~В.\,С., Кривовязь~Г.\,Р., Конушин~А.\,С.} Алгоритм распознавания людей 
в видео-\linebreak
\vspace*{-12pt}\\
\hspace*{23pt}последовательности по одежде$\dotfill$&1&74\\
\textbf{Корепанов~Э.\, Р.} см.~Синицын~И.\,Н.&&\\
\textbf{Королев~В.\,Ю.} см.~Соколов~И.\,А.&&\\
\textbf{Королев~Р.\,А.} см.~Бенинг~В.\,Е.&&\\
\textbf{Коротышева~А.\,В.} см.~Зейфман~А.\,И.&&\\
\hangindent=23pt\noindent\textbf{Кривенко~М.\,П.} Непараметрическое оценивание элементов байесовского 
клас\-си-\linebreak
\vspace*{-12pt}\\
\hspace*{23pt}фикатора$\dotfill$&2&13\\
\textbf{Кривовязь~Г.\,Р.} см.~Конушин~В.\,С.&&\\
\textbf{Крылов~А.\,С.} см.~Павельева~Е.\,А.&&\\
\hangindent=23pt\noindent\textbf{Крылов~В.\,А.} Моделирование и классификация многоканальных дистанционных\linebreak
\vspace*{-12pt}\\
\hspace*{23pt}изображений с использованием копул$\dotfill$&4&34\\
\hangindent=23pt\noindent\textbf{Крючин~О.\,В.} Разработка параллельных эвристических алгоритмов подбора 
весовых\linebreak
\vspace*{-12pt}\\
\hspace*{23pt}коэффициентов искусственной нейтронной сети$\dotfill$&2&53\\
\hangindent=23pt\noindent\textbf{Кудрявцев~А.\,А., Шоргин~С.\,Я.} Байесовские модели массового обслуживания и 
надеж-\linebreak
\vspace*{-12pt}\\
\hspace*{23pt}ности: характеристики среднего числа заявок в системе $M\vert M \vert 1\vert 
\infty$$\dotfill$&3&16\\
\hangindent=23pt\noindent\textbf{Кузнецов~А.\,А.} Связь между временными и структурно-топологическими 
характери-\linebreak
\vspace*{-12pt}\\
\hspace*{23pt}стиками диаграмм ритма сердца здоровых людей$\dotfill$&4&39\\
\textbf{Кузнецов~И.\,П.} см.~Козеренко~Е.\,Б.&&\\
\textbf{Ле~Пезан~Д.} см.~Бунтман~Н.\,В.&&\\
\hangindent=23pt\noindent\textbf{Лукьяненко~А.\,С., Морозов~Е.\,В., Гуртов~А.\,В.} Анализ сетевого протокола с общей 
функ-\linebreak
\vspace*{-12pt}\\
\hspace*{23pt}цией расширения окна передачи сообщения при конфликтах$\dotfill$&2&46\\
\hangindent=23pt\noindent\textbf{Лямин~О.\,О.} О предельном поведении мощностей критериев в случае обобщенного\linebreak
\vspace*{-12pt}\\
\hspace*{23pt}распределения Лапласа$\dotfill$&3&47\\
\hangindent=23pt\noindent\textbf{Маркин~А.\,В., Шестаков~О.\,В.} Асимптотики оценки риска при пороговой 
обработке\linebreak
\vspace*{-12pt}\\
\hspace*{23pt}вейвлет-вейглет коэффициентов в задаче томографии$\dotfill$&2&36\\
\hangindent=23pt\noindent\textbf{Матвеева~С.\,С., Захарова~Т.\,В.} Сети массового обслуживания с наименьшей 
длиной\linebreak
\vspace*{-12pt}\\
\hspace*{23pt}очереди$\dotfill$&3&22\\
\hangindent=23pt\noindent\textbf{Матюшенко~С.\,И.} Стационарные характеристики двухканальной системы 
обслужива-\linebreak
\vspace*{-12pt}\\
\hspace*{23pt}ния с переупорядочиванием заявок и распределениями фазового типа$\dotfill$&4&68\\
\textbf{Минель~Ж.-Л.} см.~Бунтман~Н.\,В.&&\\
\textbf{Морозов~Е.\,В.} см.~Бородина~А.\,В.&&\\
\textbf{Морозов~Е.\,В.} см.~Лукьяненко~А.\,С.&&\\
\textbf{Ососков~М.\,В.} см.~Каратеев~С.\,Л.&&\\
\hangindent=23pt\noindent\textbf{Павельева~Е.\,А., Крылов~А.\,С.} Поиск и анализ ключевых точек радужной 
оболочки\linebreak
\vspace*{-12pt}\\
\hspace*{23pt}глаза методом преобразования Эрмита$\dotfill$&1&79\\
\textbf{Печинкин~А.\,В.} см.~Френкель~С.\,Л.,&&\\
\hangindent=23pt\noindent\textbf{Протасов~В.\,И.} Составление субъективного портрета с использованием 
эволюционно-\linebreak
\vspace*{-12pt}\\
\hspace*{23pt}го морфинга и квалиметрия метода$\dotfill$&1&83\\
\hangindent=23pt\noindent\textbf{Рудаков~К.\,В., Торшин~И.\,Ю.} Вопросы разрешимости задачи распознавания 
вторичной\linebreak
\vspace*{-12pt}\\
\hspace*{23pt}структуры белка$\dotfill$&2&25\\
\textbf{Сатин~Я.\,А.} см.~Зейфман~А.\,И.&&\\
\hangindent=23pt\noindent\textbf{Сейфуль-Мулюков~Р.\,Б.} Нефть как носитель информации о своем 
происхождении,\linebreak
\vspace*{-12pt}\\
\hspace*{23pt}структуре и эволюции$\dotfill$&1&41\\
\end{tabular}
}

{\tabcolsep=3pt
\begin{tabular}{p{388pt}rr}
&\textbf{Выпуск} & \textbf{Стр.}\\[6pt]
\textbf{Семендяев~Н.\,Н.} см.~Синицын~И.\,Н.&&\\
\textbf{Серебряков~В.\,А.} см.~Захаров~А.\,А.&&\\
\textbf{Синицын~В.\,И.} см.~Синицын~И.\,Н.&&\\
\hangindent=23pt\noindent\textbf{Синицын~И.\,Н., Синицын~В.\,И., Корепанов~Э.\, Р., Белоусов~В.\,В., 
Семендяев~Н.\,Н.} Оперативное построение информационных моделей движения полюса 
Земли\linebreak
\vspace*{-12pt}\\
\hspace*{23pt}методами линейных и линеаризованных фильтров$\dotfill$&1&2\\
\textbf{Сипина~А.\,В.} см.~Бенинг~В.\,Е.&&\\
\hangindent=23pt\noindent\textbf{Соколов~И.\,А.} О работах заслуженного деятеля науки Российской Федерации 
И.\,Н.~Синицына в области информационных технологий и автоматизации (к 70-летию\linebreak
\vspace*{-12pt}\\
\hspace*{23pt}со дня рождения)$\dotfill$&3&84\\
\textbf{Соколов~И.\,А.} см.~Илюшин~Г.\,Я.&&\\
\hangindent=23pt\noindent\textbf{Соколов~И.\,А., Королев~В.\,Ю.} Предисловие$\dotfill$&2&2\\
\textbf{Солдатов~С.\,А.} см.~Колесников~А.\,В.&&\\
\hangindent=23pt\noindent\textbf{Степанов~С.\,Ю.} Использование координатного метода фрагментации 
коммутаторной\linebreak
\vspace*{-12pt}\\
\hspace*{23pt}нейронной сети для сокращения трафика$\dotfill$&2&57\\
\textbf{Тимонина~Е.\,Е.} см.~Грушо~А.\,А.&&\\
\textbf{Торшин~И.\,Ю.} см.~Рудаков~К.\,В.&&\\
\textbf{Ульянов~В.\,В.} см.~Кавагучи~Ю.&&\\
\textbf{Фазекаш~И.} см.~Чупрунов~А.\,Н.&&\\
\textbf{Френкель~С.\,Л.} см.~Баранов~С.\,И.&&\\
\hangindent=23pt\noindent\textbf{Френкель~С.\,Л., Печинкин~А.\,В.} Оценка времени самовосстановления в 
цифровых\linebreak
\vspace*{-12pt}\\
\hspace*{23pt}системах после сбоев, вызываемых переходными помехами$\dotfill$&3&2\\
\textbf{Фуджикоши~Я.} см.~Кавагучи~Ю.&&\\
\hangindent=23pt\noindent\textbf{Цискаридзе~А.\,К.} Математическая модель и метод восстановления позы человека 
по\linebreak
\vspace*{-12pt}\\
\hspace*{23pt}стереопаре силуэтных изображений$\dotfill$&4&27\\
\hangindent=23pt\noindent\textbf{Чупраков~К.\,Г.} К вопросу о размещении коллективных средств отображения в 
ситуа-\linebreak
\vspace*{-12pt}\\
\hspace*{23pt}ционном зале с заданными параметрами$\dotfill$&4&89\\
\textbf{Чупраков~К.\,Г.} см.~Зацаринный~А.\,А.&&\\
\hangindent=23pt\noindent\textbf{Чупрунов~А.\,Н., Фазекаш~И.} Законы повторного логарифма для числа 
безошибочных\linebreak
\vspace*{-12pt}\\
\hspace*{23pt}блоков при помехоустойчивом кодировании$\dotfill$&3&42\\
\textbf{Шевцова~И.\,Г.} см.~Григорьева~М.\,Е.&&\\
\hangindent=23pt\noindent\textbf{Шестаков~О.\,В.} Аппроксимация распределения оценки риска пороговой 
обработки вейвлет-коэффициентов нормальным распределением при использовании 
выбо-\linebreak
\vspace*{-12pt}\\
\hspace*{23pt}рочной дисперсии$\dotfill$&4&73\\
\textbf{Шестаков~О.\,В.} см.~Маркин~А.\,В.&&\\
\textbf{Шоргин~С.\,Я.} см.~Зейфман~А.\,И.&&\\
\textbf{Шоргин~С.\,Я.} см.~Кудрявцев~А.\,А.&&\\
\end{tabular}
}

%\thispagestyle{myheadings}
\def\leftfootline{\small{\textbf{\thepage}
\hfill ИНФОРМАТИКА И ЕЁ ПРИМЕНЕНИЯ\ \ \ том~4\ \ \ выпуск~4\ \ \ 2010}
}%
 \def\rightfootline{\small{ИНФОРМАТИКА И ЕЁ ПРИМЕНЕНИЯ\ \ \ том~4\ \ \ выпуск~4\ \ \ 2010
 \hfill \textbf{\thepage}}}
 \label{end\stat}

%
%Том 10 Выпуск 1-4 Год 2016

\def\stat{cont-e}
{%\hrule\par
%\vskip 7pt % 7pt
\raggedleft\Large \bf%\baselineskip=3.2ex
2\,0\,1\,6\ \ A\,U\,T\,H\,O\,R\ \ I\,N\,D\,E\,X \vskip 17pt
 \hrule
 \par
\vskip 21pt plus 6pt minus 3pt }

\label{st\stat}

\def\tit{\ }

\def\aut{\ }
\def\auf{\ }

\def\leftkol{\ } %2016 AUTHOR INDEX} % ENGLISH ABSTRACTS}

\def\rightkol{\ } %2016 AUTHOR INDEX} %ENGLISH ABSTRACTS}

\titele{\tit}{\aut}{\auf}{\leftkol}{\rightkol}

\def\leftfootline{\small{\textbf{\thepage}
\hfill INFORMATIKA I EE PRIMENENIYA~--- INFORMATICS AND APPLICATIONS\ \ \ 2016\
\ \ volume~10\ \ \ issue\ 4}
}%
 \def\rightfootline{\small{INFORMATIKA I EE PRIMENENIYA~--- INFORMATICS AND APPLICATIONS\ \ \ 2016\ \ \ volume~10\ \ \ issue\ 4
\hfill \textbf{\thepage}}}

\vspace*{-12pt}
\vspace*{-18pt}

{\tabcolsep=2.8pt
\begin{tabular}{p{382pt}cc}
&\textbf{Issue} & \textbf{Page}\\[6pt]
\Avtors{Agalarov~M.\,Ya.} see~Agalarov~Ya.\,M.&&\\
\Avtors{Agalarov~Ya.\,M., Agalarov~M.\,Ya., and
Shorgin~V.\,S.} About the optimal threshold of queue\linebreak
\\[-12pt]
\hspace*{23pt}length in a~particular problem of profit maximization
in the $M/G/1$ queuing system&2&70--79\\
\Avtors{Alexeyevsky~D.\,A.} BioNLP ontology extraction from 
a~restricted language corpus with\linebreak
\\[-12pt]
\hspace*{23pt}context-free grammars&1&119--128\\
\Avtors{Andreev~S.\,D.} see~Gaidamaka~Yu.\,V.&&\\
\Avtors{Andreev~S.\,D.} see~Ometov~A.\,Ya.&&\\
\Avtors{Arkhipov~O.\,P., Arkhipov~P.\,O., and Sidorkin~I.\,I.} The
option to create a~local coordinate\linebreak
\\[-12pt]
\hspace*{23pt}system for synchronization of selected images&3&91--97\\
\Avtors{Arkhipov~P.\,O.} see~Arkhipov~O.\,P.&&\\
\Avtors{Belousov~V.\,V.} see~Shnurkov~P.\,V.&&\\
\Avtors{Belousov~V.\,V.} see~Shnurkov~P.\,V.&&\\
\Avtors{Bening~V.\,E.} Calculation of~the~asymptotic deficiency
of~some statistical procedures based\linebreak
\\[-12pt]
\hspace*{23pt}on~samples with~random sizes&4&34--45\\
\Avtors{Borisov~A.\,V., Bosov~A.\,V., and Miller~G.\,B.} Modeling and
monitoring of VoIP connection&2&\hphantom{1}2--13\\
\Avtors{Bosov~A.\,V.} see~Borisov~A.\,V.&&\\
\Avtors{Briukhov~D.\,O.} see~Stupnikov~S.\,A.&&\\
\Avtors{Callaos~N.\,K.\ and Seyful-Mulyukov~R.\,B.} Complexity and
its information content&1&129--139\\
\Avtors{Chertok~A.\,V., Kadaner~A.\,I., Khazeeva~G.\,T., and
Sokolov~I.\,A.} Regime switching detection\linebreak
\\[-12pt]
\hspace*{23pt}for~the~Levy driven
Ornstein--Uhlenbeck process using CUSUM methods&4&46--56\\
\Avtors{Chichagov~V.\,V.} Asymptotic expansions of mean absolute
error of uniformly minimum variance unbiased and maximum likelihood
estimators on the one-parameter exponential\linebreak
\\[-12pt]
\hspace*{23pt}family model of lattice distributions&3&66--76\\
\Avtors{Danishevsky~V.\,I.} see~Kolesnikov A.\,V.&&\\
\Avtors{Fazliev~A.\,Z.} see~Kalinichenko~L.\,A.&&\\
\Avtors{Fedoseev~A.\,A.} What is behind the concept of ``knowledge in
small packages''&3&105--110\\
\Avtors{Gaidamaka~Yu.\,V., Andreev~S.\,D., Sopin~E.\,S.,
Samouylov~K.\,E., and Shorgin~S.\,Ya.} Interference analysis
of~the~device-to-device communications model with~regard to~a~signal\linebreak
\\[-12pt]
\hspace*{23pt}propagation environment&4&\hphantom{1}2--10\\
\Avtors{Gasilov~A.\,V.} see~Yakovlev~O.\,A.&&\\
\Avtors{Goncharov~A.\,V.\ and Strijov~V.\,V.} Metric time series
classification using weighted dynamic\linebreak
\\[-12pt]
\hspace*{23pt}warping relative to centroids of classes&2&36--47\\
\Avtors{Gordov~E.\,P.} see~Kalinichenko~L.\,A.&&\\
\Avtors{Gorshenin~A.\,K.} Concept of online service for stochastic
modeling of real processes&1&72--81\\
\Avtors{Gorshenin~A.\,K.} see~Shnurkov~P.\,V.&&\\
\Avtors{Gorshenin~A.\,K.} see~Shnurkov~P.\,V.&&\\
\Avtors{Grusho~A.\,A., Grusho~N.\,A., Zabezhailo~M.\,I., and
Timonina~E.\,E.} Integration of statistical and\linebreak
\\[-12pt]
\hspace*{23pt}deterministic methods for
analysis of information security&3&2--8\\
\Avtors{Grusho~A.\,A., Zabezhailo~M.\,I., and Zatsarinny~A.\,A.} On
the advanced procedure to reduce\linebreak
\\[-12pt]
\hspace*{23pt}calculation of Galois closures&4&\hphantom{1}96--104\\
\Avtors{Grusho~N.\,A.} see~Grusho~A.\,A.&&\\
\Avtors{Havanskov~V.\,A.} see~Minin~V.\,A.&&\\
\Avtors{Inkova~O.\,Yu.} see~Zatsman~I.\,M.&&\\
\Avtors{Isachenko~R.\,V.\ and Strijov~V.\,V.} Metric learning in
multiclass time series classification\linebreak
\\[-12pt]
\hspace*{23pt}problem&2&48--57\\
\end{tabular}
}
\pagebreak

\def\leftfootline{\small{\textbf{\thepage}
\hfill INFORMATIKA I EE PRIMENENIYA~--- INFORMATICS AND APPLICATIONS\ \ \ 2016\
\ \ volume~10\ \ \ issue\ 4}
}%
 \def\rightfootline{\small{INFORMATIKA I EE PRIMENENIYA~---
INFORMATICS AND APPLICATIONS\ \ \ 2016\ \ \ volume~10\ \ \ issue\ 4
\hfill \textbf{\thepage}}}

\def\leftkol{2016 AUTHOR INDEX} % ENGLISH ABSTRACTS}

\def\rightkol{2016 AUTHOR INDEX} %ENGLISH ABSTRACTS}


{\tabcolsep=2.83pt
\begin{tabular}{p{382pt}cc}
&\textbf{Issue} & \textbf{Page}\\[6pt]
\Avtors{Kadaner~A.\,I.} see~Chertok~A.\,V.&&\\[.255pt]
\Avtors{Kalinichenko~L.\,A., Volnova~A.\,A., Gordov~E.\,P.,
Kiselyova~N.\,N., Kovaleva~D.\,A., Malkov~O.\,Yu., Okladnikov~I.\,G.,
Podkolodnyy~N.\,L., Pozanenko~A.\,S., Ponomareva~N.\,V.,
Stupnikov~S.\,A.,} \textbf{and Fazliev~A.\,Z.} Data access challenges for data
intensive\linebreak
\\[-12pt]
\hspace*{23pt}research in Russia&1& 2--22\\[.255pt]
\Avtors{Karasikov~M.\,E.\ and Strijov~V.\,V.} Feature-based
time-series classification&4&121--131\\[.255pt]
\Avtors{Khazeeva~G.\,T.} see~Chertok~A.\,V.&&\\[.255pt]
\Avtors{Khokhlov~Yu.\,S.} Multivariate fractional Levy motion and its
applications&2&\hphantom{1}98--106\\[.255pt]
\Avtors{Kirikov~I.\,A., Kolesnikov~A.\,V., Listopad~S.\,V., and
Rumovskaya~S.\,B.} Fine-grained hybrid\linebreak
\\[-12pt]
\hspace*{23pt}intelligent systems. Part 2:
Bidirectional hybridization&1&\hphantom{1}96--105\\[.255pt]
\Avtors{Kirikov~I.\,A., Kolesnikov~A.\,V., Listopad~S.\,V., and
Rumovskaya~S.\,B.} ``Virtual council''~---\linebreak
\\[-12pt]
\hspace*{23pt}source environment
supporting complex diagnostic decision making&3&81--90\\[.255pt]
\Avtors{Kiselyova~N.\,N.} see~Kalinichenko~L.\,A.&&\\[.255pt]
\Avtors{Kolesnikov A.\,V., Listopad~S.\,V., Rumovskaya~S.\,B., and
Danishevsky~V.\,I.} Informal axiomatic\linebreak
\\[-12pt]
\hspace*{23pt}theory of~the~role visual models&4&114--120\\[.255pt]
\Avtors{Kolesnikov~A.\,V.} see~Kirikov~I.\,A.&&\\[.255pt]
\Avtors{Kolesnikov~A.\,V.} see~Kirikov~I.\,A.&&\\[.255pt]
\Avtors{Kolin~K.\,K.} Humanitarian aspects of information
security&3&111--121\\[.255pt]
\Avtors{Konovalov~M.\,G.\ and Razumchik~R.\,V.} Dispatching
to~two parallel nonobservable queues using\linebreak
\\[-12pt]
\hspace*{23pt}only static
information&4&57--67\\[.255pt]
\Avtors{Korchagin~A.\,Yu.} see~Korolev~V.\,Yu.&&\\[.255pt]
\Avtors{Korchagin~A.\,Yu.} see~Korolev~V.\,Yu.&&\\[.255pt]
\Avtors{Korepanov~E.\,R.} see~Sinitsyn~I.\,N.&&\\[.255pt]
\Avtors{Korepanov~E.\,R.} see~Sinitsyn~I.\,N.&&\\[.255pt]
\Avtors{Korolev~V.\,Yu., Korchagin~A.\,Yu., and Zeifman~A.\,I.} The
Poisson theorem for Bernoulli trials\linebreak
\\[-12pt]
\hspace*{23pt}with~a~random probability
of~success and~a~discrete analog of~the~Weibull distribution&4&11--20\\[.255pt]
\Avtors{Korolev~V.\,Yu., Zeifman~A.\,I., and Korchagin~A.\,Yu.}
Asymmetric Linnik distributions as~limit\linebreak
\\[-12pt]
\hspace*{23pt}laws for~random sums
of~independent random variables with~finite variances&4&21--33\\[.255pt]
\Avtors{Koucheryavy~E.\,A.} see~Ometov~A.\,Ya.&&\\[.255pt]
\Avtors{Kovaleva~D.\,A.} see~Kalinichenko~L.\,A.&&\\[.255pt]
\Avtors{Kovalyov~S.\,P.} Metaprogramming to increase
manufacturability of large-scale software-\linebreak
\\[-12pt]
\hspace*{23pt}intensive systems&1&56--66\\[.255pt]
\Avtors{Krivenko~M.\,P.} Significance tests of feature selection for
classification&3&32--40\\[.255pt]
\Avtors{Kruzhkov~M.\,G.} see~Zalizniak~Anna~A.&&\\[.255pt]
\Avtors{Kruzhkov~M.\,G.} see~Zatsman~I.\,M.&&\\[.255pt]
\Avtors{Kudryavtsev~A.\,A.} Bayesian queueing and reliability models:
\textit{A~priori} distributions with\linebreak
\\[-12pt]
\hspace*{23pt}compact support&1&67--71\\[.255pt]
\Avtors{Kudryavtsev~A.\,A.} Characteristics dependent on the balance
coefficient in Bayesian models\linebreak
\\[-12pt]
\hspace*{23pt}with compact support of \textit{a priori}
distributions&3&77--80\\[.255pt]
\Avtors{Kudryavtsev~A.\,A.\ and Palionnaia~S.\,I.} Bayesian recurrent
model of reliability growth:\linebreak
\\[-12pt]
\hspace*{23pt}Parabolic distribution of parameters&2&80--83\\[.255pt]
\Avtors{Kudryavtsev~A.\,A.\ and Titova~A.\,I.} Bayesian queuing
and~reliability models: Degenerate-\linebreak
\\[-12pt]
\hspace*{23pt}Weibull case&4&68--71\\[.255pt]
\Avtors{Leontyev~N.\,D.\ and Ushakov~V.\,G.} Analysis of a queueing
system with autoregressive arrivals\linebreak
\\[-12pt]
\hspace*{23pt}and nonpreemptive priority&3&15--22\\[.255pt]
\Avtors{Listopad~S.\,V.} see~Kirikov~I.\,A.&&\\[.255pt]
\Avtors{Listopad~S.\,V.} see~Kirikov~I.\,A.&&\\[.255pt]
\Avtors{Listopad~S.\,V.} see~Kolesnikov A.\,V.&&\\[.255pt]
\Avtors{Malkov~O.\,Yu.} see~Kalinichenko~L.\,A.&&\\[.255pt]
\Avtors{Markov~A.\,S., Monakhov~M.\,M., and
Ulyanov~V.\,V.} Generalized Cornish--Fisher expansions\linebreak
\\[-12pt]
\hspace*{23pt}for distributions of statistics based on samples
of random size&2&84--91\\[.255pt]
\Avtors{Melnikov~A.\,K.\ and Ronzhin~A.\,F.} Generalized statistical
method of~text analysis based\linebreak
\\[-12pt]
\hspace*{23pt}on~calculation of~probability distributions
of~statistical values&4&89--95\\
\end{tabular}
}
\pagebreak

\def\leftfootline{\small{\textbf{\thepage}
\hfill INFORMATIKA I EE PRIMENENIYA~--- INFORMATICS AND APPLICATIONS\ \ \ 2016\
\ \ volume~10\ \ \ issue\ 4}
}%
 \def\rightfootline{\small{INFORMATIKA I EE PRIMENENIYA~---
INFORMATICS AND APPLICATIONS\ \ \ 2016\ \ \ volume~10\ \ \ issue\ 4
\hfill \textbf{\thepage}}}

\def\leftkol{2016 AUTHOR INDEX} % ENGLISH ABSTRACTS}

\def\rightkol{2016 AUTHOR INDEX} %ENGLISH ABSTRACTS}


{\tabcolsep=3pt
\begin{tabular}{p{381pt}cc}
&\textbf{Issue} & \textbf{Page}\\[6pt]
\Avtors{Meykhanadzhyan~L.\,A.} Stationary characteristics of the finite
capacity queueing system with\linebreak
\\[-12pt]
\hspace*{23pt}inverse service order and generalized
probabilistic priority&2&123--131\\[.23pt]
\Avtors{Miller~G.\,B.} see~Borisov~A.\,V.&&\\[.23pt]
\Avtors{Minin~V.\,A., Zatsman~I.\,M., Havanskov~V.\,A., and
Shubnikov~S.\,K.} Intensity of citation of scientific publications in
inventions on information and computer technologies patented\linebreak
\\[-12pt]
\hspace*{23pt}in Russia by domestic and foreign applicants&2&107--122\\[.23pt]
\Avtors{Monakhov~M.\,M.} see~Markov~A.\,S.&&\\[.23pt]
\Avtors{Naumov~V.\,A.\ and Samouylov~K.\,E.} On relationship
between queuing systems with resources\linebreak
\\[-12pt]
\hspace*{23pt}and Erlang networks&3&\hphantom{1}9--14\\[.23pt]
\Avtors{Okladnikov~I.\,G.} see~Kalinichenko~L.\,A.&&\\[.23pt]
\Avtors{Ometov~A.\,Ya., Andreev~S.\,D., Turlikov~A.\,M., and
Koucheryavy~E.\,A.} Performance analysis of\linebreak
\\[-12pt]
\hspace*{23pt}a wireless data
aggregation system with contention for contemporary sensor
networks&3&23--31\\[.23pt]
\Avtors{Palionnaia~S.\,I.} see~Kudryavtsev~A.\,A.&&\\[.23pt]
\Avtors{Podkolodnyy~N.\,L.} see~Kalinichenko~L.\,A.&&\\[.23pt]
\Avtors{Ponomareva~N.\,V.} see~Kalinichenko~L.\,A.&&\\[.23pt]
\Avtors{Popkova~N.\,A.} see~Zatsman~I.\,M.&&\\[.23pt]
\Avtors{Pozanenko~A.\,S.} see~Kalinichenko~L.\,A.&&\\[.23pt]
\Avtors{Razumchik~R.\,V.} see~Konovalov~M.\,G.&&\\[.23pt]
\Avtors{Ronzhin~A.\,F.} see~Melnikov~A.\,K.&&\\[.23pt]
\Avtors{Rumovskaya~S.\,B.} see~Kirikov~I.\,A.&&\\[.23pt]
\Avtors{Rumovskaya~S.\,B.} see~Kirikov~I.\,A.&&\\[.23pt]
\Avtors{Rumovskaya~S.\,B.} see~Kolesnikov A.\,V.&&\\[.23pt]
\Avtors{Samouylov~K.\,E.} see~Gaidamaka~Yu.\,V.&&\\[.23pt]
\Avtors{Samouylov~K.\,E.} see~Naumov~V.\,A.&&\\[.23pt]
\Avtors{Serebryanskii~S.\,M.} see~Tyrsin~A.\,N.&&\\[.23pt]
\Avtors{Seyful-Mulyukov~R.\,B.} see~Callaos~N.\,K.&&\\[.23pt]
\Avtors{Shestakov~O.\,V.} Statistical properties of the denoising method
based on the stabilized hard\linebreak
\\[-12pt]
\hspace*{23pt}thresholding&2&65--69\\[.23pt]
\Avtors{Shestakov~O.\,V.} The strong law of large numbers for the risk
estimate in the problem of\linebreak
\\[-12pt]
\hspace*{23pt}tomographic image reconstruction from
projections with a correlated noise&3&41--45\\[.23pt]
\Avtors{Shestakov~O.\,V.} see~Zakharova~T.\,V.&&\\[.23pt]
\Avtors{Shnurkov~P.\,V., Gorshenin~A.\,K., and Belousov~V.\,V.}
Analytical solution of~the~optimal control\linebreak
\\[-12pt]
\hspace*{23pt}task of~a~semi-Markov
process with~finite set of~states&4&72--88\\[.23pt]
\Avtors{Shnurkov~P.\,V., Zasypko~V.\,V., Belousov~V.\,V., and
Gorshenin~A.\,K.} Development of the algorithm of numerical solution
of the optimal investment control problem\linebreak
\\[-12pt]
\hspace*{23pt}in the closed dynamical model of three-sector economy&1&82--95\\[.23pt]
\Avtors{Shorgin~S.\,Ya.} see~Gaidamaka~Yu.\,V.&&\\[.23pt]
\Avtors{Shorgin~V.\,S.} see~Agalarov~Ya.\,M.&&\\[.23pt]
\Avtors{Shubnikov~S.\,K.} see~Minin~V.\,A.&&\\[.23pt]
\Avtors{Sidorkin~I.\,I.} see~Arkhipov~O.\,P.&&\\[.23pt]
\Avtors{Sinitsyn~I.\,N.} Analytical modeling of processes in stochastic
systems with complex fractional\linebreak
\\[-12pt]
\hspace*{23pt}order Bessel nonlinearities&3&55--65\\[.23pt]
\Avtors{Sinitsyn~I.\,N.} Orthogonal supoptimal filters for nonlinear
stochastic systems on manifolds&1&34--44\\[.23pt]
\Avtors{Sinitsyn~I.\,N.\ and Korepanov~E.\,R.} Normal Pugachev
conditionally-optimal filters and extra-\linebreak
\\[-12pt]
\hspace*{23pt}polators for state linear stochastic systems&2&14--23\\[.23pt]
\Avtors{Sinitsyn~I.\,N.\ and Sinitsyn~V.\,I.} Analytical modeling of
distributions in stochastic systems on\linebreak
\\[-12pt]
\hspace*{23pt}manifolds based on ellipsoidal approximation&1&45--55\\[.23pt]
\Avtors{Sinitsyn~I.\,N., Sinitsyn~V.\,I., and
Korepanov~E.\,R.} Ellipsoidal suboptimal filters for nonlinear\linebreak
\\[-12pt]
\hspace*{23pt}stochastic systems on manifolds&2&24--35\\[.23pt]
\Avtors{Sinitsyn~V.\,I.} see~Sinitsyn~I.\,N.&&\\[.23pt]
\Avtors{Sinitsyn~V.\,I.} see~Sinitsyn~I.\,N.&&\\[.23pt]
\Avtors{Skvortsov~N.\,A.} see~Stupnikov~S.\,A.&&\\[.23pt]
\Avtors{Sokolov~I.\,A.} see~Chertok~A.\,V.&&\\
\end{tabular}
}
\pagebreak

\def\leftfootline{\small{\textbf{\thepage}
\hfill INFORMATIKA I EE PRIMENENIYA~--- INFORMATICS AND APPLICATIONS\ \ \ 2016\
\ \ volume~10\ \ \ issue\ 4}
}%
 \def\rightfootline{\small{INFORMATIKA I EE PRIMENENIYA~---
INFORMATICS AND APPLICATIONS\ \ \ 2016\ \ \ volume~10\ \ \ issue\ 4
\hfill \textbf{\thepage}}}

\def\leftkol{2016 AUTHOR INDEX} % ENGLISH ABSTRACTS}

\def\rightkol{2016 AUTHOR INDEX} %ENGLISH ABSTRACTS}


{\tabcolsep=3pt
\begin{tabular}{p{382pt}cc}
&\textbf{Issue} & \textbf{Page}\\[6pt]
\Avtors{Sopin~E.\,S.} see~Gaidamaka~Yu.\,V.&&\\
\Avtors{Strijov~V.\,V.} see~Goncharov~A.\,V.&&\\
\Avtors{Strijov~V.\,V.} see~Isachenko~R.\,V.&&\\
\Avtors{Strijov~V.\,V.} see~Karasikov~M.\,E.&&\\
\Avtors{Stupnikov~S.\,A., Briukhov~D.\,O., and Skvortsov~N.\,A.}
Co-lending systemic risk analysis over\linebreak
\\[-12pt]
\hspace*{23pt}heterogeneous data collections&1&23--33\\
\Avtors{Stupnikov~S.\,A.} see~Kalinichenko~L.\,A.&&\\
\Avtors{Suchkov~A.\,P.} see~Zatsarinny~A.\,A.&&\\
\Avtors{Timonina~E.\,E.} see~Grusho~A.\,A.&&\\
\Avtors{Titova~A.\,I.} see~Kudryavtsev~A.\,A.&&\\
\Avtors{Turlikov~A.\,M.} see~Ometov~A.\,Ya.&&\\
\Avtors{Tyrsin~A.\,N.\ and Serebryanskii~S.\,M.} Recognition of
dependences on the basis of inverse\linebreak
\\[-12pt]
\hspace*{23pt}mapping&2&58--64\\
\Avtors{Ulyanov~V.\,V.} see~Markov~A.\,S.&&\\
\Avtors{Ushakov~V.\,G.} Queueing system with working vacations and
hyperexponential input stream&2&92--97\\
\Avtors{Ushakov~V.\,G.} see~Leontyev~N.\,D.&&\\
\Avtors{Volnova~A.\,A.} see~Kalinichenko~L.\,A.&&\\
\Avtors{Yakovlev~O.\,A.\ and Gasilov~A.\,V.} Speeded-up stereo
matching using geodesic support weights&3&\hphantom{1}98--104\\
\Avtors{Zabezhailo~M.\,I.} see~Grusho~A.\,A.&&\\
\Avtors{Zabezhailo~M.\,I.} see~Grusho~A.\,A.&&\\
\Avtors{Zakharova~T.\,V.\ and Shestakov~O.\,V.} Precision analysis of
wavelet processing of aerodynamic\linebreak
\\[-12pt]
\hspace*{23pt}flow patterns&3&46--54\\
\Avtors{Zalizniak~Anna~A.\ and Kruzhkov~M.\,G.} Database
of~Russian impersonal verbal constructions&4&132--141\\
\Avtors{Zasypko~V.\,V.} see~Shnurkov~P.\,V.&&\\
\Avtors{Zatsarinny~A.\,A.\ and Suchkov~A.\,P.} Systems engineering
approaches to~the~establishment of\linebreak
\\[-12pt]
\hspace*{23pt}a~system for~decision support based
on~situational analysis&4&105--113\\
\Avtors{Zatsarinny~A.\,A.} see~Grusho~A.\,A.&&\\
\Avtors{Zatsman~I.\,M., Inkova~O.\,Yu., Kruzhkov~M.\,G., and
Popkova~N.\,A.} Representation of cross-\linebreak
\\[-12pt]
\hspace*{23pt}lingual knowledge about
connectors in supracorpora databases&1&106--118\\
\Avtors{Zatsman~I.\,M.} see~Minin~V.\,A.&&\\
\Avtors{Zeifman~A.\,I.} see~Korolev~V.\,Yu.&&\\
\Avtors{Zeifman~A.\,I.} see~Korolev~V.\,Yu.&&\\
\end{tabular}
}

%\thispagestyle{myheadings}
\def\leftfootline{\small{\textbf{\thepage}
\hfill INFORMATIKA I EE PRIMENENIYA~--- INFORMATICS AND APPLICATIONS\ \ \ 2016\
\ \ volume~10\ \ \ issue\ 4}
}%
 \def\rightfootline{\small{INFORMATIKA I EE PRIMENENIYA~---
INFORMATICS AND APPLICATIONS\ \ \ 2016\ \ \ volume~10\ \ \ issue\ 4
\hfill \textbf{\thepage}}}

 \label{end\stat}

\newpage

\vspace*{-60pt} {%\small 
{%\baselineskip=10.65pt
\section*{Правила подготовки рукописей статей для публикации в журнале
<<Информатика и её применения>>}

\thispagestyle{empty}

 Журнал <<Информатика и её применения>> публикует
теоретические, обзорные и дискуссионные статьи, посвященные научным
исследованиям и разработкам в области информатики и ее приложений. Журнал
издается на русском языке. Тематика журнала охватывает следующие направления:
\begin{itemize}
\item теоретические основы информатики;
\item математические методы исследования сложных систем и процессов;
\item информационные системы и сети;
\item информационные технологии;
\item архитектура и программное
обеспечение вычислительных комплексов и сетей.
\end{itemize}
\begin{enumerate}
\item В журнале печатаются результаты, ранее не
опубликованные и не предназначенные к одновременной публикации в других
изданиях. Публикация не должна нарушать закон об авторских правах. Направляя
свою рукопись в редакцию, авторы автоматически передают учредителям и
редколлегии неисключительные права на издание данной статьи на русском языке и
на ее распространение в России и за рубежом. При этом за авторами сохраняются
все права как собственников данной рукописи. В связи с этим авторами должно
быть представлено в редакцию письмо в следующей форме:
Соглашение о передаче права на публикацию:

\textit{<<Мы, нижеподписавшиеся, авторы рукописи <<$\qquad\qquad$>>, передаем
учредителям и редколлегии журнала <<Информатика и её применения>>
неисключительное право опубликовать данную рукопись статьи на русском языке как
в печатной, так и в электронной версиях журнала. Мы подтверждаем, что данная
публикация не нарушает авторского права других лиц или организаций. Подписи
авторов: (ф.\,и.\,о., дата, адрес)>>.}

Редколлегия вправе запросить у авторов экспертное заключение о возможности
опубликования представленной статьи в открытой печати.
\item Статья
подписывается всеми авторами. На отдельном листе представляются данные автора
(или всех авторов): фамилия, полные имя и отчество, телефон, факс, e-mail,
почтовый адрес. Если работа выполнена несколькими авторами, указывается фамилия
одного из них, ответственного за переписку с редакцией.
\item Редакция журнала
осуществляет самостоятельную экспертизу присланных статей. Возвращение рукописи
на доработку не означает, что статья уже принята к печати. Доработанный вариант
с ответом на замечания рецензента необходимо прислать в редакцию.
\item Решение
редакционной коллегии о принятии статьи к печати или ее отклонении сообщается
авторам. Редколлегия не обязуется направлять рецензию авторам отклоненной
статьи.
\item Корректура статей высылается авторам для просмотра. Редакция
просит авторов присылать свои замечания в кратчайшие сроки.
\item При
подготовке рукописи в MS Word рекомендуется использовать следующие настройки.
Параметры страницы: формат~--- А4; ориентация~--- книжная; поля (см): внутри~---
2,5, снаружи~--- 1,5, сверху~--- 2, снизу~--- 2, от края до нижнего
колонтитула~--- 1,3. Основной текст: стиль~--- <<Обычный>>: шрифт Times New
Roman, размер 14~пунктов, абзацный отступ~--- 0,5~см, 1,5 интервала,
выравнивание~--- по ширине. Рекомендуемый объем рукописи~--- не свыше
25~страниц указанного формата. Ознакомиться с шаблонами, содержащими примеры
оформления, можно по адресу в Интернете:
\textsf{http://www.ipiran.ru/journal/template.doc}.
\item К рукописи, предоставляемой в 2-х
экземплярах, обязательно прилагается электронная версия статьи (как правило, в
форматах MS WORD (.doc) или LaTex (.tex), а также~--- дополнительно~--- в
формате .pdf) на дискете, лазерном диске или по электронной почте. Сокращения
слов, кроме стандартных, не применяются. Все страницы рукописи должны быть
пронумерованы.
\item Статья должна содержать следующую информацию на русском и
английском языках: название, Ф.И.О. авторов, места работы авторов и их
электронные адреса, аннотация (не более 100~слов), ключевые слова. Ссылки на
литературу в тексте статьи нумеруются (в квадратных скобках) и располагаются в
порядке их первого упоминания. Все фамилии авторов, заглавия статей, названия
книг, конференций и~т.\,п.\ даются на языке оригинала, если этот язык
использует кириллический или латинский алфавит.
\item Присланные в редакцию
материалы авторам не возвращаются.
\item При отправке файлов по электронной
почте просим придерживаться следующих правил:
\begin{itemize}
\item указывать в поле subject (тема) название журнала и фамилию автора;
\item использовать attach (присоединение);
\item в случае больших объемов информации возможно
использование общеизвестных архиваторов (ZIP, RAR);
\item в состав электронной версии статьи должны входить: файл, содержащий текст статьи, и файл(ы),
содержащий(е) иллюстрации.
\end{itemize}
\item Журнал <<Информатика и её
применения>> является некоммерческим изданием, и гонорар авторам не
выплачивается.
\end{enumerate}
\thispagestyle{empty}

\medskip
\noindent
\textbf{Адрес редакции:} Москва 119333,
ул.~Вавилова, д.~44, корп.~2, ИПИ РАН\\
\hphantom{\textbf{Адрес редакции:} }Тел.: +7 (499) 135-86-92\ \
Факс:  +7 (495) 930-45-05\ \  E-mail:   rust@ipiran.ru }

\vfill
\begin{center}

Выпускающий редактор Т. Торжкова\\
Технический редактор Л. Кокушкина\\
Художественный редактор М. Седакова\\
Сдано в набор 15.01.10. Подписано в печать 03.03.10. Формат 60 х 84 / 8\\
Бумага офсетная. Печать цифровая. Усл.-печ. л. 12. Уч.-изд. л. 10,4. Тираж 200 экз.\\
\ \\
Заказ № \\
\ \\
Издательство <<ТОРУС ПРЕСС>>, Москва 121614, ул. Крылатская, 29-1-43\\
torus@torus-press.ru; http://www.torus-press.ru\\
\ \\
Отпечатано в ППП <<Типография <<Наука>> с готовых файлов\\
Москва 121099, Шубинский пер., д. 6.\\
\end{center}

\end{document}

%\def\stat{cont}
{%\hrule\par
%\vskip 7pt % 7pt
\raggedleft\Large \bf%\baselineskip=3.2ex
А\,В\,Т\,О\,Р\,С\,К\,И\,Й\ \ У\,К\,А\,З\,А\,Т\,Е\,Л\,Ь\ \ З\,А\ \ 2\,0\,0\,7 г. \vskip 17pt
    \hrule
    \par
\vskip 21pt plus 6pt minus 3pt }

\label{st\stat}

\def\tit{\ }

\def\aut{\ }
\def\auf{\ }

\def\leftkol{\ } % ENGLISH ABSTRACTS}

\def\rightkol{\ } %ENGLISH ABSTRACTS}

\titele{\tit}{\aut}{\auf}{\leftkol}{\rightkol}


\contentsline {chapter}{\ }{Выпуск \quad Стр.} 
\contentsline {section}{\textbf{Батракова Д.\,А., Королев В.\,Ю., Шоргин С.\,Я.}\ \ Новый метод вероятностно-ста\-ти\-сти\-че\-ско\-го анализа информационных потоков в\nobreakspace {}телекоммуникационных сетях}{\qquad 1 \qquad 40} 
\contentsline {section}{\textbf{Борисов А.\,В.}\ \ Байесовское оценивание в системах наблюдения с\nobreakspace {}марковскими скачкообразными процессами: игровой подход}{\qquad 2 \qquad 65}
\contentsline {section}{\textbf{Босов А.\,В., Иванов А.\,В.}\ \ Программная инфраструктура информационного Web-пор\-тала}{\qquad 2 \qquad 50}
\contentsline {section}{\textbf{Захаров В.\,Н., Калиниченко Л.\,А., Соколов И.\,А., Ступников С.\,А.}\ \ Конструирование канонических информационных моделей для интегрированных информационных систем}{\qquad 2 \qquad 15}
\contentsline {section}{\textbf{Захаров В.\,Н., Козмидиади В.\,А.}\ \ Средства обеспечения отказоустойчивости при\-ло\-жений}{\qquad 1 \qquad 14} 
\contentsline {section}{\textbf{Иванов А.\,В.}\ \ см. Босов А.\,В.\hfill\hfill\hfill\hfill\hfill\hfill\hfill\hfill\hfill\hfill\hfill\hfill\hfill\hfill\hfill\hfill\hfill\hfill\hfill\hfill\hfill\hfill\hfill\hfill\hfill\hfill\hfill\hfill\hfill\hfill\hfill\hfill\hfill\hfill\hfill}{\ }
\contentsline {section}{\textbf{Ильин В.\,Д., Соколов И.\,А.}\ \ Символьная модель системы знаний информатики в\nobreakspace {}че\-ло\-ве\-ко-автоматной среде}{\qquad 1 \qquad 66} 
\contentsline {section}{\textbf{Калиниченко Л.\,А.}\ \ см. Захаров В.\,Н.\hfill\hfill\hfill\hfill\hfill\hfill\hfill\hfill\hfill\hfill\hfill\hfill\hfill\hfill\hfill\hfill\hfill\hfill\hfill\hfill\hfill\hfill\hfill\hfill\hfill\hfill\hfill\hfill\hfill\hfill\hfill\hfill\hfill\hfill\hfill}{\ }
\contentsline {section}{\textbf{Козеренко Е.\,Б.}\ \ Лингвистическое моделирование для систем машинного перевода и обработки знаний}{\qquad 1 \qquad 54} 
\contentsline {section}{\textbf{Козмидиади В.\,А.}\ \ см. Захаров В.\,Н.\hfill\hfill\hfill\hfill\hfill\hfill\hfill\hfill\hfill\hfill\hfill\hfill\hfill\hfill\hfill\hfill\hfill\hfill\hfill\hfill\hfill\hfill\hfill\hfill\hfill\hfill\hfill\hfill\hfill\hfill\hfill\hfill\hfill\hfill\hfill }{\ } 
\contentsline {section}{\textbf{Королев В.\,Ю.}\ \ см. Батракова Д.\,А.\hfill\hfill\hfill\hfill\hfill\hfill\hfill\hfill\hfill\hfill\hfill\hfill\hfill\hfill\hfill\hfill\hfill\hfill\hfill\hfill\hfill\hfill\hfill\hfill\hfill\hfill\hfill\hfill\hfill\hfill\hfill\hfill\hfill\hfill\hfill}{\ } 
\contentsline {section}{\textbf{Кудрявцев А.\,А., Шоргин С.\,Я.}\ \ Байесовский подход к\nobreakspace {}анализу систем массового обслуживания и\nobreakspace {}показателей надежности}{\qquad 2 \qquad 76}
\contentsline {section}{\textbf{Печинкин А.\,В., Соколов И.\,А., Чаплыгин В.\,В.}\ \ Многолинейная система массового обслуживания с конечным накопителем и ненадежными приборами}{\qquad 1 \qquad 27} 
\contentsline {section}{\textbf{Печинкин А.\,В., Соколов И.\,А., Чаплыгин В.\,В.}\ \ Стационарные характеристики многолинейной\nobreakspace {}системы массового обслуживания с\nobreakspace {}одновременными отказами приборов}{\qquad 2 \qquad 39}
\contentsline {section}{\textbf{Синицын И.\,Н.}\ \ Корреляционные методы построения аналитических информационных моделей флуктуаций полюса Земли по априорным данным}{\qquad 2 \qquad \hphantom{9}2}
\contentsline {section}{\textbf{Синицын И.\,Н.}\ \ Развитие теории фильтров Пугачева для оперативной обработки информации в стохастических системах}{{\qquad 1 \qquad \hphantom{9}3}} 
\contentsline {section}{\textbf{Соколов И.\,А.}\ \ см. Захаров В.\,Н.\hfill\hfill\hfill\hfill\hfill\hfill\hfill\hfill\hfill\hfill\hfill\hfill\hfill\hfill\hfill\hfill\hfill\hfill\hfill\hfill\hfill\hfill\hfill\hfill\hfill\hfill\hfill\hfill\hfill\hfill\hfill\hfill\hfill\hfill\hfill}{\ }
\contentsline {section}{\textbf{Соколов И.\,А.}\ \ см. Ильин В.\,Д.\hfill\hfill\hfill\hfill\hfill\hfill\hfill\hfill\hfill\hfill\hfill\hfill\hfill\hfill\hfill\hfill\hfill\hfill\hfill\hfill\hfill\hfill\hfill\hfill\hfill\hfill\hfill\hfill\hfill\hfill\hfill\hfill\hfill\hfill\hfill}{\ } 
\contentsline {section}{\textbf{Соколов И.\,А.}\ \ см. Печинкин А.\,В.\hfill\hfill\hfill\hfill\hfill\hfill\hfill\hfill\hfill\hfill\hfill\hfill\hfill\hfill\hfill\hfill\hfill\hfill\hfill\hfill\hfill\hfill\hfill\hfill\hfill\hfill\hfill\hfill\hfill\hfill\hfill\hfill\hfill\hfill\hfill}{\ } 
\contentsline {section}{\textbf{Соколов И.\,А.}\ \ см. Печинкин А.\,В.\hfill\hfill\hfill\hfill\hfill\hfill\hfill\hfill\hfill\hfill\hfill\hfill\hfill\hfill\hfill\hfill\hfill\hfill\hfill\hfill\hfill\hfill\hfill\hfill\hfill\hfill\hfill\hfill\hfill\hfill\hfill\hfill\hfill\hfill\hfill}{\ }
\contentsline {section}{\textbf{Ступников С.\,А.}\ \ см. Захаров В.\,Н.\hfill\hfill\hfill\hfill\hfill\hfill\hfill\hfill\hfill\hfill\hfill\hfill\hfill\hfill\hfill\hfill\hfill\hfill\hfill\hfill\hfill\hfill\hfill\hfill\hfill\hfill\hfill\hfill\hfill\hfill\hfill\hfill\hfill\hfill\hfill}{\ }
\contentsline {section}{\textbf{Чаплыгин В.\,В.}\ \ см. Печинкин А.\,В.\hfill\hfill\hfill\hfill\hfill\hfill\hfill\hfill\hfill\hfill\hfill\hfill\hfill\hfill\hfill\hfill\hfill\hfill\hfill\hfill\hfill\hfill\hfill\hfill\hfill\hfill\hfill\hfill\hfill\hfill\hfill\hfill\hfill\hfill\hfill}{\ } 
\contentsline {section}{\textbf{Чаплыгин В.\,В.}\ \ см. Печинкин А.\,В.\hfill\hfill\hfill\hfill\hfill\hfill\hfill\hfill\hfill\hfill\hfill\hfill\hfill\hfill\hfill\hfill\hfill\hfill\hfill\hfill\hfill\hfill\hfill\hfill\hfill\hfill\hfill\hfill\hfill\hfill\hfill\hfill\hfill\hfill\hfill}{\ }
\contentsline {section}{\textbf{Шоргин С.\,Я.}\ \ см. Батракова Д.\,А.\hfill\hfill\hfill\hfill\hfill\hfill\hfill\hfill\hfill\hfill\hfill\hfill\hfill\hfill\hfill\hfill\hfill\hfill\hfill\hfill\hfill\hfill\hfill\hfill\hfill\hfill\hfill\hfill\hfill\hfill\hfill\hfill\hfill\hfill\hfill}{\ } 
\contentsline {section}{\textbf{Шоргин С.\,Я.}\ \ см. Кудрявцев А.\,А.\hfill\hfill\hfill\hfill\hfill\hfill\hfill\hfill\hfill\hfill\hfill\hfill\hfill\hfill\hfill\hfill\hfill\hfill\hfill\hfill\hfill\hfill\hfill\hfill\hfill\hfill\hfill\hfill\hfill\hfill\hfill\hfill\hfill\hfill\hfill}{\ }
%\thispagestyle{myheadings}
\def\leftfootline{\small{\textbf{\thepage}
\hfill ИНФОРМАТИКА И ЕЁ ПРИМЕНЕНИЯ\ \ \ том~1\ \ \ выпуск~2\ \ \ 2007}
}%
 \def\rightfootline{\small{ИНФОРМАТИКА И ЕЁ ПРИМЕНЕНИЯ\ \ \ том~1\ \ \ выпуск~2\ \ \ 2007
 \hfill \textbf{\thepage}}}
 \label{end\stat}

%\def\stat{cont-e}
{%\hrule\par
%\vskip 7pt % 7pt
\raggedleft\Large \bf%\baselineskip=3.2ex
2\,0\,0\,7\ \ A\,U\,T\,H\,O\,R\ \ I\,N\,D\,E\,X \vskip 17pt
    \hrule
    \par
\vskip 21pt plus 6pt minus 3pt }

\label{st\stat}

\def\tit{\ }

\def\aut{\ }
\def\auf{\ }

\def\leftkol{\ } % ENGLISH ABSTRACTS}

\def\rightkol{\ } %ENGLISH ABSTRACTS}

\titele{\tit}{\aut}{\auf}{\leftkol}{\rightkol}


\contentsline {chapter}{\ }{Issue \quad Page} 
\contentsline {subsection}{\textbf{Batrakova D.\,A., Korolev V.\,Yu., Shorgin S.\,Ya.}\ \ A New Method for the Probabilistic and Statistical Analysis of Information Flows in Telecommunication Networks}{\qquad 1 \qquad 40} 
\contentsline {subsection}{\textbf{Borisov A.\,V.}\ \ Bayesian Estimation in\nobreakspace {}Observation Systems with\nobreakspace {}Markov Jump Processes: Game-Theoretic Approach}{\qquad 2 \qquad 65} 
\contentsline {subsection}{\textbf{Bosov A.\,V., Ivanov A.\,V.}\ \ Linguistic Simulation for Machine Translation and Knowledge Management Systems}{\qquad 2 \qquad 50} 
\contentsline {subsection}{\textbf{Chaplygin V.\,V.} see Pechinkin A.\,V.\hfill\hfill\hfill\hfill\hfill\hfill\hfill\hfill\hfill\hfill\hfill\hfill\hfill\hfill\hfill\hfill\hfill\hfill\hfill\hfill\hfill\hfill\hfill\hfill\hfill\hfill\hfill\hfill\hfill\hfill\hfill\hfill\hfill\hfill\hfill}{\ }
\contentsline {subsection}{\textbf{Chaplygin V.\,V.} see Pechinkin A.\,V.\hfill\hfill\hfill\hfill\hfill\hfill\hfill\hfill\hfill\hfill\hfill\hfill\hfill\hfill\hfill\hfill\hfill\hfill\hfill\hfill\hfill\hfill\hfill\hfill\hfill\hfill\hfill\hfill\hfill\hfill\hfill\hfill\hfill\hfill\hfill}{\ }
\contentsline {subsection}{\textbf{Ilyin V.\,D., Sokolov I.\,A.}\ \ The Symbol Model of Informatics Knowledge System in Human-Automaton Environment}{\qquad 1 \qquad 66} 
\contentsline {subsection}{\textbf{Ivanov A.\,V.} see Bosov A.\,V.\hfill\hfill\hfill\hfill\hfill\hfill\hfill\hfill\hfill\hfill\hfill\hfill\hfill\hfill\hfill\hfill\hfill\hfill\hfill\hfill\hfill\hfill\hfill\hfill\hfill\hfill\hfill\hfill\hfill\hfill\hfill\hfill\hfill\hfill\hfill}{\ }
\contentsline {subsection}{\textbf{Kalinichenko L.\,A.} see Zakharov V.\,N.\hfill\hfill\hfill\hfill\hfill\hfill\hfill\hfill\hfill\hfill\hfill\hfill\hfill\hfill\hfill\hfill\hfill\hfill\hfill\hfill\hfill\hfill\hfill\hfill\hfill\hfill\hfill\hfill\hfill\hfill\hfill\hfill\hfill\hfill\hfill}{\ }
\contentsline {subsection}{\textbf{Korolev V.\,Yu.} see Batrakova D.\,A.\hfill\hfill\hfill\hfill\hfill\hfill\hfill\hfill\hfill\hfill\hfill\hfill\hfill\hfill\hfill\hfill\hfill\hfill\hfill\hfill\hfill\hfill\hfill\hfill\hfill\hfill\hfill\hfill\hfill\hfill\hfill\hfill\hfill\hfill\hfill}{\ }
\contentsline {subsection}{\textbf{Kozerenko E.\,B.}\ \ Linguistic Simulation for Machine Translation and Knowledge Management Systems}{\qquad 1 \qquad 54} 
\contentsline {subsection}{\textbf{Kozmidiady V.\,A.} see Zakharov V.\,N.\hfill\hfill\hfill\hfill\hfill\hfill\hfill\hfill\hfill\hfill\hfill\hfill\hfill\hfill\hfill\hfill\hfill\hfill\hfill\hfill\hfill\hfill\hfill\hfill\hfill\hfill\hfill\hfill\hfill\hfill\hfill\hfill\hfill\hfill\hfill}{\ }
\contentsline {subsection}{\textbf{Kudryavtsev A.\,A., Shorgin S.\,Ya.}\ \ Bayesian Approach to Queueing Systems and Reliability Characteristics}{\qquad 2 \qquad 76} 
\contentsline {subsection}{\textbf{Pechinkin A.\,V., Sokolov I.\,A., Chaplygin V.\,V.}\ \ Multichannel Queuing System with Finite Buffer and Unreliable Servers}{\qquad 1 \qquad 27} 
\contentsline {subsection}{\textbf{Pechinkin A.\,V., Sokolov I.\,A., Chaplygin V.\,V.}\ \ Stationary Characteristics of a Multichannel Queueing System with\nobreakspace {}Simultaneous Refusals of Servers}{\qquad 2 \qquad 39} 
\contentsline {subsection}{\textbf{Shorgin S.\,Ya.} see Batrakova D.\,A.\hfill\hfill\hfill\hfill\hfill\hfill\hfill\hfill\hfill\hfill\hfill\hfill\hfill\hfill\hfill\hfill\hfill\hfill\hfill\hfill\hfill\hfill\hfill\hfill\hfill\hfill\hfill\hfill\hfill\hfill\hfill\hfill\hfill\hfill\hfill}{\ }
\contentsline {subsection}{\textbf{Shorgin S.\,Ya.} see Kudryavtsev A.\,A.\hfill\hfill\hfill\hfill\hfill\hfill\hfill\hfill\hfill\hfill\hfill\hfill\hfill\hfill\hfill\hfill\hfill\hfill\hfill\hfill\hfill\hfill\hfill\hfill\hfill\hfill\hfill\hfill\hfill\hfill\hfill\hfill\hfill\hfill\hfill}{\ }
\contentsline {subsection}{\textbf{Sinitsyn I.\,N.}\ \ Correlational Methods for Analytical Informational Models of the Earth Pole Fluctuations Design Based on a priori Data}{\qquad 2 \qquad \hphantom{9}2}
\contentsline {subsection}{\textbf{Sinitsyn I.\,N.}\ \ Development of Pugachev Filtering for Stochastic Systems}{\qquad 1 \qquad \hphantom{9}3}
\contentsline {subsection}{\textbf{Sokolov I.\,A.} see Ilyin V.\,D.\hfill\hfill\hfill\hfill\hfill\hfill\hfill\hfill\hfill\hfill\hfill\hfill\hfill\hfill\hfill\hfill\hfill\hfill\hfill\hfill\hfill\hfill\hfill\hfill\hfill\hfill\hfill\hfill\hfill\hfill\hfill\hfill\hfill\hfill\hfill}{\ }
\contentsline {subsection}{\textbf{Sokolov I.\,A.} see Pechinkin A.\,V.\hfill\hfill\hfill\hfill\hfill\hfill\hfill\hfill\hfill\hfill\hfill\hfill\hfill\hfill\hfill\hfill\hfill\hfill\hfill\hfill\hfill\hfill\hfill\hfill\hfill\hfill\hfill\hfill\hfill\hfill\hfill\hfill\hfill\hfill\hfill}{\ }
\contentsline {subsection}{\textbf{Sokolov I.\,A.} see Pechinkin A.\,V.\hfill\hfill\hfill\hfill\hfill\hfill\hfill\hfill\hfill\hfill\hfill\hfill\hfill\hfill\hfill\hfill\hfill\hfill\hfill\hfill\hfill\hfill\hfill\hfill\hfill\hfill\hfill\hfill\hfill\hfill\hfill\hfill\hfill\hfill\hfill}{\ }
\contentsline {subsection}{\textbf{Sokolov I.\,A.} see Zakharov V.\,N.\hfill\hfill\hfill\hfill\hfill\hfill\hfill\hfill\hfill\hfill\hfill\hfill\hfill\hfill\hfill\hfill\hfill\hfill\hfill\hfill\hfill\hfill\hfill\hfill\hfill\hfill\hfill\hfill\hfill\hfill\hfill\hfill\hfill\hfill\hfill}{\ }
\contentsline {subsection}{\textbf{Stupnikov S.\,A.} see Zakharov V.\,N.\hfill\hfill\hfill\hfill\hfill\hfill\hfill\hfill\hfill\hfill\hfill\hfill\hfill\hfill\hfill\hfill\hfill\hfill\hfill\hfill\hfill\hfill\hfill\hfill\hfill\hfill\hfill\hfill\hfill\hfill\hfill\hfill\hfill\hfill\hfill}{\ }
\contentsline {subsection}{\textbf{Zakharov V.\,N., Kalinichenko L.\,A., Sokolov I.\,A., Stupnikov S.\,A.}\ \ Development of Canonical Information Models for Integrated Information Systems}{\qquad 2 \qquad 15} 
\contentsline {subsection}{\textbf{Zakharov V.\,N., Kozmidiady V.\,A.}\ \ Means Providing Applications Fault Tolerance}{\qquad 1 \qquad 14} 
\def\leftfootline{\small{\textbf{\thepage}
\hfill ИНФОРМАТИКА И ЕЁ ПРИМЕНЕНИЯ\ \ \ том~1\ \ \ выпуск~2\ \ \ 2007}
}%
 \def\rightfootline{\small{ИНФОРМАТИКА И ЕЁ ПРИМЕНЕНИЯ\ \ \ том~1\ \ \ выпуск~2\ \ \ 2007
 \hfill \textbf{\thepage}}}
 \label{end\stat}


%\tableofcontents


\end{document}