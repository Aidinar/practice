\def\stat{bening}


\def\tit{АСИМПТОТИЧЕСКОЕ
РАЗЛОЖЕНИЕ ДЛЯ МОЩНОСТИ КРИТЕРИЯ, ОСНОВАННОГО НА ВЫБОРОЧНОЙ
МЕДИАНЕ, В~СЛУЧАЕ РАСПРЕДЕЛЕНИЯ ЛАПЛАСА$^*$}
\def\titkol{Асимптотическое
разложение для мощности критерия, основанного на выборочной
медиане} %, в случае распределения Лапласа}

\def\autkol{В.\,Е.~Бенинг, А.\,В.~Сипина}
\def\aut{В.\,Е.~Бенинг$^1$, А.\,В.~Сипина$^2$}

\titel{\tit}{\aut}{\autkol}{\titkol}

{\renewcommand{\thefootnote}{\fnsymbol{footnote}}\footnotetext[1]
{Работа выполнена
при финансовой поддержке РФФИ, проекты 08-01-00567 и
08-07-00152.}}

\renewcommand{\thefootnote}{\arabic{footnote}}
\footnotetext[1]{Московский государственный университет им.\
М.\,В.~Ломоносова, факультет вычислительной математики и
кибернетики; Институт проблем информатики Российской академии наук, bening@yandex.ru}
\footnotetext[2]{Московский государственный университет им.\
М.\,В.~Ломоносова, факультет вычислительной математики и
кибернетики, anna@sipin.ru}



\Abst{В работе прямыми методами, использующими асимптотические разложения,
получена формула для предельного отклонения мощности критерия, 
основанного на выборочной медиане, от мощности наилучшего критерия в случае распределения Лапласа.}

\KW{выборочная медиана; асимптотичсекое разложение; функция мощности; распределение Лапласа}

      \vskip 18pt plus 9pt minus 6pt

      \thispagestyle{headings}

      \begin{multicols}{2}

      \label{st\stat}


\section{Введение}

Следуя работе~\cite{3ben}, рассмотрим задачу проверки гипотезы
\begin{equation*}
{\sf H}_0: \theta = 0     
%\label{e1.1b}
\end{equation*}
против последовательности сложных близких альтернатив вида
\begin{equation*}
{\sf H}_{n,1}: \theta = \fr{t}{\sqrt{n}}\,,\quad 0<t<C\,,\quad
 C > 0
% \label{e1.2b}
\end{equation*}
на основе выборки $(X_1, \ldots , X_n)$~--- независимых одинаково распределенных наблюдений, имеющих распределение Лапласа 
с плотностью
\begin{equation}
p(x, \theta) = \fr{1}{2}e^{-|x-\theta|}\,, \quad x,\:
\theta \in{\sf R}^1\,. 
\label{e1.3b}
\end{equation}
Распределение Лапласа широко применяется в прикладной статистике, например
в задачах вы\-де\-ле\-ния полезного сигнала на фоне помех~[2--4].
Естественность возникновения этого распределения обоснована в
работе~\cite{6ben}.

Для каждого фиксированного $t\in (0,C]$
обозначим через~$\beta_n^*(t)$ мощность наилучшего критерия размера
$\alpha\in (0,1)$. По лемме Неймана--Пирсона %\linebreak 
[6, с.~94]
такой критерий всегда существует и  основан на логарифме отношения правдоподобия
\begin{equation}
\Lambda_n(t) = 
\sum_{i=1}^{n}\left( \left|X_i\right|-\left|X_i-tn^{-1/2}\right|\right)\,.
 \label{e1.4b}
\end{equation}
В работах~\cite{3ben, 2ben} рассмотрен критерий, основанный на знаковой статистике,
и получена формула для предельного отклонения мощности данного
критерия от мощности наилучшего критерия, основанного на~$\Lambda_n(t)$.
Поскольку у плотности~$p(x,\theta)$ не существует производной по~$\theta$ в 
точке $\theta = 0$, то это семейство не является регулярным.
Это выражается в нарушении естественного порядка~$n^{-1}$ разности мощностей
этих критериев и приводит к порядку~$n^{-1/2}$.

В  работе рассматривается статистика
\begin{equation*}
T_n = \sqrt{2k}\, \zeta_n\,,\quad k=\left[\fr{n}{2}\right]\,, 
%\label{e1.5b}
\end{equation*}
где $\zeta_n$~--- выборочная медиана:
\begin{equation*}
\zeta_n= 
\begin{cases}
X_{(k+1)}\,, & n=2k+1\,; \\
\fr{X_{(k)}+X_{(k+1)}}{2}\,, &  n=2k\,.
\end{cases}
%\label{e1.6b}
\end{equation*}
Заметим, что в случае распределения Лапласа выборочная медиана
совпадает с оценкой максимального правдоподобия (см.~\cite{1ben}).

Обозначим через~$\beta_n(t)$ мощность критерия размера $\alpha\in (0,1)$,
основанного на статистике~$T_n$. В работе получено асимптотическое
разложение для~$\beta_n(t)$ и вычислен предел разности мощностей~$\beta_n^*(t)$ и~$\beta_n(t)$
$$
r(t)\equiv\lim_{n\to\infty}\sqrt n\left(\beta_n^*(t)-\beta_n(t)\right)
$$
критериев (см.~(\ref{e2.14b})),
основанных соответственно на статистиках~$\Lambda_n$ и~$T_n$.

В работе также приведено полное доказательство  (см.~\cite{5ben})
представления выборочной медианы в виде случайной суммы
независимых экспоненциально распределенных  случайных величин.


\section{Асимптотическое разложение для мощности критерия,
основанного на выборочной медиане}

В этом разделе будет построено  асимптотическое разложение  для мощности~$\beta_n(t)$.
Основой для его получения служит  работа~\cite{1ben} (см.\ теорему~2.1),
в которой получено разложение для функции распределения выборочной медианы.
Члены порядка~$n^{-1/2}$ в разложении для функции распределения выборочной медианы
без доказательства приведены  также в работе~\cite{9ben}.

\medskip
\noindent
\textbf{Теорема 1.} {\it Для мощности~$\beta_n(t)$ равномерно по
$t\in(0,C]$, $C>0$,
справедливо следующее асимптотическое разложение:
\begin{equation*}
\beta_n(t)=
\begin{cases}
\Phi(t-u_\alpha)-\fr{t(2u_\alpha-t)}{2\sqrt{n}}\,\varphi(u_\alpha-t)+{} \\
\hspace*{8mm}{}+o\left(n^{-1/2}\right)\,,  \quad t \le u_\alpha\,,\enskip  \alpha <\fr{1}{2}\,;\\
\Phi(t-u_\alpha)-\fr{2u_\alpha^2+t^2-2u_\alpha t}{2\sqrt{n}}\,\varphi(u_\alpha -t)+{}\\
\hspace*{8mm}{}+o\left(n^{-1/2}\right)\,, \quad t>u_\alpha\,, \enskip \alpha <\fr{1}{2}\,;\\
\Phi(t-u_\alpha)+\fr{t(2u_\alpha-t)}{2\sqrt{n}}\,\varphi(u_\alpha -t)+{}\\
\hspace*{22mm}{}+{} o\left(n^{-1/2}\right)\,, \quad 
\alpha \ge \fr{1}{2}\,,
\end{cases}\hspace*{-6pt}
%\label{e2.1b}
\end{equation*}
где  $\Phi(x)$  и~ $\varphi(x)$~---  функция распределения и
плотность стандартного нормального закона и $\Phi(u_\alpha)=1-\alpha$.}

\medskip

\noindent
Д\,о\,к\,а\,з\,а\,т\,е\,л\,ь\,с\,т\,в\,о\,.\
Для доказательства теоремы воспользуемся асимптотическим разложением
для функции распределения выборочной медианы в случае
распределения Лапласа из работы~\cite{1ben} (см.\ формулу~(1.3)):
\begin{multline}
\p_{n,\theta} \left( \sqrt{2k}(\zeta_n - \theta) < x \right) = 
\Phi(x)-\fr{x|x|}{2\sqrt{2k}}\,\varphi(x)+{}\\
{}+
\fr{x(18+10x^2-3x^4)}{48k}\,\varphi(x)+ o(n^{-1})\,.
\label{e2.2b}
\end{multline}
Подберем критическое значение~$d_n$, исходя из условия
\begin{equation*}
\p_{n,0}(T_n>d_n)=\alpha+ o(n^{-1})\,.
%\label{e2.3b}
\end{equation*}
Будем искать $d_n$ в виде
\begin{equation*}
d_n = u_\alpha +\fr{a}{\sqrt{2k}}+\fr{b}{2k}\,.
%\label{e2.4b}
\end{equation*}
Из формулы~(\ref{e2.2b}) следует, что

\noindent
\begin{multline}
\p_{n, 0} \left( T_n> d_n \right) = 1 -
\Phi(d_n)+\fr{d_n|d_n|}{2\sqrt{2k}}\varphi(d_n)-{}\\
{}-
\fr{d_n(18+10d_n^2-3d_n^4)}{48k}\,\varphi(d_n)+ o(n^{-1})\,.
\label{e2.5b}
\end{multline}
Чтобы раскрыть модуль в выражении~(\ref{e2.5b}),  рас\-смот\-рим два случая:
$\alpha<1/2$ и $\alpha \ge 1/2$.

Рассмотрим случай $\alpha < 1/2$. Это означает, что при достаточно
больших $n$ справедливо неравенство $d_n > 0$.
Подставляя выражение для~$d_n$ в формулу~(\ref{e2.5b}) и применяя следующие разложения:
\begin{multline*}
\Phi(d_n)=\Phi\left(u_\alpha+\fr{a}{\sqrt{2k}}+\fr{b}{2k}\right)=
\Phi(u_\alpha)+{}\\
{}+
\left(\fr{a}{\sqrt{2k}}+\fr{b}{2k}\right)\varphi(u_\alpha)-
\fr{u_\alpha a^2}{4k}\varphi(u_\alpha)+ o(n^{-1})\,;
\end{multline*}
\vspace*{-12pt}

\noindent
\begin{multline*}
\varphi(d_n)=\varphi\left(u_\alpha+\fr{a}{\sqrt{2k}}
+\fr{b}{2k}\right)= {}\\
{}=
\varphi(u_\alpha)-\left(\fr{a}{\sqrt{2k}}+\fr{b}{2k}\right)u_\alpha
\varphi(u_\alpha)+ o(n^{-1})\,,
\end{multline*}
получаем
\begin{multline*}
1-\Phi(u_\alpha)-\left(\fr{a}{\sqrt{2k}}+
\fr{b}{2k}\right)\varphi(u_\alpha)+\fr{u_\alpha a^2}{4k}\,\phi(u_\alpha)
+{}\\
{}+\fr{(u_\alpha+(a/\sqrt{2k})+b/(2k))^2}
{2\sqrt{2k}}\times{}\\
{}\times \left(\varphi(u_\alpha) - \fr{a}{\sqrt{2k}}\,u_\alpha
\varphi(u_\alpha)\right)-{}\\
{}-
\fr{u_\alpha(18+10u_\alpha^2-3u_\alpha^4)}{48k}\,\varphi(u_\alpha)=
\alpha + o(n^{-1})\,.
\end{multline*}
Приравнивая коэффициенты при~$1/\sqrt{2k}$ и~$1/(2k)$ к нулю,
находим выражения для~$a$ и~$b$:
\begin{gather*}
a=\fr{u_\alpha^2}{2}\,;
\\
b=-\fr{3}{4}\,u_\alpha+\fr{1}{12}\,u_\alpha^3\,;
\\
d_n = u_\alpha+\fr{u_\alpha^2}{2\sqrt{2k}}-\fr{3}{8k}\,
u_\alpha+\fr{1}{24k}\,u_\alpha^3\,.
\end{gather*}
Теперь для получения асимптотического разложения мощности критерия используем
разложение
\begin{multline*}
\p_{n,tn^{-1/2}}(T_n<x)= \Phi\left(x-t\sqrt{2k}n^{-1/2}\right) -{}\\
{}-
\fr{\left(x-t\sqrt{2k}n^{-1/2}\right)\left| x\:-\:t\sqrt{2k}\,n^{-1/2}\right|}{2\sqrt{2k}}\,
{}\times{}\\
{}\times\varphi(x-t\sqrt{2k}\,n^{-1/2})+ {}
\end{multline*}
\begin{multline*}
{}+
\fr{ x-t\sqrt{2k}\,n^{-1/2}}{48k}
\left(18+10(x-
t\sqrt{2k}\,n^{-1/2})^2-{}\right.\\
\left.{}-3(x-t\sqrt{2k}\,n^{-1/2})^4\right)\times{}
\\
{}\times\varphi\left(x-t\sqrt{2k}\,n^{-1/2}\right)+ o\left(n^{-1}\right)\,,
%\label{e2.6b}
\end{multline*}
которое  получается при подстановке $\theta=tn^{-1/2}$ в
формулу~(\ref{e2.2b}).

Имеем
\begin{multline*}
\beta_n(t)=\p_{n,tn^{-1/2}}\left(T_n>d_n\right) ={}\\
{}=
1-\Phi\left(d_n-t\right) +
\fr{\left(d_n-t\right)\left|d_n-t\right|}{2\sqrt{2k}}\,\varphi\left(d_n-t\right)-{}
\\\!
{}-\fr{d_n-t}{48k}\left(18+10\left(d_n-t\right)^2
-3(d_n-t)^4\right)\, \varphi\left(d_n-t\right)+{}\\
{}+ o\left(n^{-1}\right)\,.
%\label{e2.7b}
\end{multline*}
Аналогично предыдущему, рассмотрим  два случая: $t\le u_\alpha$ и
$t>u_\alpha$.

Пусть сначала $t \le u_\alpha$.
Используя разложения
\begin{multline*}
\Phi\left(d_n-t\right)={}\\
{}=\Phi\left(u_\alpha-t+
\fr{u_\alpha^2}{2\sqrt{2k}}-\fr{3}{8k}\,u_\alpha+
\fr{1}{24k}\,u_\alpha^3\right)={}\\
{}=\Phi\left(u_\alpha-t\right)+
\left(\fr{u_\alpha^2}{2\sqrt{2k}}-\fr{3}{8k}\,u_\alpha+
\fr{1}{24k}\,u_\alpha^3\right)\times{}\\
{}\times\varphi\left(u_\alpha-t\sqrt{2k}\,n^{-1/2}\right)-{}
\\
{}-
\fr{\left(u_\alpha-t\sqrt{2k}\,n^{-1/2}\right)\varphi\left(u_\alpha-
t\sqrt{2k}\,n^{-1/2}\right)u_\alpha^4}{16k}+{}\\
{}+ o\left(n^{-1}\right)\,; 
%\label{e2.8b}
\end{multline*}

\vspace*{-12pt}

\noindent
\begin{multline*}
\varphi\left(d_n-t\right)={}\\
{}= \varphi\left(u_\alpha-t+
\fr{u_\alpha^2}{2\sqrt{2k}}-\fr{3}{8k}\,u_\alpha+
\fr{1}{24k}\,u_\alpha^3\right)={}\\
{}=
\varphi\left(u_\alpha-t\right)-\left(u_\alpha-t\right)
\varphi\left(u_\alpha-t\right)\fr{u_\alpha^2}{2\sqrt{2k}}+{}\\
{}+
o\left(n^{-1/2}\right)\,,
%\label{e2.9}
\end{multline*}
получаем, что
\begin{multline*}
\beta_n(t)=1-\Phi\left(u_\alpha-t\right)-
\fr{u_\alpha^2}{2\sqrt{2k}}\,\varphi\left(u_\alpha-t\right)+{}\\
{}+\fr{u_\alpha^2}{2\sqrt{2k}}\,\varphi(u_\alpha-t)-
\fr{2u_\alpha t - t^2}{2\sqrt{2k}}\,\varphi(u_\alpha-t)+{}\\
{}+
o\left(n^{-1/2}\right)=
\Phi\left(t-u_\alpha\right)-\fr{t\left(2u_\alpha - t\right)}{2\sqrt{2k}}\,
\varphi\left(u_\alpha - t\right)+{}\\
{}+ o\left(n^{-1/2}\right)\,.
%\label{e2.10b}
\end{multline*}
Во втором случае при $t > u_\alpha$  выражение
для мощности приобретает вид:

\noindent
\begin{multline*}
\beta_n(t)=\Phi\left(t-u_\alpha\right)-{}\\
{}-
\fr{t\left(2u_\alpha^2+t^2 -2u_\alpha t\right)}{2\sqrt{n}}\,
\varphi\left(u_\alpha-t\right)+ o\left(n^{-1/2}\right)\,.
%\label{e2.11b}
\end{multline*}
При $\alpha \ge 1/2$  аналогичным образом имеем
\begin{multline*}
\beta_n(t)={}\\
{}=
 \Phi\left(t-u_\alpha\right)+
\fr{t\left(2u_\alpha - t\right)}{2\sqrt{n}}\,\varphi\left(u_\alpha - t\right)+
o\left(n^{-1/2}\right)\,.
%\label{e2.12b}
\end{multline*}
Из этих формул следует утверждение теоремы.~$\Box$

\medskip

В работе~\cite{2ben} было показано, что для мощ\-ности~$\beta_n^*(t)$ 
критерия размера $\alpha\in (0,1)$, осно\-ван\-но\-го на
логарифме отношения прав\-до\-подобия~$\Lambda_n(t)$~(\ref{e1.4b}),
справедливо  асимптотическое\linebreak разложение
\begin{equation*}
\beta_n^*(t)=\Phi(t-u_\alpha) - \fr{t^2}{6\sqrt{n}}\,
\varphi(t-u_\alpha)+ o(n^{-1/2})\,.
%\label{e2.13b}
\end{equation*}
Используя это разложение и теорему~1, получаем формулу
для предельного отклонения нормированной разности мощностей
рассматриваемых критериев:
\begin{multline}
r(t)= \lim_{n \to \infty}\sqrt{n}(\beta_n^*(t)-\beta_n(t))
={}\\
{}=
\begin{cases}
\left(t u_\alpha-\fr{2t^2}{3}\right)
\varphi(u_\alpha-t)\,,\\
\hspace*{30mm} t \le u_\alpha\,,\enskip \alpha < \fr{1}{2}\,; \\
\left(u_\alpha^2+\fr{t^2}{3}-u_\alpha t \right)
\varphi(u_\alpha - t)\,,\\
\hspace*{30mm}  t>u_\alpha\,,\enskip \alpha<\fr{1}{2}\,; \\
\left(\fr{t^2}{3}-t u_\alpha\right)\varphi(u_\alpha-t)\,, \quad\quad\ \  \alpha \ge \fr{1}{2}\,. 
\end{cases}
\label{e2.14b}
\end{multline}

\section{Представление выборочной медианы в~виде случайной суммы}

В этом разделе докажем лемму о представлении выборочной медианы
в случае распределения Лапласа в виде суммы случайного числа
независимых экспоненциально распределенных случайных величин.
Формулы для представления порядковых статистик в случае распределения
Лапласа в виде подобной суммы приведены в работе~[4, с.~63],
но без строгого доказательства.

\bigskip

\noindent
\textbf{Лемма 1.}
{\it В случае распределения Лапласа выборочную медиану
можно представить в следующем виде (здесь равенства по распределению):
\begin{align}
\zeta_{2k+1} &\stackrel{d}{=}\delta_{2k+1}
\sum\limits_{j=k+1}^{K_{2k+1}}{\fr{W_j}{j}}\,;
\label{e3.1b}\\
\zeta_{2k}&\stackrel{d}{=}\fr{W_1-W_2}{2k}\,\mathbf{1}(B_{2k+1}=k)+{}\notag\\[1pt]
&\!\!\!\!\!\!\!\!\!\!\!\!\!\!{}+
\left(\delta_{2k}\sum\limits_{j=k+1}^{K_{2k+1}}\fr{W_j}{j}+
\delta_{2k}\fr{W_k}{2k}\right)\mathbf{1}\left(B_{2k+1} \ne k\right)\,,
\label{e3.2b}
\end{align}
где
$$
\delta_n=\mathrm{sign}\left(B_n-k-\fr{1}{2}\right)\,,
$$
$W_j$~--- независимые экспоненциально (с параметром~1) распределенные
случайные величины; $B_n$~--- бернуллиевские случайные величины с параметрами
$p=1/2$ и~$n$, независимые от~$W_j$;
\begin{equation*}
K_n = \max\left(B_n, \bar{B_n}\right)\,,\quad
\bar{B_n}= n - B_n\,.
\end{equation*}
}

\smallskip

\noindent
Д\,о\,к\,а\,з\,а\,т\,е\,л\,ь\,с\,т\,в\,о\,.

Вначале докажем две вспомогательные формулы, справедливые для любого
действительного чис\-ла~$s$
и любых натуральных чисел~$a$ и~ $b$:
\begin{gather}
\prod\limits_{j=a}^{a+b}{\fr{1}{j+is}}=
\sum\limits_{j=0}^b \fr{(-1)^j}{(a+j+is)(b-j)!j!}\,;
\label{e3.3b}
\\[3pt]
\!\!\!\!\!\!\!\!\sum\limits_{l=0}^k\fr{k!}{l!} \prod\limits_{j=a}^{a+k-l}\fr{1}{j+is}=
\sum\limits_{l=0}^k \begin{pmatrix}
k\\ l\end{pmatrix}
\fr{(-1)^l 2^{k-l}}{a+l+is}\,.
\label{e3.4b}
\end{gather}
Формулу~(\ref{e3.3b}) докажем методом математической индукции.

При $b=1$ формула верна. Предполагая ее верной при $b\ge1$,
докажем что она  верна и  при~$b+1$:
\begin{multline*}
\prod\limits_{j=a}^{a+b+1}\fr{1}{j+is}=\fr{1}{a+b+1+is}\prod\limits_{j=a}^{a+b}
\fr{1}{j+is}={}\\[2pt]
{}=
\fr{1}{a+b+1+is}\left(\sum\limits_{l=0}^k 
\begin{pmatrix}
k\\ l
\end{pmatrix}
\fr{(-1)^l 2^{k-l}}
{a+l+is}\right)={}\\[2pt]
{}=
\sum\limits_{j=0}^{b}\fr{(-1)^j}{(b-j)!j!} \left(\fr{1}{(b+1-j)(a+j+is)}
- {}\right.\\[2pt]
\left.{}-\fr{1}{(b+1-j)(a+b+1+is)} \right)={}
\end{multline*}
\begin{multline*}
{}=
\sum\limits_{j=0}^{b}\fr{(-1)^j}{(a+j+is)(b+1-j)!j!}-{}\\
{}-
\fr{1}{a+b+1+is}\sum\limits_{j=0}^{b}\fr{(-1)^j}{(b-j+1)!j!}\,.
\end{multline*}
Заметим, что
\begin{multline*}
\!\!\sum\limits_{j=0}^b\fr{(-1)^j}{(b-j+1)!j!}=
\sum\limits_{j=0}^{b+1}\fr{(-1)^j}{(b-j+1)!j!}
-\fr{(-1)^{b+1}}{(b+1)!}={}\\
{}=
\fr{1}{(b+1)!}(1-1)^{b+1}-\fr{(-1)^{b+1}}{(b+1)!}=
-\fr{(-1)^{b+1}}{(b+1)!}\,.
\end{multline*}
И следовательно, формула~(\ref{e3.3b}) доказана.
Формула~(\ref{e3.4b}) следует  из доказанной формулы~(\ref{e3.3b}), по\-скольку
\begin{multline*}
\sum_{l=0}^k{\fr{k!}{l!}}\prod\limits_{j=a}^{a+k-l}\fr{1}{j+is}={}\\
{}=
\sum\limits_{l=0}^{k}\fr{k!}{l!}\sum\limits_{j=0}^{k-l}
\fr{(-1)^j}{(a+j+is)(k-l-j)! j!}={}\\
{}
=\sum\limits_{j=0}^{k}\fr{(-1)^j}{a+j+is}\sum\limits_{l=0}^{k-j}
\begin{pmatrix}
k\\ j
\end{pmatrix}
\begin{pmatrix}
k-j\\  l
\end{pmatrix}={}\\
{}=
\sum\limits_{j=0}^k\fr{(-1)^j}{a+j+is}
\begin{pmatrix}
k\\ j
\end{pmatrix}
2^{k-j}\,.
\end{multline*}
Теперь приступим к доказательству основного утверждения леммы.
Рассмотрим сначала случай $n=2k+1$.
Плотность $(k+1)$-й порядковой статистики, как известно,
выражается формулой (см.~\cite{4ben})
\begin{equation*}
p_{2k+1}(x) = (2k+1)
\begin{pmatrix}
2k\\  k\end{pmatrix}
f(x)(F(x)(1-F(x))^k\,,
%\label{3.5b}
\end{equation*}
где $f(x)$ и  $F(x)$~--- соответственно плотность и
функция распределения исходных случайных величин.

Найдем характеристическую функцию~$\phi_{2k+1}(s)$ выборочной
медианы~$\zeta_{2k+1}$:
\begin{multline*}
\phi_{2k+1}(s)=\e e^{is\zeta_{2k+1}}=
\int\limits_{-\infty}^{\infty}e^{isx}f(x)\,dx={}\\
{}=
(2k+1)
\begin{pmatrix}
2k\\  k\end{pmatrix}
2^{-(k+1)}\times{}\\
{}\times
\sum\limits_{j=0}^k (-1)^j 2^{-j}
\begin{pmatrix}
k\\ e j\end{pmatrix}
\fr{2(k+1+j)}{(k+1+j)^2+s^2}\,.
%\label{e3.6b}
\end{multline*}
Теперь найдем характеристическую функцию~$f_{2k+1}(s)$ случайной величины, определенной\linebreak\vspace*{-12pt}\pagebreak

\noindent
в правой части  формулы~(\ref{e3.1b}).
С учетом того, что
 характеристическая функция стандартной экспоненциальной
случайной величины равна $1/(1-is)$, имеем
\begin{multline*}
f_{2k+1}(s)={}\\
{}=
\sum\limits_{l=0}^{2k+1}\e \exp \left(is\delta_{2k+1}
\sum\limits_{j=k+1}^{K_{2k+1}}\fr{W_j}{j}\right)\mathbf{1}(B_{2k+1}=l)={}
\\
=2^{-(2k+1)}\left(\sum\limits_{l=0}^k \begin{pmatrix}
2k+1\\  l\end{pmatrix}
\prod\limits_{j=k+1}^{2k+1-l}\fr{j}{j+is}+{}\right.\\
\left.{}+
\sum\limits_{l=k+1}^{2k+1}
\begin{pmatrix}
2k+1\\ l\end{pmatrix}
\prod\limits_{j=k+1}^{l}\fr{j}{j-is}
\right)={}\\
{}
=2^{-(2k+1)}(2k+1)
\begin{pmatrix}
2k\\ k\end{pmatrix}
\sum\limits_{l=0}^k\fr{k!}{l!}
\left(\prod\limits_{j=k+1}^{2k+1-l}\fr{1}{j+is} +{}\right.\\
\left.{}+
\prod\limits_{j=k+1}^{2k+1-l}\fr{1}{j-is}\right)\,.
%\label{e3.7b}
\end{multline*}
Применяя формулу~(\ref{e3.4b}), получаем
\begin{multline*}
f_{2k+1}(s)=(2k+1)
\begin{pmatrix}
2k\\  k
\end{pmatrix}
2^{-(k+1)}\times{}\\
{}\times
\sum\limits_{j=0}^k(-1)^j 2^{-j}
\begin{pmatrix}
k\\  j\end{pmatrix}
\fr{2(k+1+j)}{(k+1+j)^2+s^2}\,.
%\label{e3.8b}
\end{multline*}
Значит, $f_{2k+1}(s)\equiv\phi_{2k+1}(s)$ и представление~(\ref{e3.1b}) доказано.
\medskip

Перейдем теперь к рассмотрению случая четного $n=2k$.
Совместная плотность двух порядковых статистик~$X_{(k)}$ и~$X_{(k+1)}$
определяется формулой (см.~\cite{4ben})
\begin{equation*}
p(x,y)=\fr{(2k)!}{((k-1)!)^2}\,(F(x)(1-F(y)))^{k-1}f(x)f(y)\,.
%\label{e3.9b}
\end{equation*}
Из этой формулы нетрудно получить, что плотность случайной величины
$$
\zeta_{2k}=\fr{X_{(k)}+X_{(k+1)}}{2}
$$
равна
\begin{multline*}
p_{2k}(x) = \fr{(2k)!}{2^k ((k-1)!)^2}\times{}\\
{}\times
\left(\sum_{j=0}^{k-2}\fr{(-1)^j
\begin{pmatrix}
k-1\\ j
\end{pmatrix}
2^{-j}}{k-1-j}
e^{-(k+1+j)|x|}\times{}\right.
\end{multline*}
\begin{multline}
\left.{}\times \left(1-e^{-(k-1-j)|x|}\right)- \right.{}\\
{}\left.
- \fr{(-1)^k}{2^{k-1}}|x|e^{-2k|x|} + \fr{1}{k2^k}e^{-2k|x|}
\vphantom{\fr{(-1)^j
\begin{pmatrix}
k-1\\ j
\end{pmatrix}
2^{-j}}{k-1-j}}\right)\,.
\label{e3.10b}
\end{multline}
Подробный вывод этой формулы приведен в работе~\cite{8ben}.
Исходя их формулы~(\ref{e3.10b}), найдем характеристическую функцию~$\phi_{2k}(s)$
выборочной медианы~$\zeta_{2k}$:
\begin{multline*}
\!\phi_{2k}(s)=
\fr{(2k)!}{2^k ((k-1)!)^2}
\left( \sum\limits_{j=0}^{k-2}
\fr{(-1)^j
\begin{pmatrix}
k-1\\ j
\end{pmatrix}
2^{-j}}{k-1-j}\times{}\right.
\\
\left.{}\times
\left(
\fr{2(k+1+j)}{(k+1+j)^2+s^2}  -
 \fr{4k}{4k^2+s^2} \right)-{}\right.\\
\left. {}- 
 \fr{(-1)^k}{2^{k-2}(4k^2+s^2)} + \fr{1}{2^{k-2}(4k^2+s^2)}
 \vphantom{\sum\limits_{j=0}^{k-2}
\fr{(-1)^j
\begin{pmatrix}
k-1\\ j
\end{pmatrix}
2^{-j}}{k-1-j}}
\right)\,.
%\label{e3.11b}
\end{multline*}
Найдем теперь характеристическую функ-\linebreak цию~$f_{2k}(s)$ случайной величины,
определенной
 в правой части формулы~(\ref{e3.2b}). Учитывая формулу~(\ref{e3.4b}), получим
\begin{multline*}
f_{2k}(s)=\sum\limits_{l=0}^{k-1}{\p(B_{2k}=l)
\fr{2k}{2k+is}\prod\limits_{j=k+1}^{2k-l}{\fr{j}{j+is}}}+{}\\
{}+
\sum\limits_{j=k+1}^{2k}{\p(B_{2k}=l)\fr{2k}{2k-is}\prod\limits_{j=k+1}^{2k-l}\fr{j}{j-is}}+{}\\
{}+
\p(B_{2k}=k)\fr{4k^2}{4k^2+s^2}={}\\
{}=
\fr{(2k)!}{2^k ((k-1)!)^2} \left( \fr{1}{2^{k-2}(4k^2+s^2)}
+{}\right.\\
\left.{}+2^{1-k}\sum\limits_{l=0}^{k-1}(-1)^l 2^{k-l-1}
\begin{pmatrix}
k-1\\ l\end{pmatrix}\times\right.{}\\
{}\times
\left( \fr{1}{(2k+is)(k+1+l-is)}+{}\right.\\
\left.\left.{}+ 
\fr{1}{(2k-is)(k+1+l-is)}\right) \right)\,.
\end{multline*}
Применяя при $l \ne k-1$ следующее соотношение:
\begin{multline*}
\fr{1}{(2k+is)(k+1+l+is)}={}\\
{}=
\fr{1}{k-1-l}\left( \fr{1}{k+1+l+is} - \fr{1}{2k+is}\right)\,,
\end{multline*}
получаем равенство

\noindent
\begin{multline*}
f_{2k}(s)=
\fr{(2k)!}{2^k ((k-1)!)^2}
\left( \sum\limits_{j=0}^{k-2}
\fr{(-1)^j 
\begin{pmatrix}
k-1\\ j
\end{pmatrix}
2^{-j}}{k-1-j}\times{}\right.\\
\left.{}\times
\left(
\fr{2\left(k+1+j\right)}{(k+1+j)^2+s^2} 
-  \fr{4k}{4k^2+s^2} \right)
-{}\right. \\
\left.{}- \fr{\left(-1\right)^k}{2^{k-2}\left(4k^2+s^2\right)} + \fr{1}{2^{k-2}(4k^2+s^2)}
\vphantom{\sum_{j=0}^{k-2}
\fr{(-1)^j 
\begin{pmatrix}
k-1\\ j
\end{pmatrix}
2^{-j}}{k-1-j}}
\right)\,.
%\label{e3.12b}
\end{multline*}
Таким образом,  $\phi_{2k}(s)\equiv f_{2k}(s)$ и утверждение~(\ref{e3.2b})
доказано.~$\Box$

{\small\frenchspacing
{%\baselineskip=10.8pt
\addcontentsline{toc}{section}{Литература}
\begin{thebibliography}{9}

\bibitem{3ben} %1
\Au{Королев Р.\,А., Тестова  А.\,В., Бенинг~В.\,Е.} 
О мощ\-ности асимптотически оптимального критерия в случае 
распределения Лапласа~// Вестник Тверского Государственного Университета, 
серия Прикладная математика, 2008. Вып.~8. №\,4(64). С.~5--23.

\bibitem{9ben} %2
\Au{Takeuchi K.} 
Asymptotic theory of statistical estimation.~---  Tokyo, 1974. (In Japanese.)

\bibitem{1ben} %3
\Au{Бурнашев М.\,В.} 
Асимптотические разложения для 
медианной оценки параметра~// Теор. вероятн. и ее
прим., 1996. Т.~41. Вып.~4. С.~738--753.

\bibitem{5ben}  %4
\Au{Kotz S., Kozubowski~T.\,J., Podgorski~K.}
The Laplace distribution and generalizations: 
A revisit with applications to communications, economics, engineering, 
and finance.~--- Birkhauser, 2001.  P.~349.

\bibitem{6ben}  %5
\Au{Бенинг В.\,Е., Королев В.\,Ю.}
Некоторые статистические  задачи, связанные с распределением Лапласа~// 
Информатика и её применения, 2008. Т.~2.  Вып.~2. С.~19--34.

\bibitem{7ben}  %6
\Au{Леман Э.} 
Проверка статистических гипотез.~--- М.: Наука, 1964. 498~с.

\bibitem{2ben} %7
\Au{Королев Р.\,А., Бенинг В.\,Е.}
Асимптотические 
разложения для мощностей критериев в случае распределения Лапласа~//
Вестник Тверского Государственного Университета, серия 
Прикладная математика, 2008. Вып.~3(10). №\,26(86). С.~97--107.

\bibitem{4ben} %8
\Au{David H.\,A., Nagaraja H.\,N.}
Order Statistics.  3rd ed.~--- New Jersey: Wiley, 2003.  P.~458.

\label{end\stat}

\bibitem{8ben} %9
\Au{Asrabadi B.\,R.} 
The exact confidence interval for 
the scale parameter and the MVUE of the Laplace distribution~// 
Communications in statistics. Theory and methods, 1985. Vol.~14. No.\,3. 
P.~713--733.

 \end{thebibliography}
}
}
\end{multicols}