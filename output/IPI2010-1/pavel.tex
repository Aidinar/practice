\def\stat{pavel}


\def\tit{ПОИСК И АНАЛИЗ КЛЮЧЕВЫХ ТОЧЕК РАДУЖНОЙ ОБОЛОЧКИ ГЛАЗА МЕТОДОМ 
ПРЕОБРАЗОВАНИЯ ЭРМИТА$^*$}
\def\titkol{Поиск и анализ ключевых точек радужной оболочки глаза методом 
преобразования Эрмита}

\def\autkol{Е.\,А.~Павельева, А.\,С.~Крылов}
\def\aut{Е.\,А.~Павельева$^1$, А.\,С.~Крылов$^2$}

\titel{\tit}{\aut}{\autkol}{\titkol}

{\renewcommand{\thefootnote}{\fnsymbol{footnote}}\footnotetext[1]
{Работа выполнена при финансовой поддержке РФФИ (проект №\,10-07-00433-а).}}

\renewcommand{\thefootnote}{\arabic{footnote}}
\footnotetext[1]{Московский государственный университет им.\ М.\,В.~Ломоносова,
факультет вычислительной математики и кибернетики, 
 paveljeva@yandex.ru}
\footnotetext[2]{Московский государственный университет 
им.\ М.\,В.~Ломоносова, факультет вычислительной математики и кибернетики,
kryl@cs.msu.ru}

\vspace*{-6pt}

\Abst{Предложен алгоритм поиска ключевых точек для распознавания человека по 
радужной оболочке глаза, основанный на локальном преобразовании Эрмита. 
Использование для распознавания только ключевых точек радужной оболочки позволяет 
хранить небольшой объем информации при достаточно хорошем качестве распознавания.}
\vspace*{1pt}

\KW{биометрия; радужная оболочка глаза; преобразование Эрмита; ключевые точки}

   \vskip 18pt plus 9pt minus 6pt

      \thispagestyle{headings}
      
       \vspace*{6pt}

      \begin{multicols}{2}

      \label{st\stat}
      
  

\section{Введение}

Преобразование Эрмита~[1] является известным методом, применяющимся для решения 
биометрических задач~[2--4]. Этот локальный метод основан на вычислении сверток функции 
интенсивности изображения с функциями преобразования Эрмита в каждой точке изображения. 
При этом в работе~[5] показано, что для задачи распознавания по радужной оболочке глаза 
наиболее информативными являются свертки с двумерной функцией 
пре\-об\-разо\-ва\-ния Эрмита~$\varphi_{1,0}$. Также широко известным методом в обработке сигналов является метод 
моментов Гаусса--Эрмита~\cite{2pav}, эквивалентный преобразованию Эрмита с 
точностью до знаков сверток с нечетными функциями преобразования Эрмита. При этом для 
решения задач распознавания формируются бинарные матрицы, составленные из знаков сверток 
в каждой точке, которые затем сравниваются с матрицами из базы данных.
{ %\looseness=1

}

В данной работе на основе преобразования Эрмита предложен метод нахождения ключевых 
точек текстуры радужной оболочки. Эти точки соответствуют наиболее значимым экстремумам 
свертки функции интенсивности изображения с функцией~$\varphi_{1,0}$. 

В разд.~2 дается описание преобразования Эрмита. В разд.~3 приведены некоторые детали 
использованного метода предобработки изображений радужной оболочки. Раздел~4 описывает 
алгоритм нахождения ключевых точек радужной оболочки, приведены результаты 
экспериментов на базе данных CASIA-IrisV3~\cite{6pav}. 

\section{Преобразование Эрмита}

Функции Эрмита определяются как
$$
\psi_n (x) = \fr{(-1)^n e^{-x^2/2}}{\sqrt{2^n n!\sqrt{\pi}}}\,H_n(x)\,,
$$
где $H_n(x)$~--- полиномы Эрмита:
\begin{gather*}
H_0(x) =1\,,\quad H_1(x) =2x\,,\\
H_n(x) =2x H_{n-1}(x) -2(n-1)H_{n-2}(x)\,.
\end{gather*}
      
Функции Эрмита являются собственными функциями преобразования Фурье и образуют полную 
ортонормированную систему функций в пространстве~$L_2(-\infty,\,\infty)$. 

Функции преобразования Эрмита (рис.~1) связаны с функциями Эрмита соотношением
$$
\varphi_n(x) =\psi_0(x)\psi_n(x) =\fr{(-1)^n e^{-x^2}}{\sqrt{2^n n!\pi}}\,H_n(x)\,.
$$
При вычислениях они являются одновременно локализованными в координатном и частотном 
пространствах. Так как~$\psi_n$~--- ортонормированная сис\-те\-ма функций, то
$$
\int\limits_{-\infty}^\infty \varphi_n(x)\,dx =\int\limits_{-\infty}^\infty \psi_0(x)\psi_n(x)\,dx =0\quad 
\forall n>0\,,
$$ 
т.\,е.\ функции~$\varphi_n(x)$ имеют среднее нулевое значение для номеров $n>0$. 
Это свойство очень важно\linebreak\vspace*{-12pt}
\pagebreak

\noindent
\begin{center} %fig1
\vspace*{3pt}
\mbox{%
\epsfxsize=77.701mm %78.39mm
\epsfbox{pav-1.eps}
}
%\end{center}
%\vspace*{6pt}

{{\figurename~1}\ \ \small{Примеры функций преобразования Эрмита}}
\end{center}
\vspace*{-6pt}


\bigskip
\addtocounter{figure}{1}


\noindent
 для методов, исполь\-зу\-ющих знаки сверток с такими функциями. 

Двумерные функции преобразования Эрмита являются произведением одномерных функций:
%\noindent
$$
\varphi_{n,m}(x,y) =\varphi_n(x)\varphi_m(y)\,.
$$


Преобразование Эрмита для изображения опреде\-ляется в каждой точке~$(x_0,y_0)$ значениями 
сверток функции интенсивности изображения~$I(x,y)$ с функциями преобразования %\linebreak 
Эрмита~$\varphi_{m,n}(x,y)$ для выбранного конечного набора индек\-сов~$(m,n)$~[1,~4]:
\begin{multline*}
M_{m,n}(x_0,y_0) =(I(x,y) * \varphi_{m,n}(x,y))(x_0,y_0)={}\\
{}= \iint\limits_G I(x,y) \varphi_{m,n}(x_0-x,y_0-y)\,dxdy\,,
\end{multline*}
где $G$~--- область сосредоточения функции~$\varphi_{m,n}$.



\section{Предобработка изображений и~контроль наличия века в~области 
параметризации}

Алгоритм нахождения радужной оболочки на изображении глаза описан в~\cite{5pav, 7pav} и 
основывается на поиске максимального скачка средней интенсивности вдоль круговых контуров 
изображения. После локализации радужная оболочка глаза переводится в прямоугольное 
нормализованное изобра-\linebreak\vspace*{-12pt}


\noindent
\begin{center} %fig2
\vspace*{6pt}
\mbox{%
\epsfxsize=78.847mm %78.39mm
\epsfbox{pav-2.eps}
}
%\end{center}
\vspace*{6pt}

{{\figurename~2}\ \ \small{Нормализация радужной оболочки}}
\end{center}
%\vspace*{-6pt}


\bigskip
\addtocounter{figure}{1}


\noindent
жение. Для дальнейшей параметризации в работе используется только 
область, включающая правую верхнюю четверть нормализованного изображения, на которую, 
как правило, не попадают ресницы и веки~\cite{5pav} (рис.~2).


Тем не менее для определения наличия века в этой области параметризации используется 
специальный алгоритм. 
Ищется максимум вертикальной производной яркости изображения 
$$
\underset{y}{\max}\sum\limits_x\left\vert \fr{\partial I(x,y)}{\partial y}\right\vert
$$
в области $[x_p-r/2,\,x_p+r/2][y_p+r,\,y_p+(R+r)/2]$, выделенной на рис.~3. 
Здесь~$(x_p,y_p)$~--- центр зрачка,
 $r$ и $R$~--- радиусы границ радужной оболочки. Если это 
значение больше порогового, то считается, что нижнее веко попало в область параметризации.

\begin{center} %fig3
\vspace*{6pt}
\mbox{%
\epsfxsize=80.281mm 
\epsfbox{pav-3.eps}
}
%\end{center}
%\vspace*{1pt}

{{\figurename~3}\ \ \small{Изображения $I(x,y)$ и $I_y^\prime(x,y)$}}
\end{center}
%\vspace*{-6pt}

%\bigskip
\addtocounter{figure}{1}

 
\section{Метод ключевых точек параметризации радужной оболочки}

В работе~\cite{5pav} показано, что наиболее информативными номерами двумерных функций 
преобразования Эрмита~$\varphi_{m,n}(x,y)$ для задачи идентификации по радужной оболочке 
являются номера~(1,\,0), (1,\,1), (2,\,0) в указанном порядке. Поэтому для параметризации данных 
радужной оболочки в данной работе была выбрана функция пре\-обра\-зо\-ва\-ния Эрмита~$\varphi_{1,0}(x,y)$. 

Рассмотрим в каждой точке области парамет\-ризации величину $F_1=M_{1,0}$. В~качестве кода 
радужной оболочки (ключевых точек) рассматри-\linebreak\vspace*{-12pt}
\pagebreak

\noindent
\begin{center} %fig4
%\vspace*{6pt}
\mbox{%
\epsfxsize=80mm %78.39mm
\epsfbox{pav-4.eps}
}
\end{center}
%\vspace*{-1pt}
{{\figurename~4}\ \ \small{Область параметризации радужной оболочки с кодом радужной оболочки}}

\vspace*{18pt}

 
\begin{center}
%\vspace*{12pt}
\mbox{%
\epsfxsize=80mm %78.39mm
\epsfbox{pav-5.eps}
}
\end{center}
%\vspace*{-1pt}
{{\figurename~5}\ \ \small{Примеры работы алгоритма выделения ключевых точек для изображений с наложением века и бликов на область 
параметризации}}
%\end{center}
%\vspace*{12pt}


\bigskip
\medskip
\addtocounter{figure}{2}

\noindent
ваются~$N$ ($N = 50$, 100, 150, 200) точек, 
разбитых на две группы: $N/2$ точек с максимальными значениями~$F_1$, удаленных друг от 
друга не менее чем на 2~пикселя, и аналогично~$N/2$~--- с минимальными значениями~$F_1$. 
Пример кода радужной оболочки для $N = 150$ ключевых точек приведен на рис.~4. 
Черными точками обозначены ключевые точки с максимальными значениями~$F_1$, белыми~--- 
с минимальными. На рис.~4 также обозначена граница возможных значений точек кода, 
отстоящая от краев области параметризации на полуширину области сосредоточения 
функции~$\varphi_1$.


Отметим, что этот метод эффективен только для изображений без наложения века на область 
параметризации, так как в области века и бликов большинство детектируемых точек не является 
точками радужной оболочки (рис.~5). 


При идентификации по радужной оболочке используются матрицы сравнения ключевых точек.
Для построения матрицы сравнения область пара\-мет\-ризации разбивается на непересекающиеся 
блоки размера $3\times 3$. Если в блок не попадает ни одной точки кода, то ему соответствует 
значение~0. Если в блок попадает хотя бы одна черная (белая) точка кода, то соответствующее 
значение матрицы сравнения равняется~1~($-1$), если и черная, и белая, то значение равняется 2. 
При сравнении двух матриц считается, что соответствующие блоки равны, если в них попали 
точки одного цвета (табл.~1).


Чтобы алгоритм был устойчив к поворотам глаза (поворот глаза соответствует циклическому 
сдвигу всего нормализованного изображения), сравниваются матрицы точек уменьшенного 
размера. Границы такой урезанной матрицы показаны на рис.~6 и сдвигаются у 
исследуемого изображения в обе стороны до границ возможных значений точек ко-\linebreak\vspace*{-12pt}
\columnbreak

%\bigskip

\noindent
\begin{center}
\noindent
\parbox{51mm}{{\tablename~1}\ \ \small{Сопоставление значений матриц сравнения: <<$+$>> означает, что блоки считаются 
равными, <<$-$>>~--- различными}}
\end{center}
%\vspace*{2ex}

\begin{center}
\tabcolsep=9pt
\begin{tabular}{|c|c|c|c|c|}
\hline
&0&1&$-1$&2\\
\hline
0&$+$&$-$&$-$&$-$\\
1&$-$&$+$&$-$&$+$\\
$-1$\hphantom{$-$}&$-$&$-$&$+$&$+$\\
2&$-$&$+$&$+$&$+$\\
\hline
\end{tabular}
\end{center}
\vspace*{12pt}

%\bigskip
\addtocounter{table}{1}

\begin{center} %fig6
\vspace*{6pt}
\mbox{%
\epsfxsize=80mm %78.39mm
\epsfbox{pav-6.eps}
}
\end{center}
%\vspace*{6pt}
{{\figurename~6}\ \ \small{Области сравнения матриц кодов радужных оболочек}}
%\end{center}
%\vspace*{-6pt}


\bigskip
\addtocounter{figure}{1}


\noindent
да. В~данной 
работе учитываются углы поворота от~$-10^\circ$ до ~$10^\circ$.



Алгоритм параметризации радужной оболочки по ключевым точкам протестирован на базе 
данных  CASIA-IrisV3~\cite{6pav}, содержащей 2655~изображений глаз. Результаты работы 
алгоритма приведены в табл.~2 и на рис.~7. Здесь CRR (Correct Recognition Rate)~--- вероятность верного распознавания.

\bigskip
%\vspace*{3pt}

%\begin{center}
\noindent
{{\tablename~2}\ \ \small{Результаты работы алгоритма ключевых точек}}
%\end{center}
%\vspace*{2ex}

{\small 
\begin{center}
\tabcolsep=8pt
\begin{tabular}{|c|c|c|c|}
\hline
\tabcolsep=0pt\begin{tabular}{c}Число\\ ключевых\\ точек $N$\end{tabular}&
\tabcolsep=0pt\begin{tabular}{c}Число\\ неверных\\ ближайших\\ изображений\\ (из 2655)\end{tabular}&
\tabcolsep=0pt\begin{tabular}{c}Неверные\\ из-за\\ наложения\\ века\\ и бликов\end{tabular}&
\tabcolsep=0pt\begin{tabular}{c}CRR,\\ \%\end{tabular}\\
\hline
200&$23 = 18 + 5$&18&99.81\\
150&$36 = 30 + 6$&30&99.77\\
100&\hphantom{9}$58 = 41 + 17$&41&99.35\\
\hphantom{9}50&$135 = 65 + 70$&65&97.36\\
\hline
\end{tabular}
\end{center}
}
\vspace*{12pt}


%\bigskip
\addtocounter{table}{1}


\begin{center} %fig7
\vspace*{6pt}
\mbox{%
\epsfxsize=74.136mm 
\epsfbox{pav-7.eps}
}
\end{center}
\vspace*{6pt}
{{\figurename~7}\ \ \small{График зависимости CRR от числа взятых ключевых точек}}
%\end{center}
%\vspace*{6pt}


%\bigskip
\addtocounter{figure}{1}

%\noindent


\section{Заключение}

В работе предложен алгоритм идентификации человека, использующий ключевые точки 
радужной оболочки глаза, найденные локальным методом преобразования Эрмита. Этот 
алгоритм позволяет получать достаточно хорошие результаты распознавания даже при 
небольшом объеме хранимой информации. Он достаточно перспективен для использования в 
мультибиометрических системах распознавания.

\bigskip
Работа выполнена при поддержке ФЦП <<Научные и научно-педагогические кадры 
инновационной России>> на 2009--2013~гг.

{\small\frenchspacing
{%\baselineskip=10.8pt
\addcontentsline{toc}{section}{Литература}
\begin{thebibliography}{9}

\bibitem{1pav}
\Au{Martens J.\,B.}
The Hermite transform-theory~// IEEE Transactions on Acoustics, Speech, and Signal Processing, 1990. 
Vol.~38. No.\,9. P.~1595--1606.

\bibitem{2pav}
\Au{Ma~L., Tan~T., Zhang~D., Wang~Y.}
Local intensity variation analysis for iris recognition~// Pattern Recognition, 2004. Vol.~37. No.\,6. 
P.~1287--1298.

\bibitem{3pav}
\Au{Wang L., Dai~M.}
Extraction of singular points in fingerprints by the distribution of Gaussian--Hermite moment~// IEEE 
1st Conference (International) DFMA Proceedings, 2005. P.~206--209.

\bibitem{4pav}
\Au{Estudillo-Romero~A., Escalante-Ramirez~B.}
The Hermite transform: An alternative image representation model for iris recognition~// LNCS, 2008. 
No.\,5197. P.~86--93.


\bibitem{5pav}
\Au{Павельева Е.\,А., Крылов~А.\,С., Ушмаев~О.\,С.}
Развитие информационной технологии идентификации человека по радужной оболочке глаза на 
основе преобразования Эрмита~// Системы высокой доступности, 2009. №\,1. С.~36--42.

\bibitem{6pav}
База данных CASIA-IrisV3. {\sf  http://www.cbsr.ia.ac.cn/ IrisDatabase.htm}.

\label{end\stat}

\bibitem{7pav}
\Au{Krylov A.\,S., Pavelyeva~E.\,A.}
Iris data parametrization by Hermite projection method~// GraphiCon'2007 Conference Proceedings, 2007. P.~147--149. 
 \end{thebibliography}
}
}
\end{multicols}


        