\def\stat{karateev}


\def\tit{АВТОМАТИЗИРОВАННЫЙ КОНТРОЛЬ КАЧЕСТВА ЦИФРОВЫХ 
ИЗОБРАЖЕНИЙ ДЛЯ ПЕРСОНАЛЬНЫХ ДОКУМЕНТОВ$^*$}
\def\titkol{Автоматизированный контроль качества цифровых 
изображений для персональных документов}

\def\autkol{С.\,Л. Каратеев, И.\,В.~Бекетова, М.\,В.~Ососков, В.\,А.~Князь, Ю.\,В.~Визильтер, 
А.\,В.~Бондаренко, С.\,Ю.~Желтов}
\def\aut{С.\,Л.~Каратеев$^1$, И.\,В.~Бекетова$^2$, М.\,В.~Ососков$^3$, В.\,А.~Князь$^3$, Ю.\,В.~Визильтер$^3$, 
А.\,В.~Бондаренко$^3$, С.\,Ю.~Желтов$^3$}

\titel{\tit}{\aut}{\autkol}{\titkol}

{\renewcommand{\thefootnote}{\fnsymbol{footnote}}\footnotetext[1]
{Работа выполнена при поддержке РФФИ, проект 09-07-13551-офи\_ц.}}

\renewcommand{\thefootnote}{\arabic{footnote}}
\footnotetext[1]{ФГУП <<Государственный научно-исследовательский институт авиационных систем>>, goga@gosniias.ru}
\footnotetext[2]{ФГУП <<Государственный научно-исследовательский институт авиационных систем>>, irus@gosniias.ru}
\footnotetext[3]{ФГУП <<Государственный научно-исследовательский институт авиационных систем>>}

\vspace*{6pt}

\Abst{Приведено описание программно-аппаратного комплекса для автоматизации процесса получения и контроля 
цифровых фотографий для биометрических документов: общая структура комплекса, функции, аппаратное и 
алгоритмическое обеспечение. Алгоритмическое обеспечение комплекса включает: детектор лица, модуль оценки 
яркостных и цветовых характеристик; детектор открытых/закрытых глаз; детектор очков; детектор бликов и теней; 
детектор основных элементов лица (рот, нос, брови), детектор поворотов и наклонов головы. Проведено исследование 
точности используемых алгоритмов косвенной оценки углов поворота лица по монокулярному цифровому изображению. 
Для этого разработана специальная методика, основанная на использовании синтезированных ракурсных изображений 
реальных лиц, реконструированных по результатам трехмерного сканирования.}

\vspace*{2pt}

\KW{биометрия; персональные документы; обнаружение лиц; бустинг; трехмерное сканирование; трехмерное 
моделирование}
\vspace*{11pt}

     \vskip 18pt plus 9pt minus 6pt

      \thispagestyle{headings}

      \begin{multicols}{2}

      \label{st\stat}

      
%      \vspace*{6pt}

\section{Введение}

Введение стандарта на цифровые фотографии лица определяет необходимость автоматизации операций контроля качества 
изображений лиц как непосредственно в процессе получения этих изоб\-ра\-же\-ний, 
так и на любом этапе подготовки паспортных, 
визовых и иных персональных доку\-ментов.
{\looseness=1

}

Для обеспечения согласованности национальных стандартов цифровых фотографий международной организацией по 
стандартизации были выработаны рекомендации ISO/IEC FCD
\mbox{19794-5}. В~России стандартом, определяющим требования к 
изображениям лиц для биометрических документов, является ГОСТ ИСО/МЭК 19794-5-2006.
{\looseness=1

}

Для автоматизации процесса получения циф\-ро\-вых фотографий, удовлетворяющих основным требованиям и рекомендациям 
ГОСТ ИСО/МЭК 19794-5-2006, в ГосНИИАС был разработан описанный в данной статье программно-аппаратный комплекс. 
Комплекс обеспечивает получение цифро\-вых фотографий лица, а также оценку в реальном времени основных характеристик 
изображения и параметров лица, что позволяет оператору с мини-\linebreak\vspace*{-12pt}
\columnbreak

\noindent
мальными усилиями, не превышающими усилия, необходимые 
для получения обычной качественной фотографии лица, получать циф\-ро\-вые фотографии лиц, гарантированно удовле\-тво\-ря\-ющие 
требованиям данного ГОСТа. Кроме того, мобильный комплекс может быть использован для контроля параметров фотографий 
лиц, полученных от других источников изображений,~--- как в цифровом, так и в бумажном виде, предоставляя возможность 
оценки пригодности фотографий для последующей биометрической обра\-ботки.
{ %\looseness=1

}

В данной статье также описаны результаты исследования возможной точности алгоритмов косвенной оценки углов поворота лица 
по монокулярному цифровому изображению. Поскольку в соответ\-ствии с требованиями ГОСТ ИСО/МЭК 19794-5-2006 на 
цифровой фотографии лица в биометрических документах допустимый угол поворота изоб\-ра\-же\-ния лица (головы) вокруг любой 
из осей координатной системы не должен превышать~5$^\circ$,\linebreak задача оценки возможной точности измерения 
пространственной ориентации головы на цифровом изоб\-ра\-же\-нии представляет существенный интерес, равно как и задача выбора 
наиболее надежно работающих косвенных методов оценки углов по\-во\-рота.
{\looseness=1

}

\section{Требования к цифровым фотографиям для биометрических документов}

ГОСТ ИСО/МЭК 19794-5-2006 определяет основные требования и дополняющие их рекомендации к цифровым изображениям 
лица и форматам сохранения данных. Общий вид и основные геометрические характеристики фотографии лица приведены на 
рис.~1.

\begin{center} %fig1
\vspace*{18pt}
\mbox{%
\epsfxsize=68.242mm
\epsfbox{kar-1.eps}
}
\end{center}
\vspace*{6pt}
{{\figurename~1}\ \ \small{Основные геометрические характеристики изображения лица}}
%\end{center}
%\vspace*{6pt}


\bigskip
\addtocounter{figure}{1}

Основные характеристики изображения лица на цифровой фотографии для биометрических документов должны соответствовать 
следующим требованиям:
\begin{itemize}
\item минимальный размер фотографии \mbox{$525\;\times$}\linebreak $\times\;420$~пикселей;
\item изображение лица на фотографии должно быть фронтальным и не иметь отклонения относительно основных осей более чем 
на 5$^\circ$;
\item соотношение ширины фотографии и ширины головы ($A$:$CC$) должно быть не менее 7:5 и не более чем 2:1;
\item расстояние от нижней границы фотографии до горизонтальной линии, проходящей через центры глаз ($BB$), должно 
составлять 50\%--70\% от высоты полного изображения;
\item площадь лица на фотографии должна со\-став\-лять 70\%--80\% от площади фотографии;
\item цвет и яркость фона должны обеспечивать надежное определение контура головы;
\item на изображениях лиц не должно быть световых бликов и сильного затенения;
\item на фоне не должно быть теней от головы или каких-либо предметов;
\item на изображениях лица не должно быть закрытых глаз, предметов, закрывающих глаза и лицо или искажающих черты лица.
\end{itemize}

Как видно из вышеприведенных требований, проверка цифровых изображений лица на соответствие требованиям ГОСТа 
является весьма нетривиальной задачей. Здесь недостаточно визуального анализа фотографии для принятия решения о\linebreak 
ее  пригодности для использования в документах,
 удос\-то\-ве\-ря\-ющих личность. Поэтому при создании
 мобильного комплекса 
биометрической\linebreak регистрации изображений лиц было разработано специ\-ализированное программное обеспече-\linebreak ние, 
автоматизирующее процесс создания и\linebreak контроля фотографий, удовлетворяющих требованиям \mbox{ГОСТа}.

\section{Алгоритмическое обеспечение комплекса}

На основании измеренных и рассчитанных характеристик изображения лица алгоритмическое обеспечение осуществляет 
диагностику наличия и причин отклонений от требований ГОСТа и вывод сообщений об этих отклонениях и их возможных 
причинах.

Для получения оценок основных параметров изображения лица в автоматическом режиме решаются следующие задачи:
\begin{itemize}
\item автоматическое обнаружение лица на изображении;
\item автоматическое определение контура и оценка параметров лица на изображении;
\item автоматическое обнаружение глаз на изображении и оценка координат центров зрачков;
\item обнаружение бликов и областей сильной затененности на изображении лица;
\item формирование фронтальных и условно-фрон\-таль\-ных цветных и монохромных изображений заданного размера для печати 
фотографий;
\item формирование изображений для систем обмена биометрическими данными.
\end{itemize}

Алгоритм автоматического обнаружения об\-ласти лица является вариантом каскадного детектора, обучаемого с помощью метода 
Adaboost~\cite{1kor, 2kor}. В~алгоритме слабые (weak) классификаторы построены на основе фильтров Хаара, однако отклик 
формируется с использованием аппроксимации распределения вероятностей амплитуды откликов.\linebreak
 Аппроксимация распределения 
вероятностей откликов представляется в виде гистограммы, которая строится по взвешенным примерам, в результате чего 
обучение слабых классификаторов\linebreak
 проводится на подвыборках одного и того же фиксированного размера, но имеющих 
различные распределения обучающих изображений. В~процессе обучения при формировании каскадного классификатора для 
каж\-до\-го последующего классификатора признаковое пространство сокращается за счет устранения признаков, на которых 
построены предыдущие классификаторы. Поскольку каждый\linebreak
 следующий классификатор обучается в другом\linebreak
  подпространстве 
признаков, он обладает уточ\-ня\-ющи\-ми свойствами и работает по принципу последовательных приближений. Классификатор 
представляет собой линейную комбинацию слабых классификаторов, число которых варьируется от~5 до~75. Классификаторы 
объединяются в каскадную структуру, число слоев каскада варьируется от~4 до~8 в зависимости от желаемого уровня 
соотношения ошибок первого и второго рода. Математическое моделирование показало~\cite{3kor}, что при работе по случайно 
выбранной совокупности тестовых изоб\-ра\-же\-ний алгоритм автоматического обнаружения изображений лица обеспечивает 
вероятность правильного обнаружения не менее~0,95 при веро\-ят\-ности ложного обнаружения лица не более~0,01. 

Алгоритм автоматического определения контура лица на изображении построен на использовании информации об оттенках кожи 
человека.\linebreak
Об\-ластью интереса алгоритма является фрагмент изоб\-ра\-же\-ния, классифицированный как лицо каскадным 
классификатором Adaboost. Каждому пикселю цветного RGB-фрагмента изображения ставится в соответствие вектор параметров 
цвета (H, S, V) в цветовом пространстве HSV ($H$ue, $S$aturation, $V$alue~--- цветовой тон, насыщенность, яркость)~\cite{6kor}. 
Распределение оттенков кожи представлено бинарной картой, хранящейся в структуре данных типа <<просмотровой таблицы>> 
(Look-Up-Table, LUT). Сегментация фрагмента изображения выполняется путем проверки принадлежности параметров цвета 
пикселей к кластеру модели оттенков кожи с помощью операции поиска по таблице. Область изображения кожи формируется из 
пикселей, векторы параметров которых вошли в один из кластеров. Для формирования одной однородной области пикселей, по 
цвету соответствующих оттенкам кожи, и удаления мелких областей, линий и отдельных пикселей к изображению применяются 
такие операции математической морфологии, как дилатация и эрозия. Линии контура лица формируются с помощью алгоритма 
сплайн-интерполяции к координатам граничных пикселей области кожи лица. Размеры прямоугольника, в который вписаны 
линии контура лица, являются линейными размерами изображения лица. На рис.~2 приведены результаты работы 
алгоритма цветовой сегментации изображения лица. Белым цветом отображены пиксели, классифицированные как кожа.

\begin{center} %fig2
\vspace*{4pt}
\mbox{%
\epsfxsize=78.293mm
\epsfbox{kar-2.eps}
}
\end{center}
\vspace*{2pt}
{{\figurename~2}\ \ \small{Пример работы алгоритма цветовой пиксельной сегментации кожи лица: (\textit{а})~
исходное изображение; (\textit{б})~результат выделения кожи на изображении}}
%\end{center}
%\vspace*{6pt}


\bigskip
\addtocounter{figure}{1}


Задача обнаружения изображений глаз решается как задача поиска и распознавания на цифро\-вом изображении лица локальных 
областей, об\-ладающих специфическими характерными для\linebreak
 изображений глаз параметрами. Областью интереса алгоритма 
является часть изображения лица, пред\-став\-ля\-ющая собой область ожидаемого расположения глаз. На основе статистической 
модели проводится предварительная оценка ожидаемого положения и размеров глаз, благодаря чему существенно возрастает 
вычислительная эффективность алгоритма. Алгоритм автоматического обнаружения области глаз с помощью каскадного 
классификатора Adaboost определяет координаты глаз в пределах радужной оболочки. Для точной локализации изображений глаз 
производится поиск координат центров зрачков с использованием операций морфологической фильтрации. Морфологический 
фильтр выделяет изображение зрачка и радужной оболочки глаза, устраняя при этом шумовые помехи и артефакты изображения, 
например блики. Координаты центров зрачков определяются путем свертки изображения с круговым фильтром, 
подчеркивающим форму зрачка. Проведенное математическое моделирование показало, что при работе по случайно выбранной 
совокупности тестовых изображений разработанный алгоритм автоматического обнаружения и локализации изображений глаз 
обеспечивает вероятность правильного обнаружения и локализации не менее~0,95 при ве\-ро\-ят\-ности ложного обнаружения глаз не 
более~0,01. Пример работы алгоритмов обнаружения изображения лица, выделения контура лица и обнаружения глаз 
представлен на рис.~3.

Определение бликов и областей затенения осуществляется путем анализа пространственных распределений яркостей на 
последовательности изоб\-ра\-же\-ний лица, сглаженных окнами различных\linebreak
 размеров. При этом бликом считается появление 
связанной области площадью более 1\% от площади лица, в которой все компоненты RGB равны и превышают по 
амплитуде~250, а признаком затенения является наличие на изображении лица об\-ласти размером более 10\% от площади лица, 
яркость которой отличается более чем на 10\% от яркости симметричной ей об\-ласти изображения. В~связи с тем, что яркостные 
отличия могут быть вызваны причинами, не связанными с наличием тени, например
 дефектами лица или небритостью, признак 
затенения носит рекомендательный характер и не влияет на вывод о соответствии изображения лица требованиям ГОСТа. 
\begin{center} %fig3
%\vspace*{-2pt}
\mbox{%
\epsfxsize=70mm
\epsfbox{kar-3.eps}
}
\end{center}
\vspace*{6pt}
{{\figurename~3}\ \ \small{Пример работы алгоритмов обнаружения лица и обнаружения глаз}}
%\end{center}
%\vspace*{6pt}


\bigskip
\addtocounter{figure}{1}

После проведения основных операций над изоб\-ра\-же\-нием лица выполняется анализ полученных данных на соответствие ГОСТу. 
Для этого производится расчет оценок характеристик изображения, производных от геометрических параметров лица, и проверка 
наличия артефактов на самом изображении. Выполняются следующие операции:
\begin{itemize}
\item определение оси симметрии лица;
\item определение центровки изображения лица;
\item определение угла поворота лица;
\item определение угла наклона лица;
\item обнаружение очков.
\end{itemize}

На заключительном этапе интерпретации\linebreak
результатов проводится проверка оцененных параметров изображения на соответствие 
требованиям стандарта. В случае несоответствия вычисленных параметров изображения требованиям стандарта выдаются 
рекомендации по изменению условий съемки. В случае формирования условно-фрон\-таль\-ных изображений система выполняет 
необходимые повороты и перемасштабирование изображения.

\section{Конструктивные особенности программно-аппаратного комплекса}
Одним из основных приоритетов при выборе конструкции комплекса было требование создания наиболее простой, мобильной и 
сравнительно дешевой конструкции, состоящей из широкодоступных готовых компонентов.

Пограммно-аппаратный комплекс включает:
\begin{itemize}
\item персональный компьютер;
\item цифровой фотоаппарат;
\item источник освещения;
\item специальный штатив для крепления фотоаппарата и источника освещения;
\item планшетный сканер;
\item специализированное программное обеспечение.
\end{itemize}

Комплекс обеспечивает выполнение сле\-ду\-ющих основных функций:
\begin{itemize}
\item захват (оцифровка) и отображение на мониторе последовательности изображений лица, получаемых от цифрового 
фотоаппарата в реальном времени;
\item сохранение изображений на жестком диске компьютера;
\item загрузка и отображение изображений с жесткого диска компьютера;
\item обнаружение изображений лиц, близких к фронтальному положению;
\item обнаружение глаз, определение контура лица, вычисление осей симметрии;
\item определение центровки изображения лица;
\item определение размеров изображения головы;
\item определение углов наклона и поворота головы;
\item обнаружение очков на изображении;
\item оценка качества изображения~--- наличие теней, бликов, оценка цвета, яркости и текстуры фона;
\item сравнение измеренных и вычисленных па\-ра\-мет\-ров изображения лица с требованиями стандартов;
\item индикация результатов сравнения в виде пиктограмм и текстовых сообщений;
\item выбор изображения, удовлетворяющего требованиям стандартов (автоматически или вручную);
\item вывод изображения на печать в заданном формате.
\end{itemize}


При установке системы предлагается выбор используемого устройства видеоввода. В качестве такого устройства может 
использоваться цифровой фотоаппарат, имеющий программный интерфейс с компьютером, или сканер. Кроме этого, в качестве 
источника данных может использоваться любой внешний носитель информации, содержащий массивы цифровых фотографий в 
форматах BMP или JPEG. На рис.~4 представлен общий вид комплекса.


Программное обеспечение работает под управ\-лением ОС Windows 2000/XP. Интерфейс программы представляет собой 
диалоговое окно, в котором помимо изображения текущей фотографии также отображаются результаты проверки требований к 
изображению лица в виде пиктограмм и текстовой информации. Если полученное изображение имеет отклонения от норм ГОСТа, 
оператор получает визуальное и звуковое оповещение. При этом изображения на пиктограммах и соответствующие текстовые 
сообщения подсказывают ему причину
 ошибки. Каждая из пиктограмм, имеющихся в окне программы, соответствует одному из 
приведенных выше требований ГОСТа по характеристикам изображения лица и фотографии. Результаты 
проверки отображаются 
в виде текстовых сообщений\linebreak\vspace*{-12pt}
%\noindent
\begin{center} %fig4
\vspace*{4pt}
\mbox{%
\epsfxsize=79.8mm
\epsfbox{kar-4.eps}
}
%\end{center}

\vspace*{9pt}
{{\figurename~4}\ \ \small{Программно-аппаратный комплекс в сборе}}
\end{center}
\vspace*{-4pt}


\bigskip
\addtocounter{figure}{1}

\noindent
в специальном окне. Кроме этого, для каждой обработанной фотографии программа сохраняет 
результаты всех проверок, выполненных в процессе обработки.


\section{Примеры работы комплекса}

На рис.~5 показаны примеры нескольких типовых ошибок, возникающих при получении циф\-ро\-вых 
фотографий для биометрических документов, и результаты обработки этих изображений про\-грам\-мным обеспечением комплекса.


На приведенных рисунках хорошо видны особенности графического интерфейса. При обнаружении отклонений от требований 
ГОСТа возле ряда\linebreak пиктограмм, содержащего пиктограмму, соответствующую данному отклонению от требований, появляется 
информирующий тревожный сигнал красного цвета, сопровождаемый звуковым сигналом.\linebreak При этом подсвечивается 
соответствующая пиктограмма и появляется текстовая надпись в информационном окне с указанием ошибки и ее фактического 
значения (если это возможно). В случае\linebreak выявления артефактов, неоднозначно трактуемых или нежелательных, но не 
запрещенных ГОСТом, выводится предупредительная визуальная сигнализация на изображении лица и делается 
соответствующая запись в информационном окне.

\section{Методика оценки точности определения пространственной ориентации головы}

Вообще говоря, построение связанной трехмерной системы координат на основании двухмерного\linebreak\vspace*{-12pt}
\pagebreak

\begin{center} %fig5
\vspace*{2pt}
\mbox{%
\epsfxsize=80mm
\epsfbox{kar-5.eps}
}
\end{center}
\vspace*{6pt}
{{\figurename~5}\ \ \small{Несоответствие требованиям: закрытые глаза~(\textit{а}); 
поворот головы~(\textit{б}); наклон головы~(\textit{в})}}
%\end{center}
%\vspace*{6pt}


%\bigskip
\addtocounter{figure}{1}
\columnbreak

\noindent
 изображения лица является 
некорректной обратной задачей. Поэтому в системах подготовки и контроля цифровых фотографий для биометрических 
документов для определения углов поворота относительно пространственных осей используются косвенные методы оценки углов 
поворота головы по изменению пропорций лица, как показано на рис.~5,\,\textit{б}  и~\textit{в}.
Вследствие этого важной 
проблемой является анализ точности определения углов поворота лица, основанного на косвенных методах измерений, и выбор 
наиболее надежно работающих косвенных методов оценки углов поворота.

Координатная система, рекомендованная в стандарте, представлена на рис.~6. Здесь $XYZ$~--- правая система 
координат с центром в точке, соответствующей кончику носа на изображении лица. Углы поворотов определяются 
относительно неподвижной системы координат~$XYZ$, соответствующей полнофронтальному изображения 
лица с углами поворота~(0, 0, 0). 
При этом точного %\linebreak\vspace*{-12pt}
%\noindent
 определения фронтального лица в ГОСТе не приводится. Для определения истинных углов поворота 
изображения необходимо ввести две системы координат: связанную с головой и опорную. Связанная с головой система 
координат $XsYsZs$ определяется двумя ортогональными плоскостями: франкфуртской (или глазнично-ушной) го\-ри\-зон\-талью~--- 
плос\-костью~$XsZs$, проходящей через верхние края отверстий наружного слухового прохода и нижнюю точку нижнего края 
левой орбиты, и плос\-костью~$YsZs$, перпендикулярной плос\-кости~$XsZs$ и проходящей через ось симметрии лица (центр 
переносицы, центр губ). Ось~$Zs$ совпадает с линией пересечения плоскостей и направлена от поверхности лица. Опорная 
система координат~$XYZ$\linebreak

\noindent 
\begin{center} %fig6
%\vspace*{6pt}
\mbox{%
\epsfxsize=66.117mm
\epsfbox{kar-8.eps}
}
\end{center}
\vspace*{2pt}
{{\figurename~6}\ \ \small{Система координат для определения  углов поворота лица}}
%\end{center}
%\vspace*{-6pt}


\bigskip
\addtocounter{figure}{1}
\pagebreak


\noindent
определяет виртуальную камеру, формирующую двумерное изображение, и представляется следующим 
образом: ось~$Y$ параллельна вертикальной оси плоскости изображения, ось~$X$ параллельна горизонтальной оси плоскости 
изображения, ось~$Z$ дополняет до правой системы координат и на\-прав\-ле\-на от модели лица, центр расположен в середине 
двумерного изображения. Изображение считается фронтальным, когда оси опорной и связанной систем координат коллинеарны.


Для оценки точности определения углов поворота изображения лица была предложена методика, основанная на использовании 
трехмерных моделей лиц, получаемых путем трехмерного сканирования. Первым шагом в оценке точности является получение 
трехмерной модели лица человека со строго вертикальным расположением головы. Эта модель считается базовой, и к ней 
привязывается опорная система координат. На следующих этапах исследования модель поворачивается вокруг осей опорной 
системы координат на заданные углы и с нее строится проекция на фокальную плоскость. Эта проекция и считается 
изображением повернутого лица, по которому осуществляются оценки углов поворота изображения. На рис.~7 
и~8 показаны трехмерные модели и соответствующие им синтетические изображения фронтального и повернутого 
лица, созданные по предлагаемой методике.

\begin{center} %fig7
\vspace*{12pt}
\mbox{%
\epsfxsize=79.88mm
\epsfbox{kar-9.eps}
}
\end{center}
\vspace*{2pt}
{{\figurename~7}\ \ \small{Трехмерная модель и синтетическое изображение фронтального лица}}
%\end{center}
%\vspace*{-6pt}


\bigskip
\addtocounter{figure}{1}

\begin{center} %fig8
\vspace*{6pt}
\mbox{%
\epsfxsize=79.88mm
\epsfbox{kar-10.eps}
}
\end{center}
\vspace*{2pt}
{{\figurename~8}\ \ \small{Трехмерная модель и синтетическое изображение искусственно развернутого лица}}
%\end{center}
%\vspace*{-6pt}


\bigskip
\addtocounter{figure}{1}


Для получения трехмерной модели лица используется специализированный комплекс бесконтактных измерений, построенный на 
фотограмметрических принципах, позволяющих рассчитать \mbox{координаты} заданной точки объекта по двум разноракурсным 
изображениям объекта, наблюдаемого стереосистемой видеокамер.

Для применения в задачах антропометрии сис\-те\-ма бесконтактных измерений должна удовлетворять ряду специфических 
требований, таких как безопасность и комфортность для объекта съемки (человека) и высокая скорость съемки, необходимая для 
устранения ошибок, вызванных невозможностью долгого сохранения неподвижности человеком. Кроме того, система 
бесконтактных антропометрических измерений должна измерять координаты поверхности с высоким разрешением и 
представлять их в форме компьютерной трехмерной модели (предпочтительно текстурированной) для последующего анализа. 
Изображение специализированного комплекса бесконтактных измерений приведено на рис.~9.



\begin{center} %fig9
\vspace*{6pt}
\mbox{%
\epsfxsize=80mm
\epsfbox{kar-11.eps}
}
\end{center}
\vspace*{2pt}
{{\figurename~9}\ \ \small{Специализированный комплекс  бесконтактных измерений}}
%\end{center}
%\vspace*{-6pt}


\bigskip
\addtocounter{figure}{1}


Система бесконтактных измерений включает:
\begin{itemize}
\item две камеры для технического зрения, предназначенные для захвата черно-белых изображений человека в 
структурированном свете и расчета трехмерных координат поверхности лица;
\item цветную фотокамеру высокого разрешения для получения цветного изображения и фотореалистичного текстурирования 
трехмерной модели;
\item портативный DLP-проектор для создания ПК-управляемого подсвета, обеспечивающего автоматизацию решения задачи 
стереоотождествления;
\item персональный компьютер.
\end{itemize}

Предварительным этапом работы с фо\-то\-грам\-мет\-ри\-ческим комплексом бесконтактных измерений является 
калибровка~\cite{4kor, 5kor}, т.\,е.\ оценка па\-ра\-мет\-ров модели камер, учитывающих нелинейные\linebreak
 искажения, возникающие при 
формировании изоб\-ра\-же\-ний камерой. Калибровка системы поз\-во\-ля\-ет обеспечить высокую точность измерений трехмерных 
координат объекта.

Внешнее ориентирование системы выполняется для оценки положения и ориентации камер в системе координат, задаваемой 
специальным тестовым объектом. В результате процедуры ориентирования, выполняемой по снимкам тестового сюжета, 
определяются координаты и углы поворота камер в заданной системе координат. В дальнейшем координаты точек поверхности 
объекта рассчитываются в системе координат, заданной ориентированием системы.

Для расчета трехмерных координат поверхности объекта и построения его трехмерной модели необходимо для каждой видимой 
точки объекта найти ее координаты на левом и правом изображениях (решить задачу стереоотождествления точек левого и 
правого изображений). Тогда с использованием результатов ориентирования стереосистемы (положения камер) рассчитываются 
трехмерные координаты точки.

Для автоматизированного решения задачи стереоотождествления соответственных точек изображения с левой и правой камер в 
системе применяется оригинальный кодированный подсвет объекта, минимизирующий число кадров, использующихся для 
расчета трехмерных координат поверхности объекта, при сохранении высокой плотности измерений. 

Основные технические характеристики сис\-темы:
\begin{itemize}
\item время сканирования: $\sim  0{,}5$~с; 
\item время расчета трехмерной модели: 5~с;
\item плотность измерения координат: 10--25 точек на мм$^2$;
\item точность измерения пространственных координат: 0,5~мм.
\end{itemize}

Система выполняет следующие функции: 
\begin{itemize}
\item сканирование и получение необходимого числа снимков лица для последующего использования при построении трехмерной 
модели лица;
\item построение высокоточной трехмерной модели лица;
\item текстурирование полученной трехмерной модели.
\end{itemize}

Для расчета трехмерных координат по\-верх\-ности\linebreak
 используются последовательности снимков с чер\-но-бе\-лых камер, а в качестве 
текстуры служит цветной снимок высокого разрешения, по\-лу\-ча\-емый с циф\-ро\-во\-го фотоаппарата. Текстурирование трехмерной 
модели выполняется автоматически на основе данных ориентирования циф\-ро\-во\-го фотоаппарата.

С помощью описанной методики непосредственных трехмерных измерений были получены оценки точности косвенного 
оценивания па\-ра\-мет\-ров трехмерного позиционирования головы человека по монокулярному изображению, осуществляемого в 
описанной выше системе контроля качества изображений для персональных документов. Полученные оценки точности 
составляют около~1$^\circ$ для углов поворота относительно осей~$Z$ и~$Y$ и около~5$^\circ$ для поворотов относительно 
оси~$X$. Худшая оценка точности при определении угла поворота относительно оси~$X$ связана с высокой вариабельностью 
вертикальных пропорций лица у разных людей.

\section{Заключение}

В статье представлен программно-аппаратный комплекс, предназначенный для автоматизации процесса получения цифровых 
фотографий, удовле\-тво\-ря\-ющих основным требованиям и рекомендациям ГОСТ ИСО/МЭК 19794-5-2006. Комплекс обеспечивает 
получение как фронтальных, так и условно-фрон\-таль\-ных цифровых фотографий лица, а также оценку в реальном времени 
основных характеристик изображения и параметров лица, подготовку изображений к печати с заданными размерами и 
разрешением, формирование изображений в биометрическом формате обмена данными, сохранение изображений в различных 
графических форматах. 

Проведено исследование точности ис\-поль\-зу\-емых алгоритмов косвенной оценки углов поворота лица относительно 
пространственных осей по монокулярному цифровому изображению. Для этого разработана специальная методика, основанная 
на использовании синтезированных ракурсных изоб\-ра\-же\-ний реальных лиц, реконструированных по результатам трехмерного 
сканирования. Разработано алгоритмическое и программное обеспечение для моделирования и измерений, собран 
специализированный комплекс для бесконтактного трехмерного сканирования лиц. С использованием данного комплекса 
получены трехмерные модели и фотореалистические текстуры тестовых лиц. Полученная в результате оценки точность 
измерений составляет в среднем около~1$^\circ$ для углов поворота относительно осей~$Z$ и~$Y$ и около~5$^\circ$ для 
поворотов относительно оси~$X$. Более низкая оценка точности при определении угла поворота относительно оси~$X$ связана с 
высокой вариабельностью вертикальных пропорций лица у разных людей.

{\small\frenchspacing
{%\baselineskip=10.8pt
\addcontentsline{toc}{section}{Литература}
\begin{thebibliography}{9}

\bibitem{1kor}
\Au{Freund Y., Schapire R.}
A short introduction to boosting~// J.~of Japanese Society for Artificial Intelligence, 1999. Vol.~14. No.\,5. P.~771--780.

\bibitem{2kor}
\Au{Viola P., Jones M.}
Robust real time object detection~// IEEE ICCV Workshop Statistical and Computational Theories of Vision, July 2001.

\bibitem{3kor}
\Au{Бекетова И.\,В., Каратеев С.\,Л., Визильтер~Ю.\,В., Бондаренко~А.\,В., Желтов~С.\,Ю.}
Автоматическое обнаружение лиц на цифровых изображениях на основе метода адаптивной классификации AdaBoost~// Вестник 
компьютерных и информационных технологий, 2007. №\,8. С.~2--6.

\label{end\stat}

\bibitem{6kor} %4
\Au{Albiol A., Torres L., Delp~E.\,J.}
Optimum color spaces for skin detection~// Conference (International) on Image Processing Proceedings, 2001. Vol.~1. P.~122--124.

\bibitem{4kor} %5
\Au{Schenk T.\,F.}
Digital photogrammetry.~--- TerraScience, 1999. 

\bibitem{5kor} %6
\Au{Luhmann T., Robson S., Kyle S., Harley I.}
Close range photogrammetry, principles, methods and applications.~--- Whittles, 2006.  510~p. 


 \end{thebibliography}
}
}
\end{multicols}