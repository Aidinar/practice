\def\stat{abstr}
{%\hrule\par
%\vskip 7pt % 7pt
\raggedleft\Large \bf%\baselineskip=3.2ex
A\,B\,S\,T\,R\,A\,C\,T\,S \vskip 17pt
    \hrule
    \par
\vskip 21pt plus 6pt minus 3pt }


\def\tit{ON-LINE INFORMATION MODEL BUILDING OF THE~ EARTH POLE MOTION BY~LINEAR AND~LINEARIZED
FILTERS}

%1
\def\aut{I.\,N.~Sinitsyn$^1$, V.\,I.~Sinitsyn$^2$, E.\,R.~Korepanov$^3$, V.\,V.~Belousov$^4$, 
and~N.\,N.~Semendyaev$^5$}

\def\auf{$^1$IPI RAN, sinitsin@dol.ru\\[1pt]
$^2$IPI RAN, vsinitsin@ipiran.ru\\[1pt]
$^3$IPI RAN, ekorepanov@ipiran.ru\\[1pt]
$^4$IPI RAN, vbelousov@ipiran.ru\\[1pt]
$^5$IPI RAN, nsemendyaev@ipiran.ru}

\def\leftkol{\ } % ENGLISH ABSTRACTS}

\def\rightkol{\ } %ENGLISH ABSTRACTS}

\titele{\tit}{\aut}{\auf}{\leftkol}{\rightkol}

\vspace*{-2pt}

\noindent 
Practical problems of on-line continuous and discrete filtering (estimation of state and parameters) 
of the Earth pole motion by linear and linearized filters are considered. Instrumental algorithms and 
software tools in MATLAB are discussed. Test examples and computer experiments based on astronometric 
measurements are presented.

\label{st\stat}

\vspace*{-5pt}

\KWN{astronometric measurements; instrumental algorithms; software tools; 
information model; linear and linearized filters; generalized Kalman filters; 
on-line filtering; Kalman--Bucy filters; Pugachev filters; fluctuations of the Earth pole
}

%\vskip 14pt plus 6pt minus 3pt


%\def\tit{ASYMPTOTIC PROPERTIES OF RISK ESTIMATE OF WAVELET-VAGUELETTE COEFFICIENTS THRESHOLDING IN TOMOGRAPHIC RECONSTRUCTION PROBLEM}

%2
%\def\aut{A.\,V.~Markin$^1$, O.\,V.~Shestakov$^2$}
%\def\auf{$^1$Department of Mathematical Statistics, Faculty of
%Computational Mathematics and Cybernetics,\\  
%\hphantom{$^1$}M.\,V.~Lomonosov Moscow State University, artem.v.markin@mail.ru\\[1pt]
%$^2$Department of Mathematical Statistics, Faculty of
%Computational Mathematics and Cybernetics,\\  
%\hphantom{$^1$}M.\,V.~Lomonosov Moscow State University,
%oshestakov@cs.msu.su}

\def\leftkol{\ } % ENGLISH ABSTRACTS}

\def\rightkol{\ } %ENGLISH ABSTRACTS}

%\titele{\tit}{\aut}{\auf}{\leftkol}{\rightkol}

%\noindent
%Tomographic image reconstruction problem using wavelet-vaguelette 
%decomposition is considered. Consistency and asymptotic normality 
%of risk estimate of vaguelette coefficients thresholding are studied.

%\label{st\stat}

%\KWN{wavelets; tomography; thresholding; risk estimate; limit distribution}

%\pagebreak

% \thispagestyle{headings}

\vskip 5pt plus 6pt minus 3pt

%\vfil

%3
\def\tit{ASYMPTOTIC DISTRIBUTIONS OF~BASIC STATISTICS IN~GEOMETRIC REPRESENTATION 
FOR~HIGH-DIMENSIONAL DATA AND~THEIR ERROR BOUNDS}

\def\aut{Y.~Kawaguchi$^1$, V.\,V.~Ulyanov$^2$, and Y.~Fujikoshi$^3$}
\def\auf{$^1$Department of Mathematics, Graduate School of Science and Engineering,
Chuo University, Bunkyo-ku, Tokyo,\linebreak
$\hphantom{^1}$Japan,
n15007@gug.math.chuo-u.ac.jp\\[1pt]
$^2$Department of Mathematical Statistics, Faculty of Computational
Mathematics and Cybernetics,\newline
$\hphantom{^1}$M.\,V.~Lomonosov Moscow State University, vulyan@gmail.com\\[1pt]
$^3$Department of Mathematics, Graduate School of Science and Engineering,
Chuo University, Bunkyo-ku, Tokyo,\linebreak
$\hphantom{^1}$Japan, fuji@math.sci.hiroshima-u.ac.jp
}


\def\leftkol{\ } % ENGLISH ABSTRACTS}

\def\rightkol{\ } %ENGLISH ABSTRACTS}

\titele{\tit}{\aut}{\auf}{\leftkol}{\rightkol}

\vspace*{-2pt}

\noindent
In geometric representation of $n$ observations on $p$ variables, it is necessary 
to examine asymptotic behaviors of the three statistics; the length of a $p$-dimensional 
observation vector, the distance between two independent observation vectors, and the angle
between these observation vectors. Hall \textit{et al}. (2005) found 
the asymptotic values of  these three statistics in  high-dimensional framework when the dimension~$p$ 
tends to infinity, while the sample size $n$ is fixed. 
In this paper, their results are extended by deriving asymptotic expansions of the distributions 
of the three statistics. Further, computable error bounds for the limiting distributions of the 
length and the distance were obtained. These results will be useful to obtain statistical 
insights in middle- as well as in high-dimensional data sets.

\vspace*{-5pt}

\KWN{asymptotic expansions; error bounds; high-dimensional data; geometric representation}
%\pagebreak


%\vfil
\vskip 6pt plus 6pt minus 3pt
%\vskip 14pt plus 9pt minus 6pt

%4
\def\tit{ASYMPTOTIC EXPANSION FOR~THE~POWER OF~TEST BASED ON~SAMPLE
MEDIAN IN~THE~CASE~OF~LAPLACE DISTRIBUTION
}

\def\aut{V.\,E.~Bening$^1$ and A.\,V.~Sipina$^2$}
\def\auf{$^1$Department of Mathematical Statistics, Faculty of
Computational Mathematics and Cybernetics,\\  
\hphantom{$^1$}M.\,V.~Lomonosov Moscow State University, bening@yandex.ru\\[1pt]
$^2$Department of Mathematical Statistics, Faculty of
Computational Mathematics and Cybernetics, \\ 
\hphantom{$^1$}M.\,V.~Lomonosov Moscow State University, anna@sipin.ru}

\def\leftkol{\ } % ENGLISH ABSTRACTS}

\def\rightkol{\ } %ENGLISH ABSTRACTS}

%\def\leftkol{ENGLISH ABSTRACTS}

%\def\rightkol{ENGLISH ABSTRACTS}

\titele{\tit}{\aut}{\auf}{\leftkol}{\rightkol}

\vspace*{-2pt}
\noindent
Using asymptotic expansions, the formula for the limit of the difference between
the powers of the test based on the sample median and the most powerful test for the case of Laplace distribution
was obtained.

\vspace*{-5pt}

\KWN{sample median; asymptotic expansion; power function; Laplace distribution
}
\pagebreak

%\vful

 %\vskip 14pt plus 6pt minus 3pt

% \vskip 24pt plus 9pt minus 6pt
\vskip 6pt plus 3pt minus 3pt
%\vspace*{12pt}

%5
\def\tit{ORGANIZATION OF USERS' MANAGEABLE ACCESS TO HETEROGENEOUS DEPARTMENTAL INFORMATIONAL RESOURCES}


\def\aut{G.\,Y.~Ilyushin$^1$ and I.\,A.~Sokolov$^2$}

\def\auf{$^1$IPI RAN, ilushin@ipiran.ru\\[1pt]
$^2$IPI RAN, isokolov@ipiran.ru}


\def\leftkol{ENGLISH ABSTRACTS}

\def\rightkol{ENGLISH ABSTRACTS}

\titele{\tit}{\aut}{\auf}{\leftkol}{\rightkol}

\vspace*{-2pt}

\noindent
The paper is devoted to the realization issues of application interoperability and  
user access to data repositories management during IT-infrastructure renovation in large 
enterprises. For the governed evolution of the legacy management information systems and data 
repositories without interruption  of existing systems operation, a technology is proposed for 
creating the middleware infrastructure based on hybrid software/hardware solutions. The 
middleware based on the service-oriented architecture and Web Services technologies solves the problem of interoperability 
of heterogeneous applications as well as the problem of providing a centralized model of management 
of user access to heterogeneous data repositories  on the basis of formalized roles. Besides 
interoperability issues, the middleware infrastructure enables to solve a broad spectrum of 
problems related to a provision of  information security of the required level.

\vspace*{-5pt}

\KWN{interoperability; access management; middleware; Web Services; legacy systems; metadata}

%\vskip 18pt plus 6pt minus 3pt

 \vskip 6pt plus 6pt minus 3pt

% \pagebreak

%6
\def\tit{PETROLEUM AS A CARRIER OF~INFORMATION ON~ITS~ORIGIN, STRUCTURE, AND~EVOLUTION}

\def\aut{R.\,B.~Seiful-Mulukov}
\def\auf{IPI RAN, rust@ipiran.ru}

\titele{\tit}{\aut}{\auf}{\leftkol}{\rightkol}

\vspace*{-2pt}

\noindent
One of the most complicated and still unambiguously unresolved problems
concerning the origin of petroleum parent substance is considered. It is
proved that the information on petroleum is contained in the petroleum
itself and the petroleum as a complex system can be treated within the frame
of the laws of informatics.

\vspace*{-5pt}

\KWN{petroleum; petroleum geneses; petroleum information carrier; 
petroleum information content; information content evaluation
}
%\pagebreak

 \vskip 6pt plus 6pt minus 3pt

%7
\def\tit{MODELING OF ELASTIC DEFORMATIONS IMPACT 
ON~FINGERPRINT RECOGNITION PERFORMANCE
}


\def\aut{A.\,R.~Arutyunyan}
\def\auf{Nuclear Safety Institute of the Russian Academy of Sciences, 
artem@ibrae.ac.ru}


\titele{\tit}{\aut}{\auf}{\leftkol}{\rightkol}

\vspace*{-2pt}

\noindent
The problem of registration and modeling of distortions in biometric systems is considered. 
A biometric system performance is measured by false acceptance and false rejection 
rate which can be calculated from score distributions in genuine and impostor matches. 
The statistical moments are used to describe these score distributions with few parameters. 
Distortions shift the distributions and, consequently, moments. Based on it,  
an arbitrary distortion is modeled as shifts in score distribution. Finally, the experiments 
with modeling of impact of elastic deformation on fingerprint recognition were carried out.

\vspace*{-5pt}

\KWN{biometrics; operational testing; nonlinear fingerprint deformations}
%\pagebreak

 \vskip 6pt plus 6pt minus 3pt

%8
\def\tit{MATHEMATICAL MODELS OF~FINGERPRINT IMAGE ON~THE~BASIS OF~LINES DESCRIPTION
}

\def\aut{V.\,Yu.~Gudkov}
\def\auf{Chelyabinsk State University, 
Department of Applied Mathematics, diana@sonda.ru
}

\titele{\tit}{\aut}{\auf}{\leftkol}{\rightkol}

\vspace*{-2pt}

\noindent
The mathematical models of fingerprint based on the 
lines as topological vectors and ridge count are presented. They are stored in the template 
with the list of minutiae. The templates are used to identify the fingerprint.

\vspace*{-5pt}


\KWN{fingerprint; minutiae; topology; event detector; ridge count}
%\pagebreak

 \vskip 14pt plus 6pt minus 3pt

%9

\def\tit{AUTOMATED QUALITY TESTING OF DIGITAL IMAGERY FOR PERSONAL DOCUMENTS
}

\def\aut{S.\,L. Karateev$^1$, I.\,V.~Beketova$^2$, M.\,V.~Ososkov$^3$, V.\,A.~Knyaz$^3$, Yu.\,V.~Vizilter$^3$,\\ 
A.\,V.~Bondarenko$^3$, and S.\,Yu.~Zheltov$^3$}
\def\auf{$^1$FGUP ``GosNIIAS,'' goga@gosniias.ru\\[1pt]
$^2$FGUP ``GosNIIAS,'' irus@gosniias.ru\\[1pt]
$^3$FGUP ``GosNIIAS''
}

\titele{\tit}{\aut}{\auf}{\leftkol}{\rightkol}

\noindent
The software-hardware system for digital face imagery acquisition and testing for requirements of 
ISO/IEC FCD 19794-5 standard is described. System contains the following algorithmic modules: face detector; 
color and intensity characteristics estimator; opened/closed eyes detector; glasses detector, reflexes, 
shines, and shadows detector; face features detector (nose, brows, mouth); face slope/rotation detector. 
The precision of face orientation estimation based on monocular digital imagery is addressed. 
The approach for precision estimation is developed based on comparison of synthesized facial 
two-dimensional images and scanned face three-dimensional model.

\KWN{biometrics; personal documents; face imagery; face detection; boosting; three-dimensional reconstruction; 
three-dimensional modeling}
%\pagebreak



\vskip 14pt plus 6pt minus 3pt

%10
\def\tit{AN ALGORITHM FOR CLOTHES-BASED HUMAN RECOGNITION IN~VIDEO
}
\def\aut{V.\,S.~Konushin$^1$, G.\,R.~Krivovyaz$^2$, and~A.\,S.~Konushin$^3$}


\def\auf{$^1$M.\,V.~Keldysh Institute of Applied Mathematics of the Russian Academy of Sciences; M.\,V.~Lomonosov Moscow\linebreak 
$\hphantom{^1}$State University, vadim@graphics.cs.msu.ru\\[1pt]
$^2$M.\,V.~Lomonosov Moscow State University, gkrivovyaz@graphics.cs.msu.ru\\[1pt]
$^3$M.\,V.~Lomonosov Moscow State University, ktosh@graphics.cs.msu.ru}


%\def\leftkol{ENGLISH ABSTRACTS}

%\def\rightkol{ENGLISH ABSTRACTS}

\titele{\tit}{\aut}{\auf}{\leftkol}{\rightkol}

\noindent
The new algorithm for clothes-based human recognition in video is presented. 
The algorithm is based on random patches classification with a random forest. 
The main advantage of the algorithm is the fact that it does not rely on human 
mask; therefore, it is able to work with video with arbitrary complex background. 
Experimental results are reported for a test set, which was acquired using authors' video 
surveillance system.


\KWN{video-based human recognition; machine learning; random forest; background subtraction
}

%\pagebreak

 
 \vskip 14pt plus 6pt minus 3pt

%10
\def\tit{HERMITE TRANSFORM BASED IRIS KEY POINTS SELECTION AND ANALYSIS
}
\def\aut{E.\,A.~Pavelyeva$^1$ and A.\,S.~Krylov$^2$}


\def\auf{$^1$Faculty of Computational Mathematics and Cybernetics,
M.\,V.~Lomonosov Moscow State University,\newline
$\hphantom{^1}$paveljeva@yandex.ru\\[1pt]
$^2${Faculty of Computational Mathematics and Cybernetics,  
\hphantom{$^1$}M.\,V.~Lomonosov Moscow State University,\newline
\hphantom{$^1$}kryl@cs.msu.ru}
}

%\def\leftkol{ENGLISH ABSTRACTS}

%\def\rightkol{ENGLISH ABSTRACTS}

\titele{\tit}{\aut}{\auf}{\leftkol}{\rightkol}

\noindent
Key points selection method based on local Hermite transform for iris recognition has been designed. 
Use of iris key points enables to obtain good recognition results with small storage capacity.


\KWN{biometrics; iris; Hermite transform; key points
}
\pagebreak
 
  \vskip 14pt plus 6pt minus 3pt

%10
\def\tit{DRAWING UP OF AN IDENTIKIT USING EVOLUTIONARY MORPHING AND~QUALIMETRY METHOD}

\def\aut{V.~Protasov}

\def\auf{Russian Center of Computing for Physics and Technology, Protvino, protonus@yandex.ru}


%\def\leftkol{ENGLISH ABSTRACTS}

%\def\rightkol{ENGLISH ABSTRACTS}

\titele{\tit}{\aut}{\auf}{\leftkol}{\rightkol}

 \label{end\stat}

\noindent
The new information technology for drawing up of an identikit is resulted. 
It is based on genetic algorithms and work of group of witnesses. The model of ``the virtual witness'' 
has been developed. The model is intended for calculation of accuracy of drawing up of an identikit. 
It includes ability of the person to drawing persons and the ability to compare different individuals 
according to their similarity to the original. The fatigue of a witness after a long work on the 
recognition and comparison of individuals is counted. The model provides setting method and qualimetry 
of evolutionary morphing.


\KWN{identikit; evolutionary morphing; genetic algorithms; decision making; qualimetry
}
 %\pagebreak