\def\stat{protasov}


\def\tit{СОСТАВЛЕНИЕ СУБЪЕКТИВНОГО ПОРТРЕТА С~ИСПОЛЬЗОВАНИЕМ 
ЭВОЛЮЦИОННОГО МОРФИНГА И~КВАЛИМЕТРИЯ МЕТОДА$^*$}
\def\titkol{Составление субъективного портрета с использованием 
эволюционного морфинга и квалиметрия метода}

\def\autkol{В.\,И.~Протасов}
\def\aut{В.\,И.~Протасов$^1$}

\titel{\tit}{\aut}{\autkol}{\titkol}

{\renewcommand{\thefootnote}{\fnsymbol{footnote}}\footnotetext[1]
{Работа выполнена при поддержке РФФИ, грант №\,08-07-00447-а.}}

\renewcommand{\thefootnote}{\arabic{footnote}}
\footnotetext[1]{Институт физико-технической информатики, Протвино, 
protonus@yandex.ru}
 

\Abst{Приведено описание нового метода составления субъективного портрета группой 
свидетелей, базирующегося на генетических алгоритмах. На основе модели <<виртуального 
свидетеля>> определяется точность составления субъективного портрета. Образ лица, 
<<вспоминаемого>> виртуальным свидетелем, моделируется набором компонент вектора, 
определяющего трехмерное изображение лица. Модель описывает художественные 
способности человека к рисованию лиц и способности к сравнению разных лиц по степени 
их похожести на оригинал. Введена характеристика, описывающая утомляемость свидетеля 
при длительной работе по распознаванию и сравнению лиц. Полученная таким образом 
модель позволяет производить настройку сетевого метода эволюционного морфинга для 
получения оптимального результата и его квалиметрию.}

\KW{субъективный портрет; эволюционный морфинг; генетические алгоритмы; принятие 
решений; квалиметрия}
   \vskip 18pt plus 9pt minus 6pt

      \thispagestyle{headings}

      \begin{multicols}{2}

      \label{st\stat}
     
\section{Введение}
     
     Одна из наиболее сложных задач в современной криминалистике~--- 
составление субъективного портрета (фоторобота) человека, увиденного и 
запомненного с разной степенью точности одним или несколькими 
свидетелями. Развитие современных информационно-коммуникационных 
технологий привело к принципиально новым возможностям решения этой 
проблемы. 
     
     <<Согласно теории отражения, мысленный образ предмета формируется 
в сознании под воздействием самого предмета. Человеческий мозг обладает 
способностью воспринимать внешнюю информацию о предметах через органы 
чувств (в нашем случае предмет~--- это некий человек, его внешний облик, а 
орган восприятия~--- глаза), и длительное время удерживать в памяти 
представление о нем\ldots\ В~связи с тем, что мысленный образ может 
забываться и он имеет место только в сознании ограниченного числа лиц 
(свидетели, очевидцы или потерпевшие), необходимо как можно быст\-рее 
закрепить его с помощью других средств и методов~--- <<актуализировать>> в 
материальной форме>>~\cite{1pr}. Известно, что существующие методы 
со\-став\-ле\-ния субъективного портрета с использованием компьютерных 
технологий или с помощью полицейских художников не всегда дают 
удовлетворительные результаты. 
     
     В настоящей работе представлена новая информационная технология 
генетического консилиума (ГК)~[2--8], предназначенная для составления 
субъективного портрета коллективом свидетелей или одиночным свидетелем с 
использованием эволюционного морфинга лица. В~основу технологии 
положены генетические алгоритмы. Для составления объемных фотороботов 
используется программа FaceGen Modeller~[9]. 
     
     Технологию ГК можно представить следующим образом. Свидетели, 
основываясь на своих воспоминаниях, составляют в первом приближении свои 
варианты фотороботов и отправляют их на сервер. Сетевая программа 
предъявляет каждому свидетелю по два варианта фотороботов, полученных на 
первой итерации его коллегами. Свидетели с помощью специальной процедуры 
скрещивают эти портреты, получая варианты-потомки. Далее они подвергают 
мутации лучший из них и выбирают из нескольких мутированных вариантов 
один, наиболее похожий на оригинал. Эти варианты вновь отправляются на 
сервер, и цикл итераций повторяется до тех пор, пока в эво\-лю\-цио\-ни\-ру\-ющей 
популяции не окажутся варианты-близнецы, в наибольшей степени похожие на 
оригинал. 
     
     Описанные в~[3--8] эксперименты показали принципиальную 
работоспособность нового метода составления субъективного портрета с 
использованием трехмерного морфинга лица группой свидетелей и одиночным 
свидетелем. Было показано, что итерационные процессы получения 
консоли-\linebreak\vspace*{-12pt}
\pagebreak

\noindent
дированного решения сходятся достаточно быстро и качество 
составления фоторобота на заключительной итерации существенно лучше 
качества вариантов первой итерации. Аналогичные результаты были получены 
независимо от этих работ группой шотландских исследователей~[10]. Они 
предложили и использовали технологию эволюционного морфинга, в 
значительной степени повторяющую технологию ГК. 
     
     Недостатками существующих методов составления субъективных 
портретов является то, что неизвестно, с какой точностью коллектив 
свидетелей или одиночный свидетель могут восстановить исходный портрет. 
Даже в случае, когда свидетели имеют достаточно времени для запоминания 
лица некоего человека, неясно, с какой точностью они смогут сделать это. 
     
      В настоящей работе впервые предпринята попытка осуществить 
квалиметрию процесса со\-ставления субъективного портрета коллективом или 
одиночным свидетелем с использованием технологии эволюционного 
морфинга. Для дости\-жения этой цели было необходимо разработать модель 
виртуального свидетеля. Модель должна %\linebreak 
пол\-ностью замещать реального 
свидетеля, так чтобы результаты деятельности виртуального свидетеля были 
неотличимы от результатов деятельности реального свидетеля с такими же 
параметрами.
     
\section{Модель виртуального свидетеля}
     
     Виртуальный свидетель представляет собой программу, содержащую ряд 
параметров, характеризующих основные свойства реального свидетеля, 
пытающегося восстановить и зафиксировать портрет ранее виденного им 
человека. В первом приближении эти свойства можно описать тремя 
параметрами. К ним относятся величина~$K_a$, характеризующая способности 
свидетеля как художника, и величины~$K_0$ и~$\alpha$, описанные ниже и 
характеризующие способности свидетеля к распознаванию лиц. Образ лица, 
<<вспоминаемого>> виртуальным свидетелем, моделируется набором $n$ 
относительных величин некоторого вектора~$G_i$, $i = 1, 2, \ldots , n$. Эти 
величины однозначно определяют трехмерное изображение лица в виде 
полигональной модели. Так, в программе FaceGen Modeller для описания 
симметричного лица без текстуры $n = 50$. Если ставить перед виртуальным 
свидетелем задачу по хранящемуся у него образу восстановить трехмерный 
портрет лица, характеризующегося вектором~$G_i$, то программа, 
моделирующая виртуального свидетеля, восстановит искаженный 
век\-тор-об\-раз~$U_i$ этого лица

\noindent
     \begin{equation}
     U_i = G_i\left[1+K_a \chi \left(1-2\xi\right)\right]\,,
     \label{e1pr}
     \end{equation}
     где $\chi$ и $\xi$~--- случайные числа от 0 до~1. Случайное число~$\chi$ 
отражает особенность свидетеля каждый раз рисовать разные портреты, 
отличающиеся от оригинала, но с постоянным, характеризующим данного 
свидетеля коэффициентом сходства. 
     
     Эксперименты показали, что такая модель достаточно реалистично 
описывает художественные способности человека. Коэффициент $K_a=0$ 
соответствует идеальному художнику, с уменьшением изобразительных 
способностей величина коэффициента растет. 
     
     Для оценки качества <<нарисованного>> портрета введем два 
коэффициента~$K_R$ и $K_S$ следующим образом:
     \begin{gather}
     K_R = \fr{1}{n(G_{\max}-G_{\min})}\sum\limits_{i=1}^n\vert U_i-
G_i\vert\,; \label{e2pr}\\
     K_S=1-K_R\,.\label{e3pr}
     \end{gather}
     Здесь $K_R$~--- коэффициент различия двух портретов, $G_{\max}$~--- 
максимальное и $G_{\min}$~--- минимальное из возможных значений вектора 
$G_i$, $K_S$~--- коэффициент сходства.
     
     По аналогии с коэффициентом художественных способностей вводится 
коэффициент способности к распознаванию лиц~$K_0$. Этот коэффициент 
определяется из эксперимента по распознаванию свидетелем ряда 
сгенерированных программой лиц, в разной степени похожих на искомое. 
     
     При длительных экспериментах, когда перед свидетелем проходит 
длинная череда рас\-по\-зна\-ва\-емых лиц, он начинает ошибаться и его способность 
к распознаванию снижается. Эту способность свидетеля можно 
охарактеризовать величиной <<утомляемости>>~$\alpha$, определяемой из 
выражения
     \begin{equation}
     \alpha = -\fr{1}{N}\ln\fr{K_{0N}}{K_0}\,,
     \label{e5pr}
     \end{equation}
     где $K_{0N}$~--- значение коэффициента~$K_0$ после предъявления 
$N$ портретов в течение одного сеанса экспериментов. 

Типичные значения величин коэффициентов  $K_a$, $K_0$ и $\alpha$, 
полученные из  экспериментов с разными людьми, приведены в 
табл.~1. 

\bigskip

{\small\begin{center} %tabl1
\noindent
{{\tablename~1}\ \ \small{Значения коэффициентов $K_a$, $K_0$ и  $\alpha$}}
\end{center}
%\vspace*{2pt}

\begin{center}
\tabcolsep=5.6pt
\begin{tabular}{|c|c|c|}
\hline
Коэффициент&\tabcolsep=0pt\begin{tabular}{c}Минимальное\\ значение\end{tabular}&
\tabcolsep=0pt\begin{tabular}{c}Максимальное\\ значение\end{tabular}\\
\hline
$K_a$ &0,02\hphantom{999}&0,3\hphantom{9999}\\
$K_0$ &0,92\hphantom{999}&0,99\hphantom{999}\\
$\alpha$ &0,00003&0,00009\\
\hline
\end{tabular}
\end{center}
%\vspace*{6pt}
}


%\bigskip
\addtocounter{table}{1}


     \begin{figure*}[b] %fig1
     \vspace*{1pt}
\begin{center}
\mbox{%
\epsfxsize=147.006mm
\epsfbox{pro-1.eps}
}
\end{center}
\vspace*{-9pt}
     \Caption{Блок-схема настройки параметров эволюционного морфинга
\label{f1pr}}
\end{figure*}

\section{Метод настройки параметров генетического консилиума}
     
     Наряду с получением аппарата для измерения способностей свидетелей к 
процессу составления субъективного портрета создание модели виртуального 
свидетеля преследует еще одну цель. Это разработка механизма настройки 
параметров ГК,\linebreak реализующего эволюционный морфинг, на достижение 
наилучшего результата при работе конкретной группы свидетелей. 
     
     На первом этапе свидетели тестируются на предмет их способностей к 
составлению субъективного портрета. Им предъявляются тестовые примеры, 
при работе над которыми измеряются значения их художественных и 
распознавательных способностей. Определяются также значения параметров 
утомляемости. Эти цифры вводятся в программу настройки параметров 
эволюционного морфинга, и программа при настройке параметров метода на 
оптимальные значения работает с виртуальными свидетелями, обладающими 
свойствами реальных. На втором этапе, отделенном от первого временем 
отдыха, свидетели приступают к работе по со\-став\-ле\-нию искомого портрета. 
     
     Анализ задачи получения оптимальных па\-ра\-мет\-ров ГК при работе 
группы свидетелей показывает, что оптимизируемая величина (степень 
соответствия составленного экспертами портрета\linebreak исходному) в пространстве 
поиска оптимальных па\-ра\-мет\-ров не может быть выражена через них в явном 
виде. Поэтому для проведения оптимизации были выбраны генетические 
алгоритмы. Параметрами настройки эволюционного морфинга являются 
величины KPM~--- чис\-ло <<прогонов>> программы для получения 
установившихся значений коэффициентов сходства, NC~--- количество 
скрещенных портретов, выдаваемых программой виртуальному свидетелю для 
выбора, NM~--- чис\-ло мутированных портретов из одного выбранного портрета 
после скрещивания, PM~--- параметр мутации, или относительная величина 
изменения генов выбранного портрета. На рис.~\ref{f1pr} приведена блок-схе\-ма 
программы настройки этих параметров.    
     
     Данный подход позволяет существенно сократить время работы 
коллектива свидетелей по со\-став\-ле\-нию субъективного портрета, поскольку 
генетическая процедура настройки параметров\linebreak эволюционного морфинга 
(нахождения оптимальной стратегии генетического консилиума) 
приспосабливает этот метод к конкретным особенностям реальных свидетелей.
     

\section{Экспериментальная часть}


     Для проверки предложенного метода настройки параметров ГК с 
использованием модели вир-\linebreak\vspace*{-12pt}
\pagebreak

\noindent
\begin{center} %fig2
%\vspace*{18pt}
\mbox{%
\epsfxsize=77.822mm
\epsfbox{pro-2.eps}
}
\end{center}
\vspace*{6pt}
{{\figurename~2}\ \ \small{Сходимость метода эволюционного морфинга при пяти различных начальных 
приближениях}}
%\end{center}
%\vspace*{6pt}


\bigskip
\bigskip
\addtocounter{figure}{1}


\noindent
туального свидетеля были проведены три группы 
экспериментов.

     
    В первой группе экспериментов проверялась сходимость результатов 
составления субъективного портрета с использованием ГК независимо от 
начальных условий~--- виртуальным свидетелям были заданы низкие 
коэффициенты художественных способностей. На рис.~2 приведены 
результаты сходимости метода для группы из девяти виртуальных свидетелей 
при пяти различных начальных приближениях. Несмотря на то что коллективы 
виртуальных свидетелей <<стартовали>> исходя из популяций портретов с 
низкими значениями~$K_S$, они неизменно получали лучшие результаты.
     
     Во второй группе экспериментов с виртуальными свидетелями 
осуществлялась проверка влияния параметров настройки ГК на качество 
восстанавливаемого портрета.

     Для коллектива из девяти виртуальных свидетелей с известными 
параметрами~$K_a$, $K_0$ и~$\alpha$ было испытано большое число 
вариантов составления портретов для разных наборов параметров 
эволюционного морфинга KPM, NC, NM и PM. Наблюдалось значительное 
влияние параметров ГК на конечный результат. 
     
     Из анализа этих экспериментов был сделан вывод о том, что настройка 
параметров ГК может существенно повысить качество составления 
субъективного портрета. 


     В третьей группе экспериментов проверялась работоспособность метода 
настройки ГК и качество его работы с реальными свидетелями. В одном из 
экспериментов, результаты которого приведены ниже, были предварительно 
измерены способности пяти свидетелей. Способности свидетелей 
характеризовались коэффициентами $0{,}03 <K_a < 0{,}12$, 
$0{,}93 <K_0< 0{,}98$ и $4\cdot 10^{-5} <\alpha< 7\cdot 10^{-5}$. С~использованием 
полученных значений была проведена настройка метода ГК.
     
     Затем реальным свидетелям на непродолжительное время были 
предъявлены фотографии в фас и в профиль неизвестного им лица 
(рис.~\ref{f3pr},\,\textit{а}). Далее свидетели с использованием ГК, 
настроенного на их способности, составляли субъективный портрет. 
Усредненный результат по пяти свидетелям после первой итерации с 
$K_S = 0{,}77$ приведен на рис.~\ref{f3pr},\,\textit{б}. На 
рис.~\ref{f3pr},\,\textit{в} приведены результаты со\-став\-ле\-ния субъективного 
портрета этой группой. Для составления субъективного портрета с 
$K_S = 0{,}92$ потребовалось всего 6~итераций. 


     Метод эволюционного морфинга, при\-ме\-ня\-емый шотландскими 
исследователями в~[10], отличается тем, что свидетели при проведении 
эволюционного морфинга работают поодиночке. При этом в качестве 
окончательного результата используется усредненный портрет. 
В~предложенном же методе свидетели работают в составе ГК совместно, в 
результате чего возникает синергетический эффект усиления 
интеллекта~\cite{4pr}, а составленные субъективные портреты обладают 
большей схо\-жестью с оригиналом. 
     
     С использованием коллектива виртуальных свидетелей была проведена 
сравнительная проверка метода ГК и шотландского метода. Для приведенного 
выше случая величина коэффициента сходства~$K_S$ составила, как было 
показано, 0,92, в то время как для метода, описанного в~[10], $K_S = 0{,}86$. 
{\looseness=1

}
     
     Эксперименты с различными коллективами виртуальных свидетелей и 
исходными портретами показали, что коэффициенты сходства у шотландского 
метода лучше, как и следовало ожидать, чем усредненные коэффициенты после 
первой итерации, но всегда хуже, чем в ГК после шести итераций. 
     
     Здесь следует отметить, что метод шотландских исследователей принят 
на вооружение английской полицией. Было бы интересно провести 
сравнительный анализ двух методов в реальных условиях криминалистической 
практики.
     
\section{Заключение}
     
     В результате проведенных исследований получены следующие 
результаты:
     \begin{itemize}
\item разработан и исследован новый метод со\-став\-ле\-ния субъективного 
портрета, основанный на применении эволюционного морфинга и модели 
виртуального свидетеля;\end{itemize}
%\pagebreak


\end{multicols}

\begin{figure} %fig3
\vspace*{1pt}
\begin{center}
\mbox{%
\epsfxsize=164.062mm
\epsfbox{pro-3.eps}
}
\end{center}
\vspace*{-9pt}
\Caption{Результаты составления субъективного портрета группой из пяти свидетелей 
методом настраиваемого эволюционного морфинга
\label{f3pr}}
\vspace*{6pt}
\end{figure}
     

\begin{multicols}{2}

\noindent
\begin{itemize}
\item использование модели виртуального свидетеля позволяет проводить 
тестирование способностей свидетелей к составлению субъективного 
портрета, настраивать параметры метода на получение оптимальных 
результатов и осуществлять квалиметрию метода;
\item показано, что настройка этих параметров для конкретного состава 
свидетелей позволяет улучшить качество субъективного портрета и 
определить точность его составления;
\item использование модели виртуальных свидетелей позволяет 
сравнивать методы составления субъективных портретов, применяемые 
разными авторами.
\end{itemize}

     Направлением дальнейших исследований является апробация метода 
эволюционного морфинга в реальных условиях криминалистической практики. 
     
{\small\frenchspacing
{%\baselineskip=10.8pt
\addcontentsline{toc}{section}{Литература}
\begin{thebibliography}{99}

\bibitem{1pr}
\Au{Муравев-Витковский А.\,В.}
Габитоскопия. {\sf http:// www.expert.aaanet.ru/rabota/gabito.htm}.

\bibitem{2pr}
\Au{Протасов В.\,И., Панфилов~Д.\,С., Здоровеющев~Ю.\,Ю.}
Генерация фоторобота с помощью сетевого че\-ло\-ве\-ко-ма\-шин\-но\-го интеллекта~// 
Международная на\-уч\-но-тех\-ни\-че\-ская конференция <<Интеллектуальные многопроцессорные 
системы ИМС-99>>.~--- Таганрог, 1999. С.~106--107.

\bibitem{3pr}
\Au{Протасов В.\,И.}
Генерация новых знаний сетевым че\-ло\-ве\-ко-машинным интеллектом. Постановка 
проб\-ле\-мы~// Нейрокомпьютеры. Разработка и применение, 2001. № \,7--8.

\bibitem{4pr}
\Au{Протасов В.\,И.}
Метасистемный эффект самоорганизации интеллекта более высокого уровня из 
искусственных и естественных компонентов~// Сб.\ научных трудов IV Всероссийской 
на\-уч\-но-тех\-ни\-че\-ской конференции <<Нейроинформатика-2002>>.~--- М., 2002. 
С.~33--40.

\bibitem{5pr}
\Au{Протасов В.\,И.}
Тестирование гибридного человеко-машинного интеллекта на шахматных задачах~//\linebreak 
Материалы международной научно-технической конференции <<Искусственный интеллект 
2002>>.~---\linebreak Кацивели, Крым, 2002. С.~348--353.

\bibitem{6pr}
\Au{Шустов Е.\,В., Протасов~В.\,И., Витиска~Н.\,И.}
Решение задачи формирования инвестиционного портфеля кластером компьютеров с 
использованием метода <<двухступенчатого усиления интеллекта>>~//\linebreak
Российский 
экономический Интернет-журнал, 2003. {\sf http://www.e-rej.ru/Articles/2003/Invest.pdf}.

\bibitem{7pr}
\Au{Протасов В.\,И., Витиска~Н.\,И., Шустов~Е.\,В.}
Решение многокритериальной задачи назначений методом генетического консилиума~// 
Российский экономический Интернет-журнал, 2003.\linebreak 
{\sf http://www.e-rej.ru/Articles/2003/Counsil.pdf}.

\bibitem{8pr}
\Au{Протасов В.\,И., Дружинин А.\,А., Михайлов~Л.\,В.}
Методика восстановления субъективного портрета коллективом свидетелей с 
использованием 3D-мор\-фин\-га~// Программные продукты и системы, 2007.  №\,1(77). 
С.~21--24. 

\bibitem{9pr}
FaceGen Modeller~3.1. {\sf http://www.facegen.com/\linebreak modeller.htm}.

\label{end\stat}

\bibitem{10pr}
\Au{Frowd C.\,D., Hancock~P.\,J.\,B., Carson~D.}
EvoFIT: A holistic, evolutionary facial identification technique for creating composites~// 
Association for Computing Machinery Transactions on Applied Psychology, 2004. Vol.~1. 
P.~1--21. 
 \end{thebibliography}
}
}
\end{multicols}