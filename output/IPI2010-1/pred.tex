\def\stat{pred}
{%\hrule\par
%\vskip 7pt % 7pt
\centering\Large \bf%\baselineskip=3.2ex
Т\,Е\,М\,А\,Т\,И\,Ч\,Е\,С\,К\,И\,Й\ \,Р\,А\,З\,Д\,Е\,Л \vskip 17pt
    \hrule
    \par
\vskip 21pt plus 6pt minus 3pt }      

\def\tit{\centerline{Труды секции <<Биометрия>> 19-й Международной конференции}\newline
\centerline{по~компьютерной графике и зрению <<ГрафиКон'2009>>}\newline
\centerline{(г.\ Москва,~Россия, 5--9~октября~2009~г.)}}
\def\titkol{\ %Тематический раздел
}

\def\autkol{\ }
\def\aut{\ }

\titel{\tit}{\aut}{\autkol}{\titkol}

\def\leftkol{\ } % ENGLISH ABSTRACTS}

\def\rightkol{\ } %ENGLISH ABSTRACTS}

                 
      Международная конференция по компьютерной графике и зрению 
<<ГрафиКон>>~--- это крупнейшая международная конференция по компьютерной 
графике и зрению на территории бывшего СССР. В~2009~г.\ конференция прошла 
      5--9~октября 2009~г.\ на базе Московского государственного университета им.\ 
М.\,В.~Ломоносова. Работа одной из секций конференции, посвященной вопросам 
биометрической идентификации личности, организовывалась с участием на\-уч\-но-об\-ра\-зо\-ва\-тель\-но\-го 
центра (НОЦ) Института проблем информатики Российской академии наук и факультета 
вычислительной математики и кибернетики МГУ <<Биометрическая информатика>>. По 
согласованию с НОЦ и руководством секции в журнале публикуется в виде 
тематического раздела цикл статей, являющихся доработанными и дополненными 
версиями докладов, представленных на указанной секции конференции 
<<ГрафиКон'2009>>.
      
В статье А.\,Р.~Арутюняна <<Моделирование влияния деформаций отпечатков пальцев на точность 
дактилоскопической идентификации>> исследована проблема влияния искажений на точность 
дактилоскопической идентификации. Предложен способ учета таких искажений и 
проведены эксперименты по моделированию изменения точности идентификации в 
зависимости от силы деформаций отпечатков пальцев.
      
      Статья В.\,Ю.~Гудкова <<Математические модели изображения отпечатка пальца на основе
описания линий>> посвящена топологической модели изображения отпечатка 
пальца. Особенностью модели является независимость от масштаба и эластичных 
деформаций отпечатков. 

\def\leftkol{\ } % ENGLISH ABSTRACTS}

\def\rightkol{\ } %ENGLISH ABSTRACTS}
      
      В статье С.\,Л.~Каратеева, И.\,В.~Бекетовой, М.\,В.~Ососкова, В.\,А.~Князя, Ю.\,В.~Визильтера, 
А.\,В.~Бондаренко и С.\,Ю.~Желтова <<Автоматизированный контроль качества цифровых 
изображений для персональных документов>> представлен программно-аппаратный комплекс интерактивного 
получения цифрового изображения лица для паспортно-визовых документов нового 
поколения.
      
      В статье В.\,C.~Конушина, Г.\,Р.~Кривовязя и А.\,С.~Конушина <<Алгоритм распознавания людей в видеопоследовательности по 
одежде>> представлен алгоритм 
идентификации личности по одежде. Такие методы идентификации относятся к новым 
<<мягким>> методам идентификации, которые не могут использоваться самостоятельно, но 
могут усилить традиционные способы идентификации в ряде прикладных задач.
      
      В статье Е.\,А.~Павельевой и А.\,С.~Крылова <<Поиск и анализ ключевых точек радужной оболочки глаза методом 
преобразования Эрмита>> предложена новая система признаков 
для идентификации по изображению радужной оболочки глаза. Для вычисления 
признаков предложен способ выявления ключевых точек изображения и их компактного 
описания на основе преобразования Эрмита. 
      
      В статье В.\,И.~Протасова <<Составление субъективного портрета с использованием 
эволюционного морфинга и квалиметрия метода>> исследуется процесс составления субъективного 
портрета (фоторобота) коллективом свидетелей для последующей автоматизированной 
идентификации. По результатам исследования предложен <<генетический>> способ 
получения новых алгоритмов получения фоторобота. Для отбора наиболее эффективных 
алгоритмов используются модели свидетелей (<<виртуальные свидетели>>). 
 \label{end\stat}      
      