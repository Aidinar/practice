\def\stat{konushin}


\def\tit{АЛГОРИТМ РАСПОЗНАВАНИЯ ЛЮДЕЙ В~ВИДЕОПОСЛЕДОВАТЕЛЬНОСТИ ПО~ОДЕЖДЕ$^*$}
\def\titkol{Алгоритм распознавания людей в видеопоследовательности по 
одежде}

\def\autkol{В.\,C.~Конушин, Г.\,Р.~Кривовязь, А.\,С.~Конушин}
\def\aut{В.\,C.~Конушин$^1$, Г.\,Р.~Кривовязь$^2$, А.\,С.~Конушин$^3$}

\titel{\tit}{\aut}{\autkol}{\titkol}

{\renewcommand{\thefootnote}{\fnsymbol{footnote}}\footnotetext[1]
{Работа выполнена при поддержке гранта РФФИ 09-01-92474-МНКС\_а.}}

\renewcommand{\thefootnote}{\arabic{footnote}}
\footnotetext[1]{Институт прикладной математики им.\ М.\,В.~Келдыша РАН; Московский государственный университет им.\ 
М.\,В.~Ломоносова, vadim@graphics.cs.msu.ru}
\footnotetext[2]{Московский государственный университет им.\ М.\,В.~Ломоносова, gkrivovyaz@graphics.cs.msu.ru}
\footnotetext[3]{Московский государственный университет им.\ М.\,В.~Ломоносова, ktosh@graphics.cs.msu.ru}

\vspace*{6pt}

\Abst{Предложен новый алгоритм распознавания людей по одежде. Алгоритм основан 
на классификации случайных регионов с помощью метода случайного леса деревьев (random 
forest). Главным достоинством предложенного алгоритма является то, что в нем не используется 
маска человека, поэтому он применим для видеоданных с произвольным сложным фоном. 
Приведены результаты тестирования алгоритма на собственной выборке 
видеопоследовательностей, полученных со стационарной камеры видеонаблюдения.}


\vspace*{2pt}

\KW{распознавание человека по видео; машинное обучение; случайный лес деревьев; вычитание 
фона}


\vspace*{6pt}

   \vskip 18pt plus 9pt minus 6pt

      \thispagestyle{headings}

      \begin{multicols}{2}

      \label{st\stat}



\section{Введение}
Идентификация личности является одной из самых бурно развивающихся областей 
компьютерного зрения, что вызвано широкой практической применимостью систем на ее основе. 
Наиболее надежными методами идентификации личности считаются методы распознавания по 
отпечаткам пальцев или по радужной оболочке глаза. Однако эти методы относятся к 
инвазивным~--- для распознавания человека они требуют его <<сотрудничества>>, например 
прикладывания пальца к устройству по считыванию отпечатков. Поэтому представляет интерес 
разработка алгоритма, способного распознавать людей лишь по фотографии/видео. 

Для идентификации личности по изображению лица было предложено множество 
алгоритмов~\cite{9konu}. Однако в связи с тем, что данный подход за\-час\-тую не обеспечивает 
требуемой надежности идентификации, много работ посвящено исследованию дополнительных 
признаков. В частности, в последнее время все чаще используется информация об одежде, о 
цвете волос и~т.\,д. Эта информация не является инвариантной для человека~--- он может 
перекрасить волосы, сменить одежду. Однако на протяжении небольшого промежутка времени, 
например одного дня, все эти признаки можно считать неизменными. В данной статье 
предлагается новый алгоритм распознавания человека по видео, полученному со стационарной 
камеры (рис.~\ref{f1konu}). 


Статья организована следующим образом. В~разд.~2 приведен обзор существующих методов. 
В~разд.~3 описывается предложенный алгоритм. Раздел~4 представляет результаты 
проведенных экспериментов. Заключение содержит основные результаты статьи.

\section{Существующие подходы}

Распознавание людей, учитывающее одежду, используется в двух задачах: аннотации 
изображений/видео и в системах видеонаблюдения. 


\begin{figure*} %fig1
\vspace*{1pt}
\begin{center}
\mbox{%
\epsfxsize=164.818mm
\epsfbox{kon-1.eps}
}
\end{center}
\vspace*{-9pt}
\Caption{Пример роликов из созданной тестовой выборки
\label{f1konu}}
\end{figure*}

В алгоритмах аннотации изображений/видео~\cite{3konu, 5konu} одна из основных проблем~--- 
сегментация изоб\-ра\-же\-ния человека. Современные алгоритмы сегментации не позволяют 
автоматически получить точную маску объекта на произвольном фоне. Поэтому стандартная 
схема устроена следующим образом: вначале на изображении находится лицо человека, после 
чего в качестве области одежды берется прямоугольник под лицом человека. Координаты и 
размер прямоугольника задаются эвристически. Однако такой прямоугольник содержит лишь 
небольшую часть от всей области одежды. Более того, в некоторых случаях этот прямоугольник 
может час\-тич\-но захватить область фона. 

В~\cite{1konu} используется более сложный алгоритм сегментации, однако ему тоже для первого 
приближения необходимо найти лицо человека. Поскольку современные алгоритмы нахождения 
лица могут относительно надежно находить лишь фронтальные лица, то в случаях, когда человек 
ни разу не посмотрит прямо в камеру (или его лицо будет чем-то загорожено), данный подход не 
сработает.
%\pagebreak

В системах видеонаблюдения~\cite{4konu, 8konu} маска человека находится с помощью 
алгоритмов вычитания фона. Однако современные алгоритмы вычитания фона хорошо работают 
лишь в относительно прос\-тых случаях. Например, при установке камеры в хорошо освещенном 
коридоре. Но, как было обнаружено на нашей тестовой выборке, в более сложных случаях эти 
алгоритмы могут давать неудовлетворительную маску.


\section{Алгоритм}

Предлагаемый алгоритм распознавания людей по одежде основан на методе классификации~--- 
случайном лесе деревьев~\cite{2konu}. В качестве признаков для классификации используются 
цвета пикселей набора квадратных регионов, случайным образом выбираемых из кадров 
видеоролика. В следующих подразделах описываются соответственно процессы обучения 
классификатора и распознавания людей в видеороликах с помощью обученного классификатора.

\subsection{Обучение классификатора}

Для обучения алгоритма из всех видеороликов обучающей выборки извлекается большое чис\-ло 
($N_{\mathrm{Train}}$) произвольных квадратных регионов (рис.~\ref{f2konu},\,\textit{а}). Размер и 
положение региона, а также номер кадра при этом выбираются абсолютно случайно. 

После этого проводится нормализация всех регионов: каждый из них масштабируется до размера 
$r\times r$, переводится в цветовое пространство HSV и представляется вектором длины $3r^2$. 

Так как все видеоролики сняты одной и той же стационарной камерой, становится возможным 
использование в качестве признаков региона еще и его пространственного положения~--- сдвига 
$(x,y)$ и ширины~$w$. Таким образом, получается обучающая выборка из $N_{\mathrm{Train}}$ векторов 
размера $3r^2 + 3$ (рис.~\ref{f2konu},\,\textit{б}). Каждому элементу выборки ставится в 
соответствие метка~--- идентификатор человека с соответствующего видео. 

Затем происходит обучение классификатора, который по произвольному региону (развернутому в 
вектор $3r^2 + 3$) выдает метку человека, присутствующего на видео.

Одним из главных критериев при выборе алгоритма машинного обучения была скорость, так как 
и размер входной выборки, и длина вектора признаков каждого элемента выборки очень велики. 

Поэтому в качестве такого алгоритма был выбран случайный лес деревьев~\cite{2konu}. При его 
обуче\-нии для каждого отдельного дерева выбирается случайная подвыборка из общей 
обуча\-ющей выборки. Используемые в узлах дерева функции~--- сравнение одной из координат 
входного вектора с порогом. При этом для скорости выбор координаты и самого порога при 
обучении происходит абсолютно случайно. Каждое дерево строится до тех пор, пока оно не 
станет правильно классифицировать свою обучающую подвыборку.

\subsection{Распознавание}

Для распознавания из тестового видеоролика также извлекается большое число, $N_{\mathrm{Test}}$, 
случай-\linebreak\vspace*{-12pt}
\pagebreak

\end{multicols}

\begin{figure} %fig2
\vspace*{1pt}
\begin{center}
\mbox{%
\epsfxsize=164.162mm
\epsfbox{kon-2.eps}
}
\end{center}
\vspace*{-9pt}
\Caption{Схема работы алгоритма: (\textit{а})~извлечение случайных квадратных регионов из 
видео; (\textit{б})~нормализация регионов, вытягивание их в векторы; (\textit{в})~классификация 
регионов с помощью случайного леса деревьев; (\textit{г})~получение финального результата 
классификации голосованием
\label{f2konu}}
\end{figure}

\begin{multicols}{2}


\noindent
но расположенных квадратных регионов, которые проходят описанную выше процедуру 
нормализации. После этого все они независимо друг от друга подаются на вход обученному 
классификатору (рис.~\ref{f2konu},\,\textit{в}). На выходе получается $N_{\mathrm{Test}}K$ меток, где 
$K$~--- число деревьев в обученном классифи\-каторе.
{\looseness=1

}

Понятно, что при этом регионы, расположенные полностью в области фона, будут помечены 
произвольными метками. Распределение неправильных меток в списке будет равномерным. Зато 
регионы, перекрывающие область объекта (человека), в целом будут классифицированы 
правильно, поэтому в среднем большинство меток будет соответствовать реальному человеку 
на видео.


В связи с этим в качестве вероятности каждой метки~$L$ можно брать относительное число раз, когда 
регион из тестового видео был классифицирован как~$L$:

\noindent
$$
p(L) =\fr{\sum\limits_{k=1}^K \sum\limits_{x_i\in X} l(T_k(x_i)=L)}{N_{\mathrm{Test}}K}\,,
$$
где $X$~--- множество из $N_{\mathrm{Test}}$ векторов, а $T_i(x)$~--- результат классификации 
вектора~$x$ $i$-м деревом (рис.~\ref{f2konu},\,\textit{г}).

\section{Эксперименты}

\subsection{Тестовые данные}

\begin{figure*}[b] %fig3
\vspace*{1pt}
\begin{center}
\mbox{%
\epsfxsize=161.135mm
\epsfbox{kon-3.eps}
}
\end{center}
\vspace*{-9pt}
\Caption{Результаты работы алгоритма по метрике <<$N$ лучших>>. Штриховой линией 
отмечен результат работы предложенного алгоритма, сплошной линией~--- средний результат 
случайной классификации: (\textit{а})~учет роликов, на которых человек входит в комнату;
(\textit{б})~учет всех роликов
\label{f3konu}}
\end{figure*}


Для экспериментов была собрана собственная выборка видеороликов. Эти видеоролики 
записывались камерой, висящей под потолком и на\-прав\-лен\-ной на входную дверь в лабораторию. 
Запись проводилась в течение 8~дней, всего было собрано 463~видеоролика. Разрешение
 видео~--- $352 \times 240$, средняя длительность ролика~--- около 10~с. 

Примеры кадров из собранных видеороликов показаны на рис.~\ref{f1konu}. Всего на разных 
видеороликах присутствует 25~разных людей.

Каждый ролик был вручную аннотирован на предмет присутствующих на нем людей (их меток), 
а также флагом~--- входит данный человек в комнату или, наоборот, выходит из нее.



Во многих случаях было невозможно разбить общее видео на несколько роликов, в каждом из 
которых присутствовал бы лишь один человек~--- например, когда два--три человека 
одновременно выходят из комнаты. В данной работе такие ролики были удалены из 
рассмотрения. Стоит отметить, что существующие методы тоже пока не способны обрабатывать 
такие случаи (надежно сегментировать людей в кадре).

С учетом исключения из рассмотрения таких роликов, в каждый конкретный день людей, 
появляющихся на двух и более роликах, в среднем было 5--8~человек.

Стоит отметить, что заснятая на видеороликах сцена является очень сложной для алгоритмов 
вычитания фона. В частности, протестированные алгоритмы~\cite{7konu, 6konu} давали 
неудовлетворительные маски объекта. Основными причинами таких результатов были 
нестабильность фона (открывающаяся дверь, полупрозрачное стекло) и нестабильное освещение. 
Таким образом, методы, явно опирающиеся на известную маску человека, здесь неприменимы.

\subsection{Проведенные эксперименты}

При проведении экспериментов использовались следующие параметры: число регионов в 
обуча\-ющей выборке $N_{\mathrm{Train}} =1\,000\,000$; нормализованный размер региона $r=16$; число 
регионов, извлекаемых из тестового видео $N_{\mathrm{Test}}=3000$; число деревьев $K=20$.

Для тестирования использовалось два сценария. В первом сценарии задействованы лишь те 
ролики, на которых человек входит в комнату. В качестве обучающих данных использовались 
первые два видеоролика каждого человека, остальные попадали в тестовую выборку. Во втором 
сценарии уже использовались все видеоролики. Обучающими были первые три ролика каждого 
человека, все остальные~--- тестовые.

В обоих случаях человек из тестовой выборки присутствовал в обучающих данных.

Результаты работы алгоритма представлены на рис.~\ref{f3konu}. Демонстрируемые результаты 
являются суммой результатов за все 8~дней наблюдения.

В качестве метрики качества использовалась мет\-ри\-ка <<$N$ лучших>> (Top$N$), которая для 
каждого конкретного $n$ показывала, для какого процента всех роликов правильная метка была 
среди первых $n$ результатов.

Так как за один день в обучающей выборке могло присутствовать максимум 14~человек, 
значение метрики при $n\geq 14$ равно 100\%.

Для ориентира на графиках приведен результат распознавания в случае, если бы классификация 
осуществлялась случайно (подбрасыванием мо\-нетки). 

Из графиков видно, что полученные результаты еще далеки от того, чтобы их можно было 
надежно использовать в реальной системе. Например, в первом сценарии процент правильно 
распознанных роликов (т.\,е.\ Top$N(1)$) составляет лишь~45\%.

Однако следует учитывать сложность входных данных. В частности, человек в один и тот же 
день может появляться на видео как в куртке, так и без нее. А значит, если, скажем, на первых 
двух (обучающих) роликах он будет в куртке, то распознать его на остальных роликах по 
используемым признакам будет практически невозможно.

Большую проблему составляет освещение. К~кон\-цу дня, когда уже темно, но еще не включили 
свет в комнате, даже человек зачастую не сразу распознает людей на видео. 

Во втором сценарии используется тот факт, что во многих случаях цвет и текстура одежды на 
спине очень похожа на одежду спереди. А значит, есть надежда, что алгоритм, обученный на 
видео, на котором человек входит, сможет распознавать его же, но когда он выходит, и наоборот. 
Как видно, пока, к сожалению, при данном сценарии алгоритм сработал значительно хуже, чем в 
первом.

\section{Заключение}

В данной статье был предложен новый алгоритм распознавания человека по видео. Основным 
его преимуществом является то, что он не опирается на маску переднего плана, полученную с 
по\-мощью алгоритмов вычитания фона, как это делает большинство существующих алгоритмов. 
Благодаря этому для его обучения достаточно предоставить лишь выборку видеороликов с 
меткой-идентификатором присутствующего на видео человека. Алгоритм был протестирован 
на собственной выборке видеопоследовательностей, полученных с камеры видеонаблюдения.

{\small\frenchspacing
{%\baselineskip=10.8pt
\addcontentsline{toc}{section}{Литература}
\begin{thebibliography}{9}

\bibitem{9konu} %1
\Au{Zhao W., Chellappa~R., Phillips~P.\,J., Rosenfeld~A.}
Face recognition: A literature survey~// ACM Computing Surveys, 2003. 
Vol.~35. No.\,4. P.~399--458.

\bibitem{3konu} %2
\Au{Jaffre G., Joly~P.}
Costume: A new feature for automatic video content indexing~// RIAO Proceedings, 2004. P.~314--325. 

\bibitem{5konu} %3
\Au{Song Y., Leung T.}
Context-aided human recognition~--- clustering~// ECCV Proceedings, 2006. P.~382--395. 

\bibitem{1konu} %4
\Au{Gallagher A., Chen~T.}
Clothing cosegmentation for recognizing people~// Proc. of CVPR, 2008. No.\,1. P.~1--8. 

\bibitem{4konu} %5
\Au{Nakajima C., Pontil M., Heisele~B., Poggio~T.}
Full-body person recognition system~// Pattern Recognition, 2003.
Vol.~36. No.\,9. P.~1997--2006.

\bibitem{8konu} %6
\Au{Yoon K., Harwood D., Davis~L.}
Appearance-based person recognition using color/path-length profile~// J.~Visual Communication Image Representation, 2006.
Vol.~17. No.\,3. P.~605--622.

\bibitem{2konu} %7
\Au{Geurts P., Ernst D., Wehenkel~L.}
Extremely randomized trees~// Machine Learning J., 2006. Vol.~63. No.\,1.

\bibitem{7konu} %8
\Au{Wren C., Azarbayejani A., Darrell~T., Pentland~A.}
Pfinder: Real-time tracking of the human body~// IEEE Trans.\ PAMI, 1997. Vol.~19. No.\,7. 
P.~780--785. 


\label{end\stat}

\bibitem{6konu} %9
\Au{Stauffer C., Grimson~W.\,E.\,L.}
Adaptive background mixture modelsfor real-time tracking~// CVPR Proceedings, 1999. P.~246--252. 



 \end{thebibliography}
}
}
\end{multicols}