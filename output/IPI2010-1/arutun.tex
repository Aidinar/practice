\def\stat{arutun}


\def\tit{МОДЕЛИРОВАНИЕ ВЛИЯНИЯ ДЕФОРМАЦИЙ ОТПЕЧАТКОВ ПАЛЬЦЕВ НА~ТОЧНОСТЬ 
ДАКТИЛОСКОПИЧЕСКОЙ ИДЕНТИФИКАЦИИ$^*$}
\def\titkol{Моделирование влияния деформаций отпечатков пальцев на~точность 
дактилоскопической идентификации} 

\def\autkol{А.\,Р.~Арутюнян}
\def\aut{А.\,Р.~Арутюнян$^1$}

\titel{\tit}{\aut}{\autkol}{\titkol}

{\renewcommand{\thefootnote}{\fnsymbol{footnote}}\footnotetext[1]
{Работа выполнена
в рамках исследований Научно-образовательного центра ИПИ РАН\,--\,ВМК МГУ 
<<Биометрическая информатика>>.}}

\renewcommand{\thefootnote}{\arabic{footnote}}
\footnotetext[1]{Институт проблем безопасного развития атомной энергетики Российской академии наук, 
artem@ibrae.ac.ru}

%\vspace*{-6pt}

\Abst{Рассмотрена проблема учета влияния искажающих факторов на точность 
био\-мет\-ри\-ческой идентификационной системы. Предложена модель искажающих факторов, 
основанная на приближении методом моментов условных плотностей распределений меры 
сходства биометрических образцов. Разработан способ их оценки и учета. Проведены 
эксперименты по моделированию влияния эластичных деформаций на точность 
дактилоскопической идентификации.}

%\vspace*{-2pt}

\KW{биометрическая идентификация; операционные испытания; нелинейные деформации 
отпечатков}

\vskip 18pt plus 9pt minus 6pt

%\vspace{6pt}

      \thispagestyle{headings}

      \begin{multicols}{2}

      \label{st\stat}

\section{Введение }

  На сегодняшний день биометрические технологии идентификации личности получили 
широкое распространение в различных областях обеспечения безопасности: от контроля и 
управления доступом в офисные помещения до гражданской идентификации и 
правоохранительных приложений~[1--3]. С~распространением таких технологий актуальной 
становится проблема выбора того или иного метода биометрической идентификации. Основным 
фактором, определяющим результаты выбора, является точность идентификации, выраженная 
ROC (Receiver Operating Characteristics~--- функциональные характеристики приемника)
кривой соотношения вероятностей ошибок 1-го (FRR~--- False Reject Rate, вероятность ложного отказа)
и 2-го (FAR~--- False Acceptance Rate, вероятность ложной идентификации) рода~[4]. При этом 
  ROC-кривая зависит от данных, которые использовались при ее оценке, т.\,е.\ фактически 
зависит от среды и ха\-рак\-те\-ри\-зу\-ющих ее искажающих факторов. На практике эти зависимости 
могут быть очень сильными. На рис.~\ref{f1ar} приведен пример измерения ROC-кривой для 
технологии NIST VTB (National Institute of Standards and
Technology Verification Test Bed)~[5] на массивах отпечатков пальцев, полученных в разное время. Видно, 
что оценки вероятности FRR при одной и той же вероятности FAR могут различаться в 5--6~раз. 
Аналогичные оценки потенциального влияния среды могут быть получены теоретически~[6,~7].
  
  Подобная ситуация типична и для других биометрических характеристик. На рис.~2 
показано изменение точности идентификации за период с 1993 по 2006~гг., измеренное в ходе 
технологических испытаний NIST (США). В~частности, эти цифры указывают на то, что с~2002 
по 2006~гг.\ FRR снизилась в 20~раз, хотя при измерении точности на одном массиве (рис.~3)
наблюдается изменение в 3--4~раза, что больше соответствует практическим наблюдениям. 
Основной причиной таких рас\-хож\-де\-ний является изменение условий испытаний. В~2006~г.\ 
использовались данные, полученные в лучших операционных условиях по сравнению с~2002~г.\ 

Обрат\-ная си\-ту\-а\-ция наблюдалась в испытаниях технологий распознавания по отпечаткам пальцев 
FVC (Fingerprint Verification Competition): данные тестовых массивов FVC2004 оказались значительно хуже массивов FVC2002. 
Поэтому наблюдалось формальное снижение точности идентификации в то время, когда 
технологии распознавания по отпечаткам пальцев с точки зрения практики значительно 
улучшились.

  Такая неточность в оценке вероятности ошибок идентификации может приводить к 
не\-пра\-вильным решениям при проектировании и эксплуатации биометрических систем и даже к 
неправильному выбору используемой биометрической характеристики.
  
  Для решения этой проблемы в статье предложена модель искажающих факторов 
биометрической идентификации и способ оценки степени их влияния на точность распознавания 
и его последующего учета при эксплуатации. Входными параметрами модели являются 
численные оценки искажающих\linebreak\vspace*{-12pt}
\pagebreak

\end{multicols}

\begin{figure} %fig1
\vspace*{1pt}
\begin{center}
\mbox{%
\epsfxsize=154.762mm
\epsfbox{aru-1.eps}
}
\end{center}
\vspace*{-9pt}
\Caption{Кривые ROC идентификации по отпечаткам пальцев для технологии NIST VTB на 
различных массивах: TAR~--- True Acceptance Rate
%\textit{1}~--- `dos-cli.dat'; \textit{2}~--- `dos-cri.dat'; \textit{3}~--- `dhs2-cli.dat'; 
%\textit{4}~--- `dhs2-cri.dat'; \textit{5}~--- `benli.dat';  \textit{6}~--- `ben??.dat'?????; 
%\textit{7}~--- `benri.dat'; \textit{8}~--- `benrt.dat'; \textit{9}~--- `dhs10li.dat'; \textit{10}~--- 
%`dhs10ri.dat'; \textit{11}~--- `dhs10rt.dat'; \textit{12}~--- `txdpsli.dat'; \textit{13}~--- `txdpslt.dat'???;
%\textit{14}~--- `txdpsri??.dat'; \textit{15}~--- `txdpsrt.dat'; \textit{16}~--- `visit\_poe\_bvali.dat';
%\textit{17}~--- `visit\_poe\_bvari.dat'; \textit{18}~--- `visit\_poeli.dat'; 
%\textit{19}~---  `visit\_poeri.dat'
\label{f1ar}}
\end{figure}

\begin{multicols}{2}

\noindent
 факторов и ROC-кривая в эталонных операционных условий. 
Выходными параметрами~--- ROC-кри\-вая в реальных условиях эксплуатации.


  Статья организована следующим образом. В~разд.~2 приведена модель искажающих факторов. 
Раздел~3 посвящен применению модели к учету упругих деформаций в задаче 
дактилоскопической идентификации. Заключение содержит основные выводы по работе.

\noindent
\begin{center} %fig2
\vspace*{3pt}
\mbox{%
\epsfxsize=79.993mm
\epsfbox{aru-3.eps}
}
\end{center}
\vspace*{3pt}
{{\figurename~2}\ \ \small{Номинальный прогресс точности идентификации по форме лица в испытаниях NIST}}
%\end{center}
%\vspace*{6pt}



%\bigskip
\addtocounter{figure}{1}

\section{Модель операционных условий}

  При оценке точности идентификации био\-мет\-ри\-че\-ская система достаточно полно 
характеризуется условными плотностями распределений: меры сходства биометрических 
образцов в <<своих>> и <<чужих>> сравнениях. Обозначим их через~$f^{\mathrm{gen}}(x)$ 
и~$f^{\mathrm{imp}}(x)$ соответственно. Принятие решения осуществляется на основе сравнения меры 
сходства с порогом. Ошибки идентификации при выбранном пороге определяются как интегралы 
от плотностей (рис.~\ref{f4ar}). Ошибки FAR и FRR определяются по следующим формулам:

\vspace*{-3pt}

\noindent
  \begin{align*}
  \mathrm{FAR}(t) & = \int\limits_t^{+\infty} f^{\mathrm{imp}}(x)\,dx\,;\\
  \mathrm{FRR}(t) & = \int\limits_{-\infty}^{t} f^{\mathrm{gen}}(x)\,dx\,,
  \end{align*}
где $t$~--- порог принятия решения.



  Под влиянием различных операционных условий плотности распределений меняются. 
В~\cite{6ar} был предложен способ грубого учета операцион-\linebreak\vspace*{-12pt}
\pagebreak

\end{multicols}

\begin{figure} %fig3
\vspace*{1pt}
\begin{center}
\mbox{%
\epsfxsize=159.966mm
\epsfbox{aru-2.eps}
}
\end{center}
\vspace*{-9pt}
\Caption{Изменение точности идентификации при испытаниях NIST FRVT2002 и NIST 
FRVT2006 на одном массиве: \textit{1}~--- $\mathrm{FAR}=10^{-2}$; \textit{2}~--- 
$\mathrm{FAR}=10^{-3}$; \textit{3}~---  $\mathrm{FAR}=10^{-4}$ 
\label{f2ar}}
\end{figure}

\begin{figure} %fig4
\vspace*{1pt}
\begin{center}
\mbox{%
\epsfxsize=108.872mm
\epsfbox{aru-4.eps}
}
\end{center}
\vspace*{-9pt}
\Caption{Примерный вид плотностей условных распределений меры сходства биометрического 
сравнения
\label{f4ar}}
\end{figure}

\begin{multicols}{2}

\noindent
ных условий на основе 
вероятностных моментов первого и второго порядка. Воспользуемся теперь более точным 
приближением плотностей био\-мет\-ри\-че\-ских тестов~\cite{8ar} методом моментов с базовой 
нормальной плотностью~\cite{9ar}. 
  
  Пусть на основе эмпирических данных оценены вероятностные моменты условных 
распределений. Обозначим их через $\{m^{\mathrm{gen}}, \sigma^{\mathrm{gen}}, \gamma_3^{\mathrm{gen}}, \ldots , 
\gamma_n^{\mathrm{gen}}\}$  и $\{m^{\mathrm{imp}}, \sigma^{\mathrm{imp}}, \gamma_3^{\mathrm{imp}}, \ldots , 
\gamma_n^{\mathrm{imp}}\}$ для <<своих>> и <<чужих>> сравнений соответственно. 
Статистика~$\gamma$ (нормированный вероятностный момент) вычисляется по следующей 
формуле:
  \begin{equation*}
  \gamma_k =M\left [ \left (\fr{x-m}{\sigma}\right )^k\right]\,.
%  \label{e3ar}
  \end{equation*}
  
  Плотности распределений приближаются следующими функциями:
  \begin{equation*}
  f(x) =\sum\limits_{i=1}^n \gamma_i R_i\left (m+\alpha x\right) f^M_{0,1}\left (m+\alpha x\right )\,,
%  \label{e4ar}
  \end{equation*}
где $R_i$~--- базисные полиномы,  $f_{0,1}^N$~--- плотность стандартного нормального 
распределения.

  Приближение достаточно точно описывает реальные биометрические данные с точки зрения 
основной технологической характеристики: ROC-кри\-вой (рис.~5). 

\setcounter{figure}{5}
\begin{figure*}[b] %fig6
\vspace*{1pt}
\begin{center}
\mbox{%
\epsfxsize=162.398mm
\epsfbox{aru-6.eps}
}
\end{center}
\vspace*{-9pt}
\Caption{Среднее изменение ROC при идентификации отпечатков под действием деформаций (\textit{1}~--- без 
деформаций; \textit{2}~--- с деформациями):
(\textit{а})~Biolink MST; (\textit{б})~прямое наложение изображений 
\label{f6ar}}
\end{figure*}
  
  Операционные условия влияют на ROC-кри\-вую. Если воздействие искажающего фактора 
приводит к изменениям в ROC-кри\-вой, то оно также\linebreak\vspace*{-12pt}
\pagebreak

\noindent
\begin{center} %fig5
\vspace*{6pt}
\mbox{%
\epsfxsize=73.258mm
\epsfbox{aru-5.eps}
}
\end{center}
\vspace*{6pt}
{{\figurename~5}\ \ \small{Приближение ROC дактилоскопической идентификации: \textit{1}~--- исходная ROC;
\textit{2}~--- нормальная аппроксимация; \textit{3}~--- аппроксимация методом моментов (база NIST 
BSSR1$\backslash$ri)~[4]}}
%\end{center}
%\vspace*{6pt}



\bigskip
%\addtocounter{figure}{1}

\noindent
приводит к изменениям в векторе 
параметров~$S$, который можно рассматривать как функцию~$S$ от операционных 
условий~$u$. Если есть численная оценка операционных факторов, то изменение распределений 
можно приближенно определить по формуле
  \begin{equation}
  S(u) =S(0) +\sum\limits_{i=1}^w \fr{\partial S}{\partial u_i}\,u_i\,.
  \label{e5ar}
  \end{equation}

\section{Моделирование искажающих факторов на примере деформаций отпечатков пальцев}

  В качестве примера использования моделирования искажающих факторов рассмотрим 
деформации отпечатков пальцев. Выбор деформаций в качестве модельного примера 
определяется тем, что это один из немногих искажающих факторов дактилоскопической 
идентификации, который может быть численно оценен. Как показано в~\cite{10ar, 11ar}, сила 
воздействия деформации может быть определена энергией деформации. 
  
  Для оценки влияния деформаций на параметры распределений были проведены эксперименты 
на базе FVC2002~DB1 с двумя алгоритмами распознавания отпечатков пальцев: Biolink MST и 
прямого наложения изображений. На рис.~\ref{f6ar} представлены примеры изменения ROC под 
воздействием деформаций отпечатков пальцев~\cite{9ar, 10ar} для каждого из этих методов 
распознавания. В табл.~\ref{t1ar} и~\ref{t2ar} приведены изменения вероятностных моментов.


  
\begin{table*}\small
\begin{center}
\parbox{105mm}{\Caption{Изменение вероятностных моментов условных распределений мер сходства под 
влиянием деформаций отпечатков пальцев (Biolink MST)
\label{t1ar}}

}

\vspace*{2ex}

\begin{tabular}{|l|c|c|c|c|c|c|}
\hline
\multicolumn{1}{|c|}{Отпечатки}&$\mu $&$\sigma$&$\gamma_3$&$\gamma_4$&$\gamma_5$&$\gamma_6$\\
\hline
Свои (без деформаций)&730&181&$-0{,}22$&2,84&$-3{,}29$&16,70\\
Свои (с деформациями) &680&198&$-0{,}78$&3,94&$-9{,}75$&36,00\\
Чужие (без деформаций)&100&\hphantom{9}73&\hphantom{9,}0,76&3,01&\hphantom{9,}6,00&18,10\\
Чужие (с деформациями)&\hphantom{9}96&\hphantom{9}73&\hphantom{9,}0,66&3,13&\hphantom{9,}5,56&19,02\\
\hline
\end{tabular}
\end{center}
\end{table*}


\begin{table*}\small
\begin{center}
\parbox{105mm}{\Caption{Изменение вероятностных моментов условных распределений мер сходства под 
влиянием деформаций отпечатков пальцев (прямое наложение изображений)
\label{t2ar}}

}

\vspace*{2ex}

\tabcolsep=6.4pt
\begin{tabular}{|l|c|c|c|c|c|c|}
\hline
\multicolumn{1}{|c|}{Отпечатки}&$\mu$&$\sigma$&$\gamma_3$&$\gamma_4$&$\gamma_5$&$\gamma_6$\\
\hline
Свои (без деформаций)&335&147&0,33&2,81&\hphantom{9}2,83&\hphantom{9}14,96\\
Свои (с деформациями) &264&138&0,83&3,35&\hphantom{9}6,81&\hphantom{9}22,47\\
Чужие (без деформаций)&\hphantom{9}54&\hphantom{9}22&1,31&6,22&23,45&116,00\\
Чужие (с деформациями)&\hphantom{9}55&\hphantom{9}21&1,23&5,79&21,59&104,76\\
\hline
\end{tabular}
\end{center}
\end{table*}


  Для уточнения измерения влияния деформаций согласно уравнению~(\ref{e5ar}) были 
построены зависимости параметров распределений от энергии деформации. На рис.~\ref{f8ar} 
представлены зависимости меры сходства в своих сравнениях от энергии деформации на основе 
эмпирических данных FVC2002DB1. В чужих сравнениях была принята гипотеза о 
независимости распределений от фактора деформаций, так как для этих сравнений моменты 
практически не изменяются (см.\ табл.~\ref{t1ar} и~\ref{t2ar}).
  
\begin{figure*} %fig7
\vspace*{1pt}
\begin{center}
\mbox{%
\epsfxsize=157.606mm
\epsfbox{aru-8.eps}
}
\end{center}
%\vspace*{-9pt}
\Caption{Выборочные зависимости меры сходства от энергии деформации на репрезентативной 
выборке в экспериментах с базой FVC2002DB1: \textit{1}~--- прямое наложение; \textit{2}~--- 
Biolink MST; \textit{3}~--- линейный (прямое наложение); \textit{4}~--- линейный (Biolink MST)
\label{f8ar}}
\end{figure*}

  Из рис.~\ref{f8ar} видно, что имеется ощутимая зависимость меры сходства от энергии 
деформации. На рис.~\ref{f9ar} представлены графики относительных изменений 
параметров распределений в своих сравнениях при росте силы деформации отпечатков. 
В~качестве единичной эталонной энергии деформации взята средняя энергия, измеренная на базе 
FVC2002 DB1.


  Результаты искусственного моделирования точности идентификации в зависимости от силы 
деформации представлены на рис.~\ref{f11ar}.

\end{multicols}

\begin{figure}  %fig8
\vspace*{1pt}
\begin{center}
\mbox{%
\epsfxsize=161.822mm
\epsfbox{aru-9.eps}
}
\end{center}
\vspace*{-9pt}
\Caption{Относительное изменение параметров распределений в зависимости от силы 
деформации (от~1 до~6): (\textit{а})~Biolink MST; (\textit{б})~прямое наложение
\label{f9ar}}
\vspace*{6pt}
\end{figure}
\begin{figure} %fig9
\vspace*{1pt}
\begin{center}
\mbox{%
\epsfxsize=162.166mm
\epsfbox{aru-11.eps}
}
\end{center}
\vspace*{-9pt}
\Caption{Изменение ROC в зависимости от силы деформации (\textit{1}~--- без деформаций (факт.);
\textit{2}~--- без деформаций (модель);
\textit{3}~--- с деформациями\;$\times$\;1 (модель);
\textit{4}~--- с деформациями (факт.);
\textit{5}~--- с деформациями\;$\times$\;0,5 (модель);
\textit{6}~--- с деформациями\;$\times$\;2 (модель)):
(\textit{а})~Biolink MST; 
(\textit{б})~прямое  наложение
\label{f11ar}}
\vspace*{6pt}
\end{figure}

\begin{multicols}{2}

\section{Заключение}

В статье представлена модель операционных условий и метод оценки показателей точности 
биометрической идентификации при их изменении. Эксперименты с дактилоскопической 
идентификацией позволяют сделать предварительное заключение о практической применимости 
модели. 

В дальнейшем модель может применяться к оценке других искажающих факторов 
биометрической идентификации, допускающих непосредственное численное измерение, 
например угла поворота лица.


{\small\frenchspacing
{%\baselineskip=10.8pt
\addcontentsline{toc}{section}{Литература}
\begin{thebibliography}{99}

\bibitem{3ar} %1
\Au{Bolle R.\,M., Connell J.\,H., Pankanti~S., Ratha~N.\,K., Senior~A.\,W.}
Guide to biometrics.~--- New York: Springer-Verlag, 2003.

\bibitem{1ar} %2
\Au{Dessimoz D., Champod~C., Richiadi~J., Drygajlo~A.}
Multimodal biometrics for identity documents. Research Report, PFS 314-08.05. UNIL, June~2006.

\bibitem{2ar} %3
\Au{Sinitsyn I.\,N., Ushmaev~O.\,S.}
Development of metrological and biometric technologies and systems~// 9th Conference (International) 
in Pattern Recognition and Image Analysis: New Information Technologies~--- PRIA-9-2002 
Proceedings.~--- Nizhni Novgorod, 2008. Vol.~2. P.~169--172.

\bibitem{4ar}
\Au{Wayman~J.\,L., Jain~A.\,K., Maltoni~D., Maio~D.}
Biometric systems: Technology, design and performance evaluation.~--- London: 
Springer Verlag, 2005.

\bibitem{5ar}
\Au{Watson C., Wilson~C., Indovina~M., Cochran~B.}
Two finger matching with Vendor SDK matchers. NISTIR 7249. July 2005.

\bibitem{6ar}
\Au{Ушмаев О.\,С.}
Адаптация биометрической системы к искажающим факторам на примере 
дактилоскопической идентификации~// Информатика и её применения, 2009. Т.~3. 
Вып.~2. С.~25--33.
\pagebreak

\bibitem{7ar}
\Au{Ушмаев~О.\,С., Арутюнян~А.\,Р.}
Метод оценки качества биометрической идентификации в операционных условиях 
на примере дактилоскопической идентификации~// Труды конференции 
ГрафиКон'2009: 19-я Международная конференция по компьютерной графике и 
зрению.~--- М.: МАКС ПРЕСС, 2009. С.~232--235.

\bibitem{8ar}
\Au{Ushmaev O., Novikov S.}
Biometric fusion: Robust approach~// 2nd Workshop on Multimodal User 
Authentication~--- MMUA'06 Proceedings.  Toulouse, France, 11--12~May 2006. {\sf  
http://mmua.cs.uchb.edu/ MMUA2006/Papers/127.pdf}.

\bibitem{9ar}
\Au{Pugachev V.\,S., Sinitsyn I.\,N.}
Stochastic systems: Theory and applications.~--- World Scientific, 2001.

\label{end\stat}

\bibitem{10ar}
\Au{Ушмаев О.\,С., Арутюнян А.\,Р.}
Влияние деформаций на качество биометрической идентификации по отпечаткам 
пальцев~// Информатика и её применения, 2009. Т.~3. Вып.~4. С.~12--21.



\bibitem{11ar}
\Au{Novikov S. Ushmaev~O.}
Registration and modelling of elastic deformations of fingerprints~// Biometric 
Authentication: ECCV 2004 International Workshop~--- BioAW2004 Proceedings.~--- 
Berlin--Heidelberg: Springer-Verlag, 2004. 
P.~80--88.
 \end{thebibliography}
}
}
\end{multicols}