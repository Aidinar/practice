\def\stat{gudkov}


\def\tit{МАТЕМАТИЧЕСКИЕ МОДЕЛИ ИЗОБРАЖЕНИЯ ОТПЕЧАТКА ПАЛЬЦА НА ОСНОВЕ
ОПИСАНИЯ ЛИНИЙ}
\def\titkol{Математические модели изображения отпечатка пальца на основе
описания линий}

\def\autkol{В.\,Ю.~Гудков}
\def\aut{В.\,Ю.~Гудков$^1$}

\titel{\tit}{\aut}{\autkol}{\titkol}

%{\renewcommand{\thefootnote}{\fnsymbol{footnote}}\footnotetext[1]
%{Работа выполнена
%при финансовой поддержке РФФИ, проекты 08-01-00567 и
%08-07-00152.}}

\renewcommand{\thefootnote}{\arabic{footnote}}
\footnotetext[1]{Челябинский государственный университет,
кафедра прикладной математики, diana@sonda.ru}


\Abst{Представлена математическая модель изображения отпечатка пальца на основе
топологических векторов для линий и векторов гребневого счета для линий. Модель сохраняется
в шаблоне изображения и используется при идентификации.}

\KW{отпечаток пальца; частный признак; топология; детектирование событий; гребневый счет}

     \vskip 18pt plus 9pt minus 6pt

      \thispagestyle{headings}

      \begin{multicols}{2}

      \label{st\stat}

\section{Введение }

В компьютеризированных системах верификация и идентификация дактилоскопических
изоб\-ра\-жений (ДИ) выполняется на основе шаблонов,\linebreak
 базис которых составляет описание частных
признаков в виде начал и окончаний, слияний и разветвлений линий [1, 2]. Их обычно
детектируют по стилизованному представлению ДИ в виде скелета линий, представленному на
рис.~1~\cite{5gud, 1gud}.


Математическая модель должна опираться на необходимое и достаточное число
признаков~\cite{1gud}. В~дактилоскопии информативными считают частные признаки и
гребневый счет между ними~\cite{5gud}.\linebreak\vspace*{-12pt}
\begin{center} %fig1
\vspace*{18pt}
\mbox{%
\epsfxsize=79.038mm
\epsfbox{gud-1.eps}
}
\end{center}
\vspace*{6pt}
{{\figurename~1}\ \ \small{Скелет и частные признаки изображения отпечатка пальца}}
%\end{center}
%\vspace*{6pt}


%\bigskip
\addtocounter{figure}{1}

\noindent 
Однако они не исчерпывают все множество
математических моделей, используемых для автоматического доказательства индивидуальности
ДИ~[1, 2, 4, 5]. Каждая из таких моделей нацелена на повышение точности
идентификации, однако неизвестно ни одной лучшей, свободной от недостатков
модели~\cite{5gud}. Например, классический гребневый счет, который в криминалистике
предписывается измерять по прямой линии, не работает в об\-ласти значительного искривления
линий узора и в об\-ласти петель, дельт и завитков~\cite{7gud}. Это очевидно при измерении
гребневого счета по прямой линии, проходящей близко к касательной к изображению
папиллярной линии (см.\ рис.~1).

\section{Шаблон изображения}

Частные признаки и гребневый счет между ними как набор данных сохраняют в файле, который
называют шаблоном~\cite{5gud}. У~разных производителей программного обеспечения шаблоны
различны~[1, 2, 6], но можно указать их общее свойство: они содержат данные,
являющиеся некоторой мет\-ри\-кой для частных признаков.

В работе предлагается шаблон изображения, синтезируемый как отображение в виде
\begin{equation*}
\Gamma:\ F_0^{(m)} \rightarrow \{ L_m, L_l, L_r\}\,,
%\label{e1gud}
\end{equation*}
где $F_0^{(m)} =\vert f_0^{(m)} (x,y)\vert$~--- скелет изображения (см.\ рис.~1);
$L_m$~--- список частных признаков; $L_l$~--- список топологических векторов для линий;
$L_r$~--- список векторов гребневого счета для линий.

Начала и окончания, слияния и разветвления линий называют частными признаками. Учет
направления частного признака в сторону увеличения числа линий позволяет выделить два типа
частных признаков: окончания и разветвления~\cite{5gud}. Они указаны стрелками на
рис.~1, а их множество образует основу для формирования списка~$L_m$.

Построение списков~$L_l$ и~$L_r$ опирается на частные признаки и скелетные линии. Однако и
топологический вектор, и вектор гребневого счета имеют много общего: они характеризуют
одинаковые свойства множества точек линии, а не окрестность одной точки. Эта интересная
особенность повышает устойчивость модели.

\subsection{Список частных признаков}

Пусть $M_i$~--- индексированный номером~$i$ частный признак. Список частных
признаков~$L_m$ находится в виде
\begin{equation}
L_m =\left \{ M_i=\{(x_i,y_i),\alpha_i,t_i\}\vert i=\overline{1,n_1}\right \}\,,
\label{e2gud}
\end{equation}
где $\vert L_m\vert =n_1$~--- мощность списка; $(x_i, y_i)$, $\alpha_i$ и~$t_i$~--- координаты,
направление и тип частного признака, детектируемого внутри информативной об\-ласти
изображения~\cite{3gud, 4gud} (на рис.~1 информативная область ДИ затемнена, а
скелет зачернен).

Координаты~$(x_i,y_i)$ определяются координатами вершины скелета~\cite{1gud}. Направление
$\alpha_i$ как угол определяется простой цепью вершин скелета для\linebreak
окончания и тремя
простыми цепями для раз\-ветвления~\cite{5gud, 2gud}. Тип $t_i\in \{0,\,1\}$ устанавливается
валентностью вершины скелета как вершины графа~\cite{2gud}, где 0~--- разветвление, а~1~---
 окончание. Координаты
  $(x_i,y_i)$, направление~$\alpha_i$ и тип~$t_i$ являются базовыми
параметрами~$M_i$, которые используются при идентификации изображения~\cite{6gud, 5gud}.

\subsection{Список топологических векторов} %2.2

Список топологических векторов для линий~$L_l$ находится на основе списка~$L_m$
по~(\ref{e2gud}) и всех вершин скелета~$F_0^{(m)}$, исключая вершины частных признаков, в
виде
\begin{equation}
L_l =\left \{ V_i =\left \{ (e_j, n_j, l_j)\right \}\vert i=\overline{1,\,n_2},\,j=\overline{1,\, m_t}\right
\}\,,
\label{e3gud}
\end{equation}
где $V_i$~--- топологический вектор для группы вершин скелета; $\vert L_l\vert =n_2$~---
мощность списка и $n_2>n_1$; $i$~--- индекс как номер топологического вектора; $j$~--- номер
связи в топологическом векторе; $e_j$~--- событие, а $l_j$~--- длина связи, сформированная
частным признаком с номером~$n_j$; $m_t =4m+2$~--- число связей с учетом центральной
линии.

Опишем процедуру синтеза списка.

В информативной области ДИ выделяют линии, строят скелет и детектируют два типа частных
признаков: окончания и разветвления (см.\ рис.~1)~\cite{5gud}.\linebreak\vspace*{-12pt} 
\begin{center} %fig2
\vspace*{1pt}
\mbox{%
\epsfxsize=74.97mm
\epsfbox{gud-2.eps}
}
%\end{center}
\vspace*{12pt}
{{\figurename~2}\ \ \small{Проекции для окончания и разветвления}}
\end{center}
\vspace*{6pt}


%\bigskip
\addtocounter{figure}{1}

\noindent
Направление частного
признака~$M_i$ указывает на область увеличения числа линий и параллельно касательной к
изображению папиллярной линии в малой его окрестности. Каждый частный признак нумеруют.
Затем от каждого частного признака фиксируют две проекции: вправо и влево от перпендикуляра
к его направлению на смежные скелетные линии. На рис.~2 проекции показаны
пунктиром, а две соответствующие вершины скелета на линиях~1 и~2 закрашены.


Выберем вершину скелета~$p_i$ (но не частный признак). Проведем через ее
координаты~$(x_i,y_i)$ вправо и влево сечение на глубину нескольких линий
перпендикулярно касательным к пе\-ре\-се\-ка\-емым линиям. Пронумеруем по спирали,
разворачивающейся по часовой стрелке, рассеченные линии, которые назовем связями. Сечение
проходит, отслеживая направление кривизны линий~\cite{3gud}. Одна линия в сечении образует
две связи. Топологический вектор определяют по сечению методом слежения за ходом каждой
связи от сечения до встречи с другим частным признаком, расположенным на связи, или с
проекцией от него на связи. На связях детектируют события, показанные на рис.~\ref{f3gud} и
представленные в двоичном коде:

\smallskip

%\noindent
1101~--- на связи проекция от окончания, расположенного справа по ходу связи, направление
окончания ориентировано навстречу ходу связи;\\[-9pt]

%\noindent
1001~--- на связи проекция от окончания, расположенного справа по ходу связи, направление
окончания ориентировано по ходу связи;\\[-9pt]

%\noindent
1110~--- на связи проекция от окончания, расположенного слева по ходу связи, направление
окончания ориентировано навстречу ходу связи;\\[-9pt]


%\noindent
1010~--- на связи проекция от окончания, расположенного слева по ходу связи, направление
окончания ориентировано по ходу связи;\\[-9pt]


%\noindent
0101~--- на связи проекция от разветвления, расположенного справа по ходу связи, направление
разветвления ориентировано навстречу ходу связи;\\[-9pt]


%\noindent
0001~--- на связи проекция от разветвления, расположенного справа по ходу связи, направление
разветвления ориентировано по ходу связи;\\[-9pt]

\begin{figure*} %fig3
\vspace*{1pt}
\begin{center}
\mbox{%
\epsfxsize=153.389mm
\epsfbox{gud-3.eps}
}
\end{center}
\vspace*{-9pt}
\Caption{События
\label{f3gud}}
\end{figure*}

%\noindent
0110~--- на связи проекция от разветвления, расположенного слева по ходу связи, направление
разветвления ориентировано навстречу ходу связи;\\[-9pt]

%\noindent
0010~--- на связи проекция от разветвления, расположенного слева по ходу связи, направление
разветвления ориентировано по ходу связи;\\[-9pt]

%\noindent
0011~--- на связи разветвление, направление которого ориентировано по ходу связи;\\[-9pt]

%\noindent
0111~--- разветвление на связи, образованной линией, касательная к которой образует
минимальный угол при повороте направления разветвления против часовой стрелки;\\[-9pt]

%\noindent
1011~--- разветвление на связи, образованной линией, касательная к которой образует
минимальный угол при повороте направления разветвления по часовой стрелке;\\[-9pt]

%\noindent
1111~--- на связи окончание, направление которого ориентировано навстречу ходу связи;\\[-9pt]

%\noindent
1100~--- связь по линии замыкается, частный признак или проекция от него отсутствует;\\[-9pt]

%\noindent
0000~--- на связи нет ни частного признака, ни проекции от него (обрыв скелета).

\smallskip


С событием как числом, детектированным на связи, ассоциируют номер частного признака,
инициирующего это событие. Событие привязано к номеру связи. Для событий~0000 и~1100
номера\linebreak частных признаков отсутствуют. Нумерованный набор связей с событиями и номерами
част\-ных приз\-на\-ков есть {\bfseries\textit{базовый топологический вектор}} (экономный).
Событие и номер частного признака образуют упорядоченную пару~$(e_j,n_j)$, которую
дополняют длиной связи от сечения до позиции, в которой детектируется это событие. Так
формируется {\bfseries\textit{расширенный топологический вектор}}. Событие, номер частного
признака и длина связи образуют упорядоченную тройку~$(e_j,n_j,l_j)$. Для событий~0000
и~1100 длины связей описывают информативные области без частных признаков. Длины связей,
обрывающихся на краю ДИ, устойчивы в том смысле, что не укорачиваются при полной
прокатке пальца~\cite{3gud}.

Местоположение бита в событии определяет тип частного признака, его направление и
местоположение по отношению к направлению хода связи и~др. (см.\ рис.~\ref{f3gud}).


Сечение разрезает линии на связи, пронумерованные по спирали, разворачивающейся по часовой
стрелке. На рис.~4 в сечении для вершины~$A$ скелета
 пронумерованы связи~0--17.
Топологический вектор вершины~$A$ представлен в виде табл.~1. На рис.~5
 в сечении для вершины~$B$ скелета пронумерованы\linebreak\vspace*{-12pt}
\begin{center} %fig4
\vspace*{9pt}
\mbox{%
\epsfxsize=72.915mm
\epsfbox{gud-4.eps}
}
%\end{center}
\vspace*{12pt}
{{\figurename~4}\ \ \small{Сечение для линии с окончанием}}
\end{center}
\pagebreak
%\vspace*{6pt}


%\bigskip
\addtocounter{figure}{1}


\noindent
\begin{center}
\vspace*{1pt}
%\begin{table*}\small
\parbox{60mm}{{{\tablename~1}\ \ \small{Топологический вектор для вершины~$A$}}}

%\label{t1gud}}
\vspace*{2ex}
%\begin{center}
{\small 
\tabcolsep=6pt
\begin{tabular}{|c|c|c|l|}
\hline
Номер&Событие&Индекс&\multicolumn{1}{c|}{Длина}\\
\hline
\hphantom{9}0&1110&22&\hspace*{3mm}$l_0$\\
\hphantom{9}1&1111&19&\hspace*{3mm}$l_1$\\
\hphantom{9}2&1110&19&\hspace*{3mm}$l_2$\\
\hphantom{9}3&1111&22&\hspace*{3mm}$l_3$\\
\hphantom{9}4&0001&21&\hspace*{3mm}$l_4$\\
\hphantom{9}5&1101&19&\hspace*{3mm}$l_5$\\
\hphantom{9}6&1010&24&\hspace*{3mm}$l_6$\\
\hphantom{9}7&0010&25&\hspace*{3mm}$l_7$\\
\hphantom{9}8&0011&21&\hspace*{3mm}$l_8$\\
\hphantom{9}9&1111&23&\hspace*{3mm}$l_9$\\
10&1010&26&\hspace*{3mm}$l_{10}$\\
11&0011&25&\hspace*{3mm}$l_{11}$\\
12&0010&21&\hspace*{3mm}$l_{12}$\\
13&1010&20&\hspace*{3mm}$l_{13}$\\
14&1111&27&\hspace*{3mm}$l_{14}$\\
15&0001&25&\hspace*{3mm}$l_{15}$\\
16&0000&---&\hspace*{3mm}---\\
17&1001&20&\hspace*{3mm}$l_{17}$\\
\hline
\end{tabular}
}
\end{center}
%\end{table*}

\bigskip
\addtocounter{table}{1}


\begin{center} %fig5
%\vspace*{9pt}
\mbox{%
\epsfxsize=74.915mm
\epsfbox{gud-5.eps}
}
%\end{center}
\vspace*{12pt}
{{\figurename~5}\ \ \small{Сечение для линии с разветвлением}}
\end{center}
%\vspace*{6pt}


%\bigskip
\addtocounter{figure}{1}

\noindent
 связи~0--17. Топологический вектор
вершины~$B$ представлен в виде табл.~2. Сечения изображены пунктиром. Рисунки
показывают {\bfseries\textit{мутацию}} окончания~19 в разветвление~19 (из-за
грязи)~\cite{3gud}. По сути вершины~$A$ и~$B$ скелета одни и те же. События, привязанные к
номерам связей, позволяют сопоставить топологию {\bfseries\textit{базовых топологических
векторов}}~\cite{3gud}.

Начало нумерации связей в сечении для вершин~$A$ и~$B$ (связи~№\,0) несущественно, так как
при развороте ДИ формируется зеркальное отоб\-ра\-же\-ние номеров связей в сечении, которое
распознается и учитывается при идентификации ДИ. При\linebreak\vspace*{-12pt}
\noindent
\begin{center}
\vspace*{1pt}
%\begin{table*}\small
\parbox{60mm}{{{\tablename~2}\ \ \small{Топологический вектор для вершины~$B$}}}

%\label{t1gud}}
\vspace*{2ex}
%\begin{center}
{\small 
\tabcolsep=6pt
\begin{tabular}{|c|c|c|l|}
\hline
Номер&Cобытие&Индекс&\multicolumn{1}{c|}{Длина}\\
\hline
\hphantom{9}0&1110&22&\hspace*{3mm}$l_0$\\
\hphantom{9}1&1011&19&\hspace*{3mm}$l_1$\\
\hphantom{9}2&0111&19&\hspace*{3mm}$l_2$\\
\hphantom{9}3&1111&22&\hspace*{3mm}$l_3$\\
\hphantom{9}4&0001&21&\hspace*{3mm}$l_4$\\
\hphantom{9}5&0101&19&\hspace*{3mm}$l_5$\\
\hphantom{9}6&0110&19&\hspace*{3mm}$l_6$\\
\hphantom{9}7&0010&25&\hspace*{3mm}$l_7$\\
\hphantom{9}8&0011&21&\hspace*{3mm}$l_8$\\
\hphantom{9}9&1111&23&\hspace*{3mm}$l_9$\\
10&1010&26&\hspace*{3mm}$l_{10}$\\
11&0011&25&\hspace*{3mm}$l_{11}$\\
12&0010&21&\hspace*{3mm}$l_{12}$\\
13&1010&20&\hspace*{3mm}$l_{13}$\\
14&1111&27&\hspace*{3mm}$l_{14}$\\
15&0001&25&\hspace*{3mm}$l_{15}$\\
16&0000&---&\hspace*{3mm}---\\
17&1001&20&$\hspace*{3mm}l_{17}$\\
\hline
\end{tabular}
}
\end{center}
%\end{table*}

\bigskip
\addtocounter{table}{1}

\noindent
 глубине сечения $m=4$\ для линии
формируется восемнадцать связей $m_t=18$.

Число топологических векторов конечно. На рис.~4 внизу двунаправленной
пунктирной стрелкой указана зона, умещающаяся между частными признаками~19 и~25, в
пределах которой для вершины~$A$ при смещении ее по скелету синтезируется один и тот же
{\bfseries\textit{базовый топологический вектор}}. Подобная зона указана для вершины~$B$
внизу на рис.~5. Топологические векторы с равными {\bfseries\textit{базовыми
топологическими векторами}} объединяют~\cite{3gud}. Их количество при этом на один--два
порядка уменьшается до величины $n_2 <1000$ по~(\ref{e3gud}) при $n_1<100$ по~(\ref{e2gud}).
Вектор~$V_i$ автоматически характеризует отрезок линии, а не частный признак. Поскольку на
содержание {\bfseries\textit{базового топологического вектора}} деформация изоб\-ра\-же\-ния
практически не влияет, вектор называют топологическим~\cite{6gud}.

На этом построение списка~(\ref{e3gud}) завершается.

Сопоставительный анализ~\cite{3gud} и~\cite{6gud} показывает, что~\cite{3gud} имеет ряд
преимуществ. \textit{Во-первых}, сечение строится по кривой, отслеживающей направление
кривизны линий. \textit{Во-вторых}, при вычислении событий используются проекции от
частных признаков, что при их мутациях предотвращает потерю информации. \textit{В-третьих},
нумерация связей разворачивается по спирали без пропуска связей, что позволяет наращивать
сечение с сохранением содержимого общей части укороченного и удлиненного векторов.
\textit{В-четвертых}, при объединении можно выбрать топологический вектор с максимальной
величиной минимальной длины связи и отодвинуть сечение от частных признаков.
\textit{В-пятых}, при объединении векторов происходит автоматическое разбиение линий на
отрезки, что обеспечивает более подробное описание ДИ в области петель, дельт, завитков и
участков повышенной кривизны линий. Это повышает устойчивость и информативность
математической модели.

\subsection{Список векторов гребневого счета} %2.3

В дактилоскопии гребневый счет является базой при доказательстве идентичности отпечатков
пальцев в суде~\cite{5gud}. Его предписывают измерять как количество линий, умещающихся на
прямой между двумя частными признаками. В электронных системах для одного частного
признака~$M_i$, как правило, определяют несколько подобных величин.

Согласно~\cite{5gud}, из списка~$L_m$ выбирают очередной частный признак~$M_i$ и
принимают его за центр вращения луча сканирования, начальное положение которого совпадает
с направлением $\theta_i \in M_i$ по~(\ref{e2gud}). Изображение сканируют, вращая луч
сканирования. При встрече луча с $M_k \in L_m$ формируют упорядоченную пару~$(r_j,n_j)$,
где $n_j=k$~--- номер частного признака~$M_k$ и $k\not= i$; $r_j$~--- гребневый счет
между~$M_i$ и~$M_k$; $j$~--- номер связи как число встреченных лучом частных признаков.
Количество таких пар не превышает числа $n_1-1$.

Эти операции повторяют $\forall M_i\in L_m$, $i=\overline{1,\,n_1}$. Геометрические
характеристики, связанные с упорядоченной парой и являющиеся производными от параметров
частных признаков и угла луча сканирования, для модели избыточны и вычисляются при
идентификации. В шаблонах число связей для~$M_i$, как правило, ограничивают, а связи
группируют по квадрантам или октантам ориентированной по $\theta_i \in M_i$ системы
координат~\cite{5gud}.

Основных недостатков гребневого счета, указанных в~[1, 2, 5, 6], четыре.
\textit{Во-первых}, величина~$r_j$ неустойчива, так как при мутации окончания в разветвление
или разветвления в окончание величина гребневого счета меняется. \textit{Во-вторых}, в области
петель, дельт, завитков и существенной кривизны линий гребневый счет недостоверен из-за
механизма измерения по прямой линии. \textit{В-третьих}, группировка величин гребневого
счета по квадрантам или октантам приводит к появлению пограничного эффекта из-за перехода
частных признаков через границы разбиения плоскости на области. \textit{В-четвертых}, выбор
частных признаков, для которых выполняется измерение гребневого счета, зависит от
деформации изображения. Это увеличивает ошибки идентификации.

Для преодоления указанных недостатков предлагается список векторов гребневого счета для
линий, который синтезируется на основе всех вершин скелета~$F_0^{(m)}$, исключая вершины
частных признаков, и списка частных признаков~$L_m$ в виде
\begin{equation}
L_r =\left \{ R_i=\left \{ (r_j,n_j)\right \}\vert i=\overline{1,\,n_3},\ j=\overline{1,\,n_4}\right \}\,,
\label{e4gud}
\end{equation}
где $R_i$~--- вектор гребневого счета для группы вершин скелета как упорядоченное по
индексу~$j$ множество упорядоченных пар~$(r_j,n_j)$; $\vert L_r\vert =n_3$~--- мощность
списка, $n_3>n_1$; $i$~--- индекс как номер вектора; $j$~--- номер связи в векторе; $n_4$~---
число связей в векторе, $n_4<n_1$; $r_j$~--- величина гребневого счета, а $n_j$~--- номер
частного признака по~(\ref{e2gud}) на $j$-й связи; $n_1=\vert L_m\vert$ по~(\ref{e2gud}).

Опишем процедуру синтеза списка.

В информативной области ДИ выделяют линии и строят скелет, по которому детектируют два
типа частных признаков: окончания и разветвления~\cite{5gud}. Направление частного признака
указывает на область увеличения числа линий и параллельно касательной к изображению
папиллярной линии в малой окрестности частного признака~$M_i$. Каж\-дый~$M_i$ нумеруют и
описывают координатами, направлением и типом по~(\ref{e2gud}).

Выберем вершину скелета~$p_g$ (но не частный признак) и примем ее за центр вращения луча
сканирования, начальное направление которого совпадает с направлением линии скелета.
Изображение сканируют, вращая луч сканирования. При встрече луча с $M_k\in L_m$
формируют упорядоченную пару~$(r_j,n_j)$, где $n_j=k$~--- номер частного признака~$M_k$;
$r_j$~--- гребневый счет между~$p_g$ и~$M_k$; $j$~--- номер связи как число встреченных
лучом частных признаков. Количество таких пар не превышает числа~$n_1$. В~результате
одного оборота луча формируется {\bfseries\textit{вектор гребневого счета}} $R_g
=\{(r_j,n_j)\vert j=\overline{1,\,n_1}\}$ как упорядоченное по индексу~$j$ множество
упорядоченных пар~$(r_j,n_j)$. Связи замыкаются по кольцу, и их можно перенумеровать, меняя
начальное направление луча сканирования, например $R_g^s
=\{(r_l,n_l)\vert l=\overline{1,\,n_1}\}$, где номер связи $l=(j+s)\mathrm{mod}\left (n_1+1\right )$.
В~этом случае для вершины~$p_g$ на основе вектора~$R_g$ можно с учетом замыкания связей
по кольцу синтезировать $\vert R_g^s \vert s=\overline{1,\,n_1}\vert =n_1$
{\bfseries\textit{эквивалентных векторов гребневого счета}}.
Для скелета формируется конечный набор векторов гребневого счета. {\bfseries\textit{Векторы
гребневого счета}}, для которых можно синтезировать одинаковые
{\bfseries\textit{эквивалентные векторы гребневого счета}}, объединяют~\cite{4gud}. Операция
объединения означает выбор, например случайным образом, одного из объединяемых векторов и
помещение его в список векторов гребневого счета $L_r =\{ R_i\vert i=\overline{1,\,n_3}\}$
по~(\ref{e4gud}). Число векторов при этом на один--два порядка уменьшается. Вектор~$R_i$ для
изображения уникален и автоматически характеризует отрезок\linebreak\vspace*{-12pt} 
\begin{center} %fig6
\vspace*{6pt}
\mbox{%
\epsfxsize=73.038mm
\epsfbox{gud-6.eps}
}
%\end{center}
\vspace*{6pt}
{{\figurename~6}\ \ \small{Гребневый счет линии}}
\end{center}
\vspace*{6pt}


%\bigskip
\addtocounter{figure}{1}

\noindent
линии, а не частный признак. На
этом построение списка~(\ref{e4gud}) завершается.

Очевидны следующие достоинства метода.

\textit{Во-первых}, гребневый счет на одном конце замыкается на вершину скелета, не
подверженную мутациям частных признаков, что повышает устойчивость величины гребневого
счета~$r_j$. \textit{Во-вторых}, линии в области петель, дельт, завитков и существенной
кривизны линий автоматически разбиваются векторами~$R_i$ на короткие отрезки, что
позволяет по-разному детализировать представление ДИ. Это уменьшает ошибки
идентификации. Чис\-ло связей в векторе можно ограничить, например, по длине луча или
плотности распределения частных признаков.

На рис.~6 вершина~$B$ выбрана за центр вращения луча сканирования~$BC$. При
вращении по часовой стрелке он поочередно встретится с частными признаками~3, 4, 6, 5, 1,~2.
Если гребневый счет определяется числом пересеченных линий вдоль путей, показанных на
рис.~6 длинным пунктиром, то для по\-сле\-до\-ва\-тель\-ности встреченных частных
признаков генерируются величины гребневого счета~0, 2, 2, 1, 3,~1. Это можно представить
упорядоченным множеством пар $\{(0,\,3)$, $(2,\,4)$, $(2,\,6), (1,\,5), (3,\,1), (1,\,2)\}$, которое для
вершины~$B$ образует вектор гребневого счета. При другом начальном положении луча
сканирования, например~$BE$ (см.\ рис.~6), определяется другое множество пар $\{(2,\,6),
(1,\,5), (3,\,1), (1,\,2), (0,\,3), (2,\,4)\}$, которое при замыкании в кольцо идентифицируется с
исходным множеством пар $\{(0,\,3)$, $(2,\,4)$, $(2,\,6)$, $(1,\,5)$, $(3,\,1)$, $(1,\,2)\}$. Действительно, для
вершины~$B$ существует 6~эквивалентных векторов гребневого счета.

При выборе вершины~$A$ с помощью луча сканирования~$AD$ (на рис.~6 показано его
начальное положение), который вращают по часовой стрелке, формируется упорядоченное
множество пар $\{(0,\,3), (2,\,4), (2,\,6), (1,\,5), (3,\,1), (1,\,2)\}$. Оно совпадает с одним из
эквивалентных векторов гребневого счета для вершины~$B$ (независимо от начального
направления луча~$AD$). Эти вычисленные векторы гребневого счета для вершины~$A$ и
вершины~$B$ объединяют.

\section{Заключение}

В работе предложены две математические модели ДИ на основе топологических векторов для
линий~(\ref{e3gud}) и векторов гребневого счета для линий~(\ref{e4gud}). Обе модели получены
операцией объединения векторов, которые строят для вершин скелета, исключая вершины
частных признаков. Операцию объединения выполняют для тех векторов, которые подобно
характеризуют некоторую область ДИ. Объединяемые векторы замещают вектором,
отражающим свойства скелетной линии или ее отрезка.

Несмотря на различный подход при по\-стро\-ении каждой из двух моделей, указывается их общее
свойство: модели ориентированы на описание линий, а не точек. Как следствие, такие модели
более полно описывают изображение по сравнению с моделями, предлагавшимися
ранее~[1, 2, 6]. Платой за это преимущество является повышенная дробность
отрезков линий, описываемых топологическими векторами и векторами гребневого счета, в
области значительной кривизны линий. Однако все это позволяет реализовать механизмы,
компенсирующие влияние мутаций частных признаков и деформации изображения на
математическую модель и повышающие ее устойчивость.

Дальнейшее направление развития видится в синтезе математических моделей, ориентированных
на описание областей изображения с помощью структур для нескольких линий. Такие
исследования помогут реализовать механизмы индексирования ДИ, многократно ускоряющие
процедуры идентификации изображений отпечатков пальцев.

{\small\frenchspacing
{%\baselineskip=10.8pt
\addcontentsline{toc}{section}{Литература}
\begin{thebibliography}{9}
\bibitem{6gud} %1
Patent  USA 5631971, Int. Cl. G~06~K~9/00. Vector based topological fingerprint matching~/
M.\,K.~Sparrow (Winchester).~--- Field: Jul. 15, 1994; Date of patent: May 20, 1997; US Cl.~382/125.
17~p.

\bibitem{5gud} %2
\Au{Maltoni D., Maio D., Jain A.\,K.}
Handbook of fingerprint recognition.~--- New York: Springer-Verlag, 2003.  348~p.

\bibitem{1gud} %3
\Au{Гонсалес Р., Вудс Р.}
Цифровая обработка изображений~/ Пер. с англ.~--- М.: Техносфера, 2006. 1072~c.

\bibitem{3gud} %4
Патент РФ~2321057, МПК G~06~K~9/52, A~61~B~5/117. Способ кодирования
отпечатка папиллярного узора~/ В.\,Ю.~Гудков.~--- №\,2006142831/09; заявл. 04.12.2006; опубл.
27.03.2008; Бюл. №\,9. 13~с.

\bibitem{4gud} %5
Патент РФ~2360286, МПК G~06~K~9/00. Способ кодирования отпечатка
папиллярного узора~/ В.\,Ю.~Гудков.~--- №\,2007118575/09; заявл. 18.05.2007; опубл. 27.06.2009;
Бюл. №\,18.  13~с.

\bibitem{7gud} %6
\Au{Sparrow M.\,K., Sparrow P.\,J.}
A topological approach to the matching of single fingerprints: Development of algorithms for use on
latent finger marks~// U.S.\ Dep. Comer. Nat. Bur. Stand. Spec. Pub., 1985. №\,500--126.  61~p.

\label{end\stat}

\bibitem{2gud} %7
\Au{Новиков Ф.\,А.}
Дискретная математика для программистов: Учебник для вузов. 3-е изд.~--- СПб.: Питер, 2008.
384~с.

 \end{thebibliography}
}
}

\end{multicols}