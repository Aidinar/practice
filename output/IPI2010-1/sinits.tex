
\def\tr{\,,\,\ldots\,,\,}
\def\rv{\right\vert\,}
\def\rrv{\right\vert}
\def\lv{\,\left\vert}
\def\rk{\,\right]}
\def\lk{\left[\,}
\def\rf{\right\}}
\def\lf{\left\{}


\def\vrp{\varphi}
\def\alp{\alpha}
\def\w{\omega}

\def\iinti{\int\limits_{-\infty}^{\infty}}
\def\mm{{\rm M}}

\def\stat{sinits}


\def\tit{ОПЕРАТИВНОЕ ПОСТРОЕНИЕ ИНФОРМАЦИОННЫХ МОДЕЛЕЙ ДВИЖЕНИЯ ПОЛЮСА ЗЕМЛИ
 МЕТОДАМИ ЛИНЕЙНЫХ И~ЛИНЕАРИЗОВАННЫХ ФИЛЬТРОВ$^*$}
\def\titkol{Оперативное построение информационных моделей движения полюса Земли}
% методами линейных и~линеаризованных фильтров}

\def\autkol{И.\,Н.~Синицын,  В.\,И.~Синицын,  Э.\,Р.~Корепанов, В.\,В.~Белоусов,
Н.\,Н.~Семендяев}
\def\aut{И.\,Н.~Синицын$^1$,  В.\,И.~Синицын$^2$,  Э.\,Р.~Корепанов$^3$, В.\,В.~Белоусов$^4$,
Н.\,Н.~Семендяев$^5$}

\titel{\tit}{\aut}{\autkol}{\titkol}

{\renewcommand{\thefootnote}{\fnsymbol{footnote}}\footnotetext[1]
{Работа выполнена при финансовой поддержке РФФИ
(проект №\,10-07-00021) и программы ОНИТ РАН <<Информационные
технологии и анализ сложных систем>> (проект~1.5).}}

\renewcommand{\thefootnote}{\arabic{footnote}}
\footnotetext[1]{Институт проблем информатики Российской академии наук, sinitsin@dol.ru}
\footnotetext[2]{Институт проблем информатики Российской академии наук, vsinitsin@ipiran.ru}
\footnotetext[3]{Институт проблем информатики Российской академии наук, ekorepanov@ipiran.ru}
\footnotetext[4]{Институт проблем информатики Российской академии наук, vbelousov@ipiran.ru}
\footnotetext[5]{Институт проблем информатики Российской академии наук, nsemendyaev@ipiran.ru}

\vspace*{12pt}

\Abst{Рассмотрены практические вопросы оперативной непрерывной и дискретной фильтрации 
(оценивания состояния и параметров) движения полюса Земли на
основе линейных и линеаризованных фильтров. Описано инструментальное алгоритмическое и 
программное обеспечение. Приведены тес\-то\-вые примеры, а также результаты вычислительных 
экспериментов по оперативной обработке данных астрометрических измерений.}

\KW{астрометрические измерения;
инструментальное алгоритмическое и программное обеспечение;
информационная модель; линейные и линеаризованные фильтры;
обобщенные фильтры Калмана; оперативная фильтрация; фильтры Калмана;
фильтры Калмана--Бьюси; фильтры Пугачёва; флуктуации полюса Земли}

      \vskip 30pt plus 9pt minus 6pt

      \thispagestyle{headings}

      \begin{multicols}{2}

      \label{st\stat}

\section{Введение}

В современной статистической информатике важное место занимают статистические методы оперативной 
обработки информации в стохастических системах (СтС) на основе неточных текущих астрометрических 
наблюдений путем использования оптимальных (по среднеквадратическому критерию) линейных фильтров 
(ЛФ) и линеаризованных нелинейных фильтров (ЛНФ)~[1--5]. Наиболее широкое распространение получили 
фильтры Калмана и Пугачёва~[2--4].

Непрерывный ЛФ Пугачёва (ЛФП) для линейных СтС обладает следующими достоинствами 
по сравнению с непрерывным ЛФ Калмана (ЛФК):
\begin{itemize}
\item
 ЛФП в отличие от ЛФК справедлив для линейных СтС, содержащих как переменные состояния, так и наблюдения;
\item
 при одинаковой точности (а для <<окрашенных>> шумов более простых в реализации) порядок ЛФП 
 может быть существенно меньше ЛФК.
 \end{itemize}

Дискретный ЛФК для дискретных линейных СтС в отличие  от дискретного ЛФП требует, чтобы наблюдения 
и оценивание проводились в один и тот же момент времени, что на практике трудно реализуемо.

В 2001--2009~гг.\ в рамках фундаментальных исследований по проблеме <<Статистическая динамика вращения Земли>> 
созданы информационные ресурсы (специализированные инструментальные\linebreak базы данных, стохастические дифференциальные 
и разностные уравнения флуктуаций полюса Земли, инструментальное алгоритмическое и  программное обеспечение в 
среде MATLAB для аналитического и статистического моделирования, фильтрации, прогноза и~др.). Соответствующая 
библиография работ приведена в~[6--9].

Как обобщение результатов~\cite{4sin, 5sin} дается изложение опыта создания ЛФ и ЛНФ 
для оперативной фильтрации (оценивания состояния и параметров) результатов наблюдений 
движения полюса Земли на интервале времени 1995--2010~гг.

\section{Уравнения флуктуаций полюса Земли и наблюдений}

Как известно~[6--9], на интервалах времени 3--5~лет уравнения флуктуаций полюса Земли пред\-став\-ля\-ют 
собой квазилинейную дифференциальную СтС следующего вида:
    \begin{align}
    \dot p + N_1 q + D_1 p &= M_1^* (t) + M_1^0(t) +\Delta M_1\,;\label{e1sin}\\
\dot q - N_2 p + D_2 q &= M_2^* (t) + M_2^0(t) +\Delta M_2\,.\label{e2sin}
\end{align}
Здесь $p$ и $q$~--- проекции мгновенной угловой ско\-рости вращения Земли на связанные с Землей
 оси $(p,q\ll r_*$, где $r_*$~--- осевая угловая скорость вращения Земли, принимаемая постоянной); 
 $N_i$ и  $D_i$ $(i=1,2)$~--- коэффициенты удельных моментов гироскопических и диссипативных сил;  
 $M_i^*(t)$ и $M_i^0(t)$ $(i=1,2)$~--- удельные регулярные и нерегулярные (флуктуирующие) возмущающие 
 моменты; $\Delta M_i=\Delta M_i (p,q,t;\pi)$ $(i=1,2)$~--- 
 квазилинейные возмущения, зависящие от~ $p,q,t$ и вектора па\-ра\-мет\-ров~$\pi$.
Переменные $X_1=p$ и $X_2 =q$  называются \textit{информационными переменными}.

Флуктуационные составляющие моментов  $M_i^0 (t)$ $(i=1,2)$ примем в виде случайных широкополосных 
и <<окрашенных>>  составляющих
\begin{gather}
M_i^0(t) = M_{i0}^0 (t) + \sum\limits_{h_1 =1}^H M_{ih_1}^0 (t)\,;\label{e3sin}\\
\dot M_{i0}^0 +\alp_{i0}^M M_{i0}^0 =\beta_{i0}^M V_i \enskip (i=1,2)\,;\label{e4sin}\\
\ddot M_{ih}^0 +2\eps_{ih}^M \w_{ih}^M \dot M_{ih}^0 +(\w_{ih}^M)^2 M_{ih}^0 ={}\notag\\
\hspace*{40mm}{}=\beta_{ih}^M V_i \enskip (i=1,2)\,.
\label{e5sin}
\end{gather}
Здесь $\alp_{i0}^M$, $\beta_{io}^M$ и~$\eps_{ih}^M$, $\w_{ih}^M$, $\beta_{ih}^M$ $(i=1,2)$, $h=$\linebreak 
$=h_1, h_2$~--- 
параметры соответственно широкополосных и  <<окрашивающих>> формирующих фильтров (ФФ); $V' =[V_1\ V_2]^{\mathrm{T}}$~--- 
вектор независимых белых шумов с нулевыми математическими ожиданиями и интенсивностями:
$$
G_t' = \mathrm{diag}\, \left(G_{1t}' G_{2t}'\right)\,;\enskip G_{1t}' = G_{11}(t)\,,\enskip G_{2t}' = G_{22}(t)\,.
$$

\smallskip
\noindent
\textbf{Замечание 1.} Во многих задачах статистической динамики Земли случайные  
широкополосные моменты~$M_{i0}^0 (t)$ принимают в виде независимых белых шумов, 
что существенно снижает размерность исходных уравнений статистической динамики.
Случайные возмущающие моменты~(\ref{e3sin})--(\ref{e5sin}) выражены через два скалярных независимых 
белых шума~$V_1$ и~$V_2$. Аналогично рассматривается более общий случай, когда можно выделить 
несколько независимых белых шумов, например~$H_1$ в~(\ref{e4sin}) и  $H_2$ в~(\ref{e5sin}).

\smallskip

Перейдем к рассмотрению уравнений наблюдения, при этом ограничимся только линейными относительно 
переменных наблюдениями. Обозна\-чим через~$Y_1$ и~$Y_2$ переменные, опи\-сы\-ва\-ющие наблю\-де\-ние 
информационных переменных~$X_1$ и~$X_2$.
Для непрерывных наблюдений общая линейная модель может быть представлена в ви\-де~\cite{4sin}
\begin{align} 
\dot Y_1 &= b_{11} Y_1 + b_{12} Y_2 + b_{1,11} X_1 + b_{1, 12} X_2 + U_1\,;\label{e6sin}\\
\dot Y_2 &= b_{21} Y_1 + b_{22} Y_2 + b_{1,21} X_1 + b_{1, 22} X_2 + U_2\,,\label{e7sin}
\end{align}
где $b_{ij}$ и $b_{1, ij}$ $(i,j=1,2)$~--- функции времени, удовлетворяющие известным условиям наблю\-да\-емости~\cite{4sin}; 
$U_1$ и~$U_2$~--- шумы в наблюдениях, в общем случае отличные от белого шума. Шумы~$U_1$ и~$U_2$ 
определяются уравнениями соответствующего ФФ для измерительного устройства.  В~случае одноканального 
линейного наблюдения уравнения~(\ref{e6sin}) и~(\ref{e7sin}) при $U_1 \equiv V_3$, $U_2 \equiv V_4$ имеют вид:
\begin{align}
\dot Y_1 &= b_{11} Y_1 + b_{1,11} X_1 + V_3\,;\label{e8sin}\\
\dot Y_2 &= b_{22} Y_2 + b_{1,22} X_2 +V_4\,.\label{e9sin}
\end{align}
Здесь $V''=[V_3\, V_4]^{\mathrm{T}}$, где шумы $V_3$ и~$V_4$  независимы между собой 
и  $V'=[V_1\,V_2]^{\mathrm{T}}$ --- белые шумы с интенсивностями
   \begin{gather*}
     G_t'' =\mathrm {diag}\, (G_{13t}'', G_{4t}'')\,;\\
     G_{3t}'' = G_{33} (t)\,;\qquad G_{4t}'' = G_{44}(t)\,.
     \end{gather*}

\smallskip

\noindent
\textbf{Замечание 2.} Если наблюдаются не только проекции угловой скорости~$p$ и~$q$, но и соответствующие 
квазиугловые переменные, то следует, во-первых, к уравнениям~(1), (2) добавить соответствующие 
инструментальные кинематические соотношения и рассматривать расширенный 4-мерный вектор информационных 
переменных $\bar X =[X_1\,X_2\,X_3\, X_4]^{\mathrm{T}}$ и, во-вторых, выписать уравнения 
ФФ для всех со\-став\-ля\-ющих расширенного вектора наблюдения~$\bar Y$.

\smallskip
 
Непрерывные ЛФ для оптимальной (по критерию минимума средней квадратической ошибки) оперативной обработки 
информации основаны на следующей общей записи уравнений состояния~(1), (2) и наблюдения~(\ref{e6sin}), 
(\ref{e7sin})~\cite{4sin}:
\begin{align}
\dot X_t &= a Y_t + a_1 X_t + a_0 +\psi V+\Delta \vrp +\Delta \psi V\,;\label{e10sin}\\
Z_t &=\dot Y_t = b Y_t + b_1 X_t + b_0 +\psi_1 V\,.\label{e11sin}
\end{align}
Здесь $X_t$~--- расширенный (посредством добавления к~ $X_{1t}, X_{2t}$ инструментальных переменных ФФ) 
вектор состояния; $Y_t$~--- расширенный (за счет инструментальных переменных ФФ) вектор наблюдения; 
$V=[V^{'\mathrm{T}} V^{''\mathrm{T}}]^{\mathrm{T}}$~--- расширенный вектор независимых белых шумов с матрицей интенсивностей 
$G_t^V =\mathrm{diag}\,(G_{1t} G_{2t}, G_{3t} G_{4t})$; 
$a$, $a_1$, $b$, $b_1$, $\psi$, $\psi_1$~--- постоянные матрицы соответствующей размерности; 
$a_0$, $b_0$~--- известные функции времени; 
$\Delta\vrp =\Delta\vrp (X_t, Y_t ;\pi)$, $\Delta\psi =\Delta \psi (X_t, Y_t, t;\pi)$~--- квазилинейные составляющие.

Уравнения~(\ref{e10sin}) и~(\ref{e11sin}) для независимых белых шумов в уравнениях состояния и наблюдения имеют вид:
\begin{align}
\dot X_t &= a Y_t + a_1 X_t + a_0 +\psi' V_1 +\Delta\vrp +\Delta\psi' V_1\,;\label{e12sin}\\
Z_t &=\dot Y_t = b Y_t + b_1 X_t + b_0 + V_2\,,\label{e13sin}
\end{align}
где $\Delta \vrp= \Delta \vrp (X_t, t;\pi)$ и $\Delta \psi' =\Delta \psi' (X_t, t;\pi)$~--- 
квазинелинейные составляющие, отвечающие $\Delta M_i (p,q,t;\pi)$ $(i=1,2)$.

Для дискретных линейных наблюдений искомые дискретные уравнения состояния и наблюдений, 
соответствующие~(\ref{e10sin}), (\ref{e11sin}) и~(\ref{e12sin}), (\ref{e13sin}), имеют следующий вид~\cite{4sin}:
\begin{align}
X_{k+1} &= a_k X_k + a_{0,k} +\psi_k V_k +\Delta \vrp_k +\Delta \psi_k V_k\,;\label{e14sin}\\
Y_k &= b_k X_k + b_{0,k} +\psi_{1,k}V_k\,;\label{e15sin}\\
X_{k+1} &={}\notag \\
&\!\!\!\!\!\!\!\!{}=a_k X_k + a_{0,k} +\psi_k' V_{1,k} +\Delta \vrp_k +\Delta \psi_k' V_{1,k}\,;\label{e16sin}\\
Y_k &= b_k X_k + b_{0,k} +V_{2,k}\,.\label{e17sin}
\end{align}
Здесь через $X_k$, $Y_k$, $V_k$, $V_{1,k}$, $V_{2,k}$, $a_{0,k}$ и $b_{0,k}$ 
обозначены значения переменных и параметров в моменты времени $k=1,2,\ldots$; 
$G_k^d$, $G_{1,k}^d$, $G_{2,k}^d$~--- ковариационные матрицы дискретных белых шумов~$V_k$, $V_{1,k}$ и $V_{2,k}$. 
При этом дискретные уравнения ФФ~(\ref{e4sin}) и~(\ref{e5sin}) будут иметь вид:
\begin{align}
M_{i0, k+1}^0 &= \left(1-\alp_{i0}^{Md}\right) M_{i0,k}^0 + \beta_{i0}^{Md} V_{i,k}\,,\label{e18sin}\\
M_{ih, k+2}^0 &={}\notag\\
&\!\!\!\!\!\!\!\!\!\!\!\!\!\!\!\!\!\!\!\!\!\!{}= -2\eps_{ih}^{Md} M_{ih, k+1}^0 -\left(\w_{ih}^{Md}\right)^2 M_{ih,k}^o +\beta_{i0}^{Md} V_{i,k}\,,\label{e19sin}
\end{align}
где верхним индексом~$d$ отмечается дискретный вариант соответствующего па\-ра\-мет\-ра~ФФ.

\smallskip

\noindent
\textbf{Замечание 3.} Следуя~\cite{4sin}, подчеркнем, что уравнения~(\ref{e10sin})--(\ref{e13sin}) 
и~(\ref{e14sin})--(\ref{e19sin}) справедливы в общем случае как для гауссовых, так и для
негауссовых белых шумов.
 \smallskip

Наконец, отметим, что вопросы оперативной экстраполяции (прогноза) и интерполяции требуют 
специального инструментального программного обеспечения на основе алгоритмов~\cite{4sin, 5sin} 
и будут рассмотрены в отдельной статье.

\section{Инструментальное алгоритмическое и~программное обеспечение непрерывной линейной и~квазилинейной фильтрации}

В основе непрерывной линейной и квазилинейной фильтрации лежат следующие утверждения~\cite{3sin, 4sin}:

\noindent
\textbf{Утверждение 1.}
\textit{Пусть векторный случайный процесс\linebreak  
$\lk X_t^{\mathrm{T}} Y_t^{\mathrm{T}}\rk^{\mathrm{T}}$ удовлетворяет линейным 
стохастическим дифференциальным уравнениям}~(\ref{e10sin}) \textit{и}~(\ref{e11sin})  
($\Delta\vrp=\Delta\psi=0$), \textit{а диффузионная матрица  $\si_1 =$\linebreak $=\psi_1 G_t\psi_1^{\mathrm{T}}$ невырождена. 
Тогда фильтрационные уравнения имеют следующий вид:}
\begin{align}
\dot{\hat X}_t &= a Y_t + a_1 \hat X_t + a_0 +{}\notag\\[2pt]
&\hspace{10mm}{}+\beta \lk \dot Y_t -(bY_t+ b_1 \hat X_t + b_0)\rk\,;\label{e20sin}\\[2pt]
\beta &= \fr{(R_t b_1^{\mathrm{T}} + \psi G_t \psi_1^{\mathrm{T}})}{(\psi_1 G_t \psi_1^{\mathrm{T}})}\,;\notag\\[2pt] %\label{e21sin}\\
\dot R_t &= a_1 R_t + R_t a_1^{\mathrm{T}} +\psi G_t \psi^{\mathrm{T}} - {}\notag\\[2pt]
&\hspace*{5mm}{}-\fr{\left(R_t b_1^{\mathrm{T}} +\psi G_t\psi_1^{\mathrm{T}}\right)
\left(b_1R_t +\psi_1G_t \psi^{\mathrm{T}}\right)}{\left(\psi_1 G_t \psi_1^{\mathrm{T}}\right)}\label{e22sin}
\end{align}
\textit{при начальных условиях}
\begin{align*}
\hat X(t_0) &=\hat X_0\,;\\
R(t_0)&=R_0 =\mm [X_0\vert Y_0]={}\\
&\hspace{10mm}{}=\mm \lk (X_0 - \hat X_0)(X_0^{\mathrm{T}} -\hat X_0^{\mathrm{T}})\vert Y_0\rk\,.
\end{align*}
\textit{Здесь $\hat X_t$~--- оптимальная средняя квадратическая оценка ошибки фильтрации переменной~$X_t$; 
$\beta$~--- мат\-рич\-ный коэффициент усиления; $R_t$~--- ковариационная матрица ошибки фильтрации; 
$G_t$~---  мат\-ри\-ца интенсивностей составляющих векторного  белого шума~$V$.}

\smallskip

Порядок ЛФ равен  $Q_{\mathrm{ЛФ}} =n_x (n_x+3)/2$. Уравнение~(\ref{e22sin}) 
не содержит результатов наблюдений. Это дает возможность определять~$R_t$ и~$\beta$ заранее, 
до получения результатов. Нелинейное уравнение Риккати~(\ref{e22sin}) явно не содержит матричных па\-ра\-мет\-ров~$a$ 
и~$b$. Таким образом, определение~$X_t$ сводится к интегрированию~(\ref{e20sin}) 
оперативно в темпе получения результатов наблюдений.

\medskip

\noindent
\textbf{Утверждение 2.}
\textit{Пусть в линейных стохастических дифференциальных уравнениях}~(\ref{e12sin}) \textit{и} (\ref{e13sin})
 ($\Delta\vrp =$\linebreak $=\Delta\psi=0$)  \textit{матрица~$G_{2t}$ интенсивностей шумов в наблюдениях невырождена. 
 Тогда непрерывный линейный фильтр Калмана--Бьюси (ЛФКБ) определяется следующими уравнениями}~\cite{4sin}:
\begin{align}
\!\!\!\dot{\hat X}_t&= a_1 \hat X_t + a_0 +\beta (Z_t - b_1 \hat X_t - b_0)\,;\label{e24sin}\\
\!\!\!\beta &= \fr{R_t b_1^{\mathrm{T}}}{G_{2t}}\,; \label{e25sin}\\
\!\!\!\dot R_t &= a_1 R_t + R_t a_1^{\mathrm{T}} +\psi' G_{1t} \psi^{'\mathrm{T}}-\beta G_{2t} \beta^{\mathrm{T}}\,,\!\\
\!\!\!\beta G_{2t} \beta^{\mathrm{T}} &= \fr{R_t b_1^{\mathrm{T}}}{G_{2t}} b_1 R_t = \beta b_1 R_t\,.
\end{align}

\smallskip

\noindent
\textbf{Замечание 4.} 
 Из~(22)--(25) следуют три важных вывода. Во-первых, если  $R_t$ <<велико>> 
 (в смысле больших собственных значений), то обновляющая разность  $Z_t - b_1 \hat X_t$ <<сильно>> взвешена 
 (т.\,е.\ текущая оценка~$\hat X_t$ известна с большой погрешностью). Тогда новая информация, 
 получаемая из наблюдений, становится очень значимой. Во-вторых, если $R_t$ <<мало>> 
 (т.\,е.\ наблюдения проводятся очень точно), тогда измерительная информация  <<сильно>> 
 взвешивается. Коэффициент~$\beta$ может рас\-смат\-ри\-вать\-ся как отношение сигнал/шум. В третьих, 
 в~(24) первые два члена определяют ста\-би\-ли\-зи\-ру\-ющие свойства системы, третий член 
 характеризует воздействие собственного шума в системе, а четвертый член выражает оставшуюся 
 ковариационную погрешность вследствие использования данных измерения.  Практически надо 
 стремиться снижать эту погрешность.

\smallskip

\noindent
\textbf{Замечание 5.} 
Порядок уравнений ЛФКБ равен  $Q_{\mathrm{КБ}}=5$. 
Для широкополосных возмущений  $M_{i0}^0(t)$ в~(\ref{e3sin}) и~(\ref{e4sin}), 
удовлетворяющих ФФ~(\ref{e5sin}),  $n_x=4$ и порядок будет равен $Q_{\mathrm{КБ}}=14$. 
Для <<окрашенных>> возмущений $M_{1h_1}^0 (t)$ и~$M_{2h_2}^0(t)$, удовлетворяющих~(\ref{e6sin}),  
$n_x=6$ и $Q_{\mathrm{КБ}}=54$. В~этих случаях с целью снижения порядка уравнений ЛФКБ 
можно заменить случайные процессы~$M_{i0}^0(t)$ и~$M_{ih}^0(t)$ в~(\ref{e3sin}) эквивалентным белым шумом 
с интенсивностью~$G^{\mathrm{Э}}(t)$ и интервалом корреляции~$\tau_{\mathrm{кор}}$~\cite{4sin}
    \begin{align*}
    G^{\mathrm{Э}}(t) &=\iinti K(t, t+\tau)\,d\tau\,;\\
    \tau_{\mathrm{кор}} &=\fr{1}{2}\, \max\limits_t \lf \iinti \fr{K(t,t+\tau)}{K(t,t)}\,d\tau\rf\,.
    \end{align*}
Здесь $K(t, t+\tau)$~--- ковариационная функция  процесса. Правомерность такого приема 
требуется проверять статистическим моделированием.


\medskip

\noindent
\textbf{Утверждение 3.}
\textit{Пусть нелинейная функция $\Delta\vrp$ в}~(12) \textit{статистически линеаризуема, т.\,е.
\begin{equation*}
\Delta\vrp \approx\Delta\vrp_{00} +\Delta\vrp_{01} X_t^0\,.
%\label{e27sin}
\end{equation*}
Здесь $\Delta\vrp_{00}=\Delta\vrp_{00} (m_t^x, K_t^x, t)$  и 
$\Delta\vrp_{01}=$\linebreak $=\Delta\vrp_{01}(m_t^x, K_t^x, t)$~--- коэффициенты статистической линеаризации нелинейной 
функции~$\Delta\vrp$ ($m_t^x$ и $K_t^x$~--- соответственно математическое ожидание и ковариационная матрица~$X_t$); 
функция~$\Delta\psi$ в}~(\ref{e13sin}) \textit{от~$X_t$ не зависит, 
$\Delta\vrp=\Delta\psi(t)$. Тогда линеаризованный ЛФКБ определяется следующими уравнениями}~\cite{4sin}:

\noindent
\begin{multline*}
\dot{\hat X}_t =\vrp_{00} -\vrp_{01} m_t^x + \vrp_{01} \hat X_t +{}\\
{}+\beta ( Z_t - b_1 \hat X_t - b_0 + b_1 m_t^x)\,; %\label{e28sin}
\end{multline*}
\begin{align*}
\dot R_t &=\vrp_{01} R_t + R_t \vrp_{01}^{\mathrm{T}} -\beta G_{2t} \beta^{\mathrm{T}}
+ \bar\psi' G_{1t}\bar\psi^{'\mathrm{T}}\,; %\label{e29sin}
\\[3pt]
\beta &= \fr{R_t b_1^{\mathrm{T}} }{G_{2t}}\,, %\label{e30sin}
\end{align*}
\textit{причем  $m_t^x$, $K_t^x$ находятся из уравнений}
\begin{align*}
\dot m_t^x &=\vrp_{00}\,; %\label{e31sin}
\\[3pt]
\dot K_t^x &=\vrp_{01} K_t^x + K_t^x \vrp_{01}^{\mathrm{T}} +\bar\psi' G_{1t}\bar\psi^{'\mathrm{T}}\,. %\label{e32sin}
\end{align*}
\textit{Здесь введены следующие обозначения:}
\begin{align*}
\vrp_{00} &=\vrp_{00}(m_t^x, K_t^x, t) = a_0 +\Delta\vrp_{00}(m_t^x, K_t^x, t)\,;\\[3pt]
\vrp_{01} &=\vrp_{01}(m_t^x, K_t^x, t) = a_1 +\Delta\vrp_{01}(m_t^x, K_t^x, t)\,;\\[3pt]
\bar\psi' &=\psi'+\Delta\psi(t)\,.
%\label{e33sin}
\end{align*}

\smallskip

\noindent
\textbf{Замечание 6.} Наряду со статистической линеаризацией~$\Delta \vrp$  в~(\ref{e12sin}) 
может быть использована другая эквивалентная линеаризация~\cite{4sin}, 
определяемая спецификой задачи~\cite{6sin, 8sin, 9sin}.

\medskip

\noindent
\textbf{Утверждение 4.}
\textit{В условиях утверждения~3, когда $\Delta\psi$ в}~(\ref{e13sin}) \textit{зависит от~$X_t$ и~$\pi$, используются 
различные методы нелинейной приближенной (субоптимальной) фильтрации}~\cite{4sin}. \textit{Первая группа методов основана на 
параметризации апостериорного распределения методами моментов, квазимоментов, семиинвариантов, коэффициентов 
ортогонального разложения апостериорной плотности и~др. Простейшим методом служит метод нормальной (гауссовой) 
аппроксимации. Ко второй группе методов относятся обобщенный ФКБ, фильтры второго порядка, гауссов фильтр и~др., 
использующие тейлоровское разложение различного порядка~$X_t$ в окрестности оценки~$\hat X_t$. 
Порядок таких фильтров существенно растет с ростом порядка аппроксимации. При этом для оперативной 
обработки информации требуются специализированные высокопроизводительные вычислительные средства}~[3--5].

\medskip

\noindent
\textbf{Утверждение 5.}
\textit{Пусть векторный случайных процесс  $\lk X_t^{\mathrm{T}} Y_t^{\mathrm{T}}\rk^{\mathrm{T}}$ 
удовлетворяет стохастическим дифференциальным 
уравнениям}~(\ref{e10sin}) \textit{и}~(\ref{e11sin}). \textit{Тогда фильтр Пугачёва (ФП) определяется следующими уравнениями}~\cite{3sin, 4sin}:
\begin{equation*}
\dot{\hat X}_t =\alp\xi (\hat X_t , Y_t,t) + \beta\eta (\hat X_t , Y_t,t)\dot Y_t +\gamma\,.
%\label{e34sin}
\end{equation*}
Здесь  $\xi=\xi (\hat X_t , Y_t,t)$ и $\eta=\eta(\hat X_t , Y_t,t)$~--- известные структурные функции ФП;  
$\alp$, $\beta$ и~$\gamma$~--- коэффициенты ФП, удовлетворяющие уравнениям
\begin{equation*}
\alp m_1 +\beta m_2 +\gamma =m_0\,,
%\label{e35sin}
\end{equation*}
где 
\pagebreak

\noindent
    \begin{equation}
    \left.
    \begin{array}{rl}
    m_0 &=\mm \vrp (X_t, Y_t, t)\,, \\[6pt]
    m_1 &=\mm \xi(Y_t, \hat X_t,  t)\,,\\[6pt] 
    m_2 &=\mm \eta (Y_t, \hat X_t, t)\,,
    \end{array}
    \right \}
    \label{e36sin}
    \end{equation}
    
    \vspace*{3pt}
    \noindent
\textit{$\mm$~--- символ математического ожидания;}

\vspace*{3pt}

\noindent
\begin{equation*}
\beta=\fr{\kappa_{02}}{\kappa_{22}}\,,
%\label{e37sin}
\end{equation*}
\textit{где}
    \begin{multline*}
    \kappa_{02} =\mm (X_t -\hat X_t) \vrp_1 (X_t, Y_t, t)^{\mathrm{T}} \eta (Y_t, \hat X_t, t)+{}\\[3pt]
    {}+
    \mm \bar \psi (X_t, Y_t, t) G_t (X_t, Y_t, t)\eta (Y_t, \hat X_t, t)\,,
    \end{multline*}
    
    \vspace*{-6pt}
    
    \noindent
\begin{multline}
\kappa_{22} =\mm \eta (Y_t, \hat X_t, t) \psi_1(X_t, Y_t, t) G_t \psi_1(X_t, Y_t, t)^{\mathrm{T}} \times{}\\[3pt]
{}\times \eta (Y_t, \hat X_t, t)\,;
\label{e38sin}
\end{multline}

\vspace*{-6pt}

    \noindent
    \begin{equation*}
    \alp =\fr{(\kappa_{01} -\beta \kappa_{21})}{\kappa_{11}}\,,
%    \label{e39sin}
    \end{equation*}
\textit{где}
\begin{equation}
\left.
\begin{array}{rl}
\kappa_{11}&= \mm \lk \xi (Y_t, \hat X_t, t) -m_1\rk \xi (Y_t, \hat X_t, t)\,,\\[6pt]
\kappa_{21} &=\mm \lk \vrp_1 (X_t, Y_t, t)-m_2\rk \xi (Y_t, \hat X_t, t)\,,\\[6pt]
\kappa_{01} &=\mm \lk \vrp (X_t, Y_t, t) -m_0\rk  \xi (Y_t, \hat X_t, t)\,,
\end{array}
\right \}
\label{e40sin}
\end{equation}

\noindent
\begin{multline*}
\dot R_t =\mm \big[ (X_t -\hat X_t)\vrp (X_t, Y_t, t)^{\mathrm{T}} +{}\\[3pt]
{}+\vrp (X_t, Y_t, t,\pi) (X_t -\hat X_t)^{\mathrm{T}}-{}\\[3pt]
{}-\beta\eta (Y_t,\hat X_t, t) \psi_1 (X_t, Y_t, t) G_t \psi_1 (X_t, Y_t, t)^{\mathrm{T}} \times{}\\[3pt]
\!\!\!{}\times \eta (Y_t,\hat X_t,  t^{\mathrm{T}})\beta^{\mathrm{T}}
+\bar\psi (X_t, Y_t, t)G_t \bar \psi (X_t, Y_t, t)^{\mathrm{T}}\big ].
%\label{e41sin}
\end{multline*}
\textit{В}~(24)--(26) \textit{приняты обозначения:}
\begin{align*}
\vrp (X_t, Y_t, t) &= a Y_t + a_1 X_t + a_0 + \Delta\vrp (X_t, Y_t, t)\,;\\[3pt]
\vrp_1 (X_t, Y_t, t)&= bY_t + b_1 X_t + b_0\,;\\[3pt]
\bar\psi (X_t, Y_t, t)&=\psi(t) +\Delta \psi (X_t, Y_t, t)\,;\\[3pt]
\psi_1 (X_t, Y_t, t)&=\psi_1(t)\,.
%\label{e42sin}
\end{align*}

Для вычисления математических ожиданий~$m_0$, $m_1$, $m_2$, $\kappa_{02}$,
$\kappa_{22}$, $\kappa_{11}$, $\kappa_{21}$ и $\kappa_{01}$  
необходимо вычисление совместного распределения вектора~$\lk Y_t^{\mathrm{T}} X_t^{\mathrm{T}} \hat X_t^{\mathrm{T}}\rk^{\mathrm{T}}$ 
путем решения соответствующей задачи статистического анализа~\cite{4sin}.


\bigskip

\noindent
\textbf{Замечание 7.} 
Для случая линейной СтС~(\ref{e10sin}) и~(\ref{e11sin})  
$(\Delta \vrp =0$, $\Delta\psi=0)$ уравнения ЛФП совпадают с уравнениями утверждения~1.

\bigskip

\noindent
\textbf{Замечание 8.} 
Если входящий в уравнения~(\ref{e10sin}) и~(\ref{e11sin}) векторный параметр~$\pi$ 
подлежит идентификации, то следует расширить вектор состояния~$X_t$, 
$\bar X_t =\lk X_t^{\mathrm{T}} \pi^{\mathrm{T}}\rk^{\mathrm{T}}$ и использовать уравнения $\dot \pi =0$.


\section{Инструментальное алгоритмическое и~программное обеспечение дискретной линейной и~квазилинейной фильтрации}

Следуя~\cite{3sin, 4sin}, приведем утверждения, лежащие в основе дискретной линейной и квазилинейной фильтрации.

\medskip

\noindent
\textbf{Утверждение 6.} 
\textit{Пусть векторный дискретный случайный процесс $\lk X_k^{\mathrm{T}} Y_k^{\mathrm{T}}\rk^{\mathrm{T}}$ 
удовлетворяет дискретным 
уравнениям}~(\ref{e16sin}) \textit{и}~(\ref{e17sin}), \textit{причем матрица интенсивностей шумов в наблюдениях~$G_{2,k}^d$ 
невырождена. Тогда дискретный ЛФК определяется рекуррентными уравнениями}~\cite{4sin}:
\begin{align*}
\hat X_{k+1} &=\hat X_{k+1\vert k} +\beta_{k+1} (Y_{k+1} -b_{1,k+1} \hat X_{k+1\vert k})\,; %\label{e43sin}
\\[1pt]
\hat X_{k+1\vert k} &= a_{1,k} \hat X_k + a_{0,k}\,;%\label{e44sin}
\\[1pt]
\beta_{k+1} &= \fr{R_{ k+1\vert k} b_{1, k+1}^{\mathrm{T}}}{\left( b_{1, k+1} R_{k+1\vert k} b_{1,k+1}^T G_{2,k+1}\right)}\,;
%\label{e45sin}
\\[1pt]
R_{k+1\vert k }&= a_{1,k} R_k a_{1,k}^{\mathrm{T}} +\psi_k' G_{1,k}\psi_k^{'T}\,; %\label{e46sin}
\\[1pt]
 R_{k+1} &= R_{k+1\vert k} -\beta_{k+1} b_{1, k+1} R_{k+1\vert k} = R_{k+1\vert k}-{}\\[3pt]
&\hspace*{10mm}{} - \fr{R_{k+1\vert k} b_{1, k+1}^{\mathrm{T}} 
 b_{1,k+1} R_{k+1\vert k}}{\left( b_{1,k+1} R_{k+1\vert k} b_{1,k+1}^T + G_{2, k+1}\right)}\,.
%\label{e47sin}
\end{align*}

\smallskip

\noindent
\textbf{Утверждение 7.}
\textit{Пусть векторный дискретный случайный процесс~ $\lk X_k^{\mathrm{T}} Y_k^{\mathrm{T}}\rk^{\mathrm{T}}$ определяется 
линейными стохастическими дискретными уравнениями}~(\ref{e14sin}) \textit{и}~(\ref{e15sin})  
\textit{$(\Delta \vrp_k=0$, $\Delta \psi_k=0)$. Тогда в основе алгоритмов дискретного 
ЛФП лежат следующие рекуррентные уравнения}~\cite{4sin}:
\begin{align*}
\hat X_{k+1} &=a_k \hat X_k + a_{0,k} +\beta_k (Y_k - b_k \hat X_k - b_{0,k})\,; %\label{e48sin}
\\[1pt]
\beta_k &= \fr{(a_k R_k b_k^{\mathrm{T}} +\psi_k G_k \psi_{1,k}^{\mathrm{T}})}{ 
(b_k R_k b_k^{\mathrm{T}} + \psi_k G_k \psi_{1,k}^{\mathrm{T}})}\,; %\label{e49sin}
\\[1pt]
R_{k+1} &= (a_k -\beta_k b_k) R_k a_k^{\mathrm{T}} + (\psi_k - \beta_k \psi_{1,k}) G_k \psi_k^{\mathrm{T}}\,. %\label{e50sin}
\end{align*}

\smallskip

\noindent
\textbf{Утверждение 8.}
\textit{Пусть дискретный случайный процесс~$\lk X_t^{\mathrm{T}} Y_t^{\mathrm{T}}\rk^{\mathrm{T}}$ удовлетворяет стохастическим разностным 
уравнениям}~(\ref{e16sin}) \textit{и}~(\ref{e17sin}). \textit{Тогда уравнения дискретного ФП определяются следующими 
уравнениями}~\cite{3sin, 4sin}:
    \begin{equation}
    \hat X_{k+1} =\alp_k \xi_k (\hat X_k) +\beta_k \eta_k (\hat X_k) Y_k +\gamma_k\,.\label{e51sin}
    \end{equation}
\textit{Здесь $\xi_k =\xi(k, \hat X_k)$ и $\eta_k=\eta(k, \hat X_k)$~--- 
известные структурные функции ФП; $\alp_k$, $\beta_k$ и~$\gamma_k$ определяются из следующей системы линейных уравнений:}
\begin{gather*}
\alp_k \kappa_{11} +\beta_k \kappa_{21} =\kappa_{01}\,; %\label{e52sin}
\\
\alp_k \kappa_{12} +\beta_k \kappa_{22} =\kappa_{02}\,; %\label{e53sin}
\\
\gamma_k = m_{0,k} -\alp_k m_{1,k} -\beta_k m_{2,k}\,, %\label{e54sin}
\end{gather*}
где 
    \begin{align*}
 \kappa_{01}&=\mm_{\cal N} \lk \vrp_k (X_k) - m_{0,k}\rk \xi_k (\hat X_k)^{\mathrm{T}}\,;\\
%    \label{e55sin}
%    \end{equation*}
 %    \vspace*{-12pt}
  %  
   % \noindent
%\begin{multline*}
\kappa_{02}&=\mm_{\cal N} \lk \vrp_k (X_k) - m_{0,k}\rk \vrp_{1,k} (\hat X_k)^{\mathrm{T}} \eta_k (\hat X_k)^{\mathrm{T}}+{}\\
&\hspace*{8mm}{}+        \mm_{\cal N} \lk \bar \psi_k' (X_k) G_{1,k} \bar \psi_k' (X_k)^{\mathrm{T}} \eta_k (\hat X_k) \rk\,;\\ %\label{e56sin}
%\end{multline*}
%    \vspace*{-12pt}
%\noindent
%\begin{equation*}
\kappa_{11}&=\mm_{\cal N} \lk \xi_k (\hat X_k) - m_{1,k}\rk \xi_k (\hat X_k)^{\mathrm{T}}\,;\\%\label{e57sin}
%\end{equation*}
%    \vspace*{-12pt}
 %  
%\noindent
%\begin{multline*}
\kappa_{12}&=\kappa_{21}^{\mathrm{T}}={}\\
&{}=\mm_{\cal N} \lk \xi_k (\hat X_k) - m_{1,k}\rk \vrp_{1,k}(X_k)^{\mathrm{T}} \eta_k 
(\hat X_k)^{\mathrm{T}}\,;\\ %\label{e58sin}
%\end{multline*}
%    \vspace*{-12pt}
 % 
%\noindent
%\begin{multline*}
\kappa_{22}&=
\mm_{\cal N} \lk \eta_k (\hat X_k) \vrp_{1,k}(X_k) - m_{2,k}\rk\times{}\\
&\hspace*{8mm}{}\times \vrp_{1,k}(X_k)^{\mathrm{T}} \eta_k (\hat X_k)^{\mathrm{T}}+{}\\
&\hspace*{-10.35pt}{}+\mm_{\cal N} \lk \eta_k (\hat X_k) \psi_{1,k}(X_k) G_{2,k} \psi_{1,k} (X_k)^{\mathrm{T}} \eta_k 
(\hat X_k)^{\mathrm{T}}\rk\,;\\ %\label{e59sin}
%\end{multline*}
%    \vspace*{-12pt}
 %\noindent
%\begin{align*}  
m_{0,k} &= \mm_{\cal N} \vrp_k (X_k)\,;\\
m_{1,k} &=\mm_{\cal N} \xi_k (\hat X_k)\,;\\
m_{2,k} &=\mm_{\cal N} \eta_k (\hat X_k)\,;\\
%\end{align*}
%\vspace*{-12pt}
%\noindent
%\begin{align*}
\vrp_k &= a_k X_k + a_{0,k} +\Delta\vrp_k (X_k)\,;\\
\vrp_{1,k} &= b_k X_k +b_{0,k}\,;\\
\bar \psi_k' &=\psi_k' +\Delta \psi (X_k)\,; \quad \psi_{1,k} = I_{n_y}\,.
%\label{e61sin}
\end{align*}

Символ $ \mm_{\cal N}$ означает, что соответствующие математические ожидания вычисляются по методу 
нормальной аппроксимации~\cite{3sin, 4sin}.  Линейный ФП (утверждение~7) при нормальном распределении является 
частным случаем~(\ref{e51sin}).

\section{Тестовые примеры}

\noindent
\textbf{Тестовый пример 1}.
Для системы
\begin{equation}
\left.
\begin{array}{rl}
    \dot X_{1t} &= M_{1t}^* - D^* X_{1t} - N^* X_{2t} +V_1\,;\\[6pt]
    \dot X_{2t} &= M_{2t}^* + N^* X_{1t} - D^* X_{2t} +V_2\,;\\[6pt]
    Z_{1t}&=\dot Y_{1t} = X_{1t} + V_3\,;\\[6pt]
Z_{2t} &=\dot Y_{2t} = X_{2t} +V_4
    \end{array}
    \right \}
    \label{e62sin}
\end{equation}
в силу уравнений разд.~3 ЛФКБ имеет вид~\cite{4sin}:

\noindent
\begin{multline*}
    \dot{\hat X}_{1t} = - D^* \hat X_{1t} - N^*\hat X_{2t} +{}\\
{}+ \fr{R_{11}}{G_{3t}} (Z_{1t} -\hat X_{1t}- M_1^*)+ {}\hspace*{10mm} %\\
\end{multline*}

\noindent
\begin{multline*}
{}+\fr{R_{12}}{G_{4t}} (Z_{2t} -\hat X_{2t}+M_2^*)= D^* \hat X_{1t} - N^*\hat X_{2t} +{}\\
    {}+ \beta_{11}(Z_{1t} -\hat X_{1t}- M_1^*)+ \beta_{12} (Z_{2t} -\hat X_{2t}+M_2^*)\,;
\end{multline*}

\vspace*{-12pt}

\noindent
\begin{multline*}
    \dot{\hat X}_{2t} = N^* \hat X_{1t} - D^*\hat X_{2t} +{}\\
   {}+ \fr{R_{12}}{G_{3t}} (Z_{1t} -\hat X_{1t}- M_1^*)+{}\\
{}+ \fr{R_{22}}{G_{4t}} (Z_{2t} -\hat X_{2t}+M_2^*)=N^* \hat X_{1t} - D^*\hat X_{2t} +{}\\
    {}+ \beta_{21}(Z_{1t} -\hat X_{1t}- M_1^*)+ \beta_{22} (Z_{2t} -\hat X_{2t}+M_2^*)\,,
 %    \label{e64sin}
    \end{multline*}
где $\beta_{11}=R_{11}/G_{3t}$;  $\beta_{12}=R_{12}/G_{4t}$;
$\beta_{21}=R_{12}/G_{3t}$;  $\beta_{22}=R_{22}/G_{4t}$;
\begin{equation*}
\beta_t =
\begin{bmatrix}
    \fr{R_{11}}{G_{3t}}&\fr{R_{12}}{G_{4t}}\\
    \fr{R_{12}}{G_{3t}}&\fr{R_{22}}{G_{4t}}
    \end{bmatrix} =
\begin{bmatrix}
    \beta_{11}&\beta_{12}\\
    \beta_{21}&\beta_{22}
    \end{bmatrix}\,;
%    \label{e65sin}
    \end{equation*}
    
    \vspace*{-12pt}
    
    \noindent
    \begin{equation*}
    \dot R_{11} =- 2(D^* R_{11} + N^* R_{12}) + G_{1t} - \fr{R_{11}^2}{G_{3t}} - \fr{R_{12}^2}{G_{4t}}\,;
    \end{equation*}
    
    \vspace*{-12pt}
    
    \noindent
    \begin{multline*}
    \dot R_{12} ={}\\
    {}=- 2D^* R_{12} - N^*(R_{22}- R_{11})  -
    \fr{R_{11} R_{12}}{ G_{3t}}  - \fr{R_{12} R_{22}}{G_{4t}}\,;
    \end{multline*}
    
    \vspace*{-12pt}
    
    \noindent
    \begin{equation*}
    \dot R_{22} = 2(N^* R_{11} -D^* R_{22}) + G_{2t} -
 \fr{R_{12}^2}{G_{3t}} - \fr{R_{22}^2}{G_{4t}}\,.
% \label{e66sin}
 \end{equation*}

\noindent
\textbf{Тестовый пример 2}.
Пусть $M_{it}^*$ $(i=1,2)$ являются неизвестными параметрами. 
Положим $M_{1t}^* = X_3$, $M_{2t}^* = X_4$, тогда уравнения  
$\dot X_{3t} =0$, $\dot X_{4t}=$\linebreak $=0$ будут совместно с уравнениями~(30) 
определять расширенный вектор состояния $\bar X_t= $\linebreak $=\lk X_{1t} X_{2t} X_{3t} X_{4t}\rk^{\mathrm{T}}$.

Для системы
\begin{align*}
    \dot X_{1t} &=  - D^* X_{1t} - N^* X_{2t} +X_{3t}+V_1\,;\\
    \dot X_{2t} &=  N^* X_{1t} - D^* X_{2t} +X_{4t}+V_2\,;\\
Z_{1t}&=\dot Y_{1t} = X_{1t} + V_3\,;\\
Z_{2t} &=\dot Y_{2t} = X_{2t} +V_4\,;\\ 
Z_{3t} &= X_{3t}+V_5\,;\\
 Z_{4t} &= X_{4t} +V_6
 \end{align*}
% \label{e68sin}
ЛФКБ имеет следующий вид:
\begin{multline*}
\dot{\hat X}_{1t} = - D^* \hat X_{1t} - N^*\hat X_{2t} +\hat X_{3t}+{}\\
{}+ \fr{R_{11} \left(Z_{1t} -\hat X_{1t}\right)}{G_{3t}}+ \fr{R_{12}}{G_{4t}} \left(Z_{2t} -\hat X_{2t}\right)+{}\\
{}+ \fr{R_{13}\left(Z_{3t} -\hat X_{3t}\right)}{G_{5t}}+\fr{R_{14}}{G_{6t}} \left(Z_{4t} -\hat X_{4t}\right)={}
\end{multline*}

%\vspace*{-12pt}

\noindent
\begin{multline*}
{}- D^* \hat X_{1t} - N^*\hat X_{2t} +\hat X_{3t}+ \beta_{11} \left(Z_{1t} -\hat X_{1t}\right)+ {}\\
{}+\beta_{12} (Z_{2t} -\hat X_{2t})+\beta_{13}
    \left(Z_{3t} -\hat X_{3t}\right)+{}\\
    {}+\beta_{14} \left(Z_{4t} -\hat X_{4t}\right)\,;
    \end{multline*}
    
    \vspace*{-12pt}
    
    \noindent
    \begin{multline*}
   \dot{\hat X}_{2t} = N^* \hat X_{1t} - 
   D^*\hat X_{2t} +\hat X_{4t}+{}\\
   {}+ \fr{R_{12}}{G_{3t}} \left(Z_{1t} -\hat X_{1t}\right)
+ \fr{R_{22}  \left(Z_{2t} -\hat X_{2t}\right)}{G_{4t}}+{}\\
{}+ \fr{R_{23}\left(Z_{3t} -\hat X_{3t}\right)}{G_{5t}}+\fr{R_{24}  \left(Z_{4t} -\hat X_{4t}\right)}{G_{6t}}={}\\
{}=- D^* \hat X_{1t} - N^*\hat X_{2t} +\hat X_{3t}+ \beta_{21} \left(Z_{1t} -\hat X_{1t}\right)+ {}\\
{}+\beta_{22} \left(Z_{2t} -\hat X_{2t}\right)+\beta_{23}
    \left(Z_{3t} -\hat X_{3t}\right)+{}\\
    {}+\beta_{24} \left(Z_{4t} -\hat X_{4t}\right)\,;
    \end{multline*}
    
    \vspace*{-12pt}

    \noindent
    \begin{multline*}
   \dot{\hat X}_{3t} =  \fr{R_{13} \left(Z_{1t} -\hat X_{1t}\right)}{G_{3t}}+ 
   \fr{R_{23} \left(Z_{2t} -\hat X_{2t}\right)}{G_{4t}}+{}\\
   {}+ \fr{R_{33} \left(Z_{3t} -\hat X_{3t}\right)}{G_{5t}}+
   \fr{R_{34}  \left(Z_{4t} -\hat X_{4t}\right)}{G_{6t}}={}\\
{}= \beta_{31} \left(Z_{1t} -\hat X_{1t}\right)+ \beta_{32} \left(Z_{2t} -\hat X_{2t}\right)+{}\\
{}+\beta_{33}     \left(Z_{3t} -\hat X_{3t}\right)
+\beta_{34} \left(Z_{4t} -\hat X_{4t}\right)\,;
    \end{multline*}
    
    \vspace*{-12pt}
    
    \noindent
\begin{multline*}
   \dot{\hat X}_{4t} =  \fr{R_{14}  \left(Z_{1t} -\hat X_{1t}\right)}{G_{3t}}+ 
   \fr{R_{24} \left(Z_{2t} -\hat X_{2t}\right)}{G_{4t}}+{}\\
   {}+ \fr{R_{34} \left(Z_{3t} -\hat X_{3t}\right)}{G_{5t}}+
   \fr{R_{44}  \left(Z_{4t} -\hat X_{4t}\right)}{G_{6t}}={}\\
{}= \beta_{41} \left(Z_{1t} -\hat X_{1t}\right)+ \beta_{42} \left(Z_{2t} -\hat X_{2t}\right)+{}\\
{}+\beta_{43}     \left(Z_{3t} -\hat X_{3t}\right)
+\beta_{44} \left(Z_{4t} -\hat X_{4t}\right)\,,
    \label{e69sin}
    \end{multline*}
где

\noindent
\begin{align*}
\beta_{11}&=\fr{R_{11}}{G_{3t}}\,;\quad  \beta_{12}=\fr{R_{12}}{G_{4t}}\,;\quad \beta_{13}=\fr{R_{13}}{G_{5t}}\,;\\[2pt]
\beta_{14}&=\fr{R_{14}}{G_{6t}}\,;\quad \beta_{21}=\fr{R_{12}}{G_{3t}}\,;\quad \beta_{22}=\fr{R_{22}}{G_{4t}}\,;\\[2pt]
\beta_{23}&=\fr{R_{23}}{G_{5t}}\,;\quad \beta_{24}=\fr{R_{24}}{G_{6t}}\,;\quad \beta_{31}=\fr{R_{13}}{G_{3t}}\,;\\[2pt]
\beta_{32}&=\fr{R_{23}}{G_{4t}}\,;\quad \beta_{33}=\fr{R_{33}}{G_{5t}}\,;\quad \beta_{34}=\fr{R_{34}}{G_{6t}}\,;\\[2pt]
\beta_{41}&=\fr{R_{14}}{G_{3t}}\,;\quad \beta_{42}=\fr{R_{24}}{G_{4t}}\,;\\[2pt]
\beta_{43}&=\fr{R_{34}}{G_{5t}}\,;\quad \beta_{44}=\fr{R_{44}}{G_{6t}}\,;
\end{align*}
%\label{e70sin}

%\vspace*{-6pt}

\noindent
\begin{multline*}
\dot R_{11} =- 2\left(D^* R_{11} + N^* R_{12}\right) +R_{13}+R_{14}+{}\\
{}+ G_{1t} - \fr{R_{11}^2}{G_{3t}} - \fr{R_{12}^2}{G_{4t}}-  \fr{R_{13}^2}{G_{5t}}- \fr{R_{14}^2}{G_{6t}}\,;
       \end{multline*}
       
       \vspace*{-12pt}
       
       \noindent
\begin{multline*}
\dot R_{22} = 2\left(N^* R_{12} -D^* R_{22}+R_{24}\right) + G_{2t} -{}\\
{}- \fr{R_{12}^2}{G_{3t}} - \fr{R_{22}^2}{G_{4t}}- \fr{R_{23}^2}{G_{5t}}- \fr{R_{24}^2}{G_{6t}}\,;
\end{multline*}

\noindent
\begin{equation*}
\dot R_{33} =- \fr{R_{13}^2}{G_{3t}} -\fr{R_{23}^2}{G_{4t}} - \fr{R_{33}^2}{G_{5t}}- \fr{R_{34}^2}{G_{6t}}\,;
\end{equation*}

%\vspace*{-12pt}

\noindent
\begin{equation*}
\dot R_{44} = -\fr{R_{14}^2}{G_{3t}} -\fr{R_{24}^2}{G_{4t}} - \fr{R_{34}^2}{G_{5t}}- \fr{R_{44}^2}{G_{6t}}\,;
\end{equation*}

\vspace*{-12pt}
 
 \noindent
 \begin{multline*}
 \dot R_{12} =N^* \left(R_{11} - R_{22}\right) -2 D^* R_{12} +R_{14}+R_{23} -{}\\
 {}- \fr{R_{11} R_{12}}{G_{3t}} - \fr{R_{12} R_{22}}{G_{4t}} - \fr{R_{13} R_{23}}{G_{5t}}- 
 \fr{R_{14} R_{24}}{G_{6t}}\,;
\end{multline*}

\vspace*{-12pt}

\noindent
\begin{multline*}
\dot R_{13} = - D^* R_{13} - N^* R_{23} +R_{33} - \fr{R_{11} R_{13}}{G_{3t}} -{}\\
{}- \fr{R_{12} R_{23}}{G_{4t}} - \fr{R_{13} R_{33}}{G_{5t}}- \fr{R_{14} R_{44}}{G_{6t}}\,;
\end{multline*}

\vspace*{-12pt}

   \noindent
\begin{multline*}
\dot R_{14} =-D^* R_{14} - N^* R_{24} +R_{34} - \fr{R_{11} R_{14}}{G_{3t}} -{}\\
{}- \fr{R_{12} R_{24}}{G_{4t}} - \fr{R_{13} R_{34}}{G_{5t}}- \fr{R_{14} R_{44}}{G_{6t}}\,;
\end{multline*}

\vspace*{-12pt}

\noindent
\begin{multline*}
\dot R_{24} =N^* R_{14} - D^* R_{24} +R_{44} - \fr{R_{11} R_{12}}{G_{3t}} -{}\\
{}- \fr{R_{22} R_{24}}{G_{4t}} - \fr{R_{23} R_{34}}{G_{5t}}- \fr{R_{24} R_{44}}{G_{6t}}\,;
\end{multline*}

\vspace*{-12pt}

\noindent
\begin{equation*}
\dot R_{34} =- \fr{R_{13} R_{14}}{G_{3t}} - \fr{R_{23} R_{24}}{G_{4t}} - \fr{R_{33} R_{34}}{G_{5t}}- \fr{R_{34} R_{44}}{G_{6t}}\,.
%\label{e71sin}
\end{equation*}

\noindent
\textbf{Тестовый пример 3.}
 Рассмотреть случай СтС~(30) при условиях, когда случайные функции 
 $X_3=V_1$ и $X_4=V_2$ представляют собой широкополосные случайные процессы, описываемые ФФ вида~(\ref{e4sin}).

\smallskip

\noindent
\textbf{Тестовый пример 4.}
 Рассмотреть случай СтС~(30)  
 при условиях, когда случайные функции $X_3=V_1$ и $X_4=V_2$ представляют собой <<окрашенные>> 
 случайные процессы, описываемые уравнениями вида~(\ref{e5sin}).

\smallskip

\noindent
\textbf{Тестовый пример 5.}
 В условиях примера~1 провести идентификацию следующих параметров: 
 $\pi_1 = N^*$, $\pi_2=D^*$, $\pi_3= M_1^*$, $\pi_4=M_2^*$.

\smallskip

\noindent
\textbf{Тестовые примеры~6--10}.
 В~\cite{4sin} для квазилинейных уравнений~(\ref{e1sin}) и~(\ref{e2sin}) 
 приведены тестовые примеры линеаризованных ФКБ и ФК обобщенных фильт\-ров 
 Калмана, гауссовых фильтров, фильтров второго порядка.
 
 \begin{figure*}[b] %fig1
\vspace*{1pt}
\begin{center}
\mbox{%
\epsfxsize=165.451mm
\epsfbox{sin-1.eps}
}
\end{center}
\vspace*{-9pt}
\Caption{Результаты наблюдения проекции угловой скорости $p=p(t)$~(\textit{а}) и
$q=q(t)$~(\textit{б}) на интервале времени 1995--2010~гг.~\cite{10sin}
\label{f1sin}}
\end{figure*}

\section{Вычислительные эксперименты. Анализ~результатов}

На рис.~\ref{f1sin} приведены результаты наблюдений проекций угловой скорости вращения 
Земли $X_1=$\linebreak $=p=p(t)$ и $X_2=q=q(t)$ на интервале времени 1995--2010~гг.~\cite{10sin}. 
Дискретность наблюдений составляет одно наблюдение в сутки. Первичный анализ экспериментальных 
данных позволяет сделать следующие выводы:

\noindent
\begin{enumerate}[(1)]
\item регулярные составляющие колебаний полюса Земли содержат наряду с вековым медленно меняющимся 
трендом и гармонические со\-став\-ля\-ющие на чандлеровской частоте, удвоенной чандлеровской частоте и 
комбинированные (годичные и чандлеровские). Как\linebreak показано в~\cite{11sin}, 
точность оценки методом\linebreak наименьших квадратов составляет величину около~0,01~угл.\,с/год 
и требует значительных временных затрат квалифицированных специалистов;
\item
наряду с регулярными колебаниями полюса Земли имеют место случайные широкополосные и <<окрашенные>> 
упомянутыми частотами колебания со сред\-ней квад\-ра\-ти\-че\-ской ошибкой порядка 0,01~угл.\,с/год .
\end{enumerate}

В рамках работ по фундаментальной проблеме <<Статистическая динамика вращения Земли>> в~2009~г.\ 
разработано инструментальное програм\-мное обеспечение <<СДВЗ-2009>> в среде MATLAB. Оно включает 
в свой состав три модуля:
\begin{enumerate}[(1)]
\item СДВЗ-2009-1~--- непрерывные ЛФ для оперативной обработки информации и идентификации параметров 
методами непрерывных ЛФКБ и ЛФП;
\item
СДВЗ-2009-2~--- дискретные ЛФ для оперативной обработки информации и идентификации параметров методами 
дискретных ЛФК и ЛФП;
\item
СДВЗ-2009-3~--- дискретный полиномиальный ФП для оперативной обработки информации и идентификации параметров методом ФП.
\end{enumerate}

Приведем некоторые результаты вычислительных экспериментов с инструментальным программным обеспечением <<СДВЗ-2009>>.


На рис.~\ref{f3sin} показаны результаты вычислительных экспериментов по оперативной 
оценке информационных переменных  $\hat X_1=\hat p_t$, $\hat X_2 = \hat q_t$ методом ЛФКБ 
(тестовый пример~1) при:
\begin{itemize}
\item
кривые \textit{1}~--- экспериментальные данные (рис.~\ref{f1sin});
\item
кривые \textit{2}~--- $M_1=M_2=0$, $G_1=G_2=G_3=$\linebreak $=G_4=0,001$, $N^*=0,01$, $D^*=0,01$;
\item
кривые \textit{3}~--- $M_1=M_2=0$, $G_1=G_2=G_3=$\linebreak $=G_4=0,001$, $N^*=0,01$, $D^*=0,1$;
\item
кривые \textit{4}~--- $M_1=M_2=0$, $G_1=G_2=G_3=$\linebreak $=G_4=0,001$, $N^*=0,01$, $D^*=1$.
\end{itemize}

Точность расчетов $\left(\sqrt{R_{ii}}\right)$ соответственно для кривых~\textit{2}, \textit{3} и~\textit{4} 
составляет 0,001, 0,01 и~0,02.
Дискретный (линейный и полиномиальный) ФП был применен для оперативной обработки информации 
и оценки параметров из тестовых примеров~3--5. Результаты оценки информационных переменных 
$\hat X_1=\hat p_t$, $\hat X_2=\hat q_t$ методом дискретного ЛФП на\linebreak\vspace*{-12pt}
\pagebreak
\end{multicols}

\begin{figure} %fig2
\vspace*{1pt}
\begin{center}
\mbox{%
\epsfxsize=165.961mm
\epsfbox{sin-3.eps}
}
\end{center}
\vspace*{-9pt}
\Caption{Оперативная оценка методом ЛФКБ проекции угловой скорости $p=p(t)$~(\textit{а}) и
$q=q(t)$~(\textit{б}) на интервале времени 1995--2010~гг.
 \label{f3sin}}
 \vspace*{12pt}
 \end{figure}

\begin{figure} %fig3
\vspace*{1pt}
\begin{center}
\mbox{%
\epsfxsize=165.055mm
\epsfbox{sin-5.eps}
}
\end{center}
\vspace*{-9pt}
\Caption{Оперативная оценка методом ЛФП проекции угловой скорости $p=p(t)$~(\textit{а})
и $q=q(t)$~(\textit{б}) (в районе 2000~г.)
\label{f5sin}}
\end{figure}

\begin{multicols}{2}

\noindent
интервале времени с 1995 по~2010~гг.\
совпадают с ЛФКБ (рис.~\ref{f3sin}). Оценка этих параметров на небольшом временном интервале в 
200~ежесуточных измерений в районе 2000~г.\ представлена на рис.~\ref{f5sin} 
для координат $p=p(t)$ и $q=q(t)$. Коэффициенты ФП при решении тестового примера~1 
принимают следующие значения: $\alp_{11}=\alp_{22}=0{,}3817$; 
$\alp_{12}=\alp_{21}=0{,}0038$; $\beta_{11}=\beta_{22}=0{,}6172$; 
$\beta_{12}=\beta_{21}=$\linebreak $=0{,}0062$;  $\gamma_1=\gamma_2=0$. 
Вычислительные эксперименты по оперативной оценке переменных 
$\hat X_3=M_1^*$, $\hat X_4 =N_2^*$ методом ЛФП для тестового 
примера~2 на интервале времени с 1995 по~2010~гг.\ и при $G_1=G_2=G_3=G_4=0{,}0001$, $N^*=D^*=0{,}01$ 
дали следующие результаты: $\hat M_1^*=\hat M_2^*=0{,}005$.



Модуль СДВЗ-2009-3 был использован для оценки параметров~$N^*$ и~$D^*$ 
с помощью дискретного полиномиального ФП на интервале времени в 500~ежесуточных измерений. 
В~результате были получены следующие оценки: $\hat N^*= 0{,}015$; $\hat D^*= 0{,}009$.

Для учета случайных возмущений, пред\-став\-ляющих собой широкополосные~(4) и <<окрашенные>>~(5) 
случайные процессы, а также инерционных свойств астрометрических систем\linebreak
 измерения~(8) и~(9) 
разработан ЛФ, описываемый уравнениями утверж\-де\-ния~1. В~соответствии с замечанием~6, 
статистическим моделированием подтверждена возможность замены рассматриваемых при $D^*/ N^*<10^{-2}$ 
возмущений белым шумом.

Как показали вычислительные эксперименты тестовых примеров~6--10, для оперативной оценки 
флуктуаций на интервале времени 3--5~лет рекомендуется использовать линейные фильтры, учитывающие 
одну широкополосную составляющую и первые две <<окрашенные>> составляющие. Для интервалов времени 
10--20~лет должны использоваться квазилинейные фильтры, учитывающие квазилинейные составляющие 
гироскопических и флук\-ту\-а\-ци\-он\-но-диссипативных возмущений.


\section{Заключение}

Проведенное тестирование и статистическое моделирование убедительно 
показывают эффективность разработанных инструментальных средств по 
оперативному оцениванию флуктуаций и параметров модели.

В ИПИ РАН начата работа по дальнейшему развитию инструментального алгоритмического и 
программного обеспечения для обработки результатов наблюдений колебаний полюса Земли 
на основе канонических представлений случайных функций~[5--12].

{\small\frenchspacing
{%\baselineskip=10.8pt
\addcontentsline{toc}{section}{Литература}
\begin{thebibliography}{99}
\bibitem{1sin}
\Au{Пугачёв В.\,С., Синицын~И.\,Н.} 
Стохастические дифференциальные системы. Анализ и фильтрация. 2-е изд.~--- М.: Наука, 1990.

\bibitem{2sin}
\Au{Синицын И.\,Н.} 
Из опыта преподавания статистических основ информатики в технических университетах~// 
Система и средства информатики. Спец.\ вып., посвященный II международному конгрессу \mbox{ЮНЕСКО}
<<Образование и информатика>>.~--- М.: Наука, 1996. Вып.~8. С.~68--73.

\bibitem{3sin}
\Au{Синицын И.\,Н.} 
Развитие теории фильтров Пугачёва для оперативной обработки информации в стохастических системах~// 
Информатика и её применения, 2007. Т.~1. Вып.~1. С.~3--13.

\bibitem{4sin}
\Au{Синицын И.\,Н.} 
Фильтры Калмана и Пугачева. 2-е изд.~--- М.: Логос, 2007.

\bibitem{5sin}
\Au{Синицын И.\,Н.} 
Канонические представления случайных функций и их применение в задачах компьютерной поддержки научных исследований.~--- 
М.: ТОРУС ПРЕСС, 2009.

\bibitem{6sin}
\Au{Синицын И.\,Н.} 
Корреляционные методы построения аналитических информационных моделей флуктуаций полюса Земли по априорным данным~// 
Информатика и её применения, 2007. Т.~1. Вып.~2. С.~2--14.

\bibitem{7sin}
\Au{Синицын И.\,Н., Корепанов~Э.\,Р., Семендяев~Н.\,Н.} 
Методическое и программное обеспечение корреляционного анализа флуктуаций полюса Земли~// 
Системы и средства информатики. Спец.\ вып.\ <<Геоинформационные технологии>>.~--- М.: ИПИ РАН, 2008. С.~379--396.

\bibitem{8sin}
\Au{Марков Ю.\,Г., Синицын~И.\,Н.} 
Корреляционная модель колебаний полюса Земли с параметрическими возмущениями~// Астроном. журнал, 2008. 
Т.~85. №\,6. C.~566--576.

\bibitem{9sin}
\Au{Синицын И.\,Н.} 
Методы построения эредитарных информационных моделей флуктуаций в стохастической динамике Земли~// 
Информатика и её применения, 2009. Т.~3. Вып.~4. C.~2--11.

\bibitem{10sin}  
IERS Annual Reports.  {\sf  http://hpiers.obspm.fr/eop-pc/}.

\bibitem{11sin} 
\Au{Акуленко Л.\,Д., Кумакшев~С.\,А., Марков~Ю.\,Г., Рыхлова~Л.\,В.} 
Высокоточный прогноз движения полюса Земли~// Астроном. журнал, 2006. Т.~83. №\,4. С.~376--384.

\label{end\stat}

\bibitem{12sin}
\Au{Синицын И.\,Н., Синицын~В.\,И., Корепанов~Э.\,Р., Белоусов~В.\,В., Конашенкова~Т.\,Д., Семендяев~Н.\,Н.,
Басилашвили~Д.\,А.} 
Инструментальное программное обеспечение анализа и синтеза стохастических систем высокой до\-ступ\-ности~(I)~// 
Системы высокой доступности, 2009. Т.~5. №\,3. С.~4--52.
 \end{thebibliography}
}
}
\end{multicols}