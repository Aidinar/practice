\def\E{\mbox{\rm E}}
%\def\erm{\mbox{\rm e}}
\def\O{\mbox{\rm O}}
\def\o{\mbox{\rm o}}

\def\P{\mbox{\rm P}}

\def\vx{\mbox{\boldmath{$x$}}}
\def\vy{\mbox{\boldmath{$y$}}}


\def\IP{\mbox{$\Phi$}}
\def\Ip{\mbox{$\phi$}}

\def\stat{kavag}


\def\tit{ПРИБЛИЖЕНИЯ ДЛЯ СТАТИСТИК, ОПИСЫВАЮЩИХ ГЕОМЕТРИЧЕСКИЕ СВОЙСТВА 
ДАННЫХ БОЛЬШОЙ РАЗМЕРНОСТИ, С~ОЦЕНКАМИ ОШИБОК$^*$}

\def\titkol{Приближения для статистик, описывающих геометрические свойства данных большой размерности} %, с~оценками ошибок}

\def\autkol{Ю.~Кавагучи, В.\,В.~Ульянов, Я.~Фуджикоши}
\def\aut{Ю.~Кавагучи$^1$, В.\,В.~Ульянов$^2$, Я.~Фуджикоши$^3$}

\titel{\tit}{\aut}{\autkol}{\titkol}

{\renewcommand{\thefootnote}{\fnsymbol{footnote}}\footnotetext[1]
{Исследования выполнены при частичной поддержке РФФИ, гранты 08-01-00567, 08-01-91205, 09-01-12180.}}

\renewcommand{\thefootnote}{\arabic{footnote}}
\footnotetext[1]{Высшая научная и инженерная школа, Университет Чуо, Токио, n15007@gug.math.chuo-u.ac.jp}
\footnotetext[2]{Московский государственный университет им.\
М.\,В.~Ломоносова, факультет вычислительной математики и кибернетики,
 vulyan@gmail.com}
\footnotetext[3]{Высшая научная и инженерная школа, 
Университет Чуо, Токио, fuji@math.sci.hiroshima-u.ac.jp}

\vspace*{2pt}

\Abst{При описании геометрических свойств~$n$  наблюдений с~$p$ признаками необходимо 
исследовать асимптотическое поведение трех статистик: длины $p$-мерного вектора наблюдений, 
расстояния между двумя векторами наблюдений и угла между ними. В~[1] найдено асимптотическое 
поведение указанных статистик, когда размерность~$p$ стремится к бесконечности, а объем выборки~$n$ 
остается постоянным. В настоящей работе результаты~[1] уточняются путем построения асимптотических разложений. 
Кроме этого, найдены вычислимые оценки погрешности приближения предельными распределениями для первых двух 
статистик. Эти результаты будут также полезны  в ситуациях, когда размерность данных не является чрезмерно большой.}

\vspace*{1pt}

\KW{данные большой размерности; асимптотические разложения; оценки погрешности; геометрические свойства}

\vspace*{3pt}

     \vskip 18pt plus 9pt minus 6pt

      \thispagestyle{headings}

      \begin{multicols}{2}

      \label{st\stat}

\section{Введение}

Рассмотрим случайную выборку  $\vx_1, \ldots, \vx_n$, взятую из  $p$-мер\-но\-го распределения. 
Набор данных можно рассматривать как  $n$~векторов или точек в $p$-мер\-ном пространстве. При анализе данных 
полезным оказывается исследование следующих трех геометрических характеристик: длины $p$-мер\-но\-го 
вектора наблюдений, расстояния между двумя векторами наблюдений и угла между ними.

В последние годы значительный интерес вызывают исследования для данных большой раз\-мер\-ности. 
Это связано с тем, что такого типа данные встречаются все в большем числе приложений, в частности 
при анализе финансовых рынков и деятельности финансовых организаций, в социологии, генетике, биологии 
и математической физике.
В асимптотической теории для данных большой размерности предполагается, что либо ($i$)~оба параметра~--- 
размерность~$p$ и объем выборки~$n$~--- стремятся к бесконечности, либо ($ii$)~$p$~стремится к бесконечности, а 
объем~$n$ фиксирован.

 Недавний результат в рамках предположения~($i$) представлен, например, в~[2], где получены также вычислимые 
 оценки точности приближений.

 Если~$\vx_i$  взяты из нормального распределения~ $N(0, I_p)$,  в~[1] в предположении~($ii$) 
 показано, что три статистики, описывающие геометрические свойства данных, удовлетворяют следующим 
 предельным соотношениям:
\begin{align*}
\|\vx_i\| &= \sqrt{p} +\O_p(1), \!\!\!& i&=1,\ldots,n\,; \\
\|\vx_i- \vx_j\| &= \sqrt{2p} +\O_p(1), \!\!\!& i,j&=1,\ldots n\,,\enskip i \neq j\,; \\
\mathrm{ang}(\vx_i,\vx_j) &= \fr{1}{2}\pi +\O_p(p^{-1/2}),\!\!\!& i,j&=1,\ldots n,~i \neq j\,,
\end{align*}
где $\| \cdot \|$~--- евклидово расстояние и  $\O_p$ обозначает стохастический порядок малости. 
Из этих  результатов вытекает, что при увеличении размерности данные сходятся к вершинам правильного симплекса.

В настоящей работе  эти результаты сначала уточняются  путем построения асимптотических разложений 
для распределений указанных трех статистик. Затем   строятся вычислимые оценки погрешности для 
предельных распределений первых двух статистик. При этом получены два типа неравенств с использованием 
идей из работы~[3], где найдена оценка порядка~$\O(p^{-1})$  хи-квадрат приближения для преобразованной 
хи-квадрат случайной величины с $p$~степенями свободы. Тем самым получена возможность судить о геометрических 
свойствах статистических данных, размерность которых не является чрезмерно большой.


\section{Асимптотические разложения для распределений статистик}

Пусть
$\vx_i = (x_{i1}, \ldots, x_{ip})^\prime$ $(i =1, \ldots, n)$ есть выборка из распределения~$N(0,I_p)$.

Рассмотрим сначала статистику
$$
\|\vx_i\|= S_{ii}^{1/2}\,,
$$
где  $S_{ii} = \sum\limits^{p}_{k=1} x^2_{ik}$. Поскольку статистика~$S_{ii}$ распределена как~$\chi_p^{2}$~--- 
хи-квадрат случайная величина с $p$~степенями свободы, она не зависит от~$i$.  Положим
\begin{equation*}
 V = \fr{S_{ii} - p}{\sqrt{2p}}\,.
\end{equation*}
Распределение~$V$ сходится к стандартному нормальному при стремлении~$p$ к бесконечности.
Поскольку  $S_{ii} = p(1 + \sqrt{2}\,p^{-1/2}V)$, имеем
\begin{multline*}
\|\vx_i\|= S_{ii}^{1/2}
     =\sqrt{p}\left(1 + \fr{1}{2}\,\sqrt{2}Vp^{-1/2} -{}\right.\\
\left.     {}- \fr{1}{4}\,Vp^{-1} + \fr{1}{8}\,\sqrt{2}V^3 p^{-3/2} + \cdots \right)\,.
\end{multline*}

Рассмотрим асимптотическое разложение для распределения случайной величины
$$
T_1 = \sqrt 2(\|\vx_i\| -\sqrt{p})\,.
$$
Характеристическая функция величины~$T_1$ запишется в виде
\begin{multline*}
C_{T_1}(t) = \E\left[\exp\left\{(it) \sqrt 2 \left(\|\vx_i\| - \sqrt{p}\right)\right\}\right] ={}\\
= C_0(t) + \fr{1}{\sqrt{p}}\,C_1(t)+ \fr{1}{p}\,C_2(t) + \O\left(p^{-3/2}\right)\,,
\end{multline*}
где
\begin{align*}
C_0 (t) &= \E\left[\exp \left \{(it) V \right\}\right]\,; \\
C_1 (t) &= \E\left[- \fr{1}{4}\,\sqrt{2}(it)V^2 \exp \left \{(it) V\right\} \right]\,;\\
C_2 (t) &= \E\left[ \left(\fr{1}{4}\,(it)V^3 + \fr{1}{16}\,(it)^2 V^4 \right) \exp\{(it)V \}\right]\,.
\end{align*}
Для вычисления функций~$C_0(t)$, $C_1(t)$ и~$C_2(t)$ используем асимптотическое разложение для плот\-ности 
случайной величины (с.в.)~$V$.
Характеристическая функция с.в.~$V$ может быть записана в виде
\begin{multline*}
C_V(t) = \exp\left\{\fr{1}{2}\left(it\right)^2\right\}
            \left[1 + \fr{1}{\sqrt{p}}\,\fr{1}{6}(it)^3\kappa_3 +{}\right.\\
\left.{}  + \fr{1}{p}\left\{\fr{1}{72}\,(it)^6\kappa_3^2 + \fr{1}{24}\,(it)^4\kappa_4 \right\} + \O\left(p^{-3/2}\right)\right]\,.
\end{multline*}
Соответственно, плотность с.в.~$V$ раскладывается как
\begin{multline*}
\Ip(x)\left[1 + \fr{1}{\sqrt p}\,\fr{\sqrt{2}}{3}\,h_3(x)
                  + {}\right.\\
\left.                  {}+\fr{1}{p}\left\{\fr{1}{9}\,h_6(x) + \fr{1}{2}\,h_4(x) \right\}\right]+ \O\left(p^{-3/2}\right)\,,
\end{multline*}
где  $h_i(x)$~--- многочлен Эрмита порядка~$i$, а $\Ip(x)$~--- плотность стандартного нормального распределения.
Используя это разложение,   можно  найти $C_0(t)$, $C_1(t)$ и~$C_2(t)$.
Асимптотическое разложение для распределения с.в.~$T_1$ можно получить, обращая ее характеристическую функцию. 
Тем самым доказана следующая теорема:

\medskip

\noindent
\textbf{Теорема 1.} %\label{th2.1}
\textit{Пусть $\vx_i$ есть $p$-мерный случайный вектор с распределением~$N(0,I_p)$.
Тогда функция распределения с.в.\ $T_1 = \sqrt 2(\|\vx_i\| -\sqrt{p})$ может быть представлена в виде}
\begin{equation*}
\IP(x) -\Ip(x)\left[\fr{1}{\sqrt p}\, \ell_1(x) + \fr{1}{p}\,\ell_2(x) \right] + \O(p^{-3/2})\,,
\end{equation*}
\textit{где $\IP(x)$~--- функция распределения стандартного нормального закона, а $\ell_1(x)$ и~$\ell_2(x)$ 
определяются формулами}
\begin{align*}
\ell_1(x) &= \fr{1}{12}\,\sqrt{2}h_2(x) - \fr{1}{4}\,\sqrt{2}h_0(x)\,;\\
\ell_2(x) &=\fr{1}{144}\,\left[  -15 h_5(x) -6 h_3(x) + 16 h_2(x) -{}\right.\\
&\hspace*{10mm}\left.{}- 81 h_1(x) + 72h_0(x)\right]\,.
\end{align*}


Распределение с.в.\
$\| \vx_i - \vx_j \|$ $(i \neq j)$ тесно связано с распределением
$\| \vx_i \|$, так как $(x_{ik} - x_{jk})/\sqrt{2}\sim$\linebreak $ \sim N(0,1)$. Точнее, распределение 
с.в. $\| \vx_i - \vx_j \|/\sqrt{2}$ совпадает с распределением с.в.~$\| \vx_i \|$. 
Следовательно, распределение~$T_1$ совпадает с распределением
\begin{equation*}
T_2 = \sqrt{2}\left\{\fr{\| \vx_i - \vx_j \|}{\sqrt{2}}-\sqrt{p}\right\}
=\| \vx_i - \vx_j \|-\sqrt{2p}
%\label{deft2}
\end{equation*}
и справедливо следствие:

\medskip

\noindent
\textbf{Следствие 1.} %\label{cor2.2}
\textit{Пусть $\vx_i$ и $\vx_j$~--- независимые случайные векторы с распределением~$N(0,I_p)$.
Тогда распределение $T_2 =\| \vx_i - \vx_j \| -\sqrt{2p}$ совпадает с распределением~ $T_1$. 
В~частности, для функции распределения~$T_2$  имеет место то же асимптотическое разложение, что и для~$T_1$.}
\medskip

Далее рассмотрим асимптотическое разложение для распределения угла~$\theta$ между двумя независимыми 
векторами~$\vx_i$ и~$\vx_j$, взятыми из нормального распределения~$N(0,I_p)$.
Поскольку
$$
\|\vx_i - \vx_j\|^2 = \sum\limits^{p}_{k = 1} \left(x_{ik} - x_{jk} \right)^2 = S_{ii} +S_{jj} -2 S_{ij}\,,
$$
где  $S_{ij} = \sum\limits^{p}_{k=1} x_{ik} x_{jk}$, получаем
\begin{multline*}
\cos \theta = \fr{\|\vx_i\|^2+\|\vx_j\|^2 - \|\vx_i -\vx_j\|^2}{2 \|\vx_i\|\|\vx_j\|}={}\\
{} =\fr{S_{ij}}{\sqrt{S_{ii}S_{jj}}}
                = r_{ij} \,,
\end{multline*}
где $r_{ij}$~--- выборочный коэффициент корреляции. Распределение~$r_{ij}$ исследовалось во многих работах. 
В~частности, если  $\rho_{ij} = 0$, для функции распределения~$r_{ij}$ имеем~[4]:
\begin{multline*}
\P\left (\sqrt{p}\cdot r_{ij} < x\right)={}\\
{}= \IP(x) +\fr{1}{p}\, \left \{- \fr{3}{4}\,x
+\fr{1}{4}\,x^3\right\} \Ip(x) 
+ \O\left(p^{-3/2}\right)\,.
\end{multline*}
Поскольку~ $\theta$ может быть разложен в терминах~$r_{ij}$ в виде
\begin{equation*}
\theta = \arccos r_{ij}
       = \fr{1}{2}\,\pi - \left(r_{ij} + \fr{1}{6}\,r_{ij}^3 + \fr{3}{40}\, r_{ij}^5 + \cdots\right)\,,
\end{equation*}
полагая   $y = \sqrt{p}\cdot r_{ij}$, имеем
\begin{equation*}
\sqrt p \left( \fr{1}{2}\,\pi - \theta \right) =  y + \fr{1}{6p}\,y^3 + \o\left(p^{-1}\right)\,.
\end{equation*}
Тем самым характеристическую функцию $T_3 =$\linebreak $= \sqrt {p} \left(\pi/2 - \theta \right)$ можно записать в виде
\begin{equation*}
C_{\mathrm{ang}}(it) = C_0(t) +\fr{1}{6p}\,C_1(t) + \o(p^{-1})\,,
\end{equation*}
где
\begin{align*}
C_0(t) &= \E \left[\exp\left\{(it) y\right\}\right ]\,;\\
C_1(t) &= \E \left[\exp\left\{(it) y\right\}(it)y^3\right ]\,.
\end{align*}
Используя асимптотическое разложение для распределения~$\sqrt{p}\cdot r_{ij}$,
вычисляем~$C_0(t)$ и~$C_1(t)$. Обращая полученную характеристическую функцию, приходим к следующей теореме:

\smallskip

\noindent
\textbf{Теорема 2.} %\label{th2.4}
\textit{Пусть $\vx_i = (x_{i1}, \ldots, x_{ip})^\prime$ есть $p$-мер\-ный случайный вектор, 
взятый из нормального распределения~$N(0,I_p)$, и~$\theta$~--- угол между двумя независимыми 
векторами~$\vx_i$ и~$\vx_j$.
Тогда функция распределения с.в.\  $T_3 = \sqrt{p}\left(\pi/2 - \theta\right)$ может быть представлена в виде}
\begin{equation*}
\IP(x) +\fr{1}{12p}\,
       \left[h_3(u) - 6h_1(x)\right]\Ip(x) + \o(p^{-1})\,.
\end{equation*}


\section{Оценки погрешностей приближений}

В этой части   получены вычислимые выражения~$B(p)$ такие, что
%\vspace*{-1pt}
\begin{equation}
| \P(T_1 \le x) - \IP(x) | \le  B(p) = \O(p^{-1/2})\,,
\label{ebound1}
\end{equation}
где $T_1 = \sqrt 2(\|\vx_i\| -\sqrt{p})$. В силу следствия~1 результат~(\ref{ebound1}) 
справедлив и для $T_2 =\| \vx_i - \vx_j \| -\sqrt{2p}$.

Идея доказательства опирается на работу~[3]. Рассмотрим случайную величину
\begin{equation*}
 V_p = \fr{\chi_p^2 - p}{\sqrt{2p}}\,.
\end{equation*}
Пусть~$h(x)$ есть действительная функция, определяемая формулой
$$
h(x) = \sqrt{2}\left[\left(\sqrt{2p}\cdot x + p \right)^{{1}/{2}} - \sqrt{p}\right]\,.
$$
Тогда
\begin{equation*}
 h(V_p)=\sqrt{2}\left(\sqrt{\chi_p^2}- \sqrt{p}\right)
       =\sqrt{2}\left(\|X\|- \sqrt{p}\right)
       =T_1\,.
\end{equation*}
Если   $x \ge -\sqrt{2p}$,    то

\vspace*{-1pt}

\noindent
\begin{multline*}
\sup\limits_{-\sqrt{2p}\; \le\; x} \left |\P(T_1 \le x) - \IP(x) \right|={}\\
{}  =\sup\limits_{-\sqrt{2p}\; \le\; x} \left| \P(h(V_p) \le x) - \IP(x)\right| ={}\\
{}=\sup\limits_{-\sqrt{2p} \;\le\;
 x} \left| \P\left(V_p \le\, x + \fr{x^2}{2\sqrt{2p}}\right) - \IP(x)\right|\leq{}\\
{} \leq\sup\limits_{-\sqrt{2p}\; \le\; x} \left(\left|I_1\right| +\left|I_2 \right|\right)\,,
\end{multline*}
где
\vspace*{-4pt}

\noindent
\begin{align*}
                 I_1&= \P\left(V_p  \le x + \fr{x^2}{2\sqrt{2p}}\right)
         - \IP \left(x + \fr{x^2}{2\sqrt{2p}}\right)\,;\\
                 I_2&= \IP \left(x + \fr{x^2}{2\sqrt{2p}}\right)
             - \IP(x) \,.
\end{align*}
Для  $I_1$   в следующем разделе (см.\ леммы~1 и~2 ниже) будут получены два типа оценок:

\vspace*{-1pt}

\noindent
\begin{multline*}
|I_1| =\left| \P\left(V_p \le x + \frac{x^2}{2\sqrt{2p}}\right)
         - \IP \left( x + \frac{x^2}{2\sqrt{2p}}\right)\right|
      \leq{}\\
      {}\leq D_i(\lambda,p) \ \mbox{для}\  i=1, 2.
\end{multline*}
  Здесь же   займемся оцениванием~$I_2$:
%  \pagebreak

\vspace*{-3pt}  
  \noindent
\begin{multline*}
|I_2|  =\left |\IP \left(V_p  \le x + \fr{x^2}{2\sqrt{2p}}\right) - \IP(x)\right|={}\\
       =\left |\fr{1}{\sqrt{2 \pi}}\,\int\limits^{x + x^2/(2\sqrt{2p})}_{x} \exp \left\{- \fr{z^2}{2}\right\}\,dz\right|\,.
\end{multline*}
\pagebreak

\noindent
Если     $x \ge 0$, то  для~$I_2$     получаем
\begin{equation*}
I_2  \leq \fr{x^2}{2\sqrt{2p}} \, \fr{1}{\sqrt{2\pi}}\exp \left\{- \fr{x^2}{2}\right\}
 \leq \fr{1}{2 e \sqrt{p \pi}}\,.
\end{equation*}
Если $-\sqrt{2p} \le x \le 0$, то $x +  {x^2}/(2\sqrt{2p}) \le x / 2$. Следовательно,
\begin{multline*}
I_2 \leq \fr{x^2}{2\sqrt{2p}} \, \fr{1}{\sqrt{2\pi}}\,
\exp \left\{-\fr{1}{2}\, \left(x + \fr{x^2}{2\sqrt{2p}} \right )^2\right\}\leq{}\\
\leq \fr{x^2}{2\sqrt{2p}}\, \fr{1}{\sqrt{2\pi}}\,
\exp \left\{-\fr{1}{2} \left( \fr{x}{2}\right )^2\right\}
 \leq \fr{2}{ e\sqrt{p \pi}}\,.
\end{multline*}
Поскольку $1/(2 e \sqrt{p \pi})< 2 /(e\sqrt{p \pi})$,
 при выполнении условия  $x \ge - \sqrt{2p}$ имеем
\begin{equation}
\sup\limits_{-\sqrt{2p} \;\le\; x} |\P(T_1 \le x) - \IP(x)|
  \leq D(\lambda,p) + \fr{2}{e \sqrt{p \pi}}\,, 
  \label{eq12}
\end{equation}
где $D(\lambda,p) = \min\left(D_1(\lambda,p), D_2(\lambda,p)\right)$.

Известно (см., например, 7.1.13 в~[5]), что для любого     $x \ge 0$
\begin{equation}
 e^{x^2} \int\limits^{\infty}_{x} e^{- t^2} \,dt \leq \fr{1}{x + \sqrt{x^2 + 4/\pi}}\,. 
 \label{eq13}
\end{equation}
Заменой переменных из~(\ref{eq13})  для всех $x > 0$ получаем
$$
   1 -\IP(x) \leq  \sqrt{\fr{2}{\pi}}\, \fr{e^{-x^2/2}}{x + \sqrt{x^2 + 8/\pi}}\,.
$$
Тогда
\begin{multline}
\IP(-\sqrt{2p})  = 1 - \IP(\sqrt{2p})\leq{}\\
{} \leq \sqrt{\fr{2}{\pi}}\, \fr{e^{-p}}{\sqrt{2p} + \sqrt{2p + 8/\pi}}\,.
 \label{eq14}
\end{multline}
 Из~(\ref{eq12}) и~(\ref{eq14})  получаем следующую оценку
\begin{multline*}
\sup\limits_{x \in R} |\P(T_1 \leq x) - \IP(x)| \leq{}\\
{}\leq
\max\left(\vphantom{\fr{e^{-p}}{\sqrt{2}}}
\min_{\lambda} D(\lambda,p) + \fr{2}{e\sqrt{p \pi}},\,{}\right.\\
\left.\sqrt{\fr{2}{\pi}}\, \fr{e^{-p}}{\sqrt{2p}+\sqrt{2p + 8/\pi}}\right)\,.
\end{multline*}
Поскольку       
$$
\fr{2}{e\sqrt{p \pi}} > \sqrt{\fr{2}{\pi}}\, \fr{e^{-p}}{\sqrt{2p}+\sqrt{2p + 8/\pi}}
$$
и 
$$\min\limits_{\lambda} \,D(\lambda,p) >0\,,
$$ 
имеем
\begin{equation*}
\sup\limits_{x \in R} |\P(T_1 \le x) - \IP(x)| \leq \underset{\lambda}{\min}\,D\left(\lambda,p\right)
+ \fr{2}{e\sqrt{p \pi}}\,.
\end{equation*}

Таким образом (см.~(\ref{D_2})), доказана следующая теорема:

\smallskip

\noindent
\textbf{Теорема 3.} %{\label{th3.1}}
\textit{Справедлива оценка}
\begin{equation*}
\sup\limits_{x \in R}|\P(T_1 \leq x) - \IP(x) | <B(p)\,,
\end{equation*}
\textit{где}
\begin{equation}
B(p)= \underset{\lambda}{\min}\,D(\lambda,p) + \fr{2}{e\sqrt{p \pi}}\,, 
\label{eq.11}
\end{equation}
\textit{минимум берется по всем $\lambda \in (0,\sqrt{3}-1)$
и}
\begin{multline*}
D(\lambda,p) =\fr{2}{\pi} \left( \sqrt{\fr{\pi}{p}}\,\fr{1}{6}
 +\fr{ 2 (1-\lambda)}{p(2-2\lambda-\lambda^2)^2}+{}\right.\\
\left. {}+ \fr{(1+
\lambda^2)}{\lambda^2 p} \left(1+\lambda^2\right)^{-p/4}
  + \fr{1}{\lambda^2 p} \,\exp \left(-\fr{\lambda^2 p}{4}\right)
  \vphantom{\sqrt{\fr{\pi}{p}}}\right)\,.
\end{multline*}

\medskip

Из следствия~1 и оценки для~$T_1$   также получается

\smallskip

\noindent
\textbf{Следствие 2.}
\textit{Для статистики} $T_2 = \|\vx_i - \vx_j\| -\sqrt{2p}$
\begin{equation*}
\sup_{x \in R}|\P(T_2 \le x) - \IP(x) | <B(p)\,,
\end{equation*}
где $B(p)$ определено в~(\ref{eq.11}).
%\smallskip

\section{Два подхода к оценке $I_1$}

Будем использовать результат из~[3] для на\-хож\-де\-ния величины~$D(\lambda, p)$,
 которая будет выступать оценкой для~$I_1$.
В~[3] доказано, что
\begin{equation*}
\sup\limits_{x \in R} |F_p(x) - \IP_p(x)| \leq D_0(\lambda,p)\,,
\end{equation*}
где
\begin{align*}
F_p(x) &= \P(V_p \le x) =\P\left(\chi^{2}_p -p \le \sqrt{2p}x\right)\,; \\
\IP_p(x) &= \IP(x) + \fr{\sqrt{2}(1-x^2)}{3\sqrt{p}}\,\Ip(x)
\end{align*}
и    
\begin{multline*}
D_0(\lambda,p)= \fr{2}{\pi p}\left(
\vphantom{\fr{(1+\lambda^2)^{1-p/4}}{\lambda^2}}
\fr{4}{9}+\fr{2(1-\lambda)}{(2-2\lambda-\lambda^2)^2}+{}\right.\\
\left.{}+
\fr{(1+\lambda^2)^{1-p/4}}{\lambda^2}+
\fr{3+\lambda}{3\lambda^2}
e^{-\lambda^2(3-\lambda)p/(12+4\lambda)} \right)\,.
\end{multline*}
 Чтобы оценить~$I_1$,  надо найти равномерную оценку для  
$F_p(x) - \IP(x)$.
С этой целью будем использовать два подхода.
Первый подход очень прост. Используя  результат из~[3]
\begin{multline*}
\sup\limits_{x \in R} |F_p(x) - \IP(x)|={}\\
{}  = \sup\limits_{x \in R} \left|F_p(x) - \left(\IP_p(x) -
  \fr{\sqrt{2}(1-x^2)}{3\sqrt{p}}\Ip(x)\right)\right|\leq{}\\
{}  \leq \sup\limits_{x \in R} \left|F_p(x) -\IP_p(x) \right|+
  \fr{\sqrt{2}}{3\sqrt{p}}\,\sup\limits_{x \in R}\left|(1-x^2)\Ip(x)\right|\leq{}\\
{}\le D_0(\lambda, p) + \fr{1}{3\sqrt{p \pi}}\,,
\end{multline*}
получаем лемму:

\smallskip

\noindent
\textbf{Лемма 1.} %\begin{lemma}
\textit{Для всех}
 $\lambda \in \left(0,\sqrt{3}-1\right)$  и целых $p >1$
\begin{equation*}
\sup\limits_{x \in R} |F_p(x) - \IP(x)| \le D_1(\lambda,p)\,,
\end{equation*}
где
\begin{multline*}
D_1(\lambda, p)= \fr{2}{\pi p}\left (\fr{4}{9} +\fr{2(1-\lambda)}{(2-2\lambda-\lambda^2)^2}
+{}\right.{}\\
{}+\fr{(1+\lambda^2)^{1-p/4}}{\lambda^2}+{}\\
\left.{}+\fr{3+\lambda}{3\lambda^2}\,
e^{-\lambda^2(3-\lambda)p/(12+4\lambda)} \right )
+ \fr{1}{3\sqrt{p \pi}}\,.
\end{multline*}

\smallskip

Второй подход основан на модификации результата из~[3].
Пусть~$f_p(t)$ и~$g(t)$~--- характеристические функции распределений~$F_p(x)$ и~$\IP(x)$ соответственно.
Положим $p^*=p/2$. Имеем
\begin{equation*}
f_p(t) = \exp(-it \sqrt{p^*}) \left(1 -\fr{it}{\sqrt{p^*}}\right)^{-p^*}\!\,\!;\enskip
g(t) = e^{-t^2/2}\,.
\end{equation*}
Из формулы обращения для характеристических функций получаем
\begin{equation*}
|F_p(x) - \IP(x)| \le \fr{1}{2\pi} (I_1 + I_2 +I_3)\,,
\end{equation*}
где
\begin{align*}
I_1 &= \int\limits_{|t|\; < \;\lambda \sqrt{p^*}} \fr{1}{|t|}\left| f_p(t) - e^{-t^2/2}\right|\,dt\,;\\
I_2 &= \int\limits_{|t|\; \ge \;\lambda \sqrt{p^*}} \fr{|f_p(t)|}{|t|}\,dt\,;\\
I_3 &= \int\limits_{|t| \;\ge\; \lambda \sqrt{p^*}} \fr{e^{-t^2/2}}{|t|}\,dt\,.
\end{align*}
Используя ту часть комплексной функции $\tau (z) =$\linebreak $= \log (1- z )$,  для которой $\tau(0)=0$,
запишем:
\begin{multline*}
f_p(t) = \exp \left\{-it \sqrt{p^*} -p^* \log\left(1 - \fr{it}{\sqrt{p^*}}\right)\right \} ={}\\
{}=
\exp \left\{- \fr{t^2}{2} - \fr{it^3}{3\sqrt{p^*}} + p^* R_{p^*}(t) \right \}={}\\
{}       = \exp \left\{- \fr{t^2}{2} - \fr{it^3}{3\sqrt{p^*}} \right \}+ S_{p^*}(t)\,,
\end{multline*}
где
\begin{align*}
 R_{p^*}(t) &= - \log \left(1 - \fr{it}{\sqrt{p^*}}\right) -\fr{it}{\sqrt{p^*}}
 - \fr{(it)^2}{2p^*} -\fr{(it)^3}{3{p^*}^{3/2}}\,;\\
 S_{p^*}(t) &= \exp\left\{-\fr{t^2}{2} - \fr{it^3}{3\sqrt{p^*}}\right\}(\exp{p^* R_{p^*}(t)} -1)\,.
\end{align*}
В~[3] для  $|t| < \lambda \sqrt{p^*}$  показано, что
\begin{equation*}
\left\vert p^* R_{p^*}(t)\right\vert \leq \fr{|t|^4}{4p^*\left(1 -|t|/\sqrt{p^*}\right)}
\end{equation*}
и
\begin{multline*}
\left\vert S_{p^*}(t)\right\vert \leq e^{-t^2/2} \left\vert \exp \left\{p^* R_{p^*}(t)\right\} - 1\right\vert \leq{}\\
{}\leq e^{-t^2/2} \fr{|t|^4}{4p^* (1-\lambda)}\,
             \exp \left\{\fr{|t|^2 \lambda^2}{4p^*(1 -\lambda)} \right\}={}\\
{}             =  \fr{|t|^4}{4p^* (1-\lambda)} e^{-(-a|t|^2)/2}
\end{multline*}
с $a = 1 - \lambda^2 /(2(1-\lambda))$.
Следовательно,
\begin{equation*}
I_1 \le I_{11} +I_{12}\,,
\end{equation*}
где
\begin{align*}
I_{11} &= \int\limits_{|t| \;<\; \lambda \sqrt{p^*}} \fr{e^{-t^2/2}}{|t|}
\left| e^{-it^3/(3\sqrt{p^*})} - 1\right|\,dt\,;\\
I_{12} &= \int\limits_{|t|\; <\; \lambda \sqrt{p^*}} \fr{|S_{p^*}|}{|t|}\,dt\,.
\end{align*}
Используя неравенство $|e^{-iz} -1| \le |z|$, получаем
$$
I_{11}\leq \sqrt{\fr{\pi}{2}}\,\fr{2}{3\sqrt{p^*}}
$$
и
\begin{multline*}
I_{12} \le \fr{1}{4p^*(1-\lambda)}\int\limits_{|t|\; <\; \lambda \sqrt{p^*}} |t|^3 e^{-at^2/2} \,dt
 \leq{}\\
 {}\leq \fr{2}{4p^*(1-\lambda)} \int\limits^{\infty}_{0} t^3 e^{-at^2/2}\,dt ={}\\
{}=\fr{4}{4p^*(1-\lambda)a^2} =\fr{2}{p(1-\lambda)a^2}\,,
\end{multline*}
где $a = 1 - \lambda^2/(2-2\lambda)$.
Заметим, что $|f_p (t)| =$\linebreak
$= (1 + t^2/p^*)^{-p^*/2}$. Оценку для~$I_2$ непосредственно заимствуем из~[3]:
\begin{multline*}
I_2 \leq \int\limits_{|t|\; \ge\; \lambda \sqrt{p^*}} \fr{1}{|t|}\, \fr{1}{(1 +t^2/p^*)^{p^*/2}}\,dt \leq{}\\
{} \leq \fr{1+ \lambda^2}{\lambda^2} \int\limits_{u \;\ge\; \lambda^2} (1+u)^{-1-m/2}\,du ={} \\
                   =\fr{4\left(1+ \lambda^2\right)}{\lambda^2 p} \left(1+\lambda^2\right)^{-p/4}\,.
\end{multline*}
Рассуждая аналогично~[6] и используя интегрирование по частям, получаем
\begin{multline*}
I_3 \leq 2 \int\limits_{|t|\; \ge\; \lambda \sqrt{p^*}} \fr{1}{t} e^{-t^2/2}\,dt ={}\\
{}= \fr{2}{\lambda^2 p^*} \exp \left(-\fr{\lambda^2 p^*}{2}\right)
                    -2 \int\limits^{\infty}_{\lambda \sqrt{p^*}} t^{-3} e^{-t^2/2}\,dt \leq{}\\
                    {}\leq \fr{4}{\lambda^2 p} \exp \left(-\fr{\lambda^2 p}{4}\right)\,.
\end{multline*}
Объединяя оценки для~$I_1$, $I_2$ и~$I_3$, завершаем доказательство следующей леммы:

\smallskip

\noindent
\textbf{Лемма 2.} %\begin{lemma}
\textit{Для всех $\lambda \in (0,\sqrt{3}-1)$ и целых $p >1$}
\begin{equation*}
\sup\limits_{x \in R} |F_p(x) - \IP(x)| \le D_2(\lambda,p)\,,
\end{equation*}
\textit{где}
\begin{multline*}
D_2(\lambda, p) =\fr{2}{\pi} \left( \sqrt{\fr{\pi}{p}}\,\fr{1}{6}
 +\fr{ 2 (1-\lambda)}{p(2-2\lambda-\lambda^2)^2}+ {}\right.\\
\left. {}+
 \fr{(1+
\lambda^2)}{\lambda^2 p} \left(1+\lambda^2\right)^{-p/4}
  + \fr{1}{\lambda^2 p} \exp \left(-\fr{\lambda^2 p}{4}\right)
  \vphantom{\sqrt{\fr{\pi}{p}}}
  \right)\,.
\end{multline*}

\smallskip

Покажем, что оценка леммы~2 точнее оценки леммы~1, т.\,е.\
\begin{equation}
D(\lambda,p) = \min\left(D_1(\lambda,p), D_2(\lambda,p)\right) = D_2(\lambda,p)\,.
 \label{D_2}
\end{equation}
Действительно, поскольку 
$$
\fr{3+\lambda}{3} > 1
$$ и 
$$
\fr{3-\lambda}{3+\lambda} < 1
$$ 
для положительных~$\lambda$, равенство~(\ref{D_2}) верно.

В табл.~1 для некоторых значений~$p$ даны величины~$\underset{\lambda}{\min}\, D(\lambda,p)$ 
и соответствующие значения~$\lambda$, при которых достигается минимум.

\bigskip

\begin{center} %tabl1
\noindent
\parbox{59mm}{{\tablename~1}\ \ \small{Минимумы $D(\lambda,p)$ при фиксированном~$p$}}
\end{center}
%\vspace*{2pt}

{\small
\begin{center}
\tabcolsep=16pt
\begin{tabular}{|c|c|c|}
\hline
$p$ & $\lambda$ &$\min\, D$\\
\hline
 10 & 0,551 & 0,460\hphantom{9}\\
 30  &  0,517 & 0,104\hphantom{9}\\
 50  &  0,483 & 0,0550\\
 100\hphantom{9} & 0,416 & 0,0276\\
 500\hphantom{9}  & 0,244 & 0,0094\\
\hline
\end{tabular}
\end{center}
}
%\vspace*{6pt}


%\bigskip
\addtocounter{table}{1}


{\small\frenchspacing
{%\baselineskip=10.8pt
\addcontentsline{toc}{section}{Литература}
\begin{thebibliography}{9}

\bibitem{1kav}
\Au{Hall~ P., Marron~J.\,S.,   Neeman~ A.} 
Geometric representation of high dimension,
low sample size data~// J.~Royal Statistical Soc. Series~B,  2005. Vol.~67. P.~427--444.

\bibitem{2kav}
\Au{Ulyanov~V.\,V., Wakaki~ H., Fujikoshi~Y.} 
Berry--Esseen bound for high dimensional asymptotic approximation of Wilks' lambda distribution~// 
Statist. Probab. Lett.,  2006. Vol.~76.  No.\,12. P.~1191--1200.

\bibitem{3kav}
\Au{Ulyanov~V.\,V., Christoph~G., Fujikoshi~Y.}  
On approximations
of transformed chi-squared distributions in statistical applications~//
 Siber.\ Math.\ J., 2006. Vol.~47. No.\,6. P.~1154--1166.

\bibitem{4kav}
\Au{Konishi S.} 
Asymptotic expansions for the distributions of functions of a correlation matrix~//
J.\ Multivariate Analysis, 1979. Vol.~9. No.\,2. P.~259--266.


\bibitem{5kav}
Справочник по специальным функциям~/ Под ред.\ М.~Абрамовица и И.~Стиган.~--- М.: Наука, 1979.

\label{end\stat}

\bibitem{6kav}
\Au{Dobric~V., Ghosh~B.\,K.} 
Some analogs of the Berry--Esseen bounds for first-order
Chebyshev--Edgeworth expansions~// Statist. Decisions, 1996. Vol.~14. No.\,4. P.~383--404.
 \end{thebibliography}
}
}

\end{multicols}