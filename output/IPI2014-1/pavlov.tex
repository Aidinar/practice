\def\stat{pavlov}

\def\tit{ОЦЕНКА НАДЕЖНОСТИ СЛОЖНЫХ СИСТЕМ С~ВОССТАНОВЛЕНИЕМ ПО~РЕЗУЛЬТАТАМ\\ ИСПЫТАНИЙ 
ЭЛЕМЕНТОВ}

\def\titkol{Оценка надежности сложных систем с~восстановлением по результатам испытаний 
элементов}

\def\autkol{И.\,В. Павлов}

\def\aut{И.\,В. Павлов$^1$}

\titel{\tit}{\aut}{\autkol}{\titkol}

%{\renewcommand{\thefootnote}{\fnsymbol{footnote}} \footnotetext[1]{Работа 
%выполнена при финансовой поддержке РФФИ (проект 11-01-00515а).}}

\renewcommand{\thefootnote}{\arabic{footnote}}
\footnotetext[1]{Московский государственный технический университет им.\ 
Н.\,Э.~Баумана, ipavlov@bmstu.ru} 



\Abst{Рассматривается задача доверительного оценивания показателей 
надежности для сложных сис\-тем с сетевой структурой с восстанавливаемыми 
элементами. Оценка надежности системы производится по результатам 
испытаний ее отдельных компонентов (элементов, подсистем). 
Существующие методы решения данной проблемы разработаны для 
относительно простых по\-сле\-до\-ва\-тель\-но-па\-рал\-лель\-ных структур и в 
предположении, что элементы системы имеют экспоненциальные 
распределения времени безотказной работы. Предлагается решение 
этой задачи для более общей модели сложных <<монотонных структур>>, а 
также для более общего случая <<стареющих>> (с монотонно возрастающей 
функцией интенсивности отказов) элементов системы. Предполагается, что в 
случае отказа элементы сис\-те\-мы восстанавливаются независимо от состояния 
других элементов. Кроме того, решение данной проблемы получено в 
естественной с прикладной точки зрения асимптотике, а именно для случая 
высокой надежности (быстрого восстановления) элементов сис\-темы.}

\KW{сложные системы; сетевые структуры; надежность; время безотказной 
работы; время восстановления; функция ресурса; функция интенсивности 
отказов}

\DOI{10.14357/19922264140103}

\vskip 14pt plus 9pt minus 6pt

      \thispagestyle{headings}

      \begin{multicols}{2}

            \label{st\stat}     
     
\section{Введение}

     Обеспечение надежности является одним из ключевых требований при 
проектировании и эксплуатации современных информационно-вы\-чис\-ли\-тель\-ных 
сис\-тем. Возникающая в связи с этим в инженерной 
практике задача оценки характеристик надежности по результатам 
испытаний сис\-те\-мы или ее отдельных компонентов (элементов,\linebreak под\-сис\-тем) 
является одной из актуальных и до настояще\-го времени в значительной 
степени нерешенных проблем. При этом основной практический интерес 
чаще всего представляет по\-стро\-ение гарантированных или доверительных 
оценок для тех или иных показателей надежности сис\-те\-мы. Задачи 
подобного типа возникают при оценке характеристик надежности сис\-тем 
связи и управления, банковских компьютерных сетей и ряда других сложных 
сис\-тем с сетевой структурой (см., например,~[1--8]). Аналогичные задачи 
возникают при развитии и наращивании сети, когда требуется дать 
гарантированный прогноз характеристик надежности для различных 
возможных вариантов сис\-те\-мы на основе статистической информации по 
элементам.
     
     Пусть в сис\-теме имеется $m$ различных элементов. С~точки зрения 
работоспособности состояние сис\-те\-мы в момент времени $t\hm\geq 0$ 
характеризуется вектором двоичных переменных $x_t\hm= [x_1(t), \ldots , 
x_m(t)]$, где $x_i(t)$~--- индикатор исправности $i$-го элемента: 
$x_i(t)\hm=1$, если в момент времени~$t$ $i$-й элемент исправен, 
$x_i(t)\hm=0$  в противном случае, $i\hm= 1, \ldots , m$. Работоспособность 
сис\-те\-мы в состоянии $x\hm= (x_1,\ldots , x_m)$ характеризуется 
<<структурной функцией>>: $\Phi(x)\hm=1$, если в состоянии~$x$  сис\-те\-ма исправна, и  
$\Phi(x)\hm=0$
в противном случае. Предполагается, что эта функция удовле\-тво\-ря\-ет 
естественному условию мо\-но\-тон\-ности $\Phi(x_1,\ldots ,x_m)\hm\geq \Phi(y_1, 
\ldots ,y_m)$,  если $x_i\hm\geq y_i$  при всех $i\hm= 1,\ldots , m$, другими 
словами, дополнительные отказы элементов не улучшают состояние 
сис\-те\-мы~\cite{1-pav, 2-pav}. Индикатор исправности сис\-те\-мы в момент 
времени~$t$ имеет вид: $X(t)\hm= \Phi(x_t)$.
{\looseness=-1

}

     Процесс функционирования $i$-го элемента $x_i(t)$  пред\-став\-ля\-ет 
собой альтернирующий процесс вос\-ста\-нов\-ле\-ния~\cite{3-pav}; другими 
словами, последовательность сменяющих друг друга независимых между 
собой (и от состояния других элементов) интервалов исправной работы и 
восстановления. 

Функции распределения интервалов исправной 
работы~$\xi_i$ и восстановления~$\eta_i$ для $i$-го элемента обозначим 
соответственно через $F_i(t)$ и $G_i(t)$, предполагая, что они имеют 
непрерывные плотности $f_i(t)$ и $g_i(t)$ и конечные вторые моменты. 
Обозначим также через $P_i(t)\hm=1-F_i(t)$ функцию надежности, $u_i\hm= 
\int\limits_0^{\infty} P_i(t)\,dt$~---  среднее время безотказной работы и 
$v_i\hm= \int\limits_0^{\infty} [1-G_i(t)]\,dt$~--- среднее время 
восстановления для $i$-го элемента, $i\hm=1, \ldots ,m$.
     
     Далее предполагается, что параметры $v_i$~--- средние времена 
восстановления (замены) элемен-\linebreak тов~--- известны и малы по сравнению со 
средним временем безотказной работы; другими словами, выполняется 
следующее условие высокой надежности (<<быстрого восстановления>>) 
элементов сис\-темы:
     \begin{equation}
     v_i \ll u_i\,,\quad i=1,\ldots , m\,.
     \label{e1-pav}
     \end{equation}

Задача заключается в оценке характеристик надеж\-ности системы по 
результатам испытаний элементов, полученным или в ходе испытаний 
отдельных элементов системы, или испытаний ее отдельных фрагментов, или 
системы в целом. При этом основной интерес для приложений чаще всего 
представляет построение гарантированных или доверительных оценок 
характеристик надежности системы. Методы решения данной проблемы в 
настоящее время разработаны для относительно простых последовательных 
и по\-сле\-до\-ва\-тель\-но-па\-рал\-лель\-ных структур для случая, когда элементы 
системы имеют экспоненциальные распределения времени безотказной 
работы~\cite{3-pav, 4-pav, 6-pav}. Далее дается решение для указанной выше 
более общей модели <<монотонных структур>>, а также для более общего 
случая, когда элементы системы имеют <<стареющие>> или 
ВФИ-рас\-пре\-де\-ле\-ния (с воз\-рас\-та\-ющей функцией интенсивности отказов 
$r_i(t)\hm= f_i(t)/P_i(t)$, $i\hm= 1,\ldots , m$) времени безотказной работы. 
Класс ВФИ-рас\-пре\-де\-ле\-ний является довольно общим и включает в себя 
целый ряд час\-то используемых параметрических семейств распределений, 
таких как экспоненциальное, Вейбулла, нормальное, распределение Эрланга, 
гам\-ма-рас\-пре\-де\-ле\-ние и~др. В~то же время предположение о старении 
элементов сис\-те\-мы является довольно естественным с физической точки 
зрения и часто используется в инженерной практике 
(см., например,~\cite{3-pav, 1-pav, 6-pav}). В~этом смысле выводы, основанные на 
предположении о старении элементов системы, являются значительно более 
корректными.

\vspace*{-6pt}

\section{Нижняя доверительная граница для~коэффициента 
готовности системы}

     Стандартный показатель надежности для систем с восстанавливаемыми 
элементами~--- коэффициент готовности (вероятность застать систему в 
исправном состоянии в стационарном режиме при $t\hm\to\infty$)~--- для 
данной модели имеет вид~\cite{1-pav, 2-pav}:
     $$
     K=K(u) = \sum \prod\limits_{i=1}^m k_i^{x_i}(u_i) \left[ 1-
k_i(u_i)\right]^{1-x_i}\,,
     $$
где $u=(u_1, \ldots , u_m)$~--- вектор параметров элементов, а сумма берется 
по всем $x\hm= (x_1,\ldots , x_m)$ таким, что $\Phi(x) \hm= 1$, $k_i\hm= 
k_i(u_i) \hm = u_i/(u_i+v_i)$~--- коэффициент готовности $i$-го элемента, 
$i\hm= 1,\ldots ,m$.
     
     Далее будем предполагать, что результаты испытаний по различным 
элементам системы пред\-став\-ле\-ны в виде стандартных статистических 
выборок с цензурированием типа $\left [N_i\, \mathrm{Б}\, r_i\right]$ (в обозначениях 
     книги~\cite{3-pav}), т.\,е.\ испытывались $N_i$ элементов $i$-го типа 
до наблюдения $r_i\hm\leq N_i$ отказов, в результате чего наблюдались 
последовательные моменты отказов $0\hm< t_1^{(i)} <  \cdots < 
t_{r_i}^{(i)}$, $i\hm=1, \ldots , m$. В~частном случае при $r_i\hm =N_i$ 
данная схема испытаний содержит классическую схему полной выборки 
объема~$N_i$, $i\hm= 1,\ldots ,m$.
     
     Обозначим через 
     
\noindent
     $$
     R_i(t) \hm = -\ln P_i(t) \hm= \int\limits_0^t r_i(z)\, dz$$ 
     функцию ресурса для $i$-го элемента, $V_i$~--- множество 
всех функций ресурса $R_i(t)$  выпуклых вниз по $t\hm\geq 0$ (при всех~$t$  
таких, что $P_i(t)\hm >0$, $i\hm= 1,\ldots , m$), $R\hm = \left\{ R_i(t),\ldots , 
R_m(t)\right\}$~--- вектор функций ресурса по всем элементам, $V\hm= 
V_1\times V_2 \times \cdots \times V_m$~--- множество всех~$R$ таких, что 
функции $R_i(t)$ выпуклы вниз по $t\hm \geq 0$. Далее будем предполагать, 
что $R\hm\in V$, т.\,е.\ все элементы сис\-те\-мы имеют 
     ВФИ-рас\-пре\-де\-ле\-ния (с воз\-рас\-таю\-щей функцией интенсивности 
отказов $r_i(t)\hm= R_i(t)$, $i\hm= 1,\ldots , m$) времени безотказной работы. 
     
     Пусть $S_i$~--- суммарное время работы (наработка) элементов $i$-го 
типа на испытаниях: 
$$
S_i\hm= t_1^{(i)}+ t_2^{(i)} +\cdots + t^{(i)}_{r_i} 
+ (N_i\hm-r_i) t_{r_i}^{(i)}\,,\enskip i\hm= 1,\ldots , m\,;
$$
$S\hm= (S_1,\ldots , 
S_m)$ и $P_R(S)$~--- вероятностное распределение на множестве результатов 
испытаний~$S$ при данном $R\hm\in V$. Требуется построить нижнюю 
$\gamma$-до\-ве\-ри\-тель\-ную границу для коэффициента го\-тов\-ности 
     сис\-те\-мы $K\hm= K(u)$, т.\,е.\ функцию результатов испытаний 
$\underline{K}(S)$  такую, что 
$$
P_R\left\{ \underline{K}(S)\leq 
K(u)\right\} \geq \gamma\,,\enskip  R\in V\,.
$$
     
     В~\cite{9-pav, 10-pav} ранее была получена нижняя 
     $\gamma$-до\-ве\-ри\-тель\-ная граница~$\underline{u}_i$ в классе 
ВФИ-рас\-пре\-де\-ле\-ний для параметра~$u_i$ (среднего времени 
безотказной работы) одного отдельно взятого $i$-го элемента:
     \begin{equation}
     P_R\left\{ \underline{u}_i\leq u_i\right\} \geq \gamma\,,\quad R\in V\,.
     \label{e2-pav}
     \end{equation}
Здесь $\underline{u}_i = b_i(\gamma)S_i$, где 
$$
b_i(\gamma)= \fr{ 1-\exp \left[ -
\Delta_{1-\gamma}(r_i-1)/N_i\right]}{\Delta_{1-\gamma}(r_i-1)}\,;
$$ 
$\Delta_{1-\gamma}(d)$~--- решение уравнения 
$$
\exp \left( - \Delta\right)\sum\limits_{j=0}^d \fr{\Delta^j}{j!}=1-\gamma,\enskip 
i\hm=1,\ldots , m. $$

Коэффициент готовности системы $K(u)\hm= K(u_1, u_2,\ldots ,u_m)$ 
монотонно возрастает по каж\-до\-му аргументу~$u_i$, что соответствует 
естественному свойству~--- улучшению характеристик системы при 
улучшении параметров надежности элементов. Поэтому нижнюю 
границу~$\underline{K}$, вообще говоря, можно найти, просто подставив 
нижние доверительные оценки для параметров отдельных 
элементов~$\underline{u}_i$ в функцию $K(u)$, полагая $\underline{K}\hm= 
K(\underline{u}_1, \underline{u}_2,\ldots , \underline{u}_m)$. Тем не менее, 
если число элементов системы велико, такой подход, использующий только 
указанное свойство монотонности, оказывается малопригодным. 
Действительно, при этом справедливы неравенства (см.\ 
также~\cite{11-pav, 12-pav}):
\begin{multline*}
P_R\left\{ K(\underline{u}_1, \underline{u}_2, \ldots , \underline{u}_m) \leq 
K(u_1, u_2, \ldots , u_m)\right\} \geq{}\\
{}\geq P_R\left\{ \mathop{\bigcap}\limits_{i=1}^m 
\left( \underline{u}_i\leq u_i\right)\right\}={}\\
{}=\prod\limits_{i=1}^m P_R\left\{ \underline{u}_i\leq u_i\right\} > \gamma^m\,, 
\enskip R\in V\,,
\end{multline*}
откуда видно, что коэффициент доверия для такой процедуры (метод Ллойда 
и Липова, или <<метод прямоугольника>> в терминологии~[11--13]) быстро 
убывает с ростом размерности задачи (числа элементов системы)~$m$.

     Обозначим через $Q(u)\hm=1-K(u)$ вероятность отказа системы и 
$q_i(u)\hm=1-k_i(u)$~--- вероятность отказа $i$-го элемента, $i\hm=1,\ldots 
,m$. В~случае высокой надежности (быстрого восстановления) 
элементов~(\ref{e1-pav}) справедлива приближенная 
     формула~\cite{1-pav, 2-pav}:
     \begin{equation}
     Q(u)\cong \sum\limits_{j=1}^M \prod_{i\in B_j} q_i(u_i) \cong 
\sum\limits_{j=1}^M \prod_{i\in B_j} \left( \fr{v_i}{u_i}\right)\,,
     \label{e3-pav}
     \end{equation}
где $B_j\subset (1,\ldots ,m)$~--- набор индексов $j$-го минимального сечения 
системы, $j\hm=1,\ldots ,M$ ($M$~--- чис\-ло минимальных сечений). 
Правая часть~(\ref{e3-pav}) при этом дает верхнюю оценку для $Q(u)$. 
Довери\-тельное оценивание коэффициента готовности сис\-те\-мы $K(u)$ в 
приближении~(\ref{e1-pav}), таким образом,\linebreak сводится к построению верхней 
доверительной границы для функции вида~(\ref{e3-pav}) от вектора 
$u\hm=(u_1,\ldots ,u_m)$ неизвестных параметров надежности элементов. 
     
     В соответствии с~(\ref{e2-pav}) для каждого типа элементов $i\hm= 
1,\ldots ,m$ справедливы неравенства:
     \begin{equation*}
     P_R\left\{ u_i\geq b_i(\gamma)S_i\right\} \geq \gamma\,,\quad R\in V\,,
%     \label{e4-pav}
     \end{equation*}
при любом $0\hm<\gamma\hm<1$. При данных фиксированных значениях 
результатов испытаний $S\hm= (S_1,\ldots , S_m)$ введем (независимые) 
случайные величины (с.в.)\ $\tilde{u}_i$, $i\hm=1,\ldots ,m$, такие, что 
\begin{equation}
P\left( \tilde{u}_i\geq b_i(\gamma) S_i\right\} =\gamma
\label{e5-pav}
\end{equation}
при всех $0<\gamma<1$, $i\hm=1,\ldots ,m$. В~соответствии с~(\ref{e5-pav}) 
функция надежности с.в.~$\tilde{u}_i$ имеет вид:
$$
\tilde{P}_i(t) = P\left\{ \tilde{u}_i\geq t\right\} = h_i\left( \fr{t}{S_i}\right)\,,
$$
где $h_i(z)$~--- функция, обратная $b_i(\gamma)$, $0\hm< \gamma\hm <1$. 
Случайная величина~$\tilde{u}_i$ может быть также задана при данном 
фиксированном~$S_i$ как $\tilde{u}_i\hm= S_i b_i(\gamma)$, где 
$\gamma$~--- с.в., равномерно распределенная на интервале $(0,\,1)$. 
     
     Пусть имеется некоторая функция $Q(u)\hm= Q(u_1, u_2, \ldots ,u_m)$, 
монотонно убывающая по\linebreak каж\-до\-му аргументу $u_i\hm> 0$, $i\hm=1,\ldots 
,m$. При данных фиксированных значениях результатов наблюдений $S\hm= 
(S_1,\ldots , S_m)$ и данном фиксированном значении коэффициента доверия 
$0\hm <\gamma\hm< 1$ определим далее верхнюю границу $\overline{Q}\hm= 
\overline{Q(S_1, \ldots , S_m)}$ как квантиль уровня~$\gamma$ для 
случайной величины $\tilde{Q}\hm= Q\left( \tilde{u}_1, \ldots , 
\tilde{u}_m\right)$, т.\,е.\ из условия 
     $$
     P\left\{ Q\left( \tilde{u}_1,\ldots ,. \tilde{u}_m\right) \leq 
\overline{Q}\right\} =\gamma
     $$
(численное значение~$\overline{Q}$ далее достаточно просто может быть 
найдено на основе стандартного метода Мон\-те-Кар\-ло), что является 
обобщением фидуциального подхода при построении доверительных границ 
для функций от~$m$ неизвестных па\-ра\-мет\-ров экспоненциальных 
распределений~\cite{6-pav, 14-pav, 15-pav}. 

В~одномерном случае 
$\gamma$-фи\-ду\-ци\-аль\-ная граница параметра одновременно является 
$\gamma$-до\-ве\-ри\-тель\-ной, что следует непосредственно из ее 
определения. 

В~многомерном случае при $m\hm>1$ это, вообще говоря, не 
обязательно так и зависит от того или иного вида оцениваемой функции 
многомерного параметра. 

Теорема~1 дает достаточные условия, при которых 
определенная таким образом величина $\overline{Q}\hm= \overline{Q}(S_1, 
\ldots , S_m)$ служит верхней доверительной\linebreak\vspace*{-12pt}

\pagebreak

\noindent
 границей с коэффициентом 
доверия не меньше~$\gamma$ для функции $Q(u)\hm= Q(u_1, \ldots , u_m)$ в 
классе стареющих распределений~$V$.
     
     \medskip
     
     \noindent
     \textbf{Теорема~1.} \textit{Пусть выполняются условия}:
     \begin{enumerate}[(1)] 
  \item \textit{функция $Q(u)\hm= Q(u_1, u_2, \ldots , u_m)$ монотонно 
убывает по каждому $u_i\hm> 0$, $i\hm= 1,\ldots ,m$};
  \item \textit{функция $Q\left( e^{-z_1}, e^{-z_2}, \ldots , e^{-z_m}\right)$ 
выпукла вниз по $z\hm= (z_1, \ldots ,z_m)\hm\in E_m$}.
  \end{enumerate}
     \textit{Тогда справедливо неравенство}
     $P_R\left\{ \overline{Q}\geq Q(u)\right\}\hm\geq\gamma$ \textit{при всех} 
$R\hm\in V$.
     
     \medskip
     
     \noindent
     Д\,о\,к\,а\,з\,а\,т\,е\,л\,ь\,с\,т\,в\,о\,.\ \ В~соответствии с~(\ref{e5-pav}) 
справедливо неравенство
     \begin{equation}
     P_R\left\{ \fr{u_i}{S_i}\geq b_i(\gamma)\right\} \geq \gamma
     \label{e6-pav}
     \end{equation}
при всех $0<\gamma <1$, $R\hm\in V$, $i\hm=1, \ldots ,m$. Введем с.в.\
 $\xi_i = u_i/S_i$, распределение которой зависит от $R\hm\in V$, 
$i\hm= 1,\ldots ,m$. Введем также независимые с.в.\ $\eta_1, \ldots ,\eta_m$, 
где $\eta_i\hm\geq 0$ имеет функцию распределения $P\left\{ \eta_i<t\right\} 
\hm= 1\hm- h_i(t)$, $i\hm= 1,\ldots , m$.
     
     В соответствии с~(\ref{e6-pav}) с.в.~$\xi_i$ стохастически больше, чем 
с.в.~$\eta_i$, в следующем смысле:
     $$
     P_R\left\{ \xi_i\geq t\right\} \geq P\left\{ \eta_i\geq t\right\}
     $$
при всех $t\geq 0$, $R\hm\in V$, $i\hm=1,\ldots , m$. Отсюда следует, что для 
любой функции $\varphi(z_1, \ldots , z_m)\hm \geq 0$, монотонно убывающей 
по каждому аргументу $z_i\hm\geq 0$, справедливо аналогичное неравенство 
$$
P_R\left\{ \varphi(\xi_1, \ldots ,\xi_m)\leq t\right\} \geq P\left\{ \varphi(\eta_1, 
\ldots ,\eta_m)\leq t\right\}
$$
при всех $t\geq 0$, $R\hm\in V$. Полагая 
$$
\varphi(z_1, \ldots ,z_m) =\prod\limits_{i=1}^m \left( \fr{1}{z_i}\right)^{a_i}\,,
$$
где $a_i\geq 0$, $i\hm=1,\ldots , m$,~--- произвольные положительные 
константы, получаем: 
\begin{equation}
P_R\left\{ \prod\limits_{i=1}^m \left( \fr{S_i}{u_i} \right)^{a_i} \leq t\right\} 
\geq P\left\{ \prod\limits_{i=1}^m \left( \fr{1}{\eta_i}\right)^{a_i}\leq t\right\}
\label{e7-pav}
\end{equation}
при всех $t\geq 0$, $R\hm\in V$. Обозначим через $t_\gamma\hm= 
t_\gamma(a)$ квантиль уровня $0\hm<\gamma\hm<1$  для с.в.\ 
$\prod\limits_{i=1}^m (1/\eta_i)^{a_i}$,  где $a\hm= (a_1, \ldots , a_m)$. 
Из~(\ref{e7-pav}) далее получаем 
$$
P_R \left\{ \prod\limits_{i=1}^m \left( \fr{S_i}{u_i}\right)^{a_i} \leq 
t_\gamma\right\} \geq P\left\{ \prod\limits_{i=1}^m \left( 
\fr{1}{\eta_i}\right)^{a_i}\!\leq t_\gamma \right\} =\gamma,
$$
а значит, справедливо неравенство
\begin{equation}
P_R\left\{ \prod\limits_{i=1}^m \left( \fr{1}{u_i}\right)^{a_i} \leq t_\gamma 
\prod\limits_{i=1}^m \left( \fr{1}{S_i}\right)^{a_i}\right\} \geq \gamma
\label{e8-pav}
\end{equation}
при любом $R\in V$ и любых положительных коэффициентах $a_i\hm\geq 0$, 
$i\hm=1, \ldots , m$.
     
     Введем замену переменных $u_i\hm =e^{-z_i}$, $i\hm= 1, \ldots , m$. 
Из~(\ref{e8-pav}) следует, что приведенная выше процедура дает верхнюю 
доверительную границу с коэффициентами доверия не меньше~$\gamma$ 
для любой функции вида
     $$
     H(u) =c_0 \prod\limits_{i=1}^m \left( \fr{1}{u_i}\right)^{a_i}\,,
     $$
где $c_0\hm>0$, $a_i\geq 0$, $i\hm= 1,\ldots ,m$,  или в переменных $z\hm= 
(z_1, \ldots , z_m)$ для любой линейной функции вида
\begin{equation}
L(z,a) =a_0+\sum\limits_{i=1}^m a_i z_i\,,
\label{e9-pav}
\end{equation}
где вектор коэффициентов $a\hm= (a_0, a_1, \ldots , a_m) \hm\in A\hm = \{ a:\ -
\infty<a_0 <\infty,\ a_i \geq 0\,,\ i=1, \ldots , m\}$. Обозначим через 
$\overline{L}\hm= \overline{L(S,a)}$ соответствующую верхнюю 
$\gamma$-до\-ве\-ри\-тель\-ную границу для линейной функции $L(z,a)$: 
$P_R\left\{ \overline{L}(S,a)\hm\geq L(z,a)\right\}\hm\geq \gamma$ при всех 
$R\hm\in V$, $a\hm\in A$. Введем функцию
\begin{equation}
f(z) =Q\left( e^{-z_1}, \ldots , e^{-z_m}\right)\,.
\label{e10-pav}
\end{equation}

Рассмотренная выше процедура построения верхней границы 
$\overline{Q}\hm= \overline{Q(S)}$ как $\gamma$-кван\-ти\-ля с.в.\ 
$Q(\tilde{u}_1, \ldots , \tilde{u}_m)$ эквивалентна построению верхней 
границы $\overline{f}\hm= \overline{f}(S)$ для функции $f(z)$ как 
$\gamma$-кван\-ти\-ля для с.в.\ $f(\tilde{z}_1, \ldots , \tilde{z}_m)$, где 
$\tilde{z}_i\hm= \ln (1/\tilde{u}_i)$, $i\hm= 1,\ldots , m$. При этом, очевидно, 
$\overline{f}(S)\hm= \overline{Q(S)}$ при любом $S\hm= (S_1,\ldots ,S_m)$. 
В~условиях теоремы функция~(\ref{e10-pav}) монотонно возрастает по 
каждому $z_i$ и выпукла вниз по $z\hm= (z_1,\ldots ,z_m)\hm\in E_m$. 
Следовательно, она может быть представлена через базовую систему 
линейных функций~(\ref{e9-pav}) в виде
\begin{equation}
f(z) =\max\limits_{a\in A_f} L(z,a)
\label{e11-pav}
\end{equation}
в каждой точке $z\in E_m$, где максимум берется по $a\hm\in A_f\subset A$. 
Отсюда следует $f(z)\hm\geq L(z,a)$ при всех $z\hm\in E_m$, $a\hm\in A_f$ и, 
соответственно, 
$$
\overline{f}(S)\geq \overline{L}(S,a)\,,\quad a\in A_f\,,
$$
при любом $S\hm= (S_1, \ldots , S_m)$. Значит, 
\begin{equation}
\overline{f}(S)\geq \overline{L}(S) = \max\limits_{a\in A_f} \overline{L}(S,a)\,.
\label{e12-pav}
\end{equation}

Пусть $z(a)\in A_f$~--- точка, в которой достигается максимум 
в~(\ref{e11-pav}):
\begin{equation}
f(z) =\max\limits_{a\in A_f} L(z,a) =L[z,a(z)]\,.
\label{e13-pav}
\end{equation}
Зафиксируем $R\in V$  и соответственно параметры $u_i\hm= u_i(R_i) \hm= 
\int\limits_0^\infty \exp \left[ -R_i(t)\right]\,dt$ и $z_i\hm= z_i(R_i) \hm= \ln \left[ 
1/u_i(R_i)\right]$, $i\hm=1,\ldots ,m$. Из~(\ref{e12-pav}), (\ref{e13-pav}) далее 
следуют неравенства 
\begin{multline*}
P_R \left\{ \overline{f}(S)\geq f(z)\right\} \geq P_R\left\{ \overline{L}(S)\geq 
f(z)\right\} \geq{}\\
{}\geq
 P_R\left\{ \overline{L}[S,a(z)]\geq f(z)\right\} ={}\\
{}= P_R\left\{ \overline{L}[S,a(z)]\geq L[z,a(z)]\right\}\geq \gamma
\end{multline*}
при любом $R\hm\in V$. Теорема доказана.

\medskip

     Вероятность отказа системы~(\ref{e3-pav}) в переменных $z\hm= 
(z_1,\ldots ,z_m)$ имеет вид:
     \begin{multline}
     f(z) = Q \left(e^{-z_1},\ldots , e^{-z_m}\right) = {}\\
     {}=\sum\limits_{j=1}^M C_j 
\exp \left( \sum\limits_{i\in B_j} z_i\right)\,,
     \label{e14-pav}
     \end{multline}
где коэффициенты 
$$
C_j=\prod\limits_{i\in B_j} v_i\,,\quad j=1,\ldots , M\,.
$$
     
     Функция~(\ref{e14-pav}) выпукла вниз по $z\hm= (z_1,\ldots ,z_m)$ (в 
том числе и в случае, когда различные минимальные сечения системы $B_j$ 
могут пересекаться, что имеет место для систем со сложной структурой). Тем 
самым функция $Q(u)$ в~(\ref{e3-pav}) удовлетворяет обоим условиям 
приведенной выше теоремы~1 (справедливость первого условия в данном 
случае очевидна). Таким образом, определенная выше 
величина~$\overline{Q}$ дает верхнюю доверительную границу с 
коэффициентом доверия не меньше заданной величины~$\gamma$  для 
вероятности отказа системы $Q(u)$ для общей модели сложных 
<<монотонных структур>> и при довольно общих непараметрических 
предположениях о том, что элементы системы~--- <<стареющие>> (с 
монотонно возрастающей функцией интенсивности отказов). 
Соответственно, величина $\underline{K}\hm=1-\overline{Q}$ дает при этом 
нижнюю 
     $\gamma$-до\-ве\-ри\-тель\-ную границу для коэффициента готовности 
системы. (Заметим также, что, как видно из доказательства теоремы~1, 
построенная выше доверительная граница $\overline{Q}\hm=\overline{Q(S)}$  
может быть улучшена, если взять в качестве такой границы величину 
$\overline{L}\hm=\overline{L(S)}$, но вычисление которой при каждом 
данном значении вектора результатов испытаний $S\hm=(S_1,\ldots , S_m)$ 
окажется значительно более сложным.) 

     В качестве примера рассмотрим систему из $m\hm=11$ элементов с 
сетевой структурой, изображенной на рисунке. Результаты испытаний $N_i$, 
$r_i$, $S_i$ по элементам различных типов $i\hm=1,\ldots ,m$ приводятся в 
таблице. В~этом случае нижняя $\gamma$-до\-ве\-ри-\linebreak\vspace*{-12pt}

\columnbreak

 \begin{center}
 \mbox{%
 \epsfxsize=74.155mm
 \epsfbox{pav-1.eps}
 }
 \end{center}
\begin{center}
{\small{Система с сетевой структурой из $m\hm=11$ элементов}}
\end{center}


%\pagebreak

\vspace*{6pt}




{\small
\begin{center}
\begin{tabular}{|c|c|c|c|}
\multicolumn{4}{c}{Результаты испытаний элементов}\\[6pt]
\hline
\ \ \ \ $i$\ \ \ \ &\ \ \ \ $N_i$\ \ \ \ &\ \ \ \ $r_i$\ \ \ \ & $S_i$\\
\hline
1&1&1&24,2\\
2& 2&1&27,9\\
3& 3& 2& 21,1\\
4& 2& 1& 26,3\\
5& 2&2&25,4\\
6& 1&1&25,2\\
7& 3&2& 23,6\\
8& 2&2&24,2\\
9& 1&1&28,8\\
10\hphantom{9}& 2& 1& 25,2\\
11\hphantom{9}& 3&3&22,3\\
\hline
\end{tabular}
\end{center}
}

\vspace*{9pt}

      


\noindent
тель\-ная граница (при 
$\gamma\hm=0{,}9$) для коэффициента
 готовности системы, вычисляемая на 
основе приведенной выше теоремы~1, $\underline{K}\hm= 1-
\overline{Q}\hm=0{,}97$. При этом аналогичная нижняя доверительная 
граница, вычисляемая указанным выше упрощенным методом (Ллойда и 
Липова), основанным на непосредственном использовании частных 
доверительных границ для параметров отдельных элементов, равна 
$\underline{K}\hm=0{,}89$. 



\section{Заключение}

      Таким образом, получено решение задачи доверительного оценивания 
по результатам испытаний элементов системы одного из основных 
показателей надежности~--- коэффициента готовности для довольно общей 
модели сложных систем (с произвольной <<монотонной структурой>>) с 
восстанавливаемыми <<стареющими>> элементами. Указанное решение 
получено в естественном с точки зрения приложений приближении~--- для 
случая высокой надежности (<<быстрого восстановления>>) элементов 
системы, а также в предположении независимого восстановления элементов. 
Существенный\linebreak интерес с прикладной точки зрения представляет также 
дальнейшее обобщение полученных результатов как для более общих 
моделей систем (в том чис\-ле с зависимым восстановлением элементов), так и 
по отношению к другим часто используемым показателям надежности 
систем с восстановлением~--- коэффициенту оперативной готовности, 
среднему времени безотказной работы системы и~др.

{\small\frenchspacing
{%\baselineskip=10.8pt
\addcontentsline{toc}{section}{References}
\begin{thebibliography}{99}

\bibitem{3-pav} %1
\Au{Гнеденко Б.\,В., Беляев Ю.\,К. Соловьев~А.\,Д.} Математические 
методы в теории надежности.~--- М.: Наука, 1965. 524~с. 
\bibitem{1-pav} %2
\Au{Барлоу Р., Прошан Ф.} Математическая теория надежности~/ Пер. с 
англ.~--- М.: Радио и связь, 1969. 488~с. (\Aue{Barlow, R., Proschan~F}. 
{Mathematical theory of reliability}.~--- N.Y.: John Wiley, 1965. 497~p.)

\bibitem{5-pav} %3
\Au{Васильев Н.\,С.} Об одной модели развития сети связи~// Известия 
Российской академии наук. Теория и системы управления, 1985. №\,6. 
С.~227--234.
\bibitem{4-pav} %4
\Au{Павлов И.\,В.} Приближенно оптимальные доверительные границы 
для показателей надежности сис\-тем с восстановлением~// Известия 
Российской академии наук. Теория и системы управления, 1988. №\,3. 
С.~109--116.

\bibitem{2-pav} %5
\Au{Павлов И.\,В., Ушаков И.\,А.} Вычисление показателей надежности для 
сложных систем с восстанавливаемыми элементами~// Известия 
Российской академии наук. Теория и системы управления, 1989. №\,6. 
С.~170--176.

\bibitem{6-pav}
\Au{Gnedenko B.\,V., Pavlov I.\,V., Ushakov~I.\,A.} Statistical reliability 
engineering.~--- N.Y.: John Wiley, 1999. 517~p.
\bibitem{7-pav}
\Au{Коновалов М.\,Г.} Организация работы вычислительного комплекса с 
помощью имитационной модели и адаптивных алгоритмов~// 
Информатика и её применения, 2012. Т.~6. Вып.~1. С.~37--48.
\bibitem{8-pav}
\Au{Павлов И.\,В.} Расчет и оптимизация некоторых характеристик для 
модели вычислительного комплекса~// Информатика и её применения, 
2012. Т.~6. Вып.~2. С.~59--62. 
\bibitem{9-pav}
\Au{Barlow R., Proschan F.} Tolerance and confidence limits for classes of 
distributions based on failure rate~// Ann. Math. Stat., 1966. Vol.~37. 
No.\,6. P.~1184--1195. 
\bibitem{10-pav}
\Au{Павлов И.\,В.} Доверительные границы в классе распределений с 
возрастающей функцией интен\-сив\-ности отказов~// Известия Российской 
академии наук. Теория и системы управления, 1977. №\,6. С.~72--84.
\bibitem{11-pav}
\Au{Ллойд Д., Липов М.} Надежность~/ Пер. с англ.~--- М.: Сов. радио, 
1964. 668~с. (\Aue{Lloyd D., Lipow M.} Reliability management, methods and 
mathematics.~--- Englewood Cliffs, N.J.: Prentice-Hall, 1962. 684~p.)
\bibitem{12-pav}
\Au{Беляев Ю.\,К.} Доверительные интервалы для функций от многих 
неизвестных параметров~// Докл.\ АН СССР, 1967. Т.~196. №\,4. С.~755--758.
\bibitem{13-pav}
\Au{Беляев Ю.\,К., Дугина Т.\,Н., Чепурин~Е.\,В.} Вычисление нижней 
доверительной оценки для веро\-ят\-ности без\-от\-каз\-ной работы сложных 
сис\-тем~// Известия Российской академии наук. Теория и системы 
управ\-ле\-ния, 1967. №\,2. С.~52--59.
\bibitem{14-pav}
\Au{Павлов И.\,В.} О корректности фидуциального подхода при 
построении доверительных границ для показателей надежности сложных 
сис\-тем~// Известия Российской академии наук. Теория и системы 
управления, 1981. №\,5. С.~46--52.
\bibitem{15-pav}
\Au{Павлов И.\,В.} О~фидуциальном подходе при вычислении 
доверительных границ для функций многих неизвестных параметров~// 
Докл. РАН, 1981. Т.~258. №\,6. С.~1314--1317.

\end{thebibliography}
} }

\end{multicols}

\hfill{\small\textit{Поступила в редакцию 24.12.13}}


\vspace*{12pt}

\hrule

\vspace*{2pt}

\hrule


\def\tit{ESTIMATION OF RELIABILITY OF~COMPLEX SYSTEMS WITH~RENEWAL BASED 
ON~ELEMENT TEST RESULTS}

\def\titkol{Estimation of reliability of~complex system with renewal based 
on~element test results}

\def\aut{I.\,V.~Pavlov }
\def\autkol{I.\,V.~Pavlov }


\titel{\tit}{\aut}{\autkol}{\titkol}

\vspace*{-9pt}

\noindent Bauman Moscow State Technical University, 5, 2nd Baumanskaya Str., Moscow 
105005, Russian Federation

 
\def\leftfootline{\small{\textbf{\thepage}
\hfill INFORMATIKA I EE PRIMENENIYA~--- INFORMATICS AND APPLICATIONS\ \ \ 2014\ 
\ \ volume~8\ \ \ issue\ 1}
}%
 \def\rightfootline{\small{INFORMATIKA I EE PRIMENENIYA~--- INFORMATICS AND APPLICATIONS\ \ \ 2014\ \ \ volume~8\ \ \ issue\ 1
\hfill \textbf{\thepage}}}   

\vspace*{6pt}

\Abste{The problem of confidence estimation of reliability of complex systems 
with network structure with repairable elements is considered. Estimation of reliability of a 
system is based on test results of its individual elements (subsystems). Existing 
methods for solving this problem are designed for relatively simple series-parallel 
structures consisting of elements with exponential distribution of time to failure.  
Solution of this problem is suggested for the more general model of ``monotone 
structures'' with independent renewable elements, as well as significantly more 
general case of ``aging'' system elements (with monotonically increasing function 
of failure rate). It is assumed that elements of the system are restored regardless 
of the state of other elements. In addition, the solution of this problem is 
obtained in the natural, from the practical point of view, asymptotic behavior for 
the case of high reliability (fast recovery) system elements.}

\KWE{complex systems; network structures; reliability; time to failure; renewal 
time; resource function; failure rate function} 

\DOI{10.14357/19922264140103}

%\Ack
%\noindent


  \begin{multicols}{2}

\renewcommand{\bibname}{\protect\rmfamily References}
%\renewcommand{\bibname}{\large\protect\rm References}

{\small\frenchspacing
{%\baselineskip=10.8pt
\addcontentsline{toc}{section}{References}
\begin{thebibliography}{99}

\bibitem{3-pav-1} %1
\Aue{Gnedenko, B.\,V., Ju.\,K. Beljaev, and A.\,D.~Solov'ev}. 1965. 
\textit{Matematicheskie metody v teorii nadezhnosti}
[\textit{Mathematical methods in reliability theory}]. Moscow: Nauka. 524~p.


\bibitem{1-pav-1} %2
\Aue{Barlow, R., and F.~Proschan}. 1965. \textit{Mathematical theory of 
reliability}. N.Y.: John Wiley\,\&\,Sons. 497~p. 

\bibitem{5-pav-1} %3
\Aue{Vasil'ev, N.\,S.} 1985. Ob odnoy modeli razvitiya seti svya\-zi
[On one model of network development] 
\textit{Izvestiya Rossiyskoy Akademii Nauk. Teoriya i Sistemy Upravleniya} 
[\textit{Bulletin of the Russian Academy of Sciences. Theory and Control Systems}]
6:227--234.


\bibitem{4-pav-1} %4
\Aue{Pavlov, I.\,V.} 1988. Priblizhenno optimal'nye doveritel'nye granitsy dlya 
pokazateley nadezhnosti sistem s vosstanovleniem 
 [Approximately optimum confidence limits for system reliability indicators with recovery].
\textit{Izvestiya Rossiyskoy 
Akademii Nauk. Teoriya i Sistemy Upravleniya} 
[\textit{Bulletin of the Russian Academy of Sciences. Theory and Control Systems}]
3:109--116.

\bibitem{2-pav-1} %5
\Aue{Pavlov, I.\,V., and I.\,A.~Ushakov}. 1989. Vychislenie pokazateley 
nadezhnosti dlya slozhnykh sistem s vosstanav\-li\-va\-emy\-mi elementami
[Calculation of reliability indices for complex systems with recoverable elements].
\textit{Izvestiya Rossiyskoy Akademii Nauk. Teoriya i Sistemy Upravleniya} 
[\textit{Bulletin of the Russian Academy of Sciences. Theory and Control Systems}]
6:170--176.



\bibitem{6-pav-1}
\Aue{Gnedenko, B.\,V., I.\,V.~Pavlov, and I.\,A.~Ushakov}. 1999. 
\textit{Statistical reliability engineering}. N.Y.: John Wiley\,\&\,Sons. 517~p.

\bibitem{7-pav-1}
\Aue{Konovalov, M.\,G.} 2012. Organizatsiya raboty vychislitel'nogo 
kompleksa s pomoshch'yu imitatsionnoy modeli i adaptivnykh algoritmov
[Organization of work of computer complex using a simulation model and adaptive algorithms].
\textit{Informatika i ee Primeneniya}~--- \textit{Inform. Appl.}  6(1):37--48.

\bibitem{8-pav-1}
\Aue{Pavlov, I.\,V.} 2012. Raschet i optimizatsiya nekotorykh kharakteristik 
dlya modeli vychislitel'nogo kompleksa
[Calculation and optimization of some characteristics of the model computer complex].
\textit{Informatika i ee 
Primeneniya}~--- \textit{Inform. Appl.} 6(2):59--62.
\bibitem{9-pav-1}
\Aue{Barlow, R., and F.~Proschan}. 1966. Tolerance and confidence limits for 
classes of distributions based on failure rate. \textit{Ann. Math. Stat.} 
37(6):1184--1195. 
\bibitem{10-pav-1}
\Aue{Pavlov, I.\,V.} 1977. Doveritel'nye granitsy v klasse raspredeleniy s 
vozrastayushchey funktsiey intensivnosti otkazov
[Confidence limits in the class of distributions with increasing failure rate function].
\textit{Izvestiya Rossiyskoy 
Akademii Nauk. Teoriya i Sistemy Upravleniya} 
[\textit{Bulletin of the Russian Academy of Sciences. Theory and Control Systems}]
6:72--84.
\bibitem{11-pav-1}
\Aue{Lloyd, D., and M.~Lipow}. 1962. \textit{Reliability management, 
methods and mathematics}. N.J.: Prentice-Hall, Englewood Cliffs. 684~p.

\bibitem{12-pav-1}
\Aue{Beljaev, Ju.\,K.} 1967. Doveritel'nye intervaly dlya funk\-tsiy ot mnogikh 
neizvestnykh parametrov
[Confidence intervals for functions of many unknown parameters].
 \textit{Dokl. AN SSSR} 196(4):755--758.
 
\bibitem{13-pav-1}
\Aue{Beljaev, Ju.\,K., T.\,N.~Dugina, and E.\,V.~Chepurin}. 1967. Vychislenie 
nizhney doveritel'noy otsenki dlya ve\-ro\-yat\-nosti bezotkaznoy raboty slozhnykh 
sistem  [Calculation of the lower confidence estimates for the probability of failure-free operation of complex systems].
\textit{Izvestiya Rossiyskoy Akademii Nauk. Teoriya i Sistemy 
Upravleniya} [\textit{Bulletin of the Russian Academy of Sciences. Theory and Control Systems}]
2:52--59.
\bibitem{14-pav-1}
\Aue{Pavlov, I.\,V.} 1981. O~korrektnosti fidutsial'nogo podkhoda pri 
postroenii doveritel'nykh granits dlya pokazateley nadezhnosti slozhnykh sistem
[On the correctness of the fiducial approach when constructing confidence limits for the indicators of reliability of complex systems].
\textit{Izvestiya Rossiyskoy Akademii Nauk. Teoriya i Sistemy Upravleniya}
[\textit{Bulletin of the Russian Academy of Sciences. Theory and Control Systems}] 
5:46--52.
\bibitem{15-pav-1}
\Aue{Pavlov, I.\,V.} 1981. O~fidutsial'nom podhode pri vychislenii 
doveritel'nykh granits dlya funktsiy mnogikh neizvestnykh parametrov
[On fiducial approach in calculating confidence limits for functions of many unknown parameters].
\textit{Dokl. RAN}  258(6):1314--1317. 

\end{thebibliography}
} }


\end{multicols}

\vspace*{-6pt}

\hfill{\small\textit{Received December 24, 2013}}

\vspace*{-18pt}

\Contrl

\noindent \textbf{Pavlov Igor V.} (b.\ 1945)~--- Doctor of Science in physics 
and mathematics, Professor, Bauman Moscow State Technical University, 
5, 2nd Baumanskaya Str., Moscow 
105005, Russian Federation; ipavlov@bmstu.ru



 \label{end\stat}
 
\renewcommand{\bibname}{\protect\rm Литература}