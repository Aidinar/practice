%\newcommand{\la}{\lambda}




\def\stat{zeifman}

\def\tit{ОБЩИЕ ОЦЕНКИ УСТОЙЧИВОСТИ ДЛЯ НЕСТАЦИОНАРНЫХ МАРКОВСКИХ ЦЕПЕЙ С НЕПРЕРЫВНЫМ
ВРЕМЕНЕМ$^*$}

\def\titkol{Общие оценки устойчивости для нестационарных марковских цепей с непрерывным
временем}

\def\autkol{А.\,И.~Зейфман,  В.\,Ю.~Королев,  А.\,В.~Коротышева, С.\,Я.~Шоргин}

\def\aut{А.\,И.~Зейфман$^1$,  В.\,Ю.~Королев$^2$,  А.\,В.~Коротышева$^3$, С.\,Я.~Шоргин$^4$}

\titel{\tit}{\aut}{\autkol}{\titkol}

{\renewcommand{\thefootnote}{\fnsymbol{footnote}}
\footnotetext[1]{Исследование поддержано РФФИ (гранты 12-07-00109, 12-07-00115, 13-07-00223,
14-07-00041).}}

\renewcommand{\thefootnote}{\arabic{footnote}}
\footnotetext[1]{Вологодский государственный университет;  Институт проблем
информатики Российской академии наук, Институт социально-экономического
развития территорий Российской академии наук, a\_zeifman@mail.ru}
\footnotetext[2]{Факультет вычислительной математики и кибернетики Московского
государственного университета им.\ М.\,В. Ломоносова; Институт проблем
информатики Российской академии наук, vkorolev@cs.msu.su}
\footnotetext[3]{Вологодский государственный  университет, a\_korotysheva@mail.ru}
\footnotetext[4]{Институт проблем информатики Российской академии наук,
SShorgin@ipiran.ru}


\Abst{Рассмотрен общий метод получения оценок
устойчивости нестационарных марковских цепей с непрерывным временем
с использованием специальных весовых норм, связанных с полной
вариацией. Метод основан на оценках, получаемых при помощи
логарифмической нормы оператора линейного дифференциального
уравнения, и специальных преобразованиях редуцированной матрицы
интенсивностей процесса.  Доказаны утверждения, дающие точные оценки
возмущений вероятностных характеристик моделей в случае отсутствия
эргодичности в равномерной операторной топологии. В~качестве
возможных приложений рассмотрены системы массового обслуживания,
описываемые процессами рождения и гибели с катастрофами, а также
системы массового обслуживания с групповым поступлением и
обслуживанием требований. Исследованы также конкретные классы таких
систем, для которых применим предлагаемый метод получения оценок их
устойчивости, в частности система обслуживания $M_t/M_t/S$ с
катастрофами, а также простейшая система обслуживания с групповым
поступлением и обслуживанием заявок. Рассмотрен численный пример
построения предельных характеристик возмущенной сис\-те\-мы
обслуживания.}

\KW{марковские цепи и модели с непрерывным
временем; нестационарные марковские цепи; оценки устойчивости;
специальные нормы; модели массового обслуживания}

\DOI{10.14357/19922264140111}

\vskip 14pt plus 9pt minus 6pt

      \thispagestyle{headings}

      \begin{multicols}{2}

            \label{st\stat}

\section{Введение}

Исследования устойчивости различных характеристик стохастических
моделей активно проводятся начиная с 1970-х~гг.~[1--3]. В~это же
время В.\,М.~Золотарев предложил подход, согласно которому предельные
теоремы теории вероятностей рассматривались как теоремы
устойчивости. Этот подход был затем развит в работах  В.\,М.~Золотарева,
В.\,В.~Калашникова, В.\,Ю.~Королева и их коллег в рамках
традиционных международных семинаров по проблемам устойчивости
стохастических моделей, основанного В.\,М.~Золотаревым (см.,
например,~[4--8]). Этот подход оказался очень продуктивным для
изучения случайных сумм в теории массового обслуживания, теории
восстановления, ветвящихся процессов~\cite{gk}.

Проблемы, связанные с оценками устойчивости марковских цепей по
отношению к возмущениям их характеристик, в разных ситуациях и для
разных классов процессов детально исследовались начиная с работ
Н.\,В.~Карташова и (параллельно) А.\,И.~Зейф\-ма\-на в 1980-х~гг.~[10--12].

В настоящей работе рассматриваются только цепи с непрерывным временем, поэтому дальнейшие
замечания относятся в основном к таким цепям. Следует отметить
прежде всего два следующих важных момента.

Во-первых, равномерная эргодичность цепи (т.\,е.\ ее эргодичность в равномерной операторной топологии)
позволяет сравнительно легко получать точные оценки возмущений предельных характеристик процесса.

Во-вторых, для получения явных и точных оценок устойчивости цепи
необходимо наличие оценок скорости сходимости цепи к предельным характеристикам
в виде явных неравенств. Более того, чем точнее удается получить оценки скорости сходимости, тем точнее
будут и оценки устой\-чи\-вости.

Такого рода оценки проще всего удается получить для конечных
стационарных (однородных по времени) марковских цепей, так что
наибольшее число результатов относится именно к этой ситуации (см.,
например,~[13--18]).

В случае же счетного пространства состояний (и тем более для нестационарных цепей)
возникают очень серьезные трудности, связанные в первую очередь с отсутствием
равномерной эргодичности для наиболее интересных, с точки зрения приложений, процессов.
Так, процессы рождения и гибели, используемые при описании систем массового обслуживания,
биологии, химии, статистической физики, как правило, {\it не} являются равномерно эргодичными.

Следуя идеям Н.\,В.~Карташова (см.\ детальное описание в~\cite{kar96}),
большинство исследователей применяет вероятностные
методы для изучения стационарных цепей (с конечным, счетным или
общим фазовым пространством), их эргодичности и устойчивости в
различных нормах~[20--22]. Близкий подход для широкого класса
стационарных цепей с дискретным временем рассмотрен в~\cite{mt}.
Обзор основных результатов и подходов, а также новейшие достижения
были представлены в~\cite{yl}.

В работах авторов настоящей статьи проблемы устойчивости исследовались
для нестационарных конечных и счетных
цепей с непрерывным временем другими методами.

Первые исследования, в которых рассматривались нестационарные модели
теории массового обслуживания, появились в 1970-х~гг.~(\cite{g1, g},
а также более поздняя работа~\cite{mw}). Более то-\linebreak го,
тогда же в~\cite{gm} была отмечена принципиальная возможность
использования логарифмической\linebreak нормы матрицы для изучения скорости
схо\-ди\-мости марковских цепей с непрерывным временем. Соответствующий
общий подход, использу-\linebreak ющий тео\-рию линейных дифференциальных\linebreak
уравнений в банаховом пространстве, был развит в серии работ~[12,
29--32] (см.\ также подробное описание в~[33--35]). А~метод
исследования устой\-чи\-вости вектора вероятностей состояний мар\-ковской
цепи с непрерывным временем по норме\linebreak полной вариации ($l_1$-нор\-ме)
относительно возмущений инфинитезимальных характеристик цепи был
впервые предложен в~\cite{z85} (см.\ так\-же~\cite{z88,z94}). %\linebreak
В~\cite{z98} оценки устойчивости по отношению к\linebreak условно малым
возмущениям были подробно исследованы для существенно нестационарных\linebreak
процессов рождения и гибели. Проблемы устойчивости и соответствующие
оценки для одного нового класса процессов были недавно рас\-смот\-ре\-ны в~\cite{z12, z13c}.
{ %\looseness=1

}

%\smallskip

Как правило, изучаемые цепи не являются равномерно эргодичными, и поэтому
для получения содержательных
оценок устойчивости приходится накладывать дополнительные условия на структуру
инфинитезимальной матрицы возмущенного процесса.

%\smallskip

В настоящей же работе предложен подход, поз\-во\-ля\-ющий получать общие оценки устойчивости
в терминах специальных
<<взвешенных>> норм, связанных с естественной (полной вариацией, $l_1$-нор\-мой)
без наложения отмеченных дополнительных условий.


\section{Предварительные сведения}

Пусть $X=X(t)$, $t\hm\geq 0$,~--- нестационарная, вообще говоря, марковская
цепь с непрерывным временем и счетным пространством
состояний $0,1,\dots$ Обозначим через $p_{ij}(s,t)\hm=
\mathrm{Pr}\left\{ X(t)\hm=j\left| X(s)\hm=i\right. \right\}$,
$i,j \hm\ge 0$, $0\hm\leq s\hm\leq t$, переходные вероятности для $X\hm=X(t)$.
Пусть $p_i(t)\hm=\mathrm{Pr}\left\{ X(t) \hm=i \right\}$~--- вероятности
состояний цепи, а ${\bf p}(t) \hm= \left(p_0(t), p_1(t), \dots\right)^{\mathrm{T}}$~---
соответствующий вектор вероятностей состояний.
Далее предполагается, что
\begin{multline*}
\mathrm{Pr}\left\{X\left( t+h\right) =j/X\left( t\right) =i\right\} ={}r\\
{}=  \begin{cases}
q_{ij}\left( t\right)  h+\alpha_{ij}\left(t, h\right) & \mbox { при }j\neq i\,;\\
1-\sum\limits_{k\neq i}q_{ik}\left( t\right)  h+\alpha_{i}\left(
t,h\right) & \mbox { при } j=i\,,
\end{cases}
%\label{1001}
\end{multline*}
где все  $\alpha_{i}(t,h)$ есть $o(h)$ равномерно по  $i$, т.\,е.\
$\sup\limits_i |\alpha_i(t,h)| \hm= o(h)$.

Кроме того, как обычно, считаем, что в нестационарном случае все интенсивности
(т.\,е.\ функции $q_{ij}\left( t\right) $) являются линейными комбинациями конечного
чис\-ла локально интегрируемых на $[0,\infty)$ не\-от\-ри\-ца\-тель\-ных функций.

Положим $a_{ij}(t) \hm=  q_{ji}(t)$ при $j\hm\neq i$ и $a_{ii}(t)\hm =
-\sum\limits_{j\neq i} a_{ji}(t) \hm= -\sum\limits_{j\neq i} q_{ij}(t)$.

Далее, для обеспечения возможности получения более обозримых оценок
(см.\ детали в~[32--34, 40]) предположим, что матрица интенсивностей
существенно ограничена, т.\,е.\
\begin{equation}
|a_{ii}(t)| \le L < \infty
\label{0102-1}
\end{equation}
\noindent почти при всех $t \hm\ge 0$.


Тогда для вероятностей состояний справедлива прямая система Колмогорова
\begin{equation} \label{ur01}
\fr{d\vp}{dt}=A(t)\vp(t)\,,
\end{equation}
\noindent где $A(t)$~--- транспонированная матрица интенсивностей процесса.

Через $\|\cdot\|$  будем обозначать  $l_1$-нор\-му, т.\,е.\  $\|{\vx}\|\hm=\sum|x_i|$, и при
$B \hm= (b_{ij})_{i,j=0}^{\infty}$ соответственно $\|B\| \hm= \sup\limits_j \sum\limits_i |b_{ij}|$.
Пусть $\Omega$~--- множество всех стохастических векторов, т.\,е.\ $l_1$-век\-то\-ров с
неотрицательными координатами и единичной нормой.

Тогда имеем
$$\|A(t)\| = 2\sup_{k}\left|a_{kk}(t)\right| \le 2 L $$
почти при всех  $t \hm\ge 0$. Следовательно, опе\-ра\-тор-функ\-ция $A(t)$ из
$l_1$ в себя ограничена почти при всех $t \hm\ge 0$ и локально интегрируема на  $[0,\infty)$.
А~значит, сис\-те\-му~(\ref{ur01}) можно отождествить с дифференциальным уравнением
в пространстве~$l_1$ с ограниченным оператором. Как известно~\cite{DK}, тогда
задача Коши для уравнения~(\ref{ur01}) имеет единственное решение при любом начальном условии
и, кроме того, если   $\vp(s)\hm \in \Omega$, то и $\vp(t) \hm\in \Omega$ при $t \hm\ge s\hm \ge 0$.

Значит, можно положить  $p_0(t) \hm= 1\hm - \sum\limits_{i \ge 1} p_i(t)$. Тогда
из~(\ref{ur01}) получаем следующее уравнение (см.\ подробное
обсуждение в~[32--34, 40]):
\begin{equation}
\fr{d\vz}{dt}= B(t)\vz(t)+\vf(t)\,, \label{2.06}
\end{equation}
где $\vf(t)=\left(a_{10}, a_{20},\dots \right)^{\mathrm{T}}$;
{ %\scriptsize
\begin{equation}
B = \begin{pmatrix}
a_{11}- a_{10}   & a_{12} - a_{10}   &  \cdots & a_{1r} - a_{10}  &\cdots \\
a_{21} - a_{20} & a_{22} - a_{20}   &    \cdots & a_{2r} - a_{20} &\cdots \\
a_{31} - a_{30}    & a_{32} - a_{30}  &    \cdots & a_{3r} - a_{30}  &\cdots \\
\vdots &\vdots &\vdots &\vdots &\vdots\\
a_{r1} - a_{r0}  & a_{r2} - a_{r0} & \cdots     &  a_{rr} -a_{r0} &\cdots \\
\vdots &\vdots &\vdots &\vdots &\vdots
\end{pmatrix}
\label{2.07}
\end{equation}
}
Через  $\bar{X}=\bar{X}(t)$ обозначим <<возмущенную>> марковскую цепь с вероятностями
со\-сто\-яний
$\bar{p}_i(t)$, транспонированной матрицей интенсивностей
$\bar{A}(t) \hm= \left(\bar{a}_{ij}(t)\right)_{i,j=0}^{\infty}$ и~т.\,д.

Через  ${\sf E}(t,k) = {\sf E}\left\{X(t)\left|X(0)=k\right.\right\}$
будем обозначать далее среднее (математическое ожидание)
процесса в момент~$t$ при начальном условии $X(0)\hm=k$.

\section{Основные результаты}

Пусть $\{d_i\}$, $i=1,2, \dots$,~--- возрастающая последовательность
положительных чисел, причем $d_1\hm = 1$.

Рассмотрим треугольную матрицу $D$ следующего вида:
\begin{equation*}
D=\begin{pmatrix}
d_1   & d_1 & d_1 & \cdots  \\
0   & d_2  & d_2  &   \cdots  \\
0   & 0  & d_3  &   \cdots  \\
& \ddots & \ddots & \ddots \\
\end{pmatrix}\,, 
%\label{2013}
\end{equation*}
а также соответствующее пространство последовательностей
$l_{1D}\hm=\left\{{\bf z} \hm= (p_1,p_2,\cdots)^{\mathrm{T}} / \|{\bf z}\|_{1D}
\hm\equiv\right.$\linebreak $\left.\equiv \|D {\bf z}\| \hm<\infty \right\}$. Отметим, что  $\|B\|_{1D} \hm=
\|DBD^{-1}\|$. Положим вдобавок $\|{\bf p}\|_{1D}\hm = \|{\bf z}\|_{1D}$.

%\bigskip

Отметим, что введенная матрица  не имеет прямой вероятностной интерпретации.
С~другой стороны, она оказывается очень полезной в проводимых исследованиях.


%\bigskip


Будем далее предполагать выполненными следующие условия:
\begin{equation*}
\|B(t)\|_{1D} \le  \mathfrak{B}< \infty\,; \quad  \|{\bf f}(t)\|_{1D} \le \mathfrak{f}< \infty
%\label{3001} 
\end{equation*}
\noindent почти при всех $t \hm\ge 0$.

\medskip

\noindent

\noindent
\textbf{Определение~1.}\
\textit{Марковская цепь $X(t)$ называется $1D$-экс\-по\-нен\-ци\-аль\-но слабо
эргодичной, если существуют
положительные~$M$ и~$a$ такие, что при любых  $s,t$ таких, что $t\hm\ge s \hm\ge 0$,
и любых начальных условиях
 ${\bf p^{*}}(s) \hm\in l_{1D}$, ${\bf p^{**}}(s) \hm\in l_{1D}$ выполнено неравенство}:
\begin{equation*}
\|{\bf p^{*}}(t) - {\bf p^{**}}(t)\|_{1D} \le M
e^{-a\left(t-s\right)} \|{\bf p^{*}}(s) - {\bf p^{**}}(s)\|_{1D}.
%\label{d-erg2}
\end{equation*}

\smallskip

\noindent
\textbf{Замечание~1.}\
Если у цепи, кроме того, существует стационарное распределение вероятностей ${\bf \pi}
\hm\in \Omega$, то  $X(t)$ является $1D$-экс\-по\-нен\-ци\-аль\-но {\it сильно}
эргодичной, т.\,е.\
\begin{equation}
\|{\bf p^{*}}(t) - {\bf \pi}\|_{1D} \le M e^{-at} \|{\bf p^{*}}(0) -
{\bf \pi}\|_{1D}
\label{d-erg1}
\end{equation}
при любом  $t \ge 0$ и любом начальном условии ${\bf p^{*}}(0)$.


\medskip

\noindent
\textbf{Определение~2.}\
\textit{Марковская цепь  $X(t) $ имеет предельное математическое ожидание
(предельное среднее) $\phi(t)$, если  $|E(t,k) \hm- \phi(t) | \hm\to
0$ при $t \to \infty$ для любого~$k$}.

\medskip

Пусть почти при всех  $t \hm\ge 0$ выполнены условия:
\begin{equation}
\left.
\begin{array}{rl}
\|B(t)-\bar{B}(t)\|_{1D} &\le  \mathfrak{B-\bar{B}}\,; \\[9pt]
\|{\bf f}(t)-\bar{{\bf f}}(t)\|_{1D} &\le \mathfrak{f-\bar{f}}\,.
\end{array}
\right\}
\label{3001'}
\end{equation}

\noindent
\textbf{Теорема~1.}\
\textit{Пусть марковская цепь   $X(t)$ $1D$-экс\-по\-нен\-ци\-аль\-но слабо эргодична.
Тогда при достаточно малых возмущениях~$(\ref{3001'})$ возмущенная цепь $\bar{X}(t)$
также $1D$-экс\-по\-нен\-ци\-аль\-но слабо эргодична и справедлива следующая оценка
устойчивости в $1D$-норме}:
\begin{equation*}
\limsup_{t \to \infty}   \|{\bf p}(t)- {\bar{\bf p}}(t)\| \le
\fr{M\left(M\left(\mathfrak{B-\bar{B}}\right) \mathfrak{f}  +
a\left(\mathfrak{f-\bar{f}}\right)\right)}{a\left(a -
M\left(\mathfrak{B-\bar{B}}\right)\right)}.
%\label{3002}
\end{equation*}
\textit{Если, кроме того, $W=\inf\limits_{i \ge 1}  {d_i}/{i} > 0$, то обе цепи  $X(t)$ и $\bar{X}(t)$
имеют предельные средние и справедлива оценка}:
\begin{multline}
\limsup_{t \to \infty}  |\phi(t) - \bar{\phi}(t)|\le{}\\
{}\le
\fr{M\left(M\left(\mathfrak{B-\bar{B}}\right) \mathfrak{f}  +
a\left(\mathfrak{f-\bar{f}}\right)\right)}{Wa\left(a -
M\left(\mathfrak{B-\bar{B}}\right)\right)}\,.
\label{3003}
\end{multline}

\noindent
Д\,о\,к\,а\,з\,а\,т\,е\,л\,ь\,с\,т\,в\,о\,.\ \
Обозначим через  $V(t, s)$ оператор Коши для уравнения~(\ref{2.06}),
а через  $\bar{V}(t, s)$~--- оператор Коши соответствующего уравнения для возмущенного процесса.
Тогда из $1D$-экс\-по\-нен\-ци\-аль\-ной эргодичности, в соответствии с~(\ref{d-erg1}),
вытекает при всех $t \hm\ge s \hm\ge 0$ оценка
\begin{equation*}
\|V(t,s)\|_{1D} \le   M e^{-a(t-s)}\,.
%\label{2015}
\end{equation*}
Теперь, применяя лемму~3.2.3 из~\cite{DK}, получаем при всех $t \hm\ge s \hm\ge 0$
оценку нормы <<возмущенного>> оператора Коши
\begin{equation*}
\|\bar{V}(t,s)\|_{1D} \le   M e^{-\left(a - M\left(\mathfrak{B-\bar{B}}\right)\right)(t-s)}\,,
%\label{2015-1}
\end{equation*}
откуда вытекает  $1D$-экс\-по\-нен\-ци\-аль\-ная эргодичность  $\bar{X}(t)$
при достаточно малых $\mathfrak{B-\bar{B}}$.
Перепишем уравнение~(\ref{2.06}) в виде:
\begin{equation*}
\fr{d\vz}{dt}=\bar{B}(t)\vz(t) + {\bf f}(t)+\left(B(t)-\bar{B}(t)\right)\vz(t)\,.
%\label{2020}
\end{equation*}
Отсюда
\begin{align*}
\vz(t)&=\bar{V}(t,0)\vz(0)+\int\limits_0^t \bar{V}(t,\tau){\bf{f}}(\tau) \,
d\tau+{}\\
&\hspace*{5mm}{}+\int\limits_0^t \bar{V}(t,\tau)\ \left(B(\tau) -\bar{B}(\tau)\right)\vz(\tau)\,d\tau\,;
\\
\bar{\vz}(t)&=\bar{V}(t,0)\bar{\vz}(0)+\int\limits_0^t
\bar{V}(t,\tau)\bar{\vf}(\tau) \, d\tau\,.
\end{align*}
Значит, при одинаковых начальных условиях исходного и возмущенного процессов получаем
следующее неравенство в $1D$-норме:
\begin{multline*}
\left\|\vz(t)-\bar{\vz}(t)\right\|\le{}\\
{}\le \int\limits_0^t \|\bar{V}(t,\tau)\|
\left(\| B (\tau)-\bar{B} (\tau)\| \|\vz(\tau)\| +{} \right.\\
\left.{}+\|
\vf(\tau) -\bar{\vf}(\tau)\|\right)\ d\tau \le {} \\
 {}\le \int\limits_0^t M e^{-\left(a - M\left(\mathfrak{B-\bar{B}}\right)\right)(t-\tau)}
\left(\left(\mathfrak{B-\bar{B}}\right) \|\vz(\tau)\| + {}\right.\\
\left.{}+
\left(\mathfrak{f-\bar{f}}\right)\right)\ d\tau.
%\label{3005}
\end{multline*}

Используя условия теоремы, можно оценить при всех $ t \hm\ge 0$
\begin{multline*}
 \|\vz (t)\|_{1D} \le  \|V(t)\|_{1D}\|\vz (0)\|_{1D} + {}\\
{} + \int\limits_0^t\! \|V(t, \tau)\|_{1D}\|{\bf f}(\tau)\|_{1D} \, d\tau      \le
 Me^{-at}\|\vz (0)\|_{1D} + \fr{M}{a}\,  \mathfrak{f}.\hspace*{-9.4267pt}
%\label{3006}
\end{multline*}
А тогда получаем в $1D$-нор\-ме при любом начальном условии  $\vz (0) \hm\in l_{1D}$ оценку
\begin{multline}
\left\|\vz(t)-\bar{\vz}(t)\right\| \le
 M\left(\left(\mathfrak{B-\bar{B}}\right)\fr{M}{a}  \mathfrak{f}  + 
 \left(\mathfrak{f-\bar{f}}\right)\right) \times{}\\
 {}\times
\int\limits_0^t e^{-\left(a - M\left(\mathfrak{B-\bar{B}}\right)\right)(t-\tau)}\, d\tau +{} \\
{} + M \int\limits_0^t e^{-\left(a - M\left(\mathfrak{B-\bar{B}}\right)\right)(t-\tau)}
\left(\mathfrak{B-\bar{B}}\right)\times{}\\
{}\times  Me^{-a\tau}\|\vz (0)\| \, d\tau \le {}\\
 {}\le \fr{M\left(M\left(\mathfrak{B-\bar{B}}\right) \mathfrak{f}  + a\left(\mathfrak{f-\bar{f}}\right)\right)}{a\left(a - M\left(\mathfrak{B-\bar{B}}\right)\right)} + o\left( 1 \right) 
  \label{aaa-a}
\end{multline}
и первое утверждение теоремы доказано.

\smallskip

Введем теперь пространство последовательностей $l_{1E}$ таких, что
$$
l_{1E}\hm=\left\{z=(p_1,p_2,\cdots)^{\mathrm{T}} / \|z\|_{1E}\hm\equiv\sum n |p_n| \hm<
\infty\right\}\,.
$$

Тогда второе утверждение и оценка~(\ref{3003}) вытекают из известного неравенства
$\|\vz\|_{1E}\hm\le W^{-1}\|\vz\|_{1D}$ (см., например,~\cite{z06}) и оценки~(\ref{aaa-a}).




\medskip

\noindent
\textbf{Следствие~1.}\
Пусть  $X(t)$~--- стационарная $1D$-экс\-по\-нен\-ци\-аль\-но сильно
эргодичная марковская цепь. Тогда при достаточно
малых возмущениях цепь $\bar{X}(t)$ также $1D$-экс\-по\-нен\-ци\-аль\-но
сильно эргодична и справедлива следующая оценка в норме $1D$:
\begin{equation*}
 \|{\bf \pi}- {\bar{\bf \pi}}\| \le
\fr{M\left(M\| B -\bar{B}\| \|{\bf f}\|  + a\|\vf
-\bar{\vf}\|\right)}{a\left(a - M\|B -\bar{B}\|\right)}\,.
%\label{2011}
\end{equation*}
Если вдобавок $W=\inf\limits_{i \ge 1} (d_i/i)\hm > 0$, то
обе марковских цепи $X(t)$ и $\bar{X}(t)$
имеют предельные средние и в норме~$1D$ выполнено неравенство:
\begin{equation*}
|\phi - \bar{\phi}|\le
\fr{M\left(M\| B -\bar{B}\| \|{\bf f}\|  + a\|\vf -\bar{\vf}\|\right)}
{Wa\left(a - M\|B -\bar{B}\|\right)}\,.
%\label{2012}
\end{equation*}

\medskip

Можно получить и оценки устойчивости в естественной $l_1$-нор\-ме.
Для этого достаточно воспользоваться известным соотношением
(см., например,~\cite{gz04} или~\cite{z06}):
$$\|\vp^*-\vp^{**}\| \le 2 \|\vz^*-\vz^{**}\| \le 4\|\vz^*-\vz^{**}\|_{1D}\,.
$$


\medskip

\noindent
\textbf{Следствие~2.}\
Пусть  $X(t)$~--- стационарная $1D$-экс\-по\-нен\-ци\-аль\-но
сильно эргодичная марковская цепь. Тогда справедлива следующая оценка устойчи\-вости:
\begin{equation*}
 \|{\bf \pi}- {\bar{\bf \pi}}\| \le
\fr{4M\left(M\| B -\bar{B}\|_{1D} \|{\bf f}\|_{1D}  + a\|\vf -\bar{\vf}\|_{1D}\right)}{a\left(a - M\|B -\bar{B}\|_{1D}\right)}.
%\label{2101}
\end{equation*}

\medskip

\noindent
\textbf{Следствие~3.}\
Пусть  $X(t)$~--- нестационарная  $1D$-экс\-по\-нен\-ци\-аль\-но слабо
эргодичная марковская цепь. Тогда справедлива следующая оценка устойчи\-вости:
\begin{multline*}
\limsup\limits_{t \to \infty}   \|{\bf p}(t)- {\bar{\bf p}}(t)\| \le{}\\
{}\le
\fr{4M\left(M\left(\mathfrak{B-\bar{B}}\right) \mathfrak{f}  + a\left(\mathfrak{f-\bar{f}}\right)\right)}
{a\left(a - M\left(\mathfrak{B-\bar{B}}\right)\right)}\,.
%\label{3011}
\end{multline*}

\section{Процессы рождения и~гибели с~катастрофами}

Пусть $X(t)$~--- процесс рождения и гибели с катастрофами (ПРГК).
Тогда ненулевыми являются интенсивности
переходов: $q_{i,i+1}\left( t\right)\hm=\lambda_i(t)$~--- интенсивности рождения,
$q_{i,i-1}\left(t\right) \hm= \mu_i(t)$~---
интенсивности гибели,  $q_{i,0}\left( t\right) \hm=\xi_i(t)$~---
интенсивности катастроф,  а все остальные $q_{ij}(t)\hm \equiv 0$ при $i\hm \neq j$.

Условия ограниченности~(\ref{0102-1}) будут выглядеть следующим образом:
\begin{equation*}
\sup\limits_{k \ge 0} \left(\lambda _k(t) +\mu_{k}(t) +\xi_{k}(t)\right) \le L < \infty
%\label{4001}
\end{equation*}
\noindent почти при всех $t \hm\ge 0$.

Для элементов матрицы~(\ref{2.07}) получаем теперь следующие выражения:
%\noindent
\begin{equation*}
b_{ij} = \begin{cases}
 -(\lambda_0+\lambda_1+\mu_1+\xi_1)  & \mbox { при }  i=j=1\,; \\
\mu_2-\lambda_0  & \mbox { при }  i=1,j=2\,; \\
-\lambda_0  & \mbox { при }  i=1,j>2\,; \\
-(\lambda_j+\mu_j+ \xi_j)  & \mbox { при } i=j >1\,; \\
\mu_j  & \mbox { при }  i=j-1 >1\,; \\
\lambda_j  & \mbox { при }  i=j+1 >1\,; \\
0  & \mbox { в остальных случаях}.
\end{cases}\hspace*{-1.87943pt}
% \label{4002}
\end{equation*}


Следовательно,

\end{multicols}

\hrule

\vspace*{6pt}

\begin{equation*}
DBD^{-1}=\begin{pmatrix}
-(\lambda_{0}+\mu_1+\xi_1) & \fr{d_{1}}{d_2}\mu_2 & 0 & \cdots & \cdots &\cdots\\
\fr{d_{2}}{d_1}\lambda_1 & -(\lambda_{1}+\mu_2+\xi_2) & \fr{d_{2}}{d_3}\mu_3 & 0 & \ddots & \ddots\\
0 & \fr{d_{3}}{d_2}\lambda_2 & -(\lambda_{2}+\mu_3+\xi_3) & \fr{d_{3}}{d_4}\mu_4 & 0 & \ddots \\
\vdots& \ddots & \ddots & \ddots & \ddots & \ddots\\
\vdots&\ddots  & \vdots & \ddots & \ddots & \ddots
\end{pmatrix}\,;
\end{equation*}
\begin{equation*}
\|B(t)\|_{1D} =  \sup\limits_{k \ge 0} \left(
\lambda _k(t) +\mu_{k+1}(t) +\xi_{k+1}(t) + {}\right.\\
\left.{}+\fr{d_{k+1}}{d_k}\lambda _{k+1}(t) +
\fr{d_{k-1}}{d_k} \mu _k(t)\right)\,.
%\label{4003}
\end{equation*}

\vspace*{6pt}

\hrule

\begin{multicols}{2}

\noindent
Отметим, что простое дополнительное условие
\begin{equation*}
\sup\limits_{k \ge 0} \fr{d_{k+1}}{d_k} = m < \infty
%\label{4004}
\end{equation*}
гарантирует справедливость оценки
\begin{equation*}
\|B(t)\|_{1D} \le \left(1+m\right)L < \infty
%\label{4005}
\end{equation*}
почти при всех $t \ge 0$.
Так как в этом разделе $X(t)$~--- ПРГК, то
$\vf(t)=\left(\lambda_{0}, 0,0, \cdots \right)^{\mathrm{T}}$ и
\begin{equation*}
\|{\bf f}(t)\|_{1D} = d_1 \lambda_0(t) \le L < \infty
%\label{4006}
\end{equation*}
почти при всех  $t \hm\ge 0$.

\medskip

В предыдущих работах изучалась $1D$-эр\-го\-дич\-ность для ПРГ и ПРГК,
соответствующие точные оценки получены в~\cite{z91, z95, z12, z13a}.
Здесь только сформулированы основные результаты, которые позволяют
получить соответствующие оценки устойчивости.

Рассмотрим выражения:
\begin{multline*}
\alpha_{k}\left( t\right) = \lambda _k\left( t\right) +\mu
_{k+1}\left( t\right) +\xi
_{k+1}\left( t\right) - {}\\
{}-\fr{d_{k+1}}{d_k} \lambda _{k+1}\left(
t\right) -\fr{d_{k-1}}{d_k} \mu _k\left( t\right)\,, \quad k \ge 0\,;
%\label{4007}
\end{multline*}
\begin{equation*}
\alpha\left( t\right) = \inf\limits_{k\geq 0} \alpha_{k}\left( t\right)\,.
%\label{4008}
\end{equation*}

\noindent
\textbf{Теорема~2.}
\textit{Пусть
\begin{equation}
\int\limits_0^{\infty}\alpha\left( t\right)\, dt = + \infty.
\label{4009}
\end{equation}
Тогда  $X(t)$ слабо эргодичен. Если же, кроме того, существуют положительные~$M$ и~$a$ такие,
что при всех $0 \hm\le s \hm\le t$
\begin{equation*}
e^{-\int\limits_s^{t}\alpha\left( \tau\right)\, d\tau} \le M e^{-a\left(t-s\right)}\,,
%\label{4010}
\end{equation*}
 то $X(t)$  $1D$-экс\-по\-нен\-ци\-аль\-но слабо эргодичен}.
 
 \begin{figure*}[b] %fig1
\vspace*{1pt}
 \begin{center}
 \mbox{%
 \epsfxsize=163.445mm
 \epsfbox{zei-1.eps}
 }
 \end{center}
 \vspace*{-16pt}
\begin{minipage}[t]{80mm}
\Caption{Предельное среднее $\bar{\phi}(t)$}
\end{minipage}
\hfill
\begin{minipage}[t]{80mm}
\Caption{Вероятность $\mathrm{Pr}\left\{ \bar{X}(t) =0 \right\}$}
\end{minipage}
\end{figure*}




\medskip

\noindent
\textbf{Следствие~4.}\
Пусть $X(t)$~--- стационарный ПРГК и пусть теперь вместо~(\ref{4009}) соответствующее
$\alpha \hm> 0$.
Тогда $X(t)$  $1D$-экс\-по\-нен\-ци\-аль\-но (сильно) эргодичен.
При этом можно выбрать $M\hm=1$ и $a\hm = \alpha$.


\medskip

\noindent
\textbf{Следствие~5.}\
Пусть теперь все интенсивности рож\-де\-ния, гибели и катастроф  $1$-пе\-ри\-о\-дич\-ны по~$t$.
Пусть
\begin{equation*}
\int\limits_0^{1}\alpha\left( t\right)\, dt = \alpha^* > 0
%\label{4011}
\end{equation*}
вместо~(\ref{4009}). Тогда  $X(t)$  $1D$-экс\-по\-нен\-ци\-аль\-но слабо эргодичен,
\begin{equation*}
a= \alpha^*, \quad M = \sup\limits_{|t-s|\le 1} e^{\int\limits_s^{t}\alpha\left( \tau\right)\, d\tau}\,.
%\label{4012}
\end{equation*}


\vspace*{-9pt}

\subsection{Система обслуживания  $M_t/M_t/S$ с~катастрофами}

Это  очень часто используемая модель системы массового обслуживания.
Число требований в сис\-те\-ме $X(t)$~--- ПРГК
с интенсивностями: $\lambda_n(t)\hm=  \lambda(t)$~---
интенсивность поступления требования (заявки) в сис\-те\-му;
$\mu_n(t)\hm=\min(n,S)\mu(t)$~--- интенсивность обслуживания требования,
если в сис\-те\-ме находится $n$ требований;
наконец,  $\xi_n(t)\hm=\xi(t)$~--- интенсивность катастрофы (т.\,е.\ обнуления числа заявок).

\medskip

Здесь рассмотрим только случай существенной интенсивности обслуживания.
Соответствующие оценки получены в~\cite{z12}.
Используя доказанные теоремы, получаем для этой системы следующий результат.

\medskip

\noindent
\textbf{Утверждение~1.}\
\textit{Пусть существует  $\delta \hm\in \left(1,{S}/(S-1)\right]$, функция $\theta(t)$
и положительные чис\-ла~$R$ и~$\nu$
такие, что при любых}  $s,t$, $0 \hm\le s \hm\le t$,
\begin{equation*}
S\mu(t)-\delta\lambda(t) \ge \theta(t)\,; \ 
e^{-\int\limits_s^t\left(1-\delta^{-1}\right)\theta(u)\, du} \le R e^{-\nu (t-s)}\,.
%\label{4013}
\end{equation*}
\textit{Тогда при  $d_k\hm=\delta^{k-1}$, $M\hm=R$, $a\hm =\nu$
процесс $X(t)$ $1D$-экс\-по\-нен\-ци\-аль\-но слабо эргодичен и,
следовательно, справедливы следующие оценки устойчивости}:
\begin{multline}
\limsup\limits_{t \to \infty}   \|{\bf p}(t)- {\bar{\bf p}}(t)\|_{1D} \le{}\\
{}\le
\fr{R\left(R\left(\mathfrak{B-\bar{B}}\right) \mathfrak{f}  + \nu\left(\mathfrak{f-\bar{f}}\right)\right)}
{\nu\left(\nu - R\left(\mathfrak{B-\bar{B}}\right)\right)}\,;
\label{4014}
\end{multline}

\vspace*{-12pt}

\begin{multline}
\limsup\limits_{t \to \infty}  |\phi(t) - \bar{\phi}(t)|\le{}\\
{}\le
\fr{R\left(R\left(\mathfrak{B-\bar{B}}\right) \mathfrak{f}  +
 \nu\left(\mathfrak{f-\bar{f}}\right)\right)}
 {W\nu\left(\nu - R\left(\mathfrak{B-\bar{B}}\right)\right)}\,,
\label{4015}
\end{multline}
\textit{где} $W=\inf\limits_{i \ge 1}  {\delta^{i-1}}/{i} \hm> 0$.
\textit{Если, кроме того, процесс $X(t)$ стационарен (т.\,е.\ интенсивности~$\lambda$,
$\mu$ и~$\xi$ не зависят от~$t$), то $R\hm=1$, $\nu \hm=
\left(1\hm-\delta^{-1}\right)\left(S\mu\hm-\delta\lambda\right)$,
$X(t)$  $1D$-экс\-по\-нен\-ци\-аль\-но
сильно эргодичен и выполняются соответствующие оценки устойчивости.}


\medskip

Рассмотрим теперь более конкретный пример. Пусть $\bar{X}(t)$~---
чис\-ло требований в системе
обслуживания $M_t/M_t/2$ без катастроф, а интенсивности поступления требований
$\bar\lambda_n(t)\hm=  \bar\lambda(t) \hm= 2 \hm+
\sin 2\pi t \hm+ \varepsilon_1 \cos t^2$, обслуживания~---
$\bar\mu_n(t)\hm=\min(n,2)\bar\mu(t) \hm=
\min(n,2)\left(3\hm+\cos 2\pi t\hm+\varepsilon_2 \sin \sqrt{t}\right)$,
где $|\varepsilon_i| \hm\le \epsilon^*$.
Выберем невозмущенный процесс $X(t)$ с интенсивностями $\lambda_n(t)\hm=  \lambda(t) \hm= 2 \hm+
\sin 2\pi t$ и   $\mu_n(t)\hm=\min(n,2)\mu(t) \hm= \min(n,2)\left(3\hm+\cos 2\pi t\right)$.
Положим $d_k\hm= 2^{k-1}$, тогда
\begin{equation*}
\alpha\left( t\right) = 1 + \cos 2\pi t - \sin 2\pi t\,.
%\label{4016}
\end{equation*}
При этом $\mathfrak{f}= 3$, $W\hm=1$,  $a\hm = \alpha^* \hm=
\int\limits_0^{1}\alpha\left( t\right)\, dt \hm= 1$,
$M\hm = \sup\limits_{|t-s|\le 1} e^{\int\limits_s^{t}\alpha\left( \tau\right)\, d\tau} \hm\le 2$
и справедливы сле\-ду\-ющие оценки устойчивости:

\noindent
\begin{align*}
\limsup\limits_{t \to \infty}   \|{\bf p}(t)- {\bar{\bf p}}(t)\|_{1D} &\le
\fr{74\epsilon^*}{1 - 12\epsilon^*}\,;
%\label{4017}
\\
\limsup_{t \to \infty}  |\phi(t) - \bar{\phi}(t)|&\le
\fr{74\epsilon^*}{1 - 12\epsilon^*}\,;
%\label{4018}
\\
\limsup_{t \to \infty}   \|{\bf p}(t)- {\bar{\bf p}}(t)\| &\le
\fr{296\epsilon^*}{1 - 12\epsilon^*}\,.
%\label{4019}
\end{align*}

С учетом оценок скорости сходимости получаем, что для построения с точностью
$10^{-6}$ при $\epsilon^* \hm= 10^{-3}$ предельных характеристик исходного процесса
достаточно брать $t \hm\ge 18$. А~используя прием, подробно изученный в~\cite{z06},
находим, что для достижения нужной точности достаточно рассмотреть <<усеченный>>
процесс с  пространством состояний от~0 до~40.

Рисунки~1 и~2 дают <<полосы>>, показывающие границы для приближенных
предельных характеристики $\bar{p}_0(t)$ и $\bar{\phi}(t)$.


\vspace*{-5pt}

\subsection{Простой процесс рождения, гибели и~иммиграции с~конечным числом состояний}

Эта модель рассмотрена в недавней статье~\cite{z13a},
где получены и оценки скорости сходимости в разных ситуациях
(см.\ также описание в~\cite{tkf} и~\cite{mb}).
Здесь $\lambda_n(t)\hm=(n\hm+1)\lambda(t)$~--- интенсивности рождения,
$\mu_n(t)\hm=n\mu(t)$~--- интенсивности гибели,
$\xi_{n}(t)\hm=\xi(t)$~--- интенсивности катастрофы (мгновенного
исчезновения популяции), а пространство возможных состояний~--- $\{0, 1, \dots,S \}$.

\smallskip

\noindent
\textbf{Утверждение~2.}\
\textit{Пусть существуют положительные~$R$ и~$\nu$ такие, что при любых }
$s,t$, $0\hm \le s\hm \le t$,
\begin{equation*}
e^{-\int\limits_s^t\left(\mu(u)-\lambda(u)+\xi(u)\right)\, du} \le R e^{-\nu (t-s)}\,.
%\label{4021}
\end{equation*}
  \textit{Тогда $X(t)$ $1D$-экс\-по\-нен\-ци\-аль\-но слабо эргодичен,
  причем $d_k\hm= 1$, $M\hm=R$, $ a\hm = \nu$, $W\hm= {1}/{S}$.
  Следовательно, выполняются оценки устой\-чи\-вости~$(\ref{4014})$ и~$(\ref{4015})$.
Если же, кроме того,  $X(t)$ стационарен (т.\,е.\ ин\-тен\-сив\-ности $\lambda$, $\mu$ и~$\xi$
не зависят от~$t$), то $R\hm=1$, $\nu \hm= \mu\hm-\lambda\hm+\xi$ и
$X(t)$ будет $1D$-экс\-по\-нен\-ци\-аль\-но сильно эргодичным}.


Оценки скорости сходимости при конкретных интенсивностях для этой модели
рас\-смот\-ре\-ны в~\cite{z13a}.

\vspace*{-4pt}


\section{Система массового обслуживания с~групповым~поступлением и~обслуживанием заявок}

Пусть теперь  $X(t)$~--- число требований в нестационарной марковской модели массового обслуживания
с групповым поступлением и обслуживанием требований. Будем предполагать, что интенсивности поступления
 $\chi_{k}(t)$ и обслуживания $\nu_{k}(t)$  требований не зависят от чис\-ла заявок в системе.
 Дополнительно считаем, что при всех~$k$ и почти всех  $t \hm\ge 0$  выполняются неравенства
 $\chi_{k+1}(t)\hm \le \chi_{k}(t)$ и $\nu_{k+1}(t)\hm \le \nu_{k}(t)$ .
Тогда процессу $X(t)$ соответствует следующая транспонированная матрица интенсивностей:

\noindent
{%\scriptsize
\begin{equation*}
A(t)=\begin{pmatrix}
a_{00}(t) & \nu_1(t)  & \nu_2(t)   &   \cdots & \nu_r(t) & \cdots\\[3pt]
\chi_1(t)   & a_{11}(t)  & \nu_1(t)  &  \cdots & \nu_{r-1}(t) & \ddots\\[3pt]
\chi_2(t)  & \chi_1(t)    & a_{22}(t)&   \cdots & \nu_{r-2}(t) & \ddots\\[3pt]
\vdots&\vdots&\vdots&\ddots&\ddots&\ddots\\[3pt]
\chi_r(t) & \chi_{r-1}(t) & \cdots &  \chi_1 (t)   &  a_{rr}(t) & \ddots\\[3pt]
\vdots&\vdots&\vdots&\vdots&\vdots&\vdots
\end{pmatrix}
%\label{5001}
\end{equation*}}
где $a_{ii}(t)\hm=-\sum\limits_{k=1}^{i}\nu_k(t) \hm- \sum\limits_{k=1}^{\infty} \chi_{k}(t)$.

Условие ограниченности~(\ref{0102-1}) приводит к неравенству
$$
\sum\limits_{k=1}^{\infty}\nu_k(t)\hm + \sum\limits_{k=1}^{\infty} \chi_{k}(t) \hm \le L
$$
почти при всех $t \hm\ge 0$.

Основные свойства таких систем изучены в  работах~[39, 45--48].

Элементы матрицы~(\ref{2.07}) здесь выглядят сле\-ду\-ющим образом:
\begin{equation*}
b_{ij} = \begin{cases}
 a_{jj}-\chi_j  & \mbox { при } \quad i=j \ge 1\,; \\
\chi_{i-j}-\chi_i  & \mbox { при } \quad 1 \le j < i\,; \\
\nu_{j-i}-\chi_i  & \mbox { при } \quad 1 \le i < j \,.
\end{cases} %\label{5002}
\end{equation*}
Следовательно,
\begin{equation*}
DBD^{-1}=\begin{pmatrix}
a_{11} & \fr{d_{1}}{d_2}(\nu_1-\nu_2) & \fr{d_1}{d_3}(\nu_2-\nu_3) & \cdots  \\
\fr{d_{2}}{d_1}\chi_1 & a_{22} & \fr{d_{2}}{d_3}(\nu_1-\nu_3) &  \ddots \\
\fr{d_{3}}{d_1}\chi_2 & \fr{d_{3}}{d_2}\chi_1 &   a_{33} &  \ddots   \\
\vdots & \vdots & \ddots & \ddots
\end{pmatrix}\,;
\end{equation*}

\vspace*{-12pt}

\noindent
\begin{multline*}
\|B(t)\|_{1D} \le {}\\
{}\le \sup\limits_{k \ge 0} \left(
|a_{kk}(t)| + \sum\limits_{i=1}^{\infty}\fr{d_{k+i}}{d_k}\,\chi_i(t)\right)
+ \sum\limits_{i=1}^{\infty} \nu_{i}(t)\,.
%\label{5003}
\end{multline*}
А значит, если потребовать, чтобы
$$
\sum\limits_{i=1}^{\infty}\fr{d_{k+i}}{d_k}\,\chi_i(t) \le  K_1 \hm< \infty\,,
$$
то получим 
$$
\|B(t)\|_{1D}\hm \le 2L+ K_1
$$
 почти при всех $t \hm\ge 0$.

Предположим еще, что почти при всех $t \hm\ge 0$
\begin{equation*}
\|{\bf f}(t)\|_{1D} =  d_{1}\chi_1(t) + \left(d_{1}+d_{2}\right)\chi_2(t) + \dots \le K_2 < \infty \,.
%\label{5004}
\end{equation*}

\medskip

Рассмотрим выражения
\begin{multline*}
\beta_{i}\left( t\right) = -a_{ii}(t) - \sum\limits_{k=1}^{i-1}
(\nu_{i-k}(t)-\nu_i(t))\fr{d_k}{d_i}-{}\\
{}-\sum\limits_{k \ge 1} \chi_k(t)\fr{d_{k+i}}{d_i}\,, \quad k \ge 0\,;
%\label{5005}
\end{multline*}
\begin{equation*}
\beta\left( t\right) = \inf_{k\geq 0} \beta_{k}\left( t\right)\,.
%\label{5006}
\end{equation*}
Тогда справедливо следующее утверждение.

\medskip

\noindent
\textbf{Теорема~3.}\
\textit{Пусть
\begin{equation}
\int\limits_0^{\infty}\beta\left( t\right)\, dt = + \infty\,.
\label{5007}
\end{equation}
Тогда $X(t)$ слабо эргодичен. Если же, кроме того, при некоторых положительных~$M$ и~$a$
и почти всех $s,t$ $(0 \hm\le s \hm\le t)$
\begin{equation*}
e^{-\int\limits_s^{t}\beta\left( \tau\right)\, d\tau} \le M e^{-a\left(t-s\right)},
%\label{5008}
\end{equation*}
то $X(t)$  $1D$-экс\-по\-нен\-ци\-аль\-но слабо эргодичен
и справедливы соответствующие оценки устойчивости.}

\medskip

\noindent
\textbf{Следствие~6.}\
Если вдобавок процесс $X(t)$ стационарен и вместо~(\ref{5007})
выполняется соответствующее неравенство $\beta \hm> 0$, то $X(t)$
$1D$-экс\-по\-нен\-ци\-аль\-но сильно эргодичен. При этом $M\hm=1$ и $a\hm = \beta$.

\medskip

\noindent
\textbf{Следствие~7.}\
Пусть все интенсивности $1$-пе\-ри\-о\-дич\-ны по~ $t$, а вместо~(\ref{5007})
выполнено условие:
\begin{equation*}
\int\limits_0^{1}\beta\left( t\right)\, dt = \beta^* > 0\,.
%\label{5009}
\end{equation*}
Тогда  $X(t)$  $1D$-экс\-по\-нен\-ци\-аль\-но слабо эргодичен и
\begin{equation*}
a= \beta^*, \quad M = \sup\limits_{|t-s|\le 1}
e^{\int\limits_s^{t}\beta\left( \tau\right)\, d\tau}\,.
%\label{5010}
\end{equation*}


\section{Простая система обслуживания с~групповым поступлением и~обслуживанием требований}

Рассмотрим здесь, следуя~\cite{z12b}, модель, в которой
возможно поступление групп требований размером не более~$m$
и обслуживание групп требований размером не более~$n$.
Интенсивность поступления группы  $k$ заявок  ($k \hm\le m$) равна $\lambda(t)$,
а интенсивность обслуживания
группы  $k$ заявок ($k \hm\le n$) равна $\mu(t)$.

\smallskip
Утверждения об эргодичности и скорости сходимости для этой модели были получены
в~\cite{z12b}. Сформулируем здесь основной результат.

\medskip

\noindent
\textbf{Утверждение~3.}\
\textit{Пусть найдется  $\delta \hm> 1$ такое, что}
\begin{equation}
 \int\limits_{0}^{\infty} \beta^*(t) \, dt = + \infty\,,
\label{5021}
\end{equation}
\textit{где}
\begin{multline*}
\beta^*(t) =  \left( \delta-1\right)\left(\left\{\fr{1}{\delta} + \fr{1+\delta}{\delta^2}
+ \dots{}\right.\right.\\
\left.{}\dots +\fr{1+\delta+\dots+\delta^{n-1}}{\delta^n}\right\}\mu(t) -{} \\
{} - \left. \left\{1 + \left(1+\delta\right)+\dots+\left(1+\delta+\dots+\delta^{m-1}\right)
\right\}\la(t) 
\vphantom{\fr{1}{\delta}}\right).\hspace*{-3.29407pt}
%\label{5022}
\end{multline*}
\textit{Тогда процесс $X\left( t\right) $ слабо эргодичен.}

\textit{Если найдутся положительные~$R^*$ и~$\nu^*$ такие, что при всех  $s,t$,
$0 \hm\le s \hm\le t$},
\begin{equation*}
  e^{-\int\limits_s^t \beta^*(u)\, du} \le R^* e^{-\nu^* (t-s)}\,,
%\label{5023}
\end{equation*}
\textit{то при  $d_k={\delta}^{k-1}$ процесс $X(t)$ $1D$-экспоненциально слабо эргодичен, $M=R^*,$ $a = \nu^*$ и справедливы соответствующие оценки устойчивости.}

\medskip

\noindent
\textbf{Замечаниие~2.}
Если интенсивности поступления и обслуживания требований постоянны, то условие~(\ref{5021})
эквивалентно неравенству
\begin{equation*}
 m\left(m+1\right)\la - n\left(n+1\right)\mu < 0\,.
%\label{5024}
\end{equation*}
В этом случае $X(t)$  $1D$-экс\-по\-нен\-ци\-аль\-но
сильно эргодичен, $M\hm=R^* \hm=1$, $a\hm = \nu^*\hm= \beta^*$.

\smallskip

Если интенсивности  $1$-пе\-ри\-о\-дич\-ны по~$t$, то условие~(\ref{5021})
эквивалентно неравенству
\begin{equation*}
m\left(m+1\right)\int\limits_0^1\lambda(t) \, dt  - n\left(n+1\right)
\int\limits_0^1\mu(t) \, dt < 0\,,
%\label{5025}
\end{equation*}
которое гарантирует $1D$-экс\-по\-нен\-ци\-аль\-но слабую эргодичность $X(t)$.

{\small\frenchspacing
{%\baselineskip=10.8pt
\addcontentsline{toc}{section}{References}
\begin{thebibliography}{99}



\bibitem{sto} %1
\Au{Штойян Д.} Качественные свойства и оценки стохастических моделей~/
Пер с нем.~--- М.: Мир, 1979. 272~с.
(\Au{Stoyan D.} Qualitative Eigenschaften und Absch$\ddot{\mbox{a}}$tzungen
stochastischer Modelle.~--- Berlin: Akademie-Verlag, 1977. 198~p.)


\bibitem{zol} %2
\Au{Zolotarev V.\,M.} Quantitative estimates for the continuity property
of queueing systems of type $G/G/\infty$~// Theory Probab. Appl., 1977. Vol.~22. P.~679--691.

\bibitem{kalash} %3
\Au{Калашников В.\,В.}
Качественный анализ сложных сис\-тем методом пробных функций.~--- М.: Наука, 1978. 248~с.


\bibitem{kz1} %4
Stability problems for stochastic models~// Lecture notes in
mathematics~/
Eds.\ V.\,V.~Kalashnikov, V.\,M.~Zolotarev.~--- N.Y.: Springer Verlag, 1983. Vol.~982. 295~p.

\bibitem{kz11} %5
Stability problems for stochastic models // Lecture notes in
mathematics~/ Eds.\ V.\,V.~Kalashnikov, B.~Penkov, V.\,M.~Zolotarev.~--- 
N.Y.: Springer Verlag, 1987. Vol.~1233. 224~p.

\bibitem{kz2} %6
Stability problems for stochastic models~// Lecture notes in mathematics~/
Eds.\ V.\,V.~Kalashnikov, V.\,M.~Zolotarev.~--- N.Y.: Springer Verlag, 1989. Vol.~1412. 380~p.

\bibitem{zkk1} %7
Stability problems for stochastic models~// Stability problems for stochastic models:
15th Perm Seminar Proceedings~/ Eds.\ V.\,M.~Zolotarev, V.\,M.~Kruglov,
V.\,Yu.~Korolev.~--- Perm, Russia, 1992. 312~p.

\bibitem{kz21} %8
Stability problems for stochastic models~// Lecture notes in mathematics~/
Eds.\ V.\,V.~Kalashnikov, V.\,M.~Zolotarev.~--- N.Y.: Springer Verlag, 1993. Vol.~1546. 229~p.



\bibitem{gk} %9
\Au{Gnedenko B.\,V., Korolev V.\,Yu.}
Random summation: Limit theorems and applications.~--- Boca Raton: CRC Press, 1996. 267~p.

\bibitem{kar81} %10
\Au{Карташов Н.\,В.} Сильно устойчивые цепи Маркова~// Проб\-ле\-мы
устой\-чи\-вости сто\-ха\-сти\-че\-ских моделей: Труды семинара.~---
М.: ВНИИСИ, 1981. С.~54--59.

\bibitem{kar85} %11
\Au{Карташов Н.\,В.} Неравенства в теоремах эргодич\-ности и устойчивости цепей
Маркова с общим фазовым пространством. I~// Теория вероятностей и ее применения, 1985.
Т.~30. Вып.~2. С.~230--240.

\bibitem{z85}   %12
\Au{Zeifman A.~I.}  Stability for contionuous-time nonhomogeneous 
Markov chains //  Lecture notes in
mathematics~/ Eds. V.\,V.~Kalashnikov, B.~Penkov, V.\,M.~Zolotarev.~---
N.Y.: Springer Verlag,  1985. Vol.~1233. P.~401--414.

\bibitem{af} %13
\Au{Aldous D.\,J., Fill~J.} Reversible Markov chains and random walks on graphs.
Ch.~8. {\sf http://www.stat.berkeley.edu/ users/aldous/RWG/book.html}. 

\bibitem{ds} %14
\Au{Diaconis P., Stroock~D.} Geometric bounds for eigenvalues of Markov chains~//
Ann. Appl. Probab., 1991. Vol.~1. P.~36--61.

\bibitem{mit03} %15
\Au{Mitrophanov A.\,Yu.}  Stability and exponential convergence of continuous-time
Markov chains~// J. Appl. Probab., 2003. Vol.~40. P.~970--979.

\bibitem{mit04} %16
\Au{Mitrophanov A.\,Yu.}  The spectral gap and perturbation
bounds for reversible continuous-time Markov chains~// J.~Appl.
Probab., 2004. Vol.~41. P.~1219--1222.

\bibitem{mit05} %17
\Au{Mitrophanov A.\,Yu.} Ergodicity coefficient and perturbation bounds for
continuous-time Markov chains~// Math. Inequal. Appl., 2005. Vol.~8. P.~159--168.

\bibitem{mit06} %18
\Au{Митрофанов А.\,Ю.} Оценки устойчивости для конечных однородных цепей
Маркова с непрерывным временем~// Теория вероятностей и ее применения, 2005. Т.~50.
Вып.~2. С.~371--379.

\bibitem{kar96}  %19
\Au{Kartashov N.~V.}  Strong stable Markov chains. -- Utrecht, Kiev: VSP, TBiMC, 1996. 138~p.

\bibitem{aan} %20
\Au{Altman E., Avrachenkov~K., N$\acute{\mbox{u}}$nez-Queija~R.}
Perturbation analysis for denumerable Markov chains with application
to queueing models~// Adv. Appl. Probab., 2004. Vol.~36.
P.~839--853.

\bibitem{ma} %21
\Au{Mouhoubi Z., A{\!\!\!\!\ptb{\"{\hspace*{-2pt}{\i}}}}ssani~D.}  New perturbation
bounds for denumerable Markov chains~//  Linear Algebra and Its Applications, 2010.
Vol.~432. P.~1627--1649.

\bibitem{fhl} %22
\Au{Ferre D., Herve L., Ledoux~J.} Regular perturbation of\linebreak
V-geometrically ergodic Markov chains~// J.~Appl. Probab., 2013. Vol.~50. P.~184--194.



\bibitem {mt} %23
\Au{Meyn S.\,P., Tweedie R.\,L.} Computable bounds for geometric convergence
rates of Markov chains~// Ann. Appl. Probab.,
1994. Vol.~4. P.~981--1012.

\bibitem{yl} %24
\Au{Yuanyuan L}.  Perturbation bounds for the stationary distributions
of Markov chains~// SIAM J. Matrix Anal.  Appl.,
2012. Vol.~33. P.~1057--1074.

\bibitem{g1} %25
\Au{Gnedenko D.\,B.} On a generalization of Erlang formulae~// Zastosow. Mat., 1972.
Vol.~12. P.~239--242.

\bibitem{g} %26
\Au{Gnedenko B.,  Soloviev A.} On the conditions of the existence of final probabilities for a Markov process~//
Math. Oper. Stat., 1973. Vol.~4. P.~379--390.



\bibitem{mw}
\Au{Massey W.\,A., Whitt~W.}  Uniform acceleration expansions for Markov
chains with time-varying rates~// Ann. Appl. Probab., 1998. Vol.~8. P.~1130--1155.

\bibitem{gm}
\Au{Гнеденко Б.\,В., Макаров И.\,П.} Свойства решений задачи с потерями
в случае периодических интенсивностей~// Дифф. уравнения, 1971. Т.~7. Вып.~9. С.~1696--1698.

\bibitem{z88}
\Au{Зейфман А.\,И.}  Качественные свойства неоднородных процессов рождения и гибели~//
Проб\-ле\-мы устойчивости стохастических моделей.~--- М.: ВНИИСИ, 1988. С.~32--40.

\bibitem{z89} \Au{Зейфман А.\,И.} Некоторые свойства сис\-те\-мы
с потерями в случае переменных интенсивностей~// Автоматика и телемеханика, 1989. Вып.~1. С.~107--113.

\bibitem{z91}
\Au{Zeifman A.\,I.} Some estimates of the rate of convergence for birth and death processes~//
J.~Appl. Probab., 1991. Vol.~28. P.~268--277.

\bibitem{z95}
\Au{Zeifman A.\,I.} 
Upper and lower bounds on the rate of convergence for nonhomogeneous
birth and death processes~// Stoch. Proc. Appl., 1995. Vol.~59. P.~157--173.

\bibitem{gz04} %33
\Au{Granovsky B.\,L., Zeifman A.\,I.} Nonstationary queues:
Estimation of the rate of convergence~// Queueing Syst., 2004. Vol.~46. P.~363--388.

\bibitem{z08} %34
\Au{Зейфман А.\,И., Бенинг В.\,Е., Соколов~И.\,А.}
Марковские цепи и модели с непрерывным временем.~--- М.: Элекс-КМ, 2008. 168~с.

\bibitem{dzp} %35
\Au{Ван Доорн Э.\,А., Зейфман~А.\,И., Панфилова~Т.\,Л.}
Оценки и асимптотика скорости сходимости для процессов рождения и гибели~//
Теория вероятностей и ее применения, 2009. Т.~54. С.~18--38.

\bibitem{z94} %36
\Au{Zeifman A.\,I.,  Isaacson~D.}
On strong ergodicity for nonhomogeneous continuous-time Markov chains~// Stoch. Proc. Appl.,
1994. Vol.~50. P.~263--273.

\bibitem{z98} %37
\Au{Zeifman A.\,I.}  Stability of birth and death processes~//  J.~Math. Sci., 1998.
Vol.~91. P.~3023--3031.

\bibitem{z12} %38
\Au{Zeifman A.\,I., Korotysheva A.}  Perturbation bounds for $M_t/M_t/N$
queue with catastrophes~// Stochastic Models, 2012. Vol.~28. P.~49--62.

\bibitem{z13c} %39
\Au{Zeifman A.\,I., Korolev~V.,  Korotysheva~A., Satin~Y.,  Bening~V.}
Perturbation bounds and truncations for a class of Markovian queues~//
Queueing Syst.,  2014 (in press).

\bibitem{z06}
\Au{Zeifman A.\,I., Leorato~S., Orsingher~E., Satin~Ya., Shilova~G.}
Some universal limits for nonhomogeneous birth and death processes~// Queueing Syst., 2006.
Vol.~52. P.~139--151.

\bibitem{DK} %41
\Au{Далецкий Ю.\,Л., Крейн М.\,Г.} Устойчивость решений
диф\-фе\-рен\-циальных уравнений в банаховом пространстве.~--- М.: Наука, 1970. 386~с.

\bibitem{z13a}
{\it Zeifman A.\,I., Satin Ya., Panfilova~T.} Limiting characteristics
for finite birth--death--catastrophe processes~// Math. Biosci., 2013.
Vol.~245. P.~96--102.

\bibitem{tkf}
\Au{Thorne J.\,L., Kishino H., Felsenstein~J.}
An evolutionary model for maximum-likelihood alignment of DNA sequences~//
J.~Mol. Evol., 1991. Vol.~33. P.~114--124.

\bibitem{mb} %44
\Au{Mitrophanov, A.\,Yu., Borodovsky~M.}  Convergence rate estimation for the TKF91
model of biological sequence length evolution~//
Math. Biosci., 2007. Vol.~209. P.~470--485.

\bibitem{s11}
\Au{Сатин Я.\,А., Зейфман А.\,И., Коротышева~А.\,В.,  Шоргин~С.\,Я.}
Об одном классе марковских систем обслуживания~//  Информатика и её применения, 2011.
Т.~5. Вып.~4. С.~18--24.


\bibitem{s12} %46
\Au{Сатин Я.\,А., Зейфман А.\,И., Коротышева~А.\,В.}
О~ско\-рости сходимости и усечениях для одного класса марковских сис\-тем обслуживания~//
Теория вероятностей и ее применения, 2012. Т.~57. Вып.~3. С.~611--621.

\bibitem{z13b} %47
\Au{Zeifman A.\,I., Korotysheva~A., Satin~Ya., Shilova~G., Panfilova~T.}
On a queueing model with group services~// Lecture notes in communications
in computer and information science. 2013. Vol.~356. P.~198--205.

 
\bibitem{z13d}  %48
\Au{Zeifman A.\,I.,  Satin~Y., Shilova~G., Korolev~V., Bening~V., Shorgin~S.}
On $M_t/M_t/S$ type queue with group services~//  ECMS 2013:
27th  Conference (European) on Modeling and Simulation Proceedings.~---
Alesund, Norway, 2013. P.~604--609.

\bibitem{z12b} \Au{ Зейфман А.\,И., Коротышева~А.\,В., Сатин~Я.\,А.,  Шоргин~С.\,Я.}
Оценки в нуль-эр\-го\-ди\-че\-ском случае для некоторых сис\-тем обслуживания~//
Информатика и её применения, 2012. Т.~6. Вып.~4. С.~27--33.

\end{thebibliography}
} }

\end{multicols}

\hfill{\small\textit{Поступила в редакцию 27.08.13}}


\vspace*{12pt}

\hrule

\vspace*{2pt}

\hrule


\def\tit{GENERAL BOUNDS FOR~NONSTATIONARY CONTINUOUS-TIME MARKOV CHAINS}

\def\titkol{General bounds for~nonstationary continuous-time Markov chains}

\def\aut{A.\,I.~Zeifman$^{1,2}$, V.\,Yu.~Korolev$^{2,3}$, A.\,V.~Korotysheva$^{1}$,
and~S.\,Ya.~Shorgin$^2$}
\def\autkol{A.\,I.~Zeifman, V.\,Yu.~Korolev, A.\,V.~Korotysheva,
and~S.\,Ya.~Shorgin}


\titel{\tit}{\aut}{\autkol}{\titkol}

\vspace*{-9pt}

\noindent
$^1$Vologda State University, 15 Lenin Str., Vologda 160000, Russian Federation

\noindent
$^2$Institute of Informatics Problems, Russian Academy of Sciences,
44-2 Vavilov Str., Moscow 119333, Russian\\
$\hphantom{^1}$Federation

\noindent
$^3$Department of Mathematical Statistics, Faculty of Computational Mathematics and
Cybernetics,\\
$\hphantom{^1}$M.\,V.~Lomonosov Moscow State University, 
1-52 Leninskiye Gory, GSP-1, Moscow 119991, Russian Federation



\def\leftfootline{\small{\textbf{\thepage}
\hfill INFORMATIKA I EE PRIMENENIYA~--- INFORMATICS AND APPLICATIONS\ \ \ 2014\ \ \ volume~8\ \ \ issue\ 1}
}%
 \def\rightfootline{\small{INFORMATIKA I EE PRIMENENIYA~--- INFORMATICS AND APPLICATIONS\ \ \ 2014\ \ \ volume~8\ \ \ issue\ 1
\hfill \textbf{\thepage}}}

\vspace*{6pt}

\Abste{A general approach for obtaining perturbation bounds of
nonstationary continuous-time  Markov chains is considered. The 
suggested approach deals with a special
weighted norms related to total variation. The method is based on the
notion of a logarithmic norm of a linear operator function and respective bounds for
the Cauchy operator of a differential equation. Special
transformations of the reduced intensity matrix of the process are applied. 
The statements are proved which provide estimates
of perturbation of probability characteristics 
for the case of absence of ergodicity
in uniform operator topology.
Birth--death--catastrophe queueing models and queueing systems
with batch arrivals and group services are also considered in the paper.  Some classes of such systems are
studied, and bounds of perturbations are obtained. Particularly, such bounds are given
for the $M_t/M_t/S$ queueing system with possible catastrophes and a simple model of
a queueing system with batch arrivals and group services is analyzed. Moreover, 
approximations of limiting characteristics  are considered for the queueing model.}


\KWE{nonstationary continuous-time chains and models; nonstationary Markov chains;
perturbation bounds; special norms; queueing models}


\DOI{10.14357/19922264140111}

\Ack
\noindent
The research is supported by the Russian Foundation for Basic Research (grants 
Nos.\,12-07-00109, 12-07-00115, 13-07-00223, and
14-07-00041).

\begin{multicols}{2}

\renewcommand{\bibname}{\protect\rmfamily References}
%\renewcommand{\bibname}{\large\protect\rm References}

{\small\frenchspacing
{%\baselineskip=10.8pt
\addcontentsline{toc}{section}{References}
\begin{thebibliography}{99}


\bibitem{2-ze} %1
\Aue{Stoyan, D.} 1977. Qualitative Eigenschaften und Absch$\ddot{\mbox{a}}$tzungen
stochastischer Modelle. Berlin: Akademie-Verlag. 198~p.
\bibitem{3-ze} %2
\Aue{Zolotarev, V.\,M.} 1978. Quantitative estimates for the continuity property
of queueing systems of type $G/G/\infty$. \textit{Theory Probab. Appl.} 22(4):679--691.
doi: 10.1137/1122083.
\bibitem{1-ze} %3
\Aue{Kalashnikov, V.\,V.} 1978.
Kachestvennyy analiz slozhnykh sistem metodom probnykh funktsiy
[Qualitative analysis of complex systems behavior by test functions method].
Moscow: Nauka. 248~p.
\bibitem{4-ze}
Kalashnikov, V.\,V., and V.\,M.~Zolotarev, eds. 1983.
Stability problems for stochastic models. \textit{Lecture notes in mathematics}.
N.Y.: Springer Verlag. Vol.~983. 295~p.
\bibitem{5-ze}
Kalashnikov, V.\,V., V.~Penkov, and V.\,M.~Zolotarev, eds.
1987. Stability problems for stochastic models.
\textit{Lecture notes in mathematics}.  N.Y.: Springer Verlag. Vol.~1233. 224~p.
\bibitem{6-ze}
Kalashnikov, V.\,V., and V.\,M.~Zolotarev, eds. 1989.
Stability problems for stochastic models. \textit{Lecture notes in mathematics}.
 N.Y.: Springer Verlag. Vol.~1412. 380~p.
 \bibitem{8-ze} %7
Zolotarev, V.\,M., V.\,M.~Kruglov, and V.\,Yu.~Korolev, eds.
1994. Stability problems for stochastic models.
\textit{15th Perm Seminar Proceedings}. Perm, Russia. 312~p.

\bibitem{7-ze} %8
Kalashnikov, V.\,V., and V.\,M.~Zolotarev, eds.
1993. Stability problems for stochastic models. \textit{Lecture notes in mathematics}.
 N.Y.: Springer Verlag. Vol.~1546. 229~p.

\bibitem{9-ze} 
\Aue{Gnedenko, B.\,V., and V.\,Yu.~Korolev}. 1996. Random summation: Limit
theorems and applications. Boca Raton: CRC Press. 267~p.
\bibitem{10-ze}
\Aue{Kartashov, N.\,V.} 1981. Sil'no ustoychivye tsepi Markova
[Strong stable Markov chains].
\textit{Problemy ustoychivosti stokhasticheskikh modeley:
Trudy seminara} [\textit{``Stability Problems for Stochastic Models'' Seminar Proceedings}].
Moscow: VNIISI. 54--59.
\bibitem{11-ze}
\Aue{Kartashov, N.\,V.} 1986. Inequalities in theorems of ergodicity and stability for
Markov chains with common phase space.~I. \textit{Theory Probab. Appl.} 30(2):247--259.
doi: 10.1137/1122083.
\bibitem{12-ze}
\Aue{Zeifman, A.\,I.} 1985. Stability for contionuous-time nonhomogeneous
Markov chains. \textit{Lecture notes in mathematics}.  N.Y.: Springer Verlag. 1155:401--414.
\bibitem{13-ze}
\Aue{Aldous, D.\,J., and J.~Fill}.
Reversible Markov chains and random walks on graphs. Ch.~8.
Available at: {\sf http://www.stat.berkeley.edu/users/aldous/RWG/ book.html} (accessed December 25, 2013).
\bibitem{14-ze}
\Aue{Diaconis, P., and D.~Stroock}. 1991.
Geometric bounds for eigenvalues of Markov chains. \textit{Ann. Appl. Probab.} 1:36--61.
doi: 10.1214/aoap/1177005980.
\bibitem{15-ze}
\Aue{Mitrophanov, A.\,Yu.} 2003. Stability and exponential
convergence of continuous-time Markov chains. \textit{J.~Appl. Probab.} 40:970--979. doi: 10.1239/jap/1067436094.
\bibitem{16-ze}
\Aue{Mitrophanov, A.\,Yu.} 2004. The spectral gap and perturbation bounds for
reversible continuous-time Markov chains. \textit{J.~Appl. Probab.} 41:1219--1222. 
doi: 10.1239/ jap/1101840568.
\bibitem{17-ze}
\Aue{Mitrophanov, A.\,Yu.} 2005. Ergodicity coefficient and perturbation
bounds for continuous-time Markov chains. \textit{Math. Inequal. Appl.} 8:159--168. doi: 10.7153/mia-08-15.
\bibitem{18-ze}
\Aue{Mitrophanov, A.\,Yu.} 2006. Stability estimates for finite homogeneous
continuous-time Markov chains. \textit{Theory Probab. Appl.} 50(2):319--326. 
doi: 10.1137/ S0040585X97981718.

\bibitem{19-ze}
\Aue{Kartashov, N.\,V.} 1996. \textit{Strong stable Markov chains}. Utrecht, Kiev: VSP, TBiMC. 138~p.
\bibitem{20-ze}
\Aue{Altman, E., K.~Avrachenkov, and R.\,N$\acute{\mbox{u}}$nez-Queija}. 2004.
Perturbation analysis for denumerable Markov chains with application to queueing models.
\textit{Adv. Appl. Probab.} 36: 839--853. doi: 10.1239/aap/1093962237.

\bibitem{22-ze} %21
\Aue{Mouhoubi, Z., and D.\,A{\!\!\!\!\!\ptb{\"{\hspace*{-2pt}{\i}}}}ssani}. 2010.
New perturbation bounds for denumerable Markov chains.
\textit{Linear Algebra and Its Applications}. 432:1627--1649. doi: 10.1016/j.laa.2009.11.020.

\bibitem{21-ze} %22
\Aue{Ferre, D., L.~Herve, and J.~Ledoux}. 2013.  Regular perturbation of
V-geometrically ergodic Markov chains. 
\textit{J. Appl. Probab.} 50:184--194. doi: 10.1239/jap/1363784432.

\bibitem{23-ze}
\Aue{Meyn, S.\,P., and R.\,L.~Tweedie}. 1994. Computable bounds for geometric
convergence rates of Markov chains. \textit{Ann. Appl. Probab}. 4: 981--1012. 
doi:10.1214/ aoap/1177004900.
\bibitem{24-ze}
\Aue{Yuanyuan, L.} 2012. Perturbation bounds for the stationary distributions of
Markov chains. \textit{SIAM J.~Matrix Anal. Appl.} 33:1057--1074. doi: 10.1137/110838753.

\bibitem{26-ze} %25
\Aue{Gnedenko, D.\,B.} 1972. On a generalization of Erlang formulae.
\textit{Zastosow. Mat.} 12:239--242.

\bibitem{25-ze} %26
\Aue{Gnedenko, B., and A.~Soloviev}. 1973. On the conditions of the existence of
final probabilities for a Markov process. \textit{Math. Oper. Stat}. 4:379--390.

\bibitem{27-ze}
\Aue{Massey, W.\,A., and W.~Whitt}. 1998. Uniform acceleration expansions for
Markov chains with time-varying rates. \textit{Ann. Appl. Probab}. 8:1130--1155. 
doi: 10.1214/ aoap/1028903375.
\bibitem{28-ze}
\Aue{Gnedenko, B.\,V., and I.\,P.~Makarov}. 1971. Svoystva re\-she\-niy zadachi s
poteryami v sluchae periodicheskikh intensivnostey
[Properties of the solution to the loss problem for periodic rates].
\textit{Differentsial'nye Uravneniya} [\textit{Differential Equations}] 7:1696--1698.
\bibitem{29-ze}
\Aue{Zeifman, A.\,I.} 1991. Qualitative properties of inhomogeneous birth and
death processes. \textit{J. Math. Sci.} 57:3217--3224. doi: 10.1007/BF01099019.
\bibitem{30-ze}
\Aue{Zeifman, A.\,I.} 1989. Properties of a system with losses in the case of
variable rates. \textit{Autom. Rem. Contr.} 50:82--87.
\bibitem{31-ze}
\Aue{Zeifman, A.\,I.} 1991. Some estimates of the rate of convergence for
birth and death processes. \textit{J. Appl. Probab.} 28:268--277.
\bibitem{32-ze}
\Aue{Zeifman, A.\,I.} 1995. Upper and lower bounds on the rate of convergence
for nonhomogeneous birth and death processes. \textit{Stoch. Proc. Appl.} 59:157--173. doi: 10.1016/0304-4149(95)00028-6.
\bibitem{33-ze}
\Aue{Granovsky, B.\,L., and A.\,I.~Zeifman}. 2004. Nonstationary queues: Estimation of the
rate of convergence. \textit{Queueing Syst.} 46:363--388. 
doi: 10.1023/B:QUES. 0000027991.19758.b4.
\bibitem{34-ze}
\Aue{Zeifman, A.\,I., V.\,E.~Bening, and I.\,A.~Sokolov}. 2008. \textit{Markovskie tsepi
i modeli s nepreryvnym vremenem} [\textit{Continuous-time Markov chains and models}].
Moscow: Elex-KM. 168~p.
\bibitem{35-ze}
\Aue{Van Doorn, E.\,A., A.\,I.~Zeifman, and T.\,L.~Panfilova}. 2010.
Bounds and asymptotics for the rate of convergence of birth--death processes.
\textit{Theory Probab. Appl.} 54:97--113. doi: 10.1137/S0040585X97984097.
\bibitem{36-ze}
\Aue{Zeifman, A.\,I., and D.~Isaacson}. 1994. On strong ergodicity for nonhomogeneous
continuous-time Markov chains. \textit{Stoch. Proc. Appl.} 50:263--273. doi: 10.1016/0304-4149(94)90123-6.
\bibitem{37-ze}
\Aue{Zeifman, A.\,I.} 1998. Stability of birth and death processes.
\textit{J. Math. Sci.} 91:3023--3031. doi: 10.1007/BF02432876.
\bibitem{38-ze}
\Aue{Zeifman, A.\,I., and A.~Korotysheva}. 2012.
Perturbation bounds for $M_t/M_t/N$ queue with catastrophes.
\textit{Stochastic models} 28:49--62. doi: 10.1080/15326349.2011.614900.
\bibitem{39-ze}
\Aue{Zeifman, A., V.~Korolev, A.~Korotysheva, Y.~Satin, and V.~Bening}.
2014 (in press). Perturbation bounds and truncations for a class of Markovian queues.
\textit{Queueing Syst.}
\bibitem{40-ze}
\Aue{Zeifman, A.\,I., S.~Leorato, E.~Orsingher, Y.~Satin, and G.~Shilova}. 2006.
Some universal limits for nonhomogeneous birth and death processes.
\textit{Queueing Syst.} 52:139--151. doi: 10.1007/s11134-006-4353-9.
\columnbreak

\bibitem{41-ze}
\Aue{Daleckij, Ju, L. and M.\,G.~Krein}. 1974. Stability of solutions of differential
equations in Banach space. \textit{Am. Math. Soc. Transl.} 43. 386~p.
\bibitem{42-ze}
\Aue{Zeifman, A.\,I., Ya.~Satin, and T.~Panfilova}. 2013. Limiting characteristics
for finite birth--death--catastrophe processes. \textit{Math. Biosci.} 245:96--102. 
doi: 10.1016/ j.mbs.2013.02.009.
\bibitem{43-ze}
\Aue{Thorne, J.\,L., H.~Kishino, and J.~Felsenstein}. 1991. An evolutionary model for
maximum-likelihood alignment of DNA sequences. \textit{J.~Mol. Evol.} 33:114--124. 
doi: 10.1007/ BF02193625.
\bibitem{44-ze}
\Aue{Mitrophanov, A.\,Yu., and M.~Borodovsky}. 2007. Convergence rate estimation for
the TKF91 model of biological sequence length evolution. \textit{Math. Biosci.}
209:470--485. doi: 10.1016/j.mbs.2007.02.011.
\bibitem{45-ze}
\Aue{Satin, Ya.\,A., A.\,I.~Zeifman, A.\,V.~Korotysheva, and S.\,Ya.~Shorgin}.
2011. Ob odnom klasse markovskikh sistem obsluzhivaniya [On a class of Markovian queues].
\textit{Informatika i ee Primeneniya}~--- \textit{Inform. Appl}. 5(4):18--24.
\bibitem{46-ze}
\Aue{Satin, Ya.\,A., A.\,I.~Zeifman, A.\,V.~Korotysheva}. 2013.
On the rate of convergence and truncations for a class of Markovian queueing systems.
\textit{Theory Probab. Appl.} 57:529--539. doi: 10.1137/S0040585X97986151.
\bibitem{47-ze}
\Aue{Zeifman, A.\,I., A.~Korotysheva, Ya.~Satin, G.~Shilova, and T.~Panfilova}.
2013. On a queueing model with group services.
\textit{Lecture notes in communications in computer and information science}. 
356:198--205. doi: 10.1007/978-3-642-35980-4\_22.
\bibitem{48-ze}
\Aue{Zeifman, A.\,I., Y.~Satin, G.~Shilova, V.~Korolev, V.~Bening, and S.~Shorgin}.
2013. On $M_t/M_t/S$ type queue with group services.
\textit{ECMS 2013: 27th  Conference (European) on Modeling and Simulation Proceedings}.
Alesund, Norway. 604--609. doi: 10.7148/2013-0604.
\bibitem{49-ze}
\Aue{Zeifman, A.\,I., A.\,V.~Korotysheva, Ya.\,A.~Satin,  and S.\,Ya.~Shorgin}. 2012.
Otsenki v nul'-ergodicheskom slu\-chae dlya nekotorykh sistem obsluzhivaniya
[Bounds in null ergodic case for some queueing systems].
\textit{Informatika i ee Primeneniya}~--- \textit{Inform. Appl.} 6(4):27--33.

\end{thebibliography}
} }

\end{multicols}

\vspace*{-9pt}

\hfill{\small\textit{Received August 27, 2013}}

\vspace*{-20pt}

\Contr


\textbf{Zeifman Alexander I.}\ (b.\ 1954)~--- Doctor of science in physics and
mathematics, professor, Head of Department  of Applied Mathematics, Vologda State University,
15 Lenin Str., Vologda 160000, Russian Federation;
senior researcher, Institute of of Informatics Problems, Russian Academy of Sciences,
44-2 Vavilov Str., Moscow 119333, Russian Federation;
principal scientist, ISEDT, Russian Academy of Sciences,
Vologda, Russian Federation; a\_zeifman@mail.ru

\vspace*{2pt}

\textbf{Korolev Victor Yu.}\ (b.\ 1954)~--- Doctor of Science in physics and mathematics, professor,
Department of Mathematical Statistics, Faculty of Computational Mathematics and Cybernetics,
M.\,V.~Lomonosov Moscow State University,
1-52 Leninskiye Gory, GSP-1, Moscow 119991, Russian Federation; leading scientist,
Institute of Informatics Problems, Russian Academy of Sciences,
44-2 Vavilov Str.,
Moscow 119333, Russian Federation; vkorolev@cs.msu.su

\vspace*{2pt}

\textbf{Korotysheva Anna V.} (b.\ 1988)~--- senior lecturer, Vologda State  University,
15 Lenin Str., Vologda 160000, Russian Federation;
a\_korotysheva@mail.ru

\vspace*{2pt}


\textbf{Shorgin Sergey Ya.}\ (b.\ 1952)~--- Doctor of science in physics and
mathematics, professor, Deputy Director, Institute of Informatics Problems,
Russian Academy of Sciences, 44-2 Vavilov Str.,
Moscow 119333, Russian Federation; SShorgin@ipiran.ru


 \label{end\stat}

\renewcommand{\bibname}{\protect\rm Литература}