\def\stat{zykin}

\def\tit{ДИНАМИЧЕСКИЕ КОНТЕКСТЫ БАЗЫ ДАННЫХ РЕЛЯЦИОННОГО ТИПА$^*$}

\def\titkol{Динамические контексты базы данных реляционного типа}

\def\autkol{С.\,В. Зыкин}

\def\aut{С.\,В. Зыкин$^1$}

\titel{\tit}{\aut}{\autkol}{\titkol}

{\renewcommand{\thefootnote}{\fnsymbol{footnote}} 
\footnotetext[1]{Работа выполнена при поддержке РФФИ (грант №\,12-07-00066-а).}}

\renewcommand{\thefootnote}{\arabic{footnote}}
\footnotetext[1]{Институт математики им.\ С.\,Л.~Соболева Сибирского отделения 
Российской академии наук, szykin@mail.ru} 
   
 

\Abst{Предложена технология динамического формирования 
представления данных. Эта технология является развитием методов 
аналитической обработки данных (OLAP~--- online analytical processing). 
Источником данных служит 
реляционная база данных (БД) с произвольной схемой (не обязательно 
иерархической). Целевое представление данных~--- композиционная таблица, 
которая позволяет представлять многомерные данные на плоскости. Эта 
таблица предполагает раздельное формирование размерностей с последующим 
сопоставлением мер размерностям в таблице. Основой промежуточных 
представлений данных является таблица связанных соединений, 
удовлетворяющая контекстным и логическим ограничениям. 
Предложены алгоритмы формирования таких таблиц и исследованы их 
свойства. Особое внимание уделено рассмотрению контекстов, используемых 
при формировании таблиц связанных соединений. Для создания контекстов 
предложен алгоритм направленного перебора и на примере выполнен 
сравнительный анализ работы алгоритмов формирования контекстов. 
Исследованные свойства контекстов и предложенные алгоритмы 
предназначены для автоматизации работы пользователя при формировании 
новых представлений данных.}

\KW{реляционная база данных; контекст; соединение без потерь информации; 
композиционная таблица}

\DOI{10.14357/19922264140108}

\vskip 20pt plus 9pt minus 6pt

      \thispagestyle{headings}

      \begin{multicols}{2}

            \label{st\stat}
  
\section{Введение}
  
  В работах, посвященных OLAP, значительное внимание уделяется 
исследованию свойств моделей многомерных представлений данных 
(гиперкубов)~[1--3] и операциям их преобразования~\cite{2-z, 4-z} с целью 
получения представления, необходимого для анализа данных. Особое внимание 
уделяется построению иерархий в размерностях~[2, 3, 5--7], %\cite{2-z, 3-z, 5-z, 6-z, 7-z}, 
что  позволяет гарантировать корректность операций агрегации данных. 
В~работах~\cite{3-z, 5-z, 7-z} рас\-смат\-ри\-ва\-ют\-ся нормальные формы для 
многомерных моделей данных, которые позволяют контролировать значения 
NULL в иерархиях размерностей. Во всех перечисленных работах 
предполагается ручное формирование и анализ корректности размерностей 
гиперкубов.
  
  В данной работе предлагается исследовать проб\-ле\-му автоматизации 
формирования размерностей с использованием свойств исходной реляционной 
БД. Это особенно актуально для сис\-тем, в которых не 
предполагается хранение многомерного пред\-став\-ле\-ния данных целиком 
(технология \mbox{MOLAP}~--- multidimensional \mbox{OLAP}). 
Кроме того, считается недопустимым преобразование 
операционной БД с целью получения иерархической схемы <<звезда>> или 
<<снежинка>> (технология \mbox{ROLAP}~--- relational \mbox{OLAP}). Отправной точкой служит предположение 
о том, что основой аналитической работы пользователя является 
необходимость формирования новых гиперкубов из исходного реляционного 
пред\-став\-ле\-ния данных, а не многократное манипулирование одним и тем же 
гиперкубом. Новые гиперкубы нужны при выявлении скрытых 
закономерностей в данных и проведения анализа данных, не предусмотренного 
при проектировании складов данных по технологии \mbox{MOLAP} либо \mbox{ROLAP}.
  
  Рассмотрим содержательную сторону постановки задачи и обсудим подходы 
к ее решению. Основной целью данной работы является повышение уровня 
автоматизации работы пользователя при формировании представления нового 
гиперкуба с возможностью его визуализации. Для демонстрации предлагаемых 
подходов к решению поставленной задачи рассмотрим пример.

\begin{table*}\small
\begin{center}
\Caption{Фрагмент расписания занятий}
\vspace*{2ex}

\tabcolsep=5pt
\begin{tabular}{|c|c|c|c|c|c|c|c|}
\hline
\multicolumn{2}{|c|}{\textbf{№ курса}} & \multicolumn{6}{c|}{2}\\
\hline
\multicolumn{2}{|c|}{\textbf{Код группы}} & \multicolumn{3}{c|}{M-210} &\multicolumn{3}{c|}{M-220}\\
\hline
\textbf{День недели}&\tabcolsep=0pt\begin{tabular}{c}\textbf{Время}\\ \textbf{начала}\\ \textbf{занятия}\end{tabular}&
\textit{Предмет}&\tabcolsep=0pt\begin{tabular}{c}\textit{ФИО}\\ \textit{преподавателя}\end{tabular}&
\tabcolsep=0pt\begin{tabular}{c}\textit{№}\\ \textit{аудитории}\end{tabular}&\textit{Предмет}&
\tabcolsep=0pt\begin{tabular}{c}\textit{ФИО} \\ \textit{преподавателя}\end{tabular}&
\tabcolsep=0pt\begin{tabular}{c}\textit{№}\\ \textit{аудитории}\end{tabular}\\
\hline
\multicolumn{1}{|c|}{\raisebox{-18pt}[0pt][0pt]{Понедельник}}&8-00&Физика&Чанышев О.\,Г.&1-330&&&\\
\cline{2-8}
&9-40&Химия&Чигишев О.\,М.&1-330&История&Дергачев А.\,С.&2-110\\
\cline{2-8}
&11-30&История&Дергачев А.\,С.&1-330&Химия&Чигишев О.\,М.&2-110\\
\cline{2-8}
&13-15&Литература&Арутян В.\,А.&1-330&&&\\
\hline
\multicolumn{1}{|c|}{\raisebox{-6pt}[0pt][0pt]{Вторник}}&8-00&&&&&&\\
\cline{2-8}
&9-40&Философия&Дергачев А.\,С.&1-330&Философия&Дергачев А.\,С.&1-330\\
\hline
\end{tabular}
\end{center}
\end{table*}

  \begin{table*}[b]\small
  \begin{center}
  \Caption{Фрагмент сводной ведомости}
  \vspace*{2ex}
  
  \begin{tabular}{|c|c|c|c|c|c|c|c|}
    \multicolumn{8}{l}{\textbf{Ограничение}: Код группы\;=\;М-220.}\\
    \hline
       \multicolumn{2}{|c|}{\textbf{Предмет}} &\multicolumn{2}{c|}{{Физика}} &
       \multicolumn{2}{c|}{{Философия} }
&\multicolumn{2}{c|}{{Химия}} \\
    \hline
    \multicolumn{2}{|c|}{\textbf{Вид аттестации}} & \multicolumn{1}{c|}{Реферат} & &
    \multicolumn{1}{c|}{Доклад} & &\multicolumn{1}{c|}{Лаб. раб.} &\\
    \cline{1-3}
    \cline{5-5}
    \cline{7-7}
    \textbf{№ студента} & \textbf{ФИО студента} & 
    \textit{Балл} & \multicolumn{1}{c|}{\raisebox{6pt}[0pt][0pt]{\textit{Оценка}}} & 
    \textit{Балл} & \multicolumn{1}{c|}{\raisebox{6pt}[0pt][0pt]{\textit{Оценка}}} & \textit{Балл} & 
    \multicolumn{1}{c|}{\raisebox{6pt}[0pt][0pt]{\textit{Оценка}}}\\
    \hline
1&Алексенко С.\,В.&44&4&46&4&&\\
2&Белоусов П.\,О.&52&5&43&4&&\\
3&Бессараб О.\,П.&40&4&53&5&&\\
4&Вяткин М.\,С.&42&4&&&&\\
5&Драница А.\,А.&&&&&&\\
6&Ефимов Е.\,С.&52&5&52&5&&\\
\hline
\end{tabular}
\end{center}
\end{table*}
  
  \medskip
  
  \noindent
  \textbf{Пример 1.} Рассмотрим упрощенный фрагмент схемы БД учебного 
заведения~\cite{8-z}, где подчеркнуты ключевые атрибуты отношений:

$R_1$\;=\;\textit{Студенты} (\underline{№~студента}, \underline{№~группы}, 
ФИО~студента);

$R_2$\;=\;\textit{Список групп} (\underline{№ группы}, Код группы, 
№~спе\-ци\-аль\-ности, № курса);

$R_3$\;=\;\textit{Предметы} (\underline{№ предмета}, Предмет);

$R_4$\;=\;\textit{Преподаватели} (\underline{№ преподавателя}, 
ФИО преподавателя);

$R_5$\;=\;\textit{Неделя} (\underline{№ дня недели}, День недели);

$R_6$\;=\;\textit{Начало занятий} (\underline{№ занятия}, 
Время начала занятия);

$R_7$\;=\;\textit{Оценки} (\underline{№ студента}, \underline{№ группы}, 
\underline{№ предмета}, Оценка);

$R_8$\;=\;\textit{Расписание} (\underline{№ группы}, \underline{№ дня недели}, 
\underline{№ занятия}, № предмета, № преподавателя, № аудитории);

$R_9$\;=\;\textit{Специальности} (\underline{№ специальности}, 
Наименование специальности);

$R_{10}$\;=\;\textit{Нагрузка} (\underline{№~спе\-ци\-аль\-ности}, 
\underline{№ предме-}\linebreak \underline{та}, Количество часов, Контроль успеваемости);

$R_{11}$\;=\;\textit{Аттестация} (\underline{№ студента}, \underline{№ группы}, 
\underline{№ предмета}, \underline{Вид аттестации}, Балл).
  
  Один из возможных гиперкубов представлен в табл.~1.
  

  
  Атрибуты размерностей в табл.~1 представлены жирным шрифтом, атрибуты 
мер~--- курсивом, значения атрибутов~--- обычным шрифтом.
  
  Другой гиперкуб представлен в табл.~2.
  

  
  В табл.~1 заданы две размерности: по горизонтали \{День недели, Время 
начала занятий\} и по вертикали \{№~курса, Код группы\}. К~раз\-мер\-ности по 
вертикали приписаны меры {Предмет, ФИО преподавателя, №\,аудитории}. 
В~табл.~2 имеется две вертикальных размерности: \{Предмет\}, к которой 
приписана мера \{Оценка\}, и размерность \{Предмет, Вид аттестации\}, к 
которой приписана мера \{Балл\}. В~этих размерностях объединены значения 
общего атрибута \{Предмет\}. Горизонтальная размерность в табл.~2 
\{№\,студента, ФИО студента\} является общей для обеих вертикальных 
размерностей. Кроме того, на значения в табл.~2 навешено логическое 
ограничение: <<Код группы\;=$\linebreak $=\;М-220>>.

Представление данных в табл.~1 и~2 
будем называть \textit{композиционной таблицей} (композиция вертикальных 
размерностей). Далее общую размерность будем обозначать символом~$X$. 
Вертикальные размерности обозначим $Y_1$, $Y_2$ и~т.\,д.
  
  Предложенная компоновка данных в табл.~1 и~2 является удобной для 
визуального анализа многомерных данных, поскольку все данные 
располагаются на плоскости, значения мер сопоставлены размерностям и 
представление данных не является разреженным.
  
  Для автоматизации построения композиционной таблицы предлагается 
следующая последовательность формирования ее представления:

\pagebreak

\noindent
  \begin{enumerate}[1.]
  \item Пользователь из списка атрибутов БД формирует множества атрибутов: 
размерности $X, Y_1, Y_2, \ldots , Y_N$ и соответствующие им меры $Z_1, Z_2, 
\ldots , Z_N$. Мера~$Z_1$ соответствует размерностям $(X, Y_1)$, мера $Z_2$ 
соответствует размерностям $(X, Y_2)$ и~т.\,д. Естественными являются 
ограничения: $X\cap Y_i \hm= \varnothing$, $(X\cup Y_i) \cap Z_i \hm=\varnothing$, 
$i\hm= 1, 2, \ldots , N$. Дополнительным технологическим ограничением 
является запрет на использование атрибута в качестве меры, если на него 
установлено ограничение в логическом выражении. Основанием для такого 
ограничения является возможность наличия значений размерностей в 
представлении и отсутствия соответствующих значений мер, хотя они есть в 
БД. Это может служить причиной для неверной интерпретации результатов.
  
  \item Формирование иерархий размерностей для множеств атрибутов~$X, 
Y_1, Y_2, \ldots , Y_N$. Иерархии формируются автоматически по правилам, 
рассмотренным в работе~\cite{9-z}, и пользователю предлагается только их 
модифицировать.
  \item По специальному шаблону задаются логические ограничения на 
размерности $F_0(X), F_1(Y_1), F_2(Y_2), \ldots , F_N(Y_N)$. По умолчанию 
каждая формула есть конъюнкция условий определенности (IS\ NOT\ NULL) 
для атрибутов размерности.
  
  \item Формирование контекстов размерностей $C_0, C_1, \ldots , C_N$ 
(некоторые контексты могут быть пустыми, а некоторые~--- 
псевдоконтекстами). Далее будут представлены соответствующие определения 
и алгоритмы формирования контекстов.
  \item Формирование контекста приложения~$C$ и соответствующей 
реализации таблицы связанных соединений~$s$. С~учетом структуры 
композиционной таблицы очевидно, что логическое ограничение на кортежи 
из~$s$ имеет следующий вид:
  $$
F(C)=F_0(X)\wedge (F_1(Y_1) \vee F_2(Y_2)\vee \cdots \vee F_N(Y_N))\,.
$$
  \item Формирование реализаций размерностей $X, Y_1, Y_2, \ldots , Y_N$ с 
сортировкой значений в соответ\-ст\-вии с иерархией. Если контекст размерности 
не пуст, то он используется для формирования, в противном случае реализация 
размерности является проекцией~$s$.
  \item Формирование реализации (представления) композиционной таблицы 
(заполнение значений мер на соответствующих местах таблицы).
  \end{enumerate}
  
  Пользователь вручную выполняет шаги~1 и~3 и осуществляет выбор 
предложенных вариантов в шагах~2, 4 и~5. Все остальные операции 
выполняются автоматически.
  
  Заметим, что в предложенной последо\-ва\-тель\-ности шагов формирования 
композиционной таб\-ли\-цы исключается необходимость ка\-ким-ли\-бо образом 
модифицировать исходную операционную БД, что делает возможным 
реализовать все принципы проектирования БД~\cite{10-z, 11-z}, в том числе 
самый важный~--- принцип независимости данных.
  
  В работах, посвященных построению гиперкубического представления 
данных, в качестве промежуточных моделей используются SQL-таб\-ли\-цы. 
Интерфейс между БД и хранилищем данных программируется. Предлагаемый в 
данной работе подход исключает затраты на программирование.
  
  В предлагаемой статье рассмотрена общая постановка задачи, включающая 
шаги от построения схемы композиционной таблицы до ее реализации. Особое 
внимание уделено правилам и алгоритмам формирования контекстов.

\vspace*{-6pt}
  
\section{Контекст}

\vspace*{-2pt}

\subsection{Определения} %2.1

\vspace*{-1pt}
  
  Пусть $\Re\hm= \{R_1, R_2, \ldots , R_k\}$~--- реляционная БД, определенная 
на множестве атрибутов $U\hm= \{A_1, A_2, \ldots , A_n\}$, где $R_i$~--- 
отношение, [$R_i$]~--- схема отношения (подмножество атрибутов, на\linebreak которых 
определено отношение~$R_i$). Рассмотрим базовые зависимости, используемые 
при проектировании схемы БД~[10--12]. Пусть DEP~--- 
мно\-жество зависимостей (функциональных, многозначных, включения, 
соединения), определенных на множестве атрибутов~$U$ и множестве 
отношений~$\Re$. Пусть $R$~--- отношение, определенное на множестве 
атрибутов~$U$ (универсальное реляционное отношение). 

Следующие четыре 
определения являются компонентами традиционной теории~БД.
{ %\looseness=1

}
  
  \medskip
  
  \noindent
  \textbf{Определение~2.1 (ФЗ).}\ Пусть $X$ и~$Y$~--- некоторые 
подмножества из множества атрибутов~$U$. Будем говорить, что $X$ 
функционально определяет $Y:\ X\hm\to Y$, если в любой реализации~$R$ не 
могут присутствовать два кортежа $t,u\hm\in R$, такие что $t[X]\hm=u[X]$ и 
$t[Y]\not= u[Y]$.

  \smallskip
  
  Пусть заданы множества атрибутов $X\subseteq U$, $Y\subseteq U$ и $X\cap 
Y\hm=\varnothing$, $Z\hm=R\backslash (X\cup Y)$.
  
  \medskip
  
  \noindent
  \textbf{Определение 2.2 (МЗ).} Множество~$X$ мультиопределяет 
множество~$Y$ в контексте $Z:\ X\hm\to Y(Z)$ (многозначная зависимость), 
когда выполнено условие, что если для произвольной реализации~$R$ 
существу-\linebreak\vspace*{-12pt}

\pagebreak

\noindent
ет два кортежа $t_1,t_2\hm\in R$ таких, что $t_1[X]\hm=t_2[X]$, то 
существует кортеж~$t_3$, для которого
  $$
t_3[X]=t_1[X]\,;\enskip  t_3[Y]=t_1[Y]\,;\enskip  t_3[Z]=t_2[Z]\,,
$$
и в силу симметрии существует кортеж~$t_4$:
$$
t_4[X]=t_1[X]\,;\enskip  t_4[Y]=t_2[Y]\,;\enskip  t_4[Z]=t_1[Z]\,.
$$

  \noindent
  \textbf{Определение 2.3 (ЗС).} Отношение $R(V_1, V_2, \ldots , V_p)$ 
удовлетворяет зависимости соединения на множествах атрибутов $V_1, V_2, 
\ldots , V_p$ тогда и только тогда, когда $R$ удовлетворяет свойству 
соединения без потерь информации (СБПИ):
  $$
R=R[V_1]\bowtie R[V_2]\bowtie \cdots\bowtie R[V_p]\,,
$$
где $\bowtie$~--- операция естественного соединения; $R[V_i]$~--- операция 
проекции отношения~$R$ по атрибутам~$V_i$~\cite{10-z}.
  
  \smallskip
  
  Заметим, что многозначная зависимость является частным случаем 
зависимости соединения, а функциональная зависимость является частным 
случаем многозначной зависимости~\cite{10-z, 11-z}.
  
  Формальным основанием для установления связей на схеме БД являются 
зависимости включения~\cite{12-z}.
  
  \smallskip
  
  \noindent
  \textbf{Определение 2.4 (ЗВ).} Пусть $R_i[A_1, \ldots , A_m]$ и $R_j[B_1, 
\ldots , B_p]$~--- схемы отношений (не обязательно различные), $V\subseteq 
\{A_1, \ldots , A_m\}$ и $W\subseteq \{B_1, \ldots , B_p\}$, $\vert V\vert \hm= \vert 
W \vert$. Тогда между отношениями~$R_i$ и~$R_j$ существует зависимость 
включения по атрибутам~$V$ и~$W$ соответственно, если $R_i[V] \subseteq  
R_j[W]$, где $\vert V\vert$~--- мощность множества~$V$.
  
  Если выполнено условие $V\hm=W$, то такой вид ЗВ называется 
типизированным (typed)~\cite{13-z, 14-z}. Это дополнительное ограничение 
вполне согласуется с общепринятым свойством связей на схеме БД: связи 
отражают количественное соотнесение кортежей в отношениях и не обладают 
ка\-кой-ли\-бо семантикой. Необходимость связывания различных по смыслу 
атрибутов, скорее всего, является признаком потери ка\-кой-ли\-бо ФЗ для 
свя\-зы\-ва\-емых атрибутов.
  
  В традиционной теории БД рассматривается свойство сохраненных 
зависимостей при декомпозиции отношений~\cite{1-z}. Однако добиться 
реализации этого свойства на практике не всегда удается. Введем усиленный 
вариант этого свойства. Пусть $C\hm=\{ R_1^*, R_2^*, \ldots , R_m^*\}$~--- 
произвольное подмножество отношений реляционной БД.
  
  \smallskip
  
  \noindent
  \textbf{Определение 2.5.} Зависимость dep$_j\hm\in \mathrm{DEP}$ будем считать 
реализованной на~$C$, когда операция дополнения, удаления или 
модификации кортежа в произвольном отношении $R_i^*\hm\in C$ будет 
заблокирована ор\-га\-ни\-за\-ци\-он\-но-тех\-ни\-че\-ски\-ми средствами, если 
при этом нарушается зависимость dep$_j$.
  \smallskip
  
  Под организационными средствами подразумевается способ проектирования 
схемы БД с указанием ограничений целостности на данные, под техническими 
средствами~--- возможности системы управления базами данных (СУБД) по 
поддержке этих ограничений целостности.
  
  Совокупность отношений, по которым строится гиперкуб, должна 
удовлетворять свойству СБПИ~\cite{15-z}, поскольку лишние кортежи в 
промежуточном представлении данных дают лишние значения в рабочей 
области гиперкуба. Для реализации этого свойства будем применять понятие 
контекста, которое впервые было использовано в работе~\cite{16-z}.
  
  \smallskip
  
  \noindent
  \textbf{Определение 2.6.} Множество~$C$ будем называть контекстом, если 
оно удовлетворяет свойству СБПИ на зависимостях DEP, реализованных 
в~$C$.
  \smallskip
  
  \noindent
  \textbf{Замечание.} В~основе контекста лежит операция естественного 
соединения, которая собирает из различных отношений БД связанные друг с 
другом по значению данные. Затем эти данные (кортежи) участвуют в 
формировании новых структур, естественным образом дополняя и ограничивая 
друг друга, что делает уместным использование термина <<контекст>> для 
совокупности таких значений. Проверка свойства СБПИ осуществляется по 
алгоритму, представленному в~\cite{10-z}.

\vspace*{-6pt}
  
\subsection{Формирование контекста} %2.2

\vspace*{-2pt}
  
  Первоначальный выбор размерностей и мер гиперкуба предлагается сделать 
в расширенном виде: $R_i,\,A_j$, где $R_i$~--- наименование отношения из 
исходной реляционной БД и $A_j$~--- наименование атрибута в этом отношении. 
Таким образом, будет задано начальное множество отношений $R^0\hm= \{ 
R_1^0, R_2^0, \ldots , R_q^0\}$, участвующее в обязательном порядке в 
формируемом контексте, так как контекст далее используется в качестве 
основы для формирования размерностей и всего многомерного представления 
данных. Дальнейшая задача состоит в дополнении $R^0$ отношениями из 
$\Re$, чтобы результирующее множество удовлетворяло свойству СБПИ на 
реализованных зависимостях, т.\,е.\ являлось контекстом. В~общем случае 
таких вариантов дополнения существует несколько. Каждый из вариантов 
(контекстов) имеет свою смысловую нагрузку, поэтому окончательный выбор 
контекста может выполнить только пользователь. Задача алгоритма 
заключается в последовательной генерации контекстов без зацикливания. Для 
сокращения числа перебираемых вариантов при формировании контекстов, 
ближайших к множеству~$R^0$, предлагается сделать этот перебор 
направленным.
  
  Сформулируем критерии, которые позволят сделать перебор отношений 
направленным.
  \begin{enumerate}[1.]
\item Замыкание первичного ключа нового отношения~$R_i$ совпадает со всем 
множеством атрибутов в выбранных отношениях. По теореме~5.8~\cite{10-z} 
полученное множество отношений обладает свойством СБПИ, что исключает 
необходимость проверки этого свойства по алгоритму. В~этом случае 
достаточно построить замыкание первичного ключа~$R_i$. Такое отношение 
получает приоритет~3.
\item Для отношения $R_i$ выполнено условие существования связи, 
соответствующей ЗВ $R_i[X]\hm\subseteq R_j[X]$ с уже выбранными 
отношениями~$R_j$, где множество атрибутов~$X$ является первичным 
ключом отношения~$R_j$. Отношение~$R_i$ является <<связующим>> и 
позволяет объединить несколько несвязанных отношений в иерархию. Однако 
это условие не гарантирует выполнения свойства СБПИ, и его надо проверять 
отдельно. Отношение~$R_i$ получает приоритет~2.
\item Дополняемое отношение~$R_i$ должно иметь непустое пересечение с уже 
выбранными отношениями. В~работе~\cite{16-z} показано, что это условие 
является необходимым для выполнения свойства СБПИ, если в DEP 
отсутствует реализованная декомпозицией (теорема Фейджина) многозначная 
зависимость с пустой левой \mbox{частью}. Такое отношение получает приоритет~1. 
Остальные отношения получают приоритет~0.
\item Формируемый контекст не должен содержать лишних отношений, 
наличие которых обуслов\-ле\-но только порядком присоединения отношений к 
контексту в алгоритме.
\end{enumerate}

  Перечисленные критерии увеличивают вероятность более быстрого 
достижения результата.
  
  Введем обозначения. Пусть $R^1\hm= \{ R_1^1, R_2^1, \ldots$\linebreak $\ldots , R_p^1\}$~--- 
множество отношений, не входящих в исходное множество~$R^0$: $R^1\hm= 
\Re\backslash R^0$.
  
  Рассмотрим схему алгоритма, удовлетворяющего сформулированным 
критериям (\textit{алгоритм на\-прав\-лен\-но\-го перебора}).
  \begin{enumerate}[1.]
\item Подсчет весов для отношений~$R^1$ и их упорядочение по убыванию 
весов.
\item Формируются сочетания без повторений из отношений~$R^1$, сначала по 
одному, затем по два и~т.\,д. Сочетания начинаются с наименьших 
значений и далее последовательно увеличиваются, например сочетания по два 
элемента: $(1, 2), (1, 3), \ldots , (2, 3)$,\ldots\ Текущее сочетание отношений 
совместно с~$R^0$ проверяется на выполнение свойства СБПИ. Если свойство 
выполнено, то полученное сочетание дополняет множество контекстов.
\item В процессе выполнения алгоритма пользователю предлагается выбрать 
нужный контекст.
  \end{enumerate}
  
  \noindent
  \textbf{Замечание.} Такая схема алгоритма на ближайших итерациях находит 
дополнительные отношения, с наибольшей вероятностью образующие 
наименьший контекст с исходным множеством отношений~$R^0$.
  
  \smallskip
  
  В табл.~1 размерности $Y_1$ соответствует контекст $C_1\hm=\{R_2\}$ 
($C_0\hm=\{R_5, R_6\}$ для размерности~$X$ является псевдоконтекстом), 
контекст приложения $C\hm=\{R_2, R_3, R_4, R_5, R_6, R_8\}$. В~табл.~2 
раз\-мер\-ности~$X$ соответствует контекст $C_0\hm=\{R_1, R_2\}$ 
(отношение~$R_2$ дополнено в контекст за счет логического ограничения). 
Размерности~$Y_1$ соответствует контекст $C_1\hm=\{R_2, R_3, R_9, R_{10}\}$. 
В~этом контексте пользователь определил, что ему требуется только список 
предметов из учебной нагрузки по специальности, а не все предметы из 
отношения~$R_3$. По аналогичным рассуждениям размерности~$Y_2$ 
соответствует контекст $C_2\hm=\{R_2, R_3, R_9, R_{10}, R_{11}\}$. Оба контекста 
удовлетворяют свойству СБПИ с учетом многозначной зависимости: 
<<№\,спе\-ци\-аль\-ности>>\;$\to$\;<<№\,груп\-пы>> (<<№\,предмета>>). 
Поскольку в обеих размерностях присутствует отношение~$R_2$, то для них 
используется логическое ограничение, указанное в заголовке таблицы. 
Контекст приложения в табл.~2: $C\hm= \{R_1, R_2, R_3, R_7, R_9, R_{10}, 
R_{11}\}$.

%\vspace*{-6pt}
  
\section{Реализация контекста}

%\vspace*{-2pt}
  
  В качестве реализации контекста будем использовать представление данных 
в виде таблицы связанных соединений, являющейся частным случаем модели 
<<таблица соединений>>~\cite{17-z}. Совокупность свойств этой таблицы, 
которые будут рассмотрены ниже, позволяет получить необходимое 
пред\-став\-ле\-ние для формирования многомерных пред\-став\-ле\-ний данных.

  \begin{table*}[b]\small
  \begin{center}
  \Caption{Пример реализации контекста}
  \vspace*{2ex}
  
  \begin{tabular}{|c|c|c|c|c|c|c|c|}
  \hline
\textbf{№ студента}&\textbf{№ группы}&\textbf{ФИО студента}&\textbf{Код группы}&
\textbf{№ специальности}&\textbf{№ курса}&{\boldmath{$l_1$}}&{\boldmath{$l_2$}}\\
\hline
1&2&Алексенко С.\,В.&М-220&5&2&1&1\\
2&2&Белоусов П.\,О.&М-220&5&2&1&1\\
3&2&Бессараб О.\,П.&М-220&5&2&1&1\\
4&2&Вяткин М.\,С.&М-220&5&2&1&1\\
5&2&Драница А.\,А.&М-220&5&2&1&1\\
6&2&Ефимов Е.\,С.&М-220&5&2&1&1\\
\hline
\end{tabular}
\end{center}
\end{table*}
  
  Рассмотрим преобразование представления реляционной БД: $\Re\hm= \{R_1, 
R_2, \ldots , R_k\}$ в таблицу связанных соединений $(S,l)$, где $S$~--- схема, 
определенная на множестве атрибутов $U\hm=\{A_1, A_2, \ldots , A_n\}$, $l$~--- 
вектор вхождения кортежей отношений длины~$k$. Определим принцип 
формирования кортежей $t\hm\in s$, где $s$~--- реализация (множество 
кортежей) схемы отношения~$S$. Рассмотрим все возможные сочетания без 
повторений отношений $R_1, R_2, \ldots , R_k$, удовлетворяющие свойству 
СБПИ. Пусть множество отношений $C^\prime\hm= \{R_{m(1)}, R_{m(2)}, \ldots , 
R_{m(p)}\}$~--- контекст, где $m(p)$~--- целочисленный массив из $p$ номеров 
отношений текущего сочетания, и $c^\prime$~--- его реализация, ограниченная 
операцией селекции $\sigma_{F}$ с логической формулой~$F$:
  $$
  c^\prime =\sigma_F (R_{m(1)}\bowtie R_{m(2)}\bowtie \cdots\bowtie  R_{m(p)})\,.
  $$
  
  Для каждого кортежа $u\hm\in c^\prime$ формируем кортеж~$t$ по 
следующим правилам: $t[A_j]\hm=u[A_j]$, если атрибут~$A_j$ принадлежит 
хотя бы одному отношению соединения, и $t[A_j]\hm=\mathrm{emp}$ в противном\linebreak 
случае, где emp~--- пустое значение. Каждому кортежу поставим в 
соответствие битовый вектор $l(t) \hm= (l_1(t), l_2(t), \ldots , l_k(t))$, где 
$l_j(t)\hm=1$, если отношение~$R_j$ участвует в текущем соединении, и 
$l_j(t)\hm=0$ в противном случае.
  
  Рассмотрим отношение частичного порядка над кортежами $t\hm\in s$.
  
  \smallskip
  
  \noindent
  \textbf{Определение 3.1.} Кортеж $t\hm\in s$ является менее определенным 
или равным кортежу $t^\prime\hm\in s$, когда для любого атрибута~$A_i$ 
выполнено условие: если $t[A_i]\not= t^\prime[A_i]$, то $t[A_i]\hm=\mathrm{emp}$ и 
$l_j(t^\prime)\hm\geq l_j(t)$, $j\hm=1, \ldots ,k$, причем $t[A_i]\hm=t^\prime[A_i]$, 
если $A_i$ принимает значение NULL в обоих кортежах. В~этом случае 
назовем кортеж~$t$ подчиненным кортежу~$t^\prime$ и будем писать $t\prec 
t^\prime$.
  
  \smallskip
  
  В представлении~$s$ достаточно хранить только кортеж~$t^\prime$, который 
содержит в себе все менее определенные либо равные кортежи. Следовательно, 
завершающим этапом построения представления~$s$ является удаление в нем 
всех подчиненных кортежей. Равенство неопределенных значений в 
определении~3.1 позволяет избавиться от подчиненных кортежей~$t$, которые 
получены из тех же кортежей БД, что и~$t^\prime$. Отличие значений NULL 
и emp в том, что первое указывает на неопределенное значение атрибута, а 
второе~--- на отсутствие со\-от\-вет\-ст\-ву\-юще\-го кортежа в текущем соединении. 
Очевидно, что отношение~$\prec$ является транзитивным.
  
  Реализации контекстов для рассмотренного примера~1 являются 
громоздкими, поэтому рассмотрим самую простую из них: $C_0\hm=\{R_1, 
R_2\}$ для табл.~2 (см.\ табл.~3).
  

  
  Пусть $X(J)=([R_{j(1)}]\cup [R_{j(2)}]\cup\cdots \cup [R_{j(m)}])$, где $J\hm=(j(1), 
j(2), \ldots , j(m))$, и $[R_{j(i)}]$~--- множество атрибутов отношения~$R_{j(i)}$. 
Определим операцию проекции на множестве~$s$.
  
  \smallskip
  
  \noindent
  \textbf{Определение 3.2.} Проекция $\pi_{X(J})(s)$ есть совокупность 
кортежей $u[X(J)]$, определенных на множестве атрибутов $X(J)$, где для 
каждого $u[X(J)]$ существует кортеж $t\hm\in s$ такой, что $u[X(J)]\hm=t[X(J)]$ 
и $l_{j(i)}(t)\hm=1$, $i\hm=1,2, \ldots ,m$.
  
  \smallskip
  
  Логическое ограничение $F(t)$ будем пред\-став\-лять в виде дизъюнктивной 
нормальной формы, что удобно реализовать в пользовательском интерфейсе в 
виде шаблонов
  $$
F=F_1\vee F_2\vee\cdots\vee  F_m\,,
$$
где каждая элементарная формула является конъюнкцией предикатов 
сравнения языка SQL:
$$
F_i=F_{i,1}\wedge F_{i,2}\wedge\cdots\wedge  F_{i,p}\,,
$$
где $F_{i,j}=[\mathrm{NOT}]A_q\Theta\langle \mbox{выражение}\rangle$, 
$\langle\mbox{выражение}\rangle$~--- константа либо атрибут~$A_s$, 
$\Theta$~--- операция сравнения, [NOT]~--- необязательный параметр. Если 
ка\-кой-ли\-бо предикат $F_{i,j}$ не определен на кортеже~$t$ (атрибуты~$A_q$ 
и/или $A_s$ имеют значение emp или NULL), то $F_{i,j}$ заменяется 
значением TRUE независимо от операции~$\Theta$. Такая подстановка 
позволяет оставить в $s$ кортежи, для которых пока не определены некоторые 
атрибуты или отсутствуют связанные по значениям кортежи в других 
отношениях, что также является предметом анализа информации с 
использованием композиционной таблицы. Формула~$F$ после подстановки 
будет принимать только два значения: TRUE и FALSE.
  
  Реализация этого свойства требует определенной последовательности 
формирования~$s$. Если его формировать сверху вниз (начиная с соединения 
всех отношений контекста, затем подмножеств отношений и~т.\,д.\ до 
одиночных отношений) и сразу отфильтровывать кортежи, используя 
формулу~$F$, то в $s$ останутся кортежи, подчиненные удаленным кортежам 
на предыдущих итерациях. Корректное формирование таблицы $s$ достигается 
в следующем алгоритме.
  
  \smallskip
  
\noindent
\textbf{А1}: Вход: $C\hm=\{R_1, R_2, \ldots .., R_k\}$~--- контекст, DEP~--- 
реализованные зависимости, $\Re$~--- БД.

  Выход: $s$~--- таблица связанных соединений.
  
\noindent
$s=\varnothing$

\noindent
for $i=k$ to~1 step $-1$

\hspace*{1mm}$m=(0,0,\ldots ,0)$

\hspace*{1mm}do while $\mathrm{Comb}\left(i,k,m\right)$

\hspace{2mm}if $\mathrm{Llj}\left(C,m,\mathrm{DEP}\right)$ then

\hspace*{3mm}$r = R_{m(1)}\bowtie R_{m(2)}\bowtie\cdots\bowtie R_{m(i)}$

\hspace*{3mm}$s^{\prime\prime}=\mathrm{Transfor}(r)$

\hspace*{3mm}$s=s \cup  s^{\prime\prime}$

\hspace*{2mm}endif

\hspace*{1mm}enddo

\noindent
endfor

\noindent
for each $t_i$ in $s$

\hspace*{1mm}for each $t_j$ in $s$

\hspace*{2mm}if $ti\not=  t_j$ then

\hspace*{3mm}if $t_i \prec t_j$ then $s=s - t_i$

\hspace*{3mm}if $t_j \prec t_i$ then $s=s - t_j$

\hspace*{2mm}endif

\hspace*{1mm}endfor

\noindent
endfor

\noindent
$s=\sigma_F(s)$



\smallskip

\noindent
где $\mathrm{Comb}\left(i,k,m\right)$~--- сочетания без повторений из $k$ элементов по~$i$, 
результат помещается в массиве~$m$, функция принимает значение FALSE, 
если текущий набор сочетаний исчерпан, в противном случае~--- TRUE; 
$\mathrm{Llj}\left(C,m,\mathrm{DEP}\right)$~--- проверка свойства СБПИ для отношений с ненулевыми 
номерами в массиве~$m$, $\mathrm{Transfor}(r)$~--- преобразование соединения в 
таблицу связанных соединений.
  
  Размер $s$ перед фильтрацией может оказаться огромным, поэтому 
предлагается последовательность формирования~$s$ снизу вверх. Сначала 
просматриваются одиночные отношения контекста, из их кортежей 
формируются кортежи~$s$ с фильтрацией по формуле~$F$. Затем 
просматриваются соединения пар отношений контекста, удовлетворяющие 
свойству СБПИ. Сформированные новые кортежи в~$s$ сначала используются 
для удаления подчиненных кортежей, а затем подвергаются фильтрации. Далее 
по аналогичным правилам просматриваются соединения трех отношений 
контекста и~т.\,д.\ до соединения всех отношений контекста.
  
  Формализуем эти рассуждения в виде алго\-ритма.
  
  \smallskip
  
\noindent
\textbf{А2}: Вход: $C\hm=\{R_1, R_2, \ldots .., R_k\}$~--- контекст, DEP~--- 
реализованные зависимости, $\Re$~--- БД.

\noindent
Выход: $s^\prime$~--- таблица связанных соединений.


\noindent
$s^\prime=\varnothing$

\noindent
for $i=1$ to $k$

$m=(0,0,\ldots ,0)$

do while $\mathrm{Comb}\left(i,k,m\right)$

\hspace*{1mm}if $\mathrm{Llj}\left(C,m,\mathrm{DEP}\right)$ then

\hspace*{2mm}$r = R_{m(1)}\bowtie R_{m(2)}\bowtie\cdots\bowtie R_{m(i)}$

\hspace*{2mm}$s^{\prime\prime}=\mathrm{Transfor}(r)$

\hspace*{2mm}for each $t_i$ in $s^\prime$

\hspace*{3mm}for each $t_j$ in $s^{\prime\prime}$

\hspace*{4mm}if $t_i \prec t_j$ then $s^\prime=s^\prime - t_i$

\hspace*{3mm}endfor

\hspace*{2mm}endfor

\hspace*{2mm}$s^\prime=s^\prime \cup  \sigma_F(s^{\prime\prime})$

\hspace*{1mm}endif

enddo

\noindent
endfor
  
  \smallskip
  
  Алгоритмы вычисления~$s$ и~$s^\prime$ совпадают по количеству вновь 
генерируемых кортежей, т.\,е.\ до мес\-та, где выполнена операция $\mathrm{Transfor}(r)$. 
Однако размер~$s^\prime$ в процессе формирования меньше~$s$ и, как 
следствие, количество сравнений кортежей также меньше. Это достигается за 
счет фильтрации ненужных кортежей в процессе формирования~$s^\prime$ и за 
счет выбранной последовательности формирования промежуточных 
соединений. Можно получить оценки для количества кортежей и операций в 
обоих алгоритмах в предположении о распределениях атрибутов, как это 
сделано в работе~\cite{18-z} для таб\-ли\-цы соединений. Однако формулы для 
оценок будут громоздкими, хотя очевидно, что алгоритм формирования 
для~$s^\prime$ является более экономичным по памяти и по числу операций. 
Осталось показать следующее.
  
  \medskip
  
  \noindent
  \textbf{Теорема 3.1.} $s=s^\prime$.
  
  \noindent
  Д\,о\,к\,а\,з\,а\,т\,е\,л\,ь\,с\,т\,в\,о\,.\ \
  {Пусть $t\in s$ и $t\not= s^\prime$. Следовательно, существует кортеж 
$t^\prime\in s^\prime$: $t\prec t^\prime$, а по предположению $t\not=t^\prime$. 
Поскольку~$s$ и~$s^\prime$ формируются из совпадающих наборов 
соединений, то существует кортеж $t^{\prime\prime}\in s$: $t^\prime\prec 
t^{\prime\prime}$. Учитывая транзитивность операции~$\prec$, имеем $t\prec 
t^{\prime\prime}$. Следовательно, кортеж~$t$ должен быть удален из~$s$: 
$s^\prime\subseteq s$. В~алгоритме~$\mathrm{A2}$ не проверяется подчиненность кортежей 
внутри одного соединения~$s^{\prime\prime}$. В~этом случае подчиненными 
могут быть только совпадающие кортежи. С~одной стороны, это не нарушает 
соотношения $s^\prime\subseteq s$, с другой~--- совпадающие кортежи могут 
появиться в~$s^{\prime\prime}$, только если в каком-либо отношении $R_{m(j)}$ 
есть совпа\-да\-ющие кортежи, что недопустимо для реляционной~БД.}
  
  \medskip
  
  Пусть $t\in s^\prime$ и $t\not\in s$. Следовательно, существует кортеж 
$t^\prime\hm\in s$: $t\prec t^\prime$, а $t\not= t^\prime$. По построению кортеж 
$t^\prime$ на некоторой итерации должен появиться в~$s^{\prime\prime}$ 
алгоритма~А2. Кортежи~$t$ и~$t^\prime$ принадлежат следующим 
соединениям: $t\hm\in  R_{m(1)}\hm\bowtie  R_{m(2)}\bowtie \cdots$\linebreak $\cdots \bowtie  
R_{m(j)}$, $t^\prime\,\in\, R_{q(1)}\hm\bowtie  R_{q(2)}\bowtie\cdots\bowtie  R_{q(i)}$.\linebreak
 По определению 
операции~$\prec$ для множеств индексов выполнено: $\{m(1), m(2), \ldots , 
m(j)\}\hm\subseteq \{m(1), m(2), \ldots , m(j)\}$. Это означает, что кортеж~$t^\prime$ 
появится в~$s^{\prime\prime}$ (алгоритм~А2) позднее, чем~$t$ в~$s^\prime$, и 
$t$ будет удален из~$s^\prime$. Следовательно, $s\subseteq s^\prime$. Теорема 
доказана.
  
  \smallskip
  
  Основываясь на способе формирования таблицы~$s$, сформулируем ее 
важные свойства.
  
  \medskip
  
  \noindent
  \textbf{Теорема 3.2.} \textit{Для любого множества отношений $R^*\hm=\{ 
R_1^*, R_2^*, \ldots , R_q^*\}\subseteq C^\prime$, удовлетворяющего свойству 
СБПИ, где $C^\prime\hm= \{R_{m(1)}, R_{m(2)}, \ldots , R_{m(p)}\}$~--- контекст, 
выполнено:}
  $$
  \pi_{R^*}(s)=\sigma_F(R_1^*\bowtie R_2^*\bowtie \cdots \bowtie R_q^*)\,.
  $$
  
  \smallskip
  
  \noindent
  Д\,о\,к\,а\,з\,а\,т\,е\,л\,ь\,с\,т\,в\,о\,.\ \ Пусть кортеж $t\hm\in 
\sigma_F(R_1^*\hm\bowtie R_2^*\bowtie \cdots\bowtie R_q^*)$. Покажем, что этот 
кортеж принадлежит $\pi_{R*}(s)$. По определению операции селекции $F(t)\hm 
=\mathrm{TRUE}$. Поскольку $R^*\subseteq C^\prime$, то сочетание, состоящее из всех 
отношений~$R^*$, также будет участвовать в формировании~$s$ с теми же 
ограничениями, задаваемыми формулой~$F$. Кортеж~$t$ может быть удален 
из~$s$, если существует кортеж $t^\prime\hm\in s$: $t\prec t^\prime$. В~этом 
случае возможны два варианта:
  \begin{enumerate}[(1)]
  \item $F(t^\prime)=\mathrm{FALSE}$;
  \item $F(t^\prime)=\mathrm{TRUE}$.
  \end{enumerate}
  
  В первом варианте кортеж~$t^\prime$ будет удален из~$s$ до реализации 
$\sigma_F(R_1^*\bowtie R_2^*\bowtie\cdots\bowtie R_q^*)$ при 
формировании~$s$ и не сможет воспрепятствовать появлению кортежа~$t$ 
в~$s$. Во втором варианте кортеж~$t^\prime$ останется в~$s$, а кортеж~$t$ 
будет удален. Однако по правилу получения проекции $\pi_{R*}(t^\prime)\hm=t$, 
что доказывает соотношение $\pi_{R*}(s)\supseteq \sigma_F(R_1^*\bowtie 
R_2^*\bowtie\cdots\bowtie R_q^*)$.
  
  Докажем включение в обратную сторону. Пусть $t\hm\in \pi_{R*}(s)$. По 
правилу построения проекции существует кортеж $u\hm\in s$ такой, что $u$ 
определен для всех отношений множества~$R^*$ и по построению 
представления~$s$ выполнено $F(u)\hm=\mathrm{TRUE}$. Поскольку $u$ определен для 
всех отношений~$R^*$, то каждое отношение $R_i^*\hm\in R^*$ содержит 
кортеж $u_i\hm=u[R_i^*]\hm=t[R_i^*]$. По правилу выполнения операции 
естественного соединения из совокупности кортежей~$u_i$ будет сформирован 
кортеж~$t$, который принадлежит $R_1^*\bowtie R_2^*\bowtie \cdots\bowtie 
R_q^*$. Поскольку $F(u)\hm=\mathrm{TRUE}$, то по правилу вычисления $F(t)$ 
получим $F(t)\hm=\mathrm{TRUE}$. Следовательно, кортеж~$t$ принадлежит 
$\sigma_F(R_1^*\bowtie R_2^*\bowtie\cdots\bowtie R_q^*)$. Теорема доказана.
  
  \medskip
  
  \noindent
  \textbf{Теорема 3.3.} \textit{Представление~$s$ всегда существует и 
единственно для любой схемы реляционной БД.}
  
  \smallskip
  
  \noindent
  Д\,о\,к\,а\,з\,а\,т\,е\,л\,ь\,с\,т\,в\,о\,.\ \ Существование~$s$ следует из его 
построения. Для доказательства един\-ст\-вен\-ности предположим, что существует 
$s^\prime\not= s$. Заметим, что таблицы~$s$ и~$s^\prime$ сформированы для 
одного состояния БД, по одинаковым сочетаниям отношений и с одинаковым 
ограничением~$F$. В~этом случае возможны два варианта:
  \begin{enumerate}[(1)]
  \item существует кортеж $t\hm\in s$ и $t\notin s^\prime$;
  \item существует кортеж $t^\prime \hm\in s^\prime$ и $t^\prime \notin s$.
  \end{enumerate}
  
  Рассмотрим первый вариант. Для того чтобы кортеж~$t$ отсутствовал 
в~$s^\prime$, необходимо, чтобы существовал кортеж $t_1\hm\in s$ такой, что 
$t\prec t_1$. Поскольку $t\not= t_1$, то $t_1$ сформирован по большему 
количеству отношений, чем~$t$. Если кортеж~$t_1$ есть в~$s$, то в~$s$ не 
должно быть кортежа~$t$. Пусть $t_1$ отсутствует в~$s$, тогда должен 
существовать кортеж $t_2\hm\in s$, для которого выполнено $t_1\prec t_2$. С~учетом 
свойства транзитивности отношения~$\prec$ получим, что $t\prec t_2$ и 
кортеж~$t$ должен быть удален из~$s$. 

Аналогичные рассуждения могут быть 
приведены и для второго варианта. Полученное противоречие доказывает 
единственность представления~$s$. Теорема доказана.
  
  \medskip
  
  Рассмотрим дополнительное обоснование выбора контекстов для реализации 
размерностей в\linebreak виде таблицы связанных соединений. Кроме отоб\-ра\-же\-ния 
частично заполненных данных в размерности таблицы можно ограничить 
присутствие только тех данных, которые связаны с другими объектами БД. 
Например, при формировании сводной ведомости по специальности в учебном 
заведении к размерности <<Предметы>> можно дополнить отношение 
<<Учебный план>>. Тогда в размерности будут отображены только те 
предметы, которые есть в учебном плане, а не весь список предметов учебного 
заведения, причем, если по ка\-ко\-му-ли\-бо предмету из учебного плана 
оценки пока отсутствуют, он все равно будет отображен в размерности 
таблицы. Такие свойства размерностей не предусмотрены во всех известных 
моделях многомерных данных.
  
  \smallskip
  
  \noindent
  \textbf{Определение 3.3.} Псевдоконтекстом $C_i$ назы-\linebreak вается множество 
отношений, для которых не\linebreak га\-рантируется выполнение свойства СБПИ. 
Реа\-лизаци\-ей размерности, соответствующей псевдоконтексту, является 
соединение отношений~$C_i$. Ограничения на такую размерность задаются 
только на атрибутах уже выбранных отношений.
  
  \smallskip
  
  Псевдоконтексты соответствуют размерностям, в которых требуются все 
допустимые комбинации значений атрибутов. Например, значения атрибутов 
<<День недели>> и <<Время начала занятия>> в табл.~1 должны быть 
сопоставлены каждый каж\-дому.
  
  \smallskip
  
  \noindent
  \textbf{Определение 3.4.} Контекст~$C_i$ называется пустым, если его 
реализация формируется в виде проекции от существующего контекста.
  
  Такие контексты соответствуют размерностям, в которых присутствуют 
только те значения атрибутов, которым сопоставлено непустое множество 
значений меры. Заметим, что аналогичный результат дают все известные 
продукты технологии ROLAP.
  
  \smallskip
  После того как сформированы контексты и псевдоконтексты размерностей и 
общий контекст приложения, задача построения композиционной таблицы 
может быть решена с использованием алгоритма, рассмотренного в 
работе~\cite{19-z}. Основная идея алгоритма заключается в раздельном 
формировании размерностей с установлением иерархий и общей таблицы 
приложения. Затем значениям размерностей сопоставляются значения меры, и 
те и другие первоначально должны находиться в одном кортеже общей 
таблицы приложения.
  
\section{Сравнительный анализ алгоритмов формирования 
контекстов}
  
  В работе~\cite{20-z} рассмотрен метод формирования контекстов по 
множествам атрибутов $X, Y_1, Y_2, \ldots , Y_N, Z_1, Z_2, \ldots , Z_N$, задающих 
размерности и меры композиционной таблицы, причем\linebreak множества атрибутов 
задаются без указания от-\linebreak ношений, которым они принадлежат (без рас\-ширения). 
Автором этого метода П.\,Г.~Редреевым\linebreak разработаны два алгоритма. Кроме 
формирования собственно контекстов в алгоритмах для каж\-до\-го контекста 
требуется еще определить начальное множество отношений исходной БД, в 
которых присутствуют атрибуты размерностей и мер.
  
Первый алгоритм автором назван \textit{алгоритмом полного перебора}, 
второй~--- \textit{эвристическим алгоритмом}. Схема функционирования 
первого алгоритма понятна из его названия, во втором алгоритме к исходному 
множеству отношений дополняются только те отношения, которые имеют 
связи с исходным множеством. Сравнительный анализ этих двух алгоритмов 
выполнен Редреевым для схемы БД примера~1, ограниченной 
отношениями $R_1$--$R_8$.

  Для каждого алгоритма подсчитано число итераций, выполненных для 
нахождения минимального контекста при заданном множестве атрибутов.
  
  \medskip
  
  \noindent
  \textbf{Пример 2.} Пусть для пользовательского приложения <<Предметы>> 
заданы следующие множества атрибутов композиционной таблицы:
  
$X$\; =\;\{№ дня недели, День недели\};

$Y$\;=\;\{Код группы\};

$Z$\;=\;\{Предмет\}.
  
  Минимальным контекстом для данного набора атрибутов является 
следующее множество отношений: $\{R_2, R_3, R_5, R_8\}$. В~алгоритме 
полного перебора по заданному набору атрибутов будет сформировано 
2~начальных множества. В~результате работы алгоритма полного перебора 
минимальный контекст найден за \textbf{2~итерации}.
  
  Для тех же начальных данных в работе эвристического алгоритма 
минимальный контекст будет найден за \textbf{1~итерацию}.
  
  \smallskip
  
  \noindent
  \textbf{Пример 3.} Пусть для пользовательского приложения <<Предметы>> 
заданы следующие множества атрибутов композиционной таблицы:
  
$X$\;=\;\{№ дня недели\};

$Y$\;=\;\{№ группы\};

$Z$\;=\;\{№ предмета\}.

  Минимальным контекстом для данного набора атрибутов является 
аналогичное множество отношений $\{R_2, R_3, R_5, R_8\}$.
  
  В результате работы алгоритма полного перебора будет сформировано 
24~начальных множества. Минимальный контекст при этом будет найден за 
\textbf{71~итерацию}.
  
  Для тех же начальных данных в работе эвристического алгоритма 
минимальный контекст будет найден за \textbf{2~итерации}.
  
  \smallskip
  
  Рассмотрим работу алгоритма направленного перебора для рассмотренных 
примеров. Начальное множество отношений будет сформировано 
пользователем: $\{R_2, R_3, R_5\}$. Далее для обоих примеров будут присвоены 
веса отношениям, которые не вошли в начальное множество: $R_8$~--- 3, $R_1$~--- 
2, $R_7$~--- 2, $R_4$~--- 0, $R_6$~--- 0. Необходимый минимальный контекст будет 
найден на \textit{первой итерации} для обоих примеров без проверки свойства 
СБПИ, так как замыкание первичного ключа отношения~$R_8$ содержит все 
атрибуты отношений контекста.
  
\section{Заключение}

  Рассмотренный в данной работе подход к формированию представлений 
данных является развитием технологии OLAP. Он ориентирован на работу 
аналитика, где не требуется быстрая (за доли секунд) реакция системы на 
запросы, поскольку в большинстве случаев аналитик должен вдумчиво 
выполнять различные виды анализа над различными представлениями данных с 
участием ИТ-спе\-ци\-а\-лис\-та. Основным методологическим принципом в данной 
работе является то, что операционная база данных должна удовлетворять 
принципам независимости, неизбыточности, непротиворечивости и~т.\,д. Эта 
БД является ядром приложений для множества пользователей, а не только для 
отдельно взятого аналитика.
  
  Сравнительный анализ алгоритмов формирования контекстов показал, что 
направленный перебор имеет преимущество перед алгоритмом полного 
перебора, хотя и содержит незначительные дополнительные расходы, 
связанные с определением весов отношений и упорядочением этих отношений. 
Это преимущество обусловлено тем, что выбор отношений вместе с атрибутами 
на начальном этапе существенно уточняет семантику приложения и сокращает 
число допустимых вариантов. Эвристический алгоритм может не найти 
существующий контекст, так как отсутствие непосредственных связей между 
отношениями не гарантирует отсутствие свойства СБПИ. Направленный 
перебор рассмотрит все варианты сочетания отношений и последовательно 
вычислит все существующие контексты аналогично алгоритму полного 
перебора. Следовательно, направленный перебор предпочтителен при 
динамическом формировании контекстов.
  
{\small\frenchspacing
{%\baselineskip=10.8pt
\addcontentsline{toc}{section}{References}
\begin{thebibliography}{99}

  \bibitem{1-z}
  \Au{Vassiliadis P., Sellis T.} A~survey of logical models for OLAP databases~// 
SIGMOD Record, 1999. Vol.~28. No.\,4. P.~64--69.
  \bibitem{2-z}
  \Au{Pedersen T.\,B., Jensen C.\,S., Dyreson~C.\,E.} A~foundation for capturing 
and querying complex multidimensional data~// Inform. Syst., 2001. Vol.~26. No.\,5. 
P.~383--423.
  \bibitem{3-z}
  \Au{Lechtenborger J., Vossen~G.} Multidimensional normal forms for data 
warehouse design~// Inform. Syst., 2003. Vol.~28. No.\,5. P.~415--434.
  \bibitem{4-z}
  \Au{Li H.-G., Yu~H., Agrawal~D., Abbadi~A.\,E.} 
  Progressive ranking of range aggregates~// Data  
Knowl. Eng., 2007. Vol.~63. No.\,1. P.~4--25.
  \bibitem{5-z}
  \Au{Lehner W., Albrecht J., Wedekind~H.} Normal forms for multidimensional 
databases~// 10th Conference (International) on Scientific and Statistical Database 
Management Proceedings.~--- Capri: IEEE Computer Society, 1998. P.~63--72.
  \bibitem{6-z}
  \Au{Giorgini P., Rizzi S., Garzetti~M.} Goal-oriented requirement analysis for 
data warehouse design~// DOLAP'05: 8th ACM Workshop (International) on Data 
Warehousing and OLAP Proceedings.~--- Bremen: ACM, 2005. P.~47--56.
  \bibitem{7-z}
  \Au{Mazon J.-N., Trujillo J., Lechtenborger~J.} Reconciling requirement-driven 
data warehouses with data sources via multidimensional normal forms~// Data  
Knowl. Eng., 2007. Vol.~63. No.\,3. P.~725--751.
  \bibitem{8-z}
  \Au{Кукин А.\,В., Зыкин С.\,В.} Построение математической модели учебного 
процесса для долгосрочного планирования~// Математические структуры и 
моделирование, 2002. Вып.~10. С.~77--86.
  \bibitem{9-z}
  \Au{Редреев П.\,Г.} Построение иерархий в многомерных моделях данных~// 
Изв. Саратовского ун-та. Сер. Математика. Механика. 
Информатика, 2009. Т.~9. Вып.~4. Ч.~1. С.~84--87.
  \bibitem{10-z}
  \Au{Ульман Дж.} Основы систем баз данных~/ Пер с англ.~--- М.: Финансы и статистика, 
1983. 334~с. (\Aue{Ullman~J.} Principles of database systems.~---
 Computer Science Press, 1980. 379~p.)
  \bibitem{11-z}
  \Au{Мейер Д.} Теория реляционных баз данных~/ Пер с англ.~--- М.: Мир, 1987. 608~с.
  (\Aue{Maier~D.} The theory of relational databases.~--- Computer Science Press,
  1983. 637~p.)
  \bibitem{12-z}
  \Au{Casanova M., Fagin R., Papadimitriou~C.} Inclusion dependencies and their 
interaction with functional dependencies~// J.~Comput. Syst. Sci., 1984. Vol.~28. 
No.\,1. P.~29--59.
  \bibitem{13-z}
  \Au{Missaoui R., Godin R.} The implication problem for inclusion dependencies: 
A~graph approach~// SIGMOD Record, 1990. Vol.~19. No.\,1. P.~36--40.
  \bibitem{14-z}
  \Au{Levene M., Vincent M.\,W.} Justification for inclusion dependency normal 
form~// IEEE Trans. Knowl. Data Eng., 2000. Vol.~12. 
No.\,2. P.~281--291.
  \bibitem{15-z}
  \Au{Miller L., Nilakanta S.} Data warehouse modeler: A~CASE tool for 
warehouse design~// 31st Annual Hawaii  Conference (International) on System 
Sciences.~--- Kohala Coast: IEEE Computer Society, 1998. Vol.~6. P.~42--48.
  \bibitem{16-z}
  \Au{Зыкин С.\,В., Полуянов А.\,Н.} Формирование пред\-став\-ле\-ний данных с 
контекстными ограничениями~// Омский научный вестник. Сер. Приборы, 
машины и технологии, 2008. №\,1(64). С.~141--144.
  \bibitem{17-z}
  \Au{Зыкин С.\,В.} Построение отображения реляционной базы данных в 
списковую модель данных~// Управляющие системы и машины, 2001. №\,3. 
С.~42--63.
  \bibitem{18-z}
  \Au{Зыкин С.\,В.} Оценка мощности списка для отображения типа 
<<частичное соединение>>~// Кибернетика и системный анализ, 1993. №\,6. 
С.~142--152.
  \bibitem{19-z}
  \Au{Зыкин С.\,В.} Формирование гиперкубического представления данных со 
списочными компонентами~// Информационные технологии и вычислительные 
сис\-те\-мы, 2010. №\,4. С.~38--46.
  \bibitem{20-z}
  \Au{Редреев П.\,Г.} Построение табличных приложений со списочными 
компонентами~// Информационные технологии, 2009. №\,5. С.~7--12.
\end{thebibliography}
} }

\end{multicols}

\hfill{\small\textit{Поступила в редакцию 09.01.13}}


%\vspace*{12pt}

%\hrule

%\vspace*{2pt}

%\hrule

\newpage


\def\tit{DYNAMIC CONTEXTS OF~RELATIONAL-TYPE DATABASE}

\def\titkol{Dynamic contexts of~relational-type database}

\def\aut{S.\,V.~Zykin}
\def\autkol{S.\,V.~Zykin}


\titel{\tit}{\aut}{\autkol}{\titkol}

\vspace*{-9pt}

\noindent
Sobolev Institute of Mathematics, Siberian
Branch of the Russian Academy of Sciences,  4 Acad.\ Koptyug Av.,\\
Novosibirsk 630090, Russian Federation
 
\def\leftfootline{\small{\textbf{\thepage}
\hfill INFORMATIKA I EE PRIMENENIYA~--- INFORMATICS AND APPLICATIONS\ \ \ 2014\ \ \ volume~8\ \ \ issue\ 1}
}%
 \def\rightfootline{\small{INFORMATIKA I EE PRIMENENIYA~--- INFORMATICS AND APPLICATIONS\ \ \ 2014\ \ \ volume~8\ \ \ issue\ 1
\hfill \textbf{\thepage}}}   

\vspace*{6pt}
  
\Abste{The technology of dynamic formation of data presentation is sugessted. 
This technology is the development of online analytical processing. Data source is a 
relational database with any scheme (not necessarily hierarchical). Target data presentation 
is a composite table which allows to present multivariate data on a plane. This table 
assumes separate formation of dimensions with the following juxtaposition of measures to 
dimensions in the table. The foundation of data presentation is a table of connected 
joins, which satisfies contextual and logic restrictions. The  
algorithms used to form such tables are suggested and their properties are investigated. Special 
attention is given to contexts which are used to form tables of connected 
joins. The algorithm of directed search for creation of contexts is proposed and 
comparative analysis of algorithms of contexts formation is performed on an example. 
The investigated properties of contexts and the offered algorithms are intended to
automate user work to form new data presentations.}


\KWE{relational database; context; lossless join; composite table} 

\DOI{10.14357/19922264140108}

\Ack
\noindent
The financial support of the Russian Foundation for Basic Research 
(grant No.\,12-07-00066-а) is acknowledged.

  \begin{multicols}{2}

\renewcommand{\bibname}{\protect\rmfamily References}
%\renewcommand{\bibname}{\large\protect\rm References}

{\small\frenchspacing
{%\baselineskip=10.8pt
\addcontentsline{toc}{section}{References}
\begin{thebibliography}{99}
\bibitem{1-z-1}
\Aue{Vassiliadis, P., and T. Sellis}. 1999. 
A~survey of logical models for OLAP databases. 
\textit{SIGMOD Rec.} 28(4):64--69. doi: 10.1145/344816.344869.
\bibitem{2-z-1}
\Aue{Pedersen, T.\,B., C.\,S.~Jensen, and C.\,E.~Dyreson}. 2001. 
A~foundation for capturing and querying complex multidimensional data. 
\textit{Inform. Syst.} 26(5):383--423. doi: 10.1016/S0306-4379(01)00023-0.
\bibitem{3-z-1}
\Aue{Lechtenborger, J., and G.~Vossen}. 2003. 
Multidimensional normal forms for data warehouse design. 
\textit{Inform. Syst.} 28(5):415--434. doi: 10.1016/S0306-4379(02)00024-8.
\bibitem{4-z-1}
\Aue{Li, H.-G., H.~Yu, D.~Agrawal, and A.\,E.~Abbadi}. 2007. 
Progressive ranking of range aggregates. \textit{Data Knowl. Eng.} 63(1):4--25. 
doi: 10.1016/j.datak.2006.10.008.
\bibitem{5-z-1}
\Aue{Lehner, W., J. Albrecht, and H.~Wedekind}. 1998. 
Normal forms for multidimensional databases. 
\textit{10th  Conference (International) on Scientific and Statistical Database Management
Proceedings}. Capri: IEEE Computer Society. 63--72.
\bibitem{6-z-1}
\Aue{Giorgini, P., S. Rizzi, and M.~Garzetti}. 2005. 
Goal-oriented requirement analysis for data warehouse design. 
\textit{8th ACM  Workshop (International) on Data Warehousing and OLAP Proceedings}. Bremen: 
ACM. 47--56.
\bibitem{7-z-1}
\Aue{Mazon, J.-N., J. Trujillo, and J.~Lechtenborger}. 2007. 
Reconciling requirement-driven data warehouses with data sources 
via multidimensional normal forms. \textit{Data Knowl. Eng.} 63(3):725--751. 
doi: 10.1016/j.datak.2007.04.004.
\bibitem{8-z-1}
\Aue{Kukin, A.\,V., and S.\,V.~Zykin}. 2002. 
Postroenie mate\-ma\-ti\-che\-skoy modeli uchebnogo protsessa dlya dol\-go\-sroch\-no\-go planirovaniya 
[Creation of the mathematical model of organization of the study process for long-term 
planning]. \textit{Matematicheskie struktury i modelirovanie} 
[\textit{Mathematical Structures and Modeling}] 10:77--86.
\bibitem{9-z-1}
\Aue{Redreev, P.\,G.} 2009. Postroenie ierarkhiy v mnogomernykh modelyakh dannykh 
[Construction of hierarchies in multidimensional data models]. 
\textit{Izvestija Saratovskogo Universiteta. Seriya Matematika. Mekhanika. Informatika} 
[\textit{News of Saratov University. Mathematics. Mechanics. 
Computer Science ser.}] 9(4-1):84--87.
\bibitem{10-z-1}
\Aue{Ullman, J.} 1980. \textit{Principles of database systems}. Computer Science Press. 379~p.
\bibitem{11-z-1}
\Aue{Maier, D.} 1983. \textit{The theory of relational databases}. Computer Science Press. 637~p.
\bibitem{12-z-1}
\Aue{Casanova, M., R. Fagin, and C.~Papadimitriou}. 1984. 
Inclusion dependencies and their interaction with functional dependencies. 
\textit{J.~Comput. Syst. Sci.} 28(1):29--59. doi: 10.1145/588111.588141.
\bibitem{13-z-1}
\Aue{Missaoui, R., and R.~Godin}. 1990. 
The implication problem for inclusion dependencies: A~graph approach. 
\mbox{\textit{SIGMOD Record}} 19(1):36--40. doi: 10.1145/382274. 382402.
\bibitem{14-z-1}
\Aue{Levene, M., and M.\,W.~Vincent}. 2000. 
Justification for inclusion dependency normal form. 
\textit{IEEE Trans. Knowl. Data Eng.} 12(2):281--291. 
doi: 10.1109/69.842267.
\bibitem{15-z-1}
\Aue{Miller, L., and S.~Nilakanta}. 1998. Data warehouse modeler: 
A~CASE tool for warehouse design. 
\textit{31st Annual Hawaii  Conference (International) on System Sciences}. Kohala Coast:
IEEE Computer Society. 42--48.
\bibitem{16-z-1}
\Aue{Zykin, S.\,V., and A.\,N.~Poluyanov}. 2008. 
Formirovanie predstavleniy dannykh s kontekstnymi ogranicheniyami 
[Formation of the data representations with contextual restrictions]. 
\textit{Omskiy Nauchnyy Vestnik} [\textit{Omsk Scientific Bulletin}] 1(64):141--144.
\bibitem{17-z-1}
\Aue{Zykin, S.\,V.} 2001. Postroenie otobrazheniya relyatsionnoy bazy dannykh 
v spiskovuyu model' dannykh [Construction of mapping of a relational database 
at the list model of data]. \textit{Upravlyayushchie Sistemy i Mashiny} 
[\textit{Control Systems and Machines}] 3:42--63.
\bibitem{18-z-1}
\Aue{Zykin, S.\,V.} 1993. Otsenka moshchnosti spiska dlya otob\-ra\-zhe\-niya tipa 
``chastichnoe soedinenie'' 
[The list size estimation for mapping of ``partial join'' type]. 
\textit{Kibernetika i Sistemnyy Analiz} [\textit{Cybernetics and Systems Analysis}] 6:142--152.
\bibitem{19-z-1}
\Aue{Zykin, S.\,V.} 2010. Formirovanie giperkubicheskogo predstavleniya dannykh 
so spisochnymi komponentami [Formation of the hypercubic data representation with 
list components]. 
\textit{Informatsionnye tekhnologii i vychislitel'nye sistemy} 
[\textit{Information Technologies and Computing Systems}] 4:38--46.
\bibitem{20-z-1}
\Aue{Redreev, P.\,G.} 2009. Postroenie tablichnykh prilozheniy so spisochnymi komponentami 
[Construction of applications with the list components]. 
\textit{Informatsionnye Tekhnologii} [\textit{Information Technologies}] 5:7--12.

\end{thebibliography}
} }


\end{multicols}

\vspace*{-6pt}

\hfill{\small\textit{Received January 9, 2013}}

\vspace*{-18pt}

\Contrl

\noindent
\textbf{Zykin Sergey V.} (b.\ 1959)~--- Doctor of sciences in technology,
professor, Head of laboratory, Sobolev Institute of Mathematics, Siberian
Branch of the Russian Academy of Sciences,  4 Acad.\ Koptyug Av., Novosibirsk 630090,
Russian Federation,  szykin@mail.ru





 \label{end\stat}
 
\renewcommand{\bibname}{\protect\rm Литература}  