\def\stat{sokolov}

\def\tit{БАЗИС РЕАЛИЗАЦИИ СУПЕР-ЭВМ ЭКСАФЛОПСНОГО КЛАССА$^*$}

\def\titkol{Базис реализации супер-ЭВМ эксафлопсного класса}

\def\autkol{И.\,А.~Соколов, Ю.\,А.~Степченков, С.\,Г.~Бобков и~др.}

\def\aut{И.\,А.~Соколов$^1$, 
Ю.\,А.~Степченков$^2$, С.\,Г.~Бобков$^3$, В.\,Н.~Захаров$^4$, 
Ю.\,Г.~Дьяченко$^5$, Ю.\,В.~Рождественский$^6$, А.\,В.~Сурков$^7$}

\titel{\tit}{\aut}{\autkol}{\titkol}

{\renewcommand{\thefootnote}{\fnsymbol{footnote}} \footnotetext[1]{Исследование выполнено при финансовой поддержке РФФИ (проекты 13-07-12062~офи\_м и 
13-07-12068~офи\_м), а также при частичной финансовой поддержке Программы 
фундаментальных исследований ОНИТ РАН за 2013~г.\ (проект~1.5).}}

\renewcommand{\thefootnote}{\arabic{footnote}}
\footnotetext[1]{Институт проблем информатики Российской академии наук, ISokolov@ipiran.ru} 
\footnotetext[2]{Институт проблем информатики Российской академии наук, YStepchenkov@ipiran.ru} 
\footnotetext[3]{Научно-исследовательский институт системных исследований Российской академии наук, 
bobkov@cs.niisi.ras.ru}
\footnotetext[4]{Институт проблем информатики Российской академии наук, VZakharov@ipiran.ru} 
\footnotetext[5]{Институт проблем информатики Российской академии наук, diaura@mail.ru} 
\footnotetext[6]{Институт проблем информатики Российской академии наук, YRogdest@ipiran.ru} 
\footnotetext[7]{Научно-исследовательский институт системных исследований Российской академии наук, 
surkov@cs.niisi.ras.ru} 



\Abst{Статья посвящена выбору схемотехнического базиса реализации 
микропроцессоров и коммуникационных сред супер-ЭВМ эксафлопсного класса. 
Проведен сравнительный анализ характеристик цифровых устройств различной 
сложности, реализованных в синхронном (С) и самосинхронном (СС, self-timed) базисе. 
Подтверждены основные преимущества 
СС-схем по сравнению с синхронными аналогами: отсутствие гонок, максимально 
возможный диапазон работоспособности, высокое быстродействие, относительно 
низкая мощность потребления. В~результате перехода от синхронной к 
квазисамосинхронной и самосинхронной реализации зона работоспособности 
устройства расширяется независимо от его сложности. В наибольшей степени эти 
преимущества проявляются при использовании СС-схем для проектирования надежной 
цифровой аппаратуры. Рассмотрены различные методологии проектирования СС-схем. 
Проведен сравнительный анализ реализации СС-схем в обобщенном базисе схем, 
нечувствительных к задержкам, развиваемом авторами, и в базисе NCL (NULL 
Convention Logic) схем. Показано, что предлагаемый базис обеспечивает получение 
схем с лучшими параметрами по быстродействию, аппаратным затратам и 
энергопотреблению при проектировании типовых цифровых устройств, составляющих 
основу для построения современных вычислительных систем и комплексов.}

\KW{синхронные схемы; самосинхронные схемы; нечувствительность к задержкам; 
NULL Convention Logic; быстродействие; энергопотребление; отказоустойчивость}

\DOI{10.14357/19922264140106}

\vskip 14pt plus 9pt minus 6pt

      \thispagestyle{headings}

      \begin{multicols}{2}

            \label{st\stat}     

\section{Введение}

       В настоящее время потенциал инженерных технологий, опирающихся на теории 
микроскопического взаимодействия в рамках моделей сплошной среды, практически 
исчерпан. Дальнейшее развитие ядерной и термоядерной энергетики, электроники, 
авиастроения, биотехнологий и~др.\ становится невозможным без проведения 
полномасштабных инженерных расчетов сложных технических и биологических систем с 
учетом атом\-но-мо\-ле\-ку\-ляр\-но\-го взаимодействия. А~это требует уже в среднесрочной 
перспективе (до 2020~г.)\ использования су\-пер-ЭВМ эксафлопсного класса (10$^{18}$~оп/с). 
Только те страны, которые будут иметь такие ЭВМ и соответствующее программное 
обеспечение, способны будут создавать принципиально новые изделия. Таким образом, 
создание супер-ЭВМ является одним из актуальнейших направлений развития техники.
       
       Основные трудности в достижении эксафлопсной производительности~--- 
необходимость эффективного и надежного функционирования 10$^8$--10$^9$ процессорных ядер и 
преодоление физических ограничений, обусловленных энергопотреблением, надежностью и 
конструктивными размерами. По оценкам авторов, энергопотребление су\-пер-ЭВМ, 
созданной по имеющимся технологиям и обладающей производительностью 10~PFLOPS, 
составит около 30~МВт. Для создания такой су\-пер-ЭВМ эксафлопсной 
производительности потребуется уже строить рядом с ЭВМ небольшую электростанцию. 

Решение проблем создания су\-пер-ЭВМ эксафлопсной производительности невозможно без 
разработки фундаментальных основ повышения на\-деж\-ности и снижения потребления 
питания требуемых су\-пер-ЭВМ. Используемые современные технологии не позволяют 
создать су\-пер-ЭВМ экса\-флоп\-сной производительности. 

Повышение надежности и снижение 
потребления питания требуемых су\-пер-ЭВМ можно реализовать только за счет реализации 
комплекса архитектурных, схемотехнических, технологических и конструктивных решений.
       
       Проблемы, встающие перед разработчиками современной вычислительной техники 
(низкое энергопотребление, надежность работы в меняющихся условиях эксплуатации 
и~т.\,д.), заставляют по-но\-во\-му взглянуть на принципы проектирования циф\-ро\-вой 
аппаратуры
и в первую очередь на задачу синхронизации. Синхронизация~--- одна из важнейших задач в 
цифровых системах, решающая проблему координации событий (сигналов, операций или 
процессов) в аппаратуре и связанная в основном с обеспечением интерфейса между 
физическим (естественным) и логическим (искусственным) временем~[1]. Координация 
событий отражает при\-чин\-но-след\-ст\-вен\-ные связи между ними и обычно определяется 
последовательностью мно\-жества событий, происходящих в системе.
       
       В середине 1950-х~гг.\ активно исследовались 
альтернативные методологии синхронизации элементов в аппаратуре: синхронная  и 
самосинхронная. В~С-ме\-то\-до\-ло\-гии интерфейс между физическим и 
логическим (сис\-тем\-ным) временем определяется сис\-тем\-ны\-ми часами: все события в 
синхронной сис\-те\-ме могут инициироваться только синхроимпульсами внешних часов. 
Действительная длительность инициированных событий никак не отслеживается. Чтобы 
синхронизируемая аппаратура работала корректно, период синхроимпульсов выбирается из 
расчета на наихудший случай~--- максимально возможное время переключения отдельных 
элементов при неблагоприятных сочетаниях условий функционирования (напряжения 
питания, температуры, параметров и характера распределения нагрузки и~т.\,п.). 
В~результате заведомо ухудшается быстродействие синхронной аппаратуры~--- до 
130\%~[2] по сравнению с номинально возможным быстродействием.
       
       Механизмы, обеспечивающие системное время в СС-под\-хо\-де, включены в модель 
системного поведения и должны быть разработаны вместе с созданием начальной 
поведенческой спецификации. Корректные СС-сис\-те\-мы базируются на механизме 
фиксации действительного окончания всех инициированных процессов. При этом 
обеспечивается их правильное функционирование независимо от задержек распространения 
сигналов в элементах схемы, отчего они также называются нечувствительными к 
задержкам~--- НЗ (delay-insensitive, DI).
       
       С момента появления теории Маллера~[3, 4] проектирование НЗ-схем было 
областью активных исследований~[5--14]. Однако ввиду сравнительной сложности их 
проектирования популярными стали лишь С-схе\-мы. Развитие средств 
автоматизации проектирования (САПР), образовательной и технологической базы также 
пошло в направлении синхронной схемотехники. В~конце XX~в.\ абсолютное большинство 
выпускаемых серийно БИС проектировалось по синхронному маршруту. 
       
       В последние годы неуклонное развитие технологий и растущие требования 
выявляют все больше сложностей в проектировании С-схем, сталкивая 
разработчиков со все большим спектром задач, ранее не изученных. Так, непрерывное 
увеличение производительности влечет за собой постоянную борьбу за снижение 
энергопотребления, а также соблюдение жестких требований к надежности и 
помехозащищенности схем~\cite{15-sok}. Поиск методов и решений этих проблем возродил 
интерес к НЗ-схе\-мо\-тех\-ни\-ке, лишенной части недостатков С-схем, таких как, 
к примеру, избыточное потребление вследствие использования тактирующих цепей. Однако, 
несмотря на многочисленные потенциальные преимущества НЗ-схем, коммерчески 
выпускаемых НЗ-изделий по-прежнему мало. 
       
       В печати приводятся результаты разработки функ\-ци\-о\-наль\-но-за\-кон\-чен\-ных 
нетактируемых изделий вплоть до уровня цифровых сигнальных про\-цес\-со\-ров
(DSP~--- digital signal processor)~\cite{15-sok, 16-sok}, 
сопроцессоров~\cite{17-sok}, самосинхронной машины потока данных DDM2 (MIT). 
Разработан широкий спектр нетактируемых микропроцессоров с архитектурами ARM 
(AMULET 1-3~\cite{18-sok}, ARM996HS~\cite{19-sok}), MIPS (MiniMIPS~\cite{20-sok}), 
Intel (HT80C51~\cite{21-sok}), а также RISC-ар\-хи\-тек\-тур собственной разработки 
(\mbox{ASPRO}~\cite{22-sok}, TengYue-1~\cite{23-sok} и~др.). Некоторые\linebreak из перечисленных 
устройств выпускаются серийно. Относительно недавно начато производст\-во первой в мире 
асинхронной ПЛИС фирмы Achronix~\cite{24-sok}. Параллельно ведутся разработки 
асинхронных \mbox{САПР}, а в ведущих институтах Америки, Англии и Китая студентам читают 
образовательные курсы по асинхронным автоматам.
       
       Однако эти реализации в действительности не являются НЗ-схемами. Они 
используют запрос-от\-вет\-ное взаимодействие (handshake) между функциональными 
блоками вместо <<дерева>> тактового\linebreak
 сигнала и за счет этого существенно сокращают 
по\-треб\-ле\-ние энергии и расширяют область работоспособности. Но они не содержат 
реального контроля окончания переходных процессов при переключениях схемы, присущего 
НЗ-схе\-мам. Контроль заменен элементами задержки~\cite{19-sok}, эмулирующими 
реальные задержки обработки данных на соответствующих участках вычислительного 
тракта. Такой подход обеспечивает аппаратные затраты на уровне синхронных аналогов, 
но не исключает возможности появления сбоев в работе схемы при разбросе параметров 
транзисторов и элементов схемы, обусловленных технологическими и эксплуатационными 
факторами, а следовательно, не может быть признан действительно нечувствительным к 
задержкам.
       
       В России активным пропагандистом НЗ-под\-хода был коллектив под руководством 
д.т.н.\linebreak В.\,И.~Варшавского. В~работах~[1, 25--28], 
развивающих положения теории Маллера, убедительно доказана целесообразность 
внедрения принципа самосинхронизации в практику проектирования цифровых СБИС. 
С~1980-х~гг.\ это направление проектирования аппаратуры активно поддерживается и 
развивается в ИПИ РАН~[29--57]. 
       
       Реализации НЗ-схем обладают рядом преимуществ по сравнению с синхронными 
аналогами~\cite{29-sok}:
       \begin{itemize}
\item устойчивая работа~--- отсутствие сбоев при любых возможных условиях эксплуатации;
\item безопасная работа~--- прекращение всех переключений в момент появления отказа 
любого элемента (константной неисправности, при которой выход элемента <<залипает>> в 
одном состоянии);
\item отсутствие периодов вынужденного простоя в ожидании очередного синхроимпульса.
       \end{itemize}
       
       Практические следствия этих преимуществ НЗ-схем:
       \begin{itemize}
\item естественная устойчивость к параметрическим отказам, вызываемым изменением 
параметров элементов из-за процессов старения и неблагоприятных воздействий 
окружающей среды;
\item естественная стопроцентная самопрове\-ря\-емость и самодиагностируемость по 
отношению к множественным константным неисправностям;
\item безопасность функционирования на основе бестес\-товой локализации неисправностей, 
т.\,е.\ прекращение работы в момент отказа элемента, исключающее выдачу недостоверной 
информации, с одновременной локализацией места события;
\item максимально возможная область эксплуатации (диапазон работоспособности), 
определяемая только физическим сохранением переключательных свойств активных 
элементов базиса реализации;
\item максимально возможное в текущих условиях эксплуатации быстродействие;
\item отсутствие накладных аппаратных и энергетических расходов, связанных с 
реализацией <<клокового дерева>>~--- разветвленной системы синхронизации, 
обеспечивающей строгую одновременность событий в разных местах проектируемой схемы.
       \end{itemize}
       
       Единственный недостаток НЗ-схем~--- большие аппаратные затраты. 
В~зависимости от класса рассматриваемого цифрового устройства его\linebreak НЗ статическая 
КМОП (комплементарная структура ме\-талл--ок\-сид--по\-лу\-про\-вод\-ник)
реализация\linebreak тре\-бует в 1,3--2,5~раза больше транзисторов, чем аналогичная 
синхронная реализация. Наихудшее соотноше\-ние аппаратных затрат наблюдается в 
комбинационных схемах из-за необходимости использования дуального представления 
каждой функции и добавления схемы контроля окончания всех переходных процессов при 
переключениях схемы.
       
       В данной статье рассматривается возможность существенного улучшения требуемых 
характеристик (до 50\%) высокопроизводительных вычислительных систем за счет перехода 
от синхронной схемотехники к самосинхронной. Возможность снижения потребления 
питания основывается на следующем:
       \begin{itemize}
       \item в современных высокопроизводительных мик\-ро\-процессорах потребление 
дерева синхронизации составляет от 30\% до 50\% потребления всей схемы, в СС-схе\-мах не 
используется дерево синхронизации;
       \item имеется значительный разброс параметров транзисторов на расстояниях свыше 
1~мм для технологических норм 28~нм и ниже, что приводит к необходимости 
дополнительных технических решений, приводящих к повышению потребления питания в 
С-схе\-мах, в НЗ-схе\-мах разброс параметров учитывается автоматически в силу 
базовых НЗ решений;
       \item наибольшие производительности достигнуты в графических процессорах 
компаний Nvidia и AMD, где вычисления организуются как потоковые процессы; 
       НЗ-ло\-ги\-ка наилучшим образом согласуется с потоковыми машинами, поскольку  
в них обоих используются состояния готовности; подобное свойство позволяет избежать 
дополнительных накладных расходов на организацию вычислений и снизить 
энергопотребление.
       \end{itemize}
       
       Ресурс повышения надежности в НЗ-схе\-мах по сравнению с С-схе\-ма\-ми 
обеспечивается их базовыми решениями, в которые закладывается дополнительная 
информация для их функционирования, используемая и для повышения надежности.
       
Схемы, нечувствительные к задержкам, органично вписываются в концепцию создания современных 
вычислительных сис\-тем, обеспечивая \mbox{низкое} энергопотребление и сохранение 
работоспособности в изменяющихся условиях эксплуатации оборудования. Данная работа 
посвящена сравнительному анализу вариантов реализации типовых представителей 
основных классов вычислительных устройств и обоснованию выбора схемотехнического 
базиса реализации мик\-ро\-про\-цес\-со\-ров и коммуникационных сред су\-пер-ЭВМ 
эксафлопсного класса.

\section{Сравнительный анализ синхронных и~нечувствительных к~задержкам схем}
       
       С практической точки зрения наиболее интересным является подкласс схем, 
нечувствительных к задержкам в элементах (НЗЭ). В~пределах эквихронной 
       зоны~\cite{28-sok} они обладают всеми свойствами и преимуществами НЗ-схем. 
Только при передаче информации отдаленному приемнику вне пределов эквихронной зоны 
необходимо предпринимать дополнительные меры по обеспечению нечувствительности к 
задержкам в соединительных проводах. При микронных нормах проектирования топологии 
микросхем эквихронная зона измерялась миллиметрами и практически покрывала всю 
площадь кристалла БИС, так как задержки переключения элементов превалировали над 
задержками распространения сигналов в соединительных проводах. Поэтому НЗЭ-схе\-ма в 
рамках одного кристалла БИС фактически являлась НЗ-схе\-мой.
       
       Однако с переходом к субмикронным нормам проектирования БИС размер 
эквихронной зоны существенно сократился из-за того, что задержки переключения 
элементов многократно уменьшились и стали сравнимы и даже меньше задержек 
распространения сигналов в проводах. В~современных цифровых СБИС эквихронная зона 
покрывает лишь малую часть площади кристалла. Поэтому НЗЭ-схе\-мы могут считаться 
       НЗ-схе\-ма\-ми, если связанные друг с другом функциональные блоки имеют 
соответствующие небольшие размеры и расположены относительно близко друг к другу.
       
       Практически целесообразными являются также ква\-зи-НЗЭ (КНЗЭ) схе\-мы. 
Основное отличие НЗЭ-схем от КНЗЭ-схем состоит в том, что НЗЭ-схе\-мы контролируют с 
помощью индикаторной подсхемы окончание переключения каждого элемента в схеме, в то 
время как КНЗЭ-схе\-мы имеют спекулятивную индикацию~--- обеспечивают индикацию 
только элементов, стоящих на критических путях\linebreak
 обработки информации. За счет этого 
многоразрядные КНЗЭ-схе\-мы оказываются более быст\-ро\-дей\-ст\-ву\-ющи\-ми и менее 
сложными. Но они не \mbox{дают} стопроцентной гарантии сохранения работоспособности схемы 
при изменении в широком диапазоне условий эксплуатации: напряжения питания, 
температуры~--- и при воздействии экстремальных факторов.
       
       В работе~\cite{55-sok} представлены результаты сравнительных испытаний С-, 
       КНЗЭ- и НЗЭ-вариантов реализации цифровых устройств различной сложности: 
       \begin{itemize}
\item 4-разрядного микроядра~\cite{33-sok}, 
аналога ядра микроконтроллера PIC16 фирмы Microchip, США, включающего типовые 
ариф\-ме\-ти\-ко-ло\-ги\-че\-ские устройства: регистровую память, аппаратный 
умножитель, сдвигатель, счетчики;
\item 8-разрядного отказоустойчивого по\-сле\-до\-ва\-тель\-но-па\-рал\-лель\-но\-го  
(ПП) порта, эмулиру\-юще\-го последовательный интерфейс между двумя цифровыми 
устройствами~\cite{34-sok};
\item 64-разрядного сопроцессора~--- устройства деления и извлечения квадратного 
корня~\cite{37-sok, 38-sok, 50-sok, 49-sok} в соответствии со стандартом 
IEEE754~\cite{58-sok}.
\end{itemize}

       Сравнение С-, КНЗЭ- и НЗЭ-ва\-ри\-ан\-тов реализации перечисленных цифровых 
устройств проводилось на основе оценки быстродействия в реаль\-ных условиях 
эксплуатации. Частота тактирования С-устройств устанавливалась из расчета на наихудший 
случай в пределах допустимой области эксплуатации. Быстродействие же НЗЭ-схем 
определялось реальными, а не наихудшими условиями эксплуатации. Именно поэтому 
       НЗЭ-устройства в нормальных условиях оказываются, как правило, быстрее 
       С-ана\-ло\-гов, что и было подтверждено результатами испытаний перечисленных 
выше вариантов цифровых устройств.

\setcounter{figure}{1}
\begin{figure*}[b] %fig2
             \vspace*{1pt}
 \begin{center}
 \mbox{%
 \epsfxsize=162.888mm
 \epsfbox{sok-2.eps}
 }
 \end{center}
 \vspace*{-9pt}
\Caption{Энергетические параметры НЗЭ-~(\textit{1}) и С-ва\-ри\-ан\-тов~(\textit{2}) 
микроядра при $T\hm=27$~$^\circ$C}
\end{figure*}

\setcounter{table}{1}
\setcounter{figure}{1}
       
       В табл.~1 приведены аппаратные затраты С- и НЗЭ-ва\-ри\-ан\-тов 
реализации микроядра и ПП-пор\-та в вентилях базового матричного кристалла (БМК) 
серии~5503 (МИЭТ, Технологический центр). Мик\-ро\-яд\-ро НЗЭ, содержащее большую 
комбинационную схему~--- умножитель $4\times 4$, построенный
 по модифицированному 
алгоритму Бута, оказалось в 1,43~раза сложнее своего синхронного аналога. Благодаря 
заметному сокращению общего числа\linebreak

\vspace*{-12pt}

{\small
\begin{center}
{{\tablename~1}\ \ \small{Аппаратные затраты}}

\vspace*{6pt}

\begin{tabular}{|l|c|c|}
\hline
\tabcolsep=0pt\begin{tabular}{c}Цифровое\\ устройство\end{tabular}&С-вариант&НЗЭ-вариант\\
\hline
Микроядро&970&1390\\
ПП-порт&443&\hphantom{9}370\\
\hline
\end{tabular}
\end{center}
\vspace*{-6pt}
}




%\pagebreak

\vspace*{12pt}


\noindent
\begin{center}  %fig1
\mbox{%
 \epsfxsize=78.608mm
 \epsfbox{sok-1.eps}
 }
  \end{center}

  \vspace*{-3pt}

\noindent
{{\figurename~1}\ \ \small{Зона работоспособности НЗЭ-~(\textit{1}) и С-об\-раз\-цов~(\textit{2}) микроядра}}

\vspace*{12pt}

      

\addtocounter{figure}{1}


\noindent
 устройств в составе отказоустойчивого 
       НЗЭ-ПП-пор\-та по сравнению с синхронным~\cite{55-sok}, его суммарные затраты 
оказались на 20\% меньше, чем в С-ва\-ри\-ан\-те. Аналогичные характеристики могут 
быть получены и при реализации данных устройств в виде функциональных блоков заказной 
БИС.



       На рис.~1 приведены результаты эксперимента по проверке работоспособности всех 
НЗЭ- и \mbox{С-об}\-разцов в диапазоне изменяющихся напряжения питания и температуры при 
пороговых напряжениях транзисторов на уровне 0,8~В. Частота синхронизации С-об\-раз\-цов 
подбиралась для каж\-дой пары значений <<напряжение 
       пи\-та\-ния\,--\,тем\-пе\-ра\-ту\-ра>>. Из рис.~1 видно, что НЗЭ-образ\-цы оказались 
работоспособными в более широком диапазоне условий эксплуатации, причем все без 
исключения. Синхронные же образцы продемонстрировали разброс параметров зоны 
работоспособности из-за флуктуации технологических параметров при изготовлении 
микросхем.
      
      
       Производительность С-реализаций микроядра при фиксированной частоте 
тактирования, рассчитанной на наихудший случай, и выполнении смеси операций постоянна для 
всех возможных условий эксплуатации в пределах гарантированной области работоспособности 
и составила 4~MOPS. Быстродействие же НЗЭ-микроядра широко изменяется в зависимости от 
условий эксплуатации. Например, в зоне работоспособности, гарантированной изготовителем 
БМК, его производительность изменяется от 10,9~MOPS (5,5~В, $-63$~$^\circ$C) до 5,2~MOPS 
(4,5~В, $+125$~$^\circ$C). В~среднем во всем реальном диапазоне работоспособности она 
оказалась выше производительности С-мик\-ро\-яд\-ра почти в 2~раза.
       
       На рис.~2,\,\textit{а} приведен график зависимости тока потребления (Icc) С- и НЗЭ-вариантов 
реализации микроядра от величины напряжения питания при температуре 
$T\hm=+27$~$^\circ$C. При одном и том же напряжении питания НЗЭ-ва\-ри\-ант 
потребляет несколько больше, чем синхронный, что объясняется его более высокой 
производительностью.
       
       
      \begin{figure*} %fig3
                   \vspace*{1pt}
 \begin{center}
 \mbox{%
 \epsfxsize=165.286mm
 \epsfbox{sok-3.eps}
 }
 \end{center}
 \vspace*{-9pt}
      \Caption{Результаты испытаний вариантов сопроцессора}
      \end{figure*}
      
       Для более корректной оценки сравнительного потребления энергии на рис.~2,\,\textit{б} 
приведен график энергетической эффективности, который показывает ток 
потребления микроядра при выполнении операций с производительностью 1~MOPS. Чем 
меньше величина~$E$, тем более эффективна реализация. Из рис.~2 видно, что 
       НЗЭ-реа\-ли\-за\-ция микроядра более эффективна, чем его С-ва\-ри\-ант. Например, 
при номинальном напряжении питания 5~В энергетическая эффективность составляет 
1,2~мА/MOPS для НЗЭ-образ\-ца и 1,8~мА/MOPS для С-образ\-ца. При напряжении питания 
12~В имеет место двукратное превосходство НЗЭ-реа\-ли\-за\-ций. Сочетание возможности 
КМОП-НЗЭ-схем работать (и потреблять энергию) только <<по требованию>> с низким 
потреблением пассивной логики создает хорошие предпосылки для создания энергетически 
эффективных аппаратных ре\-шений. 
{\looseness=1

}
       
       Ценой повышения производительности и расширения зоны работоспособности 
       НЗЭ-ва\-ри\-анта\linebreak микроядра является увеличение его аппаратных затрат. 
В~качестве интегральной оценки эффек\-тив\-ности (добротности) реализации цифрового 
устройства может служить отношение произве\-дения производительности при номинальном\linebreak 
питании на ширину зоны работоспособности к аппаратным затратам. Суммарное 
преимущество\linebreak НЗЭ-мик\-ро\-яд\-ра в сравнении с синхронным аналогом по этому параметру 
с учетом гарантированной производителем области ра\-бо\-то\-спо\-соб\-ности С-устройств 
по напряжению питания (номинал ${}\pm$\linebreak $\pm\;10\%$) со\-став\-ля\-ет 17,6~раза.
       
       В качестве способа построения отказоустойчивого НЗЭ-ПП-пор\-та было выбрано 
дублирование его основной функциональной части~--- регистра сдвига, а в 
       С-ва\-ри\-ан\-те~--- троирование регистра сдвига. Один из дубликатов изначально 
является рабочим, остальные~--- контрольными и/или резервными. Во всех случаях 
применяется постоянный контроль одинаковости результата, который получается 
независимо каждым устройством на основе общих входных данных и позволяет выявить 
возникшее несовпадение. <<Ремонт>> схемы состоит в мультиплексировании на выход 
заведомо исправного устройства. Такой способ обеспечивает оперативный саморемонт 
одного отказа в сдвиговом регистре ПП-пор\-та и достоверность определения 
работоспособности всех частей схемы.
       
       Результаты измерения показали~\cite{55-sok}, что отказоустойчивый НЗЭ-ПП-порт 
по сравнению с С-ва\-ри\-ан\-том имеет существенно лучшие характеристики: 
в~2,4~раза по быстродействию; в~1,2~раза по аппаратным затратам; в~1,3~раза по 
энергетической эффективности; в 18~раз по доб\-рот\-ности.
       
       Сравнение вариантов сопроцессора проводилось на С-ва\-ри\-ан\-тах, 
реализующих алгоритмы Ньютона (С-N) и SRT Radix4 (С-SRT), и КНЗЭ-ва\-ри\-ан\-те, 
реализующем алгоритм SRT Radix2. Все варианты сопроцессора были реализованы в составе 
тестовой микросхемы по стандартной 0,18-мик\-рон\-ной КМОП-тех\-но\-ло\-гии с шестью слоями 
металлизации. 
       
       Сравнительные результаты испытаний вариантов сопроцессора показаны на рис.~3. 
Производительность измерялась при нормальных условиях работы (напряжение питания\;=\,1,8~В, 
$T\hm=27$~$^\circ$C). Зона работоспособности определялась как произведение 
диапазонов напряжения питания и температуры, в которых сопроцессор демонстрировал 
устойчивую работу.
       

      
       Таким образом, реализация микроядра, ПП-пор\-та и сопроцессора в виде НЗЭ- или 
КНЗЭ-устрой\-ст\-ва обеспечивает их устойчивую работу при любых допустимых условиях 
эксплуатации. Устройства НЗЭ и КНЗЭ экспериментально подтвердили свои не\-оспо\-ри\-мые 
преимущества по производительности и зоне работоспособности по сравнению с 
синхронными аналогами, поэтому такой схемотехнический базис целесообразно 
использовать и для разработки современных вы\-чис\-ли\-тель\-ных устройств и комплексов.
       
\section{Варианты методологии проектирования схем, нечувствительных к~задержкам}

       В работе~\cite{14-sok} рассмотрено 10 различных методологий проектирования 
асинхронных и, в частности, СС-схем. Их можно разделить на две группы:
       \begin{enumerate}[($I$)]
\item методологии, базирующиеся на модели с ограниченной задержкой, например схемы 
Haffman'а и микроконвейеры. Некоторые из них предназначены для проектирования НЗ-схем 
(например, I-net), однако объединение таких схем требует 
использования либо линий задержки в цепях обратной связи, либо системы локальной 
синхронизации. Подобно С-схе\-мам, они вынуждены ориентироваться на 
наихудший случай условий работы схемы. Такие схемотехнические решения фактически 
являются квазисамосинхронными. Большинство наиболее известных зарубежных 
СС-мик\-ро\-схем и реализованных проектов относится именно к этому классу~[15--24, 59--61];
\item подходы, базирующиеся на модели элементов и соединительных проводов с 
неограниченной задержкой до точки разветвления. При этом предполагается, что разница в 
задержке проводов после разветвления меньше, чем минимальная задержка элемента. 
Примеры таких методологий: графы сигнальных переходов (STG~--- signal transition\linebreak
graph), диаграммы изменений 
(CD~--- change diagrams) и трансляция процессов связи %\linebreak
 Martin'а~\cite{7-sok, 9-sok}. При не\-об\-хо\-ди\-мости эти %\linebreak 
методологии могут быть расширены для разработки схем, не удовлетворяющих требованию 
изохронности ветвления, путем введения индикаторов переходных процессов в 
непосредственной близости к приемнику сигнала. 
\end{enumerate}

       Методология, разрабатываемая в ИПИ РАН, относится ко второй группе. Она имеет 
следующие особенностий:
       \begin{itemize}
\item на концептуальном уровне она базируется на теории Маллера~\cite{3-sok, 4-sok}. 
Правильная работа построенных по данной методологии схем не зависит от \textit{задержек 
составляющих их элементов} (задержка любого элемента схемы, например инвертора, 
может быть любой, но конечной величины);
\item на схемотехническом уровне использование дополнительных логических и 
топологических приемов позволяет обеспечить правильную работу СС-схем независимо 
\textit{от задержек в соединительных проводах};
\item на уровне взаимодействия с внешней средой и другими СС-схе\-ма\-ми используется 
асинхронный (запрос-от\-вет\-ный) принцип с фиксацией\linebreak действительного окончания любого 
ини\-ци\-ированного переходного процесса. Генераторы могут использоваться только для 
сугубо второстепенных целей, например для создания внут\-рен\-них таймеров.
\end{itemize}

       Перечисленные выше свойства СС-схем пред\-опре\-деляют высокую эффективность 
создания надежных изделий на их основе, в том числе и отказоустойчивых. Однако в полной 
мере данными свойствами обладают только НЗ-схе\-мы. 
Данное свойство (нечувствительность) относится к задержкам распространения сигналов 
через логические элементы и по соединительным проводам.
       
       Типичными представителями НЗ-схем среди зарубежных СС-устройств являются 
NCL-схе\-мы~[14, 62--68]. Методология NCL была предложена в 
       \mbox{1990-х~гг}.~\cite{13-sok}. В~настоящее время она развивается в основном усилиями 
компании Theseus Research, Inc. (TR) и университета в Арканзасе (University of Arkansas).
       
       Ниже представлен сравнительный анализ основных принципов проектирования 
       НЗ-схем, разрабатываемых в соответствии с методологиями ИПИ РАН и NCL. 
В~дальнейшем будем называть первые просто НЗ-схе\-ма\-ми, а последние~--- 
       NCL-схе\-мами.
       
       \subsection{Дисциплина сигналов}
       
       Схемотехника NCL основана на парафазном с нулевым спейсером кодировании всех 
информационных сигналов. Любая функция, выполняемая устройствами внутри схемы, 
реализуется путем ее дуального представления. Состояние каждого информационного 
сигнала~$A$ представляется комбинацией двух компонентов $\{A, AB\}$. В~спейсерной 
фазе (фазе NULL) $A\hm=AB\hm=0$, в рабочей фазе (фазе DATA) $\{A\hm=0$, $AB\hm=1\}$ 
или $\{A\hm=1$, $AB \hm =0\}$. Состояние $A\hm=AB\hm=1$ запрещено и при нормальной 
работе схемы никогда не формируется. Переключение информационного сигнала из 
текущего рабочего состояния в следующее всегда происходит через спейсерное состояние, 
даже если текущее и следующее рабочие состояния совпадают.
{\looseness=-1

}
       
       Дисциплина информационных сигналов в методологии проектирования НЗ-схем 
более гибкая. Она включает использование как парафазных сигналов со спейсером, 
аналогичных сигналам NCL-схем, так и других сигналов: парафазных без спейсера, 
бифазных (выходов бистабильной ячейки), унарных, управляющих (аналогов тактовых 
сигналов в С-схе\-мо\-тех\-ни\-ке), мультифазных (например, сигналов выборки 
мультиплексоров). Это позволяет строить более компактные схемы.

      \begin{figure*}[b] %fig4
                   \vspace*{-6pt}
 \begin{center}
 \mbox{%
 \epsfxsize=128.066mm
 \epsfbox{sok-4.eps}
 }
 \end{center}
 \vspace*{-9pt}
      \Caption{Функциональная схема NCL-эле\-мен\-та: статическая~(\textit{а}) и 
полустатическая~(\textit{б})}
      \end{figure*}
       
       Парафазные сигналы со спейсером своим значением отображают собственную фазу 
и закодированный бит данных. Остальные информационные сигналы своим статическим 
значением отображают только заложенный в них бит информации. Поэтому для фазового 
регулирования парафазные без спейсера, унарные и бифазные информационные сигналы в 
обязательном порядке сопровождаются сигналом управления. Сигнал управления переходит 
в рабочую фазу только после того, как сопровождаемый им информационный сигнал 
переключился в следующее рабочее состояние, тем самым давая знать приемнику 
информационного сигнала, что соответствующий информационный вход можно 
использовать. Переключение сигнала управления в спейсер инициируется приемниками 
сопровождаемого им информационного сигнала и является подтверждением факта 
<<доставки по назначению>> соответствующего информационного сигнала. 
Информационный сигнал может изменять свое состояние только во время спейсера 
со\-про\-вож\-да\-юще\-го его сигнала управления.

\vspace*{-3pt}
       
       \subsection{Схемотехнический базис}
       
       \vspace*{-2pt}
       
       Успех проектирования цифровых схем любого типа не в последнюю очередь 
определяется составом библиотеки элементов, на основе которой ведется проектирование. 
Качество CC-схем зависит от этого даже в большей степени, чем качество\linebreak С-схем. 
       
       Схемотехника NCL-схем основана на использовании функциональных схем, 
показанных на рис.~4. Разработчики NCL-схем называют элементы схемотехнического 
базиса \textit{пороговыми элементами} (threshold gates). Термин <<пороговые>> в данном 
случае относится не к потенциальному уровню сигналов на входе и выходе элемента, а к 
количеству входов, которые должны переключиться из спейсера (NULL) в рабочее состояние 
(DATA), чтобы выход элемента тоже переключился в рабочее состояние. Обратное 
переключение в спейсер возможно только тогда, когда все входы перейдут в спейсер.
      

      
       Статическая реализация (рис.~4,\,\textit{а}) состоит из подсхем на 
       КМОП-тран\-зис\-то\-рах $n$- и $p$-ти\-па проводимости, обеспечивающих 
переключение элемента в обе фазы работы и хранение текущего состояния до момента 
появления комбинации входов, вызывающей переключение в противоположное состояние. 
При этом функции, реализуемые частями <<Переход в NULL>> и <<Переход в DATA>> и 
обеспечивающие соединение выхода элемента с источниками активных уровней (питание и 
<<земля>>), ортогональны: при любой комбинации входов на выход элемента 
коммутируется только один источник логического уровня. 
       
       Полустатическая (semistatic) реализация (рис.~4,\,\textit{б}) использует слабый 
инвертор в качестве обратной связи для организации хранения состояния NULL или DATA. 
Она обладает меньшей помехоустойчивостью и характеризуется протеканием сквозного тока 
при переключении NCL-схе\-мы из текущего состояния в противоположное. В~дальнейшем 
будем считать, что для реализации NCL-схе\-мы используется статический вариант, 
показанный на рис.~4,\,\textit{а}.
       
       Статическая реализация NCL-схе\-мы фак\-ти\-чески является гистерезисным 
триггером~\cite{28-sok} со слож\-ной функциональной начинкой. В~методо\-логии 
проектирования СС-устройств ИПИ РАН гис\-те\-ре\-зис\-ный триггер (Г-триг\-гер) играет роль 
индикаторного элемента. Нагрузка его дополнительными функциями позволяет в ряде 
случаев уменьшить сложность реализации схемы, но является скорее исключением из 
правил, нежели типовым приемом проектирования СС-схем.


       
       Схемотехнический базис NCL-схем состоит из фундаментальных элементов ТНmn. 
Имя элемента обозначает <<пороговый элемент с~$n$~входами и порогом~$m$>>. Они 
воплощают в себе один из принципов проектирования NCL-схем: <<реализация любого 
       NCL-эле\-мен\-та и любой NCL-схе\-мы должна обеспечивать переключение 
выходов в спейсер (NULL) только после того, как все входы перешли в спейсер, и 
переключение выходов в рабочее состояние (DATA) только тогда, когда не меньше 
порогового числа входов у элемента или схемы перешли в рабочее состояние>>. Некоторые 
элементы имеют дополнение к имени в виде окончания <<wXXX>>. Цифры после 
буквы~<<w>> показывают вес соответствующей по порядку входной переменной. Если 
порог элемента равен~$m$, а вход~$A$ имеет вес~$k$, то для переключения элемента в фазу 
достаточно переключения входа~$A$ и еще каких-нибудь ($m$--$k$) входов, если вес 
остальных входов единичный, или переключения любых $m$ входов, кроме входа~$A$. 
Общее число фундаментальных элементов равно~27. Они перечислены в 
       табл.~2~\cite{68-sok}. Там же указаны выполняемые ими функции и количество 
КМОП-тран\-зис\-то\-ров, необходимых для их реализации.

 Функции, показанные в табл.~2, описывают блок <<Переход в DATA>> 
функциональной схемы на рис.~4,\,\textit{а}. Условное графическое обозначение (УГО) 
фундаментальных NCL-эле\-мен\-тов показано на рис.~5.

\vspace*{6pt}

{\small
\begin{center}
{{\tablename~2}\ \ \small{Состав библиотеки NCL-элементов}}

\vspace*{6pt}

\tabcolsep=3pt
\begin{tabular}{|l|l|c|}
\hline
\multicolumn{1}{|c|}{\tabcolsep=0pt\begin{tabular}{c}Имя\\ элемента\end{tabular}}&
\multicolumn{1}{|c|}{Выполняемая функция}&\tabcolsep=0pt\begin{tabular}{c}Число\\ транзи-\\ сторов\end{tabular}\\
\hline
TH12 &A\;+\;B &\hphantom{9}6\\
TH22 &AB &12\\
TH13 &A\;+\;B\;+\;C &\hphantom{9}8\\
TH23 &AB\;+\;AC\;+\;BC &18\\
TH33 &ABC &16\\
TH23w2 &A\;+\;BC &14\\
TH33w2 &AB\;+\;AC &14\\
TH14 &A\;+ B\;+\;C\;+\;D &10\\
TH24 &AB\;+\;AC\;+\;AD\;+\;BC\;+\;BD\;+\;CD &26\\
TH34 &ABC\;+\;ABD\;+\;ACD\;+\;BCD &24\\
TH44 &ABCD &20\\
TH24w2 &A\;+\;BC\;+\;BD\;+\;CD &20\\
TH34w2 &AB\;+\;AC\;+\;AD\;+\;BCD &22\\
TH44w2 &ABC\;+\;ABD\;+\;ACD &23\\
TH34w3 &A\;+\;BCD &18\\
TH44w3 &AB\;+\;AC\;+\;AD &16\\
TH24w22 &A\;+\;B\;+\;CD &16\\
TH34w22 &AB\;+\;AC\;+\;AD\;+\;BC\;+\;BD &22\\
TH44w22 &AB\;+\;ACD\;+\;BCD &22\\
TH54w22 &ABC\;+\;ABD &18\\
TH34w32 &A\;+\;BC\;+\;BD &17\\
TH54w32 &AB\;+\;ACD &20\\
TH44w322 &AB\;+\;AC\;+\;AD\;+\;BC &20\\
TH54w322 &AB\;+\;AC\;+\;BCD &21\\
THxor0 &AB\;+\;CD &20\\
THand0 &AB\;+\;BC\;+\;AD &19\\
TH24comp &AC\;+\;BC\;+\;AD\;+\;BD &18\\
\hline
\end{tabular}
\end{center}
} 


      \columnbreak
      
      \noindent
\begin{center}  %fig5
\mbox{%
 \epsfxsize=52.356mm
 \epsfbox{sok-5.eps}
 }
  \end{center}

  \vspace*{-3pt}

\noindent
{{\figurename~5}\ \ \small{Условное графическое обозначение элемента THmn}}

\vspace*{12pt}

      

\addtocounter{figure}{1}
       

      

       Фундаментальные элементы NCL-ба\-зи\-са служат основой для синтеза 
производных от них элементов, необходимых для проектирования практических цифровых 
устройств, например триггеров с асинхронным сбросом или установкой.
       
       Схемотехника НЗ-схем основана на использовании классических КМОП 
принципиальных схем.\linebreak Необходимость индицировать окончание пе\-ре\-ходных процессов в 
элементах НЗ-схе\-мы делает нежелательной сильную функциональную декомпозицию. Она 
приводит к появлению в схеме множества <<мелких>> логических элементов, каждый из 
которых требует дополнительных аппаратных затрат для реализации его индикации. Это 
делает целесообразной разработку библиотеки с широкой номенклатурой логических, 
триггерных и индикаторных элементов. При этом функциональный состав библиотеки 
определяется, в первую очередь, целесообразностью использования тех или иных элементов 
в НЗ-схе\-мах, а во вторую очередь, технологическим базисом реализации проектируемой 
БИС. 
       
       При проектировании НЗ-схем используется биб\-ли\-о\-те\-ка элементов, содержащая 
260~единиц~[69--71] и являющаяся самосинхронным дополнением 
типовых библиотек стандартных элементов. Биб\-ли\-о\-те\-ка включает логические элементы для 
формирования комбинационных схем, мультиплексоры, триггеры (D, RS, счетные), 
сумматоры, преобразователи сигналов. Условные графические обозначения 
элементов схемотехнического базиса 
       НЗ-схем~--- традиционные для С-схе\-мо\-тех\-ни\-ки, и лишь отсутствующие в 
последней элементы имеют характерные графические обозначения.  В~качестве  примера на  
рис.~6  показаны  УГО трехвходового\linebreak Г-триг\-ге\-ра (GI3) и индикаторного элемента G0B3I2, 
индицирующего входы и состояние двух связанных друг с другом бистабильных ячеек.
      
   
      

      \setcounter{figure}{6}
      \begin{figure*}[b] %fig7
 \begin{center}
 \mbox{%
 \epsfxsize=159.239mm
 \epsfbox{sok-7.eps}
 }
 \end{center}
 \vspace*{-9pt}
\Caption{Конвейер NCL}
\end{figure*}
       
       Элементы, аналогичные ТН22, используются в качестве индикаторных элементов, 
обеспечивающих контроль окончания переключений в НЗ-схеме. 
       
       Ряд триггеров разработанной библиотеки имеет уникальные свойства, 
обеспечивающие эффектив-\linebreak

\setcounter{figure}{5} \noindent
\begin{center}  %fig6
\mbox{%
 \epsfxsize=72.876mm
 \epsfbox{sok-6.eps}
 }
  \end{center}

  \vspace*{-3pt}

\noindent
{{\figurename~6}\ \ \small{Условные графические обозначения Г-триг\-ге\-ра GI3~(\textit{а})
       и индикаторного элемента G0B3I2~(\textit{б})}}

\vspace*{12pt}

\setcounter{figure}{7}
      
\noindent
ное решение корректными средствами двух проб\-лем:
       \begin{enumerate}[(1)]
\item большой нагрузочной способности выходов триггеров;
\item упрощенной реализации входного интерфейса с С-фор\-ми\-ро\-ва\-те\-ля\-ми 
входных данных.
\end{enumerate}
Использование таких триггеров в составе НЗ-схем гарантирует сохранение их 
свойств в полном объеме при приемлемых потребительских характеристиках: быстродействии 
и сложности реализации. К числу таких триггеров относятся, в первую очередь, триггер с 
мощными инверторами на информационных выходах~\cite{72-sok} и триггер с единичным 
непарафазным информационным входом~\cite{48-sok}.
       
       Библиотека элементов для проектирования НЗ-схем включена в состав САПР 
<<Ковчег>> (Технологический центр, МИЭТ) и позволяет разрабатывать НЗ-схе\-мы 
различной степени сложности с использованием БМК 
серий 5503, 5507, 5508, 5509.

\begin{figure*}[b] %fig8
              \begin{center}
 \mbox{%
 \epsfxsize=159.299mm
 \epsfbox{sok-8.eps}
 }
 \end{center}
 \vspace*{-9pt}
\Caption{Последовательная NCL-схема}
\end{figure*}
       
       Разработанная библиотека внедрена также в САПР фирмы Cadence для 
       КМОП-тех\-но\-ло\-гии 0,18~мкм~\cite{73-sok} и 65~нм (для проектирования 
заказных БИС). Она включает схемотехнические, топологические представления, а также 
Verilog и VHDL-мо\-де\-ли. Топология элементов для проектирования заказных БИС может 
быть отмасштабирована до уровня 45~нм. Для их характеризации использовались 
стандартные программные средства фирмы Cadence и разработанный в ИПИ РАН 
про\-грам\-мный комплекс СТЕРХ~\cite{74-sok}. Разработанная биб\-ли\-о\-те\-ка успешно прошла 
апробацию на ряде полузаказных и заказных БИС.

\vspace*{-3pt} 

       \subsection{Принципы построения схемы}
       
       \vspace*{-1pt}
       
       В NCL-методологии любая схема строится в виде конвейера (рис.~7). Элементы 
ТН22 на входах каждого комбинационного блока формируют парафазный код на основе 
парафазного выхода предыду\-щей ступени конвейера и сигнала разрешения, выдаваемого 
индикатором следующей ступни. Схема индикации, функционально эквивалентная элементу 
ТН48, формирует фазовый сигнал, обеспечивающий взаимодействие соседних ступеней 
конвейера.
       
       Рисунок~8 демонстрирует реализацию последовательной NCL-схе\-мы. Здесь 
регистры присутствуют и на входе схемы, и на ее выходе. Кроме того, используется регистр 
обратной связи, обеспечивающий корректное хранение состояния переменных памяти в 
схеме.
       
       Спейсер NULL одинаков для всех элементов и устройств NCL-схе\-мы: нулевое 
значение обеих составляющих каждого парафазного сигнала. Все элементы схемы, кроме 
элементов индикаторной подсхемы, имеют неинверсные выходы. С~одной стороны, это 
упрощает согласование соседних устройств в тракте обработки данных; с~другой стороны, 
создает дополнительную задержку, которая не всегда оправдана.

\begin{figure*} %fig9
             \vspace*{1pt}
 \begin{center}
 \mbox{%
 \epsfxsize=115.162mm
 \epsfbox{sok-9.eps}
 }
 \end{center}
% \vspace*{-9pt}
%\Caption{НЗ-умножитель $4\times 4$}
\end{figure*}

\addtocounter{figure}{1}
       
       В НЗ-методологии элементы библиотеки объединя\-ются в более сложные НЗ-схе\-мы 
в соответствии с дисциплиной формирования и согласования информационных, 
       управ\-ля\-ющих и индикаторных сигналов НЗ-схем:
       \begin{itemize}
\item информационные сигналы формируются с использованием одного из видов 
СС-ко\-ди\-ро\-ва\-ния (парафазного со спейсером, бифазного с управляющим 
сигналом и~т.\,д.). Число вариантов кодирования в пределах одной НЗ-схе\-мы не 
ограничено;
\begin{figure*}[b] %fig10
             \vspace*{6pt}
 \begin{center}
 \mbox{%
 \epsfxsize=145.954mm
 \epsfbox{sok-10.eps}
 }
 \end{center}
 \vspace*{-6pt}
\Caption{Схема NCL-эле\-мен\-та <<неравнозначность>>: без индицируемости входов на 
выходах~(\textit{а}) и с индицируемостью входов~(\textit{б})}
%\end{figure}
%\begin{figure} %fig11
             \vspace*{6pt}
 \begin{center}
 \mbox{%
 \epsfxsize=148.65mm
 \epsfbox{sok-11.eps}
 }
 \end{center}
 \vspace*{-9pt}
\Caption{Функциональная~(\textit{а}) и принципиальная~(\textit{б})
НЗ-схемы <<неравнозначность>> без индикации входов и выходов 
}
\end{figure*}
\item каждый рабочий набор кодированного сигнала в обязательном порядке чередуется со 
специальным самосинхронным промежуточным кодом~--- спейсером. Тип спейсера 
(нулевой или единичный) может быть произвольным, управ\-ля\-ющие сигналы схемы и ее 
окружения подчиняются запрос-от\-вет\-ной дисциплине;
\item все информационные и управляющие сиг\-на\-лы схемы должны индицироваться на ее 
выходах, т.\,е.\ любое переключение каждого сигнала должно в конечном итоге приводить 
к переключе\-нию одного или нескольких выходов \mbox{схемы}.
\end{itemize}
 




       На рис.~9 показана реализация умножителя $4\times4$~\cite{55-sok} в виде 
       НЗ-схе\-мы. Здесь регистр R1R10 используется только на выходе схемы. Основной 
блок SSMULT является комбинационной схемой, так же как и схема МХ22 преобразования 
бифазного сигнала $\{\mathrm{B}[3..0], \mathrm{NB}[3..0]\}$ в парафазный. Остальные 
элементы схемы обеспечивают индицирование схемы умножителя и запрос-от\-вет\-ное 
взаимодействие блоков умножителя с его окружением.


       
       Использование разных типов кодирования сигналов и произвольного типа спейсера 
позволяют в большинстве случаев получить менее сложную реализацию НЗ-схе\-мы с более 
высоким быстродействием.

\begin{figure*} %fig12
             \vspace*{1pt}
 \begin{center}
 \mbox{%
 \epsfxsize=110.505mm
 \epsfbox{sok-12.eps}
 }
 \end{center}
 \vspace*{-9pt}
\Caption{Схема НЗ <<неравнозначность>> с индикацией входов и выходов}
\end{figure*}
       
\vspace*{-3pt}
       \subsection{Сложность реализации}
       
       \vspace*{-2pt}
       
       Сложность реализации NCL-схе\-мы определяется характером индицируемости 
(наблюдаемости~--- observability) и ограниченной номенклатурой схемотехнического базиса. 
На рис.~10 показаны две реализации схемы, выполняющей функцию $\mathrm{Z}\hm= 
\mathrm{X}\oplus \mathrm{Y}$ (<<неравнозначность>>): (а)~без индицируемости входов на 
выходе и (б)~с ин\-ди\-ци\-ру\-емостью~\cite{68-sok}. С~точки зрения булевой алгебры обе эти 
реализации оказываются избыточными даже с учетом парафазного кодирования входов и 
выходов. Сложность их составляет 68 (см.\ рис.~10,\,\textit{а}) и 76 (см.\ рис.~10,\,\textit{б}) 
       КМОП-тран\-зис\-то\-ров в соответствии с табл.~2.
       
       Для сравнения на рис.~11 приведена функциональная схема НЗ-эле\-мен\-та, 
идентичного по выполняемым функциям и степени индицируемости входов и выходов схеме 
на рис.~10,\,\textit{а}, и его реализация на КМОП-тран\-зис\-то\-рах. 
       
       Как видно из рис.~11,\,\textit{б}, сложность адекватного НЗ-эле\-мен\-та составляет 
всего лишь 16~транзисторов. Функциональная схема НЗ-элемента, аналогичного схеме на 
рис.~10,\,\textit{б}, показана на рис.~12. Сложность ее реализации равна 
       40~КМОП-тран\-зис\-то\-рам. Но по сравнению со схемой рис.~10,\,\textit{б} она 
формирует дополнительный выход~I, индицирующий все ее входы и выходы. Если схему на 
рис.~10,\,\textit{б} дополнить аналогичной индикацией (элемент ТН12), то ее сложность 
возрастет до 82~транзисторов. 




       
       Схемы на рис.~11 и~12 имеют парафазные входы и выходы. Спейсеры входов и 
выходов совпадают. Спейсер входов и выходов схемы на рис.~11 может быть любым: 
единичным или нулевым. Спейсер сигналов в схеме на рис.~12 может быть только нулевым. 
Это определяется элементами NOR4 и NOR2, индицирующими входы и выходы схемы. Если 
эти элементы заменить элементами NAND4 и NAND2 соответственно, то тип спейсера 
станет единичным.
       
       Таким образом, НЗ-реализация элемента <<неравнозначность>> оказывается проще 
NCL-реа\-ли\-за\-ции в 2,05--4,25~раза в зависимости от степени индицируемости входов и 
выходов.


       
       На рис.~13 представлена оптимизированная функциональная NCL-схе\-ма полного 
одноразрядного сумматора~\cite{68-sok}. Она реализуется схемой из
 80~транзисторов, в то 
время как НЗ-схе\-ма однораз-\linebreak

\noindent
\begin{center}  %fig13
\mbox{%
 \epsfxsize=68.897mm
 \epsfbox{sok-13.eps}
 }
  \end{center}

  \vspace*{-2pt}

\noindent
{{\figurename~13}\ \ \small{Функциональная NCL-схе\-ма одноразрядного сумматора}}


\noindent
рядного полного сумматора (рис.~14) реализуется на 
40~транзисторах, что в 2~раза меньше, чем в NCL-схе\-ме сумматора. 

 Следует отметить, что НЗ-реа\-ли\-за\-ция сумматора на рис.~14 индицирует на 
своих выходах только рабочие состояния входных сигналов. Практика проектирования 
показала, что этого в ряде случаев достаточно для обеспечения нечувствительности к 
задержкам схемы с сумматором, так как зачастую входы сумматора в полном объеме или в 
фазе спейсера индицируются в другом месте. 

 На рис.~15 показаны соответственно NCL- и НЗ-реа\-ли\-за\-ции полного 
сумматора, имеющие выход~I, индицирующий все входы и выходы схемы в полном объеме. 
Их сравнение показывает, что NCL-реализация содержит 100~транзисторов против~84 у 
       НЗ-реализации. 





\vspace*{6pt}

\begin{center}  %fig14
\mbox{%
 \epsfxsize=77.552mm
 \epsfbox{sok-14.eps}
 }
  \end{center}

  \vspace*{-2pt}

\noindent
{{\figurename~14}\ \ \small{Функциональная НЗ-схе\-ма одноразрядного сумматора}}

%\pagebreak

      
      

\addtocounter{figure}{2}

\end{multicols}

\begin{figure} %fig15
   \vspace*{1pt}
 \begin{center}
 \mbox{%
 \epsfxsize=156.645mm
 \epsfbox{sok-15.eps}
 }
 \end{center}
 \vspace*{-12pt}
\Caption{Схемы сумматора с полной индикацией: (\textit{а})~NCL; (\textit{б})~НЗ}
\vspace*{-6pt}
\end{figure}

\begin{multicols}{2}

      
       

       
      
       
       Таким образом, реализация полного одноразрядного сумматора в виде НЗ-схе\-мы 
оказывается проще NCL-аналога в 1,19--2~раза в зависимости от степени индицируемости 
входов и выходов в самой схеме. 

\begin{figure*} %fig16
   \vspace*{1pt}
 \begin{center}
 \mbox{%
 \epsfxsize=155.444mm
 \epsfbox{sok-17.eps}
 }
 \end{center}
 \vspace*{-6pt}
\Caption{Функциональная схема однотактного NCL-триг\-ге\-ра~(\textit{а}) и 
НЗ-триг\-ге\-ра с парафазными входами и 
выходами~(\textit{б})}
\vspace*{3pt}
\end{figure*}


%\pagebreak






      
Комбинационные схемы в NCL-ба\-зи\-се наиболее избыточны в случае, если не требуется 
индицировать входы на выходах элемента. На практике в комбинационных схемах все или 
часть входных сигналов служат входами сразу нескольких элементов (разветвляются). 
Поэтому гораздо эффективнее индицировать их именно как парафазные входы схемы, а не 
как входы отдельных элементов. Это дает неоспоримое преимущество НЗ-схе\-мам, 
поскольку их элементный базис не избыточен в этом отношении.
       
       Рассмотрим NCL- и НЗ-реа\-ли\-за\-ции последовательных схем. Однотактный 
триггер в NCL-схе\-мах реализуется на двух 2-вхо\-до\-вых С-эле\-мен\-тах со сбросом и 
элементе 2ИЛИ-НЕ в качестве индикато-\linebreak

\vspace*{6pt}


\noindent
\begin{center}  %fig1

\vspace*{-6pt}
\mbox{%
 \epsfxsize=77.713mm
 \epsfbox{sok-19.eps}
 }
  \end{center}

  \vspace*{-3pt}

\noindent
{{\figurename~17}\ \ \small{Функциональная схема однотактного НЗ-триг\-ге\-ра с бифазными входами и 
выходами}}

\vspace*{12pt}

      

\addtocounter{figure}{1}

%\columnbreak


\noindent 
ра, всего 32 транзистора в статическом исполнении
(рис.~16,\,\textit{а}).  Однотактный триггер НЗ с аналогичными свойствами показан на 
рис.~16,\,\textit{б}. Его 
реализация содержит 32~транзистора, как и NCL-аналог.




       Однако в НЗ-схемах обычно используется другой принцип хранения и передачи 
информации между соседними устройствами: бифазные или унарные информационные 
сигналы с сопровождающим их сигналом управления. Поэтому однотактный триггер имеет 
другую схемотехническую реализацию, представленную на рис.~17. У~него бифазные 
информационные входы (R, S), вход асинхронного сброса (Res) и вход управления (Е), 
бифазный информационный выход (Q, QB) и индикаторный выход~(I). Такая реализация 
однотактного триггера требует 24~транзистора~--- в 1,33~раза меньше, чем 
       NCL-реа\-ли\-зация.
       
       \begin{figure*} %fig18
   \vspace*{3pt}
 \begin{center}
 \mbox{%
 \epsfxsize=155.898mm
 \epsfbox{sok-20.eps}
 }
 \end{center}
 \vspace*{-12pt}
\Caption{Функциональная NCL-схе\-ма четырехразрядного счетчика}
\end{figure*}
 \begin{figure*} %fig19
          \vspace*{1pt}
 \begin{center}
 \mbox{%
 \epsfxsize=143.5mm
 \epsfbox{sok-21.eps}
 }
 \end{center}
 \vspace*{-9pt}
       \Caption{Комбинационная часть NCL-счет\-чи\-ка (блок Increment circuitry)}
       \end{figure*}
       
          \begin{figure*} %fig20
      \vspace*{1pt}
 \begin{center}
 \mbox{%
 \epsfxsize=153.254mm
 \epsfbox{sok-22.eps}
 }
 \end{center}
 \vspace*{-9pt}
\Caption{Функциональная схема НЗ-счетчика}
\end{figure*}

       
       Двухтактный NCL-триггер реализуется на двух однотактных триггерах. И~в этом 
случае НЗ-реа\-ли\-за\-ция оказывается в 1,33~раза проще по числу транзисторов. 
Следовательно, регистры хранения и сдвига, реализуемые на однотактных и двухтактных 
триггерах, в НЗ-ис\-пол\-не\-нии будут примерно на треть проще, чем в NCL-ис\-пол\-не\-нии.
      
      Рисунок~18 демонстрирует NCL-схе\-му двоичного счетчика~\cite[рис.~36]{75-sok}. 
Она вынужденно включает в себя комбинационную схему увеличения текущего со\-сто\-яния 
счетчика на~<<1>> и три регистра на однотактных триггерах с асинхронным сбросом.

       
       На рис.~19 показана оптимизированная NCL-реа\-ли\-за\-ция комбинационной части 
счетчика. Отсутствие счетных триг\-ге\-ров в со\-ста\-ве библиотеки NCL-эле\-мен\-тов 
приводит к существенным аппаратным затратам при реализации счетчика. Аналогичный 
четырехразрядный двоичный НЗ-счет\-чик
 показан на рис.~20. В~нем используется счетный 
триггер C0R~\cite{69-sok}, функциональная схема которого показана на рис.~21. 
Сравнительный анализ схем  NCL- и НЗ-счет\-чи\-ков показывает, что по числу транзисторов, 
требующихся для реализации счетчика, НЗ-вариант проще NCL-варианта в 4,49~раза 
(134~транзистора против 602), так как использование триггера C0R в составе счетчика 
исключило необходимость применения регистров для накопления и хранения результата. 
Следовательно, и по энергопотреблению он будет намного эффективнее.


       
      Сравнение реализаций аппаратного однотактного умножителя $4\times 4$ без знака в 
NCL~\cite[рис.~59]{75-sok} и НЗ~\cite{55-sok} базисах также подтверждает преимущество 
НЗ-ва\-ри\-ан\-та, сложность которого составляет\linebreak\vspace*{-12pt}

\pagebreak

 \noindent
\begin{center}  %fig21
\mbox{%
 \epsfxsize=71.899mm
 \epsfbox{sok-23.eps}
 }
  \end{center}

  \vspace*{-3pt}

\noindent
{{\figurename~21}\ \ \small{Функциональная схема НЗ-элемента C0R}}

\vspace*{12pt}

\noindent
 1558~транзисторов, в то время как 
сложность NCL-ва\-ри\-ан\-та равна 1766~транзисторам. %\linebreak



      
Таким образом, проектирование арифметических устройств в НЗ-ба\-зи\-се оказывается 
намного эффективнее, чем в NCL-ба\-зи\-се. Из-за ограниченности функционального 
элементного базиса и типов кодирования информационных сигналов NCL-схе\-мы 
получаются более сложными (четырехразрядный счетчик~--- в 4,49~раза, умножитель 
$4\times 4$ без знака~--- в 1,13~раза), а следовательно, и более энергопотребляющими.
       

     %  \pagebreak

       Схемы NCL имеют неоспоримые преимущества в сравнении с НЗ-схемами: 
    \begin{enumerate}[1.]
       \item При реализации комбинационных схем они не требуют индикации каждого 
элемента схемы. Достаточно проиндицировать только ее последние ярусы, если каждый элемент 
полностью индицирует все свои входы на своих выходах.
       \item Строгое соблюдение парафазной дисциплины с нулевым спейсером (NULL) 
существенно упрощает построение сложных NCL-схем. 
       \item Благодаря использованию единственного способа кодирования 
информационных сигналов и единственного спейсера процесс проектирования NCL-схем 
легче поддается формализации и автоматизации. В~настоящее время уже существует как 
минимум два программных средства синтеза NCL-схем по формальному описанию на 
специальном языке~--- BALSA~\cite{76-sok} и \mbox{UNCLE}~\cite{77-sok}.
       \end{enumerate}
       
       Однако NCL-схемы обладают и существенными недостатками по сравнению с 
       НЗ-схе\-мами:
       \begin{enumerate}[1.]
       \item Индикация входов на выходах в каждом элементе приводит к большой 
избыточности аппаратных затрат.
       \item Ограниченность элементного базиса, использование единственного способа 
кодирования информационных сигналов и единственного спейсера не позволяют получать 
более компактные реализации последовательных схем.
       \item Вследствие аппаратной избыточности и наличия инвертора на выходе 
каждого элемента ухудшается быстродействие и увеличивается энергопотребление.
       \end{enumerate}
       
      

\section{Заключение}
       
       Несмотря на изначально более сложную аппаратную реализацию НЗ-схем по 
сравнению с синхронными аналогами (до 2,1~раза для регистровых структур и до 2,5~раза 
для комбинационных структур), НЗ-схе\-мы обеспечивают более высокое быстродействие 
аппаратуры в реальных условиях. В~ряде случаев они обладают и существенно меньшим 
энергопотреблением. Поэтому применение НЗ-схе\-мо\-тех\-ни\-ки может быть оправдано 
даже в областях, где высокая надежность функционирования не является определяющей, но 
требуется высокое реальное быстродействие или низкое энергопотребление.
       
       Типовые вычислительные устройства, реализованные в базисе НЗ-схем, 
оказываются в~1,5--2~раза лучше своих синхронных аналогов по энергетической 
эффективности (отношению энергии по\-треб\-ле\-ния к производительности) и в~1,7--2,6~раза 
лучше по производительности в реальных условиях. По добротности, учитывающей энергию 
потребления, производительность и допустимые диапазоны напряжения питания и 
температуры окружающей среды, НЗ-схемы оказываются лучше синхронных аналогов в~15--18~раз.
       
       Наиболее предпочтительно применение НЗ-схе\-мо\-тех\-ни\-ки в высоконадежных 
отказоустойчивых системах реального времени. Результаты испытаний отказоустойчивых 
вариантов исполнения ПП-порта показали, что НЗ-ис\-пол\-не\-ние по сравнению с 
синхронной реализацией характеризуется лучшими показателями по всем параметрам: 
в~1,2~раза по быстродействию и по аппаратным затратам, в~1,3~раза по энергетической 
эффективности и в~18~раз по добротности.
       
       Независимо от сложности реализации НЗ-схе\-мы зона ее работоспособности 
определяется физическими характеристиками транзисторов. Она гораздо шире зоны 
работоспособности традиционных С-схем с фиксированной частотой 
синхронизации и превышает аналогичную зону С-схем с адаптивной частотой 
синхронизации.
       
       Маршрут проектирования НЗ-схем поддерживается разработанными в ИПИ РАН 
программными средствами:
       \begin{itemize}
\item синтеза относительно простых НЗ-схем (\mbox{СИНТАБИБ}, СИНКОМБ);
\item анализа разрабатываемой схемы на возможное нарушение принципов построения 
НЗ-схем (АСИАН~\cite{78-sok}, АСПЕКТ~\cite{51-sok}, САМАН, ФАЗАН).
\end{itemize}

       Эти программные средства обеспечивают без\-оши\-бочное проектирование 
       НЗ-устройств и гарантируют принадлежность разрабатываемой схемы к классу 
       НЗ-схем. Программы анализа способны обработать достаточно сложные цифровые устройства, 
например 64-раз\-ряд\-ное АЛУ. 
       
       Результаты практических исследований представителей различных подклассов 
СС-схем подтвердили декларированные теоретически пре\-иму\-щества НЗ-схем 
по зоне ра\-бо\-то\-спо\-соб\-ности, быстродейст\-вию и энергетической эффективности по сравнению 
с синхронными аналогами.
       
       Схемы НЗ, разрабатываемые в соответствии с методологией, продвигаемой ИПИ 
РАН, обладают меньшими аппаратными затратами (в 4,49~раза при реализации двоичного 
счетчика, в 1,13~раза при реализации умножителя $4\times 4$, до 2~раз при\linebreak реализации 
более простых логических схем), большей производительностью и меньшим 
энергопотреблением по сравнению с NCL-схе\-ма\-ми. Поэтому именно их целесообразно 
использовать в \mbox{качестве} схемотехнического базиса для проектирования и изготовления 
       су\-пер-ЭВМ эксафлопсного класса: они обеспечат пониженное энергопотребление и 
высокую надежность проектируемых циф\-ро\-вых устройств любой сложности.

{\small\frenchspacing
{%\baselineskip=10.8pt
\addcontentsline{toc}{section}{References}
\begin{thebibliography}{99}
\bibitem{1-sok}
\Au{Varshavsky V.} Time, timing and clock in massively parallel computing systems~// Conference  
(International) on Massively Parallel Computing Systems Proceedings.~--- Colorado Springs, 
1998. P.~100--106.
\bibitem{2-sok}
\Au{Beerel P., Cortadella J., Kondratyev~A.} Bridging the gap between asynchronous design and 
designers (Tutorial)~// VLSI Design Conference Proceedings.~--- Mumbai, 2004. P.~18--20.
\bibitem{3-sok}
\Au{Muller D., Bartky W.} A~theory of asynchronous circuits~// Annals of Computation 
Laboratory of Harvard University, 1959. Vol.~29. P.~204--243.
\bibitem{4-sok}
\Au{Muller D.\,E.} Asynchronous logics and application to information processing~// Switching 
theory in space technology.~--- Stanford, CA: Stanford University Press, 1963. P.~289--297.
\bibitem{5-sok}
\Au{Seitz C.\,L.} System timing~// Introduction to VLSI Systems.~--- Reading, MA: 
Addison-Wesley, 1980. P.~218--262.
\bibitem{6-sok}
\Au{Singh N.\,P.} A~design methodology for self-timed systems. Master's Thesis. 
MIT/LCS/TR-258.~--- MIT, Laboratory for Computer Science, 1981. 98~p.
\bibitem{7-sok}
\Au{Martin A.\,J.} Compiling communicating processes into delay-insensitive VLSI circuits~// 
Distrib. Comput., 1986. Vol.~1. No.\,4. P.~226--234.
\bibitem{8-sok}
\Au{Anantharaman T.\,S.} A~delay insensitive regular expression recognizer~// IEEE VLSI Techn. 
Bull., 1986. Vol.~1. No.\,2. P.~4.
\bibitem{9-sok}
\Au{Martin A.\,J.} Programming in VLSI~// Development in concurrency and communication.~--- 
Reading, MA: Addison-Wesley, 1990. P.~1--64.
\bibitem{10-sok}
\Au{Van Berkel K.} Beware the isochronic fork~// Integration, VLSI J., 1992. Vol.~13. No.\,2. 
P.~103--128.
\bibitem{11-sok}
\Au{David I., Ginosar R., Yoeli~M.} An efficient implementation of Boolean functions as 
self-timed circuits~// IEEE Trans. Comput., 1992. Vol.~41. No.\,1. P.~2--10.
\bibitem{12-sok}
\Au{Sparso J., Staunstrup J., Dantzer-Sorensen~M.} Design of delay insensitive circuits using 
multi-ring structures~// European Design Automation Conference Proceedings, 1992. 
P.~15--20.

\bibitem{14-sok} %13
\Au{Hauck S.} Asynchronous design methodologies: An overview~//  Proc.\ IEEE, 1995. 
Vol.~83. No.\,1. P.~69--93.

\bibitem{13-sok} %14
\Au{Fant K.\,M., Brandt S.\,A.} NULL convention logic: A complete and consistent logic for 
asynchronous digital circuit synthesis~// Conference (International) on Application Specific 
Systems, Architectures, and Processors Proceedings, 1996. P.~261--273.


\bibitem{15-sok}
\Au{Paver N.\,C., Day P., Farnsworth~C., Jackson~D.\,L., Lien~W.\,A., Liu~J.} A~low-power, 
low-noise, configurable self-timed DSP~// ASYNC'98:  4th Symposium (International) on 
Advanced Research in Asynchronous Circuits and Systems Proceedings, 1998. P.~32--42.
\bibitem{16-sok}
\Au{Laiho M., Vianio O.} A~full-custom self-timed DSP processor implementation~//  European 
Solid-State Circuits Conference Proceedings, 1997. 
{\sf http://www.imec.be/ esscirc/papers-97/172.pdf}.
\bibitem{17-sok}
\Au{Matsubara G., Ide N., Tago~H., Suzuki~S., Goto~N.} \mbox{30-m} 55-b shared Radix 2 Division and 
square root using a self-timed circuit~// ARITH'95:  12th Symposium on Computer Arithmetic 
Proceedings, 1995. P.~98--105.
\bibitem{18-sok}
\Au{Garside J.\,D., Bainbridge W.\,J., Bardsley~A., \textit{et al}.}
 AMULET3i~--- an asynchronous system-on-chip~//  
ASYNC-2000 Proceedings.~--- Eilat, Israil, 2000. P.~162--175.
\bibitem{19-sok}
\Au{Bink A., York R.} ARM996HS: The first licensable, clockless 32-bit processor core~// IEEE 
Micro, 2007. Vol.~27. No.\,2. P.~58--68.
\bibitem{20-sok}
\Au{Martin A.\,J., Nystrom M., Wong~C.\,G.} Three generations of asynchronous 
microprocessors~// IEEE Des. Test Comput., 2003. Vol.~20. No.\,6. P.~9--17.
\bibitem{21-sok}
Handshake Solutions HT80C51 User Manual. {\sf 
http:// www.keil.com/dd/docs/datashts/handshake/ht80c51\_\linebreak um.pdf}.
\bibitem{22-sok}
TIMA Laboratory Annual Report 2006. 2007. {\sf 
http:// tima.imag.fr/publications/files\_reports/ann-rep-06.pdf}.
\bibitem{23-sok}
\Au{Gang J., Lei W., Zhiying~W.} The design of asynchronous microprocessor based on 
optimized NCL\_X design-flow~// IEEE Conference (International) on Networking, 
Architecture and Storage Proceedings, 2009. P.~357--364.
\bibitem{24-sok}
\Au{Ramaswamy S., Rockett L., Patel~D., Danziger~S., Manohar~R., Kelly~C.\,W., Holt~J.\,L., 
Ekanayake~V., Elftmann~D.} A~radiation hardened reconfigurable FPGA~//  IEEE Aerospace 
Conference Proceedings, 2009. P.~1--10.


\bibitem{26-sok} %25
Апериодические автоматы~/ Под ред. В.\,И.~Варшавского.~--- М.: Наука, 1976. 424~c.
\bibitem{27-sok} %26
Автоматное  управление асинхронными процессами в ЭВМ и дискретных сис\-те\-мах~/  Под 
ред. В.\,И.~Варшавского.~---  М.: Наука, 1986. 400~с.
\bibitem{28-sok} %27
\Au{Varshavsky V., Kishinevsky~M., Marakhovsky~V., \textit{et al}.} Self-timed control of 
concurrent processes.~--- Dordrecht, The Netherlands: Kluwer Acad. Publs., 1990. 
245~p.

\bibitem{25-sok} %28
\Au{Kishinevsky M., Kondratyev~A., Taubin~A., Varshavsky~V.}  Concurrent hardware: The 
theory and practice of self-timed design.~--- N.Y.: John Wiley\,\&\,Sons, 1994. 368~p.

\bibitem{29-sok}
\Au{Филин А.\,В., Степченков~Ю.\,А.} Компьютеры без синхронизации~// Системы и 
средства информатики, 1999. Вып.~9. C.~247--261.
\bibitem{30-sok}
\Au{Степченков Ю.\,А., Дьяченко~Ю.\,Г., Петрухин~В.\,С., Филин~А.\,В.} 
Цена реализации уникальных свойств самосинхронных схем~// Системы и средства 
информатики, 1999. Вып.~9. C.~261--292.
\bibitem{31-sok}
\Au{Степченков Ю.\,А., Дьяченко~Ю.\,Г., Петрухин~В.\,С., Филин~А.\,В.} Самосинхронная 
схемотехника~--- альтернатива синхронной~// Электронный сборник научных трудов 
сотрудников \mbox{ОИВТА} РАН. Разд. Элементная база, 1999. 10~с.
{\sf http://samosinhron.ru/ files/articles/native/sss\_alternative\_1999.DOC}. 
\bibitem{32-sok}
\Au{Плеханов Л.\,П., Степченков~Ю.\,А.}  Экспериментальная проверка некоторых свойств 
строго самосинхронных электронных схем~// Системы и средства информатики, 
2006. Вып.~16. С.~476--485.
\bibitem{33-sok}
\Au{Степченков Ю.\,А., Петрухин~В.\,С., Дьяченко~Ю.\,Г.} Опыт разработки 
самосинхронного ядра микроконтроллера на базовом матричном кристалле~// Нано- и 
микросистемная техника, 2006. №\,5. С.~29--36.
\bibitem{34-sok}
\Au{Степченков Ю.\,А., Дьяченко~Ю.\,Г., Петрухин~В.\,С.} Самосинхронные 
последовательностные схемы: опыт разработки и рекомендации по проектированию~// 
Системы и средства информатики, 2007. Вып.~17. С.~503--529.
\bibitem{35-sok}
\Au{Соколов И.\,А., Степченков~Ю.\,А., Петрухин~В.\,С., Дьяченко~Ю.\,Г., Захаров~В.\,Н.} 
Самосинхронная схемотехника~--- перспективный путь реализации аппаратуры~// 
Системы высокой доступности, 2007. Т.~3. №\,1-2. С.~61--72.
\bibitem{36-sok}
\Au{Степченков Ю.\,А., Дьяченко~Ю.\,Г., Петрухин~В.\,С., Плеханов~Л.\,П.} 
Самосинхронные схемы~--- ключ к построению эффективной и надежной аппаратуры 
долговременного действия~// Системы высокой доступности, 2007. Т.~3. №\,1-2. 
С.~73--88.
\bibitem{37-sok}
\Au{Дьяченко Ю.\,Г., Степченков~Ю.\,А., Бобков~С.\,Г.} Квазисамосинхронный 
вычислитель: методологические и алгоритмические аспекты~// Проблемы разработки 
перспективных микро- и наноэлектронных сис\-тем: Мат-лы конф.~--- М.: ИППМ 
РАН, 2008. С.~441--446.
\bibitem{38-sok}
\Au{Stepchenkov Y., Diachenko~Y., Zakharov~V., Rogdestvenski~Y., Morozov~N., 
Stepchenkov~D.} Quasi-delay-insensitive computing device: Methodological aspects and 
practical implementation~// PATMOS'2009: Workshop (International) on Power and Timing 
Modeling, Optimization and Simulation Proceedings.~--- Delft, The Netherlands, 2009. 
P.~276--285.
\bibitem{39-sok}
\Au{Степченков Ю.\,А., Дьяченко~Ю.\,Г., Плеханов~Л.\,П., Гринфельд~Ф.\,И., 
Степченков~Д.\,Ю.} Самосинхронный двухтактный D-триг\-гер с высоким активным 
уровнем сигнала управ\-ле\-ния: Патент РФ №\,2365031~// Офиц. бюлл. <<Изобретения 
(заявки и патенты)>>.~--- М.: ВНИИПИ, 2009. №\,23. 9~с.
\bibitem{40-sok}
\Au{Степченков Ю.\,А., Дьяченко~Ю.\,Г., Рождественскене~А.\,В., Морозов~Н.\,В., 
Петрухин~В.\,С.} Самосинхронный двухтактный D-триг\-гер с низким активным 
уровнем сигнала управления: Патент на изобретение №\,2366080~// Офиц. бюлл. 
<<Изобретения (заявки и патенты)>>.~--- М.: ВНИИПИ,  2009. №\,24. 9~с.
\bibitem{41-sok}
\Au{Дьяченко Ю.\,Г., Степченков~Ю.\,А., Гринфельд~Ф.\,И.}\linebreak Г-триг\-гер с парафазными 
входами с нулевым спейсером: Патент на изобретение №\,2366081~// Офиц. бюлл. 
<<Изобретения (заявки и патенты)>>.~--- М.: ВНИИПИ, 2009. №\,24. 7~с.
\bibitem{42-sok}
\Au{Степченков Ю.\,А., Дьяченко~Ю.\,Г., Плеханов~Л.\,П., Денисов~А.\,Н., 
Филимоненко~О.\,П.} Самосинхронный триггер для связи с удаленным приемником: 
Патент РФ №\,2382487~// Офиц. бюлл. <<Изобретения (заявки и патенты)>>.~--- М.: 
ВНИИПИ, 2010. №\,5. 7~с.
\bibitem{43-sok}
\Au{Степченков Ю.\,А., Дьяченко~Ю.\,Г., Рождественский~Ю.\,Г., Петрухин~В.\,С.} 
Однотактный самосинхронный RS-триг\-гер с предустановкой: Патент №\,2390092~// 
Офиц. бюлл. <<Изобретения (заявки и патенты)>>.~--- М.: ВНИИПИ, 2010. №\,14. 18~с.
\bibitem{44-sok}
\Au{Степченков Ю.\,А., Дьяченко~Ю.\,Г., Захаров~В.\,Н., Гринфельд~Ф.\,И.} Двухтактный 
самосинхронный RS-триг\-гер с предустановкой и входом управления: Патент РФ 
№\,2390093~// Офиц. бюлл. <<Изобретения (заявки и патенты)>>.~--- М.: ВНИИПИ, 
2010. №\,14. 20~с.
\bibitem{45-sok}
\Au{Степченков Ю.\,А., Дьяченко~Ю.\,Г., Степченков~Д.\,Ю., Плеханов~Л.\,П.} 
Двухтактный самосинхронный RS-триг\-гер с предустановкой: Патент РФ №\,2390923~// 
Офиц. бюлл. <<Изобретения (заявки и патенты)>>.~--- М.: ВНИИПИ, 2010. №\,15. 20~с.
\bibitem{46-sok}
\Au{Степченков Ю.\,А., Дьяченко~Ю.\,Г., Морозов~Н.\,В., Филин~А.\,В.} Однотактный 
самосинхронный RS-триг\-гер с предустановкой и входом управления: Патент РФ 
№\,2391772~// Офиц. бюлл. <<Изобретения (заявки и патенты)>>.~--- М.: ВНИИПИ, 
2010. №\,16. 18~с.
\bibitem{47-sok}
\Au{Степченков Ю.\,А., Дьяченко~Ю.\,Г., Плеханов~Л.\,П.} Двоичный самосинхронный 
счетчик с предустановкой: Патент РФ №\,2392735~// Офиц. бюлл. <<Изобретения (заявки 
и патенты)>>.~--- М.: ВНИИПИ, 2010. №\,17. 11~с.
\bibitem{48-sok}
\Au{Соколов И.\,А., Степченков~Ю.\,А., Дьяченко~Ю.\,Г.} Самосинхронный триггер с 
однофазным информационным входом: Патент №\,2405246~// Офиц. бюлл. 
<<Изобретения (заявки и патенты)>>.~--- М.: ВНИИПИ, 2010. №\,33. 32~с.



\bibitem{50-sok} %49
\Au{Степченков Ю.\,А., Дьяченко~Ю.\,Г., Рождественский~Ю.\,В., Морозов~Н.\,В., 
Степченков~Д.\,Ю.} Разработка вычислителя, не зависящего от задержек элементов~// 
Системы и средства информатики, 2010. Вып.~20. №\,1. С.~5--23.

\bibitem{51-sok} %50
\Au{Рождественский Ю.\,В., Морозов~Н.\,В., Рождественскене~А.\,В.} АСПЕКТ: Подсистема 
событийного анализа самосинхронных схем~//  Проблемы разработки перспективных 
микро- и наноэлектронных сис\-тем: IV~Всеросс. науч.-технич. конф. (МЭС-2010): 
Сб. науч. тр.~--- М.: ИППМ РАН, 2010. С.~26--31. 

\bibitem{49-sok} %51
\Au{Степченков Ю.\,А., Дьяченко~Ю.\,Г., Рождественский~Ю.\,В., Морозов~Н.\,В., 
Степченков~Д.\,Ю.} Самосинхронный вычислитель для высоконадежных применений~// 
Проблемы разработки перспективных микро- и наноэлектронных сис\-тем: IV Всеросс. 
науч.-технич. конф. (МЭС-2010): Сб. науч. тр.~--- М.: ИППМ РАН, 2010. 
С.~418--423. 

\bibitem{52-sok} %52
\Au{Плеханов Л.\,П.} Разработка самосинхронных схем: функциональный подход~// 
Проблемы разработки перспективных микро- и наноэлектронных сис\-тем: IV~Всеросс. 
науч.-технич. конф. (МЭС-2010): Сб. науч. тр.~--- М.: ИППМ РАН, 2010. 
С.~424--429.
\bibitem{53-sok}
\Au{Соколов И.\,А., Степченков~Ю.\,А., Дьяченко~Ю.\,Г.} Самосинхронный RS-триг\-гер с 
повышенной помехоустойчивостью (варианты): Патент РФ №\,2427955~// Офиц. бюлл. 
<<Изобретения (заявки и патенты)>>.~--- М.: ВНИИПИ, 2011. №\,24. 42~с.
\bibitem{54-sok}
\Au{Степченков Ю.\,А., Дьяченко~Ю.\,Г., Плеханов~Л.\,П., Петрухин~В.\,С., 
Степченков~Д.\,Ю.} Комбинированный Г-триг\-гер с единичным спейсером: Патент РФ 
№\,2434318~// Офиц. бюлл. <<Изобретения (заявки и патенты)>>.~--- М.: ВНИИПИ, 
2011. №\,32. 10~с.
\bibitem{55-sok}
\Au{Степченков Ю.\,А., Дьяченко~Ю.\,Г., Горелкин~Г.\,А}. Самосинхронные схемы~--- 
будущее микроэлектроники~// Вопросы радиоэлектроники, 2011. Вып.~2. С.~153--184.
\bibitem{56-sok}
\Au{Степченков Ю.\,А., Дьяченко~Ю.\,Г., Рождественский~Ю.\,В., Морозов~Н.\,В.} Анализ 
на самосинхронность некоторых типов цифровых устройств~// Сис\-те\-мы и средства 
информатики, 2011. Вып.~21. №\,1. С.~74--83.
\bibitem{57-sok}
\Au{Плеханов Л.\,П.} Основы самосинхронных электронных схем.~--- М.: Бином, 2013. 
208~с.
\bibitem{58-sok}
IEEE Computer Society. IEEE Standard for Floating-Point Arithmetic IEEE Std 
754-2008. doi:10.1109/ IEEESTD.2008.4610935.


\bibitem{60-sok} %59
\Au{Karthik S., de Souza~I., Rahmeh~J., Abraham~J.} Interlock schemes for micropipelines: 
Application to a self-timed rebound sorter~// Conference (International) on Computer Design 
Proceedings.~--- Cambridge, 1991. P.~393--396.

\bibitem{61-sok} %60
\Au{Liebchen A., Gopalakrishnan~G.} Dynamic reordering of high latency transactions using a 
modified micropipeline~// Conference (International) on Computer Design Proceedings.~--- 
Cambridge, 1992. P.~336--340.

\bibitem{59-sok} %61
\Au{Payne R.} Self-timed FPGA systems~// 5th Workshop (International) on Field Programmable 
Logic and Applications Proceedings.~--- Berlin/Heidelberg: Springer, 1995. P.~21--35.


\bibitem{63-sok} %62
\Au{Sobelman G.\,E., Fant~K.} CMOS circuit design of threshold gates with hysteresis~// 
Symposium (International) on Circuits and Systems Proceedings, 1998. P.~61--64.

\bibitem{66-sok} %63
\Au{Weng N., Yuan~J.\,S., DeMara~R.\,F., Ferguson~D., Hagedorn~M.} Glitch power reduction 
for low power IC design~// 9th Annual NASA Symposium on VLSI Design Proceedings.~--- 
Albuquerque, 2000. P.~7.5.1--7.5.7.

\bibitem{67-sok} %64
\Au{Smith S.\,C.} Completion-completeness for NULL convention digital circuits utilizing the 
bit-wise completion strategy~// Conference (International) on VLSI Proceedings.~--- Las Vegas, 
2003. P.~143--149.

\bibitem{68-sok} %65
\Au{Smith S.\,C., DeMara~R.\,F., Yuan~J.\,S., Ferguson~D., Lamb~D.} Optimization of NULL 
convention self-timed circuits~// Integration, VLSI~J., 2004. Vol.~37. No.\,3. P.~135--165.

\bibitem{62-sok} %66
\Au{Fant K.\,M.} Logically determined design: Clockless system design with NULL convention 
logic.~--- N.Y.: John Wiley\,\&\,Sons, 2005. 292~p.

\bibitem{65-sok} %67
\Au{Smith S.\,C.} Development of a large word-width high-speed asynchronous multiply and 
accumulate unit~// Integration, VLSI~J., 2005. Vol.~39. No.\,1. P.~12--28.

\bibitem{64-sok} %68
\Au{Smith S.\,C., Jia Di.} Designing asynchronous circuits using NULL Convention Logic 
(NCL)~// Synthesis Lectures Digital Circuits Syst., 2009. Vol.~4. No.\,1. P.~61--73.



\bibitem{69-sok} %69
\Au{Степченков Ю.\,А., Денисов~А.\,Н., Дьяченко~Ю.\,Г., Гринфельд~Ф.\,И., 
Филимоненко~О.\,П., Фомин~Ю.\,П.} Библиотека элементов БМК для критических 
областей применения~// Системы и средства информатики, 2004. 
Вып.~14. С.~318--361.
\bibitem{70-sok}
\Au{Степченков Ю.\,А., Денисов~А.\,Н., Дьяченко~Ю.\,Г. и~др.} Библиотека 
самосинхронных элементов для технологии БМК~// Проблемы разработки 
перспективных микроэлектронных систем~--- 2006.~--- М.: ИППМ РАН, 2006. 
С.~259--264.
\bibitem{71-sok}
\Au{Морозов Н.\,В., Степченков~Ю.\,А., Дьяченко~Ю.\,Г., Степченков~Д.\,Ю.} 
Функциональная полузаказная биб\-ли\-о\-тека самосинхронных элементов ML03: Свид. 
№\,2010611908 от 12.03.10.
\bibitem{72-sok}
\Au{Sokolov I.\,A., Stepchenkov~Y.\,A., Dyachenko~Y.\,G.} Self-timed RS-trigger with the 
enhanced noise immunity: U.S. Patent No.\,8232825. 31~p.
\bibitem{73-sok}
Artisan Components. Chartered Semiconductor 0.18~$\mu$m IB Process 1.8-Volt 
SAGE-X$^{\mathrm{TM}}$ Standard Cell Library Databook. Release 1.0. 2003. 313~p.
\bibitem{74-sok}
\Au{Дьяченко Ю.\,Г., Морозов~Н.\,В., Степченков~Д.\,Ю.} Характеризация 
псевдодинамических элементов~// Проблемы разработки перспективных микро- и 
наноэлектронных систем: IV Всеросс. науч.-технич. конф. (МЭС-2010): Сб. 
науч. тр.~--- М.: ИППМ РАН, 2010. С.~32--35.
\bibitem{75-sok}
Gate and throughput optimizations for null convention self timed digital circuits. {\sf 
http://citeseerx.ist.psu.edu/\linebreak viewdoc/download?doi=10.1.1.118.7825\&rep=rep1\&\linebreak type= pdf}.
\bibitem{76-sok}
\Au{Edwards D., Bardsley~A., Jani~L., Plana~L., Toms~W.} Balsa: A~tutorial guide. Version 
V3.5~--- Manchester, 19/5/06. 157~p. 
{\sf ftp://ftp.cs.man.ac.uk/pub/apt/\linebreak balsa/3.5/BalsaManual3.5.pdf}.
\bibitem{77-sok}
\Au{Reese R.\,B.} UNCLE (Unified NCL Environment): Technical Report MSU-ECE-10-001. {\sf 
http://www.ece. msstate.edu/$\sim$reese/uncle/UNCLE.pdf}.
\bibitem{78-sok}
\Au{Рождественский Ю.\,В., Морозов~Н.\,В., Степченков~Ю.\,А., Рождественскене~А.\,В.} 
Универсальная подсистема анализа самосинхронных схем~// Системы и средства 
информатики.~--- М.: Наука, 2006. Вып.~16. С.~463--475.

\end{thebibliography}
} }

\end{multicols}

\vspace*{-6pt}

\hfill{\small\textit{Поступила в редакцию 29.08.13}}


\vspace*{12pt}

\hrule

\vspace*{2pt}

\hrule   
\vspace*{6pt} 

\def\tit{IMPLEMENTATION BASIS OF EXAFLOPS CLASS SUPERCOMPUTER}

\def\titkol{Implementation basis of exaflops class supercomputer}

\def\aut{I.~Sokolov$^1$, Y.~Stepchenkov$^1$, S.~Bobkov$^2$, V.~Zakharov$^1$, Y.~Diachenko$^1$, 
Y.~Rogdestvenski$^1$, and~A.~Surkov$^2$}
\def\autkol{I.~Sokolov et al.}


\titel{\tit}{\aut}{\autkol}{\titkol}

\vspace*{-9pt}

\noindent
$^1$Institute of Informatics Problems, Russian Academy of Sciences,
Moscow 119333, 44-2 Vavilov Str., Russian\\
$\hphantom{^1}$Federation

\noindent
$^2$Scientific Research Institute for System Studies, Russian Academy of Sciences,
36 bld.~1, Nakhimovsky Prosp.,\\
$\hphantom{^1}$Moscow 117218, Russian Federation
 

 
\def\leftfootline{\small{\textbf{\thepage}
\hfill INFORMATIKA I EE PRIMENENIYA~--- INFORMATICS AND APPLICATIONS\ \ \ 2014\ \ \ volume~8\ \ \ issue\ 1}
}%
 \def\rightfootline{\small{INFORMATIKA I EE PRIMENENIYA~--- INFORMATICS AND APPLICATIONS\ \ \ 2014\ \ \ volume~8\ \ \ issue\ 1
\hfill \textbf{\thepage}}}   

\vspace*{6pt}
  
\Abste{The paper deals with choice of a circuitry basis for 
implementation of microprocessors and communication environment of exaflops 
supercomputers. A~comparative analysis of the characteristics of the digital 
circuits with different complexity which are implemented in the synchronous basis as 
well as in the self-timed (ST) one was performed. It has proved the fundamental 
advantages of ST circuits comparing to synchronous analogues: absence of hazards, 
a maximum reachable operability range, high performance, and relatively low
 power 
consumption. Transforming any synchronous circuit into its quasi-ST or ST 
implementation leads to extension of its operability range independently of
 its complexity. The advantages of ST circuits show up to the maximum extent
 when they are used for designing 
 reliable equipment. Various methodologies of ST circuits 
 development are discussed. A~comparative analysis of ST circuit implementation 
 in the generic basis of the delay-insensitive circuits that is
suggested by the authors and 
 in the NULL Convention Logic circuit basis is performed. It is demonstrated that the
 suggested basis makes it possible to synthesize the circuits with the best parameters  of
 performance, complexity, and power consumption while developing standard digital 
 circuits serving as the basis for designing high end computing systems and hardware.}


\KWE{synchronous circuits; self-timed circuits; delay-insensitivity; 
NULL Convention Logic; performance; power consumption; fault tolerance}


\DOI{10.14357/19922264140106}

\vspace*{-12pt}

\Ack
\noindent
This project was financially supported by the Russian Foundation for
Basic Research (projects 13-07-12062~ofi\_m and 
13-07-12068~ofi\_m) and partially supported by the Program of Basic Research of the
RAS Department for Nanotechnologies and Information
Technologies in 2013 (project~1.5).


  \begin{multicols}{2}

\renewcommand{\bibname}{\protect\rmfamily References}
%\renewcommand{\bibname}{\large\protect\rm References}

{\small\frenchspacing
{%\baselineskip=10.8pt
\addcontentsline{toc}{section}{References}
\begin{thebibliography}{99}
\bibitem{1-sok-1}
\Aue{Varshavsky, V.} 
1998. Time, timing and clock in massively parallel computing systems. 
\textit{Conference (International) on Massively Parallel Computing Systems Proceedings}.
Colorado Springs. 100--106.
\bibitem{2-sok-1}
\Aue{Beerel, P., J.~Cortadella, and A.~Kondratyev}. 
2004. Bridging the gap between asynchronous design and designers (Tutorial). 
\textit{VLSI Design Conference Proceedings}. Mumbai. 18--20.
\bibitem{3-sok-1}
\Aue{Muller, D., and W.~Bartky}. 1959. 
A~theory of asynchronous circuits. \textit{Annals of Computation Laboratory of Harvard University}.
29:204--243.
\bibitem{4-sok-1}
\Aue{Muller, D.\,E.} 1963. 
Asynchronous logics and application to information processing.  
\textit{Switching theory in space technology.} 
Stanford, CA: Stanford University Press. 289--297.
\bibitem{5-sok-1}
\Au{Seitz, C.\,L.} 1980. System timing. 
\textit{Introduction to VLSI Systems}. Addison-Wesley. 218--262.
\bibitem{6-sok-1}
\Aue{Singh, N.\,P.} 1981. A~design methodology for self-timed systems. 
Cambridge: MIT Laboratory for Computer Science, MIT. M.Sc. Thesis.  98~p.
\bibitem{7-sok-1}
\Aue{Martin, A.\,J.} 1986. Compiling communicating processes into delay-insensitive 
VLSI circuits. \textit{Distrib. Comput.} 1(4):226--234.
\bibitem{8-sok-1}
\Aue{Anantharaman, T.\,S.} 1986. A~delay insensitive regular expression recognizer. 
\textit{IEEE VLSI Technical Bulletin} 1(2):4. 
\bibitem{9-sok-1}
\Aue{Martin, A.\,J.} 1990. Programming in VLSI. 
\textit{Development in concurrency and communication}. Reading, MA: Addison-Wesley. 1--64.
\bibitem{10-sok-1}
\Aue{Van Berkel, K.} 1992. Beware the isochronic fork. 
\textit{Integration, VLSI J.} 13(2):103--128.
\bibitem{11-sok-1}
\Aue{David, I., R.~Ginosar, and M.~Yoeli}. 1992. 
An efficient implementation of Boolean functions as self-timed circuits. 
\textit{IEEE Trans. Comput.} 41(1):2--11.
\bibitem{12-sok-1}
\Aue{Sparso, J., J.~Staunstrup, and M.~Dantzer-Sorensen}. 1992. 
Design of delay insensitive circuits using multi-ring structures. 
\textit{European Design Automation Conference Proceedings}. Hamburg. 15--20.

\bibitem{14-sok-1}
\Aue{Hauck, S.} 1995. Asynchronous design methodologies: An overview. 
\textit{Proc. IEEE} 83(1):69--93.

\bibitem{13-sok-1}
\Aue{Fant, K.\,M., and S.\,A.~Brandt}. 1996. NULL convention logic: 
A~complete and consistent logic for asynchronous digital circuit synthesis.
\textit{Conference (International) on Application Specific Systems, Architectures, 
and Processors Proceedings}. Chicago. 261--273.

\bibitem{15-sok-1}
\Aue{Paver, N.\,C., P.~Day, C.~Farnsworth, D.\,L.~Jackson, W.\,A.~Lien, and  J.~Liu}. 
1998. A low-power, low-noise, configurable self-timed DSP. 
\textit{4th  Symposium (International) on Advanced Research in Asynchronous 
Circuits and Systems Proceedings}. San-Diego. 32--42.
\bibitem{16-sok-1}
\Aue{Laiho, M., and O. Vianio}. 1997. A~full-custom self-timed DSP processor implementation.
\textit{European Solid-State Circuits Conference Proceedings}. 
Available at: {\sf http://www.imec.be/esscirc/papers-97/172.pdf} 
(accessed August 18, 2013).


\bibitem{18-sok-1} %17
\Aue{Matsubara, G., N.~Ide, H.~Tago, S.~Suzuki, and N.~Goto}. 1995. 
30-ns 55-b shared Radix 2 Division and square root using a self-timed circuit. 
\textit{12th Symposium on Computer Arithmetic Proceedings}. 98--105.

\bibitem{17-sok-1} %18
\Aue{Garside, J.\,D., W.\,J.~Bainbridge, A.~Bardsley, \textit{et al}.} 2000. AMULET3i~--- 
an asynchronous system-on-chip. \textit{6th IEEE  Symposium (International)
on Asynchronous Circuits and Systems Proceedings}. Eilat. 162--175.

\bibitem{19-sok-1}
\Aue{Bink, A., and R. York}. 2007. ARM996HS: The first licensable, clockless 32-bit 
processor core. \textit{IEEE Micro} 27(2):58--68.
\bibitem{20-sok-1}
\Aue{Martin, A.\,J., M.~Nystrom, and C.\,G.~Wong.} 
2003. Three generations of asynchronous microprocessors.
\textit{IEEE Des. Test Comput.} 20(6):9--17.
\bibitem{21-sok-1}
Handshake solutions. HT80C51 User Manual. Available at: 
{\sf http://www.keil.com/dd/docs/datashts/ handshake/ht80c51\_um.pdf} 
(accessed August~27, 2013).
\bibitem{22-sok-1}
TIMA Laboratory Annual Report 2006. 
2007. Available at: 
{\sf http:// tima.imag.fr/publications/files\_reports/ann-rep-06.pdf} (accessed August~18, 2013).
\bibitem{23-sok-1}
\Aue{Jin, G., L. Wang, and Z.~Wang}. 
2009. The design of asynchronous microprocessor based on optimized NCL\_X design-flow. 
\textit{IEEE  Conference (International) on  Networking, Architecture and Storage Proceedings}.
357--364.
\bibitem{24-sok-1}
\Aue{Ramaswamy, S., L. Rockett, D.~Patel, S.~Danziger, R.~Manohar, C.\,W.~Kelly, J.\,L.~Holt, 
V.~Ekanayake, and D.~Elftmann}. 2009. 
A~radiation hardened reconfigurable FPGA. \textit{IEEE Aerospace Conference Proceedings}. 1--10.

\bibitem{26-sok-1} %25
Varshavsky, V.\,I., ed. 1976. 
\textit{Aperiodicheskie avtomaty} [\textit{Aperiodic machines}]. Moscow: Nauka Publ. 424~p.

\bibitem{27-sok-1} %26
Varshavsky, V.\,I., ed. 1986. 
\textit{Avtomatnoe upravlenie asinkhronnymi processami v EVM i diskretnykh sistemakh} 
[\textit{Automata control of concurrent processes in computers and discrete systems}]. 
Moscow: Nauka Publ. 400~p.

\bibitem{28-sok-1} %27
\Aue{Varshavsky, V., M.~Kishinevsky, V.~Marakhovsky, \textit{et al}.} 
1990. \textit{Self-timed control of concurrent processes}. Kluver Acad. Publs. 245~p.

\bibitem{25-sok-1} %28
\Aue{Kishinevsky, M., A.~Kondratyev, A.~Taubin, and V.~Varshavsky}. 
1994. Concurrent hardware: The theory and practice of self-timed design. 
New York: John Wiley\,\&\,Sons. 368~p.

\bibitem{29-sok-1}
\Aue{Filin, A.\,V., and Y.\,A.~Stepchenkov}. 1999. 
Komp'yutery bez sinkhronizatsii [Clockless computers]. 
\textit{Sistemy i Sredstva Informatiki}~--- \textit{Systems and Means of Informatics}
9:247--261.
\bibitem{30-sok-1}
\Aue{Stepchenkov, Y.\,A., Y.\,G.~Diachenko, V.\,S.~Petruhin, and A.\,V.~Filin}. 
1999. Tsena realizatsii unikal'nykh svoystv samosinkhronnykh skhem 
[The penalty of self-timed circuit's unique features implementation]. 
\textit{Sistemy i Sredstva Informatiki}~--- \textit{Systems and Means of Informatics}
9:261--292.
\bibitem{31-sok-1}
\Aue{Stepchenkov, Y.\,A., Y.\,G.~Diachenko, V.\,S.~Petruhin, and A.\,V.~Filin}. 
1999. Samosinkhronnaya skhemotekhnika~--- al'ternativa sinkhronnoy 
[Self-timed circuitry\linebreak as an alternative of synchronous one].  
Available\linebreak at:
{\sf http://samosinhron.ru/files/articles/native/sss\_\linebreak alternative\_1999.DOC} 
(accessed August~18, 2013).
\bibitem{32-sok-1}
\Aue{Plehanov, L.\,P., and Y.\,A.~Stepchenkov}. 2006. 
Ekspe\-ri\-men\-tal'\-naya proverka nekotorykh svoystv strogo samosinkhronnykh 
elektronnykh skhem [Expe\-ri\-mental test of some features of strictly self-timed electronic 
circuits]. \textit{Sistemy i Sredstva Informatiki}~--- \textit{Systems and Means of 
Informatics} 16:476--485.
\bibitem{33-sok-1}
\Aue{Stepchenkov, Y.\,A., V.\,S.~Petruhin, and Y.\,G.~Diachenko}. 2006. 
Opyt razrabotki samosinkhronnogo yadra mikrokontrollera na bazovom matrichnom kristalle 
[The experience in microcontroller's self-timed core design on FPGA]. 
\textit{Nano- i Mikrosistemnaya Tekhnika} [\textit{Nano- and Microsystem Technology}] 5:29--36.
\bibitem{34-sok-1}
\Aue{Stepchenkov, Y.\,A., Y.\,G.~Diachenko, and V.\,S.~Petruhin}. 2007. 
Samosinkhronnye posledovatel'nostnye skhemy: Opyt razrabotki i rekomendatsii 
po proektirovaniyu 
[Self-timed sequential logic: An experience and design guidelines]. 
\textit{Sistemy i Sredstva Informatiki}~--- \textit{Systems and Means of Informatics} 
17:503--529.
\bibitem{35-sok-1}
\Aue{Sokolov, I.\,A., Y.\,A.~Stepchenkov, V.\,S.~Petruhin, Y.\,G.~Diachenko, 
and V.\,N.~Zakharov}. 2007. Samosinkhronnaya skhemotekhnika~--- perspektivnyy 
put' realizatsii apparatury [Timed circuitry~--- perspective method of hardware development]. 
\textit{Sistemy Vysokoy Dostupnosti} [\textit{High Availability Systems}] 3(1-2):61--72.
\bibitem{36-sok-1}
\Aue{Stepchenkov, Y.\,A., Y.\,G.~Diachenko, V.\,S.~Petruhin, and L.\,P.~Plehanov}. 
2007. Samosinkhronnye skhemy~--- klyuch k postroeniyu effektivnoy i nadezhnoy 
apparatury dolgovremennogo deystviya 
[Self-timed circuits are a key for designing the efficient and reliable 
hardware with permanent operation]. \textit{Sistemy Vysokoy Dostupnosti} 
[\textit{High Availability Systems}] 3(1-2):73--88.
\bibitem{37-sok-1}
\Aue{Diachenko, Y.\,G., Y.\,A.~Stepchenkov, and S.\,G.~Bobkov}. 
2008. Kvazisamosinkhronnyy vychislitel': Metodo\-lo\-gi\-che\-skie i algoritmicheskie aspekty 
[Quasi-self-timed\linebreak
 coprocessor: The methodological aspects]. 
\textit{Trudy Mezhdunarodnoy Konferentsii ``Problemy Razrabotki Perspektivnykh 
Mikro- i Nanoelektronnykh Sistem''} [\textit{Problems of the Perspective Micro- and 
Nanoelectronic Systems Development~--- 2008'' Proceedings}]. Moscow. 441--446.
\bibitem{38-sok-1}
\Aue{Stepchenkov, Y., Y.~Diachenko, V.~Zakharov, Y.~Rogdestvenski, N.~Morozov, 
andD.~Stepchenkov}. 2009. 
Quasi-delay-insensitive computing device: Methodological aspects and practical implementation. 
\textit{The Workshop (International) on Power and Timing Modeling, Optimization and 
Simulation Proceedings}. Delft. 276--285.
\bibitem{39-sok-1}
\Aue{Stepchenkov, Y.\,A., Y.\,G.~Diachenko, L.\,P.~Plehanov, F.\,I.~Grinfel'd, 
and D.\,Y.~Stepchenkov}. 2009. Samosinkhronnyy dvukhtaktnyy D-trigger s vysokim aktivnym 
urovnem signala upravleniya [Self-timed D flip-flop with high level control signal]. 
Patent RF No.\,2365031.  \textit{Byulleten' Izobreteniy} 
[\textit{Bulletin of Inventions}] 23. 9~p.
\bibitem{40-sok-1}
\Aue{Stepchenkov, Y.\,A., Y.\,G.~Diachenko, A.\,V.~Rozh\-dest\-ven\-ske\-ne, N.\,V.~Morozov, 
and V.\,S.~Pet\-ru\-hin}. 2009. Samosinkhronnyy dvukhtaktnyy D-trigger s nizkim aktivnym 
urovnem signala upravleniya [Self-timed D flip-flop with low level control signal]. 
Patent RF No.\,2366080. \textit{Byulleten' Izobreteniy} [\textit{Bulletin of Inventions}] 24. 9~p.
\bibitem{41-sok-1}
\Aue{Diachenko, Y.\,G., Y.\,A.~Stepchenkov, and F.\,I.~Grinfel'd}. 
2009. G-trigger s parafaznymi vkhodami s nulevym spey\-se\-rom 
[G-trigger with null spacer dual-rail inputs]. 
Patent RF No.\,2366081. \textit{Byulleten' Izobreteniy} [\textit{Bulletin of Inventions}] 24.
7~p.
\bibitem{42-sok-1}
\Aue{Stepchenkov, Y.\,A., Y.\,G.~Diachenko, L.\,P.~Plehanov, A.\,N.~Denisov,
and O.\,P.~Filimonenko}. 2010. Sa\-mo\-sin\-khronnyy trigger dlya svyazi s udalennym 
priemnikom [Self-timed trigger for connection to remote receiver]. Patent RF No.\,2382487. 
\textit{Byulleten' Izobreteniy} [\textit{Bulletin of Inventions}] 5. 7~p.
\bibitem{43-sok-1}
\Aue{Stepchenkov, Y.\,A., Y.\,G.~Diachenko, Y.\,G.~Rogdestvenski, 
and V.\,S.~Petruhin}. 2010. Odnotaktnyy samosinkhronnyy RS-trigger s predustanovkoy 
[Self-timed RS-latch with preset]. Patent RF No.\,2390092. 
\textit{Byulleten' Izobreteniy} [\textit{Bulletin of Inventions}] 14. 18~p.
\bibitem{44-sok-1}
\Aue{Stepchenkov, Y.\,A., Y.\,G.~Diachenko, V.\,N.~Zakharov, 
and F.\,I.~Grinfel'd}. 2010. Dvukhtaktnyy samosinkhronnyy RS-trigger s predustanovkoy
i vkhodom upravleniya 
[Self-timed RS flip-flop with preset and control input]. Patent RF No.\,2390093. 
\textit{Byulleten' Izobreteniy} [\textit{Bulletin of Inventions}] 14. 20~p.
\bibitem{45-sok-1}
\Aue{Stepchenkov, Y.\,A., Y.\,G.~Diachenko, D.\,Y.~Stepchenkov, and L.\,P.~Plehanov}. 
2010. Dvukhtaktnyy samosinkhronnyy RS-trigger s predustanovkoy
[Self-timed RS flip-flop with preset]. Patent RF No.\,2390923. 
\textit{Byulleten' Izobreteniy} [\textit{Bulletin of Inventions}] 15. 20~p.
\bibitem{46-sok-1}
\Aue{Stepchenkov, Y.\,A., Y.\,G.~Diachenko, N.\,V.~Morozov, and A.\,V.~Filin}. 
2010. Odnotaktnyy samosinkhronnyy RS-trigger s pred\-usta\-nov\-koy i vkhodom upravleniya 
[Self-timed RS-latch with preset and control input]. Patent RF No.\,2391772. 
\textit{Byulleten' Izobreteniy} [\textit{Bulletin of inventions}] 16. 18~p.
\bibitem{47-sok-1}
\Aue{Stepchenkov, Y.\,A., Y.\,G.~Diachenko, and L.\,P.~Plehanov}. 
2010. Dvoichnyy samosinkhronnyy schetchik s pred\-usta\-nov\-koy
[Self-timed binary counter with preset]. Patent RF No.\,2392735. 
\textit{Byulleten' Izobreteniy} [\textit{Bulletin of Inventions}] 17. 11~p.
\bibitem{48-sok-1}
\Aue{Sokolov, I.\,A., Y.\,A.~Stepchenkov, and Y.\,G.~Diachenko}. 
2010. Samosinkhronnyy trigger s odnofaznym in\-for\-ma\-tsi\-on\-nym vkhodom 
[Self-timed trigger with single-phase data input]. Patent RF No.\,2405246. 
\textit{Byulleten' Izobreteniy} [\textit{Bulletin of Inventions}] 33. 32~p.

\bibitem{50-sok-1} %49
\Aue{Stepchenkov, Y.\,A., Y.\,G.~Diachenko, Y.\,V.~Rogdestvenski, N.\,V.~Morozov, 
and D.\,Y.~Stepchenkov}. 2010. Razrabotka vychislitelya, nezavisyashchego ot zaderzhek 
ele\-men\-tov [The design of a cell delay-insensitive coprocessor]. 
\textit{Sistemy i Sredstva Informatiki}~--- \textit{Systems and Means of Informatics} 
20:5--23.

\bibitem{51-sok-1} %50
\Aue{Rogdestvenski, Y.\,V., N.\,V.~Morozov, and A.\,V.~Rozhdestvenskene}. 
2010. ASPEKT: Podsistema sobytijnogo analiza samosinkhronnyh skhem 
[ASPECT: A~suite of self-timed event-driven analysis].  
\textit{Trudy Mezhdunarodnoj Konferentsii 
``Problemy Razrabotki Perspektivnykh Mikro- i Nanoelektronnykh Sistem''} 
[\textit{``Problems of the Perspective Micro- and Nanoelectronic Systems Development~--- 2010''
Proceedings}]. Moscow. 26--31. 

\bibitem{49-sok-1} %51
\Aue{Stepchenkov, Y.\,A., Y.\,G.~Diachenko, Y.\,V.~Rogdestvenski, N.\,V.~Morozov, 
and D.\,Y.~Stepchenkov}. 2010. Samosinkhronnyy vychislitel' dlya vysokonadezhnykh primeneniy 
[Self-timed coprocessor for high-reliable applications]. 
\textit{Trudy Mezhdunarodnoy Konferentsii ``Problemy Razrabotki Perspektivnykh Mikro- i 
Nanoelektronnykh Sistem''} [\textit{``Problems of the Perspective Micro- and Nanoelectronic 
Systems Development~--- 2010'' Proceedings}]. Moscow. 418--423. 

\bibitem{52-sok-1} 
\Aue{Plehanov, L.\,P.} 2010. Razrabotka samosinkhronnykh skhem: Funktsional'nyy podkhod 
[Self-timed circuits design: A~functional approach]. 
\textit{Trudy Mezhdunarodnoy Konferentsii ``Problemy Razrabotki Perspektivnykh Mikro- i 
Nanoelektronnykh Sistem''} [\textit{``Problems of the Perspective Micro- and 
Nanoelectronic Systems Development~--- 2010" Proceedings}]. Moscow. 424--429.
\bibitem{53-sok-1}
\Aue{Sokolov, I.\,A., Y.\,A.~Stepchenkov, and Y.\,G.~Diachenko}. 2011. 
Samosinkhronnyy RS-trigger s povyshennoy pomekhoustoychivost'yu (varianty) 
[Self-timed RS-trigger with the enhanced noise immunity]. Patent RF No.\,2427955. 
\textit{Byulleten' Izobreteniy} [\textit{Bulletin of Inventions}] 24. 42~p.
\bibitem{54-sok-1}
\Aue{Stepchenkov, Y.\,A., Y.\,G.~Diachenko, L.\,P.~Plehanov, V.\,S.~Petruhin, 
and D.\,Y.~Stepchenkov}. 2011. Kombinirovannyy G-trigger s edinichnym spey\-se\-rom 
[Composite G-trigger with the unit spacer]. Patent RF No.\,2434318. 
\textit{Byulleten' Izobreteniy} [\textit{Bulletin of Inventions}] 32. 10~p.
\bibitem{55-sok-1}
\Aue{Stepchenkov, Y.\,A., Y.\,G.~Diachenko, and G.\,A.~Gorelkin}. 2011. 
Samosinkhronnye skhemy~--- budushchee mikroelektroniki 
[Self-timed circuits are the future of microelectronics]. 
\textit{Voprosy Radioelektroniki} [\textit{The Problems of Radio Electronics}] 2:153--184.
\bibitem{56-sok-1}
\Aue{Stepchenkov, Y.\,A., Y.\,G.~Diachenko, Y.\,V.~Rogdestvenski, and N.\,V.~Morozov}. 
2011. Analiz na samosinkhronnost' nekotorykh tipov tsifrovykh ustroystv 
[Self-timed analysis of the few types of the digital units]. 
\textit{Sistemy i Sredstva Informatiki}~--- \textit{Systems and Means of Informatics}
 21(1):74--83.
\bibitem{57-sok-1}
\Aue{Plehanov, L.\,P.} 2013. \textit{Osnovy samosinkhronnykh elektronnykh skhem} 
[\textit{The base of the self-timed electronic circuits}]. Moscow: Binom Publ. 208~p.
\bibitem{58-sok-1}
IEEE Computer Society. 2008. IEEE Standard for Floating-Point Arithmetic IEEE Std 754-2008. 
doi:10.1109/IEEESTD.2008.4610935. 


\bibitem{60-sok-1} %59
\Aue{Karthik, S., I.~de~Souza, J.~Rahmeh, and J.~Abraham}. 1991. 
Interlock schemes for micropipelines: Application to a self-timed rebound sorter. 
\textit{Conference (International) on Computer Design Proceedings}.  Cambridge. 393--396.

\bibitem{61-sok-1} %60
\Aue{Liebchen, A., and G.~Gopalakrishnan}. 1992. Dynamic reordering of high 
latency transactions using a modified micropipeline. 
\textit{Conference (International) on Computer Design Proceedings}. Cambridge. 336--340.

\bibitem{59-sok-1} %61
\Aue{Payne, R.} 1995. Self-timed FPGA systems. 
\textit{5th Workshop (International) on Field Programmable Logic and Applications Proceedings}.
Berlin/Heidelberg. 21--35.

\bibitem{63-sok-1} %62
\Aue{Sobelman, G.\,E., and K.~Fant}. 1998. 
CMOS circuit design of threshold gates with hysteresis. 
\textit{Symposium (International) on Circuits and Systems Proceedings}. 61--64.

\bibitem{66-sok-1} %63
\Aue{Weng, N., J.\,S.~Yuan, R.\,F.~DeMara, D.~Ferguson, and M.~Hagedorn}. 
2000. Glitch power reduction for low power IC design. 
\textit{9th Annual NASA Symposium on VLSI Design Proceedings}. Albuquerque. 7.5.1--7.5.7.

\bibitem{67-sok-1} %64
\Aue{Smith S.\,C.} 2003.
Completion-completeness for NULL convention digital circuits utilizing the 
bit-wise completion strategy. \textit{Conference (International) on VLSI Proceedings}.
 Las Vegas. 143--149.

\bibitem{68-sok-1} %65
\Aue{Smith, S.\,C., R.\,F.~DeMara, J.\,S.~Yuan, D.~Ferguson, and D.~Lamb}. 
2004. Optimization of NULL convention self-timed circuits. 
\textit{Integration, VLSI~J.} 37(3):135--165.

\bibitem{62-sok-1} %66
\Aue{Fant, K.\,M.} 2005. Logically determined design: 
Clockless system design with NULL convention logic. New York: John Wiley\,\&\,Sons. 292~p.

\bibitem{65-sok-1} %67
\Aue{Smith, S.\,C.} 2005. {Development of a large word-width high-speed asynchronous 
multiply and accumulate unit}. \textit{Integration, VLSI~J.} 39(1):12--28.

\bibitem{64-sok-1} %68
\Aue{Smith, S.\,C., and J.~Di.} 2009. 
Designing asynchronous circuits using NULL Convention Logic (NCL). 
\textit{Synthesis Lectures Digital Circuits  Syst.} 4(1):61--73.


\bibitem{69-sok-1} %69
\Aue{Stepchenkov, Y.\,A., A.\,N.~Denisov, Y.\,G.~Diachenko, F.\,I.~Grinfel'd, 
O.\,P.~Filimonenko, and Y.\,P.~Fomin}. 2004. Biblioteka elementov BMK dlya kriticheskikh 
oblastey primeneniya [The gate array cell library for critical applications]. 
\textit{Sistemy i Sredstva Informatiki}~--- \textit{Systems and Means of Informatics}
14:318--361.
\bibitem{70-sok-1}
\Aue{Stepchenkov, Y.\,A., A.\,N.~Denisov, Y.\,G.~Diachenko, \textit{et~al.}} 
2006. Biblioteka samosinkhronnykh elementov dlya tekhnologii BMK 
[Self-timed cell library for gate array technology]. 
\textit{Trudy Mezhdunarodnoy Konferentsii ``Problemy Razrabotki Perspektivnykh Mikro- i 
Nanoelektronnykh Sistem''} [\textit{``Problems of the Perspective Micro- and 
Nanoelectronic Systems Development~--- 2006'' Proceedings}]. Moscow. 259--264.
\bibitem{71-sok-1}
\Aue{Morozov, N.\,V., Y.\,A.~Stepchenkov, Y.\,G.~Diachenko, and D.\,Y.~Stepchenkov}. 
2010. Funktsional'naya poluza\-kaz\-naya biblioteka samosinkhronnykh elementov ML03 
[The functional semicustom library of the self-timed cells]. 
Certificate on official registration of the computer program No.\,2010611908. 
(In Russian, unpublished.)
\bibitem{72-sok-1}
\Aue{Sokolov, I.\,A., Y.\,A.~Stepchenkov, and Y.\,G.~Dyachenko}. 2010. 
Self-timed RS-trigger with the enhanced noise immunity. 
U.S.\ Patent No.\,8232825. 31~p.
\bibitem{73-sok-1}
Artisan Components.  
Chartered Semiconductor 0.18~$\mu$m IB Process 1.8-Volt SAGE-XTM Standard Cell 
Library Databook. 2003. Release~1.0. 313~p.
\bibitem{74-sok-1}
\Aue{Diachenko, Y.\,G., N.\,V.~Morozov, and D.\,Y.~Stepchenkov}. 2010. 
Kharakterizatsiya psevdodinamicheskikh elementov 
[The characterization of the pseudodynamic cells]. 
\textit{Trudy Mezhdunarodnoy Konferentsii ``Problemy Razrabotki Perspektivnykh Mikro- 
i Nanoelektronnykh sistem''} 
[\textit{``Problems of the Perspective Micro- and Nanoelectronic Systems 
Development~--- 2010'' Proceedings}]. Moscow. 32--35.
\bibitem{75-sok-1}
Gate and throughput optimizations for null convention self timed digital circuits. 
Available at: 
{\sf http://citeseerx. ist.psu.edu/viewdoc/download?doi=10.1.1.118.7825\&\linebreak rep=rep1\&type=pdf}
(accessed August 18, 2013).
\bibitem{76-sok-1}
\Aue{Edwards, D., A.~Bardsley, L.~Jani, L.~Plana, and W.~Toms}. 2006. 
Balsa: A~tutorial guide. Manchester. 157~p. Available at: 
{\sf ftp://ftp.cs.man.ac.uk/pub/apt/ balsa/3.5/BalsaManual3.5.pdf}  (accessed August~18, 2013).
\bibitem{77-sok-1}
\Aue{Reese, R.\,B.} 2011. UNCLE (Unified NCL Environment). 
Technical Report MSU-ECE-10-001. Available at: 
{\sf http://www.ece.msstate.edu/$\sim$reese/uncle/UNCLE.pdf} (accessed August 18, 2013).
\bibitem{78-sok-1}
\Aue{Rogdestvenski, Y.\,V., N.\,V.~Morozov, Y.\,A.~Stepchenkov, and A.\,V.~Rozhdestvenskene}. 
2006. Universal'naya podsistema analiza samosinkhronnykh skhem 
[The universal suite for self-timed circuit analysis]. 
\textit{Sistemy i Sredstva Informatiki}~--- \textit{Systems and Means of Informatics} 
16:463--475.
{\looseness=1

}


\end{thebibliography}
} }


\end{multicols}

\vspace*{-6pt}

\hfill{\small\textit{Received August 29, 2013}}

\vspace*{-18pt}

\Contr


\noindent
\textbf{Sokolov Igor A.} (b.\ 1954)~--- Academician of the Russian Academy of Sciences,  
Doctor of Science in technology, Director,  
Institute of Informatics Problems, Russian Academy of Sciences, 44-2 Vavilov Str., 
Moscow 119333, Russian Federation;  ISokolov@ipiran.ru



\vspace*{2pt}

\noindent
\textbf{Stepchenkov Yuri A.} (b.\ 1951)~--- Candidate of Sciences (PhD) in technology, 
Head of Department, Institute of Informatics Problems, Russian Academy of Sciences,
44-2 Vavilov Str., Moscow 119333, Russian Federation;    YStepchenkov@ipiran.ru

\vspace*{2pt}


\noindent
\textbf{Bobkov Sergey G.} (b.\ 1951)~--- Doctor of Science in technology, Head of Department, 
Scientific Research Institute for System Studies, Russian Academy of Sciences,
36 bld.~1, Nakhimovsky Prosp.,
Moscow 117218, Russian Federation;  bobkov@cs.niisi.ras.ru

\vspace*{2pt}

\noindent
\textbf{Zakharov Victor N.}  (b.\ 1948)~--- Doctor of Science (PhD) in technology, 
associate professor; Scientific Secretary, Institute of Informatics Problems, 
Russian Academy of Sciences, 44-2 Vavilov Str.,
Moscow 119333, Russian Federation;  VZakharov@ipiran.ru 

\vspace*{2pt}


\noindent
\textbf{Diachenko Yuri G.} (b.\ 1958)~--- Candidate of  Sciences (PhD) in technology, senior 
scientist, Institute of Informatics Problems, 
Russian Academy of Sciences, 44-2 Vavilov Str.,
Moscow 119333, Russian Federation; diaura@mail.ru

\vspace*{2pt}

\noindent
\textbf{Rogdestvenski Yuri V.} (b.\ 1952)~--- Candidate of  Sciences (PhD) in 
technology, Head of Laboratory, Institute of Informatics Problems, 
Russian Academy of Sciences, 44-2 Vavilov Str.,
Moscow 119333, Russian Federation; YRogdest@ipiran.ru

\vspace*{2pt}

\noindent
\textbf{Surkov Alexei V.} (b.\ 1978)~---  senior scientist, 
Scientific Research Institute for System Studies, Russian Academy of Sciences,
36 bld.~1, Nakhimovsky Prosp.,
Moscow 117218, Russian Federation;  surkov@cs.niisi.ras.ru


 \label{end\stat}
 
\renewcommand{\bibname}{\protect\rm Литература}