

\def\stat{sinits}

\def\tit{АНАЛИЗ И МОДЕЛИРОВАНИЕ РАСПРЕДЕЛЕНИЙ В~ЭРЕДИТАРНЫХ  СТОХАСТИЧЕСКИХ СИСТЕМАХ$^*$}

\def\titkol{Анализ и моделирование распределений в эредитарных  стохастических системах}

\def\autkol{И.\,Н.~Синицын}

\def\aut{И.\,Н.~Синицын$^1$}

\titel{\tit}{\aut}{\autkol}{\titkol}

{\renewcommand{\thefootnote}{\fnsymbol{footnote}} 
\footnotetext[1]{Работа выполнена при финансовой поддержке Программы ОНИТ РАН 
<<Интеллектуальные информационные технологии, системный анализ и 
автоматизация>> (проект~1.7).}}

\renewcommand{\thefootnote}{\arabic{footnote}}
\footnotetext[1]{Институт проблем информатики Российской академии наук, sinitsin@dol.ru} 

\vspace*{12pt}

\Abst{Рассматриваются методы и алгоритмы анализа и моделирования 
(аналитического, статистического) одно- и многомерных распределений 
в эредитарных стохастических системах (ЭСтС) с винеровскими и пуассоновскими шумами. 
Приводятся нелинейные стохастические интегродифференциальные уравнения. 
Для затухающих физически возможных эредитарных ядер рассматриваются два способа их 
аппроксимации (на основе линейных операторных уравнений и вырожденных ядер). 
Устанавливаются алгоритмы приведения ЭСтС к дифференциальным стохастическим системам 
(ДСтС).
Приводится подробный анализ подходов к аналитическому и статистическому моделированию 
распределений в ЭСтС, приводимых к ДСтС. В~основу подходов положены как методы прямого 
численного интегрирования уравнений ДСтС, так и численного интегрирования для параметров 
ортогонального разложения плотностей (моментов, коэффициентов ортогонального разложения и~др.).
Подробно рассматриваются алгоритмы аналитического и статистического моделирования, основанные 
на методе статистической линеаризации (МСЛ) и методе нормальной аппроксимации (МНА). Получены 
условия устойчивости  алгоритмов на основе МСЛ и МНА.
Для задач МСЛ рассматриваются прямые одношаговые сильные методы и алгоритмы численного 
интегрирования (различной точности) для гладких и разрывных правых частей уравнений ЭСтС.
Разработан комплекс тестовых примеров для разрабатываемого в ИПИ РАН 
инструментального программного обеспечения ``IDStS'' в среде MATLAB. Подробно рассмотрены 
задачи анализа и моделирования колебаний осциллятора Дуффинга и релейного осциллятора в 
эредитарной стохастической среде.}

\vspace*{2pt}

\KW{аналитическое моделирование; вырожденное (сингулярное) эредитарное ядро;
дифференциальная сис\-те\-ма; интегродифференциальная сис\-те\-ма;
параметризация распределений; система, приводимая к дифференциальной;
стохастическая сис\-те\-ма; эредитарная сис\-те\-ма}

\vspace*{2pt}

\DOI{10.14357/19922264140101}

%\vspace*{24pt}

\vskip 18pt plus 9pt minus 6pt

      \thispagestyle{headings}

      \begin{multicols}{2}

            \label{st\stat}   

\section{Введение}

Как известно~[1--3], интегродифференциальные стохастические системы (СтС) 
являются подходящими математическими моделями так называемых эредитарных СтС. 
Процессы в ЭСтС, как правило, являются немарковскими.

В прикладных задачах для затухающей эредитарной памяти путем аппроксимации ядер, 
опре-\linebreak деляющих память линейными обыкновенными дифференциальными уравнениями или 
вырож-\linebreak денными ядрами, удается привести интегродиф-\linebreak ференциальные 
уравнения к дифференциаль-\linebreak ным. 
В~результате процессы в таких ДСтС становятся марковскими. 
Поэтому, как показано в~[2--8], оказывается возможным использовать богатый современный 
арсенал методов и средств аналитического и статистического моделирования.
%
Вопросам  моделирования распределений  в ДСтС (см.\ библиографические замечания в~[4--8]) 
посвящена обширная ли\-те\-ра\-тура.
{\looseness=1

}

Статья посвящена вопросам анализа и моделирования одно- и многомерных распределений 
в негауссовских ЭСтС, описываемых интегродифференциальными стохастическими уравнениями 
Ито с винеровскими и пуассоновскими шумами в конечномерных пространствах.


\section{Уравнения эредитарных стохастических систем}

Рассмотрим ЭСтС, описываемую интегродифференциальным уравнением Ито следующего вида:

\noindent
\begin{multline}
d X =\lk a (X,t) +\iii_{t_0}^t a_1 (X(\tau),\tau, t)\, d \tau\rk\, dt+{}\\
{}+\lk b(X, t) +\iii_{t_0}^t b_1 (X(\tau), \tau, t)\, d \tau\rk dW_0+{}\\
\hspace*{-4mm}  {}+\iii_{R_0^q} \lk c(X,t,v) +\iii_{t_0}^t c_1 (X(\tau),  \tau, t,v) \rk d P^0 (t, dv)\!
  \label{e2.1-s}
  \end{multline}
с начальным условием $X(t_0)= X_0$.

В~(\ref{e2.1-s}) приняты следующие обозначения и допущения:
\begin{itemize}
\item    $X=X(t)$~--- $p$-мер\-ный вектор состояния;
\item
    $W_0$~--- $r$-мер\-ный винеровский процесс интенсивности $\nu_0 = \nu_0 (t)$;
\item
    $ \iii_{\Delta_t} d P^0 (t, A)$~--- центрированная пуассоновская мера, 
    удовлетворяющая условию
$$
\iii_{\Delta_t} d P^0 (t, A)=\iii_{\Delta_t} dP (t, A)-\iii_{\Delta_t} \nu_P (t, A) \,dt\,, 
$$
где $\iii_{\Delta_t}\,dP(t, A)$~--- число скачков пуассоновского процесса в интервале~$\Delta_t$;
\item
    $\nu_P (t,A)$~--- интенсивность пуассоновского процесса $P(t, A)$;
\item
    $v$~--- $q$-мер\-ный вспомогательный параметр в пространстве~$R_0^q$ с выколотым началом;
\item
    $A$~--- некоторое борелевское множество пространства~$R_0^q$.
\end{itemize}

Функции $a\hm=a(X, t)$, $a_1 \hm= a_1(X (\tau),\tau, t)$, $b\hm=b(X, t)$, 
$b_1 \hm= b_1(X (\tau),\tau, t)$, $c\hm=c(X,t,v)$ и $c_1\hm = c_1(X (\tau),\tau, t,v)$ 
имеют размерности $p\times 1$, $p\times 1$, $p\times r$, $p\times r$, 
$p\times 1$ и $p\times 1$ и допускают представления вида:
\begin{equation}
\left.
\begin{array}{rl}
    a_1&=A(t,\tau) \vrp (X(\tau), \tau)\,;\\[9pt] 
    b_1&=B(t,\tau) \psi (X(\tau) ,  \tau)\,;\\[9pt]
    c_1&=C(t,\tau) \chi (X(\tau) ,  \tau, v)\,.
    \end{array}
    \right\}
    \label{e2.2-s}
    \end{equation}
Здесь эредитарные ядра $A(t,\tau)\hm=\lk A_{ij}(t,\tau)\rk$ $(i,j\hm=\overline{1,p})$,
$B(t,\tau)\hm=\lk B_{i l}(t,\tau)\rk$ $(i\hm=\overline{1,p}$, $l\hm=\overline{1,r})$ и
$C(t,\tau)\hm=\lk C_{ij}(t,\tau)\rk$ $(i,j\hm=\overline{1,p})$ имеют соответственно 
размерности
$p\times p$, $p\times r$ и $p\times p$ и удовлетворяют следующим условиям 
физической реализуемости и асимптотического затухания:
\begin{equation}
\left.
\begin{array}{c}
A_{ij}(t,\tau)=0\,,\enskip B_{i l}(t,\tau)=0\,;\\[9pt] 
C_{ij}(t,\tau)=0\enskip \forall \tau >t\,;
\end{array}
\right\}\label{e2.3-s}
\end{equation}
\begin{equation}
\left.
\begin{array}{c}
\displaystyle\iin \lv A_{ij} (t,\tau) \rv d\tau <\infty \,;\\[9pt]
\displaystyle\iin \lv B_{i l} (t,\tau) \rv d\tau <\infty \,;\\[9pt] 
\displaystyle\iin \lv C_{ij} (t,\tau) \rv d\tau <\infty\,.
\end{array}
\right\}\label{e2.4-s}
\end{equation}

Нелинейные в общем случае функции $\vrp\hm=\vrp(X(\tau),\tau)$,
$\psi \hm=\psi(X(\tau), \tau)$ и $\chi \hm=\chi (X(\tau),  \tau, v)$ 
отражают нелинейные свойства ЭСтС, зависят от $X(\tau)$, $\tau$ и имеют размерности 
$p\times 1$, $p\times p$ и $p\times 1$ соответственно.

В случае, если эредитарные ядра удовлетворяют условиям
\begin{equation}
\left.
\begin{array}{c}
A_{ij} (t,\tau) =\tilde A_{ij} (u)\,;\enskip 
B_{i l} (t,\tau) =\tilde B_{i l} (u)\,;\\[9pt]  
C_{ij} (t,\tau) =\tilde C_{ij} (u)\enskip (u=t-\tau)\,,
\label{e2.5-s}
\end{array}
\right\}
\end{equation}
то говорят об ЭСтС со стационарным затуханием памяти.

Важный класс ядер представляют собой вы\-рож\-ден\-ные (сингулярные) ядра, 
когда имеют место представления
    \begin{equation}
    \left.
    \begin{array}{rl}
    A_{ij} (t,\tau) &= A_{ij}^+(t) A_{ij}^-(\tau)\,;\\[9pt] 
    B_{i l} (t,\tau) &=      B_{il}^+(t) B_{il}^-(\tau)\,;\\[9pt]  
  \!  \!C_{ij} (t,\tau)& =   C_{ij}^+ ( t) C_{ij}^- (\tau)\enskip  %\\[9pt]
    (i,l= \overline{1,p}\,,\  j\hm=\overline{1,r})\!\!
    \end{array}
    \right\}
    \label{e2.6-s}
    \end{equation}



\noindent
\textbf{Замечание 2.1.}\
В случае, когда подынтегральные функции  $c(X, t, v)$ и  $c_1(X(\tau), \tau, v)$ 
в~(\ref{e2.1-s}) допускают представления
    \begin{align*}
        c(X,t, v)&=b(X, t)c'(v)\,;\\ 
    c_1(X(\tau), \tau, v)&=b(X(\tau),\tau)c'(v)\,,
%\label{e2.7-s}
    \end{align*}
ЭСтС~(\ref{e2.1-s}) приводится к виду:
\begin{multline}
    \dot X =  a(X, t)+\iii_{t_0}^t a_1 (X(\tau),\tau, t)\,d\tau 
     +{}\\
     {}+\lk b(X, t)+ \iii_{t_0}^t b_1 (X(\tau),\tau, t)\,d\tau\rk V\,,\label{e2.8-s}
     \end{multline}
если принять
    $$
    V=\dot W\,;\enskip W(t) = W_0(t) +\iii_{R_0^q} c' (v) P^0 (t, dv)\,.
    $$

\noindent
\textbf{Замечание 2.2.}\
Путем введения блочных матриц рассматриваются случаи, обобщающие 
соответственно представления~(\ref{e2.2-s}) и~(\ref{e2.6-s})
\begin{align*}
a_1 &=\sss_{k=1}^N w_k' (t,\tau) \vrp_k' (X(\tau), \tau)\,;\\
    b_1 &=\sss_{k=1}^N w_k'' (t,\tau) \psi_k' (X(\tau), \tau)\,;
\\
c_1 &=\sss_{k=1}^N w_k''' (t,\tau) \chi_k' (X(\tau), \tau,v)\\
%\end{align*}
\intertext{и}
%\begin{align*}
%\\
%\label{e2.9-s}
 a_1 &=\sss_{k=1}^N \xi_k' (t) \vrp_k' (X(\tau),\tau)\,;\\
     b_1 &=\sss_{k=1}^N \xi_k'' (t) \psi_k' (X(\tau), \tau)\,;\\
c_1 &=\sss_{k=1}^N \xi_k''' (t,\tau) \chi_k' (X(\tau), \tau,v)
%\label{e2.10-s}
\end{align*}

\noindent
\textbf{Замечание 2.3.} Очевидно, что случай ЭСтС, описываемых по одной
части переменных состояния системы стохастическим интегральным
уравнением, а второй части~--- стохастическим дифференциальным
уравнением, является частным случаем~(\ref{e2.1-s}) и~(\ref{e2.8-s}).

Решим сначала вспомогательную задачу приведения ЭСтС~(\ref{e2.1-s}) к ДСтС, 
предполагая выполненными условия~(\ref{e2.2-s})--(\ref{e2.4-s}) и~(\ref{e2.2-s}), 
(\ref{e2.3-s}) и~(\ref{e2.5-s}) 
соответственно. Для ЭСтС~(\ref{e2.8-s}) задача рассматривалась в~[2--6].

\vspace*{-2pt}

\section{Аппроксимация эредитарных ядер }

Рассмотрим ЭСтС~(\ref{e2.1-s}) при условиях~(\ref{e2.2-s})--(\ref{e2.4-s}). 
Будем считать, что эредитарные ядра  $A(t,\tau)$, $B(t,\tau)$ и $C(t,\tau)$ 
удовлетворяют следующим линейным операторным уравнениям:
\begin{align*}
F^{At}A(t,\tau) &= H^{At} \delta (t-\tau)\,;\\
F^{Bt}B(t,\tau) &= H^{Bt} \delta (t-\tau)\,;\\ 
F^{Ct}C(t,\tau) &= H^{Ct} \delta (t-\tau)\,;
%\label{e3.1-s}
\\
A(t,\tau)&= A'(t,\tau)^{\mathrm{T}} (H^{A*\tau})^{\mathrm{T}};\\
A'(t,\tau)^{\mathrm{T}} (F^{A*\tau})^{\mathrm{T}}&= I_h^A\delta(t-\tau)\,,\\
B(t,\tau)&= B'(t,\tau)^{\mathrm{T}} (H^{B*\tau})^{\mathrm{T}}\,,\\
B'(t,\tau)^{\mathrm{T}} (F^{B*\tau})^{\mathrm{T}}&= I_h^B\delta(t-\tau)\,;\\
C(t,\tau)&= C'(t,\tau)^{\mathrm{T}} (H^{C*\tau})^{\mathrm{T}}\,;\\
C'(t,\tau)^{\mathrm{T}} (F^{C*\tau})^{\mathrm{T}}&= I_h^C\delta(t-\tau)\,.
%    \label{e3.2-s}
    \end{align*}
Здесь

\noindent
\begin{equation}
\left.
\begin{array}{c}
F^A = F^A (t,D)=\displaystyle\sss_{l=0}^{n_A} \alp_l^A (t) D^l\,;\\[12pt] 
H^A=H^A(t,D) =\displaystyle\sss_{l=0}^{m_A} \beta_l^A (t) D^l\,;\\[12pt]
F^B = F^B (t,D)=\displaystyle\sss_{l=0}^{n_B} \alp_l^B (t) D^l\,;\\[12pt] 
H^B=H^B(t,D) =\displaystyle\sss_{l=0}^{m_B} \beta_l^B (t) D^l\,;\\[12pt]
F^C = F^C (t,D)=\displaystyle\sss_{l=0}^{n_C} \alp_l^C (t) D^l\,;\\[12pt] 
H^C=H^C(t,D) =\displaystyle\sss_{l=0}^{m_CA} \beta_l^C (t) D^l
   \end{array}
   \right\}
    \label{e3.3-s}
    \end{equation}
являются известными матричными дифференциальными операторами размерности  
$h_A\times h_A$, $h_B\times h_B$ и $h_C\times h_C$ порядков $n_A, m_A$,
$n_B, m_B$, $n_C, m_C$ ($n_A\hm>m_A$, $n_B\hm>m_B$, $n_C\hm>m_C$) 
соответственно; индекс~$t$ у операторов означает, что оператор действует на функцию 
от~$t$ при фиксированном~$\tau$; звездочкой обозначен символ сопряжения оператора; 
$I_h^A$, $I_h^B$ и $I_h^C$~--- единичные $h\times h$ матрицы.

Введем $h^A$-, $h^B$- и $h^C$-мер\-ные векторы посредством соотношений

\vspace*{-2pt}

\noindent
    \begin{align*}
    U' &=\iii_{t_0}^t A(t,\tau) \vrp(X(\tau), \tau)\,d\tau\,;%\label{e3.4-s}
    \\
    U'' &=\iii_{t_0}^t B(t,\tau) \psi(X(\tau), \tau)\,d\tau\,;\label{e3.5-s}
    \\
    U''' &=\iii_{t_0}^t C(t,\tau) \chi(X(\tau), \tau,v)\,d\tau\,. %\label{e3.6-s}
    \end{align*}
Тогда, как известно из теории линейных дифференциальных сис\-тем~[4--6], 
переменные  $U'$, $U''$ и~$U'''$ будут удовлетворять следующим линейным дифференциальным 
уравнениям:

\vspace*{2pt}

\noindent
\begin{equation}
\left.
\begin{array}{rl}
F^A(t, D) U' &= H^A (t, D) \vrp (X, t)\,;\\[9pt]
    F^B(t, D) U'' &= H^B (t, D) \psi (X, t)\,;\\[9pt]
F^C(t, D) U''' &= H^C (t, D) \chi (X, t,v)\,.
\end{array}
\right\}
\label{e3.7-s}
\end{equation}
Применяя стандартную технику приведения уравнений~(\ref{e3.7-s}) к форме Коши~[4--6], 
придем к искомой ДСтС для расширенного вектора состояния 
$Z\hm=\left[ X^{\mathrm{T}} \,{Z_1'}^{\mathrm{T}}\, 
{Z_1''}^{\mathrm{T}}\,{Z_1'''}^{\mathrm{T}}\right]^{\mathrm{T}}$ 
($Z_1' \hm= U'$, $Z_1''\hm= U''$, $Z_1'''\hm = U'''$):

\pagebreak

\noindent
    \begin{multline}
    dZ = a_1^z (Z, t)\, dt + b_1^z (Z, t)\,dW_0+{}\\
    {}+ 
    \iii_{R_0^q} c_1^z (Z,  t, v) \,d P^0 (t, dv)\,.
    \label{e3.8-s}
    \end{multline}
Для случая  $h_A \hm= h_B \hm= h_C\hm=h$, $n_A\hm= n_B\hm=n_C\hm=n$ и 
$m_A\hm=m_B\hm=m_C\hm=m$ 
в подробной записи функции $a^z_1 (Z,t)$, $b^z_1 (Z,t)$ и $c^z_1 (Z, t,v)$ имеют следующий вид:
    \begin{align}
    a_1^z (Z, t)&=\begin{bmatrix}
        a(X, t)+ Z_1'\\
        a'(t)Z_1'\\
        a''(t) Z_1''\\
        a'''(t)Z_1'''
        \end{bmatrix}\,;\label{e3.9-s}\\
b_1^z (Z, t)&=\begin{bmatrix}
        b(X, t)+ Z_1''\\
        b''(t)Z_1''\\
        0\\
        0\end{bmatrix}\,;\label{e3.10-s}\\
  c_1^z (Z, t,v)&=\begin{bmatrix}
        c(X, t,v)+ Z_1'''\\
        c'''(t)Z_1'''\\
        0\\
        0\end{bmatrix}\,.\label{e3.11-s}
        \end{align}
При условии существования обратных матриц $(\alp_n^A)^{-1}$,
 $(\alp_n^B)^{-1}$ и $(\alp_n^C)^{-1}$ входящие в~(\ref{e3.9-s})--(\ref{e3.11-s}) 
 переменные и коэффициенты допускают  следующую запись:
 \begin{equation}
 \left.
 \begin{array}{rl}
    Z_{j+1}' &=\dot Z_j' - q_j' \vrp(X,t)\,;\\[9pt]
    Z_{j+1}'' &=\dot Z_j'' - q_j'' \psi(X,t)\,;\\[9pt]
 Z_{j+1}''' &=\dot Z_j'''- q_j''' \chi(X,t,v)\enskip (j=\overline{1,(n-1)})\,;
 \end{array}
 \right\}
% \label{e3.12-s}
 \end{equation}


\vspace*{-12pt}

\noindent
\begin{multline}
a'(t) ={}\\
{}=\begin{bmatrix}\scriptstyle
    I_h&\scriptstyle 0& \cdots& \scriptstyle 0\\
  \scriptstyle  0&\scriptstyle I_h&\scriptstyle \vdots&\scriptstyle 0\\
   \scriptstyle  \vdots&\cdots&\scriptstyle\ddots&\cdots\\
\scriptstyle    0&\scriptstyle 0&\scriptstyle\vdots&\scriptstyle I_h\\
    \scriptstyle-(\alp_n^A)^{-1}\alp_0^A&\scriptstyle -(\alp_n^A)^{-1}\alp_1^A&
    \cdots& \scriptstyle-(\alp_n^A)^{-1} \alp_{n-1}^A
    \end{bmatrix};
    \end{multline}

\vspace*{-12pt}

\noindent
\begin{multline}
a''(t) ={}\\
{}=\begin{bmatrix}
    \scriptstyle I_h&\scriptstyle 0& \cdots&\scriptstyle 0\\
   \scriptstyle 0&\scriptstyle I_h&\scriptstyle \vdots&\scriptstyle 0\\
  \scriptstyle  \vdots& \cdots&\scriptstyle \ddots& \cdots\\
  \scriptstyle  0&\scriptstyle 0&\scriptstyle \vdots&\scriptstyle I_h\\
   \scriptstyle -(\alp_n^B)^{-1}\alp_0^B& \scriptstyle-(\alp_n^B)^{-1}\alp_1^B&
\cdots&\scriptstyle -(\alp_n^B)^{-1}\alp_{n-1}^B
    \end{bmatrix};
    \end{multline}
    
\vspace*{-12pt}

\noindent
\begin{multline}
a'''(t) ={}\\
{}=\begin{bmatrix}
    \scriptstyle I_h&\scriptstyle0&\cdots&\scriptstyle 0\\
  \scriptstyle  0&\scriptstyle I_h&\scriptstyle \vdots&\scriptstyle 0\\
    \scriptstyle\vdots&\cdots&\scriptstyle\ddots&\cdots\\
\scriptstyle    0&\scriptstyle 0&\scriptstyle \vdots&\scriptstyle I_h\\
 \scriptstyle   -(\alp_n^C)^{-1}\alp_0^C&\scriptstyle -(\alp_n^C)^{-1}\alp_1^C&
\cdots&\scriptstyle-(\alp_n^C)^{-1}\alp_{n-1}^C
    \end{bmatrix}\,;\label{e3.13-s}
\end{multline}

\noindent
\begin{equation}
\left.
\begin{array}{rl}
q_j' &= (\alp_n^A)^{-1} \left[ \vphantom{\displaystyle\sss_{l=0}^{j-i} }
\beta_{n-j}^A -{}\right.\\
&\left.\displaystyle{}-\sss_{i=0}^{j-1} \sss_{l=0}^{j-i} 
{\cal C}_{n-j-l}^{n-j} \alp_{n-j+i+l}^A {q_i'}^{(l)}\right]\,;\\[9pt]
q_n' &= (\alp_n^A)^{-1} \lk \beta_{0}^A -\displaystyle\sss_{i=0}^{n-1} \sss_{l=0}^{n-i}  \alp_{i+l}^A {q_i'}^{(l)}\rk\,;
\end{array}
\right\}
\end{equation}
    \begin{equation}
\left.
\begin{array}{rl}
q_j'' &= (\alp_n^B)^{-1} \left[ \vphantom{\displaystyle\sss_{l=0}^{j-i} }
\beta_{n-j}^B -{}\right.\\
&\left.{}-\displaystyle\sss_{i=0}^{j-1} \sss_{l=0}^{j-i} {\cal C}_{n-j-l}^{n-j} 
\alp_{n-j+i+l}^A {q_i''}^{(l)}\right]\,;\\[9pt]
q_n'' &= \displaystyle(\alp_n^B)^{-1} \lk \beta_{0}^B -\sss_{i=0}^{n-1} \sss_{l=0}^{n-i}  \alp_{i+l}^A {q_i''}^{(l)}\rk\,;
\end{array}
\right\}
\end{equation}
\begin{equation}
\left.
\begin{array}{rl}
q_j''' &= (\alp_n^C)^{-1} \left[\vphantom{\displaystyle\sss_{l=0}^{j-i} }
 \beta_{n-j}^C -{}\right.\\
 &{}-\left.\displaystyle\sss_{i=0}^{j-1} \sss_{l=0}^{j-i} 
 {\cal C}_{n-j-l}^{n-j} \alp_{n-j+i+l}^C {q_i'''}^{(l)}\right]\,;\\[9pt]
q_j''' &= (\alp_n^C)^{-1} \lk \beta_{0}^C -\displaystyle\sss_{i=0}^{n-1} \sss_{l=0}^{n-i}  \alp_{i+l}^C {q_i'''}^{(l)}\rk\,.
\end{array}
\right\}
\label{e3.14-s}
\end{equation}
Здесь $C_m^n=n!/(m!(n-m)!)$; индекс~$l$ означает, что суммирование проводится 
по всем индексам, исключая~$l$.

Таким образом, справедливо следующие утверждение.

\medskip

\noindent
\textbf{Теорема 3.1.}\ 
\textit{Пусть эредитарные ядра $A(t,\tau)$, $B(t,\tau)$ и $C(t,\tau)$ в ЭСтС~$(\ref{e2.1-s})$ 
удовлетворяют условиям~$(\ref{e2.2-s})$--$(\ref{e2.4-s})$ 
или~$(\ref{e3.7-s})$, причем мат\-ри\-цы  $\alp_n^A$, $\alp_n^B$ и $\alp_n^C$ 
в~$(\ref{e3.3-s})$ обратимы, а функции~ $\vrp$, $\psi$ и~$\chi$ 
дифференцируемы по переменным расширенного вектора состояния достаточное число раз. 
Тогда ЭСтС~$(\ref{e2.1-s})$ приводится к ДСтС~$(\ref{e3.8-s})$ на 
основе~$(\ref{e3.9-s})$--$(\ref{e3.14-s})$.
}

\medskip

\noindent
\textbf{Замечание 3.1.}\ Рассмотрим ЭСтС со стационарным затуханием, когда эредитарные 
ядра удовлетворяют условиям~(\ref{e2.5-s}). Тогда вместо выполнения условия~(\ref{e2.4-s}) 
достаточно потребовать, чтобы преобразования\linebreak Лап\-ла\-са ядер были бы рациональными 
функциями скалярной переменной~$s$:
\begin{align*}
\iii_0^\infty \tilde A(u) e^{-su} \,du &= (F^A (s))^{-1} H^A (s)\,;\\
    \iii_0^\infty \tilde B(u) e^{-su}\, du &= (F^B (s))^{-1} H^B (s)\,;\\
\iii_0^\infty \tilde C(u) e^{-su} \,du &= (F^C (s))^{-1} H^C (s)\,.
%\label{e3.15-s}
\end{align*}
Здесь порядок матричных полиномов  $H^i (s)$ равен~$m^i$, а порядок полиномов  $F^i (s)$ 
равен~$n^i$, причем $n^i\hm\ge m^i$ $(i\hm=A,B,C)$.

\medskip

\noindent
\textbf{Замечание 3.2.}
В практических задачах элементами эредитарных ядер $A(t,\tau)$, $B(t,\tau)$ и $C(t,\tau)$  
являются следующие типовые функции: $e^{-\alp |u|}$ и 
$e^{-\alp |u|} (\cos \w u \hm+\gamma \sin \w |u|)$, удовлетворяющие обыкновенным 
дифференциальным уравнениям первого и второго порядка (см.\ разд.~6).

Наконец, в том случае, когда выполнены\linebreak условия~(\ref{e2.2-s})--(\ref{e2.4-s}), а функции 
$\vrp$, $\psi$ и $\chi$ не диф\-фе\-рен\-ци\-ру\-емы по переменным расширенного вектора состояния, 
целесообразна аппроксимация вы\-рож\-ден\-ны\-ми ядрами. Тогда имеют место следующие 
соотношения:
\begin{equation}
\left.
\begin{array}{rl}
\displaystyle\iii_{t_0}^t A(t,\tau) \vrp(X(\tau),  \tau) d\tau &= A^+ Y'\,;\\[9pt]
\displaystyle    \iii_{t_0}^t B(t,\tau) \psi(X(\tau),  \tau) d\tau &= B^+ Y''\,;\\[9pt]
\displaystyle\iii_{t_0}^t C(t,\tau) \chi(X(\tau),  \tau,v) d\tau &= C^+ Y'''\,;
\end{array}
\right\}
\label{e3.16-s}
\end{equation}
\begin{equation}
\left.
\begin{array}{c}        
\dot Y'=A^-\vrp\,;\enskip Y'(t_0)=0\,;\\[9pt]
        \dot Y''=B^-\psi\,;\enskip
        Y''(t_0)=0\,,\\[9pt] 
        \dot Y'''=C^-\chi\,;\enskip Y'''(t_0)=0\,;
        \end{array}
        \right\}
        \label{e3.17-s}
        \end{equation}
\begin{equation}
Z=\lk X^{\mathrm{T}} {Y'}^{\mathrm{T}} {Y''}^{\mathrm{T}} {Y'''}^{\mathrm{T}}\rk^{\mathrm{T}}\,;\label{e3.18-s}
\end{equation}

\vspace*{-12pt}

\noindent
\begin{multline}
 dZ= a_2^z (Z,t) \,dt+ b_2^z(Z,t,t)\, dW_0 +{}\\
 {}+\iii_{R_0^q} c_2^z (Z,t,t,v)\,dP^0 (t, dv)\,;
 \label{e3.19-s}
 \end{multline}
        $$a^z_2 (Z,t)=\begin{bmatrix}
        a(X,t)+A^+\vrp\\
        A^-\vrp\\
        B^- \psi\\
        C^-\chi\end{bmatrix}\,;$$
        $$
    b^z_2 (Z,t)=\begin{bmatrix}
        b(X,t)+B^+\psi\\
        0\\
        0\\
        0\end{bmatrix}\,;$$
$$
c^z_2 (Z,t,v)=\begin{bmatrix}
        c(X,t,v)+C^+\chi\\
        0\\
        0\\
        0\end{bmatrix}\,. %\label{e3.20-s}
  $$

Таким образом, имеем следующий результат.

\medskip

\noindent
\textbf{Теорема 3.2.}\ 
\textit{Пусть эредитарные ядра $A(t,\tau)$, $B(t,\tau)$ и $C(t,\tau)$ в ЭСтС~$(\ref{e2.1-s})$ 
удовлетворяют условиям~$(\ref{e2.3-s})$, $(\ref{e2.4-s})$ и~$(\ref{e2.6-s})$, 
а функции $\vrp, \psi, \chi$ не дифференцируемы по переменным расширенного 
вектора состояния. Тогда ЭСтС~$(\ref{e2.1-s})$ 
приводится к ДСтС~$(\ref{e3.19-s})$ на основе~$(\ref{e3.16-s})$ и~$(\ref{e3.17-s})$.}

\noindent
\textbf{Замечание 3.3.}\
Аналогичная теорема устанавливается для ЭСтС~(\ref{e2.8-s}).


\section{Обзор подходов к~моделированию распределений в~эредитарных
стохастических системах}


В стохастической информатике для вычисления вероятностей событий, 
связанных со случайными функциями, достаточно знания многомерных распределений. 
Различают три подхода к анализу и моделированию таких распределений в ЭСтС.

Первый, общий, основан на прямом численном решении уравнений~(\ref{e2.1-s}) или~(\ref{e2.8-s}), 
а для ЭСтС, приводимых к ДСтС интегрированием,~--- (\ref{e3.8-s}) или~(\ref{e3.19-s}), 
с последующей статистической обработкой результатов~[5--7].

Второй подход справедлив для ЭСтС, приводимых к ДСтС, поскольку он основан на 
теории марковских процессов и предполагает аналитическое моделирование, т.\,е.\ 
решение детерминированных уравнений в функциональных пространствах 
(Фок\-ке\-ра--План\-ка--Кол\-мо\-го\-ро\-ва, Пугачева и~др.)\ 
для одно- и многомерных распределений~\cite{5-s, 6-s}.

В практических задачах для ЭСтС, приводимых к ДСтС, часто можно рекомендовать 
комбинированный метод параметрического статистического и аналитического моделирования. 
При этом будем предполагать, что, во-пер\-вых, существуют одно- и многомерные плотности 
процессов, во-вто\-рых, плотности одно- и многомерных распределений можно параметризовать 
с помощью вероятностных моментов, квазимоментов, семиинвариантов, коэффициентов ортогональных 
разложений плотностей, канонических разложений и~др. и, в-третьих, по обобщенной 
формуле Ито составить для~$\Xi_n$ стохастические дифференциальные уравнения. 
При использовании такого подхода под расширенным вектором состояния следует 
рассматривать вектор $\bar Z\hm= \lk Z^{\mathrm{T}}  \Xi_n^{\mathrm{T}}\rk^{\mathrm{T}}$.

Наконец, отметим, что всегда приведенные к ДСтС нелинейные уравнения ЭСтС 
линейны относительно интегральных переменных $Z'$, $Z''$, $Z'''$ или $Y'$, $Y''$, $Y'''$.

Как известно~[4--8], в задачах корреляционного аналитического моделирования процессов в 
СтС с аддитивными шумами широкое распространение получил МСЛ. 
Для СтС с мультипликативными шумами развит МНА.

Обобщением МНА-рас\-пре\-де\-ле\-ний являются различные
приближенные методы, основанные на параметризации распределений.
Аппроксимация одномерной характеристической функции $g_1 (\la;t)$
и соответствующей плотности $f_1 (z,t)$ известными функциями
$g_1^* (\la;\Xi_1)$ и $f_1^* (z;\Xi_1)$, зависящими от
конечномерного векторного параметра $\Xi_1$,  сводит задачу
приближенного определения одномерного распределения к выводу из
уравнения для характеристических функций обыкновенных
дифференциальных уравнений, определяющих $\Xi_1$ как функцию
времени. Это относится и к остальным многомерным распределениям.
При аппроксимации многомерных распределений целесообразно выбирать
последовательности функций $\{g_n^* (\la_1\tr \la_n;\Xi_n)\}$ и
 $\{ f_n^* (z_1,\ldots,z_n;\Xi_n)\}$, каждая пара
которых находилась бы в такой  за\-ви\-си\-мости от векторного параметра~$\Xi_n$, 
чтобы при любом~$n$ множество параметров, образующих
вектор~$\Xi_n$, включало в качестве подмножества множество
параметров, образующих вектор~$\Xi_{n-1}$. Тогда при
аппроксимации $n$-мер\-но\-го распределения придется определять только
те координаты вектора $\Xi_n$, которые не были определены ранее
при аппроксимации функций $g_1$, $f_1\tr g_{n-1}$, $f_{n-1}$.

В зависимости от того, что представляют собой параметры, от
которых зависят функции $f_n^* (z_1\tr z_n;\Xi_n)$ и $g_n^*
(\la_1\tr \la_n;\Xi_n)$, аппроксимирующие неизвестные
многомерные плотности $f_n (z_1,  \ldots,z_n; t_1 \tr t_n)$ и
характеристические функции $g_n (\la_1\tr \la_n; t_1,\ldots,t_n)$,
используются различные приближенные методы решения
 уравнений, определяющих многомерные
распределения вектора со\-сто\-яния сис\-те\-мы~$Y_t$, в частности методы
моментов, семиинвариантов, ортогональных разложений и~др.


\section{Алгоритмы аналитического моделирования эредитарных стохастических
систем, основанные на~методах нормальной аппроксимации и~статистической линеаризации}


Для ЭСтС~(\ref{e2.8-s}) при условии~(\ref{e2.2-s}) с аддитивным шумом~$V$, 
когда  $b\hm=b(t)$ и $b_1\hm=b_1 (\tau, t)$, можно рекомендовать МСЛ для 
эквивалентной линеаризации\linebreak нелинейных функций $a(X,t)$ и $a_1 (X(\tau),\tau)$. 
В~результате получим нелинейное интегральное уравнение для математических ожиданий 
и линейное уравнение относительно центрированных переменных. Применяя к последнему 
уравнению корреляционную теорию линейных сис\-тем~\cite{4-s, 5-s, 8-s}, 
получим уравнения для ковариационной матрицы  $K^X(t)$ и матрицы ковариационных 
функций $K^X (t_1, t_2)$.

Для ЭСтС, приводимых к~(\ref{e3.8-s}) или~(\ref{e3.19-s}), уравнения МНА имеют следующий вид:
\begin{multline*}
    f_1(z;t) =\lk (2\pi)^{n_z} |K|\rk^{-1/2} \times{}\\
    {}\times \exp \lk -\fr{1}{2}\, 
    (z^{\mathrm{T}}- m^{\mathrm{T}}) K^{-1} (z-m)\rk\,;
    \end{multline*}
    
    \vspace*{-12pt}
    
    
\begin{multline}
f_n (z_1\tr z_n;t_1\tr t_n) =\lk (2\pi^{nn_z}) |\bar K|\rk^{-1/2} \times{}\\
{}\times
\exp \lk \fr{1}{2} \left(\bar z_n^{\mathrm{T}} -\bar m_n^{\mathrm{T}}\right) \bar K_n^{-1} 
\left(\bar z_n- \bar m_n\right)\rk\,;\label{e5.1-s}
\end{multline}
\begin{equation}
\dot m =\Phi_1 (m,K,t)\,;\label{e5.2-s}
\end{equation}

\vspace*{-12pt}

\noindent
\begin{multline}
\dot K =\Phi_2 (m,K,t) ={}\\
\!\!{}= \Phi_{21} (m,K,t) +\Phi_{21} (m,K,t)^{\mathrm{T}} 
+\Phi_{22} (m,K,t);\!\!\label{e5.3-s}
\end{multline}

\vspace*{-12pt}

\noindent
\begin{multline*}
\fr{\prt K(t_1, t_2)}{\prt t_2} ={}\\
{}= K(t_1, t_2) K(t_2)^{-1} \Phi_{21} (m(t_2), K(t_2), t_2)^{\mathrm{T}}
%\label{e5.4-s}
\end{multline*}
при $t_1<t_2$ и начальном условии $K(t_1, t_2)\hm = K(t_1)$.
Здесь введены следующие обозначения:
\begin{equation}
\left.
\begin{array}{rl}
\Phi_1 (m,K,t)&= {\sf M}_N a^z (Z,t)\,;\\[9pt]
\Phi_{21}(m,K,t)&= {\sf M}_N a^z (Z,t) (Z-m)^{\mathrm{T}}\,;\\[9pt]
\Phi_{22}(m,K,t) &= {\sf M}_N b^z (Z,t) \nu_0 b^z (Z,t)^{\mathrm{T}} + {}\\
&\hspace*{-15mm}\displaystyle{}+{\sf M}_N  \iii_{R_0^t} c^z (Z, t,v) c^z(Z,t,v)^{\mathrm{T}} \nu_P (t, dv)\,,
\end{array}
\right\}
\label{e5.5-s}
\end{equation}
где ${\sf M}_N$~--- символ вычисления математического ожидания для нормального 
распределения~(\ref{e5.1-s}), $n\hm=2,3,\ldots$, 
$Z_n \hm= \lk Z_1^{\mathrm{T}} \cdots Z_n^{\mathrm{T}}\rk^{\mathrm{T}}$, 
$ \bar m \hm=\lk m(t_1)^{\mathrm{T}}\cdots m(t_n)^{\mathrm{T}}\rk^{\mathrm{T}}$,
    $$\bar K_n =\begin{bmatrix}
    K(t_1, t_1)&\vdots&K(t_1, t_n)\\
    \vdots&\vdots&\vdots\\
    K(t_n, t_1)&\cdots&K(t_n, t_n)\end{bmatrix}\,.
    $$

\noindent
\textbf{Замечание 5.1.}\
Необходимым условием кор\-рект\-ности алгоритма МНА для определения стационарного 
стохастического режима  $m\hm=m^*$, $K\hm=K^*$  в ЭСтС с мультипликативными шумами 
является асимптотическая устойчивость линеаризованных уравнений~(\ref{e5.2-s}) 
и~(\ref{e5.3-s}).

Для ЭСтС с аддитивными шумами алгоритм МНА совпадает с алгоритмом МСЛ и определяется 
следующими уравнениями:
    \begin{align}
    \dot m &=\Phi_{1} (m,K,t)\,;\label{e5.6-s}\\
\dot Z^0 &=\Lambda (m,K,t)Z^0 + b V\,;\label{e5.7-s}\\
    \dot K &=\Lambda (m,K,t)K+ K\Lambda (m,K,t)^{\mathrm{T}} + b\nu b^{\mathrm{T}}\,;\label{e5.8-s}
    \end{align}
    $$
\fr{\prt K(t_1, t_2)}{\prt t_2} 
= K(t_1, t_2) \Lambda  \left(m(t_2), K(t_2), t_2\right)^{\mathrm{T}} \ (t_2>t_1)\,.
%    \label{e5.9-s}
$$

\noindent
\textbf{Замечание~5.2.}\
 Для нормальных ЭСтС необходима асимптотическая устойчивость матрицы
    \begin{equation}
    \Lambda (m,K) =\lk \left( \fr{\prt}{\prt m}\right) ( {\sf M}_N a^z (Z,t))^{\mathrm{T}}\rk^{\mathrm{T}}\,.
    \label{e5.10-s}
    \end{equation}

Асимптотическая устойчивость матрицы~(\ref{e5.10-s}) 
обеспечивает устойчивость численных методов интегрирования обыкновенных 
дифференциальных уравнений~(\ref{e5.6-s}) и~(\ref{e5.7-s}).


\section{Тестовые примеры}

{\bf 1.} Рассмотрим скалярную нелинейную ЭСтС первого порядка
    $$
    \dot{X} = a_0 + a_1 X + \int\limits_{0}^{t} e^{-\alpha |t-\tau|} \varphi(X(\tau))\, d\tau + b\dot{W}, \; X_0 = 0\,,
    $$
где $a_0$, $a_1$, $\alpha$, $b$~--- константы. Полагая
    $$\int\limits_{0}^{t} e^{-\alpha |t-\tau|} \varphi(X(\tau))\,;\ d \tau = Y\,,
    $$
приведем ЭСтС к ДСтС:

\noindent
\begin{align*}
\dot{X} &= a_0 + a_1 X + \fr{1}{\alpha}\, Y + b\dot{W}\,; \\
\dot{Y} &=   -\alpha Y + \varphi(X)\,;\\
X(0) &= Y(0) = 0\,.
\end{align*}

{\bf 2.}
 Рассмотрим скалярную линейную ЭСтС второго порядка
 
 \noindent
    \begin{multline*}
    a\ddot{X}+b \int\limits_{0}^{t}\dot{X}(\tau) e^{-\beta |t-\tau|} d\tau + 
    c \int\limits_{0}^{t} X(\tau) e^{-\gamma|t-\tau|}\,d\tau ={}\\
    {}= Q(t)\,,  \enskip
    X_0 = 0\,,
\end{multline*}
где $a$, $b$, $c$, $\beta$ и $\gamma$~--- константы. Расширим вектор 
со\-сто\-яния сис\-те\-мы следующим образом:
    $X_1 = X,$ $X_2 = \dot{X}$, $\dot{X}_1 = X_2.$
Полагая 
$$
\int\limits_{0}^{t} e^{-\beta |t-\tau|} X_2(\tau) d \tau = Y_1\,;\
\int\limits_{0}^{t} e^{-\gamma |t-\tau|} X_1(\tau) d \tau = Y_2\,,
$$
приведем ЭСтС к ДСтС:

\columnbreak

\noindent
\begin{align*}
    \dot{X}_1 &= X_2\,;\\
    \dot{X}_2 &= -\fr{b}{a}Y_1 -\fr{c}{a}Y_2+\fr{1}{a}\,Q(t)\,;\\
    \dot{Y}_1 &=   -\beta Y_1+ X_2\,;\\
    \dot{Y}_2 &=   -\gamma Y_2+ X_1\,;\\
    X_1(0) = X_2(0) &= Y_1(0) = Y_2(0) = 0\,.
\end{align*}


{\bf 3.}
 Рассмотрим скалярную нелинейную ЭСтС второго порядка
    $$
    \ddot{X}+\int\limits_{0}^{t}\varphi(X(\tau),\dot{X(\tau)})
    e^{-\alpha |t-\tau|}d\tau = Q(t), \; X_0 = 0\,,
    $$
где $\alpha$~--- константа. Расширим вектор состояния системы следующим образом:
$X_1 \hm= X$, $X_2 \hm= \dot{X}$, $\dot{X}_1 \hm= X_2$.

Полагая $\int\limits_{0}^{t}\varphi(X_1(\tau),X_2(\tau))e^{-\alpha |t-\tau|}d\tau \hm= Y$, 
придем к следующей ДСтС:
\begin{align*}
    \dot{X}_1 &= X_2\,, \\
    \dot{X}_2 &= - \fr{1}{\alpha}Y + Q(t)\,,\\
    \dot{Y}   &=  - \alpha Y +  \varphi(X_1,X_2)\,,\\
    X_1(0) = X_2(0) &= Y(0) = 0\,.
\end{align*}

{\bf 4.}
 Рассмотрим скалярную нелинейную ЭСтС второго порядка
    \begin{multline*}
    \ddot{X}+\int\limits_{0}^{t}\varphi(X(\tau)\,,\enskip 
    \dot{X}(\tau))e^{-\alpha |t-\tau|} \cos (\omega_0(t-\tau))\,d\tau ={}\\
    {}= Q(t)\,,\enskip  X_0 = 0\,,
    \end{multline*}
где $\alpha$ и $\omega_0$~--- константы. Расширим вектор состояния системы:
$X_1 \hm= X$, $X_2 \hm= \dot{X}$, $\dot{X}_1 \hm= X_2$.
Положим
    $$
    \int\limits_{0}^{t}\varphi(X_1(\tau),X_2(\tau))e^{-\alpha |t-\tau|}
    \cos\left(\omega_0(t-\tau)\right)d\tau = Y
    $$
и найдем производную $\dot{Y}$:
    \begin{multline*}
\!   \!\!\! \dot{Y} =  \int\limits_{0}^{t}\!\!
    \left(-\alpha e^{-\alpha |t-\tau|} \cos(\omega_0(t-\tau)) \varphi(X_1(\tau),X_2(\tau))
    \right. -{}\\
{}-\left.\omega_0 e^{-\alpha |t-\tau|} \sin(\omega_0(t-\tau)) \varphi(X_1(\tau),X_2(\tau))
    \right)d\tau +{}\\
    {}+ \varphi(X_1,X_2)\,.
    \end{multline*}
Представляя интеграл суммы суммой интегралов, получим:
\begin{multline*}
 \dot{Y} =  -\alpha Y-{}\\
 {}- \omega_0 \int\limits_{0}^{t}
     e^{-\alpha |t-\tau|} \cos(\omega_0(t-\tau)) \varphi(X_1(\tau),X_2(\tau))
    \,d\tau + {}\\
    {}+\varphi(X_1,X_2)\,.
\end{multline*}
Введем новую переменную:
    $$
    I = \int\limits_{0}^{t}\varphi(X_1(\tau),X_2(\tau))e^{-\alpha |t-\tau|}
    \sin\left(\omega_0(t-\tau)\right)d\tau.$$
Тогда полученное уравнение запишется в виде:
    $$
    \dot{Y} =  -\alpha Y- \omega_0 I + \varphi(X_1,X_2)\,.
    $$
Выразим из последнего уравнения~$I$:
    $$I = \fr{1}{\omega_0} (-\dot{Y}-\alpha Y + \varphi (X_1, X_2)).$$
Найдем производную~$\dot{I}$:
    \begin{multline*}
    \dot{I} = {}\\
    {}= \int\limits_{0}^{t}
    \left(-\alpha e^{-\alpha |t-\tau|} \sin(\omega_0(t-\tau)) \varphi(X_1(\tau),X_2(\tau))\right. +{}\\
    {}+\left.\omega_0 e^{-\alpha |t-\tau|} \cos(\omega_0(t-\tau)) 
    \varphi(X_1(\tau),X_2(\tau))
    \right)d\tau={}\\
    {}=-\alpha I + \omega_0 Y\,.
    \end{multline*}

Найдем вторую производную~$\ddot{Y}$:
\begin{multline*}
\ddot{Y} =  -\alpha \dot{Y}- \omega_0 \dot{I} + \fr{\partial \varphi}{\partial X_1} \,
\dot{X_1} + \fr{\partial \varphi}{\partial X_2} \,\dot{X_2}={}\\
{}=  \alpha^2 Y + 2\alpha \omega_0 I   - \omega_0^2 Y - \alpha \varphi(X_1,X_2) + {}\\
{}+
\fr{\partial \varphi}{\partial X_1}\, \dot{X_1} + 
\fr{\partial \varphi}{\partial X_2}\, \dot{X_2}={}\\
  {}= -2\alpha \dot{Y} -(\alpha^2 + \omega_0^2) Y + \alpha \varphi +
     \fr{\partial \varphi}{\partial X_1}\, \dot{X_1} + 
     \fr{\partial \varphi}{\partial X_2}\, \dot{X_2}\,.
     \end{multline*}
Расширим вектор состояния
$Y_1\hm=Y$, $Y_2\hm=\dot{Y}$, $\dot{Y}_1 \hm= Y_2$, в итоге придем к ДСтС вида:
   \begin{align*}
    \dot{X}_1 &= X_2\,; \\
    \dot{X}_2 &= -Y_1 + Q(t)\,; \\
    \dot{Y}_1 &= Y_2\,; \\
    \dot{Y}_2 &= -2\alpha Y_2 -(\alpha^2 + \omega_0^2) Y_1 + \alpha \varphi +{}\\
   &{}+
 \fr{\partial \varphi}{\partial X_1} X_2 + 
 \fr{\partial \varphi}{\partial X_2} [ Q(t)-Y]\,;\\
    X_1(0) = X_2(0) &= Y_1(0) = Y_2(0) = 0\,.
\end{align*}

{\bf 5.}
 Скалярная нелинейная стохастическая ЭСтС второго порядка
       \begin{multline*}
       \ddot{X}+\int\limits_{0}^{t}\varphi(X(\tau),\dot{X}(\tau))e^{-\alpha |t-\tau|}
        \left(
            \cos (\omega_0(t-\tau)) + {}\right.\\
            \left.{}+\gamma \sin (\omega_0(t-\tau))
        \right)
        d\tau = Q(t), \enskip  X_0 = 0
        \end{multline*}
($\alpha$, $\omega_0$ и $\gamma$~--- константы) для переменных
    $X_1 \hm= X$, $X_2\hm = \dot{X}$,  
$\dot{X}_1 \hm= X_2$, $Y_1\hm=Y$, $Y_2\hm=\dot{Y}$ и $\dot{Y}_1 = Y_2$,
где
    \begin{multline*}
    \int\limits_{0}^{t}\varphi(X_1(\tau),X_2(\tau))e^{-\alpha |t-\tau|}
        \left(
            \cos (\omega_0(t-\tau)) + {}\right.\\
            \left.{}+\gamma \sin (\omega_0(t-\tau))
        \right)\,         d\tau = Y
        \end{multline*}
 имеет вид:
\begin{gather*}
\dot{X}_1 = X_2\,; \\
\dot{X}_2 = -Y_1 + Q(t)\,; \\
\dot{Y}_1 = Y_2\,; \\
\dot{Y}_2 =  -2\alpha Y_2 - 
    (\omega_0^2 + \alpha^2)Y_1 +
    (\alpha + \omega_0 \gamma) \varphi(X_1,X_2) + {}\\
\hspace*{20mm} {}+
    \fr{\partial \varphi}{\partial X_1} X_2 +
 \fr{\partial \varphi}{\partial X_2} \left[Q(t)-Y\right]\,; \\
      X_1(0) = X_2(0) = Y_1(0) = Y_2(0) = 0\,. 
   \end{gather*}

{\bf 6.}
Рассмотрим осциллятор Дуффинга  при $X(t_0) \hm= X_0$, $X\hm=X_0$ в эредитарной 
стохастической среде, описываемый уравнением
  \begin{multline}
  \ddot X +\w^2 X -\mu X^3 =-\delta \dot X +\gamma +V +{}\\
\!\!\!\!{}+\!\iii_{t_0}^t\! \lk - \w_1^2 X(\tau) +\mu_1 X^3 (\tau) -\delta_1 \dot X(\tau) \rk \!
e^{-\la |t-\tau|}\, d\tau.\!\!\!\!\label{e6.1-s}
\end{multline}
Здесь $\w$, $\w_1$, $\mu$, $\mu_1$, $\delta$, $\delta_1$, $\gamma$ и $\la$~--- 
постоянные па\-ра\-мет\-ры; $V$~--- скалярный белый шум интенсивности~$\nu$. 
Используя метод аппроксимации экспоненциального эредитарного 
ядра уравнением первого порядка при нулевом начальном условии, приведем~(\ref{e6.1-s}) 
для $Z\hm= \lk Z_1 Z_2 Z_3\rk^{\mathrm{T}}$ $(Z_1\hm=X$, $Z_2\hm =\dot X)$ к ДСтС вида:
    \begin{gather}
    \dot Z = a (Z,t) + bV\,;\label{e6.2-s}\\
Z_1 (t_0) = Z_{10}\,; \enskip Z_2 (t_0) = Z_{20}\,; \enskip Z_3 (t_0) =0\,;\notag %\label{e6.3-s}
\end{gather}

\vspace*{-6pt}

\noindent
\begin{align*}
a(Z,t) &=\begin{bmatrix}
    Z_2\\
    -\w^2 Z_1 +\mu Z_1^3 -\delta Z_2 +\gamma-\la Z_3\\
    \la(-\w_1^2 Z_1 +\mu_1 Z_1^3 -\delta_1 Z_2 - Z_3)\end{bmatrix}\,;\\[6pt]
    b(Z,t) &=\begin{bmatrix}
    0\\
    1\\
    0\end{bmatrix}\,.
%\label{e6.4-s}
    \end{align*}

Применим МСЛ к~(\ref{e6.2-s}), положив согласно~\cite{8-s}
    \begin{equation*}
    Z_1^3 \approx (m_1^2 + 3D_1) m_1 + 3 (m_1^2 + D_1) 
    Z_1^0 
    \end{equation*}
($m_1 = M_N Z_1^0$, $D_1 \hm= M_N Z_1^{02})$,
придем к уравнениям~(\ref{e5.5-s})--(\ref{e5.8-s}). Входящие в эти уравнения
количества будут определяться сле\-ду\-ющи\-ми формулами:
\begin{align}
\Phi_1 (m,K) &=\begin{bmatrix}
    m_2\\
    -{\w'}^2 m_1 -\delta m_2 +\gamma -\la m_3\\
    \la(-{\w''}^2 m_1 -\delta_1 m_2 - m_3)\end{bmatrix}\,;\notag\\[3pt]
    %\label{e6.5-s}\\
   \Lambda (m,K) &=\begin{bmatrix}
    0&1&0\cr
    {\w'}^2 &-\delta &-\la\\
    -\la {\w''}^2 &-\la \delta_1&-\la
    \end{bmatrix}\,,\label{e6.6-s}
    \end{align}
где
\begin{align*}
{\w'}^2 &=\w'(m,K)^2 = \w^2 +\w_1^2 \mu (m_1^2+3D_1)\,;\\
{\w''}^2 &= \w'' (m,K)^2 =-\w_1^2 \mu_1 (m_1^2 + 3 D_1)\,.
\end{align*}
%\label{e6.7-s}

Стационарные значения  $m^*$ и~$K^*$ определяются из уравнений~(\ref{e5.6-s}) и~(\ref{e5.8-s}):
    \begin{align*}
\Phi_1 (m^*, K^*) &=0\,;\\
\Lambda (m^*, K^*) K^* + K^* \Lambda (m^*, K^*)^{\mathrm{T}} + \nu b b^{\mathrm{T}} &=0
\end{align*}
%\label{e6.8-s}
при условии асимпотической устойчивости матрицы~(\ref{e6.6-s}).

Сравнивая результаты моделирования стационарных стохастических колебаний осциллятора 
Дуффинга в эредитарной и неэредитарных стохастических средах, приходим к выводам:
\begin{itemize}
\item эредитарные свойства среды проявляются только для моментов времени $\la^{-1}$;
\item  МСЛ для ЭСтС приводит к более точным результатам, чем МСЛ для неэредитарных сред;
\item  МСЛ может быть успешно применен и для осциллятора Дуффинга в негауссовской среде с 
автокоррелированными шумами и разрывными нелинейностями~[9]. При этом точность 
МСЛ для эредитарных сред выше, чем для неэредитарных.
\end{itemize}


\section{Заключение}

Для нелинейных ЭСтС разработаны методы и алгоритмы аналитического моделирования 
распределений, основанные на их параметризации.

Особое внимание уделено методам и алгоритмам, основанным на приведении стохастических 
интегродифференциальных уравнений к стохастическим дифференциальным уравнениям с 
винеровскими и пуассоновскими шумами.
Полученные результаты положены в основу разрабатываемого в ИПИ РАН инструментального 
программного обеспечения <<IDStS>> в среде MATLAB.

{\small\frenchspacing
{%\baselineskip=10.8pt
\addcontentsline{toc}{section}{References}
\begin{thebibliography}{9}
\bibitem{1-s}
\Au{Колмановский В.\,Б., Ноcов В.\,Р. }
Устойчивость и периодические режимы регулируемых систем с последействием.~--- М.: Наука, 1981.
386~с.

\bibitem{2-s}
\Au{Синицын И.\,Н.}
Stochastic hereditary control systems~// Проблемы управления и теории информатики, 1986. 
Т.~15. №\,4. С.~287--298.

\bibitem{3-s}
\Au{Синицын И.\,Н. }
Конечномерные распределения процессов в стохастических интегральных и 
интегродифференциальных сис\-те\-мах~// 2nd  Symposium (International) 
IFAC on Stochastic Control: Preprints.~--- Vilnius, 1986; Pergamon Press, 1987. 
Pt.~1. P.~144--153.

\bibitem{4-s}
\Au{Пугачев В.\,С., Синицын И.\,Н. }
Стохастические дифференциальные системы. Анализ и фильтрация.~--- 
М.: Наука,  1990. 632~с.
[Англ. пер.: Stochastic differential systems. Analysis and filtering.~--- 
Chichester, New York: Jonh Wiley, 1987. 549~p.]

\bibitem{5-s}
\Au{Пугачев В.\,С., Синицын И.\,Н. }
Теория стохастических систем.~--- М.: Логос, 2000; 2004. 1000~с. [Англ. пер.:
Stochastic systems. Theory and  applications.~--- Singapore: World
Scientific,  2001. 908~p.]

\bibitem{6-s}
\Au{Синицын И.\,Н.}
Канонические представления случайных функций и их применение в задачах 
компьютерной поддержки научных исследований.~--- М.: ТОРУС ПРЕСС, 2009.
768~с.

\bibitem{7-s}
\Au{Синицын И.\,Н. }
Параметрическое статистическое и аналитическое моделирование 
распределений в нелинейных стохастических системах на многообразиях~// 
Информатика и её применения, 2013. Т.~7. Вып.~2. С.~4--16.

\bibitem{8-s}
\Au{Синицын И.\,Н.,  Синицын В.\,И.}
Лекции по нормальной и эллипсоидальной аппроксимации распределений в 
стохастических сис\-те\-мах.~--- М.: ТОРУС ПРЕСС, 2013. 480~с.

\bibitem{9-s}
\Au{Синицын И.\,Н.}
Аналитическое моделирование распределений с инвариантной мерой в 
стохастических сис\-те\-мах с разрывными характеристиками~// Информатика и её 
применения, 2013. Т.~7. Вып.~1. С.~3--11.

\end{thebibliography}
} }

\end{multicols}

\hfill{\small\textit{Поступила в редакцию 23.10.13}}


%\vspace*{12pt}

%\hrule

%\vspace*{2pt}

%\hrule

\newpage


\def\tit{ANALYSIS AND MODELING  OF DISTRIBUTIONS IN~HEREDITARY  STOCHASTIC SYSTEMS}

\def\titkol{Analysis and modeling  of distributions in~hereditary  stochastic systems}

\def\aut{I.\,N.~Sinitsyn}
\def\autkol{I.\,N.~Sinitsyn}


\titel{\tit}{\aut}{\autkol}{\titkol}

\vspace*{-12pt}


\noindent
Institute of Informatics 
Problems, Russian Academy of Sciences, 44-2 Vavilov Str., Moscow 119333, Russian
Federation


 
\def\leftfootline{\small{\textbf{\thepage}
\hfill INFORMATIKA I EE PRIMENENIYA~--- INFORMATICS AND APPLICATIONS\ \ \ 2014\ \ \ volume~8\ \ \ issue\ 1}
}%
 \def\rightfootline{\small{INFORMATIKA I EE PRIMENENIYA~--- INFORMATICS AND APPLICATIONS\ \ \ 2014\ \ \ volume~8\ \ \ issue\ 1
\hfill \textbf{\thepage}}}   

%\vspace*{6pt}
  
\Abste{Methods and algorithms for statistical and analytical 
modeling of one- and multidimensional distributions in hereditary 
stochastic systems (HStS) with Wiener and Poisson noises are considered. 
Nonlinear stochastic integrodifferential equations are presented. 
For dying physically realizable hereditary kernels, two ways of approximation 
(on the basis of linear operator equations and singular kernels) are described. 
Basic reduction algorithms of HStS to differential StS (DStS) are given. 
Detailed analysis of various approaches to statistical and analytical modeling 
of distributions in HStS reducible to DStS is given. These approaches are based: 
on the direct numerical integration DStS equations and numerical integration of 
equations for parameters (moments, quasi-moments, etc.)\ of orthogonal densities expansions. 
The detailed consideration of the method of statistical linearization (MSL) and of the
method of normal approximation (MNA) in reducible HStS to DStS is presented. 
Numerical stability of MSL and MNA algorithms is investigated. For MSL problems, 
one-step strong methods and algorithms of  numerical integration (of various accuracy) 
for smooth and nonsmooth right hands of HStS equations are described. Test examples for 
the IPI RAS software tool ``IDStS'' in MATLAB are considered. Special attention is 
paid to stochastic oscillations of the Duffing oscillator and the relay oscillator 
in hereditary stochastic media.}

\vspace*{-4pt}



\KWE{analytical and statistical modeling; differential system;
hereditary kernel; hereditary system; integrodifferential system;
parametrization of distribution; reducible system; singular kernel;
stochastic system} 

\vspace*{-4pt}


\DOI{10.14357/19922264140101}

\vspace*{-18pt}

\Ack
\noindent
The work was financially supported by the Program of the RAS
Department for Nanotechnologies and Information Technologies
``Intelligent information technology, systems analysis, and automation''
(project~1.7).

%\vspace*{-2pt}


  \begin{multicols}{2}

\renewcommand{\bibname}{\protect\rmfamily References}
%\renewcommand{\bibname}{\large\protect\rm References}

{\small\frenchspacing
{%\baselineskip=10.8pt
\addcontentsline{toc}{section}{References}
\begin{thebibliography}{9}

\bibitem{1-s-1}
\Aue{Kolmanovskij, V.\,B., and V.\,R.~Nosov}. 
1981. \textit{Ustoychivost' i periodicheskie rezhimy sistem s posledeystviem} 
[\textit{Stability of hereditary systems}]. Moscow: Nauka. 386~p.
\bibitem{2-s1}
\Aue{Sinitsyn, I.\,N.}
1986. Stochstic hereditary control systems. \textit{Problems Control Infrom. Theory} 
15(4):287--298.
\bibitem{3-s-1}
\Aue{Sinitsyn, I.\,N.} 1986. Finite-dimenstional distributions in stochastic 
integral and integraldifferential systems. 
\textit{2nd  Symposium (International) IFAC on Stochastic Control Proceeding: Preprints}.
 Vilnius: Pergamon Press. Pt.\,1. P.~144--153.
\bibitem{4-s-1}
\Aue{Pugachev, V.\,S., and I.\,N.~Sinitsyn}. 
1987. \textit{Stochastic differential systems. Analysis and filtering}. 
Chichester, New York: Jonh Wiley. 549~p.
\bibitem{5-s-1}
\Aue{Pugachev, V.\,S., and I.\,N.~Sinitsyn}. 
2001. \textit{Stochastic systems. Theory and applications}. Singapore: World Scientific. 908~p.
\bibitem{6-s-1}
\Aue{Sinitsyn, I.\,N.}
2009. \textit{Kanonicheskie predstavleniya slu\-chay\-nykh funktsiy i ikh primenenie v zadachakh 
komp'yuternoy podderzhki nauchnykh issledovaniy} 
[\textit{Canonical expansions of random functions and its application to scientific 
computer-aided support}].  Moscow: TORUS PRESS. 768~p.
\bibitem{7-s-1}
\Aue{Sinitsyn, I.\,N.} 2013.
Parametricheskoe statisticheskoe i analiticheskoe modelirovanie 
raspredeleniy v nelineynykh stochasticheskikh sistemakh na mnogoobraziyakh 
[Parametrical statistical and analytical modeling of distributions in 
stochastic systems on manifolds]. \textit{Informatika i ee Primemeniya}~---
\textit{Inform. Appl}. 7(2):4--16.
\bibitem{8-s-1}
\Au{Sinitsyn, I.\,N., and V.\,I.~Sinitsyn}.  2013. 
Lektsii po nor\-mal'\-noy i ellipsoidal'noy approksimatsii raspredeleniy 
v stokhasticheskikh sistemakh 
[Lectures on normal and ellipsoidal distributions approximations]. Moscow: TORUS PRESS. 480~p.
\bibitem{9-s-1}
\Aue{Sinitsyn, I.\,N.} 2013. 
Analiticheskoe modelirovanie raspredeleniy s invariantnoy meroy v 
stokhasticheskikh sistemakh s razryvnymi kharakteristikami 
[Analytical modeling of distributions in stochastic systems with 
discontinuous characteristics]. \textit{Informatika i ee Primemeniya}~---
\textit{Inform. Appl}. 7(1):3--11.

\end{thebibliography}
} }


\end{multicols}

\vspace*{-9pt}

\hfill{\small\textit{Received October 23, 2013}}

\vspace*{-24pt}

\Contrl

\noindent
\textbf{Sinitsyn Igor N.} (b.\ 1940)~--- Doctor of Science in technology, Professor, 
Honored scientist of Russian Federation, Head of Department,
Institute of Informatics 
Problems, Russian Academy of Sciences,
44-2 Vavilov Str., Moscow 119333, Russian Federation; sinitsin@dol.ru


 \label{end\stat}
 
\renewcommand{\bibname}{\protect\rm Литература}




 