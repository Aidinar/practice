\def\stat{arkhipov}

\def\tit{ИНФОРМАЦИОННАЯ МОДЕЛЬ ТЕХНОЛОГИИ ПРЕДСТАВЛЕНИЯ НАТУРНОГО ОБЪЕКТА И~ИЗМЕНЕНИЯ 
ЕГО~ПРОСТРАНСТВЕННОГО ПОЛОЖЕНИЯ}

\def\titkol{Информационная модель технологии представления НО
%натурного объекта 
и~изменения  его пространственного положения}

\def\autkol{О.\,П.~Архипов, Ю.\,А.~Маньяков, Д.\,О.~Сиротинин}

\def\aut{О.\,П.~Архипов$^1$, Ю.\,А.~Маньяков$^2$, Д.\,О.~Сиротинин$^3$}

\titel{\tit}{\aut}{\autkol}{\titkol}

%{\renewcommand{\thefootnote}{\fnsymbol{footnote}} \footnotetext[1]{Работа 
%выполнена при финансовой поддержке РФФИ (проект 11-01-00515а).}}

\renewcommand{\thefootnote}{\arabic{footnote}}
\footnotetext[1]{Орловский филиал Института проблем информатики Российской 
академии наук, arkhipov12@yandex.ru} 
\footnotetext[2]{Орловский филиал Института проблем информатики Российской академии наук, maniakov\_yuri@mail.ru}
\footnotetext[3]{Орловский филиал Института проблем информатики Российской академии наук, vespert@mail.ru}

  
\Abst{Приводится описание информационной модели технологии 
представления натурного объекта (НО) и изменения его пространственного 
положения с последующей интеграцией полученных данных в трехмерные (3D)
изображения. Рассматриваются этапы технологии, а также основные структуры 
и потоки данных. 
  Предлагаемая технология подразумевает простой способ создания сложной 
анимации 3D объектов, содержащий как изменение положения 
различных объектов, так и их формы, не предъявляющий высоких требований к 
квалификации пользователя и не подразумевающий наличия сложного 
дорогостоящего оборудования. Технология может быть использована для 
создания обучающих программ и энциклопедий, рекламных видеороликов, 
клипов, фильмов, а также для реализации систем представления и передачи 
визуальной информации, для решения задач дистанционного управления, в 
системах дополненной реальности, пользовательских интерфейсах, системах 
поддержки принятия решений, мониторинга, контроля качества.}
   
\KW{стереоскопия; 3D-модель; обработка изображений; цветовые 
характеристики; опорные точки; трехмерная реконструкция; анимация}

\DOI{10.14357/19922264140107}

\vskip 20pt plus 9pt minus 6pt

      \thispagestyle{headings}

      \begin{multicols}{2}

            \label{st\stat}
  

\section{Введение}

  В настоящее время при создании видеороликов, фильмов, рекламы и другого 
визуального контента широко используется 3D анимация. Среди 
методов создания такой анимации выделяют технологии захва\-та движения 
(Motion capture), которая изменяет виртуальную модель на основе данных, 
по\-лу\-ча\-емых с реального объекта, и <<ручное>> создание анимации в пакетах 
3D графики. Среди недостатков технологии Motion capture можно 
выделить не\-об\-хо\-ди\-мость размещения различных датчиков, затрудняющих 
перемещения, а также отсутствие воз\-мож\-ности их размещения на 
определенных объектах, например человеческом лице. Кроме того, обе 
рассмотренные технологии требуют наличия довольно большого количества 
высококвалифицированного технического персонала и дорогостоящего 
оборудования.
  
  Вследствие этого актуальна разработка бесконтактных способов 3D
реконструкции и анимации, использующих распространенные и недорогие 
технические средства (веб-ка\-ме\-ры, фотоаппараты).
  
  Целью предлагаемой работы является создание информационной модели 
технологии представления натурного объекта и изменения его 
пространственного положения с последующей интеграцией полученных 
данных в 3D изображения.
  
  Основной идеей предлагаемой технологии является поиск особых опорных 
точек (ОТ) на изображениях объекта, составляющих стереопары. Поиск 
производится на основе сопоставления цветовых характеристик различных 
областей изображений. В~отличие от известных, данная технология 
подразумевает простой способ создания сложной анимации 3D
объектов, содержащий изменение как положения различных объектов, так и их 
формы, не предъявляющий высоких требований к квалификации пользователя. 
  
  Информационная модель, представляющая данную технологию, 
подразумевает ряд этапов:
  \begin{enumerate}[1.]
\item Построение сцены и съемка НО.
\item Поиск границ объекта на изображениях.
\item Фрагментация изображений объекта.
\item Построение локальных систем координат (ЛСК).
\item Генерация ОТ:
\begin{itemize}
    \item[(а)] аппроксимация цветового пространства изображения;

    \item[(б)] сегментация изображения на основе цветовых характеристик;
    \item[(в)] структурирование сегментированных областей.
    \end{itemize}
\item Индексирование.
\item Поиск сопряженных и соответственных ОТ.
\item Фильтрация ОТ.
\end{enumerate}

\vspace*{16pt}

\noindent
\begin{center}  %fig1
 \mbox{%
 \epsfxsize=76.034mm
 \epsfbox{arh-1.eps}
 }
  \end{center}

  \vspace*{6pt}

\noindent
{\small{Схема информационной модели технологии представления НО и 
изменения его пространственного положения}}

%\vspace*{12pt}

      

\addtocounter{figure}{1}

\noindent
\begin{enumerate}
\setcounter{enumi}{8}
\item Вычисление координат ОТ в 3D пространстве.
\item Вычисление векторов смещения ОТ.
\item Построение 3D воксельной модели.
\item Текстурирование воксельной модели.
\end{enumerate}

  Обобщенная схема информационной модели представлена на рисунке.
  
  Рассмотрим все этапы более подробно.
  
\section{Построение сцены и~съемка натурного объекта}

  Съемка натурного объекта должна обеспечивать получение полной 
информации о поверхности объекта. Для этого необходимо получить 
информацию cо всех сторон объекта. Эта задача решается путем круговой 
съемки одной камерой или несколькими камерами.
  
  Съемка объекта характеризуется следующими параметрами:
  $$
  \mathrm{SP}=\langle d, \alpha, f, \mathrm{fov}\rangle\,,
  $$
где $d$~--- расстояние до объекта;
$\alpha$~--- угол между оптическими осями камер;
$f$~--- фокусное расстояние;
fov~--- угловое поле объектива.
  
  В результате получаются серии изображений НО, полученные 
через определенные временн$\acute{\mbox{ы}}$е интервалы,~--- $I^{LR}$:
  $$
  I^{LR} = \begin{bmatrix}
  \langle I^L, I^R\rangle_{1,1} & \cdots & \langle I^L, I^R\rangle_{1,r}\\
  \vdots & \ddots & \vdots\\
  \langle I^L, I^R\rangle_{t,1} & \cdots & \langle I^L, I^R\rangle_{t,r}
  \end{bmatrix}\,.
  $$
Здесь $r$~--- индекс ракурса;
$t$~--- индекс временн$\acute{\mbox{о}}$го этапа съемки;
$\langle I^L, I^R\rangle$~--- стереопара, где $I^L$ и $I^R$ являются 
подмножествами~$I$:
$$
I=\begin{bmatrix}
\mathrm{px}_{1,1} & \cdots & \mathrm{px}_{1,w}\\
\vdots & \ddots & \vdots\\
\mathrm{px}_{h,1} &\cdots & \mathrm{px}_{h,w}
\end{bmatrix}\,,\enskip  \mathrm{px} = (r, g, b)\,,
$$
$w$ и $h$~--- соответственно ширина и высота изоб\-ра\-же\-ния в пикселах.



  
\section{Поиск границ объекта на~изображениях}

  Первым этапом обработки полученных изображений является выделение 
границ объекта. Главной информационной единицей для этого этапа служит 
маска, представляющая собой битовую матрицу и определяющая 
принадлежность каждой точки изоб\-ра\-же\-ния объекту:
  $$
  Q=\begin{bmatrix}
  q_{1,1} & q_{1,2} & \cdots & q_{1,w}\\
  q_{2,1} & q_{2,2} & \cdots & q_{2,w}\\
  \cdots & \cdots & \cdots & \cdots\\
  q_{h,1} & q_{h,2} & \cdots & q_{h,w}
  \end{bmatrix}\,,
  $$
где $q_{ij}\hm = [1, 0]$~--- переменная, равная единице в том случае, когда 
данный пиксел изображения принадлежит объекту.

  Результатом выделения границ объекта является множество точек, 
принадлежащих объекту на изображении:
  $$
  O=I\cap Q\,,\quad O\in I\,,
  $$
где $Q$~--- маска (растровое бинарное изображение, обозначающее, является 
ли каждый его пиксел \mbox{частью} объекта).
  
  У~совокупности пикселов изображения, принадлежащих объекту, 
выделяются пикселы, являющиеся граничными. Для этого применяется еще 
одна битовая маска: 
  $$
Q^\prime=\begin{bmatrix}
  q_{11} & \cdots & q_{1w}\\
  \vdots & \ddots & \vdots\\
  q_{h1} & \cdots & q_{hw}
  \end{bmatrix}\,,
  $$
где $q_{ij} = [1, 0]$~--- переменная, равная единице в том случае, если данный 
пиксел изображения является граничным пикселом объекта.

\section{Фрагментация изображений объекта}

  Этап фрагментации осуществляется пользо\-вателем в интерактивном режиме 
по принципу определения движущихся частей объекта. Цель фрагментации~--- 
создание групп ОТ, к которым относят\-ся ОТ, принадлежащие неделимой 
движущейся части (например, руке, ноге и~т.\,п.). Множество фрагментов на 
изображении можно представить в виде
  \begin{equation*}
  F=\{ f_1,\ldots , f_{10}\}\,.
  \end{equation*}
  Здесь
  $$  
   f_i=\langle (x_1,y_1), (x_2,y_2), (x_3, y_3), (x_4, 
y_4)\rangle\,,
$$
где ($x_1, y_1$), $(x_2, y_2)$, $(x_3, y_3)$ и $(x_4, y_4)$~--- координаты четырех 
вершин прямоугольника, описывающего фрагмент.

  Множество точек объекта, принадлежащих фрагментам:
  $$
  F^\prime =f(O,F)\,,\quad F^\prime\subset O\,.
  $$
  
\section{Построение локальных систем координат}

  Для оптимизации процесса обработки изменений положения точек в каждом 
из фрагментов их координаты рассчитываются относительно 
ЛСК фрагмента. Для этого используется матрица преобразования 
координат точек изображения к ЛСК:
  $$
  C=\begin{bmatrix}
  1&0&0\\
  0&1&0\\
  \Delta x & \Delta y &1
  \end{bmatrix}\,,
  $$
где $\Delta x$ и $\Delta y$~--- смещение начала ЛСК относительно точки начала 
мировой системы координат.

\section{Генерация опорных точек}

  Основными этапами генерации ОТ являются:
  \begin{itemize}
  \item[(а)] аппроксимация цветового пространства изоб\-ра\-жения:
  \begin{gather*}
  I\to A\,,\\
  A=\begin{bmatrix}
  R_1 & G_1 & B_1\\
  \vdots & \vdots & \vdots\\
  R_l & G_l & B_l
  \end{bmatrix}\,,
  \end{gather*}
где $l$~--- количество цветов палитры изображения после аппроксимации;
$R$, $G$ и $B$~--- цветовые характеристики красного, зеленого и синего 
каналов цвета;
  \item[(б)] сегментация изображения на основе цветовых характеристик:
  $$
\mathrm{SP}=\begin{bmatrix}
  \mathrm{tlx}_1 & \mathrm{tly}_1 & \mathrm{brx}_1 & \mathrm{bry}_1\\
  \mathrm{tlx}_2 & \mathrm{tly}_2 & \mathrm{brx}_2 & \mathrm{bry}_2\\
  \vdots &\vdots &\vdots & \vdots\\
  \mathrm{tlx}_n & \mathrm{tly}_n & \mathrm{brx}_n & \mathrm{bry}_n
  \end{bmatrix}\,,
  $$
где tlx и tly~--- координаты верхней левой граничной точки сегмента;
brx и bry~--- координаты нижней правой граничной точки сегмента;
  \item[(в)] структурирование сегментированных областей:
  $$
  S=\begin{bmatrix}
  \mathrm{tlx}_1 & \mathrm{tly}_1 & \mathrm{brx}_1 & \mathrm{bry}_1 & s\\
  \mathrm{tlx}_2 & \mathrm{tly}_2 & \mathrm{brx}_2 & \mathrm{bry}_2 & s\\
  \vdots & \vdots & \vdots & \vdots &\vdots\\
  \mathrm{tlx}_n & \mathrm{tly}_n & \mathrm{brx}_n & \mathrm{bry}_n & s
  \end{bmatrix}\,,
  $$
где $s$~--- состояние структурирующего прямоугольника, $s\hm\in \{0, 1\}$.
  \end{itemize}
  
  В результате получаются две совокупности ОТ для левых ($P_l$) и правых 
($P_r$) изображений стереопар:
  \begin{equation*}
  P=P_l\cup P_r\,.
  \end{equation*} 
  Здесь
  \begin{equation*}
  P_l=\begin{bmatrix}
  p_{1,1} &\cdots & p_{1,s}\\
  \vdots & \ddots & \vdots\\
  p_{t,1} & \cdots & p_{t,s}
  \end{bmatrix}\,;\enskip P_r= \begin{bmatrix}
  p_{1,1} &\cdots & p_{1,q}\\
  \vdots & \ddots & \vdots\\
  p_{t,1} & \cdots & p_{t,q}
  \end{bmatrix}\,,
  \end{equation*}
  где
$$
  p_{i,j} =\langle x, y, r, g ,b\rangle\,,
$$
$(x, y)$~--- координаты ОТ на изображении;
$(r, g, b)$~--- цветовые координаты ОТ в пространстве RGB;
$s$ и~$q$~--- количество ОТ на левых и правых изоб\-ра\-же\-ни\-ях соответственно;
$t$~--- индекс временн$\acute{\mbox{о}}$го этапа съемки.
  
  Сразу после генерации ОТ производится преобразование их координат к 
локальным:
  $$
  P_c=PC\,.
  $$
  
\vspace*{-18pt}

\section{Индексирование}
  
  Индексирование ОТ осуществляется по принципу принадлежности к тому 
или иному фрагменту. В~результате каждый элемент ($p_{ij}$) матрицы~$P$ 
преобразуется к виду:
  $$
p_{ci} = \langle x, y, r, g, b, F_i\rangle\,,
$$
где $F_i$~--- индекс фрагмента, которому принадлежит~ОТ.
  
  В результате получается преобразованная мат\-ри\-ца $P_{ci}$, 
аналогичная~$P_c$, с элементами, содержащими индексированные ОТ ($p_c$).


\vspace*{-9pt}

\section{Поиск сопряженных и~соответственных опорных~точек}
  
  Исходными данными являются сегменты изоб\-ра\-жения~$S$, полученные на 
этапе сегментации. Результатом поиска будут множества точек, полученные в 
результате применения функций сбора \mbox{окружения}~$F_n$, проверки 
соответствия~$F_{\mathrm{sp}}$ и сопряженности~$F_p$:
  \begin{align*}
P_{lr} & = F_n(P_c) * F_{sp}(P_c)\,;\\
P_s &= F_n(P_c)  * F_p(P_c)\,,
\end{align*}
где $P_{lr}$~--- совокупность сопряженных ОТ;
$P_{s}$~--- совокупность соответственных ОТ.

  Учитывая, что в результате поиска сопряженных точек каждой ОТ 
однозначно соответствует опре-\linebreak\vspace*{-12pt}

\columnbreak

\noindent
деленная точка на изображении, можно каждой 
ОТ поставить в соответствие определенное значение цветовых координат точки 
на изображении. Таким образом, $j$-ю сопряженную ОТ можно представить 
кортежем
  \begin{equation*}
  P_{lr} = \left \{ p^{lr}_1, \ldots , p_k^{lk}\right\}\,.
  \end{equation*}
  Здесь
  $$
p^{lr}_j = \langle x_1, y_1, x_2, y_2, r, g, b, F_i\rangle\,,\enskip  j = 1, \ldots, k\,,
$$
где ($x_1, y_1$)~--- координаты ОТ на левом изображении стереопары;
$(x_2, y_2)$~--- координаты ОТ на правом изображении стереопары;
$(r, g, b)$~--- усредненные цветовые координаты ОТ в пространстве RGB;
$F_i$~--- индекс фрагмента.

  Тогда соответственные точки можно представить как
  $$
  P_s=\left\{ P_{lr_1},\ldots , P_{lr_t}\right\}\,.
  $$
  
  
  \vspace*{-9pt}
  
\section{Фильтрация опорных точек}
  
  Фильтрация ОТ производится с целью уменьшения количества 
рассматриваемых ОТ. Данная возможность позволяет достаточно гибко 
варьировать вычислительную нагрузку, изменяя количество ОТ в зависимости 
от требуемой степени детализации. Результатом фильтрации является 
множество опорных точек 
  $$
\mathrm{OT}_{\mathrm{2D}}\in P_s\,.$$

\vspace*{-9pt}

\section{Вычисление координат опорных точек в~трехмерном~пространстве}

  Вычисление координат ОТ в 3D пространстве основано на модели 
стереоскопической системы общего назначения~\cite{1-arh}. Главной 
осо\-бен\-ностью такой системы является наличие модели камеры, содержащей 
матрицы параметров внешних и внут\-рен\-них камер.
  
  Матрица внутренних параметров камеры содержит только параметры 
оптической системы и фотоприемника камеры и имеет следующий вид:
  $$
  A=\begin{bmatrix}
  \fr{f}{q} & 0 &u_0\\
  0 & \fr{f}{p} & v_0\\
  0 &0&1
  \end{bmatrix}\,,
  $$
где ($u_0, v_0$)~--- координаты главной точки относительно начала координат 
фотоприемника (в естественных координатах фотоприемника); 
  $q$ и~$p$~--- рас\-сто\-яния между сенсорами матрицы камеры вдоль строк и 
столбцов.
  
  Внешние параметры описывают взаимное расположение камер в 
стереоскопической системе. Данные параметры определяются матрицами 
аффинных преобразований, посредством которых осуществляется переход от 
системы координат одной камеры к системе координат другой, и выражаются 
соотношением
  $$
  M_l=RM_r+T\,,
  $$
где $R$~--- мат\-ри\-ца поворота, описывающая ориентацию системы 
координат левой камеры относительно правой (поворот оптической оси);
$T$~--- вектор, определяющий положение оптического центра левой камеры в 
системе координат правой.
  
  Основываясь на полученной на предыдущем этапе модели камеры, можно 
вычислить координаты ОТ в 3D пространстве:
$$
\mathrm{OT}_{\mathrm{3D}} = \langle x, y, z, F_i\rangle\,.
$$

\vspace*{-9pt}
  
\section{Построение и~преобразование трехмерной воксельной 
модели}
  
  Воксельное представление подразумевает определенную упорядоченность 
объемных примитивов~--- вокселов. Построение воксельной модели основано на 
методе воксельной аппроксимации~\cite{2-arh}, исходной информацией для 
которой служит совокупность ОТ в 3D пространстве, полученная на 
предыдущем этапе, а результатом~--- воксельная модель вида
  \begin{align*}
V &= \left(v_1, v_2, \ldots , v_n\right)\,,\\
V &= \langle \mathrm{xs}_v, \mathrm{ys}_v, \mathrm{zs}_v\rangle\,,
\end{align*}
где $s_v$~--- размер воксела, определяющий 3D разрешение сцены.
  
  Преобразование модели осуществляется на основе изменений положения ОТ, 
которые в свою очередь описываются векторами смещения~($\Delta$):
  $$
  \Delta =\mathrm{OT}_{\mathrm{3D}_t}-\mathrm{OT}_{\mathrm{3D}_{t-1}}\,.
  $$
  
  Для отображения преобразования на модели при изменении положения ОТ 
необходимо вы\-чис\-лить смещение окружающих данную ОТ
вокселов в определенной области. После этого для каждой ОТ, изменившей 
свое положение, осуществить воксельную аппроксимацию.
  
\section{Текстурирование воксельной модели}

  Процесс текстурирования в рамках рассматриваемой информационной 
технологии заключается в присвоении вокселам модели цветов путем 
интерполяции на основе цветовых характеристик ОТ.
  
  Исходной информацией на данном этапе является совокупность ОТ в 
3D пространстве.
  
  На основе исходной информации о цветах ближайших ОТ осуществляется 
линейная интерполяция цветов вокселов в области, ограниченной данными ОТ. 
В~результате получается совокупность вокселов, образующих 3D
модель и содержащих информацию о цвете:
  $$
  V_T= \langle x, y, z, r, g, b, F_i\rangle\,.
  $$
  
\vspace*{-9pt}

\section{Заключение}


  Таким образом, выше описана информационная модель технологии 
представления НО и изменения его пространственного 
положения с по\-сле\-ду\-ющей интеграцией полученных данных в 3D\linebreak
изображения. Приведено описание основных этапов информационной 
технологии, структур и потоков данных.
  
  Предлагаемая технология может быть исполь\-зована для создания обучающих 
программ и энциклопедий, рекламных видеороликов, клипов, фильмов. Кроме 
того, может быть использована для реализа\-ции систем представления и 
передачи визуальной информации для решения задач дистанционного 
управ\-ле\-ния, сис\-тем дополненной реальности, пользовательских интерфейсов, 
сис\-тем поддержки принятия решений, мониторинга, контроля качества. 

\vspace*{-9pt}

{\small\frenchspacing
{%\baselineskip=10.8pt
\addcontentsline{toc}{section}{References}
\begin{thebibliography}{9}
\bibitem{1-arh}
\Au{Грузман И.\,С., Киричук~В.\,С., Косых~В.\,П., Перетягин~Г.\,И., 
Спектор~А.\,А.} Цифровая обработка изображений в информационных 
системах.~--- Новосибирск: НГТУ, 2000.
\bibitem{2-arh}
\Au{Маньяков Ю.\,А.} Технология регистрации поведения объектов в 
трехмерном пространстве~// Информационные технологии в науке, 
образовании и производстве (ИТНОП): Мат-лы Междунар. 
науч.-технич. конф.~--- Орел: ОрелГТУ, 2010. Т.~3. С.~182--186.
\end{thebibliography}
} }

\end{multicols}

\vspace*{-6pt}

\hfill{\small\textit{Поступила в редакцию 02.10.13}}

\newpage


%\vspace*{12pt}

%\hrule

%\vspace*{2pt}

%\hrule


\def\tit{INFORMATION MODEL OF~FULL-SCALE OBJECT 
AND~ITS~ATTITUDE CHANGES REPRESENTATION TECHNOLOGY}

\def\titkol{Information model of~full-scale object and~its 
attitude changes representation technology}

\def\aut{O.\,P.~Arkhipov, Y.\,A.~Maniakov, and~D.\,O.~Sirotinin}
\def\autkol{O.\,P.~Arkhipov, Y.\,A.~Maniakov, and~D.\,O.~Sirotinin}


\titel{\tit}{\aut}{\autkol}{\titkol}

\vspace*{-9pt}

\noindent
Orel Branch of Institute of Informatics Problems, Russian
Academy of Sciences, 137 Moskovskoe Highway, Orel 302025, Russian Federation


 
\def\leftfootline{\small{\textbf{\thepage}
\hfill INFORMATIKA I EE PRIMENENIYA~--- INFORMATICS AND APPLICATIONS\ \ \ 2014\ \ \ volume~8\ \ \ issue\ 1}
}%
 \def\rightfootline{\small{INFORMATIKA I EE PRIMENENIYA~--- INFORMATICS AND APPLICATIONS\ \ \ 2014\ \ \ volume~8\ \ \ issue\ 1
\hfill \textbf{\thepage}}}   

\vspace*{6pt}
  
\Abste{The article describes the information model of full-scale object and its attitude 
changes representation with the following integration of the obtained data into 
three-dimensional (3D) images technology.
  The proposed technology provides a simple way to create complex animations 
of 3D objects including changes of positions of various objects and of their 
form. The technology does not require highly qualified users or expensive equipment. It 
can be used to create educational software and encyclopedias, commercials, videos, 
movies, as well as for implementation presentation and transmission of visual 
information solutions, remote control systems, augmented reality, user interfaces, 
decision support systems, monitoring systems, and quality control.}
  
\KWE{stereoscopy; 3D model; image processing; color characteristics; 
markers; 3D reconstruction; animation}
  
\DOI{10.14357/19922264140107}

%\Ack
%\noindent


  \begin{multicols}{2}

\renewcommand{\bibname}{\protect\rmfamily References}
%\renewcommand{\bibname}{\large\protect\rm References}

{\small\frenchspacing
{%\baselineskip=10.8pt
\addcontentsline{toc}{section}{References}
\begin{thebibliography}{9}

\bibitem{1-arh-1}
\Aue{Gruzman, I.\,S., V.\,S.~Kirichuk, V.\,P.~Kosyh, G.\,I.~Peretjagin, and 
A.\,A.~Spektor}. 2000. \textit{Tsifrovaya obrabotka izobrazheniy v informatsionnykh 
sistemakh} [\textit{Digital image processing in information 
systems}]. Novosibirsk: Novosibirsk State Technical University. 
168~p.
\bibitem{2-arh-1}
\Aue{Maniakov, Y.\,A.} 2010. Tekhnologiya registratsii povedeniya ob''ektov v 
trekhmernom prostranstve [Three-dimensional space object behavior tracking technology]. 
\textit{Informatsionnye Tekhnologii v Nauke, Obrazovanii i Proizvodstve (ITNOP): 
Materialy Konferentsii} [\textit{Information 
Technologies in Science, Education and Production: Science and Technology 
Conference (International) Proceedings}]. Orel. 3:182--186.
  
\end{thebibliography}
} }


\end{multicols}

\vspace*{-6pt}

\hfill{\small\textit{Received October 2, 2013}}

\vspace*{-18pt}

\Contr

\noindent
\textbf{Arkhipov Oleg P.} (b.\ 1948)~--- Candidate of Science (PhD) in technology,
Director, Orel Branch of Institute of Informatics Problems, Russian
Academy of Sciences, 137 Moskovskoe Highway, Orel 302025, Russian Federation;
arkhipov12@yandex.ru


\vspace*{3pt}

\noindent
\textbf{Maniakov Yury A.} (b.\ 1984)~--- Candidate of Science (PhD) 
in technology,
scientist, Orel Branch of Institute of Informatics Problems, Russian
Academy of Sciences, 137 Moskovskoe Highway, Orel 302025, Russian Federation;
maniakov\_yuri@mail.ru

\vspace*{3pt}

\noindent
\textbf{Sirotinin Denis O.} (b.\ 1984)~--- Candidate of Science (PhD) 
in technology, scientist, Orel Branch of Institute of Informatics Problems, Russian
Academy of Sciences, 137 Moskovskoe Highway, Orel 302025, Russian Federation;
vespert@mail.ru


 \label{end\stat}
 
\renewcommand{\bibname}{\protect\rm Литература}



  