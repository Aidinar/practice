\def\stat{vasil}

\def\tit{ИСПОЛЬЗОВАНИЕ ПРИНЦИПА РАВНОВЕСИЯ ДЛЯ~УПРАВЛЕНИЯ 
МАРШРУТИЗАЦИЕЙ В~ТРАНСПОРТНЫХ СЕТЯХ}

\def\titkol{Использование принципа равновесия для~управления 
маршрутизацией в~транспортных сетях}

\def\autkol{Н.\,С.~Васильев}

\def\aut{Н.\,С.~Васильев$^1$}

\titel{\tit}{\aut}{\autkol}{\titkol}

%{\renewcommand{\thefootnote}{\fnsymbol{footnote}} \footnotetext[1]{Работа 
%выполнена при финансовой поддержке РФФИ (проект 11-01-00515а).}}

\renewcommand{\thefootnote}{\arabic{footnote}}
\footnotetext[1]{Московский государственный технический университет им.\ Н.\,Э.~Баумана, nik8519@yandex.ru} 
   
    
  
  \Abst{Выбор алгоритмов управления передачей должен основываться на 
принципах функциональной эффективности (увеличить быстродействие сети) и 
устойчивости передачи (принцип равновесия). В~сетях передачи данных имеется 
огромное число тяготеющих пар пользователей, каждая из которых заинтересована 
в быстроте доставки своих сообщений. Таким образом, качество функционирования 
сети необходимо оценивать с помощью векторного критерия. (Существуют также и 
другие характеристики сетей.) Поэтому проектирование системы управления 
передачей осуществляется с учетом векторных целевых функций, а принимаемые 
(реализуемые) решения должны искаться методами векторной оптимизации. 
  Стремление улучшить качество передачи с целью наилучшего (по возможности) 
удовлетворения пользователей сети стимулирует поиск новых методов 
маршрутизации сообщений. В~работе предложен метод маршрутизации, 
основанный на применении игрового принципа равновесия (по Нэшу). Игровая 
постановка задачи маршрутизации и указанное понятие решения (равновесие) 
формализуют пред\-став\-ле\-ние об оптимальности управления передачей в 
распределенной системе.
  Использование принципа равновесия предполагает наличие ответа на следующие 
главные вопросы: всегда ли равновесие достижимо, устойчиво ли оно и как его 
найти. При общих предположениях в работе доказано существование равновесия по 
Нэшу. Установлено, что равновесие обладает дополнительными свойствами~--- 
вычислительной устойчивостью и эффективностью (оптимальностью) в смысле 
Парето. Предложен быстрый параллельный (игровой) алгоритм поиска равновесной 
маршрутизации и обоснована его сходимость.} 
    
    \KW{пакетная сеть; потоки в сетях; метрика сети; маршрутизация; векторный 
критерий; многокритериальная оптимизация; игровая задача; равновесие по Нэшу; 
эффективность по Парето}
  
  
  \DOI{10.14357/19922264140104}

\vskip 20pt plus 9pt minus 6pt

      \thispagestyle{headings}

      \begin{multicols}{2}

            \label{st\stat}


  \section{Введение}
   
  Экспоненциальный рост объемов передаваемой по сетям информации 
стимулирует исследования, связанные с совершенствованием не только сетевого 
оборудования, но и алгоритмов управления передачей в пакетных сетях. В~таких 
сетях каждая тяготеющая пара пользователей заинтересована в наискорейшей 
передаче своих сообщений за счет выбора оптимальных маршрутов доставки. 
В~глобальных сетях невозможно обеспечить централизованное управление 
маршрутизацией. Поэтому применяются распределенные параллельные алгоритмы 
(сетевые протоколы). 
  
  В каждый момент времени в сети имеется огромное число тяготеющих пар. 
Значит, задача\linebreak оп\-тимального управ\-ле\-ния передачей (задача марш-\linebreak рутизации~[1]) 
является многокритериальной. Требуется найти маршрутизацию, наилучшим 
образом удовле\-тво\-ря\-ющую всех пользователей сети. Сетевые задачи с векторными 
критериями ранее исследовались, например, в работах~[2--4]. 
  
  В статье поставлена и решена \textit{игровая} многокритериальная 
оптимизационная задача маршрутизации. Одним из подходов к решению задачи 
оптимизации векторного критерия является его сворачивание в скалярный 
критерий. Недостатком этого подхода является то, что распределенная модель 
системы заменяется на централизованную. В~результате строящиеся 
алгоритмы поиска решения не обладают той степенью параллелизма, которая 
допускает распределенную реализацию. 
  
  Переход к задаче математического программирования (в случае 
дифференцируемости целевой функции) позволяет использовать градиентный 
метод поиска оптимального решения, принимаемого в качестве решения исходной 
задачи. Так вы\-нуж\-де\-ны поступать, когда не удается найти подходящие (быстрые) 
алгоритмы поиска решения исходной многокритериальной задачи. 

Указанный 
подход применен в работе~[1] для решения задачи маршрутизации. 
  
  В статье предложен метод сведения игровой задачи оптимизации к 
\textit{эквивалентной} задаче математического программирования с целью 
построения игрового (параллельного) алгоритма поиска решения (равновесия) 
исходной задачи. Этот подход основан на введении новой метрики сети, 
модифицирующей имеющуюся. После этого на итерациях алгоритма поочередно 
для любой пары абонентов сети строится набор маршрутов передачи, 
оптимизирующих время доставки каждого сообщения. Эти вычисления основаны 
на использовании принципа уравнивания Ю.\,Б.~Гермейера, применяемого при 
решении минимаксных задач. 
  
  В отличие от градиентного метода, вычисления маршрутов передачи проводятся 
поочередно для каждой тяготеющей пары в отдельности. Это позволяет реализовать 
алгоритм так, чтобы изменение маршрутизации сети проводить одновременно 
(независимо) для многих пар абонентов.
  
  Управление потоками в пакетных телекоммуникационных системах 
(транспортных сетях~--- ТС) основано на моделях сетей с переменной 
  метрикой~[1--10]. Изменяемая метрика присуща даже однопродуктовой сети 
(имеется единственная пара абонентов) из-за наличия обратной связи между 
потоками и задержкой в передаче пакетов. Напомним, что задержки в линиях связи 
определяют метрику, с помощью которой оценивается быстродействие сети. 
  
  Игнорирование указанной обратной связи при построении параллельных 
алгоритмов маршрутизации не позволяет обеспечить устойчивое управ\-ле\-ние 
потоками ТС. Это наблюдается даже в сетях, имеющих кольцевую архитектуру. 
Так, поочередный выбор кратчайшего маршрута в текущей мет\-ри\-ке сети для 
передачи сообщений между всеми тяготеющими парами может приводить к 
возникновению колебательного процесса~[1]. В~результате в сети возникают 
потоки, вызывающие ее перегрузку, хотя имеется маршрутизация, при которой сеть 
справляется с заданными входными потоками. Для поиска соответствующей 
допустимой маршрутизации сети достаточно применить адекватный (а не 
эвристический, как в~[1]) игровой алгоритм маршрутизации. 
  
  Свойства ТС с переменной метрикой~\cite{10-vasil} ра-\linebreak нее изучались в связи с 
поиском равновесной марш\-рутизации глобальной пакетной сети передачи\linebreak 
данных~\cite{2-vasil, 4-vasil, 7-vasil}. При этом существование равновесной 
маршрутизации удавалось теоретически обосно\-вать лишь для сетей с топологией, 
мало отличающейся от кольцевой. При численном моделировании удавалось 
строить равновесную маршрутизацию для весьма широкого класса сетей.\linebreak Так, в 
результате проведения вы\-чис\-ли\-тель\-ных\linebreak экспериментов равновесие по Нэшу 
достигалось в модели сети Интернет~\cite{7-vasil}. Таким образом, чис\-лен\-ный 
\mbox{поиск} равновесного решения задачи уже прошел экспериментальную апробацию, но 
не получил должного теоретического обоснования. 
  
  Данная статья посвящена доказательству общей теоремы существования 
равновесия и установлению его свойств~--- вычислительной устойчивости и 
эффективности по Парето. 
  
  Доказана сходимость алгоритма поиска равновесной маршрутизации. Простота и 
параллельные свойства алгоритма позволяют надеяться на его применение при 
создании новых сетевых протоколов транспортного уровня~[1], основанных на 
использовании принципа равновесия. 
  
  \section{Оптимизационная модель транспортной системы}
  
  Топологию ТС будем представлять в виде связного неориентированного графа 
$\Gamma\hm = (U,V)$, вдоль ребер $l\hm\in V$, $l\hm= 1, 2,\ldots , n$, которого 
расположены линии (каналы) передачи пакетов, а в узлах $u\hm\in U$ размещены 
источники и стоки передаваемых потоков. (Требование неориентированности графа 
несущественно.) 
  
  Доставка сообщений для каждой \textit{тяготеющей} $k$-й пары 
  (ис\-точ\-ник--сток), $k\hm=1, 2, \ldots ,K$, осуществляется по одному или 
нескольким выбираемым маршрутам графа сети $\{L_j^k\}$, соединяющим эти 
узлы. Входные (случайные) потоки интенсивности $\lambda_k\hm=\lambda_0^k$ 
поступают в узлы-источники, разделяются в них (алгоритмом маршрутизации) на 
маршрутные потоки величины $\{\lambda_j^k\}$ и по маршрутам $\{L_j^k\}$ 
передаются в соответствующие уз\-лы-сто\-ки, из которых покидают~ТС. 
  
   Функционирование сети происходит с задержками на линиях сети $l\hm=1, 2, 
\ldots , n$, равными значениям некоторой функции $f_l(z_l)$, зависящей от величин 
интенсивностей потоков~$z_l$ на этих ли\-ниях.
   
  Например, в пакетных сетях передачи данных величина задержки всякого пакета 
на любой линии складывается из следующих величин~\cite{1-vasil, 8-vasil}:
 \begin{itemize}
\item времени ожидания пакета в очереди; 
\item времени определения направления дальнейшей передачи (в транзитный 
узел сети) с помощью маршрутной таблицы; 
\item времени пересылки пакета по выбранной линии.
\end{itemize}

  Определение функций задержек составляет самостоятельную 
  задачу~\cite{1-vasil, 8-vasil, 9-vasil}. Во всяком случае, эти функции 
неотрицательны, монотонно и неограниченно возрастают при увеличении 
интенсивности потока по линии до величины ее пропускной способности. 
(Согласно теории массового обслуживания при совпадении интенсивностей 
поступления и обслуживания заявок наблюдается неограниченный рост очередей.)
  
  Время доставки продуктов вдоль маршрута~$L$ (или <<длина>> маршрута~$L$) 
равно сумме задержек:
  \begin{equation}
  \rho_f (L,z)=\sum\limits_{l\in L} f_l(z_l)\,.
  \label{e1-vasil}
  \end{equation}
  
  Зафиксировав векторную функцию $f\hm=(f_1,f_2,\ldots , f_n)$, опустим 
индекс~$f$ в обозначении метрики сети~(\ref{e1-vasil}). Потоки в сети задаются в 
единицах измерения интенсивности передачи. По смыслу использованных 
обозначений для всех допустимых значений индексов~$k$ и~$l$ должны 
выполняться балансовые соотношения
  \begin{equation}
  \sum\limits_{j=1}^{J_k} \lambda_j^k=\lambda_0^k\,,\enskip \sum\limits_{j,k\in 
L_j^k} \lambda_j^k=z_l\,.
  \label{e2-vasil}
  \end{equation}
    Через $z_l$ в~(\ref{e2-vasil}) обозначен суммарный поток по\linebreak $l$-й линии ТС, 
который ограничен величиной $\overline{z}_l$~--- пропускной способностью 
линии:
  \begin{equation}
  z_l\leq \overline{z}_l\,,\quad l=1, 2, \ldots ,n\,.
  \label{e3-vasil}
  \end{equation}
  
  Под \textit{допустимой маршрутизацией} (ДМ) сети (для $k$-й тяготеющей 
пары) будем понимать совокупность маршрутов $M^k\hm=\{L_j^k\}$ и 
маршрутных потоков $\{\lambda_j^k\}$ (интенсивностей передачи) для всех 
тяготеющих пар $k\hm=1, 2, \ldots , K$ такую, что выполняются 
соотношения~(\ref{e2-vasil}), (\ref{e3-vasil}) и $\rho(L_j^k,z)\hm<\infty$. 
  
  Векторный входной поток $(\lambda_0^1, \lambda_0^2, \ldots , \lambda_0^K)$ 
называется \textit{допустимым}, если для него найдется~ДМ.
  
  Задавая маршрутизацию, будем перечислять только \textit{применяемые} 
маршруты передачи, для которых маршрутные потоки положительны. В~процессе 
управ\-ле\-ния ТС за счет (принудительного) ограничения входного потока 
обеспечивается его допустимость. Выбор ДМ
  $$
  M=\left\{ \left\{ L_j^k\right\}, \left\{ \lambda_j^k\right\}, k=1, 2, \ldots ,K\right\}
  $$
однозначно задает вектор допустимых потоков по линиям ТС $z\hm= (z_1, z_2, 
\ldots , z_n)$, от которого зависит время передачи сообщений (вместе с 
вектором~$\mathbf{z}$ изменяется метрика сети~(\ref{e1-vasil})). 
  
  Любая $k$-я тяготеющая пара <<заинтересована>> в уменьшении времени 
доставки пакетов, определяемого длинами применяемых маршрутов 
передачи~(\ref{e1-vasil}). В~сети с потоками~$\mathbf{z}$ минимально возможное 
время доставки пакетов $k$-й пары равно
  $$
 \underline{T}^k(z)=\min\limits_{\{L^k\}} \rho(L^k,z)\,,
  $$
  где $\{L^k\}$~--- множество всех маршрутов, соединя\-ющих эту пару.
  
  Маршрутизацию $k$-й тяготеющей пары назовем \textit{оптимальной}, если все 
применяемые маршруты передачи имеют минимальную длину, равную 
$\underline{T}^k(z)$.
  
  Так как вектор~$\mathbf{z}$ зависит от выбираемой маршрутизации 
(см.~(\ref{e2-vasil})), то оптимальным решением этой задачи для $k$-й пары 
является выбор ее кратчайших маршрутов соединения, длина которых оценивается 
с помощью изменяющейся (вместе с сетевыми потоками) 
  мет\-ри\-кой~(\ref{e1-vasil})~\cite{5-vasil, 6-vasil, 10-vasil}.
  
  Далее исследуется \textit{игровая} задача об оп\-ти\-мальной маршрутизации ТС. 
Тяготеющие пары\linebreak $k\hm= 1, 2, \ldots , K$ рассматриваются как игроки в 
бескоалиционной игре, выбирающие свою стратегию~--- 
  маршрутизацию~\cite{2-vasil, 4-vasil, 7-vasil}.
  
  Равновесие по Нэшу в этой игре назовем \textit{равновесной} (оптимальной) 
маршрутизацией ТС. Соответствующий вектор потоков в сети также будем 
называть \textit{равновесным}. 
  
  В равновесии каждой тяготеющей паре \textit{невыгодно} отклоняться от своей 
маршрутизации из-за того, что время передачи потоков только увеличится при 
условии, что все остальные пары придерживаются своих маршрутов и 
интенсивностей передачи. В~этом заключается устойчивость (в игровом смыс\-ле) 
равновесного решения.
  
  На практике выбор маршрутизации может проводиться не самими тяготеющими 
парами, а с по\-мощью некоторого алгоритма, построенного на этом принципе. 
(Таковы сетевые протоколы~[1].)
  
  \section{Существование равновесия}

  Введем вспомогательную задачу математического программирования с целью 
замены более трудной игровой задачи маршрутизации на однокритериальную 
оптимизационную задачу. 
  
  Известные стандартные методы сведения обычно приводят к сложным 
многоэкстремальным задачам, в которых целевые функции не являются 
дифференцируемыми, даже если исходная векторная целевая функция была 
дифференцируемой. 
  
  Особенности строения сетевых критериев ка\-чества позволяют предложить 
следующий способ сведения исходной игровой задачи к выпуклой 
оптимизационной задаче, решить которую проще, чем исходную проблему. 
  
  Определим новые функции задержек 
  $$
  t_l=t_l(z_l)\,,\quad l=1,2, \ldots ,n\,,
  $$
решив набор не связанных между собой одномерных задач Коши для 
обыкновенного дифференциального уравнения (ОДУ) первого порядка:
\begin{equation}
z_l \fr{dt_l}{dz_l}+t_l =f_l(z_l)\,,\quad t_l(0)=0\,.
\label{e4-vasil}
\end{equation}
Пусть $Z$~--- многогранник допустимых потоков по линиям передачи. Ввиду 
равенств~(\ref{e2-vasil}) и (\ref{e3-vasil}) $Z$~--- выпуклый ограниченный 
многогранник. Рассмотрим следующую экстремальную задачу:
\begin{equation}
F(z)=\sum\limits_{l=1}^n z_l t_l(z_l)\to \min\limits_{z\in Z}\,.
\label{e5-vasil}
\end{equation}
  
  \noindent
  \textbf{Теорема~1.}\ \textit{Пусть все функции задержек монотонно возрастают, 
дифференцируемы и}
  $$
  (\forall\ l) f_l(z)\to \infty\,,\quad z\to \overline{z}_l\,.
  $$
  \textit{Тогда если $Z\not= \emptyset$, то минимум целевой 
  функции~$(\ref{e5-vasil})$ достигается в единственной точке~$z^*$, которой 
отвечает некоторая равновесная маршрутизация.}
  
  \medskip
  
  \noindent
  Д\,о\,к\,а\,з\,а\,т\,е\,л\,ь\,с\,т\,в\,о\,.\ \ С~помощью критерия Сильвестра проверим 
положительную определенность матрицы Якоби целевой функции $F(z)$. Условия 
теоремы и определение этой функции (см.~(\ref{e4-vasil}) и (\ref{e5-vasil})) дают: 
  \begin{gather*}
  \fr{\partial^2 F}{\partial z_l^2} = \left( z_l t_l^\prime(z_l) +t_l (z_l)\right)^\prime 
=f_l^\prime(z_l)>0\,;\\
  \fr{\partial^2 F}{\partial z_{l_1} \partial z_{l_1}}=0\,,\quad l_1\not= l_2\,.
  \end{gather*}
(штрихом обозначено дифференцирование). Все условия этого критерия 
выполнены, поэтому целевая функция $F(z)$ строго выпукла. В~условиях теоремы 
это позволяет сделать вывод о том, что решение выпуклой экстремальной 
задачи~(\ref{e5-vasil}) существует и единственно. 

  Докажем, что $z^*$~--- точке минимума~(\ref{e5-vasil})~--- отвечает равновесная 
маршрутизация. Для этого применим теорему Ку\-на--Так\-ке\-ра~\cite{11-vasil}, 
которая в данном случае служит критерием оптимальности потока~$z^*$. В~записи 
условий оптимальности учтем, что ограничения~(\ref{e3-vasil}) неактивны. 
(Согласно наложенным предположениям элемент~$z^*$ удовлетворяет строгим 
неравенствам в~(\ref{e3-vasil}).) 
  
  Выразим произвольный допустимый вектор~$z$ в виде 
$z\hm=\mathbf{A}\lambda$, где вектор $\lambda\hm=(\lambda_j^k)$ входит в 
соотношения~(\ref{e2-vasil}). Через~$\mathbf{A}$ обозначена 
  ($n\times J$)-мат\-ри\-ца всех маршрутов, соединяющих рассматриваемую 
тяготеющую пару. Напомним, что $\mathbf{A}$~--- это (0,\,1)-мат\-ри\-ца 
инциденций реб\-ра--марш\-ру\-ты. Маршруты передачи представлены столбцами 
матрицы~$\mathbf{A}$. Указанное представление вектора потоков $z\hm= 
\mathbf{A}\lambda$ всегда возможно: для неприменяемых маршрутов $L_j^k$ 
полагаем $\lambda_j^k\hm=0$. 
  
  Условия оптимальности из теоремы Ку\-на--Так\-ке\-ра в 
  задаче~(\ref{e2-vasil})--(\ref{e5-vasil}) представляют собой систему соотношений, 
распадающуюся на подсистемы, описывающие оптимальные решения для 
отдельных тяготеющих пар. Поэтому рассмотрим произвольную пару~$k$, 
зафиксировав маршрутизацию остальных пар. Для упрощения записи в 
обозначениях опустим верхний индекс~$k$. Итак, выпишем функцию 
Лагранжа~\cite{11-vasil, 12-vasil}:
  $$
  H(\mu, \lambda) =F(A\lambda)+\left\langle \mu,\lambda_0 -\sum\limits_{j=1}^J 
\lambda_j \right\rangle\,,\quad \lambda\geq 0\,.
  $$
  
  Тогда критерий оптимальности маршрутизации, определяемой вектором
  $$
  \lambda=\lambda^*\,;\quad z^*=A\lambda^*\,,
  $$
для $k$-й тяготеющей пары принимает следующий вид:
\begin{equation}
\left.
\begin{array}{c}
\nabla_\lambda H= fA -\mu (1, \ldots , 1) \geq 0\,;\\[9pt]
(\forall \ j) \lambda_j ((fA)_j-\mu)=0\,,
\end{array}
\right\}
\label{e6-vasil}
\end{equation} 
причем $f=(f_1, f_2, \ldots , f_n)$~--- вектор задержек в~(\ref{e6-vasil})~--- 
вычислен в точке~$z^*$. 

  Если $\lambda_j\not= 0$, то из~(\ref{e6-vasil}) следует равенство $(fA)_j\hm=\mu$. 
Если $\lambda_j\hm=0$, то справедливо неравенство $(fA)_j\hm\geq \mu$. По 
определению матрицы маршрутов и в соответствии с определением 
метрики~(\ref{e1-vasil}) имеем
  $$
  (fA)_j =\rho (L_j, z^*)\,.
  $$
  
  Полученные соотношения означают то, что все маршруты, применяемые 
произвольной $k$-й парой, являются кратчайшими, причем их длина равна~$\mu$. 
Следовательно, решение задачи маршрутизации, отвечающее вектору 
потоков~$z^*$, равновесно. В~соответствии с определением равновесия доказана 
оптимальность рассматриваемой маршрутизации для всех тяготеющих пар.
  
  \medskip
  
  \noindent
  \textbf{Пример 1.} При аппроксимации функций задержек~$f_l$ степенными 
функциями задержки~$t_l$, найден\-ные согласно уравнениям~(\ref{e4-vasil}), с 
точностью до постоянного множителя совпадают с~$f_l$. Таким образом, новая 
метрика сети~$\rho_t$ несущественно отличается от~$\rho_f$.
  
  \medskip
  
  \noindent
  \textbf{Пример 2.} В случае дроб\-но-ли\-ней\-ной аппроксимации
  $$
  f(z)=\fr{az}{\overline{z}-z}\,,\enskip 0\leq z< \overline{z}\,,
  $$
функции задержек вспомогательная метрика сети~$\rho_t$ заметно отличается от 
исходной, так как функция $t\hm=t(z)$, удовлетворяющая 
уравнению~(\ref{e4-vasil}), имеет вид: 
$$
t(z) =-a\left( 1+\fr{\overline{z}}{z}\,\ln (\overline{z}-z)\right)\,,\enskip 0\leq 
z<\overline{z}\,.
$$
  
  \section{Эффективность по~Парето равновесной маршрутизации}
  
   На множестве допустимых маршрутизаций введем отношение эквивалентности. 
Маршрутизации $M_1$ и $M_2$ (всей сети или какой-нибудь отдельной тяготеющей 
пары) \textit{эквивалентны}, если они приводят к одному и тому же вектору 
потоков~$z$ на линиях сети. Класс эквивалентности маршрутизации~$M$ будем 
обозначать $R_z(M)$. 
  
  Для упрощения записи будем опускать индекс~$z$, а множество применяемых 
  $k$-й парой маршрутов $\{L_j^k\}$ при маршрутизации~$M$ обозначать~$M^k$.
  
  Из проведенного доказательства теоремы~1 вытекает
  
  \smallskip
  
  \noindent
  \textbf{Следствие~1.} Равновесной является всякая маршрутизация, которая 
эквивалентна равновесной маршрутизации.
  
  Будем считать, что в сети с потоками~$z$ может реализоваться произвольная 
эквивалентная маршрутизация. (Это предположение имеет место в пакетных сетях, 
в которых управление передачей осущест\-вля\-ет\-ся с помощью маршрутных 
  таб\-лиц~\cite{1-vasil}.) 
  
  Тогда времена доставки отдельных пакетов для $k$-й тяготеющей пары 
абонентов вычисляются по следующей формуле:
  \begin{multline}
  T^k(z;M^k) =\max\limits_{M^\prime\in R_z(M^k)} \max\limits_{L_j^k\in M^\prime} 
\rho(L_j^{\prime\,k}, z)\,, \\ k=1,2,\ldots, K\,.
  \label{e7-vasil}
  \end{multline}
  Подстановка в~(\ref{e7-vasil}) равновесной маршрутизации дает значения 
критериев~(\ref{e7-vasil}), совпадающие с длинами применяемых (кратчайших) 
маршрутов (следствие~1).
  
  \medskip
  
  \noindent
  \textbf{Теорема~2.}\ \textit{Пусть все пары узлов графа сети являются 
тяготеющими и для любых попарно смежных ребер графа сети~$k, l, m$ выполнены 
неравенства треугольника
  $$
  f_k(0)\leq f_l(0)+f_m(0)\,.
  $$
  Тогда равновесная маршрутизация эффективна по Парето.} 
  
  \medskip
  
  \noindent
  Д\,о\,к\,а\,з\,а\,т\,е\,л\,ь\,с\,т\,в\,о\,.\ \ Рассуждая от противного, найдем такую 
допустимую маршрутизацию
  $$
  M=\left\{ \left\{ L_j^k\right\}, \left\{ \lambda_j^k\right\},\ k=1,2,\ldots , K\right\}\,,
  $$
  для которой 
  \begin{equation}
  \left(\forall k \right) T^k(z)\leq T^k(z^*)\,.
  \label{e8-vasil}
  \end{equation}
    В соотношениях~(\ref{e8-vasil}) хотя бы одно неравенство является 
строгим~\cite{13-vasil}, поэтому равновесные потоки~$z^*$ таковы, что 
$z\not=z^*$. 
  
  Из условия теоремы и неравенства треугольника следует, что в ситуации 
равновесия
  $$
  z_l^*> 0\,,\enskip l=1,2,\ldots ,n\,.
  $$
  
  Докажем справедливость неравенства
  \begin{equation}
  z_l\leq z_l^*\,,\enskip l=1,2,\ldots ,n\,.
  \label{e9-vasil}
  \end{equation}
  
  Если в маршрутизации $M$ линия связи~$l$ не применяется для передачи 
пакетов, то, очевидно, соответствующее неравенство в~(\ref{e9-vasil}) выполнено. 
Пусть теперь $z_l\hm> 0$, $l\hm=(a,b)$. Можно считать, что у тяготеющей пары 
$(a,b)$ с номером~$k$ имеется маршрут соединения
  $$
  L\equiv \left\{ (a,b)\right\} \in M^k\,.
  $$
  
  В самом деле, если это не так, то, покажем, существует маршрутизация 
$M^{\prime\,k} \hm\in R(M^k)$, обладающая этим свойством. (Тогда вместо 
маршрутизации~$M$ достаточно будет рассмотреть~$M^\prime$.)
  
  Так как $z_l>0$, то найдется такая пара~$k^\prime$, что $l\hm= (a,b)\hm\in 
L_{j^\prime}^{k^\prime}\hm\in M_{j^\prime}^{k^\prime}$. Выберем величину
  $$
  0<\Delta <\min \left\{ \lambda_j^k,\lambda_j^{k^\prime}\right\}\,,
  $$
в которой поток $\lambda_j^k$ проходит по такому маршруту~$L_j^k$, что $l\notin 
L_j^k\hm\in M^k$. Тогда часть маршрутного потока~$\lambda_j^k$ 
величиной~$\Delta$ перебросим с маршрута~$L_j^k$ на~$L$~--- новый для $k$-й 
пары маршрут соединения. Такую же величину~$\Delta$, являющуюся частью 
потока $\lambda_{j^\prime}^{k^\prime}$, направим по маршруту
$$
L^\prime =\left( L_{j^\prime}^{k^\prime} \backslash \{l\}\right)\cup L_j^k\,,
$$
новому для пары~$k^\prime$. Все остальные <<элементы>> маршрутизации~$M$ 
оставим без изменения. В~результате получена искомая 
маршрутизация~$M^\prime$, для которой $M^\prime\hm\in R(M)$.

  Согласно сделанному допущению~(\ref{e8-vasil}) относительно маршрутизаций 
$M$ и $M^*$ и определению критериев~(\ref{e7-vasil}) имеем:
\begin{multline*}
  \left(\forall\ l=1,2,\ldots , n\right) f_l(z_l) ={}\\
  {}=\rho(L,z)\leq T^k(z)\leq 
T^k(z^*)=f_l(z_l^*)\,.
\end{multline*}
  Так как функции задержек монотонно возрастают, отсюда получаем 
неравенства~(\ref{e9-vasil}), $z\not=z^*$. Целевая функция задачи~(\ref{e5-vasil}) 
монотонно воз\-рас\-та\-ет по каждой переменной~$z_l$. Тогда из~(\ref{e9-vasil}) 
следует неравенство $F(z)\hm<F(z^*)$, противоречащее тому, что $z^*$~--- 
минимум целевой функции~(\ref{e5-vasil}). Теорема~2 доказана.
  
  %\smallskip
  
  \section{Вычислительная устойчивость равновесного решения задачи 
маршрутизации} 
  
  Исходные данные модели обычно обладают некоторой неопределенностью. 
Возникает необходимость исследовать устойчивость искомого решения. 
В~рассматриваемой модели будем варьировать все параметры с помощью 
изменения $c\hm= (\lambda_0,\overline{z})$~--- вектора, составленного из величин 
входных потоков и пропускных способностей линий~ТС. 
  
  Задержки и целевая функция~(\ref{e5-vasil}) есть функции переменных $z$ и $c$, а 
многогранник потоков~--- значение многозначного отображения $c\hm\to Z(c)$. 
Пред\-полагается, что параметр~$c$ изменяется в пределах\linebreak множества~$C$ такого, 
что при справедливости включения $c\hm\in C$ выполняются все условия 
теоремы~1 и, кроме того, все функции задержек непрерывны по совокупности 
переменных. Тогда справедлива
  
  \smallskip
  
  \noindent
  \textbf{Теорема~3.} \textit{Метрика сети непрерывно зависит от параметров 
модели.}
  
  \smallskip
  
  \noindent
  Д\,о\,к\,а\,з\,а\,т\,е\,л\,ь\,с\,т\,в\,о\,.\ \ Из теоремы о непрерывной зависимости 
ОДУ~(\ref{e4-vasil}) от параметра следует, что целевая функция
  $
  F(z,c)$, $z \hm\in Z(c)$, $c\hm\in C$,  экстремальной задачи~(\ref{e5-vasil}) 
непрерывна. Анализ линейных соотношений~(\ref{e2-vasil}) и~(\ref{e3-vasil}) 
показывает, что отображение $c\hm\to Z(c)$ непрерывно по 
  Хаусдорфу~\cite{11-vasil}. По теореме~1 отсюда можно заключить, что 
оптимальное решение задачи~(\ref{e5-vasil}) $z^*\hm=z^*(c)$ непрерывно зависит 
от параметра~$c$. Таким образом, метрика сети также непрерывна как сумма 
непрерывных функций (см.~(\ref{e1-vasil})).
  
  \smallskip
  
  Непосредственным следствием теорем~1 и~2 является вывод о том, что решение 
задачи маршрутизации устойчиво к изменению параметров мо\-дели.
   
  \section{Алгоритм поиска равновесной маршрутизации}
  
  Для поиска оптимальных маршрутов передачи продуктов каждой тяготеющей 
пары будем применять следующую схему алгоритма. 
  
  Произвольно выберем начальную маршрутизацию (шаг $t\hm=0$).
  
  Пусть на шаге $t\hm=0,1,\ldots$ уже построена маршрутизация, обозначаемая 
$M^t$, $M^t\hm= \left( \left\{ L_j^k\right\},\left\{ \lambda_j^k\right\} \right)$.
  
  Ей отвечает вектор потоков~$z^t$. В~сети с фиксированной метрикой 
$\rho(L,z^t)$ найдем кратчайший маршрут~$L$. (Достаточно воспользоваться 
алгоритмом Дейкстры~\cite{14-vasil}.) Пусть также в текущей 
маршрутизации~$M^t$ существует маршрут~$L^t$, имеющий б$\acute{\mbox{о}}$льшую длину по 
сравнению с~$L$. (Иначе решение задачи уже найдено.) Тогда часть 
потока~$\lambda^t$, передаваемого по маршруту~$L^t$, перебросим на 
маршрут~$L$ с целью уменьшения разности длин этих маршрутов. (Это возможно 
ввиду монотонности функций задержек, см.~(\ref{e1-vasil}).) При этом либо 
уравняются длины маршрутов, либо маршрут~$L$ останется по-прежнему короче, а 
маршрут~$L^t$ перестанет использоваться (в случае 
$\lambda\vert_L\hm=\lambda^t$).
  
  Определим очередную маршрутизацию ТС $M^{t+1}$ как результат добавления к 
маршрутизации~$M^t$ пары $L,\lambda\vert_L$ и, возможно, исключения 
маршрута~$L^t$ в случае, когда он перестает применяться. 
  
  Указанные в алгоритме действия~--- применение \textit{принципа уравнивания} 
Ю.\,Б.~Гермейера как метода решения минимаксных задач~\cite{15-vasil}. 
  
  \smallskip
  
  \noindent
  \textbf{Теорема~4.} \textit{Последовательность маршрутизаций $\{M^t,\ 
t=0,1,\ldots\}$, построенная по принципу уравнивания, сходится к равновесному 
решению задачи.}
  
  \smallskip
  
  \noindent
  Д\,о\,к\,а\,з\,а\,т\,е\,л\,ь\,с\,т\,в\,о\,.\ \ Вычислим производную функции 
$F(A\lambda)$ в текущей точке $z^t\hm= A\lambda^t$ по на\-прав\-ле\-нию 
вектора~$\Delta$, все координаты которого равны нулю, за исключением тех, 
которые отвечают маршрутным потокам вдоль~$L$ и~$L^t$, равных 
соответственно~1 и~$-1$:
  $$
  \fr{dF(A\lambda)}{d\Delta} =\left\langle fA,\Delta\right\rangle=\rho(L,z^t)-
\rho(L^t,z^t)<0\,.
  $$
  
  Шаг градиентного метода в направлении~$\Delta$ с целью выравнивания длин 
этих маршрутов приводит к уменьшению значения целевой функции в 
задаче~(\ref{e5-vasil}):
  $$
  F(z^{t+1})< F(z^t)\,,\enskip t=0,1,\ldots
  $$
  
  Монотонная числовая последовательность $\{ F(z^t)\}$ сходится к~$F(z^*)$, где 
$z^*$~--- решение задачи~(\ref{e5-vasil}). Согласно теореме~1 $z^t\hm\to z^*$, 
$t\hm\to\infty$. По следствию~1 в пределе получаем равновесную маршрутизацию.
  
{\small\frenchspacing
{%\baselineskip=10.8pt
\addcontentsline{toc}{section}{References}
\begin{thebibliography}{99}
\bibitem{1-vasil}
\Au{Бертсекас Д., Галлагер Р.} Сети передачи данных~/ 
Пер с англ.~--- М.: Наука, 1989.
(\Au{Bertsecas~D.\,P., Gallager~R}. 
{Data networks}.~--- Englewood Cliffs: Prentice Hall, 1987. 544~p.)
\bibitem{2-vasil}
\Au{Васильев Н.\,С., Федоров В.\,В.} О~равновесной маршрутизации в сетях 
передачи данных~// Вестн. Моск. ун-та. Сер.~15. Вычисл. матем. и киберн., 1996. №\,4. 
С.~39--47.
\bibitem{3-vasil}
\Au{Korilis Y.\,A., Lazar A.\,A., Orda~A.} Capacity allocation under noncooperative 
routing~// IEEE Trans. Automat. Contr., 1997. Vol.~42. No.\,3. P.~309--325.
\bibitem{4-vasil}
\Au{Vasilyev N.\,S.} Nash equilibrium routing in ring networks~// Int. J.~Math. Game 
Theory  Algebra, 1998. Vol.~7. No.\,4. P.~221--234.
\bibitem{5-vasil}
\Au{Васильев Н.\,С.} О~свойствах решений задачи маршрутизации сети с 
виртуальными каналами~// Ж. вычисл. матем. и матем. физ., 1997. Т.~37. №\,7. 
С.~785--793.
\bibitem{6-vasil}
\Au{Васильев Н.\,С.} О~свойствах решений задачи динамической маршрутизации 
сети~// Ж. вычисл. матем. и матем. физ., 1998. Т.~38. №\,1. С.~42--52.

\bibitem{8-vasil}
\Au{Соколов И.\,А., Шоргин С.\,Я.} Модель и математические методы расчета 
характеристик сети, использующей технологии X.25 и Frame relay~// Системы и 
средства информатики. Спец. вып. Математические методы информатики.~--- М.: 
Наука, 2001. С.~43--66.

\bibitem{7-vasil}
\Au{Васильев Н.\,С., Федоров В.\,В.} О~построении алгоритмов маршрутизации 
пакетных сетей на основе векторных критериев~// Известия РАН. Теория и системы 
управления, 2005. №\,3. С.~36--47.

\bibitem{10-vasil} %9
\Au{Васильев Н.\,С.} Задача о кратчайших маршрутах в сетях с переменной 
метрикой~// Вестник МГТУ им.\ Н.\,Э.~Баумана. Сер. Естеств. науки, 2008. №\,1. 
С.~70--75.

\bibitem{9-vasil} %10
\Au{Коновалов М.\,Г.} Оптимизация работы вычислительного комплекса с помощью 
имитационной модели и адаптивных алгоритмов~// Информатика и её применения, 
2012. Т.~6. Вып.~1. С.~37--48.

\bibitem{11-vasil}
\Au{Васильев Ф.\,П.} Численные методы решения экстремальных задач.~--- М.: 
Наука, 1980.
\bibitem{12-vasil}
\Au{Иоффе А.\,Д., Тихомиров В.\,М.} Теория экстремальных задач.~--- М.: Наука, 
1974. 481~c.
\bibitem{13-vasil}
\Au{Подиновкий В.\,В., Ногин В.\,Д.} Па\-ре\-то-оп\-ти\-маль\-ные решения 
многокритериальных задач.~--- М.: Наука, 1982. 256~с.
\bibitem{14-vasil}
\Au{Кристофидес Н.} Теория графов. Алгоритмический подход~/
Пер с англ.~--- М.: Мир, 1978.
(\Au{Cristofides~N.} {Graph theory: An algorithmic approach}.~---
 London: Academic, 1975. 430~p.)
\bibitem{15-vasil}
\Au{Федоров В.\,В.} Численные методы максимина.~--- М.: Наука, 1979.  280~с.
\end{thebibliography}
} }

\end{multicols}

\hfill{\small\textit{Поступила в редакцию 25.05.13}}


\vspace*{12pt}

\hrule

\vspace*{2pt}

\hrule
    
\def\tit{EQUILIBRIUM PRINCIPLE APPLICATION TO~ROUTING CONTROL IN~PACKET DATA TRANSMISSION NETWORKS}

\def\titkol{Equilibrium principle application to routing control in packet data transmission networks}

\def\aut{N.\,S.~Vasilyev}
\def\autkol{N.\,S.~Vasilyev}


\titel{\tit}{\aut}{\autkol}{\titkol}

\vspace*{-12pt}

\noindent
Bauman Moscow State Technical University, 5, 2nd Baumanskaya Str., Moscow 
105005, Russian Federation

 
\def\leftfootline{\small{\textbf{\thepage}
\hfill INFORMATIKA I EE PRIMENENIYA~--- INFORMATICS AND APPLICATIONS\ \ \ 2014\ \ \ volume~8\ \ \ issue\ 1}
}%
 \def\rightfootline{\small{INFORMATIKA I EE PRIMENENIYA~--- INFORMATICS AND APPLICATIONS\ \ \ 2014\ \ \ volume~8\ \ \ issue\ 1
\hfill \textbf{\thepage}}}   

\vspace*{3pt}
  
\Abste{Annual exponential growth of data flows in large scale networks impels to 
search not only network hardware improvements but also more perfect routing control 
algorithms. In networks, it is impossible to use centralized algorithms of 
 routing control. Parallel algorithms choice must be based on the principles 
of functional effectiveness and stability (equilibrium). 
     In large-scale networks, there is a huge number of users' pairs trying to achieve 
     the maximally possible rate of data transmission by routing control. 
     Thus, control must be based on multicriteria optimization ideas and methods. 
    The Nash equilibrium (game formulation of the routing problem) formally presents optimality 
    of transmission control in distributed systems. In the present paper,
     the equilibrium 
    routing is proved to exist under general conditions. The solution is additionally shown to be effective in 
    Pareto sense and computationally stable. An effective (quick and parallel) game theory 
    algorithm is suggested and its convergence is proved.}
    
    \KWE{packet network; data flows; network metric; routing; vector criteria; 
    multicriteria optimization; game problem; Nash equilibrium; Pareto effectiveness}
    
    \DOI{10.14357/19922264140104}

%\Ack
%\noindent
%Работа выполнена при финансовой поддержке РФФИ (проект 11-01-00515а).

  \begin{multicols}{2}
  
  \renewcommand{\bibname}{\protect\rmfamily References}
%\renewcommand{\bibname}{\large\protect\rm References}

{\small\frenchspacing
{%\baselineskip=10.8pt
\addcontentsline{toc}{section}{References}
\begin{thebibliography}{99}
  
  \bibitem{1-vasil1}
  \Aue{Bertsecas,~D.\,P., and R.~Gallager}. 
  1987. \textit{Data networks}. Englewood Cliffs: Prentice Hall. 544~p.
\bibitem{2-vasil-1}
\Aue{Vasilyev, N.\,S., and V.\,V.~Fedorov}. 
1996. O~ravnovesnoy marshrutizatsii v setyakh peredachi dannykh 
[Equilibrium routing in data-transmission networks]. 
\textit{Vestn. Mosk. Univ. Ser.~15: Vychisl. Mat. Kibern.}
[\textit{Bulletin of Moscow State University. Ser.~15: Comput. Math., Cybern.}] 4:47--52.
\bibitem{3-vasil-1}
\Aue{Korilis, Y.\,A., A.\,A.~Lazar, and A.~Orda}. 
1997. Capacity allocation under noncooperative routing. 
\textit{IEEE Trans. Automat. Contr.} 42(3):309--325.
\bibitem{4-vasil-1}
\Aue{Vasilyev, N.\,S.} 1998. 
Nash equilibrious routing in ring networks. \textit{Int. J.~Math. Game Theory Algebra}
7(4):221--234.
{\looseness=1

}
\bibitem{5-vasil-1}
\Aue{Vasilyev, N.\,S.} 1997. 
O~svoystvakh resheniy zadachi marshrutizatsii seti s virtual'nymi kanalami 
[Properties of the solutions to the problem of routing in network with virtual channels]. 
\textit{Zh. Vychisl. Mat. Mat. Fiz.} [\textit{Computational Mathematics and Mathematical 
Physics}] 37(7):785--793.
\bibitem{6-vasil-1}
\Aue{Vasilyev, N.\,S.} 1998. O~svoystvakh resheniy zadachi dinamicheskoy 
marshrutizatsii  seti [Properties of the solutions to the problem of dynamic routing 
in networks]. \textit{Zh. Vychisl. Mat. Mat. Fiz}. 
[\textit{Computational Mathematics and Mathematical Physics}] 38(1):42--52.

\bibitem{8-vasil-1}
\Aue{Sokolov, I.\,A., and S.\,Ya.~Shorgin}. 2001. 
Model' i matematicheskie metody rascheta kharakteristik seti, 
ispol'zuyushchey tekhnologii X.25 i Frame relay 
[Models and mathematical methods of characteristics calculation for X.25 and 
Frame relay network]. \textit{Sistemy i Sredstva Informatiki. Spec. vyp. 
``Matematicheskie metody informatiki''} 
[\textit{Systems and Means of Informatics. Spec. ed. ``Math. Meth. of Informatics''}]. 
Moscow: Nauka, Fizmatlit. 43--66.

\bibitem{7-vasil-1}
\Aue{Vasilyev, N.\,S., and V.\,V.~Fedorov}. 2005. O~postroenii 
algoritmov marshrutizatsii paketnykh setey na osnove vektornykh kriteriev 
[On routing algorithms in packet networks on the base of vector criterias]. 
\textit{Izvestija RAN. Teorija i sistemy upravlenija} 
[\textit{Bulletin of RAS. Theory and Control Systems}] 3:36--47.

\bibitem{10-vasil-1}
\Aue{Vasilyev, N.\,S.} 2008. Zadacha o kratchayshikh marshrutakh v setyakh s 
peremennoy metrikoy [The shortest paths problem in networks with changeable metric]. 
\textit{Vestnik MGTU im. N.\,E.~Baumana, Ser. Estestv. Nauki} 
[\textit{Bulletin of Bauman MSTU. Ser. Natural Sciences}] 1:70--75.


\bibitem{9-vasil-1}
\Aue{Konovalov, M.\,G.} 2012. Optimizatsiya raboty vychislitel'nogo kompleksa s 
pomoshch'yu imitatsionnoy modeli i adaptivnykh algoritmov 
[Optimization of computational complex work on the base of imitational models and 
adaptive algorithms]. 
\textit{Informatika i Ee Primeneniya}~--- \textit{Inform. Appl.} 6(1):37--48.
\bibitem{11-vasil-1}
\Aue{Vasilyev, F.\,P.} 1980. Chislennye metody resheniya ekstremal'nykh zadach 
[Numerical methods for extremum problems]. Moscow: Nauka. 520~p.
\bibitem{12-vasil-1}
\Aue{Ioffe, A.\,D., and V.\,M.~Tihomirov}. 1974. 
\textit{Teoriya ekstremal'nykh zadach} [\textit{Theory of extremum problems}]. 
Moscow: Nauka. 481~p.
\bibitem{13-vasil-1}
\Aue{Podinovskij, V.\,V., and V.\,D.~Nogin}. 1982. 
\textit{Pareto-optimal'nye resheniya mnogokriterial'nykh zadach} 
[\textit{Pareto optimal solutions in multicriteria problems}].  Moscow: Nauka. 256~p.
\bibitem{14-vasil-1}
\Aue{Cristofides, N.} 1975. \textit{Graph theory: An algorithmic approach}.
 London: Academic. 430~p.
\bibitem{15-vasil-1}
\Aue{Fedorov, V.\,V.}  1979. 
\textit{Chislennye metody maksimina} [\textit{Numerical methods of maximin}]. 
Moscow: Nauka. 280~p.



\end{thebibliography}
} }


\end{multicols}

\vspace*{-6pt}

\hfill{\small\textit{Received May 25, 2013}}

\vspace*{-18pt}

\Contrl

\noindent
\textbf{Vasilyev Nikolai S.} (b.\ 1952)~--- Doctor of Science in physics and mathematics,
professor, Bauman Moscow State Technical University,
5, 2nd Baumanskaya Str., Moscow 105005, Russian Federation; nik8519@yandex.ru



 \label{end\stat}
 
 \renewcommand{\bibname}{\protect\rm Литература}  
 