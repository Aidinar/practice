\def\stat{kireev}

\def\tit{ОБ АППРОКСИМАЦИИ И СХОДИМОСТИ ОДНОМЕРНЫХ 
ПАРАБОЛИЧЕСКИХ ИНТЕГРОДИФФЕРЕНЦИАЛЬНЫХ МНОГОЧЛЕНОВ И~СПЛАЙНОВ}

\def\titkol{Об аппроксимации и сходимости одномерных 
параболических интегродифференциальных многочленов и~сплайнов}

\def\autkol{В.\,И.~Киреев, М.\,М.~Гершкович, Т.\,К.~Бирюкова}

\def\aut{В.\,И.~Киреев$^1$, М.\,М.~Гершкович$^2$, Т.\,К.~Бирюкова$^3$}

\titel{\tit}{\aut}{\autkol}{\titkol}

%{\renewcommand{\thefootnote}{\fnsymbol{footnote}} \footnotetext[1]{Работа 
%выполнена при финансовой поддержке РФФИ (проект 11-01-00515а).}}

\renewcommand{\thefootnote}{\arabic{footnote}}
\footnotetext[1]{Московский государственный горный университет, 
Vladimir-Kireyev@mail.ru} 
\footnotetext[2]{Институт проблем информатики Российской академии наук, makmg@mail.ru}
\footnotetext[3]{Институт проблем информатики Российской академии наук, yukonta@mail.ru}


 \vspace*{-6pt}    

  \Abst{Работа посвящена исследованию методов аппроксимации функций одномерными 
интегродифференциальными  многочленами (ИД-многочленами)
второй степени и построенными на их базе 
консервативными параболическими интегродифференциальными сплайнами (ИД-сплай\-на\-ми). 
  В~большинстве практических вычислительных задач точность исходных данных не 
превышает точности аппроксимации параболическими многочленами и сплайнами. 
Традиционные параболические сплайны, основанные на дифференциальных условиях 
согласования с аппроксимируемой функцией (дифференциальные сплайны), строятся со 
сдвигом узлов сплайна относительно узлов интерполяции для обеспечения устойчивости 
процесса аппроксимации, что существенно усложняет расчетные алгоритмы. Кроме того, 
традиционные дифференциальные сплайны не обладают свойством консервативности, т.\,е.\ 
не сохраняют интегральные свойства аппроксимируемых функций.
  Авторами разработаны новые параболические ИД-сплай\-ны, 
основанные на использовании интегральной невязки в качестве условия согласования 
сплайна и исходной функции. Такие сплайны являются устойчивыми без сдвига их узлов 
относительно узлов приближаемой функции и консервативными в смысле сохранения 
площадей под кривыми.
  В~статье доказаны теоремы об аппроксимации функций одномерными 
 параболическими ИД-мно\-го\-чле\-на\-ми и о сходимости построенных 
на их базе параболических ИД-сплай\-нов.
  Предложенные ИД-сплай\-ны рекомендуются для применения при 
построении математических моделей обработки данных в больших территориально 
распределенных информационных системах.}
  
  \KW{сплайн; многочлен; параболический; интегродифференциальный; аппроксимация; 
интерполяция; сглаживание; оценка погрешностей;  теорема сходимости; математическая 
модель обработки данных}

\DOI{10.14357/19922264140112}

\vskip 14pt plus 9pt minus 6pt

      \thispagestyle{headings}

      \begin{multicols}{2}

            \label{st\stat}        

\section{Введение}

  Одной из актуальных задач информатики является задача управления 
большими базами данных, постоянно пополняемыми в реальном времени. Под 
большой базой данных здесь понимается пространственно распределенная, 
неоднородная (с отличиями по структуре в узлах сети), иерархически 
организованная база данных, реализующая следующие функции: 
  \begin{itemize}
  \item  динамическое пополнение с первичной обработкой и идентификацией 
данных;
  \item мониторинг и анализ данных;
  \item поддержка параллельного осуществления процессов внутренней 
обработки данных и поиска ответов на внешние запросы, а также подготовки 
результатов для принятия решений.
  \end{itemize}
  
  Очевидно, что для успешной работы в реальном времени с базами данных, 
обладающими такими свойствами, необходимы специальные математические 
методы, сокращающие трудоемкость вы\-чис\-ли\-тель\-ных процессов без заметного 
ущерба для точности результатов. В~ряде случаев для такой цели весьма 
полезными оказываются сплайны. Это обусловлено тем, что сплайны могут с 
требуемой точностью восстанавливать кривые и поверхности, сохраняя их 
специфические особенности, что особенно важно, когда отсутствует строгая 
математическая модель описания поведения кривой или поверхности. 
  
  Применение сплайнов целесообразно при оптимизации функционирования 
больших баз данных, например, в задачах~[1]:\\[-14pt] 
  \begin{itemize}
  \item идентификации информационных объектов, составляющих базы 
данных, кластеризации данных при выработке суждений о степени их близости 
между собой;\\[-14pt] 
  \item поиска зависимостей между информационными объектами;
  \item оптимизации процессов передачи данных по каналам связи;
  \item построения интерполяционных и сгла\-жи\-ва\-ющих кривых экспертных 
оценок, участ\-ву\-ющих в подготовке результатов для принятия решений.
  \end{itemize}
  
  С~этой целью авторами был разработан и реализован особый вид 
ИД-сплайнов. В~статье приводятся теоретические 
результаты исследования поведения таких сплайнов, оценивается погрешность 
аппроксимации, доказывается сходимость и устойчивость сплайнов.
  
  В настоящее время наиболее развиты и математически обоснованы методы 
аппроксимации\linebreak кубическими сплайнами, которые являются устойчивыми, а по 
способу построения~--- дифференциальными~[2--4]. Однако в большинстве 
практических вычислительных задач точность исходных данных не превышает 
точности аппроксимации параболическими многочленами и сплайнами. 
  
  Для обеспечения устойчивости традиционных дифференциальных 
параболических сплайнов приходится осуществлять сдвиг узлов сплайна 
относительно узлов аппроксимируемой функции~[5], что усложняет расчетные 
алгоритмы. Кроме того, все традиционные дифференциальные сплайны, 
основанные на дифференциальных условиях согласования с аппроксимируемой 
функцией, не обладают свойством консервативности, т.\,е.\ не сохраняют 
интегральные свойства аппроксимируемых функций.
  
  В работах~[6, 7] и др.\ авторами предложены 
ИД-мно\-го\-чле\-ны и ИД-сплай\-ны четных степеней, 
обладающие свойством консервативности, которое обеспечивается 
использованием интегрального условия согласования при их построении. 
Интегродифференциальный способ приближения широко используется при 
конструировании численных методов~[8]. 
  
  В данной статье, являющейся развитием работ~[6, 7], доказана теорема об 
оценке погрешностей аппроксимации функций различных классов гладкости 
параболическими ИД-мно\-го\-чле\-на\-ми и теорема сходимости 
параболических слабосглаживающих ИД-сплай\-нов $S_{2\mathrm{ID}}(x)$ дефекта~1.
  
  Под  алгебраическим ИД-многочленом (в общем 
случае эрмитова типа) $S_{r,i}(x)$ степени~$r$, аппроксимирующим на отрезке 
$[x_i,x_{i+1}]$ некоторую сеточную функцию $f_i\hm=f(x_i)$, заданную на 
$[a,b]$ на сетке несовпадающих узлов
  \begin{multline}
  \Delta_1:\ a=x_0<x_1<\cdots < x_i<x_{i+1}<\cdots \\
{}  \cdots< x_{n-1}<x_n=b\,,
  \label{e1-kir}
  \end{multline}
понимается функция вида 
\begin{equation}
S_{r,i}(x)=\sum\limits_{k=0}^r a_{k,i}(x-x_i)^k\,.
\label{e2-kir}
\end{equation}
  Здесь $a_{k,i}$~--- коэффициенты, определяемые из совокупности 
интегральных и дифференциальных условий согласования:
  \begin{align}
  \delta S_{r,i}^{(-1)}(x_i,x_{i+1}) &= \int\limits_{x_i}^{x_{i+1}} \left[ 
S_{r,i}(x) - f(x)\right]\,dx =0\,;\label{e3-kir}\\
  \delta S_{r,i}^{(p_1)}(x_k) &= S_{r,i}^{(p_1)}(x_k) -f^{(p_1)}(x_k) =0\notag\\ 
&\hspace*{25mm}(k=i,i+1)\,,\label{e4-kir}
  \end{align}
где $p_1$~--- порядки производных, принимающие целые значения.

  Количество условий~(\ref{e4-kir}) (с различными~$p_1$) определяются 
степенью многочлена.
  
  Тогда функция $S_r^{[q]}(x)\hm= \mathop{\cup}\limits_{i=0}^{n-1} S_{r,i}(x)$, 
определенная на отрезке $[a,b]$ и принадлежащая классу гладкости 
$C^m_{[a,b]}$, составленная из звеньев $S_{r,i}(x)$, называется 
\textit{одномерным алгебраическим ИД-сплайном 
степени~$r$ дефекта}~$q$ ($0\hm\leq m\hm\leq r$, $q\hm=r\hm-m$) с узлами на 
сетке~$\Delta_1$~(\ref{e1-kir}), если каждое звено сплайна при $x\hm\in 
[x_i,x_{i+1}]$ ($i\hm=0, \ldots , n-1$) представляется в виде 
многочлена~(\ref{e2-kir}) с коэффициентами~$a_{k,i}$, определяемыми из 
совокупности интегральных и дифференциальных условий 
согласования~(\ref{e3-kir}), (\ref{e4-kir}) (где $p_1$~--- целое, $0\hm\leq 
p_1\hm\leq m$) и условий непрерывности сплайна $S_r^{[q]}(x)$ и его 
производных во внутренних узлах сетки~$\Delta_1$:
  $$
  S_{r,i-1}^{(p_2)}(x)\big\vert_{x=x_i} = S_{r,i}^{(p_2)}(x)\big\vert_{x=x_i}\ 
(i=1,\ldots, n-1)\,,
  $$ 
где $0\leq p_2\leq m$~--- порядки производных, такие что $\{p_1\}\cap 
\{p_2\}=\emptyset$ и $\{p_1\}\cup\{p_2\}\hm=\{0,1,2, \ldots ,m\}$.
  
  Ранее показано~\cite{6-kir}, что условия согласования 
  \begin{equation*}
  \int\limits_{x_i}^{x_{i+1}} S_{2,i}(x)\,dx=I_i^{i+1}\,,
\end{equation*}
где 
$$ I_i^{i+1} 
= \int\limits_{x_i}^{x_{i+1}}f(x)\,dx\,;
$$
$$
  S_{2,i}(x_k) = f_k\,;$$
  $$
  f_k=f(x_k)\ (k=i,i+1)\,,
  $$
определяют параболический интерполяционный ИД-мно\-го\-член на 
$[x_i,x_{i+1}]$:

\noindent
\begin{multline}
S_{2\mathrm{ID},i}(x) =f_i+\left( \fr{6\nabla I_i^{i+1}}{h^2_{i+1}}-\fr{2\Delta 
f_i}{h_{i+1}}\right) \left( x-x_i\right) + {}\\
{}+\left( -\fr{6\nabla I_i^{i+1}}{h^3_{i+1}}+ 
\fr{3\Delta f_i} {h^2_{i+1}} \right) \left( x-x_i\right)^2\,,
\label{e5-kir}
\end{multline}
где $h_{i+1}=x_{i+1}-x_i$; $\nabla I_i^{i+1}\hm= I_i^{i+1}\hm-f_i h_{i+1}$; 
$\Delta f_i\hm= f_{i+1}\hm- f_i$.
  
  Далее в разд.~2 проведены исследования погрешностей аппроксимации 
функций различных классов гладкости с помощью ИД-мно\-го\-чле\-нов 
$S_{2\mathrm{ID},i}(x)$. В~разд.~3 доказана теорема сходимости параболического 
  ИД-сплай\-на минимального дефекта ($q\hm=1$). В~разд.~4 перечислены 
основные выводы по данной работе.

\section{Оценки погрешностей аппроксимации сеточных~функций 
интегродифференциальными многочленами} 
  
  Проведем оценку погрешностей аппроксимации функций $f(x)$ различных 
классов глад\-кости~$m$ с помощью ИД-мно\-го\-чле\-нов 
$S_{2\mathrm{ID},i}(x)$~(\ref{e5-kir}) в предположении, что параметры $I_i^{i+1}$, 
$f_i$, $f_{i+1}$ многочлена $S_{2\mathrm{ID},i}(x)$ известны точно или вычислены с 
точ\-ностью не ниже $O(h^{m+2})$ для $I_i^{i+1}$ и $O(h^{m+1})$ для $f_i$, 
$f_{i+1}$ (в этом случае порядок аппроксимации максимален).
  
  В пространстве непрерывных на отрезке $[a,b]$ функций используется норма 
$$
\| g(x)\|_{[a,b]} \hm = \max\limits_{x\in [a,b]} \vert g(x)\vert\,.
$$ 

Обозначим через 
$$
R^{(p)}(x)\hm= S_r^{(p)}(x)\hm- f^{(p)}(x)
$$ 
($p\hm=0, 1$~--- 
порядок производной) остаточное слагаемое аппроксимации. Тогда на 
частичном отрезке $[x_i,x_{i+1}]$ в качестве погрешности принимается\linebreak норма:
  $$
  \parallel \!R^{(p)} (x)\!\parallel_{[x_i,x_{i+1}]} = \max\limits_{x\in [x_i,x_{i+1}]} 
\vert R^{(p)}(x)\vert\,.
  $$
  
  Многочлен $S_{2\mathrm{ID},i}(x)$~(\ref{e5-kir}) в форме Лагранжа имеет вид:
  \begin{multline*}
  S_{2\mathrm{ID},i}(x) =\fr{6u(1-u)}{h_{i+1}}\,I_i^{i+1}+{}\\
  {}+(1-u) (1-3u)    f_i+ u(3u-2)f_{i+1}\,,
\end{multline*}
где 
$$
u=\fr{x-x_i}{h_{i+1}}\enskip (0\leq u\leq 1)\,.
$$
  
  Остаточное слагаемое интерполяции $f(x)$ многочленом $S_{2\mathrm{ID},i}(x)$ на 
отрезке $[x_i,x_{i+1}]$ выражается разностью
  \begin{multline}
  R_{2\mathrm{ID}}(x) =S_{2\mathrm{ID},i}(x)-f(x) 
=\fr{\varphi_1(u)}{h_{i+1}}\,I_i^{i+1}+{}\\
{}+\varphi_2(u)f_i +\varphi_3(u)f_{i+1}-f(x)\,,
  \label{e6-kir}
  \end{multline}
      где 
      \begin{equation}
      \left.
      \begin{array}{rl}
      \varphi_1(u) &= 6u(1-u)\,;\\[9pt]
      \varphi_2(u) &= (1-u)(1-3u)\,;\\[9pt]
      \varphi_3(u) &= u(3u-2)\,.
      \end{array}
      \right\}
      \label{e7-kir}
      \end{equation}
    Здесь $I_i^{i+1}$, $f_i$ и $f_{i+1}$ являются параметрами мно\-го\-члена.
  
  Рассмотрим случай $f(x)\hm\in C^3_{[x_i,x_{i+1}]}$. Заменяя в~(\ref{e6-kir}) 
параметры $I_i^{i+1}\hm= F_{i+1}\hm- F_i$ и $f_i$, $f_{i+1}$ (здесь $F_i\hm= 
F(x_{i})$, $F_{i+1}\hm= F(x_{i+1})$, где $F(x)$~--- первообразная) их разложениями 
по формуле Тейлора в точке $x\hm\in (x_i,x_{i+1})$, получим: 
  \begin{multline*}
  R_{2\mathrm{ID}}(x) =\fr{h^3_{i+1}}{12}\,u^3(1-u)(-3u^2+6u-2)f^{\prime\prime\prime}
  (\xi)+{}\\
  {}+\fr{h^3_{i+1}} 
{12}\, u(1-u)^3 \left(3u^2-1\right) f^{\prime\prime\prime}(\eta)\
(\xi,\eta\in (x_i,x_{i+1})).
  \end{multline*}
  Здесь учитывается, что точка~$\xi$ ($x_i\hm< \xi\hm < x$) для 
разложения~$F_i$ та же, что и для разложения~$f_i$, и точка~$\eta$ 
($x\hm<\eta \hm< x_{i+1}$) для разложения~$F_{i+1}$ та же, что и для 
разложения~$f_{i+1}$. 

\begin{table*}\small
\begin{center}
\parbox{380pt}{\Caption{Константы в оценке погрешности интерполяции функции $f(x)\hm\in 
C^m_{[x_i,x_{i+1}]}$ ($m\hm=1,2,3$) параболическим 
ИД-мно\-го\-чле\-ном $S_{2\mathrm{ID},i}(x)$ }

}

\vspace*{2ex}

\tabcolsep=7pt
\begin{tabular}{|c|c|c|c|}
\hline
&&&\\[-9pt]
Порядок производной&\tabcolsep=0pt\begin{tabular}{c}$T_{3,0}^{(2\mathrm{ID})}$ \\
$\left(f(x)\in C^3_{[x_i,x_{i+1}]}\right)$\end{tabular}&
\tabcolsep=0pt\begin{tabular}{c}$T_{2,0}^{(2\mathrm{ID})}$\\ 
$\left(f(x)\in C^2_{[x_i,x_{i+1}]}\right)$\end{tabular}&
\tabcolsep=0pt\begin{tabular}{c}$T_{1,0}^{(2\mathrm{ID})}$ \\ 
$\left(f(x)\in C^1_{[x_i,x_{i+1}]}\right)$\end{tabular}
\\
&&&\\[-8pt]
\hline
$p=0$&$\fr{1}{72\sqrt{3}}\approx 0{,}0080$&$\fr{1}{25\sqrt{2}}\approx 
0{,}0283$&$\fr{1}{6}$\\
$p=1$&$\fr{1}{12}$&$\fr{1}{4}$&2 \\
\hline
\end{tabular}
\end{center}
\end{table*}
  
  Обозначим 
  \begin{align*}
  \psi_1(u)  &= (1-u)(-3u^2 +6u -2)\,; \\
  \psi_2(u)&= u(1-u)^3(3u^2-1)\,.
  \end{align*}
   Тогда
  \begin{multline*}
  \vert R_{2\mathrm{ID}}(x)\vert \leq{}\\
  {}\leq \fr{h^3_{i+1}}{12}\left(  \vert \psi_1(u)\vert\cdot \vert 
f^{\prime\prime\prime}(\xi)\vert+ \vert \psi_2(u)\vert\cdot \vert 
f^{\prime\prime\prime}(\eta)\vert \right)\,.
  \end{multline*}
По теореме о среднем~\cite{2-kir}, если $g(x)\hm\in C_{[\xi,\eta]}$ и $\alpha$ 
и~$\beta$ имеют одинаковые знаки, то 
$$
\exists\ \zeta \in [\xi,\eta]:\ \alpha 
g(\xi) +\beta g(\eta) = (\alpha+\beta) g(\zeta)\,.
$$
 Следовательно, если 
принять в качестве $g(x)$ функцию $\vert f^{\prime\prime\prime}(x)\vert$ и в 
качестве коэффициентов~$\alpha$ и $\beta$  значения $\vert\psi_1(u)\vert$ и 
$\vert\psi_2(u)\vert$ в данной конкретной точке~$u$ ($u\hm\in [0,1]$), то 
существует такая точка $\zeta\hm\in [x_i, x_{i+1}]$ (поскольку $\xi, \eta\hm\in 
[x_i,x_{i+1}]$), что $\forall x\hm\in [x_i,x_{i+1}]$ (т.\,е.\ $\forall u\hm\in [0,1]$) 
выполняется неравенство:
\begin{multline*}
\vert R_{2\mathrm{ID}}(x)\vert \leq \fr{h^3_{i+1}}{12}\left( \vert \psi_1(u)\vert +\vert 
\psi_2(u)\vert \right) \vert f^{\prime\prime\prime}(\zeta)\vert \leq {}\\
{}\leq
\fr{h^3_{i+1}}{12}\left(  \vert \psi_1(u)\vert +\vert \psi_2(u)\vert \right) \parallel 
f^{\prime\prime\prime}(\zeta)\parallel\,.
\end{multline*}
Правая часть неравенства является мажорантой для модуля остаточного 
слагаемого, поэтому
\begin{multline*}
\parallel\! R_{2\mathrm{ID}}(x) \!\parallel_{[x_i,x_{i+1}]} \leq {}\\
{}\leq
\fr{h^3_{i+1}}{12}\,\max\limits_{u\in [0,1]} \gamma(u)\cdot \parallel 
f^{\prime\prime\prime}(x)\parallel_{[x_i,x_{i+1}]}\,,
\end{multline*}
где $\gamma(u) = 
\vert \psi_1(u)\vert +\vert \psi_2(u)\vert.$

  
  Максимум функции $\gamma(u)$ при $0\hm\leq u\hm\leq 1$ достигается в 
двух точках, симметричных относительно середины отрезка, и равен 
$1/(6\sqrt{3})\hm\approx 0{,}0962$.
  
  Таким образом, при $f(x)\hm\in C^3_{[x_i, x_{i+1}]}$ оценка погрешности 
интерполяции функции $f(x)$ ИД-мно\-го\-чле\-ном $S_{2\mathrm{ID},i}(x)$ на отрезке 
$[x_i, x_{i+1}]$ будет иметь вид
  \begin{multline*}
  \parallel \!R_{2\mathrm{ID}}(x)\!\parallel_{[x_i,x_{i+1}]} = \parallel \!S_{2\mathrm{ID},i}(x) -
f(x)\!\parallel_{[x_i,x_{i+1}]} \leq {}\\
{}\leq\fr{h^3_{i+1}}{72\sqrt{3}}\,\parallel  \!
f^{\prime\prime\prime} (x)\!\parallel_{[x_i, x_{i+1}]}\,.
  \end{multline*}
  
  Для производной $f^\prime(x)$ при $f(x)\hm\in C^3_{[x_i,x_{i+1}]}$, а также 
для функций классов гладкости $C^2_{[x_i,x_{i+1}]}$ и $C^1_{[x_i,x_{i+1}]}$ 
(и их производных) оценивание погрешностей аппроксимации проводится по 
тому же алгоритму.
  
  Все полученные результаты обобщаются в виде следующей теоремы.
  
  \medskip
  
  \noindent
\textbf{Теорема 1.} {Об оценке погрешностей аппроксимации функций 
ИД-мно\-го\-чле\-ном $S_{2\mathrm{ID},i}(x)$}.
  
  \textit{Если параболический ИД-мно\-го\-член $S_{2\mathrm{ID},i}(x)$ на отрезке 
$[x_i,x_{i+1}]$ интерполирует функцию $f(x)\hm\in C^m_{x_i,x_{i+1}}$ 
($m\hm=1,2,3$), причем его параметры $I_i^{i+1}\hm= 
\int\limits_{x_i}^{x_{i+1}} f(x)\,dx$, $f_k\hm=f(x_k)$ $(k\hm=i,i+1)$ известны 
точно или вычислены с точностью не ниже $O(h^{m+2})$, $O(h^{m+1})$ 
соответственно, то справедливы оценки:
  \begin{multline}
  \hspace*{-3mm}\parallel \!R^{(p)}_{2\mathrm{ID}}(x)\!\parallel_{[x_i,x_{i+1}]} = \parallel 
S^{(p)}_{2\mathrm{ID},i}(x) -f^{(p)}(x)\parallel_{[x_i,x_{i+1}]}\leq {}\\
{}\leq T^{(2\mathrm{ID})}_{m,p} 
h^{m-p}_{i+1} \parallel f^{(m)}(x)\parallel_{[x_i,x_{i+1}]}\,,
  \label{e8-kir}
  \end{multline}
где $p=0,1$~--- порядок производной; $T_{m,p}^{(2\mathrm{ID})}$~--- константы, 
приведенные в табл.}~1.




\section{Построение и~сходимость глобальных параболических 
интегродифференциальных сплайнов минимального дефекта}
  
  Пусть на сетке $\Delta_1$~(\ref{e1-kir}) задана функция $\{ f_i \hm= f(x_i) 
\hm\pm \theta_i\}^n_{i=0}$, где $\theta_i$ ($i\hm=0,\ldots ,n$)~--- погрешности 
измерения или вычисления значений функции, не превышающие $O(H^3)$ 
($H\hm= \max\limits_{i=1,\ldots ,n} h_i$). 
  
  Требуется построить глобальный параболический ИД-сплайн 
$\tilde{S}_{2\mathrm{ID}}(x)$ с узлами на сетке~$\Delta_1$, имеющий погрешность 
аппроксимации 
$$
\parallel \!\tilde{S}_{2\mathrm{ID}}(x)- f(x)\!\parallel_{[a,b]} = 
\max\limits_{x\in[a,b]} \vert \tilde{S}_{2\mathrm{ID}}(x)- f(x)\vert\,,
$$ 
не превышающую 
$O(H^3)$ (для $f(x)\hm\in C^m_{[a,b]}$ ($m\hm\geq 3$)), и удовлетворяющий 
следующим условиям:
  \begin{enumerate}[(1)]
\item интегральному условию согласования 
\begin{equation}
\int\limits_{x_1}^{x_{i+1}} \tilde{S}_{2\mathrm{ID}}(x)\,dx =\hat{I}_i^{i+1}\,,\enskip 
i=0,\ldots , n-1\,;
\label{e9-kir}
\end{equation}
\item условию непрерывности сплайна и его первой производной в узлах 
сетки~$\Delta_1$:
\begin{multline}
\tilde{S}_{2\mathrm{ID}}^{(p)}(x) \big\vert_{x=x_i}^{[x_{i-1},x_i]}=\tilde{S}_{2\mathrm{ID}}^{(p)} 
(x)\big\vert_{x=x_i}^{[x_i,x_{i+1}]}\,,\\  p=0,1\,; \enskip i=1,\ldots ,n-1\,.
\label{e10-kir}
\end{multline}
\end{enumerate}
Здесь $\hat{I}_i^{i+1}$~--- заданные или предварительно вы\-чис\-лен\-ные с 
точностью не ниже $O(H^4)$ интегралы от функции~$f(x)$ на частичных 
отрезках $[x_i,x_{i+1}]$, образуемых сеткой~$\Delta_1$. Параболический 
сплайн, удовлетворяющий условиям~(\ref{e10-kir}), имеет дефект\linebreak $q\hm=1$.

  Данный метод аппроксимации будем называть \textit{методом слабого 
сглаживания}, а соответствующий сплайн~--- \textit{слабосглаживающим 
сплайном}.
  
  Как будет показано ниже, для этих сплайнов разность значений сплайна и 
функции как в узлах сетки $\Delta_1$, так и на всем рассматриваемом отрезке 
при аппроксимации функций класса $C^m_{[a,b]}$ ($m\hm\geq 3$) имеет 
порядок $O(H^3)$ и, таким образом, указанные сплайны близки к 
интерполяционным. Во многих практических задачах порядок точности 
исходных данных не превышает $O(H^3)$. В~этих случаях 
слабосглаживающие сплайны в сущности являются интерполяционными, так 
как обеспечивают выполнение условий интерполяции в узлах сеточной 
функции с порядком $O(H^3)$. 
  
  Однако, в отличие от традиционных интерполяционных параболических 
дифференциальных сплайнов, устойчивость ИД-сплай\-нов обеспечивается без 
сдвига узлов сплайна относительно узлов аппроксимируемой сеточной 
функции, что обуслов\-ле\-но применением интегрального условия согласования. 
При этом ИД-сплай\-ны обладают свойством консервативности, а алгоритмы их 
по\-стро\-ения характеризуются простотой реализации и экономичностью.
  
  Для построения слабосглаживающего ИД-сплай\-на $\tilde{S}_{2\mathrm{ID}}(x)\hm= 
\mathop{\bigcup}\limits_{i=0}^{n-1} \tilde{S}_{2\mathrm{ID},i}(x)$, удовлетворя\-ющего 
условиям~(\ref{e9-kir}) и~(\ref{e10-kir}), коэффициенты $a_{k,i}$ 
($k\hm=0,1,2$) составляющих его звеньев 
$\tilde{S}_{2\mathrm{ID},i}(x)\hm=\sum\limits_{k=0}^2 a_{k,i}(x-x_i)^k$ вычисляются из 
системы уравнений, вытекающей из совокупности соотношений~(\ref{e9-kir}) 
и~(\ref{e10-kir}) . 
  
  Звено $\tilde{S}_{2ID,i}(x)$ на отрезке $[x_i,x_{i+1}]$ с коэффициентами, 
полученными из~(\ref{e9-kir}) и~(\ref{e10-kir}) при $p\hm=0$ (условие 
интегрального согласования сплайна и аппроксимируемой функции и условие 
непрерыв\-ности сплайна соответственно), имеет вид:
 \begin{multline*}
  \tilde{S}_{2\mathrm{ID},i}(x) =\tilde{f}_i +\left( \fr{6\nabla \hat{I}_i^{i+1}} {h^2_{i+1}}-
\fr{2\Delta \tilde{f}_i}{h_{i+1}}\right)(x-x_i) + {}\\
{}+\left( -\fr{6\nabla 
\hat{I}_i^{i+1}}{h^3_{i+1}}+ \fr{3\Delta\tilde{f}_i}{h^2_{i+1}} \right) (x-x_i)^2\,,
  \end{multline*}
где $\tilde{f}_i$ ($i\hm=0,\ldots , n$)~--- параметры, равные значениям сплайна 
$\tilde{S}_{2\mathrm{ID}}(x)$ в узлах сетки~$\Delta_1$, $\nabla \hat{I}_i^{i+1}\hm= 
\hat{I}_i^{i+1} \hm- \tilde{f}_i h_{i+1}$, $\Delta \tilde{f}_i\hm= \tilde{f}_{i+1}\hm- 
\tilde{f}_i$.
  
  Выполнение условия~(\ref{e10-kir}) при $p\hm=1$ (условие непрерывности 
первой производной сплайна $\tilde{S}_{2\mathrm{ID}}(x)$) обеспечивается, если 
параметры $\tilde{f}_i$ ($i\hm=0,\ldots ,n$) удовлетворяют следующим 
соотношениям (полученным из~(\ref{e10-kir}) при $p\hm=1$ путем 
алгебраических преобразований):
  \begin{multline}
  \fr{1}{h_i} \tilde{f}_{i-1} +2\left( \fr{1}{h_i} +\fr{1}{h_{i+1}}\right) \tilde{f}_i 
+ \fr{1}{h_{i+1}}\,\tilde{f}_{i+1} ={}\\
{}=3 \left( 
\fr{\hat{I}_i^{i+1}}{h^2_{i+1}}+\fr{\hat{I}_{i-1}^{i}}{h_i^2} \right)\,,\enskip i=1,\ldots , n-
1\,.
  \label{e11-kir}
  \end{multline}
  
  Равенства~(\ref{e11-kir}) представляют собой трехдиагональную систему 
линейных алгебраических уравнений с диагональным преобладанием. Эта 
система в совокупности с краевыми условиями (например, заданными в виде 
$\tilde{f}_0\hm= f_0\hm= f(x_0)$, $\tilde{f}_n\hm= f_n\hm= f(x_n)$) имеет 
единственное решение, которое можно найти экономичным методом 
прогонки~\cite{9-kir}.
  
  Для вычисления интегральных параметров звень\-ев сплайнов могут быть 
использованы левосторонние и правосторонние квадратурные формулы, 
обеспечивающие при $f(x)\hm\in C^m_{[a,b]}$\linebreak ($m\hm\geq3$) порядок 
точности вычисления интегралов $O(H^4)$~\cite{8-kir}: 
  \begin{align}
  \hat{I}_{i-1}^i &= \fr{h_i^3}{6\overline{H}_i^{i+1}}\left(  -
\fr{1}{h_{i+1}}\,f_{i+1} +\fr{\overline{H}_i^{i+1}\overline{H}_i^{3(i+1)}} 
{h_i^2 h_{i+1}}\,f_i +{}\right.\notag\\
&\hspace*{30mm}\left.{}+\fr{\overline{H}_{2i}^{3(i+1)}}{h_i^2}\,f_{i-1}\right)\,; 
\label{e12-kir}\\
  \hat{I}_i^{i+1} &= \fr{h^3_{i+1}}{6\overline{H}_i^{i+1}}\left(  
\fr{\overline{H}_{3i}^{2(i+1)}}{h^2_{i+1}}\,f_{i+1} + \fr{\overline{H}_i^{i+1} 
\overline{H}_{3i}^{i+1}} {h_i h^2_{i+1}}\,f_i -{}\right.\notag\\
&\hspace*{30mm}\left.{}-\fr{1}{h_i}\,f_{i- 1}
\vphantom{ \fr{\overline{H}_i^{i+1} 
\overline{H}_{3i}^{i+1}} {h_i h^2_{i+1}}}
\right)\,,\label{e13-kir}
  \end{align}
где $\overline{H}_{ki}^{l(i+1)} \hm= kh_i\hm+lh_{i+1}$ ($k, l\hm>0$~--- 
натуральные числа).
  
  Для параболического ИД-сплай\-на $\tilde{S}_{2\mathrm{ID}}(x)$ верна следующая 
теорема сходимости. 
  
  \medskip
  
  \noindent
\textbf{Теорема 2.} {О сходимости одномерного глобального 
параболического ИД-сплай\-на.}

  \textit{Пусть функцию $f(x)$ $(x\hm\in [a,b])$, заданную с точностью не ниже 
$O(H^4)$ $(h_i\hm=x_i\hm-x_{i-1}$, $H\hm= \max\limits_{i=1,\ldots ,n} h_i)$ на 
сетке~$\Delta_1$~$(\ref{e1-kir})$ с параметром неравномерности сетки 
$$
Q= \fr{\max\limits_{i=1,\ldots , n} h_i }{\min\limits_{i=1,\ldots ,n} h_i}\,,
$$
 аппроксимирует 
слабосглаживающий глобальный\linebreak параболический ИД-сплайн 
$\tilde{S}_{2\mathrm{ID}}(x)$. Тогда если\linebreak $f(x)\hm\in C^3_{[a,b]}$ и параметры 
$\hat{I}_i^{i+1}$ $(i\hm=0,\ldots , n-1)$ определяются по формулам~$(\ref{e12-kir})$, 
$(\ref{e13-kir})$, а $\tilde{f}_i$ $(i\hm=0,\ldots ,n)$~--- из трехдиагональной 
системы линейных алгебраических уравнений (СЛАУ)~$(\ref{e11-kir})$ с 
краевыми условиями $\tilde{f}_0\hm= f_0\hm= f(x_0)$, $\tilde{f}_n\hm= 
f_n\hm= f(x_n)$, то справедливы оценки}
  \begin{multline}
  \parallel \!\tilde{S}^{(p)}_{2\mathrm{ID}}(x)-f^{(p)}(x)\!\parallel_{[a,b]} \leq {}\\
  {}\leq H^{3-p}\left( 
T_{3,p}^{(2\mathrm{ID})} +K_p Q^{1+p}\right) \parallel \!
f^{\prime\prime\prime}(x)\!\parallel_{[a,b]}\,,
  \label{e14-kir}
  \end{multline}
\textit{где $p\hm=0,1$~--- порядок производной, а константы имеют значения: }
$$
T^{(2\mathrm{ID})}_{3,0}=\fr{1}{72\sqrt{3}}\,;\  T^{2\mathrm{(ID)}}_{3,1}=\fr{1}{12}\,;\  
K_0=\fr{11}{48}\,;\  K_1=\fr{25}{24}\,.
$$
  
  \textit{Таким образом, при $f(x)\hm\in C^3_{[a,b]}$ сплайны 
$\tilde{S}_{2\mathrm{ID}}(x)$ равномерно сходятся к функции $f(x)$ на 
последовательности сеток~$\Delta_1^{(n)}$: $a\hm=x_0\hm<x_1<\cdots < 
x_i\hm< x_{i+1}<\cdots <x_n\hm=b$ по крайней мере со скоростью $H^3$, а их 
производные $\tilde{S}^\prime_{2\mathrm{ID}}$ равномерно сходятся к $f^\prime(x)$ по 
крайней мере со скоростью~$H^2$ с ростом~$n$.}
  
  \medskip
  
  \noindent
  Д\,о\,к\,а\,з\,а\,т\,е\,л\,ь\,с\,т\,в\,о\,.\ \ Формула звена сплайна 
$\tilde{S}_{2\mathrm{ID}}(x)$ на отрезке $[x_i,x_{i+1}]$ в форме Лагранжа записывается 
в виде ИД-мно\-го\-чле\-на: 
  \begin{multline*}
  \tilde{S}_{2\mathrm{ID},i}(x) = \fr{6u(1-u)}{h_{i+1}}\,\hat{I}_i^{i+1}+(1-u) (1-3u)\tilde{f}_i 
+{}\\
{}+u(3u-2)\tilde{f}_{i+1}\,.
\end{multline*}
  
  Для нахождения погрешности $\parallel \!\!\!\tilde{S}_{2\mathrm{ID}}^{(p)}(x)\hm- 
f^{(p)}(x)\!\parallel_{[a,b]}$ представим ее в виде:
  \begin{multline}
  \hspace*{-3mm}\parallel\! \tilde{S}_{2\mathrm{ID}}^{(p)} (x) -f^{(p)}(x)\!\parallel_{[a,b]}
  \leq \parallel \!
S_{2\mathrm{ID}}^{(p)}(x) -f^{(p)}(x)\!\parallel_{[a,b]}+{}\\
{}+ \parallel\! \tilde{S}^{(p)}_{2\mathrm{ID}}(x) -
S_{2\mathrm{ID}}^{(p)}(x)\!\parallel_{[a,b]}\,,
  \label{e15-kir}
  \end{multline}
где $S_{2\mathrm{ID}}(x)$~--- сплайн, составленный из звеньев~(\ref{e5-kir}), параметры 
которого $I_i^{i+1}$, $f_i$, $f_{i+1}$ известны точно (или вычислены с 
точностью не ниже $O(H^5)$ для $I_i^{i+1}$ и $O(H^4)$ для $f_i$, $f_{i+1}$).

  Оценки первого слагаемого правой части неравенства~(\ref{e15-kir}) 
вытекают из формулы~(\ref{e8-kir}). Поскольку оценки, приведенные в 
теореме~1, справедливы для любого частичного отрезка $[x_i,x_{i+1}]\subset 
[a,b]$, $i\hm=0,\ldots , n-1$, то верна формула
  \begin{multline}
  \parallel\! S^{(p)}_{2\mathrm{ID},i}(x) -f^{(p)}(x)\!\parallel_{[a,b]} \;\leq {}\\
  {}\leq T^{(2\mathrm{ID})}_{m,p} 
H^{m-p}\parallel\! f^{(m)}(x)\!\parallel_{[a,b]}\,.
  \label{e16-kir}
  \end{multline}
  
  Оценки второго слагаемого получаются следующим образом. На отрезке 
$[x_i,x_{i+1}]$ разность $\tilde{S}^{(p)}_{2\mathrm{ID}}(x)\hm- S_{2\mathrm{ID}}^{(p)}(x)$ при 
$p\hm=0,1$ (соответственно) имеет вид:

\noindent 
  \begin{multline}
  \tilde{S}_{2\mathrm{ID}}(x) -S_{2\mathrm{ID}}(x) = \fr{6u(1-u)}{h_{i+1}}\left( \hat{I}_i^{i+1} -
I_i^{i+1}\right) +{}\\
  {}+ (1-u)(1-3u) (\tilde{f}_i-f_i) +{}\\
  {}+u(3u-2) (\tilde{f}_{i+1}-f_{i+1})\,,
  \label{e17-kir}
  \end{multline}
  
  \vspace*{-12pt}
  
  \noindent
  \begin{multline}
  \tilde{S}^\prime_{2\mathrm{ID}}(x) -S^\prime_{2\mathrm{ID}}(x) = \fr{6(1-2u)}
  {h^2_{i+1}}\left( 
\hat{I}_i^{i+1}-I_i^{i+1}\right) +{}\\
  {}+ \fr{2(3u-2)}{h_{i+1}}\left(\tilde{f}_i-f_i\right) +{}\\
  {}+\fr{2(3u-1)}{h_{i+1}} \left( 
\tilde{f}_{i+1}-f_{i+1}\right)\,.
  \label{e18-kir}
  \end{multline}

Введем функции $\varphi_k(u)$ ($k\hm=1,2,3$) по формулам~(\ref{e7-kir}). 
Дифференцируя $\varphi_k(u)$, получим:
\begin{align*}
\varphi_1^\prime(u) &= \fr{d\varphi_1(u)}{du}=6(1-2u)\,;\\
\varphi^\prime_2(u) &= \fr{d\varphi_2(u)}{du}=2(3u-2)\,;\\
\varphi^\prime_3 &= \fr{d\varphi_3(u)}{du}=2(3u-1)\,.
\end{align*}

Из~(\ref{e17-kir}) и~(\ref{e18-kir}) $\forall\,x \hm\in [a,b]$ вытекает 
соотношение: 
\begin{multline}
\parallel\! \tilde{S}^{(p)}_{2\mathrm{ID}}(x) -S^{(p)}_{2\mathrm{ID}}(x)\!\parallel_{[a,b]} \;\leq {}\\
{}\leq
\fr{1}{\left( \min\limits_{i=1,\ldots , n} h_i\right)^{1+p}}\,\max\limits_{u\in [0,1]} 
\left\vert \varphi_1^{(p)}(u)\right\vert \times{}\\
{}\times \max\limits_{i=0,\ldots , n-1} \left\vert 
\hat{I}_i^{i+1}-I_i^{i+1}\right\vert+{}\\
{}+
\fr{1}{\left(\min\limits_{i=1,\ldots , n} h_i\right)^{p}}\,\max\limits_{u\in [0,1]}\left[  
\left\vert \varphi_2^{(p)}(u)\right\vert +{}\right.\\
\left.{}+\left\vert \varphi_3^{(p)}(u)\right\vert \right] 
\max\limits_{i=0,\ldots , n}\left\vert \tilde{f}_i-f_i\right\vert\,.
\label{e19-kir}
\end{multline}

Можно показать, что 
\begin{alignat*}{2}
\max\limits_{u\in [0,1]} \left\vert \varphi_1(u)\right\vert &=\fr{3}{2} &\mbox{ при } &
u=\fr{1}{2}\,;\\
\max\limits_{u\in[0,1]} \left\vert\varphi^\prime(u)\right\vert &=1 &\mbox{ при } &
u=0\,;\\
\max\limits_{u\in [0,1]} \left[ \left\vert \varphi_2(u)\right\vert +\left\vert 
\varphi_3(u)\right\vert\right]&=1 & \mbox{ при }& u=0,\ u=1\,;\\
\max\limits_{u\in [0,1]} \left[ \left\vert \varphi_2^\prime(u)\right\vert +\left\vert 
\varphi^\prime_3(u)\right\vert\right]&=6 &\mbox{ при } & u=0,\ u=1\,.
\end{alignat*}

\begin{table*}\small
\begin{center}
\Caption{Оценки погрешностей аппроксимации функции $f(x)\hm= 
C^m_{[x_i,x_{i+1}]}$ ($m\hm=1,2,3$) и ее первой производной $f^\prime(x)$ 
слабосглаживающим ИД-сплай\-ном $\tilde{S}_{2\mathrm{ID}}(x)$ и его производной 
$\tilde{S}^\prime_{2\mathrm{ID}}(x)$}
\vspace*{2ex}

\tabcolsep=4pt
\begin{tabular}{|c|c|c|c|}
\hline 
Погрешность& $f(x)\in C^3_{[a,b]}$ & $f(x)\in C^2_{[a,b]}$ 
& $f(x)\in C^1_{[a,b]}$\\
\hline
\tabcolsep=0pt\begin{tabular}{c}
\hspace*{-3mm}$\parallel\! \tilde{S}_{2\mathrm{ID}}^{(p)}(x)-{}$\hspace*{3mm}\\
 ${}-f^{(p)}(x)\!\parallel_{[a,b]}\leq{}$\end{tabular} & 
 $H^{3- p}(T_{3,p}+K_{3,p}Q^{1+p})M_3$ & $H^{2-p}(T_{2,p}+K_{2,p}Q^{2+p})M_2$ & $H^{1-
p}(T_{1,p}+(K_{1,p}+K_{1,p}^*Q)Q^{1+p})M_1$\\
\hline
$p=0$ &  $T_{3,0}=\fr{1}{72\sqrt{3}}$\,,\ $K_{3,0}=\fr{11}{48}$ 
&$T_{2,0}=\fr{1}{25\sqrt{2}}$\,, \ $K_{2,0}=\fr{9}{24}$ &$T_{1,0}=\fr{1}{6}$\,,\ 
$K_{1,0}=\fr{1}{2}$\,,\ $K^*_{1,0}=\fr{3}{4}$\\
\hline
$p=1$ & $T_{3,1}=\fr{1}{12}$\,,\ $K_{3,1}=\fr{25}{24}$&$T_{2,1}=\fr{1}{8}$\,,\ 
$K_{2,1}=\fr{19}{12}$ &$T_{1,1}=2$\,,\ $K_{1,1}=3$\,,\ $K^*_{1,1}=\fr{19}{6}$\\
\hline
  \multicolumn{4}{l}{\footnotesize \textbf{Примечания:} $M_3\hm=\parallel\! 
f^{\prime\prime\prime}(x)\!\parallel_{[a,b]}$, $M_2\hm= \parallel \!
f^{\prime\prime}(x)\!\parallel_{[a,b]}$, $M_1\hm= \parallel \!
f^\prime(x)\!\parallel_{[a,b]}$.}\\
  \end{tabular}
  \end{center}
  \end{table*}
  
  Оценки погрешностей вычисления интегралов $\max\limits_{i=0,\ldots ,n-1} 
\vert \hat{I}_i^{i+1}-I_i^{i+1}\vert$ при нахождении $\hat{I}_i^{i+1}$ ($i\hm=0,\ldots , 
n-1$) по формулам~(\ref{e12-kir}), (\ref{e13-kir}) и погрешностей вычисления 
значений функции в узлах сетки~$\Delta_1$ $\max\limits_{i=0,\ldots ,n} \vert 
\tilde{f}_i\hm- f_i\vert$ при нахождении~$\tilde{f}_i$ 
  ($i\hm=0,\ldots ,n$) из трехдиагональной СЛАУ~(\ref{e11-kir}) с краевы\-ми 
условиями $\tilde{f}_0\hm=f_0$, $\tilde{f}_n\hm=f_n$ найдены в~\cite{8-kir} и в 
случае $f(x)\hm\in C^3_{[a,b]}$ имеют вид:
  \begin{align}
 & \max\limits_{i=0,\ldots , n-1}\left\vert \hat{I}_i^{i+1} -I_i^{i+1}\right\vert \leq  
\fr{1}{24}\,H^4 \parallel\! f^{\prime\prime\prime}(x)\!\parallel_{[a,b]}\,; \label{e20-kir}\\ 
 & \max\limits_{i=0,\ldots , n} \left\vert \tilde{f}_i -f_i\right\vert \leq  \left( 
\fr{H^3}{24} + \fr{H^3}{8}\right) Q \parallel \!f^{\prime\prime\prime}(x)\!\parallel_{[a,b]} = {}\notag\\
&\hspace*{30mm}{}=\fr{H^3}{6}\,Q\parallel f^{\prime\prime\prime}(x)\parallel_{[a,b]}\,,
  \label{e21-kir}
  \end{align}
где 
$H=\max\limits_{i=1,\ldots , n} h_i$.
  
  Итак, из формул (\ref{e19-kir})--(\ref{e21-kir}) вытекает
  \begin{multline}
\parallel\! \tilde{S}_{2\mathrm{ID}}^{(p)} (x) -S_{2\mathrm{ID}}^{(p)}(x)\!\parallel_{[a,b]}\leq {}\\
{}\leq K_p 
H^{3-p} Q^{1+p} \parallel\! f^{\prime\prime\prime} (x)\!\parallel_{[a,b]}\\ 
\mbox{при}\ f(x)\in C^3_{[a,b]}\,,
  \label{e22-kir}
  \end{multline}
где $p=0, 1$~--- порядок производной; $K_0\hm= {11}/{48}$, $K_1\hm= 
{25}/{24}$.
  
  Таким образом, из формул~(\ref{e15-kir}), (\ref{e16-kir}), (\ref{e22-kir}) 
получается оценка~(\ref{e14-kir}). 


  
  Теорема~2 доказана.
  
  \medskip
  
  Аналогичным способом можно найти оценки погрешностей аппроксимации 
функции $f(x)$ (и ее производной) с помощью сплайна $\tilde{S}_{2\mathrm{ID}}(x)$ (и 
его производной) при $f(x)\hm\in C^m_{[x_i,x_{i+1}]}$, $m\hm=1,2$.
  
  Оценки погрешностей аппроксимации функции $f(x)\hm \in 
C^m_{[x_i,x_{i+1}]}$ ($m\hm=1,2,3$) (и ее первой производной $f^\prime(x)$) с 
помощью слабо\-сгла\-жи\-ва\-юще\-го ИД-сплай\-на $\tilde{S}_{2\mathrm{ID}}(x)$ (и 
$\tilde{S}^\prime_{2\mathrm{ID}}(x)$) при определении параметров $\hat{I}_i^{i+1}$ 
($i\hm=0, \ldots , n-1$) по формулам~(\ref{e12-kir}), (\ref{e13-kir}) и нахождении 
затем значений $\tilde{f}_i$ ($i\hm=0, \ldots ,n$) из трехдиагональной 
СЛАУ~(\ref{e11-kir}) с краевыми условиями $\tilde{f}_0\hm= f_0\hm= f(x_0)$, 
$\tilde{f}_n \hm= f_n \hm= f(x_n)$ представлены в табл.~2.
  


  
  Из оценок, приведенных в табл.~2, следует, что при $f(x)\hm\in C^m_{[a,b]}$ 
($m\hm=1,2,3$) ИД-сплай\-ны $\tilde{S}_{2\mathrm{ID}}(x)$ равномерно сходятся к 
функции $f(x)$ на последовательности сеток $\Delta_1^{(n)}:\ 
a\hm=x_0\hm<x_1<\cdots < x_i \hm< x_{i+1}<\cdots < x_n\hm=b$ по крайней 
мере со ско\-ростью~$H^m$ с ростом~$n$, а их производные при $f(x)\hm\in 
C^m_{[a,b]}$ ($m\hm=2,3$) равномерно сходятся к $f^\prime(x)$ по крайней 
мере со скоростью~$H^{m-1}$
  
  Дальнейшее увеличение степени гладкости функции $f(x)$ не приводит к 
увеличению порядка приближения относительно~$H$.

\vspace*{-6pt}

\section*{Выводы}

\noindent
\begin{enumerate}[1.]
  \item  В работе предложены методы аппроксимации функций различных 
классов гладкости с по\-мощью ИД-многочленов и 
ИД-сплай\-нов второй степени, обладающих свойством консервативности.
  \item Получены оценки погрешностей аппроксимации функций и их 
производных параболическими ИД-мно\-го\-чле\-на\-ми, доказана теорема 
сходимости глобального параболического слабосглаживающего ИД-сплай\-на 
дефекта~1. 
  \item Применение предложенных авторами алгоритмов аппроксимации 
  ИД-сплай\-на\-ми можно рекомендовать разработчикам больших 
  тер\-ри\-то\-ри\-аль\-но-рас\-пре\-де\-лен\-ных информационных %\linebreak 
  сис\-тем при 
построении математических моде\-лей обработки данных в условиях 
ограниченных временных и вычислительных ресурсов~--- реализующих, 
например, идентификацию %\linebreak 
инфор\-ма\-ци\-он\-ных объектов, определение степени 
близости данных, поиск зависимостей~\mbox{\cite{1-kir, 7-kir}}.
\end{enumerate}

{\small\frenchspacing
{%\baselineskip=10.8pt
\addcontentsline{toc}{section}{References}
\begin{thebibliography}{9}
\bibitem{1-kir}
\Au{Гершкович М.\,М., Бирюкова~Т.\,К., Синицын~В.\,И.} Проб\-ле\-мы 
идентификации и распознавания инфор\-мационных объектов при создании 
распределенных\linebreak ин\-фор\-ма\-ци\-он\-но-те\-ле\-ком\-му\-ни\-ка\-ци\-он\-ных 
сис\-тем~// Оп\-ти\-ко-элект\-рон\-ные приборы и устройства в сис\-те\-мах 
распознавания образов, обработки изображений и символьной информации 
(Распознавание-2012): Сб. мат-лов X Междунар. 
на\-уч.-технич. конф.~--- Курск: ЮЗГУ, 2012. С.~24--26. 
\bibitem{2-kir}
\Au{Завьялов Ю.\,С., Квасов Б.\,И. , Мирошниченко~В.\,Л.} Методы 
сплайн-функ\-ций.~--- Новосибирск: Наука, 1980. 350~с.
\bibitem{3-kir}
\Au{Завьялов Ю.\,С., Леус В.\,А., Скороспелов~В.\,А.} Сплайны в инженерной 
геометрии.~--- М.: Машиностроение, 1985. 224~с.
\bibitem{4-kir}
\Au{Квасов Б.\,И.} Методы изогеометрической аппроксимации сплайнами.~--- 
М.: Физматлит, 2006. 360~с.
\bibitem{5-kir}
\Au{Стечкин С.\,Б., Субботин~Ю.\,Н.} Сплайны в вычислительной 
математике.~--- М.: Наука, 1976. 248~с.
\bibitem{6-kir}
\Au{Киреев В.\,И., Бирюкова~Т.\,К.} Интегродифференциальные 
консервативные сплайны и их применение в интерполяции, чис\-лен\-ном 
дифференцировании и интегрировании~// Вычислительные технологии, 
1995. Т.~4. №\,16. С.~233--244. 
\bibitem{7-kir}
\Au{Бирюкова Т.\,К., Гершкович~М.\,М., Киреев~В.\,И.} 
Ин\-тегро-диф\-фе\-рен\-ци\-аль\-ные многочлены и сплайны произволь\-ной 
четной степени в задачах анализа па\-ра\-мет\-ров функционирования 
распределенных информационных сис\-тем~// Сис\-те\-мы компьютерной 
математики и их приложения (СКМП-2012): Мат-лы XIII~Междунар. науч. 
конф., посвященной 75-ле\-тию профессора Э.\,И.~Зверовича.~--- 
Смоленск: СмолГУ, 2012. Вып.~13. С.~67--72. 
\bibitem{8-kir}
\Au{Киреев В.\,И., Пантелеев А.\,В.} Численные методы в примерах и 
задачах.~--- М.: Высшая школа, 2008. 480~с.
\bibitem{9-kir}
\Au{Волков Е.\,А.} Численные методы.~--- М.: Наука, 1982. 254~с.
\end{thebibliography}
} }

\end{multicols}

\vspace*{-12pt}

\hfill{\small\textit{Поступила в редакцию 10.12.13}}


\vspace*{12pt}

\hrule

\vspace*{2pt}

\hrule


\def\tit{ON APPROXIMATION AND CONVERGENCE 
OF~ONE-DIMENSIONAL PARABOLIC INTEGRODIFFERENTIAL POLYNOMIALS AND~SPLINES}

\def\titkol{On approximation and convergence 
of~one-dimensional parabolic integrodifferential polynomials and~splines}

\def\aut{V.\,I.~Kireev$^1$, M.\,M.~Gershkovich$^2$, and~T.\,K.~Biryukova$^2$}
\def\autkol{V.\,I.~Kireev, M.\,M.~Gershkovich, and~T.\,K.~Biryukova}


\titel{\tit}{\aut}{\autkol}{\titkol}

\vspace*{-9pt}

\noindent
$^1$Moscow State Mining University, 6 Leninskiy Prosp., Moscow 119991, Russian 
Federation

\noindent
$^2$Institute of Informatics Problems, Russian Academy of Sciences,
44-2 Vavilov Str., Moscow 119333, Russian\\
$\hphantom{^1}$Federation


 
\def\leftfootline{\small{\textbf{\thepage}
\hfill INFORMATIKA I EE PRIMENENIYA~--- INFORMATICS AND APPLICATIONS\ \ \ 2014\ \ \ volume~8\ \ \ issue\ 1}
}%
 \def\rightfootline{\small{INFORMATIKA I EE PRIMENENIYA~--- INFORMATICS AND APPLICATIONS\ \ \ 2014\ \ \ volume~8\ \ \ issue\ 1
\hfill \textbf{\thepage}}}   

\vspace*{6pt}
  
\Abste{The methods for approximation of functions with one-dimensional (1D)
integrodifferential polynomials of the 2nd degree and derived conservative parabolic 
integrodifferential splines are considered.
In majority of applied computational tasks, accuracy of source data does not exceed 
precision of approximation by parabolic polynomials and splines.  
The nodes of conventional parabolic splines, based on differential matching conditions 
with approximated function (further named as differential splines), are shifted relatively 
to interpolation nodes in order to provide stability of approximation process. 
The shift between spline and approximation nodes complicates computational algorithms drastically.
 Additionally, traditional differential splines are not conservative, i.\,e., they
  do not maintain integral characteristics of approximated functions.   
The novel integrodifferential parabolic splines that use integral deviation as criteria 
for matching a spline with a source function are presented.  These splines are stable if 
spline nodes coincide with nodes of approximated functions and conservative with respect to sustaining area under curves.    
The theorems on approximation of mathematical functions with 1D
integrodifferential parabolic polynomials and convergence of parabolic integrodifferential 
splines are proved. 
It is suggested to apply the proposed integrodifferential splines for development 
of mathematical data processing models for large area spread information systems.
}


\KWE{spline; polynomial; integrodifferential; integrodifferential; approximation; 
interpolation; smoothing; estimation of errors; convergence theorem; 
mathematical data processing model}


\DOI{10.14357/19922264140112}

%\Ack
%\noindent


  \begin{multicols}{2}

\renewcommand{\bibname}{\protect\rmfamily References}
%\renewcommand{\bibname}{\large\protect\rm References}

{\small\frenchspacing
{%\baselineskip=10.8pt
\addcontentsline{toc}{section}{References}
\begin{thebibliography}{9}

\bibitem{1-kir-1}
\Aue{Gershkovich, M.\,M., T.\,K.~Biryukova, and V.\,I.~Sinitsyn}.
2012. 
Problemy identifikatsii i raspoznavaniya informatsionnykh ob''ektov pri 
sozdanii raspredelennykh informatsionno-telekommunikatsionnykh sistem 
[Problems of identification and recognition of information's objects 
in development of information-telecommunication systems].
\textit{Optiko-elektronnye pribory i ustroystva v sistemakh raspoznavaniya obrazov, 
obrabotki izobrazheniy i simvol'noy informatsii.  
Raspoznavanie 2012: Sbornik Materialov X Mezhdunarodnoy Nauchno-Tekhnicheskoy Kon\-fe\-ren\-tsii}. 
Kursk: Izd-vo Jugo-Zap. Gos. Un-ta. 24--26. 

\bibitem{2-kir-1}
\Aue{Zav'jalov, Ju.\,S. , B.\,I.~Kvasov, and V.\,L.~Miroshnichenko}.  
1980. \textit{Metody splayn-funktsiy} [\textit{Methods of spline functions}].
Novosibirsk: Nauka, 1980. 350~p.
\bibitem{3-kir-1}
\Aue{Zav'jalov, Ju.\,S.,  V.\,A.~Leus, and V.\,A.~Skorospelov}.
1985. \textit{Splayny v inzhenernoy geometrii} 
[\textit{Splines in engineering geometry}]. Moscow: Mashinostroenie. 224~p.
\bibitem{4-kir-1}
\Aue{Kvasov, B.\,I.} 
2006. \textit{Metody izogeometricheskoy approk\-si\-ma\-tsii splaynami} 
[\textit{Methods of izogeometric spline approximation}].  Moscow: Fizmatlit. 360~p.
\bibitem{5-kir-1}
\Aue{Stechkin, S.\,B., and Ju.\,N.~Subbotin}.
1976. \textit{Splayny v vychislitel'noy matematike}
[\textit{Splines in computing mathematics}]. 
Moscow: Nauka. 248~p.
\bibitem{6-kir-1}
\Aue{Kireev, V.\,I., and T.\,K.~Biryukova}.  
1955. Integro\-dif\-fe\-ren\-tsi\-al'nye  konservativnye splayny i ikh primenenie v in\-ter\-po\-lya\-tsii,  
chislennom differentsirovanii i integrirovanii 
[Integrodifferencial splines and their applications in interpolation, 
numerical differentiation and quadrature]. \textit{Vychislitel'nye Tekhnologii}
 4(16):233--244. 
\bibitem{7-kir-1}
\Aue{Biryukova, T.\,K., M.\,M.~Gershkovich, and V.\,I.~Kireev}. 2012.
Integro-differentsial'nye mnogochleny i splay\-ny proizvol'noy chetnoy stepeni 
v zadachakh ana\-li\-za parametrov funktsionirovaniya raspredelennykh informatsionnykh system 
[Integrodifferential polynomials and splines of arbitrary even degree in analysis of 
parameters of functioning of spread information systems].  \textit{Sistemy Komp'yuternoy 
matematiki i ikh prilozheniya (SKMP-2012): Materialy XIII Mezhdunarodnoy Nauchnoy Kon\-fe\-ren\-tsii, 
posvyashchennoy 75-letiyu Professora E.\,I.~Zverovicha}. 
Smolensk: Izd-vo SmolGU. 13:67--72. 

\bibitem{8-kir-1}
\Aue{Kireev, V.\,I., and A.\,V.~Panteleev}. 
2008. \textit{Chislennye metody v primerakh i zadachakh} 
[\textit{Numerical methods in examples and problems}].  Moscow: Vysshaya shkola. 480~p.
\bibitem{9-kir-1}
\Aue{Volkov, E.\,A.} 1982. 
\textit{Chislennye metody} [\textit{Computational methods}].  Moscow: Nauka. 254~p.  
\end{thebibliography}
} }


\end{multicols}

\vspace*{-6pt}

\hfill{\small\textit{Received December 10, 2013}}

\vspace*{-18pt}

\Contr

\noindent
\textbf{Kireev Vladimir I.} (b.\ 1938)~--- Doctor of Science in physics and 
mathematics, Professor, Moscow State Mining University, 6 Leninskiy Prosp.,
Moscow 119991, Russian Federation; Vladimir-Kireyev@mail.ru

\vspace*{2pt}

\noindent
\textbf{Gershkovich Maxim M.} (b.\ 1968)~--- senior scientist, Institute of 
Informatics Problems, Russian Academy of
Sciences, 44-2 Vavilov Str., Moscow 119333, Russian Federation; makmg@mail.ru

\vspace*{2pt}

\noindent
\textbf{Biryukova Tatiana K.} (b.\ 1968)~--- Candidate of Science (PhD)
in physics and mathematics, Institute of 
Informatics Problems, Russian Academy of
Sciences, 44-2 Vavilov Str., Moscow 119333, Russian Federation; yukonta@mail.ru



 \label{end\stat}
 
\renewcommand{\bibname}{\protect\rm Литература}