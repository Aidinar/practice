\newcommand{\Variance}{\sf D}
\newcommand{\Prob}{\sf P}

\def\stat{shestakov}

\def\tit{АСИМПТОТИЧЕСКИЕ СВОЙСТВА ОЦЕНКИ РИСКА ПРИ~ПОРОГОВОЙ ОБРАБОТКЕ 
ВЕЙВЛЕТ-КОЭФФИЦИЕНТОВ
В~МОДЕЛИ С КОРРЕЛИРОВАННЫМ ШУМОМ$^*$}

\def\titkol{Асимптотические свойства оценки риска при~пороговой обработке 
вейвлет-коэффициентов}
%в~модели с коррелированным шумом}

\def\autkol{А.\,А.~Ерошенко, О.\,В.~Шестаков}

\def\aut{А.\,А.~Ерошенко$^1$, О.\,В.~Шестаков$^2$}

\titel{\tit}{\aut}{\autkol}{\titkol}

{\renewcommand{\thefootnote}{\fnsymbol{footnote}}
\footnotetext[1]{Работа выполнена при финансовой поддержке РФФИ (проект 11-01-00515а).}}

\renewcommand{\thefootnote}{\arabic{footnote}}
\footnotetext[1]{Московский государственный университет им.\ М.\,В.~Ломоносова, 
кафедра математической статистики факультета вычислительной математики и кибернетики, 
aeroshik@gmail.com} 
\footnotetext[2]{Московский государственный университет им.\ М.\,В.~Ломоносова, 
кафедра математической статистики факультета вычислительной математики и кибернетики; 
Институт проблем информатики Российской академии наук, oshestakov@cs.msu.su}


\Abst{Вейвлет-методы подавления шума, основанные на процедуре пороговой обработки, 
широко используются при анализе сигналов и изображений. Их привлекательность заключается, 
во-пер\-вых, в быстроте алгоритмов построения оценок, а во-вто\-рых, 
в возможности лучшей, чем линейные методы, адаптации к функциям, имеющим 
на разных участках различную степень регулярности. Анализ погрешностей 
этих методов представляет собой важную практическую задачу, поскольку он 
позволяет оценить качество как самих методов, так и используемого оборудования. 
В~работе исследуются асимптотические свойства оценки среднеквадратичного риска 
при пороговой обработке коэффициентов разложения функции сигнала по вейв\-лет-ба\-зи\-су 
в модели с коррелированным шумом. Приводятся условия, при которых имеют место состоятельность 
и асимптотическая нормальность несмещенной оценки риска. Полученные результаты дают 
возможность строить асимптотические доверительные интервалы для погрешности пороговой 
обработки, используя только наблюдаемые данные.}

\KW{вейвлеты; пороговая обработка; несмещенная оценка риска; коррелированный шум; асимптотическая нормальность}

\DOI{10.14357/19922264140105}

\vskip 20pt plus 9pt minus 6pt

      \thispagestyle{headings}

      \begin{multicols}{2}

            \label{st\stat}

\section{Введение}

Статистические методы вейв\-лет-ана\-ли\-за широко применяются при анализе и обработке 
зашумленных сигналов и изображений. \mbox{Построение} вейвлет-оце\-нок 
для данных с коррелированным шумом обычно осуществляется с помощью 
применения мягкой пороговой обработки к эмпирическим вейв\-лет-ко\-эф\-фи\-ци\-ен\-там. 
Порог обычно зависит от уровня разложения, и его можно выбирать различными способами, 
исходя из постановки задачи и целей обработки (см., например,~[1--4]). Наличие шума 
неизбежно приводит к погрешностям в оцениваемом сиг\-на\-ле/изоб\-ра\-же\-нии. 
Свойства оценки таких погрешностей (риска) в модели с независимым шумом исследовались в 
работах~[1--12]. Показано, что при определенных условиях оценка риска является состоятельной 
и асимптотически нормальной. В данной работе исследуется асимптотическое поведение оценки 
риска в модели со стационарным коррелированным шумом. Для простоты изложения рассматриваются 
одномерные сигналы.

\section{Модель данных и~коэффициенты вейвлет-разложения}

Вейвлет-разложение функции $f\hm\in L^2(\mathbb{R})$, описывающей сигнал, представляет 
собой ряд
\begin{equation}
f=\sum\limits_{j,k\in Z}\langle f,\psi_{jk}\rangle\psi_{jk}\,,
\label{Wavelet_Decomp}
\end{equation}
где $\psi_{jk}(t)\hm=2^{j/2}\psi(2^jt-k)$, а $\psi(t)$~--- 
некоторая материнская вейв\-лет-функ\-ция (семейство $\{\psi_{jk}\}_{jk\in Z}$ 
образует ортонормированный базис в~$L^2(\mathbb{R})$). Индекс~$j$ 
в~\eqref{Wavelet_Decomp} называется масштабом, а индекс~$k$~--- сдвигом. 
Функция~$\psi$ должна удовлетворять определенным требованиям, однако ее 
можно выбрать таким образом, чтобы она обладала некоторыми полезными свойствами, 
например была дифференцируемой нужное число раз и имела заданное число~$M$ нулевых 
моментов~\cite{13-she}, т.\,е.
$$\int\limits_{-\infty}^{\infty}t^k\psi(t)\,dt=0\,,\enskip k=0,\ldots,M-1\,.
$$

В дальнейшем будут рассматриваться функции сигнала $f\hm\in L^2(\mathbb{R})$ 
на конечном отрезке $[a,b]$, равномерно регулярные по Липшицу с некоторым 
параметром $\gamma\hm>0$. Для таких функций известно~\cite{14-she}, 
что если вейв\-лет-функ\-ция~$M$~раз непрерывно дифференцируема 
($M\hm\geqslant\gamma$), имеет $M$ нулевых моментов и достаточно 
быстро убывает на бесконечности, т.\,е.\ существует такая константа $C_A\hm>0$, что
$$
\int\limits_{-\infty}^{\infty}\left(1+\abs{t}^{\gamma}\right)\abs{\psi(t)}\,dt\leqslant C_A\,,
$$
то найдется такая константа $A\hm>0$, что
\begin{equation}
\langle f,\psi_{jk}\rangle\leqslant\fr{A}{2^{j\left(\gamma+1/2\right)}}\,.
\label{Coeff_Decay}
\end{equation}

На практике функции сигнала всегда заданы в дискретных отсчетах на конечном отрезке. 
Не ограничивая общности, будем считать, что это отрезок $[0,1]$ и функция~$f$ задана в 
точках $i/2^J$ ($i\hm=1,\ldots, 2^J$): $f_i\hm=f\left(i/2^J\right)$.
Дискретное вейв\-лет-пре\-обра\-зо\-ва\-ние представляет собой умножение 
вектора значений функции~$f$ (обозначим его через~$\overline{f}$) на ортогональную матрицу~$W$, 
определяемую вейв\-лет-функ\-ци\-ей~$\psi$: $\overline{f}^{W}\hm=W\overline{f}$~\cite{14-she}. 
При этом дискретные вейв\-лет-ко\-эф\-фи\-ци\-ен\-ты связаны с непрерывными следующим образом: 
$f^{W}_{jk}\hm\approx 2^{J/2}\langle f,\psi_{jk}\rangle$ (см., например,~\cite{2-she} 
или~\cite{14-she}). Это приближение тем точнее, чем больше~$J$. 
Здесь не будут обсуждаться методы борьбы с краевыми эффектами, 
связанными с использованием вейв\-лет-раз\-ло\-же\-ния на конечном отрезке. 
Познакомиться с этими методами можно, например, в~\cite{15-she}.

В реальных наблюдениях всегда присутствует шум. Пусть $\{e_i, i \hm\in \mathbb{Z}\}$~--- 
стационарный гауссовский процесс с ковариационной по\-сле\-до\-ва\-тель\-ностью 
$r_k \hm= \cov (e_i,e_{i+k})$. Будем полагать, что $e_i$ имеют нулевое среднее и 
единичную дисперсию. Рас\-смот\-рим следующую модель данных:
\begin{equation*}
Y_j = f_j + e_j \qquad j = 1, \dots, 2^J\,.
%\label{Data_Model}
\end{equation*}

Для $t\in [0,1]$ определим наблюдаемый процесс
\begin{equation*}
Y_J(t) = \fr{1}{2^J} \sum\limits_{i=1}^{[2^Jt]} Y_i = 
F_J(t)+ \fr{1}{2^J} \sum\limits_{i=1}^{[2^Jt]} e_i\,,
\end{equation*}
где $F_J(t)=1/2^J \sum\limits_{i=1}^{[2^Jt]} f(i/2^J)$~--- 
<<суммарный сигнал>>. Как и в работах~\cite{16-she, 17-she}, рассмотрим два от\-дель\-ных случая.

\subsection{Модель краткосрочной зависимости}

Пусть $\sum\limits_{-\infty}^{+\infty} |r_k| < \infty$.
Положим $\tau^2 \hm= \sum\limits_{-\infty}^{+\infty} r_k$.
В~этом случае
\begin{equation}
2^{J/2}(Y_J(t)-F_J(t)) \Rightarrow \tau \mathbf{B}(t))\,, \enskip t \in [0,1]\,,
\label{SRD_limit}
\end{equation}
где $\mathbf{B}(t)$~--- процесс стандартного броуновского движения. Это следует, 
например, из~\cite{18-she} (лемма~5.1).
Проведем масштабирование $\epsilon\hm = \tau \cdot 2^{-J/2}$. Без ограничения общности 
далее будем полагать, что $\tau\hm=1$. Из~(\ref{SRD_limit}) следует, 
что можно аппроксимировать наблюдаемый процесс $Y_J(t)$ с помощью $Y(t)$ 
для $t \hm\in [0,1]$, где
\begin{equation}
Y(t) = F(t) + \epsilon \mathbf{B}(t)
\label{Scale_proc}
\end{equation}
и $F(t)= \int\limits_0^t f(s)\,d s$. Применяя к~(\ref{Scale_proc}) вейв\-лет-раз\-ло\-же\-ние 
и аппроксимируя его дискретным вейв\-лет-пре\-обра\-зо\-ва\-ни\-ем, приходим к следующей модели 
дискретных вейв\-лет-ко\-эф\-фи\-ци\-ен\-тов~\cite{16-she}:
\begin{equation}
X_{jk} = \mu_{jk} +  z_{jk}\,,
\label{Wav_SRD_model}
\end{equation}
где $\mu_{jk}=f^{W}_{jk}$, а  $z_{jk}$ независимы и имеют стандартное 
нормальное распределение.

\smallskip

\noindent
\textbf{Замечание 1.} В силу краевых эффектов нормальные случайные величины~$z_{jk}$ 
c нулевым средним, относящиеся к фиксированному уровню~$j$, на самом деле имеют 
ковариационную матрицу, удовлетворяющую условию
\begin{equation}
\sigma_a^2 I \leqslant \Gamma_j \leqslant \sigma_b^2 I,
\label{Wav_SRD_Cov}
\end{equation}
где неравенства рассматриваются в смысле неотрицательно определенных матриц и 
константы $\sigma_a$, $\sigma_b$ не зависят от~$j$, а только от выбранного вейв\-лет-ба\-зи\-са. 
Таким образом, модель~(\ref{Wav_SRD_model}) справедлива с точностью до 
ограничений~(\ref{Wav_SRD_Cov})~\cite{16-she}.

\subsection{Модель долгосрочной зависимости}

Теперь предположим, что автоковариационная функция шума убывает медленно согласно 
модели $r_k \hm\sim Ak^{-\alpha}$, где $0 \hm< \alpha \hm<1$.

Положим $\tau^2 \hm= 2A/((1\hm-\alpha)(2\hm-\alpha))$ (без ограничения общности 
далее будем полагать, что $\tau\hm=1$) и $H \hm= 1- \alpha/2 \hm\in (1/2,1)$.

Определим дробное броуновское движение $\mathbf{B}_H(t)$ как
гауссовский процесс на~$\mathbb{R}$ с нулевым средним  и
ковариационной функцией
\begin{equation*}
r(s,t) = \fr{V_H}{2}\left(|s|^{2H} + |t|^{2H} - |t-s|^{2H}\right)\,, \enskip s,t \in \mathbb{R}\,,
\end{equation*}
где
\begin{equation*}
V_H = \Variance (\mathbf{B}_H(1)) = \fr{-\Gamma(2-2H)\cos(\pi H)}{\pi H(2H-1)}\,.
\end{equation*}
Теперь лемма~5.1 из~\cite{18-she} показывает, что
\begin{equation*}
2^{\alpha J/2}(Y_J - F_J) \Rightarrow \tau\mathbf{B}_H(t)\,, \enskip t\in[0,1]\,.
\end{equation*}
Таким образом, полагая $\epsilon \hm= \tau^{1/\alpha}\cdot 2^{-J/2}$, 
можно аппроксимировать наблюдаемый процесс $Y_J(t)$ с помощью $Y(t)$ для $t\hm \in [0,1]$:
\begin{equation*}
Y(t) = F(t) + \epsilon^{\alpha} \textbf{B}_H(t)\,.
\end{equation*}

Так же как и в случае краткосрочной за\-ви\-си\-мости, можно перейти к следующей модели 
дискретных вейв\-лет-ко\-эф\-фи\-ци\-ен\-тов~\cite{16-she}:
\begin{equation}
X_{jk} = \mu_{jk} +  2^{(J-j)(1-\alpha)/2} z_{jk}\,,
\label{Wav_LRD_model}
\end{equation}
где $z_{jk}=2^{j(1-\alpha)/2} \int \psi_{jk}\, d\mathbf{B}_H$. 
Шумовые переменные $z_{jk}$ имеют стандартное нормальное распределение, 
но не являются независимыми. Однако можно показать, что они обладают 
ограниченной зависимостью, т.\,е.\ для всех~$j$ и~$k$
\begin{equation*}
0<c_0\leqslant \Variance(z_{jk}| z_{il}, i \neq j, k \neq l) \leqslant 1\,.
\end{equation*}

\section{Пороговая обработка и~оценка риска}

Смысл пороговой обработки вейв\-лет-ко\-эф\-фи\-ци\-ен\-тов заключается в 
удалении достаточно маленьких коэффициентов, которые считаются шумом. 
Будем использовать так называемую мягкую пороговую обработку с порогом~$T_j$, 
зависящим от уровня~$j$. К~каждому вейв\-лет-ко\-эф\-фи\-ци\-ен\-ту применяется 
функция $\rho_{T_j}(x)\hm=\textbf{sgn}(x)\left(\abs{x}-T_j\right)_{+}$, т.\,е.\ 
при такой пороговой обработке коэффициенты, которые по модулю меньше порога~$T_j$, 
обнуляются, а абсолютные величины остальных коэффициентов уменьшаются на величину 
порога. Погрешность (или риск) мягкой пороговой обработки определяется следующим образом:
\begin{equation}
R_J(f)=\sum\limits_{j=0}^{J-1}\sum\limits_{k=0}^{2^j-1}
\Expect\left(\mu_{jk}-\rho_{T_j}(X_{jk})\right)^2.
\label{Risk_def}
\end{equation}

В работах~\cite{2-she} и~\cite{3-she} было предложено использовать порог 
$T_j\hm=\sigma_j\sqrt{2\ln 2^J}$. Было показано, что при таком пороге риск 
близок к минимальному~\cite{2-she}. Этот порог получил название <<универсальный>>. 
В~дальнейшем будет использоваться именно такой вид порога. 
В~выражении~(\ref{Risk_def}) присутствуют неизвестные величины~$\mu_{jk}$, 
поэтому вычислить значение $R_J(f)$ нельзя. Однако его можно оценить. 
В~качестве оценки риска используется следующая величина~\cite{1-she}:
\begin{equation}
\widehat{R}_J(f)=\sum\limits_{j=0}^{J-1}\sum\limits_{k=0}^{2^j-1}F\left[X_{jk}^2,T_j,\sigma_j\right]\,,
\label{Risk_Est}
\end{equation}
где 
\begin{multline*}
F[x,T,\sigma]=\left(x-\sigma^2\right)\Ik(|x|\leqslant T^2)+{}\\
{}+
(\sigma^2+T^2)\Ik (|x|\hm>T^2)\,.
\end{multline*}

Величина $\widehat{R}_J(f)$ является несмещенной оценкой для $R_N(f)$~\cite{14-she}. 
В~работах~[6--12] исследовались асимптотические свойства оценки~(\ref{Risk_Est}) 
в модели с независимым шумом. Было показано, что при определенных условиях гладкости 
эта оценка является\linebreak состоятельной и асимптотически нормальной.\linebreak Поскольку для модели 
данных с краткосрочной зависимостью принимается модель эмпирических 
вейв\-лет-ко\-эф\-фи\-ци\-ен\-тов~(\ref{Wav_SRD_model}), с точностью до ограничений~(\ref{Wav_SRD_Cov}) 
все результаты указанных работ переносятся на этот случай. Далее будет исследовано асимптотическое 
поведение оценки~(\ref{Risk_Est}) в модели данных с долгосрочной зависимостью.

\section{Вспомогательные результаты}

В этом разделе будут получены некоторые результаты, касающиеся характера 
зависимости эмпирических вейв\-лет-ко\-эф\-фи\-ци\-ен\-тов. Всюду далее предполагается, 
что используются вейвлеты \mbox{Мейера}~\cite{14-she}, обладающие нужным количеством нулевых 
моментов и непрерывных производных. Рассмотрим ковариацию случайных величин в 
модели~(\ref{Wav_LRD_model})~\cite{16-she}:
\begin{multline*}
\cov(X_{jk},X_{il})= 2^{J(1-\alpha)} \Expect \int \psi_{jk} \,
d\mathbf{B}_H \overline{\int \psi_{il}\, d\mathbf{B}_H}={}\\
{}=
2^{J(1-\alpha)}\fr{1}{2\pi} \int \hat{\psi_{jk}}(\xi)
\overline{\hat{\psi_{il}}(\xi)}|\xi|^{-(1-\alpha)}\, d\xi\,.
\end{multline*}
Для преобразования Фурье вейв\-лет-функ\-ции справедливо
$$\hat{\psi}_{jk}(\xi) = 2^{-j/2} e^{ik\cdot 2^{-j}}\hat{\psi}(2^{-j}\xi)\,.
$$
Рассмотрим ковариацию в пределах одного уровня~$j$:
\begin{multline*}
\cov(X_{j0},X_{jk}) ={}\\
{}=
2^{(J-j)(1-\alpha)} \fr{1}{2\pi}\int e^{ik\xi}|\hat{\psi}(\xi)|^2 |\xi|^{-(1-\alpha)}\,d\xi\,.
\end{multline*}
Поскольку вейв\-лет-функ\-ция~$\psi$ имеет $M$ нулевых моментов и $M$ непрерывных 
производных, найдется такая константа $C_M\hm>0$, что
\begin{equation}
|\cov(X_{j0},X_{jk})| \leqslant  \fr{C_M 2^{(J-j)(1-\alpha)}}{k^M}\,.
\label{Cov_Single_Decay}
\end{equation}


\smallskip

\noindent
\textbf{Замечание~2.}
Для вейв\-лет-функ\-ции Мейера $\psi(x)$ при любом натуральном~$M_0$ существует 
константа $C_{M_0}\hm>0$ такая, что $|\hat{\psi}(\xi)|\hm\leqslant C_{M_0} |\xi|^{M_0} 
\mathbf{1}_{\xi \in \mathrm{supp}\left(\hat{\psi}\right)}$.

\smallskip

Теперь оценим ковариацию на разных уровнях $\cov(X_{jk},X_{il})$, предполагая, что $j\hm>i$:
\begin{multline*}
\cov(X_{jk},X_{il}) = 2^{J(1-\alpha)} \fr{1}{2\pi} \int 
2^{-{j}/{2}}e^{ik\xi\cdot 2^{-j}}\times{}\\
{}\times \hat{\psi}(2^{-j}\xi) 
2^{-{i}/{2}} \overline{e^{il\xi\cdot 2^{-i}}\hat{\psi}(2^{-i}\xi)} 
|\xi|^{-(1-\alpha)} d \xi = {}\\
{}=2^{J(1-\alpha)}\fr{1}{2\pi}2^{-(\Delta/2)-i(1-\alpha)} \times{}\\
{}\times
\int e^{i\xi(k\cdot 2^{-\Delta}-l)}\hat{\psi}(2^{-\Delta}\xi) 
\overline{\hat{\psi}(\xi)} |\xi|^{-(1-\alpha)} \,d \xi\,,
\end{multline*}
где $\Delta\hm=j-i$. Далее, в силу замечания~2
\begin{multline*}
|\cov(X_{jk},X_{il})|\leqslant{}\\
{}\leqslant
2^{J(1-\alpha)}\fr{C_{M_0}}{2\pi}2^{-({\Delta}/{2})-i(1-\alpha)}\cdot
2^{-\Delta M_0} \times{}\\
{}\times
\left\vert
\int e^{i\xi(k\cdot2^{-\Delta}-l)} \overline{\hat{\psi}(\xi)} |\xi|^{M_0-(1-\alpha)}\,
d \xi \right\vert\,.
\end{multline*}
Рассмотрим два случая:
\begin{enumerate}[(1)]
\item $|k\cdot2^{-\Delta}-l| > 1$:
тогда в силу свойств преобразования Фурье и гладкости выбранной 
вейв\-лет-функ\-ции найдется такая константа $C_1\hm>0$, что
\begin{multline*}
|\cov(X_{jk},X_{il})|
\leqslant {}\\
\hspace*{-19pt}{}\leqslant 2^{J(1-\alpha)} \fr{1}{2\pi}\,2^{-({\Delta}/{2})-i(1-\alpha)}\cdot
2^{-\Delta M_0} \fr{C_1}{(k\cdot2^{-\Delta}-l)^M}\,;\hspace*{-8.9667pt} %\notag
\end{multline*}

\item $|k\cdot2^{-\Delta}-l| \leqslant 1$: тогда найдется такая константа $C_2\hm>0$, что
\begin{equation*}
\left\vert \int e^{i\xi(k\cdot2^{-\Delta}-l)} \overline{\hat{\psi}(\xi)} 
|\xi|^{M_0-(1-\alpha)} \,d \xi \right\vert \leqslant C_2
\end{equation*}
и
\begin{multline*}
|\cov(X_{jk},X_{il})|
\leqslant {}\\
{}\leqslant C_2 \cdot 2^{J(1-\alpha)}\fr{C_{M_0}}{2\pi}\,
2^{-({\Delta}/{2})-i(1-\alpha)}\cdot 2^{-\Delta M_0}. %\notag
\end{multline*}
\end{enumerate}
Следовательно, найдется такая константа $C_{Me}\hm>0$, что

%\pagebreak

\noindent
\begin{multline}
|\cov(X_{jk},X_{il})| \leqslant{}\\
\!\!{}\leqslant
    \begin{cases}
        C_{Me}\cdot2^{J(1-\alpha)-({\Delta}/{2})-i(1-\alpha)}\cdot2^{-\Delta M_0} \times{}\\
         \hspace*{5mm}\times \fr{1}{(k\cdot2^{-\Delta}-l)^M}\,, &  \hspace*{-30mm}
         \left\vert k\cdot2^{-\Delta}-l\right\vert > 1; \\
        C_{Me}\cdot2^{J(1-\alpha)-({\Delta}/{2})-i(1-\alpha)}\cdot2^{-\Delta M_0}\,, & \\
        \hspace*{34mm}\left\vert k\cdot2^{-\Delta}-l\right\vert \leqslant 1\,. &
    \end{cases}\!\!
    \label{Cov_Multi_Decay}
\end{multline}
Рассмотрим теперь структуру дисперсии оценки риска.
Введем обозначение для последовательностей
$a_J\simeq b_j$, если $\lim ({a_J}/{b_J}) \hm= 1$ при $J\hm\rightarrow \infty$.

\bigskip

\noindent
\textbf{Лемма 1.}
\textit{Пусть $\alpha\hm>1/2$ и $\gamma\hm>(4\alpha\hm-2)^{-1}$, тогда 
$D_J^2 \hm=\Variance \widehat{R}_J(f) \simeq  C_{\alpha}\cdot2^J,$ 
где константа $C_{\alpha}$ зависит от~$\alpha$, но не зависит от функции сигнала~$f$.}

\medskip

\noindent
Д\,о\,к\,а\,з\,а\,т\,е\,л\,ь\,с\,т\,в\,о\,.\ \ 
При выполнении условий леммы $(2\gamma +1)^{-1}\hm<1\hm-(2\alpha)^{-1}$.
Выберем $p''$ такое, что $(2\gamma \hm+1)^{-1}\hm<p''\hm<1\hm-(2\alpha)^{-1}$ и 
$p''J$~--- целое число. Тогда в силу~(\ref{Coeff_Decay}) $\mu_{jk}\hm\rightarrow 0$ 
для всех $j \colon p''J\leqslant j \hm<J$ при $J\hm\rightarrow \infty$. 
Разобьем выражение~(\ref{Risk_Est}) на две суммы:
\begin{multline*}
\widehat{R}_J(f)=\sum\limits_{j=0}^{p''J}
\sum\limits_{k=0}^{2^j-1}F\left[X_{jk}^2,T_j,\sigma_j\right]+{}\\
{}+
\sum\limits_{j=p''J+1}^{J-1}\sum\limits_{k=0}^{2^j-1}F\left[X_{jk}^2,T_j,\sigma_j\right]\,.
\end{multline*}
Так как существует такая константа $C_F\hm>0$, что $F[X_{jk}^2,T_j,\sigma_j] \hm\leqslant 
C_F T^2_j\hm=C_F J\cdot2^{(J-j)(1-\alpha)}$, для первой суммы имеем
\begin{multline}
\hspace*{-9pt}\sum_{i=0}^{p''J}\sum\limits_{l=0}^{2^i-1} F[X_{jk}^2,T_j,\sigma_j]
\leqslant C_F\sum\limits_{i=0}^{p''J}\sum\limits_{l=0}^{2^i-1} С 
J\cdot2^{(J-i)(1-\alpha)} \leqslant {}\\
{}\leqslant C'_F  J \cdot2^{J(1-\alpha+\alpha p'')}
\label{Large_Coeff_Order}
\end{multline}
с некоторой константой $C'_F\hm>0$. Далее
\begin{multline}
\Variance \widehat{R}_J(f)
= \sum\limits_{j=0}^{J-1}\sum\limits_{k=0}^{2^j-1} \Variance F[X_{jk}^2,T_j,\sigma_j]
+ \sum\limits_{i=0}^{J-1} \sum\limits_{l=0}^{2^i-1}\times{}\\
\hspace*{-5mm}{}\times \sum\limits_{j=0}^{J-1}
\sum\limits_{k=0}^{2^j-1} \cov\left(F[X_{il}^2,T_i,\sigma_i]^2,F[X_{jk}^2,T_j,\sigma_j]^2\right).
\label{Var_Struct}
\end{multline}
Рассмотрим сумму дисперсий:
\begin{multline*}
\sum\limits_{j=0}^{J-1}\sum\limits_{k=0}^{2^j-1} \Variance 
F\left[X_{jk}^2,T_j,\sigma_j\right]=
\sum\limits_{j=0}^{p''J}\sum\limits_{k=0}^{2^j-1} \Variance 
F\left[X_{jk}^2,T_j,\sigma_j\right]+{}\\
{}+\sum\limits_{j=p''J+1}^{J-1}\sum\limits_{k=0}^{2^j-1} 
\Variance F\left[X_{jk}^2,T_j,\sigma_j\right].
\end{multline*}
В силу~(\ref{Large_Coeff_Order}) первая сумма не превосходит 
$C''_F  J^2\times$\linebreak $\times\; 2^{2J(1-\alpha+\alpha p'')}$, где $C''_F$~--- 
некоторая положительная константа, и, так как $p''\hm<1\hm-(2\alpha)^{-1}$, 
имеем $2(1\hm-\alpha\hm+\alpha p'')\hm<1$. Учитывая вид порога~$T_j$ 
и принимая во внимание, что $\alpha\hm>1/2$, для второй суммы имеем:
\begin{multline*}
\sum\limits_{j=p''J+1}^{J-1}\sum\limits_{k=0}^{2^j-1} \Variance 
F\left[X_{jk}^2,T_j,\sigma_j\right]\simeq
\sum\limits_{j=p''J+1}^{J-1}\sum\limits_{k=0}^{2^j-1}\Variance X_{jk}^2={} \\
{}=\sum\limits_{j=p''J+1}^{J-1}\sum\limits_{k=0}^{2^j-1} 
2\sigma_j^2(\sigma_j^2 +\mu_{jk}^2)\simeq \sum\limits_{j=p''J+1}^{J-1}
\sum\limits_{k=0}^{2^j-1} 2\sigma_j^4
= {}\\
{}= \sum\limits_{j=p''J+1}^{J-1}\sum\limits_{k=0}^{2^j-1} 
2\fr{2^{2J(1-\alpha)}}{2^{2j(1-\alpha)}} = {}\\
{}= 2^{2J(1-\alpha)+1}\!\!\! \sum\limits_{j=p''J+1}^{J-1}\!\!\! 2^{-j+2\alpha j} 
\simeq 2^{J}\fr{2}{2^{2\alpha-1}(2^{2\alpha-1}-1)}.\hspace*{-4.40202pt}
\end{multline*}
Таким образом,
\begin{multline}
\sum\limits_{j=0}^{J-1}\sum\limits_{k=0}^{2^j-1} 
\Variance F\left[X_{jk}^2,T_j,\sigma_j\right]\simeq{}\\
{}\simeq
\sum\limits_{j=p''J+1}^{J}\sum\limits_{k=0}^{2^j-1}\Variance X_{jk}^2\simeq C'_\alpha2^{J}\,,
\label{Var_Order}
\end{multline}
где $C'_\alpha$~--- положительная константа.
Рассмотрим теперь сумму ковариаций в~(\ref{Var_Struct}). Аналогично сумме дисперсий имеем:
\begin{multline*}
\sum\limits_{i=0}^{J-1} \sum\limits_{l=0}^{2^i-1}\sum\limits_{j=0}^{J-1}\times{}\\
{}\times
\sum\limits_{k=0}^{2^j-1} \!\!\cov\left(F[X_{jk}^2,T_j,\sigma_j],
F\left[X_{jk}^2,T_j,\sigma_j\right]\right)\simeq{}\\
{}\simeq
\sum\limits_{i=p''J+1}^{J-1} \sum\limits_{l=0}^{2^i-1}\sum\limits_{j=p''J+1}^{J-1}
\sum\limits_{k=0}^{2^j-1} \cov(X_{il}^2,X_{jk}^2)\,.
\end{multline*}
Известно, что если вектор $(X,Y)$ имеет двумерное нормальное распределение, то
\begin{equation}
\hspace*{-2mm}\cov(X^2,Y^2) = 4\Expect X \Expect Y\cov(X,Y) + 2\cov^2(X,Y).\!\!
\label{covrazl}
\end{equation}
Используя~(\ref{Cov_Single_Decay}), (\ref{Cov_Multi_Decay}) и~(\ref{covrazl}), получаем
\begin{multline*}
\fr{1}{2}\sum\limits_{i=p''J+1}^{J-1} \sum\limits_{l=0}^{2^i-1}
\sum\limits_{j=p''J+1}^{J-1}\sum\limits_{k=0}^{2^j-1} \cov^2(X_{il},X_{jk})={}\\
{}=\!\!\!\!
\sum\limits_{i=p''J+1}^{J-1} \sum\limits_{l=0}^{2^i-1}
\sum\limits_{\Delta=0}^{J-i-1} \sum_{k={\protect\scriptstyle\begin{cases}{\scriptstyle l+1,} 
&{\scriptstyle \Delta=0;} \\[-6pt]{\scriptstyle 0,} & 
{\scriptstyle \Delta>0.} \end{cases}}}^{2^{i+\Delta}-1}\!\!\!\!\!\! \!\!\!\!\!\!\cov^2(X_{il},X_{i+\Delta,k}) 
\leqslant{}\hspace*{-5.1336pt}
\end{multline*}

\noindent
\begin{multline}
{}
\leqslant \sum\limits_{i=p''J+1}^{J-1} \sum\limits_{l=0}^{2^i-1} \Bigl( 
\sum\limits_{\delta=1}^{2^i-l} C_M^2\delta^{-2M_0}\cdot2^{2(J-i)(1-\alpha)}+{}
\\[3pt]
{}+\sum\limits_{\Delta=1}^{J-i-1}\sum\limits_{k=0}^{2^{i+\Delta}-1} 
2^{2(J-i)(1-\alpha)}\cdot 2^{-\Delta(1+2M_0)}\times{}\\[3pt]
{}\times \fr{\mathbf{1}_{\{|k\cdot2^{-\Delta}-l|>1\}}}
{|k\cdot2^{-\Delta}-l|^{2M}}\,C^2_{Me} +
+\sum\limits_{\Delta=1}^{J-i-1}\sum\limits_{k=0}^{2^{i+\Delta}-1} 
2^{2(J-i)(1-\alpha)}\times{}\\[3pt]
{}\times 2^{-\Delta(1+2M_0)}\cdot \mathbf{1}_{\{|k\cdot2^{-\Delta}-l|\leqslant 1\}} 
C^2_{Me}\Bigr) \leqslant{}\\[3pt]
{}\leqslant \sum\limits_{i=p''J+1}^{J-1} \left( \sum\limits_{l=0}^{2^i-1}
\sum\limits_{\delta=1}^{2^i-l-1} C_M^2\delta^{-2M_0}2^{2(J-i)(1-\alpha)} +{}\right.
\\[3pt]
{}+\sum\limits_{\Delta=1}^{J-i-1}2^{2(J-i)(1-\alpha)}\cdot 2^{-\Delta(1+2M_0)} \times{}\\[3pt]
\left.{}\times
\sum\limits_{l=0}^{2^i-1}C^2_{Me}
\left( \sum\limits_{k=0}^{2^{i+\Delta}-1} 
\fr{\mathbf{1}_{\{|k\cdot2^{-\Delta}-l|>1\}}}{|k\cdot2^{-\Delta}-l|^{2M}} 
+ 2^{\Delta+1}\right)\right) \leqslant{}\\[3pt]
{}\leqslant \sum\limits_{i=p''J+1}^{J-1} 2^{2(J-i)(1-\alpha)}
 \left(\vphantom{\sum\limits_{\Delta=1}^{J-1}}
C_M^2 \cdot2^i H_0 +{}\right.\\[3pt]
\left.{}+\sum\limits_{\Delta=1}^{J-i-1} \left( 2^{i-2M_0\Delta} H_{1}
+ 2^{i-2M_0\Delta+1} H_{2}\right) 
\vphantom{\sum\limits_{\Delta=1}^{J-1}}
\right)\leqslant{}
\\[3pt]
{}\leqslant \sum\limits_{i=p''J+1}^{J-1} 2^{2(J-i)(1-\alpha)}\cdot 2^iH_3\leqslant C''_\alpha
\cdot 2^{J}\,,
\label{Cov2_Estimate}
\end{multline}
где $H_0$, $H_1$, $H_2$, $H_3$ и $C''_\alpha$~--- константы, зависящие от~$\alpha$.
Аналогично с учетом~(\ref{Coeff_Decay})
\begin{multline}
\sum\limits_{i=p''J+1}^{J-1} \sum\limits_{l=0}^{2^i-1}\sum\limits_{j=p''J+1}^{J-1}
\sum\limits_{k=0}^{2^j-1} \mu_{il}\mu_{jk}\cov(X_{il},X_{jk})\leqslant {}\\[3pt]
{}\leqslant
H_\alpha\cdot 2^{J(1-\alpha(1-p''))}\,,
\label{Cov1_Estimate}
\end{multline}
где $H_\alpha$~--- константа, зависящая от~$\alpha$.

Объединяя (\ref{Var_Order}), (\ref{Cov2_Estimate}) и~(\ref{Cov1_Estimate}), получаем, что 
$\Variance \widehat{R}_J(f) \simeq  C_{\alpha}\cdot 2^J$. Лемма доказана.

\smallskip

Докажем еще одно свойство эмпирических вейв\-лет-ко\-эф\-фи\-ци\-ен\-тов. 
Говорят, что последовательность случайных величин $\{Y_i\}_{i=1}^\infty$ 
обладает свойством $\rho$-пе\-ре\-ме\-ши\-ва\-ния, если для функции
\begin{equation*}
\rho(m) = \sup\limits_{i,j:|i-j|>m} \corr(Y_i,Y_j)
\end{equation*}
справедливо $\rho(m)\hm\rightarrow0$ при $m\hm\rightarrow\infty$.

\pagebreak

%\medskip

\noindent
\textbf{Лемма 2.}
Последовательность $\left\{F[X_{jk}^2,T_j,\sigma_j]\right\}$, $j \hm=0, \dots, J-1, k=1,
\dots,2^j,$ обладает свойством $\rho$-пе\-ре\-ме\-ши\-ва\-ния. 
Причем для некоторой положительной константы~$C_\rho$
$$
\rho(m) \leqslant \begin{cases}
       \fr{C_\rho}{(m+1)^{2M_0}}  &\ \mbox{для элементов}\\[-9pt]
       &\ \mbox{на одном уровне } (i=j); \\
       \fr{C_\rho}{2^{(m+1)\alpha}}  &\ \mbox{для элементов}\\[-9pt]
       &\ \mbox{на разных уровнях}. 
\end{cases}
$$

\medskip

\noindent
Д\,о\,к\,а\,з\,а\,т\,е\,л\,ь\,с\,т\,в\,о\,.\ \ 
Рассмотрим функцию перемешивания между элементами с номерами~$l$ и~$k$ на одном уровне~$i$, 
$i \hm=0, \dots, J-1$. Для некоторой константы $C_\rho\hm>0$ имеем:
\begin{multline}
\rho(m)={}\\
{}=
\sup\limits_{{\begin{smallmatrix}0\leqslant i\leqslant J-1,\\k,l:|k-l|>m\end{smallmatrix}}} 
\fr{\cov(F[X_{il}^2,T_i,\sigma_i],F[X_{ik}^2,T_i,\sigma_i])}
{\sqrt{\Variance F[X_{il}^2,T_i,\sigma_i] \Variance F[X_{ik}^2,T_i,\sigma_i]}}
\leqslant{}\\
{}\leqslant \sup\limits_{{\begin{smallmatrix}0\leqslant i\leqslant J-1,\\k,l:|k-l|>m\end{smallmatrix}}} 
C_\rho\fr{2^{2(J-i)(1-\alpha)} |k-l|^{-2M_0}}{\sqrt{\sigma^8_i}}\leqslant{}\\
{} \leqslant 
\sup_{{\begin{smallmatrix}0\leqslant i\leqslant J-1,\\k,l:|k-l|>m\end{smallmatrix}}}  
C_\rho\fr{2^{2(J-i)(1-\alpha)} |k-l|^{-2M_0}}{2^{2(J-i)(1-\alpha)} }
\leqslant{}\\
{}\leqslant \sup\limits_{{\begin{smallmatrix}0\leqslant 
i\leqslant J-1,\\k,l:|k-l|>m\end{smallmatrix}}}  C_\rho\fr{1}{|k-l|^{2M_0}} = 
\fr{C_\rho}{(m+1)^{2M_0}}\,.\notag
\end{multline}

Далее обратимся к функции перемешивания для элементов 
$F\left[X_{il}^2,T_i,\sigma_i\right]$ и 
$F\left[X_{jk}^2,T_j,\sigma_j\right]$ на разных уровнях $i,j: j\hm>i, j\hm-i\hm=\Delta\hm>0$.
Рассмотрим случай $|k\cdot 2^{-\Delta}-l|>1$. Имеем:
\begin{multline*}
\rho(m)
={}\\
{}=\sup\limits_{{\begin{smallmatrix}j-i=\Delta>m,\\ 
0\leqslant l\leqslant2^i-1,\\
0\leqslant k\leqslant2^j-1,\\|k\cdot2^{-\Delta}-l|>1.\end{smallmatrix}}}
\!\!\!\!\!\!\!\fr{\cov(F[X_{il}^2,T_i,\sigma_i],F[X_{jk}^2,T_j,\sigma_j])}
{\sqrt{\Variance F[X_{il}^2,T_i,\sigma_i] \Variance F[X_{jk}^2,T_j,\sigma_j]}}\leqslant{}\\
{}\leqslant \sup\limits_{{\begin{smallmatrix}j-i=\Delta>m,\\ 0\leqslant
 l\leqslant2^i-1,\\ 0\leqslant k\leqslant2^j-1,\\|k\cdot2^{-\Delta}-l|>1.\end{smallmatrix}}} 
 \!\!\!\!\!\!\!\!\!\!\!C_\rho\fr{2^{2(J-i)(1-\alpha) -\Delta} |k\cdot2^{-\Delta}-l|^{-2M_0}}
 {\sqrt{\sigma^4_j\sigma^4_i}}\leqslant{}\\
  {}\leqslant\sup\limits_{{\begin{smallmatrix}j-i=\Delta>m,\\ 0\leqslant l\leqslant2^i-1,\\ 0\leqslant k\leqslant2^j-1,\\|k\cdot2^{-\Delta}-l|>1.\end{smallmatrix}}} 
\!\!\!\!\!\!\!\!\!\!\!C_\rho\fr{2^{2(J-i)(1-\alpha) -\Delta} |k\cdot2^{-\Delta}-l|^{-2M_0}}{2^{2(J-i)(1-\alpha) -\Delta(1-\alpha)}}
\leqslant{}  %\notag
\end{multline*}

\noindent
\begin{multline*} 
{}\leqslant \sup\limits_{{\begin{smallmatrix}\Delta>m,\\ 0\leqslant l\leqslant2^i-1,\\ 0\leqslant k\leqslant2^j-1,\\|k\cdot2^{-\Delta}-l|>1.\end{smallmatrix}}} 
\!\!\!\!\!\!\!\!\!\!\!C_\rho\fr{1}{2^{\Delta\alpha}}\,\fr{1}{|k\cdot2^{-\Delta}-l|^{2M_0}}\leqslant{}\\
{}\leqslant \sup_{\Delta>m}  C_\rho\fr{1}{2^{\Delta\alpha}}
=\fr{C_\rho}{2^{(m+1)\alpha}}. %\notag
\end{multline*}

Теперь рассмотрим случай $|k\cdot2^{-\Delta}\hm-l|\hm\leqslant1$. Имеем для некоторой константы 
$C'_\rho\hm>0$:
\begin{multline*}
\rho(m)
={}\\
{}=\!\!\!\!\!\sup\limits_{{\begin{smallmatrix}j-i=\Delta>m,\\ 0\leqslant l\leqslant2^i-1,\\
 0\leqslant k\leqslant2^j-1,\\|k\cdot2^{-\Delta}-l|\leqslant1.\end{smallmatrix}}}
\!\!\!\fr{\cov(F[X_{il}^2,T_i,\sigma_i],F[X_{jk}^2,T_j,\sigma_j])}
{\sqrt{\Variance F[X_{il}^2,T_i,\sigma_i] \Variance F[X_{jk}^2,T_j,\sigma_j]}}
\leqslant{}\\
{}\leqslant\sup\limits_{j-i=\Delta>m} C'_\rho\fr{2^{2(J-i)(1-\alpha) -\Delta} C_{Me}}
{\sqrt{\sigma^4_j\sigma^4_i}}\leqslant{}\\
{}\leqslant \sup\limits_{j-i=\Delta>m} C'_\rho\fr{2^{2(J-i)(1-\alpha) -\Delta} C_{Me}}{2^{2(J-i)(1-\alpha) -\Delta(1-\alpha)}}
\leqslant{}\\
{}\leqslant \sup\limits_{\Delta,k,l:\Delta>m} C_{\rho}\fr{1}{2^{\Delta\alpha}}=\frac{C_{\rho}}{2^{(m+1)\alpha}}\,.
\end{multline*}

Следовательно, получаем
$$
\rho(m) \leqslant
\begin{cases}
       \fr{C_{\rho}}{(m+1)^{2M_0}}  &\ \mbox{на одном уровне;} \\
       \fr{C_{\rho}}{2^{(m+1)\alpha}}  &\ \mbox{на разных уровнях.} \\
\end{cases}
$$
При $m \hm\rightarrow \infty$ $\rho(m) \rightarrow 0$  что и доказывает утверждение леммы.


\section{Основные теоремы}

Докажем асимптотическую нормальность оценки риска.

\medskip

\noindent
\textbf{Теорема 1.} \textit{Пусть $\alpha>1/2$ и функция~$f$ 
регулярна с параметром $\gamma \hm> (4\alpha-2)^{-1}$. Тогда при 
пороговой обработке с <<универсальным>> порогом $T_j$ имеет место 
сходимость по распределению}:
\begin{align}
\fr{\widehat{R}_J(f) - R_J(f)}{ D_J } \Rightarrow \textbf{N}(0,1)\,, \enskip
J \rightarrow \infty\,,
\label{asnorm}
\end{align}
\textit{где $D_J^2\hm=C_\alpha 2^J$, а константа~$C_\alpha$ зависит только 
от~$\alpha$ и выбранного вейв\-лет-ба\-зиса}.

\medskip

\noindent
Д\,о\,к\,а\,з\,а\,т\,е\,л\,ь\,с\,т\,в\,о\,.\ \ Из леммы~1 следует, что 
$\Variance \widehat{R}_J(f)\simeq D_J^2\hm=C_\alpha 2^J$. 
Разобьем выражение в~(\ref{asnorm}) на две суммы, как это было сделано в лемме~1:

\noindent
\begin{multline*}
\fr{\widehat{R}_J(f) - R_J(f)}{ D_J }={}\\[3pt]
{}=
\fr{\sum\limits_{j=0}^{p''J}\sum\limits_{k=0}^{2^j-1}
\left(F[X_{jk}^2,T_j,\sigma_j]-\Expect F[X_{jk}^2,T_j,\sigma_j]\right)}{ D_J }+{}\\[3pt]
{}+\fr{\sum\limits_{j=p''J+1}^{J-1}\sum\limits_{k=0}^{2^j-1}
\left(F[X_{jk}^2,T_j,\sigma_j]-\Expect F[X_{jk}^2,T_j,\sigma_j]\right)}{ D_J }\,,
\end{multline*}
где $p''>(2\gamma+1)^{-1}$. Поскольку $\gamma\hm > (4\alpha-2)^{-1}$, 
имеем $(2\gamma+1)^{-1}\hm<1\hm-(2\alpha)^{-1}$. Следовательно, можно выбрать такое~$p''$, 
что $(2\gamma\hm+1)^{-1}\hm<p''\hm<1\hm-(2\alpha)^{-1}$. Отсюда 
$1\hm-\alpha\hm+\alpha p''\hm<1/2$, и из~(\ref{Large_Coeff_Order}) 
следует, что первая сумма стремится к нулю почти всюду (п.в.).

Далее, действуя как в лемме~1, нетрудно показать, что
\begin{multline*}
\sup_J \fr{1}{D^2_J}\sum\limits_{j=p''J+1}^{J-1}\sum\limits_{k=0}^{2^j-1} 
\Expect  \bigl( F\left[X_{jk}^2,T_j,\sigma_j\right] -{}\\[3pt]
 {}-
\Expect  F\left[X_{jk}^2,T_j,\sigma_j\right]\bigr)^2 ={}\\[3pt]
=\sup_J \fr{1}{D^2_J}\sum\limits_{j=p''J+1}^{J-1}\sum\limits_{k=0}^{2^j-1} \Variance  
F\left[X_{jk}^2,T_j,\sigma_j\right] < \infty\,.
%\label{Norm_Cond1}
\end{multline*}

Из леммы~2 следует, что последовательность $\bigl\{F[X_{jk}^2,T_j,\sigma_j]\bigr\}$, 
$j \hm=0, \dots, J-1, k=1,\dots,2^j,$ обладает свойством $\rho$-пе\-ре\-ме\-ши\-ва\-ния 
и, следовательно, обладает свойством $\alpha$-пе\-ре\-ме\-ши\-ва\-ния~\cite{19-she}.
\par Наконец, выполнено условие Линдеберга: для любого $\epsilon\hm>0$
\begin{multline}
\fr{1}{D^2_J}\sum\limits_{j=p''J+1}^{J-1}\sum\limits_{k=0}^{2^j-1} \Expect  
\left( F\left[X_{jk}^2,T_j,\sigma_j\right] - {}\right.\\[3pt]
\left.{}-
\Expect F\left[X_{jk}^2,T_j,\sigma_j\right]\right)^2\times{}\\[3pt] 
{}\times\mathbf{1}\bigl( \left\vert F\left[X_{jk}^2,T_j,\sigma_j\right] - \Expect  
F\left[X_{jk}^2,T_j,\sigma_j\right]\right\vert >\epsilon D_J\bigr)\rightarrow {}\\[3pt]
{}\rightarrow
0\,,\ J\rightarrow\infty\,.
\label{Norm_Cond2}
\end{multline}
Действительно, так как $F[X_{jk}^2,T_j,\sigma_j]\hm \leqslant C_F
J\times$\linebreak $\times\;2^{(J-j)(1-\alpha)}$ (см.\ лемму~1), а $D_J^2\hm=C_\alpha \cdot2^J$, то при
$\alpha\hm>1/2$, начиная с некоторого~$J$, все индикаторы в~(\ref{Norm_Cond2}) обращаются в ноль.

Таким образом, выполнены все условия теоремы~2.1 из работы~\cite{20-she} 
и справедлива сходимость по распределению~(\ref{asnorm}). Теорема доказана.

\medskip

Докажем теперь теорему о состоятельности оценки риска, справедливую 
при более слабых ограничениях на~$\alpha$ и~$\gamma$.

\medskip

\noindent
\textbf{Теорема 2.}
\textit{Пусть $0<\alpha<1$, функция~$f$ регулярна с параметром $\gamma\hm>0$ и 
$b\hm>1\hm-\alpha\hm+\alpha(2\gamma\hm+1)^{-1}$.
Тогда при пороговой обработке с <<универсальным>> порогом~$T_j$ выполняется}
\begin{equation}
\fr{\widehat{R}_J(f) - R_J(f)}{2^{bJ}}\xrightarrow{\Prob} 0\,,\enskip J\rightarrow \infty\,.
\label{utv2l}
\end{equation}

\medskip

\noindent
Д\,о\,к\,а\,з\,а\,т\,е\,л\,ь\,с\,т\,в\,о\,.\ \  
Условия теоремы позволяют выбрать~$p''$ таким, что 
$(2\gamma \hm+1)^{-1}\hm<p''\hm<(b\hm+\alpha\hm-1)/\alpha$. Разобьем выражение 
в~(\ref{utv2l}) на две суммы:
\begin{multline*}
\widehat{R}_J(f) - R_J(f)={}\\
{}=\sum\limits_{j=0}^{p''J}
\sum\limits_{k=0}^{2^j-1}\left(F[X_{jk}^2,T_j,\sigma_j]-
\Expect F[X_{jk}^2,T_j,\sigma_j]\right)+{}\\
{}+\sum\limits_{j=p''J+1}^{J-1}\sum\limits_{k=0}^{2^j-1}\left(F[X_{jk}^2,T_j,\sigma_j]-
\Expect F[X_{jk}^2,T_j,\sigma_j]\right)\equiv{}\\
{}\equiv R_1+R_2\,.
\end{multline*}
Повторяя рассуждения леммы~1, получаем, что $|R_1|\hm\leqslant C_1  J\cdot 
2^{J(1-\alpha+\alpha p'')}$ 
с некоторой константой $C_1\hm>0$. При этом $1\hm-\alpha\hm+\alpha p''\hm<b$ в 
силу выбора~$p''$. Следовательно, $R_1\cdot 2^{-bJ}\xrightarrow{\Prob} 0$.
Далее, так же как в лемме~1, можно показать, что для некоторой константы~$C_\alpha$
$$
\Variance R_2\simeq\begin{cases}
       C_\alpha \cdot 2^{J}, & \mbox{если}\  \alpha>\fr{1}{2}\,; \\
       C_\alpha J\cdot 2^{J}, & \mbox{если}\  \alpha=\fr{1}{2}\,; \\
       C_\alpha \cdot 2^{2J(1-\alpha)}, & \mbox{если}\  \alpha<\fr{1}{2}\,.
\end{cases}
$$
Вследствие неравенства Чебышёва имеем при любом $\delta\hm>0$
\begin{equation}
\Prob \left(\fr{|R_2|}{2^{bJ} } \geqslant \delta \right)
\leqslant\fr{\Variance\widehat{R}_J(f)}{2^{2bJ}\delta^2}\,.
\label{Tchebyshev}
\end{equation}
Для $1>\alpha\geqslant1/2$ правая часть~(\ref{Tchebyshev}) стремится к нулю при 
$b\hm>1/2$. Для $1/2\hm>\alpha\hm>0$ правая часть~(\ref{Tchebyshev}) стремится к нулю 
при $b\hm>1-\alpha$.

{\small\frenchspacing
{%\baselineskip=10.8pt
\addcontentsline{toc}{section}{References}
\begin{thebibliography}{99}


\bibitem{2-she}
\Au{Donoho D., Johnstone I.\,M.} Ideal spatial adaptation via
wavelet shrinkage~// Biometrika, 1994. Vol.~81. No.\,3.
P.~425--455.

\bibitem{1-she}
\Au{Donoho~D., Johnstone I.\,M.} Adapting to unknown
smoothness via wavelet shrinkage~// J.~Amer. Stat. Assoc., 1995.
Vol.~90. P.~1200--1224.

\bibitem{3-she}
\Au{Donoho D.\,L., Johnstone I.\,M., Kerkyacharian~G., Picard~D.} 
Wavelet shrinkage: Asymp\-to\-pia?~// J.~R. Statist. Soc. Ser.
B., 1995. Vol.~57. No.\,2. P.~301--369.

\bibitem{4-she}
\Au{Marron J.\,S., Adak~S., Johnstone~I.\,M., Neumann~M.\,H.,
Patil~P.} Exact risk analysis of wavelet regression~// J.~Comput.
Graph. Stat., 1998. Vol.~7. P.~278--309.

\bibitem{5-she}
\Au{Antoniadis A., Fan J.} Regularization of wavelet
approximations~// J.~Amer. Statist. Assoc., 2001. Vol.~96. No.\,455. P.~939--967.

\bibitem{6-she}
\Au{Маркин А.\,В.} Предельное распределение оценки риска при
пороговой обработке вейв\-лет-ко\-эф\-фи\-ци\-ен\-тов~// Информатика и её
применения, 2009. Т.~3. Вып.~4. С.~57--63.

\bibitem{7-she}
\Au{Маркин А.\,В., Шестаков О.\,В.} О~состоятельности оценки
риска при пороговой обработке вейв\-лет-ко\-эф\-фи\-ци\-ен\-тов~// Вестн. Моск.
ун-та. Сер.~15. Вычисл. матем. и киберн., 2010. №\,1. C.~26--34.

\bibitem{8-she}
\Au{Шестаков О.\,В.} Аппроксимация распределения оценки риска
пороговой обработки вейв\-лет-ко\-эф\-фи\-ци\-ен\-тов нормальным распределением
при использовании выборочной дисперсии // Информатика и её
применения, 2010. Т.~4. Вып.~4. С.~73--81.

\bibitem{9-she}
\Au{Шестаков О.\,В.} О~точности приближения распределения оценки риска пороговой обработки 
вейв\-лет-ко\-эф\-фи\-ци\-ен\-тов
сигнала нормальным законом при неизвестном уровне шума~// 
Системы и средства информатики, 2012. Т.~22. №\,1. С.~142--152.

\bibitem{10-she}
\Au{Шестаков О.\,В.}  Асимптотическая нормальность оценки риска пороговой обработки 
вейв\-лет-ко\-эф\-фи\-ци\-ен\-тов при выборе
адаптивного порога~// Доклады РАН, 2012. Т.~445. №\,5. С.~513--515.

\bibitem{11-she}
\Au{Шестаков О.\,В.} Зависимость предельного распределения
оценки риска пороговой обработки вейв\-лет-ко\-эф\-фи\-ци\-ен\-тов сигнала от
вида оценки дисперсии шума при выборе адаптивного порога~// T-Comm~--- 
Телекоммуникации и транспорт, 2012. №\,1. С.~46--51.

\bibitem{12-she}
\Au{Шестаков О.\,В.} Центральная предельная теорема для
функции обобщенной кросс-ва\-ли\-да\-ции при пороговой обработке
вейв\-лет-ко\-эф\-фи\-ци\-ен\-тов~// Информатика и её применения, 2013. Т.~7.
Вып.~2. С.~40--49.

\bibitem{13-she}
\Au{Добеши И.} Десять лекций по вейвлетам.~--- 
Ижевск: НИЦ Регулярная и хаотическая динамика, 2001. 357~с.

\bibitem{14-she}
\Au{Mallat S.} A~wavelet tour of signal processing.~--- Academic Press, 1999.
851~с.

\bibitem{15-she}
\Au{Boggess A., Narkowich~F.} A~first course in wavelets with Fourier analysis.~--- 
Prentice Hall, 2001. 283~с.

\bibitem{16-she}
\Au{Johnstone I.\,M., Silverman B.\,W.} Wavelet threshold
estimates for data with correlated noise~// J.~Roy. Statist. Soc.
Ser.~B, 1997. Vol.~59. P.~319--351.

\bibitem{17-she}
\Au{Johnstone I.\,M.} Wavelet shrinkage for correlated data and inverse problems: Adaptivity
results~// Statistica Sinica, 1999. Vol.~9. No.\,1. P.~51--83.

\bibitem{18-she}
\Au{Taqqu M.\,S.} Weak convergence to fractional Brownian motion and to the 
Rosenblatt process~// Z. Wahrscheinlichkeitsth. verw. Geb., 1975. Vol.~31. P.~287--302.

\bibitem{19-she}
\Au{Bradley R.\,C.} Basic properties of strong mixing conditions. A~survey 
and some open questions~// Probab. Surveys, 2005. Vol.~2. P.~107--144.

\bibitem{20-she}
\Au{Peligrad M.} On the asymptotic normality of sequences of weak dependent random variables~// 
J.~Theoret. Prob., 1996. Vol.~9. No.\,3. P.~703--715.
\end{thebibliography}
} }

\end{multicols}

\hfill{\small\textit{Поступила в редакцию 14.08.13}}


\vspace*{12pt}

\hrule

\vspace*{2pt}

\hrule

%\newpage

%\vspace*{-24pt}

\def\tit{ASYMPTOTIC PROPERTIES OF WAVELET THRESHOLDING RISK ESTIMATE IN~THE~MODEL 
OF~DATA WITH~CORRELATED NOISE}

\def\titkol{Asymptotic properties of wavelet thresholding risk estimate in~the~model 
of~data with~correlated noise}

\def\aut{A.\,A.~Eroshenko$^1$ and O.\,V.~Shestakov$^{1,2}$}
\def\autkol{A.\,A.~Eroshenko and O.\,V.~Shestakov}


\titel{\tit}{\aut}{\autkol}{\titkol}

\vspace*{-9pt}

\noindent
$^1$M.\,V.~Lomonosov Moscow State University, Faculty of Computational Mathematics 
and Cibernetics,\\
$\hphantom{^1}$1-52 Leninskiye Gory, GSP-1, Moscow 119991, Russian Federation\\
\noindent
$^2$Institute of Informatics 
Problems, Russian Academy of Sciences, 44-2 Vavilov Str., Moscow 119333, Russian\\
$\hphantom{^1}$Federation
 
\def\leftfootline{\small{\textbf{\thepage}
\hfill INFORMATIKA I EE PRIMENENIYA~--- INFORMATICS AND APPLICATIONS\ \ \ 2014\ \ \ volume~8\ \ \ issue\ 1}
}%
 \def\rightfootline{\small{INFORMATIKA I EE PRIMENENIYA~--- INFORMATICS AND APPLICATIONS\ \ \ 2014\ \ \ volume~8\ \ \ issue\ 1
\hfill \textbf{\thepage}}}   

\vspace*{6pt}
  
\Abste{Wavelet thresholding techniques of denoising are widely used in signal and 
image processing. These methods are easily implemented through fast algorithms; so, 
they are very appealing in practical situations. Besides, they adapt to function 
classes with different amounts of smoothness in different locations more effectively 
than the usual linear methods. Wavelet thresholding risk analysis is an important 
practical task because it allows determining the quality of techniques themselves 
and equipment which is being used. In the present paper, asymptotical 
properties of mean-square risk estimate of wavelet thresholding techniques have been studied
in the 
model of data with correlated noise. The conditions under which the unbiased 
risk estimate is consistent and asymptotically normal are given. These results allow constructing 
asymptotical confidence intervals for wavelet thresholding risk, using only the observed data.}


\KWE{wavelets; unbiased risk estimate; correlated noise; asymptotic normality} 

\DOI{10.14357/19922264140105}

\Ack
\noindent
The financial support of the Russian Foundation for Basic Research is acknowledged
(project 11-01-00515а).

  \begin{multicols}{2}

\renewcommand{\bibname}{\protect\rmfamily References}
%\renewcommand{\bibname}{\large\protect\rm References}

{\small\frenchspacing
{%\baselineskip=10.8pt
\addcontentsline{toc}{section}{References}
\begin{thebibliography}{99}


\bibitem{2-she-1}
\Aue{Donoho, D., and I.\,M. Johnstone.} 
1994. Ideal spatial adaptation via wavelet shrinkage.
\textit{Biometrika}  81(3):425--455.

\bibitem{1-she-1}
\Aue{Donoho, D., and I.\,M. Johnstone.} 
1995. Adapting to unknown smoothness via wavelet shrinkage.
\textit{J.~Amer. Stat. Assoc.} 90:1200--1224.

\bibitem{3-she-1}
\Aue{Donoho, D.\,L., I.\,M.~Johnstone, G.~Kerkyacharian, and D.~Picard.} 
1995. Wavelet shrinkage: Asymp\-to\-pia? \textit{J.~R. Statist. Soc. Ser. B.} 57(2):301--369.

\bibitem{4-she-1}
\Aue{Marron, J.\,S., S.~Adak, I.\,M.~Johnstone, M.\,H.~Neumann, and P.~Patil.} 
1998. Exact risk analysis of wavelet regression. \textit{J.~Comput. Graph. Stat.} 7:278--309.

\bibitem{5-she-1}
\Aue{Antoniadis, A., and J.~Fan.} 
2001. Regularization of wavelet approximations. \textit{J.~Amer. Statist. Assoc.}
96(455):939--967.

\bibitem{6-she-1}
\Aue{Markin, A.\,V.} 
2009. Predel'noe raspredelenie otsenki riska pri porogovoy obrabotke veyvlet-koeffitsientov 
[Limit distribution of risk estimate of wavelet coefficient thresholding].
\textit{Informatika i ee Primeneniya}~---
\textit{Inform. Appl.} 3(4):57--63.

\bibitem{7-she-1}
\Aue{Markin, A.\,V., and O.\,V.~Shestakov} 
2010. O~sostoyatel'nosti otsenki riska pri porogovoy obrabotke veyvlet-koeffitsientov 
[Consistency of risk estimation
with thresholding of wavelet coefficients]. \textit{Vestn. Mosk. Un-ta. Ser.~15. 
Vychisl. Matem. i Kibern.} [\textit{Herald of Moscow University, Computational Mathematics and 
Cybernetics}] 1:26--34.

\bibitem{8-she-1}
\Aue{Shestakov, O.\,V.} 
2010. Approksimatsiya raspredeleniya otsenki riska porogovoy obrabotki 
veyvlet-koeffitsientov normal'nym raspredeleniem pri ispol'zovanii vyborochnoy dispersii 
[Normal approximation for distribution of risk estimate for wavelet coefficients thresholding 
when using sample variance]. \textit{Informatika i ee Primeneniya}~--- 
\textit{Inform. Appl.} 4(4):73--81.

\bibitem{9-she-1}
\Aue{Shestakov,~O.\,V.} 
2012. O~tochnosti priblizheniya\linebreak raspredeleniya otsenki riska porogovoy 
obrabotki veyvlet-koeffitsientov
signala normal'nym zakonom pri neizvestnom urovne shuma 
[On the accuracy of normal approximation for risk estimate distribution when 
thresholding signal wavelet coefficients in case of unknown noise level]. 
\textit{Sistemy i Sredstva Informatiki}~--- \textit{Systems and Means of Informatics}
22(1):142--152.

\bibitem{10-she-1}
\Aue{Shestakov, O.\,V.}  
2012. Asimptoticheskaya normal'nost' otsenki riska porogovoy obrabotki veyvlet-koeffitsientov pri vybore
adaptivnogo poroga [Asymptotic normality of adaptive wavelet thresholding risk estimation].
\textit{Doklady RAN} [\textit{Doklady Mathematics}] 445(5):513--515.

\bibitem{11-she-1}
\Aue{Shestakov, O.\,V.} 
2012. Zavisimost' predel'nogo ras\-pre\-de\-leniya otsenki riska porogovoy obrabotki 
veyvlet-koef\-fit\-si\-entov signala
ot vida otsenki dispersii shuma pri vybore adaptivnogo poroga 
[The dependence of the limiting distribution of the risk assessment thresholding 
wavelet coefficients of the signal on the type of noise variance estimation 
when selecting an adaptive threshold]. \textit{T-Comm~--- Telekommunikacii i Transport} 
[\textit{T-Comm~--- Telecommunications and Transport}] 1:46--51.

\bibitem{12-she-1}
\Aue{Shestakov, O.\,V.} 
2013. Tsentral'naya predel'naya teorema dlya funktsii obobshchennoy 
kross-validatsii pri po\-ro\-go\-voy obrabotke veyvlet-koeffitsientov 
[Central limit theorem for generalized cross-validation function in wavelet 
thresholding method]. \textit{Informatika i ee Primeneniya}~--- 
\textit{Inform. Appl.} 7(2):40--49.

\bibitem{13-she-1}
\Aue{Daubechies, I.} 1992. \textit{Ten lectures on wavelets}. SIAM. 357~p.

\bibitem{14-she-1}
\Aue{Mallat, S.} 1999. \textit{A~wavelet tour of signal processing}. Academic Press.
851~p.

\bibitem{15-she-1}
\Aue{Boggess, A., and F.~Narkowich.} 
2001. A~first course in wavelets with Fourier analysis. Prentice Hall. 283~p.

\bibitem{16-she-1}
\Aue{Johnstone, I.\,M., and B.\,W.~Silverman.} 
1997. Wavelet threshold estimates for data with correlated noise.
\textit{J.~Roy. Statist. Soc. Ser. B.} 59:319--351.

\bibitem{17-she-1}
\Aue{Johnstone, I.\,M.} 
1999. Wavelet shrinkage for correlated data and inverse problems: Adaptivity
results. \textit{Statistica Sinica} 9(1):51--83.

\bibitem{18-she-1}
\Aue{Taqqu, M.\,S.} 
1975. Weak convergence to fractional Brownian motion and to the Rosenblatt process.
\textit{Z.~Wahrscheinlichkeitsth. verw. Geb.} 31:287--302.

\bibitem{19-she-1}
\Aue{Bradley, R.\,C.} 
2005. Basic properties of strong mixing conditions. A~survey and some open questions.
\textit{Probab. Surveys} 2:107--144.

\bibitem{20-she-1}
\Aue{Peligrad, M.} 
1996. On the asymptotic normality of sequences of weak dependent random variables.
\textit{J.~Theoret. Prob.} 9(3):703--715.
\end{thebibliography}
} }


\end{multicols}

\vspace*{-6pt}

\hfill{\small\textit{Received August 14, 2013}}

\vspace*{-18pt}

\Contr

\noindent
\textbf{Eroshenko Alexander A.} (b.\ 1989)~--- PhD student, Department of Mathematical Statistics, Faculty of Computational
Mathematics and Cybernetics, M.\,V.~Lomonosov
Moscow State University, 1-52 Leninskiye Gory, GSP-1, Moscow 119991, Russian Federation; 
aeroshik@gmail.com 

\vspace*{2pt}

\noindent
\textbf{Shestakov Oleg V.} (b.\ 1976)~--- Doctor of Science in physics and 
mathematics, assistant professor, Department of Mathematical Statistics, Faculty of Computational
Mathematics and Cybernetics, M.\,V.~Lomonosov
Moscow State University, 
1-52 Leninskiye Gory, GSP-1, Moscow 119991, Russian Federation; senior scientist, Institute of Informatics Problems, Russian Academy of
Sciences, 44-2 Vavilov Str., Moscow 119333, Russian Federation; oshestakov@cs.msu.su



 \label{end\stat}
 
\renewcommand{\bibname}{\protect\rm Литература}  
  