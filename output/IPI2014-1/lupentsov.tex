\def\stat{lupen}

\def\tit{РАЗРАБОТКА МОДЕЛИ УПРАВЛЕНИЯ ПРОЦЕССОМ ОБУЧЕНИЯ С~ИСПОЛЬЗОВАНИЕМ КОГНИТИВНЫХ 
ТЕХНОЛОГИЙ}

\def\titkol{Разработка модели управления процессом обучения с использованием когнитивных 
технологий}

\def\autkol{В.\,А.~Маренко, О.\,Н.~Лучко, О.\,С.~Лупенцов}

\def\aut{В.\,А.~Маренко$^1$, О.\,Н.~Лучко$^2$, О.\,С.~Лупенцов$^3$}

\titel{\tit}{\aut}{\autkol}{\titkol}

%{\renewcommand{\thefootnote}{\fnsymbol{footnote}} \footnotetext[1]{Работа 
%выполнена при финансовой поддержке РФФИ (проект 11-01-00515а).}}

\renewcommand{\thefootnote}{\arabic{footnote}}
\footnotetext[1]{Институт математики им.\ С.\,Л.~Соболева Сибирского отделения 
Российской академии наук, marenko@ofim.oscsbras.ru} 
\footnotetext[2]{Омский государственный институт сервиса, o\_luchko@rambler.ru} 
\footnotetext[3]{Омский государственный институт сервиса, lupentsov@mail.ru}
  


\Abst{Приведена когнитивная карта <<Процесс обучения>> в виде 
ориентированного графа. На дуги ориентированного графа нанесены согласованные 
экспертные оценки. Объекты на когнитивной карте делятся на целевой фактор и 
управляющие факторы. Целевым фактором является качество обучения. Управляющие 
факторы используются для корректировки образовательного процесса. Приведена 
информация, которая используется при построении модели управляющего фактора 
<<когнитивная готовность студента>>. Формализация экспериментальных данных 
осуществлена с применением метода семантического дифференциала и аппарата нечетких 
множеств. Приведена когнитивная модель образовательного процесса в виде 
функционального ориентированного графа. Показаны результаты имитационного 
эксперимента <<управление образовательным процессом>>. Показано, что 
при увеличении управляющего фактора <<уровень стабильности внешней среды>> 
увеличивается и целевой фактор <<качество обучения>>.}
     
     \KW{когнитивная карта; когнитивная модель; процесс обучения; имитационный 
эксперимент; нечеткое множество; семантический дифференциал}

\DOI{10.14357/19922264140110}

\vskip 14pt plus 9pt minus 6pt

      \thispagestyle{headings}

      \begin{multicols}{2}

            \label{st\stat}     
  
  
\section{Введение}

    Современное общество нуждается в компетентных и 
конкурентоспособных специалистах, которые должны сочетать в себе как 
профессиональные\linebreak знания, так и личностные качества, позволяющие\linebreak в 
короткие сроки решать профессиональные задачи. Реализация таких 
потребностей требует при-\linebreak менения современных подходов в управлении 
об\-разованием, например таких, как когнитивные\linebreak технологии. Их используют 
в управленческой деятельности для проведения оперативного анализа 
устой\-чи\-вости со\-ци\-аль\-но-эко\-но\-ми\-че\-ских процессов. Методологию 
когнитивного подхода успешно применяют сотрудники ИПУ РАН 
А.\,А.~Кулинич, Е.\,К.~Корноушенко, С.\,В.~Качаев и~др.~[1]. 
    
    Управлению в сфере образования посвящены работы многих ученых. 
Среди них Д.\,А.~Новиков, М.\,М.~Поташник, В.\,М.~Филиппов, которые 
рассматривали развитие образовательных систем различных уровней в целом 
с точки зрения общекибернетического подхода~[2]. 
    
    В данной работе решается частная задача: повышение эффективности 
процесса обучения в \mbox{Омском} государственном институте сервиса с 
использованием когнитивной методологии. Применение когнитивной 
методологии дает возможность с помощью активизации интеллекта 
объективизировать знания экспертов о процессах, формализовать изучаемую 
социально-экономическую проб\-ле\-му и провести ее исследование с помощью 
имитационного эксперимента. 
    
\section{Разработка когнитивной модели управления}
    
    Процесс разработки когнитивной модели состоит из последовательности 
взаимосвязанных шагов: проведения PEST (Political, Economic, Social, and Technological)
ана\-ли\-за, построения 
когнитивной карты, формирования когнитивной модели в виде 
функционального графа для проведения имитационного эксперимента. 
    
    PEST-анализ помогает исследователю увидеть картину внешней среды, 
выделить наиболее важные влияющие факторы. Внешняя среда включает 
экономическую, политическую, правовую, социальную и технологическую 
составляющие, которые оказывают различное по степени, характеру и 
периодичности влияние на исследуемый процесс. 

     \begin{table*}\small
     \begin{center}
     \begin{tabular}{|c|l|l|c|}
\multicolumn{4}{c}{Базисные факторы}\\[6pt]
\hline
№&\multicolumn{1}{c|}{Фактор}&\multicolumn{1}{c|}{Значения} &Интервал измерений\\
\hline
\multicolumn{4}{|c|}{\textbf{Целевой фактор}}\\
\hline
1&Качество образовательного процесса&\tabcolsep=0pt\begin{tabular}{l}Высокое, низкое,\\ 
не очень высокое\ldots\end{tabular}&[0, 1]\\
\hline
\multicolumn{4}{|c|}{\textbf{Управляющие факторы}}\\
\hline
2&Содержание обучения&\tabcolsep=0pt\begin{tabular}{l}Сложное, несложное,\\ очень сложное, 
не очень сложное\ldots\end{tabular}&[0, 1]\\
\hline
3&Квалификация 
преподавателей&Высокая, не очень высокая\ldots&[0, 1]\\
\hline
4&Когнитивная готовность студента&Высокая, низкая\ldots&[0, 1]\\
\hline
5&Мотивация&Сильная, умеренная, слабая\ldots&[0, 1]\\
\hline
6&Потребность общества в специалистах&Высокая, низкая\ldots&[0, 1]\\
\hline
7&Уровень стабильности внешней среды&Высокий,  невысокий\ldots&[0, 1]\\
\hline
\end{tabular}
\end{center}
\end{table*}

    
    В работах ученых Б.\,П.~Мартиросяна, О.\,А.~Уткина, С.\,В.~Ивановой 
и~др.\ учитывается воздействие внешней среды как источника ресурсов для 
образовательного учреждения~[3]. В~предлагаемой статье воздействие 
внешней среды в рамках PEST-ана\-ли\-за оценивается экспертами степенью 
стабильности. 


    
     \subsection{Построение когнитивной карты}
     
Авторы~[4] считают, что 
когнитивная карта~--- это ориентированный взвешенный граф, в котором 
вершины взаимно однозначно соответствуют базисным факторам. 
Взаимосвязи между базисными факторами определяются путем 
рассмотрения при\-чин\-но-след\-ст\-вен\-ных цепочек, описывающих 
распространение влияний одного из них на другие. Влияние вершин может 
быть положительным, отрицательным или нулевым~[4]. 
     
     По мере накопления знаний в виде текстов, таб\-лиц, математических 
структур становится возможным более детально раскрыть как характер 
связей между базисными факторами, так и содержание самих базисных 
факторов.   
     
     В данном исследовании когнитивная карта в виде ориентированного 
графа $G\hm= \langle V, E\rangle$, где $V$~--- множество вершин, 
$V_i\hm\in V$, $i \hm= 1, 2,\ldots , k$; $E$~--- множество дуг, $e_{ij}\hm\in E$, $i, j 
\hm= 1, 2, \ldots , n$, состоит из\linebreak
     
\vspace*{6pt}

\noindent
\begin{center}  %fig1
 \mbox{%
 \epsfxsize=77.519mm
 \epsfbox{lup-1.eps}
 }
  \end{center}
%  \vspace*{6pt}
\begin{center}
{{\figurename~1}\ \ \small{Фрагмент когнитивной карты}}
\end{center}

%\vspace*{12pt}

      

\addtocounter{figure}{1}

\noindent
 базисных факторов, наиболее значимых для 
управ\-ле\-ния образовательным процессом (см.\ таблицу). 


     Между базисными факторами устанавливаются при\-чин\-но-след\-ст\-вен\-ные связи. 
     На рис.~1 приведен фрагмент когнитивной карты с 
оценками специалистов в исследуемой предметной области.
     

     
     Совокупность базисных факторов, влияющих на состояние 
исследуемой системы, можно разделить на группу управляющих факторов и 
группу целевых факторов. Для их количественного описания использовались 
лингвистические переменные и данные, полученные в результате 
экспериментов.


     
     Для установления при\-чин\-но-след\-ст\-вен\-ных связей между 
управляющими факторами и целевым фактором привлечено несколько 
экспертов. Их оценки согласовывались с использованием средств 
математической статистики. 
     
     \subsection{Имитационный эксперимент}
     
     Следующий шаг исследования~--- построение когнитивной модели для 
проведения имитационного эксперимента. Когнитивная модель $\Phi\hm= 
(G,X,F)$, где $G\hm=\langle V, E\rangle$~--- ориентированный граф; $X$~--- 
множество параметров вершин~$V$, $X\hm=\left\{x^{(v_i)}\right\}$, $i \hm= 
1, 2, \ldots , k$; $x^{(v_i)} \hm=\left\{ x^{(i)}_g\right\}$, $g\hm = 1, 2, \ldots , n$; 
$x^{(i)}_g$~--- параметр вершины~$V_i$, \mbox{если} $g\hm = 1$, то $x^{(i)}_g\hm = 
x_i$; $X:\ V\hm\to R$, $R$~--- множество вещественных чисел; $F\hm=F(X, 
E)\hm=F(x_i, x_j, e_{ij})$~--- функционал преобразования\linebreak дуг, ставящий в 
соответ\-ствие каждой дуге знак, весовой коэффициент~$\omega_{ij}$ или 
функцию $f(x_i, x_j, e_{ij}) \hm= f_{ij}$~[5]. На рис.~2 представлена упрощенная 
когнитивная модель, которая задается матрицей смежности. Элементы 
матрицы~--- экспертные оценки.
     
\begin{figure*} %fig2
\vspace*{1pt}
 \begin{center}
 \mbox{%
 \epsfxsize=150mm
 \epsfbox{lup-2.eps}
 }
 \end{center}
 \vspace*{-9pt}
\Caption{Упрощенная когнитивная модель}
\end{figure*}
\begin{figure*}[b] %fig3
\vspace*{3pt}
 \begin{center}
 \mbox{%
 \epsfxsize=163.112mm
 \epsfbox{lup-3.eps}
 }
 \end{center}
 \vspace*{-9pt}
\Caption{Визуализация расчетов}
\end{figure*}

     На следующем этапе исследовался процесс распространения 
возмущений на графе. 
     
     Параметры когнитивной модели $x_i(t)$, $t\hm= 1, \ldots , n$, зависят от 
времени. Если в момент времени $t\hm- 1$ в вершину поступал импульс $p_j
     \hm\in P$, то переход системы из состояния $t\hm-1$ в~$t$ 
осуществлялся по правилу 
     $$
     x_i(t) =x_i(t-1) +\sum\limits_{j=1}^{k-1} f(x_i, x_j, e_{ij}) p_j(t-1)
     $$
     при известных начальных значениях~[5].
     
     Возмущение поступало в одну из вершин графа и актуализировало всю 
систему показателей в большей или меньшей степени. Если между двумя 
управляющими факторами связь имела величину, например, 0,5 и значение 
одного управляющего фактора увеличивалось на 10\%, то величина другого 
управляющего фактора возрастала на 5\%. 

Кривые на рис.~3 визуализируют 
изменения, происходящие на графе. Уменьшение значений управляющего 
фактора <<степень стабильности внешней среды>>~(\textit{2}) приводит к 
уменьшению значений целевого фактора <<качество образовательного 
процесса>>~(\textit{1}). На рис.~3 показано изменение значений целевого фактора 
до~(\textit{а}) и после~(\textit{б}) уменьшения значений управ\-ля\-юще\-го 
фактора <<степень стабильности внешней среды>>. 
     
\setcounter{figure}{4}
\begin{figure*}[b] %fig5
\vspace*{1pt}
 \begin{center}
 \mbox{%
 \epsfxsize=101.229mm
 \epsfbox{lup-5.eps}
 }
 \end{center}
 \vspace*{-9pt}
\Caption{Индивидуальные семантические профили }
\end{figure*}


     В настоящее время реализован простейший вариант имитационного 
эксперимента, позволя\-ющий наблюдать различные состояния 
образовательного процесса по значениям целевого фактора.
     
     \subsection{Модель <<Когнитивная готовность студента>> как 
управляющий фактор}
     
     Авторами разработана модель <<Когнитивная готовность студента>>, 
которая служит управляющим фактором. Модель имеет три составляющие: 
интеллектуальную, рефлексивную и информационную. Эти составляющие 
являются личностными характеристиками студентов.



     
     \textit{Интеллектуальная характеристика} ($A_z$) оценивается 
баллами по результату интернет-теста. Данные ин\-тер\-нет-тес\-та для 
группы студентов исполь-\linebreak

\vspace*{12pt}

\noindent
\begin{center}  %fig4
 \mbox{%
 \epsfxsize=78.633mm
 \epsfbox{lup-4.eps}
 }
  \end{center}
%  \vspace*{6pt}
\begin{center}
{{\figurename~4}\ \ \small{Нечеткое множество <<интеллектуальная характеристика>>}}
\end{center}

%\vspace*{12pt}

      


\noindent
зовались для формирования нечеткого множества 
<<интеллектуальная характеристика>>, которое характеризует степень 
выраженности исследуемой характеристики у субъекта образовательного 
процесса (рис.~4).





     \textit{Рефлексивная характеристика} ($A_x$) помогает студенту 
осознать содержание обучения за счет размышления над собственным 
познавательным процессом. Для выявления численных значений этой 
характеристики в рамках компетентностного подхода применялось 
анкетирование. Вопросы анкеты составлены в соответствии с названиями 
компетенций по направлению подготовки 230700.68 <<Прикладная 
информатика>> (всего 22~компетенции). Задача опрашиваемого состоит в 
том, чтобы зафиксировать свою оценку степени владения компетенцией на 
шкале семантического дифференциала какой-либо меткой~[6]. В~результате 
возникает субъективный семантический профиль~--- ломаная линия, 
соединяющая все метки, поставленные испытуемым. Ось абсцисс~--- номера 
компетенций, отражающих общепрофессиональные умения и навыки. Ось 
ординат~--- степень владения компетенцией (рис.~5).



     Для уточнения рефлексивного аспекта, состоящего в том, что субъект 
размышляет над своей образовательной деятельностью, анкетирование для 
каждой личности проводилось в трех вариантах в соответствии с моделью 
В.\,А.~Лефевра~[7]. Сначала каждый испытуемый осуществлял самооценку 
знаний по соответствующей компетенции в баллах. Затем выставлялась 
балльная оценка знаний испытуемому другим субъектом. А~потом также в 
баллах оценивал себя сам испытуемый, став на позицию стороннего 
наблюдателя. Из полученных
\begin{figure*}
\vspace*{1pt}
\begin{center}  %fig6+7
 \mbox{%
 \epsfxsize=157.636mm
 \epsfbox{lup-6.eps}
 }
  \end{center}
  \vspace*{-12pt}
  \begin{minipage}[t]{80mm}
\Caption{Нечеткое множество <<рефлексивная характеристика>>}
\end{minipage}
\hfill
\begin{minipage}[t]{80mm}
\Caption{Нечеткая модель <<Когнитивная готовность студента>>}
\end{minipage}
\end{figure*}
 трех чисел выбиралась медиана. Интегральный 
показатель~--- сумма баллов по всем компетенциям. По результатам 
анкетирования группы студентов построено нечеткое множество 
<<рефлексивная характеристика>> (рис.~6). 


     \textit{Информационная характеристика} ($A_y$) выявляет 
совокупность способностей студента рационально добывать информацию, 
превращая ее в знания и компетенции, умение овладевать новыми 
технологиями переработки информации. Численное\linebreak значение 
информационной характеристики устанавливается по каждому субъекту 
образовательного процесса в баллах экспертами\,--\,чле\-на\-ми методического 
совета кафедры. Интегральный показатель~--- сумма баллов по всем 
компетенциям. По результатам анкетирования группы студентов 
строится также  нечеткое множество <<информационная характеристика>>, по 
которому можно определять степень выраженности исследуемой 
характеристики у каж\-до\-го студента.
     
     Три характеристики субъекта процесса обучения составляют 
сформированную авторами статьи нечеткую модель <<Когнитивная 
готовность студента>> (рис.~7). На координатных осях модели 
откладываются степени выраженности исследуемых характеристик 
(интеллектуальной, рефлексивной и информационной) в интервале от~0 
до~1. 



     Таким образом, каждый студент представлен точкой~$A$ в 
трехмерном пространстве. Длина вектора~$OA$~--- численное значение 
управляющего фактора <<когнитивная готовность студента>>. 
В~перспективе планируется расширение разработанной нечеткой модели 
<<Когнитивная готовность студента>> до $n$-мер\-но\-го уровня для более 
полного учета индивидуальных характеристик личности.


     
     Модель <<Когнитивная готовность студента>> можно применять не 
только в качестве управ\-ля\-юще\-го фактора при когнитивном моделировании, 
но и для классификации студентов в соответствии с их характеристиками. 
Например, классифика-\linebreak ция по интеллектуальной характеристике может 
использоваться для рекомендации технологий\linebreak обуче\-ния с различной 
степенью интерактивности, а также для диагностики пригодности личности к 
определенной профессиональной деятельности.
     
     \section{Заключение}
     
     Проведенное моделирование позволяет прогнозировать качество 
процесса обучения с учетом индивидуальных характеристик студентов, 
квалификации преподавательского состава, содержания учебного 
материала, мотивации и других управляющих факторов. 
     
     Заключительный этап исследований состоит в том, чтобы 
сформировать систему научно обоснованных рекомендаций руководящему 
звену образовательного учреждения для улучшения процесса обучения.

     
{\small\frenchspacing
{%\baselineskip=10.8pt
\addcontentsline{toc}{section}{References}
\begin{thebibliography}{9}
\bibitem{1-ll}
\Au{Максимов В.\,И., Тер-Егиазарова Н.\,В.} IV Междунар. конф. 
<<Когнитивный анализ и управление развитием ситуаций>> CASC'2004~// 
Проблемы управления, 2005. №\,1. С.~83--87.
\bibitem{2-ll}
\Au{Новиков Д.\,А.} Теория управления образовательными системами.~--- М.: 
Народное образование, 2009. 416~с. 
\bibitem{3-ll}
\Au{Иванова С.\,В.} Образование в 
ор\-га\-ни\-за\-ци\-он\-но-гу\-ма\-ни\-сти\-че\-ском измерении.~--- М.: РУДН, 2007. 236~с.
\bibitem{4-ll}
\Au{Максимов В.\,И., Корноушенко Е.\,К., Качаев~С.\,В.} Когнитивные 
технологии для поддержки принятия управ\-лен\-че\-ских решений. {\sf 
http://emag.iis.ru/arc/infosoc/\linebreak emag.nsf/bpa/092aa276c601a997c32568c0003ab839}.
\bibitem{5-ll}
\Au{Горелова Г.\,В., Радченко С.\,А.} Когнитивные технологии поддержки 
управ\-лен\-че\-ских решений в со\-ци\-аль\-но-эко\-но\-ми\-че\-ских 
сис\-те\-мах~// Известия Южного федерального университета. Технические 
науки, 2003. Т.~34. №\,5. С.~95--104.
\bibitem{6-ll}
\Au{Лупенцов О.\,С., Лучко О.\,Н., Маренко~В.\,А.} Применение 
семантического дифференциала для реализации компетентностного подхода 
в вузе~// Информатизация образования и науки, 2012. №\,3(15). С.~128--134.
\bibitem{7-ll}
\Au{Лефевр В.\,А.} Формула человека: Контуры фундаментальной 
психологии~/ Пер. с англ.~--- М.: Прогресс, 1991. 108~с.
\end{thebibliography}
} }

\end{multicols}

\vspace*{-12pt}

\hfill{\small\textit{Поступила в редакцию 21.02.13}}


\vspace*{12pt}

\hrule

\vspace*{2pt}

\hrule




\def\tit{DEVELOPMENT OF~LEARNING PROCESS CONTROL MODEL WITH~COGNITIVE TECHNOLOGIES}

\def\titkol{Development of learning process control model with cognitive technologies}

\def\aut{V.\,A.~Marenko$^1$, O.\,N.~Luchko$^2$, and~O.\,S.~Lupentsov$^2$}
\def\autkol{V.\,A.~Marenko, O.\,N.~Luchko, and~O.\,S.~Lupentsov}


\titel{\tit}{\aut}{\autkol}{\titkol}

\vspace*{-9pt}

\noindent
$^1$Sobolev Institute of
Mathematics, Siberian Branch of the Russian Academy of Sciences,
4 Acad.\ Koptyug Av.,\\
$\hphantom{^1}$Novosibirsk 630090, Russian Federation

\noindent
$^2$Omsk State Institute of Service, 13 Petrov Str., 
Omsk 644099, Russian Federation


 
\def\leftfootline{\small{\textbf{\thepage}
\hfill INFORMATIKA I EE PRIMENENIYA~--- INFORMATICS AND APPLICATIONS\ \ \ 2014\ \ \ volume~8\ \ \ issue\ 1}
}%
 \def\rightfootline{\small{INFORMATIKA I EE PRIMENENIYA~--- INFORMATICS AND APPLICATIONS\ \ \ 2014\ \ \ volume~8\ \ \ issue\ 1
\hfill \textbf{\thepage}}}   

\vspace*{3pt}
  
\Abste{The paper describes the  ``Learning process'' cognitive map in the form of a 
directed graph. Arcs of the directed graph are labeled with the agreed 
expert estimates. Objects on the cognitive map are divided into the target 
factor and controlling factors. The quality of education is the target factor. 
Controlling factors are used to adjust the educational process. The paper provides 
information used to devise the model of the control factor of cognitive readiness of a 
student. Formalization of experimental data is carried out using the semantic 
differential method and fuzzy sets. The cognitive model of the educational 
process, which is a functional directed graph, is presented. The results of 
experiments simulating educational process management 
 are shown. It is also shown that 
while the level of stability of external environment (viewed as a control factor)
increases, the level of the quality of education (viewed as the target factor)
also increases.}


\KWE{control; cognitive model; cognitive map; educational process; simulation experiment; 
semantic differential; fuzzy set}


\DOI{10.14357/19922264140110}

%\Ack
%\noindent


  \begin{multicols}{2}

\renewcommand{\bibname}{\protect\rmfamily References}
%\renewcommand{\bibname}{\large\protect\rm References}

{\small\frenchspacing
{%\baselineskip=10.8pt
\addcontentsline{toc}{section}{References}
\begin{thebibliography}{9}
\bibitem{1-ll-1}
\Aue{Maksimov, V.\,I., and N.\,V.~Ter-Egiazarova}. 2005. 
IV Mezhdunarodnaya konferentsiya ``Kognitivnyy analiz i upravlenie razvitiem situatsiy'' 
CASC'2004 [IV International Conference ``Cognitive Analysis and Management of Development 
of Situations'' CASC'2004].  \textit{Problemy Upravleniya} [\textit{Management Problems}]
1:83--87.

\bibitem{2-ll-1}
\Aue{Novikov, D.\,A.} 2009. \textit{Teoriya upravleniya obrazovatel'nymi 
protsessami} [\textit{Theory of management of educational systems}]. 
Moscow:  National Education.  416~p. 
\bibitem{3-ll-1}
\Aue{Ivanova, S.\,V.} 2007. \textit{Obrazovanie v organizatsionno-gumanisticheskom izmerenii} 
[\textit{Education in organizational and humanistic measurement}]. Moscow: RUDN. 236~p.

\bibitem{4-ll-1}
\Aue{Maksimov, V.\,I., E.\,K. Kornoushenko, and S.\,V.~Kachayev}. 
Kognitivnye tekhnologii dlya podderzhki prinyatiya upravlencheskikh resheniy 
[Cognitive technologies for support of adoption of administrative decisions]. 
Available at: 
{\sf http://emag.iis.ru/arc/infosoc/emag.nsf/bpa/\linebreak 092aa276c601a997c32568c0003ab839} 
(accessed December 26, 2013).

\bibitem{5-ll-1}
\Aue{Gorelova, G.\,V., and S.\,A.~Radchenko}. 
2003. Kognitivnye tekhnologii podderzhki upravlencheskikh 
resheniy v sotsial'no-ekonomicheskikh sistemakh 
[Cognitive technologies of support of administrative decisions in social 
and economic systems]. \textit{Izvestiya Yuzhnogo Federal'nogo Universiteta. 
Tekhnicheskie Nauki} [\textit{News of the Southern Federal University. Technical Sciences}]. 
34(5):95-104.

\bibitem{6-ll-1}
\Aue{Lupentsov, O.\,S., O.\,N.~Luchko, and V.\,A.~Marenko}. 
2012. Primenenie semanticheskogo differentsiala dlya rea\-li\-za\-tsii 
kompetentnostnogo podkhoda v vuze 
[Application of semantic differential for realization of competence-based approach 
in higher education institution]. 
\textit{Informatizatsiya Obrazovaniya i Nauki} [\textit{Science and Education Informatization}] 3(15):128--134.

\bibitem{7-ll-1}
\Aue{Lefevre, V.\,A.} 1982. \textit{The formula of man: An outline of fundamental psychology}. 
Irvine: School of Social Sciences, University of California. 290~p. 

\end{thebibliography}
} }


\end{multicols}

\vspace*{-6pt}

\hfill{\small\textit{Received February 21, 2013}}

%\vspace*{-18pt}

\Contr

\noindent
\textbf{Marenko Valentina A.} (b.\ 1951)~--- Candidate of Science (PhD)
in technology, associate professor, senior scientist, Sobolev Institute of
Mathematics, Siberian Branch of the Russian Academy of Sciences,
4 Acad.\ Koptyug Av., 
Novosibirsk 630090, Russian Federation; marenko@ofim.oscsbras.ru

\vspace*{3pt}

\noindent
\textbf{Luchko Oleg N.} (b.\ 1961)~--- Candidate of Science (PhD) in
education, professor; Head of Department, Omsk State Institute of Service,
13 Petrov Str., 
Omsk 644099, Russian Federation; o\_luchko@rambler.ru

\vspace*{3pt}

\noindent
\textbf{Lupentsov Oleg S.} (b.\ 1986)~--- PhD student, Omsk State Institute of Service,
13 Petrov Str., 
Omsk 644099, Russian Federation; lupentsov@mail.ru




 \label{end\stat}
 
\renewcommand{\bibname}{\protect\rm Литература}