
\newcommand{\A}{{\mathcal A}}

\renewcommand{\bibname}{\protect\rmfamily References}
\renewcommand{\figurename}{\protect\bf Figure}

\def\stat{morozov}

\def\tit{STABILITY ANALYSIS OF AN OPTICAL SYSTEM WITH~RANDOM DELAY LINES LENGTHS}

\def\titkol{Stability analysis of an optical system with random delay lines lengths}

\def\autkol{E.~Morozov, L.~Potakhina,  and K.~De Turck}

\def\aut{E.~Morozov$^{1,2}$, L.~Potakhina$^{1,2}$,  and~K.~De Turck$^3$}

\titel{\tit}{\aut}{\autkol}{\titkol}

%{\renewcommand{\thefootnote}{\fnsymbol{footnote}}
%\footnotetext[1] {The work of first and second  authors is partially supported by the
%Program of Strategy development of Petrozavodsk State University in
%the framework of the research activity. The third author is a
%postdoctoral fellow with the Research Foundation-Flanders
%(FWO-Vlaanderen).}}

\renewcommand{\thefootnote}{\arabic{footnote}}
\footnotetext[1]{Institute of Applied Mathematical
Research, Karelian Research Center, Russian Academy of Sciences, 11 Pushkinskaya Str., Petrozavodsk 185910,
Russian Federation}
\footnotetext[2]{Petrozavodsk State University, 33 Lenin Str., Petrozavodsk 185910,
Russian Federation} 
\footnotetext[3]{Ghent University, TELIN Department, 41 Sint-Pietersnieuwstraat, 
Gent B-9000, Belgium} 


\vspace*{-6pt}

\def\leftfootline{\small{\textbf{\thepage}
\hfill INFORMATIKA I EE PRIMENENIYA~--- INFORMATICS AND APPLICATIONS\ \ \ 2014\ \ \ volume~8\ \ \ issue\ 1}
}%
 \def\rightfootline{\small{INFORMATIKA I EE PRIMENENIYA~--- INFORMATICS AND APPLICATIONS\ \ \ 2014\ \ \ volume~8\ \ \ issue\ 1
\hfill \textbf{\thepage}}}





\Abste{A new model of an optical buffer system is considered, in which the
differences $\{\Delta_n\}$ between the lengths of two adjacent fiber
delay lines (FDLs) are \textit{random}. This is an extension of the model
considered in~\cite{Call} where these differences (also referred to
as {\it granularity}) are constant, i.\,e., $\Delta_n\equiv const$. 
The system is modeled by utilizing the random-walk theory and
closely-related asymptotic results of the renewal theory, such as the
inspection paradox  and the Lorden's inequality. 
A~stability analysis is performed based on the regenerative approach.  Some
numerical results  are included  as well, showing that the obtained
conditions delimit the stability region with high accuracy.}

\KWE{optical buffer; stability; stochastic granularity;
renewal theory; regeneration; inspection paradox; simulation}

\DOI{10.14357/19922264140113}


\vskip 12pt plus 9pt minus 6pt

      \thispagestyle{myheadings}

      \begin{multicols}{2}

            \label{st\stat}

\section{Introduction}

\noindent
In networks featuring optical burst switching, data units {referred
to as bursts} travel from source to destination in the form of light
(i.\,e., without conversion to the electrical domain in intermediate
switches). As optical random access memory does not as yet exist, a
set $\cal A$ of FDLs is used as a substitute
form of buffering. Thus, the set of possible waiting times is
denumerable, with each value corresponding to the length of a
fixed delay line belonging to~$\cal A$. As a result, in general even
having an infinite set of (different) lines, arriving bursts have to
wait for service longer than in the classical case with infinite
buffer for awaiting customers. The stability of such a general
optical system with  a denumerable set~$\cal A$ with deterministic
lengths has been analyzed in recent papers~\cite{optical1, optical2}. 
The main idea of the present paper is to consider a
denumerable set of FDLs
 with {\it random lengths}.

Motivations for this extension are manifold. For heavily-loaded
modern large networks, a large number of lines is required.
These lines constitute a huge number of possible paths between hosts
and  users. Moreover, it seems appropriate to describe the
differences between their lengths as random variables to reflect 
variability of the paths. On the other hand,  to meet the high
quality of service (QoS) requirement and avoid  huge  difference in transmission times,
it seems reasonable to assume that the difference $\Delta>0$ between
the {\it adjacent increasing} FDL lengths has the same distribution
regardless of the lines, reflecting  {\it homogeneity} of the
network.
 In particular, homogeneity may be important for  reduction
of reordering in the multipath transmission of a big file by
means of a huge number of  small packets. 

One more argument to
support a common distribution of $\Delta$ is that the modern
networks are very well-connected and contain a huge number of links,
such that  each path is collected from a number of {\it
elementary (primary)} optical cables with {\it comparable} lengths.
Thus, randomness of~$\Delta$ reflects  variability  of FDL
lengths, while  the common distribution of~$\Delta$ (homogeneity)
reflects the mentioned comparability.  Such a setting seems to be
natural for a large fiber lines set (infinite in theory) in a
local network, where knowledge of the exact FDL lengths  is
unavailable (or costly) for the some reason. In this case, it is assumed
that the most important predetermined information is contained in the
distribution of $\Delta$, while other details/info related to  the
set~$\cal A$ are  hardly  available or less  important.

The final argument for the considered model is that sometimes,  it
might be easier to assume some randomness in the differences~$\Delta$ 
rather than to calculate the exact differences.
Moreover, to meet the stability condition, in fact, only knowledge of the first two
moments is required (not of the entire distribution); hence,
 simple sample  estimators can be used.

As it will be shown below, randomness of~$\Delta$ implies  considerable
changes in the technique of analysis in comparison with the model
with the deterministic FDL set~$\cal A$. As the difference
$\Delta=const$ considered in~\cite{Call} is called \textit{granularity}, 
it is natural to refer to the model considered in this
paper as the model with {\it stochastic granularity}.

Further, the term  {\it stochastic inhomogeneity} can be used to
name a system in which the distribution of the  difference~$\Delta$ 
depends on the lines.  This case will not  be further discussed; let only note that
inhomogeneity may lead to extra  loss of
capacity caused by an increase of variance of the void (the idle
time following completion of a transmission).

Note that the assumption   $\Delta>0$ reflecting the
corresponding ordering of  the set of FDL lengths $\cal A$ seems
less motivated for  large distributed networks, where the structure
of currently available different routes is not as well defined.
Nevertheless, the authors believe that such assumption  and, hence, the
approach developed is suitable for some distributed optical
networks. Besides,  it may be useful not only for optical systems
but for some other bufferless communication networks as well.

In the authors' opinion, this model has intrinsic interest from the
mathematical point of view as well.

The structure of the paper is as follows. In section~2, a description
of the optical system with random delay lines  is given. Then, in
Section~3,  stability of this system is analyzed. In particular,
sufficient stability conditions of such a system are presented using
regenerative arguments. The~key ingredient of the analysis is 
application of the so-called inspection paradox from the renewal
theory. In section~4, simulation results are presented which verify
that the  stability conditions found are tight and close to being also
necessary.


\section{Description of~the~Model}

\noindent
Based on the assumptions detailed above, define the length  of
the $n$th line as $S_n:= \Delta_1+\cdots+\Delta_n,\,S_0:=0,$ with
independent and identically distributed (i.i.d.)\ random variables
$\{\Delta_n,\,n\ge 1\}$
 distributed as a generic variable~$\Delta$.
Assume in the following that~$\Delta$ is nonlattice.
 (In a more general case, it
could  be assumed that  $S_n= a_0+ \Delta_1+\cdots+\Delta_n$ where
a constant (or, possibly, even a random variable (r.v.))\ $a_0>0$ is a minimal
given delay. However, further stability analysis is independent of
this assumption.)


Thus, for the $n$th  arriving  burst whose waiting time is $W_n=x$,
generate a path until the random walk $\{S_i\}$ exceeds the level
$x,\,\,n\ge 0$. As in previous works~\cite{optical1, optical2}, this
rule  is motivated by the requirement to keep the \mbox{FIFO} (first in, first out)
discipline. Thus,
 the following  set
 $\mathcal{A} = \{S_0,\, S_1,\,S_2,
\ldots\}$, in which  the  origin~$0$  covers  the case of a
nonwaiting burst is obtained. Define the  number of lines required to
guarantee a delay exceeding the threshold $x\ge 0$ as
\begin{equation*}
%\label{numb} 
N(x) = \inf \{n\ge 0: S_n \ge x\}\,.
\end{equation*}
(Obviously, $\{N(x)\}$  is the renewal process generated by the random
walk $\{S_n\}$.) Then, the actual (random)  delay of the  burst, which
meets the workload~$x$, is defined as
\begin{equation*}
\label{act_del} S_{N(x)}=\inf\{S_n: S_n-x\ge 0\}\,.
%\,\,\,(S_{N(x)}:=0,\,x< 0).
\end{equation*}
Note that $S_{N(0)}=N(0)=S_0=0$. For each $x\ge0$, the
overshoot of the renewal process is defined as
\begin{equation}
\label{eq:ceilx}
S_{N(x)}-x=\Delta(x)\,.
\end{equation}
By definition,  $\Delta(x)\ge0$ for all $x\geq0$ and
$\Delta(S_{k})=0$
 for any $k\ge0$.   In what
follows, it is assumed that
\begin{equation}
0<{\sf E} \Delta^{2+\varepsilon_0} < \infty \label{5}
\end{equation}
for some $\varepsilon_0>0$. This assumption is necessary to  apply the
asymptotic results from the renewal theory detailed below. In
particular, (\ref{5})  implies that {\it for any~$\delta_0$}, there
are (fixed) constants $D<\infty,\,\delta^*>0$, such that
\begin{equation}
 {\sf P}( \delta^* < \Delta \le D )\ge 1- \delta_0\,.
\label{cond2}
\end{equation}
(Obviously, having ${\sf E}\Delta<\infty$ is enough to ensure~(\ref{cond2}).)
Let introduce the following quantities,  which  are crucial for the
 further analysis:
\begin{equation}
\label{cond3} \Delta^* := \fr{\sf E \Delta^2}{{\sf E} \Delta}\,;\quad
\Delta_0 := \fr{{\sf E} \Delta^2}{2{\sf E} \Delta}\,.
\end{equation}
Note that by the  Lorden's inequality \cite{Asmus} the first term in~$(\ref{cond3})$ 
is  the upper bound of the expected overshoot $\Delta(x)$ over 
all  $x \ge 0$, while the second term is the mean stationary  overshoot in the 
random walk $\{S_n\}$ expressing the so-called {\it inspection paradox}. 
Further, assume that the input process is renewal with independent and 
identically distributed (i.i.d.)\ interarrival times $\{T_k\}$ with a generic 
time~$T$. Also assume that service (transmission)  times $\{B_n\}$ are 
i.i.d.\ with the generic element~$B$. Denote $U:=B-T$ and introduce the distribution 
$F_{U}(x)={\sf P}(U\le x)$, $x\in (-\infty,\,+\infty)$.  Obviously, by the 
independence between~$B$ and~$T$, one can express this distribution in the terms 
of given distributions of~$B$ and~$T$ as follows:
$$
F_U(x)=\int\limits_{0^-}^\infty F_B(x+y)F_T(dy)\,.
$$
Let $W_k$ be the remaining workload which the burst~$k$ meets upon 
arrival. In other words, it is the time which is required to
transmit  the  work accumulated in the systems (in the optical
buffer and server) prior to arrival of the  burst. Define  the workload
process $W=\{W_{k},\,k\ge0\}$ and
 construct  regeneration instants  $\{\beta_{n}\}$ for the
 process~$W$ in the following (conventional) way: let $\beta_{0}=0$
and
\begin{equation*}
\beta_{n+1}=\inf(k>\beta_{n}:W_{k}=0)\,,\enskip n\ge0\enskip
(\inf\emptyset:=\infty)\,. 
%\label{eq:recurrence}
\end{equation*}
In what follows, the focus will be on  the  zero-delayed regenerative process
in which case $0$ is the regeneration point, and $\beta:=\beta_1$ is the
generic regeneration period. Note that the  sequence $\{\beta_{n}\}$
is well-defined since $0\in\A$.

\section{Main Stability Result}
\setcounter{section}{3}

\noindent
The main purpose is to establish conditions ensuring the {\it positive 
recurrence} of the workload regenerative process meaning that 
\begin{equation} 
{\sf E} 
\beta<\infty.\label{eq:positiverec}
\end{equation}
The importance of  positive recurrence for the stability analysis
is well-known (see, for instance,~\cite{Asmus, 2}). Let use the
following characterization of the recurrence property of the renewal
process~$\beta$ via the limiting behavior of the forward
regeneration time at the instant~$n$, which is defined as
\begin{equation*} 
\beta(n)=\inf_{k}(\beta_{k}-n:\beta_{k}-n>0)\,,\enskip
n\ge1\,.
\end{equation*}
 It is known~\cite{Feller}
  that
 \begin{multline}
 {\sf E} \beta=\infty\ \ \mbox{if and only if}\ \
 \beta(n)\to\infty\ \ \mbox{in probability} \\
\mbox{as}\ \  n\to\infty\,.\label{eq:Feller}
\end{multline}
Thus, to establish~(\ref{eq:positiverec}), it suffices to show that
$\beta(n)\not \to\infty$ (in probability), which is exactly the approach 
used below. Now,  the main stability result can be formulated.

\medskip

\noindent
\textbf{Theorem 1.} \textit{Assume that $\Delta$ is nonlattice, ${\sf E} B <
\infty$, ${\sf E} T < \infty$ and  that condition~$(\ref{5})$  holds. 
Furthermore, assume  that the following negative drift assumption
\begin{equation}
\label{eq:condition2} 
{\sf E}U+\Delta_0  < 0
\end{equation}
and the regeneration assumption
\begin{equation}
\label{eq:condition3} 
{\sf P}(T > \Delta + B ) = \delta_1 > 0
\end{equation}
 hold. Then  ${\sf E}
\beta<\infty$.}

\medskip

P\,r\,o\,o\,f\,.\ \  Put $N(x):=0$ for $x\le 0$ and define the increments of
the workload process  as
\begin{equation*}
\Delta_{W}(k) = S_{N((W_k+U_k))}-W_k\,, \enskip k\ge 0\,.
\end{equation*}
Note  that $S_{N((W_k+U_k))}$ is the fiber line length which
is selected for the $(k+1)$th burst meeting the workload
$(W_k+B_k-T_k)^+:=(W_k+U_k)^+$, the time when  the $k$th burst brings the
transmission (service) equals~$B_k$ and the interarrival time
between the bursts~$k$ and $k+1$ (where the workload decreases)
equals~$T_k$. Thus, $\Delta_{W}(k)$ is the overshoot of the random
walk (forming line lengths) across the threshold $W_k+U_k$. Then
 as in~\cite{optical1}, using
independence between $U_{k}$ and $W_{k}$, one can write:
\begin{equation*}
{\sf E}\Delta_{W}(k)=
 \int\limits_{y\in R_{+}} {\sf E}(S_{N(y+U)}-y){\sf P}(W_{k} \in
 dy)\,,\ k\ge1\,.
% \label{2.6}
\end{equation*}
If $U\le-y$, then ${\sf E}(S_{N(y+U)}-y)=-y.$ Thus, as in~\cite{optical1}, one 
has for each  fixed~$y$:
\begin{multline}
{\sf E}(S_{N(y+U)}-y)= - y{\sf P}(U\le-y) +{}\\
{}+\int\limits_{z\ge-y} {\sf E} \Delta 
(y+z)F_{U}(dz) +\int\limits_{z\ge-y}zF_{U}(dz)\,. 
\label{eq:3terms}
\end{multline}
Let consider the three terms in the right-hand side of Eq.~(\ref{eq:3terms}) one by 
one. Since ${\sf E} U>-\infty$ , one has
\begin{equation}
0>-y{\sf P}(U\le-y) \ge\!\!
\int\limits_{-\infty}^{-y}\! \! xF_{U}(dx)\uparrow0\,\,\mbox{as}\,\,
y\to\infty.\!\!
\label{eq:PUtozero}
\end{equation}
For the third term, %of (\ref{eq:3terms})
it holds that (since $y\ge0$ and ${\sf E} U<\infty$)
 \begin{equation}
\int\limits_{z\ge-y}zF_{U}(dz)\downarrow{\sf E} U\,\,\mbox{as}\,\,
y\to\infty\,.\label{eq:EUlower}
\end{equation}
Finally, for the second term, one has
\begin{multline*}
\int\limits_{-y}^{\infty}{\sf E}\Delta(y+z)F_{U}(dz)  \\
{}=  \int\limits_{-y}^{-y/2}{\sf 
E}\Delta(y+z)F_{U}(dz)
 +\int\limits_{-y/2}^{\infty}{\sf E}\Delta(y+z)F_{U}(dz) \\
{} \le  \sup\limits_{0\le x\le y/2}{\sf E}\Delta(x){\sf P}\left(-y<U\le-\fr{y}{2}\right)\\
{}+ \sup\limits_{x\ge y/2}{\sf E}\Delta(x){\sf P}\left(U>-\fr{y}{2}\right)
\le \Delta^{*}{\sf P}\left(U\le-\fr{y}{2}\right)\\
{}+ \sup\limits_{x\ge y/2}{\sf E}\Delta(x){\sf 
P}\left(U>-\fr{y}{2}\right)\,.
\end{multline*}
 Fix now an arbitrary $\varepsilon>0$ and choose  such $y_{0}$ that
 for $y\ge y_{0}$,
\begin{align}
{\sf P}\left(U\le-\fr{y}{2}\right) & \le \fr{\varepsilon}{4\Delta^{*}}\,;\label{eq:y0A}\\
\int\limits_{z\ge-y}zF_{U}(dz) & \le  {\sf E} U+\fr{\varepsilon}{4}\,.\label{eq:y0B}
\end{align}
Such a choice  of $y_{0}$ is possible: indeed, for~(\ref{eq:y0A}), due 
to~(\ref{eq:PUtozero}); for~(\ref{eq:y0B}), due to~(\ref{eq:EUlower}). 
As~$\Delta$ is nonlattice and~(\ref{5}) holds, then $\Delta(x)\Rightarrow 
\Delta(\infty)$ in distribution, where ${\sf E}\Delta(\infty)=\Delta_0$. 
Similarly,  the renewal interval $L(x)$ covering the point $x$  converges in 
distribution, as $x\to \infty$, to a variable~$L$  with the mean~${\sf E} 
L=\Delta^*$~\cite{Feller}.
 By the modified  form of the  Lorden's inequality (see, for instance,~\cite{Chang}),
\begin{equation*}
\sup_x{\sf E}\left[\Delta(x)\right]^{1+\varepsilon_0}\le (3+\varepsilon_0){\sf E}
L^{1+\varepsilon_0}\,. %\label{20}
\end{equation*}
It is well-known~\cite{Feller}  that the density  of~$L$ is

\noindent
$$
{\sf P}(L\in dx)=\fr{1}{{\sf E} \Delta}\,xF_\Delta(dx)
$$
where $F_\Delta$ is the distribution of~$\Delta$  and, hence, ${\sf E}
L^{1+\varepsilon_0}<\infty$  if and only if  condition~(\ref{5}) holds. Thus,  
$\sup_x {\sf E}[\Delta(x)]^{1+\varepsilon_0}<\infty$, that is, the family 
$\{\Delta(x),\,x\ge 0\}$ is uniformly integrable~\cite{Billingsley}. 
It then follows from the convergence $ \Delta(x)\Rightarrow 
\Delta(\infty)$ that ${\sf E}\Delta(x)\to {\sf E} \Delta(\infty)=\Delta_0$, 
see~(\ref{cond3}).  Thus, one can take~$y_0$ in~(\ref{eq:y0A}) 
and~(\ref{eq:y0B}) in such a way that simultaneously,
\begin{equation*}
\sup\limits_{x\ge y/2}{\sf E}\Delta(x) \le 
\Delta_{0}+\fr{\varepsilon}{4}\,,\enskip y\ge y_{0}\,.
%\label{eq:y0C}
\end{equation*}
As $\varepsilon $ above is arbitrary, define it now    as
$\Delta_{0}+{\sf E} U=-2\varepsilon $ (see~(\ref{eq:condition2})).  Then,
picking together the bounds obtained above, one has
\begin{multline}
{\sf E}(\Delta_{W}(k)|W_{k}=y)={\sf E}(S_{N(y+U)}-y) \\
{}\le
\fr{\varepsilon}{4}+\Delta_{0}+\fr{\varepsilon}{4}+{\sf E}
U+\fr{\varepsilon}{4}<-\varepsilon\,, \enskip y\ge
y_{0}\,.
\label{eq:3termsB}
\end{multline} 
By the standard Lorden's inequality,  
$\sup\limits_x{\sf E}\Delta(x)$\linebreak $\le\;\Delta^*$~\cite{Asmus} and it then follows that
\begin{multline*}
 {\sf E}(S_{N(y+B)}-y)\\
 {}={\sf E}(S_{N(y+B)}-y-B)+{\sf E} B={\sf E}[\Delta(y+B)]+{\sf E} B \\
{} \le  \Delta^{*}+{\sf E} B\le\max(\varepsilon,\,\Delta^{*}+{\sf E} B):=
C<\infty\,.
%\label{17}
\end{multline*} 
(It is necessary to ensure $\varepsilon/C\le1$ in~(\ref{eq:EDWBound}) 
below.) In particular,
\begin{multline}
{\sf E}(\Delta_{W}(k)\,\vert\, W_{k}\le y_{0}) \\
{}=\int\limits_0^{y_0}{\sf E}( 
S_{N(y+U)}-y){\sf P}(W_k\in dy) \le C\,.
\label{eq:DeltaWkcond2}
\end{multline}
Now, for any $k$, use~(\ref{eq:3termsB}) and~(\ref{eq:DeltaWkcond2}) to obtain
 \begin{equation} 
 {\sf E}\Delta_{W}(k)\le C{\sf P}(W_{k}\le
y_{0})-\varepsilon{\sf P}(W_{k}>y_{0})\,.
\label{eq:CPeP}
\end{equation}
Now it will be proved, by contradiction, that $W_{k}\not\to\infty$ in
probability as $k\rightarrow\infty$.    Thus, it is  assumed that
\begin{equation}
 W_{k}\to\infty\;\mbox{in probability as }\;
k\to\infty\,.\label{eq:absurdum}
\end{equation}
 Therefore, one can find a value $k_{0}$ such that 
 $$
{\sf P}(W_{k}\le y_{0})\le\fr{\varepsilon}{4C}\,, \enskip k\ge k_{0}\,,
$$
 which allows to rewrite~(\ref{eq:CPeP}) as 
 \begin{equation}
{\sf  E}\Delta_{W}(k)\le
\fr{\varepsilon}{4}-\varepsilon\left(1-\fr{\varepsilon}{4C}\right)\le
-\fr{\varepsilon}{2}\,,\enskip 
k\ge k_{0}\,.
\label{eq:EDWBound}
\end{equation} 
Using the Lorden's inequality, one 
obtains (see~(\ref{eq:ceilx}))
\begin{multline*}
{\sf E} W_1  =  {\sf E} S_{N(U_0)}\le{\sf E} S_{N (B_0)}=
\int\limits_0^\infty {\sf E} S_{N(z)}F_B(dz)\\
{}= \int\limits_0^\infty {\sf E} \Delta(z)F_B(dz)+\int\limits_{0}^{\infty}z B(dz)
\leq \Delta^{*}+{\sf E} B:= C^{*}
 \end{multline*}
where $F_B(x):={\sf P}(B\le x)$.  Continuing  this analysis iteratively, one 
easily obtains ${\sf E} W_{k_{0}}<k_{0}C^{*}<\infty$ that contradicts~(\ref{eq:absurdum}). 
Hence, assumption~(\ref{eq:absurdum}) is false, and there  
exist a nonrandom (sub-)sequence $z_{k}\to\infty$ and constants 
$\delta^\star>0$ and  $R<\infty$ such that
 \begin{equation}
\inf\limits_{k}{\sf P}(W_{z_{k}}\le R)\ge\delta^\star.
\label{eq:sequence}
\end{equation}
Let now proceed by showing that the probability of reaching an
empty system is strictly positive. Let
discern two cases pertaining to the r.v.~$T$.
In the first case, let assume that~$T$ is unbounded, while the second
case considers bounded~$T$.

\noindent
1. If it is assumed that the r.v.~$T$ is unbounded, then ${\sf P}(T>x)>0$ for any
$x\ge0$. It is easy to see that in the event 
$
\{W_{z_{k}}\le R,\, T_{z_{k}}>R+B_{z_{k}}+\Delta(W_{z_k} + B)\}
$
the burst $z_{k}+1$ meets an empty system, and, hence,  regeneration occurs.
 In other words, the event $\{\beta(z_{k})=1\}$ occurs.
Now, it will be shown that for a given~ $z_{k}$ belonging to the sequence $\{z_{k}\}$ which 
satisfies~(\ref{eq:sequence}), the probability of this event is positive. Indeed,
\begin{multline*}
{\sf P}(\beta(z_{k})=1)\\
{}\ge {\sf P}(\{W_{z_{k}}\le R,\,
T_{z_{k}}>R+B_{z_{k}}+\Delta(W_{z_k} + B)\}) \\
{} =  \int\limits_{x = 0}^{R} \int\limits_{y = 0}^{\infty} {\sf P}\left(W_{z_k} \in dx,\, 
B_{z_k} \in dy,\right.\\
\left. T_{z_k} > R + y + \Delta(x+y) \right) \\
{}  =   \int\limits_{x = 0}^{R} \int\limits_{y = 0}^{\infty} {\sf P}(T_{z_k} > R + y + \Delta(x+y) ) 
{\sf P}(W_{z_k} \in dx) \\
{}\times {\sf P}(B_{z_k} \in dy)\\
{} \ge  \int\limits_{x = 0}^{R} \int\limits_{y = 0}^{b} {\sf P}(T_{z_k} > R + y + 
\Delta(x+y) )\\
{}\times {\sf P}(W_{z_k} \in dx) {\sf P}(B_{z_k} \in dy)
\\
 {}\ge \inf\limits_{x+y \le R+b} {\sf P}(T_{z_k} >
 R + b + \Delta(x+y) )
\delta^\star {\sf P}(B \le b) 
%\label{2.25}
\end{multline*}
where inequality~(\ref{eq:sequence}) and independence between variables 
$W_{z_{k}}, B_{z_{k}}$, and $T_{z_{k}}$ are used, and a constant $b>0$ is such  
that ${\sf P}(B\le b)>0$. Now, note that
\begin{equation}
\Delta(x+y) \le R+b+\Delta(R+b)\,,\enskip x+y \le  R+b\,. 
\label{2.26}
\end{equation}
Moreover, because ${\sf E} \Delta(R+b)<\infty$, then constants
$\varepsilon_0>0$ and $D<\infty$ exist such that
\begin{equation}
{\sf P}(\Delta(R+b)\le D)\ge 1- \varepsilon_0\,. 
\label{2.27}
\end{equation}
Then, using (\ref{2.26}) and~(\ref{2.27}), one has

\noindent
\begin{multline*}
\inf_{x+y \le R+b} {\sf P}(T_{z_k} >
 R + b + \Delta(x+y) )\\
 \ge 
 {\sf P}(T> 2(R + b) +D,\, \Delta(R+b)\le D)\\
\ge {\sf P}(T > 2(R +
 b)+D)(1- \varepsilon_0):=\hat \delta>0\,.
%\label{2.28}
\end{multline*}
Now, one obtains the following lower bound
\begin{equation*}
{\sf P}(\beta(z_{k})=1)\ge \hat \delta \,\delta^\star {\sf P}(B \le b)>0
\end{equation*}
which is uniform  in~$z_k$.

\noindent
2. Assume now that $T$ is bounded; then, it is possible that

\noindent
$$
{\sf P}(T>2(R+b)+\Delta(R+b))=0\,.
$$
In the analysis below,
the first  moment will be found when the random walk
 $S_n$ reaches the zero state, starting into the compact set $[0,\,R+D]$.
More exactly,  each next  burst finds a  shorter FDL upon arrival
until a nonwaiting burst arrives. Consider the independent events

\noindent
\begin{multline*}
\!\!\!{\cal D}(z_{k}+i):= \{T_{z_{k}+i}>\Delta_{z_k+i}+B_{z_{k}+i},\,
\delta^* \le \Delta_{z_k} \le D\} \hspace*{-0.13023pt}\\ 
{} = \!\{B_{z_{k}+i}-T_{z_{k}+i} < -\Delta , \, 
\delta^* \le \Delta_{z_k} \le D \}\,, \ i\ge0\,, %\label{eq:events}
\end{multline*}
where,  by (\ref{cond2}) and  (\ref{eq:condition3}), constants 
$D<\infty$ and $\delta^*>0$ are  chosen in  such a way that (regardless of~$z_k$ and~$i$)

\noindent
\begin{equation}
{\sf P}\left(T_{z_k+i} > B_{z_k+i} + D,\, \Delta_{z_k+i}\in (\delta^*, \, D]\right)\ge 
\fr{\delta_1}{2}\,.
\label{2.29}
\end{equation}
It is clear that on each event ${\cal D}(z_{k}+i)$, the random
walk $S_n$ decreases by at least~$\Delta$ and thus, each new
burst occupies a shorter FDL. Note that $W_{z_k}\le S_{N(R)}$ in 
the event ${\cal E}(z_{k})=\{W_{z_{k}}\le R\}$ and that in
the event ${\cal E}(z_{k})\cap{\cal D}(z_{k})$,

\noindent
\begin{multline*}
W_{z_{k}+1}  =  S_{N ((W_{z_k}+B_{z_k}-T_{z_k}))}  \le
S_{N(R)}-\Delta \\
=S_{(N(R)-1)^+}
\end{multline*}
where   ${\sf P}({\cal E}(z_{k})\cap{\cal D}(z_{k}))\ge
\delta^\star\delta_{1}/2$ in view of~(\ref{eq:sequence}) and~(\ref{2.29}). 
Continuing in such a way, one finds that in the event

\noindent
$$
{\cal B} (z_k):={\cal E}(z_{k})\cap\bigcap_{i=0}^{N(R)}{\cal D}(z_{k}+i)\,,
$$
a  burst arrives, among the bursts with the numbers $\{z_k,\ldots,
z_k+N(R)\}$,  which finds the transmission channel free, and thus 
regeneration occurs.  Moreover, it is easy to see that in the event
${\cal B}(z_k)$, 
\begin{equation*}
N(R)\le \left\lceil \fr{R+D}{\delta^{*}}\right\rceil:=C_0 <\infty\,.
\end{equation*}
Now, it follows that

\noindent
\begin{equation*}
{\sf P}(\beta(z_{k})\le C_0)\ge {\sf P}({\cal B}(z_k))\ge \delta^\star
\left[\fr{\delta}{2}\right]^{C_0}>0\,. 
%\label{2.30}
\end{equation*}
As  the lower bound holds for all $z_{k}$ satisfying~(\ref{eq:sequence}) 
and  the sequence $\{z_{k}\}$ is deterministic, then $\beta(n)\not\to\infty$, and
 it   follows from~(\ref{eq:Feller}) that ${\sf E}\beta<\infty$.\hfill$\blacksquare$
 
%\vspace*{-6pt}

\section{Verification of the~Stability Region by~Simulation}
\setcounter{section}{4}

\noindent
In this section, the  simulation
results,  which illustrate the stability/instability domain of the
described optical system and the accuracy of the  stability
conditions found above, are presented. The simulation has been carried out by means
of the {\it ``R'' toolbox}: the language and the environment for
statistical computing~\cite{r_reference}. The conditions of Theorem~1
imply the positive recurrence of the workload process~$W$. 

First, let
check the main negative drift condition~(\ref{eq:condition2})
which includes the stationary overshoot~$\Delta_0$ of the FDL
length. As simulations below suggest, condition~(\ref{eq:condition2}) is, in
fact, a {\it stability criterion}, meaning that it appears to sharply delineate
the regions of stability and instability. On the other hand,  a less tight
condition has been suggested in previous related works~\cite{optical1, optical2}. 
This condition, in adaptation to the
model described, can be written~as
\begin{equation}
\label{delta*} {\sf E} U + \Delta^* < 0
\end{equation}
where $\Delta^*$  from~(\ref{cond3}) is used. Indeed, condition~(\ref{delta*}) 
is suggested  in~\cite{optical1, optical2} for   the model  with the different 
{\it deterministic} distances $\{\Delta_n\}$, and in this case, the appearance 
of the bound $\Delta^*\ge \sup\limits_x {\sf E} \Delta(x)$ is motivated by   the 
Lorden's inequality.  However, simulations of the model with  deterministic 
granularity $\Delta=const$  have  shown that
 condition~(\ref{delta*}) is indeed 
overly restrictive. The inspection paradox presented above has revealed the 
source of this redundancy and has allowed to replace~$\Delta^*$  by the smaller 
quantity~$\Delta_0$ in~(\ref{eq:condition2}).
 It explains  why below, the 
accuracy of both conditions~(\ref{eq:condition2}) and~(\ref{delta*}) is verified.
 In particular, 
simulations confirm that in the intermediate case when condition~(\ref{delta*}) 
is  violated but condition~(\ref{eq:condition2}) holds, the system remains stable.
Visually, absence of a clear tendency in the workload process indicates
that the system is stable, while instability is characterized by an 
increase of~$W_n$ as $n\to 
\infty$.  These evident observations are used  below to determine 
stability/instability of the system.

 In the numerical experiments,
 5000~bursts  have been simulated following a Poisson input with
 rate~$\lambda$ and with   the i.i.d.\ exponential service times  with
 rate~$\mu$.  Moreover, the following distributions of
 the  difference~$\Delta$ have been  considered: ($i$)~exponential; ($ii$)~uniform; 
 and ($iii$)~Pare-\linebreak\vspace*{-12pt}

\begin{center}  %fig1
\vspace*{-3pt}
\mbox{%
 \epsfxsize=80mm
 \epsfbox{mor-1.eps}
 }
  \end{center}
  
    \vspace*{-2pt}

\noindent
{{\figurename~1}\ \ \small{The system with exponential $\Delta \sim \exp(3)$, 
$\Delta_0 = 0.333$, $\Delta^* = 0.667$, and $\rho = 0.94$: \textit{1}~--- 
${\sf E}U+\Delta_0$\protect\linebreak $=-0.268<0$, $\lambda=0.1$, and $\mu=0.1064$;
and \textit{2}~--- ${\sf E}U+\Delta_0=0.273>0$, $\lambda=1$, and $\mu=1.0638$}}


%\pagebreak

\vspace*{18pt}

\begin{center}  %fig2
\mbox{%
 \epsfxsize=79.849mm
 \epsfbox{mor-2.eps}
 }
  \end{center}

  \vspace*{-2pt}

\noindent
{{\figurename~2}\ \ \small{The system with uniform $\Delta \in [0,0.8]$, 
$\Delta_0$\protect\linebreak $ =\;0.267$, $\Delta^* = 0.533$, and $\rho = 0.94$: \textit{1}~--- 
${\sf E}U+\Delta_0$\protect\linebreak $=-0.230<0$, $\lambda=0.12$, and $\mu=0.1276$;
and \textit{2}~--- ${\sf E}U$\protect\linebreak $+\;\Delta_0\hm=0.217>0$, $\lambda=1.2$, 
and $\mu=1.2766$}}

\vspace*{18pt}

\begin{center}  %fig3
 \mbox{%
 \epsfxsize=80.111mm
 \epsfbox{mor-3.eps}
 }
  \end{center}

  \vspace*{-2pt}
  
  \noindent
{{\figurename~3}\ \ \small{The system with $\Delta$ that has Pareto distribution with parameter 
$\alpha=4$,
$\Delta_0 = 0.75$, $\Delta^* = 1.5$, and $\rho = 0.94$: \textit{1}~--- ${\sf E}U+\Delta_0=-0.527<0$,
$\lambda =0.047$, and $\mu=0.05$; and \textit{2}~--- ${\sf E}U+\Delta_0=0.622>0$,
$\lambda=0.47$, and $\mu=0.5$}}

%\vspace*{12pt}


\noindent
to,  $F_\Delta(x)=1-x^{-\alpha}$, $x\ge 1$ $(F_\Delta(x)=0,\,x<1)$.  
 The  simulations have shown  similar
results for   a range of traffic intensity $\rho:=\lambda/\mu$,
but below, the present results for the most illustrative {\large}
value $\rho = 0.94$ are presented, in which case,  an increase  of the workload
process is more easy to observe if condition~(\ref{eq:condition2}) is
violated. (For smaller values of~$\rho$, this tendency  requires much
more time to be detected.) 

Figure~1 describes  the system with exponential~$\Delta$ with 
parameter~$3$. As one can see, satisfying condition~(\ref{eq:condition2}) implies 
stability. At the same time, in  this case, ${\sf E} U + \Delta^* = 0.065$, and 
a definite decision about stability   could not be made using condition~(\ref{delta*}). 
Otherwise, if condition~(\ref{eq:condition2}) is violated, the 
system is unstable. In this case, the value  ${\sf E} U + \Delta^* = 0.607$ 
also indicates instability.


Figure~2 describes  the system with~$\Delta$ uniformly distributed 
in  the interval $[0,\, 0.8]$.  { Satisfying condition~(\ref{eq:condition2}) implies 
stability; however, in this  case, ${\sf E} U + \Delta^* = 0.037$. Otherwise, if 
 condition~(\ref{eq:condition2}) is violated, visual observation confirms 
instability and it agrees with  the positivity of the sum  ${\sf E} U + 
\Delta^* = 0.483$.}

Finally, Fig.~3 corresponds to the system where~$\Delta$ has 
Pareto distribution  with parameter $\alpha = 4$. One can see that satisfying 
condition~(\ref{eq:condition2}) implies stability,   while in  this case, ${\sf E} 
U + \Delta^* = 0.223$. Otherwise, if condition~(\ref{eq:condition2}) is violated, 
the system is unstable, and again, it is consistent with the value  ${\sf E} U + 
\Delta^* = 1.372$.
{\looseness=1

}


 Thus, in all considered cases the simulations  confirm   that
condition~(\ref{eq:condition2}) allows to delimit  the
stability/instability region with high accuracy and that condition~(\ref{eq:condition2}) 
is indeed overly strict.

\section{Concluding Remarks}
\setcounter{section}{4} 

\noindent
In this work, a new model of an optical
buffer system is suggested.  The main assumption is that the
distances $\{\Delta_n\}$ between the lengths of the adjacent
increasing FDLs are random and have the same distribution for all
lines. This is a natural extension of the model with the {\it
deterministic granularity}  $\Delta_n\equiv const$ considered in~\cite{Call}. For this reason, the new model considered above
 can be referred to as an optical system with {\it
stochastic granularity}. The key new ingredient of the stability
analysis  is the renewal theory, including such asymptotic results as
the {\it inspection paradox} and the {\it Lorden's inequality}.
The regenerative approach is applied to obtain sufficient stability
conditions of the system.

As simulations show, the stability condition found  allows to
delimit the stability domain with high accuracy suggesting
that it is in fact the stability criterion of the model.

As an extension of the model, one may  consider the case when
distributions $F_n $ of  $\{\Delta_n\}$ are different for
$n=1,\ldots,n_0$, being  the same, $F_n\equiv F$, for $n>n_0$ with a
 deterministic~$n_0$. Another modification may   assume that
$F_{n}\to F$ (weakly) as $n\to \infty$. The authors believe that the analysis
developed above can be applied in both cases. Finally, if we were to
weaken the requirement $\Delta>0$, which may be problematic for
distributed optical networks,  allowing~$\Delta$ to have  arbitrary
sign, when the fundamental rule FIFO  is violated. An analysis of such
an extension  is probably possible  within the framework of the extended
renewal theory (where the renewal ``interval'' can be negative); however,
this analysis is beyond the scope this paper.



\Ack
The work of the first and second  authors is partially supported by the
Program of Strategy development of the Petrozavodsk State University within
the framework of the research activity. The third author is a
postdoctoral fellow with the Research Foundation-Flanders
(FWO-Vlaanderen).


{\small\frenchspacing
{%\baselineskip=10.8pt
\begin{thebibliography}{9}

\bibitem{Call} %1
\Aue{Callegati, F.} 
2000. Optical buffers for variable
length packets. \textit{IEEE Comm. Lett.}
4(9):292--294.

\bibitem{optical1} %2
\Aue{Rogiest,~W., E.~Morozov, D.~Fiems, K.~Laevens, and H.~Bruneel}.
2010. Stability of single-wavelength optical buffers.
\textit{Eur. Trans.   Telecomm.} 21(3):202--212.
\bibitem{optical2}   %3
\Aue{Morozov,~E., W. Rogiest, K.~De~Turck, and D.~Fiems}. 2012. 
Stability of
multiwavelength optical buffers with delay-oriented scheduling.
\textit{Trans. Emerging Telecomm. Technol.} 23(3):217--226.

\bibitem{Asmus} %4
\Aue{Asmussen, S}. 2003. \textit{Applied probability and queues}. 2nd
ed.  NY: Springer-Verlag. 440~p.





\bibitem{2} %5
\Aue{Morozov,~E., and R.~Delgado}. 
2009. Stability analysis of regenerative
queueing  systems. \textit{Automation Remote Control} 70(12):1977--1991.

\bibitem{Feller} %6
\Aue{Feller,~W.}
1971. \textit{An introduction to probability theory and its applications}. Vol.~II.
 New York: John Wiley\,\&\,Sons. 704~p.
 {\looseness=1
 
 }



\bibitem{Chang}  %7
\Aue{Chang,~J.}
1994. Inequalities for the overshoot. \textit{Ann. Appl. Probab.}  4(4):1223--1233.

\bibitem{Billingsley} %8
\Aue{Billingsley, P.}  1968.
\textit{Convergence of probability measures}. Wiley. 296~p.

\bibitem{r_reference} %9
\textit{R Foundation for Statistical Computing}.
  Vienna, Austria.  {\sf http://www.R-project.org/}.
\end{thebibliography} } }

\end{multicols}

\vspace*{-6pt}

\hfill{\small\textit{Received November 8, 2013}}

\vspace*{-6pt}

\Contr

\noindent
\textbf{Morozov Evsei V.} (b.\ 1947)~--- Doctor of Science in physics and mathematics, professor,
leading scientist, Institute of Applied Mathematical Research of Karelian Research Center,
Russian Academy of Sciences, 11 Pushkinskaya Str., Petrozavodsk 185910,
Republic of Karelia, Russian Federation; professor, Petrozavodsk State University,
33 Lenin Str., Petrozavodsk 185910, Republic of Karelia,
Russian Federation; emorozov@karelia.ru

\vspace*{3pt}

\noindent
\textbf{Potakhina Lyubov V.} (b.\ 1989)~--- PhD student, 
Institute of Applied Mathematical Research of Karelian Research Center,
Russian Academy of Sciences, 11 Pushkinskaya Str., Petrozavodsk 185910,
Republic of Karelia, Russian Federation; engineer, Petrozavodsk State University,
33 Lenin Str., Petrozavodsk 185910, Republic of Karelia,
Russian Federation; lpotahina@gmail.com

\vspace*{3pt}

\noindent
\textbf{Koen De Turck} (b.\ 1981)~--- postdoctoral fellow with the Research Foundation
Flanders (FWO-Vlaanderen), SMACS Research Group, Department of Telecommunications
and Information Processing (TELIN), Ghent University, 41 Sint-Pietersnieuwstraat,  Gent B-9000,
Belgium;  kdeturck@telin.ugent.be




%\vspace*{24pt}

%\hrule

%\vspace*{2pt}

%\hrule

%\vspace*{12pt}

\newpage


\def\tit{АНАЛИЗ  УСТОЙЧИВОСТИ СИСТЕМЫ ПЕРЕДАЧИ ДАННЫХ  
С~ОПТИЧЕСКИМИ ЛИНИЯМИ ЗАДЕРЖКИ СЛУЧАЙНОЙ ДЛИНЫ}

\def\aut{Е.\,В. Морозов$^1$,  Л.\,В. Потахина$^2$,  К.~Де Турк$^3$}


\def\titkol{Анализ  устойчивости системы передачи данных  с~оптическими линиями задержки случайной длины}

\def\autkol{Е.\,В. Морозов,  Л.\,В. Потахина,  Де Турк~K.}


\titel{\tit}{\aut}{\autkol}{\titkol}

\vspace*{-12pt}

\noindent $^1$Институт прикладных математических исследований КарНЦ РАН,
 Россия, Республика Карелия,\\
 $\hphantom{^1}$г.~Петрозаводск 185910, ул.\ Пушкинская 11;
Петрозаводский государственный университет,\\
$\hphantom{^1}$Россия, Республика Карелия, г.~Петрозаводск 185910, пр.\ Ленина 33; emorozov@karelia.ru\\ 
\noindent $^2$Институт прикладных математических исследований КарНЦ РАН,
 Россия, Республика Карелия,\\
  $\hphantom{^1}$г.~Петрозаводск 185910, ул.\ Пушкинская 11;
Петрозаводский государственный университет,\\ 
 $\hphantom{^1}$Россия, Республика Карелия, г.~Петрозаводск 185910, пр.\ Ленина 33; lpotahina@gmail.com\\
\noindent
$^3$Университет Гента,
Sint-Pietersnieuwstraat 41,  Гент B-9000, Бельгия; kdeturck@telin.ugent.be


\vspace*{6pt}

\def\leftfootline{\small{\textbf{\thepage}
\hfill ИНФОРМАТИКА И ЕЁ ПРИМЕНЕНИЯ\ \ \ том\ 8\ \ \ выпуск\ 1\ \ \ 2014}
}%
 \def\rightfootline{\small{ИНФОРМАТИКА И ЕЁ ПРИМЕНЕНИЯ\ \ \ том\ 8\ \ \ выпуск\ 1\ \ \ 2014
\hfill \textbf{\thepage}}}
 

\Abst{Рассмотрена новая модель оптической системы, в которой разности упорядоченных  
(по возрастанию)  длин  двух  соседних оптических  кабелей  
являются независимыми одинаково распределенными случайными величинами.  
Это является обобщением рассмотренной ранее модели, где  эти разности  
были детерминированными величинами.  Эта система моделируется с использованием   
теории случайного блуждания и тесно связанных  с ней асимптотических результатов  
 теории восстановления,  таких как  парадокс инспекции  и неравенство Лордена.   
 Развит анализ устойчивости, основанный на регенеративном подходе. Включены  
 также некоторые численные результаты, которые показывают, 
что полученные условия позволяют определить область устойчивости с высокой точностью.}

\KW{оптический буфер; устойчивость;  теория восстановления; регенерация; 
парадокс инспекции; неравенство Лордена;   имитационное моделирование}


\DOI{10.14357/19922264140113}


Работа первых двух авторов частично поддержана Программой стратегического развития Петрозаводского
государственного университета.
Работа третьего автора поддержана
Research Foundation-Flanders
(FWO-Vlaanderen).


 \begin{multicols}{2}

\renewcommand{\bibname}{\protect\rmfamily Литература}
%\renewcommand{\bibname}{\large\protect\rm References}

{\small\frenchspacing
{%\baselineskip=10.8pt
\addcontentsline{toc}{section}{References}
\begin{thebibliography}{9}

\bibitem{Call-1}  %1
\Au{Callegati F.} 
Optical buffers for variable
length packets~// {IEEE Comm. Lett.}, 2000.
Vol.~4. No.\,9. P.~292--294.

\bibitem{optical1-1} %2
\Au{Rogiest~W., Morozov E., Fiems D., Laevens~K., Bruneel~H.}
Stability of single-wavelength optical buffers~//
{Eur. Trans. Telecomm.}, 2010. Vol.~21. No.\,3. P.~202--212.
\bibitem{optical2-1}  %3
\Au{Morozov~E., Rogiest W., De~Turck~K., Fiems~D}. 
Stability of multiwavelength optical buffers with delay-oriented scheduling~//
{Trans. Emerging Telecomm. Technol.}, 2012. 
Vol.~23. No.\,3. P.~217--226.

\bibitem{Asmus-1} %4
\Au{Asmussen S}.  {Applied probability and queues}.~--- 2nd
ed.~---  NY: Springer-Verlag, 2003. 440~p.


\bibitem{2-1} %5
\Au{Morozov~E., Delgado~R}. 
Stability analysis of regenerative
queueing  systems~// Automation Remote Control, 2009. Vol.~70. No.\,12.
P.~1977--1991.

\bibitem{Feller-1}  %6
\Au{Feller~W.}
{An introduction to probability theory and its applications}. Vol.~II.~--- 1971.
New York: John Wiley\,\&\,Sons. 704~p.


\bibitem{Chang-1}  %7
\Au{Chang~J.}
 Inequalities for the overshoot~// \textit{Ann. 
Appl. Probab.}, 1994.  Vol.~4. No.\,4. P.~1223--1233.

\bibitem{Billingsley-1}
\Au{Billingsley, P.} %8  
{Convergence of probability measures}.~--- Wiley, 1968. 296~p.

\bibitem{r_reference-1}  %9
{R Foundation for Statistical Computing}.
  Vienna, Austria. {\sf http://www.R-project.org/}.

\end{thebibliography}
} }

\end{multicols}

 \label{end\stat}

\hfill{\small\textit{Поступила в редакцию 08.11.2013}}
%\renewcommand{\bibname}{\protect\rm Литература}  
\renewcommand{\figurename}{\protect\bf Рис.}