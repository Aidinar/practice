\def\stat{gorshenin}

\def\tit{АНАЛИЗ ВЕРОЯТНОСТНО-СТАТИСТИЧЕСКИХ ХАРАКТЕРИСТИК ОСАДКОВ НА~ОСНОВЕ ПАТТЕРНОВ$^*$}

\def\titkol{Анализ вероятностно-статистических характеристик осадков на основе паттернов}

\def\aut{А.\,К.~Горшенин$^1$}

\def\autkol{А.\,К.~Горшенин}

\titel{\tit}{\aut}{\autkol}{\titkol}

\index{Горшенин А.\,К.}
\index{Gorshenin A.\,K.}


{\renewcommand{\thefootnote}{\fnsymbol{footnote}} \footnotetext[1]
{Работа выполнена при поддержке РФФИ (проекты 17-07-00851, 15-07-04040 и~15-07-05316).}}


\renewcommand{\thefootnote}{\arabic{footnote}}
\footnotetext[1]{Институт проблем информатики 
Федерального исследовательского центра <<Информатика и~управ\-ле\-ние>> 
Российской академии наук;
Институт океанологии им.\ П.\,П.~Ширшова Российской академии наук,
\mbox{agorshenin@frccsc.ru}}

%\vspace*{-18pt}




\Abst{Осадки входят в~число ключевых параметров гидрологических
моделей, поэтому их исследование необходимо для решения различных
прикладных задач. В~работе продемонстрировано нарушение марковского
свойства для осадков, наблюдаемых в~существенно различающихся между собой
климатических областях~--- в~Потсдаме и~Элисте. Такие сведения о~данных,
наряду с~ранее исследованными свойствами, представляют базовую информацию,
необходимую для дальнейшего корректного построения вероятностных моделей, 
в~частности для распределений объемов экстремальных осадков. Для анализа
вероятностного поведения процесса выпадения осадков и~построения прогнозов
предложено использование цепочек событий (паттернов), выделенных из данных.
При этом статистические процедуры автоматизированы с~использованием
программных инструментов пакета {\sf MATLAB}. В~качестве альтернативного
инструмента прогнозирования на основе паттернов использованы нейронные сети,
при этом наиболее точные результаты продемонстрированы в~архитектуре,
учитывающей сезонность, с~двумя скрытыми слоями нейронов и~сигмоидной
функцией активации. Предложены направления дальнейших исследований 
в~данной области.}

\KW{осадки; паттерны; прогноз; нейронные сети; 
вероятностное прогнозирование; марковское свойство}

\DOI{10.14357/19922264170405} 

\vspace*{19pt}

\vskip 10pt plus 9pt minus 6pt

\thispagestyle{headings}

\begin{multicols}{2}

\label{st\stat}


\section{Введение}

Как известно, осадки являются важной частью гидрологических моделей, поэтому
востребовано построение адекватных математических моделей (в~том
числе ве\-ро\-ят\-ност\-но-ста\-ти\-сти\-че\-ских), например для работы 
с~регионами, в~которых сети датчиков не покрывают полностью зоны
необходимого наблюдения~\cite{Strauch2012}. Такие модели могут быть
использованы для решения задач вероятностного
прогнозирования (см., например, работу~\cite{Krzysztofowicz2001}), формирования
статистических сценариев (см.\ статью~\cite{Pinson2009}, в~которой подобная
задача решается в~области ветроэнергетики). Более того, возможно построение
сложных прогностических математических моделей на основе плотностей 
и~функций распределения~\cite{Gneiting2007}.
{\looseness=1

}

Выявление и~моделирование закономерностей для экстремальных осадков
весьма важны при решении различных задач изучения климата~\cite{Kharin2007}, 
в~том числе с~целью понимания процессов его изменения. Можно анализировать
усредненные объемы осадков, кумулятивные данные за каждый дождливый период
или ежедневные наблюдения. При этом в~последнем случае результаты достаточно
сильно зависят от точности измерений, более чувствительны к~наличию
пропусков~\cite{Zolina2005}.

\begin{figure*}[b] %fig1
      \vspace*{1pt}
 \begin{center}
 \mbox{%
 \epsfxsize=120.85mm 
 \epsfbox{gor-1.eps}
 }
\end{center}
\vspace*{-11pt}
\Caption{Частоты для четырехдневных паттернов (Потсдам)
\label{FigPatternsPotsdam4}}
\end{figure*}

Настоящая работа посвящена изучению ряда вероятностно-статистических
характеристик для процесса выпадения осадков на основе анализа суточных
наблюдений для двух городов~--- Потсдама и~Элисты. Предложена замена
непрерывных значений дискретной шкалой из <<0>> и~<<1>>, со\-от\-вет\-ст\-ву\-ющих
дням, в~которых осадки не наблюдались (<<D>>, от \verb"Dry") или были
зарегистрированы ненулевые наблюдения (<<W>>, от \verb"Wet"). Наборы~<<0--1>>
удобны, например, с~точки зрения программной реализации (как аналог булевых
векторов), однако всюду далее в~статье для наглядности будут использованы
символы <<D>> и~<<W>> для обозначения со\-от\-вет\-ст\-ву\-ющих событий.
Рассмотрено по\-стро\-ение прогнозов для преобразованных таким образом данных
(в~том числе с~помощью нейронных сетей), а~также предложен ряд направлений
дальнейших исследований. 

Стоит отметить, что 
продемонстрированная достаточно высокая прогностическая точность для нейросетей~--- 
около~80\% даже для умеренного объема тестовых выборок~--- 
основана на анализе исключительно базовых статистических данных без 
привлечения ка\-ких-ли\-бо дополнительных сведений о~метеорологических условиях.

\section{Дискретизация величины дневных объемов осадков}

Рассмотрим преобразование исходных объемов суточных осадков~$V_{\mathrm{daily}}$,
представляющих собой неотрицательные данные, по следующему правилу: если 
в~данный $i$-й день наблюдалась ка\-кая-ли\-бо положительная величина, то она
заменяется на единицу ($\widetilde{V}_{\mathrm{daily}}^{(i)}=1$), иначе
$\widetilde{V}_{\mathrm{daily}}^{(i)}$ остается равной нулю. Таким образом, исходный ряд,
состоящий из непрерывных значений, становится дискретным, принимающим два
возможных значения $\{0,1\}$. Данное упрощение позволяет анализировать
непосредственно наличие или отсутствие осадков безотносительно к~их объему.
Стоит отметить, что подобная дискретизация не позволяет решать задачу
определения экстремальности величины того или иного <<дождя>>, однако ряд
представленных в~данном разделе результатов использовался в~качестве
предпосылок для построения моделей в~\mbox{статье}~\cite{Gorshenin2017a}.

Любая последовательность <<сухих>> (без осадков) и~<<дождливых>> (с~любым
ненулевым объемом) дней представляет собой цепочку~--- паттерн~--- 
из~<<0>> и~<<1>> (или~<<D>> и~<<W>>, как было отмечено в~предыду\-щем разделе). Для
каждого такого набора можно в~рамках исторических данных определить частоты
появления как отношение числа таких паттернов фиксированной длины~$N$ 
к~общему числу возможных наборов (очевидно, $2^N$), т.\,е.\ фактически
получить значения вероятностей (согласно классическому определению).



В рамках настоящего исследования были проанализированы наблюдения
примерно за $60$ лет для значений параметра~$N$ от~1 до~14. Для каждого
набора были получены значения частот (вероятностей), определен паттерн 
с~максимальным значением. Подобные данные удобно визуализировать 
в~табличной форме (см.\ табл.~1 и~2 для трехдневных
паттернов Потсдама и~Элис\-ты соответственно). Кроме того, в~наглядной форме
результаты можно представить в~виде столбчатых диаграмм (см.\
рис.~\ref{FigPatternsPotsdam4} и~\ref{FigPatternsElista4} для четырехдневных
паттернов Потсдама и~Элис\-ты соответственно).

\begin{figure*} %fig2
\vspace*{1pt}
 \begin{center}
 \mbox{%
 \epsfxsize=120.754mm 
 \epsfbox{gor-2.eps}
 }
\end{center}
\vspace*{-11pt}
\Caption{Частоты для четырехдневных паттернов (Элиста)\label{FigPatternsElista4}}
\end{figure*}

\setcounter{table}{2}

\begin{table*}[b]\small %tabl3
\vspace*{-6pt}
\begin{center}
\Caption{Таблица значений для условных вероятностей (Потсдам)}
\label{TabProbPotsdam}
\vspace*{2ex}

\tabcolsep=16pt
\begin{tabular}{|c|c|}
\hline
{Выражение}&{Значение}\\
\hline
$\mathbb{P}(\{\mathrm{DDD}\}\mid \{\mathrm{DD}\})$&$0{,}7774$\\
%\hline
$\mathbb{P}(\{\mathrm{DDW}\}\mid \{\mathrm{DD}\})$&$0{,}2226$\\
%\hline
$\mathbb{P}(\{\mathrm{DWD}\}\mid \{\mathrm{DW}\})$&$0{,}3785$\\
%\hline
$\mathbb{P}(\{\mathrm{DWW}\}\mid \{\mathrm{DW}\})$&$0{,}6215$\\
%\hline
$\mathbb{P}(\{\mathrm{WDD}\}\mid \{\mathrm{WD}\})$&$0{,}6043$\\
%\hline
$\mathbb{P}(\{\mathrm{WDW}\}\mid \{\mathrm{WD}\})$&$0{,}3957$\\
%\hline
$\mathbb{P}(\{\mathrm{WWD}\}\mid \{\mathrm{WW}\})$&$0{,}3484$\\
%\hline
$\mathbb{P}(\{\mathrm{WWW}\}\mid \{\mathrm{WW}\})$&$0{,}6516$\\
%\hline
$\left|\mathbb{P}(\{\mathrm{DDD}\}\mid \{\mathrm{DD}\})-\mathbb{P}(\{\mathrm{DD}\}\mid \{\mathrm{D}\})\right|$&$0{,}0466$\\
%\hline
$\left|\mathbb{P}(\{\mathrm{DDW}\}\mid \{\mathrm{DD}\})-\mathbb{P}(\{\mathrm{DW}\}\mid \{\mathrm{D}\})\right|$&$0{,}0466$\\
%\hline
$\left|\mathbb{P}(\{\mathrm{DWD}\}\mid \{\mathrm{DW}\})-\mathbb{P}(\{\mathrm{WD}\}\mid \{\mathrm{W}\})\right|$&$0{,}0192$\\
%\hline
$\left|\mathbb{P}(\{\mathrm{DWW}\}\mid \{\mathrm{DW}\})-\mathbb{P}(\{\mathrm{WW}\}
\mid \{\mathrm{W}\})\right|$&$0{,}0192$\\
%\hline
$\left|\mathbb{P}(\{\mathrm{WDD}\}\mid \{\mathrm{DD}\})-\mathbb{P}(\{\mathrm{DD}\}\mid \{\mathrm{D}\})\right|$&$0{,}1265$\\
%\hline
$\left|\mathbb{P}(\{\mathrm{WDW}\}\mid \{\mathrm{WD}\})-\mathbb{P}(\{\mathrm{DW}\}\mid \{\mathrm{D}\})\right|$&$0{,}1265$\\
%\hline
$\left|\mathbb{P}(\{\mathrm{WWD}\}\mid \mathrm{WW})-\mathbb{P}(\{\mathrm{WD}\}\mid \{\mathrm{W}\})\right|$&$0{,}0107$\\
%\hline
$\left|\mathbb{P}(\{\mathrm{WWW}\}\mid \{\mathrm{WW}\})-\mathbb{P}(\{\mathrm{WW}\}\mid \{\mathrm{W}\})\right|$&$0{,}0107$\\
\hline
\end{tabular}
\end{center}
\end{table*}

Представленные таблицы и~диаграммы позволяют делать определенные выводы
о~климатических зонах, в~которых расположены соответствующие города. Так, 
в~Потсдаме климат умеренный, продолжительные осадки не редкость (например,
для  трех  дней подряд частота~--- $0{,}1789$), в~то время как\linebreak\vspace*{-12pt}

%\begin{table*}
{
\begin{center}
\vspace*{3pt}
\begin{minipage}[t]{38mm}
%\Caption{Значения частот для трехдневных паттернов (Потсдам)}
%\label{TabPotsdam}
%\vspace*{2ex}
{{\tablename~1}\ \ \small{Значения час\-тот для трехдневных паттернов (Потсдам)}}

\vspace*{2ex}

\tabcolsep=5.5pt
{\small \begin{tabular}{|c|c|}
\hline
{Набор}&{Частота}\\
\hline
\verb"Dry-Dry-Dry"&$\bf 0{,}3247$\\
%\hline
\verb"Dry-Dry-Wet"&$0{,}0930$\\
%\hline
\verb"Dry-Wet-Dry"&$0{,}0582$\\
%\hline
\verb"Dry-Wet-Wet"&$0{,}0956$\\
%\hline
\verb"Wet-Dry-Dry"&$0{,}0930$\\
%\hline
\verb"Wet-Dry-Wet"&$0{,}0609$\\
%\hline
\verb"Wet-Wet-Dry"&$0{,}0957$\\
%\hline
\verb"Wet-Wet-Wet"&$0{,}1789$\\
\hline
\end{tabular}
}
%\end{center}
\end{minipage}
%\end{table*}
%\begin{table*}
\hfill
\begin{minipage}[t]{38mm}
%{\small %tabl2
%\begin{center}
%\Caption{Значения частот для трехдневных паттернов (Элиста)}
%\label{TabElista}
%\vspace*{2ex}
{{\tablename~2}\ \ \small{Значения час\-тот для трехдневных паттернов (Элиста)}}

\vspace*{2ex}

\tabcolsep=5.5pt
{\small
\begin{tabular}{|c|c|}
\hline
{Набор}&{Частота}\\
\hline
\verb"Dry-Dry-Dry"&$\bf 0{,}4864$\\
%\hline
\verb"Dry-Dry-Wet"&$0{,}1021$\\
%\hline
\verb"Dry-Wet-Dry"&$0{,}0751$\\
%\hline
\verb"Dry-Wet-Wet"&$0{,}0660$\\
%\hline
\verb"Wet-Dry-Dry"&$0{,}1022$\\
%\hline
\verb"Wet-Dry-Wet"&$0{,}0390$\\
%\hline
\verb"Wet-Wet-Dry"&$0{,}0660$\\
%\hline
\verb"Wet-Wet-Wet"&$0{,}0631$\\
\hline
\end{tabular}}
\end{minipage}
\end{center}
}
%\end{table*}



\noindent
 в~Элисте климат
резко континентальный с~умеренным числом осадков (частота <<трехдневного>>
дождя за период наблюдений составила всего~$0{,}0631$). Для
четырнадцатидневных наборов максимальную час\-то\-ту для обоих городов имеет
последовательность из всех <<сухих>> дней, при этом для Элисты
соответствующая частота составила~$0{,}1138$, а~для Потсдама~--- $0{,}0671$.

\setcounter{table}{3}

\begin{table*}\small %tabl4
\begin{center}
\Caption{Таблица значений для условных вероятностей (Элиста)}
\label{TabProbElista}
\vspace*{2ex}

\tabcolsep=14pt
\begin{tabular}{|c|c|}
\hline
{Выражение}&{Значение}\\
\hline
$\mathbb{P}(\{\mathrm{DDD}\}\mid \{\mathrm{DD}\})$&$0{,}8264$\\
%\hline
$\mathbb{P}(\{\mathrm{DDW}\}\mid \{\mathrm{DD}\})$&$0{,}1736$\\
%\hline
$\mathbb{P}(\{\mathrm{DWD}\}\mid \{\mathrm{DW}\})$&$0{,}532$\hphantom{9}\\
%\hline
$\mathbb{P}(\{\mathrm{DWW}\}\mid \{\mathrm{DW}\})$&$0{,}468$\hphantom{9}\\
%\hline
$\mathbb{P}(\{\mathrm{WDD}\}\mid \{\mathrm{WD}\})$&$0{,}2761$\\
%\hline
$\mathbb{P}(\{\mathrm{WDW}\}\mid \{\mathrm{WD}\})$&$0{,}5114$\\
%\hline
$\mathbb{P}(\{\mathrm{WWD}\}\mid \{\mathrm{WW}\})$&$0{,}4887$\\
%\hline
$\mathbb{P}(\{\mathrm{WWW}\}\mid \{\mathrm{WW}\})$&$0{,}8264$\\
%\hline
$\left|\mathbb{P}(\{\mathrm{DDD}\}\mid \{\mathrm{DD}\})-\mathbb{P}
(\{\mathrm{DD}\}\mid \{\mathrm{D}\})\right|$&$0{,}0198$\\
%\hline
$\left|\mathbb{P}(\{\mathrm{DDW}\}\mid \{\mathrm{DD}\})-\mathbb{P}
(\{\mathrm{DW}\}\mid \{\mathrm{D}\})\right|$&$0{,}0198$\\
%\hline
$\left|\mathbb{P}(\{\mathrm{DWD}\}\mid \{\mathrm{DW}\})-\mathbb{P}
(\{\mathrm{WD}\}\mid \{\mathrm{W}\})\right|$&$0{,}0098$\\
%\hline
$\left|\mathbb{P}(\{\mathrm{DWW}\}\mid \{\mathrm{DW}\})-\mathbb{P}
(\{\mathrm{WW}\}\mid \{\mathrm{W}\})\right|$&$0{,}0098$\\
%\hline
$\left|\mathbb{P}(\{\mathrm{WDD}\}\mid \{\mathrm{DD}\})-\mathbb{P}
(\{\mathrm{DD}\}\mid \{\mathrm{D}\})\right|$&$0{,}827$\hphantom{9}\\
%\hline
$\left|\mathbb{P}(\{\mathrm{WDW}\}\mid \{\mathrm{WD}\})-\mathbb{P}
(\{\mathrm{DW}\}\mid \{\mathrm{D}\})\right|$&$0{,}827$\hphantom{9}\\
%\hline
$\left|\mathbb{P}(\{\mathrm{WWD}\}\mid \mathrm{WW})-\mathbb{P}
(\{\mathrm{WD}\}\mid \{\mathrm{W}\})\right|$&$0{,}0109$\\
%\hline
$\left|\mathbb{P}(\{\mathrm{WWW}\}\mid \{\mathrm{WW}\})-\mathbb{P}
(\{\mathrm{WW}\}\mid \{\mathrm{W}\})\right|$&$0{,}0109$\\
\hline
\end{tabular}
\end{center}
\end{table*}


\vspace*{-6pt}


\section{Проверка марковского свойства}

\vspace*{-2pt}

В большинстве работ, посвященных статистическому
анализу метеорологических данных, считается, что продолжительность периода
выпадения осадков, измеренная в~сутках (т.\,е.\ число последовательных
<<дождливых>> дней), подчиняется гео\-мет\-ри\-че\-ско\-му распределению
вероятностей (см., например,~\cite{Zolina2013}). Возможно, данные предположения
базируются на классической интерпретации гео\-мет\-ри\-че\-ско\-го распределения 
в~терминах испытаний Бернулли как распределения числа последовательных
<<дождливых>> дней (<<успех>>) до первого дня без осадков (<<неудача>>). Для
изучаемых в~работе городов проверим более слабое предположение, а~именно:
наличие марковости для данных.



Для этого потребуется вычисление условных вероятностей, но, как было отмечено
в предыдущем разделе, необходимые базовые величины определены ранее,
поэтому с~помощью классической формулы
\begin{equation*}
P(A|B)=\fr{P(AB)}{P(B)}
\end{equation*}
могут быть получены требуемые значения. В~табл.~\ref{TabProbPotsdam}
и~\ref{TabProbElista} представлены условные вероятности и~модули их разностей
для Потсдама и~Элисты, демонстрирующие отсутствие свойства марковости 
у~данных.



Таким образом, последовательность <<дождливых>> и~<<сухих>> дней
не является марковской, поэтому использование схемы испытаний Бернулли
некорректно. Альтернативные вероятностные модели предложены в~статьях~\cite{Gorshenin2017a,Gorshenin2017b}.

\vspace*{-4pt}

\section{Вероятностное прогнозирование осадков 
с~использованием исторических паттернов}

\vspace*{-2pt}

Паттерны в~анализе данных могут быть использованы для повышения точности 
и~скорости построения прогнозов (см., например, статью~\cite{Gould2008}).\linebreak
 Кроме
того, они являются достаточно распространенным инструментом в~рамках
решения различных климатологических задач (см., например,\linebreak
работы~\cite{Abaurrea2005,Stopa2013}). В~данном разделе будет реализована
достаточно простая схема построения вероятностных прогнозов для
последовательностей <<сухих>> и~<<дождливых>> дней на основе исторических
паттернов для маркированных данных (<<0--1>> или <<D--W>>).

При определении частот (вероятностей) появления той или иной
последовательности событий по дням в~предыдущем разделе были вычислены
соответствующие величины для наборов различной длины. Используя эти данные,
можно вычислять значения условных вероятностей появления в~будущем
определенных комбинаций, т.\,е.\ построения прогноза. При этом, в~отличие от
стандартной для анализа данных практики, когда пред\-ска\-зы\-ва\-емое окно не должно
превышать размер входных наблюдений, для исторических значений это правило
может нарушаться.

%\vspace*{6pt}

\begin{figure*}[b] %fig3
\vspace*{1pt}
 \begin{center}
 \mbox{%
 \epsfxsize=163.425mm 
 \epsfbox{gor-3.eps}
 }
\end{center}
\vspace*{-11pt}
\Caption{Точность построения прогноза по месяцам в~процессе обучения
(пример): (\textit{а})~однодневный прогноз;
(\textit{б})~двухдневный прогноз.
Графики справа соответствуют модели с~учетом сезонности
\label{FigNN}}
\vspace*{-6pt}
\end{figure*}
 


В качестве примера рассмотрим построение вероятностного прогноза на два
следующих дня для Потсдама и~Элисты при условии текущих наблюдений
вида <<\verb"Wet-Wet-Dry-Dry">>, т.\,е.\ два дня подряд выпали осадки, в~следующие
двое суток они не регистрировались. В~табл.~5 
и~6 представлены вероятности соответствующих событий
(полужирным шрифтом выделено наиболее вероятное событие).\linebreak\vspace*{-12pt}

\vspace*{8pt}

%tabl5
\begin{center}
\parbox{68mm}{{{\tablename~5}\ \ \small{Пример: прогнозирование осадков, Потсдам}}}

\vspace*{2ex}

{\small
\begin{tabular}{|c|c|}
\hline
{Прогноз на 2 следующих дня}&{Вероятность}\\
\hline
\verb"Dry-Dry"&$\bf 0{,}4828$\\
%\hline
\verb"Dry-Wet"&$0{,}1909$\\
%\hline
\verb"Wet-Dry"&$0{,}1211$\\
%\hline
\verb"Wet-Wet"&$0{,}2053$\\
\hline
\end{tabular}
}
\vspace*{10pt}

%\begin{center}
\parbox{68mm}{{{\tablename~6}\ \ \small{Пример: прогнозирование осадков, Элис\-та}}}

\vspace*{2ex}

{\small
\begin{tabular}{|c|c|}
\hline
{Прогноз на 2 следующих дня}&{Вероятность}\\
\hline
\verb"Dry-Dry"&$\bf 0{,}5852$\\
%\hline
\verb"Dry-Wet"&$0{,}1641$\\
%\hline
\verb"Wet-Dry"&$0{,}1259$\\
%\hline
\verb"Wet-Wet"&$0{,}1249$\\
\hline
\end{tabular}
}
\vspace*{2pt}
%\end{center}
\end{center}

\noindent
 Соответственно,
можно формулировать утверждения следующего вида: <<Вероятность осадков
\textit{через}~2~дня в~Потсдаме при текущих наблюдениях
\verb"Wet-Wet-Dry-Dry" составляет~$0{,}3961$, а вероятность отсутствия осадков
\textit{через}~2~дня~--- $0{,}6039$>>; <<Вероятность осадков \textit{через}~2~дня 
в~Элисте при текущих наблюдениях \verb"Wet-Wet-Dry-Dry" составляет~$0{,}2889$, 
а~вероятность отсутствия осадков \textit{через}~2~дня~--- $0{,}7111$>>.



Стоит отметить, что поступление новых данных может существенным образом
изменить вероятности событий только при значительном объеме новых
наблюдений, что позволяет говорить об устойчивости прогнозов. При этом с~сугубо
вычислительной точки зрения обновление данных не является трудоемкой
задачей.

\section{Построение прогнозов с~помощью нейронных сетей}

В данном разделе рассмотрим технологию прогнозирования событий, связанных 
с~выпадением осадков, на основе использования нейронных сетей. При этом 
в~качестве обучающих рядов задействуются те же самые паттерны, о~которых шла
речь выше. Однако в~явном виде частота каждого из наборов не используется, 
а~соответствующие процедуры реализуются в~скрытых слоях нейронной сети.

Было проведено исследование возможности построения прогностической модели
для данных с~помощью многослойного перцептрона на основе биб\-лио\-те\-ки
\verb"PHP-ML" для языка программирования \verb"PHP"
({\sf https://github.com/php-ai/php-ml}). Исходные наблюдения случайным образом
разделялись на обуча\-ющие и~тестовые выборки в~отношении~$7:3$. В~качестве
результата работы нейронной сети получается прогноз на следующие за входным
набором~1--2~дня.

Тестировались нейронные сети с~одним и~двумя скрытыми слоями в~архитектуре,
при этом достигнутая точность составила~73\% (в~среднем) для однодневного
прогноза при объеме входной выборки в~7~элементов (дней) и~56\% для
двухдневного прогноза при размере входных данных в~10~наблюдений. Затем
модель была усложнена за счет учета сезонности как отдельной компоненты 
(к~выборке каждый раз добавлялось еще одно наблюдение с~номером месяца,
соответствующего последнему наблюдению). Такое решение позволило повысить
точность прогнозирования до~82\% для однодневного прогноза и~74\% для
двухдневного при сохранении тех же параметров входных данных. Добавление
третьего скрытого слоя не привело к~заметному улучшению  качества прогноза,
при этом скорость обучения снизилась, поэтому разумно отказаться от такого
решения в~данной конфигурации нейросети.

На рис.~\ref{FigNN} продемонстрированы графики точ\-ности прогнозов 
нейронных сетей по месяцам, построенные по выборке из~7~наблюдений 
как с~учетом сезонности (правый столбец), так и~без (левый столбец). 
На рис.~3,\,\textit{а} приводится однодневный прогноз, на рис.~3,\,\textit{б}~--- 
двухдневный. Для расчетов в~данном случае использовался укороченный набор 
данных за четырехлетний период для уменьшения времени, необходимого 
для корректной настройки нейросети.


Как видно из рис.~\ref{FigNN}, введение в~модель се\-зон\-ности 
повышает качество прогнозирования, особенно для случая двухдневного 
предсказания. Например, на рис.~3,\,\textit{б} (левый столбец)
в~феврале \mbox{точ\-ность} 
со\-став\-ля\-ла около~26\%, а~с~использованием сезонности удалось добиться 
увеличения данного показателя в~2~раза~--- до~52\% (см.\ рис.~3,\,\textit{б},
правый столбец).

Очевидно, что качество прогноза может быть повышено за счет увеличения
объемов тестовых данных, использования высокопроизводительных решений,
развития архитектуры нейросети и~т.\,д. Представленные в~данном разделе
результаты демонстрируют успешность применения инструментария нейросетей 
в~решении задачи прогнозирования в~указанной постановке, однако не являются
окончательными по своей точности. Подробнее вопрос получения высокоточных 
прогнозов будет изучаться в~дальнейших работах.

\vspace*{-9pt}

\section{Заключение}

\vspace*{-2pt}

В работе продемонстрировано нарушение марковского свойства для осадков,
наблюдаемых в~существенно различающихся между собой климатических
областях~--- в~Потсдаме и~Элисте. Такие сведения о данных, наряду 
с~исследованными в~работе~\cite{Gorshenin2017b} статистическими свойствами,
представляют базовую информацию, необходимую для дальнейшего корректного
построения вероятностных моделей, в~част\-ности для распределений объемов
экстремальных осадков~\cite{Gorshenin2017a}.

Для анализа вероятностного поведения процесса выпадения осадков 
и~построения прогнозов
предложено использование цепочек событий (паттернов), выделенных из данных.
При этом статистические процедуры автоматизированы с~использованием
программных инструментов пакета \verb"MATLAB". В~качестве альтернативного
инструмента прогнозирования на основе паттернов были использованы нейронные
сети с~учетом сезонности, двумя скрытыми слоями нейронов и~сигмоидной
функцией активации, реализованные с~помощью средств языка \verb"PHP".

В качестве одного из направлений дальнейших исследований можно предложить
переход от двоичной модели дискретизации событий к~$k$-ичной. Это позволит
решать более сложные задачи вероятностного прогнозирования, например можно
предсказывать величину осадков в~терминах попадания в~тот или иной диапазон
(соответствующие интервалы формируются с~учетом значений квантилей
распределения для объемов, см., например,
\mbox{статьи}~\cite{Zolina2009,Gorshenin2017c}).

Работа с~паттернами представляет интерес с~точки зрения верификации
ансамблей прогнозов~--- например, европейские климатологические агентства
достаточно точно предсказывают общий объем осадков, который выпадет за
некоторый период, но остается актуальной задача определения структуры\linebreak его
распределения по дням. Кроме того, данная методология может быть 
использована для прогнозирования поведения моментных 
характеристик (мате\-ма\-ти\-че\-ское ожидание, дисперсия, коэффициенты 
ассиметрии и~эксцесса) конечных смесей вероятностных 
распределений~\cite{Gorshenin2016} для определения направления 
изменения тренда (аналогия со случайным блужданием), 
например в~рамках моделирования физических~\cite{Gorshenin2011} 
и~иных процессов.

Методы прогнозирования на основе нейронных сетей вполне могут быть включены
в~состав комплекса стохастического анализа
данных~\cite{Gorshenin2015,Gorshenin2017d} как отдельный элемент. При этом
наибольший интерес представляет использование алгоритмов EM (expectation-maximization)
ти\-па, прежде
всего сеточных~\cite{Gorshenin2013a,Gorshenin2013b}, в~процессе обучения. 

Для
классического EM-ал\-го\-рит\-ма ранее продемонстрированы определенные успехи
(см., например, статьи~\cite{Freitas2000,Ng2004}) в~рамках такого использования,
кроме того, установлена взаимосвязь обобщенного EM-ал\-го\-рит\-ма с~методами
обуче\-ния нейронных сетей~\cite{Audhkhasi2016}, однако для сеточных
модификаций результаты (как теоретические, так и~эмпирические) отсутствуют.

Стоит отметить, что расширение функциональных возможностей упомянутого
комплекса стохастического анализа данных с~точки зрения включения 
в~пользовательский интерфейс~\cite{Gorshenin2016} новых\linebreak методов предусмотрено
за счет реализации в~ее рамках специализированного
фреймворка~\cite{Gorshenin2017Soft}.

\smallskip

{Автор выражает признательность чле\-ну-кор\-рес\-пон\-ден\-ту РАН, доктору
фи\-зи\-ко-ма\-те\-ма\-ти\-че\-ских наук,
профессору Сергею Константиновичу Гулеву за предоставленные данные,
доктору
фи\-зи\-ко-ма\-те\-ма\-ти\-че\-ских наук, профессору Виктору Юрьевичу Королеву за полезные обсуждения 
в~рамках совместных исследований метеорологических явлений и~Виктору Кузьмину
за помощь в~реализации обучения нейронных сетей.}

\vspace*{-4pt}

{\small\frenchspacing
 {%\baselineskip=10.8pt
 \addcontentsline{toc}{section}{References}
 \begin{thebibliography}{99}
\bibitem{Strauch2012} 
\Au{Strauch~M., Bernhofer~C., Koide~S., Volk~M., Lorz~C.,
Makeschin~F.} Using precipitation data ensemble for uncertainty analysis in SWAT
streamflow simulation~// J.~Hydrol., 2012. Vol.~414. P.~413--424.

\bibitem{Krzysztofowicz2001} 
\Au{Krzysztofowicz~R.} The case for probabilistic
forecasting in hydrology~// J.~Hydrol., 2001. Vol.~249. Iss.~1-4. P.~2--9.

\bibitem{Pinson2009} 
\Au{Pinson~P., Madsen~H., Nielsen~H.\,A., Papaefthymiou~G.,
Klockl~B.} From probabilistic forecasts to statistical scenarios of short-term wind
power production~// Wind Energy, 2009. Vol.~12. Iss.~1. P.~51--62.

\bibitem{Gneiting2007} 
\Au{Gneiting~T., Balabdaoui~F., Raftery~A.\,E.} Probabilistic
forecasts, calibration and sharpness~// J.~Roy. Stat. Soc. B, 2007. Vol.~69. P.~243--268.

\bibitem{Kharin2007} 
\Au{Kharin~V.\,V., Zwiers~F.\,W., Zhang~X., Hegerl~G.\,C.}
Changes in temperature and precipitation extremes in the IPCC ensemble of global
coupled model simulations~// J.~Climate, 2007. Vol.~20. Iss.~8. P.~1419--1444.

\bibitem{Zolina2005}  %6
\Au{Zolina~O., Simmer~C., Kapala~A., Gulev~S.\,K.} On the
robustness of the estimates of centennial-scale variability in heavy precipitation from
station data over Europe~// Geophys. Res. Lett., 2005. Vol.~32. P.~L14707-1--L14707-5.

\bibitem{Gorshenin2017a} %7
\Au{Королев~В.\,Ю., Горшенин~А.\,К.} О~распределении
вероятностей экстремальных осадков~// Докл. РАН, 2017. Т.~477.
Вып.~5. C.~604--609.

\bibitem{Zolina2013}  %8
\Au{Zolina~O., Simmer~C., Belyaev~K., Gulev~S., Koltermann~P.}
Changes in the duration of European wet and dry spells during the last~60~years~// 
J.~Climate, 2013. Vol.~26. P.~2022--2047.



\bibitem{Gorshenin2017b}  %9
\Au{Korolev~V.\,Yu., Gorshenin~A.\,K., Gulev~S.\,K.,
Belyaev~K.\,P., Grusho~A.\,A.} Statistical analysis of precipitation events~// AIP
Conf. Proc., 2017. Vol.~1863. P.~090011-1--090011-4.

\bibitem{Gould2008}
\Au{Gould~P.\,G., Koehler~A.\,B., Ord~J.\,K., Snyder~R.\,D.,
Hyndman~R.\,J., Vahid-Araghi~F.} Forecasting time series with multiple seasonal
patterns~// Eur. J.~Oper. Res., 2008. Vol.~191. Iss.~1.
P.~207--222.

\bibitem{Abaurrea2005} 
\Au{Abaurrea~J.} Forecasting local daily precipitation patterns
in a~climate change scenario~// Clim. Res., 2005. Vol.~28. Iss.~3. P.~183--197.

\bibitem{Stopa2013} 
\Au{Stopa~J.\,E., Cheung~K.\,F., Tolman~H.\,L., Chawla~A.}
Patterns and cycles in the Climate Forecast System Reanalysis wind and wave data~//
Ocean Model., 2013. Vol.~70. P.~207--220.

\bibitem{Zolina2009} 
\Au{Zolina~O., Simmer~C., Belyaev~K., Kapala~A., Gulev~S.\,K.}
Improving estimates of heavy and extreme precipitation using daily records from
European rain gauges~// J.~Hydrometeorol., 2009. Vol.~10. P.~701--716.

\bibitem{Gorshenin2017c} 
\Au{Горшенин~А.\,К.} О~некоторых математических 
и~программных методах построения структурных моделей информационных
потоков~// Информатика и~её применения, 2017. Т.~11. Вып.~1. C.~58--68.

\bibitem{Gorshenin2016} 
\Au{Gorshenin~A., Kuzmin~V.} On an interface of the online
system for a~stochastic analysis of the varied information flows~// AIP Conf.
Proc., 2016. Vol.~1738. P.~220009-1--220009-4.

\bibitem{Gorshenin2011}
\Au{Батанов~Г.\,М., Горшенин~А.\,К.,
Королев~В.\,Ю., Малахов~Д.\,В., Скворцова~Н.\,Н.} Эволюция
вероятностных характеристик низкочастотной турбулентности плазмы 
в~микроволновом поле~// Математическое моделирование, 2011.
Т.~23. №\,5. C.~35--55.

\bibitem{Gorshenin2015} 
\Au{Gorshenin~A., Kuzmin~V.} Online system for the
construction of structural models of information flows~// 7th
 Congress (International) on Ultra Modern Telecommunications and Control Systems and
Workshops Proceedings.~--- Piscataway, NJ, USA: IEEE, 2015. P.~216--219.

\bibitem{Gorshenin2017d} 
\Au{Gorshenin~A.\,K., Kuzmin~V.\,Yu.} Research support
system for stochastic data processing~// Pattern Recogn. Image Anal., 2017.
Vol.~27. No.\,3. P.~518--524.



\bibitem{Gorshenin2013b} %19
\Au{Gorshenin~A., Korolev~V., Kuzmin~V., Zeifman~A.}
Coordinate-wise versions of the grid method for the analysis of intensities of
non-stationary information flows by moving separation of mixtures of
gamma-distribution~// 27th European Conference on Modelling and
Simulation Proceedings.~--- Dudweiler, Germany: Digitaldruck Pirrot GmbHP, 2013. P.~565--568.

\bibitem{Gorshenin2013a}  %20
\Au{Gorshenin~A., Korolev~V.} Modelling of statistical
fluctuations of information flows by mixtures of gamma distributions~// 
27th European Conference on Modelling and Simulation Proceedings.~--- 
Dudweiler, Germany: Digitaldruck Pirrot GmbHP, 2013. P.~569--572.

\bibitem{Freitas2000} 
\Au{De Freitas~J., Niranjan~M., Gee~A.} Dynamic learning with
the EM algorithm for neural networks~// J.~VLSI Sig. Proc. Syst., 
2000. Vol.~26. Iss.~1-2. P.~119--131.

\bibitem{Ng2004} 
\Au{Ng~S.\,K., McLachlan~G.\,J.} Using the EM algorithm to train
neural networks: Misconceptions and a~new algorithm for multiclass classification~// 
IEEE T.~Neural Networ., 2004. Vol.~15. Iss.~3. P.~738--749.

\bibitem{Audhkhasi2016} 
\Au{Audhkhasi~K., Osoba~O., Kosko~B.} Noise-enhanced
convolutional neural networks~// Neural Networks, 2016. Vol.~78. P.~15--23.

\bibitem{Gorshenin2017Soft} 
\Au{Горшенин~А.\,К., Кузьмин~В.\,Ю.} Фреймворк
вычислительной части системы поддержки научных исследований. Свидетельство
о~государственной регистрации программ для ЭВМ №\,2017617610 от 11.07.2017.
 \end{thebibliography}

 }
 }

\end{multicols}

\vspace*{-6pt}

\hfill{\small\textit{Поступила в~редакцию 20.10.17}}

%\vspace*{8pt}

\newpage

\vspace*{-24pt}

%\hrule

%\vspace*{2pt}

%\hrule

%\vspace*{8pt}


\def\tit{PATTERN-BASED ANALYSIS OF~PROBABILISTIC AND~STATISTICAL CHARACTERISTICS 
OF~EXTREME PRECIPITATION}

\def\titkol{Pattern-based analysis of~probabilistic and~statistical characteristics 
of~extreme precipitation}

\def\aut{A.\,K.~Gorshenin$^{1,2}$}

\def\autkol{A.\,K.~Gorshenin}

\titel{\tit}{\aut}{\autkol}{\titkol}

\vspace*{-9pt}


\noindent
$^1$Institute of Informatics Problems, Federal Research Center ``Computer Science and
Control'' of the Russian\linebreak
$\hphantom{^1}$Academy of Sciences, 44-2~Vavilova Str., Moscow 119333,
Russian Federation

\noindent
$^2$P.\,P.~Shirshov Institute of Oceanology of the Russian Academy of Sciences,
36~Nakhimovski Prosp., Moscow\linebreak
$\hphantom{^1}$117997, Russian Federation


\def\leftfootline{\small{\textbf{\thepage}
\hfill INFORMATIKA I EE PRIMENENIYA~--- INFORMATICS AND
APPLICATIONS\ \ \ 2017\ \ \ volume~11\ \ \ issue\ 4}
}%
 \def\rightfootline{\small{INFORMATIKA I EE PRIMENENIYA~---
INFORMATICS AND APPLICATIONS\ \ \ 2017\ \ \ volume~11\ \ \ issue\ 4
\hfill \textbf{\thepage}}}

\vspace*{3pt}



\Abste{Precipitations are the key parameters of hydrological models; so, research
related to precipitation processes is necessary for solving various applied problems.
 The
paper demonstrates a~violation of the Markov property for precipitation observed in
essentially different climatic regions~--- in the cities of Potsdam and Elista. Such
information about the data, along with previously studied properties, 
represents the basic
information which is necessary for the further correct construction of 
probabilistic models,
in particular, for probability distribution of the volumes of extreme 
precipitation. For the
analysis of the probabilistic behavior of the precipitation process and 
the construction of
forecasts, it is suggested to use chains of events (patterns) extracted from the data. 
At the same time, statistical procedures are automated using the software 
tools of the
{\sf MATLAB} package. Neural networks were used as an alternative forecasting tool
based on patterns, and the best results were demonstrated via the 
architecture that takes
into account a seasonality, has two hidden layers of neurons and a~sigmoid activation
function. The ideas for further research in this field are suggested.}

\KWE{precipitations; patterns; forecast; neural networks; probabilistic 
forecasting; Markov property}

  \DOI{10.14357/19922264170405} 

%\vspace*{-12pt}

\Ack
\noindent
The research was supported by the Russian Foundation for Basic Research (projects
17-07-00851, 15-07-04040, and 15-07-05316).


%\vspace*{3pt}

  \begin{multicols}{2}

\renewcommand{\bibname}{\protect\rmfamily References}
%\renewcommand{\bibname}{\large\protect\rm References}

{\small\frenchspacing
 {%\baselineskip=10.8pt
 \addcontentsline{toc}{section}{References}
 \begin{thebibliography}{99}


\bibitem{1-gor}
\Aue{Strauch,~M., C.~Bernhofer, S.~Koide, M.~Volk, C.~Lorz, and F.~Makeschin.} 2012.
Using precipitation data ensemble for uncertainty analysis in SWAT streamflow
simulation. \textit{J.~Hydrol.} 414:413--424.

\bibitem{2-gor}
\Aue{Krzysztofowicz,~R.} 2001. The case for probabilistic forecasting in hydrology. 
\textit{J.~Hydrol.} 249(1-4):2--9.

\bibitem{3-gor}
\Aue{Pinson,~P., H.~Madsen, H.\,A.~Nielsen, G.~Papaefthymiou, and B.~Klockl.} 2009.
From probabilistic forecasts to statistical scenarios of short-term wind power
production. \textit{Wind Energy} 12(1):51--62.
\bibitem{4-gor}
\Aue{Gneiting,~T., F.~Balabdaoui, and A.\,E.~Raftery.} 2007. Probabilistic forecasts,
calibration and sharpness.
\textit{J.~Roy. Stat. Soc.~B} 69:243--268.
\bibitem{5-gor}
\Aue{Kharin,~V.\,V., F.\,W.~Zwiers, X.~Zhang, and G.\,C.~Hegerl.} 2007. Changes in
temperature and precipitation extremes in the IPCC ensemble of global coupled model
simulations. \textit{J.~Climate} 20(8):1419--1444.

\bibitem{6-gor}
\Aue{Zolina,~O., C.~Simmer, A.~Kapala, and S.\,K.~Gulev.} 
2005. On the robustness of the
estimates of centennial-scale variability in heavy precipitation from station data over
Europe. \textit{Geophys. Res. Lett.} 32:L14707-1--L14707-5.

\bibitem{8-gor} %7
\Aue{Korolev,~V.\,Yu., and A.\,K.~Gorshenin.} 2017. The probability distribution of extreme precipitation. 
\textit{Dokl. Earth Sci.} 477(2):1461--1466.

\bibitem{7-gor} %8
\Aue{Zolina,~O., C.~Simmer, K.~Belyaev, S.~Gulev, and P.~Koltermann.} 2013.
Changes in the duration of European wet and dry spells during the last~60~years. 
\textit{J.~Climate} 26:2022--2047.



\bibitem{9-gor}
\Aue{Korolev,~V.\,Yu., A.\,K.~Gorshenin, S.\,K.~Gulev, K.\,P.~Belyaev, and 
A.\,A.~Grusho.}
2017. Statistical analysis of precipitation events. 
\textit{AIP Conf. Proc.} 1863:090011-1--090011-4.

\bibitem{10-gor}
\Aue{Gould,~P.\,G., A.\,B.~Koehler, J.\,K.~Ord, R.\,D.~Snyder, R.\,J.~Hyndman, and
F.~Vahid-Araghi.} 2008. Forecasting time series with multiple seasonal patterns.  
\textit{European J.~Oper. Res.} 191(1):207--222.

\bibitem{11-gor}
\Aue{Abaurrea,~J.} 2005. Forecasting local daily precipitation patterns in 
a~climate change scenario. \textit{Clim. Res.} 28(3):183--197.

\bibitem{12-gor}
\Aue{Stopa,~J.\,E., K.\,F.~Cheung, H.\,L.~Tolman, and A.~Chawla.} 
2013. Patterns and cycles
in the Climate Forecast System Reanalysis wind and wave data. 
\textit{Ocean Model.} 70:207--220.

\bibitem{13-gor}
\Aue{Zolina,~O., C.~Simmer, K.~Belyaev, A.~Kapala, and S.\,K.~Gulev.} 2009. Improving
estimates of heavy and extreme precipitation using daily records from European rain
gauges. \textit{J.~Hydrometeorol.} 10:701--716.

\bibitem{14-gor}
\Aue{Gorshenin,~A.\,K.} 2017. O~nekotorykh matematicheskikh i~programmnykh metodakh
postroeniya strukturnykh modeley informatsionnykh potokov [On some mathematical and
programming methods for construction of structural models of information flows]. 
\textit{Informatika i~ee Primeneniya~---  Inform. Appl.} 11(1):58--68.

\bibitem{15-gor}
\Aue{Gorshenin,~A., and V.~Kuzmin.}  2016. 
On an interface of the online system for a
stochastic analysis of the varied information flows. 
\textit{AIP Conf. Proc.}
1738:220009-1--\mbox{220009-4}.

\bibitem{16-gor}
\Aue{Batanov,~G.\,M., A.\,K.~Gorshenin, V.\,Yu.~Korolev, D.\,V.~Malakhov, and
N.\,N.~Skvortsova.} 2012. The evolution of probability characteristics of
low-frequency plasma turbulence. \textit{Math. Models Computer
Simulations} 4(1):10--25.

\bibitem{17-gor}
\Aue{Gorshenin,~A.\,K., and V.~Kuzmin.} 2015. Online system for the construction of
structural models of information flows. \textit{7th 
Congress (International) on Ultra Modern Telecommunications and Control Systems and Workshops
Proceedings}. Piscataway, NJ: IEEE. 216--219.

\bibitem{18-gor}
\Aue{Gorshenin,~A.\,K., and V.\,Yu.~Kuzmin.} 2017. 
Research support system for stochastic
data processing. \textit{Pattern Recogn. Image Anal.} 27(3):518--524.



\bibitem{20-gor}
\Aue{Gorshenin,~A.\,K., V.~Korolev, V.~Kuzmin, and A.~Zeifman.} 2013.  Coordinate-wise
versions of the grid method for the analysis of intensities of non-stationary information
flows by moving separation of mixtures of gamma-distribution.  
\textit{27th European Conference on Modelling and Simulation
Proceedings}. Dudweiler, Germany: Digitaldruck Pirrot GmbHP. 565--568.

\bibitem{19-gor}
\Aue{Gorshenin~A.\,K., and V.~Korolev.} 2013.  
Modelling of statistical fluctuations of
information flows by mixtures of gamma distributions. 
\textit{27th
European Conference on Modelling and Simulation Proceedings}.
Dudweiler, Germany: Digitaldruck Pirrot GmbHP. 569--572.

\bibitem{21-gor}
\Aue{De Freitas,~J., M.~Niranjan, and A.~Gee.} 2000. Dynamic learning with the EM
algorithm for neural networks. \textit{J.~VLSI Sig. Proc. Syst.} 26(1-2):119--131.

\bibitem{22-gor}
\Aue{Ng,~S.\,K., and G.\,J.~McLachlan.} 2004. Using the EM algorithm to train neural
networks: Misconceptions and a~new algorithm for multiclass classification. 
\textit{IEEE T.~Neural Networ.} 15(3):738--749.

\bibitem{23-gor}
\Aue{Audhkhasi,~K., O.~Osoba, and B.~Kosko.} 2016. Noise-enhanced convolutional
neural networks. \textit{Neural Networks} 78:15--23.

\bibitem{24-gor}
\Aue{Gorshenin,~A.\,K., and V.\,Yu.~Kuzmin.} 2017. 
Freymvork vychislitel'noy chasti sistemy
podderzhki nauchnykh issledovaniy [Framework for computational part of  scientific
research support system]. Certificate RF of state registration of computer programs
No.\,2017617610.
\end{thebibliography}

 }
 }

\end{multicols}

\vspace*{-6pt}

\hfill{\small\textit{Received October 20, 2017}}

%\vspace*{-10pt}


\Contrl

\noindent
\textbf{Gorshenin Andrey K.} (b.\ 1986)~--- Candidate of Science (PhD) in physics and
mathematics, associate professor, leading scientist, Institute of Informatics Problems,
Federal Research Center ``Computer Science and Control'' of the Russian Academy of
Sciences, 44-2~Vavilova Str., Moscow 119333, Russian Federation;
senior scientist, P.\,P.~Shirshov Institute of Oceanology of the Russian Academy of Sciences,
36~Nakhimovski Prosp., Moscow 117997, Russian Federation
\mbox{agorshenin@frccsc.ru}

\label{end\stat}


\renewcommand{\bibname}{\protect\rm Литература} 