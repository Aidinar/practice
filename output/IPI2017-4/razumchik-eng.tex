\def\bmx{\mathbf}

\renewcommand{\figurename}{\protect\bf Figure}
\renewcommand{\tablename}{\protect\bf Table}

\def\stat{raz-1}

\def\tit{STATIONARY SOJOURN TIMES 
IN~$\mathrm{MAP}/\mathrm{PH}/1/r$ QUEUE WITH~BI-LEVEL HYSTERETIC CONTROL OF~ARRIVALS}

\def\titkol{Stationary sojourn times 
in~$\mathrm{MAP}/\mathrm{PH}/1/r$ queue with~bi-level hysteretic control of~arrivals}

\def\autkol{R.\,V.~Razumchik}

\def\aut{R.\,V.~Razumchik$^{1,2}$}

\titel{\tit}{\aut}{\autkol}{\titkol}

\index{Razumchik R.\,V.}
\index{Разумчик Р.\,В.}

%{\renewcommand{\thefootnote}{\fnsymbol{footnote}}
%\footnotetext[1] {This work was supported in part by the
%Russian Foundation for Basic Research (grants 15-07-03007 and 13-07-00223).}}

\renewcommand{\thefootnote}{\arabic{footnote}}
\footnotetext[1]{Institute of Informatics Problems, Federal Research 
Center ``Computer Science and Control'' of the Russian Academy of Sciences, 44-2~Vavilov Str., 
Moscow 119333, Russian Federation} 
\footnotetext[2]{Peoples' Friendship University of Russia (RUDN University), 6~Miklukho-Maklaya Str., 
Moscow 117198, Russian Federation}


%\vspace*{-6pt}

\def\leftfootline{\small{\textbf{\thepage}
\hfill INFORMATIKA I EE PRIMENENIYA~--- INFORMATICS AND APPLICATIONS\ \ \ 2017\ \ \ volume~11\ \ \ issue\ 4}
}%
 \def\rightfootline{\small{INFORMATIKA I EE PRIMENENIYA~--- INFORMATICS AND APPLICATIONS\ \ \ 2017\ \ \ volume~11\ \ \ issue\ 4
\hfill \textbf{\thepage}}}



%\def\l{\lambda}
%\def\m{\mu}
%\def\a{\alpha}
%\def\b{\beta}
%\def\g{\gamma}
%\def\d{\delta}
%\def\r{\overline r}
%\def\f{\overline f}
%\def\n{\nu}



\Abste{This paper reports some new results concerning the 
analysis of the time-related stationary characteristics
of a~finite-capacity queueing system 
operating in a~random environment
with the bi-level hysteretic control of arrivals.
The topic of the paper is motivated by 
the overload problem in networks of SIP (session initiation protocol) servers 
and the viewpoint that multilevel hysteretic control 
of arrivals in SIP servers
can be used to mitigate signalling network congestion.
The considered mathematical model of SIP server is the 
single server queueing system with Markovian arrival processes (MAP), PH (phase-type)
service,
and bi-level hysteretic control policy.
According to this policy, a~system may be in one
of the three operation modes: normal, overload, or blocking. 
The switching between modes occurs at 
instants whenever the total number of customers in the 
system changes.
The analytical method for the computation of the 
stationary sojourn times in different operation modes (in terms of 
Laplace--Stieltjes transforms (LST)), 
which utilizes the knowledge about the presence of 
hysteretic loops, is given. It is also applicable 
in the case when, in addition to the sojourn times,
one needs to account for the number of lost customers.}


\KWE{queueing system; random environment; first passage times; hysteretic control}

\DOI{10.14357/19922264170403}

%\vspace*{9pt}


\vskip 12pt plus 9pt minus 6pt

      \thispagestyle{myheadings}

      \begin{multicols}{2}

                  \label{st\stat}


\section{Introduction}

\noindent
This paper continues the analysis 
of the stationary finite-capacity queueing system 
operating in a~random environment
with hysteretic control of arrivals, 
which was started in~\cite{int0}.
Specifically, we deal with the $\mathrm{MAP}/\mathrm{PH}/1/r$ queue 
with two-level hysteretic control of 
arrival rates with nonoverlapping hysteretic
loops. For this system, the authors 
of~\cite{int0} proposed the 
new analytic method for the computation 
of the steady-state distribution, which
is different from the known general approaches 
for QBDs (quasi-birth-deathes). It exploits the knowledge about the hysteretic loops 
which are present in the system, has a~probabilistic interpretation
and leads to easy-to-implement computational procedures.  
We will not dwell on the motivation behind the 
analysis of this system (for details, refer~\cite{int0} and references therein)
and just mention that the various aspects of the 
topic of hysteretic control in 
queueing models still gains attention from the research 
community (see~\cite{int2, int3, int4}).

In order to make the picture clearer, let us assume 
that the control is only bi-level with nonoverlapping 
loops although all the results 
presented here (and in~\cite{int0}) can be 
generalized in a~straightforward manner for 
hysteretic control of arrivals with arbitrary number of 
nonoverlapping loops. 
Following the bi-level hysteretic control, 
the system changes its status (or mode) 
between ``normal,'' ``overload,'' and ``blocking'' 
(this will be made more precise in the next section). In each mode 
except for a~``normal'' one, server discards a~certain percentage 
of arriving customers. From a~practical point of view 
(at least the one mentioned in~\cite{int1}), 
it may be beneficial when the system spends 
as little time as possible in ``overload''/``blocking'' 
modes. This brings one to the analysis of time-related stationary
characteristics of the system, which was not carried out in~\cite{int0}.

In what follows, we are interested in the two performance characteristics 
of the hysteretic policy. The first one is the stationary
distribution of the time system spends in ``normal'' mode. 
The second one is the distribution of the time it takes the system to get back 
to ``normal'' mode\footnote[3]{Waiting and system sojourn time distributions
are of little interest since hysteretic loops have no influence on them in 
case of FIFO service policy (which is assumed).}. 

After giving the detailed system description in 
the next section, in section~3, 
the analytical method for the sequential computation
of these (sojourn time) distributions will be presented
in terms of LST.
The LST of the sojourn times are obtained as solutions to certain matrix difference equations
and are expressed in terms of recurrence relations.
They can be used for direct numerical implementation and 
numerical inversion with well-known methods (Fourier-series method with Euler 
summation, Talbot, etc.).

\begin{figure*}[b] %fig1
\vspace*{1pt}
 \begin{center}
 \mbox{%
 \epsfxsize=123.993mm 
 \epsfbox{raz-1.eps}
 }

\vspace*{6pt}

{\small Sketch of the bi-level hysteretic control of arrivals in the $\mathrm{MAP}/\mathrm{PH}/1/r$ 
system}

 \end{center}
\end{figure*}



\section{System Description and~Preliminaries}

\noindent
The system consists of a~single server and a~queue of finite capacity~$r$.
The arrival process is a~MAP with representation $(\mathbf{D_0}, \mathbf{D_1})$ 
of order~$N$.
Let us assume that an arrival, whenever it occurs, can be of one of the two types, either 
a~priority arrival or a~nonpriority. Thus, the matrix~$\mathbf{D_1}$ is assumed to
have the form $\mathbf{D_1}=\mathbf{D_{1,1}}+\mathbf{D_{1,2}}$
where $\mathbf{D_{1,1}}$ ($\mathbf{D_{1,2}}$) describes state transitions with an 
arrival of
priority (nonpriority) customer. Bi-level hysteretic control of 
arrivals is assumed to be implemented in the system. It operates as follows (see figure).
There are three operation modes for the system: ``normal,'' ``overload,'' 
and ``blocking.''
 Let~$L$ and~$H$ be arbitrary integers,
such that $0 < L < H < r+1$. Assume the system starts empty. 
As long as the total
number of customers in the system remains below~$H$,
the system is considered to be in ``normal`` mode and accepts 
all arrivals (both priority and nonpriority). 
When the total number of customers reaches~$H$ for the first time, the
system changes its mode to ``overload'' and stays in it as long
as the total number of customers remains between~$L$ and~$r$.
When overloaded, the system accepts only priority customers (nonpriority customers 
are lost)
till the total number of customers either
drops down below $L$ after which it changes its mode back to
``normal,'' or exceeds~$r$ after which it changes its state
to ``blocking.'' In the ``blocking'' mode the system does not accept
newly arriving customers until the total number of customers
drops down below $(H+1)$, after which the system changes mode 
back to ``overload`` and the process goes on.
The service time of both priority and
nonpriority customers is PH distributed with representation
$(\vec{f}, \mathbf{G})$ of order~$M$ and $\vec g=-\mathbf{G}\vec1$, and
the service policy is FIFO (first in, first out).



The operation of the considered queueing
system can be completely described
by continuous-time Markov chain 
${\bf{X}} \left( t \right)= {\left( \xi (t); \eta(t); \mu(t); \nu(t) \right) }$ with
four components:
$\xi(t)$~--- MAP generation phase at time~$t$;
$\eta(t)$~--- PH service phase at time~$t$;
$\mu(t)$~--- system's mode at time~$t$;
and $\nu(t)$~--- number of customers in
the system at time~$t$.
When $\nu(t)=0$, the second component $\eta(t)$ is omitted.
It is convenient to represent the state space of ${\bf{X}} \left( t \right)$ as
$\mathcal{X} = {\mathcal{X}_0} \cup {\mathcal{X}_1} \cup {\mathcal{X}_2}$
where $\mathcal{X}_{0} $ is the set of states of ''normal'' mode,
$\mathcal{X}_{1} $ is the set of states of ''overload'' mode,
and $\mathcal{X}_{2}$ is the set of states of ''blocking'' mode, i.\,e.,
\begin{align*}
{\mathcal{X}_0} &= \left\{ \left( {k,0,0} \right):  1 \le k \le N  \right\}
\cup 
\\
&\hspace*{5mm}\cup \left\{ {\left( {k,0,n} \right):  1 \le k \le NM, 1 \le n \le H - 1} \right\}\,;\\
{\mathcal{X}_1} &= \left\{ {\left( {k,1,n}\right): 1 \le k \le NM,  L \le n \le r} \right\}\,;\\
{\mathcal{X}_2} &= \left\{ {\left( {k,2,n} \right): 1 \le k \le NM, 
H\! + \! 1 \le n \le r \! + \! 1} \right\}\,.
\end{align*}

\noindent
Here, $k$ represents the state of 
the background (arrival and service) processes.
Indeed, the state $(k,m,n)$, $n>0$, means that 
there are~$n$ customers in the system, system's mode is~$m$,
and arrival and service phases are~$i$ and~$j$, but such that
$(i-1)M+j=k$; the state $(k,0,0)$ means that the system
is empty and the arrival phase is~$k$.

Let us denote by $\bmx{E}$ the identity matrix (its size each time will 
be clear from the context)
and let introduce the following transition rate matrices:
\begin{itemize}
    \item service of a~customer after which the system becomes empty: 
    $\bmx{P}_1=\bmx{E}\otimes {\vec g}$;
    \item service of a~customer after which the system remains busy: 
    $\bmx{P}=\bmx{P}^*=\bmx{P}^{\#}=\bmx{E}\otimes {\vec g}{\vec f}$;
    \item phase change when system is empty: $\bmx{Q}_0=\bmx{D_0}$;
    \item phase change when system is in the ``normal'' mode: 
    $\bmx{Q}=\bmx{D_0}\otimes \bmx{E} + \bmx{E} \otimes \bmx{G}$;
    \item phase change when system is in the ``overload'' mode: 
    $\bmx{Q}^*=(\bmx{D_0}+\bmx{D_{12}})\otimes \bmx{E} + \bmx{E} \otimes \bmx{G}$;
        \item arrival phase change when system is in the ``blocking'' mode: 
        $\bmx{Q}^{\#}=(\bmx{D_0}+\bmx{D_{1}})\otimes \bmx{E} + \bmx{E} \otimes 
        \bmx{G}$;
    \item arrival of a~customer to an empty system: $\bmx{R}_0\linebreak =\bmx{D_1} \otimes
    {\vec f}$,
    \item arrival of a~customer to the system in the ``normal'' mode: 
    $\bmx{R}=\bmx{D_1}\otimes \bmx{E}$; and
    \item arrival of a~customer to the system in the ``overload'' mode: 
    $\bmx{R}^*=\bmx{D_{11}}\otimes \bmx{E}$.
\end{itemize}

In order to be able to compute time-related characteristics,
in addition to transition rate matrices, one needs
transition probability matrices, which contain probabilities 
of possible state change of the background process.
Thus, let~$\alpha$, $\beta$, and $\gamma$ denote the matrices of
service, phase change, and arrival transition probabilities
when the system is in the ``normal'' mode,
i.\,e.,
\begin{alignat*}{3}
[\alpha]_{ij}  
&=
\fr{[\bmx{P}]_{ij} }{-{[\bmx{Q}]}_{ii}}\,, &\enskip 1 &\le i, j \le NM\,;
\\
[\beta]_{ij}  
&=\begin{cases}
\fr{[\bmx{Q}]_{ij}}{-{[\bmx{Q}]}_{ii}}\,,  & i \neq j\,; \\
0\,, & i = j\,,
\end{cases}
&\enskip 1 &\le i,j \le NM\,;
\\
[\gamma]_{ij}  
&=\fr{[\bmx{R}]_{ij} }{-{[\bmx{Q}]}_{ii}}\,, &\enskip 1 &\le i,j \le NM\,.
\end{alignat*}

\noindent
Here and henceforth, by $[\cdot]_{ij}$ we denote the $(i,j)${th}
 entry of any matrix.
By analogy, let us denote by~$\alpha^*$, $\beta^*$, and~$\gamma^*$
transition probabilities matrices in the ``overload'' mode,
by~$\alpha^{\#}$ and~$\beta^{\#}$
transition probability matrices  
in the ``blocking'' mode,
and by~$\alpha^{e}$, $\beta^{e}$, and~$\gamma^{e}$
transition probability matrices when
the system becomes or is empty.


\section{Sojourn Time Distributions}

\noindent
As it was mentioned in section~1,
we are interested in the two stationary characteristics:
distribution of the time system spends in ``normal'' mode
and the distribution of the time it takes the  system to get back 
to ``normal'' mode. These distributions are computed by 
conditioning on the number of customers in the system
and the state of the background process
and can be expressed in terms of
the following three quantities:
\begin{description}
\item[\,] $\bmx{V}_n(s)$, $n=\overline{0,H-1}$,~---
matrix of size $NM\times NM$,
which the $(i,j)${th} entry is
the LST of the first passage time 
to the ``overload'' mode 
and state of the background process~$j$, 
given that initially, the 
system was in the ``normal'' mode,
there where~$n$~customers in it,
and
the state of the background process was~$i$;
\item[\,] 
$\bmx{V}^{\#}_n(s)$, $n=\overline{H+1,R}$,~---
matrix of size $NM\times NM$,
which the $(i,j)${th} entry is
the LST of the first passage time 
to the ``normal'' mode 
and state of the background process~$j$, 
given that initially, the 
system was in the ``blocking'' mode,
there where~$n$~customers in it,
and
the state of the background process was~$i$; and
\item[\,] 
$\bmx{V}^*_n(s)$, $n=\overline{L,R-1}$,~---
matrix of size $NM\times NM$,
which the $(i,j)${th} entry is
the LST of the first passage time 
to the ``normal'' mode 
and state of the background process~$j$, 
given that initially, the 
system was in the ``overload'' mode,
there where~$n$~customers in it,
and
the state of the background process was~$i$.
\end{description}

The rest of the section is devoted to obtaining the relations for
$\bmx{V}_n(s)$, $\bmx{V}^*_n(s)$, and~$\bmx{V}^{\#}_n(s)$.
Let us begin with the calculation of $\bmx{V}_n(s)$,
$n=\overline{1,H-1}$. 
Denote by 
$\bmx{T}_k(s)$, $k=-1,0,1$, the 
matrix of size $NM \times NM$ 
which the $(i,j)${th} entry is 
LST of the first passage time
from the state $(i,0,n)$ 
to the state $(j,0,n-k)$.
Here,~$n$~may take any value from 
the set $\{2,3,\dots,H-2\}$. 
Remembering that the sojourn time 
in the state $(i,0,n)$ is exponential 
with rate $-[\bmx{Q}]_{ii}$, one has for 
$1 \le i,j \le NM$:
\begin{align*}
\left[ \bmx{T}_{-\!1}(s) \right]_{ij}
&=
\fr{[\bmx{Q}]_{ii} }{[\bmx{Q}]_{ii} - s}
\left[\gamma\right]_{ij}  \,;
\\
\left[ \bmx{T}_0(s) \right]_{ij}
&=
\fr{[\bmx{Q}]_{ii}}{[\bmx{Q}]_{ii} - s}
\left[\beta\right]_{ij}  \,;
\\
\left[ \bmx{T}_1(s) \right]_{ij}
&=
\fr{[\bmx{Q}]_{ii} }{[\bmx{Q}]_{ii} - s}
\left[\alpha\right]_{ij} \,.
\end{align*}

\noindent 
From the first-step analysis, let us find that
the LST $\bmx{V}_n(s)$ satisfies the system of matrix 
difference equations: 
\begin{equation}
\left.
\begin{array}{rl}
\bmx{V}_n(s)&=\bmx{T}_1(s) \bmx{V}_{n-1}(s)+ 
\bmx{T}_0(s) \bmx{V}_{n}(s)\\[6pt]
&\hspace*{1mm}{}+ \bmx{T}_{-1}(s) \bmx{V}_{n+1}(s)\,,\enskip
n=\overline{2,H-1}\,;
\\[6pt]
\bmx{V}_1(s)
&= \bmx{T}^e_1(s)  \bmx{V}_{0}(s) +  \bmx{T}_0(s)  \bmx{V}_{1}(s)\\[6pt]
&\hspace*{28mm}{} + 
\bmx{T}_{-1}(s)  \bmx{V}_{2}(s)\,.
\end{array}
\right\}
\label{eq1-r}
\end{equation}

\noindent The boundary conditions for the system~\eqref{eq1-r} 
have the form:
%%%%%%%%%%%%%%%%%%%%%%%
\begin{equation}
\left.
\begin{array}{rl}
\bmx{V}_{0}(s)&= \bmx{T}^e_0(s)  \bmx{V}_{0}(s)
+  \bmx{T}^e_{-\!1}(s)  \bmx{V}_{1}(s) \,;
\\[6pt]
\bmx{V}_{H}(s)
&=\bmx{E} \,,
\end{array}
\right\}
\label{eq12b}
\end{equation}

\noindent 
where the $(i,j)${th} entries of the matrices 
$\bmx{T}^e_{1}(s)$, $\bmx{T}^e_0(s)$, and $\bmx{T}^e_{-\!1}(s)$
are equal to 
\begin{align*}
\left[ \bmx{T}^e_1(s) \right]_{ij}
&=
\fr{[\bmx{Q}]_{ii} }{[\bmx{Q}]_{ii} - s}
\left[\alpha^{e}\right]_{ij}\,;
\\
\left[ \bmx{T}^e_0(s) \right]_{i,j}
&=
\fr{[\bmx{Q}_0]_{ii}}{[\bmx{Q}_0]_{ii} - s}
\left[\beta^{e}\right]_{ij}
\,;
\\
\left[ \bmx{T}^e_{-1}(s) \right]_{i,j}
&=
\fr{[\bmx{Q}_0]_{ii}}{[\bmx{Q}_0]_{ii} - s}
\left[\gamma^{e}\right]_{ij}\,.
\end{align*}

\noindent 
Note that here, $\bmx{T}^e_0(s)$ is the square matrix of size~$N$
and $\bmx{T}^e_1(s) $ and $\bmx{T}^e_1(s)$ are the
rectangular matrices of size $NM\times N$ and $N\times NM$,
correspondingly.
The solution of the system~\eqref{eq1-r}--\eqref{eq12b}
can be written as 
$$
\bmx{V}_{n}(s)=\bmx{X}_{n}(s) \bmx{V}_{n-1}(s)+\bmx{Y}_{n}(s)\,,
\enskip n=\overline{1,H-1}\,,
$$

\noindent where 
$$
\bmx{X}_{H-1}(s)=(\bmx{E}-\bmx{T}_0(s))\bmx{T}_1(s)\,;
$$
$$
\bmx{Y}_{H-1}(s)=(\bmx{E}-\bmx{T}_0(s))\bmx{T}_{-\!1}(s)\,;
$$
$$
\bmx{X}_{1}(s)=
\left(\bmx{E}-\bmx{T}_0(s)-\bmx{T}_{-1}(s)\bmx{X}_{2}(s)\right)^{-1}\bmx{T}^e_1(s)\,; 
$$

\vspace*{-12pt}

\noindent
\begin{multline*}
\bmx{X}_{n}(s)=\left(\bmx{E}-\bmx{T}_0(s)-\bmx{T}_{-1}(s)\bmx{X}_{n+1}(s)\right)^{-1}
\bmx{T}_1(s)\,, \\
n=\overline{2,H-2}\,;
\end{multline*}

\vspace*{-12pt}

\noindent
\begin{multline*}
\bmx{Y}_{n}(s)=\left(\bmx{E}-\bmx{T}_0(s)\right.\\
\left.{}-\bmx{T}_{-1}(s)\bmx{X}_{n+1}(s)\right)^{-1}
\bmx{T}_{-1}(s)\bmx{Y}_{n+1}(s)\,,
\\ 
 n=\overline{1,H-2}\,;
\end{multline*}

\vspace*{-12pt}

\noindent
\begin{multline*}
\bmx{V}_{0}(s)\\
{}=\left(\bmx{E}-\bmx{T}^e_0(s)-\bmx{T}^e_{-1}(s)\bmx{X}_{1}(s)
\right)^{-1}\bmx{T}^e_{-1}(s)\bmx{Y}_{1}(s)\,. 
\end{multline*}

If the inverse of the matrix $\bmx{T}_{-\!1}(s)$ exists, then 
it is possible to write out the solution of the system~\eqref{eq1-r}--\eqref{eq12b}
using the Kronecker expansion technique (see~\cite{Steeb,Graham,Telek}),
which is based on the identity $\mathrm{vec}\,(\bmx{A}\bmx{B})=
(\bmx{E} \otimes \bmx{A}) \mathrm{vec}\,(\bmx{B})$.
In this identity, ~$\mathrm{vec}$\ denotes the column stacking vector operator
which transforms a~matrix of size $n \times m$ into a~vector of size $nm \times 1$.
We are going to utilize the property of the~$\mathrm{vec}$ operator 
that $\mathrm{vec}\,(\bmx{A})=\bmx{A}$ for matrix~$\bmx{A}$ of size $n \times 1$.
Firstly by applying $\mathrm{vec}$ operator to~\eqref{eq1-r}--\eqref{eq12b},
we get the new system of vector-matrix difference equations:
\begin{align}
\label{H-1}
\mathrm{vec}\left(\bmx{V}_{H-1}(s)\right)
&\notag\\
&\hspace*{-10mm}{}= \bmx{X}^*(s) \mathrm{vec}\left(\bmx{V}_{H-2}(s)\right) +
\mathrm{vec}\left(\bmx{Y}^*(s)\right)\,;
\\
\label{n}
\mathrm{vec}\left(\bmx{V}_{n+1}(s)\right)&=\bmx{X}(s)
\mathrm{vec}\left(\bmx{V}_n(s)\right)\notag\\
&\hspace*{-10mm}{}+\bmx{Y}(s)
\mathrm{vec}\left(\bmx{V}_{n-1}(s)\right)\,,\enskip
n=\overline{2,H-2}\,;
\\
\label{2}
\mathrm{vec}\left(\bmx{V}_{2}(s)\right)&{}\notag\\
&\hspace*{-12mm}{}=
\bmx{X}(s) \mathrm{vec}\left(\bmx{V}_1(s)\right)+
\bmx{Y}^e(s) \mathrm{vec}\left(\bmx{V}_{0}(s)\right)\,;
\\
\label{0}
\mathrm{vec}\left(\bmx{V}_{1}(s)\right) &= \bmx{X}^e(s) \mathrm{vec}\left(\bmx{V}_{0}(s)\right)
\end{align}
where
\begin{align*}
\bmx{X}(s)&= \bmx{E} \otimes \left(\bmx{T}_{-1}(s)\right)^{-1}
\left(\bmx{E}-\bmx{T}_0(s)\right)\,; \\
 \bmx{Y}(s)&= - \left(\bmx{E} \otimes (\bmx{T}_{-1}(s) )^{-1}
\bmx{T}_1(s)\right)\,;
\\
\bmx{X}^e(s)&= \bmx{E} \otimes \left(\bmx{T}^e_{-1}(s) \right)^{-1}
\left(\bmx{E}-\bmx{T}^e_0(s)\right)\,; \\
\bmx{Y}^e(s)&=- \left(\bmx{E} \otimes (\bmx{T}_{-1}(s) )^{-1}
\bmx{T}^e_1(s)\right)\,;\\
\bmx{X}^*(s)&=
\bmx{E} \otimes \left(\bmx{E}-\bmx{T}_0(s)\right)^{-1} \bmx{T}_{1}(s) \,; \\ 
\bmx{Y}^*(s)&=\left(\bmx{E}-\bmx{T}_0(s)\right)^{-1} \bmx{T}_{-1}(s)\,.
\end{align*}

\noindent Secondly, notice that the new system~\eqref{H-1}---\eqref{0}
 consists of pairs of simultaneous equations and thus, its solution can 
 be rewritten as
\begin{multline}
\label{v1}
\begin{pmatrix}
    \mathrm{vec}\left(\bmx{V}_{n}(s)\right) \\
    \mathrm{vec}\left(\bmx{V}_{n-1}(s)\right) 
\end{pmatrix}
={}\\
{}=
\begin{pmatrix}
    \bmx{X}(s) & \bmx{Y}(s) \\
    \bmx{E} & \bmx{0}
\end{pmatrix}^{n-1}
\begin{pmatrix}
    \bmx{X}(s) & \bmx{Y}^e(s) \\
    \bmx{E} & \bmx{0}
\end{pmatrix}
\begin{pmatrix}
    \mathrm{vec}\left(\bmx{V}_{1}(s)\right) \\
    \mathrm{vec}\left(\bmx{V}_{0}(s)\right)
\end{pmatrix}\,, \\
 n=\overline{2,H-1}\,.
\end{multline}


\noindent 
Finally,~\eqref{v1} for $n=H-1$, \eqref{0} and~\eqref{H-1}
make up the system of four matrix equations,
which solution yields the values of 
$\mathrm{vec}\,(\bmx{V}_{H-1}(s))$, $\mathrm{vec}\,(\bmx{V}_{H-2}(s))$,
$\mathrm{vec}\,(\bmx{V}_{1}(s))$, and $\mathrm{vec}\,(\bmx{V}_{0}(s))$.
By virtue of~\eqref{n} and~\eqref{2}, the rest 
of $\bmx{V}_{n}(s)$ can be computed.

Now, let us proceed to the derivation of the equations for 
$\bmx{V}^*_{n}(s)$, $n=\overline{L,r}$,
and $\bmx{V}^{\#}_{n}(s)$, $n=\overline{H+1,r+1}$.
In order to do this, let us introduce the following auxiliary 
square matrices (each of size~$NM$):
\begin{description}
\item[\,] $\bmx{T}^{\#}(s)$~---  
matrix with the  $(i,j)${th} entry 
equal to the LST of the first passage time (of the Markov chain ${\bf{X}}(t)$)
from the state $(i,2,r+1)$ to the state $(j,2,r)$;
\item[\,] 
$\bmx{W}^{\#}_n(s)$, $n=\overline{H+1,r+1}$,~---
matrix with the $(i,j)${th} entry 
equal to the LST of the first passage time 
from the state $(i,2,n)$ to the state $(j,1,H)$;

\item[\,] 
$\bmx{w}^*_n(s)$, $n=\overline{H+1,r}$,~---
matrix with the $(i,j)${th} entry 
equal to the LST of the first passage time 
from the state $(i,1,n)$ to the state $(j,1,H)$
without visiting the states $(\cdot,2,r+1)$;

\item[\,] 
$\bmx{\ov w}^*_n$, $n=\overline{H+1,r}$,~---
matrix with the $(i,j)${th} entry 
equal to the LST of the first passage time 
from the state $(i,1,n)$ to the state $(j,2,r+1)$
without visiting the states $(\cdot,1,H)$; and
\item[\,] 
$\bmx{W}^*_n(s)$, $n=\overline{H+1,r}$,~---
matrix with the $(i,j)${th} entry 
equal to the LST of the first passage time 
from the state $(i,1,n)$ to the state $(j,1,H)$.
\end{description}

Let us begin with the relation for~$\bmx{T}^{\#}(s)$. 
Let~$\bmx{T}^{\#}_k(s)$, $k=0,1$, denote 
the square matrix of size $NM$ with 
the $(i,j)${th} entry 
equal to the LST of the first passage time 
from the state $(i,2,r+1)$ to the state $(j,2,r+k)$.
Since the sojourn time in the state $(i,2,r+1)$
is distributed exponentially with the rate $-[\bmx{Q}^{\#}]_{ii}$, 
one has for $1 \le i,j \le NM$:
\begin{gather*}
\left[ \bmx{T}^{\#}_0(s) \right]_{ij}
=
\fr{[\bmx{Q}^{\#}]_{ii}}{[\bmx{Q}^{\#}]_{ii} - s}
\left[\beta^{\#}\right]_{ij}  \,;
\\
\left[ \bmx{T}^{\#}_1(s) \right]_{ij}
= \fr{[\bmx{Q}^{\#}]_{ii} }{[\bmx{Q}^{\#}]_{ii} - s}
\left[\alpha^{\#}\right]_{ij}  \,.
\end{gather*}

\noindent The first-step analysis yields the following 
equation for~$\bmx{T}^{\#}(s)$:
$$
\bmx{T}^{\#}(s)= \bmx{T}^{\#}_1(s) + \bmx{T}^{\#}_0(s) \bmx{T}^{\#}(s)\,,
$$
which solution is
$$
\bmx{T}^{\#}(s)= \left(\bmx{E}-\bmx{T}^{\#}_0(s)\right)^{-1} \bmx{T}^{\#}_1(s)\,.
$$


Since in the ``blocking'' mode any arrival is lost, then the sojourn time in 
it is equal to the time needed for $(n-H)$ service completions,
given that initially, there were $n$ customers in the system, i.\,e.,
$$
\bmx{W}^{\#}_n(s)= \left(\bmx{T}^{\#}(s)\right)^{n-H}\,, \enskip
n=\overline{H+1,r+1}\,.
$$

Equations for $\bmx{w}^*_n(s)$ and $\bmx{\ov w}^*_n(s)$ can be 
derived by following the same arguments given above for~$\bmx{V}_n(s)$. Denote by 
$\bmx{T}^{*}_k(s)$, $k=-1,0,1$,  
the square matrix of size~$NM$ with 
the $(i,j)${th} entry 
equal to the LST of the first passage time 
from the state $(i,1,n)$ to the state $(j,1,n-k)$.
Then, using the fact that the sojourn time in the state $(i,1,n)$
is distributed exponentially with the rate~$-[\bmx{Q}^{*}]_{ii}$, 
one obtains for $1 \le i,j \le NM$:
\begin{align*}
\left[ \bmx{T}^{*}_{-1}(s) \right]_{ij}
&=
\fr{[\bmx{Q}^{*}]_{ii}}{[\bmx{Q}^{*}]_{ii} - s}
\left[\gamma^{*}\right]_{ij}\,;
\\
\left[ \bmx{T}^{*}_0(s) \right]_{ij}
&=\fr{[\bmx{Q}^{*}]_{ii}}{[\bmx{Q}^{*}]_{ii} - s}
\left[\beta^{*}\right]_{ij}  \,;
\\
\left[ \bmx{T}^{*}_1(s) \right]_{i,j}
&=
\fr{[\bmx{Q}^{*}]_{ii} }{[\bmx{Q}^{*}]_{ii} - s}
\left[\alpha^{*}\right]_{ij}   \,.
\end{align*}

Again, by applying the first-step analysis, one gets 
the following system of matrix difference equations 
for~$\bmx{w}^*_n(s)$, $n=\overline{H+1,r}$:
\begin{multline}
\label{w*n}
\bmx{w}^*_n(s)
= \bmx{T}^*_1(s)  \bmx{w}^*_{n-1}(s)\\
{} +  \bmx{T}^*_0(s) \bmx{w}^*_n(s) + 
\bmx{T}^*_{-\!1}(s)  \bmx{w}^*_{n+1}(s)\,,
\end{multline}

\noindent with the boundary conditions
\begin{equation}
\label{w*nbound}
\bmx{w}^*_H(s)
=
\bmx{E}\,;
\enskip  
\bmx{w}^*_{r+1}(s)
= 
\bmx{0}\,.
\end{equation}


\noindent Clearly, the matrices $\bmx{\ov w}^*_n(s)$, $n=\overline{H+1,r}$,
satisfy the system of equations, which is identical to~\eqref{w*n},
i.\,e.,
\begin{multline}
\label{w*nOV} \bmx{\ov w}^*_n(s)
= \bmx{T}^*_1(s)  \bmx{\ov w}^*_{n-1}(s)
+  \bmx{T}^*_0(s) \bmx{\ov w}^*_n(s)\\
{}+  \bmx{T}^*_{-1}(s) 
\bmx{\ov w}^*_{n+1}(s)\,,
\end{multline}

\noindent but with the ``reversed'' boundary conditions:
\begin{equation}
\label{w*nboundOV}
\bmx{\ov w}^*_H(s)
= \bmx{0}\,; \enskip  
\bmx{\ov w}^*_{r+1}(s) =  \bmx{E}\,.
\end{equation}

The structure of the systems~\eqref{w*n}, \eqref{w*nbound} 
and~\eqref{w*nOV}, \eqref{w*nboundOV} is
similar to the~\eqref{eq1-r}--\eqref{eq12b} (expect for the boundary conditions).
Thus, its solutions can be found completely in the same way and, therefore, are omitted.
Once~$\bmx{w}^*_n(s)$ and~$\bmx{\ov w}^*_n(s)$ are found, the 
matrices~$\bmx{W}^*_n(s)$ can be computed. Noticing that from the 
state $(i,1,n)$, $n=\overline{H+1,r}$, the Markov chain can enter 
the state $(j,H,1)$ either from the set of ``overload'' states or
from the set of ``blocking'' states (see figure), one has:
$$
\bmx{W}^*_n(s)= \bmx{w}^*_n(s)+\bmx{\ov w}^*_n(s)\bmx{W}^{\#}_{r+1}(s)\,,
\enskip n=\overline{H+1,r}\,.
$$

\columnbreak

Now, everything is prepared for the derivation of
the relations for the unknown quantities $\bmx{V}^*_n(s)$  
and $\bmx{V}^{\#}_n(s)$. 
If $n=\overline{L,H}$, then $\bmx{V}^*_n(s)$ satisfy the following
system of matrix difference equations:
\begin{multline}
\label{V*}
\bmx{V}^*_n(s)
= \bmx{T}^*_1(s)  \bmx{V}^*_{n-1}(s) + 
\bmx{T}^*_0(s) \bmx{V}^*_n(s)\\
{} +  \bmx{T}^*_{-\!1}(s)  \bmx{V}^*_{n+1}(s)\,,
\end{multline}

\noindent with the boundary conditions
\begin{align*}
\bmx{V}^*_{L-1}(s)
&= \bmx{E}\,;
\\[3pt]
\bmx{V}^*_{H}(s)&=  \bmx{T}^*_1(s)  \bmx{V}^*_{H-1}(s)\\
&{}  + 
\bmx{T}^*_0(s) \bmx{V}^*_H(s)
+  \bmx{T}^*_{-1}(s) 
\bmx{W}^*_{H+1}(s)
\bmx{V}^*_{H}(s)\,.
%\label{w*nboundOV1}
\end{align*}


The final expressions for the matrices
$\bmx{V}^*_n(s)$, $n=\overline{H+1,r}$
and $\bmx{V}^{\#}_n(s)$, $n=\overline{H+1,r+1}$,
have the form:
\begin{alignat*}{2}
\bmx{V}^{\#}_{n}(s)
&=  \bmx{W}^{\#}_n(s) 
\bmx{V}^*_{H}(s)\,, &\enskip n&=\overline{H+1,r+1}\,;
\\
\bmx{V}^{*}_{n}(s)
&=  \bmx{W}^{*}_n(s) 
\bmx{V}^*_{H}(s)\,, &\enskip n&=\overline{H+1,r}\,.
\end{alignat*}

The last two relations, together with~\eqref{eq1-r} and~\eqref{V*}, 
give the complete solution of the considered problem.
The matrices $\bmx{V}_n(s)$, $\bmx{V}^*_n(s)$, and $\bmx{V}^{\#}_n(s)$
allow one to calculate various performance characteristics of
the hysteretic policy such as (conditional\footnote{The corresponding 
unconditional characteristics 
are obtained by weighting according to the joint stationary distribution 
found in~\cite{int0}.}) 
mean duration of overload period (equal to $-[\bmx{V}^*_H(s)]'_{s=0}$),
(conditional) mean return time to the ``overload'' 
mode (equal to $-[\bmx{V}_{L-1}(s)]'_{s=0}- [\bmx{V}^*_H(s)]'_{s=0}$),
higher moments, etc.

No principal difficulties show up if in addition 
to the sojourn times one needs to count 
the number of lost customers. The same
argumentation applies. For example, 
let us assume that customers arrived during the
period of time when the system 
is in the ``blocking'' mode are considered as lost.
Then, the LST and the generating function~$\bmx{W}^{\#}_n(s,z)$
of the joint stationary distribution of the 
sojourn time in the ``blocking'' mode 
and the number of lost customers 
provided that the system is in the ``blocking'' mode 
and there are $n$ customers in it 
is equal to 
\begin{multline*}
\bmx{W}^{\#}_n(s,z)
=  \left \{
\left[ \bmx{E} - z ( \bmx{E}-\bmx{T}^*_0(s))^{-1}\bmx{T}^*_{-1}(s) \right]^{-1}\right.\\
\left.{}\times
\left( \bmx{E}-\bmx{T}^*_0(s)\right)^{-1}\bmx{T}^*_{1}(s)
\right \}^{H+1-n}
\times{} \\
{}\times
\left \{ \bmx{E} +  
z \left[ \bmx{E} - z \left( \bmx{E}-\bmx{T}^*_0(s)\right)^{-1}
\bmx{T}^*_{-1}(s) \right]^{-1}
\right \}\\
{}\times
\left( \bmx{E}-\bmx{T}^*_0(s)\right)^{-1}\bmx{T}^*_{1}(s)\,.
\end{multline*}

\noindent Thus, the substitution of~$\bmx{W}^{\#}_n(s,z)$ 
instead of~$\bmx{W}^{\#}_n(s)$ in the above expressions
will account not only for the sojourn time but for losses
(during the sojourn time). 



\section{Concluding Remarks}

\noindent
The approach proposed in the paper allows one to calculate
the system's sojourn time in various modes 
in terms of LST by exploiting the knowledge
about the presence of hysteretic loops. 
Minor changes are needed to adapt it to the case of overlapping loops. 
Of course, due to MAP arrivals and PH service times, it
utilizes matrix analytic techniques and, thus, possesses
the disadvantages inherent to matrix algebra.
Despite the fact there is large body of research results 
available in this topic, 
there is still a~number of open questions.
Is there any analytic approach to find the steady-state
behavior of several interconnected systems each with hysteretic policy, 
which exploits the knowledge of the presence of hysteretic loops?
What is the gain of hysteretic control of arrivals with
respect to other types of control? Just to name a~few.

\vspace*{-4pt}


\Ack
\noindent
This work was supported by the Russian Science Foundation (grant No.\,16-11-10227).

\renewcommand{\bibname}{\protect\rmfamily References}

\vspace*{-4pt}


{\small\frenchspacing
{%\baselineskip=10.8pt
\begin{thebibliography}{9}

\bibitem{int0} %1
\Aue{Razumchik, R.} 2016. Analysis of finite ${\mathrm{MAP}/\mathrm{PH}/1}$ 
queue with hysteretic control of arrivals.
\textit{Congress (International)
on Ultra Modern Telecommunications and Control Systems and Workshops Proceedings}.
 Lisbon. 288--293.



\bibitem{int2} %2
\Aue{Chesoong, K., A.~Dudin, S.~Dudin, and O.~Dudina.} 2016.
Hysteresis control by the number of active servers in queueing system ${\mathrm{MMAP}/\mathrm{PH}/N}$
with priority service. \textit{Perform. Evaluation} 101:20--33.

\bibitem{int3} %3
\Aue{Chan, C.\,W., M.~Armony, and N.~Bambos.} 2016.
Maximum weight matching with hysteresis in overloaded queues with setups.
\textit{Queueing Sy.} 82(3-4):315--351.

\bibitem{int4} %4
\Aue{Rumyantsev, A.\,S., K.\,A.~Kalinina, and T.\,E.~Morozova.} 2017.
Stokhasticheskioe modelirovanie vycheslitel'nogo klastera s~gisterezisnym upravleniem
skorost'yu obsluzhivaniya
[Stochastic modeling of a~high-performance cluster
with hysteretic control of service rate].
\textit{Trudy Karel'skogo nauchnogo tsentra RAN}
[Transactions of KarRC RAS] 8:76--85.

\bibitem{int1} %5
\Aue{Abaev, P., Y.~Gaidamaka, K.~Samouylov, A.~Pechinkin, R.~Razumchik, and S.~Shorgin.} 2014.
Hysteretic control technique for overload problem solution in network of SIP servers.
\textit{Comput. Inform.} 33(1):1--18.




\bibitem{Graham} %6
\Aue{Graham, A.} 1982.
\textit{Kronecker products and matrix calculus: With applications}. 
New York, NY: John Wiley \& Sons. 130~p.

\bibitem{Steeb} %7
\Aue{Steeb, W.\,H., and Y.~Hardy.} 2011.
\textit{Matrix calculus and Kronecker product: A~practical 
approach to linear and multilinear algebra}. 2nd ed. River Edge,
NJ: World Scientific. 324~p.

\bibitem{Telek} %8
\Aue{Razumchik, R., and M.~Telek.} 2016.
Delay analysis of a~queue with re-sequencing buffer
and Markov environment.
\textit{Queueing Sy.} 82(1-2):7--28.

\end{thebibliography} }
 }

\end{multicols}

\vspace*{-3pt}

\hfill{\small\textit{Received September 19, 2017}}

%\vspace*{-12pt}


\Contrl

\noindent
\textbf{Razumchik Rostislav V.} (b.\ 1984)~--- 
Candidate of Science (PhD) in physics and mathematics, leading scientist, 
Institute of Informatics Problems, Federal Research Center 
``Computer Science and Control'' of the Russian Academy of Sciences, 
44-2~Vavilov Str., Moscow 119333, Russian Federation; associate professor, 
Peoples' Friendship University of Russia (RUDN University), 
6~Miklukho-Maklaya Str., Moscow 117198, Russian Federation; 
\mbox{rrazumchik@ipiran.ru}
%; \mbox{razumchik\_rv@rudn.university}

%\vspace*{8pt}

%\hrule

%\vspace*{2pt}

%\hrule



\newpage

\vspace*{-28pt}



\def\tit{СТАЦИОНАРНЫЕ РАСПРЕДЕЛЕНИЯ, СВЯЗАННЫЕ СО~ВРЕМЕНЕМ ПРЕБЫВАНИЯ
В~СОСТОЯНИИ ПЕРЕГРУЗКИ СИСТЕМЫ $\mathrm{MAP}/\mathrm{PH}/1/r$ С~ГИСТЕРЕЗИСНЫМ УПРАВЛЕНИЕМ 
НАГРУЗКОЙ$^*$}

\def\aut{Р.\,В.~Разумчик}


\def\titkol{Стационарные распределения, связанные со временем пребывания
в~состоянии перегрузки системы $\mathrm{MAP}/\mathrm{PH}/1/r$}
% с~гистерезисным управлением  нагрузкой}

\def\autkol{Р.\,В.~Разумчик}

{\renewcommand{\thefootnote}{\fnsymbol{footnote}}
\footnotetext[1]{Работа выполнена при поддержке РНФ (проект 16-11-10227).}}


\titel{\tit}{\aut}{\autkol}{\titkol}

\vspace*{-12pt}

\noindent
Институт проблем информатики
Федерального исследовательского центра <<Информатика и~управ\-ле\-ние>>
Российской академии наук; Российский университет дружбы народов,
\mbox{rrazumchik@ipiran.ru}

\vspace*{6pt}

\def\leftfootline{\small{\textbf{\thepage}
\hfill ИНФОРМАТИКА И ЕЁ ПРИМЕНЕНИЯ\ \ \ том\ 11\ \ \ выпуск\ 4\ \ \ 2017}
}%
 \def\rightfootline{\small{ИНФОРМАТИКА И ЕЁ ПРИМЕНЕНИЯ\ \ \ том\ 11\ \ \ выпуск\ 4\ \ \ 2017
\hfill \textbf{\thepage}}}


\Abst{Как известно, одним из решений проблемы перегрузок
в~сетях SIP (session initiation protocol)
сиг\-на\-ли\-за\-ции является применение
в~узлах сети (SIP-сер\-ве\-рах) многоуровневого
гистерезисного управления нагрузкой.
В~данной работе представлены некоторые новые 
результаты анализа системы $\mathrm{MAP}/\mathrm{PH}/1/r$ конечной емкости
с~двумя петлями гистерезисного управления,
функционирующей в случайной среде и являющейся
моделью SIP-сер\-ве\-ра с двухуровневым
гистерезисным управлением нагрузкой.
Получен метод вычисления преобразования
Лап\-ла\-са--Стилть\-еса
функций распределения времени возврата системы из
множества состояний перегрузки в множество
состояний нормальной нагрузки
и~времени выхода системы из множества состояний
нормальной нагрузки.
Метод основан на решении матричных рекуррентных уравнений
и~применим в случае, когда помимо расчета времени
выхода из состояния перегрузки необходимо
также учитывать и число потерянных за это время
заявок.}

\KW{система массового обслуживания; случайная среда; гистерезисное управление; 
время пребывания}

\DOI{10.14357/19922264170403}

\vspace*{18pt}


 \begin{multicols}{2}

\renewcommand{\bibname}{\protect\rmfamily Литература}
%\renewcommand{\bibname}{\large\protect\rm References}

{\small\frenchspacing
{%\baselineskip=10.8pt
\begin{thebibliography}{9}

\bibitem{1-r-1}
\Au{Razumchik R.} 
Analysis of finite ${\mathrm{MAP}/\mathrm{PH}/1}$ queue with hysteretic control of arrivals~// 
 Congress (International) on Ultra Modern Telecommunications and Control 
Systems and Workshops Proceedings.~--- Lisbon, 2016.  P.~288--293.


\bibitem{3-r-1} %2
\Au{Chesoong K., Dudin~A., Dudin~S., Dudina~O.}
Hysteresis control by the number of active servers in queueing system 
$\mathrm{MMAP}/\mathrm{PH}/N$
with priority service~// Perform. Evaluation, 2016. Vol.~101. P.~20--33.

\bibitem{4-r-1} %3
\Au{Chan C.\,W., Armony~M., Bambos~N.}
Maximum weight matching with hysteresis in overloaded queues with setups~// 
Queueing Sy., 2016. Vol.~82. No.\,3-4. P.~315--351.

\bibitem{5-r-1} %4
\Au{Румянцев А.\,С., Калинина~К.\,А., Морозова~Т.\,Е.}
Стохастическое моделирование вычислительного кластера 
с~гистерезисным управлением скоростью обслуживания~//
Труды Карельского научного центра РАН, 2017. Вып.~8. С.~76--85.

\bibitem{2-r-1} %5
\Au{Abaev P., Gaidamaka~Y.,  Samouylov~K.,  Pechinkin~A.,  Razumchik~R.,  Shorgin~S.}
Hysteretic control technique for overload problem solution in network of SIP servers~//
Comput. Inform., 2014. Vol.~33. No.\,1. P.~1--18.





\bibitem{7-r-1} %6
\Au{Graham A.}
Kronecker products and matrix calculus: With applications.~--- 
New York, NY, USA: John Wiley \& Sons, 1982. 130~p.

\bibitem{6-r-1} %7
\Au{Steeb W.\,H., Hardy~Y.}
Matrix calculus and Kronecker product: A~practical
approach to linear and multilinear algebra.~--- 2nd ed.~---
River Edge, NJ, USA: World Scientific, 2011. 324~p.

\bibitem{8-r-1}
\Au{Razumchik~R., Telek~M.}
Delay analysis of a~queue with re-sequencing buffer
and Markov environment~// Queueing Sy., 2016. Vol.~82. 
No.\,1-2. P. 7--28.

\end{thebibliography}
} }

\end{multicols}

 \label{end\stat}

 \vspace*{-3pt}

\hfill{\small\textit{Поступила в~редакцию  19.09.2017}}
%\renewcommand{\bibname}{\protect\rm Литература}
\renewcommand{\figurename}{\protect\bf Рис.}
\renewcommand{\tablename}{\protect\bf Таблица} 