\def\ld{\ldots}
\def\d{\overline d}

\def\stat{raz-rus}

\def\tit{СТАЦИОНАРНЫЕ ХАРАКТЕРИСТИКИ СИСТЕМЫ ОБСЛУЖИВАНИЯ
С~ИНВЕРСИОННЫМ ПОРЯДКОМ ОБСЛУЖИВАНИЯ, ВЕРОЯТНОСТНЫМ
ПРИОРИТЕТОМ И~ГРУППОВЫМ ПОСТУПЛЕНИЕМ РАЗНОРОДНЫХ ЗАЯВОК$^*$}

\def\titkol{Стационарные характеристики системы обслуживания
с~инверсионным порядком обслуживания} %, вероятностным приоритетом и~групповым поступлением разнородных заявок}

\def\aut{Р.\,В.~Разумчик$^1$}

\def\autkol{Р.\,В. Разумчик}

\titel{\tit}{\aut}{\autkol}{\titkol}

\index{Разумчик Р.\,В.}
\index{Razumchik R.\,V.}



{\renewcommand{\thefootnote}{\fnsymbol{footnote}} \footnotetext[1]
{Работа выполнена при поддержке Российского научного фонда (проект 16-11-10227).}}


\renewcommand{\thefootnote}{\arabic{footnote}}
\footnotetext[1]{Институт проблем информатики Федерального исследовательского центра <<Информатика 
и~управ\-ле\-ние>> Российской академии наук; Российский
университет дружбы народов, \mbox{rrazumchik@ipiran.ru}}
%; \mbox{razumchik\_rv@rudn.university}}

\vspace*{-16pt}



\Abst{Статья посвящена исследованию стационарных характеристик
однолинейных систем массового обслуживания (СМО)
со специальными дисциплинами обслуживания.
Рассматриваемая дисциплина~--- инверсионный
порядок обслуживания с~вероятностным приоритетом.
Основные результаты для данной дисциплины
были получены в~предположениях, что в~систему поступает
пуассоновский поток или поток фазового типа и~времена
обслуживания имеют произвольное распределение.
Существенным также было предположение о~независимости процесса поступления заявок
от состояния системы. Здесь же показано, что это
предположение может быть определенным образом ослаблено.
Рассматривается система с~одним прибором, очередью неограниченной
емкости и~неординарным пуассоновским потоком,
интенсивность которого может зависеть от общего числа заявок, находящихся
в системе в~момент поступления группы, причем
размер  поступающей группы и~размеры заявок
в~ней имеют совместное произвольное распределение.
Получены аналитические соотношения, позволяющие
рассчитывать совместное стационарное распределение числа заявок
в~системе и~остаточных времен обслуживания.
Кроме того, в~терминах преобразований Лап\-ла\-са--Стилть\-еса (ПЛС)
находятся стационарные распределения
случайных величин, связанных с~временем ожидания начала обслуживания
и~пребывания заявки в~системе.}


\KW{инверсионный порядок обслуживания; вероятностный приоритет; 
неординарный входящий поток}

%\vspace*{-8pt}

\DOI{10.14357/19922264170402} 


\vskip 10pt plus 9pt minus 6pt

\thispagestyle{headings}

\begin{multicols}{2}

\label{st\stat}

\section{Введение}

Эта статья развивает результаты работ~\cite{n0,n1,n2,n3,n4,n5} по исследованию
стационарных характеристик однолинейных СМО
$M/G/1$ с~инверсионным порядком обслуживания и~вероятностным приоритетом.
Основные результаты этих работ были получены в~предположении, что входящий в~систему поток
является простейшим. 

Как было продемонстрировано в~\cite{nm1},
некоторые из этих результатов допускают обобщение на случай
потоков фазового типа, которые не являются рекуррентными и,~таким образом,
могут быть более привлекательными при моделировании процессов в~реальных технических системах.
Несмотря на свою общность, модель потока фазового типа
подразумевает, что процесс поступления заявок в~сис\-те\-му не зависит от состояния самой системы.
Тем\linebreak самым в~стороне осталась задача обобщения результатов на случай,
когда такая зависимость присутствует.
Не останавливаясь на возможных практических интерпретациях
связей между входящим\linebreak потоком и~состоянием системы (см.~\cite{gg2}),
 отметим лишь, что исследованию СМО с~такими зависимостями посвящено достаточно много работ
(см., например,~\cite{gg1,gg3,gg4,gg5} и~ссылки в~них). Обычно
предполагается, что в~систему поступает пуассоновский поток второго рода
(т.\,е.\ интенсивность потока зависит от общего числа заявок, находящихся в~сис\-те\-ме). 
Если же
допускается поступление групп заявок, то обычно предполагается, что размеры (остаточные времена обслуживания)
заявок в~группе являются независимыми случайными величинами (не зависящими также и~от размера группы).

В~данной статье эти предположения ослабляются следующим образом:
рассматривается неординарный пуассоновский поток,
интенсивность которого может зависеть от общего числа заявок, находящих\-ся
в~системе в~момент поступления группы, причем
размер  поступающей группы и~размеры\linebreak заявок в~ней имеют совместное произвольное распределение.
Для однолинейной СМО неограниченной емкости с~инверсионным порядком обслуживания 
и~вероятностным приоритетом при таком\linebreak входя\-щем потоке
решена задача отыскания совместного стационарного
распределения числа заявок в~сис\-те\-ме и~их остаточного времени обслуживания,
а~также стационарных распределений (в~терминах ПЛС),
связанных с~временем пребывания заявки в~системе.

\vspace*{-4pt}

\section{Описание системы}

\vspace*{-2pt}

Рассмотрим однолинейную СМО с~очередью неограниченной емкости,
на вход которой поступает групповой пуассоновский поток заявок с~переменной 
интенсивностью~$\lambda_n$, зависящей от числа заявок~$n$, находящихся в~системе.
Через $B_k(x_1,\ld,x_k)$ будем обозначать вероятность того, что в~поступившей
группе будет~$k$ заявок, причем первая заявка будет иметь
длину меньше~$x_1$, вторая~--- меньше $x_2$ и~т.\,д.;
через $b_k(x_1,\ld,x_k)\hm=
\partial^k B_k(x_1,\ld,x_k)/(\partial x_1\cdots \partial x_k)$~--- 
совместную плотность вероятностей.
Длины заявок в~различных группах независимы между собой.

Определим дисциплину обслуживания сле\-ду\-ющим образом:
в~момент прихода очередной группы заявок замеряется
остаточное время обслуживания (в дальнейшем будем называть его
длиной) первой заявки из группы.
Пусть она равна~$x$. Эта длина сравнивается с~остаточной длиной
заявки, находящейся на обслуживании. Если
оставшееся время обслуживания заявки на приборе равно~$y$,
то с~вероятностью $d(x,y)$ первая заявка из группы становится на
обслуживание, за ней (в очередь) становятся остальные заявки группы,
затем обслуживавшаяся ранее и~остальные заявки, прежде находившиеся в~системе.
С~вероятностью $\d(x,y)\hm=1\hm-d(x,y)$ обслуживавшаяся ранее заявка продолжает
обслуживаться на приборе, вновь поступившие заявки становятся (в~очередь) за ней,
затем остальные находившиеся прежде в~системе заявки.
Недообслуженные заявки дообслуживаются.

Поскольку  интерес представляют стационарные характеристики
этой системы, всюду в~дальнейшем будем предполагать,
что стационарное распределение существует.
Критерий его существования следует
из условия конечности среднего времени
возвращения в~некоторое состояние (см.\ соотношение~\eqref{uns}).
Однако в~случае произвольных функций $B_k(x_1,\ld,x_k)$
выписать его в~простом виде не удается.

\vspace*{-4pt}

\section{Марковский случайный процесс}

\vspace*{-2pt}

Обозначим через $\nu(t)$ число заявок в~системе
в момент~$t$, а через $\vec\xi(t)\hm=
(\xi_{1}(t),\ldots,\xi_{\nu(t)}(t))$~---
вектор, координатой~$\xi_{1}(t)$ которого
является остаточное время обслуживания
заявки, находящейся в~этот момент на приборе,
$\xi_{2}(t)$~--- первой заявки в~очереди$,\ldots,$ $\xi_{\nu(t)-1}(t)$~---
последней, \mbox{$(\nu(t)-1)$-й} заявки в~очереди.
При $\nu(t)\hm=0$ вектор~$\vec\xi(t)$
не определяется.
Тогда $\eta(t)\hm=(\nu(t),\vec\xi(t))$ представляет
собой марковский процесс, описывающий
поведение числа заявок в~рассматриваемой системе.

\vspace*{-4pt}

\section{Система интегродифференциальных уравнений}

\vspace*{-2pt}

Обозначим через $p_{0}\hm=\lim\nolimits_{t\to\infty}
{\bf P}\{\nu(t)\hm=0\}$,
$P_{n}(x_1,\ldots,x_{n})
\hm= \lim\nolimits_{t\to\infty} {\bf P}\{\nu(t)\hm=n,\,
\xi_{1}(t)\hm<x_{1},\ldots,\xi_{n}(t)<x_{n}\}$, $n \hm\ge 1$,
стационарное распределение процесса~$\eta(t)$,
а~через $p_n(x_1,\ld,x_n)$~---\linebreak стационарную плот\-ность
вероятности того, что в~сис\-те\-ме~$n$~заявок, причем заявка на приборе имеет
длину~$x_1$, первая в~очереди~--- длину~$x_2$ и~т.\,д.

Выпишем систему интегродифференциальных 
уравнений, которой удовлетворяют $p_n(x_1,\ldots,x_{n})$.
Воспользовавшись методом исключения со\-сто\-яний (см., например,~\cite{n1,n2,n3,n4,n5}),
получаем следующие соотношения:
\begin{multline}
\label{e1}
-p'_1(x) = \lambda_0 p_0 \left( 
\vphantom{\mathop{\int\!\cdots\!\int}\limits_{y_1,\ld,y_{k-1}>0}}
b_1(x) +{}\right.\\
\left.{}+
\sum\limits_{k=2}^\infty\!\!
\mathop{\int\!\cdots\!\int}\limits_{y_1,\ld,y_{k-1}>0}\!\!
b_k\left(y_1,\ld,y_{k-1},x\right)\, dy_1\cdots dy_{k-1}
\!\right)-{}\\
{}- \lambda_1 p_1(x) +
\lambda_1 p_1(x)\times{}\\
{}\times \sum\limits_{k=1}^\infty
\mathop{\int\!\cdots\!\int}\limits_{y_1,\ld,y_{k}>0}
b_k\left(y_1,\ld,y_{k}\right)\, d\left(y_1,x\right)  dy_1\cdots dy_{k}
+{}\\
{}+\lambda_1 \int\limits_0^\infty p_1(y) b_1(x) \d(x,y) \, dy
+{}\\
{}+
\lambda_1 \sum\limits_{k=2}^\infty
\mathop{\int\!\cdots\!\int}\limits_{y_1,\ld,y_{k}>0}
p_1\left(y_1\right)  b_k\left(y_2,\ld,y_{k},x\right) \times{}\\
{}\times\d\left(y_2,y_1\right)
\, dy_1\cdots dy_{k}\,;
\end{multline}

\vspace*{-12pt}

\noindent
\begin{multline*}
-p'_n\left(x_1,\ld,x_n\right)
=  \lambda_0 p_0 \left(
\vphantom{\mathop{\int\!\cdots\!\int}\limits_{y_1,\ld,y_{k-1}>0}\sum\limits^\infty}
b_n\left(x_1,\ld,x_n\right) +{}\right.\\
{}+ \sum\limits_{k=n+1}^\infty
\mathop{\int\ld\int}\limits_{y_1,\ld,y_{k-n}>0}\!\!\!
b_k\left(y_1,\ld,y_{k-n},x_1,\ld\right.
\end{multline*}

\noindent
\begin{multline}
\left.\left.{}\ld,x_{n}\right)\, dy_1\cdots dy_{k-n}
\vphantom{\mathop{\int\!\cdots\!\int}\limits_{y_1,\ld,y_{k-1}>0}\sum\limits^\infty}
\!\right)\!
+\!
\sum\limits_{k=1}^{n-1}\! \lambda_k d\left(x_1,x_{n-k+1}\right) 
\times{}\\
{}\times b_{n-k}\left(x_1,\ld,x_{n-k}\right) p_k\left(x_{n-k+1},\ld,x_{n}\right)
+{}
\\
{}+
 \sum\limits_{k=1}^{n-1} \lambda_k \d(x_2,x_{1})
b_{n-k}\left(x_2,\ld,x_{n-k+1}\right)\times{}\\
{}\times
p_k\left(x_1,x_{n-k+2},\ld,x_{n}\right) -
\lambda_n p_n\left(x_1,\ld,x_n\right) +{}
\\
{}+
 \sum\limits_{k=1}^{n} \int\limits_0^\infty \lambda_k
\d\left(x_{1},y\right) b_{n-k+1}\left(x_1,\ld,x_{n-k+1}\right)\times{}\\
{}\times
p_k\left(y,x_{n-k+2},\ld,x_{n}\right)\, dy +{}
\\
{}+
\lambda_n p_n(x) \sum\limits_{k=1}^\infty
\int\limits_0^\infty b_{k1}(y)\, d(y,x) dy
+{}
\\
{}+\!
\sum\limits_{k=1}^{n-1}\! \int\limits_0^\infty\!\! \lambda_k 
d\left(y,x_{n-k+1}\right)
b_{n-k+1}\left(y,x_1,\ld,x_{n-k}\right)\times{}\\
{}\times
p_k\left(x_{n-k+1},\ld,x_{n}\right)\, dy
+{}
\\
{}+
\lambda_k \sum\limits_{k=1}^{n}
\sum\limits_{m=2}^{\infty}
\mathop{\!\int\cdots\!\int}\limits_{y_1,\ld,y_{m}>0}
\d\left(y_2,y_{1}\right)\times{}\\
{}\times
b_{n-k+m}\left(y_2,\ld,y_m,x_1,\ld,x_{n-k+1}\right)\times{}\\
{}\times
p_k\left(y_1,x_{n-k+2},\ld,x_{n}\right)\, dy_1\cdots dy_m
+{}
\\
{}+
\lambda_k \sum\limits_{k=1}^{n}
\sum\limits_{m=2}^{\infty}
\mathop{\int\!\cdots\!\int}\limits_{y_1,\ld,y_{m}>0}
d\left(y_1,x_{n-k+1}\right)\times{}\\
{}\times
b_{n-k+m}\left(y_1,\ld,y_m,x_1,\ld,x_{n-k}\right)\times{}\\
{}\times
p_k\left(x_{n-k+1},\ld,x_{n}\right)\, dy_1\cdots dy_m\,,
\enskip
n\ge2\,.
\label{e2}
\end{multline}



\noindent
К этой системе уравнений нужно добавить граничные условия,
которые удобно записать в~виде:
\begin{equation*}
%\label{(3.3)}
\lim\limits_{x\to \infty} p_{1}(x) = 0\,;                         %      \eqno(3.3)
\enskip
%\label{(3.4)}
\lim\limits_{x\to \infty} p_{n}(x,x_2,\ld,x_n) =
0\,,\enskip n\ge 2\,.               %       \eqno(3.4)
\end{equation*}

\noindent
Полученные соотношения позволяют теоретически последовательно по~$n$
вычислять совместное стационарное распределение $p_n(x_1,\ldots,x_{n})$
с точностью до вероятности~$p_0$, которая находится из условия нормировки.
Если достаточно знать только маргинальные плотности

\vspace*{2pt}

\noindent
\begin{equation*}
p_n(x) = \int\limits_0^\infty\!\! \cdots\!\! \int\limits_0^\infty
p_n\left(x,x_2,\ld,x_n\right)\, dx_2\cdots dx_n\,,
\enskip n\ge 2\,,
\end{equation*}

\noindent то полученные соотношения можно упросить.
Введем обозначения:

\vspace*{-2pt}

\noindent
\begin{multline}
\label{new6}
b_{k,m}(x)
= \mathop{\int\!\cdots\!\int} \limits_{y_1,\ld,y_{k-1}>0}\!\!
b_k\left(y_1,\ld,y_{m-1},x,y_{m},\ld\right.\\
\left.\ld,y_{k-1}\right)\, dy_1\cdots dy_{k-1}\,,
\enskip k\ge 2\,,\enskip m=\overline{1,k}\,;
\end{multline}

\noindent
\begin{equation}
\label{new7-1}
b_{2,1,2}(y,x) = b_2(y,x)\,;
\end{equation}
\begin{multline}
\label{new7}
b_{k,1,m}(y,x)
=  \mathop{\int\!\cdots\!\int}
\limits_{y_1,\ld,y_{k-2}>0}\!\!
b_k\left(y,y_1,\ld\right.\\
\left.\ld,y_{m-2},x,y_{m-1},\ld,y_{k-2}\right)
\, dy_1\cdots dy_{k-2}\,,\\
k\ge 3\,,\enskip m=\overline{2,k}\,.
\end{multline}

\noindent 
Интегрируя~\eqref{e1} и~\eqref{e2} по
$x_2,\ldots ,x_n$ в~пределах от~0 до~$\infty$ и~учитывая 
обозначения~\eqref{new6}--\eqref{new7}, получаем следующую
систему интегродифференциальных уравнений
для~$p_{n}(x)$, $n\hm\ge 1$:
\begin{align}
-p'_1(x) &=a_1(x)
-\lambda_1 p_1(x)+{}\notag\\
&{}+\lambda_1\int\limits_0^\infty
p_1(y)K(x,y)\, dy+\lambda_1p_1(x)g_{1}(x)\,, \label{sys1}
\\
-p'_n(x)&=a_n(x)-\lambda_n p_n(x)+{}\notag\\
&\hspace*{-13mm}{}+\lambda_n \int\limits_0^\infty p_n(y) K(x,y) \, dy +
\sum\limits_{k=1}^{n-1} \lambda_k \left(
\vphantom{\int\limits_0^\infty}
p_k(x) g_{n,k}(x)+ {}\right.\notag\\
&\left.{}+\int\limits_0^\infty p_k(y) G_{n,k}(x,y) \, dy
\right)\,, \enskip n \ge 2\,;
\label{sys2}
\end{align}
где
$$
a_1(x) = \lambda_0 p_0
\left( b_1(x) + \sum\limits_{k=2}^\infty b_{kk}(x) \right)\,;
$$
$$
a_n(x)= \lambda_0 p_0 \left(
b_{n1}(x) + \sum\limits_{k=n+1}^\infty b_{k,k-n+1}(x)
\right)\,;
$$
$$
K(x,y)= \d(x,y) b_{1}(x) + \sum\limits_{k=2}^{\infty}
\int\limits_0^\infty \d(z,y) b_{k,1,k}(z,x) \, dz\,;
$$
$$
g_{1}(x)= \int\limits_0^\infty
 d(y,x) b_{1}(y)\, dy, + \sum\limits_{k=2}^\infty
\int\limits_0^\infty  d(y,x) b_{k1}(y)\, dy\,;
$$
$$
g_{n,n-1}(x)= \int\limits_0^\infty \d(y,x)
b_{1}(y)\, dy\,; 
$$
$$
g_{n,k}(x)= \int\limits_0^\infty \d(y,x) b_{n-k,1}(y) \, dy\,, \enskip 
k=\overline{1,n-2}\,;
$$
$$
G_{n,n-1}(x,y)= d(x,y) b_{1}(x);
$$

\vspace*{-12pt}

\noindent
\begin{multline*}
G_{n,k}(x,y)= d(x,y) b_{n-k,1}(x) + \d(x,y) b_{n-k+1,1}(x)
+{}\\
{}+ \int\limits_0^\infty d(z,y) b_{n-k+1,1,2}(z,x)\, dz +{}
\\
{}+
\sum\limits_{m=2}^{\infty}
\int\limits_0^\infty
\left( \d(z,y)
b_{n-k+m,1,m}(z,x)
+{}\right.\\
\left.{}+ d(z,y) b_{n-k+m,1,m+1}(z,x) \right)\, dz\,, \enskip
 k=\overline{1,n-2}\,.
\end{multline*}

\noindent
Отметим, что все функции, входящие в~интегральные уравнения \eqref{sys1} и~\eqref{sys2},
являются неотрицательными.
Полученная система решается рекуррентным образом.
Граничные условия имеют вид:
$$
\lim\limits_{x\to \infty} p_{n}(x)= 0\,,\enskip
n\ge 1\,.
$$

 Сначала определяем
$p_1(x)$ из~\eqref{sys1}, затем $p_2(x)$ через $p_1(x)$ из~\eqref{sys2} 
при $n\hm=2$ и~т.\,д.
Предварительно можно выполнить замену \mbox{$p_n(x) \hm= e^{\lambda x} q_n(x)$}
и~проинтегрировать новые уравнения от~0 до~$\infty$ с~учетом граничных условий.
Чис\-лен\-ное решение может быть найдено, например, методом итераций, причем
в качестве начальной итерации удобно взять нулевое приближение.

В заключение этого раздела отметим,
что, если для функции~$d(x,y)$ известна соответству\-ющая
сепарабельная аппроксимация (см., например,~\cite{n5,n6,n7,n8}), 
в~некоторых случаях (как, например, при выполнении приводимых ниже\linebreak условий~\eqref{us1})
уравнения~\eqref{sys1} и~\eqref{sys2} сводятся к~сис\-те\-ме линейных алгебраических 
уравнений.

\section{Производящая функция}

В ряде случаев решение уравнений~\eqref{sys1} и~\eqref{sys2} может быть
найдено в~терминах производящих функций (ПФ), что облегчает нахождение моментов\linebreak
чис\-ла заявок в~сис\-те\-ме.
Разберем один из них~--- случай группового пуассоновского потока постоянной
интенсивности, в~котором длины заявок в~по\-сту\-па\-ющей
группе не зависят друг от друга и~от размера группы, т.\,е.
\begin{equation}
\left.
\begin{array}{rlrl}
\hspace*{-2mm}\lambda_k&=\lambda\,, & k&\ge 0\,;\\[6pt]
\hspace*{-2mm}B_k\left(x_1,\ld,x_k\right)&=
c_k B\left(x_1\right)\cdots B\left(x_k\right)\,, &
k &\ge 1\,,
\end{array}
\right\}
\label{us1}
\end{equation}

\noindent где $B(x)$~--- непрерывная функция распределения времени обслуживания
одной заявки на приборе, $c_k \hm\ge 0$ и~$\sum\nolimits_{k=1}^\infty c_k\hm=1$.
Необходимым и~достаточным условием существования стационарного режима 
(и~это будет дополнительно показано ниже\footnote{Этот результат  
также следует из сравнения суммарной работы в~рассматриваемой системе и~классической 
системе $M/G/1$ с~групповым входящим потоком и~обслуживанием в~порядке поступления.})
является $\lambda \overline{c}  \overline{b}\hm < 1$,
где $\overline{b}\hm=\int\nolimits_0^\infty x  dB(x)$~--- 
средняя длина поступающей заявки, а~$\overline{c}\hm=C'(1)$~--- 
средний размер поступающей группы заявок.


Введем обозначения:
\begin{gather*}
H^*(z)=\sum\limits_{n=0}^\infty  P_n z^n=
P_0+H(z)\,; \\
h(z,x)=\sum\limits_{n=1}^\infty  p_n(x) z^n\,, \enskip
C(z)= \sum\limits_{n=1}^\infty  c_n z^n\,,
\end{gather*}

\noindent где $P_n\hm=P_n(\infty,\ld,\infty)$, $n \hm\ge 1$.
Умножив уравнение~\eqref{sys1} на~$z$, а~\eqref{sys2}~---
на~$z^n$, просуммировав и~проинтегрировав
с~учетом граничного условия $\lim\nolimits_{x\to \infty} h(z,x)\hm= 0$,
получаем уравнение:
\begin{multline}
h(z,x) = \lambda p_0 (1-B(x)) \fr{z (1-C(z))}{1-z}
+ {}\\[2pt]
{}+
\lambda (1-B(x))  H(z) \left ( C(z) + c_1 + \fr{z^2 - C(z)}{z(1-z)} \right )
- {}
\\[2pt]
{}- \lambda (1- C(z)) \left(
\int\limits_x^\infty \int\limits_0^\infty \d(t,y) h(z,y)\, dy  dB(t) -{}\right.\\[2pt]
\left.{}-
\int\limits_0^\infty \int\limits_x^\infty \d(t,y) h(z,y)\, dy  dB(t)
\right)
+{}
\\[2pt]
{}+
\lambda \fr{C(z)-c_1 z}{z} \int\limits_x^\infty
\int\limits_0^\infty h(z,y) \left(
\vphantom{\int\limits_0^\infty}
\d(t,y) +{}\right.\\[2pt]
\left.{}+ \int\limits_0^\infty d(u,y) \,dB(u) \right)
dy  dB(t)\,.
\label{pf}
\end{multline}

\noindent В случае ординарного потока ($c_1 \hm\equiv 1$) из этого уравнения 
немедленно следует ПФ
числа заявок в~системе, рассмотренной в~\cite{n3}. Трактуя~$z$ как параметр,
для решения уравнения~\eqref{pf} можно применить метод, описанный 
в~предыдущем разделе.

Задачу нахождения моментов стационарного распределения
числа заявок в~системе рассмотрим на примере математического ожидания
и~ограничимся лишь описанием алгоритма его расчета.

Будем считать, что операции дифференцирования, которые будут применены ниже, законны.
Проинтегрируем~\eqref{pf} по~$x$ от~0 до~$\infty$ и~найдем~$H(z)$.
Продифференцировав выражение для $(1-z)H(z)$ два раза и~положив $z\hm=1$, получим формулу
для расчета среднего числа заявок $\mathbf{E}\nu$ в~системе с~двумя
неизвестными: $h(1,x)$ и~$h'(1,x)\hm=\partial h(z,x) / \partial z |_{z=1}$.
Их нахождение осуществляется в~два этапа.
Сначала выписывается выражение для~$H(1)$,
затем, подставив $z=1$ и~найденное выражение для~$H(1)$
в~\eqref{pf},\linebreak получается интегральное уравнение для $h(1,x)$,
чис\-лен\-ное решение которого можно найти, например, как и~выше,
итерационным методом. Таким же образом, но предварительно продифференцировав~\eqref{pf} 
по~$z$, находится и~уравнение для~$h'(1,x)$.

Необходимым условием существования~$\mathbf{E}\nu$ является условие:

\pagebreak



\noindent
\begin{equation}
\label{sred}
\overline{c}
\int\limits_0^\infty
\int\limits_0^\infty
\int\limits_x^\infty
\d(t,y)
(1-B(y)) \,dy  dB(t)
dx < \infty\,.
\end{equation}


\noindent Показать это можно так же, как и~для системы из~\cite{n3}.
Предположим, что $\mathbf{E}\nu\hm< \infty$. Тогда $(H^*(1))'\hm=H'(1)<\infty$.
Поскольку
\begin{equation}
\label{hzx}
h(z,x)\le h(1,x)\le \lambda \overline{c} (1-B(x)) \,,
\end{equation}
то
$$
\int\limits_0^\infty
\int\limits_x^\infty
\int\limits_0^\infty
\d(t,y) h(z,y) \, dy 
dB(t) dx
\le \lambda  \overline{c}  \overline{b}^2 < \infty\,.
$$

\noindent Интегрируя теперь~\eqref{pf} по~$x$ от~0 до~$\infty$,
получаем:
\begin{multline*}
\fr{ \lambda \overline{b} (1-z H^*(z))}{ 1-z}+
\left ( z(1-C(z))(1-z) \right )^{-1}\times{}\\
{}\times \left \{
\vphantom{\int\limits_0^\infty}
H(z)(1-z)z - \lambda \overline{b} z (1-C(z))
\right.
+{}
\\
{}+
\lambda(1-z) (C(z)-c_1 z)\times{}\\
{}\times
\left[
\int\limits_0^\infty
\int\limits_x^\infty
\int\limits_0^\infty
h(z,y) \left(
\d(t,y) + {}\right.\right.\\
\left.\left.\left.{}+\int\limits_0^\infty d(u,y) \, dB(u)
\right)\,
dy  dB(t)
dx -
\overline{b} H(z)
\right ] \right\}
+ {}\\
{}+ \lambda \int\limits_0^\infty
\int\limits_x^\infty \int\limits_0^\infty
\d(t,y) h(z,y) \,dy dB(t) dx
={}\\
{}= \lambda \int\limits_0^\infty \int\limits_0^\infty
\int\limits_x^\infty \d(t,y) h(z,y) \,dy  dB(t) dx\,.
%\label{neob}
\end{multline*}

\noindent Поскольку левая часть ограничена и~$h(z,x) \hm\rightarrow h(1,x)$ 
при $z \hm\rightarrow 1$,
воспользовавшись теоремой Фату и~учитывая~\eqref{hzx}, приходим к~\eqref{sred}.

Достаточность показать сложнее, 
и~ввиду громоздкости выкладок на этом здесь останавливаться не будем.
Заметим, что для выполнения~\eqref{sred} достаточно (помимо конечности среднего
размера группы) существования у~распределения времени обслуживания~$B(x)$ 
второго момента.

\section{Время пребывания заявки в~системе}

Вернемся к~основной исследуемой системе.
Расчет временных характеристик поступающих в~сис\-те\-му заявок
начинается с~нахождения периода занятости (ПЗ) и~его характеристик.
Обозначим через $u_n(s;x)$, $n\hm\ge 1$, ПЛС функции распределения
времени до того момента,
когда в~системе останется $(n\hm-1)$ заявок при условии,
что на приборе начала обслуживаться заявка длины~$x$
и~в~системе находилось~$n$~заявок.
Уравнение для $u_n(s;x)$ получается из следующих рассуждений:
за время обслуживания заявки длины~$x$
с~вероятностью $e^{- \lambda_n x}$ не поступит больше ни одной
заявки, а~с~вероятностью $\lambda_n e^{- \lambda_n t}\,dt$
на интервале времени $[t,t+dt]$ может поступить
группа размером $k\hm\ge1$. В~первом случае ПЛС
равно~$e^{-sx}$, а~во втором зависит от размера
поступающей группы и~того, произошла смена заявки на приборе
или нет (и~в~каждом случае необходимо
дождаться окончания обслуживания исходной заявки длины~$(x\hm-t)$ и~$k$~новых заявок).
Рассматривая все возможные события и~воспользовавшись свойствами ПЛС, получаем:
\begin{multline}
u_{n}(s;x) = e^{- (\lambda_n+s) x}
+{}
\\
{}+
\sum\limits_{k=1}^\infty
\int\limits_0^x \lambda_n e^{- (\lambda_n+s) t} \, dt
\mathop{\int\!\cdots\!\int}\limits_{y_1,\ld,y_{k}>0}
d\left(y_1,x-t\right)\times{}\\
{}\times u_n(s;x-t)
\prod\limits_{j=1}^k u_{n+k+1-j}\left(s;y_j\right) B_k\left(dy_1,\dots,dy_k\right)
+{}
\\
{}+
\sum\limits_{k=1}^\infty \int\limits_0^x
\lambda_n e^{- (\lambda_n+s) t} \, dt
\mathop{\int\!\cdots\!\int}\limits_{y_1,\ld,y_{k}>0}
\d\left(y_1,x-t\right)\times{}\\
{}\times u_{n+k}(s;x-t) 
\prod\limits_{j=1}^k u_{n+k-j}\left(s;y_j\right)\times{}\\
{}\times
B_k\left(dy_1,\dots,dy_k\right)\,.
\label{uns}
\end{multline}

\noindent
Решение этого интегрального уравнения в~явном виде для произвольных 
функций $B_k(x_1,\ld,x_k)$
получить не удается. Однако в~некоторых частных случаях оно разрешимо, как, например,
в~случае условий~\eqref{us1}. Здесь $u_{n}(s;x)$ не зависит
от~$n$ и,~как нетрудно вывести из~\eqref{uns},
$$
u_n(s;x)=u(s;x)=e^{-\left ( \lambda + s - \lambda C(u(s))\right ) x}\,,
$$
где $\beta(s)$~--- ПЛС функции распределения~$B(x)$,
а~$u(s)$ является корнем уравнения:
$$
u(s) = \beta \left (\lambda+s-\lambda C (u(s)) \right )\,.
$$

Кроме того, ПЛС $u^*(s)$ функции распределения
ПЗ\footnote{Для основной
исследуемой системы ПЛС ПЗ равен $\sum\nolimits_{k=1}^\infty
\mathop{\int\!\cdots\!\int}\nolimits_{y_1,\ld,y_{k}>0}
\prod\nolimits_{n=1}^k u_{n}(s;y_{k-n+1})
B_k(dy_1,\dots,dy_k)$.}
удовлетворяет уравнению $u^*(s)\hm=C\left ( \beta(\lambda\hm+s\hm-\lambda u^*(s)) \right)$.

Для нахождения распределений времен ожидания начала обслуживания 
и~пребывания в~системе введем следующие функции:
\begin{description}
\item[\,] ${\tilde B}(k,i,x)$~--- вероятность того, что пришла группа из~$k$~заявок 
и~$i$-я заявка в~группе имеет длину меньше~$x$:
\begin{multline*}
{\tilde B}(k,i,x)=B_k(\infty,\dots,\infty,x,\infty, \dots, \infty)\,,\\ 
k \ge 1\,,\enskip 1 \le i \le k\,;
\end{multline*}

\item[\,] ${\bar B}(x_1,\dots,x_{i-1};k,i,x)$~---
условная 
вероятность\footnote{Здесь производная понимается
как производная Ра\-до\-на--Ни\-ко\-ди\-ма.} того, что первая заявка имеет
длину меньше~$x_1$, вторая~--- меньше~$x_2$, $\dots$, $(i-1)$-я~--- меньше $x_{i-1}$,
при условии, что пришла группа из~$k$~заявок, причем заявка на $i$-м месте имеет
длину~$x$:
\begin{multline*}
{\bar B}(x_1,\dots,x_{i-1};k,i,x)={}\\
{}=\fr{d_x B_k(x_1,\dots,x_{i-1},x,\infty, 
\dots, \infty)}{d {\tilde B}(k,i,x)}\,;
\end{multline*} 

\item[\,] ${\hat B}(x)$~---  среднее число заявок длины
меньше~$x$ в~поступающей группе:
$$
{\hat B}(x) = \sum\limits_{k=1}^\infty 
\sum\limits_{i=1}^k {\tilde B}(k,i,x)\,;
$$

\item[\,] ${\hat B}(k,i;x)$~---
условная вероятность того, что поступила группа из~$k$~заявок, среди них
есть ровно одна заявка длины~$x$ и~она находится на $i$-м месте, при условии что
поступила группа, в~которой имеются заявки длины~$x$:
$$
{\hat B}(k,i;x)=\fr{d_x {\tilde B}(k,i,x)}{{\hat B}(x)}\,, \enskip
k \ge 1,\enskip 1 \le i \le k\,.
$$

\end{description}

Определим сначала ПЛС $\omega_{k1}(s;x)$ функции распределения
времени ожидания начала обслуживания заявки длины~$x$ при условии,
что она поступила в~группе размера $k\hm\ge$ и~была на первом месте в~группе.
Ее время ожидания равно нулю, если она застала систему свободной 
и~если она, застав на приборе заявку длины~$y$, заняла ее место.
Если же она застала в~системе~$n$, $n\hm\ge 1$, заявок, на приборе~---
заявку длины~$y$ и~не заняла ее место,
то время ожидания совпадает с~ПЗ, открываемого заявкой длины~$y$,
когда в~системе находится $(n+k)$ заявок, т.\,е.\ $u_{n+k}(s;y)$. 
В~терминах ПЛС имеем
\begin{multline*}
\omega_{k1}(s;x) = p_0 + \sum\limits_{n=1}^\infty
\int\limits_0^\infty p_n(y) \left( 
\vphantom{'d}
d(x,y) +{}\right.\\
\left.{}+ \d(x,y) u_{n+k}(s;y) \right )\,
dy\,, \enskip k \ge 1\,.
\end{multline*}

Перейдем к~ПЛС времени
ожидания начала обслуживания заявки длины~$x$,
поступившей в~группе из~$k$~заявок ($k\hm\ge2$)
и~занимающей в~группе $i$-е мес\-то ($2 \hm\le i\hm\le k$).
В~случае поступления в~пустую систему
время ожидания совпадает с~суммарной дли\-тель\-ностью $(i-1)$-го
ПЗ, первый из которых от\-кры\-вается заявкой длины~$x_1$,
второй~--- $x_2$ и~т.\,д., и~в~терми\-нах ПЛС
равно $u_{k}(s;x_1)\cdots u_{2}(s;x_{i-1})$.
Длительности соответствующих ПЗ необходимо добавить к~времени
ожидания, когда по\-сту\-па\-ющая группа застает сис\-те\-му занятой.
В~итоге, вводя обозначение 

\noindent
$$
{\tilde u}_{nk}(s;x_1,\dots,x_{i-1})
=u_{n+k}(s;x_1)\cdots u_{n+2}(s;x_{i-1})\,,
$$
выражение для ПЛС функции распределения времени ожидания начала обслуживания
$\omega_{ki}(s;x_1, \dots, x_{i-1},x)$ заявки длины~$x$,
поступившей в~группе из~$k$~заявок и~занимающей в~группе $i$-е место,
можно записать так:

\noindent
\begin{multline*}
\omega_{ki}(s;x_1, \dots, x_{i-1},x) =
p_0 {\tilde u}_{0k}(s;x_1,\dots,x_{i-1})
+{}
\\
{}+
\sum\limits_{n=1}^\infty \int\limits_0^\infty p_n(y) \left (
d\left(x_1,y\right) {\tilde u}_{nk}\left(s;x_1,\dots,x_{i-1}\right)
+ {}\right.\\
\left.{}+\d\left(x_1,y\right) u_{n+k}(s;y)
{\tilde u}_{n-1,k}\left(s;x_1,\dots,x_{i-1}\right)
\right )\,dy\,,
\\
 k \ge 2\,, \enskip 2 \le i \le k\,.
\end{multline*}

\noindent
Легко видеть, что если
интенсивность входящего потока не зависит от числа заявок в~системе,
то при фиксированном~$i$ все $u_n(s;x_i)$ равны между собой
и~выражение $\omega_{ki}(s;x_1, \dots, x_{i-1},x)$ приводится к~виду:

\noindent
\begin{multline*}
\omega_{ki}(s;x_1, \dots, x_{i-1},x)={}\\
{}=
\omega_{k1}(s;x_1)u(s;x_1)\cdots u(s;x_{i-1})\,,
\end{multline*}
т.\,е.\ не зависит от числа заявок в~группе, а~только от места выделенной
заявки в~группе (и,~конечно, остаточных длин заявок, стоящих перед ней).

Теперь можно выписать ПЛС распределений,
связанных с~временем ожидания начала обслуживания и~пребывания в~системе.
Условно стационарное распределение времени ожидания
начала обслуживания заявки длины~$x$
при условии, что всего в~группе поступило $k\hm\ge2$ заявок
и~заявка длины~$x$ находится на $i$-м месте ($2 \hm\le i \hm\le k$),
имеет ПЛС, задаваемое выражением:

\noindent
\begin{multline}
\omega_{ki}(s;x)
=
\int\limits_0^\infty\!\cdots\!\int\limits_0^\infty
\omega_{ki}\left(s;x_1, \dots, x_{i-1},x\right)\times{}\\
{}\times
{\bar B}\left(dx_1,\dots,dx_{i-1};k,i,x\right)\,.
\label{w1}
\end{multline}

\noindent
Усредняя $\omega_{ki}(s;x)$ по распределению
${\hat B}(k,i;x)$, получаем формулу для ПЛС
$\omega(s;x)$ функции распре-\linebreak\vspace*{-12pt}

\pagebreak

\noindent
деления времени
ожидания начала обслуживания
заявки длины~$x$:
\begin{equation}
\label{w2}
\omega(s;x)
= \sum\limits_{k=1}^\infty \sum\limits_{i=1}^k \omega_{ki}(s;x) {\hat B}(k,i;x)\,.
\end{equation}


\noindent
Безусловное ПЛС~$\omega(s)$ функции распределения времени
ожидания начала обслуживания определяется путем усреднения по длине заявки:
\begin{equation}
\label{w3}
\omega(s)= \int\limits_0^\infty
\omega(s;x) d{\hat B}(x) ({\hat B}(\infty))^{-1}\,.
\end{equation}

\noindent В случае условий~\eqref{us1} все упрощается
и~формулу для $\omega(s;x)$ можно привести к~виду:
\begin{multline}
\omega(s;x)
= \fr{1}{\overline{c}}
\left (
\vphantom{\int\limits_{0}^\infty}
\omega^*(s;x) + {}\right.\\
\hspace*{-2mm}\left.{}+\fr{u(s)-C \left ( u(s) \right )}{u(s) (1- u(s))}
\int\limits_{0}^\infty \omega^*(s;y) u(s,y) b(y)\, dy \right ),\!
\label{sc1}
\end{multline}

\noindent где
$$
\omega^*(s;x)= p_0 + \int\limits_0^\infty
h(1,y) \left ( d(x,y) + \d(x,y) u(s;y)
\right ) \,dy\,.
$$

Аналогичным образом находится и~ПЛС $\phi(s;x)$ функции распределения
времени пребывания заявки длины~$x$ в~системе и~безусловное ПЛС~$\phi(s)$.
Обозначим через $\phi_{ki}(s;x_1, \dots, x_{i-1},x)$, $k \hm\ge 1$, $1\hm\le i \hm\le k$,
ПЛС функции распределения времени пребывания в~системе заявки длины~$x$,
поступившей в~группе из~$k$~заявок и~занимающей в~группе $i$-е место.
При $i\hm=1$ аргумент~$x_0$ опускается, т.\,е.\
 $\phi_{k1}(s;x_{0},x)\hm=\phi_{k1}(s;x)$.
Тогда
\begin{multline*}
\phi_{ki}(s;x_1, \dots, x_{i-1},x)
={}\\
{}=p_0 {\tilde u}_{0k}\left(s;x_1,\dots,x_{i-1}\right)u_{1}(s;x)
+ 
\sum\limits_{n=1}^\infty
\int\limits_0^\infty p_n(y)\times{}\\
{}\times \left(
d\left(x_1,y\right) {\tilde u}_{nk}\left(s;x_1,\dots,x_{i-1}\right)
u_{n+1}(s;x) +{}\right.
\\
{}+
\d\left(x_1,y\right) u_{n+k}(s;y) {\tilde u}_{n-1,k}\left(s;x_1,\dots,x_{i-1}\right)\times{}\\
\left.{}\times
u_{n}(s;x) %\vhpahntom*{\tilde{u}}
\right)\,dy\,,  \enskip 
k \ge 1\,, \enskip 1 \le i \le k\,.
\end{multline*}

\noindent Переход к~ПЛС $\phi(s;x)$ и~$\phi(s)$ осуществляется
по формулам~\eqref{w1}--\eqref{w3}. Если выполняются условия~\eqref{us1}, то,
вспоминая, что время пребывания заявки в~системе складывается
из времени ожидания начала обслуживания и~времени пребывания на приборе,
получаем $\phi(s;x)\hm=\omega(s;x)u(s;x)$. Дифференцируя эту формулу 
с~учетом~\eqref{sc1} необходимое число раз, нетрудно получить моменты
времени пребывания  в~системе заявки  длины~$x$.


\section{Заключение}

\vspace*{-3pt}

Используя результаты предыдущего раздела,
можно также найти совместные распределения ПЗ 
и~числа обслуженных на приборе заявок, или интервала времени,
когда в~системе находилось не менее~$n$~заявок, 
или этих обеих случайных величин и~т.\,п.

В практическом плане интерес в~дальнейшем представляет разбор
 частных случаев, т.\,е.\ анализ
стационарных характеристик системы в~различных предположениях 
о~зависимости размеров заявок внутри группы; в~теоретическом~---
обобщение полученных результатов
на случай более общего группового входящего потока, когда
в~каждой поступающей группе могут находиться подгруппы заявок
одинаковой или различной длины.

\vspace*{-18pt}

{\small\frenchspacing
 {%\baselineskip=10.8pt
 \addcontentsline{toc}{section}{References}
 \begin{thebibliography}{99}
 
 \vspace*{-3pt}
 
 \bibitem{n3}  %1
\Au{Нагоненко В.\,А.}
О~характеристиках одной нестандартной системы
массового обслуживания.~I, II~//
Изв.\ АН СССР. Технич.\ кибернет., 1981.
№\,1. С.~187--195; №\,3. С.~91--99.

\bibitem{n4} %2
\Au{Печинкин А.\,В.} Об одной
инвариантной системе массового обслуживания~//
Math.\ Operationsforsch.\ Statist.
Ser.\ Optimization, 1983. Vol.~14. No.\,3. P.~433--444.

\bibitem{n5} %3
\Au{Милованова Т.\,А., Печинкин~А.\,В.}
Стационарные характеристики системы обслуживания 
с~инверсионным порядком обслуживания, вероятностным
приоритетом и~гистерезисной политикой~//
Информатика и~её применения, 2013. Т.~7. Вып.~1. С.~22--36.




\bibitem{n1}  %4
\Au{Мейханаджян Л.\,А., Милованова~Т.\,А., Печинкин~А.\,В., Разумчик~Р.\,В.}
Стационарные вероятности состояний в~системе обслуживания с~инверсионным 
порядком обслуживания и~обобщенным вероятностным приоритетом~// Информатика 
и~её применения, 2014. Т.~8. Вып.~3. С.~16--26.

\bibitem{n2} %5
\Au{Мейханаджян Л.\,А., Милованова~Т.\,А., Разумчик~Р.\,В.}
Время ожидания в~системе обслуживания с~инверсионным 
порядком обслуживания и~обобщенным
вероятностным приоритетом~// Информатика и~её применения, 2015. Т.~9. Вып.~2. С.~14--22.

\bibitem{n0} %6
\Au{Razumchik R.} On ${M/G/1}$ queue with state-dependent heterogeneous 
batch arrivals, inverse service order and probabilistic priority~// 
AIP Conf. Proc., 2017. Vol.~1863. No.\,1. P.~090006-1--090006-3.

\bibitem{nm1} %7
\Au{Милованова Т.\,А.} Система $\mathrm{BMAP}/G/1/\infty$ с~инверсионным 
порядком обслуживания и~вероятностным приоритетом~// Автомат. телемех., 2009. 
№\,5. С.~155--168.
% Autom. Remote Control, 70:5 (2009), 885-896


\bibitem{gg2} %8
\Au{Bent N.} On a~queuing model where potential customers are discouraged by queue length~//
Scand. J.~Stat., 1975. Vol.~2. Iss.~1. P.~34--42.




\bibitem{gg3} %9
\Au{Печинкин А.\,В.} Система ${M_k/G/1}$ с~ненадежным прибором~//
Автомат. телемех., 1996. №\,9. С.~100--110.
% Autom. Remote Control, 57:9 (1996), 1302-1310



\bibitem{gg5} %10
\Au{Gupta U.\,C., Srinivasa~Rao~T.\,S.\,S.} 
On the analysis of single server finite queue with state dependent
arrival and service processes: ${M_n/G_n/1/K}$~//
OR Spektrum, 1998. Vol.~20. Iss.~2. P.~83--89.

\bibitem{gg4} %11
\Au{Kerner Y.} The conditional distribution of the residual service time in the ${M_n/G/1}$ 
queue~// Stoch. Models, 2008. Vol.~24. Iss.~3. P.~364--375.

\bibitem{gg1} %12
\Au{Abouee-Mehrizi H., Baron~O.} State-dependent ${M/G/1}$ queueing systems~//
Queueing Sy., 2016. Vol.~82. Iss.~1-2. P.~121--148.

\columnbreak

\bibitem{n6} %13
\Au{Поспелов В.\,В.}
О~погрешности приближения функции двух переменных суммами
произведений функций одного переменного~//
Ж.~вычисл. матем. матем. физ., 1978. Т.~18. Вып.~5. С.~1307--1308.
%U.S.S.R. Comput. Math. Math. Phys., 18:5 (1978), 228-230

\vspace*{7pt}

\bibitem{n8} %14
\Au{Uschmajew A.} Regularity of tensor product approximations to square
integrable functions~// Constr. Approx., 2011. Vol.~34. Iss.~3. P.~371--391.

\vspace*{7pt}

\bibitem{n7} %15
\Au{Townsend A., Trefethen~L.\,N.} An extension of Chebfun to two dimensions~// 
SIAM J.~Sci. Comput., 2013. Vol.~35. Iss.~6. P.~495--518.



 \end{thebibliography}

 }
 }

\end{multicols}

\vspace*{-3pt}

\hfill{\small\textit{Поступила в~редакцию 19.09.17}}

\vspace*{8pt}

%\newpage

%\vspace*{-24pt}

\hrule

\vspace*{2pt}

\hrule

%\vspace*{8pt}


\def\tit{$M/G/1$ QUEUE WITH~STATE-DEPENDENT 
HETEROGENEOUS BATCH ARRIVALS, 
INVERSE SERVICE ORDER, AND~PROBABILISTIC~PRIORITY}

\def\titkol{$M/G/1$ queue with~state-dependent 
heterogeneous batch arrivals, 
inverse service order, and~probabilistic priority}

\def\aut{R.\,V.~Razumchik$^{1,2}$}

\def\autkol{R.\,V.~Razumchik}

\titel{\tit}{\aut}{\autkol}{\titkol}

\vspace*{-9pt}


\noindent
$^1$Institute of Informatics Problems, Federal Research Center ``Computer Science 
and Control'' of the Russian\linebreak
$\hphantom{^1}$Academy of Sciences, 44-2~Vavilov Str., Moscow 
119333, Russian Federation

\noindent
$^2$Peoples' Friendship University of Russia (RUDN University), 
6~Miklukho-Maklaya Str., Moscow 117198, Russian\linebreak
$\hphantom{^1}$Federation



\def\leftfootline{\small{\textbf{\thepage}
\hfill INFORMATIKA I EE PRIMENENIYA~--- INFORMATICS AND
APPLICATIONS\ \ \ 2017\ \ \ volume~11\ \ \ issue\ 4}
}%
 \def\rightfootline{\small{INFORMATIKA I EE PRIMENENIYA~---
INFORMATICS AND APPLICATIONS\ \ \ 2017\ \ \ volume~11\ \ \ issue\ 4
\hfill \textbf{\thepage}}}

\vspace*{3pt}






\Abste{Consideration is given to the stationary characteristics
of single-server queues with the queue of infinite capacity,
independent and identically-distributed service times, 
LCFS (last-come-first-served) service order, and probabilistic priority discipline.
Most of the results for such type of queueing systems
have been obtained under the 
assumption of either Poisson arrivals or 
phase-type arrivals. 
Another important assumption made was that
the arrival process is independent 
from the system state. The author shows that the
latter assumption can be relaxed to some, quite large extent.
The author considers an $M/G/1/\infty$ queue  
with batch Poisson arrival flow in which ($i$)~the arrival rate depends
on the total number of customers present in the system
at the arrival instant; and ($ii$)~the size of the arriving batch~$k$ 
and the remaining service times $x_1,\dots,x_k$ of the customers in the batch
have the arbitrary continuous joint probability distribution
$B_k(x_1,\ld,x_k)$. The author obtains analytic expressions
for the computation of the joint stationary distribution 
of the total number of customers in the system 
and their remaining service times. 
Busy period, waiting and sojourn time distributions
are also given in terms of the Laplace--Stieltjes transforms.}

\KWE{queueing system; LIFO; probabilistic priority; batch 
arrival; state-dependent Poisson flow}


  \DOI{10.14357/19922264170402} 

\vspace*{-16pt}

\Ack
\noindent
This work was supported by the
Russian Science Foundation (grant 16-11-10227).



\vspace*{1pt}

  \begin{multicols}{2}

\renewcommand{\bibname}{\protect\rmfamily References}
%\renewcommand{\bibname}{\large\protect\rm References}

{\small\frenchspacing
 {%\baselineskip=10.8pt
 \addcontentsline{toc}{section}{References}
 \begin{thebibliography}{99}
 
 \vspace*{-3pt}
 
 \bibitem{Xx3-1} %1
\Aue{Nagonenko, V.\,A.} 1981.
O~kharakteristikakh odnoy nestandartnoy sistemy
massovogo obsluzhivaniya
[On the characteristics of one nonstandard queuing
system].~I, II.
\textit{Izv.\ AN SSSR. Tekhnich.\ kibernet}
[Proceedings of the Academy of Sciences of
the USSR. Technical Cybernetics]
1:187--195; 3:91--99.

\bibitem{Xx4-1} %2
\Aue{Pechinkin, A.\,V.} 1983.
Ob odnoy invariantnoy sisteme massovogo
obsluzhivaniya
[On an invariant queuing system].
\textit{Math.\ Operationsforsch.\ Statist.
Ser.\ Optimization} 14(3):433--444.

\bibitem{Xx5-1} %3
\Aue{Milovanova, T.\,A., and A.\,V.~Pechinkin.} 2013.
Sta\-tsi\-o\-nar\-nye kharakteristiki sistemy obsluzhivaniya
s~in\-ver\-si\-on\-nym poryadkom obsluzhivaniya,
veroyatnostnym pri\-ori\-te\-tom i~gisterezisnoy politikoy
[Stationary characteristics of queuing system with
an inversion procedure service probabilistic priority
and hysteresis policy].
\textit{Informatika i~ee Primeneniya~--- Inform. Appl.} 7(1): 22--35.





\bibitem{Xx1-1} %4
\Aue{Meykhanadzhyan, L.\,A., T.\,A.~Milovanova, A.\,V.~Pechinkin, 
and R.\,V.~Razumchik.} 2014.
Sta\-tsi\-o\-nar\-nye veroyatnosti sostoyaniy v~sisteme obsluzhivaniya s~inversionnym 
poryadkom ob\-slu\-zhi\-va\-niya i~obob\-shchen\-nym veroyatnostnym prioritetom
[Stationary distribution in a~queueing system with inverse service order and
generalized probabilistic priority].
\textit{Informatika i~ee Primeneniya~--- Inform. Appl.} 8(3):16--26.

\bibitem{Xx2-1} %5
\Aue{Meykhanadzhyan, L.\,A., T.\,A.~Milovanova, and R.\,V.~Razumchik.} 2015.
Vremya ozhidaniya v~sisteme obsluzhivaniya s inversionnym poryadkom obsluzhivaniya 
i~obobshchennym veroyatnostnym prioritetom
[Stationary waiting time in a~queueing system with inverse service order and 
generalized probabilistic priority].
\textit{Informatika i~ee Primeneniya~--- Inform. Appl.} 9(2):14--22.

\bibitem{Xn0-1} %6
\Aue{Razumchik, R.} 2017. On ${M/G/1}$ queue with state-dependent heterogeneous
 batch arrivals, inverse service order and probabilistic priority. 
 \textit{AIP Conf. Proc}. 1863(1):090006-1--090006-3.
 
 \bibitem{Xnm1-1} %7
\Aue{Milovanova, T.\,A.} 2009. 
${\mathrm{BMAP}/G/1/\infty}$ system with last come first served probabilistic priority.
\textit{Automat. Rem. \mbox{Contr.}} 70(5):885--896.

\bibitem{Xgg2-1} %8
\Aue{Bent, N.} On a~queuing model where potential customers are discouraged by queue length.
\textit{Scand. J.~Stat.} 2(1):34--42.

\bibitem{Xgg3-1} %9
\Aue{Pechinkin, A.\,V.} 1996. Sistema ${M_k/G/1}$ 
s~nenadezhnym priborom [An ${M_k/G/1}$  system with an unreliable device].
\textit{Avtomat. telemekh.} [Autom. Rem. Contr.] 9:100--110.



\bibitem{Xgg5-1} %10
\Aue{Gupta, U.\,C., and T.\,S.\,S.~Srinivasa Rao.} 1998.
On the analysis of single server finite queue with state dependent
arrival and service processes: ${M_n/G_n/1/K}$.
\textit{OR Spektrum} 20(2):83--89.

\bibitem{Xgg4-1}  %11
\Aue{Kerner, Y.} 2008. The conditional distribution of 
the residual service time in the ${M_n/G/1}$ queue.
\textit{Stoch. Models} 24(3):364--375.

\bibitem{Xgg1-1}  %12
\Aue{Abouee-Mehrizi, H., and O.~Baron.} 2016.
State-dependent ${M/G/1}$ queueing systems. 
\textit{Queueing Sy.} 82(1-2):121--148. 

\bibitem{Xn6-1} %13
\Aue{Pospelov, V.\,V.} 1978.
O~pogreshnosti priblizheniya funktsii dvukh peremennykh summami
proizvedeniy funktsiy odnogo peremennogo
[The error of approximation of a~function of two variables by sums 
of the products of functions of one variable]
\textit{Zh. vichisl. matem. matem fiz.}
[USSR\ Comput. Math. Math. Phys.] 18(5):1307--1308.

\bibitem{Xn8-1} %14
\Aue{Uschmajew, A.} 2011. Regularity of tensor product approximations to square
integrable functions.
\textit{Constr. Approx.} 34(3):371--391.

\bibitem{Xn7-1} %15
\Aue{Townsend, A., and L.\,N.~Trefethen.} 2013. 
An extension of Chebfun to two dimensions.
\textit{SIAM J.~Sci. Comput.} 35(6):495--518.


\end{thebibliography}

 }
 }

\end{multicols}

\vspace*{-6pt}

\hfill{\small\textit{Received September 19, 2017}}

%\vspace*{-10pt}

\Contrl

\noindent
\textbf{Razumchik Rostislav V.} (b.\ 1984)~--- 
Candidate of Science (PhD) in physics and mathematics, leading scientist, 
Institute of Informatics Problems, Federal Research Center 
``Computer Science and Control'' of the Russian Academy of Sciences, 
44-2~Vavilov Str., Moscow 119333, Russian Federation; 
associate professor, Peoples' Friendship University of Russia (RUDN University), 
6~Miklukho-Maklaya Str., Moscow 117198, Russian Federation; 
\mbox{rrazumchik@ipiran.ru}
%\mbox{razumchik\_rv@rudn.university}
\label{end\stat}


\renewcommand{\bibname}{\protect\rm Литература} 