\renewcommand{\figurename}{\protect\bf Figure}
\renewcommand{\tablename}{\protect\bf Table}

\def\stat{pagano}

\def\tit{STUDY OF THE~MMPP/GI/$\infty$ QUEUEING SYSTEM\\ WITH~RANDOM CUSTOMERS' CAPACITIES}

\def\titkol{Study of the~MMPP/GI/$\infty$ queueing system with random customers' capacities}

\def\autkol{E.~Lisovskaya, S.~Moiseeva,   M.~Pagano, and~V.~Potatueva}

\def\aut{E.~Lisovskaya$^{1}$, S.~Moiseeva$^{1}$,   M.~Pagano$^{2}$, and~V.~Potatueva$^{1}$}

\titel{\tit}{\aut}{\autkol}{\titkol}

\index{Lisovskaya E.}
\index{Moiseeva S.}
\index{Pagano M.}
\index{Potatueva V.}
\index{Лисовская Е.\,Ю.}
\index{Моисеева С.\,П.}
\index{Пагано М.}
\index{Потатуева В.\,В.}

%{\renewcommand{\thefootnote}{\fnsymbol{footnote}}
%\footnotetext[1] {This work was supported in part by the
%Russian Foundation for Basic Research (grants 15-07-03007 and 13-07-00223).}}

\renewcommand{\thefootnote}{\arabic{footnote}}
\footnotetext[1]{Tomsk State University, 36~Lenin ave., Tomsk 634050, Russian Federation}
\footnotetext[2]{University of Pisa, 16~Via Caruso, Pisa 56122, Italy}


\vspace*{12pt}

\def\leftfootline{\small{\textbf{\thepage}
\hfill INFORMATIKA I EE PRIMENENIYA~--- INFORMATICS AND APPLICATIONS\ \ \ 2017\ \ \ volume~11\ \ \ issue\ 4}
}%
 \def\rightfootline{\small{INFORMATIKA I EE PRIMENENIYA~--- INFORMATICS AND APPLICATIONS\ \ \ 2017\ \ \ volume~11\ \ \ issue\ 4
\hfill \textbf{\thepage}}}

\Abste{A~queueing system with an infinite number of 
servers is considered. Customers arrive in the system according to a~Markov 
Modulated Poisson Process (MMPP). Each customer carries a~random quantity of 
work (capacity of the customer). In this study, service time does not depend 
on the customers' capacities; the latter are used just to fix some additional 
features of the system's evolution. It is shown that the joint probability 
distribution of the customers' number and total capacities in the system is 
two-dimensional Gaussian under the asymptotic condition of an infinitely 
growing service time. 
Simulation results allow determining the applicability area of the asymptotic result.}

\KWE{infinite-server queueing system; random capacity of customers; Markov Modulated Poisson Process}

\DOI{10.14357/19922264170414}

\vspace*{9pt}


\vskip 12pt plus 9pt minus 6pt

      \thispagestyle{myheadings}

      \begin{multicols}{2}

                  \label{st\stat}

\section{Introduction}

\noindent
Queueing systems represent a~powerful mathematical tool for investigating 
the performance of a~wide variety of real-life systems, ranging 
from telecommunication networks to financial markets, from computer 
architectures to supply chain management and airplane traffic control, 
just to cite a~few. Analytical tractability of the corresponding models 
strongly depends on the nature of the underlying processes (Poisson 
arrivals have many nice features that strongly simplify the analysis) 
and on the system geometry.

Although physical resources are always finite, quite often it is easier 
to study queueing systems in which the corresponding parameters assume 
infinite values. For instance, the overflow probability is often used 
as an upper bound for the loss probability in finite-buffer queues and, 
indeed, asymptotic results are available even for strongly non-Markovian 
systems~\cite{mandjes}. Moreover, infinite-server queueing systems may be 
applicable in case of models with a~limited number of server devices 
as described in~\cite{lit2}.

In this work,  an infinite-server queueing system, fed by 
non-Poisson arrivals with random customers' capacities, is considered.  
Queues with random customers' capacities are  useful for analysis 
and design issues in high-performance computer and communication systems, 
in which service time and customer volume are the independent quantities 
(see~\cite{lit8, lit10} and references therein). For instance, in~\cite{lit10}, 
performance analysis of LTE (Long Term Evolution) networks is carried out 
in terms of flow-level dynamics and the amount of required radio resources 
does not depend on the duration of the flow. Such queues are also important in 
modeling devices, where it is necessary to calculate a~sufficient volume of buffer 
for data storing~\cite{lit9, lit12}.
The results for single-server queues with limited buffer and LIFO 
(last in, first out) service discipline 
were presented in~\cite{lit13}, where algorithms for the calculation of stationary 
characteristics were derived. 

A new trend in the study of queueing systems is the analysis of the systems with 
non-Poisson arrivals and nonexponential service time. So, in the 
works~\cite{lit1, lit2, lit4, lit5, lit11}, queues with renewal arrivals, 
Markovian Arrival processes (MAP), and MMPP 
are studied under various asymptotic conditions. 
The main contribution of this paper consists in extending such analysis, 
focusing on the properties of the bidimensional process describing the 
number of customers and the total capacity in the system when an infinite-server 
queue is fed by MMPP arrivals with random capacities and nonexponential service 
time distribution.


\section{Matematical Model}

\noindent
Consider a~queue with infinite number of servers (Fig.~1) 
and assume that customers arrive according to an MMPP. The input process is 
defined by its generator matrix $\mathbf{Q}=||q_{ij}||$ of size $K\times K$ and 
the conditional rates $\lambda_1,\ldots,\lambda_K$, typically composed into 
the diagonal matrix $\mathbf{\Lambda}=\mathrm{diag}\,\{\lambda_1,\ldots,\lambda_K\} $. 
Denote the underlying\linebreak\vspace*{-12pt}

 { \begin{center}  %fig1
 \vspace*{-1pt}
 \mbox{%
 \epsfxsize=45.705mm 
 \epsfbox{pag-1.eps}
 }


\end{center}


\noindent
{{\figurename~1}\ \ \small{Queue MMPP/GI/$\infty$ with random customers' capacities}}
}

\vspace*{9pt}

\addtocounter{figure}{1}



\noindent
 Markov chain of the MMPP as $k(t) \in 1,2,\ldots,K$. 
Let each customer has some random capacity $v>0$ with distribution function~$G(y)$. 
An arriving customer instantly occupies a~server in the system and its service 
time has distribution function $B(x)$; when the service is completed, the customer 
leaves the system. Customers' capacities and service times are mutually 
independent and do not dependent on the epochs of customers' arrivals.

 


Denote by $i(t)$ and $V(t)$ the number of customers in the system at 
time~$t$ and their total capacity, respectively. Let us obtain the probabilistic 
characteristics of two-dimensional process~$\{i(t),V(t)\}$. This process 
is not Markovian; therefore,  the dynamic screening method has been used for 
its investigation.

Consider two time axes that are numbered as~0 and~1 (Fig.~2). 
Let axis~0 shows the epochs of customers' arrivals, while axis~1 
corresponds to the screened process.



Let us introduce a~function $S(t)$ (dynamic probability) that satisfies the condition 
$0\le S(t) \le 1$.
Let us assume that a~customer, arriving in the system at time~$t$, is screened 
to axis~1 with probability $S(t)$, and not screened with probability $1-S(t)$.

Let the system be empty at moment~$t_0$ and let us fix some arbitrary moment~$T$ 
in the future. $S(t)$ represents the probability that a~customer arriving at time~$t$ 
will be serviced in the system by the moment~$T$. 
It is easy to show~\cite{lit5} that $S(t)=1-B(T-t) $ for $t_0\le t\le T$.




Denote by $n(t)$ and $W(t)$ the number of arrivals screened before the moment~$t$ 
on axis~1 and their total\linebreak\vspace*{-12pt}

{ \begin{center}  %fig2
 \vspace*{9pt}
 \mbox{%
 \epsfxsize=77.897mm 
 \epsfbox{pag-2.eps}
 }

\vspace*{6pt}

\noindent
{{\figurename~2}\ \ \small{Screening of the customers' arrivals}}


\end{center}
}

%\vspace*{9pt}

\addtocounter{figure}{1}

\noindent
 capacity, respectively. 
As it is shown in~\cite{lit4}, the probability distribution of the number of 
customers in the system at the moment~$T$ coincides with the probability 
distribution of the number of screened arrivals on the axis:
$$
P\{i(T)=m\}=P\{n(T)=m\}
$$
for all $m=0,1,2,\ldots$ It is easy to prove the same property for 
the extended process $\{i(t),V(t)\}$:
\begin{multline}
\label{eq1-p}
P\{i(T)=m,V(T)<z\}\\
{}=P\{n(T)=m,W(T)<z\}
\end{multline}
for all $m=0,1,2,\ldots$ and $z\ge 0$. 
Let us use Eqs.~\eqref{eq1-p} for the investigation of the process $\{i(t),V(t)\}$ 
via the analysis of the process $\{n(t),W(t)\}$.

\section{Kolmogorov Differential Equations}

\noindent
Let us consider the three-dimensional Markovian process $\{k(t),n(t),W(t)\}$. 
Denoting the probability distribution of this process by 
$P(k,n,w,t)=P\{k(t)\linebreak =k,n(t)=n,W(t)<w\}$ and taking into account the formula 
of total probability, one can write the following system of Kolmogorov 
differential equations:
\begin{multline*}
\hspace*{-9pt}\fr{\partial P(k,n,w,t)}{\partial t}=\lambda_kS(t)\!\left[
\int\limits_0^z\!\!\! P(k,n-1,w-y,t)\,dG(y){}\right.\hspace*{-1pt}\\
\left.{}-P(k,n,w,t)
\vphantom{\int\limits_0^z}\right]
+\sum_vP(\nu,n,w,t)q_{\nu k}
\end{multline*}
for $k=1,\ldots , K$; $n=0,1,2,\ldots$; $w>0$.

Let us introduce the partial characteristic function:
$$
h(k,u_1,u_2,t)=\sum\limits_{n=0}^{\infty}e^{ju_1n}
\int\limits_0^\infty e^{ju_2w}P(k,n,dw,t)
$$
where $j=\sqrt{-1}$ is the imaginary unit. Then, one can write the following equations:
\begin{multline*}
\fr{\partial h(k,u_1,u_2,t)}{\partial t}\\
{}=
h\left(k,u_1,u_2,t\right)\lambda_kS(t)\left( e^{ju_1}G^*(u_2)-1\right) \\
{}+
\sum\limits_{\nu}h(\nu,n,w,t)q_{\nu k}
\end{multline*}
for $k=1,\ldots,K$ where $G^*(u)=\int\nolimits_0^\infty e^{juy}dG(y)$.

Let us rewrite this system in the matrix form:
\begin{multline}
\label{eq2-p}
\fr{\partial\mathbf{h}(u_1,u_2,t)}{\partial t}\\
{}=
\mathbf{h}(u_1,u_2,t)\left[
\mathbf{\Lambda} S(t)\left( e^{ju_1}G^*\left(u_2\right)-1\right) +\mathbf{Q}\right]
\end{multline}
with the initial condition
\begin{equation}
\label{eq3}
{\mathbf{h}}\left(u_1,u_2,t_0\right)=\mathbf{r}
\end{equation}
where
$$
\mathbf{h}\left(u_1,u_2,t\right)=\left[h\left(1,u_1,u_2,t\right),\ldots,
h\left(K,u_1,u_2,t\right)\right]
$$
and ${\mathbf{r}}=[r(1),\ldots,r(K)]$ represents the stationary 
distribution of the underlying Markov chain, i.\,e., 
vector~$\bf{r}$ satisfies the following linear system:
\begin{equation}
\label{eq4}
\left.
\begin{array}{l}
\mathbf{rQ}=\mathbf{0}\,; \\[6pt]
\mathbf{re}=1
\end{array}
\right\}
\end{equation}
where $\mathbf{e}$ is the~column vector with all entries equal to~1.

\section{Asymptotic Analysis}

\noindent
In general, the exact solution of Equation~\eqref{eq2-p} is not available, 
but it may be found under asymptotic conditions. In this paper,  
the case of infinitely growing service time is considered.

Denote by
$$
b_1=\int\limits_0^\infty xdB(x)=\int\limits_0^\infty(1-B(x))\,dx
$$
the mean service time; then, the asymptotic condition is $b_1\to\infty$.

Let us solve Problem \eqref{eq2-p}--\eqref{eq3} under such asymptotic condition 
and we obtain  approximate solutions with different order of accuracy, named as 
``first-order asymptotic'' 
${\mathbf{h}}(u_1,u_2,t)\approx{\mathbf{h}}^{(1)}(u_1,u_2,t)$ and  
``second-order asymptotic'' 
${\mathbf{h}}(u_1,u_2,t)\approx{\mathbf{h}}^{(2)}(u_1,u_2,t)$.

\subsection{First-order asymptotic analysis}

\noindent
Let us formulate and prove the following statement.

\smallskip

\noindent
\textbf{Lemma.}\ 
\textit{The first-order asymptotic characteristic function of the probability 
distribution of the process $\{k(t),n(t),W(t)\}$  has the form}:
\begin{equation*}
\mathbf{h}^{(1)}(u_1,u_2,t)=\mathbf{r} 
\exp\left\{ \!\left(ju_1\kappa_1+ju_2\kappa_1a_1\right)
\!\int\limits_{t_0}^t \! S(v)\,dv\!\right\}
\end{equation*}
\textit{where  $\kappa_1=\mathbf{r\Lambda e}$ and  
$a_1=\int\limits\nolimits_0^\infty ydG(y)$ is the mean customer capacity}.

\smallskip


\noindent
P\,r\,o\,o\,f\,.\ \ 
By performing the substitutions
\begin{gather*}
\varepsilon=\fr{1}{b_1}\,;\quad
 \varepsilon t=\tau\,;\quad 
 \varepsilon t_0=\tau_0\,;\\[6pt]
  u_1=\varepsilon x_1\,;\enskip
   u_2=\varepsilon x_2\,;\enskip 
   S(t)=S_1(\tau)\,;\\[6pt] 
   \mathbf{h}(u_1,u_2,t)=\mathbf{f}_1(x_1,x_2,\tau,\varepsilon)
%\label{eq5}
\end{gather*}
in expressions~\eqref{eq2-p} and~\eqref{eq3}, one obtains
\begin{multline}
\varepsilon\fr{\partial \mathbf{f}_1(x_1,x_2,\tau,\varepsilon)}{\partial \tau}\\
\!\!\!\!\!{}=
\mathbf{f}_1(x_1,x_2,\tau,\varepsilon)\!\left[\mathbf{\Lambda} 
S_1(\tau)\left( e^{j\varepsilon x_1}G^*(\varepsilon x_2)\!-\!1\right) +\mathbf{Q}\right]\!\!
\label{eq6}
\end{multline}
with the initial condition
\begin{equation}
\label{eq7}
\mathbf{f}_1(x_1,x_2,\tau_0,\varepsilon)=\mathbf{r}\,.
\end{equation}

Let us find the asymptotic solution of Problem~\eqref{eq6}--\eqref{eq7} 
$\mathbf{f}_1(x_1,x_2,\tau)=
\lim\nolimits_{\varepsilon\to 0}\mathbf{f}_1(x_1, x_2,\tau,\varepsilon)$
in two steps.

\textit{Step~1.} Let $\varepsilon\to 0$ in~\eqref{eq6}--\eqref{eq7}; 
then, one obtains the following system of equations:
$$
\left\{ 
\begin{array}{l}
\mathbf{f}_1\left(x_1,x_2,\tau\right)\mathbf{Q}=\mathbf{0}\,;\\[6pt]
\mathbf{f}_1\left(x_1,x_2,\tau_0\right)=\mathbf{r}\,.
\end{array}
\right.
$$

Taking into account~\eqref{eq4}, one can conclude that $\mathbf{f}_1(x_1,x_2,\tau)$  
can be expressed as
\begin{equation}
\label{eq8}
\mathbf{f}_1(x_1,x_2,\tau)=\mathbf{r}\Phi_1(x_1,x_2,\tau)
\end{equation}
where $\Phi_1(x_1,x_2,\tau)$ is some scalar function which satisfies the condition
\begin{equation}
\label{eq9}
\Phi_1(x_1,x_2,\tau_0)=1\,.
\end{equation}

\textit{Step 2.} Let us multiply~\eqref{eq6} by vector~{\bf e}, substitute~\eqref{eq8}, 
divide the result by~$\varepsilon$, and perform the asymptotic transition 
$\varepsilon\to 0$. Then, taking into account that $\mathbf{Qe}=\mathbf{0}$ 
and $\mathbf{re}=1$, one obtains the following differential equation 
for the function $\Phi_1(x_1,x_2,\tau)$:
\begin{multline}
\label{eq10}
\fr{\partial\Phi_1(x_1,x_2,\tau)}{\partial\tau}\\
{}=
\Phi_1\left(x_1,x_2,\tau\right)S_1(\tau)\left(jx_1\kappa_1+jx_2\kappa_1a_1\right)\,.
\end{multline}

The solution of Problem~\eqref{eq9}--\eqref{eq10} is as follows:
$$
\Phi_1(x_1,x_2,\tau)=\exp\left\{ \!\left(jx_1\kappa_1+jx_2\kappa_1a_1\right)\!
\int\limits_{\tau_0}^{\tau}\!S_1(v)\,dv\right\}.
$$
Substituting this expression into~\eqref{eq8}, one obtains
$$
\mathbf{f}_1(x_1,x_2,\tau)=
\mathbf{r}\exp\left\{ \!\left(jx_1\kappa_1+jx_2\kappa_1a_1\right)
\!\int\limits_{\tau_0}^{\tau}\!S_1(v)\,dv\!\right\}.\hspace*{-0.69418pt}
$$

Therefore, one can write
\begin{multline*}
\mathbf{h}^{(1)}(u_1,u_2,t)=\mathbf{f}_1\left(x_1,x_2,\tau,\varepsilon\right)\approx
\mathbf{f}_1\left(x_1,x_2,\tau\right)
\\
{}=\mathbf{r}\exp\left\{ \left(jx_1\kappa_1+jx_2\kappa_1a_1\right)
\int\limits_{\tau_0}^{\tau}S_1(v)dv\right\}\\
{} =
\mathbf{r}\exp\left\{ \left( ju_1\kappa_1+ju_2\kappa_1a_1\right) 
\int\limits_{t_0}^tS(v)\,dv\right\} \,.
\end{multline*}
Thus, the proof is complete.


\subsection{Second-order asymptotic analysis}

\noindent
The main result is the following theorem.

\smallskip

\noindent
\textbf{Theorem.}\ 
\textit{The second-order asymptotic characteristic function of the 
probability distribution of the process  $\{k(t),n(t),W(t)\}$ has the form}:

\noindent
\begin{multline*}
\mathbf{h}^{(2)}\left(u_1,u_2,t\right)\\
{}=\mathbf{r}\exp\left\{ 
\left(ju_1\kappa_1+ju_2\kappa_1a_1\right)\int\limits_{t_0}^tS(v)\,dv\right.
\\
{}+\fr{(ju_1)^2}{2}\left(\kappa_1\int\limits_{t_0}^tS(v)\,dv+
\kappa_2\int\limits_{t_0}^tS^2(v)\,dv\right)
\\
{}+\fr{(ju_2)^2}{2}\left(\kappa_1a_2\int\limits_{t_0}^tS(v)\,dv+
\kappa_2a_1^2\int\limits_{t_0}^tS^2(v)\,dv\right)
\\
\left.{}+ju_1ju_2\left(\kappa_1a_1\int\limits_{t_0}^tS(v)\,dv+
\kappa_2a_1\int\limits_{t_0}^tS^2(v)\,dv\right)\right\}\hspace*{-3.166pt}
\end{multline*}
\textit{where  $\kappa_2=2\mathbf{g}(\mathbf{\Lambda}-\kappa_1\mathbf{I})\mathbf{e}$;  
$a_2=\int\nolimits_0^\infty y^2dG(y)$;  and the row vector  $\mathbf{g}$  
satisfies the linear matrix system} 
$$
\left\{
\begin{array}{rl}
\mathbf{gQ}&=\mathbf{r}(\kappa_1\mathbf{I}-\mathbf{\Lambda})\,; \\[6pt]
\mathbf{ge}&=const\,.
\end{array}
\right.
$$



\noindent
P\,r\,o\,o\,f\,.\ \  
Let $\mathbf{h}_2(x_1,x_2,t)$ be a~vector function that satisfies the equation:
\begin{multline}
\label{eq12}
\mathbf{h}\left(u_1,u_2,t\right)=
\mathbf{h}_2\left(u_1,u_2,t\right)\\
{}\times\exp
\left\{ \!\left(ju_1\kappa_1+ju_2\kappa_1a_1\right)\int\limits_{t_0}^tS(v)\,dv\right\}\,.
\end{multline}

Substituting this expression into~\eqref{eq2-p} and~\eqref{eq3}, one obtains
the following problem:
\begin{multline}
\fr{\partial {\mathbf{h}_2(u_1,u_2,t)}}{\partial t}\\
{}=
\mathbf{h}_2(u_1,u_2,t)\left[(e^{ju_1}G^*(u_2)-1)S(t)\mathbf{\Lambda}\right.
\\
\left.{}-\left(ju_1\kappa_1+ju_2\kappa_1a_1\right)S(t)\mathbf{I}+\mathbf{Q}
\vphantom{e^{ju_1}G^*(u_2)}
\right]
\label{eq13}
\end{multline}
with the initial condition
\begin{equation}
\label{eq14}
{\bf h}_2(u_1,u_2,t_0)={\bf r}
\end{equation}
where {\bf I} is the identity matrix.

Let us make the substitutions:
\begin{equation}
\left.
\begin{array}{c}
\varepsilon^2=\fr{1}{b_1}\,;\quad
\varepsilon^2 t=\tau\,;\quad
\varepsilon^2 t_0=\tau_0\,;\\[6pt] 
u_1=\varepsilon x_1\,;\enskip 
u_2=\varepsilon x_2\,;\enskip 
S(t)=S_1(t)\,;\\[6pt] 
{\bf h}_2(u_1,u_2,t)={\bf f}_2(x_1,x_2,\tau,\varepsilon)\,.
\end{array}
\right\}
\label{eq15}
\end{equation}

Using these notations, Problem~\eqref{eq13}--\eqref{eq14} can be rewritten in the form
\begin{multline}
\varepsilon^2\fr{\partial {\bf f}_2(x_1,x_2,\tau,\varepsilon)}{\partial \tau}\\
{}=
{\bf{f}}_2(x_1,x_2,\tau,\varepsilon)\left[
\mathbf{\Lambda} S_1(\tau)(e^{j\varepsilon x_1}G^*(\varepsilon x_2)-1)\right. 
\\
\left.{} -\left( j\varepsilon\kappa_1x_1+j\varepsilon\kappa_1x_2a_1\right)
S_1\left( \tau\right) \mathbf{I}+ \mathbf{Q}
\vphantom{e^{j\varepsilon x_1}G^*(\varepsilon x_2)}
\right]
\label{eq16}
\end{multline}
with the initial condition
\begin{equation}
\label{eq17}
{\bf f}_2(x_1,x_2,\tau_0,\varepsilon)={\bf r}\,.
\end{equation}

Let us find the asymptotic solution of this problem 
${\bf f}_2(x_1,x_2,\tau)=
\lim\limits_{\varepsilon\to 0}{\bf f}_2(x_1,x_2,\tau,\varepsilon)$ in three steps.

\textit{Step~1.} Letting $\varepsilon\to 0$ in~\eqref{eq16}--\eqref{eq17}, 
one obtains the following system of equations:
$$
\left\{
\begin{array}{l}
{\bf f}_2\left(x_1,x_2,\tau\right)\mathbf{Q}=\mathbf{0}\,; \\[6pt]
{\bf f}_2\left(x_1,x_2,\tau_0\right)={\bf r}\,.
\end{array}
\right.
$$
Therefore, taking into account~\eqref{eq4}, one can write:
\begin{equation}
\label{eq18}
{\bf f}_2\left(x_1,x_2,\tau\right)={\bf r}\Phi_2\left(x_1,x_2,\tau\right)
\end{equation}
where $\Phi_2(x_1,x_2,\tau)$  is some scalar function which satisfies the condition
\begin{equation}
\label{eq19}
\Phi_2\left(x_1,x_2,\tau_0\right)=1\,.
\end{equation}

\textit{Step 2.} Using~\eqref{eq18}, the function ${\bf f}_2(x_1,x_2,\tau)$  
can be represented in the expansion form:
\begin{multline}
{\bf f}_2\left(x_1,x_2,\tau,\varepsilon\right)\\
{}=\Phi_2\left(x_1,x_2,\tau\right)\left[{\bf r}+\mathbf{g}S_1(\tau)
\left(j\varepsilon x_1+j\varepsilon x_2a_1\right)\right]\\
{}+{\bf O}(\varepsilon^2)
\label{eq20}
\end{multline}
where {\bf g} is the~row vector that satisfies the condition ${\bf ge}= const$ 
and ${\bf O}(\varepsilon^2)$  is the row vector of the second-order infinitesimals. 
Let us use substitution~\eqref{eq20} and the expansion
$$
e^{j\varepsilon x}=1+j\varepsilon x+O\left( \varepsilon^2\right) 
$$
in Eq.~\eqref{eq16}. Taking into account~\eqref{eq4} and 
making the transition $\varepsilon\to 0$, one obtains the 
following matrix equation for the vector~$\mathbf{g}$:
$$
{\bf gQ}={\bf r}\left(\kappa_1\mathbf{I}-\mathbf{\Lambda}\right)\,.
$$

\textit{Step~3.} Let us multiply Eq.~\eqref{eq16} by vector~\textbf{e} 
and use expression~\eqref{eq20} and the second-order expansion:
$$
e^{j\varepsilon x}=1+j\varepsilon x+\fr{(j\varepsilon x)^2}{2}+O\left(\varepsilon^3\right)\,.
$$

After some transformations, using the notation
$$
\kappa_2=2{\bf g}\left(\mathbf{\Lambda}-\kappa_1\mathbf{I}\right){\bf{e}}\,,
$$
one obtains the following differential equation for the function $\Phi_2(x_1,x_2,\tau)$:
\begin{multline*}
\fr{\partial\Phi_2(x_1,x_2,\tau)}{\partial\tau}\\
{}=
\Phi_2(x_1,x_2,\tau) \left[\fr{(jx_1)^2}{2}\left(\kappa_1S_1(\tau)+
\kappa_2S_1^2(\tau)\right)\right.
\\
{}+\fr{(jx_2)^2}{2}\left(\kappa_1a_2S_1(\tau)+\kappa_2a_1^2S_1^2(\tau)\right)\\
\left.{}+jx_1jx_2\left(\kappa_1a_1S_1(\tau)+\kappa_2a_1S_1^2(\tau)\right)
\vphantom{\fr{(jx_1)^2}{2}}\right]\,.
\end{multline*}

The solution of this equation with initial condition~\eqref{eq19} is as follows:
\begin{multline*}
\Phi_2\left(x_1,x_2,\tau\right)\\
{}= 
\exp\left\{ \fr{(jx_1)^2}{2}\left(
\kappa_1\int\limits_{\tau_0}^{\tau}S_1(v)\,dv+\kappa_2\int\limits_{\tau_0}^{\tau}
S_1^2(v)\,dv\right)\right.
\\
{}+\fr{(jx_2)^2}{2}\left(\kappa_1a_2\int\limits_{\tau_0}^{\tau}S_1(v)\,dv+
\kappa_2a_1^2\int\limits_{\tau_0}^{\tau}\!S_1^2(v)\,dv\right)
\\
\!\left.{}+
jx_1jx_2\left(\kappa_1a_1\int\limits_{\tau_0}^{\tau}\!S_1(v)\,dv+
\kappa_2a_1\int\limits_{\tau_0}^{\tau}\!S_1^2(v)\,dv\right)\!\right\}.\hspace*{-0.39064pt}
\end{multline*}

Substituting this expression in formula~\eqref{eq18} and performing 
the substitutions that are inverse to~\eqref{eq15} and~\eqref{eq12}, one obtains
\begin{multline*}
{\bf{h}}^{(2)}\left(u_1,u_2,t\right)\\
{}=
{\bf{r}} \exp\left\{ \left(ju_1\kappa_1+ju_2\kappa_1a_1\right)
\int\limits_{t_0}^tS(v)\,dv\right. 
\\
{}+\fr{(ju_1)^2}{2}\left(\kappa_1\int\limits_{t_0}^tS(v)\,dv+
\kappa_2\int\limits_{t_0}^tS^2(v)\,dv\right)\\
{}+\fr{(ju_2)^2}{2}\left(\kappa_1a_2\int\limits_{t_0}^tS(v)\,dv+
\kappa_2a_1^2\int\limits_{t_0}^tS^2(v)\,dv\right)
\\
\left.
{}+ju_1ju_2\left(\kappa_1a_1\int\limits_{t_0}^tS(v)\,dv+
\kappa_2a_1\int\limits_{t_0}^tS^2(v)\,dv\right)\right\} 
\end{multline*}
for the asymptotic characteristic function of the process
 $\{k(t),n(t),W(t)\}$. The proof is complete.
 
\columnbreak
 
 \noindent
 \textbf{Corollary.}
Assuming $t = T$ and $t_0\to -\infty $ and using Eqs.~\eqref{eq1-p}, one obtains 
the steady-state characteristic function of the process under study $\{i(t),V(t)\}$:
\begin{multline}
h\left(u_1,u_2\right)= 
\exp\left\{ \left(ju_1\kappa_1b_1+ju_2\kappa_1a_1b_1\right)\right.\\
{}+
\fr{(ju_1)^2}{2}\left(\kappa_1b_1
+\kappa_2b_2\right)
+\fr{(ju_2)^2}{2}\left(\kappa_1a_2b_1+\kappa_2a_1^2b_2\right)\\
\left.{}+
ju_1ju_2\left(\kappa_1a_1b_1+\kappa_2a_1b_2\right)\right\}
\label{eq21}
\end{multline}
where 
$$
b_1=\int\limits_0^{\infty}(1-B(v))\,dv\,;\enskip  
b_2=\int\limits_0^{\infty}(1-B(v))^2\,dv\,.
$$


From the form of the characteristic function~\eqref{eq21}, it is clear 
that the probability distribution of the two-dimensional process $\{i(t),V(t)\}$ 
is asymptotically Gaussian with vector of means
$$
{\bf a}=\left[
\begin{array}{lr}
\kappa_1b_1 &  \kappa_1a_1b_1
\end{array}
\right]
$$
and covariance matrix
\begin{multline*}
\mathbf{K}=\left[
\begin{array}{cc}
\sigma_1^2 &  K_{12} \\
K_{12} & \sigma_2^2
\end{array}
\right]\\
{}=
\left[
\begin{array}{cc}
\kappa_1b_1+\kappa_2b_2 &  \kappa_1a_1b_1+\kappa_2a_1b_2 \\
\kappa_1a_1b_1+\kappa_2a_1b_2 & \kappa_1a_2b_1+\kappa_2a_1^2b_2
\end{array}
\right]\,.
\end{multline*}

Therefore, the correlation coefficient is given by
$$
r=\fr{K_{12}}{\sigma_1\sigma_2}=
\fr{\kappa_1a_1b_1+\kappa_2a_1b_2}{\sqrt{\kappa_1b_1+\kappa_2b_2}\,
\sqrt{\kappa_1a_2b_1+\kappa_2a_1^2b_2}}\,.
$$

\begin{figure*}[b] %fig3
\vspace*{1pt}
 \begin{center}
 \mbox{%
 \epsfxsize=163.767mm 
 \epsfbox{pag-3.eps}
 }
  \end{center}
\vspace*{-11pt}
\Caption{Distributions of the number of customers~(\textit{a})
and of the total capacity~(\textit{b}) for different values of~$N$:
left column~--- $N=10$; right column~--- $N=100$;
\textit{1}~--- theoretical results; and \textit{2}~--- simulation}
\label{fig:fig3}
\end{figure*}

\section{Numerical Example}

\noindent
Result~\eqref{eq21} is obtained under the asymptotic condition $b_1\to\infty$. 
Therefore, it may be used just as an approximation when~$b_1$ is large enough. 
To test its practical applicability, the present authors
 considered several numerical examples, 
varying all the system parameters (including the distributions of the service 
time and of the customer capacity). Since all the different simulation sets led 
to similar results, for sake of brevity, in the following, 
just one of them is discussed in detail. In particular, let us
assume that the input MMPP is characterized by the matrices:
$$
\mathbf{Q}=\left[
\begin{array}{rrr}
-0.8 & 0.4 & 0.4\\
0.3 & -0.6 & 0.3\\
0.4 & 0.4 & -0.8
\end{array}\right]
$$
and
$$
\mathbf{\Lambda}=\left[
\begin{array}{ccc}
0.5 & 0 & 0\\
0 & 1 & 0\\
0 & 0 & 1.5
\end{array}\right] \,.
$$

 %\begin{table*} %tabl1
\begin{center}
\begin{minipage}[t]{34mm}
{{\tablename~1}\ \ \small{Kolmogorov distances between simulation results and asymptotic 
values for the number of customers in the system}}

\vspace*{6pt}


{\small
\tabcolsep=14pt
\begin{tabular}{cc}
\hline
$N$ & $\Delta$\\
\hline
\hphantom{9}1 & 0.265\\
10 & 0.039\\
15 & 0.032\\
20 & \textbf{0.027}\\
25 & \textbf{0.025}\\
50 & \textbf{0.017}\\
100\hphantom{9} & \textbf{0.012}\\
\hline
\end{tabular}
}
\end{minipage}
\hfill
%\end{center}
%\end{table*}
%\begin{table*}\small %tabl2
%\begin{center}
\begin{minipage}[t]{34mm}
{{\tablename~2}\ \ \small{Kolmogorov distances between simulation results and asymptotic values for the total capacity in the system}}

\vspace*{6pt}

{\small 
\tabcolsep=14pt
\begin{tabular}{cc}
\hline
$N$ & $\Delta$\\
\hline
\hphantom{9}1 & 0.355\\
10 & 0.033\\
15 & \textbf{0.025}\\
20 & \textbf{0.021}\\
25 & \textbf{0.019}\\
50 & \textbf{0.013}\\
100\hphantom{9} &\textbf{0.010}\\
\hline
\end{tabular}
}
\end{minipage}
\end{center}

\vspace*{9pt}
%\end{table*}

\setcounter{table}{2}

Hence, the fundamental rate of arrivals is $\kappa_1\linebreak =\mathbf{r\Lambda e}=1$ 
customers per time unit. Let us also assume that customers' capacities have 
uniform distribution in the range $[0;1]$ and service time has gamma distribution 
with shape and inverse scale parameters $\alpha = 1.5$ and $\beta = \alpha/ N$, 
respectively. So, when $N \to\infty$, one obtains the asymptotic condition of 
an infinite growing service time ($b_1 = \alpha / \beta = N \to \infty$).

The goal is to find a~lower bound of parameter~$N$ for the applicability 
of the approximation~\eqref{eq21}. To this aim,  series of
 simulation experiments have been  carried out for increasing values of~$N$ and 
  the asymptotic
  distributions    have been compared with the
  empiric ones by using the Kolmogorov distance~\cite{lit3,lit6}
$$
\Delta=\sup\limits_x\left| F\left( x\right) -A\left( x\right) \right| 
$$
as an accuracy measure. Here,~$F(x)$ is the cumulative distribution function 
built on the basis of simulation results and $A(x)$ is the Gaussian 
approximation based on~\eqref{eq21}.

Let us consider the marginal distributions of the customers' number and the 
total capacity in the system.

In the first case, the asymptotic values of mean and variance are equal 
to~$N$ and~$1.144 N$, respectively, and the corresponding values of the 
Kolmogorov distance for increasing values of parameter~$N$ are presented in 
Table~1. Similarly, for the total capacity in the system, 
mean and variance are equal to~$0.5 N$ and~$0.369 N$, respectively, and 
Table~2 shows the Kolmogorov distance.


One can notice that the asymptotic results become more accurate 
when the parameter~$N$  increases. Fig-\linebreak\vspace*{-12pt}

 { \begin{center}  %fig1
 \vspace*{-1pt}
 \mbox{%
 \epsfxsize=77.763mm 
 \epsfbox{pag-5.eps}
 }


\end{center}


\noindent
{{\figurename~4}\ \ \small{Relative error for the variance of the number of customers~$i(t)$~(\textit{1}) 
and the total capacity $V(t)$~(\textit{2})}}
}

\vspace*{18pt}

\addtocounter{figure}{1}


\noindent
ure~\ref{fig:fig3}  
compares the asymptotic approximations with the empirical results for 
the number of customers and the total capacity in the system.



As typically done in the literature~\cite{lit6}, let us suppose that 
an approximation is applicable if its Kolmogorov distance is less than~0.03. 
Hence, one can  conclude that the asymptotic results are applicable for values
 of the parameter~$N$ equal to~15 or more (marked by boldface in 
 Tables~1 and~2).
 



Then, let us compare the asymptotic value of some characteristics of the 
queueing system with the corresponding empirical characteristics, 
using the relative error
$$
\delta=\fr{\left| d-a\right| }{d}
$$
where $d$ denotes the value constructed on the basis of simulation results and~$a$ 
is obtained from~\eqref{eq21}.

In more detail, the mean values of the processes~$i(t)$ and $V(t)$ are very
 close (with $\delta<10^{-5}$ for all~$N$) and the relative errors 
 of the variance decreases with~$N$ as shown in Fig.~4.



Finally, Table~3 shows the relative error for 
the correlation coefficient.

\vspace*{12pt}

%\begin{table*}\small %tabl3
\begin{center}
 \parbox{43mm}{{{\tablename~3}\ \ \small{Relative error for the correlation coefficient}}
 }
\vspace*{6pt}

        \tabcolsep=15pt
        {\small \begin{tabular}{cc}
            \hline
                        \multicolumn{2}{c}{\ }\\[-9pt]
            $N$ & $\delta$\\
            \hline
            \multicolumn{2}{c}{\ }\\[-9pt]
\hphantom{9}1 & {\boldmath{${60\cdot10^{-4}}$}}\hphantom{9}\\
            10 & {\boldmath{$11\cdot10^{-4}$}}\hphantom{9} \\
            15 &             {\boldmath{$7\cdot10^{-4}$}} \\
             20 & {\boldmath{$5\cdot10^{-4}$}} \\
             25 & {\boldmath{$4\cdot10^{-4}$}}\\
             50 &     {\boldmath{$1\cdot10^{-4}$}}\\
             100\hphantom{9}&    {\boldmath{$0.8\cdot10^{-4}$}}\hphantom{.9}\\
            \hline
        \end{tabular}}
    \end{center}
%\end{table*}

\section{Concluding Remarks}

\noindent
In the paper, the queue with MMPP arrivals, infinite number of servers, 
and nonexponential service time is considered. Moreover, random customers' capacities, 
independent of their service time, are assumed.
The analysis is performed under the asymptotic condition of an infinitely 
growing service time. It is shown that two-dimensional probability 
distribution of customers' number and total capacity in the system is 
two-dimensional Gaussian under this asymptotic condition. Numerical 
results show that asymptotic results have enough accuracy for 
the marginal distributions of number of customers and of the 
total capacity in the system when the service rate exceeds the 
fundamental rate of arrivals by at least~15~times.

\vspace*{-6pt}

\Ack
\noindent
This work is supported by the Russian Foundation for Basic research, project 16-31-00292.

\renewcommand{\bibname}{\protect\rmfamily References}

\vspace*{-6pt}

{\small\frenchspacing
{%\baselineskip=10.8pt
\begin{thebibliography}{99}

\bibitem{mandjes} %1
\Aue{Mandjes, M.} 2007. \textit{Large deviations of Gaussian queues.} 
Chichester: Wiley. 340~p.

\bibitem{lit2} %2
\Aue{Melikov, A., L.~Zadiranova, and A.~Moiseev.} 2016. 
Two asymptotic conditions in queue with MMPP arrivals and feedback. 
\textit{Comm. Com. Inf. Sc.} 678:231--240. 
doi: 10.1007/978-3-319-51917-3\_21.

\bibitem{lit10} %3
\Aue{Naumov, V., K.~Samouylov, E.~Sopin, and S.~Andreev.} 2015. 
Two approaches to analyzing dynamic cellular networks with limited resources. 
\textit{6th  Congress (International)
on Ultra Modern Telecommunications and Control Systems and Workshops.} 
St.\ Petersburg. 485--488. 
doi: 10.1109/ICUMT.2014.7002149.

\bibitem{lit8}%4
\Aue{Morozov, E., L.~Potakhina, and O.~Tikhonenko.} 2016. 
Regenerative analysis of a~system with a~random volume of customers. 
\textit{Comm. Com. Inf. Sc.} 638:261--272. 
doi: 10.1007/978-3-319-44615-8\_23.

\bibitem{lit12} %5
\Aue{Tikhonenko, O.\,M., and W.~Kempa.} 2015. 
Queueing systems with processor sharing and limited memory under control of the AQM 
mechanism. \textit{Automat. Rem. Contr.} 76(10):1784--1796. 
doi: 10.1134/S0005117915100069.

\bibitem{lit9} %6
\Aue{Naumov, V.\,A., K.\,E.~Samuilov, and A.\,K.~Samuilov}. 2016. 
On the total amount of resources occupied by serviced customers. 
\textit{Automat. Rem. Contr.} 77(8):1419--1427.



\bibitem{lit13} %7
\Aue{Tikhonenko, O.\,M.} 2010. 
Queueing system with processor sharing and limited resources. 
\textit{Automat. Rem. Contr.} 71(5):803--815.

\bibitem{lit11} %8
\Aue{Pankratova, E.\,V., and S.\,P.~Moiseeva.} 2014. 
Queueing system MAP/M/$\infty$ with $n$ types of customers. 
\textit{Comm. Com. Inf. Sc.} 487:356--366.

\bibitem{lit4} %9
\Aue{Moiseev, A., and A.~Nazarov.} 2016. 
Tandem of infinite-server queues with Markovian arrival process. 
\textit{Comm. Com. Inf. Sc.} 601:323--333. 
doi: 10.1007/978-3-319-30843-2\_34.

\bibitem{lit1} %10
\Aue{Lisovskaya, E., S.~Moiseeva, and M.~Pagano.}  2016. 
The total capacity of customers in the infinite-server queue with MMPP arrivals. 
\textit{Comm. Com. Inf. Sc.} 678:110--120. 
doi: 10.1007/978-3-319-51917-3\_11.
    


\bibitem{lit5} %11
\Aue{Moiseev, A., and A.~Nazarov.} 2016. Queueing network MAP/(GI/$\infty$)$^K$ 
with high-rate arrivals. \textit{Eur. J.~Oper. Res.} 254:161--168. 
doi: 10.1016/j.ejor.2016.04.011.





\bibitem{lit6} %12
\Aue{Moiseev, A.\,N., and M.\,V.~Sinyakov.} 2010. 
Razrabotka ob''ektno-orientirovannoy modeli sistemy imitatsionnogo modelirovaniya 
protsessov massovogo obsluzhivaniya 
[Design of object-oriented model for queueing simulation software]. 
\textit{Vestnik Tomskogo gosudarstvennogo universiteta. Upravlenie, 
vychislitel'naya tekhnika i informatika} 
[Tomsk State University. J.~Control Computer Sci.] 1:89--93.

\bibitem{lit3} %13
\Aue{Moiseev, A., A.~Demin, V.~Dorofeev, and V.~Sorokin.} 2016. 
Discrete-event approach to simulation of queueing networks. 
\textit{Key Eng. Mater.} 685:939--942. 
doi: 10.4028/www.scientific.net/KEM.685.939.

    
\end{thebibliography} }
 }

\end{multicols}

\vspace*{-6pt}

\hfill{\small\textit{Received March 16, 2017}}

\vspace*{-18pt}

\Contr

\noindent
\textbf{Lisovskaya Ekaterina Yu.} (b.\ 1992)~--- PhD student,
Department of Probability 
Theory and Mathematical Statistics, Tomsk State University, 36~Lenin Ave., 
Tomsk 634050, Russian Federation; 
\mbox{ekaterina\_lisovs@mail.ru}

\vspace*{3pt}

\noindent
\textbf{Moiseeva Svetlana P.} (b.\ 1971)~--- Doctor of Science in physics 
and mathematics, 
associate professor, professor, Department of Probability Theory
 and Mathematical Statistics, Tomsk State University, 36~Lenin ave., Tomsk 634050, 
 Russian Federation; \mbox{smoiseeva@mail.ru}

\vspace*{3pt}

\noindent
\textbf{Pagano Michele} (b.\ 1968)~--- PhD in electronics engineering,  
professor, Department of Information Engineering of University of 
Pisa, 16~Via Caruso, Pisa 56122, Italy; \mbox{m.pagano@iet.unipi.it}

\vspace*{3pt}

\noindent
\textbf{Potatueva Viktoriya V.} (b.\ 1993)~--- Master Degree student,
Department 
of Probability Theory and Mathematical Statistics, Tomsk State University, 
36~Lenin Ave., Tomsk 634050, Russian Federation; \mbox{ve-kusik@mail.ru}

\vspace*{8pt}

\hrule

\vspace*{2pt}

\hrule

%\newpage

%\vspace*{-24pt}



\def\tit{ИССЛЕДОВАНИЕ СИСТЕМЫ МАССОВОГО ОБСЛУЖИВАНИЯ MMPP/GI/$\infty$ 
С~ТРЕБОВАНИЯМИ СЛУЧАЙНОГО ОБЪЕМА$^*$}

\def\aut{Е.\,Ю.~Лисовская$^1$, С.\,П.~Моисеева$^2$, М.~Пагано$^3$, В.\,В.~Потатуева$^4$}


\def\titkol{Исследование системы массового обслуживания MMPP/GI/$\infty$ 
с~требованиями случайного объема}

\def\autkol{Е.\,Ю.~Лисовская, С.\,П.~Моисеева, М.~Пагано, В.\,В.~Потатуева}

{\renewcommand{\thefootnote}{\fnsymbol{footnote}}
\footnotetext[1]{Работа выполнена при частичной поддержке РФФИ (проект 16-31-00292).}}


\titel{\tit}{\aut}{\autkol}{\titkol}

\vspace*{-12pt}

\noindent
$^1$Национальный исследовательский Томский государственный университет,
\mbox{ekaterina\_lisovs@mail.ru}

\noindent
$^2$Национальный исследовательский Томский государственный университет,
\mbox{smoiseeva@mail.ru}

\noindent
$^3$Университет г.\ Пиза, Италия, \mbox{m.pagano@iet.unipi.it} 

\noindent
$^4$Национальный исследовательский Томский государственный университет,
\mbox{ve-kusik@mail.ru}

\vspace*{6pt}

\def\leftfootline{\small{\textbf{\thepage}
\hfill ИНФОРМАТИКА И ЕЁ ПРИМЕНЕНИЯ\ \ \ том\ 11\ \ \ выпуск\ 4\ \ \ 2017}
}%
 \def\rightfootline{\small{ИНФОРМАТИКА И ЕЁ ПРИМЕНЕНИЯ\ \ \ том\ 11\ \ \ выпуск\ 4\ \ \ 2017
\hfill \textbf{\thepage}}}


\Abst{Проведено исследование системы массового обслуживания с неограниченным 
числом приборов. Заявки поступают в систему в виде мар\-ков\-ски-мо\-ду\-ли\-ро\-ван\-но\-го 
пуассоновского потока. Каждая заявка несет в себе произвольное количество 
данных (объем заявки). В~этом исследовании время обслуживания не зависит от 
объема заявок. Показано, что совместное распределение вероятностей числа заявок в системе 
и~их суммарного объема является двумерным гауссовским при асимптотическом условии 
растущего времени обслуживания. Имитационное моделирование и численные эксперименты 
позволили определить область применимости асимптотического результата.}

\KW{бесконечнолинейная система массового обслуживания; случайный объем заявок; 
MMPP-поток}

\DOI{10.14357/19922264170414}

%\vspace*{18pt}


 \begin{multicols}{2}

\renewcommand{\bibname}{\protect\rmfamily Литература}
%\renewcommand{\bibname}{\large\protect\rm References}

{\small\frenchspacing
{%\baselineskip=10.8pt
\begin{thebibliography}{99}

\bibitem{mandjes-1} %1
\Au{Mandjes M.} Large deviations of Gaussian queues.~--- 
Chichester: Wiley, 2007. 340~p.

\bibitem{lit2-1}  %2
\Au{Melikov~A.,  Zadiranova~L., Moiseev~A.}  
Two asymptotic conditions in queue with MMPP arrivals and feedback~// 
Comm. Com. Inf. Sc., 2016. Vol.~678. P.~231--240.

\bibitem{lit10-1} %3
\Au{Naumov~V., Samouylov~K.,  Sopin~E.,  Andreev~S.} 
Two approaches to analyzing dynamic cellular networks with limited resources~//  
6th Congress (International)
on Ultra Modern Telecommunications and Control Systems and Workshops.~--- 
St.\ Petersburg, 2015. P.~485--488.
doi: 10.1007/978-3-319-44615-8\_23.

\bibitem{lit8-1} %4
\Au{Morozov  E.,  Potakhina~L.,  Tikhonenko~O.}  
Regenerative analysis of a~system with a~random volume of customers~// 
Comm. Com. Inf. Sc., 2016. Vol.~638. P.~261--272.
doi: 10.1007/978-3-319-44615-8\_23.





\bibitem{lit12-1} %5
\Au{Тихоненко О.\,М., Кемпа В\,.М.} 
Система с~разделением процессора и~ограниченным объемом памяти,
управ\-ля\-емая механизмом AQM~//
Автоматика и~телемеханика, 2015.
№\,10. С.~90--105.

\bibitem{lit9-1}  %6
\Au{Наумов В.\,А., Самуйлов~К.\,Е., Самуйлов~А.\,К.} 
О~суммарном объеме ресурсов, занимаемых обслуживаемыми заявками~// 
Автоматика и телемеханика, 2016. №\,8. C.~125--132.

\bibitem{lit13-1} %7 
\Au{Тихоненко О.\,М.} 
Система обслуживания с~разделением процессора и~ограниченными ресурсами~//
Queueing systems with processor sharing and limited resources~// 
Автоматика и~телемеханика, 2010. №\,5. С.~84--98.

\bibitem{lit11-1} %8
\Au{Pankratova  E.\,V., Moiseeva~S.\,P.} Queueing system 
MAP/M/$\infty$ with $n$ types of customers~// 
Comm. Com. Inf. Sc., 2014. Vol.~487. P.~356--366.

\bibitem{lit4-1} %9
\Au{Moiseev  A., Nazarov~A.} 
Tandem of infinite-server queues with Markovian arrival process~// 
Comm. Com. Inf. Sc., 2016. Vol.~601. P.~323--333.
doi: 10.1007/978-3-319-30843-2\_34.

\bibitem{lit1-1} %10
\Au{Lisovskaya~E., Moiseeva~S., Pagano~M.}  
The total capacity of customers in the infinite-server queue with MMPP arrivals~// 
Comm. Com. Inf. Sc., 2016. Vol.~678. P.~110--120.
doi: 10.1007/978-3-319-51917-3\_11.
    


\bibitem{lit5-1} %11
\Au{Moiseev~A., Nazarov~A.} 
Queueing network MAP/(GI/$\infty$)$^K$ with high-rate arrivals~// Eur. 
J.~Oper. Res., 2016. Vol.~254. P.~161--168.
doi: 10.1016/ j.ejor.2016.04.011.





\bibitem{lit6-1}  %12
\Au{Моисеев А.\,Н., Синяков~М.\,В.} Разработка 
объ\-ект\-но-ори\-ен\-ти\-ро\-ван\-ной модели системы имитационного\linebreak 
моделирования процессов массового обслуживания~// 
Вестник Томского государственного университета. 
Управление, вычислительная техника и информатика, 2010. №\,1. С.~89--93.

\bibitem{lit3-1}  %13
\Au{Moiseev~A., Demin~A., Dorofeev~V., Sorokin~V.} 
Discrete-event approach to simulation of queueing networks~// 
Key Eng. Mater., 2016. Vol.~685. P.~939--942.
doi: 10.4028/www.scientific.net/KEM.685.939.

\end{thebibliography}
} }

\end{multicols}

 \label{end\stat}

 \vspace*{-3pt}

\hfill{\small\textit{Поступила в~редакцию  16.03.2017}}
%\renewcommand{\bibname}{\protect\rm Литература}
\renewcommand{\figurename}{\protect\bf Рис.}
\renewcommand{\tablename}{\protect\bf Таблица} 