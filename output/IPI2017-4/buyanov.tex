\def\stat{buyanov}

\def\tit{РАЗВИТИЕ МАТЕМАТИЧЕСКОЙ МОДЕЛИ УПРАВЛЕНИЯ ГРУЗОПЕРЕВОЗКАМИ 
НА~УЧАСТКЕ ЖЕЛЕЗНОДОРОЖНОЙ СЕТИ С~УЧЕТОМ СЛУЧАЙНЫХ ФАКТОРОВ$^*$}

\def\titkol{Развитие математической модели управления грузоперевозками 
%на~участке железнодорожной сети 
с~учетом случайных факторов}

\def\aut{М.\,В.~Буянов$^1$, С.\,В.~Иванов$^2$, А.\,И.~Кибзун$^3$, А.\,В.~Наумов$^4$}

\def\autkol{М.\,В.~Буянов, С.\,В.~Иванов, А.\,И.~Кибзун, А.\,В.~Наумов}

\titel{\tit}{\aut}{\autkol}{\titkol}

\index{Буянов М.\,В.}
\index{Иванов С.\,В.}
\index{Кибзун А.\,И.}
\index{Наумов А.\,В.}
\index{Buyanov M.\,V.}
\index{Ivanov S.\,V.}
\index{Kibzun A.\,I.}
\index{Naumov A.\,V.}





{\renewcommand{\thefootnote}{\fnsymbol{footnote}} \footnotetext[1]
{Результаты работы получены в~рамках выполнения государственного задания 
Минобрнауки №\,2.2461.2017/ПЧ, а~также при поддержке РФФИ 
и~ОАО <<РЖД>> в~рамках научного проекта №\,17-20-03050~офи\_м\_РЖД.}}


\renewcommand{\thefootnote}{\arabic{footnote}}
\footnotetext[1]{Московский авиационный институт 
(национальный исследовательский 
университет), \mbox{buyanovmikhailv@gmail.com}}
\footnotetext[2]{Московский авиационный институт 
(национальный исследовательский 
университет), \mbox{sergeyivanov89@mail.ru}}
\footnotetext[3]{Московский авиационный институт 
(национальный исследовательский 
университет), \mbox{kibzun@mail.ru}}
\footnotetext[4]{Московский авиационный институт 
(национальный исследовательский 
университет), \mbox{naumovav@mail.ru}}


\vspace*{-18pt}



\Abst{Предлагается математическая модель назначения локомотивов для перевозки 
грузовых составов.
Целью оптимизации в~модели является минимизация числа задействованных для 
перевозки составов локомотивов
за счет выбора маршрутов составов и~локомотивов.
Приводится детерминированный алгоритм получения субоптимального решения,
а~также алгоритм, реализующий схему оперативного планирования. Предлагается 
использование случайного параметра,
моделирующего задержку готовности состава к~отправлению.
Проводится численный эксперимент в~условиях неполной ин\-фор\-ми\-ро\-ван\-ности.
Численный эксперимент проведен на примере данных Московской железной дороги
(МЖД).
Сравниваются результаты, полученные в~детерминированной и~стохастической постановках.}

\KW{математическое моделирование; оптимизация; планирование перевозок; 
оперативное планирование}

\DOI{10.14357/19922264170411} 

\vspace*{-8pt}


\vskip 10pt plus 9pt minus 6pt

\thispagestyle{headings}

\begin{multicols}{2}

\label{st\stat}

\section{Введение}

\vspace*{-4pt}
    
Проблема организации перевозок на железнодорожном транспорте затрагивалась 
во многих работах, среди которых можно выделить~[1--8].
%\cite{AzanovBuyanov, KibzunNaumov, isuzht2015, isuzht2016, belyi, cacchiani, lazarev1, lazarev2}.
В~этих работах описана структура грузовых перевозок на железнодорожном транспорте и~предложены
различные математические модели организации грузовых перевозок.
%В \cite{AzanovBuyanov}
%предложена детерминированная модель оптимизации назначения локомотивов на сформированные
%составы по критерию минимизации общего числа задействованных локомотивов.
В~работе развивается предложенная ранее 
в~\cite{AzanovBuyanov} детерминированная модель
оптимизации назначения локомотивов на сформированные составы.
При этом учитывается влияние на модель случайных факторов, приводящих 
к~задержке готовности состава к~отправлению.
    
Необходимо отметить, что в~процессе осуществления грузовых перевозок 
возникает множество случайных факторов,
влияющих на работу локомотивов, таких как
задержки формирования составов, задержки в~движении поездов, аварии, 
неопределенное поведение диспетчеров, ошибки машинистов и~т.\,д. 
Таким образом, детерминированное решение, полученное в~\cite{AzanovBuyanov}, 
не может быть реализовано на практике и~требуется более реалистичный 
стохастический подход. Однако учет всех существующих случайных факторов 
является очень трудоемкой задачей и~приведет к~большим трудностям при 
получении решения, в~связи с~чем предлагается рассмотреть факторы, влияющие 
на время готовности состава к~отправлению.
    

В работе описываются основные случайные факторы, приводящие к~задержкам по времени,\linebreak
влияющим на время формирования составов.
Предлагается эвристический алгоритм поиска субоптимального решения задачи.
Проводится чис\-лен\-ный эксперимент с~учетом случайных факторов на примере данных 
МЖД.
Приведенные в~статье результаты сравниваются с~решением, полученным
ранее авторами~\cite{AzanovBuyanov} в~детерминированной по\-ста\-новке.

\vspace*{-12pt}

\section{Основные определения и~постановка задачи}

\vspace*{-6pt}

В основе модели, предложенной в~\cite{AzanovBuyanov},
лежит взвешенный ориентированный граф $G \hm= (V, A)$, где $V$~--- 
множество вершин; $A$~--- множество дуг.
Вершинами графа~$G$ являются значимые станции. Значимыми называются 
станции, на которых формируются грузовые составы (сортировочные станции), 
и~станции смены локомотивной тяги.\linebreak Некото\-рые значимые станции являются 
стан\-ци\-ями-де\-по, соответствующее подмножество вершин~$V$ обозначим через~$D$. 
Множеству дуг соответствуют перегоны, соединяющие значимые станции.

Локомотивы могут передвигаться только по определенным маршрутам 
(так называемым плечам), в~связи с~чем в~\cite{AzanovBuyanov} вводятся 
следующие определения.

\smallskip

\noindent
\textbf{Определение 1.}\
    Назовем плечом~$P$ последовательность дуг $a_1,\ldots, a_{I_P}$ графа~$G$, 
     удовлетворяющую следующим условиям:
    \begin{enumerate}[(1)]

    \item
        все дуги различны: $a_i \hm\ne a_j$, $i \hm\ne j$, $i,j \hm\in \{1, I_P\}$;

    \item
        первая вершина первой дуги в~последовательности совпадает 
        с~последней вершиной последней дуги последовательности, является 
        стан\-ци\-ей-де\-по и~отлична от всех промежуточных вершин последовательности: 
        $v_1\hm = v_{I_P}\hm\in D$, $ v_i \hm\ne v_1$ для $i\hm=\overline{2, I_P-1}$.
    
    \end{enumerate}

Также рассматриваются подплечи и~простые подплечи, определенные следующим образом.
    
\smallskip

\noindent
\textbf{Определение 2.}\
  Любую подпоследовательность соседних дуг $a_i, a_{i+1},\ldots, a_j$ 
   $(1\hm\leqslant i\hm<j \hm\leqslant I_P)$, образующих плечо, назовем 
   подплечом данного плеча. Любую дугу $a_i\hm=(v_{i-1},v_i)$, 
   входящую в~некоторое плечо~$P$, назовем простым подплечом плеча~$P$.
    
\smallskip

Пусть $L$~--- множество всех локомотивов, приписанных 
к~рассматриваемым стан\-ци\-ям-де\-по~$D$. Для каждого локомотива 
$l\hm\in L$ задано множество допустимых плеч~$\overline {\mathcal P}_l$, 
по которым он может передвигаться. Каждому множеству  
$\overline {\mathcal P}_l$, $l\hm\in L$, ставится в~соответствие 
множество~$\mathcal P_l$, составленное из всех простых подплеч, 
входящих в~плечи множества~$\overline {\mathcal P}_l$. Предполагается, что 
для каждого локомотива $l\hm\in L$ задана функция весовых 
норм~$w_l (\cdot) \colon {\mathcal P}_l \hm\to \mathbb{R}$, ставящая в~соответствие 
простым подплечам плеча~$\overline {\mathcal P}_l$ максимально допустимую 
для перевозки массу состава.

Пусть $S$~--- множество грузовых составов. Каж\-дый состав $s \hm\in S$ 
характеризуется массой~$w^s$, начальной станцией~$v^s_o$, станцией назначения~$v^s_f$, 
временем формирования~$t^s_o$, временем~$\tau^s_f$, до которого необходимо 
прибыть на станцию назначения, т.\,е.\ каждому составу соответствует пятерка
 $(w^s, v^s_o, t^s_o, v^s_f, \tau^s_f)$. По сути, данные характеристики 
 определяют план перевозок.

Движение локомотивов и~составов по заданному маршруту может 
осуществляться только в~определенные промежутки времени. Совокупность 
маршру\-та и~времени будем называть ниткой. По аналогии с~подплечами и~простыми 
подплечами введем в~рассмотрение поднитки и~простые поднитки. Приведем 
определения нитки, поднитки и~простой поднитки, описанные в~\cite{AzanovBuyanov}.

\smallskip

\noindent
\textbf{Определение 3.}\
    Ниткой $N$ назовем упорядоченное множество четверок 
    $(v_1, t_1, v_2, \tau_2)$, $(v_2, t_2, v_3, \tau_3)$, \ldots, 
    $(v_{I_N-1}, t_{I_N-1}, v_{I_N}, \tau_{I_N})$, удовлетворяющее условиям:
    \begin{enumerate}[(1)]

    \item
        $v_i \in V$, $i \hm= \overline{1, I_N}$, $t_i\hm \in \mathbb{R}$, $i \hm= \overline{1, I_N-1}$, $\tau_i \in \mathbb{R}$, $i = \overline{2, I_N}$;

    \item
        $(v_{i}, v_{i+1}) \in A$, $i \hm= \overline{1, I_N-1}$;

    \item
        $t_i < \tau_{i+1}$, $i\hm = \overline{1, I_N -1}$;

    \item
        $\tau_{i} \leqslant t_{i} $, $i \hm= \overline{2,I_N}$.
        
    \end{enumerate}



Во введенном определении величина~$t_i$ соответствует времени отправления со
 станции~$v_i$, а~$\tau_{i+1}$~--- времени прибытия на станцию~$v_{i+1}$. 
 Приведенные условия выражают естественные свойства движения поездов, 
 заключающиеся в~том, что движение может осуществляться только по 
 перегонам (условия~1 и~2), время отправления со станции не может быть 
 позже времени прибытия на следующую станцию (условие~3), время прибытия 
 на станцию не может быть позже времени отправления с~той же станции (условие~4).

\smallskip

\noindent
\textbf{Определение 4.}\
    Каждую подпоследовательность соседних четверок, образующих нитку~$N$, 
    назовем подниткой. Каждую четверку $(v_i, t_i, v_{i+1}, \tau_{i+1})$, 
    $i\hm=\overline{1, I_N-1}$, составляющую нитку~$N$, назовем прос\-той подниткой.

\smallskip

Пусть имеется множество~$\overline{\mathcal N}$ ниток. Сопоставим каждому элементу~$N$ 
данного множества множество~$\mathcal F(N)$,  являющееся неупорядоченным множеством 
простых подниток, составляющих нитку~$N$. Множество всех простых подниток, 
полученных из множества ниток~$\overline {\mathcal N}$, обозначим через~$\mathcal N$, 
т.\,е.\
\begin{equation*}
%    \label{1.2}
        \mathcal N = \bigcup_{N\in \overline{\mathcal N}} \mathcal F(N)\,.
\end{equation*}

Важно отметить, что каждая простая поднитка проходит только по одной из дуг графа.
    
На множестве $2^L\times \mathcal N$, являющемся декартовым 
произведением всех возможных сочетаний локомотивов и~множества 
простых подниток, определим функцию~$W(\pi_n)$, задающую максимальную массу 
состава, которую может перевезти со\-от\-вет\-ст\-ву\-ющая комбинация локомотивов $\pi_n 
\hm\subset L$ по заданной простой поднитке $n \hm\in \mathcal N$. Очевидно, 
если $\pi_n\hm = l \hm\in L$, где $n \hm= (v,t,v',\tau)$ 
и~$(v,v')\hm\in \mathcal P_l$, то $W(\pi_n) \hm= w_l((v, v'))$. 
Комбинация локомотивов~$\pi_n$ называется составным локомотивом и~используется 
для перевозки состава посредством их совместной работы.
    
Поскольку движение локомотивов осуществляется только по ниткам и~по плечам, 
приведем определение допустимого маршрута оборота локомотива из~\cite{AzanovBuyanov}.
В данном определении учтем также, что локомотив через интервалы времени~$T^{\mathrm{TO}}$ 
(48~ч) должен проходить техосмотр (ТО) продолжительностью~$t^{\mathrm{TO}}$ (8~ч). 
Будем считать, что каждый локомотив $l\hm\in L$ в~начальный момент времени 
характеризуется временем~$\tau_l^{\mathrm{TO}}$, прошедшим с~момента последнего~ТО.
Если локомотив в~начальный момент времени находится на ТО, 
то величина~$\tau_l^{\mathrm{TO}}$ принимает отрицательное значение, равное по модулю 
времени до окончания~ТО.

\smallskip

\noindent
\textbf{Определение 5.}\ %\\label{defMl}
    Допустимым маршрутом оборота~$M_l$  локомотива~$l$ относительно множества 
    плеч~$\overline {\mathcal P}_l$ назовем последовательность прос\-тых 
    подниток $(v_1, t_1, v_2, \tau_2)$,  $(v_2, t_2, v_3, \tau_3)$, 
    \ldots\linebreak $\ldots , (v_{I_l-1}, t_{I_l-1}, v_{I_l}, \tau_{I_l})$, удовлетворяющую условиям:
\begin{enumerate}[(1)]
\item $\tau_{i} \leqslant t_{i}$, $i \hm= \overline{2,I_l -1}$;

\item $(v_i,v_{i+1}) \in\mathcal {P}_l $,  $i \hm= \overline{1,I_l -1}$;

\item существует возрастающая последовательность $i_1, \ldots, i_{f_l}$ 
чисел, выбранных из множества $\{2,3,\ldots,I_l\}$ таким образом, что
\begin{enumerate}[({3}.1)]
\item $\tau^{\mathrm{TO}}_l + \tau_{i_1} \leqslant T^{\mathrm{TO}}$;
    
     \item $t_{i_j}-\tau_{i_j} \geqslant t^{\mathrm{TO}}$, $j\hm=\overline{1, f_l-1}$;
    
        \item $\tau_{i_j} - t_{i_{j-1}} \leqslant T^{\mathrm{TO}}$, $j\hm=\overline{2, f_l}$;
    
        \item $\tau_{I_l} - t_{i_{f_l}} \leqslant T^{\mathrm{TO}}$, если $f_l\hm\ne I_l$.
        
\end{enumerate}
\end{enumerate}

Условие~1 требует, чтобы время прибытия на станцию не было раньше времени 
отправления с~той же станции. Условие~2 ограничивает возможные передвижения 
локомотива только движением по плечам. Условие~3 требует прохождения ТО через 
установленные промежутки времени. Последовательности моментов времени 
$t_{i_1}, \ldots, t_{i_{f_l}}$ соответствуют моментам начала ТО. В~условии~3.1 
требуется, чтобы время ухода на первое ТО не превышало~$T^{\mathrm{TO}}$ 
с~момента предыдущего ТО. Согласно условию~3.2 время прохождения ТО не может 
быть меньше~$t^{\mathrm{TO}}_l$. Из~условия~3.3 
следует, что время между началом движения после ТО и~уходом на сле\-ду\-ющее 
ТО не может быть больше~$T^{\mathrm{TO}}$. Согласно условию~3.4 
время начала движения после последнего ТО должно быть не позже, чем за время~$T^{\mathrm{TO}}$ 
до окончания рассматриваемого периода планирования движения.

Заметим, что маршрут оборота является про\-стран\-ст\-вен\-но-вре\-мен\-н$\acute{\mbox{ы}}$м 
понятием. Множество допустимых маршрутов оборота локомотива~$l$ обозначим 
через~$\mathcal M_l$. Начальную и~конечную станции маршрута оборота~$M_l$ 
обозначим через~$v_o(M_l)$ и~$v_f(M_l)$ соответственно, время начала первой нитки 
данного маршрута оборота обозначим через~$t_o(M_l)$, время прибытия на станцию 
назначения~--- через $\tau_f(M_l)$.

Введем определение допустимого рейса состава, которое так же, 
как маршрут оборота локомотива, является 
про\-стран\-ст\-вен\-но-вре\-мен\-н$\acute{\mbox{о}}$й характеристикой и~было 
ранее описано в~\cite{AzanovBuyanov}.

\smallskip

\noindent
\textbf{Определение~6.}\ 
    Допустимым рейсом~$R_s$ состава $s\hm\in S$ назовем последовательность 
    прос\-тых подниток $(v_1, t_1, v_2, \tau_2)$,  $(v_2, t_2, v_3, \tau_3)$, \ldots$\linebreak $\ldots,
$(v_{I_s-1}, t_{I_s-1}, v_{I_s}, \tau_{I_s})$, удовлетворяющую условиям:
\begin{enumerate}[(1)]
\item  $v_1 = v^s_o$;

\item $v_{I_s} = v^s_f$;

\item  $\tau_{i} \leqslant t_{i}$, $i \hm= \overline{2,I_s -1}$;

\item  $t^s_o \leqslant t_1$;

\item  $\tau^s_f \geqslant \tau_{I_s}$.
\end{enumerate}


Условия 1 и~2 определяют начальную и~конечную станции рейса, 
условие~3 задает естественные ограничения на время отправления и~прибытия, 
условия~4 и~5 требуют выполнения перевозок согласно плану.

Множество допустимых рейсов состава~$s$ обозначим через~$\mathcal R_s$.

Так же как и~для ниток, определим множество~$\mathcal F(M_l)$ всех простых 
подниток, со\-став\-ля\-ющих маршрут оборота~$M_l$ локомотива~$l$, $l\hm\in L$, 
и~множество~$\mathcal F(R_s)$, $s\hm\in S$, всех простых подниток, со\-став\-ля\-ющих 
рейс~$R_s$ состава~$s$.

Для каждой простой поднитки $n\hm\in\mathcal N$ и~каждого набора 
маршрутов оборота локомотивов  $M\hm=\{M_l\}_{l\in L}$ определим 
множество~$\pi_{n}(M)$, со\-став\-лен\-ное из всех локомотивов, передвигающихся 
по простой поднитке~$n$ при наборе маршрутов оборота локомотивов~$M$:
\begin{equation*}
    l\in \pi_{n}(M) \Leftrightarrow n\in\mathcal F\left(M_l\right)\,.
\end{equation*}

Рассмотрим некоторый участок железнодорожной сети с~графом $G\hm = (V, A)$, 
определенным выше. Пусть задано множество локомотивов~$L$, множество составов~$S$, 
множество ниток~$\overline{\mathcal N}$ и~соответствующих простых 
подниток~$\mathcal N$, функция~$W(\cdot)$ весовых норм составных локомотивов. 
Для каждого локомотива $l\hm\in L$ определено множество плеч~$\overline{\mathcal P}_l$ 
и~простых подплеч~$\mathcal P_l$.

В начальный момент времени некоторые локомотивы могут находиться в~движении, 
поэтому будем считать, что локомотив $l\hm\in L$ можно отправить только 
с~некоторой фиксированной станции~$v_0^l$ после момента времени~$t_0^l$.
Пусть  $|L|$~--- число локомотивов в~множестве~$L$, имеющих непустой маршрут оборота.

Пусть для каждого состава $s\hm\in S$ задано множество ниток
 $\overline{\mathcal N_s}\hm\subset \overline{\mathcal N}$, 
 по которым он может быть перевезен. Через~$\mathcal N_s$ обозначим множество 
 соответствующих простых подниток. Данные ограничения
  связаны с~тем, 
 что некоторые нитки могут быть использованы только для перевозки 
 составов определенного рода.

Пусть $M=\{M_l\}_{l\in L}$~--- выбираемый набор маршрутов оборота 
всех локомотивов; $R \hm= \{R_s\}_{s\in S}$~--- выбираемый набор 
рейсов всех со\-ста\-вов; $\mathcal{M}\hm=\{\mathcal M_l\}_{l\in L}$~---
 множество допустимых\linebreak маршрутов оборота всех локомотивов; 
 $\mathcal {R}\hm = \{\mathcal R_s\}_{s\in S}$~--- множество допустимых рейсов
  всех составов.

Требуется найти такой набор $M$ маршрутов оборота локомотивов и~такой набор~$R$ 
рейсов составов, при котором общее число~$|L|$ локомотивов, используемых 
для перевозки составов, будет минимальным, при этом все рейсы 
составов будут покрыты маршрутами локомотивов.
    
В~\cite{AzanovBuyanov} была предложена следующая постановка задачи:

\noindent
\begin{equation}
    \label{problem_main}
    |L| \to \min_{M\in \mathcal M, R\in\mathcal R}
\end{equation}
    при ограничениях
    
    \noindent
\begin{gather}
        M_{l}\in\mathcal M_l\,,\enskip l\in L\,;    \label{c1}\\
    \label{c2}
        R_s \in \mathcal R_s\,,\enskip s\in S\,;\\
    \label{c3}
           \bigcup\limits_{s\in S} \mathcal F(R_s) \subset \bigcup\limits_{l\in L} \mathcal F(M_l)\,;\\
    \label{c4}
        \mathcal F(R_s)\cap \mathcal F(R_{s'}) = \varnothing\,, 
        \enskip s \ne s', \enskip s, s'\in S\,;\\
    \label{cadd1}
        W(\pi_n)\geqslant w^s\,,\enskip n\in\mathcal F(R_s)\,,\enskip s\in S\,;\\
    \label{c5}
        \mathcal F(M_l) \subset \mathcal N\,,\enskip l \in L\,;\\
    \label{c6}
        \mathcal F(R_s) \subset \mathcal N_s\,,\enskip s\in S\,;\\
    \label{c7}
        v_0(M_l) = v_0^l\,;\\
    \label{c8}
        t_0(M_l) \geqslant t_0^l\,.
\end{gather}

Условия~(\ref{c1}) и~(\ref{c2}) значат, что рассматриваются только допустимые 
маршруты локомотивов и~рейсы составов, в~частности те, для которых 
существуют допустимые плечи. Также заметим, что допустимость 
рейсов составов требует, чтобы был выполнен план перевозок в~установленный срок.

Условие~(\ref{c5}) требует, чтобы маршруты оборота локомотивов составлялись 
только из простых подниток, поскольку множество $\bigcup\nolimits_{l\in L} 
\mathcal F(M_l)\hm \subset \mathcal N$ со\-став\-ле\-но из простых подниток, 
входящих в~ка\-кой-ли\-бо маршрут оборота локомотива. Условие~(\ref{c6})\linebreak 
задает аналогичное требование для рейсов составов, а~кроме этого оно 
ограничивает выбор допустимых ниток для перевозки состава. Условие~(\ref{c3}) 
означает, что все простые поднитки, образующие рейс некоторого состава, 
используются для движения некоторого локомотива, т.\,е.\
 все составы перевозятся локомотивами. Также из этого условия 
 следует, что локомотивы могут передвигаться по простым подниткам, 
 по которым не движутся составы. Таким образом, каждой задействованной 
 нитке со\-от\-вет\-ст\-ву\-ет либо состав с~локомотивом (возможно, 
 с~несколькими локомотивами), либо локомотив, движущийся порожняком.

Условие~(\ref{c4}) означает, что рейсы составов не могут пересекаться, т.\,е.\
 одну простую поднитку нельзя использовать для передвижения двух составов. 
 Поскольку локомотивы могут ехать в~сплотке или с~составом (так называемый 
 вспомогательный пробег), то подобное условие для локомотивов отсутствует.

Условие~(\ref{cadd1}) требует, чтобы были выполнены весовые нормы составных 
локомотивов при перевозке составов, т.\,е.\ составной локомотив~$\pi_n$, 
используемый на простой поднитке $n \hm\in \mathcal F(R_s)$, по которой 
перевозится состав~$s$, должен иметь возможность перевозить состав массой~$W(\pi_n)$, 
не меньшей чем масса~$w^s$ состава~$s$.

Условия~(\ref{c7}) и~(\ref{c8}) задают начальное состояние локомотивов.

Заметим также, что множество составов~$S$ и~множество ниток~$\overline{\mathcal N}$ 
определяются суточным планом перевозок и~количеством суток, на которые осуществляется 
планирование.

Сформулированная задача предполагает оптимизацию как по маршрутам оборота локомотивов, 
так и~по рейсам составов. Однако на практике нитки уже сформированы под конкретные 
составы, поэтому в~дальнейшем будем считать, что множество допустимых 
рейсов~$\mathcal R_s$ состава $s\hm\in S$ состоит из одного рейса. 
Таким образом, задача сводится к~назначению локомотивов для перевозки 
составов с~заданными рейсами, т.\,е.\ к~поиску набора маршрутов~$M$.


Для исследования существования решения задачи необходимо определить, 
является ли набор ниток достаточным
для осуществления плана перевозок.
Решение данной задачи сравнимо с~решением исходной задачи. Ниже приведен 
эвристический алгоритм решения задачи,
который в~ряде случаев позволяет находить допустимое решение задачи.
Вопрос о единственности решения не является актуальным с~практической точки зрения,
так как достаточно найти хотя бы одно решение, обеспечивающее минимальное значение 
целевой функции.
Вычислительная сложность данной задачи в~работе не исследуется, однако можно заметить,
что время перебора всех допустимых маршрутов локомотивов и~рейсов составов
зависит экспоненциально от объема исходных данных задачи.

\vspace*{-2pt}

\section{Алгоритм решения детерминированной задачи}

\vspace*{-2pt}

Опишем алгоритм получения субоптимального решения задачи~(\ref{problem_main}).
Будем считать, что рейсы всех составов определены, т.\,е.\ для каждого состава 
определена нитка,
по которой он движется. В~основе алгоритма лежит идея о~максимальном использовании 
локомотива
с~минимальным временем начала движения. Решение предполагает неограниченное число 
локомотивов
в~начальный момент времени в~каждом депо, однако даже такое предположение 
позволяет получить лучший,
в~сравнении с~реаль-\linebreak\vspace*{-12pt}

\pagebreak

\noindent
ным движением, результат.
Для простоты изложения алгоритмов при их построении не учитываются 
ограничения на массу перевозимых составов и~опускается описание алгоритма проведения 
ТО. При наличии ограничений на массу составов необходимо осуществлять поиск не только простых локомотивов $l\in L$,
но и~составных локомотивов, т.\,е.\ комбинаций нескольких локомотивов.

\vspace*{-6pt}
    
\subsection{Алгоритм назначения} \label{find1}

\vspace*{-2pt}
    
Пусть задано непустое множество составов $S \hm= \{s_i\mid i = \overline{1, |S|}\}$ 
с~непустыми рейсами.
Пусть~$v^l_f$ и~$\tau^l$~--- конечная станция маршрута оборота~$M_l$ (либо начальная станция в~случае пустого маршрута)
и время прибытия на эту станцию локомотива $l\hm\in L$. Для корректной работы алгоритма 
необходимо,
чтобы элементы множества~$S$ были упорядочены по возрастанию времени отправления составов.
Данный алгоритм является упрощенной версией алгоритма в~\cite{AzanovBuyanov},
при этом для ряда примеров решения, получаемые с~по\-мощью данных алгоритмов, совпадают.
    
\renewcommand{\figurename}{\protect\bf Алгоритм}

\begin{figure*} 
\hrule

\vspace*{-4pt}

\Caption{\ }  

\vspace*{3pt}  
\hrule

\vspace*{2pt}
        \begin{enumerate}[1.]
        \setcounter{enumi}{-1}
        
        \item
        Полагаем $i:=1, j:=1, k:=1$.
        
        \item
        Зафиксируем локомотив $l_k \in L$, состав $s_i \hm\in S$ и~простую поднитку 
        из рейса состава $n_j \hm= (v_o(n_j), t(n_j), v_f(n_j), \tau(n_j))$, 
        $n_j \hm\in \mathcal F(R_{s_i})$, переходим к~шагу~3.
        
        \item
        Если $k > |L|$, то берем новый локомотив $L:=L \cup \{l_k\}$, 
        $i:=1$, $j:=1$ и~повторяем шаг~2; если $i\hm > |S|$, то переходим 
        к~следующему локомотиву $k := k+1$, $i:=1$, $j:=1$ и~повторяем шаг~2; 
        если $j\hm > |\mathcal F(R_s)|$, то переходим к~следующему составу 
        $i := i+1$, $j:=1$ и~повторяем шаг~2. Переходим к~шагу~3.
        
        \item
        Если $\tau^{l_k} \leqslant t(n_j)$, $(v_o(n_j), v_f(n_j))\hm \in 
        \mathcal P_{l_k}$, переходим к~шагу~4. Иначе переходим к~шагу~2.
        
        \item
        Если $v^{l_k}_f \ne v_o(n_j)$, выполняем поиск нитки~$N^*$ 
        для перегонки локомотива~$l_k$ к~началу простой поднитки~$n_j$, согласно 
        подразд.~3.2. Если $v^{l_k}_f \hm= v_o(n_j)$, полагаем 
        $N^*:=\varnothing$. Если нитка~$N^*$ найдена, переходим к~шагу~5, 
        иначе переходим к~шагу~2.

        \item
        Внесем найденную простую поднитку~$n_j$ и,~если необходимо, соответствующую 
        ей нитку для перегонки~$N^*$ в~маршрут локомотива $M_{l_k} \hm= M_{l_k} 
        \cup N^* \cup \{n_j\}$. Уберем простую поднитку~$n$ из рейса состава 
        $R_{s_i} := R_{s_i} \setminus \{n_j\}$, если $\mathcal F(R_{s_i})\hm = 
        \varnothing$, то уберем состав из множества рассматриваемых
         $S := S \setminus \{s_i\}$. Если $S \hm= \varnothing$, 
         переходим к~шагу~6, иначе переходим к~шагу~2.

        \item
        Окончание алгоритма, получено субоптимальное решение 
        задачи~(\ref{problem_main}).

    \end{enumerate}
    \hrule
 %   \vspace*{4pt}
\end{figure*}

%\vspace*{-9pt}


\vspace*{-6pt}

\subsection{Поиск составной нитки} \label{N*}

\vspace*{-2pt}

Для осуществления перегонки локомотива необходимо найти нитку~$N^*$, 
соединяющую станцию~$v^l_f$, на которой находится локомотив~$l$,\linebreak
 и~станцию~$v_o(n)$, 
с~которой отправляется прос\-тая поднитка~$n$. Пусть~$t(n)$~--- 
время начала движения по простой поднитке~$n$, $\tau(n)$~--- 
время окончания прос\-той поднитки~$n$, а~$\tau^l$~--- 
время остановки локомотива на станции~$v^l_f$. Через~$\mathcal N_a$ 
обозначим множество прос\-тых подниток, соответствующих дуге $a\hm\in A$. 
Сопоставим каждой дуге графа~$G$ весовую характеристику, равную среднему 
времени движения по ней:
\begin{equation}
    \label{w_a}
    w_a = \fr {1}{|\mathcal N_a|}\sum\limits_{n\in \mathcal N_a} (\tau(n) - t(n))\,.
\end{equation}

Полагаем $N^*$ равной нитке, проходящей по кратчайшему пути в~графе~$G$, 
взвешенном согласно~(\ref{w_a}), соединяющему станции~$v^l_f$ и~$v_o(n)$, 
с~временем начала не ранее~$\tau^l$ и~временем окончания не позднее~$t(n)$. 
Для поиска кратчайших путей между вершинами взвешенного ориентированного графа 
можно использовать, например, алгоритм Флой\-да--Уор\-шел\-ла~\cite{Floyd}.

\vspace*{-6pt}

\section{Статистическое моделирование}

\vspace*{-2pt}

В процессе осуществления грузовых перевозок возникает множество случайных 
факторов, влияющих на работу локомотивов, таких как
задержки формирования составов, задержки в~движении поездов и~другие 
нештатные ситуации. 
В~связи с~этим детерминированное решение, полученное в~\cite{AzanovBuyanov}, 
не может быть реализовано на практике. Будем моделировать задержки во времени 
формирования состава.

\renewcommand{\figurename}{\protect\bf Алгоритм}

\begin{figure*}[b] %\label{alg:2} %fig2
\hrule

\vspace*{-4pt}

\Caption{\ }  

\vspace*{3pt}  
\hrule

\vspace*{2pt}

    \begin{enumerate}[1.]
        \setcounter{enumi}{-1}

        \item
        Полагаем $i := 0$.

        \item
        Из множества составов $S_i$ выберем подмножество~$S_i^{\Delta T}$ 
        такое, что для всех $s \hm\in S_i^{\Delta T}$
        выполнено $T_o \hm+ i\Delta T \leqslant t^s_o \hm\leqslant T_o \hm+ 
        (i+1)\Delta T$. Пусть теперь для каждого состава~$s$
        из множества~$S_i^{\Delta T}$ задано время фактической готовности 
        к~отправлению $\tau^s_o \hm= t^s_o \hm+ \xi_s$. Переходим к~шагу~2.

        \item
        Определим рейсы составов. Для каждого состава $s \hm\in S_i^{\Delta T}$ 
        выберем нитку $n \hm\in \overline{\mathcal N_s}$ такую,
        чтобы разница во времени отправления~$t(n)$ по нитке~$n$ 
        и~времени фактической готовности~$\tau_o^s$ состава~$s$
        к~отправлению была минимальной, и~внесем ее в~рейс состава 
        $R_s := R_s \cup \{n\}$.
        Для $s \hm\in S_i \setminus S_i^{\Delta T}$ выберем нитку 
        $n \hm\in \overline{\mathcal N_s}$ такую, чтобы разница
        во времени отправления~$t(n)$ по нитке~$n$ и~планируемого времени 
        формирования~$t_o^s$  состава~$s$ была минимальной,
        и~внесем ее в~рейс состава $R_s := R_s \cup \{n\}$. Переходим к~шагу~3.

        \item
        Для полученного множества составов~$S_i$ с~заданными рейсами 
        и~множества локомотивов~$L$ выполним назначение локомотивов согласно
         подразд.~3.1. Зафиксируем маршруты локомотивов на момент времени~$T_o \hm+ 
         i\Delta T$, т.\,е.\ уберем из маршрутов локомотивов все простые поднитки
          со временем отправления, превышающим $T_o \hm+ i\Delta T$. Переходим к~шагу~4.

        \item
        Примем $i := i+1$. Если $i \hm> [T_m/({\Delta T})]$, переходим к~шагу~5, 
        иначе переходим к~шагу~1.

        \item
        Окончание алгоритма, получено субоптимальное решение.
    \end{enumerate}
    \hrule
%    \vspace*{6pt}
\end{figure*}


Задержки во времени готовности состава к~отправлению могут возникать в~результате 
множества причин.
Такими причинами могут стать, например, ошибки при планировании работы станции, 
включающие в~себя как работу маневровых локомотивов,
так и,~например, неверную очередность формирования и~расформирования составов, 
задержки в~прибытии вагонов (грузов),
участвующих в~составообразовании на станцию отправления, нарушение технических 
нормативов в~результате человеческого фактора,
погодных и~иных явлений и~т.\,д. В общем случае все описанные факторы в~том или 
ином виде приводят к~изменению времени
готовности состава к~отправке. Таким образом, сведем учет всех случайных факторов,
связанных с~задержками по времени, к~одной случайной величине, моделирующей 
задержку формирования состава.
{\looseness=-1

}

Для моделирования случайных задержек по времени будем использовать 
случайную величину~$\xi_s$,
имеющую экспоненциальное распределение с~параметром~$\lambda_s$, 
которое обознается через $E(\lambda_s)$.
Для моделирования случайных задержек в~транспортных системах 
традиционно используют экспоненциально распределенные случайные величины. 
Например, в~\cite{KibzunNaumovUlanov} на основе статистического анализа данных 
был предложен экспоненциальный закон распределения времени задержки 
прибытия в~аэропорт самолета, выполняющего рейс по расписанию.

Помимо долгосрочного планирования перевозок, которое главным образом 
позволяет проводить оценку важных эксплуатационных
показателей\linebreak ра\-бо\-ты железной дороги, также существует так называемое 
оперативное планирование, которое является основным инструментом 
организации железнодорожных перевозок. Отметим, что в~работе~\cite{AzanovBuyanov} 
решается задача долгосрочного пла\-ни\-ро\-вания. 
{\looseness=-1

}

Оперативное планирование 
на железнодорожном транспорте состоит из нескольких этапов. %\\[-16pt]
\begin{enumerate}[1.]
    \item
    Разработка и~утверждение плана перевозок на период времени, равный~$T$. %\\[-16pt]
    \item
    Корректировка сформированного плана перевозок с~учетом фактического расположения 
    локомотивов и~составов,
    а~также общего состояния железнодорожной сети через интервалы времени~$\Delta T$.
\end{enumerate}

%\vspace*{-2pt}

Будем считать, что точное время готовности состава к~отправлению известно 
в~текущий момент времени на период планирования~$\Delta T$,
поэтому схема оперативного управления подвижным составом 
заключается в~последовательном решении детерминированной
задачи формирования маршрутов движения локомотивов согласно алгоритму~1 
через промежутки времени~$\Delta T$
с~учетом точного знания времени готовности составов к~отправлению на время~$\Delta T$ 
вперед и~планового времени го\-тов\-ности
составов к~отправлению в~оставшийся период времени планирования.
 
 При этом в~качестве начальных условий решения задачи о~назначении локомотивов 
каждый раз принимается
фактическое расположение локомотивов на железнодорожной сети, сложившееся на 
момент корректировки
с~учетом всех показателей функционирования локомотивов (необходимость прохождения ТО, 
плечи и~т.\,д.).

\begin{table*}\small %tabl1
\begin{center}
    \Caption{Характеристики входных данных}
\vspace*{2ex}

        \begin{tabular}{|c|c|c|c|c|c|c|c|}
         \hline
\tabcolsep=0pt\begin{tabular}{c}Число\\ станций \end{tabular}& 
\tabcolsep=0pt\begin{tabular}{c}Число\\ станций-депо\end{tabular} &
\tabcolsep=0pt\begin{tabular}{c}Число\\ сортировочных\\ станций\end{tabular}&
\tabcolsep=0pt\begin{tabular}{c}Число\\ составов\\ в~суточном\\ задании\end{tabular} &
\tabcolsep=0pt\begin{tabular}{c}Число\\ ниток\\ на сутки\end{tabular} &
\tabcolsep=0pt\begin{tabular}{c}Период\\ моделирования\\ $T_m$, сут\end{tabular} &
\tabcolsep=0pt\begin{tabular}{c}Дискретность\\ оперативного\\ управления\\
$\Delta T$, ч\end{tabular} \\
\hline
40 & 16  & 16 & 598 & 1254 & 10  & 3 \\ 
\hline
        \end{tabular}
    \end{center}
%\vspace*{-6pt}
\end{table*}

Опишем эвристический алгоритм поиска субоптимального решения задачи 
назначения локомотивов для осуществления грузоперевозок
по железнодорожной сети, реализующий схему оперативного планирования. Пусть~$T_o$~--- 
время начала моделирования.
Пусть~$S_i$~--- множество составов, которые необходимо перевезти в~интервал 
времени $[T_o\hm + i\Delta T, T_o \hm+ i\Delta T \hm+ T]$,
где $i \hm= \overline{0, [{T_m}/({\Delta T}) ]}$. Множества~$S_i$ 
становятся известными за время~$\Delta T$
до начала соответствующего интервала. Пусть для каждого состава $s \hm\in S_i$ 
задано планируемое время формирования~$t^s_o$.
Обозначим через~$\overline{\mathcal N_s}$ множество ниток, по 
которым может быть перевезен состав~$s$
из расчета планируемого времени фор\-ми\-ро\-вания.
{ %\looseness=1

}



Описанный выше алгоритм позволяет получить субоптимальное 
решение задачи~(\ref{problem_main}) при условии неопределенности
во времени готовности состава, используя принцип оперативного планирования. 
Неопределенность заключается в~отсутствии точной информации 
о~времени готовности составов к~отправлению на весь рассматриваемый период 
планирования. Согласно принципу оперативного планирования производится 
многократная корректировка плана перевозок через интервалы времени~$\Delta T$. 
В~каждый рассматриваемый интервал времени имеется точная информация 
о~времени готовности составов только из этого интервала, для остальных 
известно только планируемое время готовности, которое может сильно отличаться 
от фактического.

Проверим адекватность решения, получаемого с~помощью алгоритма~2, 
в~случае стохастической постановки
с~учетом случайного времени формирования составов. Численный эксперимент проводился 
на примере данных участка МЖД за определенный период 
времени. Характеристики входных данных приведены в~табл.~1.



Вычисления были проведены с~учетом ограничений на ТО. Предполагается, 
что локомотивы должны проходить ТО продолжительностью
не менее~8~ч не позже, чем через~48~ч после предыду\-ще\-го~ТО.

%\end{multicols}






%\begin{multicols}{2}

Согласно алгоритму~2, составление рейса состава происходит при помощи выбора 
ближайшей по времени отправления
нитки относительно фактического времени формирования.
Параметр~$\lambda_s$ распределения случайной величины~$\xi_s$ зависит 
от множества факторов, например числа вагонов в~составе,
структуры станции и~др. Так как описанная модель не включает в~себя 
процесс составообразования,
вагонопотоки и~станционные работы, вопрос выбора~$\lambda_s$ остается вне 
рамок данной работы.\linebreak\vspace*{-12pt}


\columnbreak

%\begin{table*}
{\small %tabl2
    \begin{center}
    {{\tablename~2}\ \ \small{Сравнение результатов}}
    
    \vspace*{2ex}


        \begin{tabular}{|l|c|c|c|} 
        \hline
        \multicolumn{1}{|c|}{Вариант} & 
        \tabcolsep=0pt\begin{tabular}{c}Число\\ локомо-\\ тивов\end{tabular} & 
       \tabcolsep=0pt\begin{tabular}{c} Число\\ переве-\\ зенных\\ составов\end{tabular}&
        \tabcolsep=0pt\begin{tabular}{c}Макси-\\мальное\\ время\\ задержки\\ состава, ч\end{tabular}\\
        \hline
  Решение из \cite{AzanovBuyanov} &  369& 5920& 0\hphantom{,3}\\
  $T=12$~ч &  405& 5710& 6,3\\
  $T=24$~ч &  440& 5783& 5,1\\
  $T=48$~ч & 422& 5659& 6,7\\ 
 \hline
        \end{tabular}
    \end{center}}
%\end{table*}

\vspace*{9pt}

\noindent
 Оценки параметров~$\lambda_s$ получены исходя из 
обработки реальных статистических данных.



Численный эксперимент проводился с~исходными данными, представленными в~табл.~1.
Для
 каждого $T \hm\in \{12~\mbox{ч}, 24~\mbox{ч}, 48~\mbox{ч}\}$
выполнено~100~реализаций алгоритма.
Кроме основного критерия, числа используемых локомотивов, также оценены 
выполнение плана перевозок и~максимальное время задержки составов. Для каждого~$T$ представим в~табл.~2 
средние значения для основного критерия и~описанных характеристик.



Из табл.~2 видно, что значение критерия относительно детерминированной постановки 
показало среднее изменение
в большую сторону не более чем на~20\%. При этом менее~7\%~составов не было
 перевезено, что объясняется недостатком ниток.
Максимальное время задержки составов при этом достигало около~7~ч.
В~реальных условиях управления грузовыми перевозками такие составы отправляются 
вне нормативных ниток.
Относительные доли основных характеристик движения локомотивов, а~именно: 
время, проведенное в~работе (полезный пробег);
время, затраченное на проведение ТО; время, затраченное на перегонки 
(холостой пробег), а~также время простоя~--- не изменились.

Общий локомотивный парк на МЖД составляет около~900~локомотивов, 
ежедневно используется примерно~700~локомотивов.
В~результате численного эксперимента в~худшем случае ($T = 24$~ч) 
получено~440~локомотивов,
что в~сравнении с~реальными данными является очень хорошим показателем. Однако 
следует заметить,
что такие показатели, с~одной стороны, связаны с~наличием некоторой части 
неперевезенных составов в~результате полученного решения,
а~с~другой стороны, возможно, не всеми нюансами функционирования системы
 железнодорожных перевозок,
учтенными в~виде ограничений в~рассматриваемой модели. Тем не менее 
полученные результаты показывают, что учет случайных факторов необходим в~подобных 
моделях, поскольку оказывает значительное влияние (порядка~20\%) 
на основные показатели функционирования системы даже с~учетом нового 
предложенного алгоритма оперативного управления.

\vspace*{-6pt}

\section{Заключение}

\vspace*{-2pt}

В работе описана и~исследована математическая модель назначения локомотивов 
для перевозки составов.
Результаты численных экспериментов показали, что случайные возмущения, 
связанные со временем формирования составов,
имеют  влияние порядка~20\% на значение основного критерия и~характеристики
 движения локомотивов,
что подтверждает сравнение с~результатами, полученными в~\cite{AzanovBuyanov},
а~также создает дополнительную проблему~--- нехватку ниток для перевозки 
всех составов.
В~дальнейших исследований планируется изучить влияние других случайных факторов
на эффективность использования локомотивного парка.
Также планируется учесть необходимость прохождения нескольких видов 
технического обслуживания,
ограничения на тип тяги локомотива и~массу перевозимого состава и~другие факторы.

\renewcommand{\figurename}{\protect\bf Рис.}


{\small\frenchspacing
 {%\baselineskip=10.8pt
 \addcontentsline{toc}{section}{References}
 \begin{thebibliography}{99}

\bibitem{belyi}  %1
\Au{Белый О.\,В., Кокурин И.\,М.}
Организация грузовых железнодорожных перевозок: пути оптимизации~// 
Транспорт Российской Федерации, 2011. №\,4(35). С.~28--30.

\bibitem{KibzunNaumov}  %2
\Au{Кибзун А.\,И., Наумов~А.\,В., Иванов~С.\,В.}
Двухуровневая задача оптимизации деятельности железнодорожного транспортного узла~// 
Управление большими сис\-те\-ма\-ми, 2012. №\,38. С.~140--160.

\bibitem{lazarev1} %3
\Au{Лазарев А.\,А., Мусатова Е.\,Г.}
Целочисленные постановки задачи формирования железнодорожных составов и~расписания 
их движения~// Управление большими системами, 2012. №\,38. С.~161--169.

\bibitem{lazarev2} %4
\Au{Лазарев~А.\,А., Мусатова~Е.\,Г., Гафаров~Е.\,Р., Кварацхелия~А.\,Г.}
Теория расписаний. Задачи железнодорожного планирования.~--- М.: ИПУ РАН, 2012.
92~с.

\bibitem{isuzht2015}  %5
\Au{Гайнанов Д.\,Н., Иванов С.\,В., Кибзун А.\,И., Осокин А.\,В.}
Модель оптимального назначения локомотивов при формировании грузовых составов~// 
Интеллектуальные системы управления на железнодорожном транспорте: Тр. 
4-й научн.-технич. конф. с~междунар. учас\-ти\-ем.~--- М.: НИИАС, 2015. С.~45--47.

\bibitem{cacchiani}  %6
\Au{Cacchiani V., Galli~L., Toth~P.}
A~tutorial on non-periodic train timetabling and platforming problems~// 
EURO J.~Transportation Logistics, 2015. Vol.~4. No.\,3. P.~285--320.

\bibitem{AzanovBuyanov}  %7
\Au{Azanov V.\,M., Buyanov~M.\,V., Gaynanov~D.\,N., Ivanov~S.\,V.}
Algorithm and software development to allocate locomotives for transportation 
of freight trains~//
%Вестн. ЮУрГУ. Сер. Матем. моделирование и~программирование, 9:4 (2016), 73-85.
Bull. South Ural State University. 
Ser. Math. Modelling Programming  Computer Software, 2016. 
Vol.~9. No.\,4. P.~73--85.

\bibitem{isuzht2016}  %8
\Au{Азанов В.\,М., Буянов М.\,В., Иванов С.\,В., Кибзун А.\,И., 
Наумов А.\,В., Гайнанов Д.\,Н.}
Оптимизация локомотивного парка, предназначенного для перевозки грузовых составов~// 
Интеллектуальные системы управ\-ле\-ния на железнодорожном транспорте: Тр. 5-й\linebreak 
на\-учн.-тех\-нич. конф. с~междунар. участием.~--- М.: \mbox{НИИАС}, 2016. С.~94--96.
    

\bibitem{Floyd}  %9
\Au{Floyd~R.\,W.} Algorithm~97~--- shortes path~// Comm. ACM, 1962. 
Vol.~5. No.\,6. P.~345.

\bibitem{KibzunNaumovUlanov} %10
\Au{Кибзун А.\,И., Наумов~А.\,В., Уланов~С.\,В.}
Стохастический алгоритм управления летным парком авиакомпании~// 
Автоматика и~телемеханика, 2000. №\,8. С.~126--136.

 \end{thebibliography}

 }
 }

\end{multicols}

\vspace*{-3pt}

\hfill{\small\textit{Поступила в~редакцию 17.04.17}}

\vspace*{8pt}

%\newpage

%\vspace*{-24pt}

\hrule

\vspace*{2pt}

\hrule

%\vspace*{8pt}


\def\tit{DEVELOPMENT OF~THE~MATHEMATICAL MODEL OF~CARGO TRANSPORTATION CONTROL 
ON~A~RAILWAY NETWORK SEGMENT TAKING INTO~ACCOUNT RANDOM FACTORS}

\def\titkol{Development of~the~mathematical model of~cargo transportation control 
%on~a~railway network segment 
taking into~account random factors}

\def\aut{M.\,V.~Buyanov, S.\,V.~Ivanov, A.\,I.~Kibzun, and A.\,V.~Naumov}

\def\autkol{M.\,V.~Buyanov, S.\,V.~Ivanov, A.\,I.~Kibzun, and A.\,V.~Naumov}

\titel{\tit}{\aut}{\autkol}{\titkol}

\vspace*{-9pt}


\noindent
Moscow Aviation Institute (National Research University), 
4~Volokolamskoye Highway,
Moscow 125993, Russian Federation



\def\leftfootline{\small{\textbf{\thepage}
\hfill INFORMATIKA I EE PRIMENENIYA~--- INFORMATICS AND
APPLICATIONS\ \ \ 2017\ \ \ volume~11\ \ \ issue\ 4}
}%
 \def\rightfootline{\small{INFORMATIKA I EE PRIMENENIYA~---
INFORMATICS AND APPLICATIONS\ \ \ 2017\ \ \ volume~11\ \ \ issue\ 4
\hfill \textbf{\thepage}}}

\vspace*{3pt}
    


\Abste{A mathematical model for the assignment of locomotives for the transport 
of freight trains is proposed. In the model, the purpose of optimization is to
 minimize the number of locomotives involved in transportation of trains due 
 to the choice of routes for trains and locomotives. A~deterministic algorithm 
 for obtaining a~suboptimal solution is given as well as an algorithm that 
 implements the operational planning scheme. It is proposed to use\linebreak\vspace*{-12pt}}
 
 \Abstend{a~random parameter that simulates the delay in the readiness of a~train for 
 departure. The numerical experiment was performed in conditions of incomplete 
 information using the data of the Moscow Railway. The results obtained in 
 deterministic and stochastic statements are compared.}

\KWE{mathematical modeling; optimization; transportation planning; 
operational planning}

\DOI{10.14357/19922264170411} 

\vspace*{-12pt}

\Ack
\noindent
This work is a part of Project No.\,2.2461.2017 supported 
by the Russian Ministry of Education and Science.
This work is also supported by the Russian Foundation for Basic
Research and Russian Railways (project 17-20-03050~ofi\_m\_RZhD).





%\vspace*{3pt}

  \begin{multicols}{2}

\renewcommand{\bibname}{\protect\rmfamily References}
%\renewcommand{\bibname}{\large\protect\rm References}

{\small\frenchspacing
 {%\baselineskip=10.8pt
 \addcontentsline{toc}{section}{References}
 \begin{thebibliography}{99}

\bibitem{5-bu-1} %1
\Aue{Belyy, O.\,V., and I.\,M.~Kokurin.} 2011. Organizatsiya gruzovykh 
zheleznodorozhnykh perevozok: puti optimizatsii 
[Organization of freight rail transportation: Ways to optimize].
\textit{Transport Rossiyskoy Federatsii} [Transport of the Russian Federation]
4(35):28--30.
\bibitem{2-bu-1} %2
\Aue{Kibzun, A.\,I., A.\,V.~Naumov, and S.\,V.~Ivanov.} 
2012. Dvukhurovnevaya zadacha optimizatsii deyatel'nosti 
zhe\-lez\-no\-do\-rozh\-no\-go transportnogo uzla [Bilevel optimization problem for 
railway transport hub planning]. \textit{Upravlenie bol'shimi sistemami}
[Large-Scale Systems Control] 38:140--160.
\bibitem{7-bu-1} %3
\Aue{Lazarev, A.\,A. and E.\,G.~Musatova.} 2012. Tselochislennye postanovki 
zadachi formirovaniya zheleznodorozhnykh sostavov 
i~raspisaniya ikh dvizheniya [Integer formulations of
 freight train design and scheduling problems].
 \textit{Upravlenie bol'shimi sistemami} [Large-Scale Systems Control] 38:161--169.
\bibitem{8-bu-1} %4
\Aue{Lazarev, A.\,A., E.\,G.~Musatova, E.\,R.~Gafarov, and A.\,G.~Kvarachelija.} 
2012. \textit{Teoriya raspisaniy. Zadachi zheleznodorozhnogo planirovaniya} 
[Theory of schedules. Railway planning problems]. Moscow: IPU RAN. 92~p.
\bibitem{3-bu-1} %5
\Aue{Gaynanov, D.\,N., S.\,V.~Ivanov, A.\,I.~Kibzun, and A.\,V.~Oso\-kin.} 
2015. Model' optimal'nogo naznacheniya lokomotivov pri formirovanii
 gruzovyh sostavov
[Model of the optimal assignment of locomotives in the formation of freight
trains]. \textit{Tr. 4-y nauchn.-tekhnich. konf. s~mezhdunarodnym uchastiem 
``Intellektual'nye sistemy upravleniya na zheleznodorozhnom transporte''} 
[4th Scientific and Technical Conference with International Participation 
``Intelligent Control Systems in Railway Transport'' Proceedings]. Moscow. 45--47.

\bibitem{6-bu-1} %6
\Aue{Cacchiani, V., L.~Galli, and P.~Toth.} 2015. 
A~tutorial on non-periodic train timetabling and platforming problems. 
\textit{EURO J.~Transportation Logistics} 4(3):285--320.

\bibitem{1-bu-1} %7
\Aue{Azanov, V.\,M., M.\,V.~Buyanov, D.\,N.~Gaynanov, and S.\,V.~Ivanov.} 2016.
Algorithm and software development to allocate locomotives for transportation 
of freight trains. \textit{Bull. South Ural State University. 
Ser. Math. Modelling Programming Computer Software} 9(4):73--85.
\bibitem{4-bu-1} %8
\Aue{Azanov, V.\,M., M.\,V.~Buyanov, S.\,V.~Ivanov, A.\,I.~Kibzun, A.\,V.~Naumov, 
and D.\,N.~Gaynanov.} 2016. Optimizatsiya lokomotivnogo parka, prednaznachennogo 
dlya perevozki gruzovykh sostavov [Optimization of locomotive 
fleet intended for transportation of freight trains]. 
\textit{Tr. 5-y nauchn.-tekhnich. konf. s~mezhdunarodnym uchastiem 
``Intellektual'nye sistemy upravleniya na zheleznodorozhnom transporte''} 
[5th Scientific and Technical Conference with International Participation 
``Intelligent Control Systems in Railway Transport'' Proceedings]. Moscow. 94--96.



\bibitem{9-bu-1} %9
\Aue{Floyd, R.\,W.} 1962. Algorithm~97: Shortes path. \textit{Comm. ACM} 5(6):345.
\bibitem{10-bu-1}
\Au{Kibzun, A.\,I., A.\,V.~Naumov, and S.\,V.~Ulanov.}
 2000. A~stochastic control algorithm for aircraft allocation. 
 \textit{Automat. Rem. Contr.} 61(8):1355--1363.
\end{thebibliography}

 }
 }

\end{multicols}

\vspace*{-6pt}

\hfill{\small\textit{Received April 17, 2017}}

%\vspace*{-10pt}

\Contr

\noindent
\textbf{Buyanov Mikhail V.} (b.\ 1994)~--- 
PhD student, Moscow Aviation Institute (National Research University), 
4~Volokolamskoye Highway, 
Moscow 125993, Russian Federation; \mbox{buyanovmikhailv@gmail.com}

\vspace*{3pt}

\noindent
\textbf{Ivanov Sergey V.} (b.\ 1989)~--- 
Candidate of Science (PhD) in physics and mathematics,  
associate professor, Moscow Aviation Institute (National Research University), 
4~Volokolamskoye Highway,
Moscow 125993, Russian Federation;  \mbox{sergeyivanov89@mail.ru} 


\vspace*{3pt}

\noindent
\textbf{Kibzun Andrey I.} (b.\ 1951)~--- 
Doctor of Science in physics and mathematics, professor,  
Head of Department, Moscow Aviation Institute (National Research University), 
4~Volokolamskoye Highway,
Moscow 125993, Russian Federation;  \mbox{kibzun@mail.ru} 

\vspace*{3pt}

\noindent
\textbf{Naumov Andrey V.} (b.\ 1966)~--- 
Doctor of Science in physics and mathematics, associate professor,  
professor, Moscow Aviation Institute (National Research University), 
4~Volokolamskoye Highway,
Moscow 125993, Russian Federation;  \mbox{naumovav@mail.ru} 
\label{end\stat}


\renewcommand{\bibname}{\protect\rm Литература} 