\def\stat{bitugov}

\def\tit{ПРИМЕНЕНИЕ ВЕЙВЛЕТОВ ДЛЯ РАСЧЕТА ЛИНЕЙНЫХ СИСТЕМ УПРАВЛЕНИЯ  
С~СОСРЕДОТОЧЕННЫМИ ПАРАМЕТРАМИ$^*$}

\def\titkol{Применение вейвлетов для расчета линейных систем управления  
с~сосредоточенными параметрами}

\def\aut{Ю.\,И.~Битюков$^1$, Е.\,Н.~Платонов$^2$}


\def\autkol{Ю.\,И.~Битюков, Е.\,Н.~Платонов}

\titel{\tit}{\aut}{\autkol}{\titkol}

\index{Битюков Ю.\,И.}
\index{Платонов Е.\,Н.}
\index{Bityukov Yu.\,I.}
\index{Platonov E.\,N.}



{\renewcommand{\thefootnote}{\fnsymbol{footnote}} \footnotetext[1]
{Результаты работы получены в~рамках выполнения государственного 
задания Минобрнауки №\,2.2461.2017/ПЧ.}}


\renewcommand{\thefootnote}{\arabic{footnote}}
\footnotetext[1]{Московский авиационный институт (национальный исследовательский университет),  
\mbox{yib72@mail.ru}}
\footnotetext[2]{Московский авиационный институт (национальный исследовательский 
университет),  \mbox{en.platonov@gmail.com}}

%\vspace*{-18pt}

\Abst{Задачи многих дисциплин могут привести к~дифференциальным и~интегральным 
уравнениям. В~простых случаях такие уравнения могут быть решены аналитически, 
но в~более сложных приходится находить приближенные решения этих уравнений. 
В~последнее время большую популярность получили методы, основанные на использовании 
вейвлетов. Среди применяемых были вейвлеты Дебеши, койфлеты и~т.\,д. 
Недостаток таких вейвлетов состоит в~том, что у~них нет аналитического выражения. 
Поэтому возникают большие сложности при интегрировании и~дифференцировании выражений, 
содержащих эти вейвлеты. В~данной статье представлены алгоритмы численного 
решения линейных интегральных и~дифференциальных уравнений, основанные на 
сплайн-вейв\-ле\-тах на отрезке. Представленные алгоритмы обобщают 
известные методы, основанные на вейвлетах Хаара, которые являются частным 
случаем сплайн-вейв\-ле\-тов. Результаты статьи применяются для анализа 
линейных систем управ\-ле\-ния с~сосредоточенными параметрами.}

\KW{сплайн-вейвлет; дифференциальные уравнения; интегральные уравнения}

\DOI{10.14357/19922264170412} 


\vskip 10pt plus 9pt minus 6pt

\thispagestyle{headings}

\begin{multicols}{2}

\label{st\stat}

\section{Введение} 

Для численного решения линейных интегральных уравнений традиционно 
применяется метод, основанный на замене интегрального уравнения 
алгебраической системой линейных уравнений с~помощью применения 
квадратурной формулы. Мат\-ри\-ца такой системы имеет большой размер, и,~как следствие, 
для нахождения решения требуется большое число арифметических операций. 

В~\cite{Lepik3} было предложено использовать вейвлеты Хаара 
для приближенного решения интегрального уравнения, что приводило 
к~системе линейных уравнений с~разреженной матрицей. Получаемое приближенное 
решение было ку\-соч\-но-не\-пре\-рыв\-ным. 

В~\cite{Blatov}  показано, что, если использовать вместо вейвлетов Хаара 
сплайн-вейв\-ле\-ты на отрезке, матрица системы линейных уравнений получается 
псевдоразреженной, т.\,е.\ имеет очень много малых по модулю элементов. 

В~данной статье будут обобщены результаты 
работ~\cite{Lepik1, Lepik2, Lepik3, Lepik4, Lepik} и~развиты результаты 
работы~\cite{Blatov} для получения приближенных решений любого класса гладкости 
линейных интегральных и~дифференциальных уравнений. 
В~качестве примера рассмотрим анализ линейной системы управления  с~сосредоточенными 
параметрами.

\section{Сплайн-вейвлеты на~отрезке}

В этом разделе  кратко рассмотрим подход к~построению вейв\-лет-сис\-тем 
на отрезке, предложенный в~\cite{ArticleFinkelstein}. Пусть действительная 
функция~$\varphi $ принадлежит действительному пространству 
$\mathrm{L}^{2} \left({\bf R}\right)$, удовлетво\-ряет равенству
\begin{equation} 
\label{Pr1}
\varphi \left(x\right)\hm=\sqrt{2} \sum\limits_{k\in {\bf Z}}u_{k} 
\varphi \left(2x-k\right)\,,\enskip u_{k} \in \mathbf{R}\,,
\end{equation}
и имеет компактный носитель, содержащийся в~отрезке $[a;b]$. 
Обозначим $\varphi _{jk} (x)\hm=2^{{j}/{2}}\varphi \left(2^{j} x-k\right)$,
$x\hm\in [a;b]$. Функция~$\varphi $ в~теории вейвлетов называ\-ется масштабирующей,
 а~равенство~\eqref{Pr1}~--- масштаб\-ным соотношением~\cite{Frazer}. 
 Ясно, что отличными\linebreak от нуля на отрезке $[a;b]$ будет лишь конечное чис\-ло
  таких функций. Пусть для определенности это будут функции $\varphi _{j,0},
\varphi _{j,1} ,\ldots,\varphi _{j,n_{j} -1}$. 

Если рассмотреть линейные пространства $ V_{j} \hm= \mathrm{lin}
\left\{\varphi _{j,0} ,\varphi _{j,1} ,\ldots,\varphi _{j,n_{j} -1} \right\}$,
$\dim V_{j} \hm=n_{j}$, то  в~силу равенства~\eqref{Pr1}  будет выполняться 
$V_{0} \hm\subset V_{1} \subset \cdots$\linebreak $\cdots \subset L^{2} \left[a;b\right]$. 
Поэтому $\varphi _{j-1,k} \hm=\sum\nolimits _{s=0}^{n_{j} -1}p_{s,k} \varphi _{j,s}$. 

Как и~в~\cite{ArticleFinkelstein}, введем обозначения: 

\noindent
\begin{align*}
\Phi _{j} (x)&=
\left(\varphi _{j,0} (x),\varphi _{j,1} (x),\ldots,\varphi _{j,n_{j} -1} (x)\right)\,;\\
{P}_{j} &=\left(p_{s,k} \right)_{s=0, k=0}^{n_{j} -1, n_{j-1} -1}\,.
\end{align*}
Тогда $\Phi _{j-1}\hm =\Phi _{j} {P}_{j} $. 

\pagebreak

Обозначим символом~$W_{j-1} $ 
ортогональное дополнение к~пространству~$V_{j-1} $ в~пространстве~$V_{j}$. 
Поскольку $V_{j}\hm =V_{j-1} \oplus W_{j-1} $ и~$W_{j-1} \hm\subset V_{j} $, 
то~$W_{j-1} $~--- конечномерное пространство $ W_{j} \hm = 
\mathrm{lin} \left\{\psi _{j,0} ,\psi _{j,1} ,\ldots,\psi _{j,m_{j} -1} \right\}$,
$\dim W_{j} \hm=m_{j}$ и~$\psi _{j-1,k} \hm=
\sum\nolimits_{s=0}^{n_{j} -1}q_{s,k}^{j} \varphi _{j,s}.$ 
Функции~$\psi _{j,k} $ называются вейвлетами, а~пространства~$W_{j} $ 
называются вейв\-лет-про\-стран\-ст\-вами.

Снова введем в~рассмотрение матрицы~\cite{ArticleFinkelstein}:
\begin{align*}
\Psi _{j} (x)&=\left(\psi _{j,0} (x),\psi _{j,1} (x),\ldots,\psi_{j,m_{j} -1}
 (x)\right)\,;\\
{Q}_{j}&=\left(q_{s,k}^{j} \right)_{s=0, k=0}^{n_{j}-1,m_{j-1}-1}\,.
\end{align*}
Тогда $\Psi _{j-1} \hm=\Phi _{j} {Q}_{j} $. 
Следует заметить, что $n_{j} \hm+m_{j} \hm=n_{j+1} $.
Пусть $f\hm\in {L}^{2} (X)$ и~$\Pi _{j} : {L}^{2} (X)\hm\to V_{j} $. Тогда
\begin{multline*}
\Pi _{j} f=\sum\limits_{k=0}^{n_{j} -1}c_{jk} \varphi _{jk}  =
\Pi _{j-1} f+\Pi _{j-1}^{W} f={}\\
{}=\sum\limits_{k=0}^{n_{j-1} -1}c_{j-1,k} \varphi _{j-1,k}  +
\sum\limits_{k=0}^{m_{j-1} -1}d_{j-1,k} \psi _{j-1,k}\,.
\end{multline*}
Данное равенство можно переписать в~матричном виде, если ввести в~рассмотрение 
векторы ${C}_{j} \hm=\left(c_{j,0} ,\dots,c_{j,n_{j} -1} \right)^{\mathrm{T}}$,
${D}_{j} \hm=\left(d_{j,0} ,\dots,d_{j,m_{j} -1} \right)^{\mathrm{T}}$. 
Первый вектор описывает приближение функции~$f$, а~второй вектор представляет 
собой вейв\-лет-ко\-эф\-фи\-ци\-ен\-ты, которые характеризуют отклонение
$\Pi _{j-1} f$ от~$\Pi _{j} f$. Как показано в~\cite{ArticleFinkelstein}, 
имеет место равенство
${C}_{j} \hm={P}_{j} {C}_{j-1} \hm+{Q}_{j} {D}_{j-1}.$
По данному равенству можно восстановить приближение~$\Pi _{j} f$ 
по более грубому приближению~$\Pi _{j-1} f$ и~вейв\-лет-ко\-эф\-фи\-ци\-ен\-там.

Поскольку линейные операторы (проекторы) $V_{j} \hm\to V_{j-1}$,
$V_{j} \hm\to W_{j-1} $ определяются некоторыми матрицами~${A}_{j}$,
${B}_{j}$, то ${C}_{j-1} \hm={A}_{j} {C}_{j}$,
${D}_{j-1} \hm={B}_{j} {C}_{j}$.

Под вейвлет-преобра\-зо\-ва\-ни\-ем функции~$f$ будем понимать 
нахождение векторов ${C}_{0}$, ${D}_{0}$, ${D}_{1} ,
\dots, D_{j-1}$. Матрицы~${Q}_j$ и~${P}_j$ 
известны как фильт\-ры синтеза. Матрицы~${A}_j$ и~${B}_j$ 
известны как фильт\-ры анализа. Множество 
$\{{A}_j, {B}_j, {P}_j,{Q}_j\}$ 
называется банком фильтров.

Как показано в~\cite{ArticleFinkelstein}, между 
матрицами~${A}_{j}$, ${B}_{j}$ и~${P}_{j}$, 
${Q}_{j} $ существует следующая связь:
$$
\begin{pmatrix} {A}_{j} \\ {B}_{j} \end{pmatrix}=
\begin{pmatrix} {P}_{j} {Q}_{j}\end{pmatrix}^{-1}\,.
$$

Посмотрим теперь, как определить мат\-ри\-цу~${Q}_{j}$. 
Введем следующее обозначение. Если ${f}\hm=\left(f_{1} ,\dots, f_{r} \right)$,
${g}\hm=\left(g_{1} ,\dots,g_{r} \right)$~--- 
некоторые векторы, то $[({f},{g})]\hm=
\left(\left(f_{i} ,g_{j} \right)\right)_{i,j=1}^{r} $~--- 
мат\-ри\-ца скалярных произведений.  Как показано в~\cite{ArticleFinkelstein}, 
мат\-ри\-ца~${Q}_{j} $ удовлетворяет следующему уравнению: 
${P}_{j}^{\mathrm{T}} \left[\left(\Phi _{j}, \Phi _{j} 
\right)\right]{Q}_{j} \hm=0.$

Перейдем теперь к~сплайн-вейв\-ле\-там на отрезке. Определим В-сплай\-ны порядка~$n$  
как свертку~\cite{Chui}:
$$
N_{n} =N_{n-1} *N_{0}\,,\quad
N_{0} (x)=\begin{cases} 
1\,, & x\in [0;1)\,; \\ 
0\,, & x\notin [0;1)\,.
\end{cases}
$$
Как показано в~\cite{Chui}, если определить функцию $\varphi (x)\hm=N_{n} (x)$, 
то она удовлетворяет равенству $\varphi (x)\hm=\sum\nolimits _{k=0}^{n+1}
({C_{n+1}^{k} }/{2^{n}}) \varphi (2x\hm-k)$, где $C_{n+1}^k\hm={(n+1)!}/({k!(n\hm+1\hm-k)!})$.
В~\cite{Yurgu} пред\-став\-лен банк фильт\-ров, соответствующий функции  
$\varphi (x)\hm=N_{n} (x)$, а~именно:  справедливы следующие результаты.

\smallskip

\noindent
\textbf{Лемма~2.1.}\
\textit{Функция $\varphi(x)\hm=N_n(x)$ определяет последовательность подпространств}
\begin{multline*}
V_{\alpha,0}\subset V_{\alpha,1}\subset\cdots,\\
V_{\alpha,j}=\mathrm{lin}\left\{\varphi_{j,-n},\varphi_{j,-n+1},\dots,
\varphi_{j,2^j\alpha(n+1)-1}\right\}
\end{multline*}
\textit{пространства} ${L}^2[0;\alpha(n+1)]$, $\alpha\hm=1,2,\ldots$, 
\textit{такую, что} 
$\overline{\bigcup\nolimits_{j=0}^{+\infty}V_{\alpha,j}}\hm={L}^2
[0;\alpha(n+1)]$.

\smallskip

\noindent
\textbf{Лемма~2.2.}\
\textit{Имеет место равенство} 
$\sum\nolimits_{k=-n}^{2^j\alpha (n+1)-1} \varphi_{j,k}(x) \hm\equiv 
2^{{j}/{2}}$, $x\hm\in [0;\alpha (n+1)].
$
\textit{Если}  $V_{\alpha ,j} \hm=V_{\alpha ,j-1} \oplus W_{\alpha ,j-1} $, 
\textit{то} $\dim W_{\alpha ,j-1}\hm =2^{j-1} \alpha (n+1)$.

\smallskip

Пусть $\lambda_{m,k}\hm=\int\nolimits_k^{k+1} N_n(z)N_n(z-m)\,dz$, 
$m\hm=-n,\ldots ,n$, $k\hm=0,1,\ldots ,n$, и~$\omega_{i,k}\hm=\omega_{k,i}
\hm=\sum\nolimits_{s=n-i+1}^n \lambda_{k-i,s}$, 
$\theta_{i,k}\hm=\theta_{k,i}\hm=\sum\nolimits_{s=0}^{n-k} \lambda_{i-k,s}$, 
$1\hm\leqslant i \hm\leqslant k \hm\leqslant n.$ Введем в~рассмотрение 
вектор\linebreak ${p}\hm\in \textbf{R}^{2^j\alpha (n\hm+1)\hm+n}$, 
который определим равенством:
\begin{multline*}
%\label{vecp}
{p}={}\\
{}=\begin{cases}
 \begin{pmatrix} 
 C_{n+1}^{0} \cdots  C_{n+1}^{k}&C_{n+1}^{k} \cdots  C_{n+1}^{0}&0 \cdots 0
 \end{pmatrix}^{\mathrm{T}},
 &\\
 & \hspace*{-30mm}\mbox{ если } n=2k\,;\\
\begin{pmatrix} 
C_{n+1}^{0} \cdots C_{n+1}^{k}&C_{n+1}^{k+1}&C_{n+1}^{k} \cdots  C_{n+1}^{0}&0 \cdots 0
\end{pmatrix}^{\mathrm{T}}\!,\hspace*{-10.94377pt}
&\\
&\hspace*{-30mm} \mbox{ если } n=2k+1\,.
\end{cases}
\end{multline*}
Определим оператор сдвига~$R_s$:  $\textbf{R}^m \hm\rightarrow  \textbf{R}^m$ 
сле\-ду\-ющим правилом:
\begin{multline*}
\!\!\!\!\!R_s {a}=\!\begin{cases} 
\begin{pmatrix} 
\underbrace{0 \cdots 0}_s & a_1 \cdots a_{m-s}\end{pmatrix}^{\mathrm{T}}\!, &\!\!\!\! \mbox{ если } 
0\leqslant s < m\,;\\
\begin{pmatrix} 
a_{|s|+1} \cdots  a_m&0 \cdots 0\end{pmatrix}^{\mathrm{T}}\!, &\!\!\!\! \mbox{ если }
 -m<s<0\,;\hspace*{-10.8pt}\\
 0\,, &\!\!\!\! \mbox{ если } \vert s\vert \geqslant m\,,
 \end{cases}
 \end{multline*}
 где
 $$
{a}=\left(a_1,\dots,a_m\right)^{\mathrm{T}}\,.
$$


%\smallskip

\noindent
\textbf{Лемма 2.3.}\ 
\textit{Матрицы ${P}_j$ и~$[(\Phi_j,\Phi_j)]$ 
для последовательности подпространств  $V_{\alpha,0}\hm\subset V_{\alpha,1}
\subset\cdots$ имеют вид}:

\pagebreak

\noindent
\begin{equation*}
{P}_j=\fr{1}{2^{n+{1}/{2}}}
\begin{pmatrix} 
R_{-n}{p}&R_{-n+2}{p}&\cdots&R_{n-2+2^j\alpha (n+1)}{p}
\end{pmatrix};
\end{equation*}

\vspace*{-12pt}

\noindent
\begin{multline*}
[(\Phi_j,\Phi_j)]=\left( 
{d}_1\  \cdots\  {d}_n\ {q}\ 
R_1{q}\ \cdots\right.\\
\left.\cdots  \ R_{2^j\alpha (n+1)-n-1}
{q}\ {u}_1\ \cdots\ {u}_n\right)^{\mathrm{T}}\,,
\end{multline*}
\textit{где }
\begin{align*}
&{d}_s=
\begin{pmatrix} \omega_{1,s}&\omega_{2,s}&\cdots&\omega_{n,s}&q_{n-s+1}&\cdots&
q_n&0 \cdots 0\end{pmatrix}^{\mathrm{T}}\,;
\\
&{u}_s=
\begin{pmatrix}0 \cdots 0&q_n&\cdots&q_s&\theta_{1,s}&\cdots&\theta_{n,s}\end{pmatrix}^{\mathrm{T}}\,;\\
&{q}=
\begin{pmatrix} q_n & q_{n-1}  \cdots  q_1 & q_0 & q_1  \cdots  q_{n-1} & q_n &
 0 \cdots 0 \end{pmatrix}^{\mathrm{T}}   \in{}\\
& \hspace*{8mm}{}\in  \mathbf{R}^{2^j\alpha (n+1)+n}\,,
 \enskip q_{k} =\left(N_{n}(\cdot),N_{n} (\cdot -k)\right).
\end{align*}
\textit{Матрица, транспонированная к~${T}_j\hm={P}_j^{\mathrm{T}}
 [(\Phi_j,\Phi_j)]\hm=2^{-n-{1}/{2}}
 (t_{i,s})_{i=1,s=1}^{2^{j-1}\alpha (n+1)+n,~2^j\alpha (n+1)+n}$,
имеет вид}:
\begin{multline*}
{T}_j^{\mathrm{T}}=\fr{1}{2^{n+{1}/{2}}}
\left( 
{L}_1 \cdots {L}_n\  \  {w}\ \  R_2 {w} \cdots \right.\\
\cdots
R_{2^j\alpha (n+1)-2n-2} {w}\ \  {L}_{2^{j-1}\alpha (n+1)+1} \cdots \\
\left.\cdots {L}_{2^{j-1}\alpha (n+1)+n}\right)\,,
\end{multline*}
\textit{где} 
\begin{multline*}
\hspace*{-2pt}{w}=\begin{pmatrix}
{p}^{\mathrm{T}} R_{-2n}{q}&{p}^{\mathrm{T}} R_{-2n+1}{q} 
\cdots {p}^{\mathrm{T}} R_{n+1}{q}&0 \cdots 0 
\end{pmatrix}^{\mathrm{T}}
 \in{}\\
 {}\in \mathbf{R}^{2^j\alpha (n+1)+n}\,;
\end{multline*}

\vspace*{-12pt}

\noindent
\begin{multline*}
{L}_i={}\\
{}=\begin{pmatrix} 
\left(R_{-n+2i-2}{p}\right)^{\mathrm{T}} {d}_1 \cdots 
\left(R_{-n+2i-2}{p}\right)^{\mathrm{T}} {d}_n&0 \cdots 0
\end{pmatrix}^{\mathrm{T}}+ {}\\
{}+
\left(R_n\circ R_{-3n+2i-2}\right){w},\enskip
i=1,\dots,n\,;
\end{multline*}

\vspace*{-12pt}

\noindent
\begin{multline*}
\hspace*{-8.5727pt}{L}_{i+1}=\begin{pmatrix} 
0 \cdots 0&\left(R_{-n+2i}{p}\right)^{\mathrm{T}} {u}_1 
\cdots \left(R_{-n+2i}{p}\right)^{\mathrm{T}}{u}_n
\end{pmatrix}^{\mathrm{T}}\!  
+{}\\
{}+\left(R_{-n}\circ R_{-n+2i}\right){w}\,,
\end{multline*}

\vspace*{-12pt}

\noindent
$$
\hspace*{8mm}i=2^{j-1}\alpha (n+1),\dots,n-1+2^{j-1}\alpha (n+1)\,.
$$

С использованием леммы~2.3  в~\cite{Yurgu} найдены $2^{j-1}\alpha (n+1)$ 
линейно независимых решений ${h}_s\hm=
(h_{1,s},h_{2,s},\dots,h_{2^j\alpha (n+1)+n,s})^{\mathrm{T}}$ 
системы линейных уравнений  ${T}_j {h}_s\hm=0$. 
Эти решения и~представляют собой столбцы матрицы ${Q}_j\hm=
({h}_1,\dots,{h}_{2^{j-1}\alpha (n+1)}).$
Столбцы ${h}_{s} $ выбирались таким образом, чтобы функции
$$
\psi _{j-1,s} (x)=\Phi_j(x){h}_s=
\sum\limits_{i=1}^{2^{j} \alpha (n+1)+n}h_{i,s}  \varphi _{j,-n+(i-1)}(x)
$$
по возможности представляли собой сдвинутые версии одной функции, т.\,е.\ 
имели бы одну форму (за исключением, конечно, граничных вейвлетов).  
Введем сокращенные обозначения для матриц, составленных из элементов
 матрицы~${T}_j$:
$$
T_j\left(\begin{smallmatrix} 
i_1,\dots,i_k \\ j_1,\dots,j_m \end{smallmatrix} 
\right) = 
\begin{pmatrix} t_{i_1,j_1} & \cdots & t_{i_1,j_m} \\ 
\vdots & \vdots&\vdots \\ 
t_{i_k,j_1} & \cdots & t_{i_k,j_m} 
\end{pmatrix}.
$$
Для внутренних вейвлетов (носитель содержится в~отрезке $[0;\alpha (n+1)]$):
\begin{multline*}
{h}_s=(0,\dots,0,h_{2s-n-1,s},\dots, h_{2s+2n,s},0,\dots,0)^{\mathrm{T}},
\\
 s=n+1,\dots,2^{j-1}\alpha (n+1)-n\,,
\end{multline*}
 где $ T_j\left(\begin{smallmatrix} s-n,\dots,s+2n \\ 
 2s-n-1,\dots,2n+2s \end{smallmatrix} \right)(h_{2s-n-1,s},\dots,h_{2s+2n,s})^{\mathrm{T}}\hm=0.$

Решения, соответствующие граничным вейвлетам, выбираются следующим образом. 
Для $s\hm=1,2,\dots,n$ положим
$$
{h}_{s}=(0,\dots,0,h_{s,s},\dots,h_{2n+2s,s},0,\dots,0)^{\mathrm{T}},
$$
где $T_j\left(\begin{smallmatrix} 1,\dots,s+2n \\ 
s,\dots,2s+2n\end{smallmatrix} \right)(h_{s,s},\dots,h_{2s+2n,s})^{\mathrm{T}}\hm=0.$
Для $s\hm=2^{j-1}\alpha (n+1)\hm-n+1,\dots ,2^{j-1}\alpha (n+1)$ положим 
\begin{multline*}
{h}_{s}=\left(
0,\dots,0,h_{2s-n-1,s},\dots\right.\\
\left.\dots,h_{2^{j-1}\alpha (n+1)+n+s,s},0,
\dots,0\right)^{\mathrm{T}},
\end{multline*}
где
\begin{multline*}
T_j\left(\begin{smallmatrix} 
s-n,\dots,n+2^{j-1}\alpha (n+1) \\ 2s-n-1,\dots,2^{j-1}\alpha (n+1)+n+s 
\end{smallmatrix} \right)={}\\
{}=\left(h_{2s-n-1,s},\dots,h_{2^{j-1}\alpha (n+1)+n+s,s}\right)^{\mathrm{T}}.
\end{multline*}

Кратко рассмотрим применение вейв\-лет-сис\-тем на отрезке к~построению 
двумерных вейвлетов на прямоугольной области. Пусть даны последовательности $V_{0,i} 
\hm\subset V_{1,i} \subset \cdots \subset V_{j,i} \subset \cdots$ 
конечномерных подпространств ${L}^{2} [a_{i} ;b_{i} ]$, 
масштаби\-ру\-ющие функции~$\varphi ^{(i)} $ и~банки фильтров 
${P}_{j,i}$, ${Q}_{j,i}$, ${A}_{j,i}$, ${B}_{j,i}$, 
$ i\hm=1,2$. Стандартный подход к~построению многомерных вейв\-лет-сис\-тем~--- 
это взятие тензорных произведений функций из одномерных базисов~\cite{Novikov}. 
Определим подпространства $V_{j}^{2} \hm=V_{j,1}\;\otimes$\linebreak
$\otimes\;V_{j,2} \hm= \mathrm{lin}
\left\{f_{1} \otimes f_{2} : f_{1} \hm\in V_{j,1},\ f_{2} \hm\in V_{j,2} \right\}$, 
где функция $f_{1} \otimes f_{2} $ определяется правилом 
$f_{1} \hm\otimes f_{2} \left(x,y\right)\hm=f_{1} (x)f_{2} (y)$. 
Ясно, что функции $\varphi _{j,k}^{(1)} \otimes \varphi _{j,s}^{(2)} $ 
образуют базис в~пространстве~$V_{j}^{2} $.  Вейв\-лет-про\-стран\-ст\-ва~$W_{j}^{2} $ 
определяются следующим образом: 
$$
V_{j}^{2} =V_{j-1}^{2} \oplus W_{j-1}^{2} \,.
$$

Следующие две леммы очевидны.

\begin{figure*}[b] %fig1
\vspace*{1pt}
 \begin{center}
 \mbox{%
 \epsfxsize=162.046mm 
 \epsfbox{bit-1.eps}
 }
 \end{center}
\vspace*{-9pt}
\Caption{Графики функций $w_l$ для $n\hm=5$}
\end{figure*}


\noindent
\textbf{Лемма 2.4.}\ 
\textit{Пусть $f\hm\in {L}^{2}[0;n+1]$, тогда 
$\Pi _{j} f=\Phi _{j} {C}_{j}^{*} $, где
 ${C}_{j}^{*}\hm =[(\Phi _{j} ,\Phi _{j})]^{-1}[(f,\Phi _{j})]$. 
 При этом} 
 $$
 \| f-\Pi _{j} f\| _{{L}^{2} }^{2} =
 \| f\| _{{L}^{2} }^{2} -\left[(f,\Phi _{j})\right]^{\mathrm{T}}
 \left[(\Phi _{j} ,\Phi _{j})\right]\left[(f,\Phi _{j})\right].
 $$

\smallskip

\noindent
\textbf{Лемма 2.5.}\ 
\textit{Пусть $f\in {L}^2([a_1;b_1]\times [a_2;b_2])$ 
и~$\Pi_j^{(2)} : {L}^2([a_1;b_1]\times [a_2;b_2])\hm\to V_j^2$~--- 
проектор. Если}
\begin{multline*}
%\label{G35}
{G}_j={}\\
{}=\left(\int\limits_{\,\,\,a_1}^{b_1}\,dx\!
\int\limits_{a_2}^{b_2}\!\varphi_{j,s}^{(1)}(x)\varphi_{j,k}^{(2)}(y)f(x,y)\,dy
\right)_{s,k=0}^{n_{j,1}-1,n_{j,2}-1},\hspace*{-5.4785pt}
\end{multline*}
\textit{то $\Pi_j^{(2)}f (x,y)\hm=\Phi_j^{(1)}(x)\mathrm{C}_j (\Phi_j^{(2)}
(y))^{\mathrm{T}}$, где $\Phi_j^{(i)}\hm=(\varphi_{j,0}^{(i)}\cdots \varphi_{j,n_{j,i}-1}^{(i)})$, 
а~матрица~${C}_j$ определяется равенством}:
\begin{equation*}
%\label{Cj35}
{C}_j=\left[\left(\Phi_j^{(1)},\Phi_j^{(1)}\right)\right]^{-1}
{G}_j\left[\left(\Phi_j^{(2)},\Phi_j^{(2)}\right)\right]^{-1}\,.
\end{equation*}

%\vspace*{-24pt}

\section{Интегралы от~сплайн-вейвлетов}

Пусть ${Q}_j=({h}_1^j,\dots,{h}_{2^{j-1}(n+1)}^j)$, 
где ${h}_s^j\hm=(h_{1,s}^j,h_{2,s}^j,\dots,h_{2^j (n+1)+n,s}^j)^{\mathrm{T}}$. 
Тогда,   согласно результатам предыдущего раздела,

\noindent
\begin{multline}
\label{U31}
\psi_{j-1,s}(x)={}\\
{}=
\begin{cases}
\displaystyle\sum\limits_{i=s}^{2s+2n} \! h_{i,s}^j\varphi_{j,-n+i-1}(x)\,,&\hspace*{-10mm}
s=1,\dots,n\,;\\
\displaystyle\sum\limits_{i=2s-n-1}^{2s+2n}\!\!
h_{i,s}^j\varphi_{j,-n+i-1}(x)\,,&\\ 
&\hspace*{-38mm}s=n+1,\dots,2^{j-1}(n+1)-n\,;
\\
%\label{U33}
\displaystyle\sum\limits_{i=2s-n-1}^{2^{j-1}(n+1)+n+s}\!\!\!\!
h_{i,s}^j
\varphi_{j,-n+i-1}(x)\,,&\\
&\hspace*{-50mm}s=2^{j-1}(n+1)-n+1,\dots,2^{j-1}(n+1)\,.
\end{cases}
\end{multline}


Так же, как и~в~работах~\cite{Lepik1, Lepik2, Lepik3, Lepik4, Lepik}, для удобства  
введем следующие обозначения:
\begin{align*}
&w_l(x)=\varphi_{0,l-n-1}\,,\enskip l=1,2,\dots,2n+1\,,\\
&w_l(x)=\psi_{j,s}(x)\,,\enskip l=2^j(n+1)+n+s\,, \\
&\hspace*{23mm} j=0,1,\dots\,, \ s=1,\dots,2^{j}(n+1)\,.
\end{align*}
На рис.~1 представлены графики некоторых функций~$w_l$ для случая $n\hm=5$.




Пусть $J\geqslant 0$,  $\Pi_{J} : {L}^2[0;n+1]\hm\to V_{J}$~--- 
проектор и~$M\hm=2^{J}(n+1)+n$. Обозначим ${H}_J\hm=
\begin{pmatrix}w_1 & \cdots & w_M
\end{pmatrix}$ и~введем в~рассмотрение матрицу скалярных 
произведений $[({H}_J, {H}_J)]$. 
В~лемме~2.3 представлены матрицы скалярных произведений 
$[(\Phi_k, \Phi_k)]$ для всех $k\hm=0,1,\dots$ Замечая, что $\Psi_k \hm= 
\Phi_{k+1}{Q}_{k+1}$ и~$[(\Psi_k,\Psi_k)]\hm={Q}_{k+1}^{\mathrm{T}}
[( \Phi_{k+1}, \Phi_{k+1})]{Q}_{k+1}$,
получаем матрицу:
{\small \begin{multline*}
\left[({H}_J, {H}_J)\right]={}\\
\!{}=\!
\begin{pmatrix}
\left[(\Phi_0, \Phi_0)\right] & 0 & 0 & \cdots & 0\\
0 & {Q}_{1}^{\mathrm{T}}\left[( \Phi_{1}, \Phi_{1})\right]{Q}_{1} & 0 & \cdots & 0 \\
\vdots & \ddots & \ddots & \ddots & \vdots \\
0 & 0 & 0 & \cdots & {Q}_{J}^{\mathrm{T}}\left[( \Phi_{J}, \Phi_{J})\right]{Q}_{J}
\end{pmatrix}\!.\hspace*{-12.6139pt}
\end{multline*}
}

\noindent
Так как $V_{J}\hm=V_0\oplus W_0\oplus V_1\oplus\dots\oplus W_{J-1},$ то для 
$f\hm\in {L}^2[0;n+1]$ имеем
$\Pi_{J} f\hm=\sum\nolimits_{l=1}^{M} c_l w_l \hm= {H}_{J}{C}_{J}$,
где ${C}_J\hm=
\begin{pmatrix} c_1 & \cdots\ c_{M} \end{pmatrix}^{\mathrm{T}}$.
Как и~в~работах~\cite{Lepik1, Lepik2, Lepik3, Lepik4, Lepik}, определим функции:
\begin{equation}
\label{U38}
{\xi}_{1,l}(x)=\int\limits_0^x w_l (t)\,dt\,;
\end{equation}

%\vspace*{-12pt}

\noindent
\begin{multline}
\label{U38-1}
{\xi}_{\nu+1,l}(x)=\int\limits_0^x {\xi}_{\nu,l}(t)\,dt={}\\
{}=
\fr{1}{\nu !}\int\limits_0^x (x-t)^{\nu} w_l (t)\,dt\,,
\enskip \nu = 1,2,\ldots
\end{multline}
Согласно определению функций~$w_l$ и~равенст\-вам~(\ref{U31})  
функция ${\xi}_{\nu+1,l}(x)$ представляет собой линейную комбинацию функций
$$
\eta_{n,\nu}^{j,s}(x)=\int\limits_0^x (x-t)^{\nu} N_{n}(2^j t-s)\,dt\,.
$$

\vspace*{-1pt}

\noindent
\textbf{Лемма 3.1.}\ 
\textit{Имеет место следующее рекуррентное соотношение}:

\noindent
\begin{multline}
\label{U40}
\eta_{n,\nu}^{j,s}(x)=\fr{x^{\nu+1}}{\nu+1}\, N_n(-s)+{}\\
{}+
\fr{2^j}{\nu+1}\left(\eta_{n-1,\nu+1}^{j,s}(x)-\eta_{n-1,\nu+1}^{j,s+1}(x)\right)\,,
\end{multline}
\textit{где}

\noindent
\begin{multline}
\label{U41}
\eta_{0,\nu}^{j,s}(x) = {}\\
\!\!\!{}=
\begin{cases}
\fr{(x-a)^{\nu+1} - (x-b)^{\nu+1}}{\nu+1}, &\\[6pt]
& \hspace*{-37mm}\mbox{если } [a;b]=[0;x]
\cap\left[\fr{s}{2^j};\fr{s+1}{2^j}\right]\not= \varnothing\,;\\[9pt] 
0, & \hspace*{-37mm}\mbox{если } [a;b]=[0;x]\cap\left[\fr{s}{2^j};\fr{s+1}{2^j}\right]= \varnothing\,.
\end{cases}
\end{multline}


\noindent
Д\,о\,к\,а\,з\,а\,т\,е\,л\,ь\,с\,т\,в\,о\,.\ \
По свойству В-сплай\-нов~\cite{Chui}  имеет место равенство:
\begin{equation*}
%\label{BSPL}
N_n'(x)=N_{n-1}(x)-N_{n-1}(x-1)\,.
\end{equation*}
Следовательно, по формуле интегрирования по час\-тям  получаем:

\noindent
\begin{multline*}
\eta_{0,\nu}^{j,s}(x)=\left.- \fr{(x-t)^{\nu+1}}{\nu+1} 
N_n(2^j t-s)\right|_0^x+{}\\
{}+2^j \int\limits_0^x \fr{(x-t)^{\nu+1}}{\nu+1} \left(N_{n-1}(2^jt-s)-{}\right.\\
\left.{}-
N_{n-1}(2^jt-s-1)\right)\,dt=
\fr{x^{\nu+1}}{\nu+1}\, N_n(-s)+{}\\
{}+\fr{2^j}{\nu+1}\left(\eta_{n-1,\nu+1}^{j,s}(x)-
\eta_{n-1,\nu+1}^{j,s+1}(x)\right).
\end{multline*}
Равенство~(\ref{U41}) очевидно.~\hfill$\square$

\vspace*{2pt}

Формулы (\ref{U40}) и~(\ref{U41}) позволяют находить значение 
функции $\eta_{n,\nu}^{j,s}(x)$ в~любой точке без интегрирования. Итак, 
для $l\hm=1,2,\dots, 2n+1$ получаем:

\columnbreak

\noindent
$$
{\xi}_{\nu+1,l}(x)=\fr{1}{\nu !}\, \eta_{n,\nu}^{0,l-n-1}(x)\,,\enskip
l=1,2,\dots, 2n+1\,.
$$
Для $l=2^j(n+1)\hm+n\hm+s$, $j\hm=0,1,\dots$, $s\hm=1,\ldots$\linebreak$\ldots,2^{j}(n\hm+1)$ получаем:
\begin{multline*}
{\xi}_{\nu+1,l}(x)={}\\
{}=
\begin{cases}
\fr{2^{({j+1})/{2}}}{\nu !}\sum\limits_{i=s}^{2s+2n}
h_{i,s}^{j+1}\eta_{n,\nu}^{j+1,-n+i-1}(x)\,, &\\
& \hspace*{-17mm}s=1,\ldots,n; \\
\fr{2^{({j+1})/{2}}}{\nu !}\sum\limits_{i=2s-n-1}^{2s+2n}
\!\! h_{i,s}^{j+1}\eta_{n,\nu}^{j+1,-n+i-1}(x)\,, &\\
& \hspace*{-40mm}s=n+1,\ldots,2^{j}(n+1)-n;\\
\fr{2^{({j+1})/{2}}}{\nu !}\sum\limits_{i=2s-n-1}^{2^{j}(n+1)+n+s}\!\!\!\!
h_{i,s}^{j+1}\eta_{n,\nu}^{j+1,-n+i-1}(x)\,, &\\
& \hspace*{-54mm}s=2^{j}(n+1)-n+1,\ldots,2^{j}(n+1).
\end{cases}
\end{multline*}
Полученные равенства справедливы при всех $\nu \hm= 0,1,\dots$

\section{Применение сплайн-вейвлетов к~решению 
линейных интегральных и~дифференциальных уравнений}

В проекционных методах решения линейных уравнений рассматриваются два 
уравнения~\cite{Akilov}: 
первое~--- в~полном нормированном пространстве~$X$:
\begin{equation}
\label{Ur1}
Kx\equiv x-\lambda Hx=f\,;
\end{equation}
второе~--- в~его полном подпространстве~$V_j$:
\begin{equation}
\label{Ur2}
K_j x_j\equiv x_j-\lambda H_j x_j=\Pi_j f\,,
\end{equation}
где $H$~--- непрерывный линейный оператор в~$X$; $H_j$~--- 
непрерывный линейный оператор в~$V_j$. Уравнение~(\ref{Ur1}) называется точным, 
а~уравнение~(\ref{Ur2})~--- приближенным. При этом предполагается, что 
выполнены следующие условия.

\noindent \textbf{1. Условие близости операторов $H$ и~$H_j$.} 
Для любого $x_j\hm\in V_j$ выполняется $\|\Pi_j H x_j \hm- H_j x_j\|
\hm\leqslant \rho_j \|x_j\|$.

\noindent \textbf{2. Условие хорошей аппроксимации элементов 
вида~$Hx$ элементами из~$V_j$.} Для любого $x\hm\in X$ существует $x_j\hm\in V_j$ 
такой, что $\|Hx-x_j\|\hm\leqslant \rho_{1,j} \|x\|$.

\noindent \textbf{3. Условие хорошей аппроксимации свободного члена 
точного уравнения.} Существует элемент $f_j \hm\in V_j$ такой, что $\|f-f_j\|
\hm\leqslant \rho_{2,j}\|f\|$. В~отличие от предыду\-щих условий $\rho_{2,j}$ 
здесь, вообще говоря, зависит от~$f$.

Как показано в~\cite{Akilov}, если оператор~$K$ 
имеет непрерывный обратный, уравнение~(\ref{Ur1}) имеет решение 
и~$\lim\limits_{j\to +\infty} \rho_j \hm= 0$,  $\lim\limits_{j\to +\infty} \rho_{1,j} \hm= 
0$,  $\lim\limits_{j\to +\infty} \rho_{2,j} \hm= 0$, то  
$\lim\limits_{j\to +\infty} \|x-x_j\|\hm = 0$, где $x_j$~--- 
решение уравнения~(\ref{Ur2}).

Рассмотрим сначала линейное интегральное уравнение Фредгольма 2-го рода. 
С~помощью замены переменной такое уравнение можно свести к~следующему:
\begin{equation*}
%\label{U313}
u(x)-\lambda\int\limits_{0}^{n+1}U(x,t)u(t)\,dt =f(x)\,,\enskip x,t\in [0;n+1]\,.
\end{equation*}

Пусть $\varphi(x)=N_n(x)$, $V_{0} \subset V_{1} \subset\cdots$~--- 
соответствующая последовательность конечномерных подпространств пространства
 ${L}^{2} \left[0;n+1\right]$. Пусть $X\hm={L}^2[0;n+1]$. 
 Операторы $K:X\hm\to X$, $H:X\hm\to X$ и~$H_j : V_j\hm\to V_j$ определим равенствами:
\begin{align*}
Ku(\cdot)&=u(\cdot)-\lambda\int\limits_0^{n+1} U(\cdot,t)u(t)\,dt\,;\\
Hu(\cdot)&=\int\limits_0^{n+1} U(\cdot,t)u(t)\,dt\,;\\
H_j&=\Pi_j\circ H\,,
\end{align*}
где $U\in {L}^2([0;n+1]^2)$. Условие близости операторов~$H$ и~$H_j$ 
выполняется с~$\rho_j\hm=0$. Пусть  $u\hm\in X$ и~$u_j(\cdot)\hm=\int\nolimits_0^{n+1} 
\Pi^{(2)}U(\cdot,t)u(t)\,dt\hm\in V_j$. То\-гда $\rho_{1,j}\hm=
\|U-\Pi_j^{(2)}U\|_{{L}^2([0;n+1]^2)}$ 
и~$\lim\nolimits_{j\to +\infty} \rho_{1,j}\hm=0$. Следовательно, 
условие хорошей аппроксимации элементов вида~$Hu$ элементами из~$V_j$ 
также выполняется. Наконец, для произвольного $f\hm\in X$, $f\hm\ne 0$, 
возьмем $f_j\hm=\Pi_j f$, 
а~$\rho_{2,j}\hm={\|f-\Pi_j f\|_{{L}^2([0;n+1])}}/{\|f\|_{{L}^2([0;n+1])}}$. Тогда $\lim\limits_{j\to +\infty} \rho_{2,j}=0$. 

Решение приближенного уравнения
\begin{equation}
\label{Ur3}
u_J-\lambda\Pi_j\circ H u_J=\Pi_J f
\end{equation}
будем искать в~виде $u_J\hm=\sum\nolimits_{l=1}^{M} c_l w_l 
\hm= {H}_J{C}_J $, где $M\hm=2^{J}(n+1)+n$. 
Тогда уравнение~(\ref{Ur3}) можно переписать в~виде
системы линейных уравнений для определения коэффициентов~$c_l$:
\begin{multline*}
%\label{U315}
\sum\limits_{l=1}^{M} c_l (w_l,w_s)-{}\\
{}-\lambda
\sum\limits_{l=1}^{M} c_l \int\limits_{0}^{n+1}dx
\int\limits_{0}^{n+1}U(x,t)w_l(t)w_s(x)\,dt = (f,w_s)\,,\\
s=1,2,\dots,M\,.
\end{multline*}
Это и~есть система метода Галеркина. Перепишем ее в~матричном виде:
\begin{equation}
\label{U315}
{C}_J\left([({H}_J,{H}_J)]-
\lambda{G}_J\right)={F}_J\,,
\end{equation}
где 
\begin{align*}
{C}_J&=\begin{pmatrix}
c_1&\cdots &c_M\end{pmatrix}\,;\\ 
{F}_J&=\begin{pmatrix}(f,w_1)&\cdots &(f,w_M)\end{pmatrix}^{\mathrm{T}}\,;\\
{G}_J&=\left(g_{l,s}\right)_{l,s=1}^M\,,
\\ 
&\hspace*{10mm}g_{l,s}= \int\limits_{0}^{n+1}dx\int\limits_{0}^{n+1}U(x,t)w_l(t)w_s(x)\,dt.
\end{align*}

Аналогично рассматривается линейное интегральное уравнение Вольтерра 2-го рода
$$
u(x)-\lambda\int\limits_{0}^{x}U(x,t)u(t)\,dt =f(x)\,,\enskip 
x,t\in [0;n+1]\,.
$$
Операторы $K:X\hm\to X$, $H:X\hm\to X$ и~$H_j : V_j\hm\to V_j$ определим равенствами:
\begin{align*}
Ku(x)&=u(x)-\lambda\int\limits_0^{x} U(x,t)u(t)\,dt\,;\\
Hu(x)&=\int\limits_0^{x} U(x,t)u(t)\,dt\,;\\
H_j&=\Pi_j\circ H\,.
\end{align*}
Величины $\rho_j$, $\rho_{1,j}$ и~$\rho_{2,j}$ остаются теми же, что и~для 
уравнения Фредгольма.  Таким образом,
система Галеркина для данного уравнения имеет вид:
\begin{multline}
\label{U316}
\sum\limits_{l=1}^{M} c_l (w_l,w_s)-{}\\
{}-
\lambda\sum\limits_{l=1}^{M} 
c_l \int\limits_{0}^{n+1}dx\int\limits_{0}^{x}U(x,t)w_l(t)w_s(x)\,dt ={}\\
{}=(f,w_s)\,,\enskip
s=1,2,\dots,M\,.
\end{multline}
Матричный вид системы~(\ref{U316}) совпадает с~(\ref{U315}), где
$$
{G}_J=(g_{s,l})_{s,l=1}^M,\ \ g_{s,l}= 
\int\limits_{0}^{n+1}\!dx\int\limits_{0}^{x}\!U(x,t)w_l(t)w_s(x)\,dt.
$$

Рассмотрим теперь линейное дифференциальное уравнение
\begin{equation}\label{DU1}
y^{(k)}+a_1(x)y^{(k-1)}+\dots +a_k(x)y=f(x)
\end{equation}
с непрерывными коэффициентами $a_i(x)$, $i\hm=1,2,\dots,k$ и~начальными условиями
\begin{equation}
\label{NDU1}
y(0)=y_0\,,\enskip y'(0)=y_1,\ \dots,\ y^{(k-1)}=y_{k-1}\,.
\end{equation}
Если обозначить $y^{(k)}(x)=u(x)$, то задача~(\ref{DU1})--(\ref{NDU1}) сводится 
к~интегральному уравнению Вольтерра \mbox{2-го} рода. Следовательно, 
приближенное решение задачи~(\ref{DU1})--(\ref{NDU1}) можно искать в~виде:
\begin{multline*}
y_J(x)=\sum\limits_{s=1}^M c_s {\xi}_{k,s}(x) +y_{k-1}\fr{x^{k-1}}{(k-1)!}+{}\\
{}+y_{k-2}\fr{x^{k-2}}{(k-2)!}+\dots+y_0\,,
\end{multline*}
где функции ${\xi}_{k,s}(x)$ определены равенствами~(\ref{U38})
и~(\ref{U38-1}), а~коэффициенты~$c_s$ 
определяются из системы линейных уравнений:
\begin{multline*}
\sum\limits_{s=1}^M c_s ({\xi}_{k,s}+a_1 {\xi}_{k-1,s}+\dots +a_k w_s,w_l)={}\\
{}=\left(f,w_l\right),\enskip l=1,\dots,M\,.
\end{multline*}
При этом $\lim\nolimits_{J\to +\infty} \|y_J\hm-y\|_{C^{k-1}[0;n+1]}\hm=0$, 
где $y(x)$~--- точное решение задачи~(\ref{DU1})--(\ref{NDU1}).



\noindent
\textbf{Пример 4.1.}\
Рассмотрим нестационарную систему автоматического управления, 
поведение которой описывается дифференциальным уравнением:
$$
\sum\limits_{k=0}^5 a_k(t)x^{(k)}(t)=g(t),
$$
где коэффициенты~$a_k(t)$ определяются из сле\-ду\-юще\-го выражения:
{\small
\begin{multline*}
\begin{pmatrix}
a_0(t)\\a_1(t)\\a_2(t)\\a_3(t)\\a_4(t)\\a_5(t)\end{pmatrix}={}\\
\!\!{}=\!
\begin{pmatrix}0{,}5596&1{,}8918&2{,}5825&1{,}7855&0{,}6277&\!\!0{,}0909\\
0{,}7113&2{,}3843&3{,}222\hphantom{9}&2{,}1975&0{,}7588&\!\!0{,}1065\\
0{,}3717&1{,}2333&1{,}6449&1{,}1038&0{,}3728&\!\!0{,}0507\\
0{,}1002&0{,}3278&0{,}43\hphantom{99}&0{,}2827&0{,}093\hphantom{9}&\!\!0{,}0122\\
0{,}014\hphantom{9}&0{,}0449&0{,}0576&0{,}0369&0{,}0118&\!\!0{,}0015\\
0{,}0008&0{,}0025&0{,}0031&0{,}0019&0{,}006\hphantom{9}&\!\!\hphantom{9}0{,}00007
\end{pmatrix}\!\!
\begin{pmatrix}1\\t\\t^2\\t^3\\t^4\\t^5\end{pmatrix}\!.\hspace*{-2.77pt}
\end{multline*}
}

\noindent
Найти реакцию системы на входное воздействие:
\begin{multline*}
g(t)=\left(85{,}7661+338{,}5984t+497{,}0437t^2+{}\right.\\
{}+406{,}9496t^3+186{,}9354t^4+46{,}7809t^5+{}\\
\left.{}+4{,}8258t^6\right)e^{-4t}.
\end{multline*}
Начальные условия нулевые. Интервал исследования~--- $[0;5]$~c.

\columnbreak




\smallskip

\noindent
Р\,е\,ш\,е\,н\,и\,е\,.
Так как начальные условия нулевые, приближенное решение данной задачи будем  
искать в~виде
$$
x_J(t)=\sum\limits_{s=1}^M c_s {\xi}_{5,s}(t)\,,
$$
где коэффициенты $c_1,\dots, c_M$ определяются из системы линейных уравнений:
\begin{multline*}
\sum\limits_{s=1}^{2^J(n+1)+n} c_s \left(
a_5w_{s}+\sum\limits_{k=0}^4 a_k\xi_{5-k,s},w_l\right)={}\\
{}=
(g,w_l)\,,\enskip 
l=1,\dots,2^J(n+1)+n\,.
\end{multline*}
На рис.~2 показаны графики третьего и~пятого приближений~$x_2(t)$ 
(\textit{1}), $x_4(t)$~(\textit{2}) и~график 
сеточной функции $\{(t_i,\tilde{x}_i)\}$~(\textit{3}), 
полученной методом Рун\-ге--Кутта.\hfill $\square$

 { \begin{center}  %fig2
 \vspace*{9pt}
 \mbox{%
 \epsfxsize=79.639mm 
 \epsfbox{bit-2.eps}
 }


\end{center}


\noindent
{{\figurename~2}\ \ \small{Графики приближений $x_2(t)$ 
(\textit{1}), $x_4(t)$~(\textit{2}) 
и~график сеточной функции $\{(t_i,\tilde{x}_i)\}$~(\textit{3}), 
полученной методом Рун\-ге--Кутта}}
}

\vspace*{6pt}

\addtocounter{figure}{1}

%\vspace*{-36pt}

\begin{figure*} %fig3
\vspace*{1pt}
 \begin{center}
 \mbox{%
 \epsfxsize=164.758mm 
 \epsfbox{bit-3.eps}
 }
  \end{center}
\vspace*{-9pt}
\begin{minipage}[t]{79mm}
\Caption{Графики приближений $x_0(t)$ (\textit{1}), $x_2(t)$ 
(\textit{2}) и~график сеточной функции $\{(t_i,\tilde{x}_i)\}$  \label{sx}
(\textit{3}), полученной методом Рун\-ге--Кутта}
%\end{figure*}
%\begin{figure*} %fig4
\end{minipage}
\hfill
\vspace*{-9pt}
\begin{minipage}[t]{79mm}
\Caption{Графики приближений $y_0(t)$ (\textit{1}), $y_2(t)$~(\textit{2}) 
и~график сеточной функции $\{(t_i,\tilde{y}_i)\}$~(\textit{3}), полученной методом 
Рун\-ге--Кутта}
\label{sy}
\end{minipage}
\vspace*{12pt}
\end{figure*}


\smallskip

\noindent
\textbf{Пример 4.2.}\
Поведение линейной нестационарной системы описывается следующей
 системой дифференциальных уравнений:
$$
\begin{pmatrix}
{\dot x}(t) \\ {\dot y}(t)\end{pmatrix} = 
\begin{pmatrix} t^2 & 1-t \\ 1+t & t-t^2 \end{pmatrix}
 \begin{pmatrix} {x}(t) \\ {y}(t)\end{pmatrix}
 +  
 \begin{pmatrix} t^2 & 0 \\ 1 & t \end{pmatrix} 
 \begin{pmatrix} {g}_1(t) \\ {g}_2(t)\end{pmatrix}\,.
$$
Найти реакцию системы на входное воздействие:
\begin{multline*}
g_1(t) = 0{,}23315158 t^9-3{,}89665 t^8+26{,}4309725 t^7{}-\\
{}-93{,}4794 t^6+183{,}95 t^5 -200{,}83 t^4+{}\\
{}+122{,}255277 t^3-50{,}135386 t^2+13{,}095959 t-2{,}8237;
\end{multline*}

\vspace*{-12pt}

\noindent
\begin{multline*}
g_2(t) = -0{,}071962459 t^{13}+1{,}3465024 t^{12}-{}\\
{}-10{,}98105044 t^{11} + 51{,}1908385 t^{10} - 150{,}5098287 t^9+{}\\
{}+291{,}295256 t^8-378{,}61242 t^7+336{,}683591 t^6-{}
\end{multline*}

\noindent
\begin{multline*}
{}-213{,}9681871 t^5 + 106{,}48891 t^4-47{,}3676 t^3+{}\\
{}+19{,}56997 t^2-3{,}863587 t-0{,}0004283
\end{multline*}

%\pagebreak

\noindent
для начальных условий $x(0)\hm=-1$ и~$y(0)\hm=2$ на вре\-мен\-н$\acute{\mbox{о}}$м интервале $[0;2]$~c.

\smallskip



\noindent
Р\,е\,ш\,е\,н\,и\,е\,.\ \ 
Приближенное решение будем искать в~виде: 
\begin{align*}
x_{J}(t)&=-1+\sum\limits_{s=1}^M 
c_{s} {\xi}_{1,s}(t)\,;\\
y_{J}(t)&=2+\sum\limits_{s=1}^M c_{M+s} {\xi}_{1,s}(t)\,, 
\end{align*}
где $M\hm=2^J(n+1)+n$, а~коэффициенты $c_s$, $s\hm=1,2,\ldots, 2M$, определяются 
из системы линейных уравнений:
\begin{multline*}
\sum\limits_{s=1}^M c_s\left((w_s,w_l)-
\int\limits_{0}^{n+1}\!t^2{\xi}_{1,s}(t)w_l(t)\,dt\right)-{}\\
{}-\sum\limits_{s=1}^M c_{M+s}\int\limits_0^{n+1}(1-t){\xi}_{1,s}(t)w_l(t)\,dt={}\\
{}=\!\int\limits_0^{n+1}\!\left(t^2g_1(t)-t^2+2(1-t)\right)w_l(t)\,dt\,,\\
 l=1,2,\dots,M\,;
\end{multline*}

\vspace*{-12pt}

\noindent
\begin{multline*}
\sum\limits_{s=1}^M c_s\int\limits_{0}^{n+1}(1+t){\xi}_{1,s}(t)w_l(t)\,dt+{}\\
{}+
\sum\limits_{s=1}^M \!c_{M+s}\left(\int\limits_0^{n+1}\!(t-t^2){\xi}_{1,s}(t)w_l(t)\,dt-
\left(w_s,w_l\right)\right)={}\hspace*{-6.37675pt}
\end{multline*}


\noindent
\begin{multline*}
{}=\int\limits_0^{n+1}(2t^2-t+1-g_1(t)-tg_2(t))w_l(t)\,dt\,, 
\\
 l=1,2,\dots,M\,.
\end{multline*}


На рис.~\ref{sx} и~\ref{sy} показаны графики первого и~треть\-его приближений 
$x_0(t),~y_0(t)$ (\textit{1}), $x_2(t),~y_2(t)$~(\textit{2}) 
и~графики сеточных функций $\{(t_i,\tilde{x}_i)\}$, $\{(t_i,\tilde{y}_i)\}$~(\textit{3}), 
полученные методом Рун\-ге--Кутта.~\hfill$\square$

\vspace*{-6pt}

\section{Заключение}

\vspace*{-2pt}

В данной статье были обобщены известные методы применения вейвлетов 
Хаара к~приближенному решению линейных интегральных и~дифференциальных уравнений. 
Эти методы получаются из представленных здесь при $n\hm=0$, что и~соответствует 
вейвлетам Хаара. В~отличие от вейвлетов Хаара, где приближения решения интегрального 
уравнения получаются ку\-соч\-но-по\-сто\-ян\-ны\-ми, а~приближения решения 
дифференциального уравнения принадлежат классу глад\-кости~$C^{k-1}$, где $k$~--- 
порядок уравнения, использование сплайн-вейв\-лет дает возможность 
строить приближения любого класса гладкости~$C^n$.

\vspace*{-6pt}

{\small\frenchspacing
 {%\baselineskip=10.8pt
 \addcontentsline{toc}{section}{References}
 \begin{thebibliography}{99}
 
 \vspace*{-2pt}
 
\bibitem{Lepik3} 
\Au{Lepik\,{\!\!\ptb{\"{U}}}}. 
Application of the Haar wavelet transform to solving integral and 
differential equations~// Proc. Est. Acad. Sci. Ph.,  2007. 
Vol.~56. P.~28--46.
\bibitem{Blatov} %2
\Au{Блатов И.\,А., Рогова~Н.\,В.} Полуортогональные сплайновые вейвлеты и~метод 
Галеркина численного моде-\linebreak\vspace*{-11pt}

\pagebreak

\noindent
лирования тонкопроволочных антен~// 
Вычисл. матем. и~матем. физ., 2013. Т.~53. №\,5. C.~727--736.

\bibitem{Lepik4}   %3
\Au{Lepik\,{\!\!\ptb{\"{U}}}}. 
Numerical solution of differential equations using Haar wavelets~//  
Math. Comput. Simulat., 2005. Vol.~68. P.~127--143.
\bibitem{Lepik1}  %4
\Au{Lepik\,{\!\!\ptb{\"{U}}}}. Numerical solution of evolution 
equations by the Haar wavelet method~//  Appl. Math. Comput., 2007.  Vol.~185. 
P.~695--704.
\bibitem{Lepik2} %5
\Au{Lepik\,{\!\!\ptb{\"{U}}}}. Haar wavelet method for solving higher order
 differential equations~// Int. J.~Math. Comput., 2008. Vol.~1. No.\,8. P.~84--94.

\bibitem{Lepik}  %6
\Au{Lepik\,{\!\!\ptb{\"{U}}}., Hein~H}.  Haar wavelets with applications.~---  
Berlin: Springer, 2014. 207~p.
\bibitem{ArticleFinkelstein}  
\Au{Finkelstein A., Salesin~D.\,H.} Multiresolution curves~// SIGGRAPH Proceedings.~--- 
New York, NY, USA: ACM, 1994. P.~261--268.
\bibitem{Frazer} 
\Au{Frazier M.\,W}. An introduction to wavelets through linear algebra.~--- 
New York, NY, USA: Springer,  1999.  503~p.
\bibitem{Chui}  
\Au{Chui Ch.\,К}.  An introduction to wavelets.~--- Boston, MA, USA: Academic Press, 1991. 412~p.
\bibitem{Yurgu} 
\Au{Bityukov Yu.\,I.,  Akmaeva~V.\,N. }  
The use of wavelets in the mathematical and computer modelling of manufacture 
of the complex-shaped shells made of composite materials~//  Bull. 
South Ural State University. Ser. Mathematical Modelling, 
Programming and Computer Software, 2016. Vol.~9. No.\,3. P.~5--16.

\bibitem{Novikov}   %11
\Au{Новиков И.\,Я.,  Протасов~В.\,Ю.,  Скопина~М.\,А.} Теория всплесков.~--- 
М.: Физматлит, 2005.  612~c.
\bibitem{Akilov}  %12
\Au{Канторович Л.\,В.,  Акилов~Г.\,П.} Функциональный анализ.~--- 
М.: Наука, 1977. 744~с.

%\bibitem{BookSmolencev} 
%\Au{Смоленцев Н.\,К}. Основы теории вейвлетов. Вейвлеты в~MatLab.~--- 
%М.: ДМК Пресс,  2005. 303~c.
%\bibitem{Pupkov} Пупков, К.А., Н.Д. Егупов. Методы классической и~современной теории автоматического управления. Математические модели, динамические характеристики и~анализ систем автоматического управления. М.: Издательство МГТУ им. Н.Э. Баумана. 2004. 656 с.
 \end{thebibliography}

 }
 }

\end{multicols}

\vspace*{-6pt}

\hfill{\small\textit{Поступила в~редакцию 21.03.17}}

\vspace*{8pt}

%\newpage

%\vspace*{-24pt}

\hrule

\vspace*{2pt}

\hrule

%\vspace*{8pt}


\def\tit{THE USE OF WAVELETS FOR~THE~CALCULATION OF~LINEAR CONTROL SYSTEMS 
WITH~LUMPED PARAMETERS}

\def\titkol{The use of wavelets for~the~calculation of~linear control systems 
with~lumped parameters}

\def\aut{Yu.\,I.~Bityukov and~E.\,N.~Platonov}

\def\autkol{Yu.\,I.~Bityukov and~E.\,N.~Platonov}

\titel{\tit}{\aut}{\autkol}{\titkol}

\vspace*{-9pt}


\noindent
Moscow Aviation Institute (National 
Research University), 4~Volokolamskoye Highway, Moscow 125993, Russian Federation



\def\leftfootline{\small{\textbf{\thepage}
\hfill INFORMATIKA I EE PRIMENENIYA~--- INFORMATICS AND
APPLICATIONS\ \ \ 2017\ \ \ volume~11\ \ \ issue\ 4}
}%
 \def\rightfootline{\small{INFORMATIKA I EE PRIMENENIYA~---
INFORMATICS AND APPLICATIONS\ \ \ 2017\ \ \ volume~11\ \ \ issue\ 4
\hfill \textbf{\thepage}}}

\vspace*{3pt}



\Abste{In many disciplines, problems appear which can be formulated with 
the aid of differential or integral equations.  In simpler cases, such equations 
can be solved analytically, but for more complicated cases, numerical 
procedures are needed. In recent times, the wavelet-based methods 
have gained great popularity, where different wavelet families such as 
Daubechies, Coiflet, etc.\ wavelets are applied. A~shortcoming of these wavelets 
is that they do not have an analytic expression. For this reason, differentiation 
and integration of these wavelets are very complicated. The paper presents 
algorithms for the numerical solution of linear integral and differential 
equations based on spline wavelets on the interval. The algorithms generalize 
the well-known methods based on Haar wavelets, which are a~particular case 
of spline wavelets. The results presented can be applied for 
the analysis of linear systems with lumped parameters.}

\KWE{spline wavelet; differential equation; integral equation}



\DOI{10.14357/19922264170412} 

\vspace*{-6pt}

\Ack
\noindent
This work is a part of Project No.\,2.2461.2017 
supported by the Russian Ministry of Education and Science.



\vspace*{3pt}

  \begin{multicols}{2}

\renewcommand{\bibname}{\protect\rmfamily References}
%\renewcommand{\bibname}{\large\protect\rm References}

{\small\frenchspacing
 {%\baselineskip=10.8pt
 \addcontentsline{toc}{section}{References}
 \begin{thebibliography}{99}

\bibitem{1-bit-1}
\Aue{Lepik,  $\ddot{\mbox{U}}$}. 2007. Application of the
 Haar wavelet transform to solving integral and differential equations. 
 \textit{Proc. Est. Acad. Sci. Ph.} 56:28-46.
\bibitem{2-bit-1}
\Aue{Blatov,  I.\,A., and  N.\,V.~Rogova.} 
2013.  Application of semi-orthogonal spline wavelets and Galerkin
method to the numerical simulation of thin wire antennas. 
\textit{Comp. Math. Math. Phys.} 53(5):564--572.
\bibitem{5-bit-1} %3
\Aue{Lepik,  $\ddot{\mbox{U}}$.} 2005. Numerical solution of differential 
equations using Haar wavelets. \textit{Math. Comput. Simulat.} 68:127--143.
\bibitem{3-bit-1} %4
\Aue{Lepik, $\ddot{\mbox{U}}$.}
2007. Numerical solution of evolution equations by the 
Haar wavelet method. \textit{Appl. Math. Comput.} 185:695--704.
\bibitem{4-bit-1} %5
\Aue{Lepik,  $\ddot{\mbox{U}}$.} 
2008. Haar wavelet method for solving higher order differential equations. 
\textit{Int. J.~Math. Comput.} 1(8):84--94.

\bibitem{6-bit-1}
\Aue{Lepik,  $\ddot{\mbox{U}}$, and H.~Hein.} 2014. \textit{Haar wavelets 
with applications.}  Berlin: Springer. 207~p.
\bibitem{7-bit-1}
\Aue{Finkelstein, A., and D.\,H.~Salesin}. 1994. 
{Multiresolution curves}. \textit{SIGGRAPH Proceedings}.  New York, NY: ACM. 261--268.
%\item Demko, S., W.F. Moss and P.W. Smith. Decay rates for inverses of band matrices. 1984. Math. Comp. V. 43, \textnumero{167}. 491--499.
\bibitem{8-bit-1}
\Aue{Frazier, M.\,W.} 1999. \textit{An introduction to wavelets through linear algebra.} 
New York, NY: Springer. 503~p.
\bibitem{9-bit-1}
\Aue{Chui,  Ch.\,К.} 1991. \textit{An introduction to wavelets.} Boston, MA: 
Academic Press. 412~p.
\bibitem{10-bit-1}
\Aue{Bityukov, Yu.\,I., and  V.\,N.~Akmaeva.} 2016. 
The use of wavelets in the mathematical and computer modelling of manufacture 
of the complex-shaped shells made of composite materials.   
\textit{Bull. South Ural State University. Ser. Mathematical Modelling, 
Programming and Computer Software} 9(3):5--16.

\bibitem{12-bit-1}
\Aue{Novikov, I.\,Ya.,  V.\,Yu.~Protasov, and M.\,A.~Skopina.}
2005. \textit{Teoriya vspleskov} [{The theory of wavelets}]. Moscow: Fizmatlit.  612~p.

\bibitem{11-bit-1}
\Aue{Kantorovich, L.\,V., and G.\,P.~Akilov.} 1977. \textit{Funktsional'nyy analiz} 
[{Functional analysis}]. Moscow: Nauka. 744~p.
%\bibitem{13-bit-1}
%\Aue{Smolentsev, N.\,K.} 2005. \textit{Osnovy teorii veyvletov. Veyvlety v~MatLab} 
%[{Fundamentals of the theory of wavelets. Wavelets in MatLab}].  Moscow: 
%DMK Press. 303~p.
\end{thebibliography}

 }
 }

\end{multicols}

\vspace*{-6pt}

\hfill{\small\textit{Received March 21, 2017}}

%\vspace*{-10pt}

\Contr

\noindent
\textbf{Bityukov Yuri I.} (b.\ 1972)~--- 
Doctor of Science in technology, associate professor, 
Moscow Aviation Institute (National Research University), 
4~Volokolamskoye Highway, Moscow 125993, Russian Federation; 
\mbox{yib72@mail.ru}

\vspace*{3pt}

\noindent
\textbf{Platonov Evgeny N.} (b.\ 1976)~--- Candidate of Science (PhD) in physics 
and mathematics, associate professor, Moscow Aviation Institute (National 
Research University), 4~Volokolamskoye Highway, Moscow 125993, Russian Federation;
\mbox{en.platonov@gmail.com}
\label{end\stat}


\renewcommand{\bibname}{\protect\rm Литература} 