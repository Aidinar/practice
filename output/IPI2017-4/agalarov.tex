\def\stat{agalarov}

\def\tit{ОБ ОДНОЙ ЗАДАЧЕ МАКСИМИЗАЦИИ ДОХОДА СИСТЕМЫ МАССОВОГО ОБСЛУЖИВАНИЯ
 ТИПА $G/M/1$  С~ПОРОГОВЫМ УПРАВЛЕНИЕМ ОЧЕРЕДЬЮ$^*$}

\def\titkol{Об одной задаче максимизации дохода СМО типа $G/M/1$ 
с~пороговым управлением очередью}

\def\aut{Я.\,М.~Агаларов$^1$, В.\,С.~Шоргин$^2$}

\def\autkol{Я.\,М.~Агаларов, В.\,С.~Шоргин}

\titel{\tit}{\aut}{\autkol}{\titkol}

\index{Агаларов Я.\,М.}
\index{Agalarov Ya.\,M.}
\index{Шоргин В.\,С.}
\index{Shorgin V.\,S.}




{\renewcommand{\thefootnote}{\fnsymbol{footnote}} \footnotetext[1]
{Работа выполнена при частичной финансовой поддержке РФФИ (проект 15-07-03406).}}


\renewcommand{\thefootnote}{\arabic{footnote}}
\footnotetext[1]{Институт проблем информатики 
Федерального исследовательского центра <<Информатика 
и~управ\-ле\-ние>> Российской академии наук, \mbox{agglar@yandex.ru}}
\footnotetext[2]{Институт проблем информатики 
Федерального исследовательского центра <<Информатика 
и~управ\-ле\-ние>> Российской академии наук, \mbox{VShorgin@ipiran.ru}}

\vspace*{-6pt}

  
    
  \Abst{Рассматривается задача максимизации среднего дохода 
  системы массового обслуживания
  (СМО) типа $G/M/1$ 
в~единицу времени на множестве стационарных пороговых стратегий ограничения доступа 
с~одной <<точкой переключения>>. Доход определяется следующими параметрами, 
измеряемыми в~стоимостных единицах: плата, по\-лу\-ча\-емая за обслуживание заявок; 
затраты 
на техническое обслуживание прибора; вычет из дохода за задержку заявок в~очереди; 
штраф 
за необслуженные заявки; штраф за простой системы. Сформулированы необходимые 
и~достаточные условия оптимальности конечного порогового значения, разработан метод 
последовательного спуска к~оптимальному порогу, предложен алгоритм расчета 
оптимального порога и~соответствующего значения целевой функции.} 

  
  \KW{система массового обслуживания; пороговая стратегия; оптимизация}
  
  \DOI{10.14357/19922264170407} 
  
\vspace*{6pt}


\vskip 10pt plus 9pt minus 6pt

\thispagestyle{headings}

\begin{multicols}{2}

\label{st\stat}
  
  
  \section{Введение}
  
  Для повышения эффективности работы современных вычислительных сетей 
используют алгоритмы управления потоками (ограничения нагрузки), наиболее 
применяемыми из которых являются различные модификации порогового 
управления\linebreak (пороговых стратегий)~[1]. Исследованием и~разработкой 
пороговых стратегий управления потоками занимались практически с~начала 
зарождения вычислительных сетей с~целью защиты \mbox{связных} и~вычислительных 
ресурсов от перегрузок~[2]. Одним из основных методов исследования 
эф\-фек\-тив\-ности пороговых стратегий является математическое моделирование 
с~использованием аппарата теории очередей, предметом исследования которой 
являются СМО различного типа.
  
  Большинство работ, в~которых рассмотрены СМО с~пороговой стратегией 
управления потоками, посвящено расчету характеристик системы (средней 
длины очереди, среднего времени пребывания, вероятности отклонения заявки, 
загруженности приборов и~т.\,д.)\ при заданной пороговой стратегии (краткий 
обзор некоторых из этих работ проведен в~[3]). В~ряде работ, посвященных 
данной тематике, ставится задача оптимизации пороговой стратегии в~смысле 
максимизации дохода системы, представленного в~виде стоимостного 
функционала (см., например,~[4--7]). Хотя практический интерес к~такой 
постановке задачи в~смысле повышения эффективности сис\-тем, как 
представляется, не ниже, чем к~задачам расчета характеристик систем при 
фиксированной пороговой стратегии, вопрос существования эффективных 
методов и~алгоритмов поиска оптимальных пороговых стратегий для СМО 
остается открытым, за исключением простых СМО ($M/M/1$,  
$M/M/n$~\cite{4-aga}) и~простых целевых функций (допустимая средняя 
задержка заявок в~системе, допустимая интенсивность потери). Для более 
сложных СМО (например, $G/M/1$, $M/G/1$) с~более сложными целевыми 
функциями результаты исследований ограничиваются математическими 
постановками задач и~эвристическими алгоритмами их решения. 

Из недавно 
опубликованных в~отечественной литературе работ, посвященных пороговой 
стратегии управ\-ле\-ния очередью с~одной <<точкой переключения>>, 
отметим~\cite{5-aga}, где для сис\-те\-мы $G/M/n$ сформулирована 
математическая постановка максимизации дохода сис\-те\-мы на множестве 
пороговых стратегий с~одним переключением (одной гис\-те\-ре\-зис\-ной петлей), 
фиксированными платой за своевременное обслуживание и~штрафом за 
невыполнение этого условия для допущенной в~систему заявки. В~работе 
предложен эвристический алгоритм поиска оптимальной стратегии 
и~выдвинута гипотеза о~том, что для сис\-тем $G/G/n$ существует единственное 
решение указанной задачи в~виде простой пороговой стратегии. 

Аналогичная 
задача рассмотрена в~работе~\cite{6-aga} для сис\-те\-мы $M/D/1$, где также 
приведена математическая постановка задачи и~получена нижняя оценка для 
оптимального порогового значения. Постановка задачи, наиболее близкая 
к~рас\-смат\-ри\-ва\-емой в~настоящей статье, рассмотрена в~работе~\cite{7-aga}, где 
решение задачи максимизации среднего дохода, получаемого системой 
$M/G/1$ в~единицу времени, ищется на множестве смешанных пороговых 
стратегий при входной нагрузке меньше единицы. В~работе доказано, что если 
решение задачи существует, то оно принадлежит множеству чистых стратегий. 
В~работе~\cite{8-aga} в~рамках рассмотренной в~работе~\cite{7-aga} задачи 
доказана теорема о~необходимых и~достаточных условиях оптимальности 
чистой пороговой стратегии и~приведен алгоритм, гарантирующий поиск 
оптимальной стратегии за конечное число шагов. 
  
  В данной статье рассматривается аналогичная задача максимизации среднего 
дохода СМО типа $G/M/1$ на множестве чистых пороговых стратегий 
ограничения потока заявок в~систему в~случае, когда плата за обслуживание 
и~решение о~приеме заявки принимаются в~момент ее поступления. 

Ниже 
проведены исследования, касающиеся вопросов существования решения задачи 
максимизации дохода и~метода поиска оптимальной стра\-тегии.

\vspace*{-6pt}
  
  \section{Постановка задачи}
  
  \vspace*{-2pt}
  
  Рассматривается СМО типа $G/M/1$ с~накопителем бесконечной емкости 
и~одним прибором обслуживания, на которую поступает рекуррентный поток 
заявок с~функцией распределения вероятностей~$H(t)$. Время обслуживания 
заявки распределено по экспоненциальному закону с~параметром $\mu\hm>0$. 
Поступившая заявка допускается в~накопитель системы (занимает любое 
свободное место в~накопителе), если в~момент ее поступления число занятых 
мест в~накопителе меньше~$k$, $k\hm>0$~--- некоторое заданное целое число 
(ниже тривиальный случай $k\hm=0$ не рассматривается). Такую процедуру 
доступа заявок в~систему называют в~литературе пороговой стратегией 
управления доступом с~одной <<точкой переключения>>, в~дальнейшем для 
краткости назовем стратегией. Обозначим стратегию соответствующим 
пороговым значением~$k$. Если заявка допущена в~накопитель, она занимает 
любое свободное место в~накопителе и~обслуживается на приборе в~порядке 
поступления. Заявка покидает систему только при завершении обслуживания, 
освободив одновременно прибор и~накопитель, а~на освободившийся прибор 
поступает очередная заявка из накопителя (если таковая есть). Система 
получает доход, который формируется следующими составляющими:
  \begin{description}
  \item[\,] $C_0\geq 0$~--- плата, получаемая сис\-те\-мой, если поступившая 
заявка будет обслужена сис\-те\-мой (принята в~накопитель); 
  \item[\,] $C_1\geq 0$~--- величина штрафа, который платит сис\-те\-ма, если 
поступившая заявка отклонена;
  \item[\,] $C_2\geq 0$~--- вычет из дохода системы за единицу времени 
ожидания заявки в~системе;
  \item[\,] $C_3\geq 0$~--- вычет из дохода системы за единицу времени 
простоя прибора (отсутствия заявок в~системе);
  \item[\,] $C_4\geq 0$~--- затраты системы в~единицу времени на техническое 
обслуживание системы. 
  \end{description}
  
Всюду ниже под доходом системы будем понимать суммарный доход с~учетом 
всех указанных выше составляющих.
  
  Отметим, что процесс обслуживания заявок в~данной системе описывается 
цепью Mаркова, где переходы цепи определяются моментами поступления 
заявок и~состояние системы есть число заявок, находящихся в~ней в~момент 
поступления (см., например,~\cite{10-aga, 9-aga}). Отметим также, что при 
заданной стратегии~$k$ указанная цепь Маркова имеет один положительный 
возвратный класс состояний $i\hm=0,\ldots ,k$.
  
  Введем обозначения:
  \begin{description}
  \item[\,] $\{\pi_i^k,\ 0\leq i\leq k\}$~--- стационарное распределение 
вероятностей цепи при стратегии~$k$ ($\pi_i^k$~--- стационарная вероятность 
того, что цепь находится в~состоянии~$i$);
  \item[\,] $Q^k$~---  предельное среднее значение дохода системы в~единицу 
времени;
  \item[\,] $g^k$~--- предельное среднее значение суммарного дохода системы, 
усредненного по числу поступивших заявок;
  \item[\,] $q_i^k$~--- средний доход, получаемый системой в~состоянии~$i$ 
при стратегии~$k$, $i\geq 0$;
  \item[\,] $\overline{v}=\int\nolimits_0^\infty t\,dH(t)$~--- среднее время между 
соседними моментами поступления заявок, $0\hm< \overline{v}\hm< \infty$.
  \end{description}
  
  Предельное среднее значение суммарного дохода системы, усредненного по 
числу поступивших заявок при стратегии~$k$, равно пределу
  $$
  g^k =\lim\limits_{T\to\infty} \sum\limits_{n=1}^{N_{\mathrm{вх}}(T)} 
\fr{d_n^k}{N_{\mathrm{вх}}(T)}\,,
$$
предельное среднее значение дохода системы в~единицу времени при 
стратегии~$k$  равно пределу

\pagebreak

\noindent
\begin{multline}
\hspace*{-8pt}Q^k=\lim\limits_{T\to\infty}\! \sum\limits_{n=1}^{ N_{\mathrm{вх}}(T)} \!\!
\fr{d_n^k}{T} =\lim\limits_{T\to\infty} \fr{ N_{\mathrm{вх}}(T)}{T} 
\!\sum\limits_{n=1}^{ N_{\mathrm{вх}}(T)}\!\! \fr{d_n^k}{ 
N_{\mathrm{вх}}(T)}={}\\
{}=\lim\limits_{T\to\infty} \fr{ N_{\mathrm{вх}}(T)}{T} \lim\limits_{T\to\infty} 
\sum\limits_{n=1}^{ N_{\mathrm{вх}}(T)} \fr{d_n^k}{ 
N_{\mathrm{вх}}(T)}=\fr{g^k}{\overline{v}}\,,
\label{e1-aga}
\end{multline}
где $d_n^k$~--- доход, полученный системой при стратегии~$k$ за $n$-ю 
поступившую заявку; $ N_{\mathrm{вх}}(T)$~--- число поступивших за отрезок 
времени $[0,T]$ заявок. Из определения вложенной цепи Маркова следует
\begin{equation*}
g^k=\sum\limits^k_{i=0} \pi_i^k q_i^k\,.
%\label{e2-aga}
\end{equation*}

Ставится задача максимизации функции~$Q^k$ на множестве стратегий 
$k\hm>0$, которая эквивалентна (см.~(\ref{e1-aga})) задаче: найти 
оптимальную стратегию $k^0\hm>0$ такую, что
\begin{equation}
\max\limits_{k>0} g^k =g^{k^0}\,.
\label{e3-aga}
\end{equation}
  
  \section{Метод решения}
  
  Выпишем выражения для стационарных вероятностей~$\pi_j^k$, 
$j\hm=0,\ldots , k$. Для вероятностей переходов~$p_{ij}^k$ вложенной цепи 
Маркова справедливы формулы~\cite{10-aga, 9-aga}:
  \begin{equation}
  \left.
  \begin{array}{rlrl}
  p_{ij}^k &= r_{i+1-j} & \mbox{при }& 0\leq i\leq k-1\,,\\[3pt]
  &&& 1\leq j\leq i+1\,;  
\\[6pt]
  p_{i0}^k&\displaystyle=1-\sum\limits_{m=0}^i r_m & \mbox{при }& i\leq k-1\,;\\[6pt]
  p_{kj}^k &= p^k_{k-1,j} &\ \mbox{при }& j\leq k\,,
  \end{array}
  \right\}
  \label{e4-aga}
  \end{equation}
  где
  
  \noindent
  $$
  r_m=\int\limits_0^\infty \fr{(\mu t)^m}{m!}\, e^{-\mu t}\,dH(t)\ \mbox{при } 
m\geq 0\,.
  $$
  
  Для рассматриваемой цепи Маркова при стратегии~$k$ стационарное 
распределение вероятностей является единственным решением системы 
уравнений~\cite{10-aga, 9-aga}:
  \begin{equation}
  \left.
  \begin{array}{rl}
  \pi_0^k &= \displaystyle \sum\limits_{i=0}^k \pi_i^k p_{i0}^k\,;\\[6pt]
  \pi_j^k &= \displaystyle \sum\limits_{i=j-1}^k \pi_i^k p_{ij}^k\,,\enskip 
i=1,\ldots, k-1\,;\\[6pt]
  \pi_k^k &= \pi^k_{k-1} p^k_{k-1, k}+\pi_k^k p^k_{k-1, k}\,;\\[6pt]
  \displaystyle \sum\limits_{i=0}^k \pi_i^k &= 1\,,\enskip \pi_i^k>0\,,\enskip 
i=0,\ldots, k\,.
  \end{array}
  \right\}
  \label{e5-aga}
  \end{equation}
  
  Из последних~$k$~уравнений, положив $\pi_j^k\hm= R_j^k \pi_k^k$,  
получим:

\columnbreak

\noindent
  $$
  R_k^k=1\,;\enskip R^k_{k-1}=\fr{1-r_0}{r_0}\,;
  $$
   \begin{multline}
   R^k_{j-1} = \fr{R_j^k(1-r_1) -\sum\nolimits^{k-1}_{i=j+1} R_i^k r_{i+1-j} - 
   r_{k-j}}{r_0}\,,\\ 1\leq j\leq k-1\,;
   \label{e6-aga}
   \end{multline}
   
   \vspace*{-12pt}
   
   \noindent
   \begin{equation}
   \left.
   \begin{array}{rl}
   \pi_k^k &=\displaystyle \left( 1+\sum\limits_{i=0}^{k-1} R_i^k\right)^{-1}\,;\\[14pt]
    \pi_j^k &= \displaystyle\fr{R_j^k}{1+\sum\nolimits_{i=0}^{k-1} R_i^k}\,,\
 j=0,\ldots , k-1\,.
 \end{array}
 \right\}
   \label{e7-aga}
   \end{equation}
  
  Справедливы следующие леммы (доказательства см.\ в~приложении).
  
  \smallskip
  
  \noindent
  \textbf{Лемма~1.} \textit{Среднее значение дохода, получаемого сис\-те\-мой 
при стратегии~$k$ в~состоянии~$i$, равно}
  \begin{equation}
  \left.
  \begin{array}{rl}
  \hspace*{-2mm}q_i^k &=\displaystyle\fr{C_2}{\mu}\left[ \fr{1}{2}\sum\limits_{m=1}^{i+1} (m-1) mr_m - i 
\sum\limits_{m=1}^{i+1} mr_m -{}\right.\\[6pt]
&\displaystyle\left.\hspace*{8mm}{}-\fr{1}{2}\,i(i+1) \sum\limits^\infty_{m=i+2} 
r_m\right]- {}\\[6pt]
&\hspace*{13mm}{}-\displaystyle\fr{C_3}{\mu} \sum\limits^\infty_{m=i+2} (m-i-1)r_m +{}\\
&\hspace*{18mm}{}+C_0 -
C_4\overline{v}\,,\enskip 0\leq i\leq k-1\,;
\\[6pt]
&  q_k^k=q_{k-1}^k -C_0-C_1\,.
 \end{array}
 \right\}
   \label{e8-aga}
   \end{equation}
   
  
  \noindent
  \textbf{Лемма~2.} \textit{Справедливы равенства}:
  
  \noindent
  \begin{multline}
  q^k_{i+1} =q_i^k +\fr{C_3}{\mu}\sum\limits^\infty_{m=i+1} r_m -{}\\
  {}-
\fr{C_2}{\mu} \sum\limits_{m=1}^{i+1} mr_m -\fr{C_2(i+1)}{\mu} 
\sum\limits_{m=i+2}^\infty r_m\,,\\
  0\leq i\leq k-2\,;
  \label{e9-aga}
  \end{multline}
  
  \vspace*{-6pt}
  
  \noindent
  \begin{equation}
  \pi_{i+1}^{k+1}=\left(1- \pi_0^{k+1}\right) \pi_i^k\,,\enskip i=0,\ldots, k\,.
  \label{e10-aga}
  \end{equation}
  
  {Введем обозначения} $F(k)$, $\overline{W}(k)$ и~$G(k)$, $k\hm>0$:
  \begin{equation}
  \left.
  \begin{array}{rl}
  \hspace*{-1mm}\overline{W}(k) &  = \displaystyle
  \overline{v}-\fr{1}{\mu}\left[ \sum\limits_{i=0}^{k-1} \pi_i^k \!
\sum\limits^\infty_{m=i+2} (m-i-1) r_m +{}\right.\\[6pt]
&\left.\hspace*{19mm}{}+\pi_k^k\displaystyle \sum\limits^\infty_{m=k+1} (m-
k)r_m\right];\\[6pt]
  \hspace*{-1mm}F(k) &=\displaystyle\fr{1-\pi_0^{k+1}}{\pi_0^{k+1}}\,\overline{W}(k);
   \end{array}\!\!
   \right\}\!\!
   \label{e11-aga}
   \end{equation}
   \begin{equation}
   G(k) = C_0+\fr{C_3}{\mu} \left( C_3+C_4\right) \overline{v} -C_2 F(k)\,,\enskip k>0\,.
   \label{e12-aga}
   \end{equation}
  
  \noindent
  \textbf{Лемма~3.} \textit{Справедливо соотношение}
  \begin{equation}
  g^k-g^{k+1} =\pi_0^{k+1}\left[ g^k-G(k)\right]\,,\enskip k>0\,.
  \label{e13-aga}
  \end{equation}
  
  \noindent
  \textbf{Лемма~4.} \textit{Функция $G(k)$ не возрастает по} $k\hm>0$.
  
  \smallskip
  
  Справедлива следующая теорема (доказательство см.\ в~приложении).
  
  \smallskip
  
  \noindent
  \textbf{Теорема~1.} \textit{Справедливы утверждения: $(1)$~если 
$\mathop{\mathrm{inf}}\nolimits_{k>0} G(k)\hm< \mathop{\mathrm{sup}}\nolimits_{k>0} g^k$, то 
при любых значениях параметров $C_i\hm\geq 0$, $i\hm= 0$, $1$, $3$, $4$, 
$C_2\hm>0$ cуществует стратегия $0\hm< k^0\hm<\infty$, иначе, если $g^1\hm< 
G(1)$, то $k^0\hm=\infty;$ $(2)$~если $g^1\hm\geq G(1)$, то $k^0\hm=1;$ 
$(3)$~если $C_2\hm=0$ и~$g^1\hm< G(1)$, то $k^0\hm=\infty;$ $(4)$~условие $g^{k^0-
1}\hm< g^{k^0}$, $g^{k^0+1}\hm\leq g^{k^0}$ является необходимым 
и~достаточным для оптимальности стратегии} $1\hm< k^0\hm<\infty$.
  
  \smallskip
  
  Из теоремы~1 следует следующий алгоритм поиска стратегии~$k^0$.
  \begin{enumerate}[1.]
\item Вычислить $a=g^1$ и~$b\hm=G(1)$.
\item Если $C_2=0$ и~$b\hm>a$, то положить $k^0\hm=\infty$ и~перейти 
к~п.~14.
\item Положить $k=1$.
\item Выбрать значение приращения $\Delta k$ ($\Delta k\hm\geq 1$~--- целое 
число).
\item Если $a\hm\geq b$, то положить $k^0\hm=1$ и~перейти к~п.~14.
\item Положить $k=k+\Delta k$.
\item Вычислить $a=g^k$, $b\hm= G(k)$. 
\item Если $b>a$, то перейти к~п.~6, иначе положить $k_1\hm= k\hm - \Delta k$, 
$k_2\hm=k$.
\item Вычислить $k=[(k_1+k_2)/2]$, где $[\cdot ]$~--- целая часть числа 
в~квадратных скобках.
\item Вычислить $a=g^k$, $b\hm= G(k)$.
\item Если $b>a$, то положить $k_1\hm=k$, иначе положить $k_2\hm=k$.
\item Если $k_2-k_1>1$, то перейти к~п.~9.
\item Если $k_1=k$, то положить $k^0\hm= k_2$, иначе положить $k^0\hm= k_1$. 
\item Конец алгоритма.
\end{enumerate}

\smallskip

  Трудоемкость алгоритма равна $\sim ([k^0/(\Delta k)]\hm+ \log_2\Delta k)$  
вычислений функций~$g^k$ и~$G(k)$.
  
  Ниже на численном примере проиллюстрирован принцип работы 
предложенного алгоритма и~изображено в~виде графиков поведение 
функций~$g^k$ и~$G(k)$ при изменении порогового значения. 
  
  \section{Пример}
  
  В качестве примера рассмотрим СМО $H_n/M/1$ с~функцией распределения 
входного потока 
  $H_n(t)\hm= \sum\nolimits_{i=1}^n f_i (1\hm- e^{-\lambda_i t})$, $f_i\hm\geq 0$, 
$\lambda_i\hm\geq 0$, $0\hm\leq i\hm\leq n$, $\sum\nolimits^n_{i=1} f_i\hm=1$. 
  
  Функция $F(k)$ (см.~(\ref{e11-aga})) в~данном случае имеет вид:
  
  \vspace*{-2pt}
  
  \noindent
  \begin{multline*}
  F(k) =\left(\overline{v} -\left[ \sum\limits^n_{j=1} f_j \fr{1}{\lambda_j} 
\sum\limits^{k-1}_{i=0} \pi_i \left(\fr{\mu}{\lambda_j+\mu}\right)^{i+2} +{}\right.\right.\\
\left.\left.{}+ \pi_k^k 
\sum\limits^n_{j=1} f_j \fr{1}{\lambda_j} 
\left(\fr{\mu}{\lambda_j+\mu}\right)^{k+1}\right]\right)\Bigg/ 
\left(\sum\limits^n_{j=1} f_j \times{}\right.\\
{}\times\sum\limits_{i=0}^{k-1} \pi_i 
\left(\fr{\mu}{\lambda_j+\mu}\right)^{i+2} +{}\\
\left.{}+ \pi_k^k \sum\limits^n_{j=1} f_j 
\fr{1}{\lambda_j} 
\left(\fr{\mu}{\lambda_j+\mu}\right)^{k+1}\right) - \fr{1}{\mu}\,.
  \end{multline*}
  
  
  \begin{figure*} %fig1
    \vspace*{1pt}
 \begin{center}
 \mbox{%
 \epsfxsize=100.25mm 
 \epsfbox{aga-1.eps}
 }

\vspace*{6pt}

  {\small Зависимости предельного дохода $g^k$ (залитые значки)
  и~функции~$G(k)$ (пустые значки) от порогового 
значения~$k$ при $C_0\hm=20$, 
$C_1\hm=10$, $C_2\hm=1$, $C_3\hm=C_4\hm=0{,}01$, $\mu\hm=1$, $n\hm=2$, 
$f_1\hm= 0{,}3$, $f_2\hm=0{,}7$:
\textit{1}~--- $\lambda_1\hm=1$, $\lambda_2\hm=0{,}7$; 
\textit{2}~--- $\lambda_1\hm=2$, $\lambda_2\hm=1$ 
}

\end{center}
\vspace*{-9pt}
  \end{figure*}
  
  На рисунке приведены зависимости функций~$g^k$ и~$G(k)$ от порогового 
значения и~показано отношение эквивалентности условий $g^{k+1}\hm> g^k$ 
и~$g^k\hm< G(k)$. 
Оптимальный порог на рисунке обозначен через~$k$.

\vspace*{-5pt}
  
  \section{Заключение}
  
  \vspace*{-2pt}
  
  Выше в~постановке задачи~(\ref{e3-aga}) предполагалось, что плата за 
обслуживание поступает в~систему в~момент приема заявки и~не зависит от 
времени занятия прибора. Предложенный выше метод решения  
задачи~(\ref{e3-aga}) можно применить и~в~случае, когда плату за 
обслуживание система получает в~момент окончания обслуживания заявки 
и~величина платы прямо пропорциональна длительности обслуживания на 
приборе. Заметим, что все составляющие дохода системы, кроме дохода, 
получаемого системой в~виде платы за обслуживание заявок, в~данном случае 
и~в~случае~(\ref{e8-aga}) совпадают. В~данном случае  средняя величина 
дохода~$d_i^k$, получаемого системой при стратегии~$k$ в~состоянии~$i$ как 
плату за обслуживание заявок, вычисляется по формуле 

  \vspace*{-2pt}

\noindent
  \begin{multline*}
  d_i^k=\fr{C_0}{\mu}\left[ \int\limits_0^\infty \left( \sum\limits^i_{m=1} 
\fr{m(\mu v)^m}{m!}\,e^{-\mu v} +{}\right.\right.\\
\left.\left.{}+(i+1) \sum\limits^\infty_{m=i+1} \fr{(\mu 
v)^m}{m!}\,e^{-\mu v}\right) dH(v)\right]={}\\
  {}= \fr{C_0}{\mu} \sum\limits^i_{m=1} mr_m +\fr{C_0}{\mu} (i+1) 
\sum\limits^\infty_{m=i+1} r_m
  \end{multline*}
  при $i<k$, $d_k^k=d^k_{k-1}$.
  
  \pagebreak
  
  Далее, повторив те же выкладки, что и~выше, получим выражение 
вида~(\ref{e13-aga}) с~новой функцией~$G(k)$. Алгоритм поиска оптимальной 
стратегии остается без изменения.
  
  Основными результатами данной работы являются доказательство 
необходимых и~достаточных условий оптимальности порогового значения 
и~алгоритм гарантированного поиска оптимальной стратегии. 
  
  Полученные в~данной работе результаты могут быть использованы для 
поиска оптимальных пороговых стратегий управления потоками в~системах, 
моделируемых с~помощью СМО типа $G/M/1$ ($G/M/1/r$). 

\vspace*{-16pt}
   
{\small \section*{\raggedleft Приложение}
  
  \noindent
  Д\,о\,к\,а\,з\,а\,т\,е\,л\,ь\,с\,т\,в\,о\ \ леммы~1. Фиксируем состояние 
$0\hm\leq i\hm\leq k\hm-1$, и~пусть время нахождения системы в~состоянии~$i$ 
равно~$v$. Найдем выражения для суммарного среднего времени ожидания 
всех заявок в~очереди и~среднего времени простоя прибора в~состоянии~$i$, 
т.\,е.\ в~интервале времени $(0,v]$. 
  
  Рассмотрим случайные величины вида $W_l\hm= \sum\nolimits^l_{j=1} 
\tau_j$, $l\hm\geq 1$, где $\tau_j$~--- независимые экспоненциально 
распределенные случайные величины с~параметром $\mu\hm>0$. Пусть $B_m$~--- событие 
вида ($W_m\hm\leq v$, $W_{m+1}\hm>v$), $B_0$~--- событие ($W_1\hm>v$). 
Обратим внимание, что~$B_m$, $m\hm\geq 0$,~--- несовместные события 
и~в~совокупности составляют полную группу событий. 
  
  Известно (см., например,~\cite{9-aga}), что совместное распределение 
величин~$W_l$, $l\hm\leq m$,  при условии выполнения события~$B_m$ 
совпадает с~распределением порядковых статистик из выборки~$m$, взятой из 
равномерного распределения на $(0,v]$, и~маргинальным условным 
распределением случайной величины~$W_l/v$ является бе\-та-рас\-пре\-де\-ле\-ние  
с~плот\-ностью

\columnbreak

\noindent
  \begin{multline*}
  f(x/B_m) ={}\\
  {}=\begin{cases}
  \fr{m!}{(l-1)! (m-l)!}\,x^{l-1} (1-x)^{m-l}\,, &\ 0<x<1\,;\\
  0 &\ \hspace*{-25pt}\mbox{в~противном~случае}.
  \end{cases}
 \end{multline*}
Тогда условное среднее значение $M[W_l/B_m]$ случайной величины~$W_l$ при 
условии~$B_m$ равно $(l/(m+1))v$. 

  Заметим, что общее время ожидания $l$-й в~очереди заявки при $l\hm\geq 1$ 
является случайной величиной вида~$W_l$, а~выполнение события~$B_m$ 
равносильно завершению обслуживания за время~$v$ ровно~$m$~заявок. 
  
  Обозначим через $\overline{W}_{\mathrm{обсл}/m}$~--- среднее суммарное 
время ожидания заявок, завершивших обслуживание или приступивших 
к~обслуживанию в~состоянии~$i$, при условии~$B_m$; 
$\overline{W}_{\mathrm{необсл}/m}$~--- среднее суммарное время ожидания 
заявок в~очереди, не приступивших к~обслуживанию в~состоянии~$i$, при 
условии~$B_m$; $\overline{W}_{\mathrm{пр}/m}$~--- среднее время простоя 
прибора в~состоянии~$i$ при условии~$B_m$. Тогда при $m\hm\leq i\hm\leq 
k\hm - 1$ получим: 
  \begin{align}
  \hspace*{-3mm}\overline{W}_{\mathrm{обсл}/m} &= \sum\limits^m_{l=1} M[W_l/B_m] 
=\sum\limits^m_{l=1} \fr{l}{m+1}\,v =\fr{mv}{2}\,;\label{p1-aga}\\
    \hspace*{-3mm}\overline{W}_{\mathrm{необсл}/m} &= [(i+1)-(m+1)]v =(i-m)v\,.\label{p2-aga}
  \end{align}
  
  Выполнение события~$B_m$ при $m\hm>i$ означает завершение 
обслуживания всех заявок в~очереди и~по\-сле\-ду\-ющее простаивание прибора до 
момента выхода системы из состояния~$i$, поэтому при $m\hm> i$, $1\hm\leq 
i\hm\leq k\hm-1$ верно равенство:
  \begin{equation*}
  \overline{W}_{\mathrm{обсл}/m} = \sum\limits^i_{l=1} M[W_l/B_m] = 
\sum\limits^i_{l=1} \fr{l}{m+1}\,v =\fr{i(i+1)v}{2(m+1)}\,.
  %\label{p3-aga}
  \end{equation*}
При $m>i$, $i\hm\leq k\hm-1$ и~условии~$B_m$ для времени простоя верно 
соотношение:

\pagebreak

\noindent
\begin{multline}
\overline{W}_{\mathrm{пр}/m} = v-M[W_{i+1}/B_m] =v- \fr{i+1}{m+1}\,v 
={}\\
{}=\fr{(m-i)v}{m+1}\,.
\label{p4-aga}
\end{multline}
Для состояний $i\hm=k$  и~$i\hm= k\hm-1$ значения параметров 
$\overline{W}_{\mathrm{обсл}/m}$, $\overline{W}_{\mathrm{необсл}/m}$ 
и~$\overline{W}_{\mathrm{пр}/m}$ совпадают.

  Обозначим через $q_i^k(v)$ величину суммарного дохода системы 
в~состоянии~$i$ при условии, что время пребывания в~состоянии~$i$ 
равно~$v$. Так как вероятность события~$B_m$ равна $((\mu v)^m/m!) e^{-\mu 
v}$, из формулы полной вероятности и~из~(\ref{p1-aga})--(\ref{p4-aga}) 
получим для~$q_i^k(v)$ при $i\hm\leq k\hm-1$ выражение вида:

\noindent
  \begin{multline*}
  q_i^k(v) =C_0-C_4v -{}\\
  {}-\sum\limits^i_{m=0} \fr{(\mu v)^m}{m!}\,e^{-\mu v} \left( 
C_2\overline{W}_{\mathrm{обсл}/m} 
+C_2\overline{W}_{\mathrm{необсл}/m}\right) -{}\\
  {}-
   \sum\limits^\infty_{m=i+1} \fr{(\mu v)^m}{m!}\,e^{-\mu v} \left( C_2\overline{W}_{\mathrm{обсл}/m} 
+C_3 \overline{W}_{\mathrm{пр}/m}\right) ={}\\
{}=C_0-C_4 v-    C_2\sum\limits^i_{m=0} 
\fr{(\mu v)^m}{m!}\,e^{-\mu v} \left[ \fr{mv}{2}+(i-m)v\right] -{}\\
{}- C_2 
\sum\limits^\infty_{m=i+1} \fr{(\mu v)^m}{m!}\,e^{-\mu v} \fr{i(i+1)v}{2(m+1)}-{}\\
   {}-
   C_3\sum\limits^\infty_{m=i+1} \fr{(\mu v)^m}{m!}\,e^{-\mu v} 
   \fr{(m-i)v}{m+1} ={}\\
   {}= C_0 - C_4 v +\fr{C_2}{2} 
\sum\limits^i_{m=0} \fr{\mu^m v^{m+1}}{(m-1)!}\,e^{-\mu v}-{}\\
   {}-
   C_2 i \sum\limits^i_{m=0}\! \!\fr{\mu^m v^{m+1}}{m!}\,e^{-\mu v} -
   C_2 \,
\fr{i(i+1)}{2}\!\!\sum\limits^\infty_{m=i+1}\! \!\fr{\mu^m v^{m+1}}{(m+1)!}\,e^{-\mu v} -{}\\
{}-C_3 
\sum\limits^\infty_{m=i+1} \fr{\mu^m v^{m+1}}{(m+1)!}\,e^{-\mu v} (m-i)\,.
   \end{multline*}
  Следовательно, 
  
  \noindent
  \begin{multline*}
  q_i^k=\int\limits_0^\infty q_i^k(v)\,dH(v)= C_0-C_4v+{}\\
  {}+\fr{C_2}{2\mu}\sum\limits^i_{m=0} m(m+1) \int\limits_0^\infty 
\fr{\mu^{m+1} v^{m+1}}{(m+1)!}\,e^{-\mu v}\,dH(v)-{}\\
  {}-
  \fr{C_2i}{\mu}\sum\limits^i_{m=0} (m+1) \int\limits_0^\infty \fr{\mu^{m+1} 
v^{m+1}}{(m+1)!}\,e^{-\mu v}\,dH(v)-{}\\
  {}-
  \fr{C_2 i (i+1)}{2\mu} \sum\limits^\infty_{m=i+1} \int\limits_0^\infty 
\fr{\mu^{m+1} v^{m+1}}{(m+1)!}\,e^{-\mu v}\,dH(v)-{}\\
  {}-
  \fr{C_3}{\mu}\sum\limits^\infty_{m=i+1} m \int\limits_0^\infty  \fr{\mu^{m+1} 
v^{m+1}}{(m+1)!}\,e^{-\mu v}\,dH(v) +{}\\
{}+
  \fr{C_3 i}{\mu} \sum\limits^\infty_{m=i+1} \int\limits_0^\infty \fr{\mu^{m+1} 
v^{m+1}}{(m+1)!}\,e^{-\mu v}\,dH(v)\,.
  \end{multline*}
  
 
  
  \noindent
   Отсюда и~из выражения для~$r_m$ в~(\ref{e4-aga}) следует~(\ref{e8-aga}) 
для $0\hm\leq i\hm\leq k\hm-1$. Так как в~состоянии $i\hm=k$ от поступившей 
заявки система не получает плату~$C_0$, сама система платит штраф~$C_1$, 
а~значения штрафа за простой, штрафа за задержку и~затрат на техническое 
обслуживание такие же, как в~состоянии $i\hm= k\hm-1$, то в~(\ref{e8-aga}) 
равенство для~$q_k^k$ также верно. 
  
  \vspace{2pt}
  
  \noindent
  Д\,о\,к\,а\,з\,а\,т\,е\,л\,ь\,с\,т\,в\,о\ \ леммы~2. Для $0\hm\leq i\hm\leq 
k\hm-2$, проведя преобразования, из~(\ref{e8-aga}) получим~(\ref{e9-aga}):

\vspace*{-6pt}

\noindent
  \begin{multline*}
  q^k_{i+1}-q_i^k= \fr{C_2}{2\mu}\sum\limits^{i+1}_{m=1} m(m-1) r_m +{}\\
  {}+
  \fr{C_2}{2\mu} (i+1) (i+2) r_{i+2} - \fr{C_2 (i+1)}
  {\mu} \sum\limits_{m=1}^{i+1} mr_m-{}\\
  {}- 
   \fr{C_2}{\mu} (i+1)(i+2) r_{i+2} -\fr{C_2(i+1)(i+2)}{2\mu} 
   \sum\limits^\infty_{m=i+3} \! \! r_m -{}\\
   {}-
   \fr{C_3}{\mu} \sum\limits^\infty_{m=i+3} (m-i-2) r_m-
  \fr{C_2}{2\mu} \sum\limits_{m=1}^{i+1} m(m-1) r_m +{}\\
  {}+\fr{C_2i}{\mu} 
\sum\limits_{m=1}^{i+1} mr_m +\fr{C_2(i+1)i}{2\mu} 
\sum\limits^\infty_{m=i+3} r_m + \fr{C_2}{2\mu}(i+1)ir_{i+2}+{}\\
  {}+
  \fr{C_3}{\mu}\sum\limits^\infty_{m=i+3} (m-i-1) 
r_m+\fr{C_3}{\mu}\,r_{i+2}={}\\
  {}=
  -\fr{C_2}{\mu} (i+1) r_{i+2} -\fr{C_2}{\mu} \sum\limits_{m=1}^{i+1} mr_m -
\fr{C_2(i+1)}{\mu}\sum\limits^\infty_{m=i+3} \!\! r_m+{}\\
  {}+
  \fr{C_3}{\mu}\sum\limits^\infty_{m=i+3} r_m +\fr{C_3}{\mu}\,r_{i+2}= 
\fr{C_3}{\mu}\sum\limits^\infty_{m=i+2} r_m -{}\\
{}-\fr{C_2}{\mu} 
\sum\limits_{m=1}^{i+1} mr_m -\fr{C_2(i+1)}{\mu} \sum\limits^\infty_{m=i+2} 
r_m\,.
  \end{multline*}
  
  Докажем~(\ref{e10-aga}). Подставив вместо~$\pi_i^k$, $\pi_{i+1}^{k+1}$ 
соответствующие выражения из~(\ref{e7-aga}), имеем: 
  \begin{equation}
  \pi_j^k-\pi_{j+1}^{k+1} =\fr{R_j^k}{1+\sum\nolimits_{i=0}^{k-1} R_i^k}- 
\fr{R_{j+1}^{k+1}}{1+\sum\nolimits^k_{i=0} R_i^{k+1}}\,.
  \label{p5-aga}
  \end{equation}
  
  Из~(\ref{e6-aga}) следует $R_j^k\hm= R_{j+1}^{k+1}$, $0\hm\leq j\hm\leq k$. 
Тогда~(\ref{p5-aga}) приводится к~виду:

\noindent
\begin{multline*}
  \pi_j^k -\pi_{j+1}^{k+1} =\fr{R_j^k R_0^{k+1}} {\left(1+ \!
\sum\nolimits_{i=0}^{k-1} R_i^k\right) 
\left(1+\!\sum\nolimits^k_{i=0} R_i^{k+1}\right)} ={}\\
{}=  
\pi_j^k \pi_0^{k+1}\,.
  \end{multline*}
  
  \vspace*{-1pt}
  
  \noindent
  Д\,о\,к\,а\,з\,а\,т\,е\,л\,ь\,с\,т\,в\,о\ \ леммы~3. Из  
формулы~(\ref{e1-aga}) следует

\vspace*{-2pt}

\noindent
  \begin{multline*}
  g^k-g^{k+1} =\sum\limits^k_{i=0} \pi_i^k q_i^k -\sum\limits_{i=0}^{k+1} 
\pi_i^{k+1} q_i^{k+1}={}\\
  {}=\sum\limits^k_{i=0} \pi_i^k q_i^k -\sum\limits_{i=1}^{k+1} \pi_i^{k+1} 
q_i^{k+1} -\pi_0^{k+1} q_0^{k+1}\,.
  \end{multline*}
  
  Введем для краткости изложения обозначение:
   $$
   \Delta_i^k = q_i^k - q^k_{i-1}\,,\quad 1\leq i\leq k\,.
   $$
   
   Из~(\ref{e8-aga}) и~(\ref{e9-aga}) имеем:

\noindent
  \begin{align*}
  \Delta_i^k &= \fr{C_3}{\mu} \sum\limits^\infty_{m=i+1} \! r_m -
\fr{C_2}{\mu}\sum\limits^i_{m=1} mr_m -\fr{C_2i}{\mu} 
\sum\limits^\infty_{m=i+1} \!r_m\,,\\ 
&\hspace*{48mm} 1\leq i\leq k-1\,;\\
  \Delta_k^k &= -C_1-C_0\,.
  \end{align*}
Если обратить внимание на правую часть формулы~(\ref{e8-aga}), то заметим, 
что $q_i^{k+1} \hm= q_i^k$ для $0\hm\leq i\hm\leq k\hm-1$. Поэтому $\Delta_i^k 
\hm= \Delta_i^{k+1}$ при $1\hm\leq i\hm\leq k\hm-1$. Кроме того,   
из~(\ref{e8-aga}) и~(\ref{e9-aga}) следует:
\begin{multline*}
q_{k+1}^{k+1} = q_k^{k+1} -C_1-C_0= q_{k-1}^{k+1} +\Delta_k^{k+1} -C_1-
C_0={}\\
{}= q_{k-1}^k +\Delta_k^{k+1} -C_1-C_0= q_k^k+\Delta_k^{k+1}\,.
\end{multline*}
  
  Подставив вместо $q_i^{k+1}$ сумму $q_{i-1}^k\hm+ \Delta_i^{k+1}$ при 
$1\hm\leq i\hm\leq k$, а при $i\hm= k\hm+1$  сумму $q_k^k\hm+ 
\Delta_k^{k+1}$, получим:
  \begin{multline*}
  g^k-g^{k+1} =\sum\limits^k_{i=0}\pi_i^k q_i^k -{}\\
  {}- \sum\limits_{i=1}^{k+1} \pi_i^{k+1} q^k_{i-1} - \sum\limits^k_{i=1} 
\pi_i^{k+1} \Delta_i^{k+1} - \pi_{k+1}^{k+1} \Delta_k^{k+1} -\pi_0^{k+1} 
q_0^{k+1}\,.
  \end{multline*}
    Заменив во 
второй и~третьей суммах последнего равенства $\pi_i^{k+1}$ на 
$(1\hm-\pi_0^{k+1})\pi^k_{i-1}$ (см.\ формулу~(\ref{e10-aga})),  
получим:
  \begin{multline}
  g^k-g^{k+1} =\sum\limits^k_{i=0} \pi_i^k q_i^k -\left( 1-\pi_0^{k+1}\right) 
\sum\limits^k_{i=0} \pi_i^k q_i^k-{}\\
  {}-
   \left( 1-\pi_0^{k+1}\right)\! \left( \sum\limits^{k-1}_{i=0} \pi_i^k \Delta_{i+1}^{k+1} +\pi_k^k 
\Delta_k^{k+1}\!\right) -\pi_0^{k+1} q_0^{k+1}={}\\
   {}=
  \pi_0^{k+1} \left[ 
  \vphantom{\fr{1-\pi_0^{k+1}}{\pi_0^{k+1}} \sum\limits^{k-1}_{i=0}}
  g^k-q_0^{k+1} -{}\right.\\
  \left.{}-\fr{1-\pi_0^{k+1}}{\pi_0^{k+1}}\left( 
\sum\limits^{k-1}_{i=0} \pi_i^k \Delta_{i+1}^{k+1} +\pi_k^k 
\Delta_k^{k+1}
\right)\right]\,.
  \label{p6-aga}
  \end{multline}
  
  Из первого равенства в~(\ref{e5-aga}) следует 
  $$
  \pi_0^{k+1} =\sum\limits^k_{i=0} \pi_i^{k+1} \sum\limits^\infty_{m=i+2} r_m 
+\pi_{k+1}^{k+1} \sum\limits^\infty_{m=k+1} r_m\,.
  $$
    Отсюда и~из~(\ref{e10-aga}) имеем:
  \begin{equation}
  \pi_0^{k+1} =\fr{1-\pi_0^{k+1}}{r_0}\left[  \sum\limits_{i=0}^{k-1} \!\pi_i^k 
\sum\limits^\infty_{m=i+2}\!\!\! r_m+\pi_k^k \!\sum\limits^\infty_{m=k+1} \!\!\!r_m\right].
  \label{p7-aga}
  \end{equation}
  
  Преобразуем $\Delta_{i+1}^{k+1}$ к~виду:
  \begin{multline*}
  \Delta_{i+1}^{k+1} =\fr{C_3}{\mu} \sum\limits^\infty_{m=i+2} r_m -{}\\
  {}-
\fr{C_2}{\mu} \sum\limits_{m=1}^{i+1} mr_m -\fr{C_2(i+1)}{\mu} 
\sum\limits^\infty_{m=i+2} r_m={}\\
 {}= \fr{C_3}{\mu} \sum\limits^\infty_{m=i+2} r_m -C_2\overline{v} 
+\fr{C_2}{\mu} \sum\limits^\infty_{m=i+2} (m-i-1) r_m\,.
  \end{multline*}
 
  
  Заменив в~(\ref{p6-aga}) обозначение~$\Delta_{i+1}^{k+1}$ обратно на 
соответствующее выражение и~использовав~(\ref{p7-aga}), получим: 

\columnbreak

\noindent
  \begin{multline*}
  g^k-g^{k+1} =\pi_0^{k+1}\left\{
  \vphantom{\fr{C_3}{\mu}\sum\limits^\infty_{m=k+1}}
   g^k -q_0^{k+1} -{}\right.\\[-1pt]
  {}-
   \fr{1-\pi_0^{k+1}}{\pi_0^{k+1}}\sum\limits_{i=0}^{k-1} \pi_i^k \left[ 
\fr{C_3}{\mu}\sum\limits^\infty_{m=i+2} r_m -C_2\overline{v} +{}\right.\\[-1pt]
\left.{}+\fr{C_2}{\mu} \!\!\sum\limits^\infty_{m=i+2} \!
(m-i-1) r_m\right]-
   \fr{1-\pi_0^{k+1}}{\pi_0^{k+1}}\,\pi_k^k \left[ \fr{C_3}{\mu}\!\!
   \sum\limits^\infty_{m=k+1}\!\!\! \!r_m -{}\right.\\[-1pt]
 \left.\left.  {}-
C_2\overline{v} +\fr{C_2}{\mu} \sum\limits^\infty_{m=k+1} (m-k) r_m\right]\right\} ={}\\[-1pt]
   {}= \pi_0^{k+1} \left\{ 
   \vphantom{\fr{1}{\mu}\sum\limits^\infty_{m=k+1}}
   g^k-q_0^{k+1}-\fr{C_3}{\mu}\,r_0+{}\right.\\[-1pt]
   {}+
  C_2\fr{1-\pi_0^{k+1}}{\pi_0^{k+1}}\left[ \sum\limits_{i=0}^{k-1} \pi_i^k \left[ 
\overline{v} -\fr{1}{\mu} \sum\limits^\infty_{m=i+2} (m-i-1) r_m\right] +{}\right.\\[-1pt]
  \left.\left.{}+ \pi_k^k\left[ \overline{v} - \fr{1}{\mu}\sum\limits^\infty_{m=k+1} 
(m-k) r_m\right]\right]\right\}\,.
  \end{multline*}
  
  \vspace*{-6pt}
  
  Заменив соответствующие выражения в~правой части последнего равенства 
на их обозначения $F(k)$ и~$G(k)$, получим~(\ref{e13-aga}). 
  
%\vspace{1pt}
  
  \noindent
  Д\,о\,к\,а\,з\,а\,т\,е\,л\,ь\,с\,т\,в\,о\ \ леммы~4. Найдем среднее число 
шагов вложенной цепи Маркова в~стационарном режиме, составляющих период 
занятости (интервал между моментами окончания предыдущего и~текущего 
периодов простоя). Рассмотрим процесс перехода цепи из состояния 
в~состояние только за период занятости. Обозначим через $q_j^k(n)$ 
вероятность того, что при стратегии~$k$ в~момент окончания $n$-го шага 
($n\hm\geq 1$) периода занятости цепь окажется в~состоянии~$j$, $j\hm\geq1$, 
$Q_l^k(n)$~--- вероятность того, что при стратегии~$k$ в~момент окончания  
$n$-го шага ($n\hm\geq 1$) периода занятости состояние цепи (число заявок 
в~системе) окажется не меньше~$l$, $l\hm\geq 1$. Очевидно,

\vspace*{-8pt}

\noindent
  \begin{align*}
  q_j^k(n+1) &= 
  \begin{cases}
  0 &\!\! \mbox{при } j>n+1\ \mbox{или } j>k\,;\\
  \displaystyle \sum\limits^k_{i=\max \{1,j-1\}} 
  \!\!\!\!\!\!\!\!\!\!\!\!q_i^k(n) p^k_{ij} & \!\!\mbox{при } 
n\geq 1\,,\ 1\leq j\leq k\,,\\[-6pt]
&\hspace*{19mm} j\leq n+1\,;
  \end{cases}\\
  q_1^k(1) &= r_0\,.
  \end{align*}
  
  \vspace*{-6pt}
  
  \noindent
   После подстановки~(\ref{e4-aga}) имеем: 
  
\vspace*{-12pt}
  
  \noindent
  \begin{multline}
  q_j^k (n+1) =\sum\limits^{k-1}_{i=\max\{1,j-1\}}  \!\!\!\!\!\!\!\! q_i^k(n) r_{i-j+1} +q_k^k(n) 
r_{k-j}\,,\\[-3pt]
 1\leq j\leq k\,.
  \label{p8-aga}
  \end{multline}
  
  \vspace*{-6pt}
     
  Просуммировав равенства~(\ref{p8-aga}) для $j\hm\geq l$, $n\hm\geq 1$, 
получим:

\vspace*{-12pt}


\noindent
   \begin{multline}
    Q_l^k(n+1) =\displaystyle\sum\limits^\infty_{j=l} q_j^k(n+1) ={}\\
{}=\sum\limits_{j=l}^k 
 \left[ \sum\limits^{k-1}_{i=\max\{1,j-1\}}\!\!\!\!\!\!\!\! q_i^k(n) r_{j-
j+1} +q_k^k(n)r_{k-j}\right]={}\\
{}=
    \displaystyle \sum\limits^{k-l}_{i=0} r_i Q^k_{l+i-1}(n)\,,\enskip 2\leq l\leq k\,;
    \label{p9-aga}
    \end{multline}
    
    %\pagebreak

\noindent
  \begin{equation}
   \left.
   \begin{array}{rl}
   Q_1^k(n+1)&=\displaystyle \sum\limits^{k-1}_{i=1} r_i Q_i^k(n) +r_0 Q_1^k(n)\,,\\[6pt]
   Q_1^k(1) &=q_1^k(1)=r_0\,.
   \end{array}\!
   \right\}\!
   \label{p9-aga-1}
   \end{equation}
  
  Отметим, что $q_j^k(n)\hm=0$ при $j\hm> k$, $q_j^k(n)\hm= q_j^{k+1}(n)$ 
при $1\hm\leq n\hm\leq k$, $1\hm\leq j\hm\leq k$. Отсюда следует: 
\begin{alignat*}{2}
Q_l^k(n)&=0  &\ \mbox{при}&\ l>k\,;\\
Q_l^k(n)&= Q_l^{k+1}(n)  &\ \mbox{при}&\ 1\leq n\leq k,\enskip
1\leq l \leq k\,;\\
Q_{k+1}^{k+1}(n)&>  Q^k_{k+1}(n)=0 &\ \mbox{при}&\ n>k\,. 
\end{alignat*}
  
  Из~(\ref{p9-aga}) и~(\ref{p9-aga-1}) получим:
    \begin{multline*}
  Q_l^{k+1}(k+1) =\sum\limits_{i=0}^{k+1-l} r_i Q^{k+1}_{l+i-1}(k) 
={}\\
{}=\sum\limits_{i=0}^{k-l} r_i Q^{k+1}_{l+i-1} (k) +r_{k+1-l}Q_k^{k+1}(k)={}\\
{}=
     \sum\limits^{k-l}_{i=0} r_i Q^k_{l+i-1}(k) +r_{k+1-l} Q_k^{k+1}(k)=
    Q_l^k(k+1) +{}\\
    {}+r_{k+1-l} Q_k^{k+1}(k) >    Q_l^k(k+1)\,,\enskip
   2\leq l\leq k+1\,;
   \end{multline*}
   
   \vspace*{-12pt}
   
   \noindent
   \begin{multline*}
   Q_1^{k+1}(k+1) =\sum\limits^k_{i=1} r_i Q_i^{k+1}(k) +r_0 Q_1^{k+1}(k)={}\\
   {}=
  \sum\limits^{k-1}_{i=1} r_i Q_i^{k+1}(k) +r_0 Q_1^{k+1}(k)+ r_k Q_k^{k+1}(k) 
={}\\
  {}= \sum\limits_{i=1}^{k-1} r_i Q_i^k(k)+ r_0 Q_1^k(k) +r_k Q_k^{k+1}(k) 
={}\\
{}=Q_1^k(k+1) +r_k Q_k^{k+1}(k)> Q_1^k(k+1)\,.
  \end{multline*}

Аналогично получим, что если для некоторого $n\hm>k$ справедливы 
неравенства $Q_l^{k+1}(n)\hm> Q_l^k(n)$, $1\hm\leq l\hm\leq k\hm+1$, то 
справедливы эти неравенства и~для $n\hm+1$. Отсюда по индукции следует 
справедливость этих неравенств для любого $n\hm\geq k$. Итак, имеем:
\begin{equation}
\left.
\begin{array}{rl}
Q_l^{k+1}(n) =  Q_l^k(n) &\ \mbox{при } 1\leq n\leq k\,;\\[6pt]
Q_l^{k+1}(n) = Q_l^k(n) &\ \mbox{при } n>k,\ 1\leq l\leq k+1\,.
\end{array}
\right\}
\label{p10-aga}
\end{equation}
  
  Из~(\ref{p10-aga}) следует неравенство:
  $$
  \sum\limits^\infty_{n=1} Q_1^{k+1}(n) > \sum\limits^\infty_{n=1} Q_1^k(n)\,.
  $$
Так как $Q_1^k(n)$~--- вероятность того, что длительность периода занятости  
не меньше ($n\hm+1$) шагов, $1\hm+ \sum\nolimits^\infty_{n=1} Q_1^k(n)$~--- 
среднее значение числа шагов на протяжении периода занятости при 
стратегии~$k$, то согласно последнему неравенству  среднее значение 
длительности периода занятости при стратегии $k\hm+1$ больше, чем при 
стратегии~$k$. А~это значит, что если~$N$~--- число последовательных шагов 
вложенной цепи Маркова, $M^k(N)$~--- число тех из них, в~которые при 
стратегии~$k$  наблюдался простой, то 
$$
\pi_0^k=\lim\limits_{N\to\infty} \fr{M^k(N)}{N} \geq \lim\limits_{N\to\infty} 
\fr{M^{k+1}(N)}{N} =\pi_0^{k+1}\,.
$$
Здесь равенства следуют из закона больших чисел, так как периоды простоя 
(как и~периоды занятости)~--- независимые одинаково распределенные 
случайные величины, и~из основной предельной теоремы для марковских 
цепей~\cite{10-aga}.  Следовательно, отношение $(1-\pi_0^{k+1})/\pi_0^{k+1}$ 
в~выражении для $F(k)$ в~(\ref{e11-aga}) возрастает с~увеличением~$k$. 

Обозначим выражение в~квадратных скобках в~формуле~(\ref{e11-aga}) для 
функции $W(k)$ через~$f_k$ и~методом индукции докажем, что~$f_k$ убывает 
по $k\hm\geq 1$. Имеем:
\begin{multline*}
f_{k+1} = {}\\
{}=\sum\limits^k_{i=0} \pi_i^{k+1} \!\!\!\sum\limits^\infty_{m=i+2} (m-i-1) r_m 
+\pi_{k+1}^{k+1} \!\!\!\sum\limits^\infty_{m=k+2} \!(m-k-1)r_m={}\\
{}=
\left(1-\pi_0^{k+1}\right) \sum\limits_{i=0}^{k-1} 
\pi_i^k \!\sum\limits^\infty_{m=i+2} (m-i-2)r_m 
+{}\\
{}+\pi_0^{k+1} \!\sum\limits^\infty_{m=2}\! (m-1)r_m+
\left( 1-\pi_0^{k+1}\right) \pi_k^k \!\!\sum\limits^\infty_{m=k+2} \!(m-k-1)r_m ={}\\
{}= \left( 1- \pi_0^{k+1}\right) 
\sum\limits^{k-1}_{i=0} \pi_i^k \sum\limits^\infty_{m=i+2} (m-i-1)r_m-{}\\
{}-
\left( 1-\pi_0^{k+1}\right) \sum\limits_{i=0}^{k-1} \pi_i^k \sum\limits^\infty_{m=i+2} r_m +\pi_0^{k+1} 
\sum\limits^\infty_{m=2} (m-1)r_m+{}\\
{}+ \left( 1-\pi_0^{k+1}\right) \pi_k^k \sum\limits^\infty_{m=k+1} (m-k)r_m-{}\\
{}-
\left( 1-\pi_0^{k+1}\right) \pi_k^k \sum\limits^\infty_{m=k+1}\!\! r_m ={}\\
{}=\left( 1- \pi_0^{k+1}\right) f_k -
\left( 1-\pi_0^{k+1}\right) \sum\limits_{i=0}^{k-1} \pi_i^k \sum\limits^\infty_{m=i+2} r_m+{}\\
{}+
\pi_0^{k+1} \sum\limits^\infty_{m=2} (m-1)r_m -\left( 1-\pi_0^{k+1}\right) \pi_k^k 
\sum\limits^\infty_{m=k+1} \!\!r_m\,.
\end{multline*}

Как следует из~(\ref{p7-aga}),
$$
\sum\limits_{i=0}^{k-1} \pi_i^k \sum\limits^\infty_{m=i+2} r_m +\pi_k^k 
\sum\limits^\infty_{m=k+1} r_m = \fr{r_0 \pi_0^{k+1}}{1-\pi_0^{k+1}}\,.
$$
Использовав последнее равенство в~правой части предпоследнего равенства, 
получим:

\noindent
\begin{multline}
f_{k+1} =\left(1-\pi_0^{k+1}\right) f_k -{}\\
{}-\left( 1-\pi_0^{k+1}\right) 
\fr{r_0\pi_0^{k+1}}{1-\pi_0^{k+1}}+ \pi_0^{k+1}\sum\limits^\infty_{m=2} (m-
1)r_m={}
\\
{}=\left( 1-\pi_0^{k+1}\right) f_k - r_0 \pi_0^{k+1} +\pi_0^{k+1} 
\sum\limits^\infty_{m=2} (m-1)r_m ={}\\
{}=\left( 1-\pi_0^{k+1}\right) f_k + \pi_0^{k+1} 
a\,,
\label{p11-aga}
\end{multline}
где
$$
a=\sum\limits^\infty_{m=2} (m-1) r_m -r_0 =\mu\overline{v}-1\,.
$$
Отсюда имеем:

\pagebreak

\noindent
\begin{multline}
f_k-f_{k+1} =\pi_0^{k+1} f_k +r_0\pi_0^{k+1} -\pi_0^{k+1} 
\sum\limits^\infty_{m=2} (m-1) r_m ={}\\
{}= \pi_0^{k+1}\left( f_k-a\right)\,.
\label{p12-aga}
\end{multline}
Из выражения для $f_k$ находим:

%\vspace*{-4pt}

\noindent
$$
f_1=\sum\limits^\infty_{m=2} (m-1)r_m =\mu\overline{v} -\left( 1-r_0\right) 
=a+r_0> a\,.
$$
Следовательно, выполняется неравенство $f_1\hm> f_2$ (см.~(\ref{p12-aga})). 
Пусть $f_k\hm>a$, т.\,е.\ (см.~(\ref{p12-aga})) $f_k\hm> f_{k+1}$. Докажем, что 
тогда $f_{k+1}\hm> f_{k+2}$. Из~(\ref{p12-aga}), (\ref{p11-aga}) 
и~предположения $f_k\hm>a$ следует

\vspace*{-6pt}

\noindent
\begin{multline*}
f_{k+1} =\left( 1-\pi_0^{k+1}\right) f_k +\pi_0^{k+1} a> {}\\
{}>\left( 1-\pi_0^{k+1}\right) 
a +\pi_0^{k+1} a=a\,,
\end{multline*}
т.\,е.\ $f_{k+1}\hm> f_{k+2}$. 
  
Следовательно, по индукции следует $f_k\hm> f_{k+1}$ для любых $k\hm\geq 
1$, т.\,е.~$W(k)$ возрастает по $k\hm\geq 1$. Таким образом, в~
выражении~(\ref{e11-aga}) для~$F(k)$ сомножители $(1-
\pi_0^{k+1})/\pi_0^{k+1}$ и~$\overline{W}(k)$ возрастают по~$k$ 
и,~следовательно, $F(k)$ возрастает, а~функция~$G(k)$ убывает по~$k$ 
(см.~(\ref{e12-aga})). 

\smallskip

\noindent
  Д\,о\,к\,а\,з\,а\,т\,е\,л\,ь\,с\,т\,в\,о\ \ теоремы~1. Рассмотрим 
поведение функции~$g^k$ при последовательном увеличении значения 
$k\hm>0$. Как следует из~(\ref{e13-aga}), неравенства $g^k\hm< g^{k+1}$ 
и~$g^k\hm< G(k)$ эквивалентны. Отсюда следует, что если при значении 
$k\hm> 0$ выполняется неравенство $g^k\hm< G(k)$, то при значении $k\hm+1$ 
выполняется неравенство $g^{k+1}\hm> g^k$, и~наоборот: если при значении 
$k\hm>0$ выполняется неравенство $g^k\hm> G(k)$, то при значении $k\hm+1$ 
выполняется неравенство $g^{k+1}\hm\leq g^k$, а~при $g^k\hm= G(k)$ 
выполняется равенство $g^{k+1}\hm= g^k$. 
  
  Рассмотрим сначала случай $C_2\hm>0$. Пусть $k_1$~--- произвольное 
число, $1\hm\leq k_1 \hm< \infty$. Рассмотрим два альтернативных случая: 
(1)~$g^{k_1}\hm\geq G(k_1)$; (2)~$g^{k_1}\hm< G(k_1)$. 
  
  Пусть $g^{k_1}\geq G(k_1)$. Тогда если $g^{k_1}\hm> G(k_1)$, то 
согласно~(\ref{e13-aga}) (так как $0\hm< \pi_0^{k+1}\hm<1$) выполняется 
$g^{k_1+1}\hm> G(k_1)$, а~так как~$G(k)$ не возрастает по $k\hm>0$ 
(согласно лемме~4), выполняется $g^{k_1+1}\hm> G(k_1+1)$. Следовательно, 
$g^k\hm> G(k)$, $k\hm\geq k_1$, и~согласно рассуждениям в~начале 
доказательства последовательность $\{ g^k,\ k\hm= k_1, k_1+1, \ldots\}$ 
является убывающей, т.\,е.\ $g^k\hm< g^{k_1}$, $k\hm> k_1$, откуда следует, 
что существует $k^0\hm\leq k_1$. Если $g^{k_1}\hm= G(k_1)$, то  
из~(\ref{e13-aga}) и~леммы~4 следует либо $g^k\hm= G(k)\hm= g^{k_1}$, 
$k\hm> k_1$, либо существует $k_1\hm< k_2\hm< \infty$ такое, что $g^k\hm= 
G(k)\hm= g^{k_1}$, $k_1\hm< k\hm\leq k_2$, $g^{k_2}\hm> G(k_2)$. Тогда 
в~первом случае $k^0\hm\leq k_1$, во втором случае, как уже было показано, 
существует $k^0\hm\leq k_2$ и,~так как $g^k\hm= g^{k_1}$, $k_1\hm\leq 
k\hm\leq k_2$, то $k^0\hm\leq k_1$, т.\,е.\ при $g^{k_1}\hm\geq G(k_1)$ 
утверждение~1 в~формулировке теоремы справедливо. При этом если 
$k_1\hm=1$, то $k^0\hm=1$, т.\,е.\ выполняется утверждение~2 в~формулировке 
теоремы. Заметим также, что при $1\hm< k^0\hm< \infty$ выполняется 
утверждение~4 теоремы.
  
  Пусть $g^{k_1}<G(k_1)$. Тогда согласно~(\ref{e13-aga}) из $g^{k_1}\hm< 
G(k_1)$ следует $g^{k_1+1}\hm> g^{k_1}$ и~$g^{k_1+1}\hm< G(k_1)$ (так как 
$0\hm< \pi_0^{k+1}\hm< 1$). Так как $G(k)$ не возрастает по $k\hm> 0$, то 
либо $g^{k_1+1}\hm< G(k_1+1)$, либо $g^{k_1+1}\hm\geq G(k_1+1)$. 
Следовательно, если $\mathop{\mathrm{inf}}\nolimits_{k>0} G(k)\hm< 
\mathop{\mathrm{sup}}\nolimits_{k>0} g^k$, то существует 
$k_1\hm<k_2\hm<\infty$ такое, 
что $g^{k_2}\hm\geq G(k_2)$, $g^k\hm< G(k)$, $k_1\hm\leq k\hm\leq k_2\hm-1$. 
В~этом случае, как показано выше, существует $k^0\hm\leq k_2$, а~именно: 
$k^0\hm=k_2$ (это следует из того, что $k^0\hm\leq k_2$ и~$g^k\hm< G(k)$, 
$k_1\hm\leq k\hm\leq k_2\hm-1$), т.\,е.\ утверждение~1 теоремы~1 справедливо. 

Заметим также, что при $k^0\hm>1$ выполняется и~утверждение~4 теоремы~1.  
Если $\mathop{\mathrm{inf}}\nolimits_{k>0} G(k)\hm\geq 
\mathop{\mathrm{sup}}\nolimits_{k>0} g^k$, 
то $g^k\hm< G(k)$, $k\hm\geq k_1$, и~в~этом случае 
$k^0\hm=\infty$. 
  
  Рассмотрим теперь случай $C_2\hm=0$. Из~(\ref{e12-aga}) следует 
$G(k)\hm=const$, $k\hm>0$. Рассуждая, как в~случае $C_2\hm>0$, легко 
доказать, что последовательность $\{g^k,\ k\hm\geq 1\}$ удовлетворяет 
неравенствам $g^1\hm\geq g^2\geq\cdots \geq g^n\geq \cdots\geq const$ при 
условии $g^1\hm\geq const$ и~неравенствам $g^1\hm< g^2<\cdots$\linebreak
$\cdots < g^n<\cdots < 
const$ при условии $g^1\hm< const$. Как видим, $k^0\hm=1$ при $g^1\hm\geq 
const$ и~$k^0\hm=\infty$ при $g^1\hm< const$. Следовательно, утверждение~3 
теоремы~1 верно.
}

  \vspace*{-8pt}
  
  
{\small\frenchspacing
 {%\baselineskip=10.8pt
 \addcontentsline{toc}{section}{References}
 \begin{thebibliography}{99}
\bibitem{1-aga}
\Au{Welzl M.} Network congestion control.~--- New York, NY, USA: Wiley, 
2005. 282~p.
\bibitem{2-aga}
\Au{Irland M.} Buffer management in a~packet switch~// IEEE T.~Comput., 1978. 
Vol.~28. No.\,7. P.~328--337.
\bibitem{3-aga}
\Au{Печинкин А.\,В., Разумчик~Р.\,В.}  Время пребывания в~различных 
режимах системы обслуживания с~неординарными пуассоновскими 
входящими потоками, рекуррентным обслуживанием и~гистерезисной 
политикой~// Информационные процессы, 2015. Т.~15. №\,3. C.~324--336. 
\bibitem{4-aga}
\Au{Жерновый Ю.\,В.} Решение задач оптимального синтеза для некоторых 
марковских моделей обслуживания~// Информационные процессы, 2010. 
Т.~10. №\,3. C.~257--274. 
\bibitem{5-aga}
\Au{Коновалов М.\,Г.} Об одной задаче оптимального управ\-ле\-ния нагрузкой 
на сервер~// Информатика и~её применения, 2013. Т.~7. Вып.~4. С.~34--43.
\bibitem{6-aga}
\Au{Агаларов Я.\,М.} Пороговая стратегия ограничения доступа к~ресурсам 
в~системе массового обслуживания $M/D/1$ с~функцией штрафов за несвоевременное обслуживание 
заявок~// Информатика и~её применения, 2015. Т.~9. Вып.~3. С.~56--65.
\bibitem{7-aga}
\Au{Гришунина Ю.\,Б.} Оптимальное управление очередью в~системе 
$M/G/1/\infty$ с~возможностью ограничения приема заявок~// Автоматика 
и~телемеханика, 2015. №\,3. С.~79--93. 
\bibitem{8-aga}
\Au{Агаларов Я.\,М.} Максимизация среднего стационарного дохода системы массового обслуживания 
типа $M/G/1$~// Информатика и~её применения, 2017. Т.~11. Вып.~2.  
С.~25--32.

  
\bibitem{10-aga}
\Au{Карлин С.} Основы теории случайных процессов~/ Пер. с~англ.~--- М.: 
Мир, 1971. 536~с. (\Au{Karlin~S.} A~first course in stochastic processes.~---  
New York and London: Academic Press, 1968. 502~p.) 

\bibitem{9-aga}
\Au{Бочаров П.\,П., Печинкин~А.\,В.} Теория массового обслуживания.~--- 
М.: РУДН, 1995. 529~с.

 \end{thebibliography}

 }
 }

\end{multicols}

\vspace*{-10pt}

\hfill{\small\textit{Поступила в~редакцию 16.06.17}}

%\vspace*{8pt}

\newpage

\vspace*{-24pt}

%\hrule

%\vspace*{2pt}

%\hrule

%\vspace*{8pt}


\def\tit{ABOUT THE PROBLEM OF~PROFIT MAXIMIZATION IN~$G/M/1$ QUEUING SYSTEMS 
WITH~THRESHOLD CONTROL OF~THE~QUEUE}

\def\titkol{About the problem of~profit maximization in~$G/M/1$ queuing systems 
with~threshold control of~the~queue}

\def\aut{Ya.\,M.~Agalarov and V.\,S.~Shorgin}

\def\autkol{Ya.\,M.~Agalarov and V.\,S.~Shorgin}

\titel{\tit}{\aut}{\autkol}{\titkol}

\vspace*{-9pt}


\noindent
Institute of Informatics Problems, Federal Research Center ``Computer Science 
and Control'' of the Russian Academy of Sciences, 44-2~Vavilov Str., Moscow 
119333, Russian Federation



\def\leftfootline{\small{\textbf{\thepage}
\hfill INFORMATIKA I EE PRIMENENIYA~--- INFORMATICS AND
APPLICATIONS\ \ \ 2017\ \ \ volume~11\ \ \ issue\ 4}
}%
 \def\rightfootline{\small{INFORMATIKA I EE PRIMENENIYA~---
INFORMATICS AND APPLICATIONS\ \ \ 2017\ \ \ volume~11\ \ \ issue\ 4
\hfill \textbf{\thepage}}}

\vspace*{3pt}


\Abste{The problem of maximizing the average profit per time in $G/M/1$ queuing
systems is 
considered on the set of stationary access restriction threshold strategies with one 
``switch point.'' Profit is defined as the following measures: service fee; hardware 
maintenance fee; fine for service delay; fine for unhandled requests; and
fine for system 
idle. The authors formulated the necessary and sufficient conditions for optimality of 
the finite threshold value. The authors developed a method of sequential descent to 
the optimal threshold. The authors proposed an algorithm for calculating the optimal 
threshold value and the corresponding value of the objective function.}

\KWE{queuing system; threshold strategy; optimization}

  \DOI{10.14357/19922264170407} 

%\vspace*{-12pt}

\Ack
\noindent
The work was partly supported by
the Russian Foundation for Basic Research (project 15-07-03406).



%\vspace*{3pt}

  \begin{multicols}{2}

\renewcommand{\bibname}{\protect\rmfamily References}
%\renewcommand{\bibname}{\large\protect\rm References}

{\small\frenchspacing
 {%\baselineskip=10.8pt
 \addcontentsline{toc}{section}{References}
 \begin{thebibliography}{99}
\bibitem{1-aga-1}
\Aue{Welzl, M.} 2005. \textit{Network congestion control.} New York, NY: 
Wiley. 282~p.
\bibitem{2-aga-1}
\Aue{Irland, M.} 1978. Buffer management in a~packet switch. \textit{IEEE T.~Comput.} 28(7):328--337.  
\bibitem{3-aga-1}
\Aue{Pechinkin, A.\,V., and R.\,V.~Razumchik.} 2015. Vremya prebyvaniya 
v~razlichnykh rezhimakh sistemy obsluzhivaniya s~neordinarnymi 
puassonovskimi vkhodyashchimi potokami, rekurrentnym obsluzhivaniem 
i~gisterezisnoy politikoy [First passage times between modes in the queueing 
system with batch Poisson arrivals, general service and hysteresis policy]. 
\textit{Informatsionnye protsessy} [Information Processses] 15(3):324--336.
\bibitem{4-aga-1}
\Aue{Zhernovyy, Yu.\,V.} 2010. Reshenie zadach optimal'nogo sinteza dlya 
nekotorykh markovskikh modeley obsluzhivaniya [Solution of optimum synthesis 
problem for some Markov models of service]. \textit{Informatsionnye protsessy} 
[Information Processses] 10(3):257--274.
\bibitem{5-aga-1}
\Aue{Konovalov, M.\,G.} 2013. Ob odnoy zadache optimal'nogo upravleniya 
nagruzkoy na server [About one task of over-\linebreak\vspace*{-12pt}

\columnbreak

\noindent
load control]. \textit{Informatika i~ee 
Primeneniya~--- Inform. Appl.} 7(4):34--43.
\bibitem{6-aga-1}
\Aue{Agalarov, Ya.\,M.} 2015. Porogovaya strategiya ogranicheniya dostupa 
k~resursam v~sisteme massovogo obsluzhivaniya $M/D/1$ s~funktsiey shtrafov za nesvoevremennoe 
obsluzhivanie zayavok [The threshold strategy for restricting access in the 
$M/D/1$ queueing system with penalty function for late service]. 
\textit{Informatika i~ee Primeneniya~--- Inform. Appl.} 9(3):56--65. 
\bibitem{7-aga-1}
\Aue{Grishunina, Yu.\,B.} 2015. Optimal control 
of queue in the $M/G/1/\infty$ system with possibility of customer admission 
restriction. \textit{Automat. Rem. Contr.}   76(3):433--445.
\bibitem{8-aga-1}
\Aue{Agalarov, Ya.\,M.} 2017. Maksimizatsiya srednego sta\-tsi\-o\-nar\-no\-go dokhoda 
sistemy massovogo obsluzhivaniya tipa $M/G/1$ [Maximization of average stationary profit in $M/G/1$ 
queuing system]. \textit{Informatika i~ee Primeneniya~--- Inform. Appl.}  
11(2):25--32.  


\bibitem{10-aga-1}
\Aue{Karlin, S.} 1968. \textit{A~first course in stochastic processes}. 
New York\,--\,London: Academic Press. 502~p.
\bibitem{9-aga-1}
\Aue{Bocharov, P.\,P., and A.\,V.~Pechinkin.} 1995. \textit{Teoriya massovogo 
obsluzhivaniya} [Queueing theory]. Moscow: RUDN. 529~p.
\end{thebibliography}

 }
 }

\end{multicols}

\vspace*{-6pt}

\hfill{\small\textit{Received June 16, 2017}}

%\vspace*{-10pt}
   
  
  \Contrl
  
   \noindent
   \textbf{Agalarov Yaver M.} (b.\ 1952)~--- Candidate of Science (PhD) in 
technology, associate professor; leading scientist, Institute of 
Informatics Problems, Federal Research Center ``Computer Science and 
Control'' of the Russian Academy of Sciences, 44-2~Vavilov Str., Moscow 
119333, Russian Federation; \mbox{agglar@yandex.ru}

\vspace*{3pt}


   \noindent
   \textbf{Shorgin Vsevolod S.} (b.\ 1978)~--- 
   Candidate of Science (PhD) in technology, senior scientist, 
   Institute of Informatics Problems, Federal Research Center 
   ``Computer Science and Control'' of the Russian Academy of Sciences, 
   44-2~Vavilov Str., Moscow 119333, Russian Federation; \mbox{VShorgin@ipiran.ru}



\label{end\stat}


\renewcommand{\bibname}{\protect\rm Литература} 
   