%\renewcommand{\endproof}{\hfill$\Box$}
\def\stat{korolev}

\def\tit{НЕКОТОРЫЕ СВОЙСТВА РАСПРЕДЕЛЕНИЯ МИТТАГ-ЛЕФФЛЕРА И СВЯЗАННЫХ С НИМ
ПРОЦЕССОВ$^*$}

\def\titkol{Некоторые свойства распределения Миттаг-Леффлера и~связанных с~ним
процессов}

\def\aut{В.\,Ю.~Королев$^1$}

\def\autkol{В.\,Ю.~Королев}

\titel{\tit}{\aut}{\autkol}{\titkol}

\index{Королев В.\,Ю.}
\index{Korolev V.\,Yu.}



{\renewcommand{\thefootnote}{\fnsymbol{footnote}} \footnotetext[1]
{Работа выполнена при частичной финансовой поддержке
РФФИ (проект 17-07-00717).}}


\renewcommand{\thefootnote}{\arabic{footnote}}
\footnotetext[1]{Факультет вычислительной математики и~кибернетики
Московского государственного университета им.\ М.\,В. Ломоносова; 
Институт проб\-лем информатики Федерального исследовательского центра 
<<Информатика и~управ\-ле\-ние>> 
Российской академии наук; Университет Дианьзи города Ханчжоу, Китай,
\mbox{vkorolev@cs.msu.ru}}

\vspace*{12pt}




\Abst{Данная статья содержит обзор некоторых свойств
распределения Мит\-таг-Леф\-фле\-ра. Основное внимание уделено возможности
представления этого распределения в~виде смешанного показательного
закона. Также обсуждается возможность представления распределения
Мит\-таг-Леф\-фле\-ра в~виде масштабных смесей полунормальных или
равномерных распределений. Обсуждается возможность использования
распределения Мит\-таг-Леф\-фле\-ра в~качестве асимптотической
аппроксимации для распределений некоторых статистик, построенных по
выборкам случайного объема. Описан новый двухэтапный сеточный метод
оценивания параметра распределения Мит\-таг-Леф\-фле\-ра, использующий
представление этого распределения в~виде смешанного показательного
закона. Рассмотрены два возможных способа распространения понятия
распределения Мит\-таг-Леф\-фле\-ра на случайные процессы пуассоновского
типа. Первый из этих способов приводит к~специальному смешанному
пуассоновскому процессу, другой способ ведет к~специальному процессу
восстановления~--- дважды стохастическому пуассоновскому процессу
(процессу Кокса). В предельных теоремах для случайно остановленных
случайных блужданий в~обоих этих случаях в~качестве предельных
конечномерных распределений возникают дробно-устойчивые
распределения, представимые в~виде масштабных смесей нормальных
законов с~разными смешивающими распределениями.}

\KW{распределение Миттаг-Леффлера; распределение
Линника; устойчивое распределение; распределение Вейбулла;
показательное распределение; смешанный пуассоновский процесс;
процесс восстановления; асимптотическая аппроксимация}

  \DOI{10.14357/19922264170404} 
  
  \vspace*{6pt}


\vskip 10pt plus 9pt minus 6pt

\thispagestyle{headings}

\begin{multicols}{2}

\label{st\stat}
  

\section{Введение}

Данная статья содержит обзор некоторых свойств распределения
Мит\-таг-Леф\-фле\-ра. Это рас\-пределение часто применяется как
<<тяжело-\linebreak хвостая>> модель статистических закономерностей, наблюдаемых
во многих физических явлениях, опи\-сы\-ва\-емых в~терминах редеющих
процессов восстановления, в~частности аномальной диффузии 
и~релаксации~[1--3], а~также в~астрофизике~\cite{Joseetal2010},
экономике и~финансовой математике~\cite{MeerschaertScalas2006,
Scalas2006}. 

Основное внимание уделено возможности представления
распределения Мит\-таг-Леф\-фле\-ра в~виде смешанного показательного
закона. Это пред\-став\-ле\-ние позволяет существенно облегчить
ис\-сле\-дование аналитических свойств распределения Мит\-таг-Леф\-фле\-ра, 
в~частности заметно упрощает вычисление его моментов. 

Также будет
показано, что распределение Мит\-таг-Леф\-фле\-ра может быть представлено
в~виде масштабных смесей полунормальных или равномерных
распределений. Разнообразие таких пред\-став\-ле\-ний помогает выбрать
подходящие <<основное>> и~априорное распределения при использовании
распределения Мит\-таг-Леф\-фле\-ра в~качестве байесовской модели.

Обычно в~вероятностных источниках распределение Мит\-таг-Леф\-фле\-ра
упоминается в~качестве примера геометрически устойчивого закона, т.\,е.\ 
предельного для геометрических случайных сумм независимых
одинаково распределенных (о.р.)\ слагаемых с~бесконечной дисперсией.
Однако схема геометрического суммирования является далеко не
единственной предельной схемой, в~которой распределение
Мит\-таг-Леф\-фле\-ра может быть предельным.

Как правило, распределение Мит\-таг-Леф\-фле\-ра рассматривается 
<<в~тандеме>> с~распределением Линника, поскольку вид характеристической
функции (х.ф.)\ последнего формально аналогичен виду преобразования
Лап\-ла\-са--Стилть\-еса (п.~Л.--С.) распределения Мит\-таг-Леф\-фле\-ра. На
самом деле связь этих двух законов оказывается очень глубокой. Ниже
будут приведены некоторые результаты, связывающие эти распределения
между собой, а~также с~устойчивыми законами. В~частности, недавно
показано, что распределение Мит\-таг-Леф\-фле\-ра играет роль смешивающего
закона в~пред\-став\-ле\-нии распределения Линника в~виде масштабной смеси
нормальных законов. Такое пред\-став\-ле\-ние позволило доказать аналог
центральной предельной теоремы для сумм случайного числа независимых
случайных величин (с.в.)\
с~конечными дисперсиями, в~котором предельным
является распределение Линника~\cite{KorolevZeifman2017}. При этом
сходимость распределений нормированных индексов к~распределению
Мит\-таг-Леф\-фле\-ра является \textit{необходимым и~достаточным условием}
сходимости распределений упомянутых случайных сумм к~распределению
Лин\-ника. 
{\looseness=1

}

В~данной статье обсуждается возможность использования
распределения Мит\-таг-Леф\-фле\-ра в~качестве асимптотической
аппроксимации для распределений некоторых статистик, построенных по
выборкам случайного объема, в~частности экстремальных порядковых
статистик, максимальных случайных сумм независимых 
с.в.\ с~\textit{конечными дисперсиями}, удовлетворяющих условию
Линдеберга, и~абсолютных величин таких сумм.

В связи с~тем, что плотность распределения Мит\-таг-Леф\-фле\-ра не
допускает явного пред\-став\-ле\-ния в~терминах элементарных функций,
метод максимального правдоподобия оказывается малопригодным для
статистического оценивания параметра распределения Мит\-таг-Леф\-фле\-ра.
С этой целью в~некоторых работах предлагается использовать
специальные модификации метода моментов. В~данной работе будет
описан альтернативный двухэтапный сеточный метод, использующий
представление распределения Мит\-таг-Леф\-фле\-ра в~виде смешанного
показательного закона.

Ниже также будут рассмотрены два возможных способа распространения
понятия распределения Мит\-таг-Леф\-фле\-ра на случайные процессы
пуассоновского типа. Первый из этих способов приводит к~специальному
смешанному пуассоновскому процессу (со скалярным случайным
преобразованием времени), другой способ ведет к~специальному
процессу восстановления~--- дважды стохастическому пуассоновскому
процессу (процессу Кокса). В~предельных теоремах для случайно
остановленных случайных блужданий в~обоих этих случаях в~качестве
предельных возникают подчиненные винеровские процессы с~разными
субординаторами. В~обоих случаях конечномерные распределения
оказываются дроб\-но-устой\-чи\-вы\-ми~\cite{FSD2001}, но с~разными
па\-ра\-мет\-рами.

\section{Распределения Миттаг-Леффлера и~Линника}

\subsection{Распределение Миттаг-Леффлера}

Распределение Миттаг-Леф\-фле\-ра~--- это распределение неотрицательной
с.в.~$M_{\nu}$, соответствующее п.~Л.--С.:
\begin{equation}
\psi_{\nu}(s)\equiv {\sf E}e^{-sM_{\nu}}=\fr{1}{1+\lambda
s^{\nu}}\,,\enskip s\geqslant0\,,
\label{e1-kor}
\end{equation}
где $\lambda>0$, $0\hm<\nu\hm\leqslant1$. Для краткости ниже будет
рассматриваться стандартный случай $\lambda\hm=1$.

Происхождение термина \textit{распределение Мит\-таг-Леф\-фле\-ра} связано 
с~тем, что плотность распределения, соответствующая п.~Л.--С.~(1),
имеет вид:
\begin{multline}
f_{\nu}^{M}(x)=\fr{1}{x^{1-\nu}}\sum\limits_{n=0}^{\infty}\fr{(-1)^nx^{\nu
n}}{\Gamma(\nu n+1)}=-\fr{d}{dx}E_{\nu}(-x^{\nu})\,,\\
x\geqslant0\,,
\label{e2-kor}
\end{multline}
где $E_{\nu}(z)$~--- функция Мит\-таг-Леф\-фле\-ра индекса~$\nu$,
определяемая как степенной ряд
$$
E_{\nu}(z)=\sum\limits_{n=0}^{\infty}\fr{z^n}{\Gamma(\nu n+1)}\,,\enskip
 \nu>0\,,\ z\in\mathbb{Z}\,.
$$
Здесь $\Gamma(s)$~--- эйлерова гамма-функ\-ция,
$\Gamma(s)\hm=\int\nolimits_{0}^{\infty}z^{s-1}e^{-z}\,dz$, $s\hm>0$. Функция
распределения (ф.р.), соответствующая плотности~(2), будет
обозначаться~$F_{\nu}^{M}(x)$.

При $\nu=1$ распределение Мит\-таг-Леф\-фле\-ра превращается в~стандартное
показательное распределение, т.\,е.\ $F_1^{M}(x)\hm=
[1-e^{-x}]\mathbf{1}(x\hm\geqslant 0)$, $x\hm\in\mathbb{R}$ (здесь и~далее
символ~$\mathbf{1}(C)$ обозначает индикатор множества~$C$). Но при
$0\hm<\nu\hm<1$ плотность распределения Мит\-таг-Леф\-фле\-ра имеет тяжелый
хвост, убывающий степенн$\acute{\mbox{ы}}$м образом:
$$
f_\nu^{M}(x)\sim \fr{\sin(\nu\pi)\Gamma(\nu+1)}{\pi x^{\nu+1}}
$$
при $x\to\infty$ (см., например,~\cite{Kilbas2014}).

Хорошо известно, что распределение Миттаг-Леффлера является
геометрически устойчивым. Это означает, что если $X_1,X_2,\ldots$~---
независимые о.р.\ с.в., а~$V_p$~--- с.в., независимая от
$X_1,X_2,\ldots$ и~име\-ющая геометрическое распределение
\begin{multline}
{\sf P}(V_p=n)=p(1-p)^{n-1}\,,\\
n=1,2,\ldots,\quad
p\in(0,1)\,,
\label{e3-kor}
\end{multline}
то для каждого $p\hm\in(0,1)$ существует число $a_p\hm>0$ такое, что
$a_p(X_1+\cdots+X_{V_p})\Longrightarrow M_{\nu}$ при $p\hm\to 0$ (см.,
например,~\cite{Bunge1996} или~\cite{KlebanovRachev1996}). Здесь 
и~далее символ~$\Longrightarrow$ обозначает сходимость по
рас\-пре\-де\-ле\-нию.
{\looseness=1

}

Более того, еще в~1965~г.\ И.\,Н.~Коваленко~\cite{Kovalenko1965}
показал, что распределения с~п.~Л.--С.~(1) \textit{и~только они} могут
быть предельными для нормированных геометрических случайных сумм
вида $a_p(X_1+\cdots+X_{V_p})$ при $p\hm\to0$, где $X_1,X_2,\ldots$~---
независимые о.р.\ с.в., а~$V_p$~--- с.в.\ с~геометрическим
распределением~(3), при каждом $p\hm\in(0,1)$ независимая от
$X_1,X_2,\ldots$ Доказательство этого результата затем было
воспроизведено в~книгах~\cite{GnedenkoKovalenko1968,
GnedenkoKovalenko1989, GnedenkoKorolev1996}, изданных на
английском языке. В~этих книгах распределения с~п.~Л.--С.~(1)
названы \textit{распределениями класса~$\mathcal{K}$} в~честь И.\,Н.~Коваленко.

Двадцать пять лет спустя указанное предельное свойство распределений
с~п.~Л.--С.~(1) было переоткрыто А.~Пиллаи~\cite{Pillai1989,
Pillai1990}, который и~предложил для них термин \textit{распределение
Мит\-таг-Леф\-фле\-ра}. Возможно, из-за того, что работы~[11--13] были
мало известны на Западе, термин \textit{распределения класса~$\mathcal{K}$} 
не прижился, тогда как термин \textit{распределение
Мит\-таг-Леф\-фле\-ра} стал общепринятым.
{\looseness=1

}

\subsection{Распределение Линника}

В~1953~г.\ Ю.\,В.~Линник~\cite{Linnik1953} ввел класс сим\-мет\-рич\-ных
распределений, определяемых х.ф.:
$$
\mathfrak{f}^L_{\alpha}(t)=\fr{1}{1+|t|^{\alpha}}\,,\enskip
t\in\mathbb{R}\,,
$$
где $\alpha\in(0,2]$. Позднее распределения этого класса стали
называть \textit{распределениями Линника}~\cite{Kotz2001} или 
\textit{$\alpha$-рас\-пре\-де\-ле\-ни\-ями Лапласа}~\cite{Pillai1985}. В~данной работе
используется первый термин, ставший общепринятым. При $\alpha\hm=2$
распределение Линника превращается в~распределение Лапласа,
соответствующее плот\-ности
$f^{\Lambda}(x)\hm=(1/2)e^{-|x|}$, $x\hm\in\mathbb{R}$.

Распределения Линника обладают многими интересными аналитическими
свойствами. В частности, они унимодальны~\cite{Laha1961},
безгранично делимы~\cite{Devroye1990}, обладают плотностями 
с~бесконечным пиком при $\alpha\hm\leqslant1$ \cite{Devroye1990} и~т.\,п. 
В~работах~\cite{KotzOstrovskiiHayfavi1995a,
KotzOstrovskiiHayfavi1995b} можно найти детальное исследование
асимптотических свойств плотности распределения Линника. Тем не
менее чаще всего распределения Линника упоминаются в~качестве
примеров геометрически устойчивых распределений.

Случайная величина с~распределением Линника с~параметром $\alpha\hm\in(0,2]$ будет
обозначаться~$L_{\alpha}$.

\section{Предварительные сведения}

Хотя в~данной статье рассматриваются свойства {\it распределений}
вероятности, соответствующие результаты будут сформулированы 
в~терминах соответствующих с.в.\ в~предположении, что все они заданы
на одном вероятностном пространстве $(\Omega,\,\mathfrak{A},\,{\sf
P})$.

Символ~$\eqd$ будет обозначать совпадение распределений связанных им~с.в.

Случайная величина со стандартной нормальной ф.р.~$\Phi(x)$ будет обозначаться~$X$:
$$
{\sf P}(X<x)=\Phi(x)=\fr{1}{\sqrt{2\pi}}\int\limits_{-\infty}^{x}e^{-z^2/2}dz\,,\enskip
x\in\mathbb{R}\,.
$$
Пусть $\Psi(x)$, $x\in\mathbb{R}$,~--- ф.р.\ максимума стандартного
винеровского процесса на единичном отрезке,
$\Psi(x)\hm=2\Phi(\max\{0,x\})\hm-1$, $x\hm\in\mathbb{R}$. Легко
видеть, что $\Psi(x)\hm={\sf P}(|X|<x)$. Поэтому иногда говорят, что ф.р.~$\Psi(x)$ 
определяет {\it полунормальное} или {\it сложенное~$($folded$)$ нормальное} 
распределение.

Для $r>0$, $\mu\hm>0$ пусть $\Gamma_{r,\mu}$~--- с.в.\ 
с~гам\-ма-рас\-пре\-де\-ле\-ни\-ем, определяемым плотностью
$$
f_{r,\mu}(x)=\fr{\mu^r}{\Gamma(r)}\,x^{r-1}e^{-\mu x}\,,\enskip x\geqslant 0\,.
$$
Пусть $-\infty\hm<a\hm< b\hm< \infty$. Случайная величина с~равномерным
распределением на $[a,b]$ будет обозначаться $U_{[a,b]}$.

Пусть $\gamma>0$. Распределение с.в.~$W_{\gamma}$: $ {\sf P}\left(W_{\gamma}<x\right)
\hm=\left[1\hm-e^{-x^{\gamma}}\right]\mathbf{1}(x\hm\geqslant
0)$ называется {\it распределением Вейбулла} с~параметром формы~$\gamma$.\linebreak 
Очевидно, что~$W_1$~--- с.в.\ со стандартным показательным
распределением: ${\sf P}(W_1\hm<x)\hm=\left[1\hm-e^{-x}\right]{\bf 1}(x\hm\geqslant0)$.

Функция распределения и~плотность строго устойчивого распределения 
с~характеристическим показателем~$\alpha$ и~параметром формы~$\theta$,
опре\-де\-ля\-емо\-го х.ф.
$$
\mathfrak{g}_{\alpha,\theta}(t)=\exp\left\{\!
- |t|^{\alpha}\exp\left\{-\fr{1}{2}i\pi\theta\alpha\,\mathrm{sign}\,t\right\}\right\},\enskip
t\in\r\,,
$$
с $0<\alpha\leqslant2$, $|\theta|\hm\leqslant\min\{1,({2}/{\alpha})-1\}$, будут
соответственно обозначаться $G_{\alpha,\theta}(x)$ 
и~$g_{\alpha,\theta}(x)$ (см., например,~\cite{Zolotarev1983}). Случайная величина
с~ф.р.~$G_{\alpha,\theta}(x)$ будет обозначаться~$S_{\alpha,\theta}$.

\smallskip

\noindent
\textbf{Лемма 1}~\cite{KorolevWeibull2016}. \textit{При
$\delta\hm\in(0,\alpha)$ имеет место соотношение}
$$
{\sf E}S_{\alpha,1}^{\delta}=\fr{\Gamma(1-\delta/\alpha)}{\Gamma(1-\delta)}\,.
$$

\smallskip

Пусть $S_{\nu,1}$ и~$S'_{\nu,1}$~--- независимые неотрицательные
с.в.\ с~одним и~тем же односторонним строго устойчивым
распределением с~характеристическим показателем $\nu\hm\in(0,1]$.
Обозначим $R_{\nu}\hm=S_{\nu,1}/S'_{\nu,1}$. Свойства с.в.~$R_{\nu}$
приведены в~следующем утверждении.

\smallskip

\noindent
\textbf{Лемма~2}. 1. \textit{При $\nu\hm\in(0,1]$ имеет место соотношение}
$R_{\nu}\hm\eqd R_{\nu}^{-1}$.

\noindent 2. \textit{Если $\nu=1$, то ${\sf P}(R_{\nu}\hm=1)\hm=1$.}

\noindent 3. \textit{При $\nu\hm\in(0,1)$ с.в.~$R_{\nu}$ абсолютно
непрерывна, ее плотность~$p_{\nu}(x)$ имеет вид}:
\begin{multline*}
p_{\nu}(x)=\fr{\sin(\pi\nu)x^{\nu-1}}{\pi[1+x^{2\nu}+2x^{\nu}\cos(\pi\nu)]}={}\\
{}=
\fr{\sin(\pi\nu)x^{\nu-1}}
{\pi\left[\left(x^{\nu}+\cos(\pi\nu)\right)^2+\sin^2(\pi\nu)\right]}\,,\enskip
x>0\,.
\end{multline*}

\noindent 4. \textit{Если $0<\nu<1$, то моменты с.в.~$R_{\nu}$
порядков $\delta\hm\geqslant\nu$ бесконечны. Для $0\hm<\delta\hm<\nu\hm\leqslant1$ моменты
с.в.~$R_{\nu}$ порядка~$\delta$ имеют вид}:
$$
{\sf E}R_{\nu}^{\delta}=\fr{\Gamma(1-{\delta}/{\nu})\Gamma(1+{\delta}/{\nu})}
{\Gamma(1-\delta)\Gamma(1+\delta)}=
\fr{\sin(\pi\delta)}{\nu^2\sin({\pi\delta}/{\nu})}.
$$

\smallskip

Свойства 1 и~2 очевидны. Свойство~3 доказано в~работе~\cite{KorolevZeifman2017}. 
Свойство~4 вытекает из следствия~2, приведенного ниже.

\section{Представления распределений Миттаг-Леффлера и~Линника в~виде смесей}

\subsection{Распределение Миттаг-Леффлера как~смешанное показательное}

\noindent
\textbf{Теорема~1} (фольклор). \textit{При любом $\nu\hm\in(0,1]$
распределение Миттаг-Леффлера с~параметром~$\nu$ является масштабной
смесью одностороннего строго устойчивого распределения, причем
смешивающим является распределение Вейбулла с~параметром~$\nu/2$, т.\,е.}
$$
M_{\nu}\eqd S_{\nu,1}W_{\nu}\eqd S_{\nu,1}\sqrt{W_{\nu/2}}\,,
$$
\textit{где с.в.\ в~правой части независимы}.

\smallskip

\noindent
\textbf{Теорема~2}~\cite{KorolevZeifman2017, Kozubowski1998}. \textit{Распределение 
Мит\-таг-Леф\-фле\-ра с~параметром $\nu\hm\in(0,1]$ является
смешанным показательным, причем смешивающим является распределение
с.в.~$R_{\nu}\!:$}
$$
M_{\nu}\eqd W_1 R_{\nu}\,,
$$
\textit{где с.в.\ в~правой части независимы. Более того, если
$0\hm<\nu\hm<1$, то для плотности~$f_{\nu}^{M}(x)$ распределения
Мит\-таг-Леф\-фле\-ра справедливо интегральное пред\-став\-ле\-ние}
$$
f_{\nu}^{M}(x)=\fr{\sin(\pi\nu)}{\pi}\int\limits_{0}^{\infty}\,
\fr{z^{\nu}e^{-zx}dz}{1+z^{2\nu}+2z^{\nu}\cos(\pi\nu)}\,,\ 
 x>0\,.
$$

\smallskip

Представление распределения Мит\-таг-Леф\-фле\-ра в~виде смешанного
показательного было получено в~работе~\cite{Kozubowski1998}. Там же
приведен и~вид сме\-ши\-ва\-ющей плотности~$p_{\nu}(x)$. В~работе~\cite{KorolevZeifman2017} 
было замечено, что плотность~$p_{\nu}(x)$
соответствует с.в.~$R_\nu$~--- отношению двух независимых о.р.\
неотрицательных строго устойчивых с.в.

Из теоремы~2 вытекает представление ф.р. Мит\-таг-Леф\-фле\-ра
$F_{\nu}^{M}(x)$ для $x\hm>0$ в~виде
$$
F_{\nu}^{M}(x)= 1-\fr{\sin(\pi\nu)}{\pi}\int\limits_{0}^{\infty}\fr{
z^{\nu-1}e^{-zx}dz}{1+z^{2\nu}+2z^{\nu}\cos(\pi\nu)}\,.
$$

\smallskip

С помощью теоремы~2 легко сделать следующие выводы о~моментах
распределения Мит\-таг-Леф\-флера.

\bigskip

\noindent
\textbf{Следствие~1}. Первый логарифмический момент распределения
Мит\-таг-Леф\-фле\-ра не зависит от параметра~$\nu\!:$ для любого
$\nu\hm\in(0,1]$
$$
{\sf E}\ln M_{\nu}=\int\limits_{0}^{\infty}e^{-x}\ln x\,dx=-\mathbb{C}\,,
$$
где $\mathbb{C}=0{,}57721566490\ldots$~--- постоянная Эйлера.

\smallskip

Это утверждение следует из очевидного со\-от\-ношения $\ln M_{\nu}\hm\eqd
\ln W_1\hm+\ln S_{\nu,1}\hm-\ln S'_{\nu,1}$, вы\-те\-ка\-ющего из теоремы~2, 
и~хорошо известного пред\-став\-ле\-ния постоянной Эйлера (см., например,\linebreak
формулу~8.367(3) в~\cite{GradsteinRyzhik1971}).

\smallskip

В свою очередь с~учетом неравенства Иенсена следствие~1 влечет
простое неравенство
$$
{\sf E}M_{\nu}^{\delta}={\sf E}e^{\delta\ln M_{\nu}}\geqslant
e^{\delta\,{\sf E}\ln M_{\nu}}=e^{-\delta\mathbb{C}}\,,
$$
справедливое для любого $\delta\hm>0$ равномерно по $\nu\hm\in(0,1]$. При
этом для $\delta\hm\geqslant\nu$ это неравенство тривиально. Более аккуратное
описание моментов распределения Мит\-таг-Леф\-фле\-ра содержится 
в~следующих утверждениях.

Из теоремы~1 и~леммы~1 вытекает пред\-став\-ле\-ние моментов распределения
Мит\-таг-Леф\-фле\-ра в~терминах гам\-ма-функ\-ции, изначально полученное 
в~работе~\cite{Pillai1990} более сложным методом.

\bigskip

\noindent
\textbf{Следствие~2}. Если $0<\nu<1$, то моменты распределения
Мит\-таг-Леф\-фле\-ра порядков $\delta\geqslant\nu$ бесконечны. Для
$0\hm<\delta\hm<\nu\hm\leqslant1$ моменты распределения Мит\-таг-Леф\-фле\-ра имеют вид:

\noindent
\begin{equation}
{\sf E}M_{\nu}^{\delta}={\sf E}W_{\nu}^{\delta}\cdot{\sf
E}S_{\nu,1}^{\delta}=\fr{\Gamma(1-{\delta}/{\nu})\Gamma(1+{\delta}/{\nu})}
{\Gamma(1-\delta)}\,.
\label{e4-kor}
\end{equation}


\smallskip

Еще одно представление моментов распределения Мит\-таг-Леф\-фле\-ра,
возможно, более удобное с~вычислительной точки зрения, вытекает из
теоремы~2 и~формулы~3.252(12) в~\cite{GradsteinRyzhik1971}. Это
утверждение также изначально получено в~\cite{Pillai1990}.

\bigskip

\noindent
\textbf{Следствие~3}. Для $0\hm<\delta\hm<\nu\hm\leqslant1$ моменты распределения
Мит\-таг-Леф\-фле\-ра имеют вид:
\begin{multline}
{\sf E}M_{\nu}^{\delta}={\sf E}W_1^{\delta}\cdot{\sf E}R_{\nu}^{\delta}={}\\
{}=
\fr{\delta\Gamma(\delta)\sin(\pi\nu)}{\pi\nu}
\int\limits_{0}^{\infty}\fr{x^{\delta/\nu}dx}{1+x^{2}+2x\cos(\pi\nu)}={}\\
{}=
\fr{\delta\Gamma(\delta)\sin(\pi\delta)}{\nu^2\sin({\pi\delta}/{\nu})}\,.
\label{e5-kor}
\end{multline}

\smallskip

Используя хорошо известные эйлеровы формулы дополнения для
гам\-ма-функ\-ции, легко убедиться, что правые части~(4) и~(5)
совпадают.

\subsection{Связь с~распределением Линника}

Ниже во всех произведениях с.в.\ сомножители предполагаются
независимыми.

\bigskip

\noindent
\textbf{Теорема~3}~\cite{KorolevZeifman2017}. \textit{Пусть
$\alpha\hm\in(0,2]$, $\alpha'\hm\in(0,1]$. Тогда}
$$
L_{\alpha\alpha'}\eqd S_{\alpha,0}M_{\alpha'}^{1/\alpha}\,.
$$

\bigskip

\noindent
\textbf{Следствие~4}~\cite{KorolevZeifman2017}. {Распределение
Линника с~параметром $\alpha\hm\in(0,2]$ является масштабной смесью
нормальных законов, причем смешивающим является распределение
Мит\-таг-Леф\-фле\-ра с~па\-ра\-мет\-ром~$\alpha/2$}:
$$
L_{\alpha}\eqd X\sqrt{2M_{\alpha/2}}\,.
$$

\smallskip

С учетом следствий~2 и~3 из следствия~4, в~свою очередь,
непосредственно вытекает

\bigskip

\noindent
\textbf{Следствие~5}. Если $0\hm<\alpha\hm<2$, то моменты распределения
Линника порядков $\delta\hm\geqslant\alpha$ не существуют. При
$0\hm<\delta\hm<\alpha\hm\leqslant2$ абсолютные моменты распределения Линника имеют
вид:
\begin{multline*}
{\sf E}|L_{\alpha}|^{\delta}=2^{\delta/2}{\sf E}|X|^{\delta}\cdot
{\sf E}M_{\alpha/2}^{\delta/2}={}\\
{}=\fr{2^{\delta}\Gamma(1+\delta/2)
\Gamma(1-\delta/\nu)\Gamma(1+\delta/\nu)}{\sqrt{\pi}\Gamma(1-\delta/2)}={}\\
{}=
\fr{2^{\delta}\delta^2\left(\Gamma({\delta}/{2})\right)^2\sin
({\pi\delta}/{2})}{\nu^2\sqrt{\pi}\sin({\pi\delta}/{\nu})}\,.
\end{multline*}

\subsection{Распределение Миттаг-Леффлера как~смесь равномерных распределений}

Из теорем 2 и~5 (см.\ ниже) с~учетом результата статьи~\cite{Walker1999} 
легко получить следующие представления.

\smallskip

\noindent
\textbf{Теорема~4}~\cite{Korolev2016TVP}. \textit{Распределение
Мит\-таг-Леф\-фле\-ра с~параметром $\nu\hm\in(0,1]$ является масштабной
смесью равномерных распределений}:
$$
M_{\nu}\eqd U_{[0,1]}\Gamma_{2,1/\sqrt{2}} R_{\nu} \eqd
U_{[0,1]}\sqrt{2\Gamma_{3/2,1/2}W_1}\cdot R_{\nu}\,.
$$

\smallskip

Этот результат может быть использован при моделировании
продолжительности проектов. Как правило, предполагается, что она
случайна в~пределах заданного интервала, верхняя граница которого
(<<{\it deadline}>>) определяется экспертами, чье мнение также может
зависеть от случайных факторов.

\subsection{Распределение Миттаг-Леффлера как~смесь полунормальных законов}

\noindent
\textbf{Теорема~5}~\cite{KorolevZeifman2017}. \textit{Распределение
Мит\-таг-Леф\-фле\-ра с~параметром $\nu\hm\in(0,1]$ является масштабной
смесью полунормальных законов}:
$$
M_{\nu}\eqd |X|\sqrt{2W_1}\,R_{\nu}\,.
$$

\smallskip

Рассмотрим смешивающее распределение в~тео\-ре\-ме~5. Обозначим
$H_{\nu}(x)\hm={\sf P}(W_1\,R_{\nu}^2\hm<{x}/{2})$, $x\hm\geqslant0$. Тогда
утверждение теоремы~5 можно записать\linebreak в~виде:
\begin{multline*}
F_{\nu}^{M}(x)=\int\limits_{0}^{\infty}\Psi\left(\fr{x}{\sqrt{u}}\right)\,dH_{\nu}(u)={}\\
{}=
2\int\limits_{0}^{\infty}\Phi\left(\fr{x}{\sqrt{u}}\right)\,dH_{\nu}(u)-1\,,\enskip
x\geqslant0\,.
\end{multline*}
При $0<\nu\hm<1$ плотность, соответствующая ф.р.~$H_{\nu}(x)$, имеет вид:
\begin{multline*}
h_{\nu}(x)=\fr{d}{dx}\,H_{\nu}(x)=\fr{d}{dx}\,{\sf P}
\left(\!W_1<\fr{xR_{\nu}^2}{2}\!\right)={}\\
{}=
\fr{\sin(\pi\nu)}{2\pi}\!\int\limits_{0}^{\infty}\!\!\!
\fr{z^{\nu/2-1}e^{-xz/2}\,dz}{1+z^{\nu}+2z^{\nu/2}\cos(\pi\nu)}\,,
\enskip x\geqslant0\,.
\end{multline*}
Очевидно, что если $\nu\hm=1$, то $H_1(x)\hm=1\hm-e^{-x/2}$, $x\hm\geqslant0$.

\section{Распределение Миттаг-Леффлера как~асимптотическая
аппроксимация}

Распределение Мит\-таг-Леф\-фле\-ра может быть предельным в~довольно
простых предельных схемах для случайных последовательностей со
случайными индексами, в~частности для статистик, построенных по
выборкам случайного объема.

Чтобы указать примеры ситуаций, в~которых распределения
Мит\-таг-Леф\-фле\-ра могут выступать в~качестве асимптотических
аппроксимаций, уместно привести еще одно вспомогательное
утверж\-дение.

\smallskip

Рассмотрим последовательность случайных величин $Z_1,Z_2,\ldots$ Пусть
$N_1,N_2,\ldots$~--- на\-ту\-раль\-но\-знач\-ные с.в.\
такие, что при каж\-дом~$n$ с.в.~$N_n$ независима от
последовательности $Z_1,Z_2,\ldots$ В~следующей лемме сходимость
подразумевается при $n\hm\to\infty$.

\bigskip

\noindent
\textbf{Лемма~3.} \textit{Предположим, что существуют неограниченно
возрастающая $($убывающая к~нулю$)$ последовательность положительных
чисел $\{c_n\}_{n\geqslant1}$ и~с.в.~$Z$ такие, что
$c_n^{-1}Z_n\hm\Longrightarrow Z$. Если существуют неограниченно
возрастающая $($убывающая к~нулю$)$ последовательность
по\-ло\-жи\-тельных чисел~$\{d_n\}_{n\geqslant1}$ и~с.в.~$N$
такие, что}
\begin{equation}
d_n^{-1}c_{N_n}\Longrightarrow N\,,
\label{e6-kor}
\end{equation}
\textit{то}
\begin{equation}
d_n^{-1}Z_{N_n}\Longrightarrow N  Z\,,
\label{e7-kor}
\end{equation}
\textit{причем случайные сомножители в~правой части}~(\ref{e7-kor}) 
\textit{независимы. Если
дополнительно $N_n\hm\longrightarrow\infty$ по вероятности и~семейство
масштабных смесей функции распределения с.в.~$Z$
идентифицируемо, то}\linebreak \textit{условие}~(\ref{e6-kor}) 
\textit{не только достаточно для}~(\ref{e7-kor}), \textit{но и~необходимо.}


\bigskip

\noindent
Д\,о\,к\,а\,з\,а\,т\,е\,л\,ь\,с\,т\,в\,о\ см.\  в~\cite{Korolev1994} 
(случай $c_n,d_n\hm\to\infty$),~\cite{Korolev1995} (случай $c_n,d_n\hm\to 0$).

\smallskip

Выше уже было отмечено, что распределения Мит\-таг-Леф\-фле\-ра 
с~параметром $\nu\hm\in(0,1)$ и~только они могут быть предельными для
геометрических случайных сумм независимых о.р.\ с.в. В~этом случае
в лемме~3 с.в.~$Z_n$ образованы накопленными суммами независимых 
о.р.\ с.в., распределение которых принадлежит области притяжения
строго устойчивого распределения~$G_{\nu,1}$. Тогда, полагая $Z\hm\eqd
S_{\nu,1}$, $c_n\hm=d_n\hm=n^{1/\nu}$, в~(\ref{e6-kor}) получаем $N\hm\eqd
W_1^{1/\alpha}\hm\eqd W_{\alpha}$, и~в~соответствии с~леммой~3 
и~теоремой~1 в~(\ref{e7-kor}) $N Z\hm\eqd M_{\nu}$.

Поскольку в~мультипликативных представлениях для с.в.\ 
с~распределением Мит\-таг-Леф\-фле\-ра порядок сомножителей не имеет
значения, их роли в~предельной схеме, рассмотренной в~лемме~3, могут
быть различными. Скажем, хорошо известно, что распределение Вейбулла
может быть предельным для экстремальных порядковых статистик. Тогда
на основании теоремы~2 и~леммы~3 можно утверждать, что распределение
Мит\-таг-Леф\-фле\-ра с~параметром $\nu\hm\in(0,1)$ может быть предельным для
экстремальных порядковых статистик, построенных по выборкам
случайного объема, если нормированные индексы по распределению
сходятся к~с.в.~$S_{\nu,1}$. Для индексов специального вида~-- 
с.в.\ со смешанным пуассоновским распределением~--- этот случай детально
рассмотрен в~\cite{KorolevZeifman2017}.


Еще два примера предельных схем основаны на теореме~5. Как известно,
полунормальное распределение является предельным для максимальных
сумм независимых с.в., удовлетворяющих условию Линдеберга, или
абсолютных величин сумм таких слагаемых. На основании теоремы~5 и~леммы~3 
можно заключить, что распределение Мит\-таг-Леф\-фле\-ра 
с~параметром $\nu\hm\in(0,1)$ может быть предельным для максимальных
случайных сумм независимых с.в., удовлетворяющих условию
Линдеберга, или абсолютных величин случайных сумм таких слагаемых,
если нормированные индексы по распределению сходятся к~с.в.~$\sqrt{2W_1}\,R_{\nu}$. 
Этот случай также детально рассмотрен в~\cite{KorolevZeifman2017}.


\section{Статистическое оценивание параметра распределения Миттаг-Леффлера}

\subsection{Специальные версии метода моментов}

В связи с~тем, что плотность распределения Мит\-таг-Леф\-фле\-ра не
допускает явного пред\-став\-ле\-ния в~терминах элементарных функций,
метод\linebreak максимального правдоподобия оказывается ма-\linebreak лопригодным для
статистического оценивания параметра распределения Мит\-таг--Леф\-фле\-ра.
С~этой целью в~некоторых работах предлагается использовать
специальные модификации метода моментов. Первой из таких работ,
по-ви\-ди\-мо\-му, является \mbox{статья}~\cite{Kozubowski2001}, где использованы
формулы типа~(\ref{e4-kor}) и~(\ref{e5-kor}) при сильном предположении о~том, что заранее
известен нетривиальный интервал, содержащий неизвестное значение
параметра~$\nu$. В~работе~\cite{Cahoy2013} использованы
логарифмические мо\-менты.
{ %\looseness=1

}

\subsection{Двухэтапный сеточный метод оценивания параметра
распределения Миттаг-Леффлера}

Рассмотрим следующий альтернативный двухэтапный сеточный метод,
использующий более полную информацию об эмпирическом распределении,
нежели второй эмпирический логарифмический момент, как 
в~\cite{Cahoy2013}. Этот метод основан на теореме~2, позволяющей
представить распределение Мит\-таг-Леф\-фле\-ра в~виде смешанного
показательного.

Следует отметить, что сеточные методы разделения смесей
продемонстрировали высокую эффективность при оценивании параметров
смешанных пуассоновских распределений и~при разделении конечных или
произвольных дис\-пер\-си\-он\-но-сдви\-го\-вых смесей нормальных 
законов~\cite{KorolevKorchagin2014, KorolevKorchaginZeifman2016}.

На первом этапе на положительной полупрямой выделим основную часть
носителя смешивающего распределения, т.\,е.\ ограниченный интервал,
вероятность которого, вычисленная в~соответствии со смешивающим
распределением, практически равна единице. На этот интервал накинем
конечную сетку, содержащую (возможно, очень большое чис\-ло)
$K\hm\in\mathbb{N}$ {\it известных} узлов $\lambda_1,\ldots ,\lambda_K$.
Приблизим искомое распределение Мит\-таг-Леф\-фле\-ра конечной смесью
показательных законов (гиперэкспоненциальным распределением):

\vspace*{2pt}

\noindent
\begin{equation}
F_{\nu}^{M}(x)\approx 1-\sum\limits_{i=1}^K p_ie^{-\lambda_ix}\,,\enskip
x\geqslant0\,.
\label{e8-kor}
\end{equation}

\vspace*{-2pt}

\noindent
В~смеси, стоящей в~правой части соотношения~(\ref{e8-kor}), неизвестными
являются только па\-ра\-мет\-ры $p_1,\ldots,p_{K}$. 

Пусть $x_1,\ldots,x_n$~---
анализируемая выборка значений с.в.\ с~оцениваемым
распределением Мит\-таг-Леф\-фле\-ра. Итерационный процесс, определяющий
сеточный ЕМ (expectation-maximization)
ал\-го\-ритм для данной задачи, за\-да\-ет\-ся сле\-ду\-ющим образом.
Пусть $p_1^{(m)},\ldots,p_{K-1}^{(m)}$~--- оценки параметров
$p_1,\ldots,p_{K-1}$ на $m$-й итерации,
$p_K^{(m)}\hm=1-p_1^{(m)}-\cdots-p_{K-1}^{(m)}$. Для $i=1,\ldots,K$,
$j=1,\ldots,n$ обозначим $\varphi_{ij}\hm=\lambda_ie^{-\lambda_ix_j}$. Тогда,
используя стандартные рассуждения, определяющие вычислительные
формулы EM-ал\-го\-рит\-ма для параметров конечной смеси вероятностных
распределений (см., например,~\cite[разд.~5.3.7--5.3.8]{Korolev2011}), 
следует положить:

\vspace*{-6pt}

\noindent
\begin{multline}
p_i^{(m+1)}=\fr{p_i^{(m)}}{n}\sum\limits_{j=1}^n
\fr{\varphi_{ij}}{\sum\nolimits_{r=1}^Kp_r^{(m)}\varphi_{rj}}\,,
\\[-5pt]
 i=1,\ldots,K\,.
\label{e9-kor}
\end{multline}

\vspace*{-5pt}

\noindent
Можно показать, что если узлы $\lambda_1,\ldots,\lambda_K$ сетки
различны, неотрицательны и~известны, то итера-\linebreak\vspace*{-12pt}

\columnbreak

\noindent
ционный процесс~(\ref{e9-kor})
является монотонным, т.\,е.\
 каж\-дая его итерация не уменьшает
целевую сеточную функцию правдоподобия

\vspace*{-2pt}

\noindent
$$
L(p_1,\ldots,p_K;x_1,\ldots,x_n)=\prod\limits_{j=1}^n\left[\sum\limits_{i=1}^K
p_i\varphi_{ij}\right]\,.
$$

\vspace*{-2pt}

\noindent
Как показано в~\cite[разд.~5.7.4]{Korolev2011}, сеточная
функция правдоподобия $L(p_1,\ldots,p_{K};\,x_1,\ldots,x_n)$ вогнута по
аргументам $p_1,\ldots,p_{K}$. Поэтому на каж\-дом\linebreak шаге итерационного
процесса вместо соотношения~(\ref{e9-kor}) можно использовать любой более
быстрый алгоритм максимизации функции
$L(p_1,\ldots,p_{K};\,x_1,\ldots,x_n)$ по переменным $p_1,\ldots,p_{K}$.

Таким образом, на первом этапе получаются оценки весов~$p_i$ всех
узлов~$\lambda_i$, $i\hm=1,\ldots,K$, конечной сетки, накинутой на
носитель смешивающего распределения.

На втором этапе остается применить ка\-кой-ли\-бо стандартный метод
подгонки распределения с.в.~$R_{\nu}$, определяемого
плот\-ностью~$p_{\nu}(x)$ (см.\ п.~3 леммы~2), к~эмпирическим данным
типа гистограммы $(\lambda_1, p_1),\ldots, (\lambda_K, p_K)$,
полученным на первом этапе. Например, параметр~$\nu$ можно оценить,
минимизируя некоторое расстояние между полученной гистограммой 
и~плот\-ностью~$p_{\nu}(x)$. С~этой целью в~качестве оценки параметра~$\nu$ 
разумно искать такое значение~$\nu$, которое минимизирует
расстояние Куль\-ба\-ка--Лейб\-ле\-ра. Минимизация этого расстояния
эквивалентна максимизации правдоподобия полученной гистограммы 
в~классе распределений $\{p_{\nu}(x):\ \nu\hm\in(0,1]\}$.

При фиксированной сетке двухэтапный метод дает лишь приближенные
оценки параметра распределения Мит\-таг-Леф\-фле\-ра, причем точность
приближения зависит от успешного выбора сетки, который приобретает
критическое значение. Целесообразно выбирать сетку адаптивно, сгущая
ее там, где эмпирическая плотность принимает большие значения.
Говорить о состоятельности получаемых оценок при фиксированной сетке
нельзя. Но если объем выборки неограниченно возрастает и~вмес\-те 
с~ним согласованно увеличивается число узлов, то вопрос о~со\-сто\-ятель\-ности 
получаемых оценок приобретает смысл. Эти вопросы
будут отражены в~следующих публикациях.

\vspace*{-8pt}

\section{Дважды стохастические пуассоновские процессы, связанные 
с~распределением Миттаг-Леффлера}

\vspace*{-2pt}

Так как распределение Мит\-таг-Леф\-фле\-ра является смешанным
показательным (см.\ теорему~2), можно предложить две модификации
пуассоновских процессов со случайными интенсивностями, в~которых
длины промежутков времени между последовательными скачками имеют
распределения Мит\-таг-Леф\-фле\-ра. В~первой модификации длины\linebreak этих
промежутков условно независимы при фиксированной реализации
смешивающей с.в.\ $R_{\nu}\hm=S_{\nu,1}/S'_{\nu,1}$, тогда как во
второй модификации\linebreak длины этих промежутков независимы, так что сам
процесс является процессом восстановления.
{\looseness=1

}

\subsection{$R_{\nu}$-смешанные пуассоновские процессы}

Пусть $N_1(t)$, $t\hm\geqslant 0$,~--- стандартный пуассоновский процесс
(однородный пуассоновский процесс с~единичной ин\-тен\-сив\-ностью),
независимый от с.в.~$R_{\nu}\hm=S_{\nu,1}/S'_{\nu,1}$.

Определим $R_{\nu}$-смешанный пуассоновский процесс~$N_{\nu}(t)$,
$t\hm\geqslant0$, как суперпозицию $N_{\nu}(t)\hm=N_1(R_{\nu}t)$, $t\hm\geqslant0$. Так
как $R_{\nu}\hm\eqd 1/R_{\nu}$, то легко видеть, что в~рамках такой
модели при заданной с.в.~$R_{\nu}$ длины интервалов времени~$T_i$
между последовательными скачками~$T_i$ имеют вид $W_1^{(i)}R_{\nu}$,
где $W_1^{(1)},W_1^{(2)},\ldots$~--- независимые с.в.\ с~одним и~тем
же стандартным показательным распределением. В~соответствии 
с~теоремой~2 длина~$T_i$ каждого периода имеет распределение
Мит\-таг-Леф\-фле\-ра, но с.в.\ $T_1,T_2,\ldots$ не являются независимыми,
тогда как они {\it условно независимы} при фиксированном значении с.в.~$R_{\nu}$.

Пусть $X_1,X_2,\ldots$~--- независимые с.в., общее распределение
которых принадлежит области нормального притяжения строго
устойчивого закона~$G_{\alpha,0}$ с~$0\hm<\alpha\hm\leqslant2$. Пусть $N(t)$~---
$R_{\nu}$-сме\-шан\-ный пуассоновский процесс, независимый от
последовательности $X_1,X_2,\ldots$ Обозначим
$S_{N(t)}\hm=X_1+\cdots+X_{N(t)}$ $(S_0\hm=0)$. Процесс~$S_{N(t)}$,
$t\hm\geqslant0$, является {\it обобщенным $($compound$)$ $R_{\nu}$-сме\-шан\-ным
пуассоновским процессом}.

\bigskip

\noindent
\textbf{Теорема~6}. \textit{Пусть $\nu\hm\in(0,1)$. Предположим, что с.в.\
$X_1,X_2,\ldots$ и~процесс $N(t)$ удовлетворяют приведенным выше
условиям. Тогда при $t\hm\to\infty$}
\begin{multline*}
\fr{S_{N(t)}}{t^{1/\alpha}}\Longrightarrow S_{\alpha,0} 
R^{1/\alpha}_{\nu}\eqd S_{\alpha\nu,0}S_{\nu,1}^{-1/\alpha}\eqd{}\\
{}\eqd
X \sqrt{S_{\alpha\nu/2}S_{\nu,1}^{-2/\alpha}}\,.
\end{multline*}

\smallskip

Предельное распределение в~теореме~6 является 
{\it дроб\-но-устой\-чи\-вым}~\cite{FSD2001} с~параметрами $\beta\hm=\alpha\nu$ 
и~$\nu$ и~представимо в~виде масштабной смеси нормальных законов.

\subsection{Дважды стохастический миттаг-леффлер--пуассоновский процесс}

Пусть $N(t)$, $t\geqslant0$,~--- точечный процесс восстановления, т.\,е.\
расстояния $T_1,T_2,\ldots$ между соответствующими соседними
случайными точками $Y_0=0\leqslant Y_1\leqslant Y_2\leqslant\cdots$ 
точечного процесса~$N(t)$~--- независимые о.р.\ с.в.\ с~общей ф.р.~$V(x)$. Обозначим
$$
v(s)=\int\limits_{0}^{\infty}e^{-sx}\,dV(x)\,,\enskip s\geqslant0\,.
$$

Пусть $N_1(t)$, $t\hm\geqslant 0$,~--- стандартный пуассоновский процесс
(однородный пуассоновский процесс с~единичной интенсивностью). Пусть
$Z(t)$, $t\hm\geqslant0$,~--- случайная мера, т.\,е.\ случайный процесс 
с~неубывающими непрерывными справа траек\-то\-ри\-ями, удовле\-тво\-ря\-ющий
условиям $Z(t)\hm=0$, ${\sf P}\left(Z(t)\hm<\infty\right)=1$ ($0\hm<t\hm<\infty$).
Предположим, что процессы~$N_1(t)$ и~$Z(t)$ независимы. Процесс
$N(t)\hm=N_1\left(Z(t)\right)$ называется {\it дважды стохастическим
пуассоновским процессом}, или {\it процессом Кокса}.

\smallskip

\noindent
\textbf{Лемма~4}~\cite{Grandell1976, Kingman1964}. \textit{Точечный
процесс восстановления $N(t)$ является процессом Кокса тогда и~только тогда, когда}
$$
v(s)=\fr{1}{1-\ln g(s)}\,,
$$
\textit{где $g(s)$~--- п.~Л.--С.\ некоторого безгранично делимого
распределения. При этом}
$$
g(s)={\sf E}\exp\{-sZ^{-1}(1)\}\,;\quad Z^{-1}(0)=0\,,
$$
\textit{где $Z(t)$~--- случайная мера, управляющая процессом $N(t)$ $($т.\,е.\
 $N(t)\hm=N_1\left(Z(t)\right))$ 
и~$Z^{-1}(t)\hm=\sup\{\tau:\,Z(\tau)\hm\leqslant t\}$.}

\smallskip

Если $g(s)\hm=e^{-\lambda s^{\nu}}$, $0\hm<\nu\hm<1$, т.\,е.\ $g(s)$~--- 
п.~Л.--С.\
строго устойчивой с.в.~$S_{\nu,1}$, то
$v(s)\hm=\psi_{\nu}(s)=1/(1\hm+\lambda s^{\nu})$~--- п.~Л.--С.\
распределения Мит\-таг-Леф\-фле\-ра. В~силу леммы~4 это означает, что
процесс Кокса $N(t)$ является процессом восстановления, в~котором
расстояния между соседними точками (точками восстановления) имеют
распределения Мит\-таг-Леф\-фле\-ра. Такой процесс рассмотрен 
в~работах~\cite{Meerschaert2004, Meerschaert2010}, где он назван {\it дробным
пуассоновским процессом с~обратным $\nu$-устой\-чи\-вым субординатором}.
С~учетом сказанного выше, такой процесс также можно назвать {\it
мит\-таг-леф\-флер--пу\-ас\-со\-нов\-ским}.

Пусть $X_1,X_2,\ldots$~--- независимые о.р.\ с.в., общее
распределение которых принадлежит об\-ласти притяжения строго
устойчивого распределения~$G_{\nu,1}$. Пусть $N(t)$~--- дробный
пуассоновский процесс с~обратным $\nu$-устой\-чи\-вым субординатором,
$0\hm<\nu\hm<1$, независимый от последовательности $X_1,X_2,\ldots$ Снова
обозначим $S_{N(t)}\hm=X_1+\cdots$\linebreak $\cdots +X_{N(t)}$, считая, что $S_0\hm=0$. 
В~\cite{Meerschaert2004} доказано, что при надлежащей нормировке
процесс $S_{N(t)}$ слабо сходится в~пространстве Скорохода 
к~суперпозиции $P_{\alpha}\left(P^{-1}_{\nu}(t)\right)$, где
$P_{\alpha}(t)$~--- $\alpha$-устой\-чи\-вый процесс Леви, 
а~$P_{\nu}^{-1}(t)\hm=\sup\{\tau:\,P_{\nu}(\tau)\hm\leqslant t\}$~--- обратный
$\nu$-устой\-чи\-вый субординатор, независимый от~$P_{\alpha}(t)$.

Так как

\noindent
\begin{multline*}
{\sf P}\left\{ P_{\alpha}\left(P^{-1}_{\nu}(1)\right)<x\right\}={}\\
{}=
{\sf P}\left\{P_{\alpha}^{\alpha}(1)P_{\nu}^{-\nu}(1)\hm<x^{\alpha}\right\}=
{\sf P}\left\{S_{\alpha,0}S_{\nu,1}^{-\nu/\alpha}<x\right\}= {}\\
{}=
{\sf P}\left\{X \sqrt{S_{\alpha/2,1}S_{\nu,1}^{-2\nu/\alpha}}<x\right\}\,,
\end{multline*}
то конечномерные распределения предельного процесса опять являются
дроб\-но-устой\-чи\-вы\-ми, но на сей раз с~параметрами~$\alpha$ и~$\nu$ 
и~могут быть представлены в~виде специальных смесей нормальных законов.

\smallskip

В заключение автор считает своим приятным долгом выразить искреннюю
благодарность Владимиру Васильевичу Учайкину за стимулирующие
дискуссии.

\vspace*{-6pt}

{\small\frenchspacing
 {%\baselineskip=10.8pt
 \addcontentsline{toc}{section}{References}
 \begin{thebibliography}{99}
 
 \vspace*{-2pt}
 
 \bibitem{WeronKotulski1996} %1
\Au{Weron~K., Kotulski~M.} On the Cole--Cole relaxation function and
related Mittag-Leffler distributions~// Physica A, 1996. Vol.~232.
P.~180--188.

\bibitem{GorenfloMainardi2006} %2
\Au{Gorenflo R., Mainardi~F.} Continuous time random walk,
Mittag-Leffler waiting time and fractional diffusion: Mathematical
aspects~// Anomalous transport: Foundations and applications~/ Eds.
R.~Klages, G.~Radons,  I.\,M.~Sokolov.~--- Weinheim,
Germany: Wiley-VCH, 2008. Ch.~4. P.~93--127. 
{\sf http://arxiv.org/abs/0705.0797}.

\bibitem{Kilbas2014} %3
\Au{Gorenflo R., Kilbas~A.\,A., Mainardi~F., Rogosin~S.\,V.}
{Mittag-Leffler functions, related topics and applications}.~---
Springer monographs in matematics ser.~---
Berlin\,--\,New York: Springer, 2014. 457~p.



\bibitem{Joseetal2010} %4
\Au{Jose K.\,K., Uma~P., Lekshmi~V.\,S., Haubold~H.\,J.} Generalized
Mittag-Leffler distributions and processes for applications in
astrophysics and time series modeling~// 3rd
UN/ESA/NASA Workshop on the International Heliophysical Year 2007
and Basic Space Science. Astrophysics and Space Science Proceedings.~--- 
New York, NY, USA: Springer, 2010. P.~79--92.



\bibitem{Scalas2006} %5
\Au{Scalas E.} 
The application of continuous-time random
walks in finance and economics~// Physica A, 2006. Vol.~362. P.~225--239.

\bibitem{MeerschaertScalas2006} %6
\Au{Meerschaert M.\,M., Scalas~E.}
Coupled continuous time random walks in finance~// Physica A, 2006.
Vol.~370. P.~114--118.

\bibitem{KorolevZeifman2017} %7
\Au{Korolev V.\,Yu., Zeifman~A.\,I.} Convergence of statistics
constructed from samples with random sizes to the Linnik and
Mittag-Leffler distributions and their generalizations~// 
J.~Korean Stat. Soc., 2017. Vol.~46. P.~161--181.

\bibitem{FSD2001} %8
\Au{Kolokoltsov V.\,N., Korolev~V.\,Yu., Uchaikin~V.\,V.}
Fractional stable distributions~// J.~Math. Sci., 2001. Vol.~105. No.\,6. P.~2569--2576.

\bibitem{Bunge1996} %9
\Au{Bunge J.} Compositions semigroups and random stability~//
Ann. Probab., 1996. Vol.~24. P.~1476--1489.

\bibitem{KlebanovRachev1996} %10
\Au{Klebanov L.\,B., Rachev~S.\,T.} Sums of a random number of random
variables and their approximations with $\varepsilon$-accompanying
infinitely divisible laws~// Serdica, 1996. Vol.~22. P.~471--498.

\bibitem{Kovalenko1965} %11
\Au{Коваленко И.\,Н.} О~классе предельных распределений для редеющих
потоков однородных событий~// Литовский математический сборник,
1965. Т.~5. Вып.~4. С.~569--573.

\bibitem{GnedenkoKovalenko1968} %12
\Au{Gnedenko B.\,V., Kovalenko~I.\,N.} Introduction to queueing
theory.~--- Jerusalem: Israel Program for Scientific Translations, 1968.
281~p.

\bibitem{GnedenkoKovalenko1989}
\Au{Gnedenko B.\,V., Kovalenko~I.\,N.} Introduction to queueing
theory.~--- 2nd ed.~--- Boston: Birkhauser, 1989. 314~p.

\bibitem{GnedenkoKorolev1996}
\Au{Gnedenko B.\,V., Korolev~V.\,Yu.} Random summation: Limit
theorems and applications.~--- Boca Raton: CRC Press, 1996. 288~p.

\bibitem{Pillai1989}
\Au{Pillai R.\,N.} Harmonic mixtures and geometric infinite
divisibility~// J.~Indian Stat. Assoc., 1990.
Vol.~28. P.~87--98.

\bibitem{Pillai1990}
\Au{Pillai R.\,N.} On Mittag-Leffler functions and related
distributions~// Ann.  I.~Stat. Math.,
1990. Vol.~42. P.~157--161.

\bibitem{Linnik1953}
\Au{Линник Ю.\,В.} Линейные формы и~статистические критерии, I, II~// 
Украинский математический~ж., 1953. Т.~5. С.~207--243; 247--290.

\bibitem{Kotz2001}
\Au{Kotz S., Kozubowski~T.\,J., Podgorski~K.} The Laplace
distribution and generalizations: A~revisit with applications to
communications, economics, engineering, and finance.~--- Boston, MA, USA: 
Birkh$\ddot{\mbox{a}}$user Basel, 2001.
367~p.

\bibitem{Pillai1985}
\Au{Pillai R.\,N.} Semi-$\alpha$-Laplace distributions~//
Commun. Stat. Theor.~M., 1985. Vol.~14. P.~991--1000.

\bibitem{Laha1961}
\Au{Laha R.\,G.} On a class of unimodal distributions~// P.~Am. Math. Soc., 
1961. Vol.~12. P.~181--184.

\bibitem{Devroye1990}
\Au{Devroye L.} A~note on Linnik's distribution~// Stat. 
Probabil. Lett., 1990. Vol.~9. P.~305--306.

\bibitem{KotzOstrovskiiHayfavi1995a}
\Au{Kotz S., Ostrovskii~I.\,V., Hayfavi~A.} Analytic and asymptotic
properties of Linnik's probability densities. I~// J.~Math. Anal. Appl., 1995. 
Vol.~193. P.~353--371.

\bibitem{KotzOstrovskiiHayfavi1995b}
\Au{Kotz S., Ostrovskii~I.\,V., Hayfavi~A.} Analytic and asymptotic
properties of Linnik's probability densities. II~// J.~Math. Anal. Appl., 
1995. Vol.~193. P.~497--521.

\bibitem{Zolotarev1983}
\Au{Золотарев В.\,М.} Одномерные устойчивые распределения.~--- М.: Наука, 1983.
304~с.

\bibitem{KorolevWeibull2016} 
\Au{Korolev V.\,Yu.} Product representations
for random variables with Weibull distributions and their
applications~// J.~Math. Sci., 2016. Vol.~218. No.\,3. P.~298--313.

\bibitem{Kozubowski1998}
\Au{Kozubowski T.\,J.} Mixture representation of Linnik distribution
revisited~// Stat. Probabil. Lett., 1998. Vol.~38. P.~157--160.

\bibitem{GradsteinRyzhik1971} 
\Au{Градштейн И.\,С., Рыжик~И.\,М.}
Таблицы интегралов, сумм, рядов и~призведений.~--- М.: Наука, 1971. 1108~с.

\bibitem{Walker1999} 
\Au{Walker S.\,G., Guttierez-Pena~E.} Robustifying
Bayesian procedures (with discussion)~// Bayesian statistics~/ Eds.
J.\,M.~Bernardo, J.\,O.~Berger, A.\,P.~Dawid, A.\,F.\,M.~Smith.~---
New York, NY, USA: Oxford University Press, 1999. Vol.~6. P.~685--710.

\bibitem{Korolev2016TVP}
\Au{Королев В.\,Ю.} Предельные распределения дважды стохастически
прореженных процессов восстановления и~их свойства~// Теория
вероятностей и~ее применения, 2016. Т.~61. Вып.~4. С.~753--773.

\bibitem{Korolev1994} 
\Au{Королев В.\,Ю.} Сходимость случайных
последовательностей с~независимыми случайными индексами. I~// Теория
вероятностей и~ее применения, 1994. Т.~39. Вып.~2. С.~313--333.

\bibitem{Korolev1995} 
\Au{Королев В.\,Ю.} Сходимость случайных
последовательностей с~независимыми случайными индексами. II~//
Теория вероятностей и~ее применения, 1995. Т.~40. Вып.~4. С.~907--910.

\bibitem{Kozubowski2001} 
\Au{Kozubowski T.\,J.} Fractional moment estimation of Linnik
and Mittag-Leffler parameters~// Math. Comput.
Model., 2001. Vol.~34. No.\,9-11. P.~1023--1035.

\bibitem{Cahoy2013} 
\Au{Cahoy D.\,E.} Estimation of Mittag-Leffler
parameters~// Commun. Stat. Simul.~C., 2013. Vol.~42. No.\,2. P.~303--315.

\bibitem{KorolevKorchagin2014} 
\Au{Королев В.\,Ю., Корчагин~А.\,Ю.}
Модифицированный сеточный метод разделения дис\-пер\-си\-он\-но-сдви\-го\-вых
смесей нормальных законов~// Информатика и~её применения, 2014. Т.~8. Вып.~4. С.~11--19.

\bibitem{KorolevKorchaginZeifman2016} 
\Au{Королев В.\,Ю., Корчагин~А.\,Ю., Зейфман~А.\,И.} 
Теорема Пуассона для схемы испытаний Бернулли со
случайной вероятностью успеха и~дискретный аналог распределения
Вейбулла~// Информатика и~её применения, 2016. Т.~10. Вып.~4. С.~11--20.

\bibitem{Korolev2011}
\Au{Королев В.\,Ю.} Ве\-ро\-ят\-ност\-но-ста\-ти\-сти\-че\-ские методы декомпозиции
волатильности хаотических процессов.~--- М.: Изд-во МГУ, 2011. 512~с.

\bibitem{Kingman1964} 
\Au{Kingman J.\,F.\,C.} On doubly stochastic Poisson
processes~// P.~Camb. Philos. Soc., 1964. Vol.~60. No.\,4. P.~923--930.

\bibitem{Grandell1976}
\Au{Grandell J.} {Doubly stochastic Poisson processes}.~--- 
Lecture notes in mathematics ser.~--- Berlin\,--\,Heidelberg\,--\,New York:
Springer, 1976. Vol.~529. 244~p.

\bibitem{Meerschaert2004} 
\Au{Meerschaert M.\,M., Scheffler~H.\,P.}
Limit theorems for continuous-time random walks with infinite mean
waiting times~// J.~Appl. Probab., 2004. Vol.~41. No.\,3. P.~623--638.

\bibitem{Meerschaert2010} 
\Au{Meerschaert M., Nane~E., Vellaisamy~P.} 
The fractional Poisson process and the inverse stable
subordinator~// Electron. J.~Probab., 2011. Paper No.\,59. P.~1600--1620. 
{\sf ArXiv:1007.5051v1}.

 \end{thebibliography}

 }
 }

\end{multicols}

\vspace*{-6pt}

\hfill{\small\textit{Поступила в~редакцию 19.10.17}}

\vspace*{8pt}

%\newpage

%\vspace*{-24pt}

\hrule

\vspace*{2pt}

\hrule

%\vspace*{8pt}


\def\tit{SOME PROPERTIES OF THE~MITTAG-LEFFLER DISTRIBUTION AND~RELATED PROCESSES}

\def\titkol{Some properties of the~Mittag-Leffler distribution and~related processes}

\def\aut{V.\,Yu.~Korolev$^{1,2,3}$}

\def\autkol{V.\,Yu.~Korolev}

\titel{\tit}{\aut}{\autkol}{\titkol}

\vspace*{-9pt}


\noindent
$^1$Department 
 of Mathematical Statistics, Faculty of Computational Mathematics and 
 Cybernetics, Faculty of\linebreak
 $\hphantom{^1}$Computational Mathematics and Cybernetics, M.\,V.~Lomonosov 
 Moscow State University, 1-52~Leninskiye Gory,\linebreak
 $\hphantom{^1}$GSP-1, Moscow 119991,   Russian Federation
 
 \noindent
 $^2$Institute of Informatics Problems, 
 Federal Research Center ``Computer Science and Control'' 
 of the Russian\linebreak
 $\hphantom{^1}$Academy of Sciences, 44-2~Vavilov Str., Moscow 119333,  
 Russian Federation
 
 \noindent
 $^3$Hangzhou Dianzi University, 
 Xiasha Higher Education Zone, Hangzhou 310018, China



\def\leftfootline{\small{\textbf{\thepage}
\hfill INFORMATIKA I EE PRIMENENIYA~--- INFORMATICS AND
APPLICATIONS\ \ \ 2017\ \ \ volume~11\ \ \ issue\ 4}
}%
 \def\rightfootline{\small{INFORMATIKA I EE PRIMENENIYA~---
INFORMATICS AND APPLICATIONS\ \ \ 2017\ \ \ volume~11\ \ \ issue\ 4
\hfill \textbf{\thepage}}}

\vspace*{3pt}



\Abste{The paper contains an overview of some properties of
the Mittag-Leffler distribution. Main attention is paid to its
representability as a mixed exponential law. The possibility to
represent the Mittag-Leffler distribution as a~scale mixture of
half-normal and uniform distributions is discussed as well. It is
shown that the Mittag-Leffler distribution can be used as an
asymptotic approximation to the distributions of several statistics
constructed from samples with random sizes. A~new two-stage grid
method for the estimation of the parameter of the Mittag-Leffler
distribution is described. This method is based on the
representation of the Mittag-Leffler distribution as a mixed
exponential law. Two ways are considered to extend the notion of the
Mittag-Leffler distribution to Poisson-type stochastic processes.
The first way leads to a special mixed Poisson process and the second
leads to a~special renewal process simultaneously being a~doubly
stochastic Poisson process (Cox process). In limit theorems for
randomly stopped random walks in both of these cases, the limit laws
are fractionally stable distributions representable as normal scale
mixtures with different mixing distributions.}

\KWE{Mittag-Leffler distribution; Linnik distribution;
stable distribution; Weibull distribution; exponential distribution;
mixed Poisson process; renewal process; asymptotic approximation}





 \DOI{10.14357/19922264170404} 

%\vspace*{-12pt}

\Ack
\noindent
The research was partially supported by the Russian Foundation for 
Basic Research (project No.\,17-07-00717).



%\vspace*{3pt}

  \begin{multicols}{2}

\renewcommand{\bibname}{\protect\rmfamily References}
%\renewcommand{\bibname}{\large\protect\rm References}

{\small\frenchspacing
 {%\baselineskip=10.8pt
 \addcontentsline{toc}{section}{References}
 \begin{thebibliography}{99}
 
 \bibitem{3-kor-1} %1
\Aue{Weron, K., and M.~Kotulski.} 1996. On the Cole--Cole
relaxation function and related Mittag-Leffler distributions. \textit{Physica~A} 
232: 180--188.


\bibitem{1-kor-1} %2
\Aue{Gorenflo, R., and F.~Mainardi.} 2008. Continuous time random walk, Mittag-Leffler
waiting time and fractional diffusion: Mathematical aspects.
\textit{Anomalous transport: Foundations and applications}. 
Eds.\ R.~Klages, G.~Radons, and I.\,M.~Sokolov. Weinheim,
Germany: Wiley-VCH. Ch.~4. P.~93--127.
Available at: {\sf http://arxiv.org/abs/0705.0797}
(accessed December~7, 2017).

\bibitem{2-kor-1} %3
\Aue{Gorenflo R., A.\,A.~Kilbas, F.~Mainardi, and S.\,V.~Rogosin.} 2014. 
\textit{Mittag-Leffler
functions, related topics and applications}. Springer monographs in matematics ser.
Berlin\,--\,New York:
Springer.  457~p.



\bibitem{4-kor-1}
\Aue{Jose, K.\,K., P.~Uma, V.\,S.~Lekshmi, and H.\,J.~Haubold.} 2010. Generalized
Mittag-Leffler distributions and processes for applications in
astrophysics and time series modeling. \textit{3rd
UN/ESA/NASA Workshop on the International Heliophysical Year 2007
and Basic Space Science. Astrophysics and Space Science
Proceedings.}  New York, NY: Springer. 79--92.

\bibitem{6-kor-1} %5
\Aue{Scalas, E.} 2006. The application of continuous-time random
walks in finance and economics. \textit{Physica A} 362:225--239.

\bibitem{5-kor-1} %6
\Aue{Meerschaert, M.\,M., and E.~Scalas.} 2006. Coupled continuous time random
walks in finance. \textit{Physica A} 370:114--118.



\bibitem{7-kor-1}
\Aue{Korolev, V.\,Yu., and A.\,I.~Zeifman.} 2017. Convergence of statistics constructed
from samples with random sizes to the Linnik and Mittag-Leffler
distributions and their generalizations. \textit{J.~Korean
Stat. Soc.} 46:161--181.

\bibitem{8-kor-1}
\Aue{Kolokoltsov, V.\,N., V.\,Yu.~Korolev, and V.\,V.~Uchaikin.} 2001. Fractional stable
distributions. \textit{J.~Math. Sci.} 105(6):2569--2576.

\bibitem{9-kor-1}
\Aue{Bunge, J.} 1996. Compositions semigroups and random stability.
\textit{Ann. Probab.} 24:1476--1489.

\bibitem{10-kor-1}
\Aue{Klebanov, L.\,B., and S.\,T.~Rachev.} 1996. Sums of a random number of random
variables and their approximations with $\varepsilon$-accompanying
infinitely divisible laws. \textit{Serdica} 22:471--498.

\bibitem{11-kor-1}
\Aue{Kovalenko, I.\,N.} 1965. O~klasse predel'nykh raspredeleniy dlya
redeyushchikh potokov odnorodnykh sobytiy [On the class of limit
distributions for rarefied flows of homogeneous events].
\textit{Litovskiy Matematicheskiy Sbornik} [Lith. Math.~J.] 5(4):569--573.

\bibitem{12-kor-1}
\Aue{Gnedenko, B.\,V., and I.\,N.~Kovalenko.} 1968. \textit{Introduction to queueing
theory}. Jerusalem: Israel Program for Scientific Translations. 281~p.

\bibitem{13-kor-1}
\Aue{Gnedenko, B.\,V., and I.\,N.~Kovalenko.} 1989 \textit{Introduction to queueing
theory}. 2nd ed. Boston: Birkhauser. 314~p.

\bibitem{14-kor-1}
\Aue{Gnedenko, B.\,V., and V.\,Yu.~Korolev}. 1996. \textit{ Random summation: Limit
theorems and applications}. Boca Raton: CRC Press. 288~p.

\bibitem{15-kor-1}
\Aue{Pillai, R.\,N.} 1989. Harmonic mixtures and geometric
infinite divisibility. \textit{J.~Indian Stat. Assoc.} 28:87--98.

\bibitem{16-kor-1}
\Aue{Pillai, R.\,N.} 1990. On Mittag-Leffler functions and related
distributions. \textit{Ann. I.~Stat.
Math.} 42:157--161.

\bibitem{17-kor-1}
\Aue{Linnik, Yu.\,V.} 1953. Lineynye formy i~statisticheskie kriterii. I,~II 
[Linear forms and statistical criteria. I, II]. \textit{Ukrainskiy
Matematicheskiy Zh.} [Ukr. Math.~J.] 5:207--243; 5:247--290.

\bibitem{18-kor-1}
\Aue{Kotz, S., T.\,J.~Kozubowski, and K.~Podgorski}. 2001. \textit{ The
Laplace distribution and generalizations: A revisit with
applications to communications, economics, engineering, and
finance}. Boston, MA: Birkh$\ddot{\mbox{a}}$user Basel. 367~p.

\bibitem{19-kor-1}
\Aue{Pillai, R.\,N.} 1985. Semi-$\alpha$-Laplace distributions. 
\textit{Commun. Stat. Theor.~M.} 14:991--1000.

\bibitem{20-kor-1}
\Aue{Laha, R.\,G.} 1961. On a class of unimodal distributions. \textit{P.~Am. 
Math. Soc.} 12:181--184.

\bibitem{21-kor-1}
\Aue{Devroye, L.} 1990. A~note on Linnik's distribution. \textit{Stat.
Probabil. Lett.} 9:305--306.

\bibitem{22-kor-1}
\Aue{Kotz, S., I.\,V.~Ostrovskii, and A.~Hayfavi.} 1995. Analytic and asymptotic
properties of Linnik's probability densities.~I. \textit{J.~Math. Anal. 
Appl.} 193:353--371.

\bibitem{23-kor-1}
\Aue{Kotz, S., I.\,V.~Ostrovskii, and A.~Hayfavi.} 1995. Analytic and asymptotic
properties of Linnik's probability densities.~II. \textit{J.~Math. Anal.
 Appl}. 193:497--521.

\bibitem{24-kor-1}
\Aue{Zolotarev, V.\,M.} 1983. \textit{ Odnomernye ustoychivye raspredeleniya}
[One-dimensional stable distributions]. Moscow: Nauka. 304~p.

\bibitem{25-kor-1}
\Aue{Korolev, V.\,Yu.} 2016. Product representations for random variables with the
Weibull distributions and their applications. \textit{J.~Math. Sci}. 218(3):298--313.

\bibitem{26-kor-1}
\Aue{Kozubowski, T.\,J.} 1998. Mixture representation
of Linnik distribution revisited. \textit{Stat. Probabil. Lett}. 38:157--160.

\bibitem{27-kor-1}
\Aue{Gradshtein, I.\,S., and I.\,M.~Ryzhik.} 1971. \textit{Tablitsy integralov, summ,
ryadov i~proizvedeniy} [Tables of integrals, sums, series and
products]. Moscow: Nauka. 1108~p.

\bibitem{28-kor-1}
\Aue{Walker, S.\,G., and E.~Guttierez-Pena.} 1999. Robustifying Bayesian
procedures (with discussion). 
\textit{Bayesian statistics.} Eds.\ J.\,M.~Bernardo, J.\,O.~Berger, A.\,P.~Dawid, 
and A.\,F.\,M.~Smith.
New York, NY: Oxford University Press. 6:685--710.

\bibitem{29-kor-1}
\Aue{Korolev, V.\,Yu.} 2016. Predel'nye raspredeleniya dlya dvazhdy stokhasticheski
prorezhennykh protsessov vosstanovleniya i~ikh svoystva [Limit
theorems for doubly stochastically rarefied renewal processes and
their properties]. \textit{Teoriya Veroyatnostey i~ee Primineniya}
[Theor. Probab. Appl.] 61:753--773.

\bibitem{30-kor-1}
\Aue{Korolev, V.\,Yu.} 1995. Convergence of random
sequences with the independent random indices.~I. \textit{Theor. Probab. Appl.}
39(2):282--297.

\bibitem{31-kor-1}
\Aue{Korolev, V.\,Yu.} 1996. Convergence of random
sequences with independent random indices.~II. \textit{Theor. 
Probab. Appl.} 40(4):770--772.

\bibitem{32-kor-1}
\Aue{Kozubowski, T.\,J.} 2001. Fractional moment estimation of Linnik
and Mittag-Leffler parameters. \textit{Math. Comput.
Model.} 34(9-11):1023--1035.

\bibitem{33-kor-1}
\Aue{Cahoy, D.\,E.} 2013. Estimation of Mittag-Leffler parameters.
\textit{Commun. Stat. Simul.~C.}
42(2): 303--315.

\bibitem{34-kor-1}
\Aue{Korolev, V.\,Yu., and A.\,Yu.~Korchagin.} 2014. Mo\-di\-fi\-tsi\-ro\-van\-nyy setochnyy metod
razdeleniya dis\-per\-si\-on\-no-sdvi\-go\-vykh smesey normal'nykh zakonov 
[A~modified grid method for the separation of normal variance-mean
mixtures]. \textit{Informatika i~ee Primeneniya~--- Inform. Appl.} 8(4):11--19.

\bibitem{35-kor-1}
\Aue{Korolev, V.\,Yu., A.\,Yu.~Korchagin, and A.\,I.~Zeifman.} 2016.
Teorema Puassona dlya skhemy ispytaniy Bernulli so sluchaynoy
veroyatnost'yu uspekha i~diskretnyy analog raspredeleniya Veybulla
[The Poisson theorem for the scheme of Bernoulli trials with 
a~random probability of success]. \textit{Informatika i~ee Primeneniya~---
Inform. Appl.} 10(4):11--20.

\bibitem{36-kor-1}
\Aue{Korolev, V.\,Yu.} 2011. \textit{Veroyatnostno-statisticheskie metody
dekompozitsii volatil'nosti khaoticheskikh processov} [Probabilistic
and statistical methods for the decomposition of volatility of
chaotic processes]. Moscow: MSU Publs. 512~p.

\bibitem{37-kor-1}
\Aue{Kingman, J.\,F.\,C.} 1964. On doubly stochastic Poisson processes. \textit{P.~Camb. 
Philos. Soc.} 60(4): 923--930.

\bibitem{38-kor-1}
\Aue{Grandell, J.} 1976. Doubly stochastic Poisson processes. Lecture notes in
mathematics ser.  Berlin\,--\,Heidelberg\,--\,New York: Springer. Vol.~529. 244~p.

\bibitem{39-kor-1}
\Aue{Meerschaert, M.\,M., and H.\,P.~Scheffler}. 2004. Limit theorems for
continuous-time random walks with infinite mean waiting times. \textit{J.~Appl. 
Probab.} 41(3):623--638.

\bibitem{40-kor-1}
\Aue{Meerschaert, M.\,M., E.~Nane, and P.~Vellaisamy}. 
2011. The fractional Poisson process
and the inverse stable subordinator. \textit{Electron. J.~Probab.}
Paper No.\,59. 1600--1620. 
Available at: {\sf ArXiv:1007.5051v1} (accessed December~7, 2017).
\end{thebibliography}

 }
 }

\end{multicols}

\vspace*{-6pt}

\hfill{\small\textit{Received October 19, 2017}}

%\vspace*{-10pt}

\Contrl

\noindent
\textbf{Korolev Victor Yu.} (b.\ 1954)~---
 Doctor of Science in physics and mathematics, professor, Head of the Department 
 of Mathematical Statistics, Faculty of Computational Mathematics and 
 Cybernetics, Faculty of Computational Mathematics and Cybernetics, M.\,V.~Lomonosov 
 Moscow State University, 1-52~Leninskiye Gory, GSP-1, Moscow 119991, 
 Russian Federation; leading scientist, Institute of Informatics Problems, 
 Federal Research Center ``Computer Science and Control'' 
 of the Russian Academy of Sciences, 44-2~Vavilov Str., Moscow 119333,  
 Russian Federation;  professor, Hangzhou Dianzi University, 
 Xiasha Higher Education Zone, Hangzhou 310018, China; \mbox{vkorolev@cs.msu.su}
\label{end\stat}


\renewcommand{\bibname}{\protect\rm Литература} 