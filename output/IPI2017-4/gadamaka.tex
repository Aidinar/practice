\def\stat{gadamaka}

\def\tit{МЕТОД МОДЕЛИРОВАНИЯ ХАРАКТЕРИСТИК ИНТЕРФЕРЕНЦИИ ПРИ~ПРЯМОМ 
ВЗАИМОДЕЙСТВИИ ПЕРЕМЕЩАЮЩИХСЯ УСТРОЙСТВ В~ГЕТЕРОГЕННОЙ 
БЕСПРОВОДНОЙ СЕТИ ПЯТОГО ПОКОЛЕНИЯ$^*$}

\def\titkol{Метод моделирования характеристик интерференции при~прямом 
взаимодействии перемещающихся устройств} % в~гетерогенной  беспроводной сети пятого поколения}

\def\aut{Ю.\,В.~Гайдамака$^1$, К.\,Е.~Самуйлов$^2$, С.\,Я.~Шоргин$^3$}

\def\autkol{Ю.\,В.~Гайдамака, К.\,Е.~Самуйлов, С.\,Я.~Шоргин}

\titel{\tit}{\aut}{\autkol}{\titkol}

\index{Гайдамака Ю.\,В.}
\index{Самуйлов К.\,Е.}
\index{Шоргин С.\,Я.}
\index{Gaidamaka Yu.}
\index{Samouylov K.}
\index{Shorgin S.}



{\renewcommand{\thefootnote}{\fnsymbol{footnote}} \footnotetext[1]
{Исследование выполнено при финансовой поддержке Российского научного фонда 
(проект 16-11-10227).}}


\renewcommand{\thefootnote}{\arabic{footnote}}
\footnotetext[1]{Российский университет дружбы народов; 
Институт проб\-лем информатики Федерального исследовательского 
центра <<Информатика и~управ\-ле\-ние>> 
Российской академии наук, \mbox{gaydamaka\_yuv@rudn.university}}
\footnotetext[2]{Российский университет дружбы народов; Институт проб\-лем информатики Федерального исследовательского 
центра <<Информатика и~управ\-ле\-ние>> Российской академии наук, 
\mbox{samuylov\_ke@rudn.university}}
\footnotetext[3]{Институт проб\-лем информатики Федерального исследовательского центра 
<<Информатика и~управ\-ле\-ние>> Российской академии наук, \mbox{sshorgin@ipiran.ru}}

\vspace*{-9pt}

    
    
\Abst{В общем виде показано построение модели перемещения взаимодействующих 
устройств в~гетерогенных беспроводных сетях пятого поколения с~помощью кинетического 
уравнения с~учетом скорости перемещения устройств, их пространственной плотности 
и~максимального допустимого радиуса взаимодействия. Предложен метод моделирования 
траекторий, когда при\-емо-пе\-ре\-да\-ющие устройства движутся случайным образом, 
причем блуждание в~общем случае не является стационарным, что отличает предложенную 
модель движения мобильных устройств от известных ранее моделей. Характеристики 
интерференции, в~том числе отношение сигнал/ин\-тер\-фе\-рен\-ция (SIR,
signal-to-interference ratio), исследуются 
в~виде непрерывного во времени случайного процесса, задачу расчета этих характеристик 
предлагается решать методом имитационного моделирования. Показано, что такой анализ 
дает возможность исследовать вероятностные характеристики взаимодействия устройств, 
такие как вероятность обрыва связи и~длительности периодов наличия и~отсутствия связи 
между устройствами.}

\KW{беспроводная гетерогенная сеть; отношение сигнал/ин\-тер\-фе\-рен\-ция; прямое 
взаимодействие устройств; перемещение взаимодействующих устройств; модель движения; 
кинетическое уравнение; генерация траекторий; показатели эффективности сети}

\DOI{10.14357/19922264170401} 

\vspace*{-3pt}


\vskip 10pt plus 9pt minus 6pt

\thispagestyle{headings}

\begin{multicols}{2}

\label{st\stat}

\section{Введение}

  На производительность современных сетей подвижной связи, работающих 
в~диапазоне де\-ци\-мет\-ро\-вых (300~МГц\,--\,3~ГГц) и~сантиметровых (3--30~ГГц) 
волн (например, GSM~[1], 900~МГц, WiMAX~[2], 2,5~ГГц, WCDMA~[3] 
и~LTE~[4], 1,9~ГГц и~2,1~ГГц) существенное влияние оказывает 
интерференция, т.\,е.\ взаимное деструктивное влияние радиосигналов 
мобильных станций, ра\-бо\-та\-ющих в~одном диапазоне частот~[5]. Основным 
па\-ра\-мет\-ром, определяющим качество в~таких сетях, является SIR, которое 
измеряется на приемнике и~характеризует качество беспроводного канала связи 
между приемником и~передатчиком в~ассоциированной паре  
<<пе\-ре\-дат\-чик--при\-ем\-ник>>~[1--6]. Отношение сиг\-нал/ин\-тер\-фе\-рен\-ция определяет 
скорость передачи данных и~спектральную эффективность радиоканала 
и~системы в~целом, от которых зависит пропускная способность сети, 
надежность и~связность беспроводных соединений. 
  
  Результаты работ~[7--12] по точному и~приближенному анализу SIR могут 
быть использованы для оценки скорости передачи данных, спектральной 
эффективности, числа абонентов, которым в~определенной географической 
зоне могут быть предоставлены услуги беспроводной связи, в~предположении 
о~постоянном значении SIR, т.\,е.\ для случая неподвижных 
абонентов. 

Исследования перемещающихся беспроводных устройств были 
начаты в~[13] и~получили развитие в~[14], где для случая нестационарного 
блуж\-да\-ния абонентов был предложен метод расчета SIR, позволивший 
получить плотности распределения длительности периодов наличия 
и~отсутствия связи. В~соответствии с~предложенным методом  
SIR представляет собой функционал расстояний между 
взаимодействующими устройствами, опре\-де\-ля\-емых моделью движения 
абонентов, которая описывается кинетическим уравнением. При этом\linebreak метод 
моделирования ансамбля траекторий для заданной модели движения  
в~\cite{14-g1} не излагался. Этот метод изложен ниже в~разд.~3 статьи. Авторы 
используют ту же системную модель и~обозначения, что позволило в~разд.~4 
продолжить численный анализ исследованного в~\cite{14-g1} сценария 
перемещения взаимодействующих устройств. Исследованы вероятность обрыва 
связи, распределения длительности периодов наличия и~отсутствия связи и~их 
средние значения. В~заключении приведены выводы и~определены задачи 
дальнейших исследований.
  
\section{Модель перемещения устройств}

  Траектории движения устройств моделируются для ситуации, когда 
взаимодействующие при\-емо-пе\-ре\-да\-ющие устройства движутся случайным 
\mbox{образом} в~ограниченной зоне обслуживания~$V$, пред\-став\-ля\-ющей собой 
область в~$M$-мер\-ном пространстве, причем блуждание в~общем случае не 
является стационарным. Считаем, что перемещение устройств сочетает 
целеполагающее поступательное движение и~хаотическое блуждание 
и~определяет\-ся кинетическим уравнением типа Фок\-ке\-ра--План\-ка 
с~заданной скоростью сноса~$v$ и~коэффициентом диффузии~$\alpha$.
  


  Для исследования интерференции на приемнике ассоциированной пары 
взаимодействующих устройств, создаваемой передатчиками других 
ассоциированных пар, устройства разбиты на пары, число пар обозначено~$N$. 
На рис.~1 схематически показаны два варианта перемещения приемников 
в~ассоциированной паре <<пе\-ре\-дат\-чик--при\-ем\-ник>>: в~случае~(\textit{а}) 
приемник движется вместе с~передатчиком, оставаясь на постоянном 
расстоянии~$d$ от него; в~случае~(\textit{б}) приемник движется согласно броуновскому 
движению с~диффузией~$v_{R_x}$ внутри окружности радиуса~$d$ с~центром 
в~точке расположения передатчика. Первый вариант соответствует сценарию, 
когда два абонента не отрываются\linebreak

 { \begin{center}  %fig1
 \vspace*{9pt}
 \mbox{%
 \epsfxsize=77.39mm 
 \epsfbox{gai-1.eps}
 }


\end{center}


\noindent
{{\figurename~1}\ \ \small{Сценарии взаимодействия устройств на основе 
прямых соединений D2D}}
}

%\vspace*{6pt}

\addtocounter{figure}{1}


\noindent
 друг от друга в~процессе перемещения, т.\,е.\ 
траектории их движения зависимы, второй~--- сценарию, когда траектории 
движения приемника и~передатчика взаимно независимы. При этом 
расстояние~$d$ в~первом случае имеет смысл постоянной дистанции между 
абонентами в~движущейся паре, а~во втором~--- максимального расстояния от 
приемника до передатчика в~ассоциированной паре, которое определяется из 
описанной ниже модели распространения сигнала. 

Предложенный в~данной 
статье метод не ограничен выбранными параметрами и~может быть применен 
к~другим моделям движения, описываемым уравнением диффузии. Не 
ограничивая общности метода, предполагается, что мощность сигнала на 
приемнике задается выражением $\phi(r)\hm=Ar^{-\gamma}$ и~зависит от 
расстояния~$r$ при заданной константе~$A$, учитывающей излучаемую 
мощность и~коэффициенты усиления приемной и~передающей антенны, 
и~коэффициенте распространения сигнала~$\gamma$. 
  
  Как и~в~\cite{14-g1}, для оценки SIR далее используется 
формула:
 $$
 \mathrm{SIR}= \fr{\phi(r_0)}{\sum\nolimits_{n=1}^{N-1} \phi(d_n)}\,,
 $$  
где~$r_0$~--- расстояние между приемником и~передатчиком в~исследуемой 
ассоциированной паре; $d_n$~--- расстояние между приемником из 
исследуемой пары и~передатчиком из $n$-й интерферирующей пары, 
$n\hm=1,\ldots ,N\hm-1$. Также задан порог SIR$^*$, определяющий 
минимальное значение SIR, необходимое для поддержания связи 
в~ассоциированной паре. 
  
  В описанной модели случайное расстояние между приемником 
и~передатчиком в~некоторой ассоциированной паре, а также случайные 
расстояния между приемником из этой ассоциированной пары 
и~интерферирующими передатчиками из других пар, работающими на близких 
частотах, исследуются как функции от времени. Эти расстояния определяют 
суммарную интерференцию и~SIR как функционалы на 
траекториях движения взаимодействующих устройств. При падении  
SIR на приемнике исследуемой ассоциированной пары ниже порогового 
значения~SIR$^*$ происходит так называемый <<обрыв связи>>~--- передача 
данных от передатчика к~приемнику в~этой паре прерывается до момента, когда 
 SIR вновь превысит данный порог. Для случайного процесса 
с~независимыми приращениями, описывающего изменение SIR, 
представляют интерес такие  
ве\-ро\-ят\-ност\-но-вре\-мен\-н$\acute{\mbox{ы}}$е характеристики, как 
вероятность перехода процесса в~одно из состояний ниже порогового значения 
SIR$^*$, а~также длительности периодов пребывания процесса в~множестве 
таких состояний. Таким образом, задачей моделирования является нахождение 
вероятности обрыва связи и~распределения длительности периодов наличия 
и~отсутствия связи.

\section{Метод моделирования траекторий движения устройств}

  Известно использование случайных процессов с~независимыми 
приращениями в~экономике для анализа фондового рынка. При этом подходы,\linebreak 
связанные с~моделированием траектории случайного процесса~-- 
временн$\acute{\mbox{о}}$го ряда, отражающего\linebreak флуктуации параметров 
фондового рынка (курсов ценных бумаг или биржевых фондовых  
индексов),~--- использованы при разработке метода моделирования траекторий 
для нестационарного движения устройств. В~\cite{15-g1} был предложен метод 
генерации траектории случайного процесса с~нестационарной функцией 
распределения (ФР) при известном уравнении эволюции этой функции.  
В~\cite{16-g1} этот метод был расширен на генерацию ансамбля траекторий, 
распределение которых эволюционирует в~соответствии с~заданным 
кинетическим уравнением. В~\cite{17-g1} представлена структура 
программного комплекса, реализующего задачу генерации ансамбля случайных 
нестационарных траекторий и~анализа функционалов на них. 

Опишем кратко 
методологию проводимого далее анализа в~обозначениях~\cite{14-g1}, следуя 
указанным работам.
  
  Пусть плотность ФР~$f(x,t)$ приращений 
координат~$x$ положений при\-емо-пе\-ре\-да\-ющих устройств в~момент 
времени~$t$ удовле\-тво\-ря\-ет уравнению Фок\-ке\-ра--План\-ка ($M\hm=1$):
  \begin{multline}
  \fr{\partial f(x,t)}{\partial t} +\fr{\partial}{\partial x}\left( u(x,t) 
  f(x,t)\right)- {}\\
  {}-
\fr{\alpha(t)}{2} \,\fr{\partial^2 f(x,t)}{\partial x^2}=0\,.
  \label{e1-g1}
  \end{multline}
  
  Это уравнение решается численно при заданных начальном и~граничном 
условиях. Параметры уравнения~--- скорость сноса $u(x,t)$ и~нестационарный 
в~общем случае неотрицательный коэффициент диффузии~$\alpha(t)$~--- 
определяются по наблюдаемым значениям временн$\acute{\mbox{о}}$го ряда. Для этого строится 
совместное распределение $F(x,v,t)$ значений временн$\acute{\mbox{о}}$го ряда $x(t)$ и~его 
приращений $v(t)\hm= x(t+1)\hm- x(t)$ по выборке длины, достаточной для 
конструирования такого распределения на заданном уровне 
стационарности~\cite{17-g1}, после чего величины, входящие в~(\ref{e1-g1}), 
определяются по формулам:

\noindent
  \begin{align*}
    f(x,t) &=\displaystyle \int F(x,v,t)\,dv\,;\\
  u(x,t) f(x,t) &= \displaystyle \int vF(x,v,t)\,dv\,;\\
  \alpha(t) &= 2\mathrm{cov}_{x,v}(t) -\fr{d\sigma^2(t)}{dt}\,,
  %  \label{e2-g1}
  \end{align*}
где $\mathrm{cov}_{x,v}(t)$~--- ковариация приращения координаты $x(t)$ и~скорости сноса $u(x,t)$:
\begin{multline*}
\mathrm{cov}_{x,v}(t)=\int xv F(x,v,t)\,dx dv -{}\\
{}- \int xF(x,v,t)\, dx dv  \cdot \int vF(x,v,t)\,dx dv\,,
\end{multline*}
а~дисперсия приращений координат определяется как 
$$
\sigma^2(t)= \int  \left(x-\overline{x}(t)\right)^2 f(x,t)\,dx\,.
$$

При двумерном или трехмерном 
моделировании блуж\-да\-ние по каждой координате считалось независимым. 
В~отсутствие достоверных экспериментальных данных в~данной работе 
параметры сноса и~диффузии были построены по моделям типичных 
нестационарных процессов, описывающих изменения положений случайно 
блуждающих объектов, обсуждаемым в~\cite{17-g1}. Примеры стандартных 
моделей движения, традиционно использующихся для описания перемещения 
абонентов беспроводной сети, приведены в~заключении статьи при 
формулировке задач дальнейших исследований. 

  Величина~$x$ приращения координат для удобства нормирована на~1, т.\,е.\ 
считается, что возможные изменения положений устройств равномерно 
ограничены по времени. Решение строится на временн$\acute{\mbox{о}}$м горизонте $t\hm\in 
[1,T]$ в~дискретном времени с~единичным шагом по времени. На каж\-дом шаге 
$k\hm=1,2,\ldots , T$ для каждого устройства~$n$, $n\hm=1,\ldots , N$, 
генерируется случайное число с~ФР 
$$
F_n(x,t) = F(x,t) = \int\limits_0^x  f(y,t)\,dy\,,
$$
 для чего требуется, чтобы в~результате чис\-лен\-но\-го расчета была 
получена непрерывная строго монотонная функ\-ция. В~част\-ности, если решение 
уравнения~(\ref{e1-g1}) пред\-став\-ле\-но в~виде гистограммы~$f_j(t)$, где~$j$ есть 
номер классового интервала, на которые разбита об\-ласть интегрирования, то 
непрерывная строго монотонная ФР имеет вид: 
  \begin{multline}
  F(x,t) =(Jx-j) f_{j+1}(t) +\sum\limits^j_{l=1} f_l(t)\,,\\
  x\in \left[\fr{j-1}{J}; \fr{j}{J}\right]\,,\ j=1,\ldots , J\,.
  \label{e3-g1}
  \end{multline}
  
  Пусть $\mathbf{R}^n(k)\hm= \left( R_m^n(k)\right)_{m=1,\ldots , M}$~--- 
положение $n$-й точки в~момент времени $t\hm=k$, $k\hm= 1, 2,\ldots ,T$,  
в~некоторой области~$V$ в~$M$-мер\-ном пространстве. Здесь нижний 
индекс~$m$ нумерует координаты в~этом пространстве. 

Алгоритм генерации 
случайных чисел, образующих в~совокупности точки $\mathbf{R}^n(k)\hm= 
\mathbf{R}^n(0)\hm+\sum\nolimits^k_{t=1} x^n(t)$ одной из возможных 
траекторий временн$\acute{\mbox{о}}$го ряда на заданном промежутке времени, состоит 
в~следующем. Генерируется стационарный равномерно распределенный на 
$[0;1]$ ряд чисел $\{y_k\}$ длиной~$T$. Отвечающий ему ряд приращений 
координат~$\left\{ x^n(k)\right\}$ положения передающего устройства~$n$, 
$m\hm=1,\ldots ,N$, с~распределением~$F_n(x,t)$ из~(\ref{e3-g1}) строится по 
формуле обращения соответствующей локальной по времени 
ФР, движущейся в~скользящем окне длины~$T$:
  \begin{multline}
  y_k=F_n\left (x^n(k),k\right)\,,\enskip  
  x^n(k)=F_n^{-1} \left( y_k,k\right)\,,\\
   k=1,\ldots, T\,.
  \label{e4-g1}
  \end{multline}
  
  Генерируя набор равномерно распределенных выборок  
$\{y_k\}_{k=1,\ldots, T}$, для каждого $k\hm=1,\ldots, T$ получаем 
соответствующий набор из~$N$ приращений координат $\left\{ 
x^n(k)\right\}_{n=1,\ldots, N}$, которые дают набор 
траекторий~$N$~передающих устройств на временн$\acute{\mbox{о}}$м горизонте $t\hm\in 
[1,T]$. Эти траектории можно рас\-смат\-ри\-вать как выборку из ансамбля решений 
кинетического уравнения. Полученный набор траекторий представляет 
движение передатчиков в~ассоциирован\-ных парах. 
  
  Моделирование траекторий в~ограниченной области~$V$ при трехмерном 
блуждании $M\hm=3$ осуществлялось следующим образом. В~начальный 
момент задавалось сферически симметричное распределение точек в~кубе со 
стороной~$10d$, спа\-да\-ющее ку\-соч\-но-сте\-пен\-ным образом по закону 
$1/(r\hm+2)$ от центра к~граням куба и~равномерное в~каждом кольце 
шириной~$2d$, где~$d$ есть радиус сферы на рис.~1. Это распределение 
точек~--- начальное условие для решения кинетического уравнения~(1) 
и~одновременно начальные положения для траекторий устройств. Задавалась 
скорость изменения ФР в~(1), а также 
коэффициент~$\alpha$, после чего численно решалось уравнение~(1). Затем по 
описанным выше правилам~(\ref{e3-g1}), (\ref{e4-g1}) вычислялись случайные 
значения последовательных приращений координат для каждой траектории, что 
позволяло строить сами траектории с~учетом граничных усло\-вий. При 
рассмотрении движения в~данной работе задавались усло\-вия идеального 
отражения траектории приемника от границы сферы с~цент\-ром в~точке 
расположения передатчика ас\-со\-ци\-иро\-ван\-ной пары, а~также сферы 
ассоциированной пары от границы куба, определяющего ограниченную 
область~$V$, хотя для области можно рассмотреть и~задачу с~источником 
и~стоком.

\section{Пример расчета характеристик интерференции движущихся 
устройств}

  Траектории движения устройств моделируются для сценария~\cite{14-g1}, 
соответствующего перемещению внутри торгового центра $50\times 50$~м 
абонентов, использующих прямое D2D (device-to-device) под\-клю\-че\-ние. Моделирование 
проведено на временном горизонте $t\hm\in [1,T]$ при $T\hm=10^5$ для 
различной плотности устройств ($N\hm = 10$, 30, 50, 100) и~различных средних 
скоростей сноса ($v\hm = 1$, 3, 5, 10, 40~м/c). Были выбраны типичные для 
D2D-со\-еди\-не\-ний параметры системной модели ($\alpha\hm = 2$, $A\hm = 
1$, $\gamma\hm = 3$, $d\hm = 5$~м, $v_{R_x}\hm= 1$~м/с, $\mathrm{SIR}^*\hm= 0{,}01$), 
выполнялось усреднение по реализациям. При моделировании рассматривалось 
число~$C(T)$ обрывов связи и~вероятность~$P^{-}$ обрыва связи между 
взаимодействующими устройствами:
$$
P^- = \lim\limits_{T\to \infty} 
\left(\fr{C(T)}{T}\right)\,,
$$
 т.\,е.\ выброса SIR ниже порогового значения SIR$^*$. 

На рис.~2,\,\textit{а} представлены графики ве\-ро\-ят\-ности~$P^-$ обрыва связи 
в~зависимости от средней скорости~$v$ передвижения устройств 
и~числа~$N$~пар взаимодействующих устройств. Следует отметить,\linebreak что 
увеличение как числа пар устройств, так и~средней скорости их перемещения 
оказывает не\-гативное влияние на устойчивость связи. В~обоих слу\-чаях можно 
отметить логарифмический рост\linebreak ве\-ро\-ят\-ности потери связи. Средние значения 
длительностей $\overline{\tau}^+$ периода наличия и~$\overline{\tau}^-$ 
периода отсутствия связи как функции от средней скорости~$v$ устройств 
и~числа~$N$ взаимодействующих пар представлены на рис.~2,\,\textit{б} и~2,\,\textit{в}. 


  
  
  Анализируя представленные данные для периода наличия связи, следует 
отметить, что среднее значение~$\overline{\tau}^+$ рассматриваемой метрики 
демонстрирует экспоненциальное падение с~увеличением как средней скорости 
перемещения устройств, так и~чис\-ла пар взаимодействующих устройств. Так, 
при значениях скорости перемещения, со\-от\-вет\-ст\-ву\-ющих ско\-рости пешеходов 
(3--5~м/c), средняя длительность~$\overline{\tau}^+$ периода наличия связи 
может достигать~9--10~с при~10~одновременных уста\-нов\-лен\-ных прямых 
соединениях. Однако при скоростях, соответствующих средней ско\-рости 
перемещения автомобилей (30--40~м/c), величина~$\overline{\tau}^+$ не 
превышает~1~с. 



\pagebreak

\end{multicols}

\begin{figure*} %fig2
\vspace*{1pt}
 \begin{center}
 \mbox{%
 \epsfxsize=163.477mm 
 \epsfbox{gai-2.eps}
 }
\end{center}
\vspace*{-11pt}
\Caption{Вероятность $P^-$ обрыва связи~(\textit{а}),
средняя длительность периодов наличия связи~$\overline{\tau}^+$~(\textit{б})
и~средняя длительность периодов отсутствия связи $\overline{\tau}^-$~(\textit{в}):
левый столбец~--- функция от скорости~$v$ 
($N\hm=10$); правый столбец~--- функция от числа пар~$N$ ($v\hm=5$~м/с)}
\vspace*{12pt}
\end{figure*}

\begin{multicols}{2}


  
  Период наличия связи определяет длительность интервала времени до 
прерывания соединения
 для приложений в~реальном времени. Для таких 
приложений при условии поступления запроса на уста\-нов\-ле\-ние соединения 
в~случайный момент времени в~период наличия связи этот интервал совпадает 
с~длительностью периода до первого обрыва связи и,~как отмечено  
в~\cite{14-g1}, имеет показательное распределение со средним значением, 
представленным на рис.~2,\,\textit{б}. Для кэшируемых приложений интервал 
времени до прерывания соединения определяется длительностью как периодов 
наличия связи, так и~периодов отсутствия связи. Если последняя не превышает 
некоторого порогового значения~$\tau^*$, характерного для конкретного 
приложения, то интервал времени до прерывания соединения складывается из 
нескольких последовательных периодов наличия и~отсутствия связи. 

Анализируя графики на рис.~2,\,\textit{в}, следует от\-метить, что 
зависимость средней длительности пери\-о\-да отсутствия связи демонстрирует 
разный характер изменения при увеличении ско\-рости перемещения устройств 
и~числа пар взаимодействующих устройств. Так, при увеличении скорости 
перемещения период отсутствия связи экспоненциально уменьшается. При 
увеличении числа пар взаимодействующих устройств наблюдается линейный 
рост средней длительности периода отсутствия \mbox{связи}.

\section{Заключение}

  В представленной работе предложен метод моделирования траекторий 
движения при\-емо-пе\-ре\-да\-ющих устройств в~беспроводной сети пятого 
поколения с~технологией прямого взаимодействия устройств. С~применением 
предложенного метода проведено имитационное моделирование и~исследованы 
вероятность обрыва связи и~средние значения длительности периодов наличия 
и~отсутствия связи. Полученный в~работе результат по экспоненциальному 
уменьшению средней длительности периода отсутствия связи при увеличении 
ско\-рости перемещения устройств имеет важное практическое значение: это 
уменьшение компенсируется снижением средней длительности периода 
отсутствия связи, значительно увеличивая вероятность того, что в~течение 
этого времени не произойдет прерывания со\-еди\-не\-ния. 

Задачей дальнейших 
исследований остается анализ устойчивости соединения в~зависимости от 
размера буфера оборудования пользователя для кэшируемых приложений. 

В~случае приложений в~реальном времени, для которых вероятность 
прерывания соединения не зависит от буферизации, интерес представляет 
исследование эволюции старших моментов функционала SIR, определяющих 
надежность соединения. 

Еще одной задачей дальнейших исследований является 
анализ характеристик интерференции при моделировании ансамбля траекторий 
для различных моделей движения, применимых для описания перемещения 
абонентов~\cite{18-g1}: броуновское движение (Brownian Motion), движение 
в~случайном на\-прав\-ле\-нии (Random Direction Motion) и~так на\-зы\-ва\-емый 
<<полет Леви>> (L$\acute{\mbox{e}}$vy Flight).

{\small\frenchspacing
 {%\baselineskip=10.8pt
 \addcontentsline{toc}{section}{References}
 \begin{thebibliography}{99}
\bibitem{1-g1}
\Au{Mouly M., Pautet M.\,B.} The GSM system for mobile communications.~--- Washington, DC, 
USA: Telecom Publishing, 1992. 701~p.
\bibitem{2-g1}
\Au{Вишневский В.\,М., Портной~С.\,Л., Шахнович~И.\,В.} Энциклопедия WiMAX. Путь 
к~4G.~--- М.: Техносфера, 2009. 472~с.
\bibitem{3-g1} WCDMA for UMTS: Radio access for third generation mobile communications~/
Eds. H.~Holma, A.~Toskala.~--- Chichester, UK: John Wiley \& Sons, 2005. 478~p.
\bibitem{4-g1}
\Au{Sesia S., Baker~M., Toufik~I.} LTE~--- the UMTS long term evolution: From theory to  
practice.~--- Chichester, UK: John Wiley \& Sons, 2011. 792~p.
\bibitem{5-g1}
\Au{Отт Г.} Методы подавления шумов и~помех в~электронных системах~/
Пер. с~англ.~--- М.: Мир, 1979.  318~с.
(\Au{Ott~H.\,W.} {Noise reduction techniques in electronic systems}.~--- 
New York, NY, USA: John Wiley \& Sons, 1979. 294~p.)

\bibitem{6-g1}
\Au{Rong Z., Rappaport~T.\,S.} Wireless communications: Principles and practice.~--- 1st ed.~--- 
Upper Saddle River, NJ, USA: Prentice Hall. 641~p.
\bibitem{7-g1}
\Au{Baccelli F., Blaszczyszyn~B.} Stochastic geometry and wireless networks~// Found. 
Trends Netw., 2010. Vol.~3. No.\,3-4. P.~249--449
(doi: 10.1561/1300000006); Vol.~4. No.\,1-2. P.~1--312 (doi: 10.1561/1300000026).

\bibitem{8-g1}
\Au{Haenggi M.} Stochastic geometry for wireless networks.~--- Cambridge: Cambridge University 
Press, 2012. 298~p.
\bibitem{9-g1}
\Au{Samuylov A., Ometov~A., Begishev~V., Kovalchukov~R., Moltchanov~D., Gaidamaka~Yu., 
Samouylov~K., Andreev~S., Koucheryavy~Y.} Analytical performance estimation of  
network-assisted D2D communications in urban scenarios with rectangular cells~// 
T.~Emerg. Telecommun.~T., 2015. Vol.~28. Iss.~2. P.~2999-1--2999-15.
doi: 10.1002/ett.2999.
\bibitem{10-g1}
\Au{Samuylov A. Gaidamaka~Yu., Moltchanov~D., Andreev~S., Koucheryavy~Y.} Random 
triangle: A~baseline model for interference analysis in heterogeneous networks~// IEEE 
T.~Veh. Technol., 2015. Vol.~65. Iss.~8. P.~6778--6782.
doi: 10.1109/TVT.2016.2596324.
\bibitem{11-g1}
\Au{Гайдамака Ю.\,В., Самуйлов~А.\,К.} Метод расчета характеристик интерференции двух 
взаимо\-дей\-ст\-ву\-ющих устройств в~беспроводной гетерогенной сети~// Информатика и~её 
применения, 2015. Т.~9. Вып.~1. С.~9--14. doi: 10.14357/19922264150102.
\bibitem{12-g1}
\Au{Гайдамака Ю.\,В., Андреев~С.\,Д., Сопин~Э.\,С., Самуйлов~К.\,Е., Шоргин~С.\,Я.} 
Анализ характеристик интерференции в~модели взаимодействия устройств с~учетом среды 
распространения сигнала~// Информатика и~её применения, 2016. Т.~10. 
Вып.~4. С.~2--10. doi:  10.14357/19922264160401.
\bibitem{13-g1}
\Au{Orlov Yu.\,N., Fedorov~S.\,L., Samuylov~A.\,K., Gaidamaka~Yu.\,V., Molchanov~D.\,A.} 
Simulation of devices mobility to estimate wireless channel quality metrics in 5G net-\linebreak\vspace*{-12pt}

\pagebreak

\noindent
works~// AIP 
Conf. Proc., 2017. Vol.~1863. P.~090005-1--090005-3. doi: 10.1063/1.4992270.
\bibitem{14-g1}
\Au{Гайдамака Ю.\,В., Орлов~Ю.\,Н., Молчанов~Д.\,А., Самуйлов~А.\,К.} Моделирование 
отношения сигнал/ин\-тер\-фе\-рен\-ция в~мобильной сети со случайным блуж\-да\-ни\-ем 
взаимодействующих устройств~// Информатика и~её применения, 2017. Т.~11. Вып.~2. 
С.~50--58. doi:  10.14357/19922264170206.
\bibitem{15-g1}
\Au{Босов А.\,Д., Кальметьев~Р.\,Ш., Орлов~Ю.\,Н.} Мо\-де\-ли\-рование нестационарного 
временного ряда с~за\-данными свойствами выборочного распределения~// Математическое 
моделирование, 2014. Т.~26. №\,3. С.~97--107.
\bibitem{16-g1}
\Au{Орлов Ю.\,Н., Федоров~С.\,Л.} Генерация нестационарных траекторий временного ряда 
на основе уравнения Фок\-ке\-ра--План\-ка~// Труды МФТИ, 2016. Т.~8. №\,2. С.~126--133.
\bibitem{17-g1}
\Au{Орлов Ю.\,Н., Федоров~С.\,Л.} Методы численного моделирования процессов 
нестационарного случайного блуждания.~--- М.: МФТИ, 2016. 112~с.
\bibitem{18-g1}
\Au{Orsino A., Moltchanov~D., Gapeyenko~M., Samuylov~A., Andreev~S., Militano~L., 
Araniti~G., Koucheryavy~Y.} Direct connection on the move: Characterization of user mobility in 
cellular-assisted D2D systems~// IEEE Veh. Technol. Mag., 2016. Vol.~11. Iss.~3. 
P.~38--48. doi:  10.1109/MVT.2016.2550002.
 \end{thebibliography}

 }
 }

\end{multicols}

\vspace*{-3pt}

\hfill{\small\textit{Поступила в~редакцию 07.09.17}}

\vspace*{8pt}

%\newpage

%\vspace*{-24pt}

\hrule

\vspace*{2pt}

\hrule

%\vspace*{8pt}


\def\tit{METHOD OF~MODELING INTERFERENCE CHARACTERISTICS 
IN~HETEROGENEOUS FIFTH GENERATION WIRELESS 
NETWORKS WITH~DEVICE-TO-DEVICE COMMUNICATIONS}

\def\titkol{Method of~modeling interference characteristics 
in~heterogeneous fifth generation wireless 
networks with~D2D %device-to-device 
communications}

\def\aut{Yu.~Gaidamaka$^{1,2}$, K.~Samouylov$^{1,2}$, and~S.~Shorgin$^2$}

\def\autkol{Yu.~Gaidamaka, K.~Samouylov, and~S.~Shorgin}

\titel{\tit}{\aut}{\autkol}{\titkol}

\vspace*{-9pt}


\noindent
$^1$Peoples' Friendship University of Russia (RUDN University), 6~Miklukho-Maklaya Str., 
Moscow 117198, Russian\linebreak
$\hphantom{^1}$Federation

\noindent
$^2$Institute of Informatics Problems, Federal Research Center ``Computer Science 
and Control'' of the Russian\linebreak
 $\hphantom{^1}$Academy of 
Sciences, 44-2~Vavilov Str., Moscow 119333, Russian Federation


\def\leftfootline{\small{\textbf{\thepage}
\hfill INFORMATIKA I EE PRIMENENIYA~--- INFORMATICS AND
APPLICATIONS\ \ \ 2017\ \ \ volume~11\ \ \ issue\ 4}
}%
 \def\rightfootline{\small{INFORMATIKA I EE PRIMENENIYA~---
INFORMATICS AND APPLICATIONS\ \ \ 2017\ \ \ volume~11\ \ \ issue\ 4
\hfill \textbf{\thepage}}}

\vspace*{3pt}


\Abste{The paper shows the construction of the model of the moving of 
interacting devices in heterogeneous wireless networks of the fifth generation with 
the help of the kinetic equation taking into account a~given average speed of the 
devices, their spatial density, and the maximum allowable communication radius. 
A~method for generating trajectories is proposed where the transceivers move 
randomly and the walk is not stationary in general. This is the feature of the study 
which distinguishes the proposed model from previously known models. 
Interference characteristics, including signal--interference ratio (SIR), are studied 
in the form of a~time-continuous random process, the problem of calculating these 
characteristics is proposed to be solved by simulations. It is shown that such 
analysis makes it possible to investigate the probabilistic characteristics of the 
interaction of devices such as signal interruption probability for the 
receiver--transmitter pair, the random variables for the duration of
 the availability period, and the period of absence of communication.}

\KWE{wireless heterogeneous network; signal--interference ratio; 
device-to-device (D2D); motion model; kinetic equation; trajectories generation; network 
efficiency indicators} 


  \DOI{10.14357/19922264170401} 

\vspace*{-3pt}

\Ack
\noindent
This work was financially supported by the Russian Science Foundation 
(project No.\,16-11-10227).




\vspace*{2pt}

  \begin{multicols}{2}

\renewcommand{\bibname}{\protect\rmfamily References}
%\renewcommand{\bibname}{\large\protect\rm References}

{\small\frenchspacing
 {%\baselineskip=10.8pt
 \addcontentsline{toc}{section}{References}
 \begin{thebibliography}{99}
\bibitem{1-g1-1}
\Aue{Mouly, M., and M.\,B.~Pautet.} 1992. \textit{The GSM system for mobile communications}. 
Washington, DC: Telecom Publishing. 701~p.
\bibitem{2-g1-1}
\Aue{Vishnevskiy, V.\,M., S.\,L.~Portnoy, and I.\,V.~Shakhnovich.} 2009. \textit{Entsiklopediya 
WiMAX. Put' k~4G} [Encyclopedia of WiMAX. The way to 4G]. Moscow: Tekhnosfera. 472~p.
\bibitem{3-g1-1}
Holma, H., and A.~Toskala, eds. 2005. \textit{WCDMA for UMTS: Radio access for third 
generation mobile communications}. Chichester, UK: John Wiley \& Sons. 478~p.
\bibitem{4-g1-1}
\Aue{Sesia, S., M.~Baker, and I.~Toufik}. 2011. \textit{LTE~--- the UMTS long term evolution: From 
theory to practice}. Chichester, UK: John Wiley \& Sons. 792~p.
\bibitem{5-g1-1}
\Aue{Ott, H.\,W.} 1979. \textit{Noise reduction techniques in electronic systems}. 
New York, NY: John Wiley \& Sons. 294~p.
\bibitem{6-g1-1}
\Aue{Rong, Z., and T.\,S.~Rappaport.} 1996. \textit{Wireless communications: Principles and 
practice}. 1st ed. Upper Saddle River, NJ: Prentice Hall. 641~p.
\bibitem{7-g1-1}
\Aue{Baccelli, F., and B.~Blaszczyszyn.} 2010. Stochastic geometry and wireless networks. 
\textit{Found. Trends Netw.} 3(3-4):249--449 (doi: 10.1561/1300000006); 4(1-2):1--312 
(doi: 10.1561/1300000026).
\bibitem{8-g1-1}
\Aue{Haenggi, M.} 2012. \textit{Stochastic geometry for wireless networks}. Cambridge: 
Cambridge University Press. 298~p.
\bibitem{9-g1-1}
\Aue{Samuylov, A., A.~Ometov, V.~Begishev, R.~Kovalchukov, D.~Moltchanov, 
Yu.~Gaidamaka, K.~Samouylov, S.~And\-re\-ev, and Y.~Koucheryavy.} 2015. Analytical 
performance estimation of network-assisted D2D communications in urban scenarios with 
rectangular cells. \textit{T.~Emerg. Telecommun.~T.} 28(2):2999-1--2999-15. 
doi: 10.1002/ett.2999. 
\bibitem{10-g1-1}
\Aue{Samuylov, A., Yu.~Gaidamaka, D.~Moltchanov, S.~Andreev, and Y.~Koucheryavy.} 2015. 
Random triangle: A~baseline model for interference analysis in heterogeneous networks. 
\textit{IEEE T. Veh. Technol.} 65(8):6778--6782. doi: 10.1109/TVT.2016.2596324.
\bibitem{11-g1-1}
\Aue{Gaydamaka, Yu.\,V., and A.\,K.~Samuylov.} 2009. Metod rascheta kharakteristik 
interferentsii dvukh vza\-imo\-dey\-st\-vu\-yushchikh ustroystv v~besprovodnoy geterogennoy seti [The 
method of calculation of the characteristics of the interference of two interacting devices in 
a~wireless heterogeneous network]. \textit{Informatika i~ee Primeneniya~--- Inform. Appl.} 
9(1):9--14. doi: 10.14357/19922264150102.
\bibitem{12-g1-1}
\Aue{Gaydamaka, Yu.\,V., S.\,D.~Andreev, E.\,S.~Sopin, K.\,E.~Samuylov, and S.\,Ya.~Shorgin.} 
2016. Analiz kharakteristik interferentsii v~modeli vzaimodeystviya ustroystv s~uchetom sredy 
rasprostraneniya signala [Analysis of the characteristics of the interference in the model of 
interaction between devices taking into account the signal propagation environment]. 
\textit{Informatika i~ee Primeneniya~--- Inform. Appl.} 10(4):2--10. doi: 
10.14357/19922264160401.
\bibitem{13-g1-1}
\Aue{Orlov, Yu.\,N., S.\,L.~Fedorov, A.\,K.~Samuylov, Yu.\,V.~Gaidamaka, and 
D.\,A.~Molchanov.} 2017. Simulation of devices mobility to estimate wireless channel quality 
metrics in 5G networks. \textit{AIP Conf. Proc.} 1863:090005-1--090005-3. 
doi: 10.1063/1.4992270.
\bibitem{14-g1-1}
\Aue{Gaydamaka, Yu.\,V., Yu.\,N.~Orlov, D.\,A.~Molchanov, and A.\,K.~Samuylov}. 2017. 
Modelirovanie otnosheniya signal/interferentsiya v~mobil'noy seti so sluchaynym bluzhdaniem 
vzaimodeystvuyushchikh ustroystv [Modeling the signal--interference ratio in a mobile network with 
moving devices]. \textit{Informatika i~ee Primeneniya~--- Inform. Appl.} 11(2):50--58. 
doi:  10.14357/19922264170206.
\bibitem{15-g1-1}
\Aue{Bosov, A.\,D., R.\,Sh.~Kalmetiev, and Yu.\,N.~Orlov.} 2014. Modelirovanie 
nestatsionarnogo vremennogo ryada s~zadannymi svoystvami vyborochnogo raspredeleniya 
[Sample distribution function construction for non-stationary time-series forecasting]. 
\textit{Matem. mod.} [Mathematical Simulation]. 26(3):97--107. %10.20948/prepr-2016-101.
\bibitem{16-g1-1}
\Aue{Orlov, Yu.\,N. and S.\,L.~Fedorov}. 2016. Generatsiya ne\-sta\-tsi\-o\-nar\-nykh traektoriy 
vremennogo ryada na osnove uravneniya Fokkera--Planka [Generation of non-stationary 
trajectories of the time series based on the Fokker--Planck equation]. 
\textit{Trudy MFTI} 
[MIPT Proc.] 8(2):126--133. %doi: 10.20948/prepr-2017-36.
\bibitem{17-g1-1}
\Aue{Orlov, Yu.\,N., and S.\,L.~Fedorov.} 2016. \textit{Metody chislennogo modelirovaniya 
protsessov nestatsionarnogo sluchaynogo bluzhdaniya} 
[Methods of a~numerical simulation of nonstationary random walk]. Moscow: MIPT. 112~p.
\bibitem{18-g1-1}
\Aue{Orsino, A., D.~Moltchanov, M.~Gapeyenko, A.~Samuylov, S.~Andreev, L.~Militano, 
G.~Araniti, and Y.~Koucheryavy.} 2016. Direct connection on the move: Characterization of user 
mobility in cellular-assisted D2D systems. \textit{IEEE Veh. Technol. Mag.} 11(3):38--48. 
doi:  10.1109/MVT.2016.2550002.
\end{thebibliography}

 }
 }

\end{multicols}

\vspace*{-6pt}

\hfill{\small\textit{Received September 7, 2017}}

%\vspace*{-10pt}

\Contr

\noindent
\textbf{Gaidamaka Yuliya V.} (b.\ 1971)~--- Candidate of Science (PhD) in physics and 
mathematics, associate professor, Peoples' Friendship University of Russia (RUDN University), 
6~Miklukho-Maklaya Str., Moscow 117198, Russian Federation; senior scientist, Institute of 
Informatics Problems, Federal Research Center ``Computer Science and Control'' of the Russian 
Academy of Sciences, 44-2~Vavilov Str., Moscow 119333, Russian Federation; 
\mbox{gaydamaka\_yuv@rudn.university}

\vspace*{3pt}

\noindent
\textbf{Samouylov Konstantin E.} (b.\ 1955)~--- Doctor of Science in technology, professor, 
Head of Department, 
Director of Institute of Applied Mathematics and Telecommunications, 
Peoples' Friendship University of Russia (RUDN University), 6~Miklukho-Maklaya Str., Moscow 
117198, Russian Federation; senior scientist,
 Institute of Informatics Problems, Federal Research 
Center ``Computer Science and Control'' of the Russian Academy of Sciences, 44-2~Vavilov Str., 
Moscow 119333, Russian Federation; \mbox{samuylov\_ke@rudn.university}

\vspace*{3pt}

\noindent
\textbf{Shorgin Sergey Ya.} (b.\ 1952)~--- Doctor of Science in physics and mathematics, 
professor; Deputy Director, Federal Research Center ``Computer Science and Control'' of the 
Russian Academy of Sciences (FRC CRC RAS); principal scientist, Institute of Informatics 
Problems, FRC CRC RAS; 44-2~Vavilov Str., Moscow 119333, Russian Federation; 
\mbox{sshorgin@ipiran.ru}

\label{end\stat}


\renewcommand{\bibname}{\protect\rm Литература} 