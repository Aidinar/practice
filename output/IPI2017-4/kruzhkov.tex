\renewcommand{\figurename}{\protect\bf Figure}
\renewcommand{\tablename}{\protect\bf Table}

\def\stat{kruzhkov}

\def\tit{APPROACHES TO ANNOTATION OF~DISCOURSE RELATIONS 
IN~LINGUISTIC CORPORA}

\def\titkol{Approaches to annotation of~discourse relations 
in~linguistic corpora}

\def\autkol{M.\,G.~Kruzhkov}

\def\aut{M.\,G.~Kruzhkov$^1$}

\titel{\tit}{\aut}{\autkol}{\titkol}

\index{Kruzhkov M.\,G.}
\index{Кружков М.\,Г.}

%{\renewcommand{\thefootnote}{\fnsymbol{footnote}}
%\footnotetext[1] {This work was supported in part by the
%Russian Foundation for Basic Research (grants 15-07-03007 and 13-07-00223).}}

\renewcommand{\thefootnote}{\arabic{footnote}}
\footnotetext[1]{Institute of Informatics Problems, Federal Research Center ``Computer Science and Control'' of 
the Russian Academy of Sciences, 44-2 Vavilov Str., Moscow 119333, Russian Federation}


\vspace*{12pt}

\def\leftfootline{\small{\textbf{\thepage}
\hfill INFORMATIKA I EE PRIMENENIYA~--- INFORMATICS AND APPLICATIONS\ \ \ 2017\ \ \ volume~11\ \ \ issue\ 4}
}%
 \def\rightfootline{\small{INFORMATIKA I EE PRIMENENIYA~--- INFORMATICS AND APPLICATIONS\ \ \ 2017\ \ \ volume~11\ \ \ issue\ 4
\hfill \textbf{\thepage}}}
  
   
% \vspace*{10pt} 
  
  
  
  \Abste{This paper examines the Supracorpora Database of Connectives  
(SCDB-Connectives) that is based on data from parallel corpora. The  
SCDB-Connectives provides structural and semantic annotation of Russian 
connectives and their translation correspondences in French (and, eventually, in other 
languages). The SCDB-Connectives annotation approach is compared to the latest 
developments in the area of annotation of discourse relations~--- the annotated corpus 
of discourse relations Penn Discourse Treebank (PDTB) and the proposed standard 
for annotation of semantic relations ISO 24617-8, some of the important differences 
are discussed. Penn Discourse Treebank and ISO 24617-8 support annotation of both explicit
and implicit discourse relations 
while SCDB-Connectives only annotates explicit relations, 
i.\,e., those expressed by connectives. Furthermore, PDTB and ISO 24617-8 provide 
a~superior framework for annotating text spans as relation arguments, which allows 
annotating attribution for these arguments, such as source and type of the linked 
propositions. In addition, ISO 24617-8 specifies argument roles for asymmetrical 
discourse relations. On the other hand, the principle advantage of the  
SCDB-Connectives is that it supports annotation of both connectives and their translation 
correspondences in parallel corpora, opening up new possibilities for contrastive 
studies. The SCDB-Connectives is based on a~relational database rather than on the 
XML format, which helps to manage complex cross-linguistic data efficiently. 
Benefits of semantic annotation of connectives for both theoretical and practical 
purposes are also discussed.}
  
  
  \KWE{discourse relations; discourse connectives; corpus linguistics; parallel 
corpora; supracorpora databases}

\DOI{10.14357/19922264170415}

\vspace*{9pt}


\vskip 12pt plus 9pt minus 6pt

      \thispagestyle{myheadings}

      \begin{multicols}{2}

                  \label{st\stat}
  
  \section{Introduction}
  
  \noindent
  As a~part of the project ``Logical structure of text: Means for the expression of 
logical-semantic relations in Russian, French, and Italian from the contrastive 
perspective'' (funded by the Russian Science Foundation, grant No.\,16-18-10004), 
computer-aided semantic and structural annotation of Russian connectives in 
bilingual parallel corpora of the Russian National Corpus ({\sf http://ruscorpora.ru}) is 
being performed (Russian--French and French--Russian corpora are mostly used). 
{\looseness=1

}

For 
each Russian connective, a~corresponding French fragment is annotated in the 
parallel corpus. Usually, parallel French fragments also include connectives, but in 
some cases, Russian connectives may correspond to other lexical or grammatical 
items and occasionally, there is no apparent correspondence in French. These pairs of 
Russian and French fragments (hereafter referred to as ``translation 
correspondences,'' TCs) are annotated by linguistic experts and saved in a~dedicated 
database called The Supracorpora Database of Connectives.
  
  Thus, the SCDB are created as customizable banks of 
annotated translation correspondences that can be used for contrastive analysis of 
various linguistic items. 
{\looseness=1

}

The core SCDB concepts have been described in previous 
works (see, for example,~[1--3]). The goal of this paper is to examine other existing approaches 
to annotation of connectives and discourse relations in corpora and to compare them 
to the approaches used in the SCDB-Connectives. 

In section~2, existing approaches to 
annotation of discourse relations are examined. The most prominent of these is the 
one adopted by the PDTB, a~monolingual collection of 
annotated discourse relations in a~corpus of journal articles. 
The PDTB project proposed 
a~model for annotation of discourse relations that has spawned a~number of similar 
efforts. The proposed standard ISO 24617-8 that was developed in recent years in 
order to create a~common foundation for annotation of discourse relations is also 
examined.

 In section~3, approaches of PDTB and ISO 24617-8 are compared to the 
methods adopted in the SCDB-Connectives.
  
  \section{Existing Approaches to~Annotation of~Discourse~Relations}
  
  \noindent
  In recent years, there has been a~considerable effort towards computer-aided 
annotation of complex linguistic items that at the present cannot be reliably annotated 
by automated annotation procedures (e.\,g., discourse markers, connectives, relations 
between textual segments, etc.). Morphological annotation of individual words in 
corpora is usually implemented automatically (see, for example,~[4, p.~34--37]). On the other 
hand, annotation of complex intertextual entities is often carried out manually. This is 
due to the fact that no accepted annotation standards exist for many of such entities 
and, more importantly, that currently such annotation cannot be implemented 
algorithmically since analysis of such entities often proves to be daunting even for 
seasoned linguistic researchers.
  
  Penn Discourse Treebank is one of the most prominent projects for 
annotation of intertextual entities. It is a~large-scale collection of annotated discourse 
relations, both explicit and implicit, found in a~million-word corpus of Wall Street 
Journal articles. Penn Discourse Treebank was first released 
to the public in 2006 (version PDTB~1.0) 
and in 2008, a~new version of the project was released~---  
PDTB~2.0~\cite{5-kr, 6-kr}. Being the first large-scale project dealing with 
annotation of discourse relations in texts, during the following years, PDTB served as 
a~reference model for a~number of future similar projects.
  
  In PDTB, annotation of discourse relations includes three aspects:
  \begin{enumerate}[(1)]
  \item structure: relation signals (connectives) and arguments are annotated as text 
spans in the corpus (it is assumed that all discourse relations have exactly two 
arguments);
  \item semantics: discourse relations are assigned semantic labels; and
  \item attribution: discourse properties of relations and their arguments are 
annotated, including such properties as sources of relation and their arguments 
(writer/another agent), types of the arguments (assertion/belief/event/fact, etc.), 
polarity (positive/negative), determinacy. In majority of later projects that adopted 
PDTB model, only the first two aspects (structure and semantics of discourse 
relations) have been annotated: apparently, attribution of discourse relations and their 
arguments turned out to be a~nontrivial and time-consuming task.
  \end{enumerate}
  
  Both explicit and implicit discourse relations are annotated in PDTB. Explicit 
relations are those realized by explicit connectives, such as subordinating 
conjunctions (\textit{because}, \textit{when}), coordinating conjunctions 
(\textit{and}, \textit{or}) or adverbials (\textit{for example}, \textit{instead}). 
Complex connectives are also annotated, including modified forms of connectives 
(\textit{only because}), conjoined connectives (\textit{if and when}) and parallel 
connectives (\textit{either\ldots or}, \textit{on one hand\ldots\ on the other hand}).
  
  Implicit relations are annotated between each pair of adjacent sentences within 
paragraphs where there is no connective present. The PDTB methodology introduced the 
so-called lexically-grounded approach to annotation of implicit relations: annotators 
have to label them with specific connectives that best signal these relations when 
inserted between the adjacent sentences  (see, e.\,g.,~Example~1). 
  
  Example~1. \textit{In July, the Environmental Protection Agency imposed a~gradual ban 
on virtually all uses of asbestos}. (implicit\;=\;\underline{as a~result}). \textit{By~1997, 
almost all remaining uses of cancer-causing asbestos will be outlawed}.
  
  When an annotator cannot insert a~connective between the adjacent sentences, one 
of the three relation labels is used: AltLex, EntRel, or NoRel. AltLex (Alternative 
Lexicalization) signifies that the discourse relation between the adjacent sentences is 
already signaled in the text by an alternative lexical device (not a~connective); EntRel 
signifies that only an entity-based relation can be inferred between the sentences; 
and NoRel is used when annotators cannot see neither an Alternative Lexicalization, nor 
an Entity-relation between the adjacent sentences.
  
The  PDTB concept implies that discourse relations can hold between two and only two 
arguments, each argument being a~span of text that can be interpreted as 
a~proposition, eventuality, belief, etc.\ (what Asher calls \textit{abstract objects} 
in~\cite{7-kr}). In English, such abstract objects are usually conveyed by one or 
several sentences or clauses and sometimes by nominalizations and verb phrases. 
When discourse relations are annotated in PDTB, these arguments are labeled as 
Arg1 and Arg2. For explicit relations, Arg2 is the argument that is syntactically 
bound to the explicit connective; for implicit relations, Arg1 is the argument that 
comes first and Arg2 is the argument that comes second. Arguments Arg1 and Arg2 are 
limited to the minimal text needed for interpretation of the discourse relation (this is 
known as ``the minimality principle''). When a~larger context is required for better 
understanding of a~relation, supplementary materials for the arguments can also be 
annotated (labelled as Sup1 for Arg1 and Sup2 for Arg2).
  
  For semantic annotation of discourse relations, a~three-level typology of semantic 
labels (sense tags) is adopted in PDTB (Fig.~1). The discourse relations are 
divided into four classes (temporal, contingency, comparison, and expansion). 
Inside each 
class, the relations are further partitioned into several types and most types have 
several subtypes. Each discourse relation can be\linebreak\vspace*{-12pt}

\pagebreak

\end{multicols}
  \begin{figure*} %fig1
  \vspace*{1pt}
 \begin{center}
 \mbox{%
 \epsfxsize=115.088mm 
 \epsfbox{kru-1.eps}
 }
 \end{center}
\vspace*{-13pt}
  \Caption{The PDTB 2.0 sense hierarchy~\cite{5-kr}}
  \vspace*{-4pt}
   \end{figure*}

\begin{multicols}{2}

\noindent
 assigned more than one sense tag 
from this hierarchy (see~\cite{8-kr}). When there is a~disagreement between 
annotators on a~lower level, it can be automatically resolved by rolling back to the 
next higher level (e.\,g., Juxtaposition vs.\ Opposition disagreement will be 
automatically resolved by rolling back to the Contrast semantic tag).
  

  
  The developers of PDTB acknowledge that this typology of relation senses has 
some gaps~[9, p.~931--932]. In a~number of later efforts based on PDTB 
methodology, the typology of senses has been modified and extended 
 (see, e.\,g.,~\cite{10-kr, 11-kr, 12-kr}). Variations in semantic typologies result both 
from diverse theoretical assumptions and from distinctions in the scope of research. 
For example, while PDTB covers discourse relations in the corpus of Wall Street 
Journal articles (in English only), the effort reported in~\cite{12-kr} addresses 
various types of English and French discourse markers (not only those that signal 
relation between two text spans) in written and spoken texts.
{\looseness=-1

}
  
  A number of efforts have been implemented following the model established by 
PDTB (many of them listed in~[9,  p.~934]. As a~part of these efforts, 
annotated corpora of connectives and discourse relations have been created for texts 
in different genres and languages (including Arabic, Chinese, Turkish, Hindi, Czech, 
and French). In most of these corpora, PDTB's senses typology was adjusted but 
otherwise they adopted the same general approach to annotation of discourse relation 
as proposed by the PDTB framework. In particular, PDTB's lexically-grounded approach 
was implemented for annotation of implicit relations (annotators must insert 
connectives that best signal relations between adjacent sentences). 
  
  In recent years, a~standard for annotation of discourse relations is being developed 
(ISO 24617-8,~\cite{13-kr}). The authors of the standard aim to bring together the 
best practices used in corpora with annotated discourse relations created up to date. 
They propose a~structure that should support annotation of discourse relation in 
a~uniform theory-neutral way. A~set of~20~core relations is proposed for semantic 
labeling of the most common discourse-level relations, although this set does not aim 
to be exhaustive and allows for future extensions. As opposed to PDTB senses 
typology, this list is not organized as a~hierarchy; the core relations are initially 
independent but researches are free to group them into functional domains according 
to their vision.
  
  For asymmetrical relations, the argument roles are specified in ISO 24617-8. Each 
asymmetrical relation in the core set is supplied with appropriate argument role 
labels. For example, for discourse relation \textit{Cause} argument roles 
\textit{Reason} and \textit{Result} are specified, for discourse relation 
\textit{Purpose}~--- argument roles \textit{Goal} and \textit{Enablement}. This 
allows for unification of otherwise semantically identical relations with reversed 
order of arguments (for example, in PDTB~2.0, there are such pairs of relation sense 
tags as \textit{Reason} and \textit{Result}, \textit{Precedence} and 
\textit{Succession}).
  
  In order to distinguish between ideational and rhetorical variants of the same 
relation, ISO 24617-8 proposes to specify types of arguments, which can be either 
situations (eventualities, facts, conditions, etc.)\ or dialog acts. This also allows for 
removal of doubling (rhetorical and nonrhetorical) relation sense tags present in 
PDTB~2.0 (see Fig.~1). ISO 24617-8 also allows to specify the source of relation and 
its arguments (which can be either the producer of the text or another actor).

  
  The overall metamodel for annotation of discourse relations proposed in ISO 
24617-8 is presented in Fig.~2. Indication~$0\ldots1$ at the tip of the arrow from 
discourse relation to markables signifies that the discourse relation
 can be either 
explicit (with an explicit connective in the text) or implicit (no markable in the text). 
One thing that is not quite clear with ISO 24617-8 is how it proposes to deal with 
such cases when a~discourse relation can combine properties of more than one 
relation.
  
  
   { \begin{center}  %fig2
 \vspace*{16pt}
 \mbox{%
 \epsfxsize=75.108mm 
 \epsfbox{kru-2.eps}
 }


\end{center}


\noindent
{{\figurename~2}\ \ \small{ISO 24617-8 metamodel for annotation of discourse relations}}
}

%\vspace*{6pt}

\addtocounter{figure}{1}
 
  
  A few works were dedicated to annotation of connectives and discourse markers in 
parallel and comparable corpora~\cite{11-kr, 12-kr}, although currently, there are no 
large-scale parallel corpora with annotated discourse relations. For example, 
in~\cite[p.~4--5]{11-kr},  the authors maintain that for initial assessment of  
cross-language equivalents, it is mandatory to use parallel corpora, but such 
assessment is problematic in practice because parallel corpora are not entirely reliable 
and often limited to specific genres. That is why the authors propose to use small 
parallel corpora at the initial stage of a~cross-linguistic research and to verify their 
findings at later stages using larger-scale comparable multilingual corpora.

\vspace*{-7pt}
  
  \section{Supracorpora Database of~Connectives in~Contrast with~Other Approaches}
  
  \vspace*{-2pt}
  
  \noindent
  In supracorpora databases (and in the SCDB-Connectives, in particular), the 
minimal context required for interpretation of the linguistic item (connective) is 
included in the annotation. In addition, within this minimal context, the main words 
(those that are part of the connective) and the so-called functional words (those that 
may influence semantics of the connective or/and represent pragmatics or situational 
context) are annotated. However, the actual arguments of the discourse relations 
(Arg1 and Arg2 in the PDTB framework) are not explicitly annotated in the  
SCDB-Connectives. On one hand, this is a~drawback, but on the other hand, there is 
no restriction on the number of arguments in the system, which allows researchers to 
annotate connectives that join together more than two arguments. This cannot be handled by PDTB and ISO  
24617-8, even though one must acknowledge that such multipart connectives are not 
very frequent: only~41~tokens in our corpus, including such types as \textit{не 
только \ldots но даже \ldots а иногда и \ldots}; \textit{не \ldots и не \ldots 
а~просто \ldots}; \textit{хотя \ldots хотя \ldots однако все же \ldots}; \textit{не 
только \ldots но даже и \ldots и даже \ldots}\footnote{Literal English translation: 
\textit{not only \ldots but even \ldots and sometimes also \ldots}; \textit{not \ldots and not \ldots but just 
\ldots}; \textit{though \ldots though \ldots but \mbox{still \ldots}}; \textit{not only \ldots but even also \ldots and 
even \ldots}} (see Example~2).
  
 
  Example~2.~$\langle\ldots\rangle$~\textit{ужасно стыдно мне стало, когда я наконец 
догадался (вдруг как-то), что} {\bfseries\textit{не только}} \textit{его не 
покоробило бы,} {\bfseries\textit{но даже и}} \textit{в~голову бы ему не пришло, 
что это не монументально}\ldots {\bfseries\textit{и~даже}} \textit{не понял бы он 
совсем: чего тут коробиться?}\footnote[2]{Literal English translation: \textit{I~felt terribly 
ashamed when I~finally realized (quite suddenly) that not only he would not have been shocked}, 
{\bfseries\textit{but even also}} \textit{it would not have crossed his mind that this was not 
monumental}\ldots {\bfseries\textit{and even}} \textit{he would not have understood what was there to be 
shocked about?} (\Aue{F.\,M.~Dostoyevsky}. {Crime and punishment})} 
({Ф.\,М.~Достоевский}. Преступление и наказание)
  
  As opposed to PDTB and similar to ISO 24617-8, the typology of sense labels in 
the SCDB-Connectives is not organized as a~hierarchy. The relations are initially 
independent although they may be grouped into functional domains during a~later 
stage (see, e.\,g.,~\cite{14-kr}).
{\looseness=1

}
  
  The most important distinction of the SCDB-Connectives that sets it apart from 
similar efforts is that it supports annotation of connectives in large-scale parallel 
corpora. The SCDB-Connectives operates with parallel corpora of the  
Russian National Corpus that are both comparatively large and of high quality since they only 
include literary translations made by professional translators. These corpora mostly 
include fiction, but they are gradually expanding to include other genres (official 
documents, philosophical literature, etc.). Currently, most connectives are annotated in 
Russian--French ($> 3$~million words) and French--Russian ($> 600$~thousand 
words) corpora but further annotation efforts are planned for Russian--Italian and 
Russian--German corpora. The SCDB-Connectives takes advantage of parallel corpora 
structure allowing users to create annotated TCs. Each 
TC includes annotation of a~Russian connective and annotation of the corresponding 
French fragment that often (but not always)  also includes a~connective. 
  
  Simultaneous parallel annotation of discourse relations in Russian and in other 
languages makes it possible to conduct comparative analysis of discourse relations in 
the corresponding language pairs. This kind of analysis is unique because it allows 
researchers to test validity and cross-lingual universality of the proposed 
classifications of logical-semantic relations based on data from parallel texts.
  
  As of today, over 16,5 thousand TCs of various types have been created in the 
SCDB-Connectives for more than~11,5~thousand of connectives (the reason for the 
discrepancy is that for some texts, multiple translations are available in the corpus). 
There are~853~types of Russian connectives registered in the database, which are 
organized in~136~clusters. This is significantly more types than in PDTB~2.0 where 
only~100~distinct types of explicit connectives are registered (155~types if modified 
forms and variants are taken into account). These figures emphasize a~much higher 
variability and structural complexity of Russian connectives as opposed to the English 
ones. For example, in PDTB~2.0, there are~3000 tokens of `\textit{and}' 
and~1746~tokens of `\textit{also},' but not a~single modified or conjoined form of 
either of them. Meanwhile, in Russian, there is a~great variety of widely used 
conjoined/modified forms for both `\textit{и}' (\textit{and}) and `\textit{также}' 
(\textit{also}), including such types as \textit{и также}, \textit{а~также}, 
\textit{а~также~и}, \textit{и~к~тому же}\footnote{Literal English translation: \textit{and 
also}; \textit{but also}; \textit{but also and}; 
\textit{and in addition}.}, etc.
  
  The SCDB-Connectives is based on the SCDB concept that has been described in 
more detail in previous works (see, e.\,g.,~[1--3]). As a~part of other projects, several 
SCDB banks of translation correspondences were created for cross-linguistic study of 
such items as personal verbal forms, impersonal verbal forms, language-specific 
words, and discourse markers.
  
  Penn Discourse Treebank, ISO 24617-8, and most of the other annotation efforts are based on XML 
data format. The SCDB-Connectives, like other SCDBs, was developed as 
a~relational database, which enables its data to be well organized. Relational 
databases support structured query language (SQL) offering researchers powerful features when it 
comes to searching and generating statistics. 
  
  A server-based relational database increases the accessibility of the  
SCDB-Connectives to the users. Annotators and linguistic experts can simultaneously 
work with the same data from anywhere using their favorite web-browsers. In 
addition, the structure and interface of the database allow researchers to modify the 
annotation schemes easily by inserting new properties into appropriate tables or by 
altering descriptions of the existing ones. 
{\looseness=-1

}
  
  Eventually, SCDBs will integrate new features, such as a~possibility to save history 
of changes made to annotation schemes in SCDBs, and the underlying relational 
database will significantly facilitate such tasks. By tracking changes to annotation 
schemes, SCDBs will be able to contribute to development of knowledge generation 
theory~\cite{15-kr, 16-kr, 17-kr}). 
  
  Finally, in the context of contrastive linguistic projects related to studying of 
connectives, one should also mention the GECCo (German--English Contrasts in Cohesion)
project (see, e.\,g.,~\cite{18-kr}). 
The project's goal was to produce a~German--English corpus for contrastive linguistic 
work in the area of textual cohesion. The GECCo corpus includes texts in more than 
a~dozen registers, both spoken and written, English and German, allowing to study 
relevant distribution of cohesive devices in English and German. The main difference 
between the GECCo project and SCDB-Connectives is that the GECCo covers 
a~much wider range of types of cohesion, including reference, substitution, ellipsis, 
conjunction (connectives fall into this category), and lexical cohesion, while  
SCDB-Connectives deal only with connectives. As a~result, the GECCo corpus does 
not pay enough attention to annotation of discourse relations, mostly concentrating on 
other types of textual cohesion. Only~5~semantic types of conjunctions are annotated 
in the GECCo: additive, adversative, temporal, causal, and modal. In addition, only 
a~fraction of the texts in the GECCo corpus are parallel texts: the GECCo corpus is 
created and used mostly as a~comparable corpus rather than a~parallel one. 
Accordingly, while the GECCo corpus allows studying of relative frequencies of 
cohesion devices across various registers, it does not support unidirectional 
methodology of SCDB, which is based on analysis of translation correspondences.

\vspace*{-6pt}
  
  \section{Concluding Remarks}
  
  \vspace*{-2pt}
  
  \noindent
  Interpretation of explicit connectives is vital for high-level understanding of text as 
a~whole because they signal discourse relations between larger text segments 
effectively ``binding'' the text together. For this reason, semantic annotation of 
connectives plays a~major role in such areas as natural language processing (NLP) 
and machine translation. A~number of works~\cite{19-kr, 20-kr, 21-kr} brought 
evidence that integrating annotated sense labels for discourse connectives can 
improve results of factored statistical machine translation (SMT). As a~part of the 
above-mentioned work, the authors tested algorithms for automatic disambiguation of 
discourse connectives prior to SMT, which resulted in significantly improved scores 
for quality of machine translation. The authors stressed that disambiguation of 
connectives is a~more challenging task than regular word sense disambiguation 
(WSD) because connectives labeling requires more structured and longer-range 
information. Besides, modeling of content word senses differs considerably from 
modeling of the procedural meaning of function words. The authors note that to 
improve quality of automatic disambiguation of connectives, it is vital to have access 
to reliable large-scale annotated data for training and testing purposes. 
  
  While the primary goal of this paper was to examine and compare existing 
approaches to annotation of connectives and discourse relations, it also aimed to 
bring evidence that the SCDB-Connectives provides valuable data on Russian 
connectives and their correspondences in French (and, eventually, in other languages) 
that can be useful for both theoretical and practical purposes. As far as we can say, it 
is the largest-scale corpus of annotated connectives in parallel texts. Nevertheless, 
some aspects of the SCDB-Connectives require improvement to make it more useful 
for researchers. More detailed descriptions of sense labels should be provided for the 
SCDB-Connectives, including descriptions of discourse relations and their 
arguments. It will be also useful to provide mapping between relation senses used in 
the SCDB-Connectives and the core set of relations in ISO 24617-8. Finally, 
following the guidelines of PDTB, it makes sense to explicitly annotate the 
arguments of discourse relations signaled by the connectives and to specify their roles 
and types (similar to the ISO 24617-8 proposal).

\vspace*{-6pt}
  
  \Ack
  
  \vspace*{-2pt}
  
  \noindent
  The work was carried out at the Institute of Informatics Problems (FRC CSC RAS) and funded 
by the Russian Science Foundation according to the research project No.\,16-18-10004.
  
  \renewcommand{\bibname}{\protect\rmfamily References}
  
  \vspace*{-6pt}


{\small\frenchspacing
{%\baselineskip=10.8pt
\begin{thebibliography}{99}

\vspace*{-2pt}

  \bibitem{1-kr}
  \Aue{Loiseau, S., D.\,V.~Sitchinava,  Anna~A.~Zalizniak, and I.\,M.~Zatsman}. 
2013. Information technologies for creating the database of equivalent verbal forms 
in the Russian--French multivariant parallel corpus. \textit{Informatika i~ee 
Primeneniya~--- Inform. Appl.} 7(2):100--109.
  \bibitem{2-kr}
  \Aue{Kruzhkov, M., N.~Buntman, E.~Loshchilova, D.~Sitchinava, Anna 
A.~Zalizniak, and I.\,A.~Zatsman} 2014. Database of Russian verbal forms and 
their French translation equivalents. \textit{Computational Linguistics and 
Intellectual Technologies: Conference (International) ``Dialogue 2016'' 
Proceedings}. Moscow: RGGU. 13(20):275--287.
  \bibitem{3-kr}
  \Aue{Kruzhkov, M.} 2016. Supracorpora databases as corpus-based 
superstructure for manual annotation of parallel corpora. \textit{8th 
Conference (International) on  Corpus Linguistics}. EPiC ser.\ in language and 
linguistics. 1:236--248. Available at: {\sf 
https://easychair.org/\linebreak publications/paper/270289} (accessed August~31, 2017).
  \bibitem{4-kr}
  \Aue{Mikhailov, M., and R.~Cooper}. 2016. \textit{Corpus linguistics for 
translation and contrastive studies: A~guide for research.}  
 London\,--\,New York: Routledge. 234~p.
  \bibitem{5-kr}
  \Aue{Prasad, R., N.~Dinesh, A.~Lee, E.~Miltsakaki, L.~Robaldo, A.~Joshi, and 
B.~Webber}. 2008. The Penn Discourse TreeBank~2.0. \textit{6th Conference 
(International) on Language Resources and Evaluation Proceedings}. 
Marrackech, Morocco. 2961--2968.
  \bibitem{6-kr}
  \Aue{Prasad, R., B.~Webber, and A.~Joshi}. 2017. The Penn Discourse Treebank: 
An annotated corpus of discourse relations. \textit{Handbook of linguistic 
annotation}. Springer. 1197--1217.
  \bibitem{7-kr}
  \Aue{Asher, N.} 1993. \textit{Reference to abstract objects.} Dordrecht--Boston: 
Kluwer Academic. 455~p.
  \bibitem{8-kr}
  \Aue{Webber, B.} 2016. Concurrent discourse relations. \textit{Computational 
Linguistics and Intellectual Technologies: Conference (International) ``Dialogue 
2016'' Proceedings}. Moscow. 15(22):D.  Available at: {\sf 
http://www.dialog-21.ru/media/3488/webber.pdf} (accessed August~31, 2017).
  \bibitem{9-kr}
  \Aue{Prasad, R., B.~Webber, and A.~Joshi}. 2014. Reflections on the Penn 
Discourse TreeBank, comparable corpora and complementary annotation. 
\textit{Comput. Linguist.} 40(4):921--950.
  \bibitem{10-kr}
  \Aue{Prasad, R., S.~McRoy, N.~Frid, A.~Joshi,  and H.~Yu}. 2011. The 
biomedical discourse relation bank. \textit{BMC Bioinformatics} 12:188--205.
  \bibitem{11-kr}
  \Aue{Zufferey, S., and L.~Degand}. 2013. Annotating the meaning of discourse 
connectives in multilingual corpora. \textit{Corpus Linguist. Ling.} 13(2):1--24. 
  \bibitem{12-kr}
  \Aue{Cribble, L., and S.~Zufferey}. 2015. Using a~unified taxonomy to annotate 
discourse markers in speech and writing. \textit{11th Conference (International) on 
Computational Semantics Proceedings}. London. 14--22.
  \bibitem{13-kr}
  \Aue{Bunt, H., and R.~Prasad}. 2016. ISO DR-Core (ISO 24617-8): Core 
concepts for the annotation of discourse relations. \textit{12th Joint ACL-ISO 
Workshop on Interoperable Semantic Annotation (ISA-12 Proceedings}). Portoroz. 
45--54.
  \bibitem{14-kr}
  \Aue{Zatsman, I., O.~Inkova, and V.~Nuriev}. 2017. The construction of 
classification schemes: Methods and technologies of expert formation. 
\textit{Automatic Documentation Math. Linguistics} 51(1):27--41.
  \bibitem{15-kr}
  \Aue{Zatsman, I.} 2012. Tracing emerging meanings by computer: Semiotic 
framework. \textit{13th European Conference on Knowledge Management 
Proceedings}. Reading: Academic Publishing International Ltd. 2:1298--1307.
  \bibitem{16-kr}
  \Aue{Zatsman, I., N.~Buntman, M.~Kruzhkov, V.~Nuriev, and Anna 
A.~Zalizniak}. 2014. Conceptual framework for development of computer 
technology supporting cross-linguistic knowledge discovery. \textit{15th European 
Conference on Knowledge Management Proceedings}. Reading: Academic 
Publishing International Ltd. 3:1063--1071.
  \bibitem{17-kr}
  \Aue{Zatsman, I., and N.~Buntman}. 2015. Outlining goals for discovering new 
knowledge and computerised tracing of emerging meanings. \textit{16th European 
Conference on Knowledge Management Proceedings}. Reading: Academic 
Publishing International Ltd. 851--860.
  \bibitem{18-kr}
  \Aue{Lapshinova-Koltunski,~E., and K.~Kunz}. 2014. Annotating cohesion for 
multilingual analysis. \textit{10th Joint ACL-ISO Workshop on Interoperable 
Semantic Annotation Proceedings}. Reykjavik.  57--64.
  \bibitem{19-kr}
  \Aue{Meyer, T., A.~Popescu-Belis, N.~Hajlaoui, and A.~Gesmundo}. 2012. 
Machine translation of labeled discourse connectives. \textit{10th Conference of the 
Association for Machine Translation in the Americas Proceedings}. San 
Diego, CA. Available at:  
{\sf http://publications.idiap. ch/index.php/publications/show/2391} (accessed August~31, 
2017).
  \bibitem{20-kr}
  \Aue{Meyer, T.} 2014. Discourse-level features for statistical machine 
translation. PhD thesis. $\acute{\mbox{E}}$cole Polytechnique 
F$\acute{\mbox{e}}$d$\acute{\mbox{e}}$rale de Lausanne. Available at: {\sf 
http://publications.\linebreak idiap.ch/downloads/papers/2015/Meyer\_THESIS\_2014. pdf} (accessed 
August~31, 2017).
  \bibitem{21-kr}
  \Aue{Meyer, T., N.~Hajlaoui, and A.~Popescu-Belis}. 2015. Disambiguating 
discourse connectives for statistical machine translation. \textit{IEEE-ACM 
T.~Audio Spe.} 23(7):1184--1197.
  \end{thebibliography} }
 }

\end{multicols}

\vspace*{-6pt}

\hfill{\small\textit{Received September~7, 2017}}

\vspace*{-14pt}
  
  \Contrl
  
  \noindent
\textbf{Kruzhkov Mikhail G.} (b.\ 1975)~--- senior scientist, Institute of 
Informatics Problems, Federal Research Center ``Computer Science and Control'' 
of the Russian Academy of Sciences, 44-2~Vavilov Str., Moscow 119333, Russian 
Federation; \mbox{magnit75@yandex.ru}

\vspace*{7pt}

\hrule

\vspace*{2pt}

\hrule

%\newpage

\vspace*{-4pt}



\def\tit{ПОДХОДЫ К АННОТАЦИИ ДИСКУРСИВНЫХ ОТНОШЕНИЙ В~ЛИНГВИСТИЧЕСКИХ КОРПУСАХ$^*$}

\def\aut{М.\,Г.~Кружков}


\def\titkol{Подходы к аннотации дискурсивных отношений в~лингвистических корпусах}

\def\autkol{М.\,Г.~Кружков}

{\renewcommand{\thefootnote}{\fnsymbol{footnote}}
\footnotetext[1]{Работа выполнена в Институте проб\-лем информатики
Федерального исследовательского центра
<<Информатика и~управ\-ле\-ние>> Российской академии наук
при финансовой поддержке РНФ (проект  №\,16-18-10004).}}


\titel{\tit}{\aut}{\autkol}{\titkol}

\vspace*{-10pt}

\noindent
   Институт проблем информатики Федерального исследовательского центра <<Информатика 
и~управление>> Российской академии наук, \mbox{magnit75@yandex.ru}

\vspace*{1pt}

\def\leftfootline{\small{\textbf{\thepage}
\hfill ИНФОРМАТИКА И ЕЁ ПРИМЕНЕНИЯ\ \ \ том\ 11\ \ \ выпуск\ 4\ \ \ 2017}
}%
 \def\rightfootline{\small{ИНФОРМАТИКА И ЕЁ ПРИМЕНЕНИЯ\ \ \ том\ 11\ \ \ выпуск\ 4\ \ \ 2017
\hfill \textbf{\thepage}}}
  



  \Abst{Рассматривается надкорпусная база данных (НБД), разработанная на 
основе корпуса параллельных текстов для описания русских коннекторов и~их 
переводов на французский и~другие языки. В~рамках данной НБД аннотируется 
внутренняя структура и~семантика коннекторов русского языка, а~также их 
переводных соответствий на других языках. Описание семантики коннекторов 
подразумевает описание соответствующих дискурсивных отношений между 
соединяемыми ими фрагментами текста. Используемый в~НБД подход 
к~описанию дискурсивных выражений, передаваемых коннекторами, 
сравнивается с~новейшими существующими подходами к аннотации 
дискурсивных отношений: рас\-смат\-ри\-ва\-ет\-ся аннотированный корпус 
дискурсивных отношений Penn Discourse Treebank (PDTB) и проект стандарта 
по аннотации дискурсивных отношений ISO  
24617-8. Отмечается, что PDTB и~ISO 24617-8, в~отличие от НБД, позволяют 
аннотировать как эксплицитные (выраженные коннекторами и~другими
языковыми единицами), так и~имплицитные дискурсивные отношения. Кроме 
этого, в~рамках данных подходов имеется возможность аннотировать аргументы 
дискурсивных отношений, включая их источники, типы и~роли (для 
ассиметричных отношений). С~другой стороны, преимущество НБД со\-сто\-ит 
в~том, что она позволяет одновременно аннотировать коннекторы и~их 
переводные соответствия в~параллельных корпусах, что открывает для 
исследователей новые возможности в~об\-ласти лингвистического
контрастивного анализа. В~то время как в~рамках других подходов для 
аннотации дискурсивных кон-\linebreak\vspace*{-12pt}}

\Abstend{некторов используется формат XML, НБД 
представляет собой реляционную базу данных, что повышает эффективность 
системы при работе с~кросслингвистическими объектами и~доступность для 
пользователей. Также рассматривается теоретическая и~практическая 
значимость семантической аннотации коннекторов и~выражаемых ими 
дискурсивных отношений.}
  
  \KW{дискурсивные отношения; коннекторы; корпусная лингвистика; 
параллельные корпуса; надкорпусные базы данных}

  
\DOI{10.14357/19922264170415}

%\vspace*{18pt}


 \begin{multicols}{2}

\renewcommand{\bibname}{\protect\rmfamily Литература}
%\renewcommand{\bibname}{\large\protect\rm References}

{\small\frenchspacing
{%\baselineskip=10.8pt
\begin{thebibliography}{99}
  \bibitem{1-kr-1}
  \Au{Loiseau S., Sitchinava~D.\,V., Zalizniak Anna~A., Zatsman~I.\,M.} 
Information technologies for creating the data base of equivalent verbal forms in the 
Russian--French multivariant parallel corpus~// Информатика и её применения, 
2013. Т.~7. Вып.~2. С.~100--109.
  \bibitem{2-kr-1}
  \Au{Kruzhkov M., Buntman~N., Loshchilova~E., Sitchinava~D., Zalisniak 
Anna~A., Zatsman~I.\,A.} Database of Russian verbal forms and their French 
translation equivalents~// Computational Linguistics and Intellectual Technologies:  
Conference (International) ``Dialogue 2016'' Proceedings.~--- Moscow: RGGU, 
2014. Vol.~13(20). P.~275--287.
  \bibitem{3-kr-1}
  \Au{Kruzhkov M.} Supracorpora databases as corpus-based superstructure for 
manual annotation of parallel corpora~// 8th Conference (International) 
on Corpus Linguistics.~--- EPiC ser. in language and linguistics, 2016. Vol.~1. 
P.~236--248. {\sf https://easychair.org/publications/paper/270289}.
  \bibitem{4-kr-1}
  \Au{Mikhailov M., Cooper~R.} Corpus linguistics for translation and contrastive 
studies: A~guide for research.~--- London\,--\,New York: Routledge, 2016. 234~p.
  \bibitem{5-kr-1}
  \Au{Prasad R., Dinesh~N., Lee~A., Miltsakaki~E., Robaldo~L., Joshi~A., 
Webber~B.} The Penn Discourse TreeBank~2.0~//  6th Conference (International) on 
Language Resources and Evaluation (LREC 2008) Proceedings.~--- Marrackech, 
Morocco, 2008.  P.~2961--2968.
  \bibitem{6-kr-1}
  \Au{Prasad R., Webber~B., Joshi~A.} The Penn Discourse Treebank: An 
annotated corpus of discourse relations~// Handbook of linguistic annotation.~--- 
Springer, 2017. P.~1197--1217.
  \bibitem{7-kr-1}
  \Au{Asher N.} Reference to abstract objects.~--- Dordrecht--Boston: Kluwer 
Academic, 1993. 455~p.
  \bibitem{8-kr-1}
  \Au{Webber B.} Concurrent discourse relations, computational linguistics and 
intellectual technologies~// Com\-pu\-tational 
Linguistics and Intellectual Technologies:\linebreak Conference (International) ``Dialogue 2016'' 
Pro\-ceed\-ings.~--- Moscow: RGGU, 2016. Vol.~15(22). {\sf 
http://www. dialog-21.ru/media/3488/webber.pdf}.
  \bibitem{9-kr-1}
  \Au{Prasad R., Webber~B., Joshi~A.} Reflections on the Penn Discourse 
TreeBank, comparable corpora and Ccmplementary annotation~// Comput. 
Linguist., 2014. Vol.~40. No.\,4. P.~921--950.
  \bibitem{10-kr-1}
  \Au{Prasad R., McRoy~S., Frid~N., Joshi~A., Yu~H.} The biomedical discourse 
relation bank~// BMC Bioinformatics, 2011. Vol.~12. P.~188--205.
  \bibitem{11-kr-1}
  \Au{Zufferey S., Degand~L.} Annotating the meaning of discourse connectives in 
multilingual corpora~// Corpus Linguist. Ling., 2013. Vol.~13. Iss.~2. P.~1--24. 
  \bibitem{12-kr-1}
  \Au{Cribble L., Zufferey~S.} Using a~unified taxonomy to annotate discourse 
markers in speech and writing~// 11th Conference (International) on Computational 
Semantics Proceedings.~--- London, 2015. P.~14--22.
  \bibitem{13-kr-1}
  \Au{Bunt H., Prasad~R.}ISO DR-Core (ISO 24617-8): Core concepts for the 
annotation of discourse relations~// 12th Joint ACL-ISO Workshop on Interoperable 
Semantic Annotation Proceedings.~--- Portoroz, 2016. P.~45--54.
  \bibitem{14-kr-1}
  \Au{Zatsman I., Inkova~O., Nuriev~V.} The construction of classification 
schemes: Methods and technologies of expert formation~// Automatic 
Documentation Math. Linguistics, 2017. Vol.~51. No.\,1. P.~27--41.
  \bibitem{150kr-1}
  \Au{Zatsman I.} Tracing emerging meanings by computer: Semiotic framework~// 
13th European Conference on Knowledge Management Proceedings.~--- Reading: 
Academic Publishing International Ltd., 2012. Vol.~2. P.~1298--1307.
  \bibitem{16-kr-1}
  \Au{Zatsman~I., Buntman~N., Kruzhkov~M., Nuriev~V., Zalizniak Anna~A.} 
Conceptual framework for development of computer technology supporting  
cross-linguistic knowledge discovery~// 15th European Conference on Knowledge 
Management Proceedings.~--- Reading: Academic Publishing International Ltd., 
2014. Vol.~3. P.~1063--1071.
  \bibitem{17-kr-1}
  \Au{Zatsman I., Buntman~N.} Outlining goals for discovering new knowledge and 
computerised tracing of emerging meanings~// 16th European Conference on 
Knowledge Management Proceedings.~--- Reading: Academic Publishing 
International Ltd., 2015. P.~851--860.
  \bibitem{18-kr-1}
  \Au{Lapshinova-Koltunski E., Kunz~K.} Annotating cohesion for multilingual 
analysis~// 10th Joint ACL-ISO Workshop on Interoperable Semantic Annotation 
Proceedings.~--- Reykjavik, 2014. P.~57--64.
  \bibitem{19-kr-1}
  \Au{Meyer T., Popescu-Belis~A., Hajlaoui~N., Gesmundo~A.} Machine 
translation of labeled discourse connectives~// 10th Conference of the Association for 
Machine Translation in the Americas Proceedings.~--- San Diego, CA, 
USA, 2012. {\sf http://publications.idiap.ch/index.\linebreak php/publications/show/2391}.
  \bibitem{20-kr-1}
  \Au{Meyer T.} Discourse-level features for statistical machine translation: PhD 
thesis.  $\acute{\mbox{E}}$cole Polytechnique 
F$\acute{\mbox{e}}$d$\acute{\mbox{e}}$rale de Lausanne, 2014. {\sf 
http://publications.idiap.ch/\linebreak  downloads/papers/2015/Meyer\_THESIS\_2014.pdf}.
  \bibitem{21-kr-1}
  \Au{Meyer T., Hajlaoui~N., Popescu-Belis~A.} Disambiguating discourse 
connectives for statistical machine translation~// IEEE-ACM T.~Audio 
Spe., 2015. Vol.~23. No.\,7. P.~1184--1197.
\end{thebibliography}
} }


\end{multicols}

 \label{end\stat}

 \vspace*{-12pt}

\hfill{\small\textit{Поступила в~редакцию  07.09.2017}}

\pagebreak
%\renewcommand{\bibname}{\protect\rm Литература}
\renewcommand{\figurename}{\protect\bf Рис.}
\renewcommand{\tablename}{\protect\bf Таблица} 