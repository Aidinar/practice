\documentclass[12pt, a4paper, oneside]{article}
\usepackage[utf8]{inputenc}
\usepackage[english,russian]{babel}
\usepackage{amsfonts}
\usepackage{amsmath}
\usepackage{amssymb}
\usepackage{amsthm}
\usepackage{graphicx}
\usepackage{indentfirst}%ъЁрёэр  ёЄЁюър т эрўрых Ёрчфхыют
\usepackage{tikz}
\usepackage{cite}
\usetikzlibrary{arrows,automata}
\usepackage[all]{xy} % ─ы  Ёшёютрэш  фшруЁрьь
\usepackage{color}
\usepackage{fancyhdr} % ЁрсюЄр ё ъюыюэЄшЄєырьш
\graphicspath{{picture/}}
\DeclareGraphicsExtensions{.pdf,.png,.jpg}
\textheight=24cm
\textwidth=16cm
\voffset = 1cm
\topmargin=-1.5cm % юЄёЄєя юЄ тхЁїэхую ъЁр 
\oddsidemargin=1cm % юЄёЄєя юЄ ыхтюую ъЁр 
\headsep = 1cm
\footskip = 2cm
\parindent=24pt % рсчрЎэ√щ юЄёЄєя
\parskip=0pt % шэЄхЁтры ьхцфє рсчрЎрьш
\flushbottom % т√Ёртэштрэшх т√ёюЄ√ ёЄЁрэшЎ. ┬ёх ёЄЁрэшЎ√ юфшэръютющ т√ёюЄ√
%\title{\bf ╧Ёшьхэхэшх тхщтыхЄют фы  ЁрёўхЄр ышэхщэ√ї ёшёЄхь єяЁртыхэш   ё ЁрёяЁхфхыхээ√ьш ярЁрьхЄЁрьш}
%\author{▐.╚. ┴шЄ■ъют}
\setcounter{page}{1}
\begin {document}
%\maketitle
\newtheorem{Th}{╥хюЁхьр}[section]% ╤ючфрэшх юъЁєцхэш  Th фы  ртЄюьрЄшўхёъш эєьхЁєхь√ї ЄхюЁхь
\newtheorem{Def}{╬яЁхфхыхэшх}[section] % ╤ючфрэшх юъЁєцхэш  Def фы  юяЁхфхыхэшщ ё эєьхЁрЎшхщ яюфўшэхээющ
\newtheorem{Lem}{╦хььр}[section]
\newtheorem{Ex}{╧ЁшьхЁ}[section]
\newtheorem{Zam}{╟рьхўрэшх}[section]
\newtheorem{Sle}{╤ыхфёЄтшх}[section]
\renewcommand{\abstractname}{}
\begin{center}
{\bf ╧Ёшьхэхэшх тхщтыхЄют фы  ЁрёўхЄр ышэхщэ√ї ёшёЄхь єяЁртыхэш   ё ёюёЁхфюЄюўхээ√ьш ярЁрьхЄЁрьш\footnote{╨хчєы№ЄрЄ√ ЁрсюЄ√ яюыєўхэ√ т Ёрьърї т√яюыэхэш  уюёєфрЁёЄтхээюую чрфрэш  ╠шэюсЁэрєъш \No~2.2461.2017/╧╫.}}\par
\bigskip
{▐.\,╚. ┴шЄ■ъют\footnote{╠юёъютёъшщ ртшрЎшюээ√щ шэёЄшЄєЄ (эрЎшюэры№э√щ шёёыхфютрЄхы№ёъшщ єэштхЁёшЄхЄ),  yib72@mail.ru}, ┼.\,═. ╧ырЄюэют\footnote{╠юёъютёъшщ ртшрЎшюээ√щ шэёЄшЄєЄ (эрЎшюэры№э√щ шёёыхфютрЄхы№ёъшщ єэштхЁёшЄхЄ),  en.platonov@gmail.com}}
\end{center}
\begin{abstract}
\noindent {\bf └ээюЄрЎш :} ╟рфрўш ьэюушї фшёЎшяышэ ьюуєЄ яЁштхёЄш ъ фшЇЇхЁхэЎшры№э√ь ш шэЄхуЁры№э√ь єЁртэхэш ь. ┬ яЁюёЄ√ї ёыєўр ї Єръшх єЁртэхэш  ьюуєЄ с√Є№ Ёх°хэ√ рэрышЄшўхёъш, эю т сюыхх ёыюцэ√ї яЁшїюфшЄё  эрїюфшЄ№ яЁшсышцхээ√х Ёх°хэш  ¤Єшї єЁртэхэшщ. ┬ яюёыхфэхх тЁхь  сюы№°є■ яюяєы ЁэюёЄ№ яюыєўшыш ьхЄюф√, юёэютрээ√х эр шёяюы№чютрэшш тхщтыхЄют. ╤Ёхфш яЁшьхэ хь√ї с√ыш тхщтыхЄ√ ─хсх°ш, ъющЇыхЄ√ ш Є.\,ф. ═хфюёЄрЄюъ Єръшї тхщтыхЄют ёюёЄюшЄ т Єюь, ўЄю є эшї эхЄ рэрышЄшўхёъюую т√Ёрцхэш . ╧ю¤Єюьє тючэшър■Є сюы№°шх ёыюцэюёЄш яЁш шэЄхуЁшЁютрэшш ш фшЇЇхЁхэЎшЁютрэшш т√Ёрцхэшщ, ёюфхЁцр∙шї ¤Єш тхщтыхЄ√. ┬ фрээющ ёЄрЄ№х яЁхфёЄртыхэ√ рыуюЁшЄь√ ўшёыхээюую Ёх°хэш  ышэхщэ√ї шэЄхуЁры№э√ї ш фшЇЇхЁхэЎшры№э√ї єЁртэхэшщ, юёэютрээ√х эр ёяырщэ-тхщтыхЄрї эр юЄЁхчъх. ╧ЁхфёЄртыхээ√х рыуюЁшЄь√ юсюс∙р■Є шчтхёЄэ√х ьхЄюф√, юёэютрээ√х эр тхщтыхЄрї ╒ррЁр, ъюЄюЁ√х  ты ■Єё  ўрёЄэ√ь ёыєўрхь ёяырщэ-тхщтыхЄют. ╨хчєы№ЄрЄ√ ёЄрЄ№ш яЁшьхэ ■Єё  фы  рэрышчр ышэхщэ√ї ёшёЄхь єяЁртыхэш  ё ёюёЁхфюЄюўхээ√ьш ярЁрьхЄЁрьш.

\noindent {\bf ╩ы■ўхт√х ёыютр}: ёяырщэ-тхщтыхЄ; фшЇЇхЁхэЎшры№э√х єЁртэхэш ; шэЄхуЁры№э√х єЁртэхэш 
\end{abstract}
\section{┬тхфхэшх} ─ы  ўшёыхээюую Ёх°хэш  ышэхщэ√ї шэЄхуЁры№э√ї єЁртэхэшщ ЄЁрфшЎшюээю яЁшьхэ хЄё  ьхЄюф, юёэютрээ√щ эр чрьхэх шэЄхуЁры№эюую єЁртэхэш  рыухсЁршўхёъющ ёшёЄхьющ ышэхщэ√ї єЁртэхэшщ ё яюью∙№■ яЁшьхэхэш  ътрфЁрЄєЁэющ ЇюЁьєы√. ╠рЄЁшЎр Єръющ ёшёЄхь√ шьххЄ сюы№°ющ ЁрчьхЁ, ш, ъръ ёыхфёЄтшх, фы  эрїюцфхэш  Ёх°хэш  ЄЁхсєхЄё  сюы№°юх ўшёыю рЁшЇьхЄшўхёъшї юяхЁрЎшщ. ┬ ёЄрЄ№х \cite{Lepik3} с√ыю яЁхфыюцхэю шёяюы№чютрЄ№ тхщтыхЄ√ ╒ррЁр фы  яЁшсышцхээюую Ёх°хэш  шэЄхуЁры№эюую єЁртэхэш , ўЄю яЁштюфшыю ъ ёшёЄхьх ышэхщэ√ї єЁртэхэшщ ё ЁрчЁхцхээющ ьрЄЁшЎхщ. ╧юыєўрхьюх яЁшсышцхээюх Ёх°хэшх с√ыю ъєёюўэю-эхяЁхЁ√тэ√ь. ┬ ёЄрЄ№х \cite{Blatov}  яюърчрэю, ўЄю, хёыш шёяюы№чютрЄ№ тьхёЄю тхщтыхЄют ╒ррЁр ёяырщэ-тхщтыхЄ√ эр юЄЁхчъх, ьрЄЁшЎр ёшёЄхь√ ышэхщэ√ї єЁртэхэшщ яюыєўрхЄё  яёхтфюЁрчЁхцхээющ, Є.\,х. шьххЄ юўхэ№ ьэюую ьры√ї яю ьюфєы■ ¤ыхьхэЄют. ┬ фрээющ ёЄрЄ№х сєфєЄ юсюс∙хэ√ Ёхчєы№ЄрЄ√ ЁрсюЄ \cite{Lepik1, Lepik2, Lepik3, Lepik4, Lepik} ш ЁрчтшЄ√ Ёхчєы№ЄрЄ√ ЁрсюЄ√  \cite{Blatov} фы  яюыєўхэш  яЁшсышцхээ√ї Ёх°хэшщ ы■сюую ъырёёр уырфъюёЄш ышэхщэ√ї шэЄхуЁры№э√ї ш фшЇЇхЁхэЎшры№э√ї єЁртэхэшщ. ┬ ърўхёЄтх яЁшьхЁр ЁрёёьюЄЁшь рэрышч ышэхщэющ ёшёЄхь√ єяЁртыхэш   ё ёюёЁхфюЄюўхээ√ьш ярЁрьхЄЁрьш.
%===========================================
\section{╤яырщэ-тхщтыхЄ√ эр юЄЁхчъх}
┬ ¤Єюь Ёрчфхых  ъЁрЄъю ЁрёёьюЄЁшь яюфїюф ъ яюёЄЁюхэш■ тхщтыхЄ-ёшёЄхь эр юЄЁхчъх, яЁхфыюцхээ√щ т ЁрсюЄх \cite{ArticleFinkelstein}. ╧єёЄ№ фхщёЄтшЄхы№эр  ЇєэъЎш  $\varphi $ яЁшэрфыхцшЄ фхщёЄтшЄхы№эюьє яЁюёЄЁрэёЄтє $\mathrm{L}^{2} \left({\bf R}\right)$, єфютыхЄтюЁ хЄ ЁртхэёЄтє
\begin{equation} \label{Pr1}
\varphi \left(x\right)=\sqrt{2} \sum _{k\in {\bf Z}}u_{k} \varphi \left(2x-k\right),~u_{k} \in {\bf R},
\end{equation}
ш шьххЄ ъюьяръЄэ√щ эюёшЄхы№, ёюфхЁцр∙шщё  т юЄЁхчъх $[a;b]$. ╬сючэрўшь $\varphi _{jk} (x)=2^{\frac{j}{2}}\varphi \left(2^{j} x-k\right),~x\in [a;b]$. ╘єэъЎш  $\varphi $ т ЄхюЁшш тхщтыхЄют эрч√трхЄё  ьрё°ЄрсшЁє■∙хщ, р ЁртхэёЄтю \eqref{Pr1} -- ьрё°Єрсэ√ь ёююЄэю°хэшхь \cite{Frazer}. ▀ёэю, ўЄю юЄышўэ√ьш юЄ эєы  эр юЄЁхчъх $[a;b]$ сєфхЄ ыш°№ ъюэхўэюх ўшёыю Єръшї ЇєэъЎшщ. ╧єёЄ№ фы  юяЁхфхыхээюёЄш ¤Єю сєфєЄ ЇєэъЎшш $\varphi _{j,0} ,\varphi _{j,1} ,\dots,\varphi _{j,n_{j} -1} $. ┼ёыш ЁрёёьюЄЁхЄ№ ышэхщэ√х яЁюёЄЁрэёЄтр $ V_{j} = {\rm lin}\left\{\varphi _{j,0} ,\varphi _{j,1} ,\dots,\varphi _{j,n_{j} -1} \right\},~\dim V_{j} =n_{j}$, Єю  т ёшыє ЁртхэёЄтр \eqref{Pr1}  сєфхЄ т√яюыэ Є№ё  $V_{0} \subset V_{1} \subset \ldots \subset L^{2} \left[a;b\right]$. ╧ю¤Єюьє $\varphi _{j-1,k} =\sum\limits _{s=0}^{n_{j} -1}p_{s,k} \varphi _{j,s}  $. ╩ръ т  \cite{ArticleFinkelstein}, ттхфхь юсючэрўхэш  $\Phi _{j} (x)=\left(\varphi _{j,0} (x),\varphi _{j,1} (x),\dots,\varphi _{j,n_{j} -1} (x)\right),~{\rm P}_{j} =\left(p_{s,k} \right)_{s=0, k=0}^{n_{j} -1, n_{j-1} -1}.$
╥юуфр $\Phi _{j-1} =\Phi _{j} {\rm P}_{j} $. ╬сючэрўшь ёшьтюыюь $W_{j-1} $ юЁЄюуюэры№эюх фюяюыэхэшх ъ яЁюёЄЁрэёЄтє $V_{j-1} $ т яЁюёЄЁрэёЄтх $V_{j} $. ╧юёъюы№ъє $V_{j} =V_{j-1} \oplus W_{j-1} $ ш $W_{j-1} \subset V_{j} $, Єю $W_{j-1} $ -- ъюэхўэюьхЁэюх яЁюёЄЁрэёЄтю $ W_{j}  = lin \left\{\psi _{j,0} ,\psi _{j,1} ,\dots,\psi _{j,m_{j} -1} \right\},~\dim W_{j} =m_{j}$ ш
$\psi _{j-1,k} =\sum\limits_{s=0}^{n_{j} -1}q_{s,k}^{j} \varphi _{j,s}. $ ╘єэъЎшш $\psi _{j,k} $ эрч√тр■Єё  тхщтыхЄрьш, р  яЁюёЄЁрэёЄтр $W_{j} $ эрч√тр■Єё  тхщтыхЄ-яЁюёЄЁрэёЄтрьш.
\par ╤эютр ттхфхь т ЁрёёьюЄЁхэшх ьрЄЁшЎ√ \cite{ArticleFinkelstein}
\[
\Psi _{j} (x)=\left(\psi _{j,0} (x),\psi _{j,1} (x),\dots,\psi_{j,m_{j} -1} (x)\right),~{\rm Q}_{j}=\left(q_{s,k}^{j} \right)_{s=0, k=0}^{n_{j}-1,m_{j-1}-1}.
\]
╥юуфр $\Psi _{j-1} =\Phi _{j} {\rm Q}_{j} $. ╤ыхфєхЄ чрьхЄшЄ№, ўЄю $n_{j} +m_{j} =n_{j+1} $.
╧єёЄ№ $f\in \mathrm{L}^{2} (X)$ ш $\Pi _{j} : \mathrm{L}^{2} (X)\to V_{j} $. ╥юуфр
\[
\Pi _{j} f=\sum _{k=0}^{n_{j} -1}c_{jk} \varphi _{jk}  =\Pi _{j-1} f+\Pi _{j-1}^{W} f=\sum _{k=0}^{n_{j-1} -1}c_{j-1,k} \varphi _{j-1,k}  +\sum _{k=0}^{m_{j-1} -1}d_{j-1,k} \psi _{j-1,k}.
\]
─рээюх ЁртхэёЄтю ьюцэю яхЁхяшёрЄ№ т ьрЄЁшўэюь тшфх, хёыш ттхёЄш т ЁрёёьюЄЁхэшх тхъЄюЁ√ ${\rm C}_{j} =\left(c_{j,0} ,\dots,c_{j,n_{j} -1} \right)^{T} ,~{\rm D}_{j} =\left(d_{j,0} ,\dots,d_{j,m_{j} -1} \right)^{T} $. ╧хЁт√щ тхъЄюЁ юяшё√трхЄ яЁшсышцхэшх ЇєэъЎшш $f$, р тЄюЁющ тхъЄюЁ яЁхфёЄрты хЄ ёюсющ тхщтыхЄ-ъю¤ЇЇшЎшхэЄ√, ъюЄюЁ√х їрЁръЄхЁшчє■Є юЄъыюэхэшх $\Pi _{j-1} f$ юЄ $\Pi _{j} f$. ╩ръ яюърчрэю т \cite{ArticleFinkelstein}, шьххЄ ьхёЄю ЁртхэёЄтю
${\rm C}_{j} ={\rm P}_{j} {\rm C}_{j-1} +{\rm Q}_{j} {\rm D}_{j-1} .$
╧ю фрээюьє ЁртхэёЄтє ьюцэю тюёёЄрэютшЄ№ яЁшсышцхэшх $\Pi _{j} f$ яю сюыхх уЁєсюьє яЁшсышцхэш■ $\Pi _{j-1} f$ ш тхщтыхЄ-ъю¤ЇЇшЎшхэЄрь.
 \par ╧юёъюы№ъє ышэхщэ√х юяхЁрЄюЁ√ (яЁюхъЄюЁ√) $V_{j} \to V_{j-1} ,~V_{j} \to W_{j-1} $ юяЁхфхы ■Єё  эхъюЄюЁ√ьш ьрЄЁшЎрьш ${\rm A}_{j},~{\rm B}_{j} $, Єю ${\rm C}_{j-1} ={\rm A}_{j} {\rm C}_{j},~{\rm D}_{j-1} ={\rm B}_{j} {\rm C}_{j} $.
\par ╧юф тхщтыхЄ-яЁхюсЁрчютрэшхь ЇєэъЎшш $f$ сєфхь яюэшьрЄ№ эрїюцфхэшх тхъЄюЁют ${\rm C}_{0} ,{\rm D}_{0} ,{\rm D}_{1} ,\dots,{\rm D}_{j-1} $. ╠рЄЁшЎ√ ${\rm Q}_j$ ш ${\rm P}_j$ шчтхёЄэ√ ъръ Їшы№ЄЁ√ ёшэЄхчр. ╠рЄЁшЎ√ ${\rm A}_j$ ш ${\rm B}_j$ шчтхёЄэ√ ъръ Їшы№ЄЁ√ рэрышчр. ╠эюцхёЄтю $\{{\rm A}_j, {\rm B}_j, {\rm P}_j, {\rm Q}_j\}$ эрч√трхЄё  срэъюь Їшы№ЄЁют.
\par ╩ръ яюърчрэю т ЁрсюЄх  \cite{ArticleFinkelstein}, ьхцфє ьрЄЁшЎрьш ${\rm A}_{j},~{\rm B}_{j}$ ш ${\rm P}_{j,},~{\rm Q}_{j} $ ёє∙хёЄтєхЄ ёыхфє■∙р  ёт ч№:
\[
\begin{pmatrix} {\rm A}_{j} \\ {\rm B}_{j} \end{pmatrix}=\begin{pmatrix} {\rm P}_{j}& {\rm Q}_{j}\end{pmatrix}^{-1}.
\]
\par ╧юёьюЄЁшь ЄхяхЁ№, ъръ юяЁхфхышЄ№ ьрЄЁшЎє ${\rm Q}_{j} $. ┬тхфхь ёыхфє■∙хх юсючэрўхэшх. ┼ёыш ${\rm f}=\left(f_{1} ,\dots, f_{r} \right),~{\rm g}=\left(g_{1} ,\dots,g_{r} \right)$ -- эхъюЄюЁ√х тхъЄюЁ√, Єю $[({\rm f},{\rm g})]=\left(\left(f_{i} ,g_{j} \right)\right)_{i,j=1}^{r} $ -- ьрЄЁшЎр ёъры Ёэ√ї яЁюшчтхфхэшщ.  ╩ръ яюърчрэю т ЁрсюЄх  \cite{ArticleFinkelstein}, ьрЄЁшЎр ${\rm Q}_{j} $ єфютыхЄтюЁ хЄ ёыхфє■∙хьє єЁртэхэш■: ${\rm P}_{j}^{T} \left[\left(\Phi _{j},~\Phi _{j} \right)\right]{\rm Q}_{j} =0.$
\par ╧хЁхщфхь ЄхяхЁ№ ъ ёяырщэ-тхщтыхЄрь эр юЄЁхчъх. ╬яЁхфхышь ┬-ёяырщэ√ яюЁ фър $n$  ъръ ётхЁЄъє \cite{Chui}
\[
N_{n} =N_{n-1} *N_{0},~N_{0} (x)=\begin{cases} 1,~x\in [0;1), \\ 0,~x\notin [0;1).\end{cases}
\]
╩ръ яюърчрэю т \cite{Chui}, хёыш юяЁхфхышЄ№ ЇєэъЎш■ $\varphi (x)=N_{n} (x)$, Єю юэр єфютыхЄтюЁ хЄ ЁртхэёЄтє $\varphi (x)=\sum\limits _{k=0}^{n+1}\frac{C_{n+1}^{k} }{2^{n} } \varphi (2x-k)$, уфх $C_{n+1}^k=\frac{(n+1)!}{k!(n+1-k)!}$.
┬ ёЄрЄ№х \cite{Yurgu} яЁхфёЄртыхэ срэъ Їшы№ЄЁют, ёююЄтхЄёЄтє■∙шщ ЇєэъЎшш  $\varphi (x)=N_{n} (x)$, р шьхээю  ёяЁртхфышт√ ёыхфє■∙шх Ёхчєы№ЄрЄ√.
\begin{Lem}
╘єэъЎш  $\varphi(x)=N_n(x)$ юяЁхфхы хЄ яюёыхфютрЄхы№эюёЄ№ яюфяЁюёЄЁрэёЄт
\[V_{\alpha,0}\subset V_{\alpha,1}\subset\dots, ~V_{\alpha,j}=lin\{\varphi_{j,-n},\varphi_{j,-n+1},\dots,\varphi_{j,2^j\alpha(n+1)-1}\}\]
 яЁюёЄЁрэёЄтр ${\rm L}^2[0;\alpha(n+1)], \alpha=1,2,\dots$ Єръє■, ўЄю $\overline{\bigcup\limits_{j=0}^{+\infty}V_{\alpha,j}}={\rm L}^2[0;\alpha(n+1)]$.
\end{Lem}
\begin{Lem} \label{lem32}
╚ьххЄ ьхёЄю ЁртхэёЄтю $
\sum\limits_{k=-n}^{2^j\alpha (n+1)-1} \varphi_{j,k}(x) \equiv 2^{\frac{j}{2}},~x\in [0;\alpha (n+1)].
$
┼ёыш  $V_{\alpha ,j} =V_{\alpha ,j-1} \oplus W_{\alpha ,j-1} $, Єю $\dim W_{\alpha ,j-1} =2^{j-1} \alpha (n+1)$.
\end{Lem}
\par ╧єёЄ№ $\lambda_{m,k}=\int\limits_k^{k+1} N_n(z)N_n(z-m)\,dz$, $m=-n,\dots ,n, ~k=0,1,\dots ,n$ ш $\omega_{i,k}=\omega_{k,i}=\sum\limits_{s=n-i+1}^n \lambda_{k-i,s}$, $\theta_{i,k}=\theta_{k,i}=\sum\limits_{s=0}^{n-k} \lambda_{i-k,s}$, $1\leqslant i \leqslant k \leqslant n.$ ┬тхфхь т ЁрёёьюЄЁхэшх тхъЄюЁ ${\rm p}\in \textbf{R}^{2^j\alpha (n+1)+n}$, ъюЄюЁ√щ юяЁхфхышь ЁртхэёЄтюь
\begin{equation}\label{vecp}
{\rm p}=\begin{cases}
 \begin{pmatrix} C_{n+1}^{0} \dots  C_{n+1}^{k}&C_{n+1}^{k} \dots  C_{n+1}^{0}&0 \dots 0\end{pmatrix}^T,~\text{хёыш $n=2k$;}\\
\begin{pmatrix} C_{n+1}^{0} \dots C_{n+1}^{k}&C_{n+1}^{k+1}&C_{n+1}^{k} \dots  C_{n+1}^{0}&0 \dots 0\end{pmatrix}^T,~\text{хёыш $n=2k+1$.}
\end{cases}
%{\rm p}=\begin{cases}
% \begin{pmatrix} C_{n+1}^{0}&C_{n+1}^{1} \dots  C_{n+1}^{k}&C_{n+1}^{k} \dots  C_{n+1}^{1}&C_{n+1}^{0}&0 \dots 0\end{pmatrix}^T,~\text{хёыш $n=2k$;}\\
%\begin{pmatrix} C_{n+1}^{0}&C_{n+1}^{1} \dots C_{n+1}^{k}&C_{n+1}^{k+1}&C_{n+1}^{k} \dots  C_{n+1}^{1}&C_{n+1}^{0}&0 \dots 0\end{pmatrix}^T,~\text{хёыш $n=2k+1$.}
%\end{cases}
\end{equation}
╬яЁхфхышь юяхЁрЄюЁ ёфтшур $R_s:  \textbf{R}^m \rightarrow  \textbf{R}^m$ ёыхфє■∙шь яЁртшыюь:
\[
R_s{\rm a}=\begin{cases} \begin{pmatrix} \underbrace{0 \dots 0}_s & a_1 \dots a_{m-s}\end{pmatrix}^T, ~\text{хёыш $m>s \geqslant 0$;}\\
\begin{pmatrix} a_{|s|+1} \dots  a_m&0 \dots 0\end{pmatrix}^T, ~\text{хёыш $-m<s<0$,} \end{cases} ~{\rm a}=(a_1,\dots,a_m)^T.
\]
┼ёыш $|s|\geqslant m$, Єю $R_s{\rm a}=0$.
\begin{Lem} \label{LB2}
╠рЄЁшЎ√ ${\rm P}_j$ ш $[(\Phi_j,\Phi_j)]$ фы  яюёыхфютрЄхы№эюёЄш яюфяЁюёЄЁрэёЄт  $V_{\alpha,0}\subset V_{\alpha,1}\subset\dots$ шьх■Є тшф
\begin{align}
&{\rm P}_j=\frac{1}{2^{n+\frac{1}{2}}}\begin{pmatrix} R_{-n}{\rm p}&R_{-n+2}{\rm p}&\dots&R_{n-2+2^j\alpha (n+1)}{\rm p}\end{pmatrix};\notag\\
&[(\Phi_j,\Phi_j)]=\begin{pmatrix} {\rm d}_1&\dots&{\rm d}_n&{\rm q}&R_1{\rm q}&\dots&R_{2^j\alpha (n+1)-n-1}{\rm q}&{\rm u}_1&\dots&{\rm u}_n\end{pmatrix}^T,\notag
\end{align}
уфх ${\rm d}_s=\begin{pmatrix} \omega_{1,s}&\omega_{2,s}&\dots&\omega_{n,s}&q_{n-s+1}&\dots&q_n&0 \dots 0\end{pmatrix}^T$,
\begin{align}
&{\rm u}_s=\begin{pmatrix}0 \dots 0&q_n&\dots&q_s&\theta_{1,s}&\dots&\theta_{n,s}\end{pmatrix}^T,\notag\\
&{\rm q}=\begin{pmatrix} q_n & q_{n-1}  \dots  q_1 & q_0 & q_1  \dots  q_{n-1} & q_n & 0 \dots 0 \end{pmatrix}^T  \in  {\rm\bf R}^{2^j\alpha (n+1)+n},~q_{k} =\left(N_{n}(\cdot),N_{n} (\cdot -k)\right).\notag
\end{align}
╠рЄЁшЎр, ЄЁрэёяюэшЁютрээр  ъ
${\rm T}_j={\rm P}_j^T [(\Phi_j,\Phi_j)]=2^{-n-\frac{1}{2}}(t_{i,s})_{i=1,s=1}^{2^{j-1}\alpha (n+1)+n,~2^j\alpha (n+1)+n}$,
шьххЄ тшф
\[
{\rm T}_j^T=\frac{1}{2^{n+\frac{1}{2}}}\begin{pmatrix} {\rm L}_1 \dots {\rm L}_n & {\rm w}& R_2{\rm w} \dots R_{2^j\alpha (n+1)-2n-2} {\rm w}& {\rm L}_{2^{j-1}\alpha (n+1)+1} \dots {\rm L}_{2^{j-1}\alpha (n+1)+n}\end{pmatrix},
\]
уфх ${\rm w}=\begin{pmatrix}{\rm p}^T R_{-2n}{\rm q}&{\rm p}^T R_{-2n+1}{\rm q} \dots {\rm p}^TR_{n+1}{\rm q}&0 \dots 0 \end{pmatrix}^T \in {\rm\bf R}^{2^j\alpha (n+1)+n}$,
\begin{align}
&{\rm L}_i=\begin{pmatrix} (R_{-n+2i-2}{\rm ~p})^T{\rm d}_1 \dots (R_{-n+2i-2}{\rm ~p})^T{\rm d}_n&0 \dots 0\end{pmatrix}^T+(R_n\circ R_{-3n+2i-2}){\rm w},~i=1,\dots,n,\notag\\
&{\rm L}_{i+1}=\begin{pmatrix} 0 \dots 0&(R_{-n+2i}{\rm ~p})^T{\rm u}_1 \dots (R_{-n+2i}{\rm ~p})^T{\rm u}_n\end{pmatrix}^T+(R_{-n}\circ R_{-n+2i}){\rm w},\notag\\
&i=2^{j-1}\alpha (n+1),\dots,n-1+2^{j-1}\alpha (n+1).\notag
\end{align}
\end{Lem}
\par ╤ шёяюы№чютрэшхь ыхьь√ \ref{LB2}  т ёЄрЄ№х \cite{Yurgu} эрщфхэ√ $2^{j-1}\alpha (n+1)$ ышэхщэю эхчртшёшь√ї Ёх°хэшщ ${\rm h}_s=(h_{1,s},h_{2,s},\dots,h_{2^j\alpha (n+1)+n,s})^T$ ёшёЄхь√ ышэхщэ√ї єЁртэхэшщ  ${\rm T}_j {\rm h}_s=0$. ▌Єш Ёх°хэш  ш яЁхфёЄрты ■Є ёюсющ ёЄюысЎ√ ьрЄЁшЎ√ ${\rm Q}_j=({\rm h}_1,\dots,{\rm h}_{2^{j-1}\alpha (n+1)}).$
╤ЄюысЎ√ ${\rm h}_{\, s} $ т√сшЁрышё№ Єръшь юсЁрчюь, ўЄюс√ ЇєэъЎшш
\[\psi _{j-1,s} (x)=\Phi_j(x) {\rm h}_s=\sum\limits_{i=1}^{2^{j} \alpha (n+1)+n}h_{i,s} \cdot \varphi _{j,-n+(i-1)}(x)\]
яю тючьюцэюёЄш яЁхфёЄрты ыш ёюсющ ёфтшэєЄ√х тхЁёшш юфэющ ЇєэъЎшш, Є.\,х. шьхыш с√ юфэє ЇюЁьє (чр шёъы■ўхэшхь, ъюэхўэю, уЁрэшўэ√ї тхщтыхЄют).  ┬тхфхь ёюъЁр∙хээ√х юсючэрўхэш  фы  ьрЄЁшЎ, ёюёЄртыхээ√ї шч ¤ыхьхэЄют ьрЄЁшЎ√ ${\rm T}_j$:
\[
T_j\left(\begin{smallmatrix} i_1,\dots,i_k \\ j_1,\dots,j_m \end{smallmatrix} \right) = \begin{pmatrix} t_{i_1,j_1} & \dots & t_{i_1,j_m} \\ \hdotsfor{3} \\ t_{i_k,j_1} & \dots & t_{i_k,j_m} \end{pmatrix}.
\]
─ы  тэєЄЁхээшї тхщтыхЄют (эюёшЄхы№ ёюфхЁцшЄё  т юЄЁхчъх $[0;\alpha (n+1)]$)
\[
{\rm h}_s=(0,\dots,0,h_{2s-n-1,s},\dots, h_{2s+2n,s},0,\dots,0)^T, ~s=n+1,\dots,2^{j-1}\alpha (n+1)-n,
\]
 уфх $ T_j\left(\begin{smallmatrix} s-n,\dots,s+2n \\ 2s-n-1,\dots,2n+2s \end{smallmatrix} \right)(h_{2s-n-1,s},\dots,h_{2s+2n,s})^T=0.$
 \par ╨х°хэш , ёююЄтхЄёЄтє■∙шх уЁрэшўэ√ь тхщтыхЄрь, т√сшЁр■Єё  ёыхфє■∙шь юсЁрчюь. ─ы  $s=1,2,\dots,n$ яюыюцшь
${\rm h}_{s}=(0,\dots,0,h_{s,s},\dots,h_{2n+2s,s},0,\dots,0)^T,$
уфх $ T_j\left(\begin{smallmatrix} 1,\dots,s+2n \\ s,\dots,2s+2n\end{smallmatrix} \right)(h_{s,s},\dots,h_{2s+2n,s})^T=0.$
─ы  $s=2^{j-1}\alpha (n+1)-n+1,\dots ,2^{j-1}\alpha (n+1)$ яюыюцшь ${\rm h}_{s}=(0,\dots,0,h_{2s-n-1,s},\dots,h_{2^{j-1}\alpha (n+1)+n+s,s},0,\dots,0)^T,$ уфх
\[
T_j\left(\begin{smallmatrix} s-n,\dots,n+2^{j-1}\alpha (n+1) \\ 2s-n-1,\dots,2^{j-1}\alpha (n+1)+n+s \end{smallmatrix} \right)(h_{2s-n-1,s},\dots,h_{2^{j-1}\alpha (n+1)+n+s,s})^T.
\]
\par ╩ЁрЄъю ЁрёёьюЄЁшь яЁшьхэхэшх тхщтыхЄ-ёшёЄхь эр юЄЁхчъх ъ яюёЄЁюхэш■ фтєьхЁэ√ї тхщтыхЄют эр яЁ ьюєуюы№эющ юсырёЄш. ╧єёЄ№ фрэ√ яюёыхфютрЄхы№эюёЄш $V_{0,i} \subset V_{1,i} \subset \ldots \subset V_{j,i} \subset \ldots$ ъюэхўэюьхЁэ√ї яюфяЁюёЄЁрэёЄт ${\rm L}^{2} [a_{i} ;b_{i} ]$, ьрё°ЄрсшЁє■∙шх ЇєэъЎшш $\varphi ^{(i)} $ ш срэъш Їшы№ЄЁют ${\rm P}_{j,i} ,{\rm Q}_{j,i} ,{\rm A}_{j,i} ,{\rm B}_{j,i} $, $ i=1,2$. ╤ЄрэфрЁЄэ√щ яюфїюф ъ яюёЄЁюхэш■ ьэюуюьхЁэ√ї тхщтыхЄ-ёшёЄхь -- ¤Єю тч Єшх ЄхэчюЁэ√ї яЁюшчтхфхэшщ ЇєэъЎшщ шч юфэюьхЁэ√ї срчшёют \cite{Novikov}. ╬яЁхфхышь яюфяЁюёЄЁрэёЄтр $V_{j}^{2} =V_{j,1} \otimes V_{j,2} = lin\left\{f_{1} \otimes f_{2} :~f_{1} \in V_{j,1} ,~f_{2} \in V_{j,2} \right\}$, уфх ЇєэъЎш  $f_{1} \otimes f_{2} $ юяЁхфхы хЄё  яЁртшыюь $f_{1} \otimes f_{2} \left(x,y\right)=f_{1} (x)f_{2} (y)$. ▀ёэю, ўЄю ЇєэъЎшш $\varphi _{j,k}^{(1)} \otimes \varphi _{j,s}^{(2)} $ юсЁрчє■Є срчшё т яЁюёЄЁрэёЄтх $V_{j}^{2} $.  ┬хщтыхЄ-яЁюёЄЁрэёЄтр $W_{j}^{2} $ юяЁхфхы ■Єё  ёыхфє■∙шь юсЁрчюь: $V_{j}^{2} =V_{j-1}^{2} \oplus W_{j-1}^{2} $.
\par ╤ыхфє■∙шх фтх ыхьь√ юўхтшфэ√.
\begin{Lem} \label{lem61}
╧єёЄ№ $f\in {\rm L}^{2}[0;n+1]$, Єюуфр $\Pi _{j} f=\Phi _{j} {\rm C}_{j}^{*} $, уфх ${\rm C}_{j}^{*} =[(\Phi _{j} ,\Phi _{j})]^{-1}[(f,\Phi _{j})]$. ╧Ёш ¤Єюь $\| f-\Pi _{j} f\| _{{\rm L}^{2} }^{2} =\| f\| _{{\rm L}^{2} }^{2} -[(f,\Phi _{j})]^{T}[(\Phi _{j} ,\Phi _{j})][(f,\Phi _{j})]$.
\end{Lem}
%\begin{proof} ╚ьххь
%\begin{align}
%&{\rm C}_j^{*} =\arg\min\limits_{{\rm C}_{j} }\| f-\Phi _{j} {\rm C}_{j}\|_{{\rm L}^{2} }^{2} =\notag\\
%&=\arg\min\limits_{{\rm C}_{j} } \left(\| f\| _{{\rm L}^{2} }^{2} -2\sum\limits_{i=-n}^{2^{j}(n+1) -1}(f,\varphi _{j,i})c_{j,i}  +\sum\limits_{i=-n}^{2^j(n+1) -1}\sum_{s=-n}^{2^{j}(n+1)-1}(\varphi _{j,i} \varphi _{j,s})c_{j,i} c_{j,s}\right).\notag
%\end{align}
%╬Єё■фр, ё єўхЄюь Єюую, ўЄю ьрЄЁшЎр ├Ёрьр $\left[\left(\Phi _{j} ,\Phi _{j} \right)\right]$ фы  ышэхщэю эхчртшёшьющ ёшёЄхь√ ЇєэъЎшщ $\Phi _{j} =\left(\varphi _{j,-n} ,...,\varphi _{j,2^{j}(n+1) -1} \right)$ яюыюцшЄхы№эю юяЁхфхыхээр , яюыєўрхь ${\rm C}_{j}^{*} =[(\Phi _{j} ,\Phi _{j})]^{-1}[(f,\Phi _{j})]$.
%\end{proof}
\begin{Lem}\label{Proj35} ╧єёЄ№ $f\in {\rm L}^2([a_1;b_1]\times [a_2;b_2])$ ш $\Pi_j^{(2)} : {\rm L}^2([a_1;b_1]\times [a_2;b_2])\to V_j^2$ -- яЁюхъЄюЁ. ┼ёыш
\begin{equation}\label{G35}
\mathrm{G}_j=\left(\int\limits_{a_1}^{b_1}dx\int\limits_{a_2}^{b_2}\varphi_{j,s}^{(1)}(x)\varphi_{j,k}^{(2)}(y)f(x,y)dy\right)_{s,k=0}^{n_{j,1}-1,n_{j,2}-1},
\end{equation}
Єю $\Pi_j^{(2)}f (x,y)=\Phi_j^{(1)}(x)\mathrm{C}_j (\Phi_j^{(2)}(y))^T$, уфх $\Phi_j^{(i)}=(\varphi_{j,0}^{(i)}\dots \varphi_{j,n_{j,i}-1}^{(i)})$, р ьрЄЁшЎр $\mathrm{C}_j$ юяЁхфхы хЄё  ЁртхэёЄтюь
\begin{equation}\label{Cj35}
\mathrm{C}_j=[(\Phi_j^{(1)},\Phi_j^{(1)})]^{-1}\mathrm{G}_j[(\Phi_j^{(2)},\Phi_j^{(2)})]^{-1}
\end{equation}
\end{Lem}
%\begin{proof}
%╧юёъюы№ъє
%\[
%\Pi_j^{(2)}f(x,y)=\sum\limits_s \sum\limits_{k} c_{s,k}^j \varphi_{j,s}^{(1)}(x)\varphi_{j,k}^{(2)}(y) = \Phi_j^{(1)}(x)\mathrm{C}_j (\Phi_j^{(2)}(y))^T,
%\]
%Єю  ьрЄЁшЎр  $\mathrm{C}_j$  ты хЄё  Ёх°хэшхь ёыхфє■∙хщ чрфрўш
%\[
%\mathrm{C}_j=\arg\min\limits_{\mathrm{Z}} \|f-\Phi_j^{(1)} \mathrm{Z} (\Phi_j^{(2)})^T\|_{ {\rm L}^2([a_1;b_1]\times [a_2;b_2])}^2
%\]
%╧єёЄ№ $\mathrm{Z}=(z_{s,k})_{s,k=0}^{n_{j,1}-1,n_{j,2}-1}$. ╟рьхЄшь, ўЄю
%\[
%\|f-\Phi_j^{(1)} \mathrm{Z} (\Phi_j^{(2)})^T\|^2=\|f\|^2-2\left(f,\Phi_j^{(1)} \mathrm{Z} (\Phi_j^{(2)})^T\right)+\|\Phi_j^{(1)} \mathrm{Z} (\Phi_j^{(2)})^T\|^2.
%\]
%╧ю¤Єюьє
%\begin{align}
%\mathrm{C}_j=&\arg\min\limits_{\mathrm{Z}}\left(\sum\limits_{s_1,k_1,s_2,k_2}z_{s_1,k_1}z_{s_2,k_2}\left(\varphi_{j,s_1}^{(1)},\varphi_{j,s_2}^{(1)}\right)\left (\varphi_{j,k_1}^{(2)},\varphi_{j,k_2}^{(2)}\right)-\right.\notag\\
%&\left.-2\sum\limits_{s,k}z_{s,k} \int\limits_{a_1}^{b_1}dx\int\limits_{a_2}^{b_2}\varphi_{j,s}^{(1)}(x)\varphi_{j,k}^{(2)}(y)f(x,y)dy\right) \notag
%\end{align}
%╚ч эхюсїюфшьюую єёыютш  ¤ъёЄЁхьєьр, фы  тёхї $s,k$, яюыєўрхь
%\begin{equation}\label{L35}
%2\sum\limits_{s_1,k_1}z_{s_1,k_1}\left(\varphi_{j,s_1}^{(1)},\varphi_{j,s}^{(1)}\right)\left (\varphi_{j,k_1}^{(2)},\varphi_{j,k}^{(2)}\right)-
%2\int\limits_{a_1}^{b_1}dx\int\limits_{a_2}^{b_2}\varphi_{j,s}^{(1)}(x)\varphi_{j,k}^{(2)}(y)f(x,y)dy=0
%\end{equation}
%╙ўшЄ√тр  юсючэрўхэшх (\ref{G35}), ЁртхэёЄтр (\ref{L35}) ьюцэю яхЁхяшёрЄ№ т ьрЄЁшўэюь тшфх
%\[
%[(\Phi_j^{(1)},\Phi_j^{(1)})]\mathrm{Z}[(\Phi_j^{(2)},\Phi_j^{(2)})]=\mathrm{G}_j.
%\]
%╚ч яюыєўхээюую ЁртхэёЄтр ёыхфєхЄ (\ref{Cj35}).
%\end{proof}
%%--------------------------------------------
\section{╚эЄхуЁры√ юЄ ёяырщэ-тхщтыхЄют}
╧єёЄ№ ${\rm Q}_j=({\rm h}_1^j,\dots,{\rm h}_{2^{j-1}(n+1)}^j)$, уфх ${\rm h}_s^j=(h_{1,s}^j,h_{2,s}^j,\dots,h_{2^j (n+1)+n,s}^j)^T$. ╥юуфр   ёюуырёэю Ёхчєы№ЄрЄрь яЁхф√фє∙хую Ёрчфхыр
\begin{equation}\label{U31}
\psi_{j-1,s}(x)=\sum\limits_{i=s}^{2s+2n}h_{i,s}^j\varphi_{j,-n+i-1}(x),~s=1,\dots,n,
\end{equation}
\begin{equation}\label{U32}
\psi_{j-1,s}(x)=\sum\limits_{i=2s-n-1}^{2s+2n}h_{i,s}^j\varphi_{j,-n+i-1}(x),~s=n+1,\dots,2^{j-1}(n+1)-n,
\end{equation}
\begin{equation}\label{U33}
\psi_{j-1,s}(x)=\sum\limits_{i=2s-n-1}^{2^{j-1}(n+1)+n+s}h_{i,s}^j\varphi_{j,-n+i-1}(x),~s=2^{j-1}(n+1)-n+1,\dots,2^{j-1}(n+1).
\end{equation}
\par ╥ръ цх, ъръ ш т ЁрсюЄрї \cite{Lepik1, Lepik2, Lepik3, Lepik4, Lepik}, фы  єфюсёЄтр  ттхфхь ёыхфє■∙шх юсючэрўхэш :
\begin{align}
&w_l(x)=\varphi_{0,l-n-1},~l=1,2,\dots,2n+1,\notag\\
&w_l(x)=\psi_{j,s}(x),~l=2^j(n+1)+n+s,~j=0,1,\dots,~s=1,\dots,2^{j}(n+1).\notag
\end{align}
═р Ёшё. \ref{graph_w} яЁхфёЄртыхэ√ уЁрЇшъш эхъюЄюЁ√ї ЇєэъЎшщ $w_l$ фы  ёыєўр  $n=5$.
\begin{figure}[h]
\center{\includegraphics[scale=0.6]{graph_w.pdf}}
\caption{├ЁрЇшъш ЇєэъЎшщ $w_l$ фы  $n=5$}
\label{graph_w}
\end{figure}
\par ╧єёЄ№ $J\geqslant 0$,  $\Pi_{J} : {\rm L}^2[0;n+1]\to V_{J}$ -- яЁюхъЄюЁ ш $M=2^{J}(n+1)+n$. ╬сючэрўшь $\mathrm{H}_J=\begin{pmatrix}w_1 & \dots & w_M\end{pmatrix}$ ш ттхфхь т ЁрёёьюЄЁхэшх ьрЄЁшЎє ёъры Ёэ√ї яЁюшчтхфхэшщ $[(\mathrm{H}_J, \mathrm{H}_J)]$. ┬ ыхььх \ref{LB2} яЁхфёЄртыхэ√ ьрЄЁшЎ√ ёъры Ёэ√ї яЁюшчтхфхэшщ $[(\Phi_k, \Phi_k)]$ фы  тёхї $k=0,1,\dots$ ╟рьхўр , ўЄю $\Psi_k = \Phi_{k+1}\mathrm{Q}_{k+1}$ ш
$[(\Psi_k,\Psi_k)]=\mathrm{Q}_{k+1}^T[( \Phi_{k+1}, \Phi_{k+1})]\mathrm{Q}_{k+1}$,
яюыєўрхь ьрЄЁшЎє
\[
[(\mathrm{H}_J, \mathrm{H}_J)]=\begin{pmatrix}
[(\Phi_0, \Phi_0)] & 0 & 0 & \dots & 0\\
0 & \mathrm{Q}_{1}^T[( \Phi_{1}, \Phi_{1})]\mathrm{Q}_{1} & 0 & \dots & 0 \\
\dots & \dots & \dots & \dots & \dots \\
0 & 0 & 0 & \dots & \mathrm{Q}_{J}^T[( \Phi_{J}, \Phi_{J})]\mathrm{Q}_{J}
\end{pmatrix}.
\]
╥ръ ъръ $V_{J}=V_0\oplus W_0\oplus V_1\oplus\dots\oplus W_{J-1},$ Єю фы  $f\in {\rm L}^2[0;n+1]$ шьххь
$\Pi_{J} f=\sum\limits_{l=1}^{M} c_l w_l = \mathrm{H}_{J}\mathrm{C}_{J}$,
уфх $\mathrm{C}_J=\begin{pmatrix} c_1 & \dots\ c_{M} \end{pmatrix}^T$.
╩ръ ш т ЁрсюЄрї \cite{Lepik1, Lepik2, Lepik3, Lepik4, Lepik}, юяЁхфхышь ЇєэъЎшш
\begin{equation}\label{U38}
{\xi}_{1,l}(x)=\int\limits_0^x w_l (t)dt,~
{\xi}_{\nu+1,l}(x)=\int\limits_0^x {\xi}_{\nu,l}(t)dt=\frac{1}{\nu !}\int\limits_0^x (x-t)^{\nu} w_l (t)dt,~\nu = 1,2,\dots
\end{equation}
╤юуырёэю юяЁхфхыхэш■ ЇєэъЎшщ $w_l$ ш ЁртхэёЄтрь (\ref{U31})--(\ref{U33})  ЇєэъЎш  ${\xi}_{\nu+1,l}(x)$ яЁхфёЄрты хЄ ёюсющ ышэхщэє■ ъюьсшэрЎш■ ЇєэъЎшщ
\[
\eta_{n,\nu}^{j,s}(x)=\int\limits_0^x (x-t)^{\nu} N_{n}(2^j t-s)dt.
\]
\begin{Lem} ╚ьххЄ ьхёЄю ёыхфє■∙хх ЁхъєЁЁхэЄэюх ёююЄэю°хэшх:
\begin{equation}\label{U40}
\eta_{n,\nu}^{j,s}(x)=\frac{x^{\nu+1}}{\nu+1} N_n(-s)+\frac{2^j}{\nu+1}\left(\eta_{n-1,\nu+1}^{j,s}(x)-\eta_{n-1,\nu+1}^{j,s+1}(x)\right),
\end{equation}
уфх
\begin{equation}\label{U41}
\eta_{0,\nu}^{j,s}(x) = \begin{cases}\frac{(x-a)^{\nu+1} - (x-b)^{\nu+1}}{\nu+1},~\text{хёыш}~[a;b]=[0;x]\cap\left[\frac{s}{2^j};\frac{s+1}{2^j}\right]\not= \varnothing;\\ 0,~\text{хёыш}~[a;b]=[0;x]\cap\left[\frac{s}{2^j};\frac{s+1}{2^j}\right]= \varnothing.\end{cases}
\end{equation}
\end{Lem}
\begin{proof} ╧ю ётющёЄтє ┬-ёяырщэют \cite{Chui}  шьххЄ ьхёЄю ЁртхэёЄтю
\begin{equation}\label{BSPL}
N_n'(x)=N_{n-1}(x)-N_{n-1}(x-1).
\end{equation}
╤ыхфютрЄхы№эю, яю ЇюЁьєых шэЄхуЁшЁютрэш  яю ўрёЄ ь  яюыєўрхь
\begin{align}
\eta_{0,\nu}^{j,s}(x)=&-\left. \frac{(x-t)^{\nu+1}}{\nu+1} N_n(2^j t-s)\right|_0^x+2^j\int\limits_0^x \frac{(x-t)^{\nu+1}}{\nu+1} (N_{n-1}(2^jt-s)-\notag\\
&-N_{n-1}(2^jt-s-1))dt=\frac{x^{\nu+1}}{\nu+1} N_n(-s)+\frac{2^j}{\nu+1}\left(\eta_{n-1,\nu+1}^{j,s}(x)-\eta_{n-1,\nu+1}^{j,s+1}(x)\right).\notag
\end{align}
╨ртхэёЄтю (\ref{U41}) юўхтшфэю.
\end{proof}
\par
╘юЁьєы√ (\ref{U40}), (\ref{U41}) яючтюы ■Є эрїюфшЄ№ чэрўхэшх ЇєэъЎшш $\eta_{n,\nu}^{j,s}(x)$ т ы■сющ Єюўъх схч шэЄхуЁшЁютрэш . ╚Єръ, фы  $l=1,2,\dots, 2n+1$ яюыєўрхь
\[
{\xi}_{\nu+1,l}(x)=\frac{1}{\nu !} \eta_{n,\nu}^{0,l-n-1}(x),~l=1,2,\dots, 2n+1.
\]
─ы  $l=2^j(n+1)+n+s,~j=0,1,\dots,~s=1,\dots,2^{j}(n+1)$ яюыєўрхь
\begin{align}
&{\xi}_{\nu+1,l}(x)=\frac{2^{\frac{j+1}{2}}}{\nu !}\sum\limits_{i=s}^{2s+2n}h_{i,s}^{j+1}\eta_{n,\nu}^{j+1,-n+i-1}(x),~s=1,\dots,n; \notag\\
&{\xi}_{\nu+1,l}(x)=\frac{2^{\frac{j+1}{2}}}{\nu !}\sum\limits_{i=2s-n-1}^{2s+2n}h_{i,s}^{j+1}\eta_{n,\nu}^{j+1,-n+i-1}(x),~s=n+1,\dots,2^{j}(n+1)-n;\notag\\
&{\xi}_{\nu+1,l}(x)=\frac{2^{\frac{j+1}{2}}}{\nu !}\sum\limits_{i=2s-n-1}^{2^{j}(n+1)+n+s}h_{i,s}^{j+1}\eta_{n,\nu}^{j+1,-n+i-1}(x),~s=2^{j}(n+1)-n+1,\dots,2^{j}(n+1).\notag
\end{align}
╧юыєўхээ√х ЁртхэёЄтр ёяЁртхфышт√ яЁш тёхї $\nu = 0,1,\dots$
%%==================================================================================
\section{╧Ёшьхэхэшх ёяырщэ-тхщтыхЄют ъ Ёх°хэш■\\ ышэхщэ√ї шэЄхуЁры№э√ї ш фшЇЇхЁхэЎшры№э√ї\\ єЁртэхэшщ}
┬ яЁюхъЎшюээ√ї ьхЄюфрї Ёх°хэш  ышэхщэ√ї єЁртэхэшщ ЁрёёьрЄЁштр■Єё  фтр єЁртэхэш   \cite{Akilov}: яхЁтюх -- т яюыэюь эюЁьшЁютрээюь яЁюёЄЁрэёЄтх $X$
\begin{equation}\label{Ur1}
Kx\equiv x-\lambda Hx=f,
\end{equation}
р тЄюЁюх -- т хую яюыэюь яюфяЁюёЄЁрэёЄтх $V_j$
\begin{equation}\label{Ur2}
K_j x_j\equiv x_j-\lambda H_j x_j=\Pi_j f,
\end{equation}
уфх $H$ -- эхяЁхЁ√тэ√щ ышэхщэ√щ юяхЁрЄюЁ т $X$, р $H_j$  -- эхяЁхЁ√тэ√щ ышэхщэ√щ юяхЁрЄюЁ т $V_j$. ╙Ёртэхэшх (\ref{Ur1}) эрч√трхЄё  Єюўэ√ь, р єЁртэхэшх (\ref{Ur2}) -- яЁшсышцхээ√ь. ╧Ёш ¤Єюь яЁхфяюырурхЄё , ўЄю т√яюыэхэ√ єёыютш :

\noindent \textbf{1. ╙ёыютшх сышчюёЄш юяхЁрЄюЁют $H$ ш $H_j$.} ─ы  ы■сюую $x_j\in V_j$ т√яюыэ хЄё  $\|\Pi_j H x_j - H_j x_j\|\leqslant \rho_j \|x_j\|$.

\noindent \textbf{2. ╙ёыютшх їюЁю°хщ ряяЁюъёшьрЎшш ¤ыхьхэЄют тшфр $Hx$ ¤ыхьхэЄрьш шч $V_j$.} ─ы  ы■сюую $x\in X$ ёє∙хёЄтєхЄ $x_j\in V_j$ Єръющ, ўЄю $\|Hx-x_j\|\leqslant \rho_{1,j} \|x\|$.

\noindent \textbf{3. ╙ёыютшх їюЁю°хщ ряяЁюъёшьрЎшш ётюсюфэюую ўыхэр Єюўэюую єЁртэхэш .} ╤є∙хёЄтєхЄ ¤ыхьхэЄ $f_j \in V_j$ Єръющ, ўЄю $\|f-f_j\|\leqslant \rho_{2,j}\|f\|$. ┬ юЄышўшх юЄ яЁхф√фє∙шї єёыютшщ, $\rho_{2,j}$ чфхё№, тююс∙х уютюЁ , чртшёшЄ юЄ $f$.

╩ръ яюърчрэю т \cite{Akilov}, хёыш юяхЁрЄюЁ $K$ шьххЄ эхяЁхЁ√тэ√щ юсЁрЄэ√щ, єЁртэхэшх (\ref{Ur1}) шьххЄ Ёх°хэшх ш $\lim\limits_{j\to +\infty} \rho_j = 0$,  $\lim\limits_{j\to +\infty} \rho_{1,j} = 0$,  $\lim\limits_{j\to +\infty} \rho_{2,j} = 0$, Єю  $\lim\limits_{j\to +\infty} \|x-x_j\| = 0$, уфх $x_j$ -- Ёх°хэшх єЁртэхэш  (\ref{Ur2}).
\par ╨рёёьюЄЁшь ёэрўрыр ышэхщэюх шэЄхуЁры№эюх єЁртэхэшх ╘Ёхфуюы№ьр 2-ую Ёюфр. ╤ яюью∙№■ чрьхэ√ яхЁхьхээющ Єръюх єЁртэхэшх ьюцэю ётхёЄш ъ ёыхфє■∙хьє:
\begin{equation}\label{U313}
u(x)-\lambda\int\limits_{0}^{n+1}U(x,t)u(t)dt =f(x),~x,t\in [0;n+1],
\end{equation}
\par ╧єёЄ№ $\varphi(x)=N_n(x)$, $V_{0} \subset V_{1} \subset ...$ -- ёююЄтхЄёЄтє■∙р  яюёыхфютрЄхы№эюёЄ№ ъюэхўэюьхЁэ√ї яюфяЁюёЄЁрэёЄт яЁюёЄЁрэёЄтр ${\rm L}^{2} \left[0;n+1\right]$. ╧єёЄ№ $X={\rm L}^2[0;n+1]$. ╬яхЁрЄюЁ√ $K:X\to X$, $H:X\to X$ ш $H_j : V_j\to V_j$ юяЁхфхышь ЁртхэёЄтрьш
\[
Ku(\cdot)=u(\cdot)-\lambda\int\limits_0^{n+1} U(\cdot,t)u(t)dt,~Hu(\cdot)=\int\limits_0^{n+1} U(\cdot,t)u(t)dt,~H_j=\Pi_j\circ H,
\]
уфх $U\in {\rm L}^2([0;n+1]^2)$. ╙ёыютшх сышчюёЄш юяхЁрЄюЁют $H$ ш $H_j$ т√яюыэ хЄё  ё $\rho_j=0$. ╧єёЄ№  $u\in X$ ш $u_j(\cdot)=\int\limits_0^{n+1} \Pi^{(2)}U(\cdot,t)u(t)dt\in V_j$. ╥юуфр $\rho_{1,j}=\|U-\Pi_j^{(2)}U\|_{{\rm L}^2([0;n+1]^2)}$ ш $\lim\limits_{j\to +\infty} \rho_{1,j}=0$. ╤ыхфютрЄхы№эю, єёыютшх їюЁю°хщ ряяЁюъёшьрЎшш ¤ыхьхэЄют тшфр $Hu$ ¤ыхьхэЄрьш шч $V_j$ Єръцх т√яюыэ хЄё . ═ръюэхЎ, фы  яЁюшчтюы№эюую $f\in X,~f\ne 0$, тюч№ьхь $f_j=\Pi_j f$, р $\rho_{2,j}=\frac{\|f-\Pi_j f\|_{{\rm L}^2([0;n+1])}}{\|f\|_{{\rm L}^2([0;n+1])}}$. ╥юуфр $\lim\limits_{j\to +\infty} \rho_{2,j}=0$. \par ╨х°хэшх яЁшсышцхээюую єЁртэхэш 
\begin{equation}\label{Ur3}
u_J-\lambda\Pi_j\circ H u_J=\Pi_J f
\end{equation}
сєфхь шёърЄ№ т тшфх $u_J=\sum\limits_{l=1}^{M} c_l w_l = \mathrm{H}_J\mathrm{C}_J $, уфх $M=2^{J}(n+1)+n$. ╥юуфр єЁртэхэшх (\ref{Ur3}) ьюцэю яхЁхяшёрЄ№ т тшфх
ёшёЄхь√ ышэхщэ√ї єЁртэхэшщ фы  юяЁхфхыхэш  ъю¤ЇЇшЎшхэЄют $c_l$
\begin{equation}\label{U315}
\sum\limits_{l=1}^{M} c_l (w_l,w_s)-\lambda\sum\limits_{l=1}^{M} c_l \int\limits_{0}^{n+1}dx\int\limits_{0}^{n+1}U(x,t)w_l(t)w_s(x)dt =(f,w_s),~s=1,2,\dots,M.
\end{equation}
▌Єю ш хёЄ№ ёшёЄхьр ьхЄюфр ├рыхЁъшэр. ╧хЁхяш°хь хх т ьрЄЁшўэюь тшфх
\begin{equation}\label{U315}
\mathrm{C}_J\left([(\mathrm{H}_J,\mathrm{H}_J)]-\lambda\mathrm{G}_J\right)=\mathrm{F}_J,
\end{equation}
уфх $\mathrm{C}_J=\begin{pmatrix}c_1&\dots &c_M\end{pmatrix}$, $\mathrm{F}_J=\begin{pmatrix}(f,w_1)&\dots &(f,w_M)\end{pmatrix}^T$,
\[
\mathrm{G}_J=(g_{l,s})_{l,s=1}^M,~g_{l,s}= \int\limits_{0}^{n+1}dx\int\limits_{0}^{n+1}U(x,t)w_l(t)w_s(x)dt.
\]
\par └эрыюушўэю ЁрёёьрЄЁштрхЄё  ышэхщэюх шэЄхуЁры№эюх єЁртэхэшх ┬юы№ЄхЁЁр 2-ую Ёюфр
\[
u(x)-\lambda\int\limits_{0}^{x}U(x,t)u(t)dt =f(x),~x,t\in [0;n+1].
\]
╬яхЁрЄюЁ√ $K:X\to X$, $H:X\to X$ ш $H_j : V_j\to V_j$ юяЁхфхышь ЁртхэёЄтрьш
\[
Ku(x)=u(x)-\lambda\int\limits_0^{x} U(x,t)u(t)dt,~Hu(x)=\int\limits_0^{x} U(x,t)u(t)dt,~H_j=\Pi_j\circ H.
\]
┬хышўшэ√ $\rho_j,~\rho_{1,j},~\rho_{2,j}$ юёЄр■Єё  Єхьш цх, ўЄю ш фы  єЁртэхэш  ╘Ёхфуюы№ьр.  ╥ръшь юсЁрчюь,
ёшёЄхьр ├рыхЁъшэр фы  фрээюую єЁртэхэш  шьххЄ тшф
\begin{equation}\label{U316}
\sum\limits_{l=1}^{M} c_l (w_l,w_s)-\lambda\sum\limits_{l=1}^{M} c_l \int\limits_{0}^{n+1}dx\int\limits_{0}^{x}U(x,t)w_l(t)w_s(x)dt =(f,w_s),~s=1,2,\dots,M.
\end{equation}
╠рЄЁшўэ√щ тшф ёшёЄхь√ (\ref{U316}) ёютярфрхЄ ё (\ref{U315}), уфх
\[
\mathrm{G}_J=(g_{s,l})_{s,l=1}^M,~g_{s,l}= \int\limits_{0}^{n+1}dx\int\limits_{0}^{x}U(x,t)w_l(t)w_s(x)dt.
\]
\par ╨рёёьюЄЁшь ЄхяхЁ№ ышэхщэюх фшЇЇхЁхэЎшры№эюх єЁртэхэшх
\begin{equation}\label{DU1}
y^{(k)}+a_1(x)y^{(k-1)}+\dots +a_k(x)y=f(x)
\end{equation}
ё эхяЁхЁ√тэ√ьш ъю¤ЇЇшЎшхэЄрьш $a_i(x),~i=1,2,\dots,k$ ш эрўры№э√ьш єёыютш ьш
\begin{equation}\label{NDU1}
y(0)=y_0,~y'(0)=y_1,\dots,y^{(k-1)}=y_{k-1}.
\end{equation}
┼ёыш юсючэрўшЄ№ $y^{(k)}(x)=u(x)$, Єю чрфрўр (\ref{DU1})--(\ref{NDU1}) ётюфшЄё  ъ  шэЄхуЁры№эюьє єЁртэхэш■ ┬юы№ЄхЁЁр 2-ую Ёюфр. ╤ыхфютрЄхы№эю, яЁшсышцхээюх Ёх°хэшх чрфрўш (\ref{DU1})--(\ref{NDU1}) ьюцэю шёърЄ№ т тшфх
\[y_J(x)=\sum\limits_{s=1}^M c_s {\xi}_{k,s}(x) +y_{k-1}\frac{x^{k-1}}{(k-1)!}+y_{k-2}\frac{x^{k-2}}{(k-2)!}+\dots+y_0,\]
уфх ЇєэъЎшш ${\xi}_{k,s}(x)$ юяЁхфхыхэ√ ЁртхэёЄтюь (\ref{U38}), р ъю¤ЇЇшЎшхэЄ√ $c_s$ юяЁхфхы ■Єё  шч ёшёЄхь√ ышэхщэ√ї єЁртэхэшщ
\[
\sum\limits_{s=1}^M c_s ({\xi}_{k,s}+a_1 {\xi}_{k-1,s}+\dots +a_k w_s,w_l)=(f,w_l),~l=1,\dots,M.
\]
╧Ёш ¤Єюь $\lim\limits_{J\to +\infty} \|y_J-y\|_{C^{k-1}[0;n+1]}=0$, уфх $y(x)$ -- Єюўэюх Ёх°хэшх чрфрўш (\ref{DU1})--(\ref{NDU1}).
% ------------------------------- ╧ЁшьхЁ√-----------------------------------------------------
\begin{figure}[h]
\center{\includegraphics[scale=0.43]{L_J4_J2_n2.pdf}}
\caption{├ЁрЇшъш яЁшсышцхэшщ $x_2(t)$ (яєэъЄшЁэр  ышэш ), $x_4(t)$ (ёяыю°эр  ышэш ) ш уЁрЇшъ ёхЄюўэющ ЇєэъЎшш $\{(t_i,\tilde{x}_i)\}$ (Єюўъш), яюыєўхээющ ьхЄюфюь ╨єэух--╩єЄЄр}
\label{LS5_J4_n1_T5}
\end{figure}
\begin{Ex} %{\rm \cite{Pupkov}}
╨рёёьюЄЁшь эхёЄрЎшюэрЁэє■ ёшёЄхьє ртЄюьрЄшўхёъюую єяЁртыхэш , яютхфхэшх ъюЄюЁющ юяшё√трхЄё  фшЇЇхЁхэЎшры№э√ь єЁртэхэшхь
\[
\sum\limits_{k=0}^5 a_k(t)x^{(k)}(t)=g(t),
\]
уфх ъю¤ЇЇшЎшхэЄ√ $a_k(t)$ юяЁхфхы ■Єё  шч ёыхфє■∙хую т√Ёрцхэш :
\[
\begin{pmatrix}a_0(t)\\a_1(t)\\a_2(t)\\a_3(t)\\a_4(t)\\a_5(t)\end{pmatrix}=\begin{pmatrix}0{,}5596&1{,}8918&2{,}5825&1{,}7855&0{,}6277&0{,}0909\\
0{,}7113&2{,}3843&3{,}222&2{,}1975&0{,}7588&0{,}1065\\
0{,}3717&1{,}2333&1{,}6449&1{,}1038&0{,}3728&0{,}0507\\
0{,}1002&0{,}3278&0{,}43&0{,}2827&0{,}093&0{,}0122\\
0{,}014&0{,}0449&0{,}0576&0{,}0369&0{,}0118&0{,}0015\\
0{,}0008&0{,}0025&0{,}0031&0{,}0019&0{,}006&0{,}00007\end{pmatrix}
\begin{pmatrix}1\\t\\t^2\\t^3\\t^4\\t^5\end{pmatrix}.
\]
═рщЄш ЁхръЎш■ ёшёЄхь√ эр тїюфэюх тючфхщёЄтшх
\begin{align}
g(t)=&\left(85{,}7661+338{,}5984t+497{,}0437t^2+406{,}9496t^3+\right.\notag\\
&\left.+186{,}9354t^4+46{,}7809t^5+4{,}8258t^6\right)e^{-4t}.
\end{align}
═рўры№э√х єёыютш  эєыхт√х. ╚эЄхЁтры шёёыхфютрэш  -- $[0;5]$ c.
\end{Ex}
\begin{proof}[╨х°хэшх]
╥ръ ъръ эрўры№э√х єёыютш  эєыхт√х, яЁшсышцхээюх Ёх°хэшх фрээющ чрфрўш сєфхь  шёърЄ№ т тшфх
$x_J(t)=\sum\limits_{s=1}^M c_s {\xi}_{5,s}(t)$,
уфх ъю¤ЇЇшЎшхэЄ√ $c_1,\dots, c_M$ юяЁхфхы ■Єё  шч ёшёЄхь√ ышэхщэ√ї єЁртэхэшщ
\[
\sum\limits_{s=1}^{2^J(n+1)+n} c_s \left(a_5w_{s}+\sum\limits_{k=0}^4 a_k\xi_{5-k,s},w_l\right)=(g,w_l),~l=1,\dots,2^J(n+1)+n.
\]
═р Ёшё. \ref{LS5_J4_n1_T5} яюърчрэ√ уЁрЇшъш ЄЁхЄ№хую ш я Єюую яЁшсышцхэшщ $x_2(t)$ (яєэъЄшЁэр  ышэш ), $x_4(t)$ (ёяыю°эр  ышэш ) ш уЁрЇшъ ёхЄюўэющ ЇєэъЎшш $\{(t_i,\tilde{x}_i)\}$ (Єюўъш), яюыєўхээющ ьхЄюфюь ╨єэух--╩єЄЄр.
\end{proof}
\begin{Ex}  ╧ютхфхэшх ышэхщэющ эхёЄрЎшюэрЁэющ ёшёЄхь√ юяшё√трхЄё  ёыхфє■∙хщ ёшёЄхьющ фшЇЇхЁхэЎшры№э√ї єЁртэхэшщ:
\[
\begin{pmatrix}{\dot x}(t) \\ {\dot y}(t)\end{pmatrix} = \begin{pmatrix} t^2 & 1-t \\ 1+t & t-t^2 \end{pmatrix} \begin{pmatrix} {x}(t) \\ {y}(t)\end{pmatrix}+  \begin{pmatrix} t^2 & 0 \\ 1 & t \end{pmatrix} \begin{pmatrix} {g}_1(t) \\ {g}_2(t)\end{pmatrix}.
\]
═рщЄш ЁхръЎш■ ёшёЄхь√ эр тїюфэюх тючфхщёЄтшх
\begin{align}
g_1(t) =& 0{,}23315158 t^9-3{,}89665 t^8+26{,}4309725 t^7-93{,}4794 t^6+183{,}95 t^5 - \notag\\
&-200{,}83 t^4+122{,}255277 t^3-50{,}135386 t^2+13{,}095959 t-2{,}8237;\notag\\
g_2(t) =& -0{,}071962459 t^{13}+1{,}3465024 t^{12}-10{,}98105044 t^{11} + 51{,}1908385 t^{10} -\notag\\
& - 150{,}5098287 t^9+291{,}295256 t^8-378{,}61242 t^7+336{,}683591 t^6-213{,}9681871 t^5 +\notag\\
& + 106{,}48891 t^4-47{,}3676 t^3+19{,}56997 t^2-3{,}863587 t-0{,}0004283\notag
\end{align}
фы  эрўры№э√ї єёыютшщ $x(0)=-1,~y(0)=2$ эр тЁхьхээюь шэЄхЁтрых $[0;2]$ c.
\end{Ex}
\begin{proof}[╨х°хэшх] ╧Ёшсышцхээюх Ёх°хэшх сєфхь шёърЄ№ т тшфх $x_{J}(t)=-1+\sum\limits_{s=1}^M c_{s} {\xi}_{1,s}(t),$ $y_{J}(t)=2+\sum\limits_{s=1}^M c_{M+s} {\xi}_{1,s}(t)$, уфх $M=2^J(n+1)+n$, р ъю¤ЇЇшЎшхэЄ√ $c_s,~s=1,2,\dots, 2M$, юяЁхфхы ■Єё  шч ёшёЄхь√ ышэхщэ√ї єЁртэхэшщ
\begin{align}
&\sum\limits_{s=1}^M c_s\left((w_s,w_l)-\int\limits_{0}^{n+1}t^2{\xi}_{1,s}(t)w_l(t)dt\right)-\sum\limits_{s=1}^M c_{M+s}\int\limits_0^{n+1}(1-t){\xi}_{1,s}(t)w_l(t)dt=\notag\\
&=\int\limits_0^{n+1}(t^2g_1(t)-t^2+2(1-t))w_l(t)dt,~l=1,2,\dots,M;\notag\\
&\sum\limits_{s=1}^M c_s\int\limits_{0}^{n+1}(1+t){\xi}_{1,s}(t)w_l(t)dt+\sum\limits_{s=1}^M c_{M+s}\left(\int\limits_0^{n+1}(t-t^2){\xi}_{1,s}(t)w_l(t)dt-(w_s,w_l)\right)=\notag\\
&=\int\limits_0^{n+1}(2t^2-t+1-g_1(t)-tg_2(t))w_l(t)dt,~l=1,2,\dots,M.\notag
\end{align}
\begin{figure}[h]
\center{\includegraphics[scale=0.5]{sx.pdf}}
\caption{├ЁрЇшъш яЁшсышцхэшщ $x_0(t)$ (яєэъЄшЁэр  ышэш ), $x_2(t)$ (ёяыю°эр  ышэш ) ш уЁрЇшъ ёхЄюўэющ ЇєэъЎшш $\{(t_i,\tilde{x}_i)\}$ (Єюўъш), яюыєўхээющ ьхЄюфюь ╨єэух--╩єЄЄр}
\label{sx}
\end{figure}
\begin{figure}[h]
\center{\includegraphics[scale=0.6]{sy.pdf}}
\caption{├ЁрЇшъш яЁшсышцхэшщ $y_0(t)$ (яєэъЄшЁэр  ышэш ), $y_2(t)$ (ёяыю°эр  ышэш ) ш уЁрЇшъ ёхЄюўэющ ЇєэъЎшш $\{(t_i,\tilde{y}_i)\}$ (Єюўъш), яюыєўхээющ ьхЄюфюь ╨єэух--╩єЄЄр}
\label{sy}
\end{figure}
═р Ёшё. \ref{sx} ш \ref{sy} яюърчрэ√ уЁрЇшъш яхЁтюую ш ЄЁхЄ№хую яЁшсышцхэшщ $x_0(t),~y_0(t)$ (яєэъЄшЁэр  ышэш ), $x_2(t),~y_2(t)$ (ёяыю°эр  ышэш ) ш уЁрЇшъш ёхЄюўэ√ї ЇєэъЎшщ $\{(t_i,\tilde{x}_i)\}$, $\{(t_i,\tilde{y}_i)\}$ (Єюўъш), яюыєўхээ√х ьхЄюфюь ╨єэух--╩єЄЄр.
\end{proof}
\section{╟ръы■ўхэшх}
┬ фрээющ ёЄрЄ№х с√ыш юсюс∙хэ√ шчтхёЄэ√х ьхЄюф√ яЁшьхэхэш  тхщтыхЄют ╒ррЁр ъ яЁшсышцхээюьє Ёх°хэш■ ышэхщэ√ї шэЄхуЁры№э√ї ш фшЇЇхЁхэЎшры№э√ї єЁртэхэшщ. ▌Єш ьхЄюф√ яюыєўр■Єё  шч яЁхфёЄртыхээ√ї чфхё№ яЁш $n=0$, ўЄю ш ёююЄтхЄёЄтєхЄ тхщтыхЄрь ╒ррЁр. ┬ юЄышўшх юЄ тхщтыхЄют ╒ррЁр, уфх яЁшсышцхэш  Ёх°хэш  шэЄхуЁры№эюую єЁртэхэш  яюыєўр■Єё  ъєёюўэю-яюёЄю ээ√ьш, р яЁшсышцхэш  Ёх°хэш  фшЇЇхЁхэЎшры№эюую єЁртэхэш  яЁшэрфыхцрЄ ъырёёє уырфъюёЄш $C^{k-1}$, уфх $k$ -- яюЁ фюъ єЁртэхэш , шёяюы№чютрэшх ёяырщэ-тхщтыхЄ фрхЄ тючьюцэюёЄ№ ёЄЁюшЄ№ яЁшсышцхэш  ы■сюую ъырёёр уырфъюёЄш $C^n$.
%========================================================================================
\begin{thebibliography}{99}
\bibitem{Lepik3} \textit{Lepik  $\overset{..}{U}$}. Application of the Haar wavelet transform to solving integral and differential equations // Proc. Estonian Acad. Sci. Phys. Math.,  2007. Vol.~56. P.~28--46.
\bibitem{Blatov} \textit{┴ырЄют ╚.\,└., ╨юуютр ═.\,┬.} ╧юыєюЁЄюуюэры№э√х ёяырщэют√х тхщтыхЄ√ ш ьхЄюф ├рыхЁъшэр ўшёыхээюую ьюфхышЁютрэш  ЄюэъюяЁютюыюўэ√ї рэЄхэ // ┬√ўшёы. ьрЄхь. ш ьрЄхь. Їшч., 2013. ╥.~53. \textnumero{ 5}. C.~727--736.
\bibitem{Lepik1} \textit{Lepik  $\overset{..}{U}$}. Numerical solution of evolution equations by the Haar wavelet method //  Appl. Math. Comput., 2007.  Vol.~185. P.~695--704.
\bibitem{Lepik2} \textit{Lepik  $\overset{..}{U}$}. Haar wavelet method for solving higher order differential equations // Int. J. Math. Comput., 2008. Vol.~1. No.~8. P.~84--94.
\bibitem{Lepik4}  \textit{Lepik  $\overset{..}{U}$}. Numerical solution of differential equations using Haar wavelets //  Math. Comput. Simul., 2005. Vol.~68. P.~127--143.
\bibitem{Lepik} \textit{Lepik  $\overset{..}{U}$, Hein H}.  Haar wavelets with applications. --  Berlin: Springer, 2014. 207~p.
\bibitem{ArticleFinkelstein}  \textit{Finkelstein A., Salesin D.\,H.} Multiresolution curves // Proceedings of SIGGRAPH. --  New York: ACM, 1994. P.~261--268.
%\bibitem{Wilkinson} Wilkinson, J.H. The algebraic eigenvalue problem. -- Clarendon Press, Oxford, 1965.
%\bibitem{Demko} Demko, S., W.F. Moss and P.W. Smith. Decay rates for inverses of band matrices. 1984. Math. Comp. V. 43, \textnumero{167}. 491--499.
\bibitem{Frazer} \textit{Frazier M.\,W}. An introduction to wavelets through linear algebra. -- New York: Springer,  1999.  503~p.
\bibitem{Chui}  \textit{Chui Ch. ╩}.  An introduction to wavelets. -- Boston: Academic press, 1991. 412~p.
\bibitem{Yurgu} \textit{Bityukov Yu.\,I.,  Akmaeva V.\,N. }  The use of wavelets in the mathematical and computer modelling of manufacture of the complex-shaped shells made of composite materials //  Bulletin of the South Ural State University. Ser. Mathematical Modelling, Programming and Computer Software, 2016. Vol.~9. No.~3. P.~5--16.
\bibitem{Akilov}  \textit{╩рэЄюЁютшў ╦.\,┬.,  └ъшыют ├.\,╧.} ╘єэъЎшюэры№э√щ рэрышч. -- ╠.: ═рєър, 1977. 744~ё.
\bibitem{Novikov}  \textit{═ютшъют ╚.\,▀.,  ╧ЁюЄрёют ┬.\,▐.,  ╤ъюяшэр ╠.\,└.} ╥хюЁш  тёяыхёъют. -- ╠.: ╘╚╟╠└╥╦╚╥, 2005.  612~c.
\bibitem{BookSmolencev} \textit{╤ьюыхэЎхт ═.\,╩}. ╬ёэют√ ЄхюЁшш тхщтыхЄют. ┬хщтыхЄ√ т MatLab. -- ╠.: ─╠╩ ╧Ёхёё,  2005. 303~c.
%\bibitem{Pupkov} ╧єяъют, ╩.└., ═.─. ┼уєяют. ╠хЄюф√ ъырёёшўхёъющ ш ёютЁхьхээющ ЄхюЁшш ртЄюьрЄшўхёъюую єяЁртыхэш . ╠рЄхьрЄшўхёъшх ьюфхыш, фшэрьшўхёъшх їрЁръЄхЁшёЄшъш ш рэрышч ёшёЄхь ртЄюьрЄшўхёъюую єяЁртыхэш . ╠.: ╚чфрЄхы№ёЄтю ╠├╥╙ шь. ═.▌. ┴рєьрэр. 2004. 656 ё.
\end{thebibliography}
\begin{center}
{\bf The use of wavelets for the calculation of linear control systems with lumped parameters}\par
{Y.I. Bityukov$^1$, E.N. Platonov$^2$}
\end{center}
 $^1$Moscow Aviation Institute (National Research University), Volokolamskoye Highway 4, Moscow, A-80, GSP-3, 125993; yib72@mail.ru\\
 $^2$Moscow Aviation Institute (National Research University), Volokolamskoye Highway 4, Moscow, A-80, GSP-3, 125993;  en.platonov@gmail.com

\noindent {\bf Abstract:} In many disciplines, problems appear which can be formulated with the aid of differential or integral equations.  In simpler cases, such equations can be solved analytically, but for more complicated cases, numerical procedures are needed. In recent times, the wavelet-based methods have gained great popularity, where different wavelet families such as Daubechies, Coiflet, etc., wavelets are applied. A shortcoming of these wavelets is that they do not have an analytic expression. For this reason, differentiation and integration of these wavelets are very complicated. The paper presents algorithms for the numerical solution of linear integral and differential equations based on spline wavelets on the interval. The algorithms generalize the well-known methods based on Haar wavelets, which are a particular case of spline wavelets. The results presented in the paper can be applied for the analysis of linear systems with lumped parameters.

\noindent {\bf Keywords}: spline wavelet; differential equation; integral equation
\begin{center}
{\bf Acknowledgments}
\end{center}
This work is a part of Project No 2.2461.2017 supported by the Russian Ministry of Education and Science.
\begin{center}
{\bf References}
\end{center}
\begin{enumerate}
\item Lepik,  $\overset{..}{\mathrm{U}}$. 2007. Application of the Haar wavelet transform to solving integral and differential equations. \textit{Proc. Estonian Acad. Sci. Phys. Math.} 56:28-46.
\item  Blatov,  I. A., and  N. V. Rogova. 2013.  Poluortogonal'nye splaynovye veyvlety i metod Galerkina chislennogo modelirovaniya tonkoprovolochnykh antenn [Semi-orthogonal spline wavelets and Galerkin's method of numerical simulation of thin-wire antennas]. \textit{Vychisl. matem. i matem. fiz.} [ \textit{Computational Mathematics and Mathematical Physics}] 53(5): 727--736.
\item Lepik, $\overset{..}{\mathrm{U}}$. 2007. Numerical solution of evolution equations by the Haar wavelet method. \textit{Appl. Math. Comput.} 185: 695--704.
\item Lepik,  $\overset{..}{\mathrm{U}}$. 2008. Haar wavelet method for solving higher order differential equations. \textit{Int. J. Math. Comput.} 1(8): 84--94.
\item  Lepik,  $\overset{..}{\mathrm{U}}$. 2005. Numerical solution of differential equations using Haar wavelets. \textit{ Math. Comput. Simul.} 68:127--143.
\item Lepik,  $\overset{..}{\mathrm{U}}$, and H. Hein. 2014. \textit{Haar wavelets with applications.}  Berlin: Springer. 207~p.
\item Finkelstein, A., and Salesin D. H. 1994. Multiresolution curves. Proceedings of SIGGRAPH.  New York: ACM. 261--268.
%\item Demko, S., W.F. Moss and P.W. Smith. Decay rates for inverses of band matrices. 1984. Math. Comp. V. 43, \textnumero{167}. 491--499.
\item Frazier, M. W. 1999. \textit{ An introduction to wavelets through linear algebra.} New York: Springer. 503 p.
\item Chui,  Ch. ╩. 1991. \textit{An introduction to wavelets.} Boston: Academic press. 412 p.
\item Bityukov, Yu. I., and  V. N. Akmaeva. 2016. The use of wavelets in the mathematical and computer modelling of manufacture of the complex-shaped shells made of composite materials.   \textit{Bulletin of the South Ural State University. Ser. Mathematical Modelling, Programming and Computer Software}. 9(3):5--16.
\item  Kantorovich, L. V., and G. P. Akilov. 1977. \textit{Funktsional'nyy analiz} [\textit{Functional analysis}]. Moscow: Nauka. 744 p.
\item   Novikov, I. Ya.,  V. Yu. Protasov, and M. A. Skopina. 2005. \textit{Teoriya vspleskov} [\textit{The theory of wavelets}]. Moscow: FIZMATLIT.  612 p.
\item  Smolentsev, N. K. 2005. \textit{Osnovy teorii veyvletov. Veyvlety v MatLab} [\textit{Fundamentals of the theory of wavelets. Wavelets in MatLab}].  Moscow: DMK Press. 303 p.
\end{enumerate}

\end{document} 