\def\stat{grebeshkov}

\def\tit{АНАЛИЗ ВРЕМЕНИ ПЕРЕКЛЮЧЕНИЯ СЕАНСА СВЯЗИ В~ГЕТЕРОГЕННЫХ 
БЕСПРОВОДНЫХ СЕТЯХ ПРИ~ВЕРТИКАЛЬНОМ ХЭНДОВЕРЕ$^*$}

\def\titkol{Анализ времени переключения сеанса связи в~гетерогенных 
беспроводных сетях при вертикальном хэндовере}

\def\aut{А.\,Ю.~Гребешков$^1$, Ю.\,В.~Гайдамака$^2$, О.\,Г.~Вихрова$^3$, 
Э.\,Р.~Зарипова$^4$}

\def\autkol{А.\,Ю.~Гребешков, Ю.\,В.~Гайдамака, О.\,Г.~Вихрова, 
Э.\,Р.~Зарипова}

\titel{\tit}{\aut}{\autkol}{\titkol}

\index{Гребешков А.\,Ю.}
\index{Гайдамака Ю.\,В.}
\index{Вихрова О.\,Г.} 
\index{Зарипова Э.\,Р.}
\index{Grebeshkov A.\,Yu.}
\index{Gaidamaka Yu.\,V.}
\index{Vikhrova O.\,G.} 
\index{Zaripova E.\,R.}



{\renewcommand{\thefootnote}{\fnsymbol{footnote}} \footnotetext[1]
{Публикация подготовлена при финансовой поддержке Минобрнауки России (проект 2.882.2017/4.6).}}


\renewcommand{\thefootnote}{\arabic{footnote}}
\footnotetext[1]{Поволжский государственный университет телекоммуникаций и~информатики, 
\mbox{grebeshkov-ay@psuti.ru}}
\footnotetext[2]{Российский университет дружбы народов; Институт проблем информатики Федерального исследовательского 
центра <<Информатика и~управ\-ле\-ние>> Российской академии наук, 
\mbox{gaydamaka\_yuv@rudn.university}}
\footnotetext[3]{Российский университет дружбы народов, vikhrova\_og@rudn.university}
\footnotetext[4]{Российский университет дружбы народов, zaripova\_er@rudn.university}

\vspace*{4pt}
 

\Abst{Для мобильных абонентов гетерогенной беспроводной сети в~некоторых точках могут 
быть одновременно доступны соединения в~перекрывающих друг друга областях покрытия 
радиосетей различных стандартов. Пользователь с~современным многорежимным 
абонентским устройством может переключаться между различными сетями радиодоступа 
для получения требуемых услуг связи с~помощью процедуры вертикального хэндовера 
(vertical handover, VHO). Для обеспечения качества и~непрерывности связи существенное 
значение имеет время переключения сеанса связи из текущей в~целевую сеть. 
Разработана процедура вертикального хэндовера из беспроводной локальной сети (Wireless 
Local Area Network, WLAN) в~мобильную сеть (3GPP Long Term Evolution, LTE). Проведен 
анализ среднего значения и~квантиля времени переключения с~помощью метода оценки 
времени пребывания заявок в~многофазной системе массового обслуживания (СМО) с~фоновым 
трафиком. Проведен численный эксперимент для близких к~реальным исходных данных 
процесса переключения сеанса связи.}

\KW{гетерогенная беспроводная сеть; сотовая сеть; LTE; мобильность; вертикальный 
хэндовер; надежность соединения; доступность соединения; показатель эффективности; 
процедура установления соединения}

\DOI{10.14357/19922264170409} 


\vskip 10pt plus 9pt minus 6pt

\thispagestyle{headings}

\begin{multicols}{2}

\label{st\stat}

\section{Введение}

\vspace*{-2pt}

  Тенденция развития современных сотовых сетей, приводящая к~увеличению 
емкости сетей за счет их пространственного уплотнения и~совершенствования 
методов управления распределением радиоресурса, соответствует концепции 
HetNet (Heterogeneous Network)~--- гетерогенных сетей. Реализация этой 
концепции стала возможной в~беспроводных сетях стандарта LTE/LTE-A~[1]. 
Исследования таких сетей с~точки зрения различных показателей качества 
активно ведутся в~России~[2--4]. После выбора целевой сети переключение 
сеанса связи многорежимного абонентского устройства (user equipment, UE) 
осуществляется с~использованием процедуры, которая называется 
вертикальным хэндовером (VHO). 

Использование VHO 
позволяет повысить качество обслуживания, например при использовании 
тактильного Интернета~\cite{5-gre}, для поддержки приложений, работающих 
в~реальном времени~\cite{6-gre}. Поэтому актуально проведение исследований и~анализ времени переключения с~целью предотвращения потери информации, 
что особенно важно в~приложениях реального времени. Критерии принятия 
решения о VHO рассмотрены в~\cite{7-gre}. В~\cite{8-gre} при анализе 
исполнения VHO не рассматриваются процедуры авторизации в~целевой сети, 
которые необходимы при VHO. В~\cite{9-gre} на основе теоремы Бёрке 
рассматривается модель аутентификации и~связанные с~этим временн$\acute{\mbox{ы}}$е 
задержки без учета времени на получение информации о параметрах целевой 
сети для VHO. В~\cite{10-gre, 11-gre} не учтено время подключения к~целевой 
сети и~время выделения радиоканала. 

Отличием данной работы  
от~\cite{7-gre, 8-gre, 9-gre, 10-gre, 11-gre} является детальная процедура обмена 
сигнальными сообщениями при вертикальном хэндовере~\cite{12-gre}, которая 
содержит все стадии, начиная от момента получения информации о доступных 
сетях и~заканчивая началом IP (Internet Protocol) сес\-сии абонентского устройства в~целевой 
сети. 

\begin{figure*}[b] %fig1
\vspace*{6pt}
 \begin{center}
 \mbox{%
 \epsfxsize=131.756mm 
 \epsfbox{gre-1.eps}
 }
 \end{center}
\vspace*{-9pt}
\Caption{Обмен сообщениями UE с~ANDSF и~авторизация UE в~сети LTE}
\end{figure*} 

На основе описанной в~разд.~2 процедуры обмена сигнальными 
сообщениями при вертикальном хэндовере в~разд.~3 разработан метод оценки 
времени переключения при вертикальном хэндовере~[13, 14]. С~его помощью 
в~разд.~4 проведена оценка среднего времени VHO с~учетом загрузки узлов 
текущей и~целевых сетей.



\section{Процедура обмена сигнальными сообщениями 
при~вертикальном хэндовере из~недоверенной сети в~сеть LTE}

  Для разработки процедуры принимаются исходные положения, которые не 
нарушают ее целостность и~не ограничивают применение. Вертикальный 
хэндовер осуществляется сервером ANDSF (Access Network Discovery and 
Selection Function). Для обмена данными между UE и~узлом ANDSF 
используется специфицированная партнерством 3GPP 
(3rd Generation Partnership Project) эталонная точка стыка 
S14 уровня приложений. Поддержка мобильности IP при VHO 
в~рассматриваемом примере обеспечивается применением мобильной версии 
протокола с~двойным стеком IPv6~[15, 16]. Для обеспечения без\-опас\-ности 
используется протокол IPsec, идентификация пользователя с~помощью 
шифрования открытым ключом IKE (Internet Key Exchange)~[17].

\begin{figure*}[b] %fig2
\vspace*{6pt}
 \begin{center}
 \mbox{%
 \epsfxsize=132.09mm 
 \epsfbox{gre-2.eps}
 }
 \end{center}
\vspace*{-9pt}
\Caption{Обмен сообщениями при установлении соединения UE с~целевой сетью для VHO 
WLAN-LTE}
\end{figure*}
  
  Пусть многорежимное устройство UE работает в~текущей WLAN-сети, 
авторизация и~аутентификация в~которой не соответствуют спецификациям 
3GPP. С~точки зрения оператора сети LTE текущая сеть считается 
недоверенной (nontrusted) IP-сетью доступа, поэтому UE инициирует VHO 
в~целевую сеть LTE. Функциональными устройствами, вовлеченными 
в~процедуру, являются: UE, ANDSF, шлюз пакетных данных ePDG (evolved 
Packet Data Gateway), расширенная сеть радиодоступа E-UTRAN (evolved 
UMTS (Universal Mobile Telecommunication
System) Terrestrial Radio Access Networks), узел управления мобильностью MME 
(Mobility Management Entity), обслуживающий шлюз S-GW (Serving Gateway), 
пакетный шлюз P-GW (Packet Data Network Gateway), узел выставления счетов 
абонентам hPCRF (home network Policy and Charging Rules Function) 
и~комбинированный сервер HSS/AAA (Home Subscriber 
Server\,/\,Authentication, Authorization, and Accounting). 

Подробное описание 
сигнальных сообщений, включаемых в~процедуру вертикального хэндовера, 
приведено в~\cite{14-gre}.
%
Обмен сигнальными сообщениями для данного этапа 
VHO представлен на рис.~1. 
{\looseness=1

}
  
  В завершающем этап авторизации сообщении (23) UE получает от ePDG 
данные об успешной ав\-то\-ризации в~сети LTE. После этого программное\linebreak 
обеспечение UE реконфигурируется для работы в~сети LTE по туннелю IPsec 
через эталонную точку стыка 3GPP~S2c. 
  



  Вторым этапом исполнения VHO является переключение UE в~целевую сеть 
LTE из текущей недоверенной сети WLAN. Схема обмена сигнальными 
сообщениями на этом этапе показана на рис.~2. 

\begin{figure*}[b] %fig3
\vspace*{1pt}
 \begin{center}
 \mbox{%
 \epsfxsize=151.044mm 
 \epsfbox{gre-3.eps}
 }
 \end{center}
\vspace*{-9pt}
\Caption{Модель многофазной СМО с~фоновым трафиком: \textit{1}~--- основной поток
заявок; \textit{2}~--- фоновый поток заявок}
\end{figure*}
  
  В начале этапа~2 UE синхронизировано с~расширенной сетью радиодоступа 
E-UTRAN, имеет информацию о физическом канале и~временный 
идентификатор радиосоты. Сообщения~(24)--(40)\linebreak
 отвечают за реконфигурацию 
физических каналов. После сообщения~(37), в~котором узел~\mbox{S-GW} 
подтверждает создание требуемого канала для поддержки IP-сес\-сии через 
расширенную систему пакетной передачи данных EPS (Evolved Packet System) 
вместо обмена через WLAN, UE начинает\linebreak принимать и~передавать пакеты 
данных через сеть LTE. 
  


\section{Метод оценки времени переключения с~учетом фонового 
трафика}

  В настоящей работе предлагается использовать приближенный метод оценки 
времени переключения с~учетом наличия в~сети фонового трафика~[18]. Для 
перехода к~аналитической модели пронумеруем введенные в~предыдущих 
разделах функциональные устройства: UE~(I), ANDSF~(II), ePDG~(III),  
E-UTRAN~(IV), MME~(V), S-GW~(VI), P-GW~(VII), hPCRF~(VIII) 
и~HSS/AAA~(IX). Используемая методика подразумевает разбиение потока 
обслуживаемых в~каждом узле сигнальных сообщений на основной и~фоновый 
потоки. Под сообщениями основного потока будем понимать включенные 
в~процедуру вертикального хэндовера сигнальные сообщения, под фоновым 
потоком~--- сообщения других задач. Поток сигнальных сообщений на рис.~3 
образует цепь, состоящую из $K\hm=39$~со\-сто\-яний. 



  Обозначим через~$\lambda_0$ и~$b_k$ интенсивность входящего потока 
и~среднюю длительность обслу\-жи\-вания заявок основного потока на $k$-й фазе 
в~многофаз\-ной СМО. Аналогично~$\lambda_k$ и~$d_k$~--- интенсивность 
потока и~средняя длительность обслуживания заявок фонового потока на $k$-й 
фазе. 
  
  Для расчета времени пребывания заявки в~многофазной СМО необходимо 
вычислить коэффициент вариации длительности обслуживания на $k$-й фазе 
по формуле:
  \begin{equation*}
  C_k^2 =\fr{(\lambda_0+\lambda_k) \left(\lambda_0 b_k^{(2)} +\lambda_k 
d_k^{(2)}\right)} {(\lambda_0 b_k+\lambda_k d_k)^2}-1,\enskip
  k=1,\ldots, K.
 % \label{e1-gre}
  \end{equation*}
Здесь использованы обозначения вторых моментов времени обслуживания 
заявок основного~$b_k^{(2)}$ и~фонового потока~$d_k^{(2)}$.

  Время ожидания начала обслуживания, полученное из известной формулы 
Пол\-ла\-че\-ка--Хин\-чи\-на, рассчитывается по формуле:
  $$
  \omega_k= \fr{\rho_k^2(1+C_k^2)} {2(\lambda_0+\lambda_k) (1-
\rho_k)}\,,\enskip k=1,\ldots, K\,.
  $$
Здесь $\rho_k=\lambda_0b_k\hm+\lambda_kd_k$~--- суммарная нагрузка на узел, 
соответствующий $k$-й фазе, $\rho_k\hm< 1$, $k\hm=1,\ldots, K$.

  Время пребывания~$\Delta$ заявки в~многофазной СМО равно сумме времен 
пребывания заявок основного потока на каждой фазе:
  $$
  \Delta= \sum\limits^K_{k=1} \left( \omega_k+b_k\right)\,.
  $$
  
  Время пребывания~$\Delta$ заявки в~многофазной СМО с~учетом фонового 
трафика соответствует времени переключения при вертикальном хэндовере.
  
  Заметим, что данный метод позволяет найти квантиль уровня~$\psi$ времени 
пребывания заявки в~многофазной СМО с~фоновым трафиком по формуле:
  $$
  Q_\psi= q_\psi +\sum\limits^K_{k=1} \left( \fr{\ln (\gamma_k 
\omega_k)}{\gamma_k}+b_k\right)\,,
  $$
где $q_\psi$ является единственным положительным корнем уравнения
$$
1-\psi= \sum\limits^{K-1}_{k=0} e^{-\gamma_k q_\psi} \fr{(\gamma_k q_\psi)^k} 
{k!}\,.
$$
  
  Параметры $\gamma_k$ затухания функций распределения времени 
ожидания начала обслуживания на $k$-й фазе, в~свою очередь, являются 
единственными положительными корнями уравнения 
$$
\alpha_k(\gamma_k)\beta_k(-\gamma_k)=1\,,
$$
 где $\alpha_k(s)\hm= 
(\lambda_0\hm+ \lambda_k)/(\lambda_0\hm+\lambda_k\hm+ s)$~--- 
преоб\-разование Лап\-ла\-са--Стилть\-еса (ПЛС) функции\linebreak распределения (ФР) 
интервалов времени между поступлениями заявок в~узле на $k$-й фазе, 
а~$\beta_k(s)\hm= (\lambda_0/(\lambda_0\hm+\lambda_k)) e^{-sb_k}\hm+ 
(\lambda_k/(\lambda_0\hm+ \lambda_k))e^{-sd_k}$~--- ПЛС ФР длительности 
обслуживания заявок.

  \begin{table*}[b]
  {\small \begin{center}
  
  \begin{tabular}{|l|c|c|}
  \multicolumn{3}{c}{Средние времена обслуживания}\\
  \multicolumn{3}{c}{\ }\\[-6pt]
  \hline
  \tabcolsep=0pt\begin{tabular}{c}Функциональные\\
узлы\end{tabular}&Среднее время обслуживания
$\mu_i^{-1}$, мс&  \tabcolsep=0pt\begin{tabular}{c}Источники данных\\
 для численного эксперимента\end{tabular}\\
\hline
I~--- UE&\tabcolsep=0pt\begin{tabular}{c}77,5 для (24);\\
28,5 для (26);\\
2 для остальных\end{tabular}&\cite{19-gre}\\
\hline
II~--- ANDSF&70&Считается, как HSS/AAA\\
\hline
III~--- ePDG&\hphantom{9}2&Считается, как P-GW\\
\hline
IV~--- eNB&\hphantom{9}4&\cite{20-gre}\\
\hline
V~--- MME&\tabcolsep=0pt\begin{tabular}{c}15 для (27);\\
1 для остальных\end{tabular}&\cite{20-gre, 21-gre}\\
\hline
VI~--- S-GW&\hphantom{9}2&\cite{19-gre}\\
\hline
VII~--- P-GW&\hphantom{9}2&\cite{19-gre}\\
\hline
VIII~--- hPCRF&70&Считается, как HSS/AAA\\
\hline
IX~--- HSS/AAA&70&\cite{22-gre}\\
\hline
\end{tabular}
\end{center}}
%\end{table*}
%\renewcommand{\figurename}{\protect\bf Рис.}
\renewcommand{\tablename}{\protect\bf Рис.}
\setcounter{table}{3}
%\begin{figure*} %fig4
\vspace*{6pt}
 \begin{center}
 \mbox{%
 \epsfxsize=164.269mm 
 \epsfbox{gre-4.eps}
 }
 \end{center}
\vspace*{-9pt}
\Caption{Среднее время переключения~(\textit{а}) и~95\%-ный квантиль времени переключения~(\textit{б}):
\textit{1}~--- $\lambda_k\hm=\lambda_0$; 
\textit{2}~--- $\lambda_k\hm=10\lambda_0$; \textit{3}~--- 
$\lambda_k\hm=100\lambda_0$}
%\end{figure*}
\end{table*}

\renewcommand{\figurename}{\protect\bf Рис.}
\renewcommand{\tablename}{\protect\bf Таблица}
  
  
\section{Численный эксперимент}

  Интенсивность запросов на совершение вертикального хэндовера зависит от 
местности, типа устройств UE, возможностей оператора связи, плотности 
мобильных пользователей и~других параметров. В~таблице приведены средние 
значения времен обслуживания сообщений в~узлах (основной трафик). 
Некоторые сообщения обслуживаются дольше остальных в~связи с~их 
функциональными особенностями и~б$\acute{\mbox{о}}$льшим объемом, при 
применении метода следует использовать данные статистических наблюдений. 
  

  На рис.~4 представлены результаты анализа времени переключения при 
вертикальном хэндовере для трех вариантов соотношения интенсивностей 
основного и~фонового трафиков: \textit{1}~--- $\lambda_k\hm=\lambda_0$; 
\textit{2}~--- $\lambda_k\hm=10\lambda_0$; \textit{3}~--- 
$\lambda_k\hm=100\lambda_0$. Среднее время обслуживания фонового трафика 
$d_k\hm=2$~мс.
  


  
  Анализ времени переключения показал, что дополнительный трафик со 
средним временем обслуживания несущественно влияет на время 
переключения, когда интенсивность этого трафика имеет тот же порядок, что 
и~интенсивность основного трафика (кривые~\textit{1} и~\textit{2}), 
и~проявляется, когда отношения интенсивностей основного и~фонового 
трафиков различаются на два порядка (кривые~\textit{3} на рис.~4).
  
  

На рис.~5 показана зависимость среднего значе\-ния и~95\%-ного квантиля 
времени переклю\-чения при вертикальном хэндовере для трех значений средней 
длительности обслуживания\linebreak сообщения фонового трафика $d_k\hm=10$, 20 и~50~мс, интенсивности входящего основного и~фонового трафика равны, 
$\lambda_k\hm= \lambda_0$.

\setcounter{figure}{4}

\begin{figure*} %fig5
\vspace*{1pt}
 \begin{center}
 \mbox{%
 \epsfxsize=163.863mm 
 \epsfbox{gre-6.eps}
 }
 \end{center}
\vspace*{-9pt}
\Caption{Среднее время переключения~(\textit{а})
и~95\%-ный квантиль времени переключения:
\textit{1}~--- $d_k\hm=10$~мс;  
\textit{2}~--- 20; \textit{3}~--- $d_k\hm=50$~мс}
\end{figure*}



     При одинаковой интенсивности фонового и~основного трафика 
и~увеличении среднего времени обслуживания фонового трафика до~50~мс 
наблюдается резкое увеличение времени переключения при вертикальном 
хэндовере (кривые~\textit{3} на рис.~5). Однако при среднем времени 
обслуживания фонового трафика до~20~мс фоновый трафик практически не 
влияет на время переключения при вертикальном хэндовере 
(кривые~\textit{1} и~\textit{2} на рис.~5).
     
  Рекомендуемый метод оценки позволяет учесть влияние фонового трафика 
  и~на отдельные узлы сети, участвующие в~процедуре. Тем не менее по 
результатам проведения статистических наблюдений на реальных сетях связи 
указанная рекомендация может уточняться.

\section{Заключение}

  Применение процедуры вертикального хэндовера в~гетерогенных 
беспроводных сетях в~сочетании с~использованием многорежимных 
абонентских устройств открывает широкие возможности по 
дифференцированному доступу абонентов к~ресурсам сетей. 
  
  В статье разработана процедура обмена сигнальными сообщениями для VHO 
из беспроводной локальной сети в~сеть LTE, предложен аналитический метод 
оценки времени переключения. При этом пользователями могут быть как 
устройства пользователей, так и~<<умные>> устройства, автоматически 
взаимодействующие по принципу M2M (Machine-to-Machine). 
  
  В дальнейших исследованиях планируется применить представленный метод 
для оценки времени переключения при обратной процедуре из сетей 
подвижной беспроводной связи LTE в~сеть WLAN.
  
{\small\frenchspacing
 {%\baselineskip=10.8pt
 \addcontentsline{toc}{section}{References}
 \begin{thebibliography}{99}
\bibitem{1-gre}
\Au{Astely D., Dahlman~E., Fodor~G., Parkvall~S., Sachs~J.} LTE release~12 and 
beyond~// IEEE Commun. Mag., 2013. Vol.~51. No.\,7. P.~154--160.
doi: 10.1109/MCOM. 2013.6553692. 
\bibitem{3-gre} %2
\Au{Горбунова А.\,В., Зарядов~И.\,С., Матюшенко~С.\,И., Самуйлов~К.\,Е., 
Шоргин~С.\,Я.} Аппроксимация времени отклика системы облачных 
вычислений~// Информатика и~её применения, 2015. Т.~9. Вып.~3. С.~32--38.
\bibitem{2-gre} %3
\Au{Вихрова О.\,Г., Самуйлов~К.\,Е., Сопин~Э.\,С., Шоргин~С.\,Я.} К~анализу 
показателей качества обслуживания в~современных беспроводных сетях~// 
Информатика и~её применения, 2015. Т.~9. Вып.~4. С.~48--55.

\bibitem{4-gre}
\Au{Гайдамака Ю.\,В., Андреев~С.\,Д., Сопин~Э.\,С., Самуйлов~К.\,Е., 
Шоргин~С.\,Я.} Анализ характеристик интерференции в~модели 
взаимодействия устройств с~учетом среды распространения сигнала~// 
Информатика и~её применения, 2016. Т.~10. Вып.~4. С.~2--10.
\bibitem{5-gre}
\Au{Кучерявый А.\,Е., Маколкина~М.\,А., Киричек~Р.\,В.} Тактильный 
Интернет. Сети связи со сверхмалыми задержками~// Электросвязь, 2016. №\,1. 
С.~44--46.
\bibitem{6-gre}
\Au{Kellokoski J., Koskinen~J., Nyrhinen~R., 
H$\ddot{\mbox{a}}$m$\ddot{\mbox{a}}$l$\ddot{\mbox{a}}$inen T.} Efficient 
handovers for machine-to-machine communications between IEEE 802.11 and 3GPP 
evolved packed core networks~//  IEEE  Conference (International) on Green 
Computing and Communications Proceedings.~--- 
\mbox{Besan{\!\ptb{\c{c}}}on}, 2012. P.~722--725.
\bibitem{7-gre}
\Au{Ahmed L., Boulahia~L.\,M., Gaiti~D.} Enabling vertical handover decisions in 
heterogeneous wireless networks: A~state-of-the-art and a classification~// IEEE 
Commun. Surv.  Tut., 2014. Vol.~16. No.\,2. P.~776--781.
\bibitem{8-gre}
\Au{Bukhari J., Akkari~N.} QoS based approach for LTE-WiFi handover~// 7th 
Conference (International) on Computer Science \& Information Technology 
Proceedings.~--- Elsevier, 2016. P.~1--6.
\bibitem{9-gre}
\Au{Gondim P.\,R.\,L., Trineto~J.\,B.\,M.} DSMIP and PMIP for mobility 
management of heterogeneous access networks: Evaluation of authentication 
delay~//  IEEE Globecom Workshops Proceedings.~--- Anaheim, 2012.  
P.~308--313.

\bibitem{11-gre} %10
\Au{Do-Hyung~K., Won-Tae~K., Hwan-Gu~L., Sun-Ja~K., Cheol-Hoon~L.} 
A~performance evaluation of vertical handover architecture with low latency 
handover~//  Conference (International) on Convergence and Hybrid Information 
Technology Proceedings.~--- Dusan, 2008. P.~66--69.

\bibitem{10-gre} %11
\Au{Tsagkaropoulos M., Politis~I., Tselios~C., Dagiuklas~T., Kotsopoulos~S.} 
Service continuity over intertechnology RATs~// 16th IEEE  Workshop 
(International) on Computer Aided Modeling and Design of Communication Links 
and Networks Proceedings.~--- Kyoto, 2011. P.~117--121.

\bibitem{12-gre}
\Au{M$\acute{\mbox{a}}$rquez-Barja J., Calafate~C.\,T., Cano~J.-C., Manzoni~P.} 
An overview of vertical handover techniques: Algorithms, protocols and tools~// 
Comput. Commun., 2011. Vol.~34. P.~985--997.
\bibitem{13-gre}
\Au{Гребешков А.\,Ю.} Оценка целесообразности обработки заявки для 
предоставления услуги в~реконфигурируемых сетях следующего поколения~// 
T-Comm: Телекоммуникации и~транспорт, 2014. №\,8. С.~24--27. 
\bibitem{14-gre}
\Au{Grebeshkov A., Zaripova~E., Roslyakov~A., Samouylov~K.} Modelling of 
vertical handover from untrusted WLAN network to LTE~// 31st European 
Conference on Modelling and Simulation Proceedings, 2017. P.~694--700.
\bibitem{15-gre}
\Au{Lampropoulos G., Passas~N., Mekaros~L., Kaloxylos~A.} Handover 
management architectures in integrated WLAN~// IEEE Commun. 
Surv. Tut., 2005. Vol.~7. No.\,4. P.~30--44.
\bibitem{16-gre}
3GPP TS~23.402 Technical specification 3GPP; TS Group Services and System 
Aspects; Architecture enhancements for non-3GPP accesses. Release~14, 2016.
{\sf 
https://\linebreak portal.3gpp.org/desktopmodules/Specifications/Specifi cationDetails.aspx?specificationId=850}.
\bibitem{17-gre}
3GPP TS 33.402 Technical specification 3GPP; TS Group Services and System 
Aspects; 3GPP System Architecture Evolution (SAE); Security aspects of non-3GPP 
accesses. Release~14, 2016.
{\sf  
https://portal. 3gpp.org/desktopmodules/Specifications/Specification\linebreak Details.aspx?specificationId=2297.}
\bibitem{18-gre}
\Au{Gaidamaka Yu., Zaripova~E.} Session setup delay estimation methods for  
IMS-based IPTV services~// Internet of things, smart spaces, and next generation networks
and systems~/ Eds.\ S.~Balandin, S.~Andreev, Y.~Koucheryavy.~---
Lecture notes in computer science ser.~---
Springer, 2014. Vol.~8638. 
P.~408--418.
\bibitem{19-gre}
\Au{Nikaein N., Krco~S.} Latency for real-time machine-to-machine communication 
in LTE-based system architecture~// 17th European Wireless Conference on 
Sustainable Wireless Technologies Proceedings.~--- Vienna, Austria: IEEE, 2011. 
P.~1--6.
\bibitem{20-gre}
\Au{Cardona N., Monserrat~J.\,F., Cabrejas~J.} Enabling technologies for 3GPP 
LTE-advanced networks~// LTE-advanced and next generation wireless networks~/ 
Eds.\ G.~de la~Roche, A.\,A.~Glazunov, B.~Allen.~--- Chichester, U.K.: John 
Wiley and Sons, 2013. P.~3--34.
\bibitem{21-gre}
\Au{Prados-Garzon J., Ramos-Munoz~J.\,J., Ameigeiras~P., Andres-Maldonado~P., 
Lopez-Soler~J.\,M.} Latency evaluation of a virtualized MME~//  
 Wireless Days Proceedings.~--- Toulouse, France: IEEE, 2016. P.~1--3.
\bibitem{22-gre}
\Au{Granlund D., Holmlund~P., \mbox{{\ptb{\AA}}hlund~C.}} Opportunistic mobility 
support for resource constrained sensor devices in smart cities~// Sensors, 2015. 
Vol.~15. No.\,3. P.~5112--5135.
 \end{thebibliography}

 }
 }

\end{multicols}

\vspace*{-6pt}

\hfill{\small\textit{Поступила в~редакцию 31.05.17}}

\vspace*{8pt}

%\newpage

%\vspace*{-24pt}

\hrule

\vspace*{2pt}

\hrule

%\vspace*{8pt}


\def\tit{ANALYSIS OF~VERTICAL HANDOVER TIME\\ IN~HETEROGENEOUS WIRELESS NETWORKS}

\def\titkol{Analysis of~vertical handover time in~heterogeneous wireless networks}

\def\aut{A.\,Yu.~Grebeshkov$^1$, Yu.\,V.~Gaidamaka$^{2,3}$, O.\,G.~Vikhrova$^{2}$, 
and~E.\,R.~Zaripova$^2$}

\def\autkol{A.\,Yu.~Grebeshkov, Yu.\,V.~Gaidamaka, O.\,G.~Vikhrova, 
and~E.\,R.~Zaripova}

\titel{\tit}{\aut}{\autkol}{\titkol}

\vspace*{-9pt}


\noindent
$^1$Povolzhskiy State University of Telecommunications and Informatics, 23~Tolstoy Str., Samara 443010, Russian\linebreak 
$\hphantom{^1}$Federation

\noindent
$^2$Peoples' Friendship University of Russia (RUDN University), 6~Miklukho-Maklaya Str., 
Moscow 117198,\linebreak
$\hphantom{^1}$Russian Federation


\noindent
$^3$Institute of Informatics Problems, Federal Research Center ``Computer Science and Control'' 
of the Russian\linebreak
$\hphantom{^1}$Academy of Sciences, 44-2~Vavilov Str., Moscow 119333, 
Russian Federation



\def\leftfootline{\small{\textbf{\thepage}
\hfill INFORMATIKA I EE PRIMENENIYA~--- INFORMATICS AND
APPLICATIONS\ \ \ 2017\ \ \ volume~11\ \ \ issue\ 4}
}%
 \def\rightfootline{\small{INFORMATIKA I EE PRIMENENIYA~---
INFORMATICS AND APPLICATIONS\ \ \ 2017\ \ \ volume~11\ \ \ issue\ 4
\hfill \textbf{\thepage}}}

\vspace*{3pt}



\Abste{In a heterogeneous wireless network, connectivity is simultaneously available 
using different radio networks with overlapping coverage areas. A~mobile user 
equipment with a~multiple mode card that can work under various frequency bands 
and modulation schemes can switch from one technology to another in order to 
maintain communication. This procedure known as a vertical handover (VHO)
provides the 
benefit of utilizing the higher bandwidth and lower cost of wide local area networks 
as well as better mobility support and larger coverage of cellular networks. The 
authors investigate details of the VHO procedure from WLAN (Wireless Local
Area Network) to the 
3GPP Long Term Evolution (LTE). The VHO procedure includes~40~signaling 
messages, which are responsible for authorization and resource allocation in the 
target LTE network. The authors analyze the VHO sojourn time and its~95~percent 
quantile using a~multiphase queuing system with background traffic.}

\KWE{heterogeneous wireless network; cellular network; LTE; mobility; session 
setup procedure; connection reliability; connection availability; performance 
measure}




  \DOI{10.14357/19922264170409} 

%\vspace*{-12pt}

\Ack
\noindent
The publication was supported by the Ministry of Education and Science of the 
Russian Federation (project No.\,2.882.2017/4.6). 



\vspace*{12pt}

  \begin{multicols}{2}

\renewcommand{\bibname}{\protect\rmfamily References}
%\renewcommand{\bibname}{\large\protect\rm References}

{\small\frenchspacing
 {%\baselineskip=10.8pt
 \addcontentsline{toc}{section}{References}
 \begin{thebibliography}{99}
\bibitem{1-gre-1}
\Aue{Astely, D., E.~Dahlman, G.~Fodor, S.~Parkvall, and J.~Sachs.} 2013. LTE 
release~12 and beyond. \textit{IEEE Commun. Mag.} 51(7):154--160. 
doi: 10.1109/MCOM. 2013.6553692. 


\bibitem{3-gre-1} %2
\Aue{Gorbunova, A.\,V., I.\,S.~Zaryadov, S.\,I.~Matyushenko, K.\,E.~Samouylov, 
and S.\,Ya.~Shorgin.} 2015. Approksimatsiya vremeni otklika sistemy oblachnykh 
vychisleniy [The approximation of response time of a cloud computing system]. 
\textit{Informatika i~ee Primeneniya~--- Inform. Appl.} 9(3):32--38.

\bibitem{2-gre-1} %3
\Aue{Vikhrova, O.\,G., K.\,E.~Samouylov, E.\,S.~Sopin, and S.\,Ya.~Shorgin}. 2015. 
K~analizu pokazateley kachestva obsluzhivaniya v~sovremennykh besprovodnykh 
setyakh [On performance analysis of modern wireless networks]. \textit{Informatika 
i~ee Primeneniya~--- Inform. Appl.} 9(4):48--55.

\bibitem{4-gre-1}
\Aue{Gaidamaka, Yu.\,V., S.\,D.~Andreev, E.\,S.~Sopin, K.\,E.~Samouylov, and 
S.\,Ya.~Shorgin.} 2016. Analiz kharakteristik interferentsii v~modeli 
vzaimodeystviya ustroystv s~uchetom sredy rasprostraneniya signala [Interference 
analysis of the device-to-device communications model with regard to a~signal 
propagation environment]. \textit{Informatika i~ee Primeneniya~--- Inform. Appl.} 
10(4):2--10.
\bibitem{5-gre-1}
\Aue{Koucheryavy, A.\,E., М.\,А.~Makolkina, and R.\,V.~Kirichek.} 2016. 
Taktil'niy Internet. Seti svyazi so sverkhmalymi zaderzhkami [Tactile Internet. 
Ultra-low latency networks]. \textit{Elekrosvyaz'} 
[Telecomm. Radio Eng.] 1:44--46.
\bibitem{6-gre-1}
\Aue{Kellokoski, J., J.~Koskinen, R.~Nyrhinen, and 
T.~H$\ddot{\mbox{a}}$m$\ddot{\mbox{a}}$l$\ddot{\mbox{a}}$inen.} 2012. 
Efficient handovers for machine-to-machine communications between IEEE~802.11 
and 3GPP evolved packed core networks. \textit{IEEE  Conference (International) 
on Green Computing and Communications Proceedings}. \mbox{Besan{\!\ptb{\c{c}}}on.}  
722--725.
\bibitem{7-gre-1}
\Aue{Ahmed, L., L.~M.~Boulahia, and D.~Gaiti.} 2014. Enabling vertical 
handover decisions in heterogeneous wireless networks: A~state-of-the-art and 
a~classification. \textit{IEEE Commun. Surv. Tut.} 16(2):776--781.
\bibitem{8-gre-1}
\Aue{Bukhari, J., and N.~Akkari.} 2016. QoS based approach for LTE-WiFi 
handover. \textit{7th Conference (International) on Computer Science and 
Information Technology Proceedings}. Elsevier. 1--6.
\bibitem{9-gre-1}
\Aue{Gondim, P.\,R.\,L., and J.\,B.\,M.~Trineto.} 2012. DSMIP and PMIP for 
mobility management of heterogeneous access networks: Evaluation of 
authentication delay. \textit{IEEE Globecom Workshops Proceedings}. Anaheim. 
308--313.



\bibitem{11-gre-1} %10
\Aue{Do-Hyung, K., K.~Won-Tae, L.~Hwan-Gu, K.~Sun-Ja, and  
L.\,A.~Cheol-Hoon.} 2008. Performance evaluation of vertical hanover architecture 
with low latency handover. \textit{Conference (International) on Convergence and 
Hybrid Information Technology Proceedings}. Dusan. 66--69.


\columnbreak

\bibitem{10-gre-1} %11
\Aue{Tsagkaropoulos, M., I.~Politis, C.~Tselios, T.~Dagiuklas, and 
S.~Kotsopoulos}. 2011. Service continuity over intertechnology RATs. \textit{16th 
IEEE  Workshop (International) on Computer Aided Modeling and Design of 
Communication Links and Networks Proceedings}. Kyoto. 117--121.

\bibitem{12-gre-1}
\Aue{M$\acute{\mbox{a}}$rquez-Barja,~J., C.\,T.~Calafate, J.-C.~Cano, and 
P.~Manzoni.} 2011. An overview of vertical handover techniques: Algorithms, 
protocols and tools. \textit{Comput. Commun.} 34:985--997.
\bibitem{13-gre-1}
\Aue{Grebeshkov, A.\,Yu.} 2014. Otsenka tselesoobraznosti obrabotki zayavki dlya 
predostavleniya uslugi v~re\-kon\-fi\-gu\-ri\-ru\-emykh setyakh sleduyushchego pokoleniya 
[Estimation of processing feasibility for service provision application in 
reconfigurable Next Generation Networks]. \textit{\mbox{T-Comm}: Te\-le\-kom\-mu\-ni\-ka\-tsii 
i~transport} [\mbox{T-Comm}: Telecommunications and Transport ] 8:24--27. 
\bibitem{14-gre-1}
\Aue{Grebeshkov, A., E.~Zaripova, A.~Roslyakov, and K.~Samouylov}. Modelling 
of vertical handover from untrusted WLAN network to LTE. \textit{31st European 
Conference on Modelling and Simulation Proceedings}. 694--700.
\bibitem{15-gre-1}
\Aue{Lampropoulos, G., N.~Passas, L.~Mekaros, and A.~Kaloxylos.} 2005. 
Handover management architectures in integrated WLAN. \textit{IEEE 
Commun. Surv.  Tut.} 7(4):30--44.
\bibitem{16-gre-1}
3GPP TS 23.402. 2016. Technical specification 3rd Generation Partnership Project; 
Technical Specification Group Services and System Aspects; Architecture 
enhancements for non-3GPP accesses. Release~14.  
Available at: {\sf 
https://portal.3gpp.org/desktopmodules/\linebreak Specifications/SpecificationDetails.aspx?specificationId\linebreak =850} (accessed May~20, 2017).
\bibitem{17-gre-1}
3GPP TS 33.402. 2016. Technical specification 3rd Generation Partnership Project; 
Technical Specification Group Services and System Aspects; 3GPP System 
Architecture Evolution (SAE); Security aspects of non-3GPP accesses. Release~14. 
Available at: {\sf 
https://portal. 3gpp.org/desktopmodules/Specifications/Specification\linebreak Details.aspx?specificationId=2297}
 (accessed May~20, 2017).
\bibitem{18-gre-1}
\Aue{Gaidamaka, Yu., and E.~Zaripova} 2014. Session setup delay estimation 
methods for IMS-based IPTV services. 
\textit{Internet of things, smart spaces, and next generation networks
and systems}.
Eds.\ S.~Balandin, S.~Andreev, and Y.~Koucheryavy.
{Lecture notes in computer science ser.} Springer.
8638:408--418.
\bibitem{19-gre-1}
\Aue{Nikaein, N., and S.~Krco.} 2011. Latency for real-time machine-to-machine 
communication in LTE-based system architecture. \textit{17th European Wireless 
Conferense on Sustainable Wireless Technologies Proceedings.} Vienna, Austria: 
IEEE. 1--6.
\bibitem{20-gre-1}
\Aue{Cardona, N., J.\,F.~Monserrat, and J.~Cabrejas.} 2013. Enabling technologies 
for 3GPP LTE-advanced networks.\linebreak\vspace*{-11pt}

\pagebreak

\noindent
 \textit{LTE-Advanced and next generation 
wireless networks}. Eds.\
 G.~de la~Roche, A.\,A.~Glazunov, and B.~Allen.
Chichester, U.K.: John Wiley and Sons. 3--34.
\bibitem{21-gre-1}
\Aue{Prados-Garzon, J., J.\,J.~Ramos-Munoz, P.~Ameigeiras,  
P.~Andres-Maldonado, and J.\,M.~Lopez-Soler.} 2016. La-\linebreak\vspace*{-11pt}

\columnbreak

\noindent
tency evaluation of 
a~virtualized MME. \textit{Wireless Days Proceedings}. Toulouse, 
France: IEEE. 1--3.
\bibitem{22-gre-1}
\Aue{Granlund, D., P.~Holmlund, and \mbox{C.~{\ptb{\AA}}hlund}}. 2015. 
Opportunistic mobility support for resource constrained sensor devices in smart 
cities. \textit{Sensors} 15(3):5112--5135.
{\looseness=1

}
\end{thebibliography}

 }
 }

\end{multicols}

\vspace*{-6pt}

\hfill{\small\textit{Received May 31, 2017}}

%\vspace*{-10pt}

\Contr

\noindent
\textbf{Grebeshkov Alexander Yu.} (b.\ 1967)~--- Candidate of Science (PhD) in 
technology, senior scientist, Povolzhskiy State University of Telecommunications 
and Informatics, 23~Tolstoy Str., Samara 443010, Russian Federation;  
\mbox{grebeshkov-ay@psuti.ru}

\vspace*{3pt}

\noindent
\textbf{Gaidamaka Yuliya V.} (b.\ 1971)~--- Candidate of Science (PhD) in physics and 
mathematics, associate professor, Peoples' Friendship University of Russia (RUDN University), 
6~Miklukho-Maklaya Str., Moscow 117198, Russian Federation; senior scientist, Institute of 
Informatics Problems, Federal Research Center ``Computer Science and Control'' of the Russian 
Academy of Sciences, 44-2~Vavilov Str., Moscow 119333, Russian Federation; 
\mbox{gaydamaka\_yuv@rudn.university}

\vspace*{3pt}

\noindent
\textbf{Vikhrova Olga G.} (b.\ 1990)~--- PhD student, Peoples' Friendship University of Russia 
(RUDN University), 6~Miklukho-Maklaya Str., Moscow 117198, Russian Federation; 
\mbox{vikhrova\_og@rudn.university}

\vspace*{3pt}

\noindent
\textbf{Zaripova Elvira R.} (р.\ 1979)~--- Candidate of Science (PhD) in physics and 
mathematics, associate professor, Peoples' Friendship University of Russia (RUDN University), 
6~Miklukho-Maklaya Str., Moscow 117198, Russian Federation; 
\mbox{zaripova\_er@rudn.university}

\label{end\stat}


\renewcommand{\bibname}{\protect\rm Литература} 