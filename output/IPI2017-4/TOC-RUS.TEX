
%Том 11 Выпуск 1-4 Год 2017

\def\stat{cont}
{%\hrule\par
%\vskip 7pt % 7pt
\raggedleft\Large \bf%\baselineskip=3.2ex
А\,В\,Т\,О\,Р\,С\,К\,И\,Й\ \ У\,К\,А\,З\,А\,Т\,Е\,Л\,Ь\ \ З\,А\ \ 2\,0\,1\,7 г. \vskip 17pt
 \hrule
 \par
\vskip 21pt plus 6pt minus 3pt }

\label{st\stat}

\def\tit{\ }

\def\aut{\ }
\def\auf{\ }

\def\leftkol{\ } % ENGLISH ABSTRACTS}

\def\rightkol{\ } %АВТОРСКИЙ УКАЗАТЕЛЬ ЗА 2017 г.} %ENGLISH ABSTRACTS}

\titele{\tit}{\aut}{\auf}{\leftkol}{\rightkol}
\addcontentsline{toc}{subsection}{\textrm\textbf Авторский указатель за 2017 г.}

\vspace*{-12pt}
\vspace*{-36pt}

\noindent
{\tabcolsep=3pt
\begin{tabular}{p{397pt}cc}
&\textbf{Вып.} & \textbf{Стр.}\\[6pt]
\Avtors{Агаларов~Я.\,М.} Максимизация среднего стационарного дохода системы массового\linebreak
\\[-12pt]
\hspace*{23pt}об\-слу\-жи\-вания типа $M/G/1$&2&25--32\\
\Avtors{Агаларов~Я.\,М., Шоргин~В.\,С.} Об одной задаче максимизации дохода системы массового\linebreak
\\[-12pt]
\hspace*{23pt}обслуживания типа $G/M/1$ с~пороговым управлением очередью&4&55--64\\
\Avtors{Акимов~Д.\,А.} см.~Сигов~А.\,С.&&\\
\Avtors{Алексейчук А.\,С., Пантелеев~А.\,В.} Индивидуализация процесса обучения в режиме\linebreak
\\[-12pt]
\hspace*{23pt}веб-конференции на основе иерархической нечеткой экспертной системы&1&90--99\\
\Avtors{Андрианова~Е.\,Г.} см.~Сигов~А.\,С.&&\\
\Avtors{Атаева~О.\,М., Серебряков~В.\,А.} Персональная открытая семантическая цифровая биб-\linebreak
\\[-12pt]
\hspace*{23pt}лио\-те\-ка LibMeta. Конструирование контента. Интеграция с источниками LOD&2&\hphantom{1}85--100\\
\Avtors{Басок Б.\,М.} см.~Френкель С.\,Л.&&\\
\Avtors{Битюков~Ю.\,И., Платонов~Е.\,Н.} Применение вейвлетов для расчета линейных систем\linebreak
\\[-12pt]
\hspace*{23pt}управления с~сосредоточенными параметрами&4&\hphantom{1}94--103\\
\Avtors{Борисов~А.\,В.} Классификация по непрерывным наблюдениям с мультипликативными\linebreak
\\[-12pt]
\hspace*{23pt}шумами~I: формулы байесовской оценки&1&11--19\\
\Avtors{Борисов~А.\,В.} Классификация по непрерывным наблюдениям с мультипликативными\linebreak
\\[-12pt]
\hspace*{23pt}шумами~II: алгоритм численной реализации оценки&2&33--41\\
\Avtors{Буянов~М.\,В., Иванов~С.\,В., Кибзун~А.\,И., Наумов~А.\,В.} Развитие математической модели управления грузоперевозками на~участке железнодорожной сети с~учетом\linebreak
\\[-12pt]
\hspace*{23pt}случайных факторов&4&85--93\\
\Avtors{Васильев~Н.\,С.} Информированность участников и существование равновесия в позици-\linebreak
\\[-12pt]
\hspace*{23pt}онных многошаговых играх многих лиц&2&42--49\\
\Avtors{Вихрова~О.\,Г.} см.~Гребешков~А.\,Ю.&&\\
\Avtors{Гайдамака~Ю.\,В., Орлов~Ю.\,Н., Молчанов~Д.\,А., Самуйлов~А.\,К.} Моделирование отношения сигнал/интерференция в мобильной сети со случайным блужданием\linebreak
\\[-12pt]
\hspace*{23pt}взаимодействующих устройств&2&50--58\\
\Avtors{Гайдамака~Ю.\,В., Самуйлов~К.\,Е., Шоргин~С.\,Я.} Метод моделирования характеристик интерференции при~прямом взаимодействии перемещающихся устройств в~гетеро-\linebreak
\\[-12pt]
\hspace*{23pt}генной беспроводной сети пятого поколения&4&2--9\\
\Avtors{Гайдамака~Ю.\,В.} см.~Гребешков~А.\,Ю.&&\\
\Avtors{Ганебных С.\,Н.} см.~Ланге М.\,М.&&\\
\Avtors{Горбунов~К.\,Ю., Любецкий В.\,А.} Алгоритм преобразования одного графа в другой\linebreak
\\[-12pt]
\hspace*{23pt}с~ми\-ни\-мальной ценой&1&79--89\\
\Avtors{Горшенин~А.\,К.} Анализ вероятностно-статистических характеристик осадков на~основе\linebreak
\\[-12pt]
\hspace*{23pt}паттернов&4&38--46\\
\Avtors{Горшенин~А.\,К.} О некоторых математических и программных методах построения\linebreak
\\[-12pt]
\hspace*{23pt}структурных моделей информационных потоков&1&58--68\\
\Avtors{Гребешков~А.\,Ю., Гайдамака~Ю.\,В., Вихрова~О.\,Г., Зарипова~Э.\,Р.} Анализ времени переключения сеанса связи в~гетерогенных беспроводных сетях при~вертикальном\linebreak
\\[-12pt]
\hspace*{23pt}хэндовере&4&70--78\\
\Avtors{Грушо~А.\,А., Забежайло~М.\,И., Смирнов~Д.\,В., Тимонина~Е.\,Е.} Модель множества инфор-\linebreak
\\[-12pt]
\hspace*{23pt}мационных пространств в~задаче поиска инсайдера&4&65--69\\
\Avtors{Гудкова~И.\,А., Шоргин~С.\,Я.} Вероятностная модель совместного использования ресурсов\linebreak
\\[-12pt]
\hspace*{23pt}беспроводной сети с адаптивным управлением мощностью&3&90--98\\
\Avtors{Докукин~А.\,A., Рязанов~В.\,В., Шут О.\,В.} Многоуровневые модели решения многоклассо-\linebreak
\\[-12pt]
\hspace*{23pt}вых задач распознавания&1&69--78\\
\Avtors{Драницына~М.\,А., Захарова~Т.\,В.} Сегментирование нестационарных сигналов на основе\linebreak
\\[-12pt]
\hspace*{23pt}вероятностных свойств оконной дисперсии&3&18--26\\
\end{tabular}
}

\pagebreak

\def\leftkol{АВТОРСКИЙ УКАЗАТЕЛЬ ЗА 2017 г.} % ENGLISH ABSTRACTS}

\def\rightkol{АВТОРСКИЙ УКАЗАТЕЛЬ ЗА 2017 г.} %ENGLISH ABSTRACTS}

%\thispagestyle{myheadings}
\def\leftfootline{\small{\textbf{\thepage}
\hfill ИНФОРМАТИКА И ЕЁ ПРИМЕНЕНИЯ\ \ \ том~11\ \ \ выпуск~4\ \ \ 2017}
}%
 \def\rightfootline{\small{ИНФОРМАТИКА И ЕЁ ПРИМЕНЕНИЯ\ \ \ том~11\ \ \ выпуск~4\ \ \ 2017
 \hfill \textbf{\thepage}}}


\noindent
{\tabcolsep=3pt
\begin{tabular}{p{394pt}cc}
&\textbf{Вып.} & \textbf{Стр.}\\[3pt]
\Avtors{Дюкова~Е.\,В., Никифоров~А.\,Г., Прокофьев~П.\,А.} О~распараллеливании асимптотически\linebreak
\\[-12pt]
\hspace*{23pt}оптимальных алгоритмов дуализации&3&113--122\\
\Avtors{Жуков~Д.\,О.} см.~Сигов~А.\,С.&&\\
\Avtors{Забежайло~М.\,И.} см.~Грушо~А.\,А.&&\\
\Avtors{Зализняк Анна~А., Зацман~И.\,М., Инькова~О.\,Ю.} Надкорпусная база данных коннекторов:\linebreak
\\[-12pt]
\hspace*{23pt}построение системы терминов&1&100--108\\
\Avtors{Зарипова~Э.\,Р.} см.~Гребешков~А.\,Ю.&&\\
\Avtors{Захаров~В.\,Н.} см.~Френкель С.\,Л.&&\\
\Avtors{Захарова~Т.\,В.} см.~Драницына~М.\,А.&&\\
\Avtors{Зацман~И.\,М., Лукьянов~Г.\,В., Минин~В.\,А., Хавансков~В.\,А., Шубников~С.\,К.} Индикаторное оценивание процессов переноса знаний из~области научных исследований\linebreak
\\[-12pt]
\hspace*{23pt}в~сферу технологического развития&3&132--141\\
\Avtors{Зацман~И.\,М.} см.~Зализняк Анна~А.&&\\
\Avtors{Иванов~С.\,В.} см.~Буянов~М.\,В.&&\\
\Avtors{Инькова~О.\,Ю., Попкова~Н.\,А.} Статистические данные как информационная основа\linebreak
\\[-12pt]
\hspace*{23pt}лингвистического анализа коннекторов русского языка&3&123--131\\
\Avtors{Инькова~О.\,Ю.} см.~Зализняк Анна~А.&&\\
\Avtors{Кабанов~Ю.\,М.} см.~Эль Битар~Х.&&\\
\Avtors{Кабанов~Ю.\,М.} см.~Эль~Битар~Х.&&\\
\Avtors{Кибзун~А.\,И.} см.~Буянов~М.\,В.&&\\
\Avtors{Кириков~И.\,А., Колесников~А.\,В., Листопад С.\,В.} Компьютерная модель синергии кол-\linebreak
\\[-12pt]
\hspace*{23pt}лективного принятия решений&3&34--41\\
\Avtors{Ковалёв~С.\,П.} Методы теории категорий в модельно-ориентированной системной\linebreak
\\[-12pt]
\hspace*{23pt}ин\-же\-не\-рии&3&42--50\\
\Avtors{Колесников~А.\,В.} см.~Кириков~И.\,А.&&\\
\Avtors{Корепанов~Э.\,Р.} см.~Синицын~И.\,Н.&&\\
\Avtors{Королев~В.\,Ю.} Аналоги теоремы Глезера для отрицательных биномиальных и обобщен-\linebreak
\\[-12pt]
\hspace*{23pt}ных гамма-распределений и некоторые их приложения&3&\hphantom{1}2--17\\
\Avtors{Королев~В.\,Ю.} Некоторые свойства распределения Миттаг-Леффлера и связанных с~ним\linebreak
\\[-12pt]
\hspace*{23pt}процессов&4&26--37\\
\Avtors{Кривенко М.\,П.} Многомерный референсный регион высокой плотности&2&59--64\\
\Avtors{Кривенко М.\,П.} Обучаемая классификация неполных клинических данных&3&27--33\\
\Avtors{Кружков~М.\,Г.} Подходы к аннотации дискурсивных отношений в~лингвистических\linebreak
\\[-12pt]
\hspace*{23pt}корпусах&4&118--125\\
\Avtors{Кудрявцев~А.\,А., Титова~А.\,И.} Гамма-экспоненциальная функция в байесовских моделях\linebreak
\\[-12pt]
\hspace*{23pt}массового обслуживания&4&104--108\\
\Avtors{Кузнецов~М.\,П.} см.~Сафин~К.\,Ф.&&\\
\Avtors{Кузнецова~М.\,В.} см.~Сафин~К.\,Ф.&&\\
\Avtors{Ланге А.\,М.} см.~Ланге М.\,М.&&\\
\Avtors{Ланге М.\,М., Ганебных С.\,Н., Ланге А.\,М.} Об эффективности иерархического алгоритма\linebreak
\\[-12pt]
\hspace*{23pt}поиска приближенного ближайшего соседа в заданном наборе изображений&3&51--59\\
\Avtors{Лисовская~Е.\,Ю., Моисеева~С.\,П., Пагано~М., Потатуева~В.\,В.} Исследование системы\linebreak
\\[-12pt]
\hspace*{23pt}массового обслуживания MMPP/GI/$\infty $ с~требованиями случайного объема&4&109--117\\
\Avtors{Листопад С.\,В.} см.~Кириков~И.\,А.&&\\
\Avtors{Лукашенко~О.\,В., Морозов~Е.\,В., Пагано~М.} Об эффективности оценки Монте Карло\linebreak
\\[-12pt]
\hspace*{23pt}на основе гауссовского моста&2&16--24\\
\Avtors{Лукьянов~Г.\,В.} см.~Зацман~И.\,М.&&\\
\Avtors{Любецкий В.\,А.} см.~Горбунов~К.\,Ю.&&\\
\Avtors{Малашенко~Ю.\,Е., Назарова~И.\,А., Новикова~Н.\,М.} Метод анализа функциональной\linebreak
\\[-12pt]
\hspace*{23pt}уязвимости потоковых сетевых систем&4&47--54\\
\Avtors{Матюшенко~С.\,И.} см.~Мейханаджян~Л.\,А.&&\\
\Avtors{Мейханаджян~Л.\,А., Матюшенко~С.\,И., Пяткина~Д.\,А., Разумчик~Р.\,В.} Совместное стационарное распределение числа заявок в системе с двумя очередями конечной\linebreak
\\[-12pt]
\hspace*{23pt}емкости и~общим входящим потоком&3&106--112\\
\end{tabular}
}

\pagebreak

\def\leftkol{АВТОРСКИЙ УКАЗАТЕЛЬ ЗА 2017 г.} % ENGLISH ABSTRACTS}

\def\rightkol{АВТОРСКИЙ УКАЗАТЕЛЬ ЗА 2017 г.} %ENGLISH ABSTRACTS}

%\thispagestyle{myheadings}
\def\leftfootline{\small{\textbf{\thepage}
\hfill ИНФОРМАТИКА И ЕЁ ПРИМЕНЕНИЯ\ \ \ том~11\ \ \ выпуск~4\ \ \ 2017}
}%
 \def\rightfootline{\small{ИНФОРМАТИКА И ЕЁ ПРИМЕНЕНИЯ\ \ \ том~11\ \ \ выпуск~4\ \ \ 2017
 \hfill \textbf{\thepage}}}


\noindent
{\tabcolsep=3pt
\begin{tabular}{p{397pt}cc}
&\textbf{Вып.} & \textbf{Стр.}\\[3pt]
\Avtors{Минин~В.\,А.} см.~Зацман~И.\,М.&&\\
\Avtors{Моисеева~С.\,П.} см.~Лисовская~Е.\,Ю.&&\\
\Avtors{Мокбель~Р.} см.~Эль Битар~Х.&&\\
\Avtors{Мокбель~Р.} см.~Эль~Битар~Х.&&\\
\Avtors{Мокров~Е.\,В.} см.~Наумов~В.\,А.&&\\
\Avtors{Молибог И.\,О., Мотренко А.\,П., Стрижов~В.\,В.} Повышение качества классификации\linebreak
\\[-12pt]
\hspace*{23pt}в~задаче обнаружения внутреннего плагиата&3&60--72\\
\Avtors{Молчанов~Д.\,А.} см.~Гайдамака~Ю.\,В.&&\\
\Avtors{Морозов~Е.\,В.} см.~Лукашенко~О.\,В.&&\\
\Avtors{Мотренко А.\,П.} см.~Молибог И.\,О.&&\\
\Avtors{Назарова~И.\,А.} см.~Малашенко~Ю.\,Е.&&\\
\Avtors{Наумов~А.\,В.} см.~Буянов~М.\,В.&&\\
\Avtors{Наумов~В.\,А., Мокров~Е.\,В., Самуйлов~К.\,Е.} Анализ временных характеристик процесса\linebreak
\\[-12pt]
\hspace*{23pt}передачи данных подвижным пользователям в~соте сети~LTE&4&79--84\\
\Avtors{Никифоров~А.\,Г.} см.~Дюкова~Е.\,В.&&\\
\Avtors{Новикова~Н.\,М.} см.~Малашенко~Ю.\,Е.&&\\
\Avtors{Орлов~Ю.\,Н.} см.~Гайдамака~Ю.\,В.&&\\
\Avtors{Пагано~М.} см.~Лисовская~Е.\,Ю.&&\\
\Avtors{Пагано~М.} см.~Лукашенко~О.\,В.&&\\
\Avtors{Пантелеев~А.\,В.} см.~Алексейчук А.\,С.&&\\
\Avtors{Пархоменко В.\,П.} Применение квазислучайного подхода и ансамблевых вычислений для\linebreak
\\[-12pt]
\hspace*{23pt}определения оптимальных наборов значений параметров климатической модели&2&65--73\\
\Avtors{Платонов~Е.\,Н.} см.~Битюков~Ю.\,И.&&\\
\Avtors{Попкова~Н.\,А.} см.~Инькова~О.\,Ю.&&\\
\Avtors{Потатуева~В.\,В.} см.~Лисовская~Е.\,Ю.&&\\
\Avtors{Прокофьев~П.\,А.} см.~Дюкова~Е.\,В.&&\\
\Avtors{Пяткина~Д.\,А.} см.~Мейханаджян~Л.\,А.&&\\
\Avtors{Раев~В.\,К.} см.~Сигов~А.\,С.&&\\
\Avtors{Разумчик~Р.\,В.} Стационарные распределения, связанные со~временем пребывания\linebreak
\\[-12pt]
\hspace*{23pt}в~состоянии перегрузки системы MAP/PH/1/$r$ с~гистерезисным управлением на-\linebreak
\\[-12pt]
\hspace*{23pt}грузкой&4&19--25\\
\Avtors{Разумчик~Р.\,В.} Стационарные характеристики системы обслуживания с~инверсионным порядком обслуживания, вероятностным приоритетом и~групповым поступлением\linebreak
\\[-12pt]
\hspace*{23pt}разнородных заявок&4&10--18\\
\Avtors{Разумчик~Р.\,В.} см.~Мейханаджян~Л.\,А.&&\\
\Avtors{Рудой Г.\,И.} Модификация функционала качества в задачах нелинейной регрессии\linebreak
\\[-12pt]
\hspace*{23pt}для учета гетероскедастичных погрешностей измеряемых данных&2&74--84\\
\Avtors{Рязанов~В.\,В.} см.~Докукин~А.\,A.&&\\
\Avtors{Самуйлов~А.\,К.} см.~Гайдамака~Ю.\,В.&&\\
\Avtors{Самуйлов~К.\,Е., Сопин~Э.\,С., Шоргин~С.\,Я.} Система массового обслуживания с ограниченными ресурсами и сигналами для анализа показателей эффективности беспро-\linebreak
\\[-12pt]
\hspace*{23pt}водных сетей&3&\hphantom{1}99--105\\
\Avtors{Самуйлов~К.\,Е.} см.~Гайдамака~Ю.\,В.&&\\
\Avtors{Самуйлов~К.\,Е.} см.~Наумов~В.\,А.&&\\
\Avtors{Сафин~К.\,Ф., Кузнецов~М.\,П., Кузнецова~М.\,В.} Определение заимствований в тексте\linebreak
\\[-12pt]
\hspace*{23pt}без указания источника&3&73--79\\
\Avtors{Сачков~В.\,Е.} см.~Сигов~А.\,С.&&\\
\Avtors{Сейфуль-Мулюков~Р.\,Б.} Информатика и~ее роль в~познании образования и~свойств\linebreak
\\[-12pt]
\hspace*{23pt}сложной природной системы&1&119--123\\
\Avtors{Серебряков~В.\,А.} см.~Атаева~О.\,М.&&\\
\Avtors{Сигов~А.\,С., Акимов~Д.\,А., Жуков~Д.\,О., Андрианова~Е.\,Г., Сачков~В.\,Е., Раев~В.\,К.} Психолингвистический анализ русскоязычных текстовых сообщений на основе их\linebreak
\\[-12pt]
\hspace*{23pt}фоносемантических статистических характеристик&3&80--89\\
\Avtors{Синицын~В.\,И.} см.~Синицын~И.\,Н.&&\\
\end{tabular}
}

\pagebreak

\def\leftkol{АВТОРСКИЙ УКАЗАТЕЛЬ ЗА 2017 г.} % ENGLISH ABSTRACTS}

\def\rightkol{АВТОРСКИЙ УКАЗАТЕЛЬ ЗА 2017 г.} %ENGLISH ABSTRACTS}

%\thispagestyle{myheadings}
\def\leftfootline{\small{\textbf{\thepage}
\hfill ИНФОРМАТИКА И ЕЁ ПРИМЕНЕНИЯ\ \ \ том~11\ \ \ выпуск~4\ \ \ 2017}
}%
 \def\rightfootline{\small{ИНФОРМАТИКА И ЕЁ ПРИМЕНЕНИЯ\ \ \ том~11\ \ \ выпуск~4\ \ \ 2017
 \hfill \textbf{\thepage}}}


\noindent
{\tabcolsep=3pt
\begin{tabular}{p{397pt}cc}
&\textbf{Вып.} & \textbf{Стр.}\\[6pt]
\Avtors{Синицын~И.\,Н.} Аналитическое моделирование широкополосных процессов в стохасти-\linebreak
\\[-12pt]
\hspace*{23pt}ческих системах, не разрешенных относительно производных&1&\hphantom{1}3--10\\
\Avtors{Синицын~И.\,Н., Синицын~В.\,И., Корепанов~Э.\,Р.} Модифицированные эллипсоидальные условно-оптимальные фильтры для нелинейных стохастических систем на\linebreak
\\[-12pt]
\hspace*{23pt}многообразиях&2&101--111\\
\Avtors{Смирнов~Д.\,В.} см.~Грушо~А.\,А.&&\\
\Avtors{Соколов~И.\,А.} Предисловие&1&2--2\\
\Avtors{Сопин~Э.\,С.} см.~Самуйлов~К.\,Е.&&\\
\Avtors{Стефанович А.\,И., Сушко Д.\,В.} Обратимое сжатие данных посредством универсального\linebreak
\\[-12pt]
\hspace*{23pt}арифметического кодирования&1&20--45\\
\Avtors{Стрижов~В.\,В.} см.~Молибог И.\,О.&&\\
\Avtors{Сушко Д.\,В.} см.~Стефанович А.\,И.&&\\
\Avtors{Тимонина~Е.\,Е.} см.~Грушо~А.\,А.&&\\
\Avtors{Титова~А.\,И.} см.~Кудрявцев~А.\,А.&&\\
\Avtors{Ушаков~В.\,Г., Ушаков~Н.\,Г.} Одноканальная система обслуживания с зависимыми интер-\linebreak
\\[-12pt]
\hspace*{23pt}валами времени между поступлениями требований&2&112--116\\
\Avtors{Ушаков~Н.\,Г.} см.~Ушаков~В.\,Г.&&\\
\Avtors{Френкель С.\,Л., Захаров~В.\,Н., Басок Б.\,М.} Вероятностные модели оценки устойчивости\linebreak
\\[-12pt]
\hspace*{23pt}программ к кратковременным аппаратным сбоям&1&46--57\\
\Avtors{Хавансков~В.\,А.} см.~Зацман~И.\,М.&&\\
\Avtors{Шестаков~О.\,В.} Сильная состоятельность оценки среднеквадратичной погрешности\linebreak
\\[-12pt]
\hspace*{23pt}при решении обратных статистических задач&2&117--121\\
\Avtors{Шестаков~О.\,В.} Универсальная пороговая обработка в моделях с негауссовым шумом&2&122--125\\
\Avtors{Шоргин~В.\,С.} см.~Агаларов~Я.\,М.&&\\
\Avtors{Шоргин~С.\,Я.} см.~Гайдамака~Ю.\,В.&&\\
\Avtors{Шоргин~С.\,Я.} см.~Гудкова~И.\,А.&&\\
\Avtors{Шоргин~С.\,Я.} см.~Самуйлов~К.\,Е.&&\\
\Avtors{Шубников~С.\,К.} см.~Зацман~И.\,М.&&\\
\Avtors{Шут О.\,В.} см.~Докукин~А.\,A.&&\\
\Avtors{Эль Битар~Х., Кабанов~Ю.\,М., Мокбель~Р.} Динамические модели системнoго риска\linebreak
\\[-12pt]
\hspace*{23pt}и~заражения&2&\hphantom{1}2--15\\
\Avtors{Эль~Битар~Х., Кабанов~Ю.\,М., Мокбель~Р.} О~единственности клиринговых векторов,\linebreak
\\[-12pt]
\hspace*{23pt}редуцирующих системный риск&1&109--118\\
\end{tabular}
}

%\thispagestyle{myheadings}
\def\leftfootline{\small{\textbf{\thepage}
\hfill ИНФОРМАТИКА И ЕЁ ПРИМЕНЕНИЯ\ \ \ том~11\ \ \ выпуск~4\ \ \ 2017}
}%
 \def\rightfootline{\small{ИНФОРМАТИКА И ЕЁ ПРИМЕНЕНИЯ\ \ \ том~11\ \ \ выпуск~4\ \ \ 2017
 \hfill \textbf{\thepage}}}

 \label{end\stat}