\newcommand{\G}{{\sf Ge}}

\def\stat{kudr-titova}

\def\tit{ГАММА-ЭКСПОНЕНЦИАЛЬНАЯ ФУНКЦИЯ В БАЙЕСОВСКИХ МОДЕЛЯХ МАССОВОГО ОБСЛУЖИВАНИЯ$^*$}

\def\titkol{Гамма-экспоненциальная функция в~байесовских моделях массового обслуживания}

\def\aut{А.\,А.~Кудрявцев$^1$, А.\,И.~Титова$^2$}

\def\autkol{А.\,А.~Кудрявцев, А.\,И.~Титова}

\titel{\tit}{\aut}{\autkol}{\titkol}

\index{Кудрявцев А.\,А.}
\index{Титова А.\,И.}
\index{Kudryavtsev A.\,A.}
\index{Titova A.\,I.}



{\renewcommand{\thefootnote}{\fnsymbol{footnote}} \footnotetext[1]
{Работа выполнена при частичной финансовой поддержке РФФИ (проект 17-07-00577).}}


\renewcommand{\thefootnote}{\arabic{footnote}}
\footnotetext[1]{Московский государственный университет им.\ М.\,В.~Ломоносова, 
факультет вычислительной математики и~кибернетики, \mbox{nubigena@mail.ru}}
\footnotetext[2]{Московский государственный университет им.\ М.\,В.~Ломоносова, 
факультет вычислительной математики и~кибернетики, \mbox{onkelskroot@gmail.com}}

%\vspace*{-18pt}

\vspace*{-9pt}

\Abst{Рассматривается байесовский подход к~построению мо\-де\-лей 
теории массового обслуживания и~надежности. Байесовский подход является 
целесообразным при изучении систем, характеристики которых меняются в~моменты 
времени, неизвестные исследователю, или же при изучении больших совокупностей 
однотипных систем. В~рамках этого подхода для классических постановок задач 
предполагается, что основные параметры системы не являются заданными, 
но при этом известны их априорные распределения. За счет 
рандомизации параметров системы различные ее характеристики, 
например коэффициент загрузки, также становятся случайными. 
В~работе вводится понятие гам\-ма-экс\-по\-нен\-ци\-аль\-ной 
функции, приводятся ее свойства, а~также конкретные результаты для 
вероятностных характеристик коэффициента загрузки и~вероятности <<непотери>> 
вызова в~случае, когда в~качестве пары априорных распределений параметров 
системы $M/M/1/0$ рассматриваются экспоненциальное распределение и~распределение 
Вейбулла.}


\KW{байесовский подход; системы массового обслуживания; надежность; смешанные
распределения; распределение Вейбулла; экспоненциальное распределение; 
гам\-ма-экс\-по\-нен\-ци\-аль\-ная функция}

\DOI{10.14357/19922264170413} 

\vspace*{-9pt}


\vskip 10pt plus 9pt minus 6pt

\thispagestyle{headings}

\begin{multicols}{2}

\label{st\stat}

\section{Введение}

Зачастую при описании математических мо\-де\-лей функционирования различных объектов 
их жизненный цикл зависит от параметров, <<способствующих>> и~<<препятствующих>> 
функционированию. В~моделях структур и~систем массового\linebreak обслуживания к~параметрам, 
<<способствующим>> функционированию, можно отнести интенсивность обслуживания 
запросов, а~к~параметрам, <<препятствующим>> функционированию,~--- 
интенсивность входящего потока требований. При этом нетрудно заметить, 
что для исследования результатов работы системы важны не столько значения 
параметров, сколько их соотношение.

Далее будет рассмотрена система массового обслужи\-вания $M/M/1/0$, одним из 
основных показателей которой является ее коэффициент загрузки~$\rho$. 
Значение коэффициента загрузки определяется как отношение параметра входящего 
потока~$\lambda$ к~па\-ра\-мет\-ру обслуживания~$\mu$. От величины~$\rho$ 
зависят многие характеристики разнообразных систем массового обслуживания, в~том 
числе вероятность <<непотери>> вызова $\pi \hm= {\mu/(\lambda \hm+ \mu)}
\hm = 1/(1\hm+\rho)$.

В рамках байесовского подхода к~постановкам классических задач массового 
обслуживания и~надежности предполагается, что конкретные значения параметров~$\lambda$ 
и~$\mu$ неизвестны, однако имеется информация об их априорных распределениях~\cite{KuSh2015}.

Зачастую в~байесовских постановках задач массового обслуживания результаты 
описываются в~терминах специальных функций, например в~терминах бе\-та-функ\-ции 
и~интегральной показательной функции при рассмотрении общего эрланговского 
случая~\cite{KuSh09b}. При рассмотрении общего 
бе\-та-рас\-пре\-де\-ле\-ния~\cite{ZhaKuSh} или же бе\-та-рав\-но\-мер\-но\-го~\cite{ZhaKuSh2} 
распределения параметров в~байесовской модели рекуррентного роста надежности 
результаты выражаются через обобщенную гипергеометрическую функцию.

В ходе исследований, касающихся вероятностных характеристик коэффициента 
загрузки~$\rho$  и~вероятности <<непотери>> вызова~$\pi$ в~случае, когда в~качестве 
пары априорных распределений параметров системы~$\lambda$ и~$\mu$ 
рассматриваются экспоненциальное распределение и~распределение Вейбулла, 
были получены результаты, не выражающиеся в~терминах элементарных функций. 
В~связи с~этим предлагается рассмотреть новую специальную функцию, 
упоминаний об аналитическом виде и~свойствах которой не было 
обнаружено в~классических книгах, посвященных специальным функциям 
(см., например,~\cite{Artin, BeEr,AbSt}).

\section{Основные результаты}


Введем следующие обозначения. Через~$M(\theta)$ обозначим экспоненциальное 
распределение с~параметром $\theta\hm>0$, а~через $W(p,\alpha)$~--- 
распределение Вейбулла с~плот\-ностью $w_{p,\alpha}(x)$, имеющей вид:
$$
w_{p,\alpha}(x) = \fr{px^{p-1}e^{-({x/\alpha})^p}}{\alpha^{p}} \,, \enskip
 x>0\,,\  p>0\,,\ \alpha>0\,.
 $$
Назовем функцию вида
$$
\G_{\alpha,\, \beta} (x) = \sum\limits_{k=0}^{\infty}
\fr{x^k}{k!}\, \Gamma(\alpha k + \beta)\,, \enskip
 x\in\mathbb{R}\,, \ \alpha\ge0\,, \  \beta> 0\,,
 $$
\textit{гамма-экспо\-нен\-ци\-аль\-ной функцией}.


\smallskip

\noindent
\textbf{Теорема~1.}\
\textit{Гамма-экспо\-нен\-ци\-аль\-ная функция обладает следующими свойствами}:
\begin{enumerate}
\item $\G_{\alpha, \beta} (x)$ 
\textit{сходится абсолютно на всей прямой при $0\hm\le\alpha\hm<1$, $\beta\hm>0$ и~на интервале $(-1,1)$ при $\alpha\hm=1$, $\beta\hm>0$; 
сходится условно в~точке $x = -1$ при 
$\alpha=1$, $0\hm<\beta\hm<1$; сходится только 
в~точке  $x\hm = 0$ при} $\alpha\hm>1$, $\beta\hm>0$.
\item $\G_{\alpha, \beta}(x)$ \textit{непрерывна в~области сходимости}.
\item $\G_{1,\, n+1} (x) \hm= \left({x^n}/{(1-x)}\right)^{(n)}_x$, $|x|\hm<1$. 
\textit{В~частности, $\G_{1,\, 1} (x)\hm= {1/{(1\hm-x)}}$ и}
$\G_{1,\, 2} (x)\hm= {1/{(1\hm-x)^2}}$,  $|x|\hm<1$.
\item $\G^{(n)}_{\alpha, \beta} (x) = \G_{\alpha, \alpha n + \beta}(x)$ 
\textit{в области сходимости}.
\item $\G_{\alpha, \beta} (0)\hm=\Gamma(\beta),$ $\alpha\hm\ge0,$ $\beta\hm> 0$.
\item $\G_{0, \beta} (x)= \Gamma(\beta)e^{x},$ $x\hm\in\mathbb{R},$ $\beta\hm> 0$.
\item $\G_{\alpha, 1}(x)=1\hm+\alpha x\G_{\alpha, \alpha}(x),$ $x\hm\in
\mathbb{R},$ $\alpha\hm>0$.
\item $\G_{\alpha,\, \beta}(x)=\Gamma(\beta-1)+\alpha x \G_{\alpha,\alpha+\beta-1}(x)
+(\beta\hm-1)\G_{\alpha, \beta-1}(x),$ $x\hm\in\mathbb{R},$ $\alpha\hm\ge0,$ $\beta\hm>1$.
\item $\G_{q, q+1}(-x^q)=\G_{1/q, 1/q+1 }(-{1/x})/{({qx^{q+1}})},$ $x\hm>0,$ $q\hm>1$.
\item $\sum\nolimits_{k=0}^\infty ({\alpha^k}/{k!}) \G_{1/p,\, (k+p)/p}(- \alpha)\hm= 1$, 
$\alpha\hm>0$, $p\hm>1.$
\end{enumerate}


\noindent
Д\,о\,к\,а\,з\,а\,т\,е\,л\,ь\,с\,т\,в\,о\,.\ \ 
Свойство~1 следует из соотношения (6.1.46) в~\cite{AbSt}:
$$
\lim\limits_{n \to \infty}
\fr{\Gamma(n+\alpha)n^{\beta-\alpha}}{{\Gamma(n+\beta)}}=1\,,
$$
формулы Даламбера и~признака Лейбница.
Свойства~3--8 проверяются непосредственно. Свойство~9 следует из соотношения:
\begin{multline*}
\G_{q, q+1}(-x^q)= \int\limits_0^\infty e^{-(xt)^{q}} t^q e^{-t}\, dt={}\\
{}=
\fr{1}{{qx^{q+1}}}\sum\limits_{k=0}^\infty
\fr{(-1/x)^k}{{k!}}\int\limits_0^\infty z^{(k+1)/q}e^{-z}\, dz\,.
\end{multline*}
Для обоснования свойства~10 достаточно рас\-смот\-реть случайную величину~$N$, 
имеющую смешанное пуассоновское распределение со структурным распределением 
Вейбулла, для которой справедливо:
\begin{multline*}
\p(N=k) = \fr{p}{{\alpha^p k!}} \int\limits_0^\infty 
e^{-\lambda -(\lambda/\alpha)^p} \lambda^{k+p-1} \, d\lambda={}\\
{}= \fr{\alpha^k}{k!} \sum\limits_{n=0}^\infty 
\fr{(-\alpha)^n}{{n!}} \int\limits_0^\infty t^{(n+k)/p} e^{-t} \, dt
= {}\\
{}=\fr{\alpha^k}{k!}\, \G_{1/p,\, (k+p)/p}(- \alpha)\,.
\end{multline*}

Теорема доказана.

\smallskip

Приведем два утверждения, в~которых па\-ра\-мет\-ры входящего потока~$\lambda$ 
и~обслуживания~$\mu$ в~модели~$M/M/1/0$ имеют  экспоненциальное распределение 
и~распределение Вейбулла.

\noindent
\textbf{Теорема~2.}\ 
\textit{Пусть параметр входящего потока~$\lambda$ имеет экспоненциальное 
распределение $M(\theta)$, $\theta\hm>0$, а~параметр обслуживания~$\mu$ 
имеет распределение Вейбулла $W(p,\alpha)$, $p\hm>1$, $\alpha\hm>0$, причем~$\lambda$ 
и~$\mu$ независимы. Тогда при $x\hm>0$ функция распределения, плотность 
и~моменты коэффициента загрузки~$\rho$ имеют вид}:
\begin{align*}
F_\rho(x) &= 1 - \G_{1/p,\, 1}(-\theta \alpha x)\,; \\
f_\rho(x) &= \theta \alpha \G_{1/p,\, 1/p+1}(-\theta \alpha x)\,;\\
\e\,\rho^n &= \fr{n!}{(\theta \alpha)^n}\,\Gamma\left(1-\fr{n}{p}\right)\,, \enskip 
p>n\,,
\end{align*}
\textit{a функция распределения и~плотность вероятности <<непотери>> вызова~$\pi$ 
при $x\hm\in(0,1)$ определяются соотношениями}:
\begin{align*}
F_\pi(x) &= \G_{1/p,\, 1}\left(-\fr{\theta \alpha (1-x)}{x}\right)\,; \\
f_\pi(x) &= \fr{\theta \alpha}{{x^2}}\, \G_{1/p,\, 1/p + 1}
\left(-\fr{\theta \alpha (1-x)}{x}\right)\,.
\end{align*}

\noindent
Д\,о\,к\,а\,з\,а\,т\,е\,л\,ь\,с\,т\,в\,о\,.\ \ Заметим, что при $x\hm>0$
\begin{multline*}
F_\rho(x) =
\int\limits_0^{\infty}  \fr{p}{\alpha}\left(
\fr{u}{\alpha}\right)^{p-1}e^{-(u/\alpha)^{p}}(1-e^{-ux\theta})\, du={}\\
{}= 1 - \sum\limits_{k=0}^\infty \int\limits_0^{\infty} 
\fr{(-\theta x)^k}{k!}\,u^k e^{-\left(u/\alpha\right)^p} \ d\left(\fr{u}{\alpha}\right)^p ={}\\
{}=
1 - \sum\limits_{k=0}^\infty \int\limits_0^{\infty} 
\fr{(-\theta x)^k}{k!} \left(\alpha t^{1/p}\right)^{k}e^{-t} \, dt ={}\\
{}= 1 - \G_{1/p,\, 1}(-\theta \alpha x)\,,
\end{multline*}
откуда, воспользовавшись свойством~4 теоремы~1, получаем выражение для~$f_\rho(x)$.

Вычислим момент $n$-го порядка для коэффициента загрузки. Имеем:
\begin{multline*}
\e\,\rho^n = \int\limits_0^\infty x^n \theta \alpha \G_{1/p,\, 1/p+1}
(-\theta \alpha x) \, dx = {}\\
{} = \theta \alpha \int\limits_0^\infty \int\limits_0^{\infty} 
\exp \left\lbrace -\theta \alpha t^{1/p}x\right\rbrace t^{1/p}e^{-t} x^{n}\, dt  dx={}\\
{}=\int\limits_0^\infty \fr{e^{-t}}{(\theta \alpha)^n t^{n/p}} \times{}\\
{}\times
\int\limits_0^{\infty} \exp \left\lbrace -\theta \alpha t^{1/p}x\right\rbrace 
(\theta \alpha t^{1/p} x)^n \, d(\theta \alpha t^{1/p} x) \, dt={}\\
{}= \fr{\Gamma(n+1)}{{(\theta \alpha)^n}}\int\limits_0^\infty e^{-t}t^{-n/p}  \, dt 
= \fr{n!}{{(\theta \alpha)^n}}\Gamma\left(
1-\fr{n}{{p}}\right)\,, \\
 p>n\,.
\end{multline*}

Теперь получим функцию распределения для вероятности <<непотери>> вызова~$\pi$. Имеем:
\begin{multline*}
F_\pi(x) =  1 - \p\left(\rho<\fr{{1-x}}{{x}}\right) ={}\\
{}=
  \G_{1/p, 1}\left(-\fr{\theta \alpha (1-x)}{x}\right)\,,\enskip
  x\in(0,1)\,,
\end{multline*}
откуда, воспользовавшись свойством~4 теоремы~1, получаем выражение для $f_\pi(x)$.

Теорема доказана.

\smallskip


\noindent
\textbf{Теорема 3.}\ 
\textit{Пусть параметр входящего потока~$\lambda$ имеет распределение 
Вейбулла $W(q,\theta)$, $0\hm<q\hm<1$, $\theta\hm>0$, а~параметр обслуживания~$\mu$ 
имеет экспоненциальное распределение $M(\alpha)$, $\alpha\hm>0$, причем~$\lambda$ 
и~$\mu$ независимы. Тогда при $x\hm>0$ функция распределения, плотность 
и~математическое ожидание коэффициента загрузки~$\rho$ имеют вид}:
\begin{align*}
F_\rho(x) &=  1 - \G_{q, 1}\left(-\fr{x^q}{(\alpha \theta)^q}\right)\,; \\
f_\rho(x) &= \fr{qx^{q-1}}{{(\theta \alpha)^q}} \G_{q, q+1}
\left( -\fr{x^q}{(\alpha \theta)^q}\right)\,; 
\\
\e\,\rho&=(\theta \alpha)^q \Gamma(1-q)\,,
\end{align*}
\textit{a функция распределения и~плотность вероятности <<непотери>> вызова~$\pi$ 
при $x\hm\in(0,1)$ определяются соотношениями}:
\begin{align*}
F_\pi(x) &= \G_{q,\, 1}\left(-\fr{(1-x)^q}{(\alpha \theta x)^q}\right)\,;
\\
f_\pi(x) &= \fr{q}{{(\theta \alpha)^q x^2}}\left(
\fr{1-x}{{x}}\right)^{q-1}\! \G_{q, q + 1}\left(
-\fr{(1-x)^q}{(\alpha \theta x)^q}\right).
\end{align*}


\noindent
Д\,о\,к\,а\,з\,а\,т\,е\,л\,ь\,с\,т\,в\,о\,.\ \ Для $x\hm>0$ имеем:
\begin{multline*}
F_\rho(x) =
 \int\limits_0^{\infty} \left(1-e^{-(ux/\theta)^q}\right)\alpha e^{-\alpha u}\, du={}\\
 {}=
 1 - \int\limits_0^{\infty} e^{-\alpha u}\sum\limits_{k=0}^\infty 
 \fr{(-1)^k \left(ux/\theta\right)^{qk}}{{k!}} \, d(\alpha u) ={}\\
{}= 1 - \sum\limits_{k=0}^\infty \int\limits_0^{\infty} 
\fr{(-1)^k}{{k!}} \left(
\fr{tx}{{\alpha \theta}}\right)^{qk} e^{-t} \, dt = {}\\
{}=
1 - \G_{q,\, 1}\left(-\fr{x^q}{(\alpha \theta)^q}\right)\,.
\end{multline*}
Выражение для $f_\rho(x)$ получается из свойства~4 теоремы~1.
Найдем $n$-й момент коэффициента загрузки. Имеем:
\begin{multline*}
\e\,\rho^n ={}\\
{}=
\int\limits_0^\infty 
\fr{qx^{n+q-1}}{{(\alpha\theta )^q}}\sum\limits_{k=0}^\infty
\fr{(-1)^kx^{qk}}{{(\alpha \theta)^{qk}k!}}  
 \int\limits_0^\infty t^{q(k+1)}e^{-t}\, dt dx={}\\
{}= \int\limits_0^\infty \fr{q}{{(\theta \alpha)^q}}
\int\limits_0^\infty \exp\left\lbrace -\left(
\fr{tx}{{\alpha \theta}}\right)^q\right\rbrace x^{q+n-1} t^q e^{-t}\, dt   dx={}\\
 {}= \int\limits_0^\infty \fr{t^q e^{-t}}{{(\theta \alpha)^q}}
\int\limits_0^\infty \exp\left\lbrace -\left(
\fr{t}{{\alpha \theta}}\right)^qz\right\rbrace z^{(n-1)/q + 1} \, dz  dt={}\\
{}= \int\limits_0^\infty e^{-t}\left(
\fr{\theta \alpha}{t}\right)^{n+q-1}\int\limits_0^\infty e^{-u}
 u^{(n-1)/q + 1} \, du dt ={}\\
{}= \int\limits_0^\infty e^{-t}\left(
\fr{\theta \alpha}{{t}}\right)^{n+q-1}\Gamma\left(\fr{n-1}{q} + 2\right) \, dt={}\\
{}=(\theta \alpha)^{n+q-1}\Gamma\left(\fr{n-1}{q} + 2\right) \Gamma (2-n-q)\,, 
\end{multline*}
откуда получаем, что существует только момент первого порядка при $0\hm<q\hm<1.$

Для функции распределения вероятности <<непотери>> вызова~$\pi$ справедливо:
\begin{multline*}
F_\pi(x) = 1 - \p\left(\rho<
\fr{{1-x}}{{x}}\right)={}\\
{}= \G_{q,\, 1}\left(-\fr{(1-x)^q}{(\alpha \theta x)^q}\right)\,, \enskip
x\in(0,1)\,.
\end{multline*}
Выражение для $f_\pi(x)$ получается из соответст\-ву\-юще\-го свойства 
$\G_{\alpha,\, \beta} (x)$.

Теорема доказана.

\smallskip

\noindent
\textbf{Замечание.}\ В теоремах~2 и~3 при $p\hm = 1$ и~$q \hm= 1$ распределение 
Вейбулла $W(1, \alpha)$ совпадает с~экспоненциальным распределением~$M(1/\alpha)$. 
Результаты для общего экспоненциального случая были получены 
ранее в~работе~\cite{KuSh09b}.

\smallskip

\noindent
\textbf{Следствие.}\
Из доказанных выше теорем следует, что гам\-ма-экс\-по\-нен\-ци\-аль\-ная
 функция также обладает свойствами:
\begin{enumerate}
\setcounter{enumi}{10}
\item $\lim\nolimits_{x\to-\infty} \G_{\alpha,\, 1}(x) \hm= 0,$ $0\hm\leq \alpha\hm<1$;
\item $\int\nolimits_{0}^{+\infty} \G_{\alpha,\, \alpha + 1}(-x)\,dx \hm= 1,$ $0\hm<\alpha\hm<1$;
%\item $\int\limits_{-\infty}^{0} \G_{\alpha,\, \alpha + 1}(x)dx = 1,$ $0<\alpha<1$;
\item $\G_{\alpha,\, \alpha + 1}(x)\hm > 0,$ $x\hm\in\mathbb{R},$ $0\hm<\alpha\hm<1$;
\item $\G_{\alpha,\, 1}(x)$ строго монотонна на всей прямой при $0\hm\le\alpha\hm<1$.
\end{enumerate}





{\small\frenchspacing
 {%\baselineskip=10.8pt
 \addcontentsline{toc}{section}{References}
 \begin{thebibliography}{9}

\bibitem{KuSh2015}
\Au{Кудрявцев А.\,А., Шоргин~С.\,Я.}
Байесовские модели в~тео\-рии массового обслуживания и~надежности.~--- 
М.: ФИЦ ИУ РАН, 2015. 76~с.

\bibitem{KuSh09b}
\Au{Кудрявцев А.\,А., Шоргин~В.\,С., Шоргин~С.\,Я.} Байе\-сов\-ские
модели массового обслуживания и~надежности: общий эрланговский
случай~// Информатика и~её применения, 2009. Т.~3. Вып.~4. С.~30--34.

\bibitem{ZhaKuSh}
\Au{Жаворонкова Ю.\,В., Кудрявцев~А.\,А., Шоргин~С.\,Я.}
Байесовская рекуррентная модель роста надежности: бе\-та-рас\-пре\-де\-ле\-ние
параметров~// Информатика и~её применения, 2014. Т.~8. Вып.~2.
С.~48--54.

\bibitem{ZhaKuSh2}
\Au{Жаворонкова Ю.\,В., Кудрявцев~А.\,А., Шоргин~С.\,Я.}
Байесовская рекуррентная модель роста надежности: бета-рав\-но\-мер\-ное
распределение параметров~// Информатика и~её применения, 2015. Т.~9.
Вып.~1. С.~98--105.

\bibitem{Artin}
\Au{Artin E.} The gamma function.~--- New York, NY, USA: 
Holt, Rinehart and Winston, 1964. 39~p.

\bibitem{BeEr}
\Au{Бейтмен Г., Эрдейи~А.} Высшие трансцендентные функции. Т.~1.~/
Пер. с~англ.---  М.: Наука, 1973. 296~с.
(\Au{Bateman~H., Erdelyi~A.}  
{Higher transcendental functions}. Vol.~1.~--- New York\,--\,Toronto\,--\,London: 
McGraw-Hill Book Co., Inc., 1953. 302~p.)

\bibitem{AbSt}
\Au{Abramowitz M., Stegun~I.} 
Handbook of mathematical functions with formulas, graphs, and mathematical tables.~--- 
New York, NY, USA: Dover Publications, 1974. 1046~p.
 \end{thebibliography}

 }
 }

\end{multicols}

\vspace*{-6pt}

\hfill{\small\textit{Поступила в~редакцию 12.06.17}}

\vspace*{8pt}

%\newpage

%\vspace*{-24pt}

\hrule

\vspace*{2pt}

\hrule

%\vspace*{8pt}


\def\tit{GAMMA-EXPONENTIAL FUNCTION\\ IN~BAYESIAN QUEUEING MODELS}

\def\titkol{Gamma-exponential function in~Bayesian queueing models}

\def\aut{A.\,A.~Kudryavtsev and A.\,I.~Titova}

\def\autkol{A.\,A.~Kudryavtsev and A.\,I.~Titova}

\titel{\tit}{\aut}{\autkol}{\titkol}

\vspace*{-9pt}


\noindent
\noindent
Department of Mathematical Statistics, Faculty of Computational 
Mathematics and Cybernetics, M.\,V.~Lomonosov Moscow State University, 
1-52~Leninskiye Gory, GSP-1, Moscow 119991, Russian Federation



\def\leftfootline{\small{\textbf{\thepage}
\hfill INFORMATIKA I EE PRIMENENIYA~--- INFORMATICS AND
APPLICATIONS\ \ \ 2017\ \ \ volume~11\ \ \ issue\ 4}
}%
 \def\rightfootline{\small{INFORMATIKA I EE PRIMENENIYA~---
INFORMATICS AND APPLICATIONS\ \ \ 2017\ \ \ volume~11\ \ \ issue\ 4
\hfill \textbf{\thepage}}}

\vspace*{3pt}


\Abste{This paper considers the Bayesian approach to queueing theory and 
reliability theory. The Bayesian approach is useful for studying systems 
with alternating characteristics, the changes in which happen at the moments 
of time unpredictable for a~researcher, or large groups of systems of the same type. 
In the framework of this approach, it is assumed that key parameters of 
classical systems are not given and only their \textit{a~priori} 
distributions are known. By randomizing the system's parameters, the authors 
randomize its characteristics, for instance, the traffic intensity. 
The gamma-exponential function and some of its properties are introduced 
 as well as the results for probability characteristics of 
the system's traffic intensity and the probability that the claim 
received by the system will not be lost in the cases of the exponential 
and Weibull \textit{a~priori} distributions of $M/M/1/0$ system's\linebreak parameters.}

\KWE{Bayesian approach; queuing systems; reliability; mixed distribution; 
Weibull distribution; exponential distribution; gamma-exponential function}

\DOI{10.14357/19922264170413} 

%\vspace*{-12pt}

\Ack
\noindent
The work was partly supported by the Russian Foundation for Basic 
Research (project 17-07-00577).



%\vspace*{3pt}

  \begin{multicols}{2}

\renewcommand{\bibname}{\protect\rmfamily References}
%\renewcommand{\bibname}{\large\protect\rm References}

{\small\frenchspacing
 {%\baselineskip=10.8pt
 \addcontentsline{toc}{section}{References}
 \begin{thebibliography}{9}


\bibitem{1-kud-1}
\Aue{Kudryavtsev, A.\,A., and S.\,Ya.~Shorgin.} 2015. \textit{Bayesovskie modeli 
v~teorii massovogo obsluzhivaniya i~nadezhnosti} 
[Bayesian models in mass service and reliability theories]. Moscow: FRC
CSC RAS. 76~p.

\bibitem{2-kud-1}
\Aue{Kudryavtsev, A.\,A., V.\,S.~Shorgin, and S.\,Ya.~Shorgin.} 
2009. Bayesovskie modeli massovogo obsluzhivaniya i~nadezhnosti: 
obshchiy erlangovskiy sluchay [Bayesian queueing and reliability models: 
General Erlang case]. \textit{Informatika i~ee Primeneniya~--- Inform. Appl.}
3(4):30--34.

\bibitem{3-kud-1}
\Aue{Zhavoronkova, Iu.\,V., A.\,A.~Kudryavtsev, and S.\,Ya.~Shor\-gin.} 
2014. Bayesovskaya rekurrentnaya mo\-del' ros\-ta na\-dezh\-nosti: beta-raspredelenie 
pa\-ra\-met\-rov [Bayesian recurrent model of reliability growth: 
Beta-distribution of\linebreak parameters]. \textit{Informatika i~ee Primeneniya~--- 
Inform. Appl.} 8(2):48--54.

\bibitem{4-kud-1}
\Aue{Zhavoronkova, Iu.\,V., A.\,A.~Kudryavtsev, and S.\,Ya.~Shor\-gin.} 2015.
Bayesovskaya rekurrentnaya mo\-del' ros\-ta na\-dezh\-nosti: beta-ravnomernoe
raspredelenie pa\-ra\-met\-rov [Bayesian recurrent model of reliability growth: 
Beta-uniform distribution of parameters].
\textit{Informatika i~ee Primeneniya~--- Inform. Appl.} 9(1):98--105.

\bibitem{5-kud-1}
\Aue{Artin, E.} 1964. \textit{The gamma function.} New York, NY: 
Holt, Rinehart and Winston. 39~p. 

\bibitem{6-kud-1}
\Aue{Bateman, H., and A.~Erdelyi.} 1953. 
\textit{Higher transcendental functions}. Vol.~1. New York\,--\,Toronto\,--\,London: 
McGraw-Hill Book Co., Inc. 302~p.

\bibitem{7-kud-1}
\Aue{Abramowitz, M., and I.~Stegun.} 
1974. \textit{Handbook of mathematical functions with formulas, graphs, and mathematical tables}. 
New York, NY: Dover Publications, Inc. 1046~p.
\end{thebibliography}

 }
 }

\end{multicols}

\vspace*{-6pt}

\hfill{\small\textit{Received June 12, 2017}}

%\vspace*{-10pt}

\Contr

\noindent
\textbf{Kudryavtsev Alexey A.} (b.\ 1978)~--- 
Candidate of Sciences (PhD) in physics and mathematics, associate professor, 
Department of Mathematical Statistics, Faculty of Computational Mathematics 
and Cybernetics, M.\,V.~Lomonosov Moscow State University, 1-52~Leninskiye Gory, 
GSP-1, Moscow 119991, Russian Federation; \mbox{nubigena@mail.ru}

\vspace*{3pt}

\noindent
\textbf{Titova Anastasiia I.} (b.\ 1995)~--- 
student, Department of Mathematical Statistics, Faculty of Computational 
Mathematics and Cybernetics, M.\,V.~Lomonosov Moscow State University, 
1-52~Leninskiye Gory, GSP-1, Moscow 119991, Russian Federation; 
\mbox{onkelskroot@gmail.com}
\label{end\stat}


\renewcommand{\bibname}{\protect\rm Литература} 