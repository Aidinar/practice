\def \SS{{\frak S}}
\def \SK{{\frak K}}
\def \SL{{\frak L}}
%\def {\cal S}S{{\frak S}}
%\def {\cal S}K{{\frak K}}
%\def {\cal S}L{{\frak L}}

\def\stat{malashenko}

\def\tit{МЕТОД АНАЛИЗА ФУНКЦИОНАЛЬНОЙ УЯЗВИМОСТИ
ПОТОКОВЫХ СЕТЕВЫХ СИСТЕМ}

\def\titkol{Метод анализа функциональной уязвимости
потоковых сетевых систем}

\def\aut{Ю.\,Е.~Малашенко$^1$, И.\,А.~Назарова$^2$, Н.\,М.~Новикова$^3$}

\def\autkol{Ю.\,Е.~Малашенко, И.\,А.~Назарова, Н.\,М.~Новикова}

\titel{\tit}{\aut}{\autkol}{\titkol}

\index{Малашенко Ю.\,Е.}
\index{Назарова И.\,А.}
\index{Новикова Н.\,М.}
\index{Malashenko Yu.\,E.}
\index{Nazarova I.\,A.}
\index{Novikova N.\,M.}



%{\renewcommand{\thefootnote}{\fnsymbol{footnote}} \footnotetext[1]
%{Публикация подготовлена при финансовой поддержке Минобрнауки России (проект 2.882.2017/4.6).}}


\renewcommand{\thefootnote}{\arabic{footnote}}
\footnotetext[1]{Вычислительный центр им.\ А.\,А.~Дородницына Федерального исследовательского 
центра <<Информатика и~управ\-ле\-ние>> Российской академии наук, \mbox{malash09@ccas.ru}}
\footnotetext[2]{Вычислительный центр им.\ А.\,А.~Дородницына Федерального исследовательского центра 
<<Информатика и~управ\-ле\-ние>> Российской академии наук, \mbox{irina-nazar@yandex.ru}}
\footnotetext[3]{Вычислительный центр им.\ А.\,А.~Дородницына Федерального исследовательского центра 
<<Информатика и~управ\-ле\-ние>> Российской академии наук, \mbox{n\_novikova@umail.ru}}

%\vspace*{-18pt}


\Abst{Рассматриваются математические методы
анализа сетевых систем, предназначенных для передачи потоков взаимозаменяемых 
продуктов различным пользователям. Для описания процессов, происходящих в~системе, 
используется модель однопродуктовой сети. В~соответствии с~методологией исследования
 операций изучаются изменения функциональных характеристик сис\-те\-мы после 
 це\-ле\-на\-прав\-лен\-ных разрушающих воздействий. Предлагается способ получения 
 гарантированных оценок ущерба для каждого из равноправных пользователей 
 при полном разрушении физических и~логических элементов системы. Оценки 
 формируются на основании решения последовательности задач векторной 
 оптимизации с~лексикографическим минимаксным критерием. При определении 
 стратегии перераспределения потоков  используется апостериорная информация 
 об изменениях пропускных способностей сети.}

\KW{однопродуктовая потоковая сеть; функциональная уязвимость; оценка ущерба; 
принцип гарантированного результата}

\DOI{10.14357/19922264170406} 


\vskip 10pt plus 9pt minus 6pt

\thispagestyle{headings}

\begin{multicols}{2}

\label{st\stat}

\section{Введение}

Сетевые системы являются уязвимыми по отношению к~целенаправленным 
разрушающим воздействиям. Распределенные на больших территориях нефте- и~газопроводы, 
системы связи и~электроэнергетики трудно защитить от террористических 
угроз и~природных катаклизмов. В~связи с~этим активно развивается раздел 
прикладной математики, связанный с~исследованием уязвимости реальных сетей. 
Обзор классических моделей и~подходов к~изучению уязвимости однопродуктовой 
сети можно найти, например, в~\cite{Mur13}.

В настоящей работе для анализа изменений значений функциональных 
характеристик потоковых сетевых систем~\cite{Mal17} и~оценки ущерба от 
по\-вреж\-дений используется   методология  исследования опера\-ций. Предполагается, 
что разрушающее воздействие носит целенаправленный характер, а~атакующая 
сторона располагает полной информацией о~структуре сетевой системы и~стремится 
причинить максимально возможный ущерб. Однако в~общем случае как  цели атаки, 
так и~оценка возможных последствий противником неизвестны. 
В~этих условиях обороняющаяся сторона должна оценивать ущерб по своим
 критериям и~рассматривать все возможные повреждения. 
 В~рамках модельных построений предполагается, что после воз\-му\-ща\-юще\-го воздействия 
 будет доступна вся информация о~повреждениях и~система управления 
 способна перераспределить потоки в~сети так, чтобы минимизировать\linebreak
  потери. 
 При оценке ущерба решается последовательность минимаксных задач, что 
 позволяет получить <<справедливое>>  распределение (N-яд\-ро) 
 ограничен\-ных ресурсов среди равноправных пользователей. 
 Наборы оптимальных решений являются гарантированными оценками 
 функциональных характеристик поврежденной сети. 

\section{Модель сети}

Рассмотрим проблему оценки ущерба пользователей потоковой системы 
после крупномасштабных возмущений и/или целенаправленных разрушающих воздействий.
Для формализации такой системы введем обозначения, которые используются 
при описании потоковых  моделей~\cite{ford}.
Пусть сеть передачи единственного вида продукта, или 
однопродуктовая многополюсная сетевая система,   
задается ориентированным графом $\overline {\cal G} \hm= \langle \overline{\cal V}, \overline{\cal L} \rangle$ 
без петель, который определяется множеством вершин (узлов) 
$\overline{\cal V} \hm= \{v_1,v_2,\ldots ,v_ N\}$, где $|\overline{\cal V}| \hm= N$, ${\cal N}$ --- 
множество индексов вершин,
и множеством направленных дуг
$\overline {\cal L} \hm=\{l_{ij} \ | \ i \hm\in {\cal N}, \ j \hm\in {\cal N}, i \hm\not= j \}$,
соединяющих вершины. Здесь
$l_{ij} = (v_i, v_j)$~---  дуга, ведущая из вершины~$v_i$ в~вершину~$v_j$,  
$|\overline{\cal L}| \hm= L$.

Обозначим через ${\cal V}_\SS$ и ${\cal V}_\SK$  
множества вершин графа~$\overline {\cal G}$, 
которые являются соответственно источниками и~стоками для потока, который 
передается по многополюсной  сети;
${\cal N}_\SS$ и $\cal{N}_\SK$~---  множество индексов вер\-шин-ис\-точ\-ни\-ков 
и~вер\-шин-сто\-ков. 
Тогда
${\cal V}_\SS\hm = \{v_i | \ i \hm\in {\cal N}_\SS \}$,  $ |{\cal V}_\SS| \hm= S$,
${\cal V}_\SS \subset \cal{V}$,  ${\cal N}_\SS \hm\subset {\cal N}$,  $S \hm\ge 1,$
${\cal V}_\SK = \{v_i | \ i \hm\in {\cal N}_\SK\}$,   
$|{\cal V}_\SK | \hm= K$, ${\cal V}_\SK \hm\subset {\cal V}$,  
${\cal N}_\SK \hm\subset {\cal N}$, $K \hm\ge 1,$
${\cal V}_\SS \bigcap {\cal V}_\SK \hm= \emptyset. $
Считается, что на дугах графа~$\overline {\cal G}$ заданы веса~--- значения~$d_{ij}$ 
пропускной способности дуг~$l_{ij}$. Вектор~$d$ 
ограничивает величину потока  по дугам,
$d = \{d_{ij} \ |\  d_{ij} \hm\geq 0$, $l_{ij}\hm\in \overline{\cal L}\}$.

Для того чтобы свести многополюсную модель к~модели однопродуктовой 
сети с~единственной парой ис\-точ\-ник--сток, в~граф~$\overline {\cal G}$ введем следующие вершины:
\begin{description}
\item[\,] $v_0$~---  единственный источник потока бесконечной мощности  
и~дуги $(v_0, v_j)$,  $j \hm\in {\cal N}_\SS$, 
соединяющие~$v_0$ с~каж\-дым уз\-лом-ис\-точ\-ником. 
Для каждой дуги~$l_{0j}$ определим верхнее ограничение~$d_{0j}$, которое 
соответствует величине максимального потока из~$v_j$ в~систему. 
Назовем  дуги $(v_0, v_j)$, 
$j \hm\in {\cal N}_\SS$, ду\-га\-ми-ис\-точ\-ни\-ка\-ми 
и~обозначим их множество через
$\hat{{\cal L}}\hm=\{l_{0 j} \ | \  j \in {\cal N}_\SS \}$, $| \hat{{\cal L}}| \hm=  S; $\\[-13pt]
\item[\,] 
$v_{N+1}$~---  единственный узел-сток бесконечного объема  
и~дуги $(v_i, v_{N+1})$,  
$i \hm\in {\cal N}_\SK$, соединяющие каждый узел-сток 
с~$v_{N+1}$.  Для каж\-дой дуги~$l_{i,N+1}$ определим верхнее ограничение~$d_{i,N+1}$, 
которое соответствует величине максимального потока из системы в~$v_i$. 
По предположению значение~$d_{i,N+1}$ задает верхний предел для величины потока, 
который покидает систему через узел~$v_i$.
Назовем  дуги $(v_i, v_{N+1})$, $i \hm\in {\cal N}_\SK$,  
ду\-га\-ми-сто\-ка\-ми или стоковыми дугами и~обозначим их множество через
$\tilde{{\cal L}}\hm=\{\ l_{i,N + 1} \ | \ i  \hm\in {\cal N}_\SK\}$, 
$|\tilde{{\cal L}}|\hm =  K.$
Ориентированный граф, который определяется множествами вершин 
${\cal V} \hm= \overline {\cal V} \bigcup \{v_0, v_{N+1}\}$ и~дуг 
${\cal L} \hm= \tilde{{\cal L}} \bigcup \overline {{\cal L}} \bigcup \hat {\cal L}$, обозначим 
${\cal G} \hm= \langle {{\cal V}}, {{\cal L}} \rangle $.
\end{description}

Для графа~${\cal G}$ введем обозначения:
\begin{description}
\item[\,] $x_{ij}$~--- поток по дуге $l_{ij}, l_{ij}\hm\in {{\cal L}}$, протекающий 
в~соответствии с~ее направлением;\\[-13pt]
\item[\,] ${\cal N}^{-}_j$~--- множество индексов уз\-лов-пред\-шест\-вен\-ни\-ков $j$-го 
(узлов, из которых исходят дуги, ведущие в~$j$-й узел), 
${\cal N}^{-}_j \hm\subset {\cal N} \cup \{0\} $;\\[-13pt]
\item[\,] ${\cal N}^{+}_j$~--- множество индексов уз\-лов-по\-сле\-до\-ва\-те\-лей $j$-го 
(узлов, в~которые ведут дуги, исходящие из $j$-го узла), 
${\cal N}^{+}_j \hm\subset {\cal N} \cup \{N+1\}$.
\end{description}

\vspace*{-12pt}

\columnbreak

Поток
$x= \langle x_{0j},\ldots, x_{ij},\ldots, x_{i,N+1}\rangle$,  где 

\noindent   
\begin{multline}
i \in {\cal N} \cup \{0\}, \enskip
j \in {\cal N} \cup \{N+1\}, \enskip i \neq j, \\
l_{ij} \in {{\cal L}},  \mbox{ и~если} \ i  = 0, \ \mbox{то} \ j \not = N+1\,,
\label{e1-mal}
\end{multline}
проходящий по дугам $l_{ij}\hm\in {{\cal L}}$, должен удовлетворять: 
\begin{itemize}
\item условию сохранения потока в~транзитных узлах, т.\,е.\

\noindent
\begin{equation}
\sum\limits_{i \in {\cal N}^{-}_j}^{}{x_{ij}}= \sum\limits_{i \in {\cal N}^{+}_j}^{}{x_{ji}},  \enskip
  j \in {\cal N}\,;
  \label{e2-mal}
  \end{equation}
\item ограничению на пропускную способность соответствующих дуг, т.\,е.\

\noindent
\begin{equation}
\hspace*{-8mm}0 \le x_{ij} \le d_{ij}, \enskip l_{ij}\in {{\cal L}}, \ \mbox{для } i, j\ 
\mbox{выполняется }~(1). \!\!
\label{e3-mal}
\end{equation}
\end{itemize}

Обозначим через~${\cal X}$ множество всех допустимых потоков в~сети:

\noindent
$$
{\cal X} = \{x \ | \mbox{ выполняются~(1)--(3)}\} \,.
$$

\vspace*{-9pt}

\section{Функциональные характеристики однопродуктовой сети}

При анализе функциональных возможностей системы будем рассматривать потоки 
по стоковым дугам.
Последние перенумеруем по некоторому правилу натуральными числами от~1 до~$K$  
и~введем множество~${\cal K}$ номеров стоковых дуг, т.\,е.\ установим взаимно 
однозначное  соответствие $l_{k} \hm= l _{j, N+1}$, 
$k \hm= \overline{1, K}$, $j \hm\in {\cal N}_\SK.  $

Обозначим через $d_{k}$ пропускную способность \mbox{$k$-й} стоковой дуги, 
$d_{k}\hm = d _{j, N+1}$, $j \hm\in {\cal N}_\SK$,  $k \hm= \overline{1, K};$
$\overline x_{k}$~--- поток по $k$-й стоковой дуге, 
$\overline x_{k} \hm= x _{j, N+1}$, $x _{j, N+1}\hm\geq 0$, $k \hm= \overline{1, K}$,
$j \hm\in {\cal N}_\SK. $
Таким образом, вектор
$ \overline x \hm= \langle \overline x_{1}, \ldots, \overline x_{k}, \cdots, 
\overline x_{K}\rangle$, 
$k \hm= \overline{1, K},$
покомпонентно определяет величину потока, который передается по каждой 
стоковой дуге сети  в~соответствии с~некоторым допустимым потоком $x \hm\in {\cal X}$.
Обозначим через~$\overline{{\cal X}}$ множество всех допустимых векторов~$\overline x$:

\noindent
\begin{multline}
\overline{{\cal X}} = \{ \overline x \ | \ \overline x_{k} = x _{j, N+1}\,, \\
k = \overline{1, K}, \ j \in {\cal N}_\SK, \ \ x \in {\cal X} \}\,. 
\label{e4-mal}
\end{multline}
Множество $\overline{{\cal X}}$ является проекцией множества~${\cal X}$ всех допустимых потоков в~сети и~множеством 
допустимых потоков по стоковым дугам.

В рамках формализма потоковой модели будем говорить, что 
вектор~$\overline x$ описывает функциональные возможности однопродуктовой сети 
при передаче потока~$x$. Каждый элемент множества~$\overline{{\cal X}}$ пред\-став\-ля\-ет 
собой вектор возможных значений функ\-цио\-наль\-ных характеристик потоковой 
сетевой систе-\linebreak\vspace*{-12pt}

\pagebreak

\noindent
мы, удовлетворяющий условиям~(\ref{e1-mal})--(\ref{e4-mal}).
Множество~$\overline{{\cal X}}$ выпукло и~описывает функциональные возмож\-ности изучаемой 
однопродуктовой сети. Векторы~$\overline x$, лежащие на  границе Парето~\cite{Pod} 
множества~$\overline{{\cal X}}$, т.\,е.\ неулучшаемые ни по одной компоненте без ущерба 
для других, будем называть предельными значениями функциональных характеристик 
сети.

Пусть в~момент начала наблюдений $t \hm= 0$ однопродуктовая сеть работает в~некотором 
ста\-цио\-нар\-ном режиме. При этом по стоковым дугам передается поток
 $\overline x^0 \hm= \langle \overline x_1^0, \ldots, \overline x_k^0,\ldots,  \overline x_K^0\rangle$, 
 $\overline x^0 \hm\in \overline{{\cal X}}$,\linebreak $\overline x_k^0 \hm> 0$, 
 $k \hm= \overline{1, K}.$
Вектор~$\overline x^0$ будем называть начальной функциональной характеристикой сетевой 
сис\-те\-мы (в~момент $t \hm= 0$), а~величины компонент~$\overline x_k^0$~--- 
исходными значениями~$\overline x^0$.

Будем считать, что за каждой стоковой дугой стоит некий абстрактный пользователь 
(потребитель), имеющий определенное требование (запрос)~$f_k$, $k \hm= \overline{1, K}$, 
на обеспечение продуктом, который передается по сети. Вектор
$ f \hm= \langle f_{1}, \ldots, f_{k}, \ldots, f_{K}\rangle$, $f_{k} \hm> 0$, 
$k \hm= \overline{1, K},$
описывает требования всех потребителей сети.  
В~настоящей модели предполагается, что недопоставка продукта одному 
пользователю не может быть восполнена за счет поставки дополнительного 
объема продукта другому пользователю. Указанное предположение делает 
пользователей сети в~определенном смысле равноправными, а их требования и~объемы 
имеющихся потоков~--- невзаимозаменяемыми.
Пусть в~момент  времени $t\hm = 0$ для всех стоковых дуг выполняется
$d_k \hm=  \overline x_k^0 \hm= f_k$, $k \hm= \overline{1, K},$
т.\,е.\  в~начале наблюдения за системой все требования на передачу 
потока соответствуют запросам потребителей и~полностью удовлетворяются.
Граф с~ограничениями на пропускные способности стоковых дуг~$d_k$, 
$k \hm= \overline{1, K}$, обозначим через~${\cal G}^0$.

Определим однопродуктовую двухполюсную сеть~${\cal S}^0$ с~по\-мощью ориентированного 
графа~${\cal G}^0$ и~требований~$f$ на передачу потока всех потребителей сети,
 ${\cal S}^0 \hm= \langle {\cal G}^0; f \rangle$.
Перейдем к~изучению функциональных характеристик сети~${\cal S}^0$ в~условиях 
разрушающего воздействия, а~именно: исследуем изменение величин потоков по 
стоковым дугам при полном разрушении некоторых элементов~${\cal G}$. 


\section{Анализ функциональных характеристик сети
после разрушающего воздействия}

Пусть стационарно работающая однопродуктовая сеть подвергается 
разрушающему воздействию (удару)~${\cal W}$, при котором выходят из строя 
пол\-ностью несколько дуг~$l_{ij}$ графа~${\cal G}^0$. 
В~рамках рассматриваемой модели предполагается, что могут быть повреждены любые 
дуги, кроме стоковых, распределение удара по дугам сети~${\cal S}^0$ заранее не известно, 
а~множество вершин  остается неизменным. Считается, что объем потока по 
стоковым дугам уменьшается в~зависимости от степени разрушения сети. 
Обозначим через
${\cal L}({\cal W})$ множество по\-вреж\-ден\-ных дуг, тогда
${\cal L}({\cal W}) \hm\subset \overline {\cal L} \bigcup \hat {\cal L}$, ${\cal L}({\cal W}) \bigcap  \tilde{{\cal L}} \hm= \emptyset;$
${\cal G}({\cal W})$~--- граф сети после нанесенного удара;
${\cal S}({\cal W}) \hm= \langle {\cal G}({\cal W}); f \rangle$~--- сеть, 
поврежденную разрушающим 
воздействием~${\cal W}$.
Таким образом, разрушающее воздействие полностью определяется множеством 
уничтоженных дуг~${\cal L}({\cal W})$. Пусть
$d({\cal W})$~--- вектор пропускной способности дуг поврежденной сети~${\cal S}({\cal W})$,  
для каждой компоненты~$d_{ij}({\cal W})$ которого выполняется
\begin{equation}
d_{ij}({\cal W}) =
\begin{cases}
0 , & \mbox{если } l_{ij} \in {\cal L}({\cal W})\,;  \\
d_{ij}, & \mbox{если } \ l_{ij} \in \cal L \backslash {\cal L}({\cal W}),
\end{cases}
\label{e5-mal}
\end{equation}
$x_{ij}({\cal W})$, где для $i,  j$ выполняется~(1),~--- поток  по дуге~$l_{ij}$ 
после удара. Тогда перераспределение потоков в~сети~${\cal S}({\cal W})$ описывается вектором
$$
x ({\cal W})=\langle x_{0j}({\cal W}),\ldots, x_{ij}({\cal W}),\ldots, x_{i(N+1)}({\cal W})\rangle  
$$
при условии, что для  номеров $i,  j$ выполняется~(\ref{e1-mal}).

Для любых потоков $x({\cal W})$ в~поврежденной сети~${\cal S}({\cal W})$ должны выполняться 
стандартные ограничения на передачу потока по дуге
\begin{equation}
 \left.  
 \begin{array}{l}
0 \le x_{ij}({\cal W}) \le d_{ij}({\cal W}),  \\[6pt]
l_{ij}\in {{\cal L}}, \\[6pt]
\mbox{для } d_{ij}({\cal W})\  \mbox{выполняется~(5)}, \\[6pt]
\mbox{для } i, j\ \mbox{выполняется~(1)} 
                    \end{array}
 \right \} 
 \label{e6-mal}
 \end{equation}
и закон сохранения потока в~каждом узле
\begin{equation}
\sum\limits_{i \in {\cal N}^{-}_j}^{}{x_{ij}({\cal W})}= 
\sum\limits_{i \in {\cal N}^{+}_j}^{}{x_{ji}({\cal W})},  \enskip  j \in {\cal N}. 
\label{e7-mal}
\end{equation}
Множество ${\cal X}({\cal W})$  допустимых потоков  в~по\-вреж\-ден\-ной сети~${\cal S}({\cal W})$ 
определяется условиями~(\ref{e1-mal}), (\ref{e6-mal}) и~(\ref{e7-mal}):
$$
{\cal X}({\cal W}) = \{x({\cal W}) \ | \mbox{\ выполняется~(1),~(6) и~(7)}\}.
$$

Обозначим $\overline x_k({\cal W})$~--- поток по $k$-й стоковой дуге после разрушающего 
воздействия;
$\overline x({\cal W})$~--- вектор потоков по всем стоковым дугам после удара
$\overline x({\cal W}) \hm= \langle \overline x_1({\cal W}), \ldots, \overline x_k({\cal W}),\ldots,  
\overline x_K({\cal W})\rangle$, $k \hm= \overline{1, K}.$
Множество
\begin{multline*}
 \overline{{\cal X}}({\cal W}) = \left\{ \overline x({\cal W}) \ | \ \overline x_{k}
 ({\cal W}) = 
 x _{j (N+1)}({\cal W}), \right.\\
  \left.k = \overline{1, K},  \ j \in {\cal N}_\SK, \ \ x({\cal W}) \in {\cal X} ({\cal W})
 \right\} 
 \end{multline*}
описывает функциональные возможности системы после разрушения. 

\section{Оценка ущерба пользователей сети}

Анализ функциональных характеристик системы начнем с~изучения вопроса о~принадлежности 
исходного  вектора~$\overline x^0$ множеству допустимых потоков~$\overline{{\cal X}}({\cal W})$.
Действительно, если $\overline x^0 \hm\in \overline{{\cal X}}({\cal W})$, то потоки по стоковым дугам 
остаются прежними, требования пользователей в~поврежденной сети могут быть 
удовлетворены полностью. В~этом случае для передачи запрошенного продукта, 
возможно, просто потребуется другая маршрутизация. В~противном случае 
(если $\overline x^0 \hm\not \in \overline{{\cal X}}({\cal W})$) выполнить сразу все требования 
невозможно, и~потребители несут определенный ущерб в~результате разрушения сети.

В разд.~2 было введено понятие вектора исходных функциональных 
характеристик~$\overline x^0$, для которого покомпонентно
$ x_k^0 \hm= f_k$, $k \hm= \overline {1, K}.$
При анализе ущерба каждую компоненту~$f$ 
будем трактовать как требование определенного  пользователя на восстановление 
величины соответствующего потока на прежнем уровне и~оценивать отклонение 
возможных значений функциональных характеристик поврежденной сети от исходных.

Введем величину~$u_k({\cal W})$, численно равную разности между
требованием $k$-го пользователя  и~потоком, протекающим по $k$-й 
стоковой дуге после удара:
$$
u_k({\cal W}) = f_k - \overline x_k({\cal W}) = \overline x_k^0 - \overline x_k({\cal W}), \enskip k = \overline {1, K},
$$
которую назовем недопоставкой потока $k$-му  потребителю, или $k$-й недопоставкой.
Вектор $u({\cal W}) \hm= \langle u_1({\cal W}), u_2({\cal W}), \ldots, u_k({\cal W}), \ldots, u_K({\cal W})  \rangle$
описывает недопоставки всем пользователям сети после разрушающе\-го 
воздействия.

Назовем  ущербом  $k$-го потребителя отношение величины недопоставки~$u_k({\cal W}) $ 
к~соответствующему  запрошенному количеству потока~$f_k$ и~обозначим эту величину 
через

\noindent
\begin{multline*}
\varpi_k({\cal W}) = \fr{u_k({\cal W})}{f_k}= \fr{f_k - \overline x_k({\cal W})}{f_k}\,,\\
0 \le  \varpi_k \le 1\,,  \enskip k = \overline{1, K}\,.
\end{multline*}
В модели предполагается, \ что все пользователи 
равноправны, т.\,е.\ в~любой момент времени при распределении потоков никакому 
потребителю не отдается предпочтения.
Для оценки ущерба пользователей  решим следующую  задачу оптимизации.

\smallskip

\noindent
\textbf{Задача} {\boldmath{$C$}}. Найти
$\min\limits_{x({\cal W}) \in {\cal X}({\cal W})} \max\limits_{k = \overline{1, K}} 
\varpi_{k} $
при условии
$$
\varpi_{k} = \fr{f_{k} - \overline x_{k}({\cal W})}{f_{k}}, \enskip
 0 \le \varpi_k \le 1, \enskip k = \overline{1, K}.  
 $$

В задаче~$C$ ведется поиск распределения потоков, при котором минимизируется 
максимальный ущерб любого пользователя. Последнее означает, что максимальное 
отклонение величины любого потока по стоковой дуге от начальной 
(в~процентном отношении) должно быть минимальным.
Решение задачи~$C$ эквивалентно решению следующей задачи минимизации~\cite{Dan}~--- 
поиска наименьшего гарантированного ущерба.

\smallskip

\noindent
\textbf{Задача} {\boldmath{$C_1$}}.   Найти
$\min\limits_ {x({\cal W}) \in {\cal X}({\cal W}), \ \omega} \omega $
при условиях
\begin{equation}
 \left.  
 \begin{array}{l}
\displaystyle \omega \geq \fr{f_k - \overline x_k({\cal W})}{f_k}, \enskip
 k = \overline{1, K}\,;\\[6pt]
0 \le \omega \le 1\,. 
                    \end{array}
 \right \}
 \label{e8-mal}
 \end{equation}
 
Пусть в~результате решения задачи~$C_1$ получено оптимальное значение 
параметра~$\omega$
$$
 \omega ^{*} = \min\limits_{(x({\cal W}) 
 \in {\cal X}({\cal W}), \omega) \in (8)}\omega \,. 
 $$

Если в~процессе решения задачи~$C_1$ найден наименьший гарантированный ущерб
 потребителей $\omega^* \hm= 0$, то, несмотря на разрушающее воздействие, 
 исходный  вектор потоков~$\overline x^0$ принадлежит множеству~$\overline{{\cal X}}$. 
 Если $\omega^* \hm= 1$, то разрушен разрез графа~$\overline{{\cal G}}$, 
 разделяющий хотя бы одну пару ис\-точ\-ник--сток. 
 
 Далее рассмотрим промежуточный 
 вариант.
Пусть решение задачи~$C_1$:
\begin{equation}
  \omega^* = \omega^*_1\,;\quad
  0 < \omega_1^* < 1\,, 
  \label{e9-mal}
  \end{equation}
тогда начальный вектор~$\overline x^0$ не принадлежит множеству~$\overline{{\cal X}}$, 
а~вектор значений предельных функциональных характеристик~$\overline x^*({\cal W})$ 
будет лежать на границе Парето множества~$\overline{{\cal X}}({\cal W})$. Перейдем к~построению
 вектора~$\overline x^*({\cal W})$.

Формально выполнение~(\ref{e9-mal}) означает, что существует непустое подмножество
 номеров стоковых дуг~$\overline{\cal K}^*_1$,  $\overline{\cal K}^*_1 \hm\subseteq {\cal K}$, 
 для которых ущерб  в~точности равен~$\omega_1^* $. 
 И~этот ущерб  невозможно уменьшить, не нарушив соотношения $\omega_{k} \hm\le 
 \omega_1^* $ хотя бы для одного номера  $k \hm\in {\cal K}$:
\begin{multline*}
\overline {\cal K}^*_1 =\left \{k \in {\cal K} \ | \ \fr{f_k - \overline x_k^1({\cal W})}{f_k} =\omega^*_1\right.
\\
\left. \forall \ \overline x^1({\cal W}) : \ (\overline x^1({\cal W}), \omega^*_1)\in  
\mathrm{Arg}\,\min_{(x({\cal W}) \in {\cal X}({\cal W}),\ \omega) \in~(8)} \!\! \omega
\vphantom{\fr{f_k - \overline x_k^1({\cal W})}{f_k}}
\right\}\!. \hspace*{-8.7465pt}
\end{multline*}
Множество $\overline{\cal K}^*_1$ для задачи~$C_1$ можно построить, решив серию задач 
линейного программирования~\cite{Yen, Mal99}.

Стоковые дуги, индексы которых входят в~множество~$\overline {\cal K}^*_1$, назовем дугами 
(или пользователями) с~ущер\-бом первого уровня. Если $\overline {\cal K}^*_1 \hm= {\cal K}$, то 
относительный ущерб  одинаков для всех пользователей. Если  $\overline {\cal K}^*_1 \hm\not = {\cal K}$, 
то найдутся стоковые дуги, поток по которым можно повысить без увеличения ущерба 
для потребителей с~первым уровнем.
Для поиска таких стоковых дуг зафиксируем потери для пользователей с~индексами 
из множества~$\overline{\cal K}^*_1$ на уровне~$\omega^*_1$ и~для $\overline {\cal K}_1 \hm= {\cal K} \backslash \overline{\cal K}^*_1$ 
решим следующую задачу.

\smallskip

\noindent
\textbf{Задача}  {\boldmath{$C_2$}}.  \ Найти
$ \min\limits_{x({\cal W}) \in {\cal X}({\cal W}), \ \omega} \omega$
при выполнении
\begin{equation}
 \left.  \begin{array}{l}
 \displaystyle \omega \geq \fr{f_k - \overline x_k({\cal W})}{f_k}\,, \enskip 
 k \in \overline {\cal K}_1\,; \\[6pt]
0 \le \omega \le 1\,;\\[6pt]
\displaystyle \fr{f_k - \overline x_k({\cal W})}{f_k} = \omega^*_1\,, \enskip k \in  \overline{\cal K}^*_1\,. 
                    \end{array}
 \right \} 
 \label{e10-mal}
 \end{equation}

Пусть $\omega^*_2$~--- значение минимума в~этой задаче. Аналогично 
множеству~$\overline {\cal K}^*_1$ построим 
множество~$\overline {\cal K}^*_2$ стоковых дуг-по\-тре\-би\-те\-лей, 
для которых
\begin{multline*}
\overline {\cal K}^*_2 = \left\{ k \in \overline{\cal K}_1 \ | \  
\fr{ f_{k} - \overline x^2_{k}({\cal W})}{f_{k}} =\omega^*_2 \ \forall \ \overline x^2({\cal W}) :\right.\\
\left.\left(\overline x^2({\cal W}), \omega^*_2\right)\in  \mathrm{Arg}\,\min\limits_{(x({\cal W}) \in {\cal X}({\cal W}), 
\ \omega) \in (10) } \: \omega \right\}. 
\end{multline*}
Назовем стоковые дуги, индексы которых входят в~множество~$\overline{\cal K}^*_2$, 
дугами (или потребителями) с~ущербом второго  уровня.

Если $\overline{\cal K}^*_1\bigcup \overline{\cal K}^*_2 \hm= {\cal K}$, то решение задачи завершено. 
В~частности, если гарантированный ущерб  потребителей из~${\cal K}^*_2$ не равен нулю, 
то ущерб любого потребителя соответствует  либо  первому, либо второму уровню 
(в~случае, когда $\omega^*_2 \hm= 0$, пользователи из~$\overline{\cal K}^*_2$ не пострадали). 
Если $\overline{\cal K}^*_1\bigcup \overline{\cal K}^*_2 \hm\neq {\cal K}$, то продолжим построения, 
определив пользователей с~третьим и,~если потребуется, с~последующими уровнями ущерба. 
На шаге~$p$ решается следующая задача.

\smallskip

\noindent
\textbf{Задача} \ {\boldmath{$C_{p}$}}.
Найти $\min\limits_{x({\cal W}) \in {\cal X}({\cal W}), \ \omega} \omega$
при условиях
\begin{equation}
\left.
\begin{array}{l}
 \omega \geq \displaystyle \fr{f_{k} -\overline x_{k}({\cal W})}{f_{k}},\   
 k \in \overline{\cal K}_{p-1}, \\[9pt]
\hspace*{20mm} \overline{\cal K}_{p-1} = {\cal K} \backslash 
 \bigcup_{n=1}^{p-1}  {\overline{\cal K}^*_n}, \ \overline{\cal K}_0 = {\cal K};\\
0 \le \omega \le 1; \\[9pt]
\displaystyle \fr{f_{k} - \overline x_{k}({\cal W})}{f_{k}}=\omega^*_n, \  
k \in  {\overline{\cal K}^*_n}, \quad n = \overline{1, p-1}.
\end{array}
\right\} 
                    \end{equation}
% \label{e11-mal}


В ходе ее решения формируется множество~$\overline {\cal K}^*_p$ стоковых дуг, 
или список потребителей $p$-го уровня ущерба:

\columnbreak

\noindent
\begin{multline*}
\overline {\cal K}_p^* = \left\{ k \in \overline {\cal K}_{p-1} \ | \ 
\fr{f_{k} - \overline x^p_{k}({\cal W})}{f_{k}} =\omega^*_p \  \forall \ \overline x^p({\cal W}) : \right.\\
\left. \left(\overline x^p({\cal W}), \omega^*_p\right)
\in \textrm{Arg} \min\limits_{(x({\cal W}) \in {\cal X}({\cal W}), \ \omega) \in\ (11)} \ 
\omega \right\}, 
\end{multline*}
где $\omega^*_p$~--- оптимальное значение для задачи~$C_{p}$.

Поскольку число стоковых дуг в~сети конечно, то через конечное число~$P$~шагов, 
$P \hm\le | {\cal K} |$, будет исчерпано все множество стоковых дуг, т.\,е.\
 $\overline {\cal K}_P^*$ совпадет с~$\overline {\cal K}_{P-1} \hm= {\cal K} \backslash 
 \bigcup\limits^{P-1}_{n=1}{\overline {\cal K}^*_n}$ и~$\overline {\cal K}_P \hm= \emptyset $. 
 В~результате будет построено множество~${\cal X}^*({\cal W})$ потоков, реализующих 
 последовательный лексикографический минимакс уровня ущерба  потребителей 
 и,~следовательно, удовлетворяющих условию равноправной оптимальности:
 
 \noindent
\begin{align*}
\overline{\cal X}^*({\cal W}) &=  \mathrm{Arg}\,\min  \left\{ \max\limits_{k \in \overline{\cal K}_{P-1}} 
\fr{f_{k} - \overline x_{k}^{P-1}({\cal W})}{f_{k}} \right\}\,; \\
\overline x^{P-1}({\cal W}) &\in  \mathrm{Arg}\,\min  \left\{ \max\limits_{k \in \overline {\cal K}_{P-2}} 
\fr{f_{k} - \overline x_{k}^{P-2}({\cal W})}{f_{k}} \right\}\,;\\
\ldots\\[-3pt]
\overline x^{1}({\cal W}) &\in  \mathrm{Arg}\,\min\limits_{x({\cal W}) \in {\cal X}({\cal W})}  \left\{ 
\max\limits_{k \in {\cal K}} \fr{f_{k} - \overline x_{k}({\cal W})}{f_{k}} \right\}\,.
\end{align*}

Процесс построения оптимальных потоков
остановится на $P$-м шаге, как только окажется, что множество всех 
стоковых  дуг исчерпано. Тогда ущерб  последней группы потребителей будет 
равен оптимальному значению параметра~$\omega^*_{P}$. 
В~част\-ности, если гарантированный ущерб отдельных  потребителей окажется 
равным нулю, то их требования на передачу исходного потока в~разрушенной 
сети удовлетворяются полностью. Отметим, что из условия $d_k \hm=  \overline x_k^0 \hm= f_k$,
$k \hm= \overline{1, K}$, следует, что потребитель не может получить продукта  больше, 
чем исходные значения потока.

Результатами решения задач $C_1$--$C_{P}$ являются вектор
$\overrightarrow{\omega}^*\hm = \langle  \omega^*_1,  \omega^*_2, \ldots,  
\omega^*_{P} \rangle, $
который определяет гарантированный ущерб для всех пользователей 
поврежденной сети, 
и~эффективный вектор потоков~$\overline x^*({\cal W})$, лежащий на границе 
Парето мно\-жества~$\overline{{\cal X}}({\cal W})$.

На рисунке схематично изображен процесс поиска оптимального вектора 
потоков~$\overline x^*({\cal W})$ для $K \hm= 2$. Здесь по осям отложены значения 
потоков по двум стоковым дугам. Граница множества до\-пус\-ти\-мых векторов~--- 
потоков по стоковым дугам до нанесения удара~$\overline{{\cal X}}$~--- обозначена 
штрихпунктирной линией, множество~$\overline{{\cal X}}$ заштриховано линиями с~убыванием вправо.
Вектор $(\overrightarrow{OE})$ описывает требования на передачу потока 
в~неповрежденной сети.  
Граница множества~$\overline{{\cal X}}({\cal W})$~--- допустимых 
век\-то\-ров по\-то\-ков по стоковым дугам после нанесения удара обозначена штрихпунктирной 
линией с~двумя точками, $\overline{{\cal X}}({\cal W})$ заштриховано линиями с~убыванием влево. 
Вектор~$(\overrightarrow{OE})$ лежит вне множества~$\overline{{\cal X}}({\cal W})$.

\begin{figure*} %fig1
\vspace*{1pt}
 \begin{center}
 \mbox{%
 \epsfxsize=99.743mm 
 \epsfbox{mal-1.eps}
 }

\vspace*{6pt}

{\small Множества допустимых потоков до и~после удара}

 \end{center}
\end{figure*}


Одному из оптимальных решений задачи~$C_1$ на рисунке отвечает точка~$D$, 
лежащая на границе~$\overline{{\cal X}}({\cal W})$ и~имеющая координаты $(\overline x_1^*({\cal W}), \overline x_2({\cal W}))$:
\begin{align*}
 \overline x_1^*({\cal W}) &= f_1 - u_1({\cal W}) = f_1 - f_1\omega^*_1 = f_1 (1 - \omega^*_1)\,; \\
 \overline x_2({\cal W}) &=f_2 - u_2'({\cal W}) = f_2 - f_2\omega^*_1 = f_2 (1 - \omega^*_1)\,. 
 \end{align*}
Здесь $u_1({\cal W})$ и~$u_2'({\cal W})$~--- недопоставки соответственно первому и~второму 
пользователю.
Однако расположение~$D$ позволяет сделать предположение, что на границе 
множества~$\overline{{\cal X}}({\cal W})$ существуют точки, значения 
второй координаты которых больше, 
чем~$\overline x_2({\cal W})$.
Таким образом, недопоставки второму пользователю можно сократить, увеличив 
поток по второй стоковой дуге без уменьшения потока по первой. Увеличение 
потока для второго пользователя на рисунке изображено с~помощью  
вектора~$(\overrightarrow{DC})$.
Напомним, что в~рассматриваемой модели при исследовании поврежденной 
сети изучается соотношение между требованием каждого пользователя и~реальным 
потоком, проходящим по соответствующей стоковой дуге. Следовательно, 
требования различных пользователей и~объемы потоков по стоковым дугам 
являются невзаимозаменяемыми, а~суммарный поток в~стоковую вершину~$v_{N+1}$ 
во внимание не принимается.

Перепишем результат решения оптимизационной задачи в~векторной форме:
$$
 (\overrightarrow{OC}) = (\overrightarrow{OE}) + (\overrightarrow{EC}). 
 $$
Оптимальному распределению потоков на рисунке отвечает  точка~$C$ с~координатами
$(\overline x_1^*({\cal W}), \overline x_2^*({\cal W}))$:
\begin{align}
\overline x_1^*({\cal W}) &= f_1- u_1({\cal W}) = f_1 - f_1\omega^*_1 ={}\notag\\
&\hspace*{30mm}{}= f_1 
(1 - \omega^*_1)\,; \label{e12-mal}\\
\overline x_2^*({\cal W}) &=f_2 - u_2({\cal W}) = f_2 - f_2\omega^*_2 ={}\notag\\
&\hspace*{30mm}{}= f_2 (1 - \omega^*_2).\label{e13-mal}
\end{align}
Здесь $u_1({\cal W})$ и~$u_2({\cal W})$~--- минимальные недопоставки первому и~второму 
пользователям в~условиях их равноправности;
$\omega_1^*$~--- оптимальное решение задачи~$C_1$; 
$\omega_2^*$~--- оптимальное решение задачи~$C_2$. Из соотношений~(\ref{e12-mal})
и~(\ref{e13-mal}) 
следует:
\begin{equation}
  \fr{\overline x_1^*({\cal W})}{f_1}  =  1 - \omega^*_1\,; \quad
    \fr{\overline x_2^*({\cal W})}{f_2}  =  1 - \omega^*_2\,.
    \label{e14-mal}
    \end{equation}

В работе~\cite{Mal99} было введено понятие меры обеспеченности требований $k$-го 
потребителя после разрушающего воздействия как отношения величины потока, 
протекающего по $k$-й стоковой дуге после удара, к~запрошенной:
\begin{equation}
 \eta_k({\cal W}) = \fr{\overline x_k({\cal W})}{f_k}, \enskip 0 \le \eta_k({\cal W}) \le 1, \enskip 
 k = \overline{1, K}\,.  
 \label{e15-mal}
 \end{equation}
В правых частях равенств~(\ref{e14-mal}) и~левой части равенства~(\ref{e15-mal}) 
стоят взаимосвязанные величины. Рас\-смот\-рим связь между относительным ущербом 
пользователей~$\overrightarrow{\omega}^*$ и~мерой их обеспеченности требований. 

\section{Ущерб и~обеспеченность требований пользователей}

Согласно~\cite{Mal99} для определения величины наихудшей меры обеспеченности 
требований пользователей сети сразу после удара необходимо решить следующую 
задачу оптимизации.

\smallskip

\noindent
\textbf{Задача}  {\boldmath{$B$}}. Найти
$\max\limits_{x({\cal W}) \in {\cal X}({\cal W})} \min\limits_{k = \overline{1, K}} \eta_{k}$
при условии
$$
\eta_{k} = \fr{\overline x_{k}({\cal W})}{f_{k}}, \enskip
 0 \le \eta_k \le 1, \enskip k = \overline{1, K},  
 $$
или эквивалентную ей~\cite{Dan} следующую задачу линейного программирования.

\smallskip

\noindent
\textbf{Задача} {\boldmath{$B_1$}}.   Найти
$ \max\limits_{x({\cal W}) \in {\cal X}({\cal W}), \theta} \theta$
при условии
\begin{align*}
& \displaystyle \theta \le \fr{\overline x_{k}({\cal W})}{f_{k}}, \enskip 
 k = \overline{1, K}\,;\\
&0 \le \theta \le 1. 
                    \end{align*}
% \label{e16-mal}


Процесс построения оптимального решения последовательности задач $B_1$--$B_P$  
и~соответ\-ст\-ву\-ющие рассуждения полностью повторяют по\-стро\-ения и~рассуждения 
при решении задач $C_1$--$C_P$.

Результатом решения задач $B_1$--$B_{P}$ является вектор
$ \overrightarrow{\theta}^* \hm= \langle \theta^*_1, \theta^*_2, \ldots, 
\theta^*_P \rangle, $
который определяет гарантированную обеспеченность требований пользователей 
поврежденной сети и~эффективный вектор потоков~$\overline x^*({\cal W})$, лежащий на границе 
множества~$\overline{{\cal X}}({\cal W})$.

Рассмотрим связь между оптимальным вектором обеспеченности 
требований~$\overrightarrow{\theta}^*$  и~вектором ущерба пользователей~$\overrightarrow{\omega}^*$.

Перепишем первое условие задачи~$C_1$ в~(\ref{e7-mal}) следующим образом:
$$
\omega \geq \fr{f_k - \overline x_k({\cal W})}{f_k} = 1 - \fr{\overline x_k({\cal W})}{f_k} , \enskip
 k = \overline{1, K}\,. 
$$
Далее в~условиях~(\ref{e7-mal}) поменяем местами~$\omega$ 
и~${\overline x_k({\cal W})}/{f_k}$, а~также заменим знак:
\begin{equation}
\left.  \begin{array}{l}
\displaystyle 1 - \omega \le \fr{\overline x_k({\cal W})}{f_k}, \enskip 
k = \overline{1, K}, \\[6pt]
0 \le \omega \le 1.
                    \end{array}
 \right \} 
 \label{e17-mal}
 \end{equation}
Введем параметр $\delta = 1\hm - \omega$, для которого в~силу неравенства 
$0 \hm\le \omega \hm\le 1$ также верно
$ 0\hm \le \delta \hm\le 1. $
Поскольку
\begin{multline*}
\min\limits_{x({\cal W}) \in {\cal X}({\cal W})} \omega = 
\min\limits_{x({\cal W}) \in {\cal X}({\cal W})} (1 - \delta) = {}\\
{}=1 +  \min\limits_{x({\cal W}) \in {\cal X}({\cal W})} (-  \delta) = 1 - \max\limits_{x({\cal W}) 
\in {\cal X}({\cal W})} \delta, 
\end{multline*}
то при условии выполнения~(\ref{e17-mal}) решения задач~$B_1$ и~$C_1$ 
связаны соотношением:
$$
 \omega^* = 1 - \theta^*. 
 $$
Аналогичные рассуждения можно провести для пар задач~$B_2$ и~$C_2$, \ldots, 
$B_p$ и~$C_p$, \ldots, $B_P$ и~$C_P$. Таким образом, для соответствующих компонент 
векторов~$\overrightarrow{\theta}^*$  и~$\overrightarrow{\omega}^*$ выполняется:
$$
\omega_p^* = 1 - \theta_p^*,\enskip p = \overline{1, P}.
$$
Последнее означает, что при условии эффективного распределения 
потока в~поврежденной  сети ущерб $k$-го пользователя численно равен 
единице минус величина обеспеченности требований.

\section{Заключение}

В настоящее время в~литературе большое внимание уделяется изучению уязвимости 
территори\-ально распределенных систем. Указанные системы\linebreak
 чаще всего моделируются 
с~помощью однопро\-дуктовых сетей, а~анализ уязвимости сводится к~поиску узких 
мест таких сетей. Однако не менее \mbox{важной} представляется  проблема комплексного 
анализа ущерба пользователей, который они несут вследствие частичного разрушения 
системы или падения ее работоспособности.

Для решения данной проблемы в~работе пред\-ложена универсальная процедура, 
позволяющая рассматри\-вать широкий спектр разрушающих воздействий, изучать изменение 
функциональных характеристик сис\-те\-мы и~получать достоверные оценки ущерба 
пользователей. 

Описанная процедура может быть использована для анализа 
однопродуктовых сетей большого размера с~невзаимозаменяемыми требованиями 
на поток, например транспортных сетей или Интернета.

{\small\frenchspacing
 {%\baselineskip=10.8pt
 \addcontentsline{toc}{section}{References}
 \begin{thebibliography}{9}

\bibitem{Mur13} 
\Au{Murray A.\,T.} An overview of network vulnerability modeling approaches~// 
GeoJournal, 2013. Vol.~78. P.~209--221.

\bibitem{Mal17} 
\Au{Козлов М.\,В., Малашенко~Ю.\,Е., Назарова~И.\,А. и~др.} Управление  
топ\-лив\-но-энер\-ге\-ти\-че\-ской  системой  при  крупномасштабных повреждениях. 
I.~Сетевая  модель  и~программная реализация~// Изв. РАН. ТиСУ, 2017. №\,6. С.~50--73.

\bibitem{ford}  %3
\Au{Форд Л., Фалкерсон Д.} Потоки в~сетях~/ Пер. с~англ.~--- 
М.: Мир, 1966. 277~с. (\Au{Ford~L.\,R.,   Fulkerson~D.\,R.} Flows in networks.~--- 
Princeton, NJ, USA: Princeton University Press, 1962. 332~p.)

\bibitem{Pod} 
\Au{Подиновский В.\,В., Ногин~В.\,Д.} Па\-ре\-то-оп\-ти\-маль\-ные 
решения многокритериальных задач.~--- М.: Наука, 1982. 256~с.

\bibitem{Dan}  %5
\Au{Данциг Дж.} Линейное программирование, его применения и~обобщения~/ 
Пер. с~англ. -- М.: Прогресс, 1966. 589~c. (\Au{Dantzig~G.}  
Linear programming and extensions.~--- Princeton, NJ, USA: Princeton University Press, 1963. 
600~p.)



\bibitem{Yen}  %6
\Au{Йенсен П., Барнес~Д.} Потоковое программирование~/ Пер. с~англ.~--- 
М.: Радио и~связь, 1984. 392~с. (\Au{Jensen~P.\,A., Barnes~J.\,W. } 
Network flow programming.~--- New York, NY, USA: Wiley, 1980. 408~p.)

\bibitem{Mal99}  %7
\Au{Малашенко Ю.\,Е., Новикова~Н.\,М.} 
Модели неопределенности в~многопользовательских сетях.~--- 
М: Эдиториал УРСС, 1999. 160~с.
 \end{thebibliography}

 }
 }

\end{multicols}

\vspace*{-6pt}

\hfill{\small\textit{Поступила в~редакцию 31.05.17}}

\vspace*{8pt}

%\newpage

%\vspace*{-24pt}

\hrule

\vspace*{2pt}

\hrule

%\vspace*{8pt}


\def\tit{METHOD OF~ANALYSIS OF~FUNCTIONAL VULNERABILITY OF~FLOW NETWORK SYSTEMS}

\def\titkol{Method of~analysis of~functional vulnerability of~flow network systems}

\def\aut{Yu.\,E.~Malashenko, I.\,A.~Nazarova, and N.\,M.~Novikova}

\def\autkol{Yu.\,E.~Malashenko, I.\,A.~Nazarova, and N.\,M.~Novikova}

\titel{\tit}{\aut}{\autkol}{\titkol}

\vspace*{-9pt}


\noindent
A.\,A.~Dorodnicyn Computing Centre, Federal Research Center 
``Computer Science and Control'' of the Russian Academy of Sciences, 
40~Vavilov Str., Moscow 119333, Russian Federation

\def\leftfootline{\small{\textbf{\thepage}
\hfill INFORMATIKA I EE PRIMENENIYA~--- INFORMATICS AND
APPLICATIONS\ \ \ 2017\ \ \ volume~11\ \ \ issue\ 4}
}%
 \def\rightfootline{\small{INFORMATIKA I EE PRIMENENIYA~---
INFORMATICS AND APPLICATIONS\ \ \ 2017\ \ \ volume~11\ \ \ issue\ 4
\hfill \textbf{\thepage}}}

\vspace*{3pt}



\Abste{Mathematical methods of analysis of network systems for transfer of streams 
of interchangeable products to various users are considered. For description of 
processes occurring in the system, the model of single-product flow network is used. 
Changes of functional characteristics of a system after the targeted destroying
 effects are studied according to the methodology of an operations research. 
 The method of obtaining guaranteed damage estimates for each of the equal 
 users of the complete destruction of the physical and logical elements of 
 the system is proposed. The estimates are based on the solution of the 
 sequence of vector optimization problems with lexicographic minimax criterion. 
 To determine the strategy of flow distribution, \textit{a~posteriori} information about 
 changes of network's capacity is used.}

\KWE{single-product flow network; functional vulnerability; damage assessment; 
principle of the guaranteed result}

 \DOI{10.14357/19922264170406} 

%\vspace*{-12pt}

%\Ack
%\noindent



%\vspace*{3pt}

  \begin{multicols}{2}

\renewcommand{\bibname}{\protect\rmfamily References}
%\renewcommand{\bibname}{\large\protect\rm References}

{\small\frenchspacing
 {%\baselineskip=10.8pt
 \addcontentsline{toc}{section}{References}
 \begin{thebibliography}{9}
\bibitem{1-mal-1}
\Aue{Murray, A.\,T.} 2013. An overview of network vulnerability modeling approaches. 
\textit{GeoJournal} 78:209--221.

\bibitem{2-mal-1}
\Aue{Kozlov, M.\,V., Yu.\,E.~Malashenko, I.\,A.~Nazarova, \textit{et al.}} 
2017. Fuel and energy system control at large-scale damages. I.~Network model 
and software implementation. \textit{J.~Comput. Sys. Sci. Int.}
56(6):945--968. 

\bibitem{3-mal-1}
\Aue{Ford, L.\,R., and D.\,R.~Fulkerson.} 
1962. \textit{Flows in networks}. Princeton, NJ: Princeton University Press. 332~p.

\bibitem{4-mal-1}
\Aue{Podinovskiy, V.\,V., and V.\,D.~Nogin.} 
1982. \textit{Pareto-optimal'nye resheniya mnogokriterial'nykh zadach} 
[Pareto-optimal solutions of multicriteria tasks]. Moscow: Nauka. 256~p. 

\bibitem{5-mal-1}
\Aue{Dantzig, G.} 1963.  
\textit{Linear programming and extensions.} 
Princeton, NJ: Princeton University Press. 600~p.



\bibitem{7-mal-1}
\Aue{Jensen, P.\,A., and J.\,W.~Barnes.} 
1980. \textit{Network flow programming}. New York, NY: Wiley. 408~p. 

\bibitem{6-mal-1}
\Aue{Malashenko, Yu.\,E., and N.\,M.~Novikova.} 
1999. \textit{Modeli neopredelennosti v~mnogopol'zovatel'skikh setyakh} 
[Indeterminacy models in the multiuser networks].  Moscow: URSS Publ. 160~p.
\end{thebibliography}

 }
 }

\end{multicols}

\vspace*{-6pt}

\hfill{\small\textit{Received May 31, 2017}}

\vspace*{-18pt}

\Contr

\noindent
\textbf{Malashenko Yuri E.} (b.\ 1946)~--- 
Doctor of Science in physics and mathematics, Head of Laboratory, 
A.\,A.~Dorodnicyn Computing Centre, Federal Research Center 
``Computer Science and Control'' of the Russian Academy of Sciences, 
40~Vavilov Str., Moscow 119333, Russian Federation; \mbox{malash09@ccas.ru} 

\vspace*{3pt}

\noindent
\textbf{Nazarova Irina A.} (b.\ 1966)~--- 
Candidate of Science (PhD) in physics and mathematics, scientist,
 A.\,A.~Dorodnicyn Computing Centre, Federal Research Center 
``Computer Science and Control'' of the Russian Academy of Sciences, 
40~Vavilov Str., Moscow 119333, Russian Federation;
\mbox{irina-nazar@yandex.ru}

\vspace*{3pt}

\noindent
\textbf{Novikova Natalya M.} (b.\ 1953)~--- 
Doctor of Science in physics and mathematics,  professor, leading scientist, 
A.\,A.~Dorodnicyn Computing Centre, Federal Research Center 
``Computer Science and Control'' of the Russian Academy of Sciences, 
40~Vavilov Str., Moscow 119333, Russian Federation;
\mbox{n\_novikova@umail.ru}


\label{end\stat}


\renewcommand{\bibname}{\protect\rm Литература} 