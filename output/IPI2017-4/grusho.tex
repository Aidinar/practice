\def\stat{grusho}

\def\tit{МОДЕЛЬ МНОЖЕСТВА ИНФОРМАЦИОННЫХ ПРОСТРАНСТВ В~ЗАДАЧЕ 
ПОИСКА ИНСАЙДЕРА$^*$}

\def\titkol{Модель множества информационных пространств в~задаче 
поиска инсайдера}

\def\aut{А.\,А.~Грушо$^1$, М.\,И.~Забежайло$^2$, Д.\,В.~Смирнов$^3$, 
Е.\,Е.~Тимонина$^4$}

\def\autkol{А.\,А.~Грушо, М.\,И.~Забежайло, Д.\,В.~Смирнов, 
Е.\,Е.~Тимонина}

\titel{\tit}{\aut}{\autkol}{\titkol}

\index{Грушо А.\,А.}
\index{Забежайло М.\,И.}
\index{Смирнов Д.\,В.}
\index{Тимонина Е.\,Е.}
\index{Grusho A.\,A.}
\index{Zabezhailo M.\,I.}
\index{Smirnov D.\,V.} 
\index{Timonina E.\,E.}



{\renewcommand{\thefootnote}{\fnsymbol{footnote}} \footnotetext[1]
{Работа поддержана РФФИ (проект 15-07-02053).}}


\renewcommand{\thefootnote}{\arabic{footnote}}
\footnotetext[1]{Институт проблем информатики Федерального исследовательского центра 
<<Информатика и~управ\-ле\-ние>> 
Российской академии наук, \mbox{grusho@yandex.ru}}
\footnotetext[2]{Институт проблем информатики Федерального исследовательского центра 
<<Информатика и~управ\-ле\-ние>> Российской академии наук, \mbox{m.zabezhailo@yandex.ru}}
\footnotetext[3]{ПАО Сбербанк России, \mbox{dvlsmirnov@sberbank.ru}}
\footnotetext[4]{Институт проблем информатики Федерального исследовательского центра 
<<Информатика и~управ\-ле\-ние>> 
Российской академии наук, \mbox{eltimon@yandex.ru}}


%\vspace*{-18pt}


  \Abst{В задаче поиска инсайдеров разработан подход к~объединению компрометирующих 
данных, наблюдаемых в~разных информационных пространствах. Накопление информации 
в~каждом пространстве рассматривается как вероятностный процесс. Рассматриваемый 
подход основан на запретах и~полузапретах вероятностных мер в~разных информационных 
пространствах. С~последовательностями событий, наблюдаемых в~этих информационных 
пространствах, связываются булевы переменные. Появление полузапретов соответствует 
значению~<<1>> соответствующих булевых переменных. 
  Последовательности булевых переменных в~разных информационных пространствах 
легко связываются с~помощью логических выражений. Эти выражения описывают опасные 
тенденции, наблюдаемые в~различных информационных пространствах.}

\KW{информационная безопасность; инсайдер; запреты и~полузапреты вероятностных мер; 
архитектура объединения информационных источников}

\DOI{10.14357/19922264170408} 


\vskip 10pt plus 9pt minus 6pt

\thispagestyle{headings}

\begin{multicols}{2}

\label{st\stat}

\section{Введение}

  Разработка методов выявления инсайдеров в~организации является 
актуальной задачей. Решению этой задачи посвящен ряд исследований~[1--3]. 
Эти исследования основаны на использовании различных статистических 
методов выявления аномалий в~деятельности сотрудников организации. 
Очевидно, что хорошая идентификация инсайдера основана на объединенной 
картине незначительных аномалий. При этом всегда используется информация, 
касающаяся разных аспектов деятельности потенциальных инсайдеров. Анализ 
различных типов информации и~формирование выводов из нее~--- сложная 
задача. 
  
  В работе предлагается подход к~решению этой задачи, основанный на 
запретах вероятностных мер~[4--6]. Основная идея работы состоит 
в~сле\-ду\-ющем. Рассматривается конечное множество информационных 
пространств, в~которых ищутся признаки, связанные с~целями деятельности 
инсайдера. Накопление информации в~каждом пространстве рассматривается 
как вероятностный процесс. Появление искомых признаков интерпретируется 
как запрет вероятностной меры деятельности неинсайдера. Возникновение 
запретов в~разных информационных пространствах можно связать 
дизъюнктивной формой в~единое целое. Тогда значение~<<1>> этой суммарной 
дизъюнктивной формы определяет инсайдера. 
  
\section{Общая математическая модель множества информационных 
пространств}

  Целью построения модели является внесение в~алгоритм поиска инсайдера 
самой различной информации, в~которой могут появиться признаки 
деятельности инсайдера. Необходимо отметить, что привлекаемая информация 
может содержать психологический поведенческий фон анализируемого лица, 
который усиливает малозначимые аномальные действия потенциального 
инсайдера. 
  
  Пусть $U$~--- контролируемый потенциальный инсайдер, $T\hm= \{t_1, 
t_2,\ldots , t_N\}$~--- цели инсайдерских атак. В~ходе анализа информации 
об~$U$ могут рассматриваться подмножества~$T$. Аномалии в~общем фоне 
поведения и~в~характере~$U$ описываются всем множеством~$T$. 
В~анализе~$U$ используется~$m$~информационных пространств. Элементы 
каждого информационного пространства описываются своим языком, и~пусть 
$E_1, E_2,\ldots, E_m$~--- алфавиты этих языков, а~$E_1^*, E_2^*, \ldots , 
E_m^*$~--- множества слов конечной длины в~этих алфавитах. Языки описания 
информации~$L_1, L_2,\ldots , L_m$ в~информационных пространствах 
удовлетворяют условию $L_i\hm\subseteq E_i^*$,\linebreak $i\hm=1,\ldots , m$. 
  
  Введем понятие процесса наблюдения в~информационном пространстве. 
Пусть время является дискретным и~описывается множеством натуральных 
чисел. Наблюдение за~$U$ в~$i$-м информацион\-ном пространстве к~моменту 
времени~$n$ пред\-став\-ля\-ет собой конечное множество слов в~языке~$L_i$.\linebreak 
В~силу естественных технических ограничений в~каж\-дом языке~$L_i$ 
выделяется конечное множество слов~$X_{i,n}$, которое представляет интерес 
при наблюдении за~$U$ к~моменту времени~$n$. Обозначим через 
$\Sigma(X_{i,n})\hm= \{ \sigma: \sigma\subseteq X_{i,n}\}$ множество всех 
подмножеств~$X_{i,n}$, а~через~$\Sigma_j (X_{i,j})\hm\subseteq 
\Sigma(X_{i,n})$, $n\hm\geq j$, накопленные данные к~моменту времени~$j$. 
Очевидно, что $\Sigma_1(X_{i,1})\hm\subseteq \Sigma_2(X_{i,2})\subseteq 
\cdots$. Пусть $V_n(X_{i,n})\hm= \Sigma_n(X_{i,n})\backslash  
\Sigma_{n-1}(X_{i,n-1})$~--- дополнительная информация, полученная об~$U$ 
в~момент времени~$n$. 
  
  Последовательность $\{ V_n(X_{i,n})\}_{n\geq 1}$ можно рассматривать как 
случайный процесс. Для корректного определения случайного процесса 
необходимо определить $\sigma$-ал\-геб\-ру на последовательностях 
$\{V_n(X_{i,n})\}_{n\geq 1}$. Пусть $\sigma$-ал\-геб\-ра~$\mathcal{A}_i$ 
определяется как минимальная $\sigma$-ал\-геб\-ра, порожденная 
цилиндрическими множествами. Эта алгебра также является борелевской  
$\sigma$-ал\-геб\-рой в~тихоновском произведении дискретных топологических 
пространств в~пространстве всех бесконечных 
последовательностей~$V_i^\infty$, которые могут быть траекториями 
рассматриваемого случайного процесса~\cite{7-gr, 8-gr}.
  
  Конечное множество измеримых пространств $(V_i^\infty, \mathcal{A}_i)$, 
$i\hm= 1,\ldots ,m$, порождает измеримое пространство $(V^\infty, \mathcal{A})$, 
в~котором $V^\infty\hm= \prod\nolimits_{i=1}^m V_i^\infty$,\linebreak  
а~$\sigma$-ал\-геб\-ра~$\mathcal{A}$ порождена  
$\sigma$-ал\-геб\-ра\-ми~$\mathcal{A}_i$, $i\hm= 1,\ldots , m$, на 
пространстве~$V^\infty$. 
  
  Пусть поведение неинсайдера в~пространстве $(V^\infty, \mathcal{A})$ 
описывается вероятностной мерой~$P$. Как и~в~работах~\cite{4-gr, 9-gr}, для 
меры~$P$ можно определить запреты этой вероятностной меры. Каждый 
запрет можно рассматривать как идентификацию инсайдера с~полным 
описанием предыстории. 
  
\section{Модель полузапретов}

  Получившаяся модель является общей, но сложной для практического 
применения. Возможны различные способы ее упрощения. Рассмотрим один из 
таких способов.
  
  Рассмотрим в~отдельности каждое измеримое пространство $(V_i^\infty, 
\mathcal{A}_i)$, $i\hm= 1,\ldots , m$. Пусть~$P^{(i)}$~--- проекция меры~$P$ 
на пространство $(V_i^\infty, \mathcal{A}_i)$. Распределение 
вероятностей~$P^{(i)}$ характеризует поведе\-ние неинсайдера 
в~соответствующем информационном пространстве. Для каждого 
информационного пространства введем упрощенную\linebreak
 модель. 
Пусть~$Q^{(i)}$~--- некоторое распределение на пространстве $(V_i^\infty, 
\mathcal{A}_i)$ и~пусть~$P^{(i)}$ получается из~$Q^{(i)}$ введением 
некоторого числа запретов~\cite{10-gr}. Эти запреты можно интерпретировать 
либо как признаки некоторого подозрительного поведения, либо как прямые 
указания на некоторые цели атак инсайдера из множества~$T$. Это означает, 
что можно рассматривать траектории наблюдаемого процесса 
в~информационном пространстве, несмотря на то что появляются запреты 
меры~$P^{(i)}$. Будем называть эти запреты меры~$P^{(i)}$ в~мере~$Q^{(i)}$ 
\textit{полузапретами}. 
  
  Определим множество бинарных переменных~$x_{i,j}$, 
$j\hm=\overline{1;\infty}$, следующим образом. Пусть $\forall\ n\ x_{i,n}\hm=1$, 
когда в~момент времени~$n$ появляется полузапрет меры~$P^{(i)}$. 
В~противном случае $x_{i,n}\hm=0$. Тогда появление полузапретов к~моменту 
времени~$n$ характеризуется дизъюнкцией $\vee^n_{\{j=i\}} x_{i,j}$. Отсюда 
возникает возможность конструктивного определения запрета в~совокупности 
пространств на основе полузапретов в~отдельных пространствах. Например, 
запрет можно определить как дизъюнкцию конъюнкций следующего вида:
  $$
  \vee_{\{1\leq i<l\leq m\}} \left( \left( \vee^n_{\{j=1\}} x_{i,j}\right) \wedge 
\left( \vee^n_{\{j=1\}} x_{l,j}\right)\right)\,.
  $$
  
  Эта формула означает, что запрет (идентификация инсайдера) возникает при 
появлении не менее двух полузапретов в~рассматриваемом множестве 
информационных пространств. 
  
  Ясно, что можно варьировать способ принятия решения об идентификации 
инсайдера, а~также рассматривать уточняющие цепочки таких решений. 
  
\section{Связь целей атак инсайдера с~запретами}

  Определение запретов основано на двух подходах. В~первом подходе каждая 
цель из множества~$T$ допускает возможность построения дерева атак (см., 
например,~\cite{11-gr}). Построение деревьев атак давно используется 
в~анализе уязвимостей. Путь к~корню в~дереве атак определяет 
последовательность шагов в~каждой атаке. Кроме того, этот путь определяет 
цепочку событий, возможно в~нескольких информационных пространствах. 
Эти события в~отдельных информационных пространствах характеризуются 
двумя способами:
  \begin{enumerate}[(1)]
\item прямыми или косвенными признаками прохождения пути в~дереве 
атак, которые могут отслеживаться функциями мониторинга без\-опас\-ности;
\item сопутствующими фоновыми событиями, усиливающими или 
смягчающими возможности реализаций событий без\-опас\-ности.
\end{enumerate}

  С каждой вершиной или с~каждым ребром дерева атак можно связать одно 
или несколько информационных пространств, в~которых содержатся описания 
признаков атак и/или фона. Эти описания суть полузапреты. Таким образом 
определяется привязка целей атак к~наблюдаемым данным в~разных 
информационных пространствах.
  
  Второй подход связан с~выявлением аномалий в~наблюдаемых случайных 
процессах в~разных информационных пространствах~\cite{12-gr}. Эти 
аномалии анализируются средствами интеллектуального анализа данных на 
предмет выявления эмпирических закономерностей и~эмпирических 
при\-чин\-но-след\-ст\-вен\-ных связей~\cite{13-gr, 14-gr}. Этот подход позволяет построить 
профили поведения~$U$, из них вывести новые пути атак и~новые цели атак 
инсайдера. 
  
\section{Примеры}

  Приведем примеры информационных пространств и~полузапретов в~них. 
  
  \noindent
  \textbf{Пример~1.} Пусть имеется описание детерминированной 
последовательности функций, вы\-пол\-ня\-емых~$U$. Полузапретами могут 
считаться отклонения от детерминированной последовательности функций 
в~сторону доступа к~ценной информации, не связанной с~функциями~$U$. 
Цель инсайдера~--- кража ценной информации. Этот случай рас\-смат\-ри\-вал\-ся 
в~работе~\cite{6-gr}.
  
  \smallskip
  
  \noindent
  \textbf{Пример~2.} Рассмотрим связи~$U$ с~другими пользователями 
в~социальных сетях. Полузапрет определяется наличием связанных с~$U$ 
пользователей, имевших или имеющих отношение к~криминалу. Эти 
полузапреты имеют значение фона. Однако они же могут указывать на цели 
кражи ценной информации. 
  
  \smallskip
  
  \noindent
  \textbf{Пример~3.} Рассмотрим взаимодействие~$U$ с~дружественными 
корреспондентами в~социальных сетях. Из этих данных можно получить 
информацию о~финансовых трудностях, испытываемых~$U$. Эта информация 
может считаться полузапретом в~негативном фоне, побуждающим~$U$ 
к~совершению противоправных действий. 
  
  \smallskip
  
  \noindent
  \textbf{Пример~4.} Полузапретом с~негативным фоном можно считать 
внезапный интерес к~дорогим материальным объектам (дорогие автомобили, 
дачи, квартиры и~т.\,п.). Этот полузапрет определяется появлением аномалий 
в~профиле~$U$. Если~$U$ связан с~доступом к~ценной информации, то целью 
его атаки можно считать кражу ценной информации, которую с~указанным 
полузапретом можно считать запретом. 
  
\section{Заключение}

  В задаче поиска инсайдеров разработан подход к~объединению 
компрометирующих данных,\linebreak наблю\-да\-емых в~разных информационных 
про\-стран\-ст\-вах. Этот подход основан на запретах и~полуза\-пре\-тах вероятностных 
мер в~различных\linebreak 
информационных пространствах. С~последовательностями 
событий, наблюдаемыми в~этих информационных пространствах, связываются 
булевы переменные. Появление полузапретов соответствует значению~<<1>> 
соответствующих булевых переменных. 
  
  Последовательности булевых переменных в~разных информационных 
пространствах легко связываются с~помощью логических выражений. Эти 
выражения описывают опасные тенденции, наблюдаемые в~различных 
информационных пространствах. Выбор логических выражений (функций) 
может быть построен на использовании всей предысто\-рии наблюдений за 
потенциальным инсайдером в~соответствующих информационных 
пространствах или при анализе его подозрительной деятельности в~заданный 
промежуток времени. При этом использование различных булевых функций 
позволяет проводить различные виды анализа на одних и~тех же статистических 
данных. 
  
  С учетом различных целей атак инсайдера в~дальнейшем можно перейти 
к~рассмотрению \mbox{$k$-знач}\-ных функций.
  
  Глубокие связи в~каждом информационном пространстве можно исследовать с~помощью методов интеллектуального анализа данных.
  
{\small\frenchspacing
 {%\baselineskip=10.8pt
 \addcontentsline{toc}{section}{References}
 \begin{thebibliography}{99}
     \bibitem{1-gr}
     Anomaly Detection at Multiple Scales (ADAMS).~--- General Services Administration,  
22.10.2010. 
{\sf   
https:// www.fbo.gov/download/2f6/2f6289e99a0c04942bbd89 ccf242fb4c/DARPA-BAA-11-04\_ADAMS.pdf}.
     \bibitem{2-gr}
     \Au{Yu R., He X., Liu~Y.} 
     GLAD: Group anomaly detection in social media analysis, 1940.
 {\sf \mbox{arXiv}:1410.1940}.
     \bibitem{3-gr}
     \Au{Senator T., Goldberg H.\,G., Memory~A., \textit{et al.}} Detecting insider threats in a~real corporate 
database of computer usage activity~// 19th ACM SIGKDD Conference (International) on 
Knowledge Discovery and Data Mining Proceedings.~--- New York, NY, USA: ACM, 2013. 
P.~1393--1401.
\bibitem{4-gr}
\Au{Grusho~A., N.~Grusho, and E.~Timonina.} Quality of tests defined by bans~// 16th Applied 
Stochastic Models and Data Analysis Conference (International) Proceedings.~--- 
Piraeus, Greece: ISAST, 2015. P.~289--295.
     \bibitem{5-gr}
     \Au{Grusho~A., Grusho~N., Timonina~E.} Modelling for ensuring information security of 
the distributed information systems~// 31th European Conference on Modelling and Simulation 
 Proceedings.~--- Germany: Digitaldruck Pirrot GmbHP Dudweiler, 2017.  
P.~656--660.
     \bibitem{6-gr}
     \Au{Мартьянов Е.\,А.} Возможность выявления инсайдера статистическими 
методами~// Системы и~средства информатики, 2017. Т.~27. №\,2. С.~41--47.
     \bibitem{7-gr}
     \Au{Бурбаки Н.} Общая топология. Основные структуры~/ Пер. с~фр.~--- М.: Наука, 
1968. 272~с. (\Au{Bourbaki~N.} Topologie g$\acute{\mbox{e}}$n$\acute{\mbox{e}}$rale. Ch.~1: 
Structures topologiques. Ch.~2: Structures uniformes.~--- Paris: Hermann, 1940. 129~p.)
     \bibitem{8-gr}
     \Au{Прохоров Ю.\,В., Розанов~Ю.\,А.} Теория вероятностей.~--- М.: Наука, 1993. 496~c.
     \bibitem{9-gr}
\Aue{Grusho~A., Timonina~E.}  Consistent sequences of tests defined by bans~// 
\textit{Optimization theory, decision making, and operations research applications}~/
Eds.\linebreak A.~Mig\-da\-las, A.~Sifaleras, C.\,K.~Georgiadis,
\textit{et al.}~--- Springer 
proceedings in mathematics \& statistics ser. ~---  
New York\,--\,Heidelberg\,--\,Dordrecht\,--\,London: Springer-Verlag,
2013.  Vol.~31. P.~281--291. 
     
     
     \bibitem{10-gr}
     \Au{Грушо~А.\,А., Грушо~Н.\,А., Тимонина~Е.\,Е.} Включение новых запретов 
в~случайные последовательности~// Информатика и~её применения, 2014. Т.~8. Вып.~4. 
С.~48--54. 
     \bibitem{11-gr}
     \Au{Sheyner O., Haines~J., Jha~S., Lippmann~R., Wing~J.\,M.} Automated generation and 
analysis of attack graphs~// IEEE Symposium on Security and Privacy Proceedings.~--- 
Piscataway, NJ, USA: IEEE, 2002. P.~273--284. 
     \bibitem{12-gr}
     \Au{Grusho A., Grusho~N., Timonina~E.} Detection of anomalies in non-numerical data~// 
8th Congress (International) on Ultra Modern Telecommunications and Control Systems and 
Workshops Proceedings.~--- Piscataway, NJ, USA: IEEE, 2016. P.~273--276.
     
     \bibitem{14-gr}
     \Au{Грушо А.\,А., Грушо~Н.\,А., Забежайло~М.\,И., Тимонина~Е.\,Е.} Интеграция 
статистических и~детерминистских методов анализа информационной без\-опас\-ности~// 
Информатика и~её применения, 2016. Т.~10. Вып.~3. С.~19--25.

\bibitem{13-gr}
     \Au{Grusho~A.} Data mining and information security~// 
     Computer network security~/
     Eds.\ J.~Rak, J.~Bay, I.~Kotenko, \textit{et al.}~---
     Lecture notes in computer science ser.~--- Springer, 
2017. Vol.~10446. P.~28--33. 

 \end{thebibliography}

 }
 }

\end{multicols}

\vspace*{-3pt}

\hfill{\small\textit{Поступила в~редакцию 17.10.17}}

\vspace*{8pt}

%\newpage

%\vspace*{-24pt}

\hrule

\vspace*{2pt}

\hrule

%\vspace*{8pt}


\def\tit{THE MODEL OF~THE~SET OF~INFORMATION SPACES IN~THE~PROBLEM 
OF~INSIDER DETECTION}

\def\titkol{The model of~the~set of~information spaces in~the~problem 
of~insider detection}

\def\aut{A.\,A.~Grusho$^1$, M.\,I.~Zabezhailo$^1$, D.\,V.~Smirnov$^2$, 
and~E.\,E.~Timonina$^1$}

\def\autkol{A.\,A.~Grusho, M.\,I.~Zabezhailo, D.\,V.~Smirnov, 
and~E.\,E.~Timonina}

\titel{\tit}{\aut}{\autkol}{\titkol}

\vspace*{-9pt}


\noindent
$^1$Institute of Informatics Problems, Federal Research Center ``Computer Sciences and Control'' of 
the Russian\linebreak
$\hphantom{^1}$Academy of Sciences, 44-2~Vavilov Str., Moscow 119333, Russian Federation
 

\noindent
$^2$Sberbank of Russia, 19~Vavilov Str., Moscow 117999, Russian Federation 



\def\leftfootline{\small{\textbf{\thepage}
\hfill INFORMATIKA I EE PRIMENENIYA~--- INFORMATICS AND
APPLICATIONS\ \ \ 2017\ \ \ volume~11\ \ \ issue\ 4}
}%
 \def\rightfootline{\small{INFORMATIKA I EE PRIMENENIYA~---
INFORMATICS AND APPLICATIONS\ \ \ 2017\ \ \ volume~11\ \ \ issue\ 4
\hfill \textbf{\thepage}}}

\vspace*{3pt}
     

  \Abste{In the problem of insider detection,
   the approach to combining compromising data 
observed in different information spaces is developed. Accumulation of information in each space is 
considered as a random process. The considered approach is based on bans and semibans of 
probability measures in different information spaces. Boolean variables communicate with the help 
of sequences of events observed in the information spaces. Appearance of semibans corresponds to 
value~``1'' of the appropriate Boolean variables. Sequences of Boolean variables in different 
information spaces easily communicate by means of logical expressions. The expressions describe 
dangerous tendencies observed in different information spaces.}
  
  \KWE{information security; insider; bans and semibans of probability measures; architecture of 
combining information sources}
  
  
     
\DOI{10.14357/19922264170408} 

%\vspace*{-12pt}

\Ack
\noindent
The paper was supported by the Russian Foundation for Basic Research (project 15-07-02053).

\pagebreak



%\vspace*{3pt}

  \begin{multicols}{2}

\renewcommand{\bibname}{\protect\rmfamily References}
%\renewcommand{\bibname}{\large\protect\rm References}

{\small\frenchspacing
 {%\baselineskip=10.8pt
 \addcontentsline{toc}{section}{References}
 \begin{thebibliography}{99}
     \bibitem{1-gr-1}
     General Services Administration.  
22.10.2010. 
     Anomaly Detection at Multiple Scales (ADAMS).  Available at: 
{\sf   
https://www.fbo.gov/download/2f6/2f6289e99a0c0494 2bbd89ccf242fb4c/DARPA-BAA-11-04\_ADAMS.pdf}
(accessed May~12, 2011).
     \bibitem{2-gr-1}
     \Aue{Yu, R., X.~He, and Y.~Liu.} 1940. GLAD: Group anomaly detection in social media 
analysis. Available at: {\sf \mbox{arXiv}:1410.1940} (accessed October~7, 2014).
     \bibitem{3-gr-1}
     \Aue{Senator, T.,  H.\,G.~Goldberg, A.~Memory,  \textit{et al.}} 2013. Detecting insider threats in a~real corporate 
database of computer usage activity. \textit{19th ACM SIGKDD Conference (International) on 
Knowledge Discovery and Data Mining Proceedings}. New York, NY: ACM. 1393--1401.
\bibitem{4-gr-1}
\Aue{Grusho,~A., N.~Grusho, and E.~Timonina.} 2015.
Quality of tests defined by bans. \textit{16th Applied Stochastic Models and Data Analysis 
Conference (International) Proceedings}.~--- Piraeus, Greece: ISAST, 2015. P.~289--295. 
     \bibitem{5-gr-1}
     \Aue{Grusho,~A., N.~Grusho, and E.~Timonina.} 2017. Modelling for ensuring information 
security of the distributed information systems. \textit{31th European Conference on Modelling and 
Simulation Proceedings }.  Germany: Digitaldruck Pirrot GmbHP Dudweiler. 656--660.
     \bibitem{6-gr-1}
     \Aue{Martyanov, E.\,A.} 2017. Vozmozhnost' vyyavleniya insaydera statisticheskimi 
metodami [Possibility of insider detection by statistical techniques]. \textit{Sistemy i~Sredstva 
Informatiki~--- Systems and Means of Informatics} 27(2):41--47.
     \bibitem{7-gr-1}
     \Aue{Bourbaki, N.} 1940. \textit{Topologie g$\acute{\mbox{e}}$n$\acute{\mbox{e}}$rale}. 
Ch.~1: Structures topologiques. Ch.~2: Structures uniformes. Paris: Hermann. 129~p.
     \bibitem{8-gr-1}
     \Aue{Prokhorov, Yu.\,V., and Yu.\,A.~Rozanov.} 1993. \textit{Teoriya veroyatnostey} [Theory 
of probabilities]. Moscow: Nauka. 496~p. 
     \bibitem{9-gr-1}
     \Aue{Grusho, A.,  and E.~Timonina.} 2013. Consistent sequences of tests defined by 
     bans.  \textit{Optimization theory, decision  making, and operations research applications}.  
     Eds. A.~Migdalas, A.~Sifaleras, C.\,K.~Gorgiadis,
     \textit{et al.} Springer proceedings in mathematics \& statistics ser.    
New York\,--\,Heidelberg\,--\,Dordrecht\,--\,London: Springer-Verlag.  31:281--291.
     \bibitem{10-gr-1}
     \Aue{Grusho,~A., N.~Grusho, and E.~Timonina.} 2014. Vklyu\-che\-nie novykh zapretov 
v~sluchaynye posledovatel'nosti [Switching on of new bans in random sequences]. 
\textit{Informatika i~ee Primeneniya~--- Inform. Appl.} 8(4):48--54.
     \bibitem{11-gr-1}
     \Aue{Sheyner, S., J.~Haines, S.~Jha, R.~Lippmann, and J.\,M.~Wing.} 2002. Automated 
generation and analysis of attack graphs. \textit{IEEE Symposium on Security and Privacy 
Proceedings}.  273--284.
     \bibitem{12-gr-1}
     \Aue{Grusho,~A., N.~Grusho, and E.~Ti\-mo\-ni\-na.} 2016. Detection of anomalies in  
non-numerical data. \textit{8th Congress (International) on Ultra Modern Telecommunications and 
Control Systems and Workshops Proceedings}.~---  Piscataway, NJ: IEEE.  273--276.
     
     \bibitem{14-gr-1}
     \Aue{Grusho, A.\,A., N.\,A.~Grusho, M.\,I.~Zabezhailo, and E.\,E.~Timonina.} 2016. 
Integratsiya statisticheskikh i~deterministskikh metodov analiza informatsionnoy bezopasnosti 
[Integration of statistical and deterministic methods of information security analysis]. 
\textit{Informatika i~ee Primeneniya~--- Inform. Appl.} 10(3):19--25.

\bibitem{13-gr-1}
     \Aue{Grusho,~A.} 2017. Data mining and information security. 
     \textit{Computer network security}. Eds. J.~Rak, J.~Bay, I.~Kotenko,
     \textit{et al.}
{Lecture notes in 
computer science ser.} Springer. 10446:28--33. 

\end{thebibliography}

 }
 }

\end{multicols}

\vspace*{-6pt}

\hfill{\small\textit{Received October 17, 2017}}

%\vspace*{-10pt}

\Contr


\noindent
\textbf{Grusho Alexander A.} (b.\ 1946)~--- 
Doctor of Science in physics and mathematics, professor,  Head of Laboratory, 
Institute of Informatics Problems, Federal Research Center ``Computer 
Sciences and Control'' of the Russian Academy of Sciences, 44-2~Vavilov Str., Moscow 
119333, Russian Federation; \mbox{grusho@yandex.ru}

\vspace*{3pt}

\noindent
\textbf{Zabezhailo Michael I.} (b.\ 1956)~--- 
Candidate of Science (PhD) in physics and mathematics,  Head of Laboratory, 
Institute of Informatics Problems, Federal Research Center ``Computer 
Sciences and Control'' of the Russian Academy of Sciences, 44-2~Vavilov Str., Moscow 
119333, Russian Federation; \mbox{m.zabezhailo@yandex.ru}

\vspace*{3pt}

\noindent
\textbf{Smirnov Dmitry V.} (b.\ 1984)~--- 
business partner for IT security department, Sberbank of 
Russia, 19~Vavilov Str., Moscow 117999, Russian Federation; \mbox{dvlsmirnov@sberbank.ru}

\vspace*{3pt}

\noindent
\textbf{Timonina Elena E.} (b.\ 1952)~--- 
Doctor of Science in technology, professor, leading scientist, 
Institute of Informatics Problems, Federal Research Center ``Computer 
Sciences and Control'' of the Russian Academy of Sciences, 44-2~Vavilov Str., Moscow 
119333, Russian Federation; \mbox{eltimon@yandex.ru}
     

\label{end\stat}


\renewcommand{\bibname}{\protect\rm Литература} 