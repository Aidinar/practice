\def\stat{naumov}

\def\tit{АНАЛИЗ ВРЕМЕННЫХ ХАРАКТЕРИСТИК ПРОЦЕССА ПЕРЕДАЧИ ДАННЫХ 
ПОДВИЖНЫМ ПОЛЬЗОВАТЕЛЯМ В~СОТЕ СЕТИ~LTE$^*$}

\def\titkol{Анализ временных характеристик процесса передачи данных 
подвижным пользователям в~соте сети~LTE}

\def\aut{В.\,А.~Наумов$^1$, Е.\,В.~Мокров$^2$, К.\,Е.~Самуйлов$^3$}

\def\autkol{В.\,А.~Наумов, Е.\,В.~Мокров, К.\,Е.~Самуйлов}

\titel{\tit}{\aut}{\autkol}{\titkol}

\index{Наумов В.\,А.}
\index{Мокров Е.\,В.}
\index{Самуйлов К.\,Е.}
\index{Naumov V.\,A.}
\index{Mokrov E.\,V.} 
\index{Samouylov K.\,E.}



{\renewcommand{\thefootnote}{\fnsymbol{footnote}} \footnotetext[1]
{Публикация подготовлена при финансовой поддержке Минобрнауки России (проект 2.882.2017/4.6).}}


\renewcommand{\thefootnote}{\arabic{footnote}}
\footnotetext[1]{Исследовательский институт инноваций, 
г.~Хельсинки, Финляндия, \mbox{valeriy.naumov@pfu.fi}}
\footnotetext[2]{Российский университет дружбы народов, \mbox{mokrov\_ev@rudn.university}}
\footnotetext[3]{Российский университет дружбы народов; Институт проб\-лем информатики Федерального 
исследовательского центра <<Информатика и~управ\-ле\-ние>> Российской академии наук, 
\mbox{samuylov\_ke@rudn.university}}

%\vspace*{-18pt}

 

\Abst{Целью исследования является анализ среднего времени передачи данных в~сети 
мультивещания с~учетом перемещения пользователей в~пределах соты сети LTE
(long-term evolution), разбитой на 
несколько зон с~различным уровнем качества обслуживания. Проводится анализ данной 
характеристики в~зависимости от объема передаваемых данных и~от числа пользователей 
в~соте. Рассмотрены три различные модели движения пользователей в~пределах соты. Для 
вычисления среднего времени передачи данных используется аппарат марковских 
процессов.}
 
\KW{мультивещание; гетерогенная беспроводная сеть; показатели качества обслуживания; 
управляющий марковский процесс; модели движения; среднее время передачи; двойственный 
процесс; сеть массового обслуживания}

\DOI{10.14357/19922264170410} 


\vskip 10pt plus 9pt minus 6pt

\thispagestyle{headings}

\begin{multicols}{2}

\label{st\stat}

\section{Введение}

  В данной работе рассматривается модель качества передачи данных 
подвижным пользователям в~сети мобильной связи. Основное внимание при 
этом уделяется динамике движения пользователей внутри соты мобильной 
связи. Сота разделена\linebreak на несколько зон, отличающихся качеством об\-служивания 
(channel quality indicator, CQI), между которы\-ми перемещаются пользователи, 
при этом рассмотрены три различные модели движения пользователей между 
областями CQI. Получено среднее время передачи данных пользователям соты 
в~зависимости от объема передаваемых данных и~от числа пользователей 
в~пределах соты.
  
  Данная задача решена с~помощью жидкостного процесса накопления, 
случайного процесса, различные варианты которого широко используются для 
моделирования производственных процессов и~процессов передачи 
информации~[1]. Однако большинство имеющихся результатов относятся 
к~случаю, когда управляющий скоростью накопления процесс является 
марковским. В~данном же случае, как будет показано далее, процесс, при 
наличии в~системе более одного пользователя, таковым не является. Случай 
системы с~одним пользователем изучен довольно хорошо~[2, 3], но даже этот 
простой случай описывается довольно сложным процессом. Однако в~данной 
ситуации представляет интерес случай с~несколькими пользователями, когда 
система описывается несколькими процессами рождения и~гибели.

\vspace*{-9pt}
  
\section{Описание модели}

\vspace*{-2pt}

  Рассмотрим отдельную соту мобильной связи, разбитую на~$M$ областей 
CQI с~различным уровнем качества обслуживания, по которой 
перемещаются~$N$ пользователей, причем в~момент времени~$t$ в~$m$-м CQI 
находится ${\cal{J}}_m(t)$ абонентов. Пользователи независимо перемещаются по 
соте согласно одной и~той же модели движения, не покидая пределов соты. 
В~соте ведется передача данных некоторого объема по мультивещанию всем 
абонентам, находящимся в~соте. Примером такой передачи может служить 
обновление программного обеспечения. Объем данных, переданный в~системе за 
время~$t$, описывается формулой:

\noindent
  \begin{equation*}
  F(t)=\int\limits_0^t C(u)\,du\,.
  %\label{e1-na}
  \end{equation*}
где $C(u)$~--- скорость передачи в~момент~$t$.

  В $m$-м CQI максимальная возможная скорость приема данных 
пользователем ограничена значением~$r_m$, $m\hm=1,\ldots ,M$, причем чем 
дальше находится CQI от центра соты, тем меньше максимальная возможная 
скорость приема данных для находящихся в~нем пользователей. В~связи с~тем 
что передача происходит по технологии мультивещания, ско-\linebreak рость передачи 
$C(t)\hm= \min\nolimits_m \left\{ r_m: {\cal{J}}_m(t)>0\right\}$ является  
ку\-соч\-но-по\-сто\-ян\-ной и~в~каждый момент\linebreak времени определяется как 
минимальная из максимальных возможных скоростей приема пользователями 
данных.
  
  Так как процесс $C(t)$ при наличии в~системе $N\hm>1$ пользователей не 
является марковским, в~качестве управляющего процесса вместо $C(t)$ будем 
использовать марковский процесс $\mathbf{X}(t)\hm= \left( X_1(t), X_2(t), \ldots , 
X_M(t)\right)$, зная состояние которого, можно определить состояние 
процесса~$C(t)$. Компоненты~$X_m(t)$ этого процесса указывают на число 
абонентов в~каждом CQI в~момент времени~$t$. Такой управляющий 
процесс~$\mathbf{X}(t)$ имеет пространство состояний
\begin{multline*}
  \mathcal{X}=\left\{ \vphantom{\sum\limits_{i=1}^M}
  \left.\left( n_1,\ldots, n_M\right) 
  \right\vert n_i=0,\ldots, N,\ 
i=1,\ldots , M\,,\right.\\
\left. \sum\limits_{i=1}^M n_i=N\right\}\,,
\end{multline*}
при этом число состояний процесса равно $\vert{\cal{X}}\vert \hm=  
C^N_{M+N-1}$.
  
  Процесс $\mathbf{X}(t)$ можно рассматривать как процесс, описывающий 
некоторую замкнутую сеть массового обслуживания с~$N$~заявками 
и~$M$~узлами неограниченной емкости. В~этой сети длительности 
обслуживания заявок в~узле~$i$ имеют экспоненциальное распределение 
с~параметром~$\tau_i^{-1}$, а~матрица вероятностей переходов заявок по узлам 
имеет следующий вид:
  $$
  P_{ij}=\begin{cases}
  \alpha_i\,, & j=i+1\,;\\
  \beta_i\,, & j=i-1\,;\\
  0 & \mbox{в~остальных~случаях.}
  \end{cases}
  $$ 
Отсюда можно получить вид матрицы интенсивностей переходов 
процесса~$\mathbf{X}(t)$, которая в~случае системы с~$N$~пользователями 
имеет вид:
\begin{multline*}
\mathbf{A}=\left(a_{ij}\right)= 
\begin{cases}
\displaystyle \fr{i_m\alpha_m}{\tau_m}\,, &j=i+e_m\,;\\
\displaystyle \fr{i_m\beta_m}{\tau_m}\,, & j=i-e_m\,;\\
\displaystyle -\sum\limits^M_{m=1} \fr{i_m}{\tau_m}\,, & j=i\,;\\
  0& \mbox{в~остальных~случаях,}
  \end{cases}\\
  m=1,\ldots , M,\ i,j\in {\cal{X}}\,.
  \end{multline*}
  
  Процесс $(\mathbf{X}(t), F(t))$ является марковским процессом и~обладает 
следующим свойством: 

\noindent
  \begin{multline*}
\hspace*{-8.16681pt}P\left\{ \mathbf{X}(t+h) \!=\!\mathbf{j}, F(t+h) < y\vert 
\mathbf{X}(h)=\mathbf{i}, F(h)=x\right\}\!={}\\
  {}= P\left\{ \mathbf{X}(t)=\mathbf{j},\ F(t)< y\vert \mathbf{X}(0)=\mathbf{i},\ 
F(0)=x\right\}.
  \end{multline*}
  
  \noindent
  Это означает, что $\left( \mathbf{X}(t), F(t)\right)$ является процессом, 
однородным по второй компоненте, а~$F(t)$~--- процессом с~независимыми 
приращениями, заданным на цепи Маркова~$\mathbf{X}(t)$~[4,~5].
  
  Обозначим через $T(\mathbf{x})$ момент первого достижения 
процессом~$F(t)$ уровня~$\mathbf{x}$, $T(\mathbf{x})\hm= \mathrm{inf}\left\{  
t\vert F(t)\geq \mathbf{x}\right\}$, $\mathbf{Y}(\mathbf{x})\hm= 
\mathbf{X}(T(\mathbf{x}))$, и~заметим, что неравенство $F(t)\hm\leq 
\mathbf{x}$ равносильно неравенству $T(\mathbf{x})\hm\geq t$. Таким образом, 
процесс $\left( \mathbf{Y}(\mathbf{x}), T(\mathbf{x})\right)$, так же как и~$\left( 
\mathbf{X}(t), F(t)\right)$, является процессом с~независимыми приращениями, 
заданным на цепи Маркова~[4, 5]. Далее будем называть процесс $\left( 
\mathbf{Y}(\mathbf{x}), T(\mathbf{x})\right)$ двойственным процессу $\left( 
X(t), F(t)\right)$~[6].
  
  Обозначим через~$\mathbf{B}$ матрицу интенсивностей переходов 
процесса~$\mathbf{Y}(\mathbf{x})$. Если $\mathbf{Y}(\mathbf{x})\hm={n}$ 
в~некотором интервале $a\hm\leq x\hm\leq b$, то $\mathbf{X}(t)\hm=\mathbf{n}$ и~процесс~$F(t)$ возрастает со скоростью~$r(\mathbf{n})$ в~интервале времени 
$T(a)\hm\leq t\hm\leq T(b)$. Следовательно, $T(b)\hm= T(a)+(b-a)/r(\mathbf{n})$ 
и,~значит, $T(x)$ возрастает\linebreak на интервале $a\hm\leq x\hm\leq b$ со скоростью 
$1/r(\mathbf{n})$. Поэтому среднее время пребывания процесса~$\mathbf{X}(t)$ 
в~состоянии~$\mathbf{n}$ в~$r(\mathbf{n})$ раз меньше среднего времени 
пребывания процесса~$\mathbf{Y}(\mathbf{x})$ в~этом состоянии. 
Соответственно, интенсивность выхода $a_{\mathbf{n}}\hm= -a_{\mathbf{nn}}$ 
процесса~$\mathbf{X}(t)$ из состояния~$\mathbf{n}$ связана с~интенсивностью 
выхода $b_{\mathbf{n}}\hm= -b_{\mathbf{nn}}$ процесса~$\mathbf{Y}(x)$ из 
состояния~$\mathbf{n}$ равенством $a_{\mathbf{n}}\hm= b_{\mathbf{n}} 
r_{\mathbf{n}}$. При этом процессы~$\mathbf{X}(t)$ 
и~$\mathbf{Y}(\mathbf{x})$ имеют одинаковые вероятности 
переходов~$a_{ij}/a_{\mathrm{i}}\hm= b_{ij}/b_{\mathrm{i}}$ в~момент выхода из 
состояния~$\mathbf{i}$. Таким образом, матрицы интенсивностей переходов этих 
процессов связаны простым равенством $\mathbf{B}\hm=  
\mathbf{R}^{-1}\mathbf{A}$, а матрица переходных вероятностей 
процесса~$\mathbf{Y}(\mathbf{x})$ дается формулой $\mathbf{Q}(x)\hm= 
e^{\mathbf{R}^{-1}\mathbf{A}x}$.
  
  Матрица преобразований Лапласа условных функций распределения времени 
передачи данных $\boldsymbol{\Phi}(x,s)\hm= \left( 
\phi_{\mathrm{ij}}(x,s)\right)$,

\noindent
  \begin{multline*}
  \phi_{\mathrm{ij}}=M\left( 
  I\left( \mathbf{Y}(x)=\mathbf{j}\right) \exp \left( sT(x)\right)\vert 
\mathbf{Y} (0)=\mathbf{i},\right.\\
\left. T(0)=0\right)\,,\ \mathbf{i},\mathbf{j}\in {\cal {J}}\,,
%  \label{e2-na}
  \end{multline*}
может быть записана как

\noindent
\begin{equation*}
\boldsymbol{\Phi}(x,s)=e^{(\mathbf{B}-s\mathbf{R})t}\,.
%\label{e3-na}
\end{equation*}
  
  Для матриц условных центральных моментов~$W^{(k)}(t)$ времени~$T(x)$ 
передачи информации объема~$x$

\noindent
  \begin{multline*}
  w_{\mathbf{ij}}^{(k)}(x)={}\\
  {}=M\left( I\left(\mathbf{Y}(x)=\mathbf{j}\right) T^k(x)\vert 
\mathbf{X}(0)=\mathbf{i}\,,\enskip T(0)=0\right)\,,
  \end{multline*}
согласно~\cite{6-na}, справедлива следующая рекуррентная формула:

\noindent
\begin{align*}
\mathbf{W}^{0}(x) &=\mathbf{Q}(x)\,;\\
\mathbf{W}^{(k)} (x)&=  k\int\limits_0^t \mathbf{W}^{k-1}(s) 
\mathbf{R}^{-1} \mathbf{Q}(x-s)\,ds\,,\enskip k>0\,.
%\label{e4-na}
\end{align*}
  
  Используя операцию свертки~<<*>>, матрицу $\mathbf{W}^{(k)}(t)$ можно 
в~замкнутом виде записать как
  \begin{multline*}
  \mathbf{W}^{(k)}(t)={}\\
  {}=k! \left( \mathbf{P}(t) \mathbf{R}^{-1}\right) * \left( 
\mathbf{P}(t) \mathbf{R}^{-1}\right) * \cdots * \left( \mathbf{P}(t)\mathbf{R}^{-1} 
\right)  * \mathbf{P}(t),
  %\label{e5-na}
  \end{multline*}
где матрица $(\mathbf{P}(t)\mathbf{R}^{-1})$ встречается~$k$~раз. 

Так как процессы $\mathbf{X}(t)$ и~$\mathbf{Y}(\mathbf{x})$ имеют одно и~то 
же начальное распределение~$\mathbf{p}$, формула для нахождения $k$-го 
центрального момента $w^{(k)}(x)$ момента~$T(x)$ достижения уровня~$x$ 
имеет следующий вид:
\begin{equation*}
w^{(k)}(x)=k! \mathbf{p}\left( \mathbf{Q}(x) \mathbf{R}^{-
1}\right)^{*k}\mathbf{1}\,,
%\label{e6-na}
\end{equation*}
где $\mathbf{1}$ есть вектор из единиц.

Таким образом, алгоритм нахождения среднего времени передачи данных 
размером~$x$ можно записать в~следующем виде.
\begin{description}
\item[Шаг 1.] Задать начальные данные: 
\begin{itemize}
\item число уровней качества обслуживания (CQI)~$M$;

\item число пользователей в~соте~$N$;

\item вероятности переходов каж\-до\-го пользователя между CQI $\alpha_i$, 
$\beta_i$, $i\hm=1,\ldots ,M$;

\item среднее время нахождения пользователя в~каж\-дом CQI~$\tau_i$, 
$i\hm=1,\ldots ,M$;

\item скорость передачи данных в~каж\-дом CQI~$r_i$, $i\hm=1,\ldots , M$.
\end{itemize}

Получить матрицу интенсивностей переходов пользователя меж\-ду CQI.



\item[Шаг 2.] Найти вектор стационарных вероятностей 
$\mathbf{p}\hm=(p_1, \ldots ,p_M)$ из уравнения
$$
\left\{ 
\begin{array}{l}
\mathbf{Ap}=\mathbf{0}\,,\\
\mathbf{p1} =1\,.
\end{array}
\right.
$$

\item[Шаг 3.] Найти матрицу интенсивностей переходов двойственного процесса 
$\mathbf{B}\hm= \mathbf{R}^{-1}\mathbf{A}$, где $\mathbf{R}\hm= \mathrm{diag}\,(r_i)$, 
$i\hm=1, \ldots , M$.

\item[Шаг 4.] Найти матрицу вероятностей переходов двойственного процесса
$\mathbf{Q}(x)\hm= e^{\mathbf{B}x}$.

\item[Шаг 5.] Найти математическое ожидание времени передачи данных
$w(x)\hm= \mathbf{p}\left( \mathbf{Q}(x) \mathbf{R}^{-1}\right) \mathbf{1}$.
\end{description}

\section{Численный анализ}

   В данном разделе рассмотрен численный пример расчета среднего времени 
передачи данных в~системе в~зависимости от их объема и~от числа\linebreak 
пользователей в~системе для трех моделей движения: случайного блуж\-да\-ния, 
движения Леви\linebreak и~броуновского движения. Модель случайных блуж\-да\-ний дает 
грубую оценку процесса движения.\linebreak Модель броуновского движения наиболее 
подходит для описания случайного хаотического перемещения пользователей 
в~пределах соты. Движение Леви является обобщением модели броуновского 
движения, дающим более реалистичные результаты.
   
  В табл.~1 представлены средние времена пребывания пользователей в~каждом 
CQI и~вероятности переходов в~соседние CQI для различных моделей движения, 
полученные посредством имитационного моделирования движения 
пользователей в~соте. В~численном примере рассматривается случай $M\hm=4$, 
наиболее близких к~краю соты CQI. Такое упрощение вызвано 
экспоненциальным ростом числа состояний введенного процесса и,~как 
следствие, размерности матрицы вероятностей переходов в~зависимости от 
числа пользователей и~от числа CQI, поскольку для полного хранения только 
одной такой матрицы при рассмотрении всех~15~CQI для тех же данных 
требуется порядка~200~ТБ памяти. Выбор наиболее далеких от центра соты CQI 
позволяет получить оценку худшего случая времени передачи.

  \begin{figure*}[b] %fig1
      \vspace*{9pt}
      \begin{minipage}[t]{79mm}
 \begin{center}
 \mbox{%
 \epsfxsize=77.785mm 
 \epsfbox{nau-1.eps}
 }
\end{center}
\vspace*{-11pt}
  \Caption{Зависимость времени передачи от объема передаваемых данных: \textit{1}~--- 
10~МГц; \textit{2}~--- 15; \textit{3}~---  20~МГц}
\end{minipage}
\hfill
%  \end{figure*}
%  \begin{figure*} %fig2
  \vspace*{1pt}
        \begin{minipage}[t]{79mm}
 \begin{center}
 \mbox{%
 \epsfxsize=77.891mm 
 \epsfbox{nau-2.eps}
 }
\end{center}
\vspace*{-11pt}
  \Caption{Зависимость времени передачи от числа пользователей: \textit{1}~--- движение 
Леви; \textit{2}~--- случайные блуждания; \textit{3}~--- Броуновское движение}
\end{minipage}
  \end{figure*}


  
  Для четырех дальних CQI при скоростях, взятых из табл.~2, можно получить 
графики среднего времени передачи данных для различных полос и~значении 
размера загрузочного блока, равном~5~МБ. Таким образом, соответствующие 
скорости рассчи-\linebreak\vspace*{-12pt}

\vspace*{10pt}

 {   %tabl1
\noindent
{{\tablename~1}\ \ \small{Значения переходных вероятностей для процесса $\mathbf{X}(t)$ для различных 
моделей движения пользователей}}


%\vspace*{6pt}


{\small
\begin{center}
{\tabcolsep=11.9pt
\begin{tabular}{|c|c|c|c|}
  \hline
$i$&$\beta$&$\alpha$&$\tau$\\
  \hline
\multicolumn{4}{|c|}{Движение Леви}\\
\hline
1&0&1&16,709408\hphantom{99}\\
2&0,5044487&0,4955512&2,0139282\\
3&0,4931353&0,5068646&1,9840439\\
4&1&0&1,6421568\\
\hline
\multicolumn{4}{|c|}{Случайные блуждания}\\
\hline
1&0&1&19,576702\hphantom{99}\\
2&0,5216453&0,4783546&2,4197101\\
3&0,5090238&0,4909761&2,2763908\\
4&1&0&2,375875\hphantom{9}\\
\hline
\multicolumn{4}{|c|}{Броуновское движение}\\
\hline
1&0&1&16,555193\hphantom{99}\\
2&0,5048854&0,4951146&2,0691345\\
3&0,5069506&0,4930493&2,0785105\\
4&1&0&2,0804090\\
\hline
\end{tabular}
}
\end{center}
}
%\vspace*{-15pt}
}
%\end{table*}

\pagebreak
  
  
\addtocounter{table}{1}

 {   %tabl1
\noindent
\begin{center}
\parbox{43mm}{{{\tablename~2}\ \ \small{Скорости передачи данных при различных уровнях качества}}}


\vspace*{6pt}



{\tabcolsep=14pt
\small
\begin{tabular}{|c|c|}
\hline
CQI&$c_i$, бит/с/Гц\\
\hline
1&0,1523\\
2&0,2344\\
3&0,377\hphantom{9}\\
4&0,6016\\
\hline
\end{tabular}
}
\end{center}
%\vspace*{-15pt}
}
%\end{table*}

\vspace*{12pt}


  
  
\addtocounter{table}{1}





\noindent 
тываются как $r_i\hm= c_i b$, где~$c_i$~--- это скорость из
табл.~2, а~$b$ соответствует рассматриваемой ширине полосы. Значения 
скоростей, приходящихся на~1~Гц, взяты из~[7].
  



  
  

  
  
  Рисунок~1 показывает среднее время передачи данных в~системе при наличии 
в~ней $N\hm=20$ пользователей в~зависимости от объема передаваемых данных 
для различной ширины полосы. Можно видеть, что при заданном числе 
пользователей в~соте среднее время передачи данных линейно зависит от объема 
передаваемых данных. Другими словами, среднее время передачи данных при 
неизменном числе пользователей в~соте не зависит от их модели движения. Это 
также подтверждается тем, что результаты для всех трех моделей движения для 
заданных значений полностью совпали.

  
  На рис.~2 представлены результаты для передачи данных фиксированного 
объема $x\hm=25$~MB по полосе~15~МГц в~зависимости от числа 
пользователей в~системе. Можно видеть, что при увеличении числа 
пользователей в~системе время передачи стремится к~константе. Время передачи 
данных для рассмотренного примера не превосходит~3~с. 
Также можно сделать 
вывод о~наличии максимального необходимого для рассмотрения числа 
пользователей, перемещающихся в~пределах соты~$N_0$, при превышении 
которого достаточно рассматривать систему с~числом пользователей, 
равным~$N_0$, так как дальнейшее увеличение не внесет существенных 
изменений. 

Также можно видеть, что, как говорилось ранее, модель случайных 
блужданий, являясь наиболее грубым приближением процесса для трех 
рас\-смот\-рен\-ных моделей движения, дает оценку среднего времени передачи 
сверху, в~то время как броуновское движение и~движение Леви практически 
совпадают.

\section{Заключение}

  В представленной статье с~помощью аппарата управляющих марковских 
процессов построена модель для анализа характеристики среднего времени 
передачи данных в~соте гетерогенной беспроводной сети LTE по технологии 
мультивещания с~учетом перемещения абонентов между областями с~различным 
качеством обслуживания внутри соты. Проведен анализ полученной 
характеристики в~зависимости от объема передаваемых данных и~от чис\-ла 
пользователей в~системе. 
  
  Результаты численного анализа, представленные в~работе, показывают, что 
при постоянном чис\-ле пользователей в~системе время передачи данных не 
зависит от выбранной модели движения, а зависит только от объема 
передаваемых данных. Также было показано, что при увеличении числа 
пользователей в~соте для заданного количества CQI существует максимальное 
достаточное для рассмотрения число пользователей, при превышении которого 
рассматриваемая характеристика претерпевает минимальные изменения. 
%
В~дальнейшем планируются исследования, позволяющие получить 
рассматриваемую характеристику для соты, разбитой на большее число 
областей CQI. Отдельным направлением может также стать расширение 
сценария на случай перемещения абонентов за пределы соты. Кроме того, 
можно рассмотреть другие модели движения и~другие характеристики для 
описанной системы.

%\vspace*{6pt}
  
{\small\frenchspacing
 {%\baselineskip=10.8pt
 \addcontentsline{toc}{section}{References}
 \begin{thebibliography}{9}
\bibitem{1-na}
\Au{Howard R.\,A.} Dynamic probabilistic systems. Vol.~II: Semi-Markov and decision  
processes.~--- New York, NY, USA: John Wiley \& Sons, 1971. 564~p.
\bibitem{2-na}
\Au{Blaabjerg S., Anderson~H., Anderson~H.} Approximating the heterogeneous fluid queue with 
a~birth-death fluid queue~// IEEE T.~Commun., 1995. Vol.~43. No.\,5. P.~1884--1887.
\bibitem{3-na}
\Au{Van Doorn E.\,A., Scheinhardt~W.\,R.\,W.} Analysis of birth-death fluid queues~// Applied 
Mathematics Workshop Proceedings.~--- Taejon, Korea, 1996. Vol.~5. P.~13--29.
\bibitem{4-na}
\Au{Ежов И.\,И., Скороход~А.\,В.} Марковские процессы, однородные по второй 
компоненте~// Теория вероятностей и~ее применения, 1969. Т.~14. Вып.~1. С.~3--14; Вып.~4. 
С.~679--692.
\bibitem{5-na}
\Au{Сinlar E.} Markov additive processes~// Z.~Wahrscheinlichkeit., 
1972. Vol.~24. P.~85--93; 95--121. 
\bibitem{6-na}
\Au{Naumov V., Emstad~P.} Analysis of losses in a~bufferless transmission link~// Managing traffic 
performance in converged networks~/ Eds.  L.~Mason, T.~Drwiege, J.~Yan.~--- Lecture notes in 
computer science ser.~--- Springer, 2007. Vol.~4516. P.~913--924.
\bibitem{7-na}
3GPP TS 25.214 V8.9.0.  3rd Generation Partnership Project. Technical Specification 
Group Radio Access Network. Physical layer procedures (FDD). Release~8.~--- 
Valbonne, France: 3GPP, March 2010. 94~p.
 \end{thebibliography}

 }
 }

\end{multicols}

\vspace*{-3pt}

\hfill{\small\textit{Поступила в~редакцию 31.05.17}}

\vspace*{8pt}

%\newpage

%\vspace*{-24pt}

\hrule

\vspace*{2pt}

\hrule

%\vspace*{8pt}


\def\tit{PERFORMANCE MEASURES ANALYSIS OF~DATA TRANSFER PROCESS TO~MOBILE USERS 
IN~LTE CELL}

\def\titkol{Performance measures analysis of~data transfer process to~mobile users 
in~LTE cell}

\def\aut{V.\,A.~Naumov$^1$, E.\,V.~Mokrov$^2$, 
and~K.\,E.~Samouylov$^{2,3}$}

\def\autkol{V.\,A.~Naumov, E.\,V.~Mokrov, 
and~K.\,E.~Samouylov}

\titel{\tit}{\aut}{\autkol}{\titkol}

\vspace*{-9pt}


\noindent
$^1$Service Innovation Research Institute (PIKE), 8A~Annankatu, 5th floor,  Helsinki 
00120, Finland

\noindent
$^2$Peoples' Friendship University of Russia (RUDN University), 6~Miklukho-Maklaya Str., 
Moscow 117198, Russian\linebreak
$\hphantom{^1}$Federation


\noindent
$^3$Institute of Informatics Problems, Federal Research Center ``Computer Science and 
Control'' of the Russian\linebreak
$\hphantom{^1}$Academy of Sciences, 44-2~Vavilov Str., Moscow 119333, Russian 
Federation


\def\leftfootline{\small{\textbf{\thepage}
\hfill INFORMATIKA I EE PRIMENENIYA~--- INFORMATICS AND
APPLICATIONS\ \ \ 2017\ \ \ volume~11\ \ \ issue\ 4}
}%
 \def\rightfootline{\small{INFORMATIKA I EE PRIMENENIYA~---
INFORMATICS AND APPLICATIONS\ \ \ 2017\ \ \ volume~11\ \ \ issue\ 4
\hfill \textbf{\thepage}}}

\vspace*{3pt}



\Abste{The goal of the study is to analyze the average transmission time in multicasting 
heterogeneous wireless networks, considering user movements within a~cell under the condition that 
the cell can be divided into several areas with different channel quality. This performance measure is 
analyzed against the volume of transmitted data and number of users in a~cell. Three different 
motion models describing user movements within a~cell are studied. In order to find out the average 
transmission time, an approach based on Markov control processes was implemented.}

\KWE{multicasting; heterogeneous wireless network; quality of service; Markov control process; 
motion model; average transmission time; dual process; queuing network}



  \DOI{10.14357/19922264170410} 

%\vspace*{-12pt}

\Ack
\noindent
The publication was supported by the Ministry of Education and Science of the Russian Federation 
(project No.\,2.882.2017/4.6). 



\vspace*{6pt}

  \begin{multicols}{2}

\renewcommand{\bibname}{\protect\rmfamily References}
%\renewcommand{\bibname}{\large\protect\rm References}

{\small\frenchspacing
 {%\baselineskip=10.8pt
 \addcontentsline{toc}{section}{References}
 \begin{thebibliography}{9}
\bibitem{1-na-1}
\Aue{Howard, R.\,A.} 1971. \textit{Dynamic probabilistic systems. Vol.~II: Semi-Markov and 
decision processes}. New York, NY: John Wiley \& Sons. 564~p.
\bibitem{2-na-1}
\Aue{Blaabjerg, S., H.~Anderson, and H.~Anderson.} 1995. Approximating the heterogeneous fluid 
queue with a~birth-death fluid queue. \textit{IEEE T.~Commun.} 43(5):1884--1887. doi: 
10.1109/26.387416.
\bibitem{3-na-1}
\Aue{Van Doorn, E.\,A., and W.\,R.\,W.~Scheinhardt.} 1996. Analysis of birth-death fluid 
queues, \textit{Applied Mathematics Workshop Proceedings }. Taejon, Korea. 5:13--29.
\bibitem{4-na-1}
\Aue{Ezhov, I.\,I., and A.\,V.~Skorokhod.} 1969. 
Markov processes with homogeneous second component. \textit{Theor. 
Probab. Appl.} 14(1):1--13  (doi: 10.1137/1114001);
14(4):652--667 (doi: 10.1137/1114081).



\bibitem{5-na-1}
\Aue{Сinlar, E.} 1972. Markov additive processes. 
\textit{Z.~Wahrscheinlichkeit.}  24:85--93; 95--121. 
\bibitem{6-na-1}
\Aue{Naumov, V., and P.~Emstad.} 2007. Analysis of losses in a~bufferless transmission link. 
\textit{Managing traffic performance}\linebreak\vspace*{-11pt}

\columnbreak

\noindent
\textit{in converged networks}. Eds. L.~Mason, T.~Drwiege, and 
J.~Yan. Lecture notes in computer science ser. Springer. 4516:913--924.
\bibitem{7-na-1}
3GPP TS 25.214~V8.9.0. March 2010. 3rd Generation Partnership Project. Technical Specification 
Group Radio Access Network. Physical layer procedures (FDD). Release~8. Valbonne, France: 
3GPP. 94~p.
\end{thebibliography}

 }
 }

\end{multicols}

\vspace*{-6pt}

\hfill{\small\textit{Received May 31, 2017}}

%\vspace*{-10pt}

\Contr


\noindent
\textbf{Naumov Valeriy A.} (b.\ 1950)~--- Candidate of Science (PhD) in physics 
and mathematics, 
scientific director, Service Innovation Research Institute (PIKE), 8A~Annankatu, 5th floor,  
Helsinki 00120, Finland; \mbox{valeriy.naumov@pfu.fi} 

\vspace*{3pt}

\noindent
\textbf{Mokrov Evgeniy V.} (b.\ 1988)~--- PhD student, Peoples' Friendship University of Russia 
(RUDN University), 6~Miklukho-Maklaya Str., Moscow 117198, Russian Federation; 
\mbox{mokrov\_ev@rudn.university} 

\vspace*{3pt}

\noindent
\textbf{Samouylov Konstantin E.} (b.\ 1955)~--- Doctor of Science in technology, professor, Head of 
Department, Director of Institute of Applied Mathematics and Telecommunications, Peoples' 
Friendship University of Russia (RUDN University), 6~Miklukho-Maklaya Str., Moscow 117198, 
Russian Federation; senior scientist, Institute of Informatics Problems, Federal Research Center 
``Computer Science and Control'' of the Russian Academy of Sciences, 44-2~Vavilov Str., Moscow 
119333, Russian Federation; \mbox{samuylov\_ke@rudn.university}

\label{end\stat}


\renewcommand{\bibname}{\protect\rm Литература} 