\documentclass[10pt]{book}
\usepackage[utf8]{inputenc}

\usepackage{latexsym,amssymb,amsfonts,amsmath,amsxtra,indentfirst,shapepar,%fleqn,%
picinpar,shadow,floatflt,enumerate,multicol,colortbl,moreverb,cite,ipi}

\usepackage{rotating}
\usepackage{mathrsfs}
\usepackage[noend]{algorithmic}
\usepackage{ulem}
\usepackage{graphicx}
%\usepackage{algorithm2e}
\usepackage[linesnumbered,boxed,ruled]{algorithm2e}
%\usepackage{xypic}
\usepackage{oldgerm}

\SetAlgorithmName{Алгоритм}{алгоритм}{Список алгоритмов}

%из Дюковой

\newcommand{\algKeyword}[1]{{\bf #1}}
\newcommand{\Proc}[1]{\text{\tt #1}}
\def\CALL{\algKeyword{call}~}

\newenvironment{AlgProcedure}[1]
{
    \small
    \medskip
    %    \hrule
    \medskip
    \algKeyword{PROCEDURE} #1
    \begin{algorithmic}[1]}
    {\end{algorithmic}
    %    \hrule
    \bigskip
}

\def\CALL{\algKeyword{call}~}

%конец для Дюковой

%\RequirePackage[ruled]{algorithm}


\input{epsf}

%\nofiles

%\includeonly{avtor} %+pdf
%\includeonly{obchak,avtor}
%\includeonly{pred}                 %+
%\includeonly{podgot-rus-site,podgot-eng-site}  
%\includeonly{ocherk} 
%\includeonly{nekrol} 
%\includeonly{ipi-ind} 
%\includeonly{toc-rus}
%\includeonly{toc-en} 


%\includeonly{gadamaka}  %1 pdfавт
%\includeonly{razumchik-rus}  %2pdfавт
%\includeonly{razumchik-eng}  %3pdfавт
%\includeonly{korolev}   %4pdf
%\includeonly{gorshenin}  %5pdfавт
%\includeonly{malashenko}  %6pdfавт
%\includeonly{agalarov} %7pdfавт
%\includeonly{grusho}   %8pdfавт
%\includeonly{grebeshkov} %9pdf
%\includeonly{naumov}   %10pdfавт
%\includeonly{buyanov}  %11pdfавт
%\includeonly{bitukov}  %12pdfавт
%\includeonly{kudr-tit} %13pdfавт
%\includeonly{pagano}   %14pdfавт
%\includeonly{kruzhkov} %15pdfавт


%\includeonly{toc-rus, toc-en}
%\includeonly{obchak} %,toc-en}
%\includeonly{rekl}
%\includeonly{rekl-1}
%\includeonly{reshal}  %
%\includeonly{eng-index}
%\includeonly{cover3}

\usepackage{acad}
%\usepackage{courier}
\usepackage{decor}
\usepackage{newton}
\usepackage{pragmatica}
\usepackage{zapfchan}
\usepackage{petrotex}
\usepackage{bm}                     % полужирные греческие буквы
\usepackage{upgreek}                % прямые греческие буквы
\usepackage{eufrak}
\usepackage{verbatim}

\renewcommand{\bottomfraction}{0.99}
\renewcommand{\topfraction}{0.99}
\renewcommand{\textfraction}{0.01}

\setcounter{secnumdepth}{1} %здесь - 3 + chapter = 4

\arraycolsep=1.5pt

%\usepackage[pdftex]{graphicx}

%\usepackage{oz}

%NEW COMMANDS


\renewcommand*{\hm}[1]{#1\nobreak\discretionary{}%
            {\hbox{$\mathsurround=0pt #1$}}{}} %% Дублирует знаки операций
                               %при переносе в формуле (перед знаком, который
                               %надо продублировать ставится команда \hm)

%\newcommand{\endproof}{\hfill$\Box$}
\renewcommand{\r}{\mathbb{R}}
\newcommand{\I}{{\rm I\hspace{-0.7mm}I}}
%\newcommand{\Ikl}{{\tt{1}}\hspace*{-1.44mm}\mathtt{1}}
\newcommand{\Ik}{\mbox{{\small \tt {1}}\hspace{-1.3mm}{\tt 1}}}
\newcommand{\argmin}{\mathop{\mathrm{arg}\mathrm{min}}}
\newcommand{\argmax}{\mathop{\mathrm{arg}\mathrm{max}}}
%\newcommand{\capr}{\mathop{\cap\,}}
%\newcommand{\cupr}{\mathop{\cup\,}}
%\def\argmin{\mathop{arg\,min}}

\def\vrp{\varphi}
\def\prt{\partial}
\def\mm{{\sf M}}
\def\modnop#1{\mathop{#1}\limits_{n}}
\def\eam{\mathbin{{\mathop{=}\limits^{\mathrm{def}}}}}
\def\dey#1#2{#1 (#2)}
\def\deyc#1#2{#1 \cdot  #2}
\def\ra#1{\;\mathop{\to}\limits^{#1}\;}
\def\raz#1{\;\mathop{\longrightarrow}\limits^{\!\!\!#1}\;}
\def\ral#1{\;\mathop{\longrightarrow}\limits^{#1}\;}

\newcommand{\Nor}{\mathcal{N}}
\newcommand{\T}{\mathbb{T}}
\newcommand{\Z}{\mathbb{Z}}



\newcommand{\il}[2]{\int\limits_{#1}^{#2}}%интеграл с пределами #1 и #2

\def\sm2{\mathop {\sum\limits^{n^\Theta}\sum\limits^{n^\Theta}}}
\def\sss{\sum\limits}
\def\tr{,\,\ldots\,,\,}
\def\rk{\right]}
\def\lk{\left[}
\def\rf{\right\}}
\def\lf{\left\{}
\def\lv{\,\left\vert}
\def\rv{\right\vert\,}
\def\iii{\int\limits}
\def\iin{\int\limits_{-\infty}^\infty}
\def\rrv{\right\vert}


\def\ee{{\cal E}}
\def\ww{{\cal W}}
\def\yy{{\cal Y}}
\def\vv{{\cal V}}

\newcommand{\R}{\mathbb R}
\newcommand{\E}{\mathbb E}
\newcommand{\N}{\mathbb N}

\renewcommand{\P}{\mathbb{P}}

\newcommand{\h}{{\bf H}}
\newcommand{\p}{{\sf P}}  % вероятность

\newcommand{\e}{{\sf E}}  % мат. ожидание
\newcommand{\D}{{\sf D}}  % дисперсия
\newcommand{\eps}{\varepsilon}
\newcommand{\vp}{{\mathbf p}}
\newcommand{\vz}{{\mathbf z}}
\newcommand{\vx}{{\mathbf x}}
\newcommand{\vf}{{\mathbf f}}
\newcommand{\F}{{\mathcal F}}
\def\ap{{\mathrm{ЭР}}}
\newcommand{\ud}{\Delta_n} %uniform ditance
\newcommand{\nud}{\Delta_n(x)}
\renewcommand{\Re}{\mathrm{Re}\,}

\newcommand{\abs}[1]{\left\vert#1\right\vert}

\newcommand{\norm}[1]{\left\Vert#1\right\Vert}
\def\da{(\Delta_t,A)}

\newcommand{\corr}{\mathrm{corr}}

\newcommand{\cov}{\mathrm{cov}}
\newcommand{\Expect}{\mathbb{E}}

\def\w{\omega}
\def\W{\Omega}

\def\inh{\int\limits_{nh}^{(n+1)h}}

\def\sumin{\sum_{i=1}^N}


\def\bxt{(Y,t)}
\def\xt{(y,t)}

\def\ovth{{\fr{\tau-nh}{h}}}
\def\ov{\overline}
\def\tm{\tilde m}
\def\tl{\tilde\lambda}
\def\tB{\widetilde B}
\def\tb{\tilde b}
\def\ld{\ldots}
\def\cd{\cdots}


\DeclareMathOperator{\sign}{sign}

%\newcommand{\gr}{{\geqslant}}


\newcommand{\g}{\mbox{\textit{g}}}

\renewcommand{\la}{\lambda}
\newcommand{\si}{\sigma}
\newcommand{\alp}{\alpha}

%\newcommand{\pto}{\stackrel{P}{\longrightarrow}} % сходимость по веpоятности

\newcommand{\eqd}{\stackrel{\mathrm{d}}{=}} % равенство по pаспpеделению
\newcommand{\eqdelta}{\stackrel{\triangle}{=}} % равенство по pаспpеделению

\def\be#1{\begin{equation}\label{#1}}
\def\ee{\end{equation}}
\def\re#1{(\ref{#1})}

\def\bn{\begin{enumerate}}
\def\en{\end{enumerate}}
\def\bi{\begin{itemize}}
\def\ei{\end{itemize}}
%\def\i{\item}

%\newcommand{\kp}{\kappa}
%\def\Q{{\cal Q}} \def\H{{\cal H}}
%\newcommand{\bet}{\beta_{2+\delta}}


%\newtheorem{definition}{Определение}
%\renewcommand{\thedefinition}{\arabic{definition}.}
%END NEW COMMANDS

%\renewcommand{\baselinestretch}{1.2}

%\pagestyle{myheadings}

\setlength{\textwidth}{167mm}      % 122mm
\setlength{\textheight}{658pt}
%\setlength{\textheight}{635.6pt}
\setlength{\columnsep}{4.5mm}

\setcounter{secnumdepth}{4}

%\addtolength{\headheight}{2pt}
%\addtolength{\headsep}{-2mm}

\addtolength{\topmargin}{-7mm}  % for printing


%\hoffset=-30mm  % From Yap
\hoffset=-23mm  % From Acrobat

%\voffset=0mm % From Yap
\voffset=-5mm   % From Acrobat

%\addtolength{\evensidemargin}{-2.5mm} % for printing
%\addtolength{\oddsidemargin}{2.5mm}  % for printing

\addtolength{\evensidemargin}{-12mm} % for printing
\addtolength{\oddsidemargin}{8mm}  % for printing

%\renewcommand{\thefootnote}{\fnsymbol{footnote}}
%\renewcommand{\thefootnote}{\arabic{footnote}}
\renewcommand{\figurename}{\protect\bf Рис.}
\renewcommand{\tablename}{\protect\bf Таблица}

\newcommand{\Caption}[1]{\caption{\protect\small %\baselineskip=2.5ex
#1}}

\renewcommand{\thefigure}{\arabic{figure}}
\renewcommand{\thetable}{\arabic{table}}
\renewcommand{\theequation}{\arabic{equation}}
\renewcommand{\thesection}{\arabic{section}}

\renewcommand{\contentsname}{СОДЕРЖАНИЕ}
\newcommand{\fr}[2]{\displaystyle\frac{\displaystyle #1\mathstrut}{\displaystyle #2\mathstrut}}

%\renewcommand{\thefootnote}{\fnsymbol{footnote}}
%\newcommand{\g}{\mbox{\textit{g}}}

%\newcommand{\Caption}[1]{\caption{\protect\small\baselineskip=2ex #1}}
\newcounter{razdel}
\setcounter{razdel}{0}


\newcommand{\titel}[4]{%
\

\vspace*{5pt}

\ifodd\therazdel {\raggedright\noindent\Large\textrm\textbf
 \lineskip .75em
  \baselineskip=3.2ex #1 \par}
\vskip 1em {\noindent\large\textrm\textbf #2 \par}
\addcontentsline{toc}{subsection}{{\textrm\textbf #1}\protect\newline #2}
\def\rightheadline{\underline{\noindent\hbox to \textwidth{\hfill\small\textrm{#4}
%\hfill \large\bf\thepage
}}}
\def\leftheadline{\underline{\noindent\parbox{\textwidth}{
%\raggedleft\large\bf\thepage \hfill
\small\textit{#3}\hfill}}}
\def\leftfootline{\small{\textbf{\thepage}
\hfill ИНФОРМАТИКА И ЕЁ ПРИМЕНЕНИЯ\ \ \ том~11\ \ \ выпуск 4\ \ \ 2017}
}%
 \def\rightfootline{\small{ИНФОРМАТИКА И ЕЁ ПРИМЕНЕНИЯ\ \ \ том~11\ \ \ выпуск~4\ \ \ 2017
\hfill \textbf{\thepage}}}
\vskip 2em \setcounter{figure}{0}
\setcounter{table}{0}
\setcounter{equation}{0}
\setcounter{section}{0}
\setcounter{subsection}{0}
\setcounter{subsubsection}{0}
\setcounter{footnote}{0}
\setcounter{razdel}{0}
%\end{flushleft}
\else {
 \raggedright\noindent\Large\textrm\textbf
 \lineskip .75em
\baselineskip=3.2ex #1 \par} \vskip 1em
%\begin{flushleft}
{\noindent\large\textrm\textbf #2 \par}
\addcontentsline{toc}{subsection}{{\textrm\textbf #1}\protect\newline #2}
\def\rightheadline{\underline{\noindent\hbox to \textwidth{\hfill\small\textrm{#4}
%\hfill \large\bf\thepage
}}}
\def\leftheadline{\underline{\noindent\parbox{\textwidth}{%\raggedleft\large\bf\thepage \hfill
\small\textit{#3}\hfill}}}
\def\leftfootline{\small{\textbf{\thepage}
\hfill ИНФОРМАТИКА И ЕЁ ПРИМЕНЕНИЯ\ \ \ том~11\ \ \ выпуск~4\ \ \ 2017}
}%
 \def\rightfootline{\small{ИНФОРМАТИКА И ЕЁ ПРИМЕНЕНИЯ\ \ \ том~11\ \ \ выпуск~4\ \ \ 2017
\hfill \textbf{\thepage}}} \vskip 2em \setcounter{figure}{0}
\setcounter{table}{0} \setcounter{equation}{0} \setcounter{section}{0}
\setcounter{subsection}{0} \setcounter{subsubsection}{0}
\setcounter{footnote}{0}
%\end{flushleft}
\fi}

\newcommand{\titelr}[2]{%
\

\vspace*{5pt}

\ifodd\therazdel {\raggedright\noindent%\Large\textrm\textbf
 \lineskip .75em
  \baselineskip=3.2ex #1 \par}
\vskip 1em {\noindent\normalsize\textrm\textbf #2 \par}
\else {
 \raggedright\noindent\Large\textrm\textbf
 \lineskip .75em
\baselineskip=3.2ex #1 \par} \vskip 1em
%\begin{flushleft}
{\noindent\large\textrm\textbf #2 \par
%\noindent\normalsize\textrm\textbf #2 \par
} \fi}

\newcommand{\titele}[5]{%
\

%\vspace*{5pt}

\ifodd\therazdel {\raggedright\noindent\large
\textrm\textbf
 \lineskip .75em
%  \baselineskip=3.2ex
#1 \par}
\vskip .5em {\noindent\large\textrm\textbf #2 \par}
\vskip .5em
 {\noindent\textrm #3 \par}
\addcontentsline{toc}{subsection}{{\textrm\textbf #1}\protect\newline #2}
\def\rightheadline{\underline{\noindent\hbox to \textwidth{\hfill\small\textrm{#4}
%\hfill \large\bf\thepage
}}}
\def\leftheadline{\underline{\noindent\parbox{\textwidth}{
%\raggedleft\large\bf\thepage \hfill
\small\textrm{#5}\hfill}}}
\def\leftfootline{\small{\textbf{\thepage}
\hfill ИНФОРМАТИКА И ЕЁ ПРИМЕНЕНИЯ\ \ \ том~11\ \ \ выпуск~4\ \ \ 2017}
}%
 \def\rightfootline{\small{ИНФОРМАТИКА И ЕЁ ПРИМЕНЕНИЯ\ \ \ том~11\ \ \ выпуск~4\ \ \ 2017
\hfill \textbf{\thepage}}} \vskip 1em \setcounter{figure}{0}
\setcounter{table}{0} \setcounter{equation}{0} \setcounter{section}{0}
\setcounter{subsection}{0} \setcounter{subsubsection}{0}
\setcounter{footnote}{0} \setcounter{razdel}{0}
%\end{flushleft}
\else {
 \raggedright\noindent\large
 \textrm\textbf
 \lineskip .75em
%\baselineskip=3.2ex
#1 \par} \vskip .5em
%\begin{flushleft}
{\noindent\large\textrm\textbf #2 \par} \vskip .5em
 {\noindent\textrm #3 \par}
\addcontentsline{toc}{subsection}{{\textrm\textbf #1}\protect\newline #2}
\def\rightheadline{\underline{\noindent\hbox to \textwidth{\hfill\small\textrm{#4}
%\hfill \large\bf\thepage
}}}
\def\leftheadline{\underline{\noindent\parbox{\textwidth}{%\raggedleft\large\bf\thepage \hfill
\small\textrm{#5}\hfill}}}
\def\leftfootline{\small{\textbf{\thepage}
\hfill ИНФОРМАТИКА И ЕЁ ПРИМЕНЕНИЯ\ \ \ том~11\ \ \ выпуск~4\ \ \ 2017}
}%
 \def\rightfootline{\small{ИНФОРМАТИКА И ЕЁ ПРИМЕНЕНИЯ\ \ \ том~11\ \ \ выпуск~4\ \ \ 2017
\hfill \textbf{\thepage}}} \vskip 1em \setcounter{figure}{0}
\setcounter{table}{0} \setcounter{equation}{0} \setcounter{section}{0}
\setcounter{subsection}{0} \setcounter{subsubsection}{0}
\setcounter{footnote}{0}
%\end{flushleft}
\fi}

\def\Abst#1{
\begin{center}\small\nwt
\parbox{150mm}{%\baselineskip=2.5ex
\textbf{Аннотация:}\ \
%\hspace*{\parindent}
#1}
\end{center}}
\def\Abste#1{
\begin{center}\small\nwt
\parbox{150mm}{%\baselineskip=2.5ex
\textbf{Abstract:}\ \
%\hspace*{\parindent}
#1}
\end{center}}

\def\DOI#1{
\begin{center}\small\nwt
\parbox{150mm}{%\baselineskip=2.5ex
\textbf{DOI:}\ \
%\hspace*{\parindent}
#1}
\end{center}}

\def\Abstend#1{
\begin{center}\small\nwt
\parbox{150mm}{%\baselineskip=2.5ex
%\hspace*{\parindent}
#1}
\end{center}}


\def\KW#1{
\begin{center}\small\nwt
\parbox{150mm}{%\baselineskip=2.5ex
\textbf{Ключевые слова:}\ \ #1}
\end{center}}

\def\KWE#1{
\begin{center}\small\nwt
\parbox{150mm}{%\baselineskip=2.5ex
\textbf{Keywords:}\ \ #1}
\end{center}}


\def\KWN#1{
%\begin{center}
%\small
%\parbox{150mm}\end{center}
}

\newcommand{\Avtors}[1]{%\smallskip
%\vspace*{.5pt}
\hangindent=23pt\noindent
%\nwt
{\bfseries#1}\
}


\renewcommand{\thesubsection}{\thesection.\arabic{subsection}\hspace*{-5pt}}
\renewcommand{\thesubsubsection}{\thesubsection\hspace*{5pt}.\arabic{subsubsection}\hspace*{-3pt}}

\newcommand{\Ack}{\section*{\protect\rmfamily Acknowledgments}\noindent}
\newcommand{\Contr}{\section*{\protect\rmfamily Contributors}\noindent}
\newcommand{\Contrl}{\section*{\protect\rmfamily Contributor}\noindent}

\makeindex


\begin{document}
\Rus

\nwt
%\ptb


%\renewcommand{\contentsname}{\protect\Large\bf Содержание}

\setcounter{tocdepth}{2}

%\tableofcontents

\renewcommand{\bibname}{\protect\rmfamily Литература}
  \def\Au#1{{\it #1}}
    \def\Aue#1{{#1}}

%\newcommand{\No}{№}
  \newcommand{\tg}{\,\mathrm{tg}\,}
    \newcommand{\ctg}{\,\mathrm{ctg}\,}
  \newcommand{\arctg}{\,\mathrm{arctg}\,}

\def\forallb{\mathop{\forall}}
\def\cupb{\mathop{\cup}}
\def\existsb{\mathop{\exists}}


\newpage
\addtocounter{razdel}{1}
%\def\razd{РЕГУЛИРУЕМЫЙ ЭЛЕКТРОПРИВОД ДЛЯ ЭЛЕКТРОЭНЕРГЕТИКИ}


\setcounter{page}{2}

%{ %\Large  
{ %\baselineskip=16.6pt

\vspace*{-48pt}
\begin{center}\LARGE
\textit{Уважаемый читатель!}
\end{center}

%\vspace*{2.5mm}

\vspace*{4mm}

\thispagestyle{empty}

{\small

 
В~2017~г.\ исполняется 10~лет со времени выхода в~свет первого 
номера журнала <<Информатика и~её применения>>~--- 
научного журнала Российской академии наук, издающегося под 
на\-уч\-но-ме\-то\-ди\-че\-ским руководством Отделения нанотехнологий 
и~информационных технологий Российской академии наук. Учредителем журнала 
является Федеральный исследовательский центр <<Информатика и~управ\-ле\-ние>> 
Российской академии наук (ФИЦ ИУ РАН) (до~2015~г.~--- 
Институт проб\-лем информатики РАН).

Необходимость издания такого журнала была вызвана активным развитием 
информатики и~информационных технологий, большой важностью этого научного 
направления для развития страны, проникновением информационных технологий 
во все сферы жизни современного общества.

Тематику журнала определяет тот факт, что информатика~--- это комплексная 
фундаментальная научная дисциплина, опирающаяся на достижения 
ряда других наук, в~том числе математики, физики, лингвистики и~др. 
Одновременно журнал уделяет большое внимание современным информационным технологиям, 
являющимся приложениями результатов информатики как фундаментальной науки.

За прошедшие 10~лет (2007--2016~гг.)\ издано~38~выпусков журнала. В~них 
размещено~452~публикации, в~том числе~430~научных статей и~22~информационных 
публикации (обзоры, рецензии и~др.). Среди авторов журнала представители ведущих 
научных организаций и~университетов страны, в~том числе Московского государственного 
университета им.\ М.\,В.~Ломоносова, ФИЦ ИУ РАН (в~том числе ИПИ РАН, ВЦ 
им.\ А.\,А.~Дородницына РАН, ИСА РАН), Института точной механики и~вычислительной 
техники им.\ С.\,А.~Лебедева РАН, Института космических исследований РАН, 
Института астрономии РАН, ряда институтов Сибирского отделения РАН, МФТИ, МИФИ, 
Высшей школы экономики, Санкт-Пе\-тер\-бург\-ско\-го государственного университета, 
Санкт-Пе\-тер\-бург\-ско\-го государственного политехнического университета 
Петра Великого, Санкт-Пе\-тер\-бург\-ско\-го государственного университета 
телекоммуникаций им.\ проф.\ М.\,А.~Бонч-Бруе\-ви\-ча, 
Российского университета дружбы народов, Балтийского федерального университета 
имени Иммануила Канта, Вологодского государственного университета и~др. 
Публиковались статьи зарубежных авторов, в~том числе ученых из Израиля, 
США, Финляндии, Франции, Швейцарии, Швеции и~других стран. 

В конце настоящего выпуска журнала помещен указатель статей, 
опуб\-ли\-ко\-ван\-ных в~томах~1--10 (2007--2016~гг.).

Журнал включен в~Российский индекс научного цитирования и~в~базу 
данных RSCI Web of Science, перечень ВАК, базу данных CrossRef 
и~информационную систему <<Общероссийский математический портал MathNet>>. 
С~2015~г.\ журнал индексируется в~библиографической и~реферативной базе 
данных SCOPUS.

Мы всегда будем помнить ушедших из жизни членов редакционного совета 
и~редакционной коллегии журнала: академика С.\,К.~Коровина, профессоров 
А.\,В.~Печинкина и~И.\,А.~Ушакова, которые внесли неоценимый вклад в~становление 
и~развитие журнала.

После объединения в~2015~г.\ трех учреждений Российской академии наук~--- 
Института проблем информатики, Вычислительного центра им.\ А.\,А.~Дородницына 
и~Института системного анализа~--- в~Федеральное государственное учреждение 
<<Федеральный исследовательский центр <<Информатика и~управ\-ле\-ние>> 
Российской академии наук>> (ФИЦ ИУ РАН) именно этот Центр стал базовой организацией 
для издания журнала, что существенно расширило как тематику журнала, 
так и~его возможности по привлечению новых авторов, в~том числе и~зарубежных.

В настоящее время тематику журнала в~первую очередь составляют:
\begin{itemize}
\item    теоретические основы информатики;\\[-14.5pt] 
\item    математические методы исследования сложных систем и~процессов;\\[-14.5pt]
\item    информационные системы и~сети;\\[-14.5pt]
\item    информационные технологии;\\[-14.5pt]
\item    архитектура и~программное обеспечение вычислительных комплексов и~сетей. 
\end{itemize}

Эти направления особенно важны в~связи с необходимостью решения задач 
формирования технологической базы инновационного развития, обеспечения 
на\-уч\-но-тех\-но\-ло\-ги\-че\-ско\-го прорыва в~области создания и~развития 
отечественных информационных и~коммуникационных технологий в~интересах 
достижения высокого качества и~стабильности систем управления и~предоставления 
услуг в~экономической и~социальной сферах. 

Мы, как и~ранее, приглашаем авторов представлять для публикации в~журнале 
статьи как с достижениями в~области теоретических проблем информатики, так 
и~с~изложением результатов ее практического приложения, а~также 
рецензии на наиболее интересные книжные новинки в~области информатики 
и~информационных технологий, объявления о~крупнейших международных 
и~всероссийских конференциях, различных научных мероприятиях 
по этой тематике и~другие информационные материалы.

Надеемся, что и~в~дальнейшем содержание статей, помещаемых в~журнале, 
будет вызывать интерес научной общественности. Редакционный совет, редколлегия 
и~редакция журнала, со своей стороны, сделают все для того, 
чтобы журнал и~впредь своевременно и~подробно информировал читателей 
о~новейших достижениях информатики и~ее актуальных практических приложениях.

                

      
\vfill
\noindent
Главный редактор журнала <<Информатика и~её применения>>,\\
академик  РАН\hfill
\textit{И.\,А.~Соколов}\\[-6pt]

%\noindent
%Редактор-составитель тематического выпуска, профессор кафедры математической статистики\\
%факультета вычислительной математики и~кибернетики МГУ им.~М.\,В.~Ломоносова,\\
%ведущий научный сотрудник ИПИ РАН, доктор физико-математических наук\hfill
% \textit{В.\,Ю.~Королев}


} }
}
      

\def\stat{gadamaka}

\def\tit{МЕТОД МОДЕЛИРОВАНИЯ ХАРАКТЕРИСТИК ИНТЕРФЕРЕНЦИИ ПРИ~ПРЯМОМ 
ВЗАИМОДЕЙСТВИИ ПЕРЕМЕЩАЮЩИХСЯ УСТРОЙСТВ В~ГЕТЕРОГЕННОЙ 
БЕСПРОВОДНОЙ СЕТИ ПЯТОГО ПОКОЛЕНИЯ$^*$}

\def\titkol{Метод моделирования характеристик интерференции при~прямом 
взаимодействии перемещающихся устройств} % в~гетерогенной  беспроводной сети пятого поколения}

\def\aut{Ю.\,В.~Гайдамака$^1$, К.\,Е.~Самуйлов$^2$, С.\,Я.~Шоргин$^3$}

\def\autkol{Ю.\,В.~Гайдамака, К.\,Е.~Самуйлов, С.\,Я.~Шоргин}

\titel{\tit}{\aut}{\autkol}{\titkol}

\index{Гайдамака Ю.\,В.}
\index{Самуйлов К.\,Е.}
\index{Шоргин С.\,Я.}
\index{Gaidamaka Yu.}
\index{Samouylov K.}
\index{Shorgin S.}



{\renewcommand{\thefootnote}{\fnsymbol{footnote}} \footnotetext[1]
{Исследование выполнено при финансовой поддержке Российского научного фонда 
(проект 16-11-10227).}}


\renewcommand{\thefootnote}{\arabic{footnote}}
\footnotetext[1]{Российский университет дружбы народов; 
Институт проб\-лем информатики Федерального исследовательского 
центра <<Информатика и~управ\-ле\-ние>> 
Российской академии наук, \mbox{gaydamaka\_yuv@rudn.university}}
\footnotetext[2]{Российский университет дружбы народов; Институт проб\-лем информатики Федерального исследовательского 
центра <<Информатика и~управ\-ле\-ние>> Российской академии наук, 
\mbox{samuylov\_ke@rudn.university}}
\footnotetext[3]{Институт проб\-лем информатики Федерального исследовательского центра 
<<Информатика и~управ\-ле\-ние>> Российской академии наук, \mbox{sshorgin@ipiran.ru}}

\vspace*{-9pt}

    
    
\Abst{В общем виде показано построение модели перемещения взаимодействующих 
устройств в~гетерогенных беспроводных сетях пятого поколения с~помощью кинетического 
уравнения с~учетом скорости перемещения устройств, их пространственной плотности 
и~максимального допустимого радиуса взаимодействия. Предложен метод моделирования 
траекторий, когда при\-емо-пе\-ре\-да\-ющие устройства движутся случайным образом, 
причем блуждание в~общем случае не является стационарным, что отличает предложенную 
модель движения мобильных устройств от известных ранее моделей. Характеристики 
интерференции, в~том числе отношение сигнал/ин\-тер\-фе\-рен\-ция (SIR,
signal-to-interference ratio), исследуются 
в~виде непрерывного во времени случайного процесса, задачу расчета этих характеристик 
предлагается решать методом имитационного моделирования. Показано, что такой анализ 
дает возможность исследовать вероятностные характеристики взаимодействия устройств, 
такие как вероятность обрыва связи и~длительности периодов наличия и~отсутствия связи 
между устройствами.}

\KW{беспроводная гетерогенная сеть; отношение сигнал/ин\-тер\-фе\-рен\-ция; прямое 
взаимодействие устройств; перемещение взаимодействующих устройств; модель движения; 
кинетическое уравнение; генерация траекторий; показатели эффективности сети}

\DOI{10.14357/19922264170401} 

\vspace*{-3pt}


\vskip 10pt plus 9pt minus 6pt

\thispagestyle{headings}

\begin{multicols}{2}

\label{st\stat}

\section{Введение}

  На производительность современных сетей подвижной связи, работающих 
в~диапазоне де\-ци\-мет\-ро\-вых (300~МГц\,--\,3~ГГц) и~сантиметровых (3--30~ГГц) 
волн (например, GSM~[1], 900~МГц, WiMAX~[2], 2,5~ГГц, WCDMA~[3] 
и~LTE~[4], 1,9~ГГц и~2,1~ГГц) существенное влияние оказывает 
интерференция, т.\,е.\ взаимное деструктивное влияние радиосигналов 
мобильных станций, ра\-бо\-та\-ющих в~одном диапазоне частот~[5]. Основным 
па\-ра\-мет\-ром, определяющим качество в~таких сетях, является SIR, которое 
измеряется на приемнике и~характеризует качество беспроводного канала связи 
между приемником и~передатчиком в~ассоциированной паре  
<<пе\-ре\-дат\-чик--при\-ем\-ник>>~[1--6]. Отношение сиг\-нал/ин\-тер\-фе\-рен\-ция определяет 
скорость передачи данных и~спектральную эффективность радиоканала 
и~системы в~целом, от которых зависит пропускная способность сети, 
надежность и~связность беспроводных соединений. 
  
  Результаты работ~[7--12] по точному и~приближенному анализу SIR могут 
быть использованы для оценки скорости передачи данных, спектральной 
эффективности, числа абонентов, которым в~определенной географической 
зоне могут быть предоставлены услуги беспроводной связи, в~предположении 
о~постоянном значении SIR, т.\,е.\ для случая неподвижных 
абонентов. 

Исследования перемещающихся беспроводных устройств были 
начаты в~[13] и~получили развитие в~[14], где для случая нестационарного 
блуж\-да\-ния абонентов был предложен метод расчета SIR, позволивший 
получить плотности распределения длительности периодов наличия 
и~отсутствия связи. В~соответствии с~предложенным методом  
SIR представляет собой функционал расстояний между 
взаимодействующими устройствами, опре\-де\-ля\-емых моделью движения 
абонентов, которая описывается кинетическим уравнением. При этом\linebreak метод 
моделирования ансамбля траекторий для заданной модели движения  
в~\cite{14-g1} не излагался. Этот метод изложен ниже в~разд.~3 статьи. Авторы 
используют ту же системную модель и~обозначения, что позволило в~разд.~4 
продолжить численный анализ исследованного в~\cite{14-g1} сценария 
перемещения взаимодействующих устройств. Исследованы вероятность обрыва 
связи, распределения длительности периодов наличия и~отсутствия связи и~их 
средние значения. В~заключении приведены выводы и~определены задачи 
дальнейших исследований.
  
\section{Модель перемещения устройств}

  Траектории движения устройств моделируются для ситуации, когда 
взаимодействующие при\-емо-пе\-ре\-да\-ющие устройства движутся случайным 
\mbox{образом} в~ограниченной зоне обслуживания~$V$, пред\-став\-ля\-ющей собой 
область в~$M$-мер\-ном пространстве, причем блуждание в~общем случае не 
является стационарным. Считаем, что перемещение устройств сочетает 
целеполагающее поступательное движение и~хаотическое блуждание 
и~определяет\-ся кинетическим уравнением типа Фок\-ке\-ра--План\-ка 
с~заданной скоростью сноса~$v$ и~коэффициентом диффузии~$\alpha$.
  


  Для исследования интерференции на приемнике ассоциированной пары 
взаимодействующих устройств, создаваемой передатчиками других 
ассоциированных пар, устройства разбиты на пары, число пар обозначено~$N$. 
На рис.~1 схематически показаны два варианта перемещения приемников 
в~ассоциированной паре <<пе\-ре\-дат\-чик--при\-ем\-ник>>: в~случае~(\textit{а}) 
приемник движется вместе с~передатчиком, оставаясь на постоянном 
расстоянии~$d$ от него; в~случае~(\textit{б}) приемник движется согласно броуновскому 
движению с~диффузией~$v_{R_x}$ внутри окружности радиуса~$d$ с~центром 
в~точке расположения передатчика. Первый вариант соответствует сценарию, 
когда два абонента не отрываются\linebreak

 { \begin{center}  %fig1
 \vspace*{9pt}
 \mbox{%
 \epsfxsize=77.39mm 
 \epsfbox{gai-1.eps}
 }


\end{center}


\noindent
{{\figurename~1}\ \ \small{Сценарии взаимодействия устройств на основе 
прямых соединений D2D}}
}

%\vspace*{6pt}

\addtocounter{figure}{1}


\noindent
 друг от друга в~процессе перемещения, т.\,е.\ 
траектории их движения зависимы, второй~--- сценарию, когда траектории 
движения приемника и~передатчика взаимно независимы. При этом 
расстояние~$d$ в~первом случае имеет смысл постоянной дистанции между 
абонентами в~движущейся паре, а~во втором~--- максимального расстояния от 
приемника до передатчика в~ассоциированной паре, которое определяется из 
описанной ниже модели распространения сигнала. 

Предложенный в~данной 
статье метод не ограничен выбранными параметрами и~может быть применен 
к~другим моделям движения, описываемым уравнением диффузии. Не 
ограничивая общности метода, предполагается, что мощность сигнала на 
приемнике задается выражением $\phi(r)\hm=Ar^{-\gamma}$ и~зависит от 
расстояния~$r$ при заданной константе~$A$, учитывающей излучаемую 
мощность и~коэффициенты усиления приемной и~передающей антенны, 
и~коэффициенте распространения сигнала~$\gamma$. 
  
  Как и~в~\cite{14-g1}, для оценки SIR далее используется 
формула:
 $$
 \mathrm{SIR}= \fr{\phi(r_0)}{\sum\nolimits_{n=1}^{N-1} \phi(d_n)}\,,
 $$  
где~$r_0$~--- расстояние между приемником и~передатчиком в~исследуемой 
ассоциированной паре; $d_n$~--- расстояние между приемником из 
исследуемой пары и~передатчиком из $n$-й интерферирующей пары, 
$n\hm=1,\ldots ,N\hm-1$. Также задан порог SIR$^*$, определяющий 
минимальное значение SIR, необходимое для поддержания связи 
в~ассоциированной паре. 
  
  В описанной модели случайное расстояние между приемником 
и~передатчиком в~некоторой ассоциированной паре, а также случайные 
расстояния между приемником из этой ассоциированной пары 
и~интерферирующими передатчиками из других пар, работающими на близких 
частотах, исследуются как функции от времени. Эти расстояния определяют 
суммарную интерференцию и~SIR как функционалы на 
траекториях движения взаимодействующих устройств. При падении  
SIR на приемнике исследуемой ассоциированной пары ниже порогового 
значения~SIR$^*$ происходит так называемый <<обрыв связи>>~--- передача 
данных от передатчика к~приемнику в~этой паре прерывается до момента, когда 
 SIR вновь превысит данный порог. Для случайного процесса 
с~независимыми приращениями, описывающего изменение SIR, 
представляют интерес такие  
ве\-ро\-ят\-ност\-но-вре\-мен\-н$\acute{\mbox{ы}}$е характеристики, как 
вероятность перехода процесса в~одно из состояний ниже порогового значения 
SIR$^*$, а~также длительности периодов пребывания процесса в~множестве 
таких состояний. Таким образом, задачей моделирования является нахождение 
вероятности обрыва связи и~распределения длительности периодов наличия 
и~отсутствия связи.

\section{Метод моделирования траекторий движения устройств}

  Известно использование случайных процессов с~независимыми 
приращениями в~экономике для анализа фондового рынка. При этом подходы,\linebreak 
связанные с~моделированием траектории случайного процесса~-- 
временн$\acute{\mbox{о}}$го ряда, отражающего\linebreak флуктуации параметров 
фондового рынка (курсов ценных бумаг или биржевых фондовых  
индексов),~--- использованы при разработке метода моделирования траекторий 
для нестационарного движения устройств. В~\cite{15-g1} был предложен метод 
генерации траектории случайного процесса с~нестационарной функцией 
распределения (ФР) при известном уравнении эволюции этой функции.  
В~\cite{16-g1} этот метод был расширен на генерацию ансамбля траекторий, 
распределение которых эволюционирует в~соответствии с~заданным 
кинетическим уравнением. В~\cite{17-g1} представлена структура 
программного комплекса, реализующего задачу генерации ансамбля случайных 
нестационарных траекторий и~анализа функционалов на них. 

Опишем кратко 
методологию проводимого далее анализа в~обозначениях~\cite{14-g1}, следуя 
указанным работам.
  
  Пусть плотность ФР~$f(x,t)$ приращений 
координат~$x$ положений при\-емо-пе\-ре\-да\-ющих устройств в~момент 
времени~$t$ удовле\-тво\-ря\-ет уравнению Фок\-ке\-ра--План\-ка ($M\hm=1$):
  \begin{multline}
  \fr{\partial f(x,t)}{\partial t} +\fr{\partial}{\partial x}\left( u(x,t) 
  f(x,t)\right)- {}\\
  {}-
\fr{\alpha(t)}{2} \,\fr{\partial^2 f(x,t)}{\partial x^2}=0\,.
  \label{e1-g1}
  \end{multline}
  
  Это уравнение решается численно при заданных начальном и~граничном 
условиях. Параметры уравнения~--- скорость сноса $u(x,t)$ и~нестационарный 
в~общем случае неотрицательный коэффициент диффузии~$\alpha(t)$~--- 
определяются по наблюдаемым значениям временн$\acute{\mbox{о}}$го ряда. Для этого строится 
совместное распределение $F(x,v,t)$ значений временн$\acute{\mbox{о}}$го ряда $x(t)$ и~его 
приращений $v(t)\hm= x(t+1)\hm- x(t)$ по выборке длины, достаточной для 
конструирования такого распределения на заданном уровне 
стационарности~\cite{17-g1}, после чего величины, входящие в~(\ref{e1-g1}), 
определяются по формулам:

\noindent
  \begin{align*}
    f(x,t) &=\displaystyle \int F(x,v,t)\,dv\,;\\
  u(x,t) f(x,t) &= \displaystyle \int vF(x,v,t)\,dv\,;\\
  \alpha(t) &= 2\mathrm{cov}_{x,v}(t) -\fr{d\sigma^2(t)}{dt}\,,
  %  \label{e2-g1}
  \end{align*}
где $\mathrm{cov}_{x,v}(t)$~--- ковариация приращения координаты $x(t)$ и~скорости сноса $u(x,t)$:
\begin{multline*}
\mathrm{cov}_{x,v}(t)=\int xv F(x,v,t)\,dx dv -{}\\
{}- \int xF(x,v,t)\, dx dv  \cdot \int vF(x,v,t)\,dx dv\,,
\end{multline*}
а~дисперсия приращений координат определяется как 
$$
\sigma^2(t)= \int  \left(x-\overline{x}(t)\right)^2 f(x,t)\,dx\,.
$$

При двумерном или трехмерном 
моделировании блуж\-да\-ние по каждой координате считалось независимым. 
В~отсутствие достоверных экспериментальных данных в~данной работе 
параметры сноса и~диффузии были построены по моделям типичных 
нестационарных процессов, описывающих изменения положений случайно 
блуждающих объектов, обсуждаемым в~\cite{17-g1}. Примеры стандартных 
моделей движения, традиционно использующихся для описания перемещения 
абонентов беспроводной сети, приведены в~заключении статьи при 
формулировке задач дальнейших исследований. 

  Величина~$x$ приращения координат для удобства нормирована на~1, т.\,е.\ 
считается, что возможные изменения положений устройств равномерно 
ограничены по времени. Решение строится на временн$\acute{\mbox{о}}$м горизонте $t\hm\in 
[1,T]$ в~дискретном времени с~единичным шагом по времени. На каж\-дом шаге 
$k\hm=1,2,\ldots , T$ для каждого устройства~$n$, $n\hm=1,\ldots , N$, 
генерируется случайное число с~ФР 
$$
F_n(x,t) = F(x,t) = \int\limits_0^x  f(y,t)\,dy\,,
$$
 для чего требуется, чтобы в~результате чис\-лен\-но\-го расчета была 
получена непрерывная строго монотонная функ\-ция. В~част\-ности, если решение 
уравнения~(\ref{e1-g1}) пред\-став\-ле\-но в~виде гистограммы~$f_j(t)$, где~$j$ есть 
номер классового интервала, на которые разбита об\-ласть интегрирования, то 
непрерывная строго монотонная ФР имеет вид: 
  \begin{multline}
  F(x,t) =(Jx-j) f_{j+1}(t) +\sum\limits^j_{l=1} f_l(t)\,,\\
  x\in \left[\fr{j-1}{J}; \fr{j}{J}\right]\,,\ j=1,\ldots , J\,.
  \label{e3-g1}
  \end{multline}
  
  Пусть $\mathbf{R}^n(k)\hm= \left( R_m^n(k)\right)_{m=1,\ldots , M}$~--- 
положение $n$-й точки в~момент времени $t\hm=k$, $k\hm= 1, 2,\ldots ,T$,  
в~некоторой области~$V$ в~$M$-мер\-ном пространстве. Здесь нижний 
индекс~$m$ нумерует координаты в~этом пространстве. 

Алгоритм генерации 
случайных чисел, образующих в~совокупности точки $\mathbf{R}^n(k)\hm= 
\mathbf{R}^n(0)\hm+\sum\nolimits^k_{t=1} x^n(t)$ одной из возможных 
траекторий временн$\acute{\mbox{о}}$го ряда на заданном промежутке времени, состоит 
в~следующем. Генерируется стационарный равномерно распределенный на 
$[0;1]$ ряд чисел $\{y_k\}$ длиной~$T$. Отвечающий ему ряд приращений 
координат~$\left\{ x^n(k)\right\}$ положения передающего устройства~$n$, 
$m\hm=1,\ldots ,N$, с~распределением~$F_n(x,t)$ из~(\ref{e3-g1}) строится по 
формуле обращения соответствующей локальной по времени 
ФР, движущейся в~скользящем окне длины~$T$:
  \begin{multline}
  y_k=F_n\left (x^n(k),k\right)\,,\enskip  
  x^n(k)=F_n^{-1} \left( y_k,k\right)\,,\\
   k=1,\ldots, T\,.
  \label{e4-g1}
  \end{multline}
  
  Генерируя набор равномерно распределенных выборок  
$\{y_k\}_{k=1,\ldots, T}$, для каждого $k\hm=1,\ldots, T$ получаем 
соответствующий набор из~$N$ приращений координат $\left\{ 
x^n(k)\right\}_{n=1,\ldots, N}$, которые дают набор 
траекторий~$N$~передающих устройств на временн$\acute{\mbox{о}}$м горизонте $t\hm\in 
[1,T]$. Эти траектории можно рас\-смат\-ри\-вать как выборку из ансамбля решений 
кинетического уравнения. Полученный набор траекторий представляет 
движение передатчиков в~ассоциирован\-ных парах. 
  
  Моделирование траекторий в~ограниченной области~$V$ при трехмерном 
блуждании $M\hm=3$ осуществлялось следующим образом. В~начальный 
момент задавалось сферически симметричное распределение точек в~кубе со 
стороной~$10d$, спа\-да\-ющее ку\-соч\-но-сте\-пен\-ным образом по закону 
$1/(r\hm+2)$ от центра к~граням куба и~равномерное в~каждом кольце 
шириной~$2d$, где~$d$ есть радиус сферы на рис.~1. Это распределение 
точек~--- начальное условие для решения кинетического уравнения~(1) 
и~одновременно начальные положения для траекторий устройств. Задавалась 
скорость изменения ФР в~(1), а также 
коэффициент~$\alpha$, после чего численно решалось уравнение~(1). Затем по 
описанным выше правилам~(\ref{e3-g1}), (\ref{e4-g1}) вычислялись случайные 
значения последовательных приращений координат для каждой траектории, что 
позволяло строить сами траектории с~учетом граничных усло\-вий. При 
рассмотрении движения в~данной работе задавались усло\-вия идеального 
отражения траектории приемника от границы сферы с~цент\-ром в~точке 
расположения передатчика ас\-со\-ци\-иро\-ван\-ной пары, а~также сферы 
ассоциированной пары от границы куба, определяющего ограниченную 
область~$V$, хотя для области можно рассмотреть и~задачу с~источником 
и~стоком.

\section{Пример расчета характеристик интерференции движущихся 
устройств}

  Траектории движения устройств моделируются для сценария~\cite{14-g1}, 
соответствующего перемещению внутри торгового центра $50\times 50$~м 
абонентов, использующих прямое D2D (device-to-device) под\-клю\-че\-ние. Моделирование 
проведено на временном горизонте $t\hm\in [1,T]$ при $T\hm=10^5$ для 
различной плотности устройств ($N\hm = 10$, 30, 50, 100) и~различных средних 
скоростей сноса ($v\hm = 1$, 3, 5, 10, 40~м/c). Были выбраны типичные для 
D2D-со\-еди\-не\-ний параметры системной модели ($\alpha\hm = 2$, $A\hm = 
1$, $\gamma\hm = 3$, $d\hm = 5$~м, $v_{R_x}\hm= 1$~м/с, $\mathrm{SIR}^*\hm= 0{,}01$), 
выполнялось усреднение по реализациям. При моделировании рассматривалось 
число~$C(T)$ обрывов связи и~вероятность~$P^{-}$ обрыва связи между 
взаимодействующими устройствами:
$$
P^- = \lim\limits_{T\to \infty} 
\left(\fr{C(T)}{T}\right)\,,
$$
 т.\,е.\ выброса SIR ниже порогового значения SIR$^*$. 

На рис.~2,\,\textit{а} представлены графики ве\-ро\-ят\-ности~$P^-$ обрыва связи 
в~зависимости от средней скорости~$v$ передвижения устройств 
и~числа~$N$~пар взаимодействующих устройств. Следует отметить,\linebreak что 
увеличение как числа пар устройств, так и~средней скорости их перемещения 
оказывает не\-гативное влияние на устойчивость связи. В~обоих слу\-чаях можно 
отметить логарифмический рост\linebreak ве\-ро\-ят\-ности потери связи. Средние значения 
длительностей $\overline{\tau}^+$ периода наличия и~$\overline{\tau}^-$ 
периода отсутствия связи как функции от средней скорости~$v$ устройств 
и~числа~$N$ взаимодействующих пар представлены на рис.~2,\,\textit{б} и~2,\,\textit{в}. 


  
  
  Анализируя представленные данные для периода наличия связи, следует 
отметить, что среднее значение~$\overline{\tau}^+$ рассматриваемой метрики 
демонстрирует экспоненциальное падение с~увеличением как средней скорости 
перемещения устройств, так и~чис\-ла пар взаимодействующих устройств. Так, 
при значениях скорости перемещения, со\-от\-вет\-ст\-ву\-ющих ско\-рости пешеходов 
(3--5~м/c), средняя длительность~$\overline{\tau}^+$ периода наличия связи 
может достигать~9--10~с при~10~одновременных уста\-нов\-лен\-ных прямых 
соединениях. Однако при скоростях, соответствующих средней ско\-рости 
перемещения автомобилей (30--40~м/c), величина~$\overline{\tau}^+$ не 
превышает~1~с. 



\pagebreak

\end{multicols}

\begin{figure*} %fig2
\vspace*{1pt}
 \begin{center}
 \mbox{%
 \epsfxsize=163.477mm 
 \epsfbox{gai-2.eps}
 }
\end{center}
\vspace*{-11pt}
\Caption{Вероятность $P^-$ обрыва связи~(\textit{а}),
средняя длительность периодов наличия связи~$\overline{\tau}^+$~(\textit{б})
и~средняя длительность периодов отсутствия связи $\overline{\tau}^-$~(\textit{в}):
левый столбец~--- функция от скорости~$v$ 
($N\hm=10$); правый столбец~--- функция от числа пар~$N$ ($v\hm=5$~м/с)}
\vspace*{12pt}
\end{figure*}

\begin{multicols}{2}


  
  Период наличия связи определяет длительность интервала времени до 
прерывания соединения
 для приложений в~реальном времени. Для таких 
приложений при условии поступления запроса на уста\-нов\-ле\-ние соединения 
в~случайный момент времени в~период наличия связи этот интервал совпадает 
с~длительностью периода до первого обрыва связи и,~как отмечено  
в~\cite{14-g1}, имеет показательное распределение со средним значением, 
представленным на рис.~2,\,\textit{б}. Для кэшируемых приложений интервал 
времени до прерывания соединения определяется длительностью как периодов 
наличия связи, так и~периодов отсутствия связи. Если последняя не превышает 
некоторого порогового значения~$\tau^*$, характерного для конкретного 
приложения, то интервал времени до прерывания соединения складывается из 
нескольких последовательных периодов наличия и~отсутствия связи. 

Анализируя графики на рис.~2,\,\textit{в}, следует от\-метить, что 
зависимость средней длительности пери\-о\-да отсутствия связи демонстрирует 
разный характер изменения при увеличении ско\-рости перемещения устройств 
и~числа пар взаимодействующих устройств. Так, при увеличении скорости 
перемещения период отсутствия связи экспоненциально уменьшается. При 
увеличении числа пар взаимодействующих устройств наблюдается линейный 
рост средней длительности периода отсутствия \mbox{связи}.

\section{Заключение}

  В представленной работе предложен метод моделирования траекторий 
движения при\-емо-пе\-ре\-да\-ющих устройств в~беспроводной сети пятого 
поколения с~технологией прямого взаимодействия устройств. С~применением 
предложенного метода проведено имитационное моделирование и~исследованы 
вероятность обрыва связи и~средние значения длительности периодов наличия 
и~отсутствия связи. Полученный в~работе результат по экспоненциальному 
уменьшению средней длительности периода отсутствия связи при увеличении 
ско\-рости перемещения устройств имеет важное практическое значение: это 
уменьшение компенсируется снижением средней длительности периода 
отсутствия связи, значительно увеличивая вероятность того, что в~течение 
этого времени не произойдет прерывания со\-еди\-не\-ния. 

Задачей дальнейших 
исследований остается анализ устойчивости соединения в~зависимости от 
размера буфера оборудования пользователя для кэшируемых приложений. 

В~случае приложений в~реальном времени, для которых вероятность 
прерывания соединения не зависит от буферизации, интерес представляет 
исследование эволюции старших моментов функционала SIR, определяющих 
надежность соединения. 

Еще одной задачей дальнейших исследований является 
анализ характеристик интерференции при моделировании ансамбля траекторий 
для различных моделей движения, применимых для описания перемещения 
абонентов~\cite{18-g1}: броуновское движение (Brownian Motion), движение 
в~случайном на\-прав\-ле\-нии (Random Direction Motion) и~так на\-зы\-ва\-емый 
<<полет Леви>> (L$\acute{\mbox{e}}$vy Flight).

{\small\frenchspacing
 {%\baselineskip=10.8pt
 \addcontentsline{toc}{section}{References}
 \begin{thebibliography}{99}
\bibitem{1-g1}
\Au{Mouly M., Pautet M.\,B.} The GSM system for mobile communications.~--- Washington, DC, 
USA: Telecom Publishing, 1992. 701~p.
\bibitem{2-g1}
\Au{Вишневский В.\,М., Портной~С.\,Л., Шахнович~И.\,В.} Энциклопедия WiMAX. Путь 
к~4G.~--- М.: Техносфера, 2009. 472~с.
\bibitem{3-g1} WCDMA for UMTS: Radio access for third generation mobile communications~/
Eds. H.~Holma, A.~Toskala.~--- Chichester, UK: John Wiley \& Sons, 2005. 478~p.
\bibitem{4-g1}
\Au{Sesia S., Baker~M., Toufik~I.} LTE~--- the UMTS long term evolution: From theory to  
practice.~--- Chichester, UK: John Wiley \& Sons, 2011. 792~p.
\bibitem{5-g1}
\Au{Отт Г.} Методы подавления шумов и~помех в~электронных системах~/
Пер. с~англ.~--- М.: Мир, 1979.  318~с.
(\Au{Ott~H.\,W.} {Noise reduction techniques in electronic systems}.~--- 
New York, NY, USA: John Wiley \& Sons, 1979. 294~p.)

\bibitem{6-g1}
\Au{Rong Z., Rappaport~T.\,S.} Wireless communications: Principles and practice.~--- 1st ed.~--- 
Upper Saddle River, NJ, USA: Prentice Hall. 641~p.
\bibitem{7-g1}
\Au{Baccelli F., Blaszczyszyn~B.} Stochastic geometry and wireless networks~// Found. 
Trends Netw., 2010. Vol.~3. No.\,3-4. P.~249--449
(doi: 10.1561/1300000006); Vol.~4. No.\,1-2. P.~1--312 (doi: 10.1561/1300000026).

\bibitem{8-g1}
\Au{Haenggi M.} Stochastic geometry for wireless networks.~--- Cambridge: Cambridge University 
Press, 2012. 298~p.
\bibitem{9-g1}
\Au{Samuylov A., Ometov~A., Begishev~V., Kovalchukov~R., Moltchanov~D., Gaidamaka~Yu., 
Samouylov~K., Andreev~S., Koucheryavy~Y.} Analytical performance estimation of  
network-assisted D2D communications in urban scenarios with rectangular cells~// 
T.~Emerg. Telecommun.~T., 2015. Vol.~28. Iss.~2. P.~2999-1--2999-15.
doi: 10.1002/ett.2999.
\bibitem{10-g1}
\Au{Samuylov A. Gaidamaka~Yu., Moltchanov~D., Andreev~S., Koucheryavy~Y.} Random 
triangle: A~baseline model for interference analysis in heterogeneous networks~// IEEE 
T.~Veh. Technol., 2015. Vol.~65. Iss.~8. P.~6778--6782.
doi: 10.1109/TVT.2016.2596324.
\bibitem{11-g1}
\Au{Гайдамака Ю.\,В., Самуйлов~А.\,К.} Метод расчета характеристик интерференции двух 
взаимо\-дей\-ст\-ву\-ющих устройств в~беспроводной гетерогенной сети~// Информатика и~её 
применения, 2015. Т.~9. Вып.~1. С.~9--14. doi: 10.14357/19922264150102.
\bibitem{12-g1}
\Au{Гайдамака Ю.\,В., Андреев~С.\,Д., Сопин~Э.\,С., Самуйлов~К.\,Е., Шоргин~С.\,Я.} 
Анализ характеристик интерференции в~модели взаимодействия устройств с~учетом среды 
распространения сигнала~// Информатика и~её применения, 2016. Т.~10. 
Вып.~4. С.~2--10. doi:  10.14357/19922264160401.
\bibitem{13-g1}
\Au{Orlov Yu.\,N., Fedorov~S.\,L., Samuylov~A.\,K., Gaidamaka~Yu.\,V., Molchanov~D.\,A.} 
Simulation of devices mobility to estimate wireless channel quality metrics in 5G net-\linebreak\vspace*{-12pt}

\pagebreak

\noindent
works~// AIP 
Conf. Proc., 2017. Vol.~1863. P.~090005-1--090005-3. doi: 10.1063/1.4992270.
\bibitem{14-g1}
\Au{Гайдамака Ю.\,В., Орлов~Ю.\,Н., Молчанов~Д.\,А., Самуйлов~А.\,К.} Моделирование 
отношения сигнал/ин\-тер\-фе\-рен\-ция в~мобильной сети со случайным блуж\-да\-ни\-ем 
взаимодействующих устройств~// Информатика и~её применения, 2017. Т.~11. Вып.~2. 
С.~50--58. doi:  10.14357/19922264170206.
\bibitem{15-g1}
\Au{Босов А.\,Д., Кальметьев~Р.\,Ш., Орлов~Ю.\,Н.} Мо\-де\-ли\-рование нестационарного 
временного ряда с~за\-данными свойствами выборочного распределения~// Математическое 
моделирование, 2014. Т.~26. №\,3. С.~97--107.
\bibitem{16-g1}
\Au{Орлов Ю.\,Н., Федоров~С.\,Л.} Генерация нестационарных траекторий временного ряда 
на основе уравнения Фок\-ке\-ра--План\-ка~// Труды МФТИ, 2016. Т.~8. №\,2. С.~126--133.
\bibitem{17-g1}
\Au{Орлов Ю.\,Н., Федоров~С.\,Л.} Методы численного моделирования процессов 
нестационарного случайного блуждания.~--- М.: МФТИ, 2016. 112~с.
\bibitem{18-g1}
\Au{Orsino A., Moltchanov~D., Gapeyenko~M., Samuylov~A., Andreev~S., Militano~L., 
Araniti~G., Koucheryavy~Y.} Direct connection on the move: Characterization of user mobility in 
cellular-assisted D2D systems~// IEEE Veh. Technol. Mag., 2016. Vol.~11. Iss.~3. 
P.~38--48. doi:  10.1109/MVT.2016.2550002.
 \end{thebibliography}

 }
 }

\end{multicols}

\vspace*{-3pt}

\hfill{\small\textit{Поступила в~редакцию 07.09.17}}

\vspace*{8pt}

%\newpage

%\vspace*{-24pt}

\hrule

\vspace*{2pt}

\hrule

%\vspace*{8pt}


\def\tit{METHOD OF~MODELING INTERFERENCE CHARACTERISTICS 
IN~HETEROGENEOUS FIFTH GENERATION WIRELESS 
NETWORKS WITH~DEVICE-TO-DEVICE COMMUNICATIONS}

\def\titkol{Method of~modeling interference characteristics 
in~heterogeneous fifth generation wireless 
networks with~D2D %device-to-device 
communications}

\def\aut{Yu.~Gaidamaka$^{1,2}$, K.~Samouylov$^{1,2}$, and~S.~Shorgin$^2$}

\def\autkol{Yu.~Gaidamaka, K.~Samouylov, and~S.~Shorgin}

\titel{\tit}{\aut}{\autkol}{\titkol}

\vspace*{-9pt}


\noindent
$^1$Peoples' Friendship University of Russia (RUDN University), 6~Miklukho-Maklaya Str., 
Moscow 117198, Russian\linebreak
$\hphantom{^1}$Federation

\noindent
$^2$Institute of Informatics Problems, Federal Research Center ``Computer Science 
and Control'' of the Russian\linebreak
 $\hphantom{^1}$Academy of 
Sciences, 44-2~Vavilov Str., Moscow 119333, Russian Federation


\def\leftfootline{\small{\textbf{\thepage}
\hfill INFORMATIKA I EE PRIMENENIYA~--- INFORMATICS AND
APPLICATIONS\ \ \ 2017\ \ \ volume~11\ \ \ issue\ 4}
}%
 \def\rightfootline{\small{INFORMATIKA I EE PRIMENENIYA~---
INFORMATICS AND APPLICATIONS\ \ \ 2017\ \ \ volume~11\ \ \ issue\ 4
\hfill \textbf{\thepage}}}

\vspace*{3pt}


\Abste{The paper shows the construction of the model of the moving of 
interacting devices in heterogeneous wireless networks of the fifth generation with 
the help of the kinetic equation taking into account a~given average speed of the 
devices, their spatial density, and the maximum allowable communication radius. 
A~method for generating trajectories is proposed where the transceivers move 
randomly and the walk is not stationary in general. This is the feature of the study 
which distinguishes the proposed model from previously known models. 
Interference characteristics, including signal--interference ratio (SIR), are studied 
in the form of a~time-continuous random process, the problem of calculating these 
characteristics is proposed to be solved by simulations. It is shown that such 
analysis makes it possible to investigate the probabilistic characteristics of the 
interaction of devices such as signal interruption probability for the 
receiver--transmitter pair, the random variables for the duration of
 the availability period, and the period of absence of communication.}

\KWE{wireless heterogeneous network; signal--interference ratio; 
device-to-device (D2D); motion model; kinetic equation; trajectories generation; network 
efficiency indicators} 


  \DOI{10.14357/19922264170401} 

\vspace*{-3pt}

\Ack
\noindent
This work was financially supported by the Russian Science Foundation 
(project No.\,16-11-10227).




\vspace*{2pt}

  \begin{multicols}{2}

\renewcommand{\bibname}{\protect\rmfamily References}
%\renewcommand{\bibname}{\large\protect\rm References}

{\small\frenchspacing
 {%\baselineskip=10.8pt
 \addcontentsline{toc}{section}{References}
 \begin{thebibliography}{99}
\bibitem{1-g1-1}
\Aue{Mouly, M., and M.\,B.~Pautet.} 1992. \textit{The GSM system for mobile communications}. 
Washington, DC: Telecom Publishing. 701~p.
\bibitem{2-g1-1}
\Aue{Vishnevskiy, V.\,M., S.\,L.~Portnoy, and I.\,V.~Shakhnovich.} 2009. \textit{Entsiklopediya 
WiMAX. Put' k~4G} [Encyclopedia of WiMAX. The way to 4G]. Moscow: Tekhnosfera. 472~p.
\bibitem{3-g1-1}
Holma, H., and A.~Toskala, eds. 2005. \textit{WCDMA for UMTS: Radio access for third 
generation mobile communications}. Chichester, UK: John Wiley \& Sons. 478~p.
\bibitem{4-g1-1}
\Aue{Sesia, S., M.~Baker, and I.~Toufik}. 2011. \textit{LTE~--- the UMTS long term evolution: From 
theory to practice}. Chichester, UK: John Wiley \& Sons. 792~p.
\bibitem{5-g1-1}
\Aue{Ott, H.\,W.} 1979. \textit{Noise reduction techniques in electronic systems}. 
New York, NY: John Wiley \& Sons. 294~p.
\bibitem{6-g1-1}
\Aue{Rong, Z., and T.\,S.~Rappaport.} 1996. \textit{Wireless communications: Principles and 
practice}. 1st ed. Upper Saddle River, NJ: Prentice Hall. 641~p.
\bibitem{7-g1-1}
\Aue{Baccelli, F., and B.~Blaszczyszyn.} 2010. Stochastic geometry and wireless networks. 
\textit{Found. Trends Netw.} 3(3-4):249--449 (doi: 10.1561/1300000006); 4(1-2):1--312 
(doi: 10.1561/1300000026).
\bibitem{8-g1-1}
\Aue{Haenggi, M.} 2012. \textit{Stochastic geometry for wireless networks}. Cambridge: 
Cambridge University Press. 298~p.
\bibitem{9-g1-1}
\Aue{Samuylov, A., A.~Ometov, V.~Begishev, R.~Kovalchukov, D.~Moltchanov, 
Yu.~Gaidamaka, K.~Samouylov, S.~And\-re\-ev, and Y.~Koucheryavy.} 2015. Analytical 
performance estimation of network-assisted D2D communications in urban scenarios with 
rectangular cells. \textit{T.~Emerg. Telecommun.~T.} 28(2):2999-1--2999-15. 
doi: 10.1002/ett.2999. 
\bibitem{10-g1-1}
\Aue{Samuylov, A., Yu.~Gaidamaka, D.~Moltchanov, S.~Andreev, and Y.~Koucheryavy.} 2015. 
Random triangle: A~baseline model for interference analysis in heterogeneous networks. 
\textit{IEEE T. Veh. Technol.} 65(8):6778--6782. doi: 10.1109/TVT.2016.2596324.
\bibitem{11-g1-1}
\Aue{Gaydamaka, Yu.\,V., and A.\,K.~Samuylov.} 2009. Metod rascheta kharakteristik 
interferentsii dvukh vza\-imo\-dey\-st\-vu\-yushchikh ustroystv v~besprovodnoy geterogennoy seti [The 
method of calculation of the characteristics of the interference of two interacting devices in 
a~wireless heterogeneous network]. \textit{Informatika i~ee Primeneniya~--- Inform. Appl.} 
9(1):9--14. doi: 10.14357/19922264150102.
\bibitem{12-g1-1}
\Aue{Gaydamaka, Yu.\,V., S.\,D.~Andreev, E.\,S.~Sopin, K.\,E.~Samuylov, and S.\,Ya.~Shorgin.} 
2016. Analiz kharakteristik interferentsii v~modeli vzaimodeystviya ustroystv s~uchetom sredy 
rasprostraneniya signala [Analysis of the characteristics of the interference in the model of 
interaction between devices taking into account the signal propagation environment]. 
\textit{Informatika i~ee Primeneniya~--- Inform. Appl.} 10(4):2--10. doi: 
10.14357/19922264160401.
\bibitem{13-g1-1}
\Aue{Orlov, Yu.\,N., S.\,L.~Fedorov, A.\,K.~Samuylov, Yu.\,V.~Gaidamaka, and 
D.\,A.~Molchanov.} 2017. Simulation of devices mobility to estimate wireless channel quality 
metrics in 5G networks. \textit{AIP Conf. Proc.} 1863:090005-1--090005-3. 
doi: 10.1063/1.4992270.
\bibitem{14-g1-1}
\Aue{Gaydamaka, Yu.\,V., Yu.\,N.~Orlov, D.\,A.~Molchanov, and A.\,K.~Samuylov}. 2017. 
Modelirovanie otnosheniya signal/interferentsiya v~mobil'noy seti so sluchaynym bluzhdaniem 
vzaimodeystvuyushchikh ustroystv [Modeling the signal--interference ratio in a mobile network with 
moving devices]. \textit{Informatika i~ee Primeneniya~--- Inform. Appl.} 11(2):50--58. 
doi:  10.14357/19922264170206.
\bibitem{15-g1-1}
\Aue{Bosov, A.\,D., R.\,Sh.~Kalmetiev, and Yu.\,N.~Orlov.} 2014. Modelirovanie 
nestatsionarnogo vremennogo ryada s~zadannymi svoystvami vyborochnogo raspredeleniya 
[Sample distribution function construction for non-stationary time-series forecasting]. 
\textit{Matem. mod.} [Mathematical Simulation]. 26(3):97--107. %10.20948/prepr-2016-101.
\bibitem{16-g1-1}
\Aue{Orlov, Yu.\,N. and S.\,L.~Fedorov}. 2016. Generatsiya ne\-sta\-tsi\-o\-nar\-nykh traektoriy 
vremennogo ryada na osnove uravneniya Fokkera--Planka [Generation of non-stationary 
trajectories of the time series based on the Fokker--Planck equation]. 
\textit{Trudy MFTI} 
[MIPT Proc.] 8(2):126--133. %doi: 10.20948/prepr-2017-36.
\bibitem{17-g1-1}
\Aue{Orlov, Yu.\,N., and S.\,L.~Fedorov.} 2016. \textit{Metody chislennogo modelirovaniya 
protsessov nestatsionarnogo sluchaynogo bluzhdaniya} 
[Methods of a~numerical simulation of nonstationary random walk]. Moscow: MIPT. 112~p.
\bibitem{18-g1-1}
\Aue{Orsino, A., D.~Moltchanov, M.~Gapeyenko, A.~Samuylov, S.~Andreev, L.~Militano, 
G.~Araniti, and Y.~Koucheryavy.} 2016. Direct connection on the move: Characterization of user 
mobility in cellular-assisted D2D systems. \textit{IEEE Veh. Technol. Mag.} 11(3):38--48. 
doi:  10.1109/MVT.2016.2550002.
\end{thebibliography}

 }
 }

\end{multicols}

\vspace*{-6pt}

\hfill{\small\textit{Received September 7, 2017}}

%\vspace*{-10pt}

\Contr

\noindent
\textbf{Gaidamaka Yuliya V.} (b.\ 1971)~--- Candidate of Science (PhD) in physics and 
mathematics, associate professor, Peoples' Friendship University of Russia (RUDN University), 
6~Miklukho-Maklaya Str., Moscow 117198, Russian Federation; senior scientist, Institute of 
Informatics Problems, Federal Research Center ``Computer Science and Control'' of the Russian 
Academy of Sciences, 44-2~Vavilov Str., Moscow 119333, Russian Federation; 
\mbox{gaydamaka\_yuv@rudn.university}

\vspace*{3pt}

\noindent
\textbf{Samouylov Konstantin E.} (b.\ 1955)~--- Doctor of Science in technology, professor, 
Head of Department, 
Director of Institute of Applied Mathematics and Telecommunications, 
Peoples' Friendship University of Russia (RUDN University), 6~Miklukho-Maklaya Str., Moscow 
117198, Russian Federation; senior scientist,
 Institute of Informatics Problems, Federal Research 
Center ``Computer Science and Control'' of the Russian Academy of Sciences, 44-2~Vavilov Str., 
Moscow 119333, Russian Federation; \mbox{samuylov\_ke@rudn.university}

\vspace*{3pt}

\noindent
\textbf{Shorgin Sergey Ya.} (b.\ 1952)~--- Doctor of Science in physics and mathematics, 
professor; Deputy Director, Federal Research Center ``Computer Science and Control'' of the 
Russian Academy of Sciences (FRC CRC RAS); principal scientist, Institute of Informatics 
Problems, FRC CRC RAS; 44-2~Vavilov Str., Moscow 119333, Russian Federation; 
\mbox{sshorgin@ipiran.ru}

\label{end\stat}


\renewcommand{\bibname}{\protect\rm Литература}   %1 Gai+sam+shorgin  
\def\ld{\ldots}
\def\d{\overline d}

\def\stat{raz-rus}

\def\tit{СТАЦИОНАРНЫЕ ХАРАКТЕРИСТИКИ СИСТЕМЫ ОБСЛУЖИВАНИЯ
С~ИНВЕРСИОННЫМ ПОРЯДКОМ ОБСЛУЖИВАНИЯ, ВЕРОЯТНОСТНЫМ
ПРИОРИТЕТОМ И~ГРУППОВЫМ ПОСТУПЛЕНИЕМ РАЗНОРОДНЫХ ЗАЯВОК$^*$}

\def\titkol{Стационарные характеристики системы обслуживания
с~инверсионным порядком обслуживания} %, вероятностным приоритетом и~групповым поступлением разнородных заявок}

\def\aut{Р.\,В.~Разумчик$^1$}

\def\autkol{Р.\,В. Разумчик}

\titel{\tit}{\aut}{\autkol}{\titkol}

\index{Разумчик Р.\,В.}
\index{Razumchik R.\,V.}



{\renewcommand{\thefootnote}{\fnsymbol{footnote}} \footnotetext[1]
{Работа выполнена при поддержке Российского научного фонда (проект 16-11-10227).}}


\renewcommand{\thefootnote}{\arabic{footnote}}
\footnotetext[1]{Институт проблем информатики Федерального исследовательского центра <<Информатика 
и~управ\-ле\-ние>> Российской академии наук; Российский
университет дружбы народов, \mbox{rrazumchik@ipiran.ru}}
%; \mbox{razumchik\_rv@rudn.university}}

\vspace*{-16pt}



\Abst{Статья посвящена исследованию стационарных характеристик
однолинейных систем массового обслуживания (СМО)
со специальными дисциплинами обслуживания.
Рассматриваемая дисциплина~--- инверсионный
порядок обслуживания с~вероятностным приоритетом.
Основные результаты для данной дисциплины
были получены в~предположениях, что в~систему поступает
пуассоновский поток или поток фазового типа и~времена
обслуживания имеют произвольное распределение.
Существенным также было предположение о~независимости процесса поступления заявок
от состояния системы. Здесь же показано, что это
предположение может быть определенным образом ослаблено.
Рассматривается система с~одним прибором, очередью неограниченной
емкости и~неординарным пуассоновским потоком,
интенсивность которого может зависеть от общего числа заявок, находящихся
в системе в~момент поступления группы, причем
размер  поступающей группы и~размеры заявок
в~ней имеют совместное произвольное распределение.
Получены аналитические соотношения, позволяющие
рассчитывать совместное стационарное распределение числа заявок
в~системе и~остаточных времен обслуживания.
Кроме того, в~терминах преобразований Лап\-ла\-са--Стилть\-еса (ПЛС)
находятся стационарные распределения
случайных величин, связанных с~временем ожидания начала обслуживания
и~пребывания заявки в~системе.}


\KW{инверсионный порядок обслуживания; вероятностный приоритет; 
неординарный входящий поток}

%\vspace*{-8pt}

\DOI{10.14357/19922264170402} 


\vskip 10pt plus 9pt minus 6pt

\thispagestyle{headings}

\begin{multicols}{2}

\label{st\stat}

\section{Введение}

Эта статья развивает результаты работ~\cite{n0,n1,n2,n3,n4,n5} по исследованию
стационарных характеристик однолинейных СМО
$M/G/1$ с~инверсионным порядком обслуживания и~вероятностным приоритетом.
Основные результаты этих работ были получены в~предположении, что входящий в~систему поток
является простейшим. 

Как было продемонстрировано в~\cite{nm1},
некоторые из этих результатов допускают обобщение на случай
потоков фазового типа, которые не являются рекуррентными и,~таким образом,
могут быть более привлекательными при моделировании процессов в~реальных технических системах.
Несмотря на свою общность, модель потока фазового типа
подразумевает, что процесс поступления заявок в~сис\-те\-му не зависит от состояния самой системы.
Тем\linebreak самым в~стороне осталась задача обобщения результатов на случай,
когда такая зависимость присутствует.
Не останавливаясь на возможных практических интерпретациях
связей между входящим\linebreak потоком и~состоянием системы (см.~\cite{gg2}),
 отметим лишь, что исследованию СМО с~такими зависимостями посвящено достаточно много работ
(см., например,~\cite{gg1,gg3,gg4,gg5} и~ссылки в~них). Обычно
предполагается, что в~систему поступает пуассоновский поток второго рода
(т.\,е.\ интенсивность потока зависит от общего числа заявок, находящихся в~сис\-те\-ме). 
Если же
допускается поступление групп заявок, то обычно предполагается, что размеры (остаточные времена обслуживания)
заявок в~группе являются независимыми случайными величинами (не зависящими также и~от размера группы).

В~данной статье эти предположения ослабляются следующим образом:
рассматривается неординарный пуассоновский поток,
интенсивность которого может зависеть от общего числа заявок, находящих\-ся
в~системе в~момент поступления группы, причем
размер  поступающей группы и~размеры\linebreak заявок в~ней имеют совместное произвольное распределение.
Для однолинейной СМО неограниченной емкости с~инверсионным порядком обслуживания 
и~вероятностным приоритетом при таком\linebreak входя\-щем потоке
решена задача отыскания совместного стационарного
распределения числа заявок в~сис\-те\-ме и~их остаточного времени обслуживания,
а~также стационарных распределений (в~терминах ПЛС),
связанных с~временем пребывания заявки в~системе.

\vspace*{-4pt}

\section{Описание системы}

\vspace*{-2pt}

Рассмотрим однолинейную СМО с~очередью неограниченной емкости,
на вход которой поступает групповой пуассоновский поток заявок с~переменной 
интенсивностью~$\lambda_n$, зависящей от числа заявок~$n$, находящихся в~системе.
Через $B_k(x_1,\ld,x_k)$ будем обозначать вероятность того, что в~поступившей
группе будет~$k$ заявок, причем первая заявка будет иметь
длину меньше~$x_1$, вторая~--- меньше $x_2$ и~т.\,д.;
через $b_k(x_1,\ld,x_k)\hm=
\partial^k B_k(x_1,\ld,x_k)/(\partial x_1\cdots \partial x_k)$~--- 
совместную плотность вероятностей.
Длины заявок в~различных группах независимы между собой.

Определим дисциплину обслуживания сле\-ду\-ющим образом:
в~момент прихода очередной группы заявок замеряется
остаточное время обслуживания (в дальнейшем будем называть его
длиной) первой заявки из группы.
Пусть она равна~$x$. Эта длина сравнивается с~остаточной длиной
заявки, находящейся на обслуживании. Если
оставшееся время обслуживания заявки на приборе равно~$y$,
то с~вероятностью $d(x,y)$ первая заявка из группы становится на
обслуживание, за ней (в очередь) становятся остальные заявки группы,
затем обслуживавшаяся ранее и~остальные заявки, прежде находившиеся в~системе.
С~вероятностью $\d(x,y)\hm=1\hm-d(x,y)$ обслуживавшаяся ранее заявка продолжает
обслуживаться на приборе, вновь поступившие заявки становятся (в~очередь) за ней,
затем остальные находившиеся прежде в~системе заявки.
Недообслуженные заявки дообслуживаются.

Поскольку  интерес представляют стационарные характеристики
этой системы, всюду в~дальнейшем будем предполагать,
что стационарное распределение существует.
Критерий его существования следует
из условия конечности среднего времени
возвращения в~некоторое состояние (см.\ соотношение~\eqref{uns}).
Однако в~случае произвольных функций $B_k(x_1,\ld,x_k)$
выписать его в~простом виде не удается.

\vspace*{-4pt}

\section{Марковский случайный процесс}

\vspace*{-2pt}

Обозначим через $\nu(t)$ число заявок в~системе
в момент~$t$, а через $\vec\xi(t)\hm=
(\xi_{1}(t),\ldots,\xi_{\nu(t)}(t))$~---
вектор, координатой~$\xi_{1}(t)$ которого
является остаточное время обслуживания
заявки, находящейся в~этот момент на приборе,
$\xi_{2}(t)$~--- первой заявки в~очереди$,\ldots,$ $\xi_{\nu(t)-1}(t)$~---
последней, \mbox{$(\nu(t)-1)$-й} заявки в~очереди.
При $\nu(t)\hm=0$ вектор~$\vec\xi(t)$
не определяется.
Тогда $\eta(t)\hm=(\nu(t),\vec\xi(t))$ представляет
собой марковский процесс, описывающий
поведение числа заявок в~рассматриваемой системе.

\vspace*{-4pt}

\section{Система интегродифференциальных уравнений}

\vspace*{-2pt}

Обозначим через $p_{0}\hm=\lim\nolimits_{t\to\infty}
{\bf P}\{\nu(t)\hm=0\}$,
$P_{n}(x_1,\ldots,x_{n})
\hm= \lim\nolimits_{t\to\infty} {\bf P}\{\nu(t)\hm=n,\,
\xi_{1}(t)\hm<x_{1},\ldots,\xi_{n}(t)<x_{n}\}$, $n \hm\ge 1$,
стационарное распределение процесса~$\eta(t)$,
а~через $p_n(x_1,\ld,x_n)$~---\linebreak стационарную плот\-ность
вероятности того, что в~сис\-те\-ме~$n$~заявок, причем заявка на приборе имеет
длину~$x_1$, первая в~очереди~--- длину~$x_2$ и~т.\,д.

Выпишем систему интегродифференциальных 
уравнений, которой удовлетворяют $p_n(x_1,\ldots,x_{n})$.
Воспользовавшись методом исключения со\-сто\-яний (см., например,~\cite{n1,n2,n3,n4,n5}),
получаем следующие соотношения:
\begin{multline}
\label{e1}
-p'_1(x) = \lambda_0 p_0 \left( 
\vphantom{\mathop{\int\!\cdots\!\int}\limits_{y_1,\ld,y_{k-1}>0}}
b_1(x) +{}\right.\\
\left.{}+
\sum\limits_{k=2}^\infty\!\!
\mathop{\int\!\cdots\!\int}\limits_{y_1,\ld,y_{k-1}>0}\!\!
b_k\left(y_1,\ld,y_{k-1},x\right)\, dy_1\cdots dy_{k-1}
\!\right)-{}\\
{}- \lambda_1 p_1(x) +
\lambda_1 p_1(x)\times{}\\
{}\times \sum\limits_{k=1}^\infty
\mathop{\int\!\cdots\!\int}\limits_{y_1,\ld,y_{k}>0}
b_k\left(y_1,\ld,y_{k}\right)\, d\left(y_1,x\right)  dy_1\cdots dy_{k}
+{}\\
{}+\lambda_1 \int\limits_0^\infty p_1(y) b_1(x) \d(x,y) \, dy
+{}\\
{}+
\lambda_1 \sum\limits_{k=2}^\infty
\mathop{\int\!\cdots\!\int}\limits_{y_1,\ld,y_{k}>0}
p_1\left(y_1\right)  b_k\left(y_2,\ld,y_{k},x\right) \times{}\\
{}\times\d\left(y_2,y_1\right)
\, dy_1\cdots dy_{k}\,;
\end{multline}

\vspace*{-12pt}

\noindent
\begin{multline*}
-p'_n\left(x_1,\ld,x_n\right)
=  \lambda_0 p_0 \left(
\vphantom{\mathop{\int\!\cdots\!\int}\limits_{y_1,\ld,y_{k-1}>0}\sum\limits^\infty}
b_n\left(x_1,\ld,x_n\right) +{}\right.\\
{}+ \sum\limits_{k=n+1}^\infty
\mathop{\int\ld\int}\limits_{y_1,\ld,y_{k-n}>0}\!\!\!
b_k\left(y_1,\ld,y_{k-n},x_1,\ld\right.
\end{multline*}

\noindent
\begin{multline}
\left.\left.{}\ld,x_{n}\right)\, dy_1\cdots dy_{k-n}
\vphantom{\mathop{\int\!\cdots\!\int}\limits_{y_1,\ld,y_{k-1}>0}\sum\limits^\infty}
\!\right)\!
+\!
\sum\limits_{k=1}^{n-1}\! \lambda_k d\left(x_1,x_{n-k+1}\right) 
\times{}\\
{}\times b_{n-k}\left(x_1,\ld,x_{n-k}\right) p_k\left(x_{n-k+1},\ld,x_{n}\right)
+{}
\\
{}+
 \sum\limits_{k=1}^{n-1} \lambda_k \d(x_2,x_{1})
b_{n-k}\left(x_2,\ld,x_{n-k+1}\right)\times{}\\
{}\times
p_k\left(x_1,x_{n-k+2},\ld,x_{n}\right) -
\lambda_n p_n\left(x_1,\ld,x_n\right) +{}
\\
{}+
 \sum\limits_{k=1}^{n} \int\limits_0^\infty \lambda_k
\d\left(x_{1},y\right) b_{n-k+1}\left(x_1,\ld,x_{n-k+1}\right)\times{}\\
{}\times
p_k\left(y,x_{n-k+2},\ld,x_{n}\right)\, dy +{}
\\
{}+
\lambda_n p_n(x) \sum\limits_{k=1}^\infty
\int\limits_0^\infty b_{k1}(y)\, d(y,x) dy
+{}
\\
{}+\!
\sum\limits_{k=1}^{n-1}\! \int\limits_0^\infty\!\! \lambda_k 
d\left(y,x_{n-k+1}\right)
b_{n-k+1}\left(y,x_1,\ld,x_{n-k}\right)\times{}\\
{}\times
p_k\left(x_{n-k+1},\ld,x_{n}\right)\, dy
+{}
\\
{}+
\lambda_k \sum\limits_{k=1}^{n}
\sum\limits_{m=2}^{\infty}
\mathop{\!\int\cdots\!\int}\limits_{y_1,\ld,y_{m}>0}
\d\left(y_2,y_{1}\right)\times{}\\
{}\times
b_{n-k+m}\left(y_2,\ld,y_m,x_1,\ld,x_{n-k+1}\right)\times{}\\
{}\times
p_k\left(y_1,x_{n-k+2},\ld,x_{n}\right)\, dy_1\cdots dy_m
+{}
\\
{}+
\lambda_k \sum\limits_{k=1}^{n}
\sum\limits_{m=2}^{\infty}
\mathop{\int\!\cdots\!\int}\limits_{y_1,\ld,y_{m}>0}
d\left(y_1,x_{n-k+1}\right)\times{}\\
{}\times
b_{n-k+m}\left(y_1,\ld,y_m,x_1,\ld,x_{n-k}\right)\times{}\\
{}\times
p_k\left(x_{n-k+1},\ld,x_{n}\right)\, dy_1\cdots dy_m\,,
\enskip
n\ge2\,.
\label{e2}
\end{multline}



\noindent
К этой системе уравнений нужно добавить граничные условия,
которые удобно записать в~виде:
\begin{equation*}
%\label{(3.3)}
\lim\limits_{x\to \infty} p_{1}(x) = 0\,;                         %      \eqno(3.3)
\enskip
%\label{(3.4)}
\lim\limits_{x\to \infty} p_{n}(x,x_2,\ld,x_n) =
0\,,\enskip n\ge 2\,.               %       \eqno(3.4)
\end{equation*}

\noindent
Полученные соотношения позволяют теоретически последовательно по~$n$
вычислять совместное стационарное распределение $p_n(x_1,\ldots,x_{n})$
с точностью до вероятности~$p_0$, которая находится из условия нормировки.
Если достаточно знать только маргинальные плотности

\vspace*{2pt}

\noindent
\begin{equation*}
p_n(x) = \int\limits_0^\infty\!\! \cdots\!\! \int\limits_0^\infty
p_n\left(x,x_2,\ld,x_n\right)\, dx_2\cdots dx_n\,,
\enskip n\ge 2\,,
\end{equation*}

\noindent то полученные соотношения можно упросить.
Введем обозначения:

\vspace*{-2pt}

\noindent
\begin{multline}
\label{new6}
b_{k,m}(x)
= \mathop{\int\!\cdots\!\int} \limits_{y_1,\ld,y_{k-1}>0}\!\!
b_k\left(y_1,\ld,y_{m-1},x,y_{m},\ld\right.\\
\left.\ld,y_{k-1}\right)\, dy_1\cdots dy_{k-1}\,,
\enskip k\ge 2\,,\enskip m=\overline{1,k}\,;
\end{multline}

\noindent
\begin{equation}
\label{new7-1}
b_{2,1,2}(y,x) = b_2(y,x)\,;
\end{equation}
\begin{multline}
\label{new7}
b_{k,1,m}(y,x)
=  \mathop{\int\!\cdots\!\int}
\limits_{y_1,\ld,y_{k-2}>0}\!\!
b_k\left(y,y_1,\ld\right.\\
\left.\ld,y_{m-2},x,y_{m-1},\ld,y_{k-2}\right)
\, dy_1\cdots dy_{k-2}\,,\\
k\ge 3\,,\enskip m=\overline{2,k}\,.
\end{multline}

\noindent 
Интегрируя~\eqref{e1} и~\eqref{e2} по
$x_2,\ldots ,x_n$ в~пределах от~0 до~$\infty$ и~учитывая 
обозначения~\eqref{new6}--\eqref{new7}, получаем следующую
систему интегродифференциальных уравнений
для~$p_{n}(x)$, $n\hm\ge 1$:
\begin{align}
-p'_1(x) &=a_1(x)
-\lambda_1 p_1(x)+{}\notag\\
&{}+\lambda_1\int\limits_0^\infty
p_1(y)K(x,y)\, dy+\lambda_1p_1(x)g_{1}(x)\,, \label{sys1}
\\
-p'_n(x)&=a_n(x)-\lambda_n p_n(x)+{}\notag\\
&\hspace*{-13mm}{}+\lambda_n \int\limits_0^\infty p_n(y) K(x,y) \, dy +
\sum\limits_{k=1}^{n-1} \lambda_k \left(
\vphantom{\int\limits_0^\infty}
p_k(x) g_{n,k}(x)+ {}\right.\notag\\
&\left.{}+\int\limits_0^\infty p_k(y) G_{n,k}(x,y) \, dy
\right)\,, \enskip n \ge 2\,;
\label{sys2}
\end{align}
где
$$
a_1(x) = \lambda_0 p_0
\left( b_1(x) + \sum\limits_{k=2}^\infty b_{kk}(x) \right)\,;
$$
$$
a_n(x)= \lambda_0 p_0 \left(
b_{n1}(x) + \sum\limits_{k=n+1}^\infty b_{k,k-n+1}(x)
\right)\,;
$$
$$
K(x,y)= \d(x,y) b_{1}(x) + \sum\limits_{k=2}^{\infty}
\int\limits_0^\infty \d(z,y) b_{k,1,k}(z,x) \, dz\,;
$$
$$
g_{1}(x)= \int\limits_0^\infty
 d(y,x) b_{1}(y)\, dy, + \sum\limits_{k=2}^\infty
\int\limits_0^\infty  d(y,x) b_{k1}(y)\, dy\,;
$$
$$
g_{n,n-1}(x)= \int\limits_0^\infty \d(y,x)
b_{1}(y)\, dy\,; 
$$
$$
g_{n,k}(x)= \int\limits_0^\infty \d(y,x) b_{n-k,1}(y) \, dy\,, \enskip 
k=\overline{1,n-2}\,;
$$
$$
G_{n,n-1}(x,y)= d(x,y) b_{1}(x);
$$

\vspace*{-12pt}

\noindent
\begin{multline*}
G_{n,k}(x,y)= d(x,y) b_{n-k,1}(x) + \d(x,y) b_{n-k+1,1}(x)
+{}\\
{}+ \int\limits_0^\infty d(z,y) b_{n-k+1,1,2}(z,x)\, dz +{}
\\
{}+
\sum\limits_{m=2}^{\infty}
\int\limits_0^\infty
\left( \d(z,y)
b_{n-k+m,1,m}(z,x)
+{}\right.\\
\left.{}+ d(z,y) b_{n-k+m,1,m+1}(z,x) \right)\, dz\,, \enskip
 k=\overline{1,n-2}\,.
\end{multline*}

\noindent
Отметим, что все функции, входящие в~интегральные уравнения \eqref{sys1} и~\eqref{sys2},
являются неотрицательными.
Полученная система решается рекуррентным образом.
Граничные условия имеют вид:
$$
\lim\limits_{x\to \infty} p_{n}(x)= 0\,,\enskip
n\ge 1\,.
$$

 Сначала определяем
$p_1(x)$ из~\eqref{sys1}, затем $p_2(x)$ через $p_1(x)$ из~\eqref{sys2} 
при $n\hm=2$ и~т.\,д.
Предварительно можно выполнить замену \mbox{$p_n(x) \hm= e^{\lambda x} q_n(x)$}
и~проинтегрировать новые уравнения от~0 до~$\infty$ с~учетом граничных условий.
Чис\-лен\-ное решение может быть найдено, например, методом итераций, причем
в качестве начальной итерации удобно взять нулевое приближение.

В заключение этого раздела отметим,
что, если для функции~$d(x,y)$ известна соответству\-ющая
сепарабельная аппроксимация (см., например,~\cite{n5,n6,n7,n8}), 
в~некоторых случаях (как, например, при выполнении приводимых ниже\linebreak условий~\eqref{us1})
уравнения~\eqref{sys1} и~\eqref{sys2} сводятся к~сис\-те\-ме линейных алгебраических 
уравнений.

\section{Производящая функция}

В ряде случаев решение уравнений~\eqref{sys1} и~\eqref{sys2} может быть
найдено в~терминах производящих функций (ПФ), что облегчает нахождение моментов\linebreak
чис\-ла заявок в~сис\-те\-ме.
Разберем один из них~--- случай группового пуассоновского потока постоянной
интенсивности, в~котором длины заявок в~по\-сту\-па\-ющей
группе не зависят друг от друга и~от размера группы, т.\,е.
\begin{equation}
\left.
\begin{array}{rlrl}
\hspace*{-2mm}\lambda_k&=\lambda\,, & k&\ge 0\,;\\[6pt]
\hspace*{-2mm}B_k\left(x_1,\ld,x_k\right)&=
c_k B\left(x_1\right)\cdots B\left(x_k\right)\,, &
k &\ge 1\,,
\end{array}
\right\}
\label{us1}
\end{equation}

\noindent где $B(x)$~--- непрерывная функция распределения времени обслуживания
одной заявки на приборе, $c_k \hm\ge 0$ и~$\sum\nolimits_{k=1}^\infty c_k\hm=1$.
Необходимым и~достаточным условием существования стационарного режима 
(и~это будет дополнительно показано ниже\footnote{Этот результат  
также следует из сравнения суммарной работы в~рассматриваемой системе и~классической 
системе $M/G/1$ с~групповым входящим потоком и~обслуживанием в~порядке поступления.})
является $\lambda \overline{c}  \overline{b}\hm < 1$,
где $\overline{b}\hm=\int\nolimits_0^\infty x  dB(x)$~--- 
средняя длина поступающей заявки, а~$\overline{c}\hm=C'(1)$~--- 
средний размер поступающей группы заявок.


Введем обозначения:
\begin{gather*}
H^*(z)=\sum\limits_{n=0}^\infty  P_n z^n=
P_0+H(z)\,; \\
h(z,x)=\sum\limits_{n=1}^\infty  p_n(x) z^n\,, \enskip
C(z)= \sum\limits_{n=1}^\infty  c_n z^n\,,
\end{gather*}

\noindent где $P_n\hm=P_n(\infty,\ld,\infty)$, $n \hm\ge 1$.
Умножив уравнение~\eqref{sys1} на~$z$, а~\eqref{sys2}~---
на~$z^n$, просуммировав и~проинтегрировав
с~учетом граничного условия $\lim\nolimits_{x\to \infty} h(z,x)\hm= 0$,
получаем уравнение:
\begin{multline}
h(z,x) = \lambda p_0 (1-B(x)) \fr{z (1-C(z))}{1-z}
+ {}\\[2pt]
{}+
\lambda (1-B(x))  H(z) \left ( C(z) + c_1 + \fr{z^2 - C(z)}{z(1-z)} \right )
- {}
\\[2pt]
{}- \lambda (1- C(z)) \left(
\int\limits_x^\infty \int\limits_0^\infty \d(t,y) h(z,y)\, dy  dB(t) -{}\right.\\[2pt]
\left.{}-
\int\limits_0^\infty \int\limits_x^\infty \d(t,y) h(z,y)\, dy  dB(t)
\right)
+{}
\\[2pt]
{}+
\lambda \fr{C(z)-c_1 z}{z} \int\limits_x^\infty
\int\limits_0^\infty h(z,y) \left(
\vphantom{\int\limits_0^\infty}
\d(t,y) +{}\right.\\[2pt]
\left.{}+ \int\limits_0^\infty d(u,y) \,dB(u) \right)
dy  dB(t)\,.
\label{pf}
\end{multline}

\noindent В случае ординарного потока ($c_1 \hm\equiv 1$) из этого уравнения 
немедленно следует ПФ
числа заявок в~системе, рассмотренной в~\cite{n3}. Трактуя~$z$ как параметр,
для решения уравнения~\eqref{pf} можно применить метод, описанный 
в~предыдущем разделе.

Задачу нахождения моментов стационарного распределения
числа заявок в~системе рассмотрим на примере математического ожидания
и~ограничимся лишь описанием алгоритма его расчета.

Будем считать, что операции дифференцирования, которые будут применены ниже, законны.
Проинтегрируем~\eqref{pf} по~$x$ от~0 до~$\infty$ и~найдем~$H(z)$.
Продифференцировав выражение для $(1-z)H(z)$ два раза и~положив $z\hm=1$, получим формулу
для расчета среднего числа заявок $\mathbf{E}\nu$ в~системе с~двумя
неизвестными: $h(1,x)$ и~$h'(1,x)\hm=\partial h(z,x) / \partial z |_{z=1}$.
Их нахождение осуществляется в~два этапа.
Сначала выписывается выражение для~$H(1)$,
затем, подставив $z=1$ и~найденное выражение для~$H(1)$
в~\eqref{pf},\linebreak получается интегральное уравнение для $h(1,x)$,
чис\-лен\-ное решение которого можно найти, например, как и~выше,
итерационным методом. Таким же образом, но предварительно продифференцировав~\eqref{pf} 
по~$z$, находится и~уравнение для~$h'(1,x)$.

Необходимым условием существования~$\mathbf{E}\nu$ является условие:

\pagebreak



\noindent
\begin{equation}
\label{sred}
\overline{c}
\int\limits_0^\infty
\int\limits_0^\infty
\int\limits_x^\infty
\d(t,y)
(1-B(y)) \,dy  dB(t)
dx < \infty\,.
\end{equation}


\noindent Показать это можно так же, как и~для системы из~\cite{n3}.
Предположим, что $\mathbf{E}\nu\hm< \infty$. Тогда $(H^*(1))'\hm=H'(1)<\infty$.
Поскольку
\begin{equation}
\label{hzx}
h(z,x)\le h(1,x)\le \lambda \overline{c} (1-B(x)) \,,
\end{equation}
то
$$
\int\limits_0^\infty
\int\limits_x^\infty
\int\limits_0^\infty
\d(t,y) h(z,y) \, dy 
dB(t) dx
\le \lambda  \overline{c}  \overline{b}^2 < \infty\,.
$$

\noindent Интегрируя теперь~\eqref{pf} по~$x$ от~0 до~$\infty$,
получаем:
\begin{multline*}
\fr{ \lambda \overline{b} (1-z H^*(z))}{ 1-z}+
\left ( z(1-C(z))(1-z) \right )^{-1}\times{}\\
{}\times \left \{
\vphantom{\int\limits_0^\infty}
H(z)(1-z)z - \lambda \overline{b} z (1-C(z))
\right.
+{}
\\
{}+
\lambda(1-z) (C(z)-c_1 z)\times{}\\
{}\times
\left[
\int\limits_0^\infty
\int\limits_x^\infty
\int\limits_0^\infty
h(z,y) \left(
\d(t,y) + {}\right.\right.\\
\left.\left.\left.{}+\int\limits_0^\infty d(u,y) \, dB(u)
\right)\,
dy  dB(t)
dx -
\overline{b} H(z)
\right ] \right\}
+ {}\\
{}+ \lambda \int\limits_0^\infty
\int\limits_x^\infty \int\limits_0^\infty
\d(t,y) h(z,y) \,dy dB(t) dx
={}\\
{}= \lambda \int\limits_0^\infty \int\limits_0^\infty
\int\limits_x^\infty \d(t,y) h(z,y) \,dy  dB(t) dx\,.
%\label{neob}
\end{multline*}

\noindent Поскольку левая часть ограничена и~$h(z,x) \hm\rightarrow h(1,x)$ 
при $z \hm\rightarrow 1$,
воспользовавшись теоремой Фату и~учитывая~\eqref{hzx}, приходим к~\eqref{sred}.

Достаточность показать сложнее, 
и~ввиду громоздкости выкладок на этом здесь останавливаться не будем.
Заметим, что для выполнения~\eqref{sred} достаточно (помимо конечности среднего
размера группы) существования у~распределения времени обслуживания~$B(x)$ 
второго момента.

\section{Время пребывания заявки в~системе}

Вернемся к~основной исследуемой системе.
Расчет временных характеристик поступающих в~сис\-те\-му заявок
начинается с~нахождения периода занятости (ПЗ) и~его характеристик.
Обозначим через $u_n(s;x)$, $n\hm\ge 1$, ПЛС функции распределения
времени до того момента,
когда в~системе останется $(n\hm-1)$ заявок при условии,
что на приборе начала обслуживаться заявка длины~$x$
и~в~системе находилось~$n$~заявок.
Уравнение для $u_n(s;x)$ получается из следующих рассуждений:
за время обслуживания заявки длины~$x$
с~вероятностью $e^{- \lambda_n x}$ не поступит больше ни одной
заявки, а~с~вероятностью $\lambda_n e^{- \lambda_n t}\,dt$
на интервале времени $[t,t+dt]$ может поступить
группа размером $k\hm\ge1$. В~первом случае ПЛС
равно~$e^{-sx}$, а~во втором зависит от размера
поступающей группы и~того, произошла смена заявки на приборе
или нет (и~в~каждом случае необходимо
дождаться окончания обслуживания исходной заявки длины~$(x\hm-t)$ и~$k$~новых заявок).
Рассматривая все возможные события и~воспользовавшись свойствами ПЛС, получаем:
\begin{multline}
u_{n}(s;x) = e^{- (\lambda_n+s) x}
+{}
\\
{}+
\sum\limits_{k=1}^\infty
\int\limits_0^x \lambda_n e^{- (\lambda_n+s) t} \, dt
\mathop{\int\!\cdots\!\int}\limits_{y_1,\ld,y_{k}>0}
d\left(y_1,x-t\right)\times{}\\
{}\times u_n(s;x-t)
\prod\limits_{j=1}^k u_{n+k+1-j}\left(s;y_j\right) B_k\left(dy_1,\dots,dy_k\right)
+{}
\\
{}+
\sum\limits_{k=1}^\infty \int\limits_0^x
\lambda_n e^{- (\lambda_n+s) t} \, dt
\mathop{\int\!\cdots\!\int}\limits_{y_1,\ld,y_{k}>0}
\d\left(y_1,x-t\right)\times{}\\
{}\times u_{n+k}(s;x-t) 
\prod\limits_{j=1}^k u_{n+k-j}\left(s;y_j\right)\times{}\\
{}\times
B_k\left(dy_1,\dots,dy_k\right)\,.
\label{uns}
\end{multline}

\noindent
Решение этого интегрального уравнения в~явном виде для произвольных 
функций $B_k(x_1,\ld,x_k)$
получить не удается. Однако в~некоторых частных случаях оно разрешимо, как, например,
в~случае условий~\eqref{us1}. Здесь $u_{n}(s;x)$ не зависит
от~$n$ и,~как нетрудно вывести из~\eqref{uns},
$$
u_n(s;x)=u(s;x)=e^{-\left ( \lambda + s - \lambda C(u(s))\right ) x}\,,
$$
где $\beta(s)$~--- ПЛС функции распределения~$B(x)$,
а~$u(s)$ является корнем уравнения:
$$
u(s) = \beta \left (\lambda+s-\lambda C (u(s)) \right )\,.
$$

Кроме того, ПЛС $u^*(s)$ функции распределения
ПЗ\footnote{Для основной
исследуемой системы ПЛС ПЗ равен $\sum\nolimits_{k=1}^\infty
\mathop{\int\!\cdots\!\int}\nolimits_{y_1,\ld,y_{k}>0}
\prod\nolimits_{n=1}^k u_{n}(s;y_{k-n+1})
B_k(dy_1,\dots,dy_k)$.}
удовлетворяет уравнению $u^*(s)\hm=C\left ( \beta(\lambda\hm+s\hm-\lambda u^*(s)) \right)$.

Для нахождения распределений времен ожидания начала обслуживания 
и~пребывания в~системе введем следующие функции:
\begin{description}
\item[\,] ${\tilde B}(k,i,x)$~--- вероятность того, что пришла группа из~$k$~заявок 
и~$i$-я заявка в~группе имеет длину меньше~$x$:
\begin{multline*}
{\tilde B}(k,i,x)=B_k(\infty,\dots,\infty,x,\infty, \dots, \infty)\,,\\ 
k \ge 1\,,\enskip 1 \le i \le k\,;
\end{multline*}

\item[\,] ${\bar B}(x_1,\dots,x_{i-1};k,i,x)$~---
условная 
вероятность\footnote{Здесь производная понимается
как производная Ра\-до\-на--Ни\-ко\-ди\-ма.} того, что первая заявка имеет
длину меньше~$x_1$, вторая~--- меньше~$x_2$, $\dots$, $(i-1)$-я~--- меньше $x_{i-1}$,
при условии, что пришла группа из~$k$~заявок, причем заявка на $i$-м месте имеет
длину~$x$:
\begin{multline*}
{\bar B}(x_1,\dots,x_{i-1};k,i,x)={}\\
{}=\fr{d_x B_k(x_1,\dots,x_{i-1},x,\infty, 
\dots, \infty)}{d {\tilde B}(k,i,x)}\,;
\end{multline*} 

\item[\,] ${\hat B}(x)$~---  среднее число заявок длины
меньше~$x$ в~поступающей группе:
$$
{\hat B}(x) = \sum\limits_{k=1}^\infty 
\sum\limits_{i=1}^k {\tilde B}(k,i,x)\,;
$$

\item[\,] ${\hat B}(k,i;x)$~---
условная вероятность того, что поступила группа из~$k$~заявок, среди них
есть ровно одна заявка длины~$x$ и~она находится на $i$-м месте, при условии что
поступила группа, в~которой имеются заявки длины~$x$:
$$
{\hat B}(k,i;x)=\fr{d_x {\tilde B}(k,i,x)}{{\hat B}(x)}\,, \enskip
k \ge 1,\enskip 1 \le i \le k\,.
$$

\end{description}

Определим сначала ПЛС $\omega_{k1}(s;x)$ функции распределения
времени ожидания начала обслуживания заявки длины~$x$ при условии,
что она поступила в~группе размера $k\hm\ge$ и~была на первом месте в~группе.
Ее время ожидания равно нулю, если она застала систему свободной 
и~если она, застав на приборе заявку длины~$y$, заняла ее место.
Если же она застала в~системе~$n$, $n\hm\ge 1$, заявок, на приборе~---
заявку длины~$y$ и~не заняла ее место,
то время ожидания совпадает с~ПЗ, открываемого заявкой длины~$y$,
когда в~системе находится $(n+k)$ заявок, т.\,е.\ $u_{n+k}(s;y)$. 
В~терминах ПЛС имеем
\begin{multline*}
\omega_{k1}(s;x) = p_0 + \sum\limits_{n=1}^\infty
\int\limits_0^\infty p_n(y) \left( 
\vphantom{'d}
d(x,y) +{}\right.\\
\left.{}+ \d(x,y) u_{n+k}(s;y) \right )\,
dy\,, \enskip k \ge 1\,.
\end{multline*}

Перейдем к~ПЛС времени
ожидания начала обслуживания заявки длины~$x$,
поступившей в~группе из~$k$~заявок ($k\hm\ge2$)
и~занимающей в~группе $i$-е мес\-то ($2 \hm\le i\hm\le k$).
В~случае поступления в~пустую систему
время ожидания совпадает с~суммарной дли\-тель\-ностью $(i-1)$-го
ПЗ, первый из которых от\-кры\-вается заявкой длины~$x_1$,
второй~--- $x_2$ и~т.\,д., и~в~терми\-нах ПЛС
равно $u_{k}(s;x_1)\cdots u_{2}(s;x_{i-1})$.
Длительности соответствующих ПЗ необходимо добавить к~времени
ожидания, когда по\-сту\-па\-ющая группа застает сис\-те\-му занятой.
В~итоге, вводя обозначение 

\noindent
$$
{\tilde u}_{nk}(s;x_1,\dots,x_{i-1})
=u_{n+k}(s;x_1)\cdots u_{n+2}(s;x_{i-1})\,,
$$
выражение для ПЛС функции распределения времени ожидания начала обслуживания
$\omega_{ki}(s;x_1, \dots, x_{i-1},x)$ заявки длины~$x$,
поступившей в~группе из~$k$~заявок и~занимающей в~группе $i$-е место,
можно записать так:

\noindent
\begin{multline*}
\omega_{ki}(s;x_1, \dots, x_{i-1},x) =
p_0 {\tilde u}_{0k}(s;x_1,\dots,x_{i-1})
+{}
\\
{}+
\sum\limits_{n=1}^\infty \int\limits_0^\infty p_n(y) \left (
d\left(x_1,y\right) {\tilde u}_{nk}\left(s;x_1,\dots,x_{i-1}\right)
+ {}\right.\\
\left.{}+\d\left(x_1,y\right) u_{n+k}(s;y)
{\tilde u}_{n-1,k}\left(s;x_1,\dots,x_{i-1}\right)
\right )\,dy\,,
\\
 k \ge 2\,, \enskip 2 \le i \le k\,.
\end{multline*}

\noindent
Легко видеть, что если
интенсивность входящего потока не зависит от числа заявок в~системе,
то при фиксированном~$i$ все $u_n(s;x_i)$ равны между собой
и~выражение $\omega_{ki}(s;x_1, \dots, x_{i-1},x)$ приводится к~виду:

\noindent
\begin{multline*}
\omega_{ki}(s;x_1, \dots, x_{i-1},x)={}\\
{}=
\omega_{k1}(s;x_1)u(s;x_1)\cdots u(s;x_{i-1})\,,
\end{multline*}
т.\,е.\ не зависит от числа заявок в~группе, а~только от места выделенной
заявки в~группе (и,~конечно, остаточных длин заявок, стоящих перед ней).

Теперь можно выписать ПЛС распределений,
связанных с~временем ожидания начала обслуживания и~пребывания в~системе.
Условно стационарное распределение времени ожидания
начала обслуживания заявки длины~$x$
при условии, что всего в~группе поступило $k\hm\ge2$ заявок
и~заявка длины~$x$ находится на $i$-м месте ($2 \hm\le i \hm\le k$),
имеет ПЛС, задаваемое выражением:

\noindent
\begin{multline}
\omega_{ki}(s;x)
=
\int\limits_0^\infty\!\cdots\!\int\limits_0^\infty
\omega_{ki}\left(s;x_1, \dots, x_{i-1},x\right)\times{}\\
{}\times
{\bar B}\left(dx_1,\dots,dx_{i-1};k,i,x\right)\,.
\label{w1}
\end{multline}

\noindent
Усредняя $\omega_{ki}(s;x)$ по распределению
${\hat B}(k,i;x)$, получаем формулу для ПЛС
$\omega(s;x)$ функции распре-\linebreak\vspace*{-12pt}

\pagebreak

\noindent
деления времени
ожидания начала обслуживания
заявки длины~$x$:
\begin{equation}
\label{w2}
\omega(s;x)
= \sum\limits_{k=1}^\infty \sum\limits_{i=1}^k \omega_{ki}(s;x) {\hat B}(k,i;x)\,.
\end{equation}


\noindent
Безусловное ПЛС~$\omega(s)$ функции распределения времени
ожидания начала обслуживания определяется путем усреднения по длине заявки:
\begin{equation}
\label{w3}
\omega(s)= \int\limits_0^\infty
\omega(s;x) d{\hat B}(x) ({\hat B}(\infty))^{-1}\,.
\end{equation}

\noindent В случае условий~\eqref{us1} все упрощается
и~формулу для $\omega(s;x)$ можно привести к~виду:
\begin{multline}
\omega(s;x)
= \fr{1}{\overline{c}}
\left (
\vphantom{\int\limits_{0}^\infty}
\omega^*(s;x) + {}\right.\\
\hspace*{-2mm}\left.{}+\fr{u(s)-C \left ( u(s) \right )}{u(s) (1- u(s))}
\int\limits_{0}^\infty \omega^*(s;y) u(s,y) b(y)\, dy \right ),\!
\label{sc1}
\end{multline}

\noindent где
$$
\omega^*(s;x)= p_0 + \int\limits_0^\infty
h(1,y) \left ( d(x,y) + \d(x,y) u(s;y)
\right ) \,dy\,.
$$

Аналогичным образом находится и~ПЛС $\phi(s;x)$ функции распределения
времени пребывания заявки длины~$x$ в~системе и~безусловное ПЛС~$\phi(s)$.
Обозначим через $\phi_{ki}(s;x_1, \dots, x_{i-1},x)$, $k \hm\ge 1$, $1\hm\le i \hm\le k$,
ПЛС функции распределения времени пребывания в~системе заявки длины~$x$,
поступившей в~группе из~$k$~заявок и~занимающей в~группе $i$-е место.
При $i\hm=1$ аргумент~$x_0$ опускается, т.\,е.\
 $\phi_{k1}(s;x_{0},x)\hm=\phi_{k1}(s;x)$.
Тогда
\begin{multline*}
\phi_{ki}(s;x_1, \dots, x_{i-1},x)
={}\\
{}=p_0 {\tilde u}_{0k}\left(s;x_1,\dots,x_{i-1}\right)u_{1}(s;x)
+ 
\sum\limits_{n=1}^\infty
\int\limits_0^\infty p_n(y)\times{}\\
{}\times \left(
d\left(x_1,y\right) {\tilde u}_{nk}\left(s;x_1,\dots,x_{i-1}\right)
u_{n+1}(s;x) +{}\right.
\\
{}+
\d\left(x_1,y\right) u_{n+k}(s;y) {\tilde u}_{n-1,k}\left(s;x_1,\dots,x_{i-1}\right)\times{}\\
\left.{}\times
u_{n}(s;x) %\vhpahntom*{\tilde{u}}
\right)\,dy\,,  \enskip 
k \ge 1\,, \enskip 1 \le i \le k\,.
\end{multline*}

\noindent Переход к~ПЛС $\phi(s;x)$ и~$\phi(s)$ осуществляется
по формулам~\eqref{w1}--\eqref{w3}. Если выполняются условия~\eqref{us1}, то,
вспоминая, что время пребывания заявки в~системе складывается
из времени ожидания начала обслуживания и~времени пребывания на приборе,
получаем $\phi(s;x)\hm=\omega(s;x)u(s;x)$. Дифференцируя эту формулу 
с~учетом~\eqref{sc1} необходимое число раз, нетрудно получить моменты
времени пребывания  в~системе заявки  длины~$x$.


\section{Заключение}

\vspace*{-3pt}

Используя результаты предыдущего раздела,
можно также найти совместные распределения ПЗ 
и~числа обслуженных на приборе заявок, или интервала времени,
когда в~системе находилось не менее~$n$~заявок, 
или этих обеих случайных величин и~т.\,п.

В практическом плане интерес в~дальнейшем представляет разбор
 частных случаев, т.\,е.\ анализ
стационарных характеристик системы в~различных предположениях 
о~зависимости размеров заявок внутри группы; в~теоретическом~---
обобщение полученных результатов
на случай более общего группового входящего потока, когда
в~каждой поступающей группе могут находиться подгруппы заявок
одинаковой или различной длины.

\vspace*{-18pt}

{\small\frenchspacing
 {%\baselineskip=10.8pt
 \addcontentsline{toc}{section}{References}
 \begin{thebibliography}{99}
 
 \vspace*{-3pt}
 
 \bibitem{n3}  %1
\Au{Нагоненко В.\,А.}
О~характеристиках одной нестандартной системы
массового обслуживания.~I, II~//
Изв.\ АН СССР. Технич.\ кибернет., 1981.
№\,1. С.~187--195; №\,3. С.~91--99.

\bibitem{n4} %2
\Au{Печинкин А.\,В.} Об одной
инвариантной системе массового обслуживания~//
Math.\ Operationsforsch.\ Statist.
Ser.\ Optimization, 1983. Vol.~14. No.\,3. P.~433--444.

\bibitem{n5} %3
\Au{Милованова Т.\,А., Печинкин~А.\,В.}
Стационарные характеристики системы обслуживания 
с~инверсионным порядком обслуживания, вероятностным
приоритетом и~гистерезисной политикой~//
Информатика и~её применения, 2013. Т.~7. Вып.~1. С.~22--36.




\bibitem{n1}  %4
\Au{Мейханаджян Л.\,А., Милованова~Т.\,А., Печинкин~А.\,В., Разумчик~Р.\,В.}
Стационарные вероятности состояний в~системе обслуживания с~инверсионным 
порядком обслуживания и~обобщенным вероятностным приоритетом~// Информатика 
и~её применения, 2014. Т.~8. Вып.~3. С.~16--26.

\bibitem{n2} %5
\Au{Мейханаджян Л.\,А., Милованова~Т.\,А., Разумчик~Р.\,В.}
Время ожидания в~системе обслуживания с~инверсионным 
порядком обслуживания и~обобщенным
вероятностным приоритетом~// Информатика и~её применения, 2015. Т.~9. Вып.~2. С.~14--22.

\bibitem{n0} %6
\Au{Razumchik R.} On ${M/G/1}$ queue with state-dependent heterogeneous 
batch arrivals, inverse service order and probabilistic priority~// 
AIP Conf. Proc., 2017. Vol.~1863. No.\,1. P.~090006-1--090006-3.

\bibitem{nm1} %7
\Au{Милованова Т.\,А.} Система $\mathrm{BMAP}/G/1/\infty$ с~инверсионным 
порядком обслуживания и~вероятностным приоритетом~// Автомат. телемех., 2009. 
№\,5. С.~155--168.
% Autom. Remote Control, 70:5 (2009), 885-896


\bibitem{gg2} %8
\Au{Bent N.} On a~queuing model where potential customers are discouraged by queue length~//
Scand. J.~Stat., 1975. Vol.~2. Iss.~1. P.~34--42.




\bibitem{gg3} %9
\Au{Печинкин А.\,В.} Система ${M_k/G/1}$ с~ненадежным прибором~//
Автомат. телемех., 1996. №\,9. С.~100--110.
% Autom. Remote Control, 57:9 (1996), 1302-1310



\bibitem{gg5} %10
\Au{Gupta U.\,C., Srinivasa~Rao~T.\,S.\,S.} 
On the analysis of single server finite queue with state dependent
arrival and service processes: ${M_n/G_n/1/K}$~//
OR Spektrum, 1998. Vol.~20. Iss.~2. P.~83--89.

\bibitem{gg4} %11
\Au{Kerner Y.} The conditional distribution of the residual service time in the ${M_n/G/1}$ 
queue~// Stoch. Models, 2008. Vol.~24. Iss.~3. P.~364--375.

\bibitem{gg1} %12
\Au{Abouee-Mehrizi H., Baron~O.} State-dependent ${M/G/1}$ queueing systems~//
Queueing Sy., 2016. Vol.~82. Iss.~1-2. P.~121--148.

\columnbreak

\bibitem{n6} %13
\Au{Поспелов В.\,В.}
О~погрешности приближения функции двух переменных суммами
произведений функций одного переменного~//
Ж.~вычисл. матем. матем. физ., 1978. Т.~18. Вып.~5. С.~1307--1308.
%U.S.S.R. Comput. Math. Math. Phys., 18:5 (1978), 228-230

\vspace*{7pt}

\bibitem{n8} %14
\Au{Uschmajew A.} Regularity of tensor product approximations to square
integrable functions~// Constr. Approx., 2011. Vol.~34. Iss.~3. P.~371--391.

\vspace*{7pt}

\bibitem{n7} %15
\Au{Townsend A., Trefethen~L.\,N.} An extension of Chebfun to two dimensions~// 
SIAM J.~Sci. Comput., 2013. Vol.~35. Iss.~6. P.~495--518.



 \end{thebibliography}

 }
 }

\end{multicols}

\vspace*{-3pt}

\hfill{\small\textit{Поступила в~редакцию 19.09.17}}

\vspace*{8pt}

%\newpage

%\vspace*{-24pt}

\hrule

\vspace*{2pt}

\hrule

%\vspace*{8pt}


\def\tit{$M/G/1$ QUEUE WITH~STATE-DEPENDENT 
HETEROGENEOUS BATCH ARRIVALS, 
INVERSE SERVICE ORDER, AND~PROBABILISTIC~PRIORITY}

\def\titkol{$M/G/1$ queue with~state-dependent 
heterogeneous batch arrivals, 
inverse service order, and~probabilistic priority}

\def\aut{R.\,V.~Razumchik$^{1,2}$}

\def\autkol{R.\,V.~Razumchik}

\titel{\tit}{\aut}{\autkol}{\titkol}

\vspace*{-9pt}


\noindent
$^1$Institute of Informatics Problems, Federal Research Center ``Computer Science 
and Control'' of the Russian\linebreak
$\hphantom{^1}$Academy of Sciences, 44-2~Vavilov Str., Moscow 
119333, Russian Federation

\noindent
$^2$Peoples' Friendship University of Russia (RUDN University), 
6~Miklukho-Maklaya Str., Moscow 117198, Russian\linebreak
$\hphantom{^1}$Federation



\def\leftfootline{\small{\textbf{\thepage}
\hfill INFORMATIKA I EE PRIMENENIYA~--- INFORMATICS AND
APPLICATIONS\ \ \ 2017\ \ \ volume~11\ \ \ issue\ 4}
}%
 \def\rightfootline{\small{INFORMATIKA I EE PRIMENENIYA~---
INFORMATICS AND APPLICATIONS\ \ \ 2017\ \ \ volume~11\ \ \ issue\ 4
\hfill \textbf{\thepage}}}

\vspace*{3pt}






\Abste{Consideration is given to the stationary characteristics
of single-server queues with the queue of infinite capacity,
independent and identically-distributed service times, 
LCFS (last-come-first-served) service order, and probabilistic priority discipline.
Most of the results for such type of queueing systems
have been obtained under the 
assumption of either Poisson arrivals or 
phase-type arrivals. 
Another important assumption made was that
the arrival process is independent 
from the system state. The author shows that the
latter assumption can be relaxed to some, quite large extent.
The author considers an $M/G/1/\infty$ queue  
with batch Poisson arrival flow in which ($i$)~the arrival rate depends
on the total number of customers present in the system
at the arrival instant; and ($ii$)~the size of the arriving batch~$k$ 
and the remaining service times $x_1,\dots,x_k$ of the customers in the batch
have the arbitrary continuous joint probability distribution
$B_k(x_1,\ld,x_k)$. The author obtains analytic expressions
for the computation of the joint stationary distribution 
of the total number of customers in the system 
and their remaining service times. 
Busy period, waiting and sojourn time distributions
are also given in terms of the Laplace--Stieltjes transforms.}

\KWE{queueing system; LIFO; probabilistic priority; batch 
arrival; state-dependent Poisson flow}


  \DOI{10.14357/19922264170402} 

\vspace*{-16pt}

\Ack
\noindent
This work was supported by the
Russian Science Foundation (grant 16-11-10227).



\vspace*{1pt}

  \begin{multicols}{2}

\renewcommand{\bibname}{\protect\rmfamily References}
%\renewcommand{\bibname}{\large\protect\rm References}

{\small\frenchspacing
 {%\baselineskip=10.8pt
 \addcontentsline{toc}{section}{References}
 \begin{thebibliography}{99}
 
 \vspace*{-3pt}
 
 \bibitem{Xx3-1} %1
\Aue{Nagonenko, V.\,A.} 1981.
O~kharakteristikakh odnoy nestandartnoy sistemy
massovogo obsluzhivaniya
[On the characteristics of one nonstandard queuing
system].~I, II.
\textit{Izv.\ AN SSSR. Tekhnich.\ kibernet}
[Proceedings of the Academy of Sciences of
the USSR. Technical Cybernetics]
1:187--195; 3:91--99.

\bibitem{Xx4-1} %2
\Aue{Pechinkin, A.\,V.} 1983.
Ob odnoy invariantnoy sisteme massovogo
obsluzhivaniya
[On an invariant queuing system].
\textit{Math.\ Operationsforsch.\ Statist.
Ser.\ Optimization} 14(3):433--444.

\bibitem{Xx5-1} %3
\Aue{Milovanova, T.\,A., and A.\,V.~Pechinkin.} 2013.
Sta\-tsi\-o\-nar\-nye kharakteristiki sistemy obsluzhivaniya
s~in\-ver\-si\-on\-nym poryadkom obsluzhivaniya,
veroyatnostnym pri\-ori\-te\-tom i~gisterezisnoy politikoy
[Stationary characteristics of queuing system with
an inversion procedure service probabilistic priority
and hysteresis policy].
\textit{Informatika i~ee Primeneniya~--- Inform. Appl.} 7(1): 22--35.





\bibitem{Xx1-1} %4
\Aue{Meykhanadzhyan, L.\,A., T.\,A.~Milovanova, A.\,V.~Pechinkin, 
and R.\,V.~Razumchik.} 2014.
Sta\-tsi\-o\-nar\-nye veroyatnosti sostoyaniy v~sisteme obsluzhivaniya s~inversionnym 
poryadkom ob\-slu\-zhi\-va\-niya i~obob\-shchen\-nym veroyatnostnym prioritetom
[Stationary distribution in a~queueing system with inverse service order and
generalized probabilistic priority].
\textit{Informatika i~ee Primeneniya~--- Inform. Appl.} 8(3):16--26.

\bibitem{Xx2-1} %5
\Aue{Meykhanadzhyan, L.\,A., T.\,A.~Milovanova, and R.\,V.~Razumchik.} 2015.
Vremya ozhidaniya v~sisteme obsluzhivaniya s inversionnym poryadkom obsluzhivaniya 
i~obobshchennym veroyatnostnym prioritetom
[Stationary waiting time in a~queueing system with inverse service order and 
generalized probabilistic priority].
\textit{Informatika i~ee Primeneniya~--- Inform. Appl.} 9(2):14--22.

\bibitem{Xn0-1} %6
\Aue{Razumchik, R.} 2017. On ${M/G/1}$ queue with state-dependent heterogeneous
 batch arrivals, inverse service order and probabilistic priority. 
 \textit{AIP Conf. Proc}. 1863(1):090006-1--090006-3.
 
 \bibitem{Xnm1-1} %7
\Aue{Milovanova, T.\,A.} 2009. 
${\mathrm{BMAP}/G/1/\infty}$ system with last come first served probabilistic priority.
\textit{Automat. Rem. \mbox{Contr.}} 70(5):885--896.

\bibitem{Xgg2-1} %8
\Aue{Bent, N.} On a~queuing model where potential customers are discouraged by queue length.
\textit{Scand. J.~Stat.} 2(1):34--42.

\bibitem{Xgg3-1} %9
\Aue{Pechinkin, A.\,V.} 1996. Sistema ${M_k/G/1}$ 
s~nenadezhnym priborom [An ${M_k/G/1}$  system with an unreliable device].
\textit{Avtomat. telemekh.} [Autom. Rem. Contr.] 9:100--110.



\bibitem{Xgg5-1} %10
\Aue{Gupta, U.\,C., and T.\,S.\,S.~Srinivasa Rao.} 1998.
On the analysis of single server finite queue with state dependent
arrival and service processes: ${M_n/G_n/1/K}$.
\textit{OR Spektrum} 20(2):83--89.

\bibitem{Xgg4-1}  %11
\Aue{Kerner, Y.} 2008. The conditional distribution of 
the residual service time in the ${M_n/G/1}$ queue.
\textit{Stoch. Models} 24(3):364--375.

\bibitem{Xgg1-1}  %12
\Aue{Abouee-Mehrizi, H., and O.~Baron.} 2016.
State-dependent ${M/G/1}$ queueing systems. 
\textit{Queueing Sy.} 82(1-2):121--148. 

\bibitem{Xn6-1} %13
\Aue{Pospelov, V.\,V.} 1978.
O~pogreshnosti priblizheniya funktsii dvukh peremennykh summami
proizvedeniy funktsiy odnogo peremennogo
[The error of approximation of a~function of two variables by sums 
of the products of functions of one variable]
\textit{Zh. vichisl. matem. matem fiz.}
[USSR\ Comput. Math. Math. Phys.] 18(5):1307--1308.

\bibitem{Xn8-1} %14
\Aue{Uschmajew, A.} 2011. Regularity of tensor product approximations to square
integrable functions.
\textit{Constr. Approx.} 34(3):371--391.

\bibitem{Xn7-1} %15
\Aue{Townsend, A., and L.\,N.~Trefethen.} 2013. 
An extension of Chebfun to two dimensions.
\textit{SIAM J.~Sci. Comput.} 35(6):495--518.


\end{thebibliography}

 }
 }

\end{multicols}

\vspace*{-6pt}

\hfill{\small\textit{Received September 19, 2017}}

%\vspace*{-10pt}

\Contrl

\noindent
\textbf{Razumchik Rostislav V.} (b.\ 1984)~--- 
Candidate of Science (PhD) in physics and mathematics, leading scientist, 
Institute of Informatics Problems, Federal Research Center 
``Computer Science and Control'' of the Russian Academy of Sciences, 
44-2~Vavilov Str., Moscow 119333, Russian Federation; 
associate professor, Peoples' Friendship University of Russia (RUDN University), 
6~Miklukho-Maklaya Str., Moscow 117198, Russian Federation; 
\mbox{rrazumchik@ipiran.ru}
%\mbox{razumchik\_rv@rudn.university}
\label{end\stat}


\renewcommand{\bibname}{\protect\rm Литература}   %2
\def\bmx{\mathbf}

\renewcommand{\figurename}{\protect\bf Figure}
\renewcommand{\tablename}{\protect\bf Table}

\def\stat{raz-1}

\def\tit{STATIONARY SOJOURN TIMES 
IN~$\mathrm{MAP}/\mathrm{PH}/1/r$ QUEUE WITH~BI-LEVEL HYSTERETIC CONTROL OF~ARRIVALS}

\def\titkol{Stationary sojourn times 
in~$\mathrm{MAP}/\mathrm{PH}/1/r$ queue with~bi-level hysteretic control of~arrivals}

\def\autkol{R.\,V.~Razumchik}

\def\aut{R.\,V.~Razumchik$^{1,2}$}

\titel{\tit}{\aut}{\autkol}{\titkol}

\index{Razumchik R.\,V.}
\index{Разумчик Р.\,В.}

%{\renewcommand{\thefootnote}{\fnsymbol{footnote}}
%\footnotetext[1] {This work was supported in part by the
%Russian Foundation for Basic Research (grants 15-07-03007 and 13-07-00223).}}

\renewcommand{\thefootnote}{\arabic{footnote}}
\footnotetext[1]{Institute of Informatics Problems, Federal Research 
Center ``Computer Science and Control'' of the Russian Academy of Sciences, 44-2~Vavilov Str., 
Moscow 119333, Russian Federation} 
\footnotetext[2]{Peoples' Friendship University of Russia (RUDN University), 6~Miklukho-Maklaya Str., 
Moscow 117198, Russian Federation}


%\vspace*{-6pt}

\def\leftfootline{\small{\textbf{\thepage}
\hfill INFORMATIKA I EE PRIMENENIYA~--- INFORMATICS AND APPLICATIONS\ \ \ 2017\ \ \ volume~11\ \ \ issue\ 4}
}%
 \def\rightfootline{\small{INFORMATIKA I EE PRIMENENIYA~--- INFORMATICS AND APPLICATIONS\ \ \ 2017\ \ \ volume~11\ \ \ issue\ 4
\hfill \textbf{\thepage}}}



%\def\l{\lambda}
%\def\m{\mu}
%\def\a{\alpha}
%\def\b{\beta}
%\def\g{\gamma}
%\def\d{\delta}
%\def\r{\overline r}
%\def\f{\overline f}
%\def\n{\nu}



\Abste{This paper reports some new results concerning the 
analysis of the time-related stationary characteristics
of a~finite-capacity queueing system 
operating in a~random environment
with the bi-level hysteretic control of arrivals.
The topic of the paper is motivated by 
the overload problem in networks of SIP (session initiation protocol) servers 
and the viewpoint that multilevel hysteretic control 
of arrivals in SIP servers
can be used to mitigate signalling network congestion.
The considered mathematical model of SIP server is the 
single server queueing system with Markovian arrival processes (MAP), PH (phase-type)
service,
and bi-level hysteretic control policy.
According to this policy, a~system may be in one
of the three operation modes: normal, overload, or blocking. 
The switching between modes occurs at 
instants whenever the total number of customers in the 
system changes.
The analytical method for the computation of the 
stationary sojourn times in different operation modes (in terms of 
Laplace--Stieltjes transforms (LST)), 
which utilizes the knowledge about the presence of 
hysteretic loops, is given. It is also applicable 
in the case when, in addition to the sojourn times,
one needs to account for the number of lost customers.}


\KWE{queueing system; random environment; first passage times; hysteretic control}

\DOI{10.14357/19922264170403}

%\vspace*{9pt}


\vskip 12pt plus 9pt minus 6pt

      \thispagestyle{myheadings}

      \begin{multicols}{2}

                  \label{st\stat}


\section{Introduction}

\noindent
This paper continues the analysis 
of the stationary finite-capacity queueing system 
operating in a~random environment
with hysteretic control of arrivals, 
which was started in~\cite{int0}.
Specifically, we deal with the $\mathrm{MAP}/\mathrm{PH}/1/r$ queue 
with two-level hysteretic control of 
arrival rates with nonoverlapping hysteretic
loops. For this system, the authors 
of~\cite{int0} proposed the 
new analytic method for the computation 
of the steady-state distribution, which
is different from the known general approaches 
for QBDs (quasi-birth-deathes). It exploits the knowledge about the hysteretic loops 
which are present in the system, has a~probabilistic interpretation
and leads to easy-to-implement computational procedures.  
We will not dwell on the motivation behind the 
analysis of this system (for details, refer~\cite{int0} and references therein)
and just mention that the various aspects of the 
topic of hysteretic control in 
queueing models still gains attention from the research 
community (see~\cite{int2, int3, int4}).

In order to make the picture clearer, let us assume 
that the control is only bi-level with nonoverlapping 
loops although all the results 
presented here (and in~\cite{int0}) can be 
generalized in a~straightforward manner for 
hysteretic control of arrivals with arbitrary number of 
nonoverlapping loops. 
Following the bi-level hysteretic control, 
the system changes its status (or mode) 
between ``normal,'' ``overload,'' and ``blocking'' 
(this will be made more precise in the next section). In each mode 
except for a~``normal'' one, server discards a~certain percentage 
of arriving customers. From a~practical point of view 
(at least the one mentioned in~\cite{int1}), 
it may be beneficial when the system spends 
as little time as possible in ``overload''/``blocking'' 
modes. This brings one to the analysis of time-related stationary
characteristics of the system, which was not carried out in~\cite{int0}.

In what follows, we are interested in the two performance characteristics 
of the hysteretic policy. The first one is the stationary
distribution of the time system spends in ``normal'' mode. 
The second one is the distribution of the time it takes the system to get back 
to ``normal'' mode\footnote[3]{Waiting and system sojourn time distributions
are of little interest since hysteretic loops have no influence on them in 
case of FIFO service policy (which is assumed).}. 

After giving the detailed system description in 
the next section, in section~3, 
the analytical method for the sequential computation
of these (sojourn time) distributions will be presented
in terms of LST.
The LST of the sojourn times are obtained as solutions to certain matrix difference equations
and are expressed in terms of recurrence relations.
They can be used for direct numerical implementation and 
numerical inversion with well-known methods (Fourier-series method with Euler 
summation, Talbot, etc.).

\begin{figure*}[b] %fig1
\vspace*{1pt}
 \begin{center}
 \mbox{%
 \epsfxsize=123.993mm 
 \epsfbox{raz-1.eps}
 }

\vspace*{6pt}

{\small Sketch of the bi-level hysteretic control of arrivals in the $\mathrm{MAP}/\mathrm{PH}/1/r$ 
system}

 \end{center}
\end{figure*}



\section{System Description and~Preliminaries}

\noindent
The system consists of a~single server and a~queue of finite capacity~$r$.
The arrival process is a~MAP with representation $(\mathbf{D_0}, \mathbf{D_1})$ 
of order~$N$.
Let us assume that an arrival, whenever it occurs, can be of one of the two types, either 
a~priority arrival or a~nonpriority. Thus, the matrix~$\mathbf{D_1}$ is assumed to
have the form $\mathbf{D_1}=\mathbf{D_{1,1}}+\mathbf{D_{1,2}}$
where $\mathbf{D_{1,1}}$ ($\mathbf{D_{1,2}}$) describes state transitions with an 
arrival of
priority (nonpriority) customer. Bi-level hysteretic control of 
arrivals is assumed to be implemented in the system. It operates as follows (see figure).
There are three operation modes for the system: ``normal,'' ``overload,'' 
and ``blocking.''
 Let~$L$ and~$H$ be arbitrary integers,
such that $0 < L < H < r+1$. Assume the system starts empty. 
As long as the total
number of customers in the system remains below~$H$,
the system is considered to be in ``normal`` mode and accepts 
all arrivals (both priority and nonpriority). 
When the total number of customers reaches~$H$ for the first time, the
system changes its mode to ``overload'' and stays in it as long
as the total number of customers remains between~$L$ and~$r$.
When overloaded, the system accepts only priority customers (nonpriority customers 
are lost)
till the total number of customers either
drops down below $L$ after which it changes its mode back to
``normal,'' or exceeds~$r$ after which it changes its state
to ``blocking.'' In the ``blocking'' mode the system does not accept
newly arriving customers until the total number of customers
drops down below $(H+1)$, after which the system changes mode 
back to ``overload`` and the process goes on.
The service time of both priority and
nonpriority customers is PH distributed with representation
$(\vec{f}, \mathbf{G})$ of order~$M$ and $\vec g=-\mathbf{G}\vec1$, and
the service policy is FIFO (first in, first out).



The operation of the considered queueing
system can be completely described
by continuous-time Markov chain 
${\bf{X}} \left( t \right)= {\left( \xi (t); \eta(t); \mu(t); \nu(t) \right) }$ with
four components:
$\xi(t)$~--- MAP generation phase at time~$t$;
$\eta(t)$~--- PH service phase at time~$t$;
$\mu(t)$~--- system's mode at time~$t$;
and $\nu(t)$~--- number of customers in
the system at time~$t$.
When $\nu(t)=0$, the second component $\eta(t)$ is omitted.
It is convenient to represent the state space of ${\bf{X}} \left( t \right)$ as
$\mathcal{X} = {\mathcal{X}_0} \cup {\mathcal{X}_1} \cup {\mathcal{X}_2}$
where $\mathcal{X}_{0} $ is the set of states of ''normal'' mode,
$\mathcal{X}_{1} $ is the set of states of ''overload'' mode,
and $\mathcal{X}_{2}$ is the set of states of ''blocking'' mode, i.\,e.,
\begin{align*}
{\mathcal{X}_0} &= \left\{ \left( {k,0,0} \right):  1 \le k \le N  \right\}
\cup 
\\
&\hspace*{5mm}\cup \left\{ {\left( {k,0,n} \right):  1 \le k \le NM, 1 \le n \le H - 1} \right\}\,;\\
{\mathcal{X}_1} &= \left\{ {\left( {k,1,n}\right): 1 \le k \le NM,  L \le n \le r} \right\}\,;\\
{\mathcal{X}_2} &= \left\{ {\left( {k,2,n} \right): 1 \le k \le NM, 
H\! + \! 1 \le n \le r \! + \! 1} \right\}\,.
\end{align*}

\noindent
Here, $k$ represents the state of 
the background (arrival and service) processes.
Indeed, the state $(k,m,n)$, $n>0$, means that 
there are~$n$ customers in the system, system's mode is~$m$,
and arrival and service phases are~$i$ and~$j$, but such that
$(i-1)M+j=k$; the state $(k,0,0)$ means that the system
is empty and the arrival phase is~$k$.

Let us denote by $\bmx{E}$ the identity matrix (its size each time will 
be clear from the context)
and let introduce the following transition rate matrices:
\begin{itemize}
    \item service of a~customer after which the system becomes empty: 
    $\bmx{P}_1=\bmx{E}\otimes {\vec g}$;
    \item service of a~customer after which the system remains busy: 
    $\bmx{P}=\bmx{P}^*=\bmx{P}^{\#}=\bmx{E}\otimes {\vec g}{\vec f}$;
    \item phase change when system is empty: $\bmx{Q}_0=\bmx{D_0}$;
    \item phase change when system is in the ``normal'' mode: 
    $\bmx{Q}=\bmx{D_0}\otimes \bmx{E} + \bmx{E} \otimes \bmx{G}$;
    \item phase change when system is in the ``overload'' mode: 
    $\bmx{Q}^*=(\bmx{D_0}+\bmx{D_{12}})\otimes \bmx{E} + \bmx{E} \otimes \bmx{G}$;
        \item arrival phase change when system is in the ``blocking'' mode: 
        $\bmx{Q}^{\#}=(\bmx{D_0}+\bmx{D_{1}})\otimes \bmx{E} + \bmx{E} \otimes 
        \bmx{G}$;
    \item arrival of a~customer to an empty system: $\bmx{R}_0\linebreak =\bmx{D_1} \otimes
    {\vec f}$,
    \item arrival of a~customer to the system in the ``normal'' mode: 
    $\bmx{R}=\bmx{D_1}\otimes \bmx{E}$; and
    \item arrival of a~customer to the system in the ``overload'' mode: 
    $\bmx{R}^*=\bmx{D_{11}}\otimes \bmx{E}$.
\end{itemize}

In order to be able to compute time-related characteristics,
in addition to transition rate matrices, one needs
transition probability matrices, which contain probabilities 
of possible state change of the background process.
Thus, let~$\alpha$, $\beta$, and $\gamma$ denote the matrices of
service, phase change, and arrival transition probabilities
when the system is in the ``normal'' mode,
i.\,e.,
\begin{alignat*}{3}
[\alpha]_{ij}  
&=
\fr{[\bmx{P}]_{ij} }{-{[\bmx{Q}]}_{ii}}\,, &\enskip 1 &\le i, j \le NM\,;
\\
[\beta]_{ij}  
&=\begin{cases}
\fr{[\bmx{Q}]_{ij}}{-{[\bmx{Q}]}_{ii}}\,,  & i \neq j\,; \\
0\,, & i = j\,,
\end{cases}
&\enskip 1 &\le i,j \le NM\,;
\\
[\gamma]_{ij}  
&=\fr{[\bmx{R}]_{ij} }{-{[\bmx{Q}]}_{ii}}\,, &\enskip 1 &\le i,j \le NM\,.
\end{alignat*}

\noindent
Here and henceforth, by $[\cdot]_{ij}$ we denote the $(i,j)${th}
 entry of any matrix.
By analogy, let us denote by~$\alpha^*$, $\beta^*$, and~$\gamma^*$
transition probabilities matrices in the ``overload'' mode,
by~$\alpha^{\#}$ and~$\beta^{\#}$
transition probability matrices  
in the ``blocking'' mode,
and by~$\alpha^{e}$, $\beta^{e}$, and~$\gamma^{e}$
transition probability matrices when
the system becomes or is empty.


\section{Sojourn Time Distributions}

\noindent
As it was mentioned in section~1,
we are interested in the two stationary characteristics:
distribution of the time system spends in ``normal'' mode
and the distribution of the time it takes the  system to get back 
to ``normal'' mode. These distributions are computed by 
conditioning on the number of customers in the system
and the state of the background process
and can be expressed in terms of
the following three quantities:
\begin{description}
\item[\,] $\bmx{V}_n(s)$, $n=\overline{0,H-1}$,~---
matrix of size $NM\times NM$,
which the $(i,j)${th} entry is
the LST of the first passage time 
to the ``overload'' mode 
and state of the background process~$j$, 
given that initially, the 
system was in the ``normal'' mode,
there where~$n$~customers in it,
and
the state of the background process was~$i$;
\item[\,] 
$\bmx{V}^{\#}_n(s)$, $n=\overline{H+1,R}$,~---
matrix of size $NM\times NM$,
which the $(i,j)${th} entry is
the LST of the first passage time 
to the ``normal'' mode 
and state of the background process~$j$, 
given that initially, the 
system was in the ``blocking'' mode,
there where~$n$~customers in it,
and
the state of the background process was~$i$; and
\item[\,] 
$\bmx{V}^*_n(s)$, $n=\overline{L,R-1}$,~---
matrix of size $NM\times NM$,
which the $(i,j)${th} entry is
the LST of the first passage time 
to the ``normal'' mode 
and state of the background process~$j$, 
given that initially, the 
system was in the ``overload'' mode,
there where~$n$~customers in it,
and
the state of the background process was~$i$.
\end{description}

The rest of the section is devoted to obtaining the relations for
$\bmx{V}_n(s)$, $\bmx{V}^*_n(s)$, and~$\bmx{V}^{\#}_n(s)$.
Let us begin with the calculation of $\bmx{V}_n(s)$,
$n=\overline{1,H-1}$. 
Denote by 
$\bmx{T}_k(s)$, $k=-1,0,1$, the 
matrix of size $NM \times NM$ 
which the $(i,j)${th} entry is 
LST of the first passage time
from the state $(i,0,n)$ 
to the state $(j,0,n-k)$.
Here,~$n$~may take any value from 
the set $\{2,3,\dots,H-2\}$. 
Remembering that the sojourn time 
in the state $(i,0,n)$ is exponential 
with rate $-[\bmx{Q}]_{ii}$, one has for 
$1 \le i,j \le NM$:
\begin{align*}
\left[ \bmx{T}_{-\!1}(s) \right]_{ij}
&=
\fr{[\bmx{Q}]_{ii} }{[\bmx{Q}]_{ii} - s}
\left[\gamma\right]_{ij}  \,;
\\
\left[ \bmx{T}_0(s) \right]_{ij}
&=
\fr{[\bmx{Q}]_{ii}}{[\bmx{Q}]_{ii} - s}
\left[\beta\right]_{ij}  \,;
\\
\left[ \bmx{T}_1(s) \right]_{ij}
&=
\fr{[\bmx{Q}]_{ii} }{[\bmx{Q}]_{ii} - s}
\left[\alpha\right]_{ij} \,.
\end{align*}

\noindent 
From the first-step analysis, let us find that
the LST $\bmx{V}_n(s)$ satisfies the system of matrix 
difference equations: 
\begin{equation}
\left.
\begin{array}{rl}
\bmx{V}_n(s)&=\bmx{T}_1(s) \bmx{V}_{n-1}(s)+ 
\bmx{T}_0(s) \bmx{V}_{n}(s)\\[6pt]
&\hspace*{1mm}{}+ \bmx{T}_{-1}(s) \bmx{V}_{n+1}(s)\,,\enskip
n=\overline{2,H-1}\,;
\\[6pt]
\bmx{V}_1(s)
&= \bmx{T}^e_1(s)  \bmx{V}_{0}(s) +  \bmx{T}_0(s)  \bmx{V}_{1}(s)\\[6pt]
&\hspace*{28mm}{} + 
\bmx{T}_{-1}(s)  \bmx{V}_{2}(s)\,.
\end{array}
\right\}
\label{eq1-r}
\end{equation}

\noindent The boundary conditions for the system~\eqref{eq1-r} 
have the form:
%%%%%%%%%%%%%%%%%%%%%%%
\begin{equation}
\left.
\begin{array}{rl}
\bmx{V}_{0}(s)&= \bmx{T}^e_0(s)  \bmx{V}_{0}(s)
+  \bmx{T}^e_{-\!1}(s)  \bmx{V}_{1}(s) \,;
\\[6pt]
\bmx{V}_{H}(s)
&=\bmx{E} \,,
\end{array}
\right\}
\label{eq12b}
\end{equation}

\noindent 
where the $(i,j)${th} entries of the matrices 
$\bmx{T}^e_{1}(s)$, $\bmx{T}^e_0(s)$, and $\bmx{T}^e_{-\!1}(s)$
are equal to 
\begin{align*}
\left[ \bmx{T}^e_1(s) \right]_{ij}
&=
\fr{[\bmx{Q}]_{ii} }{[\bmx{Q}]_{ii} - s}
\left[\alpha^{e}\right]_{ij}\,;
\\
\left[ \bmx{T}^e_0(s) \right]_{i,j}
&=
\fr{[\bmx{Q}_0]_{ii}}{[\bmx{Q}_0]_{ii} - s}
\left[\beta^{e}\right]_{ij}
\,;
\\
\left[ \bmx{T}^e_{-1}(s) \right]_{i,j}
&=
\fr{[\bmx{Q}_0]_{ii}}{[\bmx{Q}_0]_{ii} - s}
\left[\gamma^{e}\right]_{ij}\,.
\end{align*}

\noindent 
Note that here, $\bmx{T}^e_0(s)$ is the square matrix of size~$N$
and $\bmx{T}^e_1(s) $ and $\bmx{T}^e_1(s)$ are the
rectangular matrices of size $NM\times N$ and $N\times NM$,
correspondingly.
The solution of the system~\eqref{eq1-r}--\eqref{eq12b}
can be written as 
$$
\bmx{V}_{n}(s)=\bmx{X}_{n}(s) \bmx{V}_{n-1}(s)+\bmx{Y}_{n}(s)\,,
\enskip n=\overline{1,H-1}\,,
$$

\noindent where 
$$
\bmx{X}_{H-1}(s)=(\bmx{E}-\bmx{T}_0(s))\bmx{T}_1(s)\,;
$$
$$
\bmx{Y}_{H-1}(s)=(\bmx{E}-\bmx{T}_0(s))\bmx{T}_{-\!1}(s)\,;
$$
$$
\bmx{X}_{1}(s)=
\left(\bmx{E}-\bmx{T}_0(s)-\bmx{T}_{-1}(s)\bmx{X}_{2}(s)\right)^{-1}\bmx{T}^e_1(s)\,; 
$$

\vspace*{-12pt}

\noindent
\begin{multline*}
\bmx{X}_{n}(s)=\left(\bmx{E}-\bmx{T}_0(s)-\bmx{T}_{-1}(s)\bmx{X}_{n+1}(s)\right)^{-1}
\bmx{T}_1(s)\,, \\
n=\overline{2,H-2}\,;
\end{multline*}

\vspace*{-12pt}

\noindent
\begin{multline*}
\bmx{Y}_{n}(s)=\left(\bmx{E}-\bmx{T}_0(s)\right.\\
\left.{}-\bmx{T}_{-1}(s)\bmx{X}_{n+1}(s)\right)^{-1}
\bmx{T}_{-1}(s)\bmx{Y}_{n+1}(s)\,,
\\ 
 n=\overline{1,H-2}\,;
\end{multline*}

\vspace*{-12pt}

\noindent
\begin{multline*}
\bmx{V}_{0}(s)\\
{}=\left(\bmx{E}-\bmx{T}^e_0(s)-\bmx{T}^e_{-1}(s)\bmx{X}_{1}(s)
\right)^{-1}\bmx{T}^e_{-1}(s)\bmx{Y}_{1}(s)\,. 
\end{multline*}

If the inverse of the matrix $\bmx{T}_{-\!1}(s)$ exists, then 
it is possible to write out the solution of the system~\eqref{eq1-r}--\eqref{eq12b}
using the Kronecker expansion technique (see~\cite{Steeb,Graham,Telek}),
which is based on the identity $\mathrm{vec}\,(\bmx{A}\bmx{B})=
(\bmx{E} \otimes \bmx{A}) \mathrm{vec}\,(\bmx{B})$.
In this identity, ~$\mathrm{vec}$\ denotes the column stacking vector operator
which transforms a~matrix of size $n \times m$ into a~vector of size $nm \times 1$.
We are going to utilize the property of the~$\mathrm{vec}$ operator 
that $\mathrm{vec}\,(\bmx{A})=\bmx{A}$ for matrix~$\bmx{A}$ of size $n \times 1$.
Firstly by applying $\mathrm{vec}$ operator to~\eqref{eq1-r}--\eqref{eq12b},
we get the new system of vector-matrix difference equations:
\begin{align}
\label{H-1}
\mathrm{vec}\left(\bmx{V}_{H-1}(s)\right)
&\notag\\
&\hspace*{-10mm}{}= \bmx{X}^*(s) \mathrm{vec}\left(\bmx{V}_{H-2}(s)\right) +
\mathrm{vec}\left(\bmx{Y}^*(s)\right)\,;
\\
\label{n}
\mathrm{vec}\left(\bmx{V}_{n+1}(s)\right)&=\bmx{X}(s)
\mathrm{vec}\left(\bmx{V}_n(s)\right)\notag\\
&\hspace*{-10mm}{}+\bmx{Y}(s)
\mathrm{vec}\left(\bmx{V}_{n-1}(s)\right)\,,\enskip
n=\overline{2,H-2}\,;
\\
\label{2}
\mathrm{vec}\left(\bmx{V}_{2}(s)\right)&{}\notag\\
&\hspace*{-12mm}{}=
\bmx{X}(s) \mathrm{vec}\left(\bmx{V}_1(s)\right)+
\bmx{Y}^e(s) \mathrm{vec}\left(\bmx{V}_{0}(s)\right)\,;
\\
\label{0}
\mathrm{vec}\left(\bmx{V}_{1}(s)\right) &= \bmx{X}^e(s) \mathrm{vec}\left(\bmx{V}_{0}(s)\right)
\end{align}
where
\begin{align*}
\bmx{X}(s)&= \bmx{E} \otimes \left(\bmx{T}_{-1}(s)\right)^{-1}
\left(\bmx{E}-\bmx{T}_0(s)\right)\,; \\
 \bmx{Y}(s)&= - \left(\bmx{E} \otimes (\bmx{T}_{-1}(s) )^{-1}
\bmx{T}_1(s)\right)\,;
\\
\bmx{X}^e(s)&= \bmx{E} \otimes \left(\bmx{T}^e_{-1}(s) \right)^{-1}
\left(\bmx{E}-\bmx{T}^e_0(s)\right)\,; \\
\bmx{Y}^e(s)&=- \left(\bmx{E} \otimes (\bmx{T}_{-1}(s) )^{-1}
\bmx{T}^e_1(s)\right)\,;\\
\bmx{X}^*(s)&=
\bmx{E} \otimes \left(\bmx{E}-\bmx{T}_0(s)\right)^{-1} \bmx{T}_{1}(s) \,; \\ 
\bmx{Y}^*(s)&=\left(\bmx{E}-\bmx{T}_0(s)\right)^{-1} \bmx{T}_{-1}(s)\,.
\end{align*}

\noindent Secondly, notice that the new system~\eqref{H-1}---\eqref{0}
 consists of pairs of simultaneous equations and thus, its solution can 
 be rewritten as
\begin{multline}
\label{v1}
\begin{pmatrix}
    \mathrm{vec}\left(\bmx{V}_{n}(s)\right) \\
    \mathrm{vec}\left(\bmx{V}_{n-1}(s)\right) 
\end{pmatrix}
={}\\
{}=
\begin{pmatrix}
    \bmx{X}(s) & \bmx{Y}(s) \\
    \bmx{E} & \bmx{0}
\end{pmatrix}^{n-1}
\begin{pmatrix}
    \bmx{X}(s) & \bmx{Y}^e(s) \\
    \bmx{E} & \bmx{0}
\end{pmatrix}
\begin{pmatrix}
    \mathrm{vec}\left(\bmx{V}_{1}(s)\right) \\
    \mathrm{vec}\left(\bmx{V}_{0}(s)\right)
\end{pmatrix}\,, \\
 n=\overline{2,H-1}\,.
\end{multline}


\noindent 
Finally,~\eqref{v1} for $n=H-1$, \eqref{0} and~\eqref{H-1}
make up the system of four matrix equations,
which solution yields the values of 
$\mathrm{vec}\,(\bmx{V}_{H-1}(s))$, $\mathrm{vec}\,(\bmx{V}_{H-2}(s))$,
$\mathrm{vec}\,(\bmx{V}_{1}(s))$, and $\mathrm{vec}\,(\bmx{V}_{0}(s))$.
By virtue of~\eqref{n} and~\eqref{2}, the rest 
of $\bmx{V}_{n}(s)$ can be computed.

Now, let us proceed to the derivation of the equations for 
$\bmx{V}^*_{n}(s)$, $n=\overline{L,r}$,
and $\bmx{V}^{\#}_{n}(s)$, $n=\overline{H+1,r+1}$.
In order to do this, let us introduce the following auxiliary 
square matrices (each of size~$NM$):
\begin{description}
\item[\,] $\bmx{T}^{\#}(s)$~---  
matrix with the  $(i,j)${th} entry 
equal to the LST of the first passage time (of the Markov chain ${\bf{X}}(t)$)
from the state $(i,2,r+1)$ to the state $(j,2,r)$;
\item[\,] 
$\bmx{W}^{\#}_n(s)$, $n=\overline{H+1,r+1}$,~---
matrix with the $(i,j)${th} entry 
equal to the LST of the first passage time 
from the state $(i,2,n)$ to the state $(j,1,H)$;

\item[\,] 
$\bmx{w}^*_n(s)$, $n=\overline{H+1,r}$,~---
matrix with the $(i,j)${th} entry 
equal to the LST of the first passage time 
from the state $(i,1,n)$ to the state $(j,1,H)$
without visiting the states $(\cdot,2,r+1)$;

\item[\,] 
$\bmx{\ov w}^*_n$, $n=\overline{H+1,r}$,~---
matrix with the $(i,j)${th} entry 
equal to the LST of the first passage time 
from the state $(i,1,n)$ to the state $(j,2,r+1)$
without visiting the states $(\cdot,1,H)$; and
\item[\,] 
$\bmx{W}^*_n(s)$, $n=\overline{H+1,r}$,~---
matrix with the $(i,j)${th} entry 
equal to the LST of the first passage time 
from the state $(i,1,n)$ to the state $(j,1,H)$.
\end{description}

Let us begin with the relation for~$\bmx{T}^{\#}(s)$. 
Let~$\bmx{T}^{\#}_k(s)$, $k=0,1$, denote 
the square matrix of size $NM$ with 
the $(i,j)${th} entry 
equal to the LST of the first passage time 
from the state $(i,2,r+1)$ to the state $(j,2,r+k)$.
Since the sojourn time in the state $(i,2,r+1)$
is distributed exponentially with the rate $-[\bmx{Q}^{\#}]_{ii}$, 
one has for $1 \le i,j \le NM$:
\begin{gather*}
\left[ \bmx{T}^{\#}_0(s) \right]_{ij}
=
\fr{[\bmx{Q}^{\#}]_{ii}}{[\bmx{Q}^{\#}]_{ii} - s}
\left[\beta^{\#}\right]_{ij}  \,;
\\
\left[ \bmx{T}^{\#}_1(s) \right]_{ij}
= \fr{[\bmx{Q}^{\#}]_{ii} }{[\bmx{Q}^{\#}]_{ii} - s}
\left[\alpha^{\#}\right]_{ij}  \,.
\end{gather*}

\noindent The first-step analysis yields the following 
equation for~$\bmx{T}^{\#}(s)$:
$$
\bmx{T}^{\#}(s)= \bmx{T}^{\#}_1(s) + \bmx{T}^{\#}_0(s) \bmx{T}^{\#}(s)\,,
$$
which solution is
$$
\bmx{T}^{\#}(s)= \left(\bmx{E}-\bmx{T}^{\#}_0(s)\right)^{-1} \bmx{T}^{\#}_1(s)\,.
$$


Since in the ``blocking'' mode any arrival is lost, then the sojourn time in 
it is equal to the time needed for $(n-H)$ service completions,
given that initially, there were $n$ customers in the system, i.\,e.,
$$
\bmx{W}^{\#}_n(s)= \left(\bmx{T}^{\#}(s)\right)^{n-H}\,, \enskip
n=\overline{H+1,r+1}\,.
$$

Equations for $\bmx{w}^*_n(s)$ and $\bmx{\ov w}^*_n(s)$ can be 
derived by following the same arguments given above for~$\bmx{V}_n(s)$. Denote by 
$\bmx{T}^{*}_k(s)$, $k=-1,0,1$,  
the square matrix of size~$NM$ with 
the $(i,j)${th} entry 
equal to the LST of the first passage time 
from the state $(i,1,n)$ to the state $(j,1,n-k)$.
Then, using the fact that the sojourn time in the state $(i,1,n)$
is distributed exponentially with the rate~$-[\bmx{Q}^{*}]_{ii}$, 
one obtains for $1 \le i,j \le NM$:
\begin{align*}
\left[ \bmx{T}^{*}_{-1}(s) \right]_{ij}
&=
\fr{[\bmx{Q}^{*}]_{ii}}{[\bmx{Q}^{*}]_{ii} - s}
\left[\gamma^{*}\right]_{ij}\,;
\\
\left[ \bmx{T}^{*}_0(s) \right]_{ij}
&=\fr{[\bmx{Q}^{*}]_{ii}}{[\bmx{Q}^{*}]_{ii} - s}
\left[\beta^{*}\right]_{ij}  \,;
\\
\left[ \bmx{T}^{*}_1(s) \right]_{i,j}
&=
\fr{[\bmx{Q}^{*}]_{ii} }{[\bmx{Q}^{*}]_{ii} - s}
\left[\alpha^{*}\right]_{ij}   \,.
\end{align*}

Again, by applying the first-step analysis, one gets 
the following system of matrix difference equations 
for~$\bmx{w}^*_n(s)$, $n=\overline{H+1,r}$:
\begin{multline}
\label{w*n}
\bmx{w}^*_n(s)
= \bmx{T}^*_1(s)  \bmx{w}^*_{n-1}(s)\\
{} +  \bmx{T}^*_0(s) \bmx{w}^*_n(s) + 
\bmx{T}^*_{-\!1}(s)  \bmx{w}^*_{n+1}(s)\,,
\end{multline}

\noindent with the boundary conditions
\begin{equation}
\label{w*nbound}
\bmx{w}^*_H(s)
=
\bmx{E}\,;
\enskip  
\bmx{w}^*_{r+1}(s)
= 
\bmx{0}\,.
\end{equation}


\noindent Clearly, the matrices $\bmx{\ov w}^*_n(s)$, $n=\overline{H+1,r}$,
satisfy the system of equations, which is identical to~\eqref{w*n},
i.\,e.,
\begin{multline}
\label{w*nOV} \bmx{\ov w}^*_n(s)
= \bmx{T}^*_1(s)  \bmx{\ov w}^*_{n-1}(s)
+  \bmx{T}^*_0(s) \bmx{\ov w}^*_n(s)\\
{}+  \bmx{T}^*_{-1}(s) 
\bmx{\ov w}^*_{n+1}(s)\,,
\end{multline}

\noindent but with the ``reversed'' boundary conditions:
\begin{equation}
\label{w*nboundOV}
\bmx{\ov w}^*_H(s)
= \bmx{0}\,; \enskip  
\bmx{\ov w}^*_{r+1}(s) =  \bmx{E}\,.
\end{equation}

The structure of the systems~\eqref{w*n}, \eqref{w*nbound} 
and~\eqref{w*nOV}, \eqref{w*nboundOV} is
similar to the~\eqref{eq1-r}--\eqref{eq12b} (expect for the boundary conditions).
Thus, its solutions can be found completely in the same way and, therefore, are omitted.
Once~$\bmx{w}^*_n(s)$ and~$\bmx{\ov w}^*_n(s)$ are found, the 
matrices~$\bmx{W}^*_n(s)$ can be computed. Noticing that from the 
state $(i,1,n)$, $n=\overline{H+1,r}$, the Markov chain can enter 
the state $(j,H,1)$ either from the set of ``overload'' states or
from the set of ``blocking'' states (see figure), one has:
$$
\bmx{W}^*_n(s)= \bmx{w}^*_n(s)+\bmx{\ov w}^*_n(s)\bmx{W}^{\#}_{r+1}(s)\,,
\enskip n=\overline{H+1,r}\,.
$$

\columnbreak

Now, everything is prepared for the derivation of
the relations for the unknown quantities $\bmx{V}^*_n(s)$  
and $\bmx{V}^{\#}_n(s)$. 
If $n=\overline{L,H}$, then $\bmx{V}^*_n(s)$ satisfy the following
system of matrix difference equations:
\begin{multline}
\label{V*}
\bmx{V}^*_n(s)
= \bmx{T}^*_1(s)  \bmx{V}^*_{n-1}(s) + 
\bmx{T}^*_0(s) \bmx{V}^*_n(s)\\
{} +  \bmx{T}^*_{-\!1}(s)  \bmx{V}^*_{n+1}(s)\,,
\end{multline}

\noindent with the boundary conditions
\begin{align*}
\bmx{V}^*_{L-1}(s)
&= \bmx{E}\,;
\\[3pt]
\bmx{V}^*_{H}(s)&=  \bmx{T}^*_1(s)  \bmx{V}^*_{H-1}(s)\\
&{}  + 
\bmx{T}^*_0(s) \bmx{V}^*_H(s)
+  \bmx{T}^*_{-1}(s) 
\bmx{W}^*_{H+1}(s)
\bmx{V}^*_{H}(s)\,.
%\label{w*nboundOV1}
\end{align*}


The final expressions for the matrices
$\bmx{V}^*_n(s)$, $n=\overline{H+1,r}$
and $\bmx{V}^{\#}_n(s)$, $n=\overline{H+1,r+1}$,
have the form:
\begin{alignat*}{2}
\bmx{V}^{\#}_{n}(s)
&=  \bmx{W}^{\#}_n(s) 
\bmx{V}^*_{H}(s)\,, &\enskip n&=\overline{H+1,r+1}\,;
\\
\bmx{V}^{*}_{n}(s)
&=  \bmx{W}^{*}_n(s) 
\bmx{V}^*_{H}(s)\,, &\enskip n&=\overline{H+1,r}\,.
\end{alignat*}

The last two relations, together with~\eqref{eq1-r} and~\eqref{V*}, 
give the complete solution of the considered problem.
The matrices $\bmx{V}_n(s)$, $\bmx{V}^*_n(s)$, and $\bmx{V}^{\#}_n(s)$
allow one to calculate various performance characteristics of
the hysteretic policy such as (conditional\footnote{The corresponding 
unconditional characteristics 
are obtained by weighting according to the joint stationary distribution 
found in~\cite{int0}.}) 
mean duration of overload period (equal to $-[\bmx{V}^*_H(s)]'_{s=0}$),
(conditional) mean return time to the ``overload'' 
mode (equal to $-[\bmx{V}_{L-1}(s)]'_{s=0}- [\bmx{V}^*_H(s)]'_{s=0}$),
higher moments, etc.

No principal difficulties show up if in addition 
to the sojourn times one needs to count 
the number of lost customers. The same
argumentation applies. For example, 
let us assume that customers arrived during the
period of time when the system 
is in the ``blocking'' mode are considered as lost.
Then, the LST and the generating function~$\bmx{W}^{\#}_n(s,z)$
of the joint stationary distribution of the 
sojourn time in the ``blocking'' mode 
and the number of lost customers 
provided that the system is in the ``blocking'' mode 
and there are $n$ customers in it 
is equal to 
\begin{multline*}
\bmx{W}^{\#}_n(s,z)
=  \left \{
\left[ \bmx{E} - z ( \bmx{E}-\bmx{T}^*_0(s))^{-1}\bmx{T}^*_{-1}(s) \right]^{-1}\right.\\
\left.{}\times
\left( \bmx{E}-\bmx{T}^*_0(s)\right)^{-1}\bmx{T}^*_{1}(s)
\right \}^{H+1-n}
\times{} \\
{}\times
\left \{ \bmx{E} +  
z \left[ \bmx{E} - z \left( \bmx{E}-\bmx{T}^*_0(s)\right)^{-1}
\bmx{T}^*_{-1}(s) \right]^{-1}
\right \}\\
{}\times
\left( \bmx{E}-\bmx{T}^*_0(s)\right)^{-1}\bmx{T}^*_{1}(s)\,.
\end{multline*}

\noindent Thus, the substitution of~$\bmx{W}^{\#}_n(s,z)$ 
instead of~$\bmx{W}^{\#}_n(s)$ in the above expressions
will account not only for the sojourn time but for losses
(during the sojourn time). 



\section{Concluding Remarks}

\noindent
The approach proposed in the paper allows one to calculate
the system's sojourn time in various modes 
in terms of LST by exploiting the knowledge
about the presence of hysteretic loops. 
Minor changes are needed to adapt it to the case of overlapping loops. 
Of course, due to MAP arrivals and PH service times, it
utilizes matrix analytic techniques and, thus, possesses
the disadvantages inherent to matrix algebra.
Despite the fact there is large body of research results 
available in this topic, 
there is still a~number of open questions.
Is there any analytic approach to find the steady-state
behavior of several interconnected systems each with hysteretic policy, 
which exploits the knowledge of the presence of hysteretic loops?
What is the gain of hysteretic control of arrivals with
respect to other types of control? Just to name a~few.

\vspace*{-4pt}


\Ack
\noindent
This work was supported by the Russian Science Foundation (grant No.\,16-11-10227).

\renewcommand{\bibname}{\protect\rmfamily References}

\vspace*{-4pt}


{\small\frenchspacing
{%\baselineskip=10.8pt
\begin{thebibliography}{9}

\bibitem{int0} %1
\Aue{Razumchik, R.} 2016. Analysis of finite ${\mathrm{MAP}/\mathrm{PH}/1}$ 
queue with hysteretic control of arrivals.
\textit{Congress (International)
on Ultra Modern Telecommunications and Control Systems and Workshops Proceedings}.
 Lisbon. 288--293.



\bibitem{int2} %2
\Aue{Chesoong, K., A.~Dudin, S.~Dudin, and O.~Dudina.} 2016.
Hysteresis control by the number of active servers in queueing system ${\mathrm{MMAP}/\mathrm{PH}/N}$
with priority service. \textit{Perform. Evaluation} 101:20--33.

\bibitem{int3} %3
\Aue{Chan, C.\,W., M.~Armony, and N.~Bambos.} 2016.
Maximum weight matching with hysteresis in overloaded queues with setups.
\textit{Queueing Sy.} 82(3-4):315--351.

\bibitem{int4} %4
\Aue{Rumyantsev, A.\,S., K.\,A.~Kalinina, and T.\,E.~Morozova.} 2017.
Stokhasticheskioe modelirovanie vycheslitel'nogo klastera s~gisterezisnym upravleniem
skorost'yu obsluzhivaniya
[Stochastic modeling of a~high-performance cluster
with hysteretic control of service rate].
\textit{Trudy Karel'skogo nauchnogo tsentra RAN}
[Transactions of KarRC RAS] 8:76--85.

\bibitem{int1} %5
\Aue{Abaev, P., Y.~Gaidamaka, K.~Samouylov, A.~Pechinkin, R.~Razumchik, and S.~Shorgin.} 2014.
Hysteretic control technique for overload problem solution in network of SIP servers.
\textit{Comput. Inform.} 33(1):1--18.




\bibitem{Graham} %6
\Aue{Graham, A.} 1982.
\textit{Kronecker products and matrix calculus: With applications}. 
New York, NY: John Wiley \& Sons. 130~p.

\bibitem{Steeb} %7
\Aue{Steeb, W.\,H., and Y.~Hardy.} 2011.
\textit{Matrix calculus and Kronecker product: A~practical 
approach to linear and multilinear algebra}. 2nd ed. River Edge,
NJ: World Scientific. 324~p.

\bibitem{Telek} %8
\Aue{Razumchik, R., and M.~Telek.} 2016.
Delay analysis of a~queue with re-sequencing buffer
and Markov environment.
\textit{Queueing Sy.} 82(1-2):7--28.

\end{thebibliography} }
 }

\end{multicols}

\vspace*{-3pt}

\hfill{\small\textit{Received September 19, 2017}}

%\vspace*{-12pt}


\Contrl

\noindent
\textbf{Razumchik Rostislav V.} (b.\ 1984)~--- 
Candidate of Science (PhD) in physics and mathematics, leading scientist, 
Institute of Informatics Problems, Federal Research Center 
``Computer Science and Control'' of the Russian Academy of Sciences, 
44-2~Vavilov Str., Moscow 119333, Russian Federation; associate professor, 
Peoples' Friendship University of Russia (RUDN University), 
6~Miklukho-Maklaya Str., Moscow 117198, Russian Federation; 
\mbox{rrazumchik@ipiran.ru}
%; \mbox{razumchik\_rv@rudn.university}

%\vspace*{8pt}

%\hrule

%\vspace*{2pt}

%\hrule



\newpage

\vspace*{-28pt}



\def\tit{СТАЦИОНАРНЫЕ РАСПРЕДЕЛЕНИЯ, СВЯЗАННЫЕ СО~ВРЕМЕНЕМ ПРЕБЫВАНИЯ
В~СОСТОЯНИИ ПЕРЕГРУЗКИ СИСТЕМЫ $\mathrm{MAP}/\mathrm{PH}/1/r$ С~ГИСТЕРЕЗИСНЫМ УПРАВЛЕНИЕМ 
НАГРУЗКОЙ$^*$}

\def\aut{Р.\,В.~Разумчик}


\def\titkol{Стационарные распределения, связанные со временем пребывания
в~состоянии перегрузки системы $\mathrm{MAP}/\mathrm{PH}/1/r$}
% с~гистерезисным управлением  нагрузкой}

\def\autkol{Р.\,В.~Разумчик}

{\renewcommand{\thefootnote}{\fnsymbol{footnote}}
\footnotetext[1]{Работа выполнена при поддержке РНФ (проект 16-11-10227).}}


\titel{\tit}{\aut}{\autkol}{\titkol}

\vspace*{-12pt}

\noindent
Институт проблем информатики
Федерального исследовательского центра <<Информатика и~управ\-ле\-ние>>
Российской академии наук; Российский университет дружбы народов,
\mbox{rrazumchik@ipiran.ru}

\vspace*{6pt}

\def\leftfootline{\small{\textbf{\thepage}
\hfill ИНФОРМАТИКА И ЕЁ ПРИМЕНЕНИЯ\ \ \ том\ 11\ \ \ выпуск\ 4\ \ \ 2017}
}%
 \def\rightfootline{\small{ИНФОРМАТИКА И ЕЁ ПРИМЕНЕНИЯ\ \ \ том\ 11\ \ \ выпуск\ 4\ \ \ 2017
\hfill \textbf{\thepage}}}


\Abst{Как известно, одним из решений проблемы перегрузок
в~сетях SIP (session initiation protocol)
сиг\-на\-ли\-за\-ции является применение
в~узлах сети (SIP-сер\-ве\-рах) многоуровневого
гистерезисного управления нагрузкой.
В~данной работе представлены некоторые новые 
результаты анализа системы $\mathrm{MAP}/\mathrm{PH}/1/r$ конечной емкости
с~двумя петлями гистерезисного управления,
функционирующей в случайной среде и являющейся
моделью SIP-сер\-ве\-ра с двухуровневым
гистерезисным управлением нагрузкой.
Получен метод вычисления преобразования
Лап\-ла\-са--Стилть\-еса
функций распределения времени возврата системы из
множества состояний перегрузки в множество
состояний нормальной нагрузки
и~времени выхода системы из множества состояний
нормальной нагрузки.
Метод основан на решении матричных рекуррентных уравнений
и~применим в случае, когда помимо расчета времени
выхода из состояния перегрузки необходимо
также учитывать и число потерянных за это время
заявок.}

\KW{система массового обслуживания; случайная среда; гистерезисное управление; 
время пребывания}

\DOI{10.14357/19922264170403}

\vspace*{18pt}


 \begin{multicols}{2}

\renewcommand{\bibname}{\protect\rmfamily Литература}
%\renewcommand{\bibname}{\large\protect\rm References}

{\small\frenchspacing
{%\baselineskip=10.8pt
\begin{thebibliography}{9}

\bibitem{1-r-1}
\Au{Razumchik R.} 
Analysis of finite ${\mathrm{MAP}/\mathrm{PH}/1}$ queue with hysteretic control of arrivals~// 
 Congress (International) on Ultra Modern Telecommunications and Control 
Systems and Workshops Proceedings.~--- Lisbon, 2016.  P.~288--293.


\bibitem{3-r-1} %2
\Au{Chesoong K., Dudin~A., Dudin~S., Dudina~O.}
Hysteresis control by the number of active servers in queueing system 
$\mathrm{MMAP}/\mathrm{PH}/N$
with priority service~// Perform. Evaluation, 2016. Vol.~101. P.~20--33.

\bibitem{4-r-1} %3
\Au{Chan C.\,W., Armony~M., Bambos~N.}
Maximum weight matching with hysteresis in overloaded queues with setups~// 
Queueing Sy., 2016. Vol.~82. No.\,3-4. P.~315--351.

\bibitem{5-r-1} %4
\Au{Румянцев А.\,С., Калинина~К.\,А., Морозова~Т.\,Е.}
Стохастическое моделирование вычислительного кластера 
с~гистерезисным управлением скоростью обслуживания~//
Труды Карельского научного центра РАН, 2017. Вып.~8. С.~76--85.

\bibitem{2-r-1} %5
\Au{Abaev P., Gaidamaka~Y.,  Samouylov~K.,  Pechinkin~A.,  Razumchik~R.,  Shorgin~S.}
Hysteretic control technique for overload problem solution in network of SIP servers~//
Comput. Inform., 2014. Vol.~33. No.\,1. P.~1--18.





\bibitem{7-r-1} %6
\Au{Graham A.}
Kronecker products and matrix calculus: With applications.~--- 
New York, NY, USA: John Wiley \& Sons, 1982. 130~p.

\bibitem{6-r-1} %7
\Au{Steeb W.\,H., Hardy~Y.}
Matrix calculus and Kronecker product: A~practical
approach to linear and multilinear algebra.~--- 2nd ed.~---
River Edge, NJ, USA: World Scientific, 2011. 324~p.

\bibitem{8-r-1}
\Au{Razumchik~R., Telek~M.}
Delay analysis of a~queue with re-sequencing buffer
and Markov environment~// Queueing Sy., 2016. Vol.~82. 
No.\,1-2. P. 7--28.

\end{thebibliography}
} }

\end{multicols}

 \label{end\stat}

 \vspace*{-3pt}

\hfill{\small\textit{Поступила в~редакцию  19.09.2017}}
%\renewcommand{\bibname}{\protect\rm Литература}
\renewcommand{\figurename}{\protect\bf Рис.}
\renewcommand{\tablename}{\protect\bf Таблица}   %3

%\newcommand{\norm}[1]{\left\Vert#1\right\Vert}
%\newcommand{\abs}[1]{\left\vert#1\right\vert}
%\newcommand{\eps}{\varepsilon}
%\renewcommand{\r}{\mathbb R}
%\newcommand{\N}{\mathbb N}
%\renewcommand{\P}{{\sf P}}
%\newcommand{\E}{{\sf E}}
%\newcommand{\D}{{\sf D}}
%\newcommand{\sign}{{\rm sign}}
%\renewcommand{\le}{\leqslant}
%\renewcommand{\ge}{\geqslant}
%\newcommand{\I}{\mathbb{I}}
%\newcommand{\betm}{{\beta_{m+1+\delta}}}
%\newcommand{\bet}{\beta_{2+\delta}}
%\renewcommand{\endproof}{\hfill$\Box$}
%\renewcommand{\phi}{\varphi}
%\newcommand{\la}{\lambda}
%\newcommand{\si}{{\rm Si}\:}
%\renewcommand{\Re}{{\rm Re}\:}
%\newcommand{\eqd}{\stackrel{d}{=}}

\def\stat{korolev}

\def\tit{ОЦЕНКИ СКОРОСТИ СХОДИМОСТИ РАСПРЕДЕЛЕНИЙ НЕКОТОРЫХ СЛУЧАЙНЫХ СУММ К~УСТОЙЧИВЫМ 
ЗАКОНАМ$^*$}

\def\titkol{Оценки скорости сходимости распределений некоторых случайных сумм к устойчивым 
законам}

\def\autkol{В.\,Ю.~Королев,  Л.\,М.~Закс}

\def\aut{В.\,Ю.~Королев$^1$,  Л.\,М.~Закс$^2$}

\titel{\tit}{\aut}{\autkol}{\titkol}

{\renewcommand{\thefootnote}{\fnsymbol{footnote}}
\footnotetext[1] {Работа поддержана Российским фондом фундаментальных исследований (проекты 
12-07-00115а, 12-07-00109а,  11-01-00515а и 11-07-00112а).}}

\renewcommand{\thefootnote}{\arabic{footnote}}
\footnotetext[1]{Факультет вычислительной математики и кибернетики 
Московского государственного университета им.\ М.\,В.~Ломоносова; Институт 
проблем информатики РАН, vkorolev@cs.msu.su}
\footnotetext[2]{Альфа-банк, 
отдел моделирования и математической статистики, lily.zaks@gmail.com}

\vspace*{6pt}

\Abst{Приведены оценки скорости сходимости распределений
специальных сумм случайного числа независимых одинаково
распределенных случайных величин с конечными дисперсиями к
симметричным строго устойчивым законам. Предполагается, что
случайный индекс имеет смешанное пуассоновское распределение, в
котором смешивающее распределение является устойчивым законом,
сосредоточенным на положительной полуоси. Абсолютные константы
выписаны в явном виде.}

\vspace*{1pt}

\KW{устойчивое распределение; неравенство
Бер\-ри--Эс\-се\-ена; случайная сумма; дважды стохастический пуассоновский
процесс (процесс Кокса); смешанное пуассоновское распределение}


\vspace*{4pt}

 \vskip 14pt plus 9pt minus 6pt

      \thispagestyle{headings}

      \begin{multicols}{2}

            \label{st\stat}



Функцию распределения и плотность строго устойчивого распределения с
характеристическим показателем $\alpha$ и параметром~$\theta$,
задаваемого характеристической функцией
\begin{multline}
\mathfrak{g}_{\alpha,\theta}(t)={}\\
{}=\exp\left\{-|t|^{\alpha}\exp\left\{
-\fr{i\pi\theta\alpha}{2}\mathrm{sign}t\right\}\right\}\,,\enskip
t\in\r\,,\label{e1-kor}
\end{multline}
где $0<\alpha\hm\le2$,
$|\theta|\hm\le\theta_{\alpha}\hm=\min\{1,{2}/{\alpha}-1\}$, будем обозначать 
соответственно $G_{\alpha,\theta}(x)$ и $g_{\alpha,\theta}(x)$. Симметричным 
строго устойчивым распределениям соответствует значение $\theta\hm=0$. 
Односторонним устойчивым распределениям соответствуют значения $\theta\hm=1$ и 
$0\hm<\alpha\hm\le1$. Функцию распределения и плотность стандартного 
нормального закона ($\alpha\hm=2$, $\theta\hm=0$) будем обозначать 
соответственно $\Phi(x)$ и~$\phi(x)$:
$$
\phi(x)=\fr{1}{\sqrt{2\pi}}\,e^{-x^2/2}\,;\quad
\Phi(x)=\int\limits_{-\infty}^x\phi(z)\,dz\,.
$$

Рассмотрим последовательность независимых одинаково распределенных
случайных величин $X_1,X_2,\ldots$, заданных на некотором
вероятностном пространстве $(\Omega,\, \mathfrak{A},\,{\sf P})$.
Будем предполагать, что
\begin{equation*}
{\sf E}X_1=0\,, \enskip 0<\sigma^2={\sf D}X_1<\infty\,. %\label{e2-kor}
\end{equation*}
Для натурального $n\hm\ge1$ положим
$$
S_n=X_1+\cdots+X_n\,.
$$
Пусть $N_1,N_2,\ldots$~--- последовательность це\-ло\-чис\-лен\-ных 
неотрицательных случайных величин, заданных на том же самом вероятностном 
пространстве так, что при каждом $n\hm\ge1$ случайная величина $N_n$ независима 
от последовательности $X_1,X_2,\ldots$ Всюду далее для определенности будем 
считать, что $\sum\limits_{j=1}^0\hm=0$.

Принято считать, что случайная последовательность $N_1,N_2,\ldots$
неограниченно возрастает ($N_n\hm\longrightarrow\infty$) по
вероятности, если для любого $m\hm\in(0,\infty)$ ${\sf P}(N_n\hm\le
m)\longrightarrow 0$ при $n\hm\to\infty$. Всюду далее символы
$\Longrightarrow$ и $\eqd$ обозначают соответственно сходимость по
распределению и совпадение распределений.

В статье~\cite{Korolev1997} доказан следующий критерий схо\-димости
сумм случайного числа независимых одинако\-во распределенных случайных
величин \textit{с конечными дисперсиями} к симметричным строго
устойчивым законам.

\smallskip

\noindent
\textbf{Лемма 1.} \textit{Предположим, что случайные величины
$X_1,X_2,\ldots$ и $N_1,N_2,\ldots$ удовлетворяют указанным выше
условиям, причем $N_n\longrightarrow\infty$ по вероятности при
$n\hm\to\infty$. Для того чтобы при} $n\hm\to\infty$
$$
{\sf P}\left(\fr{S_{N_n}}{\sigma\sqrt{n}}<x\right) \Longrightarrow
G_{\alpha,0}(x)\,,
$$
\textit{необходимо и достаточно, чтобы}
$$
{\sf P}(N_n<nx)\Longrightarrow G_{\alpha/2,1}(x)\,.
$$



В лемме~1 главным условием является сходимость распределений
нормированных индексов $N_n$ к одностороннему строго устойчивому
распределению $G_{\alpha/2,1}(x)$. Далее будет рассматриваться
довольно полезная с точки зрения практических приложе\-ний специальная
ситуация, в которой это условие выполнено.

В книге~\cite{GnedenkoKorolev1996} предложено моделировать эволюцию\linebreak
неоднородных хаотических стохастических процессов, в частности
динамику цен финансовых активов, с помощью обобщенных дважды
стохастических пуассоновских процессов (обобщенных\linebreak процессов Кокса).
Этот подход получил дополнительное обоснование и развитие в книгах~[3--6]. 
В~книгах~\cite{Korolev2011, KorolevSkvortsova2006} этот
подход успешно применен к моделированию процессов плазменной
турбулентности. В~соответствии с указанным подходом поток
информативных событий, в результате каждого из которых появляется
очередное <<наблюденное>> значение рассматриваемой характеристики,
описывается с помощью точечного случайного процесса вида
$M(\Lambda(t))$, где $M(t)$, $t\hm\geq0$,~--- однородный пуассоновский
процесс с единичной ин\-тен\-сив\-ностью, а $\Lambda(t)$, $t\hm\geq0$,~---
независимый от $M(t)$ случайный процесс, обладающий следующими
свойствами: $\Lambda(0)\hm=0$, ${\sf P}(\Lambda(t)\hm<\infty)\hm=1$ для
любого $t\hm>0$, траектории $\Lambda(t)$ не убывают и непрерывны
справа. Процесс $M(\Lambda(t))$, $t\hm\geq0$, называется дважды
стохастическим пуассоновским процессом (процессом Кокса). 
В~частности, если процесс $\Lambda(t)$ допускает представление
$$
\Lambda(t)=\int\limits_{0}^{t}\lambda(\tau)\,d\tau\,,\enskip t\ge0\,,
$$
в котором $\lambda(t)$~--- положительный случайный процесс с
интегрируемыми траекториями, то $\lambda(t)$ можно интерпретировать
как мгновенную стохастическую интенсивность процесса Кокса.

В соответствии с такой моделью в каждый момент времени~$t$
распределение случайной величины $M(\Lambda(t))$ является смешанным
пуассоновским. Для большей наглядности рассмотрим случай, когда в
рассматриваемой модели время~$t$ остается фиксированным, а
$\Lambda(t)\hm=nU_{\alpha/2,1}$, где $n$~--- вспомогательный параметр,
$U_{\alpha/2,1}$~--- случайная величина c функцией распределения
$G_{\alpha/2,1}(x)$, независимая от стандартного пуассоновского
процесса $M(t)$, $t\hm\ge0$. При этом асимптотика $n\hm\to\infty$ может
интерпретироваться как то, что (случайная) интенсивность потока
информативных событий считается очень большой. Для каждого
натурального~$n$ положим
$$
N_n=M(nU_{\alpha/2,1})\,.
$$
Очевидно, что так определенная случайная величина $N_n$ имеет
смешанное пуассоновское распределение:

\noindent
\begin{multline}
{\sf P}(N_n=k)={\sf P}
\left(M(nU_{\alpha/2,1})=k\right)={}\\
{}=
\int\limits_0^{\infty}e^{-nz}\fr{(nz)^k}{k!}\,g_{\alpha/2,1}(z)\,dz\,,\enskip
k=0,1,\ldots
\label{e3-kor}
\end{multline}
Случайная величина $N_n$ может быть интер\-пре\-ти\-рована как число
событий, зарегистрированных к моменту времени~$n$ в пуассоновском
процессе со случайной интенсивностью, имеющей строго устойчивую
плотность $g_{\alpha/2,1}(z)$. Высокая адекватность устойчивых
распределений как моделей статистических закономерностей динамики
цен финансовых активов отмечается во многих работах (см., например,~\cite{McCulloch1996}).

Предположим, что случайная величина $U_{\alpha/2,1}$ и пуассоновский
процесс $M(t)$ независимы от последовательности $X_1,X_2,\ldots$
Тогда, очевидно, при каждом~$n$ случайная величина~$N_n$ также будет
независима от этой последовательности.

Обозначим $A_n(z)={\sf P}(N_n<nz)$, $z\hm\ge0$ ($A_n(z)\hm=0$ при $z\hm<0$).
Несложно видеть, что
$$
A_n(x)\Longrightarrow G_{\alpha/2,1}(x)\enskip (n\to\infty)\,.
$$
Действительно, как известно, если $\Pi(x;\ell)$~--- функция
распределения Пуассона с параметром $\ell\hm>0$ и $E(x;c)$~--- функция
распределения с единственным единичным скачком в точке $c\hm\in\r$, то
$$
\Pi(\ell x;\ell)\Longrightarrow E(x;1)\enskip (\ell\to\infty)\,.
$$
Так как для $x\in\r$
$$
A_n(x)=\int\limits_{0}^{\infty}\Pi(n x; n z)\,dG_{\alpha/2,1}(z)\,,
$$
то по теореме Лебега о мажорируемой сходимости при $n\hm\to\infty$
\begin{multline*}
A_n(x)\Longrightarrow\int\limits_{0}^{\infty}E(x/z;1)\,dG_{\alpha/2,1}(z)={}\\
{}=
\int\limits_{0}^{x}\,\,dG_{\alpha/2,1}(z)=G_{\alpha/2,1}(x)\,,
\end{multline*}
т.\,е.\ так определенные случайные величины $N_n$\linebreak удовлетворяют
условию, фигурирующему в леммe~1.

В дополнение к сформулированным выше условиям на случайные величины
$X_1,X_2,\ldots$ предположим, что
\begin{equation}
\beta^3={\sf E}|X_1|^3<\infty\,.\label{e4-kor}
\end{equation}
Обозначим
$$
D_{n,\alpha}=\sup_x\left\vert {\sf
P}\left(S_{N_n}<x\sigma\sqrt{n}\right)-G_{\alpha,0}(x)\right\vert\,.
$$


\smallskip

\noindent
\textbf{Теорема~1.} \textit{Пусть выполнены условия}~(\ref{e3-kor}) и~(\ref{e4-kor}). \textit{Для
любого $n\ge1$ справедлива оценка}
$$
D_{n,\alpha}\le0{,}2428
\fr{\Gamma\left({1}/{\alpha}\right)\beta^3}{\alpha\sigma^3\sqrt{n}}\,.
$$

\smallskip

\noindent
Д\,о\,к\,а\,з\,а\,т\,е\,л\,ь\,с\,т\,в\,о\,.\ \  Распределение случайной величины $N_n$
является смешанным пуассоновским (см.~(\ref{e3-kor})). Следовательно, по теореме
Фубини
\begin{multline}
{\sf P}\left(S_{N_n}<x\sigma\sqrt{n}\right)=
{\sf P}\left(S_{M(nU_{\alpha/2,1})}<x\sigma\sqrt{n}\right)={}\\
{}=\int\limits_0^{\infty}{\sf P}\left(S_{M(nz)}<x\sigma\sqrt{n}\right)
g_{\alpha/2,1}(z)\,dz\,.\label{e5-kor}
\end{multline}
Далее, как известно, симметричное строго устойчивое распределение с
параметром~$\alpha$ является масштабной смесью нормальных законов, в
которой смешивающим распределением является односторонний устойчивый
закон ($\theta\hm=1$) с параметром $\alpha/2$:
\begin{equation}
G_{\alpha,0}(x)=\int\limits_{0}^{\infty}\Phi\left(\fr{x}{\sqrt{z}}\right)\,dG_{\alpha/2,1}(z)\,,\enskip
x\in\r\label{e6-kor}
\end{equation}
(см., например, теорему~3.3.1 в~\cite{Zolotarev1983}). Из~(\ref{e5-kor}) и~(\ref{e6-kor})
следует, что
\begin{multline}
D_{n,\alpha}\le{}\\
\!{}\le\int\limits_0^{\infty}\!\!\sup\limits_x\left\vert{\sf P}\!
\left(\fr{S_{M(nz)}}{\sigma\sqrt{n}}<x\right)-
\Phi\left(\fr{x}{\sqrt{z}}\right)\right\vert\,dG_{\alpha/2,1}(z)={}\\
\!\!\!{}= \int\limits_0^{\infty}\sup\limits_x \left\vert{\sf P}
\left(\fr{S_{M(nz)}}{\sigma\sqrt{nz}}<x\right)-\Phi(x)\right\vert\,dG_{\alpha/2,1}(z).\!\!
\label{e7-kor}
\end{multline}
Подынтегральное выражение в~(\ref{e7-kor}) оценим с по\-мощью следующего аналога
неравенства Бер\-ри--Эс\-се\-ена для пуассоновских случайных сумм.

\medskip

\noindent
\textbf{Лемма 2.} \textit{Пусть случайные величины $X_1,X_2,\ldots$
одинаково распределены, причем ${\sf E}X_1\hm=0$ и ${\sf E}|X_1|^3\hm<\infty$. 
Пусть $N_{\lambda}$~--- пуассоновская случайная
величина с параметром $\lambda\hm>0$. Предположим, что случайные
величины $N_{\lambda},X_1,X_2,\ldots$ независимы в совокупности.
Обозначим}
$$
Z_{\lambda}=X_1+\cdots+X_{N_{\lambda}}\,.
$$
\textit{Тогда}
$$
\sup\limits_x\left\vert {\sf P}\left(\fr{Z_{\lambda}}{\sqrt{{\sf D}Z_{\lambda}}}<x \right)-
\Phi(x)\right\vert\le\fr{0{,}3041}{\sqrt{\lambda}}\,\fr{{\sf E}|X_1|^3}{({\sf E}X_1^2)^{3/2}}\,.
$$

\smallskip

\noindent
Д\,о\,к\,а\,з\,а\,т\,е\,л\,ь\,с\,т\,в\,о\ этого утверждения приведено 
в~\cite{KorolevShevtsova2010}, также см.\ теорему~2.4.3 в~\cite{KorolevBeningShorgin2011}.

\smallskip

Далее понадобится следующее утверждение, позволяющее вычислить ${\sf E}U_{\alpha/2,1}^{-1/2}$, 
несмотря на то что плот\-ность
$g_{\alpha/2,1}(z)$, вообще говоря, нельзя выписать в явном виде в
терминах элементарных функций.

\medskip

\noindent
\textbf{Лемма 3.}
\begin{equation*}
{\sf E}U_{\alpha/2,1}^{-1/2}=\fr{\sqrt{2}\Gamma\left({1}/{\alpha}\right)}{\alpha\sqrt{\pi}}\,.
%\label{e8-kor}
\end{equation*}

\smallskip

\noindent
Д\,о\,к\,а\,з\,а\,т\,е\,л\,ь\,с\,т\,в\,о\,.\ Из~(\ref{e1-kor}) вытекает, что характеристическая
функция симметричного ($\theta\hm=0$) строго устойчивого распределения
имеет вид:
\begin{equation}
\mathfrak{f}_{\alpha,0}(t)=e^{-|t|^{\alpha}}\,,\enskip t\in\r\,. \label{e9-kor}
\end{equation}
С другой стороны, записав соотношение~(\ref{e6-kor}) в терминах
характеристических функций с учетом~(\ref{e9-kor}), получим
\begin{equation}
e^{-|t|^{\alpha}}=\int\limits_0^{\infty}\exp\left\{-\fr{t^2z}{2}\right\}
g_{\alpha/2,1}(z)\,dz\,.\label{e10-kor}
\end{equation}
Обозначим
$$
h_{\alpha/2}(z)=\fr{\alpha}{\Gamma({1}/{\alpha})}\sqrt{\fr{\pi}{2}}\,
\fr{g_{\alpha/2,1}(z)}{\sqrt{z}}\,,\enskip
z\ge0\,.
$$
Обобщенным распределением Лапласа принято называть абсолютно
непрерывное распределение вероятностей, задаваемое плотностью
$$
\ell_{\alpha}(x)=\fr{\alpha}{2\Gamma({1}/{\alpha})} \,
e^{-|x|^{\alpha}}\,,\enskip -\infty< x<\infty\,.
$$
Переобозначив аргумент $t\mapsto x$ и выполнив несколько формальных
тождественных преобразований равенства~(\ref{e10-kor}), будем иметь:
\begin{multline}
\ell_{\alpha}(x)=\fr{\alpha}{2\Gamma({1}/{\alpha})}e^{-|x|^{\alpha}}={}\\
{}=
\fr{\alpha}{\Gamma({1}/{\alpha})}\sqrt{\fr{\pi}{2}}\,\int\limits_0^{\infty}
\fr{\sqrt{z}}{\sqrt{2\pi}}\,\exp\left\{-\fr{x^2z}{2}\right\}
\fr{g_{\alpha/2,1}(z)}{\sqrt{z}}\,dz={}\\
{}=
\int\limits_0^{\infty}\sqrt{z}\phi(x\sqrt{z})h_{\alpha/2}(z)\,dz\,.\label{e11-kor}
\end{multline}
Можно убедиться, что $h_{\alpha/2}(z)$~--- плотность распределения
неотрицательной случайной величины. Действительно, при каждом $z\hm>0$
$$
\int\limits_{-\infty}^{\infty}\sqrt{z}\phi(x\sqrt{z})\,dx=1\,.
$$
Поэтому из~(\ref{e11-kor}) вытекает, что
\begin{multline*}
1=\int\limits_{-\infty}^{\infty}\!\ell_{\alpha}(x)\,dx=
\int\limits_{-\infty}^{\infty}\!\int\limits_{0}^{\infty}\!\sqrt{z}\phi(x\sqrt{z})
h_{\alpha/2}(z)\,dz dx={}
\\
{}=\int\limits_{0}^{\infty}\!h_{\alpha/2}(z)\left(\,
\int\limits_{-\infty}^{\infty}\sqrt{z}\phi(x\sqrt{z})\,dx\right)\,dz={}\\
{}=
\int\limits_{0}^{\infty}\!h_{\alpha/2}(z)\,dz.
\end{multline*}
Лемма доказана.

\smallskip

Продолжив~(\ref{e7-kor}) с учетом лемм~2 и~3, получим
\begin{multline*}
D_{n,\alpha}\le0{,}3041\,\fr{\beta^3}{\sigma^3\sqrt{n}}\,
{\sf E}U_{\alpha/2,1}^{-1/2}={}\\
{}=0{,}3041
\fr{\sqrt{2}\Gamma\left({1}/{\alpha}\right)}{\alpha\sqrt{\pi}}\,
\fr{\beta^3}{\sigma^3\sqrt{n}}\,.
\end{multline*}
Теорема доказана.

\vspace*{-6pt}

{\small\frenchspacing
{%\baselineskip=10.8pt
\addcontentsline{toc}{section}{Литература}
\begin{thebibliography}{99}

\bibitem{Korolev1997} 
\Au{Королев В.\,Ю.} О~сходимости pаспpеделений случайных сумм независимых
случайных величин к устойчивым законам~// Теоpия веpоятностей и ее
пpименения, 1997. Т.~42. Вып.~4. С.~818--820.

\bibitem{GnedenkoKorolev1996} 
\Au{Gnedenko B.\,V., Korolev~V.\,Yu.} Random summation:
Limit theorems and applications.~--- Boca Raton: CRC Press, 1996.

\bibitem{BeningKorolev2002} 
\Au{Bening V., Korolev V.} Generalized Poisson models and their applications in
insurance and finance.~--- Utrecht: VSP, 2002. 434~p.

\bibitem{KorolevSokolov2008} 
\Au{Королев В.\,Ю., Соколов И.\,А.} Математические модели
неоднородных потоков экстремальных событий.~--- М.: ТОРУС ПРЕСС,
2008.

\bibitem{KorolevBeningShorgin2011} 
\Au{Королев В.\,Ю., Бенинг В.\,Е., Шоргин~С.\,Я.}
Математические основы теории риска.~--- 2-е изд., перераб. и доп.~--- М.:
Физматлит, 2011. 620~с.

\bibitem{Korolev2011} 
\Au{Королев В.\,Ю.} Вероят\-но\-ст\-но-ста\-ти\-сти\-че\-ские методы
декомпозиции волатильности хаотических процессов.~--- М.: Изд-во
Московского ун-та, 2011. 510~с.

\bibitem{KorolevSkvortsova2006} 
Stochastic models of structural plasma turbulence~/
Eds. V.~Korolev, N.~Skvortsova.~--- Utrecht: VSP, 2006. 400~p.

\bibitem{McCulloch1996} 
\Au{McCulloch J.\,H.} Financial applications of stable
distributions~// Handbook of statistics.~--- Amsterdam:
Elsevier Science, 1996.  Vol.~14. P.~393--425.

\bibitem{Zolotarev1983} 
\Au{Золотарев В.\,М.} Одномерные устойчивые
распределения.~--- М.: Наука, 1983.

\label{end\stat}

\bibitem{KorolevShevtsova2010} 
\Au{Korolev V., Shevtsova~I.} An improvement of the Berry--Esseen
inequality with applications to Poisson and mixed Poisson random
sums~// Scandinavian Actuarial~J., 2012. Vol.~2012. No.\,2.
P.~81--105. Available online since June~4, 2010.

\end{thebibliography} } }

\end{multicols}   %4
\def\stat{gorshenin}

\def\tit{ПОВЫШЕНИЕ ДОХОДНОСТИ ТОРГОВЛИ НА FOREX С~ПОМОЩЬЮ  
LSTM-ИДЕНТИФИКАЦИИ СВЕЧНЫХ ПАТТЕРНОВ И~ИНДИКАТОРА ТИКОВЫХ 
ОБЪЕМОВ$^*$}

\def\titkol{Повышение доходности торговли на FOREX с~помощью  
LSTM-идентификации свечных паттернов} % и~индикатора тиковых  объемов}

\def\aut{А.\,К.~Горшенин$^1$, Е.\,И.~Гусейнова$^2$}

\def\autkol{А.\,К.~Горшенин, Е.\,И.~Гусейнова}

\titel{\tit}{\aut}{\autkol}{\titkol}

\index{Горшенин А.\,К.}
\index{Гусейнова Е.\,И.}
\index{Gorshenin A.\,K.}
\index{Guseynova E.\,I.}


{\renewcommand{\thefootnote}{\fnsymbol{footnote}} \footnotetext[1]
{Исследование выполнено при поддержке Российского научного фонда, проект 22-11-00212. Для обучения нейронных сетей использовалась инфраструктура Центра коллективного пользования 
<<Высокопроизводительные вычисления и~большие данные>> (ЦКП <<Информатика>>) ФИЦ ИУ РАН 
(г.~Москва).}


\renewcommand{\thefootnote}{\arabic{footnote}}
\footnotetext[1]{Федеральный исследовательский центр <<Информатика и~управление>> Российской академии наук, 
\mbox{agorshenin@frccsc.ru}}
\footnotetext[2]{ Московский государственный университет имени М.\,В.~Ломоносова, 
\mbox{ei.guseynova@yandex.ru}}

\vspace*{-12pt}



\Abst{Статья посвящена исследованию эффективности использования рекуррентных нейронных 
сетей LSTM (Long--Short Term Memory~--- долгая--краткосрочная память) для свечных данных 
и~индикатора технического анализа для большого числа наиболее распространенных валютных пар 
(всего~--- 27) за длительный период времени с~целью построения автоматических торговых 
стратегий. Продемонстрировано, что средняя итоговая и~годовая доходности за 8~лет при 
проведении модельных торгов составили 286\% и~15,4\% соответственно, что более чем в~50~раз 
превышает значения для классической торговой стратегии Buy \& Hold за тот же временной период. 
Кроме того, в~работе предложен новый индикатор технического анализа на основе тиковых объемов, 
который используется как самостоятельно в~качестве альтернативной торговой стратегии (итоговая 
и~годовая доходности LSTM-мо\-де\-лей превосходят ее в~среднем в~7,2 и~2,3~раза), так и~в~качестве 
дополнительного признака для повышения доходности нейросетевой стратегии за счет 
использования ансамблирования. Установлено, что для 37\% анализируемых валютных пар 
использование именно ансамбля нейронных сетей позволяет дополнительно повысить итоговую 
доходность в~среднем на~17,2\%.}

\KW{LSTM; ансамблевое обучение; свечные данные; технический индикатор; FOREX; валютные 
пары}

\DOI{10.14357/19922264220304} 
  
\vspace*{2pt}


\vskip 10pt plus 9pt minus 6pt

\thispagestyle{headings}

\begin{multicols}{2}

\label{st\stat}

\section{Введение}

\vspace*{-4pt}

  С развитием информационных технологий и~их внедрением в~работу бирж различных 
финансовых рынков у трейдеров появилась возможность использовать программы для 
осуществления торговых операций, построенные на основе их собственных методов 
и~правил торговли. 

Ведение торговли с~помощью про\-грамм-ро\-бо\-тов получило название 
автоматической торговой системы. Такие сис\-те\-мы способствовали существенному 
повышению эффективности работы трейдеров. В~отличие от механической торговой 
системы, автоматические торговые системы способны совершать торговые операции по 
купле и~продаже финансовых активов без непосредственного участия человека за счет 
встроенного алгоритма, который отвечает за автоматическую выработку сигнала на  
от\-кры\-тие/за\-кры\-тие позиции, а~также доставку ордера (заявки) на торговую платформу 
брокера. В~основе автоматических торговых сис\-тем чаще всего лежит стратегия на основе 
комбинации различных индикаторов и~паттернов технического анализа.
  
  В последние годы все большее влияние на инструменты подобной автоматизированной 
тор\-гов\-ли, как и~повсеместно в~мировом финансовом секторе~[1], стали оказывать алгоритмы 
на основе \mbox{машинного} обучения и~нейронных сетей. В~част\-ности, одним из направлений 
применения таких подходов стало построение торговых стратегий\linebreak с~более высокой 
доходностью по сравнению с~классическими инструментами. Зачастую для этого 
используется прогнозирование цен различных финансовых инструментов с~помощью 
алгоритмов машинного обучения, причем как на основе исходных ценовых данных, так 
и~расширением признакового пространства различными характеристиками, включая 
различные индикаторы технического анализа (см., например, статью~[2]). 

В~литературе 
наиболее распространены исследования для фондового рынка, однако также 
расcматриваются криптовалюты~[3] и~вы\-со\-ко\-час\-тот\-ная торговля отдельными валютными 
парами, прежде всего евро--дол\-лар~[4].
  
  Целью данной статьи ставится исследование эффективности применения рекуррентных 
нейросетевых архитектур LSTM сразу для~27~разнообразных валютных пар. В~качестве базового уровня до\-ход\-ности 
используются показатели простой тестовой\linebreak стратегии Buy \& Hold~[5], также на\-зы\-ва\-емой 
позиционной торговлей, при которой инвестор приобретает некоторые активы для 
долгосрочного хранения в~ожидании повышения цен на них в~\mbox{пределах} не менее нескольких 
лет. Кроме того, как было упомянуто выше, весьма популярны стратегии, использующие 
индикаторы технического анализа,~--- и~в~статье предложен новый инструмент, 
ориентированный на тиковые объемы. Наконец, представляет интерес построение стратегий 
с~одновременным использованием индикаторов и~нейронных сетей, которое в~данной работе 
реализуется с~помощью ансамблевого подхода к~обучению LSTM-ар\-хи\-тек\-тур.
  
  Статья организована следующим образом. В~разд.~2 приведен обзор известных решений 
на основе машинного обучения для различных финансовых инструментов. Раздел~3 
содержит описание анализируемых данных и~используемый подход для обработки свечных 
данных. В~разд.~4 представлена динамика доходности для каждой из пар, получаемая 
в~рамках использования LSTM-стратегии. Раздел~5 посвящен описанию нового технического 
индикатора и~сравнению его доходности с~нейросетевым подходом. В~разд.~6 
продемонстрирована возможность дополнительного повышения доходности 
для~10~валютных пар в~случае использования ансамблевого обучения для свечных данных 
и~этого индикатора. В~заключительном разделе подводятся краткие итоги и~обсуждаются 
возможные на\-прав\-ле\-ния дальнейших исследований.

\vspace*{-6pt}
  
\section{Машинное обучение в~задачах финансового прогнозирования}

\vspace*{-2pt}

  Одним из популярных подходов к~прогнозированию цен на фондовом рынке 
  с~использованием нейронных сетей стал графический анализ. Так, в~статье~[6] применяется 
сверточная архитектура для обработки исходных финансовых временн$\acute{\mbox{ы}}$х рядов~--- 
15~различных технических индикаторов, преобразованных в~двумерные изображения, 
каж\-дое из которых затем помечается как Buy, Sell или Hold в~зависимости от 
предполагаемых точек входа в~рынок и~выхода из него. Тестирование на данных 
промышленного индекса Доу Джонса и~биржевых инвестиционных фондов показало 
высокую эффективность данного подхода, в~частности по сравнению со стратегией Buy \& 
Hold.
  
  В статье~[7] предложена торговая система, не ограничивающаяся известными 
техническими паттернами. Она позволяет сравнивать текущее рыночное состояние с~более 
ранними похожими паттернами с~целью получения торговых сигналов. Для ее тестирования 
использовались 560~акций Нью-Йорк\-ской фондовой биржи, при этом авторы не проводили 
автоматическую и~динамическую оптимизацию параметров торговой стратегии~--- и~тем не 
менее предложенная система приводила к~92,5\% прибыльных сделок.
  
  Весьма популярны методы на основе анализа свечных данных. В~статье~[8] предложена 
ан\-самб\-ле\-вая нейронная сеть, включающая и~сверточную архитектуру CNN (Convolutional 
Neural Network), которая позволяет автоматически идентифицировать восемь типов свечных 
паттернов со средней точностью 90,7\% в~реальных данных, превосходя LSTM-сеть. Еще 
один пример анализа с~использованием свечных данных для Тайваньской фондовой биржи 
и~индекса фондового рынка Nikkei 225 Токийской фондовой биржи приведен в~статье~[9].
  
  Другая ансамблевая архитектура, сочетающая CNN и~LSTM-се\-ти, предложена 
  в~статье~[10]. Она использует последовательность исторических данных и~опережающие 
индикаторы (опционы и~фьючерсы) для извлечения дополнительных признаков с~по\-мощью 
CNN, а затем использует их в~качестве входных данных для LSTM. Для тестирования авторы 
использовали характеристики десяти американских и~тайваньских акций. 
  
  Эффективным оказалось сочетание алгоритмов машинного обучения и~нейронных сетей 
даже достаточно простых архитектур. Так, в~статье~[5] при построении модели 
прогнозирования учитывались не только ценовые данные по торговым инструментам, но 
и~данные объема торговли, а для получения оптимального набора параметров торговой 
стратегии использована комбинация регрессионного метода опорных векторов 
и~многослойного персептрона. Полученный алгоритм продемонстрировал высокие результаты 
на шести инструментах фондового рынка в~период с~2001 по 2015~гг.

   
  
  В статье~[11] поиск оптимальных параметров технического индикатора осуществляется 
  с~по\-мощью генетических алгоритмов. Затем его значения передаются в~глубокий 
многослойный перцептрон для получения одного из трех торговых сигналов (покупка, продажа, 
удержание текущей позиции). Результаты тестирования на исторических данных цен акций 
промышленного индекса Доу Джонса в~период с~2007 по 2016~гг.\ показывают, что 
оптимизация параметров технического индикатора повышает эффективность торговли 
акциями и~дает сопоставимые или лучшие результаты по сравнению с~Buy \& Hold и~другими торговыми стратегиями. 
  
  В работе~[12] показано, что прогнозирование точки разворота цены акций может быть 
весьма эффективным при использовании только LSTM-се\-тей, для которых на основе 
комбинаций свечных индикаторов и~технических индикаторов строятся наборы 
дополнительных признаков. Для 10 китайских и~10 американских акций результаты 
превзошли и~метод опорных векторов, и~многослойный перцептрон,  
и~CNN-ар\-хи\-тек\-туру.
  
  В статье~[13] предложен метод предварительной обработки исходных ценовых свечей 
  и~значений технических индикаторов для генерации торговых сигналов, используемых для 
обучения нейронной сети LSTM. Такая замена исходных ценовых данных позволила 
повысить точность предсказаний для пяти типов торговых стратегий, однако, как 
и~в~работе~[7], оптимизация их параметров не проводилась.
  
  В статье~[2] на примере фондового рынка Китая с~2000 по 2017~гг.\ получена точность 
прогноза более 60\% для некоторых моделей с~использованием анализа свечных графиков 
и~получения признаков с~помощью ансамблевых методов машинного обуче\-ния. Отмечено, что 
дополнительные технические индикаторы могут в~разной степени повысить точ\-ность 
прогноза. 
  
  Работа с~валютной парой ев\-ро--дол\-лар рас\-смот\-ре\-на в~статье~[4], в~которой 
предлагаются инструменты прогнозирования краткосрочного тренда на валютном рынке 
FOREX с~использованием глубокого обучения и~алгоритмов обучения с~подкреплением для 
высокочастотной торговли.
  
  Таким образом, весьма актуально исследование сразу большого числа валютных пар за 
длительный период на основе свечных данных с~использованием нейронных сетей 
и~инструментов технического анализа, которому и~посвящена данная статья.


   
   \vspace*{-12pt}

\section{Анализируемые данные и~создание пространства признаков}

\vspace*{-2pt}

Тиковые данные, содержащие метки времени и~цены сделки, по наиболее торгуемым 27 валютным парам 
загружены с~помощью программного интерфейса на языке Python с~торговой платформы 
<<Metatrader>>\footnote{{\sf https://www.metaquotes.net}.}. 

Начало доступного периода для каждой валютной пары приведено в~таблице. В~качестве 
конца периода для всех пар использована дата 15~апреля 2022~г., 23:54:00. Таким 
образом, для большинства валютных пар доступный интервал составляет около 10~лет. %\linebreak\vspace*{-12pt}



  
%\begin{table*}\small
   \begin{center}
   \vspace*{-9pt}
   
{\small \begin{tabular}{|c|c|c|c|}
   \multicolumn{2}{p{63mm}}{Анализируемые валютные пары и~даты начала периода для данных}\\
   \multicolumn{2}{c}{\ }\\[-6pt]
   \hline
Валютная пара&Начало периода\\
\hline
AUDCAD&2011-12-19 00:00:00\\ 
AUDCHF&2011-12-19 00:00:00\\ 
AUDJPY&2011-12-19 21:00:00\\ 
AUDNZD&2011-12-20 21:00:00\\ 
AUDUSD&2011-12-19 21:00:00\\ 
CADCHF&2011-12-21 21:00:00\\ 
CADJPY&2013-03-19 00:00:00\\ 
CHFJPY&2011-12-26 21:00:00\\ 
GBPAUD&2013-03-19 00:00:00\\ 
GBPCHF&2012-01-02 00:00:00\\ 
GBPJPY&2012-01-09 00:00:00\\ 
GBPNZD&2014-07-30 13:47:00\\ 
GBPUSD&2011-12-19 21:00:00\\ 
EURAUD&2012-01-13 21:00:00\\
EURCAD&2011-12-28 21:00:00\\  
EURCHF&2011-12-29 21:00:00\\  
EURGBP&2011-12-28 21:00:00\\  
EURJPY&2011-12-29 21:00:00\\  
EURNZD&2012-01-12 21:00:00\\  
EURUSD&2011-12-19 21:00:00\\  
NZDCAD&2012-01-17 20:58:00\\  
NZDCHF&2012-01-17 20:58:00\\  
NZDJPY&2012-01-17 20:58:00\\  
NZDUSD&2011-12-26 21:00:00\\  
USDCAD&2011-12-28 21:00:00\\  
USDCHF&2011-12-26 21:00:00\\  
USDCNH&2015-11-17 07:38:00\\
\hline
  \end{tabular}
  }
\end{center}

\vspace*{3pt}

%\end{table*}
   
   
  %\noindent
   

В~связи со значительным объемом занимаемой 
памяти параллельно с~выгрузкой данные конвертировались в~датафрейм, в~котором одно наблюдение 
соответствовало свече на минутном таймфрейме (торговом периоде~--- интервале времени, используемом для 
группировки котировок). Данные содержат информацию о цене открытия и~закрытия свечи, ее максимуме 
и~минимуме, а~также чис\-ло тиковых колебаний цен Ask (покупка) и~Bid (продажа) (рис.~1).
  
  Метод обработки свечей состоит в~том, чтобы из количественных данных цен открытия, 
закрытия, максимума и~минимума образовать одномерный массив категориальных объектов. 
Такое упрощение структуры данных сокращает объем выборки, ускоряя обучение, 
и~позволяет нейросети находить свечные паттерны, выявляя последовательности в~них, 
используемые в~дальнейшем для формирования сигналов в~рамках торговой стратегии.

 В зависимости от длины тела свечи и~ее хвостов были выделены по десять подклассов для 
интервалов~1 и~2 (рис.~2,\,\textit{а}). Сочетание двух подклассов\linebreak\vspace*{-12pt}

\pagebreak 

\end{multicols}

\begin{figure*} %fig1
   \vspace*{1pt}
  \begin{center}  
    \mbox{%
\epsfxsize=136.465mm
\epsfbox{gor-1.eps}
}

\end{center}
\vspace*{-9pt}
   \Caption{Цены открытия, закрытия, минимума, максимума, тиковые объемы Ask и~Bid}
%   \end{figure*}
 %  \begin{figure*} %fig2
\vspace*{12pt}
  \begin{center}  
    \mbox{%
\epsfxsize=161.084mm
\epsfbox{gor-2.eps}
}

\end{center}
\vspace*{-12pt}
\Caption{Схема преобразования числовых данных в~категориальные: (\textit{а})~объекты для 
определения класса свечи; (\textit{б})~преобразование в~шаблоны из классов свечей и~зависимой 
переменной направления рынка}
\end{figure*}

\begin{multicols}{2}
  
   
 
        %
\begin{figure*} %fig3
  \vspace*{1pt}
  \begin{center}  
    \mbox{%
\epsfxsize=163mm
\epsfbox{gor-3.eps}
}

\end{center}
\vspace*{-12pt}
  \Caption{Схема разбиения данных на обучающую, тестовую и~валидационную выборки}
   \end{figure*}
%
  
  \noindent
   определяло класс 
свечи. Все свечи на пятиминутном таймфрейме были переведены в~категориальные объекты 
  (рис.~2,\,\textit{б}). Задача прогнозирования
   сводится к~определению направления следующего 
категориального объекта в~последовательности, или, другими словами, к~определению 
семантики, т.\,е.\ к~бинарной классификации направления следующей свечи. Для этого 
длинная цепочка категориальных данных была разбита на блоки, состоящие из~50~объектов. 
Каждому блоку соответствовало значение зависимой переменной, характеризующей 
семантику~--- направление рынка на следующей свече. Стоит отметить, что направление 
рынка определялось на более старшем таймфрейме, а~именно на часовом, в~то время как 
объекты блока~--- на пятиминутном. 
  
  Поскольку с~ростом обучающей выборки сеть <<знакомится>> 
  с~б$\acute{\mbox{о}}$льшим числом паттернов, то предпочтительнее обучать сеть на как 
можно большем временн$\acute{\mbox{о}}$м периоде. Одновременно с~этим, чтобы можно 
было судить о состоятельности предложенного метода обучения, тестирование также 
должно проводиться на достаточно продолжительном временн$\acute{\mbox{о}}$м отрезке. 
Учитывая эти два фактора, было решено предварительно обучить сеть на данных, 
соответствующих двум годам, а~затем до\-обучать модель на новых данных с~периодичностью 
в~два с~половиной месяца. Схема разбиения данных на обучающие, валидационные 
и~тестовые выборки изображена на рис.~3. 
  
 
   
  Для каждой валютной пары было сделано около 30~разбиений: при первом 
разбиении из 15~тыс.\  наблюдений была создана валидационная и~тестовая 
выборка, а из следующих 1500~наблюдений~--- тестовая; при втором~--- тестовая 
выборка предыдущего разбиения добавлялась к~обучающей и~валидационной, а~следующие 
1500~наблюдений составляли новую тестовую выборку и~т.\,д.

\vspace*{-6pt}
  
\section{Торговая стратегия для~свечных~данных с~использованием~LSTM-сети}

\vspace*{-3pt}

  Суть предлагаемой стратегии заключается в~следовании сигналам нейронной сети 
(фактически, прогнозам) о по\-куп\-ке/про\-да\-же на основе свечных паттернов. Удержание 
позиции происходит до генерации следующего сигнала через один модельный торговый час. 
В~ходе исследования был проведен ряд экспериментов для определения оптимальной 
конфигурации сети, в~результате которых была выбрана структура, схематично 
представленная на рис.~4 (слева внутри графика).
  
   \begin{figure*} %fig4
   \vspace*{1pt}
  \begin{center}  
    \mbox{%
\epsfxsize=162.4mm
\epsfbox{gor-4.eps}
}

\end{center}
\vspace*{-6pt}
   \Caption{Динамика баланса в~ходе тестирования LSTM-стратегии на тестовом периоде для 27 
валютных пар и~архитектура используемой нейронной сети (на графике слева внутри)}
   \end{figure*}
   
  Слой embedding~--- векторное представление категориальных объектов в~одном 
наблюдении. В~его\linebreak основе лежит векторизация каждой категории, всего~100, так как 
класс свечи определяется де\-сятью\linebreak вариантами интервала~1 и~де\-сятью вариантами 
интервала~2. Веса инициализируются случайным образом, а~затем они корректируются 
с~по\-мощью алгоритма обратного распространения ошибки.

 Слой dropout с~коэффициентом~0,5 
традиционно используется для предотвращения пе\-ре\-обуче\-ния сети. LSTM-слой 
в~архитектуре позволяет сохранять и~передавать информацию от одного шага сети 
к~другому, учитывая последовательность, в~какой данные подаются, что является ключевым 
в~распознавании паттернов. 
Слой dense~--- стандартный полносвязный. 

В~качестве 
активационной функции выходного слоя используется softmax~--- обобщение логистической 
функции для многомерного случая. В~качестве оптимизатора используется Adam, в~качестве 
функции потерь~--- бинарная кросс-энт\-ро\-пия, точ\-ность оценивается в~смысле качества 
распознавания объектов. 

При обуче\-нии экспериментальным путем были выбраны 
сле\-ду\-ющие значения основных па\-ра\-мет\-ров обуче\-ния: общее чис\-ло тренировочных объектов, 
пред\-став\-лен\-ных в~одном пакете,~--- 50; общее чис\-ло эпох обуче\-ния~--- 300. При этом, если 
значение функции потерь не уменьшалось в~течение~7~эпох, лучшая модель сохранялась 
и~обуче\-ние прекращалось. Для программной реализации был выбран язык Python, в~качестве 
основной платформы для обучения нейронных сетей~--- TensorFlow с~программным 
интерфейсом Keras. Эффективность стратегий оценивалась до\-ход\-ностью со\-от\-вет\-ст\-ву\-ющих 
финансовых инструментов.
  
  Результаты тестирования для 27 валютных пар представляют собой динамику изменения 
баланса в~процентах (см.\ рис.~4), а~также итоговую и~годовую доходность в~процентах, 
которая будет обсуждаться ниже. Период тестирования для 23 валютных пар составляет 8 
лет, для четырех он сокращен до 6--7~лет в~силу доступности меньшего объема данных. 
Визуализация динамики баланса на рис.~4 демонстрирует отсутствие существенных 
просадок ниже уровня 100\% на протяжении всего периода, что свидетельствует о 
стабильности предложенной стратегии. Наибольшая доходность у валютной пары NZDCAD 
$- 1570\%$ (41,2\% годовых), а~наименьшая~--- у~AUDJPY: $-0{,}57\%$ ($-0{,}08\%$ годовых). 
Средняя годовая доходность по всем валютным парам со\-став\-ля\-ет 15,4\%. Сравнение с~другими стратегиями будет приведено в~разд.~5 и~6. 

\begin{figure*}[b] %fig5
\vspace*{6pt}
  \begin{center}  
    \mbox{%
\epsfxsize=140.628mm
\epsfbox{gor-5.eps}
}

\end{center}
\vspace*{-6pt}
\Caption{Визуализация сигналов стратегии на основе индикатора тиковых объемов (данные 122 
сделок)}
\end{figure*}

\vspace*{-6pt}
  
\section{Индикатор тиковых объемов}

\vspace*{-2pt}

  Рассмотренная в~предыдущем разделе стратегия, основанная на нейронных сетях, не 
относится к~традиционному техническому анализу, хотя в~ее основе лежит распознавание 
последовательностей свечей. В~данном разделе сравним результаты LSTM-стра\-те\-гии 
и~подхода на основе технического индикатора. Можно отметить, что само по себе значение 
индикатора не является сигналом к~покупке или продаже. Пороговые значения, после 
которых цену можно считать подходящей для совершения сделки, определяются 
трейдерами, а значит, сигналы одного и~того же индикатора можно интерпретировать по-раз\-но\-му. 
При построении торговой стратегии помимо графических паттернов цены важ\-ную 
роль играют уровни поддержки и~сопротивления. Многие трейдеры строят свои сис\-те\-мы 
исключительно на принципах использования ценовых уровней, зон поддержки 
и~сопротивления (в~них сосредоточены заявки крупных участников рынка на покупку 
и~продажу), не уделяя внимания паттернам в~чистом виде~[14].
  
  Предлагаемый в~данной статье индикатор связан с~поиском и~обнаружением ликвидности. 
Он построен на анализе поступающих на биржу заявок с~целью выявления крупных ордеров 
(заявок) или агентов финансового рынка, которые выставляют большие объемы активов на 
покупку или продажу. Всеобщие данные объемов недоступны, но по характеру прошлых 
изменений цен Ask и~Bid можно установить, какие ценовые уровни заставляли цену менять 
свое направление и~хотя бы на время останавливали тренд. Индикатор создан для выявления 
повышенных уровней объема у свечей. Стоит обратить внимание, что если объем Ask 
и~объем Bid одновременно велики, то это значит, что силы покупателей и~продавцов не 
превалируют друг над другом, а значит, смена тренда маловероятна. По этой причине 
в~индикаторе используются не абсолютные значения ценовых объемов, а их разница~--- 
именно повышение разницы объемов Bid и~Ask в~данном индикаторе служит сигналом о 
смене тренда.  Формула индикатора тиковых объемов Bid и~Ask имеет следующий вид: 
  \begin{multline*}
  \mathrm{fracInd}_t= \fr{\mathrm{CountAsk}_t - \mathrm{CountBid}_t}{\mathrm{CountAsk}_t+\mathrm{CountBid}_t} \times{}\\
  \times \fr{(1- (\vert \mathrm{Close}_t-
\mathrm{Open}_t\vert )/(\mathrm{High}_t -\mathrm{Low}_t))}{\vert \mathrm{Close}_t -\mathrm{Open}_t\vert}\,,
  \end{multline*}
где 
\begin{description}
\item[\,] $\mathrm{fracInd}_t$~--- значение индикатора в~момент времени~$t$; 
\item[\,] $\mathrm{CountAsk}_t$~--- объем тиков Ask в~момент времени~$t$; 
\item[\,] $\mathrm{CountBid}_t$~--- объем тиков Bid в~момент времени~$t$; 
\item[\,] $\mathrm{Close}_t$~--- цена закрытия свечи в~момент времени~$t$; 
\item[\,] $\mathrm{Open}_t$~--- цена открытия свечи в~момент времени~$t$; 
\item[\,] $\mathrm{High}_t$~--- максимальное значение цены в~момент времени~$t$; 
$\mathrm{Low}_t$~--- минимальное значение цены в~момент времени~$t$.
\end{description}
 Чем больше разница тиков 
Ask и~Bid, чем больше хвосты у свечи и~чем меньше тело свечи, тем выше значение 
индикатора $\mathrm{fracInd}_t$. Для генерации сигналов $\mathrm{fracInd}_t$ был нормализован в~соответствии с~формулой:
$$
\mathrm{stoch}_t= \fr{\mathrm{fracInd}_t- \min(\mathrm{fracInd}_n)}{\max (\mathrm{fracInd}_k) -\min(\mathrm{fracInd}_n)}\,\cdot 100\,,
$$
где 
\begin{description}
\item[\,] $\mathrm{stoch}_t$~--- значение нормализованного индикатора в~момент времени~$t$; 
\item[\,] $\min(\mathrm{fracInd}_n)$~--- минимальное значение $\mathrm{fracInd}_t$ за $n$ периодов; 
 \item[\,] $\max(\mathrm{fracInd}_k)$~--- максимальное значение $\mathrm{fracInd}_t$ за $k$ периодов.
 \end{description}
  В~зависимости от 
изменения параметров~$n$ и~$k$ интерпретация индикатора может меняться. Чем выше их 
значения, тем большее окно ценовых значений рассматривается для выявления зон с~\mbox{повышенными} объемами ликвидности. Таким образом, для старших таймфреймов 
предпочтительнее использовать более высокие значения, чем для низких.

  Для тестирования была создана простейшая торговая стратегия: при пиковых значениях 
индикатора $\mathrm{stoch}_t$ в~случае восходящего тренда позиция\linebreak менялась на короткую, 
а~в~случае нисходящего~--- на длинную. Индикатор использовался для определения 
ценовых разворотов и~смены тренда. \mbox{С~помощью} созданной функции бэктестирования 
описанная стратегия была проверена на всех валютных парах на протяжении пяти лет 
с~15~апреля 2017~г.\ по 15~апреля 2022~г. Тестирование на более длительных периодах без 
проведения дополнительной оптимизации неэффективно, так как торговля с~индикаторами 
предусматривает периодическое обновление параметров стратегии.
  
  На рис.~5 представлен фрагмент сигналов, вырабатываемых предложенной стратегией: 
черточки поверх свечей обозначают сделки: красные~--- убыточные, зеленые~--- 
прибыльные. Легко заметить, что рекомендуемые цены открытия/закрытия позиции 
совпадают с~пиковыми значениями локальных трендов. Это означает, что сигналы 
индикатора можно использовать как точки входа и~выхода из позиции. Разработанный 
индикатор хорошо предсказывает развороты рынка даже с~учетом неизменных параметров 
на протяжении всего тестируемого периода. В среднем доходность по всем валютным парам 
за 5~лет составила 36,36\%.
  

\begin{figure*} %fig6
\vspace*{1pt}
  \begin{center}  
    \mbox{%
\epsfxsize=151.209mm
\epsfbox{gor-6.eps}
}

\end{center}
\vspace*{-6pt}
\Caption{Результаты сравнения стратегий на основе LSTM~(\textit{1}~--- за~5~лет и~\textit{4}~--- годовая), индикатора тиковых объемов 
(\textit{2}~--- за~5~лет и~\textit{5}~--- годовая) и~Buy \& 
Hold за 5~лет~(\textit{3})}
\end{figure*}

\begin{figure*} %fig7
\vspace*{1pt}
  \begin{center}  
    \mbox{%
\epsfxsize=151.509mm
\epsfbox{gor-7.eps}
}

\end{center}
\vspace*{-6pt}
\Caption{Сравнение доходностей LSTM-стратегии (\textit{1}~--- доходность за 8~лет; \textit{2}~---  
годовая доходность), стратегии на основе ансамблевой LSTM-ар\-хи\-тек\-ту\-ры (\textit{3}~--- доходность за 8~лет; 
\textit{4}~--- годовая доходность) и~базовой стратегии Buy \& Hold~(\textit{5}) за 
восьмилетний период для 27 валютных пар }
\end{figure*}
  
  Чтобы результаты стратегии с~нейронной сетью были сопоставимы с~результатами 
тестирования индикатора, нейронные сети были протестированы на том же временн$\acute{\mbox{о}}$м 
участке: они продемонстрированы на рис.~6. Кроме того, добавлено сравнение 
с~классической стратегией Buy \& Hold.
  
  Средняя годовая доходность LSTM-стра\-те\-гии~--- 18,2\%. Это несколько выше средней 
годовой доходности за восьмилетний период, приведенной в~разд.~4. Причина этого может 
быть связана с~лучшим обучением нейронной сети на более поздних тестируемых периодах. 
За тот же период времени средняя годовая доходность стратегии на основе индикатора 
значительно ниже~--- 6,1\%. При этом стоит отметить, что по валютным парам \mbox{AUDJPY}, 
AUDUSD и~USDCNH годовая доходность индикатора превосходит годовую доходность 
LSTM-стра\-те\-гии, причем для \mbox{AUDJPY} обе <<технические>> стратегии показали 
положительную годовую доходность. 


 \vspace*{-9pt}


\section{Ансамблевая архитектура}

 \vspace*{-2pt}

  Для улучшения результатов, описанных в~предыду\-щих разделах, возможно построение 
стратегии, которая учитывает сигналы, генерируемые и~нейронной сетью, и~индикатором. 




  
  Для этого архитектура базовой нейронной сети (см.\ рис.~4) была изменена на ансамбль из 
двух LSTM-се\-тей: она имеет два входных потока, на вход одного из которых подаются 
прежние величины, соответствующие LSTM-стра\-те\-гии, а на вход второго~--- значения 
индикатора, приведенные к~аналогичному виду категориальных объектов путем отнесения 
значения к~одному из десяти интервалов. Обе части нейронной сети включают в~себя слои 
embedding, dropout и~LSTM~--- эти фрагменты в~точности соответствуют LSTM-стра\-те\-гии. 
Далее выходные потоки попадают в~слой concatenate, который объединяет их и~единым 
потоком передает в~выходной полносвязный слой с~функцией активации softmax. Выходной 
слой генерирует итоговые вероятности, на основе которых создается прогноз движения 
\mbox{цены.}
  
  Модифицированная нейронная сеть была протестирована на всех~27~ранее 
рассмотренных валютных парах (рис.~7) за период в~6--8~лет (по этой причине на 
графиках не приводятся значения индикатора). Для этого искусственные нейронные сети 
были предварительно обучены на данных каждой валютной пары, а затем с~использованием 
полученных сигналов покупки и~продажи было осуществлено бэктестирование 
модифицированной стратегии на исторических данных. Обучение проходило по тому же 
принципу, что и~для исходной LSTM-стра\-те\-гии. 
  
  Можно заметить, что на 10 валютных парах стратегия с~применением ансамбля 
 LSTM-се\-тей превзошла первую LSTM-стра\-те\-гию, продемонстрировав более высокую 
доходность. Для этих валютных пар среднее значение итоговой доходности воз\-рос\-ло на 
17,19\%, а годовая доходность увеличилась на 0,409\%. При этом средняя годовая 
доходность, рассчитанная для всех валютных пар, составила 12,2\%, что ниже аналогичного 
показателя простой LSTM-стра\-те\-гии. Однако можно обратить внимание на ранее 
убыточную пару AUDJPY, для которой получена итоговая доходность 31,2\% и~годовая 
3,3\%, что существенно превосходит значения для базовой стратегии Buy \& Hold. Таким 
образом, в~целом ряде случаев (37\% тестируемых валютных пар) использование 
ансамблевого подхода позволяет значимым образом повысить доходность базовой  
LSTM-стра\-те\-гии. Это обстоятельство необходимо учитывать при разработке реальных 
автоматизированных торговых инструментов, в~рамках которых обычно не используется 
единый алгоритм (метод) сразу для всех объектов из портфеля.

 \vspace*{-9pt}

\section{Заключение}

 \vspace*{-2pt}

  В работе продемонстрирована высокая эффективность использования нейросетевых 
подходов к~построению автоматических торговых стратегий на основе валютных пар. 
Показано, что предложенные методы позволяют многократного превзойти базовые методы, 
в~том числе основанные на классическом техническом анализе.
  %
  При этом можно отметить, что многие валютные пары обладают высокой корреляцией~--- 
как положительной, так и~отрицательной (рис.~8). 
  
  
  
  Можно предположить, что для большей информативности признаков, созданных на 
основе значений индикатора, в~дальнейших исследованиях стоит использовать уникальное 
разбиение на классы для разных валютных пар, а~не унифицированное, как это реализовано 
в~данной работе. Потенциально подбор порогов для разбиения на классы может повысить 
доходность стратегии с~применением ан\-самб\-ля LSTM-се\-тей, что сделает модификацию 
LSTM-стра\-те\-гии перспективной об\-ластью дальнейшего исследования, включая 
использование композиционных и~ан\-самб\-ле\-вых подходов. 
  
  Принципиальное улучшение может быть достигнуто за счет привлечения сложных 
математических методов и~моделей, в~частности использования\linebreak  параметров семейств 
вероятностных распределений для расширения признакового пространства для обучения 
нейронных сетей, как предложено в~\mbox{статье}~[15] на примере различных моделей добавления 
дополнительных признаков~--- четырех первых моментов конечных нормальных смесей. 
Определенные успехи в~этом направлении для ряда валютных пар уже 
продемонстрированы~[16], что свидетельствует в~пользу перспективности подобных 
исследований.

\end{multicols}

\begin{figure*} %fig8
   \vspace*{1pt}
  \begin{center}  
    \mbox{%
\epsfxsize=130.012mm
\epsfbox{gor-8.eps}
}

\end{center}
\vspace*{-6pt}
   \Caption{Корреляционная матрица валютных пар}
 %  \vspace*{6pt}
   \end{figure*}
   
   \vspace*{-9pt}
   
   \begin{multicols}{2}
   


  
{\small\frenchspacing
 {%\baselineskip=10.8pt
 %\addcontentsline{toc}{section}{References}
 \begin{thebibliography}{99}
\bibitem{1-gor}
\Au{Bholat D., Susskind~D.} The assessment: Artificial intelligence and financial services~// 
Oxford Rev. Econ. Pol., 2021. Vol.~37. Iss.~3. P.~417--434.
\bibitem{2-gor}
\Au{Lin Y., Liu S., Yang~H., Wu~H.} Stock trend prediction using candlestick charting and 
ensemble machine learning techniques with a~novelty feature engineering scheme~// IEEE 
Access, 2021. Vol.~9. P.~101433--101446.
\bibitem{3-gor}
\Au{Lahmiri S., Bekiros~S.} Intelligent forecasting with machine learning trading systems in chaotic 
intraday bitcoin market~// Chaos Soliton. Fract., 2020. Vol.~133. Art. No.\,109641. 7~p.
\bibitem{4-gor}
\Au{Rundo F.} Deep LSTM with reinforcement learning layer for financial trend prediction in FX 
high frequency trading systems~// Appl. Sci.~--- Basel, 2019. Vol.~9. Iss.~20. Art. No.\,4460. 18~p.
\bibitem{5-gor}
\Au{Dinh T.-A., Kwon~Y.-K.} An empirical study on importance of modeling parameters and 
trading volume-based features in daily stock trading using neural networks~// Informatics, 
2018. Vol.~5. Iss.~3. Art. No.\,36. 12~p.
\bibitem{6-gor}
\Au{Sezer O.\,B., Ozbayoglu A.\,M.} Algorithmic financial trading with deep convolutional 
neural networks: Time series to image conversion approach~// Appl. Soft Comput., 2018. 
Vol.~70. P.~525--538.
\bibitem{7-gor}
\Au{Tsinaslanidis P., Guijarro~F.} What makes trading strategies based on chart recognition 
profitable?~// Expert Syst., 2020. Vol.~38. Iss.~5. Art. No.\,e12596. 17~p.
\bibitem{8-gor}
\Au{Chen J.\,H., Tsai Y.\,C.} Encoding candlesticks as images for pattern classification using 
convolutional neural networks~// Financial Innovation, 2020. Vol.~6. Art. No.\,26. 19~p.
\bibitem{9-gor}
\Au{Hung C.-C., Chen Y.-J.} DPP: Deep predictor for price movement from candlestick charts~// 
PLoS ONE, 2021. Vol.~16. Iss.~6. Art. No.\,e0252404. 22~p.
\bibitem{10-gor}
\Au{Wu J.\,M.-T., Li Z., Herencsar~N., Vo~B., Lin~J.\,C.-W.} A~graph-based CNN-LSTM stock 
price prediction algorithm with leading indicators~// Multimedia Syst., 2021. 20~p. doi: 
10.1007/s00530-021-00758-w.
\bibitem{11-gor}
\Au{Sezer O.\,B., Ozbayoglu A.\,M., Dogdu E.} A~deep neural-network based stock trading 
system based on evolutionary optimized technical analysis parameters~// Procedia Comput. 
Sci., 2017. Vol.~114. P.~473--480.
\bibitem{12-gor}
\Au{JuHyok U., PengYu~L., ChungSong~K., UnSok~R., KyongSok~P.} A~new LSTM based 
reversal point prediction method using upward/downward reversal point feature sets~// Chaos 
Soliton. Fract., 2020. Vol.~32. Art. No.\,109559. 15~p.
\bibitem{13-gor}
\Au{Tsantekidis A., Tefas~A.} Transferring trading strategy knowledge to deep learning 
models~// Knowl. Inf. Syst., 2021. Vol.~63. P.~87--104.
\bibitem{14-gor}
\Au{Edwards R.\,D., Magee~J., Bassetti~W.\,H.\,C.} Technical analysis of stock trends.~--- 11th 
ed.~--- Boca Raton, FL, USA: CRC Press, 2018. 685~p.
\bibitem{15-gor}
\Au{Gorshenin A.\,K., Kuzmin V.\,Yu}. Statistical feature construction for forecasting accuracy 
increase and its applications in neural network based analysis~// Mathematics, 2022. Vol.~10. 
Iss.~4. Art. No.\,589. 21~p.
\bibitem{16-gor}
\Au{Виляев А.\,Л., Горшенин~А.\,К.} О~моделировании торговых стратегий для валютных пар 
с~использованием глубоких нейронных сетей и~метода скользящего разделения смесей~// 
Интеллектуальные сис\-те\-мы. Тео\-рия и~приложения, 2021. T.~25. Вып.~4. С.~92--\linebreak 95. 

\end{thebibliography}

 }
 }

\end{multicols}

\vspace*{-10pt}

\hfill{\small\textit{Поступила в~редакцию 13.07.22}}

\vspace*{8pt}

%\pagebreak

%\newpage

%\vspace*{-28pt}

\hrule

\vspace*{2pt}

\hrule

%\vspace*{-2pt}

\def\tit{INCREASING FOREX TRADING PROFITABILITY\\ WITH~LSTM CANDLESTICK PATTERN 
RECOGNITION\\ AND~TICK VOLUME INDICATOR}


\def\titkol{Increasing FOREX trading profitability with~LSTM candlestick pattern 
recognition and~tick volume indicator}


\def\aut{A.\,K.~Gorshenin$^1$ and E.\,I.~Guseynova$^2$}

\def\autkol{A.\,K.~Gorshenin and E.\,I.~Guseynova}

\titel{\tit}{\aut}{\autkol}{\titkol}

\vspace*{-17pt}


\noindent
$^1$Federal Research Center ``Computer Science and Control'' of the Russian Academy of Sciences, 
44-2~Vavilov\linebreak
$\hphantom{^1}$Str., Moscow 119133, Russian Federation

\noindent
$^2$M.\,V.~Lomonosov Moscow State University, 1~Leninskie Gory, GSP-1, Moscow 119991, Russian 
Federation


\def\leftfootline{\small{\textbf{\thepage}
\hfill INFORMATIKA I EE PRIMENENIYA~--- INFORMATICS AND
APPLICATIONS\ \ \ 2022\ \ \ volume~16\ \ \ issue\ 3}
}%
 \def\rightfootline{\small{INFORMATIKA I EE PRIMENENIYA~---
INFORMATICS AND APPLICATIONS\ \ \ 2022\ \ \ volume~16\ \ \ issue\ 3
\hfill \textbf{\thepage}}}

\vspace*{2pt} 


\Abste{The paper introduces the research of the effectiveness of using LSTM (Long--Short Term Memory)
for candlestick data 
and a technical analysis indicator for a large number of the most common currency pairs (27 in 
total) over a long period in order to build automatic trading strategies. The achieved average total 
and annual return for 8~years of a model trading were 286\% and 15.4\%, respectively. It is more 
than~50~times higher than the values for the classic Buy \& Hold trading strategy for the same 
period. In addition, the paper introduces a new technical indicator based on tick volumes which is 
an alternative trading strategy (the total and annual returns of LSTM models exceed it by an 
average of~7.2 and 2.3~times) as well as an additional feature to increase the profitability of the 
neural network strategy through the use of ensemble learning. For 37\% of the analyzed currency 
pairs, the use of an ensemble of LSTMs allows one to increase further the total return by an average 
of 17.2\%.}

\KWE{LSTM; ensemble learning; candlestick; technical indicator; FOREX; currency pairs}




\DOI{10.14357/19922264220304} 

\vspace*{-20pt}

\Ack

\vspace*{-5pt}

\noindent
The research was supported by the Russian Science Foundation (grant No.\,22-11-00212). The
 research was carried out using the infrastructure of the Shared Research Facilities ``High 
Performance Computing and Big Data'' (CKP ``Informatics'') of FRC CSC RAS (Moscow). 




\vspace*{6pt}

  \begin{multicols}{2}

\renewcommand{\bibname}{\protect\rmfamily References}
%\renewcommand{\bibname}{\large\protect\rm References}

{\small\frenchspacing
 {%\baselineskip=10.8pt
 \addcontentsline{toc}{section}{References}
 
 \begin{thebibliography}{99}
 
 \vspace*{-3pt}
 
\bibitem{1-gor-1}
\Aue{Bholat, D., and D.~Susskind.} 2021. The assessment: Artificial intelligence and financial 
services. \textit{Oxford Rev. Econ. Pol.} 37(3):417--434.
\bibitem{2-gor-1}
\Aue{Lin, Y., S.~Liu, H.~Yang, and H.~Wu.} 2021. Stock trend prediction using candlestick charting 
and ensemble machine learning techniques with a novelty feature engineering scheme. \textit{IEEE Access} 
9:101433--101446.
\bibitem{3-gor-1}
\Aue{Lahmiri, S., and S.~Bekiros.} 2020. Intelligent forecasting with machine learning trading 
systems in chaotic intraday Bitcoin market. \textit{Chaos Soliton. Fract.} 133:109641.\linebreak 7~p.
\bibitem{4-gor-1}
\Aue{Rundo, F.} 2019. Deep LSTM with reinforcement learning layer for financial trend 
prediction in FX high frequency trading systems. \textit{Appl. Sci.~--- Basel} 9(20):4460. 18~p.
\bibitem{5-gor-1}
\Aue{Dinh, T.-A., and Y.-K.~Kwon.} 2018. An empirical study on importance of modeling 
parameters and trading volume-based features in daily stock trading using neural networks. 
\textit{Informatics} 5(3):36. 12~p.
\bibitem{6-gor-1}
\Aue{Sezer, O.\,B., and A.\,M.~Ozbayoglu.} 2018. Algorithmic financial trading with deep 
convolutional neural networks: Time series to image conversion approach. \textit{Appl. Soft Comput.} 
70:525--538.
\bibitem{7-gor-1}
\Aue{Tsinaslanidis, P., and F.~Guijarro.} 2020. What makes trading strategies based on chart 
recognition profitable? \textit{Expert Syst.} 38(5):e12596. 17~p.
\bibitem{8-gor-1}
\Aue{Chen, J.\,H., and Y.\,C.~Tsai.} 2020. Encoding candlesticks as images for pattern classification 
using convolutional neural networks. \textit{Financial Innovation} 6:26. 19~p.
\bibitem{9-gor-1}
\Aue{Hung, C.-C., and Y.-J.~Chen.} 2021. DPP: Deep predictor for price movement from 
candlestick charts. \textit{PLoS ONE} 16(6):e0252404. 22~p.
\bibitem{10-gor-1}
\Aue{Wu, J.\,M.-T., Z.~Li, N.~Herencsar, B.~Vo, and J.\,C.-W.~Lin.} 2021. A~graph-based CNN-LSTM 
stock price prediction algorithm with leading indicators. \textit{Multimedia Syst}. 20~p.
doi: 10.1007/s00530-021-00758-w.
\bibitem{11-gor-1}
\Aue{Sezer, O.\,B., A.\,M.~Ozbayoglu, and E.~Dogdu.} 2017. A~deep neural-network based stock 
trading system based on evolutionary optimized technical analysis parameters. \textit{Procedia Comput. 
Sci.} 114:473--480.
\bibitem{12-gor-1}
\Aue{JuHyok, U., L.~PengYu, K.~ChungSong, R.~UnSok, and P.~KyongSok.} 2020. A~new LSTM 
based reversal point prediction method using upward/downward reversal point feature sets. \textit{Chaos 
Soliton. Fract.} 32:109559. 15~p.
\bibitem{13-gor-1}
\Aue{Tsantekidis, A., and A.~Tefas.} 2021. Transferring trading strategy knowledge to deep 
learning models. \textit{Knowl. Inf. Syst.} 63:87--104.
\bibitem{14-gor-1}
\Aue{Edwards, R.\,D., J.~Magee, and W.\,H.\,C.~Bassetti.} 2018. \textit{Technical analysis of stock trends}. 
11th ed. Boca Raton, FL: CRC Press. 685~p.
\bibitem{15-gor-1}
\Aue{Gorshenin, A.\,K., and V.\,Yu.~Kuzmin.} 2022. Statistical feature construction for forecasting 
accuracy increase and its applications in neural network based analysis. \textit{Mathematics} 10(4):589. 21~p.
\bibitem{16-gor-1}
\Aue{Vilyaev, A.\,L., and A.\,K.~Gorshenin.} 2021. O modelirovanii torgovykh strategiy dlya 
valyutnykh par s ispol'zovaniem glubokikh neyronnykh setey i~metoda skol'zyashchego razdeleniya 
smesey [On modeling trading strategies for currency pairs using deep neural networks and method 
of moving separation of mixtures]. \textit{Intellektual'nye sistemy. Teoriya i~prilozheniya} [Intelligent 
Systems. Theory and Applications] 25(4):92--95.
\end{thebibliography}

 }
 }

\end{multicols}

\vspace*{-6pt}

\hfill{\small\textit{Received July 13, 2022}}

\Contr

\noindent
\textbf{Gorshenin Andrey K.} (b.\ 1986) ~--- Doctor of Science in physics and mathematics, 
associate professor, head of department, leading scientist, Federal Research Center ``Computer 
Science and Control'' of the Russian Academy of Sciences, 44-2~Vavilov Str., Moscow 119333, 
Russian Federation; \mbox{agorshenin@frccsc.ru}

\vspace*{3pt}

\noindent
\textbf{Guseynova Ekaterina I.} (b.\ 1999)~--- Master of Science, Faculty of Economics, M.\,V.~Lomonosov 
Moscow State University, 1~Leninskie Gory, GSP-1, Moscow 119991, Russian Federation; 
\mbox{ei.guseynova@yandex.ru}

\label{end\stat}

\renewcommand{\bibname}{\protect\rm Литература}      %5
\def\stat{malashenko}

\def\tit{МЕТОДЫ ОЦЕНКИ ЭФФЕКТИВНОСТИ И~ДИРЕКТИВНЫХ СРОКОВ ВЫПОЛНЕНИЯ РЕСУРСОЕМКИХ
 ВЫЧИСЛИТЕЛЬНЫХ ЗАДАНИЙ}

\def\titkol{Методы оценки эффективности и директивных сроков выполнения ресурсоемких вычислительных заданий}

\def\autkol{И.\,К.~Купалов-Ярополк, Ю.\,Е.~Малашенко, И.\,А.~Назарова, А.\,Ф.~Ронжин}

\def\aut{И.\,К.~Купалов-Ярополк$^1$, Ю.\,Е.~Малашенко$^2$, И.\,А.~Назарова$^3$, А.\,Ф.~Ронжин$^4$}

\titel{\tit}{\aut}{\autkol}{\titkol}

%{\renewcommand{\thefootnote}{\fnsymbol{footnote}}\footnotetext[1]
%{Работа
%поддержана Российским фондом фундаментальных исследований (проекты
%11-01-00515а, 11-07-00112а, 11-01-12026-офи-м), Министерством
%образования и науки РФ (госконтракт 16.740.11.0133).}}

\renewcommand{\thefootnote}{\arabic{footnote}}
\footnotetext[1]{Институт точной механики и вычислительной техники им.\ С.\,А.~Лебедева Российской академии наук,  
kupyar@rambler.ru}
\footnotetext[2]{Вычислительный центр им.\ А.\,А.~Дородницына Российской академии наук, malash09@ccas.ru}
\footnotetext[3]{Вычислительный центр им.\ А.\,А.~Дородницына Российской академии наук, irina-nazar@yandex.ru}
\footnotetext[4]{Вычислительный центр им.\ А.\,А.~Дородницына Российской академии наук, raf-zao-zt@yandex.ru}

\vspace*{4pt}

\Abst{Рассматривается проблема эффективного использования гетерогенной вычислительной 
системы при параллельной обработке разнородных заданий. В~случае нарушения сроков завершения 
работ затраченное процессорное время относится к производственным потерям. Планирование и 
оптимизация управления осуществляются на основе  гарантированных оценок, построенных для 
наихудшего случая.}

\vspace*{2pt}

\KW{параллельные вычисления; многопроцессорные системы;  оптимизация;  принцип гарантированного результата}

\vskip 14pt plus 9pt minus 6pt

      \thispagestyle{headings}

      \begin{multicols}{2}

            \label{st\stat}

\section{Введение}

В настоящей работе предлагается модель для анализа эффективности
и оперативного планирования процедуры обработки ресурсоемких
вы\-чис\-ли\-тель\-ных заданий, которые подробно описаны и определены в~\cite{Prep11} 
как citu-за\-да\-ния: computationally intensive task under
uncertainty.  Решение каж\-дой из citu-за\-дач  состоит в просмотре
большого массива исходных данных и выделении из него одного
уникального фрагмента с наперед заданными свойствами. В~ходе  поиска
реализуется один и тот же переборный алгоритм для различных
начальных данных, разбитых  на неделимые, содержательно значимые
фрагменты. Если при выполнении задания такой фрагмент найден, то
говорят, что задача решена, и обработка задания прекращается.  Если
же  просмотрены  все  предъявленные данные и  найти уникальный
фрагмент не удалось, то задание считается выполненным, но задача не
имеет решения.

Процесс просмотра происходит в условиях неопределенности, связанной
как с длительностью поиска, так и с потенциальной возможностью
получить решение. Все citu-за\-да\-ния выполняются в режиме реального
времени и  допускают распараллеливание по данным~\cite{Sour}.

В современной практике при выполнении citu-ра\-бот используются
гетерогенные высокопроизводительные специализированные
вычислительные сис\-те\-мы (СВС). Обычно СВС имеет несколько управ\-ля\-ющих
узлов, на каж\-дом из которых разворачивается ядро сис\-те\-мы управ\-ле\-ния,
и ряд вы\-чис\-ли\-тель\-ных узлов, непосредственно обрабатыва\-ющих задания.
Специализированные
вычислительные сис\-те\-мы оснащаются  специализированными устройствами, позволяющими
значительно повысить скорость исполнения отдельных вы\-чис\-ли\-тель\-ных
процедур по сравнению со стандартными реализациями. В качестве
элементной базы таких ускорителей, согласно~\cite{Kal}, могут
использоваться заказные сверхбольшие интегральные схемы (СБИС) (ASIC~--- application-specific
integrated circuits), базовые мат\-рич\-ные крис\-тал\-лы (БМК) (\mbox{eASIC}),
сис\-те\-мы-на-крис\-тал\-ле (SoC~--- systems on chip), программируемые логические интегральные схемы (ПЛИС) 
(FPGA~--- field-programmable gate array), графические ускорители
GPGPU (general-purpose graphics processing units).

Различные типы устройств выполняют одно и то же задание с разной
производительностью. Кроме того, некоторые ускорители могут
предназначаться  для работы только с определенными типами алгоритмов
и подходят для решения ограниченного класса задач. Для
специализированных устройств\linebreak в качестве ресурсной единицы
(единичного вы\-чис\-ли\-тель\-но\-го модуля) в СВС рассматривается отдельно
каж\-дая микросхема (ПЛИС, кристалл) как наимень\-ший возможный объект,
имеющий собственное множество состояний и допускающий независимое
управление.

При планировании вычислительных работ в\linebreak СВС возникает
на\-уч\-но-тех\-ни\-че\-ская проб\-ле\-ма, состоящая в отыскании способов
эффективного распределения разнотипных ресурсов между разнородными
работами, выполняющимися в заданных\linebreak  временн$\acute{\mbox{ы}}$х рамках.

В научной литературе понятие эффективности трактуется весьма широко.
В данной работе под эффективностью будем понимать отношение реально
произведенных в СВС специализированных вычислительных операций к
максимально возможному их числу, взятое на определенном интервале
времени.

Рассмотрим специализированную систему, состоящую из большого числа
вычислительных модулей различных типов. Пусть ее диспетчеру
взаимосвязанные пользователи в случайные моменты времени
предоставляют для решения разнородные  citu-за\-да\-чи. Все задания
вместе и каж\-дое в отдельности должны быть завершены  как можно
быстрее, но не позднее заранее определенных сроков. В~противном
случае полученное решение может потерять актуальность и не будет
представлять интереса, а вычислительные затраты будут отнесены к
производственным потерям (издержкам).

Массив исходных данных для каж\-дой из citu-за\-дач состоит из
одинаковых по размеру отдельных неделимых содержательно значимых
фрагментов и становится известен после предварительного анализа
задачи в  системе. Выполнение отдельной  citu-ра\-бо\-ты прекращается,
когда обнаружен фрагмент, удовлетворяющий наперед заданному
критерию,~--- уникальный фрагмент, или если просмотрен весь массив,
но ничего найти не удалось. Считается, что архитектура
вычислительной системы позволяет просматривать  фрагменты каж\-до\-го
citu-за\-да\-ния  в произвольном порядке и каж\-дая часть данных может
обрабатываться независимо, в том числе одновременно  с другими.

Администратор или опе\-ра\-тор-пла\-ни\-ров\-щик в режиме реального времени должен:
\begin{enumerate}[(1)]
\item эффективно использовать разнотипные вы\-чис\-ли\-тель\-ные модули при обработке разнородных вычислительных заданий;
\item завершать каж\-дое конкретное  задание  до наступления назначенного срока и тем самым минимизировать потери.
\end{enumerate}
При этом обработка происходит в условиях неопределенности, связанной
со случайным характером формирования текущего набора citu-за\-да\-ний,
длительностью процесса обработки каж\-до\-го из них и с потенциальной
возможностью получить решение.

В литературе при проектировании и анализе современных сис\-тем
реального времени  в основном рассматриваются <<жестко>> заданные
директивные сроки окончания, нарушение которых может иметь фатальные
последствия. В~ходе изучения таких систем вначале делается
предположение, что  директивный срок окончания может быть превышен,
а затем предлагаются меры, позволяющие этого избежать~\cite{RT2004}.
Настоящая работа посвящена изучению функционирования гетерогенной
вычислительной системы с более мягкими временн$\acute{\mbox{ы}}$ми ограничениями. 
В~рас\-смат\-ри\-ва\-емой модели считается, что при поступлении citu-за\-да\-чи в
систему  определяется\linebreak возможное время ее завершения с учетом
объек-\linebreak тивных показателей  загруженности последней и имеющихся
заданий. Если предварительная оценка~--- срок, к которому
citu-за\-да\-ние может быть\linebreak завершено,~--- устраивает пользователя, то
оно принима\-ется для обработки. При этом заданию на\-зна\-ча\-ет\-ся
директивный срок окончания (ДСО), совпада\-ющий с прогнозируемым. Таким
образом,\linebreak ДСО~--- это уста\-нов\-лен\-ный
диспетчером и согласованный с пользователем момент календарного\linebreak
времени, до наступления которого citu-за\-да\-ча должна быть решена.
Дальнейшее планирование осуществляется исходя из предположения, что
события будут развиваться по наихудшему сценарию. В~условиях
объективной неопределенности  ДСО  используется как ограничение при
формировании диспетчерских правил (политики) совместного выполнения
всех заданий, находящихся  в~СВС.

Заметим, что понятие ДСО, введенное выше, несколько отличается от
классического dead-line~\cite{Stan}, поскольку устанавливается
диспетчером, а не пользователем сис\-те\-мы, хотя и по договоренности с
последним. Однако принятое допущение больше соответствует реальной
ситуации. Действительно, в этом случае citu-ра\-бо\-та, которая не может
быть выполнена по объективным причинам, будет сразу отозвана, что
позволит решить другие задачи в срок. У~администратора появляется
возможность более гибко подходить к формированию пакета текущих
заданий и минимизировать возможные производственные потери.

Данная работа продолжает исследования, начатые в~[6--8], однако
отличается от них мето-\linebreak дами изуче\-ния и конечной целью. 
В~\cite{Gol11, Mal412} с\linebreak по\-мощью имитационного моделирования
анализировались временн$\acute{\mbox{ы}}$е показатели длительности обработки заданий.
В~настоящей статье показано, как, опираясь на  методы оптимизации~\cite{Suh} 
и принцип гарантированного результата~\cite{germ}, можно
оценить и минимизировать максимальную величину возможного превышения
установленных сроков завершения для любого набора исходных
citu-за\-да\-ний и повысить эффективность использования СВС. 

%
Перейдем к
описанию математической модели (М-мо\-де\-ли), которая может быть
использована при оперативном планировании работ и реализации
различных диспетчерских политик.


\section{Описание модели}

При описании М-модели верхний индекс любой переменной будет
всегда относиться к типу вычислительного модуля, а нижний (один или
два) в зависимости от рассматриваемого случая~--- отвечать номеру и
виду задачи соответственно. Для того чтобы избежать путаницы, второй
нижний идентификатор, соответствующий виду задачи,  помещен в
скобки. Кроме того, в зависимости от контекста  через~$t$ обозначим
либо текущий календарный момент времени,  либо контрольную точку
принятия решения. Таким образом,  $t$  не является переменной  в
традиционном смысле, это обозначение определенных моментов времени в
модельном описании процесса. Далее будем говорить об абстрактной
гетерогенной высокопроизводительной  СВС, в которой в условиях
неопределенности выполняются разнородные citu-ра\-боты.

В рамках М-модели считается, что  указанная СВС состоит из
центрального управляющего устройства (ЦУП-устрой\-ст\-ва) и набора
независимых исполняющих единичных вычислительных модулей
(ЕВ-мо\-ду\-лей) различных конструктивных типов. Введем обозначения:

$\textbf{H}$~--- ЦУП-устройство, в котором происходит анализ
поступающих заданий, определяются ДСО и порядок их выполнения, а
также осуществляется контроль процесса обработки;

$\textbf{E}$~ --- набор (множество)  ЕВ-мо\-ду\-лей нескольких типов.

Пусть
$M$~--- общее число различных типов ЕВ-мо\-ду\-лей, из которых состоит рассматриваемая СВС;
$e^m$~--- отдельный ЕВ-мо\-дуль (устройство) \mbox{$m$-го} типа;
$E^m$~ --- множество  ЕВ-мо\-ду\-лей $m$-го типа $e^m$, $m \hm= \overline{1, M}$,
следовательно,  множество $\textbf{E}$ всех ЕВ-мо\-ду\-лей в СВС можно
записать в виде объединения
$$
\textbf{E} = E^1 \bigcup E^2 \bigcup \cdots \bigcup E^M\,. 
$$

Под специализированной элементарной вы\-чис\-ли\-тельной операцией
(СЭВ-опе\-ра\-ция) будем \mbox{понимать} просмотр отдельного
неделимого содержательно значимого фрагмента данных определенного
вида  и проверку  его  уникальности. Считается, что любой ЕВ-модуль
может обрабатывать задания по крайней мере одного типа.
Производительность конкретного ЕВ-мо\-ду\-ля при выполнении citu-за\-да\-ний
разных видов  не одинакова и может меняться (перегрев оборудования,
изменение тактовой час\-то\-ты или работа с другим программным
обеспечением). Кроме того, будем считать, что ЕВ-мо\-ду\-ли с течением
времени  могут выходить из строя.

Пусть $p_k^m(t)$~--- производительность  ЕВ-мо\-ду\-ля $m$-го типа при
обработке задания $k$-го вида, т.\,е.\ модуль $e^m$ может выполнять в
единицу времени   $p_k^m(t)$  СЭВ-опе\-ра\-ций $k$-го вида, начиная с
момента $t$; $R^m(t)$~---  общее число работоспособных ЕВ-мо\-ду\-лей
$m$-го типа на момент времени~$t$.


Будем считать, что СВС может одновременно обрабатывать как одно, так
и несколько заданий на всех действующих в данный момент 
ЕВ-мо\-ду\-лях. Отдельное citu-за\-да\-ние может быть разделено на
подзадания, каж\-дое из которых, в свою очередь, выполняется
самостоятельно или в составе некоторого набора (пакета). Для решения
каж\-дой citu-за\-да\-чи достаточно найти хотя бы один уникальный
фрагмент.

Для формального описания  заданий будут использоваться следующие обозначения:

$z_{n}$~--- задание (задача) с собственным идентификационным номером~$n$;

$K$~--- общее число видов заданий, которые могут обрабатываться в
данной СВС;

$z_{n{(k)}}$~--- задание (задача) с собственным идентификационным
номером $n$, для которой явно указан ее вид~$k$, $k\hm=\overline {1,K}$.
Данное обозначение введено для удобства описания  процесса обработки
разнородных заданий;

$t_n^0$~--- календарный момент поступления задания $z_n$ в СВС;
$T_n(t)\hm=t \hm- t_n^0$~--- длительность  промежутка
времени,  в течение которого задание $z_n$ находится в СВС при
условии, что на момент~$t$ оно еще не выполнено до конца, т.\,е.\
$T_n(t)$~---  число единиц времени, которое прошло от поступления~$z_n$ 
в СВС  до текущего момента~$t$; $t_n^+$~--- момент завершения
задания $z_n$ и/или  его удаления из СВС; $T_n^+ \hm= t_n^+ \hm- t_n^0$~--- 
длительность  промежутка времени, в течение которого задание~$z_n$  обрабатывалось  СВС,\linebreak 
т.\,е.\ $T_n^+$~---  длительность пребывания~$z_n$ в сис\-теме.


Для множества $\mathcal{Z}(t)$ заданий $z_n$, находящихся в СВС в момент~$t$,
введем обозначения:

$\mathcal{N}(t)$~--- множество номеров  заданий $z_n$ из~$\mathcal{Z}(t)$;

$N(t)$~--- общее число заданий $z_n$ из  $\mathcal{Z}(t)$. Таким образом  
$|\mathcal{Z}(t)| \hm= |\mathcal{N}(t)| \hm= N(t)$;

$\mathcal{Z}_k(t)$~--- множество заданий $k$-го вида;

$\mathcal{N}_k(t)$~--- множество номеров  (индексов) заданий $k$-го вида;

$N_k(t)$~--- общее число заданий $k$-го вида.

Тогда
\begin{gather*}
\mathcal{Z}(t) = \bigcup\limits_{k = 1}^{K} \mathcal{Z}_k(t)\,; \quad 
\mathcal{N}(t) = \bigcup_{k = 1}^{K} \mathcal{N}_k(t)\,; \\
N(t) =\sum\limits_{k = 1}^{K} N_k(t)\,.
\end{gather*}

В М-модели предполагается, что для каж\-до\-го задания $z_n$ в момент
поступления $t_n^0$ становится известна нормативная величина
$\textbf{Z}_{n}$~---  общее  число СЭВ-опе\-ра\-ций, которые будет
необходимо выполнить для данного задания в случае, когда  задача~$z_{n}$  
не имеет решения, т.\,е.\ в исходной содержательной
постановке $\textbf{Z}_{n}$~--- общее число   неделимых фрагментов
данных задания~$z_{n}$, которые необходимо будет обработать, если
среди них нет уникального.

В СВС диспетчеризация процесса выполнения citu-за\-да\-ний
осуществляется с  помощью про\-грам\-мно\-го комплекса планирования и
анализа (ПК-план). Для каж\-дой поступившей  citu-за\-да\-чи в  ПК-пла\-не
определяются необходимые вычислительные  затраты и время решения  в
наихудшем случае с учетом общей загруженности СВС. После этого по
согласованию с пользователем назначается ДСО, т.\,е.\ календарный
момент времени, до наступления которого  $z_n$ должно быть
завершено. Обозначим  через $d_n$ ДСО для~$z_n$.

В СВС вся исходная и текущая информация о citu-за\-да\-ни\-ях, вновь
поступивших или находящихся в обработке, заносится  и далее хранится
в так называемой базе данных заданий (БД-за\-да\-ний). Про\-граммный комплекс планирования и
анализа
постоянно просматривает и анализирует БД-за\-да\-ний. Динамика обработки
(просмотров) массивов данных в ПК-пла\-не описывается сис\-те\-мой
неравенств и ко\-неч\-но-раз\-ност\-ных уравнений в дискретном времени. Шаг
по времени называется плановым периодом или операционным окном.

В контрольной точке принятия решения~$t$ по команде с
ЦУП-устрой\-ст\-ва ПК-план анализирует\linebreak данные о  выполнении всех
заданий, находящихся в СВС. Заметим, что  здесь $t$~--- начальный\linebreak
момент очередного планового периода. Далее диспетчер (специалист
и/или алгоритмическая процедура)  выбирает число единиц календарного
времени $\Delta(t)$~--- длительность планового периода.\linebreak  В~ПК-пла\-не
формируется пакет текущих работ\linebreak (ТР-па\-кет), состоящий из подмассивов
(поднаборов) неделимых фрагментов данных  подзаданий $z_{n}$, $n \hm\in
\mathcal{N}(t)$, которые будут  обрабатываться в СВС   в течение $\Delta(t)$.
Центральное управ\-ля\-ющее устрой\-ст\-во  помещает  ТР-пакет в специальный буфер текущих работ
(ТР-бу\-фер) и начиная с момента~$t$   СВС выполняет  его на всех
работоспособных ЕВ-мо\-дулях.

\section{Параметры планирования и~управления }

Обозначим через $z^-_{n(k)}(t)$ число СЭВ-опе\-ра\-ций $k$-го вида,
которые были завершены для задания~$z_n$ от его поступления в СВС до
момента времени~$t$, или в рамках М-мо\-де\-ли~--- чис\-ло неделимых
фрагментов, которые уже были просмотрены для задания~$z_n$ к моменту
времени~$t$;  соответственно $z^+_{n(k)}(t)$~--- число фрагментов,
которые  для~$z_n$ остались необработанными. Тогда
$$
z^+_{n(k)}(t)= \textbf{Z}_{n} - z_{n(k)}^-(t)\,, \ \ n \in \mathcal{N}(t)\,.      
$$

Как уже указывалось ранее, подзадания ТР-па\-ке\-та представляют собой
различающиеся по объему поднаборы фрагментов данных для заданий
$z_n$, $n \hm\in \mathcal{N}(t)$. Размер подзадания для  $z_{n{(k)}}$ обозначим
через $w_{n(k)}(\Delta(t))$. Таким образом, $w_{n(k)}(\Delta(t))$~--- 
чис\-ло неделимых фрагментов  в поднаборе, взятом из массива
начальных данных задания $z_{n(k)}$ в момент~$t$ для просмотра в
течение  $\Delta(t)$, или чис\-ло СЭВ-опе\-ра\-ций  $k$-го вида, которые
планируется выполнить в операционном окне $\Delta(t)$ для
$z_{n(k)}$. Размеры отобранных поднаборов данных (подзаданий)
$w_{n(k)}(\Delta(t))$  являются управ\-ле\-ни\-ями, или управляющими
параметрами, которые определяются в ПК-пла\-не с учетом текущей
диспетчерской политики соблюдения ДСО.


Поскольку в момент $t$ для $z_n$ нельзя взять в обработку больше
данных, чем осталось, управления $w_{n(k)}(\Delta(t))$ на интервале
$\Delta(t)$ должны удовлетворять ограничениям
\begin{equation}
0 \le w_{n}(\Delta(t)) \le  z^+_{n}(t)\,,  \ \ n \in \mathcal{N}(t)\,. 
\label{e1-mal}
\end{equation}

Обозначим через $W_{k} (\Delta(t))$ суммарное чис\-ло фрагментов
$k$-го вида, которые планируется включить в ТР-па\-кет. Тогда
$W (\Delta(t))$~--- общий объем ТР-па\-ке\-та  для просмотра в операционном окне $\Delta(t)$:
\begin{multline}
W (\Delta(t)) = \sum\limits_{k = 1}^K W_{k}(\Delta(t)) =  {}\\
{}=
\sum\limits_{k = 1}^K \sum\limits_{n \in \mathcal{N}_k(t)} w_{n(k)}(\Delta(t)) =  
\sum\limits_{n \in \mathcal{N}(t)} w_{n}(\Delta(t))\,.
\label{e2-mal}
\end{multline}


Введем переменную $r_{k}^m(t)$~--- общее число  ЕВ-мо\-ду\-лей  $m$-го
типа, которые начинают обрабатывать подзадания  $k$-го вида в момент
времени~$t$. $r_{k}^m(t)$ также являются управ\-ле\-ни\-ями  и
определяются исходя из   текущих установок диспетчерских правил.

Предположим, что ни один из поднаборов $w_{n(k)}(\Delta(t))$ не 
содержит уникального фрагмента, но все поднаборы планируется  
выполнить за  период $ \Delta(t)$.  Тогда в момент~$t$ для просмотра всех данных 
для подзаданий $k$-го вида необходимо выделить  $r_{k}^m(t)$  ЕВ-мо\-ду\-лей $m$-го типа,
\begin{equation}
W_{k} (\Delta(t)) = \sum\limits^M_{m = 1} p_{k}^m (t)  r_{k}^m (t)  \Delta(t)\,, 
\ \ k = \overline{1,K}\,, 
\label{e3-mal}
\end{equation}
которые за период $\Delta(t)$ должны  проделать $W_{k}(\Delta(t))$   СЭВ-опе\-ра\-ций $k$-го вида.
Поскольку число ЕВ-модулей, распределяемых для выполнения заданий, не может превышать 
общего числа работоспособных, то
значения $r_{k}^m(t)$ должны удовлетворять следующим ограничениям:
\begin{equation}
\left.  
\begin{array}{c}
\displaystyle\sum\limits_{k=1}^K r_{k}^m(t) \le R^m(t)\,, \ \ m = \overline{1,M}\,;\\[9pt]
r_{k}^m (t) \ge 0\,, \ \ k = \overline{1,K}\,, \ \ m = \overline{1,M}\,.
                    \end{array}
                    \right \} 
                    \label{e4-mal}
                    \end{equation}

\section{Интегральная производительность специализированной вычислительной системы}

При диспетчеризации СВС, где на разнотипном оборудовании с различной
скоростью выполняются разнородные СЭВ-опе\-ра\-ции, возникает проблема
эффективного использования имеющихся ресурсов.

Обозначим через $\rho(t)$ интегральную производительность СВС. Под
$\rho(t)$ будем понимать суммарную производительность СВС на
интервале $ \Delta(t)$, т.\,е.\ общее число СЭВ-операций,
которые выполняются на всех  действующих ЕВ-мо\-ду\-лях, начиная с
момента~$t$ на отрезке планирования $ \Delta(t)$:
$$
\rho(t) = \sum\limits_{m = 1}^{M} \sum\limits_{k = 1}^{K}  p_k^m(t) r_k^m(t)  \Delta(t)\,.
$$
В качестве максимальной интегральной производительности СВС $P^*(t)$  на интервале 
$\Delta(t)$ выберем скалярное решение следующей задачи линейного программирования: найти
$
P^*(t) \hm= \max\limits_{\rho, r} \rho(t) 
$
при ограничениях  на производительность СВС
\begin{equation}
\left.  
\begin{array}{rl}
\rho(t) &= \sum\limits_{m = 1}^{M} \sum\limits_{k = 1}^{K}  p_k^m(t) r_k^m(t)  \Delta(t)\,;\\[9pt]
\sum\limits_{k = 1}^{K} r_k^m (t)  &= R^m(t)\,, \  \ m = \overline{1, M}\,;\\[9pt]
r_k^m (t) &\ge 0\,, \ \ k = \overline {1, K}\,, \ \ m = \overline{1,M}\,.
                    \end{array}
                    \right \} 
                    \label{e5-mal}
\end{equation}

Значение $P^*(t) $ показывает, какое максимальное число СЭВ-операций
суммарно для всех видов заданий может быть выполнено СВС в
операционное окно $ \Delta(t)$ начиная с момента~$t$. Пусть на
интервале $ \Delta(t)$ предполагается произвести $W(\Delta(t))$
СЭВ-опе\-ра\-ций всех видов (см.~(\ref{e2-mal})). Тогда отношение
$$
\gamma(t) = \fr{ W(\Delta(t))}{P^*(t)}
$$
назовем плановой эффективностью использования СВС на интервале 
$\Delta(t)$ для данного набора заданий $\mathcal{Z}(t)$. Максимальное значение
$W(\Delta(t))$ и, соответственно, $\gamma (t)$  существенно зависит
от состава и объема заданий, находящихся на обработке в момент
времени $t$ (см.\ (\ref{e1-mal}) и~(\ref{e2-mal})). Сравнивая множества ограничений~(\ref{e1-mal})--(\ref{e4-mal}) 
и~(\ref{e5-mal}), можно утверждать, что для любого набора заданий $\mathcal{Z}(t)$
выполняются неравенства
$$
P^*(t) \ge  W(\Delta(t))\,, \quad 0 \le \gamma(t ) \le 1\,.
$$
Кроме того, при диспетчеризации следует учитывать другие ограничения
и требования пользователей, которым должны подчиняться управ\-ле\-ния
$r_k^m (t), w_{n(k)}(\Delta(t))$.

\section{Определение директивного срока окончания вновь поступившего задания}

Для управления  гетерогенными СВС в реальном времени необходимы и разрабатываются интерактивные 
сценарии диалога дис\-пет\-чер--поль\-зо\-ва\-тель в декларативной, час\-тич\-но вербальной, частично 
технологической, постановке: за\-да\-ние--тре\-бо\-ва\-ния--воз\-мож\-но\-сти--сро\-ки.
В~рамках М-мо\-де\-ли в ПК-пла\-не для каж\-до\-го вновь поступившего задания вычисляется 
гарантированная оценка времени завершения. Фактически решается задача быстродействия 
выполнения всех заданий в наихудшем случае.

Пусть в момент $t$ в СВС, которая обрабатывает пакет заданий $\mathcal{Z}(t)$,  поступает новая 
задача $\hat{z}$, например, $\hat{k}$-го вида. В ПК-пла\-не этой задаче присваивается порядковый 
номер $\hat n_{(\hat{k})}$. Кроме того,  становится известна величина  
$\mathbf{Z}_{\hat n{(\hat{k})}}$~--- общее чис\-ло фрагментов данных, которые необходимо 
будет просмотреть при условии, что среди них нет уникального.

Для каж\-до\-го $z_{n(k)}$, $n \hm\in \mathcal{N}_k(t)$, $k \hm= \overline{1,K}$, и каж\-до\-го~$m$ 
определим переменную $\nabla_{n(k)}^m (t )$~---  число единиц времени (единичных временн$\acute{\mbox{ы}}$х 
интервалов), в течение которых задание     $z_{n(k)}$  планируется  выполнять на всех 
работоспособных ЕВ-мо\-ду\-лях      $m$-го типа,    $m \hm= \overline{1, M}$, после  момента~$t$. 
При определении ДСО для вновь поступивших заданий рассматривается  наихудший случай: ни одна из 
вновь полученных и ни одна из находящихся в обработке задач не имеет решения. Другими словами, 
для окончания всех имеющихся заданий необходимо будет  просмотреть все имеющиеся фрагменты данных. 
Следовательно, для заданий, обработка которых уже начата, будет необходимо просмотреть
\begin{multline*}
\mathbf{Z}_{n(k)} - z^-_{n(k)}(t) =  \sum\limits_{ m = 1}^{M} p_k^m(t) R^m(t) \nabla _{n(k)}^m (t )\,,\\
n \in \mathcal{N}_k(t)\,,\ k = \overline{1, K}\,,
\end{multline*}
фрагментов, а для вновь поступившего задания~---
\begin{multline*}
\textbf{Z}_{\hat n(\hat{k})} =  \sum\limits_{ m = 1}^{M} p_k^m(t) R^m(t)  
\nabla_{\hat n{(\hat{k})}} ^m (t )\,,  
\\
\nabla _{n(k)}^m (t ) \ge 0\,,\ \nabla _{\hat n(\hat k)} ^m (t ) \ge 0\,,\ 
m = \overline {1, M}\,,\ n \in \mathcal{N}(t)\,. 
\end{multline*}

Введем переменную $\nabla ^m (t )$:
\begin{multline*}
\nabla ^m (t ) = \sum\limits_{k = 1}^K \ \sum_{n \in \mathcal{N}_k(t) } \nabla _{n(k)} ^m (t ) + 
 \nabla _{\hat n{(\hat{k})}}^m (t )  = {}\\
{}= \sum\limits_{n \in \mathcal{N}(t) } \nabla _{n(k)} ^m (t ) + \nabla _{\hat n{(\hat{k})}}^m (t )\,,
m = \overline {1, M}\,.
\end{multline*}

Значение $\nabla ^m (t )$ показывает, сколько единиц времени после момента~$t$ потребуется 
ЕВ-мо\-ду\-лям $m$-го типа для завершения всех задач, имеющихся в СВС, при условии, что ни 
одна из них не имеет решения и необходимо  просмотреть все данные всех заданий из $\mathcal{Z}(t)$.

В ПК-плане в качестве оценки ДСО для вновь поступившего задания  $z_{\hat n_{(\hat{k})}}$ 
используется решение  задачи \textit{быстродействия}:

\smallskip

\noindent
в момент $t$ для заданных $\mathbf{Z}_n$, $\mathbf{Z}_{\hat n{(\hat{k})}}$, 
$z^-_{n(k)}(t), \mathcal{N}(t)$,  $R^m(t)$, $d_j$ найти
\begin{equation*}
 \delta^* =\min\limits_{\delta, \nabla} \delta
%\label{e6=mal}
\end{equation*}
при условиях:
\begin{enumerate}[(1)]
\item ДСО для вновь поступившего задания будет  не меньше, чем у находящихся в обработке:
\begin{equation*}
\delta \ge d_j\,,\ j \in \mathcal{N}(t)\,;
%\label{e7-mal}
\end{equation*}
\item ДСО не меньше гарантированного времени завершения всех заданий на всех ЕВ-модулях:
\begin{equation*}
\delta \ge t + \sum\limits_{n \in \mathcal{N}(t) } \nabla _{n(k)}^m (t) +  \nabla _{\hat n{(\hat{k})}}^m 
(t)\,,\ m = \overline {1, M}\,; 
%\label{e8-mal}
\end{equation*}
\item для выполнения всех заданий необходимо будет просмотреть все имеющиеся в момент~$t$ данные
\begin{multline*}
\hspace*{-5mm}\mathbf{Z}_{n(k)} - z^-_{n(k)}(t) =  \displaystyle\sum\limits_{ m = 1}^{M} p_k^m(t) R^m(t) \nabla _{n(k)}^m (t)\,,\\
\!\nabla _{n(k)}^m (t ) \ge 0\,,  \ m = \overline {1, M}\,,\ n \in \mathcal{N}_k(t);\ 
k = \overline{1, K};
\end{multline*}

\vspace*{-18pt}

\begin{multline*}
\mathbf{Z}_{\hat n(\hat{k})} =  \sum\limits_{ m = 1}^{M} p_{\hat{k}}^m(t) R^m(t)  
\nabla_{\hat n{(\hat{k})}} ^m (t )\,, \\
\nabla_{\hat n(\hat k)}^m (t ) \ge 0\,,\   m = \overline {1, M}\,.
\end{multline*}
\end{enumerate}

Полученное $\delta ^*$ является гарантированной оценкой ДСО для вновь поступившей 
задачи. Администратор сообщает значение $\delta ^*$ пользователю, который предоставил 
данную задачу. Если пользователя не устраивают прогнозируемые сроки завершения задания, 
то он отказывается от обработки и снимает задачу со счета. В~противном случае  
задание $z_{\hat n(\hat{k})}$ помещается в общий пакет $\mathcal{Z}(t)$ с условием, 
что его обработка должна быть завершена до момента $d_{\hat n{(\hat{k})}} \hm= \delta ^*$.

\section{Гарантированные оценки потерь процессорного времени}

Рассмотрим проблему диспетчеризации citu-за\-да\-ний с фиксированными ДСО. 
В~рамках М-мо\-де\-ли предполагается, что в случае превышения ДСО произведенные 
вычислительные затраты записываются в производственные потери, которые администратору 
предписывается свести к минимуму.
На первый взгляд описанный выше способ назначения ДСО для всех заданий, принятых на обработку, 
гарантирует их окончание в срок. Однако в реальности в СВС могут происходить сбои, изменение 
приоритетов в обслуживании заявок и~др., что неизбежно приведет к нарушению ДСО.  
В~М-мо\-де\-ли в момент~$t$ анализируются достаточные условия завершения заданий  в 
срок и вычисляются гарантированные оценки возможных превышений ДСО в наихудшем случае.

В соответствии с принятыми ранее обозначениями $d_j$~ --- ДСО задачи~$z_j$. 
Каж\-до\-му   $d_j$, $j \hm\in \mathcal{N}(t)$, ставится  в соответствие
множество (список) номеров  $\mathcal{N}(t, d_j)$  всех заданий~$z_i$ из
$\mathcal{Z}(t)$, таких что каж\-дое~$z_i$ требуется завершить не позднее~$d_j$, т.\,е.
$$
\mathcal{N}(t, d_j) = \{ i \ \vert  \ d_i \le d_j\,, \  i \in  \mathcal{N}(t) \}\,,\ j \in \mathcal{N}(t)\,.   
$$
Для получения гарантированных оценок рас\-смот\-рим наихудший случай: пусть ни одна из 
задач  $z_n$,\linebreak\vspace*{-12pt}

\pagebreak

\noindent
 $n \hm\in \mathcal{N}(t)$,  не имеет решения. Для фиксированного момента времени 
$t + \Delta(t)$  введем дополнительный параметр управления
$\nabla _{n(k)}^m (t \hm+  \Delta(t))$~---  число единиц времени (единичных временных интервалов), 
в течение которых задание     $z_n$       может обрабатываться на всех работоспособных ЕВ-мо\-ду\-лях      
$m$-го типа,    $m \hm= \overline{1, M}$, начиная с  $t \hm+ \Delta(t)$.

В рассматриваемом наихудшем случае задача~$z_n$  не имеет решения, поэтому для завершения 
задания  необходимо будет перебрать все  данные
\begin{equation}
z^+_n(t + \Delta(t)) = \mathbf{Z}_{n(k)} - z_{n(k)}^-(t) - w_{n(k)}(\Delta(t))\,,
\label{e10-mal}
\end{equation}
 оставшиеся необработанными для  $z_n$ к моменту $t \hm+ \Delta(t)$.
Здесь управление  $w_{n(k)}(\Delta(t))$~--- число фрагментов данных для~$z_n$,  которые  
в момент~$t$ планируется просмотреть за время  $\Delta(t)$.

Таким образом, для оставшихся фрагментов  в момент $t \hm+ \Delta(t)$ должно выполняться равенство:
\begin{multline}
z^+_n(t + \Delta(t)) = \displaystyle\sum\limits_{m = 1}^M  p^m_k(t)  R^m(t)  \nabla _{n(k)}^m (t 
+ \Delta(t))\,,\\
 n \in \mathcal{N}_k(t)\,, k = \overline{1, K}\,,
 \label{e11-mal}
\end{multline}
и  условие неотрицательности:
$$
\nabla _{n(k)}^m (t + \Delta(t)) \ge 0\,,\ m = \overline{1, M}\,,\ n \in \mathcal{N}(t)\,.
$$

В момент $t + \Delta(t)$    рассмотрим дополнительную переменную
\begin{multline}
\nabla^m (t + \Delta(t), d_j) = {}\\
\vspace*{-3mm}{}=\sum\limits_{k = 1}^K  
\sum\limits_{n \in \mathcal{N}(t, d_j) \bigcap \mathcal{N}_k(t)} \nabla _{n(k)} ^m (t + \Delta(t)) = {}\\
{} = \!\!\!\!\sum\limits_{n \in \mathcal{N}(t, d_j)}\!\!\!\! \nabla _{n(k)}^m (t + \Delta(t))\,, \ \ m = \overline{1, M}\,, 
\ \  j \in \mathcal{N}(t).\!\!
\label{e12-mal}
\end{multline}
Значение $\nabla ^m (t + \Delta(t), d_j)$ показывает, сколько единиц времени после  
момента $t \hm+ \Delta(t)$  потребуется ЕВ-мо\-ду\-лям  $m$-го типа для выполнения всех 
заданий $z_n$, $n \hm\in \mathcal{N}(t, d_j)$, которые должны быть  завершены до наступления~$d_j$. 
В~случае, \mbox{когда}\linebreak ни одна из задач не имеет решения,  величина\linebreak   $\nabla ^m (t + \Delta(t), d_j)$~--- 
точная верхняя оценка временн$\acute{\mbox{ы}}$х затрат на обработку всех $z_n$, $n \hm\in \mathcal{N}(t, d_j)$, с помощью работоспособных  
ЕВ-мо\-ду\-лей $m$-го типа, $m \hm= \overline{1, M}$. 
В~момент  $t$ достаточные условия соблюдения всех ДСО в наихудшем случае можно записать в виде:
$$
 t + \Delta(t) + \nabla^{m}(t + \Delta(t), d_j) \le   d_j\,,  \   j \in \mathcal{N}(t)\,, \ m = \overline {1,M}\,.
 $$

Чтобы получить гарантированные оценки относительных величин возможного превышения  ДСО для $z_n$,   введем переменные:

\noindent
\begin{multline}
\omega^m(t + \Delta(t) , d_j) = {}\\
{} = \fr{t + \Delta(t) +  \nabla^{m}(t +  \Delta(t), d_j) - t^0_j}{d_j -  t_j^0 }\,, \\ 
j \in \mathcal{N}(t)\,, \ 
m = \overline {1,M}\,,
\label{e13-mal}
\end{multline}
где $ t_j^0$~--- время поступления $z_j$ в СВС.  Конкретное значение $\omega^m(t \hm+ \Delta(t), d_j)$  
характеризует возможность соблюдения ДСО следующим образом:
\begin{enumerate}[(1)]
\item если существуют управления $w_n( \Delta(t))$, $\nabla _{n(k)}^m (t \hm+ \Delta(t))$ и 
значение $\nabla ^m (t \hm+ \Delta(t), d_j)$, удовле\-тво\-ря\-ющие~(\ref{e10-mal})--(\ref{e13-mal}), 
при которых  $\omega^m (t \hm+ \Delta(t), d_j) \hm\le 1$ для всех 
$j \hm\in \mathcal{N} (t)$, $m\hm = \overline {1, M}$,
то все задания можно гарантированно завершить до наступления их ДСО даже в наихудшем случае;

\item если же при любых управлениях $w_n(\Delta(t))$, $\nabla _{n(k)}^m (t \hm+ \Delta(t))$ и значениях 
$\nabla^m (t \hm+ \Delta(t), d_j)$ в~(\ref{e10-mal})--(\ref{e13-mal})  хотя бы для одного сочетания~$m$  и~$j$ величина 
$\omega^m (t \hm+ \Delta(t), d_j) \hm> 1$, то   может реализоваться ситуация,  при которой хотя бы одно задание 
в наихудшем случае завершится после назначенного (предписанного) срока.
\end{enumerate}

На самом деле максимальная величина $\omega^m (t \hm+ \Delta(t), d_j)$ является своего рода индикатором, 
который указывает, располагает ли  СВС на момент  $t$  необходимой мощностью для того, чтобы обработать 
все имеющиеся разнородные задания до назначенных ДСО в наихудшем случае.
Для проверки достаточных условий выполнения ДСО в момент~$t$ и получения 
гарантированных оценок максимально  возможного их превышения в рамках М-модели введем параметр управления  $ \omega(t)$:
$$
\omega (t) \le 1 -  \omega^m(t + \Delta(t), d_j)\,, \ \ m = \overline {1, M}\,,\ j \in \mathcal{N}(t)\,. 
$$

Величина $\omega (t)$ характеризует максимальное относительное значение возможного превышения ДСО.

\section{Формирование  пакета  текущих работ}

Рассмотрим возможный способ формирования ТР-пакета, при котором в процессе выполнения citu-за\-да\-ний 
требуется добиться высокой эффективности использования СВС и минимизировать производственные потери, в 
данном случае за счет соблюдения ДСО.   В рамках М-модели для анализа эффективности и получения гарантированных 
оценок превышения ДСО в момент времени $t$ решается следующая задача оптимизации:

\smallskip

\noindent
для заданных $\mathbf{Z}_n$, $z^-_{n(k)}(t)$, $N(t)$,  $R^m(t)$, $d_j$, $\mathcal{N}(t)$, $P^*(\Delta(t))$ найти
\begin{equation}
\Phi^* = \max\limits_{\gamma, \omega, \nabla, w, r, W} (c_1\gamma(t) + c_2 \omega (t) + c_3 \omega ^{\Sigma} (t)) 
\label{e14-mal}
\end{equation}
при ограничениях:
\begin{enumerate}[(1)]
\item на эффективность использования  разнотипных ЕВ-модулей  в СВС
\begin{equation}
\hspace*{-10mm}0 \le \gamma (t) P^*(\Delta(t))  =  \sum\limits_{k = 1}^K \sum\limits_{m = 1}^M p^m_k(t)  r_k^m(t)  \Delta(t);
\label{e15-mal}
\end{equation}

\item на суммарную величину отклонения от заданных ДСО
\begin{equation}
\left.  
\begin{array}{rl}
\hspace*{-6mm}\displaystyle\omega^{\Sigma} (t ) &\displaystyle\le \sum\limits_{m = 1}^{M} \sum\limits_{j \in \mathcal{N}(t)} 
\fr{1-\omega^m(t+\Delta(t),d_j)}{d_j};\\[9pt] 
\hspace*{-6mm}\omega^{\Sigma} (t )& \le 0;
                    \end{array}
                    \right \}
                    \label{e16-mal}
                    \end{equation}

\item на максимальную величину отклонения от заданных ДСО
\begin{equation}
 \left.  \begin{array}{rl}
\omega (t) &\le 1 -  \omega^m(t + \Delta(t), d_j)\,,\\[9pt]
& \hspace*{10mm}j\in \mathcal{N}(t)\,, \ m = \overline{1,M}\,;\\[9pt]
\omega (t) &\le 0\,;
                    \end{array}
                    \right \} 
                    \label{e17-mal}
                    \end{equation}

\item на балансовые соотношения выполнения заданий после $t \hm+ \Delta(t)$
\begin{equation}
\left.  
\begin{array}{rl}
\displaystyle\mathbf{Z}_{n} - z^-_{n}(t) - w_{n} (\Delta(t)) &={}\\
&\hspace*{-33mm}{}= \displaystyle\sum\limits_{m = 1}^M  p^m_k (t)  R^m( t ) \nabla_{n(k)} ^m (t + \Delta(t))\,,\\[9pt]
 &\hspace*{-20mm}n \in \mathcal{N}_k(t)\,, \ \ k = \overline {1, K}\,;\\[9pt]
&\hspace*{-45mm}\nabla _{n(k)}^m (t + \Delta(t)) \ge 0\,, \ n\in \mathcal{N}(t)\,, \ m = \overline {1,M}\,;
                    \end{array}
                    \right \} \!\!
                    \label{e18-mal}
\end{equation}

\vspace*{-12pt}

\noindent
\begin{multline}
\hspace*{-8mm}\nabla^m (t + \Delta(t), d_j)  = \sum\limits_{n \in \mathcal{N}(t, d_j)} \nabla _{n(k)} ^m (t + \Delta(t))\,,\\[9pt]
m = \overline {1, M}\,,\  j \in \mathcal{N}(t)\,;
                    \label{e19-mal}
                    \end{multline}
                    
                    \vspace*{-12pt}
                    
                    \noindent
\begin{multline} 
\hspace*{-8mm}\omega^m (t + \Delta(t), d_j ) = {}\\[9pt]
{}=\displaystyle\fr{ t + \Delta(t) + \nabla^{m}(t + \Delta(t), d_j) - t^0_j}{d_j - t^0_j }\,,\\[9pt]
j \in \mathcal{N}(t)\,, \ m = \overline {1,M}\,; 
                    \label{e20-mal}
                    \end{multline}

\item на управления в операционном окне $\Delta(t)$:
\begin{itemize}
\item[(а)] суммарное число СЭВ-опе\-ра\-ций $k$-го вида, планируемых  для ТР-пакета в момент~$t$:
\begin{equation}   
 \left.  \begin{array}{c}
 \hspace*{-10mm}\displaystyle\sum\limits_{n \in \mathcal{N}_k(t)} w_n (\Delta(t)) = W_{k} (\Delta(t))\,,  \ k = \overline {1,K}\,;\\[9pt]
  \hspace*{-10mm}0 \le  w_{n} (\Delta(t)) \le \mathbf{Z}_n - z_{n}^- (t)\,, \ n\in \mathcal{N}(t)\,;
                    \end{array}
                    \right \}  \label{e21-mal}
                    \end{equation}

\item[(б)] на число ЕВ-мо\-ду\-лей, выделяемых  для обработки фрагментов для подзаданий $k$-го вида в момент $t$:
\begin{equation} 
\left.  \begin{array}{c}
 \hspace*{-13mm}\displaystyle W_{k} (\Delta(t)) \le \sum\limits_{m = 1}^{M} p_{k}^m (t) r_{k}^m(t)  \Delta(t)\,, \ k = \overline {1,K}\,;\\[9pt]
 \hspace*{-13mm}r_k^m (t) \ge 0\,, \ \ k = \overline {1, K}\,, \ \ m = \overline{1,M}\,; \\[9pt]
 \hspace*{-13mm}\displaystyle\sum\limits_{k = 1}^{K} r_{k}^m (t) \le R^m(t)\,, \ m = \overline{1,M}\,.
                    \end{array}
                    \right \} \!\! \label{e22-mal}
                    \end{equation}
                    \end{itemize}
                    \end{enumerate}

Задача~(\ref{e14-mal}) может быть решена стандартными методами линейного
программирования~\cite{Dan}. Поскольку переменные $\gamma (t)$ и
$\omega(t)$ являются относительными, безразмерными и изменяются в
ограниченном диапазоне, то с помощью выбора значений $c_1, c_2, c_3$
можно учитывать текущие приоритеты при выполнении заданий. Например,
если в момент времени~$t$ во главу угла ставится эффективность и при
этом необходимо обработать как можно больше заданий, следует выбрать
$c_1 \hm\gg c_2 \hm> c_3$. Если  в момент~$t$ в множестве $\mathcal{Z}(t)$ много
заданий, которые необходимо завершить до наступления их ДСО, то
следует выбрать $c_1 \ll c_2 \hm\simeq c_3$ и вновь решить~(\ref{e14-mal})--(\ref{e22-mal}).

\section*{Заключение}

Решение задачи~(\ref{e14-mal})--(\ref{e22-mal}) в условиях неопределенности помогает
оперативно планировать выполнение разнородных ресурсоемких заданий в
гетерогенной вычислительной системе. Очевидно, что при управлении
СВС можно рассмотреть и другие диспетчерские политики. Однако
описанная выше расчетная схема концептуально проста и позволяет
оценивать производственные потери и добиваться эффективного
использования СВС. Полученные на практике результаты показали, что
предлагаемый подход является хорошей основой для диалога
администратора и пользователей.

{\small\frenchspacing
{%\baselineskip=10.8pt
\addcontentsline{toc}{section}{Литература}
\begin{thebibliography}{99}

\bibitem{Prep11} 
\Au{Козлов М.\,В., Малашенко Ю.\,Е., Назарова~И.\,А. и~др.} 
Анализ режимов управления вычислительным комплексом в условиях неопределенности.~--- М.: ВЦ РАН, 2011.

\bibitem{Sour}  
{Sourcebook of parallel computing}.~--- San Francisco: Morgan Kaufmann Publs., 2003.

\bibitem{Kal} \Au{ Каляев И.\,А., Левин И.\,И. } 
Реконфигурируемые мультиконвейерные вычислительные структуры для решения потоковых 
задач обработки информации и управления~//  Суперкомпьютерные технологии: разработка, программирование, 
применение: Тр. Междунар. науч.-практич. конф.~-- Ростов-на-Дону: ЮФУ, 2010. Т.~1. С.~100--102.

\bibitem{RT2004} \Au{Sha L.,  Abdelzaher T.,  Arzen~K.-E., \textit{et. al}.} 
Real time scheduling theory: A~historical perspective~//  Real-Time Systems, 2004. Vol.~28.  No.\,2--3. P.~101--155.

\bibitem{Stan} \Au{Stankovic J.\,A., Spuri M., Ramamritham~K., \textit{et. al}.} 
Deadline scheduling for real-time systems, EDF and related algorithms.~--- Boston: Kluwer,  1998.

\bibitem{Kon} \Au{ Коновалов М.\,Г., Малашенко Ю.\,Е., Назарова~И.\,А. } 
Управление заданиями в гетерогенных вычислительных системах~// Изв. РАН. ТиСУ, 2011. №\,2. С.~72--90.

\bibitem{Gol11} \Au{Голосов П.\,Е., Козлов М.\,В., Малашенко~Ю.\,Е. и~др. } 
Анализ управления  специализированными вычислительными заданиями в условиях неопределенности~// 
Изв. РАН. ТиСУ, 2012. №\,1. С.~50--66.

\bibitem{Mal412} \Au{Малашенко Ю.\,Е., Назарова И.\,А. } 
Модель управления разнородными вычислительными заданиями на основе 
гарантированных оценок времени выполнения~// Изв. РАН. ТиСУ. 2012. №\,4. С.~29--38.


\bibitem{Suh} \Au{Сухарев А.\,Г., Тимохов А.\,В., Федоров~В.\,В. } Курс методов оптимизации.~--- М.: Наука, 1986.

\bibitem{germ} \Au{Гермейер Ю.\,Б.} Введение в теорию исследования операций.~--- М.: Наука, 1971.

%\bibitem{gned} \Au{Гнеденко Б.\,В., Коваленко И.\,Н.} Введение в теорию массового обслуживания.~--- М.: ЛКИ, 2011.

%\bibitem{Leung} Handbook of scheduling: Algorithms, models, and performance  analysis~/ Ed. J.\,Y.-T.~Leung.~--- 
%N.Y.: Chapman \& Hall/CRC, 2004.

\label{end\stat}

\bibitem{Dan} \Au{ Данциг Дж.\,Б. } Линейное программирование, его применения и обобщения.~--- М.: Прогресс, 1966.
\end{thebibliography}
}
}

\end{multicols}  %6
\def\stat{agalarov}

\def\tit{ОБ ОДНОПОРОГОВОМ УПРАВЛЕНИИ ОЧЕРЕДЬЮ В~СИСТЕМЕ МАССОВОГО 
ОБСЛУЖИВАНИЯ С~НЕТЕРПЕЛИВЫМИ ЗАЯВКАМИ}

\def\titkol{Об однопороговом управлении очередью в~системе массового 
обслуживания с~нетерпеливыми заявками}

\def\aut{Я.\,М.~Агаларов$^1$}

\def\autkol{Я.\,М.~Агаларов}

\titel{\tit}{\aut}{\autkol}{\titkol}

\index{Агаларов Я.\,М.}
\index{Agalarov Ya.\,M.}


%{\renewcommand{\thefootnote}{\fnsymbol{footnote}} \footnotetext[1]
%{Работа выполнялась с~использованием инфраструктуры Центра коллективного пользования <<Высокопроизводительные вы\-чис\-ле\-ния и~большие данные>> 
%(ЦКП <<Информатика>>) ФИЦ ИУ РАН.}}


\renewcommand{\thefootnote}{\arabic{footnote}}
\footnotetext[1]{Федеральный исследовательский центр <<Информатика и~управление>> Российской академии наук, 
\mbox{agglar@yandex.ru}}


\vspace*{-6pt}

  
  
  \Abst{Изложены результаты теоретического исследования управ\-ля\-емой системы 
массового обслуживания (СМО) типа $M/M/s$ с~нетерпеливыми заявками и~однопороговым 
управлением очередью. Ставится задача оптимизации однопорогового управления очередью, 
суть которой заключается в~вычислении для длины очереди некоторого порогового значения, 
максимизирующего заданную целевую функцию. В~исследуемой сис\-те\-ме заявка покидает 
систему необслуженной, если время ожидания в~очереди (или время обслуживания на приборе) 
превышает некоторый интервал времени случайной длины, распределенной по показательному 
закону с~заданным параметром. В~качестве показателя эф\-фек\-тив\-ности управ\-ле\-ния очередью 
(целевой функции) используется стоимостная функция, учитывающая потери в~единицу 
времени из-за технического обслуживания сис\-те\-мы, отклонения заявок на входе сис\-те\-мы, 
ухода заявок до завершения обслуживания. Предложены метод решения задачи максимизации 
стоимостной целевой функции на множестве однопороговых управ\-ле\-ний очередью 
и~алгоритм гарантированного вычисления оптимального порога. }
  
  
  \KW{система массового обслуживания; нетерпеливые заявки; управ\-ле\-ние очередью}
  
\DOI{10.14357/19922264240206}{JZHAKU}
  
\vspace*{6pt}


\vskip 10pt plus 9pt minus 6pt

\thispagestyle{headings}

\begin{multicols}{2}

\label{st\stat}

  
\section{Введение}

\vspace*{-6pt}


  Настоящая работа служит продолжением исследований, посвященных 
проблеме оптимизации порогового управления очередью в~СМО с~учетом 
стоимостных потерь из-за отклонения и~задержек заявок, а также затрат на 
техническое обслуживание системы. Суть порогового управления очередью 
заключается в~том, что для длины очереди задается одно или несколько 
пороговых значений, по достижении каждого из которых принимается 
соответствующее решение по сбросу нагрузки из очереди с~\mbox{целью} 
повышения эффективности работы сис\-те\-мы~[1]. 
  
  Ниже будем рассматривать оптимизационную задачу управления очередью для 
простейшей СМО, у~которой ограничено время 
пребывания заявки в~очереди или на приборе. Заявка покидает сис\-те\-му 
необслуженной, если время ожидания в~очереди или на приборе превышает 
некоторую случайную\linebreak величину с~заданным средним значением. В~прос\-тей\-шей 
модели системы такого типа предполагают, что заявки покидают очередь через 
случайные интервалы времени, распределенные по \mbox{показательному} закону, т.\,е.\ 
возникает поток уходящих из очереди с~постоянной интенсивностью заявок. 
Таким образом, каж\-дая заявка, на\-хо\-дя\-ща\-яся в~очереди или на приборе, может 
покинуть систему, не дождавшись обслуживания, через случайный интервал 
времени, распределенный по показательному закону. Заявки в~этом случае 
называют <<нетерпеливыми>>, а~СМО~--- сис\-те\-мой с~<<нетерпеливыми>> 
заявками. Такая СМО имеет четыре потока, влия\-ющих на со\-сто\-яние сис\-те\-мы: 
входной поток заявок, поток обслуженных заявок, поток заявок, по\-ки\-да\-ющих 
очередь, не дождавшись начала обслуживания, и~поток уходящих с~приборов 
заявок, не дождавшихся завершения обслуживания. Так как поток уходящих 
заявок пуассоновский, то процесс, протекающий в~сис\-те\-ме под влиянием такого 
потока, будет марковским. 
  
  С увеличением порога длины очереди, с~одной стороны, увеличивается поток 
заявок~--- потенциальных плательщиков за обслуживание, с~другой~--- 
увеличиваются потери сис\-те\-мы (из-за увеличения задержек заявок, ухода 
<<нетерпеливых>> заявок, затрат сис\-те\-мы на хранение и~обслуживание 
заявок). Возникает задача поиска значения порога длины очереди, 
максимизирующего доход сис\-темы.
  
  Результаты теоретических и~экспериментальных исследований по 
рассматриваемой в~данной \mbox{статье} проб\-ле\-ме, изложенные в~ранее 
опубликованных работах, получены для задачи оптимизации порогового 
управления очередью в~одноканальных и~многоканальных СМО с~терпеливыми 
заявками (заявки не покидают сис\-те\-му до завершения обслуживания) (см., 
например,~[2--8]). При исследовании СМО с~управ\-ля\-емы\-ми очередями 
методами математического моделирования, как правило, требуется 
предварительно решить подзадачу расчета  
ве\-ро\-ят\-ност\-но-вре\-мен\-н$\acute{\mbox{ы}}$х характеристик (показателей) исследуемой  
сис\-те\-мы~[9--13] и~использовать данную расчетную модель для разработки 
и~исследования алгоритма управ\-ле\-ния очередью (очередями), что приводит 
к~математической модели с~более сложными функциональными 
взаимозависимостями параметров системы по сравнению с~расчетной моделью. 
В~научных пуб\-ли\-ка\-ци\-ях, посвященных исследованию сис\-тем 
с~нетерпеливыми заявками, отсутствуют результаты по оптимизации  
управ\-ле\-ния очередью, в~основном в~них рас\-смот\-ре\-ны задачи по расчету  
ве\-ро\-ят\-ност\-но-вре\-мен\-н$\acute{\mbox{ы}}$х характеристик и~оптимизации структуры таких 
сис\-тем~\cite{9-ag, 10-ag}.
  
  Ниже приводим метод и~результаты теоретического исследования 
однопорогового управления очередью системы $M/M/s$ с~<<нетерпеливыми>> 
заявками при стоимостном критерии оптимальности. 
  
\section{Постановка задачи }

  Рассматривается многоканальная СМО $M/M/s$ с~управляемой очередью, 
в~которой заявки, не дожидаясь завершения обслуживания, могут покинуть 
систему по истечении некоторого времени пребывания в~очереди или на приборе. 
Предполагается, что время, через которое заявка покидает сис\-те\-му,~--- 
показательно распределенная случайная величина, при этом параметр 
распределения равен~$\alpha_i$ ($\alpha_1\leq \alpha_2\leq\cdots$), если заявка  
\mbox{$i$-я} в~очереди, а~если на приборе, то параметр равен~$\beta$. 
Поступившая извне заявка допускается в~систему, если длина очереди в~системе 
меньше, чем $h\hm\geq 0$~--- некоторая заданная величина (пороговое значение), 
иначе отклоняется и~теряется. Допущенная в~систему заявка занимает любой из 
свободных приборов, если такой есть, иначе становится в~конец очереди. Будем 
считать, что заявки обслуживаются в~порядке поступления. Отметим, что 
поведение такой сис\-те\-мы описывается цепью Маркова, в~которой 
состоянием считается число заявок в~системе~\cite{9-ag}.
  
   Введем обозначения:
  \begin{description}
  \item[\,]  $\lambda$~--- интенсивность входного потока;
  \item[\,]  $\mu$~--- интенсивность обслуживания заявки на приборе;
  \item[\,]  $s$~--- число приборов в~системе;
  \item[\,]  $h+s$~--- объем накопителя;
   \item[\,]  $C_0$~--- плата заявки, принятой в~накопитель сис\-темы; 
  \item[\,]  $C_1$~--- стоимость потерь из-за отклонения заявки на входе системы;
  \item[\,]  $C_2$~--- стоимость потерь из-за ухода $i$-й заявки, находящейся в~очереди;
  \item[\,]  $C_3$~--- стоимость потерь из-за ухода с~прибора заявки, не 
дождавшейся завершения обслуживания;
  \item[\,]  $\pi_i^{(h)}$~--- стационарная вероятность состояния $i$ сис\-те\-мы 
при пороговом значении~$h$;
%  \item[\,]  $\overline{L}^{(h)} =s- \sum\nolimits^S_{i=1} (s-i)\pi_i^{(h)}$~--- среднее 
%значение длины очереди при пороге~$h$;
  \item[\,]  $\overline{S}^{(h)} =s-\sum\nolimits^S_{i=1} (s-i)\pi_i^{(h)}$~--- среднее 
чис\-ло занятых приборов;
  \item[\,]  $Q(h)$~--- доход сис\-те\-мы в~единицу времени при пороговом 
значении~$h$.
  \end{description}
  
  В качестве целевой функции задачи оптимизации порогового управления 
рассматривается предельный средний доход системы в~единицу времени, 
вычисляемый по формуле:
  \begin{multline}
  Q(h)=\lambda C_0\left( 1-\pi^{(h)}_{s+h}\right) -\lambda C_1\pi^{(h)}_{s+h}-{}\\
  {}- d\left( \overline{L}^{(h)}\right) -\beta C_3 \overline{S}^{(h)}.
  \label{e1-ag}
  \end{multline}
  Здесь $\lambda C_0(1\hm- \pi^{(h)}_{s+h})$~--- средняя суммарная плата 
заявок, принимаемых в~накопитель в~единицу времени; $\lambda 
C_1\pi^{(h)}_{s+h}$~--- средние потери в~единицу времени  
из-за отклонения заявок; $ d( \overline{L}^{(h)})$~--- средние потери 
в~единицу времени из-за ухода заявок из очереди:
$$
 d\left(  \overline{L}^{(h)}\right)= C_2\sum\limits^h_{i=1} \sum\limits^i_{j=1} 
\alpha_j \pi_{i+S}^{(h)};
$$
 $\beta C_3\overline{S}^{(h)}$~--- средние потери в~единицу 
времени из-за ухода с~приборов заявок, не дождавшихся завершения 
обслуживания.
  
  Ставится задача оптимизации порогового значения длины очереди, т. \,е.\ 
математическая задача вида
  \begin{equation}
  h^*= \argmax\limits_{0\leq h} Q(h)\,.
  \label{e2-ag}
  \end{equation}

\section{Метод решения и~результаты}

  Для стационарных вероятностей состояний описанной в~предыдущем разделе 
СМО справедливы равенства~\cite{9-ag}:
  \begin{equation}
  \left.
  \begin{array}{rl}
  \pi_l^{(h)} &=\fr{\rho^l}{l!(1+\gamma)^i}\,\pi_0^{(h)}\ \mbox{при}\ l\in \overline{1,s}\,,
  \\[6pt]
  \pi^{(h)}_{s+l} &=\fr{\rho^s}{s!(1+\gamma)^3} \prod\limits^l_{j=1} 
\fr{\rho}{s(1+\gamma) + j\theta_j}\,\pi_0^{(h)}\\[6pt]
 &\hspace*{27mm}\mbox{при}\ l\in \overline{1, h}\,,
 \end{array}
 \right\}
   \label{e3-ag}
  \end{equation}
  где
  
  \vspace*{-6pt}
  
  \noindent
\begin{multline*}
  \pi_0^{(h)} =\left[ 
  \vphantom{\prod\limits^l_{j=1}}
  1+\sum\limits_{m=1}^s \fr{\rho^m}{m!(1+\gamma)^m} + {}\right.\\
\left.   {}+
\fr{\rho^s}{s!(1+\gamma)^s} \sum\limits^h_{l=1} \prod\limits^l_{j=1} 
\fr{\rho}{s(1+\gamma)+j\theta_j}\right]^{-1};
\end{multline*}
  $$
  \rho=\fr{\lambda}{\mu}\,;\quad \gamma= \fr{\beta}{\mu}\,;\quad \theta_j= \fr{\alpha_j}{\mu}\,.
  $$
  
  Покажем, что для стационарных вероятностей справедливы соотношения
  \begin{align}
  \pi_l^{(h+1)} &= \left(1-\pi^{(h+1)}_{s+h+1}\right)\pi_l^{(h)},\ l=\overline{0, s+h}\,;
  \label{e4-ag}\\
  \pi^{(h+1)}_{s+h+1} &={}\notag\\
  &\hspace*{-10mm}{}= \left( 1-\pi_{s+h+1}^{(h+1)} \right) \pi^{(h)}_{s+h}  
\fr{\rho}{s(1+\gamma)+(h+1)\theta_{h+1}}\,.
  \label{e5-ag}
  \end{align}
  
  Из~(\ref{e3-ag}) при $l\hm=\overline{1,  h}$ следует 
  \begin{multline*}
  \pi^{(h)}_{s+l} -\pi_{s+l}^{(h+1)} ={}\\
  {}=\fr{\rho^s}{s!(1+\gamma)^s} 
\prod\limits^l_{j=1} \fr{\rho}{s(1+\gamma)+j\theta_j} \left( \pi_0^{(h)} -
\pi_0^{(h+1)}\right) ={}\\
  {}= \left( \fr{\rho^s}{s!(1+\gamma)^s}\right)^2 \prod\limits^l_{j=1} 
\fr{\rho}{s(1+\gamma)+j\theta_j}\times{}\\
{}\times  \prod\limits^{h+1}_{j=1} 
\fr{\rho}{s(1+\gamma)+j\theta_j}\,\pi_0^{(h)} \pi_0^{(h+1)}= \pi^{(h)}_{s+l} \pi_{s+h+1}^{(h+1)}\,.
  \end{multline*}
  
  Точно так же, использовав~(\ref{e3-ag}) при $l\hm= \overline{0, s}$, получаем 
равенство 
$$
\pi_l^{(h)} - \pi_l^{(h+1)} = \pi_l^{(h)} \pi^{(h+1)}_{s+h+1}.
$$ 
Следовательно, равенства~(\ref{e4-ag}) справедливы. Аналогично, 
использовав~(\ref{e3-ag}) и~(\ref{e4-ag}), находим
  \begin{multline*}
  \pi^{(h+1)}_{s+h+1} =\fr{\rho^s}{s!(1+\gamma)^2} \prod\limits_{j=1}^{h+1} 
\fr{\rho}{s(1+\gamma)+j\theta_j}\,\pi_0^{(h+1)}={}\\
  {}= 
  \fr{\rho^s}{s!(1+\gamma)^s}\,
  \fr{\rho}{s(1+\gamma)+(h+1)\theta_{h+1}} \times{}\\
  {}\times \prod\limits^h_{j=1} \fr{\rho} 
{s(1+\gamma)+j\theta_j} \left( 1-\pi^{(h+1)}_{s+h+1}\right) \pi_0^{(h)},
  \end{multline*}
откуда следует~(\ref{e5-ag}).

  Покажем, что имеет место равенство
  \begin{equation}
  Q(h)-Q(h+1) =\pi_{s+h+1}^{(h+1)}\left[ Q(h)-G(h)\right],
  \label{e6-ag}
  \end{equation}
  
\vspace*{-12pt}
  
  \columnbreak 


\noindent
где



\noindent
\begin{multline}
G(h)=\left( C_0+C_1\right) (\mu+\beta)s +\left( C_0+C_1\right) \alpha_{h+1}(h+1) -
{}\\
{}- \sum\limits_{j=1}^{h+1} \alpha_j C_2 - C_1\lambda -C_3\beta s\,.
\label{e7-ag}
\end{multline}

\vspace*{-6pt}
  
  Использовав~(\ref{e1-ag})--(\ref{e7-ag}), получим:
  
  \vspace*{-6pt}
  
  \noindent
  \begin{multline*}
  Q(h)-Q(h+1) =\lambda C_0\left( 1-\pi^{(h)}_{s+h}\right) -{}\\
  {}- \lambda C_1 
\pi^{(h)}_{s+h} -d\left( \overline{L}^{(h)}\right) -\beta C_3 \overline{S}^{(h)}-{}\\
  {}-
  \lambda C_0\left( 1- \pi^{(h+1)}_{s+h+1}\right) +\lambda C_1 
\pi_{s+h+1}^{(h+1)} +d\left( \overline{L}^{(h+1)}\right) +{}\\
{}+\beta C_3  \overline{S}^{(h+1)}=
  -\lambda C_0\pi^{(h)}_{s+h} -\lambda C_1 \pi^{(h)}_{s+h} -d\left( 
\overline{L}^{(h)}\right) -{}\\
{}- \beta C_3 \overline{S}^{(h)} +\lambda  C_0\pi_{s+h+1}^{(h+1)}+\lambda C_1 \pi^{(h+1)}_{s+h+1}+{}\\
{}+\left( 1-\pi_{s+h+1}^{(h+1)}\right) d\left( \overline{L}^{(h)}\right)+
 C_2\sum\limits_{j=1}^{h+1} \alpha_j \pi_{s+h+1}^{(h+1)} 
+{}\\
{}+\beta s C_3 \pi_{s+h+1}^{(h+1)} +
\beta C_3\left( 1-\pi_{s+h+1}^{(h+1)}\right) \overline{S}^{(h)} ={}\\
{}= -\lambda C_0 
\pi^{(h)}_{s+h} -\lambda C_1\pi^{(h)}_{s+h} +\lambda C_0 \pi^{(h+1)}_{s+h+1} 
+{}\\
  {}+ \lambda C_1\pi_{s+h+1}^{(h+1)} +C_2\sum\limits_{j=1}^{h+1} \alpha_j 
\pi^{(h+1)}_{s+h+1}+\beta s C_3 \pi_{s+h+1}^{(h+1)} -{}\\
{}- \pi_{s+h+1}^{(h+1)} d\left( \overline{L}^{(h)}\right)-
 \beta C_3\pi^{(h+1)}_{s+h+1} \overline{S}^{(h)} ={}\\
 {}=\pi_{s+h+1}^{(h+1)} \Bigg[ 
 %\vphantom{\fr{C_0+C_1}{\pi^{(h+1)}_{s+h+1}}}
\lambda C_0+\lambda C_1 +C_2\sum\limits_{j=1}^{h+1} \alpha_j -d\left( 
\overline{L}^{(h)}\right) - {}\\
  {}- C_3\beta \overline{S}^{(h)} +\beta s C_3 -\lambda 
\fr{C_0+C_1}{\pi^{(h+1)}_{s+h+1}}\,\pi^{(h)}_{s+h}
 %\vphantom{\fr{C_0+C_1}{\pi^{(h+1)}_{s+h+1}}}
 \Bigg]={}\\
  {}=  \pi^{(h+1)}_{s+h+1} \left[
 \vphantom{\fr{C_0+C_1}{\pi^{(h+1)}_{s+h+1}}}
 \lambda C_0 \left( 1-\pi^{(h)}_{s+h}\right) -
\lambda C_1\pi^{(h)}_{s+h} +\lambda C_1 +{}\right.\\
{}+\sum\limits^{h+1}_{j=1} \alpha_j C_2 -
d\left( \overline{L}^{(h)}\right)- C_3\beta\overline{S}^{(h)} +\lambda C_1\pi^{(h)}_{s+h} +{}\\
\left.{}+\lambda 
C_0\pi^{(h)}_{s+h} +\beta s C_3 -
\lambda\fr{C_0+C_1}{\pi^{(h+1)}_{s+h+1}}\,\pi^{(h)}_{s+h}\right]={}\\
  {}= \pi^{(h+1)}_{s+h+1} \left[ 
  Q(h) +\lambda C_1+ \sum\limits_{j=1}^{h+1} 
\alpha_j C_2 +\beta s C_3- {}\right.\\[-2pt]
\left.  {}-  (C_0+C_1)\lambda \fr{s(1+\gamma)+(h+1)\theta}{\rho} 
\vphantom{\sum\limits_{j=1}^{h+1}}
\right]={}\\[-2pt]
{}=  \pi^{(h+1)}_{s+h+1} \Bigg[ Q(h)+\lambda C_1+ C_2 \sum\limits_{j=1}^{h+1} 
\alpha_j +\beta s C_3 -{}\\[-2pt]
{}-(C_0+C_1) [ s(\mu+\beta)+\alpha_{h+1} (h+1)]\Bigg].
  \end{multline*} 
  
\begin{figure*}[b] %fig1
\vspace*{1pt}
\begin{minipage}[t]{80mm}
      \begin{center}
     \mbox{%
\epsfxsize=79mm 
\epsfbox{aga-1.eps}
}
\end{center}
\vspace*{-9pt}
\Caption{Зависимости функций $Q$~(\textit{1}) и~$G$~(\textit{2}) от порогового значения~$h$: $h^*$~--- 
оптимальное пороговое значение длины очереди}
\end{minipage}
%\end{figure*}
\hfill
%\begin{figure*} %fig2
\vspace*{1pt}
\begin{minipage}[t]{80mm}
      \begin{center}
     \mbox{%
\epsfxsize=79mm 
\epsfbox{aga-2.eps}
}
\end{center}
\vspace*{-9pt}
\Caption{Зависимости функций $Q$~(\textit{1}) и~$G$~(\textit{2}) от порогового значения~$h$}
\end{minipage}
\end{figure*}
  
 \vspace*{-6pt}
  
Значит, равенство~(\ref{e6-ag}) имеет место. 

\pagebreak

Так как верно равенство 
$$
\lambda \left(1 - \pi^{(h)}_{s+h}\right) \hm= \sum\limits^h_{i=1} \sum\limits^i_{j=1} 
\alpha_j \pi_i^{(h)} +(\beta+\mu) \overline{S}^{(h)}\,,
$$
то равенства для $Q(h)$ и~$G(h)$ в~(\ref{e1-ag}) и~(\ref{e7-ag}) эквивалентны 
равенствам
\begin{align*}
Q(h) &= \left[ \left( C_0+C_1\right)(\beta+\mu) -\beta C_3\right]\overline{S}^{(h)} 
+{}\\
&{}+\left( C_0 +C_1-C_2\right) \sum\limits^h_{i=1} \sum\limits^i_{j=1} \alpha_j 
\pi_i^{(h)} -C_1\lambda\,;\\
G(h) &= \left[ \left( C_0+C_1\right)(\beta+\mu) -\beta C_3\right]s +{}\\
&{}+\left( 
C_0+C_1\right) \alpha_{h+1} (h+1) -C_2\sum\limits_{j=1}^{h+1} \alpha_j -C_1\lambda\,.
\end{align*}
При $h=0$ последние равенства примут вид:
\begin{align*}
Q(0) &= \left[ \left( C_0+C_1\right((\beta+\mu)-\beta C_3\right] \overline{S}^{(0)} -
C_1\lambda\,;\\
G(0)&= \left[ \left( C_0+C_1\right) (\beta+\mu) -\beta C_3\right] s+{}\\
&\hspace*{18mm}{}+\left( C_0+C_1- C_2\right) \alpha_1 -C_1\lambda\,.
\end{align*}
  
  
  Далее всюду будем предполагать, что при условии $C_0\hm+ C_1\hm- C_2 
\hm<0$ выполняется и~условие $(C_2/(C_0\hm+C_1)-1) 
\alpha_{i+1}/(\alpha_{i+1}\hm-\alpha_i)\hm\geq i$ для всех $i\hm\geq 1$. Обратим 
внимание, что функция $G(h)$ возрастает по~$h$ при $C_0\hm+C_1\hm- C_2\hm 
>0$ и~не возрастает, когда $C_0\hm+C_1\hm- C_2\hm\leq 0$ и~$\alpha_i$ такие, 
что условие $(C_2/(C_0\hm+C_1)-1) \alpha_{i+1}/(\alpha_{i+1}\hm-\alpha_i)\hm\geq 
i$ для всех $i\hm\geq 1$.
  
  Воспользуемся теоремой~1 из работы~\cite{14-ag}. Нетрудно заметить (см.\ 
равенство~(\ref{e6-ag})), что функция $Q(h)$ при $C_0\hm+C_1\hm\leq C_2$ и~функция $-Q(h)$ при $C_0\hm+C_1\hm> C_2$ удовлетворяют условиям 
теоремы~1 из~\cite{14-ag}. Тогда из указанной теоремы непосредственно следует 
справедливость следующего утверж\-де\-ния. 
  
  \smallskip
  
  \noindent
  \textbf{Утверждение.} \textit{При выполнении предположения, введенного 
выше относительно параметров $C_0$, $C_1$, $C_2$ и~$\alpha_i$, $i\hm\geq 1$, 
решение задачи~$(2)$ обладает следующими свойствами}: 
  \begin{enumerate}[(1)]
\item \textit{если $C_0+C_1\hm\leq C_2$, то $Q(h)$~--- унимодальная функция 
(так как $G(h)$ не возрастает по~$h$ и~$G(0)\hm\leq Q(0)$, и~если $[(C_0\hm+ 
C_1)(\beta\hm+ \mu) \hm- \beta C_3] (s\hm- \overline{S}^{(0)} )\hm\leq 
(C_0\hm+C_1\hm-C_2)\alpha_1$, то $h^*\hm=0$, иначе существует $0\hm< h^*\hm< \infty$};
\item \textit{если $C_0+C_1\hm >C_2$ и~$[(C_0\hm+ C_1)(\beta\hm+ \mu) \hm- \beta 
C_3] (s\hm- \overline{S}^{(0)} ) \hm+ (C_0\hm+ C_1\hm- C_2)\alpha_1\hm>0$, то 
$h^*\hm=\infty$ и~при этом $Q(h)$ монотонно возрастает по~$h$ $($так как 
$G(h)$ возрастает по~$h$ и~$G(0)\hm >Q(0))$};
\item \textit{если $C_0+C_1\hm> C_2$ и~$[(C_0\hm+ C_1)(\beta\hm+ \mu) \hm- \beta 
C_3] (s\hm- \overline{S}^{(0)} ) \hm+ (C_0\hm+ C_1\hm- C_2)\alpha_1\hm\leq0$, то 
функция $-Q(h)$ унимодальная (так как удовлетворяет условиям теоремы~$1$ 
из}~\cite{14-ag}) \textit{и~при этом}
$$
h^*=\begin{cases}
0\,, & \mbox{\textit{если}\ \ } Q(\infty) \leq Q(0);\\
\infty\,, & \mbox{\textit{если}\ \ } Q(\infty) > Q(0)
\end{cases}
$$
\textit{$($так как $G(h)$ возрастает по~$h$ 
и}~$G(0)\hm\leq Q(0)$$)$. 
\end{enumerate}

\smallskip

 На рис.~1 и~2 проиллюстрировано поведение функций $Q(h)$ и~$G(h)$ для двух 
наборов значений па\-ра\-мет\-ров рассматриваемой СМО: 
\begin{enumerate}[(1)]
\item рис.~1: $\lambda=8$; $\mu=2$; $\alpha_i=0{,}5$; $i\hm= \overline{1, h}$; 
$\beta\hm= 0{,}25$; $C_0\hm= 20$; $C_1\hm= 5$; $C_2\hm= 40$; $C_3\hm=10$; 
\item рис.~2:
$\lambda=8$; $\mu=1$; $\alpha_i=0{,}25$; $i\hm= \overline{1, h}$; $\beta\hm= 0{,}125$; 
$C_0\hm= 20$; $C_1\hm= 3$; $C_2\hm= 10$; $C_3\hm=15$. 
\end{enumerate}

Заметим, что в~случае, изображенном на рис.~1, целевая функция достигает 
максимума при пороговом значении $h^*\hm=18$, что согласуется 
с~утверж\-де\-ни\-ем пункта~1, а~в~случае рис.~2 оптимальное значение порога 
$h^*\hm=\infty$, что соответствует пункту~2. 



\section{Заключение}

  Практическим результатом проведенных выше исследований стал 
сле\-ду\-ющий прос\-той алгоритм оптимизации однопорогового управ\-ле\-ния 
оче\-редью для рассмотренной выше модели СМО при выполнении условий 
утверж\-де\-ния относительно па\-ра\-мет\-ров~$C_0$, $C_1$, $C_2$ и~$\alpha_i$, 
$i\hm\geq 1$.
  \begin{enumerate}[1.]
\item Если выполняется условие $C_0\hm+ C_1\hm\leq C_2$, то
\begin{enumerate}[(1)]
\item  положить $h\hm=0$;
\item  до тех пор пока выполняется неравенство $Q(h+1)\hm> Q(h)$, полагать 
$h\hm= h+1$;
\item  положить $h^*=h$.
\end{enumerate}
\item Если выполняются неравенства $C_0\hm+C_1\hm >C_2$ и~$[(C_0\hm+ 
C_1)(\beta\hm+ \mu) \hm- \beta C_3] (s\hm- \overline{S}^{(0)} ) \hm+ (C_0\hm+ 
C_1\hm- C_2)\alpha_1\hm>0$, то положить $h^*\hm= \infty$. 
\item Если $C_0\hm+C_1\hm> C_2$ и~$[(C_0\hm+ C_1)(\beta\hm+ \mu) \hm- \beta 
C_3] (s\hm- \overline{S}^{(0)} ) \hm+ (C_0\hm+ C_1\hm- C_2)\alpha_1\hm\leq 0$, то 
положить 
$$
h^*= \begin{cases}
0\,, &\mbox{если\ } Q(\infty)\leq Q(0);\\
\infty &\mbox{иначе.}
\end{cases}
$$
\end{enumerate}
  
  Обратим внимание, что при выполнении условий третьего пункта алгоритма 
справедливы неравенства $Q(0)\hm\leq 0$ и~$G(0)\hm\leq Q(0)$ и~на отрезке 
$[0,h^0]$, где~$h^0$~--- максимальное значение, такое что $Q(h^0)\hm\geq 
G(h^0)$, функция~$Q(h)$ не возрастает, а~при $h\hm\in [h^0,\infty)$ возрастает 
(так как в~случае пункта~3 функция $-Q(h)$ унимодальная). Следовательно, если 
$C_0\hm+C_1\hm\geq C_2$ и~выполняется условие $Q(0)\hm\leq Q(1)$, то 
$h^*\hm=\infty$.

{\small\frenchspacing
 {\baselineskip=11.5pt
 %\addcontentsline{toc}{section}{References}
 \begin{thebibliography}{99}
\bibitem{1-ag}
\Au{Floyd S., Jacobson~V.} Random early detection gateways for congestion avoidance~// 
IEEE ACM T. Network., 1993. Vol.~1. P. 397--413. doi: 10.1109/90.251892.

\bibitem{2-ag}
\Au{Коновалов М.\,Г.} Об одной задаче оптимального управ\-ле\-ния нагрузкой на сервер~// 
Информатика и~её применения, 2013. Т.~7. Вып.~4. С.~34--43. doi: 10.14357/19922264130404. EDN: 
RRROXB.


\bibitem{4-ag} %3
\Au{Konovalov M.\,G., Razumchik~R.\,V.} Comparison of two active queue management schemes 
through the $M/D/1/N$ queue~// Информатика и~её применения, 2018. Т.~12. Вып.~4. С.~9--15. doi: 
10.14357/19922264180402. EDN: VOGJOZ.

\bibitem{3-ag} %4
\Au{Агаларов Я.\,М.} Оптимизация объема буферной памяти узла коммутации при схеме 
полного разделения памяти~// Информатика и~её применения, 2018. Т.~12. Вып.~4. С.~25--32.
doi: 10.14357/19922264180404. EDN: YQHHGP.


\bibitem{5-ag}
\Au{Агаларов Я.\,М., Ушаков~В.\,Г.} Об унимодальности функции дохода системы массового 
обслуживания типа $G/M/s$ с~управ\-ля\-емой очередью~// Информатика и~её применения, 
2019. Т.~13. Вып.~1. С.~55--61. doi: 10.14357/19922264190108. EDN: HYAODW.
\bibitem{6-ag}
\Au{Коновалов М.\,Г., Разумчик~Р.\,В.} Комплексное управ\-ле\-ние в~одном классе систем 
с~параллельным обслуживанием~// Информатика и~её применения, 2019. Т.~13. Вып.~4.  
С.~54--59. doi: 10.14357/19922264190409. EDN: REESRH.
\bibitem{7-ag}
\Au{Агаларов Я.\,М.} Об оптимизации работы резервного прибора в~многолинейной системе 
массового обслуживания~// Информатика и~её применения, 2023. Т.~17. Вып.~1. С.~89--95.  doi: 
10.14357/19922264230112. EDN: FCYDUT.
\bibitem{8-ag}
\Au{Агаларов Я.\, М.} Оптимизация схемы распределения буферной памяти узла пакетной 
коммутации~// Информатика и~её применения, 2023. Т.~17. Вып.~3. С.~39--48. doi: 
10.14357/19922264230306. EDN: \mbox{ХQLXCKV}.

\bibitem{9-ag}
\Au{Кирпичников Ф.\,П., Флакс~Д.\,Б., Галямова~К.\,Н.} Средняя длина очереди в~сис\-те\-ме 
массового обслуживания с~ограниченным средним временем пребывания заявки в~сис\-те\-ме~// 
Вестник Технологического университета, 2017. Т.~20. №\,2. С.~81--84. EDN: 
XVFSTN.

\bibitem{10-ag}
\Au{Савинов Ю.\,Г., Табакова~Е.\,Д., Сафиуллов~И.\,Д.} Оптимизация в~СМО с~нетерпеливыми 
заявками~// Ученые записки УлГУ. Сер. Математика и~информационные технологии, 2019. 
№\,1. С.~92--98. EDN: OWOZYR.

\bibitem{11-ag}
\Au{Мейханаджян Л.\,А., Разумчик~Р.\,В.} Система массового обслуживания 
$\mathrm{Geo}/G/1/\infty$ с~инверсионным порядком обслуживания и~ресамплингом в~дискретном 
времени~// Информатика и~её применения, 2019. Т.~13. Вып.~4. С.~60--67. doi: 
10.14357/19922264190410. EDN: LNIHGC.

\bibitem{12-ag}
\Au{Милованова Т.\,А., Разумчик~Р.\,В.} Однолинейная система массового обслуживания 
с~инверсионным порядком обслуживания с~вероятностным приоритетом, групповым 
пуассоновским потоком и~фоновыми заявками~// Информатика и~её применения, 2020. Т.~14. 
Вып.~3. С.~26--34. doi: 10.14357/19922264200304. EDN: NOMSAM.
\bibitem{13-ag}
\Au{Берговин А.\,К., Ушаков~В.\,Г.} Исследование сис\-тем обслуживания со смешанными 
приоритетами~// Информатика и~её применения, 2023. Т.~17. Вып.~2. С.~57--61. doi: 
10.14357/19922264230208. EDN: \mbox{JULPWS}.



\bibitem{14-ag}
\Au{Агаларов Я.\,М.} Признак унимодальности целочисленной функции одной переменной~// 
Обозрение прикладной и~промышленной математики, 2019. Т.~26. Вып.~1. С.~65--66.


\end{thebibliography}

 }
 }

\end{multicols}

\vspace*{-6pt}

\hfill{\small\textit{Поступила в~редакцию 23.02.24}}

%\vspace*{10pt}

%\pagebreak

\newpage

\vspace*{-28pt}

%\hrule

%\vspace*{2pt}

%\hrule



\def\tit{ON SINGLE-THRESHOLD QUEUE MANAGEMENT\\ IN~A~QUEUING SYSTEM 
WITH~IMPATIENT CUSTOMERS}


\def\titkol{On single-threshold queue management in~a~queuing system 
with~impatient customers}


\def\aut{Ya.\,M.~Agalarov}

\def\autkol{Ya.\,M.~Agalarov}

\titel{\tit}{\aut}{\autkol}{\titkol}

\vspace*{-8pt}


\noindent
Federal Research Center ``Computer Science and Control'' of the Russian Academy of 
Sciences, 44-2~Vavilov Str., Moscow 119333, Russian Federation

\def\leftfootline{\small{\textbf{\thepage}
\hfill INFORMATIKA I EE PRIMENENIYA~--- INFORMATICS AND
APPLICATIONS\ \ \ 2024\ \ \ volume~18\ \ \ issue\ 2}
}%
 \def\rightfootline{\small{INFORMATIKA I EE PRIMENENIYA~---
INFORMATICS AND APPLICATIONS\ \ \ 2024\ \ \ volume~18\ \ \ issue\ 2
\hfill \textbf{\thepage}}}

\vspace*{4pt}




\Abste{The results of a theoretical study of a~managed queuing system of $M/M/k$ type 
with impatient customers and single-threshold queue management are presented. The task of optimizing single-threshold 
queue management is set, the essence of which is to calculate for the queue length a~certain threshold value that 
maximizes a given objective function. In the system under study, a~customer leaves the system unattended if 
the waiting time in the queue (or the service time on the device) exceeds a~certain time interval of random 
length distributed according to an exponential law with a~given parameter. A~cost function is used as an 
indicator of the effectiveness of queue management (objective function) which takes into account the losses per 
unit of time due to system technical maintenance, rejection of customers at the entrance of the system, and 
leaving of customers until the end of the service. A~method for solving the problem of maximizing the cost 
objective function on a~set of single-threshold queue controls and an algorithm for guaranteed calculation of the 
optimal threshold are proposed.}

\KWE{queuing system; impatient customers; queue management}

\DOI{10.14357/19922264240206}{JZHAKU}

%\vspace*{-12pt}

%\Ack

%\vspace*{-3pt}


 %    \noindent
  


  \begin{multicols}{2}

\renewcommand{\bibname}{\protect\rmfamily References}
%\renewcommand{\bibname}{\large\protect\rm References}

{\small\frenchspacing
 {%\baselineskip=10.8pt
 \addcontentsline{toc}{section}{References}
 \begin{thebibliography}{99} 
\bibitem{1-ag-1}
\Aue{Floyd, S., and V.~Jacobson.} 1993. Random early detection gateways for congestion avoidance. 
\textit{IEEE ACM T. Network.} 1:397--413. doi: 10.1109/90.251892.
\bibitem{2-ag-1}
\Aue{Konovalov, M.\,G.} 2013. Ob odnoy zadache optimal'nogo upravleniya nagruzkoy na server [About one 
task of overload control]. \textit{Informatika i~ee Primeneniya~--- Inform. Appl.} 7(4):34--43. doi: 
10.14357/19922264130404. EDN: RRROXB.

\bibitem{4-ag-1}
\Aue{Konovalov, M., and R.~Razumchik}. 2018. Comparison of two active queue management schemes 
through the $M/D/1/N$ queue. \textit{Informatika i~ee Primeneniya~--- Inform. Appl.} 12(4):9--15. doi: 
10.14357/19922264180402. EDN: VOGJOZ.

\bibitem{3-ag-1}
\Aue{Agalarov, Ya.\,M.}  2018. Optimizatsiya ob''ema bufernoy pamyati uzla kommutatsii pri skheme polnogo 
razdeleniya pamyati [Optimization of buffer memory size of switching node in mode of full memory sharing]. 
\textit{Informatika i~ee Primeneniya~--- Inform. Appl.} 12(4):25--32. doi: 10.14357/19922264180404. EDN: 
YQHHGP.

\bibitem{5-ag-1}
\Aue{Agalarov, Ya.\,M., and V.\,G.~Ushakov.} 2019. Ob unimodal'nosti funktsii dokhoda sistemy massovogo 
obsluzhivaniya tipa $G/M/s$ s~upravlyaemoy ochered'yu [On the unimodality of the income function of a~type 
$G/M/s$ queueing system with controlled queue]. \textit{Informatika i~ee Primeneniya~--- Inform. Appl.} 
13(1):55--61. doi: 10.14357/19922264190108. EDN: HYAODW.
\bibitem{6-ag-1}
\Aue{Konovalov, M.\,G., and R.\,V.~Razumchik.} 2019. Komp\-leks\-noe upravlenie v~odnom klasse sistem 
s~parallel'nym obsluzhivaniem [Mixed policies for online job allocation in one class of systems with parallel 
service]. \textit{Informatika i~ee Primeneniya~--- Inform. Appl.} 13(4):54--59. doi: 10.14357/19922264190409. 
EDN: REESRH.
\bibitem{7-ag-1}
\Aue{Agalarov, Ya.\,M.} 2023. Ob optimizatsii raboty rezervnogo pribora v~mnogolineynoy sisteme 
massovogo obsluzhivaniya [Optimization of a~queue-length dependent additional server in the multiserver 
queue]. \textit{Informatika i~ee Primeneniya~--- Inform. Appl.} 17(1):89--95. doi: 10.14357/19922264230112. 
EDN: FCYDUT.
\bibitem{8-ag-1}
\Aue{Agalarov, Ya.\,M.} 2023. Optimizatsiya skhemy raspredeleniya bufernoy pamyati uzla paketnoy 
kommutatsii [Optimization of the buffer memory allocation scheme of the packet switching node]. 
\textit{Informatika i~ee Primeneniya~--- Inform. Appl.}  17(3):39--48. doi: 10.14357/19922264230306. EDN: 
QLXCKV.
\bibitem{9-ag-1}
\Aue{Kirpichnikov, F.\,P., D.\,B.~Flaks, and K.\,N.~Galyamova.} 2017. Srednyaya dlina ocheredi v~sisteme 
massovogo obsluzhivaniya s~ogranichennym srednim vremenem prebyvaniya zayavki v~sisteme [The average 
queue length in a~queuing system with a~limited average time for the request to stay in the system]. 
\textit{Vestnik Tekhnologicheskogo universiteta} [Bulletin of  Technological University] 
20(2):81--84. EDN: XVFSTN.
\bibitem{10-ag-1}
\Aue{Savinov, Yu.\,G., E.\,D.~Tabakova, and I.\,D.~Safiullov.} 2019. Optimizatsiya v~SMO s~neterpelivymi 
zayavkami [Optimization in the queuing system with impatient customers]. \textit{Uchenyye zapiski UlGU. Ser. 
Matematika i~informatsionnye tekhnologii} [Scientific Notes of UlSU. Ser. Mathematics and Information Technology] 1:92--98. EDN: \mbox{OWOZYR}.

\bibitem{11-ag-1}
\Aue{Meykhanadzhyan, L.\,A., and R.\,V.~Razumchik.} 2019. Sistema massovogo obsluzhivaniya 
$\mathrm{Geo}/G/1/\infty$ s~inversionnym poryadkom obsluzhivaniya i~resamplingom v~diskretnom vremeni 
[Discrete-time $\mathrm{Geo}/G/1/\infty$ LIFO queue with resampling policy]. \textit{Informatika i~ee Primeneniya~--- Inform. 
Appl.} 13(4):60--67. doi: 10.14357/ 19922264190410. EDN: LNIHGC.
\bibitem{12-ag-1}
\Aue{Milovanova, T.\,A., and R.\,V.~Razumchik.} 2020. Od\-no\-li\-ney\-naya sistema massovogo obsluzhivaniya 
s~in\-ver\-si\-on\-nym poryadkom obsluzhivaniya s~veroyatnostnym prioritetom, gruppovym puassonovskim 
\mbox{potokom} i~fonovymi zayavkami [A~single-server queueing system with \mbox{LIFO} service, probabilistic priority, 
batch Poisson arrivals, and background customers]. \textit{Informatika i~ee Primeneniya~--- Inform. Appl.} 
14(3):26--34. doi: 10.14357/ 19922264200304. EDN: NOMSAM.
\bibitem{13-ag-1}
\Aue{Bergovin, A.\,K., and V.\,G.~Ushakov.} 2023. Issledovanie sistem obsluzhivaniya so smeshannymi 
prioritetami [Analysis of the queueing systems with mixed priorities]. \textit{Informatika i~ee Primeneniya~--- 
Inform. Appl.} 17(2):57--61. doi: 10.14357/19922264230208. EDN: \mbox{JULPWS}.
{\looseness=1

}
\bibitem{14-ag-1}
\Aue{Agalarov, Ya.\,M.} 2019. Priznak unimodal'nosti tselochislennoy funktsii odnoy peremennoy [A~sign of 
unimodality of an integer function of one variable]. \textit{Obozrenie prikladnoy i~promyshlennoy matematiki} 
[Surveys Applied and Industrial Mathematics] 26(1):65--66.

\end{thebibliography}

 }
 }

\end{multicols}

\vspace*{-6pt}

\hfill{\small\textit{Received February 23, 2024}} 

\vspace*{-18pt}


\Contrl

\vspace*{-3pt}

\noindent
\textbf{Agalarov Yaver M.} (b.\ 1952)~--- Candidate of Science (PhD) in technology, associate professor, 
leading scientist, Federal Research Center ``Computer Science and Control'' of the Russian Academy of 
Sciences, 44-2~Vavilov Str., Moscow 119333, Russian Federation; \mbox{agglar@yandex.ru}




\label{end\stat}

\renewcommand{\bibname}{\protect\rm Литература}  %7
\def\stat{grusho}

\def\tit{МОДЕЛЬ СЛУЧАЙНЫХ ГРАФОВ ДЛЯ~ОПИСАНИЯ~ВЗАИМОДЕЙСТВИЙ В СЕТИ$^*$}

\def\titkol{Модель случайных графов для описания взаимодействий в сети}

\def\autkol{А.\,А.~Грушо, Е.\,Е.~Тимонина}

\def\aut{А.\,А.~Грушо$^1$, Е.\,Е.~Тимонина$^2$}

\titel{\tit}{\aut}{\autkol}{\titkol}

{\renewcommand{\thefootnote}{\fnsymbol{footnote}}\footnotetext[1]
{Работа выполнена при поддержке РФФИ (проекты №\,10-01-00480, №\,11-07-00112).}}

\renewcommand{\thefootnote}{\arabic{footnote}}
\footnotetext[1]{Институт проблем информатики Российской академии наук; Московский государственный 
университет им.\ М.\,В.~Ломоносова, факультет вычислительной математики и кибернетики, 
grusho@yandex.ru}
\footnotetext[2]{Институт проблем информатики Российской академии наук, eltimon@yandex.ru}

\Abst{Рассматривается новый класс случайных графов, призванный 
моделировать функционирование сети во времени. Предполагается, что наблюдения за 
сетью ведутся с помощью <<оконного>> метода. С~целью выявления аномалий 
исследуется нормальное поведение степеней, которые можно наблюдать в <<окнах>> 
рассматриваемой модели. Исследована асимптотика максимальной степени вершин в 
графе, который порожден <<окном>> данного размера.}

\KW{случайные графы; моделирование глобальных сетей; информационная 
безопасность; аномальное поведение}

\vskip 14pt plus 9pt minus 6pt

      \thispagestyle{headings}

      \begin{multicols}{2}

            \label{st\stat}

\section{Введение}

     Возможны различные способы распространения информации в сети. Наиболее 
известным способом является обращение к информационным ресурсам, выложенным на 
тематических сайтах. Поиск таких сайтов по нужной тематике~--- задача 
     ин\-тер\-нет-по\-иско\-ви\-ков. Другое направление~---\linebreak распространение информации 
по электронной поч\-те или через налаженные и специальным образом формируемые связи. 
К~таким способам относятся спам и управление бот-се\-тями. 
     
     Формирование собрания единомышленников или предупреждение о каком-либо 
событии могут передаваться любым из перечисленных выше методов. В~связи с этим 
представляет интерес моделирование процессов последовательной передачи информации. 
Будем рассматривать второй способ передачи информации. 
     
     В предположении о дискретности времени будем описывать состояние связей хостов 
между\linebreak собой неориентированным случайным графом. Реб\-рам графа отвечают логические 
связи хостов. Вмес\-те с тем последовательности связей вершин не обязательно связаны с 
распространением инфор\-мации в рамках некоторой корпорации. В~этом \mbox{случае} 
последовательности связей формируются случайно и независимо друг от друга. 
Выявление корпоратив\-ных связей в сети или центров управления связано с анализом 
случайных связей в рамках большого случайного графа сети. 
     
     Будем рассматривать изменения в графе сети с течением времени с помощью 
<<скользящего окна>>. Граф, который получается при таком рассмотрении, формируется 
фиксацией логических связей, захватываемых <<окном>> и отображенных на одном 
графе. Такой граф получается наложением всех графов в моменты времени, 
принадлежащие <<окну>>, и объединением параллельных ребер.
     
     В работе определена математическая модель таких случайных графов и исследованы 
характеристики, связанные с указанными выше прикладными задачами. Полученные 
результаты носят асимптотический характер при условии, что число хостов стремится к 
бесконечности. 
     
     Работа имеет следующую структуру. В~разд.~2 приведены некоторые близкие 
модели случайных графов. В~разд.~3 определена основная динамическая модель. 
В~разд.~4 исследована асимптотика максимальной степени в случайном графе, связанном 
с <<окнами>>. В~разд.~5 подведены итоги и намечены дальнейшие задачи. 
     
\section{Модели случайных графов}
     
     В научной литературе рассматривались модели случайных графов, связанных с 
Интернетом. В~работе~[1] в качестве одного из примеров приводится классическая 
модель случайного графа $G_{N,p}$ с независимыми ребрами, появляющимися с одной и 
той же вероятностью~$p$. Этой модели посвящено много работ и книг~[2--13]. И хотя 
осново\-по\-ла\-га\-ющая \mbox{статья} Эрдеша и Реньи~\cite{2-gr} связана с несколько другим классом 
случайных графов, большинство аналогичных результатов было также доказано для 
графов $G_{N,p}$ в работах~[3--6, 8] и~др. Фазовые перехо-\linebreak\vspace*{-12pt}

\pagebreak

\noindent
ды в структуре таких графов 
впервые исследованы в\linebreak работах~[3--5]. Первые модели с неравновероятными ребрами 
исследовались в работах~\cite{7-gr}. В~пе\-ре\-чис\-лен\-ных моделях появление ребра не 
допускало его дальнейшего исчезновения.
     
     Изменение случайного графа во времени в связи с задачей роста сети Интернет 
рассматривалось в работах~[7, 9, 10]. 
     
     Специальный класс случайных графов, посвященный исследованию связей в 
Интернете, объединяет модели графов ин\-тер\-нет-типа~[11--13]. Они определяются 
степенями вершин, являющихся независимыми случайными величинами. При этом 
свободные концы ребер замыкаются друг на друга случайно и равновероятно. 
     
\section{Динамическая модель сетевого взаимодействия}
     
     Определим детально модель сетевого взаимодействия, кратко изложенную во 
введении. 
     
     Рассмотрим дискретное время $t \hm= 0, 1, 2, \ldots$ Множество хостов сети 
обозначим $A\hm \{a_1, \ldots , a_N\}$. Логическая связь хостов~$a_i$ и~$a_j$ в момент 
времени~$t$ означает либо наличие в этот момент времени сеанса связи по протоколу 
ТСР между~$a_i$ и~$a_j$, либо передачу одиночного пакета от одного хоста к другому в 
этот момент времени по любому протоколу без установления соединения. Для простоты 
считаем, что время прохождения пакета по сети равно~1. При этом выбрасываются из 
рассмотрения все промежуточные поддерживающие сетевые службы (провайдеры, 
маршрутизаторы, адресные службы и~т.\,д.). Из этих допущений получаем модель графа 
сети. В~каждый момент времени~$t$ определен неориентированный граф~$G_t$, 
вершины которого совпадают с множеством~$A$, а ребра соответствуют существующим 
в момент времени~$t$ логическим связям. Из определения логической связи следует, что 
в соседние моменты времени~$t$ и $t\hm+1$ существование данного ребра в графе~$G_t$ 
и графе $G_{t+1}$ являются зависимыми событиями. Однако процессы появления и 
исчезновения разных ребер можно считать независимыми. 
     
     В простейшем случае полагаем, что процесс, описывающий возникновение и 
исчезновение одного ребра, является стационарной однородной \mbox{цепью} Маркова с двумя 
состояниями: 1~--- есть ребро, 0~--- нет ребра. 
     
     Наблюдения за графами $\{G_t\}$ происходят с помощью <<оконной>> системы. 
Пусть задано натуральное число~$r$ и для любого момента времени~$t$ рассматриваются 
графы $G_t, \ldots , G_{t+r}$, появляющиеся в моменты времени $[t,\,t+r]$. Определим 
операцию объединения этих графов
     $$
     G_{t,r}=\bigcup\limits_{i=0}^r G_{t+i}\,,
     $$
где из нескольких параллельных ребер оставляется одно ребро. Граф $G_{t.r}$ несет 
информацию об активности любой вершины в заданный промежуток времени. Эти графы 
представляют интерес в задачах информационной безопасности. Например, если вершина 
$a_i$ является центром управления бот-сетью, то использование бот-сети для организации 
DDoS атаки должно порождать в некоторый промежуток времени $[t,\,t+r]$ резкое 
повышение степени вершины~$a_i$ в графе~$G_{t,r}$. При малых~$r$ и очень 
больших~$r$ при неизвестном~$t$ этот всплеск активности может оказаться незаметным 
в масштабах всей сети. Поэтому исследование модели случайных графов может позволить 
оценить возможности по выявлению неслучайных всплесков активности отдельных 
вершин и даже дать оценку для центра управления бот-сетью. 

     Пусть поведение каждого ребра описывается стационарной однородной цепью 
Маркова с мат\-ри\-цей переходных вероятностей
     \begin{multline*}
     P=\begin{pmatrix}
     p &\ \  1-p\\
     q &\ \  1-q
     \end{pmatrix}\,,\\ 1>p=p(N)>0\,,\quad 1>q=q(N)>0\,,
%     \label{e1-gr}
     \end{multline*}
и стационарным распределением 
$\left( p_0\ \  1-p_0\right)$.

     
     Тогда вероятность непоявления данного ребра за промежуток времени $[t,\,t+r]$ 
равна
     \begin{equation}
     1-p_r=(1-p_0)(1-q)^r\,.
     \label{e3-gr}
     \end{equation}
 Эта вероятность не зависит от~$t$, поэтому будем обозначать ее~$p_r$. Из~(\ref{e3-gr}) 
     получаем вероятность появления данного ребра в промежуток времени $[t,\,t+r]$ в 
графе~$G_{t,r}$:
     \begin{equation*}
     p_r=1-(1-p_0)(1-q)^r\,.
%     \label{e4-gr}
     \end{equation*}
     
\section{Асимптотические оценки максимальной степени в~графах~{\boldmath{$G_{t,r}$}}}

     Для $v\in A$ обозначим через $d(v)$ степень вершины~$v$. Определим 
индикаторную функцию события~$B$:
     $$
     I(B)=\begin{cases}
     1\,, &\ \mbox{если событие $B$ призошло};\\
     0 &\ \mbox{в противном случае.}
     \end{cases}
     $$
     
     Ожидаемое число соединений у фиксированной вершины в ограниченный 
промежуток времени мало по сравнению с общим числом вершин. Асимптотически это 
отвечает условию $p_0 N\hm\rightarrow \mu\hm>0$, $N\hm\rightarrow\infty$. Серии единиц 
связаны с режимом установления соединения. Поэтому величина ($1-p$) может не 
стремиться к~1. Пусть $q\hm\rightarrow 0$, $N\hm\rightarrow\infty$, так что 
$qN\hm\rightarrow\lambda$. Из условия стационарности следует соотношение:
     $$
     q=\fr{p_0}{1-p_0}\left( 1-p\right)\,.
     $$
Таким образом, получаем
$$
p_r=\fr{\mu}{N}+\fr{\lambda r}{N}+O\left( \fr{\lambda r}{N^2}\right)\,.
$$
Положим $\alpha_r=\mu+\lambda r$ и будем считать, что
$$
p_r=\fr{\alpha_r}{N}\,.
$$
Обозначим 
$$
X=\sum\limits_{v\in A} I(d(v)>d)\,.
$$
Тогда 
$$
\left\{ \max\limits_{v\in A} d(v)\right\} >d =\left\{ X\geq 1\right\}\,.
$$
Используя неравенство Маркова, получаем:
\begin{multline}
P\left\{ \max\limits_{v\in A} d(v)>d\right\} \leq{}\\
{}\leq N\sum\limits_{k>d}\begin{pmatrix}
N-1\\ k\end{pmatrix} p_r^k(1-p_r)^{N-1-k}\,.
\label{e5-gr}
\end{multline}
     
     Пусть $B(N-1, k, p_r)$~--- функция распределения биномиального закона, 
$\overline{B}(N-1, k, p_r)\hm=1 \hm- B(N-1, k, p_r)$. Заметим, что в формуле~(\ref{e5-gr}) 
справа стоит $N\overline{B} (N-1, d, p_r)$. 
     
     Асимптотические оценки проводим в условиях
     $$
     N\rightarrow\infty\,, \ d=\fr{C\ln N}{\ln\ln N}\,,\ C>0\,,\ p_r=\fr{\alpha_r}{N}\,.
     $$
     
     Для оценки функции $\overline{B} (N-1, d, p_r)$ воспользуемся представлением для 
неполной бе\-та-функ\-ции~\cite{14-gr}:
     $$
     \overline{B}(N-1,k,p_r)=N\begin{pmatrix}
     N-2\\ k \end{pmatrix} \int\limits_0^{p_r} z^k (1-z)^{N-k-2}dz\,.
     $$
     
     Используя формулу Тейлора, получим при некотором $0\hm<\theta\hm<1$ 
следующее представление для математического ожидания~$X$:
     $$
     {\sf E}X=N(N-1)\begin{pmatrix}
     N-2\\ d\end{pmatrix} (p_r\theta)^d p_r(1-p_r\theta)^{N-2-d}\,.
     $$
Отсюда получаем следующую асимптотическую формулу:
\begin{equation}
{\sf E}X=N^{(1-C)(1+o(1))}\alpha_r(1-e^{-\alpha_r\theta})\,.
\label{e6-gr}
     \end{equation}
     
     \noindent
     \textbf{Теорема.} \textit{При} $C\hm>1$, $N\hm\rightarrow\infty$, $d=C\ln N/(\ln \ln 
N)$, $p_r=\alpha_r/N$
     $$
     P\left\{ \max\limits_{v\in A} d(v)>d\right\}\rightarrow 0\,.
     $$
     
     \noindent
     Д\,о\,к\,а\,з\,а\,т\,е\,л\,ь\,с\,т\,в\,о\ следует из~(\ref{e6-gr}). 
     
     \smallskip
     
     Таким образом, установлена граница для максимальной степени вершины в графе 
$G_{t,r}$. На основании этого результата можно построить оценку центра управления 
бот-сетью. Если существует вершина, степень которой превосходит заданную границу, то 
с вероятностью, близкой к~1, высокая степень этой вершины получена вне условий 
стационарности и других допущений, которые были сделаны для нормального поведения 
сети. 
     
     Предположим теперь, что $C\hm<1$ и математическое ожидание 
${\sf E}X\hm\rightarrow\infty$. Построим оценку числа вершин, имеющих степень больше~$d$, 
при условии, что ${\sf E}X\hm\rightarrow\infty$. С~этой целью оценим и сравним дисперсию 
$DX$ случайной величины~$X$ и $({\sf E}X)^2$. Очевидно, что 
     \begin{multline*}
     ({\sf E}X)^2=N^2(1-B(N-1,d,p_r))^2=\\
     {}=N^2\overline{B}^2(N-1,d,p_r)\,.
     \end{multline*}
Случайную величину~$X$ можно представить в виде:
\begin{equation}
X=\sum\limits_{i=1}^N I_i\,,
\label{e7-gr}
\end{equation}
где $I_i$~--- индикатор события $d(i)\hm>d$. Тогда из~(\ref{e7-gr}) следует:
\begin{multline*}
{\sf E}X^2={\sf E}\left( \sum\limits_{i=1}^N I_i\right) +{\sf E}
\left( 2\sum\limits_{i<j}I_iI_j\right) ={}\\
{}=
N\overline{B}(N-1,d,p_r) +2\sum\limits_{i<j}P(I_iI_j=1)\,.
\end{multline*}
По формуле полной вероятности
\begin{multline*}
P(I_1I_2=1)=p_r\overline{B}^2(N-2,d-1, p_r)+{}\\
{}+(1-p_r)\overline{B}^2(N-2,d,p_r)\,.
\end{multline*}
     
     Рассмотрим разность ${\sf E}X^2$ и $({\sf E}X)^2$. Несложные вычисления приводят к 
выражению:

\pagebreak

\noindent
     \begin{multline*}
{\sf E}X^2-({\sf E}X)^2={}\\
     {}=N\overline{B}(N-1,d,p_r)(1-\overline{B}(N-1,d,p_r))+{}\\
     {}+N(N-1)(1-p_r)p_r b^2(N-2,d,p_r)\,,
     \end{multline*}
где 
$$
b(N-2,d,p_r)=\begin{pmatrix}
N-2 \\ d\end{pmatrix} p_r^d(1-p_r)^{N-2-d}\,.
$$
     
     Предполагалось, что ${\sf E}X\rightarrow\infty$. Тогда
     \begin{multline*}
     \fr{{\sf D}X}{({\sf E}X)^2}=\fr{1}{N\overline{B}(N-1,d,p_r)}\left(
     B(N-1,d,p_r)+{}\right.\\
\left.     {}+\alpha_r(1-p_r)\fr{(N-1) b^2(N-2,d,p_r)}{N\overline{B}(N-1,d,p_r)}\right)\,.
     \end{multline*}
Из предыдущих оценок имеем, что 
\begin{align*}
Nb(N-2,d,p_r) &=O\left(N^{1-C}\right)\,;\\
N\overline{B} (N-1,d,p_r) &= N^{(1-C)(1+o(1))}\alpha_r\left( 1-e^{-\alpha_r\theta}\right)\,.
\end{align*}
Отсюда следует, что
$$
\fr{{\sf D}X}{({\sf E}X)^2} =O\left(N^{C-1}\right)\,,\enskip C<1\,.
$$
     
     Воспользуемся следствиями~4.32 и~4.33 из работы~\cite{15-gr}. Получаем, что с 
вероятностью, стремящейся к~1, $0<X$ и отношение $X/({\sf E}X)\rightarrow 1$. 
     Это означает, что при $C<1$ с вероятностью, близкой к~1, существует вершина степени 
больше~$d$ и число таких вершин совпадает с математическим ожиданием~$X$. 
     

\section{Заключение}
     
     В ходе решения поставленных в данной работе задач появилось много новых 
направлений, которые заслуживают отдельного внимания. В~данной работе исследовано 
поведение больших степеней в графе, соответствующем фиксированному <<окну>>. 
Естественно, желательно обобщить эти результаты на случай скользящего <<окна>>. 
     
     Поведение графов ин\-тер\-нет-ти\-па часто нельзя считать стационарным. Возникает 
задача анализа <<оконных>> графов в условиях нестационарного поведения сети. 
Перечень проблем, возникших при данном исследовании, не исчерпывается данными 
двумя задачами.

{\small\frenchspacing
{%\baselineskip=10.8pt
\addcontentsline{toc}{section}{Литература}
\begin{thebibliography}{99}
     
\bibitem{1-gr}
\Au{Kolaczyk E.\,D.} Statistical analysis of network data: Methods and models.~--- Springer 
Science\;+\;Business Media, LLC, 2009. 386~p. 
\bibitem{2-gr}
\Au{Erd$\ddot{\mbox{o}}$s P., R$\acute{\mbox{e}}$nyi~A.} On the evolution of random 
graphs~// Publ. Math. Inst. Hungarian Acad. Sci., Ser.~A, 1960. Vol.~5. P.~17--61.
\bibitem{3-gr}
\Au{Степанов В.\,Е.} О вероятности связности случайного графа $g_m(t)$~// Теория 
вероятностей и ее применения, 1970. Т.~15. №\,1. С.~55--67.
\bibitem{4-gr}
\Au{Степанов В.\,Е.} Фазовый переход в случайных графах~// Теория вероятностей и ее 
применения, 1970. Т.~15. №\,2. С.~187--203.
\bibitem{5-gr}
\Au{Степанов В.\,Е.} Структура случайных графов $g_n(x\vert h)$~// Теория вероятностей 
и ее применения, 1972. Т.~17. №\,3. С.~227--242.

\bibitem{6-gr}
\Au{Bollobas B.} Random graphs.~--- London: Academic Press, 1985.

\bibitem{9-gr} %7
\Au{Kleinberg J., Kumar S., Raghavan~P., Rajagopalan~S., Tomkins~A.} The web as a graph: 
measurements, models, and methods~// Conference (International) on Combinatorics and 
Computing Proceedings ~--- Berlin: Springer, 1999. Lecture Notes in Computer Science. 
Vol.~1627. P.~1--18.

\bibitem{7-gr} %8
\Au{Колчин В.\,Ф.} Случайные графы.~--- М.: Физматлит, 2000. 256~с.



\bibitem{10-gr} %9
\Au{Kumar R., Raghavan P., Rajagopalan~S., Sivakumar~D., Tomkins~A., Upfal~E.}
Stochastic models for the web graph~// 42nd Annual IEEE Symposium on the Foundations of 
Computer Science Proceedings, 2000. Vol.~41. P.~57--65. 

\bibitem{8-gr} %10
\Au{Chung F., Lu L., Dewey~T., Galas~D.} Duplication models for biological networks~// 
J.~Comput. Biology, 2003. Vol.~10. No.\,5. P.~677--687. 

\bibitem{11-gr}
\Au{Павлов Ю.\,Л., Степанов М.\,М.} Об асимптотических свойствах случайных графов 
<<ин\-тер\-нет-типа>>~// Обозрение прикладной и промышленной математики, 2005. 
Т.~12. №\,3. С.~677.
\bibitem{12-gr}
\Au{Степанов М.\,М.} О~предельных распределениях степеней узлов в случайных графах 
ин\-тер\-нет-типа~// Методы математического моделирования и информационные 
технологии: Тр. Института прикладных математических исследований Карельского 
научного центра РАН.~--- Петрозаводск: КарНЦ РАН, 2005. Вып.~6. С.~235--242.
\bibitem{13-gr}
\Au{Павлов Ю.\,Л.} Предельное распределение объема гигантской компоненты в 
случайном графе ин\-тер\-нет-типа~// Дискретная математика, 2007. Т.~19. №\,3. 
С.~22--34.
\bibitem{14-gr}
\Au{Феллер В.} Введение в теорию вероятностей и ее приложения.~--- 2-е изд.~--- 
М.: Мир, 1967. Т.~1.

\label{end\stat}

\bibitem{15-gr}
\Au{Alon N., Spencer~J.} The probabilistic method.~--- 2nd ed.~--- New York: Jonh Wiley \& Sons, 
2000.
\end{thebibliography}
}
}

\end{multicols}   %8
\def\stat{grebeshkov}

\def\tit{АНАЛИЗ ВРЕМЕНИ ПЕРЕКЛЮЧЕНИЯ СЕАНСА СВЯЗИ В~ГЕТЕРОГЕННЫХ 
БЕСПРОВОДНЫХ СЕТЯХ ПРИ~ВЕРТИКАЛЬНОМ ХЭНДОВЕРЕ$^*$}

\def\titkol{Анализ времени переключения сеанса связи в~гетерогенных 
беспроводных сетях при вертикальном хэндовере}

\def\aut{А.\,Ю.~Гребешков$^1$, Ю.\,В.~Гайдамака$^2$, О.\,Г.~Вихрова$^3$, 
Э.\,Р.~Зарипова$^4$}

\def\autkol{А.\,Ю.~Гребешков, Ю.\,В.~Гайдамака, О.\,Г.~Вихрова, 
Э.\,Р.~Зарипова}

\titel{\tit}{\aut}{\autkol}{\titkol}

\index{Гребешков А.\,Ю.}
\index{Гайдамака Ю.\,В.}
\index{Вихрова О.\,Г.} 
\index{Зарипова Э.\,Р.}
\index{Grebeshkov A.\,Yu.}
\index{Gaidamaka Yu.\,V.}
\index{Vikhrova O.\,G.} 
\index{Zaripova E.\,R.}



{\renewcommand{\thefootnote}{\fnsymbol{footnote}} \footnotetext[1]
{Публикация подготовлена при финансовой поддержке Минобрнауки России (проект 2.882.2017/4.6).}}


\renewcommand{\thefootnote}{\arabic{footnote}}
\footnotetext[1]{Поволжский государственный университет телекоммуникаций и~информатики, 
\mbox{grebeshkov-ay@psuti.ru}}
\footnotetext[2]{Российский университет дружбы народов; Институт проблем информатики Федерального исследовательского 
центра <<Информатика и~управ\-ле\-ние>> Российской академии наук, 
\mbox{gaydamaka\_yuv@rudn.university}}
\footnotetext[3]{Российский университет дружбы народов, vikhrova\_og@rudn.university}
\footnotetext[4]{Российский университет дружбы народов, zaripova\_er@rudn.university}

\vspace*{4pt}
 

\Abst{Для мобильных абонентов гетерогенной беспроводной сети в~некоторых точках могут 
быть одновременно доступны соединения в~перекрывающих друг друга областях покрытия 
радиосетей различных стандартов. Пользователь с~современным многорежимным 
абонентским устройством может переключаться между различными сетями радиодоступа 
для получения требуемых услуг связи с~помощью процедуры вертикального хэндовера 
(vertical handover, VHO). Для обеспечения качества и~непрерывности связи существенное 
значение имеет время переключения сеанса связи из текущей в~целевую сеть. 
Разработана процедура вертикального хэндовера из беспроводной локальной сети (Wireless 
Local Area Network, WLAN) в~мобильную сеть (3GPP Long Term Evolution, LTE). Проведен 
анализ среднего значения и~квантиля времени переключения с~помощью метода оценки 
времени пребывания заявок в~многофазной системе массового обслуживания (СМО) с~фоновым 
трафиком. Проведен численный эксперимент для близких к~реальным исходных данных 
процесса переключения сеанса связи.}

\KW{гетерогенная беспроводная сеть; сотовая сеть; LTE; мобильность; вертикальный 
хэндовер; надежность соединения; доступность соединения; показатель эффективности; 
процедура установления соединения}

\DOI{10.14357/19922264170409} 


\vskip 10pt plus 9pt minus 6pt

\thispagestyle{headings}

\begin{multicols}{2}

\label{st\stat}

\section{Введение}

\vspace*{-2pt}

  Тенденция развития современных сотовых сетей, приводящая к~увеличению 
емкости сетей за счет их пространственного уплотнения и~совершенствования 
методов управления распределением радиоресурса, соответствует концепции 
HetNet (Heterogeneous Network)~--- гетерогенных сетей. Реализация этой 
концепции стала возможной в~беспроводных сетях стандарта LTE/LTE-A~[1]. 
Исследования таких сетей с~точки зрения различных показателей качества 
активно ведутся в~России~[2--4]. После выбора целевой сети переключение 
сеанса связи многорежимного абонентского устройства (user equipment, UE) 
осуществляется с~использованием процедуры, которая называется 
вертикальным хэндовером (VHO). 

Использование VHO 
позволяет повысить качество обслуживания, например при использовании 
тактильного Интернета~\cite{5-gre}, для поддержки приложений, работающих 
в~реальном времени~\cite{6-gre}. Поэтому актуально проведение исследований и~анализ времени переключения с~целью предотвращения потери информации, 
что особенно важно в~приложениях реального времени. Критерии принятия 
решения о VHO рассмотрены в~\cite{7-gre}. В~\cite{8-gre} при анализе 
исполнения VHO не рассматриваются процедуры авторизации в~целевой сети, 
которые необходимы при VHO. В~\cite{9-gre} на основе теоремы Бёрке 
рассматривается модель аутентификации и~связанные с~этим временн$\acute{\mbox{ы}}$е 
задержки без учета времени на получение информации о параметрах целевой 
сети для VHO. В~\cite{10-gre, 11-gre} не учтено время подключения к~целевой 
сети и~время выделения радиоканала. 

Отличием данной работы  
от~\cite{7-gre, 8-gre, 9-gre, 10-gre, 11-gre} является детальная процедура обмена 
сигнальными сообщениями при вертикальном хэндовере~\cite{12-gre}, которая 
содержит все стадии, начиная от момента получения информации о доступных 
сетях и~заканчивая началом IP (Internet Protocol) сес\-сии абонентского устройства в~целевой 
сети. 

\begin{figure*}[b] %fig1
\vspace*{6pt}
 \begin{center}
 \mbox{%
 \epsfxsize=131.756mm 
 \epsfbox{gre-1.eps}
 }
 \end{center}
\vspace*{-9pt}
\Caption{Обмен сообщениями UE с~ANDSF и~авторизация UE в~сети LTE}
\end{figure*} 

На основе описанной в~разд.~2 процедуры обмена сигнальными 
сообщениями при вертикальном хэндовере в~разд.~3 разработан метод оценки 
времени переключения при вертикальном хэндовере~[13, 14]. С~его помощью 
в~разд.~4 проведена оценка среднего времени VHO с~учетом загрузки узлов 
текущей и~целевых сетей.



\section{Процедура обмена сигнальными сообщениями 
при~вертикальном хэндовере из~недоверенной сети в~сеть LTE}

  Для разработки процедуры принимаются исходные положения, которые не 
нарушают ее целостность и~не ограничивают применение. Вертикальный 
хэндовер осуществляется сервером ANDSF (Access Network Discovery and 
Selection Function). Для обмена данными между UE и~узлом ANDSF 
используется специфицированная партнерством 3GPP 
(3rd Generation Partnership Project) эталонная точка стыка 
S14 уровня приложений. Поддержка мобильности IP при VHO 
в~рассматриваемом примере обеспечивается применением мобильной версии 
протокола с~двойным стеком IPv6~[15, 16]. Для обеспечения без\-опас\-ности 
используется протокол IPsec, идентификация пользователя с~помощью 
шифрования открытым ключом IKE (Internet Key Exchange)~[17].

\begin{figure*}[b] %fig2
\vspace*{6pt}
 \begin{center}
 \mbox{%
 \epsfxsize=132.09mm 
 \epsfbox{gre-2.eps}
 }
 \end{center}
\vspace*{-9pt}
\Caption{Обмен сообщениями при установлении соединения UE с~целевой сетью для VHO 
WLAN-LTE}
\end{figure*}
  
  Пусть многорежимное устройство UE работает в~текущей WLAN-сети, 
авторизация и~аутентификация в~которой не соответствуют спецификациям 
3GPP. С~точки зрения оператора сети LTE текущая сеть считается 
недоверенной (nontrusted) IP-сетью доступа, поэтому UE инициирует VHO 
в~целевую сеть LTE. Функциональными устройствами, вовлеченными 
в~процедуру, являются: UE, ANDSF, шлюз пакетных данных ePDG (evolved 
Packet Data Gateway), расширенная сеть радиодоступа E-UTRAN (evolved 
UMTS (Universal Mobile Telecommunication
System) Terrestrial Radio Access Networks), узел управления мобильностью MME 
(Mobility Management Entity), обслуживающий шлюз S-GW (Serving Gateway), 
пакетный шлюз P-GW (Packet Data Network Gateway), узел выставления счетов 
абонентам hPCRF (home network Policy and Charging Rules Function) 
и~комбинированный сервер HSS/AAA (Home Subscriber 
Server\,/\,Authentication, Authorization, and Accounting). 

Подробное описание 
сигнальных сообщений, включаемых в~процедуру вертикального хэндовера, 
приведено в~\cite{14-gre}.
%
Обмен сигнальными сообщениями для данного этапа 
VHO представлен на рис.~1. 
{\looseness=1

}
  
  В завершающем этап авторизации сообщении (23) UE получает от ePDG 
данные об успешной ав\-то\-ризации в~сети LTE. После этого программное\linebreak 
обеспечение UE реконфигурируется для работы в~сети LTE по туннелю IPsec 
через эталонную точку стыка 3GPP~S2c. 
  



  Вторым этапом исполнения VHO является переключение UE в~целевую сеть 
LTE из текущей недоверенной сети WLAN. Схема обмена сигнальными 
сообщениями на этом этапе показана на рис.~2. 

\begin{figure*}[b] %fig3
\vspace*{1pt}
 \begin{center}
 \mbox{%
 \epsfxsize=151.044mm 
 \epsfbox{gre-3.eps}
 }
 \end{center}
\vspace*{-9pt}
\Caption{Модель многофазной СМО с~фоновым трафиком: \textit{1}~--- основной поток
заявок; \textit{2}~--- фоновый поток заявок}
\end{figure*}
  
  В начале этапа~2 UE синхронизировано с~расширенной сетью радиодоступа 
E-UTRAN, имеет информацию о физическом канале и~временный 
идентификатор радиосоты. Сообщения~(24)--(40)\linebreak
 отвечают за реконфигурацию 
физических каналов. После сообщения~(37), в~котором узел~\mbox{S-GW} 
подтверждает создание требуемого канала для поддержки IP-сес\-сии через 
расширенную систему пакетной передачи данных EPS (Evolved Packet System) 
вместо обмена через WLAN, UE начинает\linebreak принимать и~передавать пакеты 
данных через сеть LTE. 
  


\section{Метод оценки времени переключения с~учетом фонового 
трафика}

  В настоящей работе предлагается использовать приближенный метод оценки 
времени переключения с~учетом наличия в~сети фонового трафика~[18]. Для 
перехода к~аналитической модели пронумеруем введенные в~предыдущих 
разделах функциональные устройства: UE~(I), ANDSF~(II), ePDG~(III),  
E-UTRAN~(IV), MME~(V), S-GW~(VI), P-GW~(VII), hPCRF~(VIII) 
и~HSS/AAA~(IX). Используемая методика подразумевает разбиение потока 
обслуживаемых в~каждом узле сигнальных сообщений на основной и~фоновый 
потоки. Под сообщениями основного потока будем понимать включенные 
в~процедуру вертикального хэндовера сигнальные сообщения, под фоновым 
потоком~--- сообщения других задач. Поток сигнальных сообщений на рис.~3 
образует цепь, состоящую из $K\hm=39$~со\-сто\-яний. 



  Обозначим через~$\lambda_0$ и~$b_k$ интенсивность входящего потока 
и~среднюю длительность обслу\-жи\-вания заявок основного потока на $k$-й фазе 
в~многофаз\-ной СМО. Аналогично~$\lambda_k$ и~$d_k$~--- интенсивность 
потока и~средняя длительность обслуживания заявок фонового потока на $k$-й 
фазе. 
  
  Для расчета времени пребывания заявки в~многофазной СМО необходимо 
вычислить коэффициент вариации длительности обслуживания на $k$-й фазе 
по формуле:
  \begin{equation*}
  C_k^2 =\fr{(\lambda_0+\lambda_k) \left(\lambda_0 b_k^{(2)} +\lambda_k 
d_k^{(2)}\right)} {(\lambda_0 b_k+\lambda_k d_k)^2}-1,\enskip
  k=1,\ldots, K.
 % \label{e1-gre}
  \end{equation*}
Здесь использованы обозначения вторых моментов времени обслуживания 
заявок основного~$b_k^{(2)}$ и~фонового потока~$d_k^{(2)}$.

  Время ожидания начала обслуживания, полученное из известной формулы 
Пол\-ла\-че\-ка--Хин\-чи\-на, рассчитывается по формуле:
  $$
  \omega_k= \fr{\rho_k^2(1+C_k^2)} {2(\lambda_0+\lambda_k) (1-
\rho_k)}\,,\enskip k=1,\ldots, K\,.
  $$
Здесь $\rho_k=\lambda_0b_k\hm+\lambda_kd_k$~--- суммарная нагрузка на узел, 
соответствующий $k$-й фазе, $\rho_k\hm< 1$, $k\hm=1,\ldots, K$.

  Время пребывания~$\Delta$ заявки в~многофазной СМО равно сумме времен 
пребывания заявок основного потока на каждой фазе:
  $$
  \Delta= \sum\limits^K_{k=1} \left( \omega_k+b_k\right)\,.
  $$
  
  Время пребывания~$\Delta$ заявки в~многофазной СМО с~учетом фонового 
трафика соответствует времени переключения при вертикальном хэндовере.
  
  Заметим, что данный метод позволяет найти квантиль уровня~$\psi$ времени 
пребывания заявки в~многофазной СМО с~фоновым трафиком по формуле:
  $$
  Q_\psi= q_\psi +\sum\limits^K_{k=1} \left( \fr{\ln (\gamma_k 
\omega_k)}{\gamma_k}+b_k\right)\,,
  $$
где $q_\psi$ является единственным положительным корнем уравнения
$$
1-\psi= \sum\limits^{K-1}_{k=0} e^{-\gamma_k q_\psi} \fr{(\gamma_k q_\psi)^k} 
{k!}\,.
$$
  
  Параметры $\gamma_k$ затухания функций распределения времени 
ожидания начала обслуживания на $k$-й фазе, в~свою очередь, являются 
единственными положительными корнями уравнения 
$$
\alpha_k(\gamma_k)\beta_k(-\gamma_k)=1\,,
$$
 где $\alpha_k(s)\hm= 
(\lambda_0\hm+ \lambda_k)/(\lambda_0\hm+\lambda_k\hm+ s)$~--- 
преоб\-разование Лап\-ла\-са--Стилть\-еса (ПЛС) функции\linebreak распределения (ФР) 
интервалов времени между поступлениями заявок в~узле на $k$-й фазе, 
а~$\beta_k(s)\hm= (\lambda_0/(\lambda_0\hm+\lambda_k)) e^{-sb_k}\hm+ 
(\lambda_k/(\lambda_0\hm+ \lambda_k))e^{-sd_k}$~--- ПЛС ФР длительности 
обслуживания заявок.

  \begin{table*}[b]
  {\small \begin{center}
  
  \begin{tabular}{|l|c|c|}
  \multicolumn{3}{c}{Средние времена обслуживания}\\
  \multicolumn{3}{c}{\ }\\[-6pt]
  \hline
  \tabcolsep=0pt\begin{tabular}{c}Функциональные\\
узлы\end{tabular}&Среднее время обслуживания
$\mu_i^{-1}$, мс&  \tabcolsep=0pt\begin{tabular}{c}Источники данных\\
 для численного эксперимента\end{tabular}\\
\hline
I~--- UE&\tabcolsep=0pt\begin{tabular}{c}77,5 для (24);\\
28,5 для (26);\\
2 для остальных\end{tabular}&\cite{19-gre}\\
\hline
II~--- ANDSF&70&Считается, как HSS/AAA\\
\hline
III~--- ePDG&\hphantom{9}2&Считается, как P-GW\\
\hline
IV~--- eNB&\hphantom{9}4&\cite{20-gre}\\
\hline
V~--- MME&\tabcolsep=0pt\begin{tabular}{c}15 для (27);\\
1 для остальных\end{tabular}&\cite{20-gre, 21-gre}\\
\hline
VI~--- S-GW&\hphantom{9}2&\cite{19-gre}\\
\hline
VII~--- P-GW&\hphantom{9}2&\cite{19-gre}\\
\hline
VIII~--- hPCRF&70&Считается, как HSS/AAA\\
\hline
IX~--- HSS/AAA&70&\cite{22-gre}\\
\hline
\end{tabular}
\end{center}}
%\end{table*}
%\renewcommand{\figurename}{\protect\bf Рис.}
\renewcommand{\tablename}{\protect\bf Рис.}
\setcounter{table}{3}
%\begin{figure*} %fig4
\vspace*{6pt}
 \begin{center}
 \mbox{%
 \epsfxsize=164.269mm 
 \epsfbox{gre-4.eps}
 }
 \end{center}
\vspace*{-9pt}
\Caption{Среднее время переключения~(\textit{а}) и~95\%-ный квантиль времени переключения~(\textit{б}):
\textit{1}~--- $\lambda_k\hm=\lambda_0$; 
\textit{2}~--- $\lambda_k\hm=10\lambda_0$; \textit{3}~--- 
$\lambda_k\hm=100\lambda_0$}
%\end{figure*}
\end{table*}

\renewcommand{\figurename}{\protect\bf Рис.}
\renewcommand{\tablename}{\protect\bf Таблица}
  
  
\section{Численный эксперимент}

  Интенсивность запросов на совершение вертикального хэндовера зависит от 
местности, типа устройств UE, возможностей оператора связи, плотности 
мобильных пользователей и~других параметров. В~таблице приведены средние 
значения времен обслуживания сообщений в~узлах (основной трафик). 
Некоторые сообщения обслуживаются дольше остальных в~связи с~их 
функциональными особенностями и~б$\acute{\mbox{о}}$льшим объемом, при 
применении метода следует использовать данные статистических наблюдений. 
  

  На рис.~4 представлены результаты анализа времени переключения при 
вертикальном хэндовере для трех вариантов соотношения интенсивностей 
основного и~фонового трафиков: \textit{1}~--- $\lambda_k\hm=\lambda_0$; 
\textit{2}~--- $\lambda_k\hm=10\lambda_0$; \textit{3}~--- 
$\lambda_k\hm=100\lambda_0$. Среднее время обслуживания фонового трафика 
$d_k\hm=2$~мс.
  


  
  Анализ времени переключения показал, что дополнительный трафик со 
средним временем обслуживания несущественно влияет на время 
переключения, когда интенсивность этого трафика имеет тот же порядок, что 
и~интенсивность основного трафика (кривые~\textit{1} и~\textit{2}), 
и~проявляется, когда отношения интенсивностей основного и~фонового 
трафиков различаются на два порядка (кривые~\textit{3} на рис.~4).
  
  

На рис.~5 показана зависимость среднего значе\-ния и~95\%-ного квантиля 
времени переклю\-чения при вертикальном хэндовере для трех значений средней 
длительности обслуживания\linebreak сообщения фонового трафика $d_k\hm=10$, 20 и~50~мс, интенсивности входящего основного и~фонового трафика равны, 
$\lambda_k\hm= \lambda_0$.

\setcounter{figure}{4}

\begin{figure*} %fig5
\vspace*{1pt}
 \begin{center}
 \mbox{%
 \epsfxsize=163.863mm 
 \epsfbox{gre-6.eps}
 }
 \end{center}
\vspace*{-9pt}
\Caption{Среднее время переключения~(\textit{а})
и~95\%-ный квантиль времени переключения:
\textit{1}~--- $d_k\hm=10$~мс;  
\textit{2}~--- 20; \textit{3}~--- $d_k\hm=50$~мс}
\end{figure*}



     При одинаковой интенсивности фонового и~основного трафика 
и~увеличении среднего времени обслуживания фонового трафика до~50~мс 
наблюдается резкое увеличение времени переключения при вертикальном 
хэндовере (кривые~\textit{3} на рис.~5). Однако при среднем времени 
обслуживания фонового трафика до~20~мс фоновый трафик практически не 
влияет на время переключения при вертикальном хэндовере 
(кривые~\textit{1} и~\textit{2} на рис.~5).
     
  Рекомендуемый метод оценки позволяет учесть влияние фонового трафика 
  и~на отдельные узлы сети, участвующие в~процедуре. Тем не менее по 
результатам проведения статистических наблюдений на реальных сетях связи 
указанная рекомендация может уточняться.

\section{Заключение}

  Применение процедуры вертикального хэндовера в~гетерогенных 
беспроводных сетях в~сочетании с~использованием многорежимных 
абонентских устройств открывает широкие возможности по 
дифференцированному доступу абонентов к~ресурсам сетей. 
  
  В статье разработана процедура обмена сигнальными сообщениями для VHO 
из беспроводной локальной сети в~сеть LTE, предложен аналитический метод 
оценки времени переключения. При этом пользователями могут быть как 
устройства пользователей, так и~<<умные>> устройства, автоматически 
взаимодействующие по принципу M2M (Machine-to-Machine). 
  
  В дальнейших исследованиях планируется применить представленный метод 
для оценки времени переключения при обратной процедуре из сетей 
подвижной беспроводной связи LTE в~сеть WLAN.
  
{\small\frenchspacing
 {%\baselineskip=10.8pt
 \addcontentsline{toc}{section}{References}
 \begin{thebibliography}{99}
\bibitem{1-gre}
\Au{Astely D., Dahlman~E., Fodor~G., Parkvall~S., Sachs~J.} LTE release~12 and 
beyond~// IEEE Commun. Mag., 2013. Vol.~51. No.\,7. P.~154--160.
doi: 10.1109/MCOM. 2013.6553692. 
\bibitem{3-gre} %2
\Au{Горбунова А.\,В., Зарядов~И.\,С., Матюшенко~С.\,И., Самуйлов~К.\,Е., 
Шоргин~С.\,Я.} Аппроксимация времени отклика системы облачных 
вычислений~// Информатика и~её применения, 2015. Т.~9. Вып.~3. С.~32--38.
\bibitem{2-gre} %3
\Au{Вихрова О.\,Г., Самуйлов~К.\,Е., Сопин~Э.\,С., Шоргин~С.\,Я.} К~анализу 
показателей качества обслуживания в~современных беспроводных сетях~// 
Информатика и~её применения, 2015. Т.~9. Вып.~4. С.~48--55.

\bibitem{4-gre}
\Au{Гайдамака Ю.\,В., Андреев~С.\,Д., Сопин~Э.\,С., Самуйлов~К.\,Е., 
Шоргин~С.\,Я.} Анализ характеристик интерференции в~модели 
взаимодействия устройств с~учетом среды распространения сигнала~// 
Информатика и~её применения, 2016. Т.~10. Вып.~4. С.~2--10.
\bibitem{5-gre}
\Au{Кучерявый А.\,Е., Маколкина~М.\,А., Киричек~Р.\,В.} Тактильный 
Интернет. Сети связи со сверхмалыми задержками~// Электросвязь, 2016. №\,1. 
С.~44--46.
\bibitem{6-gre}
\Au{Kellokoski J., Koskinen~J., Nyrhinen~R., 
H$\ddot{\mbox{a}}$m$\ddot{\mbox{a}}$l$\ddot{\mbox{a}}$inen T.} Efficient 
handovers for machine-to-machine communications between IEEE 802.11 and 3GPP 
evolved packed core networks~//  IEEE  Conference (International) on Green 
Computing and Communications Proceedings.~--- 
\mbox{Besan{\!\ptb{\c{c}}}on}, 2012. P.~722--725.
\bibitem{7-gre}
\Au{Ahmed L., Boulahia~L.\,M., Gaiti~D.} Enabling vertical handover decisions in 
heterogeneous wireless networks: A~state-of-the-art and a classification~// IEEE 
Commun. Surv.  Tut., 2014. Vol.~16. No.\,2. P.~776--781.
\bibitem{8-gre}
\Au{Bukhari J., Akkari~N.} QoS based approach for LTE-WiFi handover~// 7th 
Conference (International) on Computer Science \& Information Technology 
Proceedings.~--- Elsevier, 2016. P.~1--6.
\bibitem{9-gre}
\Au{Gondim P.\,R.\,L., Trineto~J.\,B.\,M.} DSMIP and PMIP for mobility 
management of heterogeneous access networks: Evaluation of authentication 
delay~//  IEEE Globecom Workshops Proceedings.~--- Anaheim, 2012.  
P.~308--313.

\bibitem{11-gre} %10
\Au{Do-Hyung~K., Won-Tae~K., Hwan-Gu~L., Sun-Ja~K., Cheol-Hoon~L.} 
A~performance evaluation of vertical handover architecture with low latency 
handover~//  Conference (International) on Convergence and Hybrid Information 
Technology Proceedings.~--- Dusan, 2008. P.~66--69.

\bibitem{10-gre} %11
\Au{Tsagkaropoulos M., Politis~I., Tselios~C., Dagiuklas~T., Kotsopoulos~S.} 
Service continuity over intertechnology RATs~// 16th IEEE  Workshop 
(International) on Computer Aided Modeling and Design of Communication Links 
and Networks Proceedings.~--- Kyoto, 2011. P.~117--121.

\bibitem{12-gre}
\Au{M$\acute{\mbox{a}}$rquez-Barja J., Calafate~C.\,T., Cano~J.-C., Manzoni~P.} 
An overview of vertical handover techniques: Algorithms, protocols and tools~// 
Comput. Commun., 2011. Vol.~34. P.~985--997.
\bibitem{13-gre}
\Au{Гребешков А.\,Ю.} Оценка целесообразности обработки заявки для 
предоставления услуги в~реконфигурируемых сетях следующего поколения~// 
T-Comm: Телекоммуникации и~транспорт, 2014. №\,8. С.~24--27. 
\bibitem{14-gre}
\Au{Grebeshkov A., Zaripova~E., Roslyakov~A., Samouylov~K.} Modelling of 
vertical handover from untrusted WLAN network to LTE~// 31st European 
Conference on Modelling and Simulation Proceedings, 2017. P.~694--700.
\bibitem{15-gre}
\Au{Lampropoulos G., Passas~N., Mekaros~L., Kaloxylos~A.} Handover 
management architectures in integrated WLAN~// IEEE Commun. 
Surv. Tut., 2005. Vol.~7. No.\,4. P.~30--44.
\bibitem{16-gre}
3GPP TS~23.402 Technical specification 3GPP; TS Group Services and System 
Aspects; Architecture enhancements for non-3GPP accesses. Release~14, 2016.
{\sf 
https://\linebreak portal.3gpp.org/desktopmodules/Specifications/Specifi cationDetails.aspx?specificationId=850}.
\bibitem{17-gre}
3GPP TS 33.402 Technical specification 3GPP; TS Group Services and System 
Aspects; 3GPP System Architecture Evolution (SAE); Security aspects of non-3GPP 
accesses. Release~14, 2016.
{\sf  
https://portal. 3gpp.org/desktopmodules/Specifications/Specification\linebreak Details.aspx?specificationId=2297.}
\bibitem{18-gre}
\Au{Gaidamaka Yu., Zaripova~E.} Session setup delay estimation methods for  
IMS-based IPTV services~// Internet of things, smart spaces, and next generation networks
and systems~/ Eds.\ S.~Balandin, S.~Andreev, Y.~Koucheryavy.~---
Lecture notes in computer science ser.~---
Springer, 2014. Vol.~8638. 
P.~408--418.
\bibitem{19-gre}
\Au{Nikaein N., Krco~S.} Latency for real-time machine-to-machine communication 
in LTE-based system architecture~// 17th European Wireless Conference on 
Sustainable Wireless Technologies Proceedings.~--- Vienna, Austria: IEEE, 2011. 
P.~1--6.
\bibitem{20-gre}
\Au{Cardona N., Monserrat~J.\,F., Cabrejas~J.} Enabling technologies for 3GPP 
LTE-advanced networks~// LTE-advanced and next generation wireless networks~/ 
Eds.\ G.~de la~Roche, A.\,A.~Glazunov, B.~Allen.~--- Chichester, U.K.: John 
Wiley and Sons, 2013. P.~3--34.
\bibitem{21-gre}
\Au{Prados-Garzon J., Ramos-Munoz~J.\,J., Ameigeiras~P., Andres-Maldonado~P., 
Lopez-Soler~J.\,M.} Latency evaluation of a virtualized MME~//  
 Wireless Days Proceedings.~--- Toulouse, France: IEEE, 2016. P.~1--3.
\bibitem{22-gre}
\Au{Granlund D., Holmlund~P., \mbox{{\ptb{\AA}}hlund~C.}} Opportunistic mobility 
support for resource constrained sensor devices in smart cities~// Sensors, 2015. 
Vol.~15. No.\,3. P.~5112--5135.
 \end{thebibliography}

 }
 }

\end{multicols}

\vspace*{-6pt}

\hfill{\small\textit{Поступила в~редакцию 31.05.17}}

\vspace*{8pt}

%\newpage

%\vspace*{-24pt}

\hrule

\vspace*{2pt}

\hrule

%\vspace*{8pt}


\def\tit{ANALYSIS OF~VERTICAL HANDOVER TIME\\ IN~HETEROGENEOUS WIRELESS NETWORKS}

\def\titkol{Analysis of~vertical handover time in~heterogeneous wireless networks}

\def\aut{A.\,Yu.~Grebeshkov$^1$, Yu.\,V.~Gaidamaka$^{2,3}$, O.\,G.~Vikhrova$^{2}$, 
and~E.\,R.~Zaripova$^2$}

\def\autkol{A.\,Yu.~Grebeshkov, Yu.\,V.~Gaidamaka, O.\,G.~Vikhrova, 
and~E.\,R.~Zaripova}

\titel{\tit}{\aut}{\autkol}{\titkol}

\vspace*{-9pt}


\noindent
$^1$Povolzhskiy State University of Telecommunications and Informatics, 23~Tolstoy Str., Samara 443010, Russian\linebreak 
$\hphantom{^1}$Federation

\noindent
$^2$Peoples' Friendship University of Russia (RUDN University), 6~Miklukho-Maklaya Str., 
Moscow 117198,\linebreak
$\hphantom{^1}$Russian Federation


\noindent
$^3$Institute of Informatics Problems, Federal Research Center ``Computer Science and Control'' 
of the Russian\linebreak
$\hphantom{^1}$Academy of Sciences, 44-2~Vavilov Str., Moscow 119333, 
Russian Federation



\def\leftfootline{\small{\textbf{\thepage}
\hfill INFORMATIKA I EE PRIMENENIYA~--- INFORMATICS AND
APPLICATIONS\ \ \ 2017\ \ \ volume~11\ \ \ issue\ 4}
}%
 \def\rightfootline{\small{INFORMATIKA I EE PRIMENENIYA~---
INFORMATICS AND APPLICATIONS\ \ \ 2017\ \ \ volume~11\ \ \ issue\ 4
\hfill \textbf{\thepage}}}

\vspace*{3pt}



\Abste{In a heterogeneous wireless network, connectivity is simultaneously available 
using different radio networks with overlapping coverage areas. A~mobile user 
equipment with a~multiple mode card that can work under various frequency bands 
and modulation schemes can switch from one technology to another in order to 
maintain communication. This procedure known as a vertical handover (VHO)
provides the 
benefit of utilizing the higher bandwidth and lower cost of wide local area networks 
as well as better mobility support and larger coverage of cellular networks. The 
authors investigate details of the VHO procedure from WLAN (Wireless Local
Area Network) to the 
3GPP Long Term Evolution (LTE). The VHO procedure includes~40~signaling 
messages, which are responsible for authorization and resource allocation in the 
target LTE network. The authors analyze the VHO sojourn time and its~95~percent 
quantile using a~multiphase queuing system with background traffic.}

\KWE{heterogeneous wireless network; cellular network; LTE; mobility; session 
setup procedure; connection reliability; connection availability; performance 
measure}




  \DOI{10.14357/19922264170409} 

%\vspace*{-12pt}

\Ack
\noindent
The publication was supported by the Ministry of Education and Science of the 
Russian Federation (project No.\,2.882.2017/4.6). 



\vspace*{12pt}

  \begin{multicols}{2}

\renewcommand{\bibname}{\protect\rmfamily References}
%\renewcommand{\bibname}{\large\protect\rm References}

{\small\frenchspacing
 {%\baselineskip=10.8pt
 \addcontentsline{toc}{section}{References}
 \begin{thebibliography}{99}
\bibitem{1-gre-1}
\Aue{Astely, D., E.~Dahlman, G.~Fodor, S.~Parkvall, and J.~Sachs.} 2013. LTE 
release~12 and beyond. \textit{IEEE Commun. Mag.} 51(7):154--160. 
doi: 10.1109/MCOM. 2013.6553692. 


\bibitem{3-gre-1} %2
\Aue{Gorbunova, A.\,V., I.\,S.~Zaryadov, S.\,I.~Matyushenko, K.\,E.~Samouylov, 
and S.\,Ya.~Shorgin.} 2015. Approksimatsiya vremeni otklika sistemy oblachnykh 
vychisleniy [The approximation of response time of a cloud computing system]. 
\textit{Informatika i~ee Primeneniya~--- Inform. Appl.} 9(3):32--38.

\bibitem{2-gre-1} %3
\Aue{Vikhrova, O.\,G., K.\,E.~Samouylov, E.\,S.~Sopin, and S.\,Ya.~Shorgin}. 2015. 
K~analizu pokazateley kachestva obsluzhivaniya v~sovremennykh besprovodnykh 
setyakh [On performance analysis of modern wireless networks]. \textit{Informatika 
i~ee Primeneniya~--- Inform. Appl.} 9(4):48--55.

\bibitem{4-gre-1}
\Aue{Gaidamaka, Yu.\,V., S.\,D.~Andreev, E.\,S.~Sopin, K.\,E.~Samouylov, and 
S.\,Ya.~Shorgin.} 2016. Analiz kharakteristik interferentsii v~modeli 
vzaimodeystviya ustroystv s~uchetom sredy rasprostraneniya signala [Interference 
analysis of the device-to-device communications model with regard to a~signal 
propagation environment]. \textit{Informatika i~ee Primeneniya~--- Inform. Appl.} 
10(4):2--10.
\bibitem{5-gre-1}
\Aue{Koucheryavy, A.\,E., М.\,А.~Makolkina, and R.\,V.~Kirichek.} 2016. 
Taktil'niy Internet. Seti svyazi so sverkhmalymi zaderzhkami [Tactile Internet. 
Ultra-low latency networks]. \textit{Elekrosvyaz'} 
[Telecomm. Radio Eng.] 1:44--46.
\bibitem{6-gre-1}
\Aue{Kellokoski, J., J.~Koskinen, R.~Nyrhinen, and 
T.~H$\ddot{\mbox{a}}$m$\ddot{\mbox{a}}$l$\ddot{\mbox{a}}$inen.} 2012. 
Efficient handovers for machine-to-machine communications between IEEE~802.11 
and 3GPP evolved packed core networks. \textit{IEEE  Conference (International) 
on Green Computing and Communications Proceedings}. \mbox{Besan{\!\ptb{\c{c}}}on.}  
722--725.
\bibitem{7-gre-1}
\Aue{Ahmed, L., L.~M.~Boulahia, and D.~Gaiti.} 2014. Enabling vertical 
handover decisions in heterogeneous wireless networks: A~state-of-the-art and 
a~classification. \textit{IEEE Commun. Surv. Tut.} 16(2):776--781.
\bibitem{8-gre-1}
\Aue{Bukhari, J., and N.~Akkari.} 2016. QoS based approach for LTE-WiFi 
handover. \textit{7th Conference (International) on Computer Science and 
Information Technology Proceedings}. Elsevier. 1--6.
\bibitem{9-gre-1}
\Aue{Gondim, P.\,R.\,L., and J.\,B.\,M.~Trineto.} 2012. DSMIP and PMIP for 
mobility management of heterogeneous access networks: Evaluation of 
authentication delay. \textit{IEEE Globecom Workshops Proceedings}. Anaheim. 
308--313.



\bibitem{11-gre-1} %10
\Aue{Do-Hyung, K., K.~Won-Tae, L.~Hwan-Gu, K.~Sun-Ja, and  
L.\,A.~Cheol-Hoon.} 2008. Performance evaluation of vertical hanover architecture 
with low latency handover. \textit{Conference (International) on Convergence and 
Hybrid Information Technology Proceedings}. Dusan. 66--69.


\columnbreak

\bibitem{10-gre-1} %11
\Aue{Tsagkaropoulos, M., I.~Politis, C.~Tselios, T.~Dagiuklas, and 
S.~Kotsopoulos}. 2011. Service continuity over intertechnology RATs. \textit{16th 
IEEE  Workshop (International) on Computer Aided Modeling and Design of 
Communication Links and Networks Proceedings}. Kyoto. 117--121.

\bibitem{12-gre-1}
\Aue{M$\acute{\mbox{a}}$rquez-Barja,~J., C.\,T.~Calafate, J.-C.~Cano, and 
P.~Manzoni.} 2011. An overview of vertical handover techniques: Algorithms, 
protocols and tools. \textit{Comput. Commun.} 34:985--997.
\bibitem{13-gre-1}
\Aue{Grebeshkov, A.\,Yu.} 2014. Otsenka tselesoobraznosti obrabotki zayavki dlya 
predostavleniya uslugi v~re\-kon\-fi\-gu\-ri\-ru\-emykh setyakh sleduyushchego pokoleniya 
[Estimation of processing feasibility for service provision application in 
reconfigurable Next Generation Networks]. \textit{\mbox{T-Comm}: Te\-le\-kom\-mu\-ni\-ka\-tsii 
i~transport} [\mbox{T-Comm}: Telecommunications and Transport ] 8:24--27. 
\bibitem{14-gre-1}
\Aue{Grebeshkov, A., E.~Zaripova, A.~Roslyakov, and K.~Samouylov}. Modelling 
of vertical handover from untrusted WLAN network to LTE. \textit{31st European 
Conference on Modelling and Simulation Proceedings}. 694--700.
\bibitem{15-gre-1}
\Aue{Lampropoulos, G., N.~Passas, L.~Mekaros, and A.~Kaloxylos.} 2005. 
Handover management architectures in integrated WLAN. \textit{IEEE 
Commun. Surv.  Tut.} 7(4):30--44.
\bibitem{16-gre-1}
3GPP TS 23.402. 2016. Technical specification 3rd Generation Partnership Project; 
Technical Specification Group Services and System Aspects; Architecture 
enhancements for non-3GPP accesses. Release~14.  
Available at: {\sf 
https://portal.3gpp.org/desktopmodules/\linebreak Specifications/SpecificationDetails.aspx?specificationId\linebreak =850} (accessed May~20, 2017).
\bibitem{17-gre-1}
3GPP TS 33.402. 2016. Technical specification 3rd Generation Partnership Project; 
Technical Specification Group Services and System Aspects; 3GPP System 
Architecture Evolution (SAE); Security aspects of non-3GPP accesses. Release~14. 
Available at: {\sf 
https://portal. 3gpp.org/desktopmodules/Specifications/Specification\linebreak Details.aspx?specificationId=2297}
 (accessed May~20, 2017).
\bibitem{18-gre-1}
\Aue{Gaidamaka, Yu., and E.~Zaripova} 2014. Session setup delay estimation 
methods for IMS-based IPTV services. 
\textit{Internet of things, smart spaces, and next generation networks
and systems}.
Eds.\ S.~Balandin, S.~Andreev, and Y.~Koucheryavy.
{Lecture notes in computer science ser.} Springer.
8638:408--418.
\bibitem{19-gre-1}
\Aue{Nikaein, N., and S.~Krco.} 2011. Latency for real-time machine-to-machine 
communication in LTE-based system architecture. \textit{17th European Wireless 
Conferense on Sustainable Wireless Technologies Proceedings.} Vienna, Austria: 
IEEE. 1--6.
\bibitem{20-gre-1}
\Aue{Cardona, N., J.\,F.~Monserrat, and J.~Cabrejas.} 2013. Enabling technologies 
for 3GPP LTE-advanced networks.\linebreak\vspace*{-11pt}

\pagebreak

\noindent
 \textit{LTE-Advanced and next generation 
wireless networks}. Eds.\
 G.~de la~Roche, A.\,A.~Glazunov, and B.~Allen.
Chichester, U.K.: John Wiley and Sons. 3--34.
\bibitem{21-gre-1}
\Aue{Prados-Garzon, J., J.\,J.~Ramos-Munoz, P.~Ameigeiras,  
P.~Andres-Maldonado, and J.\,M.~Lopez-Soler.} 2016. La-\linebreak\vspace*{-11pt}

\columnbreak

\noindent
tency evaluation of 
a~virtualized MME. \textit{Wireless Days Proceedings}. Toulouse, 
France: IEEE. 1--3.
\bibitem{22-gre-1}
\Aue{Granlund, D., P.~Holmlund, and \mbox{C.~{\ptb{\AA}}hlund}}. 2015. 
Opportunistic mobility support for resource constrained sensor devices in smart 
cities. \textit{Sensors} 15(3):5112--5135.
{\looseness=1

}
\end{thebibliography}

 }
 }

\end{multicols}

\vspace*{-6pt}

\hfill{\small\textit{Received May 31, 2017}}

%\vspace*{-10pt}

\Contr

\noindent
\textbf{Grebeshkov Alexander Yu.} (b.\ 1967)~--- Candidate of Science (PhD) in 
technology, senior scientist, Povolzhskiy State University of Telecommunications 
and Informatics, 23~Tolstoy Str., Samara 443010, Russian Federation;  
\mbox{grebeshkov-ay@psuti.ru}

\vspace*{3pt}

\noindent
\textbf{Gaidamaka Yuliya V.} (b.\ 1971)~--- Candidate of Science (PhD) in physics and 
mathematics, associate professor, Peoples' Friendship University of Russia (RUDN University), 
6~Miklukho-Maklaya Str., Moscow 117198, Russian Federation; senior scientist, Institute of 
Informatics Problems, Federal Research Center ``Computer Science and Control'' of the Russian 
Academy of Sciences, 44-2~Vavilov Str., Moscow 119333, Russian Federation; 
\mbox{gaydamaka\_yuv@rudn.university}

\vspace*{3pt}

\noindent
\textbf{Vikhrova Olga G.} (b.\ 1990)~--- PhD student, Peoples' Friendship University of Russia 
(RUDN University), 6~Miklukho-Maklaya Str., Moscow 117198, Russian Federation; 
\mbox{vikhrova\_og@rudn.university}

\vspace*{3pt}

\noindent
\textbf{Zaripova Elvira R.} (р.\ 1979)~--- Candidate of Science (PhD) in physics and 
mathematics, associate professor, Peoples' Friendship University of Russia (RUDN University), 
6~Miklukho-Maklaya Str., Moscow 117198, Russian Federation; 
\mbox{zaripova\_er@rudn.university}

\label{end\stat}


\renewcommand{\bibname}{\protect\rm Литература}  %9
\def\stat{naumov}

\def\tit{О СВЯЗИ РЕСУРСНЫХ СИСТЕМ МАССОВОГО ОБСЛУЖИВАНИЯ 
С~СЕТЯМИ ЭРЛАНГА$^*$}

\def\titkol{О связи ресурсных систем массового обслуживания 
с~сетями Эрланга}

\def\aut{В.\,А.~Наумов$^1$, К.\,Е.~Самуйлов$^2$ }

\def\autkol{В.\,А.~Наумов, К.\,Е.~Самуйлов}

\titel{\tit}{\aut}{\autkol}{\titkol}

\index{Naumov V.\,A.}
\index{Samouylov K.\,E.}
\index{Наумов В.\,А.}
\index{Самуйлов К.\,Е.}


{\renewcommand{\thefootnote}{\fnsymbol{footnote}} \footnotetext[1]
{Работа выполнена при частичной финансовой поддержке РФФИ (проекты 16-07-00766,
 15-07-03051 и~15-07-03608).}}


\renewcommand{\thefootnote}{\arabic{footnote}}
\footnotetext[1]{Исследовательский институт инноваций, 
г.\ Хельсинки, Финляндия, \mbox{valeriy.naumov@pfu.fi}}
\footnotetext[2]{Российский университет дружбы народов; Институт проблем информатики Федерального 
исследовательского центра <<Информатика и~управление>> Российской академии наук,  
\mbox{ksam@sci.pfu.edu.ru}}

\vspace*{-12pt}
  
\Abst{Рассматривается модель многолинейной системы массового 
обслуживания (СМО) с~потерями, вызванными нехваткой ресурсов, 
необходимых для обслуживания заявок. Принятая на обслуживание заявка 
занимает случайные объемы ресурсов нескольких типов с~заданными 
функциями распределения. Случайные векторы, описывающие требования 
заявок к~ресурсам, не зависят от процессов поступления и~обслуживания 
заявок, независимы в~совокупности и~одинаково распределены. Интерес, как 
и~в~задаче Эрланга, представляет вычисление вероятности потери 
поступающей заявки из-за нехватки ресурсов. Показана связь между 
мультисервисными сетями Эрланга и~ресурсными СМО, что позволяет решать 
задачу вычисления вероятности потерь в~ресурсной СМО с~помощью 
известных методов, разработанных для мультисервисных сетей.}

\KW{мультисервисная сеть; сеть Эрланга; система массового обслуживания; 
ресурсная СМО; случайный объем ресурсов; вероятность потерь; решетчатая 
функция}

\DOI{10.14357/19922264160302} 


\vskip 12pt plus 9pt minus 6pt

\thispagestyle{headings}

\begin{multicols}{2}

\label{st\stat}

\section{Введение}

  Рассмотрим многолинейные СМО с~потерями, разнотипными ресурсами 
и~пуассоновским входящим потоком, которые функционируют сле\-ду\-ющим 
образом. Поступившая заявка теряется, если в~момент поступления количество 
ка\-ко\-го-ли\-бо требуемого ей ресурса превышает количество свободного 
ресурса этого типа либо если число обслуживаемых заявок достигло 
максимума. В~момент начала обслуживания заявки суммарный объем 
свободного ресурса каждого типа уменьшается на величину ресурса, 
выделенного этой заявке. В~момент окончания обслуживания заявки 
суммарный объем свободного ресурса каждого типа увеличивается на величину 
ресурса, выделенного этой заявке при поступлении. 
  
  Ресурсные СМО с~пуассоновским входящим потоком исследуются давно. 
В~\cite{3-n} получены стационарные распределения числа заявок в~системе 
и~объема занятого ресурса для СМО с~экспоненциальной функцией 
распределения длительности обслуживания и~произвольной функцией 
распределения объемов ресурса. Эти результаты обобщены в~\cite{4-n} на 
СМО с~произвольной функцией распределения длительностей обслуживания 
и~в~\cite{1-n} на системы с~множественными ресурсами. Дальнейшие 
обобщения, рассмотренные в~\cite{5-n, 6-n}, включают системы, у которых 
время обслуживания заявки и~объемы выделенных ей ресурсов являются 
зависимыми случайными величинами. В~\cite{5-n} получено стационарное 
распределение и~вероятность потери для СМО с~произвольной совместной 
функцией распределения длительности обслуживания и~объема единственного 
ресурса. Эти результаты обобщены в~\cite{6-n} на СМО, в~которых каждая 
заявка характеризуется тремя зависимыми случайными признаками: числом 
приборов, необходимых для обслуживания, объемом ресурса и~временем 
обслуживания. Применение модели ресурсных СМО к~анализу вероятностных 
характеристик беспроводных гетерогенных сетей 5-го поколения было 
предложено в~\cite{2-n}.
  
  Хорошо изучены мультисервисные сети Эрланга~\cite{7-n, 8-n} 
с~соединениями нескольких типов, в~которых каждому соединению в~каждом 
звене сети выделяется определенное число каналов, т.\,е.\ ресурсов сети, 
удерживаемое до завершения соеди-\linebreak нения. 

В~настоящей работе исследуется 
связь \mbox{между} мультисервисными сетями Эрланга и~ресурсными СМО 
с~арифметическими функциями распределения объемов требуемых ресурсов, 
которыми сколь угодно точно можно аппроксимировать любые функции 
распределения объемов ресурсов. \mbox{Целью} работы является исследование 
приближенного подхода к~вычислению вероятностных характеристик 
ресурсных СМО. Для краткости будем опускать слово <<мультисервисные>> 
в~названии сетей Эрланга.

\section{Сети Эрланга}

  Рассмотрим сеть массового обслуживания с~потерями, состоящую из 
некоторого числа узлов, соединенных звеньями. 

Пусть общее число звеньев 
сети равно~$M$, емкость $m$-го звена равна~$N_m$, $\mathbf{N}\hm= (N_1, 
N_2, \ldots ,N_M)$ и~${\sf N}(\mathbf{n}) \hm= \{ \mathbf{i}\hm\in {\sf N}^M 
\vert \mathbf{0}\leq \mathbf{i}\leq \mathbf{n}\}$, где ${\sf N}$~--- множество 
неотрицательных целых чисел. Между узлами сети могут быть установлены 
соединения~$L$~различных классов, каждый из которых однозначно 
характеризуется своими требованиями к~емкости звеньев сети. Требование 
к~числу каналов соединений $l$-го класса задается вектором 
$\mathbf{n}_l\hm= (n_{l1}, n_{l2}, \ldots , n_{lM})$, где~$n_{lj}$ есть число 
каналов, занимаемых соединением на $j$-м звене сети. 

Предположим, что 
запросы на уста\-нов\-ле\-ние в~сети соединения $l$-го класса образуют 
пуассоновский поток интенсивности~$\lambda_l$, причем средняя 
продолжительность соединений $l$-го класса равна $b_l\hm<\infty$. Если при 
поступлении запроса на уста\-нов\-ле\-ние нового соединения в~сети недостаточно 
свободных каналов или уже установлено максимально возможное 
число~$K$~соединений, происходит блокировка запроса. 
  
  Обозначим $v_l(t)$ число соединений $l$-го класса, установленных в~сети 
в~момент~$t$, $\mathbf{v}(t)\hm= (v_1(t), v_2(t),\ldots , v_L(t))$, и~${\sf K}$~--- 
пространство состо\-яний процесса $v(t)$, представляющее со-\linebreak бой множество 
неотрицательных целочисленных\linebreak векторов $\mathbf{k}\hm= (k_1, k_2, \ldots , 
k_L)$, удовлетворяющих неравенствам $k_1\hm+k_2+\cdots + k_L\hm\leq K$ 
и~$k_1\mathbf{n}_1\hm+ k_2\mathbf{n}_2+\cdots + k_L\mathbf{n}_L\hm\leq 
\mathbf{N}$. 

Стационарное распределение $\phi(\mathbf{k})\hm= 
\lim\limits_{t\to\infty} P\{ \mathbf{v}(t)=\mathbf{k}\}$ процесса $\mathbf{v}(t)$ зависит от функций 
распределения продолжительности соединений лишь посредством средних 
значений и~дается следующей формулой~\cite{7-n}:
  \begin{equation}
  \left.
  \begin{array}{rl}
  \phi(\mathbf{0}) &=  \sum\limits_{\mathbf{k}\in{\sf K}} \fr{(\lambda_1 
b_1)^{k_1}}{k_1!} \cdots \fr{(\lambda_L b_L)^{k_L}}{k_L!}\,;\\[6pt]
  \phi(\mathbf{k})& = \phi(\mathbf{0}) \fr{(\lambda_1 b_1)^{k_1}}{k_1!}\cdots 
\fr{(\lambda_L b_L)^{k_L}}{k_L!}\,,\enskip \mathbf{k}\in{\sf K}\,.
\end{array}
\right\}
  \label{e1-n}
  \end{equation}
  
  Пусть $w_m(t)$~--- число занятых в~момент~$t$ каналов $m$-го звена сети 
и~$\mathbf{w}(t)\hm= (w_1(t), w_2(t), \ldots , w_M(t))$. Зная распределение вероятностей 
$\phi(\mathbf{k})$, легко \mbox{найти} совместное распределение числа 
установленных в~сети соединений и~занятых ими каналов:
  \begin{multline*}
  \psi_k(\mathbf{i}) ={}\\[6pt]
  {}=\lim\limits_{t\to\infty} {\sf P} \{ v_1(t)+v_2(t)+\cdots +v_L(t)=k,\ 
\mathbf{w}(t)=\mathbf{i}\}={}\hspace*{-2.2pt}
\end{multline*}

\noindent
\begin{multline}
 {}= \hspace*{-20pt}
\sum\limits_{\substack{{\mathbf{k}\in{\sf K}}\\ {k_1+\cdots +
k_L=k}\\{k_1\mathbf{n}_1+\cdots +k_L\mathbf{n}_L=\mathbf{i}}}} 
\hspace*{-21pt}\phi(\mathbf{k}) =\phi(\mathbf{0}) \hspace*{-23pt}
\sum\limits_{\substack{{\mathbf{k}\in {\sf K}}\\ {k_1+\cdots +
k_L=k}\\{k_1\mathbf{n}_1+\cdots +{k}_{{L}}\mathbf{n}_{{L}}=\mathbf{i}}}} \hspace*{-20pt}
\hspace*{-4.86pt}\fr{(\lambda_1b_1)^{k_1}}{k_1!}\cdots \fr{(\lambda_L b_L)^{k_L}}{k_L!}={}\\
{}=
\phi(\mathbf{0})\fr{\rho^k}{k!} \hspace*{-20pt}
\sum\limits_{\substack{{\mathbf{k}\in {\sf K}}\\ {k_1+\cdots+ 
k_L=k}\\{k_1\mathbf{n}_1+\cdots +{k}_{{L}}\mathbf{n}_{{L}}=\mathbf{i}}}}
\hspace*{-20pt} \fr{k!}{k_1!\!\!\cdots 
k_L!} \left( \fr{\lambda_1b_1}{\rho}\right)^{k_1}\cdots \left( 
\fr{\lambda_Lb_L}{\rho}\right)^{k_L}\!, \\
  \mathbf{i}\in {\sf N}(\mathbf{N})\,, 
\enskip  k=0, 1,\ldots, K\,,
\label{e2-n}
\end{multline}
где $\rho=\lambda_1b_1\hm+ \lambda_2b_2+\cdots +\lambda_Lb_L$. 
Формула~(\ref{e2-n}) допускает простую вероятностную интерпретацию. 
Рассмотрим распределение вероятностей~$\pi(\mathbf{i})$, $\mathbf{i}\hm\in 
{\sf N}(\mathbf{N})$, случайного целочисленного вектора, с~положительной 
вероятностью принимающего лишь значения из подмножества ${\sf L} \hm= 
\{\mathbf{n}_1, \mathbf{n}_2, \ldots ,\mathbf{n}_L\}$ множества~${\sf 
N}(\mathbf{N})$, причем значение~$\mathbf{n}_l$ принимается с~вероятностью 
$\lambda_lb_l/\rho$, т.\,е. 
\begin{equation}
\pi(\mathbf{i}) =\begin{cases}
\fr{\lambda_lb_l}{\rho}\,, &\ \mathbf{i}=\mathbf{n}_l\,;\\
0\,, &\ \mathbf{i}\not={\sf L}\,.
\end{cases}
\label{e3-n}
\end{equation}
С~учетом очевидного равенства $\varphi(\mathbf{0})\hm= \psi_0(\mathbf{0})$ формулу~(\ref{e2-n}) 
можно записать следующим образом:
\begin{equation}
\psi_k(\mathbf{i}) =\psi_0 (\mathbf{0}) \pi^{(k)} (\mathbf{i}) \fr{\rho^k}{k!}\,,\enskip 
\mathbf{i}\in{\sf N}(\mathbf{N})\,,
\label{e4-n}
\end{equation}
где $\pi^{(k)}(\mathbf{i})$ есть $k$-крат\-ная свертка распределения 
вероятностей~(\ref{e3-n}).
  
  Положим $C(\mathrm{r})\hm=0$, если вектор~$\mathbf{r}$ не является 
не\-от\-ри\-ца\-тельным, а для неотрицательных векторов~$\mathbf{r}$ определим 
величины~$C(\mathbf{r})$ следующим образом:
  \begin{equation}
  C(\mathbf{r}) = 1+\sum\limits_{k=1}^K \fr{\rho^k}{k!} 
\sum\limits_{\mathbf{i}\in {\sf N}(\mathbf{r})} \pi^{(k)} (\mathbf{i})\,.
  \label{e5-n}
\end{equation}
Эти величины играют роль нормировочных констант для распределений 
вероятностей~(\ref{e1-n}) и~(\ref{e2-n}), поскольку справедливо равенство:
\begin{equation}
\phi(\mathbf{0}) =\psi_0(\mathbf{0}) =C(\mathbf{N})^{-1}\,.
\label{e6-n}
\end{equation}
  
  Условная вероятность блокировки запроса на установление соединения при 
условии, что запрашивается соединение $l$-го класса, дается следующей 
формулой~\cite{7-n}:
  \begin{equation*}
  B_l = 1- \fr{C(\mathbf{N}-\mathbf{n}_l)}{C(\mathbf{N})}\,.
%  \label{e7-n}
  \end{equation*}
Отсюда вытекает выражение для безусловной вероятности блокировки запроса 
на установление соединения: 
\begin{equation}
B= 1-\fr{1}{C(\mathbf{N})}\sum\limits_{l=1}^L \fr{\lambda_l}{\lambda}\,C\left( 
\mathbf{N} - \mathbf{n}_l\right)\,,
\label{e8-n}
\end{equation}
где $\lambda= \lambda_1\hm+ \lambda_2+\cdots+ \lambda_L$ есть интенсивность 
суммарного потока поступающих запросов.

\section{Ресурсная система массового обслуживания}

Рассмотрим ресурсную СМО с~ресурсами~$M$ типов, на которую поступает 
пуассоновский поток заявок с~параметром~$\lambda$. Обозначим~$R_m$ 
общий объем ресурса типа~$m$ и~$\mathbf{R}\hm= (R_1, R_2,\ldots , R_M)$. 
Поступившая $j$-я заявка характеризуется длительностью обслуживания~$s_j$ 
и вектором объемов необходимых ей ресурсов $\mathbf{r}_j\hm= (r_{j1}, r_{j2}, 
\ldots , r_{jM})$. Случайные векторы $(s_j,\mathbf{r}_j)$, $j\hm=1,2,\ldots$, не 
зависят от моментов поступления заявок, независимы в~совокупности и~имеют 
одинаковую совместную функцию распределения $H(t,\mathbf{x}) \hm= {\sf P}\{ 
s_j\leq t, \mathbf{r}_j\leq \mathbf{x}\}$. \mbox{Обозначим} через $F(\mathbf{x}) \hm= 
{\sf P}\{\mathbf{r}_j\leq \mathbf{x}\}$ функцию распределения объемов требуемых 
заявке ресурсов, $B(t)\hm= {\sf P}\{s_j\leq t\}$~--- функцию распределения 
дли\-тель\-ности обслуживания, $b\hm<\infty$~--- среднюю \mbox{длительность} 
обслуживания и~$\rho\hm= \lambda b$. Для простоты будем считать, что 
$F(\mathbf{R})\hm=1$, т.\,е.\ требование заявкой любого ресурса не превосходит 
его общего объема. 
  
  Состояние рассматриваемой системы в~момент~$t$ можно описать 
случайным процессом $X(t)\hm= (\xi(t),\gamma(t))$. Здесь $\xi(t)$~--- число 
заявок в~системе и~$\gamma(t)\hm= (\gamma_1(t), \ldots , \gamma_M(t))$~--- 
вектор объемов занимаемых ими ресурсов. При условии конечности среднего 
времени обслуживания~$b$ стационарное распределение процесса $X(t)$ 
  \begin{equation}
  \left.
  \begin{array}{rl}
  p_0 &=\lim\limits_{t\to\infty} {\sf P}\{ \xi(t)=0\}\,;\\[6pt] 
  P_k(\mathbf{x}) 
&=\lim\limits_{t\to\infty} {\sf P}\{ \xi(t)=k; \gamma(t)\leq \mathbf{x}\}\,,\\[6pt]
& \hspace*{5mm}\mathbf{0}\leq  \mathbf{x} \leq \mathbf{R}\,,\enskip
 k=0,1,\ldots ,K\,,
\end{array}
\right\}
  \label{e9-n}
\end{equation}
имеет следующий вид:
\begin{equation}
\left.
\begin{array}{rl}
p_0 &=\left( 1+\sum\limits_{k=1}^K G^{(k)} (\mathbf{R}) \fr{\rho^k}{k!}  
\right)^{-1}\,;\\[6pt]
P_k(\mathbf{x}) &=p_0 G^{(k)} (\mathbf{x}) \fr{\rho^k}{k!}\,,\enskip k=1, 2,\ldots, 
K\,.
\end{array}
\right\}
\label{e10-n}
\end{equation}
Здесь $K$~--- максимальное число заявок в~системе и~$G^{(k)}(\mathbf{x})$~--- 
$k$-крат\-ная свертка функции распределения
\begin{equation}
G(\mathbf{x}) = \fr{1}{b}\int\limits_{\mathbf{y}\leq \mathbf{x}} 
\int\limits_0^\infty  tH(dt, d\mathbf{y})\,.
\label{e11-n}
\end{equation}
  
  Из формул~(\ref{e9-n}), в~частности, вытекает сле\-ду\-ющее выражение для 
вероятности потери заявки в~ресурсной СМО:
  \begin{equation*}
  B=1-p_0 \left( 1+ \sum\limits_{k=1}^{K-1} \left( G^{(k)} * F\right)  (\mathbf{R}) 
\fr{\rho^k}{k!}\right)\,,
%  \label{e12-n}
  \end{equation*}
где $(G^{(k)} * F)(\mathbf{x})$ есть свертка функций распределения 
$G^{(k)}(\mathbf{x})$ и~$F(\mathbf{x})$.
  
  Справедливость формул~(\ref{e10-n}) можно установить путем очевидного 
обобщения на произвольное чис\-ло ресурсов результатов работы~\cite{6-n}. 
Ниже будет показано, как для вычисления стационарных характеристик 
ресурсных СМО с~множественными дискретными ресурсами~\cite{2-n} можно 
использовать сети Эрланга. Попутно будут доказаны формулы~(\ref{e10-n}) 
для решетчатых функций распределения объемов ресурсов~$F(\mathbf{x})$.
  
  Определение~(\ref{e11-n}) функции распределения $G(\mathbf{x})$ станет 
понятнее, если ввести условное среднее время обслуживания заявки 
$b(\mathbf{x})$ при условии, что вектор объемов необходимых ей ресурсов 
равен~$\mathbf{x}$. Это условное среднее время обслуживания можно 
определить как функцию, которая при любом векторе~$\mathbf{x}$ 
удовлетворяет следующему уравнению~\cite{9-n}:
  \begin{equation}
  \int\limits_{\mathbf{y}\leq \mathbf{x}}\int\limits_0^\infty tH(dt,d\mathbf{y})= 
\int\limits_{\mathbf{y}\leq \mathbf{x}} b(\mathbf{y}) D(d\mathbf{y})\,.
  \label{e13-n}
  \end{equation}
Используя равенство~(\ref{e13-n}), выражение~(\ref{e11-n}) для функции 
$G(\mathbf{x})$ можно переписать следующим образом:
\begin{equation}
G(\mathbf{x}) = \fr{1}{b}\sum\limits_{\mathbf{y}\leq\mathbf{x}} b(\mathbf{y}) 
F(d\mathbf{y})\,.
\label{e14-n}
\end{equation}

\begin{figure*}
\vspace*{1pt}
 \begin{center}  
\mbox{%
 \epsfxsize=147.923mm
 \epsfbox{nau-1.eps}
 }
\end{center} 
%\vspace*{-9pt}
\noindent
{\small Состояния процесса $\gamma(t)$, объема ресурсов, занятых 
в~ресурсной СМО~(\textit{а}) и~состояния процесса $\mathbf{w}(t)$, числа занятых 
каналов в~сети Келли~(\textit{б})}
\end{figure*}

\section{Связь ресурсных систем массового обслуживания с~сетями 
Эрланга}
  
  В дальнейшем будем считать, что функция распределения требуемых 
объемов ресурсов $F(\mathbf{x})$ является решетчатой с~некоторыми 
координатными шагами $\Delta_1, \Delta_2, \ldots, \Delta_M\hm>0$ и~положим 
$\mathbf{N}\hm= (N_1,N_2, \ldots ,N_M)$, где $N_m$~--- целая часть числа 
$R_m/\Delta_m$. В~этом случае векторы требуемых объемов 
ресурсов~$\mathbf{r}_j$ c~положительной вероятностью могут принимать 
лишь значения вида $\mathbf{Dk}\hm= (k_1\Delta_1, k_2\Delta_2, \ldots , 
k_M\Delta_M)$ где $\mathbf{k}\hm= (k_1, k_2, \ldots , k_M)$~--- целочисленный 
вектор, а~$\mathbf{D}$~--- диагональная матрица с~элементами~$\Delta_i$ на 
диагонали. Поэтому вместо вероятностей $P_k(\mathbf{x})$ удобнее 
рассматривать дискретное распределение вероятностей 
  \begin{multline}
    p_k(\mathbf{i}) =\lim\limits_{t\to\infty} {\sf P}\{\xi(t)=k; \enskip
\gamma(t)=\mathbf{Di}\}\,,\\
\mathbf{i}\in{\sf N}(\mathbf{N})\,,\enskip k=1,2,\ldots ,K\,,
  \label{e15-n}
 \end{multline}
зная которое, легко вычислить вероятности~(\ref{e9-n}):

\noindent
\begin{equation}
\left.
\begin{array}{rl}
p_0 &= p_0(\mathbf{0})\,;\\[6pt] 
P_k(\mathbf{x}) &= \sum\limits_{\substack{{\mathbf{i}\in{\sf 
N}(\mathbf{N})}\\ {\mathbf{Di}\leq \mathbf{x}}}} p_k(\mathbf{i})\,,\\[6pt] 
&\mathbf{0}\leq\mathbf{x}\leq\mathbf{R}\,,\enskip k=1,2,\ldots ,K\,.
\end{array}
\right\}
\label{e16-n}
\end{equation}
  
  Обозначим $f(\mathbf{i})$ вероятность того, что $j$-й заявке требуется 
вектор объемов ресурсов $\mathbf{r}_j\hm= \mathbf{Di}$, и~$b(\mathbf{i})$~--- 
условную среднюю длительность обслуживания \mbox{$j$-й} заявки при условии, что 
$\mathbf{r}_j\hm= \mathbf{Di}$. Перенумеруем все элементы множества ${\sf 
L}\hm= \{ \mathbf{i}\in {\sf N}(\mathbf{N})\vert f(\mathbf{i})>0\}$ и~обозначим 
$\mathbf{n}_l \hm= (n_{l1}, n_{l2},\ldots ,n_{lM})$ его $l$-й элемент. B~этих 
обозначениях функции распределения $F(\mathbf{x})$ и~$G(\mathbf{x})$ 
могут быть записаны следующим образом:
  \begin{equation}
  F(\mathbf{x}) = \sum\limits_{\substack{{\mathbf{i}\in{\sf L}}\\ 
{\mathbf{Di}\leq \mathbf{x}}}} f(\mathbf{i})\,;\enskip
  G(\mathbf{x}) =\fr{1}{b}\sum\limits_{\substack{{\mathbf{i}\in{\sf L}}\\ 
{\mathbf{Di}\leq \mathbf{x}}}} f(\mathbf{i}) b(\mathbf{i})\,.
  \label{e17-n}
 \end{equation}
Таким образом, $G(\mathbf{x})$ является функцией распределения некоторого 
случайного вектора, прини\-ма\-юще\-го значение $\mathbf{Di}$ с~вероятностью 
\begin{equation}
g(\mathbf{i}) = \fr{1}{b}\,f(\mathbf{i})b(\mathbf{i})\,,\enskip \mathbf{i}\in{\sf 
L}\,.
\label{e18-n}
\end{equation}
  
  Введем в~рассмотрение вспомогательную сеть Эрланга, поведение которой 
во времени синхронизировано с~поведением исходной ресурсной СМО. 
В~момент поступления в~СМО заявки, требующей вектор ресурсов 
$\mathbf{Dn}_l$ и~обслуживание в~течение времени~$\tau$, во 
вспомогательную сеть Эрланга поступает запрос на установление соединения 
класса~$l$ продолжительностью~$\tau$, а вектор~$\mathbf{n}_l$ задает 
требование к~числу каналов этого соединения. Моменты поступления заявок 
в~СМО являются моментами установления соединений в~сети, а моменты 
ухода заявок из СМО являются моментами разъединения соединений, и~только 
они. Кроме того, если поступившая в~СМО заявка теряется, то запрос на 
установление соответствующего соединения в~сети также теряется. Ясно, что 
запросы на установление во вспомогательной сети соединений $l$-го класса 
образуют пуассоновский поток интенсивности $\lambda_l\hm= \lambda 
f(\mathbf{n}_l)$, средняя продолжительность соединений $l$-го класса равна 
$b_l\hm= b(\mathbf{n}_l)$, а~распределение вероятностей~(\ref{e3-n}) 
совпадает с~распределением вероятностей~(\ref{e18-n}), т.\,е.\ $\pi(\mathbf{i}) 
\hm= g(\mathbf{i})$, $\mathbf{i}\hm\in {\sf L}$.
  


  Нетрудно видеть, что между процессами~$\xi(t)$ и~$\gamma(t)\hm= 
(\gamma_1(t), \ldots , \gamma_M(t))$, описыва\-ющи\-ми поведение ресурсной 
СМО, и~процессами $\mathbf{v}(t) \hm= (v_1(t), v_2(t), \ldots , v_L(t))$ 
и~$\mathbf{w}(t) \hm= \left(w_1(t), w_2(t), \ldots\right.$\linebreak 
$\left.\ldots, w_M(t)\right)$, описывающими 
вспомогательную сеть Эрланга, существует простая связь, 
проиллюстрированная на рисунке:
  \begin{equation*}
  \xi(t) =v_1(t)+v_2(t)+\cdots+ v_L(t)\,,\enskip \gamma(t)=\mathbf{Dw}(t)\,.
%  \label{e19-n}
  \end{equation*}
    Поэтому для распределений вероятностей~(\ref{e2-n}) и~(\ref{e15-n}) имеет 
место равенство: 
  \begin{equation*}
  p_k(\mathbf{i}) =\psi_k(\mathbf{i})\,,\enskip \mathbf{i}\in{\sf 
N}(\mathbf{N})\,,\enskip k=0,1,\ldots ,K\,,
%  \label{e20-n}
\end{equation*}
и распределение вероятностей~(\ref{e2-n}) можно записать в~следующем виде:
\begin{equation*}
p_k(\mathbf{i}) =p_0(\mathbf{0})g^{(k)} (\mathbf{i}) \fr{\rho^k}{k!}\,,\enskip 
\mathbf{i}\in{\sf N}(\mathbf{N})\,,\enskip k=0,1,\ldots, K\,,
%\label{e21-n}
\end{equation*}
где $g^{(k)}(\mathbf{i})$ есть $k$-крат\-ная свертка распределения 
вероятностей~(\ref{e18-n}). Используя равенства~(\ref{e6-n}) и~(\ref{e8-n}), 
вероятности простоя и~потери заявки можно выразить через нормировочные 
константы~(\ref{e5-n}):
\begin{equation*}
p_0=C(\mathbf{N})^{-1}\,;\enskip B=1-\fr{1}{C(\mathbf{N})}
\sum\limits_{\mathbf{i}\in{\sf N}(\mathbf{N})} C(\mathbf{i}) f(\mathbf{N}-
\mathbf{i})\,.
%\label{e22-n}
\end{equation*}
Кроме того, из формул~(\ref{e4-n}), (\ref{e5-n}), (\ref{e16-n}) и~(\ref{e17-n}) 
вытекают следующие выражения для стационарного распределения числа 
заявок в~системе и~объемов занимаемых ими ресурсов:
\begin{multline*}
p_0^{-1} =1+\sum\limits_{k=1}^K \fr{\rho^k}{k!} \sum\limits_{\mathbf{i}\in{\sf 
N}(\mathbf{N})} g^{(k)}(\mathbf{i}) ={}\\
{}=1+\sum\limits_{k=1}^K \fr{\rho^k}{k!} 
\sum\limits_{\substack{{\mathbf{i}\in{\sf N}(\mathbf{N})}\\ {\mathbf{Di}\leq 
\mathbf{R}}}} g^{(k)} (\mathbf{i}) =1+ \sum\limits_{k=1}^K G^{(k)} 
(\mathbf{R}) \fr{\rho^k}{k!}\,;
\end{multline*}

\vspace*{-12pt}

\noindent
\begin{multline*}
P_k(\mathbf{x}) = p_0 
\sum\limits_{\substack{{\mathbf{i}\in{\sf N}(\mathbf{N})}\\ {\mathbf{Di}\leq 
\mathbf{x}}}} g^{(k)} (\mathbf{i}) \fr{\rho^k}{k!} =p_0 G^{(k)} 
(\mathbf{x})\fr{\rho^k}{k!}\,,\\ k=1,2,\ldots ,K\,.
\end{multline*}

Таким образом, для решетчатых функций распределения требуемых объемов 
ресурсов $F(\mathbf{x})$ доказаны формулы~(\ref{e10-n}). 

\section{Заключение}

  В работе показано, что каждой СМО с~арифметической функцией 
распределения множественных ресурсов соответствует некоторая 
вспомогательная\linebreak сеть Эрланга. Поскольку стационарные распределения 
случайных процессов, описывающих ресурсную СМО и~соответствующую ей 
сеть Эрланга,\linebreak связаны простыми равенствами, для анализа ресурсных СМО 
имеется принципиальная возможность применения известных алгоритмов для 
анализа сетей Эрланга. 
  
  Соответствующие ресурсным СМО сети Эрланга имеют одну особенность, 
не характерную для типичных сетей Эрланга~--- число классов соединений 
вспомогательной сетей Эрланга может быть очень большим. Так, если 
$f(\mathbf{i})\hm>0$ для всех $\mathbf{i}\hm\in {\sf N}(\mathbf{N})$, то 
множество классов соединений сетей Эрланга состоит из всех целочисленных 
векторов $\mathbf{i}\hm= (i_1, i_2, \ldots , i_M)$ в~интервале $\mathbf{0}\hm\leq 
\mathbf{i}\hm\leq \mathbf{N}$ и~число классов соединений вспомогательной 
сети равно произведению $(N_1\hm+1)(N_2\hm+1)\cdots (N_M\hm+1)$. 
Поэтому необходимы дополнительные исследования для того, чтобы выяснить, 
какие точные и~приближенные методы анализа сетей Эрланга применимы 
к~ресурсным СМО.

{\small\frenchspacing
 {%\baselineskip=10.8pt
 \addcontentsline{toc}{section}{References}
 \begin{thebibliography}{9}
 \bibitem{3-n} %1
\Au{Ромм Э.\,Л., Скитович В.\,В.} Об одном обобщении задачи Эрланга~// Автоматика 
и~телемеханика, 1971. №\,6. С.~164--167.
\bibitem{4-n} %2
\Au{Тихоненко О.\,М.} Определение характеристик систем обслуживания с~ограниченной 
памятью~// Автоматика и~телемеханика, 1997. №\,6. С.~105--110.
\bibitem{1-n} %3
\Au{Наумов В.\,А., Самуйлов К.\,Е., Самуйлов~А.\,К.} О~суммарном объеме ресурсов, 
занимаемых обслуживаемыми заявками~// Автоматика и~телемеханика, 2016. №\,8.  
С.~105--110, 125--135.


\bibitem{5-n} %4
\Au{Тихоненко О.\,М., Климович К.\,Г.} Анализ систем обслуживания требований случайной 
длины при ограниченном суммарном объеме~// Проблемы передачи информации, 2001. 
Т.~37. Вып.~1. С.~78--88.
\bibitem{6-n} %5
\Au{Тихоненко О.\,М.} Обобщенная задача Эрланга для сис\-тем обслуживания 
с~ограниченным суммарным объемом~// Проблемы передачи информации, 2005. Т.~41. 
Вып.~3. С.~64--75.
\bibitem{2-n} %6
\Au{Naumov V., Samouylov K., Yarkina~N., Sopin~E., Andreev~S., Samuylov~A.} LTE 
performance analysis using queuing systems with finite resources and random requirements~// 7th 
Congress (International) on Ultra Modern Telecommunications and Control Systems ICUMT-2015  
Proceedings.~--- Piscataway, NJ, USA: IEEE, 2015. P.~100--103.
\bibitem{7-n}
\Au{Kelly F.\,P.} Loss networks~// Ann. App. Probab., 1991. Vol.~1. No.\,3. P.~319--378.
\bibitem{8-n}
\Au{Наумов В.\,А., Самуйлов К.\,Е., Гайдамака~Ю.\,В.} Мультипликативные решения 
конечных цепей Маркова.~--- М.: РУДН, 2015. 159~с.
\bibitem{9-n}
\Au{Гихман И.\,И., Скороход А.\,В.} 
Теория случайных процессов.~--- М.: Наука, 1971.  Т.~1. 664~с.
\end{thebibliography}

 }
 }

\end{multicols}

\vspace*{-3pt}

\hfill{\small\textit{Поступила в~редакцию 29.07.16}}

%\vspace*{8pt}

\newpage

\vspace*{-24pt}

%\hrule

%\vspace*{2pt}

%\hrule

%\vspace*{-24pt}


\def\tit{ON RELATIONSHIP BETWEEN QUEUING SYSTEMS WITH~RESOURCES 
AND~ERLANG NETWORKS}

\def\titkol{On relationship between queuing systems with~resources 
and~Erlang networks}

\def\aut{V.\,A.~Naumov$^1$ and K.\,E.~Samouylov$^{2, 3}$}

\def\autkol{V.\,A.~Naumov and K.\,E.~Samouylov}

\titel{\tit}{\aut}{\autkol}{\titkol}

\vspace*{-9pt}

\noindent
$^1$Service Innovation Research Institute, 30~D L$\ddot{\mbox{o}}$nnrotinkatu, Helsinki 00180, Finland

\noindent
$^2$Peoples' Friendship University of Russia, 6~Miklukho-Maklaya Str., Moscow 117198, Russian Federation

\noindent
$^3$Institute of Informatics Problems, Federal Research Center ``Computer Science and Control'' of the Russian\linebreak 
$\hphantom{^1}$Academy of Sciences, 44-2~Vavilov Str., Moscow 119333, Russian Federation


\def\leftfootline{\small{\textbf{\thepage}
\hfill INFORMATIKA I EE PRIMENENIYA~--- INFORMATICS AND
APPLICATIONS\ \ \ 2016\ \ \ volume~10\ \ \ issue\ 3}
}%
 \def\rightfootline{\small{INFORMATIKA I EE PRIMENENIYA~---
INFORMATICS AND APPLICATIONS\ \ \ 2016\ \ \ volume~10\ \ \ issue\ 3
\hfill \textbf{\thepage}}}

\vspace*{3pt}


\Abste{The paper considers a model of a multiserver queuing system (QS) with 
losses caused by the lack of resources required to service customers. During its 
service, each customer occupies a particular amount of resources of several types. 
Random vectors, describing the requirements of customers to resources, do not 
depend on the arrival process and service times and are mutually independent and 
identically distributed with the general cumulative distribution function. Like in the 
Erlang problem, the task is to calculate the probability of losses of an arriving 
customer caused by the lack of resources. The paper shows the relationship between 
multiservice loss networks and queuing systems with resources, which makes it 
possible to solve the problem of calculating the loss probability in the queuing 
systems with resources using known methods developed for multiservice loss 
networks.}

\KWE{multiservice network; Erlang network; queuing system; queuing system with 
resources; random amount of resources; loss probability; arithmetic probability 
distribution}



\DOI{10.14357/19922264160302}

\vspace*{-9pt}

\Ack
\noindent
The work was partly supported by the Russian Foundation for
Basic Research (projects
16-07-00766, 15-07-03051,   and 15-07-03608).


%\vspace*{3pt}

  \begin{multicols}{2}

\renewcommand{\bibname}{\protect\rmfamily References}
%\renewcommand{\bibname}{\large\protect\rm References}

{\small\frenchspacing
 {%\baselineskip=10.8pt
 \addcontentsline{toc}{section}{References}
 \begin{thebibliography}{9}


\bibitem{3-n-1} %1
\Aue{Romm, E., and V. Skitovitch}. 1971. On certain generalization of problem of Erlang. 
\textit{Automation Remote Control} 32(6):1000--1003.
\bibitem{4-n-1} %2
\Aue{Tikhonenko, O.} 1997. The determination of service characteristics under limited memory. 
\textit{Automation Remote Control} 58(6):969--973.
\bibitem{1-n-1} %3
\Aue{Naumov, V., K.~Samouylov, and A.~Samuylov}. 2016. On the total amount of resources 
occupied by serviced customers. \textit{Automation Remote Control} 77(8): 1419--1427.
\bibitem{5-n-1} %4
\Aue{Tikhonenko, O., and K. Klimovitch}. 2001. Analysis of queuing systems for random-length 
arrivals with limited cumulative volume. \textit{Problems Information Transmission}  
37(1):70--79.
\bibitem{6-n-1} %5
\Aue{Tikhonenko, O.} 2005. Generalized Erlang problem for queueing systems with bounded total 
size. \textit{Problems Information Transmission} 41(3):243--253.
\bibitem{2-n-1} %6
\Aue{Naumov, V., K. Samouylov, N.~Yarkina, E.~Sopin, S.~Andreev, and A.~Samuylov}. 2015. 
LTE performance analysis using queuing systems with finite resources and random requirements. 
\textit{7th  Congress (International) on Ultra Modern Telecommunications and Control Systems 
ICUMT-2015 Proceedings}. Piscataway, NJ: IEEE. 100--103.
\bibitem{7-n-1}
\Aue{Kelly, F.\,P.} 1991. Loss networks. \textit{Ann. App. Probab.} 1(3):319--378.
\bibitem{8-n-1}
\Aue{Naumov, V., Yu.~Gaidamaka, and K.~Samouylov}. 2015. Mul'tiplikativnye resheniya 
konechnykh tsepey Markova [Product form solutiuons for finite Markov' chains]. Moscow: RUDN. 
159~p.
\bibitem{9-n-1}
\Aue{Gihman, I., and A.~Skorohod}. 1971. \textit{The theory of stochastic processes}. New  
York\,--\,Heidelberg\,--\,Berlin: Springer-Verlag. Vol.~I, 1974, 574~p. 
   \end{thebibliography}

 }
 }

\end{multicols}

\vspace*{-9pt}

\hfill{\small\textit{Received July 29, 2016}}

\vspace*{-12pt}


\Contr

\noindent
\textbf{Naumov Valeriy A.} (b.\ 1950)~--- Candidate of Science (PhD) in physics and 
mathematics, Research Director, Service Innovation Research Institute, 30~D 
L$\ddot{\mbox{o}}$nnrotinkatu, Helsinki 00180, Finland; \mbox{valeriy.naumov@pfu.fi} 

\vspace*{3pt}

\noindent
\textbf{Samouylov Konstantin E.} (b.\ 1955)~--- Doctor of Science in technology, 
professor, 
Head of Department, Peoples' Friendship University of Russia, 6~Miklukho-Maklaya Str., 
Moscow 117198, Russian Federation; Institute of Informatics Problems, Federal Research 
Center ``Computer Science and Control'' of the Russian Academy of Sciences, 44-2~Vavilov 
Str., Moscow 119333, Russian Federation; \mbox{ksam@sci.pfu.edu.ru}
\label{end\stat}


\renewcommand{\bibname}{\protect\rm Литература}   %10
\def\stat{buyanov}

\def\tit{РАЗВИТИЕ МАТЕМАТИЧЕСКОЙ МОДЕЛИ УПРАВЛЕНИЯ ГРУЗОПЕРЕВОЗКАМИ 
НА~УЧАСТКЕ ЖЕЛЕЗНОДОРОЖНОЙ СЕТИ С~УЧЕТОМ СЛУЧАЙНЫХ ФАКТОРОВ$^*$}

\def\titkol{Развитие математической модели управления грузоперевозками 
%на~участке железнодорожной сети 
с~учетом случайных факторов}

\def\aut{М.\,В.~Буянов$^1$, С.\,В.~Иванов$^2$, А.\,И.~Кибзун$^3$, А.\,В.~Наумов$^4$}

\def\autkol{М.\,В.~Буянов, С.\,В.~Иванов, А.\,И.~Кибзун, А.\,В.~Наумов}

\titel{\tit}{\aut}{\autkol}{\titkol}

\index{Буянов М.\,В.}
\index{Иванов С.\,В.}
\index{Кибзун А.\,И.}
\index{Наумов А.\,В.}
\index{Buyanov M.\,V.}
\index{Ivanov S.\,V.}
\index{Kibzun A.\,I.}
\index{Naumov A.\,V.}





{\renewcommand{\thefootnote}{\fnsymbol{footnote}} \footnotetext[1]
{Результаты работы получены в~рамках выполнения государственного задания 
Минобрнауки №\,2.2461.2017/ПЧ, а~также при поддержке РФФИ 
и~ОАО <<РЖД>> в~рамках научного проекта №\,17-20-03050~офи\_м\_РЖД.}}


\renewcommand{\thefootnote}{\arabic{footnote}}
\footnotetext[1]{Московский авиационный институт 
(национальный исследовательский 
университет), \mbox{buyanovmikhailv@gmail.com}}
\footnotetext[2]{Московский авиационный институт 
(национальный исследовательский 
университет), \mbox{sergeyivanov89@mail.ru}}
\footnotetext[3]{Московский авиационный институт 
(национальный исследовательский 
университет), \mbox{kibzun@mail.ru}}
\footnotetext[4]{Московский авиационный институт 
(национальный исследовательский 
университет), \mbox{naumovav@mail.ru}}


\vspace*{-18pt}



\Abst{Предлагается математическая модель назначения локомотивов для перевозки 
грузовых составов.
Целью оптимизации в~модели является минимизация числа задействованных для 
перевозки составов локомотивов
за счет выбора маршрутов составов и~локомотивов.
Приводится детерминированный алгоритм получения субоптимального решения,
а~также алгоритм, реализующий схему оперативного планирования. Предлагается 
использование случайного параметра,
моделирующего задержку готовности состава к~отправлению.
Проводится численный эксперимент в~условиях неполной ин\-фор\-ми\-ро\-ван\-ности.
Численный эксперимент проведен на примере данных Московской железной дороги
(МЖД).
Сравниваются результаты, полученные в~детерминированной и~стохастической постановках.}

\KW{математическое моделирование; оптимизация; планирование перевозок; 
оперативное планирование}

\DOI{10.14357/19922264170411} 

\vspace*{-8pt}


\vskip 10pt plus 9pt minus 6pt

\thispagestyle{headings}

\begin{multicols}{2}

\label{st\stat}

\section{Введение}

\vspace*{-4pt}
    
Проблема организации перевозок на железнодорожном транспорте затрагивалась 
во многих работах, среди которых можно выделить~[1--8].
%\cite{AzanovBuyanov, KibzunNaumov, isuzht2015, isuzht2016, belyi, cacchiani, lazarev1, lazarev2}.
В~этих работах описана структура грузовых перевозок на железнодорожном транспорте и~предложены
различные математические модели организации грузовых перевозок.
%В \cite{AzanovBuyanov}
%предложена детерминированная модель оптимизации назначения локомотивов на сформированные
%составы по критерию минимизации общего числа задействованных локомотивов.
В~работе развивается предложенная ранее 
в~\cite{AzanovBuyanov} детерминированная модель
оптимизации назначения локомотивов на сформированные составы.
При этом учитывается влияние на модель случайных факторов, приводящих 
к~задержке готовности состава к~отправлению.
    
Необходимо отметить, что в~процессе осуществления грузовых перевозок 
возникает множество случайных факторов,
влияющих на работу локомотивов, таких как
задержки формирования составов, задержки в~движении поездов, аварии, 
неопределенное поведение диспетчеров, ошибки машинистов и~т.\,д. 
Таким образом, детерминированное решение, полученное в~\cite{AzanovBuyanov}, 
не может быть реализовано на практике и~требуется более реалистичный 
стохастический подход. Однако учет всех существующих случайных факторов 
является очень трудоемкой задачей и~приведет к~большим трудностям при 
получении решения, в~связи с~чем предлагается рассмотреть факторы, влияющие 
на время готовности состава к~отправлению.
    

В работе описываются основные случайные факторы, приводящие к~задержкам по времени,\linebreak
влияющим на время формирования составов.
Предлагается эвристический алгоритм поиска субоптимального решения задачи.
Проводится чис\-лен\-ный эксперимент с~учетом случайных факторов на примере данных 
МЖД.
Приведенные в~статье результаты сравниваются с~решением, полученным
ранее авторами~\cite{AzanovBuyanov} в~детерминированной по\-ста\-новке.

\vspace*{-12pt}

\section{Основные определения и~постановка задачи}

\vspace*{-6pt}

В основе модели, предложенной в~\cite{AzanovBuyanov},
лежит взвешенный ориентированный граф $G \hm= (V, A)$, где $V$~--- 
множество вершин; $A$~--- множество дуг.
Вершинами графа~$G$ являются значимые станции. Значимыми называются 
станции, на которых формируются грузовые составы (сортировочные станции), 
и~станции смены локомотивной тяги.\linebreak Некото\-рые значимые станции являются 
стан\-ци\-ями-де\-по, соответствующее подмножество вершин~$V$ обозначим через~$D$. 
Множеству дуг соответствуют перегоны, соединяющие значимые станции.

Локомотивы могут передвигаться только по определенным маршрутам 
(так называемым плечам), в~связи с~чем в~\cite{AzanovBuyanov} вводятся 
следующие определения.

\smallskip

\noindent
\textbf{Определение 1.}\
    Назовем плечом~$P$ последовательность дуг $a_1,\ldots, a_{I_P}$ графа~$G$, 
     удовлетворяющую следующим условиям:
    \begin{enumerate}[(1)]

    \item
        все дуги различны: $a_i \hm\ne a_j$, $i \hm\ne j$, $i,j \hm\in \{1, I_P\}$;

    \item
        первая вершина первой дуги в~последовательности совпадает 
        с~последней вершиной последней дуги последовательности, является 
        стан\-ци\-ей-де\-по и~отлична от всех промежуточных вершин последовательности: 
        $v_1\hm = v_{I_P}\hm\in D$, $ v_i \hm\ne v_1$ для $i\hm=\overline{2, I_P-1}$.
    
    \end{enumerate}

Также рассматриваются подплечи и~простые подплечи, определенные следующим образом.
    
\smallskip

\noindent
\textbf{Определение 2.}\
  Любую подпоследовательность соседних дуг $a_i, a_{i+1},\ldots, a_j$ 
   $(1\hm\leqslant i\hm<j \hm\leqslant I_P)$, образующих плечо, назовем 
   подплечом данного плеча. Любую дугу $a_i\hm=(v_{i-1},v_i)$, 
   входящую в~некоторое плечо~$P$, назовем простым подплечом плеча~$P$.
    
\smallskip

Пусть $L$~--- множество всех локомотивов, приписанных 
к~рассматриваемым стан\-ци\-ям-де\-по~$D$. Для каждого локомотива 
$l\hm\in L$ задано множество допустимых плеч~$\overline {\mathcal P}_l$, 
по которым он может передвигаться. Каждому множеству  
$\overline {\mathcal P}_l$, $l\hm\in L$, ставится в~соответствие 
множество~$\mathcal P_l$, составленное из всех простых подплеч, 
входящих в~плечи множества~$\overline {\mathcal P}_l$. Предполагается, что 
для каждого локомотива $l\hm\in L$ задана функция весовых 
норм~$w_l (\cdot) \colon {\mathcal P}_l \hm\to \mathbb{R}$, ставящая в~соответствие 
простым подплечам плеча~$\overline {\mathcal P}_l$ максимально допустимую 
для перевозки массу состава.

Пусть $S$~--- множество грузовых составов. Каж\-дый состав $s \hm\in S$ 
характеризуется массой~$w^s$, начальной станцией~$v^s_o$, станцией назначения~$v^s_f$, 
временем формирования~$t^s_o$, временем~$\tau^s_f$, до которого необходимо 
прибыть на станцию назначения, т.\,е.\ каждому составу соответствует пятерка
 $(w^s, v^s_o, t^s_o, v^s_f, \tau^s_f)$. По сути, данные характеристики 
 определяют план перевозок.

Движение локомотивов и~составов по заданному маршруту может 
осуществляться только в~определенные промежутки времени. Совокупность 
маршру\-та и~времени будем называть ниткой. По аналогии с~подплечами и~простыми 
подплечами введем в~рассмотрение поднитки и~простые поднитки. Приведем 
определения нитки, поднитки и~простой поднитки, описанные в~\cite{AzanovBuyanov}.

\smallskip

\noindent
\textbf{Определение 3.}\
    Ниткой $N$ назовем упорядоченное множество четверок 
    $(v_1, t_1, v_2, \tau_2)$, $(v_2, t_2, v_3, \tau_3)$, \ldots, 
    $(v_{I_N-1}, t_{I_N-1}, v_{I_N}, \tau_{I_N})$, удовлетворяющее условиям:
    \begin{enumerate}[(1)]

    \item
        $v_i \in V$, $i \hm= \overline{1, I_N}$, $t_i\hm \in \mathbb{R}$, $i \hm= \overline{1, I_N-1}$, $\tau_i \in \mathbb{R}$, $i = \overline{2, I_N}$;

    \item
        $(v_{i}, v_{i+1}) \in A$, $i \hm= \overline{1, I_N-1}$;

    \item
        $t_i < \tau_{i+1}$, $i\hm = \overline{1, I_N -1}$;

    \item
        $\tau_{i} \leqslant t_{i} $, $i \hm= \overline{2,I_N}$.
        
    \end{enumerate}



Во введенном определении величина~$t_i$ соответствует времени отправления со
 станции~$v_i$, а~$\tau_{i+1}$~--- времени прибытия на станцию~$v_{i+1}$. 
 Приведенные условия выражают естественные свойства движения поездов, 
 заключающиеся в~том, что движение может осуществляться только по 
 перегонам (условия~1 и~2), время отправления со станции не может быть 
 позже времени прибытия на следующую станцию (условие~3), время прибытия 
 на станцию не может быть позже времени отправления с~той же станции (условие~4).

\smallskip

\noindent
\textbf{Определение 4.}\
    Каждую подпоследовательность соседних четверок, образующих нитку~$N$, 
    назовем подниткой. Каждую четверку $(v_i, t_i, v_{i+1}, \tau_{i+1})$, 
    $i\hm=\overline{1, I_N-1}$, составляющую нитку~$N$, назовем прос\-той подниткой.

\smallskip

Пусть имеется множество~$\overline{\mathcal N}$ ниток. Сопоставим каждому элементу~$N$ 
данного множества множество~$\mathcal F(N)$,  являющееся неупорядоченным множеством 
простых подниток, составляющих нитку~$N$. Множество всех простых подниток, 
полученных из множества ниток~$\overline {\mathcal N}$, обозначим через~$\mathcal N$, 
т.\,е.\
\begin{equation*}
%    \label{1.2}
        \mathcal N = \bigcup_{N\in \overline{\mathcal N}} \mathcal F(N)\,.
\end{equation*}

Важно отметить, что каждая простая поднитка проходит только по одной из дуг графа.
    
На множестве $2^L\times \mathcal N$, являющемся декартовым 
произведением всех возможных сочетаний локомотивов и~множества 
простых подниток, определим функцию~$W(\pi_n)$, задающую максимальную массу 
состава, которую может перевезти со\-от\-вет\-ст\-ву\-ющая комбинация локомотивов $\pi_n 
\hm\subset L$ по заданной простой поднитке $n \hm\in \mathcal N$. Очевидно, 
если $\pi_n\hm = l \hm\in L$, где $n \hm= (v,t,v',\tau)$ 
и~$(v,v')\hm\in \mathcal P_l$, то $W(\pi_n) \hm= w_l((v, v'))$. 
Комбинация локомотивов~$\pi_n$ называется составным локомотивом и~используется 
для перевозки состава посредством их совместной работы.
    
Поскольку движение локомотивов осуществляется только по ниткам и~по плечам, 
приведем определение допустимого маршрута оборота локомотива из~\cite{AzanovBuyanov}.
В данном определении учтем также, что локомотив через интервалы времени~$T^{\mathrm{TO}}$ 
(48~ч) должен проходить техосмотр (ТО) продолжительностью~$t^{\mathrm{TO}}$ (8~ч). 
Будем считать, что каждый локомотив $l\hm\in L$ в~начальный момент времени 
характеризуется временем~$\tau_l^{\mathrm{TO}}$, прошедшим с~момента последнего~ТО.
Если локомотив в~начальный момент времени находится на ТО, 
то величина~$\tau_l^{\mathrm{TO}}$ принимает отрицательное значение, равное по модулю 
времени до окончания~ТО.

\smallskip

\noindent
\textbf{Определение 5.}\ %\\label{defMl}
    Допустимым маршрутом оборота~$M_l$  локомотива~$l$ относительно множества 
    плеч~$\overline {\mathcal P}_l$ назовем последовательность прос\-тых 
    подниток $(v_1, t_1, v_2, \tau_2)$,  $(v_2, t_2, v_3, \tau_3)$, 
    \ldots\linebreak $\ldots , (v_{I_l-1}, t_{I_l-1}, v_{I_l}, \tau_{I_l})$, удовлетворяющую условиям:
\begin{enumerate}[(1)]
\item $\tau_{i} \leqslant t_{i}$, $i \hm= \overline{2,I_l -1}$;

\item $(v_i,v_{i+1}) \in\mathcal {P}_l $,  $i \hm= \overline{1,I_l -1}$;

\item существует возрастающая последовательность $i_1, \ldots, i_{f_l}$ 
чисел, выбранных из множества $\{2,3,\ldots,I_l\}$ таким образом, что
\begin{enumerate}[({3}.1)]
\item $\tau^{\mathrm{TO}}_l + \tau_{i_1} \leqslant T^{\mathrm{TO}}$;
    
     \item $t_{i_j}-\tau_{i_j} \geqslant t^{\mathrm{TO}}$, $j\hm=\overline{1, f_l-1}$;
    
        \item $\tau_{i_j} - t_{i_{j-1}} \leqslant T^{\mathrm{TO}}$, $j\hm=\overline{2, f_l}$;
    
        \item $\tau_{I_l} - t_{i_{f_l}} \leqslant T^{\mathrm{TO}}$, если $f_l\hm\ne I_l$.
        
\end{enumerate}
\end{enumerate}

Условие~1 требует, чтобы время прибытия на станцию не было раньше времени 
отправления с~той же станции. Условие~2 ограничивает возможные передвижения 
локомотива только движением по плечам. Условие~3 требует прохождения ТО через 
установленные промежутки времени. Последовательности моментов времени 
$t_{i_1}, \ldots, t_{i_{f_l}}$ соответствуют моментам начала ТО. В~условии~3.1 
требуется, чтобы время ухода на первое ТО не превышало~$T^{\mathrm{TO}}$ 
с~момента предыдущего ТО. Согласно условию~3.2 время прохождения ТО не может 
быть меньше~$t^{\mathrm{TO}}_l$. Из~условия~3.3 
следует, что время между началом движения после ТО и~уходом на сле\-ду\-ющее 
ТО не может быть больше~$T^{\mathrm{TO}}$. Согласно условию~3.4 
время начала движения после последнего ТО должно быть не позже, чем за время~$T^{\mathrm{TO}}$ 
до окончания рассматриваемого периода планирования движения.

Заметим, что маршрут оборота является про\-стран\-ст\-вен\-но-вре\-мен\-н$\acute{\mbox{ы}}$м 
понятием. Множество допустимых маршрутов оборота локомотива~$l$ обозначим 
через~$\mathcal M_l$. Начальную и~конечную станции маршрута оборота~$M_l$ 
обозначим через~$v_o(M_l)$ и~$v_f(M_l)$ соответственно, время начала первой нитки 
данного маршрута оборота обозначим через~$t_o(M_l)$, время прибытия на станцию 
назначения~--- через $\tau_f(M_l)$.

Введем определение допустимого рейса состава, которое так же, 
как маршрут оборота локомотива, является 
про\-стран\-ст\-вен\-но-вре\-мен\-н$\acute{\mbox{о}}$й характеристикой и~было 
ранее описано в~\cite{AzanovBuyanov}.

\smallskip

\noindent
\textbf{Определение~6.}\ 
    Допустимым рейсом~$R_s$ состава $s\hm\in S$ назовем последовательность 
    прос\-тых подниток $(v_1, t_1, v_2, \tau_2)$,  $(v_2, t_2, v_3, \tau_3)$, \ldots$\linebreak $\ldots,
$(v_{I_s-1}, t_{I_s-1}, v_{I_s}, \tau_{I_s})$, удовлетворяющую условиям:
\begin{enumerate}[(1)]
\item  $v_1 = v^s_o$;

\item $v_{I_s} = v^s_f$;

\item  $\tau_{i} \leqslant t_{i}$, $i \hm= \overline{2,I_s -1}$;

\item  $t^s_o \leqslant t_1$;

\item  $\tau^s_f \geqslant \tau_{I_s}$.
\end{enumerate}


Условия 1 и~2 определяют начальную и~конечную станции рейса, 
условие~3 задает естественные ограничения на время отправления и~прибытия, 
условия~4 и~5 требуют выполнения перевозок согласно плану.

Множество допустимых рейсов состава~$s$ обозначим через~$\mathcal R_s$.

Так же как и~для ниток, определим множество~$\mathcal F(M_l)$ всех простых 
подниток, со\-став\-ля\-ющих маршрут оборота~$M_l$ локомотива~$l$, $l\hm\in L$, 
и~множество~$\mathcal F(R_s)$, $s\hm\in S$, всех простых подниток, со\-став\-ля\-ющих 
рейс~$R_s$ состава~$s$.

Для каждой простой поднитки $n\hm\in\mathcal N$ и~каждого набора 
маршрутов оборота локомотивов  $M\hm=\{M_l\}_{l\in L}$ определим 
множество~$\pi_{n}(M)$, со\-став\-лен\-ное из всех локомотивов, передвигающихся 
по простой поднитке~$n$ при наборе маршрутов оборота локомотивов~$M$:
\begin{equation*}
    l\in \pi_{n}(M) \Leftrightarrow n\in\mathcal F\left(M_l\right)\,.
\end{equation*}

Рассмотрим некоторый участок железнодорожной сети с~графом $G\hm = (V, A)$, 
определенным выше. Пусть задано множество локомотивов~$L$, множество составов~$S$, 
множество ниток~$\overline{\mathcal N}$ и~соответствующих простых 
подниток~$\mathcal N$, функция~$W(\cdot)$ весовых норм составных локомотивов. 
Для каждого локомотива $l\hm\in L$ определено множество плеч~$\overline{\mathcal P}_l$ 
и~простых подплеч~$\mathcal P_l$.

В начальный момент времени некоторые локомотивы могут находиться в~движении, 
поэтому будем считать, что локомотив $l\hm\in L$ можно отправить только 
с~некоторой фиксированной станции~$v_0^l$ после момента времени~$t_0^l$.
Пусть  $|L|$~--- число локомотивов в~множестве~$L$, имеющих непустой маршрут оборота.

Пусть для каждого состава $s\hm\in S$ задано множество ниток
 $\overline{\mathcal N_s}\hm\subset \overline{\mathcal N}$, 
 по которым он может быть перевезен. Через~$\mathcal N_s$ обозначим множество 
 соответствующих простых подниток. Данные ограничения
  связаны с~тем, 
 что некоторые нитки могут быть использованы только для перевозки 
 составов определенного рода.

Пусть $M=\{M_l\}_{l\in L}$~--- выбираемый набор маршрутов оборота 
всех локомотивов; $R \hm= \{R_s\}_{s\in S}$~--- выбираемый набор 
рейсов всех со\-ста\-вов; $\mathcal{M}\hm=\{\mathcal M_l\}_{l\in L}$~---
 множество допустимых\linebreak маршрутов оборота всех локомотивов; 
 $\mathcal {R}\hm = \{\mathcal R_s\}_{s\in S}$~--- множество допустимых рейсов
  всех составов.

Требуется найти такой набор $M$ маршрутов оборота локомотивов и~такой набор~$R$ 
рейсов составов, при котором общее число~$|L|$ локомотивов, используемых 
для перевозки составов, будет минимальным, при этом все рейсы 
составов будут покрыты маршрутами локомотивов.
    
В~\cite{AzanovBuyanov} была предложена следующая постановка задачи:

\noindent
\begin{equation}
    \label{problem_main}
    |L| \to \min_{M\in \mathcal M, R\in\mathcal R}
\end{equation}
    при ограничениях
    
    \noindent
\begin{gather}
        M_{l}\in\mathcal M_l\,,\enskip l\in L\,;    \label{c1}\\
    \label{c2}
        R_s \in \mathcal R_s\,,\enskip s\in S\,;\\
    \label{c3}
           \bigcup\limits_{s\in S} \mathcal F(R_s) \subset \bigcup\limits_{l\in L} \mathcal F(M_l)\,;\\
    \label{c4}
        \mathcal F(R_s)\cap \mathcal F(R_{s'}) = \varnothing\,, 
        \enskip s \ne s', \enskip s, s'\in S\,;\\
    \label{cadd1}
        W(\pi_n)\geqslant w^s\,,\enskip n\in\mathcal F(R_s)\,,\enskip s\in S\,;\\
    \label{c5}
        \mathcal F(M_l) \subset \mathcal N\,,\enskip l \in L\,;\\
    \label{c6}
        \mathcal F(R_s) \subset \mathcal N_s\,,\enskip s\in S\,;\\
    \label{c7}
        v_0(M_l) = v_0^l\,;\\
    \label{c8}
        t_0(M_l) \geqslant t_0^l\,.
\end{gather}

Условия~(\ref{c1}) и~(\ref{c2}) значат, что рассматриваются только допустимые 
маршруты локомотивов и~рейсы составов, в~частности те, для которых 
существуют допустимые плечи. Также заметим, что допустимость 
рейсов составов требует, чтобы был выполнен план перевозок в~установленный срок.

Условие~(\ref{c5}) требует, чтобы маршруты оборота локомотивов составлялись 
только из простых подниток, поскольку множество $\bigcup\nolimits_{l\in L} 
\mathcal F(M_l)\hm \subset \mathcal N$ со\-став\-ле\-но из простых подниток, 
входящих в~ка\-кой-ли\-бо маршрут оборота локомотива. Условие~(\ref{c6})\linebreak 
задает аналогичное требование для рейсов составов, а~кроме этого оно 
ограничивает выбор допустимых ниток для перевозки состава. Условие~(\ref{c3}) 
означает, что все простые поднитки, образующие рейс некоторого состава, 
используются для движения некоторого локомотива, т.\,е.\
 все составы перевозятся локомотивами. Также из этого условия 
 следует, что локомотивы могут передвигаться по простым подниткам, 
 по которым не движутся составы. Таким образом, каждой задействованной 
 нитке со\-от\-вет\-ст\-ву\-ет либо состав с~локомотивом (возможно, 
 с~несколькими локомотивами), либо локомотив, движущийся порожняком.

Условие~(\ref{c4}) означает, что рейсы составов не могут пересекаться, т.\,е.\
 одну простую поднитку нельзя использовать для передвижения двух составов. 
 Поскольку локомотивы могут ехать в~сплотке или с~составом (так называемый 
 вспомогательный пробег), то подобное условие для локомотивов отсутствует.

Условие~(\ref{cadd1}) требует, чтобы были выполнены весовые нормы составных 
локомотивов при перевозке составов, т.\,е.\ составной локомотив~$\pi_n$, 
используемый на простой поднитке $n \hm\in \mathcal F(R_s)$, по которой 
перевозится состав~$s$, должен иметь возможность перевозить состав массой~$W(\pi_n)$, 
не меньшей чем масса~$w^s$ состава~$s$.

Условия~(\ref{c7}) и~(\ref{c8}) задают начальное состояние локомотивов.

Заметим также, что множество составов~$S$ и~множество ниток~$\overline{\mathcal N}$ 
определяются суточным планом перевозок и~количеством суток, на которые осуществляется 
планирование.

Сформулированная задача предполагает оптимизацию как по маршрутам оборота локомотивов, 
так и~по рейсам составов. Однако на практике нитки уже сформированы под конкретные 
составы, поэтому в~дальнейшем будем считать, что множество допустимых 
рейсов~$\mathcal R_s$ состава $s\hm\in S$ состоит из одного рейса. 
Таким образом, задача сводится к~назначению локомотивов для перевозки 
составов с~заданными рейсами, т.\,е.\ к~поиску набора маршрутов~$M$.


Для исследования существования решения задачи необходимо определить, 
является ли набор ниток достаточным
для осуществления плана перевозок.
Решение данной задачи сравнимо с~решением исходной задачи. Ниже приведен 
эвристический алгоритм решения задачи,
который в~ряде случаев позволяет находить допустимое решение задачи.
Вопрос о единственности решения не является актуальным с~практической точки зрения,
так как достаточно найти хотя бы одно решение, обеспечивающее минимальное значение 
целевой функции.
Вычислительная сложность данной задачи в~работе не исследуется, однако можно заметить,
что время перебора всех допустимых маршрутов локомотивов и~рейсов составов
зависит экспоненциально от объема исходных данных задачи.

\vspace*{-2pt}

\section{Алгоритм решения детерминированной задачи}

\vspace*{-2pt}

Опишем алгоритм получения субоптимального решения задачи~(\ref{problem_main}).
Будем считать, что рейсы всех составов определены, т.\,е.\ для каждого состава 
определена нитка,
по которой он движется. В~основе алгоритма лежит идея о~максимальном использовании 
локомотива
с~минимальным временем начала движения. Решение предполагает неограниченное число 
локомотивов
в~начальный момент времени в~каждом депо, однако даже такое предположение 
позволяет получить лучший,
в~сравнении с~реаль-\linebreak\vspace*{-12pt}

\pagebreak

\noindent
ным движением, результат.
Для простоты изложения алгоритмов при их построении не учитываются 
ограничения на массу перевозимых составов и~опускается описание алгоритма проведения 
ТО. При наличии ограничений на массу составов необходимо осуществлять поиск не только простых локомотивов $l\in L$,
но и~составных локомотивов, т.\,е.\ комбинаций нескольких локомотивов.

\vspace*{-6pt}
    
\subsection{Алгоритм назначения} \label{find1}

\vspace*{-2pt}
    
Пусть задано непустое множество составов $S \hm= \{s_i\mid i = \overline{1, |S|}\}$ 
с~непустыми рейсами.
Пусть~$v^l_f$ и~$\tau^l$~--- конечная станция маршрута оборота~$M_l$ (либо начальная станция в~случае пустого маршрута)
и время прибытия на эту станцию локомотива $l\hm\in L$. Для корректной работы алгоритма 
необходимо,
чтобы элементы множества~$S$ были упорядочены по возрастанию времени отправления составов.
Данный алгоритм является упрощенной версией алгоритма в~\cite{AzanovBuyanov},
при этом для ряда примеров решения, получаемые с~по\-мощью данных алгоритмов, совпадают.
    
\renewcommand{\figurename}{\protect\bf Алгоритм}

\begin{figure*} 
\hrule

\vspace*{-4pt}

\Caption{\ }  

\vspace*{3pt}  
\hrule

\vspace*{2pt}
        \begin{enumerate}[1.]
        \setcounter{enumi}{-1}
        
        \item
        Полагаем $i:=1, j:=1, k:=1$.
        
        \item
        Зафиксируем локомотив $l_k \in L$, состав $s_i \hm\in S$ и~простую поднитку 
        из рейса состава $n_j \hm= (v_o(n_j), t(n_j), v_f(n_j), \tau(n_j))$, 
        $n_j \hm\in \mathcal F(R_{s_i})$, переходим к~шагу~3.
        
        \item
        Если $k > |L|$, то берем новый локомотив $L:=L \cup \{l_k\}$, 
        $i:=1$, $j:=1$ и~повторяем шаг~2; если $i\hm > |S|$, то переходим 
        к~следующему локомотиву $k := k+1$, $i:=1$, $j:=1$ и~повторяем шаг~2; 
        если $j\hm > |\mathcal F(R_s)|$, то переходим к~следующему составу 
        $i := i+1$, $j:=1$ и~повторяем шаг~2. Переходим к~шагу~3.
        
        \item
        Если $\tau^{l_k} \leqslant t(n_j)$, $(v_o(n_j), v_f(n_j))\hm \in 
        \mathcal P_{l_k}$, переходим к~шагу~4. Иначе переходим к~шагу~2.
        
        \item
        Если $v^{l_k}_f \ne v_o(n_j)$, выполняем поиск нитки~$N^*$ 
        для перегонки локомотива~$l_k$ к~началу простой поднитки~$n_j$, согласно 
        подразд.~3.2. Если $v^{l_k}_f \hm= v_o(n_j)$, полагаем 
        $N^*:=\varnothing$. Если нитка~$N^*$ найдена, переходим к~шагу~5, 
        иначе переходим к~шагу~2.

        \item
        Внесем найденную простую поднитку~$n_j$ и,~если необходимо, соответствующую 
        ей нитку для перегонки~$N^*$ в~маршрут локомотива $M_{l_k} \hm= M_{l_k} 
        \cup N^* \cup \{n_j\}$. Уберем простую поднитку~$n$ из рейса состава 
        $R_{s_i} := R_{s_i} \setminus \{n_j\}$, если $\mathcal F(R_{s_i})\hm = 
        \varnothing$, то уберем состав из множества рассматриваемых
         $S := S \setminus \{s_i\}$. Если $S \hm= \varnothing$, 
         переходим к~шагу~6, иначе переходим к~шагу~2.

        \item
        Окончание алгоритма, получено субоптимальное решение 
        задачи~(\ref{problem_main}).

    \end{enumerate}
    \hrule
 %   \vspace*{4pt}
\end{figure*}

%\vspace*{-9pt}


\vspace*{-6pt}

\subsection{Поиск составной нитки} \label{N*}

\vspace*{-2pt}

Для осуществления перегонки локомотива необходимо найти нитку~$N^*$, 
соединяющую станцию~$v^l_f$, на которой находится локомотив~$l$,\linebreak
 и~станцию~$v_o(n)$, 
с~которой отправляется прос\-тая поднитка~$n$. Пусть~$t(n)$~--- 
время начала движения по простой поднитке~$n$, $\tau(n)$~--- 
время окончания прос\-той поднитки~$n$, а~$\tau^l$~--- 
время остановки локомотива на станции~$v^l_f$. Через~$\mathcal N_a$ 
обозначим множество прос\-тых подниток, соответствующих дуге $a\hm\in A$. 
Сопоставим каждой дуге графа~$G$ весовую характеристику, равную среднему 
времени движения по ней:
\begin{equation}
    \label{w_a}
    w_a = \fr {1}{|\mathcal N_a|}\sum\limits_{n\in \mathcal N_a} (\tau(n) - t(n))\,.
\end{equation}

Полагаем $N^*$ равной нитке, проходящей по кратчайшему пути в~графе~$G$, 
взвешенном согласно~(\ref{w_a}), соединяющему станции~$v^l_f$ и~$v_o(n)$, 
с~временем начала не ранее~$\tau^l$ и~временем окончания не позднее~$t(n)$. 
Для поиска кратчайших путей между вершинами взвешенного ориентированного графа 
можно использовать, например, алгоритм Флой\-да--Уор\-шел\-ла~\cite{Floyd}.

\vspace*{-6pt}

\section{Статистическое моделирование}

\vspace*{-2pt}

В процессе осуществления грузовых перевозок возникает множество случайных 
факторов, влияющих на работу локомотивов, таких как
задержки формирования составов, задержки в~движении поездов и~другие 
нештатные ситуации. 
В~связи с~этим детерминированное решение, полученное в~\cite{AzanovBuyanov}, 
не может быть реализовано на практике. Будем моделировать задержки во времени 
формирования состава.

\renewcommand{\figurename}{\protect\bf Алгоритм}

\begin{figure*}[b] %\label{alg:2} %fig2
\hrule

\vspace*{-4pt}

\Caption{\ }  

\vspace*{3pt}  
\hrule

\vspace*{2pt}

    \begin{enumerate}[1.]
        \setcounter{enumi}{-1}

        \item
        Полагаем $i := 0$.

        \item
        Из множества составов $S_i$ выберем подмножество~$S_i^{\Delta T}$ 
        такое, что для всех $s \hm\in S_i^{\Delta T}$
        выполнено $T_o \hm+ i\Delta T \leqslant t^s_o \hm\leqslant T_o \hm+ 
        (i+1)\Delta T$. Пусть теперь для каждого состава~$s$
        из множества~$S_i^{\Delta T}$ задано время фактической готовности 
        к~отправлению $\tau^s_o \hm= t^s_o \hm+ \xi_s$. Переходим к~шагу~2.

        \item
        Определим рейсы составов. Для каждого состава $s \hm\in S_i^{\Delta T}$ 
        выберем нитку $n \hm\in \overline{\mathcal N_s}$ такую,
        чтобы разница во времени отправления~$t(n)$ по нитке~$n$ 
        и~времени фактической готовности~$\tau_o^s$ состава~$s$
        к~отправлению была минимальной, и~внесем ее в~рейс состава 
        $R_s := R_s \cup \{n\}$.
        Для $s \hm\in S_i \setminus S_i^{\Delta T}$ выберем нитку 
        $n \hm\in \overline{\mathcal N_s}$ такую, чтобы разница
        во времени отправления~$t(n)$ по нитке~$n$ и~планируемого времени 
        формирования~$t_o^s$  состава~$s$ была минимальной,
        и~внесем ее в~рейс состава $R_s := R_s \cup \{n\}$. Переходим к~шагу~3.

        \item
        Для полученного множества составов~$S_i$ с~заданными рейсами 
        и~множества локомотивов~$L$ выполним назначение локомотивов согласно
         подразд.~3.1. Зафиксируем маршруты локомотивов на момент времени~$T_o \hm+ 
         i\Delta T$, т.\,е.\ уберем из маршрутов локомотивов все простые поднитки
          со временем отправления, превышающим $T_o \hm+ i\Delta T$. Переходим к~шагу~4.

        \item
        Примем $i := i+1$. Если $i \hm> [T_m/({\Delta T})]$, переходим к~шагу~5, 
        иначе переходим к~шагу~1.

        \item
        Окончание алгоритма, получено субоптимальное решение.
    \end{enumerate}
    \hrule
%    \vspace*{6pt}
\end{figure*}


Задержки во времени готовности состава к~отправлению могут возникать в~результате 
множества причин.
Такими причинами могут стать, например, ошибки при планировании работы станции, 
включающие в~себя как работу маневровых локомотивов,
так и,~например, неверную очередность формирования и~расформирования составов, 
задержки в~прибытии вагонов (грузов),
участвующих в~составообразовании на станцию отправления, нарушение технических 
нормативов в~результате человеческого фактора,
погодных и~иных явлений и~т.\,д. В общем случае все описанные факторы в~том или 
ином виде приводят к~изменению времени
готовности состава к~отправке. Таким образом, сведем учет всех случайных факторов,
связанных с~задержками по времени, к~одной случайной величине, моделирующей 
задержку формирования состава.
{\looseness=-1

}

Для моделирования случайных задержек по времени будем использовать 
случайную величину~$\xi_s$,
имеющую экспоненциальное распределение с~параметром~$\lambda_s$, 
которое обознается через $E(\lambda_s)$.
Для моделирования случайных задержек в~транспортных системах 
традиционно используют экспоненциально распределенные случайные величины. 
Например, в~\cite{KibzunNaumovUlanov} на основе статистического анализа данных 
был предложен экспоненциальный закон распределения времени задержки 
прибытия в~аэропорт самолета, выполняющего рейс по расписанию.

Помимо долгосрочного планирования перевозок, которое главным образом 
позволяет проводить оценку важных эксплуатационных
показателей\linebreak ра\-бо\-ты железной дороги, также существует так называемое 
оперативное планирование, которое является основным инструментом 
организации железнодорожных перевозок. Отметим, что в~работе~\cite{AzanovBuyanov} 
решается задача долгосрочного пла\-ни\-ро\-вания. 
{\looseness=-1

}

Оперативное планирование 
на железнодорожном транспорте состоит из нескольких этапов. %\\[-16pt]
\begin{enumerate}[1.]
    \item
    Разработка и~утверждение плана перевозок на период времени, равный~$T$. %\\[-16pt]
    \item
    Корректировка сформированного плана перевозок с~учетом фактического расположения 
    локомотивов и~составов,
    а~также общего состояния железнодорожной сети через интервалы времени~$\Delta T$.
\end{enumerate}

%\vspace*{-2pt}

Будем считать, что точное время готовности состава к~отправлению известно 
в~текущий момент времени на период планирования~$\Delta T$,
поэтому схема оперативного управления подвижным составом 
заключается в~последовательном решении детерминированной
задачи формирования маршрутов движения локомотивов согласно алгоритму~1 
через промежутки времени~$\Delta T$
с~учетом точного знания времени готовности составов к~отправлению на время~$\Delta T$ 
вперед и~планового времени го\-тов\-ности
составов к~отправлению в~оставшийся период времени планирования.
 
 При этом в~качестве начальных условий решения задачи о~назначении локомотивов 
каждый раз принимается
фактическое расположение локомотивов на железнодорожной сети, сложившееся на 
момент корректировки
с~учетом всех показателей функционирования локомотивов (необходимость прохождения ТО, 
плечи и~т.\,д.).

\begin{table*}\small %tabl1
\begin{center}
    \Caption{Характеристики входных данных}
\vspace*{2ex}

        \begin{tabular}{|c|c|c|c|c|c|c|c|}
         \hline
\tabcolsep=0pt\begin{tabular}{c}Число\\ станций \end{tabular}& 
\tabcolsep=0pt\begin{tabular}{c}Число\\ станций-депо\end{tabular} &
\tabcolsep=0pt\begin{tabular}{c}Число\\ сортировочных\\ станций\end{tabular}&
\tabcolsep=0pt\begin{tabular}{c}Число\\ составов\\ в~суточном\\ задании\end{tabular} &
\tabcolsep=0pt\begin{tabular}{c}Число\\ ниток\\ на сутки\end{tabular} &
\tabcolsep=0pt\begin{tabular}{c}Период\\ моделирования\\ $T_m$, сут\end{tabular} &
\tabcolsep=0pt\begin{tabular}{c}Дискретность\\ оперативного\\ управления\\
$\Delta T$, ч\end{tabular} \\
\hline
40 & 16  & 16 & 598 & 1254 & 10  & 3 \\ 
\hline
        \end{tabular}
    \end{center}
%\vspace*{-6pt}
\end{table*}

Опишем эвристический алгоритм поиска субоптимального решения задачи 
назначения локомотивов для осуществления грузоперевозок
по железнодорожной сети, реализующий схему оперативного планирования. Пусть~$T_o$~--- 
время начала моделирования.
Пусть~$S_i$~--- множество составов, которые необходимо перевезти в~интервал 
времени $[T_o\hm + i\Delta T, T_o \hm+ i\Delta T \hm+ T]$,
где $i \hm= \overline{0, [{T_m}/({\Delta T}) ]}$. Множества~$S_i$ 
становятся известными за время~$\Delta T$
до начала соответствующего интервала. Пусть для каждого состава $s \hm\in S_i$ 
задано планируемое время формирования~$t^s_o$.
Обозначим через~$\overline{\mathcal N_s}$ множество ниток, по 
которым может быть перевезен состав~$s$
из расчета планируемого времени фор\-ми\-ро\-вания.
{ %\looseness=1

}



Описанный выше алгоритм позволяет получить субоптимальное 
решение задачи~(\ref{problem_main}) при условии неопределенности
во времени готовности состава, используя принцип оперативного планирования. 
Неопределенность заключается в~отсутствии точной информации 
о~времени готовности составов к~отправлению на весь рассматриваемый период 
планирования. Согласно принципу оперативного планирования производится 
многократная корректировка плана перевозок через интервалы времени~$\Delta T$. 
В~каждый рассматриваемый интервал времени имеется точная информация 
о~времени готовности составов только из этого интервала, для остальных 
известно только планируемое время готовности, которое может сильно отличаться 
от фактического.

Проверим адекватность решения, получаемого с~помощью алгоритма~2, 
в~случае стохастической постановки
с~учетом случайного времени формирования составов. Численный эксперимент проводился 
на примере данных участка МЖД за определенный период 
времени. Характеристики входных данных приведены в~табл.~1.



Вычисления были проведены с~учетом ограничений на ТО. Предполагается, 
что локомотивы должны проходить ТО продолжительностью
не менее~8~ч не позже, чем через~48~ч после предыду\-ще\-го~ТО.

%\end{multicols}






%\begin{multicols}{2}

Согласно алгоритму~2, составление рейса состава происходит при помощи выбора 
ближайшей по времени отправления
нитки относительно фактического времени формирования.
Параметр~$\lambda_s$ распределения случайной величины~$\xi_s$ зависит 
от множества факторов, например числа вагонов в~составе,
структуры станции и~др. Так как описанная модель не включает в~себя 
процесс составообразования,
вагонопотоки и~станционные работы, вопрос выбора~$\lambda_s$ остается вне 
рамок данной работы.\linebreak\vspace*{-12pt}


\columnbreak

%\begin{table*}
{\small %tabl2
    \begin{center}
    {{\tablename~2}\ \ \small{Сравнение результатов}}
    
    \vspace*{2ex}


        \begin{tabular}{|l|c|c|c|} 
        \hline
        \multicolumn{1}{|c|}{Вариант} & 
        \tabcolsep=0pt\begin{tabular}{c}Число\\ локомо-\\ тивов\end{tabular} & 
       \tabcolsep=0pt\begin{tabular}{c} Число\\ переве-\\ зенных\\ составов\end{tabular}&
        \tabcolsep=0pt\begin{tabular}{c}Макси-\\мальное\\ время\\ задержки\\ состава, ч\end{tabular}\\
        \hline
  Решение из \cite{AzanovBuyanov} &  369& 5920& 0\hphantom{,3}\\
  $T=12$~ч &  405& 5710& 6,3\\
  $T=24$~ч &  440& 5783& 5,1\\
  $T=48$~ч & 422& 5659& 6,7\\ 
 \hline
        \end{tabular}
    \end{center}}
%\end{table*}

\vspace*{9pt}

\noindent
 Оценки параметров~$\lambda_s$ получены исходя из 
обработки реальных статистических данных.



Численный эксперимент проводился с~исходными данными, представленными в~табл.~1.
Для
 каждого $T \hm\in \{12~\mbox{ч}, 24~\mbox{ч}, 48~\mbox{ч}\}$
выполнено~100~реализаций алгоритма.
Кроме основного критерия, числа используемых локомотивов, также оценены 
выполнение плана перевозок и~максимальное время задержки составов. Для каждого~$T$ представим в~табл.~2 
средние значения для основного критерия и~описанных характеристик.



Из табл.~2 видно, что значение критерия относительно детерминированной постановки 
показало среднее изменение
в большую сторону не более чем на~20\%. При этом менее~7\%~составов не было
 перевезено, что объясняется недостатком ниток.
Максимальное время задержки составов при этом достигало около~7~ч.
В~реальных условиях управления грузовыми перевозками такие составы отправляются 
вне нормативных ниток.
Относительные доли основных характеристик движения локомотивов, а~именно: 
время, проведенное в~работе (полезный пробег);
время, затраченное на проведение ТО; время, затраченное на перегонки 
(холостой пробег), а~также время простоя~--- не изменились.

Общий локомотивный парк на МЖД составляет около~900~локомотивов, 
ежедневно используется примерно~700~локомотивов.
В~результате численного эксперимента в~худшем случае ($T = 24$~ч) 
получено~440~локомотивов,
что в~сравнении с~реальными данными является очень хорошим показателем. Однако 
следует заметить,
что такие показатели, с~одной стороны, связаны с~наличием некоторой части 
неперевезенных составов в~результате полученного решения,
а~с~другой стороны, возможно, не всеми нюансами функционирования системы
 железнодорожных перевозок,
учтенными в~виде ограничений в~рассматриваемой модели. Тем не менее 
полученные результаты показывают, что учет случайных факторов необходим в~подобных 
моделях, поскольку оказывает значительное влияние (порядка~20\%) 
на основные показатели функционирования системы даже с~учетом нового 
предложенного алгоритма оперативного управления.

\vspace*{-6pt}

\section{Заключение}

\vspace*{-2pt}

В работе описана и~исследована математическая модель назначения локомотивов 
для перевозки составов.
Результаты численных экспериментов показали, что случайные возмущения, 
связанные со временем формирования составов,
имеют  влияние порядка~20\% на значение основного критерия и~характеристики
 движения локомотивов,
что подтверждает сравнение с~результатами, полученными в~\cite{AzanovBuyanov},
а~также создает дополнительную проблему~--- нехватку ниток для перевозки 
всех составов.
В~дальнейших исследований планируется изучить влияние других случайных факторов
на эффективность использования локомотивного парка.
Также планируется учесть необходимость прохождения нескольких видов 
технического обслуживания,
ограничения на тип тяги локомотива и~массу перевозимого состава и~другие факторы.

\renewcommand{\figurename}{\protect\bf Рис.}


{\small\frenchspacing
 {%\baselineskip=10.8pt
 \addcontentsline{toc}{section}{References}
 \begin{thebibliography}{99}

\bibitem{belyi}  %1
\Au{Белый О.\,В., Кокурин И.\,М.}
Организация грузовых железнодорожных перевозок: пути оптимизации~// 
Транспорт Российской Федерации, 2011. №\,4(35). С.~28--30.

\bibitem{KibzunNaumov}  %2
\Au{Кибзун А.\,И., Наумов~А.\,В., Иванов~С.\,В.}
Двухуровневая задача оптимизации деятельности железнодорожного транспортного узла~// 
Управление большими сис\-те\-ма\-ми, 2012. №\,38. С.~140--160.

\bibitem{lazarev1} %3
\Au{Лазарев А.\,А., Мусатова Е.\,Г.}
Целочисленные постановки задачи формирования железнодорожных составов и~расписания 
их движения~// Управление большими системами, 2012. №\,38. С.~161--169.

\bibitem{lazarev2} %4
\Au{Лазарев~А.\,А., Мусатова~Е.\,Г., Гафаров~Е.\,Р., Кварацхелия~А.\,Г.}
Теория расписаний. Задачи железнодорожного планирования.~--- М.: ИПУ РАН, 2012.
92~с.

\bibitem{isuzht2015}  %5
\Au{Гайнанов Д.\,Н., Иванов С.\,В., Кибзун А.\,И., Осокин А.\,В.}
Модель оптимального назначения локомотивов при формировании грузовых составов~// 
Интеллектуальные системы управления на железнодорожном транспорте: Тр. 
4-й научн.-технич. конф. с~междунар. учас\-ти\-ем.~--- М.: НИИАС, 2015. С.~45--47.

\bibitem{cacchiani}  %6
\Au{Cacchiani V., Galli~L., Toth~P.}
A~tutorial on non-periodic train timetabling and platforming problems~// 
EURO J.~Transportation Logistics, 2015. Vol.~4. No.\,3. P.~285--320.

\bibitem{AzanovBuyanov}  %7
\Au{Azanov V.\,M., Buyanov~M.\,V., Gaynanov~D.\,N., Ivanov~S.\,V.}
Algorithm and software development to allocate locomotives for transportation 
of freight trains~//
%Вестн. ЮУрГУ. Сер. Матем. моделирование и~программирование, 9:4 (2016), 73-85.
Bull. South Ural State University. 
Ser. Math. Modelling Programming  Computer Software, 2016. 
Vol.~9. No.\,4. P.~73--85.

\bibitem{isuzht2016}  %8
\Au{Азанов В.\,М., Буянов М.\,В., Иванов С.\,В., Кибзун А.\,И., 
Наумов А.\,В., Гайнанов Д.\,Н.}
Оптимизация локомотивного парка, предназначенного для перевозки грузовых составов~// 
Интеллектуальные системы управ\-ле\-ния на железнодорожном транспорте: Тр. 5-й\linebreak 
на\-учн.-тех\-нич. конф. с~междунар. участием.~--- М.: \mbox{НИИАС}, 2016. С.~94--96.
    

\bibitem{Floyd}  %9
\Au{Floyd~R.\,W.} Algorithm~97~--- shortes path~// Comm. ACM, 1962. 
Vol.~5. No.\,6. P.~345.

\bibitem{KibzunNaumovUlanov} %10
\Au{Кибзун А.\,И., Наумов~А.\,В., Уланов~С.\,В.}
Стохастический алгоритм управления летным парком авиакомпании~// 
Автоматика и~телемеханика, 2000. №\,8. С.~126--136.

 \end{thebibliography}

 }
 }

\end{multicols}

\vspace*{-3pt}

\hfill{\small\textit{Поступила в~редакцию 17.04.17}}

\vspace*{8pt}

%\newpage

%\vspace*{-24pt}

\hrule

\vspace*{2pt}

\hrule

%\vspace*{8pt}


\def\tit{DEVELOPMENT OF~THE~MATHEMATICAL MODEL OF~CARGO TRANSPORTATION CONTROL 
ON~A~RAILWAY NETWORK SEGMENT TAKING INTO~ACCOUNT RANDOM FACTORS}

\def\titkol{Development of~the~mathematical model of~cargo transportation control 
%on~a~railway network segment 
taking into~account random factors}

\def\aut{M.\,V.~Buyanov, S.\,V.~Ivanov, A.\,I.~Kibzun, and A.\,V.~Naumov}

\def\autkol{M.\,V.~Buyanov, S.\,V.~Ivanov, A.\,I.~Kibzun, and A.\,V.~Naumov}

\titel{\tit}{\aut}{\autkol}{\titkol}

\vspace*{-9pt}


\noindent
Moscow Aviation Institute (National Research University), 
4~Volokolamskoye Highway,
Moscow 125993, Russian Federation



\def\leftfootline{\small{\textbf{\thepage}
\hfill INFORMATIKA I EE PRIMENENIYA~--- INFORMATICS AND
APPLICATIONS\ \ \ 2017\ \ \ volume~11\ \ \ issue\ 4}
}%
 \def\rightfootline{\small{INFORMATIKA I EE PRIMENENIYA~---
INFORMATICS AND APPLICATIONS\ \ \ 2017\ \ \ volume~11\ \ \ issue\ 4
\hfill \textbf{\thepage}}}

\vspace*{3pt}
    


\Abste{A mathematical model for the assignment of locomotives for the transport 
of freight trains is proposed. In the model, the purpose of optimization is to
 minimize the number of locomotives involved in transportation of trains due 
 to the choice of routes for trains and locomotives. A~deterministic algorithm 
 for obtaining a~suboptimal solution is given as well as an algorithm that 
 implements the operational planning scheme. It is proposed to use\linebreak\vspace*{-12pt}}
 
 \Abstend{a~random parameter that simulates the delay in the readiness of a~train for 
 departure. The numerical experiment was performed in conditions of incomplete 
 information using the data of the Moscow Railway. The results obtained in 
 deterministic and stochastic statements are compared.}

\KWE{mathematical modeling; optimization; transportation planning; 
operational planning}

\DOI{10.14357/19922264170411} 

\vspace*{-12pt}

\Ack
\noindent
This work is a part of Project No.\,2.2461.2017 supported 
by the Russian Ministry of Education and Science.
This work is also supported by the Russian Foundation for Basic
Research and Russian Railways (project 17-20-03050~ofi\_m\_RZhD).





%\vspace*{3pt}

  \begin{multicols}{2}

\renewcommand{\bibname}{\protect\rmfamily References}
%\renewcommand{\bibname}{\large\protect\rm References}

{\small\frenchspacing
 {%\baselineskip=10.8pt
 \addcontentsline{toc}{section}{References}
 \begin{thebibliography}{99}

\bibitem{5-bu-1} %1
\Aue{Belyy, O.\,V., and I.\,M.~Kokurin.} 2011. Organizatsiya gruzovykh 
zheleznodorozhnykh perevozok: puti optimizatsii 
[Organization of freight rail transportation: Ways to optimize].
\textit{Transport Rossiyskoy Federatsii} [Transport of the Russian Federation]
4(35):28--30.
\bibitem{2-bu-1} %2
\Aue{Kibzun, A.\,I., A.\,V.~Naumov, and S.\,V.~Ivanov.} 
2012. Dvukhurovnevaya zadacha optimizatsii deyatel'nosti 
zhe\-lez\-no\-do\-rozh\-no\-go transportnogo uzla [Bilevel optimization problem for 
railway transport hub planning]. \textit{Upravlenie bol'shimi sistemami}
[Large-Scale Systems Control] 38:140--160.
\bibitem{7-bu-1} %3
\Aue{Lazarev, A.\,A. and E.\,G.~Musatova.} 2012. Tselochislennye postanovki 
zadachi formirovaniya zheleznodorozhnykh sostavov 
i~raspisaniya ikh dvizheniya [Integer formulations of
 freight train design and scheduling problems].
 \textit{Upravlenie bol'shimi sistemami} [Large-Scale Systems Control] 38:161--169.
\bibitem{8-bu-1} %4
\Aue{Lazarev, A.\,A., E.\,G.~Musatova, E.\,R.~Gafarov, and A.\,G.~Kvarachelija.} 
2012. \textit{Teoriya raspisaniy. Zadachi zheleznodorozhnogo planirovaniya} 
[Theory of schedules. Railway planning problems]. Moscow: IPU RAN. 92~p.
\bibitem{3-bu-1} %5
\Aue{Gaynanov, D.\,N., S.\,V.~Ivanov, A.\,I.~Kibzun, and A.\,V.~Oso\-kin.} 
2015. Model' optimal'nogo naznacheniya lokomotivov pri formirovanii
 gruzovyh sostavov
[Model of the optimal assignment of locomotives in the formation of freight
trains]. \textit{Tr. 4-y nauchn.-tekhnich. konf. s~mezhdunarodnym uchastiem 
``Intellektual'nye sistemy upravleniya na zheleznodorozhnom transporte''} 
[4th Scientific and Technical Conference with International Participation 
``Intelligent Control Systems in Railway Transport'' Proceedings]. Moscow. 45--47.

\bibitem{6-bu-1} %6
\Aue{Cacchiani, V., L.~Galli, and P.~Toth.} 2015. 
A~tutorial on non-periodic train timetabling and platforming problems. 
\textit{EURO J.~Transportation Logistics} 4(3):285--320.

\bibitem{1-bu-1} %7
\Aue{Azanov, V.\,M., M.\,V.~Buyanov, D.\,N.~Gaynanov, and S.\,V.~Ivanov.} 2016.
Algorithm and software development to allocate locomotives for transportation 
of freight trains. \textit{Bull. South Ural State University. 
Ser. Math. Modelling Programming Computer Software} 9(4):73--85.
\bibitem{4-bu-1} %8
\Aue{Azanov, V.\,M., M.\,V.~Buyanov, S.\,V.~Ivanov, A.\,I.~Kibzun, A.\,V.~Naumov, 
and D.\,N.~Gaynanov.} 2016. Optimizatsiya lokomotivnogo parka, prednaznachennogo 
dlya perevozki gruzovykh sostavov [Optimization of locomotive 
fleet intended for transportation of freight trains]. 
\textit{Tr. 5-y nauchn.-tekhnich. konf. s~mezhdunarodnym uchastiem 
``Intellektual'nye sistemy upravleniya na zheleznodorozhnom transporte''} 
[5th Scientific and Technical Conference with International Participation 
``Intelligent Control Systems in Railway Transport'' Proceedings]. Moscow. 94--96.



\bibitem{9-bu-1} %9
\Aue{Floyd, R.\,W.} 1962. Algorithm~97: Shortes path. \textit{Comm. ACM} 5(6):345.
\bibitem{10-bu-1}
\Au{Kibzun, A.\,I., A.\,V.~Naumov, and S.\,V.~Ulanov.}
 2000. A~stochastic control algorithm for aircraft allocation. 
 \textit{Automat. Rem. Contr.} 61(8):1355--1363.
\end{thebibliography}

 }
 }

\end{multicols}

\vspace*{-6pt}

\hfill{\small\textit{Received April 17, 2017}}

%\vspace*{-10pt}

\Contr

\noindent
\textbf{Buyanov Mikhail V.} (b.\ 1994)~--- 
PhD student, Moscow Aviation Institute (National Research University), 
4~Volokolamskoye Highway, 
Moscow 125993, Russian Federation; \mbox{buyanovmikhailv@gmail.com}

\vspace*{3pt}

\noindent
\textbf{Ivanov Sergey V.} (b.\ 1989)~--- 
Candidate of Science (PhD) in physics and mathematics,  
associate professor, Moscow Aviation Institute (National Research University), 
4~Volokolamskoye Highway,
Moscow 125993, Russian Federation;  \mbox{sergeyivanov89@mail.ru} 


\vspace*{3pt}

\noindent
\textbf{Kibzun Andrey I.} (b.\ 1951)~--- 
Doctor of Science in physics and mathematics, professor,  
Head of Department, Moscow Aviation Institute (National Research University), 
4~Volokolamskoye Highway,
Moscow 125993, Russian Federation;  \mbox{kibzun@mail.ru} 

\vspace*{3pt}

\noindent
\textbf{Naumov Andrey V.} (b.\ 1966)~--- 
Doctor of Science in physics and mathematics, associate professor,  
professor, Moscow Aviation Institute (National Research University), 
4~Volokolamskoye Highway,
Moscow 125993, Russian Federation;  \mbox{naumovav@mail.ru} 
\label{end\stat}


\renewcommand{\bibname}{\protect\rm Литература}   %11
\def\stat{bitugov}

\def\tit{ПРИМЕНЕНИЕ ВЕЙВЛЕТОВ ДЛЯ РАСЧЕТА ЛИНЕЙНЫХ СИСТЕМ УПРАВЛЕНИЯ  
С~СОСРЕДОТОЧЕННЫМИ ПАРАМЕТРАМИ$^*$}

\def\titkol{Применение вейвлетов для расчета линейных систем управления  
с~сосредоточенными параметрами}

\def\aut{Ю.\,И.~Битюков$^1$, Е.\,Н.~Платонов$^2$}


\def\autkol{Ю.\,И.~Битюков, Е.\,Н.~Платонов}

\titel{\tit}{\aut}{\autkol}{\titkol}

\index{Битюков Ю.\,И.}
\index{Платонов Е.\,Н.}
\index{Bityukov Yu.\,I.}
\index{Platonov E.\,N.}



{\renewcommand{\thefootnote}{\fnsymbol{footnote}} \footnotetext[1]
{Результаты работы получены в~рамках выполнения государственного 
задания Минобрнауки №\,2.2461.2017/ПЧ.}}


\renewcommand{\thefootnote}{\arabic{footnote}}
\footnotetext[1]{Московский авиационный институт (национальный исследовательский университет),  
\mbox{yib72@mail.ru}}
\footnotetext[2]{Московский авиационный институт (национальный исследовательский 
университет),  \mbox{en.platonov@gmail.com}}

%\vspace*{-18pt}

\Abst{Задачи многих дисциплин могут привести к~дифференциальным и~интегральным 
уравнениям. В~простых случаях такие уравнения могут быть решены аналитически, 
но в~более сложных приходится находить приближенные решения этих уравнений. 
В~последнее время большую популярность получили методы, основанные на использовании 
вейвлетов. Среди применяемых были вейвлеты Дебеши, койфлеты и~т.\,д. 
Недостаток таких вейвлетов состоит в~том, что у~них нет аналитического выражения. 
Поэтому возникают большие сложности при интегрировании и~дифференцировании выражений, 
содержащих эти вейвлеты. В~данной статье представлены алгоритмы численного 
решения линейных интегральных и~дифференциальных уравнений, основанные на 
сплайн-вейв\-ле\-тах на отрезке. Представленные алгоритмы обобщают 
известные методы, основанные на вейвлетах Хаара, которые являются частным 
случаем сплайн-вейв\-ле\-тов. Результаты статьи применяются для анализа 
линейных систем управ\-ле\-ния с~сосредоточенными параметрами.}

\KW{сплайн-вейвлет; дифференциальные уравнения; интегральные уравнения}

\DOI{10.14357/19922264170412} 


\vskip 10pt plus 9pt minus 6pt

\thispagestyle{headings}

\begin{multicols}{2}

\label{st\stat}

\section{Введение} 

Для численного решения линейных интегральных уравнений традиционно 
применяется метод, основанный на замене интегрального уравнения 
алгебраической системой линейных уравнений с~помощью применения 
квадратурной формулы. Мат\-ри\-ца такой системы имеет большой размер, и,~как следствие, 
для нахождения решения требуется большое число арифметических операций. 

В~\cite{Lepik3} было предложено использовать вейвлеты Хаара 
для приближенного решения интегрального уравнения, что приводило 
к~системе линейных уравнений с~разреженной матрицей. Получаемое приближенное 
решение было ку\-соч\-но-не\-пре\-рыв\-ным. 

В~\cite{Blatov}  показано, что, если использовать вместо вейвлетов Хаара 
сплайн-вейв\-ле\-ты на отрезке, матрица системы линейных уравнений получается 
псевдоразреженной, т.\,е.\ имеет очень много малых по модулю элементов. 

В~данной статье будут обобщены результаты 
работ~\cite{Lepik1, Lepik2, Lepik3, Lepik4, Lepik} и~развиты результаты 
работы~\cite{Blatov} для получения приближенных решений любого класса гладкости 
линейных интегральных и~дифференциальных уравнений. 
В~качестве примера рассмотрим анализ линейной системы управления  с~сосредоточенными 
параметрами.

\section{Сплайн-вейвлеты на~отрезке}

В этом разделе  кратко рассмотрим подход к~построению вейв\-лет-сис\-тем 
на отрезке, предложенный в~\cite{ArticleFinkelstein}. Пусть действительная 
функция~$\varphi $ принадлежит действительному пространству 
$\mathrm{L}^{2} \left({\bf R}\right)$, удовлетво\-ряет равенству
\begin{equation} 
\label{Pr1}
\varphi \left(x\right)\hm=\sqrt{2} \sum\limits_{k\in {\bf Z}}u_{k} 
\varphi \left(2x-k\right)\,,\enskip u_{k} \in \mathbf{R}\,,
\end{equation}
и имеет компактный носитель, содержащийся в~отрезке $[a;b]$. 
Обозначим $\varphi _{jk} (x)\hm=2^{{j}/{2}}\varphi \left(2^{j} x-k\right)$,
$x\hm\in [a;b]$. Функция~$\varphi $ в~теории вейвлетов называ\-ется масштабирующей,
 а~равенство~\eqref{Pr1}~--- масштаб\-ным соотношением~\cite{Frazer}. 
 Ясно, что отличными\linebreak от нуля на отрезке $[a;b]$ будет лишь конечное чис\-ло
  таких функций. Пусть для определенности это будут функции $\varphi _{j,0},
\varphi _{j,1} ,\ldots,\varphi _{j,n_{j} -1}$. 

Если рассмотреть линейные пространства $ V_{j} \hm= \mathrm{lin}
\left\{\varphi _{j,0} ,\varphi _{j,1} ,\ldots,\varphi _{j,n_{j} -1} \right\}$,
$\dim V_{j} \hm=n_{j}$, то  в~силу равенства~\eqref{Pr1}  будет выполняться 
$V_{0} \hm\subset V_{1} \subset \cdots$\linebreak $\cdots \subset L^{2} \left[a;b\right]$. 
Поэтому $\varphi _{j-1,k} \hm=\sum\nolimits _{s=0}^{n_{j} -1}p_{s,k} \varphi _{j,s}$. 

Как и~в~\cite{ArticleFinkelstein}, введем обозначения: 

\noindent
\begin{align*}
\Phi _{j} (x)&=
\left(\varphi _{j,0} (x),\varphi _{j,1} (x),\ldots,\varphi _{j,n_{j} -1} (x)\right)\,;\\
{P}_{j} &=\left(p_{s,k} \right)_{s=0, k=0}^{n_{j} -1, n_{j-1} -1}\,.
\end{align*}
Тогда $\Phi _{j-1}\hm =\Phi _{j} {P}_{j} $. 

\pagebreak

Обозначим символом~$W_{j-1} $ 
ортогональное дополнение к~пространству~$V_{j-1} $ в~пространстве~$V_{j}$. 
Поскольку $V_{j}\hm =V_{j-1} \oplus W_{j-1} $ и~$W_{j-1} \hm\subset V_{j} $, 
то~$W_{j-1} $~--- конечномерное пространство $ W_{j} \hm = 
\mathrm{lin} \left\{\psi _{j,0} ,\psi _{j,1} ,\ldots,\psi _{j,m_{j} -1} \right\}$,
$\dim W_{j} \hm=m_{j}$ и~$\psi _{j-1,k} \hm=
\sum\nolimits_{s=0}^{n_{j} -1}q_{s,k}^{j} \varphi _{j,s}.$ 
Функции~$\psi _{j,k} $ называются вейвлетами, а~пространства~$W_{j} $ 
называются вейв\-лет-про\-стран\-ст\-вами.

Снова введем в~рассмотрение матрицы~\cite{ArticleFinkelstein}:
\begin{align*}
\Psi _{j} (x)&=\left(\psi _{j,0} (x),\psi _{j,1} (x),\ldots,\psi_{j,m_{j} -1}
 (x)\right)\,;\\
{Q}_{j}&=\left(q_{s,k}^{j} \right)_{s=0, k=0}^{n_{j}-1,m_{j-1}-1}\,.
\end{align*}
Тогда $\Psi _{j-1} \hm=\Phi _{j} {Q}_{j} $. 
Следует заметить, что $n_{j} \hm+m_{j} \hm=n_{j+1} $.
Пусть $f\hm\in {L}^{2} (X)$ и~$\Pi _{j} : {L}^{2} (X)\hm\to V_{j} $. Тогда
\begin{multline*}
\Pi _{j} f=\sum\limits_{k=0}^{n_{j} -1}c_{jk} \varphi _{jk}  =
\Pi _{j-1} f+\Pi _{j-1}^{W} f={}\\
{}=\sum\limits_{k=0}^{n_{j-1} -1}c_{j-1,k} \varphi _{j-1,k}  +
\sum\limits_{k=0}^{m_{j-1} -1}d_{j-1,k} \psi _{j-1,k}\,.
\end{multline*}
Данное равенство можно переписать в~матричном виде, если ввести в~рассмотрение 
векторы ${C}_{j} \hm=\left(c_{j,0} ,\dots,c_{j,n_{j} -1} \right)^{\mathrm{T}}$,
${D}_{j} \hm=\left(d_{j,0} ,\dots,d_{j,m_{j} -1} \right)^{\mathrm{T}}$. 
Первый вектор описывает приближение функции~$f$, а~второй вектор представляет 
собой вейв\-лет-ко\-эф\-фи\-ци\-ен\-ты, которые характеризуют отклонение
$\Pi _{j-1} f$ от~$\Pi _{j} f$. Как показано в~\cite{ArticleFinkelstein}, 
имеет место равенство
${C}_{j} \hm={P}_{j} {C}_{j-1} \hm+{Q}_{j} {D}_{j-1}.$
По данному равенству можно восстановить приближение~$\Pi _{j} f$ 
по более грубому приближению~$\Pi _{j-1} f$ и~вейв\-лет-ко\-эф\-фи\-ци\-ен\-там.

Поскольку линейные операторы (проекторы) $V_{j} \hm\to V_{j-1}$,
$V_{j} \hm\to W_{j-1} $ определяются некоторыми матрицами~${A}_{j}$,
${B}_{j}$, то ${C}_{j-1} \hm={A}_{j} {C}_{j}$,
${D}_{j-1} \hm={B}_{j} {C}_{j}$.

Под вейвлет-преобра\-зо\-ва\-ни\-ем функции~$f$ будем понимать 
нахождение векторов ${C}_{0}$, ${D}_{0}$, ${D}_{1} ,
\dots, D_{j-1}$. Матрицы~${Q}_j$ и~${P}_j$ 
известны как фильт\-ры синтеза. Матрицы~${A}_j$ и~${B}_j$ 
известны как фильт\-ры анализа. Множество 
$\{{A}_j, {B}_j, {P}_j,{Q}_j\}$ 
называется банком фильтров.

Как показано в~\cite{ArticleFinkelstein}, между 
матрицами~${A}_{j}$, ${B}_{j}$ и~${P}_{j}$, 
${Q}_{j} $ существует следующая связь:
$$
\begin{pmatrix} {A}_{j} \\ {B}_{j} \end{pmatrix}=
\begin{pmatrix} {P}_{j} {Q}_{j}\end{pmatrix}^{-1}\,.
$$

Посмотрим теперь, как определить мат\-ри\-цу~${Q}_{j}$. 
Введем следующее обозначение. Если ${f}\hm=\left(f_{1} ,\dots, f_{r} \right)$,
${g}\hm=\left(g_{1} ,\dots,g_{r} \right)$~--- 
некоторые векторы, то $[({f},{g})]\hm=
\left(\left(f_{i} ,g_{j} \right)\right)_{i,j=1}^{r} $~--- 
мат\-ри\-ца скалярных произведений.  Как показано в~\cite{ArticleFinkelstein}, 
мат\-ри\-ца~${Q}_{j} $ удовлетворяет следующему уравнению: 
${P}_{j}^{\mathrm{T}} \left[\left(\Phi _{j}, \Phi _{j} 
\right)\right]{Q}_{j} \hm=0.$

Перейдем теперь к~сплайн-вейв\-ле\-там на отрезке. Определим В-сплай\-ны порядка~$n$  
как свертку~\cite{Chui}:
$$
N_{n} =N_{n-1} *N_{0}\,,\quad
N_{0} (x)=\begin{cases} 
1\,, & x\in [0;1)\,; \\ 
0\,, & x\notin [0;1)\,.
\end{cases}
$$
Как показано в~\cite{Chui}, если определить функцию $\varphi (x)\hm=N_{n} (x)$, 
то она удовлетворяет равенству $\varphi (x)\hm=\sum\nolimits _{k=0}^{n+1}
({C_{n+1}^{k} }/{2^{n}}) \varphi (2x\hm-k)$, где $C_{n+1}^k\hm={(n+1)!}/({k!(n\hm+1\hm-k)!})$.
В~\cite{Yurgu} пред\-став\-лен банк фильт\-ров, соответствующий функции  
$\varphi (x)\hm=N_{n} (x)$, а~именно:  справедливы следующие результаты.

\smallskip

\noindent
\textbf{Лемма~2.1.}\
\textit{Функция $\varphi(x)\hm=N_n(x)$ определяет последовательность подпространств}
\begin{multline*}
V_{\alpha,0}\subset V_{\alpha,1}\subset\cdots,\\
V_{\alpha,j}=\mathrm{lin}\left\{\varphi_{j,-n},\varphi_{j,-n+1},\dots,
\varphi_{j,2^j\alpha(n+1)-1}\right\}
\end{multline*}
\textit{пространства} ${L}^2[0;\alpha(n+1)]$, $\alpha\hm=1,2,\ldots$, 
\textit{такую, что} 
$\overline{\bigcup\nolimits_{j=0}^{+\infty}V_{\alpha,j}}\hm={L}^2
[0;\alpha(n+1)]$.

\smallskip

\noindent
\textbf{Лемма~2.2.}\
\textit{Имеет место равенство} 
$\sum\nolimits_{k=-n}^{2^j\alpha (n+1)-1} \varphi_{j,k}(x) \hm\equiv 
2^{{j}/{2}}$, $x\hm\in [0;\alpha (n+1)].
$
\textit{Если}  $V_{\alpha ,j} \hm=V_{\alpha ,j-1} \oplus W_{\alpha ,j-1} $, 
\textit{то} $\dim W_{\alpha ,j-1}\hm =2^{j-1} \alpha (n+1)$.

\smallskip

Пусть $\lambda_{m,k}\hm=\int\nolimits_k^{k+1} N_n(z)N_n(z-m)\,dz$, 
$m\hm=-n,\ldots ,n$, $k\hm=0,1,\ldots ,n$, и~$\omega_{i,k}\hm=\omega_{k,i}
\hm=\sum\nolimits_{s=n-i+1}^n \lambda_{k-i,s}$, 
$\theta_{i,k}\hm=\theta_{k,i}\hm=\sum\nolimits_{s=0}^{n-k} \lambda_{i-k,s}$, 
$1\hm\leqslant i \hm\leqslant k \hm\leqslant n.$ Введем в~рассмотрение 
вектор\linebreak ${p}\hm\in \textbf{R}^{2^j\alpha (n\hm+1)\hm+n}$, 
который определим равенством:
\begin{multline*}
%\label{vecp}
{p}={}\\
{}=\begin{cases}
 \begin{pmatrix} 
 C_{n+1}^{0} \cdots  C_{n+1}^{k}&C_{n+1}^{k} \cdots  C_{n+1}^{0}&0 \cdots 0
 \end{pmatrix}^{\mathrm{T}},
 &\\
 & \hspace*{-30mm}\mbox{ если } n=2k\,;\\
\begin{pmatrix} 
C_{n+1}^{0} \cdots C_{n+1}^{k}&C_{n+1}^{k+1}&C_{n+1}^{k} \cdots  C_{n+1}^{0}&0 \cdots 0
\end{pmatrix}^{\mathrm{T}}\!,\hspace*{-10.94377pt}
&\\
&\hspace*{-30mm} \mbox{ если } n=2k+1\,.
\end{cases}
\end{multline*}
Определим оператор сдвига~$R_s$:  $\textbf{R}^m \hm\rightarrow  \textbf{R}^m$ 
сле\-ду\-ющим правилом:
\begin{multline*}
\!\!\!\!\!R_s {a}=\!\begin{cases} 
\begin{pmatrix} 
\underbrace{0 \cdots 0}_s & a_1 \cdots a_{m-s}\end{pmatrix}^{\mathrm{T}}\!, &\!\!\!\! \mbox{ если } 
0\leqslant s < m\,;\\
\begin{pmatrix} 
a_{|s|+1} \cdots  a_m&0 \cdots 0\end{pmatrix}^{\mathrm{T}}\!, &\!\!\!\! \mbox{ если }
 -m<s<0\,;\hspace*{-10.8pt}\\
 0\,, &\!\!\!\! \mbox{ если } \vert s\vert \geqslant m\,,
 \end{cases}
 \end{multline*}
 где
 $$
{a}=\left(a_1,\dots,a_m\right)^{\mathrm{T}}\,.
$$


%\smallskip

\noindent
\textbf{Лемма 2.3.}\ 
\textit{Матрицы ${P}_j$ и~$[(\Phi_j,\Phi_j)]$ 
для последовательности подпространств  $V_{\alpha,0}\hm\subset V_{\alpha,1}
\subset\cdots$ имеют вид}:

\pagebreak

\noindent
\begin{equation*}
{P}_j=\fr{1}{2^{n+{1}/{2}}}
\begin{pmatrix} 
R_{-n}{p}&R_{-n+2}{p}&\cdots&R_{n-2+2^j\alpha (n+1)}{p}
\end{pmatrix};
\end{equation*}

\vspace*{-12pt}

\noindent
\begin{multline*}
[(\Phi_j,\Phi_j)]=\left( 
{d}_1\  \cdots\  {d}_n\ {q}\ 
R_1{q}\ \cdots\right.\\
\left.\cdots  \ R_{2^j\alpha (n+1)-n-1}
{q}\ {u}_1\ \cdots\ {u}_n\right)^{\mathrm{T}}\,,
\end{multline*}
\textit{где }
\begin{align*}
&{d}_s=
\begin{pmatrix} \omega_{1,s}&\omega_{2,s}&\cdots&\omega_{n,s}&q_{n-s+1}&\cdots&
q_n&0 \cdots 0\end{pmatrix}^{\mathrm{T}}\,;
\\
&{u}_s=
\begin{pmatrix}0 \cdots 0&q_n&\cdots&q_s&\theta_{1,s}&\cdots&\theta_{n,s}\end{pmatrix}^{\mathrm{T}}\,;\\
&{q}=
\begin{pmatrix} q_n & q_{n-1}  \cdots  q_1 & q_0 & q_1  \cdots  q_{n-1} & q_n &
 0 \cdots 0 \end{pmatrix}^{\mathrm{T}}   \in{}\\
& \hspace*{8mm}{}\in  \mathbf{R}^{2^j\alpha (n+1)+n}\,,
 \enskip q_{k} =\left(N_{n}(\cdot),N_{n} (\cdot -k)\right).
\end{align*}
\textit{Матрица, транспонированная к~${T}_j\hm={P}_j^{\mathrm{T}}
 [(\Phi_j,\Phi_j)]\hm=2^{-n-{1}/{2}}
 (t_{i,s})_{i=1,s=1}^{2^{j-1}\alpha (n+1)+n,~2^j\alpha (n+1)+n}$,
имеет вид}:
\begin{multline*}
{T}_j^{\mathrm{T}}=\fr{1}{2^{n+{1}/{2}}}
\left( 
{L}_1 \cdots {L}_n\  \  {w}\ \  R_2 {w} \cdots \right.\\
\cdots
R_{2^j\alpha (n+1)-2n-2} {w}\ \  {L}_{2^{j-1}\alpha (n+1)+1} \cdots \\
\left.\cdots {L}_{2^{j-1}\alpha (n+1)+n}\right)\,,
\end{multline*}
\textit{где} 
\begin{multline*}
\hspace*{-2pt}{w}=\begin{pmatrix}
{p}^{\mathrm{T}} R_{-2n}{q}&{p}^{\mathrm{T}} R_{-2n+1}{q} 
\cdots {p}^{\mathrm{T}} R_{n+1}{q}&0 \cdots 0 
\end{pmatrix}^{\mathrm{T}}
 \in{}\\
 {}\in \mathbf{R}^{2^j\alpha (n+1)+n}\,;
\end{multline*}

\vspace*{-12pt}

\noindent
\begin{multline*}
{L}_i={}\\
{}=\begin{pmatrix} 
\left(R_{-n+2i-2}{p}\right)^{\mathrm{T}} {d}_1 \cdots 
\left(R_{-n+2i-2}{p}\right)^{\mathrm{T}} {d}_n&0 \cdots 0
\end{pmatrix}^{\mathrm{T}}+ {}\\
{}+
\left(R_n\circ R_{-3n+2i-2}\right){w},\enskip
i=1,\dots,n\,;
\end{multline*}

\vspace*{-12pt}

\noindent
\begin{multline*}
\hspace*{-8.5727pt}{L}_{i+1}=\begin{pmatrix} 
0 \cdots 0&\left(R_{-n+2i}{p}\right)^{\mathrm{T}} {u}_1 
\cdots \left(R_{-n+2i}{p}\right)^{\mathrm{T}}{u}_n
\end{pmatrix}^{\mathrm{T}}\!  
+{}\\
{}+\left(R_{-n}\circ R_{-n+2i}\right){w}\,,
\end{multline*}

\vspace*{-12pt}

\noindent
$$
\hspace*{8mm}i=2^{j-1}\alpha (n+1),\dots,n-1+2^{j-1}\alpha (n+1)\,.
$$

С использованием леммы~2.3  в~\cite{Yurgu} найдены $2^{j-1}\alpha (n+1)$ 
линейно независимых решений ${h}_s\hm=
(h_{1,s},h_{2,s},\dots,h_{2^j\alpha (n+1)+n,s})^{\mathrm{T}}$ 
системы линейных уравнений  ${T}_j {h}_s\hm=0$. 
Эти решения и~представляют собой столбцы матрицы ${Q}_j\hm=
({h}_1,\dots,{h}_{2^{j-1}\alpha (n+1)}).$
Столбцы ${h}_{s} $ выбирались таким образом, чтобы функции
$$
\psi _{j-1,s} (x)=\Phi_j(x){h}_s=
\sum\limits_{i=1}^{2^{j} \alpha (n+1)+n}h_{i,s}  \varphi _{j,-n+(i-1)}(x)
$$
по возможности представляли собой сдвинутые версии одной функции, т.\,е.\ 
имели бы одну форму (за исключением, конечно, граничных вейвлетов).  
Введем сокращенные обозначения для матриц, составленных из элементов
 матрицы~${T}_j$:
$$
T_j\left(\begin{smallmatrix} 
i_1,\dots,i_k \\ j_1,\dots,j_m \end{smallmatrix} 
\right) = 
\begin{pmatrix} t_{i_1,j_1} & \cdots & t_{i_1,j_m} \\ 
\vdots & \vdots&\vdots \\ 
t_{i_k,j_1} & \cdots & t_{i_k,j_m} 
\end{pmatrix}.
$$
Для внутренних вейвлетов (носитель содержится в~отрезке $[0;\alpha (n+1)]$):
\begin{multline*}
{h}_s=(0,\dots,0,h_{2s-n-1,s},\dots, h_{2s+2n,s},0,\dots,0)^{\mathrm{T}},
\\
 s=n+1,\dots,2^{j-1}\alpha (n+1)-n\,,
\end{multline*}
 где $ T_j\left(\begin{smallmatrix} s-n,\dots,s+2n \\ 
 2s-n-1,\dots,2n+2s \end{smallmatrix} \right)(h_{2s-n-1,s},\dots,h_{2s+2n,s})^{\mathrm{T}}\hm=0.$

Решения, соответствующие граничным вейвлетам, выбираются следующим образом. 
Для $s\hm=1,2,\dots,n$ положим
$$
{h}_{s}=(0,\dots,0,h_{s,s},\dots,h_{2n+2s,s},0,\dots,0)^{\mathrm{T}},
$$
где $T_j\left(\begin{smallmatrix} 1,\dots,s+2n \\ 
s,\dots,2s+2n\end{smallmatrix} \right)(h_{s,s},\dots,h_{2s+2n,s})^{\mathrm{T}}\hm=0.$
Для $s\hm=2^{j-1}\alpha (n+1)\hm-n+1,\dots ,2^{j-1}\alpha (n+1)$ положим 
\begin{multline*}
{h}_{s}=\left(
0,\dots,0,h_{2s-n-1,s},\dots\right.\\
\left.\dots,h_{2^{j-1}\alpha (n+1)+n+s,s},0,
\dots,0\right)^{\mathrm{T}},
\end{multline*}
где
\begin{multline*}
T_j\left(\begin{smallmatrix} 
s-n,\dots,n+2^{j-1}\alpha (n+1) \\ 2s-n-1,\dots,2^{j-1}\alpha (n+1)+n+s 
\end{smallmatrix} \right)={}\\
{}=\left(h_{2s-n-1,s},\dots,h_{2^{j-1}\alpha (n+1)+n+s,s}\right)^{\mathrm{T}}.
\end{multline*}

Кратко рассмотрим применение вейв\-лет-сис\-тем на отрезке к~построению 
двумерных вейвлетов на прямоугольной области. Пусть даны последовательности $V_{0,i} 
\hm\subset V_{1,i} \subset \cdots \subset V_{j,i} \subset \cdots$ 
конечномерных подпространств ${L}^{2} [a_{i} ;b_{i} ]$, 
масштаби\-ру\-ющие функции~$\varphi ^{(i)} $ и~банки фильтров 
${P}_{j,i}$, ${Q}_{j,i}$, ${A}_{j,i}$, ${B}_{j,i}$, 
$ i\hm=1,2$. Стандартный подход к~построению многомерных вейв\-лет-сис\-тем~--- 
это взятие тензорных произведений функций из одномерных базисов~\cite{Novikov}. 
Определим подпространства $V_{j}^{2} \hm=V_{j,1}\;\otimes$\linebreak
$\otimes\;V_{j,2} \hm= \mathrm{lin}
\left\{f_{1} \otimes f_{2} : f_{1} \hm\in V_{j,1},\ f_{2} \hm\in V_{j,2} \right\}$, 
где функция $f_{1} \otimes f_{2} $ определяется правилом 
$f_{1} \hm\otimes f_{2} \left(x,y\right)\hm=f_{1} (x)f_{2} (y)$. 
Ясно, что функции $\varphi _{j,k}^{(1)} \otimes \varphi _{j,s}^{(2)} $ 
образуют базис в~пространстве~$V_{j}^{2} $.  Вейв\-лет-про\-стран\-ст\-ва~$W_{j}^{2} $ 
определяются следующим образом: 
$$
V_{j}^{2} =V_{j-1}^{2} \oplus W_{j-1}^{2} \,.
$$

Следующие две леммы очевидны.

\begin{figure*}[b] %fig1
\vspace*{1pt}
 \begin{center}
 \mbox{%
 \epsfxsize=162.046mm 
 \epsfbox{bit-1.eps}
 }
 \end{center}
\vspace*{-9pt}
\Caption{Графики функций $w_l$ для $n\hm=5$}
\end{figure*}


\noindent
\textbf{Лемма 2.4.}\ 
\textit{Пусть $f\hm\in {L}^{2}[0;n+1]$, тогда 
$\Pi _{j} f=\Phi _{j} {C}_{j}^{*} $, где
 ${C}_{j}^{*}\hm =[(\Phi _{j} ,\Phi _{j})]^{-1}[(f,\Phi _{j})]$. 
 При этом} 
 $$
 \| f-\Pi _{j} f\| _{{L}^{2} }^{2} =
 \| f\| _{{L}^{2} }^{2} -\left[(f,\Phi _{j})\right]^{\mathrm{T}}
 \left[(\Phi _{j} ,\Phi _{j})\right]\left[(f,\Phi _{j})\right].
 $$

\smallskip

\noindent
\textbf{Лемма 2.5.}\ 
\textit{Пусть $f\in {L}^2([a_1;b_1]\times [a_2;b_2])$ 
и~$\Pi_j^{(2)} : {L}^2([a_1;b_1]\times [a_2;b_2])\hm\to V_j^2$~--- 
проектор. Если}
\begin{multline*}
%\label{G35}
{G}_j={}\\
{}=\left(\int\limits_{\,\,\,a_1}^{b_1}\,dx\!
\int\limits_{a_2}^{b_2}\!\varphi_{j,s}^{(1)}(x)\varphi_{j,k}^{(2)}(y)f(x,y)\,dy
\right)_{s,k=0}^{n_{j,1}-1,n_{j,2}-1},\hspace*{-5.4785pt}
\end{multline*}
\textit{то $\Pi_j^{(2)}f (x,y)\hm=\Phi_j^{(1)}(x)\mathrm{C}_j (\Phi_j^{(2)}
(y))^{\mathrm{T}}$, где $\Phi_j^{(i)}\hm=(\varphi_{j,0}^{(i)}\cdots \varphi_{j,n_{j,i}-1}^{(i)})$, 
а~матрица~${C}_j$ определяется равенством}:
\begin{equation*}
%\label{Cj35}
{C}_j=\left[\left(\Phi_j^{(1)},\Phi_j^{(1)}\right)\right]^{-1}
{G}_j\left[\left(\Phi_j^{(2)},\Phi_j^{(2)}\right)\right]^{-1}\,.
\end{equation*}

%\vspace*{-24pt}

\section{Интегралы от~сплайн-вейвлетов}

Пусть ${Q}_j=({h}_1^j,\dots,{h}_{2^{j-1}(n+1)}^j)$, 
где ${h}_s^j\hm=(h_{1,s}^j,h_{2,s}^j,\dots,h_{2^j (n+1)+n,s}^j)^{\mathrm{T}}$. 
Тогда,   согласно результатам предыдущего раздела,

\noindent
\begin{multline}
\label{U31}
\psi_{j-1,s}(x)={}\\
{}=
\begin{cases}
\displaystyle\sum\limits_{i=s}^{2s+2n} \! h_{i,s}^j\varphi_{j,-n+i-1}(x)\,,&\hspace*{-10mm}
s=1,\dots,n\,;\\
\displaystyle\sum\limits_{i=2s-n-1}^{2s+2n}\!\!
h_{i,s}^j\varphi_{j,-n+i-1}(x)\,,&\\ 
&\hspace*{-38mm}s=n+1,\dots,2^{j-1}(n+1)-n\,;
\\
%\label{U33}
\displaystyle\sum\limits_{i=2s-n-1}^{2^{j-1}(n+1)+n+s}\!\!\!\!
h_{i,s}^j
\varphi_{j,-n+i-1}(x)\,,&\\
&\hspace*{-50mm}s=2^{j-1}(n+1)-n+1,\dots,2^{j-1}(n+1)\,.
\end{cases}
\end{multline}


Так же, как и~в~работах~\cite{Lepik1, Lepik2, Lepik3, Lepik4, Lepik}, для удобства  
введем следующие обозначения:
\begin{align*}
&w_l(x)=\varphi_{0,l-n-1}\,,\enskip l=1,2,\dots,2n+1\,,\\
&w_l(x)=\psi_{j,s}(x)\,,\enskip l=2^j(n+1)+n+s\,, \\
&\hspace*{23mm} j=0,1,\dots\,, \ s=1,\dots,2^{j}(n+1)\,.
\end{align*}
На рис.~1 представлены графики некоторых функций~$w_l$ для случая $n\hm=5$.




Пусть $J\geqslant 0$,  $\Pi_{J} : {L}^2[0;n+1]\hm\to V_{J}$~--- 
проектор и~$M\hm=2^{J}(n+1)+n$. Обозначим ${H}_J\hm=
\begin{pmatrix}w_1 & \cdots & w_M
\end{pmatrix}$ и~введем в~рассмотрение матрицу скалярных 
произведений $[({H}_J, {H}_J)]$. 
В~лемме~2.3 представлены матрицы скалярных произведений 
$[(\Phi_k, \Phi_k)]$ для всех $k\hm=0,1,\dots$ Замечая, что $\Psi_k \hm= 
\Phi_{k+1}{Q}_{k+1}$ и~$[(\Psi_k,\Psi_k)]\hm={Q}_{k+1}^{\mathrm{T}}
[( \Phi_{k+1}, \Phi_{k+1})]{Q}_{k+1}$,
получаем матрицу:
{\small \begin{multline*}
\left[({H}_J, {H}_J)\right]={}\\
\!{}=\!
\begin{pmatrix}
\left[(\Phi_0, \Phi_0)\right] & 0 & 0 & \cdots & 0\\
0 & {Q}_{1}^{\mathrm{T}}\left[( \Phi_{1}, \Phi_{1})\right]{Q}_{1} & 0 & \cdots & 0 \\
\vdots & \ddots & \ddots & \ddots & \vdots \\
0 & 0 & 0 & \cdots & {Q}_{J}^{\mathrm{T}}\left[( \Phi_{J}, \Phi_{J})\right]{Q}_{J}
\end{pmatrix}\!.\hspace*{-12.6139pt}
\end{multline*}
}

\noindent
Так как $V_{J}\hm=V_0\oplus W_0\oplus V_1\oplus\dots\oplus W_{J-1},$ то для 
$f\hm\in {L}^2[0;n+1]$ имеем
$\Pi_{J} f\hm=\sum\nolimits_{l=1}^{M} c_l w_l \hm= {H}_{J}{C}_{J}$,
где ${C}_J\hm=
\begin{pmatrix} c_1 & \cdots\ c_{M} \end{pmatrix}^{\mathrm{T}}$.
Как и~в~работах~\cite{Lepik1, Lepik2, Lepik3, Lepik4, Lepik}, определим функции:
\begin{equation}
\label{U38}
{\xi}_{1,l}(x)=\int\limits_0^x w_l (t)\,dt\,;
\end{equation}

%\vspace*{-12pt}

\noindent
\begin{multline}
\label{U38-1}
{\xi}_{\nu+1,l}(x)=\int\limits_0^x {\xi}_{\nu,l}(t)\,dt={}\\
{}=
\fr{1}{\nu !}\int\limits_0^x (x-t)^{\nu} w_l (t)\,dt\,,
\enskip \nu = 1,2,\ldots
\end{multline}
Согласно определению функций~$w_l$ и~равенст\-вам~(\ref{U31})  
функция ${\xi}_{\nu+1,l}(x)$ представляет собой линейную комбинацию функций
$$
\eta_{n,\nu}^{j,s}(x)=\int\limits_0^x (x-t)^{\nu} N_{n}(2^j t-s)\,dt\,.
$$

\vspace*{-1pt}

\noindent
\textbf{Лемма 3.1.}\ 
\textit{Имеет место следующее рекуррентное соотношение}:

\noindent
\begin{multline}
\label{U40}
\eta_{n,\nu}^{j,s}(x)=\fr{x^{\nu+1}}{\nu+1}\, N_n(-s)+{}\\
{}+
\fr{2^j}{\nu+1}\left(\eta_{n-1,\nu+1}^{j,s}(x)-\eta_{n-1,\nu+1}^{j,s+1}(x)\right)\,,
\end{multline}
\textit{где}

\noindent
\begin{multline}
\label{U41}
\eta_{0,\nu}^{j,s}(x) = {}\\
\!\!\!{}=
\begin{cases}
\fr{(x-a)^{\nu+1} - (x-b)^{\nu+1}}{\nu+1}, &\\[6pt]
& \hspace*{-37mm}\mbox{если } [a;b]=[0;x]
\cap\left[\fr{s}{2^j};\fr{s+1}{2^j}\right]\not= \varnothing\,;\\[9pt] 
0, & \hspace*{-37mm}\mbox{если } [a;b]=[0;x]\cap\left[\fr{s}{2^j};\fr{s+1}{2^j}\right]= \varnothing\,.
\end{cases}
\end{multline}


\noindent
Д\,о\,к\,а\,з\,а\,т\,е\,л\,ь\,с\,т\,в\,о\,.\ \
По свойству В-сплай\-нов~\cite{Chui}  имеет место равенство:
\begin{equation*}
%\label{BSPL}
N_n'(x)=N_{n-1}(x)-N_{n-1}(x-1)\,.
\end{equation*}
Следовательно, по формуле интегрирования по час\-тям  получаем:

\noindent
\begin{multline*}
\eta_{0,\nu}^{j,s}(x)=\left.- \fr{(x-t)^{\nu+1}}{\nu+1} 
N_n(2^j t-s)\right|_0^x+{}\\
{}+2^j \int\limits_0^x \fr{(x-t)^{\nu+1}}{\nu+1} \left(N_{n-1}(2^jt-s)-{}\right.\\
\left.{}-
N_{n-1}(2^jt-s-1)\right)\,dt=
\fr{x^{\nu+1}}{\nu+1}\, N_n(-s)+{}\\
{}+\fr{2^j}{\nu+1}\left(\eta_{n-1,\nu+1}^{j,s}(x)-
\eta_{n-1,\nu+1}^{j,s+1}(x)\right).
\end{multline*}
Равенство~(\ref{U41}) очевидно.~\hfill$\square$

\vspace*{2pt}

Формулы (\ref{U40}) и~(\ref{U41}) позволяют находить значение 
функции $\eta_{n,\nu}^{j,s}(x)$ в~любой точке без интегрирования. Итак, 
для $l\hm=1,2,\dots, 2n+1$ получаем:

\columnbreak

\noindent
$$
{\xi}_{\nu+1,l}(x)=\fr{1}{\nu !}\, \eta_{n,\nu}^{0,l-n-1}(x)\,,\enskip
l=1,2,\dots, 2n+1\,.
$$
Для $l=2^j(n+1)\hm+n\hm+s$, $j\hm=0,1,\dots$, $s\hm=1,\ldots$\linebreak$\ldots,2^{j}(n\hm+1)$ получаем:
\begin{multline*}
{\xi}_{\nu+1,l}(x)={}\\
{}=
\begin{cases}
\fr{2^{({j+1})/{2}}}{\nu !}\sum\limits_{i=s}^{2s+2n}
h_{i,s}^{j+1}\eta_{n,\nu}^{j+1,-n+i-1}(x)\,, &\\
& \hspace*{-17mm}s=1,\ldots,n; \\
\fr{2^{({j+1})/{2}}}{\nu !}\sum\limits_{i=2s-n-1}^{2s+2n}
\!\! h_{i,s}^{j+1}\eta_{n,\nu}^{j+1,-n+i-1}(x)\,, &\\
& \hspace*{-40mm}s=n+1,\ldots,2^{j}(n+1)-n;\\
\fr{2^{({j+1})/{2}}}{\nu !}\sum\limits_{i=2s-n-1}^{2^{j}(n+1)+n+s}\!\!\!\!
h_{i,s}^{j+1}\eta_{n,\nu}^{j+1,-n+i-1}(x)\,, &\\
& \hspace*{-54mm}s=2^{j}(n+1)-n+1,\ldots,2^{j}(n+1).
\end{cases}
\end{multline*}
Полученные равенства справедливы при всех $\nu \hm= 0,1,\dots$

\section{Применение сплайн-вейвлетов к~решению 
линейных интегральных и~дифференциальных уравнений}

В проекционных методах решения линейных уравнений рассматриваются два 
уравнения~\cite{Akilov}: 
первое~--- в~полном нормированном пространстве~$X$:
\begin{equation}
\label{Ur1}
Kx\equiv x-\lambda Hx=f\,;
\end{equation}
второе~--- в~его полном подпространстве~$V_j$:
\begin{equation}
\label{Ur2}
K_j x_j\equiv x_j-\lambda H_j x_j=\Pi_j f\,,
\end{equation}
где $H$~--- непрерывный линейный оператор в~$X$; $H_j$~--- 
непрерывный линейный оператор в~$V_j$. Уравнение~(\ref{Ur1}) называется точным, 
а~уравнение~(\ref{Ur2})~--- приближенным. При этом предполагается, что 
выполнены следующие условия.

\noindent \textbf{1. Условие близости операторов $H$ и~$H_j$.} 
Для любого $x_j\hm\in V_j$ выполняется $\|\Pi_j H x_j \hm- H_j x_j\|
\hm\leqslant \rho_j \|x_j\|$.

\noindent \textbf{2. Условие хорошей аппроксимации элементов 
вида~$Hx$ элементами из~$V_j$.} Для любого $x\hm\in X$ существует $x_j\hm\in V_j$ 
такой, что $\|Hx-x_j\|\hm\leqslant \rho_{1,j} \|x\|$.

\noindent \textbf{3. Условие хорошей аппроксимации свободного члена 
точного уравнения.} Существует элемент $f_j \hm\in V_j$ такой, что $\|f-f_j\|
\hm\leqslant \rho_{2,j}\|f\|$. В~отличие от предыду\-щих условий $\rho_{2,j}$ 
здесь, вообще говоря, зависит от~$f$.

Как показано в~\cite{Akilov}, если оператор~$K$ 
имеет непрерывный обратный, уравнение~(\ref{Ur1}) имеет решение 
и~$\lim\limits_{j\to +\infty} \rho_j \hm= 0$,  $\lim\limits_{j\to +\infty} \rho_{1,j} \hm= 
0$,  $\lim\limits_{j\to +\infty} \rho_{2,j} \hm= 0$, то  
$\lim\limits_{j\to +\infty} \|x-x_j\|\hm = 0$, где $x_j$~--- 
решение уравнения~(\ref{Ur2}).

Рассмотрим сначала линейное интегральное уравнение Фредгольма 2-го рода. 
С~помощью замены переменной такое уравнение можно свести к~следующему:
\begin{equation*}
%\label{U313}
u(x)-\lambda\int\limits_{0}^{n+1}U(x,t)u(t)\,dt =f(x)\,,\enskip x,t\in [0;n+1]\,.
\end{equation*}

Пусть $\varphi(x)=N_n(x)$, $V_{0} \subset V_{1} \subset\cdots$~--- 
соответствующая последовательность конечномерных подпространств пространства
 ${L}^{2} \left[0;n+1\right]$. Пусть $X\hm={L}^2[0;n+1]$. 
 Операторы $K:X\hm\to X$, $H:X\hm\to X$ и~$H_j : V_j\hm\to V_j$ определим равенствами:
\begin{align*}
Ku(\cdot)&=u(\cdot)-\lambda\int\limits_0^{n+1} U(\cdot,t)u(t)\,dt\,;\\
Hu(\cdot)&=\int\limits_0^{n+1} U(\cdot,t)u(t)\,dt\,;\\
H_j&=\Pi_j\circ H\,,
\end{align*}
где $U\in {L}^2([0;n+1]^2)$. Условие близости операторов~$H$ и~$H_j$ 
выполняется с~$\rho_j\hm=0$. Пусть  $u\hm\in X$ и~$u_j(\cdot)\hm=\int\nolimits_0^{n+1} 
\Pi^{(2)}U(\cdot,t)u(t)\,dt\hm\in V_j$. То\-гда $\rho_{1,j}\hm=
\|U-\Pi_j^{(2)}U\|_{{L}^2([0;n+1]^2)}$ 
и~$\lim\nolimits_{j\to +\infty} \rho_{1,j}\hm=0$. Следовательно, 
условие хорошей аппроксимации элементов вида~$Hu$ элементами из~$V_j$ 
также выполняется. Наконец, для произвольного $f\hm\in X$, $f\hm\ne 0$, 
возьмем $f_j\hm=\Pi_j f$, 
а~$\rho_{2,j}\hm={\|f-\Pi_j f\|_{{L}^2([0;n+1])}}/{\|f\|_{{L}^2([0;n+1])}}$. Тогда $\lim\limits_{j\to +\infty} \rho_{2,j}=0$. 

Решение приближенного уравнения
\begin{equation}
\label{Ur3}
u_J-\lambda\Pi_j\circ H u_J=\Pi_J f
\end{equation}
будем искать в~виде $u_J\hm=\sum\nolimits_{l=1}^{M} c_l w_l 
\hm= {H}_J{C}_J $, где $M\hm=2^{J}(n+1)+n$. 
Тогда уравнение~(\ref{Ur3}) можно переписать в~виде
системы линейных уравнений для определения коэффициентов~$c_l$:
\begin{multline*}
%\label{U315}
\sum\limits_{l=1}^{M} c_l (w_l,w_s)-{}\\
{}-\lambda
\sum\limits_{l=1}^{M} c_l \int\limits_{0}^{n+1}dx
\int\limits_{0}^{n+1}U(x,t)w_l(t)w_s(x)\,dt = (f,w_s)\,,\\
s=1,2,\dots,M\,.
\end{multline*}
Это и~есть система метода Галеркина. Перепишем ее в~матричном виде:
\begin{equation}
\label{U315}
{C}_J\left([({H}_J,{H}_J)]-
\lambda{G}_J\right)={F}_J\,,
\end{equation}
где 
\begin{align*}
{C}_J&=\begin{pmatrix}
c_1&\cdots &c_M\end{pmatrix}\,;\\ 
{F}_J&=\begin{pmatrix}(f,w_1)&\cdots &(f,w_M)\end{pmatrix}^{\mathrm{T}}\,;\\
{G}_J&=\left(g_{l,s}\right)_{l,s=1}^M\,,
\\ 
&\hspace*{10mm}g_{l,s}= \int\limits_{0}^{n+1}dx\int\limits_{0}^{n+1}U(x,t)w_l(t)w_s(x)\,dt.
\end{align*}

Аналогично рассматривается линейное интегральное уравнение Вольтерра 2-го рода
$$
u(x)-\lambda\int\limits_{0}^{x}U(x,t)u(t)\,dt =f(x)\,,\enskip 
x,t\in [0;n+1]\,.
$$
Операторы $K:X\hm\to X$, $H:X\hm\to X$ и~$H_j : V_j\hm\to V_j$ определим равенствами:
\begin{align*}
Ku(x)&=u(x)-\lambda\int\limits_0^{x} U(x,t)u(t)\,dt\,;\\
Hu(x)&=\int\limits_0^{x} U(x,t)u(t)\,dt\,;\\
H_j&=\Pi_j\circ H\,.
\end{align*}
Величины $\rho_j$, $\rho_{1,j}$ и~$\rho_{2,j}$ остаются теми же, что и~для 
уравнения Фредгольма.  Таким образом,
система Галеркина для данного уравнения имеет вид:
\begin{multline}
\label{U316}
\sum\limits_{l=1}^{M} c_l (w_l,w_s)-{}\\
{}-
\lambda\sum\limits_{l=1}^{M} 
c_l \int\limits_{0}^{n+1}dx\int\limits_{0}^{x}U(x,t)w_l(t)w_s(x)\,dt ={}\\
{}=(f,w_s)\,,\enskip
s=1,2,\dots,M\,.
\end{multline}
Матричный вид системы~(\ref{U316}) совпадает с~(\ref{U315}), где
$$
{G}_J=(g_{s,l})_{s,l=1}^M,\ \ g_{s,l}= 
\int\limits_{0}^{n+1}\!dx\int\limits_{0}^{x}\!U(x,t)w_l(t)w_s(x)\,dt.
$$

Рассмотрим теперь линейное дифференциальное уравнение
\begin{equation}\label{DU1}
y^{(k)}+a_1(x)y^{(k-1)}+\dots +a_k(x)y=f(x)
\end{equation}
с непрерывными коэффициентами $a_i(x)$, $i\hm=1,2,\dots,k$ и~начальными условиями
\begin{equation}
\label{NDU1}
y(0)=y_0\,,\enskip y'(0)=y_1,\ \dots,\ y^{(k-1)}=y_{k-1}\,.
\end{equation}
Если обозначить $y^{(k)}(x)=u(x)$, то задача~(\ref{DU1})--(\ref{NDU1}) сводится 
к~интегральному уравнению Вольтерра \mbox{2-го} рода. Следовательно, 
приближенное решение задачи~(\ref{DU1})--(\ref{NDU1}) можно искать в~виде:
\begin{multline*}
y_J(x)=\sum\limits_{s=1}^M c_s {\xi}_{k,s}(x) +y_{k-1}\fr{x^{k-1}}{(k-1)!}+{}\\
{}+y_{k-2}\fr{x^{k-2}}{(k-2)!}+\dots+y_0\,,
\end{multline*}
где функции ${\xi}_{k,s}(x)$ определены равенствами~(\ref{U38})
и~(\ref{U38-1}), а~коэффициенты~$c_s$ 
определяются из системы линейных уравнений:
\begin{multline*}
\sum\limits_{s=1}^M c_s ({\xi}_{k,s}+a_1 {\xi}_{k-1,s}+\dots +a_k w_s,w_l)={}\\
{}=\left(f,w_l\right),\enskip l=1,\dots,M\,.
\end{multline*}
При этом $\lim\nolimits_{J\to +\infty} \|y_J\hm-y\|_{C^{k-1}[0;n+1]}\hm=0$, 
где $y(x)$~--- точное решение задачи~(\ref{DU1})--(\ref{NDU1}).



\noindent
\textbf{Пример 4.1.}\
Рассмотрим нестационарную систему автоматического управления, 
поведение которой описывается дифференциальным уравнением:
$$
\sum\limits_{k=0}^5 a_k(t)x^{(k)}(t)=g(t),
$$
где коэффициенты~$a_k(t)$ определяются из сле\-ду\-юще\-го выражения:
{\small
\begin{multline*}
\begin{pmatrix}
a_0(t)\\a_1(t)\\a_2(t)\\a_3(t)\\a_4(t)\\a_5(t)\end{pmatrix}={}\\
\!\!{}=\!
\begin{pmatrix}0{,}5596&1{,}8918&2{,}5825&1{,}7855&0{,}6277&\!\!0{,}0909\\
0{,}7113&2{,}3843&3{,}222\hphantom{9}&2{,}1975&0{,}7588&\!\!0{,}1065\\
0{,}3717&1{,}2333&1{,}6449&1{,}1038&0{,}3728&\!\!0{,}0507\\
0{,}1002&0{,}3278&0{,}43\hphantom{99}&0{,}2827&0{,}093\hphantom{9}&\!\!0{,}0122\\
0{,}014\hphantom{9}&0{,}0449&0{,}0576&0{,}0369&0{,}0118&\!\!0{,}0015\\
0{,}0008&0{,}0025&0{,}0031&0{,}0019&0{,}006\hphantom{9}&\!\!\hphantom{9}0{,}00007
\end{pmatrix}\!\!
\begin{pmatrix}1\\t\\t^2\\t^3\\t^4\\t^5\end{pmatrix}\!.\hspace*{-2.77pt}
\end{multline*}
}

\noindent
Найти реакцию системы на входное воздействие:
\begin{multline*}
g(t)=\left(85{,}7661+338{,}5984t+497{,}0437t^2+{}\right.\\
{}+406{,}9496t^3+186{,}9354t^4+46{,}7809t^5+{}\\
\left.{}+4{,}8258t^6\right)e^{-4t}.
\end{multline*}
Начальные условия нулевые. Интервал исследования~--- $[0;5]$~c.

\columnbreak




\smallskip

\noindent
Р\,е\,ш\,е\,н\,и\,е\,.
Так как начальные условия нулевые, приближенное решение данной задачи будем  
искать в~виде
$$
x_J(t)=\sum\limits_{s=1}^M c_s {\xi}_{5,s}(t)\,,
$$
где коэффициенты $c_1,\dots, c_M$ определяются из системы линейных уравнений:
\begin{multline*}
\sum\limits_{s=1}^{2^J(n+1)+n} c_s \left(
a_5w_{s}+\sum\limits_{k=0}^4 a_k\xi_{5-k,s},w_l\right)={}\\
{}=
(g,w_l)\,,\enskip 
l=1,\dots,2^J(n+1)+n\,.
\end{multline*}
На рис.~2 показаны графики третьего и~пятого приближений~$x_2(t)$ 
(\textit{1}), $x_4(t)$~(\textit{2}) и~график 
сеточной функции $\{(t_i,\tilde{x}_i)\}$~(\textit{3}), 
полученной методом Рун\-ге--Кутта.\hfill $\square$

 { \begin{center}  %fig2
 \vspace*{9pt}
 \mbox{%
 \epsfxsize=79.639mm 
 \epsfbox{bit-2.eps}
 }


\end{center}


\noindent
{{\figurename~2}\ \ \small{Графики приближений $x_2(t)$ 
(\textit{1}), $x_4(t)$~(\textit{2}) 
и~график сеточной функции $\{(t_i,\tilde{x}_i)\}$~(\textit{3}), 
полученной методом Рун\-ге--Кутта}}
}

\vspace*{6pt}

\addtocounter{figure}{1}

%\vspace*{-36pt}

\begin{figure*} %fig3
\vspace*{1pt}
 \begin{center}
 \mbox{%
 \epsfxsize=164.758mm 
 \epsfbox{bit-3.eps}
 }
  \end{center}
\vspace*{-9pt}
\begin{minipage}[t]{79mm}
\Caption{Графики приближений $x_0(t)$ (\textit{1}), $x_2(t)$ 
(\textit{2}) и~график сеточной функции $\{(t_i,\tilde{x}_i)\}$  \label{sx}
(\textit{3}), полученной методом Рун\-ге--Кутта}
%\end{figure*}
%\begin{figure*} %fig4
\end{minipage}
\hfill
\vspace*{-9pt}
\begin{minipage}[t]{79mm}
\Caption{Графики приближений $y_0(t)$ (\textit{1}), $y_2(t)$~(\textit{2}) 
и~график сеточной функции $\{(t_i,\tilde{y}_i)\}$~(\textit{3}), полученной методом 
Рун\-ге--Кутта}
\label{sy}
\end{minipage}
\vspace*{12pt}
\end{figure*}


\smallskip

\noindent
\textbf{Пример 4.2.}\
Поведение линейной нестационарной системы описывается следующей
 системой дифференциальных уравнений:
$$
\begin{pmatrix}
{\dot x}(t) \\ {\dot y}(t)\end{pmatrix} = 
\begin{pmatrix} t^2 & 1-t \\ 1+t & t-t^2 \end{pmatrix}
 \begin{pmatrix} {x}(t) \\ {y}(t)\end{pmatrix}
 +  
 \begin{pmatrix} t^2 & 0 \\ 1 & t \end{pmatrix} 
 \begin{pmatrix} {g}_1(t) \\ {g}_2(t)\end{pmatrix}\,.
$$
Найти реакцию системы на входное воздействие:
\begin{multline*}
g_1(t) = 0{,}23315158 t^9-3{,}89665 t^8+26{,}4309725 t^7{}-\\
{}-93{,}4794 t^6+183{,}95 t^5 -200{,}83 t^4+{}\\
{}+122{,}255277 t^3-50{,}135386 t^2+13{,}095959 t-2{,}8237;
\end{multline*}

\vspace*{-12pt}

\noindent
\begin{multline*}
g_2(t) = -0{,}071962459 t^{13}+1{,}3465024 t^{12}-{}\\
{}-10{,}98105044 t^{11} + 51{,}1908385 t^{10} - 150{,}5098287 t^9+{}\\
{}+291{,}295256 t^8-378{,}61242 t^7+336{,}683591 t^6-{}
\end{multline*}

\noindent
\begin{multline*}
{}-213{,}9681871 t^5 + 106{,}48891 t^4-47{,}3676 t^3+{}\\
{}+19{,}56997 t^2-3{,}863587 t-0{,}0004283
\end{multline*}

%\pagebreak

\noindent
для начальных условий $x(0)\hm=-1$ и~$y(0)\hm=2$ на вре\-мен\-н$\acute{\mbox{о}}$м интервале $[0;2]$~c.

\smallskip



\noindent
Р\,е\,ш\,е\,н\,и\,е\,.\ \ 
Приближенное решение будем искать в~виде: 
\begin{align*}
x_{J}(t)&=-1+\sum\limits_{s=1}^M 
c_{s} {\xi}_{1,s}(t)\,;\\
y_{J}(t)&=2+\sum\limits_{s=1}^M c_{M+s} {\xi}_{1,s}(t)\,, 
\end{align*}
где $M\hm=2^J(n+1)+n$, а~коэффициенты $c_s$, $s\hm=1,2,\ldots, 2M$, определяются 
из системы линейных уравнений:
\begin{multline*}
\sum\limits_{s=1}^M c_s\left((w_s,w_l)-
\int\limits_{0}^{n+1}\!t^2{\xi}_{1,s}(t)w_l(t)\,dt\right)-{}\\
{}-\sum\limits_{s=1}^M c_{M+s}\int\limits_0^{n+1}(1-t){\xi}_{1,s}(t)w_l(t)\,dt={}\\
{}=\!\int\limits_0^{n+1}\!\left(t^2g_1(t)-t^2+2(1-t)\right)w_l(t)\,dt\,,\\
 l=1,2,\dots,M\,;
\end{multline*}

\vspace*{-12pt}

\noindent
\begin{multline*}
\sum\limits_{s=1}^M c_s\int\limits_{0}^{n+1}(1+t){\xi}_{1,s}(t)w_l(t)\,dt+{}\\
{}+
\sum\limits_{s=1}^M \!c_{M+s}\left(\int\limits_0^{n+1}\!(t-t^2){\xi}_{1,s}(t)w_l(t)\,dt-
\left(w_s,w_l\right)\right)={}\hspace*{-6.37675pt}
\end{multline*}


\noindent
\begin{multline*}
{}=\int\limits_0^{n+1}(2t^2-t+1-g_1(t)-tg_2(t))w_l(t)\,dt\,, 
\\
 l=1,2,\dots,M\,.
\end{multline*}


На рис.~\ref{sx} и~\ref{sy} показаны графики первого и~треть\-его приближений 
$x_0(t),~y_0(t)$ (\textit{1}), $x_2(t),~y_2(t)$~(\textit{2}) 
и~графики сеточных функций $\{(t_i,\tilde{x}_i)\}$, $\{(t_i,\tilde{y}_i)\}$~(\textit{3}), 
полученные методом Рун\-ге--Кутта.~\hfill$\square$

\vspace*{-6pt}

\section{Заключение}

\vspace*{-2pt}

В данной статье были обобщены известные методы применения вейвлетов 
Хаара к~приближенному решению линейных интегральных и~дифференциальных уравнений. 
Эти методы получаются из представленных здесь при $n\hm=0$, что и~соответствует 
вейвлетам Хаара. В~отличие от вейвлетов Хаара, где приближения решения интегрального 
уравнения получаются ку\-соч\-но-по\-сто\-ян\-ны\-ми, а~приближения решения 
дифференциального уравнения принадлежат классу глад\-кости~$C^{k-1}$, где $k$~--- 
порядок уравнения, использование сплайн-вейв\-лет дает возможность 
строить приближения любого класса гладкости~$C^n$.

\vspace*{-6pt}

{\small\frenchspacing
 {%\baselineskip=10.8pt
 \addcontentsline{toc}{section}{References}
 \begin{thebibliography}{99}
 
 \vspace*{-2pt}
 
\bibitem{Lepik3} 
\Au{Lepik\,{\!\!\ptb{\"{U}}}}. 
Application of the Haar wavelet transform to solving integral and 
differential equations~// Proc. Est. Acad. Sci. Ph.,  2007. 
Vol.~56. P.~28--46.
\bibitem{Blatov} %2
\Au{Блатов И.\,А., Рогова~Н.\,В.} Полуортогональные сплайновые вейвлеты и~метод 
Галеркина численного моде-\linebreak\vspace*{-11pt}

\pagebreak

\noindent
лирования тонкопроволочных антен~// 
Вычисл. матем. и~матем. физ., 2013. Т.~53. №\,5. C.~727--736.

\bibitem{Lepik4}   %3
\Au{Lepik\,{\!\!\ptb{\"{U}}}}. 
Numerical solution of differential equations using Haar wavelets~//  
Math. Comput. Simulat., 2005. Vol.~68. P.~127--143.
\bibitem{Lepik1}  %4
\Au{Lepik\,{\!\!\ptb{\"{U}}}}. Numerical solution of evolution 
equations by the Haar wavelet method~//  Appl. Math. Comput., 2007.  Vol.~185. 
P.~695--704.
\bibitem{Lepik2} %5
\Au{Lepik\,{\!\!\ptb{\"{U}}}}. Haar wavelet method for solving higher order
 differential equations~// Int. J.~Math. Comput., 2008. Vol.~1. No.\,8. P.~84--94.

\bibitem{Lepik}  %6
\Au{Lepik\,{\!\!\ptb{\"{U}}}., Hein~H}.  Haar wavelets with applications.~---  
Berlin: Springer, 2014. 207~p.
\bibitem{ArticleFinkelstein}  
\Au{Finkelstein A., Salesin~D.\,H.} Multiresolution curves~// SIGGRAPH Proceedings.~--- 
New York, NY, USA: ACM, 1994. P.~261--268.
\bibitem{Frazer} 
\Au{Frazier M.\,W}. An introduction to wavelets through linear algebra.~--- 
New York, NY, USA: Springer,  1999.  503~p.
\bibitem{Chui}  
\Au{Chui Ch.\,К}.  An introduction to wavelets.~--- Boston, MA, USA: Academic Press, 1991. 412~p.
\bibitem{Yurgu} 
\Au{Bityukov Yu.\,I.,  Akmaeva~V.\,N. }  
The use of wavelets in the mathematical and computer modelling of manufacture 
of the complex-shaped shells made of composite materials~//  Bull. 
South Ural State University. Ser. Mathematical Modelling, 
Programming and Computer Software, 2016. Vol.~9. No.\,3. P.~5--16.

\bibitem{Novikov}   %11
\Au{Новиков И.\,Я.,  Протасов~В.\,Ю.,  Скопина~М.\,А.} Теория всплесков.~--- 
М.: Физматлит, 2005.  612~c.
\bibitem{Akilov}  %12
\Au{Канторович Л.\,В.,  Акилов~Г.\,П.} Функциональный анализ.~--- 
М.: Наука, 1977. 744~с.

%\bibitem{BookSmolencev} 
%\Au{Смоленцев Н.\,К}. Основы теории вейвлетов. Вейвлеты в~MatLab.~--- 
%М.: ДМК Пресс,  2005. 303~c.
%\bibitem{Pupkov} Пупков, К.А., Н.Д. Егупов. Методы классической и~современной теории автоматического управления. Математические модели, динамические характеристики и~анализ систем автоматического управления. М.: Издательство МГТУ им. Н.Э. Баумана. 2004. 656 с.
 \end{thebibliography}

 }
 }

\end{multicols}

\vspace*{-6pt}

\hfill{\small\textit{Поступила в~редакцию 21.03.17}}

\vspace*{8pt}

%\newpage

%\vspace*{-24pt}

\hrule

\vspace*{2pt}

\hrule

%\vspace*{8pt}


\def\tit{THE USE OF WAVELETS FOR~THE~CALCULATION OF~LINEAR CONTROL SYSTEMS 
WITH~LUMPED PARAMETERS}

\def\titkol{The use of wavelets for~the~calculation of~linear control systems 
with~lumped parameters}

\def\aut{Yu.\,I.~Bityukov and~E.\,N.~Platonov}

\def\autkol{Yu.\,I.~Bityukov and~E.\,N.~Platonov}

\titel{\tit}{\aut}{\autkol}{\titkol}

\vspace*{-9pt}


\noindent
Moscow Aviation Institute (National 
Research University), 4~Volokolamskoye Highway, Moscow 125993, Russian Federation



\def\leftfootline{\small{\textbf{\thepage}
\hfill INFORMATIKA I EE PRIMENENIYA~--- INFORMATICS AND
APPLICATIONS\ \ \ 2017\ \ \ volume~11\ \ \ issue\ 4}
}%
 \def\rightfootline{\small{INFORMATIKA I EE PRIMENENIYA~---
INFORMATICS AND APPLICATIONS\ \ \ 2017\ \ \ volume~11\ \ \ issue\ 4
\hfill \textbf{\thepage}}}

\vspace*{3pt}



\Abste{In many disciplines, problems appear which can be formulated with 
the aid of differential or integral equations.  In simpler cases, such equations 
can be solved analytically, but for more complicated cases, numerical 
procedures are needed. In recent times, the wavelet-based methods 
have gained great popularity, where different wavelet families such as 
Daubechies, Coiflet, etc.\ wavelets are applied. A~shortcoming of these wavelets 
is that they do not have an analytic expression. For this reason, differentiation 
and integration of these wavelets are very complicated. The paper presents 
algorithms for the numerical solution of linear integral and differential 
equations based on spline wavelets on the interval. The algorithms generalize 
the well-known methods based on Haar wavelets, which are a~particular case 
of spline wavelets. The results presented can be applied for 
the analysis of linear systems with lumped parameters.}

\KWE{spline wavelet; differential equation; integral equation}



\DOI{10.14357/19922264170412} 

\vspace*{-6pt}

\Ack
\noindent
This work is a part of Project No.\,2.2461.2017 
supported by the Russian Ministry of Education and Science.



\vspace*{3pt}

  \begin{multicols}{2}

\renewcommand{\bibname}{\protect\rmfamily References}
%\renewcommand{\bibname}{\large\protect\rm References}

{\small\frenchspacing
 {%\baselineskip=10.8pt
 \addcontentsline{toc}{section}{References}
 \begin{thebibliography}{99}

\bibitem{1-bit-1}
\Aue{Lepik,  $\ddot{\mbox{U}}$}. 2007. Application of the
 Haar wavelet transform to solving integral and differential equations. 
 \textit{Proc. Est. Acad. Sci. Ph.} 56:28-46.
\bibitem{2-bit-1}
\Aue{Blatov,  I.\,A., and  N.\,V.~Rogova.} 
2013.  Application of semi-orthogonal spline wavelets and Galerkin
method to the numerical simulation of thin wire antennas. 
\textit{Comp. Math. Math. Phys.} 53(5):564--572.
\bibitem{5-bit-1} %3
\Aue{Lepik,  $\ddot{\mbox{U}}$.} 2005. Numerical solution of differential 
equations using Haar wavelets. \textit{Math. Comput. Simulat.} 68:127--143.
\bibitem{3-bit-1} %4
\Aue{Lepik, $\ddot{\mbox{U}}$.}
2007. Numerical solution of evolution equations by the 
Haar wavelet method. \textit{Appl. Math. Comput.} 185:695--704.
\bibitem{4-bit-1} %5
\Aue{Lepik,  $\ddot{\mbox{U}}$.} 
2008. Haar wavelet method for solving higher order differential equations. 
\textit{Int. J.~Math. Comput.} 1(8):84--94.

\bibitem{6-bit-1}
\Aue{Lepik,  $\ddot{\mbox{U}}$, and H.~Hein.} 2014. \textit{Haar wavelets 
with applications.}  Berlin: Springer. 207~p.
\bibitem{7-bit-1}
\Aue{Finkelstein, A., and D.\,H.~Salesin}. 1994. 
{Multiresolution curves}. \textit{SIGGRAPH Proceedings}.  New York, NY: ACM. 261--268.
%\item Demko, S., W.F. Moss and P.W. Smith. Decay rates for inverses of band matrices. 1984. Math. Comp. V. 43, \textnumero{167}. 491--499.
\bibitem{8-bit-1}
\Aue{Frazier, M.\,W.} 1999. \textit{An introduction to wavelets through linear algebra.} 
New York, NY: Springer. 503~p.
\bibitem{9-bit-1}
\Aue{Chui,  Ch.\,К.} 1991. \textit{An introduction to wavelets.} Boston, MA: 
Academic Press. 412~p.
\bibitem{10-bit-1}
\Aue{Bityukov, Yu.\,I., and  V.\,N.~Akmaeva.} 2016. 
The use of wavelets in the mathematical and computer modelling of manufacture 
of the complex-shaped shells made of composite materials.   
\textit{Bull. South Ural State University. Ser. Mathematical Modelling, 
Programming and Computer Software} 9(3):5--16.

\bibitem{12-bit-1}
\Aue{Novikov, I.\,Ya.,  V.\,Yu.~Protasov, and M.\,A.~Skopina.}
2005. \textit{Teoriya vspleskov} [{The theory of wavelets}]. Moscow: Fizmatlit.  612~p.

\bibitem{11-bit-1}
\Aue{Kantorovich, L.\,V., and G.\,P.~Akilov.} 1977. \textit{Funktsional'nyy analiz} 
[{Functional analysis}]. Moscow: Nauka. 744~p.
%\bibitem{13-bit-1}
%\Aue{Smolentsev, N.\,K.} 2005. \textit{Osnovy teorii veyvletov. Veyvlety v~MatLab} 
%[{Fundamentals of the theory of wavelets. Wavelets in MatLab}].  Moscow: 
%DMK Press. 303~p.
\end{thebibliography}

 }
 }

\end{multicols}

\vspace*{-6pt}

\hfill{\small\textit{Received March 21, 2017}}

%\vspace*{-10pt}

\Contr

\noindent
\textbf{Bityukov Yuri I.} (b.\ 1972)~--- 
Doctor of Science in technology, associate professor, 
Moscow Aviation Institute (National Research University), 
4~Volokolamskoye Highway, Moscow 125993, Russian Federation; 
\mbox{yib72@mail.ru}

\vspace*{3pt}

\noindent
\textbf{Platonov Evgeny N.} (b.\ 1976)~--- Candidate of Science (PhD) in physics 
and mathematics, associate professor, Moscow Aviation Institute (National 
Research University), 4~Volokolamskoye Highway, Moscow 125993, Russian Federation;
\mbox{en.platonov@gmail.com}
\label{end\stat}


\renewcommand{\bibname}{\protect\rm Литература}   %12
\newcommand{\G}{{\sf Ge}}

\def\stat{kudr-titova}

\def\tit{ГАММА-ЭКСПОНЕНЦИАЛЬНАЯ ФУНКЦИЯ В БАЙЕСОВСКИХ МОДЕЛЯХ МАССОВОГО ОБСЛУЖИВАНИЯ$^*$}

\def\titkol{Гамма-экспоненциальная функция в~байесовских моделях массового обслуживания}

\def\aut{А.\,А.~Кудрявцев$^1$, А.\,И.~Титова$^2$}

\def\autkol{А.\,А.~Кудрявцев, А.\,И.~Титова}

\titel{\tit}{\aut}{\autkol}{\titkol}

\index{Кудрявцев А.\,А.}
\index{Титова А.\,И.}
\index{Kudryavtsev A.\,A.}
\index{Titova A.\,I.}



{\renewcommand{\thefootnote}{\fnsymbol{footnote}} \footnotetext[1]
{Работа выполнена при частичной финансовой поддержке РФФИ (проект 17-07-00577).}}


\renewcommand{\thefootnote}{\arabic{footnote}}
\footnotetext[1]{Московский государственный университет им.\ М.\,В.~Ломоносова, 
факультет вычислительной математики и~кибернетики, \mbox{nubigena@mail.ru}}
\footnotetext[2]{Московский государственный университет им.\ М.\,В.~Ломоносова, 
факультет вычислительной математики и~кибернетики, \mbox{onkelskroot@gmail.com}}

%\vspace*{-18pt}

\vspace*{-9pt}

\Abst{Рассматривается байесовский подход к~построению мо\-де\-лей 
теории массового обслуживания и~надежности. Байесовский подход является 
целесообразным при изучении систем, характеристики которых меняются в~моменты 
времени, неизвестные исследователю, или же при изучении больших совокупностей 
однотипных систем. В~рамках этого подхода для классических постановок задач 
предполагается, что основные параметры системы не являются заданными, 
но при этом известны их априорные распределения. За счет 
рандомизации параметров системы различные ее характеристики, 
например коэффициент загрузки, также становятся случайными. 
В~работе вводится понятие гам\-ма-экс\-по\-нен\-ци\-аль\-ной 
функции, приводятся ее свойства, а~также конкретные результаты для 
вероятностных характеристик коэффициента загрузки и~вероятности <<непотери>> 
вызова в~случае, когда в~качестве пары априорных распределений параметров 
системы $M/M/1/0$ рассматриваются экспоненциальное распределение и~распределение 
Вейбулла.}


\KW{байесовский подход; системы массового обслуживания; надежность; смешанные
распределения; распределение Вейбулла; экспоненциальное распределение; 
гам\-ма-экс\-по\-нен\-ци\-аль\-ная функция}

\DOI{10.14357/19922264170413} 

\vspace*{-9pt}


\vskip 10pt plus 9pt minus 6pt

\thispagestyle{headings}

\begin{multicols}{2}

\label{st\stat}

\section{Введение}

Зачастую при описании математических мо\-де\-лей функционирования различных объектов 
их жизненный цикл зависит от параметров, <<способствующих>> и~<<препятствующих>> 
функционированию. В~моделях структур и~систем массового\linebreak обслуживания к~параметрам, 
<<способствующим>> функционированию, можно отнести интенсивность обслуживания 
запросов, а~к~параметрам, <<препятствующим>> функционированию,~--- 
интенсивность входящего потока требований. При этом нетрудно заметить, 
что для исследования результатов работы системы важны не столько значения 
параметров, сколько их соотношение.

Далее будет рассмотрена система массового обслужи\-вания $M/M/1/0$, одним из 
основных показателей которой является ее коэффициент загрузки~$\rho$. 
Значение коэффициента загрузки определяется как отношение параметра входящего 
потока~$\lambda$ к~па\-ра\-мет\-ру обслуживания~$\mu$. От величины~$\rho$ 
зависят многие характеристики разнообразных систем массового обслуживания, в~том 
числе вероятность <<непотери>> вызова $\pi \hm= {\mu/(\lambda \hm+ \mu)}
\hm = 1/(1\hm+\rho)$.

В рамках байесовского подхода к~постановкам классических задач массового 
обслуживания и~надежности предполагается, что конкретные значения параметров~$\lambda$ 
и~$\mu$ неизвестны, однако имеется информация об их априорных распределениях~\cite{KuSh2015}.

Зачастую в~байесовских постановках задач массового обслуживания результаты 
описываются в~терминах специальных функций, например в~терминах бе\-та-функ\-ции 
и~интегральной показательной функции при рассмотрении общего эрланговского 
случая~\cite{KuSh09b}. При рассмотрении общего 
бе\-та-рас\-пре\-де\-ле\-ния~\cite{ZhaKuSh} или же бе\-та-рав\-но\-мер\-но\-го~\cite{ZhaKuSh2} 
распределения параметров в~байесовской модели рекуррентного роста надежности 
результаты выражаются через обобщенную гипергеометрическую функцию.

В ходе исследований, касающихся вероятностных характеристик коэффициента 
загрузки~$\rho$  и~вероятности <<непотери>> вызова~$\pi$ в~случае, когда в~качестве 
пары априорных распределений параметров системы~$\lambda$ и~$\mu$ 
рассматриваются экспоненциальное распределение и~распределение Вейбулла, 
были получены результаты, не выражающиеся в~терминах элементарных функций. 
В~связи с~этим предлагается рассмотреть новую специальную функцию, 
упоминаний об аналитическом виде и~свойствах которой не было 
обнаружено в~классических книгах, посвященных специальным функциям 
(см., например,~\cite{Artin, BeEr,AbSt}).

\section{Основные результаты}


Введем следующие обозначения. Через~$M(\theta)$ обозначим экспоненциальное 
распределение с~параметром $\theta\hm>0$, а~через $W(p,\alpha)$~--- 
распределение Вейбулла с~плот\-ностью $w_{p,\alpha}(x)$, имеющей вид:
$$
w_{p,\alpha}(x) = \fr{px^{p-1}e^{-({x/\alpha})^p}}{\alpha^{p}} \,, \enskip
 x>0\,,\  p>0\,,\ \alpha>0\,.
 $$
Назовем функцию вида
$$
\G_{\alpha,\, \beta} (x) = \sum\limits_{k=0}^{\infty}
\fr{x^k}{k!}\, \Gamma(\alpha k + \beta)\,, \enskip
 x\in\mathbb{R}\,, \ \alpha\ge0\,, \  \beta> 0\,,
 $$
\textit{гамма-экспо\-нен\-ци\-аль\-ной функцией}.


\smallskip

\noindent
\textbf{Теорема~1.}\
\textit{Гамма-экспо\-нен\-ци\-аль\-ная функция обладает следующими свойствами}:
\begin{enumerate}
\item $\G_{\alpha, \beta} (x)$ 
\textit{сходится абсолютно на всей прямой при $0\hm\le\alpha\hm<1$, $\beta\hm>0$ и~на интервале $(-1,1)$ при $\alpha\hm=1$, $\beta\hm>0$; 
сходится условно в~точке $x = -1$ при 
$\alpha=1$, $0\hm<\beta\hm<1$; сходится только 
в~точке  $x\hm = 0$ при} $\alpha\hm>1$, $\beta\hm>0$.
\item $\G_{\alpha, \beta}(x)$ \textit{непрерывна в~области сходимости}.
\item $\G_{1,\, n+1} (x) \hm= \left({x^n}/{(1-x)}\right)^{(n)}_x$, $|x|\hm<1$. 
\textit{В~частности, $\G_{1,\, 1} (x)\hm= {1/{(1\hm-x)}}$ и}
$\G_{1,\, 2} (x)\hm= {1/{(1\hm-x)^2}}$,  $|x|\hm<1$.
\item $\G^{(n)}_{\alpha, \beta} (x) = \G_{\alpha, \alpha n + \beta}(x)$ 
\textit{в области сходимости}.
\item $\G_{\alpha, \beta} (0)\hm=\Gamma(\beta),$ $\alpha\hm\ge0,$ $\beta\hm> 0$.
\item $\G_{0, \beta} (x)= \Gamma(\beta)e^{x},$ $x\hm\in\mathbb{R},$ $\beta\hm> 0$.
\item $\G_{\alpha, 1}(x)=1\hm+\alpha x\G_{\alpha, \alpha}(x),$ $x\hm\in
\mathbb{R},$ $\alpha\hm>0$.
\item $\G_{\alpha,\, \beta}(x)=\Gamma(\beta-1)+\alpha x \G_{\alpha,\alpha+\beta-1}(x)
+(\beta\hm-1)\G_{\alpha, \beta-1}(x),$ $x\hm\in\mathbb{R},$ $\alpha\hm\ge0,$ $\beta\hm>1$.
\item $\G_{q, q+1}(-x^q)=\G_{1/q, 1/q+1 }(-{1/x})/{({qx^{q+1}})},$ $x\hm>0,$ $q\hm>1$.
\item $\sum\nolimits_{k=0}^\infty ({\alpha^k}/{k!}) \G_{1/p,\, (k+p)/p}(- \alpha)\hm= 1$, 
$\alpha\hm>0$, $p\hm>1.$
\end{enumerate}


\noindent
Д\,о\,к\,а\,з\,а\,т\,е\,л\,ь\,с\,т\,в\,о\,.\ \ 
Свойство~1 следует из соотношения (6.1.46) в~\cite{AbSt}:
$$
\lim\limits_{n \to \infty}
\fr{\Gamma(n+\alpha)n^{\beta-\alpha}}{{\Gamma(n+\beta)}}=1\,,
$$
формулы Даламбера и~признака Лейбница.
Свойства~3--8 проверяются непосредственно. Свойство~9 следует из соотношения:
\begin{multline*}
\G_{q, q+1}(-x^q)= \int\limits_0^\infty e^{-(xt)^{q}} t^q e^{-t}\, dt={}\\
{}=
\fr{1}{{qx^{q+1}}}\sum\limits_{k=0}^\infty
\fr{(-1/x)^k}{{k!}}\int\limits_0^\infty z^{(k+1)/q}e^{-z}\, dz\,.
\end{multline*}
Для обоснования свойства~10 достаточно рас\-смот\-реть случайную величину~$N$, 
имеющую смешанное пуассоновское распределение со структурным распределением 
Вейбулла, для которой справедливо:
\begin{multline*}
\p(N=k) = \fr{p}{{\alpha^p k!}} \int\limits_0^\infty 
e^{-\lambda -(\lambda/\alpha)^p} \lambda^{k+p-1} \, d\lambda={}\\
{}= \fr{\alpha^k}{k!} \sum\limits_{n=0}^\infty 
\fr{(-\alpha)^n}{{n!}} \int\limits_0^\infty t^{(n+k)/p} e^{-t} \, dt
= {}\\
{}=\fr{\alpha^k}{k!}\, \G_{1/p,\, (k+p)/p}(- \alpha)\,.
\end{multline*}

Теорема доказана.

\smallskip

Приведем два утверждения, в~которых па\-ра\-мет\-ры входящего потока~$\lambda$ 
и~обслуживания~$\mu$ в~модели~$M/M/1/0$ имеют  экспоненциальное распределение 
и~распределение Вейбулла.

\noindent
\textbf{Теорема~2.}\ 
\textit{Пусть параметр входящего потока~$\lambda$ имеет экспоненциальное 
распределение $M(\theta)$, $\theta\hm>0$, а~параметр обслуживания~$\mu$ 
имеет распределение Вейбулла $W(p,\alpha)$, $p\hm>1$, $\alpha\hm>0$, причем~$\lambda$ 
и~$\mu$ независимы. Тогда при $x\hm>0$ функция распределения, плотность 
и~моменты коэффициента загрузки~$\rho$ имеют вид}:
\begin{align*}
F_\rho(x) &= 1 - \G_{1/p,\, 1}(-\theta \alpha x)\,; \\
f_\rho(x) &= \theta \alpha \G_{1/p,\, 1/p+1}(-\theta \alpha x)\,;\\
\e\,\rho^n &= \fr{n!}{(\theta \alpha)^n}\,\Gamma\left(1-\fr{n}{p}\right)\,, \enskip 
p>n\,,
\end{align*}
\textit{a функция распределения и~плотность вероятности <<непотери>> вызова~$\pi$ 
при $x\hm\in(0,1)$ определяются соотношениями}:
\begin{align*}
F_\pi(x) &= \G_{1/p,\, 1}\left(-\fr{\theta \alpha (1-x)}{x}\right)\,; \\
f_\pi(x) &= \fr{\theta \alpha}{{x^2}}\, \G_{1/p,\, 1/p + 1}
\left(-\fr{\theta \alpha (1-x)}{x}\right)\,.
\end{align*}

\noindent
Д\,о\,к\,а\,з\,а\,т\,е\,л\,ь\,с\,т\,в\,о\,.\ \ Заметим, что при $x\hm>0$
\begin{multline*}
F_\rho(x) =
\int\limits_0^{\infty}  \fr{p}{\alpha}\left(
\fr{u}{\alpha}\right)^{p-1}e^{-(u/\alpha)^{p}}(1-e^{-ux\theta})\, du={}\\
{}= 1 - \sum\limits_{k=0}^\infty \int\limits_0^{\infty} 
\fr{(-\theta x)^k}{k!}\,u^k e^{-\left(u/\alpha\right)^p} \ d\left(\fr{u}{\alpha}\right)^p ={}\\
{}=
1 - \sum\limits_{k=0}^\infty \int\limits_0^{\infty} 
\fr{(-\theta x)^k}{k!} \left(\alpha t^{1/p}\right)^{k}e^{-t} \, dt ={}\\
{}= 1 - \G_{1/p,\, 1}(-\theta \alpha x)\,,
\end{multline*}
откуда, воспользовавшись свойством~4 теоремы~1, получаем выражение для~$f_\rho(x)$.

Вычислим момент $n$-го порядка для коэффициента загрузки. Имеем:
\begin{multline*}
\e\,\rho^n = \int\limits_0^\infty x^n \theta \alpha \G_{1/p,\, 1/p+1}
(-\theta \alpha x) \, dx = {}\\
{} = \theta \alpha \int\limits_0^\infty \int\limits_0^{\infty} 
\exp \left\lbrace -\theta \alpha t^{1/p}x\right\rbrace t^{1/p}e^{-t} x^{n}\, dt  dx={}\\
{}=\int\limits_0^\infty \fr{e^{-t}}{(\theta \alpha)^n t^{n/p}} \times{}\\
{}\times
\int\limits_0^{\infty} \exp \left\lbrace -\theta \alpha t^{1/p}x\right\rbrace 
(\theta \alpha t^{1/p} x)^n \, d(\theta \alpha t^{1/p} x) \, dt={}\\
{}= \fr{\Gamma(n+1)}{{(\theta \alpha)^n}}\int\limits_0^\infty e^{-t}t^{-n/p}  \, dt 
= \fr{n!}{{(\theta \alpha)^n}}\Gamma\left(
1-\fr{n}{{p}}\right)\,, \\
 p>n\,.
\end{multline*}

Теперь получим функцию распределения для вероятности <<непотери>> вызова~$\pi$. Имеем:
\begin{multline*}
F_\pi(x) =  1 - \p\left(\rho<\fr{{1-x}}{{x}}\right) ={}\\
{}=
  \G_{1/p, 1}\left(-\fr{\theta \alpha (1-x)}{x}\right)\,,\enskip
  x\in(0,1)\,,
\end{multline*}
откуда, воспользовавшись свойством~4 теоремы~1, получаем выражение для $f_\pi(x)$.

Теорема доказана.

\smallskip


\noindent
\textbf{Теорема 3.}\ 
\textit{Пусть параметр входящего потока~$\lambda$ имеет распределение 
Вейбулла $W(q,\theta)$, $0\hm<q\hm<1$, $\theta\hm>0$, а~параметр обслуживания~$\mu$ 
имеет экспоненциальное распределение $M(\alpha)$, $\alpha\hm>0$, причем~$\lambda$ 
и~$\mu$ независимы. Тогда при $x\hm>0$ функция распределения, плотность 
и~математическое ожидание коэффициента загрузки~$\rho$ имеют вид}:
\begin{align*}
F_\rho(x) &=  1 - \G_{q, 1}\left(-\fr{x^q}{(\alpha \theta)^q}\right)\,; \\
f_\rho(x) &= \fr{qx^{q-1}}{{(\theta \alpha)^q}} \G_{q, q+1}
\left( -\fr{x^q}{(\alpha \theta)^q}\right)\,; 
\\
\e\,\rho&=(\theta \alpha)^q \Gamma(1-q)\,,
\end{align*}
\textit{a функция распределения и~плотность вероятности <<непотери>> вызова~$\pi$ 
при $x\hm\in(0,1)$ определяются соотношениями}:
\begin{align*}
F_\pi(x) &= \G_{q,\, 1}\left(-\fr{(1-x)^q}{(\alpha \theta x)^q}\right)\,;
\\
f_\pi(x) &= \fr{q}{{(\theta \alpha)^q x^2}}\left(
\fr{1-x}{{x}}\right)^{q-1}\! \G_{q, q + 1}\left(
-\fr{(1-x)^q}{(\alpha \theta x)^q}\right).
\end{align*}


\noindent
Д\,о\,к\,а\,з\,а\,т\,е\,л\,ь\,с\,т\,в\,о\,.\ \ Для $x\hm>0$ имеем:
\begin{multline*}
F_\rho(x) =
 \int\limits_0^{\infty} \left(1-e^{-(ux/\theta)^q}\right)\alpha e^{-\alpha u}\, du={}\\
 {}=
 1 - \int\limits_0^{\infty} e^{-\alpha u}\sum\limits_{k=0}^\infty 
 \fr{(-1)^k \left(ux/\theta\right)^{qk}}{{k!}} \, d(\alpha u) ={}\\
{}= 1 - \sum\limits_{k=0}^\infty \int\limits_0^{\infty} 
\fr{(-1)^k}{{k!}} \left(
\fr{tx}{{\alpha \theta}}\right)^{qk} e^{-t} \, dt = {}\\
{}=
1 - \G_{q,\, 1}\left(-\fr{x^q}{(\alpha \theta)^q}\right)\,.
\end{multline*}
Выражение для $f_\rho(x)$ получается из свойства~4 теоремы~1.
Найдем $n$-й момент коэффициента загрузки. Имеем:
\begin{multline*}
\e\,\rho^n ={}\\
{}=
\int\limits_0^\infty 
\fr{qx^{n+q-1}}{{(\alpha\theta )^q}}\sum\limits_{k=0}^\infty
\fr{(-1)^kx^{qk}}{{(\alpha \theta)^{qk}k!}}  
 \int\limits_0^\infty t^{q(k+1)}e^{-t}\, dt dx={}\\
{}= \int\limits_0^\infty \fr{q}{{(\theta \alpha)^q}}
\int\limits_0^\infty \exp\left\lbrace -\left(
\fr{tx}{{\alpha \theta}}\right)^q\right\rbrace x^{q+n-1} t^q e^{-t}\, dt   dx={}\\
 {}= \int\limits_0^\infty \fr{t^q e^{-t}}{{(\theta \alpha)^q}}
\int\limits_0^\infty \exp\left\lbrace -\left(
\fr{t}{{\alpha \theta}}\right)^qz\right\rbrace z^{(n-1)/q + 1} \, dz  dt={}\\
{}= \int\limits_0^\infty e^{-t}\left(
\fr{\theta \alpha}{t}\right)^{n+q-1}\int\limits_0^\infty e^{-u}
 u^{(n-1)/q + 1} \, du dt ={}\\
{}= \int\limits_0^\infty e^{-t}\left(
\fr{\theta \alpha}{{t}}\right)^{n+q-1}\Gamma\left(\fr{n-1}{q} + 2\right) \, dt={}\\
{}=(\theta \alpha)^{n+q-1}\Gamma\left(\fr{n-1}{q} + 2\right) \Gamma (2-n-q)\,, 
\end{multline*}
откуда получаем, что существует только момент первого порядка при $0\hm<q\hm<1.$

Для функции распределения вероятности <<непотери>> вызова~$\pi$ справедливо:
\begin{multline*}
F_\pi(x) = 1 - \p\left(\rho<
\fr{{1-x}}{{x}}\right)={}\\
{}= \G_{q,\, 1}\left(-\fr{(1-x)^q}{(\alpha \theta x)^q}\right)\,, \enskip
x\in(0,1)\,.
\end{multline*}
Выражение для $f_\pi(x)$ получается из соответст\-ву\-юще\-го свойства 
$\G_{\alpha,\, \beta} (x)$.

Теорема доказана.

\smallskip

\noindent
\textbf{Замечание.}\ В теоремах~2 и~3 при $p\hm = 1$ и~$q \hm= 1$ распределение 
Вейбулла $W(1, \alpha)$ совпадает с~экспоненциальным распределением~$M(1/\alpha)$. 
Результаты для общего экспоненциального случая были получены 
ранее в~работе~\cite{KuSh09b}.

\smallskip

\noindent
\textbf{Следствие.}\
Из доказанных выше теорем следует, что гам\-ма-экс\-по\-нен\-ци\-аль\-ная
 функция также обладает свойствами:
\begin{enumerate}
\setcounter{enumi}{10}
\item $\lim\nolimits_{x\to-\infty} \G_{\alpha,\, 1}(x) \hm= 0,$ $0\hm\leq \alpha\hm<1$;
\item $\int\nolimits_{0}^{+\infty} \G_{\alpha,\, \alpha + 1}(-x)\,dx \hm= 1,$ $0\hm<\alpha\hm<1$;
%\item $\int\limits_{-\infty}^{0} \G_{\alpha,\, \alpha + 1}(x)dx = 1,$ $0<\alpha<1$;
\item $\G_{\alpha,\, \alpha + 1}(x)\hm > 0,$ $x\hm\in\mathbb{R},$ $0\hm<\alpha\hm<1$;
\item $\G_{\alpha,\, 1}(x)$ строго монотонна на всей прямой при $0\hm\le\alpha\hm<1$.
\end{enumerate}





{\small\frenchspacing
 {%\baselineskip=10.8pt
 \addcontentsline{toc}{section}{References}
 \begin{thebibliography}{9}

\bibitem{KuSh2015}
\Au{Кудрявцев А.\,А., Шоргин~С.\,Я.}
Байесовские модели в~тео\-рии массового обслуживания и~надежности.~--- 
М.: ФИЦ ИУ РАН, 2015. 76~с.

\bibitem{KuSh09b}
\Au{Кудрявцев А.\,А., Шоргин~В.\,С., Шоргин~С.\,Я.} Байе\-сов\-ские
модели массового обслуживания и~надежности: общий эрланговский
случай~// Информатика и~её применения, 2009. Т.~3. Вып.~4. С.~30--34.

\bibitem{ZhaKuSh}
\Au{Жаворонкова Ю.\,В., Кудрявцев~А.\,А., Шоргин~С.\,Я.}
Байесовская рекуррентная модель роста надежности: бе\-та-рас\-пре\-де\-ле\-ние
параметров~// Информатика и~её применения, 2014. Т.~8. Вып.~2.
С.~48--54.

\bibitem{ZhaKuSh2}
\Au{Жаворонкова Ю.\,В., Кудрявцев~А.\,А., Шоргин~С.\,Я.}
Байесовская рекуррентная модель роста надежности: бета-рав\-но\-мер\-ное
распределение параметров~// Информатика и~её применения, 2015. Т.~9.
Вып.~1. С.~98--105.

\bibitem{Artin}
\Au{Artin E.} The gamma function.~--- New York, NY, USA: 
Holt, Rinehart and Winston, 1964. 39~p.

\bibitem{BeEr}
\Au{Бейтмен Г., Эрдейи~А.} Высшие трансцендентные функции. Т.~1.~/
Пер. с~англ.---  М.: Наука, 1973. 296~с.
(\Au{Bateman~H., Erdelyi~A.}  
{Higher transcendental functions}. Vol.~1.~--- New York\,--\,Toronto\,--\,London: 
McGraw-Hill Book Co., Inc., 1953. 302~p.)

\bibitem{AbSt}
\Au{Abramowitz M., Stegun~I.} 
Handbook of mathematical functions with formulas, graphs, and mathematical tables.~--- 
New York, NY, USA: Dover Publications, 1974. 1046~p.
 \end{thebibliography}

 }
 }

\end{multicols}

\vspace*{-6pt}

\hfill{\small\textit{Поступила в~редакцию 12.06.17}}

\vspace*{8pt}

%\newpage

%\vspace*{-24pt}

\hrule

\vspace*{2pt}

\hrule

%\vspace*{8pt}


\def\tit{GAMMA-EXPONENTIAL FUNCTION\\ IN~BAYESIAN QUEUEING MODELS}

\def\titkol{Gamma-exponential function in~Bayesian queueing models}

\def\aut{A.\,A.~Kudryavtsev and A.\,I.~Titova}

\def\autkol{A.\,A.~Kudryavtsev and A.\,I.~Titova}

\titel{\tit}{\aut}{\autkol}{\titkol}

\vspace*{-9pt}


\noindent
\noindent
Department of Mathematical Statistics, Faculty of Computational 
Mathematics and Cybernetics, M.\,V.~Lomonosov Moscow State University, 
1-52~Leninskiye Gory, GSP-1, Moscow 119991, Russian Federation



\def\leftfootline{\small{\textbf{\thepage}
\hfill INFORMATIKA I EE PRIMENENIYA~--- INFORMATICS AND
APPLICATIONS\ \ \ 2017\ \ \ volume~11\ \ \ issue\ 4}
}%
 \def\rightfootline{\small{INFORMATIKA I EE PRIMENENIYA~---
INFORMATICS AND APPLICATIONS\ \ \ 2017\ \ \ volume~11\ \ \ issue\ 4
\hfill \textbf{\thepage}}}

\vspace*{3pt}


\Abste{This paper considers the Bayesian approach to queueing theory and 
reliability theory. The Bayesian approach is useful for studying systems 
with alternating characteristics, the changes in which happen at the moments 
of time unpredictable for a~researcher, or large groups of systems of the same type. 
In the framework of this approach, it is assumed that key parameters of 
classical systems are not given and only their \textit{a~priori} 
distributions are known. By randomizing the system's parameters, the authors 
randomize its characteristics, for instance, the traffic intensity. 
The gamma-exponential function and some of its properties are introduced 
 as well as the results for probability characteristics of 
the system's traffic intensity and the probability that the claim 
received by the system will not be lost in the cases of the exponential 
and Weibull \textit{a~priori} distributions of $M/M/1/0$ system's\linebreak parameters.}

\KWE{Bayesian approach; queuing systems; reliability; mixed distribution; 
Weibull distribution; exponential distribution; gamma-exponential function}

\DOI{10.14357/19922264170413} 

%\vspace*{-12pt}

\Ack
\noindent
The work was partly supported by the Russian Foundation for Basic 
Research (project 17-07-00577).



%\vspace*{3pt}

  \begin{multicols}{2}

\renewcommand{\bibname}{\protect\rmfamily References}
%\renewcommand{\bibname}{\large\protect\rm References}

{\small\frenchspacing
 {%\baselineskip=10.8pt
 \addcontentsline{toc}{section}{References}
 \begin{thebibliography}{9}


\bibitem{1-kud-1}
\Aue{Kudryavtsev, A.\,A., and S.\,Ya.~Shorgin.} 2015. \textit{Bayesovskie modeli 
v~teorii massovogo obsluzhivaniya i~nadezhnosti} 
[Bayesian models in mass service and reliability theories]. Moscow: FRC
CSC RAS. 76~p.

\bibitem{2-kud-1}
\Aue{Kudryavtsev, A.\,A., V.\,S.~Shorgin, and S.\,Ya.~Shorgin.} 
2009. Bayesovskie modeli massovogo obsluzhivaniya i~nadezhnosti: 
obshchiy erlangovskiy sluchay [Bayesian queueing and reliability models: 
General Erlang case]. \textit{Informatika i~ee Primeneniya~--- Inform. Appl.}
3(4):30--34.

\bibitem{3-kud-1}
\Aue{Zhavoronkova, Iu.\,V., A.\,A.~Kudryavtsev, and S.\,Ya.~Shor\-gin.} 
2014. Bayesovskaya rekurrentnaya mo\-del' ros\-ta na\-dezh\-nosti: beta-raspredelenie 
pa\-ra\-met\-rov [Bayesian recurrent model of reliability growth: 
Beta-distribution of\linebreak parameters]. \textit{Informatika i~ee Primeneniya~--- 
Inform. Appl.} 8(2):48--54.

\bibitem{4-kud-1}
\Aue{Zhavoronkova, Iu.\,V., A.\,A.~Kudryavtsev, and S.\,Ya.~Shor\-gin.} 2015.
Bayesovskaya rekurrentnaya mo\-del' ros\-ta na\-dezh\-nosti: beta-ravnomernoe
raspredelenie pa\-ra\-met\-rov [Bayesian recurrent model of reliability growth: 
Beta-uniform distribution of parameters].
\textit{Informatika i~ee Primeneniya~--- Inform. Appl.} 9(1):98--105.

\bibitem{5-kud-1}
\Aue{Artin, E.} 1964. \textit{The gamma function.} New York, NY: 
Holt, Rinehart and Winston. 39~p. 

\bibitem{6-kud-1}
\Aue{Bateman, H., and A.~Erdelyi.} 1953. 
\textit{Higher transcendental functions}. Vol.~1. New York\,--\,Toronto\,--\,London: 
McGraw-Hill Book Co., Inc. 302~p.

\bibitem{7-kud-1}
\Aue{Abramowitz, M., and I.~Stegun.} 
1974. \textit{Handbook of mathematical functions with formulas, graphs, and mathematical tables}. 
New York, NY: Dover Publications, Inc. 1046~p.
\end{thebibliography}

 }
 }

\end{multicols}

\vspace*{-6pt}

\hfill{\small\textit{Received June 12, 2017}}

%\vspace*{-10pt}

\Contr

\noindent
\textbf{Kudryavtsev Alexey A.} (b.\ 1978)~--- 
Candidate of Sciences (PhD) in physics and mathematics, associate professor, 
Department of Mathematical Statistics, Faculty of Computational Mathematics 
and Cybernetics, M.\,V.~Lomonosov Moscow State University, 1-52~Leninskiye Gory, 
GSP-1, Moscow 119991, Russian Federation; \mbox{nubigena@mail.ru}

\vspace*{3pt}

\noindent
\textbf{Titova Anastasiia I.} (b.\ 1995)~--- 
student, Department of Mathematical Statistics, Faculty of Computational 
Mathematics and Cybernetics, M.\,V.~Lomonosov Moscow State University, 
1-52~Leninskiye Gory, GSP-1, Moscow 119991, Russian Federation; 
\mbox{onkelskroot@gmail.com}
\label{end\stat}


\renewcommand{\bibname}{\protect\rm Литература}  %13
\renewcommand{\figurename}{\protect\bf Figure}
\renewcommand{\tablename}{\protect\bf Table}

\def\stat{pagano}

\def\tit{STUDY OF THE~MMPP/GI/$\infty$ QUEUEING SYSTEM\\ WITH~RANDOM CUSTOMERS' CAPACITIES}

\def\titkol{Study of the~MMPP/GI/$\infty$ queueing system with random customers' capacities}

\def\autkol{E.~Lisovskaya, S.~Moiseeva,   M.~Pagano, and~V.~Potatueva}

\def\aut{E.~Lisovskaya$^{1}$, S.~Moiseeva$^{1}$,   M.~Pagano$^{2}$, and~V.~Potatueva$^{1}$}

\titel{\tit}{\aut}{\autkol}{\titkol}

\index{Lisovskaya E.}
\index{Moiseeva S.}
\index{Pagano M.}
\index{Potatueva V.}
\index{Лисовская Е.\,Ю.}
\index{Моисеева С.\,П.}
\index{Пагано М.}
\index{Потатуева В.\,В.}

%{\renewcommand{\thefootnote}{\fnsymbol{footnote}}
%\footnotetext[1] {This work was supported in part by the
%Russian Foundation for Basic Research (grants 15-07-03007 and 13-07-00223).}}

\renewcommand{\thefootnote}{\arabic{footnote}}
\footnotetext[1]{Tomsk State University, 36~Lenin ave., Tomsk 634050, Russian Federation}
\footnotetext[2]{University of Pisa, 16~Via Caruso, Pisa 56122, Italy}


\vspace*{12pt}

\def\leftfootline{\small{\textbf{\thepage}
\hfill INFORMATIKA I EE PRIMENENIYA~--- INFORMATICS AND APPLICATIONS\ \ \ 2017\ \ \ volume~11\ \ \ issue\ 4}
}%
 \def\rightfootline{\small{INFORMATIKA I EE PRIMENENIYA~--- INFORMATICS AND APPLICATIONS\ \ \ 2017\ \ \ volume~11\ \ \ issue\ 4
\hfill \textbf{\thepage}}}

\Abste{A~queueing system with an infinite number of 
servers is considered. Customers arrive in the system according to a~Markov 
Modulated Poisson Process (MMPP). Each customer carries a~random quantity of 
work (capacity of the customer). In this study, service time does not depend 
on the customers' capacities; the latter are used just to fix some additional 
features of the system's evolution. It is shown that the joint probability 
distribution of the customers' number and total capacities in the system is 
two-dimensional Gaussian under the asymptotic condition of an infinitely 
growing service time. 
Simulation results allow determining the applicability area of the asymptotic result.}

\KWE{infinite-server queueing system; random capacity of customers; Markov Modulated Poisson Process}

\DOI{10.14357/19922264170414}

\vspace*{9pt}


\vskip 12pt plus 9pt minus 6pt

      \thispagestyle{myheadings}

      \begin{multicols}{2}

                  \label{st\stat}

\section{Introduction}

\noindent
Queueing systems represent a~powerful mathematical tool for investigating 
the performance of a~wide variety of real-life systems, ranging 
from telecommunication networks to financial markets, from computer 
architectures to supply chain management and airplane traffic control, 
just to cite a~few. Analytical tractability of the corresponding models 
strongly depends on the nature of the underlying processes (Poisson 
arrivals have many nice features that strongly simplify the analysis) 
and on the system geometry.

Although physical resources are always finite, quite often it is easier 
to study queueing systems in which the corresponding parameters assume 
infinite values. For instance, the overflow probability is often used 
as an upper bound for the loss probability in finite-buffer queues and, 
indeed, asymptotic results are available even for strongly non-Markovian 
systems~\cite{mandjes}. Moreover, infinite-server queueing systems may be 
applicable in case of models with a~limited number of server devices 
as described in~\cite{lit2}.

In this work,  an infinite-server queueing system, fed by 
non-Poisson arrivals with random customers' capacities, is considered.  
Queues with random customers' capacities are  useful for analysis 
and design issues in high-performance computer and communication systems, 
in which service time and customer volume are the independent quantities 
(see~\cite{lit8, lit10} and references therein). For instance, in~\cite{lit10}, 
performance analysis of LTE (Long Term Evolution) networks is carried out 
in terms of flow-level dynamics and the amount of required radio resources 
does not depend on the duration of the flow. Such queues are also important in 
modeling devices, where it is necessary to calculate a~sufficient volume of buffer 
for data storing~\cite{lit9, lit12}.
The results for single-server queues with limited buffer and LIFO 
(last in, first out) service discipline 
were presented in~\cite{lit13}, where algorithms for the calculation of stationary 
characteristics were derived. 

A new trend in the study of queueing systems is the analysis of the systems with 
non-Poisson arrivals and nonexponential service time. So, in the 
works~\cite{lit1, lit2, lit4, lit5, lit11}, queues with renewal arrivals, 
Markovian Arrival processes (MAP), and MMPP 
are studied under various asymptotic conditions. 
The main contribution of this paper consists in extending such analysis, 
focusing on the properties of the bidimensional process describing the 
number of customers and the total capacity in the system when an infinite-server 
queue is fed by MMPP arrivals with random capacities and nonexponential service 
time distribution.


\section{Matematical Model}

\noindent
Consider a~queue with infinite number of servers (Fig.~1) 
and assume that customers arrive according to an MMPP. The input process is 
defined by its generator matrix $\mathbf{Q}=||q_{ij}||$ of size $K\times K$ and 
the conditional rates $\lambda_1,\ldots,\lambda_K$, typically composed into 
the diagonal matrix $\mathbf{\Lambda}=\mathrm{diag}\,\{\lambda_1,\ldots,\lambda_K\} $. 
Denote the underlying\linebreak\vspace*{-12pt}

 { \begin{center}  %fig1
 \vspace*{-1pt}
 \mbox{%
 \epsfxsize=45.705mm 
 \epsfbox{pag-1.eps}
 }


\end{center}


\noindent
{{\figurename~1}\ \ \small{Queue MMPP/GI/$\infty$ with random customers' capacities}}
}

\vspace*{9pt}

\addtocounter{figure}{1}



\noindent
 Markov chain of the MMPP as $k(t) \in 1,2,\ldots,K$. 
Let each customer has some random capacity $v>0$ with distribution function~$G(y)$. 
An arriving customer instantly occupies a~server in the system and its service 
time has distribution function $B(x)$; when the service is completed, the customer 
leaves the system. Customers' capacities and service times are mutually 
independent and do not dependent on the epochs of customers' arrivals.

 


Denote by $i(t)$ and $V(t)$ the number of customers in the system at 
time~$t$ and their total capacity, respectively. Let us obtain the probabilistic 
characteristics of two-dimensional process~$\{i(t),V(t)\}$. This process 
is not Markovian; therefore,  the dynamic screening method has been used for 
its investigation.

Consider two time axes that are numbered as~0 and~1 (Fig.~2). 
Let axis~0 shows the epochs of customers' arrivals, while axis~1 
corresponds to the screened process.



Let us introduce a~function $S(t)$ (dynamic probability) that satisfies the condition 
$0\le S(t) \le 1$.
Let us assume that a~customer, arriving in the system at time~$t$, is screened 
to axis~1 with probability $S(t)$, and not screened with probability $1-S(t)$.

Let the system be empty at moment~$t_0$ and let us fix some arbitrary moment~$T$ 
in the future. $S(t)$ represents the probability that a~customer arriving at time~$t$ 
will be serviced in the system by the moment~$T$. 
It is easy to show~\cite{lit5} that $S(t)=1-B(T-t) $ for $t_0\le t\le T$.




Denote by $n(t)$ and $W(t)$ the number of arrivals screened before the moment~$t$ 
on axis~1 and their total\linebreak\vspace*{-12pt}

{ \begin{center}  %fig2
 \vspace*{9pt}
 \mbox{%
 \epsfxsize=77.897mm 
 \epsfbox{pag-2.eps}
 }

\vspace*{6pt}

\noindent
{{\figurename~2}\ \ \small{Screening of the customers' arrivals}}


\end{center}
}

%\vspace*{9pt}

\addtocounter{figure}{1}

\noindent
 capacity, respectively. 
As it is shown in~\cite{lit4}, the probability distribution of the number of 
customers in the system at the moment~$T$ coincides with the probability 
distribution of the number of screened arrivals on the axis:
$$
P\{i(T)=m\}=P\{n(T)=m\}
$$
for all $m=0,1,2,\ldots$ It is easy to prove the same property for 
the extended process $\{i(t),V(t)\}$:
\begin{multline}
\label{eq1-p}
P\{i(T)=m,V(T)<z\}\\
{}=P\{n(T)=m,W(T)<z\}
\end{multline}
for all $m=0,1,2,\ldots$ and $z\ge 0$. 
Let us use Eqs.~\eqref{eq1-p} for the investigation of the process $\{i(t),V(t)\}$ 
via the analysis of the process $\{n(t),W(t)\}$.

\section{Kolmogorov Differential Equations}

\noindent
Let us consider the three-dimensional Markovian process $\{k(t),n(t),W(t)\}$. 
Denoting the probability distribution of this process by 
$P(k,n,w,t)=P\{k(t)\linebreak =k,n(t)=n,W(t)<w\}$ and taking into account the formula 
of total probability, one can write the following system of Kolmogorov 
differential equations:
\begin{multline*}
\hspace*{-9pt}\fr{\partial P(k,n,w,t)}{\partial t}=\lambda_kS(t)\!\left[
\int\limits_0^z\!\!\! P(k,n-1,w-y,t)\,dG(y){}\right.\hspace*{-1pt}\\
\left.{}-P(k,n,w,t)
\vphantom{\int\limits_0^z}\right]
+\sum_vP(\nu,n,w,t)q_{\nu k}
\end{multline*}
for $k=1,\ldots , K$; $n=0,1,2,\ldots$; $w>0$.

Let us introduce the partial characteristic function:
$$
h(k,u_1,u_2,t)=\sum\limits_{n=0}^{\infty}e^{ju_1n}
\int\limits_0^\infty e^{ju_2w}P(k,n,dw,t)
$$
where $j=\sqrt{-1}$ is the imaginary unit. Then, one can write the following equations:
\begin{multline*}
\fr{\partial h(k,u_1,u_2,t)}{\partial t}\\
{}=
h\left(k,u_1,u_2,t\right)\lambda_kS(t)\left( e^{ju_1}G^*(u_2)-1\right) \\
{}+
\sum\limits_{\nu}h(\nu,n,w,t)q_{\nu k}
\end{multline*}
for $k=1,\ldots,K$ where $G^*(u)=\int\nolimits_0^\infty e^{juy}dG(y)$.

Let us rewrite this system in the matrix form:
\begin{multline}
\label{eq2-p}
\fr{\partial\mathbf{h}(u_1,u_2,t)}{\partial t}\\
{}=
\mathbf{h}(u_1,u_2,t)\left[
\mathbf{\Lambda} S(t)\left( e^{ju_1}G^*\left(u_2\right)-1\right) +\mathbf{Q}\right]
\end{multline}
with the initial condition
\begin{equation}
\label{eq3}
{\mathbf{h}}\left(u_1,u_2,t_0\right)=\mathbf{r}
\end{equation}
where
$$
\mathbf{h}\left(u_1,u_2,t\right)=\left[h\left(1,u_1,u_2,t\right),\ldots,
h\left(K,u_1,u_2,t\right)\right]
$$
and ${\mathbf{r}}=[r(1),\ldots,r(K)]$ represents the stationary 
distribution of the underlying Markov chain, i.\,e., 
vector~$\bf{r}$ satisfies the following linear system:
\begin{equation}
\label{eq4}
\left.
\begin{array}{l}
\mathbf{rQ}=\mathbf{0}\,; \\[6pt]
\mathbf{re}=1
\end{array}
\right\}
\end{equation}
where $\mathbf{e}$ is the~column vector with all entries equal to~1.

\section{Asymptotic Analysis}

\noindent
In general, the exact solution of Equation~\eqref{eq2-p} is not available, 
but it may be found under asymptotic conditions. In this paper,  
the case of infinitely growing service time is considered.

Denote by
$$
b_1=\int\limits_0^\infty xdB(x)=\int\limits_0^\infty(1-B(x))\,dx
$$
the mean service time; then, the asymptotic condition is $b_1\to\infty$.

Let us solve Problem \eqref{eq2-p}--\eqref{eq3} under such asymptotic condition 
and we obtain  approximate solutions with different order of accuracy, named as 
``first-order asymptotic'' 
${\mathbf{h}}(u_1,u_2,t)\approx{\mathbf{h}}^{(1)}(u_1,u_2,t)$ and  
``second-order asymptotic'' 
${\mathbf{h}}(u_1,u_2,t)\approx{\mathbf{h}}^{(2)}(u_1,u_2,t)$.

\subsection{First-order asymptotic analysis}

\noindent
Let us formulate and prove the following statement.

\smallskip

\noindent
\textbf{Lemma.}\ 
\textit{The first-order asymptotic characteristic function of the probability 
distribution of the process $\{k(t),n(t),W(t)\}$  has the form}:
\begin{equation*}
\mathbf{h}^{(1)}(u_1,u_2,t)=\mathbf{r} 
\exp\left\{ \!\left(ju_1\kappa_1+ju_2\kappa_1a_1\right)
\!\int\limits_{t_0}^t \! S(v)\,dv\!\right\}
\end{equation*}
\textit{where  $\kappa_1=\mathbf{r\Lambda e}$ and  
$a_1=\int\limits\nolimits_0^\infty ydG(y)$ is the mean customer capacity}.

\smallskip


\noindent
P\,r\,o\,o\,f\,.\ \ 
By performing the substitutions
\begin{gather*}
\varepsilon=\fr{1}{b_1}\,;\quad
 \varepsilon t=\tau\,;\quad 
 \varepsilon t_0=\tau_0\,;\\[6pt]
  u_1=\varepsilon x_1\,;\enskip
   u_2=\varepsilon x_2\,;\enskip 
   S(t)=S_1(\tau)\,;\\[6pt] 
   \mathbf{h}(u_1,u_2,t)=\mathbf{f}_1(x_1,x_2,\tau,\varepsilon)
%\label{eq5}
\end{gather*}
in expressions~\eqref{eq2-p} and~\eqref{eq3}, one obtains
\begin{multline}
\varepsilon\fr{\partial \mathbf{f}_1(x_1,x_2,\tau,\varepsilon)}{\partial \tau}\\
\!\!\!\!\!{}=
\mathbf{f}_1(x_1,x_2,\tau,\varepsilon)\!\left[\mathbf{\Lambda} 
S_1(\tau)\left( e^{j\varepsilon x_1}G^*(\varepsilon x_2)\!-\!1\right) +\mathbf{Q}\right]\!\!
\label{eq6}
\end{multline}
with the initial condition
\begin{equation}
\label{eq7}
\mathbf{f}_1(x_1,x_2,\tau_0,\varepsilon)=\mathbf{r}\,.
\end{equation}

Let us find the asymptotic solution of Problem~\eqref{eq6}--\eqref{eq7} 
$\mathbf{f}_1(x_1,x_2,\tau)=
\lim\nolimits_{\varepsilon\to 0}\mathbf{f}_1(x_1, x_2,\tau,\varepsilon)$
in two steps.

\textit{Step~1.} Let $\varepsilon\to 0$ in~\eqref{eq6}--\eqref{eq7}; 
then, one obtains the following system of equations:
$$
\left\{ 
\begin{array}{l}
\mathbf{f}_1\left(x_1,x_2,\tau\right)\mathbf{Q}=\mathbf{0}\,;\\[6pt]
\mathbf{f}_1\left(x_1,x_2,\tau_0\right)=\mathbf{r}\,.
\end{array}
\right.
$$

Taking into account~\eqref{eq4}, one can conclude that $\mathbf{f}_1(x_1,x_2,\tau)$  
can be expressed as
\begin{equation}
\label{eq8}
\mathbf{f}_1(x_1,x_2,\tau)=\mathbf{r}\Phi_1(x_1,x_2,\tau)
\end{equation}
where $\Phi_1(x_1,x_2,\tau)$ is some scalar function which satisfies the condition
\begin{equation}
\label{eq9}
\Phi_1(x_1,x_2,\tau_0)=1\,.
\end{equation}

\textit{Step 2.} Let us multiply~\eqref{eq6} by vector~{\bf e}, substitute~\eqref{eq8}, 
divide the result by~$\varepsilon$, and perform the asymptotic transition 
$\varepsilon\to 0$. Then, taking into account that $\mathbf{Qe}=\mathbf{0}$ 
and $\mathbf{re}=1$, one obtains the following differential equation 
for the function $\Phi_1(x_1,x_2,\tau)$:
\begin{multline}
\label{eq10}
\fr{\partial\Phi_1(x_1,x_2,\tau)}{\partial\tau}\\
{}=
\Phi_1\left(x_1,x_2,\tau\right)S_1(\tau)\left(jx_1\kappa_1+jx_2\kappa_1a_1\right)\,.
\end{multline}

The solution of Problem~\eqref{eq9}--\eqref{eq10} is as follows:
$$
\Phi_1(x_1,x_2,\tau)=\exp\left\{ \!\left(jx_1\kappa_1+jx_2\kappa_1a_1\right)\!
\int\limits_{\tau_0}^{\tau}\!S_1(v)\,dv\right\}.
$$
Substituting this expression into~\eqref{eq8}, one obtains
$$
\mathbf{f}_1(x_1,x_2,\tau)=
\mathbf{r}\exp\left\{ \!\left(jx_1\kappa_1+jx_2\kappa_1a_1\right)
\!\int\limits_{\tau_0}^{\tau}\!S_1(v)\,dv\!\right\}.\hspace*{-0.69418pt}
$$

Therefore, one can write
\begin{multline*}
\mathbf{h}^{(1)}(u_1,u_2,t)=\mathbf{f}_1\left(x_1,x_2,\tau,\varepsilon\right)\approx
\mathbf{f}_1\left(x_1,x_2,\tau\right)
\\
{}=\mathbf{r}\exp\left\{ \left(jx_1\kappa_1+jx_2\kappa_1a_1\right)
\int\limits_{\tau_0}^{\tau}S_1(v)dv\right\}\\
{} =
\mathbf{r}\exp\left\{ \left( ju_1\kappa_1+ju_2\kappa_1a_1\right) 
\int\limits_{t_0}^tS(v)\,dv\right\} \,.
\end{multline*}
Thus, the proof is complete.


\subsection{Second-order asymptotic analysis}

\noindent
The main result is the following theorem.

\smallskip

\noindent
\textbf{Theorem.}\ 
\textit{The second-order asymptotic characteristic function of the 
probability distribution of the process  $\{k(t),n(t),W(t)\}$ has the form}:

\noindent
\begin{multline*}
\mathbf{h}^{(2)}\left(u_1,u_2,t\right)\\
{}=\mathbf{r}\exp\left\{ 
\left(ju_1\kappa_1+ju_2\kappa_1a_1\right)\int\limits_{t_0}^tS(v)\,dv\right.
\\
{}+\fr{(ju_1)^2}{2}\left(\kappa_1\int\limits_{t_0}^tS(v)\,dv+
\kappa_2\int\limits_{t_0}^tS^2(v)\,dv\right)
\\
{}+\fr{(ju_2)^2}{2}\left(\kappa_1a_2\int\limits_{t_0}^tS(v)\,dv+
\kappa_2a_1^2\int\limits_{t_0}^tS^2(v)\,dv\right)
\\
\left.{}+ju_1ju_2\left(\kappa_1a_1\int\limits_{t_0}^tS(v)\,dv+
\kappa_2a_1\int\limits_{t_0}^tS^2(v)\,dv\right)\right\}\hspace*{-3.166pt}
\end{multline*}
\textit{where  $\kappa_2=2\mathbf{g}(\mathbf{\Lambda}-\kappa_1\mathbf{I})\mathbf{e}$;  
$a_2=\int\nolimits_0^\infty y^2dG(y)$;  and the row vector  $\mathbf{g}$  
satisfies the linear matrix system} 
$$
\left\{
\begin{array}{rl}
\mathbf{gQ}&=\mathbf{r}(\kappa_1\mathbf{I}-\mathbf{\Lambda})\,; \\[6pt]
\mathbf{ge}&=const\,.
\end{array}
\right.
$$



\noindent
P\,r\,o\,o\,f\,.\ \  
Let $\mathbf{h}_2(x_1,x_2,t)$ be a~vector function that satisfies the equation:
\begin{multline}
\label{eq12}
\mathbf{h}\left(u_1,u_2,t\right)=
\mathbf{h}_2\left(u_1,u_2,t\right)\\
{}\times\exp
\left\{ \!\left(ju_1\kappa_1+ju_2\kappa_1a_1\right)\int\limits_{t_0}^tS(v)\,dv\right\}\,.
\end{multline}

Substituting this expression into~\eqref{eq2-p} and~\eqref{eq3}, one obtains
the following problem:
\begin{multline}
\fr{\partial {\mathbf{h}_2(u_1,u_2,t)}}{\partial t}\\
{}=
\mathbf{h}_2(u_1,u_2,t)\left[(e^{ju_1}G^*(u_2)-1)S(t)\mathbf{\Lambda}\right.
\\
\left.{}-\left(ju_1\kappa_1+ju_2\kappa_1a_1\right)S(t)\mathbf{I}+\mathbf{Q}
\vphantom{e^{ju_1}G^*(u_2)}
\right]
\label{eq13}
\end{multline}
with the initial condition
\begin{equation}
\label{eq14}
{\bf h}_2(u_1,u_2,t_0)={\bf r}
\end{equation}
where {\bf I} is the identity matrix.

Let us make the substitutions:
\begin{equation}
\left.
\begin{array}{c}
\varepsilon^2=\fr{1}{b_1}\,;\quad
\varepsilon^2 t=\tau\,;\quad
\varepsilon^2 t_0=\tau_0\,;\\[6pt] 
u_1=\varepsilon x_1\,;\enskip 
u_2=\varepsilon x_2\,;\enskip 
S(t)=S_1(t)\,;\\[6pt] 
{\bf h}_2(u_1,u_2,t)={\bf f}_2(x_1,x_2,\tau,\varepsilon)\,.
\end{array}
\right\}
\label{eq15}
\end{equation}

Using these notations, Problem~\eqref{eq13}--\eqref{eq14} can be rewritten in the form
\begin{multline}
\varepsilon^2\fr{\partial {\bf f}_2(x_1,x_2,\tau,\varepsilon)}{\partial \tau}\\
{}=
{\bf{f}}_2(x_1,x_2,\tau,\varepsilon)\left[
\mathbf{\Lambda} S_1(\tau)(e^{j\varepsilon x_1}G^*(\varepsilon x_2)-1)\right. 
\\
\left.{} -\left( j\varepsilon\kappa_1x_1+j\varepsilon\kappa_1x_2a_1\right)
S_1\left( \tau\right) \mathbf{I}+ \mathbf{Q}
\vphantom{e^{j\varepsilon x_1}G^*(\varepsilon x_2)}
\right]
\label{eq16}
\end{multline}
with the initial condition
\begin{equation}
\label{eq17}
{\bf f}_2(x_1,x_2,\tau_0,\varepsilon)={\bf r}\,.
\end{equation}

Let us find the asymptotic solution of this problem 
${\bf f}_2(x_1,x_2,\tau)=
\lim\limits_{\varepsilon\to 0}{\bf f}_2(x_1,x_2,\tau,\varepsilon)$ in three steps.

\textit{Step~1.} Letting $\varepsilon\to 0$ in~\eqref{eq16}--\eqref{eq17}, 
one obtains the following system of equations:
$$
\left\{
\begin{array}{l}
{\bf f}_2\left(x_1,x_2,\tau\right)\mathbf{Q}=\mathbf{0}\,; \\[6pt]
{\bf f}_2\left(x_1,x_2,\tau_0\right)={\bf r}\,.
\end{array}
\right.
$$
Therefore, taking into account~\eqref{eq4}, one can write:
\begin{equation}
\label{eq18}
{\bf f}_2\left(x_1,x_2,\tau\right)={\bf r}\Phi_2\left(x_1,x_2,\tau\right)
\end{equation}
where $\Phi_2(x_1,x_2,\tau)$  is some scalar function which satisfies the condition
\begin{equation}
\label{eq19}
\Phi_2\left(x_1,x_2,\tau_0\right)=1\,.
\end{equation}

\textit{Step 2.} Using~\eqref{eq18}, the function ${\bf f}_2(x_1,x_2,\tau)$  
can be represented in the expansion form:
\begin{multline}
{\bf f}_2\left(x_1,x_2,\tau,\varepsilon\right)\\
{}=\Phi_2\left(x_1,x_2,\tau\right)\left[{\bf r}+\mathbf{g}S_1(\tau)
\left(j\varepsilon x_1+j\varepsilon x_2a_1\right)\right]\\
{}+{\bf O}(\varepsilon^2)
\label{eq20}
\end{multline}
where {\bf g} is the~row vector that satisfies the condition ${\bf ge}= const$ 
and ${\bf O}(\varepsilon^2)$  is the row vector of the second-order infinitesimals. 
Let us use substitution~\eqref{eq20} and the expansion
$$
e^{j\varepsilon x}=1+j\varepsilon x+O\left( \varepsilon^2\right) 
$$
in Eq.~\eqref{eq16}. Taking into account~\eqref{eq4} and 
making the transition $\varepsilon\to 0$, one obtains the 
following matrix equation for the vector~$\mathbf{g}$:
$$
{\bf gQ}={\bf r}\left(\kappa_1\mathbf{I}-\mathbf{\Lambda}\right)\,.
$$

\textit{Step~3.} Let us multiply Eq.~\eqref{eq16} by vector~\textbf{e} 
and use expression~\eqref{eq20} and the second-order expansion:
$$
e^{j\varepsilon x}=1+j\varepsilon x+\fr{(j\varepsilon x)^2}{2}+O\left(\varepsilon^3\right)\,.
$$

After some transformations, using the notation
$$
\kappa_2=2{\bf g}\left(\mathbf{\Lambda}-\kappa_1\mathbf{I}\right){\bf{e}}\,,
$$
one obtains the following differential equation for the function $\Phi_2(x_1,x_2,\tau)$:
\begin{multline*}
\fr{\partial\Phi_2(x_1,x_2,\tau)}{\partial\tau}\\
{}=
\Phi_2(x_1,x_2,\tau) \left[\fr{(jx_1)^2}{2}\left(\kappa_1S_1(\tau)+
\kappa_2S_1^2(\tau)\right)\right.
\\
{}+\fr{(jx_2)^2}{2}\left(\kappa_1a_2S_1(\tau)+\kappa_2a_1^2S_1^2(\tau)\right)\\
\left.{}+jx_1jx_2\left(\kappa_1a_1S_1(\tau)+\kappa_2a_1S_1^2(\tau)\right)
\vphantom{\fr{(jx_1)^2}{2}}\right]\,.
\end{multline*}

The solution of this equation with initial condition~\eqref{eq19} is as follows:
\begin{multline*}
\Phi_2\left(x_1,x_2,\tau\right)\\
{}= 
\exp\left\{ \fr{(jx_1)^2}{2}\left(
\kappa_1\int\limits_{\tau_0}^{\tau}S_1(v)\,dv+\kappa_2\int\limits_{\tau_0}^{\tau}
S_1^2(v)\,dv\right)\right.
\\
{}+\fr{(jx_2)^2}{2}\left(\kappa_1a_2\int\limits_{\tau_0}^{\tau}S_1(v)\,dv+
\kappa_2a_1^2\int\limits_{\tau_0}^{\tau}\!S_1^2(v)\,dv\right)
\\
\!\left.{}+
jx_1jx_2\left(\kappa_1a_1\int\limits_{\tau_0}^{\tau}\!S_1(v)\,dv+
\kappa_2a_1\int\limits_{\tau_0}^{\tau}\!S_1^2(v)\,dv\right)\!\right\}.\hspace*{-0.39064pt}
\end{multline*}

Substituting this expression in formula~\eqref{eq18} and performing 
the substitutions that are inverse to~\eqref{eq15} and~\eqref{eq12}, one obtains
\begin{multline*}
{\bf{h}}^{(2)}\left(u_1,u_2,t\right)\\
{}=
{\bf{r}} \exp\left\{ \left(ju_1\kappa_1+ju_2\kappa_1a_1\right)
\int\limits_{t_0}^tS(v)\,dv\right. 
\\
{}+\fr{(ju_1)^2}{2}\left(\kappa_1\int\limits_{t_0}^tS(v)\,dv+
\kappa_2\int\limits_{t_0}^tS^2(v)\,dv\right)\\
{}+\fr{(ju_2)^2}{2}\left(\kappa_1a_2\int\limits_{t_0}^tS(v)\,dv+
\kappa_2a_1^2\int\limits_{t_0}^tS^2(v)\,dv\right)
\\
\left.
{}+ju_1ju_2\left(\kappa_1a_1\int\limits_{t_0}^tS(v)\,dv+
\kappa_2a_1\int\limits_{t_0}^tS^2(v)\,dv\right)\right\} 
\end{multline*}
for the asymptotic characteristic function of the process
 $\{k(t),n(t),W(t)\}$. The proof is complete.
 
\columnbreak
 
 \noindent
 \textbf{Corollary.}
Assuming $t = T$ and $t_0\to -\infty $ and using Eqs.~\eqref{eq1-p}, one obtains 
the steady-state characteristic function of the process under study $\{i(t),V(t)\}$:
\begin{multline}
h\left(u_1,u_2\right)= 
\exp\left\{ \left(ju_1\kappa_1b_1+ju_2\kappa_1a_1b_1\right)\right.\\
{}+
\fr{(ju_1)^2}{2}\left(\kappa_1b_1
+\kappa_2b_2\right)
+\fr{(ju_2)^2}{2}\left(\kappa_1a_2b_1+\kappa_2a_1^2b_2\right)\\
\left.{}+
ju_1ju_2\left(\kappa_1a_1b_1+\kappa_2a_1b_2\right)\right\}
\label{eq21}
\end{multline}
where 
$$
b_1=\int\limits_0^{\infty}(1-B(v))\,dv\,;\enskip  
b_2=\int\limits_0^{\infty}(1-B(v))^2\,dv\,.
$$


From the form of the characteristic function~\eqref{eq21}, it is clear 
that the probability distribution of the two-dimensional process $\{i(t),V(t)\}$ 
is asymptotically Gaussian with vector of means
$$
{\bf a}=\left[
\begin{array}{lr}
\kappa_1b_1 &  \kappa_1a_1b_1
\end{array}
\right]
$$
and covariance matrix
\begin{multline*}
\mathbf{K}=\left[
\begin{array}{cc}
\sigma_1^2 &  K_{12} \\
K_{12} & \sigma_2^2
\end{array}
\right]\\
{}=
\left[
\begin{array}{cc}
\kappa_1b_1+\kappa_2b_2 &  \kappa_1a_1b_1+\kappa_2a_1b_2 \\
\kappa_1a_1b_1+\kappa_2a_1b_2 & \kappa_1a_2b_1+\kappa_2a_1^2b_2
\end{array}
\right]\,.
\end{multline*}

Therefore, the correlation coefficient is given by
$$
r=\fr{K_{12}}{\sigma_1\sigma_2}=
\fr{\kappa_1a_1b_1+\kappa_2a_1b_2}{\sqrt{\kappa_1b_1+\kappa_2b_2}\,
\sqrt{\kappa_1a_2b_1+\kappa_2a_1^2b_2}}\,.
$$

\begin{figure*}[b] %fig3
\vspace*{1pt}
 \begin{center}
 \mbox{%
 \epsfxsize=163.767mm 
 \epsfbox{pag-3.eps}
 }
  \end{center}
\vspace*{-11pt}
\Caption{Distributions of the number of customers~(\textit{a})
and of the total capacity~(\textit{b}) for different values of~$N$:
left column~--- $N=10$; right column~--- $N=100$;
\textit{1}~--- theoretical results; and \textit{2}~--- simulation}
\label{fig:fig3}
\end{figure*}

\section{Numerical Example}

\noindent
Result~\eqref{eq21} is obtained under the asymptotic condition $b_1\to\infty$. 
Therefore, it may be used just as an approximation when~$b_1$ is large enough. 
To test its practical applicability, the present authors
 considered several numerical examples, 
varying all the system parameters (including the distributions of the service 
time and of the customer capacity). Since all the different simulation sets led 
to similar results, for sake of brevity, in the following, 
just one of them is discussed in detail. In particular, let us
assume that the input MMPP is characterized by the matrices:
$$
\mathbf{Q}=\left[
\begin{array}{rrr}
-0.8 & 0.4 & 0.4\\
0.3 & -0.6 & 0.3\\
0.4 & 0.4 & -0.8
\end{array}\right]
$$
and
$$
\mathbf{\Lambda}=\left[
\begin{array}{ccc}
0.5 & 0 & 0\\
0 & 1 & 0\\
0 & 0 & 1.5
\end{array}\right] \,.
$$

 %\begin{table*} %tabl1
\begin{center}
\begin{minipage}[t]{34mm}
{{\tablename~1}\ \ \small{Kolmogorov distances between simulation results and asymptotic 
values for the number of customers in the system}}

\vspace*{6pt}


{\small
\tabcolsep=14pt
\begin{tabular}{cc}
\hline
$N$ & $\Delta$\\
\hline
\hphantom{9}1 & 0.265\\
10 & 0.039\\
15 & 0.032\\
20 & \textbf{0.027}\\
25 & \textbf{0.025}\\
50 & \textbf{0.017}\\
100\hphantom{9} & \textbf{0.012}\\
\hline
\end{tabular}
}
\end{minipage}
\hfill
%\end{center}
%\end{table*}
%\begin{table*}\small %tabl2
%\begin{center}
\begin{minipage}[t]{34mm}
{{\tablename~2}\ \ \small{Kolmogorov distances between simulation results and asymptotic values for the total capacity in the system}}

\vspace*{6pt}

{\small 
\tabcolsep=14pt
\begin{tabular}{cc}
\hline
$N$ & $\Delta$\\
\hline
\hphantom{9}1 & 0.355\\
10 & 0.033\\
15 & \textbf{0.025}\\
20 & \textbf{0.021}\\
25 & \textbf{0.019}\\
50 & \textbf{0.013}\\
100\hphantom{9} &\textbf{0.010}\\
\hline
\end{tabular}
}
\end{minipage}
\end{center}

\vspace*{9pt}
%\end{table*}

\setcounter{table}{2}

Hence, the fundamental rate of arrivals is $\kappa_1\linebreak =\mathbf{r\Lambda e}=1$ 
customers per time unit. Let us also assume that customers' capacities have 
uniform distribution in the range $[0;1]$ and service time has gamma distribution 
with shape and inverse scale parameters $\alpha = 1.5$ and $\beta = \alpha/ N$, 
respectively. So, when $N \to\infty$, one obtains the asymptotic condition of 
an infinite growing service time ($b_1 = \alpha / \beta = N \to \infty$).

The goal is to find a~lower bound of parameter~$N$ for the applicability 
of the approximation~\eqref{eq21}. To this aim,  series of
 simulation experiments have been  carried out for increasing values of~$N$ and 
  the asymptotic
  distributions    have been compared with the
  empiric ones by using the Kolmogorov distance~\cite{lit3,lit6}
$$
\Delta=\sup\limits_x\left| F\left( x\right) -A\left( x\right) \right| 
$$
as an accuracy measure. Here,~$F(x)$ is the cumulative distribution function 
built on the basis of simulation results and $A(x)$ is the Gaussian 
approximation based on~\eqref{eq21}.

Let us consider the marginal distributions of the customers' number and the 
total capacity in the system.

In the first case, the asymptotic values of mean and variance are equal 
to~$N$ and~$1.144 N$, respectively, and the corresponding values of the 
Kolmogorov distance for increasing values of parameter~$N$ are presented in 
Table~1. Similarly, for the total capacity in the system, 
mean and variance are equal to~$0.5 N$ and~$0.369 N$, respectively, and 
Table~2 shows the Kolmogorov distance.


One can notice that the asymptotic results become more accurate 
when the parameter~$N$  increases. Fig-\linebreak\vspace*{-12pt}

 { \begin{center}  %fig1
 \vspace*{-1pt}
 \mbox{%
 \epsfxsize=77.763mm 
 \epsfbox{pag-5.eps}
 }


\end{center}


\noindent
{{\figurename~4}\ \ \small{Relative error for the variance of the number of customers~$i(t)$~(\textit{1}) 
and the total capacity $V(t)$~(\textit{2})}}
}

\vspace*{18pt}

\addtocounter{figure}{1}


\noindent
ure~\ref{fig:fig3}  
compares the asymptotic approximations with the empirical results for 
the number of customers and the total capacity in the system.



As typically done in the literature~\cite{lit6}, let us suppose that 
an approximation is applicable if its Kolmogorov distance is less than~0.03. 
Hence, one can  conclude that the asymptotic results are applicable for values
 of the parameter~$N$ equal to~15 or more (marked by boldface in 
 Tables~1 and~2).
 



Then, let us compare the asymptotic value of some characteristics of the 
queueing system with the corresponding empirical characteristics, 
using the relative error
$$
\delta=\fr{\left| d-a\right| }{d}
$$
where $d$ denotes the value constructed on the basis of simulation results and~$a$ 
is obtained from~\eqref{eq21}.

In more detail, the mean values of the processes~$i(t)$ and $V(t)$ are very
 close (with $\delta<10^{-5}$ for all~$N$) and the relative errors 
 of the variance decreases with~$N$ as shown in Fig.~4.



Finally, Table~3 shows the relative error for 
the correlation coefficient.

\vspace*{12pt}

%\begin{table*}\small %tabl3
\begin{center}
 \parbox{43mm}{{{\tablename~3}\ \ \small{Relative error for the correlation coefficient}}
 }
\vspace*{6pt}

        \tabcolsep=15pt
        {\small \begin{tabular}{cc}
            \hline
                        \multicolumn{2}{c}{\ }\\[-9pt]
            $N$ & $\delta$\\
            \hline
            \multicolumn{2}{c}{\ }\\[-9pt]
\hphantom{9}1 & {\boldmath{${60\cdot10^{-4}}$}}\hphantom{9}\\
            10 & {\boldmath{$11\cdot10^{-4}$}}\hphantom{9} \\
            15 &             {\boldmath{$7\cdot10^{-4}$}} \\
             20 & {\boldmath{$5\cdot10^{-4}$}} \\
             25 & {\boldmath{$4\cdot10^{-4}$}}\\
             50 &     {\boldmath{$1\cdot10^{-4}$}}\\
             100\hphantom{9}&    {\boldmath{$0.8\cdot10^{-4}$}}\hphantom{.9}\\
            \hline
        \end{tabular}}
    \end{center}
%\end{table*}

\section{Concluding Remarks}

\noindent
In the paper, the queue with MMPP arrivals, infinite number of servers, 
and nonexponential service time is considered. Moreover, random customers' capacities, 
independent of their service time, are assumed.
The analysis is performed under the asymptotic condition of an infinitely 
growing service time. It is shown that two-dimensional probability 
distribution of customers' number and total capacity in the system is 
two-dimensional Gaussian under this asymptotic condition. Numerical 
results show that asymptotic results have enough accuracy for 
the marginal distributions of number of customers and of the 
total capacity in the system when the service rate exceeds the 
fundamental rate of arrivals by at least~15~times.

\vspace*{-6pt}

\Ack
\noindent
This work is supported by the Russian Foundation for Basic research, project 16-31-00292.

\renewcommand{\bibname}{\protect\rmfamily References}

\vspace*{-6pt}

{\small\frenchspacing
{%\baselineskip=10.8pt
\begin{thebibliography}{99}

\bibitem{mandjes} %1
\Aue{Mandjes, M.} 2007. \textit{Large deviations of Gaussian queues.} 
Chichester: Wiley. 340~p.

\bibitem{lit2} %2
\Aue{Melikov, A., L.~Zadiranova, and A.~Moiseev.} 2016. 
Two asymptotic conditions in queue with MMPP arrivals and feedback. 
\textit{Comm. Com. Inf. Sc.} 678:231--240. 
doi: 10.1007/978-3-319-51917-3\_21.

\bibitem{lit10} %3
\Aue{Naumov, V., K.~Samouylov, E.~Sopin, and S.~Andreev.} 2015. 
Two approaches to analyzing dynamic cellular networks with limited resources. 
\textit{6th  Congress (International)
on Ultra Modern Telecommunications and Control Systems and Workshops.} 
St.\ Petersburg. 485--488. 
doi: 10.1109/ICUMT.2014.7002149.

\bibitem{lit8}%4
\Aue{Morozov, E., L.~Potakhina, and O.~Tikhonenko.} 2016. 
Regenerative analysis of a~system with a~random volume of customers. 
\textit{Comm. Com. Inf. Sc.} 638:261--272. 
doi: 10.1007/978-3-319-44615-8\_23.

\bibitem{lit12} %5
\Aue{Tikhonenko, O.\,M., and W.~Kempa.} 2015. 
Queueing systems with processor sharing and limited memory under control of the AQM 
mechanism. \textit{Automat. Rem. Contr.} 76(10):1784--1796. 
doi: 10.1134/S0005117915100069.

\bibitem{lit9} %6
\Aue{Naumov, V.\,A., K.\,E.~Samuilov, and A.\,K.~Samuilov}. 2016. 
On the total amount of resources occupied by serviced customers. 
\textit{Automat. Rem. Contr.} 77(8):1419--1427.



\bibitem{lit13} %7
\Aue{Tikhonenko, O.\,M.} 2010. 
Queueing system with processor sharing and limited resources. 
\textit{Automat. Rem. Contr.} 71(5):803--815.

\bibitem{lit11} %8
\Aue{Pankratova, E.\,V., and S.\,P.~Moiseeva.} 2014. 
Queueing system MAP/M/$\infty$ with $n$ types of customers. 
\textit{Comm. Com. Inf. Sc.} 487:356--366.

\bibitem{lit4} %9
\Aue{Moiseev, A., and A.~Nazarov.} 2016. 
Tandem of infinite-server queues with Markovian arrival process. 
\textit{Comm. Com. Inf. Sc.} 601:323--333. 
doi: 10.1007/978-3-319-30843-2\_34.

\bibitem{lit1} %10
\Aue{Lisovskaya, E., S.~Moiseeva, and M.~Pagano.}  2016. 
The total capacity of customers in the infinite-server queue with MMPP arrivals. 
\textit{Comm. Com. Inf. Sc.} 678:110--120. 
doi: 10.1007/978-3-319-51917-3\_11.
    


\bibitem{lit5} %11
\Aue{Moiseev, A., and A.~Nazarov.} 2016. Queueing network MAP/(GI/$\infty$)$^K$ 
with high-rate arrivals. \textit{Eur. J.~Oper. Res.} 254:161--168. 
doi: 10.1016/j.ejor.2016.04.011.





\bibitem{lit6} %12
\Aue{Moiseev, A.\,N., and M.\,V.~Sinyakov.} 2010. 
Razrabotka ob''ektno-orientirovannoy modeli sistemy imitatsionnogo modelirovaniya 
protsessov massovogo obsluzhivaniya 
[Design of object-oriented model for queueing simulation software]. 
\textit{Vestnik Tomskogo gosudarstvennogo universiteta. Upravlenie, 
vychislitel'naya tekhnika i informatika} 
[Tomsk State University. J.~Control Computer Sci.] 1:89--93.

\bibitem{lit3} %13
\Aue{Moiseev, A., A.~Demin, V.~Dorofeev, and V.~Sorokin.} 2016. 
Discrete-event approach to simulation of queueing networks. 
\textit{Key Eng. Mater.} 685:939--942. 
doi: 10.4028/www.scientific.net/KEM.685.939.

    
\end{thebibliography} }
 }

\end{multicols}

\vspace*{-6pt}

\hfill{\small\textit{Received March 16, 2017}}

\vspace*{-18pt}

\Contr

\noindent
\textbf{Lisovskaya Ekaterina Yu.} (b.\ 1992)~--- PhD student,
Department of Probability 
Theory and Mathematical Statistics, Tomsk State University, 36~Lenin Ave., 
Tomsk 634050, Russian Federation; 
\mbox{ekaterina\_lisovs@mail.ru}

\vspace*{3pt}

\noindent
\textbf{Moiseeva Svetlana P.} (b.\ 1971)~--- Doctor of Science in physics 
and mathematics, 
associate professor, professor, Department of Probability Theory
 and Mathematical Statistics, Tomsk State University, 36~Lenin ave., Tomsk 634050, 
 Russian Federation; \mbox{smoiseeva@mail.ru}

\vspace*{3pt}

\noindent
\textbf{Pagano Michele} (b.\ 1968)~--- PhD in electronics engineering,  
professor, Department of Information Engineering of University of 
Pisa, 16~Via Caruso, Pisa 56122, Italy; \mbox{m.pagano@iet.unipi.it}

\vspace*{3pt}

\noindent
\textbf{Potatueva Viktoriya V.} (b.\ 1993)~--- Master Degree student,
Department 
of Probability Theory and Mathematical Statistics, Tomsk State University, 
36~Lenin Ave., Tomsk 634050, Russian Federation; \mbox{ve-kusik@mail.ru}

\vspace*{8pt}

\hrule

\vspace*{2pt}

\hrule

%\newpage

%\vspace*{-24pt}



\def\tit{ИССЛЕДОВАНИЕ СИСТЕМЫ МАССОВОГО ОБСЛУЖИВАНИЯ MMPP/GI/$\infty$ 
С~ТРЕБОВАНИЯМИ СЛУЧАЙНОГО ОБЪЕМА$^*$}

\def\aut{Е.\,Ю.~Лисовская$^1$, С.\,П.~Моисеева$^2$, М.~Пагано$^3$, В.\,В.~Потатуева$^4$}


\def\titkol{Исследование системы массового обслуживания MMPP/GI/$\infty$ 
с~требованиями случайного объема}

\def\autkol{Е.\,Ю.~Лисовская, С.\,П.~Моисеева, М.~Пагано, В.\,В.~Потатуева}

{\renewcommand{\thefootnote}{\fnsymbol{footnote}}
\footnotetext[1]{Работа выполнена при частичной поддержке РФФИ (проект 16-31-00292).}}


\titel{\tit}{\aut}{\autkol}{\titkol}

\vspace*{-12pt}

\noindent
$^1$Национальный исследовательский Томский государственный университет,
\mbox{ekaterina\_lisovs@mail.ru}

\noindent
$^2$Национальный исследовательский Томский государственный университет,
\mbox{smoiseeva@mail.ru}

\noindent
$^3$Университет г.\ Пиза, Италия, \mbox{m.pagano@iet.unipi.it} 

\noindent
$^4$Национальный исследовательский Томский государственный университет,
\mbox{ve-kusik@mail.ru}

\vspace*{6pt}

\def\leftfootline{\small{\textbf{\thepage}
\hfill ИНФОРМАТИКА И ЕЁ ПРИМЕНЕНИЯ\ \ \ том\ 11\ \ \ выпуск\ 4\ \ \ 2017}
}%
 \def\rightfootline{\small{ИНФОРМАТИКА И ЕЁ ПРИМЕНЕНИЯ\ \ \ том\ 11\ \ \ выпуск\ 4\ \ \ 2017
\hfill \textbf{\thepage}}}


\Abst{Проведено исследование системы массового обслуживания с неограниченным 
числом приборов. Заявки поступают в систему в виде мар\-ков\-ски-мо\-ду\-ли\-ро\-ван\-но\-го 
пуассоновского потока. Каждая заявка несет в себе произвольное количество 
данных (объем заявки). В~этом исследовании время обслуживания не зависит от 
объема заявок. Показано, что совместное распределение вероятностей числа заявок в системе 
и~их суммарного объема является двумерным гауссовским при асимптотическом условии 
растущего времени обслуживания. Имитационное моделирование и численные эксперименты 
позволили определить область применимости асимптотического результата.}

\KW{бесконечнолинейная система массового обслуживания; случайный объем заявок; 
MMPP-поток}

\DOI{10.14357/19922264170414}

%\vspace*{18pt}


 \begin{multicols}{2}

\renewcommand{\bibname}{\protect\rmfamily Литература}
%\renewcommand{\bibname}{\large\protect\rm References}

{\small\frenchspacing
{%\baselineskip=10.8pt
\begin{thebibliography}{99}

\bibitem{mandjes-1} %1
\Au{Mandjes M.} Large deviations of Gaussian queues.~--- 
Chichester: Wiley, 2007. 340~p.

\bibitem{lit2-1}  %2
\Au{Melikov~A.,  Zadiranova~L., Moiseev~A.}  
Two asymptotic conditions in queue with MMPP arrivals and feedback~// 
Comm. Com. Inf. Sc., 2016. Vol.~678. P.~231--240.

\bibitem{lit10-1} %3
\Au{Naumov~V., Samouylov~K.,  Sopin~E.,  Andreev~S.} 
Two approaches to analyzing dynamic cellular networks with limited resources~//  
6th Congress (International)
on Ultra Modern Telecommunications and Control Systems and Workshops.~--- 
St.\ Petersburg, 2015. P.~485--488.
doi: 10.1007/978-3-319-44615-8\_23.

\bibitem{lit8-1} %4
\Au{Morozov  E.,  Potakhina~L.,  Tikhonenko~O.}  
Regenerative analysis of a~system with a~random volume of customers~// 
Comm. Com. Inf. Sc., 2016. Vol.~638. P.~261--272.
doi: 10.1007/978-3-319-44615-8\_23.





\bibitem{lit12-1} %5
\Au{Тихоненко О.\,М., Кемпа В\,.М.} 
Система с~разделением процессора и~ограниченным объемом памяти,
управ\-ля\-емая механизмом AQM~//
Автоматика и~телемеханика, 2015.
№\,10. С.~90--105.

\bibitem{lit9-1}  %6
\Au{Наумов В.\,А., Самуйлов~К.\,Е., Самуйлов~А.\,К.} 
О~суммарном объеме ресурсов, занимаемых обслуживаемыми заявками~// 
Автоматика и телемеханика, 2016. №\,8. C.~125--132.

\bibitem{lit13-1} %7 
\Au{Тихоненко О.\,М.} 
Система обслуживания с~разделением процессора и~ограниченными ресурсами~//
Queueing systems with processor sharing and limited resources~// 
Автоматика и~телемеханика, 2010. №\,5. С.~84--98.

\bibitem{lit11-1} %8
\Au{Pankratova  E.\,V., Moiseeva~S.\,P.} Queueing system 
MAP/M/$\infty$ with $n$ types of customers~// 
Comm. Com. Inf. Sc., 2014. Vol.~487. P.~356--366.

\bibitem{lit4-1} %9
\Au{Moiseev  A., Nazarov~A.} 
Tandem of infinite-server queues with Markovian arrival process~// 
Comm. Com. Inf. Sc., 2016. Vol.~601. P.~323--333.
doi: 10.1007/978-3-319-30843-2\_34.

\bibitem{lit1-1} %10
\Au{Lisovskaya~E., Moiseeva~S., Pagano~M.}  
The total capacity of customers in the infinite-server queue with MMPP arrivals~// 
Comm. Com. Inf. Sc., 2016. Vol.~678. P.~110--120.
doi: 10.1007/978-3-319-51917-3\_11.
    


\bibitem{lit5-1} %11
\Au{Moiseev~A., Nazarov~A.} 
Queueing network MAP/(GI/$\infty$)$^K$ with high-rate arrivals~// Eur. 
J.~Oper. Res., 2016. Vol.~254. P.~161--168.
doi: 10.1016/ j.ejor.2016.04.011.





\bibitem{lit6-1}  %12
\Au{Моисеев А.\,Н., Синяков~М.\,В.} Разработка 
объ\-ект\-но-ори\-ен\-ти\-ро\-ван\-ной модели системы имитационного\linebreak 
моделирования процессов массового обслуживания~// 
Вестник Томского государственного университета. 
Управление, вычислительная техника и информатика, 2010. №\,1. С.~89--93.

\bibitem{lit3-1}  %13
\Au{Moiseev~A., Demin~A., Dorofeev~V., Sorokin~V.} 
Discrete-event approach to simulation of queueing networks~// 
Key Eng. Mater., 2016. Vol.~685. P.~939--942.
doi: 10.4028/www.scientific.net/KEM.685.939.

\end{thebibliography}
} }

\end{multicols}

 \label{end\stat}

 \vspace*{-3pt}

\hfill{\small\textit{Поступила в~редакцию  16.03.2017}}
%\renewcommand{\bibname}{\protect\rm Литература}
\renewcommand{\figurename}{\protect\bf Рис.}
\renewcommand{\tablename}{\protect\bf Таблица}    %14
\renewcommand{\figurename}{\protect\bf Figure}
\renewcommand{\tablename}{\protect\bf Table}

\def\stat{kruzhkov}

\def\tit{APPROACHES TO ANNOTATION OF~DISCOURSE RELATIONS 
IN~LINGUISTIC CORPORA}

\def\titkol{Approaches to annotation of~discourse relations 
in~linguistic corpora}

\def\autkol{M.\,G.~Kruzhkov}

\def\aut{M.\,G.~Kruzhkov$^1$}

\titel{\tit}{\aut}{\autkol}{\titkol}

\index{Kruzhkov M.\,G.}
\index{Кружков М.\,Г.}

%{\renewcommand{\thefootnote}{\fnsymbol{footnote}}
%\footnotetext[1] {This work was supported in part by the
%Russian Foundation for Basic Research (grants 15-07-03007 and 13-07-00223).}}

\renewcommand{\thefootnote}{\arabic{footnote}}
\footnotetext[1]{Institute of Informatics Problems, Federal Research Center ``Computer Science and Control'' of 
the Russian Academy of Sciences, 44-2 Vavilov Str., Moscow 119333, Russian Federation}


\vspace*{12pt}

\def\leftfootline{\small{\textbf{\thepage}
\hfill INFORMATIKA I EE PRIMENENIYA~--- INFORMATICS AND APPLICATIONS\ \ \ 2017\ \ \ volume~11\ \ \ issue\ 4}
}%
 \def\rightfootline{\small{INFORMATIKA I EE PRIMENENIYA~--- INFORMATICS AND APPLICATIONS\ \ \ 2017\ \ \ volume~11\ \ \ issue\ 4
\hfill \textbf{\thepage}}}
  
   
% \vspace*{10pt} 
  
  
  
  \Abste{This paper examines the Supracorpora Database of Connectives  
(SCDB-Connectives) that is based on data from parallel corpora. The  
SCDB-Connectives provides structural and semantic annotation of Russian 
connectives and their translation correspondences in French (and, eventually, in other 
languages). The SCDB-Connectives annotation approach is compared to the latest 
developments in the area of annotation of discourse relations~--- the annotated corpus 
of discourse relations Penn Discourse Treebank (PDTB) and the proposed standard 
for annotation of semantic relations ISO 24617-8, some of the important differences 
are discussed. Penn Discourse Treebank and ISO 24617-8 support annotation of both explicit
and implicit discourse relations 
while SCDB-Connectives only annotates explicit relations, 
i.\,e., those expressed by connectives. Furthermore, PDTB and ISO 24617-8 provide 
a~superior framework for annotating text spans as relation arguments, which allows 
annotating attribution for these arguments, such as source and type of the linked 
propositions. In addition, ISO 24617-8 specifies argument roles for asymmetrical 
discourse relations. On the other hand, the principle advantage of the  
SCDB-Connectives is that it supports annotation of both connectives and their translation 
correspondences in parallel corpora, opening up new possibilities for contrastive 
studies. The SCDB-Connectives is based on a~relational database rather than on the 
XML format, which helps to manage complex cross-linguistic data efficiently. 
Benefits of semantic annotation of connectives for both theoretical and practical 
purposes are also discussed.}
  
  
  \KWE{discourse relations; discourse connectives; corpus linguistics; parallel 
corpora; supracorpora databases}

\DOI{10.14357/19922264170415}

\vspace*{9pt}


\vskip 12pt plus 9pt minus 6pt

      \thispagestyle{myheadings}

      \begin{multicols}{2}

                  \label{st\stat}
  
  \section{Introduction}
  
  \noindent
  As a~part of the project ``Logical structure of text: Means for the expression of 
logical-semantic relations in Russian, French, and Italian from the contrastive 
perspective'' (funded by the Russian Science Foundation, grant No.\,16-18-10004), 
computer-aided semantic and structural annotation of Russian connectives in 
bilingual parallel corpora of the Russian National Corpus ({\sf http://ruscorpora.ru}) is 
being performed (Russian--French and French--Russian corpora are mostly used). 
{\looseness=1

}

For 
each Russian connective, a~corresponding French fragment is annotated in the 
parallel corpus. Usually, parallel French fragments also include connectives, but in 
some cases, Russian connectives may correspond to other lexical or grammatical 
items and occasionally, there is no apparent correspondence in French. These pairs of 
Russian and French fragments (hereafter referred to as ``translation 
correspondences,'' TCs) are annotated by linguistic experts and saved in a~dedicated 
database called The Supracorpora Database of Connectives.
  
  Thus, the SCDB are created as customizable banks of 
annotated translation correspondences that can be used for contrastive analysis of 
various linguistic items. 
{\looseness=1

}

The core SCDB concepts have been described in previous 
works (see, for example,~[1--3]). The goal of this paper is to examine other existing approaches 
to annotation of connectives and discourse relations in corpora and to compare them 
to the approaches used in the SCDB-Connectives. 

In section~2, existing approaches to 
annotation of discourse relations are examined. The most prominent of these is the 
one adopted by the PDTB, a~monolingual collection of 
annotated discourse relations in a~corpus of journal articles. 
The PDTB project proposed 
a~model for annotation of discourse relations that has spawned a~number of similar 
efforts. The proposed standard ISO 24617-8 that was developed in recent years in 
order to create a~common foundation for annotation of discourse relations is also 
examined.

 In section~3, approaches of PDTB and ISO 24617-8 are compared to the 
methods adopted in the SCDB-Connectives.
  
  \section{Existing Approaches to~Annotation of~Discourse~Relations}
  
  \noindent
  In recent years, there has been a~considerable effort towards computer-aided 
annotation of complex linguistic items that at the present cannot be reliably annotated 
by automated annotation procedures (e.\,g., discourse markers, connectives, relations 
between textual segments, etc.). Morphological annotation of individual words in 
corpora is usually implemented automatically (see, for example,~[4, p.~34--37]). On the other 
hand, annotation of complex intertextual entities is often carried out manually. This is 
due to the fact that no accepted annotation standards exist for many of such entities 
and, more importantly, that currently such annotation cannot be implemented 
algorithmically since analysis of such entities often proves to be daunting even for 
seasoned linguistic researchers.
  
  Penn Discourse Treebank is one of the most prominent projects for 
annotation of intertextual entities. It is a~large-scale collection of annotated discourse 
relations, both explicit and implicit, found in a~million-word corpus of Wall Street 
Journal articles. Penn Discourse Treebank was first released 
to the public in 2006 (version PDTB~1.0) 
and in 2008, a~new version of the project was released~---  
PDTB~2.0~\cite{5-kr, 6-kr}. Being the first large-scale project dealing with 
annotation of discourse relations in texts, during the following years, PDTB served as 
a~reference model for a~number of future similar projects.
  
  In PDTB, annotation of discourse relations includes three aspects:
  \begin{enumerate}[(1)]
  \item structure: relation signals (connectives) and arguments are annotated as text 
spans in the corpus (it is assumed that all discourse relations have exactly two 
arguments);
  \item semantics: discourse relations are assigned semantic labels; and
  \item attribution: discourse properties of relations and their arguments are 
annotated, including such properties as sources of relation and their arguments 
(writer/another agent), types of the arguments (assertion/belief/event/fact, etc.), 
polarity (positive/negative), determinacy. In majority of later projects that adopted 
PDTB model, only the first two aspects (structure and semantics of discourse 
relations) have been annotated: apparently, attribution of discourse relations and their 
arguments turned out to be a~nontrivial and time-consuming task.
  \end{enumerate}
  
  Both explicit and implicit discourse relations are annotated in PDTB. Explicit 
relations are those realized by explicit connectives, such as subordinating 
conjunctions (\textit{because}, \textit{when}), coordinating conjunctions 
(\textit{and}, \textit{or}) or adverbials (\textit{for example}, \textit{instead}). 
Complex connectives are also annotated, including modified forms of connectives 
(\textit{only because}), conjoined connectives (\textit{if and when}) and parallel 
connectives (\textit{either\ldots or}, \textit{on one hand\ldots\ on the other hand}).
  
  Implicit relations are annotated between each pair of adjacent sentences within 
paragraphs where there is no connective present. The PDTB methodology introduced the 
so-called lexically-grounded approach to annotation of implicit relations: annotators 
have to label them with specific connectives that best signal these relations when 
inserted between the adjacent sentences  (see, e.\,g.,~Example~1). 
  
  Example~1. \textit{In July, the Environmental Protection Agency imposed a~gradual ban 
on virtually all uses of asbestos}. (implicit\;=\;\underline{as a~result}). \textit{By~1997, 
almost all remaining uses of cancer-causing asbestos will be outlawed}.
  
  When an annotator cannot insert a~connective between the adjacent sentences, one 
of the three relation labels is used: AltLex, EntRel, or NoRel. AltLex (Alternative 
Lexicalization) signifies that the discourse relation between the adjacent sentences is 
already signaled in the text by an alternative lexical device (not a~connective); EntRel 
signifies that only an entity-based relation can be inferred between the sentences; 
and NoRel is used when annotators cannot see neither an Alternative Lexicalization, nor 
an Entity-relation between the adjacent sentences.
  
The  PDTB concept implies that discourse relations can hold between two and only two 
arguments, each argument being a~span of text that can be interpreted as 
a~proposition, eventuality, belief, etc.\ (what Asher calls \textit{abstract objects} 
in~\cite{7-kr}). In English, such abstract objects are usually conveyed by one or 
several sentences or clauses and sometimes by nominalizations and verb phrases. 
When discourse relations are annotated in PDTB, these arguments are labeled as 
Arg1 and Arg2. For explicit relations, Arg2 is the argument that is syntactically 
bound to the explicit connective; for implicit relations, Arg1 is the argument that 
comes first and Arg2 is the argument that comes second. Arguments Arg1 and Arg2 are 
limited to the minimal text needed for interpretation of the discourse relation (this is 
known as ``the minimality principle''). When a~larger context is required for better 
understanding of a~relation, supplementary materials for the arguments can also be 
annotated (labelled as Sup1 for Arg1 and Sup2 for Arg2).
  
  For semantic annotation of discourse relations, a~three-level typology of semantic 
labels (sense tags) is adopted in PDTB (Fig.~1). The discourse relations are 
divided into four classes (temporal, contingency, comparison, and expansion). 
Inside each 
class, the relations are further partitioned into several types and most types have 
several subtypes. Each discourse relation can be\linebreak\vspace*{-12pt}

\pagebreak

\end{multicols}
  \begin{figure*} %fig1
  \vspace*{1pt}
 \begin{center}
 \mbox{%
 \epsfxsize=115.088mm 
 \epsfbox{kru-1.eps}
 }
 \end{center}
\vspace*{-13pt}
  \Caption{The PDTB 2.0 sense hierarchy~\cite{5-kr}}
  \vspace*{-4pt}
   \end{figure*}

\begin{multicols}{2}

\noindent
 assigned more than one sense tag 
from this hierarchy (see~\cite{8-kr}). When there is a~disagreement between 
annotators on a~lower level, it can be automatically resolved by rolling back to the 
next higher level (e.\,g., Juxtaposition vs.\ Opposition disagreement will be 
automatically resolved by rolling back to the Contrast semantic tag).
  

  
  The developers of PDTB acknowledge that this typology of relation senses has 
some gaps~[9, p.~931--932]. In a~number of later efforts based on PDTB 
methodology, the typology of senses has been modified and extended 
 (see, e.\,g.,~\cite{10-kr, 11-kr, 12-kr}). Variations in semantic typologies result both 
from diverse theoretical assumptions and from distinctions in the scope of research. 
For example, while PDTB covers discourse relations in the corpus of Wall Street 
Journal articles (in English only), the effort reported in~\cite{12-kr} addresses 
various types of English and French discourse markers (not only those that signal 
relation between two text spans) in written and spoken texts.
{\looseness=-1

}
  
  A number of efforts have been implemented following the model established by 
PDTB (many of them listed in~[9,  p.~934]. As a~part of these efforts, 
annotated corpora of connectives and discourse relations have been created for texts 
in different genres and languages (including Arabic, Chinese, Turkish, Hindi, Czech, 
and French). In most of these corpora, PDTB's senses typology was adjusted but 
otherwise they adopted the same general approach to annotation of discourse relation 
as proposed by the PDTB framework. In particular, PDTB's lexically-grounded approach 
was implemented for annotation of implicit relations (annotators must insert 
connectives that best signal relations between adjacent sentences). 
  
  In recent years, a~standard for annotation of discourse relations is being developed 
(ISO 24617-8,~\cite{13-kr}). The authors of the standard aim to bring together the 
best practices used in corpora with annotated discourse relations created up to date. 
They propose a~structure that should support annotation of discourse relation in 
a~uniform theory-neutral way. A~set of~20~core relations is proposed for semantic 
labeling of the most common discourse-level relations, although this set does not aim 
to be exhaustive and allows for future extensions. As opposed to PDTB senses 
typology, this list is not organized as a~hierarchy; the core relations are initially 
independent but researches are free to group them into functional domains according 
to their vision.
  
  For asymmetrical relations, the argument roles are specified in ISO 24617-8. Each 
asymmetrical relation in the core set is supplied with appropriate argument role 
labels. For example, for discourse relation \textit{Cause} argument roles 
\textit{Reason} and \textit{Result} are specified, for discourse relation 
\textit{Purpose}~--- argument roles \textit{Goal} and \textit{Enablement}. This 
allows for unification of otherwise semantically identical relations with reversed 
order of arguments (for example, in PDTB~2.0, there are such pairs of relation sense 
tags as \textit{Reason} and \textit{Result}, \textit{Precedence} and 
\textit{Succession}).
  
  In order to distinguish between ideational and rhetorical variants of the same 
relation, ISO 24617-8 proposes to specify types of arguments, which can be either 
situations (eventualities, facts, conditions, etc.)\ or dialog acts. This also allows for 
removal of doubling (rhetorical and nonrhetorical) relation sense tags present in 
PDTB~2.0 (see Fig.~1). ISO 24617-8 also allows to specify the source of relation and 
its arguments (which can be either the producer of the text or another actor).

  
  The overall metamodel for annotation of discourse relations proposed in ISO 
24617-8 is presented in Fig.~2. Indication~$0\ldots1$ at the tip of the arrow from 
discourse relation to markables signifies that the discourse relation
 can be either 
explicit (with an explicit connective in the text) or implicit (no markable in the text). 
One thing that is not quite clear with ISO 24617-8 is how it proposes to deal with 
such cases when a~discourse relation can combine properties of more than one 
relation.
  
  
   { \begin{center}  %fig2
 \vspace*{16pt}
 \mbox{%
 \epsfxsize=75.108mm 
 \epsfbox{kru-2.eps}
 }


\end{center}


\noindent
{{\figurename~2}\ \ \small{ISO 24617-8 metamodel for annotation of discourse relations}}
}

%\vspace*{6pt}

\addtocounter{figure}{1}
 
  
  A few works were dedicated to annotation of connectives and discourse markers in 
parallel and comparable corpora~\cite{11-kr, 12-kr}, although currently, there are no 
large-scale parallel corpora with annotated discourse relations. For example, 
in~\cite[p.~4--5]{11-kr},  the authors maintain that for initial assessment of  
cross-language equivalents, it is mandatory to use parallel corpora, but such 
assessment is problematic in practice because parallel corpora are not entirely reliable 
and often limited to specific genres. That is why the authors propose to use small 
parallel corpora at the initial stage of a~cross-linguistic research and to verify their 
findings at later stages using larger-scale comparable multilingual corpora.

\vspace*{-7pt}
  
  \section{Supracorpora Database of~Connectives in~Contrast with~Other Approaches}
  
  \vspace*{-2pt}
  
  \noindent
  In supracorpora databases (and in the SCDB-Connectives, in particular), the 
minimal context required for interpretation of the linguistic item (connective) is 
included in the annotation. In addition, within this minimal context, the main words 
(those that are part of the connective) and the so-called functional words (those that 
may influence semantics of the connective or/and represent pragmatics or situational 
context) are annotated. However, the actual arguments of the discourse relations 
(Arg1 and Arg2 in the PDTB framework) are not explicitly annotated in the  
SCDB-Connectives. On one hand, this is a~drawback, but on the other hand, there is 
no restriction on the number of arguments in the system, which allows researchers to 
annotate connectives that join together more than two arguments. This cannot be handled by PDTB and ISO  
24617-8, even though one must acknowledge that such multipart connectives are not 
very frequent: only~41~tokens in our corpus, including such types as \textit{не 
только \ldots но даже \ldots а иногда и \ldots}; \textit{не \ldots и не \ldots 
а~просто \ldots}; \textit{хотя \ldots хотя \ldots однако все же \ldots}; \textit{не 
только \ldots но даже и \ldots и даже \ldots}\footnote{Literal English translation: 
\textit{not only \ldots but even \ldots and sometimes also \ldots}; \textit{not \ldots and not \ldots but just 
\ldots}; \textit{though \ldots though \ldots but \mbox{still \ldots}}; \textit{not only \ldots but even also \ldots and 
even \ldots}} (see Example~2).
  
 
  Example~2.~$\langle\ldots\rangle$~\textit{ужасно стыдно мне стало, когда я наконец 
догадался (вдруг как-то), что} {\bfseries\textit{не только}} \textit{его не 
покоробило бы,} {\bfseries\textit{но даже и}} \textit{в~голову бы ему не пришло, 
что это не монументально}\ldots {\bfseries\textit{и~даже}} \textit{не понял бы он 
совсем: чего тут коробиться?}\footnote[2]{Literal English translation: \textit{I~felt terribly 
ashamed when I~finally realized (quite suddenly) that not only he would not have been shocked}, 
{\bfseries\textit{but even also}} \textit{it would not have crossed his mind that this was not 
monumental}\ldots {\bfseries\textit{and even}} \textit{he would not have understood what was there to be 
shocked about?} (\Aue{F.\,M.~Dostoyevsky}. {Crime and punishment})} 
({Ф.\,М.~Достоевский}. Преступление и наказание)
  
  As opposed to PDTB and similar to ISO 24617-8, the typology of sense labels in 
the SCDB-Connectives is not organized as a~hierarchy. The relations are initially 
independent although they may be grouped into functional domains during a~later 
stage (see, e.\,g.,~\cite{14-kr}).
{\looseness=1

}
  
  The most important distinction of the SCDB-Connectives that sets it apart from 
similar efforts is that it supports annotation of connectives in large-scale parallel 
corpora. The SCDB-Connectives operates with parallel corpora of the  
Russian National Corpus that are both comparatively large and of high quality since they only 
include literary translations made by professional translators. These corpora mostly 
include fiction, but they are gradually expanding to include other genres (official 
documents, philosophical literature, etc.). Currently, most connectives are annotated in 
Russian--French ($> 3$~million words) and French--Russian ($> 600$~thousand 
words) corpora but further annotation efforts are planned for Russian--Italian and 
Russian--German corpora. The SCDB-Connectives takes advantage of parallel corpora 
structure allowing users to create annotated TCs. Each 
TC includes annotation of a~Russian connective and annotation of the corresponding 
French fragment that often (but not always)  also includes a~connective. 
  
  Simultaneous parallel annotation of discourse relations in Russian and in other 
languages makes it possible to conduct comparative analysis of discourse relations in 
the corresponding language pairs. This kind of analysis is unique because it allows 
researchers to test validity and cross-lingual universality of the proposed 
classifications of logical-semantic relations based on data from parallel texts.
  
  As of today, over 16,5 thousand TCs of various types have been created in the 
SCDB-Connectives for more than~11,5~thousand of connectives (the reason for the 
discrepancy is that for some texts, multiple translations are available in the corpus). 
There are~853~types of Russian connectives registered in the database, which are 
organized in~136~clusters. This is significantly more types than in PDTB~2.0 where 
only~100~distinct types of explicit connectives are registered (155~types if modified 
forms and variants are taken into account). These figures emphasize a~much higher 
variability and structural complexity of Russian connectives as opposed to the English 
ones. For example, in PDTB~2.0, there are~3000 tokens of `\textit{and}' 
and~1746~tokens of `\textit{also},' but not a~single modified or conjoined form of 
either of them. Meanwhile, in Russian, there is a~great variety of widely used 
conjoined/modified forms for both `\textit{и}' (\textit{and}) and `\textit{также}' 
(\textit{also}), including such types as \textit{и также}, \textit{а~также}, 
\textit{а~также~и}, \textit{и~к~тому же}\footnote{Literal English translation: \textit{and 
also}; \textit{but also}; \textit{but also and}; 
\textit{and in addition}.}, etc.
  
  The SCDB-Connectives is based on the SCDB concept that has been described in 
more detail in previous works (see, e.\,g.,~[1--3]). As a~part of other projects, several 
SCDB banks of translation correspondences were created for cross-linguistic study of 
such items as personal verbal forms, impersonal verbal forms, language-specific 
words, and discourse markers.
  
  Penn Discourse Treebank, ISO 24617-8, and most of the other annotation efforts are based on XML 
data format. The SCDB-Connectives, like other SCDBs, was developed as 
a~relational database, which enables its data to be well organized. Relational 
databases support structured query language (SQL) offering researchers powerful features when it 
comes to searching and generating statistics. 
  
  A server-based relational database increases the accessibility of the  
SCDB-Connectives to the users. Annotators and linguistic experts can simultaneously 
work with the same data from anywhere using their favorite web-browsers. In 
addition, the structure and interface of the database allow researchers to modify the 
annotation schemes easily by inserting new properties into appropriate tables or by 
altering descriptions of the existing ones. 
{\looseness=-1

}
  
  Eventually, SCDBs will integrate new features, such as a~possibility to save history 
of changes made to annotation schemes in SCDBs, and the underlying relational 
database will significantly facilitate such tasks. By tracking changes to annotation 
schemes, SCDBs will be able to contribute to development of knowledge generation 
theory~\cite{15-kr, 16-kr, 17-kr}). 
  
  Finally, in the context of contrastive linguistic projects related to studying of 
connectives, one should also mention the GECCo (German--English Contrasts in Cohesion)
project (see, e.\,g.,~\cite{18-kr}). 
The project's goal was to produce a~German--English corpus for contrastive linguistic 
work in the area of textual cohesion. The GECCo corpus includes texts in more than 
a~dozen registers, both spoken and written, English and German, allowing to study 
relevant distribution of cohesive devices in English and German. The main difference 
between the GECCo project and SCDB-Connectives is that the GECCo covers 
a~much wider range of types of cohesion, including reference, substitution, ellipsis, 
conjunction (connectives fall into this category), and lexical cohesion, while  
SCDB-Connectives deal only with connectives. As a~result, the GECCo corpus does 
not pay enough attention to annotation of discourse relations, mostly concentrating on 
other types of textual cohesion. Only~5~semantic types of conjunctions are annotated 
in the GECCo: additive, adversative, temporal, causal, and modal. In addition, only 
a~fraction of the texts in the GECCo corpus are parallel texts: the GECCo corpus is 
created and used mostly as a~comparable corpus rather than a~parallel one. 
Accordingly, while the GECCo corpus allows studying of relative frequencies of 
cohesion devices across various registers, it does not support unidirectional 
methodology of SCDB, which is based on analysis of translation correspondences.

\vspace*{-6pt}
  
  \section{Concluding Remarks}
  
  \vspace*{-2pt}
  
  \noindent
  Interpretation of explicit connectives is vital for high-level understanding of text as 
a~whole because they signal discourse relations between larger text segments 
effectively ``binding'' the text together. For this reason, semantic annotation of 
connectives plays a~major role in such areas as natural language processing (NLP) 
and machine translation. A~number of works~\cite{19-kr, 20-kr, 21-kr} brought 
evidence that integrating annotated sense labels for discourse connectives can 
improve results of factored statistical machine translation (SMT). As a~part of the 
above-mentioned work, the authors tested algorithms for automatic disambiguation of 
discourse connectives prior to SMT, which resulted in significantly improved scores 
for quality of machine translation. The authors stressed that disambiguation of 
connectives is a~more challenging task than regular word sense disambiguation 
(WSD) because connectives labeling requires more structured and longer-range 
information. Besides, modeling of content word senses differs considerably from 
modeling of the procedural meaning of function words. The authors note that to 
improve quality of automatic disambiguation of connectives, it is vital to have access 
to reliable large-scale annotated data for training and testing purposes. 
  
  While the primary goal of this paper was to examine and compare existing 
approaches to annotation of connectives and discourse relations, it also aimed to 
bring evidence that the SCDB-Connectives provides valuable data on Russian 
connectives and their correspondences in French (and, eventually, in other languages) 
that can be useful for both theoretical and practical purposes. As far as we can say, it 
is the largest-scale corpus of annotated connectives in parallel texts. Nevertheless, 
some aspects of the SCDB-Connectives require improvement to make it more useful 
for researchers. More detailed descriptions of sense labels should be provided for the 
SCDB-Connectives, including descriptions of discourse relations and their 
arguments. It will be also useful to provide mapping between relation senses used in 
the SCDB-Connectives and the core set of relations in ISO 24617-8. Finally, 
following the guidelines of PDTB, it makes sense to explicitly annotate the 
arguments of discourse relations signaled by the connectives and to specify their roles 
and types (similar to the ISO 24617-8 proposal).

\vspace*{-6pt}
  
  \Ack
  
  \vspace*{-2pt}
  
  \noindent
  The work was carried out at the Institute of Informatics Problems (FRC CSC RAS) and funded 
by the Russian Science Foundation according to the research project No.\,16-18-10004.
  
  \renewcommand{\bibname}{\protect\rmfamily References}
  
  \vspace*{-6pt}


{\small\frenchspacing
{%\baselineskip=10.8pt
\begin{thebibliography}{99}

\vspace*{-2pt}

  \bibitem{1-kr}
  \Aue{Loiseau, S., D.\,V.~Sitchinava,  Anna~A.~Zalizniak, and I.\,M.~Zatsman}. 
2013. Information technologies for creating the database of equivalent verbal forms 
in the Russian--French multivariant parallel corpus. \textit{Informatika i~ee 
Primeneniya~--- Inform. Appl.} 7(2):100--109.
  \bibitem{2-kr}
  \Aue{Kruzhkov, M., N.~Buntman, E.~Loshchilova, D.~Sitchinava, Anna 
A.~Zalizniak, and I.\,A.~Zatsman} 2014. Database of Russian verbal forms and 
their French translation equivalents. \textit{Computational Linguistics and 
Intellectual Technologies: Conference (International) ``Dialogue 2016'' 
Proceedings}. Moscow: RGGU. 13(20):275--287.
  \bibitem{3-kr}
  \Aue{Kruzhkov, M.} 2016. Supracorpora databases as corpus-based 
superstructure for manual annotation of parallel corpora. \textit{8th 
Conference (International) on  Corpus Linguistics}. EPiC ser.\ in language and 
linguistics. 1:236--248. Available at: {\sf 
https://easychair.org/\linebreak publications/paper/270289} (accessed August~31, 2017).
  \bibitem{4-kr}
  \Aue{Mikhailov, M., and R.~Cooper}. 2016. \textit{Corpus linguistics for 
translation and contrastive studies: A~guide for research.}  
 London\,--\,New York: Routledge. 234~p.
  \bibitem{5-kr}
  \Aue{Prasad, R., N.~Dinesh, A.~Lee, E.~Miltsakaki, L.~Robaldo, A.~Joshi, and 
B.~Webber}. 2008. The Penn Discourse TreeBank~2.0. \textit{6th Conference 
(International) on Language Resources and Evaluation Proceedings}. 
Marrackech, Morocco. 2961--2968.
  \bibitem{6-kr}
  \Aue{Prasad, R., B.~Webber, and A.~Joshi}. 2017. The Penn Discourse Treebank: 
An annotated corpus of discourse relations. \textit{Handbook of linguistic 
annotation}. Springer. 1197--1217.
  \bibitem{7-kr}
  \Aue{Asher, N.} 1993. \textit{Reference to abstract objects.} Dordrecht--Boston: 
Kluwer Academic. 455~p.
  \bibitem{8-kr}
  \Aue{Webber, B.} 2016. Concurrent discourse relations. \textit{Computational 
Linguistics and Intellectual Technologies: Conference (International) ``Dialogue 
2016'' Proceedings}. Moscow. 15(22):D.  Available at: {\sf 
http://www.dialog-21.ru/media/3488/webber.pdf} (accessed August~31, 2017).
  \bibitem{9-kr}
  \Aue{Prasad, R., B.~Webber, and A.~Joshi}. 2014. Reflections on the Penn 
Discourse TreeBank, comparable corpora and complementary annotation. 
\textit{Comput. Linguist.} 40(4):921--950.
  \bibitem{10-kr}
  \Aue{Prasad, R., S.~McRoy, N.~Frid, A.~Joshi,  and H.~Yu}. 2011. The 
biomedical discourse relation bank. \textit{BMC Bioinformatics} 12:188--205.
  \bibitem{11-kr}
  \Aue{Zufferey, S., and L.~Degand}. 2013. Annotating the meaning of discourse 
connectives in multilingual corpora. \textit{Corpus Linguist. Ling.} 13(2):1--24. 
  \bibitem{12-kr}
  \Aue{Cribble, L., and S.~Zufferey}. 2015. Using a~unified taxonomy to annotate 
discourse markers in speech and writing. \textit{11th Conference (International) on 
Computational Semantics Proceedings}. London. 14--22.
  \bibitem{13-kr}
  \Aue{Bunt, H., and R.~Prasad}. 2016. ISO DR-Core (ISO 24617-8): Core 
concepts for the annotation of discourse relations. \textit{12th Joint ACL-ISO 
Workshop on Interoperable Semantic Annotation (ISA-12 Proceedings}). Portoroz. 
45--54.
  \bibitem{14-kr}
  \Aue{Zatsman, I., O.~Inkova, and V.~Nuriev}. 2017. The construction of 
classification schemes: Methods and technologies of expert formation. 
\textit{Automatic Documentation Math. Linguistics} 51(1):27--41.
  \bibitem{15-kr}
  \Aue{Zatsman, I.} 2012. Tracing emerging meanings by computer: Semiotic 
framework. \textit{13th European Conference on Knowledge Management 
Proceedings}. Reading: Academic Publishing International Ltd. 2:1298--1307.
  \bibitem{16-kr}
  \Aue{Zatsman, I., N.~Buntman, M.~Kruzhkov, V.~Nuriev, and Anna 
A.~Zalizniak}. 2014. Conceptual framework for development of computer 
technology supporting cross-linguistic knowledge discovery. \textit{15th European 
Conference on Knowledge Management Proceedings}. Reading: Academic 
Publishing International Ltd. 3:1063--1071.
  \bibitem{17-kr}
  \Aue{Zatsman, I., and N.~Buntman}. 2015. Outlining goals for discovering new 
knowledge and computerised tracing of emerging meanings. \textit{16th European 
Conference on Knowledge Management Proceedings}. Reading: Academic 
Publishing International Ltd. 851--860.
  \bibitem{18-kr}
  \Aue{Lapshinova-Koltunski,~E., and K.~Kunz}. 2014. Annotating cohesion for 
multilingual analysis. \textit{10th Joint ACL-ISO Workshop on Interoperable 
Semantic Annotation Proceedings}. Reykjavik.  57--64.
  \bibitem{19-kr}
  \Aue{Meyer, T., A.~Popescu-Belis, N.~Hajlaoui, and A.~Gesmundo}. 2012. 
Machine translation of labeled discourse connectives. \textit{10th Conference of the 
Association for Machine Translation in the Americas Proceedings}. San 
Diego, CA. Available at:  
{\sf http://publications.idiap. ch/index.php/publications/show/2391} (accessed August~31, 
2017).
  \bibitem{20-kr}
  \Aue{Meyer, T.} 2014. Discourse-level features for statistical machine 
translation. PhD thesis. $\acute{\mbox{E}}$cole Polytechnique 
F$\acute{\mbox{e}}$d$\acute{\mbox{e}}$rale de Lausanne. Available at: {\sf 
http://publications.\linebreak idiap.ch/downloads/papers/2015/Meyer\_THESIS\_2014. pdf} (accessed 
August~31, 2017).
  \bibitem{21-kr}
  \Aue{Meyer, T., N.~Hajlaoui, and A.~Popescu-Belis}. 2015. Disambiguating 
discourse connectives for statistical machine translation. \textit{IEEE-ACM 
T.~Audio Spe.} 23(7):1184--1197.
  \end{thebibliography} }
 }

\end{multicols}

\vspace*{-6pt}

\hfill{\small\textit{Received September~7, 2017}}

\vspace*{-14pt}
  
  \Contrl
  
  \noindent
\textbf{Kruzhkov Mikhail G.} (b.\ 1975)~--- senior scientist, Institute of 
Informatics Problems, Federal Research Center ``Computer Science and Control'' 
of the Russian Academy of Sciences, 44-2~Vavilov Str., Moscow 119333, Russian 
Federation; \mbox{magnit75@yandex.ru}

\vspace*{7pt}

\hrule

\vspace*{2pt}

\hrule

%\newpage

\vspace*{-4pt}



\def\tit{ПОДХОДЫ К АННОТАЦИИ ДИСКУРСИВНЫХ ОТНОШЕНИЙ В~ЛИНГВИСТИЧЕСКИХ КОРПУСАХ$^*$}

\def\aut{М.\,Г.~Кружков}


\def\titkol{Подходы к аннотации дискурсивных отношений в~лингвистических корпусах}

\def\autkol{М.\,Г.~Кружков}

{\renewcommand{\thefootnote}{\fnsymbol{footnote}}
\footnotetext[1]{Работа выполнена в Институте проб\-лем информатики
Федерального исследовательского центра
<<Информатика и~управ\-ле\-ние>> Российской академии наук
при финансовой поддержке РНФ (проект  №\,16-18-10004).}}


\titel{\tit}{\aut}{\autkol}{\titkol}

\vspace*{-10pt}

\noindent
   Институт проблем информатики Федерального исследовательского центра <<Информатика 
и~управление>> Российской академии наук, \mbox{magnit75@yandex.ru}

\vspace*{1pt}

\def\leftfootline{\small{\textbf{\thepage}
\hfill ИНФОРМАТИКА И ЕЁ ПРИМЕНЕНИЯ\ \ \ том\ 11\ \ \ выпуск\ 4\ \ \ 2017}
}%
 \def\rightfootline{\small{ИНФОРМАТИКА И ЕЁ ПРИМЕНЕНИЯ\ \ \ том\ 11\ \ \ выпуск\ 4\ \ \ 2017
\hfill \textbf{\thepage}}}
  



  \Abst{Рассматривается надкорпусная база данных (НБД), разработанная на 
основе корпуса параллельных текстов для описания русских коннекторов и~их 
переводов на французский и~другие языки. В~рамках данной НБД аннотируется 
внутренняя структура и~семантика коннекторов русского языка, а~также их 
переводных соответствий на других языках. Описание семантики коннекторов 
подразумевает описание соответствующих дискурсивных отношений между 
соединяемыми ими фрагментами текста. Используемый в~НБД подход 
к~описанию дискурсивных выражений, передаваемых коннекторами, 
сравнивается с~новейшими существующими подходами к аннотации 
дискурсивных отношений: рас\-смат\-ри\-ва\-ет\-ся аннотированный корпус 
дискурсивных отношений Penn Discourse Treebank (PDTB) и проект стандарта 
по аннотации дискурсивных отношений ISO  
24617-8. Отмечается, что PDTB и~ISO 24617-8, в~отличие от НБД, позволяют 
аннотировать как эксплицитные (выраженные коннекторами и~другими
языковыми единицами), так и~имплицитные дискурсивные отношения. Кроме 
этого, в~рамках данных подходов имеется возможность аннотировать аргументы 
дискурсивных отношений, включая их источники, типы и~роли (для 
ассиметричных отношений). С~другой стороны, преимущество НБД со\-сто\-ит 
в~том, что она позволяет одновременно аннотировать коннекторы и~их 
переводные соответствия в~параллельных корпусах, что открывает для 
исследователей новые возможности в~об\-ласти лингвистического
контрастивного анализа. В~то время как в~рамках других подходов для 
аннотации дискурсивных кон-\linebreak\vspace*{-12pt}}

\Abstend{некторов используется формат XML, НБД 
представляет собой реляционную базу данных, что повышает эффективность 
системы при работе с~кросслингвистическими объектами и~доступность для 
пользователей. Также рассматривается теоретическая и~практическая 
значимость семантической аннотации коннекторов и~выражаемых ими 
дискурсивных отношений.}
  
  \KW{дискурсивные отношения; коннекторы; корпусная лингвистика; 
параллельные корпуса; надкорпусные базы данных}

  
\DOI{10.14357/19922264170415}

%\vspace*{18pt}


 \begin{multicols}{2}

\renewcommand{\bibname}{\protect\rmfamily Литература}
%\renewcommand{\bibname}{\large\protect\rm References}

{\small\frenchspacing
{%\baselineskip=10.8pt
\begin{thebibliography}{99}
  \bibitem{1-kr-1}
  \Au{Loiseau S., Sitchinava~D.\,V., Zalizniak Anna~A., Zatsman~I.\,M.} 
Information technologies for creating the data base of equivalent verbal forms in the 
Russian--French multivariant parallel corpus~// Информатика и её применения, 
2013. Т.~7. Вып.~2. С.~100--109.
  \bibitem{2-kr-1}
  \Au{Kruzhkov M., Buntman~N., Loshchilova~E., Sitchinava~D., Zalisniak 
Anna~A., Zatsman~I.\,A.} Database of Russian verbal forms and their French 
translation equivalents~// Computational Linguistics and Intellectual Technologies:  
Conference (International) ``Dialogue 2016'' Proceedings.~--- Moscow: RGGU, 
2014. Vol.~13(20). P.~275--287.
  \bibitem{3-kr-1}
  \Au{Kruzhkov M.} Supracorpora databases as corpus-based superstructure for 
manual annotation of parallel corpora~// 8th Conference (International) 
on Corpus Linguistics.~--- EPiC ser. in language and linguistics, 2016. Vol.~1. 
P.~236--248. {\sf https://easychair.org/publications/paper/270289}.
  \bibitem{4-kr-1}
  \Au{Mikhailov M., Cooper~R.} Corpus linguistics for translation and contrastive 
studies: A~guide for research.~--- London\,--\,New York: Routledge, 2016. 234~p.
  \bibitem{5-kr-1}
  \Au{Prasad R., Dinesh~N., Lee~A., Miltsakaki~E., Robaldo~L., Joshi~A., 
Webber~B.} The Penn Discourse TreeBank~2.0~//  6th Conference (International) on 
Language Resources and Evaluation (LREC 2008) Proceedings.~--- Marrackech, 
Morocco, 2008.  P.~2961--2968.
  \bibitem{6-kr-1}
  \Au{Prasad R., Webber~B., Joshi~A.} The Penn Discourse Treebank: An 
annotated corpus of discourse relations~// Handbook of linguistic annotation.~--- 
Springer, 2017. P.~1197--1217.
  \bibitem{7-kr-1}
  \Au{Asher N.} Reference to abstract objects.~--- Dordrecht--Boston: Kluwer 
Academic, 1993. 455~p.
  \bibitem{8-kr-1}
  \Au{Webber B.} Concurrent discourse relations, computational linguistics and 
intellectual technologies~// Com\-pu\-tational 
Linguistics and Intellectual Technologies:\linebreak Conference (International) ``Dialogue 2016'' 
Pro\-ceed\-ings.~--- Moscow: RGGU, 2016. Vol.~15(22). {\sf 
http://www. dialog-21.ru/media/3488/webber.pdf}.
  \bibitem{9-kr-1}
  \Au{Prasad R., Webber~B., Joshi~A.} Reflections on the Penn Discourse 
TreeBank, comparable corpora and Ccmplementary annotation~// Comput. 
Linguist., 2014. Vol.~40. No.\,4. P.~921--950.
  \bibitem{10-kr-1}
  \Au{Prasad R., McRoy~S., Frid~N., Joshi~A., Yu~H.} The biomedical discourse 
relation bank~// BMC Bioinformatics, 2011. Vol.~12. P.~188--205.
  \bibitem{11-kr-1}
  \Au{Zufferey S., Degand~L.} Annotating the meaning of discourse connectives in 
multilingual corpora~// Corpus Linguist. Ling., 2013. Vol.~13. Iss.~2. P.~1--24. 
  \bibitem{12-kr-1}
  \Au{Cribble L., Zufferey~S.} Using a~unified taxonomy to annotate discourse 
markers in speech and writing~// 11th Conference (International) on Computational 
Semantics Proceedings.~--- London, 2015. P.~14--22.
  \bibitem{13-kr-1}
  \Au{Bunt H., Prasad~R.}ISO DR-Core (ISO 24617-8): Core concepts for the 
annotation of discourse relations~// 12th Joint ACL-ISO Workshop on Interoperable 
Semantic Annotation Proceedings.~--- Portoroz, 2016. P.~45--54.
  \bibitem{14-kr-1}
  \Au{Zatsman I., Inkova~O., Nuriev~V.} The construction of classification 
schemes: Methods and technologies of expert formation~// Automatic 
Documentation Math. Linguistics, 2017. Vol.~51. No.\,1. P.~27--41.
  \bibitem{150kr-1}
  \Au{Zatsman I.} Tracing emerging meanings by computer: Semiotic framework~// 
13th European Conference on Knowledge Management Proceedings.~--- Reading: 
Academic Publishing International Ltd., 2012. Vol.~2. P.~1298--1307.
  \bibitem{16-kr-1}
  \Au{Zatsman~I., Buntman~N., Kruzhkov~M., Nuriev~V., Zalizniak Anna~A.} 
Conceptual framework for development of computer technology supporting  
cross-linguistic knowledge discovery~// 15th European Conference on Knowledge 
Management Proceedings.~--- Reading: Academic Publishing International Ltd., 
2014. Vol.~3. P.~1063--1071.
  \bibitem{17-kr-1}
  \Au{Zatsman I., Buntman~N.} Outlining goals for discovering new knowledge and 
computerised tracing of emerging meanings~// 16th European Conference on 
Knowledge Management Proceedings.~--- Reading: Academic Publishing 
International Ltd., 2015. P.~851--860.
  \bibitem{18-kr-1}
  \Au{Lapshinova-Koltunski E., Kunz~K.} Annotating cohesion for multilingual 
analysis~// 10th Joint ACL-ISO Workshop on Interoperable Semantic Annotation 
Proceedings.~--- Reykjavik, 2014. P.~57--64.
  \bibitem{19-kr-1}
  \Au{Meyer T., Popescu-Belis~A., Hajlaoui~N., Gesmundo~A.} Machine 
translation of labeled discourse connectives~// 10th Conference of the Association for 
Machine Translation in the Americas Proceedings.~--- San Diego, CA, 
USA, 2012. {\sf http://publications.idiap.ch/index.\linebreak php/publications/show/2391}.
  \bibitem{20-kr-1}
  \Au{Meyer T.} Discourse-level features for statistical machine translation: PhD 
thesis.  $\acute{\mbox{E}}$cole Polytechnique 
F$\acute{\mbox{e}}$d$\acute{\mbox{e}}$rale de Lausanne, 2014. {\sf 
http://publications.idiap.ch/\linebreak  downloads/papers/2015/Meyer\_THESIS\_2014.pdf}.
  \bibitem{21-kr-1}
  \Au{Meyer T., Hajlaoui~N., Popescu-Belis~A.} Disambiguating discourse 
connectives for statistical machine translation~// IEEE-ACM T.~Audio 
Spe., 2015. Vol.~23. No.\,7. P.~1184--1197.
\end{thebibliography}
} }


\end{multicols}

 \label{end\stat}

 \vspace*{-12pt}

\hfill{\small\textit{Поступила в~редакцию  07.09.2017}}

\pagebreak
%\renewcommand{\bibname}{\protect\rm Литература}
\renewcommand{\figurename}{\protect\bf Рис.}
\renewcommand{\tablename}{\protect\bf Таблица}  %15


%%%%%%%%%%%%%%%%%%%%%%%%%%%%%%%%%%%%%%%%%%%%%%%

%\def\stat{rez}
{%\hrule\par
%\vskip 7pt % 7pt
\raggedleft\Large \bf%\baselineskip=3.2ex
Р\,Е\,Ц\,Е\,Н\,З\,И\,И \vskip 17pt
    \hrule
    \par
\vskip 6pt plus 6pt minus 3pt }

%\thispagestyle{headings} %с верхним колонтитулом
%\thispagestyle{myheadings} %с нижним колонтитулом, но в верхнем РЕЦЕНЗИИ

\def\tit{НОВАЯ КНИГА И.\,Н.~СИНИЦЫНА, А.\,С.~ШАЛАМОВА <<ЛЕКЦИИ ПО ТЕОРИИ 
ИНТЕГРИРОВАННОЙ ЛОГИСТИЧЕСКОЙ ПОДДЕРЖКИ>> (М.: ТОРУС ПРЕСС, 2012. 624~с.)}

%1
\def\aut{Д.ф.-м.н., профессор С.\,Я.~Шоргин}

\def\auf{\ }

\def\leftkol{\ % РЕЦЕНЗИИ
}

\def\rightkol{ \ } 

%\def\leftkol{\ } % ENGLISH ABSTRACTS}

%\def\rightkol{\ } %ENGLISH ABSTRACTS}

%\def\leftkol{РЕЦЕНЗИИ}

%\def\rightkol{РЕЦЕНЗИИ}

\titele{\tit}{\aut}{\auf}{\leftkol}{\rightkol}
\vspace*{-18pt}


     \label{st\stat}

     \begin{multicols}{2}
     {\small
     {\baselineskip=10.1pt
     

      В книге представлено системное изложение теоретических основ одного из новейших 
направлений в \mbox{об\-ласти} экономики послепродажного обслуживания изделий наукоемкой 
продукции (ИНП) длительного пользования~--- интегрированной логистической поддержки
(ИЛП). 
{\looseness=1

}

Приведены также результаты новых работ, выполненных в Институте проблем информатики 
Российской академии наук в рамках научного направления <<Информационные технологии и 
анализ сложных сис\-тем>>.
 {%\looseness=1

}
     
      Излагаемые в книге научные подходы позво\-ляют карди\-наль\-но реформировать 
существующие системы производства и эксплуатации ИНП путем создания и внед\-ре\-ния 
методов рационального и оптимального управ\-ле\-ния процессами расходования 
вре\-мен\-н$\acute{\mbox{ы}}$х, 
мате\-ри\-аль\-ных, трудовых и других ресурсов на всех стадиях жизненного цикла изделий (ЖЦИ) по 
критериям экономической целесообразности и эф\-фек\-тив\-ности.
  {\looseness=1

}
    
      В книге приведен краткий обзор причин возник\-новения и
      развития CALS-методологии как основы 
современных международных стандартов по созданию и функционированию глобальных 
ин\-фор\-ма\-ци\-он\-но-ком\-му\-ни\-ка\-ци\-он\-ных систем, ее ключевых возможностей и эффективности 
результатов ее использования. 
Авторы %\linebreak 
предлагают ряд научных обоснований для разработки 
единой теории проектирования и управления систем ИЛП для полноценного использования 
преимуществ %\linebreak
 суще\-ст\-ву\-ющей методологии, определяют \mbox{общую} структурную схему 
комплексной системы <<ИНП-СППО>> и необходимость разработки для ее описания 
гибридных стохастических моделей.
{%\looseness=1

}

%\columnbreak
      
      Книга состоит из пяти частей, где последовательно излагается материал по каждой из 
следующих тем: <<Интегрированная логистическая поддержка>>, <<Теория гибридных 
стохастических систем и компьютерная поддержка исследований и разработок>>, <<Основы 
математического моделирования, анализа и синтеза систем послепродажного обслуживания>>, 
<<Определение и анализ показателей экспортного потенциала ИНП при проектировании>>, 
<<Задачи управления поддержкой послепродажного обслуживания>>, а также 
<<Моделирование инвестиционных процессов ИЛП в условиях неравновесных финансовых 
рынков>>. 
   
      В конце каждой главы приведены выводы и даны вопросы и задания для 
самоконтроля. В~приложениях содержатся основные определения по программам работ по 
анализу ИЛП, логистическим базам данных и компьютерным решениям, эквивалентной статистической 
линеаризации нелинейных преобразований ИЛП, справочный материал, а также развернутые 
уравнения для вероятностных характеристик.


      \def\leftkol{РЕЦЕНЗИИ}

\def\rightkol{РЕЦЕНЗИИ} 

      
      Книга заинтересует широкий круг специалистов и может быть использована научными 
проектными организациями в сфере промышленного производства ИНП. Большое количество 
иллюстраций, примеров и вопросов, обращенных к читателю, позволяет использовать книгу 
также в качестве учебного пособия для студентов и аспирантов машиностроительных, 
транспортных и~других специальностей, а также для самостоятельного изучения. 
{%\looseness=-1

}

Книга 
представляет несомненный интерес для специалистов и студентов в области прикладной 
математики и информатики.
    

}

}
\end{multicols}

%\newpage

\def\stat{authorsrus}
{%\hrule\par
%\vskip 7pt % 7pt
\raggedleft\Large \bf%\baselineskip=3.2ex
О\,Б\ \ А\,В\,Т\,О\,Р\,А\,Х \vskip 17pt
    \hrule
    \par
\vskip 21pt plus 8pt minus 6pt }


\def\tit{\ }

\def\aut{\ }

\def\auf{\ }

\def\leftkol{ОБ АВТОРАХ}

\def\rightkol{\ }

\titele{\tit}{\aut}{\auf}{\leftkol}{\rightkol}
\addcontentsline{toc}{subsection}{\textrm\textbf ОБ АВТОРАХ}
\label{st\stat}



\vspace*{-38pt}

\begin{multicols}{2}

\noindent
\textbf{Агаларов Явер Мирзабекович} (р.\ 1952)~--- 
кандидат технических наук, доцент, ведущий научный сотрудник 
Института проб\-лем информатики Федерального исследовательского центра 
<<Информатика и~управ\-ле\-ние>> Российской академии наук

\vspace*{3pt}

\noindent
\textbf{Битюков Юрий Иванович} (р.\ 1972)~---
доктор технических наук, доцент Московского авиационного института 
(национального исследовательского университета) 

\vspace*{3pt}

\noindent
\textbf{Буянов Михаил Владимирович} (р.\ 1994)~--- 
аспирант Московского авиационного института (национального исследовательского 
университета)

\vspace*{3pt}

\noindent
\textbf{Вихрова Ольга Геннадиевна} (р.\ 1990)~---
 аспирант Российского университета дружбы народов
 
 \vspace*{3pt}
 

\noindent
\textbf{Гайдамака Юлия Васильевна} (р.\ 1971)~--- 
кандидат фи\-зи\-ко-ма\-те\-ма\-ти\-че\-ских наук, доцент Российского университета 
дружбы народов; старший научный сотрудник Института проб\-лем информатики 
Федерального исследовательского центра <<Информатика и~управ\-ле\-ние>> 
Российской академии наук 

\vspace*{3pt}

\noindent
\textbf{Горшенин Андрей Константинович} (р.\ 1986)~--- 
кандидат фи\-зи\-ко-ма\-те\-ма\-ти\-че\-ских наук, доцент, 
ведущий научный сотрудник Института проб\-лем информатики Федерального 
исследовательского\linebreak
 центра <<Информатика и~управ\-ле\-ние>> Российской академии наук;
 старший научный сотрудник
 Института океанологии им.\ П.\,П.~Ширшова Российской академии наук

\vspace*{3pt}


\noindent
\textbf{Гребешков Александр Юрьевич} (р.\ 1967)~--- 
кандидат технических наук, старший научный сотрудник Поволжского 
государственного университета телекоммуникаций и информатики

\vspace*{3pt}

\noindent
\textbf{Грушо Александр Александрович} (р.\ 1946)~--- доктор 
фи\-зи\-ко-ма\-те\-ма\-ти\-че\-ских наук, профессор, заведующий лабораторией 
Института проб\-лем информатики Федерального исследовательского центра 
<<Информатика и~управ\-ле\-ние>> Российской академии наук 

\vspace*{3pt}

\noindent
\textbf{Забежайло Михаил Иванович} (р.\ 1956)~--- 
кандидат фи\-зи\-ко-ма\-те\-ма\-ти\-че\-ских наук, доцент, заведующий лабораторией 
Института проб\-лем информатики Федерального исследовательского центра 
<<Информатика и~управ\-ле\-ние>> Российской академии наук 

%\vspace*{3pt}
\columnbreak

\noindent
\textbf{Зарипова Эльвира Ринатовна} (р.\ 1979)~--- 
кандидат фи\-зи\-ко-ма\-те\-ма\-ти\-че\-ских наук, доцент Российского университета 
дружбы народов

\vspace*{3pt}

\noindent
\textbf{Иванов Сергей Валерьевич} (р.\ 1989)~--- 
кандидат фи\-зи\-ко-ма\-те\-ма\-ти\-че\-ских наук, доцент Московского 
авиационного института (национального исследовательского университета)

\vspace*{3pt}

\noindent
\textbf{Кибзун Андрей Иванович}  (р.\ 1951)~--- 
доктор фи\-зи\-ко-ма\-те\-ма\-ти\-че\-ских наук, профессор, 
заведующий кафедрой Московского авиационного института 
(национального исследовательского университета)

\vspace*{3pt}

\noindent
\textbf{Королев Виктор Юрьевич} (р.\ 1954)~--- доктор 
фи\-зи\-ко-ма\-те\-ма\-ти\-че\-ских наук, профессор, 
заведующий кафедрой математической статистики факультета вычислительной 
математики и~кибернетики МГУ им.\ М.\,В.~Ломоносова; 
ведущий научный сотрудник Института проб\-лем информатики 
Федерального исследовательского центра <<Информатика и~управ\-ле\-ние>> 
Российской академии наук; профессор Университета Дианьзи города Ханчжоу (Китай)

\vspace*{3pt}


\noindent
\textbf{Кружков Михаил Григорьевич} (р.\ 1975)~--- 
старший научный сотрудник Института проб\-лем 
информатики Федерального исследовательского центра 
<<Информатика и~управ\-ле\-ние>> Российской академии наук

\vspace*{3pt}

\noindent
\textbf{Кудрявцев Алексей Андреевич} (p.\ 1978)~--- кандидат 
фи\-зи\-ко-ма\-те\-ма\-ти\-че\-ских наук, 
доцент кафедры математической статистики факультета вычислительной математики 
и~кибернетики Московского государственного университета им.\ М.\,В.~Ломоносова

\vspace*{3pt}

\noindent
\textbf{Лисовская Екатерина Юрьевна} (р.\ 1992)~--- 
аспирант Национального исследовательского 
Томского государственного университета 

\vspace*{3pt}

\noindent
\textbf{Малашенко Юрий Евгеньевич} (р.\ 1946)~---
доктор фи\-зи\-ко-ма\-те\-ма\-ти\-че\-ских наук, заведующий сектором 
Вычислительного центра им.\ А.\,А.~Дородницына Федерального исследовательского центра 
<<Информатика и~управ\-ле\-ние>> Российской академии \mbox{наук}

\vspace*{3pt}


\noindent
\textbf{Моисеева Светлана Петровна} (р.\ 1971)~--- 
доктор фи\-зи\-ко-ма\-те\-ма\-ти\-че\-ских наук, доцент; 
профессор Национального исследовательского Томского государственного 
университета  

%\vspace*{3pt}
\pagebreak

\noindent
\textbf{Мокров Евгений Владимирович} (р.\ 1988)~--- 
аспирант Российского университета дружбы народов 

\vspace*{3pt}

\noindent
\textbf{Назарова Ирина Александровна} (р.\ 1966)~---
 кандидат фи\-зи\-ко-ма\-те\-ма\-ти\-че\-ских наук, научный сотрудник 
 Вычислительного центра им.\ А.\,А.~Дородницына Федерального исследовательского центра 
 <<Информатика и~управ\-ле\-ние>> Российской академии наук

\vspace*{3pt}

\noindent
\textbf{Наумов Андрей Викторович} (р.\ 1966)~--- 
доктор фи\-зи\-ко-ма\-те\-ма\-ти\-че\-ских наук, доцент, 
профессор\linebreak Московского авиационного института (национального исследовательского 
университета)

\vspace*{3pt}

\noindent
\textbf{Наумов Валерий Арсентьевич} (р.\ 1950)~--- 
кандидат фи\-зи\-ко-ма\-те\-ма\-ти\-че\-ских наук, 
научный руководитель Исследовательского института инноваций, 
г.~Хельсинки, Финляндия

\vspace*{3pt}

\noindent
\textbf{Новикова Наталья Михайловна} (р.\ 1953)~--- 
доктор фи\-зи\-ко-ма\-те\-ма\-ти\-че\-ских наук, профессор, ведущий научный сотрудник 
Вычислительного центра им.\ А.\,А.~Дородницына Федерального исследовательского центра 
<<Информатика и~управ\-ле\-ние>> Российской академии наук

\vspace*{3pt}

\noindent
\textbf{Пагано Микеле} (р.\ 1968)~---
PhD по информационным технологиям, профессор Университета 
г.\ Пиза (Италия) 

\vspace*{3pt}

\noindent
\textbf{Платонов Евгений Николаевич} (р.\ 1976)~---  
кандидат фи\-зи\-ко-ма\-те\-ма\-ти\-че\-ских наук, 
доцент Московского авиационного института (национального исследовательского 
университета)

\vspace*{3pt}

\noindent
\textbf{Потатуева Виктория Владимировна} (р.\ 1993)~---  
студентка магистратуры Национального исследовательского 
Томского государственного университета

\vspace*{3pt}


\noindent
\textbf{Разумчик Ростислав Валерьевич} (р.\ 1984)~--- 
кандидат фи\-зи\-ко-ма\-те\-ма\-ти\-че\-ских наук, 
ведущий научный сотрудник Института проб\-лем 
информатики Федерального исследовательского центра <<Информатика и~управ\-ле\-ние>>
Российской академии наук;  доцент Российского университета дружбы народов

\vspace*{3pt}

\noindent
\textbf{Самуйлов Константин Евгеньевич} (р.\ 1955)~---
доктор технических наук, профессор, заведующий ка\-фед\-рой Российского 
университета дружбы наро-\linebreak дов, директор Института прикладной математики\linebreak 
и~телекоммуникаций Российского университета дружбы народов; 
старший научный сотрудник Института проб\-лем информатики Федерального 
исследовательского центра <<Информатика и~управ\-ле\-ние>> 
Российской академии наук

\vspace*{3pt}

\noindent
\textbf{Смирнов Дмитрий Владимирович} (р.\ 1984)~--- 
биз\-нес-парт\-нер по информационным технологиям Департамента безопасности ПАО 
<<Сбербанк России>>

\vspace*{3pt}

\noindent
\textbf{Тимонина Елена Евгеньевна} (р.\ 1952)~--- 
доктор технических наук, профессор, ведущий научный\linebreak сотрудник 
Института проб\-лем информатики Федерального исследовательского центра 
<<Информатика и~управ\-ле\-ние>> Российской академии наук 

\vspace*{3pt}

\noindent
\textbf{Титова Анастасия Игоревна} (p.\ 1995)~--- 
студентка кафедры математической статистики факультета вычисли\-тельной математики 
и~кибернетики Московского государственного университета им.\ М.\,В.~Ломоносова

\vspace*{3pt}

\noindent
\textbf{Шоргин Всеволод Сергеевич} (р.\ 1978)~---
кандидат технических наук, старший научный сотрудник Института проб\-лем 
информатики Федерального исследовательского центра <<Информатика и~управ\-ле\-ние>> 
Российской академии наук

\vspace*{3pt}

\noindent
\textbf{Шоргин Сергей Яковлевич} (р.\ 1952)~--- 
доктор фи\-зи\-ко-ма\-те\-ма\-ти\-че\-ских наук, профессор, заместитель директора 
Федерального исследовательского цент\-ра <<Информатика и~управ\-ле\-ние>> 
Российской академии наук (ФИЦ ИУ РАН); главный научный сотрудник Института проб\-лем 
информатики ФИЦ ИУ РАН
 



 \label{end\stat}

%\def\leftfootline{\small{\textbf{\thepage}
%\hfill ИНФОРМАТИКА И ЕЁ ПРИМЕНЕНИЯ\ \ \ том~11\ \ \ выпуск~4\ \ \ 2017}
%}%
% \def\rightfootline{\small{ИНФОРМАТИКА И ЕЁ ПРИМЕНЕНИЯ\ \ \ том~11\ \ \ выпуск~4\ \ \ 2017
%\hfill \textbf{\thepage}}}


%\thispagestyle{myheadings}



\end{multicols}

\newpage

%\end{document}

%
\def\stat{rekl}
%\label{preobr}

%\def\tit{АКАДЕМИК ПУГАЧЁВ  ВЛАДИМИР СЕМЁНОВИЧ\\
%25.03.1911--25.03.1998}


%   \vspace*{-48pt}
%   \begin{center}\LARGE
%Академик Пугачёв  Владимир Семёнович\\ (25.03.1911--25.03.1998)
%   \end{center}

   %\vspace*{2.5mm}

   \begin{center}

{\prgsh\LARGE
ЮБИЛЕИ}

\end{center}
%\hrule

\vspace*{6pt}


   \vspace*{8mm}

   \thispagestyle{empty}


%\def\stat{emel}


\section*{К 70-летию заместителя директора ИПИ РАН,\\ члена редколлегии журнала
<<Информатика и её применения>>\\ доктора технических наук В.\,И.~Будзко}

\vspace*{18pt}




          \begin{multicols}{2}

%            \label{st\stat}

\begin{center}
\vspace*{1pt}
\mbox{%
\epsfxsize=78mm
\epsfbox{bud-1.eps}
}
\end{center}

\vspace*{12pt}

      14 августа 2014~г.\ исполнилось 70~лет за\-мес\-ти\-те\-лю директора ИПИ РАН по
научной работе доктору технических наук Владимиру Игоревичу Будзко.

      Владимир Игоревич Будзко родился в г.~Москве. Высшее образование получил на факультете
элект\-рон\-но-вы\-чис\-ли\-тель\-ных устройств в Московском
ин\-же\-нер\-но-фи\-зи\-че\-ском институте
(МИФИ), который он окончил в 1968~г., после чего был на\-прав\-лен для прохождения
службы в одну из войс\-ко\-вых частей, где прошел путь от инженера до первого заместителя
командира войсковой части.

      С приходом В.\,И.~Будзко в ИПИ РАН (2001~г.)\ в институте
сформировалось новое научное на\-прав\-ле\-ние теоретических исследований~--- <<Постро\-ение
ин\-фор\-ма\-ци\-он\-но-те\-ле\-ком\-му\-ни\-ка\-ци\-он\-ных\linebreak сис\-тем
высокой до\-ступ\-ности>>. В~рамках этого
направления выполнен широкий круг фундаментальных исследований по поиску подходов и
определению принципов построения средств обеспечения доступности, конфиденциальности
и целостности современных крупномасштабных
ин\-фор\-ма\-ци\-он\-но-те\-ле\-ком\-му\-ни\-ка\-ци\-он\-ных
сис\-тем (ИТС). Разработаны основные сис\-тем\-но-тех\-ни\-че\-ские принципы и базовые
архитектурные решения построения перспективных для условий России ИТС с
централизованной обработкой и хранением информации, сочетающих в себе свойства
высокой доступности, отказо- и катастрофоустойчивости, информационной защищенности.
Определены принципы, методы и математические основы рационального построения и
оптимизации средств восстановления функционирования центров обработки данных (ЦОД)
после возникновения отказов и катастроф, передачи и хранения данных, обеспечения
информационной безопасности при достижении минимальной совокупной стоимости
владения такими системами. Результаты нашли практическое воплощение при реализации
проектов в интересах ряда отечественных государственных и негосударственных
организаций, таких как Банк России (БР), Внешторгбанк, ОАО <<ГМК <<Норильский Никель>>,
<<Газпром>>, Минэкономразвития России, Правительство Москвы, а также ряд силовых
ведомств.

      Под руководством В.\,И.~Будзко начиная с 2001~г.\ выполнен комплекс
      на\-уч\-но-ис\-сле\-до\-ва\-тель\-ских и
      опыт\-но-кон\-ст\-рук\-тор\-ских работ (свыше 100~проектов),
направленных на развитие электронной информационной технологии БР.
Разработаны концепции развития ИТС БР сначала до 2008~г., а затем до 2013~г., которые
были приняты в качестве основы проведения технической политики. За реализацию проекта
<<Катастрофоустойчивая тер\-ри\-то\-ри\-аль\-но-рас\-пре\-де\-лен\-ная
      ин\-фор\-ма\-ци\-он\-но-те\-ле\-ком\-му\-ни\-ка\-ци\-он\-ная сис\-те\-ма централизованной
обработки банковской информации>> В.\,И.~Будзко удостоен Премии Правительства РФ в
области науки и техники за 2010~г.

      В.\,И.~Будзко возглавлял и возглавляет работы по ряду других прикладных проектов,
связанных с созданием, совершенствованием и развитием крупномасштабных ИТС.

      В.\,И.~Будзко~--- генерал-майор, доктор технических наук, член-кор\-рес\-пон\-дент
Академии криптографии РФ, известный ученый в области информатики и применения
информационных технологий при построении территориально распределенных ИТС
различного назначения. Является автором свыше 250~научных работ, опубликованных в
на\-уч\-но-тех\-ни\-че\-ских и специальных изданиях.

    \thispagestyle{empty}

      В.\,И.~Будзко уделяет большое внимание подготовке научных кадров. Под его
руководством защищено 6~диссертаций на соискание ученой степени кандидата
технических наук. Свыше 30~лет он читает лекции в ИКСИ Академии ФСБ, профессор
кафедры НИЯУ МИФИ. Является членом двух диссертационных советов, главным
редактором журнала <<Системы высокой доступности>> и членом редколлегии журнала
<<Информатика и её применения>>.

      \bigskip

      Редакционный совет и Редакционная коллегия журнала <<Информатика и её
применения>> сердечно поздравляют Владимира Игоревича Будзко с 70-ле\-ти\-ем и желают
крепкого здоровья и новых научных достижений.

\end{multicols}

\def\stat{cont}
{%\hrule\par
%\vskip 7pt % 7pt
\raggedleft\Large \bf%\baselineskip=3.2ex
А\,В\,Т\,О\,Р\,С\,К\,И\,Й\ \ У\,К\,А\,З\,А\,Т\,Е\,Л\,Ь\ \ З\,А\ \ 2\,0\,1\,0 г. \vskip 17pt
    \hrule
    \par
\vskip 21pt plus 6pt minus 3pt }

\label{st\stat}

\def\tit{\ }

\def\aut{\ }
\def\auf{\ }

\def\leftkol{\ } % ENGLISH ABSTRACTS}

\def\rightkol{\ } %АВТОРСКИЙ УКАЗАТЕЛЬ ЗА 2010 г.} %ENGLISH ABSTRACTS}

\titele{\tit}{\aut}{\auf}{\leftkol}{\rightkol}

\vspace*{-12pt}

{\tabcolsep=3pt
\begin{tabular}{p{388pt}rr}
&\textbf{Выпуск} & \textbf{Стр.}\\[6pt]
\hangindent=23pt\noindent\textbf{Арутюнян~А.\,Р.} Моделирование влияния деформаций отпечатков пальцев на 
точность\linebreak
\vspace*{-12pt}\\
\hspace*{23pt}дактилоскопической идентификации$\dotfill$&1&51\\
\hangindent=23pt\noindent\textbf{Архипов~О.\,П., Зыкова~З.\,П.} Интеграция гетерогенной информации о цветных 
пикселях\linebreak
\vspace*{-12pt}\\
\hspace*{23pt}и их цветовосприятии$\dotfill$&4&15\\
\hangindent=23pt\noindent\textbf{Баранов~С.\,И., Френкель~С.\,Л., Захаров~В.\,Н.} Полуформальная верификация 
цифрового устройства с конвейером, основанная на использовании алгоритмических машин\linebreak
\vspace*{-12pt}\\
\hspace*{23pt}состояния$\dotfill$&4&49\\
\textbf{Бекетова~И.\,В.} см.~Каратеев~С.\,Л.&&\\
\textbf{Белоусов~В.\,В.} см.~Синицын~И.\,Н.&&\\
\hangindent=23pt\noindent\textbf{Бенинг~В.\,Е., Королев~Р.\,А.} О предельном поведении мощностей критериев в 
случае\linebreak
\vspace*{-12pt}\\
\hspace*{23pt}распределения Лапласа$\dotfill$&2&63\\
\hangindent=23pt\noindent\textbf{Бенинг~В.\,Е., Сипина~А.\,В.} Асимптотическое разложение для мощности 
критерия,\linebreak
\vspace*{-12pt}\\
\hspace*{23pt}основанного на выборочной медиане, в случае распределения Лапласа$\dotfill$&1&18\\
\textbf{Бондаренко~А.\,В.} см.~Каратеев~С.\,Л.&&\\
\hangindent=23pt\noindent\textbf{Бородина~А.\,В., Морозов~Е.\,В.} Об оценивании асимптотики вероятности 
большого\linebreak
\vspace*{-12pt}\\
\hspace*{23pt}уклонения стационарной регенеративной очереди с одним прибором$\dotfill$&3&29\\
\hangindent=23pt\noindent\textbf{Бунтман~Н.\,В., Минель~Ж.-Л., Ле~Пезан~Д., Зацман~И.\,М.} Типология и 
компьютерное\linebreak
\vspace*{-12pt}\\
\hspace*{23pt}моделирование трудностей перевода$\dotfill$&3&77\\
\textbf{Визильтер~Ю.\,В.} см.~Каратеев~С.\,Л.&&\\
\hangindent=23pt\noindent\textbf{Гавриленко~С.\,В.} Оценки скорости сходимости распределений случайных сумм с 
безгранично делимыми индексами к нормальному закону$\dotfill$&4&81\\
\hangindent=23pt\noindent\textbf{Григорьева~М.\,Е., Шевцова~И.\,Г.} Уточнение неравенства 
Каца--Берри--Эссеена$\dotfill$&2&75\\
\hangindent=23pt\noindent\textbf{Грушо~А.\,А., Грушо~Н.\,А., Тимонина~Е.\,Е.} Поиск конфликтов в политиках 
безопасности: модель случайных графов$\dotfill$&3&38\\
\textbf{Грушо~Н.\,А.} см.~Грушо~А.\,А.&&\\
\hangindent=23pt\noindent\textbf{Гудков~В.\,Ю.} Математические модели изображения отпечатка пальца на основе 
описания линий$\dotfill$&1&58\\
\textbf{Гуртов~А.\,В.} см.~Лукьяненко~А.\,С.&&\\
\textbf{Желтов~С.\,Ю.} см.~Каратеев~С.\,Л.&&\\
\hangindent=23pt\noindent\textbf{Захаров~А.\,А., Серебряков~В.\,А.} Система управления электронной библиотекой 
LibMeta$\dotfill$&4&2\\
\textbf{Захаров~В.\,Н.} см.~Баранов~С.\,И.&&\\
\textbf{Захарова~Т.\,В.} см.~Матвеева~С.\,С.&&\\
\hangindent=23pt\noindent\textbf{Зацаринный~А.\,А., Чупраков~К.\,Г.} Некоторые аспекты выбора технологии для 
постро-\linebreak
\vspace*{-12pt}\\
\hspace*{23pt}ения систем отображения информации ситуационного центра$\dotfill$&3&59\\
\textbf{Зацман~И.\,М.} см.~Бунтман~Н.\,В.&&\\
\hangindent=23pt\noindent\textbf{Зейфман~А.\,И., Коротышева~А.\,В., Сатин~Я.\,А., Шоргин~С.\,Я.} Об 
устойчивости нестаци-\linebreak
\vspace*{-12pt}\\
\hspace*{23pt}онарных систем обслуживания с катастрофами$\dotfill$&3&9\\
\textbf{Зыкова~З.\,П.} см.~Архипов~О.\,П.&&\\
\hangindent=23pt\noindent\textbf{Илюшин~Г.\,Я., Соколов~И.\,А.} Организация управляемого доступа пользователей 
к\linebreak
\vspace*{-12pt}\\
\hspace*{23pt}разнородным ведомственным информационным ресурсам$\dotfill$&1&24\\
\hangindent=23pt\noindent\textbf{Кавагучи~Ю., Ульянов~В.\,В., Фуджикоши~Я.} Приближения для статистик, 
описывающих\linebreak
\vspace*{-12pt}\\
\hspace*{23pt}геометрические свойства данных большой размерности, с оценками 
ошибок$\dotfill$&1&12\\
\hangindent=23pt\noindent\textbf{Каратеев~С.\,Л., Бекетова~И.\,В., Ососков~М.\,В., Князь~В.\,А., 
Визильтер~Ю.\,В., Бондаренко~А.\,В., Желтов~С.\,Ю.} Автоматизированный контроль 
качества цифровых\linebreak
\vspace*{-12pt}\\
\hspace*{23pt}изображений для персональных документов$\dotfill$&1&65\\
\end{tabular}
}

\pagebreak

\def\leftkol{АВТОРСКИЙ УКАЗАТЕЛЬ ЗА 2010 г.} % ENGLISH ABSTRACTS}

\def\rightkol{АВТОРСКИЙ УКАЗАТЕЛЬ ЗА 2010 г.} %ENGLISH ABSTRACTS}

{\tabcolsep=3pt
\begin{tabular}{p{388pt}rr}
&\textbf{Выпуск} & \textbf{Стр.}\\[3pt]
\hangindent=23pt\noindent\textbf{Козеренко~Е.\,Б.} Лингвистические фильтры в статистических моделях машинного\linebreak
\vspace*{-12pt}\\
\hspace*{23pt}перевода$\dotfill$&2&83\\
\hangindent=23pt\noindent\textbf{Козеренко~Е.\,Б., Кузнецов~И.\,П.} Когнитивно-лингвистические представления в 
систе-\linebreak
\vspace*{-12pt}\\
\hspace*{23pt}мах обработки текстов$\dotfill$&3&69\\
\textbf{Князь~В.\,А.} см.~Каратеев~С.\,Л.&&\\
\hangindent=23pt\noindent\textbf{Колесников~А.\,В., Солдатов~С.\,А.} Алгоритм координации для гибридной 
интеллектуальной системы решения сложной задачи оперативно-производственного\linebreak
\vspace*{-12pt}\\
\hspace*{23pt}планирования$\dotfill$&4&61\\
\hangindent=23pt\noindent\textbf{Коновалов~М.\,Г.} О планировании потоков в системах вычислительных 
ресурсов$\dotfill$&2&3\\
\textbf{Конушин~А.\,С.} см.~Конушин~В.\,С.&&\\
\hangindent=23pt\noindent\textbf{Конушин~В.\,С., Кривовязь~Г.\,Р., Конушин~А.\,С.} Алгоритм распознавания людей 
в видео-\linebreak
\vspace*{-12pt}\\
\hspace*{23pt}последовательности по одежде$\dotfill$&1&74\\
\textbf{Корепанов~Э.\, Р.} см.~Синицын~И.\,Н.&&\\
\textbf{Королев~В.\,Ю.} см.~Соколов~И.\,А.&&\\
\textbf{Королев~Р.\,А.} см.~Бенинг~В.\,Е.&&\\
\textbf{Коротышева~А.\,В.} см.~Зейфман~А.\,И.&&\\
\hangindent=23pt\noindent\textbf{Кривенко~М.\,П.} Непараметрическое оценивание элементов байесовского 
клас\-си-\linebreak
\vspace*{-12pt}\\
\hspace*{23pt}фикатора$\dotfill$&2&13\\
\textbf{Кривовязь~Г.\,Р.} см.~Конушин~В.\,С.&&\\
\textbf{Крылов~А.\,С.} см.~Павельева~Е.\,А.&&\\
\hangindent=23pt\noindent\textbf{Крылов~В.\,А.} Моделирование и классификация многоканальных дистанционных\linebreak
\vspace*{-12pt}\\
\hspace*{23pt}изображений с использованием копул$\dotfill$&4&34\\
\hangindent=23pt\noindent\textbf{Крючин~О.\,В.} Разработка параллельных эвристических алгоритмов подбора 
весовых\linebreak
\vspace*{-12pt}\\
\hspace*{23pt}коэффициентов искусственной нейтронной сети$\dotfill$&2&53\\
\hangindent=23pt\noindent\textbf{Кудрявцев~А.\,А., Шоргин~С.\,Я.} Байесовские модели массового обслуживания и 
надеж-\linebreak
\vspace*{-12pt}\\
\hspace*{23pt}ности: характеристики среднего числа заявок в системе $M\vert M \vert 1\vert 
\infty$$\dotfill$&3&16\\
\hangindent=23pt\noindent\textbf{Кузнецов~А.\,А.} Связь между временными и структурно-топологическими 
характери-\linebreak
\vspace*{-12pt}\\
\hspace*{23pt}стиками диаграмм ритма сердца здоровых людей$\dotfill$&4&39\\
\textbf{Кузнецов~И.\,П.} см.~Козеренко~Е.\,Б.&&\\
\textbf{Ле~Пезан~Д.} см.~Бунтман~Н.\,В.&&\\
\hangindent=23pt\noindent\textbf{Лукьяненко~А.\,С., Морозов~Е.\,В., Гуртов~А.\,В.} Анализ сетевого протокола с общей 
функ-\linebreak
\vspace*{-12pt}\\
\hspace*{23pt}цией расширения окна передачи сообщения при конфликтах$\dotfill$&2&46\\
\hangindent=23pt\noindent\textbf{Лямин~О.\,О.} О предельном поведении мощностей критериев в случае обобщенного\linebreak
\vspace*{-12pt}\\
\hspace*{23pt}распределения Лапласа$\dotfill$&3&47\\
\hangindent=23pt\noindent\textbf{Маркин~А.\,В., Шестаков~О.\,В.} Асимптотики оценки риска при пороговой 
обработке\linebreak
\vspace*{-12pt}\\
\hspace*{23pt}вейвлет-вейглет коэффициентов в задаче томографии$\dotfill$&2&36\\
\hangindent=23pt\noindent\textbf{Матвеева~С.\,С., Захарова~Т.\,В.} Сети массового обслуживания с наименьшей 
длиной\linebreak
\vspace*{-12pt}\\
\hspace*{23pt}очереди$\dotfill$&3&22\\
\hangindent=23pt\noindent\textbf{Матюшенко~С.\,И.} Стационарные характеристики двухканальной системы 
обслужива-\linebreak
\vspace*{-12pt}\\
\hspace*{23pt}ния с переупорядочиванием заявок и распределениями фазового типа$\dotfill$&4&68\\
\textbf{Минель~Ж.-Л.} см.~Бунтман~Н.\,В.&&\\
\textbf{Морозов~Е.\,В.} см.~Бородина~А.\,В.&&\\
\textbf{Морозов~Е.\,В.} см.~Лукьяненко~А.\,С.&&\\
\textbf{Ососков~М.\,В.} см.~Каратеев~С.\,Л.&&\\
\hangindent=23pt\noindent\textbf{Павельева~Е.\,А., Крылов~А.\,С.} Поиск и анализ ключевых точек радужной 
оболочки\linebreak
\vspace*{-12pt}\\
\hspace*{23pt}глаза методом преобразования Эрмита$\dotfill$&1&79\\
\textbf{Печинкин~А.\,В.} см.~Френкель~С.\,Л.,&&\\
\hangindent=23pt\noindent\textbf{Протасов~В.\,И.} Составление субъективного портрета с использованием 
эволюционно-\linebreak
\vspace*{-12pt}\\
\hspace*{23pt}го морфинга и квалиметрия метода$\dotfill$&1&83\\
\hangindent=23pt\noindent\textbf{Рудаков~К.\,В., Торшин~И.\,Ю.} Вопросы разрешимости задачи распознавания 
вторичной\linebreak
\vspace*{-12pt}\\
\hspace*{23pt}структуры белка$\dotfill$&2&25\\
\textbf{Сатин~Я.\,А.} см.~Зейфман~А.\,И.&&\\
\hangindent=23pt\noindent\textbf{Сейфуль-Мулюков~Р.\,Б.} Нефть как носитель информации о своем 
происхождении,\linebreak
\vspace*{-12pt}\\
\hspace*{23pt}структуре и эволюции$\dotfill$&1&41\\
\end{tabular}
}

{\tabcolsep=3pt
\begin{tabular}{p{388pt}rr}
&\textbf{Выпуск} & \textbf{Стр.}\\[6pt]
\textbf{Семендяев~Н.\,Н.} см.~Синицын~И.\,Н.&&\\
\textbf{Серебряков~В.\,А.} см.~Захаров~А.\,А.&&\\
\textbf{Синицын~В.\,И.} см.~Синицын~И.\,Н.&&\\
\hangindent=23pt\noindent\textbf{Синицын~И.\,Н., Синицын~В.\,И., Корепанов~Э.\, Р., Белоусов~В.\,В., 
Семендяев~Н.\,Н.} Оперативное построение информационных моделей движения полюса 
Земли\linebreak
\vspace*{-12pt}\\
\hspace*{23pt}методами линейных и линеаризованных фильтров$\dotfill$&1&2\\
\textbf{Сипина~А.\,В.} см.~Бенинг~В.\,Е.&&\\
\hangindent=23pt\noindent\textbf{Соколов~И.\,А.} О работах заслуженного деятеля науки Российской Федерации 
И.\,Н.~Синицына в области информационных технологий и автоматизации (к 70-летию\linebreak
\vspace*{-12pt}\\
\hspace*{23pt}со дня рождения)$\dotfill$&3&84\\
\textbf{Соколов~И.\,А.} см.~Илюшин~Г.\,Я.&&\\
\hangindent=23pt\noindent\textbf{Соколов~И.\,А., Королев~В.\,Ю.} Предисловие$\dotfill$&2&2\\
\textbf{Солдатов~С.\,А.} см.~Колесников~А.\,В.&&\\
\hangindent=23pt\noindent\textbf{Степанов~С.\,Ю.} Использование координатного метода фрагментации 
коммутаторной\linebreak
\vspace*{-12pt}\\
\hspace*{23pt}нейронной сети для сокращения трафика$\dotfill$&2&57\\
\textbf{Тимонина~Е.\,Е.} см.~Грушо~А.\,А.&&\\
\textbf{Торшин~И.\,Ю.} см.~Рудаков~К.\,В.&&\\
\textbf{Ульянов~В.\,В.} см.~Кавагучи~Ю.&&\\
\textbf{Фазекаш~И.} см.~Чупрунов~А.\,Н.&&\\
\textbf{Френкель~С.\,Л.} см.~Баранов~С.\,И.&&\\
\hangindent=23pt\noindent\textbf{Френкель~С.\,Л., Печинкин~А.\,В.} Оценка времени самовосстановления в 
цифровых\linebreak
\vspace*{-12pt}\\
\hspace*{23pt}системах после сбоев, вызываемых переходными помехами$\dotfill$&3&2\\
\textbf{Фуджикоши~Я.} см.~Кавагучи~Ю.&&\\
\hangindent=23pt\noindent\textbf{Цискаридзе~А.\,К.} Математическая модель и метод восстановления позы человека 
по\linebreak
\vspace*{-12pt}\\
\hspace*{23pt}стереопаре силуэтных изображений$\dotfill$&4&27\\
\hangindent=23pt\noindent\textbf{Чупраков~К.\,Г.} К вопросу о размещении коллективных средств отображения в 
ситуа-\linebreak
\vspace*{-12pt}\\
\hspace*{23pt}ционном зале с заданными параметрами$\dotfill$&4&89\\
\textbf{Чупраков~К.\,Г.} см.~Зацаринный~А.\,А.&&\\
\hangindent=23pt\noindent\textbf{Чупрунов~А.\,Н., Фазекаш~И.} Законы повторного логарифма для числа 
безошибочных\linebreak
\vspace*{-12pt}\\
\hspace*{23pt}блоков при помехоустойчивом кодировании$\dotfill$&3&42\\
\textbf{Шевцова~И.\,Г.} см.~Григорьева~М.\,Е.&&\\
\hangindent=23pt\noindent\textbf{Шестаков~О.\,В.} Аппроксимация распределения оценки риска пороговой 
обработки вейвлет-коэффициентов нормальным распределением при использовании 
выбо-\linebreak
\vspace*{-12pt}\\
\hspace*{23pt}рочной дисперсии$\dotfill$&4&73\\
\textbf{Шестаков~О.\,В.} см.~Маркин~А.\,В.&&\\
\textbf{Шоргин~С.\,Я.} см.~Зейфман~А.\,И.&&\\
\textbf{Шоргин~С.\,Я.} см.~Кудрявцев~А.\,А.&&\\
\end{tabular}
}

%\thispagestyle{myheadings}
\def\leftfootline{\small{\textbf{\thepage}
\hfill ИНФОРМАТИКА И ЕЁ ПРИМЕНЕНИЯ\ \ \ том~4\ \ \ выпуск~4\ \ \ 2010}
}%
 \def\rightfootline{\small{ИНФОРМАТИКА И ЕЁ ПРИМЕНЕНИЯ\ \ \ том~4\ \ \ выпуск~4\ \ \ 2010
 \hfill \textbf{\thepage}}}
 \label{end\stat}





%Том 10 Выпуск 1-4 Год 2016

\def\stat{cont-e}
{%\hrule\par
%\vskip 7pt % 7pt
\raggedleft\Large \bf%\baselineskip=3.2ex
2\,0\,1\,6\ \ A\,U\,T\,H\,O\,R\ \ I\,N\,D\,E\,X \vskip 17pt
 \hrule
 \par
\vskip 21pt plus 6pt minus 3pt }

\label{st\stat}

\def\tit{\ }

\def\aut{\ }
\def\auf{\ }

\def\leftkol{\ } %2016 AUTHOR INDEX} % ENGLISH ABSTRACTS}

\def\rightkol{\ } %2016 AUTHOR INDEX} %ENGLISH ABSTRACTS}

\titele{\tit}{\aut}{\auf}{\leftkol}{\rightkol}

\def\leftfootline{\small{\textbf{\thepage}
\hfill INFORMATIKA I EE PRIMENENIYA~--- INFORMATICS AND APPLICATIONS\ \ \ 2016\
\ \ volume~10\ \ \ issue\ 4}
}%
 \def\rightfootline{\small{INFORMATIKA I EE PRIMENENIYA~--- INFORMATICS AND APPLICATIONS\ \ \ 2016\ \ \ volume~10\ \ \ issue\ 4
\hfill \textbf{\thepage}}}

\vspace*{-12pt}
\vspace*{-18pt}

{\tabcolsep=2.8pt
\begin{tabular}{p{382pt}cc}
&\textbf{Issue} & \textbf{Page}\\[6pt]
\Avtors{Agalarov~M.\,Ya.} see~Agalarov~Ya.\,M.&&\\
\Avtors{Agalarov~Ya.\,M., Agalarov~M.\,Ya., and
Shorgin~V.\,S.} About the optimal threshold of queue\linebreak
\\[-12pt]
\hspace*{23pt}length in a~particular problem of profit maximization
in the $M/G/1$ queuing system&2&70--79\\
\Avtors{Alexeyevsky~D.\,A.} BioNLP ontology extraction from 
a~restricted language corpus with\linebreak
\\[-12pt]
\hspace*{23pt}context-free grammars&1&119--128\\
\Avtors{Andreev~S.\,D.} see~Gaidamaka~Yu.\,V.&&\\
\Avtors{Andreev~S.\,D.} see~Ometov~A.\,Ya.&&\\
\Avtors{Arkhipov~O.\,P., Arkhipov~P.\,O., and Sidorkin~I.\,I.} The
option to create a~local coordinate\linebreak
\\[-12pt]
\hspace*{23pt}system for synchronization of selected images&3&91--97\\
\Avtors{Arkhipov~P.\,O.} see~Arkhipov~O.\,P.&&\\
\Avtors{Belousov~V.\,V.} see~Shnurkov~P.\,V.&&\\
\Avtors{Belousov~V.\,V.} see~Shnurkov~P.\,V.&&\\
\Avtors{Bening~V.\,E.} Calculation of~the~asymptotic deficiency
of~some statistical procedures based\linebreak
\\[-12pt]
\hspace*{23pt}on~samples with~random sizes&4&34--45\\
\Avtors{Borisov~A.\,V., Bosov~A.\,V., and Miller~G.\,B.} Modeling and
monitoring of VoIP connection&2&\hphantom{1}2--13\\
\Avtors{Bosov~A.\,V.} see~Borisov~A.\,V.&&\\
\Avtors{Briukhov~D.\,O.} see~Stupnikov~S.\,A.&&\\
\Avtors{Callaos~N.\,K.\ and Seyful-Mulyukov~R.\,B.} Complexity and
its information content&1&129--139\\
\Avtors{Chertok~A.\,V., Kadaner~A.\,I., Khazeeva~G.\,T., and
Sokolov~I.\,A.} Regime switching detection\linebreak
\\[-12pt]
\hspace*{23pt}for~the~Levy driven
Ornstein--Uhlenbeck process using CUSUM methods&4&46--56\\
\Avtors{Chichagov~V.\,V.} Asymptotic expansions of mean absolute
error of uniformly minimum variance unbiased and maximum likelihood
estimators on the one-parameter exponential\linebreak
\\[-12pt]
\hspace*{23pt}family model of lattice distributions&3&66--76\\
\Avtors{Danishevsky~V.\,I.} see~Kolesnikov A.\,V.&&\\
\Avtors{Fazliev~A.\,Z.} see~Kalinichenko~L.\,A.&&\\
\Avtors{Fedoseev~A.\,A.} What is behind the concept of ``knowledge in
small packages''&3&105--110\\
\Avtors{Gaidamaka~Yu.\,V., Andreev~S.\,D., Sopin~E.\,S.,
Samouylov~K.\,E., and Shorgin~S.\,Ya.} Interference analysis
of~the~device-to-device communications model with~regard to~a~signal\linebreak
\\[-12pt]
\hspace*{23pt}propagation environment&4&\hphantom{1}2--10\\
\Avtors{Gasilov~A.\,V.} see~Yakovlev~O.\,A.&&\\
\Avtors{Goncharov~A.\,V.\ and Strijov~V.\,V.} Metric time series
classification using weighted dynamic\linebreak
\\[-12pt]
\hspace*{23pt}warping relative to centroids of classes&2&36--47\\
\Avtors{Gordov~E.\,P.} see~Kalinichenko~L.\,A.&&\\
\Avtors{Gorshenin~A.\,K.} Concept of online service for stochastic
modeling of real processes&1&72--81\\
\Avtors{Gorshenin~A.\,K.} see~Shnurkov~P.\,V.&&\\
\Avtors{Gorshenin~A.\,K.} see~Shnurkov~P.\,V.&&\\
\Avtors{Grusho~A.\,A., Grusho~N.\,A., Zabezhailo~M.\,I., and
Timonina~E.\,E.} Integration of statistical and\linebreak
\\[-12pt]
\hspace*{23pt}deterministic methods for
analysis of information security&3&2--8\\
\Avtors{Grusho~A.\,A., Zabezhailo~M.\,I., and Zatsarinny~A.\,A.} On
the advanced procedure to reduce\linebreak
\\[-12pt]
\hspace*{23pt}calculation of Galois closures&4&\hphantom{1}96--104\\
\Avtors{Grusho~N.\,A.} see~Grusho~A.\,A.&&\\
\Avtors{Havanskov~V.\,A.} see~Minin~V.\,A.&&\\
\Avtors{Inkova~O.\,Yu.} see~Zatsman~I.\,M.&&\\
\Avtors{Isachenko~R.\,V.\ and Strijov~V.\,V.} Metric learning in
multiclass time series classification\linebreak
\\[-12pt]
\hspace*{23pt}problem&2&48--57\\
\end{tabular}
}
\pagebreak

\def\leftfootline{\small{\textbf{\thepage}
\hfill INFORMATIKA I EE PRIMENENIYA~--- INFORMATICS AND APPLICATIONS\ \ \ 2016\
\ \ volume~10\ \ \ issue\ 4}
}%
 \def\rightfootline{\small{INFORMATIKA I EE PRIMENENIYA~---
INFORMATICS AND APPLICATIONS\ \ \ 2016\ \ \ volume~10\ \ \ issue\ 4
\hfill \textbf{\thepage}}}

\def\leftkol{2016 AUTHOR INDEX} % ENGLISH ABSTRACTS}

\def\rightkol{2016 AUTHOR INDEX} %ENGLISH ABSTRACTS}


{\tabcolsep=2.83pt
\begin{tabular}{p{382pt}cc}
&\textbf{Issue} & \textbf{Page}\\[6pt]
\Avtors{Kadaner~A.\,I.} see~Chertok~A.\,V.&&\\[.255pt]
\Avtors{Kalinichenko~L.\,A., Volnova~A.\,A., Gordov~E.\,P.,
Kiselyova~N.\,N., Kovaleva~D.\,A., Malkov~O.\,Yu., Okladnikov~I.\,G.,
Podkolodnyy~N.\,L., Pozanenko~A.\,S., Ponomareva~N.\,V.,
Stupnikov~S.\,A.,} \textbf{and Fazliev~A.\,Z.} Data access challenges for data
intensive\linebreak
\\[-12pt]
\hspace*{23pt}research in Russia&1& 2--22\\[.255pt]
\Avtors{Karasikov~M.\,E.\ and Strijov~V.\,V.} Feature-based
time-series classification&4&121--131\\[.255pt]
\Avtors{Khazeeva~G.\,T.} see~Chertok~A.\,V.&&\\[.255pt]
\Avtors{Khokhlov~Yu.\,S.} Multivariate fractional Levy motion and its
applications&2&\hphantom{1}98--106\\[.255pt]
\Avtors{Kirikov~I.\,A., Kolesnikov~A.\,V., Listopad~S.\,V., and
Rumovskaya~S.\,B.} Fine-grained hybrid\linebreak
\\[-12pt]
\hspace*{23pt}intelligent systems. Part 2:
Bidirectional hybridization&1&\hphantom{1}96--105\\[.255pt]
\Avtors{Kirikov~I.\,A., Kolesnikov~A.\,V., Listopad~S.\,V., and
Rumovskaya~S.\,B.} ``Virtual council''~---\linebreak
\\[-12pt]
\hspace*{23pt}source environment
supporting complex diagnostic decision making&3&81--90\\[.255pt]
\Avtors{Kiselyova~N.\,N.} see~Kalinichenko~L.\,A.&&\\[.255pt]
\Avtors{Kolesnikov A.\,V., Listopad~S.\,V., Rumovskaya~S.\,B., and
Danishevsky~V.\,I.} Informal axiomatic\linebreak
\\[-12pt]
\hspace*{23pt}theory of~the~role visual models&4&114--120\\[.255pt]
\Avtors{Kolesnikov~A.\,V.} see~Kirikov~I.\,A.&&\\[.255pt]
\Avtors{Kolesnikov~A.\,V.} see~Kirikov~I.\,A.&&\\[.255pt]
\Avtors{Kolin~K.\,K.} Humanitarian aspects of information
security&3&111--121\\[.255pt]
\Avtors{Konovalov~M.\,G.\ and Razumchik~R.\,V.} Dispatching
to~two parallel nonobservable queues using\linebreak
\\[-12pt]
\hspace*{23pt}only static
information&4&57--67\\[.255pt]
\Avtors{Korchagin~A.\,Yu.} see~Korolev~V.\,Yu.&&\\[.255pt]
\Avtors{Korchagin~A.\,Yu.} see~Korolev~V.\,Yu.&&\\[.255pt]
\Avtors{Korepanov~E.\,R.} see~Sinitsyn~I.\,N.&&\\[.255pt]
\Avtors{Korepanov~E.\,R.} see~Sinitsyn~I.\,N.&&\\[.255pt]
\Avtors{Korolev~V.\,Yu., Korchagin~A.\,Yu., and Zeifman~A.\,I.} The
Poisson theorem for Bernoulli trials\linebreak
\\[-12pt]
\hspace*{23pt}with~a~random probability
of~success and~a~discrete analog of~the~Weibull distribution&4&11--20\\[.255pt]
\Avtors{Korolev~V.\,Yu., Zeifman~A.\,I., and Korchagin~A.\,Yu.}
Asymmetric Linnik distributions as~limit\linebreak
\\[-12pt]
\hspace*{23pt}laws for~random sums
of~independent random variables with~finite variances&4&21--33\\[.255pt]
\Avtors{Koucheryavy~E.\,A.} see~Ometov~A.\,Ya.&&\\[.255pt]
\Avtors{Kovaleva~D.\,A.} see~Kalinichenko~L.\,A.&&\\[.255pt]
\Avtors{Kovalyov~S.\,P.} Metaprogramming to increase
manufacturability of large-scale software-\linebreak
\\[-12pt]
\hspace*{23pt}intensive systems&1&56--66\\[.255pt]
\Avtors{Krivenko~M.\,P.} Significance tests of feature selection for
classification&3&32--40\\[.255pt]
\Avtors{Kruzhkov~M.\,G.} see~Zalizniak~Anna~A.&&\\[.255pt]
\Avtors{Kruzhkov~M.\,G.} see~Zatsman~I.\,M.&&\\[.255pt]
\Avtors{Kudryavtsev~A.\,A.} Bayesian queueing and reliability models:
\textit{A~priori} distributions with\linebreak
\\[-12pt]
\hspace*{23pt}compact support&1&67--71\\[.255pt]
\Avtors{Kudryavtsev~A.\,A.} Characteristics dependent on the balance
coefficient in Bayesian models\linebreak
\\[-12pt]
\hspace*{23pt}with compact support of \textit{a priori}
distributions&3&77--80\\[.255pt]
\Avtors{Kudryavtsev~A.\,A.\ and Palionnaia~S.\,I.} Bayesian recurrent
model of reliability growth:\linebreak
\\[-12pt]
\hspace*{23pt}Parabolic distribution of parameters&2&80--83\\[.255pt]
\Avtors{Kudryavtsev~A.\,A.\ and Titova~A.\,I.} Bayesian queuing
and~reliability models: Degenerate-\linebreak
\\[-12pt]
\hspace*{23pt}Weibull case&4&68--71\\[.255pt]
\Avtors{Leontyev~N.\,D.\ and Ushakov~V.\,G.} Analysis of a queueing
system with autoregressive arrivals\linebreak
\\[-12pt]
\hspace*{23pt}and nonpreemptive priority&3&15--22\\[.255pt]
\Avtors{Listopad~S.\,V.} see~Kirikov~I.\,A.&&\\[.255pt]
\Avtors{Listopad~S.\,V.} see~Kirikov~I.\,A.&&\\[.255pt]
\Avtors{Listopad~S.\,V.} see~Kolesnikov A.\,V.&&\\[.255pt]
\Avtors{Malkov~O.\,Yu.} see~Kalinichenko~L.\,A.&&\\[.255pt]
\Avtors{Markov~A.\,S., Monakhov~M.\,M., and
Ulyanov~V.\,V.} Generalized Cornish--Fisher expansions\linebreak
\\[-12pt]
\hspace*{23pt}for distributions of statistics based on samples
of random size&2&84--91\\[.255pt]
\Avtors{Melnikov~A.\,K.\ and Ronzhin~A.\,F.} Generalized statistical
method of~text analysis based\linebreak
\\[-12pt]
\hspace*{23pt}on~calculation of~probability distributions
of~statistical values&4&89--95\\
\end{tabular}
}
\pagebreak

\def\leftfootline{\small{\textbf{\thepage}
\hfill INFORMATIKA I EE PRIMENENIYA~--- INFORMATICS AND APPLICATIONS\ \ \ 2016\
\ \ volume~10\ \ \ issue\ 4}
}%
 \def\rightfootline{\small{INFORMATIKA I EE PRIMENENIYA~---
INFORMATICS AND APPLICATIONS\ \ \ 2016\ \ \ volume~10\ \ \ issue\ 4
\hfill \textbf{\thepage}}}

\def\leftkol{2016 AUTHOR INDEX} % ENGLISH ABSTRACTS}

\def\rightkol{2016 AUTHOR INDEX} %ENGLISH ABSTRACTS}


{\tabcolsep=3pt
\begin{tabular}{p{381pt}cc}
&\textbf{Issue} & \textbf{Page}\\[6pt]
\Avtors{Meykhanadzhyan~L.\,A.} Stationary characteristics of the finite
capacity queueing system with\linebreak
\\[-12pt]
\hspace*{23pt}inverse service order and generalized
probabilistic priority&2&123--131\\[.23pt]
\Avtors{Miller~G.\,B.} see~Borisov~A.\,V.&&\\[.23pt]
\Avtors{Minin~V.\,A., Zatsman~I.\,M., Havanskov~V.\,A., and
Shubnikov~S.\,K.} Intensity of citation of scientific publications in
inventions on information and computer technologies patented\linebreak
\\[-12pt]
\hspace*{23pt}in Russia by domestic and foreign applicants&2&107--122\\[.23pt]
\Avtors{Monakhov~M.\,M.} see~Markov~A.\,S.&&\\[.23pt]
\Avtors{Naumov~V.\,A.\ and Samouylov~K.\,E.} On relationship
between queuing systems with resources\linebreak
\\[-12pt]
\hspace*{23pt}and Erlang networks&3&\hphantom{1}9--14\\[.23pt]
\Avtors{Okladnikov~I.\,G.} see~Kalinichenko~L.\,A.&&\\[.23pt]
\Avtors{Ometov~A.\,Ya., Andreev~S.\,D., Turlikov~A.\,M., and
Koucheryavy~E.\,A.} Performance analysis of\linebreak
\\[-12pt]
\hspace*{23pt}a wireless data
aggregation system with contention for contemporary sensor
networks&3&23--31\\[.23pt]
\Avtors{Palionnaia~S.\,I.} see~Kudryavtsev~A.\,A.&&\\[.23pt]
\Avtors{Podkolodnyy~N.\,L.} see~Kalinichenko~L.\,A.&&\\[.23pt]
\Avtors{Ponomareva~N.\,V.} see~Kalinichenko~L.\,A.&&\\[.23pt]
\Avtors{Popkova~N.\,A.} see~Zatsman~I.\,M.&&\\[.23pt]
\Avtors{Pozanenko~A.\,S.} see~Kalinichenko~L.\,A.&&\\[.23pt]
\Avtors{Razumchik~R.\,V.} see~Konovalov~M.\,G.&&\\[.23pt]
\Avtors{Ronzhin~A.\,F.} see~Melnikov~A.\,K.&&\\[.23pt]
\Avtors{Rumovskaya~S.\,B.} see~Kirikov~I.\,A.&&\\[.23pt]
\Avtors{Rumovskaya~S.\,B.} see~Kirikov~I.\,A.&&\\[.23pt]
\Avtors{Rumovskaya~S.\,B.} see~Kolesnikov A.\,V.&&\\[.23pt]
\Avtors{Samouylov~K.\,E.} see~Gaidamaka~Yu.\,V.&&\\[.23pt]
\Avtors{Samouylov~K.\,E.} see~Naumov~V.\,A.&&\\[.23pt]
\Avtors{Serebryanskii~S.\,M.} see~Tyrsin~A.\,N.&&\\[.23pt]
\Avtors{Seyful-Mulyukov~R.\,B.} see~Callaos~N.\,K.&&\\[.23pt]
\Avtors{Shestakov~O.\,V.} Statistical properties of the denoising method
based on the stabilized hard\linebreak
\\[-12pt]
\hspace*{23pt}thresholding&2&65--69\\[.23pt]
\Avtors{Shestakov~O.\,V.} The strong law of large numbers for the risk
estimate in the problem of\linebreak
\\[-12pt]
\hspace*{23pt}tomographic image reconstruction from
projections with a correlated noise&3&41--45\\[.23pt]
\Avtors{Shestakov~O.\,V.} see~Zakharova~T.\,V.&&\\[.23pt]
\Avtors{Shnurkov~P.\,V., Gorshenin~A.\,K., and Belousov~V.\,V.}
Analytical solution of~the~optimal control\linebreak
\\[-12pt]
\hspace*{23pt}task of~a~semi-Markov
process with~finite set of~states&4&72--88\\[.23pt]
\Avtors{Shnurkov~P.\,V., Zasypko~V.\,V., Belousov~V.\,V., and
Gorshenin~A.\,K.} Development of the algorithm of numerical solution
of the optimal investment control problem\linebreak
\\[-12pt]
\hspace*{23pt}in the closed dynamical model of three-sector economy&1&82--95\\[.23pt]
\Avtors{Shorgin~S.\,Ya.} see~Gaidamaka~Yu.\,V.&&\\[.23pt]
\Avtors{Shorgin~V.\,S.} see~Agalarov~Ya.\,M.&&\\[.23pt]
\Avtors{Shubnikov~S.\,K.} see~Minin~V.\,A.&&\\[.23pt]
\Avtors{Sidorkin~I.\,I.} see~Arkhipov~O.\,P.&&\\[.23pt]
\Avtors{Sinitsyn~I.\,N.} Analytical modeling of processes in stochastic
systems with complex fractional\linebreak
\\[-12pt]
\hspace*{23pt}order Bessel nonlinearities&3&55--65\\[.23pt]
\Avtors{Sinitsyn~I.\,N.} Orthogonal supoptimal filters for nonlinear
stochastic systems on manifolds&1&34--44\\[.23pt]
\Avtors{Sinitsyn~I.\,N.\ and Korepanov~E.\,R.} Normal Pugachev
conditionally-optimal filters and extra-\linebreak
\\[-12pt]
\hspace*{23pt}polators for state linear stochastic systems&2&14--23\\[.23pt]
\Avtors{Sinitsyn~I.\,N.\ and Sinitsyn~V.\,I.} Analytical modeling of
distributions in stochastic systems on\linebreak
\\[-12pt]
\hspace*{23pt}manifolds based on ellipsoidal approximation&1&45--55\\[.23pt]
\Avtors{Sinitsyn~I.\,N., Sinitsyn~V.\,I., and
Korepanov~E.\,R.} Ellipsoidal suboptimal filters for nonlinear\linebreak
\\[-12pt]
\hspace*{23pt}stochastic systems on manifolds&2&24--35\\[.23pt]
\Avtors{Sinitsyn~V.\,I.} see~Sinitsyn~I.\,N.&&\\[.23pt]
\Avtors{Sinitsyn~V.\,I.} see~Sinitsyn~I.\,N.&&\\[.23pt]
\Avtors{Skvortsov~N.\,A.} see~Stupnikov~S.\,A.&&\\[.23pt]
\Avtors{Sokolov~I.\,A.} see~Chertok~A.\,V.&&\\
\end{tabular}
}
\pagebreak

\def\leftfootline{\small{\textbf{\thepage}
\hfill INFORMATIKA I EE PRIMENENIYA~--- INFORMATICS AND APPLICATIONS\ \ \ 2016\
\ \ volume~10\ \ \ issue\ 4}
}%
 \def\rightfootline{\small{INFORMATIKA I EE PRIMENENIYA~---
INFORMATICS AND APPLICATIONS\ \ \ 2016\ \ \ volume~10\ \ \ issue\ 4
\hfill \textbf{\thepage}}}

\def\leftkol{2016 AUTHOR INDEX} % ENGLISH ABSTRACTS}

\def\rightkol{2016 AUTHOR INDEX} %ENGLISH ABSTRACTS}


{\tabcolsep=3pt
\begin{tabular}{p{382pt}cc}
&\textbf{Issue} & \textbf{Page}\\[6pt]
\Avtors{Sopin~E.\,S.} see~Gaidamaka~Yu.\,V.&&\\
\Avtors{Strijov~V.\,V.} see~Goncharov~A.\,V.&&\\
\Avtors{Strijov~V.\,V.} see~Isachenko~R.\,V.&&\\
\Avtors{Strijov~V.\,V.} see~Karasikov~M.\,E.&&\\
\Avtors{Stupnikov~S.\,A., Briukhov~D.\,O., and Skvortsov~N.\,A.}
Co-lending systemic risk analysis over\linebreak
\\[-12pt]
\hspace*{23pt}heterogeneous data collections&1&23--33\\
\Avtors{Stupnikov~S.\,A.} see~Kalinichenko~L.\,A.&&\\
\Avtors{Suchkov~A.\,P.} see~Zatsarinny~A.\,A.&&\\
\Avtors{Timonina~E.\,E.} see~Grusho~A.\,A.&&\\
\Avtors{Titova~A.\,I.} see~Kudryavtsev~A.\,A.&&\\
\Avtors{Turlikov~A.\,M.} see~Ometov~A.\,Ya.&&\\
\Avtors{Tyrsin~A.\,N.\ and Serebryanskii~S.\,M.} Recognition of
dependences on the basis of inverse\linebreak
\\[-12pt]
\hspace*{23pt}mapping&2&58--64\\
\Avtors{Ulyanov~V.\,V.} see~Markov~A.\,S.&&\\
\Avtors{Ushakov~V.\,G.} Queueing system with working vacations and
hyperexponential input stream&2&92--97\\
\Avtors{Ushakov~V.\,G.} see~Leontyev~N.\,D.&&\\
\Avtors{Volnova~A.\,A.} see~Kalinichenko~L.\,A.&&\\
\Avtors{Yakovlev~O.\,A.\ and Gasilov~A.\,V.} Speeded-up stereo
matching using geodesic support weights&3&\hphantom{1}98--104\\
\Avtors{Zabezhailo~M.\,I.} see~Grusho~A.\,A.&&\\
\Avtors{Zabezhailo~M.\,I.} see~Grusho~A.\,A.&&\\
\Avtors{Zakharova~T.\,V.\ and Shestakov~O.\,V.} Precision analysis of
wavelet processing of aerodynamic\linebreak
\\[-12pt]
\hspace*{23pt}flow patterns&3&46--54\\
\Avtors{Zalizniak~Anna~A.\ and Kruzhkov~M.\,G.} Database
of~Russian impersonal verbal constructions&4&132--141\\
\Avtors{Zasypko~V.\,V.} see~Shnurkov~P.\,V.&&\\
\Avtors{Zatsarinny~A.\,A.\ and Suchkov~A.\,P.} Systems engineering
approaches to~the~establishment of\linebreak
\\[-12pt]
\hspace*{23pt}a~system for~decision support based
on~situational analysis&4&105--113\\
\Avtors{Zatsarinny~A.\,A.} see~Grusho~A.\,A.&&\\
\Avtors{Zatsman~I.\,M., Inkova~O.\,Yu., Kruzhkov~M.\,G., and
Popkova~N.\,A.} Representation of cross-\linebreak
\\[-12pt]
\hspace*{23pt}lingual knowledge about
connectors in supracorpora databases&1&106--118\\
\Avtors{Zatsman~I.\,M.} see~Minin~V.\,A.&&\\
\Avtors{Zeifman~A.\,I.} see~Korolev~V.\,Yu.&&\\
\Avtors{Zeifman~A.\,I.} see~Korolev~V.\,Yu.&&\\
\end{tabular}
}

%\thispagestyle{myheadings}
\def\leftfootline{\small{\textbf{\thepage}
\hfill INFORMATIKA I EE PRIMENENIYA~--- INFORMATICS AND APPLICATIONS\ \ \ 2016\
\ \ volume~10\ \ \ issue\ 4}
}%
 \def\rightfootline{\small{INFORMATIKA I EE PRIMENENIYA~---
INFORMATICS AND APPLICATIONS\ \ \ 2016\ \ \ volume~10\ \ \ issue\ 4
\hfill \textbf{\thepage}}}

 \label{end\stat}

\newpage

%\documentclass[10pt]{book}
\usepackage[utf8]{inputenc}

\usepackage{latexsym,amssymb,amsfonts,amsmath,indentfirst,shapepar,%fleqn,%
picinpar,shadow,floatflt,enumerate,multicol,ipi}

\input{epsf}

%\nofiles

\usepackage{acad}
\usepackage{courier}
\usepackage{decor}
\usepackage{newton}
\usepackage{pragmatica}
\usepackage{zapfchan}
\usepackage{petrotex}
\usepackage{bm}                     % полужирные греческие буквы
\usepackage{upgreek}                % прямые греческие буквы
%\usepackage{verbatim}

\renewcommand{\bottomfraction}{0.99}
\renewcommand{\topfraction}{0.99}
\renewcommand{\textfraction}{0.01}

%NEW COMMANDS
\renewcommand{\r}{{\rm I\hspace{-0.7mm}\rm R}}

\newcommand{\il}[2]{\int\limits_{#1}^{#2}}%интеграл с пределами #1 и #2

%\pagestyle{myheadings}

\setlength{\textwidth}{167mm}      % 122mm
\setlength{\textheight}{658pt}
%\setlength{\textheight}{635.6pt}
\setlength{\columnsep}{4.5mm}

\setcounter{secnumdepth}{4}

%\addtolength{\headheight}{2pt}
%\addtolength{\headsep}{-2mm}

%\addtolength{\topmargin}{-20mm}  % for printing


\hoffset=-30mm  % From Yap
%\hoffset=-20mm  % From Acrobat

%\voffset=0mm % From Yap
%\voffset=-15mm   % From Acrobat

\addtolength{\evensidemargin}{-9.5mm} % for printing
\addtolength{\oddsidemargin}{9.5mm}  % for printing

\renewcommand\labelitemi{$\bullet$}

\begin{document}
\Rus

\nwt
%\ptb

%\vspace*{-12pt}

\begin{center}

{\prgsh\LARGE
ОБЪЯВЛЕНИЯ О КОНФЕРЕНЦИЯХ}

\end{center}
%\hrule

\vspace*{6pt}
%\begin{center}
%\mbox{%
%\epsfxsize=167mm
%\epsfbox{recl-dia-b.eps}
%}
%\end{center}
\begin{flushright}
{\prg http://www.tvp.ru/conferen/20091001\_1.htm}
\end{flushright}

%\vspace*{6pt}

\begin{center}\prg
\Large
X Всероссийский симпозиум\\ по прикладной и промышленной математике\\
(осенняя открытая сессия)

\end{center}

\begin{center}\prg
%Очередная 15-я международная конференция <<Диалог>> состоится c
1--8 октября 2009~г., Сочи--Дагомыс
\end{center}


%Всероссийские симпозиумы по прикладной и промышленной математике проводятся ежегодно с 2000~года. 

\smallskip

{\centering Симпозиум проводится по следующим направлениям:

\begin{multicols}{2}
\begin{itemize}
\item Безопасность компьютерных систем\item 
Геометрическая нелинейная оптика\item 
Инженерно-технологическая математика\item 
Информационные технологии и задачи связи\item 
Квантовые вычисления\item 
Математические методы биологических и экологических систем\item 
Математические модели в жидких кристаллах\item 
Математические методы в педагогических исследованиях\item 
Математические модели в теории оболочек\item 
Математическое моделирование процессов рассеяния примесей в турбулентной атмосфере\item 
Математическое моделирование свойств материалов и конструкций\item 
Математическое образование\item  
Медицина\item  
Метод конечных элементов\item 
Механика жидкости и газа\item  
Механика природных процессов\item  
Механика разрушения\item 
Модели горения и взрыва\item  
Нанотехнологии: математические модели\item 
Науки о Земле, геология, геофизика\item 
 Неклассические задачи для уравнений математической физики
  \item Нелинейное моделирование и управление\item 
  Обработка данных, анализ и обработка изображений\item 
  Прикладная вероятность и статистика\item 
  Прикладная геометрия\item 
  Обработка и распознавание образов\item 
  Прикладная дискретная математика\item 
  Обработка и защита информации\item 
  Системы поддержки принятия решений для регионального управления\item 
  Социология\item 
  Психология\item 
  Специальные функции и ортогональные многочлены\item 
  Супер-, нейро-, биокомпьютеры\item
  Эволюционные и мембранные вычисления\item
  Теория управления и системные исследования\item
Процессы принятия решений\item 
Тепло- и массоперенос\item 
Физика океана и атмосферы\item 
Фракталы и масштабный эффект\item 
Экономика, страховая и финансовая матема\-тика\item 
Энергетика и передача энергии\item
Юриспруденция\item 
Криминалистика
\end{itemize}
\end{multicols}
}

\smallskip

{\centering В программу симпозиума входят также:
\begin{itemize}
\item учредительное собрание общероссийского <<Общества прикладной и промышленной математики>> (ОППМ);\\[-13pt] 
\item минисимпозиумы по предложенным позднее темам; \\[-13pt]
\item круглые столы по продвижению современных фундаментальных математических методов в различные сферы науки и технологий, по междисциплинарному сотрудничеству; \\[-13pt]
\item выставка-продажа научных изданий, демонстрация программного обеспечения.
\end{itemize}

{\large Организаторы симпозиума: }
%\noindent
%\begin{tabular}{p{7mm}p{420pt}}
%\hspace*{7mm}
\begin{itemize}
\item Управление по науке и образованию г.~Сочи
\item Сочинский государственный университет туризма и курортного дела
\item Академия криптографии Российской Федерации
\item Институт проблем информатики Российской академии наук
\item Редакции журналов <<Информатика и её применения>>, <<Обозрение прикладной и промышленной математики>> (ОПиПМ), <<Прикладная информатика>>, <<Теория вероятностей и ее применения>> (Научное издательство <<ТВП>>) 
\item Экономический факультет Санкт-Петербургского государственного университета (ЭФ СПбГУ)
\end{itemize}
%\end{tabular}

%\vspace*{6pt}
\bigskip

{\large Организационный комитет:}
\vspace*{6pt}

Академик Ю.\,В.~Прохоров (председатель)
\vspace*{6pt}

\tabcolsep=15pt
\begin{tabular}{p{60mm}p{60mm}}
академик В.\,А.~Бабешко\newline
академик А.\,А.~Боровков\newline 
академик С.\,С.~Григорян\newline 
академик И.\,А.~Ибрагимов\newline 
академик В.\,И.~Колесников\newline 
академик А.\,Б.~Куржанский\newline 
академик В.\,П.~Маслов\newline
академик И.\,А.~Соколов\newline
академик В.\,П.~Шорин\newline 
член-корр.\ РАН А.\,Б.~Жижченко\newline
член-корр.\ РАН С.\,В.~Кисляков\newline 
член-корр.\ РАН В.\,В.~Русанов\newline 
член-корр.\ РАН Б.\,А.~Севастьянов 
&
член-корр.\ РАН В.\,А.~Сойфер\newline
член-корр.\ РАН А.\,Н.~Ширяев\newline
И.\,П.~Бойко\newline 
А.\,А.~Емельянов\newline
А.\,М.~Зубков\newline 
В.\,Ф.~Колчин\newline
 А.\,С.~Максимов\newline
О.\,Н.~Медведева\newline 
Г.\,М.~Романова\newline 
В.\,В.~Сапожников\newline 
А.\,Р.~Симонян\newline 
В.\,И.~Хохлов\newline
С.\,Я.~Шоргин
\end{tabular}


}
\end{document}


%\noindent
%\begin{tabular}{p{7mm}p{420pt}}
%\hspace*{7mm}
%&\begin{itemize}
%\item Филологический факультет МГУ
%\item Институт лингвистики РГГУ
%\item Институт проблем информатики РАН
%\item Институт проблем передачи информации РАН
%\item Российский НИИ искусственного интеллекта
%\item Яндекс (Москва)
%\end{itemize}
%\end{tabular}

%Конференция проводится по следующим направлениям, сочетающим теоретические
%исследования и приложения:
%\vspace*{-6pt}

%\noindent
%\begin{tabular}{p{7mm}p{420pt}}
%\hspace*{7mm}&
%\begin{itemize}
%\item Лингвистическая семантика и семантический анализ
%\item Формальные модели языка и их применение
%\item Теоретическая и компьютерная лексикография
%\item Создание и применение компьютерных лексических ресурсов
%\item Корпусная лингвистика. Создание, применение, оценка корпусов
%\item Интернет как лингвистический ресурс. Лингвистические технологии в
%интернете
%\item Извлечение знаний из текстов
%\item Модели общения. Коммуникация, диалог и речевой акт
%\item Анализ и синтез речи
%\item Компьютерный анализ документов: реферирование, классификация,
%поиск
%\item Машинный перевод
%\item Вопросно-ответные системы
%\end{itemize}
%\end{tabular}

%Сайт конференции: {\prg http://www.dialog-21.ru/ }

%\newpage

\begin{center}

{\prgsh\LARGE
ОБЪЯВЛЕНИЯ О КОНФЕРЕНЦИЯХ}

\end{center}
%\hrule

\vspace*{12pt}

%\hrule
\begin{center}
\mbox{%
\epsfxsize=167mm
\epsfbox{recl-RL-b.eps}
}
\end{center}
\begin{flushright}
{\prg http://rcdl2009.krc.karelia.ru/}
\end{flushright}

%\vspace*{6pt}

{\begin{center}\prg
{\Large
XI Всероссийская научная конференция RCDL 2009\\
Электронные библиотеки: перспективные методы и технологии,\\ электронные
коллекции}
\end{center}}

{\begin{center}\prg
17--21 сентября 2009 г.,
Петрозаводск, Россия
\end{center}}

\vspace*{12pt}

Электронные библиотеки~--- область исследований и разработок, направленных 
на развитие теории и практики обработки, распространения, хранения, поиска 
и анализа цифровых данных различной природы. 

Основная цель серии конференций RCDL заключается в том, чтобы способствовать 
формированию сообщества специалистов России, ведущих исследования и разработки 
в области электронных библиотек. Конференция также способствует изучению 
зарубежного опыта, развитию международного сотрудничества в области электронных 
библиотек. 

Значительное внимание в тематике RCDL уделяется практическим проектам, в 
которых решаются сложные задачи. RCDL придает большое значение исследованиям 
в области создания крупномасштабных электронных библиотек (Very Large Digital 
Libraries~--- VLDL), включая использование сервисных архитектур, архитектур, 
основанных на грид, и обеспечение их качества, развитие техники 
интероперабельности и устойчивости VLDL, а также разработке организационных 
моделей крупных электронных библиотек. Особый интерес представляет применение 
современных научных подходов в контексте высоких нагрузок: сотни тысяч 
пользователей, десятки гигабайт данных, терабайты трафика.


%RCDL'2009~---  одиннадцатая конференция в серии Всероссийских научных
%конференций <<Электронные библиотеки: перспективные методы и технологии,
%электронные коллекции>>.
За 10 лет проведения RCDL в работе конференции приняло участие несколько
сотен ученых из ведущих российских и зарубежных научных центров Австрии,
Германии, Греции, Италии, Новой Зеландии, США, Украины и других стран.

Традиционно совместно с RCDL проводятся Всероссийские научные семинары по оценке
методов текстового поиска РОМИП. В 2009 году планируется совмещение с RCDL
Семинара РОМИП и\linebreak
 Третьей Российской летней школы по информационному поиску RuSSIR'2009,
во время которой ведущие российские и зарубежные ученые прочитают
обзорные лекции по актуальным проблемам развития поиска цифровых данных для
решения фундаментальных и прикладных задач.
\vspace*{6pt}

Организаторы конференции RCDL'2009:
\vspace*{-6pt}

\noindent
\begin{tabular}{p{7mm}p{420pt}}
\hspace*{7mm}&
\begin{itemize}
\item Российская академия наук
\item Российский фонд фундаментальных исследований
\item Карельский научный центр РАН
\item Институт прикладных математических исследований
\item Петрозаводский государственный университет
\item Институт проблем информатики РАН
\item Московская секция АСМ SIGMOD
\end{itemize}
\end{tabular}

%Сайт конференции: {\prg http://rcdl2009.krc.karelia.ru/}

\end{document}

%\include{rekl-1}

%\end{document}

%   \vspace*{-36pt}

\begin{center}
\vspace*{6pt}
\mbox{%
\epsfxsize=79.5mm
\epsfbox{korov-tg.eps}
}
\end{center} 

\vspace*{12pt} %Академик


   \begin{center}
\fbox{\Large\textbf{Академик Сергей Константинович Коровин}}\\[12pt]
\textbf{\large 24.05.1945--7.12.2011}
   \end{center}
   
   %\vspace*{2.5mm}
   
   \vspace*{5mm}
   
   \thispagestyle{empty}

%\

%\vspace*{-12pt}


Редакционная коллегия журнала <<Информатика и её применения>> с глубоким 
прискорбием извещает, что 7 декабря~2011~года на 67-м году жизни скоропостижно 
скончался выдающийся российский ученый в области теории управления сложными 
динамическими системами, член редколлегии журнала <<Информатика и её применения>> 
академик КОРОВИН Сергей Константинович.

Коровин Сергей Константинович окончил факультет радиотехники и 
кибернетики Московского физико-технического института в 1969~г. 
С~1969~г.\ по 1975~г.\ работал в Институте проблем управления АН СССР. 
Здесь же без отрыва от производства учился в аспирантуре (1971--1974), 
защитил диссертацию на степень кандидата технических наук по теме 
<<Алгоритмы оптимизации на скользящих режимах>> (1975). С~1975 по 2011~гг.\ 
работал в Институте системного анализа Российской академии наук 
в должностях от ведущего инженера 
до главного научного сотрудника, заведующего лабораторией. В~1985~г.\ защитил 
диссертацию на степень доктора технических наук по теме <<Системы управления 
с автоматически регулируемыми связями>>, в 1990~г.\ ему присвоено ученое звание профессора.

С 1989~г.\ С.\,К.~Коровин работал в МГУ им.\ М.\,В.~Ломоносова, с 1996~г.\ 
являлся профессором кафедры нелинейных динамических систем и процессов 
управ\-ле\-ния факультета вычислительной математики и кибернетики. 

В 1994~г.\ избран членом-кор\-рес\-пон\-ден\-том РАН, в 2000~г.~--- действительным членом 
РАН (2000). Лауреат Государственной премии РФ (1994), премии Совета Министров СССР (1981), 
премии Правительства РФ (2009), премии РАН им.\ А.\,А.~Андронова (2000), Ломоносовской 
премии МГУ I~степени в области науки (2002). С.\,К.~Коровин~--- автор 260~научных работ, в том 
чис\-ле 15~книг, 50~авторских свидетельств. 

Сергей Константинович~Коровин являлся членом редколлегии журнала <<Информатика и её применения>> с 
момента основания журнала и принимал активное участие в формировании редакционной 
политики журнала. 


%\end{document}

%\include{IPPM-25}

\vspace*{-60pt} %{ %small
{ %\baselineskip=9.1pt
\section*{Правила подготовки рукописей  для публикации в журнале
<<Информатика и её применения>>}

\thispagestyle{empty}

\noindent
\begin{enumerate}[1.]
\item В журнале печатаются статьи, содержащие результаты, ранее не опубликованные и 
не предназначенные к одновременной публикации в других изданиях. 
 
Публикация не должна нарушать закон об авторских правах. 
 
Направляя рукопись в редакцию, авторы сохраняют все права собственников данной 
рукописи и при этом передают учредителям и редколлегии неисключительные права на 
издание статьи на русском языке (или на языке статьи, если он отличен от русского) и на 
ее распространение в России и за рубежом. Авторы должны представить в редакцию 
письмо в следующей форме: 


{\bfseries\textit{Соглашение о передаче права на публикацию:}}

<<\textit{Мы, нижеподписавшиеся, авторы рукописи} <<\ldots>>, 
\textit{передаем учредителям и редколлегии журнала <<Информатика и её 
применения>> неисключительное право опубликовать данную рукопись 
статьи на русском языке как в печатной, так и в электронной версиях 
журнала. Мы подтверждаем, что данная публикация не нарушает 
авторского права других лиц или организаций. }
 
\textit{Подписи авторов: (ф. и. о., дата, адрес)}>>.  
 
Это соглашение может быть представлено в бумажном виде или в виде 
отсканированной копии (с подписями авторов).  
 
Редколлегия вправе запросить у авторов экспертное заключение о 
возможности публикации пред\-став\-лен\-ной статьи в открытой печати. 

\item К статье прилагаются данные автора (авторов) (см.\ п.~8). При наличии нескольких 
авторов указывается фамилия автора, ответственного за переписку с редакцией. 

\item Редакция журнала осуществляет экспертизу присланных статей в соответствии с 
принятой в журнале процедурой рецензирования.

Возвращение рукописи на доработку не означает ее принятия к печати.  

Доработанный вариант с ответом на замечания рецензента необходимо прислать в 
редакцию. 

\item Решение редколлегии о публикации статьи или ее отклонении сообщается авторам.  
Редколлегия может также направить авторам текст рецензии на их статью. Дискуссия по 
поводу отклоненных статей не ведется. 

\item Редактура статей высылается авторам для просмотра. Замечания к редактуре должны 
быть присланы авторами в кратчайшие сроки. 

\item Рукопись предоставляется в электронном виде в форматах MS WORD (.doc или 
.docx) или \LaTeX\ (.tex), дополнительно~--- в формате .pdf, на дискете, лазерном диске 
или электронной почтой. Предоставление бумажной рукописи необязательно.

\item При подготовке рукописи в MS Word рекомендуется использовать следующие 
настройки.

Параметры страницы:  
формат~--- А4; ориентация~--- книжная; поля (см): внутри~--- 2,5, снаружи~--- 1,5, 
сверху~--- 2, снизу~--- 2, от края до нижнего колонтитула~--- 1,3.  

Основной текст: стиль~--- <<Обычный>>, шрифт ~--- Times New Roman, размер~--- 
14~пунктов, абзацный отступ~--- 0,5~см, 1,5~интервала, выравнивание~--- по ширине.  
 
Рекомендуемый объем рукописи~--- не свыше 20 страниц указанного формата.  

Сокращения слов, помимо стандартных, не допускаются. Допускается минимальное 
количество аббревиатур. 

Все страницы рукописи нумеруются. 

Шаблоны примеров оформления, представлены в Интернете: 

{\sf http://www.ipiran.ru/journal/template.doc}.

\item Статья должна содержать следующую информацию на {\bfseries\textit{русском и 
английском языках:}} 
\begin{itemize}
\item название статьи; 
\item Ф.И.О.\ авторов, на английском можно только имя и фамилию;
\item место работы, с указанием города и страны и электронного адреса каждого 
автора; 

\pagebreak  

\thispagestyle{empty}


\vspace*{-36pt}


\item сведения об авторах, в соответствии с форматом, образцы которого 
представлены на страницах: 

{\sf http://www.ipiran.ru/journal/issues/2013\_07\_01/authors.asp}  и

 {\sf 
http://www.ipiran.ru/journal/issues/2013\_07\_01\_eng/authors.asp};

\item аннотация (не менее 100 слов на каждом из языков). Аннотация~--- это 
краткое резюме работы, которое может публиковаться отдельно. Она является 
основным источником информации в информационных системах и базах данных; 
Английская аннотация должна быть оригинальной, может не быть дословным 
переводом русского текста и должна быть написана хорошим английским языком.
\item ключевые слова, желательно из принятых в мировой научно-технической 
литературе тематических тезаурусов. Предложения не могут быть ключевыми 
словами.
\end{itemize}

\item Литература. По включенным в список литературы работам на русском языке 
информация в списке представляется как в кириллице, так и с использованием латинской 
транслитерации, а по работам, написанным латиницей,~--- на языке оригинала. 

Ссылки на литературу в тексте статьи нумеруются (в квадратных скобках) и 
располагаются в списке литературы в порядке упоминания.  

В списке литературы не должно быть позиций, на которые нет ссылки в тексте статьи.  

\item Присланные в редакцию материалы авторам не возвращаются. 

\item При отправке файлов по электронной почте просим придерживаться следующих 
правил: 
\begin{itemize}
\item указывать в поле subject (тема) название журнала и фамилию автора; 
\item использовать attach (присоединение); 
\item в состав электронной версии статьи должны входить: файл, содержащий 
текст статьи, и файл(ы), содержащий(е) иллюстрации. 
\end{itemize}
\item  Журнал <<Информатика и её применения>> является некоммерческим изданием. 
Плата за публикацию не взимается, гонорар авторам не выплачивается. 
\end{enumerate}


\thispagestyle{empty}

\noindent
\textbf{Адрес редакции:} Москва 119333, ул.~Вавилова, д.~44, корп.~2, ИПИ РАН

\noindent
\hphantom{\textbf{Адрес редакции: }}Тел.: +7 (499) 135-86-92\ \ Факс:  +7 (495) 930-45-05  

\noindent
\hphantom{\textbf{Адрес редакции: }}e-mail:   {\sf rust@ipiran.ru} (Сейфуль-Мулюков Рустем Бадриевич)
}
%}


%\vfill
%\begin{center}


%Технический редактор Л. Кокушкина\\
%%Выпускающий редактор Т. Торжкова\\
%Художественный редактор М. Седакова\\
%Сдано в набор 03.04.13. Подписано в печать 14.06.13. Формат 60 х 84 / 8\\
%Бумага офсетная. Печать цифровая. Усл.-печ.\ л.\ ??,?. Уч.-изд.\ л.\ ??,?. Тираж 100 экз.\\
%\ \\
%Заказ №\,\\
%\ \\
%Издательство <<ТОРУС ПРЕСС>>, Москва 121614, ул. Крылатская, 29-1-43\\
%torus@torus-press.ru; http://www.torus-press.ru\\
%\ \\
%Отпечатано в Академиздатцентре <<Наука>> РАН с готовых файлов\\
%Москва 121099, Шубинский пер., д.~6\\
%\end{center}
\vspace*{-60pt} %{ %small
{ %\baselineskip=9.1pt
\section*{Requirements for manuscripts submitted to Journal ``Informatics and~Applications''}

\thispagestyle{empty}



\noindent
\begin{enumerate}[1.]
\item The Journal publishes original articles which have not been published before and are not 
intended for publication in other editions. An article submitted to the Journal must not violate 
the Copyright law. Sending the manuscript to the Editorial Board, the authors retain all rights 
of the owners of the manuscript and transfer the nonexclusive rights to publish the article in Russian 
(or the language of the article, if not Russian) and its distribution in Russia and abroad to the Founders 
and the Editorial Board. Authors should submit a letter to the Editorial Board in the following form:


{\bfseries\textit{Agreement on the transfer of rights to publish:}}

<<\textit{We, the undersigned authors of the manuscript ``\ldots,'' pass to the 
Founder and the Editorial Board of the Journal ``Informatics and Applications'' 
the nonexclusive right to publish the manuscript of the article in Russian (or English) 
 in both print and electronic versions of the Journal. We affirm that 
this publication does not violate the Copyright of other persons or organizations. }
 
\textit{Author(s) signature(s): (name(s), address(es), date)}>>.  
 
This agreement should be submitted in paper form or in the form of a scanned copy (signed by the authors). 

The Editorial Board has the right to request from the authors an official expert conclusion 
that the submitted article does not have secret data prohibited for publication. 


\item A submitted article should be attached with \textbf{the data on the author(s)} (see p.~ 8). 
If there are several authors, the contact person should be indicated who is responsible 
for correspondence with the Editorial Board. 

\item The Editorial Board of the Journal examines the article according to the established reviewing procedure. 
If authors receive their article for correction after reviewing it does not mean that the article is approved 
to be published. The corrected article should be sent to the Editorial Board for the subsequent review and approval.

\item The decision on the article publication or its rejection is communicated to the authors. The Editorial Board may 
also send the reviews on the submitted articles to the authors. Any discussion upon the rejected articles is not possible.

\item The edited articles will be sent to the authors for proofread. The comments of the authors to the edited 
text of the article should be sent to the Editorial Board as soon as possible. 

\item The manuscript of the article should be presented electronically in the MS WORD (.doc or .docx) 
or \LaTeX\ (.tex) formats and, additionally, in the .pdf format. All documents may be sent by e-mail or 
on a CD or diskette. A~hard copy submission is not necessary.

\item The recommended typesetting instructions for manuscript.  

Pages parameters: format A4, portrait orientation, document margins (cm): 
left~--- 2.5, right~--- 1.5, above~--- 2.0, below~--- 2.0, footer~1.3. 

Text: font~--- Times New Roman, font size~--- 14, paragraph~--- 0.5, line spacing~--- 1.5, justified alignment. 
 
The recommended manuscript size: no more than 20 pages of the specified format. 

Word abbreviations are not allowed except the standard ones. 

Abbreviations should be minimal. All pages of the manuscript should be numbered.

The templates for the manuscript typesetting are presented on site: 


{\sf http://www.ipiran.ru/journal/template.doc}.

\item Articles should enclose data both in \textbf{Russian and English}: 
\begin{itemize}
\item title;
\item  author(s) name(s) and surname(s); 
\item affiliation~--- organization, its address with ZIP code, city, country, and e-mail address; 
\item   data on authors according to the format (see site):

{\sf http://www.ipiran.ru/journal/issues/2013\_07\_01/authors.asp}  and

{\sf 
http://www.ipiran.ru/journal/issues/2013\_07\_01\_eng/authors.asp};

\item abstract (not less than 100~words) both in Russian and in English. Abstract is a short 
summary of the article that can be published separately from the article. The abstract is 
the main source of information on the article and it could be included in leading information 
systems and data bases. The abstract in English has to be an original text and should not be 
an exact translation of the Russian one. Good English is required;

\pagebreak

\thispagestyle{empty}

\vspace*{-36pt}


\item indexing is performed on the basis of key words. The use of key words from the internationally 
accepted thematic Thesauri is recommended. 

Important! Key words must not be sentences.
\end{itemize}

\item References. Russian references have to be presented both in Cyrillic and Latin transliteration. 

References in Latin transcript are presented in original language.

References in the text are numbered according to the order of the appearance 
and the number is placed in square bracket. References absent in the text should not 
be included into the list of references.
 

\item Manuscripts and additional materials are not returned to authors by the Editorial Board. 

\item Submissions of files by e-mail must include: 
\begin{itemize}
\item  the journal title and author(s) name(s) in the ``Subject'' field; 
\item  an article and additional materials have to be attached using the ``attach'' function;
\item  an electronic version of the article should contain the file with the text and (a) 
separate file(s) with figures. 
\end{itemize}
\item  ``Informatics and Applications'' Journal is not a profit publication. There are no charges 
for the authors as well as there are no royalties.
\end{enumerate}

\noindent
\textbf{Editorial Board address:}\  119333, IPIRAN, Vavilova St., 44, block 2, Moscow, Russia

\noindent
\hphantom{\textbf{Editorial Board address:}\ }Ph.: +7(499) 135 8692,\ \  Fax: +7 (495) 930 4505

\noindent
\hphantom{\textbf{Editorial Board address:}\ }e-mail: {\sf rust@ipiran.ru} 
(To Prof.\ Rustem Seyfoul-Mulyukov)


\vfill
\begin{center}


Технический редактор Л. Кокушкина\\
%Выпускающий редактор Т. Торжкова\\
Художественный редактор М. Седакова\\
Сдано в набор 01.07.13. Подписано в печать 19.09.13. Формат 60 х 84 / 8\\
Бумага офсетная. Печать цифровая. Усл.-печ.\ л.\ 17,25. Уч.-изд.\ л.\ 15,0. Тираж 100 экз.\\
\ \\
Заказ №\,3951\\
\ \\
Издательство <<ТОРУС ПРЕСС>>, Москва 121614, ул. Крылатская, 29-1-43\\
torus@torus-press.ru; http://www.torus-press.ru\\
\ \\
Отпечатано в Академиздатцентре <<Наука>> РАН с готовых файлов\\
Москва 121099, Шубинский пер., д.~6\\
\end{center}

%\vspace*{-60pt} {\small
{\baselineskip=9.1pt
\section*{Правила подготовки рукописей статей для публикации в журнале
<<Информатика и её применения>>}

\thispagestyle{empty}

 Журнал <<Информатика и её применения>> публикует
теоретические, обзорные и дискуссионные статьи, посвященные научным
исследованиям и разработкам в области информатики и ее приложений. Журнал
издается на русском языке. По специальному решению редколлегии отдельные статьи,
в виде исключения, могут печататься на английском языке.
Тематика журнала охватывает следующие направления:
\begin{itemize}
\item теоретические основы информатики; %\\[-13.5pt]
\item математические методы исследования сложных систем и процессов; %\\[-13.5pt]
\item информационные системы и сети; %\\[-13.5pt]
\item информационные технологии; %\\[-13.5pt]
\item архитектура и программное
обеспечение вычислительных комплексов и сетей.
\end{itemize}
\begin{enumerate}
\item В журнале печатаются результаты, ранее не
опубликованные и не предназначенные к одновременной публикации в других
изданиях. Публикация не должна нарушать закон об авторских правах. Направляя
свою рукопись в редакцию, авторы автоматически передают учредителям и
редколлегии неисключительные права на издание данной статьи на русском языке и
на ее распространение в России и за рубежом. При этом за авторами сохраняются
все права как собственников данной рукописи. В связи с этим авторами должно
быть представлено в редакцию письмо в следующей форме:
Соглашение о передаче права на публикацию:

\textit{<<Мы, нижеподписавшиеся, авторы рукописи <<$\qquad\qquad$>>, передаем
учредителям и редколлегии журнала <<Информатика и её применения>>
неисключительное право опубликовать данную рукопись статьи на русском языке как
в печатной, так и в электронной версиях журнала. Мы подтверждаем, что данная
публикация не нарушает авторского права других лиц или организаций. Подписи
авторов: (ф.\,и.\,о., дата, адрес)>>.}

Указанное соглашение может быть представлено 
как в бумажном виде, так и в виде отсканированной копии (с подписями авторов).


Редколлегия вправе запросить у авторов экспертное заключение о возможности
опубликования представленной статьи в открытой печати. %\\[-13.5pt]
\item Статья
подписывается всеми авторами. На отдельном листе представляются данные автора
(или всех авторов): фамилия, полные имя и отчество, телефон, факс, e-mail,
почтовый адрес. Если работа выполнена несколькими авторами, указывается фамилия
одного из них, ответственного за переписку с редакцией. %\\[-13.5pt]
\item Редакция журнала
осуществляет самостоятельную экспертизу присланных статей. Возвращение рукописи
на доработку не означает, что статья уже принята к печати. Доработанный вариант
с ответом на замечания рецензента необходимо прислать в редакцию. %\\[-13.5pt]
\item Решение
редакционной коллегии о принятии статьи к печати или ее отклонении сообщается
авторам. Редколлегия не обязуется направлять рецензию авторам отклоненной
статьи; дискуссия с авторами по поводу отклоненных статей не ведется. %\\[-13.5pt]
\item Корректура статей высылается авторам для просмотра. Редакция
просит авторов присылать свои замечания в кратчайшие сроки. %\\[-13.5pt]
\item При
подготовке рукописи в MS Word рекомендуется использовать следующие настройки.
Параметры страницы: формат~--- А4; ориентация~--- книжная; поля (см): внутри~---
2,5, снаружи~--- 1,5, сверху~--- 2, снизу~--- 2, от края до нижнего
колонтитула~--- 1,3. Основной текст: стиль~--- <<Обычный>>: шрифт Times New
Roman, размер 14~пунктов, абзацный отступ~--- 0,5~см, 1,5 интервала,
выравнивание~--- по ширине. Рекомендуемый объем рукописи~--- не свыше
25~страниц указанного формата. Ознакомиться с шаблонами, содержащими примеры
оформления, можно по адресу в Интернете:
\textsf{http://www.ipiran.ru/journal/template.doc}.
\item К рукописи, предоставляемой в 2-х
экземплярах, обязательно прилагается электронная версия статьи (как правило, в
форматах MS WORD (.doc) или \LaTeX\ (.tex), а также~--- дополнительно~--- в
формате .pdf) на дискете, лазерном диске или по электронной почте. Сокращения
слов, кроме стандартных, не применяются. Все страницы рукописи должны быть
пронумерованы. %\\[-13.5pt]
\item Статья должна содержать следующую информацию на русском и
английском языках: название, Ф.И.О. авторов, места работы авторов и их
электронные адреса, подробные сведения об авторах, оформленные в соответствии с форматом, 
определяемым файлами {\sf http://www.ipiran.ru/journal/issues/2011\_05\_01/authors.asp} и 
{\sf http://www.ipiran.ru/journal/issues/2011\_01\_eng/authors.asp},
аннотация (не более 100~слов), ключевые слова. Ссылки на
литературу в тексте статьи нумеруются (в квадратных скобках) и располагаются в
порядке их первого упоминания. В~списке литературы не должно быть позиций, на которые нет ссылки в тексте статьи.
Все фамилии авторов, заглавия статей, названия
книг, конференций и~т.\,п.\ даются на языке оригинала, если этот язык
использует кириллический или латинский алфавит. %\\[-13.5pt]
\item Присланные в редакцию материалы авторам не возвращаются.
\item При отправке файлов по электронной
почте просим придерживаться следующих правил:
\begin{itemize}
\item указывать в поле subject (тема) название журнала и фамилию автора; %\\[-13.5pt]
\item использовать attach (присоединение); %\\[-13.5pt]
\item в случае больших объемов информации возможно
использование общеизвестных архиваторов (ZIP, RAR); %\\[-13.5pt]
\item в состав электронной версии статьи должны входить: файл, содержащий текст статьи, и файл(ы),
содержащий(е) иллюстрации. %\\[-13.5pt]
\end{itemize}
\item Журнал <<Информатика и её применения>> является некоммерческим изданием. 
Плата за публикацию с авторов не взимается, гонорар авторам не выплачивается.
\end{enumerate}
\thispagestyle{empty}
\textbf{Адрес редакции:} Москва 119333,
ул.~Вавилова, д.~44, корп.~2, ИПИ РАН\\
\hphantom{\textbf{Адрес редакции:} }Тел.: +7 (499) 135-86-92\ \
Факс:  +7 (495) 930-45-05\ \  E-mail:   rust@ipiran.ru }
}

%\include{ipi-ind}

%\tableofcontents

\end{document}


%\tableofcontents

%\end{document}





%\def\stat{cont}
{%\hrule\par
%\vskip 7pt % 7pt
\raggedleft\Large \bf%\baselineskip=3.2ex
А\,В\,Т\,О\,Р\,С\,К\,И\,Й\ \ У\,К\,А\,З\,А\,Т\,Е\,Л\,Ь\ \ З\,А\ \ 2\,0\,0\,7 г. \vskip 17pt
    \hrule
    \par
\vskip 21pt plus 6pt minus 3pt }

\label{st\stat}

\def\tit{\ }

\def\aut{\ }
\def\auf{\ }

\def\leftkol{\ } % ENGLISH ABSTRACTS}

\def\rightkol{\ } %ENGLISH ABSTRACTS}

\titele{\tit}{\aut}{\auf}{\leftkol}{\rightkol}


\contentsline {chapter}{\ }{Выпуск \quad Стр.} 
\contentsline {section}{\textbf{Батракова Д.\,А., Королев В.\,Ю., Шоргин С.\,Я.}\ \ Новый метод вероятностно-ста\-ти\-сти\-че\-ско\-го анализа информационных потоков в\nobreakspace {}телекоммуникационных сетях}{\qquad 1 \qquad 40} 
\contentsline {section}{\textbf{Борисов А.\,В.}\ \ Байесовское оценивание в системах наблюдения с\nobreakspace {}марковскими скачкообразными процессами: игровой подход}{\qquad 2 \qquad 65}
\contentsline {section}{\textbf{Босов А.\,В., Иванов А.\,В.}\ \ Программная инфраструктура информационного Web-пор\-тала}{\qquad 2 \qquad 50}
\contentsline {section}{\textbf{Захаров В.\,Н., Калиниченко Л.\,А., Соколов И.\,А., Ступников С.\,А.}\ \ Конструирование канонических информационных моделей для интегрированных информационных систем}{\qquad 2 \qquad 15}
\contentsline {section}{\textbf{Захаров В.\,Н., Козмидиади В.\,А.}\ \ Средства обеспечения отказоустойчивости при\-ло\-жений}{\qquad 1 \qquad 14} 
\contentsline {section}{\textbf{Иванов А.\,В.}\ \ см. Босов А.\,В.\hfill\hfill\hfill\hfill\hfill\hfill\hfill\hfill\hfill\hfill\hfill\hfill\hfill\hfill\hfill\hfill\hfill\hfill\hfill\hfill\hfill\hfill\hfill\hfill\hfill\hfill\hfill\hfill\hfill\hfill\hfill\hfill\hfill\hfill\hfill}{\ }
\contentsline {section}{\textbf{Ильин В.\,Д., Соколов И.\,А.}\ \ Символьная модель системы знаний информатики в\nobreakspace {}че\-ло\-ве\-ко-автоматной среде}{\qquad 1 \qquad 66} 
\contentsline {section}{\textbf{Калиниченко Л.\,А.}\ \ см. Захаров В.\,Н.\hfill\hfill\hfill\hfill\hfill\hfill\hfill\hfill\hfill\hfill\hfill\hfill\hfill\hfill\hfill\hfill\hfill\hfill\hfill\hfill\hfill\hfill\hfill\hfill\hfill\hfill\hfill\hfill\hfill\hfill\hfill\hfill\hfill\hfill\hfill}{\ }
\contentsline {section}{\textbf{Козеренко Е.\,Б.}\ \ Лингвистическое моделирование для систем машинного перевода и обработки знаний}{\qquad 1 \qquad 54} 
\contentsline {section}{\textbf{Козмидиади В.\,А.}\ \ см. Захаров В.\,Н.\hfill\hfill\hfill\hfill\hfill\hfill\hfill\hfill\hfill\hfill\hfill\hfill\hfill\hfill\hfill\hfill\hfill\hfill\hfill\hfill\hfill\hfill\hfill\hfill\hfill\hfill\hfill\hfill\hfill\hfill\hfill\hfill\hfill\hfill\hfill }{\ } 
\contentsline {section}{\textbf{Королев В.\,Ю.}\ \ см. Батракова Д.\,А.\hfill\hfill\hfill\hfill\hfill\hfill\hfill\hfill\hfill\hfill\hfill\hfill\hfill\hfill\hfill\hfill\hfill\hfill\hfill\hfill\hfill\hfill\hfill\hfill\hfill\hfill\hfill\hfill\hfill\hfill\hfill\hfill\hfill\hfill\hfill}{\ } 
\contentsline {section}{\textbf{Кудрявцев А.\,А., Шоргин С.\,Я.}\ \ Байесовский подход к\nobreakspace {}анализу систем массового обслуживания и\nobreakspace {}показателей надежности}{\qquad 2 \qquad 76}
\contentsline {section}{\textbf{Печинкин А.\,В., Соколов И.\,А., Чаплыгин В.\,В.}\ \ Многолинейная система массового обслуживания с конечным накопителем и ненадежными приборами}{\qquad 1 \qquad 27} 
\contentsline {section}{\textbf{Печинкин А.\,В., Соколов И.\,А., Чаплыгин В.\,В.}\ \ Стационарные характеристики многолинейной\nobreakspace {}системы массового обслуживания с\nobreakspace {}одновременными отказами приборов}{\qquad 2 \qquad 39}
\contentsline {section}{\textbf{Синицын И.\,Н.}\ \ Корреляционные методы построения аналитических информационных моделей флуктуаций полюса Земли по априорным данным}{\qquad 2 \qquad \hphantom{9}2}
\contentsline {section}{\textbf{Синицын И.\,Н.}\ \ Развитие теории фильтров Пугачева для оперативной обработки информации в стохастических системах}{{\qquad 1 \qquad \hphantom{9}3}} 
\contentsline {section}{\textbf{Соколов И.\,А.}\ \ см. Захаров В.\,Н.\hfill\hfill\hfill\hfill\hfill\hfill\hfill\hfill\hfill\hfill\hfill\hfill\hfill\hfill\hfill\hfill\hfill\hfill\hfill\hfill\hfill\hfill\hfill\hfill\hfill\hfill\hfill\hfill\hfill\hfill\hfill\hfill\hfill\hfill\hfill}{\ }
\contentsline {section}{\textbf{Соколов И.\,А.}\ \ см. Ильин В.\,Д.\hfill\hfill\hfill\hfill\hfill\hfill\hfill\hfill\hfill\hfill\hfill\hfill\hfill\hfill\hfill\hfill\hfill\hfill\hfill\hfill\hfill\hfill\hfill\hfill\hfill\hfill\hfill\hfill\hfill\hfill\hfill\hfill\hfill\hfill\hfill}{\ } 
\contentsline {section}{\textbf{Соколов И.\,А.}\ \ см. Печинкин А.\,В.\hfill\hfill\hfill\hfill\hfill\hfill\hfill\hfill\hfill\hfill\hfill\hfill\hfill\hfill\hfill\hfill\hfill\hfill\hfill\hfill\hfill\hfill\hfill\hfill\hfill\hfill\hfill\hfill\hfill\hfill\hfill\hfill\hfill\hfill\hfill}{\ } 
\contentsline {section}{\textbf{Соколов И.\,А.}\ \ см. Печинкин А.\,В.\hfill\hfill\hfill\hfill\hfill\hfill\hfill\hfill\hfill\hfill\hfill\hfill\hfill\hfill\hfill\hfill\hfill\hfill\hfill\hfill\hfill\hfill\hfill\hfill\hfill\hfill\hfill\hfill\hfill\hfill\hfill\hfill\hfill\hfill\hfill}{\ }
\contentsline {section}{\textbf{Ступников С.\,А.}\ \ см. Захаров В.\,Н.\hfill\hfill\hfill\hfill\hfill\hfill\hfill\hfill\hfill\hfill\hfill\hfill\hfill\hfill\hfill\hfill\hfill\hfill\hfill\hfill\hfill\hfill\hfill\hfill\hfill\hfill\hfill\hfill\hfill\hfill\hfill\hfill\hfill\hfill\hfill}{\ }
\contentsline {section}{\textbf{Чаплыгин В.\,В.}\ \ см. Печинкин А.\,В.\hfill\hfill\hfill\hfill\hfill\hfill\hfill\hfill\hfill\hfill\hfill\hfill\hfill\hfill\hfill\hfill\hfill\hfill\hfill\hfill\hfill\hfill\hfill\hfill\hfill\hfill\hfill\hfill\hfill\hfill\hfill\hfill\hfill\hfill\hfill}{\ } 
\contentsline {section}{\textbf{Чаплыгин В.\,В.}\ \ см. Печинкин А.\,В.\hfill\hfill\hfill\hfill\hfill\hfill\hfill\hfill\hfill\hfill\hfill\hfill\hfill\hfill\hfill\hfill\hfill\hfill\hfill\hfill\hfill\hfill\hfill\hfill\hfill\hfill\hfill\hfill\hfill\hfill\hfill\hfill\hfill\hfill\hfill}{\ }
\contentsline {section}{\textbf{Шоргин С.\,Я.}\ \ см. Батракова Д.\,А.\hfill\hfill\hfill\hfill\hfill\hfill\hfill\hfill\hfill\hfill\hfill\hfill\hfill\hfill\hfill\hfill\hfill\hfill\hfill\hfill\hfill\hfill\hfill\hfill\hfill\hfill\hfill\hfill\hfill\hfill\hfill\hfill\hfill\hfill\hfill}{\ } 
\contentsline {section}{\textbf{Шоргин С.\,Я.}\ \ см. Кудрявцев А.\,А.\hfill\hfill\hfill\hfill\hfill\hfill\hfill\hfill\hfill\hfill\hfill\hfill\hfill\hfill\hfill\hfill\hfill\hfill\hfill\hfill\hfill\hfill\hfill\hfill\hfill\hfill\hfill\hfill\hfill\hfill\hfill\hfill\hfill\hfill\hfill}{\ }
%\thispagestyle{myheadings}
\def\leftfootline{\small{\textbf{\thepage}
\hfill ИНФОРМАТИКА И ЕЁ ПРИМЕНЕНИЯ\ \ \ том~1\ \ \ выпуск~2\ \ \ 2007}
}%
 \def\rightfootline{\small{ИНФОРМАТИКА И ЕЁ ПРИМЕНЕНИЯ\ \ \ том~1\ \ \ выпуск~2\ \ \ 2007
 \hfill \textbf{\thepage}}}
 \label{end\stat}

%\def\stat{cont-e}
{%\hrule\par
%\vskip 7pt % 7pt
\raggedleft\Large \bf%\baselineskip=3.2ex
2\,0\,0\,7\ \ A\,U\,T\,H\,O\,R\ \ I\,N\,D\,E\,X \vskip 17pt
    \hrule
    \par
\vskip 21pt plus 6pt minus 3pt }

\label{st\stat}

\def\tit{\ }

\def\aut{\ }
\def\auf{\ }

\def\leftkol{\ } % ENGLISH ABSTRACTS}

\def\rightkol{\ } %ENGLISH ABSTRACTS}

\titele{\tit}{\aut}{\auf}{\leftkol}{\rightkol}


\contentsline {chapter}{\ }{Issue \quad Page} 
\contentsline {subsection}{\textbf{Batrakova D.\,A., Korolev V.\,Yu., Shorgin S.\,Ya.}\ \ A New Method for the Probabilistic and Statistical Analysis of Information Flows in Telecommunication Networks}{\qquad 1 \qquad 40} 
\contentsline {subsection}{\textbf{Borisov A.\,V.}\ \ Bayesian Estimation in\nobreakspace {}Observation Systems with\nobreakspace {}Markov Jump Processes: Game-Theoretic Approach}{\qquad 2 \qquad 65} 
\contentsline {subsection}{\textbf{Bosov A.\,V., Ivanov A.\,V.}\ \ Linguistic Simulation for Machine Translation and Knowledge Management Systems}{\qquad 2 \qquad 50} 
\contentsline {subsection}{\textbf{Chaplygin V.\,V.} see Pechinkin A.\,V.\hfill\hfill\hfill\hfill\hfill\hfill\hfill\hfill\hfill\hfill\hfill\hfill\hfill\hfill\hfill\hfill\hfill\hfill\hfill\hfill\hfill\hfill\hfill\hfill\hfill\hfill\hfill\hfill\hfill\hfill\hfill\hfill\hfill\hfill\hfill}{\ }
\contentsline {subsection}{\textbf{Chaplygin V.\,V.} see Pechinkin A.\,V.\hfill\hfill\hfill\hfill\hfill\hfill\hfill\hfill\hfill\hfill\hfill\hfill\hfill\hfill\hfill\hfill\hfill\hfill\hfill\hfill\hfill\hfill\hfill\hfill\hfill\hfill\hfill\hfill\hfill\hfill\hfill\hfill\hfill\hfill\hfill}{\ }
\contentsline {subsection}{\textbf{Ilyin V.\,D., Sokolov I.\,A.}\ \ The Symbol Model of Informatics Knowledge System in Human-Automaton Environment}{\qquad 1 \qquad 66} 
\contentsline {subsection}{\textbf{Ivanov A.\,V.} see Bosov A.\,V.\hfill\hfill\hfill\hfill\hfill\hfill\hfill\hfill\hfill\hfill\hfill\hfill\hfill\hfill\hfill\hfill\hfill\hfill\hfill\hfill\hfill\hfill\hfill\hfill\hfill\hfill\hfill\hfill\hfill\hfill\hfill\hfill\hfill\hfill\hfill}{\ }
\contentsline {subsection}{\textbf{Kalinichenko L.\,A.} see Zakharov V.\,N.\hfill\hfill\hfill\hfill\hfill\hfill\hfill\hfill\hfill\hfill\hfill\hfill\hfill\hfill\hfill\hfill\hfill\hfill\hfill\hfill\hfill\hfill\hfill\hfill\hfill\hfill\hfill\hfill\hfill\hfill\hfill\hfill\hfill\hfill\hfill}{\ }
\contentsline {subsection}{\textbf{Korolev V.\,Yu.} see Batrakova D.\,A.\hfill\hfill\hfill\hfill\hfill\hfill\hfill\hfill\hfill\hfill\hfill\hfill\hfill\hfill\hfill\hfill\hfill\hfill\hfill\hfill\hfill\hfill\hfill\hfill\hfill\hfill\hfill\hfill\hfill\hfill\hfill\hfill\hfill\hfill\hfill}{\ }
\contentsline {subsection}{\textbf{Kozerenko E.\,B.}\ \ Linguistic Simulation for Machine Translation and Knowledge Management Systems}{\qquad 1 \qquad 54} 
\contentsline {subsection}{\textbf{Kozmidiady V.\,A.} see Zakharov V.\,N.\hfill\hfill\hfill\hfill\hfill\hfill\hfill\hfill\hfill\hfill\hfill\hfill\hfill\hfill\hfill\hfill\hfill\hfill\hfill\hfill\hfill\hfill\hfill\hfill\hfill\hfill\hfill\hfill\hfill\hfill\hfill\hfill\hfill\hfill\hfill}{\ }
\contentsline {subsection}{\textbf{Kudryavtsev A.\,A., Shorgin S.\,Ya.}\ \ Bayesian Approach to Queueing Systems and Reliability Characteristics}{\qquad 2 \qquad 76} 
\contentsline {subsection}{\textbf{Pechinkin A.\,V., Sokolov I.\,A., Chaplygin V.\,V.}\ \ Multichannel Queuing System with Finite Buffer and Unreliable Servers}{\qquad 1 \qquad 27} 
\contentsline {subsection}{\textbf{Pechinkin A.\,V., Sokolov I.\,A., Chaplygin V.\,V.}\ \ Stationary Characteristics of a Multichannel Queueing System with\nobreakspace {}Simultaneous Refusals of Servers}{\qquad 2 \qquad 39} 
\contentsline {subsection}{\textbf{Shorgin S.\,Ya.} see Batrakova D.\,A.\hfill\hfill\hfill\hfill\hfill\hfill\hfill\hfill\hfill\hfill\hfill\hfill\hfill\hfill\hfill\hfill\hfill\hfill\hfill\hfill\hfill\hfill\hfill\hfill\hfill\hfill\hfill\hfill\hfill\hfill\hfill\hfill\hfill\hfill\hfill}{\ }
\contentsline {subsection}{\textbf{Shorgin S.\,Ya.} see Kudryavtsev A.\,A.\hfill\hfill\hfill\hfill\hfill\hfill\hfill\hfill\hfill\hfill\hfill\hfill\hfill\hfill\hfill\hfill\hfill\hfill\hfill\hfill\hfill\hfill\hfill\hfill\hfill\hfill\hfill\hfill\hfill\hfill\hfill\hfill\hfill\hfill\hfill}{\ }
\contentsline {subsection}{\textbf{Sinitsyn I.\,N.}\ \ Correlational Methods for Analytical Informational Models of the Earth Pole Fluctuations Design Based on a priori Data}{\qquad 2 \qquad \hphantom{9}2}
\contentsline {subsection}{\textbf{Sinitsyn I.\,N.}\ \ Development of Pugachev Filtering for Stochastic Systems}{\qquad 1 \qquad \hphantom{9}3}
\contentsline {subsection}{\textbf{Sokolov I.\,A.} see Ilyin V.\,D.\hfill\hfill\hfill\hfill\hfill\hfill\hfill\hfill\hfill\hfill\hfill\hfill\hfill\hfill\hfill\hfill\hfill\hfill\hfill\hfill\hfill\hfill\hfill\hfill\hfill\hfill\hfill\hfill\hfill\hfill\hfill\hfill\hfill\hfill\hfill}{\ }
\contentsline {subsection}{\textbf{Sokolov I.\,A.} see Pechinkin A.\,V.\hfill\hfill\hfill\hfill\hfill\hfill\hfill\hfill\hfill\hfill\hfill\hfill\hfill\hfill\hfill\hfill\hfill\hfill\hfill\hfill\hfill\hfill\hfill\hfill\hfill\hfill\hfill\hfill\hfill\hfill\hfill\hfill\hfill\hfill\hfill}{\ }
\contentsline {subsection}{\textbf{Sokolov I.\,A.} see Pechinkin A.\,V.\hfill\hfill\hfill\hfill\hfill\hfill\hfill\hfill\hfill\hfill\hfill\hfill\hfill\hfill\hfill\hfill\hfill\hfill\hfill\hfill\hfill\hfill\hfill\hfill\hfill\hfill\hfill\hfill\hfill\hfill\hfill\hfill\hfill\hfill\hfill}{\ }
\contentsline {subsection}{\textbf{Sokolov I.\,A.} see Zakharov V.\,N.\hfill\hfill\hfill\hfill\hfill\hfill\hfill\hfill\hfill\hfill\hfill\hfill\hfill\hfill\hfill\hfill\hfill\hfill\hfill\hfill\hfill\hfill\hfill\hfill\hfill\hfill\hfill\hfill\hfill\hfill\hfill\hfill\hfill\hfill\hfill}{\ }
\contentsline {subsection}{\textbf{Stupnikov S.\,A.} see Zakharov V.\,N.\hfill\hfill\hfill\hfill\hfill\hfill\hfill\hfill\hfill\hfill\hfill\hfill\hfill\hfill\hfill\hfill\hfill\hfill\hfill\hfill\hfill\hfill\hfill\hfill\hfill\hfill\hfill\hfill\hfill\hfill\hfill\hfill\hfill\hfill\hfill}{\ }
\contentsline {subsection}{\textbf{Zakharov V.\,N., Kalinichenko L.\,A., Sokolov I.\,A., Stupnikov S.\,A.}\ \ Development of Canonical Information Models for Integrated Information Systems}{\qquad 2 \qquad 15} 
\contentsline {subsection}{\textbf{Zakharov V.\,N., Kozmidiady V.\,A.}\ \ Means Providing Applications Fault Tolerance}{\qquad 1 \qquad 14} 
\def\leftfootline{\small{\textbf{\thepage}
\hfill ИНФОРМАТИКА И ЕЁ ПРИМЕНЕНИЯ\ \ \ том~1\ \ \ выпуск~2\ \ \ 2007}
}%
 \def\rightfootline{\small{ИНФОРМАТИКА И ЕЁ ПРИМЕНЕНИЯ\ \ \ том~1\ \ \ выпуск~2\ \ \ 2007
 \hfill \textbf{\thepage}}}
 \label{end\stat}


%\tableofcontents


\end{document}

\newcommand{\Ack}{\subsection*{\protect\large\bf Acknowledgments}}