\documentclass[10pt]{book}
\usepackage[utf8]{inputenc}

\usepackage{latexsym,amssymb,amsfonts,amsmath,amsxtra,indentfirst,shapepar,%fleqn,%
picinpar,shadow,floatflt,enumerate,multicol,colortbl,moreverb,cite,ipi}

\usepackage{rotating}
\usepackage{mathrsfs}
\usepackage[noend]{algorithmic}
\usepackage{ulem}
\usepackage{graphicx}
%\usepackage{algorithm2e}
\usepackage[linesnumbered,boxed,ruled]{algorithm2e}
%\usepackage{xypic}
\usepackage{oldgerm}

\SetAlgorithmName{Алгоритм}{алгоритм}{Список алгоритмов}

%из Дюковой

\newcommand{\algKeyword}[1]{{\bf #1}}
\newcommand{\Proc}[1]{\text{\tt #1}}
\def\CALL{\algKeyword{call}~}

\newenvironment{AlgProcedure}[1]
{
    \small
    \medskip
    %    \hrule
    \medskip
    \algKeyword{PROCEDURE} #1
    \begin{algorithmic}[1]}
    {\end{algorithmic}
    %    \hrule
    \bigskip
}

\def\CALL{\algKeyword{call}~}

%конец для Дюковой

%\RequirePackage[ruled]{algorithm}


\input{epsf}

%\nofiles

%\includeonly{avtor} %+pdf
%\includeonly{obchak,avtor}
%\includeonly{pred}                 %+
%\includeonly{podgot-rus-site,podgot-eng-site}  
%\includeonly{ocherk} 
%\includeonly{nekrol} 
%\includeonly{ipi-ind} 
%\includeonly{toc-rus}
%\includeonly{toc-en} 


%\includeonly{gadamaka}  %1 pdfавт
%\includeonly{razumchik-rus}  %2pdfавт
%\includeonly{razumchik-eng}  %3pdfавт
%\includeonly{korolev}   %4pdf
%\includeonly{gorshenin}  %5pdfавт
%\includeonly{malashenko}  %6pdfавт
%\includeonly{agalarov} %7pdfавт
%\includeonly{grusho}   %8pdfавт
%\includeonly{grebeshkov} %9pdf
%\includeonly{naumov}   %10pdfавт
%\includeonly{buyanov}  %11pdfавт
%\includeonly{bitukov}  %12pdfавт
%\includeonly{kudr-tit} %13pdfавт
%\includeonly{pagano}   %14pdfавт
%\includeonly{kruzhkov} %15pdfавт


%\includeonly{toc-rus, toc-en}
%\includeonly{obchak} %,toc-en}
%\includeonly{rekl}
%\includeonly{rekl-1}
%\includeonly{reshal}  %
%\includeonly{eng-index}
%\includeonly{cover3}

\usepackage{acad}
%\usepackage{courier}
\usepackage{decor}
\usepackage{newton}
\usepackage{pragmatica}
\usepackage{zapfchan}
\usepackage{petrotex}
\usepackage{bm}                     % полужирные греческие буквы
\usepackage{upgreek}                % прямые греческие буквы
\usepackage{eufrak}
\usepackage{verbatim}

\renewcommand{\bottomfraction}{0.99}
\renewcommand{\topfraction}{0.99}
\renewcommand{\textfraction}{0.01}

\setcounter{secnumdepth}{1} %здесь - 3 + chapter = 4

\arraycolsep=1.5pt

%\usepackage[pdftex]{graphicx}

%\usepackage{oz}

%NEW COMMANDS


\renewcommand*{\hm}[1]{#1\nobreak\discretionary{}%
            {\hbox{$\mathsurround=0pt #1$}}{}} %% Дублирует знаки операций
                               %при переносе в формуле (перед знаком, который
                               %надо продублировать ставится команда \hm)

%\newcommand{\endproof}{\hfill$\Box$}
\renewcommand{\r}{\mathbb{R}}
\newcommand{\I}{{\rm I\hspace{-0.7mm}I}}
%\newcommand{\Ikl}{{\tt{1}}\hspace*{-1.44mm}\mathtt{1}}
\newcommand{\Ik}{\mbox{{\small \tt {1}}\hspace{-1.3mm}{\tt 1}}}
\newcommand{\argmin}{\mathop{\mathrm{arg}\mathrm{min}}}
\newcommand{\argmax}{\mathop{\mathrm{arg}\mathrm{max}}}
%\newcommand{\capr}{\mathop{\cap\,}}
%\newcommand{\cupr}{\mathop{\cup\,}}
%\def\argmin{\mathop{arg\,min}}

\def\vrp{\varphi}
\def\prt{\partial}
\def\mm{{\sf M}}
\def\modnop#1{\mathop{#1}\limits_{n}}
\def\eam{\mathbin{{\mathop{=}\limits^{\mathrm{def}}}}}
\def\dey#1#2{#1 (#2)}
\def\deyc#1#2{#1 \cdot  #2}
\def\ra#1{\;\mathop{\to}\limits^{#1}\;}
\def\raz#1{\;\mathop{\longrightarrow}\limits^{\!\!\!#1}\;}
\def\ral#1{\;\mathop{\longrightarrow}\limits^{#1}\;}

\newcommand{\Nor}{\mathcal{N}}
\newcommand{\T}{\mathbb{T}}
\newcommand{\Z}{\mathbb{Z}}



\newcommand{\il}[2]{\int\limits_{#1}^{#2}}%интеграл с пределами #1 и #2

\def\sm2{\mathop {\sum\limits^{n^\Theta}\sum\limits^{n^\Theta}}}
\def\sss{\sum\limits}
\def\tr{,\,\ldots\,,\,}
\def\rk{\right]}
\def\lk{\left[}
\def\rf{\right\}}
\def\lf{\left\{}
\def\lv{\,\left\vert}
\def\rv{\right\vert\,}
\def\iii{\int\limits}
\def\iin{\int\limits_{-\infty}^\infty}
\def\rrv{\right\vert}


\def\ee{{\cal E}}
\def\ww{{\cal W}}
\def\yy{{\cal Y}}
\def\vv{{\cal V}}

\newcommand{\R}{\mathbb R}
\newcommand{\E}{\mathbb E}
\newcommand{\N}{\mathbb N}

\renewcommand{\P}{\mathbb{P}}

\newcommand{\h}{{\bf H}}
\newcommand{\p}{{\sf P}}  % вероятность

\newcommand{\e}{{\sf E}}  % мат. ожидание
\newcommand{\D}{{\sf D}}  % дисперсия
\newcommand{\eps}{\varepsilon}
\newcommand{\vp}{{\mathbf p}}
\newcommand{\vz}{{\mathbf z}}
\newcommand{\vx}{{\mathbf x}}
\newcommand{\vf}{{\mathbf f}}
\newcommand{\F}{{\mathcal F}}
\def\ap{{\mathrm{ЭР}}}
\newcommand{\ud}{\Delta_n} %uniform ditance
\newcommand{\nud}{\Delta_n(x)}
\renewcommand{\Re}{\mathrm{Re}\,}

\newcommand{\abs}[1]{\left\vert#1\right\vert}

\newcommand{\norm}[1]{\left\Vert#1\right\Vert}
\def\da{(\Delta_t,A)}

\newcommand{\corr}{\mathrm{corr}}

\newcommand{\cov}{\mathrm{cov}}
\newcommand{\Expect}{\mathbb{E}}

\def\w{\omega}
\def\W{\Omega}

\def\inh{\int\limits_{nh}^{(n+1)h}}

\def\sumin{\sum_{i=1}^N}


\def\bxt{(Y,t)}
\def\xt{(y,t)}

\def\ovth{{\fr{\tau-nh}{h}}}
\def\ov{\overline}
\def\tm{\tilde m}
\def\tl{\tilde\lambda}
\def\tB{\widetilde B}
\def\tb{\tilde b}
\def\ld{\ldots}
\def\cd{\cdots}


\DeclareMathOperator{\sign}{sign}

%\newcommand{\gr}{{\geqslant}}


\newcommand{\g}{\mbox{\textit{g}}}

\renewcommand{\la}{\lambda}
\newcommand{\si}{\sigma}
\newcommand{\alp}{\alpha}

%\newcommand{\pto}{\stackrel{P}{\longrightarrow}} % сходимость по веpоятности

\newcommand{\eqd}{\stackrel{\mathrm{d}}{=}} % равенство по pаспpеделению
\newcommand{\eqdelta}{\stackrel{\triangle}{=}} % равенство по pаспpеделению

\def\be#1{\begin{equation}\label{#1}}
\def\ee{\end{equation}}
\def\re#1{(\ref{#1})}

\def\bn{\begin{enumerate}}
\def\en{\end{enumerate}}
\def\bi{\begin{itemize}}
\def\ei{\end{itemize}}
%\def\i{\item}

%\newcommand{\kp}{\kappa}
%\def\Q{{\cal Q}} \def\H{{\cal H}}
%\newcommand{\bet}{\beta_{2+\delta}}


%\newtheorem{definition}{Определение}
%\renewcommand{\thedefinition}{\arabic{definition}.}
%END NEW COMMANDS

%\renewcommand{\baselinestretch}{1.2}

%\pagestyle{myheadings}

\setlength{\textwidth}{167mm}      % 122mm
\setlength{\textheight}{658pt}
%\setlength{\textheight}{635.6pt}
\setlength{\columnsep}{4.5mm}

\setcounter{secnumdepth}{4}

%\addtolength{\headheight}{2pt}
%\addtolength{\headsep}{-2mm}

\addtolength{\topmargin}{-7mm}  % for printing


%\hoffset=-30mm  % From Yap
\hoffset=-23mm  % From Acrobat

%\voffset=0mm % From Yap
\voffset=-5mm   % From Acrobat

%\addtolength{\evensidemargin}{-2.5mm} % for printing
%\addtolength{\oddsidemargin}{2.5mm}  % for printing

\addtolength{\evensidemargin}{-12mm} % for printing
\addtolength{\oddsidemargin}{8mm}  % for printing

%\renewcommand{\thefootnote}{\fnsymbol{footnote}}
%\renewcommand{\thefootnote}{\arabic{footnote}}
\renewcommand{\figurename}{\protect\bf Рис.}
\renewcommand{\tablename}{\protect\bf Таблица}

\newcommand{\Caption}[1]{\caption{\protect\small %\baselineskip=2.5ex
#1}}

\renewcommand{\thefigure}{\arabic{figure}}
\renewcommand{\thetable}{\arabic{table}}
\renewcommand{\theequation}{\arabic{equation}}
\renewcommand{\thesection}{\arabic{section}}

\renewcommand{\contentsname}{СОДЕРЖАНИЕ}
\newcommand{\fr}[2]{\displaystyle\frac{\displaystyle #1\mathstrut}{\displaystyle #2\mathstrut}}

%\renewcommand{\thefootnote}{\fnsymbol{footnote}}
%\newcommand{\g}{\mbox{\textit{g}}}

%\newcommand{\Caption}[1]{\caption{\protect\small\baselineskip=2ex #1}}
\newcounter{razdel}
\setcounter{razdel}{0}


\newcommand{\titel}[4]{%
\

\vspace*{5pt}

\ifodd\therazdel {\raggedright\noindent\Large\textrm\textbf
 \lineskip .75em
  \baselineskip=3.2ex #1 \par}
\vskip 1em {\noindent\large\textrm\textbf #2 \par}
\addcontentsline{toc}{subsection}{{\textrm\textbf #1}\protect\newline #2}
\def\rightheadline{\underline{\noindent\hbox to \textwidth{\hfill\small\textrm{#4}
%\hfill \large\bf\thepage
}}}
\def\leftheadline{\underline{\noindent\parbox{\textwidth}{
%\raggedleft\large\bf\thepage \hfill
\small\textit{#3}\hfill}}}
\def\leftfootline{\small{\textbf{\thepage}
\hfill ИНФОРМАТИКА И ЕЁ ПРИМЕНЕНИЯ\ \ \ том~11\ \ \ выпуск 4\ \ \ 2017}
}%
 \def\rightfootline{\small{ИНФОРМАТИКА И ЕЁ ПРИМЕНЕНИЯ\ \ \ том~11\ \ \ выпуск~4\ \ \ 2017
\hfill \textbf{\thepage}}}
\vskip 2em \setcounter{figure}{0}
\setcounter{table}{0}
\setcounter{equation}{0}
\setcounter{section}{0}
\setcounter{subsection}{0}
\setcounter{subsubsection}{0}
\setcounter{footnote}{0}
\setcounter{razdel}{0}
%\end{flushleft}
\else {
 \raggedright\noindent\Large\textrm\textbf
 \lineskip .75em
\baselineskip=3.2ex #1 \par} \vskip 1em
%\begin{flushleft}
{\noindent\large\textrm\textbf #2 \par}
\addcontentsline{toc}{subsection}{{\textrm\textbf #1}\protect\newline #2}
\def\rightheadline{\underline{\noindent\hbox to \textwidth{\hfill\small\textrm{#4}
%\hfill \large\bf\thepage
}}}
\def\leftheadline{\underline{\noindent\parbox{\textwidth}{%\raggedleft\large\bf\thepage \hfill
\small\textit{#3}\hfill}}}
\def\leftfootline{\small{\textbf{\thepage}
\hfill ИНФОРМАТИКА И ЕЁ ПРИМЕНЕНИЯ\ \ \ том~11\ \ \ выпуск~4\ \ \ 2017}
}%
 \def\rightfootline{\small{ИНФОРМАТИКА И ЕЁ ПРИМЕНЕНИЯ\ \ \ том~11\ \ \ выпуск~4\ \ \ 2017
\hfill \textbf{\thepage}}} \vskip 2em \setcounter{figure}{0}
\setcounter{table}{0} \setcounter{equation}{0} \setcounter{section}{0}
\setcounter{subsection}{0} \setcounter{subsubsection}{0}
\setcounter{footnote}{0}
%\end{flushleft}
\fi}

\newcommand{\titelr}[2]{%
\

\vspace*{5pt}

\ifodd\therazdel {\raggedright\noindent%\Large\textrm\textbf
 \lineskip .75em
  \baselineskip=3.2ex #1 \par}
\vskip 1em {\noindent\normalsize\textrm\textbf #2 \par}
\else {
 \raggedright\noindent\Large\textrm\textbf
 \lineskip .75em
\baselineskip=3.2ex #1 \par} \vskip 1em
%\begin{flushleft}
{\noindent\large\textrm\textbf #2 \par
%\noindent\normalsize\textrm\textbf #2 \par
} \fi}

\newcommand{\titele}[5]{%
\

%\vspace*{5pt}

\ifodd\therazdel {\raggedright\noindent\large
\textrm\textbf
 \lineskip .75em
%  \baselineskip=3.2ex
#1 \par}
\vskip .5em {\noindent\large\textrm\textbf #2 \par}
\vskip .5em
 {\noindent\textrm #3 \par}
\addcontentsline{toc}{subsection}{{\textrm\textbf #1}\protect\newline #2}
\def\rightheadline{\underline{\noindent\hbox to \textwidth{\hfill\small\textrm{#4}
%\hfill \large\bf\thepage
}}}
\def\leftheadline{\underline{\noindent\parbox{\textwidth}{
%\raggedleft\large\bf\thepage \hfill
\small\textrm{#5}\hfill}}}
\def\leftfootline{\small{\textbf{\thepage}
\hfill ИНФОРМАТИКА И ЕЁ ПРИМЕНЕНИЯ\ \ \ том~11\ \ \ выпуск~4\ \ \ 2017}
}%
 \def\rightfootline{\small{ИНФОРМАТИКА И ЕЁ ПРИМЕНЕНИЯ\ \ \ том~11\ \ \ выпуск~4\ \ \ 2017
\hfill \textbf{\thepage}}} \vskip 1em \setcounter{figure}{0}
\setcounter{table}{0} \setcounter{equation}{0} \setcounter{section}{0}
\setcounter{subsection}{0} \setcounter{subsubsection}{0}
\setcounter{footnote}{0} \setcounter{razdel}{0}
%\end{flushleft}
\else {
 \raggedright\noindent\large
 \textrm\textbf
 \lineskip .75em
%\baselineskip=3.2ex
#1 \par} \vskip .5em
%\begin{flushleft}
{\noindent\large\textrm\textbf #2 \par} \vskip .5em
 {\noindent\textrm #3 \par}
\addcontentsline{toc}{subsection}{{\textrm\textbf #1}\protect\newline #2}
\def\rightheadline{\underline{\noindent\hbox to \textwidth{\hfill\small\textrm{#4}
%\hfill \large\bf\thepage
}}}
\def\leftheadline{\underline{\noindent\parbox{\textwidth}{%\raggedleft\large\bf\thepage \hfill
\small\textrm{#5}\hfill}}}
\def\leftfootline{\small{\textbf{\thepage}
\hfill ИНФОРМАТИКА И ЕЁ ПРИМЕНЕНИЯ\ \ \ том~11\ \ \ выпуск~4\ \ \ 2017}
}%
 \def\rightfootline{\small{ИНФОРМАТИКА И ЕЁ ПРИМЕНЕНИЯ\ \ \ том~11\ \ \ выпуск~4\ \ \ 2017
\hfill \textbf{\thepage}}} \vskip 1em \setcounter{figure}{0}
\setcounter{table}{0} \setcounter{equation}{0} \setcounter{section}{0}
\setcounter{subsection}{0} \setcounter{subsubsection}{0}
\setcounter{footnote}{0}
%\end{flushleft}
\fi}

\def\Abst#1{
\begin{center}\small\nwt
\parbox{150mm}{%\baselineskip=2.5ex
\textbf{Аннотация:}\ \
%\hspace*{\parindent}
#1}
\end{center}}
\def\Abste#1{
\begin{center}\small\nwt
\parbox{150mm}{%\baselineskip=2.5ex
\textbf{Abstract:}\ \
%\hspace*{\parindent}
#1}
\end{center}}

\def\DOI#1{
\begin{center}\small\nwt
\parbox{150mm}{%\baselineskip=2.5ex
\textbf{DOI:}\ \
%\hspace*{\parindent}
#1}
\end{center}}

\def\Abstend#1{
\begin{center}\small\nwt
\parbox{150mm}{%\baselineskip=2.5ex
%\hspace*{\parindent}
#1}
\end{center}}


\def\KW#1{
\begin{center}\small\nwt
\parbox{150mm}{%\baselineskip=2.5ex
\textbf{Ключевые слова:}\ \ #1}
\end{center}}

\def\KWE#1{
\begin{center}\small\nwt
\parbox{150mm}{%\baselineskip=2.5ex
\textbf{Keywords:}\ \ #1}
\end{center}}


\def\KWN#1{
%\begin{center}
%\small
%\parbox{150mm}\end{center}
}

\newcommand{\Avtors}[1]{%\smallskip
%\vspace*{.5pt}
\hangindent=23pt\noindent
%\nwt
{\bfseries#1}\
}


\renewcommand{\thesubsection}{\thesection.\arabic{subsection}\hspace*{-5pt}}
\renewcommand{\thesubsubsection}{\thesubsection\hspace*{5pt}.\arabic{subsubsection}\hspace*{-3pt}}

\newcommand{\Ack}{\section*{\protect\rmfamily Acknowledgments}\noindent}
\newcommand{\Contr}{\section*{\protect\rmfamily Contributors}\noindent}
\newcommand{\Contrl}{\section*{\protect\rmfamily Contributor}\noindent}

\makeindex


\begin{document}
\Rus

\nwt
%\ptb


%\renewcommand{\contentsname}{\protect\Large\bf Содержание}

\setcounter{tocdepth}{2}

%\tableofcontents

\renewcommand{\bibname}{\protect\rmfamily Литература}
  \def\Au#1{{\it #1}}
    \def\Aue#1{{#1}}

%\newcommand{\No}{№}
  \newcommand{\tg}{\,\mathrm{tg}\,}
    \newcommand{\ctg}{\,\mathrm{ctg}\,}
  \newcommand{\arctg}{\,\mathrm{arctg}\,}

\def\forallb{\mathop{\forall}}
\def\cupb{\mathop{\cup}}
\def\existsb{\mathop{\exists}}


\newpage
\addtocounter{razdel}{1}
%\def\razd{РЕГУЛИРУЕМЫЙ ЭЛЕКТРОПРИВОД ДЛЯ ЭЛЕКТРОЭНЕРГЕТИКИ}


\setcounter{page}{2}

%   { %\Large  
   { %\baselineskip=16.6pt
   
   \vspace*{-48pt}
   \begin{center}\LARGE
   \textit{Предисловие}
   \end{center}
   
   %\vspace*{2.5mm}
   
   \vspace*{25mm}
   
   \thispagestyle{empty}
   
   { %\small 

    
Вниманию читателей журнала <<Информатика и её применения>> предлагается 
очередной тематический выпуск <<Вероятностно-статистические методы и 
задачи информатики и информационных технологий>>. Предыдущие тематические 
выпуски журнала по данному направлению вышли в 2008~г.\ (т.~2, вып.~2), 
в 2009~г.\ (т.~3, вып.~3) и в 2010~г.\ (т.~4, вып.~2). 

Статьи, собранные в данном журнале, посвящены разработке новых вероятностно-статистических 
методов, ориентированных на применение к решению конкретных задач информатики и информационных 
технологий, а также~--- в ряде случаев~--- и других прикладных задач. Проблематика, охватываемая 
публикуемыми работами, развивается в рамках научного сотрудничества между Институтом проблем 
информатики Российской академии наук (ИПИ РАН) и Факультетом вычислительной математики и 
кибернетики Московского государственного университета им.\ М.\,В.~Ломоносова в ходе работ 
над совместными научными проектами (в том числе в рамках функционирования 
Научно-образовательного центра <<Вероятностно-статистические методы анализа рисков>>). 
Многие из авторов статей, включенных в данный номер журнала, являются активными участниками 
традиционного международного семинара по проблемам устойчивости стохастических моделей, 
руководимого В.\,М.~Золотаревым и В.\,Ю.~Королевым; регулярные сессии этого семинара 
проводятся под эгидой МГУ и ИПИ РАН (в 2011~г.\ указанный семинар проводится в октябре 
в Калининградской области РФ). 

Наряду с представителями ИПИ РАН и МГУ в число авторов данного выпуска журнала входят 
ученые из Научно-исследовательского института системных исследований РАН, Института 
проблем технологии микроэлектроники и особочистых материалов РАН, Института 
прикладных математических исследований Карельского НЦ РАН, Московского 
авиационного института, Вологодского государственного педагогического университета, 
НИИММ им.\ Н.\,Г.~Чеботарева, Казанского государственного университета, Дебреценского 
университета (Венгрия).

Несколько статей выпуска посвящено разработке и применению стохастических методов и 
информационных технологий для решения различных прикладных задач. В~работе В.\,Г.~Ушакова 
и О.\,В.~Шестакова рассмотрена задача определения вероятностных характеристик случайных 
функций по распределениям интегральных преобразований, возникающих в задачах эмиссионной 
томографии. В~статье Д.\,О.~Яковенко и М.\,А.~Целищева рассмотрены некоторые вопросы 
математической теории риска и предложен новый подход к диверсификации инвестиционных 
портфелей. Работа И.\,А.~Кудрявцевой и А.\,В.~Пантелеева посвящена построению и 
исследованию математической модели, описывающей динамику сильноионизованной плазмы. 
В~статье П.\,П.~Кольцова изучается качество работы ряда алгоритмов сегментации изображений. 
Статья А.\,Н.~Чупрунова и И.~Фазекаша посвящена вероятностному анализу числа без\-оши\-бочных 
блоков при помехоустойчивом кодировании; получены усиленные законы больших чисел для указанных 
величин.

В данном выпуске традиционно присутствует тематика, весьма активно разрабатываемая в течение 
многих лет специалистами ИПИ РАН и МГУ,~--- методы моделирования и управления для 
информационно-телекоммуникационных и вычислительных систем, в частности методы 
теории массового обслуживания. В~статье А.\,И.~Зейфмана с соавторами рассматриваются 
модели обслуживания, описываемые марковскими цепями с непрерывным временем в случае 
наличия катастроф. В~работе М.\,М.~Лери и И.\,А.~Чеплюковой рассматриваются случайные 
графы Интернет-типа, т.\,е.\ графы, степени вершин которых имеют степенные распределения; 
такие задачи находят применение при исследовании глобальных сетей передачи данных. 
Работа Р.\,В.~Разумчика посвящена исследованию систем массового обслуживания специального 
вида~--- с отрицательными заявками и хранением вытесненных заявок.

Ряд статей посвящен развитию перспективных теоретических 
вероятностно-статистических методов, которые находят широкое применение в различных 
задачах информатики и информационных технологий. В~работе В.\,Е.~Бенинга, А.\,К.~Горшенина 
и В.\,Ю.~Королева рассмотрена задача статистической проверки гипотез о числе компонент 
смеси вероятностных распределений, приводится конструкция асимптотически наиболее мощного 
критерия. Результаты этой работы найдут применение в ряде прикладных задач, использующих 
математическую модель смеси вероятностных распределений (в информатике, моделировании 
финансовых рынков, физике турбулентной плазмы и~т.\,д.). В~статье В.\,Ю.~Королева, 
И.\,Г.~Шевцовой и С.\,Я.~Шоргина строится новая, улучшенная оценка точности нормальной 
аппроксимации для пуассоновских случайных сумм; как известно, указанные случайные суммы 
широко используются в качестве моделей многих реальных объектов, в том числе в информатике, 
физике и других прикладных областях. Работа В.\,Г.~Ушакова и Н.\,Г.~Ушакова посвящена 
исследованию ядерной оценки плотности распределения; эти результаты могут применяться, 
в част\-ности, при анализе трафика в телекоммуникационных системах. Серьезные приложения 
в статистике могут получить результаты работы О.\,В.~Шестакова, в которой доказаны оценки 
скорости сходимости распределения выборочного абсолютного медианного отклонения к нормальному 
закону. 

\smallskip

Редакционная коллегия журнала выражает надежду, что данный тематический  выпуск 
будет интересен специалистам в области теории вероятностей и математической статистики 
и их применения к решению задач информатики и информационных технологий.
     
     %\vfill 
     \vspace*{20mm}
     \noindent
     Заместитель главного редактора журнала <<Информатика и её 
применения>>,\\
     директор ИПИ РАН, академик  \hfill
     \textit{И.\,А.~Соколов}\\
     
     \noindent
     Редактор-составитель тематического выпуска,\\
     профессор кафедры математической статистики факультета\\
      вычислительной математики и кибернетики МГУ им.\ М.\,В.~Ломоносова,\\
     ведущий научный сотрудник ИПИ РАН,\\ 
доктор физико-математических наук \hfill
      \textit{В.\,Ю.~Королев}
     
     } }
     }

\def\stat{gaidamaka}

\def\tit{ЗАДАЧИ ОПТИМАЛЬНОГО ПЛАНИРОВАНИЯ МЕЖУРОВНЕВОГО 
ИНТЕРФЕЙСА В БЕСПРОВОДНЫХ СЕТЯХ$^*$}

\def\titkol{Задачи оптимального планирования межуровневого 
интерфейса в беспроводных сетях}

\def\autkol{Ю.\,В.~Гайдамака, Т.\,В.~Ефимушкина, А.\,К.~Самуйлов, 
К.\,Е.~Самуйлов}
\def\aut{Ю.\,В.~Гайдамака$^1$, Т.\,В.~Ефимушкина$^2$, А.\,К.~Самуйлов$^3$, 
К.\,Е.~Самуйлов$^4$}

\titel{\tit}{\aut}{\autkol}{\titkol}

{\renewcommand{\thefootnote}{\fnsymbol{footnote}}\footnotetext[1]
{Работа выполнена при частичной поддержке РФФИ (гранты 10-07-00487-a и 12-07-00108)
и Рособразования 
(проект 020619-1-174).}}

\renewcommand{\thefootnote}{\arabic{footnote}}
\footnotetext[1]{Российский университет дружбы народов, кафедра систем телекоммуникаций, ygaidamaka@sci.pfu.edu.ru}
\footnotetext[2]{Российский университет дружбы народов, кафедра систем телекоммуникаций, tefimushkina@gmail.com}
\footnotetext[3]{Российский университет дружбы народов, кафедра систем телекоммуникаций, asam1988@gmail.com}
\footnotetext[4]{Российский университет дружбы народов, кафедра систем телекоммуникаций, ksam@sci.pfu.edu.ru}
 
 
   \Abst{В данном обзоре проведено исследование современного состояния задач оптимального 
планирования межуровневого интерфейса на базе механизма мультиплексирования с 
ортогональным частотным разделением (OFDM, Orthogonal Frequency Division Multiplexing) для 
нисходящего канала в сетевой технологии LTE (Long-Term Evolution). При этом рассматривается 
понятие межуровневой оптимизации, подробно описаны оптимизационные задачи и ограничения, 
возникающие при разделении радиоресурсов в нисходящем канале, дан краткий обзор 
планировщиков и соответствующих им функций полезности, определяющих уровень 
удовлетворенности пользователей схемой распределения радиоресурсов при заданных 
ограничениях.}

%\vspace*{2pt}
   
   \KW{технология OFDM; межуровневая оптимизация; функция полезности; планировщик; 
эффективное распределение частот}

%\vspace*{6pt}

\vskip 14pt plus 9pt minus 6pt

      \thispagestyle{headings}

      \begin{multicols}{2}

            \label{st\stat}
   
\section{Введение}
  
  К основным задачам в беспроводных сетях относится оптимизация распределения 
ограниченного числа радиоресурсов между пользователями. Различные типы пакетного 
трафика, передаваемого по сети, предполагают динамическое выделение ресурсов 
пользователям. Решением задач планирования ресурсов, назначения приоритетов доступа 
в\linebreak зависимости от типа трафика с заданными требованиями к качеству обслуживания (QoS, 
Quality of\linebreak Service) занимаются модули управления радиоресурсами, называемые 
планировщиками (англ.\ \textit{schedulers}).
  
  Динамичное изменение загруженности канала\linebreak в беспроводной сети определяет 
требования к планиров\-щику, одним из которых является меж\-уров\-не\-вый (англ.\ 
\textit{crosslayer}) подход к решению\linebreak задачи оптимального распределения ресурсов. 
Основным принципом межуровневой оптимизации является комплексное решение задачи 
эффективного использования ограниченного числа радиоресурсов, учитывающее ряд 
первостепенных факторов: повышение пропускной способности; обеспечение 
равнодоступности~--- справедливого (англ.~\textit{fair}) разделения ресурсов между 
пользователями; достижение требуемого или, по крайней мере, наилучшего возможного 
качества обслуживания~[1].
  
  В обзоре в общем виде сформулированы основные задачи оптимизации, возникающие 
при планировании ресурсов в беспроводных сетях в целях повышения эффективности 
работы сети с большим числом несущих, сети, построенной на базе механизма OFDM, 
характерного для нисходящего канала в технологии LTE~[2, 3]. Вводится понятие 
межуровневой оптимизации, подробно описаны оптимизационные задачи и ограничения, 
возникающие при разделении ресурсов в нисходящем канале. Исследованы два алгоритма 
межуровневой оптимизации, предназначенных для максимизации функции полезности в 
различных условиях~--- алгоритм динамического назначения поднесущих DSA (Dynamic 
Subcarrier Assignment) и алгоритм адап\-тив\-но\-го распределения мощности APA (Adaptive 
Power Allocation). Для рассматриваемых алгоритмов сформулированы задачи 
максимизации функции полезности и получены их решения.

\section{Виды межуровневой оптимизации}

  С точки зрения терминологии межуровневая оптимизация заключается в объединении 
нескольких уровней модели взаимодействия открытых сис\-тем (OSI, Open Systems 
Interconnection) для полу\-чения более качественных решений и \mbox{эффективных} алгоритмов 
без лишних межуровневых обменов информа\-ции. Межуровневый подход к решению 
задачи оптимального распределения ресурсов позволяет в динамическом режиме учесть 
изменения типов трафика в беспроводной сети, потребности в услугах, значений 
параметров в канале связи и мобильность абонентов.
  
  Выделим три основных вида межуровневой оптимизации. Главной задачей 
  ка\-наль\-но-ориен\-ти\-ро\-ван\-но\-го вида межуровневой оптимизации является 
эффективное использование ограниченного числа изменяющихся во времени 
радиоресурсов с целью обеспечения высокой пропускной способности, заданных 
требований к качеству и равнодоступности. С~точки зрения модели OSI данная задача 
рассматривается между физическим и канальным уровнями.
  
  Второй вид, ориентированный на качество пред\-остав\-ле\-ния услуг абоненту <<из конца 
в конец>>, решает задачи адаптации протоколов верхних уровней к нестабильным, 
изменяющимся во времени канальным ресурсам для достижения заданных требований к 
QoS параметрам, например к производительности и задержкам. Заметим, что 
эффективность функционирования протокола TCP в беспроводных сетях связи~--- одна из 
типичных задач данного вида оптимизации~[4--6].
  
  Выбор наилучшего маршрута определяет третий вид межуровневой оптимизации, 
рассматриваемый в~[7--9]. При этом поиск наиболее эффективного маршрута происходит 
с учетом сетевого и физического или канального уровней. Далее в статье рассматриваются 
задачи оптимизации только с точки зрения ка\-наль\-но-ориен\-ти\-ро\-ван\-но\-го 
межуровневого подхода.

\section{Задачи оптимизации}

\subsection{Задача минимизации мощности}
  
  В~[10] сформулирована задача минимизации общей выделяемой пользователям сети 
мощ\-ности,\linebreak учитывающая распределение поднесущих с определением числа бит и уровня 
выделяемой мощ\-ности для каждой из поднесущих на основе мгновенных 
(\textit{instantaneous}) характеристик состояния\linebreak канала, измеренных для каждого из 
пользователей сети. В~рамках данной задачи предложен и реализован итерационный 
алгоритм распределения поднесущих между пользователями, а также обобщенный 
алгоритм задания числа бит и уровня мощ\-ности для поднесущих, назначенных 
пользователям. В~[10] рассматривается сеть связи с $K$ пользователями, в которой 
  $k$-поль\-зо\-ва\-тель имеет скорость передачи, равную $R_k$ бит на OFDM-сим\-вол 
(далее бит/символ), $k\hm=\overline{1,K}$. На передатчике реализован алгоритм 
назначения $n$-под\-не\-су\-щей $k$-поль\-зо\-ва\-те\-лю, после применения которого\linebreak на 
основе характеристик состояния канала для\linebreak $k$-поль\-зо\-ва\-те\-ля применяется 
обобщенный алгоритм задания $c_{k,n}$ чис\-ла бит/символ для $n$-под\-не\-су\-щей 
(здесь и далее $n\hm=\overline{1,N}$).
  
  В зависимости от числа назначенных $c_{k,n}$ бит из множества 
$\mathcal{D}=\{0,1,2,\ldots ,M\}$, где $M$~--- максимально возможное для передачи 
число бит/символ, адаптивный модулятор выбирает соответствующую схему модуляции, 
при этом уровень выделяемой мощности адаптируется согласно обобщенному алгоритму. 
Заметим, что $n$-поднесущая предоставляется только одному пользователю, т.\,е.\ при 
$c_{k^\prime,n}\not=0$, $c_{k,n}=0$ для всех $k\not=k^\prime$.
  
  В частотно-селективном канале с замираниями $n$-поднесущая характеризуется 
уровнем мощ\-ности сигнала $a_{k,n}$ по отношению к $k$-пользователю. При этом 
дисперсия уровня спектральной плот\-ности шума $\sigma_{k,n}^2$ принята равной 
единице для всех поднесущих. Для поддержания требуемого качества услуги на 
приемнике выделяемая мощность для передачи $k$-пользователю на $n$-поднесущей 
рассчитывается по формуле 
$$
P_{k,n}=\fr{f_k(c_{k,n})}{a_{k,n}^2}\,,
$$ 
где $f_k(c_{k,n})$~--- 
требуемая мощность для приема данных и их последующей демодуляции. Таким образом, 
задача минимизации мощности представляется в виде 
$$
P^*=\min\limits_{c_{k,n}\in\mathcal{D}}\sum\limits_{n=1}^N \sum\limits_{k=1}^K 
\fr{f_k(c_{k,n})}{a_{k,n}^2}
$$ 
c ограничением для $k$-поль\-зо\-ва\-те\-ля по числу бит для передачи 
$R_k=\sum\limits_{n=1}^N c_{k,n}$.
  
  Исследованный в~[10--12] алгоритм задания чис\-ла бит и уровня мощности в сети с 
одним пользователем служит основой для решения задачи минимиза\-ции мощ\-ности для 
случая многопользовательской сети. Этот алгоритм относится к так называемым 
<<жад\-ным>> алгоритмам и назначает бит под\-не\-су\-щей, требующей выделения 
наименьшей дополнительной мощности. Процесс назначения происходит по одному биту 
за один раз до тех пор, пока $R$ бит не будут распределены между $N$ поднесущими.
  
  Решение оптимизационной задачи для случая многопользовательской сети 
предусматривает использование действительных значений для числа бит/символ 
$c_{k,n}\in \mathbb{R}[0,\,M]$, а также введение функции назначения $k$-поль\-зо\-ва\-те\-лю 
  $n$-под\-не\-су\-щей, $\rho_{k,n}\in \mathbb{R}[0,\,1]$. Тогда задача минимизации 
мощности принимает вид:
$$
p^*=\min\limits_{c_{k,n},\rho_{k,n}} \sum\limits_{n=1}^N 
\sum\limits_{k=1}^K \fr{\rho_{k,n} f_k(c_{k,n})}{a_{k,n}^2}
$$
c ограничениями $\sum\limits_{n=1}^N \rho_{k,n}c_{k,n}=R_k$ и $\sum\limits_{k=1}^K 
\rho_{k,n}=1$. Данное предположение позволяет решить задачу назначения поднесущей 
  $k$-пользователю методом множителей Лагранжа по алгоритму, приведенному в~[10]. 
В~рамках решения задачи оптимизации полученные величины определяют нижнюю 
границу искомого минимального значения выделяемой мощности. Однако из-за 
принятого ранее предположения о величинах $c^*_{k,n}\not\in \mathbb{Z}$ и 
$\rho_{k,n}^*\in \mathbb{R}[0,\,1]$ предполагается разделение поднесущей между 
несколькими пользователями. Решение данной проблемы методом квантования 
полученных величин может не удовлетворять требованию $k$-пользователя к скорости 
передачи~$R_k$. В~предположении, что $\rho_{k^\prime,n}^*=1$, 
$k^\prime\hm=\mathrm{arg}\,\max\limits_k \rho_{k,n}^*$ и $\rho^*_{k,n}=0$, 
$k\not=k^\prime$, алгоритм назначения $n$-поднесущей в~[10] дополняется алгоритмом 
задания числа бит и уровня мощности в сети с одним пользователем. В~результате в~[10] 
предложен многопользовательский адаптивный алгоритм (MAO, Multiuser Adaptive 
OFDM).
  
  Данный поход к решению задачи оптимизации, согласно~\cite{13-gai}, относят к 
методу релаксаций. Использование нецелого числа бит и разделения поднесущей между 
пользователями позволяет эффективно решать задачу оптимизации, однако требует 
применения дополнительных процедур для получения целых величин, являющихся 
целесообразными с точки зрения функционирования сети. Двумя другими методами, 
предложенными в~\cite{13-gai}, являются разбиение задачи на несколько более прос\-тых и 
эвристический алгоритм. Первый метод предполагает определение числа поднесущих для 
$k$-поль\-зо\-ва\-те\-ля с учетом требований к скорости передачи $R_k$ и далее назначение 
конкретных, выбранных по некоторому алгоритму, поднесущих. Эвристический подход 
основывается на методе сортировки и представляет собой реализацию двухэтапного 
аналитического метода, описанного выше. Решение задачи оптимизации с помощью 
эвристического метода представлено также в~\cite{14-gai}.

\subsection{Задача максимизации скорости передачи}

  В~[15] нелинейная оптимизационная задача преобразована в линейную задачу 
максимизации скорости передачи путем равномерного разделения общей мощности 
$p_{tot}$ между пользователями в сети для каждой из поднесущих: 

\noindent
$$
p_{k,n}=\fr{p_{tot}}{N}\,.
$$ 
Задача максимизации общей пропускной способности сети представлена в виде:

\noindent
$$
R^*=\max\limits_{c_{k,n}\in\mathcal{D}} \sum\limits_{n=1}^N \sum\limits_{k=1}^K 
c_{k,n} \rho_{k,n}\,,
$$ 
принимая во внимание требование $r_k$, предъявляемое $k$-поль\-зо\-ва\-те\-лем 
к минимальной ско\-рости передачи чис\-ла бит на один OFDM-сим\-вол, 

\noindent
$$
\sum\limits_{n=1}^N c_{k,n}\rho_{k,n}\geq r_k\,.
$$
  
  Фиксируя уровень выделяемой мощности $p_{k,n}$ и допуская, что значения 
заданного для $k$-поль\-зо\-ва\-те\-ля коэффициента ошибок BER (Bit Error Rate) и состояния 
канала $a_{k,n}$ известны на базовой станции для всех пользователей, находятся 
значения числа бит/символ 

\noindent
$$
c_{k,n}=f(\mathrm{BER}, p_{k,n}, a_{k,n})\,.
$$

Данный подход позволяет 
эффективно решить линейную задачу оптимизации методом це\-ло\-чис\-лен\-но\-го линейного 
программирования, однако предусматривает экспоненциальный рост уровня сложности с 
увеличением числа поднесущих и пользователей в сети.
  
  В~\cite{15-gai} предложен алгоритм понижения слож\-ности, состоящий из двух этапов: 
назначение под-\linebreak несущих пользователям с наибольшим воз\-мож-\linebreak ным числом бит для 
передачи и перераспределение поднесу\-щих для соблюдения требований~$r_k$. 
%
Поднесущие на первом этапе распределяются между пользователями с целью достижения 
максимальной пропускной способности без учета требований~$r_k$ к минимальной 
скорости передачи. 
%
Для перераспределения пользователей на втором этапе требуется 
соблюдение следующих условий:
  \begin{enumerate}[(1)]
\item выделенная на первом этапе $k_n^*$-поль\-зо\-ва\-те\-лю $n$-поднесущая не может 
быть переназначена другим пользователям, если переназначение повлечет возможное 
нарушение требования~$r_k$ к минимальной скорости передачи 
$k_n^*$-поль\-зо\-ва\-те\-ля, $R_{k_n^*,n}\hm-c_{k_n^*,n}\hm<r_{k_n^*}$;
\item каждое переназначение поднесущих должно минимально сокращать общую 
пропускную способность сис\-темы;
\item число переназначений должно быть наименьшим.
\end{enumerate}

  Для выполнения последних двух условий вводится функция $e_{k,n}= (c_{k_n^*,n}-
c_{k,n})/c_{k,n}$ оцен-\linebreak ки переназначения $n$-поднесущей $k$-пользователю.\linebreak
 Согласно 
алгоритму~[15] перераспределение происходит поочередно для всех пользователей в сети, 
которым назначены поднесущие после первого этапа, не удовлетворяющие требованиям 
по скорости передачи. При этом для $k$-поль\-зо\-ва\-те\-ля выбирается поднесущая с 
наименьшей функцией оценки переназначения. Перераспределение $k$-поль\-зо\-ва\-те\-ля на 
$n^\prime$-под\-не\-су\-щую происходит, если $R_{k^*_{n^\prime},n^\prime}-
c_{k^*_{n^\prime},n}\geq r_{k^*_{n^\prime}}$. В~противном случае выбирается другая 
поднесущая с минимальной функцией перераспределения.

\subsection{Задача обеспечения равнодоступности}

  В~[16] приведены три из наиболее известных схем распределения поднесущих между 
пользователями. 

Согласно первой из них обеспечение максимальной пропускной 
способности (maxBR, maximum bit-rate) достигается за счет предоставления\linebreak 
  $n$-поднесущей $k$-пользователю, находящемуся в лучших канальных условиях, т.\,е.\ 
обладающему наибольшим частотным откликом (англ.\ \textit{frequency response}) 
$H_{k,n}$. Следует отметить, что данный метод не решает задачу обеспечения 
равнодоступности. Однако, рассматривая величину частотного отклика в качестве 
единственного параметра при распределении поднесущих, метод maxBR определяет 
верхнюю границу возможной скорости передачи данных.
  
  Вторая схема распределения канальных ресурсов, изложенная в~[17], предполагает 
решение задачи обеспечения равнодоступности путем предо\-став\-ле\-ния одинаковой 
скорости передачи всем пользователям. Помимо ограничения по мощности передачи 
$\sum\limits_{k=1}^K \sum\limits_{n=1}^N p_{k,n}\leq p_{\mathrm{tot}}$ в~[17] также вводится 
пропорциональное ограничение: 
$$
N_1:N_2:\ldots:N_K=\phi_1:\phi_2:\ldots\phi_K\,,
$$ 
где 
$N_k\hm=\phi_kN$~--- чис\-ло поднесущих, назначенных $k$-поль\-зо\-ва\-те\-лю, и $\phi_k$~--- 
нормированная пропорциональная постоянная скорости передачи данных для 
  $k$-поль\-зо\-ва\-те\-ля. При этом $\tilde{n}=N/K$ определяет максимально возможное чис\-ло 
выделяемых пользователю поднесущих.
  
  Третья схема назначения поднесущих, предложенная в~[18], заключается в выборе 
$\tilde{n}$ поднесущих с наибольшими значениями частотных откликов для пользователя. 
Данная процедура повторяется для всех пользователей в сети.
  
  Введем величину отношения сигнал-шум (SNR, Signal-to-Noise Ratio), используемую 
для постановки оптимизационной задачи:
$$
\mathrm{SNR}_{k,n}=\fr{\vert H_{k,n}\vert^2 
p_{k,n}}{\sigma_{k,n}^2}\,.
$$ 
Задача максимизации общей пропускной способности всех 
пользователей в сети формулируется в виде:
$$
F^*=\max\limits_{\mathrm{SNR}_{k,n}} 
\sum\limits_{k=1}^K \sum\limits_{n=1}^N f(\mathrm{SNR}_{k,m})\,.
$$
Следует отметить, что подобные 
задачи не учитывают улучшения пропускной способности отдельных пользователей. 
В~[16] приводится решение данной оптимизационной задачи с учетом обеспечения 
рав\-но\-до\-ступ\-ности путем назначения равного чис\-ла поднесущих по одному из двух 
алгоритмов, кратко охарактеризованных ниже, а далее путем использования обобщенного 
алгоритма задания чис\-ла бит/символ для каждой поднесущей.
  
  Первый алгоритм заключается в сравнении и выборе поднесущей с наибольшим 
частотным откликом, а также ее назначении пользователю. В~ходе данного назначения 
пользователь, получивший оптимальное число поднесущих, удаляется из 
рассматриваемого множества. Данная процедура продолжается для всех оставшихся 
пользователей и поднесущих. 

Второй алгоритм заключается в поиске для первого 
пользователя максимального частотного отклика и назначении ему соответствующей 
поднесущей. После выделения по одной поднесущей каждому из $K$ пользователей 
данный алгоритм повторяется в противоположном порядке: от $K$-го до 1-го 
пользователя.

\subsection{Задача максимизации полезности}

  Следует отметить, что все рассмотренные выше задачи подразумевают оптимизацию 
некоторой функции полезности, описывающей тот или иной уровень удовлетворенности 
пользователей для определенной схемы распределения радиоресурсов при некоторых 
ограничениях. Тем не менее, как будет показано ниже, оптимизация полезности может 
оказаться задачей, представляющей самостоятельный интерес.
  
  В~[19, 20] предложены два алгоритма межуровневой оптимизации, предназначенные для 
максимизации функции полезности (Utility Function) в различных условиях~--- алгоритм 
динамического назначения поднесущих DSA (Dynamic Subcarrier Assignment) и 
алгоритм адаптивного распределения мощности APA, а 
также комбинация этих алгоритмов. Эффективность алгоритмов оценивалась с помощью 
имитационного моделирования, при котором они сравнивались с алгоритмом 
фиксированного назначения поднесущих FSA (Fixed Subcarrier Assignment). Ниже для 
алгоритмов DSA и APA сформулированы две задачи нелинейного целочисленного 
программирования, для которых получены условия оптимальности. Вводятся следующие 
обозначения: $\mathcal{N}$~--- множество поднесущих $\{1,\ldots ,N\}$; 
$\mathcal{K}$~--- множество пользователей $\{1,\ldots ,K\}$; $\beta$~--- коэффициент 
побитовой ошибки (BER); $\mathbf{p}=(p[1],\ldots ,p[N])$~--- вектор мощностей\linebreak 
поднесущих; $\rho$~--- состояние поднесущей (отношение сиг\-нал--шум); $\Delta f$~--- 
ширина полосы пропускания поднесущей; $c_k^p[n]$~--- достижимая эффективность 
передачи (бит/символ); $r_k$~--- скорость \mbox{передачи} $k$-пользователя, $k\in\mathcal{K}$.
  
  Предполагается, что нисходящий канал базовой станции ячейки сети OFDM
используется всеми пользователями, причем базовой станции известно со\-сто\-яние 
назначенной пользователю поднесущей. Достижимая скорость передачи данных зависит 
от отношения сиг\-нал--шум и мощности передачи: 

\vspace*{4pt}

\noindent
$$
c_k^p[n]= f(\log_2(1+\beta 
p[n]\rho_k[n])\,,
$$
где функция $f(\cdot)$ зависит от выбранной схемы адап\-та\-ции скорости. 
Например, если применять непрерывную адаптацию скорости, то $f(x)=x$ и 
$c_k^p[n]=\log_2(1+\beta p[n]\rho_k[n])$.
  
  Введем $x_{kn}\in \{0,\,1\}$ состояние $n$-поднесущей так, что $x_{kn}\hm=1$, если 
  $n$-под\-не\-су\-щая назначена $k$-поль\-зо\-ва\-те\-лю, и $x_{kn}\hm=0$ в противном случае. Тогда 
$x_k\hm=(x_{kn})_{n\in\mathcal{N}}$~--- вектор состояния поднесущих для 
  $k$-пользователя, причем условие $\sum\limits_{k\in\mathcal{K}} x_{kn}=1$ означает, 
что $n$-поднесущая может быть назначена только одному пользователю. Множество 
$\mathrm{D}_k(\mathbf{x}_k) =\{n:\ x_{kn}=1\}$ включает все поднесущие, назначенные 
$k$-пользователю в состоянии $\mathbf{x}_k$, а набор множеств 
$\mathrm{D}(\mathbf{x})=(\mathrm{D}_k (\mathbf{x}_k))_{k\in \mathcal{K}}$ определяет 
распределение поднесущих по пользователям, когда система находится в состоянии 
$\mathbf{x}=(\mathbf{x}_k)_{k\in\mathcal{K}}$. Тогда множество состояний системы 
можно определить в виде:

\vspace*{-4pt}

\noindent
  \begin{multline*}
  \mathrm{X}=\left\{ \vphantom{\mathop{\bigcup}\limits_{k\in \mathcal{K}}}
  \mathbf{x} =(\mathbf{x}_k)_{k\in\mathcal{K}}:\ 
\mathrm{D}_i(\mathbf{x}_i) \cap \mathrm{D}_j(\mathbf{x}_j) =\varnothing\,,\right.\\[-3pt]
  \left. i\not=j\in\mathcal{K}\,,\enskip \mathop{\bigcup}\limits_{k\in \mathcal{K}} 
\mathrm{D}_k(\mathbf{x}_k) \subseteq \mathcal{N}\right\}\,.
  \end{multline*}
  
%  \pagebreak
  
     Введем множество всех возможных наборов поднесущих  
$\mathcal{D}\hm=\{D(\mathbf{x}):\ \mathbf{x}\hm\in \mathcal{X}\}$ и множество возможных 
вариантов распределения мощностей 
$$
\mathcal{P}=\{\mathbf{p}:\ 0\leq p(n)\leq P, \ 
\sum\limits_{n\in N} p(n)=P\,, \  n\in \mathcal{N}\}\,.
$$

%\columnbreak

Скорость передачи данных $r_k$ [бит/с] 
для\linebreak $k$-поль\-зо\-вателя в состоянии $\mathbf{x}_k$ представима в виде:
     \begin{multline*}
     r_k {:=} r_k (\mathbf{x}_k,\mathbf{p}) =\sum\limits_{n\in\mathcal{N}} 
c_k^{\mathbf{p}}(n) \Delta f x_{kn}={}\\
     {}=\sum\limits_{n\in \mathcal{D}_k(\mathbf{x}_k)}  c_k^{\mathbf{p}}(n)\Delta f =r_k 
(\mathcal{D}_k (\mathbf{x}_k),\mathbf{p}) {=:} r_k(\mathcal{D}_k, \mathbf{p})\,,\\ k\in 
\mathcal{K}\,.
     \end{multline*}
     
     Пусть $U_k(\cdot)$~--- функция полезности для $k$-поль\-зо\-ва\-те\-ля, $k\hm\in 
\mathcal{K}$. Будем рассматривать в качестве основного блага для пользователя величину 
скорости передачи данных $r_k$ и определим функцию полезностив следующем виде:
     $$
     U(r(d,\mathbf{p})) {:=} \sum\limits_{k\in \mathcal{K}} 
U_k\left(r_k\left(D_k,\mathbf{p}\right)\right)\,.
     $$
     
     Таким образом, задача межуровневой оптимизации в общем случае может быть 
сформулирована как максимизация функции полезности для ячейки сети OFDM в виде:
     $$
     \max\limits_{d,\mathbf{p}}\sum\limits_{k\in \mathcal{K}} U_k(r(d,\mathbf{p}))
     $$
с ограничениями $d\in \mathcal{D}$ и $\mathbf{p}\in \mathcal{P}$.
     
     Пусть вектор \textbf{p} распределения мощностей фиксирован, т.\,е.\ 
$\mathbf{p}=\tilde{\mathbf{p}}$. Тогда функцию полезности для алгоритма 
динамического назначения поднесущих DSA можно определить в виде:
    $$
     U(r(d)){:=} U\left(r\left(d,\tilde{\mathbf{p}}\right)\right)\,.
     $$
     
     В случае алгоритма адаптивного распределения мощностей APA фиксированным 
является набор $d$ множеств поднесущих $d=\tilde{d}$, и, следовательно, функция 
полезности имеет вид:
     $$
     U(r(\mathbf{p})) {:=}  U\left( r\left( \tilde{d},\mathbf{p}\right)\right)\,.
     $$
  
  Отметим, что данная задача относится к классу задач целочисленного нелинейного 
программирования. В~[19] для решения этой задачи используется метод релаксаций, идея 
которого заключается в том, что при разработке метода решения задач отбрасывается 
требование к целочисленности переменных. Предполагается, что функция полезности 
$U_k(r_k)$ для $k$-пользователя является неубывающей выпуклой и существует ее 
производная $U^\prime_k(r_k)$.

\bigskip

\noindent
\textit{Алгоритм динамического назначения поднесущих} DSA
    
    
    \vspace*{2pt}
     
     С учетом введенных обозначений задача максимизации полезности для алгоритма 
DSA записывается в виде $\max\limits_{d\in \mathcal{D}} U(r(d))$.
     
     Используя метод математической индукции можно доказать, что максимум функции 
по\-лез\-ности $U(r(d))$ достигается в сформулированных ниже достаточных условиях 
оптимальности.


\medskip

\noindent
\textbf{Утверждение 1.} Если для набора $d^*=(D^*_k)_{k\in \mathcal{K}}$ выполняется 
условие:
$$
U^\prime_k (r^*_k) c_k^{\tilde{\mathbf{p}}}(n)\geq U^\prime_j (r^*_j) 
c_j^{\tilde{\mathbf{p}}}(n)\,,\enskip  k\not=j\in \mathcal{K}\,,\ n\in D_k^*\,,
$$
где $r_k^*=\sum\limits_{n\in D_k^*} c_n^{\tilde{\mathbf{p}}}(n)\Delta f$, тогда функция 
полезности $U(r(d))$ достигает глобального максимума на наборе $d=d^*\in 
\mathcal{D}$.
     
     \medskip
     
     Из полученных условий оптимальности получаем правило назначения поднесущей 
пользователю. Для заданного вектора $\mathbf{p}=\tilde{\mathbf{p}}$ распределения 
мощности номер пользователя, которому назначается $n$-поднесущая, определяется 
формулой:
     $$
     k(n) =\mathrm{arg}\max\limits_{k\in\mathcal{K}} \left\{ U^\prime_k (r_k^*) 
c_k^{\tilde{p}}(n)\right\}\,.
     $$

\medskip

\noindent
\textit{Алгоритм адаптивного распределения мощности} APA

\smallskip

     Алгоритм адаптивного распределения мощ\-ности APA состоит в назначении 
каждой поднесущей $n\in \mathcal{N}$ определенной мощности передачи при условии 
фиксированного набора $d=\tilde{d}$ распределения поднесущих между пользователями.
     
  Задача межуровневой оптимизации для алгоритма APA может быть сформулирована 
в виде $\max\limits_{p\in \mathcal{P}} U(r(\mathbf{p}))$. Ниже сформулированы 
необходимые условия достижения максимума функцией полезности $U(r(\mathbf{p}))$.
  
  \medskip
  
  \noindent
\textbf{Утверждение 2.} Если $p^*(n)$  является решением задачи 
$\max\limits_{\mathbf{p}\in \mathcal{P}} U(r(\mathbf{p}))$, тогда
$$
p^*(n)=\left[ \fr{U^\prime_k(r^*_k)}{\lambda}-\fr{1}{\beta \rho_k(n)}\right]^*\,,\enskip k\in 
\mathcal{K}\,,\ n\in \tilde{D}_k\,,
$$
где $\lambda>0$~--- нормирующая константа оптимального распределения мощностей.
     
     Из полученных условий оптимальности функции $U(r(\mathbf{p}))$ очевидным 
образом следует правило назначения мощностей поднесущих для алгоритма APA.

\section{Заключение}
  
  Планирование межуровневого интерфейса является наиболее эффективным подходом к 
согласованию возможностей современных беспроводных технологий и возрастающих 
требований по обслуживанию больших объемов трафика пользователей с заданным 
качеством. 
  
  В обзоре технические и алгоритмические проблемы создания планировщиков 
межуровневого интерфейса иллюстрированы постановками оптимизационных задач, 
возникающих при распределении ресурсов в сетях с большим числом несущих 
радиочастот. Рассмотрены наиболее известные задачи и ограничения, характерные для 
технологии LTE, даны краткие комментарии по их решению, алгоритмам поиска 
оптимального решения и условиям оптимальности. Приведен типичный пример задачи 
оптимизации функции полезности как наиболее общей задачи оптимального 
планирования межуровневого интерфейса. Исследовано и сформулировано достаточное 
условие нахождения глобального максимума функции полезности для задачи DSA, а 
также необходимое условие для задачи APA.

{\small\frenchspacing
{%\baselineskip=10.8pt
\addcontentsline{toc}{section}{Литература}
\begin{thebibliography}{99}

\bibitem{1-gai}
\Au{Shariat M., Quddus A.\,U., Ghorashi~S.\,A., Tafazolli~R.}
 Scheduling as an important cross-layer operation for emerging broadband wireless systems~// 
IEEE Commun. Surveys Tuts., 2009. Vol.~11. No.\,2. P.~74--86.
\bibitem{2-gai}
\Au{Вишневский В.\,М., Ляхов А.\,И., Портной~С.\,Л., Шахнович~И.\,В.}
Широкополосные беспроводные сети передачи информации.~--- М.: Техносфера, 2005. 
597~c.
\bibitem{3-gai}
\Au{Тихвинский В.\,О., Терентьев С.\,В., Юрчук~А.\,Б.} Сети мобильной связи LTE: 
технология и архитектура.~--- М.: Эко-Трендз, 2010. 284~с.
\bibitem{4-gai}
\Au{Wu G., Bai~Y., Lai~J., Ogielski~A.} Interaction between TCP and RLP in wireless 
Internet~// IEEE Global Communication Conference Proceedings, 1999. Vol.~1b. P.~661--666.
\bibitem{5-gai}
\Au{Kim B.\,J.} A~network service providing wireless channel information for adaptive mobile 
applications: Part~I: Proposals~// IEEE  Conference (International) on Communications 
Proceedings, 2001. Vol.~5. P.~1345--1351.
\bibitem{6-gai}
\Au{Sudame P., Badrinath~B.\,R.}
On providing support for protocol adaptation in mobile networks~// Mobile Networks 
Applications, 2001. Vol.~6. No.\,1. P.~43--55.
\bibitem{7-gai}
\Au{Chiang M.} To layer or not to layer: Balancing transport and physical layers in wireless 
multihop networks~// IEEE J.~Selected Areas  Commun., 2005. Vol.~23. No.\,1. 
P.~104--116. 
\bibitem{8-gai}
\Au{Kawadia V., Kumar P.\,R.}
A~cautionary perspective on cross-layer design~// IEEE Wireless Commun., 2005. Vol.~12. 
No.\,1. P.~3--11.
\bibitem{9-gai}
\Au{Iannone L., Fdida S.} Evaluating a cross-layer approach for routing in wireless mesh 
networks~// Telecommunication Systems J. (Springer) Special issue: Next Generation 
Networks~--- Architectures, Protocols, Performance, 2006. Vol.~31. No.\,2--3. P.~173--193.
\bibitem{10-gai}
\Au{Wong C.\,Y., Cheng R.\,S., Letaief~K.\,B.} Multiuser OFDM with adaptive subcarrier, bit, 
and power allocation~// IEEE J. Selected Areas  Commun., 1999. Vol.~17. No.\,10. 
P.~1747--1757.
\bibitem{11-gai}
\Au{Hughes-Hartogs D.} Ensemble modem structure for imperfect transmission media. U.S.\ 
Patents No.\,4679227, July 1987; No.\,4731816, March 1988; No.\,4833796, May 1989.
\bibitem{12-gai}
\Au{Lai S.\,K., Cheng R.\,S., Letaief K.\,B., Murch~R.\,D.} Adaptive trellis coded MQAM and 
power optimization for OFDM transmission~// IEEE Trans. Commun., 1999. Vol.~47. 
P.~538--545.
\bibitem{13-gai} %12
\Au{Bohge M., Gross J., Wolisz~A., Meyer~M.} Dynamic resource allocation in OFDM Systems: 
an overview of cross-layer optimization principles and techniques~// IEEE Networks, 2007. 
Vol.~21. No.\,1. P.~53--59.
\bibitem{14-gai}
\Au{Kivanc D., Li G., Liu~H.}
Computationally efficient bandwidth allocation and power control for OFDMA~// IEEE 
Trans. Wireless Commun., 2003. Vol.~2. No.\,6. P.~1150--1158.
\bibitem{15-gai}
\Au{Zhang Y.\,J., Letaief K.\,B.} Multiuser adaptive subcarrier and bit allocation with adaptive 
cell selection for OFDM systems~// IEEE Trans. Wireless Commun., 2004. 
Vol.~3. No.\,5. P.~1566--1575.
\bibitem{16-gai}
\Au{Otani Y., Ohno S., Teo K., Teo~D., Hinamoto~T.}
Subcarrier allocation for multi-user OFDM system~// Asia-Pacific Communication Conference 
Proceedings, 2005. P.~1073--1077.
\bibitem{17-gai}
\Au{Wong C., Shen Z., Evans L., Andrews~J.\,G.} A~low complexity algorithm for proportional 
resource allocation in OFDMA systems~// IEEE Workshop on Signal Processing Systems 
Proceedings.~--- Texas, USA, 2004. P.~1--6.
\bibitem{18-gai}
\Au{Fu J., Karasawa Y.} Fundamental analysis on throughput characteristics of orthogonal 
frequency division multiple access OFDMA in multipath propagation environments~// IEICE  
Trans., 2002. Vol.~J85-B. No.\,11. P.~1884--1894.

\label{end\stat}
\bibitem{19-gai}
\Au{Song G., Li~Ye.}
Cross-layer optimization for OFDM wireless networks~--- Part~I: Theoretical framework~// 
IEEE Trans. Wireless Commun., 2005. Vol.~4. No.\,2. P.~614--624.


\bibitem{20-gai}
\Au{Song G., Li Y.} Cross-layer optimization for OFDM wireless networks~--- Part~II: 
Algorithm development~// IEEE Trans. Wireless Commun., 2005. Vol.~4. 
No.\,2. P.~625--634.
%\bibitem{21-gai}
%\Au{Глебов Н.\,И., Кочетов Ю.\,А., Плясунов~А.\,В.}
%Методы оптимизации: Учебное пособие.~--- Новосибирск: НГУ, 2000. 105~с.
 \end{thebibliography}
}
}


\end{multicols}  %1 Gai+sam+shorgin  
\def\ld{\ldots}
\def\d{\overline d}

\def\stat{raz-rus}

\def\tit{СТАЦИОНАРНЫЕ ХАРАКТЕРИСТИКИ СИСТЕМЫ ОБСЛУЖИВАНИЯ
С~ИНВЕРСИОННЫМ ПОРЯДКОМ ОБСЛУЖИВАНИЯ, ВЕРОЯТНОСТНЫМ
ПРИОРИТЕТОМ И~ГРУППОВЫМ ПОСТУПЛЕНИЕМ РАЗНОРОДНЫХ ЗАЯВОК$^*$}

\def\titkol{Стационарные характеристики системы обслуживания
с~инверсионным порядком обслуживания} %, вероятностным приоритетом и~групповым поступлением разнородных заявок}

\def\aut{Р.\,В.~Разумчик$^1$}

\def\autkol{Р.\,В. Разумчик}

\titel{\tit}{\aut}{\autkol}{\titkol}

\index{Разумчик Р.\,В.}
\index{Razumchik R.\,V.}



{\renewcommand{\thefootnote}{\fnsymbol{footnote}} \footnotetext[1]
{Работа выполнена при поддержке Российского научного фонда (проект 16-11-10227).}}


\renewcommand{\thefootnote}{\arabic{footnote}}
\footnotetext[1]{Институт проблем информатики Федерального исследовательского центра <<Информатика 
и~управ\-ле\-ние>> Российской академии наук; Российский
университет дружбы народов, \mbox{rrazumchik@ipiran.ru}}
%; \mbox{razumchik\_rv@rudn.university}}

\vspace*{-16pt}



\Abst{Статья посвящена исследованию стационарных характеристик
однолинейных систем массового обслуживания (СМО)
со специальными дисциплинами обслуживания.
Рассматриваемая дисциплина~--- инверсионный
порядок обслуживания с~вероятностным приоритетом.
Основные результаты для данной дисциплины
были получены в~предположениях, что в~систему поступает
пуассоновский поток или поток фазового типа и~времена
обслуживания имеют произвольное распределение.
Существенным также было предположение о~независимости процесса поступления заявок
от состояния системы. Здесь же показано, что это
предположение может быть определенным образом ослаблено.
Рассматривается система с~одним прибором, очередью неограниченной
емкости и~неординарным пуассоновским потоком,
интенсивность которого может зависеть от общего числа заявок, находящихся
в системе в~момент поступления группы, причем
размер  поступающей группы и~размеры заявок
в~ней имеют совместное произвольное распределение.
Получены аналитические соотношения, позволяющие
рассчитывать совместное стационарное распределение числа заявок
в~системе и~остаточных времен обслуживания.
Кроме того, в~терминах преобразований Лап\-ла\-са--Стилть\-еса (ПЛС)
находятся стационарные распределения
случайных величин, связанных с~временем ожидания начала обслуживания
и~пребывания заявки в~системе.}


\KW{инверсионный порядок обслуживания; вероятностный приоритет; 
неординарный входящий поток}

%\vspace*{-8pt}

\DOI{10.14357/19922264170402} 


\vskip 10pt plus 9pt minus 6pt

\thispagestyle{headings}

\begin{multicols}{2}

\label{st\stat}

\section{Введение}

Эта статья развивает результаты работ~\cite{n0,n1,n2,n3,n4,n5} по исследованию
стационарных характеристик однолинейных СМО
$M/G/1$ с~инверсионным порядком обслуживания и~вероятностным приоритетом.
Основные результаты этих работ были получены в~предположении, что входящий в~систему поток
является простейшим. 

Как было продемонстрировано в~\cite{nm1},
некоторые из этих результатов допускают обобщение на случай
потоков фазового типа, которые не являются рекуррентными и,~таким образом,
могут быть более привлекательными при моделировании процессов в~реальных технических системах.
Несмотря на свою общность, модель потока фазового типа
подразумевает, что процесс поступления заявок в~сис\-те\-му не зависит от состояния самой системы.
Тем\linebreak самым в~стороне осталась задача обобщения результатов на случай,
когда такая зависимость присутствует.
Не останавливаясь на возможных практических интерпретациях
связей между входящим\linebreak потоком и~состоянием системы (см.~\cite{gg2}),
 отметим лишь, что исследованию СМО с~такими зависимостями посвящено достаточно много работ
(см., например,~\cite{gg1,gg3,gg4,gg5} и~ссылки в~них). Обычно
предполагается, что в~систему поступает пуассоновский поток второго рода
(т.\,е.\ интенсивность потока зависит от общего числа заявок, находящихся в~сис\-те\-ме). 
Если же
допускается поступление групп заявок, то обычно предполагается, что размеры (остаточные времена обслуживания)
заявок в~группе являются независимыми случайными величинами (не зависящими также и~от размера группы).

В~данной статье эти предположения ослабляются следующим образом:
рассматривается неординарный пуассоновский поток,
интенсивность которого может зависеть от общего числа заявок, находящих\-ся
в~системе в~момент поступления группы, причем
размер  поступающей группы и~размеры\linebreak заявок в~ней имеют совместное произвольное распределение.
Для однолинейной СМО неограниченной емкости с~инверсионным порядком обслуживания 
и~вероятностным приоритетом при таком\linebreak входя\-щем потоке
решена задача отыскания совместного стационарного
распределения числа заявок в~сис\-те\-ме и~их остаточного времени обслуживания,
а~также стационарных распределений (в~терминах ПЛС),
связанных с~временем пребывания заявки в~системе.

\vspace*{-4pt}

\section{Описание системы}

\vspace*{-2pt}

Рассмотрим однолинейную СМО с~очередью неограниченной емкости,
на вход которой поступает групповой пуассоновский поток заявок с~переменной 
интенсивностью~$\lambda_n$, зависящей от числа заявок~$n$, находящихся в~системе.
Через $B_k(x_1,\ld,x_k)$ будем обозначать вероятность того, что в~поступившей
группе будет~$k$ заявок, причем первая заявка будет иметь
длину меньше~$x_1$, вторая~--- меньше $x_2$ и~т.\,д.;
через $b_k(x_1,\ld,x_k)\hm=
\partial^k B_k(x_1,\ld,x_k)/(\partial x_1\cdots \partial x_k)$~--- 
совместную плотность вероятностей.
Длины заявок в~различных группах независимы между собой.

Определим дисциплину обслуживания сле\-ду\-ющим образом:
в~момент прихода очередной группы заявок замеряется
остаточное время обслуживания (в дальнейшем будем называть его
длиной) первой заявки из группы.
Пусть она равна~$x$. Эта длина сравнивается с~остаточной длиной
заявки, находящейся на обслуживании. Если
оставшееся время обслуживания заявки на приборе равно~$y$,
то с~вероятностью $d(x,y)$ первая заявка из группы становится на
обслуживание, за ней (в очередь) становятся остальные заявки группы,
затем обслуживавшаяся ранее и~остальные заявки, прежде находившиеся в~системе.
С~вероятностью $\d(x,y)\hm=1\hm-d(x,y)$ обслуживавшаяся ранее заявка продолжает
обслуживаться на приборе, вновь поступившие заявки становятся (в~очередь) за ней,
затем остальные находившиеся прежде в~системе заявки.
Недообслуженные заявки дообслуживаются.

Поскольку  интерес представляют стационарные характеристики
этой системы, всюду в~дальнейшем будем предполагать,
что стационарное распределение существует.
Критерий его существования следует
из условия конечности среднего времени
возвращения в~некоторое состояние (см.\ соотношение~\eqref{uns}).
Однако в~случае произвольных функций $B_k(x_1,\ld,x_k)$
выписать его в~простом виде не удается.

\vspace*{-4pt}

\section{Марковский случайный процесс}

\vspace*{-2pt}

Обозначим через $\nu(t)$ число заявок в~системе
в момент~$t$, а через $\vec\xi(t)\hm=
(\xi_{1}(t),\ldots,\xi_{\nu(t)}(t))$~---
вектор, координатой~$\xi_{1}(t)$ которого
является остаточное время обслуживания
заявки, находящейся в~этот момент на приборе,
$\xi_{2}(t)$~--- первой заявки в~очереди$,\ldots,$ $\xi_{\nu(t)-1}(t)$~---
последней, \mbox{$(\nu(t)-1)$-й} заявки в~очереди.
При $\nu(t)\hm=0$ вектор~$\vec\xi(t)$
не определяется.
Тогда $\eta(t)\hm=(\nu(t),\vec\xi(t))$ представляет
собой марковский процесс, описывающий
поведение числа заявок в~рассматриваемой системе.

\vspace*{-4pt}

\section{Система интегродифференциальных уравнений}

\vspace*{-2pt}

Обозначим через $p_{0}\hm=\lim\nolimits_{t\to\infty}
{\bf P}\{\nu(t)\hm=0\}$,
$P_{n}(x_1,\ldots,x_{n})
\hm= \lim\nolimits_{t\to\infty} {\bf P}\{\nu(t)\hm=n,\,
\xi_{1}(t)\hm<x_{1},\ldots,\xi_{n}(t)<x_{n}\}$, $n \hm\ge 1$,
стационарное распределение процесса~$\eta(t)$,
а~через $p_n(x_1,\ld,x_n)$~---\linebreak стационарную плот\-ность
вероятности того, что в~сис\-те\-ме~$n$~заявок, причем заявка на приборе имеет
длину~$x_1$, первая в~очереди~--- длину~$x_2$ и~т.\,д.

Выпишем систему интегродифференциальных 
уравнений, которой удовлетворяют $p_n(x_1,\ldots,x_{n})$.
Воспользовавшись методом исключения со\-сто\-яний (см., например,~\cite{n1,n2,n3,n4,n5}),
получаем следующие соотношения:
\begin{multline}
\label{e1}
-p'_1(x) = \lambda_0 p_0 \left( 
\vphantom{\mathop{\int\!\cdots\!\int}\limits_{y_1,\ld,y_{k-1}>0}}
b_1(x) +{}\right.\\
\left.{}+
\sum\limits_{k=2}^\infty\!\!
\mathop{\int\!\cdots\!\int}\limits_{y_1,\ld,y_{k-1}>0}\!\!
b_k\left(y_1,\ld,y_{k-1},x\right)\, dy_1\cdots dy_{k-1}
\!\right)-{}\\
{}- \lambda_1 p_1(x) +
\lambda_1 p_1(x)\times{}\\
{}\times \sum\limits_{k=1}^\infty
\mathop{\int\!\cdots\!\int}\limits_{y_1,\ld,y_{k}>0}
b_k\left(y_1,\ld,y_{k}\right)\, d\left(y_1,x\right)  dy_1\cdots dy_{k}
+{}\\
{}+\lambda_1 \int\limits_0^\infty p_1(y) b_1(x) \d(x,y) \, dy
+{}\\
{}+
\lambda_1 \sum\limits_{k=2}^\infty
\mathop{\int\!\cdots\!\int}\limits_{y_1,\ld,y_{k}>0}
p_1\left(y_1\right)  b_k\left(y_2,\ld,y_{k},x\right) \times{}\\
{}\times\d\left(y_2,y_1\right)
\, dy_1\cdots dy_{k}\,;
\end{multline}

\vspace*{-12pt}

\noindent
\begin{multline*}
-p'_n\left(x_1,\ld,x_n\right)
=  \lambda_0 p_0 \left(
\vphantom{\mathop{\int\!\cdots\!\int}\limits_{y_1,\ld,y_{k-1}>0}\sum\limits^\infty}
b_n\left(x_1,\ld,x_n\right) +{}\right.\\
{}+ \sum\limits_{k=n+1}^\infty
\mathop{\int\ld\int}\limits_{y_1,\ld,y_{k-n}>0}\!\!\!
b_k\left(y_1,\ld,y_{k-n},x_1,\ld\right.
\end{multline*}

\noindent
\begin{multline}
\left.\left.{}\ld,x_{n}\right)\, dy_1\cdots dy_{k-n}
\vphantom{\mathop{\int\!\cdots\!\int}\limits_{y_1,\ld,y_{k-1}>0}\sum\limits^\infty}
\!\right)\!
+\!
\sum\limits_{k=1}^{n-1}\! \lambda_k d\left(x_1,x_{n-k+1}\right) 
\times{}\\
{}\times b_{n-k}\left(x_1,\ld,x_{n-k}\right) p_k\left(x_{n-k+1},\ld,x_{n}\right)
+{}
\\
{}+
 \sum\limits_{k=1}^{n-1} \lambda_k \d(x_2,x_{1})
b_{n-k}\left(x_2,\ld,x_{n-k+1}\right)\times{}\\
{}\times
p_k\left(x_1,x_{n-k+2},\ld,x_{n}\right) -
\lambda_n p_n\left(x_1,\ld,x_n\right) +{}
\\
{}+
 \sum\limits_{k=1}^{n} \int\limits_0^\infty \lambda_k
\d\left(x_{1},y\right) b_{n-k+1}\left(x_1,\ld,x_{n-k+1}\right)\times{}\\
{}\times
p_k\left(y,x_{n-k+2},\ld,x_{n}\right)\, dy +{}
\\
{}+
\lambda_n p_n(x) \sum\limits_{k=1}^\infty
\int\limits_0^\infty b_{k1}(y)\, d(y,x) dy
+{}
\\
{}+\!
\sum\limits_{k=1}^{n-1}\! \int\limits_0^\infty\!\! \lambda_k 
d\left(y,x_{n-k+1}\right)
b_{n-k+1}\left(y,x_1,\ld,x_{n-k}\right)\times{}\\
{}\times
p_k\left(x_{n-k+1},\ld,x_{n}\right)\, dy
+{}
\\
{}+
\lambda_k \sum\limits_{k=1}^{n}
\sum\limits_{m=2}^{\infty}
\mathop{\!\int\cdots\!\int}\limits_{y_1,\ld,y_{m}>0}
\d\left(y_2,y_{1}\right)\times{}\\
{}\times
b_{n-k+m}\left(y_2,\ld,y_m,x_1,\ld,x_{n-k+1}\right)\times{}\\
{}\times
p_k\left(y_1,x_{n-k+2},\ld,x_{n}\right)\, dy_1\cdots dy_m
+{}
\\
{}+
\lambda_k \sum\limits_{k=1}^{n}
\sum\limits_{m=2}^{\infty}
\mathop{\int\!\cdots\!\int}\limits_{y_1,\ld,y_{m}>0}
d\left(y_1,x_{n-k+1}\right)\times{}\\
{}\times
b_{n-k+m}\left(y_1,\ld,y_m,x_1,\ld,x_{n-k}\right)\times{}\\
{}\times
p_k\left(x_{n-k+1},\ld,x_{n}\right)\, dy_1\cdots dy_m\,,
\enskip
n\ge2\,.
\label{e2}
\end{multline}



\noindent
К этой системе уравнений нужно добавить граничные условия,
которые удобно записать в~виде:
\begin{equation*}
%\label{(3.3)}
\lim\limits_{x\to \infty} p_{1}(x) = 0\,;                         %      \eqno(3.3)
\enskip
%\label{(3.4)}
\lim\limits_{x\to \infty} p_{n}(x,x_2,\ld,x_n) =
0\,,\enskip n\ge 2\,.               %       \eqno(3.4)
\end{equation*}

\noindent
Полученные соотношения позволяют теоретически последовательно по~$n$
вычислять совместное стационарное распределение $p_n(x_1,\ldots,x_{n})$
с точностью до вероятности~$p_0$, которая находится из условия нормировки.
Если достаточно знать только маргинальные плотности

\vspace*{2pt}

\noindent
\begin{equation*}
p_n(x) = \int\limits_0^\infty\!\! \cdots\!\! \int\limits_0^\infty
p_n\left(x,x_2,\ld,x_n\right)\, dx_2\cdots dx_n\,,
\enskip n\ge 2\,,
\end{equation*}

\noindent то полученные соотношения можно упросить.
Введем обозначения:

\vspace*{-2pt}

\noindent
\begin{multline}
\label{new6}
b_{k,m}(x)
= \mathop{\int\!\cdots\!\int} \limits_{y_1,\ld,y_{k-1}>0}\!\!
b_k\left(y_1,\ld,y_{m-1},x,y_{m},\ld\right.\\
\left.\ld,y_{k-1}\right)\, dy_1\cdots dy_{k-1}\,,
\enskip k\ge 2\,,\enskip m=\overline{1,k}\,;
\end{multline}

\noindent
\begin{equation}
\label{new7-1}
b_{2,1,2}(y,x) = b_2(y,x)\,;
\end{equation}
\begin{multline}
\label{new7}
b_{k,1,m}(y,x)
=  \mathop{\int\!\cdots\!\int}
\limits_{y_1,\ld,y_{k-2}>0}\!\!
b_k\left(y,y_1,\ld\right.\\
\left.\ld,y_{m-2},x,y_{m-1},\ld,y_{k-2}\right)
\, dy_1\cdots dy_{k-2}\,,\\
k\ge 3\,,\enskip m=\overline{2,k}\,.
\end{multline}

\noindent 
Интегрируя~\eqref{e1} и~\eqref{e2} по
$x_2,\ldots ,x_n$ в~пределах от~0 до~$\infty$ и~учитывая 
обозначения~\eqref{new6}--\eqref{new7}, получаем следующую
систему интегродифференциальных уравнений
для~$p_{n}(x)$, $n\hm\ge 1$:
\begin{align}
-p'_1(x) &=a_1(x)
-\lambda_1 p_1(x)+{}\notag\\
&{}+\lambda_1\int\limits_0^\infty
p_1(y)K(x,y)\, dy+\lambda_1p_1(x)g_{1}(x)\,, \label{sys1}
\\
-p'_n(x)&=a_n(x)-\lambda_n p_n(x)+{}\notag\\
&\hspace*{-13mm}{}+\lambda_n \int\limits_0^\infty p_n(y) K(x,y) \, dy +
\sum\limits_{k=1}^{n-1} \lambda_k \left(
\vphantom{\int\limits_0^\infty}
p_k(x) g_{n,k}(x)+ {}\right.\notag\\
&\left.{}+\int\limits_0^\infty p_k(y) G_{n,k}(x,y) \, dy
\right)\,, \enskip n \ge 2\,;
\label{sys2}
\end{align}
где
$$
a_1(x) = \lambda_0 p_0
\left( b_1(x) + \sum\limits_{k=2}^\infty b_{kk}(x) \right)\,;
$$
$$
a_n(x)= \lambda_0 p_0 \left(
b_{n1}(x) + \sum\limits_{k=n+1}^\infty b_{k,k-n+1}(x)
\right)\,;
$$
$$
K(x,y)= \d(x,y) b_{1}(x) + \sum\limits_{k=2}^{\infty}
\int\limits_0^\infty \d(z,y) b_{k,1,k}(z,x) \, dz\,;
$$
$$
g_{1}(x)= \int\limits_0^\infty
 d(y,x) b_{1}(y)\, dy, + \sum\limits_{k=2}^\infty
\int\limits_0^\infty  d(y,x) b_{k1}(y)\, dy\,;
$$
$$
g_{n,n-1}(x)= \int\limits_0^\infty \d(y,x)
b_{1}(y)\, dy\,; 
$$
$$
g_{n,k}(x)= \int\limits_0^\infty \d(y,x) b_{n-k,1}(y) \, dy\,, \enskip 
k=\overline{1,n-2}\,;
$$
$$
G_{n,n-1}(x,y)= d(x,y) b_{1}(x);
$$

\vspace*{-12pt}

\noindent
\begin{multline*}
G_{n,k}(x,y)= d(x,y) b_{n-k,1}(x) + \d(x,y) b_{n-k+1,1}(x)
+{}\\
{}+ \int\limits_0^\infty d(z,y) b_{n-k+1,1,2}(z,x)\, dz +{}
\\
{}+
\sum\limits_{m=2}^{\infty}
\int\limits_0^\infty
\left( \d(z,y)
b_{n-k+m,1,m}(z,x)
+{}\right.\\
\left.{}+ d(z,y) b_{n-k+m,1,m+1}(z,x) \right)\, dz\,, \enskip
 k=\overline{1,n-2}\,.
\end{multline*}

\noindent
Отметим, что все функции, входящие в~интегральные уравнения \eqref{sys1} и~\eqref{sys2},
являются неотрицательными.
Полученная система решается рекуррентным образом.
Граничные условия имеют вид:
$$
\lim\limits_{x\to \infty} p_{n}(x)= 0\,,\enskip
n\ge 1\,.
$$

 Сначала определяем
$p_1(x)$ из~\eqref{sys1}, затем $p_2(x)$ через $p_1(x)$ из~\eqref{sys2} 
при $n\hm=2$ и~т.\,д.
Предварительно можно выполнить замену \mbox{$p_n(x) \hm= e^{\lambda x} q_n(x)$}
и~проинтегрировать новые уравнения от~0 до~$\infty$ с~учетом граничных условий.
Чис\-лен\-ное решение может быть найдено, например, методом итераций, причем
в качестве начальной итерации удобно взять нулевое приближение.

В заключение этого раздела отметим,
что, если для функции~$d(x,y)$ известна соответству\-ющая
сепарабельная аппроксимация (см., например,~\cite{n5,n6,n7,n8}), 
в~некоторых случаях (как, например, при выполнении приводимых ниже\linebreak условий~\eqref{us1})
уравнения~\eqref{sys1} и~\eqref{sys2} сводятся к~сис\-те\-ме линейных алгебраических 
уравнений.

\section{Производящая функция}

В ряде случаев решение уравнений~\eqref{sys1} и~\eqref{sys2} может быть
найдено в~терминах производящих функций (ПФ), что облегчает нахождение моментов\linebreak
чис\-ла заявок в~сис\-те\-ме.
Разберем один из них~--- случай группового пуассоновского потока постоянной
интенсивности, в~котором длины заявок в~по\-сту\-па\-ющей
группе не зависят друг от друга и~от размера группы, т.\,е.
\begin{equation}
\left.
\begin{array}{rlrl}
\hspace*{-2mm}\lambda_k&=\lambda\,, & k&\ge 0\,;\\[6pt]
\hspace*{-2mm}B_k\left(x_1,\ld,x_k\right)&=
c_k B\left(x_1\right)\cdots B\left(x_k\right)\,, &
k &\ge 1\,,
\end{array}
\right\}
\label{us1}
\end{equation}

\noindent где $B(x)$~--- непрерывная функция распределения времени обслуживания
одной заявки на приборе, $c_k \hm\ge 0$ и~$\sum\nolimits_{k=1}^\infty c_k\hm=1$.
Необходимым и~достаточным условием существования стационарного режима 
(и~это будет дополнительно показано ниже\footnote{Этот результат  
также следует из сравнения суммарной работы в~рассматриваемой системе и~классической 
системе $M/G/1$ с~групповым входящим потоком и~обслуживанием в~порядке поступления.})
является $\lambda \overline{c}  \overline{b}\hm < 1$,
где $\overline{b}\hm=\int\nolimits_0^\infty x  dB(x)$~--- 
средняя длина поступающей заявки, а~$\overline{c}\hm=C'(1)$~--- 
средний размер поступающей группы заявок.


Введем обозначения:
\begin{gather*}
H^*(z)=\sum\limits_{n=0}^\infty  P_n z^n=
P_0+H(z)\,; \\
h(z,x)=\sum\limits_{n=1}^\infty  p_n(x) z^n\,, \enskip
C(z)= \sum\limits_{n=1}^\infty  c_n z^n\,,
\end{gather*}

\noindent где $P_n\hm=P_n(\infty,\ld,\infty)$, $n \hm\ge 1$.
Умножив уравнение~\eqref{sys1} на~$z$, а~\eqref{sys2}~---
на~$z^n$, просуммировав и~проинтегрировав
с~учетом граничного условия $\lim\nolimits_{x\to \infty} h(z,x)\hm= 0$,
получаем уравнение:
\begin{multline}
h(z,x) = \lambda p_0 (1-B(x)) \fr{z (1-C(z))}{1-z}
+ {}\\[2pt]
{}+
\lambda (1-B(x))  H(z) \left ( C(z) + c_1 + \fr{z^2 - C(z)}{z(1-z)} \right )
- {}
\\[2pt]
{}- \lambda (1- C(z)) \left(
\int\limits_x^\infty \int\limits_0^\infty \d(t,y) h(z,y)\, dy  dB(t) -{}\right.\\[2pt]
\left.{}-
\int\limits_0^\infty \int\limits_x^\infty \d(t,y) h(z,y)\, dy  dB(t)
\right)
+{}
\\[2pt]
{}+
\lambda \fr{C(z)-c_1 z}{z} \int\limits_x^\infty
\int\limits_0^\infty h(z,y) \left(
\vphantom{\int\limits_0^\infty}
\d(t,y) +{}\right.\\[2pt]
\left.{}+ \int\limits_0^\infty d(u,y) \,dB(u) \right)
dy  dB(t)\,.
\label{pf}
\end{multline}

\noindent В случае ординарного потока ($c_1 \hm\equiv 1$) из этого уравнения 
немедленно следует ПФ
числа заявок в~системе, рассмотренной в~\cite{n3}. Трактуя~$z$ как параметр,
для решения уравнения~\eqref{pf} можно применить метод, описанный 
в~предыдущем разделе.

Задачу нахождения моментов стационарного распределения
числа заявок в~системе рассмотрим на примере математического ожидания
и~ограничимся лишь описанием алгоритма его расчета.

Будем считать, что операции дифференцирования, которые будут применены ниже, законны.
Проинтегрируем~\eqref{pf} по~$x$ от~0 до~$\infty$ и~найдем~$H(z)$.
Продифференцировав выражение для $(1-z)H(z)$ два раза и~положив $z\hm=1$, получим формулу
для расчета среднего числа заявок $\mathbf{E}\nu$ в~системе с~двумя
неизвестными: $h(1,x)$ и~$h'(1,x)\hm=\partial h(z,x) / \partial z |_{z=1}$.
Их нахождение осуществляется в~два этапа.
Сначала выписывается выражение для~$H(1)$,
затем, подставив $z=1$ и~найденное выражение для~$H(1)$
в~\eqref{pf},\linebreak получается интегральное уравнение для $h(1,x)$,
чис\-лен\-ное решение которого можно найти, например, как и~выше,
итерационным методом. Таким же образом, но предварительно продифференцировав~\eqref{pf} 
по~$z$, находится и~уравнение для~$h'(1,x)$.

Необходимым условием существования~$\mathbf{E}\nu$ является условие:

\pagebreak



\noindent
\begin{equation}
\label{sred}
\overline{c}
\int\limits_0^\infty
\int\limits_0^\infty
\int\limits_x^\infty
\d(t,y)
(1-B(y)) \,dy  dB(t)
dx < \infty\,.
\end{equation}


\noindent Показать это можно так же, как и~для системы из~\cite{n3}.
Предположим, что $\mathbf{E}\nu\hm< \infty$. Тогда $(H^*(1))'\hm=H'(1)<\infty$.
Поскольку
\begin{equation}
\label{hzx}
h(z,x)\le h(1,x)\le \lambda \overline{c} (1-B(x)) \,,
\end{equation}
то
$$
\int\limits_0^\infty
\int\limits_x^\infty
\int\limits_0^\infty
\d(t,y) h(z,y) \, dy 
dB(t) dx
\le \lambda  \overline{c}  \overline{b}^2 < \infty\,.
$$

\noindent Интегрируя теперь~\eqref{pf} по~$x$ от~0 до~$\infty$,
получаем:
\begin{multline*}
\fr{ \lambda \overline{b} (1-z H^*(z))}{ 1-z}+
\left ( z(1-C(z))(1-z) \right )^{-1}\times{}\\
{}\times \left \{
\vphantom{\int\limits_0^\infty}
H(z)(1-z)z - \lambda \overline{b} z (1-C(z))
\right.
+{}
\\
{}+
\lambda(1-z) (C(z)-c_1 z)\times{}\\
{}\times
\left[
\int\limits_0^\infty
\int\limits_x^\infty
\int\limits_0^\infty
h(z,y) \left(
\d(t,y) + {}\right.\right.\\
\left.\left.\left.{}+\int\limits_0^\infty d(u,y) \, dB(u)
\right)\,
dy  dB(t)
dx -
\overline{b} H(z)
\right ] \right\}
+ {}\\
{}+ \lambda \int\limits_0^\infty
\int\limits_x^\infty \int\limits_0^\infty
\d(t,y) h(z,y) \,dy dB(t) dx
={}\\
{}= \lambda \int\limits_0^\infty \int\limits_0^\infty
\int\limits_x^\infty \d(t,y) h(z,y) \,dy  dB(t) dx\,.
%\label{neob}
\end{multline*}

\noindent Поскольку левая часть ограничена и~$h(z,x) \hm\rightarrow h(1,x)$ 
при $z \hm\rightarrow 1$,
воспользовавшись теоремой Фату и~учитывая~\eqref{hzx}, приходим к~\eqref{sred}.

Достаточность показать сложнее, 
и~ввиду громоздкости выкладок на этом здесь останавливаться не будем.
Заметим, что для выполнения~\eqref{sred} достаточно (помимо конечности среднего
размера группы) существования у~распределения времени обслуживания~$B(x)$ 
второго момента.

\section{Время пребывания заявки в~системе}

Вернемся к~основной исследуемой системе.
Расчет временных характеристик поступающих в~сис\-те\-му заявок
начинается с~нахождения периода занятости (ПЗ) и~его характеристик.
Обозначим через $u_n(s;x)$, $n\hm\ge 1$, ПЛС функции распределения
времени до того момента,
когда в~системе останется $(n\hm-1)$ заявок при условии,
что на приборе начала обслуживаться заявка длины~$x$
и~в~системе находилось~$n$~заявок.
Уравнение для $u_n(s;x)$ получается из следующих рассуждений:
за время обслуживания заявки длины~$x$
с~вероятностью $e^{- \lambda_n x}$ не поступит больше ни одной
заявки, а~с~вероятностью $\lambda_n e^{- \lambda_n t}\,dt$
на интервале времени $[t,t+dt]$ может поступить
группа размером $k\hm\ge1$. В~первом случае ПЛС
равно~$e^{-sx}$, а~во втором зависит от размера
поступающей группы и~того, произошла смена заявки на приборе
или нет (и~в~каждом случае необходимо
дождаться окончания обслуживания исходной заявки длины~$(x\hm-t)$ и~$k$~новых заявок).
Рассматривая все возможные события и~воспользовавшись свойствами ПЛС, получаем:
\begin{multline}
u_{n}(s;x) = e^{- (\lambda_n+s) x}
+{}
\\
{}+
\sum\limits_{k=1}^\infty
\int\limits_0^x \lambda_n e^{- (\lambda_n+s) t} \, dt
\mathop{\int\!\cdots\!\int}\limits_{y_1,\ld,y_{k}>0}
d\left(y_1,x-t\right)\times{}\\
{}\times u_n(s;x-t)
\prod\limits_{j=1}^k u_{n+k+1-j}\left(s;y_j\right) B_k\left(dy_1,\dots,dy_k\right)
+{}
\\
{}+
\sum\limits_{k=1}^\infty \int\limits_0^x
\lambda_n e^{- (\lambda_n+s) t} \, dt
\mathop{\int\!\cdots\!\int}\limits_{y_1,\ld,y_{k}>0}
\d\left(y_1,x-t\right)\times{}\\
{}\times u_{n+k}(s;x-t) 
\prod\limits_{j=1}^k u_{n+k-j}\left(s;y_j\right)\times{}\\
{}\times
B_k\left(dy_1,\dots,dy_k\right)\,.
\label{uns}
\end{multline}

\noindent
Решение этого интегрального уравнения в~явном виде для произвольных 
функций $B_k(x_1,\ld,x_k)$
получить не удается. Однако в~некоторых частных случаях оно разрешимо, как, например,
в~случае условий~\eqref{us1}. Здесь $u_{n}(s;x)$ не зависит
от~$n$ и,~как нетрудно вывести из~\eqref{uns},
$$
u_n(s;x)=u(s;x)=e^{-\left ( \lambda + s - \lambda C(u(s))\right ) x}\,,
$$
где $\beta(s)$~--- ПЛС функции распределения~$B(x)$,
а~$u(s)$ является корнем уравнения:
$$
u(s) = \beta \left (\lambda+s-\lambda C (u(s)) \right )\,.
$$

Кроме того, ПЛС $u^*(s)$ функции распределения
ПЗ\footnote{Для основной
исследуемой системы ПЛС ПЗ равен $\sum\nolimits_{k=1}^\infty
\mathop{\int\!\cdots\!\int}\nolimits_{y_1,\ld,y_{k}>0}
\prod\nolimits_{n=1}^k u_{n}(s;y_{k-n+1})
B_k(dy_1,\dots,dy_k)$.}
удовлетворяет уравнению $u^*(s)\hm=C\left ( \beta(\lambda\hm+s\hm-\lambda u^*(s)) \right)$.

Для нахождения распределений времен ожидания начала обслуживания 
и~пребывания в~системе введем следующие функции:
\begin{description}
\item[\,] ${\tilde B}(k,i,x)$~--- вероятность того, что пришла группа из~$k$~заявок 
и~$i$-я заявка в~группе имеет длину меньше~$x$:
\begin{multline*}
{\tilde B}(k,i,x)=B_k(\infty,\dots,\infty,x,\infty, \dots, \infty)\,,\\ 
k \ge 1\,,\enskip 1 \le i \le k\,;
\end{multline*}

\item[\,] ${\bar B}(x_1,\dots,x_{i-1};k,i,x)$~---
условная 
вероятность\footnote{Здесь производная понимается
как производная Ра\-до\-на--Ни\-ко\-ди\-ма.} того, что первая заявка имеет
длину меньше~$x_1$, вторая~--- меньше~$x_2$, $\dots$, $(i-1)$-я~--- меньше $x_{i-1}$,
при условии, что пришла группа из~$k$~заявок, причем заявка на $i$-м месте имеет
длину~$x$:
\begin{multline*}
{\bar B}(x_1,\dots,x_{i-1};k,i,x)={}\\
{}=\fr{d_x B_k(x_1,\dots,x_{i-1},x,\infty, 
\dots, \infty)}{d {\tilde B}(k,i,x)}\,;
\end{multline*} 

\item[\,] ${\hat B}(x)$~---  среднее число заявок длины
меньше~$x$ в~поступающей группе:
$$
{\hat B}(x) = \sum\limits_{k=1}^\infty 
\sum\limits_{i=1}^k {\tilde B}(k,i,x)\,;
$$

\item[\,] ${\hat B}(k,i;x)$~---
условная вероятность того, что поступила группа из~$k$~заявок, среди них
есть ровно одна заявка длины~$x$ и~она находится на $i$-м месте, при условии что
поступила группа, в~которой имеются заявки длины~$x$:
$$
{\hat B}(k,i;x)=\fr{d_x {\tilde B}(k,i,x)}{{\hat B}(x)}\,, \enskip
k \ge 1,\enskip 1 \le i \le k\,.
$$

\end{description}

Определим сначала ПЛС $\omega_{k1}(s;x)$ функции распределения
времени ожидания начала обслуживания заявки длины~$x$ при условии,
что она поступила в~группе размера $k\hm\ge$ и~была на первом месте в~группе.
Ее время ожидания равно нулю, если она застала систему свободной 
и~если она, застав на приборе заявку длины~$y$, заняла ее место.
Если же она застала в~системе~$n$, $n\hm\ge 1$, заявок, на приборе~---
заявку длины~$y$ и~не заняла ее место,
то время ожидания совпадает с~ПЗ, открываемого заявкой длины~$y$,
когда в~системе находится $(n+k)$ заявок, т.\,е.\ $u_{n+k}(s;y)$. 
В~терминах ПЛС имеем
\begin{multline*}
\omega_{k1}(s;x) = p_0 + \sum\limits_{n=1}^\infty
\int\limits_0^\infty p_n(y) \left( 
\vphantom{'d}
d(x,y) +{}\right.\\
\left.{}+ \d(x,y) u_{n+k}(s;y) \right )\,
dy\,, \enskip k \ge 1\,.
\end{multline*}

Перейдем к~ПЛС времени
ожидания начала обслуживания заявки длины~$x$,
поступившей в~группе из~$k$~заявок ($k\hm\ge2$)
и~занимающей в~группе $i$-е мес\-то ($2 \hm\le i\hm\le k$).
В~случае поступления в~пустую систему
время ожидания совпадает с~суммарной дли\-тель\-ностью $(i-1)$-го
ПЗ, первый из которых от\-кры\-вается заявкой длины~$x_1$,
второй~--- $x_2$ и~т.\,д., и~в~терми\-нах ПЛС
равно $u_{k}(s;x_1)\cdots u_{2}(s;x_{i-1})$.
Длительности соответствующих ПЗ необходимо добавить к~времени
ожидания, когда по\-сту\-па\-ющая группа застает сис\-те\-му занятой.
В~итоге, вводя обозначение 

\noindent
$$
{\tilde u}_{nk}(s;x_1,\dots,x_{i-1})
=u_{n+k}(s;x_1)\cdots u_{n+2}(s;x_{i-1})\,,
$$
выражение для ПЛС функции распределения времени ожидания начала обслуживания
$\omega_{ki}(s;x_1, \dots, x_{i-1},x)$ заявки длины~$x$,
поступившей в~группе из~$k$~заявок и~занимающей в~группе $i$-е место,
можно записать так:

\noindent
\begin{multline*}
\omega_{ki}(s;x_1, \dots, x_{i-1},x) =
p_0 {\tilde u}_{0k}(s;x_1,\dots,x_{i-1})
+{}
\\
{}+
\sum\limits_{n=1}^\infty \int\limits_0^\infty p_n(y) \left (
d\left(x_1,y\right) {\tilde u}_{nk}\left(s;x_1,\dots,x_{i-1}\right)
+ {}\right.\\
\left.{}+\d\left(x_1,y\right) u_{n+k}(s;y)
{\tilde u}_{n-1,k}\left(s;x_1,\dots,x_{i-1}\right)
\right )\,dy\,,
\\
 k \ge 2\,, \enskip 2 \le i \le k\,.
\end{multline*}

\noindent
Легко видеть, что если
интенсивность входящего потока не зависит от числа заявок в~системе,
то при фиксированном~$i$ все $u_n(s;x_i)$ равны между собой
и~выражение $\omega_{ki}(s;x_1, \dots, x_{i-1},x)$ приводится к~виду:

\noindent
\begin{multline*}
\omega_{ki}(s;x_1, \dots, x_{i-1},x)={}\\
{}=
\omega_{k1}(s;x_1)u(s;x_1)\cdots u(s;x_{i-1})\,,
\end{multline*}
т.\,е.\ не зависит от числа заявок в~группе, а~только от места выделенной
заявки в~группе (и,~конечно, остаточных длин заявок, стоящих перед ней).

Теперь можно выписать ПЛС распределений,
связанных с~временем ожидания начала обслуживания и~пребывания в~системе.
Условно стационарное распределение времени ожидания
начала обслуживания заявки длины~$x$
при условии, что всего в~группе поступило $k\hm\ge2$ заявок
и~заявка длины~$x$ находится на $i$-м месте ($2 \hm\le i \hm\le k$),
имеет ПЛС, задаваемое выражением:

\noindent
\begin{multline}
\omega_{ki}(s;x)
=
\int\limits_0^\infty\!\cdots\!\int\limits_0^\infty
\omega_{ki}\left(s;x_1, \dots, x_{i-1},x\right)\times{}\\
{}\times
{\bar B}\left(dx_1,\dots,dx_{i-1};k,i,x\right)\,.
\label{w1}
\end{multline}

\noindent
Усредняя $\omega_{ki}(s;x)$ по распределению
${\hat B}(k,i;x)$, получаем формулу для ПЛС
$\omega(s;x)$ функции распре-\linebreak\vspace*{-12pt}

\pagebreak

\noindent
деления времени
ожидания начала обслуживания
заявки длины~$x$:
\begin{equation}
\label{w2}
\omega(s;x)
= \sum\limits_{k=1}^\infty \sum\limits_{i=1}^k \omega_{ki}(s;x) {\hat B}(k,i;x)\,.
\end{equation}


\noindent
Безусловное ПЛС~$\omega(s)$ функции распределения времени
ожидания начала обслуживания определяется путем усреднения по длине заявки:
\begin{equation}
\label{w3}
\omega(s)= \int\limits_0^\infty
\omega(s;x) d{\hat B}(x) ({\hat B}(\infty))^{-1}\,.
\end{equation}

\noindent В случае условий~\eqref{us1} все упрощается
и~формулу для $\omega(s;x)$ можно привести к~виду:
\begin{multline}
\omega(s;x)
= \fr{1}{\overline{c}}
\left (
\vphantom{\int\limits_{0}^\infty}
\omega^*(s;x) + {}\right.\\
\hspace*{-2mm}\left.{}+\fr{u(s)-C \left ( u(s) \right )}{u(s) (1- u(s))}
\int\limits_{0}^\infty \omega^*(s;y) u(s,y) b(y)\, dy \right ),\!
\label{sc1}
\end{multline}

\noindent где
$$
\omega^*(s;x)= p_0 + \int\limits_0^\infty
h(1,y) \left ( d(x,y) + \d(x,y) u(s;y)
\right ) \,dy\,.
$$

Аналогичным образом находится и~ПЛС $\phi(s;x)$ функции распределения
времени пребывания заявки длины~$x$ в~системе и~безусловное ПЛС~$\phi(s)$.
Обозначим через $\phi_{ki}(s;x_1, \dots, x_{i-1},x)$, $k \hm\ge 1$, $1\hm\le i \hm\le k$,
ПЛС функции распределения времени пребывания в~системе заявки длины~$x$,
поступившей в~группе из~$k$~заявок и~занимающей в~группе $i$-е место.
При $i\hm=1$ аргумент~$x_0$ опускается, т.\,е.\
 $\phi_{k1}(s;x_{0},x)\hm=\phi_{k1}(s;x)$.
Тогда
\begin{multline*}
\phi_{ki}(s;x_1, \dots, x_{i-1},x)
={}\\
{}=p_0 {\tilde u}_{0k}\left(s;x_1,\dots,x_{i-1}\right)u_{1}(s;x)
+ 
\sum\limits_{n=1}^\infty
\int\limits_0^\infty p_n(y)\times{}\\
{}\times \left(
d\left(x_1,y\right) {\tilde u}_{nk}\left(s;x_1,\dots,x_{i-1}\right)
u_{n+1}(s;x) +{}\right.
\\
{}+
\d\left(x_1,y\right) u_{n+k}(s;y) {\tilde u}_{n-1,k}\left(s;x_1,\dots,x_{i-1}\right)\times{}\\
\left.{}\times
u_{n}(s;x) %\vhpahntom*{\tilde{u}}
\right)\,dy\,,  \enskip 
k \ge 1\,, \enskip 1 \le i \le k\,.
\end{multline*}

\noindent Переход к~ПЛС $\phi(s;x)$ и~$\phi(s)$ осуществляется
по формулам~\eqref{w1}--\eqref{w3}. Если выполняются условия~\eqref{us1}, то,
вспоминая, что время пребывания заявки в~системе складывается
из времени ожидания начала обслуживания и~времени пребывания на приборе,
получаем $\phi(s;x)\hm=\omega(s;x)u(s;x)$. Дифференцируя эту формулу 
с~учетом~\eqref{sc1} необходимое число раз, нетрудно получить моменты
времени пребывания  в~системе заявки  длины~$x$.


\section{Заключение}

\vspace*{-3pt}

Используя результаты предыдущего раздела,
можно также найти совместные распределения ПЗ 
и~числа обслуженных на приборе заявок, или интервала времени,
когда в~системе находилось не менее~$n$~заявок, 
или этих обеих случайных величин и~т.\,п.

В практическом плане интерес в~дальнейшем представляет разбор
 частных случаев, т.\,е.\ анализ
стационарных характеристик системы в~различных предположениях 
о~зависимости размеров заявок внутри группы; в~теоретическом~---
обобщение полученных результатов
на случай более общего группового входящего потока, когда
в~каждой поступающей группе могут находиться подгруппы заявок
одинаковой или различной длины.

\vspace*{-18pt}

{\small\frenchspacing
 {%\baselineskip=10.8pt
 \addcontentsline{toc}{section}{References}
 \begin{thebibliography}{99}
 
 \vspace*{-3pt}
 
 \bibitem{n3}  %1
\Au{Нагоненко В.\,А.}
О~характеристиках одной нестандартной системы
массового обслуживания.~I, II~//
Изв.\ АН СССР. Технич.\ кибернет., 1981.
№\,1. С.~187--195; №\,3. С.~91--99.

\bibitem{n4} %2
\Au{Печинкин А.\,В.} Об одной
инвариантной системе массового обслуживания~//
Math.\ Operationsforsch.\ Statist.
Ser.\ Optimization, 1983. Vol.~14. No.\,3. P.~433--444.

\bibitem{n5} %3
\Au{Милованова Т.\,А., Печинкин~А.\,В.}
Стационарные характеристики системы обслуживания 
с~инверсионным порядком обслуживания, вероятностным
приоритетом и~гистерезисной политикой~//
Информатика и~её применения, 2013. Т.~7. Вып.~1. С.~22--36.




\bibitem{n1}  %4
\Au{Мейханаджян Л.\,А., Милованова~Т.\,А., Печинкин~А.\,В., Разумчик~Р.\,В.}
Стационарные вероятности состояний в~системе обслуживания с~инверсионным 
порядком обслуживания и~обобщенным вероятностным приоритетом~// Информатика 
и~её применения, 2014. Т.~8. Вып.~3. С.~16--26.

\bibitem{n2} %5
\Au{Мейханаджян Л.\,А., Милованова~Т.\,А., Разумчик~Р.\,В.}
Время ожидания в~системе обслуживания с~инверсионным 
порядком обслуживания и~обобщенным
вероятностным приоритетом~// Информатика и~её применения, 2015. Т.~9. Вып.~2. С.~14--22.

\bibitem{n0} %6
\Au{Razumchik R.} On ${M/G/1}$ queue with state-dependent heterogeneous 
batch arrivals, inverse service order and probabilistic priority~// 
AIP Conf. Proc., 2017. Vol.~1863. No.\,1. P.~090006-1--090006-3.

\bibitem{nm1} %7
\Au{Милованова Т.\,А.} Система $\mathrm{BMAP}/G/1/\infty$ с~инверсионным 
порядком обслуживания и~вероятностным приоритетом~// Автомат. телемех., 2009. 
№\,5. С.~155--168.
% Autom. Remote Control, 70:5 (2009), 885-896


\bibitem{gg2} %8
\Au{Bent N.} On a~queuing model where potential customers are discouraged by queue length~//
Scand. J.~Stat., 1975. Vol.~2. Iss.~1. P.~34--42.




\bibitem{gg3} %9
\Au{Печинкин А.\,В.} Система ${M_k/G/1}$ с~ненадежным прибором~//
Автомат. телемех., 1996. №\,9. С.~100--110.
% Autom. Remote Control, 57:9 (1996), 1302-1310



\bibitem{gg5} %10
\Au{Gupta U.\,C., Srinivasa~Rao~T.\,S.\,S.} 
On the analysis of single server finite queue with state dependent
arrival and service processes: ${M_n/G_n/1/K}$~//
OR Spektrum, 1998. Vol.~20. Iss.~2. P.~83--89.

\bibitem{gg4} %11
\Au{Kerner Y.} The conditional distribution of the residual service time in the ${M_n/G/1}$ 
queue~// Stoch. Models, 2008. Vol.~24. Iss.~3. P.~364--375.

\bibitem{gg1} %12
\Au{Abouee-Mehrizi H., Baron~O.} State-dependent ${M/G/1}$ queueing systems~//
Queueing Sy., 2016. Vol.~82. Iss.~1-2. P.~121--148.

\columnbreak

\bibitem{n6} %13
\Au{Поспелов В.\,В.}
О~погрешности приближения функции двух переменных суммами
произведений функций одного переменного~//
Ж.~вычисл. матем. матем. физ., 1978. Т.~18. Вып.~5. С.~1307--1308.
%U.S.S.R. Comput. Math. Math. Phys., 18:5 (1978), 228-230

\vspace*{7pt}

\bibitem{n8} %14
\Au{Uschmajew A.} Regularity of tensor product approximations to square
integrable functions~// Constr. Approx., 2011. Vol.~34. Iss.~3. P.~371--391.

\vspace*{7pt}

\bibitem{n7} %15
\Au{Townsend A., Trefethen~L.\,N.} An extension of Chebfun to two dimensions~// 
SIAM J.~Sci. Comput., 2013. Vol.~35. Iss.~6. P.~495--518.



 \end{thebibliography}

 }
 }

\end{multicols}

\vspace*{-3pt}

\hfill{\small\textit{Поступила в~редакцию 19.09.17}}

\vspace*{8pt}

%\newpage

%\vspace*{-24pt}

\hrule

\vspace*{2pt}

\hrule

%\vspace*{8pt}


\def\tit{$M/G/1$ QUEUE WITH~STATE-DEPENDENT 
HETEROGENEOUS BATCH ARRIVALS, 
INVERSE SERVICE ORDER, AND~PROBABILISTIC~PRIORITY}

\def\titkol{$M/G/1$ queue with~state-dependent 
heterogeneous batch arrivals, 
inverse service order, and~probabilistic priority}

\def\aut{R.\,V.~Razumchik$^{1,2}$}

\def\autkol{R.\,V.~Razumchik}

\titel{\tit}{\aut}{\autkol}{\titkol}

\vspace*{-9pt}


\noindent
$^1$Institute of Informatics Problems, Federal Research Center ``Computer Science 
and Control'' of the Russian\linebreak
$\hphantom{^1}$Academy of Sciences, 44-2~Vavilov Str., Moscow 
119333, Russian Federation

\noindent
$^2$Peoples' Friendship University of Russia (RUDN University), 
6~Miklukho-Maklaya Str., Moscow 117198, Russian\linebreak
$\hphantom{^1}$Federation



\def\leftfootline{\small{\textbf{\thepage}
\hfill INFORMATIKA I EE PRIMENENIYA~--- INFORMATICS AND
APPLICATIONS\ \ \ 2017\ \ \ volume~11\ \ \ issue\ 4}
}%
 \def\rightfootline{\small{INFORMATIKA I EE PRIMENENIYA~---
INFORMATICS AND APPLICATIONS\ \ \ 2017\ \ \ volume~11\ \ \ issue\ 4
\hfill \textbf{\thepage}}}

\vspace*{3pt}






\Abste{Consideration is given to the stationary characteristics
of single-server queues with the queue of infinite capacity,
independent and identically-distributed service times, 
LCFS (last-come-first-served) service order, and probabilistic priority discipline.
Most of the results for such type of queueing systems
have been obtained under the 
assumption of either Poisson arrivals or 
phase-type arrivals. 
Another important assumption made was that
the arrival process is independent 
from the system state. The author shows that the
latter assumption can be relaxed to some, quite large extent.
The author considers an $M/G/1/\infty$ queue  
with batch Poisson arrival flow in which ($i$)~the arrival rate depends
on the total number of customers present in the system
at the arrival instant; and ($ii$)~the size of the arriving batch~$k$ 
and the remaining service times $x_1,\dots,x_k$ of the customers in the batch
have the arbitrary continuous joint probability distribution
$B_k(x_1,\ld,x_k)$. The author obtains analytic expressions
for the computation of the joint stationary distribution 
of the total number of customers in the system 
and their remaining service times. 
Busy period, waiting and sojourn time distributions
are also given in terms of the Laplace--Stieltjes transforms.}

\KWE{queueing system; LIFO; probabilistic priority; batch 
arrival; state-dependent Poisson flow}


  \DOI{10.14357/19922264170402} 

\vspace*{-16pt}

\Ack
\noindent
This work was supported by the
Russian Science Foundation (grant 16-11-10227).



\vspace*{1pt}

  \begin{multicols}{2}

\renewcommand{\bibname}{\protect\rmfamily References}
%\renewcommand{\bibname}{\large\protect\rm References}

{\small\frenchspacing
 {%\baselineskip=10.8pt
 \addcontentsline{toc}{section}{References}
 \begin{thebibliography}{99}
 
 \vspace*{-3pt}
 
 \bibitem{Xx3-1} %1
\Aue{Nagonenko, V.\,A.} 1981.
O~kharakteristikakh odnoy nestandartnoy sistemy
massovogo obsluzhivaniya
[On the characteristics of one nonstandard queuing
system].~I, II.
\textit{Izv.\ AN SSSR. Tekhnich.\ kibernet}
[Proceedings of the Academy of Sciences of
the USSR. Technical Cybernetics]
1:187--195; 3:91--99.

\bibitem{Xx4-1} %2
\Aue{Pechinkin, A.\,V.} 1983.
Ob odnoy invariantnoy sisteme massovogo
obsluzhivaniya
[On an invariant queuing system].
\textit{Math.\ Operationsforsch.\ Statist.
Ser.\ Optimization} 14(3):433--444.

\bibitem{Xx5-1} %3
\Aue{Milovanova, T.\,A., and A.\,V.~Pechinkin.} 2013.
Sta\-tsi\-o\-nar\-nye kharakteristiki sistemy obsluzhivaniya
s~in\-ver\-si\-on\-nym poryadkom obsluzhivaniya,
veroyatnostnym pri\-ori\-te\-tom i~gisterezisnoy politikoy
[Stationary characteristics of queuing system with
an inversion procedure service probabilistic priority
and hysteresis policy].
\textit{Informatika i~ee Primeneniya~--- Inform. Appl.} 7(1): 22--35.





\bibitem{Xx1-1} %4
\Aue{Meykhanadzhyan, L.\,A., T.\,A.~Milovanova, A.\,V.~Pechinkin, 
and R.\,V.~Razumchik.} 2014.
Sta\-tsi\-o\-nar\-nye veroyatnosti sostoyaniy v~sisteme obsluzhivaniya s~inversionnym 
poryadkom ob\-slu\-zhi\-va\-niya i~obob\-shchen\-nym veroyatnostnym prioritetom
[Stationary distribution in a~queueing system with inverse service order and
generalized probabilistic priority].
\textit{Informatika i~ee Primeneniya~--- Inform. Appl.} 8(3):16--26.

\bibitem{Xx2-1} %5
\Aue{Meykhanadzhyan, L.\,A., T.\,A.~Milovanova, and R.\,V.~Razumchik.} 2015.
Vremya ozhidaniya v~sisteme obsluzhivaniya s inversionnym poryadkom obsluzhivaniya 
i~obobshchennym veroyatnostnym prioritetom
[Stationary waiting time in a~queueing system with inverse service order and 
generalized probabilistic priority].
\textit{Informatika i~ee Primeneniya~--- Inform. Appl.} 9(2):14--22.

\bibitem{Xn0-1} %6
\Aue{Razumchik, R.} 2017. On ${M/G/1}$ queue with state-dependent heterogeneous
 batch arrivals, inverse service order and probabilistic priority. 
 \textit{AIP Conf. Proc}. 1863(1):090006-1--090006-3.
 
 \bibitem{Xnm1-1} %7
\Aue{Milovanova, T.\,A.} 2009. 
${\mathrm{BMAP}/G/1/\infty}$ system with last come first served probabilistic priority.
\textit{Automat. Rem. \mbox{Contr.}} 70(5):885--896.

\bibitem{Xgg2-1} %8
\Aue{Bent, N.} On a~queuing model where potential customers are discouraged by queue length.
\textit{Scand. J.~Stat.} 2(1):34--42.

\bibitem{Xgg3-1} %9
\Aue{Pechinkin, A.\,V.} 1996. Sistema ${M_k/G/1}$ 
s~nenadezhnym priborom [An ${M_k/G/1}$  system with an unreliable device].
\textit{Avtomat. telemekh.} [Autom. Rem. Contr.] 9:100--110.



\bibitem{Xgg5-1} %10
\Aue{Gupta, U.\,C., and T.\,S.\,S.~Srinivasa Rao.} 1998.
On the analysis of single server finite queue with state dependent
arrival and service processes: ${M_n/G_n/1/K}$.
\textit{OR Spektrum} 20(2):83--89.

\bibitem{Xgg4-1}  %11
\Aue{Kerner, Y.} 2008. The conditional distribution of 
the residual service time in the ${M_n/G/1}$ queue.
\textit{Stoch. Models} 24(3):364--375.

\bibitem{Xgg1-1}  %12
\Aue{Abouee-Mehrizi, H., and O.~Baron.} 2016.
State-dependent ${M/G/1}$ queueing systems. 
\textit{Queueing Sy.} 82(1-2):121--148. 

\bibitem{Xn6-1} %13
\Aue{Pospelov, V.\,V.} 1978.
O~pogreshnosti priblizheniya funktsii dvukh peremennykh summami
proizvedeniy funktsiy odnogo peremennogo
[The error of approximation of a~function of two variables by sums 
of the products of functions of one variable]
\textit{Zh. vichisl. matem. matem fiz.}
[USSR\ Comput. Math. Math. Phys.] 18(5):1307--1308.

\bibitem{Xn8-1} %14
\Aue{Uschmajew, A.} 2011. Regularity of tensor product approximations to square
integrable functions.
\textit{Constr. Approx.} 34(3):371--391.

\bibitem{Xn7-1} %15
\Aue{Townsend, A., and L.\,N.~Trefethen.} 2013. 
An extension of Chebfun to two dimensions.
\textit{SIAM J.~Sci. Comput.} 35(6):495--518.


\end{thebibliography}

 }
 }

\end{multicols}

\vspace*{-6pt}

\hfill{\small\textit{Received September 19, 2017}}

%\vspace*{-10pt}

\Contrl

\noindent
\textbf{Razumchik Rostislav V.} (b.\ 1984)~--- 
Candidate of Science (PhD) in physics and mathematics, leading scientist, 
Institute of Informatics Problems, Federal Research Center 
``Computer Science and Control'' of the Russian Academy of Sciences, 
44-2~Vavilov Str., Moscow 119333, Russian Federation; 
associate professor, Peoples' Friendship University of Russia (RUDN University), 
6~Miklukho-Maklaya Str., Moscow 117198, Russian Federation; 
\mbox{rrazumchik@ipiran.ru}
%\mbox{razumchik\_rv@rudn.university}
\label{end\stat}


\renewcommand{\bibname}{\protect\rm Литература}   %2
\def\bmx{\mathbf}

\renewcommand{\figurename}{\protect\bf Figure}
\renewcommand{\tablename}{\protect\bf Table}

\def\stat{raz-1}

\def\tit{STATIONARY SOJOURN TIMES 
IN~$\mathrm{MAP}/\mathrm{PH}/1/r$ QUEUE WITH~BI-LEVEL HYSTERETIC CONTROL OF~ARRIVALS}

\def\titkol{Stationary sojourn times 
in~$\mathrm{MAP}/\mathrm{PH}/1/r$ queue with~bi-level hysteretic control of~arrivals}

\def\autkol{R.\,V.~Razumchik}

\def\aut{R.\,V.~Razumchik$^{1,2}$}

\titel{\tit}{\aut}{\autkol}{\titkol}

\index{Razumchik R.\,V.}
\index{Разумчик Р.\,В.}

%{\renewcommand{\thefootnote}{\fnsymbol{footnote}}
%\footnotetext[1] {This work was supported in part by the
%Russian Foundation for Basic Research (grants 15-07-03007 and 13-07-00223).}}

\renewcommand{\thefootnote}{\arabic{footnote}}
\footnotetext[1]{Institute of Informatics Problems, Federal Research 
Center ``Computer Science and Control'' of the Russian Academy of Sciences, 44-2~Vavilov Str., 
Moscow 119333, Russian Federation} 
\footnotetext[2]{Peoples' Friendship University of Russia (RUDN University), 6~Miklukho-Maklaya Str., 
Moscow 117198, Russian Federation}


%\vspace*{-6pt}

\def\leftfootline{\small{\textbf{\thepage}
\hfill INFORMATIKA I EE PRIMENENIYA~--- INFORMATICS AND APPLICATIONS\ \ \ 2017\ \ \ volume~11\ \ \ issue\ 4}
}%
 \def\rightfootline{\small{INFORMATIKA I EE PRIMENENIYA~--- INFORMATICS AND APPLICATIONS\ \ \ 2017\ \ \ volume~11\ \ \ issue\ 4
\hfill \textbf{\thepage}}}



%\def\l{\lambda}
%\def\m{\mu}
%\def\a{\alpha}
%\def\b{\beta}
%\def\g{\gamma}
%\def\d{\delta}
%\def\r{\overline r}
%\def\f{\overline f}
%\def\n{\nu}



\Abste{This paper reports some new results concerning the 
analysis of the time-related stationary characteristics
of a~finite-capacity queueing system 
operating in a~random environment
with the bi-level hysteretic control of arrivals.
The topic of the paper is motivated by 
the overload problem in networks of SIP (session initiation protocol) servers 
and the viewpoint that multilevel hysteretic control 
of arrivals in SIP servers
can be used to mitigate signalling network congestion.
The considered mathematical model of SIP server is the 
single server queueing system with Markovian arrival processes (MAP), PH (phase-type)
service,
and bi-level hysteretic control policy.
According to this policy, a~system may be in one
of the three operation modes: normal, overload, or blocking. 
The switching between modes occurs at 
instants whenever the total number of customers in the 
system changes.
The analytical method for the computation of the 
stationary sojourn times in different operation modes (in terms of 
Laplace--Stieltjes transforms (LST)), 
which utilizes the knowledge about the presence of 
hysteretic loops, is given. It is also applicable 
in the case when, in addition to the sojourn times,
one needs to account for the number of lost customers.}


\KWE{queueing system; random environment; first passage times; hysteretic control}

\DOI{10.14357/19922264170403}

%\vspace*{9pt}


\vskip 12pt plus 9pt minus 6pt

      \thispagestyle{myheadings}

      \begin{multicols}{2}

                  \label{st\stat}


\section{Introduction}

\noindent
This paper continues the analysis 
of the stationary finite-capacity queueing system 
operating in a~random environment
with hysteretic control of arrivals, 
which was started in~\cite{int0}.
Specifically, we deal with the $\mathrm{MAP}/\mathrm{PH}/1/r$ queue 
with two-level hysteretic control of 
arrival rates with nonoverlapping hysteretic
loops. For this system, the authors 
of~\cite{int0} proposed the 
new analytic method for the computation 
of the steady-state distribution, which
is different from the known general approaches 
for QBDs (quasi-birth-deathes). It exploits the knowledge about the hysteretic loops 
which are present in the system, has a~probabilistic interpretation
and leads to easy-to-implement computational procedures.  
We will not dwell on the motivation behind the 
analysis of this system (for details, refer~\cite{int0} and references therein)
and just mention that the various aspects of the 
topic of hysteretic control in 
queueing models still gains attention from the research 
community (see~\cite{int2, int3, int4}).

In order to make the picture clearer, let us assume 
that the control is only bi-level with nonoverlapping 
loops although all the results 
presented here (and in~\cite{int0}) can be 
generalized in a~straightforward manner for 
hysteretic control of arrivals with arbitrary number of 
nonoverlapping loops. 
Following the bi-level hysteretic control, 
the system changes its status (or mode) 
between ``normal,'' ``overload,'' and ``blocking'' 
(this will be made more precise in the next section). In each mode 
except for a~``normal'' one, server discards a~certain percentage 
of arriving customers. From a~practical point of view 
(at least the one mentioned in~\cite{int1}), 
it may be beneficial when the system spends 
as little time as possible in ``overload''/``blocking'' 
modes. This brings one to the analysis of time-related stationary
characteristics of the system, which was not carried out in~\cite{int0}.

In what follows, we are interested in the two performance characteristics 
of the hysteretic policy. The first one is the stationary
distribution of the time system spends in ``normal'' mode. 
The second one is the distribution of the time it takes the system to get back 
to ``normal'' mode\footnote[3]{Waiting and system sojourn time distributions
are of little interest since hysteretic loops have no influence on them in 
case of FIFO service policy (which is assumed).}. 

After giving the detailed system description in 
the next section, in section~3, 
the analytical method for the sequential computation
of these (sojourn time) distributions will be presented
in terms of LST.
The LST of the sojourn times are obtained as solutions to certain matrix difference equations
and are expressed in terms of recurrence relations.
They can be used for direct numerical implementation and 
numerical inversion with well-known methods (Fourier-series method with Euler 
summation, Talbot, etc.).

\begin{figure*}[b] %fig1
\vspace*{1pt}
 \begin{center}
 \mbox{%
 \epsfxsize=123.993mm 
 \epsfbox{raz-1.eps}
 }

\vspace*{6pt}

{\small Sketch of the bi-level hysteretic control of arrivals in the $\mathrm{MAP}/\mathrm{PH}/1/r$ 
system}

 \end{center}
\end{figure*}



\section{System Description and~Preliminaries}

\noindent
The system consists of a~single server and a~queue of finite capacity~$r$.
The arrival process is a~MAP with representation $(\mathbf{D_0}, \mathbf{D_1})$ 
of order~$N$.
Let us assume that an arrival, whenever it occurs, can be of one of the two types, either 
a~priority arrival or a~nonpriority. Thus, the matrix~$\mathbf{D_1}$ is assumed to
have the form $\mathbf{D_1}=\mathbf{D_{1,1}}+\mathbf{D_{1,2}}$
where $\mathbf{D_{1,1}}$ ($\mathbf{D_{1,2}}$) describes state transitions with an 
arrival of
priority (nonpriority) customer. Bi-level hysteretic control of 
arrivals is assumed to be implemented in the system. It operates as follows (see figure).
There are three operation modes for the system: ``normal,'' ``overload,'' 
and ``blocking.''
 Let~$L$ and~$H$ be arbitrary integers,
such that $0 < L < H < r+1$. Assume the system starts empty. 
As long as the total
number of customers in the system remains below~$H$,
the system is considered to be in ``normal`` mode and accepts 
all arrivals (both priority and nonpriority). 
When the total number of customers reaches~$H$ for the first time, the
system changes its mode to ``overload'' and stays in it as long
as the total number of customers remains between~$L$ and~$r$.
When overloaded, the system accepts only priority customers (nonpriority customers 
are lost)
till the total number of customers either
drops down below $L$ after which it changes its mode back to
``normal,'' or exceeds~$r$ after which it changes its state
to ``blocking.'' In the ``blocking'' mode the system does not accept
newly arriving customers until the total number of customers
drops down below $(H+1)$, after which the system changes mode 
back to ``overload`` and the process goes on.
The service time of both priority and
nonpriority customers is PH distributed with representation
$(\vec{f}, \mathbf{G})$ of order~$M$ and $\vec g=-\mathbf{G}\vec1$, and
the service policy is FIFO (first in, first out).



The operation of the considered queueing
system can be completely described
by continuous-time Markov chain 
${\bf{X}} \left( t \right)= {\left( \xi (t); \eta(t); \mu(t); \nu(t) \right) }$ with
four components:
$\xi(t)$~--- MAP generation phase at time~$t$;
$\eta(t)$~--- PH service phase at time~$t$;
$\mu(t)$~--- system's mode at time~$t$;
and $\nu(t)$~--- number of customers in
the system at time~$t$.
When $\nu(t)=0$, the second component $\eta(t)$ is omitted.
It is convenient to represent the state space of ${\bf{X}} \left( t \right)$ as
$\mathcal{X} = {\mathcal{X}_0} \cup {\mathcal{X}_1} \cup {\mathcal{X}_2}$
where $\mathcal{X}_{0} $ is the set of states of ''normal'' mode,
$\mathcal{X}_{1} $ is the set of states of ''overload'' mode,
and $\mathcal{X}_{2}$ is the set of states of ''blocking'' mode, i.\,e.,
\begin{align*}
{\mathcal{X}_0} &= \left\{ \left( {k,0,0} \right):  1 \le k \le N  \right\}
\cup 
\\
&\hspace*{5mm}\cup \left\{ {\left( {k,0,n} \right):  1 \le k \le NM, 1 \le n \le H - 1} \right\}\,;\\
{\mathcal{X}_1} &= \left\{ {\left( {k,1,n}\right): 1 \le k \le NM,  L \le n \le r} \right\}\,;\\
{\mathcal{X}_2} &= \left\{ {\left( {k,2,n} \right): 1 \le k \le NM, 
H\! + \! 1 \le n \le r \! + \! 1} \right\}\,.
\end{align*}

\noindent
Here, $k$ represents the state of 
the background (arrival and service) processes.
Indeed, the state $(k,m,n)$, $n>0$, means that 
there are~$n$ customers in the system, system's mode is~$m$,
and arrival and service phases are~$i$ and~$j$, but such that
$(i-1)M+j=k$; the state $(k,0,0)$ means that the system
is empty and the arrival phase is~$k$.

Let us denote by $\bmx{E}$ the identity matrix (its size each time will 
be clear from the context)
and let introduce the following transition rate matrices:
\begin{itemize}
    \item service of a~customer after which the system becomes empty: 
    $\bmx{P}_1=\bmx{E}\otimes {\vec g}$;
    \item service of a~customer after which the system remains busy: 
    $\bmx{P}=\bmx{P}^*=\bmx{P}^{\#}=\bmx{E}\otimes {\vec g}{\vec f}$;
    \item phase change when system is empty: $\bmx{Q}_0=\bmx{D_0}$;
    \item phase change when system is in the ``normal'' mode: 
    $\bmx{Q}=\bmx{D_0}\otimes \bmx{E} + \bmx{E} \otimes \bmx{G}$;
    \item phase change when system is in the ``overload'' mode: 
    $\bmx{Q}^*=(\bmx{D_0}+\bmx{D_{12}})\otimes \bmx{E} + \bmx{E} \otimes \bmx{G}$;
        \item arrival phase change when system is in the ``blocking'' mode: 
        $\bmx{Q}^{\#}=(\bmx{D_0}+\bmx{D_{1}})\otimes \bmx{E} + \bmx{E} \otimes 
        \bmx{G}$;
    \item arrival of a~customer to an empty system: $\bmx{R}_0\linebreak =\bmx{D_1} \otimes
    {\vec f}$,
    \item arrival of a~customer to the system in the ``normal'' mode: 
    $\bmx{R}=\bmx{D_1}\otimes \bmx{E}$; and
    \item arrival of a~customer to the system in the ``overload'' mode: 
    $\bmx{R}^*=\bmx{D_{11}}\otimes \bmx{E}$.
\end{itemize}

In order to be able to compute time-related characteristics,
in addition to transition rate matrices, one needs
transition probability matrices, which contain probabilities 
of possible state change of the background process.
Thus, let~$\alpha$, $\beta$, and $\gamma$ denote the matrices of
service, phase change, and arrival transition probabilities
when the system is in the ``normal'' mode,
i.\,e.,
\begin{alignat*}{3}
[\alpha]_{ij}  
&=
\fr{[\bmx{P}]_{ij} }{-{[\bmx{Q}]}_{ii}}\,, &\enskip 1 &\le i, j \le NM\,;
\\
[\beta]_{ij}  
&=\begin{cases}
\fr{[\bmx{Q}]_{ij}}{-{[\bmx{Q}]}_{ii}}\,,  & i \neq j\,; \\
0\,, & i = j\,,
\end{cases}
&\enskip 1 &\le i,j \le NM\,;
\\
[\gamma]_{ij}  
&=\fr{[\bmx{R}]_{ij} }{-{[\bmx{Q}]}_{ii}}\,, &\enskip 1 &\le i,j \le NM\,.
\end{alignat*}

\noindent
Here and henceforth, by $[\cdot]_{ij}$ we denote the $(i,j)${th}
 entry of any matrix.
By analogy, let us denote by~$\alpha^*$, $\beta^*$, and~$\gamma^*$
transition probabilities matrices in the ``overload'' mode,
by~$\alpha^{\#}$ and~$\beta^{\#}$
transition probability matrices  
in the ``blocking'' mode,
and by~$\alpha^{e}$, $\beta^{e}$, and~$\gamma^{e}$
transition probability matrices when
the system becomes or is empty.


\section{Sojourn Time Distributions}

\noindent
As it was mentioned in section~1,
we are interested in the two stationary characteristics:
distribution of the time system spends in ``normal'' mode
and the distribution of the time it takes the  system to get back 
to ``normal'' mode. These distributions are computed by 
conditioning on the number of customers in the system
and the state of the background process
and can be expressed in terms of
the following three quantities:
\begin{description}
\item[\,] $\bmx{V}_n(s)$, $n=\overline{0,H-1}$,~---
matrix of size $NM\times NM$,
which the $(i,j)${th} entry is
the LST of the first passage time 
to the ``overload'' mode 
and state of the background process~$j$, 
given that initially, the 
system was in the ``normal'' mode,
there where~$n$~customers in it,
and
the state of the background process was~$i$;
\item[\,] 
$\bmx{V}^{\#}_n(s)$, $n=\overline{H+1,R}$,~---
matrix of size $NM\times NM$,
which the $(i,j)${th} entry is
the LST of the first passage time 
to the ``normal'' mode 
and state of the background process~$j$, 
given that initially, the 
system was in the ``blocking'' mode,
there where~$n$~customers in it,
and
the state of the background process was~$i$; and
\item[\,] 
$\bmx{V}^*_n(s)$, $n=\overline{L,R-1}$,~---
matrix of size $NM\times NM$,
which the $(i,j)${th} entry is
the LST of the first passage time 
to the ``normal'' mode 
and state of the background process~$j$, 
given that initially, the 
system was in the ``overload'' mode,
there where~$n$~customers in it,
and
the state of the background process was~$i$.
\end{description}

The rest of the section is devoted to obtaining the relations for
$\bmx{V}_n(s)$, $\bmx{V}^*_n(s)$, and~$\bmx{V}^{\#}_n(s)$.
Let us begin with the calculation of $\bmx{V}_n(s)$,
$n=\overline{1,H-1}$. 
Denote by 
$\bmx{T}_k(s)$, $k=-1,0,1$, the 
matrix of size $NM \times NM$ 
which the $(i,j)${th} entry is 
LST of the first passage time
from the state $(i,0,n)$ 
to the state $(j,0,n-k)$.
Here,~$n$~may take any value from 
the set $\{2,3,\dots,H-2\}$. 
Remembering that the sojourn time 
in the state $(i,0,n)$ is exponential 
with rate $-[\bmx{Q}]_{ii}$, one has for 
$1 \le i,j \le NM$:
\begin{align*}
\left[ \bmx{T}_{-\!1}(s) \right]_{ij}
&=
\fr{[\bmx{Q}]_{ii} }{[\bmx{Q}]_{ii} - s}
\left[\gamma\right]_{ij}  \,;
\\
\left[ \bmx{T}_0(s) \right]_{ij}
&=
\fr{[\bmx{Q}]_{ii}}{[\bmx{Q}]_{ii} - s}
\left[\beta\right]_{ij}  \,;
\\
\left[ \bmx{T}_1(s) \right]_{ij}
&=
\fr{[\bmx{Q}]_{ii} }{[\bmx{Q}]_{ii} - s}
\left[\alpha\right]_{ij} \,.
\end{align*}

\noindent 
From the first-step analysis, let us find that
the LST $\bmx{V}_n(s)$ satisfies the system of matrix 
difference equations: 
\begin{equation}
\left.
\begin{array}{rl}
\bmx{V}_n(s)&=\bmx{T}_1(s) \bmx{V}_{n-1}(s)+ 
\bmx{T}_0(s) \bmx{V}_{n}(s)\\[6pt]
&\hspace*{1mm}{}+ \bmx{T}_{-1}(s) \bmx{V}_{n+1}(s)\,,\enskip
n=\overline{2,H-1}\,;
\\[6pt]
\bmx{V}_1(s)
&= \bmx{T}^e_1(s)  \bmx{V}_{0}(s) +  \bmx{T}_0(s)  \bmx{V}_{1}(s)\\[6pt]
&\hspace*{28mm}{} + 
\bmx{T}_{-1}(s)  \bmx{V}_{2}(s)\,.
\end{array}
\right\}
\label{eq1-r}
\end{equation}

\noindent The boundary conditions for the system~\eqref{eq1-r} 
have the form:
%%%%%%%%%%%%%%%%%%%%%%%
\begin{equation}
\left.
\begin{array}{rl}
\bmx{V}_{0}(s)&= \bmx{T}^e_0(s)  \bmx{V}_{0}(s)
+  \bmx{T}^e_{-\!1}(s)  \bmx{V}_{1}(s) \,;
\\[6pt]
\bmx{V}_{H}(s)
&=\bmx{E} \,,
\end{array}
\right\}
\label{eq12b}
\end{equation}

\noindent 
where the $(i,j)${th} entries of the matrices 
$\bmx{T}^e_{1}(s)$, $\bmx{T}^e_0(s)$, and $\bmx{T}^e_{-\!1}(s)$
are equal to 
\begin{align*}
\left[ \bmx{T}^e_1(s) \right]_{ij}
&=
\fr{[\bmx{Q}]_{ii} }{[\bmx{Q}]_{ii} - s}
\left[\alpha^{e}\right]_{ij}\,;
\\
\left[ \bmx{T}^e_0(s) \right]_{i,j}
&=
\fr{[\bmx{Q}_0]_{ii}}{[\bmx{Q}_0]_{ii} - s}
\left[\beta^{e}\right]_{ij}
\,;
\\
\left[ \bmx{T}^e_{-1}(s) \right]_{i,j}
&=
\fr{[\bmx{Q}_0]_{ii}}{[\bmx{Q}_0]_{ii} - s}
\left[\gamma^{e}\right]_{ij}\,.
\end{align*}

\noindent 
Note that here, $\bmx{T}^e_0(s)$ is the square matrix of size~$N$
and $\bmx{T}^e_1(s) $ and $\bmx{T}^e_1(s)$ are the
rectangular matrices of size $NM\times N$ and $N\times NM$,
correspondingly.
The solution of the system~\eqref{eq1-r}--\eqref{eq12b}
can be written as 
$$
\bmx{V}_{n}(s)=\bmx{X}_{n}(s) \bmx{V}_{n-1}(s)+\bmx{Y}_{n}(s)\,,
\enskip n=\overline{1,H-1}\,,
$$

\noindent where 
$$
\bmx{X}_{H-1}(s)=(\bmx{E}-\bmx{T}_0(s))\bmx{T}_1(s)\,;
$$
$$
\bmx{Y}_{H-1}(s)=(\bmx{E}-\bmx{T}_0(s))\bmx{T}_{-\!1}(s)\,;
$$
$$
\bmx{X}_{1}(s)=
\left(\bmx{E}-\bmx{T}_0(s)-\bmx{T}_{-1}(s)\bmx{X}_{2}(s)\right)^{-1}\bmx{T}^e_1(s)\,; 
$$

\vspace*{-12pt}

\noindent
\begin{multline*}
\bmx{X}_{n}(s)=\left(\bmx{E}-\bmx{T}_0(s)-\bmx{T}_{-1}(s)\bmx{X}_{n+1}(s)\right)^{-1}
\bmx{T}_1(s)\,, \\
n=\overline{2,H-2}\,;
\end{multline*}

\vspace*{-12pt}

\noindent
\begin{multline*}
\bmx{Y}_{n}(s)=\left(\bmx{E}-\bmx{T}_0(s)\right.\\
\left.{}-\bmx{T}_{-1}(s)\bmx{X}_{n+1}(s)\right)^{-1}
\bmx{T}_{-1}(s)\bmx{Y}_{n+1}(s)\,,
\\ 
 n=\overline{1,H-2}\,;
\end{multline*}

\vspace*{-12pt}

\noindent
\begin{multline*}
\bmx{V}_{0}(s)\\
{}=\left(\bmx{E}-\bmx{T}^e_0(s)-\bmx{T}^e_{-1}(s)\bmx{X}_{1}(s)
\right)^{-1}\bmx{T}^e_{-1}(s)\bmx{Y}_{1}(s)\,. 
\end{multline*}

If the inverse of the matrix $\bmx{T}_{-\!1}(s)$ exists, then 
it is possible to write out the solution of the system~\eqref{eq1-r}--\eqref{eq12b}
using the Kronecker expansion technique (see~\cite{Steeb,Graham,Telek}),
which is based on the identity $\mathrm{vec}\,(\bmx{A}\bmx{B})=
(\bmx{E} \otimes \bmx{A}) \mathrm{vec}\,(\bmx{B})$.
In this identity, ~$\mathrm{vec}$\ denotes the column stacking vector operator
which transforms a~matrix of size $n \times m$ into a~vector of size $nm \times 1$.
We are going to utilize the property of the~$\mathrm{vec}$ operator 
that $\mathrm{vec}\,(\bmx{A})=\bmx{A}$ for matrix~$\bmx{A}$ of size $n \times 1$.
Firstly by applying $\mathrm{vec}$ operator to~\eqref{eq1-r}--\eqref{eq12b},
we get the new system of vector-matrix difference equations:
\begin{align}
\label{H-1}
\mathrm{vec}\left(\bmx{V}_{H-1}(s)\right)
&\notag\\
&\hspace*{-10mm}{}= \bmx{X}^*(s) \mathrm{vec}\left(\bmx{V}_{H-2}(s)\right) +
\mathrm{vec}\left(\bmx{Y}^*(s)\right)\,;
\\
\label{n}
\mathrm{vec}\left(\bmx{V}_{n+1}(s)\right)&=\bmx{X}(s)
\mathrm{vec}\left(\bmx{V}_n(s)\right)\notag\\
&\hspace*{-10mm}{}+\bmx{Y}(s)
\mathrm{vec}\left(\bmx{V}_{n-1}(s)\right)\,,\enskip
n=\overline{2,H-2}\,;
\\
\label{2}
\mathrm{vec}\left(\bmx{V}_{2}(s)\right)&{}\notag\\
&\hspace*{-12mm}{}=
\bmx{X}(s) \mathrm{vec}\left(\bmx{V}_1(s)\right)+
\bmx{Y}^e(s) \mathrm{vec}\left(\bmx{V}_{0}(s)\right)\,;
\\
\label{0}
\mathrm{vec}\left(\bmx{V}_{1}(s)\right) &= \bmx{X}^e(s) \mathrm{vec}\left(\bmx{V}_{0}(s)\right)
\end{align}
where
\begin{align*}
\bmx{X}(s)&= \bmx{E} \otimes \left(\bmx{T}_{-1}(s)\right)^{-1}
\left(\bmx{E}-\bmx{T}_0(s)\right)\,; \\
 \bmx{Y}(s)&= - \left(\bmx{E} \otimes (\bmx{T}_{-1}(s) )^{-1}
\bmx{T}_1(s)\right)\,;
\\
\bmx{X}^e(s)&= \bmx{E} \otimes \left(\bmx{T}^e_{-1}(s) \right)^{-1}
\left(\bmx{E}-\bmx{T}^e_0(s)\right)\,; \\
\bmx{Y}^e(s)&=- \left(\bmx{E} \otimes (\bmx{T}_{-1}(s) )^{-1}
\bmx{T}^e_1(s)\right)\,;\\
\bmx{X}^*(s)&=
\bmx{E} \otimes \left(\bmx{E}-\bmx{T}_0(s)\right)^{-1} \bmx{T}_{1}(s) \,; \\ 
\bmx{Y}^*(s)&=\left(\bmx{E}-\bmx{T}_0(s)\right)^{-1} \bmx{T}_{-1}(s)\,.
\end{align*}

\noindent Secondly, notice that the new system~\eqref{H-1}---\eqref{0}
 consists of pairs of simultaneous equations and thus, its solution can 
 be rewritten as
\begin{multline}
\label{v1}
\begin{pmatrix}
    \mathrm{vec}\left(\bmx{V}_{n}(s)\right) \\
    \mathrm{vec}\left(\bmx{V}_{n-1}(s)\right) 
\end{pmatrix}
={}\\
{}=
\begin{pmatrix}
    \bmx{X}(s) & \bmx{Y}(s) \\
    \bmx{E} & \bmx{0}
\end{pmatrix}^{n-1}
\begin{pmatrix}
    \bmx{X}(s) & \bmx{Y}^e(s) \\
    \bmx{E} & \bmx{0}
\end{pmatrix}
\begin{pmatrix}
    \mathrm{vec}\left(\bmx{V}_{1}(s)\right) \\
    \mathrm{vec}\left(\bmx{V}_{0}(s)\right)
\end{pmatrix}\,, \\
 n=\overline{2,H-1}\,.
\end{multline}


\noindent 
Finally,~\eqref{v1} for $n=H-1$, \eqref{0} and~\eqref{H-1}
make up the system of four matrix equations,
which solution yields the values of 
$\mathrm{vec}\,(\bmx{V}_{H-1}(s))$, $\mathrm{vec}\,(\bmx{V}_{H-2}(s))$,
$\mathrm{vec}\,(\bmx{V}_{1}(s))$, and $\mathrm{vec}\,(\bmx{V}_{0}(s))$.
By virtue of~\eqref{n} and~\eqref{2}, the rest 
of $\bmx{V}_{n}(s)$ can be computed.

Now, let us proceed to the derivation of the equations for 
$\bmx{V}^*_{n}(s)$, $n=\overline{L,r}$,
and $\bmx{V}^{\#}_{n}(s)$, $n=\overline{H+1,r+1}$.
In order to do this, let us introduce the following auxiliary 
square matrices (each of size~$NM$):
\begin{description}
\item[\,] $\bmx{T}^{\#}(s)$~---  
matrix with the  $(i,j)${th} entry 
equal to the LST of the first passage time (of the Markov chain ${\bf{X}}(t)$)
from the state $(i,2,r+1)$ to the state $(j,2,r)$;
\item[\,] 
$\bmx{W}^{\#}_n(s)$, $n=\overline{H+1,r+1}$,~---
matrix with the $(i,j)${th} entry 
equal to the LST of the first passage time 
from the state $(i,2,n)$ to the state $(j,1,H)$;

\item[\,] 
$\bmx{w}^*_n(s)$, $n=\overline{H+1,r}$,~---
matrix with the $(i,j)${th} entry 
equal to the LST of the first passage time 
from the state $(i,1,n)$ to the state $(j,1,H)$
without visiting the states $(\cdot,2,r+1)$;

\item[\,] 
$\bmx{\ov w}^*_n$, $n=\overline{H+1,r}$,~---
matrix with the $(i,j)${th} entry 
equal to the LST of the first passage time 
from the state $(i,1,n)$ to the state $(j,2,r+1)$
without visiting the states $(\cdot,1,H)$; and
\item[\,] 
$\bmx{W}^*_n(s)$, $n=\overline{H+1,r}$,~---
matrix with the $(i,j)${th} entry 
equal to the LST of the first passage time 
from the state $(i,1,n)$ to the state $(j,1,H)$.
\end{description}

Let us begin with the relation for~$\bmx{T}^{\#}(s)$. 
Let~$\bmx{T}^{\#}_k(s)$, $k=0,1$, denote 
the square matrix of size $NM$ with 
the $(i,j)${th} entry 
equal to the LST of the first passage time 
from the state $(i,2,r+1)$ to the state $(j,2,r+k)$.
Since the sojourn time in the state $(i,2,r+1)$
is distributed exponentially with the rate $-[\bmx{Q}^{\#}]_{ii}$, 
one has for $1 \le i,j \le NM$:
\begin{gather*}
\left[ \bmx{T}^{\#}_0(s) \right]_{ij}
=
\fr{[\bmx{Q}^{\#}]_{ii}}{[\bmx{Q}^{\#}]_{ii} - s}
\left[\beta^{\#}\right]_{ij}  \,;
\\
\left[ \bmx{T}^{\#}_1(s) \right]_{ij}
= \fr{[\bmx{Q}^{\#}]_{ii} }{[\bmx{Q}^{\#}]_{ii} - s}
\left[\alpha^{\#}\right]_{ij}  \,.
\end{gather*}

\noindent The first-step analysis yields the following 
equation for~$\bmx{T}^{\#}(s)$:
$$
\bmx{T}^{\#}(s)= \bmx{T}^{\#}_1(s) + \bmx{T}^{\#}_0(s) \bmx{T}^{\#}(s)\,,
$$
which solution is
$$
\bmx{T}^{\#}(s)= \left(\bmx{E}-\bmx{T}^{\#}_0(s)\right)^{-1} \bmx{T}^{\#}_1(s)\,.
$$


Since in the ``blocking'' mode any arrival is lost, then the sojourn time in 
it is equal to the time needed for $(n-H)$ service completions,
given that initially, there were $n$ customers in the system, i.\,e.,
$$
\bmx{W}^{\#}_n(s)= \left(\bmx{T}^{\#}(s)\right)^{n-H}\,, \enskip
n=\overline{H+1,r+1}\,.
$$

Equations for $\bmx{w}^*_n(s)$ and $\bmx{\ov w}^*_n(s)$ can be 
derived by following the same arguments given above for~$\bmx{V}_n(s)$. Denote by 
$\bmx{T}^{*}_k(s)$, $k=-1,0,1$,  
the square matrix of size~$NM$ with 
the $(i,j)${th} entry 
equal to the LST of the first passage time 
from the state $(i,1,n)$ to the state $(j,1,n-k)$.
Then, using the fact that the sojourn time in the state $(i,1,n)$
is distributed exponentially with the rate~$-[\bmx{Q}^{*}]_{ii}$, 
one obtains for $1 \le i,j \le NM$:
\begin{align*}
\left[ \bmx{T}^{*}_{-1}(s) \right]_{ij}
&=
\fr{[\bmx{Q}^{*}]_{ii}}{[\bmx{Q}^{*}]_{ii} - s}
\left[\gamma^{*}\right]_{ij}\,;
\\
\left[ \bmx{T}^{*}_0(s) \right]_{ij}
&=\fr{[\bmx{Q}^{*}]_{ii}}{[\bmx{Q}^{*}]_{ii} - s}
\left[\beta^{*}\right]_{ij}  \,;
\\
\left[ \bmx{T}^{*}_1(s) \right]_{i,j}
&=
\fr{[\bmx{Q}^{*}]_{ii} }{[\bmx{Q}^{*}]_{ii} - s}
\left[\alpha^{*}\right]_{ij}   \,.
\end{align*}

Again, by applying the first-step analysis, one gets 
the following system of matrix difference equations 
for~$\bmx{w}^*_n(s)$, $n=\overline{H+1,r}$:
\begin{multline}
\label{w*n}
\bmx{w}^*_n(s)
= \bmx{T}^*_1(s)  \bmx{w}^*_{n-1}(s)\\
{} +  \bmx{T}^*_0(s) \bmx{w}^*_n(s) + 
\bmx{T}^*_{-\!1}(s)  \bmx{w}^*_{n+1}(s)\,,
\end{multline}

\noindent with the boundary conditions
\begin{equation}
\label{w*nbound}
\bmx{w}^*_H(s)
=
\bmx{E}\,;
\enskip  
\bmx{w}^*_{r+1}(s)
= 
\bmx{0}\,.
\end{equation}


\noindent Clearly, the matrices $\bmx{\ov w}^*_n(s)$, $n=\overline{H+1,r}$,
satisfy the system of equations, which is identical to~\eqref{w*n},
i.\,e.,
\begin{multline}
\label{w*nOV} \bmx{\ov w}^*_n(s)
= \bmx{T}^*_1(s)  \bmx{\ov w}^*_{n-1}(s)
+  \bmx{T}^*_0(s) \bmx{\ov w}^*_n(s)\\
{}+  \bmx{T}^*_{-1}(s) 
\bmx{\ov w}^*_{n+1}(s)\,,
\end{multline}

\noindent but with the ``reversed'' boundary conditions:
\begin{equation}
\label{w*nboundOV}
\bmx{\ov w}^*_H(s)
= \bmx{0}\,; \enskip  
\bmx{\ov w}^*_{r+1}(s) =  \bmx{E}\,.
\end{equation}

The structure of the systems~\eqref{w*n}, \eqref{w*nbound} 
and~\eqref{w*nOV}, \eqref{w*nboundOV} is
similar to the~\eqref{eq1-r}--\eqref{eq12b} (expect for the boundary conditions).
Thus, its solutions can be found completely in the same way and, therefore, are omitted.
Once~$\bmx{w}^*_n(s)$ and~$\bmx{\ov w}^*_n(s)$ are found, the 
matrices~$\bmx{W}^*_n(s)$ can be computed. Noticing that from the 
state $(i,1,n)$, $n=\overline{H+1,r}$, the Markov chain can enter 
the state $(j,H,1)$ either from the set of ``overload'' states or
from the set of ``blocking'' states (see figure), one has:
$$
\bmx{W}^*_n(s)= \bmx{w}^*_n(s)+\bmx{\ov w}^*_n(s)\bmx{W}^{\#}_{r+1}(s)\,,
\enskip n=\overline{H+1,r}\,.
$$

\columnbreak

Now, everything is prepared for the derivation of
the relations for the unknown quantities $\bmx{V}^*_n(s)$  
and $\bmx{V}^{\#}_n(s)$. 
If $n=\overline{L,H}$, then $\bmx{V}^*_n(s)$ satisfy the following
system of matrix difference equations:
\begin{multline}
\label{V*}
\bmx{V}^*_n(s)
= \bmx{T}^*_1(s)  \bmx{V}^*_{n-1}(s) + 
\bmx{T}^*_0(s) \bmx{V}^*_n(s)\\
{} +  \bmx{T}^*_{-\!1}(s)  \bmx{V}^*_{n+1}(s)\,,
\end{multline}

\noindent with the boundary conditions
\begin{align*}
\bmx{V}^*_{L-1}(s)
&= \bmx{E}\,;
\\[3pt]
\bmx{V}^*_{H}(s)&=  \bmx{T}^*_1(s)  \bmx{V}^*_{H-1}(s)\\
&{}  + 
\bmx{T}^*_0(s) \bmx{V}^*_H(s)
+  \bmx{T}^*_{-1}(s) 
\bmx{W}^*_{H+1}(s)
\bmx{V}^*_{H}(s)\,.
%\label{w*nboundOV1}
\end{align*}


The final expressions for the matrices
$\bmx{V}^*_n(s)$, $n=\overline{H+1,r}$
and $\bmx{V}^{\#}_n(s)$, $n=\overline{H+1,r+1}$,
have the form:
\begin{alignat*}{2}
\bmx{V}^{\#}_{n}(s)
&=  \bmx{W}^{\#}_n(s) 
\bmx{V}^*_{H}(s)\,, &\enskip n&=\overline{H+1,r+1}\,;
\\
\bmx{V}^{*}_{n}(s)
&=  \bmx{W}^{*}_n(s) 
\bmx{V}^*_{H}(s)\,, &\enskip n&=\overline{H+1,r}\,.
\end{alignat*}

The last two relations, together with~\eqref{eq1-r} and~\eqref{V*}, 
give the complete solution of the considered problem.
The matrices $\bmx{V}_n(s)$, $\bmx{V}^*_n(s)$, and $\bmx{V}^{\#}_n(s)$
allow one to calculate various performance characteristics of
the hysteretic policy such as (conditional\footnote{The corresponding 
unconditional characteristics 
are obtained by weighting according to the joint stationary distribution 
found in~\cite{int0}.}) 
mean duration of overload period (equal to $-[\bmx{V}^*_H(s)]'_{s=0}$),
(conditional) mean return time to the ``overload'' 
mode (equal to $-[\bmx{V}_{L-1}(s)]'_{s=0}- [\bmx{V}^*_H(s)]'_{s=0}$),
higher moments, etc.

No principal difficulties show up if in addition 
to the sojourn times one needs to count 
the number of lost customers. The same
argumentation applies. For example, 
let us assume that customers arrived during the
period of time when the system 
is in the ``blocking'' mode are considered as lost.
Then, the LST and the generating function~$\bmx{W}^{\#}_n(s,z)$
of the joint stationary distribution of the 
sojourn time in the ``blocking'' mode 
and the number of lost customers 
provided that the system is in the ``blocking'' mode 
and there are $n$ customers in it 
is equal to 
\begin{multline*}
\bmx{W}^{\#}_n(s,z)
=  \left \{
\left[ \bmx{E} - z ( \bmx{E}-\bmx{T}^*_0(s))^{-1}\bmx{T}^*_{-1}(s) \right]^{-1}\right.\\
\left.{}\times
\left( \bmx{E}-\bmx{T}^*_0(s)\right)^{-1}\bmx{T}^*_{1}(s)
\right \}^{H+1-n}
\times{} \\
{}\times
\left \{ \bmx{E} +  
z \left[ \bmx{E} - z \left( \bmx{E}-\bmx{T}^*_0(s)\right)^{-1}
\bmx{T}^*_{-1}(s) \right]^{-1}
\right \}\\
{}\times
\left( \bmx{E}-\bmx{T}^*_0(s)\right)^{-1}\bmx{T}^*_{1}(s)\,.
\end{multline*}

\noindent Thus, the substitution of~$\bmx{W}^{\#}_n(s,z)$ 
instead of~$\bmx{W}^{\#}_n(s)$ in the above expressions
will account not only for the sojourn time but for losses
(during the sojourn time). 



\section{Concluding Remarks}

\noindent
The approach proposed in the paper allows one to calculate
the system's sojourn time in various modes 
in terms of LST by exploiting the knowledge
about the presence of hysteretic loops. 
Minor changes are needed to adapt it to the case of overlapping loops. 
Of course, due to MAP arrivals and PH service times, it
utilizes matrix analytic techniques and, thus, possesses
the disadvantages inherent to matrix algebra.
Despite the fact there is large body of research results 
available in this topic, 
there is still a~number of open questions.
Is there any analytic approach to find the steady-state
behavior of several interconnected systems each with hysteretic policy, 
which exploits the knowledge of the presence of hysteretic loops?
What is the gain of hysteretic control of arrivals with
respect to other types of control? Just to name a~few.

\vspace*{-4pt}


\Ack
\noindent
This work was supported by the Russian Science Foundation (grant No.\,16-11-10227).

\renewcommand{\bibname}{\protect\rmfamily References}

\vspace*{-4pt}


{\small\frenchspacing
{%\baselineskip=10.8pt
\begin{thebibliography}{9}

\bibitem{int0} %1
\Aue{Razumchik, R.} 2016. Analysis of finite ${\mathrm{MAP}/\mathrm{PH}/1}$ 
queue with hysteretic control of arrivals.
\textit{Congress (International)
on Ultra Modern Telecommunications and Control Systems and Workshops Proceedings}.
 Lisbon. 288--293.



\bibitem{int2} %2
\Aue{Chesoong, K., A.~Dudin, S.~Dudin, and O.~Dudina.} 2016.
Hysteresis control by the number of active servers in queueing system ${\mathrm{MMAP}/\mathrm{PH}/N}$
with priority service. \textit{Perform. Evaluation} 101:20--33.

\bibitem{int3} %3
\Aue{Chan, C.\,W., M.~Armony, and N.~Bambos.} 2016.
Maximum weight matching with hysteresis in overloaded queues with setups.
\textit{Queueing Sy.} 82(3-4):315--351.

\bibitem{int4} %4
\Aue{Rumyantsev, A.\,S., K.\,A.~Kalinina, and T.\,E.~Morozova.} 2017.
Stokhasticheskioe modelirovanie vycheslitel'nogo klastera s~gisterezisnym upravleniem
skorost'yu obsluzhivaniya
[Stochastic modeling of a~high-performance cluster
with hysteretic control of service rate].
\textit{Trudy Karel'skogo nauchnogo tsentra RAN}
[Transactions of KarRC RAS] 8:76--85.

\bibitem{int1} %5
\Aue{Abaev, P., Y.~Gaidamaka, K.~Samouylov, A.~Pechinkin, R.~Razumchik, and S.~Shorgin.} 2014.
Hysteretic control technique for overload problem solution in network of SIP servers.
\textit{Comput. Inform.} 33(1):1--18.




\bibitem{Graham} %6
\Aue{Graham, A.} 1982.
\textit{Kronecker products and matrix calculus: With applications}. 
New York, NY: John Wiley \& Sons. 130~p.

\bibitem{Steeb} %7
\Aue{Steeb, W.\,H., and Y.~Hardy.} 2011.
\textit{Matrix calculus and Kronecker product: A~practical 
approach to linear and multilinear algebra}. 2nd ed. River Edge,
NJ: World Scientific. 324~p.

\bibitem{Telek} %8
\Aue{Razumchik, R., and M.~Telek.} 2016.
Delay analysis of a~queue with re-sequencing buffer
and Markov environment.
\textit{Queueing Sy.} 82(1-2):7--28.

\end{thebibliography} }
 }

\end{multicols}

\vspace*{-3pt}

\hfill{\small\textit{Received September 19, 2017}}

%\vspace*{-12pt}


\Contrl

\noindent
\textbf{Razumchik Rostislav V.} (b.\ 1984)~--- 
Candidate of Science (PhD) in physics and mathematics, leading scientist, 
Institute of Informatics Problems, Federal Research Center 
``Computer Science and Control'' of the Russian Academy of Sciences, 
44-2~Vavilov Str., Moscow 119333, Russian Federation; associate professor, 
Peoples' Friendship University of Russia (RUDN University), 
6~Miklukho-Maklaya Str., Moscow 117198, Russian Federation; 
\mbox{rrazumchik@ipiran.ru}
%; \mbox{razumchik\_rv@rudn.university}

%\vspace*{8pt}

%\hrule

%\vspace*{2pt}

%\hrule



\newpage

\vspace*{-28pt}



\def\tit{СТАЦИОНАРНЫЕ РАСПРЕДЕЛЕНИЯ, СВЯЗАННЫЕ СО~ВРЕМЕНЕМ ПРЕБЫВАНИЯ
В~СОСТОЯНИИ ПЕРЕГРУЗКИ СИСТЕМЫ $\mathrm{MAP}/\mathrm{PH}/1/r$ С~ГИСТЕРЕЗИСНЫМ УПРАВЛЕНИЕМ 
НАГРУЗКОЙ$^*$}

\def\aut{Р.\,В.~Разумчик}


\def\titkol{Стационарные распределения, связанные со временем пребывания
в~состоянии перегрузки системы $\mathrm{MAP}/\mathrm{PH}/1/r$}
% с~гистерезисным управлением  нагрузкой}

\def\autkol{Р.\,В.~Разумчик}

{\renewcommand{\thefootnote}{\fnsymbol{footnote}}
\footnotetext[1]{Работа выполнена при поддержке РНФ (проект 16-11-10227).}}


\titel{\tit}{\aut}{\autkol}{\titkol}

\vspace*{-12pt}

\noindent
Институт проблем информатики
Федерального исследовательского центра <<Информатика и~управ\-ле\-ние>>
Российской академии наук; Российский университет дружбы народов,
\mbox{rrazumchik@ipiran.ru}

\vspace*{6pt}

\def\leftfootline{\small{\textbf{\thepage}
\hfill ИНФОРМАТИКА И ЕЁ ПРИМЕНЕНИЯ\ \ \ том\ 11\ \ \ выпуск\ 4\ \ \ 2017}
}%
 \def\rightfootline{\small{ИНФОРМАТИКА И ЕЁ ПРИМЕНЕНИЯ\ \ \ том\ 11\ \ \ выпуск\ 4\ \ \ 2017
\hfill \textbf{\thepage}}}


\Abst{Как известно, одним из решений проблемы перегрузок
в~сетях SIP (session initiation protocol)
сиг\-на\-ли\-за\-ции является применение
в~узлах сети (SIP-сер\-ве\-рах) многоуровневого
гистерезисного управления нагрузкой.
В~данной работе представлены некоторые новые 
результаты анализа системы $\mathrm{MAP}/\mathrm{PH}/1/r$ конечной емкости
с~двумя петлями гистерезисного управления,
функционирующей в случайной среде и являющейся
моделью SIP-сер\-ве\-ра с двухуровневым
гистерезисным управлением нагрузкой.
Получен метод вычисления преобразования
Лап\-ла\-са--Стилть\-еса
функций распределения времени возврата системы из
множества состояний перегрузки в множество
состояний нормальной нагрузки
и~времени выхода системы из множества состояний
нормальной нагрузки.
Метод основан на решении матричных рекуррентных уравнений
и~применим в случае, когда помимо расчета времени
выхода из состояния перегрузки необходимо
также учитывать и число потерянных за это время
заявок.}

\KW{система массового обслуживания; случайная среда; гистерезисное управление; 
время пребывания}

\DOI{10.14357/19922264170403}

\vspace*{18pt}


 \begin{multicols}{2}

\renewcommand{\bibname}{\protect\rmfamily Литература}
%\renewcommand{\bibname}{\large\protect\rm References}

{\small\frenchspacing
{%\baselineskip=10.8pt
\begin{thebibliography}{9}

\bibitem{1-r-1}
\Au{Razumchik R.} 
Analysis of finite ${\mathrm{MAP}/\mathrm{PH}/1}$ queue with hysteretic control of arrivals~// 
 Congress (International) on Ultra Modern Telecommunications and Control 
Systems and Workshops Proceedings.~--- Lisbon, 2016.  P.~288--293.


\bibitem{3-r-1} %2
\Au{Chesoong K., Dudin~A., Dudin~S., Dudina~O.}
Hysteresis control by the number of active servers in queueing system 
$\mathrm{MMAP}/\mathrm{PH}/N$
with priority service~// Perform. Evaluation, 2016. Vol.~101. P.~20--33.

\bibitem{4-r-1} %3
\Au{Chan C.\,W., Armony~M., Bambos~N.}
Maximum weight matching with hysteresis in overloaded queues with setups~// 
Queueing Sy., 2016. Vol.~82. No.\,3-4. P.~315--351.

\bibitem{5-r-1} %4
\Au{Румянцев А.\,С., Калинина~К.\,А., Морозова~Т.\,Е.}
Стохастическое моделирование вычислительного кластера 
с~гистерезисным управлением скоростью обслуживания~//
Труды Карельского научного центра РАН, 2017. Вып.~8. С.~76--85.

\bibitem{2-r-1} %5
\Au{Abaev P., Gaidamaka~Y.,  Samouylov~K.,  Pechinkin~A.,  Razumchik~R.,  Shorgin~S.}
Hysteretic control technique for overload problem solution in network of SIP servers~//
Comput. Inform., 2014. Vol.~33. No.\,1. P.~1--18.





\bibitem{7-r-1} %6
\Au{Graham A.}
Kronecker products and matrix calculus: With applications.~--- 
New York, NY, USA: John Wiley \& Sons, 1982. 130~p.

\bibitem{6-r-1} %7
\Au{Steeb W.\,H., Hardy~Y.}
Matrix calculus and Kronecker product: A~practical
approach to linear and multilinear algebra.~--- 2nd ed.~---
River Edge, NJ, USA: World Scientific, 2011. 324~p.

\bibitem{8-r-1}
\Au{Razumchik~R., Telek~M.}
Delay analysis of a~queue with re-sequencing buffer
and Markov environment~// Queueing Sy., 2016. Vol.~82. 
No.\,1-2. P. 7--28.

\end{thebibliography}
} }

\end{multicols}

 \label{end\stat}

 \vspace*{-3pt}

\hfill{\small\textit{Поступила в~редакцию  19.09.2017}}
%\renewcommand{\bibname}{\protect\rm Литература}
\renewcommand{\figurename}{\protect\bf Рис.}
\renewcommand{\tablename}{\protect\bf Таблица}   %3
\def\stat{kor-kor}



\def\tit{МОДИФИЦИРОВАННЫЙ СЕТОЧНЫЙ МЕТОД РАЗДЕЛЕНИЯ ДИСПЕРСИОННО-СДВИГОВЫХ
СМЕСЕЙ НОРМАЛЬНЫХ ЗАКОНОВ$^*$}



\def\titkol{Модифицированный сеточный метод разделения дисперсионно-сдвиговых
смесей нормальных законов}

\def\aut{В.\,Ю.~Королев$^1$,  А.\,Ю.~Корчагин$^2$}

\def\autkol{В.\,Ю.~Королев,  А.\,Ю.~Корчагин}

\titel{\tit}{\aut}{\autkol}{\titkol}

{\renewcommand{\thefootnote}{\fnsymbol{footnote}} \footnotetext[1]
{Работа поддержана Российским научным фондом (проект 14-11-00364).}}


\renewcommand{\thefootnote}{\arabic{footnote}}
\footnotetext[1]{Факультет
вычислительной математики и кибернетики Московского государственного
университета им.\ М.\,В.~Ломоносова; Институт проблем информатики
Российской академии наук; victoryukorolev@yandex.ru}
\footnotetext[2]{Факультет вычислительной математики и кибернетики
Московского государственного университета им.\ М.\,В.~Ломоносова;
sasha.korchagin@gmail.com}

%\vspace*{2pt}



\Abst{Описывается модифицированный двухэтапный
сеточный метод разделения дис\-пер\-си\-он\-но-сдви\-го\-вых смесей нормальных
законов, представляющий собой альтернативу чистому ЕМ (expectation-maximization)
ал\-го\-рит\-му. На
первом этапе этого алгоритма строится дискретная аппроксимация для
смешивающего распределения, на втором этапе подбирается абсолютно
непрерывное распределение из заранее заданного семейства, например,
обобщенных обратных гауссовских законов, ближайшее к~дискретному
распределению, полученному на первом этапе. Обсуждаются вопросы
сходимости этого двухэтапного алгоритма. Доказана монотонность
сеточного итерационного метода, используемого на первом этапе.
Подробно обсуждается вопрос оптимального выбора параметров метода,
прежде всего сетки, накидываемой на носитель смешивающего
распределения. С~этой целью предложены статистические оценки
квантилей смешивающего распределения. Эффективность метода
иллюстрируется примерами конкретных вычислений оценок параметров
обобщенных гиперболических распределений.}

\KW{смесь распределений вероятностей;
дис\-пер\-си\-он\-но-сдви\-го\-вая смесь нормальных законов; обобщенное
гиперболическое распределение; ЕМ-ал\-го\-ритм; сеточный метод
разделения смесей}

\vspace*{1pt}

%\vspace*{2pt}

\DOI{10.14357/19922264140402}


\vskip 12pt plus 9pt minus 6pt

\thispagestyle{headings}

\begin{multicols}{2}

\label{st\stat}

\section{Введение}

При {\it практическом} решении задачи моделирования и исследования
волатильности (изменчивости) хаотических стохастических процессов
ключевым этапом является статистическое разделение смесей
вероятностных распределений. Задача разделения смесей~---
статистического оценивания параметров смесей вероятностных
распределений~--- в~деталях разобрана, например, в~книге~\cite{k2011}.

Для решения задачи разделения смесей вероятностных распределений
традиционно используются итерационные процедуры типа ЕМ-ал\-го\-рит\-ма.
К~сожалению, классический ЕМ-ал\-го\-ритм обладает рядом серьезных
недостатков при его применении к~смесям нормальных законов, а~именно:
он демонстрирует крайнюю неустойчивость по отношению к~исходным
данным и~начальным приближениям.

Для преодоления этих недостатков
предложено много модификаций ЕМ-ал\-го\-рит\-ма (см., например,~\cite{k2011}).
Вместе с тем в~указанной книге предложен и~исследован
принципиально новый~--- сеточный~--- метод приближенного решения
задачи разделения смесей. В~работе~\cite{n2013} подробно исследованы
вопросы сходимости сеточных методов разделения смесей.

В соответствии с подходом к~статистическому анализу хаотических
стохастических процессов, в~частности к~решению задачи декомпозиции
волатильности таких процессов, развитом в~книге~\cite{k2011},
в~общем случае на практике приходится решать задачу разделения
конечных смесей нормальных законов с~произвольно большим числом
неизвестных параметров (параметров компонент и~их весов).
И~хотя в~большинстве приложений возникают смеси не более чем с~пятью--семью
компонентами, даже при использовании таких смесей, скажем, в~задачах
анализа и~прогнозирования финансовых рисков приходится моделировать
траекторию движения точки в~пространствах, размерность которых
соответственно лежит в~пределах от~14 (для пятикомпонентных смесей)
до~20 (для семикомпонентных смесей), что существенно увеличивает
вычислительные и~временн$\acute{\mbox{ы}}$е ресурсы, необходимые для практического
решения указанных задач.

Поскольку во многих ситуациях (например,
при прогнозировании на основе высокочастотных данных) эти задачи
необходимо решать в~режиме, близком к~реальному времени, для
создания эффективных методов статистического анализа на основе
смешанных моделей на первый план выходит проб\-ле\-ма снижения
размерности решаемой задачи, т.\,е.\ параметрического пространства.

Одним из возможных подходов к~снижению размерности является
априорное сужение классов допусти\-мых смесей. К~примеру, при решении
многих задач, связанных с~анализом процессов атмосферной или
плазменной турбулентности, а~так\-же процессов, описывающих эволюцию
различных финансовых индексов, высочайшую адекватность
продемонстрировали модели, основанные на дис\-пер\-си\-он\-но-сдви\-го\-вых
смесях нормальных законов. Класс таких смесей очень обширен
и,~в~част\-ности, включает в~себя обобщенные гиперболические распределения,
которые были введены О.-Е.~Барн\-дорфф-Ниль\-се\-ном в~1977--1978~гг.\ как
класс специальных сдвиг-мас\-штаб\-ных смесей нормальных законов~\cite{BN1977, BN1978}.
Пусть $\alpha\hm\in\r$, $\beta\hm\in\r$. Если
функцию распределения обобщенного гиперболического закона
с~параметрами~$\alpha$, $\beta$, $\nu$, $\mu$, $\lambda$ обозначить
$P_{GH}(x;\alpha,\beta,\nu,\mu,\lambda)$, то по определению
\begin{multline}
P_{GH}(x;\alpha,\beta,\nu,\mu,\lambda)={}\\
{}=
\int\limits_{0}^{\infty}\Phi\left(\fr{x-\beta-\alpha
z}{\sqrt{z}}\right)\,p_{GIG}(z;\nu,\mu,\lambda)\,dz\,,\\
x\in\r\,,
\label{e1-kor}
\end{multline}
где $\Phi(x)$~--- стандартная нормальная функция распределения:
$$
\Phi(x)=\int\limits_{-\infty}^{x}\varphi(z)\,dz\,,\enskip
\varphi(x)=\fr{1}{\sqrt{2\pi}}e^{-x^2/2}\,,\enskip  x\in\mathbb{R}\,;
$$
$p_{GIG}(x;\nu,\mu,\lambda)$~--- плот\-ность обобщенного обратного
гауссовского распределения:
\begin{multline*}
p_{GIG}(x;\nu,\mu,\lambda)={}\\
{}=\fr{\lambda^{\nu/2}}{2\mu^{\nu/2}
K_{\nu}\left(\sqrt{\mu\lambda}\right)}\,
x^{\nu-1}\exp\left\{-\fr{1}{2}\left(\fr{\mu}{x}+\lambda
x\right)\right\}\,,\\ x>0\,.
\end{multline*}
Здесь $\nu\in\r$;
$$
\begin{array}{lll}
\mu>0\,, & \lambda\geqslant0\,, & \mbox{если }\nu<0\,;\\[6pt]
\mu>0\,, & \lambda>0\,, & \mbox{если }\nu=0\,;\\[6pt]
\mu\geqslant0\,, & \lambda>0\,, & \mbox{если }\nu>0\,;
\end{array}
$$
$K_{\nu}(z)$~--- модифицированная бесселева функция третьего рода
порядка~$\nu$:

\noindent
\begin{multline*}
K_{\nu}(z)=\fr{1}{2}\int\limits_{0}^{\infty}y^{\nu-1}\exp
\left\{-\fr{z}{2}\left(y+\fr{1}{y}\right)\right\}\,dy\,,\\
z\in\mathbb{C}\,,\enskip \mathrm{Re}\,z>0\,.
\end{multline*}
Обратим внимание, что в~(1) смешивание происходит одновременно и~по
параметру сдвига, и~по параметру масштаба, но так как эти параметры
в~(1)  связаны жесткой зависимостью, так что параметр сдвига
смешиваемого распределения пропорционален его дисперсии, то
фактически смесь~(1) является {\it однопараметрической} и~поэтому
называется {\it дис\-пер\-си\-он\-но-сдви\-го\-вой} (см., например,~\cite{BN1982}).

Другим примером дис\-пер\-си\-он\-но-сдви\-го\-вых смесей нормальных законов
являются обобщенные дисперсионные гам\-ма-рас\-пре\-де\-ле\-ния, в~которых
смешивающими являются обобщенные гам\-ма-рас\-пре\-де\-ле\-ния~\cite{ks2012, zk2013}.

В указанных семействах смесей число неизвестных параметров равно
пяти или шести (если\linebreak учитывать неслучайный сдвиг). Вместе
с~тем у~подоб\-ных моделей имеются довольно серьезные тео\-ре\-ти\-че\-ские
обоснования: в~работах~\cite{zk2013, k2013} показано, что указанные
модели являются асимптотическими аппроксимациями в~простой
предельной схеме случайного суммирования и~потому могут успешно
применяться для анализа процессов типа остановленных случайных
блужданий. Эти выводы подтверждены статистическим анализом
вы\-со\-ко\-час\-тот\-ных финансовых данных, в~результате которого выявлен
синхронизированный характер изменения интенсивностей потоков заявок
в~сис\-те\-мах электронных торгов, что естественно приводит к~синхронизированному
поведению па\-ра\-мет\-ров сдвига и~диффузии в~соответствующих моделях вида смесей
нормальных законов~\cite{kckg2013}.

\section{Описание моди\-фи\-ци\-ро\-ван\-но\-го
сеточного ме\-то\-да разделения дисперсионно-сдвиговых смесей
нормальных законов и~его свойства}

Оказывается, что сеточные методы разделения смесей довольно
эффективны не только при разделении конечных смесей нормальных
законов, но и~при разделении произвольных дис\-пер\-си\-он\-но-сдви\-го\-вых
смесей нормальных законов. Поясним сказанное на примере задачи
оценивания па\-ра\-мет\-ров обобщенных гиперболических распределений.

Для решения задачи оценивания параметров обобщенных гиперболических
распределений традиционно используется метод, предложенный в~статье~\cite{p2004}
и~по сути являющийся классическим ЕМ-ал\-го\-рит\-мом,
приспособленным к~конкретной задаче, и,~соответственно, наследующий
присущие ЕМ-ал\-го\-рит\-мам недостатки.

Рассмотрим следующий альтернативный двухэтапный метод. На первом
этапе на поло\-жи\-тельной полупрямой выделим основную часть носителя
смешивающего распределения, т.\,е.\ \mbox{ограниченный} интервал,
вероятность которого, вычисленная в~соответствии со смешивающим
распределением, практически равна единице. На этот интервал накинем
конечную сетку, содержащую, возможно, очень много {\it известных}
узлов $u_1,\ldots,u_K$. Считая параметр сдвига~$\beta$ равным нулю,
приблизим искомое обобщенное гиперболическое распределение конечной
смесью нормальных законов:

\noindent
\begin{multline}
P_{GH}(x;\,\alpha,0,\nu,\mu,\lambda)\approx{}\\
{}\approx \sum\limits_{i=1}^K
p_i\Phi\left(\fr{x-\alpha u_i}{\sqrt{u_i}}\right)\,,\enskip
x\in\mathbb{R}\,.\label{e2-kor}
\end{multline}
В смеси, стоящей в~правой части соотношения~(2), неизвестными
являются только параметры $p_1,\ldots,p_{K-1}$ и~$\alpha$. Пусть
$x_1,\ldots,x_n$~--- анализируемая выборка значений случайной
величины с~оцениваемым обобщенным гиперболическим распределением.
Итерационный процесс, определяющий сеточный ЕМ-ал\-го\-ритм для данной
задачи, задается следующим образом. Пусть
$p_1^{(m)},\ldots,p_{K-1}^{(m)}$ и~$\alpha^{(m)}$~--- оценки параметров
$p_1,\ldots,p_{K-1}$ и~$\alpha$ на $m$-й итерации,
$p_K^{(m)}\hm=1\hm-p_1^{(m)}-\cdots-p_{K-1}^{(m)}$. Обозначим

\noindent
\begin{align*}
\varphi_{ij}^{(m)}&=\fr{1}{\sqrt{u_i}}\varphi\left(\fr{x_j-\alpha^{(m)}u_i}{\sqrt{u_i}}\right)\,;
\\
g_{ij}^{(m)}&=\fr{p_i^{(m)}\varphi_{ij}^{(m)}}{\sum\limits_{r=1}^K
p_r^{(m)}\varphi_{rj}^{(m)}}\,,\\
&\hspace*{14mm}i=1,\ldots,K\,;\enskip j=1,\ldots,n\,.
\end{align*}
Тогда, используя стандартные рассуждения, определяющие
вычислительные формулы EM-ал\-го\-рит\-ма для параметров конечной смеси
нормальных законов (см, например,~[1, разд.~5.3.7--5.3.8]),
следует положить

\noindent
\begin{equation}
p_i^{(m+1)}=\fr{1}{n}\sum\limits_{j=1}^n g_{ij}^{(m)}\,, \enskip
i=1,\ldots,K\,.\label{e3-kor}
\end{equation}
Обозначим $\overline{x}=(1/n)\sum\limits_{j=1}^nx_j$. Используя
соотношение~(5.3.24) в~\cite{k2011}, с~учетом очевидного равенства
$\sum\limits_{i=1}^K g_{ij}^{(m)}\hm=1$ можно заметить, что уточненная
оценка параметра~$\alpha$ имеет вид:

\columnbreak

\noindent
\begin{equation}
\alpha^{(m+1)}=\fr{\overline{x}}{\sum\limits_{i=1}^K u_ip_i^{(m+1)}}\,,
\label{e4-kor}
\end{equation}
т.\,е.\ равна отношению генерального выборочного среднего и~текущего
эмпирического среднего смешивающего распределения, что вполне
согласуется с~тем, что в~соответствии с~приводимым ниже соотношением~(\ref{e5-kor})
в~данном случае ${\sf E}X\hm=\alpha{\sf E}U$.

В силу монотонности классического ЕМ-ал\-го\-рит\-ма справедливо следующее
утверждение.

\smallskip

\noindent
\textbf{Теорема~1.} {\it Пусть узлы $u_1,\ldots,u_K$ сетки различны,
неотрицательны и~известны. Тогда итерационный процесс $(3)$--$(4)$
является монотонным, т.\,е.\ каждая его итерация не уменьшает
целевую сеточную функцию правдоподобия}
\begin{multline*}
L(p_1,\ldots,p_K,\alpha;x_1,\ldots,x_n)={}\\
{}=
\prod\nolimits_{j=1}^n\left[\sum\nolimits_{i=1}^K
\fr{p_i}{\sqrt{u_i}}\,\varphi\left(\fr{x_j-\alpha^{(m)}u_i}{\sqrt{u_i}}\right)\right].
\end{multline*}

\smallskip

\noindent
\textbf{Замечание~1.} В~разд.~5.7.4 книги~\cite{k2011} показано, что
при каждом фиксированном значении параметра~$\alpha$ сеточная
функция правдоподобия\linebreak
$L(p_1,\ldots,p_{K-1},\alpha;\,x_1,\ldots,x_n)$ вогнута по
аргументам $p_1,\ldots,p_{K-1}$. Поэтому на каждом шаге
итерационного процесса вместо соотношения~(3) можно\linebreak использо\-вать
любой более быстрый алгоритм максимизации функции
$L(p_1,\ldots,p_{K-1},\alpha^{(m)};\,x_1,\ldots$\linebreak $\ldots,x_n)$ по переменным
$p_1,\ldots,p_{K-1}$. Например, оценки весов $p_1,\ldots,p_K$ можно
искать методом условного градиента~\cite{k2011, kn2010}.

\smallskip

Таким образом, на первом этапе получаются оценки параметра~$\alpha$
и~весов всех узлов~$u_i$ конечной сетки, накинутой на носитель
смешивающего обобщенного обратного гауссовского распределения
$P_{\mathrm{GIG}}(z;\,\nu,\mu,\lambda)$.

На втором этапе остается применить ка\-кой-ли\-бо стандартный метод
подгонки обобщенного обратного гауссовского распределения
$P_{\mathrm{GIG}}(z;\,\nu,\mu,\lambda)$ к~эмпирическим данным типа
гистограммы $(u_1, p_1),\ldots, (u_K, p_K)$. Например, параметры~$\nu$,
$\mu$ и~$\lambda$ можно оценить, минимизируя соответствующую
статистику хи-квад\-рат. Или же, например, можно решить задачу
наименьших квад\-ратов:
\begin{multline*}
(\nu^*,\mu^*,\lambda^*)={}\\
{}=\arg\min\limits_{\nu,\mu,\lambda}\sum\limits_{i=1}^K
\left[p_i- \!\!\!\!\!
\int\limits_{(1/2)\left(u_{i-1}+u_i\right)}^{(1/2)(u_i+u_{i+1})}\!\!\!\!\!\!\!\!\!\!\!\!\!\!\!
p_{GIG}(u;\,\nu,\mu,\lambda)\,du\right]^2,
\end{multline*}
где $u_0=0$; $u_{K+1}\hm=\infty$.

На практике хорошие результаты показал подход с решением задачи
наименьших квадратов. Для поиска параметров использовался алгоритм
ns2sol, описанный в~книге~\cite{DSch1983}. Указанный алгоритм
доступен во многих статистических пакетах, отличается высоким
быстродействием и~возможностью при желании задавать разумные
интервалы для поиска параметров.

%\vspace*{-9pt}

\section{О практическом выборе сетки
на~первом этапе моди\-фи\-ци\-ро\-ван\-но\-го
сеточного метода разделения дисперсионно-сдвиговых смесей нормальных
законов}

Естественно, что при использовании указанного двухэтапного метода
в~динамическом режиме крайне важным становится вопрос о~выборе
наиболее эффективных и~быстродействующих численных процедур и~их
параметров. В~частности, исключительную важность приобретает
правильный выбор сетки на первом этапе. Рассмотрим этот вопрос
подробнее.

Формально рассматриваемая задача выглядит так: по наблюдаемым
значениям $x_1,\ldots,x_n$ требуется построить статистическую оценку
верхней границы квантилей заданного порядка сме\-ши\-ва\-юще\-го закона так,
чтобы как можно точнее оценить носитель смешивающего распределения.

В дальнейшем будем считать, что $x_1,\ldots,x_n$~--- независимые
реализации случайной величины $X\hm=Y\sqrt{U}+\alpha U$, где $Y$~---
случайная величина со стандартным нормальным распределением, а~$U$~---
независимая от нее случайная величина с~обобщенным обратным
гауссовским распределением. Тогда, очевидно, распределение случайной
величины~$X$ имеет вид~(1). Предположим, что у~случайной величины~$U$
существуют моменты первых двух порядков. Тогда, как несложно видеть,
\begin{equation}
{\sf E}X={\sf E}Y\cdot{\sf E}\sqrt{U}+\alpha{\sf E}U=\alpha{\sf
E}U\,.\label{e5-kor}
\end{equation}
При этом по усиленному закону больших чисел с~вероятностью единица
$\overline x\hm\longrightarrow {\sf E}X$ $(n\hm\to\infty)$, так что при
больших~$n$ справедливо приближенное равенство ${\sf E}X\hm\approx\overline x$
и~с учетом~(\ref{e5-kor})
\begin{equation}
{\sf E}U\approx\fr{\overline x}{\alpha}\,.\label{e6-kor}
\end{equation}
Далее, очевидно,

\columnbreak

\noindent
\begin{multline}
{\sf E}X^2={\sf E}Y^2\cdot{\sf E}U+2\alpha{\sf E}X\cdot{\sf E}U^{3/2}+{}\\
{}+
\alpha^2{\sf E}U^2={\sf E}U+\alpha^2{\sf E}U^2\,.
\label{e7-kor}
\end{multline}

\noindent
Поэтому, обозначив
$$
m^2=\fr{1}{n}\sum\limits_{i=1}^nx_i^2\,,
$$
получаем приближенное равенство ${\sf E}X^2\hm\approx m^2$, так что
с~учетом~(\ref{e6-kor}) и~(\ref{e7-kor}) имеем:
\begin{equation}
{\sf E}U^2\approx\fr{1}{\alpha^2}\left(m^2-\fr{\overline
x}{\alpha}\right)\,.\label{e8-kor}
\end{equation}
Если параметр~$\alpha$ известен, то для определения верхней границы~$u^*$
сетки, накидываемой на носитель распределения случайной
величины~$U$, можно задать малое положительное число~$\varepsilon$
и~воспользоваться требованием
\begin{equation}
{\sf P}(U\geqslant u^*)\leqslant\varepsilon\,.\label{e9-kor}
\end{equation}
А~для гарантированного выполнения требования~(\ref{e9-kor}) можно использовать
неравенство Маркова:
$$
{\sf P}(U\geqslant u^*)\leqslant\fr{{\sf E}U^2}{(u^*)^2}\leqslant \varepsilon\,,
$$
откуда с учетом~(\ref{e8-kor})
$$
(u^*)^2\geqslant\fr{{\sf E}U^2}{\varepsilon}\approx
\fr{1}{\alpha^2\varepsilon}\left( m^2-\fr{\overline x}{\alpha}\right)
$$
или
\begin{equation}
u^*\approx\fr{1}{\alpha\sqrt{\varepsilon}}\sqrt{m^2-
\fr{\overline x}{\alpha}}\,.\label{e10-kor}
\end{equation}

\begin{figure*}[b] %fig1
\vspace*{1pt}
 \begin{center}
 \mbox{%
 \epsfxsize=161.718mm
 \epsfbox{kor-1.eps}
 }
 \end{center}
 \vspace*{-9pt}
\Caption{Примеры применения модифицированного двухэтапного сеточного
ЕМ-ал\-го\-рит\-ма для подгонки обобщенного гиперболического распределения
к искусственным данным, $\beta\hm=0$: (\textit{a})~$n\hm=1000$, $\alpha\hm=0{,}3$,
$\nu\hm=1{,}3$, $\mu\hm=1{,}6$, $\lambda\hm=0{,}2$;
(\textit{б})~$n\hm=1000$, $\alpha\hm=0{,}5$, $\nu\hm=1$, $\mu\hm=1$,
$\lambda\hm=3$;
(\textit{в})~$n\hm=1000$, $\alpha\hm=3$,
 $\nu\hm=1{,}3$, $\mu\hm=1{,}6$, $\lambda\hm=2$;
(\textit{г})~$n\hm=10\,000$,
$\alpha\hm=0{,}3$, $\nu\hm=1{,}3$, $\mu\hm=1{,}6$, $\lambda\hm=0{,}2$}
\end{figure*}


Если же параметр~$\alpha$, определяющий асим\-мет\-рию распределения
случайной величины~$X$, неизвестен, то можно воспользоваться
следующими рассуждениями. Обозначим
$$
q_n=\fr{1}{n}\sum\limits_{i=1}^n{\bf 1}(x_i<0)\,,
$$
где ${\bf 1}(A)$~--- индикаторная функция множества (события)~$A$.
При этом по усиленному закону больших чисел с~вероятностью единица
$q_n\hm\longrightarrow {\sf P}(X\hm<0)$ $(n\hm\to\infty)$, так что при
больших~$n$ справедливо приближенное равенство
\begin{equation}
q_n\approx{\sf P}(X<0)\,.\label{e11-kor}
\end{equation}
Но
\begin{multline}
{\sf P}(X<0)=\int\limits_{0}^{\infty}\Phi
\left(-\alpha\sqrt{u}\right) p_{\mathrm{GIG}}(u;\nu,\mu,\lambda)\,du={}\\
{}=
{\sf E}\Phi\left(-\alpha\sqrt{U}\right)\,.\label{e12-kor}
\end{multline}

\pagebreak

\noindent
Предположим сначала, что $q_n\hm<1/2$. Если~$n$ достаточно велико,
то можно с~большой степенью
 уверенности утверж\-дать, что тогда
$\overline x\hm>0$ и~$-\alpha\hm<0$, т.\,е.
 $\alpha\hm>0$ и,~стало быть, на
положительной полуоси значений аргумента~$u$ функция $\Phi(\alpha u)$
вогнута, т.\,е.\ выпукла вверх. Тогда из~(\ref{e11-kor}) и~(\ref{e12-kor}), дважды
применяя неравенство Иенсена, в~силу монотонности функции~$\Phi$
получаем:
\begin{multline}
1-q_n\approx 1-{\sf E}\Phi\left(-\alpha\sqrt{U}\right)=
          {\sf E}\Phi\left(\alpha\sqrt{U}\right)\leqslant{}\\
          {}\leqslant\Phi
          \left(\alpha{\sf E}\sqrt{U}\right)\leqslant
          \Phi\left(\alpha\sqrt{{\sf E}U}\right)\,.\label{e13-kor}
\end{multline}
Если теперь для $t\hm\in(0,1)$ символом~$v_t$ обозначить $t$-кван\-тиль
стандартного нормального закона, то из~(\ref{e13-kor}) и~(\ref{e6-kor}) вытекает
<<приближенное неравенство>>
$$
v_{1-q_n}\hm\leqslant \alpha\sqrt{{\sf E}U}\,,
$$
т.\,е.
$$
\alpha\geqslant\fr{v_{1-q_n}}{\sqrt{{\sf E}U}}\approx
\fr{v_{1-q_n}\sqrt{\alpha}}{\sqrt{\overline x}}\,,
$$
откуда получаем, что при достаточно больших~$n$
\begin{equation}
\alpha\geqslant\fr{v_{1-q_n}^2}{\overline x}\,.\label{e14-kor}
\end{equation}
Если теперь задать малое положительное число~$\varepsilon$, то
для определения верхней границы~$u^*$ сетки, накидываемой на
носитель распределения случайной величины~$U$, можно воспользоваться
требованием~(\ref{e9-kor}), для гарантированного выполнения которого
с~учетом~(\ref{e6-kor}) и~(\ref{e14-kor}) можно использовать неравенство Маркова:
$$
{\sf P}(U\geqslant u^*)\leqslant \fr{{\sf E}U}{u^*}\approx\fr{\overline
x}{\alpha u^*}\leqslant \fr{(\overline x)^2}{v_{1-q_n}^2 u^*}\leqslant
\varepsilon\,,
$$
откуда окончательно вытекает оценка
\begin{equation}
u^*\approx\fr{(\overline x)^2}{v_{1-q_n}^2 \varepsilon}\,.\label{e15-kor}
\end{equation}

\begin{figure*}[b] %fig2
\vspace*{18pt}
 \begin{center}
 \mbox{%
 \epsfxsize=162.433mm
 \epsfbox{kor-3.eps}
 }
 \end{center}
 \vspace*{-9pt}
\Caption{Примеры применения модифицированного двухэтапного
сеточного ЕМ-ал\-го\-рит\-ма для подгонки обобщенного гиперболического
распределения к~искусственным данным, $n=10\,000$, $\beta\hm=0$:
(\textit{а})~$\alpha\hm=0{,}3$,
$\nu\hm=2$, $\mu\hm=2$, $\lambda\hm=2{,}5$;
(\textit{б})~$\alpha\hm=0{,}5$,  $\nu\hm=1$, $\mu\hm=1$, $\lambda\hm=3$;
(\textit{в})~$\alpha\hm=0{,}8$,
$\nu\hm=1{,}3$, $\mu\hm=1{,}6$, $\lambda\hm=2$;
(\textit{г})~$\alpha\hm=1{,}3$, $\nu\hm=2$, $\mu\hm=2$, $\lambda\hm=2{,}5$}
\end{figure*}



В случае $q_n\hm\geqslant1/2$, если $n$ достаточно велико, то можно
с~большой степенью уверенности утверж\-дать, что $\overline x\hm\leqslant 0$
и~$-\alpha\hm\geqslant 0$, т.\,е.\ на положительной\linebreak\vspace*{-12pt}

\pagebreak

%\end{multicols}


%\begin{multicols}{2}

\noindent
 полуоси значений аргумента~$u$
функция $\Phi(-\alpha u)$ вогнута, т.\,е.\ выпукла вверх. Тогда
из~(\ref{e11-kor}) и~(\ref{e12-kor}), дважды применяя неравенство Иенсена, в~силу
монотонности функции~$\Phi$ получаем
$$
q_n\approx {\sf E}\Phi\left(-\alpha\sqrt{U}\right)\leqslant
\Phi\left(-\alpha\sqrt{{\sf E}U}\right)\,,
$$
откуда вытекает <<приближенное неравенство>> $v_{q_n}\hm \leqslant
-\alpha\sqrt{{\sf E}U}$,
т.\,е.
$$
-\alpha\geqslant\fr{v_{q_n}}{\sqrt{{\sf E}U}}\approx
\fr{v_{q_n}\sqrt{|\alpha|}}{\sqrt{|\overline x|}}
$$
и при достаточно больших~$n$
\begin{equation}
|\alpha|\geqslant\fr{v_{q_n}^2}{|\overline x|}\,.\label{e16-kor}
\end{equation}
Для определения верхней границы~$u^*$ сетки, накидываемой на
носитель распределения случайной величины~$U$, снова зададим малое
положительное число~$\varepsilon$ и~потребуем, чтобы было
справедливо условие~(\ref{e9-kor}), для гарантированного выполнения которого
с~учетом~(\ref{e6-kor}) и~(\ref{e16-kor}) используем неравенство Маркова и~тот факт, что
$\mathrm{sign}\, \overline x\hm=\mathrm{sign}\,\alpha$ при достаточно
больших~$n$:
\begin{multline}
{\sf P}(U\geqslant u^*)\leqslant \fr{{\sf E}U}{u^*}\approx
\fr{\overline x}{\alpha u^*}=
\fr{|\overline x|}{|\alpha| u^*} \leqslant{}\\
{}\leqslant
\fr{(\overline x)^2}{v_{q_n}^2 u^*}\leqslant
\varepsilon\,.\label{e17-kor}
\end{multline}
В силу симметричности нормального распределения $v_{t}\hm=-v_{1-t}$ для
любого $t\hm\in(0,1)$, поэтому $v_{q_n}^2\hm=v_{1-q_n}^2$ и~в~случае
$q_n\hm\geqslant1/2$ соотношение~(\ref{e17-kor}) снова приводит к~оценке~(\ref{e15-kor}).

Справедливости ради необходимо отметить, что оценки~(\ref{e10-kor}) и~(\ref{e15-kor})
являются завышенными, но они гарантируют, что
$(1-\varepsilon)$-почти-весь носитель распределения случайной
величины~$U$ будет лежать внутри интервала $[0, u^*]$.

\section{Результаты численных экспериментов}

Приводимые в~данном разделе графики иллюстрируют качество работы
модифицированного сеточного метода разделения дис\-пер\-си\-он\-но-сдви\-го\-вых
смесей нормальных законов на примере его\linebreak применения к~оцениванию
параметров обоб\-щенных гиперболических распределений с~ис\-поль\-зованием
указанного алгоритма выбора сетки\linebreak с~умеренным чис\-лом узлов $K\hm=40$.
Для вы\-чис\-ле\-ний использовались искусственно сгенерированные выборки
объемов $n\hm=1000$ и~$n\hm=10\,000$ с~разными наборами параметров, значения
которых указаны на рисунках. На рис.~1 и~2 изображены гистограммы
(серые столбики) и~графики
истинной плот\-ности (штриховые линии), промежуточной
оценки, полученной сеточным ЕМ-ал\-го\-рит\-мом (пунктирные линии)
и~итоговой оценки (непрерывные линии). На рис.~1 и~2 так\-же указаны
значения полученных оценок параметров. Как видно из приводимых
рисунков, параметры~$\alpha$ оцениваются очень точно. Точность
оценок остальных параметров удовлетворительная и~может быть повышена
за счет использования более частых сеток и~более чувствительных
критериев остановки ЕМ-ал\-го\-рит\-ма на первом этапе. Следует отметить,
что даже в~тех случаях, в~которых наблюдаются заметные расхождения
оценок параметров и~их точных значений, оценки самих плотностей
довольно \mbox{точны}.




{\small\frenchspacing
 {%\baselineskip=10.8pt
 \addcontentsline{toc}{section}{References}
 \begin{thebibliography}{99}
\bibitem{k2011}
\Au{Королев В.\,Ю.} Ве\-ро\-ят\-но\-ст\-но-ста\-ти\-сти\-че\-ские методы
декомпозиции волатильности хаотических процессов.~--- М.: Изд-во
Московского ун-та, 2011.

\bibitem{n2013}
\Au{Назаров А.\,Л.} Приближенные методы разделения смесей
вероятностных распределений: Дисс.\ \ldots\  канд. физ.-мат. наук.~--- М.:
МГУ им.\ М.\,В.~Ломоносова, 2013.

\bibitem{BN1977}
\Au{Barndorff-Nielsen~O.-E.} Exponentially decreasing distributions
for the logarithm of particle size~// Proc. Roy. Soc. Lond.~A,
1977. Vol.~353. P.~401--419.

\bibitem{BN1978}
\Au{Barndorff-Nielsen~O.-E.} Hyperbolic distributions and
distributions of hyperbolae~// Scand. J. Statist., 1978. Vol.~5.
P.~151--157.

\bibitem{BN1982}
\Au{Barndorff-Nielsen~O.-E., Kent~J., S\!{\!\ptb{\!\o}}\,rensen~M.} Normal
variance-mean mixtures and $z$-distributions~// Int. Statist. Rev.,
1982. Vol.~50. No.\,2. P.~145--159.

\bibitem{ks2012}
\Aue{Королев В.\,Ю., Соколов И.\,А.} Скошенные распределения
Стьюдента, дисперсионные гам\-ма-рас\-пре\-де\-ле\-ния и~их обобщения как
асимптотические аппроксимации~// Информатика и~её применения, 2012.
Т.~6. Вып.~1. С.~2--10.

\bibitem{zk2013}
\Au{Закс Л.\,М., Королев В.\,Ю.} Обобщенные дисперсионные
гам\-ма-рас\-пре\-де\-ле\-ния как предельные для случайных сумм~// Информатика
и её применения, 2013. Т.~7. Вып.~1. С.~105--115.

\bibitem{k2013}
\Au{Королев В.\,Ю.} Обобщенные гиперболические
распределения как предельные для случайных сумм~// Тео\-рия
вероятностей и~ее применения, 2013. Т.~58. Вып.~1. С.~117--132.

\bibitem{kckg2013}
\Au{Королев В.\,Ю., Черток А.\,В., Корчагин~А.\,Ю.,
Горшенин~А.\,К.} Ве\-ро\-ят\-но\-ст\-но-ста\-ти\-сти\-че\-ское моделирование
информационных потоков в~сложных финансовых системах на основе
высокочастотных данных~// Информатика и~её применения, 2013. Т.~7.
Вып.~1. С.~12--21.

\bibitem{p2004}
\Au{Protassov R.\,S.} EM-based maximum likelihood parameter
estimation for a~multivariate generalized hyperbolic distribution
with fixed~$\lambda$~// Statistics Computing, 2004. Vol.~14.
P.~67--77.

\bibitem{kn2010}
\Au{Королев В.\,Ю., Назаров А.\,Л.} Разделение смесей
вероятностных распределений при помощи сеточных методов моментов и~максимального правдоподобия~//
Автоматика и~телемеханика, 2010. Вып.~3. С.~98--116.

\bibitem{DSch1983}
\Au{Dennis J.\,E., Schnabel R.\,B.} Numerical methods for
unconstrained optimization and nonlinear equations.~--- Englewood
Cliffs: Prentice-Hall, 1983. 378~p.
 \end{thebibliography}

 }
 }

\end{multicols}

\vspace*{-6pt}

\hfill{\small\textit{Поступила в редакцию 01.10.14}}

\newpage

%\vspace*{12pt}

%\hrule

%\vspace*{2pt}

%\hrule

%\vspace*{12pt}

\def\tit{A MODIFIED GRID METHOD FOR~STATISTICAL SEPARATION
OF~NORMAL VARIANCE-MEAN MIXTURES}

\def\titkol{A modified grid method for statistical separation
of~normal variance-mean mixtures}

\def\aut{V.\,Yu.~Korolev$^{1,2}$ and~A.\,Yu.~Korchagin$^1$}

\def\autkol{V.\,Yu.~Korolev and~A.\,Yu.~Korchagin}

\titel{\tit}{\aut}{\autkol}{\titkol}

\vspace*{-9pt}


\noindent
$^1$Faculty of Computational Mathematics and Cybernetics,
M.\,V.~Lomonosov Moscow State University,\linebreak
$\hphantom{^1}$1-52 Leninskiye Gory, GSP-1, Moscow 119991, Russian Federation


\noindent
$^2$Institute of Informatics Problems, Russian Academy of Sciences,
44-2~Vavilov Str., Moscow 119333, Russian\linebreak
$\hphantom{^1}$Federation

\def\leftfootline{\small{\textbf{\thepage}
\hfill INFORMATIKA I EE PRIMENENIYA~--- INFORMATICS AND
APPLICATIONS\ \ \ 2014\ \ \ volume~8\ \ \ issue\ 4}
}%
 \def\rightfootline{\small{INFORMATIKA I EE PRIMENENIYA~---
INFORMATICS AND APPLICATIONS\ \ \ 2014\ \ \ volume~8\ \ \ issue\ 4
\hfill \textbf{\thepage}}}

\vspace*{3pt}

\Abste{A~modified two-stage grid method for
statistical separation of normal variance-mean mixtures is described
as an alternative to a pure EM (expectation-maximization) algorithm.
At the first stage of this
algorithm, a~discrete approximation is constructed to the mixing
distribution. At the second stage, the obtained discrete
distribution is approximated by an absolutely continuous
distribution from a~predetermined family, say, by a generalized
inverse Gaussian distribution. The convergence of this two-stage
procedure is discussed. The monotonicity of the grid procedure used
at the first stage is proved. The problem of the optimal choice of
the parameters of the method is discussed in detail. First of all,
the problem of the optimal choice of the grid thrown on the support
of the mixing distribution is considered. Statistical estimators are
proposed for the quantiles of the mixing law. The efficiency of the
method is illustrated by examples of its application to the
estimation of the parameters of generalized hyperbolic
distributions.}

\smallskip

\KWE{mixture of probability distributions; normal
variance-mean mixture; generalized hyperbolic distribution;
EM-algorithm; grid method of separation of mixtures}

\DOI{10.14357/19922264140402}

\Ack
\noindent
The research was supported by the Russian Science Foundation (project 14-11-00364).

%\vspace*{3pt}

  \begin{multicols}{2}

\renewcommand{\bibname}{\protect\rmfamily References}
%\renewcommand{\bibname}{\large\protect\rm References}



{\small\frenchspacing
 {%\baselineskip=10.8pt
 \addcontentsline{toc}{section}{References}
 \begin{thebibliography}{99}
 \bibitem{k2011eng}
 \Aue{Korolev, V.\,Yu.} 2011.
\textit{Veroyatnostno-statisticheskie metody dekompozitsii
volatil'nosti khaoticheskikh protsessov}
[Probabilistic and statistical methods for the decomposition of volatility
of chaotic processes].
Moscow: Moscow University Press. 510~p.

\bibitem{n2013eng}
\Aue{Nazarov, A.\,L.} 2013.
{Priblizhennye metody razdeleniya smesey veroyatnostnykh raspredeleniy}
[Approximate methods for the decomposition of volatility of chaotic processes].
Ph.D. Thesis. Moscow: Moscow State University.

\bibitem{BN1977eng}
\Aue{Barndorff-Nielsen, O.\,E.} 1977.
Exponentially decreasing distributions for the logarithm of particle size.
\textit{Proc. Roy. Soc. Lond. A} 353:401--419.

\bibitem{BN1978eng}
\Aue{Barndorff-Nielsen, O.\,E.} 1978.
Hyperbolic distributions and distributions of hyperbolae.
\textit{Scand. J. Statist.} 5:151--157.

\bibitem{BN1982eng}
\Aue{Barndorff-Nielsen, O.\,E., J.~Kent, and M.~S\!{\ptb{\o}}rensen}. 1982.
Normal variance-mean mixtures and $z$-distributions.
\textit{Int. Statist. Rev.} 50(2):145--159.

\bibitem{ks2012eng}
\Aue{Korolev, V.\,Yu., and I.\,A. Sokolov}. 2012.
{Skoshennye raspredeleniya St'yudenta, dispersionnye
gam\-ma-ras\-pre\-de\-le\-niya i~ikh obobshcheniya kak asimptoticheskie
approksimatsii}
[Skewed Student's distributions, variance gamma distributions, and their
generalizations as asymptotic approximations].
\textit{Informatika i ee Primeneniya}~--- \textit{Inform. Appl.} 6(1):2--10.

\bibitem{zk2013eng}
\Aue{Korolev, V.\,Yu., and L.\,M.~Zaks}. 2013.
{Obobshchennye dispersionnye gam\-ma-ras\-pre\-de\-le\-niya kak
predel'nye dlya sluchaynykh summ}
[Generalized variance gamma distributions as limiting for random sums].
\textit{Informatika i ee Primeneniya}~--- \textit{Inform. Appl.} 7(1):105--115.

\bibitem{k2013eng} \Aue{Korolev, V.\,Yu.} 2013.
{Obobshchennye giperbolicheskie raspredeleniya kak predel'nye dlya sluchaynykh summ}
[Generalized hyperbolic distributions as limiting for random sums]
\textit{Theory Probab. Appl.} 58(1):117--132.

\bibitem{kckg2013eng}
\Aue{Korolev, V.\,Yu., A.\,V. Chertok, A.\,Yu.~Korchagin, and A.\,K.~Gorshenin}.
2013. {Ve\-ro\-yat\-no\-st\-no-sta\-ti\-sti\-che\-skoe
mo\-de\-li\-ro\-va\-nie informatsionnykh potokov v~slozhnykh finansovykh sistemakh
na osnove vysokochastotnykh dannykh}
[Probability and statistical modeling of information flows in complex
financial systems from high-frequency data].
\textit{Informatika i~ee Primeneniya}~--- \textit{Inform.  Appl.} 7(1):12--21.

\bibitem{p2004eng-1}
\Aue{Protassov, R.\,S.} 2004.
EM-based maximum likelihood parameter estimation for a multivariate
generalized hyperbolic distribution with fixed~$\lambda$.
\textit{Statistics Computing} 14:67--77.

\bibitem{kn2010eng-1}
\Aue{Korolev, V.\,Yu., and A.\,L.~Nazarov}. 2010.
{Razdelenie smesey veroyatnostnykh raspredeleniy pri pomoshchi
setochnykh metodov momentov i~maksimal'nogo pravdopodobiya}
[Separation of mixtures using grid moment-based methods and maximum likelihood].
\textit{Avtomatika i~Telemekhanika} [Automatics and Telemechanics] 3:98--116.

\bibitem{DSch1983eng}
\Aue{Dennis, J.\,E., and R.\,B.~Schnabel}. 1983.
\textit{Numerical methods for unconstrained optimization and nonlinear equations}.
Englewood Cliffs: Prentice-Hall. 378~p.


\end{thebibliography}

 }
 }

\end{multicols}

\vspace*{-6pt}

\hfill{\small\textit{Received October 01, 2014}}

\vspace*{-18pt}

\Contr

\noindent
\textbf{Korolev Victor Yu.} (b.\ 1954)~---
Doctor of Science in physics and mathematics, professor,
Department of Mathematical Statistics, Faculty of Computational Mathematics
and Cybernetics, M.\,V.~Lomonosov Moscow State University,
1-52 Leninskiye Gory, GSP-1, Moscow 119991, Russian Federation;
leading scientist, Institute of Informatics Problems,
Russian Academy of Sciences, 44-2~Vavilov Str., Moscow 119333, Russian
Federation; victoryukorolev@yandex.ru

\vspace*{3pt}

\noindent
\textbf{Korchagin Alexander Yu.} (b.\ 1989)~---
PhD student, Faculty of Computational Mathematics and Cybernetics,
M.\,V.~Lomonosov Moscow State University,
1-52 Leninskiye Gory, GSP-1, Moscow 119991, Russian Federation;
sasha.korchagin@gmail.com


\label{end\stat}

\renewcommand{\bibname}{\protect\rm Литература}   %4
\def\stat{gorshenin}

\def\tit{ЗАШУМЛЕНИЕ ДАННЫХ КОНЕЧНЫМИ СМЕСЯМИ НОРМАЛЬНЫХ 
И~ГАММА-РАСПРЕДЕЛЕНИЙ\\ С~ПРИМЕНЕНИЕМ К~ЗАДАЧЕ ОКРУГЛЕНИЯ НАБЛЮДЕНИЙ$^*$}

\def\titkol{Зашумление данных конечными смесями нормальных 
и~гамма-распределений с~применением к~задаче округления} % наблюдений}

\def\aut{А.\,К.~Горшенин$^1$}

\def\autkol{А.\,К.~Горшенин}

\titel{\tit}{\aut}{\autkol}{\titkol}

\index{Горшенин А.\,К.}
\index{Gorshenin A.\,K.}


{\renewcommand{\thefootnote}{\fnsymbol{footnote}} \footnotetext[1]
{Работа выполнена при поддержке РНФ (проект 18-71-00156).}}


\renewcommand{\thefootnote}{\arabic{footnote}}
\footnotetext[1]{Институт проблем информатики Федерального исследовательского центра 
<<Информатика и~управление>> Российской академии наук, \mbox{agorshenin@frccsc.ru}}

\vspace*{-12pt}




\Abst{Во многих реальных задачах проводится статистический анализ данных, 
содержащих дополнительные ошибки измерения, в~том числе в~виде округления, 
что в~ряде ситуаций может приводить к~достаточно существенным искажениям. 
В~настоящей статье для одной из возможных моделей округления получены оценки 
для неизвестного математического ожидания наблюдений в~предположении, что 
исходные данные дополнительно зашумлены с~по\-мощью случайных величин, 
име\-ющих распределения типа конечных смесей нормальных и~гам\-ма-за\-ко\-нов. 
Построены доверительные интервалы для неизвестного математического ожидания 
с~использованием уточненной оценки для дисперсии целой части случайной величины. 
Обсуждается алгоритм определения значения параметра для искусственного шума, 
добавление которого к~исходным данным способствует повышению качества работы 
метода скользящего разделения смесей.}

\KW{зашумленные данные; округленные наблюдения; конечные смеси нормальных 
распределений; конечные смеси гам\-ма-рас\-пре\-де\-ле\-ний; доверительные интервалы;  
метод скользящего разделения смесей}

\DOI{10.14357/19922264180304}
  
\vspace*{-4pt}


\vskip 10pt plus 9pt minus 6pt

\thispagestyle{headings}

\begin{multicols}{2}

\label{st\stat}


\section{Введение}

Во многих реальных задачах данные, являющиеся непрерывными по своей сути, 
регистрируются с~помощью инструментов, вносящих дополнительные ошибки 
измерения, в~том чис\-ле в~виде округления. Таким образом, статистический 
анализ проводится не для исходных, а для преобразованных некоторым 
случайным образом наблюдений, что в~ряде ситуаций может приводить к~достаточно
 существенным искажениям.

Для преодоления данной проблемы развивались различные подходы, в~том числе 
на основе смешанных моделей (см., например, статью~\cite{Wright2003}, в~которой 
различные компоненты  используются для пред\-став\-ле\-ния уровней округления). 
В~работе~\cite{Bai2009} приводятся результаты для моделей авторегрессии и~скользящего 
среднего для округленных данных, а~в~статье~\cite{Zhang2010} эти результаты 
развиваются и~исследуются их асимптотические свойства. 
В~статье~\cite{Zhao2012} исследован метод оценивания па\-ра\-мет\-ров конечных смесей 
вероятностных распределений (в~том чис\-ле, и~многомерных) 
на основе использования EM (expectation-maximization) 
алгоритма~\cite{Korolev2011-i} с~\mbox{целью} получения состоятельных 
и~асимптотически нормальных оценок.

В настоящей статье развиваются результаты для моделей округления, 
описанных в~работах~\cite{Ushakov2015,Ushakov2017a,Ushakov2017b}. 
В~их рамках будут получены оценки для неизве\-ст\-ного математического ожидания 
наблюдений в~предположении, что исходные данные зашумлены с~по\-мощью случайных 
величин, имеющих распределения типа конечных смесей нормальных и~гам\-ма-за\-ко\-нов. 
Это позволяет учесть большее количество случайных факторов, влия\-ющих на величину 
<<дополнительной>> ошибки. Также будут построены доверительные интервалы для 
неизвестного математического ожидания. Выражения для гам\-ма-рас\-пре\-де\-ле\-ний 
получены впервые. Также обсуждается алгоритм определения значения па\-ра\-мет\-ра для 
искусственного шума, добавление которого к~исходным данным способствует 
повышению качества работы метода скользящего разделения смесей~\cite{Gorshenin2016}.

\vspace*{-12pt}

\section{Предположения и~базовые отношения}

Для сокращения формулировок теорем в~сле\-ду\-ющих разделах сделаем ряд 
предположений, на которые будем ссылаться в~дальнейшем. Итак, пусть:
\begin{itemize}
\item[(A)] $X_1,X_2,\ldots$~--- независимые одинаково распределенные 
случайные величины с~неизвестным математическим ожиданием ${\sf E}_X\hm<+\infty$;
\item[(B)] $\varepsilon_1,\varepsilon_2,\ldots$~--- независимые одинаково 
распределенные случайные величины с~математическим ожиданием 
${\sf E}_\varepsilon\hm<+\infty$; %\label{B}
\item[(C)] $X_1,X_2,\ldots$ и~$\varepsilon_1,\varepsilon_2,\ldots$ 
являются независимыми;
\item[(D)] $Y_j=\left[X_j+\varepsilon_j+1/2\right]$ для всех $j\hm=1,2,\ldots$ 
представляют собой округление значения суммы случайных величин $X_j\hm+\varepsilon_j$ 
до ближайшего целого сверху (при этом запись~$[\cdot]$ соответствует целой 
части выражения).
\end{itemize}

В рамках данных предположений в~статье будут рассмотрены вопросы качества 
приближения неизвестного математического ожидания~${\sf E}_X$ для исходных данных 
в~ситуации, когда наблюдения для анализа получены с~аддитивной ошибкой c известными 
распределениями (см.\ предположение~(B)) и~дополнительно округляются до 
ближайшего целого (см.\ предположение~(D)).

Заметим, что в~силу усиленного закона больших чисел справедливы следующие выражения:
\begin{multline}
\fr{1}{n}\sum\limits_{j=1}^n Y_j\xrightarrow[n\to\infty]{\text{п.н.}}
{\sf E}_Y\equiv\mathbb{E}\left[X_1+\varepsilon_1+\fr{1}{2}\right]={}\\
{}=\mathbb{E}\left(X_j+\varepsilon_j+\fr{1}{2}\right)-\mathbb{E}
\left\{X_j+\varepsilon_j+\fr{1}{2}\right\}={}\\
{}={\sf E}_X+{\sf E}_\varepsilon+\fr{1}{2}-\mathbb{E}\left\{X_j+\varepsilon_j+\fr{1}{2}\right\}. 
\label{Law}
\end{multline}

Запись $\{\cdot\}$ в~формуле~\eqref{Law} соответствует дробной 
части выражения, а~п.н.\ обозначает сходимость в~смысле почти наверное.

Для доказательства результатов в~дальнейшем потребуется следующее 
представления для дробной части  абсолютно непрерывной случайной величины~$Z$ 
с~абсолютно  интегрируемой характеристической функцией~$\varphi_Z(t)$
 (см., например, Лемму~4 в~работе~\cite{Ushakov2017b}):
\begin{equation}
\label{Fract}
\mathbb{E}\{Z\}=\fr{1}{2}-\sum\limits_{n=1}^\infty 
\fr{\mathrm{Im}\left (\varphi_Z(2\pi n)\right)}{\pi n}\,.
\end{equation}

Через $\mathrm{Im}\,(\cdot)$ в~формуле~\eqref{Fract} обозначена мнимая часть 
соответствующей функции.

При построении доверительных интервалов в~дальнейшем будет 
использована следующая оценка, справедливая для любой случайной величины~$Z$:
\begin{equation}
\mathbb{D}[Z]\leqslant \left(\sqrt{\mathbb{D} Z}+\fr{1}{2}\right)^2.
\label{Var}
\end{equation}
Она может быть проверена непосредственно с~учетом представления 
$\mathbb{D} [Z]\hm=\mathbb{D}\left(Z\hm-\{Z\}\right)$, неравенства 
Ко\-ши--Бу\-ня\-ков\-ско\-го для ковариации и~соотношения 
 $\mathbb{D}\{Z\}\hm\leqslant 1/4$, справедливого для любой случайной величины~$Z$ 
 (см., например, статью~\cite{Ushakov2017b}). Отметим, что данная оценка 
 является более точной по сравнению с~использованным для аналогичных 
 целей в~работе~\cite{Ushakov2017b} соотношением 
 $\mathbb{D} [Z]\hm\leqslant 2\mathbb{D} Z\hm+1/2$. Действительно,
\begin{equation*}
2\mathbb{D} Z+\fr{1}{2}-\left(\sqrt{\mathbb{D} Z}+\fr{1}{2}\right)^2=
\left(\sqrt{\mathbb{D} Z}-\fr{1}{2}\right)^2\geqslant0\,,
\end{equation*}
причем для всех $\sqrt{\mathbb{D} Z}\hm\neq {1}/{2}$ 
данное неравенство является строгим.

\section{Конечные смеси нормальных законов}

Для случайной величины~$X$, имеющей распределение типа 
конечной смеси нормальных законов~\cite{Korolev2011-i} с~параметрами 
${\bf a}\hm=(a_1,\ldots, a_k)$, $a_j\hm\in \mathbb{R}$, 
$\boldsymbol{\sigma}\hm=(\sigma_1,\ldots, \sigma_k)$, 
$\sigma_j\hm>0$, ${\bf p}\hm=(p_1,\ldots, p_k)$, $p_j\hm\geqslant 0$, 
$\sum\nolimits_{j=1}^{k}p_j\hm=1$, плот\-ность которого задается выражением
\begin{equation}
f_X(x)=\sum\limits_{j=1}^{k}\fr{p_j}{\sigma_j\sqrt{2\pi}}\,e^{-(x-a_j)^2/(2\sigma_j^2)}\,,
\label{FinNormMixt}
\end{equation}
характеристическая функция имеет вид:
\begin{equation}
\varphi_X(t)=\int\limits_{-\infty}^{+\infty}\!\!e^{itx} f_X(x)\, dx = 
\sum\limits_{j=1}^{k}p_j e^{ita_j-\sigma_j^2 t^2/2}.
\label{ChiFinNormMixt}
\end{equation}

Абсолютная интегрируемость  $\varphi_X(t)$ вытекает из свойств 
характеристической функции нормального распределения. 
Заметим, что в~точке $t\hm=2\pi n$ выражение~\eqref{ChiFinNormMixt} принимает 
сле\-ду\-ющий вид:
\begin{equation}
\label{ChiFinNormMixt2npi}
\varphi_X(2\pi n)= \sum\limits_{j=1}^{k}p_j e^{-2\pi^2 \sigma_j^2 n^2}\,.
\end{equation}

Рассмотрим вопрос точности оценивания неизвестного математического ожидания~${\sf E}_X$ 
при до\-бав\-ле\-нии зашумления.

\smallskip

\noindent
\textbf{Теорема~1.}\ 
\textit{Пусть выполнены предположения}~(A)--(D), 
\textit{причем случайные величины~$\varepsilon_j$, $j\hm=1,2,\ldots$, 
имеют распределение типа конечной $k$-ком\-по\-нент\-ной смеси нормальных законов 
вида}~\eqref{FinNormMixt} \textit{с~па\-ра\-мет\-ра\-ми~${\bf a}$, $\boldsymbol{\sigma}$ 
и~${\bf p}$. Тогда}
\begin{equation}
\label{Th1Eq}
\left\lvert {\sf E}_Y-{\sf E}_X\right\rvert \leqslant 
A+\fr{1}{\pi}\left(1+\fr{1}{4\pi^2\sigma^2}\right)e^{-2\pi^2\sigma^2}\,, 
\end{equation}
\textit{где} $A=\max(|a_1|,\ldots,|a_k|)$, $\sigma\hm=\min(\sigma_1,\ldots,\sigma_k)$.

\smallskip


\noindent
Д\,о\,к\,а\,з\,а\,т\,е\,л\,ь\,с\,т\,в\,о\,.\ \
С~учетом пред\-став\-ле\-ний~\eqref{Law},~\eqref{Fract} и~\eqref{ChiFinNormMixt2npi}, 
ограниченности модуля характеристической функции, а~также не\-за\-ви\-си\-мости 
случайных величин~$X_j$ и~$\varepsilon_j$ имеем:
\begin{multline*}
\left\lvert {\sf E}_Y-{\sf E}_X\right\rvert =
\left\lvert {\sf E}_\varepsilon+\fr{1}{2}-\mathbb{E}\left\{X_j+
\varepsilon_j+\fr{1}{2}\right\}\right\rvert={}\\
{}=\left\lvert {\sf E}_\varepsilon+\sum\limits_{n=1}^\infty
\fr{\mathrm{Im} \left(\varphi_{X_j}(2\pi n)\varphi_{\varepsilon_j}(2\pi n)
\varphi_{1/2}(2\pi n)\right)}{\pi n}\right\rvert={}\\
=\left\lvert 
\vphantom{\fr{(-1)^n\sum\nolimits_{j=1}^{k}p_j e^{-2\pi^2 \sigma_j^2 n^2} 
\mathrm{Im} \left(\varphi_{X_j}(2\pi n)\right)}{\pi n}}
{\sf E}_\varepsilon+{}\right.\\
\left.{}+\sum\limits_{n=1}^\infty
\fr{\mathrm{Im} \left(\varphi_{X_j}(2\pi n) 
\sum\nolimits_{j=1}^{k}p_j e^{-2\pi^2 \sigma_j^2 n^2} 
e^{\pi n}\right)}{\pi n}\right\rvert={}\\
{}=\left\lvert 
\vphantom{\fr{(-1)^n\sum\nolimits_{j=1}^{k}p_j e^{-2\pi^2 \sigma_j^2 n^2} 
\mathrm{Im} \left(\varphi_{X_j}(2\pi n)\right)}{\pi n}}
{\sf E}_\varepsilon+{}\right.\\
\left.{}+\sum\limits_{n=1}^\infty
\fr{(-1)^n\sum\nolimits_{j=1}^{k}p_j e^{-2\pi^2 \sigma_j^2 n^2} 
\mathrm{Im} \left(\varphi_{X_j}(2\pi n)\right)}{\pi n}\right\rvert\leqslant{}\\
{}\leqslant \left\lvert {\sf E}_\varepsilon\right\rvert+\left\lvert\
\sum\limits_{j=1}^{k}p_j\sum\limits_{n=1}^\infty 
\fr{1}{\pi n} e^{-2\pi^2 \sigma_j^2 n^2}\right\rvert\leqslant {}\\
\\
{}\leqslant
\max\left(|a_1|,\ldots,|a_k|\right)+{}\\
{}+\sum\limits_{j=1}^{k} 
\fr{p_j}{\pi} \left(\!1+\fr{1}{4\pi^2\sigma_j^2}\!\right)\!e^{-2\pi^2\sigma_j^2}\leqslant{}\\
{}\leqslant
A+\fr{1}\pi\left(1+\fr{1}{4\pi^2\sigma^2}\right)e^{-2\pi^2\sigma^2}\,.
\end{multline*}

Справедливость использованной оценки 
\begin{equation*}
\sum\limits_{n=1}^\infty
\fr{e^{-2\pi^2 \sigma_j^2 n^2}}{n}\leqslant 
\left(1+\fr{1}{4\pi^2\sigma_j^2}\right)e^{-2\pi^2\sigma_j^2}
\end{equation*}
может быть проверена непосредственно (например, см.\ доказательство Теоремы~6 
в~статье~\cite{Ushakov2017b}).~\hfill$\square$

\smallskip

\noindent
\textbf{Замечание~1.}
В~случае, если зашумление производится нормально распределенными случайными 
величинами c нулевыми средними (т.\,е.\ в~формуле~\eqref{Th1Eq} необходимо считать 
$A\hm=0$, $k\hm=1$), то Тео\-ре\-ма~1 совпадает с~результатом, 
полученным в~работе~\cite{Ushakov2017b}.


\smallskip

Рассмотрим вопросы построения доверительного интервала для неизвестного 
математического ожидания~${\sf E}_X$ в~предположении, что случайные величины~$X_j$ не 
содержат ошибок измерения, а~все погрешности учтены исключительно в~за\-шум\-ля\-ющих 
элементах~$\varepsilon_j$.

\smallskip

\noindent
\textbf{Теорема~2.}\ 
\textit{Пусть выполнены предположения}~(A)--(D), 
\textit{причем случайные величины~$\varepsilon_j$, $j\hm=1,2,\ldots$, имеют 
распределение типа конечной $k$-ком\-по\-нент\-ной смеси нормальных законов 
вида}~\eqref{FinNormMixt} \textit{с~параметрами~${\bf a}$, $\boldsymbol{\sigma}$ 
и~${\bf p}$, а~случайные величины} $X_j\stackrel{\text{п.н.}}{=}{\sf E}_X$, $j\hm=1,2,\ldots$ 
\textit{Тогда доверительный интервал для~${\sf E}_X$ при условии $0\hm<\alpha\hm<1$ имеет вид}:
\begin{equation} 
\label{Th2Eq}
\hat{{\sf E}}_X - f({\bf a},\boldsymbol{\sigma},\alpha,n) 
\leqslant {\sf E}_X \leqslant  \hat{{\sf E}}_X + f({\bf a},\boldsymbol{\sigma},\alpha,n),
\end{equation}
\textit{где}

\vspace*{-2pt}

\noindent
\begin{align}
\hat{{\sf E}}_X&=\fr{1}{n} \sum\limits_{j=1}^{n} Y_j\,; \label{Th2hatE}\\
f({\bf a},\boldsymbol{\sigma},\alpha,n)&=
\fr{z_{1-{\alpha}/2}}{\sqrt{n}} \left(\sqrt{A^2+\Sigma^2}+\fr{1}{2}\right) +{}\notag\\
&{}+A+\fr{1}\pi\left(1+\fr{1}{4\pi^2\sigma^2}\right)e^{-2\pi^2\sigma^2}\,;
  \label{Th2f}
\end{align}
\textit{$z_{1-{\alpha}/2}$~--- $\left(1-{\alpha}/2\right)$-кван\-тиль 
стандартного нормального распределения; $A\hm=\max(|a_1|,\ldots,|a_k|)$; 
$\Sigma\hm=\max(\sigma_1,\ldots,\sigma_k)$; $\sigma\hm=\min(\sigma_1,\ldots,\sigma_k)$}. 


\smallskip

\noindent
\noindent
Д\,о\,к\,а\,з\,а\,т\,е\,л\,ь\,с\,т\,в\,о\,.\ \
Из центральной предельной тео\-ре\-мы с~учетом условия~(A) следует, 
что величина~$\hat{{\sf E}}_X$~\eqref{Th2hatE} асимптотически нормальна с~математическим 
ожиданием 
\begin{equation}
{\sf E}_Y\equiv \mathbb{E}\left[{\sf E}_X+\varepsilon_1+\fr{1}{2}\right] \label{EY}
\end{equation}
и дисперсией
\begin{equation}
\fr{1}{n} {\sf D}_Y\equiv \fr{1}{n}\mathbb{D}\left[{\sf E}_X+\varepsilon_1+
\fr{1}{2}\right]. \label{DY}
\end{equation}

Воспользовавшись оценкой~\eqref{Var}, получим:

\vspace*{-2pt}

\noindent
\begin{multline*}
{\sf D}_Y \leqslant  \left(\sqrt{\mathbb{D} \left({\sf E}_X+\varepsilon_1+\fr{1}{2}\right)}+
\fr{1}{2}\right)^2={}\\
{}=
\left(\sqrt{\mathbb{D}\varepsilon_1}+\fr{1}{2}\right)^2= {}\\
{}= \left(\sqrt{\sum\limits_{j=1}^{k}p_j\left(\left(a_j-\sum\limits_{t=1}^{k}
p_t a_t\right)^2+\sigma_j^2\right)}+\fr{1}{2}\right)^2\leqslant {}\\ 
{}\leqslant \left(\sqrt{A^2+\Sigma^2}+\fr{1}{2}\right)^2\,.
\end{multline*}
Тогда доверительный интервал уровня $1\hm-\alpha$ для математического ожидания~${\sf E}_Y$ 
имеет вид:
\begin{equation*}
\mathbb{P}\left(\left\lvert \hat{{\sf E}}_X-{\sf E}_Y\right\rvert \leqslant 
\fr{z_{1-{\alpha}/2}}{\sqrt{n}} 
\left(\sqrt{A^2+\Sigma^2}+\fr{1}{2}\right)\right)\geqslant 1-\alpha\,.
\end{equation*}

\begin{table*}[b]\small
\begin{center}

\begin{tabular}{|c|c|c|c|c|c|c|c|}
\multicolumn{7}{p{100mm}}{Численные решения уравнений~\eqref{f1} и~\eqref{f2} относительно 
параметра~$\sigma$ для некоторых значений~$n$ и~$\alpha$}\\
\multicolumn{7}{c}{\ }\\[-6pt]
\hline
\multicolumn{1}{|c|}{Размер}  & \multicolumn{2}{c|}{Уровень $\alpha=0{,}1$}& 
\multicolumn{2}{c|}{Уровень $\alpha=0{,}05$}& 
\multicolumn{2}{c|}{Уровень $\alpha=0{,}01$}\\
\cline{2-7}
\multicolumn{1}{|c|}{выборки $n$}&$\sigma_1$&$\sigma_2$&$\sigma_1$&$\sigma_2$&$\sigma_1$&$\sigma_2$\\
\hline
$\hphantom{000}100$&$0{,}4302$&$0{,}435$&$0{,}419$&$0{,}425$&$0{,}4002$&$0{,}408$\\
%\hline
$\hphantom{000}200$&$0{,}452$&$0{,}455$ &$0{,}441$&$0{,}445$&$0{,}424$&$0{,}429$\\
%\hline
$\hphantom{00}1000$&$0{,}499$&$0{,}499$ &$0{,}489$&$0{,}489$&$0{,}473$&$0{,}475$\\
%\hline
$\hphantom{0}10000$&$0{,}558$&$0{,}556$ &$0{,}549$&$0{,}547$&$0{,}536$&$0{,}534$\\
%\hline
$100000$&$0{,}611$&$0{,}607$ &$0{,}603$&$0{,}599$&$0{,}591$&$0{,}588$\\
\hline
\end{tabular}
\end{center}
\end{table*}


\noindent
Откуда следует справедливость соотношения~\eqref{Th2Eq} c~уче\-том 
очевидного неравенства

\pagebreak

\noindent
\begin{equation*}
\left\lvert \hat{{\sf E}}_X-{\sf E}_X\right\rvert \leqslant 
\left\lvert \hat{{\sf E}}_X-{\sf E}_Y\right\rvert +\left\lvert {\sf E}_Y-{\sf E}_X\right\rvert 
\end{equation*}
и оценки~\eqref{Th1Eq} из Теоремы~1.~\hfill$\square$

\smallskip

\noindent
\textbf{Замечание~2.}
В~работе~\cite{Gorshenin2016} было продемонстрировано повышение точ\-ности 
работы метода скользящего разделения конечных нормальных смесей за счет 
введения дополнительной компоненты, имеющей нормальное 
распределение $\mathcal{N}(0,\sigma^2)$ с~математическим ожиданием, равным~$0$, 
и~стандартным отклонением~$\sigma$. При этом была отмечена сложность выбора 
параметра~$\sigma$ для сохранения структуры выборки, близкой к~исходной. 
Результат Теоремы~2 может быть использован с~данной целью, если положить $k\hm=1$, 
$a_j\hm=0$ для всех $j\hm=1,2,\ldots$ и~выбирать величину~$\sigma$ как 
минимизирующую длину доверительного интервала~\eqref{Th2Eq}. Для 
этого необходимо найти производную функции $f(0,\sigma,\alpha,n)$~\eqref{Th2f} 
и~численно решить уравнение
\begin{multline}
f_\sigma'(0,\sigma,\alpha,n)\equiv \fr{z_{1-{\alpha}/2}}{\sqrt{n}} - {}\\
{}-
e^{-2\pi^2\sigma^2}\left(4\pi\sigma+\fr{1}{2\pi^3\sigma^3}+
\fr{1}{\pi\sigma}\right)=0
\label{f1}
\end{multline}
относительно неизвестного параметра~$\sigma$ при выбранных значениях величин~$n$ 
и~$\alpha$. В~качестве альтернативы можно использовать вид доверительного интервала 
из статьи~\cite{Ushakov2017b}, полученный с~помощью неравенства $\mathbb{D} [Z]
\hm\leqslant 2\mathbb{D} Z\hm+{1}/{2}$, и~искать решение уравнения вида:
\begin{multline}
\hspace*{-2.90578pt}\fr{2\sigma z_{1-{\alpha}/2}}{\sqrt{n (2\sigma^2+{1}/{2})}} -
 e^{-2\pi^2\sigma^2}\left(4\pi\sigma+\fr{1}{2\pi^3\sigma^3}+
 \fr{1}{\pi\sigma}\right)={}\\
 {}=0\,.\label{f2}
\end{multline}

Примеры найденных значений~$\sigma$ для типичных размеров выборок в~методе 
скользящего разделения смесей (учитываются как возможная ширина окна, 
так и~общее количество наблюдений в~анализируемом ряде) приведены в~таблице 
(использован метод оптимизации \verb"Trust-Region Dogleg" пакета \verb"MATLAB" 
c~настройками по умолчанию), в~которой через~$\sigma_1$ обозначено приближенное  
решение уравнения~\eqref{f1}, a~через $\sigma_2$~--- уравнения~\eqref{f2}.


Проверка практической эффективности данного подхода в~качестве 
критерия выбора параметров зашумляющего распределения для повышения 
точности работы метода скользящего разделения смесей может быть отмечена 
как задача для дальнейших исследований.


\section{Конечные смеси гамма-распределений}

Для случайной величины~$X$, имеющей распределение типа конечной смеси 
гам\-ма-рас\-пре\-де\-ле\-ний с~параметрами ${\bf r}\hm=(r_1,\ldots, r_k)$,
 $r_j\hm>0$, $\boldsymbol{\lambda}\hm=(\lambda_1,\ldots, \lambda_k)$, $\lambda_j\hm>0$, 
 ${\bf p}\hm=(p_1,\ldots, p_k)$, $p_j\hm\geqslant 0$, $\sum\nolimits_{j=1}^{k}p_j\hm=1$, 
 плот\-ность которого задается выражением
\begin{equation}
f_X(x)=\sum\limits_{j=1}^{k}p_j\fr{\lambda_j^{r_j} e^{-\lambda_j x}}
{\Gamma(r_j)}\,x^{r_j-1}\,,
\label{FinGammaMixt}
\end{equation}
характеристическая функция имеет следующий вид:
%характеристическая функция задается следующим выражением:
\begin{equation}
\varphi_X(t)=\!\int\limits_{-\infty}^{+\infty}\!\!\!e^{itx} f_X(x)\, dx = \!
\sum\limits_{j=1}^{k}p_j \left(\!1-\fr{it}{\lambda_j}\right)^{-r_j}\!.\!
\label{ChiFinGammaMixt}
\end{equation}

Отметим, что подобные модели зашумления разумно использовать в~случае, 
если известно, что данные сосредоточены на положительной полуоси, например 
при анализе различных информационных потоков (см., в~част\-ности, 
 работу~\cite{Gorshenin2013}). 

Проверим абсолютную интегрируемость функции $\varphi_X(t)$~\eqref{ChiFinGammaMixt}. 
Имеем:
\begin{multline*}
\int\limits_{-\infty}^{+\infty}\left\lvert\varphi_X(t)\right\rvert\, dt\leqslant 
\sum\limits_{j=1}^{k}p_j \int\limits_{-\infty}^{+\infty}\left\lvert \left(
1-\fr{it}{\lambda_j}\right)^{-r_j}\right\rvert \, dt={}\\
{}=\sum\limits_{j=1}^{k}p_j \int\limits_{-\infty}^{+\infty} \left\lvert\left(
\fr{\lambda_j(\lambda_j+it)}{\lambda_j^2+t^2}\right)^{r_j}\right\rvert\, dt \leqslant{}\\
{}\leqslant\sum\limits_{j=1}^{k}p_j \lambda_j \int\limits_{-\infty}^{+\infty}\left(
1+y^2\right)^{-{r_j}/{2}}\, dy\,.
\end{multline*}

Подынтегральное выражение при $r_j\hm\geqslant 2$ может быть оценено сверху 
функцией $1/({1+y^2})$, при этом соответствующий интеграл равен~$\pi$, что влечет 
абсолютную интегрируемость характеристической функции для конечной смеси 
гам\-ма-рас\-пре\-де\-ле\-ний. Поэтому в~дальнейшем будем предполагать,
 что $r_j\hm\geqslant 2$ для всех возможных значений $j\hm=1,2,\ldots$

Рассмотрим вопрос точ\-ности оценивания неизвестного математического ожидания ${\sf E}_X\hm>0$ 
при добавлении зашумления.

\smallskip

\noindent
\textbf{Теорема~3.}
\textit{Пусть выполнены предположения}~(A)--(D), 
\textit{причем случайные величины~$\varepsilon_j$, $j\hm=1,2,\ldots$, имеют 
распределение типа конечной $k$-ком\-по\-нент\-ной смеси 
гам\-ма-рас\-пре\-де\-ле\-ний вида}~\eqref{FinGammaMixt} 
\textit{с~па\-ра\-мет\-ра\-ми~${\bf r}$, $\boldsymbol{\lambda}$ и~${\bf p}$. Тогда}
\begin{equation}
\label{Th3Eq}
\left\lvert {\sf E}_Y-{\sf E}_X\right\rvert \leqslant \fr{R}{\lambda}+
\fr{\Lambda^{R}}{2^{r}\pi^{r+1}}\left(1+\frac1{r}\right)\,,
\end{equation}
\textit{где} $r=\min(r_1, \ldots,r_k)$; $R\hm=\max(r_1, \ldots,r_k)$; 
$\lambda\hm=\max(\lambda_1, \ldots,\lambda_k)$; 
$\Lambda\hm=\max(\lambda_1, \ldots,\lambda_k)$.

\smallskip

\noindent
Д\,о\,к\,а\,з\,а\,т\,е\,л\,ь\,с\,т\,в\,о\,.\ \
С~учетом пред\-став\-ле\-ний~\eqref{Law} и~\eqref{Fract}, ограниченности 
модуля характеристической функции, перехода от тригонометрической к~показательной 
записи комплексных чисел, а~также независимости случайных величин~$X_j$ 
и~$\varepsilon_j$ \mbox{имеем}:
\begin{multline*}
\left\lvert {\sf E}_Y-{\sf E}_X\right\rvert
\leqslant \left\lvert {\sf E}_\varepsilon\right\rvert+ {}\\
{}+\left\lvert\sum\limits_{n=1}^\infty
\left(
(-1)^n\mathrm{Im} \left(\sum\limits_{j=1}^{k}p_j \varphi_{X_j}(2\pi n)\left(
\vphantom{\fr{2\pi n}{\lambda_j}}
1-{}\right.\right.\right.\right.\\
\left.\left.\left.\left.{}-i\left(\fr{2\pi n}{\lambda_j}\right)\right)^{-r_j}\right)
\Bigg/ ({\pi n})
\vphantom{\sum\limits_{j=1}^{k}}
\right)\right\rvert={}\\
{}=\left\lvert {\sf E}_\varepsilon\right\rvert+ 
\left\lvert\sum\limits_{n=1}^\infty
\left(\!(-1)^n\mathrm{Im} \!\left(\sum\limits_{j=1}^{k}p_j \left(\!
1+\fr{4\pi^2 n^2}{\lambda_j^2}\right)^{- {r_j}/2}\!\times{}\right.\right.\right.\hspace*{-2.8663pt}\\
\left.\left.\left.{}\times \varphi_{X_j}(2\pi n)\,
e^{-ir_j\mathrm{arctan}\,({{t}/{\lambda_j}})}\right)
\Bigg/
({\pi n})
\vphantom{\left(
1+\fr{4\pi^2 n^2}{\lambda_j^2}\right)^{- {r_j}/2}}
\right)\right\rvert\leqslant{}\\
{}\leqslant \left\lvert {\sf E}_\varepsilon\right\rvert+\sum\limits_{j=1}^{k}
p_j\sum\limits_{n=1}^\infty\fr{1}{\pi n}\left(
1+\fr{4\pi^2 n^2}{\lambda_j^2}\right)^{-{r_j}/2}\leqslant{}\\
{}\leqslant  \fr{R}\lambda + \sum\limits_{j=1}^{k}p_j
\sum\limits_{n=1}^\infty\left(\fr{1}{\pi n}\,
\fr{\lambda_j^{r_j}}{(2\pi)^{r_j} n^{r_j}}\right)\leqslant {}
\\
{}\leqslant  \fr{R}{\lambda} + \sum\limits_{j=1}^{k}p_j 
\fr{\lambda_j^{r_j}}{2^{r_j}\pi^{r_j+1}}\left(1+\int\limits_{1}^{\infty}
\fr{1}{ x^{r_j+1}}\,dx\right)
\leqslant{}\\
{}\leqslant \fr{R}{\lambda}+\fr{\Lambda^{R}}{2^{r}\pi^{r+1}}\left(1+\fr{1}{r}\right).
\end{multline*}

При переходе от суммы к~интегралу используется факт убывания функции как переменной~$n$ 
(или~$x$).~\hfill$\square$


\smallskip

\noindent
\textbf{Замечание~3.}\
Теорема~3 описывает соответ\-ст\-ву\-ющий результат для гам\-ма-рас\-пре\-де\-лен\-ных 
за\-шум\-ля\-ющих случайных величин, если положить $k\hm=1$ в~выражении~\eqref{Th3Eq}. 
При этом, очевидно, $r\hm\equiv R$ и~$\lambda\hm\equiv \Lambda$.


\smallskip

Рассмотрим вопросы построения доверительного интервала для неизвестного 
математического ожидания ${\sf E}_X\hm>0$ в~предположении, что случайные величины~$X_j$ 
не содержат ошибок измерения, а все погрешности учтены исключительно в~за\-шум\-ля\-ющих 
элементах~$\varepsilon_j$.

\smallskip

\noindent
\textbf{Теорема~4.}
\textit{Пусть выполнены предположения}~(A)--(D), 
\textit{причем случайные величины~$\varepsilon_j$, $j\hm=1,2,\ldots$, имеют 
распределение типа конечной $k$-ком\-по\-нент\-ной смеси 
гам\-ма-рас\-пре\-де\-ле\-ний вида}~\eqref{FinGammaMixt} 
\textit{с~па\-ра\-мет\-ра\-ми~${\bf r}$, $\boldsymbol{\lambda}$ и~${\bf p}$, 
а~случайные величины} $X_j\stackrel{\text{п.н.}}{=}{\sf E}_X$, $j=1,2,\ldots$ 
\textit{Тогда доверительный интервал для~${\sf E}_X$ при условии $0\hm<\alpha\hm<1$ имеет вид}:
\begin{equation} 
\label{Th4Eq}
\left\lvert {\sf E}_X - \hat{{\sf E}}_X\right\rvert \leqslant  
f({\bf r},\boldsymbol{\lambda},\alpha,n),
\end{equation}
\textit{где}

\vspace*{-9pt}

\noindent
\begin{align}
\hat{{\sf E}}_X&=\fr{1}{n} \sum\limits_{j=1}^{n} Y_j\,; \label{Th4hatE}\\[-4pt]
f({\bf r}, \boldsymbol{\lambda},\alpha,n)&=\fr{z_{1-{\alpha}/2}}{\sqrt{n}} \left(
\sqrt{\fr{R(R+1)}{\lambda^2}-\fr{r^2}{\Lambda^2}}+\fr{1}{2}\right) +{}\notag\\[-1pt]
&\hspace*{20mm}{}+
\fr{R}{\lambda}+\fr{\Lambda^{R}}{2^{r}\pi^{r+1}}\left(1+\fr{1}{r}\right); \notag
\end{align}
\textit{$z_{1-{\alpha}/2}$~--- $\left(1-{\alpha}/2\right)$-кван\-тиль 
стандартного нормального распределения; $r\hm=\min(r_1, \ldots,r_k)$; 
$R\hm=\max(r_1, \ldots,r_k)$; $\lambda\hm=\max(\lambda_1, \ldots,\lambda_k)$; 
$\Lambda\hm=\max(\lambda_1, \ldots,\lambda_k)$}. 

\smallskip

\noindent
Д\,о\,к\,а\,з\,а\,т\,е\,л\,ь\,с\,т\,в\,о\,.\ \
Из центральной предельной теоремы с~учетом условия~(A) 
следует, что величина~$\hat{{\sf E}}_X$~\eqref{Th4hatE} асимптотически нормальна 
с~математическим ожиданием~${\sf E}_Y$~\eqref{EY} и~дисперсией $(1/n){\sf D}_Y$~\eqref{DY}. 
Пользуясь определением и~свойствами гам\-ма-функ\-ции, а~также оценкой~\eqref{Var} 
получим:

\noindent
\begin{multline*}
{\sf D}_Y \leqslant \left(\sqrt{\sum\limits_{j=1}^k p_j
\fr{\lambda_j^{r_j}}{\Gamma(r_j)} \int\limits_{0}^{+\infty} 
e^{\lambda_j x}x^{r_j+1}\, dx}+\fr{1}{2}\right)^2= {}\\[-0.5pt]
= \left(\sqrt{\sum\limits_{j=1}^{k}p_j
\fr{r_j(r_j+1)}{\lambda_j^2}-\left(\sum\limits_{j=1}^{k}p_j
\fr{r_j}{\lambda_j}\right) ^2}+\fr{1}{2}\right)^2\leqslant {}\\[-1.5pt]
{}\leqslant \left(\sqrt{\fr{R(R+1)}{\lambda^2}-\fr{r^2}{\Lambda^2}}+\fr{1}{2}\right)^2\,.
\end{multline*}

Аналогично доказательству Тео\-ре\-мы~2 с~учетом оценки~\eqref{Th3Eq} 
отсюда следует справедливость соотношения~\eqref{Th4Eq}.~\hfill$\square$

\vspace*{-12pt}

\section{Заключение}

Итак, в~работе получены оценки для математического ожидания наблюдений в~предположении 
зашумления конечными смесями нормальных\linebreak (Тео\-ре\-ма~1) 
и~гам\-ма-рас\-пре\-де\-ле\-ний (Тео\-ре\-ма~3). 
%
Построены доверительные интервалы 
для неизвестного математического ожидания в~этих случаях с~использованием 
уточненной оценки~\eqref{Var} 
(Тео\-ре\-мы~2 и~4 соответственно). Отметим, что соответствующие соотношения 
зависят только от <<экстремальных>> значений параметров смесей, но не от числа 
компонент и~весов в~распределении зашумляющих наблюдений. 
%
Замечание~2 
предлагает подход, который  может быть использован для определения неизвестного 
параметра искусственно добавляемого к~исходным данным шума для улучшения качества 
работы метода скользящего разделения смесей.

\smallskip
Автор выражает признательность доктору фи\-зи\-ко-ма\-те\-ма\-ти\-че\-ских наук, 
профессору Виктору Юрьевичу Королеву за идею использования оценки 
дисперсии вида~\eqref{Var} и~другие полезные обсуждения в~рамках 
работы над данной статьей.

\vspace*{-12pt}

{\small\frenchspacing
 {%\baselineskip=10.8pt
 \addcontentsline{toc}{section}{References}
 \begin{thebibliography}{99}
\bibitem{Wright2003} \Au{Wright~D.\,E., Bray~I.} 
A~mixture model for rounded data~// J.~Roy. Stat. Soc.~D 
Sta., 2003. Vol.~52. P.~3--13.

\columnbreak

\bibitem{Bai2009} \Au{Bai~Z., Zheng~S., Zhang~B., Hu~G.} 
Statistical analysis for rounded data~// J.~Stat. Plan.  Infer., 2009. 
Vol.~139. Iss.~8. P.~2526--2542.

\bibitem{Zhang2010} \Au{Zhang~B., Liu~T., Bai~Z.\,D.} 
Analysis of rounded data from dependent sequences~// 
Ann. I.~Stat. Math., 2010. Vol.~62. Iss.~6. P.~1143--1173.

\bibitem{Zhao2012} \Au{Zhao~N., Bai~Z.} 
Analysis of rounded data in mixture normal model~// Stat. Pap., 2012. 
Vol.~53. P.~895--914.

\bibitem{Korolev2011-i} \Au{Королев~В.\,Ю.} 
Ве\-ро\-ят\-но\-ст\-но-ста\-ти\-сти\-че\-ские методы декомпозиции волатильности 
хаотических процессов.~--- М.: Изд-во Моск. ун-та, 2011. 512~с.

\bibitem{Ushakov2015} \Au{Ушаков В.\,Г., Ушаков Н.\,Г.} 
Об усреднении округленных данных~// Информатика и~её применения, 2015. Т.~9. 
Вып.~4. С.~106--109.

\bibitem{Ushakov2017a} \Au{Ушаков~В.\,Г., Ушаков~Н.\,Г.} 
Границы точ\-ности восстановления информации, 
теряемой при округлении результатов наблюдений~// 
Вестник Московского университета. Серия~15: Вычислительная математика и~кибернетика, 
2017. №\,2. С.~26--30.

\bibitem{Ushakov2017b} \Au{Ushakov~N.\,G., Ushakov~V.\,G.} 
Statistical analysis of rounded data: Recovering of information lost due to rounding~// 
J.~Korean Stat. Soc., 2017.  Vol.~46. No.\,3. P.~426--437.

\bibitem{Gorshenin2016} \Au{Gorshenin~A.\,K., Korolev~V.\,Yu.} 
A~noising method for the identification of the stochastic structure of 
information flows~// Comm. Com. Inf. Sc., 2017. 
Vol.~678. P.~279--289.

\bibitem{Gorshenin2013} 
\Au{Gorshenin~A., Korolev~V.} Modelling of statistical
fluctuations of information flows by mixtures of gamma distributions~// 
27th European Conference on Modelling and Simulation Proceedings.~--- 
Dudweiler, Germany: Digitaldruck Pirrot GmbHP, 2013. P.~569--572.
 \end{thebibliography}

 }
 }

\end{multicols}

\vspace*{-6pt}

\hfill{\small\textit{Поступила в~редакцию 03.08.18}}

\vspace*{6pt}

%\newpage

%\vspace*{-24pt}

\hrule

\vspace*{2pt}

\hrule

\vspace*{-2pt}


\def\tit{DATA NOISING BY FINITE NORMAL AND~GAMMA MIXTURES WITH~APPLICATION 
TO~THE~PROBLEM OF~ROUNDED OBSERVATIONS}


\def\titkol{Data noising by finite normal and~gamma mixtures with~application 
to~the~problem of~rounded observations}



\def\aut{A.\,K.~Gorshenin}

\def\autkol{A.\,K.~Gorshenin}

\titel{\tit}{\aut}{\autkol}{\titkol}

\vspace*{-11pt}


\noindent
Institute of Informatics Problems, Federal Research Center ``Computer Science and
Control'' of the Russian Academy of Sciences, 44-2~Vavilov Str., Moscow 119333,
Russian Federation


\def\leftfootline{\small{\textbf{\thepage}
\hfill INFORMATIKA I EE PRIMENENIYA~--- INFORMATICS AND
APPLICATIONS\ \ \ 2018\ \ \ volume~12\ \ \ issue\ 3}
}%
 \def\rightfootline{\small{INFORMATIKA I EE PRIMENENIYA~---
INFORMATICS AND APPLICATIONS\ \ \ 2018\ \ \ volume~12\ \ \ issue\ 3
\hfill \textbf{\thepage}}}

\vspace*{3pt}



\Abste{In many real problems, statistical analysis of data containing additional 
measurement errors, including rounding, is performed, which in some situations can 
lead to sufficiently significant distortions. In this paper, estimates for an 
unknown expectation of observations are obtained for one of the possible 
rounding models under the assumption that the original data are additionally 
noised with random variables having distributions of the type of finite 
mixtures of normal and gamma laws. Confidence intervals for an 
unknown expectation are constructed using the refined estimate for 
the variance of the integer part of the random variable. An algorithm 
for determining the value of the parameter of artificial noise, which 
can be added to the initial data to improve the quality of the 
method of moving separation of mixtures, is discussed.}


\KWE{noisy data; rounded data; finite normal mixtures; finite gamma mixtures; 
confidence intervals; moving separation of mixtures}



\DOI{10.14357/19922264180304}

%\vspace*{-14pt}

\Ack
\noindent
The research was supported by the Russian Science Foundation (project 18-71-00156).



%\vspace*{6pt}

  \begin{multicols}{2}

\renewcommand{\bibname}{\protect\rmfamily References}
%\renewcommand{\bibname}{\large\protect\rm References}

{\small\frenchspacing
 {%\baselineskip=10.8pt
 \addcontentsline{toc}{section}{References}
 \begin{thebibliography}{99}
\bibitem{1-gor-1}
\Aue{Wright,~D.\,E., and I.~Bray.} 2003.
A~mixture model for rounded data.  \textit{J.~Roy. Stat. Soc.~D Sta.} 52:3--13.

\bibitem{2-gor-1}
\Aue{Bai,~Z., S.~Zheng, B.~Zhang, and G.~Hu.} 2009. 
Statistical analysis for rounded data. \textit{J.~Stat. Plan. 
Infer.} 139(8):2526--2542.

\bibitem{3-gor-1}
\Aue{Zhang,~B., T.~Liu, and Z.\,D.~Bai.} 2010. 
Analysis of rounded data from dependent sequences. 
\textit{Ann. I.~Stat. Math.} 62(6):1143--1173.

\bibitem{4-gor-1}
\Aue{Zhao,~N., and Z.~Bai.} 2012. Analysis of rounded data in mixture normal model. 
\textit{Stat. Pap.} 53:895--914.

\bibitem{5-gor-1}
\Aue{Korolev, V.\,Yu.} 2011. 
\textit{Veroyatnostno-statisticheskie metody dekompozitsii volatil'nosti 
khaoticheskikh protsessov} [Probabilistic and statistical methods of 
decomposition of volatility of chaotic processes]. 
Moscow: Moscow University Publishing House. 512~p.

\bibitem{6-gor-1}
\Aue{Ushakov, V.\,G., and N.\,G.~Ushakov.} 
2015. Ob usrednenii okruglennykh dannykh [On averaging of rounded data].
\textit{Informatika i~ee Primeneniya~--- Inform. Appl.} 9(4):106--109.

\bibitem{7-gor-1}
\Aue{Ushakov,~V.\,G., and N.\,G.~Ushakov.} 2017. 
Boundaries of the precision of restoring information lost after rounding
 the results from observations. 
 \textit{Moscow University Computational Math. Cybernetics} 41(2):76--80.

\bibitem{8-gor-1}
\Aue{Ushakov,~N.\,G., and  V.\,G.~Ushakov.} 2017. 
Statistical analysis of rounded data: Recovering of information lost due to rounding. 
\textit{J.~Korean Stat. Soc.} 46(3):426--437.

\bibitem{9-gor-1}
\Aue{Gorshenin,~A.\,K., and V.\,Yu.~Korolev.} 2016. 
A~noising method for the identification of the stochastic structure of information 
flows. \textit{Comm. Com. Inf. Sc.} 678:279--289.

\bibitem{10-gor-1}
\Aue{Gorshenin,~A., and V.~Korolev.} 2013.  Modelling of statistical fluctuations of
information flows by mixtures of gamma distributions. 
\textit{27th European Conference on Modelling and Simulation Proceedings}. 
Dudweiler, Germany: Digitaldruck Pirrot GmbHP. 569--572.

\end{thebibliography}

 }
 }

\end{multicols}

\vspace*{-6pt}

\hfill{\small\textit{Received August 3, 2018}}

%\pagebreak

%\vspace*{-18pt}

\Contrl

\noindent
\textbf{Gorshenin Andrey K.} (b.\ 1986)~--- Candidate of Science (PhD) in physics and
mathematics, associate professor, leading scientist, Institute of Informatics Problems,
Federal Research Center ``Computer Science and Control'' of the Russian Academy of
Sciences, 44-2 Vavilov Str., Moscow 119333, Russian Federation; 
\mbox{agorshenin@frccsc.ru}
\label{end\stat}

\renewcommand{\bibname}{\protect\rm Литература}        %5
\def\stat{malashenko}

\def\tit{ПОСЛЕДОВАТЕЛЬНЫЙ АНАЛИЗ И~МЕТРИЧЕСКИЕ ОЦЕНКИ ПРЕДЕЛЬНЫХ
РАСПРЕДЕЛЕНИЙ МЕЖУЗЛОВЫХ ПОТОКОВ В~МНОГОПОЛЬЗОВАТЕЛЬСКОЙ СЕТИ}

\def\titkol{Последовательный анализ и~метрические оценки предельных
распределений межузловых потоков в %~многопользовательской 
сети}

\def\aut{Ю.\,Е. Малашенко$^1$}

\def\autkol{Ю.\,Е. Малашенко}

\titel{\tit}{\aut}{\autkol}{\titkol}

\index{Малашенко Ю.\,Е.}
\index{Malashenko Yu.\,E.}


%{\renewcommand{\thefootnote}{\fnsymbol{footnote}} \footnotetext[1]
%{Исследование выполнено при финансовой поддержке Российского научного фонда (проект 
%<<Информатика>> ФИЦ ИУ РАН, Москва).}}


\renewcommand{\thefootnote}{\arabic{footnote}}
\footnotetext[1]{Федеральный исследовательский центр <<Информатика и~управление>> Российской академии 
\mbox{mala-yur@yandex.ru}}


%\vspace*{-6pt}



\Abst{Для оценки функциональных возможностей
многопользовательской сети связи аналилизируется множество векторов межузловых потоков при предельных распределениях ресурсов
сети. В~рамках многопродуктовой модели про\-пуск\-ные спо\-соб\-ности ребер рас\-смат\-ри\-ва\-ют\-ся 
как компоненты вектора ресурсов различных
типов, которые требуются для передачи потоков различных видов.
При проведении вычислительных экспериментов на каждой итерации вычисляются нормы векторов совместно допустимых межузловых
потоков, при передаче которых полностью используется пропускная спо\-соб\-ность всех ребер сети. Полученные последовательности
метрических оценок позволяют анализировать особенности множества до\-сти\-жи\-мости и~эф\-фек\-тив\-ность использования ресурсов сети при
уравнительном распределении про\-пуск\-ной спо\-соб\-ности между корреспондентами.}

\KW{многопродуктовая потоковая сетевая
модель; множество достижимых межузловых потоков; предельные
распределения пропускной способности}

\DOI{10.14357/19922264220306} 
  
%\vspace*{-3pt}


\vskip 10pt plus 9pt minus 6pt

\thispagestyle{headings}

\begin{multicols}{2}

\label{st\stat}

\section{Введение}

Данная работа продолжает исследования функциональных характеристик
сетевых сис\-тем связи~[1]. В~настоящее время математические модели
передачи многопродуктового потока применяются для поиска
распределений потоков и~ресурсов в~многопользовательских
телекоммуникационных\linebreak сетях~[2--4]. Разрабатываются методы анализа
с~учетом вектора требований всех \textit{равноправных} 
и~невзаимозаменяемых корреспондентов~[5]. С~позиций\linebreak методологии
исследования операций изучаются справедливые распределения потоков
и~ресурсов~[6].

Соответствующие \textit{недискриминирующие} правила управления
потоками являются решениями задач на максмин и/или получаются 
в~результате использования процедур последовательной
лексикографически упорядоченной оптимизации~[7].

В~настоящей работе пути соединения корреспондентов прокладываются
через со\-от\-вет\-ст\-ву\-ющие минимальные разрезы. Указанный метод\linebreak \mbox{можно}
рассматривать как возможный вариант решения задачи о~построении
SPLIT-марш\-ру\-тов~[8,~9]. В~рамках вычислительных экспериментов\linebreak на
многопродуктовой модели анализируются распределения межузловых
потоков  и~пропускной способ\-ности сети.  Для оценки функциональных
возможностей многопользовательской сети используется вектор
совместно допустимых межузловых потоков. Под ресурсом, выделяемым
некоторой паре узлов-кор\-рес\-пон\-ден\-тов,  понимается суммарное
значение тре\-бу\-емых пропускных способностей на всех ребрах,
расположенных на всех маршрутах при прохождении межузлового\linebreak потока
данного вида.  Сумма соответствующих реберных потоков трактуется
как полная нагрузка на сеть, возникающая при передаче заданного
межузлового потока. Рас\-смат\-ри\-ва\-ют\-ся распределения пропускной
способности и~межузловых потоков при предельной загрузке сети.
При проведении вычислительных экспериментов на каждой  итерации
вычисляется норма  вектора совместно допустимых межузловых
потоков.   Для оценки величины требуемых ресурсов при соединении
корреспондентов по различным путям для каж\-дой пары узлов
определяется максимальный однопродуктовый поток. Марш\-ру\-ты передачи
всех допустимых межузловых потоков  проходят по ребрам
соответствующих минимальных разрезов. Вычислительные эксперименты
проводились  для получения последовательности  мет\-ри\-че\-ских оценок
векторов межузловых потоков, принадлежащих множеству до\-сти\-жи\-мости
многопользовательской сети.

\section{Математическая модель}

В качестве математической модели для описания
многопользовательской сетевой системы используется следующая
формальная запись условий и~ограничений, которые должны
выполняться при одновременной передаче потоков различных видов
между всеми парами улов-корреспондентов:

Сеть $G(\mathbf{d})$ задается множествами $\langle V,
R,U,P\rangle$:
\begin{itemize}
\item  узлов (вершин) сети 
$$
V=\left \{v_{1}, v_{2},\dots,v_{n},\dots,v_{N}\right\};
$$
\item  неориентированных ребер 
$$
R=\left\{r_{1}, r_{2}, \dots, r_{k}, \dots,
r_{E}\right\}.
$$
\end{itemize}

Ребро $r_{k}$ \textit{соединяет} концевые вершины~$v_{n_k}$ и~$v_{j_k}$. 
Реб\-ру~$r_{k}$ ставятся в~соответствие две
ориентированные дуги $\{u_{k},u_{k+E}\}$ из множества
ориентированных дуг $U\hm=\{u_{1}, u_{2}, \dots, u_{k}, \dots,
u_{2E}\}$. Дуги $\{u_{k}, u_{k+E}\}$ определяют прямое и~обратное
на\-прав\-ле\-ние передачи потока по реб\-ру~$r_{k}$ между концевыми
вершинами $\{v_{n_k}, v_{j_k}\}$.

В многопользовательской сети~$G(\mathbf{d})$ рассматривается
$M\hm=N(N\hm-1)$ независимых, невзаимозаменяемых и~равноправных потоков
различных видов, которые передаются между уз\-ла\-ми-кор\-рес\-пон\-ден\-та\-ми
из множества 
$$
P=\left\{p_{1}, p_{2}, \dots, p_{M}\right\}.
$$

По определению, каждой паре уз\-лов-кор\-рес\-пон\-ден\-тов~$p_{m}$
соответствуют:
\begin{itemize}
\item вершина-ис\-точ\-ник с~номером~$s_{m}$, через которую входной поток
$m$-го вида~$z_{m}$ поступает в~сеть;
\item  вершина-при\-ем\-ник с~номером~$t_{m}$, из которой поток $m$-го
вида~$z_{m}$ покидает сеть.
\end{itemize}

В множестве~$P$ выделяется подмножество $P(R^{+})$ пар
уз\-лов-кор\-рес\-пон\-ден\-тов, расположенных в~концевых вершинах
ребра~$r_{k}$, $k\hm=\overline{1,E}$. Вводятся сле\-ду\-ющие обозначения:
пусть ребро~$r_{k}$  с~номером~$k$ соединяет вершины с~номерами~$n$ и~$j$ такими, что $n\hm< j$. Для со\-от\-вет\-ст\-ву\-ющей пары
уз\-лов-кор\-рес\-пон\-ден\-тов~$p_{k}$, расположенных в~узлах $\{v_{n},
v_{j}\}$, узел~$v_{n}$ считается источником, а узел~$v_{j}$~---
приемником потока $z_{k}$ $k$-го вида, который передается из узла
c номером~$n$ в~узел с~номером~$j$ для пары~$p_{k}$ с~номером~$k$.
Для пары $p^{}_{k+E} \Longleftrightarrow \{v_{j},v_{n}\}$ узел~$v_{j}$ 
считается источником~$s_{k+E}$, а~узел $v_m$~--- приемником~$t_{k+E}$ для пары с~номером~$p_{k+E}$. Формируется
$R^+\hm=\{1,2,3,\dots,E,E+1,\dots,2E\}$~--- список номеров смежных
пар.

Пары $p_{k}$ из подмножества~$P(R^{+})$ называются
\textit{смежными} уз\-ла\-ми-кор\-рес\-пон\-ден\-та\-ми. Все остальные
\textit{несмежные} пары уз\-лов-кор\-рес\-пон\-ден\-тов относятся к~множеству~$P(R^{-})$:
\begin{equation*}
P=P(R^{+})\cup P(R^{-});\quad
P(R^{+}) \cap P(R^{-}) = \emptyset.
\end{equation*}

Введем обозначения:
\begin{description}
\item[\,]
$z_{m}$~--- величина \textit{межузлового} потока $m$-го вида,
который поступает в~сеть из узла с~номером~$s_{m }$ и~покидает из
узла с~номером~$t_{m}$;
\item[\,]
$S(v_{n})$~--- множество номеров исходящих дуг, по которым поток
покидает узел~$v_{n}$;
\item[\,]
$T(v_{n})$~--- множество номеров входящих дуг, по которым поток
поступает в~узел~$v_{n}$.
\end{description}

Во всех узлах $v_{n}\in V$, $n\hm=\overline{1,N}$, для всех видов
потоков должны выполняться условия сохранения потоков:
\begin{multline}
\label{eq1} 
\sum\limits_{i\in S(v_n )} x_{mi}-\sum\limits_{i\in T(v_n )} x_{mi}
={}\\
{}=\begin{cases}
z_m, & \mbox{если } v=v^{}_{S_m}; \\
-z_m,&\mbox{если } v=v_{t_m}; \\
0&\mbox{в остальных случаях}, \\
\end{cases}
\end{multline}
$n=\overline{1,N}$, $m\hm=\overline{1,M}$, $x_{mi}\hm\ge 0$,
$z_{m}\hm\ge0$.

Величина {z}$_{m}$ равна входному потоку $m$-го вида, который
пропускается от источника к~приемнику пары $p_{m}$ при
распределении потоков $x_{mi}$ по дугам сети.

Каждому ребру $r_{k}\hm\in R$ приписывается неотрицательное число~$d_{k}$, 
определяющее суммарный предельно допустимый поток,
который можно передать по реб\-ру~$r_{k}$ в~обоих на\-прав\-ле\-ни\-ях. 
В~исходной сети компоненты вектора про\-пуск\-ных способностей
$\mathbf{d}\hm=(d_{1}, d_{2},\dots, d_{k}, \dots, d_{E})$~--- наперед
заданные положительные числа $d_{k}
\hm> 0$. Вектором $\mathbf{d}$ определяются сле\-ду\-ющие ограничения на сумму
дуговых потоков всех видов, пе\-ре\-да\-ва\-емых по реб\-ру~$r_{k}$:
\begin{multline}
\sum\limits_{m=1}^M (x_{mk}+x_{m(k+E)}) \le d_k,\\
 x_{mk}\ge 0\,,\enskip
 x_{m(k+E)}\ge 0\,, \enskip k=\overline {1,E}\,.
 \label{eq2} 
\end{multline}
В рамках данной модели пропускная спо\-соб\-ность ребер сети~--- вектор~$\mathbf{d}$~--- трактуется как <<\textit{ресурсное ограничение}>>,
а~сумма дуговых
 потоков рас\-смат\-ри\-ва\-ет\-ся как показатель использования
<<\textit{ресурсов}>> сети при передаче межузловых потоков
различных видов.

Для всех $z_{m}$ и~$x_{mi}$, удовлетворяющих
условиям~\eqref{eq1} и~\eqref{eq2}, вычисляются суммарные потоки:
\begin{equation}
 y_{m }=\sum\limits_{i=1}^{2E} {x}_{mi},\quad
m=\overline{1,M}\,.
\label{eq3}
\end{equation}

Суммарный реберный поток~$y_{m}$ характеризует
<<\textit{нагрузку}>> на сеть при передаче межузлового потока
величины $z_{m}$ из уз\-ла-ис\-точ\-ни\-ка~$s_{m}$ в~узел-при\-ем\-ник~$t_{m}$. 
Величина~$y_{m}$ показывает, какой суммарный
\textit{ресурс}~-- пропускная спо\-соб\-ность сети~-- требуется для
передачи межузлового потока~$z_{m}$, а~отношение
$w_{m}\hm={y_m}/{z_m}$,  $m\hm=\overline{1,M},$
показывает, какие \textit{ресурсы} необходимы для передачи
единичного потока $m$-го вида между узлами~$s_{m}$ и~$t_{m}$.

Ограничения~\eqref{eq1}--\eqref{eq3} задают подмножество
допустимых значений компонент вектора межузловых потоков
$\mathbf{z}\hm=\left(z_{1}, z_{2},\dots,z_{m},\dots,z_{M}\right)$:
\begin{equation*}
{Z}(\mathbf{d})=\left\{\mathbf{z} \ge 0 \mid
(\mathbf{z},\mathbf{x},\mathbf{y}) \ \mbox{удовлетворяют~\eqref{eq1}--\eqref{eq3}}
\right\}\!,
\!\!
%\label{eq4} 
\end{equation*}
а все допустимые распределения ресурсов принадлежат подмножеству
\begin{equation*}
{Y}(\mathbf{d})=\left\{\mathbf{y} \ge 0 \mid
(\mathbf{z},\mathbf{x},\mathbf{y}) \ \mbox{удовлетворяют~\eqref{eq1}--\eqref{eq3}}\right\}\!.
%\!\!\!\label{eq5}
\end{equation*}


\section{Метрические оценки предельных распределений}

Для оценки функциональных возможностей сис\-те\-мы рассматриваются
допустимые распределения межузловых потоков при предельных
загрузках ребер сети.

В рамках данного модельного описания монопольным режимом
называется способ управления, при котором все ресурсы сети
используются для передачи потока одной выделенной пары
уз\-лов-кор\-рес\-пон\-ден\-тов $p_{a}\hm\in P(R^-)$, а для всех
остальных потоки полагаются равными нулю.

Предельно допустимый поток, который можно передать между
фиксированной парой уз\-лов-кор\-рес\-пон\-ден\-тов $p_{a}$ в~монопольном
режиме, является решением стандартной, в~данном случае
однопродуктовой, задачи о~максимальном потоке.

\smallskip

\noindent
\textbf{Задача 1.} Найти
$z_a^0\hm=\max\limits_{\langle z,x\rangle \in Z(d)} z_a
$
при условии $z_{i}=0$, $i\hm=\overline{1,M}$, $i\hm\ne a$.

При решении задачи~1 для пары $p_{a}$ вы\-чис\-ля\-ют\-ся: межузловой
поток~$z_a^0$; дуговые потоки $\{x^{0}_{ak};x^{0}_{a(k+E)}\}$,
$k\hm=\overline{1,E}$; суммарное значение реберного
потока~$y_{a}^{0}\hm=\sum\nolimits_{i=1}^{2E} {x}_{ai}^{0}$.

Поток величины $z_a^0$ является \textit{максимальным потоком},
пе\-ре\-да\-ва\-емым в~\textit{монопольном режиме} для пары
уз\-лов-кор\-рес\-пон\-ден\-тов~$p_{a}$, $p_{a}\hm\in P(R^-)$, в~сети~$G(d)$.

Задача~1 решается последовательно для всех $p_{m}\in P(R^-)$,
вы\-чис\-ля\-ют\-ся значения $z_{m}^{0}(t)$.

При проведении вычислительных экспериментов использовалась
итерационная процедура для оценки функциональных возможностей
сис\-те\-мы при передаче межузловых потоков по нескольким маршрутам.
На предварительном этапе шага~$t$ в~сети~$G(t)$ при заданных
значениях пропускной спо\-соб\-ности ребер~$d_k(t)$ для каждой \mbox{пары}
уз\-лов-кор\-рес\-пон\-ден\-тов $p_m\hm\in P(R^-)$ определяется максимально
допустимый однопродуктовый поток~$z^0_m(t)$, со\-от\-вет\-ст\-ву\-ющие
дуговые потоки $(x_{mk}^0(t),x_{m(k+E)}^0(t))$, $p_m\hm\in P(R^-)$, и~коэффициенты нормировки
$\xi_m^0(t)\hm={1}/{z_m^0(t)}$ для всех  $p_m\hm \in P(R^-)$,
таких что $z^0_m(t)\hm>0$, $y_m^0(t)\hm>0$.
Коэффициенты~$\xi_m^0(t)$ используются для поиска текущих
совместно допустимых квот на передачу потоков одновременно между
всеми парами $p_m\in P(R^-)$.

\smallskip

\noindent
\textbf{Задача 2.} Найти $\alpha^*(t)=\max\limits_\alpha \alpha$
при условиях
$$
\alpha\!\!\sum\limits_{m\in R^-}\! \xi_m^0\left(x_{mk}^0(t)+x_{m(k+E)}^0(t)\right)\le d_k(t),\enskip
k=\overline{1,E}\,.
$$

На основании $\alpha^*(t)$ вычисляются совместно допустимые
дуговые потоки:
\begin{multline*}
x_{mk}^*(t)=\alpha^*(t)\xi^0_m(t)x^0_{mk}(t),\\
x^*_{m(k+E)}(t)=\alpha^*(t)\xi^0_m(t)x^0_{m(k+E)}(t),
\\
m=\overline{1,M}\,,\enskip k=\overline{1,E}\,,
\end{multline*}
и остаточная пропускная способность ребер в~сети $G(t+1)$:
\begin{multline*}
d_k(t+1)=d_k(t)-\sum_{m\in R^-} (x_{mk}^*(t)+x_{m(k+E)}(t)),\\
k=\overline{1,E}\,,\enskip p_m\in P(R^-).
\end{multline*}
Формируется вектор допустимых межузловых потоков:
\begin{align*}
z_k^+(t)&=d_k(t+1),\enskip p_k\in P(R^+),\enskip k=\overline{1,E}\,;
\\
z_m^-(t)&=\sum\limits_{\tau=1}^t\alpha^*(\tau)\xi_m^0(\tau) z_m^0(\tau), \enskip p_m\in P(R^-).
\end{align*}

По построению, на шаге~$t$ при передаче вектора межузлового потока
$\mathbf{z}(t)=\{\mathbf{z}^+(t), \mathbf{z}^-(t)\}$ достигается
предельная загрузка, и~пропускная способность всех ребер  сети
используется полностью.

Точка с~координатами $\mathbf{z}(t)$ принадлежит множеству~$Z(d)$.

Расстояние точки от начала координат определяется как норма
соответствующего вектора:
\begin{align*}
\rho^+(t)&=\|\mathbf{z}^+(t)\|=
\left[\,\sum\limits_{k=1}(\mathbf{z}^+(t))^2\right]^{1/2};
\\
\rho^-(t)&=\|\mathbf{z}^-(t)\|= \left[\sum\limits_{p_m\in P(R^-)}(\mathbf{z}_m^-(t))^2\right]^{1/2}.
\end{align*}

Если при выполнении шага $(t+1)$ окажется, что $z_m^0(t+1)=0$ для
всех $p_m\in P(R^-)$, то про\-изойдет останов и~сформируются
массивы финальных данных:
\begin{align*}
z_m^-(T)&=\sum\limits_{\tau=1}^t \alpha^*(\tau)\xi_m^0(\tau) z_m^0(\tau),\enskip 
p_m\in P(R^-),\\
z_k^+(T)&=d_k(t+1),\enskip p_k\in P(R^+),\enskip k=\overline{1,E}\,.
\end{align*}

Вышеописанная вычислительная процедура далее обозначается как
MFPL-про\-це\-ду\-ра (от англ.\ \textit{max-flow-peak-load}).

При проведении второй серии вычислительных экспериментов
MFPL-про\-це\-ду\-ра использовалась для оценки функциональных
характеристик сис\-те\-мы при \textit{уравнительном} поэтапном
распределении пропускной способности между всеми
па\-ра\-ми-кор\-рес\-пон\-ден\-тами.

При реализации MFPL-процедуры выполнение каждого шага разбивается
на несколько этапов. На предварительном этапе шага~$t$ 
в~сети~$G(t)$ при заданных значениях пропускной способности ребер~$d_k(t)$ 
для каждой пары уз\-лов-кор\-рес\-пон\-ден\-тов $p_m\hm\in P(R^-)$
определяется максимально допустимый однопродуктовый
поток~$z_m^0(t)$, соответствующие дуговые потоки
$\left(x_{mk}^0(t),x_{m(k+E)}^0(t)\right)$, $p_m\hm\in P(R^-)$, и~суммарная
реберная нагрузка
$$
y_m^0(t)=\sum\limits_{k=1}^E (x_{mk}^0(t),x_{m(k+E)}^0(t)),\enskip p_m\in P(R^-).
$$

Для каждой пары $p_m\hm\in P(R^-)$ вычисляются коэффициенты
нормировки
$\theta_m^0(t)\hm={1}/{y_m^0(t)}$ для всех  
$p_m\hm\in P(R^-)$, таких что  $z^0_m(t)\hm>0$,
$y_m^0(t)\hm>0$.
Коэффициенты~$\theta_m^0(t)$ используются для поиска совместно
допустимых дуговых потоков для всех $p_m\hm\in P(R^-)$.

\smallskip

\noindent
\textbf{Задача 3.} Найти $\beta^*(t)=\max\nolimits_\beta \beta$ при
условиях
$$
\beta\!\!\!\!\sum\limits_{p_m\in P(R^-)}\!\!
\theta_m^0(x_{mk}^0(t)+x_{m(k+E)}^0(t))\le d_k(t),\enskip
k=\overline{1,E}\,.
$$

 С помощью $\beta^*(t)$ (решения задачи~3) вычисляются текущие допустимые значения дуговых потоков:
\begin{multline*}
x_{mk}^*(t)=\beta^*(t)\theta^0_m(t)x^0_{mk}(t),\\
x^*_{m(k+E)}(t)=\beta^*(t)\theta^0_m(t)x^0_{m(k+E)}(t), \enskip
k=\overline{1,E},
\end{multline*}
и реберных нагрузок при одновременной передаче межузловых потоков:

\noindent
\begin{multline*}
y_m^*(t)=\sum\limits_{i=1}^E
\left[x_{mi}^*(t)+x^*_{m(i+E)}(t)\right]={}\\
{}= \fr{\beta^*(t)}{y_m^0(t)} \sum\limits_{i=1}^E
\left[x_{mi}^0(t)+x^0_{m(i+E)}(t)\right]=\beta^*(t), \\
 p_m\in P(R^-).
\end{multline*}
Таким образом на каждом шаге определенная часть имеющегося ресурса
(пропускной спо\-соб\-ности) делится строго по\-ров\-ну меж\-ду всеми
корреспондентами $p_m\in P(R^-)$, для которых существует путь
передачи в~$G(t)$.

Формируется вектор допустимых межузловых потоков:
\begin{gather*}
\hspace*{-30mm}z_k^{++}(t)=d_k(t+1)={}\hspace*{10mm}\\
{}=d_k(t)-\!\!\! \sum\limits_{p_m\in P(R^-)}\!\!\!
\left(x_{mk}^*(t)+x_{m(k+E)}(t)\right),\\
\hspace*{35mm}k=\overline{1,E}, \enskip
p_k\in P(R^+);\\
z_m^{(=)}(t)\overset{\Delta}{=}\sum\limits_{\tau=1}^t\beta^*(\tau)
\theta_m^0(\tau) z_m^0(\tau), \enskip p_m\in P(R^-).
\end{gather*}

\noindent
Определяются расстояния:
\begin{align*}
\rho^{++}(t)&=\|\mathbf{z}^{++}(t)\|\overset{\Delta}{=}
\left[\sum\limits_{k=1}^E\left(d_k(t+1)\right)^2\right]^{1/2};\\
\rho^{(=)}(t)&=\|\mathbf{z}^{=}(t)\|= \left[\sum\limits_{p_m\in
P(R^-)}\left(z_m^{(=)}(t)\right)^2\right]^{1/2}.
\end{align*}

Если на предварительном этапе на шаге $(t+1)$ окажется, что в~сети~$G(t+1)$ для всех $p_m\hm\in P(R^-)$ все значения
$z_m^0(t+1)\hm=0$, то произойдет останов и~сформируются финальные
массивы:
\begin{align*}
z_k^{(++)}(T)&=d_k(t+1), \enskip
p_k\in P(R^+), \enskip k=\overline{1,E};
\\
z_m^{(=)}(t)&=\sum\limits_{\tau=1}^{t+1}\beta^*(\tau)
\theta_m^0(\tau) z_m^0(\tau), \enskip p_m\in P(R^-).
\end{align*}



\section{Вычислительный эксперимент}

Результаты вычислительных экспериментов, описанные ниже, служат
продолжением исследований, начатых в~[1]. Вычислительные
эксперименты проводились на моделях сетевых сис\-тем, пред\-став\-лен\-ных
на рис.~1 и~2. В~каждой сети~69~узлов. Пропускные спо\-соб\-но\-сти
ребер~-- значения $d_k$~-- выбирались случайным образом из отрезка
$[900,999]$ и~совпадали для ребер, при\-сут\-ст\-ву\-ющих в~обеих сетях.
В~кольцевой сети пропускная спо\-соб\-ность каждого из добавленных
ребер равнялась~900.

\begin{figure*} %fig1
\vspace*{1pt}
\begin{minipage}[t]{80mm}
  \begin{center}  
    \mbox{%
\epsfxsize=69.408mm
\epsfbox{mal-1.eps}
}

\end{center}
\vspace*{-6pt}
\Caption{Базовая сеть}
\end{minipage}
%\end{figure*}
\hfill
%\begin{figure*} %fig2
\vspace*{1pt}
\begin{minipage}[t]{80mm}
  \begin{center}  
    \mbox{%
\epsfxsize=69.408mm
\epsfbox{mal-2.eps}
}

\end{center}
\vspace*{-6pt}
\Caption{Кольцевая сеть}
\end{minipage}
\end{figure*}

\begin{table*}[b]\small %tabl1
\vspace*{-12pt}
\begin{center}

%\renewcommand{\arraystretch}{1.1}
\Caption{Базовая сеть}
\vspace*{2ex}

\begin{tabular}{|c||c|c|c||c|c|c|} 
\hline
&&&&&&\\[-9pt]
$t$  & $\rho^{-}(t)$ & $\rho^{+}(t)$ & $d^{+}(t+1)$ &
$\rho^{=}(t)$ & $\rho^{++}(t)$&  $d^{++}(t+1)$ \\ 
\hline
\hphantom{99}0  & \hphantom{99}0   & 8048&  68256&  \hphantom{9}0   &  8048&   68256\\
1  & \hphantom{9}63  & 4182&  26544&  \hphantom{9}95  &  3881&   24476\\
$\cdots$  & $\cdots$   & $\cdots$   &  $\cdots$    &  $\cdots$   &  $\cdots$   &   $\cdots$\\
11 & \hphantom{9}70  & 3975&  21469&  \hphantom{9}101\hphantom{9} &  3707&   20155\\
$\cdots$& $\cdots$   & $\cdots$   &  $\cdots$    & $\cdots$   &  $\cdots$   &  $\cdots$\\
22 & \hphantom{9}83  & 3861&  19623&  \hphantom{9}122\hphantom{9} &  3586&   18260\\
$\cdots$ & $\cdots$  & $\cdots$   &  $\cdots$   &  $\cdots$   &  $\cdots$  &   $\cdots$\\
33 & \hphantom{9}103\hphantom{9} & 3778&  18827&  \hphantom{9}139\hphantom{9} &  3522&   17601\\
$\cdots$ &$\cdots$  &$\cdots$  & $\cdots$  & $\cdots$   &  $\cdots$  &  $\cdots$\\
44 & \hphantom{9}\bf 190\hphantom{9} & \bf3553&  \bf17503&  \hphantom{9}\bf203\hphantom{9} &  \bf3285&   \bf16201\\
45 & \hphantom{9}\bf1452\hphantom{99}& \bf2166&  \hphantom{9}\bf7069 &  \hphantom{9}\bf1376\hphantom{99}&  \bf2020&   \hphantom{9}\bf6584\\
46 & \hphantom{9}\bf1498\hphantom{99}& \bf2158&  \hphantom{9}\bf6707 &  \hphantom{9}\bf1388\hphantom{99}&  \bf2017&   \hphantom{9}\bf6483\\
$\cdots$ & $\cdots$   & $\cdots$   &  $\cdots$    & $\cdots$   &  $\cdots$   &  $\cdots$\\
52 & \hphantom{9}1535\hphantom{99}& 2155&  \hphantom{9}6413 & \hphantom{9}1442\hphantom{99} &  2011&   \hphantom{9}6059\\
\hline
\end{tabular}
\end{center}
 %\end{table*}
% \begin{table*}\small %tabl2
\begin{center}
\Caption{Кольцевая сеть}
\vspace*{2ex}


\begin{tabular}{|c||c|c|c||c|c|c|} 
\hline
&&&&&&\\[-9pt]
$t$  & $\rho^{-}(t)$ & $\rho^{+}(t)$ & $d^{+}(t+1)$ &
$\rho^{=}(t)$ & $\rho^{++}(t)$&  $d^{++}(t+1)$ \\
 \hline
\hphantom{9}0  &\hphantom{99}0    & 8440  & 75456   &\hphantom{9}0      &8440   &75456\\
\hphantom{9}1  &\hphantom{9}68   & 5317  & 43038   &92     &5045   &40716 \\ 
$\cdots$ &$\cdots$    & $\cdots$     & $\cdots$   &$\cdots$      &$\cdots$      &$\cdots$      \\
11 &\hphantom{9}95   & 3608  & 20459   &124    &3397   &19080  \\
$\cdots$ &$\cdots$   & $\cdots$    & $\cdots$      &$\cdots$     &$\cdots$     &$\cdots$   \\
22 &\hphantom{9}101\hphantom{9}  & 3540  & 19530   &130    &3350   &18338 \\
$\cdots$ &$\cdots$  & $\cdots$   &$\cdots$      &$\cdots$     &$\cdots$   &$\cdots$    \\
33 &\hphantom{9}135\hphantom{9}  & 3346  & 17561   &154    &3220   &17003 \\
$\cdots$  &$\cdots$   & $\cdots$    & $\cdots$      &$\cdots$     &$\cdots$    &$\cdots$    \\
44 &\hphantom{9}234\hphantom{9}  & 3094  & 14881   &269    &2918   &13848 \\
$\cdots$ &$\cdots$   & $\cdots$    &$\cdots$      &$\cdots$     &$\cdots$     &$\cdots$    \\
50 &\hphantom{9}\bf 413\hphantom{9}  & \bf2770  & \bf12901   &\bf329    &\bf2792   &\bf13079 \\
51 &\hphantom{9}\bf1040\hphantom{99} & \bf2299  & \hphantom{9}\bf8801    &\bf334    &\bf2784   &\bf13034 \\
52 &\hphantom{9}\bf1062\hphantom{99} & \bf2297  & \hphantom{9}\bf8672    &\bf974    &\bf2262   &\hphantom{9}\bf8768  \\
$\cdots$ &$\cdots$   &$\cdots$    & $\cdots$      &$\cdots$      &$\cdots$     &$\cdots$    \\
55 &\hphantom{9}1069\hphantom{99} & 2297  & \hphantom{9}8630    &1010\hphantom{9}   &2259   &\hphantom{9}8553  \\
\hline
 \end{tabular}
\end{center}
 \end{table*}




Для базовой сети исходная сумма пропускных способностей:
$D^+(0)\hm=68\,256$, а~для кольцевой сети $D^{++}(0)=75\,456$.
Соответствующие значения $\rho^+(0)$ и~$\rho^{++}(0)$ указаны в~<<нулевой>> строке 
в~табл.~1 и~2, где собраны результаты
вычислительных экспериментов. В~ходе эксперимента при
уравнительном распределении остаточных ресурсов соблюдается
\textit{равномерное} убывание остаточной пропускной спо\-соб\-ности и~<<\textit{длины}>> вектора~$\rho^+(t)$. 
Однако между 44--46
итерациями для базовой и~50--52 для кольцевой сети наблюдается
резкий скачок величин~$\rho^-(t)$, $\rho^{=}(t)$ и~$d^+(t)$,
$d^{++}(t)$.

На указанных шагах полностью используется пропускная способность
ребер в~центральной час\-ти сети. Сеть \textit{распадается} на
несвязные компоненты, и~для $80\%$ корреспондентов пропадают пути
соединения, а~остаточный ресурс распределяется поровну между
оставшимися парами узлов.

Анализ результатов показал, что почти равные значения потоков
достигаются для~80\% корреспондентов и~требуют 60\%--70\%
ресурсов. Однако для~2\% смежных  пар узлов межузловые потоки на
два порядка выше медианных значений, а~затраты пропускной
способности  со\-став\-ля\-ют~20\%--30\%.








\section{Заключение}

Предложенный метод и~проведенные вычислительные эксперименты
показали, что уравнительное поэтапное распределение   приводит 
к~неравномерному  распределению   потоков  для разных групп\linebreak
корреспондентов.    Метрические оценки, полученные  в~ходе
экспериментов, продемонстрировали\linebreak \textit{деформацию} множества
достижимых потоков. В~рамках модели   предполагалось, что  все
корреспонденты  равноправны, а~потоки невзаимозаменяемы,  однако
при уравнительном предельном  распределении  смежные  пары узлов
оказывались в~привилегированном положении при использовании
остаточной пропускной способности. Пропускные способности  ребер
рассматривались  как вектор   ресурсов  различных типов,  которые
распределяются между корреспондентами   при передаче  потоков
различных видов.  По построению, на каж\-дом шаге норма вектора
смежных   межузловых    потоков численно равна   модулю вектора
остаточных  пропускных способностей.   Полученные мет\-ри\-че\-ские
значения  можно использовать  для   оценки функциональных
возможностей сети  в~режиме  предельной загрузки.

{\small\frenchspacing
 {%\baselineskip=10.8pt
 %\addcontentsline{toc}{section}{References}
 \begin{thebibliography}{9}

\bibitem{1-mal}
\Au{Малашенко Ю.\,Е., Назарова И.\,А.} Неоднородность
распределения   потоков при предельной  загрузке
многопользовательской сети~//  Известия РАН. Теория и~сис\-те\-мы
управления,  2022. №\,3. С.~81--96.

\bibitem{4-mal} %2
\Au{Luss H.} Equitable resource allocation: Models,
algorithms, and applications.~--- Hoboken, NJ, USA: John Wiley \& Sons, 2012.
420~p.

\bibitem{2-mal} %3
\Au{Ogryczak W., Luss~H., Pioro~M., Nace~D., Tomaszewski~A.}   Fair
optimization and networks: A~aurvey~// J.~Appl. Math., 2014. Vol.~2014. Art.~ID~612018. 25~p. doi: 10.1155/ 2014/612018.

\bibitem{3-mal} %4
\Au{Salimifard K., Bigharaz~S.} The multicommodity network
flow problem: State of the art classification, applications, and
solution methods~// J.~Oper. Res., 2020. Vol.~18. Iss.~3. P.~1--47.



\bibitem{5-mal}
\Au{Balakrishnan A., Li~G., Mirchandani~P.}  Optimal
network design with end-to-end service requirements~// Oper. Res.,
2017. Vol.~65. Iss.~3. P.~729--750.

\bibitem{6-mal}
\Au{Nace D., Doan~L.\,N., Klopfenstein~O., Bashllari~A.} Max-min
fairness in multicommodity flows~// Comput. Oper. Res., 2008.
Vol.~35. Iss.~2. P.~557--573.

\bibitem{7-mal}
\Au{Ros-Giralt J., Tsai~W.\,K.} A~lexicographic optimization
framework to the flow control problem~// IEEE T.
Inform. Theory, 2010. Vol.~56. Iss.~6. P.~2875--2886.

\bibitem{8-mal}
\Au{Baier G., Kohler~E., Skutella~M.}  The \mbox{k-splittable}
flow problem~//  Algorithmica, 2005. Vol.~42. Iss.~3-4.
P.~231--248.

\bibitem{9-mal}
\Au{Bialon P.\,A.} Randomized rounding approach to 
a~\mbox{k-splittable} multicommodity flow problem with lower path flow
bounds affording solution quality guarantees~// Telecommun. Syst.,
2017. Vol.~64. Iss.~3. P.~525--542.
\end{thebibliography}

 }
 }

\end{multicols}

\vspace*{-6pt}

\hfill{\small\textit{Поступила в~редакцию 10.06.22}}

\vspace*{8pt}

%\pagebreak

%\newpage

%\vspace*{-28pt}

\hrule

\vspace*{2pt}

\hrule

%\vspace*{-2pt}

\def\tit{SEQUENTIAL ANALYSIS AND METRIC ESTIMATES\\ OF~PEAK LOAD FLOWS IN~THE~MULTIUSER NETWORK}


\def\titkol{Sequential analysis and metric estimates of~peak load flows in~the~multiuser network}


\def\aut{Yu.\,E.~Malashenko}

\def\autkol{Yu.\,E.~Malashenko}

\titel{\tit}{\aut}{\autkol}{\titkol}

\vspace*{-8pt}


\noindent
Federal Research Center ``Computer Science and Control'' of the Russian Academy of Sciences, 
44-2~Vavilov Str., Moscow 119333, Russian Federation



\def\leftfootline{\small{\textbf{\thepage}
\hfill INFORMATIKA I EE PRIMENENIYA~--- INFORMATICS AND
APPLICATIONS\ \ \ 2022\ \ \ volume~16\ \ \ issue\ 3}
}%
 \def\rightfootline{\small{INFORMATIKA I EE PRIMENENIYA~---
INFORMATICS AND APPLICATIONS\ \ \ 2022\ \ \ volume~16\ \ \ issue\ 3
\hfill \textbf{\thepage}}}

\vspace*{3pt} 



\Abste{The set of vectors of internodal flows in a~multiuser communication network under peak load is analyzed. Within the framework of
 the multicommodity model, the throughput capacities of edges are considered as the components of a~vector of resources of various types that 
 are required for the transmission of various kinds of
 flows. When conducting computational experiments, at each iteration, the
  norms of vectors of jointly permissible internodal flows are calculated, during the transmission of which the capacity of 
  all network edges is fully used.\linebreak\vspace*{-12pt}}
 
 \Abstend{The proposed method and computational experiments have shown that the equalizing phased 
  distribution leads to an uneven distribution of flows for different groups of correspondents. Metric values obtained during experiments 
  indicate deformation of the sets of accessible flows. Within the framework of the model, all correspondents are tantamount 
  and the flows are noninterchangeable; however, in the case of an equalizing peak load distribution, adjacent pairs 
  of nodes are in a privileged position when using residual capacity. The obtained metric values can be used to 
  evaluate the functional characteristics of the transmission network in the finite capacity loading mode.}

\KWE{multicommodity flow network model; set of achievable internodal flows; peak load distribution}


\DOI{10.14357/19922264220306} 

%\vspace*{-16pt}

%\Ack
%\noindent



%\vspace*{4pt}

  \begin{multicols}{2}

\renewcommand{\bibname}{\protect\rmfamily References}
%\renewcommand{\bibname}{\large\protect\rm References}

{\small\frenchspacing
 {%\baselineskip=10.8pt
 \addcontentsline{toc}{section}{References}
 \begin{thebibliography}{9}
\bibitem{1-mal-1}
\Aue{Malashenko, Yu.\,E., and I.\,A.~Nazarova.}
2022. Heterogeneous flow distribution at the peak load in the multiuser network. \textit{J.~Comput. Sys. Sc. Int.} 61:372--387.

\bibitem{4-mal-1} %2
\Aue{Luss, H.} 2012. \textit{Equitable resource allocation: Models, algorithms, and applications}.
Hoboken, NJ: John Wiley \& Sons. 420~p.

\bibitem{2-mal-1} %3
\Aue{Ogryczak, W., H.~Luss, M.~Pioro, D.~Nace, and A.~Tomaszewski.}
 2014. Fair optimization and networks: A~survey. \textit{J.~Appl. Math.} 2014:612018. 25~p. doi: 10.1155/ 2014/612018.
\bibitem{3-mal-1} %4
\Aue{Salimifard, K., and S.~Bigharaz.}
 2020. The multicommodity network flow problem: State of the art classification, applications, and solution methods. 
 \textit{J.~Oper. Res.} 18(3):\linebreak 1--47.

\bibitem{5-mal-1}
\Aue{Balakrishnan, A., G.~Li, and P.~Mirchandani.} 2017. Optimal network design with end-to-end service requirements. 
\textit{Oper. Res.} 65(3):729--750.
\bibitem{6-mal-1}
\Aue{Nace, D., L.\,N.~Doan, O.~Klopfenstein, and A.~Bashllari.} 2008. Max-min fairness in multicommodity flows. 
\textit{Comput. Oper. Res.} 35(2):557--573.
\bibitem{7-mal-1}
\Aue{Ros-Giralt, J., and W.\,K.~Tsai.} 2010. A~lexicographic optimization framework to the flow control problem. 
\textit{IEEE T.~Inform. Theory} 56(6):2875--2886.
\bibitem{8-mal-1}
\Aue{Baier, G., E.~Kohler, and M.~Skutella.}
 2005. The k-splittable flow problem. \textit{Algorithmica} 42(3-4):231--248.
\bibitem{9-mal-1}
\Aue{Bialon, P.} 2017. A~randomized rounding approach to a~\mbox{k-splittable} multicommodity flow problem with lower path flow bounds affording solution quality guarantees. 
\textit{Telecommun. Syst.} 64(3):525--542.
 \end{thebibliography}

 }
 }

\end{multicols}

\vspace*{-6pt}

\hfill{\small\textit{Received June 10, 2022}}

\Contrl

\noindent
\textbf{Malashenko Yuri E.} (b.\ 1946)~--- 
Doctor of Science in physics and mathematics, principal scientist, Federal Research Center ``Computer Science and Control'' 
of the Russian Academy of Sciences, 44-2~Vavilov Str., Moscow 119333, Russian Federation; \mbox{malash09@ccas.ru} 


\label{end\stat}

\renewcommand{\bibname}{\protect\rm Литература}    %6
\def\stat{agalarov}


\def\tit{ПРИБЛИЖЕННЫЙ МЕТОД ВЫЧИСЛЕНИЯ ХАРАКТЕРИСТИК УЗЛА 
ТЕЛЕКОММУНИКАЦИОННОЙ СЕТИ С~ПОВТОРНЫМИ ПЕРЕДАЧАМИ}
\def\titkol{Приближенный метод вычисления характеристик узла 
телекоммуникационной сети с~повторными передачами} 

\def\autkol{Я.\,М.~Агаларов}
\def\aut{Я.\,М.~Агаларов$^1$}

\titel{\tit}{\aut}{\autkol}{\titkol}

%{\renewcommand{\thefootnote}{\fnsymbol{footnote}}\footnotetext[1]
%{Работа выполнена при поддержке РФФИ, проекты 08--07--00152 и 08--01--00567.}}

\renewcommand{\thefootnote}{\arabic{footnote}}
\footnotetext[1]{Институт проблем
информатики Российской академии наук, agglar@yandex.ru}

%\vspace*{-6pt}


\Abst{Рассмотрена модель узла коммутации пакетов c повторными передачами для двух 
схем распределения буферной памяти: полнодоступной и полного разделения. Предложен 
приближенный метод вычисления интенсивностей потоков и вероятностей блокировок узла. 
Получены необходимые и достаточные условия существования и единственности решения 
уравнения для потоков в узле при установившемся режиме работы и доказана сходимость 
итерационного метода решения указанного уравнения.}

\KW{узел коммутации пакетов; буферная память; повторные передачи; вероятности 
блокировок; итерационный метод}

      \vskip 18pt plus 9pt minus 6pt

      \thispagestyle{headings}

      \begin{multicols}{2}

      \label{st\stat}


\section{Введение}

    Одной из основных задач предварительного анализа 
телекоммуникационных сетей коммутации пакетов с ограниченной буферной 
памятью является расчет характеристик потоков и вероятностей блокировок в 
узлах связи. Важность указанных характеристик определяется тем, что от их 
значений существенным образом зависят другие основные показатели сети 
(пропускная способность, задержки пакетов и~др.). 

    Существует множество различных моделей узлов коммутации пакетов и 
методов их расчета (см., например,~[1--6]). Для моделей, рассматривающих 
узел с ограниченной буферной памятью как систему массового обслуживания 
(CMO) типа 
$
\begin{matrix}
M \\ \lambda
\end{matrix}
\left |
\begin{matrix}
M \\ \lambda
\end{matrix}
\right |
\overline{m} \vert N
$ или  $\vert PH\vert PH\vert 1\vert r$, в предположении отсутствия повторных 
передач пакетов получены точные методы вычисления характеристик 
узлов~[1, 3, 4, 6]. Приближенные методы расчета узлов, учитывающие повторные 
попытки передачи, используют модели типа $\vert PH\vert PH\vert 1\vert r$ или 
$
\begin{matrix}
M \\ \lambda
\end{matrix}
\left |
\begin{matrix}
M \\ \lambda
\end{matrix}
\right |
1 \vert N
$ и являются 
итерационными~[2, 3, 5, 7]. Для моделей типа 
$BM\!AP\vert PH\vert 1$, $M\vert G\vert 1\vert r$ и $M\!AP\vert 
(PH,PH)\vert 1$ с повторными заявками получены точные методы вычисления 
характеристик (например, в работах~[8--10]), которые также могут быть 
использованы при расчете узлов.

    Ниже будут рассмотрены модели узла коммутации пакетов с повторными 
передачами для двух схем распределения буферной памяти: с 
полнодоступными буферами и с полным разделением буферной памяти. 
Предлагается приближенный метод расчета характеристик, который в качестве 
модели узла использует СМО типа $
\begin{matrix}
M \\ \lambda
\end{matrix}
\left |
\begin{matrix}
M \\ \lambda
\end{matrix}
\right |
\overline{m} \vert N
$ с повторными заявками. Доказаны утверждения о 
достаточных и необходимых условиях существования и единственности 
решения уравнения для вероятности блокировки в установившемся режиме 
работы и сходимости предлагаемого итерационного метода. 

\section{Модель узла}

    Математическая модель узла представляется в виде СМО с ограниченной 
буферной памятью и различными потоками заявок, каждая из которых требует 
обслуживания только на одной из многоканальных линий связи. 

    Пусть $0<N<\infty$~--- число мест хранения в буферной памяти, $u$~--- 
узел связи, $v$~--- линия связи, $\Omega_u^+$~--- множество исходящих из 
узла~$u$ линий, $c_v$~--- канальная емкость линии~$v$. Поток заявок, 
тре\-бу\-ющих обслуживания на линии~$v$, назовем $v$-по\-то\-ком, заявки этого 
потока~--- $v$-за\-яв\-ка\-ми.


    Пусть выполняются следующие предположения: 
\begin{enumerate}[1.]
\item Места в буферной памяти распределяются согласно одной из двух 
схем:
\begin{enumerate}[($i$)]
\item полнодоступная схема~--- каждое свободное место хранения доступно 
любой заявке;
\item схема полного разделения памяти~--- $v$-за\-яв\-кам доступны всего 
$N_v$ мест, где $\sum\limits_{v\in\Omega_u^+} N_v=N$.
\end{enumerate}
\item Если в момент поступления $v$-заявки в буферной памяти есть 
доступное свободное место, то она сразу занимает это место. Если в момент 
поступления $v$-заявки в системе нет свободного доступного места 
хранения, то поступившая заявка через некоторое время повторно поступает 
на систему, оставаясь $v$-заявкой. 
\item Интенсивности первичных потоков $v$-заявок~--- заданные величины 
$0<\Lambda_v<\infty$, $v\in \Omega_u^+$. Суммарные потоки первичных и 
повторных $v$-заявок являются независимыми в совокупности 
пуассоновскими потоками. Для обслуживания $v$-заявки требуется 
одновременно одно место хранения и один канал типа~$v$, $v\in 
\Omega_u^+$.
\item Первичные нагрузки~--- реализуемые, т.\,е.\ в данном случае 
интенсивности входных первичных потоков равны интенсивностям 
выходных потоков выполненных заявок. 
\item Принятые в СМО $v$-заявки обслуживаются линией~$v$ в порядке 
поступления. 
\item Время занятия канала $v$-заявкой~--- экспоненциально 
распределенная случайная величина с параметром $0<\mu_v<\infty$, 
$v\in\Omega_u^+$, независимая от других случайных событий в узле.
\item Выполненная $v$-заявка с вероятностью~$B_v$ повторяется через 
заданное время~$\tau_v$ (тайм-аут) и с вероятностью $1-B_v$ покидает 
систему через время~$t_v$ навсегда, сразу освободив занятый канал и место 
буферной памяти.
\end{enumerate}

   Будем говорить, что узел блокирован для $v$-за\-яв\-ки, если в буферной 
памяти отсутствует доступное место хранения. Ставится задача вычисления 
вероятностей блокировок и интенсивностей потоков в узле.

\section{Вычисление вероятности блокировки и~интенсивностей~потоков} 

   Пусть $\Lambda_v^*$~--- интенсивность суммарного потока внешних 
заявок, требующих передачи по линии~$v$, $\pi_v$~--- вероятность блокировки 
узла для заявок, требующих передачи по исходящей из узла линии~$v$. 

    Пусть в узле используется полнодоступная схема распределения 
буферной памяти. Тогда, как следует из описания модели, $\pi_v 
=\pi_{v^\prime},\,v,\,v^\prime\in \Omega_u^+$, и для 
интенсивностей~$\Lambda_v^*$, $v\in\Omega_u^+$, справедливы соотношения:
\begin{equation*}
\Lambda_v^* = \fr{\Lambda_v}{1-\pi}\,,
%\label{e1aga}
\end{equation*}
    где
    $\pi =\pi_v$, $v\in\Omega_u^+$.

    Пусть 
    $\overline{k} = \{\overline{k}_v$, $v\in\Omega_u^+\}$~--- состояние 
буферной памяти узла, $\overline{k}_v =\left ( k_v,\,k_v^\prime,\,k_v^{\prime\prime}\right )$; 
$k_v$~--- число $v$-заявок в буферной 
памяти, ожидающих выполнения линией~$v$; $k^\prime_v$~--- число 
$v$-заявок в буферной памяти, ожидающих тайм-аут и неуспешно переданных 
в последующий узел; $k_v^{\prime\prime}$~--- число $v$-за\-явок в буферной 
памяти, успешно переданных в последующий узел и ожидающих 
потверждения; 
$A_m = \left \{ \overline{k}:\ \sum\limits_{v\in\Omega_u^+} \left ( 
k_v+k_v^\prime + k_v^{\prime\prime}\right ) =m \right \}$~--- множество различных 
состояний, при которых в памяти узла занято ровно $m$~буферов. Тогда с 
учетом введенных выше обозначений и предположений для ве\-ро\-ят\-ности 
блокировки узла можно написать формулу~\cite{1aga, 2aga}:
\begin{equation}
\pi = \fr{1}{G_N}\sum\limits_{\overline{k}\in A_N} 
p\left (\overline{k},\overline{\rho}^*\right )\,,
\label{e2aga}
\end{equation}
где  
\begin{gather}
p(\overline{k},\overline{\rho}^*) = \prod\limits_{v\in\Omega_u^+} z_v (\pi, 
\rho_v , k_v , k_v^\prime , k_v^{\prime\prime})\,;\\
z_v (\pi, \rho_v , k_v , k_v^\prime , k_v^{\prime\prime}) ={}\notag\\
\!\!{}=
\begin{cases}
 \fr{\rho_v^{\prime *k_v^\prime}}{k_v^{\prime}!}\,
\fr{\rho_v^{\prime\prime * k_v^{\prime\prime}}}{ k_v^{\prime\prime}!}  \,
\fr{\rho_v^{*k_v}}{ k_{v}!} 
&\mbox{при}\ k_v<c_v\,,\\
 \fr{\rho_v^{\prime * k_v^\prime}}{k_v^{\prime}!} \,
\fr{\rho_v^{\prime\prime * k_v^{\prime\prime}}} { k_v^{\prime\prime}!} 
\fr{\rho_v^{*k_v}}{ c_{v}!c_v^{k_v- c_v}} 
& \mbox{при}\ k_v\geq c_v\,;
\end{cases}\\
G_N = \sum\limits_{m=0}^N\sum\limits_{\overline{k}\in A_m}
p(\overline{k},\overline{\rho}^*)\,;\\ 
\overline{\rho}^*=\{\rho_v^*,\,v\in\Omega_u^+\}\,;\\
\rho_v^* = \fr{\rho_v}{1-\pi}\,;\quad \rho_v =\fr{\Lambda_v}{\mu_v(1- B_v)}\,;\\
\rho_v^{\prime *} =\rho_v^*\mu_v\tau_vB_v\,;\quad \rho_v^{\prime\prime *}=
p_v^* \mu_vt_v,\,\quad  v\in \Omega_u^+\,.\label{e3aga}
\end{gather}

Переобозначив $1-\pi$ через $y$, выражение в правой части равенства~(2)~--- через 
$p_{\overline{k}}(\overline{\rho},y)$, выражение в правой части равенства~(4)~--- 
через $g_N(\overline{\rho},y)$, а выражение в правой 
части равенства~(1)~--- через $1-q_N (\overline{\rho},y)$, 
где $\overline{\rho} = (\rho_v,\,v\in \Omega_u^+)$, $\rho_v = \rho_v^*y\;=$\linebreak 
$=\;\Lambda_v/(\mu_v(1-B_v))$, $v\in\Omega_u^+$, получим нелинейное уравнение 
относительно неизвестной переменной~$y$:
\begin{equation}
y=q_N(\overline{\rho},y)\,.
\label{e4aga}
\end{equation}

    Решим уравнение~(8). Как следует из~(2)--(7), верно 
равенство
\begin{equation}
q_N(\overline{\rho},y) = \fr{g_{N-1}(\overline{\rho},y )}{g_N(\overline{\rho},y)}\,.
\label{e5aga}
\end{equation}
Введем функцию  $d_n(\overline{\rho} ,y)$ среднего числа заявок в узле с 
буферной памятью емкости $n\geq 0$:
$$
d_n(\overline{\rho} ,y) = 
\fr{1}{g_n(\overline{\rho},y)}\,\sum\limits_{m=0}^n m\sum\limits_{\overline{k}\in 
A_m} p_{\overline{k}}(\overline{\rho},y)\,.
$$
Заметим, что $g_n$, $d_n$ и $q_n$, 
$n\geq 0$,~--- непрерывно-дифференцируемые функции по $y\in (0,\,1]$. Взяв 
производную функции~$g_n$ по~$y$, из~(2)--(7) получим
\begin{multline}
\fr{\partial g_n(\overline{\rho},y)}{\partial y} ={}\\
{}= -\sum\limits_{m=0}^n m 
\sum\limits_{\overline{k}\in A_m}\fr{\prod\limits_{v\in\Omega_u^+} z_n 
(0,\rho_v, k_v, k_v^\prime , k_v^{\prime\prime})}{y^{m+1}}={}\\
{}= -\fr{1}{y}\,g_n (\overline{\rho},y)d_n(\overline{\rho},y)\,.
\label{e6aga}
\end{multline}
Взяв производную функции $q_N$ по $y$, из~(\ref{e5aga}) и~(\ref{e6aga}) 
получим
\begin{equation}
\fr{\partial q_N(\overline{\rho},y)}{\partial y} = \fr{q_N(\overline{\rho},y)}{y}\left 
[ d_N (\overline{\rho},y)-d_{N-1}(\overline{\rho},y)\right ]\,.
\label{e7aga}
\end{equation}
    Докажем несколько утверждений о свойствах 
функции~$q_N(\overline{\rho},y)$.
\medskip

\noindent
\textbf{Утверждение 1.} \textit{Справедливы неравенства}
\begin{multline}
0<d_{n+1}(\overline{\rho},y)-d_n(\overline{\rho},y) <1\,,\\
\ \ \ \ \ \ \ \ \ \ \ \ \ \ \ \ \ \ \ \ y\in (0,\,1]\,, \ n\geq 0\,.
\label{e8aga}
\end{multline}


\noindent

Д\,о\,к\,а\,з\,а\,т\,е\,л\,ь\,с\,т\,в\,о\,.\ Подставив выражение для функции 
$d_n(\overline{\rho},y)$ и проведя преобразования, получим
\begin{multline*}
d_{n+1}(\overline{\rho},y) -d_n(\overline{\rho},y) = 
\fr{\sum\limits_{m=0}^{n+1}m\sum\limits_{\overline{k}\in A_m} 
p_{\overline{k}}(\overline{\rho},y)}
{\sum\limits_{m=0}^{n+1}
\sum\limits_{\overline{k}\in A_m} p_{\overline{k}}(\overline{\rho},y)} - {}\\
{}-
\fr{\sum\limits_{m=0}^n m \sum\limits_{\overline{k}\in A_m} p_{\overline{k}} 
(\overline{\rho},y)}{\sum\limits_{m=0}^n
\sum\limits_{\overline{k}\in A_m}p_{\overline{k}}(\overline{\rho},y)}={}\\
{}=\fr{\sum\limits_{m=1}^n m \sum\limits_{\overline{k}\in 
A_m}p_{\overline{k}}(\overline{\rho},y)+(n+1)\sum\limits_{\overline{k}\in 
A_{n+1}}  p_{\overline{k}}(\overline{\rho},y)}{\sum\limits_{m=0}^n\sum\limits_{\overline{k
}\in A_m}p_{\overline{k}}(\overline{\rho},y)+\sum\limits_{\overline{k}\in 
A_{n+1}}p_{\overline{k}}(\overline{\rho},y)} -{}
\end{multline*}
\begin{multline}
{}-
\fr{\sum\limits_{m=0}^n m 
\sum\limits_{\overline{k}\in A_m}p_{\overline{k}}(\overline{\rho},y)}
{\sum\limits_{m=0}^n\sum\limits_{\overline{k}\in A_m} 
p_{\overline{k}}(\overline{\rho},y)}={}\\
{}=\fr{(n+1)\sum\limits_{\overline{k}\in 
A_{n+1}}p_{\overline{k}}(\overline{\rho},y)g_n(\overline{\rho},y)}{g_{n+1}(\overline{\rho},y) g_n(\overline{\rho},y)} -{}\\
{}-
\fr{\sum\limits_{\overline{k}\in 
A_{n+1}}p_{\overline{k}}(\overline{\rho},y)\sum\limits_{m=0}^n  m 
\sum\limits_{\overline{k}\in A_m} p_{\overline{k}}(\overline{\rho},y) }
{g_{n+1}(\overline{\rho},y) g_n(\overline{\rho},y)}
={}\\
{}=\left [ 1-q_{n+1}(\overline{\rho},y)\right ] \left [n+1-d_n(\overline{\rho},y)\right ]\,.
\label{e9aga}
\end{multline}


    Докажем утверждение~1 методом индукции. При $n = 0$, как следует 
из~(\ref{e9aga}), имеем
$$
d_2(\overline{\rho},y) - d_1 (\overline{\rho},y) =1-q_1(\overline{\rho},y)\,,
$$
    т.\,е.\ утверждение~1 при $n = 0$ справедливо. 

    Пусть неравенства~(\ref{e8aga}) справедливы для некоторого $n > 0$. 
Докажем, что они справедливы и для $n + 1$. Из~(\ref{e9aga}) получаем
\begin{multline*}
d_{n+1}(\overline{\rho},y)- d_n(\overline{\rho},y)={}\\
{}=\left [ 1-
q_{n+1}(\overline{\rho},y)\right ] \left [n+1-d_n(\overline{\rho},y)\right ] ={}\\
{}= \left [ 1-
1-q_{n+1}(\overline{\rho},y)\right ] \left [ n-{}\right.\\
{}-\left. d_{n-1}(\overline{\rho},y)+d_{n-1}(\overline{\rho},y)-
d_n(\overline{\rho},y)+1\right ] ={}\\
{}=\left [ 1-q_{n+1}(\overline{\rho},y)\right ] 
\left [ n-d_{n-1}(\overline{\rho},y)-{}\right.\\
{}-\left. \left ( d_n(\overline{\rho},y)-d_{n-1}(\overline{\rho},y)\right )+1\right] = {}\\
{}=
\left [ 1-q_{n+1}(\overline{\rho},y)\right ]
\left [ 
\fr{d_n(\overline{\rho},y) -d_{n-1}(\overline{\rho},y)}{1-
q_n(\overline{\rho},y)}\right.-{}\\
{}-\left.
\left ( d_n(\overline{\rho},y)-d_{n-1}(\overline{\rho},y)\right )+1
\vphantom{\fr{d_n(\overline{\rho})}{(q_n)}}
\right ]={}\\
{}=
\left [ 1-q_{n+1}(\overline{\rho},y)\right ]
\left [ 
\vphantom{\fr{d_n(\overline{\rho})}{(q_n)}}
\left ( d_n(\overline{\rho},y\right)\right. -{}\\
 {}-\left.
d_{n-1}\left(\overline{\rho},y)\right )\fr{q_n(\overline{\rho},y)}{1-
q_n(\overline{\rho},y)}+1\right ]\,.
\end{multline*}
Так как по предположению $d_n (\overline{\rho},y) -d_{n-1}(\overline{\rho},y) 
>0$, то правая часть последнего равенства больше нуля; следовательно, 
$d_{n+1}(\overline{\rho},y)-d_n(\overline{\rho},y)>0$. 

    Продолжив преобразование правой части последнего равенства и 
учитывая предположение $d_n(\overline{\rho},y) -d_{n-1}(\overline{\rho},y)<1$, 
получим
\begin{multline*}
d_{n+1}((\overline{\rho},y) -d_n(\overline{\rho},y)<{}\\
{}< \left [ 1-
q_{n+1}(\overline{\rho},y)\right ]
\left ( \fr{q_n(\overline{\rho},y)}{1-q_n(\overline{\rho},y)}+1\right )={}\\
{}=
\fr{1-q_{n+1}(\overline{\rho},y)}{1-q_n(\overline{\rho},y)}<1\,,
\end{multline*}
так как $0< q_n(\overline{\rho},y)<q_{n+1}(\overline{\rho},y)<1$, $n>0$, $y\in 
(0,\,1]$.

Следовательно, утверждение~1 доказано.

\medskip

\noindent
\textbf{Утверждение 2.} $q_N(\overline{\rho},y)$~--- \textit{монотонно-воз\-рас\-та\-ющая 
функция по $y\in (0,\,1]$. При этом $0< q_N(\overline{\rho},y)\;\leq $\linebreak 
$\leq\;q_N(\overline{\rho},1) <1$, $y\in (0,\,1]$,  и $\underset{y\rightarrow 
0}{\mathrm{lim}}\,q_N(\overline{\rho},y) =0$}.

\medskip

\noindent
Д\,о\,к\,а\,з\,а\,т\,е\,л\,ь\,с\,т\,в\,о\,.\  Возрастание функции 
$q_N(\overline{\rho},y)$ следует непосредственно из~(\ref{e7aga}) и 
утверж\-де\-ния~1. Доказательство неравенств в условии утверждения очевидно 
следует из~(\ref{e5aga}) и вида функции $g_n (\overline{\rho},y)$, $n\geq 0$. 
Для предела имеем:
\begin{multline*}
\underset{y\rightarrow 0}{\mathrm{lim}}\,q_N(\overline{\rho},y) 
=\underset{y\rightarrow 0}{\mathrm{lim}}\,\fr{g_{N- 1}(\overline{\rho},y)}{g_N(\overline{\rho},y)} = {}\\
{}= \underset{y\rightarrow 0}{\mathrm{lim}}\,\left (
g_{N-1}(\overline{\rho},y)\Bigg / \left ( 
\vphantom{\prod\limits_{v\in\Omega_u^+}}
g_{N-1}(\overline{\rho},y)\right.\right.+{}\\
{}+\left.\left.\sum\limits_{\overline{k}\in A_N}\prod\limits_{v\in\Omega_u^+} 
\fr{z_v(0,\rho_v,k_v,k^\prime_v,k^{\prime\prime}_v)}{y^N}\right )\right ) = {}\\
{}= \underset{y\rightarrow 0}{\mathrm{lim}}\,\left (
y^N g_{N-1}(\overline{\rho},y)\Bigg / 
\left ( 
\vphantom{\prod\limits_{v\in\Omega_u^+}}
y^N g_{N-1}(\overline{\rho},y)+{}\right.\right.\\
{}+\left.\left.\sum\limits_{\overline{k}\in A_N}
\prod\limits_{v\in\Omega_u^+} z_v(0,\rho_v,k_v,k_v^\prime , k_v^{\prime\prime}) 
\right ) \right )=0\,.
\end{multline*}
    
\medskip

\noindent
\textbf{Утверждение 3.} \textit{Производная функции~$q_N (\overline{\rho},y)$ по 
$y\in (0,\,1]$ удовлетворяет следующим соотношениям}:
\begin{align}
\underset{y\rightarrow 0}{\mathrm{lim}}\fr{\partial q_N(\overline{p},y)}
{\partial  y} &= \fr{\sum\limits_{\overline{k}\in A_{N-1}} 
p_{\overline{k}}(\overline{\rho},1)}{\sum\limits_{\overline{k}\in 
A_N}p_{\overline{k}}(\overline{\rho},1)}\,;\label{e10aga}\\
\fr{\partial q_N(\overline{\rho},y)}{\partial y}\Big |_{y=1}&<1\,.\label{e11aga}
\end{align}

\medskip

\noindent
Д\,о\,к\,а\,з\,а\,т\,е\,л\,ь\,с\,т\,в\,о\,.\ Проведя преобразования 
функции~$q_N(\overline{\rho},y)$, получим:
\begin{multline*}
\underset{y\rightarrow 0}{\mathrm{lim}}\fr{q_N(\overline{\rho},y)}{y} = {}\\
\!\!{}=
\underset{y\rightarrow 0}{\mathrm{lim}}
\fr{\sum\limits_{m=0}^{N-1}\sum\limits_{\overline{k}\in A_m}
\!\!\left (\prod\limits_{v\in\Omega_u^+}\!\! 
z_v(0,\rho_v,k_v,k_v^\prime , k_v^{\prime\prime})\right )\!\!\Bigg /\!\! y^m}
{y\sum\limits_{m=0}^{N}\sum\limits_{\overline{k}\in A_m}
\!\!\left(\prod\limits_{v\in\Omega_u^+}\!\! z_v\left (0,\rho_v,k_v,k_v^\prime , 
k_v^{\prime\prime}\right )\right )\!\!\Bigg /\!\!y^m} = \!\!\!
\end{multline*}
\begin{multline*}
\!\!\!\!\!\!{}=\underset{y\rightarrow 0}{\mathrm{lim}}\,
\fr{\sum\limits_{m=0}^{N-1}\sum\limits_{\overline{k}\in A_m}
y^{N-1-m}\prod\limits_{v\in\Omega_u^+} z_v(0,\rho_v,k_v,k_v^\prime , 
k_v^{\prime\prime})}{\sum\limits_{m=0}^{N}\sum\limits_{\overline{k}
\in A_m} y^{N-m}
\prod\limits_{v\in\Omega_u^+} z_v(0,\rho_v,k_v,k_v^\prime , 
k_v^{\prime\prime})}={}\!\\
{}=\fr{\sum\limits_{\overline{k}\in A_{N-1}} p_{\overline{k}}(\overline{\rho},1)}{ 
\sum\limits_{\overline{k}\in A_{N}} p_{\overline{k}}(\overline{\rho},1)}\,.
\end{multline*}
Очевидно, $\underset{y\rightarrow 0}{\mathrm{lim}} \,[d_N (\overline{\rho},y) -
d_{N-1} (\overline{\rho},y)]=1$, так как $\underset{y\rightarrow 
0}{\mathrm{lim}}\,d_n (\overline{\rho},y)=n$, $n>0$.

Следовательно, учитывая~(\ref{e7aga}), получаем~(\ref{e10aga}). 
Справедливость~(\ref{e11aga}) непосредственно следует из~(\ref{e7aga}) и 
утверждения~1.

\medskip

\noindent
\textbf{Утверждение 4.} \textit{Пусть $y^*\in (0,\,1]$~--- решение 
уравнения}~(\ref{e4aga}). \textit{Тогда}
\begin{equation*}
\fr{\partial q_N(\overline{\rho},y)}{\partial y}\Big |_{y=y^*}<1\,.
%\label{e12aga}
\end{equation*}

\medskip

\noindent
Д\,о\,к\,а\,з\,а\,т\,е\,л\,ь\,с\,т\,в\,о\,.\ \ Доказательство следует из~(\ref{e7aga}), 
так как $q_N(\overline{\rho},y^*)/y^* =1$.
\medskip

\noindent
\textbf{Утверждение 5.} \textit{Уравнение}~(\ref{e4aga}) \textit{имеет решение $y^*\in 
(0,\,1)$ тогда и только тогда, когда} 
\begin{equation}
\fr{\sum\limits_{\overline{k}\in A_{N-1}} p_{\overline{k}}(\overline{\rho},1)}{ 
\sum\limits_{\overline{k}\in A_{N}} p_{\overline{k}}(\overline{\rho},1)} >1\,.
\label{e13aga}
\end{equation}
\textit{Если уравнение}~(\ref{e4aga}) \textit{имеет решение $y^*\in (0,\,1)$, то оно 
единственное положительное решение}.
\medskip

\noindent
Д\,о\,к\,а\,з\,а\,т\,е\,л\,ь\,с\,т\,в\,о\,.\ Пусть выполняется 
неравенство~(\ref{e13aga}). Тогда, как следует из утверждения~3, 
$\underset{y\rightarrow 0}{\mathrm{lim}} (\partial q_N(\overline{\rho},y)/\partial y) 
>1$. Кроме того, как следует из утверждения~2, 
$\underset{y\rightarrow 0}{\mathrm{lim}} q_N(\overline{\rho},y)=0$. Тогда, так 
как $q_N(\overline{\rho},y)$~--- непрерывно-дифференцируемая функция по 
$y\in (0,\,1]$, существует значение $y^\prime \in (0,\,1)$ такое, что 
$q_N(\overline{\rho},y)>y$ для всех $y\in (0,\,y^\prime]$ (следует из теоремы о 
конечном приращении~\cite{11aga}). В то же время, согласно утверждению~2, 
$q_N(\overline{\rho},y)<y$ в окрестности точки $y=1$ (рис.~\ref{f1aga},\,\textit{а}). 
Следовательно, кривая $x=q_N(\overline{\rho},y)$ пересекает прямую $x=y$ 
хотя бы в одной точке $y=y^*\in (0,\,1)$, т.\,е.\ уравнение~(\ref{e4aga}) имеет 
хотя бы одно решение $y^*\in (0,\,1)$.

\begin{figure*}
\vspace*{1pt}
\begin{center}
\vspace*{1pt}
\mbox{%
\epsfxsize=158mm
\epsfbox{aga-1.eps}
}
\end{center}
\vspace*{-9pt}
\Caption{Примеры кривых $x=q_N(\overline{\rho},y)$ и $x=y$ (\textit{а})~при существовании решения 
уравнения~(\ref{e5aga}) и (\textit{б})~при выполнении условий~(17)
\label{f1aga}}
\vspace*{6pt}
\end{figure*}

Пусть уравнение~(\ref{e4aga}) имеет решение $y^*\in (0,\,1)$ и 
\begin{equation}
\fr{\sum\limits_{\overline{k}\in A_{N-1}}p_{\overline{k}}(\overline{\rho},1)}{ 
\sum\limits_{\overline{k}\in A_{N}}p_{\overline{k}}(\overline{\rho},1)}\leq 
1\,.\label{e14aga}
\end{equation}
Тогда из условий утверждений~2 и~3 следует, что 
уравнение~(\ref{e4aga}) в интервале $(0,\,1)$ имеет более одного решения, что 
может быть только при существовании решения $y^\prime \in (0,\,1)$ такого, 
что в окрестности точки $y=y^\prime$ выполняются неравенства: 
$q_N(\overline{\rho},y)<y$ при $y<y^\prime$ и $q_N(\overline{\rho},y)>y$ при 
$y>y^\prime$, где $y$ принадлежит указанной окрест\-ности точки~$y^\prime$ 
(рис.~\ref{f1aga},\,\textit{б}). Тогда в точке $y=y^\prime$ производная функции 
$q_N(\overline{\rho},y)$ по $y$ больше~1, что противоречит утверждению~4. 
Следовательно, неравенство~(\ref{e13aga}) справедливо.


Пусть уравнение~(\ref{e4aga}) имеет более одного положительного 
решения. Рассуждая точно так же, как и выше (в случае выполнения 
условий~(\ref{e14aga})), получим противоречие утверждению~4. 
Следовательно, утверждение~5 справедливо.
\medskip

\noindent
\textbf{Следствие.} \textit{Неравенства}
\begin{gather*}
\fr{\mu_v c_v (1-B_v)}{\Lambda_v}>1\,,\quad \fr{1-B_v}{\Lambda_v \tau_v B_v}>1\,,\\ 
\fr{1-B_v}{\Lambda_v t_v}>1\,,\ v\in\Omega_u^+\,,
\end{gather*}
\textit{являются необходимым условием существования решения 
уравнения}~(\ref{e4aga}).

\medskip
\noindent
Д\,о\,к\,а\,з\,а\,т\,е\,л\,ь\,с\,т\,в\,о\,.\ Пусть $\overline{k}_v$~--- это 
набор~$\overline{k}$, у которого $k_v=0$. Преобразовав левую 
часть~(\ref{e13aga}), получим

\noindent
\begin{multline*}
\fr{\sum\limits_{\overline{k}\in A_{N-1}} p_{\overline{k}} (\overline{\rho},1)}
{ \sum\limits_{\overline{k}\in A_{N}} 
 p_{\overline{k}}(\overline{\rho},1)} 
={}
\\
{}=
\fr{\sum\limits_{\overline{k}\in A_{N-1}}\prod\limits_{v\in \Omega_u^+} 
z_v\left(0,\rho_v,k_v,k_v^\prime , k_v^{\prime\prime}\right)}
{\sum\limits_{\overline{k}\in A_{N}}
\prod\limits_{v\in \Omega_u^+} z_v\left (0,\rho_v,k_v,k_v^\prime , k_v^{\prime\prime}\right )} \leq{}
\\
{}\leq
\left ( 
\vphantom{\prod\limits_{v^\prime\in\Omega_u^+\backslash v}}
\fr{\mu_v c_v(1-B_v)}{\Lambda_v}\right. \times{}\\
{}\times \sum\limits_{k_v=0}^{N-1}\sum\limits_{\overline{k}_v\in A_{N-1-k_v}} z_v\left(0,\rho_v,k_v+1,k_v^\prime , 
k_v^{\prime\prime}\right )\times{}\\
{}\times \left.\prod\limits_{v^\prime\in\Omega_u^+\backslash v} z_v^\prime 
\left(0,\rho_v,k_v,k_v^\prime , k_v^{\prime\prime}\right) \right)
\Bigg /{}\\
\Bigg / \left ( 
\vphantom{\prod\limits_{v^\prime\in\Omega_u^+\backslash v}}
\sum\limits_{k_v=0}^{N-1} \sum\limits_{\overline{k}_v\in A_{N-1-k_v}} z_v 
\left (0,\rho_v,k_v+1,k_v^\prime , 
k_v^{\prime\prime}\right )\right. \times{}\\
{}\times \prod\limits_{v^\prime\in\Omega_u^+\backslash v} 
z_{v^\prime}\left(0,\rho_v,k_v,k^\prime , k_v^{\prime\prime}\right)+{}\\
{}+
\sum\limits_{\overline{k}_v\in A_N} z_v\left (0,\rho_v, 0,k_v^\prime , 
k_v^{\prime\prime}\right)\times{}\\
\left.{}\times \prod\limits_{v^\prime\in\Omega_u^+\backslash v}z_{v^\prime} 
\left(0,\rho_v,k_v,k_v^\prime , k_v^{\prime\prime}\right )\right )\,.
\end{multline*}
Как следует из правой части последнего неравенства, если 
$\mu_v c_v (1-B_v)/\Lambda_v \leq 1$, то она меньше~1. Поэтому, чтобы 
выполнилось условие~(\ref{e13aga}), необходимо выполнение первого 
неравенства в условии следствия для каждого $v\in\Omega_u^+$. Точно так же 
доказывается необходимость выполнения второго и третьего неравенств в 
условии следствия.

    Пусть $y[n]$, $n\geq 0$, последовательность, полученная по формуле 
$y[n+1]=q_N(\overline{\rho},y[n])$, $y[0]=1$.

\medskip

\noindent
\textbf{Утверждение 6.} \textit{Пусть $y^*\in (0,\,1)$~--- решение 
уравнения}~(8). \textit{Тогда последовательность $y[n]$, $n\geq 0$, сходится 
к решению~$y^*$}.

\medskip

\noindent
Д\,о\,к\,а\,з\,а\,т\,е\,л\,ь\,с\,т\,в\,о\,.\ Отметим, что $y[1]<y[0]$ (это следует из 
утверждения~2, так как $y[0]=1$). Пусть для некоторого $n>1$ выполняется 
условие $y[n]<y[n-1]$. Тогда, как следует из утверждения~2, указанное условие 
выполняется и для $n+1$, т.\,е.\ по индукции следует, что последовательность 
$y[n]$, $n\geq 0$, монотонно убывает. 

    Пусть для некоторого $n>0$ $y[n]>y^*$ (существование такого $n$ 
следует из равенства $y[0]=1$). Тогда, как следует из утверждения~2, 
$y[n+1]\;=$\linebreak $=\;q_N(\overline{\rho},y[n])>q_N(\overline{\rho},y^*) =y^*$, т.\,е.\ 
последовательность ограничена снизу величиной~$y^*$. Значит, существует 
$\underset{n\rightarrow \infty}{\mathrm{lim}}\,y[n]=y^0\geq y^*$. Так как 
$q_n(\overline{\rho},y)$~--- непрерывная по~$y$ функция, то можно написать 
$\underset{n\rightarrow 
\infty}{\mathrm{lim}}\,q_N(\overline{\rho},y[n])=q_N(\overline{\rho},y^0)=y^0$, 
т.\,е.\ $y^0$~--- решение уравнения~(\ref{e4aga}). Из единственности 
положительного решения уравнения~(\ref{e4aga}) получаем $y^0=y^*$.

    Пусть в узле используется схема полного разделения буферной памяти. 
Тогда для интенсив\-ностей~$\Lambda_v^*$, $v\in\Omega_u^+$, справедливы 
соотношения:
$$
\Lambda_v^* = \fr{\Lambda_v}{1-\pi_v}\,,
$$
где $v\in\Omega_u^+$.


Фиксируем произвольную линию сети~$v$. Пусть $\overline{k}_v = (k_v, 
k_v^\prime, k_v^{\prime\prime})$~--- состояние буферной памяти линии~$v$; 
$k_v$, $k_v^\prime$, $k_v^{\prime\prime}$ определены выше. Тогда с 
учетом введенных ранее предположений и обозначений для вероятности 
блокировки линии справедлива формула~\cite{4aga}:
\begin{equation}
\pi_v = \fr{1}{G_{N_v}}\sum\limits_{k_v=N_v} 
z_v(\pi_v,\rho_v,\overline{k}_v)\,,
\label{e15aga}
\end{equation}
где 
\begin{multline*}
z_v(\pi_v, \rho_v, \overline{k}_v)={}\\
{}=
\begin{cases}
\fr{\rho_v^{\prime * k_v^\prime}}{k_v^\prime !}\,
 \fr{\rho_v^{\prime\prime * k_v^{\prime\prime}}}{k_v^{\prime\prime}!}\,
 \fr{\rho_v^{*k_v}}{k_v !} & \mbox{при}\ k_v<c_v\,,\\
 \fr{\rho_v^{\prime *k_v^\prime}}{k_v^{\prime }! }
 \fr{\rho_v^{\prime\prime * k_v^{\prime\prime}}}{k_v^{\prime\prime}!}
\fr{\rho_v^{*k_v}}{c_v !c_v^{k_v-c_v}} & \mbox{при}\ k_v\geq c_v\,;
\end{cases}
\end{multline*}
\begin{align*}
G_{N_v} &= \sum\limits_{m=0}^{N_v} z_v (\pi_v ,\rho_v , \overline{k}_v)\,;\\ 
\rho_v^*&=\fr{\rho_v}{1-\pi_v}\,;
\end{align*}
$\rho_v$, $\rho_v^{\prime *}$, 
$\rho_v^{\prime\prime *}$, $v\in\Omega_u^+$ определены выше.
    
Пусть $y_v=1-\pi_v$, а $q_{N_v} (\rho_v, y_v)$~--- выражение в правой 
части~(\ref{e15aga}). Тогда из равенств~(\ref{e15aga}), взяв~$y_v$ в качестве 
неизвестной переменной, получим систему независимых уравнений:
\begin{equation}
y_v = q_{N_v}(\rho_v, y_v)\,, \quad v\in \Omega_u^+\,.
\label{e16aga}
\end{equation}
    
    Заметим, что для фиксированной $v$ и заданных параметров $\Lambda_v$, 
$\mu_v$, $\tau_v$, $t_v$, $N_v$, $v\in\Omega_u^+$, уравнение в~(\ref{e16aga}) 
является частным случаем уравнения~(\ref{e4aga}) и совпадает с последним, 
когда число исходящих линий из узла равно~1. Следовательно, для схемы 
полного разделения памяти также справедливы все приведенные выше 
утверждения~1--6 и следствие. Заметим, что неравенство~(\ref{e13aga}) в 
условии утверждения~5 при $B_v=0$ и $t_v=0$ преобразуется в неравенство 
$\Lambda_v / (\mu_v c_v) >1$, $v\in\Omega_u^+$. Последовательность 
$\overline{y}[n]$, $n\geq 0$, в утверждении~6 определяется по формуле:
    \begin{gather*}
    \overline{y}[n] =\{y_v[n],\ v\in\Omega_u^+\}\,,\
    y_v[n+1]=q_{N_v} (\rho_v,\,y_v[n])\,,\\
    y_v[0] =1\,,\quad n\geq 0\,,\quad v\in \Omega_u^+\,.
    \end{gather*}


\section{Алгоритм расчета} %4

    Для вычисления интенсивностей потоков и вероятностей блокировок в 
узле предлагается следующий алгоритм, описывающий изложенную выше 
итерационную процедуру. Введем обозначения:
$y_u^v$~--- вероятность блокировки узла для заявок, направляемых на 
линию~$v$,
\begin{gather*}
y_u^v  = 
\begin{cases}
y_u & \mbox{для}\ v\in\Omega_u^+\ \mbox{при}\\
&\mbox{полнодоступной схеме},\\
y_v & \mbox{при схеме полного распределения}\\
&\mbox{памяти};
\end{cases}
\\
q_N^v(\overline{\rho}_u^{-v}, y_u^v)  = 
\begin{cases}
q_N(\overline{\rho},y) & \mbox{для}\ v\in\Omega_u^+\ \mbox{при пол-}\\ 
&\mbox{нодоступной схеме},\\
q_{N_v}(\rho_v, y_v) & \mbox{при схеме полного}\\
&\mbox{распределения}\\ 
&\mbox{памяти},  v\in\Omega_u^+\,.
\end{cases}
\end{gather*}



Тогда уравнения~(\ref{e4aga}) и~(\ref{e16aga}) записываются в виде:
$$
y_u^v = q_N^v (\overline{\rho}^v_u, y^v_u)\,,\quad v\in \Omega_u^+\,.
$$
Для значений, вычисляемых на $k$-м шаге алгоритма, к 
обозначениям соответствующих параметров приписывается знак~$[k]$.
\pagebreak

\textbf{Шаг 0.} 
\begin{enumerate}[1.]
\item  \textit{Инициализация}. Вычисление начальных значений 
параметров~$\rho_v$, $v\in\Omega_u^+$: $\Lambda_v[0]=\Lambda_v$, 
$\rho_v[0]=\Lambda_v[0]/(\mu_v(1-B_v))$, $y_u^v[0]=1$.
\item \textit{Проверка условий существования решения}. Если для некоторой 
линии $v\in\Omega_u^+$ выполняется хотя бы одно неравенство $(c_v\mu_v(1-
B_v))/\Lambda_v[0]\;\leq$\linebreak $\leq\;1$, или $(1-B_v)/(\Lambda_v\tau_v B_v) \leq 1$, или 
$(t_v(1\;-$\linebreak $-\;B_v))/\Lambda_v[0] \leq 1$, то алгоритм заканчивает работу с 
результатом <<нагрузка не реализуема>>. Если в узле используется 
полнодоступная схема и $(c_v\mu_v(1-B_v))/\Lambda_v[0] > 1$, $(1-
B_v)/(\Lambda_v\tau_v B_v)\;>$\linebreak $>\;1$, $(t_v(1-B_v))/\Lambda_v[0] > 1$ для всех 
$v\in\Omega_u^+$, то проверяется условие~(\ref{e13aga}) для $\Lambda_v =
\Lambda_v[0]$, $v\in\Omega_u^+$, и при невыполнении этого условия алгоритм 
заканчивает работу с результатом <<нагрузка не реализуема>>.
\end{enumerate}

    При вычислении левой части неравенства~(\ref{e13aga}) рекомендуется 
использовать метод свертки Базена (см.~\cite{12aga}), позволяющий 
производить рекуррентные вычисления (подробно этот метод описан также 
в~[1, 3--6]).



\medskip
\textbf{Шаг~$k$} ($k > 0$):
\begin{enumerate}[1.]
\item \textit{Вычисление вероятностей блокировок}. Используя значения 
параметров $\overline{\rho}_u^v[k-1]$, $y_u^v[k-1]$, $v\in\Omega_u^+$, 
вычисление с помощью формул~(1)--(7) значений 
вероятностей $y[k]=1- \pi [k]$~--- в случае полнодоступной памяти, или 
$y_v[k]=1- \pi_v[k]$, $v\in\Omega_u^+$, с помощью формул~(\ref{e15aga})~--- в 
случае полного разделения памяти. При вычислении этих значений 
рекомендуется использовать метод свертки Базена.
    \item \textit{Проверка условий останова алгоритма}. Если хотя бы для 
одной $v\in\Omega_u^+$ для заданного значения точности   выполняется 
условие
$$
\fr{\vert \Lambda_v^*[k]-\Lambda_v^*[k-1]\vert}{\Lambda_v^*[k]}> \varepsilon\,,
$$
то вычисление параметров $\overline{\rho}_u^v[k]$, $v\in\Omega_u^+$, и 
переход к шагу~$k$, положив $k$ равным $k+1$, иначе алгоритм завершает 
работу. 
\end{enumerate}

    По завершении алгоритма либо выявится, что нагрузка в системе не 
реализуема, либо будут вычислены интенсивности потоков, поступающих на 
линии узла, и стационарные вероятности блокировок для заявок каждого типа. 
    
\section{Примеры расчета}

    Для проверки точности вычисления результатов с помощью 
предложенного выше алгоритма и приемлемости введенных предположений 
были проведены вычислительные эксперименты с использованием 
аналитических и имитационных моделей. Во всех рассмотренных ниже 
примерах потоки внешних заявок считаются пуассоновскими. 
В~табл.~1 приведены значения вероятности блокировок вновь 
поступивших извне заявок, полученные на основании точной формулы, 
приведенной в~\cite{4aga} для СМО типа $M\vert M\vert 1\vert 0$ с повторными 
заявками при экспоненциальном распределении интервала времени между 
повторными попытками (первая строка таблицы), алгоритма из подраздела~5 
настоящей статьи (вторая строка) и имитационной модели при постоянном 
интервале времени между повторными попытками, равном~10 (третья строка). 
Расчет табл.~1 проведен для узла с одной исходящей одноканальной 
линией при интенсивности первичного потока $\Lambda =1$ и емкости 
накопителя $N_v=1$. Таблицы~2 и~3 вычислены с помощью 
алгоритма из подраздела~5 и имитационной модели соответственно при одной 
исходящей линии с числом каналов~10.


    В табл.~\ref{t4aga} и~\ref{t5aga} приведены значения вероятности 
блокировки узла с тремя исходящими линиями канальной емкости~10 каждая 
при $\mu_v =0{,}2$, $v\in\Omega_u^+$,  вычисленные с помощью алгоритма из 
подраздела~5 и имитационной модели с интервалом повторной попытки, 
равным~10, соответственно. В табл.~\ref{t4aga} и~\ref{t5aga} знак <<--->> в 
ячейках означает, что предложенная нагрузка $\Lambda_v$, $v\in\Omega_u^+$, 
не реализуема.



В табл.~\ref{t6aga} отражены вероятности блокировки такого же узла с 
накопителем $N = 35$ при экспоненциальном распределении интервала 
времени между повторными попытками со средним значением~$\tau$. 


Результаты вычислительного эксперимента показывают, что с  увеличением 
длины интервала между повторными попытками  вероятность блокировки 
увеличивается и приближается к значению,\linebreak
вычисленному с помощью 
алгоритма из подраздела~5 (см.\ табл.~\ref{t4aga} и~\ref{t6aga}), т.\,е.\ при 
пуассоновском внешнем потоке заявок предположение, что суммарный 
входной поток заявок  является пуассоновским, вполне приемлемо для 
предварительного анализа характеристик узла (например, при  $\tau c_v\mu_v > 
10$). Как показывают табл.~1--3, вероятность блокировки 
узла существенно зависит от\linebreak 

\vspace*{6pt}
\noindent
%\begin{table*}\small %tabl1
{\small
{{\tablename~1}\ \ \small{Вероятности блокировок при одной исходящей одноканальной линии}}
%\label{t1aga}}
\vspace*{-3pt}

\begin{center}
{\tabcolsep=7.3pt
\begin{tabular}{|c|c|c|c|c|c|}
\hline
&\multicolumn{5}{c|}{$\mu$}\\
\cline{2-6}
\multicolumn{1}{|c|}{\raisebox{4pt}[0pt][0pt]{№}}&1{,}1&1{,}2&2&3&4\\
\hline
1&0,9091&0,8333&0,5000&0,3333&0,2500\\
2&0,9091&0,8333&0,5000&0,3333&0,2500\\
3&0,8867&0,8452&0,4944&0,3167&0,2396\\
\hline
\end{tabular}}
\end{center}
%\vspace*{-6pt}
%\end{table*}
}
%\bigskip
%\medskip
\addtocounter{table}{1}
\pagebreak

\end{multicols}

\renewcommand{\figurename}{\protect\bf Таблица}
%\renewcommand{\tablename}{\protect\bf Рис.}
\begin{figure*}
{\small
\begin{minipage}[t]{76mm}
%\begin{table*}\small %tabl2
\begin{center}
\Caption{Вероятности блокировок при одной исходящей многоканальной линии ($\varepsilon 
=0{,}0001$)
\label{t2aga}}
\vspace*{2ex}

\tabcolsep=6.5pt
\begin{tabular}{|c|c|c|c|c|c|}
\hline
&\multicolumn{5}{c|}{$\mu$}\\
\cline{2-6}
\multicolumn{1}{|c|}{\raisebox{4pt}[0pt][0pt]{$N$}}&0{,}11&0{,}12&0{,}2&0{,}3&0{,}4\\
\hline
10&0,4845&0,2935&0,0204&0,0017&0,0002\\
15&0,1181&0,0545&0,0006&0,0000&0,0000\\
20&0,0489&0,0167&0,0000&0,0000&0,0000\\
\hline
\end{tabular}
\end{center}
%\end{table*}
\end{minipage}
\hfill
\begin{minipage}[t]{76mm}
%\begin{table*}\small %tabl3
\begin{center}
\Caption{Вероятности блокировок при одной исходящей линии
\label{t3aga}}
\vspace*{2ex}

\tabcolsep=6.5pt
\begin{tabular}{|c|c|c|c|c|c|}
\hline
&\multicolumn{5}{c|}{$\mu_v$}\\
\cline{2-6}
\multicolumn{1}{|c|}{\raisebox{4pt}[0pt][0pt]{$N$}}&0{,}11&0{,}12&0{,}2&0{,}3&0{,}4\\
\hline
10&0,5247&0,3238&0,0219&0,0019&0,0001\\
15&0,1726&0,0912&0,0004&0,0001&0,0000\\
20&0,1180&0,0371&0,0000&0,0000&0,0000\\
\hline
\end{tabular}
\end{center}
%\end{table*}
\end{minipage}
}
\vspace*{6pt}
\end{figure*}

\renewcommand{\figurename}{\protect\bf Рис.}
\renewcommand{\tablename}{\protect\bf Таблица}
\addtocounter{table}{2}

\begin{table}\small %tabl4
\begin{center}
\parbox{400pt}{\Caption{Вероятности блокировок при трех исходящих линиях, вычисленные алгоритмом из 
подраздела~5 ($\varepsilon =0{,}0001$)
\label{t4aga}}
}

\vspace*{2ex}

\tabcolsep=8pt
\begin{tabular}{|c|c|c|c|c|c|c|c|c|c|}
\hline
&\multicolumn{9}{c|}{$\Lambda_v$}\\
\cline{2-10}
\multicolumn{1}{|c|}{\raisebox{4pt}[0pt][0pt]{$N$}}&1&1{,}1&1{,}2&1{,}3&1{,}4&1{,}5&1{,}6&1{,}7&1{,}8\\
\hline
20&0,0677&0,1423&0,2975&0,7653&---&---&---&---&---\\
25&0,0065&0,0173&0,0394&0,0827&0.1690&0.3827&---&---&---\\
30&0,0005&0,0019&0,0059&0,0155&0.0361&0.0790&0.1792&0,7259&---\\
35&0,0000&0,0002&0,0008&0,0030&0,0089&0,0234&0,0574&0,1505&---\\
40&0,0000&0,0000&0,0001&0,0005&0,0022&0,0075&0,0220&0,0617&0,2449\\
\hline
\end{tabular}
\end{center}
%\end{table}
\vspace*{6pt}
%\begin{table}\small %tabl5
\begin{center}
\parbox{400pt}{\Caption{Вероятности блокировок при трех исходящих линиях, вычисленные с помощью 
имитационной модели
\label{t5aga}}
}

\vspace*{2ex}

\tabcolsep=8pt
\begin{tabular}{|c|c|c|c|c|c|c|c|c|c|}
\hline
&\multicolumn{9}{c|}{$\Lambda_v$}\\
\cline{2-10}
\multicolumn{1}{|c|}{\raisebox{4pt}[0pt][0pt]{$N$}}&1&1{,}1&1{,}2&1{,}3&1{,}4&1{,}5&1{,}6&1{,}7&1{,}8\\
\hline
20&0,0786&0,1695&0,3549&0,7056&---&---&---&---&---\\
25&0,0069&0,0190&0,0441&0,0998&0,2266&0,4583&---&---&---\\
30&0,0007&0,0024&0,0075&0,0184&0,0462&0,1025&0,2380&0,6931&---\\
35&0,0000&0,0003&0,0007&0,0040&0,0129&0,0307&0,0890&0,2981&---\\
40&0,0000&0,0000&0,0000&0,0011&0,0041&0,0095&0,0346&0,0790&0,3179\\
\hline
\end{tabular}
\end{center}
%\end{table}
\vspace*{6pt}
%\begin{table}\small %tabl6
\begin{center}
\parbox{356pt}{\Caption{Зависимость вероятности блокировки при трех исходящих линиях, вы\-чис\-лен\-ные с 
помощью имитационной модели со случайным интервалом между повторными попытками
\label{t6aga}}
}

\vspace*{2ex}

\tabcolsep=8pt
\begin{tabular}{|c|c|c|c|c|c|c|c|c|}
\hline
&\multicolumn{8}{c|}{$\Lambda_v$}\\
\cline{2-9}
\multicolumn{1}{|c|}{\raisebox{6pt}[0pt][0pt]{$\tau$}}&1&1{,}1&1{,}2&1{,}3&1{,}4&1{,}5&1{,}6&1{,}7\\
\hline
\hphantom{9}1&0.0001&0,0001&0,0017&0,0063&0,0210&0,0733&0,1996&0,4222\\
\hphantom{9}5&0.0000&0,0002&0,0016&0,0036&0,0446&0,0159&0,1360&0,3273\\
10&0.0000&0,0002&0,0011&0,0036&0,0101&0,0430&0,0818&0,2774\\
20&0.0000&0,0003&0,0007&0,0029&0,0089&0,0257&0,0863&0,2045\\
     \hline
\end{tabular}
\end{center}
\end{table}


\begin{multicols}{2}


\noindent
числа каналов в линии при равной суммарной 
производительности. Кроме того, как видно из табл.~\ref{t5aga} и~\ref{t6aga}, 
вероятность блокировки в большей степени зависит от среднего значения 
длины интервала между повторными попытками передачи, чем от закона 
распределения длины интервала. Таким образом, предложенный в работе 
алгоритм позволяет вы\-чис\-лить с достаточной точностью вероятность 
блокировки узла, интенсивности повторных передач и предельную величину 
реализуемой нагрузки. Отметим, что полученные в данной статье результаты 
могут быть использованы для расчета нагрузок в телекоммуникационной сети с 
повторами заявок в предыдущем узле или из источника. 


{\small\frenchspacing
{%\baselineskip=10.8pt
\addcontentsline{toc}{section}{Литература}
\begin{thebibliography}{99}    
\bibitem{1aga}
\Au{Kamoun~F., Kleinrock~L.}
Analysis of shared finite storage in a computer networks node environment under 
general traffic conditions~// IEEE Trans. on Commun., 1980. Vol.~28. No.\,7. 
P.~992--1003.

\bibitem{6aga} %2
\Au{Агаларов~Я.\,М., Шоргин~С.\,Я.}
Рекуррентный метод вычисления параметров сетей связи~// Техника средств 
связи, 1986. Сер. <<Системы связи>>. Вып.~6. С.~42--46.

\bibitem{3aga}
\Au{Башарин Г.\,П., Бочаров~П.\,П., Коган~Я.\,А.}
Анализ очередей в вычислительных сетях.~--- М.: Наука, 1989. 

\bibitem{4aga}
\Au{Бочаров~П.\,П., Печинкин~А.\,В.}
Теория массового обслуживания.~--- М.: Изд-во РУДН, 1995. 

\bibitem{5aga}
\Au{Вишневский~В.\,М.} 
Теоретические основы проектирования компьютерных сетей.~--- М.: 
Техносфера, 2003. 

\bibitem{2aga} %6
\Au{Башарин Г.\,П.}
Лекции по математической теории телетрафика.~--- М.: Изд-во РУДН, 2007. 

\bibitem{7aga}
\Au{Таранцев~А.\,А.}
Инженерные методы теории массового обслуживания.~--- М.: Наука, 2007.

\bibitem{9aga} %8
\Au{D'Apice~C., De~Simone~T., Manzo~R., Rizelian~G.}
$M\vert G\vert 1\vert r$ retrial queueing system with priority service of primary 
customers and a customers-searching server~// Distributed Computer and 
Communication Networks. Stochastic Modelling and Optimization.~--- М.: 
Техносфера, 2003. P.~106--117.

\bibitem{8aga} %9
\Au{Klimenok~V.\,I., Kim~C.\,S.}
$BM\!AP$/$PH$/1 retrial system operating in random environment~// Proceedings of 
the 5th St.-Petersburg Workshop on Simulation, St.-Petersburg, June~26\,--\,July~2, 
2005.~--- St.-Petersburg: NII Chemistry St.-Petersburg University Publs., 
2005. P.~367--372.   

\bibitem{10aga}
\Au{Krishnamoorthy~A., Babu~S.}
$M\!AP\vert (PH,PH)/c$ retrial queue with selegeneration of priorities 
and non-preemptive service~// Proceedings of the 14th International Conference on 
Analytical and Stochastic Modeling Techniques and Applications, June~4--6, 
2007. Prague, Czech Republic.~--- Sbr.-Dudweiler: Digitaldruck Pirrot GmbH, 
2007. P.~70--74.

\bibitem{11aga}
\Au{Корн~Г., Корн~Т.}
Справочник по математике.~--- М.: Наука, 1974.

\label{end\stat}


\bibitem{12aga}
\Au{Buzen~J.\,P.}
Computational algorithm for closed queuing networks with exponential servers~// 
Communications ACM, 1973. Vol.~16. No.\,9. P.~527--531.
 \end{thebibliography}
}
}
\end{multicols}
 
 
  %7
\def\stat{grusho}

\def\tit{АРХИТЕКТУРНЫЕ РЕШЕНИЯ В~ЗАДАЧЕ ВЫЯВЛЕНИЯ МОШЕННИЧЕСТВА ПРИ~АНАЛИЗЕ 
ИНФОРМАЦИОННЫХ ПОТОКОВ В~ЦИФРОВОЙ ЭКОНОМИКЕ$^*$}

\def\titkol{Архитектурные решения в~задаче выявления мошенничества при~анализе 
информационных потоков в
%~цифровой 
экономике}

\def\aut{А.\,А.~Грушо$^1$, М.\,И.~Забежайло$^2$, Н.\,А.~Грушо$^3$, 
Е.\,Е.~Тимонина$^4$}

\def\autkol{А.\,А.~Грушо, М.\,И.~Забежайло, Н.\,А.~Грушо, 
Е.\,Е.~Тимонина}

\titel{\tit}{\aut}{\autkol}{\titkol}

\index{Грушо А.\,А.}
\index{Забежайло М.\,И.}
\index{Грушо Н.\,А.}
\index{Тимонина Е.\,Е.}
\index{Grusho A.\,A.}
\index{Zabezhailo M.\,I.}
\index{Grusho N.\,A.}
\index{Timonina E.\,E.}


{\renewcommand{\thefootnote}{\fnsymbol{footnote}} \footnotetext[1]
{Работа частично поддержана РФФИ (проекты 18-29-03081 и~18-07-00274).}}


\renewcommand{\thefootnote}{\arabic{footnote}}
\footnotetext[1]{Институт проблем информатики Федерального исследовательского центра <<Информатика и~управление>> 
Российской академии наук, grusho@yandex.ru}
\footnotetext[2]{Институт проблем информатики Федерального исследовательского центра <<Информатика и~управление>> 
Российской академии наук, m.zabezhailo@yandex.ru}
\footnotetext[3]{Институт проблем информатики Федерального исследовательского центра <<Информатика и~управление>> 
Российской академии наук, info@itake.ru}
\footnotetext[4]{Институт проблем информатики Федерального исследовательского центра <<Информатика и~управление>> 
Российской академии наук, eltimon@yandex.ru}

\vspace*{-12pt}
   

 
  
  \Abst{Cформулирован подход к~исследованию некоторых видов мошенничества в~цифровой 
экономике с~использованием причинно-следственных связей. Во всех видах рассматриваемых 
мошенничеств должно наблюдаться несоответствие между целями финансовых транзакций 
и~реальной стоимостью достижения этих целей. Данные о транзакциях можно собирать, 
наблюдая информационные потоки, в~которых отражаются эти транзакции. Архитектура сбора 
данных и~их анализа может быть организована с~помощью распределенных реестров 
с~централизованным консенсусом, что позволяет создать аналог электронной бухгалтерской 
книги, фиксирующей финансово-экономическую деятельность субъектов цифровой экономики в~регионе. 
  Рассматриваемые методы выявления мошенничества основаны на противоречиях 
между действиями, описанными в~транзакциях, и~информацией, содержащейся в~планах, 
стандартах, прецедентах и~др. Рассмотрен метод, основанный на некоторой упрощенной схеме 
реализации абстрактного проекта. Для выявления противоречий необходимо проводить анализ 
от следствия к~причине, т.\,е.\ искать аномалии в~информации, описывающей порождение 
наблюдаемых следствий. 
  Показано, как в~реализации проекта можно выделять простые <<необходимые условия>> 
нарушения при\-чин\-но-след\-ст\-вен\-ных связей, т.\,е.\ множество <<необходимых условий>>, 
нарушение которых свидетельствует о наличии мошенничества. Это множество <<необходимых 
условий>> можно назвать метаданными для контроля проекта на выявление мошенничества.} 
 
 
  \KW{цифровая экономика; информационные потоки; при\-чин\-но-след\-ст\-вен\-ные связи; 
выявление мошеннических схем} 

\DOI{10.14357/19922264190204}
  
\vspace*{-4pt}


\vskip 10pt plus 9pt minus 6pt

\thispagestyle{headings}

\begin{multicols}{2}

\label{st\stat}

\section{Введение}

\vspace*{3pt}

  В работе сформулирован подход к~исследованию некоторых видов 
мошенничества в~цифровой экономике с~использованием  
при\-чин\-но-след\-ст\-вен\-ных связей. Рассматриваются три вида мошенничества, 
а именно:
  \begin{enumerate}[(1)]
\item отмыв денег; 
\item обман при выполнении договорных обязательств при реализации 
технических проектов (строительные проекты и~др.); 
\item незаконный вывод денег. 
\end{enumerate}

  Названные виды мошенничества могут быть сведены к~решению одного типа 
задач. Для отмывания денег источник должен заключать фиктивные контракты, 
в~соответствии с~которыми будут переводиться средства за заведомо ненужную 
работу и~материалы. 
  
  Мошенничество, связанное с~невыполнением договорных обязательств, связано 
со снижением качества услуг, качества и~количества закупаемых 
материалов, выполнением работ с~ненадлежащим качеством. 
  
  Вывод денег связан с~переводом средств фир\-мам-од\-но\-днев\-кам, которые 
заведомо не могут выполнить обязательства по контрактам, за которые им 
переводятся средства. 
  
  Таким образом, во всех трех видах рассматриваемых мошенничеств должно 
наблюдаться несоответствие между целями финансовых транзакций и~реальной 
стоимостью достижения этих целей. Данные о транзакциях можно собирать, 
наблюдая информационные потоки, в~которых отражаются эти транзакции. 
  
  Однако для наблюдения таких информационных потоков необходимо создавать 
архитектуру\linebreak телекоммуникационной системы, позволяющей перехватывать 
и~собирать данные о всех транзакциях. Например, такая архитектура может быть 
организована с~помощью распределенных реестров с~централизованным 
консенсусом, т.\,е.\ все информационные потоки, сформированные в~цифровой 
экономике и~несущие информацию о транзакциях, проходят через некоторый 
центральный узел, запоминающий их в~форме распределенного реестра. Такие 
реестры могут дублироваться в~аналогичных центрах различных регионов, что 
позволяет создать аналог электронной бухгалтерской книги, фиксирующей 
фи\-нан\-со\-во-эко\-но\-ми\-че\-скую деятельность субъектов цифровой экономики. Такой 
подход предложено реализовать на базе системы ситуационных центров, что 
отражено в~работах~[1, 2].
  
  Собранная из информационных потоков информация о~транзакциях, т.\,е.\ 
о~контрактах, договорах, платежах, отчетах, закупленных материалах, 
характеристиках исполнителей работ и~др., собирается в~базе данных в~указанном 
центре. Согласно теории интеллектуальных сис\-тем~[3], эту базу данных можно 
называть базой фактов (БФ). Базу фактов можно представить как бинарную мат\-ри\-цу, 
строки которой описывают характеристики, входящие в~транзакции, а столбцы 
нумеруются характеристиками. Строки матрицы будем называть 
\textit{объектами}~[4, 5]. 
  
  Рассматриваемые в~работе методы выявления мошенничества будут основаны 
на противоречиях между действиями, описанными в~транзакциях, и~информацией, 
содержащейся в~планах, стандартах, прецедентах и~др. Для нахождения 
противоречий в~архитектуре центра предусмотрена другая база данных~--- база 
знаний (БЗ)~\cite{3-gr, 6-gr}, которая устроена так же, как БФ. 
  
  Информация в~БЗ собирается на основе положительного опыта или расчетов. 
Используя БЗ, можно выводить факты нарушения при\-чин\-но-след\-ст\-вен\-ных 
связей. Нарушения при\-чин\-но-след\-ст\-вен\-ных связей будем называть 
\textit{аномалиями}. 
  
  Для упрощения дальнейшее изложение будет вестись в~рамках поиска 
противоречий при выполнении некоторого абстрактного проекта. Выявление 
аномалий будет происходить на основе фактов из БФ с~помощью знаний из БЗ 
методами искусственного интеллекта и~интеллектуального анализа 
данных~\cite{6-gr}. 

\vspace*{-10pt}
  
  \section{Модели}
  
  \vspace*{-3pt}
  
  Наиболее сложная из рассмотренных выше задач~--- выявление противоречий, 
т.\,е.\ использование БЗ для получения новых знаний и~выявление аномалий из 
полученных фактов. 
  
  Все способы выявления противоречий основаны на определении 
  причинно-следственных связей. При этом противоречия в~параметрах транзакций по 
отношению к~требуемым в~БЗ составляют сущность аномалий. 
  
   Далее будет рассмотрен метод, основанный на некоторой упрощенной схеме 
реализации абстрактного проекта. 
  
  Каждый проект имеет цель: например, цель представляет собой построение 
некоторой системы. Воспользуемся структурным подходом, который позволяет 
строить проект на основе разбиения системы на подсистемы и~определения 
взаимодействий подсистем~\cite{7-gr}. При этом каждая подсистема также 
представима структурной моделью. 
  
  Как сама система, так и~каждая ее подсистема имеют свой функционал 
и~спецификацию, па\-ра\-мет\-ры настройки и~домены параметров настройки. Кроме 
этих характеристик существует множество характеристик, связанных 
с~<<жизненным циклом>> создания системы. Сюда входят работы, ресурсы, 
сроки выполнения работ по созданию подсистем и~самой системы, стоимости 
компонентов и~материалов, стоимости работ, схемы поставок, договорные 
обязательства и~др. Все характеристики связаны между собой, поэтому можно 
говорить о стоимости и~времени изготовления структурных компонентов системы. 
  
  Одной из важнейших характеристик является смета (система смет для 
подсистем). Смета сопоставляет каждому компоненту системы стоимость его 
изготовления и~настройки. 
  
  Схема построения системы может быть пред\-став\-ле\-на диаграммой, 
изображенной на рис.~1. 

{ \begin{center}  %fig1
 \vspace*{9pt}
   \mbox{%
 \epsfxsize=79mm 
 \epsfbox{gru-1.eps}
 }


\vspace*{9pt}


\noindent
{{\figurename~1}\ \ \small{Диаграмма достижения цели}}
\end{center}
}

\vspace*{9pt}

\addtocounter{figure}{1}
  
  


  Представленная на рис.~1 диаграмма позволяет описать основные классы 
возможных противоречий при достижении цели. Противоречия возникают, когда 
данные БФ не соответствуют требуемым характеристикам. 
  
  
  \section{Потенциальные классы аномалий при~достижении цели}
  
  Выделим четыре потенциальных класса противоречий, которые показывают, 
каким образом нужно искать эти противоречия.
  
 
  Противоречие цели и~проекта (рис.~2) возникает при отсутствии обоснования 
или в~случае логического противоречия между возможностями проектируемого 
функционала и~целью системы. Отметим, что в~проект входят сроки, перечень 
работ, материалы, настройки, которые описываются соответствующими 
параметрами и~допустимыми значениями этих параметров. Проект формируется 
на основе БЗ и~расчетов, исходя из информации, полученной по аналогии 
с~другими проектами и~решениями, которые считаются апробированными. 
  
  Отметим, что цель порождает проект и~в этом смысле является причиной 
проекта. Однако для анализа противоречий необходимо двигаться по штриховой 
стрелке диаграммы (см.\ рис.~2) от проекта к~цели. В~самом деле, любой компонент 
проекта направлен на теоретическое достижение цели. Цель~--- сложный объект, 
поэтому в~проекте могут возникнуть характеристики, противоречащие хотя бы 
некоторым характеристикам цели. Это делает проект противоречивым, но вывод 
об этом может быть сделан только на уровне описания цели. 
  

  Противоречия между проектом и~его реализацией, исключая настройки 
(рис.~3), могут возникать, например, при закупке исполнителем материалов более 
низкого качества по более низким ценам, при попытках достижения требуемых 
сроков работы за счет снижения качества выполнения работ, за счет нахождения 
<<объективных>> причин для увеличения сроков работы и,~следовательно, 
увеличения цены реализации проекта. 


  Для выявления указанных противоречий необходимо двигаться по диаграмме 
(см.\ рис.~3) в~обратную сторону в~соответствии со~штриховыми стрелками. 
Действительно, выявить противоречия между характеристиками закупленных 
материалов и~требуемыми по проекту можно только при обращении к~проекту 
и~его спецификациям. Манипуляции со сроками работы также можно выявить 
только при обращении к~соответствующим расчетам в~проекте. Задержки в~сроках 
работы, связанные с~поставками материалов, можно определить только на 
предыдущем этапе диаграммы (см.\ рис.~3) в~описании проекта. 


  


  Противоречия между реализацией проекта и~его настройкой (рис.~4) возникает, 
когда не удается добиться требуемых значений параметров функционала, не 
удается обеспечить необходимый уровень\linebreak\vspace*{-12pt}

{ \begin{center}  %fig2
 \vspace*{-6pt}
   \mbox{%
 \epsfxsize=16mm 
 \epsfbox{gru-2.eps}
 }


\vspace*{6pt}


\noindent
{{\figurename~2}\ \ \small{Противоречия цели и~проекта}}
\end{center}
}

%\vspace*{9pt}

\addtocounter{figure}{1}

{ \begin{center}  %fig3
 \vspace*{6pt}
    \mbox{%
 \epsfxsize=79mm 
 \epsfbox{gru-3.eps}
 }


\end{center}

\vspace*{-2pt}


\noindent
{{\figurename~3}\ \ \small{Противоречия проекта и~его реализации (без настройки)}}
}

\vspace*{6pt}

\addtocounter{figure}{1}

{ \begin{center}  %fig4
 \vspace*{1pt}
   \mbox{%
 \epsfxsize=54.5mm 
 \epsfbox{gru-4.eps}
 }


\end{center}


\noindent
{{\figurename~4}\ \ \small{Противоречия реализации проекта и~его на\-стройки}}
}

%\vspace*{9pt}

\addtocounter{figure}{1}

{ \begin{center}  %fig5
 \vspace*{5pt}
    \mbox{%
 \epsfxsize=79mm 
 \epsfbox{gru-5.eps}
 }


\end{center}



\noindent
{{\figurename~5}\ \ \small{Противоречия цели и~достигнутой реализации проекта}}
}

\vspace*{6pt}

\addtocounter{figure}{1}

\noindent
 качества реализации проекта. Для 
определения противоречия в~настройках надо опять же двигаться по диаграмме 
(см.\ рис.~4) в~обратную сторону по штриховым стрелкам, так как для выявления 
характеристик результатов работы, которые не дают возможности реализации 
определенного функционала, необходимо иметь информацию о результатах этой 
работы. 


  



  Противоречие между целью и~достигнутой реализацией проекта (рис.~5) 
возникает, когда реализованная система не позволяет достичь цели. В~этом случае 
опять противоречие нужно искать, двигаясь от цели к~реальному достигнутому 
функционалу по штриховой стрелке (см.\ рис.~5).
  
  Суммируя положения, изложенные в~данном разделе, приходим к~выводу, что 
для выявления противоречий необходимо проводить анализ от следствия 
к~причине, т.\,е.\ искать аномалии в~информации, описывающей порождение 
наблюдаемых следствий. 
  
  
  \section{Связь противоречий и~причин}
  
  Прежде чем построить связь между причинами и~противоречиями, кратко 
опишем простейшую модель связи этих понятий. Причины и~противоречия будут 
сформулированы для представления компонентов системы как объектов, 
обладающих наборами известных характеристик~\cite{4-gr, 5-gr}. 
  
  Пусть $U\hm=\{\alpha, \beta, \ldots\}$~--- совокупность характеристик 
(пространство характеристик). Согласно~\cite{4-gr} \textit{объектом}~$O$ 
называется любое подмножество характеристик $O\hm\subseteq U$. Рассмотрим 
последовательность объектов, возможно в~различных пространствах 
характеристик. 
  
  \smallskip
  
  \noindent
  \textbf{Определение~1.}\ Объект~$P$ с~числом характеристик, большим или 
равным~2, является \textit{причиной} объекта (\textit{свойства})~$B$ в~цепочке 
наблюдаемых объектов тогда и~только тогда, когда выполнены следующие 
условия:
  \begin{enumerate}[(1)]
\item для каждого объекта~$C$, если $P\hm\subseteq C$, то $C\mapsto B$, где 
$C\mapsto B$ означает, что объект~$B$ присутствует в~объекте, следующем за 
объектом~$C$;
\item объект~$P$ является минимальным объектом, удовлетворяющим 
условию~1, а~именно: $\forall \alpha\hm\in P$ объект~$P\backslash \{\alpha\}$ 
не является причиной, т.\,е.\ $\exists C:\ \alpha\not\in C$, $P\backslash 
\{\alpha\}\hm\subseteq C$ и~$C\not\mapsto B$, где $C\not\mapsto B$ означает, 
что~$B$ не может содержаться в~объекте, следующем за объектом~$C$. 
\end{enumerate}

  Приведенное определение причины является упрощением причин, 
возникающих в~реальном мире. Например, реальные причины могут возникать\linebreak 
как совокупность характеристик из разных пространств. Одно следствие может 
порождаться разными причинами или возникать из внешних\linebreak и~ненаблюдаемых 
характеристик. Однако пред\-став\-лен\-ная далее формализация позволяет доступно 
изложить при\-чин\-но-след\-ст\-вен\-ные истоки противоречий, которые 
инициируют в~дальнейшем глубокое исследование рассматриваемых процессов.
  
  Будем считать, что для любого интересующего нас свойства~$B$ существует 
причина. Тогда справедлива следующая теорема.
  
  \smallskip
  
  \noindent
  \textbf{Теорема~1.}\ \textit{Для любого свойства~$B$ существует 
единственная причина}. 
  
  \smallskip
  
  \noindent
  Д\,о\,к\,а\,з\,а\,т\,е\,л\,ь\,с\,т\,в\,о\,.\ \ Доказательство будем вести от противного, 
т.\,е.\ предположим, что существуют две причины свойства~$B$: $P$ 
и~$P^\prime$, $P\hm\not= P^\prime$. Тогда существует $\alpha\hm\in U$, которое 
удовлетворяет одному из двух условий:
  \begin{itemize}
\item[(а)] $\alpha\in P$, $\alpha\notin P^\prime$;
\item[(б)] $\alpha\notin P$, $\alpha \in P^\prime$.
\end{itemize}

  Пусть выполняется условие~(б). Тогда $P^\prime\backslash \{\alpha\}$ не 
является причиной по условию~2 определения~1, т.\,е.\ $\exists C$ такое, что 
$\alpha\notin C$, $P^\prime\backslash \{\alpha\}\hm\subseteq C$ и~$C\not\mapsto B$. 
Но если~$B$ произошло и~$P$ его причина, то $C\mapsto B$, что противоречит 
предположению. Теорема~1 доказана.
  
  \smallskip
  
  \noindent
  \textbf{Лемма.} \textit{Если $P$~--- причина появления свойства~$B$, то 
объект~$B$ определяет существование свойства~$P$ в~объекте, 
предшествующем~$B$. }
  
  \smallskip
  
  \noindent
  Д\,о\,к\,а\,з\,а\,т\,е\,л\,ь\,с\,т\,в\,о\,.\ \ Из предположения, что у~каж\-до\-го 
свойства~$B$ есть причина, и~условия, что~$P$ является причиной~$B$, следует, 
что при появлении в~данных свойства~$B$ объект~$C$, предшествующий 
появлению~$B$, содержит как часть объект~$P$. Это следует из теоремы~1 
и~определения причины. 
  
  Докажем принцип <<необходимого условия>>, который, несмотря на простоту 
доказательства, будет играть в~дальнейшем существенную роль.
  
  \smallskip
  
  \noindent
  \textbf{Теорема~2.} \textit{Если~$P$~--- причина появления свойства~$B$ 
и~$A\hm\subseteq P$, то объект~$B$ определяет наличие свойства~$A$ 
в~объекте, предшествующем~$B$}. 
  
  \smallskip
  
  \noindent
  Д\,о\,к\,а\,з\,а\,т\,е\,л\,ь\,с\,т\,в\,о\,.\ \ Пусть в~данных имеется объект~$B$ 
и~$P\mapsto B$, тогда в~силу существования и~единственности причины~$B$ 
в~данных должен существовать объект~$C$, предшествующий~$B$ 
и~содержащий причину~$P$. Поскольку $A\hm\subseteq P$ и~$B$ содержит 
причину~$P$, то $B\mapsto A$. С~учетом леммы теорема~2 доказана.
  
  \smallskip
  
  Пусть даны пространства $U_1, U_2,\ldots$ и~имеется последовательность 
данных (процесс выполнения этапов проекта в~соответствии с~рис.~1) $A, B, 
\ldots$, где каждый объект является подмножеством некоторого 
пространства~$U_i$, $i\hm=1,\ldots$ Тогда в~объекте~$A$ присутствует 
причина~$P$ появления интересующего нас свойства~$C$ в~объекте~$B$. Пусть 
$P\hm\subseteq A$, тогда по теореме~2 $\forall \alpha\hm\in P$:  
$C\mapsto \{\alpha\}$, т.\,е.\ из появления~$C$ следует появление 
характеристики~$\alpha$ в~предшествующем объекте. Это необходимое условие 
того, что~$C$ удовлетворяет причинно-следственным связям развития процесса 
выполнения проекта. Если для~$C$ нет характеристики~$\alpha$, которую можно 
отнести к~причине~$C$, то можно считать, что нарушена  
при\-чин\-но-след\-ст\-вен\-ная связь и~$C$~--- аномальный объект. 
  
  \smallskip
  
  \noindent
  \textbf{Пример.} Если объект~$C$ состоит в~получении суммы~$a$ 
фирмой~$K$, то согласно теореме~2 в~пред\-шест\-ву\-ющем объекте должна 
существовать причина перевода суммы~$a$ на фирму~$K$. Если эта причина 
в~проекте отсутствует, то это можно считать признаком мошеннической схемы. 
Все проекты по предположению собираются из <<кубиков>>, содержащихся в~БЗ. 
Тогда можно сравнить цену объекта~$C$, породившего получение суммы~$a$, 
и~сумму, присутствующую в~смете проекта. Если разница велика, то это либо 
ошибка проекта, либо признак мошеннической схемы.
  
  \section{Поиск противоречий на~основе~принципа <<необходимых~условий>>}
   
  Как было показано в~разд.~3, нахождение противоречий соответствуют 
движению от следствия к~причине. Для каждого объекта в~наблюдаемых данных 
выявление причин его появления является трудоемкой задачей. Кроме того, при 
реализации контроля соблюдения при\-чин\-но-след\-ст\-вен\-ных связей на 
большом множестве участников экономической деятельности задача анализа 
причин становится трудоемкой. Поэтому процедуру контроля необходимо разбить 
на два этапа, где первый этап состоит в~анализе простых <<необходимых 
условий>> проявления мошенничества, когда используется хотя бы одна 
известная характеристика причины. Второй этап (в~режиме офлайн) состоит 
в~выявлении причин, позволяющих провести анализ источников мошеннических 
схем. 
  
  Один из подходов к~выбору <<необходимых условий>> состоит в~построении 
множества подцелей исходной цели проекта (структурный метод построения 
проекта~\cite{7-gr}). Каждая подцель описывается диаграммой на рис.~1, 
и~реализации подцелей должны образовывать полный функционал цели. Это 
является необходимым, но не достаточным условием достижения цели, так как 
при таком подходе отсутствует компонент согласования всех подцелей в~единую 
систему. Однако такой подход значительно упрощает анализ выполнения проекта 
на предмет поиска мошенничества. Если признаки мошенничества будут 
обнаружены в~реализации хотя бы одной из подцелей, то это значит, что 
мошенничество присутствует в~реализации всего проекта. 
  
  Аналогично в~реализации каждого этапа в~любой из подцелей можно выделять 
простые <<необходимые условия>> нарушения при\-чин\-но-след\-ст\-венн\-ых 
связей. 
  
  Таким образом, получается множество <<необходимых условий>>, нарушение 
которых свидетельствует о наличии мошенничества. Это множество 
<<необходимых условий>> можно назвать метаданными~[8, 9] для контроля 
проекта на выявление мошенничества. 
  
  
  \section{Заключение }
  
  В поиске противоречий необходимо от транзакций, соответствующих 
следствиям при\-чин\-но-след\-ст\-вен\-ных связей, переходить к~анализу причин 
наблюдаемых следствий. Это сложная задача, которая связана с~описанием причин 
определенных свойств. 
  
  В работе представлена модель, позволяющая строить множество необходимых 
условий соответствия наблюдаемого следствия вызвавшей его причине. Этот 
подход делает поиск противоречий вполне вычислимой задачей, но не гарантирует 
успех. 
  
  {\small\frenchspacing
 {%\baselineskip=10.8pt
 \addcontentsline{toc}{section}{References}
 \begin{thebibliography}{9}
\bibitem{1-gr}
\Au{Грушо А.\,А., Зацаринный~А.\,А., Тимонина~Е.\,Е.} Блокчейны цифровой экономики на базе 
системы ситуационных центров и~централизованного консенсуса~// Радиолокация, навигация, 
связь: Мат-лы XXV Междунар. научн.-технич. конф.~---
Воронеж: Издательский дом ВГУ, 2019. Т.~6. С.~183--191. 
\bibitem{2-gr}
\Au{Grusho A., Zatsarinny~A., Timonina~E.} A~system approach to information security in 
distributed ledgers on the situational centers platform.~---
Lecture notes in computer science ser.~--- Springer, 2019 
(in press).
\bibitem{3-gr}
\Au{Финн В.\,К.} Искусственный интеллект: Методология, применения, философия.~--- М.: 
Красанд, 2011. 448~с.

\bibitem{5-gr} %4
\Au{Аншаков~О.\,М., Фабрикантова~Е.\,Ф.} ДСМ-ме\-тод автоматического порождения 
гипотез: Логические и~эпистемологические основания.~--- М.: Либроком, 2009. 432~с.

\bibitem{4-gr} %5
\Au{Poelmans J., Elzinga~P., Viaene~S., Dedene~G.} Formal concept analysis in knowledge 
discovery: A~survey~// Conceptual structures: From information to intelligence~/ Eds.\ M.~Croitoru, 
S.~Ferr$\acute{\mbox{e}}$, and D.~Lukose.~--- Lecture notes in computer science 
ser.~--- Berlin--Heidelberg: Springer, 2010. Vol.~6208.  P.~139--153.

\bibitem{6-gr}
\Au{Панкратова~Е.\,С., Финн~В.\,К.} Автоматическое по\-рож\-де\-ние гипотез в~интеллектуальных 
системах.~--- М.: Либроком, 2009. 528~с. 
\bibitem{7-gr}
\Au{Денисов А.\,А., Колесников~Д.\,Н.} Теория больших систем управления.~--- Л.: Энергоиздат, 1982. 488~с.

\bibitem{9-gr}
\Au{Грушо А.\,А., Грушо Н.\,А., Забежайло~М.\,И., Смирнов~Д.\,В., Тимонина~Е.\,Е.} 
Параметризация в~прикладных задачах поиска эмпирических причин~// Информатика и~её 
применения, 2018. Т.~12. Вып.~3. С.~62--66.

\bibitem{8-gr}
\Au{Грушо А.\,А., Грушо Н.\,А., Левыкин~М.\,В., Тимонина~Е.\,Е.} Методы идентификации 
захвата хоста в~распределенной ин\-фор\-ма\-ци\-он\-но-вы\-чис\-ли\-тель\-ной сис\-те\-ме, 
защищенной с~помощью метаданных~// Информатика и~её применения, 2018. Т.~12. Вып.~4. 
С.~41--45.

 \end{thebibliography}

 }
 }

\end{multicols}

\vspace*{-3pt}

\hfill{\small\textit{Поступила в~редакцию 03.04.19}}

%\vspace*{8pt}

%\pagebreak

\newpage

\vspace*{-28pt}

%\hrule

%\vspace*{2pt}

%\hrule

%\vspace*{-2pt}

\def\tit{ARCHITECTURAL DECISIONS IN~THE~PROBLEM 
OF~IDENTIFICATION OF~FRAUD IN~THE~ANALYSIS 
OF~INFORMATION FLOWS IN~DIGITAL ECONOMY\\[-5pt]}


\def\titkol{Architectural decisions in~the~problem 
of~identification of~fraud in~the~analysis 
of~information flows in~digital economy}

\def\aut{A.\,A.~Grusho, M.\,I.~Zabezhailo, N.\,A.~Grusho, and~E.\,E.~Timonina}

\def\autkol{A.\,A.~Grusho, M.\,I.~Zabezhailo, N.\,A.~Grusho, and~E.\,E.~Timonina}

\titel{\tit}{\aut}{\autkol}{\titkol}

\vspace*{-13pt}


 \noindent
   Institute of Informatics Problems, Federal Research Center ``Computer Sciences and 
Control'' of the Russian Academy of Sciences; 44-2~Vavilov Str., Moscow 119133, 
Russian Federation

\def\leftfootline{\small{\textbf{\thepage}
\hfill INFORMATIKA I EE PRIMENENIYA~--- INFORMATICS AND
APPLICATIONS\ \ \ 2019\ \ \ volume~13\ \ \ issue\ 2}
}%
 \def\rightfootline{\small{INFORMATIKA I EE PRIMENENIYA~---
INFORMATICS AND APPLICATIONS\ \ \ 2019\ \ \ volume~13\ \ \ issue\ 2
\hfill \textbf{\thepage}}}

\vspace*{3pt}


   
     
   \Abste{An approach to a~research of some types of fraud in digital economy with the usage of relationships of 
cause and effect is formulated. In all types of the considered frauds, the discrepancy between the 
purposes of financial transactions and actual cost of achievement of these purposes
has to be observed. Data on 
transactions can be collected by observing information flows in which these transactions are reflected. 
The architecture of data collection and their analysis can be organized by means of the distributed 
ledgers with the centralized consensus that allows creating an analog of the electronic account book 
fixing financial and economic activity of subjects of digital economy in the region. 
   The methods of fraud identification considered are based on the contradictions 
between actions described in transactions and information, which is contained in plans, standards, 
precedents, etc. 
   The method based on a~simplified scheme of implementation of the abstract project is considered. 
For identification of contradictions, it is necessary to carry out the analysis from the effect to the cause, 
i.\,e., to look for anomalies in information describing the generation of the observed effects. 
   It is shown how in implementation of the project it is possible to allocate simple ``necessary 
conditions'' of violation of cause and effect relationships, i.\,e., a~set of ``necessary conditions'' 
violation of which demonstrates fraud existence. It is possible to call this set of "necessary conditions" 
by metadata for control of the project for fraud identification.} 
   
   \KWE{digital economy; information flows; relationships of reason and effect; detection of 
fraudulent schemes}
   
  

 \DOI{10.14357/19922264190204}

\vspace*{-20pt}

 \Ack
   \noindent
   The work was partially supported by the Russian Foundation for Basic Research (projects  
18-29-03081 and 18-07-00274).



%\vspace*{6pt}

  \begin{multicols}{2}

\renewcommand{\bibname}{\protect\rmfamily References}
%\renewcommand{\bibname}{\large\protect\rm References}

{\small\frenchspacing
 {\baselineskip=10.5pt
 \addcontentsline{toc}{section}{References}
 \begin{thebibliography}{9}
\bibitem{1-gr-1}
\Aue{Grusho, A.\,A., A.\,A.~Zatsarinny, and E.\,E.~Timonina.} 2019. Blokcheyny tsifrovoy ekonomiki 
na baze sistemy situatsionnykh tsentrov i~tsentralizovannogo konsensusa [Blockchains of digital 
economy on the basis of the system of the situational centres and the centralized consensus]. 
\textit{25th Scientific and Technical Conference (International) ``Radar-Location, Navigation, 
Communication'' Proceedings}. Voronezh: VSU Publs. 6:183--191.
\bibitem{2-gr-1}
\Aue{Grusho, A., A.~Zatsarinny, and E.~Timonina.} 2019 (in press). 
A~system approach to information security 
in distributed ledgers on the situational centers platform. 
Lecture notes in computer science ser. Springer.
\bibitem{3-gr-1}
\Aue{Finn, V.\,K.} 2011. \textit{Iskusstvennyy intellekt: Metodologiya, primeneniya, filosofiya} 
[Artificial intelligence: Methodology, applications, philosophy]. Moscow: KRASAND. 448~p.

\bibitem{5-gr-1}
\Aue{Anshakov, O.\,M., and E.\,F.~Fabrikantova}. 2009. \textit{DSM-metod avtomaticheskogo porozhdeniya gipotez: Logicheskie 
i~epistemologicheskie osnovaniya} [JSM-method of automatic hypothesis generation: Logical and 
epistemological]. Moscow: KD LIBROKOM. 432~p.
\bibitem{4-gr-1} %5
\Aue{Poelmans, J., P.~Elzinga, S.~Viaene, and G.~Dedene.} 2010. Formal concept analysis in 
knowledge discovery: A~survey. \textit{Conceptual structures: From information to intelligence}. 
Eds.\ M.~Croitoru, S.~Ferr$\acute{\mbox{e}}$, and D.~Lukose. Lecture notes in 
computer science ser. Berlin--Heidelberg: Springer. 6208:139--153.

\bibitem{6-gr-1}
\Aue{Pankratov, E.\,S., and V.\,K.~Finn}. 
2009. \textit{Avtomaticheskoe porozhdenie gipotez v~intellektual'nykh 
sistemakh} [Automatic hypotheses generation in intelligent systems]. Moscow: KD 
\mbox{LIBROKOM}.  528~p. 
\bibitem{7-gr-1}
\Aue{Denisov, A.\,A., and D.\,N.~Kolesnikov.} 1982. \textit{Teoriya bol'shikh 
sistem upravleniya} [Theory of big control systems]. Leningrad: Energoizdat. 488~p.

\bibitem{9-gr-1}
\Aue{Grusho, A.\,A., N.\,A.~Grusho, M.\,I.~Zabezhailo, D.\,V.~Smirnov, and 
E.\,E.~Timonina.} 2018. 
Parametrizatsiya v~prikladnykh zadachakh poiska empiricheskikh prichin 
[Parametrization in applied 
problems of search of the empirical reasons]. 
\textit{Informatika i~ee Primeneniya~--- 
Inform. Appl.} 12(3):62--66.

\bibitem{8-gr-1}
\Aue{Grusho, A.\,A., N.\,A.~Grusho, M.\,V.~Levykin, and E.\,E.~Timonina.} 2018. Metody 
identifikatsii zakhvata khosta v~raspredelennoy informatsionno-vychislitel'noy sisteme, 
zashchishchennoy s~pomoshch'yu metadannykh [Methods of identification of host capture 
in the  distributed information system which is protected on the base of meta data].
\textit{Informatika i~ee 
Primeneniya~--- Inform. Appl.} 12(4):41--45.
{ %\looseness=1

}

\end{thebibliography}

 }
 }

\end{multicols}

\vspace*{-12pt}

\hfill{\small\textit{Received April 3, 2019}}

%\pagebreak

%\vspace*{-18pt}

\Contr

\noindent
\textbf{Grusho Alexander A.} (b.\ 1946)~--- Doctor of Science in physics and 
mathematics, professor, principal scientist, Institute of Informatics Problems, 
Federal Research Center ``Computer Sciences and Control'' of the Russian 
Academy of Sciences; 44-2~Vavilov Str., Moscow 119133, Russian Federation; 
\mbox{grusho@yandex.ru} 

\vspace*{3pt}

\noindent
\textbf{Zabezhailo Michael I.} (b.\ 1956)~--- Doctor of Science in physics and 
mathematics, principal scientist, Institute of Informatics Problems, Federal Research 
Center ``Computer Sciences and Control'' of the Russian Academy of Sciences;  
44-2~Vavilov Str., Moscow 119133, Russian Federation; 
\mbox{m.zabezhailo@yandex.ru} 

\vspace*{3pt}


\noindent
\textbf{Grusho Nikolai A.} (b.\ 1982)~--- Candidate of Science (PhD) in physics 
and mathematics, senior scientist, Institute of Informatics Problems, Federal 
Research Center ``Computer Sciences and Control'' of the Russian Academy of 
Sciences; 44-2~Vavilov Str., Moscow 119133, Russian Federation; 
\mbox{info@itake.ru} 

\vspace*{3pt}


\noindent
\textbf{Timonina Elena E.} (b.\ 1952)~--- Doctor of Science in technology, 
professor, leading scientist, Institute of Informatics Problems, Federal Research 
Center ``Computer Sciences and Control'' of the Russian Academy of Sciences;  
44-2~Vavilov Str., Moscow 119133, Russian Federation; 
\mbox{eltimon@yandex.ru} 

\label{end\stat}

\renewcommand{\bibname}{\protect\rm Литература}     %8
\def\stat{grebeshkov}

\def\tit{АНАЛИЗ ВРЕМЕНИ ПЕРЕКЛЮЧЕНИЯ СЕАНСА СВЯЗИ В~ГЕТЕРОГЕННЫХ 
БЕСПРОВОДНЫХ СЕТЯХ ПРИ~ВЕРТИКАЛЬНОМ ХЭНДОВЕРЕ$^*$}

\def\titkol{Анализ времени переключения сеанса связи в~гетерогенных 
беспроводных сетях при вертикальном хэндовере}

\def\aut{А.\,Ю.~Гребешков$^1$, Ю.\,В.~Гайдамака$^2$, О.\,Г.~Вихрова$^3$, 
Э.\,Р.~Зарипова$^4$}

\def\autkol{А.\,Ю.~Гребешков, Ю.\,В.~Гайдамака, О.\,Г.~Вихрова, 
Э.\,Р.~Зарипова}

\titel{\tit}{\aut}{\autkol}{\titkol}

\index{Гребешков А.\,Ю.}
\index{Гайдамака Ю.\,В.}
\index{Вихрова О.\,Г.} 
\index{Зарипова Э.\,Р.}
\index{Grebeshkov A.\,Yu.}
\index{Gaidamaka Yu.\,V.}
\index{Vikhrova O.\,G.} 
\index{Zaripova E.\,R.}



{\renewcommand{\thefootnote}{\fnsymbol{footnote}} \footnotetext[1]
{Публикация подготовлена при финансовой поддержке Минобрнауки России (проект 2.882.2017/4.6).}}


\renewcommand{\thefootnote}{\arabic{footnote}}
\footnotetext[1]{Поволжский государственный университет телекоммуникаций и~информатики, 
\mbox{grebeshkov-ay@psuti.ru}}
\footnotetext[2]{Российский университет дружбы народов; Институт проблем информатики Федерального исследовательского 
центра <<Информатика и~управ\-ле\-ние>> Российской академии наук, 
\mbox{gaydamaka\_yuv@rudn.university}}
\footnotetext[3]{Российский университет дружбы народов, vikhrova\_og@rudn.university}
\footnotetext[4]{Российский университет дружбы народов, zaripova\_er@rudn.university}

\vspace*{4pt}
 

\Abst{Для мобильных абонентов гетерогенной беспроводной сети в~некоторых точках могут 
быть одновременно доступны соединения в~перекрывающих друг друга областях покрытия 
радиосетей различных стандартов. Пользователь с~современным многорежимным 
абонентским устройством может переключаться между различными сетями радиодоступа 
для получения требуемых услуг связи с~помощью процедуры вертикального хэндовера 
(vertical handover, VHO). Для обеспечения качества и~непрерывности связи существенное 
значение имеет время переключения сеанса связи из текущей в~целевую сеть. 
Разработана процедура вертикального хэндовера из беспроводной локальной сети (Wireless 
Local Area Network, WLAN) в~мобильную сеть (3GPP Long Term Evolution, LTE). Проведен 
анализ среднего значения и~квантиля времени переключения с~помощью метода оценки 
времени пребывания заявок в~многофазной системе массового обслуживания (СМО) с~фоновым 
трафиком. Проведен численный эксперимент для близких к~реальным исходных данных 
процесса переключения сеанса связи.}

\KW{гетерогенная беспроводная сеть; сотовая сеть; LTE; мобильность; вертикальный 
хэндовер; надежность соединения; доступность соединения; показатель эффективности; 
процедура установления соединения}

\DOI{10.14357/19922264170409} 


\vskip 10pt plus 9pt minus 6pt

\thispagestyle{headings}

\begin{multicols}{2}

\label{st\stat}

\section{Введение}

\vspace*{-2pt}

  Тенденция развития современных сотовых сетей, приводящая к~увеличению 
емкости сетей за счет их пространственного уплотнения и~совершенствования 
методов управления распределением радиоресурса, соответствует концепции 
HetNet (Heterogeneous Network)~--- гетерогенных сетей. Реализация этой 
концепции стала возможной в~беспроводных сетях стандарта LTE/LTE-A~[1]. 
Исследования таких сетей с~точки зрения различных показателей качества 
активно ведутся в~России~[2--4]. После выбора целевой сети переключение 
сеанса связи многорежимного абонентского устройства (user equipment, UE) 
осуществляется с~использованием процедуры, которая называется 
вертикальным хэндовером (VHO). 

Использование VHO 
позволяет повысить качество обслуживания, например при использовании 
тактильного Интернета~\cite{5-gre}, для поддержки приложений, работающих 
в~реальном времени~\cite{6-gre}. Поэтому актуально проведение исследований и~анализ времени переключения с~целью предотвращения потери информации, 
что особенно важно в~приложениях реального времени. Критерии принятия 
решения о VHO рассмотрены в~\cite{7-gre}. В~\cite{8-gre} при анализе 
исполнения VHO не рассматриваются процедуры авторизации в~целевой сети, 
которые необходимы при VHO. В~\cite{9-gre} на основе теоремы Бёрке 
рассматривается модель аутентификации и~связанные с~этим временн$\acute{\mbox{ы}}$е 
задержки без учета времени на получение информации о параметрах целевой 
сети для VHO. В~\cite{10-gre, 11-gre} не учтено время подключения к~целевой 
сети и~время выделения радиоканала. 

Отличием данной работы  
от~\cite{7-gre, 8-gre, 9-gre, 10-gre, 11-gre} является детальная процедура обмена 
сигнальными сообщениями при вертикальном хэндовере~\cite{12-gre}, которая 
содержит все стадии, начиная от момента получения информации о доступных 
сетях и~заканчивая началом IP (Internet Protocol) сес\-сии абонентского устройства в~целевой 
сети. 

\begin{figure*}[b] %fig1
\vspace*{6pt}
 \begin{center}
 \mbox{%
 \epsfxsize=131.756mm 
 \epsfbox{gre-1.eps}
 }
 \end{center}
\vspace*{-9pt}
\Caption{Обмен сообщениями UE с~ANDSF и~авторизация UE в~сети LTE}
\end{figure*} 

На основе описанной в~разд.~2 процедуры обмена сигнальными 
сообщениями при вертикальном хэндовере в~разд.~3 разработан метод оценки 
времени переключения при вертикальном хэндовере~[13, 14]. С~его помощью 
в~разд.~4 проведена оценка среднего времени VHO с~учетом загрузки узлов 
текущей и~целевых сетей.



\section{Процедура обмена сигнальными сообщениями 
при~вертикальном хэндовере из~недоверенной сети в~сеть LTE}

  Для разработки процедуры принимаются исходные положения, которые не 
нарушают ее целостность и~не ограничивают применение. Вертикальный 
хэндовер осуществляется сервером ANDSF (Access Network Discovery and 
Selection Function). Для обмена данными между UE и~узлом ANDSF 
используется специфицированная партнерством 3GPP 
(3rd Generation Partnership Project) эталонная точка стыка 
S14 уровня приложений. Поддержка мобильности IP при VHO 
в~рассматриваемом примере обеспечивается применением мобильной версии 
протокола с~двойным стеком IPv6~[15, 16]. Для обеспечения без\-опас\-ности 
используется протокол IPsec, идентификация пользователя с~помощью 
шифрования открытым ключом IKE (Internet Key Exchange)~[17].

\begin{figure*}[b] %fig2
\vspace*{6pt}
 \begin{center}
 \mbox{%
 \epsfxsize=132.09mm 
 \epsfbox{gre-2.eps}
 }
 \end{center}
\vspace*{-9pt}
\Caption{Обмен сообщениями при установлении соединения UE с~целевой сетью для VHO 
WLAN-LTE}
\end{figure*}
  
  Пусть многорежимное устройство UE работает в~текущей WLAN-сети, 
авторизация и~аутентификация в~которой не соответствуют спецификациям 
3GPP. С~точки зрения оператора сети LTE текущая сеть считается 
недоверенной (nontrusted) IP-сетью доступа, поэтому UE инициирует VHO 
в~целевую сеть LTE. Функциональными устройствами, вовлеченными 
в~процедуру, являются: UE, ANDSF, шлюз пакетных данных ePDG (evolved 
Packet Data Gateway), расширенная сеть радиодоступа E-UTRAN (evolved 
UMTS (Universal Mobile Telecommunication
System) Terrestrial Radio Access Networks), узел управления мобильностью MME 
(Mobility Management Entity), обслуживающий шлюз S-GW (Serving Gateway), 
пакетный шлюз P-GW (Packet Data Network Gateway), узел выставления счетов 
абонентам hPCRF (home network Policy and Charging Rules Function) 
и~комбинированный сервер HSS/AAA (Home Subscriber 
Server\,/\,Authentication, Authorization, and Accounting). 

Подробное описание 
сигнальных сообщений, включаемых в~процедуру вертикального хэндовера, 
приведено в~\cite{14-gre}.
%
Обмен сигнальными сообщениями для данного этапа 
VHO представлен на рис.~1. 
{\looseness=1

}
  
  В завершающем этап авторизации сообщении (23) UE получает от ePDG 
данные об успешной ав\-то\-ризации в~сети LTE. После этого программное\linebreak 
обеспечение UE реконфигурируется для работы в~сети LTE по туннелю IPsec 
через эталонную точку стыка 3GPP~S2c. 
  



  Вторым этапом исполнения VHO является переключение UE в~целевую сеть 
LTE из текущей недоверенной сети WLAN. Схема обмена сигнальными 
сообщениями на этом этапе показана на рис.~2. 

\begin{figure*}[b] %fig3
\vspace*{1pt}
 \begin{center}
 \mbox{%
 \epsfxsize=151.044mm 
 \epsfbox{gre-3.eps}
 }
 \end{center}
\vspace*{-9pt}
\Caption{Модель многофазной СМО с~фоновым трафиком: \textit{1}~--- основной поток
заявок; \textit{2}~--- фоновый поток заявок}
\end{figure*}
  
  В начале этапа~2 UE синхронизировано с~расширенной сетью радиодоступа 
E-UTRAN, имеет информацию о физическом канале и~временный 
идентификатор радиосоты. Сообщения~(24)--(40)\linebreak
 отвечают за реконфигурацию 
физических каналов. После сообщения~(37), в~котором узел~\mbox{S-GW} 
подтверждает создание требуемого канала для поддержки IP-сес\-сии через 
расширенную систему пакетной передачи данных EPS (Evolved Packet System) 
вместо обмена через WLAN, UE начинает\linebreak принимать и~передавать пакеты 
данных через сеть LTE. 
  


\section{Метод оценки времени переключения с~учетом фонового 
трафика}

  В настоящей работе предлагается использовать приближенный метод оценки 
времени переключения с~учетом наличия в~сети фонового трафика~[18]. Для 
перехода к~аналитической модели пронумеруем введенные в~предыдущих 
разделах функциональные устройства: UE~(I), ANDSF~(II), ePDG~(III),  
E-UTRAN~(IV), MME~(V), S-GW~(VI), P-GW~(VII), hPCRF~(VIII) 
и~HSS/AAA~(IX). Используемая методика подразумевает разбиение потока 
обслуживаемых в~каждом узле сигнальных сообщений на основной и~фоновый 
потоки. Под сообщениями основного потока будем понимать включенные 
в~процедуру вертикального хэндовера сигнальные сообщения, под фоновым 
потоком~--- сообщения других задач. Поток сигнальных сообщений на рис.~3 
образует цепь, состоящую из $K\hm=39$~со\-сто\-яний. 



  Обозначим через~$\lambda_0$ и~$b_k$ интенсивность входящего потока 
и~среднюю длительность обслу\-жи\-вания заявок основного потока на $k$-й фазе 
в~многофаз\-ной СМО. Аналогично~$\lambda_k$ и~$d_k$~--- интенсивность 
потока и~средняя длительность обслуживания заявок фонового потока на $k$-й 
фазе. 
  
  Для расчета времени пребывания заявки в~многофазной СМО необходимо 
вычислить коэффициент вариации длительности обслуживания на $k$-й фазе 
по формуле:
  \begin{equation*}
  C_k^2 =\fr{(\lambda_0+\lambda_k) \left(\lambda_0 b_k^{(2)} +\lambda_k 
d_k^{(2)}\right)} {(\lambda_0 b_k+\lambda_k d_k)^2}-1,\enskip
  k=1,\ldots, K.
 % \label{e1-gre}
  \end{equation*}
Здесь использованы обозначения вторых моментов времени обслуживания 
заявок основного~$b_k^{(2)}$ и~фонового потока~$d_k^{(2)}$.

  Время ожидания начала обслуживания, полученное из известной формулы 
Пол\-ла\-че\-ка--Хин\-чи\-на, рассчитывается по формуле:
  $$
  \omega_k= \fr{\rho_k^2(1+C_k^2)} {2(\lambda_0+\lambda_k) (1-
\rho_k)}\,,\enskip k=1,\ldots, K\,.
  $$
Здесь $\rho_k=\lambda_0b_k\hm+\lambda_kd_k$~--- суммарная нагрузка на узел, 
соответствующий $k$-й фазе, $\rho_k\hm< 1$, $k\hm=1,\ldots, K$.

  Время пребывания~$\Delta$ заявки в~многофазной СМО равно сумме времен 
пребывания заявок основного потока на каждой фазе:
  $$
  \Delta= \sum\limits^K_{k=1} \left( \omega_k+b_k\right)\,.
  $$
  
  Время пребывания~$\Delta$ заявки в~многофазной СМО с~учетом фонового 
трафика соответствует времени переключения при вертикальном хэндовере.
  
  Заметим, что данный метод позволяет найти квантиль уровня~$\psi$ времени 
пребывания заявки в~многофазной СМО с~фоновым трафиком по формуле:
  $$
  Q_\psi= q_\psi +\sum\limits^K_{k=1} \left( \fr{\ln (\gamma_k 
\omega_k)}{\gamma_k}+b_k\right)\,,
  $$
где $q_\psi$ является единственным положительным корнем уравнения
$$
1-\psi= \sum\limits^{K-1}_{k=0} e^{-\gamma_k q_\psi} \fr{(\gamma_k q_\psi)^k} 
{k!}\,.
$$
  
  Параметры $\gamma_k$ затухания функций распределения времени 
ожидания начала обслуживания на $k$-й фазе, в~свою очередь, являются 
единственными положительными корнями уравнения 
$$
\alpha_k(\gamma_k)\beta_k(-\gamma_k)=1\,,
$$
 где $\alpha_k(s)\hm= 
(\lambda_0\hm+ \lambda_k)/(\lambda_0\hm+\lambda_k\hm+ s)$~--- 
преоб\-разование Лап\-ла\-са--Стилть\-еса (ПЛС) функции\linebreak распределения (ФР) 
интервалов времени между поступлениями заявок в~узле на $k$-й фазе, 
а~$\beta_k(s)\hm= (\lambda_0/(\lambda_0\hm+\lambda_k)) e^{-sb_k}\hm+ 
(\lambda_k/(\lambda_0\hm+ \lambda_k))e^{-sd_k}$~--- ПЛС ФР длительности 
обслуживания заявок.

  \begin{table*}[b]
  {\small \begin{center}
  
  \begin{tabular}{|l|c|c|}
  \multicolumn{3}{c}{Средние времена обслуживания}\\
  \multicolumn{3}{c}{\ }\\[-6pt]
  \hline
  \tabcolsep=0pt\begin{tabular}{c}Функциональные\\
узлы\end{tabular}&Среднее время обслуживания
$\mu_i^{-1}$, мс&  \tabcolsep=0pt\begin{tabular}{c}Источники данных\\
 для численного эксперимента\end{tabular}\\
\hline
I~--- UE&\tabcolsep=0pt\begin{tabular}{c}77,5 для (24);\\
28,5 для (26);\\
2 для остальных\end{tabular}&\cite{19-gre}\\
\hline
II~--- ANDSF&70&Считается, как HSS/AAA\\
\hline
III~--- ePDG&\hphantom{9}2&Считается, как P-GW\\
\hline
IV~--- eNB&\hphantom{9}4&\cite{20-gre}\\
\hline
V~--- MME&\tabcolsep=0pt\begin{tabular}{c}15 для (27);\\
1 для остальных\end{tabular}&\cite{20-gre, 21-gre}\\
\hline
VI~--- S-GW&\hphantom{9}2&\cite{19-gre}\\
\hline
VII~--- P-GW&\hphantom{9}2&\cite{19-gre}\\
\hline
VIII~--- hPCRF&70&Считается, как HSS/AAA\\
\hline
IX~--- HSS/AAA&70&\cite{22-gre}\\
\hline
\end{tabular}
\end{center}}
%\end{table*}
%\renewcommand{\figurename}{\protect\bf Рис.}
\renewcommand{\tablename}{\protect\bf Рис.}
\setcounter{table}{3}
%\begin{figure*} %fig4
\vspace*{6pt}
 \begin{center}
 \mbox{%
 \epsfxsize=164.269mm 
 \epsfbox{gre-4.eps}
 }
 \end{center}
\vspace*{-9pt}
\Caption{Среднее время переключения~(\textit{а}) и~95\%-ный квантиль времени переключения~(\textit{б}):
\textit{1}~--- $\lambda_k\hm=\lambda_0$; 
\textit{2}~--- $\lambda_k\hm=10\lambda_0$; \textit{3}~--- 
$\lambda_k\hm=100\lambda_0$}
%\end{figure*}
\end{table*}

\renewcommand{\figurename}{\protect\bf Рис.}
\renewcommand{\tablename}{\protect\bf Таблица}
  
  
\section{Численный эксперимент}

  Интенсивность запросов на совершение вертикального хэндовера зависит от 
местности, типа устройств UE, возможностей оператора связи, плотности 
мобильных пользователей и~других параметров. В~таблице приведены средние 
значения времен обслуживания сообщений в~узлах (основной трафик). 
Некоторые сообщения обслуживаются дольше остальных в~связи с~их 
функциональными особенностями и~б$\acute{\mbox{о}}$льшим объемом, при 
применении метода следует использовать данные статистических наблюдений. 
  

  На рис.~4 представлены результаты анализа времени переключения при 
вертикальном хэндовере для трех вариантов соотношения интенсивностей 
основного и~фонового трафиков: \textit{1}~--- $\lambda_k\hm=\lambda_0$; 
\textit{2}~--- $\lambda_k\hm=10\lambda_0$; \textit{3}~--- 
$\lambda_k\hm=100\lambda_0$. Среднее время обслуживания фонового трафика 
$d_k\hm=2$~мс.
  


  
  Анализ времени переключения показал, что дополнительный трафик со 
средним временем обслуживания несущественно влияет на время 
переключения, когда интенсивность этого трафика имеет тот же порядок, что 
и~интенсивность основного трафика (кривые~\textit{1} и~\textit{2}), 
и~проявляется, когда отношения интенсивностей основного и~фонового 
трафиков различаются на два порядка (кривые~\textit{3} на рис.~4).
  
  

На рис.~5 показана зависимость среднего значе\-ния и~95\%-ного квантиля 
времени переклю\-чения при вертикальном хэндовере для трех значений средней 
длительности обслуживания\linebreak сообщения фонового трафика $d_k\hm=10$, 20 и~50~мс, интенсивности входящего основного и~фонового трафика равны, 
$\lambda_k\hm= \lambda_0$.

\setcounter{figure}{4}

\begin{figure*} %fig5
\vspace*{1pt}
 \begin{center}
 \mbox{%
 \epsfxsize=163.863mm 
 \epsfbox{gre-6.eps}
 }
 \end{center}
\vspace*{-9pt}
\Caption{Среднее время переключения~(\textit{а})
и~95\%-ный квантиль времени переключения:
\textit{1}~--- $d_k\hm=10$~мс;  
\textit{2}~--- 20; \textit{3}~--- $d_k\hm=50$~мс}
\end{figure*}



     При одинаковой интенсивности фонового и~основного трафика 
и~увеличении среднего времени обслуживания фонового трафика до~50~мс 
наблюдается резкое увеличение времени переключения при вертикальном 
хэндовере (кривые~\textit{3} на рис.~5). Однако при среднем времени 
обслуживания фонового трафика до~20~мс фоновый трафик практически не 
влияет на время переключения при вертикальном хэндовере 
(кривые~\textit{1} и~\textit{2} на рис.~5).
     
  Рекомендуемый метод оценки позволяет учесть влияние фонового трафика 
  и~на отдельные узлы сети, участвующие в~процедуре. Тем не менее по 
результатам проведения статистических наблюдений на реальных сетях связи 
указанная рекомендация может уточняться.

\section{Заключение}

  Применение процедуры вертикального хэндовера в~гетерогенных 
беспроводных сетях в~сочетании с~использованием многорежимных 
абонентских устройств открывает широкие возможности по 
дифференцированному доступу абонентов к~ресурсам сетей. 
  
  В статье разработана процедура обмена сигнальными сообщениями для VHO 
из беспроводной локальной сети в~сеть LTE, предложен аналитический метод 
оценки времени переключения. При этом пользователями могут быть как 
устройства пользователей, так и~<<умные>> устройства, автоматически 
взаимодействующие по принципу M2M (Machine-to-Machine). 
  
  В дальнейших исследованиях планируется применить представленный метод 
для оценки времени переключения при обратной процедуре из сетей 
подвижной беспроводной связи LTE в~сеть WLAN.
  
{\small\frenchspacing
 {%\baselineskip=10.8pt
 \addcontentsline{toc}{section}{References}
 \begin{thebibliography}{99}
\bibitem{1-gre}
\Au{Astely D., Dahlman~E., Fodor~G., Parkvall~S., Sachs~J.} LTE release~12 and 
beyond~// IEEE Commun. Mag., 2013. Vol.~51. No.\,7. P.~154--160.
doi: 10.1109/MCOM. 2013.6553692. 
\bibitem{3-gre} %2
\Au{Горбунова А.\,В., Зарядов~И.\,С., Матюшенко~С.\,И., Самуйлов~К.\,Е., 
Шоргин~С.\,Я.} Аппроксимация времени отклика системы облачных 
вычислений~// Информатика и~её применения, 2015. Т.~9. Вып.~3. С.~32--38.
\bibitem{2-gre} %3
\Au{Вихрова О.\,Г., Самуйлов~К.\,Е., Сопин~Э.\,С., Шоргин~С.\,Я.} К~анализу 
показателей качества обслуживания в~современных беспроводных сетях~// 
Информатика и~её применения, 2015. Т.~9. Вып.~4. С.~48--55.

\bibitem{4-gre}
\Au{Гайдамака Ю.\,В., Андреев~С.\,Д., Сопин~Э.\,С., Самуйлов~К.\,Е., 
Шоргин~С.\,Я.} Анализ характеристик интерференции в~модели 
взаимодействия устройств с~учетом среды распространения сигнала~// 
Информатика и~её применения, 2016. Т.~10. Вып.~4. С.~2--10.
\bibitem{5-gre}
\Au{Кучерявый А.\,Е., Маколкина~М.\,А., Киричек~Р.\,В.} Тактильный 
Интернет. Сети связи со сверхмалыми задержками~// Электросвязь, 2016. №\,1. 
С.~44--46.
\bibitem{6-gre}
\Au{Kellokoski J., Koskinen~J., Nyrhinen~R., 
H$\ddot{\mbox{a}}$m$\ddot{\mbox{a}}$l$\ddot{\mbox{a}}$inen T.} Efficient 
handovers for machine-to-machine communications between IEEE 802.11 and 3GPP 
evolved packed core networks~//  IEEE  Conference (International) on Green 
Computing and Communications Proceedings.~--- 
\mbox{Besan{\!\ptb{\c{c}}}on}, 2012. P.~722--725.
\bibitem{7-gre}
\Au{Ahmed L., Boulahia~L.\,M., Gaiti~D.} Enabling vertical handover decisions in 
heterogeneous wireless networks: A~state-of-the-art and a classification~// IEEE 
Commun. Surv.  Tut., 2014. Vol.~16. No.\,2. P.~776--781.
\bibitem{8-gre}
\Au{Bukhari J., Akkari~N.} QoS based approach for LTE-WiFi handover~// 7th 
Conference (International) on Computer Science \& Information Technology 
Proceedings.~--- Elsevier, 2016. P.~1--6.
\bibitem{9-gre}
\Au{Gondim P.\,R.\,L., Trineto~J.\,B.\,M.} DSMIP and PMIP for mobility 
management of heterogeneous access networks: Evaluation of authentication 
delay~//  IEEE Globecom Workshops Proceedings.~--- Anaheim, 2012.  
P.~308--313.

\bibitem{11-gre} %10
\Au{Do-Hyung~K., Won-Tae~K., Hwan-Gu~L., Sun-Ja~K., Cheol-Hoon~L.} 
A~performance evaluation of vertical handover architecture with low latency 
handover~//  Conference (International) on Convergence and Hybrid Information 
Technology Proceedings.~--- Dusan, 2008. P.~66--69.

\bibitem{10-gre} %11
\Au{Tsagkaropoulos M., Politis~I., Tselios~C., Dagiuklas~T., Kotsopoulos~S.} 
Service continuity over intertechnology RATs~// 16th IEEE  Workshop 
(International) on Computer Aided Modeling and Design of Communication Links 
and Networks Proceedings.~--- Kyoto, 2011. P.~117--121.

\bibitem{12-gre}
\Au{M$\acute{\mbox{a}}$rquez-Barja J., Calafate~C.\,T., Cano~J.-C., Manzoni~P.} 
An overview of vertical handover techniques: Algorithms, protocols and tools~// 
Comput. Commun., 2011. Vol.~34. P.~985--997.
\bibitem{13-gre}
\Au{Гребешков А.\,Ю.} Оценка целесообразности обработки заявки для 
предоставления услуги в~реконфигурируемых сетях следующего поколения~// 
T-Comm: Телекоммуникации и~транспорт, 2014. №\,8. С.~24--27. 
\bibitem{14-gre}
\Au{Grebeshkov A., Zaripova~E., Roslyakov~A., Samouylov~K.} Modelling of 
vertical handover from untrusted WLAN network to LTE~// 31st European 
Conference on Modelling and Simulation Proceedings, 2017. P.~694--700.
\bibitem{15-gre}
\Au{Lampropoulos G., Passas~N., Mekaros~L., Kaloxylos~A.} Handover 
management architectures in integrated WLAN~// IEEE Commun. 
Surv. Tut., 2005. Vol.~7. No.\,4. P.~30--44.
\bibitem{16-gre}
3GPP TS~23.402 Technical specification 3GPP; TS Group Services and System 
Aspects; Architecture enhancements for non-3GPP accesses. Release~14, 2016.
{\sf 
https://\linebreak portal.3gpp.org/desktopmodules/Specifications/Specifi cationDetails.aspx?specificationId=850}.
\bibitem{17-gre}
3GPP TS 33.402 Technical specification 3GPP; TS Group Services and System 
Aspects; 3GPP System Architecture Evolution (SAE); Security aspects of non-3GPP 
accesses. Release~14, 2016.
{\sf  
https://portal. 3gpp.org/desktopmodules/Specifications/Specification\linebreak Details.aspx?specificationId=2297.}
\bibitem{18-gre}
\Au{Gaidamaka Yu., Zaripova~E.} Session setup delay estimation methods for  
IMS-based IPTV services~// Internet of things, smart spaces, and next generation networks
and systems~/ Eds.\ S.~Balandin, S.~Andreev, Y.~Koucheryavy.~---
Lecture notes in computer science ser.~---
Springer, 2014. Vol.~8638. 
P.~408--418.
\bibitem{19-gre}
\Au{Nikaein N., Krco~S.} Latency for real-time machine-to-machine communication 
in LTE-based system architecture~// 17th European Wireless Conference on 
Sustainable Wireless Technologies Proceedings.~--- Vienna, Austria: IEEE, 2011. 
P.~1--6.
\bibitem{20-gre}
\Au{Cardona N., Monserrat~J.\,F., Cabrejas~J.} Enabling technologies for 3GPP 
LTE-advanced networks~// LTE-advanced and next generation wireless networks~/ 
Eds.\ G.~de la~Roche, A.\,A.~Glazunov, B.~Allen.~--- Chichester, U.K.: John 
Wiley and Sons, 2013. P.~3--34.
\bibitem{21-gre}
\Au{Prados-Garzon J., Ramos-Munoz~J.\,J., Ameigeiras~P., Andres-Maldonado~P., 
Lopez-Soler~J.\,M.} Latency evaluation of a virtualized MME~//  
 Wireless Days Proceedings.~--- Toulouse, France: IEEE, 2016. P.~1--3.
\bibitem{22-gre}
\Au{Granlund D., Holmlund~P., \mbox{{\ptb{\AA}}hlund~C.}} Opportunistic mobility 
support for resource constrained sensor devices in smart cities~// Sensors, 2015. 
Vol.~15. No.\,3. P.~5112--5135.
 \end{thebibliography}

 }
 }

\end{multicols}

\vspace*{-6pt}

\hfill{\small\textit{Поступила в~редакцию 31.05.17}}

\vspace*{8pt}

%\newpage

%\vspace*{-24pt}

\hrule

\vspace*{2pt}

\hrule

%\vspace*{8pt}


\def\tit{ANALYSIS OF~VERTICAL HANDOVER TIME\\ IN~HETEROGENEOUS WIRELESS NETWORKS}

\def\titkol{Analysis of~vertical handover time in~heterogeneous wireless networks}

\def\aut{A.\,Yu.~Grebeshkov$^1$, Yu.\,V.~Gaidamaka$^{2,3}$, O.\,G.~Vikhrova$^{2}$, 
and~E.\,R.~Zaripova$^2$}

\def\autkol{A.\,Yu.~Grebeshkov, Yu.\,V.~Gaidamaka, O.\,G.~Vikhrova, 
and~E.\,R.~Zaripova}

\titel{\tit}{\aut}{\autkol}{\titkol}

\vspace*{-9pt}


\noindent
$^1$Povolzhskiy State University of Telecommunications and Informatics, 23~Tolstoy Str., Samara 443010, Russian\linebreak 
$\hphantom{^1}$Federation

\noindent
$^2$Peoples' Friendship University of Russia (RUDN University), 6~Miklukho-Maklaya Str., 
Moscow 117198,\linebreak
$\hphantom{^1}$Russian Federation


\noindent
$^3$Institute of Informatics Problems, Federal Research Center ``Computer Science and Control'' 
of the Russian\linebreak
$\hphantom{^1}$Academy of Sciences, 44-2~Vavilov Str., Moscow 119333, 
Russian Federation



\def\leftfootline{\small{\textbf{\thepage}
\hfill INFORMATIKA I EE PRIMENENIYA~--- INFORMATICS AND
APPLICATIONS\ \ \ 2017\ \ \ volume~11\ \ \ issue\ 4}
}%
 \def\rightfootline{\small{INFORMATIKA I EE PRIMENENIYA~---
INFORMATICS AND APPLICATIONS\ \ \ 2017\ \ \ volume~11\ \ \ issue\ 4
\hfill \textbf{\thepage}}}

\vspace*{3pt}



\Abste{In a heterogeneous wireless network, connectivity is simultaneously available 
using different radio networks with overlapping coverage areas. A~mobile user 
equipment with a~multiple mode card that can work under various frequency bands 
and modulation schemes can switch from one technology to another in order to 
maintain communication. This procedure known as a vertical handover (VHO)
provides the 
benefit of utilizing the higher bandwidth and lower cost of wide local area networks 
as well as better mobility support and larger coverage of cellular networks. The 
authors investigate details of the VHO procedure from WLAN (Wireless Local
Area Network) to the 
3GPP Long Term Evolution (LTE). The VHO procedure includes~40~signaling 
messages, which are responsible for authorization and resource allocation in the 
target LTE network. The authors analyze the VHO sojourn time and its~95~percent 
quantile using a~multiphase queuing system with background traffic.}

\KWE{heterogeneous wireless network; cellular network; LTE; mobility; session 
setup procedure; connection reliability; connection availability; performance 
measure}




  \DOI{10.14357/19922264170409} 

%\vspace*{-12pt}

\Ack
\noindent
The publication was supported by the Ministry of Education and Science of the 
Russian Federation (project No.\,2.882.2017/4.6). 



\vspace*{12pt}

  \begin{multicols}{2}

\renewcommand{\bibname}{\protect\rmfamily References}
%\renewcommand{\bibname}{\large\protect\rm References}

{\small\frenchspacing
 {%\baselineskip=10.8pt
 \addcontentsline{toc}{section}{References}
 \begin{thebibliography}{99}
\bibitem{1-gre-1}
\Aue{Astely, D., E.~Dahlman, G.~Fodor, S.~Parkvall, and J.~Sachs.} 2013. LTE 
release~12 and beyond. \textit{IEEE Commun. Mag.} 51(7):154--160. 
doi: 10.1109/MCOM. 2013.6553692. 


\bibitem{3-gre-1} %2
\Aue{Gorbunova, A.\,V., I.\,S.~Zaryadov, S.\,I.~Matyushenko, K.\,E.~Samouylov, 
and S.\,Ya.~Shorgin.} 2015. Approksimatsiya vremeni otklika sistemy oblachnykh 
vychisleniy [The approximation of response time of a cloud computing system]. 
\textit{Informatika i~ee Primeneniya~--- Inform. Appl.} 9(3):32--38.

\bibitem{2-gre-1} %3
\Aue{Vikhrova, O.\,G., K.\,E.~Samouylov, E.\,S.~Sopin, and S.\,Ya.~Shorgin}. 2015. 
K~analizu pokazateley kachestva obsluzhivaniya v~sovremennykh besprovodnykh 
setyakh [On performance analysis of modern wireless networks]. \textit{Informatika 
i~ee Primeneniya~--- Inform. Appl.} 9(4):48--55.

\bibitem{4-gre-1}
\Aue{Gaidamaka, Yu.\,V., S.\,D.~Andreev, E.\,S.~Sopin, K.\,E.~Samouylov, and 
S.\,Ya.~Shorgin.} 2016. Analiz kharakteristik interferentsii v~modeli 
vzaimodeystviya ustroystv s~uchetom sredy rasprostraneniya signala [Interference 
analysis of the device-to-device communications model with regard to a~signal 
propagation environment]. \textit{Informatika i~ee Primeneniya~--- Inform. Appl.} 
10(4):2--10.
\bibitem{5-gre-1}
\Aue{Koucheryavy, A.\,E., М.\,А.~Makolkina, and R.\,V.~Kirichek.} 2016. 
Taktil'niy Internet. Seti svyazi so sverkhmalymi zaderzhkami [Tactile Internet. 
Ultra-low latency networks]. \textit{Elekrosvyaz'} 
[Telecomm. Radio Eng.] 1:44--46.
\bibitem{6-gre-1}
\Aue{Kellokoski, J., J.~Koskinen, R.~Nyrhinen, and 
T.~H$\ddot{\mbox{a}}$m$\ddot{\mbox{a}}$l$\ddot{\mbox{a}}$inen.} 2012. 
Efficient handovers for machine-to-machine communications between IEEE~802.11 
and 3GPP evolved packed core networks. \textit{IEEE  Conference (International) 
on Green Computing and Communications Proceedings}. \mbox{Besan{\!\ptb{\c{c}}}on.}  
722--725.
\bibitem{7-gre-1}
\Aue{Ahmed, L., L.~M.~Boulahia, and D.~Gaiti.} 2014. Enabling vertical 
handover decisions in heterogeneous wireless networks: A~state-of-the-art and 
a~classification. \textit{IEEE Commun. Surv. Tut.} 16(2):776--781.
\bibitem{8-gre-1}
\Aue{Bukhari, J., and N.~Akkari.} 2016. QoS based approach for LTE-WiFi 
handover. \textit{7th Conference (International) on Computer Science and 
Information Technology Proceedings}. Elsevier. 1--6.
\bibitem{9-gre-1}
\Aue{Gondim, P.\,R.\,L., and J.\,B.\,M.~Trineto.} 2012. DSMIP and PMIP for 
mobility management of heterogeneous access networks: Evaluation of 
authentication delay. \textit{IEEE Globecom Workshops Proceedings}. Anaheim. 
308--313.



\bibitem{11-gre-1} %10
\Aue{Do-Hyung, K., K.~Won-Tae, L.~Hwan-Gu, K.~Sun-Ja, and  
L.\,A.~Cheol-Hoon.} 2008. Performance evaluation of vertical hanover architecture 
with low latency handover. \textit{Conference (International) on Convergence and 
Hybrid Information Technology Proceedings}. Dusan. 66--69.


\columnbreak

\bibitem{10-gre-1} %11
\Aue{Tsagkaropoulos, M., I.~Politis, C.~Tselios, T.~Dagiuklas, and 
S.~Kotsopoulos}. 2011. Service continuity over intertechnology RATs. \textit{16th 
IEEE  Workshop (International) on Computer Aided Modeling and Design of 
Communication Links and Networks Proceedings}. Kyoto. 117--121.

\bibitem{12-gre-1}
\Aue{M$\acute{\mbox{a}}$rquez-Barja,~J., C.\,T.~Calafate, J.-C.~Cano, and 
P.~Manzoni.} 2011. An overview of vertical handover techniques: Algorithms, 
protocols and tools. \textit{Comput. Commun.} 34:985--997.
\bibitem{13-gre-1}
\Aue{Grebeshkov, A.\,Yu.} 2014. Otsenka tselesoobraznosti obrabotki zayavki dlya 
predostavleniya uslugi v~re\-kon\-fi\-gu\-ri\-ru\-emykh setyakh sleduyushchego pokoleniya 
[Estimation of processing feasibility for service provision application in 
reconfigurable Next Generation Networks]. \textit{\mbox{T-Comm}: Te\-le\-kom\-mu\-ni\-ka\-tsii 
i~transport} [\mbox{T-Comm}: Telecommunications and Transport ] 8:24--27. 
\bibitem{14-gre-1}
\Aue{Grebeshkov, A., E.~Zaripova, A.~Roslyakov, and K.~Samouylov}. Modelling 
of vertical handover from untrusted WLAN network to LTE. \textit{31st European 
Conference on Modelling and Simulation Proceedings}. 694--700.
\bibitem{15-gre-1}
\Aue{Lampropoulos, G., N.~Passas, L.~Mekaros, and A.~Kaloxylos.} 2005. 
Handover management architectures in integrated WLAN. \textit{IEEE 
Commun. Surv.  Tut.} 7(4):30--44.
\bibitem{16-gre-1}
3GPP TS 23.402. 2016. Technical specification 3rd Generation Partnership Project; 
Technical Specification Group Services and System Aspects; Architecture 
enhancements for non-3GPP accesses. Release~14.  
Available at: {\sf 
https://portal.3gpp.org/desktopmodules/\linebreak Specifications/SpecificationDetails.aspx?specificationId\linebreak =850} (accessed May~20, 2017).
\bibitem{17-gre-1}
3GPP TS 33.402. 2016. Technical specification 3rd Generation Partnership Project; 
Technical Specification Group Services and System Aspects; 3GPP System 
Architecture Evolution (SAE); Security aspects of non-3GPP accesses. Release~14. 
Available at: {\sf 
https://portal. 3gpp.org/desktopmodules/Specifications/Specification\linebreak Details.aspx?specificationId=2297}
 (accessed May~20, 2017).
\bibitem{18-gre-1}
\Aue{Gaidamaka, Yu., and E.~Zaripova} 2014. Session setup delay estimation 
methods for IMS-based IPTV services. 
\textit{Internet of things, smart spaces, and next generation networks
and systems}.
Eds.\ S.~Balandin, S.~Andreev, and Y.~Koucheryavy.
{Lecture notes in computer science ser.} Springer.
8638:408--418.
\bibitem{19-gre-1}
\Aue{Nikaein, N., and S.~Krco.} 2011. Latency for real-time machine-to-machine 
communication in LTE-based system architecture. \textit{17th European Wireless 
Conferense on Sustainable Wireless Technologies Proceedings.} Vienna, Austria: 
IEEE. 1--6.
\bibitem{20-gre-1}
\Aue{Cardona, N., J.\,F.~Monserrat, and J.~Cabrejas.} 2013. Enabling technologies 
for 3GPP LTE-advanced networks.\linebreak\vspace*{-11pt}

\pagebreak

\noindent
 \textit{LTE-Advanced and next generation 
wireless networks}. Eds.\
 G.~de la~Roche, A.\,A.~Glazunov, and B.~Allen.
Chichester, U.K.: John Wiley and Sons. 3--34.
\bibitem{21-gre-1}
\Aue{Prados-Garzon, J., J.\,J.~Ramos-Munoz, P.~Ameigeiras,  
P.~Andres-Maldonado, and J.\,M.~Lopez-Soler.} 2016. La-\linebreak\vspace*{-11pt}

\columnbreak

\noindent
tency evaluation of 
a~virtualized MME. \textit{Wireless Days Proceedings}. Toulouse, 
France: IEEE. 1--3.
\bibitem{22-gre-1}
\Aue{Granlund, D., P.~Holmlund, and \mbox{C.~{\ptb{\AA}}hlund}}. 2015. 
Opportunistic mobility support for resource constrained sensor devices in smart 
cities. \textit{Sensors} 15(3):5112--5135.
{\looseness=1

}
\end{thebibliography}

 }
 }

\end{multicols}

\vspace*{-6pt}

\hfill{\small\textit{Received May 31, 2017}}

%\vspace*{-10pt}

\Contr

\noindent
\textbf{Grebeshkov Alexander Yu.} (b.\ 1967)~--- Candidate of Science (PhD) in 
technology, senior scientist, Povolzhskiy State University of Telecommunications 
and Informatics, 23~Tolstoy Str., Samara 443010, Russian Federation;  
\mbox{grebeshkov-ay@psuti.ru}

\vspace*{3pt}

\noindent
\textbf{Gaidamaka Yuliya V.} (b.\ 1971)~--- Candidate of Science (PhD) in physics and 
mathematics, associate professor, Peoples' Friendship University of Russia (RUDN University), 
6~Miklukho-Maklaya Str., Moscow 117198, Russian Federation; senior scientist, Institute of 
Informatics Problems, Federal Research Center ``Computer Science and Control'' of the Russian 
Academy of Sciences, 44-2~Vavilov Str., Moscow 119333, Russian Federation; 
\mbox{gaydamaka\_yuv@rudn.university}

\vspace*{3pt}

\noindent
\textbf{Vikhrova Olga G.} (b.\ 1990)~--- PhD student, Peoples' Friendship University of Russia 
(RUDN University), 6~Miklukho-Maklaya Str., Moscow 117198, Russian Federation; 
\mbox{vikhrova\_og@rudn.university}

\vspace*{3pt}

\noindent
\textbf{Zaripova Elvira R.} (р.\ 1979)~--- Candidate of Science (PhD) in physics and 
mathematics, associate professor, Peoples' Friendship University of Russia (RUDN University), 
6~Miklukho-Maklaya Str., Moscow 117198, Russian Federation; 
\mbox{zaripova\_er@rudn.university}

\label{end\stat}


\renewcommand{\bibname}{\protect\rm Литература}  %9
\def\stat{naumov}

\def\tit{О МАРКОВСКИХ И~РАЦИОНАЛЬНЫХ ПОТОКАХ 
СЛУЧАЙНЫХ СОБЫТИЙ.~II$^*$} % Часть~2$^*$}

\def\titkol{О марковских и рациональных потоках случайных 
событий. II} %Часть 2}

\def\aut{В.\,А.~Наумов$^1$, К.\,Е.~Самуйлов$^2$}

\def\autkol{В.\,А.~Наумов, К.\,Е.~Самуйлов}

\titel{\tit}{\aut}{\autkol}{\titkol}

\index{Наумов В.\,А.}
\index{Самуйлов К.\,Е.}
\index{Naumov V.\,A.}
\index{Samouylov К.\,Е.}


{\renewcommand{\thefootnote}{\fnsymbol{footnote}} \footnotetext[1]
{Исследование выполнено при финансовой поддержке РФФИ в рамках научного проекта №\,19-17-50126.}}


\renewcommand{\thefootnote}{\arabic{footnote}}
\footnotetext[1]{Исследовательский институт инноваций, г.~Хельсинки, Финляндия, 
\mbox{valeriy.naumov@pfu.fi}}
\footnotetext[2]{Российский университет дружбы народов; Институт проблем информатики Федерального 
исследовательского центра <<Информатика и~управ\-ле\-ние>> Российской академии наук, \mbox{samouylov-ke@rudn.ru}}

%\vspace*{6pt}

  \Abst{Статья представляет собой вторую часть обзора, выполненного в рамках проекта 
РФФИ 
  №\,19-17-50126. Цель обзора~--- ознакомление заинтересованных читателей с основами 
теории марковских потоков событий для более подробного изучения и облегчения 
применения этих моделей на практике. В~первой части приведены свойства общих 
марковских потоков событий и показана их связь с марковскими аддитивными процессами и 
процессами марковского восстановления. Во второй части обзора рассмотрены важные для 
приложений частные случаи таких потоков~--- подклассы марковских потоков событий, а~именно:
 простые и групповые потоки однородных и неоднородных событий. Показано, 
как свойства марковских потоков событий связаны с мультипликативностью стационарных 
распределений марковских систем. Обсуждаются  
мат\-рич\-но-экс\-по\-нен\-ци\-аль\-ные распределения и рациональные потоки событий, 
расширяющие возможности марковских потоков для моделирования сложных систем, при 
этом сохраняющие удобство их анализа с помощью вычислительной техники.}
  
  \KW{марковские процессы; марковские аддитивные процессы; потоки без последействия; 
  МС-по\-то\-ки}
  
\DOI{10.14357/19922264200406} 
  
\vspace*{6pt}


\vskip 10pt plus 9pt minus 6pt

\thispagestyle{headings}

\begin{multicols}{2}

\label{st\stat}


\section{Введение}

Настоящий обзор, состоящий из двух частей, включает изложение основ 
теории марковских потоков и снабжен ссылками на большое число работ, 
посвященных марковским и~рациональным потокам событий. Он начался с 
рассмотрения в первой части случайных величин фазового типа, определения 
марковских потоков общего вида и их связи с~марковскими аддитивными 
процессами и процессами марковского восстановления. Во второй части 
обзора  перейдем к важным для приложений подклассам марковских потоков 
однородных и неоднородных событий в разд.~2, а~в~завершение в~разд.~3 
обсудим  
мат\-рич\-но-экс\-по\-нен\-ци\-аль\-ные распределения и~в~разд.~4 
рациональные потоки событий, которые расширяют возможности марковских 
потоков для моделирования сложных систем и~при этом сохраняют удобство 
их анализа. 

Как и в первой части обзора, далее в работе жирные строчные буквы 
обозначают векторы, а~жирные прописные буквы обозначают матрицы. 
Кроме того, используются следующие обозначения: 
$$
\delta(i,j)= \begin{cases}
1, &\mbox{если } i=j\,;\\
0 & \mbox{в~противном\ случае};
\end{cases}
$$
 у~вектора~$\mathbf{e}_i$ 
$i$-я координата равна единице, а остальные равны нулю; $\mathbf{I}\hm= \left[ 
\delta(i,j)\right]$~--- единичная матрица; $\mathbf{u}$~---  
век\-тор-стол\-бец из единиц; $\boldsymbol{\mathcal{N}}^K$~--- множество 
неотрицательных целочисленных векторов длины~$K$, 
$\boldsymbol{\mathcal{N}}^K _0\hm= \boldsymbol{\mathcal{N}}^K \backslash 
\{\mathbf{0}\}$. Для краткости вмес\-то <<наступило $n_1$ событий типа~1, 
$n_2$ событий ти-\linebreak па~2,~\ldots , $n_K$ событий типа~$K$>> будем писать 
<<наступило $\mathbf{n}$ событий>>, где $\mathbf{n}\hm= \left( n_1, n_2, 
\ldots , n_K\right)$.

\section{Важные для~приложений частные случаи марковских потоков 
событий}

\subsection{Простой марковский поток однородных событий}

  Рассмотрим некоторый поток случайных неоднородных событий и 
обозначим через $N_k(t)$ чис\-ло событий типа~$k$, наступивших за время~$t$, 
$\mathbf{N}(t)\hm= (N_1(t), N_2(t), \ldots, N_K(t))$. Поток случайных событий 
называется марковским, если для некоторого случайного процесса~$X(t)$ с 
конечным \mbox{множеством} состояний $\boldsymbol{\mathcal{X}}\hm= \{1,2,\ldots , 
L\}$ процесс $\xi(t)\hm= (X(t), \mathbf{N}(t))$ является марковским процессом, 
однородным во времени и по второй компоненте, т.\,е.\ если для любых~$t, 
h\hm>0$ справедливы равенства
  \begin{multline*}
  {\sf P}\left(X(h+t)=j, \mathbf{N}(h+t)=\mathbf{k}+\mathbf{n}\vert X(h)=i, \right.\\
\left.\mathbf{N}(h)=\mathbf{k}\right)=p_{\mathbf{n}}(i,j,t)\,,\enskip
  \mathbf{k}, \mathbf{n} \in \boldsymbol{\mathcal{N}}^K,\enskip i,j\in 
\boldsymbol{\mathcal{X}}\,.
  \end{multline*}
Матрицы вероятностей переходов $\mathbf{P}_{\mathbf{n}}(t)\hm= 
[p_{\mathbf{n}}(i,j,t)]$ однозначно определяются матрицами интенсивностей 
переходов $\mathbf{A}_{\mathbf{n}}\hm= \left[ a_{\mathbf{n}}(i,j)\right]$, 
$\mathbf{n}\hm\geq \mathbf{0}$, где
\begin{align*}
a_{\mathbf{0}}(i,j) &=\lim\limits_{t\to0} \fr{1}{t}\left( p_{\mathbf{0}}(i,j,t)-\delta(i,j)\right)\,,\enskip
 i,j\in  \boldsymbol{\mathcal{X}}\,;\\
a_{\mathbf{n}}(i,j) &=\lim\limits_{t\to0} \fr{1}{t}\, p_{\mathbf{n}}(i,j,t)\,,\enskip i,j\in 
\boldsymbol{\mathcal{X}}\,,\enskip \mathbf{n}\in \boldsymbol{\mathcal{N}}^K_0,
\end{align*}
при этом фазовый процесс~$X(t)$ является однородным марковским 
процессом с матрицей интенсивностей переходов $\mathbf{A}\hm= 
\sum\nolimits_{\mathbf{n}\in \boldsymbol{\mathcal{N}}^K} 
\mathbf{A}_{\mathbf{n}}$.
  
  В первой части обзора определен процесс марковского восстановления 
$(X_l,\boldsymbol{\sigma}_l, \tau_l)$, где $X_l\hm= X(t_l)$~--- состояния 
фазового процесса~$X(t)$ марковского потока в моменты после наступления\linebreak 
событий потока, $X(t)\hm\in \boldsymbol{\mathcal{X}} \hm= \{1,2,\ldots ,L\}$, 
$0\hm< t_1\hm< t_2
  <\cdots$~--- моменты наступления событий, также называемые 
вызывающими моментами; $\tau_l\hm= t_l\hm- t_{l-1}$~--- длины интервалов 
между \mbox{моментами} наступления событий; $\boldsymbol{\sigma}_l$~--- вектор, 
$\boldsymbol{\sigma}_l\hm= (\sigma_{l,1}, \ldots , \sigma_{l,K})$, 
в~котором~$\sigma_{l,k}$ есть размер группы событий типа~$k$, наступивших 
в~момент~$t_l$, $l\hm=1, 2, \ldots$ Матрицы $\mathbf{G}_{\mathbf{n}}(x)\hm= 
[G_{\mathbf{n}}(i,j,x)]$, описывающие связанный с марковским потоком 
процесс марковского восстановления $(X_l, \boldsymbol{\sigma}_l, \tau_l)$, и 
их преобразования Лап\-ла\-са--Стилть\-еса имеют следующий вид:

\noindent
  \begin{align}
  \mathbf{G}_{\mathbf{n}}(x)&=\int\limits_0^x \exp 
(z\mathbf{A}_0)\mathbf{A}_{\mathbf{n}}\,dz={}\notag\\
&\hspace*{-10mm}{}=\left( \exp 
(x\mathbf{A}_{\mathbf{0}}))-\mathbf{I}\right)\mathbf{A}_0^{-1} \mathbf{A}_{\mathbf{n}}\,,\ \mathbf{n}\in 
\boldsymbol{\mathcal{N}}_0^K\,;
  \label{e1-nau}\\
  \int\limits_0^x e^{-\nu x}d\mathbf{G}_{\mathbf{n}}(x)&= (\nu\mathbf{I}-
\mathbf{A}_{\mathbf{0}})^{-1}\mathbf{A}_{\mathbf{n}}\,,\ \mathbf{n}\in 
\boldsymbol{\mathcal{N}}_0^K\,.
  \label{e2-nau}
  \end{align}
Используя матрицы $\mathbf{G}_{\mathbf{n}}(x)$, можно найти совместное 
распределение числа~$\boldsymbol{\sigma}_l$ наступивших событий и 
длин~$\tau_l$ интервалов между вызывающими моментами 
\begin{multline}
F_{\mathbf{k}_1, \mathbf{k}_2, \ldots , \mathbf{k}_m} \left(x_1, x_2, \ldots , 
x_m\right)={}\\
{}={\sf P}\left(
\boldsymbol{\sigma}_l=\mathbf{k}_l\,, \tau_l<x_l\,, l=1,2,\ldots, m\right)={}\\
{}=\bm{\alpha}\mathbf{G}_{\mathbf{k}_1}(x_1) \mathbf{G}_{\mathbf{k}_2}(x_2)\cdots 
\mathbf{G}_{\mathbf{k}_m}(x_m)\mathbf{u}\,,
\label{e3-nau}
\end{multline}
а также плотность этого распределения

\columnbreak

\noindent
\begin{multline}
f_{\mathbf{k}_1, \mathbf{k}_2, \ldots , \mathbf{k}_m} (x_1, x_2, \ldots , 
x_m)={}\\
{}=\bm{\alpha}\exp \left( x_1\mathbf{A}_{\mathbf{0}}\right) 
\mathbf{A}_{\mathbf{k}_1}\exp \left( x_2\mathbf{A}_{\mathbf{0}}\right) 
\mathbf{A}_{\mathbf{k}_2}\cdots\\
\cdots \exp \left( x_m\mathbf{A}_{\mathbf{0}}\right) 
\mathbf{A}_{\mathbf{k}_m}\mathbf{u}\,,\quad
\mathbf{k}_1, \mathbf{k}_2, \ldots , \mathbf{k}_m\in 
\boldsymbol{\mathcal{N}}_0^K\,,\\
 x_0, x_1, \ldots , x_m>0\,,\enskip m=1,2,\ldots
\label{e4-nau}
\end{multline}

\vspace*{-6pt}

\noindent
где $\bm{\alpha}$~--- начальное распределение фазового про\-цесса.


  
  Простой марковский поток однородных событий~--- это марковский поток 
событий одного типа, причем в каждый вызывающий момент наступает ровно 
одно событие. Он характеризуется двумя мат\-ри\-ца\-ми интенсивностей 
переходов $\mathbf{S}\hm= \mathbf{A}_0$ и~$\mathbf{R}\hm= \mathbf{A}_1$, 
а~остальные матрицы~$\mathbf{A}_k$, $k\hm\geq 2$, для такого потока~--- 
нулевые. Первыми работами, посвященными простым марковским потокам 
однородных событий, стали~[1--5]. Их применение к~решению задач теории 
телетрафика рассматривается  
в~\cite{6-nau, 7-nau}. Поток вызывающих моментов любого марковского 
потока~--- это простой марковский поток, характеризуемый матрицами 
$\mathbf{S}\hm= \mathbf{A}\hm-\mathbf{R}$ и~$\mathbf{R}\hm= 
\sum\nolimits_{\mathbf{n}\in \boldsymbol{\mathcal{N}}_0^K} 
\mathbf{A}_{\mathbf{n}}$. К~простым марковским потокам относятся также 
процессы восстановления фазового типа~\cite{8-nau}. Для таких потоков ранг 
матрицы~$\mathbf{R}$ равен единице и~она имеет вид $\mathbf{R}\hm= 
\mathbf{sq}$, где $\mathbf{s}\hm= -\mathbf{Su}$. Верно и~обратное~\cite{7-nau}. 
В~англоязычной литературе простые марковские потоки называют 
Markovian arrival process и~используют для их обозначения сокращение МАР 
или MArP.
  
  Простой марковский поток однородных событий является 
полумарковским, поскольку последовательность $(X_l, \tau_l)$, $l\hm=1, 
2,\ldots,$~--- процесс марковского восстановления. Из~(1) и~(2) вытекают 
следующие формулы для полумарковской матрицы $\mathbf{G}(x)\hm= \left[ 
G(i,j,x)\right]$ процесса $(X_l,\tau_l)$ марковского восстановления с 
элементами 

\vspace*{3pt}

\noindent
  $$
  G(i,j,x)={\sf P} \left( X_l=j,\ \tau_l<x\vert X_{l-1}=i\right)
  $$
  
  \vspace*{-1pt}
  
  \noindent
 и для ее преобразования Лап\-ла\-са--Стилть\-еса:
 
 \vspace*{2pt}
 
 \noindent
\begin{equation}
\left.
\begin{array}{rl}
\mathbf{G}(x)&=\left( \exp (x\mathbf{S})-\mathbf{I}\right) \mathbf{S}^{-
1}\mathbf{R}\,;\\
\displaystyle\int\limits_0^x e^{-\nu x}d\mathbf{G}(x)&=(\nu\mathbf{I}-\mathbf{S})^{-1}\mathbf{R}\,.
\end{array}
\right\}
\label{e5-nau}
\end{equation}

\vspace*{-2pt}
  
  Из~(\ref{e4-nau}) вытекает следующее выражение для плотности функции 
распределения длин интервалов~$\tau_l$ между моментами наступления 
событий простого марковского потока однородных событий:

\vspace*{-8pt}

\noindent
  \begin{multline}
  f\left( x_1, x_2, \ldots, x_m\right)={}\\
  {}=\bm{\alpha}\exp \left(x_1\mathbf{S}\right) 
\mathbf{R}\exp \left( x_2\mathbf{S}\right)\mathbf{R}\cdots \exp \left( 
x_m\mathbf{S}\right) \mathbf{Ru}\,,\\
  x_0, x_1,\ldots , x_m>0\,,\enskip m=1,2,\ldots
  \label{e6-nau}
  \end{multline}
  
  \vspace*{-2pt}
  
  Поскольку простой марковский поток является полумарковским, при 
анализе систем массового обслуживания с такими поступающими потоками 
можно использовать результаты, полученные для систем с полумарковским 
входящим потоком,  
например~[9--12].
   
  В первом разделе обзора указано, что стационарные распределения 
$\mathbf{q}\hm=[q(i)]$ и $\mathbf{q}_{\mathbf{n}}\hm= [q_{\mathbf{n}}(i)]$, 
$\mathbf{n}\hm\in \boldsymbol{\mathcal{N}}_0^K$, вложенных цепей 
Маркова~$X_l$ и~$(X_l, \boldsymbol{\sigma}_l)$ связаны со стационарным 
распределением~$\mathbf{p}$ фазового процесса~$X(t)$ следующими 
равенствами:
\begin{multline*}
  \mathbf{q}=\fr{1}{\lambda}\,\mathbf{p}\boldsymbol{\Lambda}\,,\
  \mathbf{p}=-\lambda \mathbf{q}\mathbf{A}_0^{-1}\,,\ 
  \mathbf{q}=\sum\limits_{\mathbf{n}\in \boldsymbol{\mathcal{N}}_0^K} 
\mathbf{q}_{\mathbf{n}}\,,\\
 \mathbf{q}_{\mathbf{n}}=\fr{1}{\lambda}\,\mathbf{p}
  \mathbf{A}_{\mathbf{n}}\,,\enskip \mathbf{n}\in \boldsymbol{\mathcal{N}}_0^K\,.
  \end{multline*}

  Если вектор из единиц~$\mathbf{u}$ является правым собственным 
вектором каждой из матриц~$\mathbf{A}_{\mathbf{n}}$ и выполняются 
равенства 
  \begin{equation}
  \mathbf{A}_{\mathbf{n}}\mathbf{u}=\lambda_{\mathbf{u}}\mathbf{u}\,,\quad
  \mathbf{n}\in \boldsymbol{\mathcal{N}}_0^K\,,
  \label{e7-nau}
  \end{equation}
то из~(\ref{e3-nau}) следует, что при любом начальном 
распределении~$\mathbf{s}$ марковский поток будет стационарным потоком 
без последействия. Аналогично, если вектор стационарных 
вероятностей~$\mathbf{p}$ является левым собственным вектором 
матриц~$\mathbf{A}_{\mathbf{n}}$ и выполняются равенства 
\begin{equation}
\mathbf{pA}_{\mathbf{n}}=\lambda_{\mathbf{n}}\mathbf{p}\,,\quad
\mathbf{n}\in \boldsymbol{\mathcal{N}}_0^K\,.
\label{e8-nau}
\end{equation}
    
Условия~(\ref{e7-nau}) и~(\ref{e8-nau}), достаточные для того чтобы 
марковский поток был пуассоновским, для простого марковского потока 
приобретают вид $\mathbf{Ru}\hm= \lambda\mathbf{u}$ и~$\mathbf{pR}\hm= 
\lambda\mathbf{p}$ соответственно, где $\lambda\hm= \mathbf{pRu}$~--- 
интенсивность потока. Проверка необходимых и~достаточных условий 
пуассоновости простого марковского потока более сложна и~требует знания 
собственных векторов матрицы~$\mathbf{S}$~\cite{13-nau}.
  
  Считающий процесс $N(t)$ стационарной версии простого марковского 
потока является асимптотически нормальным с~математическим ожиданием 
${\sf M}(t)\hm=\lambda t$ и дисперсией
  $$
  {\sf D}(t)=\left( 2\mathbf{d}_1\mathbf{s}-\lambda\right) t +2\left( 
\mathbf{d}_2\mathbf{s}-\lambda\right) +o(1)\,,
  $$
где векторы-столб\-цы~$\mathbf{d}_1$ и~$\mathbf{d}_2$~--- единственные 
решения систем линейных уравнений~\cite{2-nau}:
\begin{alignat*}{2}
\mathbf{d}_1\mathbf{A} &=\mathbf{p}(\lambda \mathbf{I}-\mathbf{R})\,,&\quad
\mathbf{d}_1\mathbf{u}&=1\,;\\
\mathbf{d}_2\mathbf{A}&=\mathbf{d}_1 -\mathbf{p}\,, &\quad
\mathbf{d}_2\mathbf{u}&=1\,.
\end{alignat*}
    
\subsection{Простой марковский поток неоднородных событий}

  Простой марковский поток неоднородных событий~--- это марковский 
поток событий нескольких типов, в каждый вызывающий момент которого 
наступает ровно одно событие. Такой поток характеризуется $K\hm+1$ 
матрицами интенсивностей переходов $\mathbf{S}\hm= \mathbf{A}_0$ и 
$\mathbf{R}_k\hm= \mathbf{A}_{\mathbf{e}_k}$, $k\hm=1,2,\ldots ,K$, 
а~остальные матрицы~$\mathbf{A}_{\mathbf{n}}$~--- нулевые. При этом поток 
событий одного типа, например типа~$i$, является простым марковским 
потоком однородных событий, описываемым матрицами 
$\mathbf{S}_i\hm=\mathbf{A}\hm- \mathbf{A}_{\mathbf{e}_i}$ 
и~$\mathbf{R}_i$. Первыми работами, посвященными прос\-тым марковским 
потокам неоднородных событий, считаются~[14--16]. В~англоязычной 
литературе такой поток называют Markovian Arrival Process with marked arrivals 
и~используют для его обозначения сокращение ММАР.  
Из~(\ref{e5-nau}) вытекает следующее выражение для плотности совместного 
распределения ${\sf P}(\omega_l=k_l,\tau_l<x_l, l\hm=1,2,\ldots ,m)$ типов 
$\omega_l$ событий, наступивших в~момент~$t_l$, и~длин~$\tau_l$ интервалов 
между вызывающими моментами: 
  \begin{multline}
  f_{{k}_1, {k}_2, \ldots , {k}_m}\left( x_1, x_2, \ldots , x_m\right)={}\\
  {}=\bm{\alpha}\exp \left( x_1\mathbf{S}\right)\mathbf{R}_{k_1}\exp\left( 
x_2\mathbf{S}\right) \mathbf{R}_{k_2}\cdots\\
\cdots \exp \left( x_m\mathbf{S}\right) 
\mathbf{R}_{k_m}\mathbf{u}\,,\quad
 1\leq k_1, k_2, \ldots , k_m\leq K\,,\\
x_0, x_1, \ldots , x_m>0\,,\quad  m=1,2,\ldots
 \label{e9-nau}
\end{multline}

\subsection{Марковский поток групп однородных событий}

  Марковский поток групп однородных событий~--- это марковский поток 
событий одного типа, в каждый вызывающий момент которого \mbox{может} 
наступить несколько событий. Такие марковские потоки впервые 
исследовались в~\cite{8-nau, 17-nau, 18-nau}, а их описание с помощью 
матриц~$\mathbf{A}_{\mathbf{n}}$ впервые появилось в~\cite{19-nau}. 
В~англоязычной литературе такой поток сейчас называют batch Markovian 
arrival process и используют для его обозначения сокращение BMAP. 
В~\cite{20-nau} получены формулы и асимптотики для первых двух моментов 
считающего процесса~$N(t)$, а~в~\cite{21-nau}~--- для старших моментов~$N(t)$.

\section{Матрично-экспоненциальные распределения}

  Функция распределения $F(t)$ неотрицательной случайной величины 
называется мат\-рич\-но-экс\-по\-нен\-ци\-аль\-ной, если $F(0)\hm<1$ и она 
представима в~виде 
  \begin{equation}
  F(t)=1-\mathbf{q}\exp (t\mathbf{S})\mathbf{u}
  \label{e10-nau}
  \end{equation}
с некоторым вектором~$\mathbf{q}$ и матрицей~$\mathbf{S}$, име\-ющей 
собственные числа лишь с отрицательными действительными частями. Для 
того чтобы функция распределения~$F(t)$ неотрицательной случайной 
величины была  
мат\-рич\-но-экс\-по\-нен\-ци\-аль\-ной, необходимо и достаточно, чтобы она 
имела рациональное преобразование 
Лап\-ла\-са--Стилть\-еса $\tilde{F}(\nu)$. Минимальный порядок 
матрицы~$\mathbf{S}$  
в~мат\-рич\-но-экс\-по\-нен\-ци\-аль\-ном представлении~(\ref{e10-nau}) равен 
чис\-лу полюсов функции $\tilde{F}(\nu)$ с учетом их кратности. Представление 
с~матрицей~$\mathbf{S}$ минимального порядка называется минимальным. 

  В некоторых работах по мат\-рич\-но-экс\-по\-нен\-ци\-аль\-ным функциям  
распределения~\cite{22-nau, 23-nau, 24-nau}, а~также в книгах~\cite{25-nau, 26-nau}, 
чтобы подчеркнуть аналогию с экспоненциальными функциями 
\mbox{распределения},  
вмес\-то~(\ref{e10-nau}) использовалось пред\-став\-ле\-ние $F(t)\hm= 1\hm - 
\mathbf{q}\exp (-t\mathbf{B})\mathbf{u}$ со знаком минус перед~$t$ 
и~мат\-ри\-цей~$\mathbf{B}$, име\-ющей собственные чис\-ла с~положительными 
действительными частями. В~настоящее\linebreak время используются только 
представления вида~(\ref{e10-nau}). Иногда допускается, что 
вектор~$\mathbf{u}$ в~(\ref{e10-nau}) может быть любым, а~не состоящим из 
единиц, как в~рас\-смат\-ри\-ва\-емом случае. Однако в~\cite{24-nau, 27-nau} 
было показано, что всегда можно подобрать мат\-рич\-но-экс\-по\-нен\-ци\-аль\-ное 
пред\-став\-ле\-ние с~$\mathbf{u}\hm=(1,1,\ldots , 1)$. 
  
  Идея матрично-экс\-по\-нен\-ци\-аль\-ных функций распределения восходит 
к работе~\cite{28-nau}, в которой показано, что рациональные преобразования  
Лап\-ла\-са--Стилть\-еса неотрицательных функций распределения 
представимы в виде:
  $$
  \tilde{F}(s)=p_0+\sum\limits^L_{l=1} q_0\cdots q_{l-1} p_l \prod\limits^l_{i=1} 
\fr{\lambda_i}{\lambda_{i}+s}\,,
  $$
где $p_i+q_i\hm=1$, $i\hm=1, \ldots , L$, $p_L\hm=1$, и~$-\lambda_i$, $i\hm=1, 
\ldots , L$,~--- полюсы~$\tilde{F}(s)$. Такое представление можно записать в 
мат\-рич\-но-экс\-по\-нен\-ци\-аль\-ном виде~(\ref{e10-nau}), полагая 
\begin{align*}
\mathbf{q}&=(1,0,\ldots ,0)\,;\\
\mathbf{S}&=\begin{bmatrix}
-\lambda_1&q_1\lambda_1&0&\cdots&0\\
0&-\lambda_2&q_2\lambda_2&\ddots &\vdots\\
0&0&\ddots& \ddots& 0\\
\vdots& \ddots& \ddots& -\lambda_{L-1}&q_{L-1}\lambda_{L-1}\\
0&\cdots & 0&0&-\lambda_L
\end{bmatrix}\,,
\end{align*}
%
  при этом элементы матрицы~$\mathbf{S}$ могут быть комплексными. 
В~\cite{22-nau} показано, что вектор~$\mathbf{q}$ и~мат\-ри\-ца~$\mathbf{S}$  
в~мат\-рич\-но-экс\-по\-нен\-ци\-аль\-ном  
пред\-ставлении~(\ref{e10-nau}) всегда могут быть выбраны действительными. 
  
  Из~(\ref{e10-nau}) вытекают формулы для начальных моментов
  \begin{equation*}
  \int\limits_0^\infty t^n dF(t)=n! \mathbf{q}(-\mathbf{S})^{-n}\mathbf{u}\,,\enskip 
n=1,2,\ldots
  %\label{e11-nau}
  \end{equation*}
и для преобразования Лап\-ла\-са--Стилть\-еса функции распределения~$F(t)$ 
\begin{multline*}
\tilde{F}(\nu)=\int\limits_0^\infty e^{-\nu t}dF(t)={}\\
{}=1-
\mathbf{q}\mathbf{u}+\mathbf{q}(\nu\mathbf{I}-\mathbf{S})^{-1} \mathbf{s}=1-
\nu\mathbf{q}(\nu\mathbf{I}-\mathbf{S})^{-1}\mathbf{u}\,,
%\label{e12-nau}
\end{multline*}
где $\mathbf{s}=-\mathbf{Su}$. Кроме того,  
мат\-рич\-но-экс\-по\-нен\-ци\-аль\-ные функции распределения обладают 
сле\-ду\-ющи\-ми свойствами~\cite{24-nau}.
\begin{enumerate}[1.]
\item Пусть $F_i(t)=1\hm- \mathbf{q}_i\exp(t\mathbf{S}_i) \mathbf{u}$, 
$i\hm=1,2$,~--- мат\-рич\-но-экс\-по\-нен\-ци\-аль\-ные функции 
распределения и $p_1\hm+p_2\hm=1$. Тогда
\begin{align*}
p_1F_1(t)+p_2F_2(t)&={}\\
&\hspace*{-15mm}{}=1-(p_1\mathbf{q}_1, p_2\mathbf{q}_2)\exp \left( 
t\begin{bmatrix} \mathbf{S}_1 & \mathbf{0}\\
\mathbf{0}& \mathbf{S}_2\end{bmatrix}
\right) \mathbf{u}\,; %\label{e13-naum}
\\
\left( F_1*F_2\right) (t) &={}\\
&\hspace*{-25mm}{}= 1-\left(\mathbf{q}_1,F_1(0) \mathbf{q}_2\right) \exp 
\left( t \begin{bmatrix}
\mathbf{S}_1 & -\mathbf{S}_1\mathbf{uq}_2\\
\mathbf{0} & \mathbf{S}_2\end{bmatrix} \right) \mathbf{u}\,.
%\label{e14-nau}
\end{align*}
\item Пусть $\tau$ и~$\gamma$~--- независимые неотрицательные случайные 
величины с функциями распределения~$F(t)$ и~$G(t)$ соответственно, 
причем~$F(t)$ имеет  
мат\-рич\-но-экс\-по\-нен\-ци\-аль\-ное представление~(\ref{e10-nau}). Тогда 
функция распределения~$H(t)$ случайной величины  $(\tau\hm-\gamma)^+$ 
имеет  
мат\-рич\-но-экс\-по\-нен\-ци\-аль\-ное пред\-став\-ление 
\begin{equation*}
H(t)=1-\mathbf{qU}\exp (t\mathbf{S})\mathbf{u}\,,
%\label{e15-nau}
\end{equation*}
где 
\begin{equation}
\mathbf{U}=\int\limits_0^\infty e^{t\mathbf{S}}dG(t)\,.
\label{e16-nau}
\end{equation}

\item Пусть $F(t)$ имеет мат\-рич\-но-экс\-по\-нен\-ци\-аль\-ное  
представление~(\ref{e10-nau}), а~у~квад\-рат\-ной мат\-ри\-цы~$\mathbf{V}$ 
все собственные числа имеют неотрицательные вещественные части. Тогда
\begin{multline*}
\int\limits_0^\infty e^{-t\mathbf{V}}dF(t)=(1-\mathbf{qu}) \mathbf{I}+
(\mathbf{q}\otimes \mathbf{I})\boldsymbol{\Psi}(\mathbf{Su}\otimes 
\mathbf{I})={}\\
{}=\mathbf{I}-(\mathbf{q}\otimes 
\mathbf{I})\boldsymbol{\Psi}(\mathbf{u}\otimes \mathbf{V})\,,
%\label{e17-nau}
\end{multline*}
где $\boldsymbol{\Psi}=(\mathbf{I}\otimes \mathbf{V}\hm- \mathbf{S}\otimes 
\mathbf{I})^{-1}$. 
  \end{enumerate}
  
  Последнее свойство можно использовать для вычисления 
матриц~$\mathbf{U}$ в~(\ref{e16-nau}) для мат\-рич\-но-экс\-по\-нен\-ци\-аль\-ных 
функций распределения~$G(t)$.
  
  Ясно, что функции распределения фазового типа являются  
мат\-рич\-но-экс\-по\-нен\-ци\-аль\-ны\-ми. Однако их  
мат\-рич\-но-экс\-по\-нен\-ци\-аль\-ные представления 
  \begin{multline*}
  F(t)-1-\mathbf{q}\exp (t\mathbf{S})\mathbf{u}\,,\enskip
  %\label{e18-nau}
    F(0)=1-\mathbf{qu}\,,\\
     \fr{d}{dt}\,F(t)= \mathbf{q}\exp 
(t\mathbf{S})\mathbf{s}\,,\ t>0\,,
  \end{multline*}
с ограничениями
\begin{equation}
\hspace*{-2mm}\left.
\begin{array}{rlrlrl}
\!\!\displaystyle 0<\sum\limits_{j\in \boldsymbol{\mathcal{X}}} q(j)&\leq 1\,,&\ q(i)&\geq0\,,&\ i&\in 
\boldsymbol{\mathcal{X}};
\\[9pt]
\!\!\displaystyle \sum\limits_{j\in \boldsymbol{\mathcal{X}}} \!\!s(i,j)&\leq 0\,,&\ s(i,j)&\geq 0\,,&\ 
i&\not= j\,,\ i, j\in \boldsymbol{\mathcal{X}},
\end{array}\!
\right\}\!\!\!\!
\label{e19-nau}
\end{equation}
где $\mathbf{S}=[s(i,j)]$, следует отличать от мат\-рич\-но-экс\-по\-нен\-ци\-аль\-ных 
представлений этих же функций, но без ограничений~(\ref{e19-nau}). 
Порядок  
мат\-рич\-но-экс\-по\-нен\-ци\-аль\-но\-го представления, удовлетворяющего 
ограничениям~(\ref{e19-nau}), будем называть числом этапов этого 
представления, а~порядок мат\-рич\-но-экс\-по\-нен\-ци\-аль\-но\-го 
представления, не удовлетворяющего этим ограничениям, следуя~\cite{28-nau}, 
будем называть\linebreak
 числом \textit{фиктивных} этапов. Необходимые и 
достаточные условия того, чтобы для функции распределения 
с~рациональным преобразованием Лап\-ла\-са--Стилть\-еса существовало 
представление, \mbox{удовлетворяющее} ограничениям~(\ref{e19-nau}), получены 
в~\cite{29-nau}. Для этого надо, чтобы (а)~функция распределения имела 
непрерывную положительную плотность на правой полуоси и~(б)~ее 
преобразование Лап\-ла\-са--Стилть\-еса имело единственный полюс 
с~максимальной вещественной частью. 

\section{Рациональные потоки событий}

  Рациональный поток групп неоднородных событий 
$(t_l,\boldsymbol{\sigma}_l)$, $l\hm=1,2,\ldots$, можно определить как поток, 
для которого совместное распределение чис\-ла~$\boldsymbol{\sigma}_l$ 
наступивших событий и~длин~$\tau_l$ интервалов между моментами~$t_l$ 
наступления событий дается формулами~(\ref{e1-nau}) и~(\ref{e3-nau}) 
с~матрицами~$\mathbf{A}_{\mathbf{n}}$, $\mathbf{n}\hm\in 
\boldsymbol{\mathcal{N}}^K $, обладающими следующими свойствами:
  \begin{enumerate}[(1)]
\item действительные части собственных чисел мат\-ри\-цы~$\mathbf{A}_{\mathbf{0}}$ 
отрицательны;
\item действительные части собственных чисел мат\-ри\-цы 
$\mathbf{A}\hm= \sum\nolimits_{\mathbf{n}\in 
\boldsymbol{\mathcal{N}}^K} \mathbf{A}_{\mathbf{n}}$ 
неположительны;
\item $\mathbf{Au}=\mathbf{0}$.
\end{enumerate}
  
  Для стационарных версий рациональных потоков дополнительно требуется, 
чтобы начальный вектор~$\bm{\alpha}$ совпадал с решением~$\mathbf{p}$ 
системы линейных уравнений $\mathbf{pA}\hm=0$, $\mathbf{pu}\hm=1$.
  
  Простой рациональный поток однородных событий, также называемый  
мат\-рич\-но-экс\-по\-нен\-циальным потоком~\cite{30-nau},~--- это поток 
событий одного типа, в каждый вызывающий момент которого наступает 
ровно одно событие и для которого плотность совместного распределения 
длин~$\tau_l$ интервалов между моментами наступления событий дается 
формулой~(\ref{e6-nau}) с матрицами~$\mathbf{S}$ и~$\mathbf{R}$, 
обладающими следующими свойствами~\cite{31-nau}:
\begin{itemize}
\item[(а)] вещественные части собственных чисел матрицы~$\mathbf{S}$ 
отрицательны;
\item[(б)] вещественные части собственных чисел матрицы 
$\mathbf{S}\hm+\mathbf{R}$ неположительны; 
\item[(в)] $(\mathbf{S}+\mathbf{R})\mathbf{u}=\mathbf{0}$. 
  \end{itemize}
  
  Примерами простых рациональных потоков однородных событий могут 
служить полумарковские потоки~\cite{22-nau} и процессы 
восстановления~\cite{27-nau}  
с~мат\-рич\-но-экс\-по\-нен\-ци\-аль\-ны\-ми функциями распределения длин 
интервалов между наступлениями событий.
  
  Рациональный поток неоднородных событий~--- это поток событий 
нескольких типов, в каждый вызывающий момент которого наступает ровно 
одно событие. Для такого потока совместное распределение типов 
наступивших событий~$\omega_l$ и длин~$\tau_l$ интервалов между 
моментами наступления событий дается формулой~(\ref{e9-nau}), а на 
матрицы~$\mathbf{S}$ 
и~$\mathbf{R}\hm=\mathbf{R}_1\hm+\mathbf{R}_2+\cdots  + \mathbf{R}_K$ 
накладываются перечисленные выше ограничения~(a)--(в)~\cite{32-nau}. 

\section{Заключение}

  Метод этапов Эрланга~\cite{33-nau} более 100~лет применяется при анализе 
стохастических систем. К~его широкому распространению привело открытие  
мат\-рич\-но-экс\-по\-нен\-ци\-аль\-но\-го представления для функций 
распределения фазового типа~\cite{34-nau} и моделей марковских потоков 
событий~\cite{1-nau, 17-nau}. Эти модели хорошо подходят для анализа 
стохастических систем с~по\-мощью вычислительной техники, 
приспособленной к обработке векторов и матриц, что привело к развитию 
специальных матричных методов анализа стохастических систем.
  
  Метод фиктивных этапов, предложенный в~\cite{28-nau}, позволил 
распространить метод Эрланга на любые распределения с рациональным 
преобразованием 
  Лап\-ла\-са--Стилть\-еса. Использование мат\-рич\-но-экс\-по\-нен\-ци\-аль\-ных 
  представлений для функций распределения~\cite{22-nau, 23-nau, 25-nau} 
  и~потоков случайных событий~\cite{31-nau} с произвольными рациональными 
преобразованиями 
  Лап\-ла\-са--Стилть\-еса упрощает применение метода фиктивных этапов. 
Формальное применение метода фиктивных этапов приводит\linebreak 
к~решению, 
в~котором вероятности, со\-от\-вет\-ст\-ву\-ющие фиктивным этапам, могут оказаться 
отрицательными, б$\acute{\mbox{о}}$льшими единицы или даже 
комплекс\-ны\-ми. Однако вероятности, соответствующие \mbox{не\-фик\-тив\-ным} 
состояниям, будут неотрицательными числами, не превосходящими единицы. 
Существуют различные интерпретации понятий отрицательных вероятностей 
и интенсивностей \mbox{переходов} \cite{35-nau, 36-nau, 37-nau, 38-nau}. Более 
детально ознакомиться с~{марковским} и~рациональным потоками событий, 
а~также с~матричными методами анализа стохастических систем можно 
 в~обзорах~\cite{39-nau, 40-nau, 41-nau, 42-nau, 43-nau, 44-nau}  
и~монографиях~[18, 25, 26, 45--57]. 

{\small\frenchspacing
 {%\baselineskip=10.8pt
 %\addcontentsline{toc}{section}{References}
 \begin{thebibliography}{99}

\bibitem{1-nau}
\Au{Наумов В.\,А.} О~независимой работе подсистем сложной системы~// Тр.~III 
Всесоюзной  
шко\-лы-се\-ми\-на\-ра по теории массового обслуживания.~--- 
М.: МГУ, 1976. №\,2. С.~169--177.
\bibitem{2-nau}
\Au{Бочаров П.\,П., Наумов В.\,А.} Анализ гиперэкспоненциальной двухфазной системы с 
ограниченным накопителем~// Информационные сети и их структура.~--- М.: Наука, 
1976.  
С.~168--180.
\bibitem{3-nau}
\Au{Наумов В.\,А.} Об обслуженной и избыточной нагрузках полнодоступного пучка с 
ограниченной очередью~// Численные методы решения задач математической физики и 
теории систем.~--- М.: УДН, 1977. С.~51--55.
\bibitem{4-nau}
\Au{Наумов В.\,А.} Исследование некоторых многофазных систем массового 
обслуживания: Дис.\ \ldots\ канд. физ.-мат. наук.~--- М.: УДН, 1978.  98~с.
\bibitem{5-nau}
\Au{Lucantoni D.\,M., Meier-Hellstern~K., Neuts M.\,F.} A~single-server queue with server 
vacations and a class of non-renewal arrival processes~// Adv. Appl. Probab., 1990. 
Vol.~22. Iss.~3. P.~676--705.
\bibitem{6-nau}
\Au{Башарин Г.\,П., Кокотушкин~В.\,А., Наумов~В.\,А.} О~методе эквивалентных замен 
расчета фрагментов сетей связи для ЦВМ~// Изв. АН \mbox{СССР}. Техническая кибернетика, 1979. №\,6. С.~92--99.
\bibitem{7-nau}
\Au{Basharin G., Naumov V.} Simple matrix description of peaked and smooth traffic and 
its applications~// 3rd ITC Specialist Seminar on Fundamentals of Teletraffic Theory.~--- M.: 
VINITI, 1984. P.~38--44. 
\bibitem{8-nau}
\Au{Neuts M.\,F.} Renewal processes of phase type~// Nav. Res. Logist.~Q., 1978. 
Vol.~25. Iss.~3. P.~445--454.
\bibitem{9-nau}
\Au{Cinlar E.} Queues with semi-Markovian arrivals~// J.~Appl. Probab., 1967. Vol.~4. Iss.~2.  
P.~365--379.
\bibitem{10-nau}
\Au{Franken P.} Erlangsche Formeln f$\ddot{\mbox{u}}$r semimarkowschen Eingang // 
Elektronische Informationsverarbeitung Kybernetik, 1968. Vol.~4. Iss.~3. P.~197--204.
\bibitem{11-nau}
\Au{Franken P., Kerstan~J.} Bedienungssysteme mit unendlich vielen Bedienungsapparaten~// 
Operationsforschung Mathematische Statistik.~--- Berlin: Akademie-Verlag, 1968. Vol.~I. 
P.~67--76.
\bibitem{12-nau}
\Au{Neuts M.\,F., Chen~S.-Z.} The infinite server queue with semi-Markovian arrivals and negative 
exponential services~// J.~Appl. Probab., 1972. Vol.~9. Iss.~1. P.~178--184.
\bibitem{13-nau}
\Au{Bean N.\,G., Green D.\,A., Taylor~P.\,G.} When is a MAP poisson?~// 2nd Australia--Japan 
Workshop on Stochastic Models in Engineering,
Technology 
and Management Proceedings~/ Eds. J.~Wilson, D.\,N.\,P.~Murthy, S.~Osaki.~--- 
Brisbane: Technology Management Center, University of Queensland, 1996. P.~34--43.
\bibitem{14-nau}
\Au{Наумов В.\,А.} Матричный аналог формулы Эрланга~// Модели распределения 
информации и методы их анализа.~--- М.: ВИНИТИ, 1988. C.~39--43.
\bibitem{15-nau}
\Au{He Q.-M.} Queues with marked customers~// Adv. Appl. Probab., 1996. Vol.~28. 
Iss.~2. P.~567--587.
\bibitem{16-nau}
\Au{He Q.-M., Neuts M.\,F.} Markov chains with marked transitions~// Stoch. Proc. 
Appl., 1998. Vol.~74. P.~37--52.
\bibitem{17-nau}
\Au{Neuts M.\,F.} A versatile Markovian point process.~--- Newark, DE: 
University of Delaware, Department of Statistics and Computer Science, 1977.
 Technical Report 77/13. 29~p.
\bibitem{18-nau}
\Au{Neuts M.\,F.} Structured stochastic matrices of $M/G/1$ type and their applications.~--- New 
York, NY, USA: Marcel Dekker, 1989. 512~p.
\bibitem{19-nau}
\Au{Lucantoni D.\,M.} New results on the single server queue with a batch Markovian arrival 
process~// Communications Statistics. Stochastic Models, 1991. Vol.~7. Iss.~1. P.~1--46. 
\bibitem{20-nau}
\Au{Narayana S., Neuts M.\,F.} The first two moment matrices of the counts for the Markovian 
arrival processes~// Communications Statistics. Stochastic Models, 1992. Vol.~8. Iss.~3. P.~459--477. 
\bibitem{21-nau}
\Au{Nielsen B.\,F., Nilsson L.\,A.\,F., Thygesen~U.\,H., Beyer~J.\,E.} Higher order moments and 
conditional asymptotics of the batch Markovian arrival process~// Stoch. Models, 2007. Vol.~23. 
Iss.~1. P.~1--26.
\bibitem{22-nau}
\Au{Бочаров П.\,П., Наумов В.\,А.} O~некоторых системах массового обслуживания 
конечной емкости~// Проб\-ле\-мы передачи информации, 1977. Т.~13. №\,4. С.~96--104.
\bibitem{23-nau}
\Au{Наумов В.\,А.} Об однолинейной системе с ограниченным накопителем и заявками 
нескольких видов~// Модели систем распределения информации и их анализ.~--- М.: 
Наука, 1982. C.~77--82.
\bibitem{24-nau}
\Au{Наумов В.\,А.} О~функциях распределения с рациональным преобразованием  
Лап\-ла\-са--Стилть\-еса~// Анализ информационно-вычислительных систем.~--- М.: 
УДН, 1986. С.~47--56.
\bibitem{25-nau}
\Au{Бочаров П.\,П., Печинкин~А.\,В.} Теория массового обслуживания.~--- М.: РУДН, 
1995. 528~с.
\bibitem{26-nau}
\Au{Bocharov P.\,P., D'Apice~C., Pechinkin~A.\,V., Salerno~S.} Queueing theory.~--- Utrecht--Boston: 
VSP, 2004. 446~p.
\bibitem{27-nau}
\Au{Asmussen S., Bladt M.} Renewal theory and queueing algorithms for matrix-exponential 
distributions~// Matrix-analytic methods in stochastic models~/
Eds. A.\,S.~Alfa, S.~Chakravarty.~--- New York, NY, USA: Marcel 
Dekker, 1996. P.~313--341.
\bibitem{28-nau}
\Au{Cox D.\,R.} A use of complex probabilities in the theory of stochastic processes~// Math. 
Proc. Cambridge, 1955. Vol.~51. Iss.~2. P.~313--319. 
\bibitem{29-nau}
\Au{O'Cinneide C.\,A.} Characterization of phase-type distributions~// Communications Statistics. 
Stochastic Models, 1990. Vol.~6. Iss.~1. P.~1--57.
\bibitem{30-nau}
\Au{Bodrog L., Horv$\acute{\mbox{a}}$th~A., Telek~M.} On the properties of moments of matrix 
exponential distributions and matrix exponential processes~// Dagstuhl Seminar Proceedings, 2008. 
Vol.~07461. Paper~1394.
\bibitem{31-nau}
\Au{Asmussen S., Bladt M.} Point processes with finite-dimensional conditional probabilities~// 
Stoch. Proc. \mbox{Appl.}, 1999. Vol.~82. Iss.~1. P.~127--142.
\bibitem{32-nau}
\Au{Horvath G., Telek M.} Acceptance-rejection methods for generating random variables from 
matrix exponential distribution and rational arrival processes~// Matrix-analytic methods in stochastic 
models~/ Eds. G.~Latouche, V.~Ramaswami, J.~Sethuraman, \textit{et al.}~--- 
New York, NY, USA: Springer, 2012. P.~123--144.
\bibitem{33-nau}
\Au{Erlang A.\,K.} \mbox{L{\!\ptb{\o}}sning} af nogle Problemer fra Sandsynlighedsregningen af 
Betydning for de automatiske Telefoncentraler~// Elektroteknikeren, 1917. Iss.~13. P.~5--13.
\bibitem{34-nau}
\Au{Neuts M.\,F.} Probability distribution of phase type~// Liber Amicorum Professor Emeritus 
H.~Florin.~--- Ottignies-Louvain-la-Neuve, Belgium: University of Louvain, Department of Mathematics, 
1975. P.~173--206.
\bibitem{35-nau}
\Au{Bartlett M.\,S.} Negative probability~// Math. Proc. Cambridge, 1945. Vol.~41. Iss.~1. P.~71--73.
\bibitem{36-nau}
\Au{Cox D.\,R.} The analysis of non-Markovian stochastic processes by the inclusion of 
supplementary variables~// Math. Proc. Cambridge, 1955. Vol.~51. Iss.~3. P.~433--441. 
\bibitem{37-nau}
\Au{Bladt M., Neuts M.\,F.} Matrix-exponential distributions: Calculus and interpretations via flows~// 
Stoch. Models, 2003. Vol.~19. Iss.~1. P.~113--124.
\bibitem{38-nau}
\Au{Khrennikov A.} Interpretations of probability.~--- 2nd ed.~--- Berlin: Walter de Gruyter, 2009. 
237~p.
\bibitem{39-nau}
\Au{Наумов В.\,А.} Марковские модели потоков требований~// Системы массового 
обслуживания и информатика.~--- М.: УДН, 1987. С.~67--73.
\bibitem{40-nau}
\Au{Asmussen S.} Matrix-analytic models and their analysis~// Scand. J.~Stat., 2000. 
Vol.~27. Iss.~2. P.~193--226.
\bibitem{41-nau}
\Au{Bladt M.} A~review on phase-type distributions and their use in risk theory~// ASTIN Bull., 
2005. Vol.~35. No.\,1. P.~145--161.
\bibitem{42-nau}
\Au{Artalejo J.\,R., G$\acute{\mbox{o}}$mez-Corral~A.} Markovian arrivals in stochastic 
modelling: A~survey and some new results~// SORT~--- Stat. Oper. Res.~T., 2010. 
Vol.~34. Iss.~2. P.~101--144.
\bibitem{43-nau}
\Au{Вишневский В.\,М., Дудин~А.\,Н.} Системы массового обслуживания с 
коррелированными входными потоками и их применение для моделирования 
телекоммуникационных сетей~// Автоматика и телемеханика, 2017. №\,8. С.~3--59.
\bibitem{44-nau}
\Au{Basharin G., Naumov~V., Samouylov~K.} On Markovian modelling of arrival processes~// 
Stat. Pap., 2018. Vol.~59. Iss.~4. P.~1533--1540. 
\bibitem{45-nau}
\Au{Neuts M.\,F.} Matrix-geometric solutions in stochastic models: An algorithmic approach.~--- 
Baltimore, MA, USA: The John Hopkins University Press, 1981. 332~p.
\bibitem{46-nau}
\Au{Latouche G., Ramaswami~V.} Introduction to matrix analytic methods in stochastic modeling.~--- 
Philadelphia, PA, USA: ASA \& SIAM, 1999. 334~p.
\bibitem{47-nau}
\Au{Asmussen S.} Applied probability and queues.~--- New York, NY, USA: Springer, 2003. 
438~p.
\bibitem{48-nau}
\Au{Breuer L., Baum D.} An introduction to queueing theory and matrix-analytic methods.~--- 
Dordrecht: Springer, 2005. 272~p.
\bibitem{49-nau}
\Au{Bini D.\,A., Latouche~G., Meini~B.} Numerical methods for structured Markov chains.~--- 
New York, NY, USA: Oxford University Press, 2005. 336~p.
\bibitem{50-nau}
\Au{Asmussen S., O'Cinneide~C.\,A.} Matrix-exponential distributions~// Encyclopedia of statistical 
sciences~/ Eds. S.~Kotz, C.\,B.~Read, N.~Balakrishnan, 
B.~Vidakovic, N.\,L.~Johnson.~--- Hoboken, NJ, USA: John Wiley \& Sons, 2006. Vol.~3. P.~1--5.
doi: 10.1002/0471667196.ess1092.
\bibitem{51-nau}
\Au{Li Q.-L.} Constructive computation in stochastic models with applications.~--- Berlin: Springer-Verlag, 2009. 650~p.
\bibitem{52-nau}
\Au{Lipsky L.} Queueing theory: A~linear algebraic approach.~--- 2nd ed.~--- New York, NY, 
USA: Springer, 2009. 548~p.
\bibitem{53-nau}
\Au{Alfa A.\,S.} Queueing theory for telecommunications.~--- New York, NY, USA: Springer, 2010. 
238~p.
\bibitem{54-nau}
\Au{He Q.-M.} Fundamentals of matrix-analytic methods.~--- New York, NY, USA: Springer, 2014. 
349~p.
\bibitem{55-nau}
\Au{Buchholz P., Kriege~J., Felko~I.} Input modeling with phase-type distributions and Markov 
models. Theory and applications.~--- New York, NY, USA: Springer, 2014. 127~p.
\bibitem{56-nau}
\Au{Наумов В.\,А., Самуйлов~В.\,А., Гайдамака~Ю.\,В.} Мультипликативные решения 
конечных цепей Маркова.~--- М.: РУДН, 2015. 159~с.
\bibitem{57-nau}
\Au{Bladt M., Nielsen B.\,F.} Matrix-exponential distributions in applied probability.~--- Boston, MA, USA: 
Springer, 2017. 736~p.
\end{thebibliography}

 }
 }

\end{multicols}

\vspace*{-12pt}

\hfill{\small\textit{Поступила в~редакцию 02.07.20}}

\vspace*{8pt}

%\pagebreak

\newpage

\vspace*{-28pt}

%\hrule

%\vspace*{2pt}

%\hrule

%\vspace*{-2pt}

\def\tit{ON MARKOVIAN AND RATIONAL ARRIVAL PROCESSES.~II}


\def\titkol{On Markovian and rational arrival processes.~II}


\def\aut{V.\,A.~Naumov$^1$ and~К.\,Е.~Samouylov$^{2,3}$}

\def\autkol{V.\,A.~Naumov and~К.\,Е.~Samouylov}

\titel{\tit}{\aut}{\autkol}{\titkol}

\vspace*{-11pt}


   \noindent
   $^1$Service Innovation Research Institute, 8A Annankatu, Helsinki 00120, Finland

\noindent
$^2$Peoples' Friendship University of Russia (RUDN University), 6~Miklukho-Maklaya Str., Moscow 
117198, Russian\linebreak
$\hphantom{^1}$Federation

\noindent
$^3$Institute of Informatics Problems, Federal Research Center ``Computer Science and Control'' 
of the Russian\linebreak
$\hphantom{^1}$Academy of Sciences, 44-2~Vavilov Str., Moscow 119333, Russian Federation

  

\def\leftfootline{\small{\textbf{\thepage}
\hfill INFORMATIKA I EE PRIMENENIYA~--- INFORMATICS AND
APPLICATIONS\ \ \ 2020\ \ \ volume~14\ \ \ issue\ 4}
}%
 \def\rightfootline{\small{INFORMATIKA I EE PRIMENENIYA~---
INFORMATICS AND APPLICATIONS\ \ \ 2020\ \ \ volume~14\ \ \ issue\ 4
\hfill \textbf{\thepage}}}

\vspace*{3pt} 
  
  
   
   
  \Abste{This article is the second part of the review carried out within the framework of the RFBR 
project No.\,19-17-50126. The purpose of this review is to get the interested readers familiar with the 
basics of the theory of Markovian arrival processes to facilitate the application of these models in practice 
and, if necessary, to study them in detail. In the first part of the review, the properties of the general 
Markovian arrival processes are presented and their relationship with Markov additive processes and 
Markov renewal processes is shown. In the second part of the review, the important for applications 
subclasses of Markovian arrival processes, i.\,e., simple and batch arrival processes of homogeneous and 
heterogeneous arrivals, are considered. It is shown how the properties of Markovian arrival processes are 
associated with the product form of stationary distributions of Markov systems. In conclusion, 
matrix-exponential distributions and rational arrival processes are discussed that expand the capabilities of 
Markovian arrival processes for modeling complex systems, while preserving the convenience of analyzing 
them using computations.}
  
  \KWE{Markov chain; Markovian arrival process; Markov additive process; MAP; MArP}
  
  
\DOI{10.14357/19922264200406} 

%\vspace*{-20pt}

  \Ack
  \noindent
  The reported study was funded by RFBR, project No.\,19-17-50126. 
  

%\vspace*{6pt}

  \begin{multicols}{2}

\renewcommand{\bibname}{\protect\rmfamily References}
%\renewcommand{\bibname}{\large\protect\rm References}

{\small\frenchspacing
 {%\baselineskip=10.8pt
 \addcontentsline{toc}{section}{References}
 \begin{thebibliography}{99}
  
  \bibitem{1-nau-1}
  \Aue{Naumov, V.\,A.} 1976. O~nezavisimoy rabote podsistem slozhnoy sistemy [About independent 
operation of subsystems of a complex system]. \textit{Tr. III Vsesoyuznoy shkoly-seminara po teorii 
massovogo obsluzhivaniya} [3th All-Russian School-Seminar of Queuing Theory Proceedings]. 
Moscow. 2:169--177.
  \bibitem{2-nau-1}
  \Aue{Bocharov, P.\,P., and V.\,A.~Naumov.} 1976. Analiz gipereksponentsial'noy dvukhfaznoy sistemy 
s~ogranichennym nakopitelem [Analysis of a hyperexponential two-phase system with a limited storage]. 
\textit{Informatsionnye seti i~ikh struktura} [Information networks and their structure]. Moscow: 
Nauka. 
  168--180.
  \bibitem{3-nau-1}
  \Aue{Naumov, V.\,A.} 1977. Ob obsluzhennoy i~izbytochnoy nagruzkakh polnodostupnogo puchka 
s~ogranichennoy ochered'yu [About serviced and excessive loads of a fully accessible bundle with a limited 
queue]. \textit{Chislennye metody resheniya zadach matematicheskoy fiziki i~teorii system} 
[Numerical methods for solving problems of mathematical physics and systems theory]. Moscow: UDN. 
51--55.
  \bibitem{4-nau-1}
  \Aue{Naumov, V.\,A.} 1978. Issledovanie nekotorykh mnogofaznykh sistem massovogo obsluzhivaniya 
[Research of some multiphase queuing systems].  Moscow: UDN.  PhD Thesis. 98~p.
  \bibitem{5-nau-1}
  \Aue{Lucantoni, D.\,M., K.~Meier-Hellstern, and M.\,F.~Neuts.} 1990. A single-server queue with 
server vacations and a~class of non-renewal arrival processes. \textit{Adv. Appl. Probab.} 
22(3):676--705.
  \bibitem{6-nau-1}
  \Aue{Basharin, G.\,P., V.\,A.~Kokotushkin, and V.\,A.~Naumov.} 1979. O~metode ekvivalentnykh 
zamen rascheta fragmentov setey svyazi dlya TsVM [On the method of equivalent substitutions for 
calculating fragments of communication networks for a central computer]. \textit{Engineering Cybernetics}
 6:92--99.
  \bibitem{7-nau-1}
  \Aue{Basharin, G.\,P., and V.\,A.~Naumov.} 1984. Simple matrix description of peaked and smooth 
traffic and its applications. \textit{3rd ITC Specialist Seminar on Fundamentals of Teletraffic Theory}. 
Moscow: VINITI. 
  38--44. 
  \bibitem{8-nau-1}
  \Aue{Neuts, M.\,F.} 1978. Renewal processes of phase type. \textit{Nav. Res. Logist.~Q.} 25(3):445--454.
  \bibitem{9-nau-1}
  \Aue{Cinlar, E.} 1967. Queues with semi-Markovian arrivals. \textit{J.~Appl. Probab.} 4(2):365--379.
  \bibitem{10-nau-1}
  \Aue{Franken, P.} 1968. Erlangsche Formeln f$\ddot{\mbox{u}}$r semimarkowschen Eingang. 
\textit{Elektronische Informationsverarbeitung  Kybernetik} 4(3):197--204.
  \bibitem{11-nau-1}
  \Aue{Franken, P., and J.~Kerstan.} 1968. Bedienungssysteme mit unendlich vielen 
Bedienungsapparaten. \textit{Operationsforschung Mathematische Statistik} 1:67--76.
  \bibitem{12-nau-1}
  \Aue{Neuts, M.\,F., and S.-Z.~Chen.} 1972. The infinite server queue with semi-Markovian arrivals 
and negative exponential services. \textit{J.~Appl. Probab.} 9(1):178--184.
  \bibitem{13-nau-1}
  \Aue{Bean, N.\,G., D.\,A.~Green, and P.\,G.~Taylor.} 1996. When is a MAP poisson? \textit{2nd 
  Australia--Japan Workshop on Stochastic Models in Engineering, Technology and Management 
Proceedings}. Eds. J.~Wilson, D.\,N.\,P.~Murthy, and S.~Osaki. 
Brisbane: Technology Management Center, University of Queensland. 34--43.
  \bibitem{14-nau-1}
  \Aue{Naumov, V.\,A.} 1988. Matrichnyy analog formuly Erlanga [The matrix analogue of a formula of 
Erlang]. \textit{Modeli raspredeleniya informatsii i~metody ikh analiza} [Information distribution 
models and methods for their analysis]. Moscow: VINITI. 39--43.
  \bibitem{15-nau-1}
  \Aue{He, Q.-M.} 1996. Queues with marked customers. \textit{Adv. Appl. Probab.} 
28(2):567--587.
  \bibitem{16-nau-1}
  \Aue{He, Q.-M., and M.\,F. Neuts.} 1998. Markov chains with marked transitions. \textit{Stoch. 
Proc. Appl.} 74:37--52.
  \bibitem{17-nau-1}
  \Aue{Neuts, M.\,F.} 1977. A~versatile Markovian point process.  
Newark, DE: University of Delaware, Department of Statistics and Computer Science.
Technical Report 77/13. 29~p.
  \bibitem{18-nau-1}
  \Aue{Neuts, M.\,F.} 1989. \textit{Structured stochastic matrices of $M/G/1$ type and their 
applications}. New York, NY: Marcel Dekker. 512~p.
  \bibitem{19-nau-1}
  \Aue{Lucantoni, D.\,M.} 1991. New results on the single server queue with a batch Markovian arrival 
process. \textit{Communications Statistics. Stochastic Models} 7(1):1--46. 
  \bibitem{20-nau-1}
  \Aue{Narayana, S., and M.\,F.~Neuts.} 1992. The first two moment matrices of the counts for the 
Markovian arrival processes. \textit{Communications Statistics. Stochastic Models} 8(3):459--477. 
  \bibitem{21-nau-1}
  \Aue{Nielsen, B.\,F., L.\,A.\,F.~Nilsson, U.\,H.~Thygesen, and J.\,E.~Beyer}. 2007. Higher order 
moments and conditional asymptotics of the batch Markovian arrival process. \textit{Stoch. Models} 
23(1):1--26.
  \bibitem{22-nau-1}
  \Aue{Bocharov, P.\,P., and V.\,A.~Naumov.} 1977. O~nekotorykh sistemakh massovogo 
obsluzhivaniya konechnoy emkosti [On some queueing systems of finite capacity]. \textit{Problemy 
peredachi informatsii} [Problems of Information Transmission] 13(4):96--104.
  \bibitem{23-nau-1}
  \Aue{Naumov, V.\,A.} 1982. Ob odnolineynoy sisteme s~ogranichennym nakopitelem i~zayavkami 
neskol'kikh vidov [About a single-line system with limited storage and multiple types of requests]. 
\textit{Modeli sistem raspredeleniya informatsii i~ikh analiz} [Models of information distribution 
systems and methods for their analysis]. Moscow: Nauka. 77--82.
  \bibitem{24-nau-1}
  \Aue{Naumov, V.\,A.} 1986. O~funktsiyakh raspredeleniya s~ratsio\-nal'nym preobrazovaniem  
Laplasa--Stilt'esa [On distribu\-tion functions with rational Laplace--Stiltjes transformation]. \textit{Analiz 
  informatsionno-vychislitel'nykh \mbox{system}}
   [\mbox{Analysis} of information and computing systems]. Moscow: 
UDN. 47--56.
  \bibitem{25-nau-1}
  \Aue{Bocharov, P.\,P., and A.\,V.~Pechinkin.} 1995. \textit{Teoriya massovogo obsluzhivaniya} 
[Queueing theory]. Moscow: RUDN. 528~p.
  \bibitem{26-nau-1}
  \Aue{Bocharov, P.\,P., C.~D'Apice, A.\,V.~Pechinkin, and S.~Salerno.} 2004. \textit{Queueing 
theory}. Utrecht--Boston: VSP. 446~p.
  \bibitem{27-nau-1}
  \Aue{Asmussen, S., and M.~Bladt}. 1996. Renewal theory and queueing algorithms for 
  matrix-exponential distributions. \textit{Matrix-analytic methods in stochastic models}. 
  Eds. A.\,S.~Alfa and 
S.~Chakravarty. New York, NY: Marcel Dekker. 313--341.
  \bibitem{28-nau-1}
  \Aue{Cox, D.\,R.} 1955. A~use of complex probabilities in the theory of stochastic processes. 
\textit{Math. Proc. Cambridge} 51(2):313--319.
  \bibitem{29-nau-1}
  \Aue{O'Cinneide, C.\,A.} 1990. Characterization of phase-type distributions. \textit{Communications 
Statistics. Stochastic Models} 6(1):1--57.
  \bibitem{30-nau-1}
  \Aue{Bodrog, L., A.~Horv$\acute{\mbox{a}}$th, and M.~Telek.} 2008. On the properties of 
moments of matrix exponential distributions and matrix exponential processes. 
\textit{Dagstuhl Seminar Proceedings} 07461:1394.
  \bibitem{31-nau-1}
  \Aue{Asmussen, S., and M.~Bladt.} 1999. Point processes with finite-dimensional conditional 
probabilities. \textit{Stoch. Proc. Appl.} 82(1):127--142.
  \bibitem{32-nau-1}
  \Aue{Horvath, G., and M.~Telek.} 2012. Acceptance-rejection methods for generating random 
variables from matrix exponential distribution and rational arrival processes. \textit{Matrix-analytic 
methods in stochastic models.} Eds. G.~Latouche, V.~Ramaswami, J.~Sethuraman, \textit{et al.} New York, NY: Springer. 123--144.
  \bibitem{33-nau-1}
  \Aue{Erlang, A.\,K.} 1917. \mbox{L{\!\ptb{\o}}sning} af nogle Problemer fra 
Sandsynlighedsregningen af Betydning for de automatiske Telefoncentraler. \textit{Elektroteknikeren} 
13:5--13.
  \bibitem{34-nau-1}
  \Aue{Neuts, M.\,F.} 1975. Probability distribution of phase type. \textit{Liber Amicorum Professor 
Emeritus H.~Florin}. Ottignies-Louvain-la-Neuve, Belgium: University of Louvain, Department of 
Mathematics.  
173--206.
  \bibitem{35-nau-1}
  \Aue{Bartlett, M.\,S.} 1945. Negative probability. \textit{Math. Proc. 
Cambridge} 41(1):71--73.
  \bibitem{36-nau-1}
  \Aue{Cox, D.\,R.} 1955. The analysis of non-Markovian stochastic processes by the inclusion of 
supplementary variables. \textit{Math. Proc. Cambridge} 
51(3):433--441.
  \bibitem{37-nau-1}
  \Aue{Bladt, M., and M.\,F.~Neuts.} 2003. Matrix-exponential distributions: Calculus and 
interpretations via flows. \textit{Stoch. Models} 19(1):113--124.
  \bibitem{38-nau-1}
  \Aue{Khrennikov, A.} 2009. \textit{Interpretations of probability}. 2nd ed. Berlin: Walter de 
Gruyter. 237~p.
  \bibitem{39-nau-1}
  \Aue{Naumov, V.\,A.} 1987. Markovskie modeli potokov trebovaniy [Markov models of demand 
flows]. \textit{Sistemy massovogo obsluzhivaniya i~informatika} [Queuing systems and computer 
science]. Moscow: UDN. 67--73.
  \bibitem{40-nau-1}
  \Aue{Asmussen, S.} 2000. Matrix-analytic models and their analysis. \textit{Scand. 
J.~Stat.} 27(2):193--226.
  \bibitem{41-nau-1}
  \Aue{Bladt, M.} 2005. A~review on phase-type distributions and their use in risk theory. \textit{ASTIN 
Bull.} 35(1):145--161.
  \bibitem{42-nau-1}
  \Aue{Artalejo, J.\,R., and A.~G$\acute{\mbox{o}}$mez-Corral.} 2010. Markovian arrivals in 
stochastic modelling: A~survey and some new results. \textit{SORT~--- Stat. Oper. Res.~T.}  
34(2):101--144.
  \bibitem{43-nau-1}
  \Aue{Vishnevskiy, V.\,M., and A.\,N.~Dudin.} 2017. Queueing systems with correlated arrival flows 
and their applications to modeling telecommunication networks. \textit{Automat. Rem. Contr.} 
78(8):1361--1403.
  \bibitem{44-nau-1}
  \Aue{Basharin, G., V.~Naumov, and K.~Samouylov.} 2018. On Markovian modelling of arrival 
processes. \textit{Stat. Pap.} 59(4):1533--1540.
  \bibitem{45-nau-1}
  \Aue{Neuts, M.\,F.} 1981. \textit{Matrix-geometric solutions in stochastic models: An algorithmic 
approach.} Baltimore, MA: The John Hopkins University Press. 332~p.
  \bibitem{46-nau-1}
  \Aue{Latouche, G., and V.~Ramaswami.} 1999. \textit{Introduction to matrix analytic methods in 
stochastic modeling}. Philadelphia, PA: ASA \& SIAM. 334~p.
  \bibitem{47-nau-1}
  \Aue{Asmussen, S.} 2003. \textit{Applied probability and queues}. New  York, NY: Springer. 
438~p.
  \bibitem{48-nau-1}
  \Aue{Breuer, L., and D.~Baum.} 2005. \textit{An introduction to queueing theory and 
  matrix-analytic methods.} Dordrecht: Springer. 272~p.
  \bibitem{49-nau-1}
  \Aue{Bini, D.\,A., G.~Latouche, and B.~Meini.} 2005. \textit{Numerical methods for structured 
Markov chains}. New  York, NY: Oxford University Press. 336~p.
  \bibitem{50-nau-1}
  \Aue{Asmussen, S., and C.\,A.~O'Cinneide}. 2006. Matrix-exponential distributions. 
\textit{Encyclopedia of statistical sciences.} Eds. S.~Kotz, C.\,B.~Read, N.~Balakrishnan, 
B.~Vidakovic, and N.\,L.~Johnson. Hoboken, NJ: John Wiley \&~Sons. 3:1--5. doi: 10.1002/0471667196.ess1092.pub2.
  \bibitem{51-nau-1}
  \Aue{Li, Q.-L.} 2009. \textit{Constructive computation in stochastic models with applications}. 
Berlin: 
  Springer-Verlag. 650~p.
  \bibitem{52-nau-1}
  \Aue{Lipsky, L.} 2009. \textit{Queueing theory: A~linear algebraic approach}. 2nd ed. New York, 
NY: Springer. 548~p.
  \bibitem{53-nau-1}
  \Aue{Alfa, A.\,S.} 2010. \textit{Queueing theory for telecommunications}. New York, NY: 
Springer. 238 p.
  \bibitem{54-nau-1}
  \Aue{He, Q.-M.} 2014. \textit{Fundamentals of matrix-analytic methods.} New York, NY: 
Springer. 349 p.
  \bibitem{55-nau-1}
  \Aue{Buchholz, P., J.~Kriege, and I.~Felko.} 2014. \textit{Input modeling with phase-type 
distributions and Markov models. Theory and applications.} New York, NY: Springer. 127~p.
  \bibitem{56-nau-1}
  \Aue{Naumov, V.\,A., K.\,E.~Samuylov, and Yu.\,V.~Gaidamaka.} 2015. \textit{Mul'tiplikativnye 
resheniya konechnykh tsepey Markova} [Multiplicative solutions of finite Markov chains]. Moscow: 
RUDN. 159~p.
  \bibitem{57-nau-1}
  \Aue{Bladt, M., and B.\,F.~Nielsen.} 2017. \textit{Matrix-exponential distributions in applied 
probability}. Boston, MA: Springer. 736~p.
\end{thebibliography}

 }
 }

\end{multicols}

\vspace*{-3pt}

\hfill{\small\textit{Received July 2, 2020}}

%\pagebreak

  %\vspace*{-24pt}
  
  \Contr
  
  \noindent
  \textbf{Naumov Valeriy A.} (b.\ 1950)~--- Candidate of Science (PhD) in physics and mathematics, 
scientific director, Service Innovation Research Institute, 8A~Annankatu, Helsinki 00120, Finland; 
\mbox{valeriy.naumov@pfu.fi}
  
  \vspace*{3pt}
  
  \noindent
  \textbf{Samouylov Konstantin E.} (b.\ 1955)~--- Doctor of Science in technology, professor, Head of 
Department,  Peoples' Friendship 
University of Russia (RUDN University), 6~Miklukho-Maklaya Str., Moscow 117198, Russian 
Federation; senior scientist, Institute of Informatics Problems, Federal Research Center ``Computer 
Science and Control'' of the Russian Academy of Sciences, 44-2~Vavilov Str., Moscow 119333, Russian 
Federation; 
  \mbox{samuylov-ke@rudn.university}
  
\label{end\stat}

\renewcommand{\bibname}{\protect\rm Литература} 
  
     %10
\def\stat{buyanov}

\def\tit{РАЗВИТИЕ МАТЕМАТИЧЕСКОЙ МОДЕЛИ УПРАВЛЕНИЯ ГРУЗОПЕРЕВОЗКАМИ 
НА~УЧАСТКЕ ЖЕЛЕЗНОДОРОЖНОЙ СЕТИ С~УЧЕТОМ СЛУЧАЙНЫХ ФАКТОРОВ$^*$}

\def\titkol{Развитие математической модели управления грузоперевозками 
%на~участке железнодорожной сети 
с~учетом случайных факторов}

\def\aut{М.\,В.~Буянов$^1$, С.\,В.~Иванов$^2$, А.\,И.~Кибзун$^3$, А.\,В.~Наумов$^4$}

\def\autkol{М.\,В.~Буянов, С.\,В.~Иванов, А.\,И.~Кибзун, А.\,В.~Наумов}

\titel{\tit}{\aut}{\autkol}{\titkol}

\index{Буянов М.\,В.}
\index{Иванов С.\,В.}
\index{Кибзун А.\,И.}
\index{Наумов А.\,В.}
\index{Buyanov M.\,V.}
\index{Ivanov S.\,V.}
\index{Kibzun A.\,I.}
\index{Naumov A.\,V.}





{\renewcommand{\thefootnote}{\fnsymbol{footnote}} \footnotetext[1]
{Результаты работы получены в~рамках выполнения государственного задания 
Минобрнауки №\,2.2461.2017/ПЧ, а~также при поддержке РФФИ 
и~ОАО <<РЖД>> в~рамках научного проекта №\,17-20-03050~офи\_м\_РЖД.}}


\renewcommand{\thefootnote}{\arabic{footnote}}
\footnotetext[1]{Московский авиационный институт 
(национальный исследовательский 
университет), \mbox{buyanovmikhailv@gmail.com}}
\footnotetext[2]{Московский авиационный институт 
(национальный исследовательский 
университет), \mbox{sergeyivanov89@mail.ru}}
\footnotetext[3]{Московский авиационный институт 
(национальный исследовательский 
университет), \mbox{kibzun@mail.ru}}
\footnotetext[4]{Московский авиационный институт 
(национальный исследовательский 
университет), \mbox{naumovav@mail.ru}}


\vspace*{-18pt}



\Abst{Предлагается математическая модель назначения локомотивов для перевозки 
грузовых составов.
Целью оптимизации в~модели является минимизация числа задействованных для 
перевозки составов локомотивов
за счет выбора маршрутов составов и~локомотивов.
Приводится детерминированный алгоритм получения субоптимального решения,
а~также алгоритм, реализующий схему оперативного планирования. Предлагается 
использование случайного параметра,
моделирующего задержку готовности состава к~отправлению.
Проводится численный эксперимент в~условиях неполной ин\-фор\-ми\-ро\-ван\-ности.
Численный эксперимент проведен на примере данных Московской железной дороги
(МЖД).
Сравниваются результаты, полученные в~детерминированной и~стохастической постановках.}

\KW{математическое моделирование; оптимизация; планирование перевозок; 
оперативное планирование}

\DOI{10.14357/19922264170411} 

\vspace*{-8pt}


\vskip 10pt plus 9pt minus 6pt

\thispagestyle{headings}

\begin{multicols}{2}

\label{st\stat}

\section{Введение}

\vspace*{-4pt}
    
Проблема организации перевозок на железнодорожном транспорте затрагивалась 
во многих работах, среди которых можно выделить~[1--8].
%\cite{AzanovBuyanov, KibzunNaumov, isuzht2015, isuzht2016, belyi, cacchiani, lazarev1, lazarev2}.
В~этих работах описана структура грузовых перевозок на железнодорожном транспорте и~предложены
различные математические модели организации грузовых перевозок.
%В \cite{AzanovBuyanov}
%предложена детерминированная модель оптимизации назначения локомотивов на сформированные
%составы по критерию минимизации общего числа задействованных локомотивов.
В~работе развивается предложенная ранее 
в~\cite{AzanovBuyanov} детерминированная модель
оптимизации назначения локомотивов на сформированные составы.
При этом учитывается влияние на модель случайных факторов, приводящих 
к~задержке готовности состава к~отправлению.
    
Необходимо отметить, что в~процессе осуществления грузовых перевозок 
возникает множество случайных факторов,
влияющих на работу локомотивов, таких как
задержки формирования составов, задержки в~движении поездов, аварии, 
неопределенное поведение диспетчеров, ошибки машинистов и~т.\,д. 
Таким образом, детерминированное решение, полученное в~\cite{AzanovBuyanov}, 
не может быть реализовано на практике и~требуется более реалистичный 
стохастический подход. Однако учет всех существующих случайных факторов 
является очень трудоемкой задачей и~приведет к~большим трудностям при 
получении решения, в~связи с~чем предлагается рассмотреть факторы, влияющие 
на время готовности состава к~отправлению.
    

В работе описываются основные случайные факторы, приводящие к~задержкам по времени,\linebreak
влияющим на время формирования составов.
Предлагается эвристический алгоритм поиска субоптимального решения задачи.
Проводится чис\-лен\-ный эксперимент с~учетом случайных факторов на примере данных 
МЖД.
Приведенные в~статье результаты сравниваются с~решением, полученным
ранее авторами~\cite{AzanovBuyanov} в~детерминированной по\-ста\-новке.

\vspace*{-12pt}

\section{Основные определения и~постановка задачи}

\vspace*{-6pt}

В основе модели, предложенной в~\cite{AzanovBuyanov},
лежит взвешенный ориентированный граф $G \hm= (V, A)$, где $V$~--- 
множество вершин; $A$~--- множество дуг.
Вершинами графа~$G$ являются значимые станции. Значимыми называются 
станции, на которых формируются грузовые составы (сортировочные станции), 
и~станции смены локомотивной тяги.\linebreak Некото\-рые значимые станции являются 
стан\-ци\-ями-де\-по, соответствующее подмножество вершин~$V$ обозначим через~$D$. 
Множеству дуг соответствуют перегоны, соединяющие значимые станции.

Локомотивы могут передвигаться только по определенным маршрутам 
(так называемым плечам), в~связи с~чем в~\cite{AzanovBuyanov} вводятся 
следующие определения.

\smallskip

\noindent
\textbf{Определение 1.}\
    Назовем плечом~$P$ последовательность дуг $a_1,\ldots, a_{I_P}$ графа~$G$, 
     удовлетворяющую следующим условиям:
    \begin{enumerate}[(1)]

    \item
        все дуги различны: $a_i \hm\ne a_j$, $i \hm\ne j$, $i,j \hm\in \{1, I_P\}$;

    \item
        первая вершина первой дуги в~последовательности совпадает 
        с~последней вершиной последней дуги последовательности, является 
        стан\-ци\-ей-де\-по и~отлична от всех промежуточных вершин последовательности: 
        $v_1\hm = v_{I_P}\hm\in D$, $ v_i \hm\ne v_1$ для $i\hm=\overline{2, I_P-1}$.
    
    \end{enumerate}

Также рассматриваются подплечи и~простые подплечи, определенные следующим образом.
    
\smallskip

\noindent
\textbf{Определение 2.}\
  Любую подпоследовательность соседних дуг $a_i, a_{i+1},\ldots, a_j$ 
   $(1\hm\leqslant i\hm<j \hm\leqslant I_P)$, образующих плечо, назовем 
   подплечом данного плеча. Любую дугу $a_i\hm=(v_{i-1},v_i)$, 
   входящую в~некоторое плечо~$P$, назовем простым подплечом плеча~$P$.
    
\smallskip

Пусть $L$~--- множество всех локомотивов, приписанных 
к~рассматриваемым стан\-ци\-ям-де\-по~$D$. Для каждого локомотива 
$l\hm\in L$ задано множество допустимых плеч~$\overline {\mathcal P}_l$, 
по которым он может передвигаться. Каждому множеству  
$\overline {\mathcal P}_l$, $l\hm\in L$, ставится в~соответствие 
множество~$\mathcal P_l$, составленное из всех простых подплеч, 
входящих в~плечи множества~$\overline {\mathcal P}_l$. Предполагается, что 
для каждого локомотива $l\hm\in L$ задана функция весовых 
норм~$w_l (\cdot) \colon {\mathcal P}_l \hm\to \mathbb{R}$, ставящая в~соответствие 
простым подплечам плеча~$\overline {\mathcal P}_l$ максимально допустимую 
для перевозки массу состава.

Пусть $S$~--- множество грузовых составов. Каж\-дый состав $s \hm\in S$ 
характеризуется массой~$w^s$, начальной станцией~$v^s_o$, станцией назначения~$v^s_f$, 
временем формирования~$t^s_o$, временем~$\tau^s_f$, до которого необходимо 
прибыть на станцию назначения, т.\,е.\ каждому составу соответствует пятерка
 $(w^s, v^s_o, t^s_o, v^s_f, \tau^s_f)$. По сути, данные характеристики 
 определяют план перевозок.

Движение локомотивов и~составов по заданному маршруту может 
осуществляться только в~определенные промежутки времени. Совокупность 
маршру\-та и~времени будем называть ниткой. По аналогии с~подплечами и~простыми 
подплечами введем в~рассмотрение поднитки и~простые поднитки. Приведем 
определения нитки, поднитки и~простой поднитки, описанные в~\cite{AzanovBuyanov}.

\smallskip

\noindent
\textbf{Определение 3.}\
    Ниткой $N$ назовем упорядоченное множество четверок 
    $(v_1, t_1, v_2, \tau_2)$, $(v_2, t_2, v_3, \tau_3)$, \ldots, 
    $(v_{I_N-1}, t_{I_N-1}, v_{I_N}, \tau_{I_N})$, удовлетворяющее условиям:
    \begin{enumerate}[(1)]

    \item
        $v_i \in V$, $i \hm= \overline{1, I_N}$, $t_i\hm \in \mathbb{R}$, $i \hm= \overline{1, I_N-1}$, $\tau_i \in \mathbb{R}$, $i = \overline{2, I_N}$;

    \item
        $(v_{i}, v_{i+1}) \in A$, $i \hm= \overline{1, I_N-1}$;

    \item
        $t_i < \tau_{i+1}$, $i\hm = \overline{1, I_N -1}$;

    \item
        $\tau_{i} \leqslant t_{i} $, $i \hm= \overline{2,I_N}$.
        
    \end{enumerate}



Во введенном определении величина~$t_i$ соответствует времени отправления со
 станции~$v_i$, а~$\tau_{i+1}$~--- времени прибытия на станцию~$v_{i+1}$. 
 Приведенные условия выражают естественные свойства движения поездов, 
 заключающиеся в~том, что движение может осуществляться только по 
 перегонам (условия~1 и~2), время отправления со станции не может быть 
 позже времени прибытия на следующую станцию (условие~3), время прибытия 
 на станцию не может быть позже времени отправления с~той же станции (условие~4).

\smallskip

\noindent
\textbf{Определение 4.}\
    Каждую подпоследовательность соседних четверок, образующих нитку~$N$, 
    назовем подниткой. Каждую четверку $(v_i, t_i, v_{i+1}, \tau_{i+1})$, 
    $i\hm=\overline{1, I_N-1}$, составляющую нитку~$N$, назовем прос\-той подниткой.

\smallskip

Пусть имеется множество~$\overline{\mathcal N}$ ниток. Сопоставим каждому элементу~$N$ 
данного множества множество~$\mathcal F(N)$,  являющееся неупорядоченным множеством 
простых подниток, составляющих нитку~$N$. Множество всех простых подниток, 
полученных из множества ниток~$\overline {\mathcal N}$, обозначим через~$\mathcal N$, 
т.\,е.\
\begin{equation*}
%    \label{1.2}
        \mathcal N = \bigcup_{N\in \overline{\mathcal N}} \mathcal F(N)\,.
\end{equation*}

Важно отметить, что каждая простая поднитка проходит только по одной из дуг графа.
    
На множестве $2^L\times \mathcal N$, являющемся декартовым 
произведением всех возможных сочетаний локомотивов и~множества 
простых подниток, определим функцию~$W(\pi_n)$, задающую максимальную массу 
состава, которую может перевезти со\-от\-вет\-ст\-ву\-ющая комбинация локомотивов $\pi_n 
\hm\subset L$ по заданной простой поднитке $n \hm\in \mathcal N$. Очевидно, 
если $\pi_n\hm = l \hm\in L$, где $n \hm= (v,t,v',\tau)$ 
и~$(v,v')\hm\in \mathcal P_l$, то $W(\pi_n) \hm= w_l((v, v'))$. 
Комбинация локомотивов~$\pi_n$ называется составным локомотивом и~используется 
для перевозки состава посредством их совместной работы.
    
Поскольку движение локомотивов осуществляется только по ниткам и~по плечам, 
приведем определение допустимого маршрута оборота локомотива из~\cite{AzanovBuyanov}.
В данном определении учтем также, что локомотив через интервалы времени~$T^{\mathrm{TO}}$ 
(48~ч) должен проходить техосмотр (ТО) продолжительностью~$t^{\mathrm{TO}}$ (8~ч). 
Будем считать, что каждый локомотив $l\hm\in L$ в~начальный момент времени 
характеризуется временем~$\tau_l^{\mathrm{TO}}$, прошедшим с~момента последнего~ТО.
Если локомотив в~начальный момент времени находится на ТО, 
то величина~$\tau_l^{\mathrm{TO}}$ принимает отрицательное значение, равное по модулю 
времени до окончания~ТО.

\smallskip

\noindent
\textbf{Определение 5.}\ %\\label{defMl}
    Допустимым маршрутом оборота~$M_l$  локомотива~$l$ относительно множества 
    плеч~$\overline {\mathcal P}_l$ назовем последовательность прос\-тых 
    подниток $(v_1, t_1, v_2, \tau_2)$,  $(v_2, t_2, v_3, \tau_3)$, 
    \ldots\linebreak $\ldots , (v_{I_l-1}, t_{I_l-1}, v_{I_l}, \tau_{I_l})$, удовлетворяющую условиям:
\begin{enumerate}[(1)]
\item $\tau_{i} \leqslant t_{i}$, $i \hm= \overline{2,I_l -1}$;

\item $(v_i,v_{i+1}) \in\mathcal {P}_l $,  $i \hm= \overline{1,I_l -1}$;

\item существует возрастающая последовательность $i_1, \ldots, i_{f_l}$ 
чисел, выбранных из множества $\{2,3,\ldots,I_l\}$ таким образом, что
\begin{enumerate}[({3}.1)]
\item $\tau^{\mathrm{TO}}_l + \tau_{i_1} \leqslant T^{\mathrm{TO}}$;
    
     \item $t_{i_j}-\tau_{i_j} \geqslant t^{\mathrm{TO}}$, $j\hm=\overline{1, f_l-1}$;
    
        \item $\tau_{i_j} - t_{i_{j-1}} \leqslant T^{\mathrm{TO}}$, $j\hm=\overline{2, f_l}$;
    
        \item $\tau_{I_l} - t_{i_{f_l}} \leqslant T^{\mathrm{TO}}$, если $f_l\hm\ne I_l$.
        
\end{enumerate}
\end{enumerate}

Условие~1 требует, чтобы время прибытия на станцию не было раньше времени 
отправления с~той же станции. Условие~2 ограничивает возможные передвижения 
локомотива только движением по плечам. Условие~3 требует прохождения ТО через 
установленные промежутки времени. Последовательности моментов времени 
$t_{i_1}, \ldots, t_{i_{f_l}}$ соответствуют моментам начала ТО. В~условии~3.1 
требуется, чтобы время ухода на первое ТО не превышало~$T^{\mathrm{TO}}$ 
с~момента предыдущего ТО. Согласно условию~3.2 время прохождения ТО не может 
быть меньше~$t^{\mathrm{TO}}_l$. Из~условия~3.3 
следует, что время между началом движения после ТО и~уходом на сле\-ду\-ющее 
ТО не может быть больше~$T^{\mathrm{TO}}$. Согласно условию~3.4 
время начала движения после последнего ТО должно быть не позже, чем за время~$T^{\mathrm{TO}}$ 
до окончания рассматриваемого периода планирования движения.

Заметим, что маршрут оборота является про\-стран\-ст\-вен\-но-вре\-мен\-н$\acute{\mbox{ы}}$м 
понятием. Множество допустимых маршрутов оборота локомотива~$l$ обозначим 
через~$\mathcal M_l$. Начальную и~конечную станции маршрута оборота~$M_l$ 
обозначим через~$v_o(M_l)$ и~$v_f(M_l)$ соответственно, время начала первой нитки 
данного маршрута оборота обозначим через~$t_o(M_l)$, время прибытия на станцию 
назначения~--- через $\tau_f(M_l)$.

Введем определение допустимого рейса состава, которое так же, 
как маршрут оборота локомотива, является 
про\-стран\-ст\-вен\-но-вре\-мен\-н$\acute{\mbox{о}}$й характеристикой и~было 
ранее описано в~\cite{AzanovBuyanov}.

\smallskip

\noindent
\textbf{Определение~6.}\ 
    Допустимым рейсом~$R_s$ состава $s\hm\in S$ назовем последовательность 
    прос\-тых подниток $(v_1, t_1, v_2, \tau_2)$,  $(v_2, t_2, v_3, \tau_3)$, \ldots$\linebreak $\ldots,
$(v_{I_s-1}, t_{I_s-1}, v_{I_s}, \tau_{I_s})$, удовлетворяющую условиям:
\begin{enumerate}[(1)]
\item  $v_1 = v^s_o$;

\item $v_{I_s} = v^s_f$;

\item  $\tau_{i} \leqslant t_{i}$, $i \hm= \overline{2,I_s -1}$;

\item  $t^s_o \leqslant t_1$;

\item  $\tau^s_f \geqslant \tau_{I_s}$.
\end{enumerate}


Условия 1 и~2 определяют начальную и~конечную станции рейса, 
условие~3 задает естественные ограничения на время отправления и~прибытия, 
условия~4 и~5 требуют выполнения перевозок согласно плану.

Множество допустимых рейсов состава~$s$ обозначим через~$\mathcal R_s$.

Так же как и~для ниток, определим множество~$\mathcal F(M_l)$ всех простых 
подниток, со\-став\-ля\-ющих маршрут оборота~$M_l$ локомотива~$l$, $l\hm\in L$, 
и~множество~$\mathcal F(R_s)$, $s\hm\in S$, всех простых подниток, со\-став\-ля\-ющих 
рейс~$R_s$ состава~$s$.

Для каждой простой поднитки $n\hm\in\mathcal N$ и~каждого набора 
маршрутов оборота локомотивов  $M\hm=\{M_l\}_{l\in L}$ определим 
множество~$\pi_{n}(M)$, со\-став\-лен\-ное из всех локомотивов, передвигающихся 
по простой поднитке~$n$ при наборе маршрутов оборота локомотивов~$M$:
\begin{equation*}
    l\in \pi_{n}(M) \Leftrightarrow n\in\mathcal F\left(M_l\right)\,.
\end{equation*}

Рассмотрим некоторый участок железнодорожной сети с~графом $G\hm = (V, A)$, 
определенным выше. Пусть задано множество локомотивов~$L$, множество составов~$S$, 
множество ниток~$\overline{\mathcal N}$ и~соответствующих простых 
подниток~$\mathcal N$, функция~$W(\cdot)$ весовых норм составных локомотивов. 
Для каждого локомотива $l\hm\in L$ определено множество плеч~$\overline{\mathcal P}_l$ 
и~простых подплеч~$\mathcal P_l$.

В начальный момент времени некоторые локомотивы могут находиться в~движении, 
поэтому будем считать, что локомотив $l\hm\in L$ можно отправить только 
с~некоторой фиксированной станции~$v_0^l$ после момента времени~$t_0^l$.
Пусть  $|L|$~--- число локомотивов в~множестве~$L$, имеющих непустой маршрут оборота.

Пусть для каждого состава $s\hm\in S$ задано множество ниток
 $\overline{\mathcal N_s}\hm\subset \overline{\mathcal N}$, 
 по которым он может быть перевезен. Через~$\mathcal N_s$ обозначим множество 
 соответствующих простых подниток. Данные ограничения
  связаны с~тем, 
 что некоторые нитки могут быть использованы только для перевозки 
 составов определенного рода.

Пусть $M=\{M_l\}_{l\in L}$~--- выбираемый набор маршрутов оборота 
всех локомотивов; $R \hm= \{R_s\}_{s\in S}$~--- выбираемый набор 
рейсов всех со\-ста\-вов; $\mathcal{M}\hm=\{\mathcal M_l\}_{l\in L}$~---
 множество допустимых\linebreak маршрутов оборота всех локомотивов; 
 $\mathcal {R}\hm = \{\mathcal R_s\}_{s\in S}$~--- множество допустимых рейсов
  всех составов.

Требуется найти такой набор $M$ маршрутов оборота локомотивов и~такой набор~$R$ 
рейсов составов, при котором общее число~$|L|$ локомотивов, используемых 
для перевозки составов, будет минимальным, при этом все рейсы 
составов будут покрыты маршрутами локомотивов.
    
В~\cite{AzanovBuyanov} была предложена следующая постановка задачи:

\noindent
\begin{equation}
    \label{problem_main}
    |L| \to \min_{M\in \mathcal M, R\in\mathcal R}
\end{equation}
    при ограничениях
    
    \noindent
\begin{gather}
        M_{l}\in\mathcal M_l\,,\enskip l\in L\,;    \label{c1}\\
    \label{c2}
        R_s \in \mathcal R_s\,,\enskip s\in S\,;\\
    \label{c3}
           \bigcup\limits_{s\in S} \mathcal F(R_s) \subset \bigcup\limits_{l\in L} \mathcal F(M_l)\,;\\
    \label{c4}
        \mathcal F(R_s)\cap \mathcal F(R_{s'}) = \varnothing\,, 
        \enskip s \ne s', \enskip s, s'\in S\,;\\
    \label{cadd1}
        W(\pi_n)\geqslant w^s\,,\enskip n\in\mathcal F(R_s)\,,\enskip s\in S\,;\\
    \label{c5}
        \mathcal F(M_l) \subset \mathcal N\,,\enskip l \in L\,;\\
    \label{c6}
        \mathcal F(R_s) \subset \mathcal N_s\,,\enskip s\in S\,;\\
    \label{c7}
        v_0(M_l) = v_0^l\,;\\
    \label{c8}
        t_0(M_l) \geqslant t_0^l\,.
\end{gather}

Условия~(\ref{c1}) и~(\ref{c2}) значат, что рассматриваются только допустимые 
маршруты локомотивов и~рейсы составов, в~частности те, для которых 
существуют допустимые плечи. Также заметим, что допустимость 
рейсов составов требует, чтобы был выполнен план перевозок в~установленный срок.

Условие~(\ref{c5}) требует, чтобы маршруты оборота локомотивов составлялись 
только из простых подниток, поскольку множество $\bigcup\nolimits_{l\in L} 
\mathcal F(M_l)\hm \subset \mathcal N$ со\-став\-ле\-но из простых подниток, 
входящих в~ка\-кой-ли\-бо маршрут оборота локомотива. Условие~(\ref{c6})\linebreak 
задает аналогичное требование для рейсов составов, а~кроме этого оно 
ограничивает выбор допустимых ниток для перевозки состава. Условие~(\ref{c3}) 
означает, что все простые поднитки, образующие рейс некоторого состава, 
используются для движения некоторого локомотива, т.\,е.\
 все составы перевозятся локомотивами. Также из этого условия 
 следует, что локомотивы могут передвигаться по простым подниткам, 
 по которым не движутся составы. Таким образом, каждой задействованной 
 нитке со\-от\-вет\-ст\-ву\-ет либо состав с~локомотивом (возможно, 
 с~несколькими локомотивами), либо локомотив, движущийся порожняком.

Условие~(\ref{c4}) означает, что рейсы составов не могут пересекаться, т.\,е.\
 одну простую поднитку нельзя использовать для передвижения двух составов. 
 Поскольку локомотивы могут ехать в~сплотке или с~составом (так называемый 
 вспомогательный пробег), то подобное условие для локомотивов отсутствует.

Условие~(\ref{cadd1}) требует, чтобы были выполнены весовые нормы составных 
локомотивов при перевозке составов, т.\,е.\ составной локомотив~$\pi_n$, 
используемый на простой поднитке $n \hm\in \mathcal F(R_s)$, по которой 
перевозится состав~$s$, должен иметь возможность перевозить состав массой~$W(\pi_n)$, 
не меньшей чем масса~$w^s$ состава~$s$.

Условия~(\ref{c7}) и~(\ref{c8}) задают начальное состояние локомотивов.

Заметим также, что множество составов~$S$ и~множество ниток~$\overline{\mathcal N}$ 
определяются суточным планом перевозок и~количеством суток, на которые осуществляется 
планирование.

Сформулированная задача предполагает оптимизацию как по маршрутам оборота локомотивов, 
так и~по рейсам составов. Однако на практике нитки уже сформированы под конкретные 
составы, поэтому в~дальнейшем будем считать, что множество допустимых 
рейсов~$\mathcal R_s$ состава $s\hm\in S$ состоит из одного рейса. 
Таким образом, задача сводится к~назначению локомотивов для перевозки 
составов с~заданными рейсами, т.\,е.\ к~поиску набора маршрутов~$M$.


Для исследования существования решения задачи необходимо определить, 
является ли набор ниток достаточным
для осуществления плана перевозок.
Решение данной задачи сравнимо с~решением исходной задачи. Ниже приведен 
эвристический алгоритм решения задачи,
который в~ряде случаев позволяет находить допустимое решение задачи.
Вопрос о единственности решения не является актуальным с~практической точки зрения,
так как достаточно найти хотя бы одно решение, обеспечивающее минимальное значение 
целевой функции.
Вычислительная сложность данной задачи в~работе не исследуется, однако можно заметить,
что время перебора всех допустимых маршрутов локомотивов и~рейсов составов
зависит экспоненциально от объема исходных данных задачи.

\vspace*{-2pt}

\section{Алгоритм решения детерминированной задачи}

\vspace*{-2pt}

Опишем алгоритм получения субоптимального решения задачи~(\ref{problem_main}).
Будем считать, что рейсы всех составов определены, т.\,е.\ для каждого состава 
определена нитка,
по которой он движется. В~основе алгоритма лежит идея о~максимальном использовании 
локомотива
с~минимальным временем начала движения. Решение предполагает неограниченное число 
локомотивов
в~начальный момент времени в~каждом депо, однако даже такое предположение 
позволяет получить лучший,
в~сравнении с~реаль-\linebreak\vspace*{-12pt}

\pagebreak

\noindent
ным движением, результат.
Для простоты изложения алгоритмов при их построении не учитываются 
ограничения на массу перевозимых составов и~опускается описание алгоритма проведения 
ТО. При наличии ограничений на массу составов необходимо осуществлять поиск не только простых локомотивов $l\in L$,
но и~составных локомотивов, т.\,е.\ комбинаций нескольких локомотивов.

\vspace*{-6pt}
    
\subsection{Алгоритм назначения} \label{find1}

\vspace*{-2pt}
    
Пусть задано непустое множество составов $S \hm= \{s_i\mid i = \overline{1, |S|}\}$ 
с~непустыми рейсами.
Пусть~$v^l_f$ и~$\tau^l$~--- конечная станция маршрута оборота~$M_l$ (либо начальная станция в~случае пустого маршрута)
и время прибытия на эту станцию локомотива $l\hm\in L$. Для корректной работы алгоритма 
необходимо,
чтобы элементы множества~$S$ были упорядочены по возрастанию времени отправления составов.
Данный алгоритм является упрощенной версией алгоритма в~\cite{AzanovBuyanov},
при этом для ряда примеров решения, получаемые с~по\-мощью данных алгоритмов, совпадают.
    
\renewcommand{\figurename}{\protect\bf Алгоритм}

\begin{figure*} 
\hrule

\vspace*{-4pt}

\Caption{\ }  

\vspace*{3pt}  
\hrule

\vspace*{2pt}
        \begin{enumerate}[1.]
        \setcounter{enumi}{-1}
        
        \item
        Полагаем $i:=1, j:=1, k:=1$.
        
        \item
        Зафиксируем локомотив $l_k \in L$, состав $s_i \hm\in S$ и~простую поднитку 
        из рейса состава $n_j \hm= (v_o(n_j), t(n_j), v_f(n_j), \tau(n_j))$, 
        $n_j \hm\in \mathcal F(R_{s_i})$, переходим к~шагу~3.
        
        \item
        Если $k > |L|$, то берем новый локомотив $L:=L \cup \{l_k\}$, 
        $i:=1$, $j:=1$ и~повторяем шаг~2; если $i\hm > |S|$, то переходим 
        к~следующему локомотиву $k := k+1$, $i:=1$, $j:=1$ и~повторяем шаг~2; 
        если $j\hm > |\mathcal F(R_s)|$, то переходим к~следующему составу 
        $i := i+1$, $j:=1$ и~повторяем шаг~2. Переходим к~шагу~3.
        
        \item
        Если $\tau^{l_k} \leqslant t(n_j)$, $(v_o(n_j), v_f(n_j))\hm \in 
        \mathcal P_{l_k}$, переходим к~шагу~4. Иначе переходим к~шагу~2.
        
        \item
        Если $v^{l_k}_f \ne v_o(n_j)$, выполняем поиск нитки~$N^*$ 
        для перегонки локомотива~$l_k$ к~началу простой поднитки~$n_j$, согласно 
        подразд.~3.2. Если $v^{l_k}_f \hm= v_o(n_j)$, полагаем 
        $N^*:=\varnothing$. Если нитка~$N^*$ найдена, переходим к~шагу~5, 
        иначе переходим к~шагу~2.

        \item
        Внесем найденную простую поднитку~$n_j$ и,~если необходимо, соответствующую 
        ей нитку для перегонки~$N^*$ в~маршрут локомотива $M_{l_k} \hm= M_{l_k} 
        \cup N^* \cup \{n_j\}$. Уберем простую поднитку~$n$ из рейса состава 
        $R_{s_i} := R_{s_i} \setminus \{n_j\}$, если $\mathcal F(R_{s_i})\hm = 
        \varnothing$, то уберем состав из множества рассматриваемых
         $S := S \setminus \{s_i\}$. Если $S \hm= \varnothing$, 
         переходим к~шагу~6, иначе переходим к~шагу~2.

        \item
        Окончание алгоритма, получено субоптимальное решение 
        задачи~(\ref{problem_main}).

    \end{enumerate}
    \hrule
 %   \vspace*{4pt}
\end{figure*}

%\vspace*{-9pt}


\vspace*{-6pt}

\subsection{Поиск составной нитки} \label{N*}

\vspace*{-2pt}

Для осуществления перегонки локомотива необходимо найти нитку~$N^*$, 
соединяющую станцию~$v^l_f$, на которой находится локомотив~$l$,\linebreak
 и~станцию~$v_o(n)$, 
с~которой отправляется прос\-тая поднитка~$n$. Пусть~$t(n)$~--- 
время начала движения по простой поднитке~$n$, $\tau(n)$~--- 
время окончания прос\-той поднитки~$n$, а~$\tau^l$~--- 
время остановки локомотива на станции~$v^l_f$. Через~$\mathcal N_a$ 
обозначим множество прос\-тых подниток, соответствующих дуге $a\hm\in A$. 
Сопоставим каждой дуге графа~$G$ весовую характеристику, равную среднему 
времени движения по ней:
\begin{equation}
    \label{w_a}
    w_a = \fr {1}{|\mathcal N_a|}\sum\limits_{n\in \mathcal N_a} (\tau(n) - t(n))\,.
\end{equation}

Полагаем $N^*$ равной нитке, проходящей по кратчайшему пути в~графе~$G$, 
взвешенном согласно~(\ref{w_a}), соединяющему станции~$v^l_f$ и~$v_o(n)$, 
с~временем начала не ранее~$\tau^l$ и~временем окончания не позднее~$t(n)$. 
Для поиска кратчайших путей между вершинами взвешенного ориентированного графа 
можно использовать, например, алгоритм Флой\-да--Уор\-шел\-ла~\cite{Floyd}.

\vspace*{-6pt}

\section{Статистическое моделирование}

\vspace*{-2pt}

В процессе осуществления грузовых перевозок возникает множество случайных 
факторов, влияющих на работу локомотивов, таких как
задержки формирования составов, задержки в~движении поездов и~другие 
нештатные ситуации. 
В~связи с~этим детерминированное решение, полученное в~\cite{AzanovBuyanov}, 
не может быть реализовано на практике. Будем моделировать задержки во времени 
формирования состава.

\renewcommand{\figurename}{\protect\bf Алгоритм}

\begin{figure*}[b] %\label{alg:2} %fig2
\hrule

\vspace*{-4pt}

\Caption{\ }  

\vspace*{3pt}  
\hrule

\vspace*{2pt}

    \begin{enumerate}[1.]
        \setcounter{enumi}{-1}

        \item
        Полагаем $i := 0$.

        \item
        Из множества составов $S_i$ выберем подмножество~$S_i^{\Delta T}$ 
        такое, что для всех $s \hm\in S_i^{\Delta T}$
        выполнено $T_o \hm+ i\Delta T \leqslant t^s_o \hm\leqslant T_o \hm+ 
        (i+1)\Delta T$. Пусть теперь для каждого состава~$s$
        из множества~$S_i^{\Delta T}$ задано время фактической готовности 
        к~отправлению $\tau^s_o \hm= t^s_o \hm+ \xi_s$. Переходим к~шагу~2.

        \item
        Определим рейсы составов. Для каждого состава $s \hm\in S_i^{\Delta T}$ 
        выберем нитку $n \hm\in \overline{\mathcal N_s}$ такую,
        чтобы разница во времени отправления~$t(n)$ по нитке~$n$ 
        и~времени фактической готовности~$\tau_o^s$ состава~$s$
        к~отправлению была минимальной, и~внесем ее в~рейс состава 
        $R_s := R_s \cup \{n\}$.
        Для $s \hm\in S_i \setminus S_i^{\Delta T}$ выберем нитку 
        $n \hm\in \overline{\mathcal N_s}$ такую, чтобы разница
        во времени отправления~$t(n)$ по нитке~$n$ и~планируемого времени 
        формирования~$t_o^s$  состава~$s$ была минимальной,
        и~внесем ее в~рейс состава $R_s := R_s \cup \{n\}$. Переходим к~шагу~3.

        \item
        Для полученного множества составов~$S_i$ с~заданными рейсами 
        и~множества локомотивов~$L$ выполним назначение локомотивов согласно
         подразд.~3.1. Зафиксируем маршруты локомотивов на момент времени~$T_o \hm+ 
         i\Delta T$, т.\,е.\ уберем из маршрутов локомотивов все простые поднитки
          со временем отправления, превышающим $T_o \hm+ i\Delta T$. Переходим к~шагу~4.

        \item
        Примем $i := i+1$. Если $i \hm> [T_m/({\Delta T})]$, переходим к~шагу~5, 
        иначе переходим к~шагу~1.

        \item
        Окончание алгоритма, получено субоптимальное решение.
    \end{enumerate}
    \hrule
%    \vspace*{6pt}
\end{figure*}


Задержки во времени готовности состава к~отправлению могут возникать в~результате 
множества причин.
Такими причинами могут стать, например, ошибки при планировании работы станции, 
включающие в~себя как работу маневровых локомотивов,
так и,~например, неверную очередность формирования и~расформирования составов, 
задержки в~прибытии вагонов (грузов),
участвующих в~составообразовании на станцию отправления, нарушение технических 
нормативов в~результате человеческого фактора,
погодных и~иных явлений и~т.\,д. В общем случае все описанные факторы в~том или 
ином виде приводят к~изменению времени
готовности состава к~отправке. Таким образом, сведем учет всех случайных факторов,
связанных с~задержками по времени, к~одной случайной величине, моделирующей 
задержку формирования состава.
{\looseness=-1

}

Для моделирования случайных задержек по времени будем использовать 
случайную величину~$\xi_s$,
имеющую экспоненциальное распределение с~параметром~$\lambda_s$, 
которое обознается через $E(\lambda_s)$.
Для моделирования случайных задержек в~транспортных системах 
традиционно используют экспоненциально распределенные случайные величины. 
Например, в~\cite{KibzunNaumovUlanov} на основе статистического анализа данных 
был предложен экспоненциальный закон распределения времени задержки 
прибытия в~аэропорт самолета, выполняющего рейс по расписанию.

Помимо долгосрочного планирования перевозок, которое главным образом 
позволяет проводить оценку важных эксплуатационных
показателей\linebreak ра\-бо\-ты железной дороги, также существует так называемое 
оперативное планирование, которое является основным инструментом 
организации железнодорожных перевозок. Отметим, что в~работе~\cite{AzanovBuyanov} 
решается задача долгосрочного пла\-ни\-ро\-вания. 
{\looseness=-1

}

Оперативное планирование 
на железнодорожном транспорте состоит из нескольких этапов. %\\[-16pt]
\begin{enumerate}[1.]
    \item
    Разработка и~утверждение плана перевозок на период времени, равный~$T$. %\\[-16pt]
    \item
    Корректировка сформированного плана перевозок с~учетом фактического расположения 
    локомотивов и~составов,
    а~также общего состояния железнодорожной сети через интервалы времени~$\Delta T$.
\end{enumerate}

%\vspace*{-2pt}

Будем считать, что точное время готовности состава к~отправлению известно 
в~текущий момент времени на период планирования~$\Delta T$,
поэтому схема оперативного управления подвижным составом 
заключается в~последовательном решении детерминированной
задачи формирования маршрутов движения локомотивов согласно алгоритму~1 
через промежутки времени~$\Delta T$
с~учетом точного знания времени готовности составов к~отправлению на время~$\Delta T$ 
вперед и~планового времени го\-тов\-ности
составов к~отправлению в~оставшийся период времени планирования.
 
 При этом в~качестве начальных условий решения задачи о~назначении локомотивов 
каждый раз принимается
фактическое расположение локомотивов на железнодорожной сети, сложившееся на 
момент корректировки
с~учетом всех показателей функционирования локомотивов (необходимость прохождения ТО, 
плечи и~т.\,д.).

\begin{table*}\small %tabl1
\begin{center}
    \Caption{Характеристики входных данных}
\vspace*{2ex}

        \begin{tabular}{|c|c|c|c|c|c|c|c|}
         \hline
\tabcolsep=0pt\begin{tabular}{c}Число\\ станций \end{tabular}& 
\tabcolsep=0pt\begin{tabular}{c}Число\\ станций-депо\end{tabular} &
\tabcolsep=0pt\begin{tabular}{c}Число\\ сортировочных\\ станций\end{tabular}&
\tabcolsep=0pt\begin{tabular}{c}Число\\ составов\\ в~суточном\\ задании\end{tabular} &
\tabcolsep=0pt\begin{tabular}{c}Число\\ ниток\\ на сутки\end{tabular} &
\tabcolsep=0pt\begin{tabular}{c}Период\\ моделирования\\ $T_m$, сут\end{tabular} &
\tabcolsep=0pt\begin{tabular}{c}Дискретность\\ оперативного\\ управления\\
$\Delta T$, ч\end{tabular} \\
\hline
40 & 16  & 16 & 598 & 1254 & 10  & 3 \\ 
\hline
        \end{tabular}
    \end{center}
%\vspace*{-6pt}
\end{table*}

Опишем эвристический алгоритм поиска субоптимального решения задачи 
назначения локомотивов для осуществления грузоперевозок
по железнодорожной сети, реализующий схему оперативного планирования. Пусть~$T_o$~--- 
время начала моделирования.
Пусть~$S_i$~--- множество составов, которые необходимо перевезти в~интервал 
времени $[T_o\hm + i\Delta T, T_o \hm+ i\Delta T \hm+ T]$,
где $i \hm= \overline{0, [{T_m}/({\Delta T}) ]}$. Множества~$S_i$ 
становятся известными за время~$\Delta T$
до начала соответствующего интервала. Пусть для каждого состава $s \hm\in S_i$ 
задано планируемое время формирования~$t^s_o$.
Обозначим через~$\overline{\mathcal N_s}$ множество ниток, по 
которым может быть перевезен состав~$s$
из расчета планируемого времени фор\-ми\-ро\-вания.
{ %\looseness=1

}



Описанный выше алгоритм позволяет получить субоптимальное 
решение задачи~(\ref{problem_main}) при условии неопределенности
во времени готовности состава, используя принцип оперативного планирования. 
Неопределенность заключается в~отсутствии точной информации 
о~времени готовности составов к~отправлению на весь рассматриваемый период 
планирования. Согласно принципу оперативного планирования производится 
многократная корректировка плана перевозок через интервалы времени~$\Delta T$. 
В~каждый рассматриваемый интервал времени имеется точная информация 
о~времени готовности составов только из этого интервала, для остальных 
известно только планируемое время готовности, которое может сильно отличаться 
от фактического.

Проверим адекватность решения, получаемого с~помощью алгоритма~2, 
в~случае стохастической постановки
с~учетом случайного времени формирования составов. Численный эксперимент проводился 
на примере данных участка МЖД за определенный период 
времени. Характеристики входных данных приведены в~табл.~1.



Вычисления были проведены с~учетом ограничений на ТО. Предполагается, 
что локомотивы должны проходить ТО продолжительностью
не менее~8~ч не позже, чем через~48~ч после предыду\-ще\-го~ТО.

%\end{multicols}






%\begin{multicols}{2}

Согласно алгоритму~2, составление рейса состава происходит при помощи выбора 
ближайшей по времени отправления
нитки относительно фактического времени формирования.
Параметр~$\lambda_s$ распределения случайной величины~$\xi_s$ зависит 
от множества факторов, например числа вагонов в~составе,
структуры станции и~др. Так как описанная модель не включает в~себя 
процесс составообразования,
вагонопотоки и~станционные работы, вопрос выбора~$\lambda_s$ остается вне 
рамок данной работы.\linebreak\vspace*{-12pt}


\columnbreak

%\begin{table*}
{\small %tabl2
    \begin{center}
    {{\tablename~2}\ \ \small{Сравнение результатов}}
    
    \vspace*{2ex}


        \begin{tabular}{|l|c|c|c|} 
        \hline
        \multicolumn{1}{|c|}{Вариант} & 
        \tabcolsep=0pt\begin{tabular}{c}Число\\ локомо-\\ тивов\end{tabular} & 
       \tabcolsep=0pt\begin{tabular}{c} Число\\ переве-\\ зенных\\ составов\end{tabular}&
        \tabcolsep=0pt\begin{tabular}{c}Макси-\\мальное\\ время\\ задержки\\ состава, ч\end{tabular}\\
        \hline
  Решение из \cite{AzanovBuyanov} &  369& 5920& 0\hphantom{,3}\\
  $T=12$~ч &  405& 5710& 6,3\\
  $T=24$~ч &  440& 5783& 5,1\\
  $T=48$~ч & 422& 5659& 6,7\\ 
 \hline
        \end{tabular}
    \end{center}}
%\end{table*}

\vspace*{9pt}

\noindent
 Оценки параметров~$\lambda_s$ получены исходя из 
обработки реальных статистических данных.



Численный эксперимент проводился с~исходными данными, представленными в~табл.~1.
Для
 каждого $T \hm\in \{12~\mbox{ч}, 24~\mbox{ч}, 48~\mbox{ч}\}$
выполнено~100~реализаций алгоритма.
Кроме основного критерия, числа используемых локомотивов, также оценены 
выполнение плана перевозок и~максимальное время задержки составов. Для каждого~$T$ представим в~табл.~2 
средние значения для основного критерия и~описанных характеристик.



Из табл.~2 видно, что значение критерия относительно детерминированной постановки 
показало среднее изменение
в большую сторону не более чем на~20\%. При этом менее~7\%~составов не было
 перевезено, что объясняется недостатком ниток.
Максимальное время задержки составов при этом достигало около~7~ч.
В~реальных условиях управления грузовыми перевозками такие составы отправляются 
вне нормативных ниток.
Относительные доли основных характеристик движения локомотивов, а~именно: 
время, проведенное в~работе (полезный пробег);
время, затраченное на проведение ТО; время, затраченное на перегонки 
(холостой пробег), а~также время простоя~--- не изменились.

Общий локомотивный парк на МЖД составляет около~900~локомотивов, 
ежедневно используется примерно~700~локомотивов.
В~результате численного эксперимента в~худшем случае ($T = 24$~ч) 
получено~440~локомотивов,
что в~сравнении с~реальными данными является очень хорошим показателем. Однако 
следует заметить,
что такие показатели, с~одной стороны, связаны с~наличием некоторой части 
неперевезенных составов в~результате полученного решения,
а~с~другой стороны, возможно, не всеми нюансами функционирования системы
 железнодорожных перевозок,
учтенными в~виде ограничений в~рассматриваемой модели. Тем не менее 
полученные результаты показывают, что учет случайных факторов необходим в~подобных 
моделях, поскольку оказывает значительное влияние (порядка~20\%) 
на основные показатели функционирования системы даже с~учетом нового 
предложенного алгоритма оперативного управления.

\vspace*{-6pt}

\section{Заключение}

\vspace*{-2pt}

В работе описана и~исследована математическая модель назначения локомотивов 
для перевозки составов.
Результаты численных экспериментов показали, что случайные возмущения, 
связанные со временем формирования составов,
имеют  влияние порядка~20\% на значение основного критерия и~характеристики
 движения локомотивов,
что подтверждает сравнение с~результатами, полученными в~\cite{AzanovBuyanov},
а~также создает дополнительную проблему~--- нехватку ниток для перевозки 
всех составов.
В~дальнейших исследований планируется изучить влияние других случайных факторов
на эффективность использования локомотивного парка.
Также планируется учесть необходимость прохождения нескольких видов 
технического обслуживания,
ограничения на тип тяги локомотива и~массу перевозимого состава и~другие факторы.

\renewcommand{\figurename}{\protect\bf Рис.}


{\small\frenchspacing
 {%\baselineskip=10.8pt
 \addcontentsline{toc}{section}{References}
 \begin{thebibliography}{99}

\bibitem{belyi}  %1
\Au{Белый О.\,В., Кокурин И.\,М.}
Организация грузовых железнодорожных перевозок: пути оптимизации~// 
Транспорт Российской Федерации, 2011. №\,4(35). С.~28--30.

\bibitem{KibzunNaumov}  %2
\Au{Кибзун А.\,И., Наумов~А.\,В., Иванов~С.\,В.}
Двухуровневая задача оптимизации деятельности железнодорожного транспортного узла~// 
Управление большими сис\-те\-ма\-ми, 2012. №\,38. С.~140--160.

\bibitem{lazarev1} %3
\Au{Лазарев А.\,А., Мусатова Е.\,Г.}
Целочисленные постановки задачи формирования железнодорожных составов и~расписания 
их движения~// Управление большими системами, 2012. №\,38. С.~161--169.

\bibitem{lazarev2} %4
\Au{Лазарев~А.\,А., Мусатова~Е.\,Г., Гафаров~Е.\,Р., Кварацхелия~А.\,Г.}
Теория расписаний. Задачи железнодорожного планирования.~--- М.: ИПУ РАН, 2012.
92~с.

\bibitem{isuzht2015}  %5
\Au{Гайнанов Д.\,Н., Иванов С.\,В., Кибзун А.\,И., Осокин А.\,В.}
Модель оптимального назначения локомотивов при формировании грузовых составов~// 
Интеллектуальные системы управления на железнодорожном транспорте: Тр. 
4-й научн.-технич. конф. с~междунар. учас\-ти\-ем.~--- М.: НИИАС, 2015. С.~45--47.

\bibitem{cacchiani}  %6
\Au{Cacchiani V., Galli~L., Toth~P.}
A~tutorial on non-periodic train timetabling and platforming problems~// 
EURO J.~Transportation Logistics, 2015. Vol.~4. No.\,3. P.~285--320.

\bibitem{AzanovBuyanov}  %7
\Au{Azanov V.\,M., Buyanov~M.\,V., Gaynanov~D.\,N., Ivanov~S.\,V.}
Algorithm and software development to allocate locomotives for transportation 
of freight trains~//
%Вестн. ЮУрГУ. Сер. Матем. моделирование и~программирование, 9:4 (2016), 73-85.
Bull. South Ural State University. 
Ser. Math. Modelling Programming  Computer Software, 2016. 
Vol.~9. No.\,4. P.~73--85.

\bibitem{isuzht2016}  %8
\Au{Азанов В.\,М., Буянов М.\,В., Иванов С.\,В., Кибзун А.\,И., 
Наумов А.\,В., Гайнанов Д.\,Н.}
Оптимизация локомотивного парка, предназначенного для перевозки грузовых составов~// 
Интеллектуальные системы управ\-ле\-ния на железнодорожном транспорте: Тр. 5-й\linebreak 
на\-учн.-тех\-нич. конф. с~междунар. участием.~--- М.: \mbox{НИИАС}, 2016. С.~94--96.
    

\bibitem{Floyd}  %9
\Au{Floyd~R.\,W.} Algorithm~97~--- shortes path~// Comm. ACM, 1962. 
Vol.~5. No.\,6. P.~345.

\bibitem{KibzunNaumovUlanov} %10
\Au{Кибзун А.\,И., Наумов~А.\,В., Уланов~С.\,В.}
Стохастический алгоритм управления летным парком авиакомпании~// 
Автоматика и~телемеханика, 2000. №\,8. С.~126--136.

 \end{thebibliography}

 }
 }

\end{multicols}

\vspace*{-3pt}

\hfill{\small\textit{Поступила в~редакцию 17.04.17}}

\vspace*{8pt}

%\newpage

%\vspace*{-24pt}

\hrule

\vspace*{2pt}

\hrule

%\vspace*{8pt}


\def\tit{DEVELOPMENT OF~THE~MATHEMATICAL MODEL OF~CARGO TRANSPORTATION CONTROL 
ON~A~RAILWAY NETWORK SEGMENT TAKING INTO~ACCOUNT RANDOM FACTORS}

\def\titkol{Development of~the~mathematical model of~cargo transportation control 
%on~a~railway network segment 
taking into~account random factors}

\def\aut{M.\,V.~Buyanov, S.\,V.~Ivanov, A.\,I.~Kibzun, and A.\,V.~Naumov}

\def\autkol{M.\,V.~Buyanov, S.\,V.~Ivanov, A.\,I.~Kibzun, and A.\,V.~Naumov}

\titel{\tit}{\aut}{\autkol}{\titkol}

\vspace*{-9pt}


\noindent
Moscow Aviation Institute (National Research University), 
4~Volokolamskoye Highway,
Moscow 125993, Russian Federation



\def\leftfootline{\small{\textbf{\thepage}
\hfill INFORMATIKA I EE PRIMENENIYA~--- INFORMATICS AND
APPLICATIONS\ \ \ 2017\ \ \ volume~11\ \ \ issue\ 4}
}%
 \def\rightfootline{\small{INFORMATIKA I EE PRIMENENIYA~---
INFORMATICS AND APPLICATIONS\ \ \ 2017\ \ \ volume~11\ \ \ issue\ 4
\hfill \textbf{\thepage}}}

\vspace*{3pt}
    


\Abste{A mathematical model for the assignment of locomotives for the transport 
of freight trains is proposed. In the model, the purpose of optimization is to
 minimize the number of locomotives involved in transportation of trains due 
 to the choice of routes for trains and locomotives. A~deterministic algorithm 
 for obtaining a~suboptimal solution is given as well as an algorithm that 
 implements the operational planning scheme. It is proposed to use\linebreak\vspace*{-12pt}}
 
 \Abstend{a~random parameter that simulates the delay in the readiness of a~train for 
 departure. The numerical experiment was performed in conditions of incomplete 
 information using the data of the Moscow Railway. The results obtained in 
 deterministic and stochastic statements are compared.}

\KWE{mathematical modeling; optimization; transportation planning; 
operational planning}

\DOI{10.14357/19922264170411} 

\vspace*{-12pt}

\Ack
\noindent
This work is a part of Project No.\,2.2461.2017 supported 
by the Russian Ministry of Education and Science.
This work is also supported by the Russian Foundation for Basic
Research and Russian Railways (project 17-20-03050~ofi\_m\_RZhD).





%\vspace*{3pt}

  \begin{multicols}{2}

\renewcommand{\bibname}{\protect\rmfamily References}
%\renewcommand{\bibname}{\large\protect\rm References}

{\small\frenchspacing
 {%\baselineskip=10.8pt
 \addcontentsline{toc}{section}{References}
 \begin{thebibliography}{99}

\bibitem{5-bu-1} %1
\Aue{Belyy, O.\,V., and I.\,M.~Kokurin.} 2011. Organizatsiya gruzovykh 
zheleznodorozhnykh perevozok: puti optimizatsii 
[Organization of freight rail transportation: Ways to optimize].
\textit{Transport Rossiyskoy Federatsii} [Transport of the Russian Federation]
4(35):28--30.
\bibitem{2-bu-1} %2
\Aue{Kibzun, A.\,I., A.\,V.~Naumov, and S.\,V.~Ivanov.} 
2012. Dvukhurovnevaya zadacha optimizatsii deyatel'nosti 
zhe\-lez\-no\-do\-rozh\-no\-go transportnogo uzla [Bilevel optimization problem for 
railway transport hub planning]. \textit{Upravlenie bol'shimi sistemami}
[Large-Scale Systems Control] 38:140--160.
\bibitem{7-bu-1} %3
\Aue{Lazarev, A.\,A. and E.\,G.~Musatova.} 2012. Tselochislennye postanovki 
zadachi formirovaniya zheleznodorozhnykh sostavov 
i~raspisaniya ikh dvizheniya [Integer formulations of
 freight train design and scheduling problems].
 \textit{Upravlenie bol'shimi sistemami} [Large-Scale Systems Control] 38:161--169.
\bibitem{8-bu-1} %4
\Aue{Lazarev, A.\,A., E.\,G.~Musatova, E.\,R.~Gafarov, and A.\,G.~Kvarachelija.} 
2012. \textit{Teoriya raspisaniy. Zadachi zheleznodorozhnogo planirovaniya} 
[Theory of schedules. Railway planning problems]. Moscow: IPU RAN. 92~p.
\bibitem{3-bu-1} %5
\Aue{Gaynanov, D.\,N., S.\,V.~Ivanov, A.\,I.~Kibzun, and A.\,V.~Oso\-kin.} 
2015. Model' optimal'nogo naznacheniya lokomotivov pri formirovanii
 gruzovyh sostavov
[Model of the optimal assignment of locomotives in the formation of freight
trains]. \textit{Tr. 4-y nauchn.-tekhnich. konf. s~mezhdunarodnym uchastiem 
``Intellektual'nye sistemy upravleniya na zheleznodorozhnom transporte''} 
[4th Scientific and Technical Conference with International Participation 
``Intelligent Control Systems in Railway Transport'' Proceedings]. Moscow. 45--47.

\bibitem{6-bu-1} %6
\Aue{Cacchiani, V., L.~Galli, and P.~Toth.} 2015. 
A~tutorial on non-periodic train timetabling and platforming problems. 
\textit{EURO J.~Transportation Logistics} 4(3):285--320.

\bibitem{1-bu-1} %7
\Aue{Azanov, V.\,M., M.\,V.~Buyanov, D.\,N.~Gaynanov, and S.\,V.~Ivanov.} 2016.
Algorithm and software development to allocate locomotives for transportation 
of freight trains. \textit{Bull. South Ural State University. 
Ser. Math. Modelling Programming Computer Software} 9(4):73--85.
\bibitem{4-bu-1} %8
\Aue{Azanov, V.\,M., M.\,V.~Buyanov, S.\,V.~Ivanov, A.\,I.~Kibzun, A.\,V.~Naumov, 
and D.\,N.~Gaynanov.} 2016. Optimizatsiya lokomotivnogo parka, prednaznachennogo 
dlya perevozki gruzovykh sostavov [Optimization of locomotive 
fleet intended for transportation of freight trains]. 
\textit{Tr. 5-y nauchn.-tekhnich. konf. s~mezhdunarodnym uchastiem 
``Intellektual'nye sistemy upravleniya na zheleznodorozhnom transporte''} 
[5th Scientific and Technical Conference with International Participation 
``Intelligent Control Systems in Railway Transport'' Proceedings]. Moscow. 94--96.



\bibitem{9-bu-1} %9
\Aue{Floyd, R.\,W.} 1962. Algorithm~97: Shortes path. \textit{Comm. ACM} 5(6):345.
\bibitem{10-bu-1}
\Au{Kibzun, A.\,I., A.\,V.~Naumov, and S.\,V.~Ulanov.}
 2000. A~stochastic control algorithm for aircraft allocation. 
 \textit{Automat. Rem. Contr.} 61(8):1355--1363.
\end{thebibliography}

 }
 }

\end{multicols}

\vspace*{-6pt}

\hfill{\small\textit{Received April 17, 2017}}

%\vspace*{-10pt}

\Contr

\noindent
\textbf{Buyanov Mikhail V.} (b.\ 1994)~--- 
PhD student, Moscow Aviation Institute (National Research University), 
4~Volokolamskoye Highway, 
Moscow 125993, Russian Federation; \mbox{buyanovmikhailv@gmail.com}

\vspace*{3pt}

\noindent
\textbf{Ivanov Sergey V.} (b.\ 1989)~--- 
Candidate of Science (PhD) in physics and mathematics,  
associate professor, Moscow Aviation Institute (National Research University), 
4~Volokolamskoye Highway,
Moscow 125993, Russian Federation;  \mbox{sergeyivanov89@mail.ru} 


\vspace*{3pt}

\noindent
\textbf{Kibzun Andrey I.} (b.\ 1951)~--- 
Doctor of Science in physics and mathematics, professor,  
Head of Department, Moscow Aviation Institute (National Research University), 
4~Volokolamskoye Highway,
Moscow 125993, Russian Federation;  \mbox{kibzun@mail.ru} 

\vspace*{3pt}

\noindent
\textbf{Naumov Andrey V.} (b.\ 1966)~--- 
Doctor of Science in physics and mathematics, associate professor,  
professor, Moscow Aviation Institute (National Research University), 
4~Volokolamskoye Highway,
Moscow 125993, Russian Federation;  \mbox{naumovav@mail.ru} 
\label{end\stat}


\renewcommand{\bibname}{\protect\rm Литература}   %11
\def\stat{bitugov}

\def\tit{ПРИМЕНЕНИЕ ВЕЙВЛЕТОВ ДЛЯ РАСЧЕТА ЛИНЕЙНЫХ СИСТЕМ УПРАВЛЕНИЯ  
С~СОСРЕДОТОЧЕННЫМИ ПАРАМЕТРАМИ$^*$}

\def\titkol{Применение вейвлетов для расчета линейных систем управления  
с~сосредоточенными параметрами}

\def\aut{Ю.\,И.~Битюков$^1$, Е.\,Н.~Платонов$^2$}


\def\autkol{Ю.\,И.~Битюков, Е.\,Н.~Платонов}

\titel{\tit}{\aut}{\autkol}{\titkol}

\index{Битюков Ю.\,И.}
\index{Платонов Е.\,Н.}
\index{Bityukov Yu.\,I.}
\index{Platonov E.\,N.}



{\renewcommand{\thefootnote}{\fnsymbol{footnote}} \footnotetext[1]
{Результаты работы получены в~рамках выполнения государственного 
задания Минобрнауки №\,2.2461.2017/ПЧ.}}


\renewcommand{\thefootnote}{\arabic{footnote}}
\footnotetext[1]{Московский авиационный институт (национальный исследовательский университет),  
\mbox{yib72@mail.ru}}
\footnotetext[2]{Московский авиационный институт (национальный исследовательский 
университет),  \mbox{en.platonov@gmail.com}}

%\vspace*{-18pt}

\Abst{Задачи многих дисциплин могут привести к~дифференциальным и~интегральным 
уравнениям. В~простых случаях такие уравнения могут быть решены аналитически, 
но в~более сложных приходится находить приближенные решения этих уравнений. 
В~последнее время большую популярность получили методы, основанные на использовании 
вейвлетов. Среди применяемых были вейвлеты Дебеши, койфлеты и~т.\,д. 
Недостаток таких вейвлетов состоит в~том, что у~них нет аналитического выражения. 
Поэтому возникают большие сложности при интегрировании и~дифференцировании выражений, 
содержащих эти вейвлеты. В~данной статье представлены алгоритмы численного 
решения линейных интегральных и~дифференциальных уравнений, основанные на 
сплайн-вейв\-ле\-тах на отрезке. Представленные алгоритмы обобщают 
известные методы, основанные на вейвлетах Хаара, которые являются частным 
случаем сплайн-вейв\-ле\-тов. Результаты статьи применяются для анализа 
линейных систем управ\-ле\-ния с~сосредоточенными параметрами.}

\KW{сплайн-вейвлет; дифференциальные уравнения; интегральные уравнения}

\DOI{10.14357/19922264170412} 


\vskip 10pt plus 9pt minus 6pt

\thispagestyle{headings}

\begin{multicols}{2}

\label{st\stat}

\section{Введение} 

Для численного решения линейных интегральных уравнений традиционно 
применяется метод, основанный на замене интегрального уравнения 
алгебраической системой линейных уравнений с~помощью применения 
квадратурной формулы. Мат\-ри\-ца такой системы имеет большой размер, и,~как следствие, 
для нахождения решения требуется большое число арифметических операций. 

В~\cite{Lepik3} было предложено использовать вейвлеты Хаара 
для приближенного решения интегрального уравнения, что приводило 
к~системе линейных уравнений с~разреженной матрицей. Получаемое приближенное 
решение было ку\-соч\-но-не\-пре\-рыв\-ным. 

В~\cite{Blatov}  показано, что, если использовать вместо вейвлетов Хаара 
сплайн-вейв\-ле\-ты на отрезке, матрица системы линейных уравнений получается 
псевдоразреженной, т.\,е.\ имеет очень много малых по модулю элементов. 

В~данной статье будут обобщены результаты 
работ~\cite{Lepik1, Lepik2, Lepik3, Lepik4, Lepik} и~развиты результаты 
работы~\cite{Blatov} для получения приближенных решений любого класса гладкости 
линейных интегральных и~дифференциальных уравнений. 
В~качестве примера рассмотрим анализ линейной системы управления  с~сосредоточенными 
параметрами.

\section{Сплайн-вейвлеты на~отрезке}

В этом разделе  кратко рассмотрим подход к~построению вейв\-лет-сис\-тем 
на отрезке, предложенный в~\cite{ArticleFinkelstein}. Пусть действительная 
функция~$\varphi $ принадлежит действительному пространству 
$\mathrm{L}^{2} \left({\bf R}\right)$, удовлетво\-ряет равенству
\begin{equation} 
\label{Pr1}
\varphi \left(x\right)\hm=\sqrt{2} \sum\limits_{k\in {\bf Z}}u_{k} 
\varphi \left(2x-k\right)\,,\enskip u_{k} \in \mathbf{R}\,,
\end{equation}
и имеет компактный носитель, содержащийся в~отрезке $[a;b]$. 
Обозначим $\varphi _{jk} (x)\hm=2^{{j}/{2}}\varphi \left(2^{j} x-k\right)$,
$x\hm\in [a;b]$. Функция~$\varphi $ в~теории вейвлетов называ\-ется масштабирующей,
 а~равенство~\eqref{Pr1}~--- масштаб\-ным соотношением~\cite{Frazer}. 
 Ясно, что отличными\linebreak от нуля на отрезке $[a;b]$ будет лишь конечное чис\-ло
  таких функций. Пусть для определенности это будут функции $\varphi _{j,0},
\varphi _{j,1} ,\ldots,\varphi _{j,n_{j} -1}$. 

Если рассмотреть линейные пространства $ V_{j} \hm= \mathrm{lin}
\left\{\varphi _{j,0} ,\varphi _{j,1} ,\ldots,\varphi _{j,n_{j} -1} \right\}$,
$\dim V_{j} \hm=n_{j}$, то  в~силу равенства~\eqref{Pr1}  будет выполняться 
$V_{0} \hm\subset V_{1} \subset \cdots$\linebreak $\cdots \subset L^{2} \left[a;b\right]$. 
Поэтому $\varphi _{j-1,k} \hm=\sum\nolimits _{s=0}^{n_{j} -1}p_{s,k} \varphi _{j,s}$. 

Как и~в~\cite{ArticleFinkelstein}, введем обозначения: 

\noindent
\begin{align*}
\Phi _{j} (x)&=
\left(\varphi _{j,0} (x),\varphi _{j,1} (x),\ldots,\varphi _{j,n_{j} -1} (x)\right)\,;\\
{P}_{j} &=\left(p_{s,k} \right)_{s=0, k=0}^{n_{j} -1, n_{j-1} -1}\,.
\end{align*}
Тогда $\Phi _{j-1}\hm =\Phi _{j} {P}_{j} $. 

\pagebreak

Обозначим символом~$W_{j-1} $ 
ортогональное дополнение к~пространству~$V_{j-1} $ в~пространстве~$V_{j}$. 
Поскольку $V_{j}\hm =V_{j-1} \oplus W_{j-1} $ и~$W_{j-1} \hm\subset V_{j} $, 
то~$W_{j-1} $~--- конечномерное пространство $ W_{j} \hm = 
\mathrm{lin} \left\{\psi _{j,0} ,\psi _{j,1} ,\ldots,\psi _{j,m_{j} -1} \right\}$,
$\dim W_{j} \hm=m_{j}$ и~$\psi _{j-1,k} \hm=
\sum\nolimits_{s=0}^{n_{j} -1}q_{s,k}^{j} \varphi _{j,s}.$ 
Функции~$\psi _{j,k} $ называются вейвлетами, а~пространства~$W_{j} $ 
называются вейв\-лет-про\-стран\-ст\-вами.

Снова введем в~рассмотрение матрицы~\cite{ArticleFinkelstein}:
\begin{align*}
\Psi _{j} (x)&=\left(\psi _{j,0} (x),\psi _{j,1} (x),\ldots,\psi_{j,m_{j} -1}
 (x)\right)\,;\\
{Q}_{j}&=\left(q_{s,k}^{j} \right)_{s=0, k=0}^{n_{j}-1,m_{j-1}-1}\,.
\end{align*}
Тогда $\Psi _{j-1} \hm=\Phi _{j} {Q}_{j} $. 
Следует заметить, что $n_{j} \hm+m_{j} \hm=n_{j+1} $.
Пусть $f\hm\in {L}^{2} (X)$ и~$\Pi _{j} : {L}^{2} (X)\hm\to V_{j} $. Тогда
\begin{multline*}
\Pi _{j} f=\sum\limits_{k=0}^{n_{j} -1}c_{jk} \varphi _{jk}  =
\Pi _{j-1} f+\Pi _{j-1}^{W} f={}\\
{}=\sum\limits_{k=0}^{n_{j-1} -1}c_{j-1,k} \varphi _{j-1,k}  +
\sum\limits_{k=0}^{m_{j-1} -1}d_{j-1,k} \psi _{j-1,k}\,.
\end{multline*}
Данное равенство можно переписать в~матричном виде, если ввести в~рассмотрение 
векторы ${C}_{j} \hm=\left(c_{j,0} ,\dots,c_{j,n_{j} -1} \right)^{\mathrm{T}}$,
${D}_{j} \hm=\left(d_{j,0} ,\dots,d_{j,m_{j} -1} \right)^{\mathrm{T}}$. 
Первый вектор описывает приближение функции~$f$, а~второй вектор представляет 
собой вейв\-лет-ко\-эф\-фи\-ци\-ен\-ты, которые характеризуют отклонение
$\Pi _{j-1} f$ от~$\Pi _{j} f$. Как показано в~\cite{ArticleFinkelstein}, 
имеет место равенство
${C}_{j} \hm={P}_{j} {C}_{j-1} \hm+{Q}_{j} {D}_{j-1}.$
По данному равенству можно восстановить приближение~$\Pi _{j} f$ 
по более грубому приближению~$\Pi _{j-1} f$ и~вейв\-лет-ко\-эф\-фи\-ци\-ен\-там.

Поскольку линейные операторы (проекторы) $V_{j} \hm\to V_{j-1}$,
$V_{j} \hm\to W_{j-1} $ определяются некоторыми матрицами~${A}_{j}$,
${B}_{j}$, то ${C}_{j-1} \hm={A}_{j} {C}_{j}$,
${D}_{j-1} \hm={B}_{j} {C}_{j}$.

Под вейвлет-преобра\-зо\-ва\-ни\-ем функции~$f$ будем понимать 
нахождение векторов ${C}_{0}$, ${D}_{0}$, ${D}_{1} ,
\dots, D_{j-1}$. Матрицы~${Q}_j$ и~${P}_j$ 
известны как фильт\-ры синтеза. Матрицы~${A}_j$ и~${B}_j$ 
известны как фильт\-ры анализа. Множество 
$\{{A}_j, {B}_j, {P}_j,{Q}_j\}$ 
называется банком фильтров.

Как показано в~\cite{ArticleFinkelstein}, между 
матрицами~${A}_{j}$, ${B}_{j}$ и~${P}_{j}$, 
${Q}_{j} $ существует следующая связь:
$$
\begin{pmatrix} {A}_{j} \\ {B}_{j} \end{pmatrix}=
\begin{pmatrix} {P}_{j} {Q}_{j}\end{pmatrix}^{-1}\,.
$$

Посмотрим теперь, как определить мат\-ри\-цу~${Q}_{j}$. 
Введем следующее обозначение. Если ${f}\hm=\left(f_{1} ,\dots, f_{r} \right)$,
${g}\hm=\left(g_{1} ,\dots,g_{r} \right)$~--- 
некоторые векторы, то $[({f},{g})]\hm=
\left(\left(f_{i} ,g_{j} \right)\right)_{i,j=1}^{r} $~--- 
мат\-ри\-ца скалярных произведений.  Как показано в~\cite{ArticleFinkelstein}, 
мат\-ри\-ца~${Q}_{j} $ удовлетворяет следующему уравнению: 
${P}_{j}^{\mathrm{T}} \left[\left(\Phi _{j}, \Phi _{j} 
\right)\right]{Q}_{j} \hm=0.$

Перейдем теперь к~сплайн-вейв\-ле\-там на отрезке. Определим В-сплай\-ны порядка~$n$  
как свертку~\cite{Chui}:
$$
N_{n} =N_{n-1} *N_{0}\,,\quad
N_{0} (x)=\begin{cases} 
1\,, & x\in [0;1)\,; \\ 
0\,, & x\notin [0;1)\,.
\end{cases}
$$
Как показано в~\cite{Chui}, если определить функцию $\varphi (x)\hm=N_{n} (x)$, 
то она удовлетворяет равенству $\varphi (x)\hm=\sum\nolimits _{k=0}^{n+1}
({C_{n+1}^{k} }/{2^{n}}) \varphi (2x\hm-k)$, где $C_{n+1}^k\hm={(n+1)!}/({k!(n\hm+1\hm-k)!})$.
В~\cite{Yurgu} пред\-став\-лен банк фильт\-ров, соответствующий функции  
$\varphi (x)\hm=N_{n} (x)$, а~именно:  справедливы следующие результаты.

\smallskip

\noindent
\textbf{Лемма~2.1.}\
\textit{Функция $\varphi(x)\hm=N_n(x)$ определяет последовательность подпространств}
\begin{multline*}
V_{\alpha,0}\subset V_{\alpha,1}\subset\cdots,\\
V_{\alpha,j}=\mathrm{lin}\left\{\varphi_{j,-n},\varphi_{j,-n+1},\dots,
\varphi_{j,2^j\alpha(n+1)-1}\right\}
\end{multline*}
\textit{пространства} ${L}^2[0;\alpha(n+1)]$, $\alpha\hm=1,2,\ldots$, 
\textit{такую, что} 
$\overline{\bigcup\nolimits_{j=0}^{+\infty}V_{\alpha,j}}\hm={L}^2
[0;\alpha(n+1)]$.

\smallskip

\noindent
\textbf{Лемма~2.2.}\
\textit{Имеет место равенство} 
$\sum\nolimits_{k=-n}^{2^j\alpha (n+1)-1} \varphi_{j,k}(x) \hm\equiv 
2^{{j}/{2}}$, $x\hm\in [0;\alpha (n+1)].
$
\textit{Если}  $V_{\alpha ,j} \hm=V_{\alpha ,j-1} \oplus W_{\alpha ,j-1} $, 
\textit{то} $\dim W_{\alpha ,j-1}\hm =2^{j-1} \alpha (n+1)$.

\smallskip

Пусть $\lambda_{m,k}\hm=\int\nolimits_k^{k+1} N_n(z)N_n(z-m)\,dz$, 
$m\hm=-n,\ldots ,n$, $k\hm=0,1,\ldots ,n$, и~$\omega_{i,k}\hm=\omega_{k,i}
\hm=\sum\nolimits_{s=n-i+1}^n \lambda_{k-i,s}$, 
$\theta_{i,k}\hm=\theta_{k,i}\hm=\sum\nolimits_{s=0}^{n-k} \lambda_{i-k,s}$, 
$1\hm\leqslant i \hm\leqslant k \hm\leqslant n.$ Введем в~рассмотрение 
вектор\linebreak ${p}\hm\in \textbf{R}^{2^j\alpha (n\hm+1)\hm+n}$, 
который определим равенством:
\begin{multline*}
%\label{vecp}
{p}={}\\
{}=\begin{cases}
 \begin{pmatrix} 
 C_{n+1}^{0} \cdots  C_{n+1}^{k}&C_{n+1}^{k} \cdots  C_{n+1}^{0}&0 \cdots 0
 \end{pmatrix}^{\mathrm{T}},
 &\\
 & \hspace*{-30mm}\mbox{ если } n=2k\,;\\
\begin{pmatrix} 
C_{n+1}^{0} \cdots C_{n+1}^{k}&C_{n+1}^{k+1}&C_{n+1}^{k} \cdots  C_{n+1}^{0}&0 \cdots 0
\end{pmatrix}^{\mathrm{T}}\!,\hspace*{-10.94377pt}
&\\
&\hspace*{-30mm} \mbox{ если } n=2k+1\,.
\end{cases}
\end{multline*}
Определим оператор сдвига~$R_s$:  $\textbf{R}^m \hm\rightarrow  \textbf{R}^m$ 
сле\-ду\-ющим правилом:
\begin{multline*}
\!\!\!\!\!R_s {a}=\!\begin{cases} 
\begin{pmatrix} 
\underbrace{0 \cdots 0}_s & a_1 \cdots a_{m-s}\end{pmatrix}^{\mathrm{T}}\!, &\!\!\!\! \mbox{ если } 
0\leqslant s < m\,;\\
\begin{pmatrix} 
a_{|s|+1} \cdots  a_m&0 \cdots 0\end{pmatrix}^{\mathrm{T}}\!, &\!\!\!\! \mbox{ если }
 -m<s<0\,;\hspace*{-10.8pt}\\
 0\,, &\!\!\!\! \mbox{ если } \vert s\vert \geqslant m\,,
 \end{cases}
 \end{multline*}
 где
 $$
{a}=\left(a_1,\dots,a_m\right)^{\mathrm{T}}\,.
$$


%\smallskip

\noindent
\textbf{Лемма 2.3.}\ 
\textit{Матрицы ${P}_j$ и~$[(\Phi_j,\Phi_j)]$ 
для последовательности подпространств  $V_{\alpha,0}\hm\subset V_{\alpha,1}
\subset\cdots$ имеют вид}:

\pagebreak

\noindent
\begin{equation*}
{P}_j=\fr{1}{2^{n+{1}/{2}}}
\begin{pmatrix} 
R_{-n}{p}&R_{-n+2}{p}&\cdots&R_{n-2+2^j\alpha (n+1)}{p}
\end{pmatrix};
\end{equation*}

\vspace*{-12pt}

\noindent
\begin{multline*}
[(\Phi_j,\Phi_j)]=\left( 
{d}_1\  \cdots\  {d}_n\ {q}\ 
R_1{q}\ \cdots\right.\\
\left.\cdots  \ R_{2^j\alpha (n+1)-n-1}
{q}\ {u}_1\ \cdots\ {u}_n\right)^{\mathrm{T}}\,,
\end{multline*}
\textit{где }
\begin{align*}
&{d}_s=
\begin{pmatrix} \omega_{1,s}&\omega_{2,s}&\cdots&\omega_{n,s}&q_{n-s+1}&\cdots&
q_n&0 \cdots 0\end{pmatrix}^{\mathrm{T}}\,;
\\
&{u}_s=
\begin{pmatrix}0 \cdots 0&q_n&\cdots&q_s&\theta_{1,s}&\cdots&\theta_{n,s}\end{pmatrix}^{\mathrm{T}}\,;\\
&{q}=
\begin{pmatrix} q_n & q_{n-1}  \cdots  q_1 & q_0 & q_1  \cdots  q_{n-1} & q_n &
 0 \cdots 0 \end{pmatrix}^{\mathrm{T}}   \in{}\\
& \hspace*{8mm}{}\in  \mathbf{R}^{2^j\alpha (n+1)+n}\,,
 \enskip q_{k} =\left(N_{n}(\cdot),N_{n} (\cdot -k)\right).
\end{align*}
\textit{Матрица, транспонированная к~${T}_j\hm={P}_j^{\mathrm{T}}
 [(\Phi_j,\Phi_j)]\hm=2^{-n-{1}/{2}}
 (t_{i,s})_{i=1,s=1}^{2^{j-1}\alpha (n+1)+n,~2^j\alpha (n+1)+n}$,
имеет вид}:
\begin{multline*}
{T}_j^{\mathrm{T}}=\fr{1}{2^{n+{1}/{2}}}
\left( 
{L}_1 \cdots {L}_n\  \  {w}\ \  R_2 {w} \cdots \right.\\
\cdots
R_{2^j\alpha (n+1)-2n-2} {w}\ \  {L}_{2^{j-1}\alpha (n+1)+1} \cdots \\
\left.\cdots {L}_{2^{j-1}\alpha (n+1)+n}\right)\,,
\end{multline*}
\textit{где} 
\begin{multline*}
\hspace*{-2pt}{w}=\begin{pmatrix}
{p}^{\mathrm{T}} R_{-2n}{q}&{p}^{\mathrm{T}} R_{-2n+1}{q} 
\cdots {p}^{\mathrm{T}} R_{n+1}{q}&0 \cdots 0 
\end{pmatrix}^{\mathrm{T}}
 \in{}\\
 {}\in \mathbf{R}^{2^j\alpha (n+1)+n}\,;
\end{multline*}

\vspace*{-12pt}

\noindent
\begin{multline*}
{L}_i={}\\
{}=\begin{pmatrix} 
\left(R_{-n+2i-2}{p}\right)^{\mathrm{T}} {d}_1 \cdots 
\left(R_{-n+2i-2}{p}\right)^{\mathrm{T}} {d}_n&0 \cdots 0
\end{pmatrix}^{\mathrm{T}}+ {}\\
{}+
\left(R_n\circ R_{-3n+2i-2}\right){w},\enskip
i=1,\dots,n\,;
\end{multline*}

\vspace*{-12pt}

\noindent
\begin{multline*}
\hspace*{-8.5727pt}{L}_{i+1}=\begin{pmatrix} 
0 \cdots 0&\left(R_{-n+2i}{p}\right)^{\mathrm{T}} {u}_1 
\cdots \left(R_{-n+2i}{p}\right)^{\mathrm{T}}{u}_n
\end{pmatrix}^{\mathrm{T}}\!  
+{}\\
{}+\left(R_{-n}\circ R_{-n+2i}\right){w}\,,
\end{multline*}

\vspace*{-12pt}

\noindent
$$
\hspace*{8mm}i=2^{j-1}\alpha (n+1),\dots,n-1+2^{j-1}\alpha (n+1)\,.
$$

С использованием леммы~2.3  в~\cite{Yurgu} найдены $2^{j-1}\alpha (n+1)$ 
линейно независимых решений ${h}_s\hm=
(h_{1,s},h_{2,s},\dots,h_{2^j\alpha (n+1)+n,s})^{\mathrm{T}}$ 
системы линейных уравнений  ${T}_j {h}_s\hm=0$. 
Эти решения и~представляют собой столбцы матрицы ${Q}_j\hm=
({h}_1,\dots,{h}_{2^{j-1}\alpha (n+1)}).$
Столбцы ${h}_{s} $ выбирались таким образом, чтобы функции
$$
\psi _{j-1,s} (x)=\Phi_j(x){h}_s=
\sum\limits_{i=1}^{2^{j} \alpha (n+1)+n}h_{i,s}  \varphi _{j,-n+(i-1)}(x)
$$
по возможности представляли собой сдвинутые версии одной функции, т.\,е.\ 
имели бы одну форму (за исключением, конечно, граничных вейвлетов).  
Введем сокращенные обозначения для матриц, составленных из элементов
 матрицы~${T}_j$:
$$
T_j\left(\begin{smallmatrix} 
i_1,\dots,i_k \\ j_1,\dots,j_m \end{smallmatrix} 
\right) = 
\begin{pmatrix} t_{i_1,j_1} & \cdots & t_{i_1,j_m} \\ 
\vdots & \vdots&\vdots \\ 
t_{i_k,j_1} & \cdots & t_{i_k,j_m} 
\end{pmatrix}.
$$
Для внутренних вейвлетов (носитель содержится в~отрезке $[0;\alpha (n+1)]$):
\begin{multline*}
{h}_s=(0,\dots,0,h_{2s-n-1,s},\dots, h_{2s+2n,s},0,\dots,0)^{\mathrm{T}},
\\
 s=n+1,\dots,2^{j-1}\alpha (n+1)-n\,,
\end{multline*}
 где $ T_j\left(\begin{smallmatrix} s-n,\dots,s+2n \\ 
 2s-n-1,\dots,2n+2s \end{smallmatrix} \right)(h_{2s-n-1,s},\dots,h_{2s+2n,s})^{\mathrm{T}}\hm=0.$

Решения, соответствующие граничным вейвлетам, выбираются следующим образом. 
Для $s\hm=1,2,\dots,n$ положим
$$
{h}_{s}=(0,\dots,0,h_{s,s},\dots,h_{2n+2s,s},0,\dots,0)^{\mathrm{T}},
$$
где $T_j\left(\begin{smallmatrix} 1,\dots,s+2n \\ 
s,\dots,2s+2n\end{smallmatrix} \right)(h_{s,s},\dots,h_{2s+2n,s})^{\mathrm{T}}\hm=0.$
Для $s\hm=2^{j-1}\alpha (n+1)\hm-n+1,\dots ,2^{j-1}\alpha (n+1)$ положим 
\begin{multline*}
{h}_{s}=\left(
0,\dots,0,h_{2s-n-1,s},\dots\right.\\
\left.\dots,h_{2^{j-1}\alpha (n+1)+n+s,s},0,
\dots,0\right)^{\mathrm{T}},
\end{multline*}
где
\begin{multline*}
T_j\left(\begin{smallmatrix} 
s-n,\dots,n+2^{j-1}\alpha (n+1) \\ 2s-n-1,\dots,2^{j-1}\alpha (n+1)+n+s 
\end{smallmatrix} \right)={}\\
{}=\left(h_{2s-n-1,s},\dots,h_{2^{j-1}\alpha (n+1)+n+s,s}\right)^{\mathrm{T}}.
\end{multline*}

Кратко рассмотрим применение вейв\-лет-сис\-тем на отрезке к~построению 
двумерных вейвлетов на прямоугольной области. Пусть даны последовательности $V_{0,i} 
\hm\subset V_{1,i} \subset \cdots \subset V_{j,i} \subset \cdots$ 
конечномерных подпространств ${L}^{2} [a_{i} ;b_{i} ]$, 
масштаби\-ру\-ющие функции~$\varphi ^{(i)} $ и~банки фильтров 
${P}_{j,i}$, ${Q}_{j,i}$, ${A}_{j,i}$, ${B}_{j,i}$, 
$ i\hm=1,2$. Стандартный подход к~построению многомерных вейв\-лет-сис\-тем~--- 
это взятие тензорных произведений функций из одномерных базисов~\cite{Novikov}. 
Определим подпространства $V_{j}^{2} \hm=V_{j,1}\;\otimes$\linebreak
$\otimes\;V_{j,2} \hm= \mathrm{lin}
\left\{f_{1} \otimes f_{2} : f_{1} \hm\in V_{j,1},\ f_{2} \hm\in V_{j,2} \right\}$, 
где функция $f_{1} \otimes f_{2} $ определяется правилом 
$f_{1} \hm\otimes f_{2} \left(x,y\right)\hm=f_{1} (x)f_{2} (y)$. 
Ясно, что функции $\varphi _{j,k}^{(1)} \otimes \varphi _{j,s}^{(2)} $ 
образуют базис в~пространстве~$V_{j}^{2} $.  Вейв\-лет-про\-стран\-ст\-ва~$W_{j}^{2} $ 
определяются следующим образом: 
$$
V_{j}^{2} =V_{j-1}^{2} \oplus W_{j-1}^{2} \,.
$$

Следующие две леммы очевидны.

\begin{figure*}[b] %fig1
\vspace*{1pt}
 \begin{center}
 \mbox{%
 \epsfxsize=162.046mm 
 \epsfbox{bit-1.eps}
 }
 \end{center}
\vspace*{-9pt}
\Caption{Графики функций $w_l$ для $n\hm=5$}
\end{figure*}


\noindent
\textbf{Лемма 2.4.}\ 
\textit{Пусть $f\hm\in {L}^{2}[0;n+1]$, тогда 
$\Pi _{j} f=\Phi _{j} {C}_{j}^{*} $, где
 ${C}_{j}^{*}\hm =[(\Phi _{j} ,\Phi _{j})]^{-1}[(f,\Phi _{j})]$. 
 При этом} 
 $$
 \| f-\Pi _{j} f\| _{{L}^{2} }^{2} =
 \| f\| _{{L}^{2} }^{2} -\left[(f,\Phi _{j})\right]^{\mathrm{T}}
 \left[(\Phi _{j} ,\Phi _{j})\right]\left[(f,\Phi _{j})\right].
 $$

\smallskip

\noindent
\textbf{Лемма 2.5.}\ 
\textit{Пусть $f\in {L}^2([a_1;b_1]\times [a_2;b_2])$ 
и~$\Pi_j^{(2)} : {L}^2([a_1;b_1]\times [a_2;b_2])\hm\to V_j^2$~--- 
проектор. Если}
\begin{multline*}
%\label{G35}
{G}_j={}\\
{}=\left(\int\limits_{\,\,\,a_1}^{b_1}\,dx\!
\int\limits_{a_2}^{b_2}\!\varphi_{j,s}^{(1)}(x)\varphi_{j,k}^{(2)}(y)f(x,y)\,dy
\right)_{s,k=0}^{n_{j,1}-1,n_{j,2}-1},\hspace*{-5.4785pt}
\end{multline*}
\textit{то $\Pi_j^{(2)}f (x,y)\hm=\Phi_j^{(1)}(x)\mathrm{C}_j (\Phi_j^{(2)}
(y))^{\mathrm{T}}$, где $\Phi_j^{(i)}\hm=(\varphi_{j,0}^{(i)}\cdots \varphi_{j,n_{j,i}-1}^{(i)})$, 
а~матрица~${C}_j$ определяется равенством}:
\begin{equation*}
%\label{Cj35}
{C}_j=\left[\left(\Phi_j^{(1)},\Phi_j^{(1)}\right)\right]^{-1}
{G}_j\left[\left(\Phi_j^{(2)},\Phi_j^{(2)}\right)\right]^{-1}\,.
\end{equation*}

%\vspace*{-24pt}

\section{Интегралы от~сплайн-вейвлетов}

Пусть ${Q}_j=({h}_1^j,\dots,{h}_{2^{j-1}(n+1)}^j)$, 
где ${h}_s^j\hm=(h_{1,s}^j,h_{2,s}^j,\dots,h_{2^j (n+1)+n,s}^j)^{\mathrm{T}}$. 
Тогда,   согласно результатам предыдущего раздела,

\noindent
\begin{multline}
\label{U31}
\psi_{j-1,s}(x)={}\\
{}=
\begin{cases}
\displaystyle\sum\limits_{i=s}^{2s+2n} \! h_{i,s}^j\varphi_{j,-n+i-1}(x)\,,&\hspace*{-10mm}
s=1,\dots,n\,;\\
\displaystyle\sum\limits_{i=2s-n-1}^{2s+2n}\!\!
h_{i,s}^j\varphi_{j,-n+i-1}(x)\,,&\\ 
&\hspace*{-38mm}s=n+1,\dots,2^{j-1}(n+1)-n\,;
\\
%\label{U33}
\displaystyle\sum\limits_{i=2s-n-1}^{2^{j-1}(n+1)+n+s}\!\!\!\!
h_{i,s}^j
\varphi_{j,-n+i-1}(x)\,,&\\
&\hspace*{-50mm}s=2^{j-1}(n+1)-n+1,\dots,2^{j-1}(n+1)\,.
\end{cases}
\end{multline}


Так же, как и~в~работах~\cite{Lepik1, Lepik2, Lepik3, Lepik4, Lepik}, для удобства  
введем следующие обозначения:
\begin{align*}
&w_l(x)=\varphi_{0,l-n-1}\,,\enskip l=1,2,\dots,2n+1\,,\\
&w_l(x)=\psi_{j,s}(x)\,,\enskip l=2^j(n+1)+n+s\,, \\
&\hspace*{23mm} j=0,1,\dots\,, \ s=1,\dots,2^{j}(n+1)\,.
\end{align*}
На рис.~1 представлены графики некоторых функций~$w_l$ для случая $n\hm=5$.




Пусть $J\geqslant 0$,  $\Pi_{J} : {L}^2[0;n+1]\hm\to V_{J}$~--- 
проектор и~$M\hm=2^{J}(n+1)+n$. Обозначим ${H}_J\hm=
\begin{pmatrix}w_1 & \cdots & w_M
\end{pmatrix}$ и~введем в~рассмотрение матрицу скалярных 
произведений $[({H}_J, {H}_J)]$. 
В~лемме~2.3 представлены матрицы скалярных произведений 
$[(\Phi_k, \Phi_k)]$ для всех $k\hm=0,1,\dots$ Замечая, что $\Psi_k \hm= 
\Phi_{k+1}{Q}_{k+1}$ и~$[(\Psi_k,\Psi_k)]\hm={Q}_{k+1}^{\mathrm{T}}
[( \Phi_{k+1}, \Phi_{k+1})]{Q}_{k+1}$,
получаем матрицу:
{\small \begin{multline*}
\left[({H}_J, {H}_J)\right]={}\\
\!{}=\!
\begin{pmatrix}
\left[(\Phi_0, \Phi_0)\right] & 0 & 0 & \cdots & 0\\
0 & {Q}_{1}^{\mathrm{T}}\left[( \Phi_{1}, \Phi_{1})\right]{Q}_{1} & 0 & \cdots & 0 \\
\vdots & \ddots & \ddots & \ddots & \vdots \\
0 & 0 & 0 & \cdots & {Q}_{J}^{\mathrm{T}}\left[( \Phi_{J}, \Phi_{J})\right]{Q}_{J}
\end{pmatrix}\!.\hspace*{-12.6139pt}
\end{multline*}
}

\noindent
Так как $V_{J}\hm=V_0\oplus W_0\oplus V_1\oplus\dots\oplus W_{J-1},$ то для 
$f\hm\in {L}^2[0;n+1]$ имеем
$\Pi_{J} f\hm=\sum\nolimits_{l=1}^{M} c_l w_l \hm= {H}_{J}{C}_{J}$,
где ${C}_J\hm=
\begin{pmatrix} c_1 & \cdots\ c_{M} \end{pmatrix}^{\mathrm{T}}$.
Как и~в~работах~\cite{Lepik1, Lepik2, Lepik3, Lepik4, Lepik}, определим функции:
\begin{equation}
\label{U38}
{\xi}_{1,l}(x)=\int\limits_0^x w_l (t)\,dt\,;
\end{equation}

%\vspace*{-12pt}

\noindent
\begin{multline}
\label{U38-1}
{\xi}_{\nu+1,l}(x)=\int\limits_0^x {\xi}_{\nu,l}(t)\,dt={}\\
{}=
\fr{1}{\nu !}\int\limits_0^x (x-t)^{\nu} w_l (t)\,dt\,,
\enskip \nu = 1,2,\ldots
\end{multline}
Согласно определению функций~$w_l$ и~равенст\-вам~(\ref{U31})  
функция ${\xi}_{\nu+1,l}(x)$ представляет собой линейную комбинацию функций
$$
\eta_{n,\nu}^{j,s}(x)=\int\limits_0^x (x-t)^{\nu} N_{n}(2^j t-s)\,dt\,.
$$

\vspace*{-1pt}

\noindent
\textbf{Лемма 3.1.}\ 
\textit{Имеет место следующее рекуррентное соотношение}:

\noindent
\begin{multline}
\label{U40}
\eta_{n,\nu}^{j,s}(x)=\fr{x^{\nu+1}}{\nu+1}\, N_n(-s)+{}\\
{}+
\fr{2^j}{\nu+1}\left(\eta_{n-1,\nu+1}^{j,s}(x)-\eta_{n-1,\nu+1}^{j,s+1}(x)\right)\,,
\end{multline}
\textit{где}

\noindent
\begin{multline}
\label{U41}
\eta_{0,\nu}^{j,s}(x) = {}\\
\!\!\!{}=
\begin{cases}
\fr{(x-a)^{\nu+1} - (x-b)^{\nu+1}}{\nu+1}, &\\[6pt]
& \hspace*{-37mm}\mbox{если } [a;b]=[0;x]
\cap\left[\fr{s}{2^j};\fr{s+1}{2^j}\right]\not= \varnothing\,;\\[9pt] 
0, & \hspace*{-37mm}\mbox{если } [a;b]=[0;x]\cap\left[\fr{s}{2^j};\fr{s+1}{2^j}\right]= \varnothing\,.
\end{cases}
\end{multline}


\noindent
Д\,о\,к\,а\,з\,а\,т\,е\,л\,ь\,с\,т\,в\,о\,.\ \
По свойству В-сплай\-нов~\cite{Chui}  имеет место равенство:
\begin{equation*}
%\label{BSPL}
N_n'(x)=N_{n-1}(x)-N_{n-1}(x-1)\,.
\end{equation*}
Следовательно, по формуле интегрирования по час\-тям  получаем:

\noindent
\begin{multline*}
\eta_{0,\nu}^{j,s}(x)=\left.- \fr{(x-t)^{\nu+1}}{\nu+1} 
N_n(2^j t-s)\right|_0^x+{}\\
{}+2^j \int\limits_0^x \fr{(x-t)^{\nu+1}}{\nu+1} \left(N_{n-1}(2^jt-s)-{}\right.\\
\left.{}-
N_{n-1}(2^jt-s-1)\right)\,dt=
\fr{x^{\nu+1}}{\nu+1}\, N_n(-s)+{}\\
{}+\fr{2^j}{\nu+1}\left(\eta_{n-1,\nu+1}^{j,s}(x)-
\eta_{n-1,\nu+1}^{j,s+1}(x)\right).
\end{multline*}
Равенство~(\ref{U41}) очевидно.~\hfill$\square$

\vspace*{2pt}

Формулы (\ref{U40}) и~(\ref{U41}) позволяют находить значение 
функции $\eta_{n,\nu}^{j,s}(x)$ в~любой точке без интегрирования. Итак, 
для $l\hm=1,2,\dots, 2n+1$ получаем:

\columnbreak

\noindent
$$
{\xi}_{\nu+1,l}(x)=\fr{1}{\nu !}\, \eta_{n,\nu}^{0,l-n-1}(x)\,,\enskip
l=1,2,\dots, 2n+1\,.
$$
Для $l=2^j(n+1)\hm+n\hm+s$, $j\hm=0,1,\dots$, $s\hm=1,\ldots$\linebreak$\ldots,2^{j}(n\hm+1)$ получаем:
\begin{multline*}
{\xi}_{\nu+1,l}(x)={}\\
{}=
\begin{cases}
\fr{2^{({j+1})/{2}}}{\nu !}\sum\limits_{i=s}^{2s+2n}
h_{i,s}^{j+1}\eta_{n,\nu}^{j+1,-n+i-1}(x)\,, &\\
& \hspace*{-17mm}s=1,\ldots,n; \\
\fr{2^{({j+1})/{2}}}{\nu !}\sum\limits_{i=2s-n-1}^{2s+2n}
\!\! h_{i,s}^{j+1}\eta_{n,\nu}^{j+1,-n+i-1}(x)\,, &\\
& \hspace*{-40mm}s=n+1,\ldots,2^{j}(n+1)-n;\\
\fr{2^{({j+1})/{2}}}{\nu !}\sum\limits_{i=2s-n-1}^{2^{j}(n+1)+n+s}\!\!\!\!
h_{i,s}^{j+1}\eta_{n,\nu}^{j+1,-n+i-1}(x)\,, &\\
& \hspace*{-54mm}s=2^{j}(n+1)-n+1,\ldots,2^{j}(n+1).
\end{cases}
\end{multline*}
Полученные равенства справедливы при всех $\nu \hm= 0,1,\dots$

\section{Применение сплайн-вейвлетов к~решению 
линейных интегральных и~дифференциальных уравнений}

В проекционных методах решения линейных уравнений рассматриваются два 
уравнения~\cite{Akilov}: 
первое~--- в~полном нормированном пространстве~$X$:
\begin{equation}
\label{Ur1}
Kx\equiv x-\lambda Hx=f\,;
\end{equation}
второе~--- в~его полном подпространстве~$V_j$:
\begin{equation}
\label{Ur2}
K_j x_j\equiv x_j-\lambda H_j x_j=\Pi_j f\,,
\end{equation}
где $H$~--- непрерывный линейный оператор в~$X$; $H_j$~--- 
непрерывный линейный оператор в~$V_j$. Уравнение~(\ref{Ur1}) называется точным, 
а~уравнение~(\ref{Ur2})~--- приближенным. При этом предполагается, что 
выполнены следующие условия.

\noindent \textbf{1. Условие близости операторов $H$ и~$H_j$.} 
Для любого $x_j\hm\in V_j$ выполняется $\|\Pi_j H x_j \hm- H_j x_j\|
\hm\leqslant \rho_j \|x_j\|$.

\noindent \textbf{2. Условие хорошей аппроксимации элементов 
вида~$Hx$ элементами из~$V_j$.} Для любого $x\hm\in X$ существует $x_j\hm\in V_j$ 
такой, что $\|Hx-x_j\|\hm\leqslant \rho_{1,j} \|x\|$.

\noindent \textbf{3. Условие хорошей аппроксимации свободного члена 
точного уравнения.} Существует элемент $f_j \hm\in V_j$ такой, что $\|f-f_j\|
\hm\leqslant \rho_{2,j}\|f\|$. В~отличие от предыду\-щих условий $\rho_{2,j}$ 
здесь, вообще говоря, зависит от~$f$.

Как показано в~\cite{Akilov}, если оператор~$K$ 
имеет непрерывный обратный, уравнение~(\ref{Ur1}) имеет решение 
и~$\lim\limits_{j\to +\infty} \rho_j \hm= 0$,  $\lim\limits_{j\to +\infty} \rho_{1,j} \hm= 
0$,  $\lim\limits_{j\to +\infty} \rho_{2,j} \hm= 0$, то  
$\lim\limits_{j\to +\infty} \|x-x_j\|\hm = 0$, где $x_j$~--- 
решение уравнения~(\ref{Ur2}).

Рассмотрим сначала линейное интегральное уравнение Фредгольма 2-го рода. 
С~помощью замены переменной такое уравнение можно свести к~следующему:
\begin{equation*}
%\label{U313}
u(x)-\lambda\int\limits_{0}^{n+1}U(x,t)u(t)\,dt =f(x)\,,\enskip x,t\in [0;n+1]\,.
\end{equation*}

Пусть $\varphi(x)=N_n(x)$, $V_{0} \subset V_{1} \subset\cdots$~--- 
соответствующая последовательность конечномерных подпространств пространства
 ${L}^{2} \left[0;n+1\right]$. Пусть $X\hm={L}^2[0;n+1]$. 
 Операторы $K:X\hm\to X$, $H:X\hm\to X$ и~$H_j : V_j\hm\to V_j$ определим равенствами:
\begin{align*}
Ku(\cdot)&=u(\cdot)-\lambda\int\limits_0^{n+1} U(\cdot,t)u(t)\,dt\,;\\
Hu(\cdot)&=\int\limits_0^{n+1} U(\cdot,t)u(t)\,dt\,;\\
H_j&=\Pi_j\circ H\,,
\end{align*}
где $U\in {L}^2([0;n+1]^2)$. Условие близости операторов~$H$ и~$H_j$ 
выполняется с~$\rho_j\hm=0$. Пусть  $u\hm\in X$ и~$u_j(\cdot)\hm=\int\nolimits_0^{n+1} 
\Pi^{(2)}U(\cdot,t)u(t)\,dt\hm\in V_j$. То\-гда $\rho_{1,j}\hm=
\|U-\Pi_j^{(2)}U\|_{{L}^2([0;n+1]^2)}$ 
и~$\lim\nolimits_{j\to +\infty} \rho_{1,j}\hm=0$. Следовательно, 
условие хорошей аппроксимации элементов вида~$Hu$ элементами из~$V_j$ 
также выполняется. Наконец, для произвольного $f\hm\in X$, $f\hm\ne 0$, 
возьмем $f_j\hm=\Pi_j f$, 
а~$\rho_{2,j}\hm={\|f-\Pi_j f\|_{{L}^2([0;n+1])}}/{\|f\|_{{L}^2([0;n+1])}}$. Тогда $\lim\limits_{j\to +\infty} \rho_{2,j}=0$. 

Решение приближенного уравнения
\begin{equation}
\label{Ur3}
u_J-\lambda\Pi_j\circ H u_J=\Pi_J f
\end{equation}
будем искать в~виде $u_J\hm=\sum\nolimits_{l=1}^{M} c_l w_l 
\hm= {H}_J{C}_J $, где $M\hm=2^{J}(n+1)+n$. 
Тогда уравнение~(\ref{Ur3}) можно переписать в~виде
системы линейных уравнений для определения коэффициентов~$c_l$:
\begin{multline*}
%\label{U315}
\sum\limits_{l=1}^{M} c_l (w_l,w_s)-{}\\
{}-\lambda
\sum\limits_{l=1}^{M} c_l \int\limits_{0}^{n+1}dx
\int\limits_{0}^{n+1}U(x,t)w_l(t)w_s(x)\,dt = (f,w_s)\,,\\
s=1,2,\dots,M\,.
\end{multline*}
Это и~есть система метода Галеркина. Перепишем ее в~матричном виде:
\begin{equation}
\label{U315}
{C}_J\left([({H}_J,{H}_J)]-
\lambda{G}_J\right)={F}_J\,,
\end{equation}
где 
\begin{align*}
{C}_J&=\begin{pmatrix}
c_1&\cdots &c_M\end{pmatrix}\,;\\ 
{F}_J&=\begin{pmatrix}(f,w_1)&\cdots &(f,w_M)\end{pmatrix}^{\mathrm{T}}\,;\\
{G}_J&=\left(g_{l,s}\right)_{l,s=1}^M\,,
\\ 
&\hspace*{10mm}g_{l,s}= \int\limits_{0}^{n+1}dx\int\limits_{0}^{n+1}U(x,t)w_l(t)w_s(x)\,dt.
\end{align*}

Аналогично рассматривается линейное интегральное уравнение Вольтерра 2-го рода
$$
u(x)-\lambda\int\limits_{0}^{x}U(x,t)u(t)\,dt =f(x)\,,\enskip 
x,t\in [0;n+1]\,.
$$
Операторы $K:X\hm\to X$, $H:X\hm\to X$ и~$H_j : V_j\hm\to V_j$ определим равенствами:
\begin{align*}
Ku(x)&=u(x)-\lambda\int\limits_0^{x} U(x,t)u(t)\,dt\,;\\
Hu(x)&=\int\limits_0^{x} U(x,t)u(t)\,dt\,;\\
H_j&=\Pi_j\circ H\,.
\end{align*}
Величины $\rho_j$, $\rho_{1,j}$ и~$\rho_{2,j}$ остаются теми же, что и~для 
уравнения Фредгольма.  Таким образом,
система Галеркина для данного уравнения имеет вид:
\begin{multline}
\label{U316}
\sum\limits_{l=1}^{M} c_l (w_l,w_s)-{}\\
{}-
\lambda\sum\limits_{l=1}^{M} 
c_l \int\limits_{0}^{n+1}dx\int\limits_{0}^{x}U(x,t)w_l(t)w_s(x)\,dt ={}\\
{}=(f,w_s)\,,\enskip
s=1,2,\dots,M\,.
\end{multline}
Матричный вид системы~(\ref{U316}) совпадает с~(\ref{U315}), где
$$
{G}_J=(g_{s,l})_{s,l=1}^M,\ \ g_{s,l}= 
\int\limits_{0}^{n+1}\!dx\int\limits_{0}^{x}\!U(x,t)w_l(t)w_s(x)\,dt.
$$

Рассмотрим теперь линейное дифференциальное уравнение
\begin{equation}\label{DU1}
y^{(k)}+a_1(x)y^{(k-1)}+\dots +a_k(x)y=f(x)
\end{equation}
с непрерывными коэффициентами $a_i(x)$, $i\hm=1,2,\dots,k$ и~начальными условиями
\begin{equation}
\label{NDU1}
y(0)=y_0\,,\enskip y'(0)=y_1,\ \dots,\ y^{(k-1)}=y_{k-1}\,.
\end{equation}
Если обозначить $y^{(k)}(x)=u(x)$, то задача~(\ref{DU1})--(\ref{NDU1}) сводится 
к~интегральному уравнению Вольтерра \mbox{2-го} рода. Следовательно, 
приближенное решение задачи~(\ref{DU1})--(\ref{NDU1}) можно искать в~виде:
\begin{multline*}
y_J(x)=\sum\limits_{s=1}^M c_s {\xi}_{k,s}(x) +y_{k-1}\fr{x^{k-1}}{(k-1)!}+{}\\
{}+y_{k-2}\fr{x^{k-2}}{(k-2)!}+\dots+y_0\,,
\end{multline*}
где функции ${\xi}_{k,s}(x)$ определены равенствами~(\ref{U38})
и~(\ref{U38-1}), а~коэффициенты~$c_s$ 
определяются из системы линейных уравнений:
\begin{multline*}
\sum\limits_{s=1}^M c_s ({\xi}_{k,s}+a_1 {\xi}_{k-1,s}+\dots +a_k w_s,w_l)={}\\
{}=\left(f,w_l\right),\enskip l=1,\dots,M\,.
\end{multline*}
При этом $\lim\nolimits_{J\to +\infty} \|y_J\hm-y\|_{C^{k-1}[0;n+1]}\hm=0$, 
где $y(x)$~--- точное решение задачи~(\ref{DU1})--(\ref{NDU1}).



\noindent
\textbf{Пример 4.1.}\
Рассмотрим нестационарную систему автоматического управления, 
поведение которой описывается дифференциальным уравнением:
$$
\sum\limits_{k=0}^5 a_k(t)x^{(k)}(t)=g(t),
$$
где коэффициенты~$a_k(t)$ определяются из сле\-ду\-юще\-го выражения:
{\small
\begin{multline*}
\begin{pmatrix}
a_0(t)\\a_1(t)\\a_2(t)\\a_3(t)\\a_4(t)\\a_5(t)\end{pmatrix}={}\\
\!\!{}=\!
\begin{pmatrix}0{,}5596&1{,}8918&2{,}5825&1{,}7855&0{,}6277&\!\!0{,}0909\\
0{,}7113&2{,}3843&3{,}222\hphantom{9}&2{,}1975&0{,}7588&\!\!0{,}1065\\
0{,}3717&1{,}2333&1{,}6449&1{,}1038&0{,}3728&\!\!0{,}0507\\
0{,}1002&0{,}3278&0{,}43\hphantom{99}&0{,}2827&0{,}093\hphantom{9}&\!\!0{,}0122\\
0{,}014\hphantom{9}&0{,}0449&0{,}0576&0{,}0369&0{,}0118&\!\!0{,}0015\\
0{,}0008&0{,}0025&0{,}0031&0{,}0019&0{,}006\hphantom{9}&\!\!\hphantom{9}0{,}00007
\end{pmatrix}\!\!
\begin{pmatrix}1\\t\\t^2\\t^3\\t^4\\t^5\end{pmatrix}\!.\hspace*{-2.77pt}
\end{multline*}
}

\noindent
Найти реакцию системы на входное воздействие:
\begin{multline*}
g(t)=\left(85{,}7661+338{,}5984t+497{,}0437t^2+{}\right.\\
{}+406{,}9496t^3+186{,}9354t^4+46{,}7809t^5+{}\\
\left.{}+4{,}8258t^6\right)e^{-4t}.
\end{multline*}
Начальные условия нулевые. Интервал исследования~--- $[0;5]$~c.

\columnbreak




\smallskip

\noindent
Р\,е\,ш\,е\,н\,и\,е\,.
Так как начальные условия нулевые, приближенное решение данной задачи будем  
искать в~виде
$$
x_J(t)=\sum\limits_{s=1}^M c_s {\xi}_{5,s}(t)\,,
$$
где коэффициенты $c_1,\dots, c_M$ определяются из системы линейных уравнений:
\begin{multline*}
\sum\limits_{s=1}^{2^J(n+1)+n} c_s \left(
a_5w_{s}+\sum\limits_{k=0}^4 a_k\xi_{5-k,s},w_l\right)={}\\
{}=
(g,w_l)\,,\enskip 
l=1,\dots,2^J(n+1)+n\,.
\end{multline*}
На рис.~2 показаны графики третьего и~пятого приближений~$x_2(t)$ 
(\textit{1}), $x_4(t)$~(\textit{2}) и~график 
сеточной функции $\{(t_i,\tilde{x}_i)\}$~(\textit{3}), 
полученной методом Рун\-ге--Кутта.\hfill $\square$

 { \begin{center}  %fig2
 \vspace*{9pt}
 \mbox{%
 \epsfxsize=79.639mm 
 \epsfbox{bit-2.eps}
 }


\end{center}


\noindent
{{\figurename~2}\ \ \small{Графики приближений $x_2(t)$ 
(\textit{1}), $x_4(t)$~(\textit{2}) 
и~график сеточной функции $\{(t_i,\tilde{x}_i)\}$~(\textit{3}), 
полученной методом Рун\-ге--Кутта}}
}

\vspace*{6pt}

\addtocounter{figure}{1}

%\vspace*{-36pt}

\begin{figure*} %fig3
\vspace*{1pt}
 \begin{center}
 \mbox{%
 \epsfxsize=164.758mm 
 \epsfbox{bit-3.eps}
 }
  \end{center}
\vspace*{-9pt}
\begin{minipage}[t]{79mm}
\Caption{Графики приближений $x_0(t)$ (\textit{1}), $x_2(t)$ 
(\textit{2}) и~график сеточной функции $\{(t_i,\tilde{x}_i)\}$  \label{sx}
(\textit{3}), полученной методом Рун\-ге--Кутта}
%\end{figure*}
%\begin{figure*} %fig4
\end{minipage}
\hfill
\vspace*{-9pt}
\begin{minipage}[t]{79mm}
\Caption{Графики приближений $y_0(t)$ (\textit{1}), $y_2(t)$~(\textit{2}) 
и~график сеточной функции $\{(t_i,\tilde{y}_i)\}$~(\textit{3}), полученной методом 
Рун\-ге--Кутта}
\label{sy}
\end{minipage}
\vspace*{12pt}
\end{figure*}


\smallskip

\noindent
\textbf{Пример 4.2.}\
Поведение линейной нестационарной системы описывается следующей
 системой дифференциальных уравнений:
$$
\begin{pmatrix}
{\dot x}(t) \\ {\dot y}(t)\end{pmatrix} = 
\begin{pmatrix} t^2 & 1-t \\ 1+t & t-t^2 \end{pmatrix}
 \begin{pmatrix} {x}(t) \\ {y}(t)\end{pmatrix}
 +  
 \begin{pmatrix} t^2 & 0 \\ 1 & t \end{pmatrix} 
 \begin{pmatrix} {g}_1(t) \\ {g}_2(t)\end{pmatrix}\,.
$$
Найти реакцию системы на входное воздействие:
\begin{multline*}
g_1(t) = 0{,}23315158 t^9-3{,}89665 t^8+26{,}4309725 t^7{}-\\
{}-93{,}4794 t^6+183{,}95 t^5 -200{,}83 t^4+{}\\
{}+122{,}255277 t^3-50{,}135386 t^2+13{,}095959 t-2{,}8237;
\end{multline*}

\vspace*{-12pt}

\noindent
\begin{multline*}
g_2(t) = -0{,}071962459 t^{13}+1{,}3465024 t^{12}-{}\\
{}-10{,}98105044 t^{11} + 51{,}1908385 t^{10} - 150{,}5098287 t^9+{}\\
{}+291{,}295256 t^8-378{,}61242 t^7+336{,}683591 t^6-{}
\end{multline*}

\noindent
\begin{multline*}
{}-213{,}9681871 t^5 + 106{,}48891 t^4-47{,}3676 t^3+{}\\
{}+19{,}56997 t^2-3{,}863587 t-0{,}0004283
\end{multline*}

%\pagebreak

\noindent
для начальных условий $x(0)\hm=-1$ и~$y(0)\hm=2$ на вре\-мен\-н$\acute{\mbox{о}}$м интервале $[0;2]$~c.

\smallskip



\noindent
Р\,е\,ш\,е\,н\,и\,е\,.\ \ 
Приближенное решение будем искать в~виде: 
\begin{align*}
x_{J}(t)&=-1+\sum\limits_{s=1}^M 
c_{s} {\xi}_{1,s}(t)\,;\\
y_{J}(t)&=2+\sum\limits_{s=1}^M c_{M+s} {\xi}_{1,s}(t)\,, 
\end{align*}
где $M\hm=2^J(n+1)+n$, а~коэффициенты $c_s$, $s\hm=1,2,\ldots, 2M$, определяются 
из системы линейных уравнений:
\begin{multline*}
\sum\limits_{s=1}^M c_s\left((w_s,w_l)-
\int\limits_{0}^{n+1}\!t^2{\xi}_{1,s}(t)w_l(t)\,dt\right)-{}\\
{}-\sum\limits_{s=1}^M c_{M+s}\int\limits_0^{n+1}(1-t){\xi}_{1,s}(t)w_l(t)\,dt={}\\
{}=\!\int\limits_0^{n+1}\!\left(t^2g_1(t)-t^2+2(1-t)\right)w_l(t)\,dt\,,\\
 l=1,2,\dots,M\,;
\end{multline*}

\vspace*{-12pt}

\noindent
\begin{multline*}
\sum\limits_{s=1}^M c_s\int\limits_{0}^{n+1}(1+t){\xi}_{1,s}(t)w_l(t)\,dt+{}\\
{}+
\sum\limits_{s=1}^M \!c_{M+s}\left(\int\limits_0^{n+1}\!(t-t^2){\xi}_{1,s}(t)w_l(t)\,dt-
\left(w_s,w_l\right)\right)={}\hspace*{-6.37675pt}
\end{multline*}


\noindent
\begin{multline*}
{}=\int\limits_0^{n+1}(2t^2-t+1-g_1(t)-tg_2(t))w_l(t)\,dt\,, 
\\
 l=1,2,\dots,M\,.
\end{multline*}


На рис.~\ref{sx} и~\ref{sy} показаны графики первого и~треть\-его приближений 
$x_0(t),~y_0(t)$ (\textit{1}), $x_2(t),~y_2(t)$~(\textit{2}) 
и~графики сеточных функций $\{(t_i,\tilde{x}_i)\}$, $\{(t_i,\tilde{y}_i)\}$~(\textit{3}), 
полученные методом Рун\-ге--Кутта.~\hfill$\square$

\vspace*{-6pt}

\section{Заключение}

\vspace*{-2pt}

В данной статье были обобщены известные методы применения вейвлетов 
Хаара к~приближенному решению линейных интегральных и~дифференциальных уравнений. 
Эти методы получаются из представленных здесь при $n\hm=0$, что и~соответствует 
вейвлетам Хаара. В~отличие от вейвлетов Хаара, где приближения решения интегрального 
уравнения получаются ку\-соч\-но-по\-сто\-ян\-ны\-ми, а~приближения решения 
дифференциального уравнения принадлежат классу глад\-кости~$C^{k-1}$, где $k$~--- 
порядок уравнения, использование сплайн-вейв\-лет дает возможность 
строить приближения любого класса гладкости~$C^n$.

\vspace*{-6pt}

{\small\frenchspacing
 {%\baselineskip=10.8pt
 \addcontentsline{toc}{section}{References}
 \begin{thebibliography}{99}
 
 \vspace*{-2pt}
 
\bibitem{Lepik3} 
\Au{Lepik\,{\!\!\ptb{\"{U}}}}. 
Application of the Haar wavelet transform to solving integral and 
differential equations~// Proc. Est. Acad. Sci. Ph.,  2007. 
Vol.~56. P.~28--46.
\bibitem{Blatov} %2
\Au{Блатов И.\,А., Рогова~Н.\,В.} Полуортогональные сплайновые вейвлеты и~метод 
Галеркина численного моде-\linebreak\vspace*{-11pt}

\pagebreak

\noindent
лирования тонкопроволочных антен~// 
Вычисл. матем. и~матем. физ., 2013. Т.~53. №\,5. C.~727--736.

\bibitem{Lepik4}   %3
\Au{Lepik\,{\!\!\ptb{\"{U}}}}. 
Numerical solution of differential equations using Haar wavelets~//  
Math. Comput. Simulat., 2005. Vol.~68. P.~127--143.
\bibitem{Lepik1}  %4
\Au{Lepik\,{\!\!\ptb{\"{U}}}}. Numerical solution of evolution 
equations by the Haar wavelet method~//  Appl. Math. Comput., 2007.  Vol.~185. 
P.~695--704.
\bibitem{Lepik2} %5
\Au{Lepik\,{\!\!\ptb{\"{U}}}}. Haar wavelet method for solving higher order
 differential equations~// Int. J.~Math. Comput., 2008. Vol.~1. No.\,8. P.~84--94.

\bibitem{Lepik}  %6
\Au{Lepik\,{\!\!\ptb{\"{U}}}., Hein~H}.  Haar wavelets with applications.~---  
Berlin: Springer, 2014. 207~p.
\bibitem{ArticleFinkelstein}  
\Au{Finkelstein A., Salesin~D.\,H.} Multiresolution curves~// SIGGRAPH Proceedings.~--- 
New York, NY, USA: ACM, 1994. P.~261--268.
\bibitem{Frazer} 
\Au{Frazier M.\,W}. An introduction to wavelets through linear algebra.~--- 
New York, NY, USA: Springer,  1999.  503~p.
\bibitem{Chui}  
\Au{Chui Ch.\,К}.  An introduction to wavelets.~--- Boston, MA, USA: Academic Press, 1991. 412~p.
\bibitem{Yurgu} 
\Au{Bityukov Yu.\,I.,  Akmaeva~V.\,N. }  
The use of wavelets in the mathematical and computer modelling of manufacture 
of the complex-shaped shells made of composite materials~//  Bull. 
South Ural State University. Ser. Mathematical Modelling, 
Programming and Computer Software, 2016. Vol.~9. No.\,3. P.~5--16.

\bibitem{Novikov}   %11
\Au{Новиков И.\,Я.,  Протасов~В.\,Ю.,  Скопина~М.\,А.} Теория всплесков.~--- 
М.: Физматлит, 2005.  612~c.
\bibitem{Akilov}  %12
\Au{Канторович Л.\,В.,  Акилов~Г.\,П.} Функциональный анализ.~--- 
М.: Наука, 1977. 744~с.

%\bibitem{BookSmolencev} 
%\Au{Смоленцев Н.\,К}. Основы теории вейвлетов. Вейвлеты в~MatLab.~--- 
%М.: ДМК Пресс,  2005. 303~c.
%\bibitem{Pupkov} Пупков, К.А., Н.Д. Егупов. Методы классической и~современной теории автоматического управления. Математические модели, динамические характеристики и~анализ систем автоматического управления. М.: Издательство МГТУ им. Н.Э. Баумана. 2004. 656 с.
 \end{thebibliography}

 }
 }

\end{multicols}

\vspace*{-6pt}

\hfill{\small\textit{Поступила в~редакцию 21.03.17}}

\vspace*{8pt}

%\newpage

%\vspace*{-24pt}

\hrule

\vspace*{2pt}

\hrule

%\vspace*{8pt}


\def\tit{THE USE OF WAVELETS FOR~THE~CALCULATION OF~LINEAR CONTROL SYSTEMS 
WITH~LUMPED PARAMETERS}

\def\titkol{The use of wavelets for~the~calculation of~linear control systems 
with~lumped parameters}

\def\aut{Yu.\,I.~Bityukov and~E.\,N.~Platonov}

\def\autkol{Yu.\,I.~Bityukov and~E.\,N.~Platonov}

\titel{\tit}{\aut}{\autkol}{\titkol}

\vspace*{-9pt}


\noindent
Moscow Aviation Institute (National 
Research University), 4~Volokolamskoye Highway, Moscow 125993, Russian Federation



\def\leftfootline{\small{\textbf{\thepage}
\hfill INFORMATIKA I EE PRIMENENIYA~--- INFORMATICS AND
APPLICATIONS\ \ \ 2017\ \ \ volume~11\ \ \ issue\ 4}
}%
 \def\rightfootline{\small{INFORMATIKA I EE PRIMENENIYA~---
INFORMATICS AND APPLICATIONS\ \ \ 2017\ \ \ volume~11\ \ \ issue\ 4
\hfill \textbf{\thepage}}}

\vspace*{3pt}



\Abste{In many disciplines, problems appear which can be formulated with 
the aid of differential or integral equations.  In simpler cases, such equations 
can be solved analytically, but for more complicated cases, numerical 
procedures are needed. In recent times, the wavelet-based methods 
have gained great popularity, where different wavelet families such as 
Daubechies, Coiflet, etc.\ wavelets are applied. A~shortcoming of these wavelets 
is that they do not have an analytic expression. For this reason, differentiation 
and integration of these wavelets are very complicated. The paper presents 
algorithms for the numerical solution of linear integral and differential 
equations based on spline wavelets on the interval. The algorithms generalize 
the well-known methods based on Haar wavelets, which are a~particular case 
of spline wavelets. The results presented can be applied for 
the analysis of linear systems with lumped parameters.}

\KWE{spline wavelet; differential equation; integral equation}



\DOI{10.14357/19922264170412} 

\vspace*{-6pt}

\Ack
\noindent
This work is a part of Project No.\,2.2461.2017 
supported by the Russian Ministry of Education and Science.



\vspace*{3pt}

  \begin{multicols}{2}

\renewcommand{\bibname}{\protect\rmfamily References}
%\renewcommand{\bibname}{\large\protect\rm References}

{\small\frenchspacing
 {%\baselineskip=10.8pt
 \addcontentsline{toc}{section}{References}
 \begin{thebibliography}{99}

\bibitem{1-bit-1}
\Aue{Lepik,  $\ddot{\mbox{U}}$}. 2007. Application of the
 Haar wavelet transform to solving integral and differential equations. 
 \textit{Proc. Est. Acad. Sci. Ph.} 56:28-46.
\bibitem{2-bit-1}
\Aue{Blatov,  I.\,A., and  N.\,V.~Rogova.} 
2013.  Application of semi-orthogonal spline wavelets and Galerkin
method to the numerical simulation of thin wire antennas. 
\textit{Comp. Math. Math. Phys.} 53(5):564--572.
\bibitem{5-bit-1} %3
\Aue{Lepik,  $\ddot{\mbox{U}}$.} 2005. Numerical solution of differential 
equations using Haar wavelets. \textit{Math. Comput. Simulat.} 68:127--143.
\bibitem{3-bit-1} %4
\Aue{Lepik, $\ddot{\mbox{U}}$.}
2007. Numerical solution of evolution equations by the 
Haar wavelet method. \textit{Appl. Math. Comput.} 185:695--704.
\bibitem{4-bit-1} %5
\Aue{Lepik,  $\ddot{\mbox{U}}$.} 
2008. Haar wavelet method for solving higher order differential equations. 
\textit{Int. J.~Math. Comput.} 1(8):84--94.

\bibitem{6-bit-1}
\Aue{Lepik,  $\ddot{\mbox{U}}$, and H.~Hein.} 2014. \textit{Haar wavelets 
with applications.}  Berlin: Springer. 207~p.
\bibitem{7-bit-1}
\Aue{Finkelstein, A., and D.\,H.~Salesin}. 1994. 
{Multiresolution curves}. \textit{SIGGRAPH Proceedings}.  New York, NY: ACM. 261--268.
%\item Demko, S., W.F. Moss and P.W. Smith. Decay rates for inverses of band matrices. 1984. Math. Comp. V. 43, \textnumero{167}. 491--499.
\bibitem{8-bit-1}
\Aue{Frazier, M.\,W.} 1999. \textit{An introduction to wavelets through linear algebra.} 
New York, NY: Springer. 503~p.
\bibitem{9-bit-1}
\Aue{Chui,  Ch.\,К.} 1991. \textit{An introduction to wavelets.} Boston, MA: 
Academic Press. 412~p.
\bibitem{10-bit-1}
\Aue{Bityukov, Yu.\,I., and  V.\,N.~Akmaeva.} 2016. 
The use of wavelets in the mathematical and computer modelling of manufacture 
of the complex-shaped shells made of composite materials.   
\textit{Bull. South Ural State University. Ser. Mathematical Modelling, 
Programming and Computer Software} 9(3):5--16.

\bibitem{12-bit-1}
\Aue{Novikov, I.\,Ya.,  V.\,Yu.~Protasov, and M.\,A.~Skopina.}
2005. \textit{Teoriya vspleskov} [{The theory of wavelets}]. Moscow: Fizmatlit.  612~p.

\bibitem{11-bit-1}
\Aue{Kantorovich, L.\,V., and G.\,P.~Akilov.} 1977. \textit{Funktsional'nyy analiz} 
[{Functional analysis}]. Moscow: Nauka. 744~p.
%\bibitem{13-bit-1}
%\Aue{Smolentsev, N.\,K.} 2005. \textit{Osnovy teorii veyvletov. Veyvlety v~MatLab} 
%[{Fundamentals of the theory of wavelets. Wavelets in MatLab}].  Moscow: 
%DMK Press. 303~p.
\end{thebibliography}

 }
 }

\end{multicols}

\vspace*{-6pt}

\hfill{\small\textit{Received March 21, 2017}}

%\vspace*{-10pt}

\Contr

\noindent
\textbf{Bityukov Yuri I.} (b.\ 1972)~--- 
Doctor of Science in technology, associate professor, 
Moscow Aviation Institute (National Research University), 
4~Volokolamskoye Highway, Moscow 125993, Russian Federation; 
\mbox{yib72@mail.ru}

\vspace*{3pt}

\noindent
\textbf{Platonov Evgeny N.} (b.\ 1976)~--- Candidate of Science (PhD) in physics 
and mathematics, associate professor, Moscow Aviation Institute (National 
Research University), 4~Volokolamskoye Highway, Moscow 125993, Russian Federation;
\mbox{en.platonov@gmail.com}
\label{end\stat}


\renewcommand{\bibname}{\protect\rm Литература}   %12
\newcommand{\G}{{\sf Ge}}

\def\stat{kudr-titova}

\def\tit{ГАММА-ЭКСПОНЕНЦИАЛЬНАЯ ФУНКЦИЯ В БАЙЕСОВСКИХ МОДЕЛЯХ МАССОВОГО ОБСЛУЖИВАНИЯ$^*$}

\def\titkol{Гамма-экспоненциальная функция в~байесовских моделях массового обслуживания}

\def\aut{А.\,А.~Кудрявцев$^1$, А.\,И.~Титова$^2$}

\def\autkol{А.\,А.~Кудрявцев, А.\,И.~Титова}

\titel{\tit}{\aut}{\autkol}{\titkol}

\index{Кудрявцев А.\,А.}
\index{Титова А.\,И.}
\index{Kudryavtsev A.\,A.}
\index{Titova A.\,I.}



{\renewcommand{\thefootnote}{\fnsymbol{footnote}} \footnotetext[1]
{Работа выполнена при частичной финансовой поддержке РФФИ (проект 17-07-00577).}}


\renewcommand{\thefootnote}{\arabic{footnote}}
\footnotetext[1]{Московский государственный университет им.\ М.\,В.~Ломоносова, 
факультет вычислительной математики и~кибернетики, \mbox{nubigena@mail.ru}}
\footnotetext[2]{Московский государственный университет им.\ М.\,В.~Ломоносова, 
факультет вычислительной математики и~кибернетики, \mbox{onkelskroot@gmail.com}}

%\vspace*{-18pt}

\vspace*{-9pt}

\Abst{Рассматривается байесовский подход к~построению мо\-де\-лей 
теории массового обслуживания и~надежности. Байесовский подход является 
целесообразным при изучении систем, характеристики которых меняются в~моменты 
времени, неизвестные исследователю, или же при изучении больших совокупностей 
однотипных систем. В~рамках этого подхода для классических постановок задач 
предполагается, что основные параметры системы не являются заданными, 
но при этом известны их априорные распределения. За счет 
рандомизации параметров системы различные ее характеристики, 
например коэффициент загрузки, также становятся случайными. 
В~работе вводится понятие гам\-ма-экс\-по\-нен\-ци\-аль\-ной 
функции, приводятся ее свойства, а~также конкретные результаты для 
вероятностных характеристик коэффициента загрузки и~вероятности <<непотери>> 
вызова в~случае, когда в~качестве пары априорных распределений параметров 
системы $M/M/1/0$ рассматриваются экспоненциальное распределение и~распределение 
Вейбулла.}


\KW{байесовский подход; системы массового обслуживания; надежность; смешанные
распределения; распределение Вейбулла; экспоненциальное распределение; 
гам\-ма-экс\-по\-нен\-ци\-аль\-ная функция}

\DOI{10.14357/19922264170413} 

\vspace*{-9pt}


\vskip 10pt plus 9pt minus 6pt

\thispagestyle{headings}

\begin{multicols}{2}

\label{st\stat}

\section{Введение}

Зачастую при описании математических мо\-де\-лей функционирования различных объектов 
их жизненный цикл зависит от параметров, <<способствующих>> и~<<препятствующих>> 
функционированию. В~моделях структур и~систем массового\linebreak обслуживания к~параметрам, 
<<способствующим>> функционированию, можно отнести интенсивность обслуживания 
запросов, а~к~параметрам, <<препятствующим>> функционированию,~--- 
интенсивность входящего потока требований. При этом нетрудно заметить, 
что для исследования результатов работы системы важны не столько значения 
параметров, сколько их соотношение.

Далее будет рассмотрена система массового обслужи\-вания $M/M/1/0$, одним из 
основных показателей которой является ее коэффициент загрузки~$\rho$. 
Значение коэффициента загрузки определяется как отношение параметра входящего 
потока~$\lambda$ к~па\-ра\-мет\-ру обслуживания~$\mu$. От величины~$\rho$ 
зависят многие характеристики разнообразных систем массового обслуживания, в~том 
числе вероятность <<непотери>> вызова $\pi \hm= {\mu/(\lambda \hm+ \mu)}
\hm = 1/(1\hm+\rho)$.

В рамках байесовского подхода к~постановкам классических задач массового 
обслуживания и~надежности предполагается, что конкретные значения параметров~$\lambda$ 
и~$\mu$ неизвестны, однако имеется информация об их априорных распределениях~\cite{KuSh2015}.

Зачастую в~байесовских постановках задач массового обслуживания результаты 
описываются в~терминах специальных функций, например в~терминах бе\-та-функ\-ции 
и~интегральной показательной функции при рассмотрении общего эрланговского 
случая~\cite{KuSh09b}. При рассмотрении общего 
бе\-та-рас\-пре\-де\-ле\-ния~\cite{ZhaKuSh} или же бе\-та-рав\-но\-мер\-но\-го~\cite{ZhaKuSh2} 
распределения параметров в~байесовской модели рекуррентного роста надежности 
результаты выражаются через обобщенную гипергеометрическую функцию.

В ходе исследований, касающихся вероятностных характеристик коэффициента 
загрузки~$\rho$  и~вероятности <<непотери>> вызова~$\pi$ в~случае, когда в~качестве 
пары априорных распределений параметров системы~$\lambda$ и~$\mu$ 
рассматриваются экспоненциальное распределение и~распределение Вейбулла, 
были получены результаты, не выражающиеся в~терминах элементарных функций. 
В~связи с~этим предлагается рассмотреть новую специальную функцию, 
упоминаний об аналитическом виде и~свойствах которой не было 
обнаружено в~классических книгах, посвященных специальным функциям 
(см., например,~\cite{Artin, BeEr,AbSt}).

\section{Основные результаты}


Введем следующие обозначения. Через~$M(\theta)$ обозначим экспоненциальное 
распределение с~параметром $\theta\hm>0$, а~через $W(p,\alpha)$~--- 
распределение Вейбулла с~плот\-ностью $w_{p,\alpha}(x)$, имеющей вид:
$$
w_{p,\alpha}(x) = \fr{px^{p-1}e^{-({x/\alpha})^p}}{\alpha^{p}} \,, \enskip
 x>0\,,\  p>0\,,\ \alpha>0\,.
 $$
Назовем функцию вида
$$
\G_{\alpha,\, \beta} (x) = \sum\limits_{k=0}^{\infty}
\fr{x^k}{k!}\, \Gamma(\alpha k + \beta)\,, \enskip
 x\in\mathbb{R}\,, \ \alpha\ge0\,, \  \beta> 0\,,
 $$
\textit{гамма-экспо\-нен\-ци\-аль\-ной функцией}.


\smallskip

\noindent
\textbf{Теорема~1.}\
\textit{Гамма-экспо\-нен\-ци\-аль\-ная функция обладает следующими свойствами}:
\begin{enumerate}
\item $\G_{\alpha, \beta} (x)$ 
\textit{сходится абсолютно на всей прямой при $0\hm\le\alpha\hm<1$, $\beta\hm>0$ и~на интервале $(-1,1)$ при $\alpha\hm=1$, $\beta\hm>0$; 
сходится условно в~точке $x = -1$ при 
$\alpha=1$, $0\hm<\beta\hm<1$; сходится только 
в~точке  $x\hm = 0$ при} $\alpha\hm>1$, $\beta\hm>0$.
\item $\G_{\alpha, \beta}(x)$ \textit{непрерывна в~области сходимости}.
\item $\G_{1,\, n+1} (x) \hm= \left({x^n}/{(1-x)}\right)^{(n)}_x$, $|x|\hm<1$. 
\textit{В~частности, $\G_{1,\, 1} (x)\hm= {1/{(1\hm-x)}}$ и}
$\G_{1,\, 2} (x)\hm= {1/{(1\hm-x)^2}}$,  $|x|\hm<1$.
\item $\G^{(n)}_{\alpha, \beta} (x) = \G_{\alpha, \alpha n + \beta}(x)$ 
\textit{в области сходимости}.
\item $\G_{\alpha, \beta} (0)\hm=\Gamma(\beta),$ $\alpha\hm\ge0,$ $\beta\hm> 0$.
\item $\G_{0, \beta} (x)= \Gamma(\beta)e^{x},$ $x\hm\in\mathbb{R},$ $\beta\hm> 0$.
\item $\G_{\alpha, 1}(x)=1\hm+\alpha x\G_{\alpha, \alpha}(x),$ $x\hm\in
\mathbb{R},$ $\alpha\hm>0$.
\item $\G_{\alpha,\, \beta}(x)=\Gamma(\beta-1)+\alpha x \G_{\alpha,\alpha+\beta-1}(x)
+(\beta\hm-1)\G_{\alpha, \beta-1}(x),$ $x\hm\in\mathbb{R},$ $\alpha\hm\ge0,$ $\beta\hm>1$.
\item $\G_{q, q+1}(-x^q)=\G_{1/q, 1/q+1 }(-{1/x})/{({qx^{q+1}})},$ $x\hm>0,$ $q\hm>1$.
\item $\sum\nolimits_{k=0}^\infty ({\alpha^k}/{k!}) \G_{1/p,\, (k+p)/p}(- \alpha)\hm= 1$, 
$\alpha\hm>0$, $p\hm>1.$
\end{enumerate}


\noindent
Д\,о\,к\,а\,з\,а\,т\,е\,л\,ь\,с\,т\,в\,о\,.\ \ 
Свойство~1 следует из соотношения (6.1.46) в~\cite{AbSt}:
$$
\lim\limits_{n \to \infty}
\fr{\Gamma(n+\alpha)n^{\beta-\alpha}}{{\Gamma(n+\beta)}}=1\,,
$$
формулы Даламбера и~признака Лейбница.
Свойства~3--8 проверяются непосредственно. Свойство~9 следует из соотношения:
\begin{multline*}
\G_{q, q+1}(-x^q)= \int\limits_0^\infty e^{-(xt)^{q}} t^q e^{-t}\, dt={}\\
{}=
\fr{1}{{qx^{q+1}}}\sum\limits_{k=0}^\infty
\fr{(-1/x)^k}{{k!}}\int\limits_0^\infty z^{(k+1)/q}e^{-z}\, dz\,.
\end{multline*}
Для обоснования свойства~10 достаточно рас\-смот\-реть случайную величину~$N$, 
имеющую смешанное пуассоновское распределение со структурным распределением 
Вейбулла, для которой справедливо:
\begin{multline*}
\p(N=k) = \fr{p}{{\alpha^p k!}} \int\limits_0^\infty 
e^{-\lambda -(\lambda/\alpha)^p} \lambda^{k+p-1} \, d\lambda={}\\
{}= \fr{\alpha^k}{k!} \sum\limits_{n=0}^\infty 
\fr{(-\alpha)^n}{{n!}} \int\limits_0^\infty t^{(n+k)/p} e^{-t} \, dt
= {}\\
{}=\fr{\alpha^k}{k!}\, \G_{1/p,\, (k+p)/p}(- \alpha)\,.
\end{multline*}

Теорема доказана.

\smallskip

Приведем два утверждения, в~которых па\-ра\-мет\-ры входящего потока~$\lambda$ 
и~обслуживания~$\mu$ в~модели~$M/M/1/0$ имеют  экспоненциальное распределение 
и~распределение Вейбулла.

\noindent
\textbf{Теорема~2.}\ 
\textit{Пусть параметр входящего потока~$\lambda$ имеет экспоненциальное 
распределение $M(\theta)$, $\theta\hm>0$, а~параметр обслуживания~$\mu$ 
имеет распределение Вейбулла $W(p,\alpha)$, $p\hm>1$, $\alpha\hm>0$, причем~$\lambda$ 
и~$\mu$ независимы. Тогда при $x\hm>0$ функция распределения, плотность 
и~моменты коэффициента загрузки~$\rho$ имеют вид}:
\begin{align*}
F_\rho(x) &= 1 - \G_{1/p,\, 1}(-\theta \alpha x)\,; \\
f_\rho(x) &= \theta \alpha \G_{1/p,\, 1/p+1}(-\theta \alpha x)\,;\\
\e\,\rho^n &= \fr{n!}{(\theta \alpha)^n}\,\Gamma\left(1-\fr{n}{p}\right)\,, \enskip 
p>n\,,
\end{align*}
\textit{a функция распределения и~плотность вероятности <<непотери>> вызова~$\pi$ 
при $x\hm\in(0,1)$ определяются соотношениями}:
\begin{align*}
F_\pi(x) &= \G_{1/p,\, 1}\left(-\fr{\theta \alpha (1-x)}{x}\right)\,; \\
f_\pi(x) &= \fr{\theta \alpha}{{x^2}}\, \G_{1/p,\, 1/p + 1}
\left(-\fr{\theta \alpha (1-x)}{x}\right)\,.
\end{align*}

\noindent
Д\,о\,к\,а\,з\,а\,т\,е\,л\,ь\,с\,т\,в\,о\,.\ \ Заметим, что при $x\hm>0$
\begin{multline*}
F_\rho(x) =
\int\limits_0^{\infty}  \fr{p}{\alpha}\left(
\fr{u}{\alpha}\right)^{p-1}e^{-(u/\alpha)^{p}}(1-e^{-ux\theta})\, du={}\\
{}= 1 - \sum\limits_{k=0}^\infty \int\limits_0^{\infty} 
\fr{(-\theta x)^k}{k!}\,u^k e^{-\left(u/\alpha\right)^p} \ d\left(\fr{u}{\alpha}\right)^p ={}\\
{}=
1 - \sum\limits_{k=0}^\infty \int\limits_0^{\infty} 
\fr{(-\theta x)^k}{k!} \left(\alpha t^{1/p}\right)^{k}e^{-t} \, dt ={}\\
{}= 1 - \G_{1/p,\, 1}(-\theta \alpha x)\,,
\end{multline*}
откуда, воспользовавшись свойством~4 теоремы~1, получаем выражение для~$f_\rho(x)$.

Вычислим момент $n$-го порядка для коэффициента загрузки. Имеем:
\begin{multline*}
\e\,\rho^n = \int\limits_0^\infty x^n \theta \alpha \G_{1/p,\, 1/p+1}
(-\theta \alpha x) \, dx = {}\\
{} = \theta \alpha \int\limits_0^\infty \int\limits_0^{\infty} 
\exp \left\lbrace -\theta \alpha t^{1/p}x\right\rbrace t^{1/p}e^{-t} x^{n}\, dt  dx={}\\
{}=\int\limits_0^\infty \fr{e^{-t}}{(\theta \alpha)^n t^{n/p}} \times{}\\
{}\times
\int\limits_0^{\infty} \exp \left\lbrace -\theta \alpha t^{1/p}x\right\rbrace 
(\theta \alpha t^{1/p} x)^n \, d(\theta \alpha t^{1/p} x) \, dt={}\\
{}= \fr{\Gamma(n+1)}{{(\theta \alpha)^n}}\int\limits_0^\infty e^{-t}t^{-n/p}  \, dt 
= \fr{n!}{{(\theta \alpha)^n}}\Gamma\left(
1-\fr{n}{{p}}\right)\,, \\
 p>n\,.
\end{multline*}

Теперь получим функцию распределения для вероятности <<непотери>> вызова~$\pi$. Имеем:
\begin{multline*}
F_\pi(x) =  1 - \p\left(\rho<\fr{{1-x}}{{x}}\right) ={}\\
{}=
  \G_{1/p, 1}\left(-\fr{\theta \alpha (1-x)}{x}\right)\,,\enskip
  x\in(0,1)\,,
\end{multline*}
откуда, воспользовавшись свойством~4 теоремы~1, получаем выражение для $f_\pi(x)$.

Теорема доказана.

\smallskip


\noindent
\textbf{Теорема 3.}\ 
\textit{Пусть параметр входящего потока~$\lambda$ имеет распределение 
Вейбулла $W(q,\theta)$, $0\hm<q\hm<1$, $\theta\hm>0$, а~параметр обслуживания~$\mu$ 
имеет экспоненциальное распределение $M(\alpha)$, $\alpha\hm>0$, причем~$\lambda$ 
и~$\mu$ независимы. Тогда при $x\hm>0$ функция распределения, плотность 
и~математическое ожидание коэффициента загрузки~$\rho$ имеют вид}:
\begin{align*}
F_\rho(x) &=  1 - \G_{q, 1}\left(-\fr{x^q}{(\alpha \theta)^q}\right)\,; \\
f_\rho(x) &= \fr{qx^{q-1}}{{(\theta \alpha)^q}} \G_{q, q+1}
\left( -\fr{x^q}{(\alpha \theta)^q}\right)\,; 
\\
\e\,\rho&=(\theta \alpha)^q \Gamma(1-q)\,,
\end{align*}
\textit{a функция распределения и~плотность вероятности <<непотери>> вызова~$\pi$ 
при $x\hm\in(0,1)$ определяются соотношениями}:
\begin{align*}
F_\pi(x) &= \G_{q,\, 1}\left(-\fr{(1-x)^q}{(\alpha \theta x)^q}\right)\,;
\\
f_\pi(x) &= \fr{q}{{(\theta \alpha)^q x^2}}\left(
\fr{1-x}{{x}}\right)^{q-1}\! \G_{q, q + 1}\left(
-\fr{(1-x)^q}{(\alpha \theta x)^q}\right).
\end{align*}


\noindent
Д\,о\,к\,а\,з\,а\,т\,е\,л\,ь\,с\,т\,в\,о\,.\ \ Для $x\hm>0$ имеем:
\begin{multline*}
F_\rho(x) =
 \int\limits_0^{\infty} \left(1-e^{-(ux/\theta)^q}\right)\alpha e^{-\alpha u}\, du={}\\
 {}=
 1 - \int\limits_0^{\infty} e^{-\alpha u}\sum\limits_{k=0}^\infty 
 \fr{(-1)^k \left(ux/\theta\right)^{qk}}{{k!}} \, d(\alpha u) ={}\\
{}= 1 - \sum\limits_{k=0}^\infty \int\limits_0^{\infty} 
\fr{(-1)^k}{{k!}} \left(
\fr{tx}{{\alpha \theta}}\right)^{qk} e^{-t} \, dt = {}\\
{}=
1 - \G_{q,\, 1}\left(-\fr{x^q}{(\alpha \theta)^q}\right)\,.
\end{multline*}
Выражение для $f_\rho(x)$ получается из свойства~4 теоремы~1.
Найдем $n$-й момент коэффициента загрузки. Имеем:
\begin{multline*}
\e\,\rho^n ={}\\
{}=
\int\limits_0^\infty 
\fr{qx^{n+q-1}}{{(\alpha\theta )^q}}\sum\limits_{k=0}^\infty
\fr{(-1)^kx^{qk}}{{(\alpha \theta)^{qk}k!}}  
 \int\limits_0^\infty t^{q(k+1)}e^{-t}\, dt dx={}\\
{}= \int\limits_0^\infty \fr{q}{{(\theta \alpha)^q}}
\int\limits_0^\infty \exp\left\lbrace -\left(
\fr{tx}{{\alpha \theta}}\right)^q\right\rbrace x^{q+n-1} t^q e^{-t}\, dt   dx={}\\
 {}= \int\limits_0^\infty \fr{t^q e^{-t}}{{(\theta \alpha)^q}}
\int\limits_0^\infty \exp\left\lbrace -\left(
\fr{t}{{\alpha \theta}}\right)^qz\right\rbrace z^{(n-1)/q + 1} \, dz  dt={}\\
{}= \int\limits_0^\infty e^{-t}\left(
\fr{\theta \alpha}{t}\right)^{n+q-1}\int\limits_0^\infty e^{-u}
 u^{(n-1)/q + 1} \, du dt ={}\\
{}= \int\limits_0^\infty e^{-t}\left(
\fr{\theta \alpha}{{t}}\right)^{n+q-1}\Gamma\left(\fr{n-1}{q} + 2\right) \, dt={}\\
{}=(\theta \alpha)^{n+q-1}\Gamma\left(\fr{n-1}{q} + 2\right) \Gamma (2-n-q)\,, 
\end{multline*}
откуда получаем, что существует только момент первого порядка при $0\hm<q\hm<1.$

Для функции распределения вероятности <<непотери>> вызова~$\pi$ справедливо:
\begin{multline*}
F_\pi(x) = 1 - \p\left(\rho<
\fr{{1-x}}{{x}}\right)={}\\
{}= \G_{q,\, 1}\left(-\fr{(1-x)^q}{(\alpha \theta x)^q}\right)\,, \enskip
x\in(0,1)\,.
\end{multline*}
Выражение для $f_\pi(x)$ получается из соответст\-ву\-юще\-го свойства 
$\G_{\alpha,\, \beta} (x)$.

Теорема доказана.

\smallskip

\noindent
\textbf{Замечание.}\ В теоремах~2 и~3 при $p\hm = 1$ и~$q \hm= 1$ распределение 
Вейбулла $W(1, \alpha)$ совпадает с~экспоненциальным распределением~$M(1/\alpha)$. 
Результаты для общего экспоненциального случая были получены 
ранее в~работе~\cite{KuSh09b}.

\smallskip

\noindent
\textbf{Следствие.}\
Из доказанных выше теорем следует, что гам\-ма-экс\-по\-нен\-ци\-аль\-ная
 функция также обладает свойствами:
\begin{enumerate}
\setcounter{enumi}{10}
\item $\lim\nolimits_{x\to-\infty} \G_{\alpha,\, 1}(x) \hm= 0,$ $0\hm\leq \alpha\hm<1$;
\item $\int\nolimits_{0}^{+\infty} \G_{\alpha,\, \alpha + 1}(-x)\,dx \hm= 1,$ $0\hm<\alpha\hm<1$;
%\item $\int\limits_{-\infty}^{0} \G_{\alpha,\, \alpha + 1}(x)dx = 1,$ $0<\alpha<1$;
\item $\G_{\alpha,\, \alpha + 1}(x)\hm > 0,$ $x\hm\in\mathbb{R},$ $0\hm<\alpha\hm<1$;
\item $\G_{\alpha,\, 1}(x)$ строго монотонна на всей прямой при $0\hm\le\alpha\hm<1$.
\end{enumerate}





{\small\frenchspacing
 {%\baselineskip=10.8pt
 \addcontentsline{toc}{section}{References}
 \begin{thebibliography}{9}

\bibitem{KuSh2015}
\Au{Кудрявцев А.\,А., Шоргин~С.\,Я.}
Байесовские модели в~тео\-рии массового обслуживания и~надежности.~--- 
М.: ФИЦ ИУ РАН, 2015. 76~с.

\bibitem{KuSh09b}
\Au{Кудрявцев А.\,А., Шоргин~В.\,С., Шоргин~С.\,Я.} Байе\-сов\-ские
модели массового обслуживания и~надежности: общий эрланговский
случай~// Информатика и~её применения, 2009. Т.~3. Вып.~4. С.~30--34.

\bibitem{ZhaKuSh}
\Au{Жаворонкова Ю.\,В., Кудрявцев~А.\,А., Шоргин~С.\,Я.}
Байесовская рекуррентная модель роста надежности: бе\-та-рас\-пре\-де\-ле\-ние
параметров~// Информатика и~её применения, 2014. Т.~8. Вып.~2.
С.~48--54.

\bibitem{ZhaKuSh2}
\Au{Жаворонкова Ю.\,В., Кудрявцев~А.\,А., Шоргин~С.\,Я.}
Байесовская рекуррентная модель роста надежности: бета-рав\-но\-мер\-ное
распределение параметров~// Информатика и~её применения, 2015. Т.~9.
Вып.~1. С.~98--105.

\bibitem{Artin}
\Au{Artin E.} The gamma function.~--- New York, NY, USA: 
Holt, Rinehart and Winston, 1964. 39~p.

\bibitem{BeEr}
\Au{Бейтмен Г., Эрдейи~А.} Высшие трансцендентные функции. Т.~1.~/
Пер. с~англ.---  М.: Наука, 1973. 296~с.
(\Au{Bateman~H., Erdelyi~A.}  
{Higher transcendental functions}. Vol.~1.~--- New York\,--\,Toronto\,--\,London: 
McGraw-Hill Book Co., Inc., 1953. 302~p.)

\bibitem{AbSt}
\Au{Abramowitz M., Stegun~I.} 
Handbook of mathematical functions with formulas, graphs, and mathematical tables.~--- 
New York, NY, USA: Dover Publications, 1974. 1046~p.
 \end{thebibliography}

 }
 }

\end{multicols}

\vspace*{-6pt}

\hfill{\small\textit{Поступила в~редакцию 12.06.17}}

\vspace*{8pt}

%\newpage

%\vspace*{-24pt}

\hrule

\vspace*{2pt}

\hrule

%\vspace*{8pt}


\def\tit{GAMMA-EXPONENTIAL FUNCTION\\ IN~BAYESIAN QUEUEING MODELS}

\def\titkol{Gamma-exponential function in~Bayesian queueing models}

\def\aut{A.\,A.~Kudryavtsev and A.\,I.~Titova}

\def\autkol{A.\,A.~Kudryavtsev and A.\,I.~Titova}

\titel{\tit}{\aut}{\autkol}{\titkol}

\vspace*{-9pt}


\noindent
\noindent
Department of Mathematical Statistics, Faculty of Computational 
Mathematics and Cybernetics, M.\,V.~Lomonosov Moscow State University, 
1-52~Leninskiye Gory, GSP-1, Moscow 119991, Russian Federation



\def\leftfootline{\small{\textbf{\thepage}
\hfill INFORMATIKA I EE PRIMENENIYA~--- INFORMATICS AND
APPLICATIONS\ \ \ 2017\ \ \ volume~11\ \ \ issue\ 4}
}%
 \def\rightfootline{\small{INFORMATIKA I EE PRIMENENIYA~---
INFORMATICS AND APPLICATIONS\ \ \ 2017\ \ \ volume~11\ \ \ issue\ 4
\hfill \textbf{\thepage}}}

\vspace*{3pt}


\Abste{This paper considers the Bayesian approach to queueing theory and 
reliability theory. The Bayesian approach is useful for studying systems 
with alternating characteristics, the changes in which happen at the moments 
of time unpredictable for a~researcher, or large groups of systems of the same type. 
In the framework of this approach, it is assumed that key parameters of 
classical systems are not given and only their \textit{a~priori} 
distributions are known. By randomizing the system's parameters, the authors 
randomize its characteristics, for instance, the traffic intensity. 
The gamma-exponential function and some of its properties are introduced 
 as well as the results for probability characteristics of 
the system's traffic intensity and the probability that the claim 
received by the system will not be lost in the cases of the exponential 
and Weibull \textit{a~priori} distributions of $M/M/1/0$ system's\linebreak parameters.}

\KWE{Bayesian approach; queuing systems; reliability; mixed distribution; 
Weibull distribution; exponential distribution; gamma-exponential function}

\DOI{10.14357/19922264170413} 

%\vspace*{-12pt}

\Ack
\noindent
The work was partly supported by the Russian Foundation for Basic 
Research (project 17-07-00577).



%\vspace*{3pt}

  \begin{multicols}{2}

\renewcommand{\bibname}{\protect\rmfamily References}
%\renewcommand{\bibname}{\large\protect\rm References}

{\small\frenchspacing
 {%\baselineskip=10.8pt
 \addcontentsline{toc}{section}{References}
 \begin{thebibliography}{9}


\bibitem{1-kud-1}
\Aue{Kudryavtsev, A.\,A., and S.\,Ya.~Shorgin.} 2015. \textit{Bayesovskie modeli 
v~teorii massovogo obsluzhivaniya i~nadezhnosti} 
[Bayesian models in mass service and reliability theories]. Moscow: FRC
CSC RAS. 76~p.

\bibitem{2-kud-1}
\Aue{Kudryavtsev, A.\,A., V.\,S.~Shorgin, and S.\,Ya.~Shorgin.} 
2009. Bayesovskie modeli massovogo obsluzhivaniya i~nadezhnosti: 
obshchiy erlangovskiy sluchay [Bayesian queueing and reliability models: 
General Erlang case]. \textit{Informatika i~ee Primeneniya~--- Inform. Appl.}
3(4):30--34.

\bibitem{3-kud-1}
\Aue{Zhavoronkova, Iu.\,V., A.\,A.~Kudryavtsev, and S.\,Ya.~Shor\-gin.} 
2014. Bayesovskaya rekurrentnaya mo\-del' ros\-ta na\-dezh\-nosti: beta-raspredelenie 
pa\-ra\-met\-rov [Bayesian recurrent model of reliability growth: 
Beta-distribution of\linebreak parameters]. \textit{Informatika i~ee Primeneniya~--- 
Inform. Appl.} 8(2):48--54.

\bibitem{4-kud-1}
\Aue{Zhavoronkova, Iu.\,V., A.\,A.~Kudryavtsev, and S.\,Ya.~Shor\-gin.} 2015.
Bayesovskaya rekurrentnaya mo\-del' ros\-ta na\-dezh\-nosti: beta-ravnomernoe
raspredelenie pa\-ra\-met\-rov [Bayesian recurrent model of reliability growth: 
Beta-uniform distribution of parameters].
\textit{Informatika i~ee Primeneniya~--- Inform. Appl.} 9(1):98--105.

\bibitem{5-kud-1}
\Aue{Artin, E.} 1964. \textit{The gamma function.} New York, NY: 
Holt, Rinehart and Winston. 39~p. 

\bibitem{6-kud-1}
\Aue{Bateman, H., and A.~Erdelyi.} 1953. 
\textit{Higher transcendental functions}. Vol.~1. New York\,--\,Toronto\,--\,London: 
McGraw-Hill Book Co., Inc. 302~p.

\bibitem{7-kud-1}
\Aue{Abramowitz, M., and I.~Stegun.} 
1974. \textit{Handbook of mathematical functions with formulas, graphs, and mathematical tables}. 
New York, NY: Dover Publications, Inc. 1046~p.
\end{thebibliography}

 }
 }

\end{multicols}

\vspace*{-6pt}

\hfill{\small\textit{Received June 12, 2017}}

%\vspace*{-10pt}

\Contr

\noindent
\textbf{Kudryavtsev Alexey A.} (b.\ 1978)~--- 
Candidate of Sciences (PhD) in physics and mathematics, associate professor, 
Department of Mathematical Statistics, Faculty of Computational Mathematics 
and Cybernetics, M.\,V.~Lomonosov Moscow State University, 1-52~Leninskiye Gory, 
GSP-1, Moscow 119991, Russian Federation; \mbox{nubigena@mail.ru}

\vspace*{3pt}

\noindent
\textbf{Titova Anastasiia I.} (b.\ 1995)~--- 
student, Department of Mathematical Statistics, Faculty of Computational 
Mathematics and Cybernetics, M.\,V.~Lomonosov Moscow State University, 
1-52~Leninskiye Gory, GSP-1, Moscow 119991, Russian Federation; 
\mbox{onkelskroot@gmail.com}
\label{end\stat}


\renewcommand{\bibname}{\protect\rm Литература}  %13
\renewcommand{\figurename}{\protect\bf Figure}
\renewcommand{\tablename}{\protect\bf Table}

\def\stat{pagano}

\def\tit{STUDY OF THE~MMPP/GI/$\infty$ QUEUEING SYSTEM\\ WITH~RANDOM CUSTOMERS' CAPACITIES}

\def\titkol{Study of the~MMPP/GI/$\infty$ queueing system with random customers' capacities}

\def\autkol{E.~Lisovskaya, S.~Moiseeva,   M.~Pagano, and~V.~Potatueva}

\def\aut{E.~Lisovskaya$^{1}$, S.~Moiseeva$^{1}$,   M.~Pagano$^{2}$, and~V.~Potatueva$^{1}$}

\titel{\tit}{\aut}{\autkol}{\titkol}

\index{Lisovskaya E.}
\index{Moiseeva S.}
\index{Pagano M.}
\index{Potatueva V.}
\index{Лисовская Е.\,Ю.}
\index{Моисеева С.\,П.}
\index{Пагано М.}
\index{Потатуева В.\,В.}

%{\renewcommand{\thefootnote}{\fnsymbol{footnote}}
%\footnotetext[1] {This work was supported in part by the
%Russian Foundation for Basic Research (grants 15-07-03007 and 13-07-00223).}}

\renewcommand{\thefootnote}{\arabic{footnote}}
\footnotetext[1]{Tomsk State University, 36~Lenin ave., Tomsk 634050, Russian Federation}
\footnotetext[2]{University of Pisa, 16~Via Caruso, Pisa 56122, Italy}


\vspace*{12pt}

\def\leftfootline{\small{\textbf{\thepage}
\hfill INFORMATIKA I EE PRIMENENIYA~--- INFORMATICS AND APPLICATIONS\ \ \ 2017\ \ \ volume~11\ \ \ issue\ 4}
}%
 \def\rightfootline{\small{INFORMATIKA I EE PRIMENENIYA~--- INFORMATICS AND APPLICATIONS\ \ \ 2017\ \ \ volume~11\ \ \ issue\ 4
\hfill \textbf{\thepage}}}

\Abste{A~queueing system with an infinite number of 
servers is considered. Customers arrive in the system according to a~Markov 
Modulated Poisson Process (MMPP). Each customer carries a~random quantity of 
work (capacity of the customer). In this study, service time does not depend 
on the customers' capacities; the latter are used just to fix some additional 
features of the system's evolution. It is shown that the joint probability 
distribution of the customers' number and total capacities in the system is 
two-dimensional Gaussian under the asymptotic condition of an infinitely 
growing service time. 
Simulation results allow determining the applicability area of the asymptotic result.}

\KWE{infinite-server queueing system; random capacity of customers; Markov Modulated Poisson Process}

\DOI{10.14357/19922264170414}

\vspace*{9pt}


\vskip 12pt plus 9pt minus 6pt

      \thispagestyle{myheadings}

      \begin{multicols}{2}

                  \label{st\stat}

\section{Introduction}

\noindent
Queueing systems represent a~powerful mathematical tool for investigating 
the performance of a~wide variety of real-life systems, ranging 
from telecommunication networks to financial markets, from computer 
architectures to supply chain management and airplane traffic control, 
just to cite a~few. Analytical tractability of the corresponding models 
strongly depends on the nature of the underlying processes (Poisson 
arrivals have many nice features that strongly simplify the analysis) 
and on the system geometry.

Although physical resources are always finite, quite often it is easier 
to study queueing systems in which the corresponding parameters assume 
infinite values. For instance, the overflow probability is often used 
as an upper bound for the loss probability in finite-buffer queues and, 
indeed, asymptotic results are available even for strongly non-Markovian 
systems~\cite{mandjes}. Moreover, infinite-server queueing systems may be 
applicable in case of models with a~limited number of server devices 
as described in~\cite{lit2}.

In this work,  an infinite-server queueing system, fed by 
non-Poisson arrivals with random customers' capacities, is considered.  
Queues with random customers' capacities are  useful for analysis 
and design issues in high-performance computer and communication systems, 
in which service time and customer volume are the independent quantities 
(see~\cite{lit8, lit10} and references therein). For instance, in~\cite{lit10}, 
performance analysis of LTE (Long Term Evolution) networks is carried out 
in terms of flow-level dynamics and the amount of required radio resources 
does not depend on the duration of the flow. Such queues are also important in 
modeling devices, where it is necessary to calculate a~sufficient volume of buffer 
for data storing~\cite{lit9, lit12}.
The results for single-server queues with limited buffer and LIFO 
(last in, first out) service discipline 
were presented in~\cite{lit13}, where algorithms for the calculation of stationary 
characteristics were derived. 

A new trend in the study of queueing systems is the analysis of the systems with 
non-Poisson arrivals and nonexponential service time. So, in the 
works~\cite{lit1, lit2, lit4, lit5, lit11}, queues with renewal arrivals, 
Markovian Arrival processes (MAP), and MMPP 
are studied under various asymptotic conditions. 
The main contribution of this paper consists in extending such analysis, 
focusing on the properties of the bidimensional process describing the 
number of customers and the total capacity in the system when an infinite-server 
queue is fed by MMPP arrivals with random capacities and nonexponential service 
time distribution.


\section{Matematical Model}

\noindent
Consider a~queue with infinite number of servers (Fig.~1) 
and assume that customers arrive according to an MMPP. The input process is 
defined by its generator matrix $\mathbf{Q}=||q_{ij}||$ of size $K\times K$ and 
the conditional rates $\lambda_1,\ldots,\lambda_K$, typically composed into 
the diagonal matrix $\mathbf{\Lambda}=\mathrm{diag}\,\{\lambda_1,\ldots,\lambda_K\} $. 
Denote the underlying\linebreak\vspace*{-12pt}

 { \begin{center}  %fig1
 \vspace*{-1pt}
 \mbox{%
 \epsfxsize=45.705mm 
 \epsfbox{pag-1.eps}
 }


\end{center}


\noindent
{{\figurename~1}\ \ \small{Queue MMPP/GI/$\infty$ with random customers' capacities}}
}

\vspace*{9pt}

\addtocounter{figure}{1}



\noindent
 Markov chain of the MMPP as $k(t) \in 1,2,\ldots,K$. 
Let each customer has some random capacity $v>0$ with distribution function~$G(y)$. 
An arriving customer instantly occupies a~server in the system and its service 
time has distribution function $B(x)$; when the service is completed, the customer 
leaves the system. Customers' capacities and service times are mutually 
independent and do not dependent on the epochs of customers' arrivals.

 


Denote by $i(t)$ and $V(t)$ the number of customers in the system at 
time~$t$ and their total capacity, respectively. Let us obtain the probabilistic 
characteristics of two-dimensional process~$\{i(t),V(t)\}$. This process 
is not Markovian; therefore,  the dynamic screening method has been used for 
its investigation.

Consider two time axes that are numbered as~0 and~1 (Fig.~2). 
Let axis~0 shows the epochs of customers' arrivals, while axis~1 
corresponds to the screened process.



Let us introduce a~function $S(t)$ (dynamic probability) that satisfies the condition 
$0\le S(t) \le 1$.
Let us assume that a~customer, arriving in the system at time~$t$, is screened 
to axis~1 with probability $S(t)$, and not screened with probability $1-S(t)$.

Let the system be empty at moment~$t_0$ and let us fix some arbitrary moment~$T$ 
in the future. $S(t)$ represents the probability that a~customer arriving at time~$t$ 
will be serviced in the system by the moment~$T$. 
It is easy to show~\cite{lit5} that $S(t)=1-B(T-t) $ for $t_0\le t\le T$.




Denote by $n(t)$ and $W(t)$ the number of arrivals screened before the moment~$t$ 
on axis~1 and their total\linebreak\vspace*{-12pt}

{ \begin{center}  %fig2
 \vspace*{9pt}
 \mbox{%
 \epsfxsize=77.897mm 
 \epsfbox{pag-2.eps}
 }

\vspace*{6pt}

\noindent
{{\figurename~2}\ \ \small{Screening of the customers' arrivals}}


\end{center}
}

%\vspace*{9pt}

\addtocounter{figure}{1}

\noindent
 capacity, respectively. 
As it is shown in~\cite{lit4}, the probability distribution of the number of 
customers in the system at the moment~$T$ coincides with the probability 
distribution of the number of screened arrivals on the axis:
$$
P\{i(T)=m\}=P\{n(T)=m\}
$$
for all $m=0,1,2,\ldots$ It is easy to prove the same property for 
the extended process $\{i(t),V(t)\}$:
\begin{multline}
\label{eq1-p}
P\{i(T)=m,V(T)<z\}\\
{}=P\{n(T)=m,W(T)<z\}
\end{multline}
for all $m=0,1,2,\ldots$ and $z\ge 0$. 
Let us use Eqs.~\eqref{eq1-p} for the investigation of the process $\{i(t),V(t)\}$ 
via the analysis of the process $\{n(t),W(t)\}$.

\section{Kolmogorov Differential Equations}

\noindent
Let us consider the three-dimensional Markovian process $\{k(t),n(t),W(t)\}$. 
Denoting the probability distribution of this process by 
$P(k,n,w,t)=P\{k(t)\linebreak =k,n(t)=n,W(t)<w\}$ and taking into account the formula 
of total probability, one can write the following system of Kolmogorov 
differential equations:
\begin{multline*}
\hspace*{-9pt}\fr{\partial P(k,n,w,t)}{\partial t}=\lambda_kS(t)\!\left[
\int\limits_0^z\!\!\! P(k,n-1,w-y,t)\,dG(y){}\right.\hspace*{-1pt}\\
\left.{}-P(k,n,w,t)
\vphantom{\int\limits_0^z}\right]
+\sum_vP(\nu,n,w,t)q_{\nu k}
\end{multline*}
for $k=1,\ldots , K$; $n=0,1,2,\ldots$; $w>0$.

Let us introduce the partial characteristic function:
$$
h(k,u_1,u_2,t)=\sum\limits_{n=0}^{\infty}e^{ju_1n}
\int\limits_0^\infty e^{ju_2w}P(k,n,dw,t)
$$
where $j=\sqrt{-1}$ is the imaginary unit. Then, one can write the following equations:
\begin{multline*}
\fr{\partial h(k,u_1,u_2,t)}{\partial t}\\
{}=
h\left(k,u_1,u_2,t\right)\lambda_kS(t)\left( e^{ju_1}G^*(u_2)-1\right) \\
{}+
\sum\limits_{\nu}h(\nu,n,w,t)q_{\nu k}
\end{multline*}
for $k=1,\ldots,K$ where $G^*(u)=\int\nolimits_0^\infty e^{juy}dG(y)$.

Let us rewrite this system in the matrix form:
\begin{multline}
\label{eq2-p}
\fr{\partial\mathbf{h}(u_1,u_2,t)}{\partial t}\\
{}=
\mathbf{h}(u_1,u_2,t)\left[
\mathbf{\Lambda} S(t)\left( e^{ju_1}G^*\left(u_2\right)-1\right) +\mathbf{Q}\right]
\end{multline}
with the initial condition
\begin{equation}
\label{eq3}
{\mathbf{h}}\left(u_1,u_2,t_0\right)=\mathbf{r}
\end{equation}
where
$$
\mathbf{h}\left(u_1,u_2,t\right)=\left[h\left(1,u_1,u_2,t\right),\ldots,
h\left(K,u_1,u_2,t\right)\right]
$$
and ${\mathbf{r}}=[r(1),\ldots,r(K)]$ represents the stationary 
distribution of the underlying Markov chain, i.\,e., 
vector~$\bf{r}$ satisfies the following linear system:
\begin{equation}
\label{eq4}
\left.
\begin{array}{l}
\mathbf{rQ}=\mathbf{0}\,; \\[6pt]
\mathbf{re}=1
\end{array}
\right\}
\end{equation}
where $\mathbf{e}$ is the~column vector with all entries equal to~1.

\section{Asymptotic Analysis}

\noindent
In general, the exact solution of Equation~\eqref{eq2-p} is not available, 
but it may be found under asymptotic conditions. In this paper,  
the case of infinitely growing service time is considered.

Denote by
$$
b_1=\int\limits_0^\infty xdB(x)=\int\limits_0^\infty(1-B(x))\,dx
$$
the mean service time; then, the asymptotic condition is $b_1\to\infty$.

Let us solve Problem \eqref{eq2-p}--\eqref{eq3} under such asymptotic condition 
and we obtain  approximate solutions with different order of accuracy, named as 
``first-order asymptotic'' 
${\mathbf{h}}(u_1,u_2,t)\approx{\mathbf{h}}^{(1)}(u_1,u_2,t)$ and  
``second-order asymptotic'' 
${\mathbf{h}}(u_1,u_2,t)\approx{\mathbf{h}}^{(2)}(u_1,u_2,t)$.

\subsection{First-order asymptotic analysis}

\noindent
Let us formulate and prove the following statement.

\smallskip

\noindent
\textbf{Lemma.}\ 
\textit{The first-order asymptotic characteristic function of the probability 
distribution of the process $\{k(t),n(t),W(t)\}$  has the form}:
\begin{equation*}
\mathbf{h}^{(1)}(u_1,u_2,t)=\mathbf{r} 
\exp\left\{ \!\left(ju_1\kappa_1+ju_2\kappa_1a_1\right)
\!\int\limits_{t_0}^t \! S(v)\,dv\!\right\}
\end{equation*}
\textit{where  $\kappa_1=\mathbf{r\Lambda e}$ and  
$a_1=\int\limits\nolimits_0^\infty ydG(y)$ is the mean customer capacity}.

\smallskip


\noindent
P\,r\,o\,o\,f\,.\ \ 
By performing the substitutions
\begin{gather*}
\varepsilon=\fr{1}{b_1}\,;\quad
 \varepsilon t=\tau\,;\quad 
 \varepsilon t_0=\tau_0\,;\\[6pt]
  u_1=\varepsilon x_1\,;\enskip
   u_2=\varepsilon x_2\,;\enskip 
   S(t)=S_1(\tau)\,;\\[6pt] 
   \mathbf{h}(u_1,u_2,t)=\mathbf{f}_1(x_1,x_2,\tau,\varepsilon)
%\label{eq5}
\end{gather*}
in expressions~\eqref{eq2-p} and~\eqref{eq3}, one obtains
\begin{multline}
\varepsilon\fr{\partial \mathbf{f}_1(x_1,x_2,\tau,\varepsilon)}{\partial \tau}\\
\!\!\!\!\!{}=
\mathbf{f}_1(x_1,x_2,\tau,\varepsilon)\!\left[\mathbf{\Lambda} 
S_1(\tau)\left( e^{j\varepsilon x_1}G^*(\varepsilon x_2)\!-\!1\right) +\mathbf{Q}\right]\!\!
\label{eq6}
\end{multline}
with the initial condition
\begin{equation}
\label{eq7}
\mathbf{f}_1(x_1,x_2,\tau_0,\varepsilon)=\mathbf{r}\,.
\end{equation}

Let us find the asymptotic solution of Problem~\eqref{eq6}--\eqref{eq7} 
$\mathbf{f}_1(x_1,x_2,\tau)=
\lim\nolimits_{\varepsilon\to 0}\mathbf{f}_1(x_1, x_2,\tau,\varepsilon)$
in two steps.

\textit{Step~1.} Let $\varepsilon\to 0$ in~\eqref{eq6}--\eqref{eq7}; 
then, one obtains the following system of equations:
$$
\left\{ 
\begin{array}{l}
\mathbf{f}_1\left(x_1,x_2,\tau\right)\mathbf{Q}=\mathbf{0}\,;\\[6pt]
\mathbf{f}_1\left(x_1,x_2,\tau_0\right)=\mathbf{r}\,.
\end{array}
\right.
$$

Taking into account~\eqref{eq4}, one can conclude that $\mathbf{f}_1(x_1,x_2,\tau)$  
can be expressed as
\begin{equation}
\label{eq8}
\mathbf{f}_1(x_1,x_2,\tau)=\mathbf{r}\Phi_1(x_1,x_2,\tau)
\end{equation}
where $\Phi_1(x_1,x_2,\tau)$ is some scalar function which satisfies the condition
\begin{equation}
\label{eq9}
\Phi_1(x_1,x_2,\tau_0)=1\,.
\end{equation}

\textit{Step 2.} Let us multiply~\eqref{eq6} by vector~{\bf e}, substitute~\eqref{eq8}, 
divide the result by~$\varepsilon$, and perform the asymptotic transition 
$\varepsilon\to 0$. Then, taking into account that $\mathbf{Qe}=\mathbf{0}$ 
and $\mathbf{re}=1$, one obtains the following differential equation 
for the function $\Phi_1(x_1,x_2,\tau)$:
\begin{multline}
\label{eq10}
\fr{\partial\Phi_1(x_1,x_2,\tau)}{\partial\tau}\\
{}=
\Phi_1\left(x_1,x_2,\tau\right)S_1(\tau)\left(jx_1\kappa_1+jx_2\kappa_1a_1\right)\,.
\end{multline}

The solution of Problem~\eqref{eq9}--\eqref{eq10} is as follows:
$$
\Phi_1(x_1,x_2,\tau)=\exp\left\{ \!\left(jx_1\kappa_1+jx_2\kappa_1a_1\right)\!
\int\limits_{\tau_0}^{\tau}\!S_1(v)\,dv\right\}.
$$
Substituting this expression into~\eqref{eq8}, one obtains
$$
\mathbf{f}_1(x_1,x_2,\tau)=
\mathbf{r}\exp\left\{ \!\left(jx_1\kappa_1+jx_2\kappa_1a_1\right)
\!\int\limits_{\tau_0}^{\tau}\!S_1(v)\,dv\!\right\}.\hspace*{-0.69418pt}
$$

Therefore, one can write
\begin{multline*}
\mathbf{h}^{(1)}(u_1,u_2,t)=\mathbf{f}_1\left(x_1,x_2,\tau,\varepsilon\right)\approx
\mathbf{f}_1\left(x_1,x_2,\tau\right)
\\
{}=\mathbf{r}\exp\left\{ \left(jx_1\kappa_1+jx_2\kappa_1a_1\right)
\int\limits_{\tau_0}^{\tau}S_1(v)dv\right\}\\
{} =
\mathbf{r}\exp\left\{ \left( ju_1\kappa_1+ju_2\kappa_1a_1\right) 
\int\limits_{t_0}^tS(v)\,dv\right\} \,.
\end{multline*}
Thus, the proof is complete.


\subsection{Second-order asymptotic analysis}

\noindent
The main result is the following theorem.

\smallskip

\noindent
\textbf{Theorem.}\ 
\textit{The second-order asymptotic characteristic function of the 
probability distribution of the process  $\{k(t),n(t),W(t)\}$ has the form}:

\noindent
\begin{multline*}
\mathbf{h}^{(2)}\left(u_1,u_2,t\right)\\
{}=\mathbf{r}\exp\left\{ 
\left(ju_1\kappa_1+ju_2\kappa_1a_1\right)\int\limits_{t_0}^tS(v)\,dv\right.
\\
{}+\fr{(ju_1)^2}{2}\left(\kappa_1\int\limits_{t_0}^tS(v)\,dv+
\kappa_2\int\limits_{t_0}^tS^2(v)\,dv\right)
\\
{}+\fr{(ju_2)^2}{2}\left(\kappa_1a_2\int\limits_{t_0}^tS(v)\,dv+
\kappa_2a_1^2\int\limits_{t_0}^tS^2(v)\,dv\right)
\\
\left.{}+ju_1ju_2\left(\kappa_1a_1\int\limits_{t_0}^tS(v)\,dv+
\kappa_2a_1\int\limits_{t_0}^tS^2(v)\,dv\right)\right\}\hspace*{-3.166pt}
\end{multline*}
\textit{where  $\kappa_2=2\mathbf{g}(\mathbf{\Lambda}-\kappa_1\mathbf{I})\mathbf{e}$;  
$a_2=\int\nolimits_0^\infty y^2dG(y)$;  and the row vector  $\mathbf{g}$  
satisfies the linear matrix system} 
$$
\left\{
\begin{array}{rl}
\mathbf{gQ}&=\mathbf{r}(\kappa_1\mathbf{I}-\mathbf{\Lambda})\,; \\[6pt]
\mathbf{ge}&=const\,.
\end{array}
\right.
$$



\noindent
P\,r\,o\,o\,f\,.\ \  
Let $\mathbf{h}_2(x_1,x_2,t)$ be a~vector function that satisfies the equation:
\begin{multline}
\label{eq12}
\mathbf{h}\left(u_1,u_2,t\right)=
\mathbf{h}_2\left(u_1,u_2,t\right)\\
{}\times\exp
\left\{ \!\left(ju_1\kappa_1+ju_2\kappa_1a_1\right)\int\limits_{t_0}^tS(v)\,dv\right\}\,.
\end{multline}

Substituting this expression into~\eqref{eq2-p} and~\eqref{eq3}, one obtains
the following problem:
\begin{multline}
\fr{\partial {\mathbf{h}_2(u_1,u_2,t)}}{\partial t}\\
{}=
\mathbf{h}_2(u_1,u_2,t)\left[(e^{ju_1}G^*(u_2)-1)S(t)\mathbf{\Lambda}\right.
\\
\left.{}-\left(ju_1\kappa_1+ju_2\kappa_1a_1\right)S(t)\mathbf{I}+\mathbf{Q}
\vphantom{e^{ju_1}G^*(u_2)}
\right]
\label{eq13}
\end{multline}
with the initial condition
\begin{equation}
\label{eq14}
{\bf h}_2(u_1,u_2,t_0)={\bf r}
\end{equation}
where {\bf I} is the identity matrix.

Let us make the substitutions:
\begin{equation}
\left.
\begin{array}{c}
\varepsilon^2=\fr{1}{b_1}\,;\quad
\varepsilon^2 t=\tau\,;\quad
\varepsilon^2 t_0=\tau_0\,;\\[6pt] 
u_1=\varepsilon x_1\,;\enskip 
u_2=\varepsilon x_2\,;\enskip 
S(t)=S_1(t)\,;\\[6pt] 
{\bf h}_2(u_1,u_2,t)={\bf f}_2(x_1,x_2,\tau,\varepsilon)\,.
\end{array}
\right\}
\label{eq15}
\end{equation}

Using these notations, Problem~\eqref{eq13}--\eqref{eq14} can be rewritten in the form
\begin{multline}
\varepsilon^2\fr{\partial {\bf f}_2(x_1,x_2,\tau,\varepsilon)}{\partial \tau}\\
{}=
{\bf{f}}_2(x_1,x_2,\tau,\varepsilon)\left[
\mathbf{\Lambda} S_1(\tau)(e^{j\varepsilon x_1}G^*(\varepsilon x_2)-1)\right. 
\\
\left.{} -\left( j\varepsilon\kappa_1x_1+j\varepsilon\kappa_1x_2a_1\right)
S_1\left( \tau\right) \mathbf{I}+ \mathbf{Q}
\vphantom{e^{j\varepsilon x_1}G^*(\varepsilon x_2)}
\right]
\label{eq16}
\end{multline}
with the initial condition
\begin{equation}
\label{eq17}
{\bf f}_2(x_1,x_2,\tau_0,\varepsilon)={\bf r}\,.
\end{equation}

Let us find the asymptotic solution of this problem 
${\bf f}_2(x_1,x_2,\tau)=
\lim\limits_{\varepsilon\to 0}{\bf f}_2(x_1,x_2,\tau,\varepsilon)$ in three steps.

\textit{Step~1.} Letting $\varepsilon\to 0$ in~\eqref{eq16}--\eqref{eq17}, 
one obtains the following system of equations:
$$
\left\{
\begin{array}{l}
{\bf f}_2\left(x_1,x_2,\tau\right)\mathbf{Q}=\mathbf{0}\,; \\[6pt]
{\bf f}_2\left(x_1,x_2,\tau_0\right)={\bf r}\,.
\end{array}
\right.
$$
Therefore, taking into account~\eqref{eq4}, one can write:
\begin{equation}
\label{eq18}
{\bf f}_2\left(x_1,x_2,\tau\right)={\bf r}\Phi_2\left(x_1,x_2,\tau\right)
\end{equation}
where $\Phi_2(x_1,x_2,\tau)$  is some scalar function which satisfies the condition
\begin{equation}
\label{eq19}
\Phi_2\left(x_1,x_2,\tau_0\right)=1\,.
\end{equation}

\textit{Step 2.} Using~\eqref{eq18}, the function ${\bf f}_2(x_1,x_2,\tau)$  
can be represented in the expansion form:
\begin{multline}
{\bf f}_2\left(x_1,x_2,\tau,\varepsilon\right)\\
{}=\Phi_2\left(x_1,x_2,\tau\right)\left[{\bf r}+\mathbf{g}S_1(\tau)
\left(j\varepsilon x_1+j\varepsilon x_2a_1\right)\right]\\
{}+{\bf O}(\varepsilon^2)
\label{eq20}
\end{multline}
where {\bf g} is the~row vector that satisfies the condition ${\bf ge}= const$ 
and ${\bf O}(\varepsilon^2)$  is the row vector of the second-order infinitesimals. 
Let us use substitution~\eqref{eq20} and the expansion
$$
e^{j\varepsilon x}=1+j\varepsilon x+O\left( \varepsilon^2\right) 
$$
in Eq.~\eqref{eq16}. Taking into account~\eqref{eq4} and 
making the transition $\varepsilon\to 0$, one obtains the 
following matrix equation for the vector~$\mathbf{g}$:
$$
{\bf gQ}={\bf r}\left(\kappa_1\mathbf{I}-\mathbf{\Lambda}\right)\,.
$$

\textit{Step~3.} Let us multiply Eq.~\eqref{eq16} by vector~\textbf{e} 
and use expression~\eqref{eq20} and the second-order expansion:
$$
e^{j\varepsilon x}=1+j\varepsilon x+\fr{(j\varepsilon x)^2}{2}+O\left(\varepsilon^3\right)\,.
$$

After some transformations, using the notation
$$
\kappa_2=2{\bf g}\left(\mathbf{\Lambda}-\kappa_1\mathbf{I}\right){\bf{e}}\,,
$$
one obtains the following differential equation for the function $\Phi_2(x_1,x_2,\tau)$:
\begin{multline*}
\fr{\partial\Phi_2(x_1,x_2,\tau)}{\partial\tau}\\
{}=
\Phi_2(x_1,x_2,\tau) \left[\fr{(jx_1)^2}{2}\left(\kappa_1S_1(\tau)+
\kappa_2S_1^2(\tau)\right)\right.
\\
{}+\fr{(jx_2)^2}{2}\left(\kappa_1a_2S_1(\tau)+\kappa_2a_1^2S_1^2(\tau)\right)\\
\left.{}+jx_1jx_2\left(\kappa_1a_1S_1(\tau)+\kappa_2a_1S_1^2(\tau)\right)
\vphantom{\fr{(jx_1)^2}{2}}\right]\,.
\end{multline*}

The solution of this equation with initial condition~\eqref{eq19} is as follows:
\begin{multline*}
\Phi_2\left(x_1,x_2,\tau\right)\\
{}= 
\exp\left\{ \fr{(jx_1)^2}{2}\left(
\kappa_1\int\limits_{\tau_0}^{\tau}S_1(v)\,dv+\kappa_2\int\limits_{\tau_0}^{\tau}
S_1^2(v)\,dv\right)\right.
\\
{}+\fr{(jx_2)^2}{2}\left(\kappa_1a_2\int\limits_{\tau_0}^{\tau}S_1(v)\,dv+
\kappa_2a_1^2\int\limits_{\tau_0}^{\tau}\!S_1^2(v)\,dv\right)
\\
\!\left.{}+
jx_1jx_2\left(\kappa_1a_1\int\limits_{\tau_0}^{\tau}\!S_1(v)\,dv+
\kappa_2a_1\int\limits_{\tau_0}^{\tau}\!S_1^2(v)\,dv\right)\!\right\}.\hspace*{-0.39064pt}
\end{multline*}

Substituting this expression in formula~\eqref{eq18} and performing 
the substitutions that are inverse to~\eqref{eq15} and~\eqref{eq12}, one obtains
\begin{multline*}
{\bf{h}}^{(2)}\left(u_1,u_2,t\right)\\
{}=
{\bf{r}} \exp\left\{ \left(ju_1\kappa_1+ju_2\kappa_1a_1\right)
\int\limits_{t_0}^tS(v)\,dv\right. 
\\
{}+\fr{(ju_1)^2}{2}\left(\kappa_1\int\limits_{t_0}^tS(v)\,dv+
\kappa_2\int\limits_{t_0}^tS^2(v)\,dv\right)\\
{}+\fr{(ju_2)^2}{2}\left(\kappa_1a_2\int\limits_{t_0}^tS(v)\,dv+
\kappa_2a_1^2\int\limits_{t_0}^tS^2(v)\,dv\right)
\\
\left.
{}+ju_1ju_2\left(\kappa_1a_1\int\limits_{t_0}^tS(v)\,dv+
\kappa_2a_1\int\limits_{t_0}^tS^2(v)\,dv\right)\right\} 
\end{multline*}
for the asymptotic characteristic function of the process
 $\{k(t),n(t),W(t)\}$. The proof is complete.
 
\columnbreak
 
 \noindent
 \textbf{Corollary.}
Assuming $t = T$ and $t_0\to -\infty $ and using Eqs.~\eqref{eq1-p}, one obtains 
the steady-state characteristic function of the process under study $\{i(t),V(t)\}$:
\begin{multline}
h\left(u_1,u_2\right)= 
\exp\left\{ \left(ju_1\kappa_1b_1+ju_2\kappa_1a_1b_1\right)\right.\\
{}+
\fr{(ju_1)^2}{2}\left(\kappa_1b_1
+\kappa_2b_2\right)
+\fr{(ju_2)^2}{2}\left(\kappa_1a_2b_1+\kappa_2a_1^2b_2\right)\\
\left.{}+
ju_1ju_2\left(\kappa_1a_1b_1+\kappa_2a_1b_2\right)\right\}
\label{eq21}
\end{multline}
where 
$$
b_1=\int\limits_0^{\infty}(1-B(v))\,dv\,;\enskip  
b_2=\int\limits_0^{\infty}(1-B(v))^2\,dv\,.
$$


From the form of the characteristic function~\eqref{eq21}, it is clear 
that the probability distribution of the two-dimensional process $\{i(t),V(t)\}$ 
is asymptotically Gaussian with vector of means
$$
{\bf a}=\left[
\begin{array}{lr}
\kappa_1b_1 &  \kappa_1a_1b_1
\end{array}
\right]
$$
and covariance matrix
\begin{multline*}
\mathbf{K}=\left[
\begin{array}{cc}
\sigma_1^2 &  K_{12} \\
K_{12} & \sigma_2^2
\end{array}
\right]\\
{}=
\left[
\begin{array}{cc}
\kappa_1b_1+\kappa_2b_2 &  \kappa_1a_1b_1+\kappa_2a_1b_2 \\
\kappa_1a_1b_1+\kappa_2a_1b_2 & \kappa_1a_2b_1+\kappa_2a_1^2b_2
\end{array}
\right]\,.
\end{multline*}

Therefore, the correlation coefficient is given by
$$
r=\fr{K_{12}}{\sigma_1\sigma_2}=
\fr{\kappa_1a_1b_1+\kappa_2a_1b_2}{\sqrt{\kappa_1b_1+\kappa_2b_2}\,
\sqrt{\kappa_1a_2b_1+\kappa_2a_1^2b_2}}\,.
$$

\begin{figure*}[b] %fig3
\vspace*{1pt}
 \begin{center}
 \mbox{%
 \epsfxsize=163.767mm 
 \epsfbox{pag-3.eps}
 }
  \end{center}
\vspace*{-11pt}
\Caption{Distributions of the number of customers~(\textit{a})
and of the total capacity~(\textit{b}) for different values of~$N$:
left column~--- $N=10$; right column~--- $N=100$;
\textit{1}~--- theoretical results; and \textit{2}~--- simulation}
\label{fig:fig3}
\end{figure*}

\section{Numerical Example}

\noindent
Result~\eqref{eq21} is obtained under the asymptotic condition $b_1\to\infty$. 
Therefore, it may be used just as an approximation when~$b_1$ is large enough. 
To test its practical applicability, the present authors
 considered several numerical examples, 
varying all the system parameters (including the distributions of the service 
time and of the customer capacity). Since all the different simulation sets led 
to similar results, for sake of brevity, in the following, 
just one of them is discussed in detail. In particular, let us
assume that the input MMPP is characterized by the matrices:
$$
\mathbf{Q}=\left[
\begin{array}{rrr}
-0.8 & 0.4 & 0.4\\
0.3 & -0.6 & 0.3\\
0.4 & 0.4 & -0.8
\end{array}\right]
$$
and
$$
\mathbf{\Lambda}=\left[
\begin{array}{ccc}
0.5 & 0 & 0\\
0 & 1 & 0\\
0 & 0 & 1.5
\end{array}\right] \,.
$$

 %\begin{table*} %tabl1
\begin{center}
\begin{minipage}[t]{34mm}
{{\tablename~1}\ \ \small{Kolmogorov distances between simulation results and asymptotic 
values for the number of customers in the system}}

\vspace*{6pt}


{\small
\tabcolsep=14pt
\begin{tabular}{cc}
\hline
$N$ & $\Delta$\\
\hline
\hphantom{9}1 & 0.265\\
10 & 0.039\\
15 & 0.032\\
20 & \textbf{0.027}\\
25 & \textbf{0.025}\\
50 & \textbf{0.017}\\
100\hphantom{9} & \textbf{0.012}\\
\hline
\end{tabular}
}
\end{minipage}
\hfill
%\end{center}
%\end{table*}
%\begin{table*}\small %tabl2
%\begin{center}
\begin{minipage}[t]{34mm}
{{\tablename~2}\ \ \small{Kolmogorov distances between simulation results and asymptotic values for the total capacity in the system}}

\vspace*{6pt}

{\small 
\tabcolsep=14pt
\begin{tabular}{cc}
\hline
$N$ & $\Delta$\\
\hline
\hphantom{9}1 & 0.355\\
10 & 0.033\\
15 & \textbf{0.025}\\
20 & \textbf{0.021}\\
25 & \textbf{0.019}\\
50 & \textbf{0.013}\\
100\hphantom{9} &\textbf{0.010}\\
\hline
\end{tabular}
}
\end{minipage}
\end{center}

\vspace*{9pt}
%\end{table*}

\setcounter{table}{2}

Hence, the fundamental rate of arrivals is $\kappa_1\linebreak =\mathbf{r\Lambda e}=1$ 
customers per time unit. Let us also assume that customers' capacities have 
uniform distribution in the range $[0;1]$ and service time has gamma distribution 
with shape and inverse scale parameters $\alpha = 1.5$ and $\beta = \alpha/ N$, 
respectively. So, when $N \to\infty$, one obtains the asymptotic condition of 
an infinite growing service time ($b_1 = \alpha / \beta = N \to \infty$).

The goal is to find a~lower bound of parameter~$N$ for the applicability 
of the approximation~\eqref{eq21}. To this aim,  series of
 simulation experiments have been  carried out for increasing values of~$N$ and 
  the asymptotic
  distributions    have been compared with the
  empiric ones by using the Kolmogorov distance~\cite{lit3,lit6}
$$
\Delta=\sup\limits_x\left| F\left( x\right) -A\left( x\right) \right| 
$$
as an accuracy measure. Here,~$F(x)$ is the cumulative distribution function 
built on the basis of simulation results and $A(x)$ is the Gaussian 
approximation based on~\eqref{eq21}.

Let us consider the marginal distributions of the customers' number and the 
total capacity in the system.

In the first case, the asymptotic values of mean and variance are equal 
to~$N$ and~$1.144 N$, respectively, and the corresponding values of the 
Kolmogorov distance for increasing values of parameter~$N$ are presented in 
Table~1. Similarly, for the total capacity in the system, 
mean and variance are equal to~$0.5 N$ and~$0.369 N$, respectively, and 
Table~2 shows the Kolmogorov distance.


One can notice that the asymptotic results become more accurate 
when the parameter~$N$  increases. Fig-\linebreak\vspace*{-12pt}

 { \begin{center}  %fig1
 \vspace*{-1pt}
 \mbox{%
 \epsfxsize=77.763mm 
 \epsfbox{pag-5.eps}
 }


\end{center}


\noindent
{{\figurename~4}\ \ \small{Relative error for the variance of the number of customers~$i(t)$~(\textit{1}) 
and the total capacity $V(t)$~(\textit{2})}}
}

\vspace*{18pt}

\addtocounter{figure}{1}


\noindent
ure~\ref{fig:fig3}  
compares the asymptotic approximations with the empirical results for 
the number of customers and the total capacity in the system.



As typically done in the literature~\cite{lit6}, let us suppose that 
an approximation is applicable if its Kolmogorov distance is less than~0.03. 
Hence, one can  conclude that the asymptotic results are applicable for values
 of the parameter~$N$ equal to~15 or more (marked by boldface in 
 Tables~1 and~2).
 



Then, let us compare the asymptotic value of some characteristics of the 
queueing system with the corresponding empirical characteristics, 
using the relative error
$$
\delta=\fr{\left| d-a\right| }{d}
$$
where $d$ denotes the value constructed on the basis of simulation results and~$a$ 
is obtained from~\eqref{eq21}.

In more detail, the mean values of the processes~$i(t)$ and $V(t)$ are very
 close (with $\delta<10^{-5}$ for all~$N$) and the relative errors 
 of the variance decreases with~$N$ as shown in Fig.~4.



Finally, Table~3 shows the relative error for 
the correlation coefficient.

\vspace*{12pt}

%\begin{table*}\small %tabl3
\begin{center}
 \parbox{43mm}{{{\tablename~3}\ \ \small{Relative error for the correlation coefficient}}
 }
\vspace*{6pt}

        \tabcolsep=15pt
        {\small \begin{tabular}{cc}
            \hline
                        \multicolumn{2}{c}{\ }\\[-9pt]
            $N$ & $\delta$\\
            \hline
            \multicolumn{2}{c}{\ }\\[-9pt]
\hphantom{9}1 & {\boldmath{${60\cdot10^{-4}}$}}\hphantom{9}\\
            10 & {\boldmath{$11\cdot10^{-4}$}}\hphantom{9} \\
            15 &             {\boldmath{$7\cdot10^{-4}$}} \\
             20 & {\boldmath{$5\cdot10^{-4}$}} \\
             25 & {\boldmath{$4\cdot10^{-4}$}}\\
             50 &     {\boldmath{$1\cdot10^{-4}$}}\\
             100\hphantom{9}&    {\boldmath{$0.8\cdot10^{-4}$}}\hphantom{.9}\\
            \hline
        \end{tabular}}
    \end{center}
%\end{table*}

\section{Concluding Remarks}

\noindent
In the paper, the queue with MMPP arrivals, infinite number of servers, 
and nonexponential service time is considered. Moreover, random customers' capacities, 
independent of their service time, are assumed.
The analysis is performed under the asymptotic condition of an infinitely 
growing service time. It is shown that two-dimensional probability 
distribution of customers' number and total capacity in the system is 
two-dimensional Gaussian under this asymptotic condition. Numerical 
results show that asymptotic results have enough accuracy for 
the marginal distributions of number of customers and of the 
total capacity in the system when the service rate exceeds the 
fundamental rate of arrivals by at least~15~times.

\vspace*{-6pt}

\Ack
\noindent
This work is supported by the Russian Foundation for Basic research, project 16-31-00292.

\renewcommand{\bibname}{\protect\rmfamily References}

\vspace*{-6pt}

{\small\frenchspacing
{%\baselineskip=10.8pt
\begin{thebibliography}{99}

\bibitem{mandjes} %1
\Aue{Mandjes, M.} 2007. \textit{Large deviations of Gaussian queues.} 
Chichester: Wiley. 340~p.

\bibitem{lit2} %2
\Aue{Melikov, A., L.~Zadiranova, and A.~Moiseev.} 2016. 
Two asymptotic conditions in queue with MMPP arrivals and feedback. 
\textit{Comm. Com. Inf. Sc.} 678:231--240. 
doi: 10.1007/978-3-319-51917-3\_21.

\bibitem{lit10} %3
\Aue{Naumov, V., K.~Samouylov, E.~Sopin, and S.~Andreev.} 2015. 
Two approaches to analyzing dynamic cellular networks with limited resources. 
\textit{6th  Congress (International)
on Ultra Modern Telecommunications and Control Systems and Workshops.} 
St.\ Petersburg. 485--488. 
doi: 10.1109/ICUMT.2014.7002149.

\bibitem{lit8}%4
\Aue{Morozov, E., L.~Potakhina, and O.~Tikhonenko.} 2016. 
Regenerative analysis of a~system with a~random volume of customers. 
\textit{Comm. Com. Inf. Sc.} 638:261--272. 
doi: 10.1007/978-3-319-44615-8\_23.

\bibitem{lit12} %5
\Aue{Tikhonenko, O.\,M., and W.~Kempa.} 2015. 
Queueing systems with processor sharing and limited memory under control of the AQM 
mechanism. \textit{Automat. Rem. Contr.} 76(10):1784--1796. 
doi: 10.1134/S0005117915100069.

\bibitem{lit9} %6
\Aue{Naumov, V.\,A., K.\,E.~Samuilov, and A.\,K.~Samuilov}. 2016. 
On the total amount of resources occupied by serviced customers. 
\textit{Automat. Rem. Contr.} 77(8):1419--1427.



\bibitem{lit13} %7
\Aue{Tikhonenko, O.\,M.} 2010. 
Queueing system with processor sharing and limited resources. 
\textit{Automat. Rem. Contr.} 71(5):803--815.

\bibitem{lit11} %8
\Aue{Pankratova, E.\,V., and S.\,P.~Moiseeva.} 2014. 
Queueing system MAP/M/$\infty$ with $n$ types of customers. 
\textit{Comm. Com. Inf. Sc.} 487:356--366.

\bibitem{lit4} %9
\Aue{Moiseev, A., and A.~Nazarov.} 2016. 
Tandem of infinite-server queues with Markovian arrival process. 
\textit{Comm. Com. Inf. Sc.} 601:323--333. 
doi: 10.1007/978-3-319-30843-2\_34.

\bibitem{lit1} %10
\Aue{Lisovskaya, E., S.~Moiseeva, and M.~Pagano.}  2016. 
The total capacity of customers in the infinite-server queue with MMPP arrivals. 
\textit{Comm. Com. Inf. Sc.} 678:110--120. 
doi: 10.1007/978-3-319-51917-3\_11.
    


\bibitem{lit5} %11
\Aue{Moiseev, A., and A.~Nazarov.} 2016. Queueing network MAP/(GI/$\infty$)$^K$ 
with high-rate arrivals. \textit{Eur. J.~Oper. Res.} 254:161--168. 
doi: 10.1016/j.ejor.2016.04.011.





\bibitem{lit6} %12
\Aue{Moiseev, A.\,N., and M.\,V.~Sinyakov.} 2010. 
Razrabotka ob''ektno-orientirovannoy modeli sistemy imitatsionnogo modelirovaniya 
protsessov massovogo obsluzhivaniya 
[Design of object-oriented model for queueing simulation software]. 
\textit{Vestnik Tomskogo gosudarstvennogo universiteta. Upravlenie, 
vychislitel'naya tekhnika i informatika} 
[Tomsk State University. J.~Control Computer Sci.] 1:89--93.

\bibitem{lit3} %13
\Aue{Moiseev, A., A.~Demin, V.~Dorofeev, and V.~Sorokin.} 2016. 
Discrete-event approach to simulation of queueing networks. 
\textit{Key Eng. Mater.} 685:939--942. 
doi: 10.4028/www.scientific.net/KEM.685.939.

    
\end{thebibliography} }
 }

\end{multicols}

\vspace*{-6pt}

\hfill{\small\textit{Received March 16, 2017}}

\vspace*{-18pt}

\Contr

\noindent
\textbf{Lisovskaya Ekaterina Yu.} (b.\ 1992)~--- PhD student,
Department of Probability 
Theory and Mathematical Statistics, Tomsk State University, 36~Lenin Ave., 
Tomsk 634050, Russian Federation; 
\mbox{ekaterina\_lisovs@mail.ru}

\vspace*{3pt}

\noindent
\textbf{Moiseeva Svetlana P.} (b.\ 1971)~--- Doctor of Science in physics 
and mathematics, 
associate professor, professor, Department of Probability Theory
 and Mathematical Statistics, Tomsk State University, 36~Lenin ave., Tomsk 634050, 
 Russian Federation; \mbox{smoiseeva@mail.ru}

\vspace*{3pt}

\noindent
\textbf{Pagano Michele} (b.\ 1968)~--- PhD in electronics engineering,  
professor, Department of Information Engineering of University of 
Pisa, 16~Via Caruso, Pisa 56122, Italy; \mbox{m.pagano@iet.unipi.it}

\vspace*{3pt}

\noindent
\textbf{Potatueva Viktoriya V.} (b.\ 1993)~--- Master Degree student,
Department 
of Probability Theory and Mathematical Statistics, Tomsk State University, 
36~Lenin Ave., Tomsk 634050, Russian Federation; \mbox{ve-kusik@mail.ru}

\vspace*{8pt}

\hrule

\vspace*{2pt}

\hrule

%\newpage

%\vspace*{-24pt}



\def\tit{ИССЛЕДОВАНИЕ СИСТЕМЫ МАССОВОГО ОБСЛУЖИВАНИЯ MMPP/GI/$\infty$ 
С~ТРЕБОВАНИЯМИ СЛУЧАЙНОГО ОБЪЕМА$^*$}

\def\aut{Е.\,Ю.~Лисовская$^1$, С.\,П.~Моисеева$^2$, М.~Пагано$^3$, В.\,В.~Потатуева$^4$}


\def\titkol{Исследование системы массового обслуживания MMPP/GI/$\infty$ 
с~требованиями случайного объема}

\def\autkol{Е.\,Ю.~Лисовская, С.\,П.~Моисеева, М.~Пагано, В.\,В.~Потатуева}

{\renewcommand{\thefootnote}{\fnsymbol{footnote}}
\footnotetext[1]{Работа выполнена при частичной поддержке РФФИ (проект 16-31-00292).}}


\titel{\tit}{\aut}{\autkol}{\titkol}

\vspace*{-12pt}

\noindent
$^1$Национальный исследовательский Томский государственный университет,
\mbox{ekaterina\_lisovs@mail.ru}

\noindent
$^2$Национальный исследовательский Томский государственный университет,
\mbox{smoiseeva@mail.ru}

\noindent
$^3$Университет г.\ Пиза, Италия, \mbox{m.pagano@iet.unipi.it} 

\noindent
$^4$Национальный исследовательский Томский государственный университет,
\mbox{ve-kusik@mail.ru}

\vspace*{6pt}

\def\leftfootline{\small{\textbf{\thepage}
\hfill ИНФОРМАТИКА И ЕЁ ПРИМЕНЕНИЯ\ \ \ том\ 11\ \ \ выпуск\ 4\ \ \ 2017}
}%
 \def\rightfootline{\small{ИНФОРМАТИКА И ЕЁ ПРИМЕНЕНИЯ\ \ \ том\ 11\ \ \ выпуск\ 4\ \ \ 2017
\hfill \textbf{\thepage}}}


\Abst{Проведено исследование системы массового обслуживания с неограниченным 
числом приборов. Заявки поступают в систему в виде мар\-ков\-ски-мо\-ду\-ли\-ро\-ван\-но\-го 
пуассоновского потока. Каждая заявка несет в себе произвольное количество 
данных (объем заявки). В~этом исследовании время обслуживания не зависит от 
объема заявок. Показано, что совместное распределение вероятностей числа заявок в системе 
и~их суммарного объема является двумерным гауссовским при асимптотическом условии 
растущего времени обслуживания. Имитационное моделирование и численные эксперименты 
позволили определить область применимости асимптотического результата.}

\KW{бесконечнолинейная система массового обслуживания; случайный объем заявок; 
MMPP-поток}

\DOI{10.14357/19922264170414}

%\vspace*{18pt}


 \begin{multicols}{2}

\renewcommand{\bibname}{\protect\rmfamily Литература}
%\renewcommand{\bibname}{\large\protect\rm References}

{\small\frenchspacing
{%\baselineskip=10.8pt
\begin{thebibliography}{99}

\bibitem{mandjes-1} %1
\Au{Mandjes M.} Large deviations of Gaussian queues.~--- 
Chichester: Wiley, 2007. 340~p.

\bibitem{lit2-1}  %2
\Au{Melikov~A.,  Zadiranova~L., Moiseev~A.}  
Two asymptotic conditions in queue with MMPP arrivals and feedback~// 
Comm. Com. Inf. Sc., 2016. Vol.~678. P.~231--240.

\bibitem{lit10-1} %3
\Au{Naumov~V., Samouylov~K.,  Sopin~E.,  Andreev~S.} 
Two approaches to analyzing dynamic cellular networks with limited resources~//  
6th Congress (International)
on Ultra Modern Telecommunications and Control Systems and Workshops.~--- 
St.\ Petersburg, 2015. P.~485--488.
doi: 10.1007/978-3-319-44615-8\_23.

\bibitem{lit8-1} %4
\Au{Morozov  E.,  Potakhina~L.,  Tikhonenko~O.}  
Regenerative analysis of a~system with a~random volume of customers~// 
Comm. Com. Inf. Sc., 2016. Vol.~638. P.~261--272.
doi: 10.1007/978-3-319-44615-8\_23.





\bibitem{lit12-1} %5
\Au{Тихоненко О.\,М., Кемпа В\,.М.} 
Система с~разделением процессора и~ограниченным объемом памяти,
управ\-ля\-емая механизмом AQM~//
Автоматика и~телемеханика, 2015.
№\,10. С.~90--105.

\bibitem{lit9-1}  %6
\Au{Наумов В.\,А., Самуйлов~К.\,Е., Самуйлов~А.\,К.} 
О~суммарном объеме ресурсов, занимаемых обслуживаемыми заявками~// 
Автоматика и телемеханика, 2016. №\,8. C.~125--132.

\bibitem{lit13-1} %7 
\Au{Тихоненко О.\,М.} 
Система обслуживания с~разделением процессора и~ограниченными ресурсами~//
Queueing systems with processor sharing and limited resources~// 
Автоматика и~телемеханика, 2010. №\,5. С.~84--98.

\bibitem{lit11-1} %8
\Au{Pankratova  E.\,V., Moiseeva~S.\,P.} Queueing system 
MAP/M/$\infty$ with $n$ types of customers~// 
Comm. Com. Inf. Sc., 2014. Vol.~487. P.~356--366.

\bibitem{lit4-1} %9
\Au{Moiseev  A., Nazarov~A.} 
Tandem of infinite-server queues with Markovian arrival process~// 
Comm. Com. Inf. Sc., 2016. Vol.~601. P.~323--333.
doi: 10.1007/978-3-319-30843-2\_34.

\bibitem{lit1-1} %10
\Au{Lisovskaya~E., Moiseeva~S., Pagano~M.}  
The total capacity of customers in the infinite-server queue with MMPP arrivals~// 
Comm. Com. Inf. Sc., 2016. Vol.~678. P.~110--120.
doi: 10.1007/978-3-319-51917-3\_11.
    


\bibitem{lit5-1} %11
\Au{Moiseev~A., Nazarov~A.} 
Queueing network MAP/(GI/$\infty$)$^K$ with high-rate arrivals~// Eur. 
J.~Oper. Res., 2016. Vol.~254. P.~161--168.
doi: 10.1016/ j.ejor.2016.04.011.





\bibitem{lit6-1}  %12
\Au{Моисеев А.\,Н., Синяков~М.\,В.} Разработка 
объ\-ект\-но-ори\-ен\-ти\-ро\-ван\-ной модели системы имитационного\linebreak 
моделирования процессов массового обслуживания~// 
Вестник Томского государственного университета. 
Управление, вычислительная техника и информатика, 2010. №\,1. С.~89--93.

\bibitem{lit3-1}  %13
\Au{Moiseev~A., Demin~A., Dorofeev~V., Sorokin~V.} 
Discrete-event approach to simulation of queueing networks~// 
Key Eng. Mater., 2016. Vol.~685. P.~939--942.
doi: 10.4028/www.scientific.net/KEM.685.939.

\end{thebibliography}
} }

\end{multicols}

 \label{end\stat}

 \vspace*{-3pt}

\hfill{\small\textit{Поступила в~редакцию  16.03.2017}}
%\renewcommand{\bibname}{\protect\rm Литература}
\renewcommand{\figurename}{\protect\bf Рис.}
\renewcommand{\tablename}{\protect\bf Таблица}    %14
\def\stat{inkova}

\def\tit{КРИТЕРИИ ОПРЕДЕЛЕНИЯ СЕМАНТИЧЕСКОЙ БЛИЗОСТИ ДИСКУРСИВНЫХ 
ОТНОШЕНИЙ$^*$}

\def\titkol{Критерии определения семантической близости дискурсивных 
отношений}

\def\aut{О.\,Ю.~Инькова$^1$, М.\,Г.~Кружков$^2$}

\def\autkol{О.\,Ю.~Инькова, М.\,Г.~Кружков}

\titel{\tit}{\aut}{\autkol}{\titkol}

\index{Инькова О.\,Ю.}
\index{Кружков М.\,Г.}
\index{Inkova O.\,Yu.}
\index{Kruzhkov M.\,G.}


{\renewcommand{\thefootnote}{\fnsymbol{footnote}} \footnotetext[1]
{Работа выполнялась с~использованием инфраструктуры Центра коллективного пользования <<Высокопроизводительные вы\-чис\-ле\-ния и~большие данные>> 
(ЦКП <<Информатика>>) ФИЦ ИУ РАН (г.~Москва).}}


\renewcommand{\thefootnote}{\arabic{footnote}}
\footnotetext[1]{Федеральный исследовательский центр <<Информатика и~управ\-ле\-ние>> Российской академии наук; Женевский 
университет, \mbox{olyainkova@yandex.ru}}
\footnotetext[2]{Федеральный исследовательский центр <<Информатика и~управ\-ле\-ние>> Российской академии наук, 
\mbox{magnit75@yandex.ru}}


\vspace*{-12pt}

  \Abst{Работа посвящена результатам разработки структурированных определений 
дискурсивных отношений на осно\-ве их классификации, а~так\-же критериям, поз\-во\-ля\-ющим 
определить их семантическую бли\-зость. Авторы показывают недостатки су\-щест\-ву\-ющих 
подходов, которые приводят к~противоречивым или час\-то необоснованным результатам, 
а~так\-же раскрывают преимущества альтернативного решения: классификации дискурсивных 
отношений на осно\-ве их структурированных определений. Приводятся примеры таких 
определений, сформированных в~Надкорпусной базе данных коннекторов (НБДК), а~так\-же 
критерии, поз\-во\-ля\-ющие определить семантическую бли\-зость дискурсивных отношений. 
Поскольку структурированные определения пред\-став\-ля\-ют собой набор различительных 
признаков, авторы об\-суж\-да\-ют проб\-ле\-му при\-сво\-ения коэффициента бли\-зости каж\-до\-му из 
признаков. Полученные данные, в~том чис\-ле количественные, поз\-во\-ля\-ют вы\-дви\-нуть 
гипотезу, со\-глас\-но которой из трех групп признаков: <<Уровень>>, <<Базовая 
операция>> и~<<Семейство признаков>>~--- наибольший вес имеет по\-след\-няя. 
Предлагаются пути дальнейшего исследования этой проб\-ле\-мы, в~част\-ности с~учетом таких 
факторов, как данные по со\-че\-та\-е\-мости дискурсивных отношений, по соответствиям 
дискурсивных отношений и~их показателей в~текс\-те оригинала и~в~текс\-те перевода, а~так\-же 
тех случаев, когда один показатель может выражать несколько дискурсивных отношений.} 
  
  \KW{надкорпусная база данных; ло\-ги\-ко-се\-ман\-ти\-че\-ские отношения; коннекторы; 
аннотирование; фасетная классификация}

\DOI{10.14357/19922264230314}{UJZJZI}
  
%\vspace*{-4pt}


\vskip 10pt plus 9pt minus 6pt

\thispagestyle{headings}

\begin{multicols}{2}

\label{st\stat}

\section{Определения дискурсивных отношений и~вопросы их~семантической близости}


  Вопросы, связанные с~изучением дискурсивных\linebreak отношений и~их показателей, 
коннекторов, не те\-ря\-ют своей ак\-ту\-аль\-ности в~силу того, что эти от\-но\-ше\-ния 
играют значительную роль в~обеспечении связ\-ности текс\-та~[1--3], а~список их 
показателей постоянно расширяется (ср.\ для русского языка, например,~[4, 
с.~663--686; 5] и~др.), что за\-труд\-ня\-ет автоматическую обработку текс\-та. Кроме 
того, в~су\-щест\-ву\-ющих подходах дискурсивные отношения определены, как 
правило, довольно противоречиво, по\-сколь\-ку нет чет\-ких критериев для их 
выделения (см., например, наиболее известную в~данной об\-ласти тео\-рию 
риторических отношений~[6], а~так\-же~[7--9]). 

В~некоторых 
классификациях~[10, 11] отношения оформлены в~виде иерархической 
структуры, однако критерии объ\-еди\-не\-ния отношения в~группы и,~в~част\-ности, 
критерий выделения того или иного чис\-ла выс\-ших иерархических уров\-ней, 
отсутствуют. Так, в~классификации, ис\-поль\-зу\-емой в~Пенсильванском 
аннотированном корпусе (Penn Discourse Treebank, RDTB)~[10], выс\-ших клас\-си\-фи\-ци\-ру\-ющих уров\-ней четыре: 
Temporal, Contingency, Comparison, Expansion. Если основания для выделения 
уров\-ней Temporal и~Contingency интуитивно ясны, то основания для 
объединения отношений в~классы Comparison и~Expansion не\-оче\-вид\-ны. 

Вызывает, например, воз\-ра\-же\-ние отнесение к~группе Comparison 
уступительных отношений (\textit{Хотя на улице дождь, Петя не захотел 
взять зонтик}), в~основе интерпретации которых лежит не сравнение, 
а~импликация (Если на улице идет дождь, то мы, как правило, берем зонтик), 
а~значит, им мес\-то в~классе Contingency вмес\-те с~причинными и~условными 
отношениями (подробнее см.~[12]). 

В~результате в~одном классе оказываются 
семантически далекие отношения: например, в~классе Expansion оказывается 
отношение альтернативы и~спецификации; ср.\ \textit{Она учится 
в~университете или работает?} с~отношением альтернативы и~\textit{Подари 
ей четвероного друга, например хомяка} с~отношением спецификации. 

И~наоборот: семантически близ\-кие отношения оказываются в~раз\-ных группах, 
как в~случае усту\-питель\-ных и~услов\-ных отношений. Отсутствие критериев для 
выделения иерархических уровней и~отнесения к~ним того или иного набора 
ло\-ги\-ко-се\-ман\-ти\-че\-ских отношений (ЛСО) приводит так\-же к~тому, что 
классификация, ис\-поль\-зу\-емая в~PDTB, 
претерпевает по\-сто\-ян\-ные изменения; ср.\ две ее версии, пред\-став\-лен\-ные в~[10] 
и~[13].
  
  Основные проблемы классификации дискурсивных отношений описаны 
в~предыду\-щей работе авторов~[14], где пред\-став\-ле\-ны так\-же пер\-вые результаты 
ее альтернативного решения, в~част\-ности классификация, ис\-поль\-зу\-емая 
в~%Надкорпусной базе данных коннекторов (
НБДК, разработанной в~ИПИ ФИЦ 
ИУ РАН\footnote{Подробнее о структуре НБД, ее возможностях и~результатах, 
полученных с~ее использованием, см., например, [15--17]. Пред\-ста\-ви\-тель\-ный фрагмент НБД 
до\-сту\-пен по адресу: {\sf http://a179.frccsc.ru/RFH41002/main.aspx}.}, и~созданные на ее основе 
структурированные определения дис\-кур\-сив\-ных отношений, или, в~терминологии авторов, 
ЛСО. Коротко пе\-ре\-чис\-лим основные положения данного под\-хода.
  
  В основе классификации лежат четыре базовые семантические операции, на 
которые опирается то или иное ЛСО: импликация; расположение на шкале 
времени; срав\-не\-ние; соотнесение част\-но\-го и~общего или элемента и~множества. 
Классификация различает так\-же уровни, на которых может быть уста\-нов\-ле\-но 
ЛСО: пропозициональный уровень, уровень вы\-ска\-зы\-ва\-ния (иллокутивный), 
метаязыковой (по\-дроб\-нее см.~[12]). Каждое ЛСО определяется, следовательно, 
на основе этих двух критериев, к~которым до\-бав\-ля\-ет\-ся критерий, 
характеризующий ЛСО, основанные на импликации и~сравнении: по\-ляр\-ность, 
т.\,е.\ уста\-нав\-ли\-ва\-ет\-ся ли ЛСО непосредственно между положениями 
вещей~$p$ и~$q$, описанными в~свя\-зы\-ва\-емых им фрагментах текс\-та, или же 
при его интерпретации долж\-ны быть учтены также их отрицательные 
корреляты $\neg p$ и~$\neg q$. 
Учитываются и~семантические, 
и~праг\-ма\-ти\-че\-ские характеристики кон\-текста.
  
\section{Структурированные определения логико-семантических отношений}

  Разработанная концепция классификации дает воз\-мож\-ность описывать ЛСО 
при помощи структурированных определений, пред\-став\-ля\-ющих собой набор 
раз\-ли\-чи\-тель\-ных признаков. На момент на\-писания \mbox{статьи} в~НБДК 
описаны~26~ЛСО. Определения ЛСО пропозициональной альтернативы 
и~спецификации приведены в~табл.~1 (другие определения см.\ в~[14, 18] 
и~ниже).
  
  
  Структурированные определения ЛСО пропозициональной альтернативы 
и~спецификации\linebreak показывают, что они имеют лишь один общий 
различительный признак: оба они уста\-нов\-ле\-ны на пропозициональном уров\-не. 
Этого недостаточно для того, чтобы отнести их к~единому иерархическому 
уров\-ню, как предлагается в~PDTB (см.\ обоснование в~разд.~3). Напротив, 
у~уступительных и~условных ЛСО, ока\-зы\-ва\-ющих\-ся в~PDTB на раз\-ных уров\-нях, 
общих признаков больше (табл.~2). В~част\-ности, общим для них является помимо 
пропозиционального уров\-ня базовая операция.
  
  
 В этой связи следует заметить, что если некоторые различительные при\-зна\-ки, 
например <<пропозициональный уровень>> или <<операция импликации>>, 
характеризуют несколько ЛСО, то другие\linebreak высту\-па\-ют уникальными свойствами 
того или иного ЛСО, поз\-во\-ля\-ющи\-ми его идентифицировать и~отличать от 
других. К~ним относится, например, при\-знак <<$Y$ содержит более част\-ное 
понятие~$q$,\linebreak су\-жа\-ющее экстенсионал~$p$>>\footnote{Строчные буквы $p$ и~$q$ 
обозначают положение дел, прописные~$P$ и~$Q$~--- акты высказывания, прописные~$X$ 
и~$Y$~--- фрагменты текста.}, ха\-рак\-те\-ри\-зу\-ющий ЛСО 
спецификации. В~табл.~3 
приведены данные по использованию различительных при\-зна\-ков для ЛСО, 
получивших определения в~НБД. 

\end{multicols}


\begin{table*}[h]\small %tabl1
\vspace*{-15pt}

  \begin{center}
  \Caption{Примеры структурированных определений ЛСО}
  \vspace*{2ex}
  
\begin{tabular}{|l|l|l|l|}
\hline
\multicolumn{1}{|c|}{ЛСО}&
\multicolumn{1}{c|}{\tabcolsep=0pt\begin{tabular}{c}Базовая\\ семантическая операция\end{tabular}}&
\multicolumn{1}{c|}{Уровень}&
\multicolumn{1}{c|}{\tabcolsep=0pt\begin{tabular}{c}Дополнительные\\ характеристики\end{tabular}}\\
\hline
\tabcolsep=0pt\begin{tabular}{l}Пропозицио-\\ нальная\\ альтернатива\end{tabular}&
\tabcolsep=0pt\begin{tabular}{l}$\bullet$~Операция сравнения, устанав-\\ \hphantom{$\bullet$~}ливающая сходство~$p$ и~$q$\end{tabular}&
\tabcolsep=0pt\begin{tabular}{l}$\bullet$~Пропозицио-\\ \hphantom{$\bullet$~}нальный\end{tabular}&
\tabcolsep=0pt\begin{tabular}{l}$\bullet$~$p$ и~$q$~---  положения вещей, имеющие\\ \hphantom{$\bullet$~}статус гипотезы;\\
$\bullet$~говорящий предлагает сделать выбор\\ \hphantom{$\bullet$~}между~$p$ и~$q$\end{tabular}\\
\hline
Спецификация&
\tabcolsep=0pt\begin{tabular}{l}$\bullet$~Операция соотнесения общего\\ \hphantom{$\bullet$~}и~частного\end{tabular}&
\tabcolsep=0pt\begin{tabular}{l}$\bullet$~Пропозицио-\\ \hphantom{$\bullet$~}нальный\end{tabular} &
\tabcolsep=0pt\begin{tabular}{l}$\bullet$~$X$ содержит обобщенное понятие или\\ \hphantom{$\bullet$~}положение вещей~$p$;\\
$\bullet$~$Y$ содержит более частное понятие~$q$,\\ \hphantom{$\bullet$~}сужающее экстенсионал~$p$\end{tabular}\\
\hline
\end{tabular}
\end{center}
\vspace*{-6pt}
\end{table*}

 \begin{table*}\small %tabl2
\begin{center}
\Caption{Структурированные определения уступительных и~условных ЛСО}
\vspace*{2ex}

\begin{tabular}{|l|l|l|l|}
\hline
\multicolumn{1}{|c|}{ЛСО}&
\multicolumn{1}{c|}{\tabcolsep=0pt\begin{tabular}{c}Базовая\\ семантическая операция\end{tabular}}&
\multicolumn{1}{c|}{Уровень}&
\multicolumn{1}{c|}{\tabcolsep=0pt\begin{tabular}{c}Дополнительные\\ характеристики\end{tabular}}\\
\hline
Уступительные&
\tabcolsep=0pt\begin{tabular}{l}$\bullet$~Операция импликации\end{tabular}&
\tabcolsep=0pt\begin{tabular}{l}$\bullet$~Пропозицио-\\ \hphantom{$\bullet$~}нальный\end{tabular}&
\tabcolsep=0pt\begin{tabular}{l}$\bullet$~$p$ и~$q$~--- положения вещей;\\
$\bullet$~как правило, если имеет место~$q$, то не имеет\\ \hphantom{$\bullet$~}места~$p$\end{tabular}\\
\hline
Условные&
\tabcolsep=0pt\begin{tabular}{l}$\bullet$~Операция импликации\end{tabular} &
\tabcolsep=0pt\begin{tabular}{l}$\bullet$~Пропозицио-\\ \hphantom{$\bullet$~}нальный\end{tabular}&
\tabcolsep=0pt\begin{tabular}{l}$\bullet$~$p$ и~$q$~--- 
положения вещей, имеющие статус\\ \hphantom{$\bullet$~}гипотезы или неосуществившегося положе-\\ \hphantom{$\bullet$~}ния вещей;\\
$\bullet$~если имеет место~$p$, то имеет место~$q$\end{tabular}\\
\hline
\end{tabular}
\end{center}
\vspace*{-6pt}
\end{table*}
  
\begin{table*}\small %tabl3
\begin{center}
\Caption{Использование признаков в~описаниях ЛСО}
\vspace*{2ex}

%\tabcolsep=3pt
\begin{tabular}{|l|c|}
\hline
\multicolumn{1}{|c|}{Признак ЛСО}&\tabcolsep=0pt\begin{tabular}{c}Количество\\ ЛСО\end{tabular}\\
\hline
Пропозициональный уровень&14\hphantom{9}\\
\hline
Операция сравнения, устанавливающая несходство~$p$ и~$q$&12\hphantom{9}\\
\hline
Метаязыковой уровень&6\\
\hline
Уровень высказывания&6\\
\hline
$p$ отвергается, принимается~$q$&5\\
\hline
Операция сравнения, устанавливающая сходство между $p$ и~$q$&5\\
\hline
Говорящий предлагает сделать выбор между $p$ и~$q$&5\\
\hline
\multicolumn{1}{|c|}{$\cdots$}&$\cdots$\\
\hline
Как правило, если имеет место $q$, то не имеет места~$p$&1\\
\hline
$p$ и~$q$~--- положения вещей, не связанные никаким другим ЛСО&1\\
\hline
$p$ верно, только если исключить осуществление~$q$&1\\
\hline
Положение вещей $q$ служит аргументом в~пользу ожидаемого вывода не-$r$&1\\
\hline
Положение вещей р служит аргументом в~пользу ожидаемого вывода~$r$&1\\
\hline
$p$ и~$q$ имеют одинаковый экстенсионал&1\\
\hline
$q$~--- возможное описание того же положения вещей~$r$&1\\
\hline
\tabcolsep=0pt\begin{tabular}{l}$q$~--- обобщенное (<<без частностей>>) представление положения вещей,\\ сделанное на 
основании свойств~$p$\end{tabular}&1\\
\hline
\multicolumn{1}{|c|}{$\cdots$}&$\cdots$\\
\hline
\end{tabular}
\end{center}
\vspace*{-6pt}
\end{table*}




\begin{multicols}{2}
 
  Из 52 использованных в~НБДК различительных при\-зна\-ков~19~характеризуют 
более одного ЛСО, а~33~уникальны. Тем не менее эти уникальные при\-зна\-ки 
могут быть объединены в~семейства, фик\-си\-ру\-ющие концептуальную бли\-зость 
при\-зна\-ков, не\-смот\-ря на их формальные раз\-ли\-чия. Например, семейст\-во 
при\-зна\-ков <<используется отрицательный коррелят $p/P$>> включает в~себя 
сле\-ду\-ющий набор признаков: 
\begin{enumerate}[(1)]
\item %1)~
$P$ отвергается, принимается~$q$; 
\item %2)~
$q$ имеет мес\-то, если не имеет мес\-та~$p$; 
\item %3)~
$q$ имеет мес\-то, если не имеет мес\-та положение вещей, описанное в~$P$; 
\item %4)~
$p$ отвергается, принимается~$q$; 
\item %5)~
как правило, если имеет мес\-то~$q$, то не имеет мес\-та~$p$.
\end{enumerate}
  
\begin{table*}\small %tabl4
\vspace*{-3pt}
\begin{center}
\Caption{Структурированные определения ЛСО од\-но\-вре\-мен\-ности, со\-пут\-ст\-во\-ва\-ния и~со\-по\-став\-ления}
\vspace*{2ex}

\tabcolsep=4.3pt
\begin{tabular}{|l|l|l|l|}
\hline
\multicolumn{1}{|c|}{ЛСО}&
\multicolumn{1}{c|}{\tabcolsep=0pt\begin{tabular}{c}Базовая\\ семантическая операция\end{tabular}}&
\multicolumn{1}{c|}{Уровень}&
\multicolumn{1}{c|}{\tabcolsep=0pt\begin{tabular}{c}Дополнительные\\ характеристики\end{tabular}}\\
\hline
Одновременность&
\tabcolsep=0pt\begin{tabular}{l}$\bullet$~Расположение на оси времени\end{tabular} &
\tabcolsep=0pt\begin{tabular}{l}$\bullet$~Пропозицио-\\ \hphantom{$\bullet$~}нальный\end{tabular}&
\tabcolsep=0pt\begin{tabular}{l}$\bullet$~$p$ и~$q$~--- положения вещей;\\
$\bullet$~$Tp$ включает в~себя~$Tq$\end{tabular}\\
\hline
Сопутствование&
\tabcolsep=0pt\begin{tabular}{l}$\bullet$~Операция сравнения, устанав-\\ \hphantom{$\bullet$~}ливающая сходство между~$p$ и~$q$\end{tabular} &
\tabcolsep=0pt\begin{tabular}{l}$\bullet$~Пропозицио-\\ \hphantom{$\bullet$~}нальный\end{tabular}&
\tabcolsep=0pt\begin{tabular}{l}$\bullet$~$q$~--- положение вещей, зависимое от~$p$;\\
$\bullet$~$Tp$ включает в~себя $Tq$\end{tabular}\\
\hline
Сопоставление&
\tabcolsep=0pt\begin{tabular}{l}$\bullet$~Операция сравнения, устанав-\\ \hphantom{$\bullet$~}ливающая несходство~$p$ и~$q$\end{tabular}&
\tabcolsep=0pt\begin{tabular}{l}$\bullet$~Пропозицио-\\ \hphantom{$\bullet$~}нальный\end{tabular}&
\tabcolsep=0pt\begin{tabular}{l}$\bullet$~$p$ и~$q$ актуальны для говорящего\\ \hphantom{$\bullet$~}в~момент речи~$Td$;\\
$\bullet$~сходство~$p$ и~$q$ относительно некоторого\\ \hphantom{$\bullet$~}<<общего знаменателя>>\end{tabular}\\
\hline
\end{tabular}
\end{center}
\vspace*{-9pt}
\end{table*}

\begin{table*}[b]\small %tabl5
\vspace*{-12pt}
\begin{center}
\Caption{Распределение аннотаций с~ЛСО альтернативы и~коррекции по уровням}
\vspace*{2ex}

\tabcolsep=10pt
\begin{tabular}{|l|c|c|c|}  
\hline
&\multicolumn{3}{c|}{Уровень}\\
\cline{2-4}
\multicolumn{1}{|c|}{\raisebox{6pt}[0pt][0pt]{ЛСО}}&Пропозициональный&Иллокутивный&Метаязыковой\\
\hline
%&&\multicolumn{1}{p{80pt}|}{\hphantom{80pt}}&\multicolumn{1}{p{80pt}|}{\hphantom{80pt}}\\[-12pt]
Альтернатива&2138&15&144\\
Коррекция&\hphantom{9}231&36&\hphantom{9}36\\
\hline
\end{tabular}
\end{center}
\vspace*{-6pt}
\end{table*}

\section{Критерии определения степени семантической близости 
логико-семантических отношений }

  Структурированные определения ЛСО поз\-во\-ля\-ют создать их 
непротиворечивую классификацию, обосновав чис\-ло высших иерархических 
уров\-ней и~объединение ЛСО в~семантические классы общ\-ностью лежащей в~их 
основе семантической операции. Общ\-ность этого признака не всегда, однако, 
служит необходимым и~достаточным условием для того, чтобы считать ЛСО 
семантически близ\-ки\-ми. Было замечено, что, с~одной стороны, показателю ЛСО 
в~одном языке может соответствовать в~переводе показатель ЛСО, 
относящийся к~другому иерархическому уров\-ню, причем как при наличии сис\-тем\-но\-го 
переводного эквивалента у~данного показателя~(a), так и~при его 
отсутствии~(б), и~такие случаи не единичны. Например, в~высказывании~(а) 
итальянский коннектор \textit{intanto} `меж\-ду тем' переведен русским 
со\-по\-ста\-ви\-тель\-ным союзом~\textit{а}:\\[-15pt] 
  \begin{itemize}
  \item[(a)] Si volse, e prese ad arrampicarsi di traverso lungo la proda assolata 
dell'argine. Si aiutava con la mano destra, afferrandosi ai ciuffi dell'erba; 
\textit{intanto}, la sinistra levata all'altezza del capo, veniva togliendosi 
e~rimettendosi il cerchietto ferma-capelli.~--- Она повернулась и~стала 
карабкаться по залитому солнцем спус\-ку, хватаясь правой рукой за траву, 
\textit{а}~левой, поднятой над головой, по\-прав\-ля\-ла обруч на волосах. [Giorgio 
Bassani. Il giardino dei Finzi-Contini (1962), перевод И.~Соболева (2008)]. 
  \end{itemize}

  В~(б) тот же итальянский коннектор переводит русский показатель ЛСО 
со\-пут\-ст\-во\-ва\-ния:\\[-15pt] 
  \begin{itemize}
  \item[(б)] Он говорил громко \textit{и при этом} делал такие удив\-ленные 
глаза, что мож\-но было подумать, что он лгал.~--- Egli parlava ad alta voce 
\textit{e~intanto} faceva degli occhi cos$\grave{\mbox{\!\ptb{\i}}}$ meravigliati che 
si pensava ch'egli mentisse. [А.\,П.~Чехов. Палата №\,6 (1892), перевод 
F.~Malcovati]. 
  \end{itemize}
  
  С другой стороны, известно, что показатели ЛСО многозначны: на это 
указывают, в~част\-ности, словари и~грамматики. Так, для русского коннектора 
\textit{между тем}, вы\-сту\-па\-юще\-го сис\-тем\-ным эквивалентом итальянского 
\textit{intanto}, словарь~[19] дает три значения: од\-но\-вре\-мен\-ность, 
со\-по\-став\-ле\-ние и~про\-ти\-ви\-тель\-ность (ЛСО <<вопреки ожи\-да\-емо\-му>>).
  
  Если, однако, сравнить определения ЛСО со\-пут\-ст\-во\-ва\-ния, со\-по\-став\-ле\-ния 
  и~од\-но\-вре\-мен\-ности, то увидим, что они, не\-смот\-ря на то что в~их основе лежат 
разные семантические операции, имеют общие различительные признаки (табл.~4). 

  
  Общим для трех отношений является при\-знак <<пропозициональный 
уровень>>, для ЛСО од\-но\-вре\-мен\-ности и~со\-пут\-ст\-во\-ва\-ния~--- при\-знак <<$Tp$ 
включает в~себя $Tq$>>, который входит в~то же семейство признаков 
(<<единство временн$\acute{\mbox{o}}$го интервала>>), что и~при\-знак <<$p$ и~$q$ актуальны для 
говорящего в~момент речи $Td$>>, ха\-рак\-те\-ри\-зу\-ющий ЛСО со\-по\-став\-ле\-ния. 
Таким образом, с~одной стороны, в~качестве переводного эквивалента 
показателя некоторого ЛСО выбираются показатели, раз\-де\-ля\-ющие с~ним 
различительные признаки, а~с~другой~--- набор значений, вы\-ра\-жа\-емых 
коннектором, та\-кже не является произвольным, а~определяется семантической 
бли\-зостью ЛСО.
  
  Однако при определении семантической близости ЛСО не все 
различительные при\-зна\-ки имеют одинаковый вес, или <<коэффициент  
бли\-зости>>. Так, признак <<пропозициональный уровень>> не может иметь 
высокий коэффициент, поскольку не является дис\-кри\-ми\-ни\-ру\-ющим. Он входит 
в~группу при\-зна\-ков (<<уровень, на котором установлено ЛСО>>), 
вклю\-ча\-ющую всего три элемента, а~значит, он становится общим для большого 
чис\-ла ЛСО, которые в~по\-дав\-ля\-ющем большинстве уста\-нав\-ли\-ва\-ют\-ся именно 
между пропозициями. Кроме того, ЛСО, если оно может уста\-нав\-ли\-вать\-ся на 
всех трех уров\-нях, в~большинстве случаев уста\-нав\-ли\-ва\-ет\-ся именно на уров\-не 
пропозиций. В~табл.~5 приводятся данные для ЛСО альтернативы и~коррекции, 
для которых в~НБДК сделана сплош\-ная вы\-борка.

  
  Признаки из группы <<Базовая семантическая операция>> так\-же, по 
име\-ющим\-ся данным, не имеют ре\-ша\-юще\-го значения для определения 
семантической бли\-зости ЛСО, хотя их коэффициенты, по-ви\-ди\-мо\-му, 
долж\-ны быть выше, чем у~признаков группы <<Уровень>>, поскольку они 
являются классифицирующими и~служат для обосно\-ва\-ния того, почему ЛСО 
объединяются в~семантические \mbox{классы}. 
{\looseness=-1

}
  
  Проанализированные данные поз\-во\-ля\-ют предположить, что на\-и\-выс\-ший 
коэффициент дол\-жен быть присвоен тем случаям, когда различительные 
при\-зна\-ки ЛСО относятся к~одному семейству признаков. Это хорошо вид\-но на 
примере ЛСО од\-но\-вре\-мен\-ности, со\-по\-став\-ле\-ния и~со\-пут\-ст\-во\-ва\-ния (см.\ табл.~4), 
а~так\-же на примере других ЛСО, получивших определения в~НБДК. 
Дальнейшее исследование поз\-во\-лит присвоить чис\-ло\-вые па\-ра\-мет\-ры 
коэффициентам бли\-зости для различительных при\-знаков.
{\looseness=-1

}

\vspace*{-6pt}

\section{Заключение}

  Разработанная классификация ЛСО и~сформированные на ее основе в~НБДК 
структурированные определения поз\-во\-ля\-ют не только избежать противоречий 
или необоснованных решений в~классификации ЛСО, но и~определить степень 
бли\-зости ЛСО с~учетом общ\-ности их раз\-ли\-чи\-тель\-ных при\-зна\-ков. 
В~дальнейшем при определении коэффициентов бли\-зости, при\-сва\-и\-ва\-емых 
при\-зна\-кам, мож\-но учитывать данные по со\-че\-та\-е\-мости ЛСО~[20], по 
соответствиям ЛСО в~оригинальном и~переводном текс\-те, а~так\-же те случаи, 
когда раз\-ные ЛСО выражаются одним и~тем же показателем. Полученные 
результаты поз\-во\-лят улучшить автоматическую обработку текс\-та, а~так\-же 
качество машинного пе\-ре\-вода.

\vspace*{-6pt}
  
{\small\frenchspacing
 { %\baselineskip=12pt
 %\addcontentsline{toc}{section}{References}
 \begin{thebibliography}{99}

\bibitem{1-kr}
\Au{Hobbs J.\,R.} A~computational approach to discourse analysis.~--- 
New York, NY, USA: Department of Computer Science, City College, City University of New 
York, 1976. Research Report~76-2. P.~28--38.
\bibitem{2-kr}
\Au{Hobbs J.\,R.} Why is discourse coherent?~--- Menlo Park, CA, 
USA: SRI International, 1978.  SRI Technical Note~176. 44~p.
\bibitem{3-kr}
\Au{Hobbs J.\,R.} Coherence and coreference~// Cognitive Sci., 1979. Vol.~3. No.\,1.  
P.~67--90. 
\bibitem{4-kr}
Русская грамматика.~/ Под ред. Н.\,Ю.~Шведовой.~--- М.: Наука, 1980. Т.~2. 714~с.
\bibitem{5-kr}
Лексикографические порт\-ре\-ты служебных слов~/ Под ред. Е.\,А.~Ста\-ро\-ду\-мо\-вой, 
Е.\,С.~Ше\-ре\-меть\-евой, В.\,Н.~Завья\-ло\-ва.~--- Владивосток: ДВФУ, 2022. 322~с.
\bibitem{6-kr}
\Au{Mann W.\,C., Thompson S.\,A.} Rhetorical structure theory: Towards a functional theory of text 
organization~// Text, 1988. Vol.~8. No.\,3. P.~243--281. doi: 10.1515/text.1. 1988.8.3.243.
\bibitem{7-kr}
\Au{Knott A., Dale~R.} Using linguistic phenomena to motivate a~set of coherence relations~// 
Discourse Process., 1994. Vol.~18. No.\,1. P.~35--62. doi: 10.1080/ 01638539409544883.
\bibitem{8-kr}
\Au{Rudolph E.} Contrast: Adversative and concessive expressions on sentence and text  
level.~--- Berlin/Boston: Walter de Gruyter, 1996. 564~p.
\bibitem{9-kr}
\Au{Fraser B.} An account of discourse markers~// International Review Pragmatics, 2009. Vol.~1. 
No.\,2. P.~293--320. doi: 10.1163/187730909X12538045489818.
\bibitem{10-kr}
PDTB Research Group. The Penn Discourse Treebank~2.0 annotation manual.~--- Philadelphia, PA, USA: 
Institute for Research in Cognitive Science, University of Pennsylvania, 2008.  Technical Report 
IRCS-08-01. 99~p. {\sf https://www.cis.upenn.edu/$\sim$elenimi/pdtb-manual.pdf}.
\bibitem{11-kr}
\Au{Breindl E., Volodina~A., \mbox{Wa\!{\ptb{\!\ss}}\,ner}~U.\,H.} Handbuch der deutschen Konnektoren~2. 
Semantik der deutschen Satzverkn$\ddot{\mbox{u}}$pfer.~--- Berlin: Walter de Gruyter, 2014. 1327~p.
\bibitem{12-kr}
\Au{Инькова О.\,Ю.} Ло\-ги\-ко-се\-ман\-ти\-че\-ские отношения: проблемы 
классификации~// Связ\-ность текста: мереологические ло\-ги\-ко-се\-ман\-ти\-че\-ские 
отношения.~--- М.: ЯСК, 2019. С.~11--98.
\bibitem{13-kr}
\Au{Webber B., Prasad~R., Lee~A., Joshi~A.} The Penn Discourse Treebank~3.0 annotation 
manual, 2019. 81~p. {\sf  
https:// catalog.ldc.upenn.edu/docs/LDC2019T05/ PDTB3-Annotation-Manual.pdf}.
\bibitem{14-kr}
\Au{Инькова О.\,Ю., Кружков М.\,Г.} Структурированные определения дискурсивных 
отношений в~Надкорпусной базе данных коннекторов~// Информатика и~её применения, 
2021. Т.~15. Вып.~4. С.~27--32. doi: 10.14357/19922264210404.
\bibitem{15-kr}
Семантика коннекторов: контрастивное исследование~/ Под. ред. О.\,Ю.~Иньковой.~--- 
М.: ТОРУС ПРЕСС, 2018. 368~с.
\bibitem{16-kr}
Структура коннекторов и~методы ее описания~/ Под. ред. О.\,Ю.~Иньковой.~--- М.: 
ТОРУС ПРЕСС, 2019. 316~с.
\bibitem{17-kr}
\Au{Кружков М.\,Г.} Концепция по\-стро\-ения надкорпусных баз данных~// Сис\-те\-мы 
и~средства информатики, 2021. T.~31. №\,3. С.~101--112. doi: 
10.14357/ 08696527210309.
\bibitem{18-kr}
\Au{Инькова О.\,Ю.} Определения дискурсивных отно\-шений: опыт Надкорпусной базы 
данных коннекторов~// Компьютерная линг\-ви\-сти\-ка и~интеллектуальные технологии: По 
мат-лам ежегодной\linebreak Междунар. конф. <<Диалог>>.~--- М.: РГГУ, 2021. Вып.~20(27). 
С.~328--338.
\bibitem{19-kr}
Словарь структурных слов русского языка~/ Под ред. В.\,В.~Морковкина.~--- М.: Лазурь, 
1997. 422~с.
\bibitem{20-kr}
\Au{Инькова О.\,Ю., Кружков~М.\,Г.} Со\-че\-та\-емость ло\-ги\-ко-се\-ман\-ти\-че\-ских 
отношений: количественные методы анализа~// Информатика и~её применения, 2019. Т.~13. 
Вып.~2. С.~83--91. doi: 10.14357/19922264190212.

\end{thebibliography}

 }
 }

\end{multicols}

\vspace*{-8pt}

\hfill{\small\textit{Поступила в~редакцию 10.07.23}}

%\vspace*{8pt}

%\pagebreak

\newpage

\vspace*{-28pt}



\def\tit{EVALUATION CRITERIA FOR~DISCOURSE~RELATIONS~SEMANTIC~AFFINITY}


\def\titkol{Evaluation criteria for~discourse relations semantic 
affinity}


\def\aut{O.\,Yu.~Inkova$^{1,2}$ and~M.\,G.~Kruzhkov$^1$}

\def\autkol{O.\,Yu.~Inkova and~M.\,G.~Kruzhkov}

\titel{\tit}{\aut}{\autkol}{\titkol}

\vspace*{-10pt}


\noindent
$^1$Federal Research Center ``Computer Science and Control'' of the Russian 
Academy of Sciences, 44-2~Vavilov\linebreak
$\hphantom{^1}$Str., Moscow 119333, Russian Federation

\noindent
$^2$University of Geneva, 22 Bd des Philosophes, CH-1205 Geneva 4, Switzerland

\def\leftfootline{\small{\textbf{\thepage}
\hfill INFORMATIKA I EE PRIMENENIYA~--- INFORMATICS AND
APPLICATIONS\ \ \ 2023\ \ \ volume~17\ \ \ issue\ 3}
}%
 \def\rightfootline{\small{INFORMATIKA I EE PRIMENENIYA~---
INFORMATICS AND APPLICATIONS\ \ \ 2023\ \ \ volume~17\ \ \ issue\ 3
\hfill \textbf{\thepage}}}

\vspace*{3pt}



\Abste{The paper presents an overview of structured definitions of discourse 
relations created based on classification principles and criteria for evaluating their 
semantic affinity. The authors point out the shortcomings of existing classification 
approaches that are sometimes inconsistent or contradictory and outline the benefits 
of an alternative approach to classification of discourse relations which is based on 
their structured definition. The paper provides examples of such definitions created 
within the Supracorpora Database of Connectives and discusses the criteria 
for evaluating their semantic closeness. As the structured definitions are represented 
by sets of distinguishing features, the authors discuss the problem of identifying 
proximity factors for each of these features. The gathered data suggest a~hypothesis 
that among the three groups of features: ``Level,'' ``Basic operation,'' and ``Feature 
family'', it is the last one that should have the most impact. Finally, directions for 
further research of this problem are considered, namely, the option of taking into 
account such factors as compatibility of discourse relations, correspondence of 
relations between the source text and its translation, and such cases where certain 
relation markers may express different discourse relations in various contexts.}

\KWE{supracorpora database; logical-semantic relations; connectives; 
annotation; faceted 
classification}



\DOI{10.14357/19922264230314}{UJZJZI}

%\vspace*{-20pt}

 \Ack
\noindent
The research was carried out using the infrastructure of the Shared Research 
Facilities ``High Performance 
Computing and Big Data'' (CKP ``Informatics'') of FRC CSC RAS (Moscow).

%\vspace*{6pt}

  \begin{multicols}{2}

\renewcommand{\bibname}{\protect\rmfamily References}
%\renewcommand{\bibname}{\large\protect\rm References}

{\small\frenchspacing
 {%\baselineskip=10.8pt
 \addcontentsline{toc}{section}{References}
 \begin{thebibliography}{99} 
\bibitem{1-kr-1}
\Aue{Hobbs, J.\,R.} 1976. A~computational approach to discourse analyses. New 
York, NY: Department of Computer Science, City College, City University of New 
York. Research Report~76-2. 28--38.
\bibitem{2-kr-1}
\Aue{Hobbs, J.\,R.} 1978. Why is discourse coherent? Menlo Park, CA: SRI 
International. SRI Technical Note~176. 44~p.
\bibitem{3-kr-1}
\Aue{Hobbs, J.\,R.} 1979. Coherence and coreference. \textit{Cognitive Sci.} 
3(1):67--90.
\bibitem{4-kr-1}
Shvedova, N.\,Yu., ed. 1980. \textit{Rus\-skaya gram\-ma\-ti\-ka.} 
[Russian grammar]. Moscow: Nauka. Vol.~2. 714~p.
\bibitem{5-kr-1}
Starodumova, E.\,A., E.\,S.~She\-re\-met'\-eva, and V.\,N.~Zav'ya\-lov, eds. 2022.  
\textit{Lek\-si\-ko\-gra\-fi\-che\-skie port\-re\-ty slu\-zheb\-nykh slov} 
[Lexicographic portraits of auxiliary words]. Vladivostok: FEFU. 322~p.
\bibitem{6-kr-1}
\Aue{Mann, W.\,C., and S.\,A.~Thomp\-son}. 1988. Rhetorical structure theory: 
Towards a~functional theory of text organization. \textit{Text} 8(3):243--281. doi: 
10.1515/text.1.1988.8.3.243.
\bibitem{7-kr-1}
\Aue{Knott, A., and R.~Dale.} 1994. Using linguistic phenomena to motivate a~set 
of coherence relations. \textit{Discourse Process.} 18(1):35--62. doi: 10.1080/01638539409544883.
\bibitem{8-kr-1}
\Aue{Rudolph, E.} 1996. \textit{Contrast: Adversative and concessive expressions on 
sentence and text level}. Berlin/Boston: Walter de Gruyter. 564~p.
\bibitem{9-kr-1}
\Aue{Fraser, B.} 2009. An account of discourse markers. \textit{International Review Pragmatics} 1(2):293--320. doi: 10.1163/187730909X12538045489818.
\bibitem{10-kr-1}
PDTB Research Group. 2008. The Penn Discourse Treebank~2.0 annotation manual. 
Philadelphia, PA: Institute for Research in Cognitive 
Science, University of Pennsylvania. Technical Report  IRCS-08-01. 99~p.\linebreak Available at: {\sf 
https://www.cis.upenn.edu/$\sim$elenimi/\linebreak pdtb-manual.pdf} (accessed August~1, 
2023).
\bibitem{11-kr-1}
\Aue{Breindl, E., A.~Vo\-lo\-di\-na, and U.\,H.~Wa{\ptb{\!\ss}}ner.} 2014. 
\textit{Handbuch der deutschen Konnektoren~2. Semantik der deutschen Satzverkn$\ddot{\mbox{u}}$pfer}. Berlin: Walter de Gruyter. 
1327~p.
\bibitem{12-kr-1}
\Aue{Inkova, O.\,Yu.} 2019. Logiko-semanticheskie otno\-she\-niya: prob\-le\-my klas\-si\-fi\-ka\-tsii 
[Logical-semantic relations: Classification problems]. \textit{Svyaz\-nost' teks\-ta: 
me\-reo\-lo\-gi\-che\-skie  logiko-semanticheskie ot\-no\-she\-niya} [Text Coherence: Mereological Logical Semantic 
Relations]. Moscow: LRC 
Publs. 11--98.
\bibitem{13-kr-1}
\Aue{Webber, B., R.~Prasad, A.~Lee, and A.~Joshi.} 2019. The Penn Discourse Treebank~3.0 annotation 
manual. 81~p. Available at: {\sf https://catalog.ldc.upenn.edu/docs/
LDC2019T05/PDTB3-Annotation-Manual.pdf} 
(accessed August~1, 2023).
\bibitem{14-kr-1}
\Aue{Inkova, O.\,Yu., and M.\,G.~Kruzh\-kov.} 2021. Struk\-tu\-ri\-ro\-van\-nye opre\-de\-le\-niya  
dis\-kur\-siv\-nykh ot\-no\-she\-niy v~Nad\-kor\-pus\-noy ba\-ze dan\-nykh  
kon\-nek\-to\-rov [Structured definitions of discourse relations in the Supracorpora Database of Connectives]. 
\textit{In\-for\-ma\-ti\-ka i~ee Pri\-me\-ne\-niya~--- Inform. Appl.} 15(4):27--32. doi: 10.14357/19922264210404.
\bibitem{15-kr-1}
Inkova, O.\,Yu., ed. 2018. \textit{Se\-man\-ti\-ka kon\-nek\-to\-rov:  
Kont\-ras\-tiv\-noe is\-sle\-do\-va\-nie} [Semantics of connectives: Contrastive study]. Moscow: TORUS PRESS. 368~p.
\bibitem{16-kr-1}
Inkova, O.\,Yu., ed. 2019. \textit{Struk\-tu\-ra kon\-nek\-to\-rov i~me\-to\-dy ee  
opi\-sa\-niya} [Structure of connectors and methods of its description]. Moscow: TORUS PRESS. 316~p.
\bibitem{17-kr-1}
\Aue{Kruzhkov, M.\,G.} 2021. Kon\-tsep\-tsiya po\-stro\-eniya nad\-kor\-pus\-nykh 
baz dan\-nykh [Conceptual framework for supracorpora databases]. 
 \textit{Sis\-te\-my i~Sred\-st\-va In\-for\-ma\-ti\-ki~--- Systems and Means of Informatics} 31(3):101--112.
  doi: 10.14357/08696527210309.
\bibitem{18-kr-1}
\Aue{Inkova, O.\,Yu.} 2021. Opre\-de\-le\-niya dis\-kur\-siv\-nykh ot\-no\-she\-niy: 
opyt Nad\-kor\-pus\-noy bazy dan\-nykh kon\-nek\-to\-rov [Definition of discursive relations: The experience 
of the supracorpora database of connectors]. \textit{Komp'yu\-ter\-naya 
 ling\-vi\-sti\-ka i~in\-tel\-lek\-tu\-al'\-nye tekh\-no\-lo\-gii: 
Po mat-m ezhegodnoy Mezhdunar.  konf. ``Dialog''} [Computational Linguistics and 
Intellectual Technologies: Papers from the Annual Conference (International) 
``Dialogue'']. Moscow: RGGU. 20(27):328--338.
\bibitem{19-kr-1}
Morkovkin, V.\,V., ed. 1997. \textit{Slo\-var' struk\-tur\-nykh slov rus\-sko\-go 
yazyka} [Dictionary of structural words of the Russian language]. Moscow: Lazur'. 422~p. 
\bibitem{20-kr-1}
\Aue{Inkova, O.\,Yu., and M.\,G.~Kruzh\-kov.} 2019. So\-che\-ta\-e\-most'  
logiko-semanticheskikh ot\-no\-she\-niy: ko\-li\-chest\-ven\-nye me\-to\-dy ana\-li\-za 
[Compatibility of logical semantic relations: Methods of quantitative analysis]. \textit{In\-for\-ma\-ti\-ka i~ee  
Pri\-me\-ne\-niya~--- Inform. Appl.} 
 13(2):83--91. doi: 10.14357/ 19922264190212.
 
 \end{thebibliography}

 }
 }

\end{multicols}

\vspace*{-6pt}

\hfill{\small\textit{Received July 10, 2023}} 

\vspace*{-18pt}

\Contr

\vspace*{-4pt}

\noindent
\textbf{Inkova Olga Yu.} (b.\ 1965)~--- Doctor of Science in philology, senior 
scientist, Institute of 
Informatics Problems, Federal Research Center ``Computer Science and Control'' of 
the Russian Academy of 
Sciences, 44-2~Vavilov Str., Moscow 119333, Russian Federation; faculty member, 
University of Geneva, 22~Bd des Philosophes, CH-1205 Geneva~4, Switzerland; 
\mbox{olyainkova@yandex.ru}

\vspace*{3pt}

\noindent
\textbf{Kruzhkov Mikhail G.} (b.\ 1975)~--- senior scientist, Institute of Informatics 
Problems, Federal Research Center ``Computer Science and Control'' of the Russian Academy of 
Sciences, 44-2~Vavilov Str., 
Moscow 119333, Russian Federation; \mbox{magnit75@yandex.ru}





\label{end\stat}

\renewcommand{\bibname}{\protect\rm Литература}  %15


%%%%%%%%%%%%%%%%%%%%%%%%%%%%%%%%%%%%%%%%%%%%%%%

%\def\stat{rez}
{%\hrule\par
%\vskip 7pt % 7pt
\raggedleft\Large \bf%\baselineskip=3.2ex
Р\,Е\,Ц\,Е\,Н\,З\,И\,И \vskip 17pt
    \hrule
    \par
\vskip 6pt plus 6pt minus 3pt }

%\thispagestyle{headings} %с верхним колонтитулом
%\thispagestyle{myheadings} %с нижним колонтитулом, но в верхнем РЕЦЕНЗИИ

\def\tit{НОВАЯ КНИГА И.\,Н.~СИНИЦЫНА, А.\,С.~ШАЛАМОВА <<ЛЕКЦИИ ПО ТЕОРИИ 
ИНТЕГРИРОВАННОЙ ЛОГИСТИЧЕСКОЙ ПОДДЕРЖКИ>> (М.: ТОРУС ПРЕСС, 2012. 624~с.)}

%1
\def\aut{Д.ф.-м.н., профессор С.\,Я.~Шоргин}

\def\auf{\ }

\def\leftkol{\ % РЕЦЕНЗИИ
}

\def\rightkol{ \ } 

%\def\leftkol{\ } % ENGLISH ABSTRACTS}

%\def\rightkol{\ } %ENGLISH ABSTRACTS}

%\def\leftkol{РЕЦЕНЗИИ}

%\def\rightkol{РЕЦЕНЗИИ}

\titele{\tit}{\aut}{\auf}{\leftkol}{\rightkol}
\vspace*{-18pt}


     \label{st\stat}

     \begin{multicols}{2}
     {\small
     {\baselineskip=10.1pt
     

      В книге представлено системное изложение теоретических основ одного из новейших 
направлений в \mbox{об\-ласти} экономики послепродажного обслуживания изделий наукоемкой 
продукции (ИНП) длительного пользования~--- интегрированной логистической поддержки
(ИЛП). 
{\looseness=1

}

Приведены также результаты новых работ, выполненных в Институте проблем информатики 
Российской академии наук в рамках научного направления <<Информационные технологии и 
анализ сложных сис\-тем>>.
 {%\looseness=1

}
     
      Излагаемые в книге научные подходы позво\-ляют карди\-наль\-но реформировать 
существующие системы производства и эксплуатации ИНП путем создания и внед\-ре\-ния 
методов рационального и оптимального управ\-ле\-ния процессами расходования 
вре\-мен\-н$\acute{\mbox{ы}}$х, 
мате\-ри\-аль\-ных, трудовых и других ресурсов на всех стадиях жизненного цикла изделий (ЖЦИ) по 
критериям экономической целесообразности и эф\-фек\-тив\-ности.
  {\looseness=1

}
    
      В книге приведен краткий обзор причин возник\-новения и
      развития CALS-методологии как основы 
современных международных стандартов по созданию и функционированию глобальных 
ин\-фор\-ма\-ци\-он\-но-ком\-му\-ни\-ка\-ци\-он\-ных систем, ее ключевых возможностей и эффективности 
результатов ее использования. 
Авторы %\linebreak 
предлагают ряд научных обоснований для разработки 
единой теории проектирования и управления систем ИЛП для полноценного использования 
преимуществ %\linebreak
 суще\-ст\-ву\-ющей методологии, определяют \mbox{общую} структурную схему 
комплексной системы <<ИНП-СППО>> и необходимость разработки для ее описания 
гибридных стохастических моделей.
{%\looseness=1

}

%\columnbreak
      
      Книга состоит из пяти частей, где последовательно излагается материал по каждой из 
следующих тем: <<Интегрированная логистическая поддержка>>, <<Теория гибридных 
стохастических систем и компьютерная поддержка исследований и разработок>>, <<Основы 
математического моделирования, анализа и синтеза систем послепродажного обслуживания>>, 
<<Определение и анализ показателей экспортного потенциала ИНП при проектировании>>, 
<<Задачи управления поддержкой послепродажного обслуживания>>, а также 
<<Моделирование инвестиционных процессов ИЛП в условиях неравновесных финансовых 
рынков>>. 
   
      В конце каждой главы приведены выводы и даны вопросы и задания для 
самоконтроля. В~приложениях содержатся основные определения по программам работ по 
анализу ИЛП, логистическим базам данных и компьютерным решениям, эквивалентной статистической 
линеаризации нелинейных преобразований ИЛП, справочный материал, а также развернутые 
уравнения для вероятностных характеристик.


      \def\leftkol{РЕЦЕНЗИИ}

\def\rightkol{РЕЦЕНЗИИ} 

      
      Книга заинтересует широкий круг специалистов и может быть использована научными 
проектными организациями в сфере промышленного производства ИНП. Большое количество 
иллюстраций, примеров и вопросов, обращенных к читателю, позволяет использовать книгу 
также в качестве учебного пособия для студентов и аспирантов машиностроительных, 
транспортных и~других специальностей, а также для самостоятельного изучения. 
{%\looseness=-1

}

Книга 
представляет несомненный интерес для специалистов и студентов в области прикладной 
математики и информатики.
    

}

}
\end{multicols}

%\newpage

\def\stat{authorsrus}
{%\hrule\par
%\vskip 7pt % 7pt
\raggedleft\Large \bf%\baselineskip=3.2ex
О\,Б\ \ А\,В\,Т\,О\,Р\,А\,Х \vskip 17pt
    \hrule
    \par
\vskip 21pt plus 8pt minus 4pt }


\def\tit{\ }

\def\aut{\ }

\def\auf{\ }

\def\leftkol{\ } % ENGLISH ABSTRACTS}

\def\rightkol{ОБ АВТОРАХ} %ENGLISH ABSTRACTS}

\titele{\tit}{\aut}{\auf}{\leftkol}{\rightkol}
      
            \label{st\stat}



\vspace*{24pt}

\begin{multicols}{2}




\noindent
\textbf{Архипов Олег Петрович} (р.\ 1948)~---
кандидат технических наук, директор Орловского филиала Института проб\-лем информатики
Российской академии наук
%302025, г.Орел, Московское шоссе, д.137

\vspace*{3pt}

\noindent
\textbf{Бирюкова Татьяна Константиновна} (р.\ 1968)~---
кандидат фи\-зи\-ко-ма\-те\-ма\-ти\-че\-ских наук, старший научный сотрудник Института проб\-лем информатики
Российской академии наук

\vspace*{3pt}

\noindent 
\textbf{Бобков  Сергей Геннадьевич} (р.\ 1955)~---
доктор технических наук,  заведующий отделением На\-уч\-но-ис\-сле\-до\-ва\-тель\-ско\-го 
института системных исследований Российской академии наук
%117218, Москва, Нахимовский просп., 36, к.1 

\vspace*{3pt}

\noindent \textbf{Васильев Николай Семенович} (р.\ 1952)~--- доктор 
фи\-зи\-ко-ма\-те\-ма\-ти\-че\-ских наук, профессор, 
МГТУ им.\ Н.\,Э.~Баумана 
%, Москва 105005, 2-я Бауманская ул., д.~5,

\vspace*{3pt}

\noindent
\textbf{Гершкович Максим Михайлович} (р.\ 1968)~---
старший научный сотрудник Института проб\-лем информатики
Российской академии наук

\vspace*{3pt}

\noindent 
\textbf{Дьяченко Юрий Георгиевич} (р.\ 1958)~--- кандидат технических наук, 
старший научный сотрудник Института проб\-лем информатики
Российской академии наук

\vspace*{3pt}

\noindent 
\textbf{Ерошенко Александр Андреевич} (р.\ 1989)~--- аспирант кафедры 
математической статистики факультета вычисли\-тельной математики и кибернетики 
Московского государственного университета им.\ М.\,В.~Ломоносова
%119991, Москва ГСП-1, Ленинские горы, д.\ 1, стр. 52

\vspace*{3pt}
 
\noindent 
\textbf{Захаров Виктор Николаевич} (р.\ 1948)~--- 
доктор технических наук, доцент, ученый секретарь Института проб\-лем информатики
Российской академии наук

\vspace*{3pt}

\noindent
\textbf{Зейфман Александр Израилевич} (р.\ 1954)~---
доктор фи\-зи\-ко-ма\-те\-ма\-ти\-че\-ских наук, профессор, 
заведующий кафедрой Вологодского государственного университета; 
старший научный сотрудник Института проб\-лем информатики
Российской академии наук; главный научный сотрудник ИСЭРТ Российской академии наук

\vspace*{3pt}

\noindent
\textbf{Зыкин Сергей Владимирович} (р.\ 1959)~--- 
доктор технических наук, профессор, заведующий лабораторией Института математики 
им.\ С.\,Л.~Соболева Сибирского отделения Российской академии наук, Новосибирск 
%630090, пр.\ ак.\ Коптюга, 4 

\vspace*{4pt}

\noindent
\textbf{Киреев Владимир Иванович} (р.\ 1938)~---
доктор фи\-зи\-ко-ма\-те\-ма\-ти\-че\-ских наук, профессор Московского 
государственного горного университета
%Адрес: Россия, 119991, г. Москва, Ленинский проспект, д. 6

%\columnbreak

\vspace*{4pt}

\noindent
\textbf{Козеренко Елена Борисовна} (р.\ 1959)~---
кандидат филологических наук, заведующая лабораторией Института проб\-лем информатики
Российской академии наук

\vspace*{4pt}

\noindent
\textbf{Королев Виктор Юрьевич} (р.\ 1954)~--- доктор
фи\-зи\-ко-ма\-те\-ма\-ти\-че\-ских наук, профессор кафедры математической 
статистики факультета вычисли\-тельной математики и кибернетики 
Московского государственного университета; 
ведущий научный сотрудник Института проб\-лем информатики
Российской академии наук

\vspace*{4pt}

\noindent
\textbf{Коротышева Анна Владимировна} (р.\ 1988)~---
старший преподаватель Вологодского государственного университета

\vspace*{4pt}

\noindent 
\textbf{Кун Де Турк} (р.\ 1981)~--- научный сотрудник 
исследовательской группы SMACS факультета телекоммуникаций и обработки информации
Университета Гента, Бельгия
%В-9000 Гент, Бельгия

\vspace*{4pt}

\noindent
\textbf{Лупенцов Олег Сергеевич} (р.\ 1986)~---
аспирант Омского государственного института сервиса
%Омск 644043, ул.\ Певцова 13

\vspace*{4pt}

\noindent
\textbf{Лучко Олег Николаевич} (р.\ 1961)~---
кандидат педагогических наук, профессор, заведующий кафедрой 
Омского государственного института сервиса
%Омск 644043, ул.\ Певцова 13

\vspace*{4pt}

\noindent
\textbf{Малашенко Юрий Евгеньевич} (р.\ 1946)~---
доктор фи\-зи\-ко-ма\-те\-ма\-ти\-че\-ских наук, заведующий сектором 
Вычислительного центра им.\ А.\,А.~Дородницына Российской академии наук
%Адрес: 119333, Москва, ул. Вавилова, 40,

\vspace*{4pt}

\noindent
\textbf{Маньяков Юрий Анатольевич} (р.\ 1984)~---
кандидат технических наук, научный сотрудник Орловского филиала Института проб\-лем информатики
Российской академии наук
%302025, г.Орел, Московское шоссе, д.137

\vspace*{4pt}

\noindent
\textbf{Маренко Валентина Афанасьевна} (р.\ 1951)~---
кандидат технических наук, доцент, старший научный сотрудник 
Института математики им.\ С.\,Л.~Соболева Сибирского отделения Российской академии наук
%Новосибирск 630090, пр. ак. Коптюга, 4 

\vspace*{3pt}

\noindent 
\textbf{Морозов Евсей Викторович} (р.\ 1947)~--- доктор 
фи\-зи\-ко-ма\-те\-ма\-ти\-че\-ских, профессор, ведущий научный сотрудник 
Института прикладных математических исследований Карельского научного центра Российской
академии наук; 
%%185910 Россия, Республика Карелия, г.\ Петрозаводск, ул.\ Пушкинская, 11
профессор Петрозаводского государственного университета, Петрозаводск
%185910 Россия, Республика Карелия, г.\ Петрозаводск, пр.\ Ленина, 33

%\pagebreak

\vspace*{3pt}

\noindent
\textbf{Назарова Ирина Александровна} (р.\ 1966)~---
кандидат фи\-зи\-ко-ма\-те\-ма\-ти\-че\-ских наук, 
научный сотрудник Вычислительного центра им.\ А.\,А.~Дородницына Российской академии наук 
%Адрес: 119333, Москва, ул. Вавилова, 40

\vspace*{3pt}

\noindent
\textbf{Павлов Игорь Валерианович} (р.\ 1945)~--- 
доктор фи\-зи\-ко-ма\-те\-ма\-ти\-че\-ских наук, профессор МГТУ им.\ Н.\,Э.~Баумана 
%Москва 105005, 2-я Бауманская ул., д.~5 

%\pagebreak

\vspace*{3pt}

\noindent 
\textbf{Потахина Любовь Викторовна} (р.\ 1989)~--- аспирантка
Института прикладных математических исследований Карельского научного центра
Российской академии наук; 
%%185910 Россия, Республика Карелия, г.\ Петрозаводск, ул.\ Пушкинская, 11
инженер Петрозаводского государственного университета, Петрозаводск
%185910 Россия, Республика Карелия, г.\ Петрозаводск, пр.\ Ленина, 33

\vspace*{3pt}

\noindent 
\textbf{Рождественский Юрий Владимирович} (р.\ 1952)~--- 
кандидат технических наук, заведующий сектором Института проб\-лем информатики
Российской академии наук

\vspace*{3pt}

\noindent 
\textbf{Синицын Игорь Николаевич} (р.\ 1940)~--- доктор технических наук,
профессор, заслуженный деятель\linebreak\vspace*{-12pt}

\columnbreak

\noindent
 науки РФ, заведующий отделом Института проб\-лем информатики
Российской академии наук

\vspace*{7pt}


\noindent
\textbf{Сиротинин Денис Олегович} (р.\ 1984)~---
кандидат технических наук, научный сотрудник Орловского филиала Института проб\-лем информатики
Российской академии наук
%302025, г.Орел, Московское шоссе, д.137

\vspace*{7pt}

%\columnbreak

\noindent 
\textbf{Соколов  Игорь Анатольевич} (р.\ 1954)~--- академик (действительный член) Российской 
академии наук, доктор технических наук, директор Института проб\-лем информатики
Российской академии наук

\vspace*{7pt}

\noindent
\textbf{Степченков Юрий Афанасьевич} (р.\ 1951)~---
кандидат технических наук, заведующий отделом Института проб\-лем информатики
Российской академии наук

\vspace*{7pt}

\noindent
\textbf{Сурков Алексей Викторович} (р.\ 1978)~--- 
старший научный сотрудник На\-уч\-но-ис\-сле\-до\-ва\-тель\-ско\-го 
института системных исследований Российской академии наук
%117218, Москва, Нахимовский просп., 36, к.1 

\vspace*{7pt}

\noindent 
\textbf{Шестаков Олег Владимирович} (р.\ 1976)~--- доктор 
фи\-зи\-ко-ма\-те\-ма\-ти\-че\-ских, доцент кафедры математической статистики 
факультета вычисли\-тельной математики и кибернетики Московского 
государственного университета им.\ М.\,В.~Ломоносова; 
%119991, Москва ГСП-1, Ленинские горы, д.\ 1, стр. 52
старший научный сотрудник Института проб\-лем информатики
Российской академии наук
%, Москва 119333, ул. Вавилова, д.~44, корп.~2

\vspace*{7pt}

\noindent 
\textbf{Шоргин Сергей Яковлевич} (р.\ 1952.)~--- доктор
фи\-зи\-ко-ма\-те\-ма\-ти\-че\-ских наук, профессор, заместитель директора Института 
проб\-лем информатики Российской академии наук





%%%%%%%%%%%%%%%%%%%%%%%%%%%%%%%%%%%%%%%%%%%%%%%%%%%%%%%%%%%%%%%%%%%%%%%%%%%%%%%




%\def\rightkol{ОБ АВТОРАХ}
%\def\leftkol{ОБ АВТОРАХ}

 \label{end\stat}





%\def\leftfootline{\small{\textbf{\thepage}
%\hfill ИНФОРМАТИКА И ЕЁ ПРИМЕНЕНИЯ\ \ \ том~7\ \ \ выпуск~1\ \ \ 2013}
%}%
% \def\rightfootline{\small{ИНФОРМАТИКА И ЕЁ ПРИМЕНЕНИЯ\ \ \ том~7\ \ \ выпуск~1\ \ \ 2013
%\hfill \textbf{\thepage}}}


%\thispagestyle{myheadings}



\end{multicols}

\newpage

%\end{document}

%
\def\stat{rekl}
%\label{preobr}

%\def\tit{АКАДЕМИК ПУГАЧЁВ  ВЛАДИМИР СЕМЁНОВИЧ\\
%25.03.1911--25.03.1998}


%   \vspace*{-48pt}
%   \begin{center}\LARGE
%Академик Пугачёв  Владимир Семёнович\\ (25.03.1911--25.03.1998)
%   \end{center}

   %\vspace*{2.5mm}

   \begin{center}

{\prgsh\LARGE
ЮБИЛЕИ}

\end{center}
%\hrule

\vspace*{6pt}


   \vspace*{8mm}

   \thispagestyle{empty}


%\def\stat{emel}


\section*{К 70-летию заместителя директора ИПИ РАН,\\ члена редколлегии журнала
<<Информатика и её применения>>\\ доктора технических наук В.\,И.~Будзко}

\vspace*{18pt}




          \begin{multicols}{2}

%            \label{st\stat}

\begin{center}
\vspace*{1pt}
\mbox{%
\epsfxsize=78mm
\epsfbox{bud-1.eps}
}
\end{center}

\vspace*{12pt}

      14 августа 2014~г.\ исполнилось 70~лет за\-мес\-ти\-те\-лю директора ИПИ РАН по
научной работе доктору технических наук Владимиру Игоревичу Будзко.

      Владимир Игоревич Будзко родился в г.~Москве. Высшее образование получил на факультете
элект\-рон\-но-вы\-чис\-ли\-тель\-ных устройств в Московском
ин\-же\-нер\-но-фи\-зи\-че\-ском институте
(МИФИ), который он окончил в 1968~г., после чего был на\-прав\-лен для прохождения
службы в одну из войс\-ко\-вых частей, где прошел путь от инженера до первого заместителя
командира войсковой части.

      С приходом В.\,И.~Будзко в ИПИ РАН (2001~г.)\ в институте
сформировалось новое научное на\-прав\-ле\-ние теоретических исследований~--- <<Постро\-ение
ин\-фор\-ма\-ци\-он\-но-те\-ле\-ком\-му\-ни\-ка\-ци\-он\-ных\linebreak сис\-тем
высокой до\-ступ\-ности>>. В~рамках этого
направления выполнен широкий круг фундаментальных исследований по поиску подходов и
определению принципов построения средств обеспечения доступности, конфиденциальности
и целостности современных крупномасштабных
ин\-фор\-ма\-ци\-он\-но-те\-ле\-ком\-му\-ни\-ка\-ци\-он\-ных
сис\-тем (ИТС). Разработаны основные сис\-тем\-но-тех\-ни\-че\-ские принципы и базовые
архитектурные решения построения перспективных для условий России ИТС с
централизованной обработкой и хранением информации, сочетающих в себе свойства
высокой доступности, отказо- и катастрофоустойчивости, информационной защищенности.
Определены принципы, методы и математические основы рационального построения и
оптимизации средств восстановления функционирования центров обработки данных (ЦОД)
после возникновения отказов и катастроф, передачи и хранения данных, обеспечения
информационной безопасности при достижении минимальной совокупной стоимости
владения такими системами. Результаты нашли практическое воплощение при реализации
проектов в интересах ряда отечественных государственных и негосударственных
организаций, таких как Банк России (БР), Внешторгбанк, ОАО <<ГМК <<Норильский Никель>>,
<<Газпром>>, Минэкономразвития России, Правительство Москвы, а также ряд силовых
ведомств.

      Под руководством В.\,И.~Будзко начиная с 2001~г.\ выполнен комплекс
      на\-уч\-но-ис\-сле\-до\-ва\-тель\-ских и
      опыт\-но-кон\-ст\-рук\-тор\-ских работ (свыше 100~проектов),
направленных на развитие электронной информационной технологии БР.
Разработаны концепции развития ИТС БР сначала до 2008~г., а затем до 2013~г., которые
были приняты в качестве основы проведения технической политики. За реализацию проекта
<<Катастрофоустойчивая тер\-ри\-то\-ри\-аль\-но-рас\-пре\-де\-лен\-ная
      ин\-фор\-ма\-ци\-он\-но-те\-ле\-ком\-му\-ни\-ка\-ци\-он\-ная сис\-те\-ма централизованной
обработки банковской информации>> В.\,И.~Будзко удостоен Премии Правительства РФ в
области науки и техники за 2010~г.

      В.\,И.~Будзко возглавлял и возглавляет работы по ряду других прикладных проектов,
связанных с созданием, совершенствованием и развитием крупномасштабных ИТС.

      В.\,И.~Будзко~--- генерал-майор, доктор технических наук, член-кор\-рес\-пон\-дент
Академии криптографии РФ, известный ученый в области информатики и применения
информационных технологий при построении территориально распределенных ИТС
различного назначения. Является автором свыше 250~научных работ, опубликованных в
на\-уч\-но-тех\-ни\-че\-ских и специальных изданиях.

    \thispagestyle{empty}

      В.\,И.~Будзко уделяет большое внимание подготовке научных кадров. Под его
руководством защищено 6~диссертаций на соискание ученой степени кандидата
технических наук. Свыше 30~лет он читает лекции в ИКСИ Академии ФСБ, профессор
кафедры НИЯУ МИФИ. Является членом двух диссертационных советов, главным
редактором журнала <<Системы высокой доступности>> и членом редколлегии журнала
<<Информатика и её применения>>.

      \bigskip

      Редакционный совет и Редакционная коллегия журнала <<Информатика и её
применения>> сердечно поздравляют Владимира Игоревича Будзко с 70-ле\-ти\-ем и желают
крепкого здоровья и новых научных достижений.

\end{multicols}

\def\stat{cont}
{%\hrule\par
%\vskip 7pt % 7pt
\raggedleft\Large \bf%\baselineskip=3.2ex
А\,В\,Т\,О\,Р\,С\,К\,И\,Й\ \ У\,К\,А\,З\,А\,Т\,Е\,Л\,Ь\ \ З\,А\ \ 2\,0\,1\,0 г. \vskip 17pt
    \hrule
    \par
\vskip 21pt plus 6pt minus 3pt }

\label{st\stat}

\def\tit{\ }

\def\aut{\ }
\def\auf{\ }

\def\leftkol{\ } % ENGLISH ABSTRACTS}

\def\rightkol{\ } %АВТОРСКИЙ УКАЗАТЕЛЬ ЗА 2010 г.} %ENGLISH ABSTRACTS}

\titele{\tit}{\aut}{\auf}{\leftkol}{\rightkol}

\vspace*{-12pt}

{\tabcolsep=3pt
\begin{tabular}{p{388pt}rr}
&\textbf{Выпуск} & \textbf{Стр.}\\[6pt]
\hangindent=23pt\noindent\textbf{Арутюнян~А.\,Р.} Моделирование влияния деформаций отпечатков пальцев на 
точность\linebreak
\vspace*{-12pt}\\
\hspace*{23pt}дактилоскопической идентификации$\dotfill$&1&51\\
\hangindent=23pt\noindent\textbf{Архипов~О.\,П., Зыкова~З.\,П.} Интеграция гетерогенной информации о цветных 
пикселях\linebreak
\vspace*{-12pt}\\
\hspace*{23pt}и их цветовосприятии$\dotfill$&4&15\\
\hangindent=23pt\noindent\textbf{Баранов~С.\,И., Френкель~С.\,Л., Захаров~В.\,Н.} Полуформальная верификация 
цифрового устройства с конвейером, основанная на использовании алгоритмических машин\linebreak
\vspace*{-12pt}\\
\hspace*{23pt}состояния$\dotfill$&4&49\\
\textbf{Бекетова~И.\,В.} см.~Каратеев~С.\,Л.&&\\
\textbf{Белоусов~В.\,В.} см.~Синицын~И.\,Н.&&\\
\hangindent=23pt\noindent\textbf{Бенинг~В.\,Е., Королев~Р.\,А.} О предельном поведении мощностей критериев в 
случае\linebreak
\vspace*{-12pt}\\
\hspace*{23pt}распределения Лапласа$\dotfill$&2&63\\
\hangindent=23pt\noindent\textbf{Бенинг~В.\,Е., Сипина~А.\,В.} Асимптотическое разложение для мощности 
критерия,\linebreak
\vspace*{-12pt}\\
\hspace*{23pt}основанного на выборочной медиане, в случае распределения Лапласа$\dotfill$&1&18\\
\textbf{Бондаренко~А.\,В.} см.~Каратеев~С.\,Л.&&\\
\hangindent=23pt\noindent\textbf{Бородина~А.\,В., Морозов~Е.\,В.} Об оценивании асимптотики вероятности 
большого\linebreak
\vspace*{-12pt}\\
\hspace*{23pt}уклонения стационарной регенеративной очереди с одним прибором$\dotfill$&3&29\\
\hangindent=23pt\noindent\textbf{Бунтман~Н.\,В., Минель~Ж.-Л., Ле~Пезан~Д., Зацман~И.\,М.} Типология и 
компьютерное\linebreak
\vspace*{-12pt}\\
\hspace*{23pt}моделирование трудностей перевода$\dotfill$&3&77\\
\textbf{Визильтер~Ю.\,В.} см.~Каратеев~С.\,Л.&&\\
\hangindent=23pt\noindent\textbf{Гавриленко~С.\,В.} Оценки скорости сходимости распределений случайных сумм с 
безгранично делимыми индексами к нормальному закону$\dotfill$&4&81\\
\hangindent=23pt\noindent\textbf{Григорьева~М.\,Е., Шевцова~И.\,Г.} Уточнение неравенства 
Каца--Берри--Эссеена$\dotfill$&2&75\\
\hangindent=23pt\noindent\textbf{Грушо~А.\,А., Грушо~Н.\,А., Тимонина~Е.\,Е.} Поиск конфликтов в политиках 
безопасности: модель случайных графов$\dotfill$&3&38\\
\textbf{Грушо~Н.\,А.} см.~Грушо~А.\,А.&&\\
\hangindent=23pt\noindent\textbf{Гудков~В.\,Ю.} Математические модели изображения отпечатка пальца на основе 
описания линий$\dotfill$&1&58\\
\textbf{Гуртов~А.\,В.} см.~Лукьяненко~А.\,С.&&\\
\textbf{Желтов~С.\,Ю.} см.~Каратеев~С.\,Л.&&\\
\hangindent=23pt\noindent\textbf{Захаров~А.\,А., Серебряков~В.\,А.} Система управления электронной библиотекой 
LibMeta$\dotfill$&4&2\\
\textbf{Захаров~В.\,Н.} см.~Баранов~С.\,И.&&\\
\textbf{Захарова~Т.\,В.} см.~Матвеева~С.\,С.&&\\
\hangindent=23pt\noindent\textbf{Зацаринный~А.\,А., Чупраков~К.\,Г.} Некоторые аспекты выбора технологии для 
постро-\linebreak
\vspace*{-12pt}\\
\hspace*{23pt}ения систем отображения информации ситуационного центра$\dotfill$&3&59\\
\textbf{Зацман~И.\,М.} см.~Бунтман~Н.\,В.&&\\
\hangindent=23pt\noindent\textbf{Зейфман~А.\,И., Коротышева~А.\,В., Сатин~Я.\,А., Шоргин~С.\,Я.} Об 
устойчивости нестаци-\linebreak
\vspace*{-12pt}\\
\hspace*{23pt}онарных систем обслуживания с катастрофами$\dotfill$&3&9\\
\textbf{Зыкова~З.\,П.} см.~Архипов~О.\,П.&&\\
\hangindent=23pt\noindent\textbf{Илюшин~Г.\,Я., Соколов~И.\,А.} Организация управляемого доступа пользователей 
к\linebreak
\vspace*{-12pt}\\
\hspace*{23pt}разнородным ведомственным информационным ресурсам$\dotfill$&1&24\\
\hangindent=23pt\noindent\textbf{Кавагучи~Ю., Ульянов~В.\,В., Фуджикоши~Я.} Приближения для статистик, 
описывающих\linebreak
\vspace*{-12pt}\\
\hspace*{23pt}геометрические свойства данных большой размерности, с оценками 
ошибок$\dotfill$&1&12\\
\hangindent=23pt\noindent\textbf{Каратеев~С.\,Л., Бекетова~И.\,В., Ососков~М.\,В., Князь~В.\,А., 
Визильтер~Ю.\,В., Бондаренко~А.\,В., Желтов~С.\,Ю.} Автоматизированный контроль 
качества цифровых\linebreak
\vspace*{-12pt}\\
\hspace*{23pt}изображений для персональных документов$\dotfill$&1&65\\
\end{tabular}
}

\pagebreak

\def\leftkol{АВТОРСКИЙ УКАЗАТЕЛЬ ЗА 2010 г.} % ENGLISH ABSTRACTS}

\def\rightkol{АВТОРСКИЙ УКАЗАТЕЛЬ ЗА 2010 г.} %ENGLISH ABSTRACTS}

{\tabcolsep=3pt
\begin{tabular}{p{388pt}rr}
&\textbf{Выпуск} & \textbf{Стр.}\\[3pt]
\hangindent=23pt\noindent\textbf{Козеренко~Е.\,Б.} Лингвистические фильтры в статистических моделях машинного\linebreak
\vspace*{-12pt}\\
\hspace*{23pt}перевода$\dotfill$&2&83\\
\hangindent=23pt\noindent\textbf{Козеренко~Е.\,Б., Кузнецов~И.\,П.} Когнитивно-лингвистические представления в 
систе-\linebreak
\vspace*{-12pt}\\
\hspace*{23pt}мах обработки текстов$\dotfill$&3&69\\
\textbf{Князь~В.\,А.} см.~Каратеев~С.\,Л.&&\\
\hangindent=23pt\noindent\textbf{Колесников~А.\,В., Солдатов~С.\,А.} Алгоритм координации для гибридной 
интеллектуальной системы решения сложной задачи оперативно-производственного\linebreak
\vspace*{-12pt}\\
\hspace*{23pt}планирования$\dotfill$&4&61\\
\hangindent=23pt\noindent\textbf{Коновалов~М.\,Г.} О планировании потоков в системах вычислительных 
ресурсов$\dotfill$&2&3\\
\textbf{Конушин~А.\,С.} см.~Конушин~В.\,С.&&\\
\hangindent=23pt\noindent\textbf{Конушин~В.\,С., Кривовязь~Г.\,Р., Конушин~А.\,С.} Алгоритм распознавания людей 
в видео-\linebreak
\vspace*{-12pt}\\
\hspace*{23pt}последовательности по одежде$\dotfill$&1&74\\
\textbf{Корепанов~Э.\, Р.} см.~Синицын~И.\,Н.&&\\
\textbf{Королев~В.\,Ю.} см.~Соколов~И.\,А.&&\\
\textbf{Королев~Р.\,А.} см.~Бенинг~В.\,Е.&&\\
\textbf{Коротышева~А.\,В.} см.~Зейфман~А.\,И.&&\\
\hangindent=23pt\noindent\textbf{Кривенко~М.\,П.} Непараметрическое оценивание элементов байесовского 
клас\-си-\linebreak
\vspace*{-12pt}\\
\hspace*{23pt}фикатора$\dotfill$&2&13\\
\textbf{Кривовязь~Г.\,Р.} см.~Конушин~В.\,С.&&\\
\textbf{Крылов~А.\,С.} см.~Павельева~Е.\,А.&&\\
\hangindent=23pt\noindent\textbf{Крылов~В.\,А.} Моделирование и классификация многоканальных дистанционных\linebreak
\vspace*{-12pt}\\
\hspace*{23pt}изображений с использованием копул$\dotfill$&4&34\\
\hangindent=23pt\noindent\textbf{Крючин~О.\,В.} Разработка параллельных эвристических алгоритмов подбора 
весовых\linebreak
\vspace*{-12pt}\\
\hspace*{23pt}коэффициентов искусственной нейтронной сети$\dotfill$&2&53\\
\hangindent=23pt\noindent\textbf{Кудрявцев~А.\,А., Шоргин~С.\,Я.} Байесовские модели массового обслуживания и 
надеж-\linebreak
\vspace*{-12pt}\\
\hspace*{23pt}ности: характеристики среднего числа заявок в системе $M\vert M \vert 1\vert 
\infty$$\dotfill$&3&16\\
\hangindent=23pt\noindent\textbf{Кузнецов~А.\,А.} Связь между временными и структурно-топологическими 
характери-\linebreak
\vspace*{-12pt}\\
\hspace*{23pt}стиками диаграмм ритма сердца здоровых людей$\dotfill$&4&39\\
\textbf{Кузнецов~И.\,П.} см.~Козеренко~Е.\,Б.&&\\
\textbf{Ле~Пезан~Д.} см.~Бунтман~Н.\,В.&&\\
\hangindent=23pt\noindent\textbf{Лукьяненко~А.\,С., Морозов~Е.\,В., Гуртов~А.\,В.} Анализ сетевого протокола с общей 
функ-\linebreak
\vspace*{-12pt}\\
\hspace*{23pt}цией расширения окна передачи сообщения при конфликтах$\dotfill$&2&46\\
\hangindent=23pt\noindent\textbf{Лямин~О.\,О.} О предельном поведении мощностей критериев в случае обобщенного\linebreak
\vspace*{-12pt}\\
\hspace*{23pt}распределения Лапласа$\dotfill$&3&47\\
\hangindent=23pt\noindent\textbf{Маркин~А.\,В., Шестаков~О.\,В.} Асимптотики оценки риска при пороговой 
обработке\linebreak
\vspace*{-12pt}\\
\hspace*{23pt}вейвлет-вейглет коэффициентов в задаче томографии$\dotfill$&2&36\\
\hangindent=23pt\noindent\textbf{Матвеева~С.\,С., Захарова~Т.\,В.} Сети массового обслуживания с наименьшей 
длиной\linebreak
\vspace*{-12pt}\\
\hspace*{23pt}очереди$\dotfill$&3&22\\
\hangindent=23pt\noindent\textbf{Матюшенко~С.\,И.} Стационарные характеристики двухканальной системы 
обслужива-\linebreak
\vspace*{-12pt}\\
\hspace*{23pt}ния с переупорядочиванием заявок и распределениями фазового типа$\dotfill$&4&68\\
\textbf{Минель~Ж.-Л.} см.~Бунтман~Н.\,В.&&\\
\textbf{Морозов~Е.\,В.} см.~Бородина~А.\,В.&&\\
\textbf{Морозов~Е.\,В.} см.~Лукьяненко~А.\,С.&&\\
\textbf{Ососков~М.\,В.} см.~Каратеев~С.\,Л.&&\\
\hangindent=23pt\noindent\textbf{Павельева~Е.\,А., Крылов~А.\,С.} Поиск и анализ ключевых точек радужной 
оболочки\linebreak
\vspace*{-12pt}\\
\hspace*{23pt}глаза методом преобразования Эрмита$\dotfill$&1&79\\
\textbf{Печинкин~А.\,В.} см.~Френкель~С.\,Л.,&&\\
\hangindent=23pt\noindent\textbf{Протасов~В.\,И.} Составление субъективного портрета с использованием 
эволюционно-\linebreak
\vspace*{-12pt}\\
\hspace*{23pt}го морфинга и квалиметрия метода$\dotfill$&1&83\\
\hangindent=23pt\noindent\textbf{Рудаков~К.\,В., Торшин~И.\,Ю.} Вопросы разрешимости задачи распознавания 
вторичной\linebreak
\vspace*{-12pt}\\
\hspace*{23pt}структуры белка$\dotfill$&2&25\\
\textbf{Сатин~Я.\,А.} см.~Зейфман~А.\,И.&&\\
\hangindent=23pt\noindent\textbf{Сейфуль-Мулюков~Р.\,Б.} Нефть как носитель информации о своем 
происхождении,\linebreak
\vspace*{-12pt}\\
\hspace*{23pt}структуре и эволюции$\dotfill$&1&41\\
\end{tabular}
}

{\tabcolsep=3pt
\begin{tabular}{p{388pt}rr}
&\textbf{Выпуск} & \textbf{Стр.}\\[6pt]
\textbf{Семендяев~Н.\,Н.} см.~Синицын~И.\,Н.&&\\
\textbf{Серебряков~В.\,А.} см.~Захаров~А.\,А.&&\\
\textbf{Синицын~В.\,И.} см.~Синицын~И.\,Н.&&\\
\hangindent=23pt\noindent\textbf{Синицын~И.\,Н., Синицын~В.\,И., Корепанов~Э.\, Р., Белоусов~В.\,В., 
Семендяев~Н.\,Н.} Оперативное построение информационных моделей движения полюса 
Земли\linebreak
\vspace*{-12pt}\\
\hspace*{23pt}методами линейных и линеаризованных фильтров$\dotfill$&1&2\\
\textbf{Сипина~А.\,В.} см.~Бенинг~В.\,Е.&&\\
\hangindent=23pt\noindent\textbf{Соколов~И.\,А.} О работах заслуженного деятеля науки Российской Федерации 
И.\,Н.~Синицына в области информационных технологий и автоматизации (к 70-летию\linebreak
\vspace*{-12pt}\\
\hspace*{23pt}со дня рождения)$\dotfill$&3&84\\
\textbf{Соколов~И.\,А.} см.~Илюшин~Г.\,Я.&&\\
\hangindent=23pt\noindent\textbf{Соколов~И.\,А., Королев~В.\,Ю.} Предисловие$\dotfill$&2&2\\
\textbf{Солдатов~С.\,А.} см.~Колесников~А.\,В.&&\\
\hangindent=23pt\noindent\textbf{Степанов~С.\,Ю.} Использование координатного метода фрагментации 
коммутаторной\linebreak
\vspace*{-12pt}\\
\hspace*{23pt}нейронной сети для сокращения трафика$\dotfill$&2&57\\
\textbf{Тимонина~Е.\,Е.} см.~Грушо~А.\,А.&&\\
\textbf{Торшин~И.\,Ю.} см.~Рудаков~К.\,В.&&\\
\textbf{Ульянов~В.\,В.} см.~Кавагучи~Ю.&&\\
\textbf{Фазекаш~И.} см.~Чупрунов~А.\,Н.&&\\
\textbf{Френкель~С.\,Л.} см.~Баранов~С.\,И.&&\\
\hangindent=23pt\noindent\textbf{Френкель~С.\,Л., Печинкин~А.\,В.} Оценка времени самовосстановления в 
цифровых\linebreak
\vspace*{-12pt}\\
\hspace*{23pt}системах после сбоев, вызываемых переходными помехами$\dotfill$&3&2\\
\textbf{Фуджикоши~Я.} см.~Кавагучи~Ю.&&\\
\hangindent=23pt\noindent\textbf{Цискаридзе~А.\,К.} Математическая модель и метод восстановления позы человека 
по\linebreak
\vspace*{-12pt}\\
\hspace*{23pt}стереопаре силуэтных изображений$\dotfill$&4&27\\
\hangindent=23pt\noindent\textbf{Чупраков~К.\,Г.} К вопросу о размещении коллективных средств отображения в 
ситуа-\linebreak
\vspace*{-12pt}\\
\hspace*{23pt}ционном зале с заданными параметрами$\dotfill$&4&89\\
\textbf{Чупраков~К.\,Г.} см.~Зацаринный~А.\,А.&&\\
\hangindent=23pt\noindent\textbf{Чупрунов~А.\,Н., Фазекаш~И.} Законы повторного логарифма для числа 
безошибочных\linebreak
\vspace*{-12pt}\\
\hspace*{23pt}блоков при помехоустойчивом кодировании$\dotfill$&3&42\\
\textbf{Шевцова~И.\,Г.} см.~Григорьева~М.\,Е.&&\\
\hangindent=23pt\noindent\textbf{Шестаков~О.\,В.} Аппроксимация распределения оценки риска пороговой 
обработки вейвлет-коэффициентов нормальным распределением при использовании 
выбо-\linebreak
\vspace*{-12pt}\\
\hspace*{23pt}рочной дисперсии$\dotfill$&4&73\\
\textbf{Шестаков~О.\,В.} см.~Маркин~А.\,В.&&\\
\textbf{Шоргин~С.\,Я.} см.~Зейфман~А.\,И.&&\\
\textbf{Шоргин~С.\,Я.} см.~Кудрявцев~А.\,А.&&\\
\end{tabular}
}

%\thispagestyle{myheadings}
\def\leftfootline{\small{\textbf{\thepage}
\hfill ИНФОРМАТИКА И ЕЁ ПРИМЕНЕНИЯ\ \ \ том~4\ \ \ выпуск~4\ \ \ 2010}
}%
 \def\rightfootline{\small{ИНФОРМАТИКА И ЕЁ ПРИМЕНЕНИЯ\ \ \ том~4\ \ \ выпуск~4\ \ \ 2010
 \hfill \textbf{\thepage}}}
 \label{end\stat}





%Том 10 Выпуск 1-4 Год 2016

\def\stat{cont-e}
{%\hrule\par
%\vskip 7pt % 7pt
\raggedleft\Large \bf%\baselineskip=3.2ex
2\,0\,1\,6\ \ A\,U\,T\,H\,O\,R\ \ I\,N\,D\,E\,X \vskip 17pt
 \hrule
 \par
\vskip 21pt plus 6pt minus 3pt }

\label{st\stat}

\def\tit{\ }

\def\aut{\ }
\def\auf{\ }

\def\leftkol{\ } %2016 AUTHOR INDEX} % ENGLISH ABSTRACTS}

\def\rightkol{\ } %2016 AUTHOR INDEX} %ENGLISH ABSTRACTS}

\titele{\tit}{\aut}{\auf}{\leftkol}{\rightkol}

\def\leftfootline{\small{\textbf{\thepage}
\hfill INFORMATIKA I EE PRIMENENIYA~--- INFORMATICS AND APPLICATIONS\ \ \ 2016\
\ \ volume~10\ \ \ issue\ 4}
}%
 \def\rightfootline{\small{INFORMATIKA I EE PRIMENENIYA~--- INFORMATICS AND APPLICATIONS\ \ \ 2016\ \ \ volume~10\ \ \ issue\ 4
\hfill \textbf{\thepage}}}

\vspace*{-12pt}
\vspace*{-18pt}

{\tabcolsep=2.8pt
\begin{tabular}{p{382pt}cc}
&\textbf{Issue} & \textbf{Page}\\[6pt]
\Avtors{Agalarov~M.\,Ya.} see~Agalarov~Ya.\,M.&&\\
\Avtors{Agalarov~Ya.\,M., Agalarov~M.\,Ya., and
Shorgin~V.\,S.} About the optimal threshold of queue\linebreak
\\[-12pt]
\hspace*{23pt}length in a~particular problem of profit maximization
in the $M/G/1$ queuing system&2&70--79\\
\Avtors{Alexeyevsky~D.\,A.} BioNLP ontology extraction from 
a~restricted language corpus with\linebreak
\\[-12pt]
\hspace*{23pt}context-free grammars&1&119--128\\
\Avtors{Andreev~S.\,D.} see~Gaidamaka~Yu.\,V.&&\\
\Avtors{Andreev~S.\,D.} see~Ometov~A.\,Ya.&&\\
\Avtors{Arkhipov~O.\,P., Arkhipov~P.\,O., and Sidorkin~I.\,I.} The
option to create a~local coordinate\linebreak
\\[-12pt]
\hspace*{23pt}system for synchronization of selected images&3&91--97\\
\Avtors{Arkhipov~P.\,O.} see~Arkhipov~O.\,P.&&\\
\Avtors{Belousov~V.\,V.} see~Shnurkov~P.\,V.&&\\
\Avtors{Belousov~V.\,V.} see~Shnurkov~P.\,V.&&\\
\Avtors{Bening~V.\,E.} Calculation of~the~asymptotic deficiency
of~some statistical procedures based\linebreak
\\[-12pt]
\hspace*{23pt}on~samples with~random sizes&4&34--45\\
\Avtors{Borisov~A.\,V., Bosov~A.\,V., and Miller~G.\,B.} Modeling and
monitoring of VoIP connection&2&\hphantom{1}2--13\\
\Avtors{Bosov~A.\,V.} see~Borisov~A.\,V.&&\\
\Avtors{Briukhov~D.\,O.} see~Stupnikov~S.\,A.&&\\
\Avtors{Callaos~N.\,K.\ and Seyful-Mulyukov~R.\,B.} Complexity and
its information content&1&129--139\\
\Avtors{Chertok~A.\,V., Kadaner~A.\,I., Khazeeva~G.\,T., and
Sokolov~I.\,A.} Regime switching detection\linebreak
\\[-12pt]
\hspace*{23pt}for~the~Levy driven
Ornstein--Uhlenbeck process using CUSUM methods&4&46--56\\
\Avtors{Chichagov~V.\,V.} Asymptotic expansions of mean absolute
error of uniformly minimum variance unbiased and maximum likelihood
estimators on the one-parameter exponential\linebreak
\\[-12pt]
\hspace*{23pt}family model of lattice distributions&3&66--76\\
\Avtors{Danishevsky~V.\,I.} see~Kolesnikov A.\,V.&&\\
\Avtors{Fazliev~A.\,Z.} see~Kalinichenko~L.\,A.&&\\
\Avtors{Fedoseev~A.\,A.} What is behind the concept of ``knowledge in
small packages''&3&105--110\\
\Avtors{Gaidamaka~Yu.\,V., Andreev~S.\,D., Sopin~E.\,S.,
Samouylov~K.\,E., and Shorgin~S.\,Ya.} Interference analysis
of~the~device-to-device communications model with~regard to~a~signal\linebreak
\\[-12pt]
\hspace*{23pt}propagation environment&4&\hphantom{1}2--10\\
\Avtors{Gasilov~A.\,V.} see~Yakovlev~O.\,A.&&\\
\Avtors{Goncharov~A.\,V.\ and Strijov~V.\,V.} Metric time series
classification using weighted dynamic\linebreak
\\[-12pt]
\hspace*{23pt}warping relative to centroids of classes&2&36--47\\
\Avtors{Gordov~E.\,P.} see~Kalinichenko~L.\,A.&&\\
\Avtors{Gorshenin~A.\,K.} Concept of online service for stochastic
modeling of real processes&1&72--81\\
\Avtors{Gorshenin~A.\,K.} see~Shnurkov~P.\,V.&&\\
\Avtors{Gorshenin~A.\,K.} see~Shnurkov~P.\,V.&&\\
\Avtors{Grusho~A.\,A., Grusho~N.\,A., Zabezhailo~M.\,I., and
Timonina~E.\,E.} Integration of statistical and\linebreak
\\[-12pt]
\hspace*{23pt}deterministic methods for
analysis of information security&3&2--8\\
\Avtors{Grusho~A.\,A., Zabezhailo~M.\,I., and Zatsarinny~A.\,A.} On
the advanced procedure to reduce\linebreak
\\[-12pt]
\hspace*{23pt}calculation of Galois closures&4&\hphantom{1}96--104\\
\Avtors{Grusho~N.\,A.} see~Grusho~A.\,A.&&\\
\Avtors{Havanskov~V.\,A.} see~Minin~V.\,A.&&\\
\Avtors{Inkova~O.\,Yu.} see~Zatsman~I.\,M.&&\\
\Avtors{Isachenko~R.\,V.\ and Strijov~V.\,V.} Metric learning in
multiclass time series classification\linebreak
\\[-12pt]
\hspace*{23pt}problem&2&48--57\\
\end{tabular}
}
\pagebreak

\def\leftfootline{\small{\textbf{\thepage}
\hfill INFORMATIKA I EE PRIMENENIYA~--- INFORMATICS AND APPLICATIONS\ \ \ 2016\
\ \ volume~10\ \ \ issue\ 4}
}%
 \def\rightfootline{\small{INFORMATIKA I EE PRIMENENIYA~---
INFORMATICS AND APPLICATIONS\ \ \ 2016\ \ \ volume~10\ \ \ issue\ 4
\hfill \textbf{\thepage}}}

\def\leftkol{2016 AUTHOR INDEX} % ENGLISH ABSTRACTS}

\def\rightkol{2016 AUTHOR INDEX} %ENGLISH ABSTRACTS}


{\tabcolsep=2.83pt
\begin{tabular}{p{382pt}cc}
&\textbf{Issue} & \textbf{Page}\\[6pt]
\Avtors{Kadaner~A.\,I.} see~Chertok~A.\,V.&&\\[.255pt]
\Avtors{Kalinichenko~L.\,A., Volnova~A.\,A., Gordov~E.\,P.,
Kiselyova~N.\,N., Kovaleva~D.\,A., Malkov~O.\,Yu., Okladnikov~I.\,G.,
Podkolodnyy~N.\,L., Pozanenko~A.\,S., Ponomareva~N.\,V.,
Stupnikov~S.\,A.,} \textbf{and Fazliev~A.\,Z.} Data access challenges for data
intensive\linebreak
\\[-12pt]
\hspace*{23pt}research in Russia&1& 2--22\\[.255pt]
\Avtors{Karasikov~M.\,E.\ and Strijov~V.\,V.} Feature-based
time-series classification&4&121--131\\[.255pt]
\Avtors{Khazeeva~G.\,T.} see~Chertok~A.\,V.&&\\[.255pt]
\Avtors{Khokhlov~Yu.\,S.} Multivariate fractional Levy motion and its
applications&2&\hphantom{1}98--106\\[.255pt]
\Avtors{Kirikov~I.\,A., Kolesnikov~A.\,V., Listopad~S.\,V., and
Rumovskaya~S.\,B.} Fine-grained hybrid\linebreak
\\[-12pt]
\hspace*{23pt}intelligent systems. Part 2:
Bidirectional hybridization&1&\hphantom{1}96--105\\[.255pt]
\Avtors{Kirikov~I.\,A., Kolesnikov~A.\,V., Listopad~S.\,V., and
Rumovskaya~S.\,B.} ``Virtual council''~---\linebreak
\\[-12pt]
\hspace*{23pt}source environment
supporting complex diagnostic decision making&3&81--90\\[.255pt]
\Avtors{Kiselyova~N.\,N.} see~Kalinichenko~L.\,A.&&\\[.255pt]
\Avtors{Kolesnikov A.\,V., Listopad~S.\,V., Rumovskaya~S.\,B., and
Danishevsky~V.\,I.} Informal axiomatic\linebreak
\\[-12pt]
\hspace*{23pt}theory of~the~role visual models&4&114--120\\[.255pt]
\Avtors{Kolesnikov~A.\,V.} see~Kirikov~I.\,A.&&\\[.255pt]
\Avtors{Kolesnikov~A.\,V.} see~Kirikov~I.\,A.&&\\[.255pt]
\Avtors{Kolin~K.\,K.} Humanitarian aspects of information
security&3&111--121\\[.255pt]
\Avtors{Konovalov~M.\,G.\ and Razumchik~R.\,V.} Dispatching
to~two parallel nonobservable queues using\linebreak
\\[-12pt]
\hspace*{23pt}only static
information&4&57--67\\[.255pt]
\Avtors{Korchagin~A.\,Yu.} see~Korolev~V.\,Yu.&&\\[.255pt]
\Avtors{Korchagin~A.\,Yu.} see~Korolev~V.\,Yu.&&\\[.255pt]
\Avtors{Korepanov~E.\,R.} see~Sinitsyn~I.\,N.&&\\[.255pt]
\Avtors{Korepanov~E.\,R.} see~Sinitsyn~I.\,N.&&\\[.255pt]
\Avtors{Korolev~V.\,Yu., Korchagin~A.\,Yu., and Zeifman~A.\,I.} The
Poisson theorem for Bernoulli trials\linebreak
\\[-12pt]
\hspace*{23pt}with~a~random probability
of~success and~a~discrete analog of~the~Weibull distribution&4&11--20\\[.255pt]
\Avtors{Korolev~V.\,Yu., Zeifman~A.\,I., and Korchagin~A.\,Yu.}
Asymmetric Linnik distributions as~limit\linebreak
\\[-12pt]
\hspace*{23pt}laws for~random sums
of~independent random variables with~finite variances&4&21--33\\[.255pt]
\Avtors{Koucheryavy~E.\,A.} see~Ometov~A.\,Ya.&&\\[.255pt]
\Avtors{Kovaleva~D.\,A.} see~Kalinichenko~L.\,A.&&\\[.255pt]
\Avtors{Kovalyov~S.\,P.} Metaprogramming to increase
manufacturability of large-scale software-\linebreak
\\[-12pt]
\hspace*{23pt}intensive systems&1&56--66\\[.255pt]
\Avtors{Krivenko~M.\,P.} Significance tests of feature selection for
classification&3&32--40\\[.255pt]
\Avtors{Kruzhkov~M.\,G.} see~Zalizniak~Anna~A.&&\\[.255pt]
\Avtors{Kruzhkov~M.\,G.} see~Zatsman~I.\,M.&&\\[.255pt]
\Avtors{Kudryavtsev~A.\,A.} Bayesian queueing and reliability models:
\textit{A~priori} distributions with\linebreak
\\[-12pt]
\hspace*{23pt}compact support&1&67--71\\[.255pt]
\Avtors{Kudryavtsev~A.\,A.} Characteristics dependent on the balance
coefficient in Bayesian models\linebreak
\\[-12pt]
\hspace*{23pt}with compact support of \textit{a priori}
distributions&3&77--80\\[.255pt]
\Avtors{Kudryavtsev~A.\,A.\ and Palionnaia~S.\,I.} Bayesian recurrent
model of reliability growth:\linebreak
\\[-12pt]
\hspace*{23pt}Parabolic distribution of parameters&2&80--83\\[.255pt]
\Avtors{Kudryavtsev~A.\,A.\ and Titova~A.\,I.} Bayesian queuing
and~reliability models: Degenerate-\linebreak
\\[-12pt]
\hspace*{23pt}Weibull case&4&68--71\\[.255pt]
\Avtors{Leontyev~N.\,D.\ and Ushakov~V.\,G.} Analysis of a queueing
system with autoregressive arrivals\linebreak
\\[-12pt]
\hspace*{23pt}and nonpreemptive priority&3&15--22\\[.255pt]
\Avtors{Listopad~S.\,V.} see~Kirikov~I.\,A.&&\\[.255pt]
\Avtors{Listopad~S.\,V.} see~Kirikov~I.\,A.&&\\[.255pt]
\Avtors{Listopad~S.\,V.} see~Kolesnikov A.\,V.&&\\[.255pt]
\Avtors{Malkov~O.\,Yu.} see~Kalinichenko~L.\,A.&&\\[.255pt]
\Avtors{Markov~A.\,S., Monakhov~M.\,M., and
Ulyanov~V.\,V.} Generalized Cornish--Fisher expansions\linebreak
\\[-12pt]
\hspace*{23pt}for distributions of statistics based on samples
of random size&2&84--91\\[.255pt]
\Avtors{Melnikov~A.\,K.\ and Ronzhin~A.\,F.} Generalized statistical
method of~text analysis based\linebreak
\\[-12pt]
\hspace*{23pt}on~calculation of~probability distributions
of~statistical values&4&89--95\\
\end{tabular}
}
\pagebreak

\def\leftfootline{\small{\textbf{\thepage}
\hfill INFORMATIKA I EE PRIMENENIYA~--- INFORMATICS AND APPLICATIONS\ \ \ 2016\
\ \ volume~10\ \ \ issue\ 4}
}%
 \def\rightfootline{\small{INFORMATIKA I EE PRIMENENIYA~---
INFORMATICS AND APPLICATIONS\ \ \ 2016\ \ \ volume~10\ \ \ issue\ 4
\hfill \textbf{\thepage}}}

\def\leftkol{2016 AUTHOR INDEX} % ENGLISH ABSTRACTS}

\def\rightkol{2016 AUTHOR INDEX} %ENGLISH ABSTRACTS}


{\tabcolsep=3pt
\begin{tabular}{p{381pt}cc}
&\textbf{Issue} & \textbf{Page}\\[6pt]
\Avtors{Meykhanadzhyan~L.\,A.} Stationary characteristics of the finite
capacity queueing system with\linebreak
\\[-12pt]
\hspace*{23pt}inverse service order and generalized
probabilistic priority&2&123--131\\[.23pt]
\Avtors{Miller~G.\,B.} see~Borisov~A.\,V.&&\\[.23pt]
\Avtors{Minin~V.\,A., Zatsman~I.\,M., Havanskov~V.\,A., and
Shubnikov~S.\,K.} Intensity of citation of scientific publications in
inventions on information and computer technologies patented\linebreak
\\[-12pt]
\hspace*{23pt}in Russia by domestic and foreign applicants&2&107--122\\[.23pt]
\Avtors{Monakhov~M.\,M.} see~Markov~A.\,S.&&\\[.23pt]
\Avtors{Naumov~V.\,A.\ and Samouylov~K.\,E.} On relationship
between queuing systems with resources\linebreak
\\[-12pt]
\hspace*{23pt}and Erlang networks&3&\hphantom{1}9--14\\[.23pt]
\Avtors{Okladnikov~I.\,G.} see~Kalinichenko~L.\,A.&&\\[.23pt]
\Avtors{Ometov~A.\,Ya., Andreev~S.\,D., Turlikov~A.\,M., and
Koucheryavy~E.\,A.} Performance analysis of\linebreak
\\[-12pt]
\hspace*{23pt}a wireless data
aggregation system with contention for contemporary sensor
networks&3&23--31\\[.23pt]
\Avtors{Palionnaia~S.\,I.} see~Kudryavtsev~A.\,A.&&\\[.23pt]
\Avtors{Podkolodnyy~N.\,L.} see~Kalinichenko~L.\,A.&&\\[.23pt]
\Avtors{Ponomareva~N.\,V.} see~Kalinichenko~L.\,A.&&\\[.23pt]
\Avtors{Popkova~N.\,A.} see~Zatsman~I.\,M.&&\\[.23pt]
\Avtors{Pozanenko~A.\,S.} see~Kalinichenko~L.\,A.&&\\[.23pt]
\Avtors{Razumchik~R.\,V.} see~Konovalov~M.\,G.&&\\[.23pt]
\Avtors{Ronzhin~A.\,F.} see~Melnikov~A.\,K.&&\\[.23pt]
\Avtors{Rumovskaya~S.\,B.} see~Kirikov~I.\,A.&&\\[.23pt]
\Avtors{Rumovskaya~S.\,B.} see~Kirikov~I.\,A.&&\\[.23pt]
\Avtors{Rumovskaya~S.\,B.} see~Kolesnikov A.\,V.&&\\[.23pt]
\Avtors{Samouylov~K.\,E.} see~Gaidamaka~Yu.\,V.&&\\[.23pt]
\Avtors{Samouylov~K.\,E.} see~Naumov~V.\,A.&&\\[.23pt]
\Avtors{Serebryanskii~S.\,M.} see~Tyrsin~A.\,N.&&\\[.23pt]
\Avtors{Seyful-Mulyukov~R.\,B.} see~Callaos~N.\,K.&&\\[.23pt]
\Avtors{Shestakov~O.\,V.} Statistical properties of the denoising method
based on the stabilized hard\linebreak
\\[-12pt]
\hspace*{23pt}thresholding&2&65--69\\[.23pt]
\Avtors{Shestakov~O.\,V.} The strong law of large numbers for the risk
estimate in the problem of\linebreak
\\[-12pt]
\hspace*{23pt}tomographic image reconstruction from
projections with a correlated noise&3&41--45\\[.23pt]
\Avtors{Shestakov~O.\,V.} see~Zakharova~T.\,V.&&\\[.23pt]
\Avtors{Shnurkov~P.\,V., Gorshenin~A.\,K., and Belousov~V.\,V.}
Analytical solution of~the~optimal control\linebreak
\\[-12pt]
\hspace*{23pt}task of~a~semi-Markov
process with~finite set of~states&4&72--88\\[.23pt]
\Avtors{Shnurkov~P.\,V., Zasypko~V.\,V., Belousov~V.\,V., and
Gorshenin~A.\,K.} Development of the algorithm of numerical solution
of the optimal investment control problem\linebreak
\\[-12pt]
\hspace*{23pt}in the closed dynamical model of three-sector economy&1&82--95\\[.23pt]
\Avtors{Shorgin~S.\,Ya.} see~Gaidamaka~Yu.\,V.&&\\[.23pt]
\Avtors{Shorgin~V.\,S.} see~Agalarov~Ya.\,M.&&\\[.23pt]
\Avtors{Shubnikov~S.\,K.} see~Minin~V.\,A.&&\\[.23pt]
\Avtors{Sidorkin~I.\,I.} see~Arkhipov~O.\,P.&&\\[.23pt]
\Avtors{Sinitsyn~I.\,N.} Analytical modeling of processes in stochastic
systems with complex fractional\linebreak
\\[-12pt]
\hspace*{23pt}order Bessel nonlinearities&3&55--65\\[.23pt]
\Avtors{Sinitsyn~I.\,N.} Orthogonal supoptimal filters for nonlinear
stochastic systems on manifolds&1&34--44\\[.23pt]
\Avtors{Sinitsyn~I.\,N.\ and Korepanov~E.\,R.} Normal Pugachev
conditionally-optimal filters and extra-\linebreak
\\[-12pt]
\hspace*{23pt}polators for state linear stochastic systems&2&14--23\\[.23pt]
\Avtors{Sinitsyn~I.\,N.\ and Sinitsyn~V.\,I.} Analytical modeling of
distributions in stochastic systems on\linebreak
\\[-12pt]
\hspace*{23pt}manifolds based on ellipsoidal approximation&1&45--55\\[.23pt]
\Avtors{Sinitsyn~I.\,N., Sinitsyn~V.\,I., and
Korepanov~E.\,R.} Ellipsoidal suboptimal filters for nonlinear\linebreak
\\[-12pt]
\hspace*{23pt}stochastic systems on manifolds&2&24--35\\[.23pt]
\Avtors{Sinitsyn~V.\,I.} see~Sinitsyn~I.\,N.&&\\[.23pt]
\Avtors{Sinitsyn~V.\,I.} see~Sinitsyn~I.\,N.&&\\[.23pt]
\Avtors{Skvortsov~N.\,A.} see~Stupnikov~S.\,A.&&\\[.23pt]
\Avtors{Sokolov~I.\,A.} see~Chertok~A.\,V.&&\\
\end{tabular}
}
\pagebreak

\def\leftfootline{\small{\textbf{\thepage}
\hfill INFORMATIKA I EE PRIMENENIYA~--- INFORMATICS AND APPLICATIONS\ \ \ 2016\
\ \ volume~10\ \ \ issue\ 4}
}%
 \def\rightfootline{\small{INFORMATIKA I EE PRIMENENIYA~---
INFORMATICS AND APPLICATIONS\ \ \ 2016\ \ \ volume~10\ \ \ issue\ 4
\hfill \textbf{\thepage}}}

\def\leftkol{2016 AUTHOR INDEX} % ENGLISH ABSTRACTS}

\def\rightkol{2016 AUTHOR INDEX} %ENGLISH ABSTRACTS}


{\tabcolsep=3pt
\begin{tabular}{p{382pt}cc}
&\textbf{Issue} & \textbf{Page}\\[6pt]
\Avtors{Sopin~E.\,S.} see~Gaidamaka~Yu.\,V.&&\\
\Avtors{Strijov~V.\,V.} see~Goncharov~A.\,V.&&\\
\Avtors{Strijov~V.\,V.} see~Isachenko~R.\,V.&&\\
\Avtors{Strijov~V.\,V.} see~Karasikov~M.\,E.&&\\
\Avtors{Stupnikov~S.\,A., Briukhov~D.\,O., and Skvortsov~N.\,A.}
Co-lending systemic risk analysis over\linebreak
\\[-12pt]
\hspace*{23pt}heterogeneous data collections&1&23--33\\
\Avtors{Stupnikov~S.\,A.} see~Kalinichenko~L.\,A.&&\\
\Avtors{Suchkov~A.\,P.} see~Zatsarinny~A.\,A.&&\\
\Avtors{Timonina~E.\,E.} see~Grusho~A.\,A.&&\\
\Avtors{Titova~A.\,I.} see~Kudryavtsev~A.\,A.&&\\
\Avtors{Turlikov~A.\,M.} see~Ometov~A.\,Ya.&&\\
\Avtors{Tyrsin~A.\,N.\ and Serebryanskii~S.\,M.} Recognition of
dependences on the basis of inverse\linebreak
\\[-12pt]
\hspace*{23pt}mapping&2&58--64\\
\Avtors{Ulyanov~V.\,V.} see~Markov~A.\,S.&&\\
\Avtors{Ushakov~V.\,G.} Queueing system with working vacations and
hyperexponential input stream&2&92--97\\
\Avtors{Ushakov~V.\,G.} see~Leontyev~N.\,D.&&\\
\Avtors{Volnova~A.\,A.} see~Kalinichenko~L.\,A.&&\\
\Avtors{Yakovlev~O.\,A.\ and Gasilov~A.\,V.} Speeded-up stereo
matching using geodesic support weights&3&\hphantom{1}98--104\\
\Avtors{Zabezhailo~M.\,I.} see~Grusho~A.\,A.&&\\
\Avtors{Zabezhailo~M.\,I.} see~Grusho~A.\,A.&&\\
\Avtors{Zakharova~T.\,V.\ and Shestakov~O.\,V.} Precision analysis of
wavelet processing of aerodynamic\linebreak
\\[-12pt]
\hspace*{23pt}flow patterns&3&46--54\\
\Avtors{Zalizniak~Anna~A.\ and Kruzhkov~M.\,G.} Database
of~Russian impersonal verbal constructions&4&132--141\\
\Avtors{Zasypko~V.\,V.} see~Shnurkov~P.\,V.&&\\
\Avtors{Zatsarinny~A.\,A.\ and Suchkov~A.\,P.} Systems engineering
approaches to~the~establishment of\linebreak
\\[-12pt]
\hspace*{23pt}a~system for~decision support based
on~situational analysis&4&105--113\\
\Avtors{Zatsarinny~A.\,A.} see~Grusho~A.\,A.&&\\
\Avtors{Zatsman~I.\,M., Inkova~O.\,Yu., Kruzhkov~M.\,G., and
Popkova~N.\,A.} Representation of cross-\linebreak
\\[-12pt]
\hspace*{23pt}lingual knowledge about
connectors in supracorpora databases&1&106--118\\
\Avtors{Zatsman~I.\,M.} see~Minin~V.\,A.&&\\
\Avtors{Zeifman~A.\,I.} see~Korolev~V.\,Yu.&&\\
\Avtors{Zeifman~A.\,I.} see~Korolev~V.\,Yu.&&\\
\end{tabular}
}

%\thispagestyle{myheadings}
\def\leftfootline{\small{\textbf{\thepage}
\hfill INFORMATIKA I EE PRIMENENIYA~--- INFORMATICS AND APPLICATIONS\ \ \ 2016\
\ \ volume~10\ \ \ issue\ 4}
}%
 \def\rightfootline{\small{INFORMATIKA I EE PRIMENENIYA~---
INFORMATICS AND APPLICATIONS\ \ \ 2016\ \ \ volume~10\ \ \ issue\ 4
\hfill \textbf{\thepage}}}

 \label{end\stat}

\newpage

%\def\stat{rekl}
%\label{preobr}

%\def\tit{АКАДЕМИК ПУГАЧЁВ  ВЛАДИМИР СЕМЁНОВИЧ\\
%25.03.1911--25.03.1998}


%   \vspace*{-48pt}
%   \begin{center}\LARGE
%Академик Пугачёв  Владимир Семёнович\\ (25.03.1911--25.03.1998)
%   \end{center}
   
   %\vspace*{2.5mm}
   
   \begin{center}

{\prgsh\LARGE
ОБЪЯВЛЕНИЯ О КОНФЕРЕНЦИЯХ}

\end{center}
%\hrule

\vspace*{6pt}

   
   \vspace*{10mm}
   
   \thispagestyle{empty}

\noindent
\begin{tabular}{cc}
%\begin{center}
\multicolumn{1}{c}{\raisebox{-40pt}[0pt][0pt]{\mbox{%
\epsfxsize=33mm
\epsfbox{vspu.eps}
}}}
%\end{center}
&
\tabcolsep=0pt\begin{tabular}{c}
{\prg{\Large\textbf{XII Всероссийское совещание}}}\\[6pt]
{\prg{\Large\textbf{по проблемам управления}}}\\[12pt]
{\prg{\large 16--19 июня 2014~г.}}\\[6pt] 
{\prg{\large Институт проблем управления имени В.\,А.~Трапезникова РАН}}\\[6pt]
{\prg{\large Москва, Россия}}
\end{tabular}
\end{tabular}

\vspace*{60pt}

     
 { %\large    
 XII Всероссийское совещание по проблемам управления (ВСПУ XII), посвященное 75-летию 
Института проблем управления (ИПУ) имени В.\,А.~Трапезникова РАН, проводится 16--19~июня 
2014~г.\ 
в ИПУ РАН (г.~Москва, Россия). ВСПУ XII организуется ИПУ РАН при поддержке РФФИ, Отделения 
энергетики, машиностроения, механики и процессов управления Российской академии наук, 
Российского 
национального комитета по автоматическому управлению, Академии навигации и управ\-ле\-ния 
движением, 
Научного совета РАН по комплексным проблемам управления и автоматизации, Совета по 
мехатронике и робототехнике РАН. Официальный язык Совещания~--- русский.

\vspace*{24pt}
     
     \textbf{Направления работы}
     \begin{enumerate}[1.]
\item Теория систем управления
\item Управление подвижными объектами и навигация
\item Интеллектуальные системы управления
\item Управление в промышленности, транспортом и логистикой
\item Управление системами междисциплинарной природы
\item Средства измерения, вычислений и контроля в управлении
\item Системный анализ и принятие решений в задачах управления
\item Информационные технологии в управлении
\item Проблемы образования в области управления: современное содержание и технологии обучения
\end{enumerate}

\vspace*{24pt}

     Подробная информация о Совещании находится на сайте {\sf http://vspu2014.ipu.ru}. Срок 
окончательной подачи докладов через систему подачи докладов на сайте~--- \textbf{30~ноября} 
2013~г.
}

%\include{rekl-1}

%\end{document}

%   \vspace*{-48pt}

\begin{center}
\vspace*{6pt}
\mbox{%
\epsfxsize=53.502mm
\epsfbox{foto-1.eps}
}
\end{center}

\vspace*{6pt} %Академик


   \begin{center}
\fbox{\Large\textbf{Профессор Игорь Алексеевич Ушаков}}\\[12pt]
\textbf{\large 22.01.1935--27.02.2015}
   \end{center}


   %\vspace*{2.5mm}

   \vspace*{5mm}

   \thispagestyle{empty}

%\

%\vspace*{-12pt}


Редакционный совет и редакционная коллегия журнала <<Информатика и~её применения>> с~глубоким прискорбием извещают, что 27~февраля 2015~г.\ после тяжелой
и~продолжительной болезни скончался Игорь Алексеевич Ушаков~--- доктор технических наук, профессор, член редколлегии журнала <<Информатика и ее применения>>.

Игорь Алексеевич Ушаков окончил Московский авиационный институт, в~1963~г.\ защитил кандидатскую, а~в~1968~г.~--- докторскую диссертацию. С~1958 по 1989~гг.\ работал в~ряде научно-исследовательских организаций СССР, в~том числе руководил отделами в~НИИ АА и~ВЦ АН СССР; с 1969 по 1989 гг. преподавал в~МФТИ (был профессором, а~затем заведующим кафедрой) и~в~МЭИ. С~1989~г.~---- в~США: являлся профессором университета Дж.\ Вашингтона, университета Дж.\ Мэйсона и~Калифорнийского университета, сотрудником компаний MCI, Qualcomm и Hughes.

И.\,А.~Ушаков с момента основания журнала <<Надежность и~контроль качества>> был заместителем ответственного редактора, а~затем на протяжении многих лет членом редколлегии. В~2006~г.\ основал электронный международный журнал ``Reliability: Theory \& Application'', главным редактором которого оставался до конца жизни.

Учебниками и справочниками по теории надежности, написанными И.\,А.~Ушаковым, пользовались и~пользуются несколько поколений ученых и~специалистов в~разных странах мира.

Игорь Алексеевич всегда уделял огромное внимание работе с~молодежью; более~50 его учеников защитили докторские и~кандидатские диссертации.

И.\,А.~Ушаков вел активную научно-про\-све\-ти\-тель\-скую деятельность. В~частности, он был одним из организаторов и~руководителей Московского кабинета качества и~надежности при Политехническом музее (целью этого Кабинета было оказание консультаций работникам промышленных предприятий и~чтение курсов лекций для инженеров, занимающихся проблемой надежности). Находясь в~США, И.\,А.~Ушаков создал международный ин\-тер\-нет-фо\-рум им.\ Б.\,В.~Гнеденко, объединивший около~400~видных специалистов по приложениям теории вероятностей и~математической статистики, преимущественно в~об\-ласти теории надежности и~анализа риска, из десятков стран мира; коллективным членов этого Форума является и~наш журнал. Цели Форума~--- содействие контактам между специалистами из разных стран, организация обмена профессиональными 
новостями и~информацией (новые публикации, предстоящие события и~др.). Также необходимо отметить большое число на\-уч\-но-по\-пу\-ляр\-ных работ, опубликованных И.\,А.~Ушаковым.

И.\,А.~Ушаков обладал большим личным обаянием, имел широкий круг интересов. Все знавшие И.\,А.~Ушакова всегда будут помнить его как замечательного ученого и~прекрасного человека.

\bigskip

Редакционный совет и редакционная коллегия журнала <<Информатика и~её применения>> 
выражают глубокие соболезнования родным и близким покойного, всем, кто его знал и~работал с~ним.



%\end{document}

%\include{IPPM-25}

\def\stat{cont-rus}
{%\hrule\par
%\vskip 7pt % 7pt
\vspace*{-24pt}
\raggedleft\Large \bf%\baselineskip=3.2ex
Правила подготовки рукописей  для публикации в журнале
<<Информатика~и~её~применения>> \vskip 8pt
    \hrule
    \par
\vskip 14pt plus 6pt minus 3pt }

\label{st\stat}

\def\tit{\ }

\def\aut{\ }
\def\auf{\ }

\def\leftkol{\ }
% Правила подготовки рукописей  для публикации в журнале
%<<Информатика и её применения>>

\def\rightkol{\ }
%Правила подготовки рукописей  для публикации в журнале
%<<Информатика и её применения>>}


\titele{\tit}{\aut}{\auf}{\leftkol}{\rightkol}


\vspace*{-60pt}
{ %\small

Журнал <<Информатика и её применения>>
публикует теоретические, обзорные и дискуссионные статьи,
посвященные научным исследованиям и разработкам в области
информатики и ее приложений.

Журнал издается на русском языке. По специальному решению
редколлегии отдельные статьи могут печататься на английском языке.

Тематика журнала охватывает следующие направления:
\begin{itemize}
\item теоретические основы информатики;\\[-15pt]
      \item
математические методы исследования сложных систем и процессов;\\[-15pt]
           \item
информационные системы и сети;\\[-15pt]
                \item
информационные технологии;\\[-15pt]
                     \item
архитектура и программное обеспечение вычислительных комплексов и сетей.\\[-15pt]
\end{itemize}


\noindent
\begin{enumerate}[1.]
\item В журнале печатаются статьи, содержащие результаты, ранее не опубликованные и
не предназначенные к одновременной публикации в других изданиях.

%Публикация не должна нарушать закон об авторских правах.
Публикация предоставленной автором(ами) рукописи не должна нарушать 
положений глав~69, 70 раздела~VII части~IV Гражданского кодекса, 
которые определяют права на результаты интеллектуальной деятельности 
и~средства индивидуализации, в~том числе авторские права, в~РФ.

Ответственность за нарушение авторских прав, в~случае предъявления претензий к~редакции журнала,  
несут авторы статей.



Направляя рукопись в редакцию, авторы сохраняют свои права на данную
рукопись и при этом передают учредителям и редколлегии журнала неисключительные права на
издание статьи на русском языке 
(или на языке статьи, если он отличен от рус\-ско\-го) и~на перевод ее на английский
язык, а~также на
ее распространение в России и за рубежом. 
Каждый автор должен представить в~редакцию подписанный 
с~его стороны <<Лицензионный договор о~передаче неисключительных прав 
на использование произведения>>, текст которого размещен по адресу 
{\sf http://www.ipiran.ru/publications/licence.doc}. 
Этот договор может быть пред\-став\-лен в~бумажном (в~2-х экз.)\ 
или в~электронном виде (отсканированная копия заполненного и~подписанного документа).




Редколлегия вправе запросить у авторов экспертное заключение о возможности
пуб\-ли\-ка\-ции пред\-став\-лен\-ной статьи в открытой печати.\\[-13.5pt]

\item К статье прилагаются данные автора (авторов) (см.\ п.~8). При наличии нескольких
авторов указывается фамилия автора, ответственного за переписку с редакцией.\\[-13.5pt]

\item Редакция журнала осуществляет экспертизу присланных статей в соответствии с
принятой в журнале процедурой рецензирования.

Возвращение рукописи на доработку не означает ее принятия к печати.

Доработанный вариант с ответом на замечания рецензента необходимо прислать в
редакцию.\\[-13.5pt]

\item Решение редколлегии о публикации статьи или ее отклонении сообщается авторам.

Редколлегия может также направить авторам текст рецензии на их статью. Дискуссия по
поводу отклоненных статей не ведется.\\[-13.5pt]

%\pagebreak

\item Редактура статей высылается авторам для просмотра. Замечания к редактуре должны
быть присланы авторами в кратчайшие сроки.\\[-13.5pt]

\item Рукопись предоставляется в электронном виде в форматах MS WORD (.doc или
.docx) или \LaTeX\  (.tex), дополнительно~--- в формате .pdf, на дискете, лазерном диске
или электронной почтой. Предоставление бумажной рукописи необязательно.\\[-13.5pt]

\item При подготовке рукописи в MS Word рекомендуется использовать следующие
настройки.

Параметры страницы:
формат~--- А4; ориентация~--- книжная; поля (см): внутри~--- 2,5, снаружи~--- 1,5,
сверху~--- 2, снизу~--- 2, от края до нижнего колонтитула~--- 1,3.

Основной текст: стиль~--- <<Обычный>>, шрифт~--- Times New Roman, размер~---
14~пунк\-тов, абзацный отступ~--- 0,5~см, 1,5~интервала, выравнивание~--- по ширине.

\pagebreak

\def\leftkol{Правила подготовки рукописей  для публикации в журнале
<<Информатика и её применения>>}

\def\rightkol{Правила подготовки рукописей  для публикации в журнале
<<Информатика и её применения>>}



Рекомендуемый объем рукописи~--- не свыше 10~страниц указанного формата.
При превышении указанного объема редколлегия вправе потребовать от 
автора сокращения объема рукописи.


Сокращения слов, помимо стандартных, не допускаются. Допускается минимальное
количество аббревиатур.


Все страницы рукописи нумеруются.

Шаблоны оформления представлены в интернете:

\noindent
 {\sf
http://www.ipiran.ru/journal/template\_iiep\_ssi\_2024.zip}\\[-14pt]

\item Статья должна содержать следующую информацию на {\bfseries\textit{русском и
английском языках}}:\\[-16pt]

\begin{itemize}
\item название статьи;\\[-15pt]
\item Ф.И.О.\ авторов, на английском можно только имя и фамилию;\\[-15pt]
\item место работы, с указанием почтового адреса организации и электронного адреса каждого
автора;\\[-15pt]
\item сведения об авторах, в соответствии с форматом, образцы которого
представлены на страницах:



\def\leftfootline{\small{\textbf{\thepage}
\hfill ИНФОРМАТИКА И ЕЁ ПРИМЕНЕНИЯ\ \ \ том\ 18\ \ \ выпуск\ 3\ \ \ 2024}
}%
 \def\rightfootline{\small{ИНФОРМАТИКА И ЕЁ ПРИМЕНЕНИЯ\ \ \ том\ 18\ \ \ выпуск\ 3\ \ \ 2024
\hfill \textbf{\thepage}}}



{\sf http://www.ipiran.ru/journal/issues/2013\_07\_01/authors.asp} и

{\sf http://www.ipiran.ru/journal/issues/2013\_07\_01\_eng/authors.asp};
\item аннотация (не менее 100~слов на каждом из языков). Аннотация~--- это краткое
резюме работы, которое может публиковаться отдельно. Она является основным
источником информации в~ин\-фор\-ма\-ци\-он\-ных системах и базах данных. Английская
аннотация должна быть оригинальной, может не быть дословным переводом русского
текста и должна быть написана хорошим английским языком. В~аннотации не должно
быть ссылок на литературу и, по возможности, формул;\\[-15pt]
\item ключевые слова~--- желательно из принятых в мировой
на\-уч\-но-тех\-ни\-че\-ской литературе тематических тезаурусов. Предложения не
могут быть ключевыми словами;\\[-15pt]
\item источники финансирования работы (ссылки на гранты, проекты,
поддерживающие организации и~т.\,п.).
\end{itemize}



%\pagebreak

\item  Требования к спискам литературы.\\[-14pt]

Ссылки на литературу в тексте статьи нумеруются (в квадратных скобках) и
располагаются в каждом из списков литературы в порядке  первых упоминаний. Если источник имеет DOI и/или EDN,
то их необходимо указывать.

Списки литературы представляются в двух вариантах:\\[-14pt]


\noindent
\begin{enumerate}[(1)]
\item \textbf{Список литературы к русскоязычной части}. Русские и английские
работы~---  на языке и в алфавите оригинала;\\[-14.5pt]
\item  \textbf{References}. Русские работы и работы на других языках~--- в латинской
транслитерации с переводом на английский язык; английские работы и работы на других
языках~--- на языке оригинала.
\end{enumerate}

Необходимо для составления списка ``References'' пользоваться размещенной на сайте
{\sf http://www. translit.net/ru/bgn/} бесплатной программой транслитерации русского
 текста в~латиницу. %, при этом в~за\-клад\-ке <<варианты\ldots>> следует выбратьопцию BGN.

Список литературы ``References'' приводится полностью отдельным блоком, повторяя все
позиции из списка литературы к русскоязычной части, независимо от того, имеются или
нет в нем иностранные источники. Если в списке литературы к русскоязычной части есть
ссылки на иностранные публикации, набранные латиницей, они полностью повторяются в
списке ``References''.

Ниже приведены примеры ссылок на различные виды публикаций в списке ``References''.

\def\leftfootline{\small{\textbf{\thepage}
\hfill ИНФОРМАТИКА И ЕЁ ПРИМЕНЕНИЯ\ \ \ том\ 18\ \ \ выпуск\ 3\ \ \ 2024}
}%
 \def\rightfootline{\small{ИНФОРМАТИКА И ЕЁ ПРИМЕНЕНИЯ\ \ \ том\ 18\ \ \ выпуск\ 3\ \ \ 2024
\hfill \textbf{\thepage}}}

{\small

\noindent
\textbf{Описание статьи из журнала:}

\Aue{Zagurenko, A.\,G., V.\,A.~Korotovskikh, A.\,A.~Kolesnikov, A.\,V.~Timonov, and D.\,V.~Kardymon}. 2008.
Tekhniko-ekonomicheskaya optimizatsiya dizayna gidrorazryva plasta [Technical and
economic optimization of the design
of hydraulic fracturing]. \textit{Neftyanoe hozyaystvo} [\textit{Oil Industry}] 11:54--57.

\Aue{Zhang, Z., and D.~Zhu}. 2008. Experimental research on the localized
electrochemical micromachining. \textit{Russ. J.~Electrochem.}  44(8):926--930.
{\sf doi:10.1134/S1023193508080077}.

\noindent
\textbf{Описание статьи из электронного журнала:}

\Aue{Swaminathan, V., E.~Lepkoswka-White, and B.\,P.~Rao}. 1999. Browsers or buyers in cyberspace? An
investigation of electronic factors influencing electronic exchange. \textit{JCMC}
5(2). Available at: {\sf http://www.ascusc.org/jcmc/vol5/issue2/} (accessed April~28, 2011).

\def\leftkol{Правила подготовки рукописей  для публикации в журнале
<<Информатика и её применения>>}

\def\rightkol{Правила подготовки рукописей  для публикации в журнале
<<Информатика и её применения>>}


\noindent
\textbf{Описание статьи из продолжающегося издания (сборника трудов):}

\Aue{Astakhov, M.\,V., and T.\,V.~Tagantsev}. 2006. Eksperimental'noe
issledovanie prochnosti soedineniy ``stal'--kompozit'' [Experimental study of
the strength of joints ``steel--composite'']. \textit{Trudy MGTU
``Matematicheskoe modelirovanie slozhnykh tekh\-ni\-che\-skikh sistem''}
[\textit{Bauman MSTU ``Mathematical Modeling of Complex Technical
Systems'' Proceedings}]. 593:125--130.


\pagebreak



\noindent
\textbf{Описание материалов конференций:}

\Aue{Usmanov, T.\,S., A.\,A.~Gusmanov, I.\,Z.~Mullagalin, R.\,Ju.~Muhametshina, A.\,N.~Chervyakova, and
A.\,V.~Sveshnikov}. 2007. Osobennosti proektirovaniya razrabotki mestorozhdeniy
s primeneniem gidrorazryva
plasta [Features of the design of field development with the use of hydraulic fracturing].
\textit{Trudy 6-go
Mezhdu\-na\-rod\-no\-go Simpoziuma ``Novye resursosberegayushchie tekhnologii nedropol'zovaniya i povysheniya
neftegazootdachi''} [\textit{6th  Symposium (International) ``New Energy Saving Subsoil Technologies and
the Increasing of the Oil and Gas Impact'' Proceedings}]. Moscow. 267--272.



\def\leftfootline{\small{\textbf{\thepage}
\hfill ИНФОРМАТИКА И ЕЁ ПРИМЕНЕНИЯ\ \ \ том\ 18\ \ \ выпуск\ 3\ \ \ 2024}
}%
 \def\rightfootline{\small{ИНФОРМАТИКА И ЕЁ ПРИМЕНЕНИЯ\ \ \ том\ 18\ \ \ выпуск\ 3\ \ \ 2024
\hfill \textbf{\thepage}}}



\noindent
\textbf{Описание книги (монографии, сборники):}



Lindorf, L.\,S., and L.\,G.~Mamikoniants, eds. 1972.
\textit{Ekspluatatsiya turbogeneratorov s neposredstvennym
okhlazhdeniem} [\textit{Operation of turbine generators with direct cooling}].
Moscow: Energy Publs. 352~p.


\Aue{Latyshev, V.\,N.} 2009. \textit{Tribologiya rezaniya. Kn.~1: Friktsionnye protsessy
pri rezanii metallov}
[\textit{Tribology of cutting. Vol.~1: Frictional processes in metal cutting}]. Ivanovo: Ivanovskii
State Univ. 108~p.

\def\leftkol{Правила подготовки рукописей  для публикации в журнале
<<Информатика и её применения>>}

\def\rightkol{Правила подготовки рукописей  для публикации в журнале
<<Информатика и её применения>>}

\noindent
\textbf{Описание переводной книги}
(в списке литературы к русскоязычной части необходимо указать:~/ Пер.\ с англ.~---
после названия книги, а в конце ссылки указать оригинал книги в круглых скобках):
\begin{enumerate}[1.]
\item  В русскоязычной части:

\def\leftfootline{\small{\textbf{\thepage}
\hfill ИНФОРМАТИКА И ЕЁ ПРИМЕНЕНИЯ\ \ \ том\ 18\ \ \ выпуск\ 3\ \ \ 2024}
}%
 \def\rightfootline{\small{ИНФОРМАТИКА И ЕЁ ПРИМЕНЕНИЯ\ \ \ том\ 18\ \ \ выпуск\ 3\ \ \ 2024
\hfill \textbf{\thepage}}}

\Au{Тимошенко С.\,П., Янг Д.\,Х., Уивер~У.}
Колебания в инженерном деле~/ Пер.\ с англ.~--- М.: Машиностроение, 1985. 472~с.
(\Au{Timoshenko~S.\,P., Young~D.\,H., Weaver~W.}
Vibration problems in engineering.~--- 4th ed.~--- New York, NY, USA: Wiley, 1974. 521~p.)\\[-13.5pt]
\item  В англоязычной части:

\Aue{Timoshenko, S.\,P., D.\,H.~Young, and W.~Weaver}.
1974. \textit{Vibration problems in engineering}. 4th ed. New York: 
Wiley. 521~p.
\end{enumerate}

\vspace*{-3pt}


\noindent
\textbf{Описание неопубликованного документа:}


\Aue{Latypov, A.\,R., M.\,M.~Khasanov, and V.\,A.~Baikov}.
2004 (unpubl.). Geologiya i~dobycha (NGT GiD) [Geology and production (NGT GiD)]. Certificate on official registration of the computer program
No.\,2004611198. 

\noindent
\textbf{Описание интернет-ресурса:}


Pravila tsitirovaniya istochnikov [Rules for the citing of sources]. Available at: {\sf
http://www.scribd.com/doc/1034528/} (accessed February~7, 2011).

%\pagebreak

\noindent
\textbf{Описание диссертации или автореферата диссертации:}

\Aue{Semenov, V.\,I.}
2003. Matematicheskoe modelirovanie plazmy v sisteme kompaktnyy tor [Mathematical
modeling of the plasma in the compact torus].  Moscow.  D.Sc.\ Diss. 272~p.

\Aue{Kozhunova, O.\,S.} 2009. Tekhnologiya razrabotki semanticheskogo
slovarya informatsionnogo monitoringa [Technology of development of
semantic dictionary of information monitoring system].  Moscow: IPI RAN. PhD Thesis. 23~p.


\noindent
\textbf{Описание ГОСТа:}

GOST 8.586.5-2005. 2007. Metodika vypolneniya izmereniy. Izmerenie raskhoda i~kolichestva zhidkostey i~gazov
s~pomoshch'yu standartnykh suzhayushchikh ustroystv [Method of measurement.
Measurement of flow rate and volume of liquids and gases by means of orifice devices]. Moscow:
Standardinform  Publs. 10~p.

\noindent
\textbf{Описание патента:}

\Aue{Bolshakov, M.\,V., A.\,V.~Kulakov, A.\,N.~Lavrenov, and M.\,V.~Palkin}.
2006. Sposob orientirovaniya po krenu letatel'nogo
apparata s opti\-che\-skoy golovkoy
samonavedeniya [The way to orient on the roll of aircraft with optical homing head].
Patent RF No.\,2280590.
}

\item Присланные в редакцию материалы авторам не возвращаются.\\[-13.5pt]

\item При отправке файлов по электронной почте просим придерживаться следующих
правил:
\begin{itemize}
\item указывать в поле subject (тема) название журнала и фамилию автора;\\[-13.5pt]
\item указывать в тексте письма название статьи, авторов и~журнал, в~который направляется статья;\\[-13.5pt]
\item использовать attach (присоединение);\\[-13.5pt]
\item в состав электронной версии статьи должны входить: файл, содержащий текст
статьи, и файл(ы), содержащий(е) иллюстрации.\\[-13.5pt]
\end{itemize}

\item Журнал <<Информатика и её применения>> является некоммерческим изданием.
Плата за публикацию не взимается, гонорар авторам не выплачивается.
\end{enumerate}



\def\leftfootline{\small{\textbf{\thepage}
\hfill ИНФОРМАТИКА И ЕЁ ПРИМЕНЕНИЯ\ \ \ том\ 18\ \ \ выпуск\ 3\ \ \ 2024}
}%
 \def\rightfootline{\small{ИНФОРМАТИКА И ЕЁ ПРИМЕНЕНИЯ\ \ \ том\ 18\ \ \ выпуск\ 3\ \ \ 2024
\hfill \textbf{\thepage}}}


\vspace*{-1mm}

\begin{center}

\textbf{Адрес редакции журнала <<Информатика и её применения>>:} \\




Москва 119333, ул.~Вавилова, д.~44, корп.~2, ФИЦ ИУ РАН\\[-10pt]

\

Тел.: +7\,(499)\,135-86-92\ \ Факс:  +7\,(495)\,930-45-05\\[-10pt]

 \

e-mail:   {\sf iiep@frccsc.ru} (Стригина Светлана Николаевна)\\[-10pt]

\

{\sf http://www.ipiran.ru/journal/issues/}
\end{center}
}


\def\leftkol{Правила подготовки рукописей  для публикации в журнале
<<Информатика и её применения>>}

\def\rightkol{Правила подготовки рукописей  для публикации в журнале
<<Информатика и её применения>>}


\def\leftfootline{\small{\textbf{\thepage}
\hfill ИНФОРМАТИКА И ЕЁ ПРИМЕНЕНИЯ\ \ \ том\ 18\ \ \ выпуск\ 3\ \ \ 2024}
}%
 \def\rightfootline{\small{ИНФОРМАТИКА И ЕЁ ПРИМЕНЕНИЯ\ \ \ том\ 18\ \ \ выпуск\ 3\ \ \ 2024
\hfill \textbf{\thepage}}} 
\def\stat{podg-e}
{%\hrule\par
%\vskip 7pt % 7pt
\vspace*{-24pt}
\raggedleft\Large \bf%\baselineskip=3.2ex
Requirements for manuscripts submitted to Journal
``Informatics~and~Applications'' \vskip 8pt
    \hrule
    \par
\vskip 21pt plus 6pt minus 3pt }

\label{st\stat}

\def\tit{\ }

\def\aut{\ }
\def\auf{\ }

\def\leftkol{\ }

\def\rightkol{\ }
%Requirements for manuscripts submitted to Journal
%``Informatics~and~Applications''}

\titele{\tit}{\aut}{\auf}{\leftkol}{\rightkol}

\def\leftfootline{\small{\textbf{\thepage}
\hfill INFORMATIKA I EE PRIMENENIYA~--- INFORMATICS AND APPLICATIONS\ \ \ 2019\
\ \ volume~13\ \ \ issue\ 4}
}%
 \def\rightfootline{\small{INFORMATIKA I EE PRIMENENIYA~--- INFORMATICS AND APPLICATIONS\ \ \ 2019\ \ \ volume~13\ \ \ issue\ 4
\hfill \textbf{\thepage}}}

\vspace*{-60pt}

{\small

\noindent
Journal ``Informatics and Applications'' (Inform.\ Appl.)
publishes theoretical, review, and discussion
articles on the research and development in the
field of informatics and its applications.

The journal is published in Russian.
By a special decision of the editorial
board, some articles can be published in English.


The topics covered include the following areas:
\begin{itemize}
               \item
     theoretical fundamentals of informatics; \\[-14pt]
\item
mathematical methods for studying complex systems and processes; \\[-14pt]
\item
information systems and networks;\\[-14pt]
\item
information technologies; and \\[-14pt]
\item
architecture and software of computational complexes and networks. \\[-14pt]
\end{itemize}

\noindent
\begin{enumerate}[1.]
\item The Journal publishes original articles which have not been published before and are not
intended for simultaneous publication in other editions. An article submitted to the Journal must not violate the
Copyright law. Sending the manuscript to the Editorial Board, the authors retain all rights of the
owners of the manuscript and transfer the nonexclusive rights to publish the article in Russian
(or the language of the article, if not Russian) and its distribution in Russia and abroad to the
Founders and the Editorial Board. Authors should submit a letter to the Editorial Board in the
following form:

{\bfseries\textit{Agreement on the transfer of rights to publish:}}

``\textit{We, the undersigned authors of the manuscript ``\ldots'', pass to the
Founder and the Editorial Board of the Journal ``Informatics and Applications''
the nonexclusive right to publish the manuscript of the article in Russian (or
in English) in both print and electronic versions of the Journal. We affirm
that this publication does not violate the Copyright of other persons or
organizations.}

\textit{Author(s) signature(s): (name(s), address(es), date).}

This agreement should be submitted in paper form or in the form of a scanned copy (signed by
the authors).


%The Editorial Board has the right to request from the authors an official expert conclusion that
%the submitted article has no secret data prohibited for publication. \\[-13.5pt]
\item
A submitted article should be attached with \textbf{the data on the author(s)} (see item~8). If
there are several authors, the contact person should be indicated who is responsible for
correspondence with the Editorial Board and other authors about revisions and final approval
of the proofs.\\[-13.5pt]

\item The Editorial Board of the Journal examines the article according to the established
reviewing procedure. If the authors receive their article for correction after reviewing, it does not
mean that the article is approved for publication. The corrected article should be sent to the
Editorial Board for the subsequent review and approval.\\[-13.5pt]

\item The decision on the article publication or its rejection is communicated to the authors. The
Editorial Board may also send the reviews on the submitted articles to the authors. Any
discussion upon the rejected articles is not possible.\\[-13.5pt]

\item The edited articles will be sent to the authors for proofread. The comments of the authors
to the edited text of the article should be sent to the Editorial Board as soon as possible.\\[-13.5pt]

\item The manuscript of the article should be presented electronically in the MS WORD (.doc or
.docx) or \LaTeX\ (.tex) formats, and additionally in the .pdf format. All documents
 may be sent
by e-mail or provided on a CD or diskette. A~hard copy submission is not necessary.\\[-13.5pt]

\item The recommended typesetting instructions for manuscript.

Pages parameters: format A4, portrait orientation, document margins (cm): left~--- 2.5, right~---
1.5, above~--- 2.0, below~--- 2.0, footer 1.3.

Text: font~---Times New Roman, font size~--- 14, paragraph indent~--- 0.5, line spacing~--- 1.5,
justified alignment.

The recommended manuscript size: not more than 15~pages of the specified format.
If the specified size exceeded, the editorial board is entitled to require the author
to reduce the manuscript.

Use only standard abbreviations. Avoid  abbreviations in the title and
abstract. The full term for which an abbreviation stands should precede
its first use in the text unless it is a standard unit of measurement.

All pages of the manuscript should be numbered.

The templates for the manuscript typesetting are presented on site: {\sf
http://www.ipiran.ru/journal/template.doc}.\\[-13.5pt]


%\def\leftkol{Requirements for manuscripts submitted to Journal
%``Informatics~and~Applications''}

\item The articles should enclose data both in \textbf{Russian and English}:
\begin{itemize}
\item title;\\[-13.5pt]
\item author's name and surname;\\[-13.5pt]
\item affiliation~--- organization, its address with ZIP code, city, country, and
official e-mail address;\\[-13.5pt]
\item data on authors according to the format: (see site)

{\sf http://www.ipiran.ru/journal/issues/2013\_07\_01/authors.asp}  and

{\sf  http://www.ipiran.ru/journal/issues/2013\_07\_01\_eng/authors.asp};\\[-13.5pt]

\pagebreak

\def\leftfootline{\small{\textbf{\thepage}
\hfill INFORMATIKA I EE PRIMENENIYA~--- INFORMATICS AND APPLICATIONS\ \ \ 2019\
\ \ volume~13\ \ \ issue\ 4}
}%
 \def\rightfootline{\small{INFORMATIKA I EE PRIMENENIYA~--- INFORMATICS AND APPLICATIONS\ \ \ 2019\ \ \ volume~13\ \ \ issue\ 4
\hfill \textbf{\thepage}}}


%\def\leftkol{Requirements for manuscripts submitted to Journal
%``Informatics~and~Applications''}

%\def\rightkol{Requirements for manuscripts submitted to Journal
%``Informatics~and~Applications''}



\item abstract (not less than 100 words) both in Russian and in English. Abstract is a short
summary of the article that can be published separately. The abstract is the
main source of information on the article and it could be included in leading information
systems and data bases. The abstract in English has to be an original text and should
not be an exact translation of the Russian one. Good English is required.
In abstracts, avoid references and formulae;\\[-13.5pt]
\item indexing is performed on the basis of keywords. The use of keywords from the
internationally accepted thematic Thesauri is recommended.

%\def\leftkol{Requirements for manuscripts submitted to Journal
%``Informatics~and~Applications''}

%\def\rightkol{Requirements for manuscripts submitted to Journal
%``Informatics~and~Applications''}

Important! Keywords must not be sentences;
\item Acknowledgments.
\end{itemize}

\item References. Russian references have to be presented both in English translation and Latin
transliteration (refer {\sf http://www.translit.net/ru/bgn/}).

Please take into account the following examples of Russian references appearance:

\noindent
\textbf{Article in journal:}

\Aue{Zhang, Z., and D.~Zhu}. 2008. Experimental research on the localized electrochemical
micromachining.
\textit{Rus. J.~Electrochem.}  44(8):926--930. {\sf doi:10.1134/S1023193508080077}.


\noindent
\textbf{Journal article in electronic format:}

\Aue{Swaminathan, V., E.~Lepkoswka-White, and B.\,P.~Rao}. 1999. Browsers or buyers in
cyberspace? An
investigation of electronic factors influencing electronic exchange. \textit{JCMC}
5(2). Available at: {\sf http://www.ascusc.org/jcmc/vol5/issue2/} (accessed April~28, 2011).




\noindent
\textbf{Article from the continuing publication (collection of works, proceedings):}

\Aue{Astakhov, M.\,V., and T.\,V.~Tagantsev}. 2006. Eksperimental'noe
issledovanie prochnosti soedineniy ``stal'--kompozit'' [Experimental study of
the strength of joints ``steel--composite'']. \textit{Trudy MGTU
``Matematicheskoe modelirovanie slozhnykh tekh\-ni\-che\-skikh sistem''}
[\textit{Bauman MSTU ``Mathematical Modeling of Complex Technical
Systems'' Proceedings}]. 593:125--130.

\def\leftfootline{\small{\textbf{\thepage}
\hfill INFORMATIKA I EE PRIMENENIYA~--- INFORMATICS AND APPLICATIONS\ \ \ 2019\
\ \ volume~13\ \ \ issue\ 4}
}%
 \def\rightfootline{\small{INFORMATIKA I EE PRIMENENIYA~--- INFORMATICS AND APPLICATIONS\ \ \ 2019\ \ \ volume~13\ \ \ issue\ 4
\hfill \textbf{\thepage}}}

\def\leftkol{Requirements for manuscripts submitted to Journal
``Informatics~and~Applications''}

\def\rightkol{Requirements for manuscripts submitted to Journal
``Informatics~and~Applications''}

\noindent
\textbf{Conference proceedings:}

\Aue{Usmanov, T.\,S., A.\,A.~Gusmanov, I.\,Z.~Mullagalin, R.\,Ju.~Muhametshina,
A.\,N.~Chervyakova, and
A.\,V.~Sveshnikov}. 2007. Osobennosti proektirovaniya razrabotki mestorozhdeniy
s primeneniem gidrorazryva
plasta [Features of the design of field development with the use of hydraulic fracturing].
\textit{Trudy 6-go
Mezhdu\-na\-rod\-no\-go Simpoziuma ``Novye resursosberegayushchie tekhnologii
nedropol'zovaniya i povysheniya
neftegazootdachi''} [\textit{6th  Symposium (International) ``New Energy Saving Subsoil
Technologies and
the Increasing of the Oil and Gas Impact'' Proceedings}]. Moscow. 267--272.


\noindent
\textbf{Books and other monographs:}




Lindorf, L.\,S., and L.\,G.~Mamikoniants, eds. 1972.
\textit{Ekspluatatsiya turbogeneratorov s neposredstvennym
okhlazhdeniem} [\textit{Operation of turbine generators with direct cooling}].
Moscow: Energy Publs. 352~p.


%\Aue{Latyshev, V.\,N.} 2009. \textit{Tribologiya rezaniya. Kn.~1: Frikcionnye prosessy
%pri rezanii metallov}
%[\textit{Tribology of cutting. Vol.~1: Frictional processes in metal cutting}]. Ivanovo: Ivanovskii
%State Univ. 108~p.


%\noindent
%\textbf{Unpublished material:}

%\Aue{Latypov, A.\,R., M.\,M.~Khasanov, and V.\,A.~Baikov}.
%2004. Geology and production (NGT GiD). Certificate on official registration of the computer
%program
%No.\,2004611198. (In Russian, unpubl.)

%\noindent
%\textbf{Internet-source:}

%APA Style. 2011. Available at: {\sf http://www.apastyle.org/apa-style-help.aspx} (accessed
%February~5, 2011).

%Pravila citirovaniya istochnikov [Rules for the citing of sources]. Available at: {\sf
%http://www.scribd.com/doc/1034528/} (accessed February~7, 2011).


\noindent
\textbf{Dissertation and Thesis:}

%\Aue{Semenov, V.\,I.}
%2003. Matematicheskoe modelirovanie plazmy v sisteme kompaktnyy tor. [Mathematical
%modeling of the plasma in the compact torus]. D.Sc.\ Diss. Moscow. 272~p.

\Aue{Kozhunova, O.\,S.} 2009. Tekhnologiya razrabotki semanticheskogo
slovarya informatsionnogo monitoringa [Technology of development of
semantic dictionary of information monitoring system]. PhD Thesis. Moscow: IPI RAN. 23~p.


\noindent
\textbf{State standards and patents:}

GOST 8.586.5-2005. 2007. Metodika vypolneniya izmereniy. Izmerenie raskhoda i~kolichestva
zhidkostey i gazov 
s~pomoshch'yu standartnykh suzhayushchikh ustroystv [Method of measurement.
Measurement of flow rate and volume of liquids and gases by means of orifice devices]. M.:
Standardinform
Publs. 10~p.

%\noindent
%\textbf{Patent:}

\Aue{Bolshakov, M.\,V., A.\,V.~Kulakov, A.\,N.~Lavrenov, and M.\,V.~Palkin}.
2006. Sposob orientirovaniya po krenu letatel'nogo
apparata s opti\-che\-skoy golovkoy
samonavedeniya [The way to orient on the roll of aircraft with optical homing head].
Patent RF No.\,2280590.

References in Latin transcription are presented in the original language.

References in the text are numbered according to the order of their
first appearance; the number is
placed in square brackets. All items from the reference list should be
cited.\\[-13.5pt]

\item Manuscripts and additional materials are not returned to Authors by the Editorial Board.\\[-13.5pt]

\item Submissions of files by e-mail must include:\\[-13.5pt]
\begin{itemize}
\item   the journal title and author's name in the ``Subject'' field; \\[-13.5pt]
\item   an article and additional materials have to be attached using the ``attach'' function;\\[-13.5pt]
\item   an electronic version of the article should contain the file with the text and a separate file
with figures.\\[-13.5pt]
\end{itemize}

\item ``Informatics and Applications'' journal is not a profit publication. There are no
charges for the authors as well as there are no royalties.\\[-13.5pt]
\end{enumerate}

\def\leftfootline{\small{\textbf{\thepage}
\hfill INFORMATIKA I EE PRIMENENIYA~--- INFORMATICS AND APPLICATIONS\ \ \ 2019\
\ \ volume~13\ \ \ issue\ 4}
}%
 \def\rightfootline{\small{INFORMATIKA I EE PRIMENENIYA~--- INFORMATICS AND APPLICATIONS\ \ \ 2019\ \ \ volume~13\ \ \ issue\ 4
\hfill \textbf{\thepage}}}

\def\leftkol{Requirements for manuscripts submitted to Journal
``Informatics~and~Applications''}

\def\rightkol{Requirements for manuscripts submitted to Journal
``Informatics~and~Applications''}


%\vspace*{5mm}


\begin{center}
\textbf{Editorial Board address:} \\

%ABOUT AUTHORS



FRC CSC RAS, 44, block~2, Vavilov Str., Moscow 119333, Russia\\[-10pt]

\

Ph.: +7\,(499)\,135\,86\,92,\ \ Fax: +7\,(495)\,930\,45\,05\\[-10pt]

\

 e-mail: {\sf rust@ipiran.ru} (to Prof.\ Rustem Seyful-Mulyukov)\\[-10pt]

\

 {\sf http://www.ipiran.ru/english/journal.asp}
\end{center}
 }
%\thispagestyle{myheadings}

\def\leftkol{Requirements for manuscripts submitted to Journal
``Informatics~and~Applications''}

\def\rightkol{Requirements for manuscripts submitted to Journal
``Informatics~and~Applications''}

\def\leftfootline{\small{\textbf{\thepage}
\hfill INFORMATIKA I EE PRIMENENIYA~--- INFORMATICS AND APPLICATIONS\ \ \ 2019\
\ \ volume~13\ \ \ issue\ 4}
}%
 \def\rightfootline{\small{INFORMATIKA I EE PRIMENENIYA~--- INFORMATICS AND APPLICATIONS\ \ \ 2019\ \ \ volume~13\ \ \ issue\ 4
\hfill \textbf{\thepage}}}

 \label{end\stat}

\newpage

%\vspace*{-60pt} {\small
{\baselineskip=9.1pt
\section*{Правила подготовки рукописей статей для публикации в журнале
<<Информатика и её применения>>}

\thispagestyle{empty}

 Журнал <<Информатика и её применения>> публикует
теоретические, обзорные и дискуссионные статьи, посвященные научным
исследованиям и разработкам в области информатики и ее приложений. Журнал
издается на русском языке. По специальному решению редколлегии отдельные статьи,
в виде исключения, могут печататься на английском языке.
Тематика журнала охватывает следующие направления:
\begin{itemize}
\item теоретические основы информатики; %\\[-13.5pt]
\item математические методы исследования сложных систем и процессов; %\\[-13.5pt]
\item информационные системы и сети; %\\[-13.5pt]
\item информационные технологии; %\\[-13.5pt]
\item архитектура и программное
обеспечение вычислительных комплексов и сетей.
\end{itemize}
\begin{enumerate}
\item В журнале печатаются результаты, ранее не
опубликованные и не предназначенные к одновременной публикации в других
изданиях. Публикация не должна нарушать закон об авторских правах. Направляя
свою рукопись в редакцию, авторы автоматически передают учредителям и
редколлегии неисключительные права на издание данной статьи на русском языке и
на ее распространение в России и за рубежом. При этом за авторами сохраняются
все права как собственников данной рукописи. В связи с этим авторами должно
быть представлено в редакцию письмо в следующей форме:
Соглашение о передаче права на публикацию:

\textit{<<Мы, нижеподписавшиеся, авторы рукописи <<$\qquad\qquad$>>, передаем
учредителям и редколлегии журнала <<Информатика и её применения>>
неисключительное право опубликовать данную рукопись статьи на русском языке как
в печатной, так и в электронной версиях журнала. Мы подтверждаем, что данная
публикация не нарушает авторского права других лиц или организаций. Подписи
авторов: (ф.\,и.\,о., дата, адрес)>>.}

Указанное соглашение может быть представлено 
как в бумажном виде, так и в виде отсканированной копии (с подписями авторов).


Редколлегия вправе запросить у авторов экспертное заключение о возможности
опубликования представленной статьи в открытой печати. %\\[-13.5pt]
\item Статья
подписывается всеми авторами. На отдельном листе представляются данные автора
(или всех авторов): фамилия, полные имя и отчество, телефон, факс, e-mail,
почтовый адрес. Если работа выполнена несколькими авторами, указывается фамилия
одного из них, ответственного за переписку с редакцией. %\\[-13.5pt]
\item Редакция журнала
осуществляет самостоятельную экспертизу присланных статей. Возвращение рукописи
на доработку не означает, что статья уже принята к печати. Доработанный вариант
с ответом на замечания рецензента необходимо прислать в редакцию. %\\[-13.5pt]
\item Решение
редакционной коллегии о принятии статьи к печати или ее отклонении сообщается
авторам. Редколлегия не обязуется направлять рецензию авторам отклоненной
статьи. %\\[-13.5pt]
\item Корректура статей высылается авторам для просмотра. Редакция
просит авторов присылать свои замечания в кратчайшие сроки. %\\[-13.5pt]
\item При
подготовке рукописи в MS Word рекомендуется использовать следующие настройки.
Параметры страницы: формат~--- А4; ориентация~--- книжная; поля (см): внутри~---
2,5, снаружи~--- 1,5, сверху~--- 2, снизу~--- 2, от края до нижнего
колонтитула~--- 1,3. Основной текст: стиль~--- <<Обычный>>: шрифт Times New
Roman, размер 14~пунктов, абзацный отступ~--- 0,5~см, 1,5 интервала,
выравнивание~--- по ширине. Рекомендуемый объем рукописи~--- не свыше
25~страниц указанного формата. Ознакомиться с шаблонами, содержащими примеры
оформления, можно по адресу в Интернете:
\textsf{http://www.ipiran.ru/journal/template.doc}.
\item К рукописи, предоставляемой в 2-х
экземплярах, обязательно прилагается электронная версия статьи (как правило, в
форматах MS WORD (.doc) или \LaTeX\ (.tex), а также~--- дополнительно~--- в
формате .pdf) на дискете, лазерном диске или по электронной почте. Сокращения
слов, кроме стандартных, не применяются. Все страницы рукописи должны быть
пронумерованы. %\\[-13.5pt]
\item Статья должна содержать следующую информацию на русском и
английском языках: название, Ф.И.О. авторов, места работы авторов и их
электронные адреса, подробные сведения об авторах, оформленные в соответствии с форматом, 
определяемым файлами {\sf http://www.ipiran.ru/journal/issues/2011\_05\_01/authors.asp} и 
{\sf http://www.ipiran.ru/journal/issues/2011\_01\_eng/authors.asp},
аннотация (не более 100~слов), ключевые слова. Ссылки на
литературу в тексте статьи нумеруются (в квадратных скобках) и располагаются в
порядке их первого упоминания. В~списке литературы не должно быть позиций, на которые нет ссылки в тексте статьи.
Все фамилии авторов, заглавия статей, названия
книг, конференций и~т.\,п.\ даются на языке оригинала, если этот язык
использует кириллический или латинский алфавит. %\\[-13.5pt]
\item Присланные в редакцию материалы авторам не возвращаются.
\item При отправке файлов по электронной
почте просим придерживаться следующих правил:
\begin{itemize}
\item указывать в поле subject (тема) название журнала и фамилию автора; %\\[-13.5pt]
\item использовать attach (присоединение); %\\[-13.5pt]
\item в случае больших объемов информации возможно
использование общеизвестных архиваторов (ZIP, RAR); %\\[-13.5pt]
\item в состав электронной версии статьи должны входить: файл, содержащий текст статьи, и файл(ы),
содержащий(е) иллюстрации. %\\[-13.5pt]
\end{itemize}
\item Журнал <<Информатика и её применения>> является некоммерческим изданием. 
Плата за публикацию с авторов не взимается, гонорар авторам не выплачивается.
\end{enumerate}
\thispagestyle{empty}
\textbf{Адрес редакции:} Москва 119333,
ул.~Вавилова, д.~44, корп.~2, ИПИ РАН\\
\hphantom{\textbf{Адрес редакции:} }Тел.: +7 (499) 135-86-92\ \
Факс:  +7 (495) 930-45-05\ \  E-mail:   rust@ipiran.ru }
}

%\include{ipi-ind}

%\tableofcontents

\end{document}


%\tableofcontents

%\end{document}





%\def\stat{cont}
{%\hrule\par
%\vskip 7pt % 7pt
\raggedleft\Large \bf%\baselineskip=3.2ex
А\,В\,Т\,О\,Р\,С\,К\,И\,Й\ \ У\,К\,А\,З\,А\,Т\,Е\,Л\,Ь\ \ З\,А\ \ 2\,0\,0\,7 г. \vskip 17pt
    \hrule
    \par
\vskip 21pt plus 6pt minus 3pt }

\label{st\stat}

\def\tit{\ }

\def\aut{\ }
\def\auf{\ }

\def\leftkol{\ } % ENGLISH ABSTRACTS}

\def\rightkol{\ } %ENGLISH ABSTRACTS}

\titele{\tit}{\aut}{\auf}{\leftkol}{\rightkol}


\contentsline {chapter}{\ }{Выпуск \quad Стр.} 
\contentsline {section}{\textbf{Батракова Д.\,А., Королев В.\,Ю., Шоргин С.\,Я.}\ \ Новый метод вероятностно-ста\-ти\-сти\-че\-ско\-го анализа информационных потоков в\nobreakspace {}телекоммуникационных сетях}{\qquad 1 \qquad 40} 
\contentsline {section}{\textbf{Борисов А.\,В.}\ \ Байесовское оценивание в системах наблюдения с\nobreakspace {}марковскими скачкообразными процессами: игровой подход}{\qquad 2 \qquad 65}
\contentsline {section}{\textbf{Босов А.\,В., Иванов А.\,В.}\ \ Программная инфраструктура информационного Web-пор\-тала}{\qquad 2 \qquad 50}
\contentsline {section}{\textbf{Захаров В.\,Н., Калиниченко Л.\,А., Соколов И.\,А., Ступников С.\,А.}\ \ Конструирование канонических информационных моделей для интегрированных информационных систем}{\qquad 2 \qquad 15}
\contentsline {section}{\textbf{Захаров В.\,Н., Козмидиади В.\,А.}\ \ Средства обеспечения отказоустойчивости при\-ло\-жений}{\qquad 1 \qquad 14} 
\contentsline {section}{\textbf{Иванов А.\,В.}\ \ см. Босов А.\,В.\hfill\hfill\hfill\hfill\hfill\hfill\hfill\hfill\hfill\hfill\hfill\hfill\hfill\hfill\hfill\hfill\hfill\hfill\hfill\hfill\hfill\hfill\hfill\hfill\hfill\hfill\hfill\hfill\hfill\hfill\hfill\hfill\hfill\hfill\hfill}{\ }
\contentsline {section}{\textbf{Ильин В.\,Д., Соколов И.\,А.}\ \ Символьная модель системы знаний информатики в\nobreakspace {}че\-ло\-ве\-ко-автоматной среде}{\qquad 1 \qquad 66} 
\contentsline {section}{\textbf{Калиниченко Л.\,А.}\ \ см. Захаров В.\,Н.\hfill\hfill\hfill\hfill\hfill\hfill\hfill\hfill\hfill\hfill\hfill\hfill\hfill\hfill\hfill\hfill\hfill\hfill\hfill\hfill\hfill\hfill\hfill\hfill\hfill\hfill\hfill\hfill\hfill\hfill\hfill\hfill\hfill\hfill\hfill}{\ }
\contentsline {section}{\textbf{Козеренко Е.\,Б.}\ \ Лингвистическое моделирование для систем машинного перевода и обработки знаний}{\qquad 1 \qquad 54} 
\contentsline {section}{\textbf{Козмидиади В.\,А.}\ \ см. Захаров В.\,Н.\hfill\hfill\hfill\hfill\hfill\hfill\hfill\hfill\hfill\hfill\hfill\hfill\hfill\hfill\hfill\hfill\hfill\hfill\hfill\hfill\hfill\hfill\hfill\hfill\hfill\hfill\hfill\hfill\hfill\hfill\hfill\hfill\hfill\hfill\hfill }{\ } 
\contentsline {section}{\textbf{Королев В.\,Ю.}\ \ см. Батракова Д.\,А.\hfill\hfill\hfill\hfill\hfill\hfill\hfill\hfill\hfill\hfill\hfill\hfill\hfill\hfill\hfill\hfill\hfill\hfill\hfill\hfill\hfill\hfill\hfill\hfill\hfill\hfill\hfill\hfill\hfill\hfill\hfill\hfill\hfill\hfill\hfill}{\ } 
\contentsline {section}{\textbf{Кудрявцев А.\,А., Шоргин С.\,Я.}\ \ Байесовский подход к\nobreakspace {}анализу систем массового обслуживания и\nobreakspace {}показателей надежности}{\qquad 2 \qquad 76}
\contentsline {section}{\textbf{Печинкин А.\,В., Соколов И.\,А., Чаплыгин В.\,В.}\ \ Многолинейная система массового обслуживания с конечным накопителем и ненадежными приборами}{\qquad 1 \qquad 27} 
\contentsline {section}{\textbf{Печинкин А.\,В., Соколов И.\,А., Чаплыгин В.\,В.}\ \ Стационарные характеристики многолинейной\nobreakspace {}системы массового обслуживания с\nobreakspace {}одновременными отказами приборов}{\qquad 2 \qquad 39}
\contentsline {section}{\textbf{Синицын И.\,Н.}\ \ Корреляционные методы построения аналитических информационных моделей флуктуаций полюса Земли по априорным данным}{\qquad 2 \qquad \hphantom{9}2}
\contentsline {section}{\textbf{Синицын И.\,Н.}\ \ Развитие теории фильтров Пугачева для оперативной обработки информации в стохастических системах}{{\qquad 1 \qquad \hphantom{9}3}} 
\contentsline {section}{\textbf{Соколов И.\,А.}\ \ см. Захаров В.\,Н.\hfill\hfill\hfill\hfill\hfill\hfill\hfill\hfill\hfill\hfill\hfill\hfill\hfill\hfill\hfill\hfill\hfill\hfill\hfill\hfill\hfill\hfill\hfill\hfill\hfill\hfill\hfill\hfill\hfill\hfill\hfill\hfill\hfill\hfill\hfill}{\ }
\contentsline {section}{\textbf{Соколов И.\,А.}\ \ см. Ильин В.\,Д.\hfill\hfill\hfill\hfill\hfill\hfill\hfill\hfill\hfill\hfill\hfill\hfill\hfill\hfill\hfill\hfill\hfill\hfill\hfill\hfill\hfill\hfill\hfill\hfill\hfill\hfill\hfill\hfill\hfill\hfill\hfill\hfill\hfill\hfill\hfill}{\ } 
\contentsline {section}{\textbf{Соколов И.\,А.}\ \ см. Печинкин А.\,В.\hfill\hfill\hfill\hfill\hfill\hfill\hfill\hfill\hfill\hfill\hfill\hfill\hfill\hfill\hfill\hfill\hfill\hfill\hfill\hfill\hfill\hfill\hfill\hfill\hfill\hfill\hfill\hfill\hfill\hfill\hfill\hfill\hfill\hfill\hfill}{\ } 
\contentsline {section}{\textbf{Соколов И.\,А.}\ \ см. Печинкин А.\,В.\hfill\hfill\hfill\hfill\hfill\hfill\hfill\hfill\hfill\hfill\hfill\hfill\hfill\hfill\hfill\hfill\hfill\hfill\hfill\hfill\hfill\hfill\hfill\hfill\hfill\hfill\hfill\hfill\hfill\hfill\hfill\hfill\hfill\hfill\hfill}{\ }
\contentsline {section}{\textbf{Ступников С.\,А.}\ \ см. Захаров В.\,Н.\hfill\hfill\hfill\hfill\hfill\hfill\hfill\hfill\hfill\hfill\hfill\hfill\hfill\hfill\hfill\hfill\hfill\hfill\hfill\hfill\hfill\hfill\hfill\hfill\hfill\hfill\hfill\hfill\hfill\hfill\hfill\hfill\hfill\hfill\hfill}{\ }
\contentsline {section}{\textbf{Чаплыгин В.\,В.}\ \ см. Печинкин А.\,В.\hfill\hfill\hfill\hfill\hfill\hfill\hfill\hfill\hfill\hfill\hfill\hfill\hfill\hfill\hfill\hfill\hfill\hfill\hfill\hfill\hfill\hfill\hfill\hfill\hfill\hfill\hfill\hfill\hfill\hfill\hfill\hfill\hfill\hfill\hfill}{\ } 
\contentsline {section}{\textbf{Чаплыгин В.\,В.}\ \ см. Печинкин А.\,В.\hfill\hfill\hfill\hfill\hfill\hfill\hfill\hfill\hfill\hfill\hfill\hfill\hfill\hfill\hfill\hfill\hfill\hfill\hfill\hfill\hfill\hfill\hfill\hfill\hfill\hfill\hfill\hfill\hfill\hfill\hfill\hfill\hfill\hfill\hfill}{\ }
\contentsline {section}{\textbf{Шоргин С.\,Я.}\ \ см. Батракова Д.\,А.\hfill\hfill\hfill\hfill\hfill\hfill\hfill\hfill\hfill\hfill\hfill\hfill\hfill\hfill\hfill\hfill\hfill\hfill\hfill\hfill\hfill\hfill\hfill\hfill\hfill\hfill\hfill\hfill\hfill\hfill\hfill\hfill\hfill\hfill\hfill}{\ } 
\contentsline {section}{\textbf{Шоргин С.\,Я.}\ \ см. Кудрявцев А.\,А.\hfill\hfill\hfill\hfill\hfill\hfill\hfill\hfill\hfill\hfill\hfill\hfill\hfill\hfill\hfill\hfill\hfill\hfill\hfill\hfill\hfill\hfill\hfill\hfill\hfill\hfill\hfill\hfill\hfill\hfill\hfill\hfill\hfill\hfill\hfill}{\ }
%\thispagestyle{myheadings}
\def\leftfootline{\small{\textbf{\thepage}
\hfill ИНФОРМАТИКА И ЕЁ ПРИМЕНЕНИЯ\ \ \ том~1\ \ \ выпуск~2\ \ \ 2007}
}%
 \def\rightfootline{\small{ИНФОРМАТИКА И ЕЁ ПРИМЕНЕНИЯ\ \ \ том~1\ \ \ выпуск~2\ \ \ 2007
 \hfill \textbf{\thepage}}}
 \label{end\stat}

%\def\stat{cont-e}
{%\hrule\par
%\vskip 7pt % 7pt
\raggedleft\Large \bf%\baselineskip=3.2ex
2\,0\,0\,7\ \ A\,U\,T\,H\,O\,R\ \ I\,N\,D\,E\,X \vskip 17pt
    \hrule
    \par
\vskip 21pt plus 6pt minus 3pt }

\label{st\stat}

\def\tit{\ }

\def\aut{\ }
\def\auf{\ }

\def\leftkol{\ } % ENGLISH ABSTRACTS}

\def\rightkol{\ } %ENGLISH ABSTRACTS}

\titele{\tit}{\aut}{\auf}{\leftkol}{\rightkol}


\contentsline {chapter}{\ }{Issue \quad Page} 
\contentsline {subsection}{\textbf{Batrakova D.\,A., Korolev V.\,Yu., Shorgin S.\,Ya.}\ \ A New Method for the Probabilistic and Statistical Analysis of Information Flows in Telecommunication Networks}{\qquad 1 \qquad 40} 
\contentsline {subsection}{\textbf{Borisov A.\,V.}\ \ Bayesian Estimation in\nobreakspace {}Observation Systems with\nobreakspace {}Markov Jump Processes: Game-Theoretic Approach}{\qquad 2 \qquad 65} 
\contentsline {subsection}{\textbf{Bosov A.\,V., Ivanov A.\,V.}\ \ Linguistic Simulation for Machine Translation and Knowledge Management Systems}{\qquad 2 \qquad 50} 
\contentsline {subsection}{\textbf{Chaplygin V.\,V.} see Pechinkin A.\,V.\hfill\hfill\hfill\hfill\hfill\hfill\hfill\hfill\hfill\hfill\hfill\hfill\hfill\hfill\hfill\hfill\hfill\hfill\hfill\hfill\hfill\hfill\hfill\hfill\hfill\hfill\hfill\hfill\hfill\hfill\hfill\hfill\hfill\hfill\hfill}{\ }
\contentsline {subsection}{\textbf{Chaplygin V.\,V.} see Pechinkin A.\,V.\hfill\hfill\hfill\hfill\hfill\hfill\hfill\hfill\hfill\hfill\hfill\hfill\hfill\hfill\hfill\hfill\hfill\hfill\hfill\hfill\hfill\hfill\hfill\hfill\hfill\hfill\hfill\hfill\hfill\hfill\hfill\hfill\hfill\hfill\hfill}{\ }
\contentsline {subsection}{\textbf{Ilyin V.\,D., Sokolov I.\,A.}\ \ The Symbol Model of Informatics Knowledge System in Human-Automaton Environment}{\qquad 1 \qquad 66} 
\contentsline {subsection}{\textbf{Ivanov A.\,V.} see Bosov A.\,V.\hfill\hfill\hfill\hfill\hfill\hfill\hfill\hfill\hfill\hfill\hfill\hfill\hfill\hfill\hfill\hfill\hfill\hfill\hfill\hfill\hfill\hfill\hfill\hfill\hfill\hfill\hfill\hfill\hfill\hfill\hfill\hfill\hfill\hfill\hfill}{\ }
\contentsline {subsection}{\textbf{Kalinichenko L.\,A.} see Zakharov V.\,N.\hfill\hfill\hfill\hfill\hfill\hfill\hfill\hfill\hfill\hfill\hfill\hfill\hfill\hfill\hfill\hfill\hfill\hfill\hfill\hfill\hfill\hfill\hfill\hfill\hfill\hfill\hfill\hfill\hfill\hfill\hfill\hfill\hfill\hfill\hfill}{\ }
\contentsline {subsection}{\textbf{Korolev V.\,Yu.} see Batrakova D.\,A.\hfill\hfill\hfill\hfill\hfill\hfill\hfill\hfill\hfill\hfill\hfill\hfill\hfill\hfill\hfill\hfill\hfill\hfill\hfill\hfill\hfill\hfill\hfill\hfill\hfill\hfill\hfill\hfill\hfill\hfill\hfill\hfill\hfill\hfill\hfill}{\ }
\contentsline {subsection}{\textbf{Kozerenko E.\,B.}\ \ Linguistic Simulation for Machine Translation and Knowledge Management Systems}{\qquad 1 \qquad 54} 
\contentsline {subsection}{\textbf{Kozmidiady V.\,A.} see Zakharov V.\,N.\hfill\hfill\hfill\hfill\hfill\hfill\hfill\hfill\hfill\hfill\hfill\hfill\hfill\hfill\hfill\hfill\hfill\hfill\hfill\hfill\hfill\hfill\hfill\hfill\hfill\hfill\hfill\hfill\hfill\hfill\hfill\hfill\hfill\hfill\hfill}{\ }
\contentsline {subsection}{\textbf{Kudryavtsev A.\,A., Shorgin S.\,Ya.}\ \ Bayesian Approach to Queueing Systems and Reliability Characteristics}{\qquad 2 \qquad 76} 
\contentsline {subsection}{\textbf{Pechinkin A.\,V., Sokolov I.\,A., Chaplygin V.\,V.}\ \ Multichannel Queuing System with Finite Buffer and Unreliable Servers}{\qquad 1 \qquad 27} 
\contentsline {subsection}{\textbf{Pechinkin A.\,V., Sokolov I.\,A., Chaplygin V.\,V.}\ \ Stationary Characteristics of a Multichannel Queueing System with\nobreakspace {}Simultaneous Refusals of Servers}{\qquad 2 \qquad 39} 
\contentsline {subsection}{\textbf{Shorgin S.\,Ya.} see Batrakova D.\,A.\hfill\hfill\hfill\hfill\hfill\hfill\hfill\hfill\hfill\hfill\hfill\hfill\hfill\hfill\hfill\hfill\hfill\hfill\hfill\hfill\hfill\hfill\hfill\hfill\hfill\hfill\hfill\hfill\hfill\hfill\hfill\hfill\hfill\hfill\hfill}{\ }
\contentsline {subsection}{\textbf{Shorgin S.\,Ya.} see Kudryavtsev A.\,A.\hfill\hfill\hfill\hfill\hfill\hfill\hfill\hfill\hfill\hfill\hfill\hfill\hfill\hfill\hfill\hfill\hfill\hfill\hfill\hfill\hfill\hfill\hfill\hfill\hfill\hfill\hfill\hfill\hfill\hfill\hfill\hfill\hfill\hfill\hfill}{\ }
\contentsline {subsection}{\textbf{Sinitsyn I.\,N.}\ \ Correlational Methods for Analytical Informational Models of the Earth Pole Fluctuations Design Based on a priori Data}{\qquad 2 \qquad \hphantom{9}2}
\contentsline {subsection}{\textbf{Sinitsyn I.\,N.}\ \ Development of Pugachev Filtering for Stochastic Systems}{\qquad 1 \qquad \hphantom{9}3}
\contentsline {subsection}{\textbf{Sokolov I.\,A.} see Ilyin V.\,D.\hfill\hfill\hfill\hfill\hfill\hfill\hfill\hfill\hfill\hfill\hfill\hfill\hfill\hfill\hfill\hfill\hfill\hfill\hfill\hfill\hfill\hfill\hfill\hfill\hfill\hfill\hfill\hfill\hfill\hfill\hfill\hfill\hfill\hfill\hfill}{\ }
\contentsline {subsection}{\textbf{Sokolov I.\,A.} see Pechinkin A.\,V.\hfill\hfill\hfill\hfill\hfill\hfill\hfill\hfill\hfill\hfill\hfill\hfill\hfill\hfill\hfill\hfill\hfill\hfill\hfill\hfill\hfill\hfill\hfill\hfill\hfill\hfill\hfill\hfill\hfill\hfill\hfill\hfill\hfill\hfill\hfill}{\ }
\contentsline {subsection}{\textbf{Sokolov I.\,A.} see Pechinkin A.\,V.\hfill\hfill\hfill\hfill\hfill\hfill\hfill\hfill\hfill\hfill\hfill\hfill\hfill\hfill\hfill\hfill\hfill\hfill\hfill\hfill\hfill\hfill\hfill\hfill\hfill\hfill\hfill\hfill\hfill\hfill\hfill\hfill\hfill\hfill\hfill}{\ }
\contentsline {subsection}{\textbf{Sokolov I.\,A.} see Zakharov V.\,N.\hfill\hfill\hfill\hfill\hfill\hfill\hfill\hfill\hfill\hfill\hfill\hfill\hfill\hfill\hfill\hfill\hfill\hfill\hfill\hfill\hfill\hfill\hfill\hfill\hfill\hfill\hfill\hfill\hfill\hfill\hfill\hfill\hfill\hfill\hfill}{\ }
\contentsline {subsection}{\textbf{Stupnikov S.\,A.} see Zakharov V.\,N.\hfill\hfill\hfill\hfill\hfill\hfill\hfill\hfill\hfill\hfill\hfill\hfill\hfill\hfill\hfill\hfill\hfill\hfill\hfill\hfill\hfill\hfill\hfill\hfill\hfill\hfill\hfill\hfill\hfill\hfill\hfill\hfill\hfill\hfill\hfill}{\ }
\contentsline {subsection}{\textbf{Zakharov V.\,N., Kalinichenko L.\,A., Sokolov I.\,A., Stupnikov S.\,A.}\ \ Development of Canonical Information Models for Integrated Information Systems}{\qquad 2 \qquad 15} 
\contentsline {subsection}{\textbf{Zakharov V.\,N., Kozmidiady V.\,A.}\ \ Means Providing Applications Fault Tolerance}{\qquad 1 \qquad 14} 
\def\leftfootline{\small{\textbf{\thepage}
\hfill ИНФОРМАТИКА И ЕЁ ПРИМЕНЕНИЯ\ \ \ том~1\ \ \ выпуск~2\ \ \ 2007}
}%
 \def\rightfootline{\small{ИНФОРМАТИКА И ЕЁ ПРИМЕНЕНИЯ\ \ \ том~1\ \ \ выпуск~2\ \ \ 2007
 \hfill \textbf{\thepage}}}
 \label{end\stat}


%\tableofcontents


\end{document}

\newcommand{\Ack}{\subsection*{\protect\large\bf Acknowledgments}}