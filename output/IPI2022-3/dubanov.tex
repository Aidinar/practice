\def\stat{dubanov}

\def\tit{КИНЕМАТИЧЕСКИЕ МОДЕЛИ ЗАДАЧ ПРЕСЛЕДОВАНИЯ НА~ПЛОСКОСТИ МЕТОДАМИ 
ПАРАЛЛЕЛЬНОГО СБЛИЖЕНИЯ И~ПОГОНИ}

\def\titkol{Кинематические модели задач преследования на~плоскости методами 
параллельного сближения и~погони}

\def\aut{А.\,А.~Дубанов$^1$, В.\,А.~Нефедова$^2$}

\def\autkol{А.\,А.~Дубанов, В.\,А.~Нефедова}

\titel{\tit}{\aut}{\autkol}{\titkol}

\index{Дубанов А.\,А.}
\index{Нефедова В.\,А.}
\index{Dubanov A.\,A.}
\index{Nefedova V.\,A.}


%{\renewcommand{\thefootnote}{\fnsymbol{footnote}} \footnotetext[1]
%{Работа выполнена при поддержке Министерства науки и~высшего образования Российской Федерации (проект 
%075-15-2020-799).}}


\renewcommand{\thefootnote}{\arabic{footnote}}
\footnotetext[1]{Бурятский государственный университет, alandubanov@mail.ru}
\footnotetext[2]{Бурятский государственный университет, emelyanovatorik@gmail.com}


\vspace*{-12pt}



  

    \Abst{Рассматриваются модели задачи преследования на плоскости методами 
параллельного сближения и погони. Цель работы~--- модификация методов параллельного 
сближения и погони на случаи, когда в момент начала преследования вектор скорости 
преследователя направлен не в точку на окружности Аполлония в методе параллельного 
сближения и не на цель в методе погони. В рассматриваемых моделях преследователь не 
может мгновенно изменять направление движения. Было наложено условие, что радиус 
кривизны траектории движения преследователя не может быть меньше определенной 
величины. Предлагаемый метод основан на том, что преследователь, выбирая шаг на этапе 
итераций, будет стараться следовать прогнозируемым траекториям. По материалам статьи 
написаны тестовые программы, которые рассчитывают траектории преследователя, 
учитывая изложенные условия. Выполненные анимированные изображения визуализируют 
изменение координат преследователя, цели и прогнозируемых траекторий во времени.}
    
    \KW{цель; преследователь; траектория; сближение; моделирование}
    
  \DOI{10.14357/19922264220314}  
  
\vspace*{-3pt}


\vskip 10pt plus 9pt minus 6pt

\thispagestyle{headings}

\begin{multicols}{2}

\label{st\stat}
    
\section{Введение}

   При расчете траекторий преследователя в задачах преследования 
в~пространстве и на плос\-кости используются методы параллельного сбли\-же\-ния 
и~погони. В~описании задачи \mbox{преследования} методом параллельного 
сближения в трудах Л.\,О.~Пет\-ро\-ся\-на~[1, 2] на\-прав\-ле\-ние вектора ско\-рости 
в~точке на\-хож\-де\-ния преследователя~$P_i$ и на\-прав\-ле\-ние вектора ско\-рости 
в~точке на\-хож\-де\-ния цели~$T_i$ пересекаются в точке~$K_i$, принадлежащей 
окружности Аполлония (рис.~1), соответствующей данному моменту времени.

Для точек~$P$ и~$T$ точка~$K$ окружности Аполлония\linebreak характерна тем, что 
отношение длин
  $\vert PK\vert/\vert QK\vert \hm= \vert V_P\vert/\vert V_T\vert$ есть отношение 
модулей скоростей преследователя и цели. При дискретном моделировании 
точек траектории преследователя $\{P_i\}$
\mbox{можно} предложить сле\-ду\-ющую 
итерационную схему (рис.~2):



\end{multicols}
   
   \begin{figure*}[h] %fig1
   \vspace*{-12pt}
  \begin{center}  
    \mbox{%
\epsfxsize=99.151mm
\epsfbox{dub-1.eps}
}

\end{center}
\vspace*{-9pt}
   \Caption{Итерационная схема метода параллельного сближения}
   \end{figure*}
   


   \begin{multicols}{2}
   
   { \begin{center}  %fig2
 \vspace*{-3pt}
     \mbox{%
\epsfxsize=60.497mm
\epsfbox{dub-2.eps}
}

\vspace*{6pt}


\noindent
{{\figurename~2}\ \ \small{
Итерационная схема метода погони
}}
\end{center}}

\vspace*{8pt}

\addtocounter{figure}{1}
   
  
   
  \noindent
      $$
  P_i= P_{i-1} +V_P \Delta T \fr{K_{i-1} -P_{i-1}}{\vert K_{i-1} -P_{i-1}\vert}\,.
  $$
    Радиусы окружностей Аполлония определяются выражением
  $$
  R_i= \fr{V_T^2}{V_P^2-V_T^2}\left\vert T_i - P_i\right\vert\,.
  $$
    Центры окружностей Аполлония рассчитываются по формуле:
  $$
  Q_i= T_i +\fr{V_i^2}{V_P^2 -V_T^2}\left( T_i -P_i\right).
  $$
  
  Координаты точки~$K_i$ получаются в результате решения системы 
уравнений относительно непрерывного параметра~$t$:
  \begin{align*}
  \left(K_i-Q_i\right)^2 &=R_i^2\,;\\
  K_i&=T_i +V_T \fr{T_{i+1} -T_i}{\vert T_{i+1} -T_i\vert}\,t\,.
  \end{align*}
  
  Такова одна из моделей построения траектории преследователя в методе 
параллельного сближения на плоскости. При этом требуется, чтобы 
на\-прав\-ле\-ния векторов движения пре\-сле\-до\-ва\-те\-ля и~цели пересекались в~точках, 
при\-над\-ле\-жа\-щих окруж\-но\-стям Аполлония. 
  
  Для метода погони в задаче преследования на плоскости с постоянной 
скоростью характерно то, что вектор скорости преследователя точно совпадает 
с направлением на цель. В~такой постановке задача имеет как непрерывную 
модель для решения, так и квазидискретную. В~непрерывной модели решение 
основывается на численном или аналитическом решении системы 
дифференциальных уравнений:
  \begin{gather}
  \left(T_x(t)\!-\!P_x(t)\right) \fr{dP_y(t)}{dt} \!=\!\left( T_y(t)\!-\!P_y(t)\right) 
\fr{dP_x(t)}{dt};
\label{e1-dub}\\
  \left( \fr{dP_x(t)}{dt}\right)^2 \!+\!\left( \fr{dP_y(t)}{dt}\right)^2 =V^2,
  \label{e2-dub}
  \end{gather}
где $\begin{bmatrix} P_x \\ P_y \end{bmatrix}$~--- координаты преследователя; 
$\begin{bmatrix} T_x\\ T_y \end{bmatrix}$~--- координаты цели; $V$~--- 
скорость цели.

  Уравнение~(1) системы говорит о том, что вектор скорости 
преследователя сонаправлен с~вектором линии, соединяющей преследователя 
и~цель. Уравнение~(2) сис\-те\-мы говорит о~постоянстве модуля ско\-рости.
  
 




  В качестве простого примера квазидискретной модели можно привести одну 
из итерационных схем, которая вычисляет координаты точек следующего 
положения преследователя (см.\ рис.~2): 
$$
P_{i-1}= P_i+ V_P \Delta t \left (T_i-  P_i\right),
$$
 где $V_P$~--- скорость преследователя; $\Delta t$~--- период 
дискретизации. Как видно из постановки задачи преследования методом погони 
с постоянной скоростью, вектор скорости преследователя должен быть всегда 
направлен на цель, даже в момент начала преследования. В~обеих описанных 
моделях задается начальное направление вектора скорости преследователя. 
Цель данной статьи~--- представить модифицированные модели преследования, 
в~которых направление вектора скорости преследователя произвольно.

\vspace*{-2pt}
   
\section{Кинематическая модель метода параллельного 
сближения}

\vspace*{-2pt}

   Итерационную схему, пред\-став\-лен\-ную на рис.~1, можно интерпретировать 
иначе. На рис.~3 пред\-став\-лен итерационный процесс, определяющий 
координаты точки~$P_i$ при известных координатах точек $P_{i-1}$, $T_{i-1}$ и~$T_i$ и~скоростей преследователя и цели~$V_P$ и~$V_T$ 
соответственно.
   
  Cначала определяется единичный вектор 
  $$
  \vec{\tau} =\fr{T_{i-1} -P_{i-1}}{\left\vert T_{i-1} -P_{i-1}\right\vert}\,.
  $$
   Координаты точки $T_i\hm= T_{i-1}\hm+ \vec{V}_T \Delta T$, где $\Delta 
T$~--- период дискретизации, предопределены поведением цели. Значит, 
прямую линию, которая будет соединять точки~$P_i$ и~$T_i$, можно 
представить в виде 
$$
L(\mu) = T_i+ \mu\vec{\tau}\,. 
$$

\setcounter{figure}{3}
\begin{figure*} %fig4
\vspace*{1pt}
  \begin{center}  
    \mbox{%
\epsfxsize=126.516mm
\epsfbox{dub-4.eps}
}

\end{center}
\vspace*{-6pt}
\Caption{Квазидискретная модель параллельного сближения}
\end{figure*}


{ \begin{center}  %fig3
 \vspace*{3pt}
     \mbox{%
\epsfxsize=66.187mm
\epsfbox{dub-3.eps}
}

\end{center}

\vspace*{-3pt}

\noindent
{{\figurename~3}\ \ \small{
Интерпретация итерационной схемы метода параллельного сближения
}}}

%\vspace*{6pt}

\addtocounter{figure}{1}

\pagebreak

\noindent
Тогда 
координаты~$P_i$ сле\-ду\-юще\-го шага итераций траектории преследователя 
можно интерпретировать как точку пересечения окруж\-ности радиуса $V_P 
\Delta T$ с центром в точке~$P_{i-1}$ и прямой линии~$L(\mu)$:
  \begin{align*}
  \left( L(\mu) -P_{i-1}\right)^2 &=\left( V_P \Delta T\right)^2\,;\\
  L(\mu)& =T_i +\mu \vec{\tau}\,.
  \end{align*}
  
  
  
  Решение вышеприведенной системы уравнений относительно 
параметра~$\mu$ даст такое значение параметра, при котором будет 
выполняться сле\-ду\-ющее: 
$$
P_i= T_i+ \mu \vec{\tau}\,. 
$$
  
  Итерационную схему параллельного сближения предлагается 
модифицировать следующим образом. Пусть в момент начала сближения 
вектор скорости преследователя~$P_{i-1}$ направлен произвольным образом, 
но не в точку на окружности Аполлония (рис.~4). В~силу инертности 
преследователя минимальный радиус кривизны траектории не может быть 
меньше определенного значения~$r_c$. Точке преследователя $P_{i-1}$ 
соответствуют вектор скорости~$\vec{V}_{P_{i-1}}$ и вектор единичной 
нормали~$\vec{n}_{i-1}$, $\vec{V}_{P_{i-1}} \vec{n}_{i-1}\hm=0$. Далее 
находится центр окружности радиуса~$r_c$:
  $$
  C_{i-1} =P_{i-1} +V_P \Delta T \vec{n}_{i-1}\,.
  $$
  
  К построенной окружности строится касательная из точки~$T_{i-1}$ для 
нахождения точки~$P_{t_{i-1}}$ сопряжения прямой и окружности (см.\ рис.~4). 
Дугу окружности $\PRDN$ и отрезок $[P_{t_{i-1}} 
T_{i-1}]$ будем считать одной составной кривой линией $f_{i-1}(s)$, где 
па\-ра\-мет\-ром~$s$ служит длина дуги данной параметрической кривой.
  

   
  В предлагаемой тестовой программе отсчет длины дуги начинается от точки 
$T_{i-1}$: $f_{i-1}(0)\hm= T_{i-1}$. Проведем параллельный перенос линии 
$f_{i-1}(s)$ на вектор $T_i\hm- T_{i-1}$. Положение точки~$T_i$ известно 
и~пол\-ностью определяется поведением цели. В~рамках решения данной задачи 
будем считать поведение цели полностью детерминированным. Уравнение 
параллельной линии $f_i(s)\hm= f_{i-1}(s)\hm+ T_i\hm- T_{i-1}$ будем считать 
известным, и для нахождения точки~$P_i$ следующего шага преследователя 
необходимо решение следующей системы уравнений и неравенств 
относительно параметра~$s$ ($0\hm\leq s\hm\leq s_{i-1}$):
  \begin{align*}
  \left( f_i(s)-P_{i-1}\right)^2 &= \left( V_P \Delta T\right)^2\,;\\
  f_i(s)&=f_{i-1}(s)+T_i-T_{i-1}\,,
    \end{align*}
где $s_{i-1}$~--- это значение параметра~$s$, со\-от\-вет\-ст\-ву\-ющее точке~$P_{i-
1}$. Первое, что сделано в~пред\-ла\-га\-емой программе,~--- для со\-став\-ной кривой 
в~момент начала преследования, состоящей из дуги и~отрезка, выполнена 
параметризация от длины дуги. Для этого было необходимо получить 
упорядоченный набор точек $\{x_i, y_i\}$. По каждой координате встроенными 
средствами MathCAD выполнена кубическая сплайн-ин\-тер\-по\-ля\-ция от 
формального параметра~$\delta$ и получены функции $X(\delta)$, $Y(\delta)$, 
$i\hm\in [0,N\hm-1]$, $\delta_i\hm\leq \delta\hm\leq \delta_{i+1}$, где 
$\delta_i\hm= i$; $N$~--- число элементов массивов $\{x_i, y_i\}$.

  Далее был составлен якобиан для передачи во встроенные решатели системы 
MathCAD:
  $$
  D(s,\delta)=\fr{1}{\sqrt{dX^2/d\delta + dY^2/d\delta}}\,.
  $$

  Полученное решение выражает зависимость массивов $\{x_i, y_i\}$ от 
параметра длины дуги~$s$. Таким образом, можно считать, что уравнение 
базовой кривой, с которой будет совершаться параллельный перенос, получено 
(рис.~5). Далее предстоит создать вычислительный цикл, в котором решается 
система уравнений и неравенств. Данная задача сводится к численному 
решению уравнения поиском нулей функции методом секущей в заданном 
диапазоне. 
  
      \setcounter{figure}{5}
\begin{figure*}[b] %fig6
\vspace*{6pt}
  \begin{center}  
    \mbox{%
\epsfxsize=128.8mm
\epsfbox{dub-6.eps}
}

\end{center}
\vspace*{-6pt}
\Caption{Моделирование траектории преследователя}
\end{figure*}  


  Встроенные средства численного решения уравнений системы MathCAD 
позволяют решить уравнение
 
 
 \pagebreak
  
  { \begin{center}  %fig5
 \vspace*{-4pt}
     \mbox{%
\epsfxsize=79mm
\epsfbox{dub-5.eps}
}

\end{center}



\noindent
{{\figurename~5}\ \ \small{
Кинематическая модель параллельного сближения: \textit{1}~--- линии параллельного переноса $Y_p(X_p)$;
\textit{2}~--- окружности числа $\mathrm{CC}^{\langle 1\rangle} (\mathrm{CC}^{\langle 0\rangle})$;
\textit{3}~--- траектория цепи $\mathrm{TT}^{\langle 1\rangle} (\mathrm{TT}^{\langle 0\rangle})$;
\textit{4}~--- траектория преследователя $\mathrm{PP}^{\langle 1\rangle} (\mathrm{PP}^{\langle 0\rangle})$;
\textit{5}~--- базовая линия предполагаемых траекторий $\mathrm{Fr}^{\langle 1\rangle} (\mathrm{Fr}^{\langle 0\rangle})$
}}}

\vspace*{15pt}



 \noindent
 $$
  \left( f_{i-1}(s) +T_i-T_{i-1} -P_{i-1}\right)^2 -\left( V_P \Delta T\right)^2=0
  $$
в диапазоне $s\hm\in [0, s_{i-1}]$. На анимированном изоб\-ра\-же\-нии рис.~5 
представлены результаты моделирования тестовой программы~\cite{11-dub}. 

\section{Кинематическая модель метода погони}


\vspace*{-12pt}



Рассмотрим следующую итерационную схему. Будем считать, что в~момент 
времени~$t_i$ известно положение цели~$T_i$, положение 
преследователя~$P_i$ и~векторное уравнение~$F_i(s)$ прогнозируемой на 
данный момент времени траектории движения преследователя (рис.~6). 
В~таком итерационном процессе ставится задача рассчитать 
координаты~$P_{i+1}$ следующего шага преследователя и выполнить 
аффинные преобразования векторной функции $F_i(s)$, чтобы найти 
выражения для функции $F_{i+1}(s)$. Чтобы найти координаты  
точки~$P_{i+1}$, необходимо решить уравнение
$$
\left\vert F_i\left(s_{i+1}\right) -F_i(s_i)\right\vert =V_P \Delta t
$$
относительно параметра~$s_{i+1}$. Когда разрабатывалась модель 
итерационного процесса, задавались начальные координаты~$T$ и~$P$ 
и~начальный вектор скорости движения преследователя~$V_P$. Также 
определялась векторная функция~$F(s)$ в момент начала преследования. На 
рис.~1 это составная кривая из дуги $\prdn$ и прямолинейного 
сегмента $[P_t T]$, где параметром служит длина дуги этой кривой. На $i$-м 
итерационном шаге происходит сле\-ду\-ющее. Строится окружность с~цент\-ром 
в~точке~$P_i$ радиуса~$V_P\Delta t$. Точка пересечения этой окружности\linebreak\vspace*{-12pt}

{ \begin{center}  %fig7
 \vspace*{-4pt}
    \mbox{%
\epsfxsize=79mm
\epsfbox{dub-7.eps}
}

\end{center}



\noindent
{{\figurename~7}\ \ \small{
Кинематическая модель метода погони: 
\textit{1}~---  точки предполагаемых траекторий $Y_{\mathrm{arr}} (X_{\mathrm{arr}})$;
\textit{2}~--- точки траекторий преследователя и~цели $(\mathrm{T}_2)^{\langle 1\rangle} ((T_2)^{\langle 0\rangle})$;
\textit{3}~--- динамическая предполагаемая траектория $y_f(x_f)$;
\textit{4}~--- динамическая точка преследователя $P_{fY} (P_{\mathrm{fX}})$
}}}

\vspace*{9pt}


\noindent
 и~траектории $F_i(s)$ будет точкой следующего шага преследователя~$P_{i+1}$. 
Потом находится точка пересечения$P^\prime_{i+1}$ параметрической 
функции $F_i(s)$ с~окруж\-ностью с~цент\-ром в~точ\-ке~$T_i$ и радиуса $\vert 
T_{i+1}\hm- P_{i+1}\vert$. Затем формируется локальный базис $(h_1^\prime, 
h_2^\prime)$ с центром координат в точке~$P^\prime_{i+1}$. Пересчитывается 
функция~$F_i(s)$. В~базисе $(h_1^\prime, h_2^\prime)$ она будет выглядеть как 
$F^\prime_i(s)$. Компоненты базиса $(h_1^\prime, h_2^\prime)$ таковы: 
$$
h_1^\prime= \fr{T_i-P^\prime_{i+1}}{\vert T_i -P^\prime_{i+1}\vert}\,,\enskip
h_2^\prime= \begin{bmatrix}
-h_{1_y}^\prime\\ h^\prime_{1_x}
\end{bmatrix}\,.
$$
Функция $F_i^\prime(s)$ будет иметь вид:
$$
F^\prime_i(s)= \begin{bmatrix}
\left( F_i(s)-P^\prime_{i+1}\right) h_1^\prime\\[3pt]
\left( F_i(s) -P^\prime_{i+1}\right) h_2^\prime
\end{bmatrix}\,.
$$
Сформируем базис $(h_1, h_2)$ с центром координат в~точке~$P_{i+1}$. 
Компоненты базиса $(h_1, h_2)$ будут выглядеть как 
$$
h_1= \fr{T_{i+1}-P_{i+1}}{\vert T_{i+1} -P_{i+1}\vert}\,;\enskip
h_2= \begin{bmatrix}
-h_{1_y}\\ h_{1_x}
\end{bmatrix}\,.
$$
Отметим очень важный момент, что $\vert T_{i+1}\hm- P_{i+1}\vert \hm= \vert 
T_i\hm- P^\prime_{i+1}\vert$. Отсюда можно утверждать, что локальное 
представление в локальном базисе $(h_1, h_2)$ с центром в точке~$P_{i+1}$ 
кривой~$F_{i+1}(s)$ будет совпадать с локальным представлением базиса 
$(h_1^\prime, h_2^\prime)$: $F^\prime_{i+1}(s)\hm= F_i^\prime(s)$. Базис $(E_1, 
E_2)$ в базисе $(h_1, h_2)$ выглядит как
$$
e_1= \begin{bmatrix}
E_1  h_1\\
E_1  h_2\end{bmatrix}\,;\enskip
e_2= \begin{bmatrix}
E_2  h_1\\
E_2 h_2\end{bmatrix}\,.
$$
Уравнение линии $F_{i+1}(s)$ будет иметь вид:
$$
F_{i+1}(s) = \begin{bmatrix} F^\prime_{i+1}(s) e_1\\[3pt]
F^\prime_{i+1}(s) e_2\end{bmatrix}
+P_{i+1}\,.
$$



  Итак, на $i$-м шаге итерации имеем сле\-ду\-ющее: шаг~$T_{i+1}$ выбирается 
целью, шаг преследователя~$P_{i+1}$ рассчитывается как точка пересечения 
окружности $(P_i, V_P \Delta t)$ и ранее рассчитанной линии $F_i(s)$, а~на 
основе уже имеющихся данных рассчитывается новая прогнозируемая 
траектория преследователя~$F_i(s)$. Анимированное изображение рис.~7 как 
раз демонстрирует результаты работы программы~[4]. 

\section{Заключение}

   В настоящей статье рассматриваются кинематические модели задачи 
преследования на плос\-кости методами параллельного сбли\-же\-ния и~погони. 
В~момент начала преследования ско\-рость преследователя на\-прав\-ле\-на 
произвольно. Данные методики возможны для использования при разработке 
геометрической модели группового пре\-сле\-до\-ва\-ния с~одновременным 
достижением цели или целей. По предложенным моделям и~алгоритмам 
написаны тес\-то\-вые программы расчета траекторий в~сис\-те\-ме компьютерной 
математики MathCAD. Тексты программ доступны на ресурсе~\cite{13-dub}. При написании статьи за основу приняты теоретические 
результаты, полученные в~\cite{1-dub, 2-dub, 3-dub}. Также приняты во 
внимание результаты  
работ~\cite{4-dub, 5-dub, 6-dub, 7-dub, 8-dub, 9-dub, 10-dub}.
   
{\small\frenchspacing
 {%\baselineskip=10.8pt
 %\addcontentsline{toc}{section}{References}
 \begin{thebibliography}{99}


\bibitem{2-dub} %1
\Au{Петросян Л.\,А., Рихсиев~Б.\,Б.} Преследование на плоскости.~--- М.: Наука, 1991. 96~с. 
%(Попул. лекции по мат.; Вып.~61).

\bibitem{1-dub} %2
\Au{Петросян Л.\,А.} Дифференциальные игры преследования~// Соросовский 
образовательный ж., 1995. №\,1. С.~88--91.

\bibitem{11-dub} %3
Метод параллельного сближения на плос\-кости с~ограничениями на кривизну. {\sf 
https://www.youtube. com/watch?v=qNXdykK21Z8}.
\bibitem{12-dub} %4
Метод погони на плос\-кости с~ограничениями на кривизну. {\sf 
https://www.youtube.com/watch?v=UQ5 bVKjVqZ4}.
\bibitem{13-dub} %5
Программный код в~системе MathCAD. {\sf http:// dubanov.exponenta.ru/books.htm}.

\bibitem{3-dub} %6
\Au{Айзекс Р.} Дифференциальные игры~/ Пер. с~англ.~--- М.: Мир, 1967. 480~с.
(\Au{Isaacs~R.} {Differential games: A~mathematical theory with applications to warfare and pursuit, control and optimization}.~---
Dover books on mathematics ser.~--- Wiley, 1965. 384~p.)

\bibitem{10-dub} %7
\Au{Ibragimov G., Hussin~N.\,A.} A~Pursuit-evasion differential game with many pursuers and one 
evader~// Malaysian J.~Mathematical Sciences, 2010. Vol.~4. Iss.~2. P.~183--194.

\bibitem{5-dub} %8
\Au{Кузьмина Л.\,И., Осипов~Ю.\,В.} Расчет длины траектории для задачи преследования~// 
Вестник МГСУ, 2013. №\,12. С.~20--26.
\bibitem{6-dub} %9
\Au{Саматов Б.\,Т.} Задача преследования убегания при  
ин\-тег\-раль\-но-гео\-мет\-ри\-че\-ских ограничениях на управ\-ле\-ния преследователя~// 
Автоматика и телемеханика, 2013. Вып.~7. С.~17--28.

\bibitem{8-dub} %10
\Au{Ibragimov G., Norshakila~A.\,R., Kuchkarov~A., Ismail~F.} Multi pursuer differential game of 
optimal approach with integral constraints on controls of players~// Taiwan. J.~Math., 
2015. Vol.~19. Iss.~3. P.~963--976.
\bibitem{9-dub} %11
\Au{Petrov N.\,N., Solov'eva~N.\,A.} Group pursuit with phase constraints in recurrent Pontryagin's 
example~// Int. J.~Pure Applied Mathematics, 2015. Vol.~100. Iss.~2. P.~263--278.

\bibitem{7-dub} %12
\Au{Романников Д.\,О.} Пример решения минимаксной задачи преследования 
с~использованием нейронных сетей~// Сборник научных трудов НГТУ, 2018. №\,2(92). 
С.~108--116. doi: 10.17212/2307-6879-2018-2-108-116.
\bibitem{4-dub} %13
\Au{Ахметжанов А.\,Р.} Динамические игры преследования на поверхностях: Автореф. дис.\ 
\ldots\ канд. физ.-мат. наук.~--- М.: МФТИ, 2019. 28~с.


\end{thebibliography}

 }
 }

\end{multicols}

\vspace*{-6pt}

\hfill{\small\textit{Поступила в~редакцию 18.08.20}}

\vspace*{8pt}

%\pagebreak

%\newpage

%\vspace*{-28pt}

\hrule

\vspace*{2pt}

\hrule

%\vspace*{-2pt}

\def\tit{KINEMATIC MODELS OF~PURSUIT PROBLEMS ON~THE~PLANE BY~THE~METHODS 
OF~PARALLEL APPROACH AND~PURSUIT}


\def\titkol{Kinematic models of~pursuit problems on~the~plane by~the~methods 
of~parallel approach and~pursuit}


\def\aut{A.\,A.~Dubanov and V.\,A.~Nefedova}

\def\autkol{A.\,A.~Dubanov and V.\,A.~Nefedova}

\titel{\tit}{\aut}{\autkol}{\titkol}

\vspace*{-8pt}


\noindent
Banzarov Buryat State University, 6a~Ranzhurov Str., Ulan-Ude 670000, Russian Federation


\def\leftfootline{\small{\textbf{\thepage}
\hfill INFORMATIKA I EE PRIMENENIYA~--- INFORMATICS AND
APPLICATIONS\ \ \ 2022\ \ \ volume~16\ \ \ issue\ 3}
}%
 \def\rightfootline{\small{INFORMATIKA I EE PRIMENENIYA~---
INFORMATICS AND APPLICATIONS\ \ \ 2022\ \ \ volume~16\ \ \ issue\ 3
\hfill \textbf{\thepage}}}

\vspace*{3pt} 




\Abste{This article provides accurate pursuit models based on the parallel approach and 
chase methods. This article is a modification of the methods of parallel rapprochement and 
chasing what happens when the pursuit begins. It cannot instantly change the direction of 
movement. This should be a less affordable option. The proposed method is based on the fact 
that the pursuer chooses a step at the iteration stage and will try to follow the predicted 
trajectories. Based on the materials of the article, test programs have been written that 
calculate the trajectories of the pursuer taking into account the stated conditions. Execution 
of animated images visualizes the change in the coordinates of the pursuer, target, and 
predicted time trajectories.}

\KWE{target; pursuer; trajectory; convergence; modeling}

  \DOI{10.14357/19922264220314}  

%\vspace*{-16pt}

%\Ack
%\noindent




\vspace*{6pt}

  \begin{multicols}{2}

\renewcommand{\bibname}{\protect\rmfamily References}
%\renewcommand{\bibname}{\large\protect\rm References}

{\small\frenchspacing
 { %\baselineskip=12pt
 \addcontentsline{toc}{section}{References}
 \begin{thebibliography}{99}
 
 \vspace*{-2pt}

\bibitem{2-dub-1} %1
\Aue{Petrosyan, L.\,A., and B.\,B.~Rihsiev.} 1961. \textit{Presledovanie na ploskosti} [Flat 
pursuit]. Moscow: Nauka. 96~p.

\bibitem{1-dub-1} %2
\Aue{Petrosyan, L.\,A.} 1995. Differentsial'nye igry presledovaniya [Differential pursuit 
games]. \textit{Sorosovskiy obrazovatel'nyy zh.} [Soros Educational~J.] 1:88--91.

\bibitem{11-dub-1}  %3
Metod parallel'nogo sblizheniya na ploskosti s~og\-ra\-ni\-che\-ni\-yami na kriviznu [Method of 
parallel approach on a~plane with restrictions on curvature]. Available at: {\sf 
https://www.youtube.com/watch?v=qNXdykK21Z8} (accessed May~25, 2022).
\bibitem{12-dub-1} %4
Metod pogoni na ploskosti s ogranicheniyami na kriviznu [Plane chase method with curvature 
constraints]. Available  at:  {\sf https://www.youtube.com/watch?v=UQ5 bVKjVqZ4} 
(accessed May~25, 2022).
\bibitem{13-dub-1} %5
Programmnyy kod v~sisteme MathCAD [Program code in the MathCAD system]. Available 
at: {\sf http://\linebreak dubanov.exponenta.ru/books.htm} (accessed May~25, 2022).
\bibitem{3-dub-1} %6
\Aue{Isaacs, R.} 1965.
\textit{Differential games: A~mathematical theory with applications to warfare and pursuit, control and optimization}.
Dover books on mathematics ser. Wiley. 384~p.
{\looseness=1

}

\bibitem{10-dub-1} %7
\Aue{Ibragimov, G.\,A., and N.\,A.~Hussin.} 2010. Pursuit-evasion differential game with 
many pursuers and one evader. \textit{Malaysian J.~Mathematical Sciences} 4(2):183--194.

\bibitem{5-dub-1} %8
\Aue{Kuzmina, L.\,I., and Y.\,Y.~Osipov.} 2013. Raschet dliny traektorii dlya zadachi 
presledovaniya [Calculation of the path length in the pursuit problem]. \textit{Vestnik MGSU} 
[Bulletin of MGSU] 12:20--26.
\bibitem{6-dub-1} %9
\Aue{Samatov, B.\,T.} 2013. The pursuit-evasion problem under integral-geometric 
constraints on pursuer controls. \textit{Automat. Rem. Contr.} 74:1072--1081.

\bibitem{8-dub-1} %10
\Aue{Ibragimov, G., A.\,R.~Norshakila, A.~Kuchkarov, and F.~Ismail.} 2015. Multi pursuer 
differential game of optimal approach with integral constraints on controls of players. 
\textit{Taiwan. J.~Math.} 19 (3):963--976.
\bibitem{9-dub-1} %11
\Aue{Petrov, N.\,N., and N.\,A.~Solov'eva.} 2015. Group pursuit with phase constraints in 
recurrent Pontryagin's example. \textit{Int. J.~Pure Applied Mathematics} 100(2):263--278.

\pagebreak 

\bibitem{7-dub-1} %12
\Aue{Romannikov, D.\,O.} 2018. Primer resheniya mi\-ni\-maks\-noy zadachi presledovaniya 
s~ispol'zovaniem ney\-ron\-nykh se\-tey [An example of solving a minimax pursuit problem using 
neural networks]. \textit{Sbornik nauchnykh trudov NGTU} [Transactions of scientific papers of the Novosibirsk State 
Technical University] 2(92):108--116. doi: 10.17212/2307-6879-2018-2-108-116.

\vspace*{-2pt}

\bibitem{4-dub-1} %13
\Aue{Ahmetzhanov, A.\,R.} 2019. Dinamicheskie igry presledovaniya na poverkhnostyakh 
[Dynamic pursuit games on surfaces].  Moscow: MIPT.  PhD Thesis. 28~p.
\end{thebibliography}

 }
 }

\end{multicols}

\vspace*{-6pt}

\hfill{\small\textit{Received August 18, 2020}}

\Contr


\noindent
\textbf{Dubanov Alexander A.} (b.\ 1967)~--- Candidate of Science (PhD) in technology, 
associate professor, Department of Geometry and Methods of Teaching Mathematics, 
Institute of Mathematics and Informatics, Banzarov Buryat State University, 6a~Ranzhurov 
Str., Ulan-Ude 670000, Russian Federation; \mbox{alandubanov@mail.ru}

\vspace*{3pt}

\noindent
\textbf{Nefedova Victoria A.} (b.\ 2000)~--- student, Institute of Mathematics and 
Informatics, Banzarov Buryat State University, 6a~Ranzhurov Str., Ulan-Ude 670000, 
Russian Federation; \mbox{emelyanovatorik@gmail.com}



\label{end\stat}

\renewcommand{\bibname}{\protect\rm Литература}    