\def\stat{vasil}

\def\tit{О ДОСТАТОЧНЫХ УСЛОВИЯХ ЭКСТРЕМУМА В~МНОГОМЕРНЫХ~ВАРИАЦИОННЫХ~ЗАДАЧАХ}

\def\titkol{О достаточных условиях экстремума в~многомерных вариационных задачах}

\def\aut{Н.\,С.~Васильев$^1$}

\def\autkol{Н.\,С.~Васильев}

\titel{\tit}{\aut}{\autkol}{\titkol}

\index{Васильев Н.\,С.}
\index{Vasilyev N.\,S.}


%{\renewcommand{\thefootnote}{\fnsymbol{footnote}} \footnotetext[1]
%{Работа выполнена при поддержке Министерства науки и~высшего образования Российской Федерации (проект 
%075-15-2020-799).}}


\renewcommand{\thefootnote}{\arabic{footnote}}
\footnotetext[1]{Московский государственный технический университет имени Н.\,Э.~Баумана, \mbox{nik8519@yandex.ru}}

%\vspace*{-6pt}



     
     \Abst{Формализуя системные представления о теориях, вариационные принципы дают 
общий подход к~моделированию широкого круга явлений из многих областей знания. Эти модели 
описывают искомые процессы как обеспечивающие стационарность некоторого 
<<универсального>> функционала. Соответствующие решения находят с~помощью необходимых 
условий оптимальности, а достаточные условия позволяют ее доказывать. В~естествознании это 
служит обоснованию принципов минимума энергии и~действия Гамильтона. В~статье получены 
достаточные условия минимальности экстремалей в~многомерных вариационных задачах. К~их 
числу относятся задачи, функционал которых содержит старшие производные нескольких 
искомых функций многих переменных. Доказательство теорем основано на использовании 
инвариантных поверхностных интегралов. Это позволило обобщить классические результаты. }
     
     \KW{экстремаль; экстремальная гиперповерхность; поле нормалей; дивергенция и~поток 
векторного поля; дифференциальная форма; внешнее дифференцирование; инвариантность 
интеграла; множитель Лагранжа}

\DOI{10.14357/19922264220305} 
  
%\vspace*{-3pt}


\vskip 10pt plus 9pt minus 6pt

\thispagestyle{headings}

\begin{multicols}{2}

\label{st\stat}
    
\section{Введение}

    Рассмотрены многомерные вариационные задачи, в~которых требуется найти 
слабый, в~пространстве $C^1(S)$ $S\hm\subset R^n$ или сильный, в~пространстве 
$C(S)$, $S\hm\subset R^n$, локальный минимум интегрального функционала 
    $J: C^1(S)\hm\to R$ вида 
    \begin{equation}
    J(z)=\int\limits_S f(x,z,z_x)\,dx\to \min
    \label{e1.1}
    \end{equation}
при заданном краевом условии
\begin{equation}
z \vert_{\gamma S} =u\,.
\label{e1.2}
\end{equation}
    
    В~(\ref{e1.1}) использовано краткое обозначение 
    $$
    z_x= \left(z_{x_1}, \ldots ,  z_{x_n}\right)
    $$ 
    для градиента функции 
    $$
    z: S\to R\,.
    $$
     Предполагается, что 
подынтегральная функция является гладкой $f\hm\in C^2[S\times W]$. Будем 
выбирать окрестности рассматриваемых экстремалей так, чтобы они не покидали 
множество $S\times W \hm\subset R^{2n+1}$, $(z,z_x)\hm\in W$. Кроме того, 
считаем, что замкнутая односвязная область~$S$ имеет ку\-соч\-но-глад\-кую 
границу~$\gamma S$. Оптимизация~(\ref{e1.1}), (\ref{e1.2}) проводится на классе 
допустимых функций $z\hm\in C^1(S)$, графики которых представляют собой 
гладкие гиперповерхности $\Pi_z\hm\subset R^{n+1}$ с~краем $(x,u)\hm\subset 
R^{n+1}$, $x\hm\in \gamma S$. 
    
    Одномерным аналогом вариационной задачи~(\ref{e1.1}), (\ref{e1.2}), 
$n\hm=1$, служит простейшая вариационная задача, решаемая на отрезке $[a,b]$ 
с~фиксированными краевыми ограничениями 
$$
z(a)= z\vert_a\,;\quad z(b)=  z\vert_b\,.
$$
 Для нее известны необходимые условия оптимальности решения $z\hm= 
z^*$, называемого экстремалью. Во-пер\-вых, это условие Лежандра: квадратичная 
форма, определяемая матрицей Гессе $(f_{z_{x_i}z_{x_j}})\vert_{z=z^*}$, является 
неотрицательно определенной. Во-вто\-рых, сама экстремаль~$z^*$ находится как 
решение уравнения Эйлера~[1, 2]. 
    
    В теоремах Якоби, Гильберта и~Вейерштрасса сформулированы достаточные 
условия локальной оптимальности~$z^*$, которые опираются на наличие поля 
экстремалей~[1, 2]. Условие Якоби обеспечивает существование этого поля 
в~некоторой области плоскости~$R^2$, включающей график функции~$z^*$. Это 
требование отсутствия сопряженной точки на полуинтервале~$[a, b)$.
    
    В многомерной вариационной задаче~(\ref{e1.1}), (\ref{e1.2}) необходимые 
условия оптимальности выражаются в~форме краевой задачи для уравнения  
Эй\-ле\-ра--Ост\-ро\-град\-ско\-го, решаемой при ограничении~(\ref{e1.2}) (см.\ [1, 
2]). 
    
    В статье получены достаточные условия локального минимума в~многомерной 
вариационной задаче~(\ref{e1.1}), (\ref{e1.2}). Обоснование опирается на свойство 
инвариантности поверхностных интегралов, определенных в~разд.~2 и~3. Это 
обобщение криволинейного инвариантного интеграла Гильберта~[1, 2]. В~разд.~3 
и~4 свойство инвариантности применено к~вариационным задачам, функционал 
которых зависит от нескольких искомых функций многих переменных и~их старших 
производных.
    
\section{Достаточные условия минимума }

    Рассмотрим семейство вариационных задач с~функционалом~(\ref{e1.1}) 
и~различными краевыми условиями 
    \begin{equation}
    z\vert_{\gamma S} =v\,,\enskip v\vert_{\Gamma}=u\vert_{\Gamma}\,.
    \label{e2.1}
    \end{equation}
Здесь лишь на части границы $\Gamma\hm\subset \gamma S$ исходной области~$S$ 
значения искомой функции совпадают с~функцией~$u$, заданной в~граничном 
условии~(\ref{e1.2}). Это приводит к~тому, что имеется множество решений 
задач~(\ref{e1.1}), (\ref{e2.1}), определяемое выбором функции~$v$. Все функции 
семейства $\{z\}$ суть решения вариационных задач с~функционалом~(\ref{e1.1}) 
и~граничными условиями $z\vert_{\gamma S}\hm= \tilde{v}$. Здесь~$\tilde{v}$~--- 
гладкое продолжение заданной функции~$u\vert_\Gamma$ с~части 
границы~$\Gamma$, $\Gamma\hm\subset \gamma S$, на всю границу~$\gamma S$ 
области~$S$. Все указанные функции удовлетворяют необходимому условию 
минимальности функционала~(\ref{e1.1}). Их можно найти как решения краевой 
задачи для уравнения Эй\-ле\-ра--Ост\-ро\-град\-ско\-го~[1, 2]:
\begin{equation}
\fr{\partial f}{\partial z} -\sum\limits_i \fr{d}{dx_i} \fr{\partial f}{\partial z_{x_i}}=0\,,\enskip z\vert_{\gamma S}=\tilde{v}\,.
\label{e2.2}
\end{equation}
    
    В семействе $\{z\}$ содержится искомое решение~$z^*$, удовлетворяющее 
исходному ограничению~(\ref{e1.2}) при $\tilde{v}\hm=u$. \textit{Требуется} 
проверить его оптимальность. 
    
    Пусть существует область $G\hm\subset R^{n+1}$, содержащая поверхность 
$\Pi_{z^*}$, через каждую точку которой проходит единственная экстремальная 
поверхность~$\Pi_z$, которая является графиком функции\linebreak $z\hm\in \{z\}$ 
(см.~(\ref{e2.2}). В~этом случае будем говорить, что в~области~$G$ имеется 
\textit{поле экстремальных поверхностей}, включающее~$\Pi_{z^*}$. Благодаря 
этому в~об\-ласти~$G$ существует поле нормалей к~поверхностям~$\Pi_z$:
    \begin{equation}
    n_z(x,z)=(-z_x, 1)\,, n_z \perp \Pi_z\,.
    \label{e2.3}
    \end{equation}
В каждой точке $(x,z)\hm\in G$ существует наклон поля~--- вектор $p\hm= 
p(x,z)\hm= z_x$ (см.~(\ref{e2.3})), характеризующий касательное пространство 
к~поверхности~$\Pi_z$. С~его помощью в~области~$G$ определим векторное поле, 
строящееся по функционалу~(\ref{e1.1}):
    \begin{multline}
    V(x,z) =\left(
    \vphantom{\sum\limits_i}
    -f_{z_x} \left(x,z,p(x,z)\right), f(x,z,p(x,z))-{}\right.\\
\left.    {}-\sum\limits_i p^i 
(x,z)f_{z_{x_i}}(x,z,p(x,z))\right).
    \label{e2.4}
    \end{multline}
В поле $V\equiv (V_1, V_2)$ имеются векторная $V_1(x,z)\hm\in R^n$ и~скалярная 
$V_2(x,z)\hm\in R$ составляющие 
$$
V_1= -f_{z_x} (x,z,p)\,;\
V_2= f(x,z,p)-\left(p,f_{z_x}(x,z,p)\right)\,.
$$
Здесь сумма, входящая в~определение~(\ref{e2.4}), записана кратко в~виде 
скалярного произведения $(p, f_{z_x})$ указанных векторов. Вычислим 
дивергенцию поля~$V$~--- его инвариант вида~[3]:
$$
\mathrm{div}\, V =\sum\limits_i \fr{\partial}{\partial x_i}\,V_1^i +\fr{\partial}{\partial z}\,V_2\,.
$$
    
    \noindent
    \textbf{Лемма~2.1.}\ \textit{Справедливо равенство} $\mathrm{div}\,V(x,z)\hm\equiv 0$, 
$(x,z)\hm\in G$.
    \smallskip
    
    \noindent
    Д\,о\,к\,а\,з\,а\,т\,е\,л\,ь\,с\,т\,в\,о\,.\ \ Через всякую точку $(x,z)\hm\in G$ 
проходит некоторая экстремаль $z\hm= \overline{z}(x)$, для которой, согласно~(\ref{e2.3}), имеют место соотношения:
    $$
    z_{x_i}= p^i (x,z),\enskip z_{x_ix_i} = p^i_{x^i} +p^i_z p^i\,.
    $$
Всякая экстремаль удовлетворяет уравнению~(\ref{e2.2}), а~значит, выполнены 
равенства:
\begin{multline*}
0=\sum\limits_i \fr{d}{dx_i}\,V_1^i\vert_{z=\overline{z}(x)} +f_z= {}\\
{}=-\sum\limits_i \left( 
f_{z_{x_i} x_i} +f_{z_{x_i}z} z_{x_i} +f_{z_{x_i}z_{x_i}} z_{x_i x_i}\right) 
+f_z={}\\
{}= \sum\limits_i \fr{\partial}{\partial x_i}\,V_1^i +f_z -\sum\limits_i \left(  z_{z_{x_i}z} 
p^i +f_{z_{x_i}z_{x_i}} p^i_z p^i\right) ={}\\
{}=\sum\limits_i \fr{\partial}{\partial x_i}\,V_1^i 
+\fr{\partial}{\partial z}\,V_2 =\mathrm{div}\,V(x,z)\,.
\end{multline*}
Это и~требовалось доказать.
    
    \smallskip
    
    Для краткости внешнее произведение $dx_1\wedge \cdots\linebreak \cdots \wedge dx_n$  
1-форм $dx_i$ обозначим $dx$~[3]. С~учетом равенств 
$$
n_{z_i}= \fr{\partial z}{\partial x_i},\enskip i= 1,\ldots , n\,,
$$
 проинтегрируем дифференциальную $n$-фор\-му 
$\omega\hm= (V,n)\,d\varphi$ по поверхности~$\Phi$~[3]. В~результате получим 
величину $I(\Phi)\hm= \int\nolimits_\Phi \omega$~--- поток поля~$V$  через 
поверхность~$\Phi$: 
    \begin{multline}
    I(\Phi) =\int\limits_\Phi \left(
    \vphantom{\sum\limits_i}
    f(x,z,p(x,z))-{}\right.\\
\left.    {}-\sum\limits_i p^i(x,z) 
f_{z_{x_i}}(x,z,p(x,z))\right)dx+{}\\
    {}+ \sum\limits_k (-1)^{k+1} f_{z_{x_k}} (x,z,p(x,z)) \,dx_1\wedge \cdots\\
    \cdots \wedge 
dx_{k-1} \wedge dz\wedge dx_{k+1}\wedge \cdots \wedge dx_n\,.
    \label{e2.5}
    \end{multline}
     
     \noindent
     \textbf{Лемма~2.2.}\ \textit{Поверхностный интеграл второго 
рода}~(\ref{e2.5}) \textit{инвариантен на множестве всех поверхностей 
$\Phi\hm=\Pi$ с~одинаковой границей} $\gamma\Pi\hm=\gamma$.
     
     \smallskip
     
    \noindent
    Д\,о\,к\,а\,з\,а\,т\,е\,л\,ь\,с\,т\,в\,о\,.\ \ Склеим две поверхности $\Pi_1$ и~$\Pi_2$, 
являющиеся графиками функций $z_1$ и~$z_2$, по общей границе~$\gamma$. 
Полученная замкнутая поверхность $\Phi\hm=\Pi_{z_1}\cup \Pi_{z_2}$, $z_1\hm\geq 
z_2$, ограничивает некоторое тело~T. <<Внешние>> нормали к~$\Phi\hm=\gamma 
\mathrm{Т}$ в~точке $(x,z)$ задаются векторами $\pm n_z(x,z)$ вида~(\ref{e2.3}), 
от\-ли\-ча\-ющи\-ми\-ся только выбором знака в~зависимости от участка~$\Pi_{z_1}$ или 
$\Pi_{z_2}$ поверхности~$\Phi$, на котором лежит эта точка. Внешнее 
дифференцирование~[3] формы $\omega\hm= (V,n)\,d\varphi$ дает $d\omega\hm= \mathrm{div}\,V\,dt$. Поэтому 
    $$
    \oint\limits_{\gamma\mathrm{Т}^+}\omega =\int\limits_{\mathrm{Т}^+} \mathrm{div}\,V\,dt\,.
    $$
    
    Согласно лемме~2.1, выполнено равенство
    $$
    I(\Pi_1) -I(\Pi_2) =\oint\limits_\Phi \omega =\int\limits_{\mathrm{Т}^+} \mathrm{div}\,V\,dx 
=0\,,
    $$
означающее инвариантность интеграла: $I(\Pi_1)\hm = I(\Pi_2)$. Лемма~2.2 доказана.

\smallskip

    
    По доказанному, интеграл $I(\Pi)$, $\gamma\Pi\hm=\gamma$, естественно 
обозначать~$I_\gamma$, указывая лишь край поверхности. Особый интерес 
представляют собой поверхности~$\Pi$, задаваемые графиками гладких 
\mbox{функций}~$z$ из семейства $\{z\}$. Они однозначно проецируются на фигуру~$S$. 
Тогда инвариантный интеграл~(\ref{e2.5}) допускает запись вида
    \begin{multline}
    I_\gamma =I(\Pi) =\int\limits_S \left( f(x,z,p(x,z))+{}\right.\\
\left.    {}+\left( z_x -p(x,z), 
f_{z_x}(x,z,p(x,z))\right)\right)dx\,.
    \label{e2.5prime}
    \end{multline}
    
    \noindent
    \textbf{Теорема~2.1.}\ \textit{Пусть в~области~$G$ существует поле 
экстремальных поверхностей с~нормалями $(-p(x,z),1)$, вклю\-ча\-ющее 
экстремаль~$z^*$ задачи}~(\ref{e1.1}), (\ref{e1.2}). \textit{Если выполнено усиленное 
условие Лежандра $\forall \,(x,z)\hm\in G\hm\Rightarrow 
(f_{z_{x_i}z_{x_j}})\vert_{z_x=p(x,z)}\hm>0$, то~$z^*$ является слабым локальным 
минимумом функционала}~(\ref{e1.1}). 
    
    \textit{Если дополнительно для любых $(x,z)\hm\in G$ подынтегральная 
функция~$f$ выпукла по переменной~$z_x$, то $z^*$~--- сильный локальный 
минимум в}~$G$.
    
    \smallskip
    
    \noindent
    Д\,о\,к\,а\,з\,а\,т\,е\,л\,ь\,с\,т\,в\,о\,.\ \  Так как $z_x^*\hm= p(x,z^*)$, то 
$J(z^*)\hm= I(\Pi_{z^*})$ (см.~(\ref{e2.5})). По лемме~2.2, $I(\Pi_{z^*})\hm= I(\Pi)$, 
$\gamma\Pi\hm= \gamma S$, для любой функции~$z$, $z\vert_{\gamma S}\hm=u$, 
имеющей график $\Pi\hm\subset G$. Тогда приращение функционала~(\ref{e1.1}) 
может быть преобразовано к~виду

\noindent
\begin{multline*}
    J(z)-J(z^*) =\int\limits_S f(x,z,z_x)\, dx- I(\Pi_{z^*}) ={}\\
    {}=\int\limits_S f(x,z,z_x) \,dx -
I(\Pi)\,.
    \end{multline*}
    
Учитывая~(\ref{e2.5prime}), перепишем правую часть этого равенства и~получим 
\begin{multline}
J(z)-J(z^*) =\int\limits_S \left( f(x,z,z_x) -f(x,z,p(x,z)\right) -{}\\
{}-
\left( z_x -p(x,z), f_{z_x}(x,z,p(x,z))\right) dx\geq 0\,.
\label{e2.6}
\end{multline}


Выполнение условия Лежандра обеспечивает неотрицательность подынтегральной 
функции в~выражении~(\ref{e2.6}). Это неравенство справедливо для любой 
функции $f(z,x,z_x)$, локально выпуклой по переменной~$z_x$. Таким образом, 
слабая локальная оптимальность точки~$z^*$ доказана. В~случае вы\-пук\-лости~$f$ 
по этой переменной неравенство~(\ref{e2.6}) имеет место при подстановке любых 
гладких функций~$z$, лишь бы их графики лежали в~области~$G$. Тогда 
неравенство $J(z)\hm\geq J(z^*)$ справедливо и~в~сильной окрестности 
точки~$z^*$. Это завершает доказательство теоремы~2.1.

\smallskip

\noindent
\textbf{Пример~2.1.} Пусть имеется вариационная задача, в~которой 
\begin{align*}
J(z)&=\int\limits_S \left( \left(\fr{\partial z}{\partial x}\right)^2+\left(\fr{\partial z}{\partial y}\right)^2-2z^2\right) dxdy\to {}\\
&{}\hspace*{50mm}\to \min\limits_{z: z\vert_{\gamma S}=u}\,;
\\
        S&= \left\{ (x,y): 0\leq x,\ y\leq 1\right\}\,;
    \\
    u&=\begin{cases} \sin(y-1)\,, & x=0\,;\\
    -\sin 1 \cos (x)\,, & y=0\,;\\
    x\,, & y=1\,;\\
    y\,, & x=1\,.
    \end{cases}
   \end{align*}
    
    Согласно уравнению Эй\-ле\-ра--Ост\-ро\-град\-ско\-го~(\ref{e2.2}), 
необходимо, чтобы все экстремальные поверхности были решениями следующей 
первой краевой задачи для уравнения Пуассона ($\Delta$~--- оператор Лапласа): 
    $$
    \Delta z  =-2z\,,\enskip z\vert_\Gamma =x\vert_\Gamma\,.
    $$
    
    В качестве~$\Gamma$ выберем часть границы квадрата, выделяемой 
уравнением $y\hm=1$. Покажем, что экстремаль $z^*\hm= xy \hm+ \sin (y\hm-
1)\cos(x)$ удовлетворяет всем условиям теоремы~2.1. Действительно, 
однопараметрическое семейство функций
    $$
    z=Cxy+(1-C)x+\sin (y-1)\cos(x)
    $$
образует поле экстремальных поверхностей в~об\-ласти $G\hm= S\times R$, 
включающее~$z^*$. По теореме~2.1, отображение~$z^*$ является даже глобальным 
сильным минимумом.

\section{Многомерные задачи с~несколькими искомыми функциями}
 
    Рассмотрим вариационную задачу относительно двух неизвестных функций 
$z,y$:
     \begin{equation}
     J(z,y) =\int\limits_S f(x,z,z_x, y, y_x)\,dx\to \min\limits_{\substack{{z\vert_{\gamma S}=u_1}\\
     {y\vert_{\gamma 
S} =u_2}}}\,.
     \label{e3.1}
     \end{equation}
    
    Предположим, что условия теоремы~2.1 выполняются для каждого из 
функционалов $J(z,y^*)$, $J(z^*, y)$, имеющих одну неизвестную функцию~$z$ 
или~$y$, причем их экстремалями являются~$z^*$ и~$y^*$ соответственно. Тогда 
точку $(z^*, y^*)$ назовем \textit{раздельным} слабым локальным минимумом 
в~задаче~(\ref{e3.1}). Раздельное условие Лежандра означает то, что оно выполнено 
по каждой из переменных~$z_x$, $y_x$ для обоих интегрантов из функционалов 
$J(z, y^*)$, $J(z^*,y)$.  Итак, доказано
    
    \smallskip
    
    \noindent
    \textbf{Следствие~3.1.} Пусть выполнено раздельное усиленное условие 
Лежандра и~в~некоторых областях, содержащих экстремали~$z^*$ и~$y^*$, 
существуют поля экстремальных поверхностей $p(x,z)$, $(x,z)\hm\in G_1$, 
и~$q(x,y)$, $q(x,y)\hm\in G_2$. Тогда~$z^*$, $y^*$~--- раздельный слабый локальный 
минимум в~(\ref{e3.1}).
    
    \smallskip
    Вместо поиска минимума в~задаче~(\ref{e3.1}) зачастую требуется найти 
минимакс этого функционала по переменным~$z$ и~$y$. Например, это так, если 
рас\-смат\-ри\-ва\-ет\-ся вариационная задача на условный экстремум, имеющая 
дополнительные фазовые ограничения, которые снимаются с~помощью множителей 
Лагранжа~$y$~[4]. Тем самым исходная задача сводится к~задаче на минимакс для 
функции Лагранжа.
    \smallskip
    
    \noindent
    \textbf{Следствие~3.2.}\ Если существуют поля экстремальных поверхностей 
$p(x,z)$ и~$q(x,y)$ в~некоторых областях $G_i$, $i\hm= 1,2$, и~для любых 
$(x,z)\hm\in G_1$ и~$(x,y)\hm\in G_2$ подынтегральная функция~$f$ выпукла по\linebreak 
переменной~$z_x$ и~вогнута по переменной~$y_x$, то пара $z^*, y^*$~--- седловая 
точка функционала~(\ref{e3.1}).
    
    \smallskip
    
    Для получения более сильных результатов рассмотрим фазовое пространство 
$G\hm= G_1\times G_2$. В~частности, это означает, что введены дополнительные 
переменные $x_{n+1}^\prime \hm= x_1,\ldots , x_{2n}^\prime\hm= x_n$, отвечающие 
исходному вектору~$x$. По предположению, имеется поле наклонов экстремальных 
поверхностей ($p\hm= p(x,z)$, $q\hm= q(x^\prime, y)$), снова позволяющее 
определить векторное поле, исходя из функционала вариационной 
задачи~(\ref{e3.1}): 
    \begin{multline*}
    V\equiv \left( V_1^1, V_1^2; V_2^1, V_2^2\right)\,,\\
     V_1=\left( V_1^1, 
V_1^2\right)\,,\ V_2= \left( V_2^1, V_2^2\right)\,,
\end{multline*}
где
\begin{align*}
    V_1^1 &=-f_{z_x}(x,z,y,p,q)\,;\\
    V_1^2&= f(x,z,y,p,q)-\left(p,f_{z_x}(x,z,y,p,q)\right);\\
    V_2^1 &=-f_{y_x} (x,z,y,p,q)\,;\\
     V_2^2 &= f(x,z,y,p,q) -\left( q, f_{u_x}(x,z,y,p,q)\right).
    \end{align*}
    
    В области $G$ находится поверхность $S\hm= z\times y$, через которую 
<<проходит>> поток поля~$V$. Так как нормаль к~этой поверхности равна $n\hm= 
(-z_x,1,-y_x,1)$, то этот поток равен
   \begin{multline*}
    I(S)= \int\limits_S (n,V)\,ds ={}\\
    {}=\int\limits_S \left(2f(x,z,y,p,q) +\left (z_x -p, f_{z_x} 
(x,z,y,p,q)\right) +{}\right.\\
\left.{}+\left( y_x-q, f_{y_x}(x,z,y,p,q)\right)\right)ds\,.
    \end{multline*}
   %
       Рассмотрим две гиперповерхности $S_1$ и~$S_2$, определяемые парами 
допустимых функций $(z_1,y_1)$ и~$(z_2,y_2)$. Как и~при доказательстве леммы~2.2, 
склеим поверхности~$S_1$ и~$S_2$ по общей границе, определив некоторое тело~T, 
имеющее границу $S_1\cup S_2$. 
    
    \smallskip
    
    \noindent
    \textbf{Лемма~3.1.} \textit{Интеграл  $I(S)$ инвариантен}. 
    
    \smallskip
    
    \noindent
    Д\,о\,к\,а\,з\,а\,т\,е\,л\,ь\,с\,т\,в\,о\,.\ \ Как и~в~лемме~2.2, воспользуемся 
равенством интегралов от дифференциальных форм $\omega\hm= (V,n)\,d\varphi$ 
и~$d\omega \hm= \mathrm{div}\,V\,dt$, взятых соответственно по поверхности $S_1\cup S_2$ 
и~по телу~T. По лемме~2.1, $\mathrm{div}\, V_1\hm= \mathrm{div}\, V_2\hm=0$. Так как $\mathrm{div}\, V\hm= \mathrm{div}\, 
V_1\hm+ \mathrm{div}\, V_2$, то $\mathrm{div}\, V\hm=0$, поэтому интеграл по замкнутой поверхности 
$S_1\cup S_2$ равен нулю, что доказывает его инвариантность: $I(S_1)\hm= I(S_2)$.
     
     \smallskip
     
     \noindent
     \textbf{Теорема~3.1.} \textit{Пусть в~области~$G$ существует поле 
экстремальных поверхностей с~коэффициентами $(p(x,z), q(x^\prime,y))$, включающее 
экстремаль~$z^*, y^*$ задачи}~(\ref{e3.1}). \textit{Если выполнено усиленное 
условие Ле\-жанд\-ра по переменным $w\hm= (z_x, y_x)$, то эта экстремаль является 
слабым локальным минимумом функционала}~(\ref{e3.1}). 
    
    \smallskip
    
    \noindent
    Д\,о\,к\,а\,з\,а\,т\,е\,л\,ь\,с\,т\,в\,о\,.\ \  Рассмотрим величину
    
    \vspace*{-2pt}
    
    \noindent
 \begin{multline*}
 \Pi(S)= {}\\
 {}=\int\limits_S \left(f(x,z,y,p,q)+\left( z_x-p, f_{z_x}\left(x,z,y,p,q\right)\right)+{}\right.\\
\left. {}+ \left( y_x-q, f_{y_x}(x,z,y,p,q)\right)\right)ds,
 \end{multline*}

\vspace*{-2pt}


\noindent
являющуюся аналогом выражения~(\ref{e2.5prime}). Не ограничивая общности, 
можно считать, что $f(x,z,y,p,q)\hm\leq 0$ в~области~$G$. Тогда найденный 
инвариант $I(S)$ (см.\ лемму~3.1) связан с~интегралом $\Pi(S)$ соотношением 
$\Pi(S)\hm\geq I(S)$. Отсюда приращение функционала задачи~(\ref{e3.1}) 
оценивается с~помощью следующего неравенства:

\vspace*{-2pt}

\noindent
\begin{multline}
J(z,y)-J(z^*, y^*) = J(z,y)-I(S)\geq{}\\
{}\geq J(z,y)-\Pi(S)\,.
\label{e3.2}
\end{multline}

\vspace*{-2pt}

\noindent
Как показано при доказательстве теоремы~2.1, применение усиленного условия 
Лежандра к~интегранту функционала из правой части неравенства~(\ref{e3.2}) 
обосновывает его неотрицательность (см.\ также~(\ref{e2.6}). В~итоге $J(z,y)\hm\geq 
J(z^*, y^*)$, что и~требовалось обосно\-вать.

\section{Вариационные задачи, содержащие старшие производные}

    Пусть функционал вариационной задачи имеет вид:
\begin{equation}
J(z,y) =\int\limits_S f(x,z,z_x, z_{xx})\,dx \to \min\limits_{\substack{{z\vert_{\gamma S}=u_1;}\\
{z_x\vert_{\gamma S}=u_2}}}\,.
\label{e4.1}
\end{equation}
 %   
    Введем вспомогательные переменные $y\hm =z_x$ и~вмес\-то~(\ref{e4.1}) 
рассмотрим задачу условной оптимизации
    \begin{equation}
    \left.
    \begin{array}{rl}
    J(z,y) &=\displaystyle \int\limits_S f(x,z,y,y_x)\,dx\to \min\limits_{\substack{{z\vert_{\gamma 
S}=u_1;}\\ {z_x\vert_{\gamma S}=u_2}}}\,;\\[6pt]
    y&=z_x\,.
    \end{array}
    \right\}
    \label{e4.2}
    \end{equation}
    
    
    Для снятия фазового ограничения в~(\ref{e4.2}) воспользуемся правилом 
множителей Лагранжа~[4]. Получим вариационную задачу на минимакс, 
эквивалентную исходной задаче~(\ref{e4.1}):
    \begin{equation}
    \left.
    \begin{array}{c}
    \hspace*{-3mm}J(z,y,\lambda) =\displaystyle \!\int\limits_S \left[ f(x,z,y,y_x)+\lambda(z_x-
y)\right]dx\to {}\\[-6pt]
{}\hspace*{45mm}\to \min\limits_{z,y:}\mathop{\mathrm{sup}}\limits_{\lambda}\,;\\
    z\vert_{\gamma S} =u_1\,;\quad y\vert_{\gamma S} =u_2\,.
    \end{array}
    \!\right\}\!
    \label{e4.3}
    \end{equation}
В ней требуется найти седловую точку $(z^*, y^*; \lambda^*)$.

\columnbreak 
    
    Наряду с~искомым решением задачи рас\-смот\-рим экстремали 
критерия~(\ref{e4.3}), удовлетворяющие варьированным граничным условиям 
    $z\vert_\Gamma \hm= u_1\vert_\Gamma$, $z_x\vert_\Gamma\hm= 
u_2\vert_\Gamma$, $\Gamma\hm\subset \gamma S$. Тогда, применив следствие~3.2 
и~теорему~3.1, получим обобщение следствия~3.1. 
    
    \smallskip
    
    \noindent
    \textbf{Теорема~4.1.} \textit{Пусть существуют поля экстремалей $p(x,z)$, 
$q(x,y)$ и~$r(x,\lambda)$ функционала}~(\ref{e4.3}), \textit{вклю\-ча\-ющие экстремаль 
$(z^*, y^*; \lambda^*)$. Если подынтегральная функция в}~(\ref{e4.3}) 
\textit{выпукла по переменной $w\hm= (z_x, y_z)$, то эта экстремаль является 
седловой точкой, а~пара $(z^*, y^*)$~--- локальным минимумом 
в~задаче}~(\ref{e4.1}).
    
    \smallskip
    
    \noindent
    \textbf{Пример~4.1.} Рассмотрим следующую вариационную задачу:
    $$
    \left\{ 
    \begin{array}{l}
    J(z)=\displaystyle \int\limits_0^\pi \left[ -z^2 -
\fr{z_x^2}{2}+\fr{z^2_{xx}}{2}\right]dx\to \min\,;\\[6pt]
    z(0)=z_z(0)=0\,;\ z(\pi) =1\,;\ z_x(\pi)=0\,.
    \end{array}
    \right.
    $$ 
Требуется исследовать на локальный минимум экстремаль $z\hm= z^*$,
\begin{equation}
z=C_1 e^x +C_2 e^{-x}\cos(x) +C_3 e^{-x}\sin(x)\,,
\label{e4.4}
\end{equation}
в которой вектор констант $C\hm= \overline{C}$ равен
\begin{equation}
\left.
\begin{array}{rl}
\overline{C}_1&=\fr{e^\pi}{2(1+e^{2\pi})}\,;\\[9pt]
\overline{C}_2&= -\fr{2+e^{2\pi}}{2}\,\overline{C}_1\,;\\[6pt]
\overline{C}_3&= -\fr{\overline{C}_1}{2}\,.
\end{array}
\right\}
\label{e4.5}
\end{equation}
    
    В самом деле, необходимое условие седловой точки функционала~(\ref{e4.3}) 
выражается с~помощью системы уравнений Эйлера 
    \begin{equation}
    \left.
    \begin{array}{rl}
    y_{xx}+\lambda+y&=0\,,\\[3pt]
    \lambda_x+2z&=0\,,\\
    z_x-y&=0\,.
    \end{array}
    \right\}    \label{e4.6}
    \end{equation}
Ищутся решения, удовлетворяющие однородным граничным условиям  $z(0)\hm=y(0)\hm=0$. 
Тогда поля\linebreak $p(x,z)$ и~$q(x,y)$, определяемые этими экстремальными поверхностями, 
задаются с~по\-мощью однопараметрического семейства функций~(\ref{e4.4}), 
$C_3\hm= -2C_1$, $C_2\hm= -C_1$, являющихся решениями \mbox{краевых} задач для 
системы дифференциальных уравнений~(\ref{e4.6}). 

     В свою очередь, поле $r(x,\lambda)$ вычисляется с~по\-мощью семейства 
функций $\lambda_C\hm= \lambda_C(x)$, по\-лу\-ча\-емых из первого уравнения 
системы~(\ref{e4.6}) подстановкой функции $y\hm= z_x$ (см.~(\ref{e4.4})). Таким 
образом, выполнены все условия тео\-ре\-мы~4.1 и~следствия~3.2. Поэтому 
экстремаль~(\ref{e4.4}), (\ref{e4.5}), $C\hm= \overline{C}$, является слабым 
локальным минимумом рассмотренного функционала.

\section{Заключение }

    Вариационное исчисление не утратило своей\linebreak актуальности. Например, 
в~современных исследованиях ищут обобщенные решения дифференциальных 
нелинейных уравнений не интегрированием, а с~применением вспомогательной 
\mbox{вариационной} задачи. Достаточные условия служат проведению более тонкого 
анализа свойств найден\-ной экстремальной поверхности с~целью установления ее 
оптимальности. Существование поля экстремалей обеспечено следующим 
обобщением условия Якоби. Требуется, чтобы возмущение краевых ограничений 
исходной вариационной задачи приводило не только к~однозначности 
возникновения новых экстремалей, но и~исключало их попарные пересечения.
     
{\small\frenchspacing
 {%\baselineskip=10.8pt
 %\addcontentsline{toc}{section}{References}
 \begin{thebibliography}{9}
\bibitem{1-vas}
\Au{Эльсгольц Л.\,Э.} Дифференциальные уравнения и~вариационное исчисление.~--- 
М.: Наука, 1969. 424~с.
\bibitem{2-vas}
\Au{Янг Л.} Лекции по вариационному исчислению и~оптимальному управлению~/ Пер. с~англ.~--- 
М.: Мир, 1974. 488~с.
(\Au{Young~L.\,C.} {Lectures on the calculus of variations and optimal control theory}.~--- Saunders, 1969. 331~p.)
\bibitem{3-vas}
\Au{Ильин В.\,А., Позняк~Э.\,Г.} Основы математического анализа.~--- М.: 
Наука, 2000.  Ч.~II. 448~с.
\bibitem{4-vas}
\Au{Васильев Ф.\,П.} Численные методы решения экстремальных задач.~--- М.: 
Наука, 1980. 520~с. 
\end{thebibliography}

 }
 }

\end{multicols}

\vspace*{-6pt}

\hfill{\small\textit{Поступила в~редакцию 13.01.22}}

\vspace*{8pt}

%\pagebreak

%\newpage

%\vspace*{-28pt}

\hrule

\vspace*{2pt}

\hrule

%\vspace*{-2pt}

\def\tit{ON EXTREMUM SUFFICIENT CONDITIONS IN~MULTIDIMENSIONAL VARIATION 
CALCULUS PROBLEMS}


\def\titkol{On extremum sufficient conditions in~multidimensional variation 
calculus problems}


\def\aut{N.\,S.~Vasilyev}

\def\autkol{N.\,S.~Vasilyev}

\titel{\tit}{\aut}{\autkol}{\titkol}

\vspace*{-8pt}


\noindent
N.\,E.~Bauman Moscow State Technical University, 5-1, 2nd Baumanskaya Str., Moscow 105005, 
Russian Federation


\def\leftfootline{\small{\textbf{\thepage}
\hfill INFORMATIKA I EE PRIMENENIYA~--- INFORMATICS AND
APPLICATIONS\ \ \ 2022\ \ \ volume~16\ \ \ issue\ 3}
}%
 \def\rightfootline{\small{INFORMATIKA I EE PRIMENENIYA~---
INFORMATICS AND APPLICATIONS\ \ \ 2022\ \ \ volume~16\ \ \ issue\ 3
\hfill \textbf{\thepage}}}

\vspace*{3pt} 





\Abste{Variation principles give formalization and general approach to construct and study the models from 
different fields of knowledge. They provide system presentations about theories origins. In the models, 
sought-for solution is a~stationary point of a~criterion. Its search on the basis of necessary conditions 
should not accomplish problem investigation. Sufficient conditions are needed to assert its optimality. In 
natural sciences, such results substantiate principles of energy or Hamilton's action minimization. In the 
paper, invariant surface integrals discovery gave possibility to prove minimum availability in  
multidimensional variation calculus problems. The functional in the problems may depend on several 
unknown functions of many variables and their high-order derivatives. Classical theorems are 
generalized.}

\KWE{extremal; extreme hypersurface; field of normals; divergence and flow of a vector field; 
differential form; external differentiation; integral invariance; Lagrange multiplier}

\DOI{10.14357/19922264220305} 

%\vspace*{-16pt}

%\Ack
%\noindent




%\vspace*{4pt}

  \begin{multicols}{2}

\renewcommand{\bibname}{\protect\rmfamily References}
%\renewcommand{\bibname}{\large\protect\rm References}

{\small\frenchspacing
 {%\baselineskip=10.8pt
 \addcontentsline{toc}{section}{References}
 \begin{thebibliography}{9}
\bibitem{1-vas-1}
\Aue{El'sgol'ts, L.\,E.} 1969. \textit{Differentsial'nye uravneniya i~va\-ri\-a\-tsi\-on\-noe ischislenie} 
[Differential equations and the calculus of variations]. Moscow: Nauka. 424~p.
\bibitem{2-vas-1}
\Aue{Young, L.\,C.} 1969. \textit{Lectures on the calculus of variations and optimal control theory}. Saunders. 331~p.
\bibitem{3-vas-1}
\Aue{Il'in, V.\,A., and E.\,G.~Poznyak.} 2000. \textit{Osnovy ma\-te\-ma\-ti\-che\-sko\-go analiza} [Basics 
of mathematical analysis]. Moscow: Nauka.   Part~II. 448~p.
\bibitem{4-vas-1}
\Aue{Vasil'ev, F.\,P.} 1980. \textit{Chislennye metody resheniya eks\-tre\-mal'\-nykh zadach} [Numerical 
methods for solving extremal problems]. Moscow: Nauka. 520~p.
\end{thebibliography}

 }
 }

\end{multicols}

\vspace*{-6pt}

\hfill{\small\textit{Received January 13, 2022}}

\Contrl

\noindent
\textbf{Vasilyev Nikolai S.} (b.\ 1952)~--- Doctor of Science in physics and mathematics, professor, 
N.\,E.~Bauman Moscow State Technical University, 5-1, 2nd Baumanskaya Str., Moscow 105005, 
Russian Federation; \mbox{nik8519@yandex.ru}

  

\label{end\stat}

\renewcommand{\bibname}{\protect\rm Литература}    