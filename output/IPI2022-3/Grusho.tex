\def\stat{grusho}

\def\tit{МЕТАДАННЫЕ В ЗАЩИЩЕННОМ ЭЛЕКТРОННОМ ДОКУМЕНТООБОРОТЕ}

\def\titkol{Метаданные в защищенном электронном документообороте}

\def\aut{А.\,А.~Грушо$^1$, Н.\,А.~Грушо$^2$, Е.\,Е.~Тимонина$^3$}

\def\autkol{А.\,А.~Грушо, Н.\,А.~Грушо, Е.\,Е.~Тимонина}

\titel{\tit}{\aut}{\autkol}{\titkol}

\index{Грушо А.\,А.}
\index{Грушо Н.\,А.}
\index{Тимонина Е.\,Е.}
\index{Grusho A.\,A.}
\index{Grusho N.\,A.}
\index{Timonina E.\,E.}


%{\renewcommand{\thefootnote}{\fnsymbol{footnote}} \footnotetext[1]
%{Публикация выполнена при поддержке Программы стратегического академического лидерства РУДН.}}


\renewcommand{\thefootnote}{\arabic{footnote}}
\footnotetext[1]{Федеральный исследовательский центр <<Информатика и управление>> Российской академии наук, \mbox{grusho@yandex.ru}}
\footnotetext[2]{Федеральный исследовательский центр <<Информатика и управление>> Российской академии наук, \mbox{info@itake.ru}}
\footnotetext[3]{Федеральный исследовательский центр <<Информатика и управление>> Российской академии наук, 
\mbox{eltimon@yandex.ru}}

%\vspace*{-6pt}





  
  

  
  \Abst{Работа посвящена исследованию проблем информационной безопасности (ИБ)
сетецентрических систем электронного документооборота (ЭДО) в~распределенной информационной 
системе (РИС) предприятия. Электронный документооборот представляет собой защищенную 
распределенную систему информационных технологий (ИТ), каждая из которых порождается частью 
организационной структуры предприятия, т.\,е.\ существует защищенная связь, обеспечивающая 
функционал документооборота.
  Для обеспечения ИБ использована концепция метаданных (МД), 
определенная ранее в~работах авторов. В~отличие от ранних приемов использования 
МД, в~ЭДО применяется множество вариантов 
ИТ, определяемых подмножеством участвующих в~работе над 
документом пользователей и~структурой поручений, по\-рож\-да\-емых исполнением документов. 
  Авторами предложена модульная конструкция МД, которая позволяет решать 
проблему синтеза и управления многообразием возникающих вариантов ИТ ЭДО и~обеспечением их~ИБ.}
   
  \KW{информационная безопасность; метаданные; электронный документооборот}
  
  \DOI{10.14357/19922264220313} 
  
%\vspace*{-3pt}


\vskip 10pt plus 9pt minus 6pt

\thispagestyle{headings}

\begin{multicols}{2}

\label{st\stat}
  
  \section{Введение }
  
  Одним из распространенных классов РИС стали сетецентрические системы~[1], поэтому проб\-ле\-мы контроля 
и~обеспечения ИБ в~сетецентрической сис\-те\-ме 
особенно актуальны. Для исследования этих проблем выбрана сетецентрическая 
сис\-те\-ма ЭДО РИС предприятия. 

Электронный документооборот 
представляет собой защищенную распределенную сис\-те\-му ИТ, в~которых каждая ИТ-струк\-ту\-ра порождается \mbox{частью} 
организационной структуры предприятия и~существует защищенная связь, 
обеспечивающая функционал ЭДО. 
  
  Функционал ЭДО состоит из:
  \begin{itemize}
\item защищенных функций взаимодействия с~внешним окружением, не 
принадлежащим РИС предприятия (получение и отсылка документов);
\item базы данных документов (архив документов);
\item функций управления документооборотом;
\item функций формирования поручений;
\item функций доставки поручений исполнителям, контроля исполнения 
поручений и времени исполнения;
\item функций идентификации и аутентификации, функций электронной 
подписи (ЭП), удосто\-ве\-ря\-юще\-го центра, центра генерации и распределения 
криптографических ключей;
\item функций обеспечения целостности и неотказуемости движения 
документов. 
\end{itemize}

  Предполагается следующая модель угроз.
  \begin{enumerate}[1.]
  \item Угрозы конфиденциальности:
  \begin{itemize}
\item содержание документов и файлов с отчетами по поручениям, 
связанным с документами, считается конфиденциальной информацией, 
подлежащей защите;
\item полный маршрут движения документа считается конфиденциальной 
информацией, подлежащей защите;
\item содержание документов архива и текущих баз данных ЭДО считается 
конфиденциальной информацией, подлежащей защите.
\end{itemize}
  \item Угрозы целостности:
  \begin{itemize}
\item содержание документов и файлов с отчетами по поручениям, 
связанным с документами, считается информацией, подлежащей защите 
целостности;
\item целостность маршрута движения документа должна быть защищена;
\item должна быть обеспечена безотказность при работе с документами;
\item функции идентификации и аутентификации сотрудников 
(пользователей компьютерных рабочих мест), работающих с документами, 
не должны нарушаться.
\end{itemize}
  \item Угрозы доступности:
  \begin{itemize}
\item доступность реализации маршрутов документов должна быть 
обеспечена; 
\item должны быть зарезервированы средства доступности к 
документам;
\item должен быть обеспечен контроль времени работы над поручениями, 
связанными с~каж\-дым документом.
\end{itemize}
  \item Угрозы потери ответственности при работе с~документами:
  \begin{itemize}
  \item необходимо обеспечить идентификацию авторства любых действий при 
работе с документами или отсутствия определенных действий.
  \end{itemize}
  
\end{enumerate}

  Для реализации ЭДО в РИС необходима дополнительная информация и 
инструменты защиты ЭДО как распределенной ИТ. Для 
обеспечения ИБ распределенных ИТ в~работах~[2, 3] разработан подход, 
основанный на МД. Суть подхода состоит в~том, что управление 
соединениями и~функции контроля выполнения ИТ реализуются централизованно 
на основе модели ИТ~[4]. Такое управление организовано на знаниях 
о~взаимодействиях субъектов ИТ и~их расположении на хостах распределенной 
системы компьютеров. Эти знания формируются и~хранятся 
в~задаче~$\mathcal{M}$, изолированой от хостов и~функций, которые на них 
выполняются. Такая изоляция служит основой безопас\-ности управ\-ле\-ния 
выполнением ИТ, использующей даже небезопасную сеть связи. Используя 
данные задачи~$\mathcal{M}$, следующая задача~$\mathcal{N}$ управляет 
соединениями и~контролем развития ИТ. Обмен информацией между 
задачами~$\mathcal{M}$ и~$\mathcal{N}$ не является однонаправленным 
обменом, но это не может нарушить безопасность управления~[5].
  
  В рассматриваемом случае отличие от исследованной ранее модели ИТ состоит 
в том, что сама ИТ ЭДО представляет собой динамически изменяемую структуру, 
решающую ограниченный набор задач с постоянно меняющимся набором 
исполнителей, поэтому постоянно меняется модель маршрутов выполнения ИТ.
  
  Цель работы состоит в построении архитектуры ЭДО на основе МД и 
обоснование безопасности предложенной модели ЭДО.
  
  \section{Метаданные в~электронном документообороте}
  
  Рассмотрим следующую дополнительную информацию (МД) для 
автоматической реализации функций ЭДО.
  \begin{enumerate}[1.]
\item  Матрица $M_1$~--- $(0, 1)$-мат\-ри\-ца, строки и~столб\-цы которой 
помечены идентификаторами пользователей $U_1, U_2,\ldots$, потенциально 
участвующих в ЭДО, где~1 на месте ($U_k, U_m$) означает право 
передавать поручение от пользователя~$U_k$ пользователю~$U_m$. Поручением 
может быть реализация поручения вышестоящего начальника либо продолжение 
ИТ, которая утверждена в~организации.
  \item Далее пользователь отождествляется с рабочей станцией, на которой этот 
пользователь работает, рабочая станция является хостом сети организации, на 
которой установлен HSM (Host Security Module)~[6]. В HSM установлен 
уникальный для пользователя ключ общего для всех алгоритма шифрования и 
комплекс для проверки ЭП и~установки ЭП данного 
пользователя. При работе HSM должна проводиться идентификация 
и~аутентификация пользователя.
  \item Для каждого документа при создании поручения также устанавливается 
глубина распространения документа, т.\,е.\ число пользователей, которые могут 
далее передавать сам документ. По умолчанию глубина равна~1 или соответствует 
нормативу для исполняемой ИТ.
  \item Для каждого документа при создании поручения также устанавливается 
время исполнения поручения. По умолчанию время исполнения не 
регламентируется или соответствует нормативу для исполняемой ИТ.
  \item Первое поручение на дальнейшую работу с поступившим извне 
документом устанавливает пользователь, указанный в адресе поступившего 
документа.
  \item  Каждое поручение включает два изолированных информационных 
фрагмента: 
  \begin{itemize}
\item[(а)] первый фрагмент содержит: указатель, кому дается поручение, время 
выдачи поручения и время его исполнения (по умолчанию время не 
регламентируется); 
\item[(б)] второй фрагмент содержит документ (опционально) и сообщение с 
содержанием поручения или отчета. 
\end{itemize}
  При отсутствии содержания в сообщении по умолчанию предполагается 
ознакомление или априорное знание пользователем, получившим документ, что 
нужно делать далее. Последнее чаще относится к выполнению утвержденной в 
организации ИТ. Сообщение и документ подписываются ЭП. Первый фрагмент 
всегда присутствует в открытом виде, а второй всегда защищен шифром и ЭП.
  \item Если глубина распространения документа исчерпана или поручения 
выполнены до наступления предельной глубины, а других нет, то документ и все 
сообщения поступают в архив или документ и приложенное к нему непустое 
сообщение (отчет) поступает к источнику поручения.
  \end{enumerate}
  
\section{Архитектура электронного документооборота}

   В сетецентрической системе, по определению, существует центр управления 
ЭДО. Этот центр в~ЭДО состоит из четырех подсистем. Первая подсистема~$Y$ 
управляет поручениями и движением документов; вторая подсистема решает 
задачу~$\mathcal{M}$ в метаданных, т.\,е.\ управляет адресацией движения 
документов и временем исполнения (составляет и~контролирует первый фрагмент 
объекта, направляемого для выполнения поручения); третья под\-сис\-те\-ма 
соответствует задаче~$\mathcal{N}$ в~МД и~управ\-ля\-ет соединениями 
с~пользователями; четвертая под\-сис\-те\-ма управ\-ля\-ет криптографией, т.\,е.\ генерирует 
и~распределяет ключи в~HSM и~контролирует генерацию и~функции ЭП 
(выполняет функции удосто\-ве\-ря\-юще\-го центра). 
  
  Основной принцип в обеспечении ИБ ЭДО состоит в том, что любое 
взаимодействие с пользователем состоит из простого ориентированного цикла:
  %
  задача~$\mathcal{N}$ организует связь с пользователем и отправляет ему 
документы и сообщения, которые сопровождали последовательность сообщений, 
предшествовавших данной связи с пользователем. 

Вся передаваемая информация 
второго фрагмента зашифрована на ключе пользователя и подписана ЭП третьей 
подсистемы (задача~$\mathcal{N}$). Первый фрагмент фактически образует 
заголовок передаваемого объекта. 
  
  Использование криптографических функций относится к задаче~$\mathcal{N}$ 
в управлении ЭДО и к блокам HSM у пользователей. Задача~$\mathcal{N}$ имеет 
перечень ключей пользователей. Перед организацией сеанса с пользователем весь 
фрагмент содержания шифруется на ключе пользователя и подписывается. 
Реакция пользователя после расшифрования и проверки подписи состоит в том, 
чтобы создать сообщение и отправить его задаче~$\mathcal{N}$ в зашифрованном 
виде, включая хеш принятого сообщения, подписав сообщение ЭП пользователя. 
Этот протокол также обеспечивает аутентификацию пользователя. 
Задача~$\mathcal{N}$ проверяет ЭП, расшифровывает сообщение и открывает 
хеш от посланного с поручением сообщения, что служит аутентификацией 
пользователя, приславшего сообщение. Затем задача~$\mathcal{N}$ направляет 
полученное сообщение задаче~$\mathcal{M}$. Задача~$\mathcal{M}$ знает, кто 
поручил передать пользователю информацию, и направляет ответ 
в~подсистему~$Y$ этому пользователю~--- создателю поручения. Здесь возникает 
причинно-следственная связь между отправителем и получателем поручения.
  
  Основной элемент архитектуры связан с созданием и исполнением поручений. 
Предлагается создавать эту подсистему в~$Y$ на модульном принципе. Каждый 
пользователь, который получает поручение, автоматически с поручением получает 
изолированный программный модуль с уникальным идентификатором. Вместе с 
поручением в этот модуль записываются исходный документ и все сообщения, 
которые связаны с предыдущей работой других пользователей. 
  
  Создание поручения и сообщения к нему может потребовать дополнительной 
информации из архива или из других источников. Информация от внешних 
источников обрабатывается на предмет безопасности по другим технологиям, не 
входящим в ЭДО (например, отражается на отдельном компьютере). 
  
  Каждый модуль имеет два входа и один выход. Вход, порождающий модуль, 
содержит поручение пользователю. Второй вход служит для получения реакции 
получателя поручения (отчетов может быть много). Выход служит для передачи 
информации следующему исполнителю или для передачи команды завершения 
работы. В~последнем случае исходный документ и все последующие сообщения 
заносятся в архив, который сам устроен на принципах блокчейна~[7]. 
Отработанный модуль уничтожается.
  
  Порождение поручений в общем случае описывается корневым деревом, 
вершинами которого\linebreak служат пользователи, а именно: пользователи, получающие 
поручения с метками времени, и пользователи, создающие эти поручения (идущие 
в соответствующие модули); ребра дерева \mbox{соответствуют} поручениям 
с~уникальными идентификаторами и~меткой времени, а~также указателем 
идентификатора документа (идентификатора модуля), откуда поступило 
поручение. 
  
  Даже если пользователи повторяются, то цик\-лы не могут возникнуть из-за 
меток времени и~идентификаторов документов и~сообщений, поступивших от 
источников поручений. Такие графы носят временный характер и уничтожаются 
после окончания работы с~документом и~сохранением его следа в архиве. Граф 
ограничен возможностями мат\-ри\-цы~$M_1$ и~отсутствием пользователей, 
которым должно быть направлено поручение. В~этих случаях пользователь 
должен изменить адресацию поручения и~найти новый способ дальнейшей работы 
над поручениями предыду\-щих пользователей. 

\vspace*{-6pt}
  
  \section{Информационная безопасность электронного документооборота}
  
  \vspace*{-2pt}
  
  Конфиденциальность содержания документов обеспечивается 
ответственностью пользователей и~выполнением правил доступа. В~частности, 
ответственность за разглашение конфиденциальной информации сле\-ду\-юще\-му 
участнику цепочки поручений несет пользователь, дающий этому участнику 
поручение. Когда ответственность определена, можно опираться на правила 
политик ограничения доступа, пользуясь достоверно опре\-де\-ля\-емой 
ответственностью за нарушение этих правил. Индивидуальные ключи каждого 
пользователя обеспечивают конфиденциальность при сетевом взаимодействии. 
Работа в~индивидуальном модуле изолирует пользователя от других 
пользователей, возможно заинтересованных в~ознакомлении с~конфиденциальной 
информацией. Защита конфиденциальной информации в~графах поручений 
других полностью автоматических компонентах зависит от проверенности 
программного обеспечения в подсистеме~$Y$ и~на рабочих станциях 
пользователей, физической защиты подсистемы~$Y$ и HSM. Открытый текст 
документов и~сообщений к~ним появляется у~нескольких пользователей. Кража 
конфиденциальной информации одним из них создает условия отсутствия 
ответственности за кражу. Эта проб\-ле\-ма решается ответственностью 
пользователя, давшего поручение. Однако поиск и выявление истинного 
виновника кражи требуют дополнительной информации, выходящей за пределы 
ЭДО. 
  
  Полный маршрут исполнения поручений доступен только на уровне~$Y$, 
и~опять считается, что этот уровень безопасен. Защита маршрута не всегда 
служит качественной защитой от разрушения персональной ответственности, так 
как пользователю, совершившему кражу, достоверно не известно, кто еще имел 
доступ к конфиденциальной информации в цепочке его уровня и далее. Однако 
постороннему пользователю маршрут движения документа может дать частичную 
информацию о содержании конфиденциальной информации.
  
  Конфиденциальность баз данных и архива обеспечивается средствами, не 
входящими в ЭДО.
  
  Защита целостности обеспечивается сохранением всех документов и их ЭП в 
ходе выполнения ИТ и при их хранении в архиве. Блокчейн-технологии 
обеспечивают невозможность изъятия документов или их частей. Наличие ЭП и 
стандартных протоколов обеспечивает безотказность всех компонентов 
технологии~ЭДО.
  
  Целостность графов поручений обеспечивается безопасностью подсистемы~$Y$ и~надежностью автоматических действий управления ЭДО. 
  Целостность и~надежность идентификации и аутентификации определяется надежностью 
задач~$\mathcal{M}$ и~$\mathcal{N}$, а~также работой HSM.
  
  Доступность обеспечивается надежностью всех компонент ЭДО. Выявление 
сбоев доступности на сетевом и уровне рабочих станций возможно за счет 
модульности взаимодействий пользователя, да\-юще\-го поручения, 
и~поль\-зо\-ва\-те\-лей-ис\-пол\-ни\-те\-лей поручений. В~условиях стихийных или 
технологических катастроф всегда возможен переход к~ручному исполнению 
документооборота.
  
  Нарушения ответственности частично рас\-смат\-ри\-ва\-лись выше. Модульный 
характер при\-чин\-но-след\-ст\-вен\-ных отношений дающего поручения 
и~исполняющего поручения позволяют создавать данные, которые в~памяти 
блокчейна однозначно позволяют выявить нарушения правил работы 
с~документами, в частности пропуски выполнения необходимых действий, 
и~определить виновного.
  
  Приведенный анализ показывает, что пе\-ре\-чис\-лен\-ные штатные средства защиты 
информации и~особенности построенной архитектуры позволяют обеспечить ИБ 
от заданного перечня угроз. Использование концепции МД, специальной 
архитектуры ЭДО, криптографии, модульной конструкции управ\-ле\-ния позволило 
обосновать безопас\-ность модели ЭДО. 

\vspace*{-6pt}
  
  \section{Заключение }
  
  \vspace*{-2pt}
  
  В работе построена модель ЭДО в виде множества защищенных ИТ в РИС. Для 
обеспечения ИБ использована концепция МД, определенная ранее в работах 
авторов. В~отличие от ранних приемов использования МД, в ЭДО используется 
множество вариантов ИТ, определяемых подмножеством задействованных 
пользователей и структурой поручений, порождаемых исполнением документов. 
  
  Авторами предложена модульная конструкция МД, которая позволяет решать 
проблему синтеза и~управ\-ле\-ния многообразием возникающих вариантов ИТ ЭДО 
и~обеспечением их ИБ. Пользователи через соответствующие модули создают 
поручения другим пользователям. Отображение взаимодействий этих 
пользователей формирует корневые деревья. Меняющийся состав модулей 
и~деревьев позволяет реализовать различные варианты работы с~документами. 
Ребра деревьев соответствуют поручениям и~отчетам об их выполнении. При этом 
отчеты могут носить комплексный характер, т.\,е.\ собираться из сообщений, 
составленных всеми пользователями, выполнявшими поручения. Иными словами, 
сборка каждого отчета возможна только по цепочкам путей от вершин к~корню 
дерева.
  
{\small\frenchspacing
 {%\baselineskip=10.8pt
 %\addcontentsline{toc}{section}{References}
 \begin{thebibliography}{9}
\bibitem{1-gru}
\Au{Володин Р.\,С., Золотарева~Е.\,С., Мешков~А.\,М.} Сетецентрическое управление: понятие 
и сущность~// Ж.~У. Экономика. Управление. Финансы, 2018. Вып.~2. С.~51--64. 
\bibitem{2-gru}
\Au{Grusho A., Grusho~N., Zabezhailo~M., Zatsarinny~A., Timonina~E.} Information security of 
SDN on the basis of metadata~// Computer network security~/ Eds. J.~Rak, J.~Bay, I.\,V.~Kotenko, 
\textit{et al.}~--- Lecture notes in computer science ser.~--- Springer, 2017. Vol.~10446. P.~339--347.
\bibitem{3-gru}
\Au{Grusho A., Grusho~N., Timonina~E.} Information flow control on the basis of meta data~// 
Distributed computer and communication networks~/ Eds. V.\,M.~Vishnevskiy, K.\,E.~Samouylov, D.\,V.~Kozyrev.~--- Lecture notes in computer 
science ser.~--- Springer, 2019. Vol.~11965. P.~548--562. 
\bibitem{4-gru}
\Au{Grusho A.\,A., Timonina~E.\,E., Shorgin~S.\,Ya}. Modelling for ensuring information security of 
the distributed information systems~// 31st European Conference on Modelling and Simulation 
Proceedings.~--- Dudweiler, Germany: Digitaldruck Pirrot GmbH, 2017. P.~656--660.
\bibitem{5-gru}
\Au{Грушо А.\,А., Применко~Э.\,А., Тимонина~Е.\,Е.} Теоретические основы компьютерной 
безопасности.~--- М.: Академия, 2009. 272~с.
\bibitem{6-gru}
\Au{Meyer C.\,H., Matyas~S.\,M.} Cryptography: A~new dimension in computer data security~--- 
a~guide for the design and implementation of secure systems.~--- Hoboken, NJ, USA: Wiley, 1982. 
755~p.
\bibitem{7-gru}
\Au{Grusho A., Piskovsky~V., Zabezhailo~M.} System to track access in digital economy systems~// 
Law Digital Technologies, 2021. Vol.~1. Iss.~1. Р.~10--18.

  \end{thebibliography}

 }
 }

\end{multicols}

\vspace*{-6pt}

\hfill{\small\textit{Поступила в~редакцию 15.07.22}}

\vspace*{8pt}

%\pagebreak

%\newpage

%\vspace*{-28pt}

\hrule

\vspace*{2pt}

\hrule

%\vspace*{-2pt}

\def\tit{METADATA IN~SECURE ELECTRONIC DOCUMENT MANAGEMENT}


\def\titkol{Metadata in~secure electronic document management}


\def\aut{A.\,A.~Grusho, N.\,A.~Grusho, and~E.\,E.~Timonina}

\def\autkol{A.\,A.~Grusho, N.\,A.~Grusho, and~E.\,E.~Timonina}

\titel{\tit}{\aut}{\autkol}{\titkol}

\vspace*{-8pt}


\noindent
Federal Research Center ``Computer Science and Control'' of the Russian Academy of Sciences,  
44-2~Vavilov Str., Moscow 119133, Russian Federation



\def\leftfootline{\small{\textbf{\thepage}
\hfill INFORMATIKA I EE PRIMENENIYA~--- INFORMATICS AND
APPLICATIONS\ \ \ 2022\ \ \ volume~16\ \ \ issue\ 3}
}%
 \def\rightfootline{\small{INFORMATIKA I EE PRIMENENIYA~---
INFORMATICS AND APPLICATIONS\ \ \ 2022\ \ \ volume~16\ \ \ issue\ 3
\hfill \textbf{\thepage}}}

\vspace*{3pt} 



\Abste{The paper is devoted to the study of information security problems of network-centric 
electronic document management systems in the distributed information system of the enterprise. 
Electronic document management is a secure distributed system of information technologies, each of 
which is generated by part of the organizational structure of the enterprise, i.\,e., there is a secure connection 
that provides the functionality of document management. To ensure information security, the concept 
of metadata defined earlier in the works of the authors was used. Unlike earlier works that used the 
concept of metadata, electronic document management uses many variants of information technologies 
determined by a subset of users participating in the work on the document and the structure of orders 
generated by the execution of documents. New in the work is the modular design of metadata which 
allows solving the problem of synthesis and management of a variety of emerging variants of 
information technologies of electronic document management and ensuring their information 
security.}

\KWE{information security; metadata; electronic document management}

\DOI{10.14357/19922264220313} 

%\vspace*{-16pt}

% \Ack
 % \noindent


\vspace*{4pt}

  \begin{multicols}{2}

\renewcommand{\bibname}{\protect\rmfamily References}
%\renewcommand{\bibname}{\large\protect\rm References}

{\small\frenchspacing
 {%\baselineskip=10.8pt
 \addcontentsline{toc}{section}{References}
 \begin{thebibliography}{9}
 
 \vspace*{-4pt}
 
\bibitem{1-gru-1}
\Aue{Volodin, R.\,S., E. \,S.~Zolotareva, and A. \,M.~Meshkov.} 2018. Setetsentricheskoe 
upravlenie: ponyatie i~sushchnost' [Network-centric control: Concept and essence]. \textit{Zh. 
U. Ekonomika. Upravlenie. Finansy} [J.~U. Economy. Management. Finance]  
2:51--64.
\bibitem{2-gru-1}
\Aue{Grusho, A., N.~Grusho, M.~Zabezhailo, A.~Zatsarinny, and E.~Timonina.} 2017. Information 
security of SDN on the basis of meta data. \textit{Computer network security}. Eds. J.~Rak, J.~Bay, 
I.\,V.~Kotenko, \textit{et al.} Lecture notes in computer science ser. Springer. 10446:339--347.
\bibitem{3-gru-1}
\Aue{Grusho, A.\,A., N.\,A.~Grusho, and E.\,E.~Timonina.} 2019. Information flow control on the 
basis of meta data. \textit{Distributed computer and communication networks}. Eds. V.\,M.~Vishnevskiy, K.\,E.~Samouylov, and 
D.\,V.~Kozyrev. Lecture notes in computer science ser. Springer. 11965:548--562.
\bibitem{4-gru-1}
\Aue{Grusho, A. \,A., E. \,E.~Timonina, and S. \,Ya.~Shorgin.} 2017. Modelling for ensuring 
information security of the\linebreak\vspace*{-12pt}

\pagebreak

\noindent
 distributed information systems. \textit{31st European Conference on 
Modelling and Simulation Proceedings}. Dudweiler, Germany: Digitaldruck Pirrot GmbHP. 656--660. 
\bibitem{5-gru-1}
\Aue{Grusho, A. \,A., E. \,A.~Primenko, and E. \,E.~Timonina.} 2009. \textit{Teoreticheskie osnovy 
komp'yuternoy bezopasnosti} [Theoretical foundations of computer security]. Moscow: Akademiya. 
272~p.
\bibitem{6-gru-1}
\Aue{Meyer, C. \,H., and S. \,M.~Matyas.} 1982. \textit{Cryptography: A~new dimension in computer 
data security~--- a~guide for the design and implementation of secure systems}. Hoboken, NJ: Wiley. 
755~p.
\bibitem{7-gru-1}
\Aue{Grusho, A., V.~Piskovsky, and M.~Zabezhailo.}  2021. System to track access in digital economy 
systems. \textit{Law  Digital Technologies} 1(1):10--18.
\end{thebibliography}

 }
 }

\end{multicols}

\vspace*{-6pt}

\hfill{\small\textit{Received July 15, 2022}}

\Contr

\noindent
\textbf{Grusho Alexander A.} (b.\ 1946)~--- Doctor of Science in physics and mathematics, professor, 
principal scientist, Institute of Informatics Problems, Federal Research Center ``Computer Science and 
Control'' of the Russian Academy of Sciences, 44-2~Vavilov Str., Moscow 119333, Russian 
Federation; \mbox{grusho@yandex.ru}

\vspace*{3pt}

\noindent
\textbf{Grusho Nikolai A.} (b.\ 1982)~--- Candidate of Science (PhD) in physics and mathematics, 
senior scientist, Institute of Informatics Problems, Federal Research Center ``Computer Science and 
Control'' of the Russian Academy of Sciences, 44-2~Vavilov Str., Moscow 119133, Russian 
Federation; \mbox{info@itake.ru}


\vspace*{3pt}

\noindent
\textbf{Timonina Elena E.} (b.\ 1952)~--- Doctor of Science in technology, professor, leading scientist, 
Institute of Informatics Problems, Federal Research Center ``Computer Science and Control'' of the 
Russian Academy of Sciences, 44-2~Vavilov Str., Moscow 119133, Russian Federation; 
\mbox{eltimon@yandex.ru}


\label{end\stat}

\renewcommand{\bibname}{\protect\rm Литература}    