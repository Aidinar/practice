
\def\stat{zeifman}

\def\tit{ОБ ОДНОМ ПОДХОДЕ К~ОЦЕНИВАНИЮ СКОРОСТИ СХОДИМОСТИ НЕСТАЦИОНАРНЫХ МАРКОВСКИХ МОДЕЛЕЙ СИСТЕМ 
ОБСЛУЖИВАНИЯ$^*$}

\def\titkol{Об одном подходе к~оцениванию скорости сходимости нестационарных марковских моделей систем 
обслуживания}

\def\aut{И.\,А.~Ковалёв$^1$, Я.\,А.~Сатин$^2$, А.\,В.~Синицина$^3$,  А.\,И.~Зейфман$^4$}

\def\autkol{И.\,А.~Ковалёв, Я.\,А.~Сатин, А.\,В.~Синицина,  А.\,И.~Зейфман}

\titel{\tit}{\aut}{\autkol}{\titkol}

\index{Ковалёв И.\,А.}
\index{Сатин Я.\,А.}
\index{Синицина А.\,В.}
\index{Зейфман А.\,И.}
\index{Kovalev I.\,A.}
\index{Satin Y.\,A.}
\index{Sinitcina A.\,V.}
\index{Zeifman A.\,I.}


{\renewcommand{\thefootnote}{\fnsymbol{footnote}} \footnotetext[1]
{Исследования в разд.~4 выполнены А.\,В.\,Синициной за 
счет гранта  Российского научного фонда (проект 21-71-30011).}}


\renewcommand{\thefootnote}{\arabic{footnote}}
\footnotetext[1]{Вологодский 
государственный университет; Московский центр фундаментальной и прикладной 
математики, \mbox{kovalev.iv96@yandex.ru}}
\footnotetext[2]{Вологодский государственный  университет,     \mbox{yacovi@mail.ru}}
\footnotetext[3]{Ярославский государственный университет им.\ П.\,Г.~Демидова,    \mbox{a\_korotysheva@mail.ru} }
\footnotetext[4]{Вологодский государственный  университет; Федеральный исследовательский центр <<Информатика и~управ\-ле\-ние>> 
Российской академии наук; Вологодский научный центр Российской академии наук; 
Московский центр фундаментальной и~прикладной математики, \mbox{a\_zeifman@mail.ru}}


%\vspace*{-6pt}


  
\Abst{Рассмотрено преобразование прямой системы 
Колмогорова, позволяющее получать простые оценки скорости сходимости для 
марковских цепей (м.\,ц.)\ с непрерывным временем, описывающих системы массового 
обслуживания. Кроме того, используется понятие логарифмической нормы операторной 
функции и связанные с ней оценки нормы матрицы Коши. Полученные результаты  
позволяют оценивать скорость сходимости для новых классов моделей, у которых 
преобразованная матрица не является существенно неотрицательной и  применение   
метода логарифмической нормы не гарантирует возможность получения оценок 
скорости сходимости.  Ранее для таких ситуаций применялся достаточно трудоемкий 
общий метод неравенств. 
     Сформулирована теорема о получении скорости сходимости при изменении 
интенсивностей матрицы, при сравнении с процессом рождения и гибели. В качестве 
примера исследована специальная нестационарная модель с групповым обслуживанием 
требований (обслуживание парами).}
    

    
\KW{скорость сходимости; эргодичность; 
логарифмическая норма; системы массового обслуживания}

\DOI{10.14357/19922264220310}
  
%\vspace*{-3pt}


\vskip 10pt plus 9pt minus 6pt

\thispagestyle{headings}

\begin{multicols}{2}

\label{st\stat}
    
        \section{Введение}
        
Одна из важнейших проблем при исследовании марковских моделей систем массового 
обслуживания~--- построение их вероятностных характеристик. Точное их вычисление 
является достаточно трудной задачей даже для стационарных моделей, 
а~в~нестационарном случае ситуация становится гораздо более сложной, в связи с чем 
неизбежно применение аппроксимационных подходов с теми или иными гарантиями их 
точности. В~частности, многие авторы используют методы дискретизации для 
построения переходных вероятностей неоднородных марковских цепей на конечных 
временн$\acute{\mbox{ы}}$х интервалах~[1--5]
%\cite{Arns2010}-\cite{Inoue}
или применяют предельные вероятностные характеристики~\cite{chak}. 

Первый подход 
в~принципе не гарантирует точность на произвольных временн$\acute{\mbox{ы}}$х интервалах, 
особенно в случае медленной сходимости, а второй требует умения вычислять эти 
предельные характеристики и, самое главное, наличия оценок, гарантирующих 
близость переходных характеристик к предельным.
Для их получения необходимо количественное исследование скорости сходимости для 
эргодических марковских цепей с непрерывным временем. Ранее (см., например, 
обзор в~\cite{Zeifman2020porto}) был развит подход, при котором прямая система 
Колмогорова
\begin{equation}
\label{eq_1}
\fr{d\vp}{dt}=A(t)\vp(t)
\end{equation}
для вероятностей состояний цепи <<подвергается>> следующим стандартным 
преобразованиям:
вначале исключается нулевое состояние, полагая
$$
p_0\left( t\right) =1-\sum\limits_{i\geq 1}p_i\left( t\right),
$$
получаем из~(\ref{eq_1})
\begin{equation*}
%\label{eq_4}
\fr{d\vz}{dt}=B(t)\vz(t)+\vf(t),
\end{equation*}
затем изучается соответствующая <<однородная>> сис\-те\-ма 
${d\vx}/{dt}=B(t)\vx(t)$,
затем применяется треугольное преобразование $w=Dx$, где

\noindent
\begin{equation*}
%\label{eq_9}
D={ \left(
\begin{array}{cccc}
d_{1}   & d_{1}  & d_{1}   &   \cdots  \\
0       & d_{2}  & d_{2}   &   \cdots  \\
0       & 0      & d_{3}   &   \cdots  \\
\vdots  & \vdots & \vdots  &   \ddots  \\
\end{array}
\right) }
\end{equation*}
\noindent 
с положительными~$d_k$,
после чего изучается уже сис\-те\-ма 
$$
\fr{d\vw}{dt}=DB(t)D^{-1}\vw(t).
$$
%
Если при этом матрица $B_1(t)\hm=DB(t)D^{-1}$ оказывается существенно 
неотрицательной, то метод логарифмической нормы (см.~\cite{Zeifman2020porto}) 
поз\-во\-ля\-ет получить точные оценки для нормы решений этой сис\-те\-мы, а~значит, и~для 
ско\-рости схо\-ди\-мости ис\-ход\-ной\linebreak м.\,ц.

Если же ее матрица не является существенно неотрицательной, то  применение   
метода логарифмической нормы не гарантирует возможность получения оценок 
скорости сходимости, для таких ситуаций был разработан более общий метод 
неравенств (см., например,~[8--10]),
%\cite{Zeifman2020AMCS}-\cite{Zeifman2021MDPI}
применение которого очень трудоемко.

В настоящей работе описывается другой, существенно более прос\-той подход, 
поз\-во\-ля\-ющий в~ряде случаев получать оценки ско\-рости схо\-ди\-мости, 
а~значит, и~да\-ющий воз\-мож\-ность исследовать и~строить предельные характеристики моделей.

Впервые этот подход, без подробностей, был описан в~работе~\cite{Satin2021} и~применен к~исследованию одного класса суперкомпьютерных сис\-тем в~\cite{Razumchik2022}.
        
    \section{Основные понятия и~описание подхода}
    
Пусть $X(t)$~--- вообще говоря, неоднородная марковская цепь с непрерывным 
временем $t \ge 0$ и~не более чем счетным пространством состояний.  Обозначим 
через 
$$
\vp\left(t\right) = \left(p_{0}\left(t\right), p_{1}\left(t\right), 
\ldots\right)^{\mathrm{T}}
$$ 
ее вектор вероятностей состояний. Положим  
$$
a_{ii}\left(t\right)  = -\sum\limits_{j\neq i} a_{ji}\left(t\right)
$$
 и рассмотрим транспонированную матрицу 
интенсивностей
        \begin{equation*}
    A(t)=       \left(
            \begin{array}{cccccccccccccc}
                a_{00} & a_{01}  & a_{02}  & a_{03}  & \cdots \\
                \\
                a_{10} & a_{11} & a_{12}   & a_{13}  & \cdots \\
                \\
                a_{20} & a_{21} & a_{22} & a_{23}    & \cdots \\
                \vdots   & \vdots & \vdots  & \vdots    & \ddots \\
            \end{array}
            \right),
%            \label{m1}
        \end{equation*}
    
\noindent 
предполагая, что все  $a_{ij}\left(t\right)$ при $i\hm \neq j$ 
локально интегрируемы на $[0,\infty)$ и неотрицательны при всех $t\hm \ge 0$, а 
кроме того, выполнено условие ограниченности
    \begin{equation*}
        \sup\limits_i|a_{ii}\left(t\right)| \le  L < \infty 
        %\label{f1}
    \end{equation*}
    \noindent почти для всех $t \hm\ge 0$.
    
    Обозначим через $ \|\cdot\| $ $ l_1 $-нор\-му вектора, 
$\|{\vx}\|\hm=\sum |x_i|$,
    $\|H\| \hm= \sup\nolimits_j \sum\nolimits_i |h_{ij}|$, если $H\hm = (h_{ij})_{i,j=0}^{\infty}$, и 
обозначим через~$\Omega $ множество всех векторов из~$l_1$ с неотрицательными 
координатами и единичной нормой. Тогда имеем 
$$
\|A(t)\| = 
2\sup\limits_{k}\left|a_{kk}\left(t\right)\right| \le 2L
$$ 
почти для всех $t \hm\ge 0$.
    
    Можем рассматривать прямую систему Колмогорова как дифференциальное 
уравнение в пространстве последовательностей~$l_1$, где~$A\left (t\right)$ 
является ограниченным почти для всех $ t \hm\ge 0 $ линейным оператором из~$l_1$ в~себя.
    
    
Напомним, что марковская цепь $X(t)$ называется \textit{слабо 
эргодичной}, если
$$ 
\lim\limits_{t\rightarrow \infty }\left\| \mathbf{p}^1\left( t\right) -\mathbf{p}^2\left( t\right) \right\| =0 
$$ 
для любых начальных условий 
$\mathbf{p}^1\left( 0\right) \hm=\mathbf{p}^1\in \Omega$ и~$\mathbf{p}^2\left( 
0\right) =\mathbf{p}^2\in \Omega.$
    
    Напомним также, что логарифмическая норма операторной функции в~$l_1$ 
вычисляется по формуле
    \begin{equation*}
        \gamma \left( H\left(t\right) \right)_{1} = \sup\limits_i 
\left(h_{ii}\left(t\right) + \sum\limits_{j\neq i} \left\vert h_{ji}\left(t\right)\right\vert \right),
     %   \label{k11}
    \end{equation*}
    \noindent 
    а  для оператора Коши соответствующего дифференциального 
уравнения ${d\vx}/{dt}\hm=H(t)\vx$ справедлива оценка
    \begin{equation*}
        \|U\left(t,s\right)\| \le e^{\int\nolimits_s^t \gamma\left(H(\tau)\right)\, 
d\tau}.     
%\label{k12}
    \end{equation*}

\bigskip

Пусть $ \vp^*(t) $  и $ \vp^{**}(t) $ являются решениями прямой системы 
Колмогорова~(\ref{eq_1}) с соответствующими (различными)
начальными условиями~$ \vp^*(0)\hm \in \Omega $ и~$ \vp^{**}(0) \in \Omega $, тогда 
их разность $ \vz(t) \hm= \vp^*(t) \hm- \vp^{**}(t)$ также является решением системы~(\ref{eq_1}), 
но при этом сумма всех координат вектора~$z(t)$ равна нулю при 
всех $t\hm \ge 0$, $ \sum\nolimits_{i\ge 0} z_i(t) \hm= 0 $.

Возьмем теперь положительное число~$c$ и вычтем из правой части уравнения 
$z'_0(t)\hm = \sum\nolimits_{j \ge 0} a_{0j}(t)z_j $ выражение  $ c\sum\nolimits_{j \ge 0} z_j \hm= 0$.
Запишем по\-лу\-чив\-шу\-юся при этом систему в виде
\begin{equation}
    \fr{d\vz }{dt}=W(t)\vz \left(t\right), \quad t\ge 0\,, 
    \label{s22}
\end{equation}
где  $ W(t) = A\left( t\right)-C $ и матрица~$C$ имеет вид:
\begin{equation}
 \left(
    \begin{array}{ccccc}
        c & c  & c  & c  & \cdots \\
        \\
        0 & 0 &0  & 0  & \cdots \\
        \\
        0 & 0 & 0 & 0    & \cdots \\
        \vdots   & \vdots & \vdots  & \vdots    & \ddots \\
    \end{array}
    \right).
    \label{m2}
\end{equation}


Для получения простейших оценок скорости сходимости можно применять и 
исследовать сис\-те\-му~(\ref{s22}).

    \section{Оценки скорости сходимости}

Для получения нужных свойств и оценок потребуются некоторые вспомогательные 
<<взвешенные>> нормы. Рассмотрим  последовательность положительных чисел  $\{d_i\}$ 
такую, что $d\hm=\inf d_i >0$, диагональную матрицу $ \textsf{D} \hm= 
\mathrm{diag}\left(d_0, d_1, d_2, \ldots \right)$ и пространство последовательностей 
$l_{1\textsf{D}} \hm= \left\{\vz / \|\vz\|_{1\textsf{D}}\hm=\|\textsf{D}\vz\|_1 \hm< 
\infty\right\}$.

\smallskip

Положим
\begin{equation*}
    \gamma_{*}(t) =  \inf\limits_i \left(|w_{ii}(t)| - \sum\limits_{j\neq i}
    \fr{d_j}{d_i}|w_{ji}(t)|\right). 
  %  \label{f8}
\end{equation*}
Рассмотрим~(\ref{s22}) как дифференциальное уравнение в пространстве 
последовательностей~$l_{1\textsf{D}}$. Тогда
\begin{multline*}
    \|W(t)\|_{1\textsf{D}} = \|\textsf{D} W(t) \textsf{D}^{-1}\| ={}\\
    {}=
    \sup\limits_i \left( \left\vert w_{ii}(t)\right\vert + \sum\limits_{j\neq i} 
\fr{d_j}{d_i} \left\vert w_{ji}(t)\right\vert \right) = {} \\
  {}  = \sup\limits_i \left(2|w_{ii}(t)| + \sum\limits_{j\neq i} \fr{d_j}{d_i}\left\vert w_{ji}(t)\right\vert - 
\left\vert w_{ii}(t)\right\vert \right) \le{}  \\
{} \le 2 \sup\limits_i   \left\vert w_{ii}(t)\right\vert - \gamma_{*}(t)\le 2L -\gamma_{*}(t);
   % \label{f9}
\end{multline*}
следовательно, операторная функция $W(t)$ ограничена в пространстве 
$l_{1\textsf{D}}$. Получаем
\begin{multline*}
\gamma\left(W(t)\right)_{1\textsf{D}} = \gamma \left(\textsf{D} 
W(t)\textsf{D}^{-1}\right)  = {}\\
{}=\sup\limits_i \left(w_{ii}(t) + \sum\limits_{j\neq i}\fr{d_j}{d_i}|w_{ji}(t)|\right) =  - \gamma_{*}(t).
    %\label{f10}
\end{multline*}


Далее оценим логарифмическую норму оператора $ \textsf{D} W(t) \textsf{D}^{-1}$. 
Обозначим через~$ -\alpha_{k}(t) $ сумму всех элементов $k$-го столбца 
матрицы~$ \textsf{D} W(t) \textsf{D}^{-1} $, тогда в предположении, что почти 
при всех $t \hm\ge 0$ $c \hm\le a_{01}(t)$,  получаем
\begin{equation}
    \alpha_{k}\left( t\right) =
    \begin{cases}
            \displaystyle   \sum\limits_{i > 0}\left(1-\fr{d_i}{d_0}\right)a_{i0} + c, & k= 0\,;\\
             \displaystyle  \sum\limits_{i \neq 1}\left(1-\fr{d_i}{d_1}\right)a_{i1} + c\fr{d_0}{d_1}, & k = 1\,;\\
           \displaystyle    \sum\limits_{i \neq k}\left(1-\fr{d_i}{d_k}\right)a_{ik} -  c\fr{d_0}{d_k}, & k > 1\\
    \end{cases}
    \label{2112}
\end{equation}
и $ \gamma_{*}(t) = \inf\limits_k \alpha_{k} $, а~значит, справедливо  следующее 
утверждение.

\smallskip

\noindent
\textbf{Теорема~1.}\ %\label{t2}
\textit{Пусть найдутся число $c\hm >0$, $c \hm\le  a_{01}(t)$, а 
также последовательность положительных чисел~ $\{d_i\}$, для которой $d\hm=\inf d_i 
\hm>0$ такие, что  
$$
\int\limits_0^\infty \gamma_{*}(t) dt = +\infty\,.
$$
 Тогда~$X(t)$ слабо  
эргодична, причем}

\vspace*{-2pt}

\noindent
    \begin{multline}
        \left\|\vp^{1}(t)-\vp^{2}(t)\right\|_{1\textsf{D}} \le{}\\
        {}\le
        e^{-\int_0^t\gamma_{*}(u) \,du}\left\| \vp^{1}(0)-
\vp^{2}(0)\right\|_{1\textsf{D}} 
\label{f12}
    \end{multline}
\textit{для любых начальных условий $ \vp^{1}(0) \hm\in \Omega$, 
$\vp^{2}(0)\hm\in \Omega$ и~всех $t \hm\ge 0$}.


\smallskip

\noindent
\textbf{Замечание~1.}\
 Отметим, что описываемый подход  не гарантирует точных оценок. Например, для 
однородной цепи с двумя состояниями и интенсивностями переходов~$\lambda$ и~$\mu$ матрица~$A$ имеет вид:
\begin{equation*}
A = \left(
    \begin{array}{cc}
        -\lambda & \mu  \\[2pt]
                \lambda & -\mu  \\
    \end{array}
    \right).
   % \label{mу2}
\end{equation*}


\noindent
Одно из собственных значений этой матрицы нулевое, а второе равно $-\left(\lambda \hm+ \mu\right)$, так что скорость сходимости к стационарному режиму 
имеет порядок $e^{-\left(\lambda\hm + \mu\right)t}$. С~другой стороны, пользуясь 
теоремой~1 и~выбирая максимально возможное $c\hm=\mu$, получаем оценку 
скорости сходимости~$e^{-\mu t}$.

\smallskip


\noindent
\textbf{Замечание~2.}\ В~общем случае возможно использование как другой строки в матрице 
$C$, так и~рас\-смот\-ре\-ние функции~$c(t)$ вместо положительной константы (см.\ 
пример далее).

\smallskip


\noindent
\textbf{Замечание~3.}\ В~частности, в случае марковской цепи с катастрофами (т.\,е.\ при 
возможности перехода из любого состояния в нулевое с ненулевой  интенсивностью $a_{0j}(t) \hm= \gamma_j(t) $ для $ j\hm \ge 1 $) 
можно выбрать в~качестве  $ c(t) $ функцию  $ c(t)\hm= \inf\nolimits_k \gamma_k(t) $.

\smallskip

\noindent
\textbf{Теорема~2.}\ %\label{t3}
\textit{Рассмотрим марковскую цепь~$X^*(t)$ с интенсивностями 
$a^*_{i,j}(t)$. Пусть при выполнении условий тео\-ре\-мы~$1$ последовательность 
$\{d_k\}$ монотонно возрастает.  Если для всех $t\hm \ge 0$ выполнены условия}:

\pagebreak

\noindent
\begin{enumerate}[(1)]
\item $a_{i,j}(t) \hm\le a^*_{i,j}(t)$ \textit{при} $i\hm<j$; 
\item $a_{i,j}(t) \hm\ge a^*_{i,j}(t)$ \textit{при} 
$i\hm>j$,
\end{enumerate}
 \textit{то цепь~$X^*(t)$ также слабо эргодична, причем $\gamma_{*}^*(t) \hm\ge  \gamma_{*}(t)$}.


%\smallskip

\noindent
Д\,о\,к\,а\,з\,а\,т\,е\,л\,ь\,с\,т\,в\,о\,.\ \  Достаточно убедиться, что  $\alpha^*_{k}\left( t\right) \hm\ge 
\alpha_{k}\left( t\right) $. Имеем
\begin{multline*}
    \alpha^*_{k}\left( t\right) =
    \begin{cases}
             \displaystyle  \sum\limits_{i > 0}\left(1-\fr{d_i}{d_0}\right)a^*_{i0} + c, & k= 0\,;\\
             \displaystyle  \sum\limits_{i \neq 1}\left(1-\fr{d_i}{d_1}\right)a^*_{i1} + 
c\fr{d_0}{d_1}, & k = 1\,;\\
             \displaystyle  \sum\limits_{i \neq k}\left(1-\fr{d_i}{d_k}\right)a^*_{ik} - 
c\fr{d_0}{d_k}, & k > 1\\
    \end{cases} 
    ={}\\
    {}=
    \begin{cases}
         \displaystyle  \sum\limits_{i > 0}\left(1-\fr{d_i}{d_0}\right)a^*_{i0} + c, & k= 0\,;\\[12pt]
          \displaystyle \left(1-\fr{d_0}{d_1}\right)a^*_{01} + {}&\\[6pt]
          \displaystyle {}+\sum\limits_{i > 1}\left(1-
\fr{d_i}{d_1}\right)a^*_{i1} + c\fr{d_0}{d_1}, & k = 1\,;\\[6pt]
          \displaystyle \sum\limits_{i < k}\left(1-\fr{d_i}{d_k}\right)a^*_{ik} +{}&\\[12pt]
          \displaystyle {}+ \sum\limits_{i > k}\left(1-
\fr{d_i}{d_k}\right)a^*_{ik} - c\fr{d_0}{d_k}, & k > 1\\[6pt]
\end{cases} \ge{}
   \\
\!\!\!\!  {}  \ge
    \begin{cases}
       \displaystyle \sum\limits_{i > 0}\left(1-\fr{d_i}{d_0}\right)a_{i0} + c, & \hspace*{-15mm}k= 0\,;\\[12pt]
        \left(1-\fr{d_0}{d_1}\right)a_{01} +        \displaystyle\sum\limits_{i > 1}\left(1-
\fr{d_i}{d_1}\right)a_{i1} + {}\\[12pt]
\hspace{25mm}{}+c\fr{d_0}{d_1}, & \hspace*{-15mm} k = 1\,;\\[6pt]
             \displaystyle  \sum\limits_{i < k}\left(1-\fr{d_i}{d_k}\right)a_{ik} +{}\\[12pt]
{}+\displaystyle \sum\limits_{i > k}\left(1-\fr{d_i}{d_k}\right)a_{ik} - c\fr{d_0}{d_k}, & \hspace*{-15mm}k > 0  \\
    \end{cases} \ge \alpha_{k}\left( t\right).
    %\label{2112w}
\end{multline*}
Из этого следует и утверждение теоремы.

\smallskip

Пусть теперь имеется процесс рождения и гибели с интенсивностями рождения 
$\lambda_k(t)$ и гибели~$\mu_k(t)$ и пусть для него подобраны нужные параметры 
так, что выполнены условия теоремы~1, а~последователь\-ность~$\{d_k\}$ 
монотонно возрастает. Тогда можно применить подход из рассуждений тео\-ре\-мы~2 и~получить оценки скорости сходимости для так называемых single birth 
processes (см., например,~\cite{mufa,mao}), для которых интенсивности $a_{ij}(t) 
\hm\equiv 0$ при всех $i\hm-j \hm\ge 2$,
%\columnbreak
а~именно: будем предполагать,\linebreak\vspace*{-12pt}

\columnbreak

\noindent
 что транспонированная матрица интенсивностей для 
марковской цепи~$X^*(t)$ имеет вид
    \begin{equation*}
        A^*(t)=\left(
        \begin{array}{cccccccccccccc}
            a_{00}^*(t) & a_{01}^*(t)  & a_{02}^*(t)  & a_{03}^*(t)  & 
\cdots \\
            \\
            \lambda_0(t) & a_{11}^*(t) & a_{12}^*(t)   & a_{13}^*(t)  & 
\cdots \\
            \\
            0 & \lambda_1(t) & a_{22}^*(t) & a_{23}^*(t)    & \cdots \\
            \vdots   & \vdots & \vdots  & \vdots    & \ddots \\
        \end{array}
        \right).
%        \label{m1}
    \end{equation*}


\noindent
\textbf{Теорема~3.}\
\textit{Пусть интенсивности марковской цепи~$X^*(t)$ удовлетворяют условию 
$a_{ii}^*(t)\hm \equiv a_{ii}(t)$ при всех~$i$. Пусть при выполнении условий 
теоремы~$1$ последовательность~$\{d_k\}$ монотонно возрастает. Тогда~$X^*(t)$ 
слабо эргодична и справедлива оценка скорости сходимости}~(\ref{f12}).

\smallskip

Для доказательства достаточно отметить, что все суммы, входящие в~(\ref{2112}), 
в условиях теоремы~3 заведомо не уменьшатся.
В частности, если взять процесс рождения и~гибели с~интенсивностями 
$\lambda_k\hm=\lambda$ и~$\mu_k\hm=\mu$ так, что 
$$
K=\left(\sqrt{\lambda}- \sqrt{\mu}\right)^2-\fr{\lambda\left(\mu-\sqrt{\lambda 
\mu}\right)}{\lambda+\mu} >0\,,
$$
 то можно взять 
\begin{gather*}
 c=\fr{\lambda\left(\mu-\sqrt{\lambda \mu}\right)}{\lambda+\mu}\,;\
d_k=\left(\sqrt{\fr{\mu}{\lambda}}\right)^k\,,
\end{gather*}
 тогда получится $\gamma_*\hm=K$.



    \section{Пример}
    
    Рассмотрим модель системы обслуживания, изученную в~\cite{Satin2019a}, 
где требования поступают по одному, а обслуживаются только парами. 
Транспонированная матрица интенсивностей~$ A(t) $ имеет вид:
            \begin{multline*}
            \left(
            \begin{array}{cccc}
                -\lambda(t) & 0  & \mu(t)  & 0  \\                             
                           \lambda(t) &-\lambda(t) & 0   & \mu(t) \\                          
                           0 &\lambda(t) & -\left(\lambda(t) + \mu(t) \right) & 0\\           
                        0 & 0 & \lambda(t) & -\left(\lambda(t) + \mu(t) \right) \\         
                        0 & 0 & 0 & \lambda(t) \\ 
            \vdots   & \vdots   &   \vdots   & \vdots\\                        
            \end{array}\right.
            \\
          \left.  
          \begin{array}{ccc}
                0  & 0  & \cdots \\                
                0  & 0  & \cdots \\                
                \mu(t)  & 0  & \cdots \\           
                0  & \mu(t)  & \cdots \\           
                      -\left(\lambda(t) + \mu(t) \right) & 0 &\cdots\\  
                  \vdots  & \vdots    & \ddots \\    
                        \end{array}
            \right).
           % \label{m22}
        \end{multline*}
    %}

В работе~\cite{Satin2019a} показано, что стандартный метод, упомянутый в начале 
настоящей статьи, не позволяет получить содержательную оценку скорости 
сходимости и применен метод неравенств, который  удалось применить к данной 
модели (однако в общей ситуации для этого метода требуется перебор бесконечного 
числа случаев комбинаций знаков).

Попробуем применить описанный выше метод, выбрав для этого в матрице~$C(t)$ 
ненулевой первую строку (считая с нуля), т.\,е.\
\begin{equation*}
    C(t) = \left(
    \begin{array}{ccccc}
        0 & 0 &0  & 0  & \cdots \\
        \\
        c(t) & c(t)  & c(t)  & c(t)  & \cdots \\
        \\
        0 & 0 & 0 & 0    & \cdots \\
        \vdots   & \vdots & \vdots  & \vdots    & \ddots \\
    \end{array}
    \right),
   % \label{mq2}
\end{equation*}
и выпишем суммы по столбцам для получившейся матрицы ${\sf D}W(t){\sf D}^{-1} $, взятые 
с~противоположным знаком:

\vspace*{-22pt}
\begin{multline}
    \alpha_{k}\left( t\right) ={}\\
 \!\!  \!{}=\!
    \begin{cases}
        \lambda(t) -\fr{d_1}{d_0} |\lambda(t) - c(t)|, &       \hspace*{-15mm}  k= 0\,;\\[9pt]
        c(t) - \left(\fr{d_2}{d_1} - 1\right) \lambda(t), & \hspace*{-15mm} k = 1\,;\\[9pt]
              \displaystyle \left(1-\fr{d_0}{d_2}\right) \mu(t) -\left(\fr{d_3}{d_2} - 
1\right) \lambda(t) - c(t) \fr{d_1}{d_2}, &\\
& \hspace*{-15mm}k = 2\,;\\
             \displaystyle  \mu(t) -\left(\fr{d_4}{d_3} - 1\right) \lambda(t) - 
\fr{d_1}{d_3}|\mu(t) - c(t)|, &\\
& \hspace*{-15mm}k = 3\,;\\
            \displaystyle  \left(1-\fr{d_{k-2}}{d_k}\right) \mu(t) -\left(\fr{d_{k+1}}{d_k} - 
1\right) \lambda(t) - {}&\\[9pt]
\hspace*{30mm}{}-c(t) \fr{d_1}{d_k}, & \hspace*{-15mm}k > 3\,.
    \end{cases}\!\!\!\!\!
 \label{t3q}
\end{multline}

\vspace*{-2pt}

 Пусть $ d_0=d_1=1$ и $ d_k \hm= \delta^{k-1} $ для $ k\hm \ge 2 $ и~$ \delta\hm > 1 $ 
и $ c \hm=({\delta-1})\mu(t)/({\delta+1}) \hm< \lambda(t)$. Подставим эти значения в~(\ref{t3q}) 
и получим:

\vspace*{-22pt}

 \begin{multline*}
    \alpha_{k}\left( t\right) ={}\\
    {}=
    \begin{cases}
        c(t), &  \hspace*{-17mm}k= 0\,;\\
        c(t) - \left(\delta - 1\right) \lambda(t), & \hspace*{-17mm}k = 1\,;\\
        \left(1-\fr{1}{\delta}\right) \mu(t) -\left(\delta - 1\right) 
\lambda(t) - c(t) \fr{1}{\delta}, &\\
&\hspace*{-17mm}k = 2\,;\\
        \mu(t) -\left(\delta - 1\right) \lambda(t) - 
\fr{1}{\delta^2}|\mu(t) - c(t)|, &\\
&\hspace*{-17mm}k = 3\,;\\
        \left(1-\fr{1}{\delta^2}\right) \mu(t) -\left(\delta - 1\right) 
\lambda(t) - c(t) \fr{1}{\delta^{k-1}}, &\\
&\hspace*{-17mm}k > 3\\
    \end{cases} \!\!\!\!\!\!\!\!\!\!={}
    \end{multline*}
    
    \noindent
    \begin{multline*}
    {}=
 \begin{cases}
          \displaystyle  \fr{\delta-1}{\delta+1}\,\mu(t), & \hspace*{-10mm}k= 0\,;\\
          \displaystyle  \fr{\delta-1}{\delta+1}\,\mu(t) - \left(\delta - 1\right) \lambda(t), & \hspace*{-10mm}k  = 1\,;\\
    \fr{\delta-1}{\delta+1}\,\mu(t) - \left(\delta - 1\right) \lambda(t), & \hspace*{-10mm}k  = 2\,;\\
          \displaystyle \mu(t) -\left(\delta - 1\right) \lambda(t) - {}&\\
          \hspace*{10mm}{}-\fr{1}{\delta^2}\,\fr{2}{\delta+1}\,\mu(t), & \hspace*{-10mm}k = 3\,;\\
    \left(1-\fr{1}{\delta^2}\right) \mu(t) -\left(\delta - 1\right) 
\lambda(t) - {}&\\
\hspace*{10mm}{}- \fr{\delta-1}{\delta+1}\,\mu(t) \fr{1}{\delta^{k-1}}, & \hspace*{-10mm}k > 3
 \end{cases} \ge{}
 %   \label{t23}
\\
  {}  \ge
    \begin{cases}
            \displaystyle   \fr{\delta-1}{\delta+1}\,\mu(t) - \left(\delta - 1\right) 
\lambda(t)\,; \\[9pt]
           \displaystyle    \!\left(\!1\!-\!\fr{1}{\delta^2}\!\right)\!\mu(t) \!-\!\left(\delta - 1\right) 
\lambda(t) + \fr{1}{\delta^2}\, \fr{\delta-1}{\delta+1}\mu(t)\,; \\[9pt]
          \displaystyle     \!\left(\!1\!-\!\fr{1}{\delta^2}\!\right)\! \mu(t) \!-\!\left(\delta - 1\right) 
\lambda(t) -  \fr{1}{\delta^{3}}\,\fr{\delta-1}{\delta+1}\,\mu(t)  
    \end{cases}  \!\!\!\!\!\! \ge{}\\
    {}\ge
\begin{cases}
       \displaystyle\fr{\delta-1}{\delta+1}\,\mu(t) - \left(\delta - 1\right) \lambda(t)\,; \\
       \displaystyle\!\left(\!1-\fr{1}{\delta^2}\!\right)\! \mu(t) -\left(\delta - 1\right) \lambda(t) -  
\fr{1}{\delta^{3}}\,\fr{\delta-1}{\delta+1}\mu(t)  
\end{cases}
   % \label{t232}
\end{multline*}
 и $ \alpha_{k} \ge ({\delta-1})\mu(t)/({\delta+1})\hm - \left(\delta \hm- 1\right) 
\lambda(t) $.

\smallskip

\noindent
\textbf{Теорема~4.}\
\textit{Пусть существует $ \delta\hm > 1 $ такое, что 
$$\left(\delta - 1\right) \lambda(t) \le \fr{(\delta-1)\mu(t)}{\delta+1} \le \lambda(t)
$$
 и 
 $$ 
 \int\limits_0^\infty \left(\mu(t) - (\delta+1)\lambda(t)\right) dt= +\infty\,.
 $$
  Тогда процесс, описывающий число требований в сис\-те\-ме, слабо  
эргодичен, причем} 
$$
\gamma_*(t) \ge \fr{(\delta-1)\mu(t)}{\delta+1} - \left(\delta - 1\right) \lambda(t).
$$

\smallskip

Рассмотрим теперь конкретную модель с интенсивностями 
$$
\lambda(t) = 2 + \fr{1}{2}\sin(2\pi t),\enskip  \mu(t) = 8 - 2\cos(2\pi t).
$$

 Положим $ \delta \hm= 
{9}/{7}$. Тогда 
$$
 \alpha_{k} \ge \fr{1}{8}(8 - 2\cos(2\pi t))- \fr{2}{7} 
(2 + \fr{1}{2}\sin(2\pi t)) \ge \fr{3}{22} 
$$
и оценка скорости сходимости имеет вид:
$$
\left \|p^{*1}(t)-p^{*2}(t)\right\|_{1D} \le  e^{-{3t}/22} \left\| p^{*1}(0)-p^{*2}(0)\right\|_{1D}.
$$

\begin{figure*}[b] %fig1-6
\vspace*{1pt}
  \begin{center}  
    \mbox{%
\epsfxsize=162.5mm
\epsfbox{zei-1-6.eps}
}

\end{center}
\vspace*{-15pt}
\begin{minipage}[t]{80mm}
    \Caption{Вероятность пустой системы массового 
обслуживания~(\textit{а}), вероятность $ p_1(t) $ системы 
массового обслуживания~(\textit{б}) и~среднее $E(t,k)$~(\textit{в}) для $ t \hm\in    [0,15]$}
\end{minipage}
\hfill
\vspace*{-15pt}
\begin{minipage}[t]{80mm}
\Caption{Аппроксимации предельной вероятности 
пустой системы массового обслуживания~(\textit{а}),
предельной вероятности 
$ p_1(t)$ системы массового обслуживания~(\textit{б}) и~предельного среднего 
значения $E(t,k)$~(\textit{в})  для $ t \hm\in [15,16]$}
\end{minipage}
\vspace*{12pt}
        \end{figure*}    


    Далее следуем методу, который был подробно описан в~\cite{Zeifman2014}, а 
именно: выбираем размерность усеченного процесса (в~данном случае~100), 
интервал, на котором достигается желаемая точность ($[0,15]$), и~сам предельный 
интервал ($[15,16]$), строим графики ожидаемого числа требований в~сис\-те\-ме 
и~некоторых наиболее вероятных состояний (рис.~1 и~2).


        
        
        
        {\small\frenchspacing
 {%\baselineskip=10.8pt
 %\addcontentsline{toc}{section}{References}
 \begin{thebibliography}{99}
 
 \bibitem{u1} %1
\Au{Sidje R.\,B., Burrage K., Macnamara~S.} Inexact uniformization method for 
computing transient distributions of Markov chains~// SIAM J.~Sci. Comput., 
2007. Vol.~29. P.~2562--2580.

\bibitem{Arns2010} %2
\Au{Arns M., Buchholz~P.,  Panchenko~A.} On the numerical analysis of 
inhomogeneous continuous-time Markov chains~// Informs J.~Comput., 2010. 
Vol.~22. P.~416--432.

\bibitem{Andreychenko2018} %3
\Au{Andreychenko A., Sandmann~W., Wolf~V.} Approximate adaptive uniformization 
of continuous-time Markov chains~// Appl. Math. Model., 2018. Vol.~61. P.~561--576.

\bibitem{Burak2020} %4
\Au{Burak M.\,R.,  Korytkowski~P.} Inhomogeneous CTMC birth-and-death models 
solved by uniformization with steady-state detection~// ACM T. Model. 
Comput.~S., 2020. Vol.~30. P.~1--18.



\bibitem{Inoue} %5
\Au{Inoue Y.} A new approach to computing the transient-state probabilities in 
time-inhomogeneous Markov chains~// J.~Oper. Res. Soc.~Jpn., 2022. Vol.~65. No.\,1. P.~48--66.


\bibitem{chak}
\Au{Chakravarthy S.\,R.} A~catastrophic queueing model with delayed action~// 
Appl. Math. Model., 2017. Vol.~46. P.~631--649.

\bibitem{Zeifman2020porto} %7
\Au{Zeifman A.} On the study of forward Kolmogorov system and the 
corresponding problems for inhomogeneous continuous-time Markov chains~//
Differential and difference equations with applications~/
 Eds.\ S.~Pinelas, J.\,R.~Graef, S.~Hilger, P.~Kloeden, and C.~Schinas.~--- 
Springer proceedings in mathematics and statistics ser.~--- Springer, 2020. Vol.~333. P.~21--39.

\bibitem{Satin2019a}        %8 
\Au{Satin Ya., Zeifman~A., Kryukova~A.} On the rate of convergence and 
limiting characteristics for a nonstationary queueing model~// Mathematics, 
2019. Vol.~7. Iss.~8. Art. No.\,678. 11~p.

\bibitem{Zeifman2020AMCS} %9
\Au{Zeifman A., Satin Y., Kryukova~A., Razumchik~R., Kiseleva~K., Shilova~G.} 
On three methods for bounding the rate of convergence for some continuous-time 
Markov chains~// Int. J.~Appl. Math. Comp., 2020. Vol.~30. No.\,2. P.~251--266.



\bibitem{Zeifman2021MDPI}
\Au{Zeifman A., Satin~Y.,  Sipin~A.} Bounds on the rate of convergence for $  
M_t^X/M_t^X/1 $ queueing models~// Mathematics, 2021. Vol.~9. Iss.~1. Art. 
No.\,1752. 11~p.

\bibitem{Satin2021}
\Au{Сатин Я.\,А.} Исследование модели типа $  M_t/M_t/1 $ с~двумя различными 
классами требований~// Сис\-те\-мы и~средства информатики, 2021. Т.~31. №\,1.\linebreak 
С.~17--27.
        
\bibitem{Razumchik2022} %12
\Au{Razumchik R., Rumyantsev~A.} Some ergodicity and truncation bounds for a 
small scale Markovian supercomputer model~// 36th ECMS  Conference (International) on Modelling and 
Simulation Proceedings.~--- Saarbrucken--Dudweiler, Germany: Digitaldruck Pirrot 
GmbH, 2022. P.~324--330.

\bibitem{mufa}
\Au{Chen M.} Single birth processes~// Chinese Annals Mathematics, 1999. 
Vol.~20. No.~1. P.~77--82.

\bibitem{mao}
\Au{Mao Y.\,H.,  Zhang~Y.\,H.} Exponential ergodicity for single-birth 
processes~// J.~Appl. Probab., 2004. Vol.~41. No.\,4. P.~1022--1032.

\bibitem{Zeifman2014}
\Au{Zeifman A., Satin~Y., Korolev~V.,  Shorgin~S.} On truncations for weakly 
ergodic inhomogeneous birth and death processes~// Int. J.~Appl. Math. Comp., 2014. Vol.~24. No.\,3. P.~503--518.
  \end{thebibliography}

 }
 }

\end{multicols}

\vspace*{-9pt}

\hfill{\small\textit{Поступила в~редакцию 25.06.22}}

\vspace*{6pt}

%\pagebreak

%\newpage

%\vspace*{-28pt}

\hrule

\vspace*{2pt}

\hrule

%\vspace*{-2pt}

\def\tit{ON AN APPROACH FOR ESTIMATING THE~RATE OF~CONVERGENCE FOR~NONSTATIONARY MARKOV MODELS OF~QUEUEING SYSTEMS}


\def\titkol{On an approach for estimating the~rate of~convergence for~nonstationary Markov models of~queueing systems}


\def\aut{I.\,A.~Kovalev$^{1,2}$, Y.\,A.~Satin$^1$, A.\,V.~Sinitcina$^3$, and~A.\,I.~Zeifman$^{1,2,4,5}$}

\def\autkol{I.\,A.~Kovalev, Y.\,A.~Satin, A.\,V.~Sinitcina, and~A.\,I.~Zeifman}

\titel{\tit}{\aut}{\autkol}{\titkol}

\vspace*{-15pt}


\noindent
$^1$Department of Applied Mathematics, Vologda State University, 15~Lenin Str., Vologda 160000, Russian Federation

\noindent
$^2$Moscow Center for Fundamental and Applied Mathematics, M.\,V.~Lomonosov Moscow State University,\linebreak
$\hphantom{^1}$1-52~Leninskie Gory, GSP-1, Moscow 119991, Russian Federation

\noindent
$^3$P.\,G.~Demidov Yaroslavl State University, 14~Sovetskaya Str., Yaroslavl 150003, Russian Federation

\noindent
$^4$Federal Research Center ``Computer Science and Control'' of the Russian Academy of Sciences, 44-2~Vavilov\linebreak
$\hphantom{^1}$Str., Moscow 119333, Russian Federation

\noindent
$^5$Vologda Research Center of the Russian Academy of Sciences, 56A~Gorky Str., Vologda 160014, Russian\linebreak
$\hphantom{^1}$Federation


\def\leftfootline{\small{\textbf{\thepage}
\hfill INFORMATIKA I EE PRIMENENIYA~--- INFORMATICS AND
APPLICATIONS\ \ \ 2022\ \ \ volume~16\ \ \ issue\ 3}
}%
 \def\rightfootline{\small{INFORMATIKA I EE PRIMENENIYA~---
INFORMATICS AND APPLICATIONS\ \ \ 2022\ \ \ volume~16\ \ \ issue\ 3
\hfill \textbf{\thepage}}}

\vspace*{3pt} 

\Abste{The transformation of the forward Kolmogorov system is considered which allows one to obtain simple estimates on 
the rate of convergence  for Markov chains with continuous time describing queuing systems. In addition, the concept 
of the logarithmic norm of the operator function and the associated estimates of the norm of the Cauchy matrix are used. 
The results obtained make it possible to estimate the rate of convergence  for new
classes of models in which the matrix
 is not significantly nonnegative and the use of the logarithmic norm method does not guarantee the possibility of 
 obtaining estimates of the rate of convergence. Previously, a rather laborious more general method of inequalities was 
 used for such situations. A~theorem is formulated on obtaining the rate of convergence when the intensities of the matrix change. 
 An estimate was obtained for the process of\linebreak\vspace*{-12pt}}

\Abstend{birth and death with constant intensities. As an example, a~special nonstationary model 
 with group service of requirements (service in pairs) is investigated.}

\KWE{rate of convergence; ergodicity bounds; logarithmic norm; queuing systems}



\DOI{10.14357/19922264220310}

\vspace*{-18pt}

\Ack

\vspace*{-3pt}

\noindent
The results of Section~4 were obtained by A.\,V.~Sinitcina supported by the Russian Science Foundation (grant No.\,21-71-30011).


%\vspace*{4pt}

  \begin{multicols}{2}

\renewcommand{\bibname}{\protect\rmfamily References}
%\renewcommand{\bibname}{\large\protect\rm References}

{\small\frenchspacing
 {%\baselineskip=10.8pt
 \addcontentsline{toc}{section}{References}
 \begin{thebibliography}{99}
 
 \bibitem{4-zei} %1
\Aue{Sidje, R.\,B., K.~Burrage, and S.~Macnamara.}
 2007. Inexact uniformization method for computing transient distributions of Markov chains. 
 \textit{SIAM J.~Sci. Comput.} 29:2562--2580.
 
\bibitem{1-zei} %2
\Aue{Arns, M., P.~Buchholz, and A.~Panchenko.}
 2010. On the numerical analysis of inhomogeneous continuous-time Markov chains. \textit{Informs J.~Comput.} 22:416--432.
\bibitem{2-zei} %3
\Aue{Andreychenko, A., W.~Sandmann, and V.~Wolf.}
 2018. Approximate adaptive uniformization of continuous-time Markov chains. \textit{Appl. Math. Model.} 61:561--576.
\bibitem{3-zei} %4
\Aue{Burak, M.\,R., and P.~Korytkowski.}
 2020. Inhomogeneous CTMC birth-and-death models solved by uniformization with steady-state detection.
 \textit{ACM T. Model. Comput.~S.} 30:1--18.

\bibitem{5-zei}
\Aue{Inoue, Y.} 2022. A~new approach to computing the transient-state probabilities in time-inhomogeneous Markov chains. 
\textit{J.~Oper. Res. Soc. Jpn.} 65(1):48--66.
\bibitem{6-zei}
\Aue{Chakravarthy, S.\,R.} 2017. A~catastrophic queueing model with delayed action. \textit{Appl. Math. Model.} 46:631--649.
\bibitem{7-zei}
\Aue{Zeifman, A.} 2020. On the study of forward Kolmogorov system and the corresponding problems for inhomogeneous continuous-time Markov chains.
\textit{Differential and difference equations with applications.}
 Eds. S.~Pinelas, J.\,R.~Graef, S.~Hilger, P.~Kloeden, and C.~Schinas. Springer proceedings in mathematics and statistics ser. Springer. 333:21--39. 

\bibitem{9-zei} %8
\Aue{Satin, Y., A. Zeifman, and A.~Kryukova.}
 2019. On the rate of convergence and limiting characteristics for a nonstationary queueing model.  \textit{Mathematics} 7(8):678. 11~p.
 
 \bibitem{8-zei} %9
\Aue{Zeifman, A., Y.~Satin, A.~Kryukova, R.~Razumchik, K.~Kiseleva, and G.~Shilova.}
 2020. On the three methods for bounding the rate of convergence for some continuous time Markov chains. 
 \textit{Int. J.~Appl. Math. Comp.} 30(2):251--266.
 
\bibitem{10-zei}
\Aue{Zeifman, A., Y.~Satin, and A.~Sipin.} 2021. Bounds on the rate of convergence for $ M_t^X/M_t^X/1$ queueing models.
 \textit{Mathematics} 9(1):1752. 11~p.
\bibitem{11-zei}
\Aue{Satin, Y.\,A.} 2021. Issledovanie modeli tipa $M_t/M_t/1$ s~dvumya razlichnymi klassami trebovaniy 
[On the bounds of the rate of convergence for $M_t/M_t/1$ model with two different requests].
 \textit{Sistemy i~Sredstva Informatiki~--- Systems and Means of Informatics} 31(1):17--27.
\bibitem{12-zei}
\Aue{Razumchik, R., and A.~Rumyantsev.} 2022. Some ergodicity and truncation bounds for a small scale markovian supercomputer model.
\textit{36th ECMS  Conference\linebreak (International) on Modelling and 
Simulation Proceedings}. Saarbrucken--Dudweiler, Germany: Digitaldruck Pirrot 
GmbH. 324--330.
\bibitem{13-zei}
\Aue{Chen, M.} 1999. Single birth processes.  \textit{Chinese Annals Mathematics} 20(1):77--82.
\bibitem{14-zei}
\Aue{Mao, Y.\,H., and Y.\,H.~Zhang.}
 2004. Exponential ergodicity for single-birth processes.  \textit{J.~Appl. Probab.} 41(4):1022--1032.
\bibitem{15-zei}
\Aue{Zeifman, A., Y.~Satin, V.~Korolev, and S.~Shorgin.}
 2014. On truncations for weakly ergodic inhomogeneous birth and death processes.  \textit{Int. J.~Appl. Math. Comp.} 24(3):503--518.
 \end{thebibliography}

 }
 }

\end{multicols}

\vspace*{-7.5pt}

\hfill{\small\textit{Received June 25, 2022}}

\vspace*{-22pt}


 \Contr
 
 \vspace*{-4pt}
 
 \noindent
 \textbf{Kovalev Ivan A.} (b.\ 1996)~--- 
 PhD student, Vologda State University, 15~Lenin Str., Vologda 160000, Russian Federation; 
 scientist, Moscow Center for Fundamental and Applied Mathematics, M.\,V.~Lomonosov Moscow State University, 
 1~Leninskie Gory, GSP-1, Moscow 119991, Russian Federation; \mbox{kovalev.iv96@yandex.ru}
 
 \vspace*{1.5pt}
 
 \noindent
 \textbf{Satin Yacov A.} (b.\ 1978)~--- 
 Candidate of Science (PhD) in physics and mathematics, associate professor, Department of Applied Mathematics, Vologda State University, 
 15~Lenin Str., Vologda 160000; \mbox{yacovi@mail.ru} 
 
 \vspace*{1.5pt}
 
 \noindent
 \textbf{Sinitcina Anna V.} (b.\ 1988)~--- 
 Candidate of Science (PhD) in physics and mathematics, scientist, P.\,G.~Demidov Yaroslavl State University, 14~Sovetskaya Str., Yaroslavl 150003, 
 Russian Federation; \mbox{a\_korotysheva@mail.ru}

 \vspace*{1.5pt}
 
 \noindent
 \textbf{Zeifman Alexander I.} (b.\ 1954)~--- 
 Doctor of Science in physics and mathematics, professor, head of department, Vologda State University, 15~Lenin Str., Vologda 160000, Russian Federation;
  senior scientist, Institute of Informatics Problems, Federal Research Center ``Computer Science and Control'' 
  of the Russian Academy of Sciences, 44-2~Vavilov Str., Moscow 119133, Russian Federation; 
  principal scientist, Vologda Research Center of the Russian Academy of Sciences, 56A~Gorky Str., Vologda 160014, Russian Federation; 
  senior scientist, Moscow Center for Fundamental and Applied Mathematics, M.\,V.~Lomonosov Moscow State University, 1-52~Leninskie Gory, GSP-1, 
  Moscow 119991, Russian Federation; \mbox{a\_zeifman@mail.ru}
\label{end\stat}

\renewcommand{\bibname}{\protect\rm Литература}    