
\def\stat{shnurkov}

\def\tit{НЕКОТОРЫЕ РЕЗУЛЬТАТЫ АНАЛИЗА ПРОЦЕССА ИЗМЕНЕНИЯ ЦЕНЫ БИВАЛЮТНОЙ 
КОРЗИНЫ НА~ОСНОВЕ МЕТОДОВ СТАТИСТИКИ СЛУЧАЙНЫХ ПРОЦЕССОВ}

\def\titkol{Некоторые результаты анализа процесса изменения цены бивалютной 
корзины на~основе методов} % статистики случайных процессов}

\def\aut{П.\,В.~Шнурков$^1$, М.\,А.~Мигуля$^2$}

\def\autkol{П.\,В.~Шнурков, М.\,А.~Мигуля}

\titel{\tit}{\aut}{\autkol}{\titkol}

\index{Шнурков П.\,В.}
\index{Мигуля М.\,А.}
\index{Shnurkov P.\,V.}
\index{Migulya M.\,A.}


%{\renewcommand{\thefootnote}{\fnsymbol{footnote}} \footnotetext[1]
%{Исследование выполнено при финансовой поддержке Российского научного фонда (проект 
 %<<Информатика>> ФИЦ ИУ РАН, Москва).}}


\renewcommand{\thefootnote}{\arabic{footnote}}
\footnotetext[1]{Национальный исследовательский университет <<Высшая школа экономики>>, pshnurkov@hse.ru}
\footnotetext[2]{Национальный исследовательский университет <<Высшая школа экономики>>, maxim.migulya@gmail.com}


\vspace*{6pt}

  
  \Abst{Работа посвящена исследованию процесса изменения цены так называемой 
бивалютной корзины на валютном рынке Российской Федерации при наличии интервенций, 
проводимых Центральным банком. Основная цель работы заключается в~проверке 
соответствия наблюдаемого процесса стохастической марковской модели с~дискретным 
временем и~дискретным множеством состояний. Теоретическую основу исследования 
составляют методы статистики марковских случайных процессов. В~результате установлены 
условия, при которых реальный процесс может быть достаточно адекватно описан указанной 
марковской моделью.}
  
  \KW{цепь Маркова с~дискретным временем и~дискретным множеством состояний; 
статистика марковских случайных процессов; стохастические модели эволюции процессов 
на финансовых рынках; стохастические модели интервенций; цена бивалютной корзины}

\DOI{10.14357/19922264220303}
  
\vspace*{11pt}


\vskip 10pt plus 9pt minus 6pt

\thispagestyle{headings}

\begin{multicols}{2}

\label{st\stat}
  
\section{Введение}

  В работах~[1, 2] была решена так называемая стохастическая задача о 
настройке, связанная с~управ\-ле\-ни\-ем случайным процессом, которое 
осуществляется в~моменты его выходов на границу заданного подмножества 
множества состояний. 

В~дальнейшем идея решения задачи о настройке была 
использована при анализе проблемы управления интервенциями, проводимыми 
Центральным банком РФ на валютном рынке~[3]. Однако для применения 
теоретических результатов решения задачи о~настройке необходимо изучить 
характер основного процесса на интервалах времени между управлениями. 
В~частности, необходимо подтвердить, что этот процесс является марковским. 
Со\-от\-вет\-ст\-ву\-ющее исследование проведено в~настоящей работе.
  
  В качестве методов исследования были использованы методы 
математической статистики марковских случайных процессов~\cite{4-shn, 5-shn}. 
По своей сущности эти методы представляют собой проверку различных 
статистических гипотез, связанных со\linebreak свойствами марковости и~однородности 
на\-блю\-да\-емых процессов. При этом реальные статистические данные об 
изменении цены бивалютной корзины за соответствующие периоды времени 
\mbox{приведены} на сайте Центрального банка РФ~\cite{6-shn}.
  
  Особая сложность наблюдаемых процессов заключается в~том, что 
множество их возможных значений не является фиксированным, а меняется во\linebreak 
времени. Это связано как с~объективным характером самих процессов, так 
и~с~тем, что на их эволю\-цию оказывали влияние действия Централь\-ного банка 
РФ, который непосредственно \mbox{устанавливал} границы так называемых 
валютных коридоров и~проводил периодические внешние воздействия на цену 
бивалютной корзины в~форме валютных интервенций. В~связи с~этим 
в~исследовании были рассмотрены несколько способов формирования 
множества состояний предлагаемых математических моделей.
  
  В ходе исследования проверено большое число статистических гипотез, 
связанных со свойствами марковости и~однородности. Эти гипотезы 
проверялись при различных параметрах (число наблюдений, длительность 
периода наблюдений, уровень значимости критерия или вероятность ошибки 
при принятии гипотезы). В~своей совокупности проведенные проверки гипотез 
дали возможность установить ряд объективных закономерностей, связанных 
с~поведением реальных процессов. 

Кроме того, были построены 
статистические оценки мат\-риц вероятностей перехода для рас\-смат\-ри\-ва\-емых 
моделей и~сделаны некоторые выводы о~характере их изменения во времени. 

\section{Дискретизация множества состояний процесса и~общий 
подход к~формированию марковской стохастической модели}

  Начнем изложение с~разработки методики создания стохастической модели 
с~дискретным временем и~дискретным множеством состояний. Отметим, что 
параметр времени изначально \mbox{дискретен}, поскольку наблюдаемые значения 
фиксируются ежедневно, когда проводятся биржевые операции. Наблюдаемые 
значения цены соответствующего финансового актива (цены бивалютной 
корзины) принадлежат некоторому подмножеству множества действительных 
чисел. Будем сначала предполагать, что это подмножество задано. При этом 
предположении осуществим переход к~стохастической модели с~дискретным 
множеством состояний.
  
  Обозначим через $\{ \widetilde{\xi_k}\}^\infty_{k=0}$ наблюдаемую 
случайную последовательность, которая описывает эволюцию состояния 
системы во времени. Данный процесс будем считать исходным стохастическим 
объектом. 
  
  В рассматриваемой стохастической модели управляющая система, 
проводящая внешние воздействия или интервенции, устанавливает конкретные 
границы, в~которых должны находиться состояния исходного процесса. 
Предположим, что\linebreak
 множество допустимых значений этого процесса 
представляет собой конечный интервал во множестве неотрицательных вещественных чисел 
$(x_0, x_1)\hm\subset [0,\infty)$. В~дальнейшем будем называть это множество 
дискретизируемым. Таким образом, множество недопустимых значений 
данного процесса является объединением интервалов
  $(0, x_0]\cup [x_1,\infty)$.
  
  Выберем достаточно большое целое положительное число~$N$ и~обозначим 
$$
\Delta= \fr{x_1-x_0}{N-1}\,.
$$ 

Разобьем интервал $(x_0, x_1)$ на малые 
интервалы длины~$\Delta$. Рассмотрим следующие множества вещественных 
чисел:
  \begin{multline*}
  \tilde{X}_0=[0,x_0]\,,\ \tilde{X}_1=\left( x_0, x_0+\Delta\right],\ldots\\
  \ldots , \tilde{X}_s= \left( x_0+(s-1)\Delta, x_0+s\Delta\right]\,,\\
   s=1,2,\ldots , N- 2\,;
  \end{multline*}
  
  \vspace*{-12pt}
  
  \noindent
\begin{multline*}
  \tilde{X}_{N-1} =\left( x_0 +(N-2)\Delta, x_0+(N-1)\Delta=x_1\right]\,,\\ 
\tilde{X}_N=\left[ x_1,\infty\right).
\end{multline*}
  
  Проведем преобразование введенных множеств по правилу
  \begin{multline*}
  X_0=\tilde{X}_0, \ X_2=\tilde{X}_1,\ X_{s+1}=\tilde{X}_s,\\ s=1,2,\ldots, N-1;\quad 
X_1=\tilde{X}_N\,.
\end{multline*}
  
  Теперь определим новую случайную последовательность $\left\{ 
\xi_k\right\}^\infty_{k=0}$ при помощи соотношений: 
  $$
  \xi_k=\begin{cases}
  0\,,&\ \mbox{если } \tilde{\xi}_k\in \tilde{X}_0\,;\\
  s\,, &\ \mbox{если } \tilde{\xi}_k\in \tilde{X}_s\,,\ s=2,3,\ldots, N\,;\\
  1\,, &\ \mbox{если } \xi_k\in X_1\,.
  \end{cases}
  $$ 
  
  Из проведенного построения следует, что случайная последовательность 
$\left\{ \xi_k\right\}^\infty_{k=0}$ принимает значения в~конечном множестве 
$\{ 0,2,\ldots , N, 1\}$, в~котором состояния $\{2,3,\ldots , N\}$ являются 
внутренними и~допустимыми, а состояния $\{0\}$ и~$\{1\}$~--- граничными 
и~недопустимыми. Предположим, что проб\-ле\-ма управ\-ле\-ния этой случайной 
последовательностью\linebreak рассматривается как стохастическая задача 
о~настройке~\cite{1-shn}. Для применения теоретических результатов решения 
задачи о~настройке с~дискретным временем~[1, 2] необходимо при помощи 
\mbox{статистических} методов убедиться, что данная последовательность образует 
однородную цепь Маркова, в~которой состояния $\{0\}$ и~$\{1\}$ являются 
поглощающими.
  
  Предположим, что в~распоряжении исследователя имеется достаточно 
большое число наблюдений за реализациями случайной по\-сле\-до\-ва\-тель\-ности на 
интервалах времени между \mbox{последовательными} интервенциями. Зафиксируем 
один из таких интервалов. Обозначим через~$v_{ij}$ чис\-ло переходов процесса 
за один шаг на указанном интервале из подмножества со\-сто\-яний~$\tilde{X}_i$ 
в~подмножество~$\tilde{X}_j$. Обозначим так\-же через~$v_i$ общее число 
переходов за один шаг из состояния~$i$ во все со\-сто\-яния $\tilde{X}\hm= 
\mathop{\cup}\nolimits^N_{i=0} \tilde{X}_i\hm= [0,\infty)$. Рассмотрим 
сле\-ду\-ющие статистические характеристики, опре\-де\-ля\-емые по наблюдениям: 
  $$
  \hat{p}_{ij} =\fr{v_{ij}}{v_i}\,,\enskip i \in \{2,3,\ldots , N\}\,,\ j\in \{0,1,\ldots , 
N\}.
  $$
  
  При этом выполняется условие нормировки $\sum\nolimits^N_{j=0} 
\hat{p}_{ij} \hm=1$, $i\hm\in \{2,3,\ldots, N\}$. Полагая дополнительно 
  \begin{gather*}
  \hat{p}_{00} =1\,,\enskip \hat{p}_{0j}=0\,,\enskip j\in \{1,2,\ldots , N\}\,;\\
  \hat{p}_{11}=1\,,\enskip \hat{p}_{1j}=0\,,\enskip j\in \{1,2,\ldots N\}\,,
  \end{gather*}
получим стохастическую матрицу $\mathbf{P}\hm= \left( \hat{p}_{ij}\right)$. 
Заметим, что структура этой матрицы соответствует структуре матрицы 
вероятностей перехода поглощающей цепи Маркова. Известно~\cite{4-shn, 5-shn}, 
что элементы этой матрицы представляют собой несмещенные и~состоятельные 
оценки вероятностей перехода введенной случайной последовательности 
$\{\xi_k\}^\infty_{k=0}$. 
  
  Заметим, что изложенная выше процедура дискретизации может быть 
с~незначительными изменениями использована при переходе от исходного 
наблюдаемого процесса к~модели общей цепи Маркова с~дискретным  
(в~част\-ности, конечным) множеством состояний. При этом в~качестве 
состояний новой цепи могут рассматриваться не только номера 
соответствующих подмножеств исходного дискретизируемого множества 
состояний, но и~произвольные элементы, принадлежащие этим подмножествам.
  
  Описанная выше процедура дискретизации связана с~разбиением исходного 
заданного множества возможных значений наблюдаемого процесса.\linebreak Однако, 
как уже отмечалось во введении, это множество может изменяться по ходу 
эволюции процесса. В~связи с~этим в~данной работе \mbox{предлагаются} три варианта 
определения конечного множества состояний, которые могут рассматриваться 
как множество состояний модели после дискретизации, т.\,е.\ множество 
состояний марковского процесса.
{\looseness=1

}

\smallskip
  
  \textbf{Первый вариант.} В~качестве исходного (дискретизируемого) 
множества состояний наблюдаемого процесса рассматривается некоторое 
достаточно <<широкое>> множество действительных чисел, которое включает 
в~себя все возможные валютные коридоры, т.\,е.\ все возможные значения, 
которые может принимать этот наблюдаемый процесс. Интервал времени, на 
котором наблюдаются значения процесса и~проверяются необходимые 
статистические гипотезы, может быть произвольным, однако при этом он не 
включает в~себя интервалы, на которых процесс выходит за границы валютного 
коридора. В~то же время сами границы валютного коридора на данном 
временн$\acute{\mbox{о}}$м интервале могут изменяться. Тогда очевидно, что на каждом 
интервале времени, на котором установлены конкретные границы валютного 
коридора, наблюдаемый процесс будет принимать только значения из 
соответствующего подмножества состояний. Дискретные значения состояний, 
не соответствующие установленному валютному коридору, в~определенной 
таким образом марковской модели будут недостижимы. Построенные оценки 
матриц вероятностей перехода за один шаг процесса будут отличаться друг от 
друга на тех интервалах времени, на которых установлены различные границы 
валютного коридора. Иллюстрация данного варианта дискретизации 
представлена на рис.~1,\,\textit{а}.



 

{ \begin{center}  %fig1
 \vspace*{-1pt}
    \mbox{%
\epsfxsize=79.101mm
\epsfbox{shn-1.eps}
}


\end{center}

\noindent
{{\figurename~1}\ \ \small{
Первый~(\textit{а}), второй~(\textit{б}) и~третий~(\textit{в}) подходы к~дискретизации
}}}

\vspace*{18pt}

\addtocounter{figure}{1}
  

\noindent
 Несмотря на очевидные недостатки такого подхода к~дискретизации, авторы 
считают целесообразным использовать его в~данном исследовании, поскольку 
он позволяет сделать некоторые общие выводы о характере наблюдаемого 
процесса.

\smallskip
  
  \textbf{Второй вариант.} Траектории наблюдаемого процесса 
рассматриваются только на интервалах времени, на которых границы 
валютного коридора остаются постоянными. Именно на этих интервалах 
времени проверяются статистические гипотезы, необходимые для того, чтобы 
установить наличие соответствующих свойств процесса. Исходное 
(дискретизируемое) множество состояний наблюдаемого процесса совпадает 
с~интервалом, граничными точками которого служат границы валютного 
коридора. При этом число элементов множества состояний после 
дискретизации постоянно для всех рассматриваемых интервалов времени, т.\,е.\ 
не зависит от конкретного валютного коридора. Иллюстрация данного способа 
дискретизации пред\-став\-ле\-на на рис.~1,\,\textit{б}.

\smallskip

  
  \textbf{Третий вариант.} Как и~в~предыдущем варианте, траектории 
наблюдаемого процесса рассматриваются только на интервалах времени, на 
которых границы валютного коридора остаются постоянными. Необходимые 
статистические гипотезы проверяются на этих интервалах времени. Однако, 
в~отличие от второго варианта, исходное (дискретизируемое) множество 
состояний наблюдаемого процесса совпадает с~множеством значений, которые 
реально принимает наблюдаемый процесс. Это множество принадлежит 
валютному коридору. Естественно, как и~для варианта~2, число элементов 
множества состояний после дискретизации постоянно для всех 
рассматриваемых интервалов времени, т.\,е.\ для всех соответствующих этим 
интервалам валютных коридоров. Иллюстрация данного варианта 
дискретизации представлена на рис.~1,\,\textit{в}. 


\section{Общая схема статистического исследования свойств 
наблюдаемого процесса}

  Приведем описание общей структуры данного исследования, включая 
используемые статистические методы.
  \begin{enumerate}[1.]
  \item Проводится дискретизация стохастической модели, т.\,е.\ переход 
  к~модели с~дискретным (конечным) множеством состояний. Процедура 
дискретизации подробно изложена в~разд.~2.
  \item Для каждого из трех вариантов дискретизации проводятся сбор 
и~необходимая обработка имеющихся статистических данных. После этого\linebreak по 
известному статистическому методу проверяется наличие общего марковского 
свойства у~наблюдаемого процесса. Иными словами, уста\-нав\-ли\-ва\-ет\-ся, 
обладает ли наблюдаемый процесс свойством марковости некоторого 
порядка~$s$, где $s\hm\geq 1$~--- некоторое целое число.
  \item После подтверждения общего марковского свойства для каждого из 
трех вариантов дискретизации проверяется гипотеза о том, что наблюдаемый 
процесс обладает марковским \mbox{свойством} первого порядка, т.\,е.\ образует 
классическую цепь Маркова с~дискретным временем и~дискретным (конечным) 
множеством состояний. 
  \item После подтверждения марковского свойства первого порядка для 
каждого из трех вариантов дискретизации проверяется гипотеза о том, что 
наблюдаемый процесс обладает свойством однородности, т.\,е.\ образует 
однородную цепь Маркова с~дискретным временем и~конечным множеством 
состояний.
  \end{enumerate}
  
  Уточним, что перечисленные выше свойства проверяются последовательно. 
Каждое последующее свойство проверяется при тех же условиях, при которых 
было подтверждено предыдущее.



\section{Проверка статистических гипотез о~независимости 
и~марковости некоторого фиксированного порядка}
  
  В данном разделе будут изложены результаты проверки общего марковского 
свойства на\-блю\-да\-емой случайной последовательности.
  
  Формально проверяется основная статистическая гипотеза, заключающаяся 
  в~том, что наблюдаемая случайная последовательность состоит из независимых 
случайных величин, против конкурирующей гипотезы, которая заключается 
в~том, что значения наблюдаемой последовательности зависимы и~образуют 
общую цепь Маркова порядка~$s$, где $s\hm\geq 1$~--- некоторое целое число. 
Теоретические основы такой проверки, а также метод ее проведения изложены в~работах~[4,  \S\,5.5; 5]. 
  
  В рамках исследования по данной задаче различения гипотез были проведены 
более 10\,000 статистических тестов. В~результате можно сделать несколько 
общих эвристических выводов.
  \begin{enumerate}[1.]
\item При уменьшении числа дискретных разбиений, возрастает число 
выборок, обладающих марковским свойством.
\item При увеличении временн$\acute{\mbox{о}}$го интервала воз\-рас\-та\-ет 
число выборок, обладающих марковским свойством.
\item Для первого способа дискретизации достаточно выбрать промежуток 
в~3~мес.\ или более с~раз\-би\-ени\-ем на~74~состояния или менее для 
достижения результата в~виде 100\% выборок, обла\-да\-ющих марковскими 
свойствами.
\end{enumerate}

{ \begin{center}  %fig2
 \vspace*{-1pt}
   \mbox{%
\epsfxsize=70.203mm
\epsfbox{shn-4.eps}
}

\end{center}

\vspace*{-3pt}

\noindent
{{\figurename~2}\ \ \small{
Выводы о марковости процесса при первом~(\textit{а}), втором~(\textit{б}) и~третьем~(\textit{в}) способах дискретизации
}}}

\vspace*{6pt}

\addtocounter{figure}{1}



\noindent
\begin{enumerate}[1.]
\setcounter{enumi}{3}
\item  Для второго способа дискретизации достаточно выбрать промежуток 
в~4~мес.\ или более с~раз\-би\-ени\-ем на~74~состояния или менее для 
достижения результата в~виде 100\% выборок, обла\-да\-ющих марковскими 
свойствами.
\item Для третьего способа дискретизации достаточно выбрать промежуток 
в~6~мес.\ или более с~раз\-би\-ени\-ем на~74~состояния или менее для 
достижения результата в~виде 100\% выборок, обла\-да\-ющих марковскими 
свойствами.
\end{enumerate}
  
  Сформулированные выводы проиллюстрированы диаграммами на рис.~2.
    Заметим, что в~этих\linebreak диаграммах по горизонтали при помощи исполь-\linebreak зования различных цветов 
    и~их насыщенности отоб\-ра\-жа\-ет\-ся доля выборок, в~которых соответствующая проверяемая гипотеза принимается, 
    а~также длительность интервалов времени, на которых формируются проверяемые выборки. Аналогичная структура характерна 
    и~для диаграмм, приводимых на рис.~3 и~4.

\vspace*{-6pt}

\section{Проверка статистических гипотез о~классической 
марковской зависимости и~марковской зависимости более высокого 
порядка} 

\vspace*{-2pt}
  
  В данном разделе изложены результаты проверки наличия классического 
марковского свойства первого порядка у наблюдаемой случайной 
последовательности. Формально проверяется основная статистическая 
гипотеза, заключающаяся в~том, что наблюдаемая случайная 
последовательность является цепью Маркова первого порядка, против 
конкурирующей гипотезы, которая заключается в~том, что значения 
наблюдаемой последовательности образуют общую цепь Маркова порядка~$s$, 
где $s$~--- некоторое целое число, большее единицы. Заметим, что в~настоящем 
исследовании конкурирующая гипотеза была проверена только для варианта 
$s\hm=2$. Теоретические основы такой проверки, а также метод ее проведения 
изложены в~работах [4, \S\,5.5;~5].
  
  По данной проблеме были проведены более 5\,000 статистических тестов. На 
основе анализа их результатов можно сделать несколько эмпирических 
выводов.
  \begin{enumerate}[1.]
\item При увеличении числа дискретных разбиений возрастает число 
выборок с~порядком, равным единице.
\item При увеличении временн$\acute{\mbox{о}}$го интервала воз\-рас\-та\-ет 
число выборок, обладающих единичным порядком.
\item Для первого способа дискретизации достаточно выбрать 
промежуток в~6~мес.\ или более с~раз\-би\-ени\-ем на~74 или более 
состояния для достижения результата в~виде 100\% выборок, обла\-да\-ющих 
единичным порядком.
\item Для второго способа дискретизации достаточно выбрать 
промежуток в~4~мес.\ или более с~раз\-би\-ени\-ем на~74 или более 
состояния для достижения результата в~виде 100\% выборок, обла\-да\-ющих 
единичным порядком.
\item Для третьего способа дискретизации достаточно выбрать 
промежуток в~2~мес.\ или более с~раз\-би\-ени\-ем на~74 или более 
состояния для достижения результата в~виде 100\% выборок, обла\-да\-ющих 
единичным порядком.
\end{enumerate}
  
  Сформулированные выводы проиллюстрированы диаграммами на рис.~3.
  
 \pagebreak

\end{multicols}

\begin{figure*} %fig3
\vspace*{1pt}
\begin{minipage}[t]{80mm}
  \begin{center}  
    \mbox{%
\epsfxsize=70.086mm
\epsfbox{shn-5.eps}
}

\end{center}
\vspace*{-11pt}
\Caption{Выводы о единичном порядке марковской цепи при первом~(\textit{а}), втором~(\textit{б}) и~третьем~(\textit{в}) способах дискретизации}
\end{minipage}
%\end{figure*}
\hfill
%\begin{figure*} %fig4
\vspace*{1pt}
\begin{minipage}[t]{80mm}
  \begin{center}  
    \mbox{%
\epsfxsize=70.086mm
\epsfbox{shn-6.eps}
}

\end{center}
\vspace*{-11pt}
\Caption{Выводы об однородности марковской цепи при первом~(\textit{а}), втором~(\textit{б}) и~третьем~(\textit{в}) способах дискретизации}
\end{minipage}
\end{figure*}


\begin{multicols}{2}
 


\section{Проверка статистических гипотез об однородности 
и~неоднородности наблюдаемого процесса}

\vspace*{-2pt}
  
  Завершающим этапом проводимого исследования стала проверка свойства 
однородности. Поскольку предполагается, что предварительно уже было 
установлено марковское свойство первого порядка у наблюдаемой случайной 
по\-сле\-до\-ва\-тель\-ности, то проверка гипотезы об однородности сводится 
к~сле\-ду\-юще\-му. Необходимо установить, что 
  случайные последовательности, 
об\-ра\-зу\-ющие цепи Маркова на разных интервалах времени, имеют одинаковые 
распределения вероятностей переходов. Такая задача также была решена 
в~статистической теории марковских цепей~[4, \S\,5.5, п.~5; 5]. Для 
проверки гипотезы однородности используется специальная статистика  
хи-квад\-рат с~соответствующим предельным распределением.
  
  В результате проведенных статистических испытаний установлено, что все 
временн$\acute{\mbox{ы}}$е выборки, обладающие свойством марковости первого порядка, 
в~100\% случаев обладают также и~свойством однородности. Результаты 
исследования проиллюстрированы диаграммами, представленными на рис.~4.

\vspace*{-8pt}

\section{Общие выводы о~вероятностных свойствах процесса 
изменения цены бивалютной корзины}

\vspace*{-3pt}
  
  Теперь можно сформулировать выводы, связанные с~исследованием в~целом, 
которые обобщают установленные опытным путем статистические 
закономерности. 
  \begin{enumerate}[1.]
\item При уменьшении числа элементов дискретных\linebreak разбиений возрастает 
число выборок, обла\-да\-ющих марковским свойством, порядком,\linebreak равным 
единице, и~свойством од\-но\-род\-ности. Заметим, что это свойство не является 
устойчивым для небольших временн$\acute{\mbox{ы}}$х интервалов 
и~со\-от\-вет\-ст\-ву\-ющих малых выборок.

Данную закономерность можно объяснить усложнением математической 
модели при уве-\linebreak\vspace*{-12pt}
\end{enumerate}

\pagebreak

\end{multicols}

\begin{figure*} %fig5
  \vspace*{1pt}
  \begin{center}  
    \mbox{%
\epsfxsize=162.869mm
\epsfbox{shn-7.eps}
}

\end{center}
\vspace*{-13pt}
   \Caption{Общие эмпирические выводы исследования
   при первом~(\textit{а}), втором~(\textit{б}) и~третьем~(\textit{в}) способах дискретизации}
   %\vspace*{-38pt}
   \end{figure*}
   
   \begin{multicols}{2}


\begin{enumerate}[1.]
\setcounter{enumi}{1}
\item[\,]
  личении чис\-ла элементов разбиений и~при\-бли\-же\-нии 
множества со\-сто\-яний к~исходному (непрерывному).
\item При увеличении временн$\acute{\mbox{о}}$го интервала воз\-рас\-та\-ет 
число выборок, об\-ла\-да\-ющих всеми тремя свойствами.

  
  Данную закономерность можно объяснить зависимостью результатов от 
объема данных. На сравнительно малых интервалах времени при\linebreak 
небольших объемах наблюдений выводы о~характере процесса неустойчивы. 
При увеличении рас\-смат\-ри\-ва\-емых интервалов времени и~соответствующем 
рос\-те объемов наблюдений ис\-поль\-зу\-емые статистические методы позволяют 
обнаружить некоторые устойчивые закономерности, характерные для 
ис\-сле\-ду\-емо\-го про-\linebreak цесса.
\end{enumerate}
  
  Сформулированные выше основные выводы проиллюстрированы 
диаграммами на рис.~5.

\begin{figure*} %fig6
\vspace*{1pt}
  \begin{center}  
    \mbox{%
\epsfxsize=157.357mm
\epsfbox{shn-8.eps}
}

\end{center}
\vspace*{-9pt}
\Caption{Матрицы оценок вероятностей перехода для марковской цепи с~37~состояниями, 
по\-стро\-ен\-ные на различных временных интервалах: (\textit{а})~01.01.2012--01.12.2013; (\textit{б})~01.11.2010--25.09.2011}
%\vspace*{-6pt}
\end{figure*}

  \vspace*{-6pt}

\section{Построение статистических оценок матриц вероятностей 
перехода}

  \vspace*{-2pt}
  
  Если в~результате проведенного исследования\linebreak установлено, что можно 
принять гипотезу о классическом марковском характере наблюдаемого 
стохастического процесса (случайный процесс описывается классической 
цепью Маркова первого \mbox{порядка}), а~также гипотезу об однородности процесса, 
то целесообразно построить статистическую оценку матрицы вероятностей 
перехода данного процесса, которая, как известно, служит основной 
вероятностной характеристикой цепи Маркова и~позволяет полностью 
описывать его эволюцию вероятностными методами. Для оценок элементов 
этой матрицы используются относительные частоты переходов процесса за 
одну единицу времени (или за один шаг процесса) из фиксированного 
состояния~$i$ во все остальные состояния, которые может принимать процесс. 

Методика по\-стро\-ения таких наборов относительных час\-тот описана в~разд.~2. 
Там же приведена классическая формула для оценки ве\-ро\-ят\-ности перехода. 

Такую мат\-ри\-цу оценок мож\-но по\-стро\-ить на любом интервале времени, на 
котором приняты гипотезы о~мар\-ко\-вости и~однородности на\-блю\-да\-емо\-го 
процесса. Это создаст воз\-мож\-ность сравнить оценки мат\-риц вероятностей 
перехода на разных интервалах времени и~охарактеризовать динамику 
изменения классической марковской мо\-дели.


  
  При построении оценок естественно использовать второй и~третий подходы 
  к~дискретизации, при которых множество состояний построенной дискретной 
марковской цепи не изменяется. Еще раз подчеркнем, что при принятых 
способах построения оценок границы валютных коридоров на 
рассматриваемых интервалах времени могут быть различными.
  
  Так как матрицы вероятностей перехода имеют большие размерности, 
используем их визуализацию в~виде рисунков, при создании которых принято 
следующее соглашение: чем ярче цвет, тем больше вероятность перехода.
  
  Для краткости ограничимся рассмотрением только второго подхода 
  к~дискретизации. При этом будем использовать данные наблюдений из 
сле\-ду\-ющих временн$\acute{\mbox{ы}}$х интервалов: 01.01.2012--01.12.2013  
и~01.11.2010--25.09.2011.
  
  Вначале рассмотрим первый временной интервал. Матрица оценок 
вероятностей перехода для дискретной марковской цепи,  
име\-ющей~37~состояний, визуально представлена на рис.~6,\,\textit{а}.


  
  Далее перейдем ко второму временн$\acute{\mbox{о}}$му интервалу. 
Матрица оценок вероятностей перехода для дискретной марковской цепи, 
также име\-ющей~37~состояний, визуально представлена на рис.~6,\,\textit{б}.




  
  Для более наглядного сравнения построим мат\-ри\-цу, определяющую разность 
по абсолютной величине между всеми элементами матриц вероятностей 
перехода на двух рассмотренных временн$\acute{\mbox{ы}}$х интервалах. 
Соответствующая матрица визуально представлена на рис.~7.
  

  
  Качественные выводы из сравнения матриц оценок вероятностей перехода 
заключаются в~следующем. В~приведенном примере число элементов этих 
матриц, которые сильно отличаются друг от друга (выражаемое числом ярких 
светлых точек на визуализации матрицы разностей, см.\ рис.~7), сравнительно невелико. 
В~то же время проблема изменения матрицы оценок вероятностей перехода 
марковской цепи для разных интервалов времени, на которых построены 
соответствующие оценки, нуждается в~дальнейшем, более глубоком 
иссле-\linebreak\vspace*{-12pt}

{ \begin{center}  %fig7
 \vspace*{-4pt}
    \mbox{%
\epsfxsize=75.89mm
\epsfbox{shn-10.eps}
}
\end{center}

\vspace*{-3pt}

\noindent
{{\figurename~7}\ \ \small{
Матрица, характеризующая разности элементов матриц оценок вероятностей перехода 
на разных временных интервалах 
}}}

\vspace*{9pt}

\addtocounter{figure}{1}

\noindent
довании. Матрицы вероятностей перехода можно использовать для 
решения многих задач, связанных с~исследованием марковских случайных 
процессов, и, в~частности, для исследования проблем прогнозирования. 
В~связи с~этим исследование влияния изменчивости матрицы оценок 
вероятностей перехода на вероятностные характеристики созданной 
марковской модели представляет собой важную и~актуальную проблему. 



\vspace*{-9pt}

\section{Заключение}

\vspace*{-2pt}

  Настоящую работу необходимо рассматривать как первую часть большого 
научного исследования реальных стохастических процессов, происходящих на 
валютном рынке РФ. В~част\-ности, при продолжении исследований можно 
дополнительно использовать результаты и~методы классической 
математической статистики, не связанные непосредственно со статистическими 
методами, разработанными для анализа марковских процессов. Комплексное 
использование различных методов математической статистики может оказаться 
весьма плодотворным.
  
  В то же время используемые в~работе статистические методы можно 
применять для анализа не только экономических процессов с~интервенциями, 
но и~многих других стохастических процессов, происходящих на товарных, 
фондовых и~валютных рынках.

\vspace*{-9pt}
  
{\small\frenchspacing
 {%\baselineskip=10.8pt
 %\addcontentsline{toc}{section}{References}
 \begin{thebibliography}{9}
 
 \vspace*{-2pt}
 
  \bibitem{1-shn}
  \Au{Shnurkov P.\,V.} Optimal control problem in a stochastic model with periodic hits on the 
boundary of a~given subset of the state set (tuning problem).~---  Cornell 
University, 2017. arXiv: 1709.03442 [math.OC]. 16~p.  {\sf 
https://arxiv.\linebreak org/abs/1709.03442v1}.
  \bibitem{2-shn}
  \Au{Shnurkov P.\,V., Novikov~D.\,A.} Analysis of the problem of intervention control in the 
economy on the basis of solving the problem of tuning.~--- Cornell University, 2018. arXiv: 1811.10993 [q-fin.GN]. 15~p. 
{\sf 
https://arxiv.\linebreak org/abs/1811.10993}.
  \bibitem{3-shn}
   \Au{Шнурков П.\,В., Новиков~Д.\,A.} О~концепции стохастической модели 
   с~управлением 
в~моменты выхода процесса на границу заданного подмножества множества состояний~// 
Информатика и~её применение, 2020. Т.~14. Вып.~3. С.~104--111.

  \bibitem{5-shn} %4
   \Au{Ивченко Г.\,И., Медведев~Ю.\,И.} Введение в~математическую статистику.~--- М.: 
ЛКИ, 2010. 600~с.

  \bibitem{4-shn} %5
   \Au{Ивченко Г.\,И., Медведев~Ю.\,И.} Дискретные распределения.  
Ве\-ро\-ят\-ност\-но-ста\-ти\-сти\-че\-ский справочник: Многомерные распределения.~--- М.: 
ЛЕНАНД, 2016. 336~с. 

  \bibitem{6-shn}
   Стоимость бивалютной корзины~// Банк России. {\sf https://www.cbr.ru/archive/db/bicurbase}.
  \end{thebibliography}

 }
 }

\end{multicols}

\vspace*{-9pt}

\hfill{\small\textit{Поступила в~редакцию 13.07.22}}

\vspace*{6pt}

%\pagebreak

%\newpage

%\vspace*{-28pt}

\hrule

\vspace*{2pt}

\hrule

%\vspace*{-2pt}

\def\tit{SOME RESULTS OF~THE~ANALYSIS OF~THE~PROCESS\\ OF~CHANGING THE~PRICE 
OF~A~DUAL CURRENCY BASKET\\ BASED ON~RANDOM PROCESS STATISTICS METHODS}


\def\titkol{Some results of~the~analysis of~the~process of changing the~price 
of~a~dual currency basket based on~random process statistics} % methods}


\def\aut{P.\,V.~Shnurkov and M.\,A.~Migulya}

\def\autkol{P.\,V.~Shnurkov and M.\,A.~Migulya}

\titel{\tit}{\aut}{\autkol}{\titkol}

\vspace*{-15pt}


\noindent
National Research University Higher School of Economics, 34~Tallinskaya Str., Moscow 123458, 
Russian Federation


\def\leftfootline{\small{\textbf{\thepage}
\hfill INFORMATIKA I EE PRIMENENIYA~--- INFORMATICS AND
APPLICATIONS\ \ \ 2022\ \ \ volume~16\ \ \ issue\ 3}
}%
 \def\rightfootline{\small{INFORMATIKA I EE PRIMENENIYA~---
INFORMATICS AND APPLICATIONS\ \ \ 2022\ \ \ volume~16\ \ \ issue\ 3
\hfill \textbf{\thepage}}}

\vspace*{2pt} 
  




\Abste{The work is devoted to the study of the process of changing the price of the so-called 
dual-currency basket in the foreign exchange market of the Russian Federation in the presence of interventions 
carried out by
the Central Bank. The main goal of the work is to verify the compliance of the observed 
process with a stochastic\linebreak\vspace*{-12pt}}

\Abstend{Markov model with discrete time and a discrete set of states. The theoretical basis 
of the study is the methods of statistics of Markov random processes. As a result, conditions are established 
under which the real process can be adequately described by the indicated Markov model.}

\KWE{Markov chain with discrete time and discrete set of states; statistics of Markov random 
processes; stochastic models of the evolution of processes in financial markets; stochastic models of 
interventions; price of a~dual-currency basket}

\DOI{10.14357/19922264220303}

%\vspace*{-16pt}

%\Ack
%\noindent





\vspace*{-4pt}

  \begin{multicols}{2}

\renewcommand{\bibname}{\protect\rmfamily References}
%\renewcommand{\bibname}{\large\protect\rm References}

{\small\frenchspacing
 {%\baselineskip=10.8pt
 \addcontentsline{toc}{section}{References}
 \begin{thebibliography}{9}
 
 \vspace*{-4pt}

  \bibitem{1-shn-1}
\Aue{Shnurkov, P.\,V.} 2017. Optimal control problem in a stochastic model with periodic hits on the 
boundary of a given subset of the state set (tuning problem). Cornell University. arXiv.org. 16~p. Available at: {\sf 
https://arxiv.org/\linebreak  abs/1709.03442v1} (accessed July~24, 2022).
  \bibitem{2-shn-1}
\Aue{Shnurkov, P.\,V., and D.\,A.~Novikov.} 2018. Analysis of the problem of intervention control in the 
economy on the basis of solving the problem of tuning. Cornell University.\linebreak arXiv.org. 15~p. Available at: {\sf 
https://arxiv.org/abs/ 1811.10993} (accessed July~24, 2022). 

\bibitem{3-shn-1}
   \Aue{Shnurkov, P.\,V., Novikov~D.\,A.} 2020. O~kontseptsii sto\-kha\-sti\-che\-skoy modeli 
s~upravlrniem v~momenty vykhoda pro\-tses\-sa na granitsu zadannogo podmnozhestva mno\-zhe\-st\-va 
sostoyaniy [On the concept of~a~stochastic model with~control at~the~moments of~the~process 
at~the~border of~a~presented subset of~multiple states]. \textit{Informatika i~ee Primeneniya~--- 
Inform. Appl.} 14(3):104--111.

%\columnbreak

  
  \bibitem{5-shn-1}
\Aue{Ivchenko, G.\,I., and Y.\,I.~Medvedev}. 2010. \textit{Vvedenie v~matematicheskuyu statistiku} 
[Introduction to mathematical statistics]. Moscow: LKI. 600~p.

\bibitem{4-shn-1}
\Aue{Ivchenko, G.\,I., and Y.\,I.~Medvedev.} 2016. \textit{Diskretnye raspredeleniya.  
Veroyatnostno-statisticheskiy spra\-voch\-nik: Mnogomernye raspredeleniya} [Discrete distributions. 
Probabilistic-statistical handbook: Multivariate distributions]. Moscow: Lenand. 336~p.

  \bibitem{6-shn-1}
Stoimost' bivalyutnoy korziny [Dual currency basket value]. Bank of Russia. Available at:  {\sf 
https://www.cbr.ru/\linebreak archive/db/bicurbase/} (accessed July~24, 2022).
\end{thebibliography}

 }
 }

\end{multicols}

\vspace*{-6pt}

\hfill{\small\textit{Received July 13, 2022}}

\Contr

\noindent
\textbf{Shnurkov Peter V.} (b.\ 1953)~--- Candidate of Science (PhD) in physics and mathematics, associate 
professor, National Research University Higher School of Economics, 34~Tallinskaya Str., Moscow 123458, 
Russian Federation; \mbox{pshnurkov@hse.ru}

\vspace*{3pt}

\noindent
\textbf{Migulya Maxim A.} (b.\ 2001)~--- bachelor, National Research University Higher School of 
Economics, 34~Tallinskaya Str., Moscow 123458, Russian Federation; \mbox{maxim.migulya@gmail.com}
  
\label{end\stat}

\renewcommand{\bibname}{\protect\rm Литература}    