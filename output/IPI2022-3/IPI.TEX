\documentclass[10pt]{book}
\usepackage[utf8]{inputenc}

\usepackage{latexsym,amssymb,amsfonts,amsmath,amsxtra,dsfont,
indentfirst,shapepar,%fleqn,%
picinpar,shadow,floatflt,enumerate,multicol,colortbl,moreverb,cite,ipi}

\usepackage{rotating}
\usepackage{mathrsfs}
\usepackage[noend]{algorithmic}
\usepackage{ulem}
\usepackage{graphicx}
%\usepackage{algorithm2e}
\usepackage[linesnumbered,boxed,ruled]{algorithm2e}
%\usepackage{xypic}
\usepackage{oldgerm}
\usepackage{epic}
\usepackage{eepic}

\SetAlgorithmName{Algorithm}{алгоритм}{Список алгоритмов}

%из Дюковой

\newcommand{\algKeyword}[1]{{\bf #1}}
\newcommand{\Proc}[1]{\text{\tt #1}}
\def\CALL{\algKeyword{call}~}

\newenvironment{AlgProcedure}[1]
{
\small
\medskip
%    \hrule
\medskip
\algKeyword{PROCEDURE} #1
\begin{algorithmic}[1]}
{\end{algorithmic}
%    \hrule
\bigskip
}

\def\CALL{\algKeyword{call}~}

%конец для Дюковой

%\RequirePackage[ruled]{algorithm}


\input{epsf}

%\nofiles

%\includeonly{avtor}    %pdf
%\includeonly{podgot-rus-site,podgot-eng-site}  
%\includeonly{podgot-rus,podgot-eng}  
%\includeonly{ipi-ind} 
%\includeonly{index-15i}
%\includeonly{toc-rus, toc-en}
%\includeonly{toc-rus}
%\includeonly{toc-en} 
%\includeonly{popravka}



%\includeonly{shvedov}        %1                           %1pdf+авт+
%\includeonly{bosov}                      %2pdf+авт+?? новые правки
%\includeonly{shnurkov}       %3          %3pdf+авт+??
%\includeonly{gorshenin}      %4  %pdf???????
%\includeonly{vasiliev}                                    %5pdf+авт+
%\includeonly{malashenko}                                  %6рdf+авт+
%\includeonly{krivenko}       %7                           %7pdf+авт+
%\includeonly{zatsman}        %8                           %8pdf+авт+
%\includeonly{nuriev}                                      %9pdf+авт+
%\includeonly{zeifman}        %10                          %10pdf+авт+
%\includeonly{aliu}           %11                          %pdf+авт+
%\includeonly{burtseva}       %12                          %pdf+авт+
%\includeonly{grusho}         %13                          %13pdf+авт+
%\includeonly{dubanov}        %14                          %14pdf+авт+


%%%%%%%%%%%%%%%%%%%\includeonly{nekrolog-new}



%\includeonly{rekl}




\usepackage{acad}
%\usepackage{courier}
\usepackage{decor}
\usepackage{newton}
\usepackage{pragmatica}
\usepackage{zapfchan}
\usepackage{petrotex}
\usepackage{bm}                     % полужирные греческие буквы
\usepackage{upgreek}                % прямые греческие буквы \upalpha
\usepackage{eufrak}
\usepackage{verbatim}

\renewcommand{\bottomfraction}{0.99}
\renewcommand{\topfraction}{0.99}
\renewcommand{\textfraction}{0.01}

\setcounter{secnumdepth}{1} %здесь - 3 + chapter = 4

\arraycolsep=1.5pt

%\usepackage[pdftex]{graphicx}

%\usepackage{oz}

%NEW COMMANDS



\renewcommand*{\hm}[1]{#1\nobreak\discretionary{}%
            {\hbox{$\mathsurround=0pt #1$}}{}} %% Дублирует знаки операций
                               %при переносе в формуле (перед знаком, который
                               %надо продублировать ставится команда \hm)
                               
                               \newcommand{\PRB}{\begin{picture}(22.5,11)
      \spline(1,8)(4,10)(7,10.5)(10,11)(13,11)(16,10.5)(19,10)(22,8)
               \put(0,0){$P_{i-1}P_{t_{t-1}}$} \end{picture}}

\newcommand{\prb}{\begin{picture}(15.5,9)
      \spline(1,6)(3,8)(5,8.5)(7,9)(9,9)(11,8.5)(13,8)(15,6)
               \put(0,0){$PP_t$} \end{picture}}
               
                 \newcommand{\PRDN}{\begin{picture}(40,11)
      \spline(4,11.5)(7,10.5)(12,10)(16,9)(20,9)(24,10)(29,10.5)(32,11.5)
               \put(0,0){$P_{i-1}P_{t_{t-1}}$} \end{picture}}

\newcommand{\prdn}{\begin{picture}(18,11)
      \spline(3,10.5)(4,10)(6,9)(8,8.5)(10,8.5)(12,9)(14,10)(15,10.5)
               \put(0,0){$PP_t$} \end{picture}}




%\newcommand{\endproof}{\hfill$\Box$}
\renewcommand{\r}{\mathbb{R}}
%\newcommand{\I}{{\rm I\hspace{-0.7mm}I}}
%\newcommand{\Ikl}{{\tt{1}}\hspace*{-1.44mm}\mathtt{1}}
\newcommand{\Ik}{\mbox{{\small \tt {1}}\hspace{-1.3mm}{\tt 1}}}
\newcommand{\argmin}{\mathop{\mathrm{arg}\,\mathrm{min}}}
\newcommand{\argmax}{\mathop{\mathrm{arg}\,\mathrm{max}}}
%\newcommand{\capr}{\mathop{\cap\,}}
%\newcommand{\cupr}{\mathop{\cup\,}}
%\def\argmin{\mathop{arg\,min}}

\def\vrp{\varphi}
\def\prt{\partial}
\def\mm{{\sf M}}
\def\modnop#1{\mathop{#1}\limits_{n}}
\def\eam{\mathbin{{\mathop{=}\limits^{\mathrm{def}}}}}
\def\dey#1#2{#1 (#2)}
\def\deyc#1#2{#1 \cdot  #2}
\def\ra#1{\;\mathop{\to}\limits^{#1}\;}
\def\raz#1{\;\mathop{\longrightarrow}\limits^{\!\!\!#1}\;}
\def\ral#1{\;\mathop{\longrightarrow}\limits^{#1}\;}

\newcommand{\Nor}{\mathcal{N}}
\newcommand{\T}{\mathbb{T}}
\newcommand{\Z}{\mathbb{Z}}



\newcommand{\il}[2]{\int\limits_{#1}^{#2}}%интеграл с пределами #1 и #2

\def\sm2{\mathop {\sum\limits^{n^\Theta}\sum\limits^{n^\Theta}}}
\def\sss{\sum\limits}
\def\tr{,\,\ldots\,,\,}
\def\rk{\right]}
\def\lk{\left[}
\def\rf{\right\}}
\def\lf{\left\{}
\def\lv{\,\left\vert}
\def\rv{\right\vert\,}
\def\iii{\int\limits}
\def\iin{\int\limits_{-\infty}^\infty}
\def\rrv{\right\vert}


\def\ee{{\cal E}}
\def\ww{{\cal W}}
\def\yy{{\cal Y}}
\def\vv{{\cal V}}

\newcommand{\R}{\mathbb R}
\newcommand{\E}{\mathbb E}
\newcommand{\N}{\mathbb N}

\renewcommand{\P}{\mathbb{P}}

\newcommand{\h}{{\bf H}}
\newcommand{\p}{{\sf P}}  % вероятность

\newcommand{\e}{{\sf E}}  % мат. ожидание
\newcommand{\D}{{\sf D}}  % дисперсия
\newcommand{\eps}{\varepsilon}
\newcommand{\vw}{{\mathbf w}}
\newcommand{\vp}{{\mathbf p}}
\newcommand{\vz}{{\mathbf z}}
\newcommand{\vx}{{\mathbf x}}
\newcommand{\vf}{{\mathbf f}}
\newcommand{\F}{{\mathcal F}}
\def\ap{{\mathrm{ЭР}}}
\newcommand{\ud}{\Delta_n} %uniform ditance
\newcommand{\nud}{\Delta_n(x)}
%\renewcommand{\Re}{\mathrm{Re}\,}

\newcommand{\abs}[1]{\left\vert#1\right\vert}

\newcommand{\norm}[1]{\left\Vert#1\right\Vert}
\def\da{(\Delta_t,A)}

\newcommand{\corr}{\mathrm{corr}}

\newcommand{\cov}{\mathrm{cov}}
\newcommand{\Expect}{\mathbb{E}}

\def\w{\omega}
\def\W{\Omega}

\def\inh{\int\limits_{nh}^{(n+1)h}}

\def\sumin{\sum_{i=1}^N}


\def\bxt{(Y,t)}
\def\xt{(y,t)}

\def\ovth{{\fr{\tau-nh}{h}}}
\def\ov{\overline}
\def\tm{\tilde m}
\def\tl{\tilde\lambda}
\def\tB{\widetilde B}
\def\tb{\tilde b}
\def\ld{\ldots}
\def\cd{\cdots}


\DeclareMathOperator{\sign}{sign}

%\newcommand{\gr}{{\geqslant}}


\newcommand{\g}{\mbox{\textit{g}}}

\renewcommand{\la}{\lambda}
\newcommand{\si}{\sigma}
\newcommand{\alp}{\alpha}

\newcommand{\pto}{\stackrel{P}{\longrightarrow}} % сходимость по веpоятности

\newcommand{\eqd}{\stackrel{\mathrm{d}}{=}} % равенство по pаспpеделению
\newcommand{\eqdelta}{\stackrel{\triangle}{=}} % равенство по pаспpеделению

\def\be#1{\begin{equation}\label{#1}}
\def\ee{\end{equation}}
\def\re#1{(\ref{#1})}

\def\bn{\begin{enumerate}}
\def\en{\end{enumerate}}
\def\bi{\begin{itemize}}
\def\ei{\end{itemize}}
%\def\i{\item}

%\newcommand{\kp}{\kappa}
%\def\Q{{\cal Q}} \def\H{{\cal H}}
%\newcommand{\bet}{\beta_{2+\delta}}


%\newtheorem{definition}{Определение}
%\renewcommand{\thedefinition}{\arabic{definition}.}
%END NEW COMMANDS

%\renewcommand{\baselinestretch}{1.2}

%\pagestyle{myheadings}

\setlength{\textwidth}{167mm}      % 122mm
\setlength{\textheight}{658pt}
%\setlength{\textheight}{635.6pt}
\setlength{\columnsep}{4.5mm}

\setcounter{secnumdepth}{4}

%\addtolength{\headheight}{2pt}
%\addtolength{\headsep}{-2mm}

\addtolength{\topmargin}{-7mm}  % for printing


%\hoffset=-30mm  % From Yap
\hoffset=-23mm  % From Acrobat

%\voffset=0mm % From Yap
\voffset=-5mm   % From Acrobat

%\addtolength{\evensidemargin}{-2.5mm} % for printing
%\addtolength{\oddsidemargin}{2.5mm}  % for printing

\addtolength{\evensidemargin}{-12mm} % for printing
\addtolength{\oddsidemargin}{8mm}  % for printing

%\renewcommand{\thefootnote}{\fnsymbol{footnote}}
%\renewcommand{\thefootnote}{\arabic{footnote}}
\renewcommand{\figurename}{\protect\bf Рис.}
\renewcommand{\tablename}{\protect\bf Таблица}

\newcommand{\Caption}[1]{\caption{\protect\small %\baselineskip=2.5ex
#1}}

\renewcommand{\thefigure}{\arabic{figure}}
\renewcommand{\thetable}{\arabic{table}}
\renewcommand{\theequation}{\arabic{equation}}
\renewcommand{\thesection}{\arabic{section}}

\renewcommand{\contentsname}{СОДЕРЖАНИЕ}
\newcommand{\fr}[2]{\displaystyle\frac{\displaystyle #1\mathstrut}{\displaystyle #2\mathstrut}}

%\renewcommand{\thefootnote}{\fnsymbol{footnote}}
%\newcommand{\g}{\mbox{\textit{g}}}

%\newcommand{\Caption}[1]{\caption{\protect\small\baselineskip=2ex #1}}
\newcounter{razdel}
\setcounter{razdel}{0}

\def\god{2022}
\def\tom{16}
\def\vyp{3}


\newcommand{\titel}[4]{%
\

\vspace*{5pt}

\ifodd\therazdel {\raggedright\noindent\Large\textrm\textbf
 \lineskip .75em
  \baselineskip=3.2ex #1 \par}
\vskip 1em {\noindent\large\textrm\textbf #2 \par}
\addcontentsline{toc}{subsection}{{\textrm\textbf #1}\protect\newline #2}
\def\rightheadline{\underline{\noindent\hbox to \textwidth{\hfill\small\textrm{#4}
%\hfill \large\bf\thepage
}}}
\def\leftheadline{\underline{\noindent\parbox{\textwidth}{
%\raggedleft\large\bf\thepage \hfill
\small\textit{#3}\hfill}}}
\def\leftfootline{\small{\textbf{\thepage}
\hfill ИНФОРМАТИКА И ЕЁ ПРИМЕНЕНИЯ\ \ \ том~\tom\ \ \ выпуск~\vyp\ \ \ \god}
}%
 \def\rightfootline{\small{ИНФОРМАТИКА И ЕЁ ПРИМЕНЕНИЯ\ \ \ том~\tom\ \ \ выпуск~\vyp\ \ \ \god
\hfill \textbf{\thepage}}}
\vskip 2em \setcounter{figure}{0}
\setcounter{table}{0}
\setcounter{equation}{0}
\setcounter{section}{0}
\setcounter{subsection}{0}
\setcounter{subsubsection}{0}
\setcounter{footnote}{0}
\setcounter{razdel}{0}
%\end{flushleft}
\else {
 \raggedright\noindent\Large\textrm\textbf
 \lineskip .75em
\baselineskip=3.2ex #1 \par} \vskip 1em
%\begin{flushleft}
{\noindent\large\textrm\textbf #2 \par}
\addcontentsline{toc}{subsection}{{\textrm\textbf #1}\protect\newline #2}
\def\rightheadline{\underline{\noindent\hbox to \textwidth{\hfill\small\textrm{#4}
%\hfill \large\bf\thepage
}}}
\def\leftheadline{\underline{\noindent\parbox{\textwidth}{%\raggedleft\large\bf\thepage \hfill
\small\textit{#3}\hfill}}}
\def\leftfootline{\small{\textbf{\thepage}
\hfill ИНФОРМАТИКА И ЕЁ ПРИМЕНЕНИЯ\ \ \ том~\tom\ \ \ выпуск~\vyp\ \ \ \god}
}%
 \def\rightfootline{\small{ИНФОРМАТИКА И ЕЁ ПРИМЕНЕНИЯ\ \ \ том~16\ \ \ выпуск~\vyp\ \ \ 2022
\hfill \textbf{\thepage}}} \vskip 2em \setcounter{figure}{0}
\setcounter{table}{0} \setcounter{equation}{0} \setcounter{section}{0}
\setcounter{subsection}{0} \setcounter{subsubsection}{0}
\setcounter{footnote}{0}
%\end{flushleft}
\fi}

\newcommand{\titelr}[2]{%
\

\vspace*{5pt}

\ifodd\therazdel {\raggedright\noindent%\Large\textrm\textbf
 \lineskip .75em
  \baselineskip=3.2ex #1 \par}
\vskip 1em {\noindent\normalsize\textrm\textbf #2 \par}
\else {
 \raggedright\noindent\Large\textrm\textbf
 \lineskip .75em
\baselineskip=3.2ex #1 \par} \vskip 1em
%\begin{flushleft}
{\noindent\large\textrm\textbf #2 \par
%\noindent\normalsize\textrm\textbf #2 \par
} \fi}

\newcommand{\titele}[5]{%
\

%\vspace*{5pt}

\ifodd\therazdel {\raggedright\noindent\large
\textrm\textbf
 \lineskip .75em
%  \baselineskip=3.2ex
#1 \par}
\vskip .5em {\noindent\large\textrm\textbf #2 \par}
\vskip .5em
 {\noindent\textrm #3 \par}
\addcontentsline{toc}{subsection}{{\textrm\textbf #1}\protect\newline #2}
\def\rightheadline{\underline{\noindent\hbox to \textwidth{\hfill\small\textrm{#4}
%\hfill \large\bf\thepage
}}}
\def\leftheadline{\underline{\noindent\parbox{\textwidth}{
%\raggedleft\large\bf\thepage \hfill
\small\textrm{#5}\hfill}}}
\def\leftfootline{\small{\textbf{\thepage}
\hfill ИНФОРМАТИКА И ЕЁ ПРИМЕНЕНИЯ\ \ \ том~16\ \ \ выпуск~3\ \ \ 2022}
}%
 \def\rightfootline{\small{ИНФОРМАТИКА И ЕЁ ПРИМЕНЕНИЯ\ \ \ том~16\ \ \ выпуск~3\ \ \ 2022
\hfill \textbf{\thepage}}} \vskip 1em \setcounter{figure}{0}
\setcounter{table}{0} \setcounter{equation}{0} \setcounter{section}{0}
\setcounter{subsection}{0} \setcounter{subsubsection}{0}
\setcounter{footnote}{0} \setcounter{razdel}{0}
%\end{flushleft}
\else {
 \raggedright\noindent\large
 \textrm\textbf
 \lineskip .75em
%\baselineskip=3.2ex
#1 \par} \vskip .5em
%\begin{flushleft}
{\noindent\large\textrm\textbf #2 \par} \vskip .5em
 {\noindent\textrm #3 \par}
\addcontentsline{toc}{subsection}{{\textrm\textbf #1}\protect\newline #2}
\def\rightheadline{\underline{\noindent\hbox to \textwidth{\hfill\small\textrm{#4}
%\hfill \large\bf\thepage
}}}
\def\leftheadline{\underline{\noindent\parbox{\textwidth}{%\raggedleft\large\bf\thepage \hfill
\small\textrm{#5}\hfill}}}
\def\leftfootline{\small{\textbf{\thepage}
\hfill ИНФОРМАТИКА И ЕЁ ПРИМЕНЕНИЯ\ \ \ том~16\ \ \ выпуск~3\ \ \ 2022}
}%
 \def\rightfootline{\small{ИНФОРМАТИКА И ЕЁ ПРИМЕНЕНИЯ\ \ \ том~16\ \ \ выпуск~3\ \ \ 2022
\hfill \textbf{\thepage}}} \vskip 1em \setcounter{figure}{0}
\setcounter{table}{0} \setcounter{equation}{0} \setcounter{section}{0}
\setcounter{subsection}{0} \setcounter{subsubsection}{0}
\setcounter{footnote}{0}
%\end{flushleft}
\fi}

\def\Abst#1{
\begin{center}\small\nwt
\parbox{150mm}{%\baselineskip=2.5ex
\textbf{Аннотация:}\ \
%\hspace*{\parindent}
#1}
\end{center}}
\def\Abste#1{
\begin{center}\small\nwt
\parbox{150mm}{%\baselineskip=2.5ex
\textbf{Abstract:}\ \
%\hspace*{\parindent}
#1}
\end{center}}

\def\DOI#1{
\begin{center}\small\nwt
\parbox{150mm}{%\baselineskip=2.5ex
\textbf{DOI:}\ \
%\hspace*{\parindent}
#1}
\end{center}}

\def\Abstend#1{
\begin{center}\small\nwt
\parbox{150mm}{%\baselineskip=2.5ex
%\hspace*{\parindent}
#1}
\end{center}}


\def\KW#1{
\begin{center}\small\nwt
\parbox{150mm}{%\baselineskip=2.5ex
\textbf{Ключевые слова:}\ \ #1}
\end{center}}

\def\KWE#1{
\begin{center}\small\nwt
\parbox{150mm}{%\baselineskip=2.5ex
\textbf{Keywords:}\ \ #1}
\end{center}}


\def\KWN#1{
%\begin{center}
%\small
%\parbox{150mm}\end{center}
}

\newcommand{\Avtors}[1]{%\smallskip
%\vspace*{.5pt}
\hangindent=23pt\noindent
%\nwt
{\bfseries#1}\
}


\renewcommand{\thesubsection}{\thesection.\arabic{subsection}\hspace*{-5pt}}
\renewcommand{\thesubsubsection}{\thesubsection\hspace*{5pt}.\arabic{subsubsection}\hspace*{-3pt}}

\newcommand{\Ack}{\section*{\protect\rmfamily Acknowledgments}\noindent}
\newcommand{\Contr}{\section*{\protect\rmfamily Contributors}\noindent}
\newcommand{\Contrl}{\section*{\protect\rmfamily Contributor}\noindent}

\makeindex


\begin{document}
\Rus

\nwt
%\ptb


%\renewcommand{\contentsname}{\protect\Large\bf Содержание}

\setcounter{tocdepth}{2}

%\tableofcontents

\renewcommand{\bibname}{\protect\rmfamily Литература}
  \def\Au#1{{\it #1}}
    \def\Aue#1{{#1}}

%\newcommand{\No}{№}
  \newcommand{\tg}{\,\mathrm{tg}\,}
    \newcommand{\ctg}{\,\mathrm{ctg}\,}
  \newcommand{\arctg}{\,\mathrm{arctg}\,}

\def\forallb{\mathop{\forall}}
\def\cupb{\mathop{\cup}}
\def\existsb{\mathop{\exists}}


\newpage
\addtocounter{razdel}{1}
%\def\razd{РЕГУЛИРУЕМЫЙ ЭЛЕКТРОПРИВОД ДЛЯ ЭЛЕКТРОЭНЕРГЕТИКИ}


\setcounter{page}{2}

%   { %\Large  
   { %\baselineskip=16.6pt
   
   \vspace*{-48pt}
   \begin{center}\LARGE
   \textit{Предисловие}
   \end{center}
   
   %\vspace*{2.5mm}
   
   \vspace*{25mm}
   
   \thispagestyle{empty}
   
   { %\small 

    
Вниманию читателей журнала <<Информатика и её применения>> предлагается 
очередной тематический выпуск <<Вероятностно-статистические методы и 
задачи информатики и информационных технологий>>. Предыдущие тематические 
выпуски журнала по данному направлению вышли в 2008~г.\ (т.~2, вып.~2), 
в 2009~г.\ (т.~3, вып.~3) и в 2010~г.\ (т.~4, вып.~2). 

Статьи, собранные в данном журнале, посвящены разработке новых вероятностно-статистических 
методов, ориентированных на применение к решению конкретных задач информатики и информационных 
технологий, а также~--- в ряде случаев~--- и других прикладных задач. Проблематика, охватываемая 
публикуемыми работами, развивается в рамках научного сотрудничества между Институтом проблем 
информатики Российской академии наук (ИПИ РАН) и Факультетом вычислительной математики и 
кибернетики Московского государственного университета им.\ М.\,В.~Ломоносова в ходе работ 
над совместными научными проектами (в том числе в рамках функционирования 
Научно-образовательного центра <<Вероятностно-статистические методы анализа рисков>>). 
Многие из авторов статей, включенных в данный номер журнала, являются активными участниками 
традиционного международного семинара по проблемам устойчивости стохастических моделей, 
руководимого В.\,М.~Золотаревым и В.\,Ю.~Королевым; регулярные сессии этого семинара 
проводятся под эгидой МГУ и ИПИ РАН (в 2011~г.\ указанный семинар проводится в октябре 
в Калининградской области РФ). 

Наряду с представителями ИПИ РАН и МГУ в число авторов данного выпуска журнала входят 
ученые из Научно-исследовательского института системных исследований РАН, Института 
проблем технологии микроэлектроники и особочистых материалов РАН, Института 
прикладных математических исследований Карельского НЦ РАН, Московского 
авиационного института, Вологодского государственного педагогического университета, 
НИИММ им.\ Н.\,Г.~Чеботарева, Казанского государственного университета, Дебреценского 
университета (Венгрия).

Несколько статей выпуска посвящено разработке и применению стохастических методов и 
информационных технологий для решения различных прикладных задач. В~работе В.\,Г.~Ушакова 
и О.\,В.~Шестакова рассмотрена задача определения вероятностных характеристик случайных 
функций по распределениям интегральных преобразований, возникающих в задачах эмиссионной 
томографии. В~статье Д.\,О.~Яковенко и М.\,А.~Целищева рассмотрены некоторые вопросы 
математической теории риска и предложен новый подход к диверсификации инвестиционных 
портфелей. Работа И.\,А.~Кудрявцевой и А.\,В.~Пантелеева посвящена построению и 
исследованию математической модели, описывающей динамику сильноионизованной плазмы. 
В~статье П.\,П.~Кольцова изучается качество работы ряда алгоритмов сегментации изображений. 
Статья А.\,Н.~Чупрунова и И.~Фазекаша посвящена вероятностному анализу числа без\-оши\-бочных 
блоков при помехоустойчивом кодировании; получены усиленные законы больших чисел для указанных 
величин.

В данном выпуске традиционно присутствует тематика, весьма активно разрабатываемая в течение 
многих лет специалистами ИПИ РАН и МГУ,~--- методы моделирования и управления для 
информационно-телекоммуникационных и вычислительных систем, в частности методы 
теории массового обслуживания. В~статье А.\,И.~Зейфмана с соавторами рассматриваются 
модели обслуживания, описываемые марковскими цепями с непрерывным временем в случае 
наличия катастроф. В~работе М.\,М.~Лери и И.\,А.~Чеплюковой рассматриваются случайные 
графы Интернет-типа, т.\,е.\ графы, степени вершин которых имеют степенные распределения; 
такие задачи находят применение при исследовании глобальных сетей передачи данных. 
Работа Р.\,В.~Разумчика посвящена исследованию систем массового обслуживания специального 
вида~--- с отрицательными заявками и хранением вытесненных заявок.

Ряд статей посвящен развитию перспективных теоретических 
вероятностно-статистических методов, которые находят широкое применение в различных 
задачах информатики и информационных технологий. В~работе В.\,Е.~Бенинга, А.\,К.~Горшенина 
и В.\,Ю.~Королева рассмотрена задача статистической проверки гипотез о числе компонент 
смеси вероятностных распределений, приводится конструкция асимптотически наиболее мощного 
критерия. Результаты этой работы найдут применение в ряде прикладных задач, использующих 
математическую модель смеси вероятностных распределений (в информатике, моделировании 
финансовых рынков, физике турбулентной плазмы и~т.\,д.). В~статье В.\,Ю.~Королева, 
И.\,Г.~Шевцовой и С.\,Я.~Шоргина строится новая, улучшенная оценка точности нормальной 
аппроксимации для пуассоновских случайных сумм; как известно, указанные случайные суммы 
широко используются в качестве моделей многих реальных объектов, в том числе в информатике, 
физике и других прикладных областях. Работа В.\,Г.~Ушакова и Н.\,Г.~Ушакова посвящена 
исследованию ядерной оценки плотности распределения; эти результаты могут применяться, 
в част\-ности, при анализе трафика в телекоммуникационных системах. Серьезные приложения 
в статистике могут получить результаты работы О.\,В.~Шестакова, в которой доказаны оценки 
скорости сходимости распределения выборочного абсолютного медианного отклонения к нормальному 
закону. 

\smallskip

Редакционная коллегия журнала выражает надежду, что данный тематический  выпуск 
будет интересен специалистам в области теории вероятностей и математической статистики 
и их применения к решению задач информатики и информационных технологий.
     
     %\vfill 
     \vspace*{20mm}
     \noindent
     Заместитель главного редактора журнала <<Информатика и её 
применения>>,\\
     директор ИПИ РАН, академик  \hfill
     \textit{И.\,А.~Соколов}\\
     
     \noindent
     Редактор-составитель тематического выпуска,\\
     профессор кафедры математической статистики факультета\\
      вычислительной математики и кибернетики МГУ им.\ М.\,В.~Ломоносова,\\
     ведущий научный сотрудник ИПИ РАН,\\ 
доктор физико-математических наук \hfill
      \textit{В.\,Ю.~Королев}
     
     } }
     }

\include{shvedov}    %1  
\def\stat{bosov+stef}

\def\tit{УПРАВЛЕНИЕ ВЫХОДОМ СТОХАСТИЧЕСКОЙ ДИФФЕРЕНЦИАЛЬНОЙ СИСТЕМЫ 
ПО~КВАДРАТИЧНОМУ КРИТЕРИЮ. I.~ОПТИМАЛЬНОЕ РЕШЕНИЕ МЕТОДОМ 
ДИНАМИЧЕСКОГО ПРОГРАММИРОВАНИЯ$^*$}

\def\titkol{Управление выходом стохастической дифференциальной системы 
по~квадратичному критерию. I}
%.~Оптимальное решение методом 
%динамического программирования}

\def\aut{А.\,В.~Босов$^1$, А.\,И.~Стефанович$^2$}

\def\autkol{А.\,В.~Босов, А.\,И.~Стефанович}

\titel{\tit}{\aut}{\autkol}{\titkol}

\index{Босов А.\,В.}
\index{Стефанович А.\,И.}
\index{Bosov A.\,V.}
\index{Stefanovich A.\,I.}




{\renewcommand{\thefootnote}{\fnsymbol{footnote}} \footnotetext[1]
{Работа выполнена при частичной поддержке РФФИ (проект 16-07-00677).}}


\renewcommand{\thefootnote}{\arabic{footnote}}
\footnotetext[1]{Институт проблем информатики Федерального исследовательского центра <<Информатика 
и~управление>> Российской академии наук, \mbox{AVBosov@ipiran.ru}}
\footnotetext[2]{Институт проблем информатики Федерального исследовательского центра <<Информатика 
и~управление>> Российской академии наук, \mbox{AStefanovich@frccsc.ru}}

%\vspace*{8pt}



  
  \Abst{Решается задача оптимального управления для диффузионного процесса 
Ито и~линейного управ\-ля\-емо\-го выхода. Рассматриваемая постановка близка 
к~классической ли\-ней\-но-квад\-ра\-тич\-ной гауссовской задаче управления 
(linear-quadratic Gaussian (LQG) control). Отличия состоят в~том, что состояние описывается нелинейным 
дифференциальным уравнение Ито $dy_t\hm= A_t(y_t) \,dt\hm+ \Sigma_t(y_t)\,dv_t$ 
и~не зависит от управ\-ле\-ния~$u_t$, оптимизации подлежит управ\-ля\-емый 
линейный выход $dz_t\hm= a_t y_t\,dt\hm+ b_t z_t \,dt\hm+ c_t u_t \,dt\hm+ \sigma_t\, 
dw_t$. Дополнительные обобщения внесены в~квад\-ра\-тич\-ный критерий качества 
с~целью воз\-мож\-ности постановки таких задач, как отслеживание выходом 
состояния или управ\-ле\-ни\-ем~--- линейной комбинации состояния и~выхода. Для 
решения используется метод динамического программирования. Функцию 
Беллмана позволяет найти предположение о~ее структуре вида $V_t(y,z)\hm= 
\alpha_t z^2\hm+ \beta_t(y)z \hm+\gamma_t(y)$. Решение дают три 
дифференциальных уравнения для коэффициентов~$\alpha_t$, $\beta_t(y)$ 
и~$\gamma_t(y)$. Эти уравнения со\-став\-ля\-ют оптимальное решение 
рас\-смат\-ри\-ва\-емой задачи.}
  
  \KW{стохастическое дифференциальное уравнение; оптимальное управ\-ле\-ние; 
динамическое программирование; функция Беллмана; уравнение Риккати; 
линейные уравнения параболического типа}

\DOI{10.14357/19922264180314}
  
%\vspace*{4pt}


\vskip 10pt plus 9pt minus 6pt

\thispagestyle{headings}

\begin{multicols}{2}

\label{st\stat}

\section{Введение}

     Ключевые результаты в~области оптимизации стохастических 
динамических систем, со\-став\-ля\-ющие классическую теорию управления, 
получены более~40~лет назад (такова работа~[1] в~отношении задачи 
управ\-ле\-ния ли\-ней\-но-гаус\-сов\-ски\-ми стохастическими сис\-те\-ма\-ми по 
квад\-ра\-тич\-но\-му критерию). К~классической тео\-рии следует относить 
линейные модели стохастических сис\-тем и~квадратичный критерий качества. 
Это исходный базис, на котором основано множество успешно 
исследованных и~решенных задач стохастического управ\-ле\-ния 
и~оптимизации. 

Дальнейшее развитие~--- это новые модели и~критерии, но 
прежде всего это новые методы: от тео\-рии линейных регуляторов, метода 
динамического программирования и~принципа максимума к~адаптивному 
и~минимаксному подходу, импульсному управ\-ле\-нию и~т.\,д. Множество 
инноваций как в~час\-ти моделей, так и~в~час\-ти математического аппарата, 
имевших мес\-то в~по\-сле\-ду\-ющие годы, существенно обогатили тео\-рию 
управ\-ле\-ния. Но и~до настоящего времени линейные модели и~квадратичный 
критерий, несмотря на всю справедливую критику в~отношении их 
аде\-кват\-ности и~гиб\-кости, сохраняют исследовательский интерес и~находят 
современные области приложения.
     
     Не претендуя на сколь\-ко-ни\-будь полное обосно\-ва\-ние последнего 
тезиса, приведем несколько примеров, показавшихся наиболее ин\-те\-рес\-ными. 

Так, в~[2] решается ли\-ней\-но-квад\-ра\-тич\-ная за\-да\-ча в~игровой 
постановке с~запаздыванием. В~близ\-кой по модели работе~[3] задача 
управ\-ле\-ния ставится в~терминах $H_\infty$-ро\-баст\-ности. Точнее \mbox{называть} 
эту тематику $H_2/H_\infty$-управ\-ле\-ни\-ем, и~работ по этой теме очень 
много. Аккуратности ради следует уточнить, что под линейными 
понимаются модели с~мультипликативными по состоянию воз\-му\-ще\-ниями. 

Совсем другой класс моделей, особо популярных в~по\-след\-ние годы, 
составляют скачкообразные процессы. Например, линейные уравнения 
в~сочетании с~пуассоновскими скачками в~[4] используются в~моделях, 
описывающих различные показатели функционирования сетевых протоколов 
передачи данных транспортного уровня. Телекоммуникации представляют 
в~последние годы самый популярный прикладной материал для 
исследований, работ по этой проб\-ле\-ма\-ти\-ке множество, математические 
техники привлекаются самые разные и~самые современные, но и~линейным 
моделям место находится. Еще один любопытный пример исследования 
скачкообразного процесса и~оптимизации на основе квад\-ра\-тич\-но\-го критерия 
можно найти в~[5] применительно к~задаче инвестирования на финансовом 
рынке. Наконец, упомянем еще работу~[6], подводящую итог исследований 
в~отношении классической детерминированной  
ли\-ней\-но-квад\-ра\-тич\-ной задачи с~использованием техники матричных 
неравенств.
     
     В данной работе также эксплуатируются привлекательные свойства 
линейных моделей и~квад\-ра\-тич\-но\-го критерия, причем в~стохастической 
постановке. На\-прав\-ле\-ни\-ем для обобщения \mbox{выбрана} модель динамики 
сис\-те\-мы: основные усилия на\-прав\-ле\-ны на то, чтобы сделать ее нелинейной. 
Кроме того, пред\-став\-лен\-ная постановка может рас\-смат\-ри\-вать\-ся и~как 
обобщение ранее решенной задачи в~дискретном времени~[7, 8] на время 
непрерывное. В~упомянутых работах помимо собственно модельной 
постановки важна еще и~привлекаемая прикладная об\-ласть~--- 
функционирование сложных программных сис\-тем. Результатов, 
ориентированных непосредственно на такие приложения, к~настоящему 
времени пренебрежимо мало, поэтому~[7, 8]~--- это еще и~прикладное 
обоснование рас\-смат\-ри\-ва\-емой далее задачи.
     
     Оптимизируемая динамическая сис\-те\-ма описывается двумя 
уравнениями. Состояние задается нелинейным стохастическим 
дифференциальным уравнением Ито, не содержащим управ\-ля\-емой 
переменной. Возмущение здесь описывается стандартным винеровским 
процессом, накладываются простые условия существования 
и~един\-ст\-вен\-ности решения. Поскольку состояние не управ\-ля\-ет\-ся, то уместно 
его интерпретировать как слож\-ное внешнее возмущение. Вторая 
переменная~--- управ\-ля\-емый выход~--- задается линейным стохастическим 
дифференциальным уравнением. Цель оптимизации выхода формируется 
квадратичным критерием общего вида. Формальная постановка задачи 
приведена в~сле\-ду\-ющем разделе.
     
     Для решения задачи используется метод динамического 
программирования, решается уравнение Беллмана~[9]. Соответственно, 
в~результате получаются аналитические выражения и~для оптимального 
управ\-ле\-ния, и~для значения функционала качества. Технически 
традиционный, стандартный подход к~задаче обременен, пожалуй, 
единственной проблемой~--- поиском верного пред\-став\-ле\-ния структуры 
функции Беллмана. Справиться с~этой проблемой в~большей степени удается 
за счет результата, полученного при решении дискретного по времени 
аналога рассматриваемой постановки~\cite{8-bos}. Конечные соотношения 
для оптимального решения, как и~во всех подобных задачах, включая 
классическую ли\-ней\-но-квад\-ра\-тич\-ную, содержат решения 
определенных дифференциальных уравнений (обыкновенных и~в~частных 
производных). Вывод этих уравнений и~со\-став\-ля\-ет содержание первой час\-ти 
данной работы. Во второй части будет обсуждаться их приближенное 
чис\-лен\-ное решение и~компьютерные эксперименты.
     
     Кратко обозначим основные положения, при\-вле\-ка\-емые далее 
к~решению задачи, следуя в~основном обозначениям 
и~терминологии~\cite{9-bos}, а~именно: будем рассматривать задачу 
оптимального управления в~стохастической динамической сис\-те\-ме по полной 
информации, применяя метод динамического программирования. В~качестве 
целевого функционала, опре\-де\-ля\-юще\-го качество управ\-ле\-ния $U_0^T\hm= \{ 
u_t,\ 0\leq t\leq T\}$, выступает
     \begin{equation}
     J\left(U_0^T\right)={\sf E}\left\{ \int\limits_0^T L_t \left(x_t, u_t\right)\,dt+ 
l\left(x_T\right)\right\}\,.
     \label{e1-bos}
     \end{equation}
Здесь ${\sf E}\{\cdot\}$~--- оператор математического ожидания; $x_t$~--- 
случайный процесс, описываемый стохастическим дифференциальным 
уравнением Ито
     \begin{equation}
     dx_t=m_t\left( x_t, u_t\right) dt+ \sigma_t\left( x_t\right)dW_t\,,\enskip 
x_0=X\,,
     \label{e2-bos}
     \end{equation}
где $W_t$~--- стандартный винеровский процесс подходящей раз\-мер\-ности; 
$X$~--- случайный вектор.

     $U_0^T$ будем выбирать из класса допустимых неупреждающих (по 
отношению к~$W_t$) управлений~\cite{9-bos}. Соответственно, 
относительно функций сноса и~диффузии~$m_t$ и~$\sigma_t$  
в~(\ref{e2-bos}) будем предполагать выполненными ка\-кие-ли\-бо условия 
существования сильного решения для заданного до\-пус\-ти\-мо\-го управ\-ле\-ния. 
Например, для управ\-ле\-ния с~обратной связью $u_t\hm= u_t(x_t)$ будем 
считать, что $m_t(x,u_t(x))$ и~$\sigma_t(x)$ удовлетворяют условию 
линейного рос\-та и~локальному условию Липшица по~$x$ равномерно 
по~$t$ (т.\,е.\ условиям Ито).
     
     Для поиска оптимального управления, минимизирующего $J(U_0^T)$, 
рас\-смат\-ри\-ва\-ет\-ся функция Беллмана
     \begin{equation}
     V_t(x)=\left.\mathop{\mathrm{inf}}\limits_{U_t^T} {\sf E} \left\{ \int\limits_t^T 
L_t \left( x_t, u_t\right)\,dt+l\left( x_T\right) \right\vert \mathcal{F}_t^x\right\}\,,
     \label{e3-bos}
     \end{equation}
где $\mathcal{F}_t^x$~--- $\sigma$-ал\-геб\-ра, по\-рож\-ден\-ная~$x_\tau$, 
$0\hm\leq \tau\hm\leq t$, ${\sf E}\{\cdot\vert \mathcal{F}\}$~--- оператор условного 
математического ожидания относительно~$\mathcal{F}$. Соответственно, 
в~качестве достаточного условия оп\-ти\-маль\-ности воспользуемся уравнением 
динамического программирования
\begin{multline}
\fr{\partial V_t(x)}{\partial t} +\fr{1}{2}\sum\limits^n_{i,j=1} \sigma^2_{t_{ij}}
\fr{\partial^2 V_t(x)}{\partial x_i \partial x_j}+{}\\
{}+\min\limits_u\left[  
\sum\limits^n_{i=1} m_{t_i} \fr{\partial V_t(x)}{\partial x_i} + L_t(x,u)\right] 
=0\,,\\
V_T(x)=l(x)\,,
\label{e4-bos}
\end{multline}
где $m_{t_i}$~--- $i$-й элемент век\-тор-функ\-ции~$m_t(x,u)$; 
$\sigma^2_{t_{ij}} \hm= \sum\nolimits^m_{k=1} 
\sigma_{t_{ik}}\sigma_{t_{ki}}$, $\sigma_{t_{ij}}$~--- $i$-й по строке, $j$-й 
по столб\-цу элемент мат\-рич\-ной функции~$\sigma_t(x)$; $n$ и~$m$~--- 
размерности~$x_t$ и~$W_t$ соответственно.

     Традиционно в~рамках применения метода динамического 
программирования будем предполагать, что функции~$L_t$, $l$, $m_t$ 
и~$\sigma_t$ обеспечивают существование хотя бы одного решения 
уравнения~(\ref{e4-bos}), а~следовательно, и~оптимального 
управления~$u_t^*$, $0\hm\leq t\hm\leq T$, до\-став\-ля\-юще\-го минимум 
целевому функционалу~(\ref{e1-bos}). Задача оптимизации далее получается 
путем указания конкретных выражений для~$L_t$, $l$, $m_t$ и~$\sigma_t$.

\section{Постановка задачи управления выходом}

     Рассматриваемые далее случайные функции будут предполагаться 
скалярными. Такое упрощение позволит разгрузить выкладки и~итоговые 
выражения от не самых существенных деталей.
     
     Рассмотрим стохастическую дифференциальную сис\-те\-му, со\-сто\-яние 
которой представляет диффузи\-он\-ный процесс~$y_t$, описываемый 
нелинейным стохастическим дифференциальным уравнением Ито
     \begin{equation}
     dy_t=A_t\left( y_t\right) dt +\Sigma_t \left( y_t\right) dv_t\,,\enskip 
y_0=Y\,,
     \label{e5-bos}
     \end{equation}
где $v_t$~--- стандартный (одномерный) винеровский процесс; $Y$~--- 
случайная величина с~конечным вторым моментом; функции~$A_t$ 
и~$\Sigma_t$ удовлетворяют условиям Ито:
\begin{equation*}
\left\vert A_t(y)\right\vert +\left\vert \Sigma_t(y)\right\vert \leq C(1+\vert y\vert )\ 
\mbox{для\ всех } 0\leq t\leq T\,;
\end{equation*}

\vspace*{-12pt}

\noindent
\begin{multline*}
\hspace*{-2.10051pt}\left\vert A_t\left(y_1\right) -A_t \left( y_2\right) \right\vert +\left\vert 
\Sigma_t\left( y_1\right) -\Sigma_t \left(y_2\right)\right\vert \leq
C\left\vert y_1-y_2\right\vert\\
 \mbox{для\ всех\ } 0\leq t\leq T\ \mbox{и } 
y_1,y_2\in \mathbb{R}^1\,,
\end{multline*}
обеспечивающим существование единственного сильного (потраекторного) 
решения уравнения.
     
     Будем считать, что~$y_t$ описывает состояние некоторой 
динамической системы. Соответственно, поведение этой сис\-те\-мы опишем 
выходом, линейно связанным с~со\-сто\-янием:
     \begin{equation}
     dz_t=a_t y_t \,dt+ b_t z_t \,dt+ c_t u_t \,dt+\sigma_t \,dw_t\,,\enskip
     z_0=Z\,.
     \label{e6-bos}
     \end{equation}
Здесь $w_t$~--- не зависящий от~$v_t$, $Y$ и~$Z$ стандартный (одномерный) 
винеровский процесс; $Z$~--- случайная величина с~конечным вторым 
моментом; $u_t$~--- допустимое неупреждающее управ\-ле\-ние, качество 
которого определяется целевым функционалом следующего вида:
\begin{multline}
\!\hspace*{-3.98538pt}J\left( U_0^T\right) ={\sf E}\left\{ \int\limits_0^T \!\left( S_t\left( s_ty_t-g_t z_t -h_t 
u_t\right)^2 +G_t z_t^2+{}\right.\right.\\
\left.\left.{}+ H_t u_t^2
\vphantom{S_t\left( s_ty_t-g_t z_t -h_t 
u_t\right)^2}
\right) dt+S_T\left( s_T y_T -g_T 
z_T\right)^2+G_T z_T^2
\vphantom{\int\limits_0^T}\right\}\,,
\label{e7-bos}
\end{multline}
где $S_t$, $G_t$ и~$H_t$~--- неотрицательные функции\linebreak
$0\hm\leq t\hm\leq T$. 
Такой критерий отражает физический смысл задачи распределения ресурсов 
со\-глас\-но аналогичной~(\ref{e5-bos})--(\ref{e7-bos}) задаче для дис\-крет\-но\-го 
времени, рас\-смот\-рен\-ной в~\cite{7-bos}. В~част\-ности,  
функци\-онал~(\ref{e7-bos}) поз\-во\-ля\-ет ставить задачи отслеживания
 выходом 
со\-сто\-яния сис\-те\-мы, используя сла\-га\-емое $(y_t\hm- z_t)^2$, или 
управлением~--- линейной комбинации со\-сто\-яния и~выхода, сла\-га\-емое типа\linebreak 
$(y_t\hm+ z_t\hm- u_t)^2$. Поскольку задача формулируется 
в~предположении наличия пол\-ной информации о~со\-сто\-янии~$y_t$ 
и~выходе~$z_t$ (соответствующую $\sigma$-ал\-геб\-ру 
обозначим~$\mathcal{F}_t^{y,z}$), то допустимое управ\-ле\-ние ищется 
в~классе~$\mathcal{F}_t^{y,z}$-из\-ме\-ри\-мых неупреждающих функций 
(и,~как будет показано далее, оказывается управ\-ле\-ни\-ем с~обратной связью).

     Функции~$a_t$, $b_t$, $c_t$ и~$\sigma_t$ будем предполагать 
ограниченными: $\vert a_t\vert \hm+ \vert b_t\vert \hm+\vert c_t\vert \hm+ \vert 
\sigma_t \vert \hm\leq C$ для всех $0\hm\leq t\hm\leq T$, процесс  
управления~--- допустимым не\-упреж\-да\-ющим~\cite{9-bos}, обеспечивая, 
таким образом, существование сильного решения урав\-не\-ния~(\ref{e6-bos}) 
для любого допустимого управ\-ления.
     
     Задачу составляет поиск~$u_t^*$~--- допустимого управ\-ле\-ния, 
доставляющего минимум квад\-ра\-тич\-но\-му функционалу~$J(U_0^T)$.
      
     Поставленная задача очевидным образом формулируется в~терминах 
введенных выше в~(\ref{e1-bos})--(\ref{e3-bos}) обозначений, а~именно: 
     требуется обозначить
     \begin{gather*}
      x_t=\begin{pmatrix}
     y_t\\ z_t\end{pmatrix};\quad  m_t(x_t, u_t)=\begin{pmatrix}
     A_t(y_t)\\ a_t y_t +b_t z_t +c_t u_t\end{pmatrix};\\
     \sigma_t(x_t)= \begin{pmatrix}
     \Sigma_t(y_t)& 0\\
     0& \sigma_t\end{pmatrix};\quad W_t=\begin{pmatrix}
     v_t \\ w_t\end{pmatrix}
     %     \label{e8-bos}
     \end{gather*}
для записи уравнения со\-сто\-яния типа~(\ref{e2-bos}) и
\begin{align*}
L_t(x,u)&= L_t(y,z,u) ={}\\
&\hspace*{3mm}{}=S_t\left( s_t y-g_t z -h_t u\right)^2 +G_t z^2 +H_t  u^2\,;\\
l(x)&= l(y,z) =S_T \left( S_T y-g_T z\right)^2 +G_T z^2
%\label{e9-bos}
\end{align*}
для записи целевого функционала в~виде~(\ref{e1-bos}).

     Функция Беллмана~(\ref{e3-bos}) принимает вид 
     $V_t(x)\hm= V_t(y,z)$. Для записи со\-от\-вет\-ст\-ву\-юще\-го~(\ref{e4-bos}) 
уравнения Беллмана для~$V_t(y,z)$ заметим, что
     $$
     \left( \sigma^2_{t_{ij}}\right)_{i,j=1,2}= \begin{pmatrix}
     \Sigma_t^2(y) & 0\\
     0 & \sigma_t^2\end{pmatrix}\,.
     $$
     
     С~учетом перечисленных обозначений урав\-не\-ние динамического 
программирования~(\ref{e4-bos}) принимает вид:
     \begin{multline}
     \fr{\partial V_t(y,z)}{\partial t} +\fr{1}{2}\left( \Sigma_t^2(y) \fr{\partial^2 
V_t(y,z)} {\partial y^2}+\sigma_t^2\fr{\partial^2 V_t(y,z)} {\partial 
z^2}\right)+{}\\
    {}+\min\limits_u\! \left[ A_t(y) \fr{\partial V_t(y,z)}{\partial y}+\left( a_t 
y+b_t z+c_t u\right) \fr{\partial V_t(y,z)}{\partial z} +{}\right.\hspace*{-3pt}\\
\left.{}+ S_t\left( s_t y-g_t z-h_t 
u\right)^2+G_t z^2+H_t u^2
     \vphantom{\fr{\partial V_t(y,z)}{\partial y}}\right] =0\,,\\
     V_T(y,z)=S_T\left( s_T y-g_T z\right)^2+G_T z^2\,.
     \label{e10-bos}
     \end{multline}
     Это и~есть то самое уравнение, которое требуется решить: 
существование решения данного урав\-не\-ния суть достаточное условие 
оптимальности; оптимальное управ\-ле\-ние при этом~--- точ\-ка минимума 
со\-от\-вет\-ст\-ву\-юще\-го сла\-га\-емого.
     
\section{Динамическое программирование и~оптимальное 
управление}

     В рассматриваемой постановке линейность\linebreak выхода и~квадратичность 
критерия дают те же преимущества, что и~в~классической  
ли\-ней\-но-квад\-ра\-тич\-ной задаче управ\-ле\-ния~\cite{1-bos}, а~именно: 
позволяют сразу определить вид оптимального управ\-ле\-ния и~фактические 
условия его существования. Действительно, со\-хра\-няя в~(\ref{e10-bos}) под 
знаком $\min\nolimits_u$ только члены, зависящие от~$u$, получаем
     \begin{multline*}
     \fr{\partial V_t(y,z)}{\partial t} +\fr{1}{2}\left( \Sigma_t^2(y) \fr{\partial^2 
V_t(y,z)} {\partial y^2}+\sigma_t^2\fr{\partial^2 V_t(y,z)} {\partial 
z^2}\right)+{}\\
     {}+A_t(y)\fr{\partial V_t(y,z)}{\partial y}+\left( a_t y+b_t z\right) 
\fr{\partial V_t(y,z)}{\partial z}+{}\\
{}+S_t\left( s_t y-g_t z\right)^2 +G_t z^2+{}
\end{multline*}

\noindent
\begin{multline*}
     {}+\min\limits_u \left[ \left( c_t \fr{\partial V_t(y,z)}{\partial z}-2S_t \left( 
s_t y-g_t z\right) h_t\right)u +{}\right.\\
\left.{}+\left( S_t h_t^2+H_t\right) u^2
\vphantom{\fr{\partial V_t(y,z)}{\partial z}}
\right]=0\,,
     %\label{e11-bos}
     \end{multline*}
откуда в~предположении $S_t h_t^2\hm+ H_t\hm>0$ следует, что существует 
оптимальное управ\-ле\-ние, которое определяется равенством
\begin{multline}
u_t^* = u_t^*(y,z)=-\fr{1}{2}\left( S_t h_t^2 +H_t\right)^{-1} \left( c_t 
\fr{\partial V_t(y,z)}{\partial z}-{}\right.\\
\left.{}-2S_t\left( s_t y-g_t z\right) h_t
\vphantom{\fr{\partial V_t(y,z)}{\partial z}}
\right)
\label{e12-bos}
\end{multline}
и доставляет минимум соответствующему сла\-га\-емо\-му в~урав\-не\-нии Беллмана, 
равный
$-\left( S_t h_t^2\hm+\right.$\linebreak
$\left.{}+H_t\right)^{-1} \left( c_t 
{\partial V_t(y,z)}/{\partial 
z}\hm-2S_t\left( s_t y \hm-g_t z\right) h_t \right)^2/4.
$ 
     
     Отметим, что, как и~в~классической ли\-ней\-но-квад\-ра\-тич\-ной 
задаче, управ\-ле\-ние из класса до\-пус\-ти\-мых не\-упреж\-да\-ющих получилось 
управ\-ле\-ни\-ем с~обратной связью.
     
     Таким образом, функция Беллмана описывается сле\-ду\-ющим 
дифференциальным уравнением:
     \begin{multline}
     \fr{\partial V_t(y,z)}{\partial t} +\fr{1}{2}\left( \Sigma_t^2(y) \fr{\partial^2 
V_t(y,z)} {\partial y^2}+\sigma_t^2\fr{\partial^2 V_t(y,z)} {\partial 
z^2}\right)+{}\\
     {}+ A_t(y) \fr{\partial V_t(y,z)}{\partial y}+\left( a_t y+b_t z\right) 
\fr{\partial V_t(y,z)}{\partial z}+{}\\
{}+ S_t \left( s_t y- g_t z\right)^2 +G_t z^2-
 \fr{1}{4}\left( S_t h_t^2+H_t\right)^{-1}\times{}\\
 {}\times \left( c_t \fr{\partial V_t(y,z)} 
{\partial z}-2S_t\left( s_t y -g_t z\right) h_t \right)^2=0\,.
     \label{e13-bos}
     \end{multline}
     
     Возводя в~квадрат по\-след\-нее сла\-га\-емое в~(\ref{e13-bos}), перепишем 
его в~виде:
     \begin{multline}
     \fr{\partial V_t(y,z)}{\partial t} +\fr{1}{2}\left( \Sigma_t^2(y) \fr{\partial^2 
V_t(y,z)} {\partial y^2}+\sigma_t^2\fr{\partial^2 V_t(y,z)} {\partial 
z^2}\!\right)+{}\\
{}+A_t(y) \fr{\partial V_t(y,z)}{\partial y}
+ \left( 
\vphantom{\left( S_t h_t^2 +H_t\right)^{-1}}
a_t y+b_t z+{}\right.\\
\left.{}+\left( S_t h_t^2 +H_t\right)^{-1}
 c_t S_t \left( s_t y-g_t z\right) h_t
\right) 
     \fr{\partial V_t(y,z)}{\partial z}+{}\\
     {}+\left( S_t-\left( S_t h_t^2 +H_t\right)^{-1} S_t^2 h_t^2\right)\left( s_t y -
g_t z\right)^2+{}\\
     \!\!{}+
     G_t z^2 -\fr{1}{4}\left( S_t h_t^2+H_t\right)^{-1}\! c_t^2
     \left(\fr{\partial V_t(y,z)}{\partial z}\right)^{\!2}=0\,.\!\!
     \label{e14-bos}
     \end{multline}
     
     Рассматривая полученное уравнение, заметим, что его решение может 
быть пред\-став\-ле\-но в~виде:
   \begin{equation}
     V_t(y,z)= \alpha_t z^2+\beta_t(y) z +\gamma_t(y)\,,
     \label{e15-bos}
     \end{equation}
т.\,е.\ будем искать решение при дополнительном предположении 
о~квад\-ра\-тич\-ности функции Белл\-ма\-на по переменной~$z$, и~сведем, таким 
образом, поиск оптимального решения к~уравнениям относительно функций 
$\alpha_t$, $\beta_t(y)$ и~$\gamma_t(y)$. Отметим сразу, что явный вид 
функции~$\gamma_t(y)$ для реализации оптимального управ\-ле\-ния не 
требуется, однако далее будет предложен вариант вы\-чис\-ле\-ния и~этой 
функции, что пред\-став\-ля\-ет\-ся небесполезным, поскольку позволит выполнять 
расчет минимума целевого функционала. Источником для 
предложения~(\ref{e15-bos}) является уже упоминавшаяся аналогичная 
задача для случая дис\-крет\-но\-го времени~\cite{7-bos, 8-bos}. В~той задаче 
выражение для функции Беллмана получается формально без 
дополнительных усилий. При этом форма~(\ref{e15-bos}) обнаруживается 
как свойство оптимального решения. В~рассматриваемом случае 
непрерывного времени~(\ref{e15-bos}) постулируется, а~пра\-виль\-ность 
постулата под\-тверж\-да\-ет\-ся далее ре\-зуль\-ти\-ру\-ющи\-ми уравнениями 
для~$\alpha_t$, $\beta_t(y)$ и~$\gamma_t(y)$ Кроме того, данное 
предположение пред\-став\-ля\-ет\-ся вы\-те\-ка\-ющим из линейной структуры задачи 
в~отношении переменной~$z$, в~част\-ности, тем фактом, что такой вид 
функции Беллмана обеспечивает линейность оптимального 
управ\-ле\-ния~(\ref{e12-bos}) по~$z$.

     Граничное условие при выбранном предположении~(\ref{e15-bos}) 
принимает вид:

\noindent
     \begin{multline*}
     V_T(y,z)= S_T\left( s_T y- g_T z\right)^2+G_T z^2 ={}\\[-0.5pt]
     {}=\alpha_T z^2 
+\beta_T(y) z +\gamma_T(y)\,,
    \end{multline*}
т.\,е.

\noindent
\begin{align*}
\alpha_T&= S_T g_T^2 +G_T\,;\\[-0.5pt]
\beta_T(y)&=-2S_T s_T g_T y\,;\\[-0.5pt]
\gamma_T(y)&=S_T s_T^2 y^2\,.
%\label{e16-bos}
\end{align*}
          При этом само оптимальное управ\-ле\-ние, определенное 
выражением~(\ref{e12-bos}), оказывается управ\-ле\-ни\-ем с~обратной связью 
по~$y_t$ и~$z_t$:

\noindent
     \begin{multline}
     u_t^*=u_t^*(y,z) ={}\\[-0.5pt]
     {}=
     -\fr{1}{2}\left( S_t h_t^2 +H_t\right)^{-1}
     \left( c_t \left( 2\alpha_t z +\beta_t(y)\right) +{}\right.\\[-0.5pt]
    \left. {}+2S_t\left( s_t y-g_t z\right) 
h_t\right)\,.
     \label{e17-bos}
     \end{multline}
          Подставляем $V_t(y,z)\hm= \alpha_t z^2 \hm+ \beta_t(y) 
z\hm+\gamma_t(y)$ в~(\ref{e14-bos}):

\noindent
     \begin{multline*}
     \fr{\partial \alpha_t}{\partial t}\, z^2 +
     \fr{\partial \beta_t(y)}{\partial t}\,z +
     \fr{\partial \gamma_t(y)}{\partial t}+{}\\[-0.5pt]
     {}+\fr{1}{2}\left( \Sigma_t^2(y) \left( 
\fr{\partial^2\beta_t(y)}{\partial y^2}\,z +\fr{\partial^2 \gamma_t(y)}{\partial 
y^2}\right) +2\sigma_t^2\alpha_t\right)+{}\\[-0.5pt]
 {}+A_t(y)\left(\fr{\partial \beta_t(y)}{\partial y}\,z + \fr{\partial 
\gamma_t(y)}{\partial y}\right) +{}\\[-0.5pt]
\hspace*{-0.22987pt}{}+\left( a_t y+b_t z+\left( S_t h_t^2 +H_t\right)^{-1} c_t S_t \left( s_t y-
g_t z\right) h_t\right)\times{}
\end{multline*}

\noindent
\begin{multline*}
         {}\times \left( 2\alpha_t z+\beta_t(y)\right)+{}\\
     {}+\left( S_t-\left( S_t h_t^2 +H_t\right)^{-1} S_t^2 h_t^2\right)\left( s_t y-
g_t z\right)^2+{}\\
     {}+ G_t z^2 -\fr{1}{4}\left( S_t h_t^2 +H_t\right)^{-1} c_t^2 \left( 
2\alpha_t z+\beta_t(y)\right)^2=0\,.
     \end{multline*}
          Далее выделяем слагаемые при~$z^2$, $z$ и~$z^0$
          
          \noindent
     \begin{multline*}
     \fr{\partial \alpha_t}{\partial t}\, z^2 +\fr{\partial \beta_t(y)}{\partial t}\,z +
     \fr{\partial \gamma_t(y)}{\partial 
t}+\fr{1}{2}\,\Sigma_t^2(y)\fr{\partial^2\beta_t(y)}{\partial y^2}\,z+ {}\\
{}+
\fr{1}{2}\,\Sigma_t^2(y)\fr{\partial^2\gamma_t(y)}{\partial 
y^2}+\sigma_t^2\alpha_t+A_t(y)\fr{\partial \beta_t(y)}{\partial y}\,z +{}\\
{}+A_t(y) \fr{\partial 
\gamma_t(y)}{\partial y}+{}\\
{}+ 2\alpha_t \left( b_t -\left( S_t h_t^2+H_t\right)^{-1} c_t 
S_t h_t g_t \right)z^2+{}\\
     {}+
     \left( 2\alpha_t\left( \alpha_t+\left( S_t h_t^2+H_t\right)^{-1} c_t S_t h_t 
s_t\right)y +{}\right.\\
\left.{}+\beta_t(y) \left( b_t-\left( S_t h_t^2+H_t\right)^{-1} c_t S_t h_t 
g_t\right) \right) z+{}\\
     {}+\beta_t(y)\left( a_t +\left( S_t h_t^2+H_t\right)^{-1} c_t S_t h_t s_t\right) 
y+{}\\
{}+ \left( S_t -\left( S_t h_t^2+H_t\right)^{-1} S_t^2 h_t^2\right) g_t^2 z^2-{}\\
     {}- 2\left( S_t -\left( S_t h_t^2+H_t\right)^{-1} S_t^2 h_t^2\right) s_t g_t yz 
+{}\\
{}+
     \left( S_t-\left( S_t h_t^2+H_t\right)^{-1} S_t^2 h_t^2\right) s_t^2 y^2+{}\\
     {}+G_t z^2 -\left( S_t h_t^2 +H_t\right)^{-1} c_t^2 \alpha_t^2 z^2 -{}\\
     {}-\left( 
S_t h_t^2+H_t\right)^{-1} c_t^2 \alpha_t \beta_t(y) z-{}\\
{}-
\fr{1}{4}\left( S_t h_t^2+H_t\right)^{-1}  c_t^2 \beta_t^2(y)=0\,,
     \end{multline*}
группируем их и~получаем сле\-ду\-ющие уравнения:
\begin{itemize}
\item  для~$\alpha_t$:

\noindent
\begin{multline}
\fr{\partial\alpha_t}{\partial t}+2\alpha_t\left( b_t-\left( S_t h_t^2+H_t\right)^{-1} c_t 
S_t h_t g_t\right)+{}\\
{}+ \left( S_t- \left( S_t h_t^2+H_t\right)^{-1} S_t^2 h_t^2\right) 
g_t^2+G_t-{}\\
\hspace*{-8mm}{}-\left( S_t h_t^2+H_t\right)^{-1} c_t^2 \alpha_t^2 =0\,,\enskip \alpha_T=S_T 
g_t^2+G_T\,;\!\!
\label{e18-bos}
\end{multline}
\item для $\beta_t$:

\noindent
\begin{multline}
\fr{\partial\beta_t(y)}{\partial 
t}+\fr{1}{2}\,\Sigma_t^2(y)\fr{\partial^2\beta_t(y)}{\partial y^2} 
+A_t(y)\fr{\partial \beta_t(y)}{\partial y}+{}\\
{}+ 2\alpha_t\left( a_t +\left( S_t h_t^2+H_t\right)^{-1} c_t S_t h_t s_t\right) y+{}\\
{}+
\beta_t(y)\left( b_t -\left( S_t h_t^2 +H_t\right)^{-1} c_t S_t h_t g_t\right)-{}\\
{}-2\left( S_t-\left( S_t h_t^2+H_t\right)^{-1} S_t^2 h_t^2\right) s_t g_t y-{}
\\
{}-
\left( S_t h_t^2+H_t\right)^{-1} c_t^2 \alpha_t \beta_t(y)=0\,,\\
\beta_T(y)=-2S_T s_T g_T y\,;
\label{e19-bos}
\end{multline}
\item  для $\gamma_t$:
\begin{multline}
\hspace*{-0.8pt}\fr{\partial \gamma_t(y)}{\partial t}+\fr{1}{2}\,\Sigma_t^2(y)
\fr{\partial^2 \gamma_t(y)}{\partial y^2} +\sigma_t^2 \alpha_t +A_t(y)
\fr{\partial \gamma_t(y)}{\partial y}+{}\\
{}+ \beta_t(y)\left( a_t +\left( S_t h_t^2+H_t\right)^{-1} c_t S_t h_t s_t\right) y+{}\\
{}+
\left( S_t-\left( S_t h_t^2+H_t\right)^{-1} S_t^2 h_t^2\right)  s_t^2 y^2-{}\\
{}-\fr{1}{4}\left( S_t h_t^2+H_t\right)^{-1} c_t^2 \beta_t^2(y) =0\,,\\
\gamma_T(y)=S_T s_T^2 y^2\,.
\label{e20-bos}
\end{multline}
\end{itemize}
     
     Уравнение~(\ref{e18-bos}), легко заметить, является уравнением 
Риккати, которое в~силу сформулированного выше условия   
имеет единственное неотрицательное решение для всех $0\hm\leq t\hm\leq T$. 
Этот факт требует дополнительного комментария. Для получения 
уравнения~(\ref{e18-bos}) рас\-смот\-рим исходную задачу при дополнительных 
условиях $a_t\hm=0$ и~$s_t\hm=0$ для всех $0\hm\leq t\hm\leq T$. Нетрудно 
видеть, что эти условия рассматриваемую по\-ста\-нов\-ку сводят фактически 
к~классической ли\-ней\-но-квад\-ра\-тич\-ной задаче. Имеющуюся 
в~рассматриваемой формулировке чуть более общую форму целевой 
функции (принципиального значения это обобщение, конечно, не имеет) 
сведем к~классической еще одним предположением: $S_t\hm=0$ для всех 
$0\hm\leq t\hm\leq T$. Теперь уравнение для~$\alpha_t$ принимает хорошо 
известный вид:
     \begin{equation}
     \fr{\partial \alpha_t}{\partial t}+2\alpha_t b_t +G_t- H_t^{-1} c_t^2 
\alpha_t^2=0\,,\enskip \alpha_T=G_T\,.
     \label{e21-bos}
     \end{equation}

     В таком случае, как известно~\cite{10-bos}, существует единственное 
оптимальное управление~--- линейное с~обратной связью по выходу~$z_t$, 
с~коэффициентом усиления, опи\-сы\-ва\-емым уравнением  
Риккати~(\ref{e21-bos}). Именно этот результат дают  
уравнения~(\ref{e18-bos})--(\ref{e20-bos}) и~описываемая ими функция 
Беллмана~(\ref{e15-bos}), так как из $a_t\hm=0$ и~$s_t\hm=0$ немедленно 
следует, что $\beta_t(y)\hm=0$, откуда, в~свою очередь, с~учетом 
не\-за\-ви\-си\-мости решения от~$y_t$ следует, что $\gamma_t(y)\hm=\gamma_t$, 
т.\,е.\ не зависит от~$y$ и~задается уравнением: 
     $$
     \fr{\partial \gamma_t(y)}{\partial t} +\sigma^2_t \alpha_t=0\,,\enskip 
\gamma_T=0\,.
     $$ 
     Оптимальное управ\-ле\-ние при этом 
     $$
     u_t^*= -H_t^{-1} c_t \alpha_t z_t\,,
     $$
      т.\,е.\ все полностью совпадает с~известным классическим решением.
     
     С уравнениями~(\ref{e19-bos}) и~(\ref{e20-bos}) ситуация, естественно, 
обстоит сложнее. Это линейные уравнения второго порядка параболического 
типа, поскольку\linebreak
 $\Sigma_t^2(y)\hm>0$. Фактически отсутствуют 
конструктивные условия, гарантирующие существование их\linebreak
 решений 
(требовать, чтобы все фигурирующие в~уравнениях коэффициенты были 
представлены аналитическими функциями на всем пространстве значений, 
вряд ли целесообразно), поэтому далее будем предполагать, что данные 
уравнения имеют на рас\-смат\-ри\-ва\-емом интервале $0\hm\leq t\hm\leq T$ хотя 
бы одно ограниченное решение и~именно эти условия будем рас\-смат\-ри\-вать 
как достаточные условия существования оптимального решения 
рассматриваемой задачи.
     
     Таким образом, доказана следующая тео\-рема.
     
     \smallskip
     
     \noindent
     \textbf{Теорема.}\ \textit{Пусть для диффузионного 
процесса}~(\ref{e5-bos}) \textit{выполнены условия Ито, для 
     процесса}~(\ref{e6-bos})~--- \textit{ограничены коэффициенты, 
уравнения}~(\ref{e18-bos})--(\ref{e20-bos}) \textit{имеют ограниченные 
решения для $0\hm\leq t\hm\leq T$. Тогда минимум  
функционалу}~(\ref{e7-bos}) \textit{доставляет оптимальное 
управ\-ле\-ние}~(\ref{e17-bos}), \textit{где} $y\hm= y_t$; $z\hm=z_t$.
     
\section{Заключение}

     Рассмотренная задача оптимизации в~целом близка и~по модели, и~по 
критерию к~классической ли\-ней\-но-квад\-ра\-тич\-ной постановке. 
Принципиальным отличием является нелинейная модель для описания 
со\-сто\-яния динамической сис\-те\-мы, в~которой отсутствует управ\-ля\-ющее 
воздействие.\linebreak
 Такую модель наряду с~традиционной интер\-пре\-тацией  
<<со\-сто\-яние--вы\-ход>> мож\-но понимать как\linebreak модель неконтролируемого 
слож\-но\-го внешнего воздействия. Небольшое дополнительное отличие дает 
предложенная форма квад\-ра\-тич\-но\-го критерия, поз\-во\-ля\-ющая, в~част\-ности, 
ставить такие задачи, как отслеживание выходом или управ\-ле\-ни\-ем со\-сто\-яния 
сис\-те\-мы или ее выхода.
     
     Поскольку обсуждать возможности точного решения уравнений, 
определяющих оптимальное управ\-ле\-ние, не имеет смыс\-ла, наиболее 
актуальной далее является задача их приближенного чис\-лен\-но\-го решения 
и~анализа воз\-мож\-ности практической реализации. Этому посвящена вторая 
часть данной работы, пла\-ни\-ру\-емая к~выходу в~ближайшее время.

{\small\frenchspacing
 {%\baselineskip=10.8pt
 \addcontentsline{toc}{section}{References}
 \begin{thebibliography}{99}
\bibitem{1-bos}
\Au{Athans M.} Editorial on the LQG problem~// IEEE~T. Automat. Contr., 1971. Vol.~16. 
No.\,6. P.~528--552. doi: 10.1109/TAC.1971.1099845.
\bibitem{2-bos}
\Au{Wu Z.} Forward-backward stochastic differential equations, linear quadratic stochastic 
optimal control and nonzero sum differential games~// J.~Syst. Sci. Complex., 2005. Vol.~18. 
No.\,2. P.~179--192.
\bibitem{3-bos}
\Au{Chen B.\,S., Zhang~W.} Stochastic H2/H1 control with state-dependent noise~// IEEE 
T.~Automat. Contr., 2004. Vol.~49. No.\,1. P.~45--56. doi: 10.1109/TAC.2003.821400.
\bibitem{4-bos}
\Au{Bohacek S.} A~stochastic model of TCP and fair video transmission~// IEEE 
INFOCOM, 2003. Vol.~2. P.~1134--1144. doi: 10.1109/INFCOM.2003.1208950.
\bibitem{5-bos}
\Au{Домбровский В.\,В., Объедко~Т.\,Ю.} Управление с~прогнозированием системами 
с~марковскими скачками при ограничениях и~применение к~оптимизации 
инвестиционного портфеля~// Автомат. телемех., 2011. №\,5. С.~96--112. doi: 
10.1134/S0005117911050079.
\bibitem{6-bos}
\Au{Баландин Д.\,В., Коган~М.\,М.} Оптимальное линейно-квад\-ра\-тич\-ное управление: от 
матричных уравнений к~линейным матричным неравенствам~// Автомат. телемех., 2011. 
№\,11. С.~60--69. doi: 10.1134/ S0005117911110038.
\bibitem{7-bos}
\Au{Босов А.\,В.} Обобщенная задача распределения ресурсов программной системы~// 
Информатика и~её применения, 2014. Т.~8. Вып.~2. С.~39--47. doi: 
10.14357/19922264140204.
\bibitem{8-bos}
\Au{Босов А.\,В.} Управление линейным выходом дискретной стохастической системы по 
квадратичному критерию~// Изв. РАН. Теория и~системы управления, 2016. №\,3.  
С.~19--35. doi: 10.1134/S1064230716030060.
\bibitem{9-bos}
\Au{Флеминг У., Ришел~Р.} Оптимальное управление детерминированными 
и~стохастическими системами~/ Пер. с~англ.~--- М.: Мир, 1978. 316~с. 
(\Au{Fleming~W.\,H., Rishel~R.\,W.} Deterministic and stochastic optimal control.~--- New 
York, NY, USA: Springer-Verlag, 1975. 222~p.)
\bibitem{10-bos}
\Au{Девис М.\,Х.\,А.} Линейное оценивание и~стохастическое управление~/ Пер. с~англ.~--- 
М.: Наука, 1984. 206~с. (\Au{Davis~M.\,H.\,A.} Linear estimation and stochastic control.~--- 
London: Chapman and Hall, 1977. 224~p.)

 \end{thebibliography}

 }
 }

\end{multicols}

\vspace*{-6pt}

\hfill{\small\textit{Поступила в~редакцию 30.03.18}}

\vspace*{4pt}

%\newpage

%\vspace*{-24pt}

\hrule

\vspace*{2pt}

\hrule

\vspace*{-2pt}


\def\tit{STOCHASTIC DIFFERENTIAL SYSTEM OUTPUT CONTROL 
BY~THE~QUADRATIC CRITERION.~I.~DYNAMIC\\ PROGRAMMING 
OPTIMAL SOLUTION}


\def\titkol{Stochastic differential system output control 
by~the~quadratic criterion. I.~Dynamic programming 
optimal solution}

\def\aut{A.\,V.~Bosov and~A.\,I.~Stefanovich}

\def\autkol{A.\,V.~Bosov and~A.\,I.~Stefanovich}

\titel{\tit}{\aut}{\autkol}{\titkol}

\vspace*{-11pt}


\noindent
Institute of Informatics Problems, Federal Research Center ``Computer Science 
and Control'' of the Russian Academy of Sciences, 44-2~Vavilov Str., Moscow 
119333, Russian Federation


\def\leftfootline{\small{\textbf{\thepage}
\hfill INFORMATIKA I EE PRIMENENIYA~--- INFORMATICS AND
APPLICATIONS\ \ \ 2018\ \ \ volume~12\ \ \ issue\ 3}
}%
 \def\rightfootline{\small{INFORMATIKA I EE PRIMENENIYA~---
INFORMATICS AND APPLICATIONS\ \ \ 2018\ \ \ volume~12\ \ \ issue\ 3
\hfill \textbf{\thepage}}}

\vspace*{3pt}



\Abste{The problem of optimal control for the Ito diffusion 
process and a~controlled linear output is solved. The considered 
statement is close to the classical linear-quadratic Gaussian 
control  (LQG control) problem. Differences consist in the fact 
that the state is described by the nonlinear differential Ito equation  $dy_y = A_t(y_t) 
\,dt+\Sigma_t(y_t)\,dv_t$ and does not depend on the control~$u_t$, 
optimization subject is controlled linear output 
 $dz_t=a_ty_t\,dt +b_tz_t\,dt +c_t u_t\,dt +\sigma_t \,dw_t$. 
Additional generalizations are included in the quadratic 
quality criterion for the purpose of statement such problems 
as state tracking by output or a linear combination of state 
and output tracking by control. The method of dynamic programming 
is used for the solution. 
The assumption about Bellman function in the form  $V_t(y,z)= \alpha_t 
z^2+\beta_t(y) z+\gamma_t(y)$ allows one to find it. 
Three differential equations for the coefficients $\alpha_t$,  $\beta_t(y)$,
and $\gamma_t(y)$ give the solution. 
These equations constitute the optimal solution of the problem under consideration.}

\KWE{stochastic differential equation; optimal control; dynamic programming; 
Bellman function; Riccati equation; linear differential equations of parabolic type}


\DOI{10.14357/19922264180314}

\vspace*{-12pt}

\Ack
\noindent
This work was partially supported by the Russian Science Foundation (grant  
16-07-00677).



%\vspace*{6pt}

  \begin{multicols}{2}

\renewcommand{\bibname}{\protect\rmfamily References}
%\renewcommand{\bibname}{\large\protect\rm References}

{\small\frenchspacing
 {%\baselineskip=10.8pt
 \addcontentsline{toc}{section}{References}
 \begin{thebibliography}{99}
\bibitem{1-bos-1}
\Aue{Athans, M.} 1971. Editorial on the LQG problem. \textit{IEEE~T. 
Automat. Contr.} 16(6):528--552. doi: 10.1109/ TAC.1971.1099845.
\bibitem{2-bos-1}
\Aue{Wu, Z.} 2005. Forward-backward stochastic differential equations, linear 
quadratic stochastic optimal control and\linebreak\vspace*{-12pt}

\columnbreak

\noindent
 nonzero sum differential games. 
\textit{J.~Syst. Sci. Complex.} 18(2):179--192.
\bibitem{3-bos-1}
\Aue{Chen, B.\,S. and W.~Zhang.} 2004. Stochastic H2/H1 control with  
state-dependent noise. \textit{IEEE~T. Automat. Contr.} 49(1):45--56.
doi: 10.1109/TAC.2003.821400.
\bibitem{4-bos-1}
\Aue{Bohacek, S.} 2003. A~stochastic model of TCP and fair video 
transmission. \textit{IEEE INFOCOM}. 2:1134--1144.
doi: 10.1109/INFCOM.2003.1208950.
\bibitem{5-bos-1}
\Aue{Dombrovskii, V.\,V., and T.\,Yu.~Ob''edko.} 2011. Predictive control of 
systems with Markovian jumps under constraints and its application to the 
investment portfolio optimization. \textit{Automat. Rem. Contr.}  
72(5):989--1003.
\bibitem{6-bos-1}
\Aue{Balandin, D.\,V., and M.\,M.~Kogan.} 2011. Optimal linear-quadratic 
control: From matrix equations to linear matrix inequalities. \textit{Automat. 
Rem. Contr.} 72(11):2276--2284.
\bibitem{7-bos-1}
\Aue{Bosov, A.\,V.} 2014. Obobshchennaya zadacha raspredeleniya resursov 
programmnoy sistemy [The generalized problem of software system resources 
distribution]. \textit{Informatika i~ee Primeneniya~--- Inform. Appl.}  
8(2):39--47. doi: 
10.14357/19922264140204.
\bibitem{8-bos-1}
\Aue{Bosov, A.\,V.} 2016. Discrete stochastic system linear output control 
with respect to a quadratic criterion. \textit{J.~Comput. Syst. Sc. 
Int.} 55(3):349--364.
\bibitem{9-bos-1}
\Aue{Fleming, W.\,H., and R.\,W.~Rishel.} 1975. \textit{Deterministic and 
stochastic optimal control.} New York, NY: Springer-Verlag. 222~p.
\bibitem{10-bos-1}
\Aue{Davis, M.\,H.\,A.} 1977. \textit{Linear estimation and stochastic 
control.} London: Chapman and Hall. 224~p.
\end{thebibliography}

 }
 }

\end{multicols}

\vspace*{-6pt}

\hfill{\small\textit{Received March 30, 2018}}

%\pagebreak

%\vspace*{-18pt}
     
     \Contr
     
       \noindent
       \textbf{Bosov Alexey V.} (b.\ 1969)~--- Doctor of Science in technology, 
principal scientist, Institute of Informatics Problems, Federal Research 
Center ``Computer Science and Control'' of the Russian Academy of Sciences, 
44-2~Vavilov Str., Moscow 119333, Russian Federation; 
\mbox{AVBosov@ipiran.ru}
       
       \vspace*{3pt}
       
       \noindent
       \textbf{Stefanovich Alexey I.} (b.\ 1983)~--- principal specialist, 
Institute of Informatics Problems, Federal Research Center ``Computer Science 
and Control'' of the Russian Academy of Sciences, 44-2~Vavilov Str., Moscow 
119333, Russian Federation; \mbox{AStefanovich@frccsc.ru}
\label{end\stat}

\renewcommand{\bibname}{\protect\rm Литература}       

            %2 

\def\stat{shnurkov}

\def\tit{АНАЛИТИЧЕСКОЕ РЕШЕНИЕ ЗАДАЧИ ОПТИМАЛЬНОГО УПРАВЛЕНИЯ ПОЛУМАРКОВСКИМ ПРОЦЕССОМ\\ 
С~КОНЕЧНЫМ МНОЖЕСТВОМ СОСТОЯНИЙ$^*$}

\def\titkol{Аналитическое решение задачи оптимального управления полумарковским 
процессом} %с~конечным множеством состояний}

\def\aut{П.\,В.~Шнурков$^1$, А.\,К.~Горшенин$^2$, В.\,В.~Белоусов$^3$}

\def\autkol{П.\,В.~Шнурков, А.\,К.~Горшенин, В.\,В.~Белоусов}

\titel{\tit}{\aut}{\autkol}{\titkol}

\index{Шнурков П.\,В.}
\index{Горшенин А.\,К.}
\index{Белоусов В.\,В.}
\index{Shnurkov P.\,V.}
\index{Gorshenin A.\,K.}
\index{Belousov V.\,V.}


{\renewcommand{\thefootnote}{\fnsymbol{footnote}} \footnotetext[1]
{Работа выполнена при частичной поддержке РФФИ (проект 15-07-05316).}}


\renewcommand{\thefootnote}{\arabic{footnote}}
\footnotetext[1]{Национальный исследовательский университет <<Высшая школа экономики>>, 
\mbox{pshnurkov@hse.ru}}
\footnotetext[2]{Институт проблем информатики Федерального исследовательского центра <<Информатика 
и~управ\-ле\-ние>> Российской академии наук, \mbox{agorshenin@frccsc.ru}}
\footnotetext[3]{Институт проблем информатики Федерального исследовательского центра <<Информатика 
и~управление>> Российской академии наук, \mbox{vbelousov@ipiran.ru}}

%\vspace*{-6pt}

\Abst{Настоящее исследование посвящено теоретическому обоснованию нового метода 
нахождения оптимальной стратегии управления полумарковским процессом с~конечным 
множеством состояний. Рассматриваются марковские рандомизированные стратегии 
управления, определяемые конечным набором вероятностных мер, соответствующих 
каждому состоянию. Характеристикой качества управления служит стационарный 
стоимостной показатель. Данный показатель представляет собой дроб\-но-ли\-ней\-ный 
интегральный функционал от набора вероятностных мер, задающих стратегию управления. 
Для этого функционала известны явные аналитические представления подынтегральных 
функций числителя и~знаменателя. Дальнейшие результаты основываются на новой 
усиленной и~обобщенной форме теоремы об экстремуме дроб\-но-ли\-ней\-но\-го интегрального 
функционала. Доказывается, что проблемы существования оптимальной стратегии управления 
полумарковским процессом и~ее нахождения сводятся к~задаче численного исследования 
на глобальный экстремум заданной функции от конечного числа вещественных переменных.}

\KW{оптимальное управление полумарковским процессом; стационарный стоимостной 
показатель качества управления; дроб\-но-ли\-ней\-ный интегральный функционал}

\DOI{10.14357/19922264160408} 

\vspace*{9pt}


\vskip 10pt plus 9pt minus 6pt

\thispagestyle{headings}

\begin{multicols}{2}

\label{st\stat}

\section{Введение}

Теория оптимального управления марковскими и~полумарковскими случайными 
процессами интенсивно развивается с~начала 1960-х~гг. Еще в~первых 
основополагающих исследованиях рассматривались не только проблемы существования 
оптимальных стратегий управления, но и~способы нахождения этих стратегий. 

Для решения таких проблем, имеющих алгоритмическое содержание, использовались 
открытые незадолго до этого мощные методы прикладной математики: линейное 
программирование и~динамическое программирование. Отметим, прежде всего, 
классическую работу Р.~Ховарда~\cite{1}, в~которой метод динамического 
программирования был применен для решения проблемы оптимального управления 
марковским процессом с~непрерывным временем. В~дальнейшем В.\,В.~Рыков~\cite{2} 
доказал, что для аналогичной модели управления марковским процессом с~учетом 
переоценки оптимальной стратегией также является стационарная.

Важную роль в~развитии теории управления случайными процессами сыграла работа 
В.~Джевелла~\cite{3}, в~которой были впервые рассмотрены полумарковские модели 
управления для вариантов с~переоценкой и~без переоценки. Данная работа была 
переведена на русский язык и~послужила основой для многих последующих работ 
отечественных и~зарубежных специалистов. В~частности, Б.~Фокс показал~\cite{4}, 
что оптимальной стратегией управления полумарковским процессом в~варианте без 
переоценки является стационарная; аналогичные результаты были получены Э.~Денардо 
и~для варианта с~переоценкой~\cite{5}.

Среди последующих исследований алгоритмической направленности отметим работы 
Р.~Ховарда~\cite{6}, Б.~Фокса~\cite{4}, а также С.~Осаки и~Х.~Майна~\cite{7}. 
В~этих работах для нахождения оптимальных стратегий управления полумарковскими 
процессами использовался метод линейного программирования.

В 1970~г.\ была опубликована фундаментальная монография Х.~Майна и~С.~Осаки~\cite{8}, 
переведенная на русский язык в~1977~г., в~которой были систе\-ма\-ти\-зи\-ро\-ва\-ны и~изложены 
основные результаты по теории оптимального управления марковскими и~полумарковскими 
случайными процессами. Фактически данная книга стала итогом исследований по проблемам 
стохастического управления\linebreak
 за~10~лет. Отметим, что в~этой монографии рас\-смат\-ри\-ва\-лись 
марковские и~полумарковские модели управления с~конечными множествами состояний 
и~допустимых решений, принимаемых \mbox{в~каждом} состоянии. Были получены принципиальные 
тео\-ре\-ти\-че\-ские результаты, заключающиеся в~том, что оптимальные стратегии управ\-ле\-ния 
для основных видов рас\-смат\-ри\-ва\-емых моделей с~переоценкой и~без переоценки являются 
детерминированными и~стационарными. Были разработаны и~обоснованы процедуры нахождения 
оптимальных стратегий управления. В~частности, для модели управления полумарковским 
процессом без переоценки, когда множество со\-сто\-яний образует один эргодический класс, 
а~показатель качества управления пред\-став\-ля\-ет собой стационарный средний удельный 
доход (см.~[8, гл.~5, п.~5.5]), процедура поиска оптимальной рандомизированной 
стратегии осуществлялась методом линейного программирования. Обратим особое внимание 
на данный результат, поскольку аналогичная модель управления полумарковским 
процессом будет рассмотрена в~настоящей работе.

Принципиальную роль в~развитии теории стохастического управления сыграла 
монография И.\,И.~Гихмана и~А.\,В.~Скорохода~\cite{9}. В~этой книге были впервые 
систематически изложены основы теории оптимального управления случайными процессами 
с~дискретным и~непрерывным временем, включая теорию управления процессами, которые 
описываются стохастическими дифференциальными уравнениями. Отдельно были рас\-смот\-ре\-ны 
проблемы управления марковскими процессами с~дискретным временем и~скачкообразными 
марковскими процессами с~непрерывным временем. Роли множеств состояний и~допустимых 
управ\-ле\-ний играли пространства весьма общей структуры. Для широких классов функционалов 
качества управ\-ле\-ния (так называемых эволюционных функционалов в~марковских моделях 
с~дискретным временем и~интегральных функционалов накопления в~марковских моделях 
с~непрерывным временем) были доказаны теоремы о~существовании и~формах пред\-став\-ле\-ния 
оптимальных стратегий управ\-ле\-ния. Было установлено, что для однородных марковских 
моделей оптимальные стратегии управ\-ле\-ния существуют, являются стационарными 
и~детерминированными. Иначе говоря, такие стратегии задаются детерминированными 
функциями, аргументом которых является со\-сто\-яние сис\-те\-мы в~момент принятия решения, 
и~не зависящими от самого момента принятия решения. Что же касается важного вопроса 
о~формах представления этих функций, то их можно охарактеризовать следующим образом. 
Были найдены функциональные уравнения, осложненные условием экстремума, которым 
удовле\-тво\-ря\-ют упомянутые функции. По существу эти соотношения пред\-став\-ля\-ют собой 
уравнения Беллмана для соответствующих динамических стохастических моделей.

Особо отметим, что в~монографии~\cite{9} не рас\-смат\-ри\-ва\-лись проблемы управления 
полумарковскими процессами. Однако дальнейшее развитие общей теории управления 
такими процессами шло по пути, идейно намеченному в~указанной книге.

В последующие годы развитие теории управ\-ле\-ния полумарковскими процессами 
осуществля-\linebreak лось по направлению усложнения моделей и~обобщения исходных предположений. 
Например,\linebreak в~работах~\cite{10, 11} рассмотрены управляемые по\-лумарковские процессы при 
весьма общих предположениях относительно характера пространств состояний и~управлений. 
Проблемы управления исследовались по отношению к~различным видам целевых показателей, 
обобщающих упомянутый выше стационарный показатель средней удельной прибыли. В~этих 
работах доказывается, что оптимальная стратегия управления по отношению к~каж\-до\-му из 
показателей существует и~является одной и~той же стационарной детерминированной 
стратегией, определяемой некоторой функцией, заданной на множестве со\-сто\-яний процесса. 
Об этой функции известно лишь то, что она удовлетворяет некоторому интегральному 
уравнению, которое по содержанию пред\-став\-ля\-ет собой уравнение Бел\-лма\-на для 
соответствующей задачи управ\-ления.

Среди исследований, предшествовавших настоящему, отметим работу 
В.\,А.~Каштанова~[12, гл. 13]. В этом разделе коллективной монографии~\cite{12} 
автором была рассмотрена проблема оптимального управления полумарковским 
процессом с~конечным множеством состояний и~множеством возможных решений, 
которое представляет собой произвольный интервал множества вещественных чисел. 
Модель относится к~виду моделей без переоценки, показателем качества управления 
служит стационарное значение среднего удельного дохода, определяемое аналогично 
классическим работам~\cite{3, 8}. Рандомизированное управление в~каждом состоянии 
определяется в~соответствии с~вероятностным распределением, совокупность которых 
задает\linebreak
 стратегию управления. В.\,А.~Каш\-та\-но\-вым было\linebreak сформулировано утверждение о том, 
что стацио\-нарное значение среднего удельного дохода представляет собой 
дроб\-но-ли\-ней\-ный интегральный функционал от набора вероятностных распределений, 
образующих стратегию управления. При этом\linebreak ранее~[12, гл.~10; 13] было уста\-нов\-ле\-но, 
что дроб\-но-ли\-ней\-ный функционал достигает экстремума на вырожденных распределениях. 
Отсюда естест-\linebreak венно следует, что оптимальная стратегия управ\-ле-ния является 
детерминированной и~должна\linebreak определяться точкой экстремума функции, представляющей 
собой отношение подынтегральных функций чис\-ли\-те\-ля и~знаменателя данного 
дроб\-но-ли\-ней\-но\-го функционала. Однако в~\cite{12} не были получены явные 
представления для указан-\linebreak ных функций. Кроме того, приведенный в~гл.~10 
монографии~\cite{12} вариант теоремы об экстремуме дроб\-но-ли\-ней\-но\-го 
интегрального функционала требовал проверки выполнения условия существования 
этого экстремума. Такие условия указаны не были. В~связи с~этими обстоятельствами 
использовать полученные в~\cite{12} результаты для доказательства существования 
оптимальной детерминированной стратегии управ\-ле\-ния полумарковским процессом и~для 
строгого обоснования способа нахождения такой стратегии оказалось невозможным.

Настоящее исследование посвящено теоретическому обоснованию нового метода 
нахождения\linebreak оптимальной стратегии управления полумарковским процессом с~конечным 
множеством со\-сто\-яний. Рассматриваются марковские рандомизи\-рованные стратегии 
управления, определяемые конеч\-ным набором вероятностных мер, соответствующих 
каждому состоянию. Показателем качества управления служит уже упоминавшийся 
классический  показатель~--- стационарное значение средней удельной прибыли. 
Доказано, что этот показатель представляет собой дроб\-но-ли\-ней\-ный интегральный 
функционал от набора вероятностных мер, задающих стратегию управления. При этом, 
в~отличие от~\cite{12}, получены явные аналитические представления для подынтегральных 
функций числителя и~знаменателя этого дроб\-но-ли\-ней\-но\-го\linebreak
 функционала. Дальнейшие 
результаты основываются на новой усиленной и~обобщенной форме\linebreak
 теоремы об экстремуме 
дроб\-но-ли\-ней\-но\-го интегрального функционала, впервые опубликованной 
в~работе П.\,В.~Шнуркова~\cite{14}. Согласно\linebreak
 утверж\-де\-нию этой теоремы, если 
существует глобальный экстремум так называемой основной функции дроб\-но-ли\-ней\-но\-го 
функционала, которая пред\-став\-ля\-ет собой отношение подынтегральных функций чис\-ли\-те\-ля 
и~знаменателя, то существует безусловный экстремум самого дроб\-но-ли\-ней\-но\-го 
функционала, который достигается на наборе вырожденных вероятностных распределений, 
сосредоточенных в~точке глобального экстремума. В~этом случае оптимальная стратегия 
управ\-ле\-ния существует, является стационарной и~детерминированной и~определяется точкой, 
в~которой основная\linebreak функция достигает глобального экстремума. Таким\linebreak образом, проблемы 
существования оптимальной стратегии управ\-ле\-ния полумарковским процессом и~ее 
нахождения сводятся к~задаче чис\-лен\-но\-го исследования на глобальный экстремум 
заданной функции от конечного чис\-ла вещественных переменных.

\section{Общее описание модели управления полумарковским случайным процессом}

Построим модель управления полумарковским случайным процессом, следуя общему 
подходу, принятому в~классических работах~\cite{3, 8}. Пусть $\xi(t)$~--- 
случайный полумарковский процесс с~конечным множеством состояний
$X\hm=\{1,2,\ldots, N\}$, $N\hm< \infty$. Обозначим через~$t_n$, $n=0,1,2,\ldots$, 
$t_0\hm=0$, случайные моменты изменения состояний данного процесса, 
$\theta_n\hm=t_{n+1}-t_n$, $n\hm=0,1,2,\ldots$, $\xi_n\hm=\xi(t_n)\hm=\xi(t_n+0)$, 
$n\hm=0,1,2,\ldots$ (предполагается, что траектории процесса~$\xi(t)$ 
непрерывны справа). Случайная последовательность~$\{\xi_n\}$
образует цепь Маркова, вложенную в~полумарковский процесс~$\xi(t)$.
Зададим набор измеримых пространств\linebreak $(U_1, \mathscr{B}_1), 
(U_2, \mathscr{B}_2), \ldots, (U_N, \mathscr{B}_N)$, где $U_i$~--- 
множество возможных допустимых управ\-ле\-ний, $\mathscr{B}_i$~--- $\sigma$-ал\-геб\-ра 
подмножеств множества~$U_i$, вклю\-ча\-ющая в~себя все одноточечные подмножества\linebreak  
множества~$U_i$, т.\,е.\ если $u_i \hm\in U_i$, то $\{u_i\} \hm\in \mathscr{B}_i$, 
$i\hm=1,2,\ldots, N$.
Пусть $\Gamma_i$~--- некоторое множество всевозможных вероятностных мер $\Psi_i 
\hm \in \Gamma_i$, заданных на $\sigma$-ал\-геб\-ре~$\mathscr{B}_i$, $i\hm=1,2,\ldots,N$.

Поскольку идейное содержание и~свойства вероятностных мер существенно используются 
в~данной работе, укажем на некоторые фундаментальные издания, в~которых 
изложена соответствующая тео\-рия. Понятие и~основные свойства вероятностной 
меры определены и~подробно проанализированы в~книге А.\,Н.~Ширяева~\cite[гл.~II]{15}. 
Глубокое изложение основ теории вероятностных мер имеется также в~книге 
А.\,А.~Боровкова~\cite{16}. Заметим попутно, что в~книге~\cite{16} имеются разделы, 
посвященные изложению основ теории полумарковских и~регенерирующих случайных процессов. 
Из зарубежных изданий отметим фундаментальную работу П.~Хеннекена и~А.~Тортра~\cite{17}, 
основная часть которой посвящена изложению математических основ теории вероятностей.

Введем специальное понятие вырожденной вероятностной меры, которое будет часто 
использоваться в~дальнейшем. Пусть $(U_0, \mathscr{B}_0)$~--- некоторое измеримое 
пространство, $\mathscr{B}_0$~--- $\sigma$-ал\-геб\-ра подмножеств множества~$U_0$, 
включающая в~себя все одноточечные подмножества этого множества.

\medskip

\noindent
\textbf{Определение 1.}\ Вероятностная мера~$\Psi^*$, заданная 
на $\sigma$-ал\-геб\-рe~$\mathscr{B}_0$, называется вырожденной, если существует 
такой элемент $u^* \hm\in U_0$, для которого выполняются условия $\Psi^*(\{u^*\})\hm=
1$, $\Psi^*(U_0 \setminus \{u^*\})\hm=0$, где $\{u^*\}=u^*$~--- 
множество, состоящее из единственной точки $u^* \hm\in U_0$. Соответствующая 
точка $u^* \hm\in U_0$ будет называться точкой сосредоточения вырожденной 
вероятностной меры~$\Psi^*$.
Таким образом, всякая вырожденная вероятностная мера~$\Psi^*$ определяется 
своей точкой сосредоточения~$u^*$. В~дальнейшем будем использовать 
обозначение~$\Psi_{u^*}^{*}$, имея в~виду, что вырожденная вероятностная мера~$\Psi^*$ 
сосредоточена в~точке~$u^*$.
Отметим также, что вырожденная вероятностная мера~$\Psi_{u^*}^{*}$ соответствует 
детерминированной величине, которая принимает фиксированное значение $u\hm=u^*$ 
с~вероятностью, равной единице.

\medskip

Обозначим через $\Gamma_0$ множество всех  вероятностных мер, заданных 
на измеримом пространстве ($U_0, \mathscr{B}_0$), а через~$\Gamma_0^*$~--- 
множество всех вырожденных вероятностных мер, заданных на этом пространстве, 
$\Gamma_0^*\hm\in \Gamma_0$. Аналогичные обозначения будут использоваться 
и~в~дальнейшем. Заметим, что множество~$\Gamma_0^*$ находится во взаимно
 однозначном соответствии с~множеством точек сосредоточения вырожденных 
 вероятностных мер, т.\,е.\ с~множеством~$U_0$.

Пусть $\Gamma_i^{*}$~--- множество всех вырожденных мер, заданных на 
$\sigma$-ал\-геб\-ре~$\mathscr{B}_i$, $\Gamma_i^{*}\hm\subset \Gamma_i$.
Произвольная вероятностная мера~$\Psi_i$ описывает случайную величину, 
принимающую значения в~$U_i$, а вырожденная мера~$\Psi_i^*$, сосредоточенная 
в~точке~$u_i^*$, соответствует детерминированной величине $u_i^*\hm\in U_i$.
Предполагается, что соответствующие конструкции определены на всех измеримых 
пространствах управлений $(U_1, \mathscr{B}_1), (U_2, \mathscr{B}_2), \ldots, 
(U_N,\mathscr{B}_N)$.

Предположим, что управления случайным полумарковским процессом~$\xi(t)$ 
осуществляются в~моменты времени~$t_n,$ $n\hm=0,1,2,\ldots,$
непосредственно после изменения состояния процесса. Если\linebreak 
$\xi_n\hm=\xi(t_n)\hm=i \hm\in X$, то значение управления представляет 
собой случайную величину~$u_n$, принимающую значения в~множестве допустимых 
управ\-ле\-ний~$U_i$ и~описываемую вероятностной мерой (распределе\-ни\-ем 
вероятностей) $\Psi_i \hm\in \Gamma_i$.
Будем предполагать, что при фиксированном условии $\xi_n\hm=\xi(t_n)=i$ 
управ\-ле\-ние определяется независимо от прошлого поведения процесса~$\xi(t)$ 
и~вероятностная мера~$\Psi_i$,
описывающая стохастическое управление~$u_n$, зависит только от состояния $i\hm\in X$.
Тогда выбор управ\-ле\-ний в~моменты изменения состояний $\{t_n, n\hm=0,1,2,\ldots \}$ 
описывается набором вероятностных мер (распределений вероятностей) 
$(\Psi_1, \Psi_2,\ldots, \Psi_N)$, 
$\Psi_i \hm\in \Gamma_i$, $i\hm=1,2,\ldots,N$.
Назовем любой такой набор стратегией управ\-ле\-ния полумарковским процессом~$\xi(t)$. 
По своим свойствам такая стратегия является марковской, однородной 
и~рандомизированной.

Следуя классической монографии П.~Халмоша~\cite[гл.~VII]{18}, 
рассмотрим декартово произведение пространств $U\hm=U_1\times U_2\times \cdots\times U_N$ 
и~соответствующих $\sigma$-ал\-гебр $\mathscr{B}\hm=\mathscr{B}_1\times \mathscr{B}_2
\times \cdots \times\mathscr{B}_N$. Обозначим через $\Psi\hm=\Psi_1\times \Psi_2\times \cdots
\times \Psi_N$ вероятностную меру на~$(U,\mathscr{B})$, определяемую как 
произведение мер $\Psi_1,\Psi_2, \ldots , \Psi_N$, где $\Psi_i \hm\in \Gamma_i$, 
$i\hm=1,2,\ldots,N$. Обозначим также через~$\Gamma$ множество вероятностных мер~$\Psi$, 
заданных на~$(U,\mathscr{B})$, которые пред\-став\-ля\-ют собой произведение 
мер $\Psi_1,\Psi_2, \ldots , \Psi_N$, где $\Psi_i \hm\in \Gamma_i$, $i\hm=1,2,\ldots,N$.
Множество~$\Gamma$ можно отож\-де\-ст\-вить с~множеством всех стратегий управ\-ле\-ния 
полумарковским процессом~$\xi(t)$.

Проблема оптимального управления полумар\-ковским процессом~$\xi(t)$ будет в~дальнейшем 
сформулирована в~виде задачи безусловного экстремума некоторого функционала 
$I(\Psi)\hm=I(\Psi_1,\Psi_2, \ldots , \Psi_N)$, заданного на множестве 
допустимых стратегий управления. Содержание показателя качества управления~$I(\Psi)$, 
аналитическое представление для него, а~также описание множества допустимых 
стратегий управления будут приведены в~последующих разделах данной работы.

Для получения дальнейших результатов потребуются различные вероятностные 
характеристики управляемого полумарковского процесса~$\xi(t)$. Как известно из
 общей теории полумарковских процессов~\cite{19, 20}, 
 основной вероятностной характеристикой такого процесса является так называемая 
 полумарковская функция. Определим эту функцию для процесса с~управлением 
 (см.~\cite[гл.~5]{8}):
\begin{multline}
Q_{ij}(t,u)=
{\sf P}\left(\xi_{n+1}=j,\theta_n<t \mid \xi_n=i, u_n=u\right)\,,\\
t\in [0,\infty)\,,\ u\in U_i\,;\ i,j\in X=\{1,2,\ldots,N\}\,. \label{e1}
\end{multline}
Используя полумарковские функции, можно получить вероятности перехода 
управляемой цепи Маркова~$\{\xi_n\}$:
\begin{multline}
p_{ij}(u)={\sf P}\left(\xi_{n+1}=j \mid \xi_n=i, u_n=u\right)= {}\\
{}=
\lim\limits_{t\rightarrow\infty}Q_{ij}(t,u)\,,\enskip
u\in U_i\,;\enskip i,j\in X\,, 
\label{e2}
\end{multline}
а также функции распределения длительностей пребывания полумарковского 
процесса~$\xi(t)$ в~соответствующих состояниях:

\noindent
\begin{multline}
H_{i}(t,u)={\sf P}\left(\theta_n<t \mid \xi_n=i, u_n=u\right)={}\\
{}=
\sum\limits_{j\in X}Q_{ij}(t,u)\,,\enskip
t\in [0,\infty)\,,\  u\in U_i\,; \  i\in X\,. 
\label{e3}
\end{multline}

Обозначим через
\begin{multline}
T_{i}(u)=\mathbf{E}\left[\theta_n \mid \xi_n=i, u_n=u\right]={}\\
{}=
\int\limits_0^{\infty}\left[1-H_i(t,u)\right]\,dt\,,\enskip
u\in U_i\,,\ i\in X\,, 
\label{e4}
\end{multline}
математические ожидания длительностей пребывания полумарковского процесса~$\xi(t)$ 
в~каждом из состояний.

Введенные выше характеристики~(1)--(4) определены для случая, когда 
в~момент изменения состояния~$t_n$ процесс оказывается в~состоянии~$i$ 
и~принимается решение $u\hm\in U_i$. При заданной стратегии управления 
$\Psi\hm=\left(\Psi_1,\Psi_2, \ldots , \Psi_N\right)$ можно записать 
соответствующие вероятностные характеристики без условия на управление, а~именно:
\begin{multline*}
Q_{ij}(t)={\sf P}\left(\xi_{n+1}=j,\theta_n<t \mid \xi_n=i\right)={}\\
{}=
\int\limits_{U_i}Q_{ij}(t,u) \,d\Psi_i(u)\,,\enskip 
t\in [0,\infty)\,,\ i,j\in X\,; %\label{e5}
\end{multline*}

\vspace*{-12pt}

\noindent
\begin{multline}
p_{ij}={\sf P}\left(\xi_{n+1}=j \mid \xi_n=i\right)=
\int\limits_{U_i}p_{ij}(u)\, d\Psi_i(u)\,,\\  
i,j\in X\,; 
\label{e6}
\end{multline}

\vspace*{-9pt}

\noindent
\begin{equation}
T_{i}=\mathbf{E}\left[\theta_n \mid \xi_n=i\right]=
\int\limits_{U_i}T_{i}(u)\,d\Psi_i(u)\,,\enskip i\in X\,. 
\label{e7}
\end{equation}
В дальнейшем будем предполагать, что для рас\-смат\-ри\-ва\-емой 
полумарковской модели заданы вероятностные характеристики 
$p_{ij}(u)$, $u\hm\in U_i$, $i,j\hm\in X$, и~$T_i(u)$, $u\hm\in U_i$, $i\hm\in X$, 
определяемые соотношениями~(\ref{e2}) и~(\ref{e4}). 
Для фиксированной стратегии управления $\Psi\hm=(\Psi_1, \Psi_2,\ldots, \Psi_N)$ 
соответствующие вероятностные характеристики~$p_{ij}$ и~ $T_i$, $i,j\hm\in X,$ 
определены равенствами~(\ref{e6}) и~(\ref{e7}) без условий на управление.

\section{Стационарный стоимостной показатель качества управления}

Определим некоторый стоимостной аддитивный функционал, связанный 
с~рассматриваемым полумарковским процессом~$\xi(t)$. По содержанию этот функционал 
представляет собой случайный\linebreak доход или прибыль, накопленную за период времени $[0,t]$. 
Определения такого функционала приведены в~основополагающих работах~[3; 8, гл.~5].\linebreak 
Обозначим через $\widetilde{v}(t)$, $t\hm\geq 0$, значение этого аддитивного 
функционала в~момент времени~$t$; $\widetilde{v}_n\hm=\widetilde{v}(t_n\hm+0)$~--- 
соответствующее значение непосредственно после очередного момента изменения\linebreak 
состояния~$t_n$, $n\hm=0,1,2,\ldots$; $\widetilde{v}_0\hm=v_0$~--- 
заданное начальное значение в~момент $t\hm=0$. Рассмотрим величину
\begin{multline}
d_i(u)=\mathbf{E}\left[\widetilde{v}_{n+1}-\widetilde{v}_n \mid \xi_n=i\,, 
u_n=u\right]\,,\\
u\in U_i\,, \enskip i\in X\,, \label{e8}
\end{multline}
представляющую собой математическое ожидание приращения стоимостного 
аддитивного функционала за период времени между последовательными 
изменениями состояния полумарковского процесса~$\xi(t)$. Тогда соответствующее 
математическое ожидание, вычисляемое без условия на решение, 
принимаемое в~момент времени~$t_n$, представляется в~виде:
\begin{equation*}
d_i=\mathbf{E}\left[\widetilde{v}_{n+1}-\widetilde{v}_n \mid \xi_n=i\right]=
\!\int\limits_{U_i}\!d_i(u)\,d\Psi_i(u)\,,\ i\in X \,. %\label{e9}
\end{equation*}

Предположим, что для заданной стратегии управ\-ле\-ния 
$\Psi\hm=(\Psi_1,\Psi_2,\ldots,\Psi_N)$ вложенная цепь Маркова~$\{\xi_n\}$ 
имеет ровно один класс возвратных положительных состояний (по терминологии, 
принятой в~\cite{8}, такое множество состояний называется эргодическим классом). 
Как известно~\cite[гл.~VIII]{15}, данное условие является необходимым 
и~достаточным для существования единственного\linebreak стационарного распределения. 
Обозначим это стационарное распределение цепи Маркова~$\{\xi_n\}$ через 
$\pi\hm=(\pi_1, \pi_2,\ldots, \pi_N)$. Заметим, что данное\linebreak распределение зависит  
от стратегии управления $\Psi\hm=(\Psi_1,\Psi_2,\ldots,\Psi_N)$. При указанном 
условии имеет место следующий результат, называемый эргодической теоремой 
для аддитивного стоимостного функционала:
\begin{equation}
I=\lim\limits_{t\rightarrow\infty}\fr{\mathbf{E}\widetilde{v}(t)}{t}=
\fr{\sum\nolimits_{i=1}^N d_i\pi_i}{\sum\nolimits_{i=1}^N T_i\pi_i}\,. 
\label{e10}
\end{equation}

Соотношение~(\ref{e10}) доказано в~работе~\cite[гл.~5]{8}. Заметим, что аналогичные 
результаты имеют мес\-то для гораздо более общих полумарковских моделей~\cite{10, 11}.

По своему прикладному содержанию величина, определяемая соотношением~(\ref{e10}), 
представляет собой
среднюю удельную прибыль, связанную с~эволюцией системы в~стационарном
режиме. Кроме того, величина~$I$ представляет собой функционал от
набора вероятностных распределений~$\Psi_{i}$, $i\hm\in\lbrace 1,\ldots
,N\rbrace $, определяющих стратегию управле-\linebreak\vspace*{-12pt}

\pagebreak

\noindent
ния системой. 
В~дальнейшем будем рассматривать стационарный стоимостной функционал 
$I\hm=I(\Psi_{1},\Psi_{2},\ldots , \Psi_{N})$ как
показатель качества управ\-ле\-ния системой и~построенным полумарковским
процессом~$\xi (t)$.

\section{Представление стационарного показателя в~форме
дробно-линейного интегрального функционала}

В данном разделе будет приведено утверждение об аналитическом
представлении стационарного стоимостного функционала~(\ref{e10}), 
служащего критерием качества управления в~рассматриваемой задаче управления 
полумарковским процессом.

\smallskip

\noindent
\textbf{Теорема 1.} \textit{Стационарный стоимостной показатель, 
определяемый равенством}~(\ref{e10}), \textit{представляет собой дроб\-но-ли\-ней\-ный
функционал от вероятностных распределений~$\Psi_{i}(u_{i})$,
$i\hm\in\{1,\dots,N\}$. Данный функционал задается
аналитически следующей формулой:}
\begin{multline}
I=I(\Psi_{1},\ldots, \Psi_{N})={}\\
\hspace*{-2mm}{}=\!
\fr{\int\nolimits_{U_1}\!{\cdots\! 
\int\nolimits_{U_N}\!{A(u_{1},\ldots ,u_{N})d\Psi_{1}(u_{1})\cdots
\,d\Psi_{N}(u_{N})}}}{\int\nolimits_{U_1}{\!\cdots\! \int\nolimits_{U_N}\!{B(u_{1},\ldots ,u_{N})
\,d\Psi_{1}(u_{1})\ldots
d\Psi_{N}(u_{N})}}},\!\!\! \label{e11}
\end{multline}
\textit{где подынтегральные функции числителя и~знаменателя выражаются
соотношениями}:
\begin{align}
A(u_{1},\ldots
,u_{N})&={}\notag\\
&\hspace*{-20mm}{}=\sum\limits_{i=1}^{N}{d_{i}(u_{i})}{\widehat{D}}^{(i)}(u_{1}, \ldots
,u_{i-1},u_{i+1}, \ldots , u_{N})\,;  \label{e12}\\
 B(u_{1},\ldots
,u_{N})&={}\notag\\
&\hspace*{-20mm}{}=\sum\limits_{i=1}^{N}{T_{i}(u_{i})}{\widehat{D}}^{(i)}(u_{1}, \ldots
,u_{i-1},u_{i+1}, \ldots , u_{N})\,.  \label{e13}
\end{align}
\textit{В свою очередь, функции} ${\widehat{D}}^{(i)}(u_{1}, \ldots
,u_{i-1},u_{i+1}, \ldots$\linebreak $\ldots , u_{N})$, $i\hm\in\{1,\dots,N\}$, 
\textit{входящие в~правые части формул}~(\ref{e12}) и~(\ref{e13}), 
\textit{определяются следующим образом:}

\noindent
\begin{multline}
{\widehat{D}}^{(i)}(u_{1}, \ldots ,u_{i-1},u_{i+1}, \ldots , u_{N})={}
\\
{}=(-1)^{N+i+2}\sum\limits_{\alpha ^{(N),i}}{(-1)}^{\delta (\alpha
^{(N),i}) }{\widehat{D}}_{0}^{(i)}\left(\alpha ^{(N),i},u_{1}, \ldots\right.\\
\left.\ldots , u_{i-1},u_{i+1}, \ldots , u_{N}\right)\,. \label{e14}
\end{multline}
\textit{Здесь} $\alpha ^{(N),i}=(\alpha _{1}, \ldots , \alpha _{i-1},\alpha_{i+1}, \ldots , 
\alpha _{N})$~\textit{--- произвольная
перестановка чисел }$(1, \ldots , i-1, i+1, \ldots , N)$;
$\delta
(\alpha ^{(N),i})$~\textit{--- число инверсий в~перестановке} 
$\alpha ^{(N),i}$;

\noindent
\begin{multline}
{\widehat{D}}_{0}^{(i)}(\alpha ^{(N),i},u_{1}, \ldots ,u_{i-1},u_{i+1},
\ldots , u_{N})={}\\
{} ={\widetilde{p}}_{1,\alpha _{1}}\left(u_{1}\right)\cdots {\widetilde{p}}_{i-1,\alpha
_{i-1}}\left(u_{i-1}\right){\widetilde{p}}_{i+1,\alpha _{i+1}}\left(u_{i+1}\right)\cdots\\
\cdots
{\widetilde{p}}_{N,\alpha _{N}}\left(u_{N}\right)\,, 
\label{e15}
\end{multline}
где
\begin{multline}
 {\widetilde{p}}_{k,\alpha _{k}}(u_{k})=
\begin{cases}
p_{kk}(u_{k})-1,\  & \alpha _{k}=k\,; \\
p_{k,\alpha _{k}}(u_{k}),\  & \alpha _{k}\ne k, \\
\end{cases}\\
 k=1, \ldots , i-1, i+1, \ldots ,N\,. \label{e16}
 \end{multline}
\textit{Функции $p_{ij}(u_i)$, $T_{i}(u_{i})$ и~$d_{i}(u_{i})$,
$u_i\hm\in U_i$, $i,j\hm\in \{1,2,\ldots,N\}$, 
входящие в~соотношения}~(\ref{e12})--(\ref{e16}), 
\textit{определяются равенствами}~(\ref{e2}), (\ref{e4}) \textit{и}~(\ref{e8}) \textit{соответственно.}

\smallskip

\noindent
Д\,о\,к\,а\,з\,а\,т\,е\,л\,ь\,с\,т\,в\,о\ теоремы~1 
в~весьма сжатой форме приведено в~работе~\cite{21}. Читателю, интересующемуся 
более подробным обоснованием данного результата, порекомендуем обратиться к~тексту 
кандидатской диссертации А.\,В.~Иванова~\cite[гл.~3]{22}.

\smallskip

Итак, теорема~1 позволяет получить явное аналитическое представление 
для стационарного стоимостного показателя вида~(\ref{e10}) в~форме 
дроб\-но-ли\-ней\-но\-го интегрального функционала от набора\linebreak вероятностных мер 
$\Psi\hm=(\Psi_{1},\Psi_{2},\ldots , \Psi_{N})$, за\-да\-ющих стратегию управления 
полумарковским процессом~$\xi(t)$. При этом подынтегральные функции числителя 
и~знаменателя задаются формулами~(\ref{e12}), (\ref{e13}) 
и~вспомогательными равенствами~(\ref{e14})--(\ref{e16}). Таким образом, функция
\begin{equation}
C\left(u_1, u_2,\ldots, u_N\right)=\fr{A(u_1, u_2,\ldots, u_N)}{B(u_1, u_2,\ldots, u_N)}\,,
\label{e17}
\end{equation}
которая в~дальнейшем будет называться основной функцией дроб\-но-ли\-ней\-но\-го 
интегрального функционала~(\ref{e11}) и~которая будет играть важную роль 
в~дальнейшем исследовании, также явно определяется формулами~(\ref{e17}), 
(\ref{e12}), (\ref{e13}).

\section{Формальная постановка оптимизационной задачи 
и~условия существования оптимальной стратегии управления полумарковским процессом}

Будем рассматривать проблему управления полумарковским процессом~$\xi(t)$ в~форме 
экстремальной задачи
\begin{multline}
I(\Psi)=I\left(\Psi_1, \Psi_2,\ldots,\Psi_N\right)\rightarrow \mathrm{extr}\,,
\\
\Psi=\left(\Psi_1, \Psi_2,\ldots,\Psi_N\right)\in\Gamma\,. \label{e18}
\end{multline}
При этом показатель качества управления~$I(\Psi)$ представляет собой 
дроб\-но-ли\-ней\-ный интегральный функционал вида~(\ref{e11}).

Для решения экстремальной задачи~(\ref{e18}) воспользуемся некоторым утверждением 
об экстремуме дроб\-но-ли\-ней\-но\-го интегрального функционала. Прежде 
чем сформулировать данное утверждение, отметим, что в~теории оптимизации 
хорошо известны задачи, в~которых целевая функция представляет собой 
отношение двух линейных отображений, а имеющиеся ограничения также линейны. 
Такой раздел называется дроб\-но-ли\-ней\-ным программированием. Основные
 теоретические результаты данного направления изложены в~работе~\cite{23},
  там же приведена подробная библиография. В~дальнейшем потребуется некоторый 
  специальный результат о безусловном экстремуме дроб\-но-ли\-ней\-но\-го 
  интегрального функционала вида~(\ref{e11}), который был впервые сформулирован 
  в~работе~\cite{14}. Заметим, что для использования этого результата необходимо, 
  чтобы выполнялись некоторые предварительные условия, которые в~данном случае 
  можно сформулировать следующим образом:
\begin{enumerate}[1.]
\item Интегральные выражения
\begin{align*}
I_1(\Psi)&=I_1\left(\Psi_1,\Psi_2,\ldots,\Psi_N\right)={}&\\
&\hspace*{-13mm}{}=\int\limits_{U_1}\!\cdots\!
\int\limits_{U_N}\!\!A\left(u_1,\ldots ,u_N\right)\,
d\Psi_1\left(u_1\right) %d\Psi_2\left(u_2\right)
\cdots
 d\Psi_N\left(u_N\right)\,;
\\
I_2(\Psi)&=I_2\left(\Psi_1,\Psi_2,\ldots,\Psi_N\right)={}&\\
&\hspace*{-13mm}{}=\int\limits_{U_1}\!\cdots\!\int\limits_{U_N}\!\!
B\left(u_1,\ldots,u_N\right)\,
d\Psi_1\left(u_1\right)% d\Psi_2\left(u_2\right)\cdots\\
\cdots d\Psi_N\left(u_N\right)
\end{align*}
определены для всех стратегий управления $\Psi\hm=(\Psi_1, \ldots,\Psi_N)
\hm\in \Gamma$.

\item Функционал $I_2(\Psi)=I_2(\Psi_1, \ldots,\Psi_N)\hm\neq 0$ 
для всех $\Psi\hm=(\Psi_1, \ldots,\Psi_N)\hm\in \Gamma$.

\item Множество $\Gamma$ включает в~себя множество всех вырожденных 
вероятностных мер: $\Gamma^* \hm\subset \Gamma$.
\end{enumerate}

Сделаем несколько важных замечаний по поводу введенных предварительных условий.

\smallskip

\noindent
\textbf{Замечание~1.}\ Из условия~2 следует, что функция $B(u_1, u_2,\ldots, u_N)$ 
не может принимать значения разных знаков. С~учетом условия~3 
получаем, что указанная функция должна обладать \mbox{свойством} строгой 
знакопостоянности на всем множестве~$U$. С~другой стороны, если выполняется 
условие строгой знакопостоянности функции $B(u_1, u_2,\ldots, u_N), 
(u_1, u_2,\ldots, u_N)\hm\in U$, то условие~2 выполняется автоматически.

\smallskip

\noindent
\textbf{Замечание~2.}\ Если рассматривать в~качестве целевого функционала 
$I(\Psi_1, \Psi_2,\ldots,\Psi_N)$ экстремальной задачи~(\ref{e18}) 
стационарный стоимостной пока\-затель~(\ref{e10}), то функция $B(u_1,u_2,\ldots,u_N)$ 
имеет\linebreak следующее теоретическое содержание. Данная функция представляет собой условное 
математическое ожидание длительности периода времени между соседними моментами 
изменения со\-сто\-яния полумарковского процесса~$\xi(t)$ при условии, что стратегия 
его управ\-ле\-ния является детерминированной и~задается набором значений аргументов 
$(u_1,u_2,\ldots,u_N)$. Тогда условие строгой положительности функции 
$B(u_1,u_2,\ldots,u_N)$ при всех $(u_1,u_2,\ldots,u_N)\hm\in U$ является естественным 
и~фактически означает, что при любой заданной детерминированной стратегии 
управ\-ле\-ния процесс~$\xi(t)$ не имеет мгновенных со\-сто\-яний, длительность пребывания 
в~которых равна нулю.

\smallskip

\noindent
\textbf{Замечание~3.}\ Сделаем некоторые замечания, связан\-ные с~подынтегральной 
функцией числителя дроб\-но-ли\-ней\-но\-го интегрального функционала~(\ref{e11}). 
Как и~ранее, будем рассматривать в~качестве целевого функционала $I(\Psi_1, \Psi_2,\ldots,\Psi_N)$\linebreak 
экстремальной задачи~(\ref{e18}) стационарный стоимостной показатель~(\ref{e10}). 
Тогда для любого фиксированного набора значений аргументов $(u_1,u_2,\ldots,u_N)\hm\in U$ 
значение функции $A(u_1,u_2,\ldots\linebreak \ldots,u_N)$ представляет собой условное математическое
 ожидание приращения рассматриваемого стоимостного функционала, 
 происшедшее за время пребывания полумарковского процесса~$\xi(t)$ в~некотором 
 фиксированном  состоянии при условии, что стратегия управления является 
 детерминированной и~задается указанным набором $(u_1,u_2,\ldots,u_N)\hm\in U$. 
 Отметим, что в~теореме об экстремуме дроб\-но-ли\-ней\-но\-го интегрального 
 функционала, доказанной в~работе~\cite[гл.~10]{12}, 
 на подынтегральную функцию числителя накладываются условия ограниченности на 
 всем множестве значений аргумента. Для многих математических моделей и~связанных 
 с~ними задач оптимального управления такое условие является излишне ограничительным. 
 В~качестве примера можно привести модели оптимального управления запасом непрерывного 
 продукта, рассмотренные в~работах~\cite{27, 28}. 
 В~настоящем исследовании на функцию $A(u_1,u_2,\ldots,u_N)$ накладывается только 
 условие интегрируемости по любому заданному набору вероятностных мер 
 $\Psi\hm=(\Psi_1, \Psi_2,\ldots,\Psi_N)$, образующему стратегию управления 
 полумарковским процессом~$\xi(t)$ (условие~1 системы предварительных условий).

\smallskip

\noindent
\textbf{Замечание~4.} Условия~1--3 являются необходимыми для корректной 
постановки задачи безусловного экстремума дроб\-но-ли\-ней\-но\-го интегрального 
функционала. Если этот функционал служит показателем качества в~задаче оптимального 
управления случайным процессом, то необходимо добавить к~этим условиям дополнительное, 
связанное с~некоторой регулярностью самого управляемого процесса, а~именно: некоторый 
содержательный показатель, связанный с~поведением этого процесса, должен существовать 
и~быть представимым в~виде дроб\-но-ли\-ней\-но\-го интегрального функционала. 
Если потребовать, чтобы выполнялось эргодическое соотношение~(\ref{e10}), 
то можно использовать\linebreak теорему~1 и~сформулировать задачу оптимального управ\-ле\-ния 
в~виде~(\ref{e18}) для дроб\-но-ли\-ней\-но\-го\linebreak интегрального функционала~(\ref{e11}). 
Таким образом, необходимо ввести условие, обеспечивающее существование единственного 
стационарного распределения вложенной цепи Маркова и~выполнение\linebreak соотношения~(\ref{e10}). 
По аналогии с~[8, гл.~5] сформулируем это дополнительное условие в~следующем виде:
\begin{enumerate}
\setcounter{enumi}{3}
\item Для любой рассматриваемой стратегии управ\-ле\-ния $\Psi\hm=
(\Psi_1, \Psi_2,\ldots,\Psi_N)\hm\in \Gamma$ вложенная цепь Маркова 
полумарковского процесса $\xi(t)$ имеет ровно один класс возвратных 
положительных состояний.
\end{enumerate}

Теперь определим понятие допустимой стратегии управления полумарковским процессом 
с~конечным множеством состояний.

\smallskip

\noindent
\textbf{Определение~2.}\ Назовем стратегию управления 
$\Psi\hm=(\Psi_1, \Psi_2,\ldots,\Psi_N)$ 
допустимой в~данной задаче, если она удовлетворяет условиям~1--4.


\smallskip

\noindent
\textbf{Замечание~5.}\ Как следует из замечания~1, если потребовать, 
чтобы функция $B(u_1, u_2,\ldots,u_N)$ являлась строго знакопостоянной при 
всех $(u_1, u_2,\ldots,u_N)\hm\in U$, то можно считать допустимыми стратегии 
$(\Psi_1, \Psi_2,\ldots,\Psi_N)$, удовлетворяющие условиям~1, 3, 4. С~учетом замечания~2 
о~естественном характере условия строгой знакопостоянности функции $B(u_1,u_2,\ldots,u_N)$ 
при всех значениях аргументов $(u_1, u_2,\ldots,u_N)\hm\in U$ будем требовать 
выполнения этого условия в~формулировке приводимой в~дальнейшем основной 
теоремы об оптимальной стратегии управления полумарковским процессом.

\smallskip

\noindent
\textbf{Замечание~6.}\ Ниже будет сформулирована и~доказана основная 
теорема об оптимальной стра\-тегии управления полумарковским процессом с~конеч\-ным 
множеством состояний. Будем формулировать эту теорему по отношению к~экстремальной 
задаче~(\ref{e18}), в~которой целевой функционал $I(\Psi_1, \Psi_2,\ldots,\Psi_N)$ 
имеет вид дроб\-но-ли\-ней\-но\-го интегрального функционала. 
Это обстоятельство связано с~тем, что целевой функционал в~задаче 
оптимального управления необязательно должен иметь характер стационарного 
стоимостного показателя вида~(\ref{e10}). В~частности, еще в~1983~г.\ П.\,В.~Шнурковым 
было установлено~\cite{24}, что ряд показателей, связанных 
с~временем пребывания управляемого полумарковского процесса в~заданном конечном 
подмножестве состояний, имеет структуру дроб\-но-ли\-ней\-но\-го интегрального 
функционала от набора вероятностных мер, определяющих стратегию управления. 
Таким образом, рассматриваемая задача управления имеет более общий характер, 
чем задача, в~которой целевой функционал представляет собой стационарный 
стоимостной показатель вида~(\ref{e10}).






\smallskip

\noindent
\textbf{Замечание~7.}\ Если рассматривать задачу оптимального управления 
полумарковским процессом, в~кото\-рой целевой функционал не совпадает 
со стационарным стоимостным показателем~(\ref{e10}), то возможно, что могут 
потребоваться другие дополнительные условия, обеспечивающие существование этого 
показателя и~его представление в~форме~(\ref{e11}). В~связи с~этим в~формулировке 
основной теоремы будем использовать термин допустимые стратегии в~широком смысле, 
имея в~виду выполнение всех необходимых условий для каждого рассмат\-ри\-ва\-емо\-го 
показателя качества управления.

\smallskip


\noindent
\textbf{Замечание 8.} Множество допустимых стратегий может 
не совпадать с~множеством всех возможных стратегий управления. 
В~частности, допустимые стратегии могут состоять только из дискретных вероятностных 
мер $\Psi_1, \Psi_2,\ldots,\Psi_N$, т.\,е.\ таких, которые сосредоточены на дискретных 
множествах точек пространств $U_1, U_2,\ldots,U_N$.

\section{Теоретическое решение задачи оптимального управления}

Перейдем к~формулировке и~доказательству тео\-ре\-мы об 
оптимальной стратегии управ\-ле\-ния полумарковским процессом с~конечным 
множеством состояний.

\smallskip

\noindent
\textbf{Теорема~2.} \textit{Рассмотрим проблему оптимального управ\-ле\-ния 
полумарковским процессом~$\xi(t)$ в~виде экстремальной задачи}~(\ref{e18}), 
\textit{определенной на множестве допустимых стратегий $\Gamma$, 
для дроб\-но-ли\-ней\-но\-го 
функционала}~(\ref{e11}). \textit{Пусть функция $B(u_1,u_2,\ldots,u_N)$, 
входящая в~определение функционала}~(\ref{e11}),
\textit{является строго знакопостоянной (строго положительной или строго отрицательной) 
при всех значениях аргументов $(u_1,u_2,\ldots,u_N)\hm\in U$.
Тогда справедливы сле\-ду\-ющие утверждения}:
\begin{enumerate}[1.]
\item \textit{Если функция} $C(u_1,u_2,\ldots,u_N)\hm=A(u_1,u_2,\ldots$\linebreak
$\ldots,u_N)/{B(u_1,u_2,\ldots,u_N)}$ 
\textit{ограничена сверху или снизу и~достигает глобального экст\-ре\-му\-ма на множестве
$U\hm=U_1\times U_2\times \cdots \times U_N$ (максимума или минимума), 
то оптимальная стратегия управления полумарковским процессом~$\xi(t)$ существует, 
является детерминированной и~определяется
вырожденной вероятностной мерой $\Psi^*\hm\in \Gamma^*$, сосредоточенной в~точке, 
в~которой достига\-ет соответствующего экстремума функция $C(u_1,u_2,\ldots,u_N)$,
и~при этом выполняются соотношения}:
\begin{multline}  %{\substack{{i=\overline{1,n}}\\ {j=\overline{1,l}}}}
\max\limits_{\Psi \in \Gamma} I(\Psi)=
\max\limits_{\substack{{\Psi_i \in \Gamma_i\,,}\\ 
{i=\overline{1,N}}}}
I\left(\Psi_1,\Psi_2,\ldots,\Psi_N\right)={}\\
{}=
\max\limits_{\substack{{\Psi_i^* \in \Gamma_i^*,}\\ 
{i=\overline{1,N}}}}
 I\left(\Psi_1^*,\Psi_2^*,\ldots,\Psi_N^*\right)={}\\
{}=\max\limits_{(u_1,u_2,\ldots,u_N)\in U}\fr{A(u_1,u_2,\ldots,u_N)}
{B(u_1,u_2,\ldots,u_N)}\,; \label{e19}
\end{multline}

\vspace*{-12pt}

\noindent
\begin{multline*}
\min\limits_{\Psi \in \Gamma} I(\Psi)=
\min\limits_{\substack{{\Psi_i \in \Gamma_i\,,}\\ 
{i=\overline{1,N}}}} I\left(\Psi_1,\Psi_2,\ldots,\Psi_N\right)={}\\
{}=
\min\limits_{\substack{{\Psi_i^* \in \Gamma_i^*,}\\ 
{i=\overline{1,N}}}}
I\left(\Psi_1^*,\Psi_2^*,\ldots,\Psi_N^*\right)={}\\
{}=\min\limits_{(u_1,u_2,\ldots,u_N)\in U}\fr{A(u_1,u_2,\ldots,u_N)}
{B(u_1,u_2,\ldots,u_N)}\,. %\label{e20}
\end{multline*}
\item \textit{Если функция $C(u_1,u_2,\ldots,u_N)\hm=
{A(u_1,u_2,\ldots,u_N)}/{B(u_1,u_2,\ldots,u_N)}$ ограничена сверху или снизу, 
но не достигает глобального экстремума на множестве $U\hm=U_1\times U_2\times\cdots
\times U_N$,
то для любого $\varepsilon\hm > 0$ можно выбрать $\varepsilon$-оп\-ти\-маль\-ную 
детерминированную стратегию управления полумарковским процессом~$\xi(t)$, 
которая определяется вырожденной
вероятностной мерой $\Psi^{*(+)}(\varepsilon)\hm\in \Gamma^*$ или вырожденной
вероятностной мерой $\Psi^{*(-)}(\varepsilon)\hm\in \Gamma^*$, в~зависимости от 
вида экстремума (максимума или минимума) в~задаче}~(\ref{e18}). 
\textit{При этом вероятностная мера $\Psi^{*(+)}(\varepsilon)\hm\in \Gamma^*$ может быть 
сосредоточена в~любой точке $\left(u_1^{(+)}(\varepsilon),u_2^{(+)}(\varepsilon),\ldots,
u_N^{(+)}(\varepsilon)\right)$, удовлетворяющей соотношению}:
\begin{multline}
\sup\limits_{(u_1,u_2,\ldots,u_N) \in U}
\fr{A(u_1,u_2,\ldots,u_N)}{B(u_1,u_2,\ldots,u_N)}-\varepsilon <{}\\
{}<
\fr{A\left(u_1^{(+)}(\varepsilon),u_2^{(+)}(\varepsilon),\ldots,u_N^{(+)}
(\varepsilon)\right)}
{B\left(u_1^{(+)}(\varepsilon),u_2^{(+)}(\varepsilon),\ldots,u_N^{(+)}
(\varepsilon)\right)}<{}\\
{}<\sup\limits_{(u_1,u_2,\ldots,u_N) \in U}
\fr{A(u_1,u_2,\ldots,u_N)}{B(u_1,u_2,\ldots,u_N)}<\infty\,, 
\label{e21}
\end{multline}
\textit{если функция $C(u_1,u_2,\ldots,u_N)$ ограничена сверху 
и~экстремальная задача}~(\ref{e18}) 
\textit{представляет собой задачу на максимум. Аналогично вероятностная мера 
$\Psi^{*(-)}(\varepsilon)\hm\in \Gamma^*$ может быть сосредоточена в~любой точке 
$\left(u_1^{(-)}(\varepsilon),u_2^{(-)}(\varepsilon),\ldots,u_N^{(-)}(\varepsilon)
\right)$, удовлетворяющей соотношению}:

\noindent
\begin{multline*}
-\infty<\inf\limits_{(u_1,u_2,\ldots,u_N) \in U}\fr{A(u_1,u_2,\ldots,u_N)}
{B(u_1,u_2,\ldots,u_N)} <{}\\
{}<
\fr{A\left(u_1^{(-)}(\varepsilon),u_2^{(-)}
(\varepsilon),\ldots,u_N^{(-)}(\varepsilon)\right)}
{B\left(u_1^{(-)}(\varepsilon),u_2^{(-)}(\varepsilon),\ldots,
u_N^{(-)}(\varepsilon)\right)}<{}\\
{}<\inf\limits_{(u_1,u_2,\ldots,u_N) \in U}
\fr{A(u_1,u_2,\ldots,u_N)}{B(u_1,u_2,\ldots,u_N)}+\varepsilon\,, 
%\label{e22}
\end{multline*}
\textit{если функция $C(u_1,u_2,\ldots,u_N)$ ограничена снизу и~экстремальная 
задача}~(\ref{e18})  \textit{представляет собой задачу на минимум}.
\item \textit{Если функция $C(u_1,u_2,\ldots,u_N)\hm=
{A(u_1,u_2,\ldots,u_N)}/{B(u_1,u_2,\ldots,u_N)}$ не ограничена сверху 
или снизу, то оптимальной стратегии управления в~смысле
соответствующей экстремальной задачи не существует. 
При этом найдется такая последовательность вырожденных вероятностных
мер~$\Psi^{*(+)}(n)$, сосредоточенных в~точках 
$\left(u_1^{(+)}(n),u_2^{(+)}(n),\ldots,u_N^{(+)}(n)\right)$, $n\hm=1,2,\dots $, 
для которых выполняется соотношение}:
\begin{multline*}
I\left(\Psi^*(n)\right)={}\\
{}=
I\left(\Psi_1^{*(+)}(n),\Psi_2^{*(+)}(n),\ldots,\Psi_N^{*(+)}(n)\right)={}\\
{}=\fr{A\left(u_1^{(+)}(n),u_2^{(+)}(n),\ldots,u_N^{(+)}(n)\right)}
{B\left(u_1^{(+)}(n),u_2^{(+)}(n),\ldots,u_N^{(+)}(n)\right)}\to 
\infty\\
\mbox{при}\ n\to\infty\,, 
%\label{e23}
\end{multline*}
\textit{если функция $C(u_1,u_2,\ldots,u_N)$ не ограничена сверху. 
Аналогично найдется такая последовательность вырожденных вероятностных
мер~$\Psi^{*(-)}(n)$, сосредоточенных в~точках 
$\left(u_1^{(-)}(n),u_2^{(-)}(n),\ldots,u_N^{(-)}(n)\right)$, 
$n\hm=1,2,\dots $, для которых выполняется соотношение}:
\begin{multline*}
I\left(\Psi^{*(-)}(n)\right)={}\\
{}= I
\left(\Psi_1^{*(-)}(n),\Psi_2^{*(-)}(n),\ldots,\Psi_N^{*(-)}(n)\right)={}\\
{}=\fr{A\left(u_1^{(-)}(n),u_2^{(-)}(n),\ldots,u_N^{(-)}(n)\right)}
{B\left(u_1^{(-)}(n),u_2^{(-)}(n),\ldots,u_N^{(-)}(n)\right)}\to 
-\infty\\
\mbox{при}~~n\to\infty\,,  
%\label{e24}
\end{multline*}
\textit{если функция $C(u_1,u_2,\ldots,u_N)$ не ограничена \mbox{снизу}}.
\end{enumerate}
\textit{При этом сформулированные утверждения каждого пункта теоремы~$2$ 
могут выполняться как по отдельности, для одного из двух
видов экстремума, так и~совместно, для обоих видов экстремума.}

\smallskip

Прежде чем непосредственно доказывать теорему~2, докажем некоторые 
вспомогательные утверждения.

\smallskip

\noindent
\textbf{Лемма~1.}\ 
\textit{Рассмотрим дроб\-но-ли\-ней\-ный интегральный функционал 
$I(\Psi_1, \Psi_2,\ldots, \Psi_N)$ вида}~(\ref{e11}), 
\textit{заданный на некотором множестве наборов вероятностных мер 
$\Psi\hm=(\Psi_1, \Psi_2,\ldots, \Psi_N)\hm \in \Gamma$. Предположим, что на 
множестве~$\Gamma$ выполняется условие~$1$ из набора предварительных условий 
и~функция $B(u_1, u_2,\ldots, u_N)$  обладает свойством строгой знакопостоянности 
при всех $(u_1, u_2,\ldots, u_N) \hm\in U$. Тогда справедливы следующие утверждения}:
\begin{enumerate}[1.]
\item \textit{Если основная функция 
$C(u_1, u_2,\ldots, u_N)\hm={A(u_1, u_2,\ldots, u_N)}/{B(u_1, u_2,\ldots, u_N)}$ 
ограничена сверху, т.\,е.\ выполняется условие}
\begin{multline}
C\left(u_1, u_2,\ldots, u_N\right)=
\fr{A(u_1, u_2,\ldots, u_N)}{B(u_1, u_2,\ldots, u_N)}\leq {}\\
{}\leq
c_0^{(+)}<\infty \,, \enskip \left(u_1, u_2,\ldots, u_N\right) \in U\,, \label{e25}
\end{multline}
\textit{то имеет место неравенство}:
\begin{equation}
I\left(\Psi_1, \Psi_2,\ldots, \Psi_N\right)\leq c_0^{(+)} 
\label{e26}
\end{equation}
\textit{для всех} $(\Psi_1, \Psi_2,\ldots, \Psi_N) \in \Gamma$.
\item \textit{Если основная функция 
$C(u_1, u_2,\ldots, u_N)\hm={A(u_1, u_2,\ldots, u_N)}/{B(u_1, u_2,\ldots, u_N)}$ 
ограничена снизу, т.\,е.\ выполняется условие}
\begin{multline*}
C\left(u_1, u_2,\ldots, u_N\right)=\fr{A(u_1, u_2,\ldots, u_N)}{B(u_1, u_2,\ldots, 
u_N)}\geq{}\\
{}\geq c_0^{(-)}>-\infty \,, 
\left(u_1, u_2,\ldots, u_N\right) \in U\,, 
%\label{e27}
\end{multline*}
\textit{то имеет место неравенство}:
\begin{equation*}
I\left(\Psi_1, \Psi_2,\ldots, \Psi_N\right)\geq c_0^{(-)} 
%\label{e28}
\end{equation*}
\textit{для всех} $(\Psi_1, \Psi_2,\ldots, \Psi_N) \hm\in \Gamma$.
\end{enumerate}

\noindent
Д\,о\,к\,а\,з\,а\,т\,е\,л\,ь\,с\,т\,в\,о\ \ леммы~1.\ 
Докажем первое утверждение леммы. Предположим сначала, 
что функция $B(u_1, u_2,\ldots,  u_N)$ строго положительна:
\begin{equation}
B\left(u_1, u_2,\ldots, u_N\right)>0\,,\enskip
\left(u_1, u_2,\ldots, u_N\right)\in U\,. \label{e29}
\end{equation}
Заметим, что в~таком случае по свойству интеграла~\cite[гл.~V]{18}
\begin{multline}
\hspace*{-2mm}\int\limits_{U_1}\!\!\cdots\! \!\int\limits_{U_N}\!\!B(u_1, \ldots,u_N) \,
d\Psi_1(u_1)%d\Psi_2(u_2)\cdots\\
\cdots d\Psi_N(u_N)>0 \!\!\!\!\label{e30}
\end{multline}
для любого фиксированного набора $\Psi\hm=(\Psi_1, \ldots, \Psi_N)\hm\in \Gamma$.
Из неравенства~(\ref{e25}) с~уче\-том~(\ref{e29}) получаем:
\begin{multline}
\hspace*{-4mm}A\left(u_1,\ldots, u_N\right)\leq{}\\
\hspace*{-4mm}{}\leq c_0^{(+)} B\left(u_1, \ldots, u_N\right)\,, 
\left(u_1, \ldots, u_N\right)\in U\,. \label{e31}
\end{multline}
В свою очередь, из неравенства~(\ref{e31}) и~свойств интеграла следует:
\begin{multline}
\int\limits_{U_1}\!\!\cdots\! \!\int\limits_{U_N}\!\!A(u_1,\ldots, u_N) \,
d\Psi_1\left(u_1\right)%d\Psi_2\left(u_2\right)\cdots\\
\cdots d\Psi_N\left(u_N\right)\leq\\
\hspace*{-24pt}\leq 
c_0^{(+)}\!\!\int\limits_{U_1}\!\!\cdots\!\! \int\limits_{U_N}\!\!\!B\!\left(u_1,\ldots, u_N\right)
 d\Psi_1\!\left(u_1\right)\!%d\Psi_2\left(u_2\right)\cdots\\
 \cdots d\Psi_N\!\left(u_N\right)\!\! 
 \label{e32}
\end{multline}
для любого фиксированного набора $\Psi\hm=(\Psi_1, \ldots, \Psi_N)\hm\in \Gamma$. 
Но тогда из~(\ref{e32}) с~учетом~(\ref{e30}) получаем:
\begin{multline}
I(\Psi_1, \ldots, \Psi_N)={}\\
{}=
\fr{\int\nolimits_{U_1}\!\cdots\! \int\nolimits_{U_N}\!\!A\left(u_1, \ldots, u_N\right)\,
 d\Psi_1(u_1)\cdots d\Psi_N(u_N)}{
\int\nolimits_{U_1}\!\cdots\! \int\nolimits_{U_N}\!\!B\left(u_1, \ldots, u_N\right)\,
 d\Psi_1(u_1)
 \cdots d\Psi_N(u_N)}\leq{}\\
 {}\leq c_0^{(+)} 
 \label{e33}
\end{multline}
для любого фиксированного набора $(\Psi_1, \ldots\linebreak\ldots, \Psi_N)\hm\in \Gamma$.

Предположим теперь, что функция $B(u_1,\ldots, u_N)$ строго отрицательна:
\begin{equation}
B(u_1,\ldots, u_N)<0 \quad \left(u_1, \ldots, u_N\right)\in U\,. 
\label{e34}
\end{equation}
Тогда
\begin{multline}
\hspace*{-6pt}\int\limits_{U_1}\!\!\cdots\!\! \int\limits_{U_N}\!\!B\!\left(u_1,\ldots, u_N\right)\!
 d\Psi_1(u_1) \cdots d\Psi_N(u_N)<0 \!\!\!
 \label{e35}
\end{multline}
для любого фиксированного набора $(\Psi_1, \ldots\linebreak \ldots, \Psi_N)\hm\in \Gamma$.

Как и~ранее, будем исходить из неравенства~(\ref{e25}). 
При выполнении условий~(\ref{e34}) и~(\ref{e35}) характер неравенств~(\ref{e31}) 
и~(\ref{e32}) меняется на противоположный, но характер неравенства~(\ref{e33}) 
остается неизменным. Таким образом, для любой функции 
$B(u_1, u_2,\ldots, u_N)$, обладающей свойством строгой знакопостоянности, 
из условия~(\ref{e25}) следует выполнение неравенства~(\ref{e33}), 
которое совпадает с~(\ref{e26}). Первое утверждение леммы~1 доказано. 
Второе утверждение доказывается аналогично. Лемма~1 доказана.

\smallskip

\noindent
\textbf{Лемма 2.} \textit{Рассмотрим дроб\-но-ли\-ней\-ный интегральный функционал 
$I(\Psi_1, \Psi_2,\ldots, \Psi_N)$ вида}~(\ref{e11}), 
\textit{заданный на некотором множестве наборов вероятностных мер 
$\Psi\hm=(\Psi_1, \Psi_2,\ldots, \Psi_N)\hm\in \Gamma$. Предпо\-ложим, что на 
множестве~$\Gamma$ выполняется условие~$1$ из набора предварительных условий 
и~функция $B(u_1, u_2,\ldots, u_N)$ обладает свойством строгой знакопостоянности 
при всех $(u_1, u_2,\ldots, u_N)\hm\in U$. Тогда справедливы следующие утверждения}:
\begin{enumerate}[1.]
\item \textit{Если основная функция $C(u_1, u_2,\ldots, u_N)\hm=
{A(u_1, u_2,\ldots, u_N)}/{B(u_1, u_2,\ldots, u_N)}$ ограничена сверху, 
но не достигает своего максимального 
значения, то имеет место неравенство}:
\begin{multline}
I\left(\Psi_1, \Psi_2,\ldots, \Psi_N\right)<{}\\
{}< \sup\limits_{(u_1, u_2,\ldots, u_N)\in U}
 C\left(u_1, u_2,\ldots, u_N\right)<\infty \label{e36}
\end{multline}
\textit{для всех} $(\Psi_1, \Psi_2,\ldots, \Psi_N)\in \Gamma$.
\item \textit{Если основная функция $C(u_1, u_2,\ldots, u_N)\hm=
{A(u_1, u_2,\ldots, u_N)}/{B(u_1, u_2,\ldots, u_N)}$ ограничена снизу, 
но не достигает своего минимального значения, то имеет место неравенство}:
\begin{multline*}
I\left(\Psi_1, \Psi_2,\ldots, \Psi_N\right)>{}\\
{}> \inf\limits_{(u_1, u_2,\ldots, u_N)\in U} 
C\left(u_1, u_2,\ldots, u_N\right)>-\infty 
%\label{e37}
\end{multline*}
\textit{для всех} $(\Psi_1, \Psi_2,\ldots, \Psi_N)\hm\in \Gamma$.
\end{enumerate}

\noindent
Д\,о\,к\,а\,з\,а\,т\,е\,л\,ь\,с\,т\,в\,о\ \ леммы~2. 
Докажем первое утверждение леммы. Поскольку множество значений 
основной функции $C(u_1, u_2,\ldots, u_N)$ ограничено сверху, оно имеет конечную 
верхнюю грань:
$$
\exists \sup\limits_{(u_1, u_2,\ldots, u_N)\in U} 
C\left(u_1, u_2,\ldots, u_N\right)<\infty
$$
(см.~\cite[гл.~1, \S3, п.~3.4, теорема~1]{25}).

По условию функция $C(u_1, u_2,\ldots, u_N)$ не достигает своего максимального 
значения. Следовательно, выполняется неравенство:
\begin{multline}
C(u_1, u_2,\ldots, u_N)=\fr{A(u_1, u_2,\ldots, u_N)}{B(u_1, u_2,\ldots, u_N)}<{}\\
{}< 
\sup\limits_{(u_1, u_2,\ldots, u_N)\in U} C(u_1, u_2,\ldots, u_N)<\infty\,, 
\\
\left(u_1, u_2,\ldots, u_N\right)\in U\,.
\label{e38}
\end{multline}
Взяв за основу строгое неравенство~(\ref{e38}), проведем рассуждения, аналогичные тем, 
которые были проведены в~лемме~1 по отношению к~неравенству~(\ref{e25}). 
В~результате получим строгое неравенство~(\ref{e36}).

Второе утверждение леммы~2 доказывается аналогично. Лемма~2 доказана.

\noindent
Д\,о\,к\,а\,з\,а\,т\,е\,л\,ь\,с\,т\,в\,о\ 
\ теоремы~2.
Начнем с~доказательства утверждения~1. Предположим сначала, что основная 
функция $C(u_1, u_2,\ldots, u_N)={A(u_1, u_2,\ldots, u_N)}/{B(u_1, u_2,\ldots, u_N)}$ 
ограничена сверху и~достигает глобального максимума на множестве~$U$ 
в~некоторой точке $u^{(+)}\hm=\left(u^{(+)}_1,u^{(+)}_2,\ldots,u^{(+)}_N\right)\hm\in U$,
а~именно:
\begin{multline*}
\max\limits_{(u_1, u_2,\ldots, u_N)\in U} C\left(u_1, u_2,\ldots, u_N\right) = {}\\
{}=
C\left(u^{(+)}_1,u^{(+)}_2,\ldots,u^{(+)}_N\right)<\infty\,.
\end{multline*}
Тогда выполняется соотношение:
\begin{multline}
C(u_1, u_2,\ldots, u_N)=\fr{A(u_1, u_2,\ldots, u_N)}{B(u_1, u_2,\ldots, u_N)}
\leq{}\\
{}\leq C\left(u^{(+)}_1,u^{(+)}_2,\ldots,u^{(+)}_N\right)<\infty\,, 
\\
\left(u_1, u_2,\ldots, u_N\right)\in U\,.
\label{e39}
\end{multline}
Условия леммы~1 выполнены, и~можно воспользоваться ее утверждениями. 
Согласно первому из них, если выполняется неравенство~(\ref{e39}), 
то имеет место соотношение:
\begin{equation*}
I(\Psi_1, \Psi_2,\ldots, \Psi_N)\leq 
C\left(u^{(+)}_1,u^{(+)}_2,\ldots,u^{(+)}_N\right)<\infty 
%\label{e40}
\end{equation*}
для всех стратегий управления $\Psi\hm=(\Psi_1, \Psi_2,\ldots\linebreak
\ldots, \Psi_N)\hm\in \Gamma$.

Таким образом, множество значений дроб\-но-ли\-ней\-но\-го интегрального 
функционала $I(\Psi_1, \Psi_2,\ldots, \Psi_N)$ ограничено сверху при всех 
$\Psi\hm=(\Psi_1, \Psi_2,\ldots, \Psi_N)\hm\in \Gamma$. Тогда существует верхняя 
грань этого множества и~выполняется неравенство:
\begin{multline}
\sup\limits_{(\Psi_1, \Psi_2,\ldots, \Psi_N)\in \Gamma} 
I\left(\Psi_1, \Psi_2,\ldots, \Psi_N\right)\leq {}\\
{}\leq
C\left(u^{(+)}_1,u^{(+)}_2,\ldots,u^{(+)}_N\right). \label{e41}
\end{multline}
Рассмотрим детерминированную стратегию управ\-ле\-ния 
$\Psi^{*(+)}\hm=\left(\Psi_1^{*(+)}, \Psi_2^{*(+)},\ldots, \Psi_N^{*(+)}\right)$, 
в~которой каждая вероятностная мера~$\Psi_i^{*(+)}$ является вы\-рож\-ден\-ной 
и~сосредоточена в~точке $u_i^{(+)}$, $i\hm=\overline{1, N}$.
По свойству интеграла
\begin{multline}
I\left(\Psi_1^{*(+)}, \Psi_2^{*(+)},\ldots ,\Psi_N^{*(+)}\right)={}\\
{}=
C\left(u^{(+)}_1,u^{(+)}_2,\ldots,u^{(+)}_N\right). \label{e42}
\end{multline}
Из соотношений~(\ref{e41}) и~(\ref{e42}) получаем:
\begin{multline}
\sup\limits_{(\Psi_1, \Psi_2,\ldots, \Psi_N)\in \Gamma} 
I\left(\Psi_1, \Psi_2,\ldots, \Psi_N\right)\leq{}\\
{}\leq
 I\left(\Psi_1^{*(+)}, 
\Psi_2^{*(+)},\ldots, \Psi_N^{*(+)}\right). \label{e43}
\end{multline}
Заметим дополнительно, что выполняются отношения принадлежности:
\begin{equation}
\Psi^{*(+)}=\left(\Psi_1^{*(+)}, \Psi_2^{*(+)},\ldots, \Psi_N^{*(+)}\right) 
\in \Gamma^* \subset \Gamma\,. \label{e44}
\end{equation}
Из~(\ref{e44}) и~свойства верхней грани следует:
\begin{multline}
\sup\limits_{\left(\Psi_1^{*}, \Psi_2^{*},\ldots, \Psi_N^{*}\right) \in \Gamma^*} 
I\left(\Psi_1^{*}, \Psi_2^{*},\ldots, \Psi_N^{*}\right)\leq {}\\
{}\leq
\sup\limits_{\left(\Psi_1, \Psi_2,\ldots, \Psi_N\right) 
\in \Gamma} I\left(\Psi_1, \Psi_2,\ldots, \Psi_N\right)\,. 
\label{e45}
\end{multline}
Объединяя~(\ref{e42}), (\ref{e43}) и~(\ref{e45}), получаем соотношение:
\begin{multline}
\sup\limits_{\left(\Psi_1^{*}, \Psi_2^{*},\ldots, \Psi_N^{*}\right) 
\in \Gamma^*} I\left(\Psi_1^{*}, \Psi_2^{*},\ldots, 
\Psi_N^{*}\right)\leq{}\\
{}\leq \sup\limits_{\left(\Psi_1, \Psi_2,\ldots, \Psi_N\right) 
\in \Gamma} I\left(\Psi_1, \Psi_2,\ldots, \Psi_N\right)\leq{}\\
{}\leq I\left(\Psi_1^{*(+)}, \Psi_2^{*(+)},\ldots, \Psi_N^{*(+)}\right)={}\\
{}=
\fr{A\left(u^{(+)}_1,u^{(+)}_2,\ldots,u^{(+)}_N\right)}{B\left(u^{(+)}_1,u^{(+)}_2,
\ldots,u^{(+)}_N\right)}\,.
 \label{e46}
\end{multline}
Из соотношения~(\ref{e46}) с~учетом~(\ref{e44}) получаем, что максимум 
функционала $I(\Psi_1, \Psi_2,\ldots, \Psi_N)$ на множестве допустимых стратегий 
$\Psi\hm=(\Psi_1, \Psi_2,\ldots, \Psi_N)\hm\in \Gamma$ существует и~достигается 
на детерминированной стратегии $\left(\Psi_1^{*(+)}, \Psi_2^{*(+)},\ldots, 
\Psi_N^{*(+)}\right)$.

Кроме того, выполняются соотношения~(\ref{e19}). Таким образом, утверждение~1 
в~случае, когда основная функция $C(u_1, u_2,\ldots, u_N)$ достигает глобального 
максимума, доказано. Соответствующее утверждение в~случае, когда основная функция 
$C(u_1, u_2,\ldots, u_N)$ достигает глобального минимума, доказывается аналогично. 
При этом используется второе утверждение леммы~1.

\smallskip

Перейдем к~доказательству второго утверждения теоремы~2. Предположим, что основная 
функция $C(u_1, u_2,\ldots, u_N)\hm=A(u_1, u_2,\ldots$\linebreak
$\ldots, u_N)/{B(u_1, u_2,\ldots, u_N)}$ 
ограничена сверху, но не достигает глобального максимума на множестве 
$U \hm= U_1 \times U_2 \times \cdots \times U_N$. Тогда множество значений 
основной функции имеет конечную верхнюю грань:

\noindent
\begin{multline*}
C\left(u_1, u_2,\ldots, u_N\right)=\fr{A(u_1, u_2,\ldots, u_N)}
{B(u_1, u_2,\ldots, u_N)}<{}\\
{}<
\sup\limits_{(u_1, u_2,\ldots, u_N)\in U} \fr{A(u_1, u_2,\ldots, u_N)}
{B(u_1, u_2,\ldots, u_N)}<\infty\,, 
\\
\left(u_1, u_2,\ldots, u_N\right)\in U\,.
%\label{e47}
\end{multline*}
По определению верхней грани для любого фиксированного $\varepsilon \hm>0$ 
существует точка $(u_1^{(+)}(\varepsilon), u_2^{(+)}(\varepsilon),\ldots, 
u_N^{(+)}(\varepsilon))$ такая, что выполняется двойное неравенство~(\ref{e21}) 
(см.~\cite[гл.~1, \S\,3, п.~3.4]{25}). Иначе говоря, значение основной функции 
в~указанной точке лежит в~левой \mbox{$\varepsilon$-окрест}\-ности верхней грани. 
Рассмотрим детерминированную стратегию управления 
$\Psi^{*(+)}(\varepsilon)\hm=\!\left(\Psi_1^{*(+)}(\varepsilon), 
\Psi_2^{*(+)}(\varepsilon),\ldots, \Psi_N^{*(+)}(\varepsilon)\!\right)$, компонентами\linebreak 
которой являются вырожденные вероятностные меры $\Psi_1^{*(+)}(\varepsilon), 
\Psi_2^{*(+)}(\varepsilon),\ldots, \Psi_N^{*(+)}(\varepsilon)$, причем вырожденная 
мера~$\Psi_i^{*(+)}(\varepsilon)$ сосредоточена в~точке~$u_i^{(+)}(\varepsilon)$,
$i\hm=1,2,\ldots,N$.

По свойству интеграла
\begin{multline}
I\left(\Psi_1^{*(+)}(\varepsilon), \Psi_2^{*(+)}(\varepsilon),\ldots,
 \Psi_N^{*(+)}(\varepsilon)\right)={}\\
 {}=
 C\left(u_1^{(+)}(\varepsilon), u_2^{(+)}(\varepsilon),\ldots, 
 u_N^{(+)}(\varepsilon)\right)\,. 
 \label{e48}
\end{multline}
Из соотношения~(\ref{e48}) с~учетом указанного свойства основной функции получаем:
\begin{multline}
\sup\limits_{(u_1, u_2,\ldots, u_N)\in U} \fr{A(u_1, u_2,\ldots, u_N)}
{B(u_1, u_2,\ldots, u_N)}-\varepsilon<{}\\
{}< I\left(\Psi_1^{*(+)}(\varepsilon), 
\Psi_2^{*(+)}(\varepsilon),\ldots, \Psi_N^{*(+)}(\varepsilon)\right)<{}
\\
{}< \sup\limits_{(u_1, u_2,\ldots, u_N)\in U} \fr{A(u_1, u_2,\ldots, u_N)}
{B(u_1, u_2,\ldots, u_N)}<\infty\,. 
\label{e49}
\end{multline}
Заметим также, что в~рассматриваемом случае выполнены условия леммы~2. 
Воспользуемся первым утверждением этой леммы, а~именно соотношением~(\ref{e36}):
\begin{multline}
I(\Psi_1, \Psi_2,\ldots, \Psi_N)< {}\\
{}<\sup\limits_{(u_1, u_2,\ldots, u_N)
\in U} \fr{A(u_1, u_2,\ldots, u_N)}{B(u_1, u_2,\ldots, u_N)}<\infty 
\label{e50}
\end{multline}
для всех $(\Psi_1, \Psi_2,\ldots, \Psi_N)\in\Gamma$.

Из соотношений~(\ref{e49}) и~(\ref{e50}) следует, что детерминированная стратегия 
$\Psi^{*(+)}(\varepsilon)\hm=\left(\Psi_1^{*(+)}(\varepsilon), \Psi_2^{*(+)}(\varepsilon),
\ldots, \Psi_N^{*(+)}(\varepsilon)\right)$, опре\-де\-ля\-емая набором вырожденных 
вероятностных мер, сосредоточенных в~соответствующих точках 
$\left(u_1^{(+)}(\varepsilon), u_2^{(+)}(\varepsilon),\ldots, 
u_N^{(+)}(\varepsilon)\right)$, является $\varepsilon$-оп\-ти\-маль\-ной. 
Вторая часть утверждения~2 теоремы~2, связанная со свойствами нижней грани, 
доказывается аналогично.

Докажем третье утверждение теоремы~2. Предположим, что множество значений 
основной функции $C(u_1, u_2,\ldots, u_N)\hm=
A(u_1, u_2,\ldots$\linebreak $\ldots, u_N)/{B(u_1, u_2,\ldots, u_N)}$
не является ограниченным сверху на множестве $U\hm=U_1\times U_2 \times \cdots $\linebreak
$\cdots \times U_N$.
Тогда существует последовательность\linebreak точек $\left(u_1^{(+)}(n), u_2^{(+)}(n),
\ldots,u_N^{(+)}(n)\right)\hm\in U$, $n\hm=1,2,\ldots$, для которой
\begin{multline}
C\left(u_1^{(+)}(n), u_2^{(+)}(n),\ldots,u_N^{(+)}(n)\right)={}\\
{}=
\fr{A\left(u_1^{(+)}(n), u_2^{(+)}(n),\ldots,u_N^{(+)}(n)\right)}
{B\left(u_1^{(+)}(n), u_2^{(+)}(n),\ldots,u_N^{(+)}(n)\right)}
\longrightarrow \infty \,,\\
n\rightarrow \infty\,.
\label{e51}
\end{multline}
Зафиксируем некоторую последовательность точек $\left(u_1^{(+)}(n), u_2^{(+)}(n),
\ldots,u_N^{(+)}(n)\right)\hm\in U$, $n\hm=1,2,\ldots$, обладающих указанным свойством, 
и~рассмотрим последовательность детерминированных  стратегий управления 
$\Psi^{*(+)}(n)\hm=\left(\Psi_1^{*(+)}(n), \Psi_2^{*(+)}(n),\ldots, 
\Psi_N^{*(+)}(n)\right)$, $n\hm=1,2,\ldots$, определяемых набором вырожденных 
вероятностных мер, сосредоточенных в~соответствующих точках 
$\left(u_1^{(+)}(n), u_2^{(+)}(n),\ldots,u_N^{(+)}(n)\right)$, $n\hm=1,2,\ldots$ 
По свойству интеграла для любого фиксированного значения $n=1,2,\ldots$ 
выполняется равенство:
\begin{multline}
I \left(\Psi^{*(+)}(n)\right)={}\\
{}=I\left(\Psi_1^{*(+)}(n), \Psi_2^{*(+)}(n),\ldots,
 \Psi_N^{*(+)}(n)\right)={}\\
{}=\fr{A\left(u_1^{(+)}(n), u_2^{(+)}(n),\ldots,u_N^{(+)}(n)\right)}
{B\left(u_1^{(+)}(n), u_2^{(+)}(n),\ldots,u_N^{(+)}(n)\right)}\,. 
\label{e52}
\end{multline}
Из соотношений~(\ref{e51}) и~(\ref{e52}) следует, что
\begin{multline}
I\left(\Psi^{*(+)}(n)\right)={}\\
{}=I\left(\Psi_1^{*(+)}(n), \Psi_2^{*(+)}(n),\ldots, 
\Psi_N^{*(+)}(n)\right)\longrightarrow\infty\,,\\ 
n \rightarrow\infty\,.
 \label{e53}
\end{multline}
Соотношение~(\ref{e53}) означает, что множество значе\-ний дроб\-но-ли\-ней\-но\-го 
интегрального функциона\-ла $I(\Psi_1, \Psi_2,\ldots, \Psi_N)$ вида~(\ref{e11}) 
не ограничено сверху\linebreak на множестве наборов вырожденных вероятностных мер 
$\left(\Psi_1^{*(+)}(n), \Psi_2^{*(+)}(n),\ldots, \Psi_N^{*(+)}(n)\right)\hm\in\Gamma^*$, 
а~следовательно, и~на более широком\linebreak множестве наборов вероятностных 
мер $(\Psi_1, \Psi_2,\ldots$\linebreak $\ldots, \Psi_N)\hm\in\Gamma$. В~таком случае решения экстремальной 
задачи~(\ref{e18}) в~форме задачи на максимум не существует. Соответствующее утвержде\-ние 
для варианта, когда множество значений основной функции $C(u_1, u_2,\ldots,u_N)
\hm=A(u_1, u_2,\ldots$\linebreak $\ldots,u_N)/{B(u_1, u_2,\ldots,u_N)}$ 
не является ограниченным снизу, доказывается аналогично. Третье утверж\-де\-ние теоремы~2 
доказано. Тем самым тео\-ре\-ма~2 доказана полностью.

\smallskip

Применим теорему~2 для решения поставленной задачи оптимального управления. 
Из утверждения этой теоремы следует, что для доказательства су-\linebreak ществования 
оптимального управ\-ле\-ния и~его нахождения необходимо исследовать на 
глобальный экстремум основную функцию дроб\-но-ли\-ней\-но\-го интегрального 
функционала $C(u_1,u_2,\ldots,u_N)$, определяемую формулой~(\ref{e17}) с~учетом 
равенств~(\ref{e12})--(\ref{e16}). В~некоторых случаях, например когда основной 
процесс~$\xi(t)$ является регенерирующим, а~стоимостные характеристики 
модели задаются линейными функциями, такое исследование можно провести 
аналитически. Однако для подавляющего большинства полумарковских моделей 
для этого необходимо использовать численные методы.

\section{Заключение}

В заключительной части работы приведем \mbox{краткое} описание теоретической 
основы метода решения задачи оптимального управления полумарковским 
процессом с~конечным множеством состояний.

\begin{enumerate}[1.]
\item Исходная проблема оптимального управления формулируется в~виде 
экстремальной задачи~(\ref{e18}). Целевым показателем качества управ\-ле\-ния в~данной задаче 
служит величина~(\ref{e10}), которая имеет характер средней удельной прибыли.
\item Доказывается, что стационарный показатель~(\ref{e10}) представим в~виде 
дроб\-но-ли\-ней\-но\-го интегрального функционала~(\ref{e11}), для которого явно 
определяются подынтегральные функции числителя и~знаменателя, а~следовательно, 
и~основная функция данного функционала.
\item Используется теорема об экстремуме дроб\-но-ли\-ней\-но\-го интегрального 
функционала. На основании утверждений этой теоремы уста\-нав\-ли\-ва\-ет\-ся, что 
исходная задача оптимального управления сводится к~исследованию на глобальный 
экстремум основной функции этого функционала, для которой получено явное 
аналитическое представление.
\end{enumerate}

Заметим, что такое исследование задач оптимального управления 
стохастическими системами фактически уже было проведено в~ряде работ П.\,В.~Шнуркова 
и~его соавторов. В~частности, в~работе~\cite{26} была рассмотрена модель 
управления для обрывающегося процесса восстановления, описывающего функционирование 
некоторой технической системы. Задача управления решалась для различных показателей 
эффективности и~надежности этой системы, имеющих структуру дроб\-но-ли\-ней\-но\-го 
интегрального функционала.

В работах~\cite{27, 28} рассматривались модели регенерирующих процессов 
для исследования сис\-тем управления запасами. Различные показатели качества 
управления были представлены в~форме дроб\-но-ли\-ней\-ных интегральных функционалов. 
Основные функции этих функционалов были найде\-ны в~явной форме и~исследовались 
на глобальный экстремум. В~работах~\cite{21,29} рассматривалась достаточно 
сложная полумарковская модель с~конечным множеством состояний, описывающая 
сис\-те\-му управления запасом непрерывного продукта. Показатели качества управления в~этой 
модели также имели структуру дроб\-но-ли\-ней\-ных интегральных функционалов, 
для основных функций которых были найдены явные аналитические представления. 
Упомянем также работы~\cite{30, 31}, в~которых была исследована полумарковская 
модель с~дис\-крет\-но-не\-пре\-рыв\-ным фазовым пространством. Показатели 
качества управления в~этой  модели были найдены в~явной форме как функции от 
двух непрерывных параметров управления.

Фактически во всех упомянутых работах уже был использован метод решения задачи 
оптимального управления регенерирующим или полумарковским случайным процессом, 
основанный на исследовании экстремальных свойств основной функции соответствующего 
дроб\-но-ли\-ней\-но\-го интегрального функционала. Из соображений, изложенных 
во\linebreak введении, следует, что в~период написания и~пуб\-ли\-кации этих работ данный метод 
не имел стро\-гого обоснования. Однако после публикации\linebreak работы~\cite{14} и~настоящего 
исследования можно утверж\-дать, что полученные в~них результаты полностью теоретически 
обоснованы.

Таким образом, изложенный выше метод решения проблемы оптимального управления 
полумарковскими процессами с~конечными множествами состояний может быть успешно 
реализован для многих задач, рассматриваемых в~различных областях прикладной 
теории вероятностей.

Практическая реализация численной процедуры поиска оптимального решения на примере\linebreak 
полумарковской модели управления запасом непрерывного продукта (подробнее 
см.~\cite{21, 29}), ба\-зи\-ру\-юща\-яся на изложенных выше результатах (в~частности, 
теореме~1), была осуществлена А.\,К.~Горшениным и~соавторами 
в~статье~\cite{Gorshenin2015}. Коротко опишем наиболее важные аспекты этой работы.

Для решения поставленной задачи опти\-мального управления была создана 
специальная программа \verb"Inventory" на встроенном языке программирования 
пакета \verb"MATLAB", ее возможности\linebreak кратко представле\-ны в~упомянутой ранее 
\mbox{статье}~\cite{Gorshenin2015}. В~программе \verb"Inventory" реализованы функции 
для оценивания через заданные исходные параметры вероятностных и~стоимостных 
характеристик модели, которые в~дальнейшем используются для поиска значений 
основной функции дроб\-но-ли\-ней\-но\-го функционала~(\ref{e17}). Точка глобального 
экстремума этой функции и~определяет оптимальное управление.

В качестве начальных данных необходимо задание следующих параметров:
\begin{itemize}
\item спрос и~вместимость склада;
\item разбиение множества значений объема запаса;
\item вероятностные характеристики, описывающие модель пополнения запаса;
\item условные математические ожидания длительностей задержек пополнения запаса;
\item функции для характеризации затрат и~доходов.
\end{itemize}

По итогам работы программы \verb"Inventory" ряд вспомогательных функций 
представляется в~аналитической форме (в частности, с~использованием аппарата 
символьных вычислений  \verb"Symbolic Toolbox"\linebreak пакета \verb"MATLAB"), выводится 
точка глобального экстремума функции нескольких вещественных переменных~(\ref{e17}), 
найденная с~помощью применения численных и~при\-бли\-жен\-но-ана\-ли\-ти\-че\-ских\linebreak 
аппроксимаций. 
Также формируются графики оценок значений ве\-ро\-ят\-ност\-но-сто\-и\-мост\-ных 
характеристик 
и~основной функции дроб\-но-ли\-ней\-но\-го функционала~(\ref{e17}), либо трехмерных 
сечений в~случае наличия более трех параметров управления (переменных).

Функциональность пакета \verb"Inventory" может быть расширена для практической 
реализации метода решения задачи поиска оптимального управ\-ле\-ния полумарковскими 
процессами с~конечными множествами состояний, рассмотренного в~данной статье.


 {\small\frenchspacing
 {%\baselineskip=10.8pt
 \addcontentsline{toc}{section}{References}
 \begin{thebibliography}{99}
 \bibitem{1}
\Au{Ховард Р.} Динамическое программирование и~марковские процессы~/ 
Пер. с~англ.~--- М.: Сов. радио, 1964. 189~с.
(\Au{Howard~R.\,A.} Dynamic programming and Markov processes.~--- 
Cambridge, MA, USA: MIT Press, 1960. 136~p.)
\bibitem{2} 
\Au{Рыков В.\,В.} Управляемые марковские процессы с~конечными пространствами 
состояний и~управлений~// Теория вероятностей и~ее применения, 1966. Т.~11. 
Вып.~2. С.~343--351.
\bibitem{3} 
\Au{Джевелл В.} Управляемые полумарковские процессы~// Кибернетич. сборник.~--- 
М.: Мир, 1967. Вып.~4. С.~97--134.
%{\em Jewell W.\,S.} Markov-renewal programming~// Operations Research, 1963. Vol.~11. P.~938--971.
\bibitem{4} 
\Au{Fox B.} Markov renewal programming by linear fractional programming~// 
SIAM J.~Appl. Math., 1966. Vol.~14. P.~1418--1432.
\bibitem{5} 
\Au{Denardo E.\,V.} Contraction mappings in the theory underlying dinamic programming~// 
SIAM Rev., 1967. Vol.~9. P.~165--177.

\bibitem{6} 
\Au{Howard R.\,A.} Research in semi-Markovian decision structures~// 
J.~Oper. Res. Soc. Japan, 1963. Vol.~6. P.~163--199.
\bibitem{7} 
\Au{Osaki S., Mine H.} Linear programming algorithms for Markovian decision processes~//
 J.~Math. Anal. Appl., 1968. Vol.~22. P.~356--381.
\bibitem{8} 
\Au{Майн Х., Осаки С.} Марковские процессы принятия решений~/ Пер. с~англ.~--- 
М.: Наука, 1977. 176~с.
(\Au{Mine~H., Osaki~S.} 
Markovian decision processes.~--- New York, NY, USA: 
American Elsevier Publishing Co., 1970. 142~p.)
\bibitem{9} 
\Au{Гихман И.\,И., Скороход А.\,В.} Управляемые случайные процессы.~--- 
Киев: Наукова думка, 1977. 251~с.
\bibitem{10} 
\Au{Luque-Vasquez F., Herndndez-Lerma~О.} Semi-Markov control models with average costs~// 
Appl. Math., 1999. Vol.~26. No.\,3. P.~315--331.
\bibitem{11} 
\Au{Vega-Amaya O., Luque-Vasquez~F.} Sample-path average cost optimality for 
semi-Markov control processes on Borel spaces: Unbounded costs and mean holding times~// 
Appl. Math., 2000. Vol.~27. No.\,3. P.~343--367.
\bibitem{12} 
Вопросы математической теории надежности~/ Под ред. Б.\,В. Гнеденко.~--- 
М.: Радио и~связь, 1983. 376~с.
\bibitem{13} 
\Au{Барзилович Е.\,Ю., Каштанов~В.\,А.} Некоторые математические вопросы теории 
обслуживания сложных систем.~---  М.: Сов. радио, 1971. 272~с.
\bibitem{14} 
\Au{Шнурков П.\,В.} О~решении проблемы безусловного экстремума для 
дроб\-но-ли\-ней\-но\-го интегрального функционала на множестве вероятностных мер~// 
Докл. РАН. Сер. Математика, 2016. Т.~470. №\,4. C.~387--392.
\bibitem{15} 
\Au{Ширяев А.\,Н.}  Вероятность.~--- М.:~МЦНМО, 2011. Кн.~1. 552~с. Кн.~2. 968~с.
\bibitem{16} 
\Au{Боровков А.\,А.} Теория вероятностей.~--- М.: Либроком, 2009. 656~c.
\bibitem{17} 
\Au{Хеннекен П.\,Л., Тортра А.} Теория вероятностей 
и~некоторые ее приложения.~--- М.: Наука, 1974. 472~c.
\bibitem{18} 
\Au{Халмош П.} Теория меры~/ Пер. с~англ.~--- М.: ИЛ, 1953. 282~c.
(\Au{Halmos~P.} Measure theory.~--- Litton Educational Publishing, Inc. 1950. 304~p.)
\bibitem{19} 
\Au{Королюк В.\,С., Турбин~А.\,Ф.} Полумарковские процессы и~их приложения.~--- 
Киев:~Наукова думка, 1976. 184~с.
\bibitem{20} 
\Au{Janssen J., Manca R.} Applied semi-Markov processes.~--- New York,
NY, USA: Springer, 2006. 309~p.
\bibitem{21} 
\Au{Шнурков П.\,В., Иванов~А.\,В.} Анализ дискретной полумарковской модели
 управления запасом непрерывного продукта при периодическом прекращении потребления~// 
 Дискретная математика, 2014. Т.~26. Вып.~1. С.~143--154.
\bibitem{22} 
\Au{Иванов~А.\,В.} Анализ дискретной полумарковской модели
 управления запасом непрерывного продукта при периодическом прекращении 
 потребления.~--- М.: НИУ ВШЭ, 2014.  Дисс.\ \ldots\ канд. физ.-мат. наук. 120~с.
\bibitem{23}  %23
\Au{Bajalinov~E.\,B.} Linear-fractional programming. 
Theory, methods, applications and software.~--- 
Boston/\linebreak Dordrecht/London: Kluwer Academic Publs., 2003. 423~p.

\bibitem{27} %27
\Au{Шнурков П.\,В., Мельников~Р.\,В.} Оптимальное управление запасом 
непрерывного продукта в~модели регенерации~// Обозрение прикладной 
и~промышленной математики, 2006. Т.~13. Вып.~3. С.~434--452.
\bibitem{28} 
\Au{Шнурков П.\,В., Мельников~Р.\,В.} 
Исследование проб\-ле\-мы управления запасом непрерывного продукта при детерминированной 
задержке поставки~// Автоматика и~телемеханика, 2008. Т.~10. С.~93--113.


\bibitem{24}  %26
\Au{Шнурков П.\,В.} Методы исследования задач оптимального обслуживания 
в~математической теории надежности.~--- 
М.: МИЭМ, 1983.  Дисс.\ \ldots\ канд. физ.-мат. наук.

 \bibitem{25}  %25
\Au{Кудрявцев Л.\,Д.} Курс математического анализа. Т.~1.~--- 
М.: Дрофа, 2006. 704~с.

\bibitem{26} %24
\Au{Шнурков П.\,В.} Оптимальное обслуживание на периоде 
до первого отказа системы~// Применение аналитических методов в~вероятностных
 задачах.~--- Киев: Институт математики АН УССР, 1986. С.~121--129.

\bibitem{29} 
\Au{Шнурков П.\,В., Иванов~А.\,В.} Исследование задачи оптимизации в~дискретной 
полумарковской модели управления непрерывным запасом~// Вестник МГТУ им.\ 
Н.\,Э. Баумана. Сер.\ Естественные науки, 2013. Т.~3. Вып.~50. С.~62--87.
\bibitem{30} 
\Au{Shnourkoff P.\,V.} The two-element system with one 
restoring device optimum maintenance~// Stoch. Anal. Appl., 1997. 
Vol.~15. No.\,5. P.~823--837.
\bibitem{31} 
\Au{Shnourkoff P.\,V.} The two-element system optimum maintenance tills the first fail~// 
Stoch. Anal. Appl., 2001. Vol.~19. No.\,6. P.~1005--1024.
\bibitem{Gorshenin2015} 
\Au{Gorshenin~A.\,K., Belousov~V.\,V., Shnourkoff~P.\,V.,
Ivanov~A.\,V.} Numerical research of the optimal control problem in the semi-Markov 
inventory model~// AIP Conference Proceedings, 2015. Vol.~1648. {250007}. 4~p.
%\bibitem{33} {\em Горшенин А.\,К., Белоусов В.\,В., Шнурков П.\,В.} 2016. Система управления запасами на основе стохастических полумарковских моделей. Свидетельство о государственной регистрации программы для ЭВМ \textnumero 2016619021.
 \end{thebibliography}

 }
 }

\end{multicols}

\vspace*{-6pt}

\hfill{\small\textit{Поступила в~редакцию 15.07.16}}

%\vspace*{8pt}

\newpage

\vspace*{-24pt}

%\hrule

%\vspace*{2pt}

%\hrule

%\vspace*{8pt}


\def\tit{ANALYTICAL SOLUTION OF~THE~OPTIMAL CONTROL TASK OF~A~SEMI-MARKOV 
PROCESS WITH~FINITE SET OF~STATES}

\def\titkol{Analytical solution of~the~optimal control task of~a~semi-Markov 
process with~finite set of~states}

\def\aut{P.\,V.~Shnurkov$^{1}$, A.\,K.~Gorshenin$^{2}$, and~V.\,V.~Belousov$^{2}$}

\def\autkol{P.\,V.~Shnurkov, A.\,K.~Gorshenin, and~V.\,V.~Belousov}

\titel{\tit}{\aut}{\autkol}{\titkol}

\vspace*{-9pt}


    
\noindent
$^1$National Research University Higher School of Economics, 34~Tallinskaya Str., 
Moscow, 123458, Russian\linebreak
$\hphantom{^9}$Federation

\noindent
$^2$Institute of Informatics Problems, Federal Research Center 
``Computer Science and Control'' of the Russian\linebreak
$\hphantom{^9}$Academy of Sciences, 44-2~Vavilova Str., 
Moscow 119333, Russian Federation



\def\leftfootline{\small{\textbf{\thepage}
\hfill INFORMATIKA I EE PRIMENENIYA~--- INFORMATICS AND
APPLICATIONS\ \ \ 2016\ \ \ volume~10\ \ \ issue\ 4}
}%
 \def\rightfootline{\small{INFORMATIKA I EE PRIMENENIYA~---
INFORMATICS AND APPLICATIONS\ \ \ 2016\ \ \ volume~10\ \ \ issue\ 4
\hfill \textbf{\thepage}}}

\vspace*{3pt}


\Abste{The theoretical verification of the new method of finding 
the optimal strategy of control of a~semi-Markov process with finite set of states is 
presented. The paper considers Markov randomized strategies of control, determined by 
a~finite collection of probability measures, corresponding to each state. The quality 
characteristic is the stationary cost index. This index is a~linear-fractional integral 
functional, depending on collection of probability measures, giving the strategy of control. 
Explicit analytical forms of integrands of numerator and denominator of this 
linear-fractional integral functional are known. The basis of consequent results is 
the new generalized and strengthened form of the theorem about an extremum of 
a~linear-fractional integral functional. It is proved that problems of existence 
of an optimal control strategy of a~semi-Markov process and finding this strategy 
can be reduced to the task of numerical analysis of global extremum for 
the given function, depending on finite number of real arguments.}

\KWE{optimal control of a~semi-Markov process; stationary cost index of quality control; 
linear-fractional integral functional}




\DOI{10.14357/19922264160408} 

\vspace*{-16pt}

\Ack
\noindent
The research was partially supported by the Russian Foundation 
for Basic Research (project 15-07-05316).



%\vspace*{3pt}

  \begin{multicols}{2}

\renewcommand{\bibname}{\protect\rmfamily References}
%\renewcommand{\bibname}{\large\protect\rm References}

{\small\frenchspacing
 {%\baselineskip=10.8pt
 \addcontentsline{toc}{section}{References}
 \begin{thebibliography}{99}
\bibitem{1-1}
\Aue{Howard,~R.\,A.} 1960. \textit{Dynamic programming and Markov processes}. 
Cambridge, MA: MIT Press. 136~p.
\bibitem{2-1}
\Aue{Rykov,~V.\,V.} 1966. Upravlyaemye markovskie protsessy 
s~konechnymi prostranstvami sostoyaniy i~upravleniy 
[Controlled Markov processes with finite spaces of states and controls ]. 
\textit{Teoriya veroyatnostey i~ee primeneniya} 
[Theory of Probability and Its Applications] 11(2):343--351.
\bibitem{3-1}
\Aue{Jewell,~W.\,S.} 1963. Markov-renewal programming. 
\textit{Oper. Res.} 11:938--971.
\bibitem{4-1}
\Aue{Fox,~B.} 1966. Markov renewal programming by linear fractional programming. 
\textit{SIAM J.~Appl. Math.} 14:1418--1432.
\bibitem{5-1}
\Aue{Denardo, E.\,V.} 1967. Contraction mappings in the theory underlying dinamic 
programming. \textit{SIAM Rev.} 9:165--177.
\bibitem{6-1}
\Aue{Howard,~R.\,A.} 1963. Research in semi-Markovian decision structures. 
\textit{J.~Oper. Res. Soc. Japan} 6:163--199.
\bibitem{7-1}
\Aue{Osaki,~S., and H.~Mine.} 1968. Linear programming algorithms 
for Markovian decision processes. \textit{J.~Math. Anal. Appl.} 22:356--381.
\bibitem{8-1}
\Aue{Mine,~H., and S.~Osaki.} 1970. 
\textit{Markovian decision processes}. New York, NY: Elsevier. 142~p.
\bibitem{9-1}
\Aue{Gikhman,~I.\,I., and A.\,V.~Skorokhod.} 1977. 
\textit{Upravlyaemye sluchaynye protsessy} 
[Controlled random processes]. Kiev: Naukova Dumka. 251~p.
\bibitem{10-1}
\Aue{Luque-Vasquez,~F., and О.~Herndndez-Lerma.} 1999. 
Semi-Markov control models with average costs. \textit{Appl. Math.} 26(3):315--331.
\bibitem{11-1}
\Aue{Vega-Amaya,~O., and  F.~Luque-Vasquez.} 2000.  
Sample-path average cost optimality for semi-Markov control processes on Borel spaces: 
Unbounded costs and mean holding times. \textit{Appl. Math.} 27(3):343--367.
\bibitem{12-1}
Gnedenko,~B.~V., ed. 1983. 
\textit{Voprosy matematicheskoy teorii nadezhnosti} 
[Problems of the mathematical theory of reliability].  Moscow: Radio i~svyaz'. 376~p.
\bibitem{13-1}
\Aue{Barzilovich,~E.\,Yu., and V.\,A.~Kashtanov.} 1971. 
\textit{Nekotorye matematicheskie voprosy teorii obsluzhivaniya slozhnykh sistem}  
[Some mathematical questions in theory of complex systems maintenance]. 
Moscow: Sovetskoe radio. 272~p.
\bibitem{14-1}
\Aue{Shnurkov,~P.\,V.} 2016. Solution of the unconditional extremum problem for 
a~linear-fractional 
integral functional on a~set of probability measures. 
\textit{Dokl. Math.} 94(2):550--554.
\bibitem{15-1} %14
\Aue{Shiryaev,~A.\,N.} 2016. 
\textit{Probability-1}. Graduate texts in mathematics ser.
New York, NY: Springer. Vol.~95. 503~p.;
2017. \textit{Probability-2.} Vol.~900. 500~p.
\bibitem{16-1}
\Aue{Borovkov,~А.\,А.} 2009. 
\textit{Teoriya veroyatnostey} [Probability theory]. Moscow: Librokom. 656~p.
\bibitem{17-1}
\Aue{Khenneken,~P.\,L., and A.~Tortra.} 1974. 
\textit{Teoriya veroyatnostey i~nekotorye ee prilozheniya} 
[Probability theory and some of its applications]. Moscow: Nauka. 472~p.
\bibitem{18-1}
\Aue{Halmos,~P.} 1950. \textit{Measure theory}. Litton Educational Publishing. 304~p.
\bibitem{19-1}
\Aue{Korolyuk, V.\,S., and A.\,F.~Turbin.} 1976. 
\textit{Polumarkovskie protsessy i~ikh prilozheniya} 
[Semi-Markov processes and their applications]. Kiev: Naukova Dumka. 184~p.
\bibitem{20-1}
\Aue{Janssen,~J., and R.~Manca.} 2006. 
\textit{Applied semi-Markov processes}. New York, NY: Springer. 309~p.
\bibitem{21-1}
\Aue{Shnurkov,~P.\,V, and A.\,V~Ivanov.} 2015. Analysis of a~discrete semi-Markov model of continuous inventory 
control with periodic interruptions of consumption. 
\textit{Discrete Math. \mbox{Appl}.} 25(1):59--67.
\bibitem{22-1} %21
\Aue{Ivanov,~A.\,V.} 2014. Analiz diskretnoy polumarkovskoy modeli upravleniya 
zapasom nepreryvnogo produkta pri periodicheskom prekrashchenii potrebleniya 
[Analysis of a~discrete semi-Markov control model of continuous product inventory 
in a~periodic cessation of consumption].  
Moscow: Natsional'nyy Issledovatel'skiy Universitet ``Vysshaya Shkola Ekonomiki.'' 
PhD Thesis. 120~p.
\bibitem{23-1} %22
\Aue{Bajalinov,~E.\,B.} 2003. 
\textit{Linear-fractional programming. Theory, methods, applications and software}. 
Boston/\linebreak Dordrecht/London: Kluwer Academic Publs. 423~p.
\bibitem{26-1} %24
\Aue{Shnurkov,~P.\,V., and R.\,V.~Mel'nikov.} 2006. Optimal'noe upravlenie 
zapasom nepreryvnogo produkta v modeli regeneratsii [Optimal control of 
a~continuous product inventory in the regeneration model]. 
\textit{Obozrenie prikladnoy i~promyshlennoy matematiki} [Rev. Appl. Ind. Math.]
13(3):434--452.

\bibitem{25-1} %25
\Aue{Shnurkov,~P.\,V., and R.\,V.~Mel'nikov.} 2008. 
Analysis of the problem of continuous-product inventory control under deterministic 
lead time. \textit{Automat. Rem. Contr.} 69(10):1734--1751.

\columnbreak

\bibitem{24-1} %26
\Aue{Shnurkov,~P.\,V.} 1983. Metody issledovaniya zadach optimal'nogo obsluzhivaniya 
v~matematicheskoy teorii nadezhnosti [Research methods of optimal service problems 
in the mathematical theory of reliability].  
Moscow: Moskovskiy Institut Elektronnogo Mashinostroeniya.  PhD Thesis. 


\bibitem{27-1} %27
\Aue{Kudryavtsev,~L.\,D.} 2006. 
\textit{Kurs matematicheskogo analiza} 
[A~course of mathematical analysis]. Vol.~1. Moscow: Drofa. 704~p.

\bibitem{28-1}
\Aue{Shnurkov,~P.\,V.} 1986. Optimal'noe obsluzhivanie na periode do 
pervogo otkaza sistemy [The optimum service period until the first system failure]. 
\textit{Primenenie analiticheskikh metodov v~veroyatnostnykh zadachakh} 
[The application of analytical methods in probabilistic tasks]. Kiev:
Institute of Mathematics of the Academy of Sciences of the USSR. 121--129.

\bibitem{29-1}
\Aue{Shnurkov,~P.\,V., and A.\,V.~Ivanov.} 2013. Issledovanie zadachi optimizatsii 
v~diskretnoy polumarkovskoy modeli upravleniya nepreryvnym zapasom 
[Study of the optimization problem in discrete semi-Markov model of continuous 
inventory control]. \textit{Vestnik MGTU im.\ N.\,E.~Baumana. Ser. 
Estestvennye nauki} [Vestnik of MSTU named after N.\,E.~Bauman. Ser. Natural sciences] 
3(50):62--87.
\bibitem{30-1}
\Aue{Shnourkoff,~P.\,V.} 1997. The two-element system with one restoring device 
optimum maintenance.  \textit{Stoch. Anal. Appl.} 15(5):823--837.
\bibitem{31-1}
\Aue{Shnourkoff,~P.\,V.} 2001. The two-element system optimum maintenance tills 
the first fail. \textit{Stoch. Anal. Appl.} 19(6):1005--1024.
\bibitem{32-1}
\Aue{Gorshenin,~A.\,K., V.\,V.~Belousov, P.\,V.~Shnourkoff, and A.\,V.~Ivanov.}
2015. Numerical research of the optimal control problem in the semi-Markov 
inventory model. \textit{AIP Conference Proceedings} 1648:250007.
\end{thebibliography}

 }
 }

\end{multicols}

\vspace*{-3pt}

\hfill{\small\textit{Received July 15, 2016}}

\Contr

\noindent
\textbf{Shnurkov Peter V.} (b.\ 1953)~---
 Candidate of Science (PhD) in physics and mathematics, 
 associate professor, National Research University Higher School of Economics, 
 34~Tallinskaya Str., Moscow 123458, Russian Federation; \mbox{pshnurkov@hse.ru} 
 
 \vspace*{3pt}
 
 \noindent
\textbf{Gorshenin Andrey K.}  (b.\ 1986)~---
Candidate of Science (PhD) in physics and mathematics, leading scientist, 
Institute of Informatics Problems, Federal Research Center ``Computer Science 
and Control'' of the Russian Academy of Sciences, 44-2~Vavilov Str., Moscow 119333, 
Russian Federation; associate professor, Federal State Budget Educational 
Institution of Higher Education ``Moscow Technological University,'' 
78~Vernadskogo Ave., Moscow 119454, Russian Federation;
\mbox{agorshenin@frccsc.ru}

\vspace*{3pt}

\noindent
\textbf{Belousov Vasiliy V.} (b.\ 1977)~---
Candidate of Science (PhD) in technology, senior scientist, Institute of 
Informatics Problems, Federal Research Center ``Computer Science and Control'' 
of the Russian Academy of Sciences, 44-2~Vavilov Str., Moscow 119333, Russian 
Federation; \mbox{VBelousov@ipiran.ru}
\label{end\stat}


\renewcommand{\bibname}{\protect\rm Литература}    %3
\def\stat{gorshenin}

\def\tit{ЗАШУМЛЕНИЕ ДАННЫХ КОНЕЧНЫМИ СМЕСЯМИ НОРМАЛЬНЫХ 
И~ГАММА-РАСПРЕДЕЛЕНИЙ\\ С~ПРИМЕНЕНИЕМ К~ЗАДАЧЕ ОКРУГЛЕНИЯ НАБЛЮДЕНИЙ$^*$}

\def\titkol{Зашумление данных конечными смесями нормальных 
и~гамма-распределений с~применением к~задаче округления} % наблюдений}

\def\aut{А.\,К.~Горшенин$^1$}

\def\autkol{А.\,К.~Горшенин}

\titel{\tit}{\aut}{\autkol}{\titkol}

\index{Горшенин А.\,К.}
\index{Gorshenin A.\,K.}


{\renewcommand{\thefootnote}{\fnsymbol{footnote}} \footnotetext[1]
{Работа выполнена при поддержке РНФ (проект 18-71-00156).}}


\renewcommand{\thefootnote}{\arabic{footnote}}
\footnotetext[1]{Институт проблем информатики Федерального исследовательского центра 
<<Информатика и~управление>> Российской академии наук, \mbox{agorshenin@frccsc.ru}}

\vspace*{-12pt}




\Abst{Во многих реальных задачах проводится статистический анализ данных, 
содержащих дополнительные ошибки измерения, в~том числе в~виде округления, 
что в~ряде ситуаций может приводить к~достаточно существенным искажениям. 
В~настоящей статье для одной из возможных моделей округления получены оценки 
для неизвестного математического ожидания наблюдений в~предположении, что 
исходные данные дополнительно зашумлены с~по\-мощью случайных величин, 
име\-ющих распределения типа конечных смесей нормальных и~гам\-ма-за\-ко\-нов. 
Построены доверительные интервалы для неизвестного математического ожидания 
с~использованием уточненной оценки для дисперсии целой части случайной величины. 
Обсуждается алгоритм определения значения параметра для искусственного шума, 
добавление которого к~исходным данным способствует повышению качества работы 
метода скользящего разделения смесей.}

\KW{зашумленные данные; округленные наблюдения; конечные смеси нормальных 
распределений; конечные смеси гам\-ма-рас\-пре\-де\-ле\-ний; доверительные интервалы;  
метод скользящего разделения смесей}

\DOI{10.14357/19922264180304}
  
\vspace*{-4pt}


\vskip 10pt plus 9pt minus 6pt

\thispagestyle{headings}

\begin{multicols}{2}

\label{st\stat}


\section{Введение}

Во многих реальных задачах данные, являющиеся непрерывными по своей сути, 
регистрируются с~помощью инструментов, вносящих дополнительные ошибки 
измерения, в~том чис\-ле в~виде округления. Таким образом, статистический 
анализ проводится не для исходных, а для преобразованных некоторым 
случайным образом наблюдений, что в~ряде ситуаций может приводить к~достаточно
 существенным искажениям.

Для преодоления данной проблемы развивались различные подходы, в~том числе 
на основе смешанных моделей (см., например, статью~\cite{Wright2003}, в~которой 
различные компоненты  используются для пред\-став\-ле\-ния уровней округления). 
В~работе~\cite{Bai2009} приводятся результаты для моделей авторегрессии и~скользящего 
среднего для округленных данных, а~в~статье~\cite{Zhang2010} эти результаты 
развиваются и~исследуются их асимптотические свойства. 
В~статье~\cite{Zhao2012} исследован метод оценивания па\-ра\-мет\-ров конечных смесей 
вероятностных распределений (в~том чис\-ле, и~многомерных) 
на основе использования EM (expectation-maximization) 
алгоритма~\cite{Korolev2011-i} с~\mbox{целью} получения состоятельных 
и~асимптотически нормальных оценок.

В настоящей статье развиваются результаты для моделей округления, 
описанных в~работах~\cite{Ushakov2015,Ushakov2017a,Ushakov2017b}. 
В~их рамках будут получены оценки для неизве\-ст\-ного математического ожидания 
наблюдений в~предположении, что исходные данные зашумлены с~по\-мощью случайных 
величин, имеющих распределения типа конечных смесей нормальных и~гам\-ма-за\-ко\-нов. 
Это позволяет учесть большее количество случайных факторов, влия\-ющих на величину 
<<дополнительной>> ошибки. Также будут построены доверительные интервалы для 
неизвестного математического ожидания. Выражения для гам\-ма-рас\-пре\-де\-ле\-ний 
получены впервые. Также обсуждается алгоритм определения значения па\-ра\-мет\-ра для 
искусственного шума, добавление которого к~исходным данным способствует 
повышению качества работы метода скользящего разделения смесей~\cite{Gorshenin2016}.

\vspace*{-12pt}

\section{Предположения и~базовые отношения}

Для сокращения формулировок теорем в~сле\-ду\-ющих разделах сделаем ряд 
предположений, на которые будем ссылаться в~дальнейшем. Итак, пусть:
\begin{itemize}
\item[(A)] $X_1,X_2,\ldots$~--- независимые одинаково распределенные 
случайные величины с~неизвестным математическим ожиданием ${\sf E}_X\hm<+\infty$;
\item[(B)] $\varepsilon_1,\varepsilon_2,\ldots$~--- независимые одинаково 
распределенные случайные величины с~математическим ожиданием 
${\sf E}_\varepsilon\hm<+\infty$; %\label{B}
\item[(C)] $X_1,X_2,\ldots$ и~$\varepsilon_1,\varepsilon_2,\ldots$ 
являются независимыми;
\item[(D)] $Y_j=\left[X_j+\varepsilon_j+1/2\right]$ для всех $j\hm=1,2,\ldots$ 
представляют собой округление значения суммы случайных величин $X_j\hm+\varepsilon_j$ 
до ближайшего целого сверху (при этом запись~$[\cdot]$ соответствует целой 
части выражения).
\end{itemize}

В рамках данных предположений в~статье будут рассмотрены вопросы качества 
приближения неизвестного математического ожидания~${\sf E}_X$ для исходных данных 
в~ситуации, когда наблюдения для анализа получены с~аддитивной ошибкой c известными 
распределениями (см.\ предположение~(B)) и~дополнительно округляются до 
ближайшего целого (см.\ предположение~(D)).

Заметим, что в~силу усиленного закона больших чисел справедливы следующие выражения:
\begin{multline}
\fr{1}{n}\sum\limits_{j=1}^n Y_j\xrightarrow[n\to\infty]{\text{п.н.}}
{\sf E}_Y\equiv\mathbb{E}\left[X_1+\varepsilon_1+\fr{1}{2}\right]={}\\
{}=\mathbb{E}\left(X_j+\varepsilon_j+\fr{1}{2}\right)-\mathbb{E}
\left\{X_j+\varepsilon_j+\fr{1}{2}\right\}={}\\
{}={\sf E}_X+{\sf E}_\varepsilon+\fr{1}{2}-\mathbb{E}\left\{X_j+\varepsilon_j+\fr{1}{2}\right\}. 
\label{Law}
\end{multline}

Запись $\{\cdot\}$ в~формуле~\eqref{Law} соответствует дробной 
части выражения, а~п.н.\ обозначает сходимость в~смысле почти наверное.

Для доказательства результатов в~дальнейшем потребуется следующее 
представления для дробной части  абсолютно непрерывной случайной величины~$Z$ 
с~абсолютно  интегрируемой характеристической функцией~$\varphi_Z(t)$
 (см., например, Лемму~4 в~работе~\cite{Ushakov2017b}):
\begin{equation}
\label{Fract}
\mathbb{E}\{Z\}=\fr{1}{2}-\sum\limits_{n=1}^\infty 
\fr{\mathrm{Im}\left (\varphi_Z(2\pi n)\right)}{\pi n}\,.
\end{equation}

Через $\mathrm{Im}\,(\cdot)$ в~формуле~\eqref{Fract} обозначена мнимая часть 
соответствующей функции.

При построении доверительных интервалов в~дальнейшем будет 
использована следующая оценка, справедливая для любой случайной величины~$Z$:
\begin{equation}
\mathbb{D}[Z]\leqslant \left(\sqrt{\mathbb{D} Z}+\fr{1}{2}\right)^2.
\label{Var}
\end{equation}
Она может быть проверена непосредственно с~учетом представления 
$\mathbb{D} [Z]\hm=\mathbb{D}\left(Z\hm-\{Z\}\right)$, неравенства 
Ко\-ши--Бу\-ня\-ков\-ско\-го для ковариации и~соотношения 
 $\mathbb{D}\{Z\}\hm\leqslant 1/4$, справедливого для любой случайной величины~$Z$ 
 (см., например, статью~\cite{Ushakov2017b}). Отметим, что данная оценка 
 является более точной по сравнению с~использованным для аналогичных 
 целей в~работе~\cite{Ushakov2017b} соотношением 
 $\mathbb{D} [Z]\hm\leqslant 2\mathbb{D} Z\hm+1/2$. Действительно,
\begin{equation*}
2\mathbb{D} Z+\fr{1}{2}-\left(\sqrt{\mathbb{D} Z}+\fr{1}{2}\right)^2=
\left(\sqrt{\mathbb{D} Z}-\fr{1}{2}\right)^2\geqslant0\,,
\end{equation*}
причем для всех $\sqrt{\mathbb{D} Z}\hm\neq {1}/{2}$ 
данное неравенство является строгим.

\section{Конечные смеси нормальных законов}

Для случайной величины~$X$, имеющей распределение типа 
конечной смеси нормальных законов~\cite{Korolev2011-i} с~параметрами 
${\bf a}\hm=(a_1,\ldots, a_k)$, $a_j\hm\in \mathbb{R}$, 
$\boldsymbol{\sigma}\hm=(\sigma_1,\ldots, \sigma_k)$, 
$\sigma_j\hm>0$, ${\bf p}\hm=(p_1,\ldots, p_k)$, $p_j\hm\geqslant 0$, 
$\sum\nolimits_{j=1}^{k}p_j\hm=1$, плот\-ность которого задается выражением
\begin{equation}
f_X(x)=\sum\limits_{j=1}^{k}\fr{p_j}{\sigma_j\sqrt{2\pi}}\,e^{-(x-a_j)^2/(2\sigma_j^2)}\,,
\label{FinNormMixt}
\end{equation}
характеристическая функция имеет вид:
\begin{equation}
\varphi_X(t)=\int\limits_{-\infty}^{+\infty}\!\!e^{itx} f_X(x)\, dx = 
\sum\limits_{j=1}^{k}p_j e^{ita_j-\sigma_j^2 t^2/2}.
\label{ChiFinNormMixt}
\end{equation}

Абсолютная интегрируемость  $\varphi_X(t)$ вытекает из свойств 
характеристической функции нормального распределения. 
Заметим, что в~точке $t\hm=2\pi n$ выражение~\eqref{ChiFinNormMixt} принимает 
сле\-ду\-ющий вид:
\begin{equation}
\label{ChiFinNormMixt2npi}
\varphi_X(2\pi n)= \sum\limits_{j=1}^{k}p_j e^{-2\pi^2 \sigma_j^2 n^2}\,.
\end{equation}

Рассмотрим вопрос точности оценивания неизвестного математического ожидания~${\sf E}_X$ 
при до\-бав\-ле\-нии зашумления.

\smallskip

\noindent
\textbf{Теорема~1.}\ 
\textit{Пусть выполнены предположения}~(A)--(D), 
\textit{причем случайные величины~$\varepsilon_j$, $j\hm=1,2,\ldots$, 
имеют распределение типа конечной $k$-ком\-по\-нент\-ной смеси нормальных законов 
вида}~\eqref{FinNormMixt} \textit{с~па\-ра\-мет\-ра\-ми~${\bf a}$, $\boldsymbol{\sigma}$ 
и~${\bf p}$. Тогда}
\begin{equation}
\label{Th1Eq}
\left\lvert {\sf E}_Y-{\sf E}_X\right\rvert \leqslant 
A+\fr{1}{\pi}\left(1+\fr{1}{4\pi^2\sigma^2}\right)e^{-2\pi^2\sigma^2}\,, 
\end{equation}
\textit{где} $A=\max(|a_1|,\ldots,|a_k|)$, $\sigma\hm=\min(\sigma_1,\ldots,\sigma_k)$.

\smallskip


\noindent
Д\,о\,к\,а\,з\,а\,т\,е\,л\,ь\,с\,т\,в\,о\,.\ \
С~учетом пред\-став\-ле\-ний~\eqref{Law},~\eqref{Fract} и~\eqref{ChiFinNormMixt2npi}, 
ограниченности модуля характеристической функции, а~также не\-за\-ви\-си\-мости 
случайных величин~$X_j$ и~$\varepsilon_j$ имеем:
\begin{multline*}
\left\lvert {\sf E}_Y-{\sf E}_X\right\rvert =
\left\lvert {\sf E}_\varepsilon+\fr{1}{2}-\mathbb{E}\left\{X_j+
\varepsilon_j+\fr{1}{2}\right\}\right\rvert={}\\
{}=\left\lvert {\sf E}_\varepsilon+\sum\limits_{n=1}^\infty
\fr{\mathrm{Im} \left(\varphi_{X_j}(2\pi n)\varphi_{\varepsilon_j}(2\pi n)
\varphi_{1/2}(2\pi n)\right)}{\pi n}\right\rvert={}\\
=\left\lvert 
\vphantom{\fr{(-1)^n\sum\nolimits_{j=1}^{k}p_j e^{-2\pi^2 \sigma_j^2 n^2} 
\mathrm{Im} \left(\varphi_{X_j}(2\pi n)\right)}{\pi n}}
{\sf E}_\varepsilon+{}\right.\\
\left.{}+\sum\limits_{n=1}^\infty
\fr{\mathrm{Im} \left(\varphi_{X_j}(2\pi n) 
\sum\nolimits_{j=1}^{k}p_j e^{-2\pi^2 \sigma_j^2 n^2} 
e^{\pi n}\right)}{\pi n}\right\rvert={}\\
{}=\left\lvert 
\vphantom{\fr{(-1)^n\sum\nolimits_{j=1}^{k}p_j e^{-2\pi^2 \sigma_j^2 n^2} 
\mathrm{Im} \left(\varphi_{X_j}(2\pi n)\right)}{\pi n}}
{\sf E}_\varepsilon+{}\right.\\
\left.{}+\sum\limits_{n=1}^\infty
\fr{(-1)^n\sum\nolimits_{j=1}^{k}p_j e^{-2\pi^2 \sigma_j^2 n^2} 
\mathrm{Im} \left(\varphi_{X_j}(2\pi n)\right)}{\pi n}\right\rvert\leqslant{}\\
{}\leqslant \left\lvert {\sf E}_\varepsilon\right\rvert+\left\lvert\
\sum\limits_{j=1}^{k}p_j\sum\limits_{n=1}^\infty 
\fr{1}{\pi n} e^{-2\pi^2 \sigma_j^2 n^2}\right\rvert\leqslant {}\\
\\
{}\leqslant
\max\left(|a_1|,\ldots,|a_k|\right)+{}\\
{}+\sum\limits_{j=1}^{k} 
\fr{p_j}{\pi} \left(\!1+\fr{1}{4\pi^2\sigma_j^2}\!\right)\!e^{-2\pi^2\sigma_j^2}\leqslant{}\\
{}\leqslant
A+\fr{1}\pi\left(1+\fr{1}{4\pi^2\sigma^2}\right)e^{-2\pi^2\sigma^2}\,.
\end{multline*}

Справедливость использованной оценки 
\begin{equation*}
\sum\limits_{n=1}^\infty
\fr{e^{-2\pi^2 \sigma_j^2 n^2}}{n}\leqslant 
\left(1+\fr{1}{4\pi^2\sigma_j^2}\right)e^{-2\pi^2\sigma_j^2}
\end{equation*}
может быть проверена непосредственно (например, см.\ доказательство Теоремы~6 
в~статье~\cite{Ushakov2017b}).~\hfill$\square$

\smallskip

\noindent
\textbf{Замечание~1.}
В~случае, если зашумление производится нормально распределенными случайными 
величинами c нулевыми средними (т.\,е.\ в~формуле~\eqref{Th1Eq} необходимо считать 
$A\hm=0$, $k\hm=1$), то Тео\-ре\-ма~1 совпадает с~результатом, 
полученным в~работе~\cite{Ushakov2017b}.


\smallskip

Рассмотрим вопросы построения доверительного интервала для неизвестного 
математического ожидания~${\sf E}_X$ в~предположении, что случайные величины~$X_j$ не 
содержат ошибок измерения, а~все погрешности учтены исключительно в~за\-шум\-ля\-ющих 
элементах~$\varepsilon_j$.

\smallskip

\noindent
\textbf{Теорема~2.}\ 
\textit{Пусть выполнены предположения}~(A)--(D), 
\textit{причем случайные величины~$\varepsilon_j$, $j\hm=1,2,\ldots$, имеют 
распределение типа конечной $k$-ком\-по\-нент\-ной смеси нормальных законов 
вида}~\eqref{FinNormMixt} \textit{с~параметрами~${\bf a}$, $\boldsymbol{\sigma}$ 
и~${\bf p}$, а~случайные величины} $X_j\stackrel{\text{п.н.}}{=}{\sf E}_X$, $j\hm=1,2,\ldots$ 
\textit{Тогда доверительный интервал для~${\sf E}_X$ при условии $0\hm<\alpha\hm<1$ имеет вид}:
\begin{equation} 
\label{Th2Eq}
\hat{{\sf E}}_X - f({\bf a},\boldsymbol{\sigma},\alpha,n) 
\leqslant {\sf E}_X \leqslant  \hat{{\sf E}}_X + f({\bf a},\boldsymbol{\sigma},\alpha,n),
\end{equation}
\textit{где}

\vspace*{-2pt}

\noindent
\begin{align}
\hat{{\sf E}}_X&=\fr{1}{n} \sum\limits_{j=1}^{n} Y_j\,; \label{Th2hatE}\\
f({\bf a},\boldsymbol{\sigma},\alpha,n)&=
\fr{z_{1-{\alpha}/2}}{\sqrt{n}} \left(\sqrt{A^2+\Sigma^2}+\fr{1}{2}\right) +{}\notag\\
&{}+A+\fr{1}\pi\left(1+\fr{1}{4\pi^2\sigma^2}\right)e^{-2\pi^2\sigma^2}\,;
  \label{Th2f}
\end{align}
\textit{$z_{1-{\alpha}/2}$~--- $\left(1-{\alpha}/2\right)$-кван\-тиль 
стандартного нормального распределения; $A\hm=\max(|a_1|,\ldots,|a_k|)$; 
$\Sigma\hm=\max(\sigma_1,\ldots,\sigma_k)$; $\sigma\hm=\min(\sigma_1,\ldots,\sigma_k)$}. 


\smallskip

\noindent
\noindent
Д\,о\,к\,а\,з\,а\,т\,е\,л\,ь\,с\,т\,в\,о\,.\ \
Из центральной предельной тео\-ре\-мы с~учетом условия~(A) следует, 
что величина~$\hat{{\sf E}}_X$~\eqref{Th2hatE} асимптотически нормальна с~математическим 
ожиданием 
\begin{equation}
{\sf E}_Y\equiv \mathbb{E}\left[{\sf E}_X+\varepsilon_1+\fr{1}{2}\right] \label{EY}
\end{equation}
и дисперсией
\begin{equation}
\fr{1}{n} {\sf D}_Y\equiv \fr{1}{n}\mathbb{D}\left[{\sf E}_X+\varepsilon_1+
\fr{1}{2}\right]. \label{DY}
\end{equation}

Воспользовавшись оценкой~\eqref{Var}, получим:

\vspace*{-2pt}

\noindent
\begin{multline*}
{\sf D}_Y \leqslant  \left(\sqrt{\mathbb{D} \left({\sf E}_X+\varepsilon_1+\fr{1}{2}\right)}+
\fr{1}{2}\right)^2={}\\
{}=
\left(\sqrt{\mathbb{D}\varepsilon_1}+\fr{1}{2}\right)^2= {}\\
{}= \left(\sqrt{\sum\limits_{j=1}^{k}p_j\left(\left(a_j-\sum\limits_{t=1}^{k}
p_t a_t\right)^2+\sigma_j^2\right)}+\fr{1}{2}\right)^2\leqslant {}\\ 
{}\leqslant \left(\sqrt{A^2+\Sigma^2}+\fr{1}{2}\right)^2\,.
\end{multline*}
Тогда доверительный интервал уровня $1\hm-\alpha$ для математического ожидания~${\sf E}_Y$ 
имеет вид:
\begin{equation*}
\mathbb{P}\left(\left\lvert \hat{{\sf E}}_X-{\sf E}_Y\right\rvert \leqslant 
\fr{z_{1-{\alpha}/2}}{\sqrt{n}} 
\left(\sqrt{A^2+\Sigma^2}+\fr{1}{2}\right)\right)\geqslant 1-\alpha\,.
\end{equation*}

\begin{table*}[b]\small
\begin{center}

\begin{tabular}{|c|c|c|c|c|c|c|c|}
\multicolumn{7}{p{100mm}}{Численные решения уравнений~\eqref{f1} и~\eqref{f2} относительно 
параметра~$\sigma$ для некоторых значений~$n$ и~$\alpha$}\\
\multicolumn{7}{c}{\ }\\[-6pt]
\hline
\multicolumn{1}{|c|}{Размер}  & \multicolumn{2}{c|}{Уровень $\alpha=0{,}1$}& 
\multicolumn{2}{c|}{Уровень $\alpha=0{,}05$}& 
\multicolumn{2}{c|}{Уровень $\alpha=0{,}01$}\\
\cline{2-7}
\multicolumn{1}{|c|}{выборки $n$}&$\sigma_1$&$\sigma_2$&$\sigma_1$&$\sigma_2$&$\sigma_1$&$\sigma_2$\\
\hline
$\hphantom{000}100$&$0{,}4302$&$0{,}435$&$0{,}419$&$0{,}425$&$0{,}4002$&$0{,}408$\\
%\hline
$\hphantom{000}200$&$0{,}452$&$0{,}455$ &$0{,}441$&$0{,}445$&$0{,}424$&$0{,}429$\\
%\hline
$\hphantom{00}1000$&$0{,}499$&$0{,}499$ &$0{,}489$&$0{,}489$&$0{,}473$&$0{,}475$\\
%\hline
$\hphantom{0}10000$&$0{,}558$&$0{,}556$ &$0{,}549$&$0{,}547$&$0{,}536$&$0{,}534$\\
%\hline
$100000$&$0{,}611$&$0{,}607$ &$0{,}603$&$0{,}599$&$0{,}591$&$0{,}588$\\
\hline
\end{tabular}
\end{center}
\end{table*}


\noindent
Откуда следует справедливость соотношения~\eqref{Th2Eq} c~уче\-том 
очевидного неравенства

\pagebreak

\noindent
\begin{equation*}
\left\lvert \hat{{\sf E}}_X-{\sf E}_X\right\rvert \leqslant 
\left\lvert \hat{{\sf E}}_X-{\sf E}_Y\right\rvert +\left\lvert {\sf E}_Y-{\sf E}_X\right\rvert 
\end{equation*}
и оценки~\eqref{Th1Eq} из Теоремы~1.~\hfill$\square$

\smallskip

\noindent
\textbf{Замечание~2.}
В~работе~\cite{Gorshenin2016} было продемонстрировано повышение точ\-ности 
работы метода скользящего разделения конечных нормальных смесей за счет 
введения дополнительной компоненты, имеющей нормальное 
распределение $\mathcal{N}(0,\sigma^2)$ с~математическим ожиданием, равным~$0$, 
и~стандартным отклонением~$\sigma$. При этом была отмечена сложность выбора 
параметра~$\sigma$ для сохранения структуры выборки, близкой к~исходной. 
Результат Теоремы~2 может быть использован с~данной целью, если положить $k\hm=1$, 
$a_j\hm=0$ для всех $j\hm=1,2,\ldots$ и~выбирать величину~$\sigma$ как 
минимизирующую длину доверительного интервала~\eqref{Th2Eq}. Для 
этого необходимо найти производную функции $f(0,\sigma,\alpha,n)$~\eqref{Th2f} 
и~численно решить уравнение
\begin{multline}
f_\sigma'(0,\sigma,\alpha,n)\equiv \fr{z_{1-{\alpha}/2}}{\sqrt{n}} - {}\\
{}-
e^{-2\pi^2\sigma^2}\left(4\pi\sigma+\fr{1}{2\pi^3\sigma^3}+
\fr{1}{\pi\sigma}\right)=0
\label{f1}
\end{multline}
относительно неизвестного параметра~$\sigma$ при выбранных значениях величин~$n$ 
и~$\alpha$. В~качестве альтернативы можно использовать вид доверительного интервала 
из статьи~\cite{Ushakov2017b}, полученный с~помощью неравенства $\mathbb{D} [Z]
\hm\leqslant 2\mathbb{D} Z\hm+{1}/{2}$, и~искать решение уравнения вида:
\begin{multline}
\hspace*{-2.90578pt}\fr{2\sigma z_{1-{\alpha}/2}}{\sqrt{n (2\sigma^2+{1}/{2})}} -
 e^{-2\pi^2\sigma^2}\left(4\pi\sigma+\fr{1}{2\pi^3\sigma^3}+
 \fr{1}{\pi\sigma}\right)={}\\
 {}=0\,.\label{f2}
\end{multline}

Примеры найденных значений~$\sigma$ для типичных размеров выборок в~методе 
скользящего разделения смесей (учитываются как возможная ширина окна, 
так и~общее количество наблюдений в~анализируемом ряде) приведены в~таблице 
(использован метод оптимизации \verb"Trust-Region Dogleg" пакета \verb"MATLAB" 
c~настройками по умолчанию), в~которой через~$\sigma_1$ обозначено приближенное  
решение уравнения~\eqref{f1}, a~через $\sigma_2$~--- уравнения~\eqref{f2}.


Проверка практической эффективности данного подхода в~качестве 
критерия выбора параметров зашумляющего распределения для повышения 
точности работы метода скользящего разделения смесей может быть отмечена 
как задача для дальнейших исследований.


\section{Конечные смеси гамма-распределений}

Для случайной величины~$X$, имеющей распределение типа конечной смеси 
гам\-ма-рас\-пре\-де\-ле\-ний с~параметрами ${\bf r}\hm=(r_1,\ldots, r_k)$,
 $r_j\hm>0$, $\boldsymbol{\lambda}\hm=(\lambda_1,\ldots, \lambda_k)$, $\lambda_j\hm>0$, 
 ${\bf p}\hm=(p_1,\ldots, p_k)$, $p_j\hm\geqslant 0$, $\sum\nolimits_{j=1}^{k}p_j\hm=1$, 
 плот\-ность которого задается выражением
\begin{equation}
f_X(x)=\sum\limits_{j=1}^{k}p_j\fr{\lambda_j^{r_j} e^{-\lambda_j x}}
{\Gamma(r_j)}\,x^{r_j-1}\,,
\label{FinGammaMixt}
\end{equation}
характеристическая функция имеет следующий вид:
%характеристическая функция задается следующим выражением:
\begin{equation}
\varphi_X(t)=\!\int\limits_{-\infty}^{+\infty}\!\!\!e^{itx} f_X(x)\, dx = \!
\sum\limits_{j=1}^{k}p_j \left(\!1-\fr{it}{\lambda_j}\right)^{-r_j}\!.\!
\label{ChiFinGammaMixt}
\end{equation}

Отметим, что подобные модели зашумления разумно использовать в~случае, 
если известно, что данные сосредоточены на положительной полуоси, например 
при анализе различных информационных потоков (см., в~част\-ности, 
 работу~\cite{Gorshenin2013}). 

Проверим абсолютную интегрируемость функции $\varphi_X(t)$~\eqref{ChiFinGammaMixt}. 
Имеем:
\begin{multline*}
\int\limits_{-\infty}^{+\infty}\left\lvert\varphi_X(t)\right\rvert\, dt\leqslant 
\sum\limits_{j=1}^{k}p_j \int\limits_{-\infty}^{+\infty}\left\lvert \left(
1-\fr{it}{\lambda_j}\right)^{-r_j}\right\rvert \, dt={}\\
{}=\sum\limits_{j=1}^{k}p_j \int\limits_{-\infty}^{+\infty} \left\lvert\left(
\fr{\lambda_j(\lambda_j+it)}{\lambda_j^2+t^2}\right)^{r_j}\right\rvert\, dt \leqslant{}\\
{}\leqslant\sum\limits_{j=1}^{k}p_j \lambda_j \int\limits_{-\infty}^{+\infty}\left(
1+y^2\right)^{-{r_j}/{2}}\, dy\,.
\end{multline*}

Подынтегральное выражение при $r_j\hm\geqslant 2$ может быть оценено сверху 
функцией $1/({1+y^2})$, при этом соответствующий интеграл равен~$\pi$, что влечет 
абсолютную интегрируемость характеристической функции для конечной смеси 
гам\-ма-рас\-пре\-де\-ле\-ний. Поэтому в~дальнейшем будем предполагать,
 что $r_j\hm\geqslant 2$ для всех возможных значений $j\hm=1,2,\ldots$

Рассмотрим вопрос точ\-ности оценивания неизвестного математического ожидания ${\sf E}_X\hm>0$ 
при добавлении зашумления.

\smallskip

\noindent
\textbf{Теорема~3.}
\textit{Пусть выполнены предположения}~(A)--(D), 
\textit{причем случайные величины~$\varepsilon_j$, $j\hm=1,2,\ldots$, имеют 
распределение типа конечной $k$-ком\-по\-нент\-ной смеси 
гам\-ма-рас\-пре\-де\-ле\-ний вида}~\eqref{FinGammaMixt} 
\textit{с~па\-ра\-мет\-ра\-ми~${\bf r}$, $\boldsymbol{\lambda}$ и~${\bf p}$. Тогда}
\begin{equation}
\label{Th3Eq}
\left\lvert {\sf E}_Y-{\sf E}_X\right\rvert \leqslant \fr{R}{\lambda}+
\fr{\Lambda^{R}}{2^{r}\pi^{r+1}}\left(1+\frac1{r}\right)\,,
\end{equation}
\textit{где} $r=\min(r_1, \ldots,r_k)$; $R\hm=\max(r_1, \ldots,r_k)$; 
$\lambda\hm=\max(\lambda_1, \ldots,\lambda_k)$; 
$\Lambda\hm=\max(\lambda_1, \ldots,\lambda_k)$.

\smallskip

\noindent
Д\,о\,к\,а\,з\,а\,т\,е\,л\,ь\,с\,т\,в\,о\,.\ \
С~учетом пред\-став\-ле\-ний~\eqref{Law} и~\eqref{Fract}, ограниченности 
модуля характеристической функции, перехода от тригонометрической к~показательной 
записи комплексных чисел, а~также независимости случайных величин~$X_j$ 
и~$\varepsilon_j$ \mbox{имеем}:
\begin{multline*}
\left\lvert {\sf E}_Y-{\sf E}_X\right\rvert
\leqslant \left\lvert {\sf E}_\varepsilon\right\rvert+ {}\\
{}+\left\lvert\sum\limits_{n=1}^\infty
\left(
(-1)^n\mathrm{Im} \left(\sum\limits_{j=1}^{k}p_j \varphi_{X_j}(2\pi n)\left(
\vphantom{\fr{2\pi n}{\lambda_j}}
1-{}\right.\right.\right.\right.\\
\left.\left.\left.\left.{}-i\left(\fr{2\pi n}{\lambda_j}\right)\right)^{-r_j}\right)
\Bigg/ ({\pi n})
\vphantom{\sum\limits_{j=1}^{k}}
\right)\right\rvert={}\\
{}=\left\lvert {\sf E}_\varepsilon\right\rvert+ 
\left\lvert\sum\limits_{n=1}^\infty
\left(\!(-1)^n\mathrm{Im} \!\left(\sum\limits_{j=1}^{k}p_j \left(\!
1+\fr{4\pi^2 n^2}{\lambda_j^2}\right)^{- {r_j}/2}\!\times{}\right.\right.\right.\hspace*{-2.8663pt}\\
\left.\left.\left.{}\times \varphi_{X_j}(2\pi n)\,
e^{-ir_j\mathrm{arctan}\,({{t}/{\lambda_j}})}\right)
\Bigg/
({\pi n})
\vphantom{\left(
1+\fr{4\pi^2 n^2}{\lambda_j^2}\right)^{- {r_j}/2}}
\right)\right\rvert\leqslant{}\\
{}\leqslant \left\lvert {\sf E}_\varepsilon\right\rvert+\sum\limits_{j=1}^{k}
p_j\sum\limits_{n=1}^\infty\fr{1}{\pi n}\left(
1+\fr{4\pi^2 n^2}{\lambda_j^2}\right)^{-{r_j}/2}\leqslant{}\\
{}\leqslant  \fr{R}\lambda + \sum\limits_{j=1}^{k}p_j
\sum\limits_{n=1}^\infty\left(\fr{1}{\pi n}\,
\fr{\lambda_j^{r_j}}{(2\pi)^{r_j} n^{r_j}}\right)\leqslant {}
\\
{}\leqslant  \fr{R}{\lambda} + \sum\limits_{j=1}^{k}p_j 
\fr{\lambda_j^{r_j}}{2^{r_j}\pi^{r_j+1}}\left(1+\int\limits_{1}^{\infty}
\fr{1}{ x^{r_j+1}}\,dx\right)
\leqslant{}\\
{}\leqslant \fr{R}{\lambda}+\fr{\Lambda^{R}}{2^{r}\pi^{r+1}}\left(1+\fr{1}{r}\right).
\end{multline*}

При переходе от суммы к~интегралу используется факт убывания функции как переменной~$n$ 
(или~$x$).~\hfill$\square$


\smallskip

\noindent
\textbf{Замечание~3.}\
Теорема~3 описывает соответ\-ст\-ву\-ющий результат для гам\-ма-рас\-пре\-де\-лен\-ных 
за\-шум\-ля\-ющих случайных величин, если положить $k\hm=1$ в~выражении~\eqref{Th3Eq}. 
При этом, очевидно, $r\hm\equiv R$ и~$\lambda\hm\equiv \Lambda$.


\smallskip

Рассмотрим вопросы построения доверительного интервала для неизвестного 
математического ожидания ${\sf E}_X\hm>0$ в~предположении, что случайные величины~$X_j$ 
не содержат ошибок измерения, а все погрешности учтены исключительно в~за\-шум\-ля\-ющих 
элементах~$\varepsilon_j$.

\smallskip

\noindent
\textbf{Теорема~4.}
\textit{Пусть выполнены предположения}~(A)--(D), 
\textit{причем случайные величины~$\varepsilon_j$, $j\hm=1,2,\ldots$, имеют 
распределение типа конечной $k$-ком\-по\-нент\-ной смеси 
гам\-ма-рас\-пре\-де\-ле\-ний вида}~\eqref{FinGammaMixt} 
\textit{с~па\-ра\-мет\-ра\-ми~${\bf r}$, $\boldsymbol{\lambda}$ и~${\bf p}$, 
а~случайные величины} $X_j\stackrel{\text{п.н.}}{=}{\sf E}_X$, $j=1,2,\ldots$ 
\textit{Тогда доверительный интервал для~${\sf E}_X$ при условии $0\hm<\alpha\hm<1$ имеет вид}:
\begin{equation} 
\label{Th4Eq}
\left\lvert {\sf E}_X - \hat{{\sf E}}_X\right\rvert \leqslant  
f({\bf r},\boldsymbol{\lambda},\alpha,n),
\end{equation}
\textit{где}

\vspace*{-9pt}

\noindent
\begin{align}
\hat{{\sf E}}_X&=\fr{1}{n} \sum\limits_{j=1}^{n} Y_j\,; \label{Th4hatE}\\[-4pt]
f({\bf r}, \boldsymbol{\lambda},\alpha,n)&=\fr{z_{1-{\alpha}/2}}{\sqrt{n}} \left(
\sqrt{\fr{R(R+1)}{\lambda^2}-\fr{r^2}{\Lambda^2}}+\fr{1}{2}\right) +{}\notag\\[-1pt]
&\hspace*{20mm}{}+
\fr{R}{\lambda}+\fr{\Lambda^{R}}{2^{r}\pi^{r+1}}\left(1+\fr{1}{r}\right); \notag
\end{align}
\textit{$z_{1-{\alpha}/2}$~--- $\left(1-{\alpha}/2\right)$-кван\-тиль 
стандартного нормального распределения; $r\hm=\min(r_1, \ldots,r_k)$; 
$R\hm=\max(r_1, \ldots,r_k)$; $\lambda\hm=\max(\lambda_1, \ldots,\lambda_k)$; 
$\Lambda\hm=\max(\lambda_1, \ldots,\lambda_k)$}. 

\smallskip

\noindent
Д\,о\,к\,а\,з\,а\,т\,е\,л\,ь\,с\,т\,в\,о\,.\ \
Из центральной предельной теоремы с~учетом условия~(A) 
следует, что величина~$\hat{{\sf E}}_X$~\eqref{Th4hatE} асимптотически нормальна 
с~математическим ожиданием~${\sf E}_Y$~\eqref{EY} и~дисперсией $(1/n){\sf D}_Y$~\eqref{DY}. 
Пользуясь определением и~свойствами гам\-ма-функ\-ции, а~также оценкой~\eqref{Var} 
получим:

\noindent
\begin{multline*}
{\sf D}_Y \leqslant \left(\sqrt{\sum\limits_{j=1}^k p_j
\fr{\lambda_j^{r_j}}{\Gamma(r_j)} \int\limits_{0}^{+\infty} 
e^{\lambda_j x}x^{r_j+1}\, dx}+\fr{1}{2}\right)^2= {}\\[-0.5pt]
= \left(\sqrt{\sum\limits_{j=1}^{k}p_j
\fr{r_j(r_j+1)}{\lambda_j^2}-\left(\sum\limits_{j=1}^{k}p_j
\fr{r_j}{\lambda_j}\right) ^2}+\fr{1}{2}\right)^2\leqslant {}\\[-1.5pt]
{}\leqslant \left(\sqrt{\fr{R(R+1)}{\lambda^2}-\fr{r^2}{\Lambda^2}}+\fr{1}{2}\right)^2\,.
\end{multline*}

Аналогично доказательству Тео\-ре\-мы~2 с~учетом оценки~\eqref{Th3Eq} 
отсюда следует справедливость соотношения~\eqref{Th4Eq}.~\hfill$\square$

\vspace*{-12pt}

\section{Заключение}

Итак, в~работе получены оценки для математического ожидания наблюдений в~предположении 
зашумления конечными смесями нормальных\linebreak (Тео\-ре\-ма~1) 
и~гам\-ма-рас\-пре\-де\-ле\-ний (Тео\-ре\-ма~3). 
%
Построены доверительные интервалы 
для неизвестного математического ожидания в~этих случаях с~использованием 
уточненной оценки~\eqref{Var} 
(Тео\-ре\-мы~2 и~4 соответственно). Отметим, что соответствующие соотношения 
зависят только от <<экстремальных>> значений параметров смесей, но не от числа 
компонент и~весов в~распределении зашумляющих наблюдений. 
%
Замечание~2 
предлагает подход, который  может быть использован для определения неизвестного 
параметра искусственно добавляемого к~исходным данным шума для улучшения качества 
работы метода скользящего разделения смесей.

\smallskip
Автор выражает признательность доктору фи\-зи\-ко-ма\-те\-ма\-ти\-че\-ских наук, 
профессору Виктору Юрьевичу Королеву за идею использования оценки 
дисперсии вида~\eqref{Var} и~другие полезные обсуждения в~рамках 
работы над данной статьей.

\vspace*{-12pt}

{\small\frenchspacing
 {%\baselineskip=10.8pt
 \addcontentsline{toc}{section}{References}
 \begin{thebibliography}{99}
\bibitem{Wright2003} \Au{Wright~D.\,E., Bray~I.} 
A~mixture model for rounded data~// J.~Roy. Stat. Soc.~D 
Sta., 2003. Vol.~52. P.~3--13.

\columnbreak

\bibitem{Bai2009} \Au{Bai~Z., Zheng~S., Zhang~B., Hu~G.} 
Statistical analysis for rounded data~// J.~Stat. Plan.  Infer., 2009. 
Vol.~139. Iss.~8. P.~2526--2542.

\bibitem{Zhang2010} \Au{Zhang~B., Liu~T., Bai~Z.\,D.} 
Analysis of rounded data from dependent sequences~// 
Ann. I.~Stat. Math., 2010. Vol.~62. Iss.~6. P.~1143--1173.

\bibitem{Zhao2012} \Au{Zhao~N., Bai~Z.} 
Analysis of rounded data in mixture normal model~// Stat. Pap., 2012. 
Vol.~53. P.~895--914.

\bibitem{Korolev2011-i} \Au{Королев~В.\,Ю.} 
Ве\-ро\-ят\-но\-ст\-но-ста\-ти\-сти\-че\-ские методы декомпозиции волатильности 
хаотических процессов.~--- М.: Изд-во Моск. ун-та, 2011. 512~с.

\bibitem{Ushakov2015} \Au{Ушаков В.\,Г., Ушаков Н.\,Г.} 
Об усреднении округленных данных~// Информатика и~её применения, 2015. Т.~9. 
Вып.~4. С.~106--109.

\bibitem{Ushakov2017a} \Au{Ушаков~В.\,Г., Ушаков~Н.\,Г.} 
Границы точ\-ности восстановления информации, 
теряемой при округлении результатов наблюдений~// 
Вестник Московского университета. Серия~15: Вычислительная математика и~кибернетика, 
2017. №\,2. С.~26--30.

\bibitem{Ushakov2017b} \Au{Ushakov~N.\,G., Ushakov~V.\,G.} 
Statistical analysis of rounded data: Recovering of information lost due to rounding~// 
J.~Korean Stat. Soc., 2017.  Vol.~46. No.\,3. P.~426--437.

\bibitem{Gorshenin2016} \Au{Gorshenin~A.\,K., Korolev~V.\,Yu.} 
A~noising method for the identification of the stochastic structure of 
information flows~// Comm. Com. Inf. Sc., 2017. 
Vol.~678. P.~279--289.

\bibitem{Gorshenin2013} 
\Au{Gorshenin~A., Korolev~V.} Modelling of statistical
fluctuations of information flows by mixtures of gamma distributions~// 
27th European Conference on Modelling and Simulation Proceedings.~--- 
Dudweiler, Germany: Digitaldruck Pirrot GmbHP, 2013. P.~569--572.
 \end{thebibliography}

 }
 }

\end{multicols}

\vspace*{-6pt}

\hfill{\small\textit{Поступила в~редакцию 03.08.18}}

\vspace*{6pt}

%\newpage

%\vspace*{-24pt}

\hrule

\vspace*{2pt}

\hrule

\vspace*{-2pt}


\def\tit{DATA NOISING BY FINITE NORMAL AND~GAMMA MIXTURES WITH~APPLICATION 
TO~THE~PROBLEM OF~ROUNDED OBSERVATIONS}


\def\titkol{Data noising by finite normal and~gamma mixtures with~application 
to~the~problem of~rounded observations}



\def\aut{A.\,K.~Gorshenin}

\def\autkol{A.\,K.~Gorshenin}

\titel{\tit}{\aut}{\autkol}{\titkol}

\vspace*{-11pt}


\noindent
Institute of Informatics Problems, Federal Research Center ``Computer Science and
Control'' of the Russian Academy of Sciences, 44-2~Vavilov Str., Moscow 119333,
Russian Federation


\def\leftfootline{\small{\textbf{\thepage}
\hfill INFORMATIKA I EE PRIMENENIYA~--- INFORMATICS AND
APPLICATIONS\ \ \ 2018\ \ \ volume~12\ \ \ issue\ 3}
}%
 \def\rightfootline{\small{INFORMATIKA I EE PRIMENENIYA~---
INFORMATICS AND APPLICATIONS\ \ \ 2018\ \ \ volume~12\ \ \ issue\ 3
\hfill \textbf{\thepage}}}

\vspace*{3pt}



\Abste{In many real problems, statistical analysis of data containing additional 
measurement errors, including rounding, is performed, which in some situations can 
lead to sufficiently significant distortions. In this paper, estimates for an 
unknown expectation of observations are obtained for one of the possible 
rounding models under the assumption that the original data are additionally 
noised with random variables having distributions of the type of finite 
mixtures of normal and gamma laws. Confidence intervals for an 
unknown expectation are constructed using the refined estimate for 
the variance of the integer part of the random variable. An algorithm 
for determining the value of the parameter of artificial noise, which 
can be added to the initial data to improve the quality of the 
method of moving separation of mixtures, is discussed.}


\KWE{noisy data; rounded data; finite normal mixtures; finite gamma mixtures; 
confidence intervals; moving separation of mixtures}



\DOI{10.14357/19922264180304}

%\vspace*{-14pt}

\Ack
\noindent
The research was supported by the Russian Science Foundation (project 18-71-00156).



%\vspace*{6pt}

  \begin{multicols}{2}

\renewcommand{\bibname}{\protect\rmfamily References}
%\renewcommand{\bibname}{\large\protect\rm References}

{\small\frenchspacing
 {%\baselineskip=10.8pt
 \addcontentsline{toc}{section}{References}
 \begin{thebibliography}{99}
\bibitem{1-gor-1}
\Aue{Wright,~D.\,E., and I.~Bray.} 2003.
A~mixture model for rounded data.  \textit{J.~Roy. Stat. Soc.~D Sta.} 52:3--13.

\bibitem{2-gor-1}
\Aue{Bai,~Z., S.~Zheng, B.~Zhang, and G.~Hu.} 2009. 
Statistical analysis for rounded data. \textit{J.~Stat. Plan. 
Infer.} 139(8):2526--2542.

\bibitem{3-gor-1}
\Aue{Zhang,~B., T.~Liu, and Z.\,D.~Bai.} 2010. 
Analysis of rounded data from dependent sequences. 
\textit{Ann. I.~Stat. Math.} 62(6):1143--1173.

\bibitem{4-gor-1}
\Aue{Zhao,~N., and Z.~Bai.} 2012. Analysis of rounded data in mixture normal model. 
\textit{Stat. Pap.} 53:895--914.

\bibitem{5-gor-1}
\Aue{Korolev, V.\,Yu.} 2011. 
\textit{Veroyatnostno-statisticheskie metody dekompozitsii volatil'nosti 
khaoticheskikh protsessov} [Probabilistic and statistical methods of 
decomposition of volatility of chaotic processes]. 
Moscow: Moscow University Publishing House. 512~p.

\bibitem{6-gor-1}
\Aue{Ushakov, V.\,G., and N.\,G.~Ushakov.} 
2015. Ob usrednenii okruglennykh dannykh [On averaging of rounded data].
\textit{Informatika i~ee Primeneniya~--- Inform. Appl.} 9(4):106--109.

\bibitem{7-gor-1}
\Aue{Ushakov,~V.\,G., and N.\,G.~Ushakov.} 2017. 
Boundaries of the precision of restoring information lost after rounding
 the results from observations. 
 \textit{Moscow University Computational Math. Cybernetics} 41(2):76--80.

\bibitem{8-gor-1}
\Aue{Ushakov,~N.\,G., and  V.\,G.~Ushakov.} 2017. 
Statistical analysis of rounded data: Recovering of information lost due to rounding. 
\textit{J.~Korean Stat. Soc.} 46(3):426--437.

\bibitem{9-gor-1}
\Aue{Gorshenin,~A.\,K., and V.\,Yu.~Korolev.} 2016. 
A~noising method for the identification of the stochastic structure of information 
flows. \textit{Comm. Com. Inf. Sc.} 678:279--289.

\bibitem{10-gor-1}
\Aue{Gorshenin,~A., and V.~Korolev.} 2013.  Modelling of statistical fluctuations of
information flows by mixtures of gamma distributions. 
\textit{27th European Conference on Modelling and Simulation Proceedings}. 
Dudweiler, Germany: Digitaldruck Pirrot GmbHP. 569--572.

\end{thebibliography}

 }
 }

\end{multicols}

\vspace*{-6pt}

\hfill{\small\textit{Received August 3, 2018}}

%\pagebreak

%\vspace*{-18pt}

\Contrl

\noindent
\textbf{Gorshenin Andrey K.} (b.\ 1986)~--- Candidate of Science (PhD) in physics and
mathematics, associate professor, leading scientist, Institute of Informatics Problems,
Federal Research Center ``Computer Science and Control'' of the Russian Academy of
Sciences, 44-2 Vavilov Str., Moscow 119333, Russian Federation; 
\mbox{agorshenin@frccsc.ru}
\label{end\stat}

\renewcommand{\bibname}{\protect\rm Литература}        %4
\def\stat{vasiliev}

\def\tit{О ФУНКТОРНОМ ПРЕДСТАВЛЕНИИ ОПТИМИЗИРУЕМЫХ ДИНАМИЧЕСКИХ 
МУЛЬТИАГЕНТНЫХ СИСТЕМ}

\def\titkol{О функторном представлении оптимизируемых динамических 
мультиагентных систем}

\def\aut{Н.\,С.~Васильев$^1$}

\def\autkol{Н.\,С.~Васильев}

\titel{\tit}{\aut}{\autkol}{\titkol}

\index{Васильев Н.\,С.}
\index{Vasilyev N.\,S.}


%{\renewcommand{\thefootnote}{\fnsymbol{footnote}} \footnotetext[1]
%{Исследование выполнено за счет гранта Российского научного фонда №\,22-28-00588, {\sf 
%https://rscf.ru/project/22-28-00588/}. Работа проводилась с~использованием инфраструктуры Центра 
%коллективного пользования <<Высокопроизводительные вычисления и~большие данные>> (ЦКП 
%<<Информатика>> ФИЦ ИУ РАН, Москва).}}


\renewcommand{\thefootnote}{\arabic{footnote}}
\footnotetext[1]{Московский государственный технический университет имени Н.\,Э.~Баумана, \mbox{nik8519@yandex.ru}}

\vspace*{2pt}






    \Abst{Топос функторов выбран в~качестве компьютерного инструмента синтеза 
динамических игр многих лиц. Задаваемая шкала упорядочивает объекты, 
отвечающие сопутствующим статическим подыграм. Последние служат 
состояниями динамической мультиагентной сис\-те\-мы (ДМАС). Исходная 
динамическая игра и~все статические подзадачи пред\-став\-ля\-ют\-ся в~моноидальной 
категории бинарных отношений. Под рациональным решением игры понимается 
равновесие. Композициональное строение оп\-ти\-ми\-зи\-ру\-емой ДМАС выражено 
в~форме динамического результирующего отношения (ДРО) игры. Поиску равновесия 
отвечает максимизация ДРО. Это делается методом Беллмана, обобщенным на 
задачи оптимального управления, поставленные в~форме отношений. Программная 
реализация предложенного подхода может быть основана на нейросетевых 
вычислениях ввиду согласованности архитектур применяемых графов отношений и~нейросетей.}
    
    \KW{категория функторов; композициональность; моноидальная категория; 
обратный образ; динамическое отношение игры; статическая подыгра; отношение 
предпочтения; динамическое результирующее отношение; рациональное решение; 
морфизм Беллмана}

\DOI{10.14357/19922264240201}{CLMBXC}
  
%\vspace*{-6pt}


\vskip 10pt plus 9pt minus 6pt

\thispagestyle{headings}

\begin{multicols}{2}

\label{st\stat}
    
\section{Введение}

    В ДМАС агенты принимают 
решения в~каждом состоянии системы в~за\-ви\-си\-мости от рас\-по\-ла\-га\-емой ими 
информации о~поведении других участников конфликта~[1--4]. Воздействие 
стратегий на ДМАС распределено во времени и~связано с~выбором ситуаций 
в~череде со\-пут\-ст\-ву\-ющих статических игровых подзадач. 
    
    Процесс изменения состояний ДМАС традиционно задают с~по\-мощью 
дифференциальных или итерационных уравнений, в~которые входят управ\-ля\-ющие 
воздействия игроков. От этого описания динамики приходится отходить даже 
в~антагонистических дифференциальных играх с~полной ин\-фор\-ми\-ро\-ван\-ностью 
игроков. Управляемую систему представляют дифференциальным уравнением 
в~контингенциях~\cite{3-vas}, т.\,е.\ применяют многозначные отоб\-ра\-же\-ния. 
В~стохастических постановках так\-же возникает дополнительная не\-опре\-де\-лен\-ность, 
связанная с~прогнозированием будущих со\-сто\-яний сис\-те\-мы~\cite{4-vas}. 
    
    Зачастую целесообразно отказаться от функционального описания динамики 
игры и~перейти на\linebreak язык отношений~[5]. Благодаря этому значительно расширяется 
круг приложений~\cite{4-vas, 6-vas, 7-vas, 8-vas, 9-vas}, приобретается свойство 
композициональности моделей ДМАС, удобное для модификации игровых \mbox{задач}. 
Этот подход поддерживается компьютерной алгеброй тео\-рии категорий~\cite{10-vas, 11-vas}. 
Кроме того, графовая структура отношений и~сетевая структура игровой задачи 
допускают эффективную нейросетевую программную реализацию~\cite{6-vas, 7-vas, 12-vas, 13-vas}. 
    
    Категорный подход охватывает общий случай динамических игр многих лиц 
    с~разнообразными классами допустимых стратегий игроков~\cite{5-vas, 11-vas}. 
В~качестве со\-сто\-яний динамической системы рассматриваются со\-пут\-ст\-ву\-ющие 
статические игры, связанные динамическим отношением.
    
    Традиционное моделирование игровой задачи проводится средствами 
категории множеств SET. Естественным обобщением классического подхода 
становится формализация игровой операции, вы\-пол\-ня\-емая на языке моноидальной 
категории бинарных отношений REL. Применяемый аппарат позволяет 
единообразно выражать и~модифицировать интересы агентов, проводить 
эквивалентные преобразования игровой задачи, описывать поиск рационального 
решения~\cite{5-vas}. С~по\-мощью введения ДРО игры исследование ДМАС может быть сведено 
к~последовательной максимизации ДРО.
    
    Итак, вместо многошагового процесса игры предлагается перейти 
    к~представлению динамики задачи с~по\-мощью функтора $\tau: S\hm\to \mathrm{SET}_T$, 
который преобразует вы\-би\-ра\-емую шкалу~$S$ в~категорию $\mathrm{SET}_T$ монады 
$(T,\eta,\psi)$, $T: A\hm\to 2^A$~\cite{11-vas}. Объектами монады служат конечные 
множества, а~морфизмами~--- отношения между ними. Равенство $\mathrm{SET}_T\hm=\mathrm{REL}$ 
означает, что функторная модель ДМАС строится на языке отношений. 
Единообразно выражаются правила игры, интересы участников, классы стратегий 
агентов, динамика задачи и~даже алгоритм ее решения. Строгость функтора 
вложения $\mathrm{SET}\hm\to \mathrm{SET}_T$ обеспечивает пре\-емст\-вен\-ность новой формы 
пред\-став\-ле\-ния задачи и~классической.
    
    Функтор $\tau$ порождает динамическое отношение игры $R\hm= R(\tau)$, 
регламентирующее смену со\-сто\-яний~--- ре\-ша\-емых в~текущий момент времени 
статических подыгр $\Gamma(X)$ с~множеством \mbox{до\-пус\-ти\-мых} ситуаций~$X$. 
Динамическое отношение определяет по\-сле\-до\-ва\-тель\-ность смены со\-сто\-яний 
$\Gamma(X_0), \ldots , \Gamma(X_T)$ и~вводит отношение предшествования подыгр 
$\Gamma(X)\overset{R}{\prec} \Gamma(Y)$. В~соответствии с~ним определена 
преемственность ситуаций $x\overset{R}{\prec} y$, причем выбор любой пары 
ситуаций $x\hm\in X$ и~$y\hm\in Y$, относящихся к~разным состояниям системы, 
возможен, только если выполнено включение $(x,y)\hm\in R$. Таким образом, из 
решений подзадач $\Gamma(X_0), \ldots , \Gamma(X_T)$ строится 
по\-сле\-до\-ва\-тель\-ность $x_0, \ldots , x_T$ вы\-би\-ра\-емых ситуаций подобно тому, как это 
делается в~многошаговых играх с~по\-мощью уравнений. 
    
    Интересы агентов в~задаче заданы в~форме отношений предпочтения для 
терминального со\-сто\-яния сис\-те\-мы~$\Gamma(X^T)$: 
    \begin{equation}
    \tilde{\rho}^{kT} \in \mathrm{REL} \left( X^T, X^T\right)\,,\enskip k\in K\,.
    \label{e1.1-vas}
    \end{equation}
    
    Правила функционирования ДМАС формулируются во всех подыграх 
$\Gamma(X_i)$. Во-пер\-вых, определены цели игроков~--- максимизация 
отношений предпочтения $\rho_i^k\hm\in \mathrm{REL}\left( X_i, X_i\right)$, $k\hm\in 
K(X_i)$, по\-лу\-ча\-емых из заданных~(\ref{e1.1-vas}) с~по\-мощью вы\-чис\-ле\-ния 
обратных образов морфизмов~$\tilde{\rho}^{kT}$ относительно динамического 
отношения~$R$. Во-вто\-рых, агенты выбирают классы допустимых стратегий 
посредством графа коммуникаций и~функции, размечающей его дуги. Тем самым 
задается сетевая структура в~статических подыграх~\cite{12-vas, 13-vas}. Разметка 
графа детализирует информацию, которой обмениваются партнеры при совершении 
ходов и~формировании коалиций~\cite{5-vas}. Стратегии агентов в~исходной 
динамической задаче включают процедуру принятия решений в~каж\-дом состоянии 
сис\-темы. 
    
    Далее изучено композициональное стро\-ение динамического результирующего 
отношения игры, поз\-во\-ля\-ющее искать рациональное решение, т.\,е.\ такое, которое 
оптимизирует функционирование ДМАС. Поиск проводится методом 
динамического программирования, обоб\-щен\-ным на задачи оптимального 
управ\-ле\-ния, по\-став\-лен\-ные в~форме отношений. 

\section{Постановка задачи}

    В качестве шкалы~$S$ выберем категорию 
    \begin{equation}
 S: L \overset{\underset{\mathrm{out}}{\longrightarrow}}{\underset{\overset{\mathrm{in}}{\longrightarrow}}
 {\vphantom{{\mbox{\scriptsize{$\circ$}}}} \ }} M\,.
    \label{e2.1-vas}
    \end{equation}
    
    Динамику мультиагентной системы зададим функтором $\tau: S\hm\to \mathrm{SET}_T$, 
которому отвечает ориентированный граф динамического отношения~$\Gamma_R$, 
$R\hm=R(\tau)$. Граф должен быть ацик\-ли\-че\-ским и~связ\-ным. В~его вершинах 
находятся множества до\-пус\-ти\-мых ситуаций статических подыгр~$\Gamma(X_i)$, 
а~дуги отвечают отношениям $R_{ij}\hm\in \mathrm{REL}\left( X_i, X_j\right)$. 
    
    \smallskip
    
    \noindent
    \textbf{Определение~2.1.}\  Скажем, что ситуация $x_i\hm\in X_i$ 
непосредственно предшествует ситуации $x_j\hm\in X_j$, если и~только если $(x_i, 
x_j)\hm\in R_{ij}$. Отношением предшествования со\-сто\-яний назовем транзитивное 
замыкание $\overline{R}\hm= \overline{R}_{ij}$ динамического отношения. 
    
    \smallskip
    
    Последовательная смена состояний ДМАС происходит в~соответствии 
с~отношением непосредственного предшествования $\Gamma(X_i) 
\overset{R_{ij}}{\prec} \Gamma(X_j)$. Каж\-до\-му со\-сто\-янию соответствует некоторая 
\mbox{статическая} подыгра. Процесс функционирования сис\-те\-мы складывается из 
последовательного выбора игроками ситуаций $x_0, \ldots , x_T$, где  
$x_0$ и~$x_T$~--- решения начальной и~терминальной подыгр. Итак, имеем 
множество всех допустимых ситуаций исходной динамической задачи вида
 \begin{multline*}
    \tilde{X} ={}\\
    {}=\!\left\{\! \tilde{x} =\left( x_0, \ldots , x_T\right): \left( \forall_i x_i\in 
X_i\right) \wedge x_0\overset{R}{\prec} \cdots \overset{R}{\prec} x_T\!\right\}\!.\hspace*{-5.73415pt}
    \end{multline*}
    
    Правила проведения каждой подыгры $\Gamma(X_i)$ задаются сле\-ду\-ющим 
функтором и~функ\-цией: 
    \begin{equation}
    g_i: S\to \mathrm{SET}\,;\enskip h_i: E\to \overline{2}\,.
    \label{e2.2-vas}
    \end{equation}
Игроки стремятся по возможности максимизировать свои отношения предпочтения 
$\rho_l^k: X_l\hm\to X_l$,\linebreak $k\hm\in K_l$. (Иначе рас\-смат\-ри\-ва\-ют\-ся противоположные 
отношения $(\rho_i^k)^{\mathrm{op}}$.) Функторы~(\ref{e2.1-vas}) и~(\ref{e2.2-vas}) 
опре\-де\-ля\-ют сетевую структуру игры. Шкала~$S$ выделяет множество дуг~$E$, по 
которым участники\linebreak $k\hm\in K(X_i)$ статической игры осуществляют 
коммуникацию. Функциями in и~out в~(\ref{e2.1-vas}) вводят порядок ходов. 
Функция~$h_i$ из формулы~(\ref{e2.2-vas}) осуществляет разметку дуг графа 
коммуникаций~$\gamma_i$, определяя данные, которыми обмениваются игроки при 
своих ходах~\cite{5-vas}. Не\-пус\-тые сообщения могут содержать либо выбранную 
стра\-те\-гию-конс\-тан\-ту, либо стра\-те\-гию-функ\-цию~\cite{1-vas, 5-vas}. 
Предполагается так\-же, что никто из агентов не блефует. 
    
    Понятие рационального поведения игроков зависит от сетевой 
структуры~(\ref{e2.2-vas}) статических игр~\cite{5-vas, 12-vas, 13-vas}. Будем 
применять принцип равновесия как к~исходной динамической задаче, так и~ко всем 
по отдельности подыграм~$\Gamma(X_i)$.
    
    \smallskip
    
    \noindent
    \textbf{Пример~2.1.}\ Рассмотрим функтор~(\ref{e2.1-vas}), которому отвечает 
граф
    \begin{equation}
    \Gamma_R: \mbox{\fbox{$X_1$}} \xrightarrow{R_{12}} \mbox{\fbox{$X_2$}} 
\xrightarrow{R_{23}} \cdots \xrightarrow{R_{n-1,n}} \mbox{\fbox{$X_n$}}\,.
    \label{e2.3-vas}
    \end{equation}
    
    Функции out и~in определяют начало $X_i\hm\in M$ и~конец $X_j\hm\in M$ 
каж\-дой дуги, обозначенной стрелкой $R_{i,i+1}\hm\in L$. В~моменты времени 
$i\hm= 1,2,\ldots , n$ решаются игры~$\Gamma(X_i)$.
    
    Схема~(\ref{e2.3-vas}) свойственна многошаговым опе\-ра\-циям.
    
    \smallskip
    
    \noindent
    \textbf{Пример~2.2.}\ Пусть в~позиционной многошаговой игре двух лиц 
известна сле\-ду\-ющая динамическая сис\-те\-ма, начальное условие и~фазовое 
ограничение:
    \begin{gather*}
    \left\{
    \begin{array}{c}
    y_{t+1}^1 =y_t^1+u_t^1\,;\\[6pt]
    y_{t+1}^2 = y_t^2-u_t^2\,;
    \end{array}
    \right.\\[2pt]
    y_0^1 =0\,,\enskip y_0^2 =0\,;\\[2pt]
    \left\vert y_t^1 -y_t^2\right\vert \leq 2\,,\enskip t=\overline{0, T}\,.
    \end{gather*}
Агенты $k=1, 2$ стремятся по возможности максимизировать функции выигрыша
\begin{equation*}
J_1=-\sum\limits_{t=1}^{T-1} y_t u_t^1;\quad J_2= \sum\limits_{t=1}^{T-1} y_t u_t^2,
\end{equation*}
где
$$
y_t=y_t^1 - y_t^2,\quad u_t^k \in U_0\equiv \{ -1, 0, 1\}.
$$
    
    Интересы игроков $\tilde{\rho}^k \hm\in \mathrm{REL}( \tilde{X}, 
\tilde{X})$~(\ref{e1.1-vas}) заданы функционалами~$J_k$. В~статических 
подыг\-рах~$\Gamma(X_t)$, $X_t\hm= \cup y_t X(y_t)$, $X_0\hm= U_0^2$, с~по\-мощью 
функций $f^1\hm= -y_t u_t^1$, $f^2\hm= y_t u_t^2$ введем отношения предпочтения 
игроков~$\rho_t^k$. Видно, что интересы участников подыгр противоположны 
$\rho_t^2\hm= \rho_t^{1\mathrm{op}}$, а~множества допустимых ситуаций расслоены 
в~за\-ви\-си\-мости от позиций $y_t\hm= y_t^1\hm- y_t^2$, $\vert y_t\vert \hm\leq 2$, 
в~которых может пребывать динамическая сис\-те\-ма. Ситуация $(u_t^1, u_t^2)\hm\in 
X(y_t)$ допустима, только если ее выбор не нарушает заданное фазовое 
ограничение. 
    
    Во все моменты времени отношения предпочтения $\rho_t^k(y_t): X(y_t) \hm\to 
X(y_t)$, $k\hm=1,2$,  одинаковы: $\rho_t^k \hm= \rho_1^k$, $X_t\hm= X_1$, $t\hm= 
\overline{1,T-1}$, причем
    $$
    \rho_t^k\triangleq \bigcup\limits_{y_t} \rho_t^k (y_t).
    $$

    
    Вычисления показывают, что при $t\hm= \overline{1, T-1}$ имеем
    
    \vspace*{-8pt}
    
    \noindent
    \begin{multline*}
    X_t=\coprod\limits^2_{y_t=-2} X(y_t),\ X(\pm 2) = \{ (\pm 1,\pm1)\},\\
     X(\pm1) = \{ (\pm1,0), (0,\pm1)\},\\
      X(0)=\{ (1,-1), (0,0), (-1,1)\}\,;
    \end{multline*}
    
    \vspace*{-13pt}
    
    \noindent
    \begin{multline*}
    \rho_t^1(0): (0,-1)\sim (-1,0) \sim (0,0),\\
     \rho_t^1(1)=\{ ((0,-1), (-1,0))\},\\
    \rho^1_t(-1) =\{ ((-1,0), (0,-1))\}\,,\\ 
    \rho_1^1(\pm2) =\{ (\pm 1, \pm1)\}.
    \end{multline*}
    
    \vspace*{-4pt}
    
    Динамическое отношение игры, равное $R_{t,t+1}: X_t\hm\to X_{t+1}$, $t\hm= 
\overline{1, T-1}$, связывает до\-пус\-ти\-мую текущую $u_m\hm\equiv (u_m^1, u_m^2) 
\hm\in X(y_m)$ и~выбираемую сле\-ду\-ющую $u_{m+1}\hm\in X(y_{m+1})$ ситуации 
в~состояниях сис\-те\-мы $\Gamma(X_m)$, $m\hm= t, t\hm+1$. Таким образом, 
отношение непосредственного пред\-шест\-во\-ва\-ния ситуаций определено включением
    $$
    u_t \xrightarrow{R_{tt+1}} u_{t+1},\ u_{t+1} \in X(y_{t+1}),\ y_{t+1} \!=\! y_t+\left( 
u_t^1+ u_t^2\right).
    $$
    
    \vspace*{-2pt}
    
    Пусть сетевая структура~(\ref{e2.2-vas}) такова, что во всех играх 
$\Gamma(X_t)$ один из участников сообщает другому свое управ\-ля\-ющее 
воздействие~$u_t^k$. Тогда рациональное решение игры состоит в~выборе 
ситуации, ста\-би\-ли\-зи\-ру\-ющей траекторию сис\-те\-мы $y_t\hm\equiv 0$, а~многошаговая 
игра имеет седловые точки  $\forall_t u_t^1 \hm= -u_t^2$.
    
    При поиске рационального решения игровых задач применяются 
вспомогательные морфизмы, стро\-ящи\-еся из исходных~(\ref{e1.1-vas}) с~по\-мощью 
операций алгебры отношений $A\hm= \left( \circ, \cup, \cap, {}^{\mathrm{op}}, \times; 
\sigma,\varnothing \right)$~\cite{5-vas, 10-vas, 11-vas}. Получение ка\-ким-ли\-бо 
игроком дополнительной информации уменьшает неопределенность выбора 
подходящей стратегии (см.\ пример~2.2). Выбор сетевой структуры  
игры~(\ref{e2.2-vas}) изменяет отношения предпочтений участников~$\rho$. Игроки 
руководствуются сужениями $\rho\vert_A\hm= \rho \cap A^2$, $A\hm\subset X$, 
исходных отношений~$\rho$ на некоторые, вполне определенные подмножества 
ситуаций. 
    
    При условии сделанных предположений граф динамического отношения 
содержит минимальные и~максимальные элементы~--- вершины $X^0$ и~$X^T$ 
(см.~пример~2.1). Процесс функционирования ДМАС начинается с~решения 
статических \mbox{подыгр}~$\Gamma(X^0)$ и~заканчивается подзадачами~$\Gamma(X^T)$. 
Они названы начальным и~терминальным со\-сто\-яни\-ями сис\-те\-мы соответственно. 
В~отличие от примера~2.2, в~рас\-смат\-ри\-ва\-емой по\-ста\-нов\-ке задаются терминальные 
отношения предпочтения агентов~(\ref{e1.1-vas}).

     \begin{figure*}[b] %fig1
\vspace*{1pt}
\begin{minipage}[t]{79mm}
\begin{center}  
    \mbox{%
\epsfxsize=17.323mm 
\epsfbox{vas-1.eps}
}
\end{center}
\end{minipage}
\hfill
\vspace*{1pt}
\begin{minipage}[t]{79mm}
\begin{center}
     \mbox{%
\epsfxsize=25.462mm 
\epsfbox{vas-2.eps}
}
\end{center}
\end{minipage}
\begin{minipage}[t]{79mm}
\vspace*{-12pt}
\Caption{Образ $r=R\circ r^\prime$ и~обратный образ $r^\prime \hm= R^{\mathrm{op}}\circ r$ 
относительно функтора~$R$}
\end{minipage}
%\end{figure*}
\hfill
% \begin{figure*} %fig3.2
      \begin{minipage}[t]{79mm}
      \vspace*{-12pt}
           \Caption{Граф $\Gamma_R$ динамического отношения~$R$: $R_{ij}\hm\in \mathrm{REL}\,(X_i, X_j)$ }
     \label{f3.2-vas}
     \end{minipage}
%     \vspace*{-24pt}
     \end{figure*} 

    
    Рациональное поведение игроков заключается в~том, что они стремятся 
максимизировать свои 
 предпочтения на множестве всех допустимых ситуаций:
    \begin{equation}
    \forall_k \tilde{\rho}^{kT} \to \mathop{\mathrm{MAX}}\limits_{\tilde{X}}\,.
    \label{e2.4-vas}
    \end{equation}
В~задаче~(\ref{e1.1-vas}), (\ref{e2.1-vas}), (\ref{e2.2-vas}), (\ref{e2.4-vas}) требуется 
найти ситуацию равновесия~\cite{1-vas, 5-vas}. 
    
    Существование равновесия во многом зависит от свойств 
морфизмов~$\rho_i^k$, $k\hm\in K$. Наложим одно из таких требований. Пусть 
партнеры предлагают некоторому игроку~$k$ принять решение в~ситуации~$s$, для 
которой $\rho_i^k\vert_{s_k} \hm= \varnothing$, $s\hm\in X_i$. Предполагается, что 
возникшую неопределенность выбора каж\-дый игрок $k\hm \in K$ разрешит  
с~по\-мощью изменения своего отношения предпочтения, положив 
$\rho_i^k\vert_{s_k} \hm= \{ s\}$. (Игрок соглашается с~выбором ситуации~$s$.) Это 
требование выполняется для рефлексивных отношений.
    
\section{Эквивалентные преобразования динамических~игр}

    Категорное представление допускает применение моноидальных операций для 
преобразования игровой задачи~\cite{11-vas}. Морфизмы $R: \mathrm{REL}\left(Y\right)\hm\to 
\mathrm{REL}\,(X)$ представляют собой функторы, срав\-ни\-ва\-ющие однообъектные 
подкатегории в~категории REL. Образ морфизма $\rho\hm\in \mathrm{REL}\,(Y,Y)$ 
относительно~$R$ будем записывать как композицию $R\circ \rho \hm\in \mathrm{REL}\,(X,X)$.
    
    \smallskip
    
    \noindent
    \textbf{Определение~3.1.} Обратным образом (или кообразом) отношения $r: 
X\hm\to X$ относительно морфизма $R: Y\hm\to X$ назовем стрелку $r^\prime: 
Y\hm\to Y$, превращающую сле\-ду\-ющий квад\-рат в~коммутативный  
(рис.~1).
    
    
     
     
    Ввиду того что морфизм $R^{\mathrm{op}} \hm\in \mathrm{REL}\,(X,Y)$ сохраняет композиции 
стрелок, он сам становится функтором $R^{\mathrm{op}}: \mathrm{REL}\,(X)\hm\to \mathrm{REL}\,(Y)$. Так как 
$\forall_{ij} (x,y) \hm\in R_{ij}^{\mathrm{op}} \hm\Leftrightarrow  (y,x)\hm\in R_{ij}$, то 
назовем~$R^{\mathrm{op}}$ противоположным к~$\{ R_{ij}\}$ динамическим отношением. 
Его можно изобразить, изменив на\-прав\-ле\-ния всех дуг в~графе~$\Gamma_R$ 
(вертикальных стрелок на рис.~1). 




    
    Пусть вершины $X_1$ и~$X_l$ графа динамического отношения~$\Gamma_R$ 
связаны некоторым путем $L\hm= (X_1, \ldots, X_l)$. Взяв композицию $R_L\hm= 
R_{l-1}\circ\cdots\linebreak \cdots \circ R_1$ морфизмов $R_i\hm \in \mathrm{REL}\,(X_i, X_{i+1})$ вдоль цепи
$L\hm= (X_1, \ldots , X_l)$, мож\-но <<опустить>> произвольное бинарное отношение 
$r_l: X_l\hm\to X_l$ с~$X_l$ на множество~$X_1$ и~по\-стро\-ить его обратный образ 
$r_1\hm= R_L^{\mathrm{op}} \circ r_l$.
    
    Пусть теперь множество $\{ L\hm= (X_1, \ldots , X_l)\}$ всех попарно 
различных путей, свя\-зы\-ва\-ющих вершины $X_1$ и~$X_l$, состоит более чем из одного 
элемента. Под обратным образом отношения $\rho_l: X_l\hm\to X_l$, переносимого 
на множество~$X_1$, будем понимать копроизведение
    \begin{multline}
    \rho_1^\prime =R^{\mathrm{op}}\circ \rho_l =\coprod\limits_{\{L\}} R_L^{\mathrm{op}} \circ 
\rho_l;\\
 \rho_1^\prime: \coprod\limits_{\{ L\}} X_1\to \coprod\limits_{\{L\}} X_1\,.
    \label{e3.1-vas}
    \end{multline}
    
    <<Поднятием>> морфизма~$\rho_1$ вдоль путей $\{L\}$ мож\-но по\-стро\-ить 
образ $\rho_L\hm= R\circ \rho_1$. Это отношение $\coprod_{\{L\}} R_L\circ \rho_1$ на 
копроизведении объектов $\coprod_{\{L\}} X_l$ с~числом сомножителей, рав\-ным 
мощ\-ности набора~$\{L\}$.
    
    По формуле~(\ref{e3.1-vas}) заданные в~терминальных со\-сто\-яни\-ях сис\-те\-мы 
морфизмы~(\ref{e1.1-vas}) переносятся во все остальные со\-сто\-яния. Так вводятся 
отношения предпочтения игроков~$\rho_i^k$ в~играх~$\Gamma(X_i)$:
    \begin{equation*}
    \forall_{ik} \rho_i^k =R^{\mathrm{op}}\circ \tilde{\rho}_i^{kT}.
%    \label{e3.2-vas}
    \end{equation*}
    
    \noindent
    \textbf{Замечание~3.1.}\ Общее определение кообраза~(\ref{e3.1-vas}) мож\-но 
заменить двойственной конструкцией, основанной на произведении 
отношений~\cite{11-vas}: 
    \begin{equation}
    \rho^{\prime\prime} \!=\!R^{\mathrm{op}} \circ \rho \!=\!\prod\limits_{\{L\}} R^{\mathrm{op}}_L \circ \rho\,;\ \rho\! =\!R\circ \rho^{\prime\prime}= 
\prod\limits_{\{L\}} R_L\circ \rho^{\prime\prime}\!.\!
    \label{e3.3-vas}
    \end{equation}
Получим морфизмы~(\ref{e3.3-vas}) вида 
$$
\rho: \prod\limits_{\{L\}} X_1\to \prod\limits_{\{L\}} X_1,\ \rho^{\prime\prime}: \prod\limits_{\{L\}} X_l\to \prod\limits _{\{L\}} X_l.
$$

    \smallskip
    
    \noindent
    \textbf{Пример~3.1.}\ Воспользуемся формулой~(\ref{e3.1-vas}) 
применительно к~ДМАС, пред\-став\-лен\-ной на рис.~2. 
    
    % \end{multicols}
     


     
  %   \begin{multicols}{2}
    
     
Кообразом $R^{\mathrm{op}}\circ \rho_4$ отношения $\rho_4: X_4\hm\to X_4$ служит морфизм 
\begin{multline*}
\rho_1^\prime: X_1\coprod X_1 \to X_1\coprod X_1;
\\
\rho_1^\prime =\left( R_{13}^{\mathrm{op}} \circ R_{34}^{\mathrm{op}}\coprod R_{12}^{\mathrm{op}} \circ 
R_{24}^{\mathrm{op}}\right)\circ \rho_4.
\end{multline*}
Опуская~$\rho_4$ на множества $X_2$ и~$X_3$, получим $\rho_2^\prime \hm= 
R_{24}^{\mathrm{op}} \circ \rho_4$, $\rho_3^\prime\hm= R_{34}^{\mathrm{op}} \circ \rho_4$ 
соответственно. Поднятием отношения $\rho_1^\prime: X_1\hm\to X_1$ на 
множество~$X_4$ строится образ  $R\circ \rho_1^\prime$ как
\begin{multline*}
\rho_4: X_4\coprod X_4 \to X_4\coprod X_4\,;\\
 \rho_4=\left( R_{24}\circ R_{12} \coprod 
R_{34} \circ R_{13}\right) \circ \rho_1^\prime.
\end{multline*}
    
    Конструкция~(\ref{e3.3-vas}) дает структуру
\begin{multline*}
    \rho_4: X_4\times X_4 \to X_4\times X_4\,;\\
     \rho_4=\left( R_{24}\circ R_{12} \prod 
R_{34} \circ R_{13}\right) \circ \rho_1^{\prime\prime}.
\end{multline*}
    
    Формулы~(\ref{e3.1-vas}) и~(\ref{e3.3-vas}) позволяют строить эквивалентные 
модели игры. Свойство универсальности объектов $X_2\times X_3$ и~$X_2\coprod 
X_3$, вы\-ра\-жа\-емое по\-средст\-вом коммутативных диаграмм~\cite{11-vas}, однозначно 
определяет новое динамическое отношение с~со\-от\-вет\-ст\-ву\-ющи\-ми морфизмами. 
Например, вместо графа из рис.~2 мож\-но работать с~более прос\-той 
графовой структурой, пред\-став\-лен\-ной в~любой из сле\-ду\-ющих форм:
    \begin{gather}
    \mbox{\fbox{$X_1$}} \xrightarrow{\left\langle R_{12}, R_{13}\right\rangle} 
\mbox{\fbox{$X_2\times X_3$}} \xrightarrow{\left\langle R_{24}^{\mathrm{op}}, 
R_{34}^{\mathrm{op}}\right\rangle^{\mathrm{op}}} \mbox{\fbox{$X_4$}}\,;
    \label{e3.4-vas}\\
    \mbox{\fbox{$X_1$}} \xrightarrow{[ R_{12}, R_{13}]^{\mathrm{op}}} 
\mbox{\fbox{$X_2\coprod X_3$}} \xrightarrow{[ R_{24}, R_{34}]} 
\mbox{\fbox{$X_4$}}\,.
    \label{e3.5-vas}
    \end{gather}
Возможность эквивалентного перехода от произвольного графа~$\Gamma_R$ (см.\ 
рис.~2) к~цепи~(\ref{e2.3-vas}) (см.\ (\ref{e3.4-vas}) и~(\ref{e3.5-vas})) 
обоснована в~тео\-ре\-ме~3.1. 
    
    \smallskip
    
    \noindent
    \textbf{Теорема~3.1.} \textit{Граф всякого динамического отношения 
приводится к~виду}~(\ref{e2.3-vas}).

\vspace*{-6pt}

\section{Результирующее отношение динамической игры}

    Введением результирующего отношения игра сводится к~проблеме 
оптимального управ\-ле\-ния, по\-став\-лен\-ной в~форме отношений. Для ее решения\linebreak 
мож\-но применить обобщенный метод динамического программирования. 
Напомним~\cite{5-vas}, что в~любой статической игре $\Gamma(Y)$ существует 
ре\-зуль\-ти\-ру\-ющее отношение~$P_Y$. Оно выражает \mbox{композициональное} свойство 
игры, учи\-ты\-ва\-ющее интересы всех участников операции, сетевую структуру их 
взаимодействия и~ра\-ци\-о\-наль\-ность поведения. Решение игры $\Gamma(Y)$ сводится
 к~поиску максимальных элементов $P_Y\hm\to \mathrm{MAX}_Y$. Благодаря\linebreak этому, мож\-но 
по-но\-во\-му определить состояния \mbox{исходной} ДМАС. Вместо игр $\Gamma(X_i)$ 
будем рас\-смат\-ри\-вать оптимизационные задачи $(\tilde{X}_i, P_{\tilde{X}_i})$, 
$P_{\tilde{X}_i} \hm\in \mathrm{REL}\,(\tilde{X}_i, \tilde{X}_i)$. Иначе говоря, теперь во всех 
со\-сто\-яни\-ях сис\-те\-мы решение принимает единственный агент. 
    
    %\smallskip
    
    \noindent
    \textbf{Определение~4.1.}\ Пусть $(\tilde{Y}_1, P_{\tilde{Y}_1} ) 
\overset{R}{\prec} (\tilde{Y}_2, P_{\tilde{Y}_2})$; $\tilde{y}_1, \tilde{y}_2 \hm\in 
\tilde{Y}_1$. Ситуация~$\tilde{y}_2$ называется более перспективной по сравнению 
с~$\tilde{y}_1$, если выполнено свойство
    \begin{equation}
    \left( \tilde{y}_1, \tilde{y}_2\right) \in \left( R^{\mathrm{op}}\circ P_{\tilde{Y}_2} \right)\circ 
P_{\tilde{Y}_1}.
    \label{e4.1-vas}
    \end{equation}
    
    В каждом состоянии ДМАС игрокам целесообразно использовать более 
перспективные ситуации и~из них формировать рациональное решение 
задачи~(\ref{e1.1-vas}), (\ref{e2.1-vas}), (\ref{e2.2-vas}), (\ref{e2.4-vas}). В~этом 
заключается прин\-цип Бел\-лмана. 
    
    В случае динамического отношения с~графом~(\ref{e2.3-vas}) (см.\ 
тео\-ре\-му~3.1) рас\-смот\-рим сле\-ду\-ющую итерационную схему по\-стро\-ения 
<<оп\-ти\-ми\-зи\-ру\-ющих>> морфизмов $\{ \tilde{P}_{\tilde{X}_k}\}$:

\vspace*{-4pt}

\noindent
    \begin{multline}
    \tilde{P}_{\tilde{X}_T} =P_{X_T},\ \tilde{P}_{\tilde{X}_{T-k}} =\left( 
R^{\mathrm{op}}\circ \tilde{P}_{\tilde{X}_{T-k+1}}\right) \circ P_{\tilde{X}_{T-k}},\\
    k=\overline{1, T}\,.
    \label{e4.2-vas}
    \end{multline}
    
    \vspace*{-4pt}

\noindent
Назовем~(\ref{e4.2-vas}) уравнениями Беллмана в~форме отношений. 
    
    \smallskip
    
    \noindent
    \textbf{Определение~4.2.} Динамическим ре\-зуль\-ти\-ру\-ющим отношением 
называется семейство морфизмов Беллмана $\{ \tilde{P}_{\tilde{X}_k},\ k\hm= 
\overline{1, T}\}$ из уравнений~(\ref{e4.2-vas}).
    
    \smallskip
    
    \noindent
    \textbf{Теорема~4.1.} \textit{Во всякой ДМАС существует ре\-зу\-ль\-ти\-ру\-ющее 
отношение.}
    
    \smallskip
    
    \noindent
    Д\,о\,к\,а\,з\,а\,т\,е\,л\,ь\,с\,т\,в\,о\,.\ \ Построение ДРО проведем методом 
математической индукции. На начальном шаге, в~терминальном со\-сто\-янии сис\-те\-мы, 
ДРО совпадает с~ре\-зуль\-ти\-ру\-ющим отношением $\tilde{P}_{X^T} \triangleq 
P_{X^T}$, $\tilde{X}^T\hm= X^T$, статической игры~$\Gamma(X^T)$. Опус\-тим 
этот морфизм на все множества $\tilde{X}^{T-1}$, для которых имеет мес\-то 
непосредственное предшествование $\Gamma(\tilde{X}^{T-1}) \overset{R}{\prec} 
\Gamma(\tilde{X}^T)$. Композиция морфизмов~(\ref{e4.1-vas}), $\tilde{Y}_1\hm= 
\tilde{X}^{T-1}$, $\tilde{Y}_2\hm= \tilde{X}^T$, определяет компоненту ДРО 
$\tilde{P}_{\tilde{X}^{T-1}}$, от\-ве\-ча\-ющую со\-сто\-янию сис\-те\-мы $(\tilde{X}^{T-1}, 
P_{\tilde{X}^{T-1}})$. Пусть отношение~$\tilde{P}_{\tilde{X}_i}$ уже построено. 
Тогда опять по формуле~(\ref{e4.1-vas}) его можно продолжить на все со\-сто\-яния 
$\Gamma(\tilde{X}_j) \overset{R_{ij}}{\prec} \Gamma(\tilde{X}_i)$ исходной ДМАС, 
т.\,е.\ построить очередные морфизмы~$\tilde{P}_{\tilde{X}_j}$. Процесс 
завершается в~начальных состояниях сис\-темы. 

\vspace*{-11pt}

\section{Поиск рационального решения игры}

\vspace*{-1pt}

    Для решения задачи~(\ref{e1.1-vas}), (\ref{e2.1-vas}), (\ref{e2.2-vas}),  
(\ref{e2.4-vas}) предложим сле\-ду\-ющее обобщение метода динамического\linebreak 
программирования. Сначала из системы уравнений Беллмана~(\ref{e4.2-vas}) 
найдем ДРО игры. Затем последовательно для моментов времени $1,\ldots, T$ 
вычислим сле\-ду\-ющие множества максимальных \mbox{элементов} отношения Беллмана:

\pagebreak



\noindent
    \begin{multline}
    X_1^*=\mathop{\mathrm{ARGMAX}}\,\tilde{P}_{\tilde{X}_1};\enskip 
    X_2^*=\mathop{\mathrm{ARGMAX}}\limits_{RX_1^*}\, \tilde{P}_{\tilde{X}_2}; \ldots  \\
\ldots ;    X_T^* = \mathop{\mathrm{ARGMAX}}\limits_{RX^*_{T-1}}\,\tilde{P}_{\tilde{X}_T}.
       \label{e5.1-vas}
    \end{multline}
    
\vspace*{-3pt}

    \noindent
    \textbf{Теорема~5.1.}\ \textit{Рациональному решению динамической игровой 
задачи}~(\ref{e1.1-vas})--(\ref{e2.4-vas}) \textit{отвечает ситуация 
$\tilde{x}^*\hm= (x_1^*, \ldots , x_T^*)$; $x_s^*\hm\in X_s^*$, $s\hm= \overline{1, T}$, 
найденная методом Беллмана}~(\ref{e4.2-vas}), (\ref{e5.1-vas}).
    
    \smallskip
    
    \noindent
    \textbf{Пример~5.1.} Графом~$\Gamma_R$~(\ref{e2.3-vas}), $n\hm=2$, 
динамического отношения задана ДМАС 

\vspace*{-4pt}

\noindent
    \begin{multline*}
    R: (x_1, x_2, x_3) \to (x_3, x_1, x_1\vee x_2) \cup (x_2, x_3, x_1\wedge x_2);\\
    x=(x_1, x_2, x_3)\in\underline{8}\simeq \{0,1\}^3.
    \end{multline*}
    
    \vspace*{-3pt}
    
    \noindent
Терминальная задача представляет собой игру Гермейера трех лиц 
$\Gamma^{2,3}$~\cite{1-vas, 5-vas}, где $X^T\hm=\underline{8}$~--- бинарный куб. 
Отношения предпочтений агентов $\tilde{\rho}_2^k \hm\in \mathrm{REL}\,(\underline{8},\underline{8})$ равны:
\begin{align*}
\tilde{\rho}_2^1 &=\{ 01, 05, 40, 34, 32, 76\};\\
\tilde{\rho}_2^2&= \{ 02, 10, 24, 32, 35, 45, 64, 67\};\\
\tilde{\rho}_2^3 &=\{ 20, 40, 35, 31, 64, 76\}.
\end{align*}
Найдем интересы игроков $\rho_1^k\hm= R^{\mathrm{op}}(\tilde{\rho}_2^k)$ в~начальном 
состоянии сис\-те\-мы~$\Gamma^{1,3}$, $X^0\hm= \underline{8}$: 

\vspace*{-4pt}

\noindent
\begin{multline*}
\rho_1^1= \{ 01, 02, 03, 04, 06, 12, 13, 14, 16, 20, 21, 62, 63,\\
 64, 65, 72, 73, 74, 75\}\,;
 \end{multline*}
 
 \vspace*{-14pt}
 
 \noindent
 \begin{multline*}
 \rho_1^2=\{ 03, 04, 05, 13, 14, 15, 20, 21,23, 26, 32, 40, 41,\\
  42, 52, 61, 63, 64, 65, 71, 73, 74, 75, 76\};
  \end{multline*}
  
  \vspace*{-14pt}
 
 \noindent
 \begin{multline*}
\rho_1^3 = \{ 20, 21, 30, 31, 40, 41, 50, 51, 61, 62, 63, 64, 71,\\
 72, 73, 74, 76\}.
\end{multline*}

\vspace*{-5pt}

\noindent
Вычислим результирующие отношения статических подыгр $\Gamma(X^T)$, 
$\Gamma(X^0)$ ~\cite{5-vas}:

\vspace*{-4pt}

\noindent
\begin{multline*}
P_{X^T} =\tilde{\rho}_1^3 \vert_{x_3} \circ \tilde{\rho}_1^2 \circ \left( \tilde{\rho}_1^1 
\cup \tilde{\rho}_1^{2G}\right),\\
\rho_{1}^{2G} =\tilde{\rho}_1^2 \vert_{\mathrm{MIN} \tilde{\rho}_1^2\vert_{{x_1}}}
\Rightarrow P_{X^T} =\{ 40, 05, 42, 35, 30, 70\},\\
P_{X^0} =\rho_1^1\vert_{x_1} \circ \rho_1^2\vert_{x_2}\circ\rho_1^3\vert_{x_3} =\{ 41, 53, 61, 73\}.
\end{multline*}

\vspace*{-4pt}

\noindent
Из уравнений Беллмана~(\ref{e4.2-vas}) найдем искомое ДРО игры:
$$
\tilde{P}_{X^T} =\{ 40, 05, 42, 35, 30, 70\},\ \tilde{P}_{X^0} = \{40, 00, 60, 70\}.
$$

\vspace*{-3pt}

\noindent
По формулам~(\ref{e5.1-vas}) и~тео\-ре\-ме~5.1 рациональное решение игры равно 
$\tilde{x}^*\hm=(0,0)$.

\vspace*{-12pt}

\section{Заключение}

\vspace*{-4pt}

    Функторная модель стала наследником традиционной формы представления 
ДМАС. Обладая композициональной структурой, она создает условия для 
модификации игровой задачи и~выполнения эквивалентных преобразований 
средствами алгебры тео\-рии категорий. Доказано существование ре\-зуль\-ти\-ру\-юще\-го 
отношения динамической игры многих лиц. Для задач оптимального управ\-ле\-ния 
в~форме динамических отношений предложено обобщение метода Беллмана.  
С~по\-мощью ДРО этим методом строится рациональное решение задачи. 
Функторный подход поддерживается компьютерной алгеброй тео\-рии категорий. 
Сетевая архитектура применяемых морфизмов допускает эффективную 
нейросетевую программную реализацию, которую еще только пред\-сто\-ит 
осуществить. 

\vspace*{-12pt}

{\small\frenchspacing
 {\baselineskip=10.6pt
 %\addcontentsline{toc}{section}{References}
 \begin{thebibliography}{99}
 
\vspace*{-3pt}

\bibitem{2-vas}
\Au{Моисеев~Н.\,Н.} Элементы теории оптимальных сис\-тем.~--- М.: Наука, 1974. 526~с.

\bibitem{1-vas} %2
\Au{Гермейер~Ю.\,Б.} Игры с~непротивоположными интересами.~--- М.: Наука, 1976. 326~с.

\bibitem{3-vas}
\Au{Красовский Н.\,Н., Субботин~А.\,И.} Позиционные дифференциальные игры.~--- M.: Наука, 
1976. 456~с.
\bibitem{4-vas}
\Au{Dockner E.\,J., Jorgensen~S., Long~N.\,V., Sorger~G.} Differential games in economics and management
science.~--- Cambridge: Cambridge University 
Press, 2000. 382~p. doi: 10.1017/CBO9780511805127.
\bibitem{5-vas}
\Au{Васильев Н.\,С.} Композициональное представление структуры игры многих лиц 
в~моноидальной категории бинарных отношений~// Информатика и~её применения, 2023. Т.~17. 
Вып.~2. С.~18--26. doi: 10.14357/19922264230203. EDN: GPMZTS.
\bibitem{9-vas}  %6
\Au{Dixit~A.\,K., Natebuff~B.\,J.} The art of strategy.~--- New York; London: W.\,W.~Norton~\&~Co., 
2008. 446~p.

\bibitem{7-vas}
\Au{Shoham~Y., Leyton-Brown~R.} Multiagent systems: Algorithmic, game-theoretic, and logical 
foundations.~--- Cambridge: Cambridge University Press, 2010. 532~p.
\bibitem{6-vas} %8
\Au{Bai~Q., Ren~F., Fujita~K., Znang~M.} Multi-agent and complex systems.~--- Studies in computational 
intelligence ser.~--- Springer Singapore, 2016. Vol.~670. 210~p.
\bibitem{8-vas} %9
\Au{Dixit~A.\,K., Skeath~S., Reiley~W.\,W., Jr.} Games of strategy.~--- New York; London: 
W.\,W.~Norton \&~Co., 2017. 880~p.

\bibitem{10-vas}
\Au{Скорняков Л.\,А.} Элементы общей алгебры.~--- М.: Наука, 1983. 272~с.

%\pagebreak

\bibitem{11-vas}
\Au{Маклейн~С.} Категории для работающего математика~/ Пер.\ с~англ.~--- М.: Физматлит, 2004. 352~с. 
(\Au{Mac Lane~S.} Categories for the working mathematician.~--- 2nd ed.~--- New York, NY, USA: 
Springer, 1998. 318~p.)
\bibitem{12-vas}
\Au{Губко М.\,В.} Управление организационными системами с~сетевым взаимодействием агентов. 
Обзор теории сетевых игр~// Автоматика и~телемеханика, 2004. №\,8. С.~115--132.
\bibitem{13-vas}
Group formation in economics: Networks, clubs, and coalitions~/ Eds.\ G.~Demange, M.~Wooders.~--- 
Cambridge: Cambridge University Press, 2005. 475~p.

\end{thebibliography}

 }
 }

\end{multicols}

\vspace*{-9pt}

\hfill{\small\textit{Поступила в~редакцию 02.02.24}}

%\vspace*{6pt}

\pagebreak

%\newpage

\vspace*{-28pt}

%\hrule

%\vspace*{2pt}

%\hrule



\def\tit{ON FUNCTOR REPRESENTATION OF~OPTIMIZED\\ DYNAMIC~MULTIAGENT~SYSTEMS}


\def\titkol{On functor representation of~optimized dynamic multiagent systems}


\def\aut{N.\,S.~Vasilyev}

\def\autkol{N.\,S.~Vasilyev}

\titel{\tit}{\aut}{\autkol}{\titkol}

\vspace*{-15pt}


\noindent
N.\,E.~Bauman Moscow State Technical University, 5-1~Baumanskaya 2nd Str., Moscow 105005, Russian 
Federation

\def\leftfootline{\small{\textbf{\thepage}
\hfill INFORMATIKA I EE PRIMENENIYA~--- INFORMATICS AND
APPLICATIONS\ \ \ 2024\ \ \ volume~18\ \ \ issue\ 2}
}%
 \def\rightfootline{\small{INFORMATIKA I EE PRIMENENIYA~---
INFORMATICS AND APPLICATIONS\ \ \ 2024\ \ \ volume~18\ \ \ issue\ 2
\hfill \textbf{\thepage}}}

\vspace*{4pt}



\Abste{Functors' topoi is chosen as a computational tool for synthesizing dynamic multiagent systems (DMAS). The scale orders the objects as 
multiagent system states to solve attendant static subgames in them. 
The initial dynamic game and all static subproblems are represented in the monoidal category of binary relations. 
Players' preference relations might be maximized in DMAS. The game rational solution is understood as
 equilibrium. The 
compositional structure of the optimized DMAS can be described in the form of the game dynamic resulting relation (DRR). 
Players' rational behavior search is reduced to DRR subsequent maximization. For this purpose, the Bellman's method 
is generalized to solve control problems stated in the form of relations. 
The program implementation of the approach can be based on neural networks due to the consistency of the architectures 
of the applied relation graphs and neural networks.}

\KWE{functor category; compositionality; monoidal category; opposite image; game dynamic relation; 
static subgame; preference relation; dynamic resulting relation; rational solution; Bellman morphism}

\DOI{10.14357/19922264240201}{CLMBXC}

%\vspace*{-12pt}

%\Ack

%\vspace*{-3pt}

%    \noindent
 

  \begin{multicols}{2}

\renewcommand{\bibname}{\protect\rmfamily References}
%\renewcommand{\bibname}{\large\protect\rm References}

{\small\frenchspacing
 {%\baselineskip=10.8pt
 \addcontentsline{toc}{section}{References}
 \begin{thebibliography}{99} 

\bibitem{2-vas-1}
\Aue{Moiseev, N.\,N.} 1975. \textit{Elementy teorii optimal'nykh sistem} [Elements of optimal systems 
theory]. Moscow: Nauka. 527~p.

\bibitem{1-vas-1}
\Aue{Germeyer, Yu.\,B.} 1976. \textit{Igry s~neprotivopolozhnymi interesami} [Games with  
nonopposing interests]. Moscow: Nauka. 326~p.

\bibitem{3-vas-1}
\Aue{Krasovskiy, N.\,N., and A.\,I.~Subbotin.} 1974. \textit{Pozitsionnye differentsial'nye igry} 
[Positional differential games]. Moscow: Nauka. 456~p.
\bibitem{4-vas-1}
\Aue{Dockner, E.\,J., S.~Jorgensen, N.\,V.~Long, and G.~Sorger.} 2000. \textit{Differential games in 
economics and management science}. Cambridge: Cambridge University Press. 382~p. doi: 
10.1017/CBO9780511805127.
\bibitem{5-vas-1}
\Aue{Vasilyev, N.\,S.} 2023. Kompozitsional'noe predstavlenie struktury igry mnogikh lits 
v~monoidal'noy kategorii binarnykh otnosheniy [Multiplayers' games compositional structure in the 
monoidal category of binary relations]. \textit{Informatika i~ee Primeneniya~--- Inform. Appl.} 
 17(2):18--26. doi: 10.14357/19922264230203. EDN: GPMZTS.
 
 \bibitem{9-vas-1} %6
\Aue{Dixit, A.\,K., and B.\,J.~Nalebuff.} 2008. \textit{The art of strategy}. New York, London: 
W.\,W.~Norton \&~Co. 446~p.


\bibitem{7-vas-1}
\Aue{Shoham, Y., and R.~Leyton-Brown.} 2010. \textit{Multiagent systems: Algorithmic,  
game-theoretic, and logical foundations}. Cambridge University Press. 532~p.

\bibitem{6-vas-1} %8
\Aue{Bai, Q., F.~Ren, K.~Fujita, and M.~Znang.} 2016. \textit{Multi-agent and complex systems}. Studies 
in computational intelligence ser.  Springer Singapore. 210~p.

\bibitem{8-vas-1} %9
\Aue{Dixit, A.\,K., S.~Skeath, and D.\,H.~Reiley, Jr.} 2017. \textit{Games of strategy}. New York, 
London: W.\,W.~Norton \&~Co.\linebreak 880~p.

\bibitem{10-vas-1}
\Aue{Skornyakov, L.\,A.} 1983. \textit{Elementy obshchey algebry} [Elements of general algebra]. 
Moscow: Nauka. 272~p.
\bibitem{11-vas-1}
\Aue{Mac Lane, S.} 1998. \textit{Categories for the working mathematician}. 2nd ed. New York, NY: 
Springer. 318~p. 
\bibitem{12-vas-1}
\Aue{Gubko, M.\,V.} 2004. Control of organizational systems with network interaction of agents. 
II.~Stimulation problems. \textit{Automat. Rem. Contr.} 65(9):1470--1485. doi: 
10.1023/B:AURC.0000041425.34118.7d. EDN: \mbox{LFMUCG}.
\bibitem{13-vas-1}
Demange, G., and M.~Wooders, eds. 2005. \textit{Group formation in economics: Networks, clubs, and 
coalitions}. Cambridge: Cambridge University Press. 475~p. 

\end{thebibliography}

 }
 }

\end{multicols}

\vspace*{-8pt}

\hfill{\small\textit{Received February 2, 2024}} 

\vspace*{-18pt}


\Contrl

\vspace*{-3pt}

\noindent
\textbf{Vasilyev Nikolai S.} (b.\ 1952)~--- Doctor of Science in physics and mathematics, professor, 
N.\,E.~Bauman Moscow State Technical University, 5-1 Baumanskaya 2nd Str., Moscow 105005, Russian 
Federation; \mbox{nik8519@yandex.ru}





\label{end\stat}

\renewcommand{\bibname}{\protect\rm Литература}    %5
\def\stat{malashenko}

\def\tit{ПОСЛЕДОВАТЕЛЬНЫЙ АНАЛИЗ И~МЕТРИЧЕСКИЕ ОЦЕНКИ ПРЕДЕЛЬНЫХ
РАСПРЕДЕЛЕНИЙ МЕЖУЗЛОВЫХ ПОТОКОВ В~МНОГОПОЛЬЗОВАТЕЛЬСКОЙ СЕТИ}

\def\titkol{Последовательный анализ и~метрические оценки предельных
распределений межузловых потоков в %~многопользовательской 
сети}

\def\aut{Ю.\,Е. Малашенко$^1$}

\def\autkol{Ю.\,Е. Малашенко}

\titel{\tit}{\aut}{\autkol}{\titkol}

\index{Малашенко Ю.\,Е.}
\index{Malashenko Yu.\,E.}


%{\renewcommand{\thefootnote}{\fnsymbol{footnote}} \footnotetext[1]
%{Исследование выполнено при финансовой поддержке Российского научного фонда (проект 
%<<Информатика>> ФИЦ ИУ РАН, Москва).}}


\renewcommand{\thefootnote}{\arabic{footnote}}
\footnotetext[1]{Федеральный исследовательский центр <<Информатика и~управление>> Российской академии 
\mbox{mala-yur@yandex.ru}}


%\vspace*{-6pt}



\Abst{Для оценки функциональных возможностей
многопользовательской сети связи аналилизируется множество векторов межузловых потоков при предельных распределениях ресурсов
сети. В~рамках многопродуктовой модели про\-пуск\-ные спо\-соб\-ности ребер рас\-смат\-ри\-ва\-ют\-ся 
как компоненты вектора ресурсов различных
типов, которые требуются для передачи потоков различных видов.
При проведении вычислительных экспериментов на каждой итерации вычисляются нормы векторов совместно допустимых межузловых
потоков, при передаче которых полностью используется пропускная спо\-соб\-ность всех ребер сети. Полученные последовательности
метрических оценок позволяют анализировать особенности множества до\-сти\-жи\-мости и~эф\-фек\-тив\-ность использования ресурсов сети при
уравнительном распределении про\-пуск\-ной спо\-соб\-ности между корреспондентами.}

\KW{многопродуктовая потоковая сетевая
модель; множество достижимых межузловых потоков; предельные
распределения пропускной способности}

\DOI{10.14357/19922264220306} 
  
%\vspace*{-3pt}


\vskip 10pt plus 9pt minus 6pt

\thispagestyle{headings}

\begin{multicols}{2}

\label{st\stat}

\section{Введение}

Данная работа продолжает исследования функциональных характеристик
сетевых сис\-тем связи~[1]. В~настоящее время математические модели
передачи многопродуктового потока применяются для поиска
распределений потоков и~ресурсов в~многопользовательских
телекоммуникационных\linebreak сетях~[2--4]. Разрабатываются методы анализа
с~учетом вектора требований всех \textit{равноправных} 
и~невзаимозаменяемых корреспондентов~[5]. С~позиций\linebreak методологии
исследования операций изучаются справедливые распределения потоков
и~ресурсов~[6].

Соответствующие \textit{недискриминирующие} правила управления
потоками являются решениями задач на максмин и/или получаются 
в~результате использования процедур последовательной
лексикографически упорядоченной оптимизации~[7].

В~настоящей работе пути соединения корреспондентов прокладываются
через со\-от\-вет\-ст\-ву\-ющие минимальные разрезы. Указанный метод\linebreak \mbox{можно}
рассматривать как возможный вариант решения задачи о~построении
SPLIT-марш\-ру\-тов~[8,~9]. В~рамках вычислительных экспериментов\linebreak на
многопродуктовой модели анализируются распределения межузловых
потоков  и~пропускной способ\-ности сети.  Для оценки функциональных
возможностей многопользовательской сети используется вектор
совместно допустимых межузловых потоков. Под ресурсом, выделяемым
некоторой паре узлов-кор\-рес\-пон\-ден\-тов,  понимается суммарное
значение тре\-бу\-емых пропускных способностей на всех ребрах,
расположенных на всех маршрутах при прохождении межузлового\linebreak потока
данного вида.  Сумма соответствующих реберных потоков трактуется
как полная нагрузка на сеть, возникающая при передаче заданного
межузлового потока. Рас\-смат\-ри\-ва\-ют\-ся распределения пропускной
способности и~межузловых потоков при предельной загрузке сети.
При проведении вычислительных экспериментов на каждой  итерации
вычисляется норма  вектора совместно допустимых межузловых
потоков.   Для оценки величины требуемых ресурсов при соединении
корреспондентов по различным путям для каж\-дой пары узлов
определяется максимальный однопродуктовый поток. Марш\-ру\-ты передачи
всех допустимых межузловых потоков  проходят по ребрам
соответствующих минимальных разрезов. Вычислительные эксперименты
проводились  для получения последовательности  мет\-ри\-че\-ских оценок
векторов межузловых потоков, принадлежащих множеству до\-сти\-жи\-мости
многопользовательской сети.

\section{Математическая модель}

В качестве математической модели для описания
многопользовательской сетевой системы используется следующая
формальная запись условий и~ограничений, которые должны
выполняться при одновременной передаче потоков различных видов
между всеми парами улов-корреспондентов:

Сеть $G(\mathbf{d})$ задается множествами $\langle V,
R,U,P\rangle$:
\begin{itemize}
\item  узлов (вершин) сети 
$$
V=\left \{v_{1}, v_{2},\dots,v_{n},\dots,v_{N}\right\};
$$
\item  неориентированных ребер 
$$
R=\left\{r_{1}, r_{2}, \dots, r_{k}, \dots,
r_{E}\right\}.
$$
\end{itemize}

Ребро $r_{k}$ \textit{соединяет} концевые вершины~$v_{n_k}$ и~$v_{j_k}$. 
Реб\-ру~$r_{k}$ ставятся в~соответствие две
ориентированные дуги $\{u_{k},u_{k+E}\}$ из множества
ориентированных дуг $U\hm=\{u_{1}, u_{2}, \dots, u_{k}, \dots,
u_{2E}\}$. Дуги $\{u_{k}, u_{k+E}\}$ определяют прямое и~обратное
на\-прав\-ле\-ние передачи потока по реб\-ру~$r_{k}$ между концевыми
вершинами $\{v_{n_k}, v_{j_k}\}$.

В многопользовательской сети~$G(\mathbf{d})$ рассматривается
$M\hm=N(N\hm-1)$ независимых, невзаимозаменяемых и~равноправных потоков
различных видов, которые передаются между уз\-ла\-ми-кор\-рес\-пон\-ден\-та\-ми
из множества 
$$
P=\left\{p_{1}, p_{2}, \dots, p_{M}\right\}.
$$

По определению, каждой паре уз\-лов-кор\-рес\-пон\-ден\-тов~$p_{m}$
соответствуют:
\begin{itemize}
\item вершина-ис\-точ\-ник с~номером~$s_{m}$, через которую входной поток
$m$-го вида~$z_{m}$ поступает в~сеть;
\item  вершина-при\-ем\-ник с~номером~$t_{m}$, из которой поток $m$-го
вида~$z_{m}$ покидает сеть.
\end{itemize}

В множестве~$P$ выделяется подмножество $P(R^{+})$ пар
уз\-лов-кор\-рес\-пон\-ден\-тов, расположенных в~концевых вершинах
ребра~$r_{k}$, $k\hm=\overline{1,E}$. Вводятся сле\-ду\-ющие обозначения:
пусть ребро~$r_{k}$  с~номером~$k$ соединяет вершины с~номерами~$n$ и~$j$ такими, что $n\hm< j$. Для со\-от\-вет\-ст\-ву\-ющей пары
уз\-лов-кор\-рес\-пон\-ден\-тов~$p_{k}$, расположенных в~узлах $\{v_{n},
v_{j}\}$, узел~$v_{n}$ считается источником, а узел~$v_{j}$~---
приемником потока $z_{k}$ $k$-го вида, который передается из узла
c номером~$n$ в~узел с~номером~$j$ для пары~$p_{k}$ с~номером~$k$.
Для пары $p^{}_{k+E} \Longleftrightarrow \{v_{j},v_{n}\}$ узел~$v_{j}$ 
считается источником~$s_{k+E}$, а~узел $v_m$~--- приемником~$t_{k+E}$ для пары с~номером~$p_{k+E}$. Формируется
$R^+\hm=\{1,2,3,\dots,E,E+1,\dots,2E\}$~--- список номеров смежных
пар.

Пары $p_{k}$ из подмножества~$P(R^{+})$ называются
\textit{смежными} уз\-ла\-ми-кор\-рес\-пон\-ден\-та\-ми. Все остальные
\textit{несмежные} пары уз\-лов-кор\-рес\-пон\-ден\-тов относятся к~множеству~$P(R^{-})$:
\begin{equation*}
P=P(R^{+})\cup P(R^{-});\quad
P(R^{+}) \cap P(R^{-}) = \emptyset.
\end{equation*}

Введем обозначения:
\begin{description}
\item[\,]
$z_{m}$~--- величина \textit{межузлового} потока $m$-го вида,
который поступает в~сеть из узла с~номером~$s_{m }$ и~покидает из
узла с~номером~$t_{m}$;
\item[\,]
$S(v_{n})$~--- множество номеров исходящих дуг, по которым поток
покидает узел~$v_{n}$;
\item[\,]
$T(v_{n})$~--- множество номеров входящих дуг, по которым поток
поступает в~узел~$v_{n}$.
\end{description}

Во всех узлах $v_{n}\in V$, $n\hm=\overline{1,N}$, для всех видов
потоков должны выполняться условия сохранения потоков:
\begin{multline}
\label{eq1} 
\sum\limits_{i\in S(v_n )} x_{mi}-\sum\limits_{i\in T(v_n )} x_{mi}
={}\\
{}=\begin{cases}
z_m, & \mbox{если } v=v^{}_{S_m}; \\
-z_m,&\mbox{если } v=v_{t_m}; \\
0&\mbox{в остальных случаях}, \\
\end{cases}
\end{multline}
$n=\overline{1,N}$, $m\hm=\overline{1,M}$, $x_{mi}\hm\ge 0$,
$z_{m}\hm\ge0$.

Величина {z}$_{m}$ равна входному потоку $m$-го вида, который
пропускается от источника к~приемнику пары $p_{m}$ при
распределении потоков $x_{mi}$ по дугам сети.

Каждому ребру $r_{k}\hm\in R$ приписывается неотрицательное число~$d_{k}$, 
определяющее суммарный предельно допустимый поток,
который можно передать по реб\-ру~$r_{k}$ в~обоих на\-прав\-ле\-ни\-ях. 
В~исходной сети компоненты вектора про\-пуск\-ных способностей
$\mathbf{d}\hm=(d_{1}, d_{2},\dots, d_{k}, \dots, d_{E})$~--- наперед
заданные положительные числа $d_{k}
\hm> 0$. Вектором $\mathbf{d}$ определяются сле\-ду\-ющие ограничения на сумму
дуговых потоков всех видов, пе\-ре\-да\-ва\-емых по реб\-ру~$r_{k}$:
\begin{multline}
\sum\limits_{m=1}^M (x_{mk}+x_{m(k+E)}) \le d_k,\\
 x_{mk}\ge 0\,,\enskip
 x_{m(k+E)}\ge 0\,, \enskip k=\overline {1,E}\,.
 \label{eq2} 
\end{multline}
В рамках данной модели пропускная спо\-соб\-ность ребер сети~--- вектор~$\mathbf{d}$~--- трактуется как <<\textit{ресурсное ограничение}>>,
а~сумма дуговых
 потоков рас\-смат\-ри\-ва\-ет\-ся как показатель использования
<<\textit{ресурсов}>> сети при передаче межузловых потоков
различных видов.

Для всех $z_{m}$ и~$x_{mi}$, удовлетворяющих
условиям~\eqref{eq1} и~\eqref{eq2}, вычисляются суммарные потоки:
\begin{equation}
 y_{m }=\sum\limits_{i=1}^{2E} {x}_{mi},\quad
m=\overline{1,M}\,.
\label{eq3}
\end{equation}

Суммарный реберный поток~$y_{m}$ характеризует
<<\textit{нагрузку}>> на сеть при передаче межузлового потока
величины $z_{m}$ из уз\-ла-ис\-точ\-ни\-ка~$s_{m}$ в~узел-при\-ем\-ник~$t_{m}$. 
Величина~$y_{m}$ показывает, какой суммарный
\textit{ресурс}~-- пропускная спо\-соб\-ность сети~-- требуется для
передачи межузлового потока~$z_{m}$, а~отношение
$w_{m}\hm={y_m}/{z_m}$,  $m\hm=\overline{1,M},$
показывает, какие \textit{ресурсы} необходимы для передачи
единичного потока $m$-го вида между узлами~$s_{m}$ и~$t_{m}$.

Ограничения~\eqref{eq1}--\eqref{eq3} задают подмножество
допустимых значений компонент вектора межузловых потоков
$\mathbf{z}\hm=\left(z_{1}, z_{2},\dots,z_{m},\dots,z_{M}\right)$:
\begin{equation*}
{Z}(\mathbf{d})=\left\{\mathbf{z} \ge 0 \mid
(\mathbf{z},\mathbf{x},\mathbf{y}) \ \mbox{удовлетворяют~\eqref{eq1}--\eqref{eq3}}
\right\}\!,
\!\!
%\label{eq4} 
\end{equation*}
а все допустимые распределения ресурсов принадлежат подмножеству
\begin{equation*}
{Y}(\mathbf{d})=\left\{\mathbf{y} \ge 0 \mid
(\mathbf{z},\mathbf{x},\mathbf{y}) \ \mbox{удовлетворяют~\eqref{eq1}--\eqref{eq3}}\right\}\!.
%\!\!\!\label{eq5}
\end{equation*}


\section{Метрические оценки предельных распределений}

Для оценки функциональных возможностей сис\-те\-мы рассматриваются
допустимые распределения межузловых потоков при предельных
загрузках ребер сети.

В рамках данного модельного описания монопольным режимом
называется способ управления, при котором все ресурсы сети
используются для передачи потока одной выделенной пары
уз\-лов-кор\-рес\-пон\-ден\-тов $p_{a}\hm\in P(R^-)$, а для всех
остальных потоки полагаются равными нулю.

Предельно допустимый поток, который можно передать между
фиксированной парой уз\-лов-кор\-рес\-пон\-ден\-тов $p_{a}$ в~монопольном
режиме, является решением стандартной, в~данном случае
однопродуктовой, задачи о~максимальном потоке.

\smallskip

\noindent
\textbf{Задача 1.} Найти
$z_a^0\hm=\max\limits_{\langle z,x\rangle \in Z(d)} z_a
$
при условии $z_{i}=0$, $i\hm=\overline{1,M}$, $i\hm\ne a$.

При решении задачи~1 для пары $p_{a}$ вы\-чис\-ля\-ют\-ся: межузловой
поток~$z_a^0$; дуговые потоки $\{x^{0}_{ak};x^{0}_{a(k+E)}\}$,
$k\hm=\overline{1,E}$; суммарное значение реберного
потока~$y_{a}^{0}\hm=\sum\nolimits_{i=1}^{2E} {x}_{ai}^{0}$.

Поток величины $z_a^0$ является \textit{максимальным потоком},
пе\-ре\-да\-ва\-емым в~\textit{монопольном режиме} для пары
уз\-лов-кор\-рес\-пон\-ден\-тов~$p_{a}$, $p_{a}\hm\in P(R^-)$, в~сети~$G(d)$.

Задача~1 решается последовательно для всех $p_{m}\in P(R^-)$,
вы\-чис\-ля\-ют\-ся значения $z_{m}^{0}(t)$.

При проведении вычислительных экспериментов использовалась
итерационная процедура для оценки функциональных возможностей
сис\-те\-мы при передаче межузловых потоков по нескольким маршрутам.
На предварительном этапе шага~$t$ в~сети~$G(t)$ при заданных
значениях пропускной спо\-соб\-ности ребер~$d_k(t)$ для каждой \mbox{пары}
уз\-лов-кор\-рес\-пон\-ден\-тов $p_m\hm\in P(R^-)$ определяется максимально
допустимый однопродуктовый поток~$z^0_m(t)$, со\-от\-вет\-ст\-ву\-ющие
дуговые потоки $(x_{mk}^0(t),x_{m(k+E)}^0(t))$, $p_m\hm\in P(R^-)$, и~коэффициенты нормировки
$\xi_m^0(t)\hm={1}/{z_m^0(t)}$ для всех  $p_m\hm \in P(R^-)$,
таких что $z^0_m(t)\hm>0$, $y_m^0(t)\hm>0$.
Коэффициенты~$\xi_m^0(t)$ используются для поиска текущих
совместно допустимых квот на передачу потоков одновременно между
всеми парами $p_m\in P(R^-)$.

\smallskip

\noindent
\textbf{Задача 2.} Найти $\alpha^*(t)=\max\limits_\alpha \alpha$
при условиях
$$
\alpha\!\!\sum\limits_{m\in R^-}\! \xi_m^0\left(x_{mk}^0(t)+x_{m(k+E)}^0(t)\right)\le d_k(t),\enskip
k=\overline{1,E}\,.
$$

На основании $\alpha^*(t)$ вычисляются совместно допустимые
дуговые потоки:
\begin{multline*}
x_{mk}^*(t)=\alpha^*(t)\xi^0_m(t)x^0_{mk}(t),\\
x^*_{m(k+E)}(t)=\alpha^*(t)\xi^0_m(t)x^0_{m(k+E)}(t),
\\
m=\overline{1,M}\,,\enskip k=\overline{1,E}\,,
\end{multline*}
и остаточная пропускная способность ребер в~сети $G(t+1)$:
\begin{multline*}
d_k(t+1)=d_k(t)-\sum_{m\in R^-} (x_{mk}^*(t)+x_{m(k+E)}(t)),\\
k=\overline{1,E}\,,\enskip p_m\in P(R^-).
\end{multline*}
Формируется вектор допустимых межузловых потоков:
\begin{align*}
z_k^+(t)&=d_k(t+1),\enskip p_k\in P(R^+),\enskip k=\overline{1,E}\,;
\\
z_m^-(t)&=\sum\limits_{\tau=1}^t\alpha^*(\tau)\xi_m^0(\tau) z_m^0(\tau), \enskip p_m\in P(R^-).
\end{align*}

По построению, на шаге~$t$ при передаче вектора межузлового потока
$\mathbf{z}(t)=\{\mathbf{z}^+(t), \mathbf{z}^-(t)\}$ достигается
предельная загрузка, и~пропускная способность всех ребер  сети
используется полностью.

Точка с~координатами $\mathbf{z}(t)$ принадлежит множеству~$Z(d)$.

Расстояние точки от начала координат определяется как норма
соответствующего вектора:
\begin{align*}
\rho^+(t)&=\|\mathbf{z}^+(t)\|=
\left[\,\sum\limits_{k=1}(\mathbf{z}^+(t))^2\right]^{1/2};
\\
\rho^-(t)&=\|\mathbf{z}^-(t)\|= \left[\sum\limits_{p_m\in P(R^-)}(\mathbf{z}_m^-(t))^2\right]^{1/2}.
\end{align*}

Если при выполнении шага $(t+1)$ окажется, что $z_m^0(t+1)=0$ для
всех $p_m\in P(R^-)$, то про\-изойдет останов и~сформируются
массивы финальных данных:
\begin{align*}
z_m^-(T)&=\sum\limits_{\tau=1}^t \alpha^*(\tau)\xi_m^0(\tau) z_m^0(\tau),\enskip 
p_m\in P(R^-),\\
z_k^+(T)&=d_k(t+1),\enskip p_k\in P(R^+),\enskip k=\overline{1,E}\,.
\end{align*}

Вышеописанная вычислительная процедура далее обозначается как
MFPL-про\-це\-ду\-ра (от англ.\ \textit{max-flow-peak-load}).

При проведении второй серии вычислительных экспериментов
MFPL-про\-це\-ду\-ра использовалась для оценки функциональных
характеристик сис\-те\-мы при \textit{уравнительном} поэтапном
распределении пропускной способности между всеми
па\-ра\-ми-кор\-рес\-пон\-ден\-тами.

При реализации MFPL-процедуры выполнение каждого шага разбивается
на несколько этапов. На предварительном этапе шага~$t$ 
в~сети~$G(t)$ при заданных значениях пропускной способности ребер~$d_k(t)$ 
для каждой пары уз\-лов-кор\-рес\-пон\-ден\-тов $p_m\hm\in P(R^-)$
определяется максимально допустимый однопродуктовый
поток~$z_m^0(t)$, соответствующие дуговые потоки
$\left(x_{mk}^0(t),x_{m(k+E)}^0(t)\right)$, $p_m\hm\in P(R^-)$, и~суммарная
реберная нагрузка
$$
y_m^0(t)=\sum\limits_{k=1}^E (x_{mk}^0(t),x_{m(k+E)}^0(t)),\enskip p_m\in P(R^-).
$$

Для каждой пары $p_m\hm\in P(R^-)$ вычисляются коэффициенты
нормировки
$\theta_m^0(t)\hm={1}/{y_m^0(t)}$ для всех  
$p_m\hm\in P(R^-)$, таких что  $z^0_m(t)\hm>0$,
$y_m^0(t)\hm>0$.
Коэффициенты~$\theta_m^0(t)$ используются для поиска совместно
допустимых дуговых потоков для всех $p_m\hm\in P(R^-)$.

\smallskip

\noindent
\textbf{Задача 3.} Найти $\beta^*(t)=\max\nolimits_\beta \beta$ при
условиях
$$
\beta\!\!\!\!\sum\limits_{p_m\in P(R^-)}\!\!
\theta_m^0(x_{mk}^0(t)+x_{m(k+E)}^0(t))\le d_k(t),\enskip
k=\overline{1,E}\,.
$$

 С помощью $\beta^*(t)$ (решения задачи~3) вычисляются текущие допустимые значения дуговых потоков:
\begin{multline*}
x_{mk}^*(t)=\beta^*(t)\theta^0_m(t)x^0_{mk}(t),\\
x^*_{m(k+E)}(t)=\beta^*(t)\theta^0_m(t)x^0_{m(k+E)}(t), \enskip
k=\overline{1,E},
\end{multline*}
и реберных нагрузок при одновременной передаче межузловых потоков:

\noindent
\begin{multline*}
y_m^*(t)=\sum\limits_{i=1}^E
\left[x_{mi}^*(t)+x^*_{m(i+E)}(t)\right]={}\\
{}= \fr{\beta^*(t)}{y_m^0(t)} \sum\limits_{i=1}^E
\left[x_{mi}^0(t)+x^0_{m(i+E)}(t)\right]=\beta^*(t), \\
 p_m\in P(R^-).
\end{multline*}
Таким образом на каждом шаге определенная часть имеющегося ресурса
(пропускной спо\-соб\-ности) делится строго по\-ров\-ну меж\-ду всеми
корреспондентами $p_m\in P(R^-)$, для которых существует путь
передачи в~$G(t)$.

Формируется вектор допустимых межузловых потоков:
\begin{gather*}
\hspace*{-30mm}z_k^{++}(t)=d_k(t+1)={}\hspace*{10mm}\\
{}=d_k(t)-\!\!\! \sum\limits_{p_m\in P(R^-)}\!\!\!
\left(x_{mk}^*(t)+x_{m(k+E)}(t)\right),\\
\hspace*{35mm}k=\overline{1,E}, \enskip
p_k\in P(R^+);\\
z_m^{(=)}(t)\overset{\Delta}{=}\sum\limits_{\tau=1}^t\beta^*(\tau)
\theta_m^0(\tau) z_m^0(\tau), \enskip p_m\in P(R^-).
\end{gather*}

\noindent
Определяются расстояния:
\begin{align*}
\rho^{++}(t)&=\|\mathbf{z}^{++}(t)\|\overset{\Delta}{=}
\left[\sum\limits_{k=1}^E\left(d_k(t+1)\right)^2\right]^{1/2};\\
\rho^{(=)}(t)&=\|\mathbf{z}^{=}(t)\|= \left[\sum\limits_{p_m\in
P(R^-)}\left(z_m^{(=)}(t)\right)^2\right]^{1/2}.
\end{align*}

Если на предварительном этапе на шаге $(t+1)$ окажется, что в~сети~$G(t+1)$ для всех $p_m\hm\in P(R^-)$ все значения
$z_m^0(t+1)\hm=0$, то произойдет останов и~сформируются финальные
массивы:
\begin{align*}
z_k^{(++)}(T)&=d_k(t+1), \enskip
p_k\in P(R^+), \enskip k=\overline{1,E};
\\
z_m^{(=)}(t)&=\sum\limits_{\tau=1}^{t+1}\beta^*(\tau)
\theta_m^0(\tau) z_m^0(\tau), \enskip p_m\in P(R^-).
\end{align*}



\section{Вычислительный эксперимент}

Результаты вычислительных экспериментов, описанные ниже, служат
продолжением исследований, начатых в~[1]. Вычислительные
эксперименты проводились на моделях сетевых сис\-тем, пред\-став\-лен\-ных
на рис.~1 и~2. В~каждой сети~69~узлов. Пропускные спо\-соб\-но\-сти
ребер~-- значения $d_k$~-- выбирались случайным образом из отрезка
$[900,999]$ и~совпадали для ребер, при\-сут\-ст\-ву\-ющих в~обеих сетях.
В~кольцевой сети пропускная спо\-соб\-ность каждого из добавленных
ребер равнялась~900.

\begin{figure*} %fig1
\vspace*{1pt}
\begin{minipage}[t]{80mm}
  \begin{center}  
    \mbox{%
\epsfxsize=69.408mm
\epsfbox{mal-1.eps}
}

\end{center}
\vspace*{-6pt}
\Caption{Базовая сеть}
\end{minipage}
%\end{figure*}
\hfill
%\begin{figure*} %fig2
\vspace*{1pt}
\begin{minipage}[t]{80mm}
  \begin{center}  
    \mbox{%
\epsfxsize=69.408mm
\epsfbox{mal-2.eps}
}

\end{center}
\vspace*{-6pt}
\Caption{Кольцевая сеть}
\end{minipage}
\end{figure*}

\begin{table*}[b]\small %tabl1
\vspace*{-12pt}
\begin{center}

%\renewcommand{\arraystretch}{1.1}
\Caption{Базовая сеть}
\vspace*{2ex}

\begin{tabular}{|c||c|c|c||c|c|c|} 
\hline
&&&&&&\\[-9pt]
$t$  & $\rho^{-}(t)$ & $\rho^{+}(t)$ & $d^{+}(t+1)$ &
$\rho^{=}(t)$ & $\rho^{++}(t)$&  $d^{++}(t+1)$ \\ 
\hline
\hphantom{99}0  & \hphantom{99}0   & 8048&  68256&  \hphantom{9}0   &  8048&   68256\\
1  & \hphantom{9}63  & 4182&  26544&  \hphantom{9}95  &  3881&   24476\\
$\cdots$  & $\cdots$   & $\cdots$   &  $\cdots$    &  $\cdots$   &  $\cdots$   &   $\cdots$\\
11 & \hphantom{9}70  & 3975&  21469&  \hphantom{9}101\hphantom{9} &  3707&   20155\\
$\cdots$& $\cdots$   & $\cdots$   &  $\cdots$    & $\cdots$   &  $\cdots$   &  $\cdots$\\
22 & \hphantom{9}83  & 3861&  19623&  \hphantom{9}122\hphantom{9} &  3586&   18260\\
$\cdots$ & $\cdots$  & $\cdots$   &  $\cdots$   &  $\cdots$   &  $\cdots$  &   $\cdots$\\
33 & \hphantom{9}103\hphantom{9} & 3778&  18827&  \hphantom{9}139\hphantom{9} &  3522&   17601\\
$\cdots$ &$\cdots$  &$\cdots$  & $\cdots$  & $\cdots$   &  $\cdots$  &  $\cdots$\\
44 & \hphantom{9}\bf 190\hphantom{9} & \bf3553&  \bf17503&  \hphantom{9}\bf203\hphantom{9} &  \bf3285&   \bf16201\\
45 & \hphantom{9}\bf1452\hphantom{99}& \bf2166&  \hphantom{9}\bf7069 &  \hphantom{9}\bf1376\hphantom{99}&  \bf2020&   \hphantom{9}\bf6584\\
46 & \hphantom{9}\bf1498\hphantom{99}& \bf2158&  \hphantom{9}\bf6707 &  \hphantom{9}\bf1388\hphantom{99}&  \bf2017&   \hphantom{9}\bf6483\\
$\cdots$ & $\cdots$   & $\cdots$   &  $\cdots$    & $\cdots$   &  $\cdots$   &  $\cdots$\\
52 & \hphantom{9}1535\hphantom{99}& 2155&  \hphantom{9}6413 & \hphantom{9}1442\hphantom{99} &  2011&   \hphantom{9}6059\\
\hline
\end{tabular}
\end{center}
 %\end{table*}
% \begin{table*}\small %tabl2
\begin{center}
\Caption{Кольцевая сеть}
\vspace*{2ex}


\begin{tabular}{|c||c|c|c||c|c|c|} 
\hline
&&&&&&\\[-9pt]
$t$  & $\rho^{-}(t)$ & $\rho^{+}(t)$ & $d^{+}(t+1)$ &
$\rho^{=}(t)$ & $\rho^{++}(t)$&  $d^{++}(t+1)$ \\
 \hline
\hphantom{9}0  &\hphantom{99}0    & 8440  & 75456   &\hphantom{9}0      &8440   &75456\\
\hphantom{9}1  &\hphantom{9}68   & 5317  & 43038   &92     &5045   &40716 \\ 
$\cdots$ &$\cdots$    & $\cdots$     & $\cdots$   &$\cdots$      &$\cdots$      &$\cdots$      \\
11 &\hphantom{9}95   & 3608  & 20459   &124    &3397   &19080  \\
$\cdots$ &$\cdots$   & $\cdots$    & $\cdots$      &$\cdots$     &$\cdots$     &$\cdots$   \\
22 &\hphantom{9}101\hphantom{9}  & 3540  & 19530   &130    &3350   &18338 \\
$\cdots$ &$\cdots$  & $\cdots$   &$\cdots$      &$\cdots$     &$\cdots$   &$\cdots$    \\
33 &\hphantom{9}135\hphantom{9}  & 3346  & 17561   &154    &3220   &17003 \\
$\cdots$  &$\cdots$   & $\cdots$    & $\cdots$      &$\cdots$     &$\cdots$    &$\cdots$    \\
44 &\hphantom{9}234\hphantom{9}  & 3094  & 14881   &269    &2918   &13848 \\
$\cdots$ &$\cdots$   & $\cdots$    &$\cdots$      &$\cdots$     &$\cdots$     &$\cdots$    \\
50 &\hphantom{9}\bf 413\hphantom{9}  & \bf2770  & \bf12901   &\bf329    &\bf2792   &\bf13079 \\
51 &\hphantom{9}\bf1040\hphantom{99} & \bf2299  & \hphantom{9}\bf8801    &\bf334    &\bf2784   &\bf13034 \\
52 &\hphantom{9}\bf1062\hphantom{99} & \bf2297  & \hphantom{9}\bf8672    &\bf974    &\bf2262   &\hphantom{9}\bf8768  \\
$\cdots$ &$\cdots$   &$\cdots$    & $\cdots$      &$\cdots$      &$\cdots$     &$\cdots$    \\
55 &\hphantom{9}1069\hphantom{99} & 2297  & \hphantom{9}8630    &1010\hphantom{9}   &2259   &\hphantom{9}8553  \\
\hline
 \end{tabular}
\end{center}
 \end{table*}




Для базовой сети исходная сумма пропускных способностей:
$D^+(0)\hm=68\,256$, а~для кольцевой сети $D^{++}(0)=75\,456$.
Соответствующие значения $\rho^+(0)$ и~$\rho^{++}(0)$ указаны в~<<нулевой>> строке 
в~табл.~1 и~2, где собраны результаты
вычислительных экспериментов. В~ходе эксперимента при
уравнительном распределении остаточных ресурсов соблюдается
\textit{равномерное} убывание остаточной пропускной спо\-соб\-ности и~<<\textit{длины}>> вектора~$\rho^+(t)$. 
Однако между 44--46
итерациями для базовой и~50--52 для кольцевой сети наблюдается
резкий скачок величин~$\rho^-(t)$, $\rho^{=}(t)$ и~$d^+(t)$,
$d^{++}(t)$.

На указанных шагах полностью используется пропускная способность
ребер в~центральной час\-ти сети. Сеть \textit{распадается} на
несвязные компоненты, и~для $80\%$ корреспондентов пропадают пути
соединения, а~остаточный ресурс распределяется поровну между
оставшимися парами узлов.

Анализ результатов показал, что почти равные значения потоков
достигаются для~80\% корреспондентов и~требуют 60\%--70\%
ресурсов. Однако для~2\% смежных  пар узлов межузловые потоки на
два порядка выше медианных значений, а~затраты пропускной
способности  со\-став\-ля\-ют~20\%--30\%.








\section{Заключение}

Предложенный метод и~проведенные вычислительные эксперименты
показали, что уравнительное поэтапное распределение   приводит 
к~неравномерному  распределению   потоков  для разных групп\linebreak
корреспондентов.    Метрические оценки, полученные  в~ходе
экспериментов, продемонстрировали\linebreak \textit{деформацию} множества
достижимых потоков. В~рамках модели   предполагалось, что  все
корреспонденты  равноправны, а~потоки невзаимозаменяемы,  однако
при уравнительном предельном  распределении  смежные  пары узлов
оказывались в~привилегированном положении при использовании
остаточной пропускной способности. Пропускные способности  ребер
рассматривались  как вектор   ресурсов  различных типов,  которые
распределяются между корреспондентами   при передаче  потоков
различных видов.  По построению, на каж\-дом шаге норма вектора
смежных   межузловых    потоков численно равна   модулю вектора
остаточных  пропускных способностей.   Полученные мет\-ри\-че\-ские
значения  можно использовать  для   оценки функциональных
возможностей сети  в~режиме  предельной загрузки.

{\small\frenchspacing
 {%\baselineskip=10.8pt
 %\addcontentsline{toc}{section}{References}
 \begin{thebibliography}{9}

\bibitem{1-mal}
\Au{Малашенко Ю.\,Е., Назарова И.\,А.} Неоднородность
распределения   потоков при предельной  загрузке
многопользовательской сети~//  Известия РАН. Теория и~сис\-те\-мы
управления,  2022. №\,3. С.~81--96.

\bibitem{4-mal} %2
\Au{Luss H.} Equitable resource allocation: Models,
algorithms, and applications.~--- Hoboken, NJ, USA: John Wiley \& Sons, 2012.
420~p.

\bibitem{2-mal} %3
\Au{Ogryczak W., Luss~H., Pioro~M., Nace~D., Tomaszewski~A.}   Fair
optimization and networks: A~aurvey~// J.~Appl. Math., 2014. Vol.~2014. Art.~ID~612018. 25~p. doi: 10.1155/ 2014/612018.

\bibitem{3-mal} %4
\Au{Salimifard K., Bigharaz~S.} The multicommodity network
flow problem: State of the art classification, applications, and
solution methods~// J.~Oper. Res., 2020. Vol.~18. Iss.~3. P.~1--47.



\bibitem{5-mal}
\Au{Balakrishnan A., Li~G., Mirchandani~P.}  Optimal
network design with end-to-end service requirements~// Oper. Res.,
2017. Vol.~65. Iss.~3. P.~729--750.

\bibitem{6-mal}
\Au{Nace D., Doan~L.\,N., Klopfenstein~O., Bashllari~A.} Max-min
fairness in multicommodity flows~// Comput. Oper. Res., 2008.
Vol.~35. Iss.~2. P.~557--573.

\bibitem{7-mal}
\Au{Ros-Giralt J., Tsai~W.\,K.} A~lexicographic optimization
framework to the flow control problem~// IEEE T.
Inform. Theory, 2010. Vol.~56. Iss.~6. P.~2875--2886.

\bibitem{8-mal}
\Au{Baier G., Kohler~E., Skutella~M.}  The \mbox{k-splittable}
flow problem~//  Algorithmica, 2005. Vol.~42. Iss.~3-4.
P.~231--248.

\bibitem{9-mal}
\Au{Bialon P.\,A.} Randomized rounding approach to 
a~\mbox{k-splittable} multicommodity flow problem with lower path flow
bounds affording solution quality guarantees~// Telecommun. Syst.,
2017. Vol.~64. Iss.~3. P.~525--542.
\end{thebibliography}

 }
 }

\end{multicols}

\vspace*{-6pt}

\hfill{\small\textit{Поступила в~редакцию 10.06.22}}

\vspace*{8pt}

%\pagebreak

%\newpage

%\vspace*{-28pt}

\hrule

\vspace*{2pt}

\hrule

%\vspace*{-2pt}

\def\tit{SEQUENTIAL ANALYSIS AND METRIC ESTIMATES\\ OF~PEAK LOAD FLOWS IN~THE~MULTIUSER NETWORK}


\def\titkol{Sequential analysis and metric estimates of~peak load flows in~the~multiuser network}


\def\aut{Yu.\,E.~Malashenko}

\def\autkol{Yu.\,E.~Malashenko}

\titel{\tit}{\aut}{\autkol}{\titkol}

\vspace*{-8pt}


\noindent
Federal Research Center ``Computer Science and Control'' of the Russian Academy of Sciences, 
44-2~Vavilov Str., Moscow 119333, Russian Federation



\def\leftfootline{\small{\textbf{\thepage}
\hfill INFORMATIKA I EE PRIMENENIYA~--- INFORMATICS AND
APPLICATIONS\ \ \ 2022\ \ \ volume~16\ \ \ issue\ 3}
}%
 \def\rightfootline{\small{INFORMATIKA I EE PRIMENENIYA~---
INFORMATICS AND APPLICATIONS\ \ \ 2022\ \ \ volume~16\ \ \ issue\ 3
\hfill \textbf{\thepage}}}

\vspace*{3pt} 



\Abste{The set of vectors of internodal flows in a~multiuser communication network under peak load is analyzed. Within the framework of
 the multicommodity model, the throughput capacities of edges are considered as the components of a~vector of resources of various types that 
 are required for the transmission of various kinds of
 flows. When conducting computational experiments, at each iteration, the
  norms of vectors of jointly permissible internodal flows are calculated, during the transmission of which the capacity of 
  all network edges is fully used.\linebreak\vspace*{-12pt}}
 
 \Abstend{The proposed method and computational experiments have shown that the equalizing phased 
  distribution leads to an uneven distribution of flows for different groups of correspondents. Metric values obtained during experiments 
  indicate deformation of the sets of accessible flows. Within the framework of the model, all correspondents are tantamount 
  and the flows are noninterchangeable; however, in the case of an equalizing peak load distribution, adjacent pairs 
  of nodes are in a privileged position when using residual capacity. The obtained metric values can be used to 
  evaluate the functional characteristics of the transmission network in the finite capacity loading mode.}

\KWE{multicommodity flow network model; set of achievable internodal flows; peak load distribution}


\DOI{10.14357/19922264220306} 

%\vspace*{-16pt}

%\Ack
%\noindent



%\vspace*{4pt}

  \begin{multicols}{2}

\renewcommand{\bibname}{\protect\rmfamily References}
%\renewcommand{\bibname}{\large\protect\rm References}

{\small\frenchspacing
 {%\baselineskip=10.8pt
 \addcontentsline{toc}{section}{References}
 \begin{thebibliography}{9}
\bibitem{1-mal-1}
\Aue{Malashenko, Yu.\,E., and I.\,A.~Nazarova.}
2022. Heterogeneous flow distribution at the peak load in the multiuser network. \textit{J.~Comput. Sys. Sc. Int.} 61:372--387.

\bibitem{4-mal-1} %2
\Aue{Luss, H.} 2012. \textit{Equitable resource allocation: Models, algorithms, and applications}.
Hoboken, NJ: John Wiley \& Sons. 420~p.

\bibitem{2-mal-1} %3
\Aue{Ogryczak, W., H.~Luss, M.~Pioro, D.~Nace, and A.~Tomaszewski.}
 2014. Fair optimization and networks: A~survey. \textit{J.~Appl. Math.} 2014:612018. 25~p. doi: 10.1155/ 2014/612018.
\bibitem{3-mal-1} %4
\Aue{Salimifard, K., and S.~Bigharaz.}
 2020. The multicommodity network flow problem: State of the art classification, applications, and solution methods. 
 \textit{J.~Oper. Res.} 18(3):\linebreak 1--47.

\bibitem{5-mal-1}
\Aue{Balakrishnan, A., G.~Li, and P.~Mirchandani.} 2017. Optimal network design with end-to-end service requirements. 
\textit{Oper. Res.} 65(3):729--750.
\bibitem{6-mal-1}
\Aue{Nace, D., L.\,N.~Doan, O.~Klopfenstein, and A.~Bashllari.} 2008. Max-min fairness in multicommodity flows. 
\textit{Comput. Oper. Res.} 35(2):557--573.
\bibitem{7-mal-1}
\Aue{Ros-Giralt, J., and W.\,K.~Tsai.} 2010. A~lexicographic optimization framework to the flow control problem. 
\textit{IEEE T.~Inform. Theory} 56(6):2875--2886.
\bibitem{8-mal-1}
\Aue{Baier, G., E.~Kohler, and M.~Skutella.}
 2005. The k-splittable flow problem. \textit{Algorithmica} 42(3-4):231--248.
\bibitem{9-mal-1}
\Aue{Bialon, P.} 2017. A~randomized rounding approach to a~\mbox{k-splittable} multicommodity flow problem with lower path flow bounds affording solution quality guarantees. 
\textit{Telecommun. Syst.} 64(3):525--542.
 \end{thebibliography}

 }
 }

\end{multicols}

\vspace*{-6pt}

\hfill{\small\textit{Received June 10, 2022}}

\Contrl

\noindent
\textbf{Malashenko Yuri E.} (b.\ 1946)~--- 
Doctor of Science in physics and mathematics, principal scientist, Federal Research Center ``Computer Science and Control'' 
of the Russian Academy of Sciences, 44-2~Vavilov Str., Moscow 119333, Russian Federation; \mbox{malash09@ccas.ru} 


\label{end\stat}

\renewcommand{\bibname}{\protect\rm Литература}   %6
\def\stat{krivenko}

\def\tit{МНОГОМЕРНЫЙ РЕФЕРЕНСНЫЙ РЕГИОН\\ ВЫСОКОЙ ПЛОТНОСТИ}

\def\titkol{Многомерный референсный регион высокой плотности}

\def\aut{М.\,П.~Кривенко$^1$}

\def\autkol{М.\,П.~Кривенко}

\titel{\tit}{\aut}{\autkol}{\titkol}

\index{Кривенко М.\,П.}
\index{Krivenko M.\,P.}


%{\renewcommand{\thefootnote}{\fnsymbol{footnote}} \footnotetext[1]
%{Работа выполнена при финансовой поддержке РФФИ (проекты 16-07-00677 
%и~15-37-20611-мол\_а\_вед).}}


\renewcommand{\thefootnote}{\arabic{footnote}}
\footnotetext[1]{Институт проблем информатики Федерального исследовательского центра <<Информатика и~управление>> Российской академии наук,
\mbox{mkrivenko@ipiran.ru}}

\vspace*{4pt}



\Abst{Рассматриваются принципы построения многомерных референсных регионов
(MRR~--- multivariate reference region). 
Предложен оригинальный метод построения региона на основе областей с~высокой 
плотностью точек и~аппроксимации распределения данных с~помощью смеси нормальных 
распределений. Для оценки порога для плотности распределения используется  
бут\-стреп-ме\-тод. В~качестве эксперимента рассмотрена задача построения 
и~использования эталонной области для прогнозирования типа мочевого камня. Обработка 
реальных данных продемонстрировала преимущества предлагаемых решений.}

\KW{многомерный референсный регион; область высокой плотности; бут\-стреп-ме\-тод; 
смесь многомерных нормальных распределений}

\vspace*{6pt}

\DOI{10.14357/19922264170207} 


\vskip 10pt plus 9pt minus 6pt

\thispagestyle{headings}

\begin{multicols}{2}

\label{st\stat}

\section{Введение}

     Многомерный референсный регион 
был предложен в~литературе по клинической химии в~начале 1970-х~гг.\ как 
альтернатива одномерным референсным интервалам~[1]. Там излагались 
преимущества предлагаемых множественных тестов, хоть и~имеющих 
упрощенный вид, но снижающих (по отношению к~одномерным вариантам) 
число ложных положительных результатов. Появление MRR оказалось 
особенно привлекательным для интерпретации результатов наборов 
медицинских тестов. Тем не менее возникали трудности в~построении 
и~использовании процедур многомерного анализа (см., например,~[2]), 
связанные, в~частности, с~быстрым увеличением числа параметров, которые 
должны быть оценены. Немногие лаборатории использовали MRR в~своей 
практике, причем в~экспериментальном режиме, и,~как следствие, на 
сегодняшний день имеется относительно малое количество соответствующих 
публикаций. 

\vspace*{-6pt}

\section{Многомерный референсный регион на основе расстояния Махалонобиса}

\vspace*{-2pt}

     Одномерный референсный интервал, полученный статистическим путем, 
использует центральную часть значений анализируемого показателя, обычно 
соответствующую~95\% некоторой популяции~--- совокупности особей 
определенного вида (например, здоровой части населения определенного пола 
из некоторого диапазона возрастов). Одномерные референсные интервалы 
применялись в~течение многих лет в~качестве стандартного приема 
интерпретации лабораторных данных. Они легко формируются, хранятся, 
извлекаются и~передаются в~лабораторных информационных системах, просты 
в~понимании, хорошо воспринимаются медицинским сообществом в~ходе 
длительного использования. Тем не менее одномерные референсные интервалы 
при классификации данных могут дать большое число ложно аномальных 
результатов. Этот далеко не единственный недостаток однофакторного 
референсного интервала может быть полностью или частично устранен 
с~помощью MRR.
     
     Простейшим и~весьма распространенным способом построения MRR 
является использование прямого произведения отдельных референсных 
интервалов в~предположении, что они статистически независимы. Пусть 
$(1\hm-\alpha)$~--- вероятность попадания в~MRR, а~$p_0$~--- вероятность 
попадания в~референсный интервал для любого из~$d$~признаков, тогда 
$p_0\hm= \sqrt[d]{1-\alpha}$. С~ростом размерности~$d$ значения~$p_0$ 
быстро приближаются к~1, что фактически лишает смысла применение MRR.
     
     Как и~в одномерном случае, отправной точкой для построения MRR 
может стать нормальное распределение. Идеи центрального расположения 
референсного региона и~заданной вероятности попадания в~него приводят для 
$d$-мер\-но\-го нормального распределения, имеющего плотность 
распределения
     \begin{multline*}
     \varphi(y,\mu,\Sigma) ={}\\
     {}=(2\pi)^{-d/2}\vert\Sigma\vert^{-1/2}\exp \left( -\fr{\left(y-
\mu\right)^{\mathrm{T}} \Sigma^{-1}(y-\mu)}{2}\right),
   \end{multline*}
где величина $(y-\mu)^{\mathrm{T}} \Sigma^{-1} (y-\mu)$ есть квадрат так 
называемого расстояния Махаланобиса между~$y$ и~$\mu$, к~использованию 
многомерного эллипсоида
\begin{multline*}
(2\pi)^{-d/2}\vert\Sigma\vert^{-1/2}\exp \left( -\fr{\left(y-\mu\right)^{\mathrm{T}}
\Sigma^{-1} 
(y-\mu)}{2}\right) ={}\\
{}=const
\end{multline*}
или, что то же самое, 
$$ 
(y-\mu)^{\mathrm{T}} \Sigma^{-1}(y-\mu)=const\,.
$$
Его называют эллипсоидом равной плотности распределения (или просто 
эллипсоидом равной вероятности). 
     
     Если задаться вероятностью $(1\hm-\alpha)$ попадания в~эллипсоид 
равной вероятности вида $(y\hm-\mu)^{\mathrm{T}}\Sigma^{-1} (y\hm-\mu)\hm= 
\rho$, то параметр~$\rho$ удовлетворяет уравнению $\mathrm{Pr}\left\{ 
\chi_d^2\leq \rho\right\} \hm=1\hm-\alpha$.
     
     Использование эллипсоида в~качестве MRR будет оправдано только 
тогда, когда исходное распределение данных есть многомерное нормаль-\linebreak ное. 
Поэтому становятся актуальными критерии\linebreak подгонки, а~также использование 
процедур норма\-ли\-зации распределения данных в~многомерном\linebreak случае.
 Если 
с~помощью тестов выявляется, что распределение не является нормальным, то 
Международная федерация клинической химии и~лабораторной медицины 
рекомендует, согласно~[3], использовать двухступенчатую процедуру 
нормализации. Следует обратить внимание, что многошаговость здесь 
относится не к~многомерности, а касается лишь покоординатного 
преобразования распределения данных к~нормальному.
     
     Первые же попытки применения MRR на основе расстояния 
Махалонобиса (фактически это означает принятие модели нормального 
распределения референсных значений) выявили ряд недостатков (более 
подробно смотри в~\cite[разд.~6.2]{4-kri}):
     \begin{itemize}
\item проявление <<проклятий>> размерности при механическом 
увеличении~$d$, в~особенности если игнорируется этап анализа состава 
признаков~[1, 5, 6];
\item из-за небольших объемов обучающей выборки невысокая устойчивость 
при применении, в~частности чувствительность к~увеличению неточностей 
измерений после того, как регион был установлен~\cite{5-kri, 7-kri}. 
\item предположение о нормальном распределении и~попытки <<подправить>> 
действительность с~помощью преобразований реальных данных для их 
нормализации при увеличении размерности данных становятся все более 
шаткими~\cite{5-kri};
\item представление и~интерпретация выводов на основе MRR трудно 
понимаемы не только для специалистов в~предметной области~[8].
\end{itemize}

\vspace*{-9pt}

\section{Многомерный референсный регион высокой плотности}

\vspace*{-2pt}

     Заметим, что в~случае нормального распределения референсных значений 
для точек внут\-ри построенного эллипсоида значения плотности\linebreak распределения 
больше, чем на границе, а~вне~--- меньше. Это замечание позволяет 
предложить другой подход к~построению MRR.
     
     \smallskip
     
     \noindent
     \textbf{Определение.}\ Eсли плотность распределения референсных 
значений есть $f(y)$, то MRR есть область $A_t\hm= \left\{ y\in 
\mathcal{R}^d\vert f(y)\hm\geq t\right\}$ для некоторого порогового 
значения~$t$. 
     
     \smallskip
     
     Для нормального распределения это уже упомянутый эллипсоид равной 
вероятности. Если задается вероятность $(1\hm-\alpha)$ попадания в~$A_t$, то 
пороговое значение~$t$ есть решение уравнения $\int\nolimits_{A_t} 
f(u)\,du\hm=1\hm-\alpha$, получить которое аналитически в~случае 
произвольной плотности распределения вряд ли возможно. Здесь присутствуют 
две проблемы: вычисление многомерного интеграла и~зависимость области 
интегрирования от неизвестного значения. Для решения их предлагается 
привлечь метод моделирования.
     
     Сгенерируем выборку из $f(y)$, которую обозначим как $Y^f\hm= \left\{ 
y_1^f, \ldots, y_m^f\right\}$. Для оценки $\int\nolimits_{A_t} f(u)\,du$ 
используем отношение:

\noindent
\begin{multline*}
     \fr{\left\vert \left\{ y_i^f\vert y_i^f\in A_t\right\}\right\vert }{m} =
      \fr{\left\vert\left\{ y_i^f\vert 
f\left(y_i^f\right) \geq t\right\}\right\vert }{m} ={}\\
{}= 1-\fr{\left\vert \left\{ y_i^f\vert f(y_i^f)<t\right\}\right\vert }{m}=1-
F_m(t)\,,
     \end{multline*}
где $F_m(t)$~--- эмпирическая функция распределения случайной 
величины~$f(y)$, т.\,е.\ случайной величины, являющейся результатом 
преобразования с~помощью функции~$f(\cdot)$ случайной величины, име\-ющей 
плотность распределения~$f(u)$.

     Таким образом, искомая оценка~$t^*$ должна удовле\-тво\-рять уравнению 
$F_m(t^*)\hm=\alpha$ и~может быть получена как непараметрическая оценка 
квантиля\linebreak\vspace*{-12pt}

\pagebreak

\noindent
 порядка~$\alpha$ из распределения $F_m(\cdot)$. Если обозначить 
$f_i\hm= f(y_i^f)$, то~$t^*$ есть~$f_{(r)}$, где
     $$
     r= \begin{cases}
     m\alpha, &\ m\alpha~\mbox{---~целое}\,;\\
     \lfloor m\alpha+1\rfloor\,, & m\alpha~\mbox{--- не целое}\,.
     \end{cases}
     $$
     Заметим, что для такой оценки можно указать доверительный интервал.
     
     Для построения MRR необходимо знать распределение данных. При 
реализации принципа точек высокой плотности в~первую очередь следует 
обратиться к~параметрическим моделям, в~част\-ности к~смеси нормальных 
распределений, име\-ющей плотность распределения
     $$
     f(u) =\sum\limits_{j=1}^k p_j \varphi\left (u,\mu_j, \Sigma_j\right)\,.
     $$
Если $\hat{f}(u)$~--- оценка смеси, то~$t^*$ строится сле\-ду\-ющим образом:
\begin{itemize}
\item генерируется выборка $\left\{ y_1^f,\ldots , y_m^f\right\}$ из $\hat{f}(u)$ и~
для каждого ее $i$-го элемента подсчитывается значение $\hat{f}\left( 
y_i^f\right)$;
\item в~качестве~$t^*$ берется непараметрическая оценка квантиля 
порядка~$\alpha$ (в случае необходимости дополнительно находится 
непараметрическая оценка доверительного интервала для~$t^*$, что 
может характеризовать правильность выбранного объема для 
генерируемой выборки).
\end{itemize}

     Пусть для $f(u)$ имеется~$A_t$, а также получена $\hat{f}(u)$ 
и~соответствующий MRR вида~$\hat{A}_t$. Качество аппроксимации~$A_t$ 
с~по\-мощью~$\hat{A}_t$ можно оценить с~по\-мощью вероятности совпадения 
этих областей, т.\,е. 
     $$
     P_c= \int\limits_{\{ u\in A_t\}\cup \{u\in \hat{A}_t\}} \hspace*{-6mm}
f(u)\,du+\int\limits_{\{u\not\in A_t\} \cup\{ u\not\in \hat{A}_t\}}\hspace*{-6mm} f(u)\,du\,.
     $$
     
     Для оценки  $P_c$ можно использовать величину
     \begin{multline*}
     \hat{P}_c= \fr{\left\vert \left\{ 
     y_i^f\vert y_i^f \in \left\{\left\{ y_i^f\in A_t\right\}\cup \left\{y_i^f\in 
\hat{A}_t\right\}\right\}\right\}\right\vert}{m}+{}\\
{}+\fr{\left\vert \left\{ y_i^f\vert y_i^f \in \left\{\left\{ y_i^f\not\in A_t\right\}\cup 
\left\{ y_i^f\not\in \hat{A}_t\right\}\right\}\right\}\right\vert}{m}\,.
     \end{multline*}
     
     Использование MRR высокой плотности для диагностирования сводится 
к~реализации так называемого слабого критерия значимости для наблюденного 
значения~$x$: нулевая гипотеза заключается в~том, что $x\hm\in A_t$, 
статистика критерия есть $\hat{f}(x)$ и~решение о~принадлежности 
критической об\-ласти~$A_t$ принимается при больших значениях~$\hat{f}(x)$.
     
     Для медицинской практики важна возможность использования 
референсного региона при интерпретации результатов обследования 
некоторого пациента с~вектором признаков~$x$. В~подобных случаях 
сложившейся практикой для слабых критериев значимости является 
использование критического уровня~$\alpha_{\mathrm{cr}}$ (более распространенным 
в~медицине является употребление термина $p$-зна\-че\-ние)  $\alpha_{\mathrm{cr}}\hm= 
\mathrm{Pr}\left\{ \hat{f}(y)\hm\leq \hat{f}(x)\right\}$, где $y$~--- случайная 
величина, имеющая плотность распределения~$\hat{f}(u)$, а $\hat{f}(x)$~--- 
значение плотности распределения~$\hat{f}(u)$ в~точке~$x$. Эта 
характеристика дает представление о~том, насколько сильно данное 
наблюденное значение~$x$ противоречит гипотезе (или подкрепляет ее) 
о~принадлежности данных MRR. При выбранном же заранее уровне 
значимости с~помощью~$\alpha_{\mathrm{cr}}$ сразу же можно принять конкретное 
решение. 

\vspace*{-9pt}

\section{Эксперименты}

\vspace*{-2pt}

     Для демонстрации возможностей MRR использовались данные по 
прогнозу химического состава мочевых камней по метаболическим 
показателям мочи и~сыворотки крови, а также антропологическим 
характеристикам пациентов~[9]. В качестве исходной классификации камней 
рассматривалась следующая: чисто оксалатные (далее обозначены как O), чисто 
уратные (U), чисто фосфатные (P), смесь только оксалатных и~уратных (OU), 
смесь только оксалатных и~фосфатных (OP), смесь только уратных 
и~фосфатных (UP), все остальные. Данная классификация была построена 
в~[10] на основе доминирующих частот встречаемости основных компонентов. 
В~качестве референсных значений рассматривались наборы метаболических 
и~антропологических показателей (их всего было~14), соответствующих 
определенному классу камней.

\begin{table*}\small
\begin{center}


\begin{tabular}{|c|c|c|c|c|c|c|}
\multicolumn{7}{c}{Качество классификации с~помощью MRR}\\
\multicolumn{7}{c}{\ }\\[-6pt]
\hline
\multicolumn{1}{|c|}{\raisebox{-6pt}[0pt][0pt]{\tabcolsep=0pt\begin{tabular}{c}Тип\\ камня\end{tabular}}}&
\multicolumn{1}{c|}{\raisebox{-6pt}[0pt][0pt]{$N$}}&$(1-\alpha)$, 
&\multicolumn{2}{c|}{MRR(5)}&\multicolumn{2}{c|}{MRR(1)}\\
\cline{4-7}
&&&&&&\\[-9pt]
&&\%&$(1-\hat{\alpha})$, \%&$\hat{\beta}$, \%&$(1-\hat{\alpha})$, \%&$\hat{\beta}$, \%\\
\hline
\multicolumn{1}{|c|}{\raisebox{-18pt}[0pt][0pt]{O}}&
\multicolumn{1}{c|}{\raisebox{-18pt}[0pt][0pt]{82}}
&95&100\hphantom{9}&71&90&24\\
&&85&96&78&89&36\\
&&75&91&85&77&44\\
&&65&76&88&74&50\\
\hline
\multicolumn{1}{|c|}{\raisebox{-18pt}[0pt][0pt]{U}}&
\multicolumn{1}{c|}{\raisebox{-18pt}[0pt][0pt]{76}}&95&100\hphantom{9}&75&91&24\\
&&85&99&85&80&35\\
&&75&82&89&74&48\\
&&65&71&91&68&56\\
\hline
\multicolumn{1}{|c|}{\raisebox{-18pt}[0pt][0pt]{P}}&
\multicolumn{1}{c|}{\raisebox{-18pt}[0pt][0pt]{83}}&95&100\hphantom{9}&66&87&25\\
&&85&94&78&86&33\\
&&75&86&82&82&41\\
&&65&77&87&75&47\\
\hline
\end{tabular}
\end{center}
\end{table*}
     
     
     Для каждого из основных классов O, U, P, OU, OP и~UP перед построением 
MRR проводилась селекция признаков и~принималось то значение размерности 
признакового пространства~$d$ и~соответствующий набор показателей, 
которые позволяли прогнозировать состав камней без потери качества 
(методика описана в~\cite{9-kri} и~привела к~значению $d\hm=9$). В~качестве 
модели данных в~первую очередь рассматривалась смесь многомерных 
нормальных распределений из пяти элементов (подбор числа элементов смеси 
проводился с~по\-мощью AIC~--- Akaike information criterion), для соответствующего региона было принято 
обозначение MRR(5). Для сравнения также использовалась модель 
нормального распределения, которой соответствовал MRR(1). Полученные 
результаты приводятся час\-тич\-но в~таблице, где $N$~--- объем 
классифицируемых данных; $\hat{\alpha}$~--- оценка для~$\alpha$; 
$\hat{\beta}$~--- оценка мощности критерия при определении типа камня на 
основании MRR.


     Одной из базовых характеристик является вероятность попадания в~MRR 
$(1\hm-\alpha)$ и~ее оценка $(1\hm-\hat{\alpha})$. Сравнение соответствующих 
столбцов с~учетом значений~$N$ и~ориентировочных значений разброса 
(стандартные отклонения на основе биномиального распределения) не 
позволило выявить явных отклонений. Необходимо, правда, отметить, что во 
всех проанализированных случаях для MRR(5) оказалось, что $1\hm-
\hat{\alpha}\hm\geq 1\hm-\alpha$.
     
     Назначение MRR, заключающееся в~сжатом представлении референсных 
значений, в~многомерном случае практически не проявляется. Для задания 
MRR(5) необходимо указать следующие величины: $1\hm-\alpha$, $t$, 
$p_1,\ldots, p_{k-1}$, $\mu_1, \Sigma_1,\ldots , \mu_k,\Sigma_k$, общее 
количество которых равно  $[2\hm+ (k\hm-1)\hm+ k(d\hm+ d(d\hm+1)/2)]$ 
и,~в~частности, в~рассматриваемых экспериментах~--- 276. Для MRR(1) это 
значение меньше и~равно~56. При этом для обрабатываемой обучающей 
выборки в~зависимости от класса камней речь идет о~порядка~10$^2$ векторах 
данных (см.\ столбец со значениями~$N$), что приблизительно 
дает~10$^3$~скалярных величин.
     
     Другое назначение MRR состоит в~его использовании для 
диагностирования (классификации). В~этой связи в~первую очередь 
проводился сравнительный анализ MRR(1) (фактически это означает, что 
построение региона осуществляется на основе расстояния Махаланобиса) 
и~MRR(5) (модель смеси нормальных распределений и~предложенный 
в~данной работе метод оценивания па\-ра\-мет\-ров региона). Показателем 
информативности метода построения многомерного региона выступала 
мощность соответствующего слабого критерия значимости, а~именно: 
вероятность не попасть в~MRR при условии, что данные берутся из дополнения 
к~классу, для которого построена MRR. Сравнение соответствующих столбцов 
говорит о~явном преимуществе двух предложенных моментов: усложнение 
модели данных путем перехода от нормального распределения к~смеси 
нормальных распределений и~построение региона высокой плотности.
     
     Использование критического уровня можно продемонстрировать  
с~по\-мощью зависимости результатов сравнения двух классов от того, какой 
класс взять за основу. Введем для возможных значений $p$-ве\-ли\-чи\-ны три 
интервала: $(-\infty, 1\%)$, $[1\%, 5\%)$, $[5\%, 100\%)$ с~соответствующей 
интерпретацией положения наблюденного набора показателей для пациента 
относительно построенного MRR: уверенное непопадание, неуверенное 
попадание, уверенное попадание. Если MRR построить для оксалатных камней, 
то результаты для анализа пациентов с~фосфатными камнями дадут следующий 
вектор относительных частот попадания $p$-ве\-ли\-чин в~указанные 
интервалы: $(60\%, 18\%, 22\%)$. Если же MRR строить для фосфатных 
камней, то получим $(71\%, 5\%, 24\%)$. Таким образом, для классификации 
указанных камней при приблизительно одинаковых частотах попадания в~MRR 
(22\% или~24\%) уверенный отказ от референсного региона происходит чаще, 
если принять за базовый MRR регион для фосфатных камней. Построение 
шкалы, подобной рассмотренной, является прерогативой специалистов 
в~предметной области, в~данной работе она использовалась только для 
иллюстрации. 

\vspace*{-6pt}

\section{Заключение}

\vspace*{-2pt}

     На настоящий момент имеется относительно мало примеров применения 
MRR в~клинической практике. Тому есть несколько причин. Математическое 
обеспечение, необходимое для получения и~применения MRR, не отвечает 
возможностям большинства клинических лабораторий. Лаборатории слабо 
оснащены программными средствами\linebreak для реализации достаточно сложного 
математического аппарата многомерного анализа, а~еще важнее, что 
отсутствуют методики, инструкции по\linebreak использованию соответствующих 
средств. Лишь немногие клинические применения демонстрируют 
преимущества MRR, хотя свидетельств неудачных попыток больше.
     
     Несмотря на сложности внедрения мно\-го\-мерно\-го анализа референсных 
значений, можно сформулировать некоторые рекомендации по иссле\-до\-ва\-нию 
и~разработке MRR. Во-пер\-вых, эффективная размерность в~MRR должна 
быть как можно меньше, чтобы избежать затенения диагностически полезной 
информации тестами, со\-зда\-ющи\-ми шум. Низкая размерность также должна 
уменьшить неблагоприятные последствия увеличения неточности результатов 
в~связи с~ростом числа анализируемых показателей. Во-вто\-рых, показатели 
(тес\-ты), включенные в~MRR, должны быть физиологически релевантными 
исследуемому кругу расстройств, чтобы максимизировать информацию, 
полученную от MRR. В-треть\-их, чтобы учесть эффекты долгосрочной 
лабораторной из\-мен\-чи\-вости, данные, используемые для получения MRR, 
долж\-ны быть собраны и~проанализированы в~течение достаточно большого 
периода времени (от нескольких недель до нескольких месяцев).  
В-чет\-вер\-тых, представление результатов лабораторных исследований 
следует осуществлять в~графическом виде, чтобы помочь врачам лучше понять 
MRR. Различные подходы к~уменьшению размерности помогут выполнить это 
требование.
     
     Необходима дальнейшая разработка пояснительных инструментов, 
способных воспринять результаты анализа MRR. При этом дополнительно 
необходима информация о~том, какие именно тес\-ты являются важнейшими 
факторами нарушения нормы. Надо признать, что соответствующий 
математический аппарат еще предстоит разработать. Решение перечисленных 
вопросов играет важную роль для обеспечения постоянного клинического 
применения MRR. 

\vspace*{-6pt}
     
{\small\frenchspacing
 {%\baselineskip=10.8pt
 \addcontentsline{toc}{section}{References}
 \begin{thebibliography}{99}
 
 \vspace*{-2pt}
 
\bibitem{1-kri}
\Au{Boyd J.\,C.} Reference regions of two or more dimensions~// Clin. Chem. Lab. 
Med., 2004. Vol.~42. No.\,7. P.~739--746.
\bibitem{2-kri}
\Au{Winkel P.} Patterns and clusters~--- multivariate approach for interpreting 
clinical chemistry results~// Clin. Chem., 1973. Vol.~19. No.\,12. P.~1329--1333.
\bibitem{3-kri}
IFCC. Expert panel on theory of reference values. Approved recommendation on the 
theory of reference values. Part~5. Statistical treatment of collected reference values. 
Determination of reference limits~// J.~Clin. Chem. Clin. Biochem., 1987. Vol.~25. 
No.\,9. P.~645--656.
\bibitem{4-kri}
\Au{Кривенко М.\,П.} Статистические методы представления и~предварительной 
обработки референсных значений.~--- М.: ФИЦ ИУ РАН, 2016. 160~с.
\bibitem{5-kri}
\Au{Boyd J.\,C., Lacher~D.\,A.} The multivariate reference range: An alternative 
interpretation of multi-test profiles~// Clin. Chem., 1982. Vol.~28. No.\,2.  
P.~259--265.
\bibitem{6-kri}
\Au{Albert A., Harris~E.\,K.} Multivariate interpretation of clinical laboratory  
data.~--- New York, NY, USA: CRC Press, 1987. 328~p.
\bibitem{7-kri}
\Au{Linnet K.} Influence of sampling variation and analytical errors on the 
performance of the multivariate reference region~// Meth. Inf. Med., 1988. Vol.~27. 
No.\,1. P.~37--42.
\bibitem{8-kri}
\Au{Durbridge T.\,C.} Clinical acceptance of a multi-test reference region for 
biochemical-panel results~// Clin. Chem., 1983. Vol.~29. No.\,10. P.~1724--1726.
\bibitem{9-kri}
\Au{Кривенко М.\,П.} Критерии значимости отбора признаков классификации~// 
Информатика и~её применения, 2016. Т.~10. Вып.~3. С.~32--40.
\bibitem{10-kri}
\Au{Кривенко М.\,П., Голованов~С.\,А., Сивков~А.\,В.} Анализ однородности 
данных о химическом составе камней при уролитиазе~// Информатика и~её 
применения, 2013. Т.~7. Вып.~4. С.~94--104.
 \end{thebibliography}

 }
 }

\end{multicols}

\vspace*{-10pt}

\hfill{\small\textit{Поступила в~редакцию 5.12.16}}

\vspace*{4pt}

%\newpage

%\vspace*{-24pt}

\hrule

\vspace*{2pt}

\hrule

\vspace*{-3pt}


\def\tit{HIGH-DENSITY MULTIVARIATE REFERENCE REGION\\[-5pt]}

\def\titkol{High-density multivariate reference region}

\def\aut{M.\,P.~Krivenko\\[-7pt]}

\def\autkol{M.\,P.~Krivenko}

\titel{\tit}{\aut}{\autkol}{\titkol}

\vspace*{-16pt}


\noindent
Institute of Informatics Problems, Federal Research Center 
``Computer Science and Control'' of the Russian
Academy of Sciences,  44-2~Vavilov Str., Moscow 119333, Russian Federation



\def\leftfootline{\small{\textbf{\thepage}
\hfill INFORMATIKA I EE PRIMENENIYA~--- INFORMATICS AND
APPLICATIONS\ \ \ 2017\ \ \ volume~11\ \ \ issue\ 2}
}%
 \def\rightfootline{\small{INFORMATIKA I EE PRIMENENIYA~---
INFORMATICS AND APPLICATIONS\ \ \ 2017\ \ \ volume~11\ \ \ issue\ 2
\hfill \textbf{\thepage}}}

\vspace*{2pt}




\Abste{The paper considers the principles of construction of multivariate 
reference regions. An original method of construction of 
a~region on the basis of areas of high density of points and approximation 
of data distribution with a~mixture of normal distributions is suggested. 
To estimate the threshold for the probability density, the bootstrap method is used. 
As an experiment, the paper considers the problem of description and use of 
the reference region for predicting the type of urinary stones. 
Real data treatment demonstrated the benefits of the proposed solutions.}

\KWE{multivariate reference region; high-density region; bootstrap method; 
multivariate normal mixture}

\DOI{10.14357/19922264170207} 

%\vspace*{-18pt}

%\Ack
%\noindent



%\vspace*{3pt}

  \begin{multicols}{2}

\renewcommand{\bibname}{\protect\rmfamily References}
%\renewcommand{\bibname}{\large\protect\rm References}

{\small\frenchspacing
 {%\baselineskip=10.8pt
 \addcontentsline{toc}{section}{References}
 \begin{thebibliography}{99}
\bibitem{1-kri-1}
\Aue{Boyd, J.\,C.} 2004. Reference regions of two or more dimensions. \textit{Clin. 
Chem. Lab. Med.} 42(7):739--746.

\bibitem{2-kri-1}
\Aue{Winkel, P.} 1973. Patterns and clusters~--- multivariate approach for interpreting 
clinical chemistry results. \textit{Clin. Chem.} 19(12):1329--1333.
\bibitem{3-kri-1}
IFCC. 1987. Expert panel on theory of reference values. Approved recommendation on the 
theory of reference values. Part~5. Statistical treatment of collected reference values. 
Determination of reference limits. \textit{J.~Clin. Chem. Clin. Biochem.} 
25(9):645--656.
\bibitem{4-kri-1}
\Aue{Krivenko, M.\,P.} 2016. \textit{Statisticheskie metody predstavleniya 
i~predvaritel'noy obrabotki referensnykh znacheniy}
[Statistical methods for representation and preliminary processing of
reference values]. Moscow: FRC CSC RAS. 160~p.

\bibitem{5-kri-1}
\Aue{Boyd, J.\,C., and D.\,A.~Lacher.} 1982. The multivariate reference range: An 
alternative interpretation of multi-test profiles. \textit{Clin. Chem.}  
28(2):259--265.
\bibitem{6-kri-1}
\Aue{Albert, A., and E.\,K.~Harris.} 1987. \textit{Multivariate interpretation of 
clinical laboratory data}. New York, NY: CRC Press. 328~p.
\bibitem{7-kri-1}
\Aue{Linnet, K.} 1988. Influence of sampling variation and analytical errors on the 
performance of the multivariate reference region. \textit{Meth. Inf. Med.}  
27(1):37--42.
\bibitem{8-kri-1}
\Aue{Durbridge, T.\,C.} 1983. Clinical acceptance of a multi-test reference region 
for biochemical-panel results. \textit{Clin. Chem.} 29(10):1724--1726.
\bibitem{9-kri-1}
\Aue{Krivenko, M.\,P.} 2016. Kriterii znachimosti otbora priznakov klassifikatsii
[Significance tests of feature selection for~classification]. \textit{Informatika i~ee 
Primeneniya~--- Inform. Appl.} 10(3):32--40.
\bibitem{10-kri-1}
\Aue{Krivenko, M.\,P., S.\,A.~Golovanov, and A.\,V.~Sivkov}. 2013. Analiz 
odnorodnosti dannykh o~khimicheskom sostave kamney pri urolitiaze
[Analysis of data homogeneity of~the~chemical compositions 
of~stones in~case of~urolithiasis]. \textit{Informatika i~ee Primeneniya~---
Inform Appl.} 7(4):94--104.
\end{thebibliography}

 }
 }

\end{multicols}

\vspace*{-3pt}

\hfill{\small\textit{Received December 5, 2016}}


\Contrl

\noindent
\textbf{Krivenko Michail P.} (b.\ 1946)~--- Doctor of Science in technology, 
professor, leading scientist, Institute of Informatics Problems, Federal Research 
Center ``Computer Science and Control'' of the Russian Academy of Sciences, 
\mbox{44-2}~Vavilov Str., Moscow 119333, Russian Federation; \mbox{mkrivenko@ipiran.ru}

\label{end\stat}


\renewcommand{\bibname}{\protect\rm Литература}    %7
\def\stat{zatsman}

\def\tit{ТРАНСФОРМАЦИИ ОБЪЕКТОВ ПЕРВОГО И~ВТОРОГО ПОРЯДКА 
В~ЛЕКСИКОГРАФИЧЕСКОЙ ИНФОРМАЦИОННОЙ СИСТЕМЕ$^*$}

\def\titkol{Трансформации объектов первого и~второго порядка 
в~лексикографической информационной системе}

\def\aut{И.\,М.~Зацман$^1$}

\def\autkol{И.\,М.~Зацман}

\titel{\tit}{\aut}{\autkol}{\titkol}

\index{Зацман И.\,М.}
\index{Zatsman I.\,M.}


{\renewcommand{\thefootnote}{\fnsymbol{footnote}} \footnotetext[1]
{Исследование выполнено в~ФИЦ ИУ РАН за счет гранта Российского научного фонда №\,24-18-00155, {\sf 
https://rscf.ru/project/24-18-00155}. Работа выполнялась с~использованием инфраструктуры Центра 
коллективного пользования <<Высокопроизводительные вычисления и~большие данные>> (ЦКП 
<<Информатика>>) ФИЦ ИУ РАН (г.\ Москва).}}


\renewcommand{\thefootnote}{\arabic{footnote}}
\footnotetext[1]{ Федеральный исследовательский центр <<Информатика и~управление>> Российской академии наук, 
\mbox{izatsman@yandex.ru}}

\vspace*{-12pt}


  
  \Abst{Рассматриваются теоретические основания проектирования информационных 
технологий (ИТ) интеграции двуязычных словарей и~параллельных корпусов. Дано описание 
первых результатов создания третьего уровня классификации трансформаций объектов 
предметной области информатики, которую предполагается использовать при создании 
концепции лексикографической информационной системы, обеспечивающей интеграцию. 
Все сущности информатики в~статье разделены на два глобальных класса: объекты и~их 
трансформации. Для каждого такого класса конструируется своя классификация. Ранее были 
описаны два верхних уровня классификации трансформаций объектов предметной области. 
В~данной статье рассматривается третий уровень этой классификации. Основанием для 
построения самого верхнего ее уровня служило деление предметной области информатики 
на среды (ментальная, сенсорно воспринимаемая, цифровая и~ряд других сред), каждая из 
которых по определению включает объекты одной природы. Основанием для построения 
второго уровня классификации трансформаций объектов служила типология знаковых  
сис\-тем А.~Соломоника. Цель статьи состоит в~систематизации трансформаций первого 
и~второго порядка объектов предметной области на третьем уровне этой классификации. 
Основанием для систематизации служит средовая версия иерархии Акоффа.}
  
  \KW{объекты предметной области; трансформации объектов; классификация; данные; 
информация; знание; лексикографическая информационная сис\-тема}

\DOI{10.14357/19922264240211}{VZTGVV}
  
\vspace*{3pt}


\vskip 10pt plus 9pt minus 6pt

\thispagestyle{headings}

\begin{multicols}{2}

\label{st\stat}
  
\section{Введение}

\vspace*{-9pt}

  Возникновение параллельных корпусов, в~которых предложениям 
оригинального текста со\-по\-став\-ле\-ны предложения его перевода, обеспечило 
возможность контрастивного лингвистического\linebreak \mbox{анализа} на принципиально 
новом уровне полноты и~точности, недостижимом в~докорпусную эпоху. 
Пионерскими в~этой области стали работы \mbox{1990-х~гг}. Стига Йоханссона  
с~анг\-ло-нор\-веж\-ским корпусом~[1]. В России параллельные корпусы стали 
формироваться в~начале XXI~века в~рамках Национального корпуса русского 
языка~[2].
  
  Создатели двуязычных словарей используют параллельные корпусы для 
сбора материала и~эмпирической проверки своих гипотез, касающихся 
межъязы\-ко\-вой эквивалентности. Ценность параллельных корпусов 
определяется тем, что в~лингвистике этап сбора исходного материала считается 
наиболее трудоемким и~наименее творческим, а~параллельные корпусы 
позволяют значительно сэкономить время и~силы для творческого этапа 
создания словарей~[3].
 % 
  При этом двуязычные словари, создаваемые на основе исходного материала, 
извлеченного из параллельных корпусов, сейчас формируются без связей с~их 
текстами. Другими словами, онлайновые связи созданных словарей 
с~параллельными корпусами, которые служили источниками исходного 
материала, отсутствуют. 

Параллельные корпусы постоянно пополняются 
новыми текстами, в~предложениях которых можно обнаружить новые значения 
слов и~устойчивых словосочетаний. Однако при этом отсутствуют методы 
и~средства оперативного обновления словарей по корпусным данным. 
В~настоящее время проблема установления связей между двуязычными 
словарями и~параллельными корпусами (далее~--- проблема интеграции) 
находится на стадии поиска концептуальных подходов к~их интеграции на 
уровне значений.
  
  Подход к~решению проблемы интеграции, предлагаемый в~статье, учитывает 
  и~появление новых значений слов и~устойчивых словосочетаний, и~динамику 
смысловых значений, которая обусловлена развитием и~пополнением знания 
лингвистов, фиксирующих эти значения в~результате семантического анализа 
пополняемых корпусных данных. Проведенные эксперименты показали, что 
обнаружение нового лингвистического знания обусловливает и~формирование 
дефиниций новых значений, и~пересмотр уже существующих дефиниций~[4, 5].
  
  Например, в~проведенных экспериментах с~использованием ЦКП 
<<Информатика>> ФИЦ ИУ РАН фиксировалась эволюция значений немецких 
модальных глаголов, исходное состояние значений которых было описано 
в~не\-мец\-ко-рус\-ском словаре. В~экспериментальном массиве текстов как 
потенциальных источниках нового знания 16\,268 предложений содержали 
немецкие модальные глаголы и~в~2041 из них встречался глагол sollen. 
В~начале эксперимента в~словаре были описаны~12~значений этого модального 
глагола. По окончании эксперимента лингвисты обнаружили два новых его 
значения, согласовали их дефиниции и~описали эволюцию дефиниций~[6, 7].
  
  Таким образом, для решения проблемы интеграции требуется фиксировать 
новое знание, обнаруженное лингвистами в~текстовых данных параллельных 
корпусов, отслеживать эволюцию знания, представленного в~виде дефиниций 
значений слов и~устойчивых словосочетаний, и,~соответственно, 
актуализировать электронные двуязычные словари. Предлагаемый 
концептуальный подход к~интеграции, который планируется реализовать 
в~процессе проектирования лексикографической информационной сис\-те\-мы, 
фиксирующей эволюцию лингвистического знания, основан на решении 
следующих задач:\\[-14pt]
  \begin{itemize}
  \item категоризация трех базовых понятий информатики, включенных 
  в~иерархию Акоффа~[8] (данные, информация, знание), на объекты 
проектируемой сис\-те\-мы, которая необходима, чтобы фиксировать 
<<кванты>> нового знания и~отслеживать его эволюцию в~этой сис\-теме;\\[-15pt]
  \item  систематизация трансформаций объектов этой сис\-темы.\\[-14pt]
  \end{itemize}
  
  Цель статьи и~состоит в~решении двух задач: категоризации трех базовых 
понятий информатики на объекты лексикографической информационной  
сис\-те\-мы и~сис\-те\-ма\-ти\-за\-ции трансформаций первого и~второго порядка 
ее объектов.
  
  Трансформациями первого порядка, о которых сказано в~формулировке цели 
статьи, называются взаимные преобразования между двумя объектами  
сис\-те\-мы одной природы. Например, перевод в~сис\-те\-ме текста с~русского 
языка на английский относится к~ним. Трансформациями второго порядка 
и~выше называются взаимные преобразования между двумя и~более объектами 
разной природы. Например, кодирование символов текс\-та компьютерными 
кодами и~их декодирование относятся по определению к~трансформациям 
второго порядка.

%\vspace*{-9pt}
  
\section{Процессы трансформаций в~информатике}

%\vspace*{-3pt}

Процессы трансформаций, рассматриваемые в~статье, относятся к~теоретическому ядру информатики, а~не 
только к~проектированию лексикографической информационной сис\-те\-мы. Например, из трех основных 
подходов к~описанию предметной об\-ласти информатики\footnote{В статье предметная область информатики 
трактуется согласно концепции полиадического компьютинга Пола Розенблума~\cite{9-zac}.} (объектный, 
трансформационный и~синтетический) сис\-те\-ма\-ти\-за\-ция трансформаций ближе всего ко второму 
подходу. Примерами первого подхода, в~рамках которого основное внимание уделяется объектам предметной 
области информатики и~в~меньшей степени отношениям\linebreak между ними, могут служить  
работы~\cite{8-zac, 10-zac, 11-zac}; \mbox{примерами} второго подхода, в~рамках которого основное внимание 
уделяется трансформациям и~в~меньшей степени трансформируемым объектам,~---  
работы~\cite{12-zac, 13-zac}; примерами третьего, синтетического подхода, в~котором уделяется внимание 
и~объектам предметной об\-ласти информатики, и~отношениям между ними, могут служить работы~\cite{14-zac, 
15-zac, 16-zac, 17-zac, 18-zac}.

  Таким образом, для описания трансформаций объектов лексикографической 
информационной\linebreak системы предпочтительнее всего трансформационный 
подход, который упоминается и~в определениях информатики. Например, 
в~2009~г.\ П.~Деннинг и~П.~Розенблум сформулировали суть \mbox{информатики} как 
компьютинга следующим образом: <<$\ldots$информатика~--- это не просто 
алгоритмы и~структуры данных; это преобразования [трансформации] 
представлений>>~\cite{12-zac}. Чуть позже, в~контексте краткого описания 
парадигмы информатики как компьютинга, П.~Деннинг и~П.~Фриман изменили 
эту формулировку на такую: <<Центральный объект внимания в~информатике 
можно определить как информационные процессы~--- \textit{естественные или 
искусственные процессы, преобразующие информацию} (курсив мой~--- 
И.\,З.)>>~\cite{13-zac}. Согласно парадигме, предлагаемой авторами этой 
статьи, на начальном этапе проектирования автоматизированных систем 
базовыми элементами моделей их функционирования служат 
\textit{информационные про\-цессы}.
  
  Однако если 15~лет назад в~формулировке из работы~\cite{13-zac} шла речь 
о~процессах, преобразующих информацию, то в~последние~10~лет в~спектр 
процессов трансформаций все чаще стали включать процессы, преобразующие 
не только информацию, но также и~другие объекты автоматизированных 
систем, в~первую очередь данные и~знания~[19--21]. Например, Виктория 
Стодден, позиционируя науку о~данных как одну из дисциплин информатики, 
говорит, что центральный объект исследований в~науке о~данных~--- это 
<<изучение обобщаемого извлечения знания из данных>>~\cite{21-zac}. 
Увеличение и~чис\-ла объектов, и~спект\-ра процессов их трансформаций 
в~автоматизированных сис\-те\-мах обуслов\-ли\-ва\-ет не\-об\-хо\-ди\-мость 
систематизации и~объектов, и~процессов их трансформаций на начальном этапе 
проектирования сис\-тем.
  
  Для создания концепции лексикографической информационной сис\-те\-мы 
и~проектирования ИТ, обеспечивающих интеграцию 
двуязычных словарей и~параллельных корпусов, сначала выполним 
категоризацию на объекты этой сис\-те\-мы трех базовых понятий информатики 
(данные, информация, знание) в~контексте построения классификаций 
сущностей ее предметной об\-ласти.
  
  Необходимость использования классификаций информатики в~процессе 
создания концепции проиллюстрируем, используя иерархию  
Акоффа~\cite{8-zac}. Он использовал принцип их вертикального размещения 
в~иерархии снизу вверх: данные, информация и~знание. Еще в~ней есть термин 
<<мудрость>>, который в~статье не рассматривается. Такое размещение Акофф 
прокомментировал так: <<Каждое из пе\-ре\-чис\-лен\-ных понятий [кроме данных] 
содержит в~себе нижестоящие$\ldots$>>~\cite{8-zac}.
  
  Этому принципу размещения и~комментарию Акоффа свойственны 
недостатки, проанализированные, в~частности, в~работе~\cite{10-zac}. Главный 
вывод, к~которому пришла Роули после изучения иерархии Акоффа, 
заключается в~следующем: <<$\ldots$информация определяется в~терминах 
данных, знание~--- в~терминах информации$\ldots$ но существует меньше 
консенсуса в~описании трансформаций, которые преобразуют сущности, 
расположенные ниже в~иерархии, в~те, которые находятся над ними, что 
приводит к~их терминологической неопределенности>>~\cite{10-zac}. Причина 
этой неопределенности, скорее всего, в~том, что базовые понятия информатики 
включены в~иерархию Акоффа изолированно от общего контекста 
классификаций сущностей ее предметной об\-ласти.

%\vspace*{-9pt}
  
\section{Классификации сущностей информатики}


%\vspace*{-2pt}

  Все сущности предметной области информатики в~работах~[22, 23] 
разделены на два глобальных класса: ее объекты и~их трансформации. Для 
каждого такого класса была предложена своя классификация. 
В~работе~\cite{22-zac} дано описание классификации объектов предметной 
области информатики, первый уровень которой содержит базовые понятия ее 
предметной области (данные, информация, знания и~др.).  
В~работе~\cite{23-zac} дано описание двух верхних уровней классификации 
трансформаций объектов предметной об\-ласти (см.\ рисунок 
в~работе~\cite{23-zac}). Основанием для построения самого верхнего ее уровня послужило деление 
предметной области информатики на среды\footnote{В~работе~\cite{24-zac} дано описание пяти сред 
предметной области информатики (ментальная; сенсорно воспринимаемая, или информационная; 
цифровая; нейро- и~ДНК-среда), каждая из которых по определению включает объекты одной и~той же 
природы.} и~степень разнообразия природы объектов, вовлеченных в~трансформации:
\begin{itemize}
\item  первый класс верхнего уровня классификации включает 
трансформации объектов в~пределах среды только одной природы 
(трансформации первого порядка);
\item  второй класс включает трансформации объектов, относящихся 
к~двум средам разной природы (трансформации второго порядка);
\item третий и~последующие классы включают трансформации объектов, 
относящихся к~трем и~более средам разной природы (трансформации 
третьего и~более высоких порядков).
\end{itemize}

  В работе~\cite{23-zac} были приведены примеры для трех первых классов 
трансформаций, включая пример трансформаций объектов, относящихся 
к~двум средам разной природы (компьютерное кодирование символов текстов 
с~по\-мощью таб\-лиц Unicode).
  
Основанием для построения второго уровня классификации трансформаций объектов послужила типология 
знаковых сис\-тем А.~Соломоника~\cite[c.~131]{25-zac}: естественные знаковые сис\-те\-мы, образные,  
ес\-тест\-вен\-но-язы\-ко\-в$\acute{\mbox{ы}}$е,  
вер\-баль\-но-не\-сло\-вес\-ные сис\-те\-мы записи\footnote{Под системой записи понимается знаковая 
система, сочетающая вербальные знаки с~несловесными (языки нотной записи, карт, таблиц и~др.).} 
и~формализованные знаковые сис\-те\-мы, включая математические. Введем понятие обобщенного текста~--- 
это текст, который может быть создан в~любой из перечисленных знаковых систем. Тогда обобщенные тексты 
могут быть естественными, образными, ес\-тест\-вен\-но-язы\-ко\-в$\acute{\mbox{ы}}$\-ми,  
вер\-баль\-но-не\-сло\-вес\-ны\-ми и~формализованными. Второй уровень классификации трансформаций 
охватывает не все виды объектов предметной  
об\-ласти информатики, а~только перечисленные~5~видов текс\-тов и~их представления, вовлеченные 
в~процессы трансформаций в~одной или более средах вместе с~данными, знанием и~его концептами.

\begin{figure*}[b] %fig1
\vspace*{6pt}
      \begin{center}
     \mbox{%
\epsfxsize=121.191mm 
\epsfbox{zac-1.eps}
}
\end{center}
\vspace*{-6pt}
\Caption{Средовая версия иерархии Акоффа}
\end{figure*}

\section{Классификация трансформаций: построение~третьего 
уровня}

  Основанием для систематизации трансформаций первого и~второго порядка 
на третьем уровне этой классификации служит иерархия Акоффа~\cite{8-zac}, 
на основе которой и~была создана ее средов$\acute{\mbox{а}}$я версия~[26, 
27]. Для создания средов$\acute{\mbox{о}}$й версии была выполнена 
категоризация трех базовых понятий информатики (данные, информация, 
знания) на объекты лексикографической информационной сис\-те\-мы 
в~процессе создания ее концепции\linebreak (рис.~1).
  


  В отличие от классической иерархии Акоффа, в~ее 
средов$\acute{\mbox{о}}$й версии различаются три вида данных: сенсорно 
воспринимаемые, цифровые и~те данные, которые генерируются 
искусственными нейронными сетями (ИНС) в~системах искусственного интеллекта 
(далее~--- ИИ-дан\-ные). Последний вид данных необходим, например, для 
различения входа и~выхода процесса применения обученной 
ИНС в~цифровой модели генерации знания, описанию которой 
посвящена работа~\cite{27-zac}.
  
  Также предлагается различать два вида информации: сенсорно 
воспринимаемая и~цифровая. Кроме знания в~средов$\acute{\mbox{у}}$ю 
версию добавлены концепты и~ментальные образы сенсорно воспринимаемых 
данных. Последние служат промежуточной сущностью между сенсорно 
воспринимаемыми данными и~генерируемым знанием при описании процессов 
извлечения знания из текстовых данных лексикографической информационной 
системы. Описание объектов средов$\acute{\mbox{о}}$й версии иерархии 
Акоффа (см.\ рис.~1) и~отношений между ними дано в~работах~\cite{26-zac, 28-zac}.
  
  В средов$\acute{\mbox{о}}$й версии число объектов равно восьми. Если 
учитывать направления трансформаций, то между восемью объектами на 
рис.~1 она включает~16 их видов (трансформации на границе между сенсорно 
воспринимаемыми данными и~информацией, обозначенные символом~<<?>>, 
в~статье не рас\-смат\-ри\-ва\-ют\-ся). В~будущем число объектов 
в~средов$\acute{\mbox{о}}$й версии, которая выбрана как основание для 
сис\-те\-ма\-ти\-за\-ции трансформаций первого и~второго порядка, может быть 
увеличено. Для построения классификации трансформаций 
важ\-но не возможное увеличение числа объектов 
и~трансформаций между ними, а то, что их виды в~средов$\acute{\mbox{о}}$й 
версии распределены между трансформациями первого и~второго порядка. Из 
16~видов на рис.~1 шесть относятся к~трансформациям первого порядка, это\linebreak 
виды с~номерами~7, 8, 13--16 (далее~--- типология трансформаций первого 
порядка), а~десять~--- к~трансформациям второго порядка, это виды 
с~\mbox{номерами}~1--6 и~9--12 (далее~--- типология трансформаций второго 
порядка). Разместим обе типологии на третьем уровне классификации (см.\ ее 
схему на рис.~2). Перечислим виды трансформаций первой типологии, вводя 
в~скобках их краткие названия, используемые ниже на рис.~3:
  \begin{description}
  \item[\,] 7~--- членение знания на концепты с~помощью одной или нескольких 
знаковых систем (далее~--- членение знания);
  \item[\,] 8~--- формирование знания на основе концептов (формирование 
знания);
  \item[\,] 13~--- обучение ИНС;
  \end{description}
  
  \vspace*{-6pt}
  
  \pagebreak
  
  \end{multicols}
  
  \begin{figure*} %fig2
\vspace*{1pt}
      \begin{center}
     \mbox{%
\epsfxsize=127.513mm 
\epsfbox{zac-2.eps}
}
\end{center}
\vspace*{-9pt}
\Caption{Схема трех верхних уровней классификации трансформаций объектов (объединены 
по три слоя и~для второго, и~для третьего уровней этой классификации)}
\end{figure*}
  
  \begin{multicols}{2}
  
  \noindent
  \begin{description}
  \item[\,] 14~--- восстановление обучающей информации на основе 
содержания обученной ИНС (обращение ИНС);
  \item[\,] 15~--- использование обученной ИНС (использование ИНС);



  \item[\,] 16~--- восстановление исходных данных, соответствующих 
полученным результатам работы обучен\-ной ИНС (восстановление исходных данных 
по результатам ИНС).
  \end{description}
  
  
  Не все виды трансформаций 13--16 поддерживаются в~конкретных системах 
искусственного интеллекта, но с~теоретической точки зрения все их 
предлагается включить в~первую типологию для полноты спектра видов 
трансформаций.
  
  Перечислим виды трансформаций второй типологии:
  \begin{description}
  \item[\,] 1~--- декодирование цифровых данных в~компьютерных системах 
(декодирование данных);
  \item[\,]  2~--- кодирование сенсорно воспринимаемых данных (кодирование 
данных);
  \item[\,] 3~--- ментальное копирование сенсорно воспринимаемых данных 
(ментальное копирование);
  \item[\,] 4~--- восстановление сенсорно воспринимаемых данных по 
ментальным образам (восстановление по образам);
  \item[\,] 5~--- смысловая интерпретация без деления на концепты ментальных 
образов сенсорно воспринимаемых данных (смысловая интерпретация);
  \item[\,] 6~--- восстановление ментальных образов (восстановление образов);
  \item[\,] 9~--- представление концептов в~виде сенсорно воспринимаемой 
информации, например текс\-та\-ми, формулами, таблицами, рисунками и~т.\,д.\ 
(представление концептов);
  \item[\,] 10~--- понимание смысла сенсорно воспринимаемой информации 
(понимание смысла);
  \item[\,] 11~--- кодирование сенсорно воспринимаемой информации 
(кодирование информации);
\end{description}

\vspace*{-6pt}

\pagebreak

\end{multicols}

\begin{figure*} %fig3
\vspace*{1pt}
      \begin{center}
     \mbox{%
\epsfxsize=163mm 
\epsfbox{zac-3.eps}
}
\end{center}
\vspace*{-9pt}
\Caption{Схема частного случая классификации трансформаций объектов (трансформации 
пронумерованы согласно рис.~1)}
\end{figure*}

\begin{multicols}{2}

\noindent
\begin{description}

  \item[\,] 12~--- декодирование цифровой информации (декодирование 
информации).
  \end{description}
  
  Отметим, что в~существующих ИТ
  и~компьютерных системах наиболее часто используются виды 
трансформаций~13 и~15 типологии первого порядка и~1, 2, 11 и~12 типологии 
второго порядка. На рис.~2 в~первом слое третьего уровня классификации 
показаны типологии первого порядка без указания числа трансформаций в~них 
и~без детализации трансформируемых объектов.
  
  Во втором слое третьего уровня классификации условно (без названий) 
показаны типологии второго порядка. Также на рис.~2 в~третьем слое третьего 
уровня классификации условно (также без названий) показаны типологии 
третьего порядка, которые планируется рассмотреть в~отдельной статье. По 
определению они должны включать трансформации между тремя объектами 
разной природы, но средов$\acute{\mbox{а}}$я версия иерархии Акоффа 
включает трансформации только между двумя объектами разной природы. 
Поэтому потребуется другое основание для их систематизации (ранее были 
рассмотрены отдельные примеры трансформаций третьего 
порядка\footnote{Далеко не всегда трансформации третьего и~более высоких порядков можно 
рассматривать как последовательность трансформаций второго порядка. Примером этого могут 
служить трансформации в~процессе обучения пациента пользованию роботизированной рукой, 
охватывающие личностные концепты пациента, релевантные его намерениям, сигналы активности 
мозга как объекты нейросреды и~компьютерные коды~\cite{29-zac}.}~\cite{29-zac}).

\section{Классификация трансформаций: частный~случай}

  Выше было отмечено, что в~будущем число объектов 
в~средов$\acute{\mbox{о}}$й версии иерархии Акоффа может быть увеличено. 
Это означает, что увеличатся и~чис\-ло объектов, и~чис\-ло трансформаций между 
ними в~классификации трансформаций, так как эта средов$\acute{\mbox{а}}$я 
версия служит по определению основанием для систематизации 
трансформаций первого и~второго порядка. Поэтому на третьем уровне рис.~2 
указаны типологии без детализации объектов и~без указания числа 
трансформаций в~каждой из них. С~одной стороны, при таком подходе 
получаем достаточно общий вид этой классификации, так как она не зависит от 
числа объектов в~том или ином варианте средов$\acute{\mbox{о}}$й версии 
(и~это существенно упрощает рис.~2). С~другой стороны, на третьем уровне 
такой общей классификации подразумевается, но не эксплицируется природа 
трансформируемых объектов и~их возможные сочетания в~трансформациях. 

При проектировании лексикографической информационной системы важно 
эксплицировать природу трансформируемых объектов и~их возможные 
сочетания.
  %
  Поэтому в~парадигму информатики~\cite{30-zac} кроме общей 
классификации трансформаций предлагается включать и~ее частные случаи, 
эксплицирующие природу трансформируемых объектов. 

В~этом разделе 
рассмотрим один частный случай, когда используются только естественные 
знаковые сис\-те\-мы из типологии А.~Соломоника~\cite{25-zac} вместе 
с~данными, знанием и~его концептами. Чис\-ло естественных языков при этом не 
ограничено. И~этот частный случай классификации включает только три 
класса природных трансформаций (первого, второго и~третьего порядка, см.\ 
схему классификации на рис.~3).
  
  Первый и~второй уровни схемы общей классификации (см.\ рис.~2) можно 
объединить в~один уровень в~этом частном случае. Ниже этого уровня 
приведено содержание типологий первого и~второго порядка без содержания 
типологий третьего по\-рядка.




  Наполнение типологий первого и~второго порядка соответствует 
средов$\acute{\mbox{о}}$й версии иерархии Акоффа на рис.~1, содержащей 
6~видов трансформаций типологии первого порядка и~10~видов 
трансформаций типологии второго порядка (на рис.~3 стрелки указывают 
направления трансформаций согласно средов$\acute{\mbox{о}}$й версии на рис.~1).
  
  Таким образом, частный случай классификации содержит для этих двух 
типологий 16~теоретически возможных трансформаций, 6 из которых 
в~настоящее время в~существующих ИТ применяются наиболее часто: виды 
трансформаций~1, 2, 11 и~12 типологии второго порядка реализуются 
с~помощью тех или иных методов ко\-ди\-ро\-ва\-ния/де\-ко\-ди\-ро\-ва\-ния 
(например, с~использованием таблиц Unicode), а~виды трансформаций~13 и~15
 в~типологии первого порядка реализуются полностью с~по\-мощью процессов 
цифровой обработки компьютерами.
  
  Остальные виды трансформаций или применяются намного реже (это 
виды~3, 5, 7, 9 и~10), или находятся в~стадии поиска и~разработки (14 и~16) или 
в~настоящее время носят только теоретический характер, обеспечивая полноту 
первой и~второй типологий (4, 6 и~8). Знаком~<<?>> обозначены те виды 
трансформаций, которые по определению не существуют в~используемой 
парадигме информатики~\cite{30-zac}. Однако возможно, что в~других 
будущих подходах к~построению ее парадигмы эти виды трансформаций будут 
существовать.
  
\section{Заключение}

  На сегодняшний день процесс построения классификаций объектов 
предметной области информатики~\cite{22-zac} и~их  
трансформаций~\cite{23-zac} еще не завершен. Однако первые результаты их 
построения уже используются для создания концепции лексикографической 
информационной сис\-те\-мы, обеспечивающей интеграцию двуязычных 
словарей и~параллельных корпусов.
  
  \bigskip
  
  
  Автор признателен рецензентам за помощь в~улучшении статьи.
  
{\small\frenchspacing
 { %\baselineskip=10.6pt
 %\addcontentsline{toc}{section}{References}
 \begin{thebibliography}{99}
\bibitem{1-zac}
\Au{Aijmer K., Altenberg~B.} Advances in corpus-based contrastive linguistics. Studies in honour 
of Stig Johansson.~--- Amsterdam: John Benjamins, 2013. 295~p.  doi: 10.1075/scl.54.
\bibitem{2-zac}
\Au{Добровольский Д.\,О., Кретов~А.\, А., Шаров~С.\,А.} Корпус параллельных текстов~// 
Научная и~техническая информация. Сер.~2: Информационные процессы и~сис\-те\-мы, 2005. 
№\,6. С.~16--27.
\bibitem{3-zac}
\Au{Добровольский Д.\,О.} Корпус параллельных текстов и~сопоставительная 
лексикология~// Труды Института русского языка им.\ В.\,В.~Виноградова, 2015. №\,6. 
С.~413--449. EDN: VJQBHP.
\bibitem{4-zac}
\Au{Гончаров А.\,А., Зацман~И.\,М., Кружков~М.\,Г.} Эволюция классификаций 
в~надкорпусных базах данных~// Информатика и~её применения, 2020. Т.~14. Вып.~4. 
С.~108--116. doi: 10.14357/19922264200415.  
EDN: \mbox{GKWBZT}.
\bibitem{5-zac}
\Au{Гончаров А.\, А., Зацман И. \,М., Кружков~М.\, Г}. Представление новых 
лексикографических знаний в~динамических классификационных сис\-те\-мах~// 
Информатика и~её применения, 2021. Т.~15. Вып.~1. С.~86--93.  doi: 10.14357/19922264210112. EDN: OPEFXW.
\bibitem{6-zac}
\Au{Zatsman I.} Finding and filling lacunas in linguistic typologies~// 15th Forum (International) 
on Knowledge Asset Dynamics Proceedings.~--- Matera, Italy: Institute of Knowledge Asset 
Management, 2020. P.~780--793.
\bibitem{7-zac}
\Au{Zatsman I.} Three-dimensional encoding of emerging meanings in AI-systems~// 21st 
European Conference on Knowledge Management Proceedings.~--- Reading, U.K.: Academic 
Publishing International Ltd., 2020. P.~878--887.
\bibitem{8-zac}
\Au{Ackoff R.} From data to wisdom~// J.~Applied Systems Analysis, 1989. Vol.~16. No.\,1. P.~3--9.
\bibitem{9-zac}
\Au{Rosenbloom P.\,S.} On computing: The fourth great scientific domain.~--- Cambridge, MA, 
USA: MIT Press, 2013. 307~p.
\bibitem{10-zac}
\Au{Rowley J.} The wisdom hierarchy: Representations of the DIKW hierarchy~// J.~Inf. 
Sci., 2007. Vol.~33. Iss.~2. P.~163--180. doi: 10.1177/0165551506070706.
\bibitem{11-zac} 
\Au{Frick$\acute{\mbox{e}}$~M.\,H.} Data--Information--Knowledge--Wisdom (DIKW) pyramid, 
framework, continuum~// Encyclopedia of big data~/ Eds. L.~Schintler, C.~McNeely.~--- Cham: 
Springer, 2018. 4~p. doi: 10.1007/978-3-319-32001-4\_331-1.
\bibitem{12-zac}
\Au{Denning P., Rosenbloom~P.} Computing: The fourth great domain of science~// Commun. 
ACM, 2009. Vol.~52. Iss.~9. P.~27--29.
\bibitem{13-zac}
\Au{Denning P., Freeman~P.} Computing's paradigm~// Commun.  ACM, 2009. Vol.~52. 
Iss.~12. P.~28--30. doi: 10.1145/ 1610252.1610265.
\bibitem{17-zac} %14
\Au{Farradane J.} Knowledge, information, and information science~// J.~Inf. Sci., 
1980. Vol.~2. Iss.~2. P.~75--80. doi: 10.1177/01655515800020020.

\bibitem{15-zac}
\Au{Шрейдер Ю.\,А.} Информация и~знание~// Сис\-тем\-ная концепция информационных 
процессов.~--- М.: ВНИИСИ, 1988. С.~47--52.
\bibitem{16-zac}
\Au{Ingwersen P.} Information and information science~// Enclyclopaedie of library and 
information science~/ Eds. J.\,D.~McDonald, 
M.~Levine-Clark.~--- New York, NY, USA: Marcel Dekker Inc., 1992. Vol.~56. Sup.~19. 
P.~137--174.

\bibitem{14-zac} %17
Информатика как наука об информации: Информационный, документальный, 
технологический, экономический, социальный и~организационный аспекты~/ Под ред. 
Р.\,С.~Гиляревского.~--- М.: Фаир-Пресс, 2006. 592~с.

\bibitem{18-zac}
\Au{Hjorland B.} Library and information science: practice, theory, and philosophical basis~// 
Inform. Process. Manag., 2000. Vol.~36. Iss.~3. P.~501--531. doi:  
10.1016/S0306-\mbox{4573(99)00038-2}.
\bibitem{19-zac}
Deep shift~--- technology tipping points and societal impact.~--- Geneva: WE Forum, 2015. 44~p. 
{\sf http://www3.weforum.org/docs/WEF\_GAC15\_ Technological\_Tipping\_Points\_report\_2015.pdf}.
\bibitem{20-zac}
\Au{Berman F., Rutenbar~R., Hailpern~B., Christensen~H., Davidson~S., Estrin~D., 
Franklin~M., Martonosi~M., Raghavan~P., Stodden~V., Szalay~A.\,S.} Realizing the potential of 
data science~// Commun.  ACM, 2018. Vol.~61. Iss.~4. P.~67--72. doi: 10.1145/3188721.

\bibitem{21-zac}
\Au{Stodden V.} The data science life cycle: A~disciplined approach to advancing data science as 
a~science~// Commun.  ACM, 2020. Vol.~63. Iss.~7. P.~58--66. doi: 10.1145/ 3360646.


\bibitem{23-zac} %22
\Au{Зацман И.\,М.} Научная парадигма информатики: классификация трансформаций 
объектов предметной об\-ласти~// Системы и~средства информатики, 2023. Т.~33. №\,4. 
С.~126--138. doi: 10.14357/08696527230412. EDN: ZIKUWO.

\bibitem{22-zac} %23
\Au{Зацман И.\,М.} Научная парадигма информатики: классификация объектов предметной  
об\-ласти~// Информатика и~её применения, 2023. Т.~17. Вып.~4. С.~96--103. doi: 
10.14357/19922264230413. EDN: FIUQAT.

\bibitem{24-zac}
\Au{Зацман И.\,М.} О~научной парадигме информатики: верхний уровень классификации 
объектов ее предметной об\-ласти~// Информатика и~её применения, 2022. Т.~16. Вып.~4. 
С.~73--79. doi: 10.14357/ 19922264220411. EDN: XZNKVI.

\bibitem{25-zac}
\Au{Соломоник А.\,Б.} Философия знаковых систем и~язык.~--- М.: ЛКИ, 2011. 408~с.
\bibitem{26-zac}
\Au{Зацман И.\,М.} Трансформация иерархии Акоффа в~научной парадигме информатики~// 
Информатика и~её применения, 2023. Т.~17. Вып.~3. С.~107--113. doi: 
10.14357/19922264230315. EDN: UMVRRV.

\bibitem{27-zac}
\Au{Zatsman I.} Building digital spiral models of knowledge generation~// 19th Forum 
(International) on Knowledge Asset Dynamics Proceedings.~--- Matera, Italy: Arts for Business 
Institute, 2024. P.~2185--2196.
\bibitem{28-zac}
\Au{Zatsman I.} Digital spiral model of knowledge creation and encoding its dynamics~// 18th 
Forum (International) on Knowledge Asset Dynamics Proceedings.~--- Matera, Italy: Arts for 
Business Institute, 2023. P.~581--596.
\bibitem{29-zac}
\Au{Зацман И.\,М.} Интерфейсы третьего порядка в~информатике~// Информатика и~её 
применения, 2019. Т.~13. Вып.~3. С.~82--89. doi: 10.14357/19922264190312. EDN: 
EHRQLF.

\bibitem{30-zac}
\Au{Зацман И.\,М.} Научная парадигма информатики как третьей культуры~//  
На\-уч\-но-тех\-ни\-че\-ская информация. Сер.~1: Организация и~методика информационной 
работы, 2023. №\,11. С.~1--14.

\end{thebibliography}

 }
 }

\end{multicols}

\vspace*{-9pt}

\hfill{\small\textit{Поступила в~редакцию 14.04.24}}

\vspace*{4pt}

%\pagebreak

%\newpage

%\vspace*{-28pt}

\hrule

\vspace*{2pt}

\hrule



\def\tit{OBJECT TRANSFORMATIONS OF~THE~FIRST AND~SECOND ORDER
IN~A~LEXICOGRAPHIC INFORMATION SYSTEM\\[-5pt]}


\def\titkol{Object transformations of~the~first and~second order
in~a~lexicographic information system}


\def\aut{I.\,M.~Zatsman}

\def\autkol{I.\,M.~Zatsman}

\titel{\tit}{\aut}{\autkol}{\titkol}

\vspace*{-13pt}


\noindent
Federal Research Center ``Computer Science and Control'' of the Russian Academy of Sciences, 
44-2~Vavilov Str., Moscow 119133, Russian Federation


\def\leftfootline{\small{\textbf{\thepage}
\hfill INFORMATIKA I EE PRIMENENIYA~--- INFORMATICS AND
APPLICATIONS\ \ \ 2024\ \ \ volume~18\ \ \ issue\ 2}
}%
 \def\rightfootline{\small{INFORMATIKA I EE PRIMENENIYA~---
INFORMATICS AND APPLICATIONS\ \ \ 2024\ \ \ volume~18\ \ \ issue\ 2
\hfill \textbf{\thepage}}}

\vspace*{2pt}



\Abste{The theoretical foundations of the design of information technologies used for 
the integration of bilingual dictionaries and parallel corpora are considered. The 
description of the first outcomes of the creation of the third\linebreak\vspace*{-12pt}}

\Abstend{ level of object 
transformations classification in the subject domain of informatics, which is supposed 
to be used
in creating the lexicographic information system providing integration, is 
given. All the entities of informatics are divided into two global classes: objects and 
their transformations. For each such class, its own classification is constructed. 
Previously, the two upper levels of the object transformation classification in the subject 
domain have been described. The present paper discusses the third level of this classification. The 
basis for the construction of its highest level was the division of the subject domain of 
informatics into media (mental, sensory, digital, and a~number of other media), each 
of which by definition includes objects of the same nature. The Solomonick's 
typology of sign systems served as the basis for constructing the second level of the 
object transformation classification. The aim of the paper is to systematize object 
transformations of the first and second orders at the third level of this classification. 
The basis for systematization is the medium version of the Ackoff's hierarchy.}

\KWE{subject domain objects; object transformations; classification; data; 
information; knowledge; lexicographic information system}


\DOI{10.14357/19922264240211}{VZTGVV}

\vspace*{-12pt}

\Ack

\vspace*{-3pt}


\noindent
The reported study was funded by the Russian Science Foundation, project  
No.\,24-18-00155, {\sf 
https://rscf.ru/project/24-18-00155}. The research was carried out using the infrastructure of the Shared 
Research Facilities ``High Performance Computing and Big Data'' (CKP 
``Informatics'') of FRC CSC RAS (Moscow) .
   


  \begin{multicols}{2}

\renewcommand{\bibname}{\protect\rmfamily References}
%\renewcommand{\bibname}{\large\protect\rm References}

{\small\frenchspacing
 {%\baselineskip=10.8pt
 \addcontentsline{toc}{section}{References}
 \begin{thebibliography}{99} 
\bibitem{1-zac-1}
\Aue{Aijmer, K., and B.~Altenberg.} 2013. \textit{Advances in corpus-based 
contrastive linguistics. Studies in honour of Stig Johansson}. Amsterdam: John 
Benjamins. 295~p. doi: 10.1075/scl.54.
\bibitem{2-zac-1}
\Aue{Dobrovolskiy, D.\,O., A.\,A.~Kretov, and S.\,A.~Sharov.} 2005. Korpus 
parallel'nykh tekstov [Corpus of parallel texts]. \textit{Nauchnaya i~tekhnicheskaya 
informatsiya. Ser. 2. Informatsionnye protsessy i~sistemy} [Scientific and Technical 
Information. Ser.~2: Information Processes and Systems] 6:16--27.
\bibitem{3-zac-1}
\Aue{Dobrovolskiy, D.\,O.} 2015. Korpus parallel'nykh tekstov i~sopostavitel'naya 
leksikologiya [The corpus of parallel texts and contrastive lexicology]. \textit{Trudy 
Instituta russkogo yazyka im. V.\,V.~Vinogradova} [Proceedings of the 
V.\,V.~Vinogradov Russian Language Institute] 6:413--449. EDN: VJQBHP.
\bibitem{4-zac-1}
\Aue{Goncharov, A.\,A., I.\,M.~Zatsman, and M.\,G.~Kruzhkov.} 2020. Evolyutsiya 
klassifikatsiy v~nadkorpusnykh ba\-zakh dannykh [Evolution of classifications in 
supracorpora databases]. \textit{Informatika i~ee Primeneniya~--- Inform. \mbox{Appl.}}  
14(4):108--116. doi: 10.14357/19922264200415.  
EDN: GKWBZT.
\bibitem{5-zac-1}
\Aue{Goncharov, A.\,A., I.\,M.~Zatsman, and M.\,G.~Kruzhkov.} 2021. 
Predstavlenie novykh leksikograficheskikh znaniy v~dinamicheskikh 
klassifikatsionnykh sistemakh [Representation of new lexicographical knowledge in 
dynamic classification systems]. \textit{Informatika i~ee Primeneniya~--- Inform. 
Appl.} 15(1):86--93. doi: 10.14357/19922264210112. EDN: OPEFXW.
\bibitem{6-zac-1}
\Aue{Zatsman, I.} 2020. Finding and filling lacunas in linguistic typologies. 
\textit{15th Forum (International) on Knowledge Asset Dynamics Proceedings}. 
Matera, Italy: Institute of Knowledge Asset Management. 780--793.
\bibitem{7-zac-1}
\Aue{Zatsman, I.} 2020. Three-dimensional encoding of emerging meanings in  
AI-systems. \textit{21st European Conference on Knowledge Management 
Proceedings}. Reading, U.K.: Academic Publishing International Ltd. 878--887.
\bibitem{8-zac-1}
\Aue{Ackoff, R.} 1989. From data to wisdom. \textit{J.~Applied Systems Analysis} 
16(1):3--9.
\bibitem{9-zac-1}
\Aue{Rosenbloom, P.\,S.} 2013. \textit{On computing: The fourth great scientific 
domain}. Cambridge, MA: MIT Press. 307~p.
\bibitem{10-zac-1}
\Aue{Rowley, J.} 2007. The wisdom hierarchy: Representations of the DIKW 
hierarchy. \textit{J.~Inf. Sci.} 33(2):163--180. doi: 10.1177/0165551506070706.
\bibitem{11-zac-1}
\Aue{Frick$\acute{\mbox{e}}$, M.\,H.} 2018.  
Data-Information-Knowledge-Wisdom (DIKW) pyramid, framework, continuum. 
\textit{Encyclopedia of big data}. Eds. L.~Schintler and C.~McNeely. Cham: 
Springer. 4~p. doi: 10.1007/978-3-319-32001- 4\_331-1.
\bibitem{12-zac-1}
\Aue{Denning, P., and P.~Rosenbloom.} 2009. Computing: The fourth great domain 
of science. \textit{Commun. ACM} 52(9):27--29.
\bibitem{13-zac-1}
\Aue{Denning, P., and P.~Freeman.} 2009. Computing's paradigm. \textit{Commun. 
ACM} 52(12):28--30. doi: 10.1145/ 1610252.1610265.

\bibitem{17-zac-1} %14
\Aue{Farradane, J.} 1980. Knowledge, information, and information science. 
\textit{J.~Inf. Sci.} 2(2):75--80. doi: 10.1177/ 01655515800020020.

\bibitem{15-zac-1}
\Aue{Shreyder, Yu.\,A.} 1988. Informatsiya i~znanie [Information and knowledge]. 
\textit{Sistemnaya kontseptsiya in\-for\-ma\-tsi\-on\-nykh protsessov} [System concept of 
information processes]. Moscow: VNIISI. 47--52.
\bibitem{16-zac-1}
\Aue{Ingwersen, P.} 1995. Information and information science. 
\textit{Encyclopedia of library and information science}. Eds. J.\,D.~McDonald and 
M.~Levine-Clark. New York, NY: Marcel Dekker Inc. 56(19):137--174.

\bibitem{14-zac-1} %17
Gilyarevskiy, R.\,S., ed. 2006. \textit{Informatika kak nauka ob informatsii: 
informatsionnyy, dokumental'nyy, tekh\-no\-lo\-gi\-che\-skiy, ekonomicheskiy, sotsial'nyy 
i~organizatsionnyy aspekty} [Informatics as information science: Informational, 
documentary, technological, economic, social, and organizational dimensions]. 
Moscow: FAIR-PRESS. 592~p.

\bibitem{18-zac-1}
\Aue{Hjorland, B.} 2000. Library and information science: Practice, theory, and 
philosophical basis. \textit{Inform. Process. Manag.} 36(3):501--531. doi:  
10.1016/S0306-\mbox{4573(99)00038-2}.
\bibitem{19-zac-1}
Deep shift~--- technology tipping points and societal impact. 2015. \textit{World Economic 
Forum}. Geneva. 44~p. Available at: {\sf 
http://www3.weforum.org/docs/WEF\_ GAC15\_Technological\_Tipping\_Points\_report\_2015.pdf} (accessed May~20, 
2024).
\bibitem{20-zac-1}
\Aue{Berman, F., R.~Rutenbar, B.~Hailpern, H.~Christensen, S.~Davidson, 
D.~Estrin, M.~Franklin, M.~Martonosi, P.~Raghavan, V.~Stodden, and 
A.\,S.~Szalay.} 2018. Realizing the potential of data science. \textit{Commun. ACM} 
61(4):67--72. doi: 10.1145/3188721.
\bibitem{21-zac-1}
\Aue{Stodden, V.} 2020. The data science life cycle: A~disciplined approach to 
advancing data science as a~science. \textit{Commun. ACM} 
 63(7):58--66. doi: 10.1145/3360646.

\bibitem{23-zac-1} %22
\Aue{Zatsman, I.\,M.} 2023. Nauchnaya paradigma informatiki: klassifikatsiya 
transformatsiy ob''ektov predmetnoy oblasti [Scientific paradigm of informatics: 
Transformation classification of domain objects]. \textit{Sistemy i~Sredstva 
Informatiki~--- Systems and Means of Informatics} 33(4):126--138. doi: 
10.14357/08696527230412. EDN: ZIKUWO.

\bibitem{22-zac-1} %23
\Aue{Zatsman, I.\,M.} 2023. Nauchnaya paradigma informatiki: klassifikatsiya 
ob''ektov predmetnoy oblasti [Scientific paradigm of informatics: Classification of 
domain objects]. \textit{Informatika i~ee Primeneniya~--- Inform. Appl.} 
 17(4):96--103. doi: 10.14357/19922264230413. EDN: FIUQAT.
 
\bibitem{24-zac-1}
\Aue{   Zatsman, I.\,M.} 2022. O nauchnoy paradigme informatiki: verkhniy uroven' 
klassifikatsii ob''ektov ee predmetnoy oblasti [On the scientific paradigm of 
informatics: The classification high level of its objects]. \textit{Informatika i~ee 
Primeneniya~--- Inform. Appl.} 16(4):73--79. doi: 10.14357/19922264220411. EDN: 
XZNKVI.
\bibitem{25-zac-1}
\Aue{Solomonick, A.\,B.} 2011. \textit{Filosofiya znakovykh system i~yazyk} 
[Philosophy of sign systems and language]. Moscow: LKI. 408~p.
\bibitem{26-zac-1}
\Aue{Zatsman, I.\,M.} 2023. Transformatsiya ierarkhii Akoffa v~nauchnoy 
paradigme informatiki [Transformation of the Ackoff's hierarchy in the scientific 
paradigm of informatics]. \textit{Informatika i~ee Primeneniya~--- Inform. \mbox{Appl.}} 
17(3):107--113. doi: 10.14357/19922264230315. EDN: UMVRRV.
\bibitem{27-zac-1}
\Aue{Zatsman, I.} 2024. Building digital spiral models of knowledge 
generation. \textit{19th Forum (International) on Knowledge Asset Dynamics 
Proceedings}. Matera, Italy: Arts for Business Institute. 2185--2196.
\bibitem{28-zac-1}
\Aue{Zatsman, I.} 2023. Digital spiral model of knowledge creation and encoding its 
dynamics. \textit{18th Forum (International) on Knowledge Asset Dynamics 
Proceedings}. Matera, Italy: Arts for Business Institute. 581--596.
\bibitem{29-zac-1}
\Aue{Zatsman, I.\,M.} 2019. Interfeysy tret'ego poryadka v~informatike 
 [Third-order interfaces in informatics]. \textit{Informatika i~ee Primeneniya~--- 
Inform. Appl.} 13(3):82--89. doi: 10.14357/19922264190312. EDN: EHRQLF.
\bibitem{30-zac-1}
\Aue{Zatsman, I.} 2023. Scientific paradigm of informatics as a~third culture. 
\textit{Scientific Technical Information Processing} 50(4):246--258. doi: 
10.3103/S0147688223040111. EDN: CKHMYS.

\end{thebibliography}

 }
 }

\end{multicols}

\vspace*{-6pt}

\hfill{\small\textit{Received April 14, 2024}} 


\vspace*{-12pt}


\Contrl

\vspace*{-3pt}

\noindent
\textbf{Zatsman Igor M.} (b.\ 1952)~--- Doctor of Science in technology, head of 
department, Federal Research Center ``Computer Science and Control'' of the 
Russian Academy of Sciences, 44-2~Vavilov Str., Moscow 119333, Russian 
Federation; \mbox{izatsman@yandex.ru}





\label{end\stat}

\renewcommand{\bibname}{\protect\rm Литература}     %8
\def\stat{nuriev}

\def\tit{МЕТОДОЛОГИЯ КОРПУСНО-ОРИЕНТИРОВАННОГО ИССЛЕДОВАНИЯ 
В~ОБЛАСТИ КОНТРАСТИВНОЙ ПУНКТУАЦИИ$^*$\\[-5pt]}

\def\titkol{Методология корпусно-ориентированного исследования 
в~области контрастивной пунктуации}

\def\aut{В.\,А.~Нуриев$^1$, В.\,И.~Карпов$^2$}

\def\autkol{В.\,А.~Нуриев, В.\,И.~Карпов}

\titel{\tit}{\aut}{\autkol}{\titkol}

\index{Нуриев В.\,А.}
\index{Карпов В.\,И.}
\index{Nuriev V.\,A.}
\index{Karpov V.\,I.}


{\renewcommand{\thefootnote}{\fnsymbol{footnote}} \footnotetext[1]
{Работа выполнена за счет гранта Российского научного фонда (проект 23-28-00548) с~использованием инфраструктуры 
Центра коллективного пользования <<Высокопроизводительные вычисления и~большие данные>> (ЦКП 
<<Информатика>>) ФИЦ ИУ РАН (г.~Москва).}}


\renewcommand{\thefootnote}{\arabic{footnote}}
\footnotetext[1]{Федеральный исследовательский центр <<Информатика и~управление>> Российской академии наук, 
\mbox{nurieff.v@gmail.com}}
\footnotetext[2]{Институт языкознания Российской академии наук; Федеральный исследовательский центр <<Информатика 
и~управ\-ле\-ние>> Российской академии наук, \mbox{wi.karpow@gmail.com}}

\vspace*{-3pt}

  
  
    
  \Abst{Уточняется  подход к~современным исследованиям 
в~об\-ласти контрастивной пунктуации с~точки зрения методологии. С~учетом новейших достижений информатики, 
компьютерной лингвистики и~теории перевода такие исследования очевидным образом 
должны иметь кор\-пус\-но-ори\-ен\-ти\-ро\-ван\-ный характер. В~данной статье представлена 
методологическая схема подобного исследования, направленного на выявление 
межъязыковой пунктуационной асим\-мет\-рии посредством сравнения функционального 
диапазона одного и~того же знака препинания в~разных языках. Показываются основные 
методологические тенденции, характерные для этой научной об\-ласти. Внимание 
уделяется особенностям корпусной методологии при контрастивном изучении 
пунктуации. В~качестве одного из современных методологических инструментов 
предлагаются надкорпусные базы данных (НБД), раз\-ра\-ба\-ты\-ва\-емые в~ФИЦ ИУ РАН.}

%\vspace*{-6pt}
  
  \KW{контрастивная пунктуация; перевод; корпусное переводоведение; кор\-пус\-но-ори\-ен\-ти\-ро\-ван\-ное 
  исследование; параллельный корпус; надкорпусная база данных; 
межъязыковая асим\-мет\-рия; методология}

%\vspace*{-6pt}

\DOI{10.14357/19922264230213}{VBOZAO} 
  
%\vspace*{-3pt}


\vskip 10pt plus 9pt minus 6pt

\thispagestyle{headings}

\begin{multicols}{2}

\label{st\stat}
    
    \section{Введение}
    
    \vspace*{-3pt}
    
  Важность и~необходимость исследований в~области контрастивной 
пунктуации в~научной литературе отмечалась неоднократно (см., 
например,~[1--7]). Обычно эта необходимость выводится из нужд 
переводческой практики, которая предполагает при обработке письменного 
текста обязательную речемыслительную программу, связанную с~исходным 
пунктуационным компонентом и~его переносом в~сис\-те\-му переводящего 
языка. Так, Ньюмарк в~своем <<Учебнике перевода>> пишет, что 
<<пунктуация может быть мощнейшим инструментом, но ее настолько легко 
упус\-тить из виду, что я~советую переводчикам: специально сравнивайте, где 
у~вас рас\-став\-ле\-ны знаки препинания, а~где они стоят 
в~оригинале>>~\cite[с.~58]{4-nu}. В~работе <<Переводчик в~текс\-те: 
о~чтении русской литературы  
по-анг\-лий\-ски>> значение пунктуации отмечает Мей, критикуя 
англоязычных переводчиков за недостаточное внимание к~межъязыковой 
пунктуационной асим\-мет\-рии~--- за <<игнорирование отличительных 
особенностей, присущих знакам препинания>>~\cite[с.~121]{2-nu}. 
О~пунктуации в~переводе говорит Юдейл, выделяя три аспекта:
%\begin{enumerate}[(1)]
%\item 
(1)~<<знаки препинания~--- важ\-ная часть перевода, но, концентрируясь на 
общем смыс\-ле переводимого, ее час\-то не замечают>>; 
%\item 
(2)~<<изменения 
в~пунктуации при переводе могут значительно по\-вли\-ять на вы\-ра\-зи\-тель\-ность 
текс\-та, его свя\-зан\-ность и~ритм>>; 
%\item 
(3)~<<час\-то возникает впечатление, что 
литературные переводчики наделили себя правом менять границы исходного 
предложения и~пунктуационные знаки, как им 
заблагорассудится>>~\cite[с.~121]{5-nu}.
%\end{enumerate}
 Гораздо реже 
в~специализированной литературе подчеркивается роль, которую 
исследования в~об\-ласти контрастивной пунктуации играют при обучении 
иностранным языкам, в~част\-ности при обуче\-нии иноязычной письменной 
речи~\cite{7-nu}.
  
  Признавая безусловную зна\-чи\-мость данного научного на\-прав\-ле\-ния и~его 
дальнейшего развития, необходимо предметно разрабатывать методологию 
исследования в~об\-ласти контрастивной пунктуации, которая учитывала бы 
новейшие достижения информатики, компьютерной лингвистики 
и~корпусного переводоведения. Пред\-став\-ля\-ет\-ся, что такая методология 
долж\-на основываться на использовании современных информационных 
корпусных инструментов, поз\-во\-ля\-ющих автоматизированным образом 
обрабатывать пред\-ста\-ви\-тель\-ные массивы текс\-то\-вых данных, 
и,~следовательно, носить кор\-пус\-но-ори\-ен\-ти\-ро\-ван\-ный характер 
(о~корпусных данных при контрастивном изуче\-нии пунктуации  
см.~\cite{6-nu}).

%\vspace*{-6pt}
    
    \section{Методологические модели  
корпусно-ориентированного исследования контрастивной 
пунктуации}

\vspace*{-3pt}
  
  В мае 2019~г.\ в~Регенсбурге (Германия) про\-шла научная конференция 
под названием <<Punctuation Seen Internationally. System--Norm--Practice>> 
(<<Пунктуация в~мировом мас\-шта\-бе: 
 сис\-те\-ма--нор\-ма--прак\-ти\-ка>>)~--- первая конференция, пол\-ностью\linebreak 
по\-свя\-щен\-ная проб\-ле\-мам контрастивной пунктуации. Оргкомитет, собирая 
заявки на участие, справедливо отмечал, что до на\-сто\-яще\-го времени 
пунктуации едва ли уделялось внимание в~рамках \mbox{типологии}, контрастивной 
лингвистики, прагмалингвистики, а~так\-же в~исследованиях индивидуальной 
языковой манеры на фоне языкового стандарта. Сейчас появляются 
отдельные работы, где проводится сопоставительное изуче\-ние пунктуации, 
однако по-преж\-не\-му ощущается острая не\-об\-хо\-ди\-мость в~исследованиях по 
контрастивной пунктуации, которые бы учитывали типологические 
(сис\-тем\-ные), социолингвистические (нормативные) и~прагматические 
(речевые) ас\-пекты.
  
  Итогом конференции стала коллективная монография~\cite{8-nu}, 
со\-сто\-ящая из шестнадцати статей, которые пред\-став\-ля\-ют собой пио\-нер\-ские 
исследования, на\-прав\-лен\-ные на формирование целостной па\-ра\-диг\-мы 
контрастивного изучения пунктуации и~борьбу с~маргинализацией важ\-ной 
научной от\-расли. Все статьи услов\-но мож\-но разделить на~4~категории, 
первые две из которых имеют в~большей степени тео\-ре\-ти\-че\-ский характер и~связаны с~сис\-те\-мой и~нормой, а~вторые~--- более практической 
на\-прав\-лен\-ности~--- с~узусом и~освоением пунктуационных навыков. 
В~пред\-став\-лен\-ных работах доминируют два подхода к~исследованию 
конт\-растив\-ной пунк\-ту\-ации:
  \begin{enumerate}[(1)]
\item интралингвистический (контрастивный анализ знаков препинания 
и~кон\-ку\-ри\-ру\-ющих с~ними маркеров синтаксических отношений в~рамках 
одного языка)~\cite[с.~110]{9-nu};
  \item  интерлингвистический (контрастивный анализ знаков препинания 
  и~конкурирующих с~ними средств в~разных языках, конт\-растив\-ная пунктуация 
рас\-смат\-ри\-ва\-ет\-ся в~том чис\-ле как часть методики обуче\-ния неродному языку, 
например при интеграции трудовых мигрантов в~иноязычную 
среду)~\cite[с.~57--73]{10-nu}.
  \end{enumerate}
  
  Интралингвистический подход час\-то носит смешанный характер: если 
речь идет об эволюции пунктуационной сис\-те\-мы отдельно взятого языка на 
фоне развития аналогичных сис\-тем других языков, контрастивный анализ 
со\-про\-вож\-да\-ет\-ся 
 ис\-то\-ри\-ко-эти\-мо\-ло\-ги\-че\-ским~\cite[с.~187--206]{11-nu}. В~рамках 
этого подхода в~указанной монографии имеются психолингвистические 
исследования с~нетривиальным корпусным материалом. Так, 
в~статье~\cite[с.~163--186]{12-nu} корпусные данные привлекаются для 
контрастивного анализа пунктуационных предпочтений двух групп 
ис\-пы\-ту\-емых. Автор использует корпус \mbox{CoPaDocs} (Corpus of Patient 
Documents), основу которого со\-ста\-ви\-ли письма и~другие личные документы 
бывших пациентов психиатрических учреж\-де\-ний Германии на рубеже  
XIX--XX~вв. Корпус поз\-во\-ля\-ет установить, зависит ли языковое оформление 
пись\-ма от лич\-ности адресата~--- происходит ли переключение ре\-гист\-ров 
сознательно. Данный корпус создан с~целью разработки интегративной 
методики анализа языковой ва\-риа\-тивн\-ости, в~том чис\-ле и~в~об\-ласти 
пунктуации. \mbox{Изучив} специфику расстановки~12~знаков препинания, 
Эбер-Хам\-мерль приходит к~выводу, что пациенты, чей род де\-я\-тель\-ности 
прежде не был связан с~письменной сферой, использовали больше 
пунктуационных маркеров (но с~меньшей ва\-риа\-тив\-ностью), чем 
представители второй опытной группы~--- канцелярские служащие. 
В~лич\-ной переписке участники обеих групп к~знакам препинания прибегали 
гораздо реже, чем в~документах, адресованных официальным лицам.
  
 В статье~\cite[с.~57--73]{10-nu} представлено контрастивное исследование, выполненное в~интерлигвистическом 
ключе. Со\-по\-став\-ле\-ние 
пунктуации в~италь\-ян\-ском и~немецком языках здесь проводится на основе 
комплексной методологии, вклю\-ча\-ющей приемы дескриптивного, 
просодического, синтаксического и~ком\-му\-ни\-ка\-тив\-но-текс\-то\-во\-го 
анализа. Примеры приводятся из различных источников, причем 
к~корпусным данным в~статье отсылают не напрямую, а~опосредованно~--- 
через более раннюю работу~\cite{13-nu}. По мнению авторов, 
пунктуирование в~этих языках организовано по-раз\-но\-му, что объясняется 
резкими различиями в~пунктуационном узусе: если в~итальянском знаки 
препинания коммуникативно на\-гру\-же\-ны, то в~немецком они подчинены 
фор\-маль\-но-син\-так\-си\-че\-ско\-му принципу. Иначе говоря, итальянская 
пунктуация выполняет не формальную функцию, а~сигнализиру-\linebreak ет о~тон\-ких 
смыс\-ло\-вых нюансах, которых нельзя\linebreak достичь другими языковыми 
средствами (аргументативный конфликт, полифонические эффекты, 
метатекстовые комментарии). В~этом же духе\linebreak выполнена и~другая 
интерлингвистическая работа~\cite{14-nu}, по\-свя\-щен\-ная контрастивному 
исследованию многоточия и~тире в~италь\-ян\-ском и~анг\-лий\-ском языках 
и~продуктивно ис\-поль\-зу\-ющая \mbox{корпусный} метод сбора и~обработки 
эмпирических данных.
     
     Объединенные в~коллективную монографию рабо\-ты позволяют 
вывести обобщенную ме\-то\-до\-ло\-гическую схему контрастивного изуче\-ния 
пунктуации. Она имеет трехфазную структуру. Первая\linebreak фаза включает 
тео\-ре\-ти\-че\-ское описание пунктуации в~изуча\-емом языке с~привлечением 
исторических и~современных нормативных грамматик и~справочников. 
Вторая фаза на\-прав\-ле\-на на описание трансформаций в~других языках, 
оказавших существенное влияние на статус и~мес\-то пунктуации в~сис\-те\-ме 
конкретного языка. Обе фазы нацелены на создание такого 
исследовательского поля, которое поз\-во\-лит выявить значение пунктуации 
для языковой культуры. Это, в~свою очередь, долж\-но стать задачей треть\-ей 
фазы. Вторая и~\mbox{третья} фазы предполагают межъязыковое сравнение как 
функционального диапазона отдельно взятых знаков препинания, так 
и~пунктуационного репертуара в~целом. На этих стадиях применяется 
корпусный метод. Контрастивный анализ в~за\-ви\-си\-мости от по\-став\-лен\-ных 
целей и~задач наряду со знаками препинания может охватывать 
и~кон\-ку\-ри\-ру\-ющие с~ними языковые средства. На\-прав\-ле\-ние контрастивного 
исследования пунктуации может быть и~синхронным, и~диахроническим.

\vspace*{-6pt}
    
    \section{Методологические особенности  
корпусно-ориентированного исследования в~области 
контрастивной пунктуации}

\vspace*{-3pt}
  
  Особенности методологии при корпусном контрастивном изуче\-нии 
пунктуации, как, впрочем, и~при любом 
 кор\-пус\-но-ори\-ен\-ти\-ро\-ван\-ном исследовании, связаны прежде всего 
со стремлением получить непротиворечивые, валидные и~на\-деж\-ные данные. 
Электронный корпус, будучи методологически новаторским инструментом 
для получения научного знания, поз\-во\-ля\-ет, с~одной стороны, автоматическим 
образом обрабатывать большие массивы данных и~тем самым серьезно 
сокращает временные издержки на поиск эмпирического материала. 
С~другой стороны, электронные корпусные ресурсы имеют свои 
особенности, и~без их над\-ле\-жа\-ще\-го учета пользователь рискует получить 
искаженные результаты.
  
  Например, в~указанной выше работе~\cite[с.~291]{14-nu} авторы, описывая 
методологию своего исследования, отмечают, что итальянские примеры 
заимствованы из корпуса, хранящегося в~Базельском университете 
и~со\-сто\-яще\-го из двух частей~--- 33~современных  
ро\-ма\-на-бест\-сел\-ле\-ра (1~млн словоупотреблений) 
и~нехудожественных текс\-та разной на\-прав\-лен\-ности (1~млн 40~тыс.\ 
словоупотреблений), в~то время как англоязычные примеры извлечены из 
подкорпуса <<Книги и~периодические издания>> Британского 
национального корпуса (80~млн словоупотреблений). Итальянский материал, 
по словам авторов, был проанализирован весь, а~для английского из-за 
гораздо большего объема ограничились анализом случайной выборки, объем 
которой со\-по\-ста\-вим с~выборкой из итальянского корпуса. Очевидным 
образом ва\-лид\-ность выводов по результатам анализа англоязычного 
материала здесь может оказаться под вопросом в~силу методологически 
неоднородных установок применительно к~процедуре обработки данных, 
полученных по двум языкам. Примечательно к~тому же, что базельский 
корпус, в~отличие от британского, за\-крыт для общественного пользования.
  
  О подобных ограничениях рассуждает На\-двор\-ни\-ко\-ва в~своей работе, 
по\-свя\-щен\-ной корпусной методологии контрастивного изучения 
пунктуации~\cite{15-nu}, где анализируется час\-тот\-ность упо\-треб\-ле\-ния шести 
знаков препинания (запятой, точки, двоеточия, точ\-ки с~запятой, 
вопросительного и~восклицательного знака) в~английском, французском 
и~чешском языках. Для сбора данных используются со\-по\-ста\-ви\-мые 
веб-кор\-пу\-сы, моноязычные общие (референтные) и~параллельные корпусы. Цель 
автора~--- определить, какой из трех типов корпусных ресурсов наиболее 
подходит для исследований в~об\-ласти контрастивной пунктуации.
  
  Полученные данные показывают, что при изучении пунктуации показатели 
час\-тот\-ности проявляют высокую чув\-ст\-ви\-тель\-ность к~типу текс\-та; 
следовательно, веб-кор\-пу\-сы, которые, как правило, отличают стихийное 
наполнение, не\-упо\-ря\-до\-чен\-ность и~низ\-кая степень струк\-ту\-ри\-ро\-ван\-ности, не 
могут служить источником до\-сто\-вер\-ной информации об упо\-треб\-ле\-нии 
знаков препинания в~том или ином языке. Моноязычный общий корпус, 
наоборот, содержит специальную раз\-мет\-ку (морфологическую, 
синтаксическую и~т.\,д.)\ и~поз\-во\-ля\-ет гиб\-ко настраивать поиск (в~том чис\-ле 
выбирать соответствующий тип текс\-та) в~за\-ви\-си\-мости от конкретных 
исследовательских задач. Такие корпусы располагают большими массивами 
данных, поскольку призваны пред\-ста\-вить язык во всей его пол\-но\-те 
и~многообразии, что, казалось бы, обеспечивает на\-деж\-ность и~ва\-лид\-ность 
полученных результатов. Меж\-ду тем этот тип корпусов имеет существенный 
недостаток~--- ограниченную межъязыковую со\-по\-ста\-ви\-мость. Как правило, 
моноязычные общие корпусы разных языков разительно отличаются по 
объему данных и~их со\-ста\-ву и~поэтому не подходят в~качестве основного 
инструмента контрастивного исследования, а~могут служить лишь 
референтным (проверочным) источником для дополнительной верификации 
ре\-зуль\-ти\-ру\-ющих данных. Кроме того, со\-по\-ста\-ви\-тель\-ный анализ 
относительной час\-тот\-ности упо\-треб\-ле\-ния знаков препинания в~разных 
языках на основе данных, извлеченных из корпусов этого типа, так\-же имеет 
свои ограничения. Он не применим для изучения пунк\-ту\-а\-ции в~языках 
разного строя, которым для кодирования информации требуется 
количественно больше (аналитические языки типа французского) или 
меньше слов (синтетические языки типа русского). Таким образом, лучше 
всего для контрастивного изуче\-ния пунк\-ту\-а\-ции подходят параллельные 
корпусы, которые, не\-смот\-ря на свой сравнительно небольшой объем, 
пред\-став\-ля\-ют существенно больше воз\-мож\-но\-стей для качественного анализа 
упо\-треб\-ле\-ния знаков препинания и~непосредственного со\-по\-став\-ле\-ния их 
абсолютной час\-тот\-ности в~параллельных текс\-тах~--- оригинале и~переводе. 
Однако и~этот тип информационного ресурса не может служить 
универсальным исследовательским инструментом. При его использовании 
необходимо учитывать, что пунктуационные рас\-хож\-де\-ния в~исходном 
и~переводном текс\-те могут быть не результатом сис\-тем\-ных дифференциаций, 
а~возникнуть под влиянием переводческих предпочтений. Следовательно, 
чтобы избежать искажения ре\-зуль\-ти\-ру\-ющих данных, надо следовать 
некоторым методологическим принципам: %\\[-13pt] 
\begin{enumerate}[(1)]
\item данные собираются в~обоих 
переводных на\-прав\-ле\-ни\-ях; %\\[-13pt] 
\item выявленные тенденции проходят 
обязательную проверку с~по\-мощью референтного моноязычного корпуса; %\\[-13pt] 
\item контрастивное изуче\-ние пунктуации с~применением параллельных 
корпусов требует сис\-тем\-но\-го подхода в~том смыс\-ле, что в~функциональном 
диапазоне разных знаков препинания могут быть общие зоны, ука\-зы\-ва\-ющие 
на их потенциальную внут\-ри\-язы\-ко\-вую и~межъ\-язы\-ко\-вую конкуренцию. %\\[-13pt]
\end{enumerate}
    
 \vspace*{-12pt}
 
    \section{Заключение}
    
    \vspace*{-3pt}
    
  В статье представлена обобщенная методологическая схема 
  кор\-пус\-но-ори\-ен\-ти\-ро\-ван\-но\-го 
  исследования в~об\-ласти контрастивной пунктуации~--- 
от\-расли научного знания, интенсивно \mbox{раз\-ви\-ва\-ющей\-ся} и~при\-вле\-ка\-ющей 
внимание специалистов самого широкого профиля. Несмотря на то что 
появляются работы, где описываются сопоставительные исследования 
пунктуации на примере одного произведения или литературного наследия 
отдельно взятого писателя (см., например,~\cite{16-nu,17-nu}), очевидно, что 
для ка\-ких-ли\-бо существенных, круп\-но\-мас\-штаб\-ных обобщений относительно 
межъязыковой пунктуационной асимметрии и~специфики функционирования 
знаков препинания в~разных языках требуется привлечение корпусного 
материала.
  
  Дальнейшее изучение контрастивной пунктуации видится в~нескольких 
направлениях. Необходимо качественное углубление со\-по\-ста\-ви\-тель\-но\-го 
анализа, чтобы его тонкая нюансировка \mbox{поз\-во\-ли\-ла} установить, в~какой мере 
совпадает и~разнится функциональный диапазон того или иного знака 
препинания в~кон\-так\-ти\-ру\-ющих языках в~за\-ви\-си\-мости от жанровой 
при\-над\-леж\-ности текс\-та. Этот анализ целесообразно проводить комплексно, 
охватывая всю со\-во\-куп\-ность синтаксических изменений, которые влекут за 
собой отказ от исходного пунктуирования при переводе с~одного языка на 
другой. Такая ком\-плекс\-ность поможет выявить и~с~большей пол\-но\-той 
описать су\-щест\-ву\-ющие межъ\-язы\-ко\-вые структурные различия, что 
необходимо и~для переводческой практики, и~для обуче\-ния иностранным 
языкам. Требует дальнейшего уточ\-не\-ния вопрос, как на пунктуационные 
преференции переводчика влияет род\-ная языковая культура, 
пунктуационные уста\-нов\-ки которой могут меняться со временем. По мере 
наращивания опыта и~мастерства могут меняться пунктуационные 
предпочтения и~самого переводчика, и~это так\-же пред\-став\-ля\-ет определенный 
научный интерес.
  
  В заключение следует отметить, что одним из современных 
информационных инструментов корпусного исследования в~об\-ласти 
контрастивной пунктуации могут быть НБД, 
раз\-ра\-ба\-ты\-ва\-емые в~отделе~54 Федерального исследовательского цент\-ра 
<<Информатика и~управ\-ле\-ние>> Российской академии наук (о~возможностях 
НБД см.~\cite{6-nu}). В~данный момент этот методологический инструмент 
проходит апро\-ба\-цию в~контрастивном исследовании двоеточия и~многоточия в~трех языках~--- русском, французском и~немецком.

\vspace*{-9pt}
  
{\small\frenchspacing
 {\baselineskip=11.5pt
 %\addcontentsline{toc}{section}{References}
 \begin{thebibliography}{99}
 
 \vspace*{-3pt}
 
 \bibitem{4-nu} %1
\Au{Newmark P.} A~textbook of translation.~--- New York, London, Toronto, Sydney, Tokyo: Prentice 
Hall, 1988. 402~p.
 

\bibitem{2-nu} %2
\Au{May R.} The translator in the text: On reading Russian literature in English.~--- Evanston, IL, USA: 
Northwestern University Press, 1994. 209 p.
\bibitem{3-nu}
\Au{Munday J.} Systems in translation: A~systemic model for descriptive translation studies~// 
Crosscultural transgressions: Research models in translation studies II~--- historical and 
ideological issues~/ Ed. T.~Hermans.~---  Manchester, U.K.: St.\ Jerome, 2002. P.~76--92.
\bibitem{1-nu} %4
\Au{Baker M.} In other words.~--- 2nd ed.~--- London, New York: Routledge, 2011. 352~p.

\bibitem{7-nu} %5
\Au{Сигал К.\,Я.} Контрастивная пунктуация в~начале XXI века~// Язык. Текст. Дискурс: 
Научный альманах Ставропольского отделения РАЛК.~--- Ставрополь: СКФУ, 
2019.  Вып.~17. С.~69--78.
\bibitem{5-nu} %6
\Au{Youdale R.} Using computers in the translation of literary style: Challenges and 
opportunities.~--- London, New York: Routledge, 2020. 242~p.
\bibitem{6-nu} %7
\Au{Нуриев В.\,А., Кружков~М.\,Г.} Корпусные данные при контрастивном изуче\-нии 
пунктуации~// Сис\-те\-мы и~средства информатики, 2023. Т.~33. №\,1. С.~14--23. doi: 10.14357/08696527230102.

\bibitem{8-nu}
Vergleichende Interpunktion~--- comparative punctuation~/ Eds. P.~R$\ddot{\mbox{o}}$ssler, P.~Besl, A.~Saller.~--- 
Berlin, Boston: De Gruyter, 2021. 454~p.
\bibitem{9-nu}
\Au{Rinas K.} Vom genormten Satzbau zur genormten Interpunktion. Zur Funktion der 
Zeichensetzung in $\ddot{\mbox{a}}$lterer und neuerer Zeit~// Vergleichende Interpunktion~--- comparative 
punctuation~/ Eds. P.~R$\ddot{\mbox{o}}$ssler, P.~Besl, A.~Saller.~---
 Berlin, Boston: De Gruyter, 2021. P.~109--136. doi: 10.1515/9783110756319-006.
\bibitem{10-nu}
\Au{Ferrari~A., Stojmenova Weber R.} Das Komma in kontrastiver Perspektive Italienisch-Deutsch~// Vergleichende Interpunktion~--- 
comparative punctuation / Eds. P.~R$\ddot{\mbox{o}}$ssler, P.~Besl, 
A.~Saller.~--- Berlin, Boston: De Gruyter, 2021. P.~57--73. doi: 10.1515/9783110756319-003.

\columnbreak

\bibitem{11-nu}
\Au{Besch W.} Zur Entwicklung der deutschen Interpunktion seit dem sp$\ddot{\mbox{a}}$ten Mittelalter~// 
Interpretation und Edition deutscher Texte des Mittelalters. Festschrift f$\ddot{\mbox{u}}$r John Asher zum 60. 
Geburtstag~/ Eds. K.~Smits, W.~Besch, V.~Lange.~--- Berlin: Erich Schmidt, 1981. P.~187--206.
\bibitem{12-nu}
\Au{Eber-Hammerl F.} Interpunktion in historischen Patientenbriefen // Vergleichende
Interpunktion~--- comparative punctuation~/ Eds. P.~R$\ddot{\mbox{o}}$ssler, 
P.~Besl, A.~Saller.~--- Berlin, Boston: De Gruyter, 2021. P.~163--186.
\bibitem{13-nu}
\Au{Ferrari A.} Leggere la virgola. Una prima ricognizione~// Chimera Romance Corpora 
Linguistic Studies, 2017. Vol.~4. Iss.~2. P.~145--162. doi: 
10.15366/chimera2017. 4.2.001.
\bibitem{14-nu}
\Au{Pecorari F., Longo~F.} The ellipsis and the dash in Italian and English: A~contrastive 
perspective~// Vergleichende Interpunktion~--- comparative punctuation~/ Eds.
 P.~R$\ddot{\mbox{o}}$ssler, P.~Besl, A.~Saller.~--- Berlin, Boston: De Gruyter, 2021. P.~289--314.
 doi: 10.1515/9783110756319-013.
\bibitem{15-nu}
\Au{N$\acute{\mbox{a}}$dvorn$\acute{{\iota}}$kov$\acute{\mbox{a}}$~O.}
The use of English, Czech and French punctuation marks in reference, 
parallel and comparable web corpora: A~question of methodology~// 
Linguist. Prag.,  2020. Vol.~30. Iss.~2. P.~30--50. doi: 
10.14712/ 18059635.2020.1.2.
\bibitem{16-nu}
\Au{Сигал К.\,Я.} Пунктуация как средство создания эмоционального под\-текс\-та (на 
материале рассказа М.\,А.~Шолохова <<Судьба человека>> и~его переводов на английский 
язык)~// Известия РАН. Серия литературы и~языка, 2014. Т.~73. №\,6. С.~38--50.
\bibitem{17-nu}
\Au{Богданов К.\,А.} Пунктуация как мотив: многоточие и~тире~// НЛО, 2022. №\,2(174). С.~241--253.
doi: 0.53953/ 08696365\_2022\_174\_2\_241.

\end{thebibliography}

 }
 }

\end{multicols}

\vspace*{-8pt}

\hfill{\small\textit{Поступила в~редакцию 15.04.23}}

\vspace*{6pt}

%\pagebreak

%\newpage

%\vspace*{-28pt}

\hrule

\vspace*{2pt}

\hrule

\vspace*{-2pt}

\def\tit{METHODOLOGY OF~THE~CORPUS-BASED STUDIES\\ 
IN~THE~FIELD OF~CONTRASTIVE PUNCTUATION}


\def\titkol{Methodology of~the~corpus-based studies 
in~the~field of~contrastive punctuation}


\def\aut{V.\,A.~Nuriev$^1$ and~V.\,I.~Karpov$^{1,2}$}

\def\autkol{V.\,A.~Nuriev and~V.\,I.~Karpov}

\titel{\tit}{\aut}{\autkol}{\titkol}

\vspace*{-14pt}


\noindent
      $^1$Federal Research Center ``Computer Science and Control'' of the Russian 
Academy of Sciences, 44-2~Vavilov\linebreak
$\hphantom{^1}$Str., Moscow 119333, Russian Federation
      
      \noindent
      $^2$Institute of Linguistics of the Russian Academy of Sciences, 1~bld.~1 
Bolshoy Kislovsky Lane, Moscow 125009,\linebreak
$\hphantom{^1}$Russian Federation

\def\leftfootline{\small{\textbf{\thepage}
\hfill INFORMATIKA I EE PRIMENENIYA~--- INFORMATICS AND
APPLICATIONS\ \ \ 2023\ \ \ volume~17\ \ \ issue\ 2}
}%
 \def\rightfootline{\small{INFORMATIKA I EE PRIMENENIYA~---
INFORMATICS AND APPLICATIONS\ \ \ 2023\ \ \ volume~17\ \ \ issue\ 2
\hfill \textbf{\thepage}}}

\vspace*{3pt}
      
      
    
    \Abste{The paper refines the methodological approach to the contrastive 
studies of punctuation. Given the recent achievements of information science, 
computer linguistics, and translation theory, such studies are most likely to be 
corpus-based. The paper presents a~methodological model of research into 
interlingual punctuation asymmetry, the aim of which is to shed light on the 
functional scope of the same punctuation marks in different languages. It shows 
what methodological trends are characteristic of this research area. The focus is 
also on the specificities of corpus methodology in the contrastive study of 
punctuation. It is argued that one of the methodological tools, tailored specifically 
to the needs of contrastive punctuation research, may be the supracorpora 
databases developed at the Federal Research Center ``Computer Science and 
Control'' of the Russian Academy of Sciences.}
    
    \KWE{contrastive punctuation; translation; corpus-based translation studies; 
corpus-based studies; parallel corpus; supracorpora database; asymmetry between 
languages; methodology}
    
    
    
\DOI{10.14357/19922264230213}{VBOZAO}

%\vspace*{-18pt}

\Ack
    \noindent
    The research was carried out using the infrastructure of the Shared Research 
Facilities ``High Performance Computing and Big Data'' (CKP ``Informatics'') of 
FRC CSC RAS (Moscow). The research was supported by the Russian Science Foundation (project  
No.\,23-28-00548).
 
%\vspace*{4pt}

  \begin{multicols}{2}

\renewcommand{\bibname}{\protect\rmfamily References}
%\renewcommand{\bibname}{\large\protect\rm References}

{\small\frenchspacing
 {%\baselineskip=10.8pt
 \addcontentsline{toc}{section}{References}
 \begin{thebibliography}{99}
 
 \bibitem{4-nu-1} %1
\Aue{Newmark, P.} 1988. \textit{A~textbook of translation}. New York, London, Toronto, Sydney, Tokyo: 
Prentice Hall. 402~p.   

\bibitem{2-nu-1}
\Aue{May, R.} 1994. \textit{The translator in the text: On reading Russian 
literature in English}. Evanston, IL: Northwestern University Press. 209~p.
\bibitem{3-nu-1}
\Aue{Munday, J.} 2002. Systems in translation: A~systemic model for 
descriptive translation studies. \textit{Crosscultural transgressions: Research models in 
translation studies II~--- historical and ideological issues}. Ed. T.~Hermans. 
Manchester, U.K.: St.\ Jerome. 76--92.

\bibitem{1-nu-1} %4
\Aue{Baker, M.} 2011. \textit{In other words}. 2nd ed. London, New York: 
Routledge. 352~p.

\bibitem{7-nu-1} %5
\Aue{Seagal, K.\,Ya.} 2019. Kont\-ras\-tiv\-naya punk\-tu\-a\-tsiya v~na\-cha\-le XXI~veka 
[Contrastive punctuation at the beginning of the XXI century]. \textit{Yazyk. Tekst. 
Diskurs: Nauchnyy al'manakh Stavropol'skogo otdeleniya RALK} [Language. Text. 
Discourse: Scientific almanac of Stavropol Branch of the Russian Cognitive 
Linguists Association].  Stavropol': SKFU. 17:69--78.

\bibitem{5-nu-1} %6
\Aue{Youdale, R.} 2020. \textit{Using computers in the translation of literary style: 
Challenges and opportunities}. London, New York: Routledge. 242~p.
\bibitem{6-nu-1} %7
\Aue{Nuriev, V.\,A., and M.\,G.~Kruzhkov.} 2023. Kor\-pus\-nye dan\-nye pri 
kont\-ras\-tiv\-nom izu\-che\-nii punk\-tu\-a\-tsii [The parallel corpora perspective on studying 
contrastive punctuation]. \textit{Sistemy i~Sredstva Informatiki~--- Systems and Means of 
Informatics} 33(1):14--23. doi: 10.14357/08696527230102.

  \bibitem{8-nu-1}
R$\ddot{\mbox{o}}$ssler, P., P.~Besl, and A.~Saller, eds. 2021. \textit{Vergleichende 
Interpunktion~--- comparative punctuation}. Berlin, Boston: De Gruyter. 454~p.
\bibitem{9-nu-1}
\Aue{Rinas, K.} 2021. Vom genormten satzbau zur genormten interpunktion. 
Zur funktion der zeichensetzung in $\ddot{\mbox{a}}$lterer und neuerer zeit. \textit{Vergleichende 
Interpunktion~--- comparative punctuation}. Eds.\ P.~R$\ddot{\mbox{o}}$ssler, 
P.~Besl, and A.~Saller. 
Berlin, Boston: De Gruyter. 109--136. doi: 10.1515/ 9783110756319-006.
\bibitem{10-nu-1}
\Aue{Ferrari, A., and R.~Stojmenova.} 2021. Weber das komma in kontrastiver 
perspektive Italienisch-Deutsch. \textit{Vergleichende Interpunktion~--- comparative 
punctuation}. Eds. P.~R$\ddot{\mbox{o}}$ssler, P.~Besl, and A.~Saller. Berlin, Boston: De Gruyter.  
57--73. doi: 10.1515/9783110756319-003.
 \bibitem{11-nu-1}
\Aue{Besch, W.} 1981. Zur entwicklung der deutschen interpunktion seit 
dem sp$\ddot{\mbox{a}}$ten mittelalter. \textit{Interpretation und Edition deutscher Texte des Mittelalters. 
Festschrift f$\ddot{\mbox{u}}$r John Asher zum 60. Geburtstag}. Eds. K.~Smits, W.~Besch, and 
V.~Lange. Berlin: Erich Schmidt. 187--206.
 \bibitem{12-nu-1}
\Aue{Eber-Hammerl, F.} 2021. Interpunktion in historischen 
Patientenbriefen. \textit{Vergleichende Interpunktion~--- comparative punctuation}. Eds. 
P.~R$\ddot{\mbox{o}}$ssler, P.~Besl, and A.~Saller. Berlin, Boston: De Gruyter. 163--186.
\bibitem{13-nu-1}
\Aue{Ferrari, A.} 2017. Leggere la virgola. Una prima ricognizione. 
\textit{Chimera Romance Corpora Linguistic Studies} 4(2):145--162. doi: 
10.15366/chimera2017.4.2.001.
\bibitem{14-nu-1}
\Aue{Pecorari, F., and F.~Longo.} 2021. The ellipsis and the dash in Italian 
and English: A~contrastive perspective. \textit{Vergleichende Interpunktion~--- 
comparative punctuation}. Eds. P.~R$\ddot{\mbox{o}}$ssler, P.~Besl, and A.~Saller. Berlin, Boston: 
De Gruyter. 289--314. doi: 10.1515/9783110756319-013.
\bibitem{15-nu-1}
\Aue{N$\acute{\mbox{a}}$dvorn$\!\acute{\mbox{\ptb{\i}}}$kov$\acute{\mbox{a}}$,~O.} 2020. The use of English, Czech and French 
punctuation marks in reference, parallel and comparable web corpora: A~question 
of methodology. \textit{Linguist. Prag.} 30(2):30--50. doi: 
10.14712/18059635.2020.1.2.
\bibitem{16-nu-1}
\Aue{Seagal, K.\,Ya.} 2014. Punk\-tu\-a\-tsiya kak sred\-st\-vo so\-zda\-niya 
emo\-tsi\-o\-nal'\-no\-go pod\-teks\-ta (na ma\-te\-ri\-ale ras\-ska\-za M.\,A.~Sho\-lo\-kho\-va ``Sud'\-ba 
che\-lo\-ve\-ka'' i~ego pe\-re\-vo\-dov na ang\-liy\-skiy yazyk) [Punctuation as a means of 
revealing the emotional subtext (the case of Mikhail Sholokhov's short story ``The 
Fate of a~Man'' and its translations into English)]. \textit{Izvestiya RAN. Seriya literatury i~yazyka}
 [The Bulletin of the Russian Academy of Sciences: Studies in Literature 
and Language]. 73(6):38--50.
\bibitem{17-nu-1}
\Aue{Bogdanov, K.\,A.} 2022. Punk\-tu\-a\-tsiya kak mo\-tiv: mno\-go\-to\-chie i~ti\-re 
[Punctuation as a~motive: The ellipsis and the dash]. \textit{NLO} [New Literary Observer] 
2(174):241--253. doi: 0.53953/08696365\_2022\_174\_2\_241.
\end{thebibliography}

 }
 }

\end{multicols}

\vspace*{-6pt}

\hfill{\small\textit{Received April 15, 2023}} 

\vspace*{-18pt}
    
    
    \Contr
    
    
    \vspace*{-3pt}
    
    \noindent
    \textbf{Nuriev Vitaly A.} (b.\ 1980)~--- Doctor of Science in philology, leading 
scientist, Institute of Informatics Problems, Federal Research Center ``Computer 
Science and Control'' of the Russian Academy of Sciences, 44-2~Vavilov Str., 
Moscow 119333, Russian Federation; \mbox{nurieff.v@gmail.com}
    
    \vspace*{3pt}
    
    \noindent
    \textbf{Karpov Vladimir I.} (b.\ 1971)~--- Candidate of Science (PhD) in 
philology, leading scientist, Institute of Linguistics of the Russian Academy of 
Sciences, 1~bld.~1 Bolshoy Kislovsky lane, Moscow 125009, Russian Federation; 
scientist, Institute of Informatics Problems, Federal Research Center ``Computer 
Science and Control'' of the Russian Academy of Sciences, 44-2~Vavilov Str., 
Moscow 119333, Russian Federation; \mbox{wi.karpow@gmail.com}
     
      
\label{end\stat}

\renewcommand{\bibname}{\protect\rm Литература}      %9
%\newcommand{\A}{{\mathbf A}}
%\newcommand{\B}{{\mathbf B}}
%\newcommand{\la}{{\lambda}}
%\newcommand{\be}{\begin{equation}}
%\newcommand{\ee}{\end{equation}}
%\newcommand{\ber}{\begin{eqnarray}}
%\newcommand{\eer}{\end{eqnarray}}

%\newcommand{\nin}{\noindent}
%\newcommand{\non}{\nonumber}
%\newcommand{\half}{\frac{1}{2}}
%\newcommand{\quarter}{\frac{1}{4}}

\def\stat{zeifman}

\def\tit{ОБ ОДНОМ КЛАССЕ МАРКОВСКИХ СИСТЕМ ОБСЛУЖИВАНИЯ$^*$}

\def\titkol{Об одном классе марковских систем обслуживания}

\def\autkol{Я.\,А.~Сатин, А.\,И.~Зейфман, А.\,В.~Коротышева, С.\,Я.~Шоргин}
\def\aut{Я.\,А.~Сатин$^1$, А.\,И.~Зейфман$^2$, А.\,В.~Коротышева$^3$, С.\,Я.~Шоргин$^4$}

\titel{\tit}{\aut}{\autkol}{\titkol}

{\renewcommand{\thefootnote}{\fnsymbol{footnote}}\footnotetext[1]
{Исследование поддержано РФФИ, гранты 11-07-00112-а и 11-01-12026-офи-м.}}


\renewcommand{\thefootnote}{\arabic{footnote}}
\footnotetext[1]{Вологодский государственный педагогический
университет, yacovi@mail.ru}
\footnotetext[2]{Вологодский государственный педагогический университет;  
Институт проблем информатики Российской академии наук; 
Институт социально-экономического развития территорий Российской академии наук,  a\_zeifman@mail.ru}
\footnotetext[3]{Вологодский государственный педагогический
университет,  a\_korotysheva@mail.ru}
\footnotetext[4]{Институт проблем информатики Российской академии наук, SShorgin@ipiran.ru}


\Abst{Рассматриваются модели обслуживания, описываемые конечными марковскими 
цепями с непрерывным временем. При этом предполагается,  что интенсивности 
поступления и обслуживания требований не зависят от числа требований в сис\-те\-ме. 
Получены оценки скорости сходимости и устойчивости различных характеристик таких сис\-тем.}

\KW{нестационарные марковские системы
обслуживания; скорость сходимости; устойчивость; оценки}

 \vskip 14pt plus 9pt minus 6pt

      \thispagestyle{headings}

      \begin{multicols}{2}
      
            \label{st\stat}

\section{Введение}

Классы систем массового обслуживания, описываемых процессами
рождения и гибели (стационарными и нестационарными, с катастрофами)
изучались начиная с 1970-х~гг.\ многими авторами
(см., например,~[1--6]). С~помощью методов,
разработанных одним из авторов настоящей \mbox{статьи}\linebreak (подробное изложение
этих методов приведено в~[7--9]), для таких сис\-тем
удалось получить точные оценки скорости сходимости и устойчивости.

Оказывается, этот же подход можно применить и к существенно более 
общему классу систем обслуживания.

Рассмотрим систему массового обслуживания, число требований в которой 
описывается нестационарной марковской цепью с непрерывным временем и 
конечным пространством состояний, причем требования могут поступать и 
обслуживаться группами.

Пусть $X=X(t)$, $t\geq 0$,~--- число требований в системе обслуживания ($0 \hm\le X(t) \hm\le r$).

Обозначим через 
\begin{gather*}
p_{ij}(s,t)=\mathrm{Pr}\left\{ X(t)=j\left| X(s)=i\right.
\right\}\,,\\
i,j \ge 0\,,\ 0\leq s\leq t\,,
\end{gather*}
переходные вероятности
процесса $X\hm=X(t)$, а через  $p_i(t)\hm=\mathrm{Pr}\left\{ X(t) \hm=i \right\}$~---
его вероятности состояний.

Будем предполагать, что интенсивности поступления и обслуживания $k$ требований в 
момент~$t$ в сис\-те\-ме об\-слу\-жи\-ва\-ния ($\lambda_{k}(t)$ и  $\mu_{k}(t)$ соответственно)  
не зависят от числа требований, находящихся в системе в момент~$t$, являются локально 
интегрируемыми на $[0,\infty)$ функциями времени~$t$ и, кроме того, 
$\lambda_{k+1}(t) \hm\le \lambda_{k}(t)$ и  $\mu_{k+1}(t) \hm\le \mu_{k}(t)$ при всех~$k$ 
и почти при всех $t \hm\ge 0$.

Тогда для описания вероятностной динамики процесса получаем прямую систему Колмогорова в виде
\begin{equation} 
\fr{d\vp}{dt}=A(t)\vp(t)\,,
\label{ur_1}
\end{equation}
 где
 {\footnotesize
\begin{multline*}
A(t)={}\\
{}=
\begin{pmatrix}
a_{00}(t) & \mu_1(t)  & \mu_2(t)   & \mu_3(t)  & \mu_4(t) & \cdots & \mu_r(t) \\
\la_1(t)   & a_{11}(t)  & \mu_1(t)  & \mu_2(t)   & \mu_3(t)  & \cdots & \mu_{r-1}(t) \\
\la_2(t)  & \la_1(t)    & a_{22}(t)& \mu_1(t)  & \mu2(t)    &  \cdots & \mu_{r-2}(t) \\
\cdots&\cdots&\cdots&\cdots&\cdots&\cdots&\cdots \\
\la_r(t) & \la_{r-1}(t) & \la_{r-2}(t) & \cdots & \la_2(t)  & \la_1 (t)   &  a_{rr}(t)
\end{pmatrix}\,,
\end{multline*}}
причем  
$$
a_{ii}(t)=-\sum\limits_{k=1}^{i}\mu_k(t) - \sum\limits_{k=1}^{r-i} \la_{r-k}(t)\,.
$$

Далее будем обозначать через $\|\bullet\|$  $l_1$-нор\-му, т.\,е.\ 
$\|{\vx}\|\hm=\sum|x_i|$, а $\|B\| \hm= \max\limits_j \sum\limits_i |b_{ij}|$, 
если $B \hm= (b_{ij})_{i,j=0}^{r}$.
%
Тогда, в частности, имеем 
$$
\|A(t)\| \le 2\sum\limits_{k=1}^{r}(\la_{k}(t)+ \mu_k(t))
$$ 
при  всех $t \hm\ge 0$.

Через 
$$
E(t,k) = E\left\{X(t)\left|X(0)\hm=k\right.\right\}
$$ 
будем далее обозначать математическое ожидание процесса (среднее число требований) в момент~$t$ 
при условии, что в нулевой момент времени он находится в состоянии~$k$, 
а через $E_{\bf p}(t)$ обозначим математическое ожидание процесса в момент~$t$ 
при начальном распределении вероятностей состояний $\mathbf{p}(0) \hm= \mathbf{p}$.

\section{Оценки скорости сходимости}

Рассмотрим вспомогательную последовательность положительных чисел $\{d_i\}$, $i\hm=1, \dots,r$.

Положим
\begin{equation*}
d=\min\limits_{1 \le i \le r} d_i\,; \enskip 
G=\sum\limits_{i=1}^r d_i\,; \enskip W=\min\limits_k \fr{d_k}{k}\,.
%\label{2.01}
\end{equation*}

Рассмотрим величины
\begin{multline*}
\alpha_i(t)= -a_{ii}(t)+\la_{r-i+1}(t)-\sum\limits_{k=1}^{i-1}(\mu_{i-k}(t)-{}\\
{}-
\mu_i(t))\fr{d_k}{d_i}-\sum\limits_{k=1}^{r-i}(\la_k(t)-\la_{i+r-1}(t))\fr{d_{k+i}}{d_i}\,,
%\label{2.02}
\end{multline*}

\noindent
\begin{equation*}
\alpha(t)=\min\limits_{1 \le i \le r}\alpha_i(t)\,.
%\label{2.03}
\end{equation*}

\smallskip

\noindent
\textbf{Теорема~1.} \textit{Пусть существует последовательность положительных 
чисел  $\{d_j\}$ такая, что}
\begin{equation}
\int\limits_0^{\infty} \alpha(t)\, dt = + \infty\,.
\label{2.031}
\end{equation}
\textit{Тогда $X(t)$ слабо эргодичен, при
любых начальных условиях} $\mathbf{p}^*(s)$, $\mathbf{p}^{**}(s)$ 
\textit{и любых $s$, $t$, $0\le s\le t$, справедлива оценка
\begin{equation} 
\label{2.04}
\|\vp^*(t)-\vp^{**}(t)\| \le \fr{8G}{d}\,e^{-\int\limits_s^t {\alpha(u)\,du}}\,.
\end{equation}
Кроме того,  $X(t)$ имеет предельное среднее $\phi(t)$ и при любых~$k$ и $t \hm\ge 0$ справедливо неравенство}:
\begin{equation}
\label{2.05}
|E(t,k)-\phi(t)|\le \fr{4G}{W}\,e^{-\int\limits_0^t {\alpha(u)\,du}}\,.
\end{equation}


\smallskip


\noindent
Д\,о\,к\,а\,з\,а\,т\,е\,л\,ь\,с\,т\,в\,о\,.\

Пользуясь предложенным в предыдущих работах способом, 
выразим 
$$
p_0=1-\sum\limits_{1\le i \le r}{p_i}\,.
$$

Тогда получим неоднородное уравнение:
\begin{equation} 
\label{ur_per}
\fr{d\vz}{dt}= B(t)\vz(t)+\vf(t)\,, 
%\label{2.06}
\end{equation}
\noindent
где $\vf(t)=\left(\la_1, \la_2,\cdots,\la_r \right)^{\mathrm{T}}$;

\end{multicols}


\hrule

\vspace*{6pt}

\begin{equation*}
B = \left(
\begin{array}{cccccccc}
a_{11}- \la_1   & \mu_1 - \la_1   & \mu_2 - \la_1   & \mu_3 -\la_1   & \cdots& \cdots & \mu_{r-1}- \la_1  \\
\la_1 -\la_2    & a_{22} -\la_2  & \mu_1-\la_2   & \mu_2 -\la_2     & \cdots&  \cdots & \mu_{r-2} -\la_2 \\
\la_2 -\la_3    & \la_1 -\la_3   & a_{33} -\la_3  & \mu_1-\la_2   & \cdots&  \cdots & \mu_{r-3} -\la_3 \\
\cdots&\cdots&\cdots&\cdots&\cdots&\cdots&\cdots \\
\la_{r-1} -\la_r  &\la_{r-2} -\la_r & \cdots & \cdots & \la_2 -\la_r   & \la_1 -\la_r     &  a_{rr} -\la_r
\end{array}
\right)\,.
%\label{2.07}
\end{equation*}

Рассмотрим треугольную матрицу
\begin{equation*}
D=\begin{pmatrix}
d_1   & d_1 & d_1 & \cdots & d_1 \\
0   & d_2  & d_2  &   \cdots & d_2 \\
\cdots&\cdots&\cdots&\cdots&\cdots \\
0  & 0 & \cdots & 0 &  d_r
\end{pmatrix}
%\label{2.08}
\end{equation*}
и соответствующую норму $\|{\bf z}\|_{D}\hm=\|D {\bf z}\|_1$.

Тогда имеем:
\begin{equation*}
 D BD^{-1}=\left(
\begin{array}{ccccccc}
a_{11}-\la_r  &  (\mu_1-\mu_2) \fr{d_1}{d_2}  & (\mu_2-\mu_3)\fr{d_1}{d_3}  & \cdots &  (\mu_{r-1}-\mu_r)\fr{d_1}{d_r} \\
(\la_1-\la_r) \fr{d_2}{d_1} &  a_{22}-\la_{r-1}  &(\mu_1-\mu_3)\fr{d_2}{d_3}  & \cdots &  (\mu_{r-2}-\mu_r)\fr{d_2}{d_r} \\
(\la_2-\la_r) \fr{d_3}{d_1} &  (\la_1-\la_{r-1})\fr{d_3}{d_2}   &a_{33}-\la_{r-2}   & \cdots &  (\mu_{r-3}-\mu_r)
\fr{d_3}{d_r}  \\
\cdots&\cdots&\cdots&\cdots&\cdots \\
(\la_{r-1} -\la_r) \fr{d_r}{d_1} & (\la_{r-2} -\la_{r-1}) \fr{d_r}{d_2}  & (\la_{r-3} -\la_{r-2}) \fr{d_r}{d_3}  & \cdots & a_{rr}-\la_1 \\
\end{array}
\right)\,.
%\label{2.09}
\end{equation*}


\begin{multicols}{2}


Далее, оценивая логарифмическую норму оператора~$B(t)$ (см., например, 
подробное рассмотрение в~[8--10]), получаем
\begin{multline*}
\gamma \left(B(t)\right)_{1D} = \gamma \left(DB(t)D^{-1}\right)_{1}={}\\
{}=
\max \left(\vphantom{\sum\limits_{k=1}^{i-1}}
a_{ii}(t) - \la_{r-i+1}(t) + \sum\limits_{k=1}^{i-1}\left(\mu_{i-k}(t)-{}\right.\right.\\
\left.\left.{}-\mu_i(t)\right)
\fr{d_k}{d_i} +
\sum\limits_{k=1}^{r-i}(\la_k(t)-\la_{i+r-1}(t))\fr{d_{k+i}}{d_i}\right) ={}\\
{}=
 - \min \alpha_i(t) = - \alpha(t)\,.
% \label{2.10}
\end{multline*}
Тогда\\[-7.9pt]
\begin{equation*}
\|\vz^*(t)-\vz^{**}(t)\|_{1D}\le  e^{-\int\limits_s^t {\alpha(u)du}}\|\vz^*(s)-\vz^{**}(s)\|_{1D}
%\label{2.11}
\end{equation*}
для всех $0 \le s \le t$ и любых начальных условий $\vz^*(s)$, $\vz^{**}(s)$.

Теперь, учитывая оценки для сравнения норм (см., например,~\cite{z08b}), получаем:
\begin{multline*}
\|\vp^*(t)-\vp^{**}(t)\| \le 2\|\vz^*(t)-\vz^{**}(t)\| \le{}\\
{}\le  \fr{4}{d}\|\vz^*(t)-\vz^{**}(t)\|_{1D}\le{} \\
{} \le \fr{4}{d}\,e^{-\int\limits_s^t {\alpha(u)\,du}}\|\vz^*(s)-\vz^{**}(s)\|_{1D} 
\le{}\\
{}\le
 \fr{4G}{d}\,e^{-\int\limits_s^t {\alpha(u)\,du}}\|\vz^*(s)-\vz^{**}(s)\| \le{} \\
{} \le  \fr{4G}{d}\,e^{-\int\limits_s^t {\alpha(u)\,du}}\|\vp^*(s)-\vp^{**}(s)\| \le 
\fr{8G}{d}\,e^{-\int\limits_s^t {\alpha(u)\,du}} 
%\label{2.11-a}
\end{multline*}
для любых начальных условий ${\bf p^*}(s)$, ${\bf p^{**}}(s)$ и любых $s,t$, $0\hm\le s\hm\le t$.

Из слабой эргодичности процесса с конечным пространством состояний 
вытекает существование предельного среднего, начальные условия для которого можно 
в общем случае выбрать произвольно.
Для оценки средних воспользуемся неравенством, приведенным в параграфе~2.3 из~\cite{z08b}:
\begin{multline*}
\|{\bf z}\|_{1D} = d_0 \left|\sum\limits_{i=1}^{\infty} p_i \right|
+ d_1 \left|\sum\limits_{i=2}^{\infty} p_i \right| + \dots \ge{}\\
{}\ge 
 W \sum\limits_{k \ge 1} k \left|\sum\limits_{i \ge k} p_i\right| \ge \fr{W}{2}
\sum\limits_{k \ge 1} k \left|p_k\right|\,.  
%\label{2.12}
\end{multline*}
Получаем теперь
\begin{multline*}
|E(t,k)-\phi(t)|\le \fr{2}{W}\,\|\vp^*(t)-\vp^{**}(t)\|_{1D}\le {} \\
{}\le\fr{2}{W}\,e^{-\int\limits_0^t {\alpha(u)\,du}}\|{\bf e}_k -
\vp^{**}(0)\|_{1D} \le \frac{4G}{W}e^{-\int\limits_0^t
{\alpha(u)\,du}}\,,
%\label{2.13}
\end{multline*}
что и требовалось доказать.
\columnbreak

%\smallskip

\noindent
\textbf{Замечание~1.} {Положим в условиях теоремы~1 
$$
\beta(t)=\max\limits_{1 \le i \le r}\alpha_i(t)\,.
$$ 
Тогда, пользуясь внедиагональной неотрицательностью матрицы $DB(t)D^{-1}$ 
с помощью методики, описанной в~\cite{z08b, z95b}, получаем справедливость неравенства

\noindent
\begin{equation*} 
%\label{2.14}
\|\vp^*(t)-\vp^{**}(t)\| \ge \fr{d}{8G}\,e^{-\int\limits_s^t {\beta(u)\,du}}
\end{equation*}
при любых $s$, $t$, $0\le s\le t$ и уже не при любых начальных условиях~${\bf p^*}(s)$, 
${\bf p^{**}}(s)$, а таких, что  $D\left({\bf p^*}(s) \hm-{\bf p^{**}}(s)\right) \hm\ge 0.$ 
Следовательно, оценки тео\-ре\-мы~1 будут заведомо иметь точный по времени порядок, если удастся 
выбрать вспомогательную последовательность $\{d_i\}$ так, что $\alpha(t)\hm=\beta(t)$, т.\,е.\ 
все $\alpha_i(t)$ одинаковы (не зависят от индекса~$i$)}.



\smallskip

Введем теперь в рассмотрение величины

\vspace*{-1pt}

\noindent
\begin{multline*}
\zeta_i(t)= -a_{ii}(t)+\la_{r-i+1}(t)+{}\\
{}+\sum\limits_{k=1}^{i-1}\left(\mu_{i-k}(t)-
\mu_i(t)\right) \fr{d_k}{d_i}+{}\\
{}+\sum\limits_{k=1}^{r-i}\left(\la_k(t)-\la_{i+r-1}(t)\right)\fr{d_{k+i}}{d_i}\,;
%\label{2.0211}
\end{multline*}
\begin{equation*}
\chi(t)=\max\limits_{1 \le i \le r}\zeta_i(t)\,.
%\label{2.0311}
\end{equation*}

\noindent
\textbf{Замечание 2.} {В условиях теоремы~1 при любых начальных условиях 
${\bf p^*}(s)$, ${\bf p^{**}}(s)$ и любых $s,t$,  $0\le s\le t$, 
справедлива следующая двухсторонняя оценка скорости сходимости:

\vspace*{-1pt}

\noindent
\begin{multline*} 
%\label{2.041}
\!\!\!\fr{d}{4G}\,e^{-\int\limits_s^t {\chi(u)\,du}}\|\vp^*(s)-\vp^{**}(s)\| \le
 \|\vp^*(t)-\vp^{**}(t)\| \le {}\\
 {}\le\fr{4G}{d}\,e^{-\int\limits_s^t {\alpha(u)\,du}}\|\vp^*(s)-\vp^{**}(s)\|.
\end{multline*}
Таким образом, можно оценить и сверху и снизу время  вхождения 
сис\-те\-мы обслуживания в предельный режим. Более подробно о получении 
нижних оценок см., например, в~\cite{z95b, gz05}.}

\smallskip

Рассмотрим два частных случая теоремы.

\smallskip

\noindent
\textbf{Следствие 1}. \textit{Пусть при выполнении остальных условий теоремы~1 
вместо}~(\ref{2.031}) \textit{выполняется условие $\alpha(t) \hm\ge \alpha \hm> 0$ 
почти при всех $t \hm\ge 0$. Тогда вместо}~(\ref{2.04}) \textit{и}~(\ref{2.05}) 
\textit{справедливы оценки}:

\vspace*{-1pt}

\noindent
\begin{align*} 
%\label{2.15}
\|\vp^*(t)-\vp^{**}(t)\| &\le \fr{8G}{d}\,e^{-\alpha \left(t-s\right)}\,;
\\
%\label{2.16}
|E(t,k)-\phi(t)|&\le \fr{4G}{W}\,e^{- \alpha t}\,.
\end{align*}

\pagebreak

%\smallskip

Положим 
\begin{gather*}
M_0=\max\limits_{|t-s|\le 1}\int\limits_s^t \alpha(u)\,du;\\
\alpha^* = \int\limits_0^1 \alpha(t)\, dt\,; \quad
M=e^{M_0+\alpha^*}\,.
\end{gather*}
С учетом неравенства 
$$
e^{-\int\limits_s^t {\alpha(u)\,du}} \hm\le M e^{-\alpha^* (t-s)}
$$ 
получаем следующее утверждение.

\smallskip

\noindent
\textbf{Следствие~2.} \textit{Пусть все $\lambda_k(t)$ и $\mu_k(t)$ 1-пе\-ри\-одич\-ны,  
а при выполнении остальных условий теоремы~1 вместо}~(\ref{2.031}) 
\textit{выполняется условие  $\alpha^* \hm> 0$.  Тогда предельный режим (скажем, $\vp^*(t)$) 
и соответствующее ему предельное среднее $\phi^*(t)$ можно выбрать 
1-пе\-ри\-оди\-че\-ски\-ми, а вместо}~(\ref{2.04}) \textit{и}~(\ref{2.05}) 
\textit{справедливы оценки}:
\begin{equation*} 
%\label{2.17}
\|\vp(t) - \vp^*(t)\| \le \fr{8GM}{d}\,e^{-\alpha^*t}
\end{equation*}
\textit{и, кроме того,}
\begin{equation*}
|E(t,k)-\phi^*(t)|\le \fr{4GM}{W}\,e^{-\alpha^*t}
%\label{2.18}
\end{equation*}
\textit{при любом $k$ и $t \ge 0$}.



\section{Устойчивость}

Рассмотрим также <<возмущенный>> процесс обслуживания $\bar{X}\hm=\bar{X}(t)$, $t\hm\geq 0$, 
в котором интенсивности поступления и обслуживания требований также не зависят от чис\-ла 
требований в системе, обозначая его соответствующие характеристики теми же буквами с 
чертой сверху. Для прос\-то\-ты записи оценок будем предполагать, что возмущения 
<<равномерно малы>>, т.\,е.\ выполняется неравенство $\| A(t)-\bar{A}(t)\| \hm\le \varepsilon$. 
Первые результаты для нестационарных цепей с непрерывным временем получены в~\cite{z85}, 
а детальное рассмотрение для более общего случая неравномерных оценок можно без труда 
провести так же, как это сделано в~\cite{z98, ae}. Для получения требуемых равномерных 
оценок устойчивости необходима экспоненциальная эргодичность соответствующего процесса, 
т.\,е.\ существование положительных констант $N$, $a$ таких, что  для правой части~(\ref{2.04}) 
справедливо неравенство:
\begin{equation}
e^{-\int\limits_s^t {\alpha(u)\,du}} \le Ne^{-a\left(t-s\right)}\,.
\label{3.01}
\end{equation}
Оценка~(\ref{3.01}) заведомо имеет место, в частности, если выполнены условия одного из следствий 
предыду\-ще\-го параграфа.

\smallskip

\noindent
\textbf{Теорема~2.}
\textit{Пусть выполнены условия теоремы~1 и}~(\ref{3.01}). \textit{Тогда при
 любых начальных условиях ${\bf p}(s)$ и ${\bar{\bf p}}(s)$ для процессов~$X(t)$ 
 и $\bar{X}(t)$ соответственно справедливы следующие оценки устойчивости:}
\begin{align*} 
%\label{3.02}
\limsup_{t \to \infty}  \|{\bf p}(t)- \bar{\bf p}(t)\| &\le
\fr{\varepsilon(1+\ln(4GN/d))}{a}\,;
\\
% \label{3.03}
\limsup\limits_{t \to \infty}   |E_{\bf p}(t)- \bar{E}_{\bar{\bf p}(t)}|&\le 
\fr{r \varepsilon(1+\ln(4GN/d))}{a}\,.
\end{align*}


\smallskip

\noindent
Д\,о\,к\,а\,з\,а\,т\,е\,л\,ь\,с\,т\,в\,о\ основано на подходе, 
введенном для стационарных процессов в~\cite{mit03} и описанном для нестационарной 
ситуации в~\cite{z11}.
Если  при любых начальных условиях для исходного процесса справедлива оценка
\begin{equation*} 
%\label{3.04}
\|\vp(t) - \vp^*(t)\| \le ce^{-b\left(t-s\right)}\,,
\end{equation*}
то, полагая
\begin{multline*}
\beta (t, s)=\sup\limits_{ \| {\bf v} \| =1, \sum {v_i}=0}
{\|V(t,s){\bf v}(t,s)\|} ={}\\
{}= \fr{1}{2} \max_{i,j} \sum\limits_k {|p_{ik}(t,
s)-p_{jk}(t, s)|}\,, 
\end{multline*}
где $V(t, s)$~--- матрица Коши
уравнения~(\ref{ur_1}), получаем в итоге следующее неравенство:
\begin{equation*}
\|{\bf p}(t)-\bar{\bf p}(t)\| \le{}
\begin{cases}
\|{\bf p}(s)-{\bf \bar{p}}(s)\|+ (t-s)\varepsilon \,, &\\
&\hspace*{-35mm} 0<t< b^{-1} \ln \left(\fr{c}{2}\right)\,; \\
b^{-1}\left(\ln \fr{c}{2} +1-\fr{c}{2}\,e^{-b(t-s)}\right)\varepsilon +{}&\\
{}+
\fr{c}{2}\,e^{-b(t-s)} \|{\bf p}(s)-{\bf \bar{p}}(s)\|\,, &\\
&\hspace*{-30mm}t\ge b^{-1}\ln \left(\fr{c}{2}\right)
\end{cases}
%\label{3.05}
\end{equation*}
для любых начальных условий ${\bf p}(s)$ и $\bar{\bf p}(s)$.
Из неравенств~(\ref{2.04}) и~(\ref{3.01}) вытекает, что $b=a$, $c={8GN}/{d}$.  
Устремив $t \hm\to \infty$ и взяв $s\hm=0$, получаем требуемые оценки.


\smallskip

\noindent
\textbf{Замечание~3.} 
В полученную оценку устойчивости для математического ожидания процесса 
в качестве множителя входит размерность~$r$, поэтому иногда лучший результат 
удается получить при помощи другого подхода, описанного в работе~\cite{z11}.

\smallskip

Положим 
$$
S=\max\limits_{{1 \le i, j \le r}} \fr{d_i}{d_j}\,,
$$ 
и пусть числа $K, L$ таковы, что 

\noindent
$$
d_1\la_1(t) + (d_1+d_2)\la_2(t) + \dots + 
\left(\sum\limits_{1 \le i \le r}d_i\right) \la_r(t) \le K\,,
$$ 
а 

\noindent
\begin{multline*}
d_1(\la_1(t)-\bar{\la}_1(t)) + (d_1+d_2)(\la_2(t)-\bar{\la}_2(t)) + \dots\\
\dots + 
\left(\sum\limits_{1 \le i \le r}d_i\right) (\la_r(t)-\bar{\la}_r(t)) \le 
L\varepsilon
\end{multline*} 
почти при всех $t \ge 0.$

\smallskip

\noindent
\textbf{Теорема~3.}
\textit{Пусть  выполнены условия теоремы~2 и, кроме того, при всех~$k$ 
и почти всех $t \hm\ge 0$ $\la_k(t) \hm< \infty$. Тогда при любых начальных условиях 
${\bf p}(s)$ и ${\bar{\bf p}}(s)$ для процессов $X(t)$ и $\bar{X}(t)$ 
соответственно справедливо неравенство}

\noindent
\begin{equation*}
\limsup\limits_{t \to \infty}   |E_{\bf p}(t)- \bar{E}_{\bar{\bf p}(t)}|\le 
\fr{ N\varepsilon\left(L a+ 2KNS\right)}{W a \left(a-2\varepsilon S\right)}\,.
\end{equation*}


\smallskip

\noindent
Д\,о\,к\,а\,з\,а\,т\,е\,л\,ь\,с\,т\,в\,о.\
 Перепишем исходную систему~(\ref{ur_per}) для невозмущенного процесса в следующем виде:
 \noindent
 
\begin{equation*}
\fr{d\vp}{dt}=\bar{B}(t)\vp(t) + {\bf f}(t)+\left(B(t)-\bar{B}(t)\right)\vp(t)\,.
%\label{eq112-n}
\end{equation*}
Тогда

\noindent
\begin{multline*}
\vp(t)=\bar{U}(t,0)\vp(0)+\int\limits_0^t \bar{U}(t,\tau){\bf{f}}(\tau) \, d\tau+{}\\
{}+\int\limits_0^t \bar{U}(t,\tau) \left(B(\tau)-\bar{B}(\tau)\right)\vp(\tau)\, d\tau\,;
\end{multline*}

\vspace*{-9pt}

\begin{equation*}
\hspace*{-15mm}\bar{\vp}(t)=\bar{U}(t,0)\bar{\vp}(0)+\int\limits_0^t \bar{U}(t,\tau){\bf{f}}(\tau) \, d\tau,
\end{equation*}
где $U(t,s)$~--- матрица Коши для уравнения~(\ref{ur_per}).
В любой норме при одинаковых начальных условиях получаем следующую оценку:
%\noindent
\begin{multline}
 \label{3000}
\!\!\!\!\!\!\left\|\vp(t)-\bar{\vp}(t)\right\|\le \!\!\int\limits_0^t \!\!\|\bar{U}(t,\tau)\|
\left(\| B(\tau)-\bar{B}(\tau)\| \|\vp(\tau)\| +\right.\\
\left.{}+ \| \vf(\tau)-\bar{\vf}(\tau)\|\right)\,d\tau\,.\!
\end{multline}
Имеем почти при всех $t \ge 0$:
\begin{equation*}
\|B(t)-\bar{B}(t)\|_{1D}=\|D(B(t)-\bar{B}(t))D^{-1}\| \le 2S\varepsilon\,;
%\label{3002}
\end{equation*}
%
%\vspace*{-14pt}
%
%\noindent
\begin{multline*}
\|{\bf f}(t)\|_{1D} \le d_1\la_1(t) + (d_1+d_2)\la_2(t) + \dots + {}\\
{}+
\left(\sum\limits_{1 \le i \le r}d_i\right) \la_r(t) \le K\,, 
\quad \|\vf(\tau)-\bar{\vf}(\tau)\|_{1D} \le L\varepsilon\,.
%\label{3002-a}
\end{multline*}
А тогда
\begin{multline*}
\gamma(\bar{B}(t))_{1D} \le \gamma(DB(t)D^{-1})+\|B(t)-\bar{B}(t)\|_{1D} \le  {}\\
{}\le -
\alpha(t)+2S \varepsilon \,.
% \label{3003}
\end{multline*}

Оценим теперь
\begin{multline*} 
%\label{8402}
\!\|{\bf p}(t)\|_{1D} \le
\|U(t){\bf p}(0) \|_{1D} +
 \int\limits_0^t \!\!\| U(t,\tau){\bf f}(\tau)\, d\tau \|_{1D} \le {}\\
 {}\le
 N e^{-a t} \| \vp(0)\|_{1D}  + \fr{K N}{a}.
\end{multline*}

 Тогда с учетом~(\ref{3000}) получаем:
\begin{multline*} 
%\label{3004}
\left\|\vp(t)-\bar{\vp}(t)\right\|_{1D}\le N\int\limits_0^t e^{-(a - 2\varepsilon S)(t-\tau)}\times{}\\
{}\times
\left(2S\varepsilon (N e^{-a \tau} \| \vp(0)\|_{1D}  + \fr{K N}{a}) +  L\varepsilon \right)\, d\tau  \le {} \\
{}\le  o(1)+\fr{ N\varepsilon(L+{2KNS}/{a})}{a-2\varepsilon S}\,. 
\end{multline*}

\vspace*{-9pt}

\section{Примеры}

\noindent
\textbf{Пример 1.}

Рассмотрим исходный процесс обслуживания с интенсивностями 
$\la_1(t)\hm=\la_2(t)\hm=\la_3(t)\hm=\la(t) \hm= 3\hm+\sin{2\pi t}$, 
$\mu_1(t)\hm=\mu_2(t)\hm=  \mu(t) \hm= 2\hm+\cos{2\pi t}$, 
$\la_4(t)=\ldots=\la_r(t)\hm=\mu_3(t)=\ldots=\mu_r(t)\hm=0$. Выберем последовательность  
$d_k\hm=h^k$, где $0{,}82 \hm< h \hm<1$. Тогда имеем
$$
d=h^r\,; \quad G \le \fr{h}{1-h}\,; \quad W=\fr{h^r}{r}\,.
$$

Будем предполагать, что возмущенный процесс имеет такую же структуру 
мат\-ри\-цы интенсивностей, причем $|\la(t)\hm-\bar{\la}(t)| \hm\le \varepsilon$ 
и  $|\mu(t)\hm-\bar{\mu}(t)| \hm\le \varepsilon$ почти при всех $t \hm\ge 0$. 
Отметим кстати, что при этом $\| A(t)\hm-\bar{A}(t)\| \hm\le 10 \varepsilon$ почти при 
всех $t \hm\ge 0$. Рассмотрим дальнейшие оценки:
$$
S=\fr{1}{h^2}\,; \ K=4 \left(3h+2h^2+h^3\right)\,; \ L=3h+2h^2+h^3\,;
$$
$$
\alpha(t) \ge \la(t)\left(3 - h - h^2 -h^3\right)-\mu(t)\left(\fr{1}{h^2}+\fr{1}{h}-2\right)\,;
$$
$$
\alpha^*= 3\left(3 - h - h^2 -h^3\right)-2\left(\fr{1}{h^2}+\fr{1}{h}-2\right)\,;
$$


\noindent
\begin{multline*}
M_0 \le \int\limits_0^1 |\alpha(t)|\, dt \le 4\left(3 - h - h^2 -h^3\right)+{}\\
{}+
3\left(\fr{1}{h^2}+\fr{1}{h}-2\right)\,;
\end{multline*}

\vspace*{-9pt}

\noindent
$$
M=e^{\alpha^*+M_0}\,.
$$

Если, например, взять 
$h\hm=0{,}9$, то $\alpha^*\hm=0{,}992$, $M_0\hm=3{,}281$, $M\hm=71{,}737$.

Тогда получаем следующие оценки.

По следствию~2
\begin{align*}
 \|{\bf p}(t)- {\bf p^{*}}(t)\| &\le \fr{8Me^{-\alpha^*t}}{h^{r-1}(1-h)}\,;\\
|E_{\bf p}(t)-\phi^*(t)| &\le  \fr{4Mre^{-\alpha^*t}}{h^{r-1}(1-h)}\,.
\end{align*}

По теореме~2 ($N=M$, $a=\alpha^*$) с использованием оценок следствия~2
\begin{align*}
\limsup\limits_{t \to \infty} \|{\bf p}(t)- \bar{\bf p}(t)\| &\le{} \notag\\
&\hspace*{-15mm}{}\le \fr{\varepsilon(1+\ln({4M}/({h^{r-1}(1-h)})))}{\alpha^*}\,;\\
\limsup\limits_{t \to \infty}   |E_{\bf p}(t)- \bar{E}_{\bar{\bf p}(t)}| &\le \notag\\
&\hspace*{-15mm}{}\le\fr{r\varepsilon(1+\ln(4M/(h^{r-1}(1-h))))}{\alpha^*}\,.
\end{align*}

По теореме~3 с использованием оценок следствия~2
\begin{multline*}
\limsup\limits_{t \to \infty}   |E_{\bf p}(t)- \bar{E}_{\bar{\bf p}(t)}| \le {}\\
{}\le
\fr{rM\varepsilon(3h+2h^2+h^3)(\alpha^* h^2+8M)}{h^r\alpha^*(\alpha^* h^2-2\varepsilon)}\,.
\end{multline*}

\noindent

\textbf{Пример 2.}

Рассмотрим процесс с интенсивностями 
$\la_1(t)\hm=\la_2(t)\hm=\ldots=\la_r(t) \hm= \la(t) \hm= 3\hm+\sin{2\pi t}$; 
$\mu_1(t)\hm=\mu_2(t)\hm= \mu(t) \hm= 2+\cos{2\pi t}$;
$\mu_3(t)=\ldots=\mu_r(t)=0$.

Будем предполагать, что возмущенный процесс имеет такую же структуру 
мат\-ри\-цы интен\-сив\-ностей, причем $|\la(t)-\bar{\la}(t)| \hm\le \varepsilon$ и  
$|\mu(t)-\bar{\mu}(t)| \hm\le \varepsilon$ почти при всех $t \hm\ge 0$. 
При этом будем иметь $\| A(t)\hm-\bar{A}(t)\| \hm\le 2r \varepsilon$ почти при всех $t \hm\ge 0$.

Выберем последовательность $d_k\hm=1$. Тогда  
\begin{gather*}
d=1\,; \enskip G=r\,; \enskip W=\fr{1}{r}\,; \enskip S=1\,; \\
K=\fr{4r(1+r)}{2}\,; \quad L=\fr{r(1+r)}{2}\,;
\\
\alpha(t)=\la(t)\,; \ \alpha=2\,; \ \alpha^*=3\,; M_0 \le 4\,; \ M \le  e^{7}\,.
\end{gather*}

И получаем следующие оценки.

\columnbreak

По следствию~1
\begin{align*}
 \|{\bf p^*}(t)- {\bf p^{**}}(t)\| &\le 8re^{-2t}\,;\\
|E_{\bf p}(t)- \phi(t)|&\le  4r^2 e^{-2t}\,.
\end{align*}

По следствию~2
\begin{align*}
\|{\bf p}(t)- {\bf p^{*}}(t)\| &\le 8re^{7-3t}\,;
\\[6pt]
|E_{\bf p}(t)- \phi^*(t)| &\le 4r^2 e^{7-3t}\,.
\end{align*}

По теореме~2 ($N=1$, $a=\alpha$) с учетом оценок следствия~1
\begin{align*}
\limsup\limits_{t \to \infty} \|{\bf p}(t)- \bar{\bf p}(t)\| &\le 
\fr{\varepsilon(1+\ln{4r})}{2}\,;
\\[6pt]
\limsup\limits_{t \to \infty}   |E_{\bf p}(t)- \bar{E}_{\bar{\bf p}(t)}|
&\le \fr{r\varepsilon(1+\ln{4r})}{2}\,.
\end{align*}

По теореме~2 ($N=M$, $a=\alpha^*$) с учетом оценок следствия~2
\begin{align*}
\limsup\limits_{t \to \infty} \|{\bf p}(t)- \bar{\bf p}(t)\| &\le 
\fr{\varepsilon(8+\ln{4r})}{3}\,;
\\
\limsup\limits_{t \to \infty}   \left|E_{\bf p}(t)- \bar{E}_{\bar{\bf p}(t)}\right| &\le 
\fr{r\varepsilon(8 + \ln{4r})}{3}\,.
\end{align*}

По теореме~3 с учетом оценок следствия~1
\begin{equation*}
\limsup\limits_{t \to \infty}   \left|E_{\bf p}(t)- \bar{E}_{\bar{\bf p}(t)}\right| \le 
\fr{5 \varepsilon r^2 (1+r)}{4(1- \varepsilon)}\,.
\end{equation*}

По теореме~3 с учетом оценок следствия~2
\begin{equation*}
\limsup\limits_{t \to \infty}   \left|E_{\bf p}(t)- \bar{E}_{\bar{\bf p}(t)}\right| \le 
\fr{\varepsilon e^{7} r^2 (1+r) (3+8e^{7})}{6(3-2\varepsilon)}\,.
\end{equation*}

{\small\frenchspacing
{%\baselineskip=10.8pt
\addcontentsline{toc}{section}{Литература}
\begin{thebibliography}{99}

 \bibitem{b} %1
\Au{Баруча-Рид~А.\,Т.} Элементы теории марковских процессов и их
приложения.~--- М.: Наука, 1969.

\bibitem{gm}  %2
\Au{Гнеденко~Б.\,В., Макаров~И.\,П.} Свойства решений задачи с потерями
в случае периодических интенсивностей~// Дифф. уравнения, 1971.
Вып.~7. С.~1696--1698.

\bibitem{g1}   %3
\Au{Gnedenko~D.\,B.} On a generalization of Erlang formulae~// 
Zastosow. Mat., 1971. Vol.~12. P.~239--242.

\bibitem{S}  %4
\Au{Саати~Т.\,Л.} Элементы теории массового обслуживания
 и ее приложения.~--- М.: Сов. радио, 1971.

\bibitem{g}  %5
\Au{Gnedenko~B., Soloviev~A.} On the conditions of the
existence of final probabilities for a Markov process~// Math.
Operations. Stat., 1973. P.~379--390.

\bibitem{gk} %6
\Au{Гнеденко~Б.\,В., Коваленко~И.\,Н.} Введение в теорию массового
обслуживания.~--- М.: Наука, 1987.
\pagebreak

\bibitem{gz00}   %7
\Au{Granovsky~B.\,L., Zeifman~A.\,I.}  The N-limit of spectral gap of 
a class of birth-death Markov chains~//
 Appl. Stoch. Models Business Ind., 2000. Vol.~16. P.~235--248.

\bibitem{z08b}  %8
\Au{Зейфман~А.\,И., Бенинг~В.\,Е., Соколов~И.\,А.} 
Марковские цепи и модели с непрерывным временем.~--- М.: Элекс-КМ, 2008.

\bibitem{dzp} %9
\Au{Van Doorn~E.\,A., Zeifman~A.\,I., Panfilova~T.\,L.}  
Bounds and asymptotics for the rate of convergence of birth-death processes~//  
Th. Prob. Appl., 2010. Vol.~54. P.~97--113.

\bibitem{z95b}   %10
\Au{Zeifman~A.\,I.} Upper and lower bounds on the rate of
convergence for nonhomogeneous birth and death processes~//  Stoch.
Proc. Appl., 1995. Vol.~59. P.~157--173.

\bibitem{gz05}  %11
\Au{Granovsky~B.\,L., Zeifman~A.\,I.} On the lower bound of the spectrum
 of some mean-field models~// Theory Prob. Appl., 2005. Vol.~49. P.~148--155.
 
\bibitem{z85}  %12
\Au{Zeifman~A.\,I.} Stability for contionuous-time
nonhomogeneous Markov chains~// Lect. Notes Math.,  1985. Vol.~1155.
P.~401--414.

\bibitem{z98} %13
\Au{Zeifman~A.} Stability of birth and death processes~// 
J.~Math. Sci., 1998. Vol.~91. P.~3023--3031.

\bibitem{ae} %14
\Au{Андреев~Д., Елесин~М., Кузнецов~А., Крылов~Е., Зейфман~А.}
Эргодичность и устойчивость нестационарных систем обслуживания~//
Теория вероятностей и математическая статистика, 2003. Т.~68.
С.~1--11.

\bibitem{mit03} %15
\Au{Mitrophanov~A.\,Yu.} Stability and exponential convergence of continuous-time 
Markov chains~//  J. Appl. Prob., 2003. Vol.~40. P.~970--979.

\label{end\stat} 

\bibitem{z11} %16
\Au{Зейфман~А.\,И., Коротышева~А.\,В., Панфилова~Т.\,Л., Шоргин~С.\,Я.} 
Оценки устойчивости  для некоторых систем обслуживания с катастрофами~//  
Информатика и её применения, 2011. Т.~5. Вып.~3. С.~27--33.
 \end{thebibliography}
}
}


\end{multicols}           %10
\include{aliu}       %11
\include{burtseva}   %12
\def\stat{grusho}

\def\tit{АРХИТЕКТУРНЫЕ РЕШЕНИЯ В~ЗАДАЧЕ ВЫЯВЛЕНИЯ МОШЕННИЧЕСТВА ПРИ~АНАЛИЗЕ 
ИНФОРМАЦИОННЫХ ПОТОКОВ В~ЦИФРОВОЙ ЭКОНОМИКЕ$^*$}

\def\titkol{Архитектурные решения в~задаче выявления мошенничества при~анализе 
информационных потоков в
%~цифровой 
экономике}

\def\aut{А.\,А.~Грушо$^1$, М.\,И.~Забежайло$^2$, Н.\,А.~Грушо$^3$, 
Е.\,Е.~Тимонина$^4$}

\def\autkol{А.\,А.~Грушо, М.\,И.~Забежайло, Н.\,А.~Грушо, 
Е.\,Е.~Тимонина}

\titel{\tit}{\aut}{\autkol}{\titkol}

\index{Грушо А.\,А.}
\index{Забежайло М.\,И.}
\index{Грушо Н.\,А.}
\index{Тимонина Е.\,Е.}
\index{Grusho A.\,A.}
\index{Zabezhailo M.\,I.}
\index{Grusho N.\,A.}
\index{Timonina E.\,E.}


{\renewcommand{\thefootnote}{\fnsymbol{footnote}} \footnotetext[1]
{Работа частично поддержана РФФИ (проекты 18-29-03081 и~18-07-00274).}}


\renewcommand{\thefootnote}{\arabic{footnote}}
\footnotetext[1]{Институт проблем информатики Федерального исследовательского центра <<Информатика и~управление>> 
Российской академии наук, grusho@yandex.ru}
\footnotetext[2]{Институт проблем информатики Федерального исследовательского центра <<Информатика и~управление>> 
Российской академии наук, m.zabezhailo@yandex.ru}
\footnotetext[3]{Институт проблем информатики Федерального исследовательского центра <<Информатика и~управление>> 
Российской академии наук, info@itake.ru}
\footnotetext[4]{Институт проблем информатики Федерального исследовательского центра <<Информатика и~управление>> 
Российской академии наук, eltimon@yandex.ru}

\vspace*{-12pt}
   

 
  
  \Abst{Cформулирован подход к~исследованию некоторых видов мошенничества в~цифровой 
экономике с~использованием причинно-следственных связей. Во всех видах рассматриваемых 
мошенничеств должно наблюдаться несоответствие между целями финансовых транзакций 
и~реальной стоимостью достижения этих целей. Данные о транзакциях можно собирать, 
наблюдая информационные потоки, в~которых отражаются эти транзакции. Архитектура сбора 
данных и~их анализа может быть организована с~помощью распределенных реестров 
с~централизованным консенсусом, что позволяет создать аналог электронной бухгалтерской 
книги, фиксирующей финансово-экономическую деятельность субъектов цифровой экономики в~регионе. 
  Рассматриваемые методы выявления мошенничества основаны на противоречиях 
между действиями, описанными в~транзакциях, и~информацией, содержащейся в~планах, 
стандартах, прецедентах и~др. Рассмотрен метод, основанный на некоторой упрощенной схеме 
реализации абстрактного проекта. Для выявления противоречий необходимо проводить анализ 
от следствия к~причине, т.\,е.\ искать аномалии в~информации, описывающей порождение 
наблюдаемых следствий. 
  Показано, как в~реализации проекта можно выделять простые <<необходимые условия>> 
нарушения при\-чин\-но-след\-ст\-вен\-ных связей, т.\,е.\ множество <<необходимых условий>>, 
нарушение которых свидетельствует о наличии мошенничества. Это множество <<необходимых 
условий>> можно назвать метаданными для контроля проекта на выявление мошенничества.} 
 
 
  \KW{цифровая экономика; информационные потоки; при\-чин\-но-след\-ст\-вен\-ные связи; 
выявление мошеннических схем} 

\DOI{10.14357/19922264190204}
  
\vspace*{-4pt}


\vskip 10pt plus 9pt minus 6pt

\thispagestyle{headings}

\begin{multicols}{2}

\label{st\stat}

\section{Введение}

\vspace*{3pt}

  В работе сформулирован подход к~исследованию некоторых видов 
мошенничества в~цифровой экономике с~использованием  
при\-чин\-но-след\-ст\-вен\-ных связей. Рассматриваются три вида мошенничества, 
а именно:
  \begin{enumerate}[(1)]
\item отмыв денег; 
\item обман при выполнении договорных обязательств при реализации 
технических проектов (строительные проекты и~др.); 
\item незаконный вывод денег. 
\end{enumerate}

  Названные виды мошенничества могут быть сведены к~решению одного типа 
задач. Для отмывания денег источник должен заключать фиктивные контракты, 
в~соответствии с~которыми будут переводиться средства за заведомо ненужную 
работу и~материалы. 
  
  Мошенничество, связанное с~невыполнением договорных обязательств, связано 
со снижением качества услуг, качества и~количества закупаемых 
материалов, выполнением работ с~ненадлежащим качеством. 
  
  Вывод денег связан с~переводом средств фир\-мам-од\-но\-днев\-кам, которые 
заведомо не могут выполнить обязательства по контрактам, за которые им 
переводятся средства. 
  
  Таким образом, во всех трех видах рассматриваемых мошенничеств должно 
наблюдаться несоответствие между целями финансовых транзакций и~реальной 
стоимостью достижения этих целей. Данные о транзакциях можно собирать, 
наблюдая информационные потоки, в~которых отражаются эти транзакции. 
  
  Однако для наблюдения таких информационных потоков необходимо создавать 
архитектуру\linebreak телекоммуникационной системы, позволяющей перехватывать 
и~собирать данные о всех транзакциях. Например, такая архитектура может быть 
организована с~помощью распределенных реестров с~централизованным 
консенсусом, т.\,е.\ все информационные потоки, сформированные в~цифровой 
экономике и~несущие информацию о транзакциях, проходят через некоторый 
центральный узел, запоминающий их в~форме распределенного реестра. Такие 
реестры могут дублироваться в~аналогичных центрах различных регионов, что 
позволяет создать аналог электронной бухгалтерской книги, фиксирующей 
фи\-нан\-со\-во-эко\-но\-ми\-че\-скую деятельность субъектов цифровой экономики. Такой 
подход предложено реализовать на базе системы ситуационных центров, что 
отражено в~работах~[1, 2].
  
  Собранная из информационных потоков информация о~транзакциях, т.\,е.\ 
о~контрактах, договорах, платежах, отчетах, закупленных материалах, 
характеристиках исполнителей работ и~др., собирается в~базе данных в~указанном 
центре. Согласно теории интеллектуальных сис\-тем~[3], эту базу данных можно 
называть базой фактов (БФ). Базу фактов можно представить как бинарную мат\-ри\-цу, 
строки которой описывают характеристики, входящие в~транзакции, а столбцы 
нумеруются характеристиками. Строки матрицы будем называть 
\textit{объектами}~[4, 5]. 
  
  Рассматриваемые в~работе методы выявления мошенничества будут основаны 
на противоречиях между действиями, описанными в~транзакциях, и~информацией, 
содержащейся в~планах, стандартах, прецедентах и~др. Для нахождения 
противоречий в~архитектуре центра предусмотрена другая база данных~--- база 
знаний (БЗ)~\cite{3-gr, 6-gr}, которая устроена так же, как БФ. 
  
  Информация в~БЗ собирается на основе положительного опыта или расчетов. 
Используя БЗ, можно выводить факты нарушения при\-чин\-но-след\-ст\-вен\-ных 
связей. Нарушения при\-чин\-но-след\-ст\-вен\-ных связей будем называть 
\textit{аномалиями}. 
  
  Для упрощения дальнейшее изложение будет вестись в~рамках поиска 
противоречий при выполнении некоторого абстрактного проекта. Выявление 
аномалий будет происходить на основе фактов из БФ с~помощью знаний из БЗ 
методами искусственного интеллекта и~интеллектуального анализа 
данных~\cite{6-gr}. 

\vspace*{-10pt}
  
  \section{Модели}
  
  \vspace*{-3pt}
  
  Наиболее сложная из рассмотренных выше задач~--- выявление противоречий, 
т.\,е.\ использование БЗ для получения новых знаний и~выявление аномалий из 
полученных фактов. 
  
  Все способы выявления противоречий основаны на определении 
  причинно-следственных связей. При этом противоречия в~параметрах транзакций по 
отношению к~требуемым в~БЗ составляют сущность аномалий. 
  
   Далее будет рассмотрен метод, основанный на некоторой упрощенной схеме 
реализации абстрактного проекта. 
  
  Каждый проект имеет цель: например, цель представляет собой построение 
некоторой системы. Воспользуемся структурным подходом, который позволяет 
строить проект на основе разбиения системы на подсистемы и~определения 
взаимодействий подсистем~\cite{7-gr}. При этом каждая подсистема также 
представима структурной моделью. 
  
  Как сама система, так и~каждая ее подсистема имеют свой функционал 
и~спецификацию, па\-ра\-мет\-ры настройки и~домены параметров настройки. Кроме 
этих характеристик существует множество характеристик, связанных 
с~<<жизненным циклом>> создания системы. Сюда входят работы, ресурсы, 
сроки выполнения работ по созданию подсистем и~самой системы, стоимости 
компонентов и~материалов, стоимости работ, схемы поставок, договорные 
обязательства и~др. Все характеристики связаны между собой, поэтому можно 
говорить о стоимости и~времени изготовления структурных компонентов системы. 
  
  Одной из важнейших характеристик является смета (система смет для 
подсистем). Смета сопоставляет каждому компоненту системы стоимость его 
изготовления и~настройки. 
  
  Схема построения системы может быть пред\-став\-ле\-на диаграммой, 
изображенной на рис.~1. 

{ \begin{center}  %fig1
 \vspace*{9pt}
   \mbox{%
 \epsfxsize=79mm 
 \epsfbox{gru-1.eps}
 }


\vspace*{9pt}


\noindent
{{\figurename~1}\ \ \small{Диаграмма достижения цели}}
\end{center}
}

\vspace*{9pt}

\addtocounter{figure}{1}
  
  


  Представленная на рис.~1 диаграмма позволяет описать основные классы 
возможных противоречий при достижении цели. Противоречия возникают, когда 
данные БФ не соответствуют требуемым характеристикам. 
  
  
  \section{Потенциальные классы аномалий при~достижении цели}
  
  Выделим четыре потенциальных класса противоречий, которые показывают, 
каким образом нужно искать эти противоречия.
  
 
  Противоречие цели и~проекта (рис.~2) возникает при отсутствии обоснования 
или в~случае логического противоречия между возможностями проектируемого 
функционала и~целью системы. Отметим, что в~проект входят сроки, перечень 
работ, материалы, настройки, которые описываются соответствующими 
параметрами и~допустимыми значениями этих параметров. Проект формируется 
на основе БЗ и~расчетов, исходя из информации, полученной по аналогии 
с~другими проектами и~решениями, которые считаются апробированными. 
  
  Отметим, что цель порождает проект и~в этом смысле является причиной 
проекта. Однако для анализа противоречий необходимо двигаться по штриховой 
стрелке диаграммы (см.\ рис.~2) от проекта к~цели. В~самом деле, любой компонент 
проекта направлен на теоретическое достижение цели. Цель~--- сложный объект, 
поэтому в~проекте могут возникнуть характеристики, противоречащие хотя бы 
некоторым характеристикам цели. Это делает проект противоречивым, но вывод 
об этом может быть сделан только на уровне описания цели. 
  

  Противоречия между проектом и~его реализацией, исключая настройки 
(рис.~3), могут возникать, например, при закупке исполнителем материалов более 
низкого качества по более низким ценам, при попытках достижения требуемых 
сроков работы за счет снижения качества выполнения работ, за счет нахождения 
<<объективных>> причин для увеличения сроков работы и,~следовательно, 
увеличения цены реализации проекта. 


  Для выявления указанных противоречий необходимо двигаться по диаграмме 
(см.\ рис.~3) в~обратную сторону в~соответствии со~штриховыми стрелками. 
Действительно, выявить противоречия между характеристиками закупленных 
материалов и~требуемыми по проекту можно только при обращении к~проекту 
и~его спецификациям. Манипуляции со сроками работы также можно выявить 
только при обращении к~соответствующим расчетам в~проекте. Задержки в~сроках 
работы, связанные с~поставками материалов, можно определить только на 
предыдущем этапе диаграммы (см.\ рис.~3) в~описании проекта. 


  


  Противоречия между реализацией проекта и~его настройкой (рис.~4) возникает, 
когда не удается добиться требуемых значений параметров функционала, не 
удается обеспечить необходимый уровень\linebreak\vspace*{-12pt}

{ \begin{center}  %fig2
 \vspace*{-6pt}
   \mbox{%
 \epsfxsize=16mm 
 \epsfbox{gru-2.eps}
 }


\vspace*{6pt}


\noindent
{{\figurename~2}\ \ \small{Противоречия цели и~проекта}}
\end{center}
}

%\vspace*{9pt}

\addtocounter{figure}{1}

{ \begin{center}  %fig3
 \vspace*{6pt}
    \mbox{%
 \epsfxsize=79mm 
 \epsfbox{gru-3.eps}
 }


\end{center}

\vspace*{-2pt}


\noindent
{{\figurename~3}\ \ \small{Противоречия проекта и~его реализации (без настройки)}}
}

\vspace*{6pt}

\addtocounter{figure}{1}

{ \begin{center}  %fig4
 \vspace*{1pt}
   \mbox{%
 \epsfxsize=54.5mm 
 \epsfbox{gru-4.eps}
 }


\end{center}


\noindent
{{\figurename~4}\ \ \small{Противоречия реализации проекта и~его на\-стройки}}
}

%\vspace*{9pt}

\addtocounter{figure}{1}

{ \begin{center}  %fig5
 \vspace*{5pt}
    \mbox{%
 \epsfxsize=79mm 
 \epsfbox{gru-5.eps}
 }


\end{center}



\noindent
{{\figurename~5}\ \ \small{Противоречия цели и~достигнутой реализации проекта}}
}

\vspace*{6pt}

\addtocounter{figure}{1}

\noindent
 качества реализации проекта. Для 
определения противоречия в~настройках надо опять же двигаться по диаграмме 
(см.\ рис.~4) в~обратную сторону по штриховым стрелкам, так как для выявления 
характеристик результатов работы, которые не дают возможности реализации 
определенного функционала, необходимо иметь информацию о результатах этой 
работы. 


  



  Противоречие между целью и~достигнутой реализацией проекта (рис.~5) 
возникает, когда реализованная система не позволяет достичь цели. В~этом случае 
опять противоречие нужно искать, двигаясь от цели к~реальному достигнутому 
функционалу по штриховой стрелке (см.\ рис.~5).
  
  Суммируя положения, изложенные в~данном разделе, приходим к~выводу, что 
для выявления противоречий необходимо проводить анализ от следствия 
к~причине, т.\,е.\ искать аномалии в~информации, описывающей порождение 
наблюдаемых следствий. 
  
  
  \section{Связь противоречий и~причин}
  
  Прежде чем построить связь между причинами и~противоречиями, кратко 
опишем простейшую модель связи этих понятий. Причины и~противоречия будут 
сформулированы для представления компонентов системы как объектов, 
обладающих наборами известных характеристик~\cite{4-gr, 5-gr}. 
  
  Пусть $U\hm=\{\alpha, \beta, \ldots\}$~--- совокупность характеристик 
(пространство характеристик). Согласно~\cite{4-gr} \textit{объектом}~$O$ 
называется любое подмножество характеристик $O\hm\subseteq U$. Рассмотрим 
последовательность объектов, возможно в~различных пространствах 
характеристик. 
  
  \smallskip
  
  \noindent
  \textbf{Определение~1.}\ Объект~$P$ с~числом характеристик, большим или 
равным~2, является \textit{причиной} объекта (\textit{свойства})~$B$ в~цепочке 
наблюдаемых объектов тогда и~только тогда, когда выполнены следующие 
условия:
  \begin{enumerate}[(1)]
\item для каждого объекта~$C$, если $P\hm\subseteq C$, то $C\mapsto B$, где 
$C\mapsto B$ означает, что объект~$B$ присутствует в~объекте, следующем за 
объектом~$C$;
\item объект~$P$ является минимальным объектом, удовлетворяющим 
условию~1, а~именно: $\forall \alpha\hm\in P$ объект~$P\backslash \{\alpha\}$ 
не является причиной, т.\,е.\ $\exists C:\ \alpha\not\in C$, $P\backslash 
\{\alpha\}\hm\subseteq C$ и~$C\not\mapsto B$, где $C\not\mapsto B$ означает, 
что~$B$ не может содержаться в~объекте, следующем за объектом~$C$. 
\end{enumerate}

  Приведенное определение причины является упрощением причин, 
возникающих в~реальном мире. Например, реальные причины могут возникать\linebreak 
как совокупность характеристик из разных пространств. Одно следствие может 
порождаться разными причинами или возникать из внешних\linebreak и~ненаблюдаемых 
характеристик. Однако пред\-став\-лен\-ная далее формализация позволяет доступно 
изложить при\-чин\-но-след\-ст\-вен\-ные истоки противоречий, которые 
инициируют в~дальнейшем глубокое исследование рассматриваемых процессов.
  
  Будем считать, что для любого интересующего нас свойства~$B$ существует 
причина. Тогда справедлива следующая теорема.
  
  \smallskip
  
  \noindent
  \textbf{Теорема~1.}\ \textit{Для любого свойства~$B$ существует 
единственная причина}. 
  
  \smallskip
  
  \noindent
  Д\,о\,к\,а\,з\,а\,т\,е\,л\,ь\,с\,т\,в\,о\,.\ \ Доказательство будем вести от противного, 
т.\,е.\ предположим, что существуют две причины свойства~$B$: $P$ 
и~$P^\prime$, $P\hm\not= P^\prime$. Тогда существует $\alpha\hm\in U$, которое 
удовлетворяет одному из двух условий:
  \begin{itemize}
\item[(а)] $\alpha\in P$, $\alpha\notin P^\prime$;
\item[(б)] $\alpha\notin P$, $\alpha \in P^\prime$.
\end{itemize}

  Пусть выполняется условие~(б). Тогда $P^\prime\backslash \{\alpha\}$ не 
является причиной по условию~2 определения~1, т.\,е.\ $\exists C$ такое, что 
$\alpha\notin C$, $P^\prime\backslash \{\alpha\}\hm\subseteq C$ и~$C\not\mapsto B$. 
Но если~$B$ произошло и~$P$ его причина, то $C\mapsto B$, что противоречит 
предположению. Теорема~1 доказана.
  
  \smallskip
  
  \noindent
  \textbf{Лемма.} \textit{Если $P$~--- причина появления свойства~$B$, то 
объект~$B$ определяет существование свойства~$P$ в~объекте, 
предшествующем~$B$. }
  
  \smallskip
  
  \noindent
  Д\,о\,к\,а\,з\,а\,т\,е\,л\,ь\,с\,т\,в\,о\,.\ \ Из предположения, что у~каж\-до\-го 
свойства~$B$ есть причина, и~условия, что~$P$ является причиной~$B$, следует, 
что при появлении в~данных свойства~$B$ объект~$C$, предшествующий 
появлению~$B$, содержит как часть объект~$P$. Это следует из теоремы~1 
и~определения причины. 
  
  Докажем принцип <<необходимого условия>>, который, несмотря на простоту 
доказательства, будет играть в~дальнейшем существенную роль.
  
  \smallskip
  
  \noindent
  \textbf{Теорема~2.} \textit{Если~$P$~--- причина появления свойства~$B$ 
и~$A\hm\subseteq P$, то объект~$B$ определяет наличие свойства~$A$ 
в~объекте, предшествующем~$B$}. 
  
  \smallskip
  
  \noindent
  Д\,о\,к\,а\,з\,а\,т\,е\,л\,ь\,с\,т\,в\,о\,.\ \ Пусть в~данных имеется объект~$B$ 
и~$P\mapsto B$, тогда в~силу существования и~единственности причины~$B$ 
в~данных должен существовать объект~$C$, предшествующий~$B$ 
и~содержащий причину~$P$. Поскольку $A\hm\subseteq P$ и~$B$ содержит 
причину~$P$, то $B\mapsto A$. С~учетом леммы теорема~2 доказана.
  
  \smallskip
  
  Пусть даны пространства $U_1, U_2,\ldots$ и~имеется последовательность 
данных (процесс выполнения этапов проекта в~соответствии с~рис.~1) $A, B, 
\ldots$, где каждый объект является подмножеством некоторого 
пространства~$U_i$, $i\hm=1,\ldots$ Тогда в~объекте~$A$ присутствует 
причина~$P$ появления интересующего нас свойства~$C$ в~объекте~$B$. Пусть 
$P\hm\subseteq A$, тогда по теореме~2 $\forall \alpha\hm\in P$:  
$C\mapsto \{\alpha\}$, т.\,е.\ из появления~$C$ следует появление 
характеристики~$\alpha$ в~предшествующем объекте. Это необходимое условие 
того, что~$C$ удовлетворяет причинно-следственным связям развития процесса 
выполнения проекта. Если для~$C$ нет характеристики~$\alpha$, которую можно 
отнести к~причине~$C$, то можно считать, что нарушена  
при\-чин\-но-след\-ст\-вен\-ная связь и~$C$~--- аномальный объект. 
  
  \smallskip
  
  \noindent
  \textbf{Пример.} Если объект~$C$ состоит в~получении суммы~$a$ 
фирмой~$K$, то согласно теореме~2 в~пред\-шест\-ву\-ющем объекте должна 
существовать причина перевода суммы~$a$ на фирму~$K$. Если эта причина 
в~проекте отсутствует, то это можно считать признаком мошеннической схемы. 
Все проекты по предположению собираются из <<кубиков>>, содержащихся в~БЗ. 
Тогда можно сравнить цену объекта~$C$, породившего получение суммы~$a$, 
и~сумму, присутствующую в~смете проекта. Если разница велика, то это либо 
ошибка проекта, либо признак мошеннической схемы.
  
  \section{Поиск противоречий на~основе~принципа <<необходимых~условий>>}
   
  Как было показано в~разд.~3, нахождение противоречий соответствуют 
движению от следствия к~причине. Для каждого объекта в~наблюдаемых данных 
выявление причин его появления является трудоемкой задачей. Кроме того, при 
реализации контроля соблюдения при\-чин\-но-след\-ст\-вен\-ных связей на 
большом множестве участников экономической деятельности задача анализа 
причин становится трудоемкой. Поэтому процедуру контроля необходимо разбить 
на два этапа, где первый этап состоит в~анализе простых <<необходимых 
условий>> проявления мошенничества, когда используется хотя бы одна 
известная характеристика причины. Второй этап (в~режиме офлайн) состоит 
в~выявлении причин, позволяющих провести анализ источников мошеннических 
схем. 
  
  Один из подходов к~выбору <<необходимых условий>> состоит в~построении 
множества подцелей исходной цели проекта (структурный метод построения 
проекта~\cite{7-gr}). Каждая подцель описывается диаграммой на рис.~1, 
и~реализации подцелей должны образовывать полный функционал цели. Это 
является необходимым, но не достаточным условием достижения цели, так как 
при таком подходе отсутствует компонент согласования всех подцелей в~единую 
систему. Однако такой подход значительно упрощает анализ выполнения проекта 
на предмет поиска мошенничества. Если признаки мошенничества будут 
обнаружены в~реализации хотя бы одной из подцелей, то это значит, что 
мошенничество присутствует в~реализации всего проекта. 
  
  Аналогично в~реализации каждого этапа в~любой из подцелей можно выделять 
простые <<необходимые условия>> нарушения при\-чин\-но-след\-ст\-венн\-ых 
связей. 
  
  Таким образом, получается множество <<необходимых условий>>, нарушение 
которых свидетельствует о наличии мошенничества. Это множество 
<<необходимых условий>> можно назвать метаданными~[8, 9] для контроля 
проекта на выявление мошенничества. 
  
  
  \section{Заключение }
  
  В поиске противоречий необходимо от транзакций, соответствующих 
следствиям при\-чин\-но-след\-ст\-вен\-ных связей, переходить к~анализу причин 
наблюдаемых следствий. Это сложная задача, которая связана с~описанием причин 
определенных свойств. 
  
  В работе представлена модель, позволяющая строить множество необходимых 
условий соответствия наблюдаемого следствия вызвавшей его причине. Этот 
подход делает поиск противоречий вполне вычислимой задачей, но не гарантирует 
успех. 
  
  {\small\frenchspacing
 {%\baselineskip=10.8pt
 \addcontentsline{toc}{section}{References}
 \begin{thebibliography}{9}
\bibitem{1-gr}
\Au{Грушо А.\,А., Зацаринный~А.\,А., Тимонина~Е.\,Е.} Блокчейны цифровой экономики на базе 
системы ситуационных центров и~централизованного консенсуса~// Радиолокация, навигация, 
связь: Мат-лы XXV Междунар. научн.-технич. конф.~---
Воронеж: Издательский дом ВГУ, 2019. Т.~6. С.~183--191. 
\bibitem{2-gr}
\Au{Grusho A., Zatsarinny~A., Timonina~E.} A~system approach to information security in 
distributed ledgers on the situational centers platform.~---
Lecture notes in computer science ser.~--- Springer, 2019 
(in press).
\bibitem{3-gr}
\Au{Финн В.\,К.} Искусственный интеллект: Методология, применения, философия.~--- М.: 
Красанд, 2011. 448~с.

\bibitem{5-gr} %4
\Au{Аншаков~О.\,М., Фабрикантова~Е.\,Ф.} ДСМ-ме\-тод автоматического порождения 
гипотез: Логические и~эпистемологические основания.~--- М.: Либроком, 2009. 432~с.

\bibitem{4-gr} %5
\Au{Poelmans J., Elzinga~P., Viaene~S., Dedene~G.} Formal concept analysis in knowledge 
discovery: A~survey~// Conceptual structures: From information to intelligence~/ Eds.\ M.~Croitoru, 
S.~Ferr$\acute{\mbox{e}}$, and D.~Lukose.~--- Lecture notes in computer science 
ser.~--- Berlin--Heidelberg: Springer, 2010. Vol.~6208.  P.~139--153.

\bibitem{6-gr}
\Au{Панкратова~Е.\,С., Финн~В.\,К.} Автоматическое по\-рож\-де\-ние гипотез в~интеллектуальных 
системах.~--- М.: Либроком, 2009. 528~с. 
\bibitem{7-gr}
\Au{Денисов А.\,А., Колесников~Д.\,Н.} Теория больших систем управления.~--- Л.: Энергоиздат, 1982. 488~с.

\bibitem{9-gr}
\Au{Грушо А.\,А., Грушо Н.\,А., Забежайло~М.\,И., Смирнов~Д.\,В., Тимонина~Е.\,Е.} 
Параметризация в~прикладных задачах поиска эмпирических причин~// Информатика и~её 
применения, 2018. Т.~12. Вып.~3. С.~62--66.

\bibitem{8-gr}
\Au{Грушо А.\,А., Грушо Н.\,А., Левыкин~М.\,В., Тимонина~Е.\,Е.} Методы идентификации 
захвата хоста в~распределенной ин\-фор\-ма\-ци\-он\-но-вы\-чис\-ли\-тель\-ной сис\-те\-ме, 
защищенной с~помощью метаданных~// Информатика и~её применения, 2018. Т.~12. Вып.~4. 
С.~41--45.

 \end{thebibliography}

 }
 }

\end{multicols}

\vspace*{-3pt}

\hfill{\small\textit{Поступила в~редакцию 03.04.19}}

%\vspace*{8pt}

%\pagebreak

\newpage

\vspace*{-28pt}

%\hrule

%\vspace*{2pt}

%\hrule

%\vspace*{-2pt}

\def\tit{ARCHITECTURAL DECISIONS IN~THE~PROBLEM 
OF~IDENTIFICATION OF~FRAUD IN~THE~ANALYSIS 
OF~INFORMATION FLOWS IN~DIGITAL ECONOMY\\[-5pt]}


\def\titkol{Architectural decisions in~the~problem 
of~identification of~fraud in~the~analysis 
of~information flows in~digital economy}

\def\aut{A.\,A.~Grusho, M.\,I.~Zabezhailo, N.\,A.~Grusho, and~E.\,E.~Timonina}

\def\autkol{A.\,A.~Grusho, M.\,I.~Zabezhailo, N.\,A.~Grusho, and~E.\,E.~Timonina}

\titel{\tit}{\aut}{\autkol}{\titkol}

\vspace*{-13pt}


 \noindent
   Institute of Informatics Problems, Federal Research Center ``Computer Sciences and 
Control'' of the Russian Academy of Sciences; 44-2~Vavilov Str., Moscow 119133, 
Russian Federation

\def\leftfootline{\small{\textbf{\thepage}
\hfill INFORMATIKA I EE PRIMENENIYA~--- INFORMATICS AND
APPLICATIONS\ \ \ 2019\ \ \ volume~13\ \ \ issue\ 2}
}%
 \def\rightfootline{\small{INFORMATIKA I EE PRIMENENIYA~---
INFORMATICS AND APPLICATIONS\ \ \ 2019\ \ \ volume~13\ \ \ issue\ 2
\hfill \textbf{\thepage}}}

\vspace*{3pt}


   
     
   \Abste{An approach to a~research of some types of fraud in digital economy with the usage of relationships of 
cause and effect is formulated. In all types of the considered frauds, the discrepancy between the 
purposes of financial transactions and actual cost of achievement of these purposes
has to be observed. Data on 
transactions can be collected by observing information flows in which these transactions are reflected. 
The architecture of data collection and their analysis can be organized by means of the distributed 
ledgers with the centralized consensus that allows creating an analog of the electronic account book 
fixing financial and economic activity of subjects of digital economy in the region. 
   The methods of fraud identification considered are based on the contradictions 
between actions described in transactions and information, which is contained in plans, standards, 
precedents, etc. 
   The method based on a~simplified scheme of implementation of the abstract project is considered. 
For identification of contradictions, it is necessary to carry out the analysis from the effect to the cause, 
i.\,e., to look for anomalies in information describing the generation of the observed effects. 
   It is shown how in implementation of the project it is possible to allocate simple ``necessary 
conditions'' of violation of cause and effect relationships, i.\,e., a~set of ``necessary conditions'' 
violation of which demonstrates fraud existence. It is possible to call this set of "necessary conditions" 
by metadata for control of the project for fraud identification.} 
   
   \KWE{digital economy; information flows; relationships of reason and effect; detection of 
fraudulent schemes}
   
  

 \DOI{10.14357/19922264190204}

\vspace*{-20pt}

 \Ack
   \noindent
   The work was partially supported by the Russian Foundation for Basic Research (projects  
18-29-03081 and 18-07-00274).



%\vspace*{6pt}

  \begin{multicols}{2}

\renewcommand{\bibname}{\protect\rmfamily References}
%\renewcommand{\bibname}{\large\protect\rm References}

{\small\frenchspacing
 {\baselineskip=10.5pt
 \addcontentsline{toc}{section}{References}
 \begin{thebibliography}{9}
\bibitem{1-gr-1}
\Aue{Grusho, A.\,A., A.\,A.~Zatsarinny, and E.\,E.~Timonina.} 2019. Blokcheyny tsifrovoy ekonomiki 
na baze sistemy situatsionnykh tsentrov i~tsentralizovannogo konsensusa [Blockchains of digital 
economy on the basis of the system of the situational centres and the centralized consensus]. 
\textit{25th Scientific and Technical Conference (International) ``Radar-Location, Navigation, 
Communication'' Proceedings}. Voronezh: VSU Publs. 6:183--191.
\bibitem{2-gr-1}
\Aue{Grusho, A., A.~Zatsarinny, and E.~Timonina.} 2019 (in press). 
A~system approach to information security 
in distributed ledgers on the situational centers platform. 
Lecture notes in computer science ser. Springer.
\bibitem{3-gr-1}
\Aue{Finn, V.\,K.} 2011. \textit{Iskusstvennyy intellekt: Metodologiya, primeneniya, filosofiya} 
[Artificial intelligence: Methodology, applications, philosophy]. Moscow: KRASAND. 448~p.

\bibitem{5-gr-1}
\Aue{Anshakov, O.\,M., and E.\,F.~Fabrikantova}. 2009. \textit{DSM-metod avtomaticheskogo porozhdeniya gipotez: Logicheskie 
i~epistemologicheskie osnovaniya} [JSM-method of automatic hypothesis generation: Logical and 
epistemological]. Moscow: KD LIBROKOM. 432~p.
\bibitem{4-gr-1} %5
\Aue{Poelmans, J., P.~Elzinga, S.~Viaene, and G.~Dedene.} 2010. Formal concept analysis in 
knowledge discovery: A~survey. \textit{Conceptual structures: From information to intelligence}. 
Eds.\ M.~Croitoru, S.~Ferr$\acute{\mbox{e}}$, and D.~Lukose. Lecture notes in 
computer science ser. Berlin--Heidelberg: Springer. 6208:139--153.

\bibitem{6-gr-1}
\Aue{Pankratov, E.\,S., and V.\,K.~Finn}. 
2009. \textit{Avtomaticheskoe porozhdenie gipotez v~intellektual'nykh 
sistemakh} [Automatic hypotheses generation in intelligent systems]. Moscow: KD 
\mbox{LIBROKOM}.  528~p. 
\bibitem{7-gr-1}
\Aue{Denisov, A.\,A., and D.\,N.~Kolesnikov.} 1982. \textit{Teoriya bol'shikh 
sistem upravleniya} [Theory of big control systems]. Leningrad: Energoizdat. 488~p.

\bibitem{9-gr-1}
\Aue{Grusho, A.\,A., N.\,A.~Grusho, M.\,I.~Zabezhailo, D.\,V.~Smirnov, and 
E.\,E.~Timonina.} 2018. 
Parametrizatsiya v~prikladnykh zadachakh poiska empiricheskikh prichin 
[Parametrization in applied 
problems of search of the empirical reasons]. 
\textit{Informatika i~ee Primeneniya~--- 
Inform. Appl.} 12(3):62--66.

\bibitem{8-gr-1}
\Aue{Grusho, A.\,A., N.\,A.~Grusho, M.\,V.~Levykin, and E.\,E.~Timonina.} 2018. Metody 
identifikatsii zakhvata khosta v~raspredelennoy informatsionno-vychislitel'noy sisteme, 
zashchishchennoy s~pomoshch'yu metadannykh [Methods of identification of host capture 
in the  distributed information system which is protected on the base of meta data].
\textit{Informatika i~ee 
Primeneniya~--- Inform. Appl.} 12(4):41--45.
{ %\looseness=1

}

\end{thebibliography}

 }
 }

\end{multicols}

\vspace*{-12pt}

\hfill{\small\textit{Received April 3, 2019}}

%\pagebreak

%\vspace*{-18pt}

\Contr

\noindent
\textbf{Grusho Alexander A.} (b.\ 1946)~--- Doctor of Science in physics and 
mathematics, professor, principal scientist, Institute of Informatics Problems, 
Federal Research Center ``Computer Sciences and Control'' of the Russian 
Academy of Sciences; 44-2~Vavilov Str., Moscow 119133, Russian Federation; 
\mbox{grusho@yandex.ru} 

\vspace*{3pt}

\noindent
\textbf{Zabezhailo Michael I.} (b.\ 1956)~--- Doctor of Science in physics and 
mathematics, principal scientist, Institute of Informatics Problems, Federal Research 
Center ``Computer Sciences and Control'' of the Russian Academy of Sciences;  
44-2~Vavilov Str., Moscow 119133, Russian Federation; 
\mbox{m.zabezhailo@yandex.ru} 

\vspace*{3pt}


\noindent
\textbf{Grusho Nikolai A.} (b.\ 1982)~--- Candidate of Science (PhD) in physics 
and mathematics, senior scientist, Institute of Informatics Problems, Federal 
Research Center ``Computer Sciences and Control'' of the Russian Academy of 
Sciences; 44-2~Vavilov Str., Moscow 119133, Russian Federation; 
\mbox{info@itake.ru} 

\vspace*{3pt}


\noindent
\textbf{Timonina Elena E.} (b.\ 1952)~--- Doctor of Science in technology, 
professor, leading scientist, Institute of Informatics Problems, Federal Research 
Center ``Computer Sciences and Control'' of the Russian Academy of Sciences;  
44-2~Vavilov Str., Moscow 119133, Russian Federation; 
\mbox{eltimon@yandex.ru} 

\label{end\stat}

\renewcommand{\bibname}{\protect\rm Литература}       %13
\def\stat{dubanov}

\def\tit{КИНЕМАТИЧЕСКИЕ МОДЕЛИ ЗАДАЧ ПРЕСЛЕДОВАНИЯ НА~ПЛОСКОСТИ МЕТОДАМИ 
ПАРАЛЛЕЛЬНОГО СБЛИЖЕНИЯ И~ПОГОНИ}

\def\titkol{Кинематические модели задач преследования на~плоскости методами 
параллельного сближения и~погони}

\def\aut{А.\,А.~Дубанов$^1$, В.\,А.~Нефедова$^2$}

\def\autkol{А.\,А.~Дубанов, В.\,А.~Нефедова}

\titel{\tit}{\aut}{\autkol}{\titkol}

\index{Дубанов А.\,А.}
\index{Нефедова В.\,А.}
\index{Dubanov A.\,A.}
\index{Nefedova V.\,A.}


%{\renewcommand{\thefootnote}{\fnsymbol{footnote}} \footnotetext[1]
%{Работа выполнена при поддержке Министерства науки и~высшего образования Российской Федерации (проект 
%075-15-2020-799).}}


\renewcommand{\thefootnote}{\arabic{footnote}}
\footnotetext[1]{Бурятский государственный университет, alandubanov@mail.ru}
\footnotetext[2]{Бурятский государственный университет, emelyanovatorik@gmail.com}


\vspace*{-12pt}



  

    \Abst{Рассматриваются модели задачи преследования на плоскости методами 
параллельного сближения и погони. Цель работы~--- модификация методов параллельного 
сближения и погони на случаи, когда в момент начала преследования вектор скорости 
преследователя направлен не в точку на окружности Аполлония в методе параллельного 
сближения и не на цель в методе погони. В рассматриваемых моделях преследователь не 
может мгновенно изменять направление движения. Было наложено условие, что радиус 
кривизны траектории движения преследователя не может быть меньше определенной 
величины. Предлагаемый метод основан на том, что преследователь, выбирая шаг на этапе 
итераций, будет стараться следовать прогнозируемым траекториям. По материалам статьи 
написаны тестовые программы, которые рассчитывают траектории преследователя, 
учитывая изложенные условия. Выполненные анимированные изображения визуализируют 
изменение координат преследователя, цели и прогнозируемых траекторий во времени.}
    
    \KW{цель; преследователь; траектория; сближение; моделирование}
    
  \DOI{10.14357/19922264220314}  
  
\vspace*{-3pt}


\vskip 10pt plus 9pt minus 6pt

\thispagestyle{headings}

\begin{multicols}{2}

\label{st\stat}
    
\section{Введение}

   При расчете траекторий преследователя в задачах преследования 
в~пространстве и на плос\-кости используются методы параллельного сбли\-же\-ния 
и~погони. В~описании задачи \mbox{преследования} методом параллельного 
сближения в трудах Л.\,О.~Пет\-ро\-ся\-на~[1, 2] на\-прав\-ле\-ние вектора ско\-рости 
в~точке на\-хож\-де\-ния преследователя~$P_i$ и на\-прав\-ле\-ние вектора ско\-рости 
в~точке на\-хож\-де\-ния цели~$T_i$ пересекаются в точке~$K_i$, принадлежащей 
окружности Аполлония (рис.~1), соответствующей данному моменту времени.

Для точек~$P$ и~$T$ точка~$K$ окружности Аполлония\linebreak характерна тем, что 
отношение длин
  $\vert PK\vert/\vert QK\vert \hm= \vert V_P\vert/\vert V_T\vert$ есть отношение 
модулей скоростей преследователя и цели. При дискретном моделировании 
точек траектории преследователя $\{P_i\}$
\mbox{можно} предложить сле\-ду\-ющую 
итерационную схему (рис.~2):



\end{multicols}
   
   \begin{figure*}[h] %fig1
   \vspace*{-12pt}
  \begin{center}  
    \mbox{%
\epsfxsize=99.151mm
\epsfbox{dub-1.eps}
}

\end{center}
\vspace*{-9pt}
   \Caption{Итерационная схема метода параллельного сближения}
   \end{figure*}
   


   \begin{multicols}{2}
   
   { \begin{center}  %fig2
 \vspace*{-3pt}
     \mbox{%
\epsfxsize=60.497mm
\epsfbox{dub-2.eps}
}

\vspace*{6pt}


\noindent
{{\figurename~2}\ \ \small{
Итерационная схема метода погони
}}
\end{center}}

\vspace*{8pt}

\addtocounter{figure}{1}
   
  
   
  \noindent
      $$
  P_i= P_{i-1} +V_P \Delta T \fr{K_{i-1} -P_{i-1}}{\vert K_{i-1} -P_{i-1}\vert}\,.
  $$
    Радиусы окружностей Аполлония определяются выражением
  $$
  R_i= \fr{V_T^2}{V_P^2-V_T^2}\left\vert T_i - P_i\right\vert\,.
  $$
    Центры окружностей Аполлония рассчитываются по формуле:
  $$
  Q_i= T_i +\fr{V_i^2}{V_P^2 -V_T^2}\left( T_i -P_i\right).
  $$
  
  Координаты точки~$K_i$ получаются в результате решения системы 
уравнений относительно непрерывного параметра~$t$:
  \begin{align*}
  \left(K_i-Q_i\right)^2 &=R_i^2\,;\\
  K_i&=T_i +V_T \fr{T_{i+1} -T_i}{\vert T_{i+1} -T_i\vert}\,t\,.
  \end{align*}
  
  Такова одна из моделей построения траектории преследователя в методе 
параллельного сближения на плоскости. При этом требуется, чтобы 
на\-прав\-ле\-ния векторов движения пре\-сле\-до\-ва\-те\-ля и~цели пересекались в~точках, 
при\-над\-ле\-жа\-щих окруж\-но\-стям Аполлония. 
  
  Для метода погони в задаче преследования на плоскости с постоянной 
скоростью характерно то, что вектор скорости преследователя точно совпадает 
с направлением на цель. В~такой постановке задача имеет как непрерывную 
модель для решения, так и квазидискретную. В~непрерывной модели решение 
основывается на численном или аналитическом решении системы 
дифференциальных уравнений:
  \begin{gather}
  \left(T_x(t)\!-\!P_x(t)\right) \fr{dP_y(t)}{dt} \!=\!\left( T_y(t)\!-\!P_y(t)\right) 
\fr{dP_x(t)}{dt};
\label{e1-dub}\\
  \left( \fr{dP_x(t)}{dt}\right)^2 \!+\!\left( \fr{dP_y(t)}{dt}\right)^2 =V^2,
  \label{e2-dub}
  \end{gather}
где $\begin{bmatrix} P_x \\ P_y \end{bmatrix}$~--- координаты преследователя; 
$\begin{bmatrix} T_x\\ T_y \end{bmatrix}$~--- координаты цели; $V$~--- 
скорость цели.

  Уравнение~(1) системы говорит о том, что вектор скорости 
преследователя сонаправлен с~вектором линии, соединяющей преследователя 
и~цель. Уравнение~(2) сис\-те\-мы говорит о~постоянстве модуля ско\-рости.
  
 




  В качестве простого примера квазидискретной модели можно привести одну 
из итерационных схем, которая вычисляет координаты точек следующего 
положения преследователя (см.\ рис.~2): 
$$
P_{i-1}= P_i+ V_P \Delta t \left (T_i-  P_i\right),
$$
 где $V_P$~--- скорость преследователя; $\Delta t$~--- период 
дискретизации. Как видно из постановки задачи преследования методом погони 
с постоянной скоростью, вектор скорости преследователя должен быть всегда 
направлен на цель, даже в момент начала преследования. В~обеих описанных 
моделях задается начальное направление вектора скорости преследователя. 
Цель данной статьи~--- представить модифицированные модели преследования, 
в~которых направление вектора скорости преследователя произвольно.

\vspace*{-2pt}
   
\section{Кинематическая модель метода параллельного 
сближения}

\vspace*{-2pt}

   Итерационную схему, пред\-став\-лен\-ную на рис.~1, можно интерпретировать 
иначе. На рис.~3 пред\-став\-лен итерационный процесс, определяющий 
координаты точки~$P_i$ при известных координатах точек $P_{i-1}$, $T_{i-1}$ и~$T_i$ и~скоростей преследователя и цели~$V_P$ и~$V_T$ 
соответственно.
   
  Cначала определяется единичный вектор 
  $$
  \vec{\tau} =\fr{T_{i-1} -P_{i-1}}{\left\vert T_{i-1} -P_{i-1}\right\vert}\,.
  $$
   Координаты точки $T_i\hm= T_{i-1}\hm+ \vec{V}_T \Delta T$, где $\Delta 
T$~--- период дискретизации, предопределены поведением цели. Значит, 
прямую линию, которая будет соединять точки~$P_i$ и~$T_i$, можно 
представить в виде 
$$
L(\mu) = T_i+ \mu\vec{\tau}\,. 
$$

\setcounter{figure}{3}
\begin{figure*} %fig4
\vspace*{1pt}
  \begin{center}  
    \mbox{%
\epsfxsize=126.516mm
\epsfbox{dub-4.eps}
}

\end{center}
\vspace*{-6pt}
\Caption{Квазидискретная модель параллельного сближения}
\end{figure*}


{ \begin{center}  %fig3
 \vspace*{3pt}
     \mbox{%
\epsfxsize=66.187mm
\epsfbox{dub-3.eps}
}

\end{center}

\vspace*{-3pt}

\noindent
{{\figurename~3}\ \ \small{
Интерпретация итерационной схемы метода параллельного сближения
}}}

%\vspace*{6pt}

\addtocounter{figure}{1}

\pagebreak

\noindent
Тогда 
координаты~$P_i$ сле\-ду\-юще\-го шага итераций траектории преследователя 
можно интерпретировать как точку пересечения окруж\-ности радиуса $V_P 
\Delta T$ с центром в точке~$P_{i-1}$ и прямой линии~$L(\mu)$:
  \begin{align*}
  \left( L(\mu) -P_{i-1}\right)^2 &=\left( V_P \Delta T\right)^2\,;\\
  L(\mu)& =T_i +\mu \vec{\tau}\,.
  \end{align*}
  
  
  
  Решение вышеприведенной системы уравнений относительно 
параметра~$\mu$ даст такое значение параметра, при котором будет 
выполняться сле\-ду\-ющее: 
$$
P_i= T_i+ \mu \vec{\tau}\,. 
$$
  
  Итерационную схему параллельного сближения предлагается 
модифицировать следующим образом. Пусть в момент начала сближения 
вектор скорости преследователя~$P_{i-1}$ направлен произвольным образом, 
но не в точку на окружности Аполлония (рис.~4). В~силу инертности 
преследователя минимальный радиус кривизны траектории не может быть 
меньше определенного значения~$r_c$. Точке преследователя $P_{i-1}$ 
соответствуют вектор скорости~$\vec{V}_{P_{i-1}}$ и вектор единичной 
нормали~$\vec{n}_{i-1}$, $\vec{V}_{P_{i-1}} \vec{n}_{i-1}\hm=0$. Далее 
находится центр окружности радиуса~$r_c$:
  $$
  C_{i-1} =P_{i-1} +V_P \Delta T \vec{n}_{i-1}\,.
  $$
  
  К построенной окружности строится касательная из точки~$T_{i-1}$ для 
нахождения точки~$P_{t_{i-1}}$ сопряжения прямой и окружности (см.\ рис.~4). 
Дугу окружности $\PRDN$ и отрезок $[P_{t_{i-1}} 
T_{i-1}]$ будем считать одной составной кривой линией $f_{i-1}(s)$, где 
па\-ра\-мет\-ром~$s$ служит длина дуги данной параметрической кривой.
  

   
  В предлагаемой тестовой программе отсчет длины дуги начинается от точки 
$T_{i-1}$: $f_{i-1}(0)\hm= T_{i-1}$. Проведем параллельный перенос линии 
$f_{i-1}(s)$ на вектор $T_i\hm- T_{i-1}$. Положение точки~$T_i$ известно 
и~пол\-ностью определяется поведением цели. В~рамках решения данной задачи 
будем считать поведение цели полностью детерминированным. Уравнение 
параллельной линии $f_i(s)\hm= f_{i-1}(s)\hm+ T_i\hm- T_{i-1}$ будем считать 
известным, и для нахождения точки~$P_i$ следующего шага преследователя 
необходимо решение следующей системы уравнений и неравенств 
относительно параметра~$s$ ($0\hm\leq s\hm\leq s_{i-1}$):
  \begin{align*}
  \left( f_i(s)-P_{i-1}\right)^2 &= \left( V_P \Delta T\right)^2\,;\\
  f_i(s)&=f_{i-1}(s)+T_i-T_{i-1}\,,
    \end{align*}
где $s_{i-1}$~--- это значение параметра~$s$, со\-от\-вет\-ст\-ву\-ющее точке~$P_{i-
1}$. Первое, что сделано в~пред\-ла\-га\-емой программе,~--- для со\-став\-ной кривой 
в~момент начала преследования, состоящей из дуги и~отрезка, выполнена 
параметризация от длины дуги. Для этого было необходимо получить 
упорядоченный набор точек $\{x_i, y_i\}$. По каждой координате встроенными 
средствами MathCAD выполнена кубическая сплайн-ин\-тер\-по\-ля\-ция от 
формального параметра~$\delta$ и получены функции $X(\delta)$, $Y(\delta)$, 
$i\hm\in [0,N\hm-1]$, $\delta_i\hm\leq \delta\hm\leq \delta_{i+1}$, где 
$\delta_i\hm= i$; $N$~--- число элементов массивов $\{x_i, y_i\}$.

  Далее был составлен якобиан для передачи во встроенные решатели системы 
MathCAD:
  $$
  D(s,\delta)=\fr{1}{\sqrt{dX^2/d\delta + dY^2/d\delta}}\,.
  $$

  Полученное решение выражает зависимость массивов $\{x_i, y_i\}$ от 
параметра длины дуги~$s$. Таким образом, можно считать, что уравнение 
базовой кривой, с которой будет совершаться параллельный перенос, получено 
(рис.~5). Далее предстоит создать вычислительный цикл, в котором решается 
система уравнений и неравенств. Данная задача сводится к численному 
решению уравнения поиском нулей функции методом секущей в заданном 
диапазоне. 
  
      \setcounter{figure}{5}
\begin{figure*}[b] %fig6
\vspace*{6pt}
  \begin{center}  
    \mbox{%
\epsfxsize=128.8mm
\epsfbox{dub-6.eps}
}

\end{center}
\vspace*{-6pt}
\Caption{Моделирование траектории преследователя}
\end{figure*}  


  Встроенные средства численного решения уравнений системы MathCAD 
позволяют решить уравнение
 
 
 \pagebreak
  
  { \begin{center}  %fig5
 \vspace*{-4pt}
     \mbox{%
\epsfxsize=79mm
\epsfbox{dub-5.eps}
}

\end{center}



\noindent
{{\figurename~5}\ \ \small{
Кинематическая модель параллельного сближения: \textit{1}~--- линии параллельного переноса $Y_p(X_p)$;
\textit{2}~--- окружности числа $\mathrm{CC}^{\langle 1\rangle} (\mathrm{CC}^{\langle 0\rangle})$;
\textit{3}~--- траектория цепи $\mathrm{TT}^{\langle 1\rangle} (\mathrm{TT}^{\langle 0\rangle})$;
\textit{4}~--- траектория преследователя $\mathrm{PP}^{\langle 1\rangle} (\mathrm{PP}^{\langle 0\rangle})$;
\textit{5}~--- базовая линия предполагаемых траекторий $\mathrm{Fr}^{\langle 1\rangle} (\mathrm{Fr}^{\langle 0\rangle})$
}}}

\vspace*{15pt}



 \noindent
 $$
  \left( f_{i-1}(s) +T_i-T_{i-1} -P_{i-1}\right)^2 -\left( V_P \Delta T\right)^2=0
  $$
в диапазоне $s\hm\in [0, s_{i-1}]$. На анимированном изоб\-ра\-же\-нии рис.~5 
представлены результаты моделирования тестовой программы~\cite{11-dub}. 

\section{Кинематическая модель метода погони}


\vspace*{-12pt}



Рассмотрим следующую итерационную схему. Будем считать, что в~момент 
времени~$t_i$ известно положение цели~$T_i$, положение 
преследователя~$P_i$ и~векторное уравнение~$F_i(s)$ прогнозируемой на 
данный момент времени траектории движения преследователя (рис.~6). 
В~таком итерационном процессе ставится задача рассчитать 
координаты~$P_{i+1}$ следующего шага преследователя и выполнить 
аффинные преобразования векторной функции $F_i(s)$, чтобы найти 
выражения для функции $F_{i+1}(s)$. Чтобы найти координаты  
точки~$P_{i+1}$, необходимо решить уравнение
$$
\left\vert F_i\left(s_{i+1}\right) -F_i(s_i)\right\vert =V_P \Delta t
$$
относительно параметра~$s_{i+1}$. Когда разрабатывалась модель 
итерационного процесса, задавались начальные координаты~$T$ и~$P$ 
и~начальный вектор скорости движения преследователя~$V_P$. Также 
определялась векторная функция~$F(s)$ в момент начала преследования. На 
рис.~1 это составная кривая из дуги $\prdn$ и прямолинейного 
сегмента $[P_t T]$, где параметром служит длина дуги этой кривой. На $i$-м 
итерационном шаге происходит сле\-ду\-ющее. Строится окружность с~цент\-ром 
в~точке~$P_i$ радиуса~$V_P\Delta t$. Точка пересечения этой окружности\linebreak\vspace*{-12pt}

{ \begin{center}  %fig7
 \vspace*{-4pt}
    \mbox{%
\epsfxsize=79mm
\epsfbox{dub-7.eps}
}

\end{center}



\noindent
{{\figurename~7}\ \ \small{
Кинематическая модель метода погони: 
\textit{1}~---  точки предполагаемых траекторий $Y_{\mathrm{arr}} (X_{\mathrm{arr}})$;
\textit{2}~--- точки траекторий преследователя и~цели $(\mathrm{T}_2)^{\langle 1\rangle} ((T_2)^{\langle 0\rangle})$;
\textit{3}~--- динамическая предполагаемая траектория $y_f(x_f)$;
\textit{4}~--- динамическая точка преследователя $P_{fY} (P_{\mathrm{fX}})$
}}}

\vspace*{9pt}


\noindent
 и~траектории $F_i(s)$ будет точкой следующего шага преследователя~$P_{i+1}$. 
Потом находится точка пересечения$P^\prime_{i+1}$ параметрической 
функции $F_i(s)$ с~окруж\-ностью с~цент\-ром в~точ\-ке~$T_i$ и радиуса $\vert 
T_{i+1}\hm- P_{i+1}\vert$. Затем формируется локальный базис $(h_1^\prime, 
h_2^\prime)$ с центром координат в точке~$P^\prime_{i+1}$. Пересчитывается 
функция~$F_i(s)$. В~базисе $(h_1^\prime, h_2^\prime)$ она будет выглядеть как 
$F^\prime_i(s)$. Компоненты базиса $(h_1^\prime, h_2^\prime)$ таковы: 
$$
h_1^\prime= \fr{T_i-P^\prime_{i+1}}{\vert T_i -P^\prime_{i+1}\vert}\,,\enskip
h_2^\prime= \begin{bmatrix}
-h_{1_y}^\prime\\ h^\prime_{1_x}
\end{bmatrix}\,.
$$
Функция $F_i^\prime(s)$ будет иметь вид:
$$
F^\prime_i(s)= \begin{bmatrix}
\left( F_i(s)-P^\prime_{i+1}\right) h_1^\prime\\[3pt]
\left( F_i(s) -P^\prime_{i+1}\right) h_2^\prime
\end{bmatrix}\,.
$$
Сформируем базис $(h_1, h_2)$ с центром координат в~точке~$P_{i+1}$. 
Компоненты базиса $(h_1, h_2)$ будут выглядеть как 
$$
h_1= \fr{T_{i+1}-P_{i+1}}{\vert T_{i+1} -P_{i+1}\vert}\,;\enskip
h_2= \begin{bmatrix}
-h_{1_y}\\ h_{1_x}
\end{bmatrix}\,.
$$
Отметим очень важный момент, что $\vert T_{i+1}\hm- P_{i+1}\vert \hm= \vert 
T_i\hm- P^\prime_{i+1}\vert$. Отсюда можно утверждать, что локальное 
представление в локальном базисе $(h_1, h_2)$ с центром в точке~$P_{i+1}$ 
кривой~$F_{i+1}(s)$ будет совпадать с локальным представлением базиса 
$(h_1^\prime, h_2^\prime)$: $F^\prime_{i+1}(s)\hm= F_i^\prime(s)$. Базис $(E_1, 
E_2)$ в базисе $(h_1, h_2)$ выглядит как
$$
e_1= \begin{bmatrix}
E_1  h_1\\
E_1  h_2\end{bmatrix}\,;\enskip
e_2= \begin{bmatrix}
E_2  h_1\\
E_2 h_2\end{bmatrix}\,.
$$
Уравнение линии $F_{i+1}(s)$ будет иметь вид:
$$
F_{i+1}(s) = \begin{bmatrix} F^\prime_{i+1}(s) e_1\\[3pt]
F^\prime_{i+1}(s) e_2\end{bmatrix}
+P_{i+1}\,.
$$



  Итак, на $i$-м шаге итерации имеем сле\-ду\-ющее: шаг~$T_{i+1}$ выбирается 
целью, шаг преследователя~$P_{i+1}$ рассчитывается как точка пересечения 
окружности $(P_i, V_P \Delta t)$ и ранее рассчитанной линии $F_i(s)$, а~на 
основе уже имеющихся данных рассчитывается новая прогнозируемая 
траектория преследователя~$F_i(s)$. Анимированное изображение рис.~7 как 
раз демонстрирует результаты работы программы~[4]. 

\section{Заключение}

   В настоящей статье рассматриваются кинематические модели задачи 
преследования на плос\-кости методами параллельного сбли\-же\-ния и~погони. 
В~момент начала преследования ско\-рость преследователя на\-прав\-ле\-на 
произвольно. Данные методики возможны для использования при разработке 
геометрической модели группового пре\-сле\-до\-ва\-ния с~одновременным 
достижением цели или целей. По предложенным моделям и~алгоритмам 
написаны тес\-то\-вые программы расчета траекторий в~сис\-те\-ме компьютерной 
математики MathCAD. Тексты программ доступны на ресурсе~\cite{13-dub}. При написании статьи за основу приняты теоретические 
результаты, полученные в~\cite{1-dub, 2-dub, 3-dub}. Также приняты во 
внимание результаты  
работ~\cite{4-dub, 5-dub, 6-dub, 7-dub, 8-dub, 9-dub, 10-dub}.
   
{\small\frenchspacing
 {%\baselineskip=10.8pt
 %\addcontentsline{toc}{section}{References}
 \begin{thebibliography}{99}


\bibitem{2-dub} %1
\Au{Петросян Л.\,А., Рихсиев~Б.\,Б.} Преследование на плоскости.~--- М.: Наука, 1991. 96~с. 
%(Попул. лекции по мат.; Вып.~61).

\bibitem{1-dub} %2
\Au{Петросян Л.\,А.} Дифференциальные игры преследования~// Соросовский 
образовательный ж., 1995. №\,1. С.~88--91.

\bibitem{11-dub} %3
Метод параллельного сближения на плос\-кости с~ограничениями на кривизну. {\sf 
https://www.youtube. com/watch?v=qNXdykK21Z8}.
\bibitem{12-dub} %4
Метод погони на плос\-кости с~ограничениями на кривизну. {\sf 
https://www.youtube.com/watch?v=UQ5 bVKjVqZ4}.
\bibitem{13-dub} %5
Программный код в~системе MathCAD. {\sf http:// dubanov.exponenta.ru/books.htm}.

\bibitem{3-dub} %6
\Au{Айзекс Р.} Дифференциальные игры~/ Пер. с~англ.~--- М.: Мир, 1967. 480~с.
(\Au{Isaacs~R.} {Differential games: A~mathematical theory with applications to warfare and pursuit, control and optimization}.~---
Dover books on mathematics ser.~--- Wiley, 1965. 384~p.)

\bibitem{10-dub} %7
\Au{Ibragimov G., Hussin~N.\,A.} A~Pursuit-evasion differential game with many pursuers and one 
evader~// Malaysian J.~Mathematical Sciences, 2010. Vol.~4. Iss.~2. P.~183--194.

\bibitem{5-dub} %8
\Au{Кузьмина Л.\,И., Осипов~Ю.\,В.} Расчет длины траектории для задачи преследования~// 
Вестник МГСУ, 2013. №\,12. С.~20--26.
\bibitem{6-dub} %9
\Au{Саматов Б.\,Т.} Задача преследования убегания при  
ин\-тег\-раль\-но-гео\-мет\-ри\-че\-ских ограничениях на управ\-ле\-ния преследователя~// 
Автоматика и телемеханика, 2013. Вып.~7. С.~17--28.

\bibitem{8-dub} %10
\Au{Ibragimov G., Norshakila~A.\,R., Kuchkarov~A., Ismail~F.} Multi pursuer differential game of 
optimal approach with integral constraints on controls of players~// Taiwan. J.~Math., 
2015. Vol.~19. Iss.~3. P.~963--976.
\bibitem{9-dub} %11
\Au{Petrov N.\,N., Solov'eva~N.\,A.} Group pursuit with phase constraints in recurrent Pontryagin's 
example~// Int. J.~Pure Applied Mathematics, 2015. Vol.~100. Iss.~2. P.~263--278.

\bibitem{7-dub} %12
\Au{Романников Д.\,О.} Пример решения минимаксной задачи преследования 
с~использованием нейронных сетей~// Сборник научных трудов НГТУ, 2018. №\,2(92). 
С.~108--116. doi: 10.17212/2307-6879-2018-2-108-116.
\bibitem{4-dub} %13
\Au{Ахметжанов А.\,Р.} Динамические игры преследования на поверхностях: Автореф. дис.\ 
\ldots\ канд. физ.-мат. наук.~--- М.: МФТИ, 2019. 28~с.


\end{thebibliography}

 }
 }

\end{multicols}

\vspace*{-6pt}

\hfill{\small\textit{Поступила в~редакцию 18.08.20}}

\vspace*{8pt}

%\pagebreak

%\newpage

%\vspace*{-28pt}

\hrule

\vspace*{2pt}

\hrule

%\vspace*{-2pt}

\def\tit{KINEMATIC MODELS OF~PURSUIT PROBLEMS ON~THE~PLANE BY~THE~METHODS 
OF~PARALLEL APPROACH AND~PURSUIT}


\def\titkol{Kinematic models of~pursuit problems on~the~plane by~the~methods 
of~parallel approach and~pursuit}


\def\aut{A.\,A.~Dubanov and V.\,A.~Nefedova}

\def\autkol{A.\,A.~Dubanov and V.\,A.~Nefedova}

\titel{\tit}{\aut}{\autkol}{\titkol}

\vspace*{-8pt}


\noindent
Banzarov Buryat State University, 6a~Ranzhurov Str., Ulan-Ude 670000, Russian Federation


\def\leftfootline{\small{\textbf{\thepage}
\hfill INFORMATIKA I EE PRIMENENIYA~--- INFORMATICS AND
APPLICATIONS\ \ \ 2022\ \ \ volume~16\ \ \ issue\ 3}
}%
 \def\rightfootline{\small{INFORMATIKA I EE PRIMENENIYA~---
INFORMATICS AND APPLICATIONS\ \ \ 2022\ \ \ volume~16\ \ \ issue\ 3
\hfill \textbf{\thepage}}}

\vspace*{3pt} 




\Abste{This article provides accurate pursuit models based on the parallel approach and 
chase methods. This article is a modification of the methods of parallel rapprochement and 
chasing what happens when the pursuit begins. It cannot instantly change the direction of 
movement. This should be a less affordable option. The proposed method is based on the fact 
that the pursuer chooses a step at the iteration stage and will try to follow the predicted 
trajectories. Based on the materials of the article, test programs have been written that 
calculate the trajectories of the pursuer taking into account the stated conditions. Execution 
of animated images visualizes the change in the coordinates of the pursuer, target, and 
predicted time trajectories.}

\KWE{target; pursuer; trajectory; convergence; modeling}

  \DOI{10.14357/19922264220314}  

%\vspace*{-16pt}

%\Ack
%\noindent




\vspace*{6pt}

  \begin{multicols}{2}

\renewcommand{\bibname}{\protect\rmfamily References}
%\renewcommand{\bibname}{\large\protect\rm References}

{\small\frenchspacing
 { %\baselineskip=12pt
 \addcontentsline{toc}{section}{References}
 \begin{thebibliography}{99}
 
 \vspace*{-2pt}

\bibitem{2-dub-1} %1
\Aue{Petrosyan, L.\,A., and B.\,B.~Rihsiev.} 1961. \textit{Presledovanie na ploskosti} [Flat 
pursuit]. Moscow: Nauka. 96~p.

\bibitem{1-dub-1} %2
\Aue{Petrosyan, L.\,A.} 1995. Differentsial'nye igry presledovaniya [Differential pursuit 
games]. \textit{Sorosovskiy obrazovatel'nyy zh.} [Soros Educational~J.] 1:88--91.

\bibitem{11-dub-1}  %3
Metod parallel'nogo sblizheniya na ploskosti s~og\-ra\-ni\-che\-ni\-yami na kriviznu [Method of 
parallel approach on a~plane with restrictions on curvature]. Available at: {\sf 
https://www.youtube.com/watch?v=qNXdykK21Z8} (accessed May~25, 2022).
\bibitem{12-dub-1} %4
Metod pogoni na ploskosti s ogranicheniyami na kriviznu [Plane chase method with curvature 
constraints]. Available  at:  {\sf https://www.youtube.com/watch?v=UQ5 bVKjVqZ4} 
(accessed May~25, 2022).
\bibitem{13-dub-1} %5
Programmnyy kod v~sisteme MathCAD [Program code in the MathCAD system]. Available 
at: {\sf http://\linebreak dubanov.exponenta.ru/books.htm} (accessed May~25, 2022).
\bibitem{3-dub-1} %6
\Aue{Isaacs, R.} 1965.
\textit{Differential games: A~mathematical theory with applications to warfare and pursuit, control and optimization}.
Dover books on mathematics ser. Wiley. 384~p.
{\looseness=1

}

\bibitem{10-dub-1} %7
\Aue{Ibragimov, G.\,A., and N.\,A.~Hussin.} 2010. Pursuit-evasion differential game with 
many pursuers and one evader. \textit{Malaysian J.~Mathematical Sciences} 4(2):183--194.

\bibitem{5-dub-1} %8
\Aue{Kuzmina, L.\,I., and Y.\,Y.~Osipov.} 2013. Raschet dliny traektorii dlya zadachi 
presledovaniya [Calculation of the path length in the pursuit problem]. \textit{Vestnik MGSU} 
[Bulletin of MGSU] 12:20--26.
\bibitem{6-dub-1} %9
\Aue{Samatov, B.\,T.} 2013. The pursuit-evasion problem under integral-geometric 
constraints on pursuer controls. \textit{Automat. Rem. Contr.} 74:1072--1081.

\bibitem{8-dub-1} %10
\Aue{Ibragimov, G., A.\,R.~Norshakila, A.~Kuchkarov, and F.~Ismail.} 2015. Multi pursuer 
differential game of optimal approach with integral constraints on controls of players. 
\textit{Taiwan. J.~Math.} 19 (3):963--976.
\bibitem{9-dub-1} %11
\Aue{Petrov, N.\,N., and N.\,A.~Solov'eva.} 2015. Group pursuit with phase constraints in 
recurrent Pontryagin's example. \textit{Int. J.~Pure Applied Mathematics} 100(2):263--278.

\pagebreak 

\bibitem{7-dub-1} %12
\Aue{Romannikov, D.\,O.} 2018. Primer resheniya mi\-ni\-maks\-noy zadachi presledovaniya 
s~ispol'zovaniem ney\-ron\-nykh se\-tey [An example of solving a minimax pursuit problem using 
neural networks]. \textit{Sbornik nauchnykh trudov NGTU} [Transactions of scientific papers of the Novosibirsk State 
Technical University] 2(92):108--116. doi: 10.17212/2307-6879-2018-2-108-116.

\vspace*{-2pt}

\bibitem{4-dub-1} %13
\Aue{Ahmetzhanov, A.\,R.} 2019. Dinamicheskie igry presledovaniya na poverkhnostyakh 
[Dynamic pursuit games on surfaces].  Moscow: MIPT.  PhD Thesis. 28~p.
\end{thebibliography}

 }
 }

\end{multicols}

\vspace*{-6pt}

\hfill{\small\textit{Received August 18, 2020}}

\Contr


\noindent
\textbf{Dubanov Alexander A.} (b.\ 1967)~--- Candidate of Science (PhD) in technology, 
associate professor, Department of Geometry and Methods of Teaching Mathematics, 
Institute of Mathematics and Informatics, Banzarov Buryat State University, 6a~Ranzhurov 
Str., Ulan-Ude 670000, Russian Federation; \mbox{alandubanov@mail.ru}

\vspace*{3pt}

\noindent
\textbf{Nefedova Victoria A.} (b.\ 2000)~--- student, Institute of Mathematics and 
Informatics, Banzarov Buryat State University, 6a~Ranzhurov Str., Ulan-Ude 670000, 
Russian Federation; \mbox{emelyanovatorik@gmail.com}



\label{end\stat}

\renewcommand{\bibname}{\protect\rm Литература}        %14



%%%%%%%%%%%%%%%%%%%%%%%%%%%%%%%%%%%%%%%%

%\def\stat{rez}
{%\hrule\par
%\vskip 7pt % 7pt
\raggedleft\Large \bf%\baselineskip=3.2ex
Р\,Е\,Ц\,Е\,Н\,З\,И\,И \vskip 17pt
    \hrule
    \par
\vskip 6pt plus 6pt minus 3pt }

%\thispagestyle{headings} %с верхним колонтитулом
%\thispagestyle{myheadings} %с нижним колонтитулом, но в верхнем РЕЦЕНЗИИ

\def\tit{НОВАЯ КНИГА И.\,Н.~СИНИЦЫНА, А.\,С.~ШАЛАМОВА <<ЛЕКЦИИ ПО ТЕОРИИ 
ИНТЕГРИРОВАННОЙ ЛОГИСТИЧЕСКОЙ ПОДДЕРЖКИ>> (М.: ТОРУС ПРЕСС, 2012. 624~с.)}

%1
\def\aut{Д.ф.-м.н., профессор С.\,Я.~Шоргин}

\def\auf{\ }

\def\leftkol{\ % РЕЦЕНЗИИ
}

\def\rightkol{ \ } 

%\def\leftkol{\ } % ENGLISH ABSTRACTS}

%\def\rightkol{\ } %ENGLISH ABSTRACTS}

%\def\leftkol{РЕЦЕНЗИИ}

%\def\rightkol{РЕЦЕНЗИИ}

\titele{\tit}{\aut}{\auf}{\leftkol}{\rightkol}
\vspace*{-18pt}


     \label{st\stat}

     \begin{multicols}{2}
     {\small
     {\baselineskip=10.1pt
     

      В книге представлено системное изложение теоретических основ одного из новейших 
направлений в \mbox{об\-ласти} экономики послепродажного обслуживания изделий наукоемкой 
продукции (ИНП) длительного пользования~--- интегрированной логистической поддержки
(ИЛП). 
{\looseness=1

}

Приведены также результаты новых работ, выполненных в Институте проблем информатики 
Российской академии наук в рамках научного направления <<Информационные технологии и 
анализ сложных сис\-тем>>.
 {%\looseness=1

}
     
      Излагаемые в книге научные подходы позво\-ляют карди\-наль\-но реформировать 
существующие системы производства и эксплуатации ИНП путем создания и внед\-ре\-ния 
методов рационального и оптимального управ\-ле\-ния процессами расходования 
вре\-мен\-н$\acute{\mbox{ы}}$х, 
мате\-ри\-аль\-ных, трудовых и других ресурсов на всех стадиях жизненного цикла изделий (ЖЦИ) по 
критериям экономической целесообразности и эф\-фек\-тив\-ности.
  {\looseness=1

}
    
      В книге приведен краткий обзор причин возник\-новения и
      развития CALS-методологии как основы 
современных международных стандартов по созданию и функционированию глобальных 
ин\-фор\-ма\-ци\-он\-но-ком\-му\-ни\-ка\-ци\-он\-ных систем, ее ключевых возможностей и эффективности 
результатов ее использования. 
Авторы %\linebreak 
предлагают ряд научных обоснований для разработки 
единой теории проектирования и управления систем ИЛП для полноценного использования 
преимуществ %\linebreak
 суще\-ст\-ву\-ющей методологии, определяют \mbox{общую} структурную схему 
комплексной системы <<ИНП-СППО>> и необходимость разработки для ее описания 
гибридных стохастических моделей.
{%\looseness=1

}

%\columnbreak
      
      Книга состоит из пяти частей, где последовательно излагается материал по каждой из 
следующих тем: <<Интегрированная логистическая поддержка>>, <<Теория гибридных 
стохастических систем и компьютерная поддержка исследований и разработок>>, <<Основы 
математического моделирования, анализа и синтеза систем послепродажного обслуживания>>, 
<<Определение и анализ показателей экспортного потенциала ИНП при проектировании>>, 
<<Задачи управления поддержкой послепродажного обслуживания>>, а также 
<<Моделирование инвестиционных процессов ИЛП в условиях неравновесных финансовых 
рынков>>. 
   
      В конце каждой главы приведены выводы и даны вопросы и задания для 
самоконтроля. В~приложениях содержатся основные определения по программам работ по 
анализу ИЛП, логистическим базам данных и компьютерным решениям, эквивалентной статистической 
линеаризации нелинейных преобразований ИЛП, справочный материал, а также развернутые 
уравнения для вероятностных характеристик.


      \def\leftkol{РЕЦЕНЗИИ}

\def\rightkol{РЕЦЕНЗИИ} 

      
      Книга заинтересует широкий круг специалистов и может быть использована научными 
проектными организациями в сфере промышленного производства ИНП. Большое количество 
иллюстраций, примеров и вопросов, обращенных к читателю, позволяет использовать книгу 
также в качестве учебного пособия для студентов и аспирантов машиностроительных, 
транспортных и~других специальностей, а также для самостоятельного изучения. 
{%\looseness=-1

}

Книга 
представляет несомненный интерес для специалистов и студентов в области прикладной 
математики и информатики.
    

}

}
\end{multicols}

%\newpage

%\def\stat{popravka}



\def\tit{ПОПРАВКА К СТАТЬЕ О.\,В.~ШЕСТАКОВА 
<<ПОРОГОВЫЕ ФУНКЦИИ В~МЕТОДАХ ПОДАВЛЕНИЯ ШУМА, ОСНОВАННЫХ~НА~ВЕЙВЛЕТ-РАЗЛОЖЕНИИ СИГНАЛА>>\\
(Информатика и её применения, 2021. Т.\ 15.  Вып.\,3. C.\ 51--56)}



\def\titkol{Поправка к статье О.\,В.~Шестакова\\
<<Пороговые функции в~методах подавления шума, основанных
на~вейвлет-разложении сигнала>>}



  \def\aut{\ }

  \def\autkol{\ } 

\titel{\tit}{\aut}{\autkol}{\titkol}

\def\leftfootline{\small{\textbf{\thepage}
\hfill INFORMATIKA I EE PRIMENENIYA~--- INFORMATICS AND
APPLICATIONS\ \ \ 2021\ \ \ volume~15\ \ \ issue\ 4}
}%
 \def\rightfootline{\small{INFORMATIKA I EE PRIMENENIYA~---
INFORMATICS AND APPLICATIONS\ \ \ 2021\ \ \ volume~15\ \ \ issue\ 4
\hfill \textbf{\thepage}}}


 \label{st\stat}

 \thispagestyle{headings}
 
 \vspace*{-24pt}  

\noindent
{\textbf{DOI:} 10.14357/19922264210307}

\vspace*{20pt}

\def\leftfootline{\small{\textbf{\thepage}
\hfill INFORMATIKA I EE PRIMENENIYA~--- INFORMATICS AND
APPLICATIONS\ \ \ 2021\ \ \ volume~15\ \ \ issue\ 4}
}%
 \def\rightfootline{\small{INFORMATIKA I EE PRIMENENIYA~---
INFORMATICS AND APPLICATIONS\ \ \ 2021\ \ \ volume~15\ \ \ issue\ 4
\hfill \textbf{\thepage}}}


%%%%%%%%%

\medskip

\noindent
С.~55, вместо 

\bigskip

\noindent
{\large ANALYSIS OF THE UNBIASED MEAN-SQUARE RISK ESTIMATE\\[6pt]
 OF~THE~BLOCK THRESHOLDING METHOD}

 



\bigskip

\noindent
должно быть

\bigskip

\noindent
{\large THRESHOLDING FUNCTIONS IN~THE~NOISE SUPPRESSION METHODS\\[6pt] 
BASED ON~THE~WAVELET EXPANSION OF~THE~SIGNAL}

 



 
\vskip 10pt plus 9pt minus 6pt

 \thispagestyle{headings}
 
 %\vspace*{-22pt}
  

\label{end\stat}

\renewcommand{\bibname}{\protect\rm Литература} 


\vspace*{8pt}

\hrule

\vspace*{2pt}

\hrule 

\vspace*{12pt}


\def\stat{popravka-1}



\def\tit{ПОПРАВКА К СТАТЬЕ А.\,А.~КУДРЯВЦЕВА, О.\,В.~ШЕСТАКОВА, С.\,Я.~ШОРГИНА
<<МЕТОД ОЦЕНИВАНИЯ ПАРАМЕТРОВ ИЗГИБА, ФОРМЫ И~МАСШТАБА
ГАММА-ЭКСПОНЕНЦИАЛЬНОГО РАСПРЕДЕЛЕНИЯ>>\\
(Информатика и её применения, 2021. Т.\ 15.  Вып.\,3. C.\ 57--62)}



\def\titkol{Поправка к статье А.\,А.~Кудрявцева, О.\,В.~Шестакова, С.\,Я.~Шоргина
<<Метод оценивания параметров изгиба, формы и масштаба
гамма-экспоненциального распределения>>}



  \def\aut{\ }

  \def\autkol{\ } 

\titel{\tit}{\aut}{\autkol}{\titkol}


 \label{st\stat}

 \thispagestyle{headings}
 
 \vspace*{-24pt}  

\noindent
{\textbf{DOI:} 10.14357/19922264210308}

\vspace*{20pt}




%%%%%%%%%

\medskip

\noindent
С.~61, вместо 

\bigskip

\noindent
{\large PROBABILISTIC CHARACTERISTICS OF~BALANCE INDEX
OF~FACTORS\\[6pt] 
WITH~GENERALIZED GAMMA DISTRIBUTION}



 



\bigskip

\noindent
должно быть

\bigskip

\noindent
{\large A METHOD FOR~ESTIMATING BENT, SHAPE AND~SCALE PARAMETERS\\[6pt] 
OF~THE~GAMMA-EXPONENTIAL DISTRIBUTION} 



 



 
\vskip 10pt plus 9pt minus 6pt

 \thispagestyle{headings}
 
 %\vspace*{-22pt}
  

\label{end\stat}

\renewcommand{\bibname}{\protect\rm Литература}  
%\include{popravka-1}

\def\stat{authorsrus}
{%\hrule\par
%\vskip 7pt % 7pt
\raggedleft\Large \bf%\baselineskip=3.2ex
О\,Б\ \ А\,В\,Т\,О\,Р\,А\,Х \vskip 17pt
    \hrule
    \par
\vskip 21pt plus 8pt minus 4pt }


\def\tit{\ }

\def\aut{\ }

\def\auf{\ }

\def\leftkol{\ } % ENGLISH ABSTRACTS}

\def\rightkol{ОБ АВТОРАХ} %ENGLISH ABSTRACTS}

\titele{\tit}{\aut}{\auf}{\leftkol}{\rightkol}
      
            \label{st\stat}



\vspace*{24pt}

\begin{multicols}{2}




\noindent
\textbf{Архипов Олег Петрович} (р.\ 1948)~---
кандидат технических наук, директор Орловского филиала Института проб\-лем информатики
Российской академии наук
%302025, г.Орел, Московское шоссе, д.137

\vspace*{3pt}

\noindent
\textbf{Бирюкова Татьяна Константиновна} (р.\ 1968)~---
кандидат фи\-зи\-ко-ма\-те\-ма\-ти\-че\-ских наук, старший научный сотрудник Института проб\-лем информатики
Российской академии наук

\vspace*{3pt}

\noindent 
\textbf{Бобков  Сергей Геннадьевич} (р.\ 1955)~---
доктор технических наук,  заведующий отделением На\-уч\-но-ис\-сле\-до\-ва\-тель\-ско\-го 
института системных исследований Российской академии наук
%117218, Москва, Нахимовский просп., 36, к.1 

\vspace*{3pt}

\noindent \textbf{Васильев Николай Семенович} (р.\ 1952)~--- доктор 
фи\-зи\-ко-ма\-те\-ма\-ти\-че\-ских наук, профессор, 
МГТУ им.\ Н.\,Э.~Баумана 
%, Москва 105005, 2-я Бауманская ул., д.~5,

\vspace*{3pt}

\noindent
\textbf{Гершкович Максим Михайлович} (р.\ 1968)~---
старший научный сотрудник Института проб\-лем информатики
Российской академии наук

\vspace*{3pt}

\noindent 
\textbf{Дьяченко Юрий Георгиевич} (р.\ 1958)~--- кандидат технических наук, 
старший научный сотрудник Института проб\-лем информатики
Российской академии наук

\vspace*{3pt}

\noindent 
\textbf{Ерошенко Александр Андреевич} (р.\ 1989)~--- аспирант кафедры 
математической статистики факультета вычисли\-тельной математики и кибернетики 
Московского государственного университета им.\ М.\,В.~Ломоносова
%119991, Москва ГСП-1, Ленинские горы, д.\ 1, стр. 52

\vspace*{3pt}
 
\noindent 
\textbf{Захаров Виктор Николаевич} (р.\ 1948)~--- 
доктор технических наук, доцент, ученый секретарь Института проб\-лем информатики
Российской академии наук

\vspace*{3pt}

\noindent
\textbf{Зейфман Александр Израилевич} (р.\ 1954)~---
доктор фи\-зи\-ко-ма\-те\-ма\-ти\-че\-ских наук, профессор, 
заведующий кафедрой Вологодского государственного университета; 
старший научный сотрудник Института проб\-лем информатики
Российской академии наук; главный научный сотрудник ИСЭРТ Российской академии наук

\vspace*{3pt}

\noindent
\textbf{Зыкин Сергей Владимирович} (р.\ 1959)~--- 
доктор технических наук, профессор, заведующий лабораторией Института математики 
им.\ С.\,Л.~Соболева Сибирского отделения Российской академии наук, Новосибирск 
%630090, пр.\ ак.\ Коптюга, 4 

\vspace*{4pt}

\noindent
\textbf{Киреев Владимир Иванович} (р.\ 1938)~---
доктор фи\-зи\-ко-ма\-те\-ма\-ти\-че\-ских наук, профессор Московского 
государственного горного университета
%Адрес: Россия, 119991, г. Москва, Ленинский проспект, д. 6

%\columnbreak

\vspace*{4pt}

\noindent
\textbf{Козеренко Елена Борисовна} (р.\ 1959)~---
кандидат филологических наук, заведующая лабораторией Института проб\-лем информатики
Российской академии наук

\vspace*{4pt}

\noindent
\textbf{Королев Виктор Юрьевич} (р.\ 1954)~--- доктор
фи\-зи\-ко-ма\-те\-ма\-ти\-че\-ских наук, профессор кафедры математической 
статистики факультета вычисли\-тельной математики и кибернетики 
Московского государственного университета; 
ведущий научный сотрудник Института проб\-лем информатики
Российской академии наук

\vspace*{4pt}

\noindent
\textbf{Коротышева Анна Владимировна} (р.\ 1988)~---
старший преподаватель Вологодского государственного университета

\vspace*{4pt}

\noindent 
\textbf{Кун Де Турк} (р.\ 1981)~--- научный сотрудник 
исследовательской группы SMACS факультета телекоммуникаций и обработки информации
Университета Гента, Бельгия
%В-9000 Гент, Бельгия

\vspace*{4pt}

\noindent
\textbf{Лупенцов Олег Сергеевич} (р.\ 1986)~---
аспирант Омского государственного института сервиса
%Омск 644043, ул.\ Певцова 13

\vspace*{4pt}

\noindent
\textbf{Лучко Олег Николаевич} (р.\ 1961)~---
кандидат педагогических наук, профессор, заведующий кафедрой 
Омского государственного института сервиса
%Омск 644043, ул.\ Певцова 13

\vspace*{4pt}

\noindent
\textbf{Малашенко Юрий Евгеньевич} (р.\ 1946)~---
доктор фи\-зи\-ко-ма\-те\-ма\-ти\-че\-ских наук, заведующий сектором 
Вычислительного центра им.\ А.\,А.~Дородницына Российской академии наук
%Адрес: 119333, Москва, ул. Вавилова, 40,

\vspace*{4pt}

\noindent
\textbf{Маньяков Юрий Анатольевич} (р.\ 1984)~---
кандидат технических наук, научный сотрудник Орловского филиала Института проб\-лем информатики
Российской академии наук
%302025, г.Орел, Московское шоссе, д.137

\vspace*{4pt}

\noindent
\textbf{Маренко Валентина Афанасьевна} (р.\ 1951)~---
кандидат технических наук, доцент, старший научный сотрудник 
Института математики им.\ С.\,Л.~Соболева Сибирского отделения Российской академии наук
%Новосибирск 630090, пр. ак. Коптюга, 4 

\vspace*{3pt}

\noindent 
\textbf{Морозов Евсей Викторович} (р.\ 1947)~--- доктор 
фи\-зи\-ко-ма\-те\-ма\-ти\-че\-ских, профессор, ведущий научный сотрудник 
Института прикладных математических исследований Карельского научного центра Российской
академии наук; 
%%185910 Россия, Республика Карелия, г.\ Петрозаводск, ул.\ Пушкинская, 11
профессор Петрозаводского государственного университета, Петрозаводск
%185910 Россия, Республика Карелия, г.\ Петрозаводск, пр.\ Ленина, 33

%\pagebreak

\vspace*{3pt}

\noindent
\textbf{Назарова Ирина Александровна} (р.\ 1966)~---
кандидат фи\-зи\-ко-ма\-те\-ма\-ти\-че\-ских наук, 
научный сотрудник Вычислительного центра им.\ А.\,А.~Дородницына Российской академии наук 
%Адрес: 119333, Москва, ул. Вавилова, 40

\vspace*{3pt}

\noindent
\textbf{Павлов Игорь Валерианович} (р.\ 1945)~--- 
доктор фи\-зи\-ко-ма\-те\-ма\-ти\-че\-ских наук, профессор МГТУ им.\ Н.\,Э.~Баумана 
%Москва 105005, 2-я Бауманская ул., д.~5 

%\pagebreak

\vspace*{3pt}

\noindent 
\textbf{Потахина Любовь Викторовна} (р.\ 1989)~--- аспирантка
Института прикладных математических исследований Карельского научного центра
Российской академии наук; 
%%185910 Россия, Республика Карелия, г.\ Петрозаводск, ул.\ Пушкинская, 11
инженер Петрозаводского государственного университета, Петрозаводск
%185910 Россия, Республика Карелия, г.\ Петрозаводск, пр.\ Ленина, 33

\vspace*{3pt}

\noindent 
\textbf{Рождественский Юрий Владимирович} (р.\ 1952)~--- 
кандидат технических наук, заведующий сектором Института проб\-лем информатики
Российской академии наук

\vspace*{3pt}

\noindent 
\textbf{Синицын Игорь Николаевич} (р.\ 1940)~--- доктор технических наук,
профессор, заслуженный деятель\linebreak\vspace*{-12pt}

\columnbreak

\noindent
 науки РФ, заведующий отделом Института проб\-лем информатики
Российской академии наук

\vspace*{7pt}


\noindent
\textbf{Сиротинин Денис Олегович} (р.\ 1984)~---
кандидат технических наук, научный сотрудник Орловского филиала Института проб\-лем информатики
Российской академии наук
%302025, г.Орел, Московское шоссе, д.137

\vspace*{7pt}

%\columnbreak

\noindent 
\textbf{Соколов  Игорь Анатольевич} (р.\ 1954)~--- академик (действительный член) Российской 
академии наук, доктор технических наук, директор Института проб\-лем информатики
Российской академии наук

\vspace*{7pt}

\noindent
\textbf{Степченков Юрий Афанасьевич} (р.\ 1951)~---
кандидат технических наук, заведующий отделом Института проб\-лем информатики
Российской академии наук

\vspace*{7pt}

\noindent
\textbf{Сурков Алексей Викторович} (р.\ 1978)~--- 
старший научный сотрудник На\-уч\-но-ис\-сле\-до\-ва\-тель\-ско\-го 
института системных исследований Российской академии наук
%117218, Москва, Нахимовский просп., 36, к.1 

\vspace*{7pt}

\noindent 
\textbf{Шестаков Олег Владимирович} (р.\ 1976)~--- доктор 
фи\-зи\-ко-ма\-те\-ма\-ти\-че\-ских, доцент кафедры математической статистики 
факультета вычисли\-тельной математики и кибернетики Московского 
государственного университета им.\ М.\,В.~Ломоносова; 
%119991, Москва ГСП-1, Ленинские горы, д.\ 1, стр. 52
старший научный сотрудник Института проб\-лем информатики
Российской академии наук
%, Москва 119333, ул. Вавилова, д.~44, корп.~2

\vspace*{7pt}

\noindent 
\textbf{Шоргин Сергей Яковлевич} (р.\ 1952.)~--- доктор
фи\-зи\-ко-ма\-те\-ма\-ти\-че\-ских наук, профессор, заместитель директора Института 
проб\-лем информатики Российской академии наук





%%%%%%%%%%%%%%%%%%%%%%%%%%%%%%%%%%%%%%%%%%%%%%%%%%%%%%%%%%%%%%%%%%%%%%%%%%%%%%%




%\def\rightkol{ОБ АВТОРАХ}
%\def\leftkol{ОБ АВТОРАХ}

 \label{end\stat}





%\def\leftfootline{\small{\textbf{\thepage}
%\hfill ИНФОРМАТИКА И ЕЁ ПРИМЕНЕНИЯ\ \ \ том~7\ \ \ выпуск~1\ \ \ 2013}
%}%
% \def\rightfootline{\small{ИНФОРМАТИКА И ЕЁ ПРИМЕНЕНИЯ\ \ \ том~7\ \ \ выпуск~1\ \ \ 2013
%\hfill \textbf{\thepage}}}


%\thispagestyle{myheadings}



\end{multicols}

\newpage  

%\def\stat{cont}
{%\hrule\par
%\vskip 7pt % 7pt
\raggedleft\Large \bf%\baselineskip=3.2ex
А\,В\,Т\,О\,Р\,С\,К\,И\,Й\ \ У\,К\,А\,З\,А\,Т\,Е\,Л\,Ь\ \ З\,А\ \ 2\,0\,0\,7 г. \vskip 17pt
    \hrule
    \par
\vskip 21pt plus 6pt minus 3pt }

\label{st\stat}

\def\tit{\ }

\def\aut{\ }
\def\auf{\ }

\def\leftkol{\ } % ENGLISH ABSTRACTS}

\def\rightkol{\ } %ENGLISH ABSTRACTS}

\titele{\tit}{\aut}{\auf}{\leftkol}{\rightkol}


\contentsline {chapter}{\ }{Выпуск \quad Стр.} 
\contentsline {section}{\textbf{Батракова Д.\,А., Королев В.\,Ю., Шоргин С.\,Я.}\ \ Новый метод вероятностно-ста\-ти\-сти\-че\-ско\-го анализа информационных потоков в\nobreakspace {}телекоммуникационных сетях}{\qquad 1 \qquad 40} 
\contentsline {section}{\textbf{Борисов А.\,В.}\ \ Байесовское оценивание в системах наблюдения с\nobreakspace {}марковскими скачкообразными процессами: игровой подход}{\qquad 2 \qquad 65}
\contentsline {section}{\textbf{Босов А.\,В., Иванов А.\,В.}\ \ Программная инфраструктура информационного Web-пор\-тала}{\qquad 2 \qquad 50}
\contentsline {section}{\textbf{Захаров В.\,Н., Калиниченко Л.\,А., Соколов И.\,А., Ступников С.\,А.}\ \ Конструирование канонических информационных моделей для интегрированных информационных систем}{\qquad 2 \qquad 15}
\contentsline {section}{\textbf{Захаров В.\,Н., Козмидиади В.\,А.}\ \ Средства обеспечения отказоустойчивости при\-ло\-жений}{\qquad 1 \qquad 14} 
\contentsline {section}{\textbf{Иванов А.\,В.}\ \ см. Босов А.\,В.\hfill\hfill\hfill\hfill\hfill\hfill\hfill\hfill\hfill\hfill\hfill\hfill\hfill\hfill\hfill\hfill\hfill\hfill\hfill\hfill\hfill\hfill\hfill\hfill\hfill\hfill\hfill\hfill\hfill\hfill\hfill\hfill\hfill\hfill\hfill}{\ }
\contentsline {section}{\textbf{Ильин В.\,Д., Соколов И.\,А.}\ \ Символьная модель системы знаний информатики в\nobreakspace {}че\-ло\-ве\-ко-автоматной среде}{\qquad 1 \qquad 66} 
\contentsline {section}{\textbf{Калиниченко Л.\,А.}\ \ см. Захаров В.\,Н.\hfill\hfill\hfill\hfill\hfill\hfill\hfill\hfill\hfill\hfill\hfill\hfill\hfill\hfill\hfill\hfill\hfill\hfill\hfill\hfill\hfill\hfill\hfill\hfill\hfill\hfill\hfill\hfill\hfill\hfill\hfill\hfill\hfill\hfill\hfill}{\ }
\contentsline {section}{\textbf{Козеренко Е.\,Б.}\ \ Лингвистическое моделирование для систем машинного перевода и обработки знаний}{\qquad 1 \qquad 54} 
\contentsline {section}{\textbf{Козмидиади В.\,А.}\ \ см. Захаров В.\,Н.\hfill\hfill\hfill\hfill\hfill\hfill\hfill\hfill\hfill\hfill\hfill\hfill\hfill\hfill\hfill\hfill\hfill\hfill\hfill\hfill\hfill\hfill\hfill\hfill\hfill\hfill\hfill\hfill\hfill\hfill\hfill\hfill\hfill\hfill\hfill }{\ } 
\contentsline {section}{\textbf{Королев В.\,Ю.}\ \ см. Батракова Д.\,А.\hfill\hfill\hfill\hfill\hfill\hfill\hfill\hfill\hfill\hfill\hfill\hfill\hfill\hfill\hfill\hfill\hfill\hfill\hfill\hfill\hfill\hfill\hfill\hfill\hfill\hfill\hfill\hfill\hfill\hfill\hfill\hfill\hfill\hfill\hfill}{\ } 
\contentsline {section}{\textbf{Кудрявцев А.\,А., Шоргин С.\,Я.}\ \ Байесовский подход к\nobreakspace {}анализу систем массового обслуживания и\nobreakspace {}показателей надежности}{\qquad 2 \qquad 76}
\contentsline {section}{\textbf{Печинкин А.\,В., Соколов И.\,А., Чаплыгин В.\,В.}\ \ Многолинейная система массового обслуживания с конечным накопителем и ненадежными приборами}{\qquad 1 \qquad 27} 
\contentsline {section}{\textbf{Печинкин А.\,В., Соколов И.\,А., Чаплыгин В.\,В.}\ \ Стационарные характеристики многолинейной\nobreakspace {}системы массового обслуживания с\nobreakspace {}одновременными отказами приборов}{\qquad 2 \qquad 39}
\contentsline {section}{\textbf{Синицын И.\,Н.}\ \ Корреляционные методы построения аналитических информационных моделей флуктуаций полюса Земли по априорным данным}{\qquad 2 \qquad \hphantom{9}2}
\contentsline {section}{\textbf{Синицын И.\,Н.}\ \ Развитие теории фильтров Пугачева для оперативной обработки информации в стохастических системах}{{\qquad 1 \qquad \hphantom{9}3}} 
\contentsline {section}{\textbf{Соколов И.\,А.}\ \ см. Захаров В.\,Н.\hfill\hfill\hfill\hfill\hfill\hfill\hfill\hfill\hfill\hfill\hfill\hfill\hfill\hfill\hfill\hfill\hfill\hfill\hfill\hfill\hfill\hfill\hfill\hfill\hfill\hfill\hfill\hfill\hfill\hfill\hfill\hfill\hfill\hfill\hfill}{\ }
\contentsline {section}{\textbf{Соколов И.\,А.}\ \ см. Ильин В.\,Д.\hfill\hfill\hfill\hfill\hfill\hfill\hfill\hfill\hfill\hfill\hfill\hfill\hfill\hfill\hfill\hfill\hfill\hfill\hfill\hfill\hfill\hfill\hfill\hfill\hfill\hfill\hfill\hfill\hfill\hfill\hfill\hfill\hfill\hfill\hfill}{\ } 
\contentsline {section}{\textbf{Соколов И.\,А.}\ \ см. Печинкин А.\,В.\hfill\hfill\hfill\hfill\hfill\hfill\hfill\hfill\hfill\hfill\hfill\hfill\hfill\hfill\hfill\hfill\hfill\hfill\hfill\hfill\hfill\hfill\hfill\hfill\hfill\hfill\hfill\hfill\hfill\hfill\hfill\hfill\hfill\hfill\hfill}{\ } 
\contentsline {section}{\textbf{Соколов И.\,А.}\ \ см. Печинкин А.\,В.\hfill\hfill\hfill\hfill\hfill\hfill\hfill\hfill\hfill\hfill\hfill\hfill\hfill\hfill\hfill\hfill\hfill\hfill\hfill\hfill\hfill\hfill\hfill\hfill\hfill\hfill\hfill\hfill\hfill\hfill\hfill\hfill\hfill\hfill\hfill}{\ }
\contentsline {section}{\textbf{Ступников С.\,А.}\ \ см. Захаров В.\,Н.\hfill\hfill\hfill\hfill\hfill\hfill\hfill\hfill\hfill\hfill\hfill\hfill\hfill\hfill\hfill\hfill\hfill\hfill\hfill\hfill\hfill\hfill\hfill\hfill\hfill\hfill\hfill\hfill\hfill\hfill\hfill\hfill\hfill\hfill\hfill}{\ }
\contentsline {section}{\textbf{Чаплыгин В.\,В.}\ \ см. Печинкин А.\,В.\hfill\hfill\hfill\hfill\hfill\hfill\hfill\hfill\hfill\hfill\hfill\hfill\hfill\hfill\hfill\hfill\hfill\hfill\hfill\hfill\hfill\hfill\hfill\hfill\hfill\hfill\hfill\hfill\hfill\hfill\hfill\hfill\hfill\hfill\hfill}{\ } 
\contentsline {section}{\textbf{Чаплыгин В.\,В.}\ \ см. Печинкин А.\,В.\hfill\hfill\hfill\hfill\hfill\hfill\hfill\hfill\hfill\hfill\hfill\hfill\hfill\hfill\hfill\hfill\hfill\hfill\hfill\hfill\hfill\hfill\hfill\hfill\hfill\hfill\hfill\hfill\hfill\hfill\hfill\hfill\hfill\hfill\hfill}{\ }
\contentsline {section}{\textbf{Шоргин С.\,Я.}\ \ см. Батракова Д.\,А.\hfill\hfill\hfill\hfill\hfill\hfill\hfill\hfill\hfill\hfill\hfill\hfill\hfill\hfill\hfill\hfill\hfill\hfill\hfill\hfill\hfill\hfill\hfill\hfill\hfill\hfill\hfill\hfill\hfill\hfill\hfill\hfill\hfill\hfill\hfill}{\ } 
\contentsline {section}{\textbf{Шоргин С.\,Я.}\ \ см. Кудрявцев А.\,А.\hfill\hfill\hfill\hfill\hfill\hfill\hfill\hfill\hfill\hfill\hfill\hfill\hfill\hfill\hfill\hfill\hfill\hfill\hfill\hfill\hfill\hfill\hfill\hfill\hfill\hfill\hfill\hfill\hfill\hfill\hfill\hfill\hfill\hfill\hfill}{\ }
%\thispagestyle{myheadings}
\def\leftfootline{\small{\textbf{\thepage}
\hfill ИНФОРМАТИКА И ЕЁ ПРИМЕНЕНИЯ\ \ \ том~1\ \ \ выпуск~2\ \ \ 2007}
}%
 \def\rightfootline{\small{ИНФОРМАТИКА И ЕЁ ПРИМЕНЕНИЯ\ \ \ том~1\ \ \ выпуск~2\ \ \ 2007
 \hfill \textbf{\thepage}}}
 \label{end\stat} 
                     
%\def\stat{cont-e}
{%\hrule\par
%\vskip 7pt % 7pt
\raggedleft\Large \bf%\baselineskip=3.2ex
2\,0\,0\,7\ \ A\,U\,T\,H\,O\,R\ \ I\,N\,D\,E\,X \vskip 17pt
    \hrule
    \par
\vskip 21pt plus 6pt minus 3pt }

\label{st\stat}

\def\tit{\ }

\def\aut{\ }
\def\auf{\ }

\def\leftkol{\ } % ENGLISH ABSTRACTS}

\def\rightkol{\ } %ENGLISH ABSTRACTS}

\titele{\tit}{\aut}{\auf}{\leftkol}{\rightkol}


\contentsline {chapter}{\ }{Issue \quad Page} 
\contentsline {subsection}{\textbf{Batrakova D.\,A., Korolev V.\,Yu., Shorgin S.\,Ya.}\ \ A New Method for the Probabilistic and Statistical Analysis of Information Flows in Telecommunication Networks}{\qquad 1 \qquad 40} 
\contentsline {subsection}{\textbf{Borisov A.\,V.}\ \ Bayesian Estimation in\nobreakspace {}Observation Systems with\nobreakspace {}Markov Jump Processes: Game-Theoretic Approach}{\qquad 2 \qquad 65} 
\contentsline {subsection}{\textbf{Bosov A.\,V., Ivanov A.\,V.}\ \ Linguistic Simulation for Machine Translation and Knowledge Management Systems}{\qquad 2 \qquad 50} 
\contentsline {subsection}{\textbf{Chaplygin V.\,V.} see Pechinkin A.\,V.\hfill\hfill\hfill\hfill\hfill\hfill\hfill\hfill\hfill\hfill\hfill\hfill\hfill\hfill\hfill\hfill\hfill\hfill\hfill\hfill\hfill\hfill\hfill\hfill\hfill\hfill\hfill\hfill\hfill\hfill\hfill\hfill\hfill\hfill\hfill}{\ }
\contentsline {subsection}{\textbf{Chaplygin V.\,V.} see Pechinkin A.\,V.\hfill\hfill\hfill\hfill\hfill\hfill\hfill\hfill\hfill\hfill\hfill\hfill\hfill\hfill\hfill\hfill\hfill\hfill\hfill\hfill\hfill\hfill\hfill\hfill\hfill\hfill\hfill\hfill\hfill\hfill\hfill\hfill\hfill\hfill\hfill}{\ }
\contentsline {subsection}{\textbf{Ilyin V.\,D., Sokolov I.\,A.}\ \ The Symbol Model of Informatics Knowledge System in Human-Automaton Environment}{\qquad 1 \qquad 66} 
\contentsline {subsection}{\textbf{Ivanov A.\,V.} see Bosov A.\,V.\hfill\hfill\hfill\hfill\hfill\hfill\hfill\hfill\hfill\hfill\hfill\hfill\hfill\hfill\hfill\hfill\hfill\hfill\hfill\hfill\hfill\hfill\hfill\hfill\hfill\hfill\hfill\hfill\hfill\hfill\hfill\hfill\hfill\hfill\hfill}{\ }
\contentsline {subsection}{\textbf{Kalinichenko L.\,A.} see Zakharov V.\,N.\hfill\hfill\hfill\hfill\hfill\hfill\hfill\hfill\hfill\hfill\hfill\hfill\hfill\hfill\hfill\hfill\hfill\hfill\hfill\hfill\hfill\hfill\hfill\hfill\hfill\hfill\hfill\hfill\hfill\hfill\hfill\hfill\hfill\hfill\hfill}{\ }
\contentsline {subsection}{\textbf{Korolev V.\,Yu.} see Batrakova D.\,A.\hfill\hfill\hfill\hfill\hfill\hfill\hfill\hfill\hfill\hfill\hfill\hfill\hfill\hfill\hfill\hfill\hfill\hfill\hfill\hfill\hfill\hfill\hfill\hfill\hfill\hfill\hfill\hfill\hfill\hfill\hfill\hfill\hfill\hfill\hfill}{\ }
\contentsline {subsection}{\textbf{Kozerenko E.\,B.}\ \ Linguistic Simulation for Machine Translation and Knowledge Management Systems}{\qquad 1 \qquad 54} 
\contentsline {subsection}{\textbf{Kozmidiady V.\,A.} see Zakharov V.\,N.\hfill\hfill\hfill\hfill\hfill\hfill\hfill\hfill\hfill\hfill\hfill\hfill\hfill\hfill\hfill\hfill\hfill\hfill\hfill\hfill\hfill\hfill\hfill\hfill\hfill\hfill\hfill\hfill\hfill\hfill\hfill\hfill\hfill\hfill\hfill}{\ }
\contentsline {subsection}{\textbf{Kudryavtsev A.\,A., Shorgin S.\,Ya.}\ \ Bayesian Approach to Queueing Systems and Reliability Characteristics}{\qquad 2 \qquad 76} 
\contentsline {subsection}{\textbf{Pechinkin A.\,V., Sokolov I.\,A., Chaplygin V.\,V.}\ \ Multichannel Queuing System with Finite Buffer and Unreliable Servers}{\qquad 1 \qquad 27} 
\contentsline {subsection}{\textbf{Pechinkin A.\,V., Sokolov I.\,A., Chaplygin V.\,V.}\ \ Stationary Characteristics of a Multichannel Queueing System with\nobreakspace {}Simultaneous Refusals of Servers}{\qquad 2 \qquad 39} 
\contentsline {subsection}{\textbf{Shorgin S.\,Ya.} see Batrakova D.\,A.\hfill\hfill\hfill\hfill\hfill\hfill\hfill\hfill\hfill\hfill\hfill\hfill\hfill\hfill\hfill\hfill\hfill\hfill\hfill\hfill\hfill\hfill\hfill\hfill\hfill\hfill\hfill\hfill\hfill\hfill\hfill\hfill\hfill\hfill\hfill}{\ }
\contentsline {subsection}{\textbf{Shorgin S.\,Ya.} see Kudryavtsev A.\,A.\hfill\hfill\hfill\hfill\hfill\hfill\hfill\hfill\hfill\hfill\hfill\hfill\hfill\hfill\hfill\hfill\hfill\hfill\hfill\hfill\hfill\hfill\hfill\hfill\hfill\hfill\hfill\hfill\hfill\hfill\hfill\hfill\hfill\hfill\hfill}{\ }
\contentsline {subsection}{\textbf{Sinitsyn I.\,N.}\ \ Correlational Methods for Analytical Informational Models of the Earth Pole Fluctuations Design Based on a priori Data}{\qquad 2 \qquad \hphantom{9}2}
\contentsline {subsection}{\textbf{Sinitsyn I.\,N.}\ \ Development of Pugachev Filtering for Stochastic Systems}{\qquad 1 \qquad \hphantom{9}3}
\contentsline {subsection}{\textbf{Sokolov I.\,A.} see Ilyin V.\,D.\hfill\hfill\hfill\hfill\hfill\hfill\hfill\hfill\hfill\hfill\hfill\hfill\hfill\hfill\hfill\hfill\hfill\hfill\hfill\hfill\hfill\hfill\hfill\hfill\hfill\hfill\hfill\hfill\hfill\hfill\hfill\hfill\hfill\hfill\hfill}{\ }
\contentsline {subsection}{\textbf{Sokolov I.\,A.} see Pechinkin A.\,V.\hfill\hfill\hfill\hfill\hfill\hfill\hfill\hfill\hfill\hfill\hfill\hfill\hfill\hfill\hfill\hfill\hfill\hfill\hfill\hfill\hfill\hfill\hfill\hfill\hfill\hfill\hfill\hfill\hfill\hfill\hfill\hfill\hfill\hfill\hfill}{\ }
\contentsline {subsection}{\textbf{Sokolov I.\,A.} see Pechinkin A.\,V.\hfill\hfill\hfill\hfill\hfill\hfill\hfill\hfill\hfill\hfill\hfill\hfill\hfill\hfill\hfill\hfill\hfill\hfill\hfill\hfill\hfill\hfill\hfill\hfill\hfill\hfill\hfill\hfill\hfill\hfill\hfill\hfill\hfill\hfill\hfill}{\ }
\contentsline {subsection}{\textbf{Sokolov I.\,A.} see Zakharov V.\,N.\hfill\hfill\hfill\hfill\hfill\hfill\hfill\hfill\hfill\hfill\hfill\hfill\hfill\hfill\hfill\hfill\hfill\hfill\hfill\hfill\hfill\hfill\hfill\hfill\hfill\hfill\hfill\hfill\hfill\hfill\hfill\hfill\hfill\hfill\hfill}{\ }
\contentsline {subsection}{\textbf{Stupnikov S.\,A.} see Zakharov V.\,N.\hfill\hfill\hfill\hfill\hfill\hfill\hfill\hfill\hfill\hfill\hfill\hfill\hfill\hfill\hfill\hfill\hfill\hfill\hfill\hfill\hfill\hfill\hfill\hfill\hfill\hfill\hfill\hfill\hfill\hfill\hfill\hfill\hfill\hfill\hfill}{\ }
\contentsline {subsection}{\textbf{Zakharov V.\,N., Kalinichenko L.\,A., Sokolov I.\,A., Stupnikov S.\,A.}\ \ Development of Canonical Information Models for Integrated Information Systems}{\qquad 2 \qquad 15} 
\contentsline {subsection}{\textbf{Zakharov V.\,N., Kozmidiady V.\,A.}\ \ Means Providing Applications Fault Tolerance}{\qquad 1 \qquad 14} 
\def\leftfootline{\small{\textbf{\thepage}
\hfill ИНФОРМАТИКА И ЕЁ ПРИМЕНЕНИЯ\ \ \ том~1\ \ \ выпуск~2\ \ \ 2007}
}%
 \def\rightfootline{\small{ИНФОРМАТИКА И ЕЁ ПРИМЕНЕНИЯ\ \ \ том~1\ \ \ выпуск~2\ \ \ 2007
 \hfill \textbf{\thepage}}}
 \label{end\stat} 


%\end{document}

%
\def\stat{rekl}
%\label{preobr}

%\def\tit{АКАДЕМИК ПУГАЧЁВ  ВЛАДИМИР СЕМЁНОВИЧ\\
%25.03.1911--25.03.1998}


%   \vspace*{-48pt}
%   \begin{center}\LARGE
%Академик Пугачёв  Владимир Семёнович\\ (25.03.1911--25.03.1998)
%   \end{center}

   %\vspace*{2.5mm}

   \begin{center}

{\prgsh\LARGE
ЮБИЛЕИ}

\end{center}
%\hrule

\vspace*{6pt}


   \vspace*{8mm}

   \thispagestyle{empty}


%\def\stat{emel}


\section*{К 70-летию заместителя директора ИПИ РАН,\\ члена редколлегии журнала
<<Информатика и её применения>>\\ доктора технических наук В.\,И.~Будзко}

\vspace*{18pt}




          \begin{multicols}{2}

%            \label{st\stat}

\begin{center}
\vspace*{1pt}
\mbox{%
\epsfxsize=78mm
\epsfbox{bud-1.eps}
}
\end{center}

\vspace*{12pt}

      14 августа 2014~г.\ исполнилось 70~лет за\-мес\-ти\-те\-лю директора ИПИ РАН по
научной работе доктору технических наук Владимиру Игоревичу Будзко.

      Владимир Игоревич Будзко родился в г.~Москве. Высшее образование получил на факультете
элект\-рон\-но-вы\-чис\-ли\-тель\-ных устройств в Московском
ин\-же\-нер\-но-фи\-зи\-че\-ском институте
(МИФИ), который он окончил в 1968~г., после чего был на\-прав\-лен для прохождения
службы в одну из войс\-ко\-вых частей, где прошел путь от инженера до первого заместителя
командира войсковой части.

      С приходом В.\,И.~Будзко в ИПИ РАН (2001~г.)\ в институте
сформировалось новое научное на\-прав\-ле\-ние теоретических исследований~--- <<Постро\-ение
ин\-фор\-ма\-ци\-он\-но-те\-ле\-ком\-му\-ни\-ка\-ци\-он\-ных\linebreak сис\-тем
высокой до\-ступ\-ности>>. В~рамках этого
направления выполнен широкий круг фундаментальных исследований по поиску подходов и
определению принципов построения средств обеспечения доступности, конфиденциальности
и целостности современных крупномасштабных
ин\-фор\-ма\-ци\-он\-но-те\-ле\-ком\-му\-ни\-ка\-ци\-он\-ных
сис\-тем (ИТС). Разработаны основные сис\-тем\-но-тех\-ни\-че\-ские принципы и базовые
архитектурные решения построения перспективных для условий России ИТС с
централизованной обработкой и хранением информации, сочетающих в себе свойства
высокой доступности, отказо- и катастрофоустойчивости, информационной защищенности.
Определены принципы, методы и математические основы рационального построения и
оптимизации средств восстановления функционирования центров обработки данных (ЦОД)
после возникновения отказов и катастроф, передачи и хранения данных, обеспечения
информационной безопасности при достижении минимальной совокупной стоимости
владения такими системами. Результаты нашли практическое воплощение при реализации
проектов в интересах ряда отечественных государственных и негосударственных
организаций, таких как Банк России (БР), Внешторгбанк, ОАО <<ГМК <<Норильский Никель>>,
<<Газпром>>, Минэкономразвития России, Правительство Москвы, а также ряд силовых
ведомств.

      Под руководством В.\,И.~Будзко начиная с 2001~г.\ выполнен комплекс
      на\-уч\-но-ис\-сле\-до\-ва\-тель\-ских и
      опыт\-но-кон\-ст\-рук\-тор\-ских работ (свыше 100~проектов),
направленных на развитие электронной информационной технологии БР.
Разработаны концепции развития ИТС БР сначала до 2008~г., а затем до 2013~г., которые
были приняты в качестве основы проведения технической политики. За реализацию проекта
<<Катастрофоустойчивая тер\-ри\-то\-ри\-аль\-но-рас\-пре\-де\-лен\-ная
      ин\-фор\-ма\-ци\-он\-но-те\-ле\-ком\-му\-ни\-ка\-ци\-он\-ная сис\-те\-ма централизованной
обработки банковской информации>> В.\,И.~Будзко удостоен Премии Правительства РФ в
области науки и техники за 2010~г.

      В.\,И.~Будзко возглавлял и возглавляет работы по ряду других прикладных проектов,
связанных с созданием, совершенствованием и развитием крупномасштабных ИТС.

      В.\,И.~Будзко~--- генерал-майор, доктор технических наук, член-кор\-рес\-пон\-дент
Академии криптографии РФ, известный ученый в области информатики и применения
информационных технологий при построении территориально распределенных ИТС
различного назначения. Является автором свыше 250~научных работ, опубликованных в
на\-уч\-но-тех\-ни\-че\-ских и специальных изданиях.

    \thispagestyle{empty}

      В.\,И.~Будзко уделяет большое внимание подготовке научных кадров. Под его
руководством защищено 6~диссертаций на соискание ученой степени кандидата
технических наук. Свыше 30~лет он читает лекции в ИКСИ Академии ФСБ, профессор
кафедры НИЯУ МИФИ. Является членом двух диссертационных советов, главным
редактором журнала <<Системы высокой доступности>> и членом редколлегии журнала
<<Информатика и её применения>>.

      \bigskip

      Редакционный совет и Редакционная коллегия журнала <<Информатика и её
применения>> сердечно поздравляют Владимира Игоревича Будзко с 70-ле\-ти\-ем и желают
крепкого здоровья и новых научных достижений.

\end{multicols}

%%Информатика и её применения
%Том 14 Выпуск 1-4 Год 2020

\def\stat{cont}
{%\hrule\par
%\vskip 7pt % 7pt
\raggedleft\Large \bf%\baselineskip=3.2ex
А\,В\,Т\,О\,Р\,С\,К\,И\,Й\ \ У\,К\,А\,З\,А\,Т\,Е\,Л\,Ь\ \ З\,А\ \ 2\,0\,2\,0 г. \vskip 17pt
 \hrule
 \par
\vskip 21pt plus 6pt minus 3pt }

\label{st\stat}

\def\tit{\ }

\def\aut{\ }
\def\auf{\ }

\def\leftkol{\ } % ENGLISH ABSTRACTS}

\def\rightkol{\ } %АВТОРСКИЙ УКАЗАТЕЛЬ ЗА 2020 г.} %ENGLISH ABSTRACTS}

\titele{\tit}{\aut}{\auf}{\leftkol}{\rightkol}
\addcontentsline{toc}{subsection}{\textrm\textbf Авторский указатель за 2020 г.}

\vspace*{-24pt}

\noindent
{\tabcolsep=3pt
\begin{tabular}{p{397pt}cc}
&\textbf{Вып.} & \textbf{Стр.}\\[6pt]
\Avtors{Абгарян~К.\,К., Гаврилов~Е.\,С.} Интеграционная платформа для многомасштабного моде-\linebreak
\\[-12pt]
\hspace*{23pt}лирования нейроморфных систем&2&104--110\\
\Avtors{Абгарян~К.\,К., Колбин~И.\,С.} Применение многомасштабного подхода и методов анализа\linebreak
\\[-12pt]
\hspace*{23pt}данных для моделирования теплопроводности в слоистых структурах&4&91--99\\
\Avtors{Агаларов~Я.\,М.} Оптимизация емкости основного накопителя в системе массового\linebreak
\\[-12pt]
\hspace*{23pt}обслуживания типа $G/M/1/K$ с дополнительным накопителем&2&72--79\\
\Avtors{Агасандян~Г.\,А.} Вычислительные аспекты применения CC-VaR на совокупности рынков&3&62--70\\
\Avtors{Агеев~К.\,А., Сопин~Э.\,С., Яркина~Н.\,В., Самуйлов~К.\,Е., Шоргин~С.\,Я.} Анализ механизмов\linebreak
\\[-12pt]
\hspace*{23pt}нарезки сети с учетом гарантий для различных типов трафика&3&\hphantom{1}94--100\\
\Avtors{Адамова~К.\,А.} см.\ Шнурков~П.\,В.&&\\
\Avtors{Базилевский~М.\,П.} Многофакторные модели полносвязной линейной регрессии без\linebreak
\\[-12pt]
\hspace*{23pt}ограничений на соотношения дисперсий ошибок переменных&2&92--97\\
\Avtors{Бахтеев~О.\,Ю.} см.\ Грабовой~А.\,В.&&\\
\Avtors{Беленков~В.\,Г.} см.\ Будзко~В.\,И.&&\\
\Avtors{Бетелин~В.\,Б., Кушниренко~А.\,Г., Леонов~А.\,Г.} Основные понятия программирования\linebreak
\\[-12pt]
\hspace*{23pt}в изложении для дошкольников&3&55--61\\
\Avtors{Бетелин~В.\,Б., Кушниренко~А.\,Г., Семенов~А.\,Л., Сопрунов~С.\,Ф.} О цифровой грамотности\linebreak
\\[-12pt]
\hspace*{23pt}и средах ее формирования&4&100--107\\
\Avtors{Борисов~А.\,В.} Численные схемы фильтрации марковских скачкообразных процессов по\linebreak
\\[-12pt]
\hspace*{23pt}дискретизованным наблюдениям II: случай аддитивных шумов&1&17--23\\
\Avtors{Борисов~А.\,В.} Численные схемы фильтрации марковских скачкообразных процессов по\linebreak
\\[-12pt]
\hspace*{23pt}дискретизованным наблюдениям III: случай мультипликативных шумов&2&10--18\\
\Avtors{Босов~А.\,В.} Управление выходом стохастической дифференциальной системы по квад-\linebreak
\\[-12pt]
\hspace*{23pt}ратичному критерию. V. Случай неполной информации о состоянии&2&19--25\\
\Avtors{Босов~А.\,В., Мартюшова~Я.\,Г., Наумов~А.\,В., Сапунова~А.\,П.} Байесовский подход к~по\-стро\-ению индивидуальной траектории пользователя в~системе дистанционного\linebreak
\\[-12pt]
\hspace*{23pt}обучения&3&86--93\\
\Avtors{Босов~А.\,В., Стефанович~А.\,И.} Управление выходом стохастической дифференциальной\linebreak
\\[-12pt]
\hspace*{23pt}системы по квадратичному критерию. IV. Альтернативное численное решение&1&24--30\\
\Avtors{Брюхов~Д.\,О., Ступников~С.\,А., Ковалёв~Д.\,Ю., Шанин~И.\,А.} Нейрофизиология как\linebreak
\\[-12pt]
\hspace*{23pt}предметная область для решения задач с интенсивным использованием данных&1&40--47\\
\Avtors{Будзко~В.\,И., Ядринцев~В.\,В., Соченков~И.\,В., Королёв~В.\,И., Беленков~В.\,Г.} Об одном подходе
 к формированию в условиях высокой неопределенности марке-\linebreak
\\[-12pt]
\hspace*{23pt}ров конфиденциальности в системах интенсивного использования данных&4&69--76\\
\Avtors{Вайсер~К.\,О.} см.\ Потанин~М.\,С.&&\\
\Avtors{Вохминцев~А.\,В., Мельников~А.\,В., Пачганов~C.\,А.} Метод навигации и составления карты в трехмерном пространстве на основе комбинированного решения вариационной\linebreak
\\[-12pt]
\hspace*{23pt}подзадачи точка--точка ICP для аффинных преобразований&1&101--112\\
\Avtors{Гаврилов~Е.\,С.} см.\ Абгарян~К.\,К.&&\\
\Avtors{Гайдамака~Ю.\,В.} см.\  Москалева~Ф.\,А.&&\\
\Avtors{Голембиовский~Д.\,Ю.} см.\ Данилишин~А.\,Р.&&\\
\Avtors{Голембиовский~Д.\,Ю.} см.\ Данилишин~А.\,Р.&&\\
\Avtors{Гончаров~А.\,А., Зацман~И.\,М., Кружков~М.\,Г.} Эволюция классификаций в надкорпусных\linebreak
\\[-12pt]
\hspace*{23pt}базах данных&4&108--116\\
\Avtors{Гончаров~А.\,В., Стрижов~В.\,В.} Выравнивание декартовых произведений упорядоченных\linebreak
\\[-12pt]
\hspace*{23pt}множеств&1&31--39\\
\end{tabular}
}

\pagebreak

\def\leftkol{АВТОРСКИЙ УКАЗАТЕЛЬ ЗА 2020 г.} % ENGLISH ABSTRACTS}

\def\rightkol{АВТОРСКИЙ УКАЗАТЕЛЬ ЗА 2020 г.} %ENGLISH ABSTRACTS}

%\thispagestyle{myheadings}
\def\leftfootline{\small{\textbf{\thepage}
\hfill ИНФОРМАТИКА И ЕЁ ПРИМЕНЕНИЯ\ \ \ том~14\ \ \ выпуск~4\ \ \ 2020}
}%
 \def\rightfootline{\small{ИНФОРМАТИКА И ЕЁ ПРИМЕНЕНИЯ\ \ \ том~14\ \ \ выпуск~4\ \ \ 2020
 \hfill \textbf{\thepage}}}


\noindent
{\tabcolsep=3pt
\begin{tabular}{p{394pt}cc}
&\textbf{Вып.} & \textbf{Стр.}\\[3pt]
\Avtors{Горшенин~А.\,К., Королев~В.\,Ю.} Аппроксимация распределений размеров частиц лунного\linebreak
\\[-12pt]
\hspace*{23pt}реголита на основе метода статистической симуляции выборок&2&50--57\\
\Avtors{Горшенин~А.\,К., Королев~В.\,Ю., Щербинина~А.\,А.} Статистическое оценивание распределений случайных коэффициентов стохастического дифференциального уравнения\linebreak
\\[-12pt]
\hspace*{23pt}Ланжевена&3&\hphantom{1}3--12\\
\Avtors{Горшенин~А.\,К., Кузьмин~В.\,Ю.} Анализ конфигураций LSTM-сетей для построения\linebreak
\\[-12pt]
\hspace*{23pt}среднесрочных векторных прогнозов&1&10--16\\
\Avtors{Грабовой~А.\,В., Бахтеев~О.\,Ю., Стрижов~В.\,В.} Введение отношения порядка на множестве\linebreak
\\[-12pt]
\hspace*{23pt}параметров аппроксимирующих моделей&2&58--65\\
\Avtors{Грушо~А.\,А., Забежайло~М.\,И., Смирнов~Д.\,В., Тимонина~Е.\,Е.} О вероятностных оценках\linebreak
\\[-12pt]
\hspace*{23pt}достоверности эмпирических выводов&4&3--8\\
\Avtors{Грушо~А.\,А., Забежайло~М.\,И., Смирнов~Д.\,В., Тимонина~Е.\,Е., Шоргин~С.\,Я.} Методы\linebreak
\\[-12pt]
\hspace*{23pt}математической статистики в задаче поиска инсайдера&3&71--75\\
\Avtors{Грушо~А.\,А., Забежайло~М.\,И., Тимонина~Е.\,Е.} О каузальной репрезентативности обуча-\linebreak
\\[-12pt]
\hspace*{23pt}ющих выборок прецедентов в задачах диагностического типа&1&80--86\\
\Avtors{Грушо~А.\,А., Тимонина~Е.\,Е., Грушо~Н.\,А., Терехина~И.\,Ю.} Выявление аномалий с по-\linebreak
\\[-12pt]
\hspace*{23pt}мощью метаданных&3&76--80\\
\Avtors{Грушо~А.\,А.} см.\ Грушо~Н.\,А.&&\\
\Avtors{Грушо~Н.\,А., Грушо~А.\,А., Забежайло~М.\,И., Тимонина~Е.\,Е.} Методы нахождения причин\linebreak
\\[-12pt]
\hspace*{23pt}сбоев в информационных технологиях  с помощью метаданных&2&33--39\\
\Avtors{Грушо~Н.\,А.} см.\ Грушо~А.\,А.&&\\
\Avtors{Данилишин~А.\,Р., Голембиовский~Д.\,Ю.} Оценка стоимости опционов на основе моделей\linebreak
\\[-12pt]
\hspace*{23pt}ARIMA--GARCH с ошибками, распределенными по закону $S_u$ Джонсона&4&83--90\\
\Avtors{Данилишин~А.\,Р., Голембиовский~Д.\,Ю.} Риск-нейтральная динамика для модели ARIMA-\linebreak
\\[-12pt]
\hspace*{23pt}GARCH с ошибками, распределенными по закону $S_U$ Джонсона&1&48--55\\
\Avtors{Диментов~А.\,В.} см.\ Краснов~Ф.\,В.&&\\
\Avtors{Донской~В.\,И.} Извлечение оптимизационных моделей из данных&3&109--118\\
\Avtors{Дубнов~Ю.\,А.} см.\ Попков~Ю.\,С.&&\\
\Avtors{Дулин~С.\,К., Дулина~Н.\,Г., Ермаков~П.\,В.} Информационный синтез документов&1&128--135\\
\Avtors{Дулина~Н.\,Г.} см.\ Дулин~С.\,К.&&\\
\Avtors{Дьяченко~Ю.\,Г.} см.\ Соколов~И.\,А.&&\\
\Avtors{Ермаков~П.\,В.} см.\ Дулин~С.\,К.&&\\
\Avtors{Ефросинин~Д.\,В.} см.\ Харин~П.\,А.&&\\
\Avtors{Жолобов~В.\,А.} см.\ Потанин~М.\,С.&&\\
\Avtors{Забежайло~М.\,И.} см.\ Грушо~А.\,А.&&\\
\Avtors{Забежайло~М.\,И.} см.\ Грушо~А.\,А.&&\\
\Avtors{Забежайло~М.\,И.} см.\ Грушо~А.\,А.&&\\
\Avtors{Забежайло~М.\,И.} см.\ Грушо~Н.\,А.&&\\
\Avtors{Захаров В. Н.} см.\ Френкель С. Л.&&\\
\Avtors{Зацман~И.\,М.} Проблемно-ориентированная верификация полноты темпоральных\linebreak
\\[-12pt]
\hspace*{23pt}онтологий и заполнение понятийных лакун&3&119--128\\
\Avtors{Зацман~И.\,М.} см.\ Гончаров~А.\,А.&&\\
\Avtors{Зацман~И.\,М.} см.\ Нуриев~В.\,А.&&\\
\Avtors{Зейфман~А.\,И.} см.\ Сатин~Я.\,А.&&\\
\Avtors{Кириков~И.\,А.} см.\ Румовская~С.\,Б.&&\\
\Avtors{Кирилюк~И.\,Л., Сенько~О.\,В.} Выбор моделей оптимальной сложности методами Монте-Карло (на примере моделей производственных функций регионов Российской\linebreak
\\[-12pt]
\hspace*{23pt}Федерации)&2&111--118\\
\Avtors{Ковалёв~Д.\,Ю.} см.\ Брюхов~Д.\,О.&&\\
\Avtors{Козеренко~Е.\,Б., Михеев~М.\,Ю., Сомин~Н.\,В., Эрлих~Л.\,И., Кузнецов~К.\,И.} Аналити\-че\-ская
текс\-тология в системах интеллектуальной обработки неструктурированных\linebreak
\\[-12pt]
\hspace*{23pt}данных&1&113--120\\
\Avtors{Колбин~И.\,С.} см.\ Абгарян~К.\,К.&&\\
\end{tabular}
}

\pagebreak

\def\leftkol{АВТОРСКИЙ УКАЗАТЕЛЬ ЗА 2020 г.} % ENGLISH ABSTRACTS}

\def\rightkol{АВТОРСКИЙ УКАЗАТЕЛЬ ЗА 2020 г.} %ENGLISH ABSTRACTS}

%\thispagestyle{myheadings}
\def\leftfootline{\small{\textbf{\thepage}
\hfill ИНФОРМАТИКА И ЕЁ ПРИМЕНЕНИЯ\ \ \ том~14\ \ \ выпуск~4\ \ \ 2020}
}%
 \def\rightfootline{\small{ИНФОРМАТИКА И ЕЁ ПРИМЕНЕНИЯ\ \ \ том~14\ \ \ выпуск~4\ \ \ 2020
 \hfill \textbf{\thepage}}}


\noindent
{\tabcolsep=3pt
\begin{tabular}{p{394pt}cc}
&\textbf{Вып.} & \textbf{Стр.}\\[3pt]
\Avtors{Королев~В.\,Ю.} О распределении отношения суммы элементов выборки, превосходящих\linebreak
\\[-12pt]
\hspace*{23pt}некоторый порог, к сумме всех элементов выборки.~I&3&35--43\\
\Avtors{Королев~В.\,Ю.} О распределении отношения суммы элементов выборки, превосходящих\linebreak
\\[-12pt]
\hspace*{23pt}некоторый порог, к сумме всех элементов выборки.~II&4&33--36\\
\Avtors{Королев~В.\,Ю.} см.\ Горшенин~А.\,К&&\\
\Avtors{Королев~В.\,Ю.} см.\ Горшенин~А.\,К.&&\\
\Avtors{Королёв~В.\,И.} см.\ Будзко~В.\,И.&&\\
\Avtors{Костина~А.\,А., Мирин~А.\,Ю., Молдовян~Д.\,Н., Фахрутдинов~Р.\,Ш.} Метод задания конечных некоммутативных ассоциативных алгебр произвольной четной размерности\linebreak
\\[-12pt]
\hspace*{23pt}для построения постквантовых криптосхем&1&\hphantom{1}94--100\\
\Avtors{Кочеткова~И.\,А.} см.\ Харин~П.\,А.&&\\
\Avtors{Краснов~Ф.\,В., Диментов~А.\,В., Шварцман~М.\,Е.} Использование тематических моделей\linebreak
\\[-12pt]
\hspace*{23pt}для парного сравнения  коллекций научных статей&3&129--135\\
\Avtors{Кривенко~М.\,П.} Последовательный анализ серий данных на основе многомерных ре-\linebreak
\\[-12pt]
\hspace*{23pt}фе\-рен\-с\-ных регионов&2&86--91\\
\Avtors{Кружков~М.\,Г.} см.\ Гончаров~А.\,А.&&\\
\Avtors{Кудрявцев~А.\,А., Шестаков~О.\,В.} Метод логарифмических моментов для оценивания\linebreak
\\[-12pt]
\hspace*{23pt}параметров гамма-экспоненциального распределения&3&49--54\\
\Avtors{Кузнецов~К.\,И.} см.\ Козеренко~Е.\,Б.&&\\
\Avtors{Кузьмин~В.\,Ю.} см.\ Горшенин~А.\,К.&&\\
\Avtors{Кушниренко~А.\,Г.} см.\ Бетелин~В.\,Б.&&\\
\Avtors{Кушниренко~А.\,Г.} см.\ Бетелин~В.\,Б.&&\\
\Avtors{Леонов~А.\,Г.} см.\ Бетелин~В.\,Б.&&\\
\Avtors{Макеева~Е.\,Д.} см.\ Харин~П.\,А.&&\\
\Avtors{Малашенко~Ю.\,Е., Назарова~И.\,А.} Аппроксимация множества достижимых потоков\linebreak
\\[-12pt]
\hspace*{23pt}многопользовательской сети&3&81--85\\
\Avtors{Мартюшова~Я.\,Г.} см.\ Босов~А.\,В.&&\\
\Avtors{Матюшенко~С.\,И., Разумчик~Р.\,В.} Стационарные характеристики системы Geo$/G/1/\infty $\linebreak
\\[-12pt]
\hspace*{23pt}с неординарным входящим потоком, управляющим размером очереди&4&25--32\\
\Avtors{Мейханаджян~Л.\,А., Разумчик~Р.\,В.} Стационарные характеристики системы $M/G/2/\infty$ с одним частным случаем дисциплины инверсионного порядка обслуживания\linebreak
\\[-12pt]
\hspace*{23pt}с обобщенным  вероятностным приоритетом&2&66--71\\
\Avtors{Мельников~А.\,В.} см.\ Вохминцев~А.\,В.&&\\
\Avtors{Мельников~С.\,Ю., Самуйлов~К.\,Е.} Статистические свойства двоичных неавтономных\linebreak
\\[-12pt]
\hspace*{23pt}регистров сдвига  с внутренним суммированием&2&80--85\\
\Avtors{Милованова~Т.\,А., Разумчик~Р.\,В.} Однолинейная система массового обслуживания с инверсионным порядком обслуживания с вероятностным приоритетом, групповым\linebreak
\\[-12pt]
\hspace*{23pt}пуассоновским потоком и фоновыми заявками&3&26--34\\
\Avtors{Мирин~А.\,Ю.} см.\ Костина~А.\,А.&&\\
\Avtors{Михеев~М.\,Ю.} см.\ Козеренко~Е.\,Б.&&\\
\Avtors{Молдовян~Д.\,Н.} см.\ Костина~А.\,А.&&\\
\Avtors{Москалева~Ф.\,А., Гайдамака~Ю.\,В., Шоргин~В.\,С.} Влияние параметров изоляции на\linebreak
\\[-12pt]
\hspace*{23pt}разделение ресурсов при нарезке сети&4&\hphantom{1}9--16\\
\Avtors{Назарова~И.\,А.} см.\ Малашенко~Ю.\,Е.&&\\
\Avtors{Наумов~А.\,В.} см.\ Босов~А.\,В.&&\\
\Avtors{Наумов~В.\,А., Самуйлов~К.\,Е.} О марковских и рациональных потоках случайных со-\linebreak
\\[-12pt]
\hspace*{23pt}бытий.~I&3&13--19\\
\Avtors{Наумов~В.\,А., Самуйлов~К.\,Е.} О марковских и рациональных потоках случайных со-\linebreak
\\[-12pt]
\hspace*{23pt}бытий.~II&4&37--46\\
\Avtors{Новиков~Д.\,А.} см.\ Шнурков~П.\,В.&&\\
\Avtors{Нуриев~В.\,А., Зацман~И.\,М.} Редуцирование спектра моделей перевода в надкорпусных\linebreak
\\[-12pt]
\hspace*{23pt}базах данных&2&119--126\\
\Avtors{Пачганов~C.\,А.} см.\ Вохминцев~А.\,В.&&\\
\end{tabular}
}

\pagebreak

\def\leftkol{АВТОРСКИЙ УКАЗАТЕЛЬ ЗА 2020 г.} % ENGLISH ABSTRACTS}

\def\rightkol{АВТОРСКИЙ УКАЗАТЕЛЬ ЗА 2020 г.} %ENGLISH ABSTRACTS}

%\thispagestyle{myheadings}
\def\leftfootline{\small{\textbf{\thepage}
\hfill ИНФОРМАТИКА И ЕЁ ПРИМЕНЕНИЯ\ \ \ том~14\ \ \ выпуск~4\ \ \ 2020}
}%
 \def\rightfootline{\small{ИНФОРМАТИКА И ЕЁ ПРИМЕНЕНИЯ\ \ \ том~14\ \ \ выпуск~4\ \ \ 2020
 \hfill \textbf{\thepage}}}


\noindent
{\tabcolsep=3pt
\begin{tabular}{p{394pt}cc}
&\textbf{Вып.} & \textbf{Стр.}\\[3pt]
\Avtors{Попков~А.\,Ю.} см.\ Попков~Ю.\,С.&&\\
\Avtors{Попков~Ю.\,С., Попков~А.\,Ю., Дубнов~Ю.\,А.} Методы детерминированных и рандомизи-\linebreak
\\[-12pt]
\hspace*{23pt}рованных энтропийных проекций для редукции размерности матрицы данных&4&47--54\\
\Avtors{Попов~Г.\,А., Симаворян~С.\,Ж., Симонян~А.\,Р., Улитина~Е.\,И.} Моделирование процесса мониторинга систем информационной безопасности на основе систем массового\linebreak
\\[-12pt]
\hspace*{23pt}обслуживания&1&71--79\\
\Avtors{Попов~М.\,В., Посыпкин~М.\,А.} Аппроксимация множества решений систем нелинейных\linebreak
\\[-12pt]
\hspace*{23pt}неравенств с использованием графических ускорителей&3&20--25\\
\Avtors{Посыпкин~М.\,А.} см.\ Попов~М.\,В.&&\\
\Avtors{Потанин~М.\,С., Вайсер~К.\,О., Жолобов~В.\,А., Стрижов~В.\,В.} Оптимизация структуры\linebreak
\\[-12pt]
\hspace*{23pt}сетей глубокого обучения&4&55--62\\
\Avtors{Разумчик~Р.\,В.} см.\ Матюшенко~С.\,И.&&\\
\Avtors{Разумчик~Р.\,В.} см.\ Мейханаджян~Л.\,А.&&\\
\Avtors{Разумчик~Р.\,В.} см.\ Милованова~Т.\,А.&&\\
\Avtors{Рождественский~Ю.\,В.} см.\ Соколов~И.\,А.&&\\
\Avtors{Румовская~С.\,Б., Кириков~И.\,А.} Метод визуального представления конфликтов в гибрид-\linebreak
\\[-12pt]
\hspace*{23pt}ных интеллектуальных многоагентных системах&4&77--82\\
\Avtors{Самуйлов~К.\,Е.} см.\ Агеев~К.\,А.&&\\
\Avtors{Самуйлов~К.\,Е.} см.\ Мельников~С.\,Ю.&&\\
\Avtors{Самуйлов~К.\,Е.} см.\ Наумов~В.\,А.&&\\
\Avtors{Самуйлов~К.\,Е.} см.\ Наумов~В.\,А.&&\\
\Avtors{Сапунова~А.\,П.} см.\ Босов~А.\,В.&&\\
\Avtors{Сатин~Я.\,А., Зейфман~А.\,И., Шилова~Г.\,Н.} О подходах к построению предельных режимов\linebreak
\\[-12pt]
\hspace*{23pt}для некоторых моделей массового обслуживания&2&3--9\\
\Avtors{Севастьянов~Л.\,А., Щетинин~Е.\,Ю.} О методах повышения точности многоклассовой\linebreak
\\[-12pt]
\hspace*{23pt}классификации на несбалансированных данных&1&63--70\\
\Avtors{Семенов~А.\,Л.} см.\ Бетелин~В.\,Б.&&\\
\Avtors{Сенько~О.\,В.} см.\ Кирилюк~И.\,Л.&&\\
\Avtors{Серебрянский~С.\,М., Тырсин~А.\,Н.} Повышение точности решения обратных задач за\linebreak
\\[-12pt]
\hspace*{23pt}счет уточнения граничных условий&1&56--62\\
\Avtors{Симаворян~С.\,Ж.} см.\ Попов~Г.\,А.&&\\
\Avtors{Симонян~А.\,Р.} см.\ Попов~Г.\,А.&&\\
\Avtors{Смирнов~Д.\,В.} см.\ Грушо~А.\,А.&&\\
\Avtors{Смирнов~Д.\,В.} см.\ Грушо~А.\,А.&&\\
\Avtors{Соколов~И.\,А., Степченков~Ю.\,А., Дьяченко~Ю.\,Г., Рождественский~Ю.\,В.} Повышение\linebreak
\\[-12pt]
\hspace*{23pt}сбоеустойчивости самосинхронных схем&4&63--68\\
\Avtors{Сомин~Н.\,В.} см.\ Козеренко~Е.\,Б.&&\\
\Avtors{Сопин~Э.\,С.} см.\ Агеев~К.\,А.&&\\
\Avtors{Сопрунов~С.\,Ф.} см.\ Бетелин~В.\,Б.&&\\
\Avtors{Соченков~И.\,В.} см.\ Будзко~В.\,И.&&\\
\Avtors{Степченков~Ю.\,А.} см.\ Соколов~И.\,А.&&\\
\Avtors{Стефанович~А.\,И.} см.\ Босов~А.\,В.&&\\
\Avtors{Стрижов~В.\,В.} см.\ Гончаров~А.\,В.&&\\
\Avtors{Стрижов~В.\,В.} см.\ Грабовой~А.\,В.&&\\
\Avtors{Стрижов~В.\,В.} см.\ Потанин~М.\,С.&&\\
\Avtors{Ступников~С.\,А.} см.\ Брюхов~Д.\,О.&&\\
\Avtors{Терехина~И.\,Ю.} см.\ Грушо~А.\,А.&&\\
\Avtors{Тимонина~Е.\,Е.} см.\  Грушо~А.\,А.&&\\
\Avtors{Тимонина~Е.\,Е.} см.\ Грушо~А.\,А.&&\\
\Avtors{Тимонина~Е.\,Е.} см.\ Грушо~А.\,А.&&\\
\Avtors{Тимонина~Е.\,Е.} см.\ Грушо~А.\,А.&&\\
\Avtors{Тимонина~Е.\,Е.} см.\ Грушо~Н.\,А.&&\\
\Avtors{Тырсин~А.\,Н.} см.\ Серебрянский~С.\,М.&&\\
\Avtors{Улитина~Е.\,И.} см.\ Попов~Г.\,А.&&\\
\end{tabular}
}

\pagebreak

\def\leftkol{АВТОРСКИЙ УКАЗАТЕЛЬ ЗА 2020 г.} % ENGLISH ABSTRACTS}

\def\rightkol{АВТОРСКИЙ УКАЗАТЕЛЬ ЗА 2020 г.} %ENGLISH ABSTRACTS}

%\thispagestyle{myheadings}
\def\leftfootline{\small{\textbf{\thepage}
\hfill ИНФОРМАТИКА И ЕЁ ПРИМЕНЕНИЯ\ \ \ том~14\ \ \ выпуск~4\ \ \ 2020}
}%
 \def\rightfootline{\small{ИНФОРМАТИКА И ЕЁ ПРИМЕНЕНИЯ\ \ \ том~14\ \ \ выпуск~4\ \ \ 2020
 \hfill \textbf{\thepage}}}


\noindent
{\tabcolsep=3pt
\begin{tabular}{p{394pt}cc}
&\textbf{Вып.} & \textbf{Стр.}\\[3pt]
\Avtors{Фахрутдинов~Р.\,Ш.} см.\ Костина~А.\,А.&&\\
\Avtors{Френкель С. Л., Захаров В. Н.} Совместная оценка предсказуемости данных и качества\linebreak
\\[-12pt]
\hspace*{23pt}предикторов&2&40--49\\
\Avtors{Харин~П.\,А., Макеева~Е.\,Д., Кочеткова~И.\,А., Ефросинин~Д.\,В., Шоргин~С.\,Я.} 
Система массового обслуживания с орбитами для анализа совместного обслуживания трафика 
с малыми задержками URLLC и~широкополосного доступа eMBB в~беспроводных\linebreak
\\[-12pt]
\hspace*{23pt}сетях пятого поколения&4&17--24\\
\Avtors{Хусаинов~А.\,А.} Производительность ограниченного конвейера&1&87--93\\
\Avtors{Шанин~И.\,А.} см.\ Брюхов~Д.\,О.&&\\
\Avtors{Шварцман~М.\,Е.} см.\ Краснов~Ф.\,В.&&\\
\Avtors{Шестаков~О.\,В.} Асимптотика оценки среднеквадратичного риска в задаче обращения\linebreak
\\[-12pt]
\hspace*{23pt}преобразования Радона по проекциям, регистрируемым на случайной сетке&2&26--32\\
\Avtors{Шестаков~О.\,В.} Асимптотическая регулярность вейвлет-методов обращения линейных однородных операторов по наблюдениям, регистрируемым в случайные моменты\linebreak
\\[-12pt]
\hspace*{23pt}времени&1&3--9\\
\Avtors{Шестаков~О.\,В.} О статистических свойствах оценки риска в задаче обращения преобра-\linebreak
\\[-12pt]
\hspace*{23pt}зования Радона при случайном объеме проекционных данных&3&44--48\\
\Avtors{Шестаков~О.\,В.} см.\ Кудрявцев~А.\,А.&&\\
\Avtors{Шилова~Г.\,Н.} см.\ Сатин~Я.\,А.&&\\
\Avtors{Шихиев~Ф.\,Ш.} см.\ Шихиев~Ш.\,Б.&&\\
\Avtors{Шихиев~Ш.\,Б., Шихиев~Ф.\,Ш.} Инкапсуляция семантических представлений в элементы\linebreak
\\[-12pt]
\hspace*{23pt}грамматики&1&121--127\\
\Avtors{Шнурков~П.\,В., Адамова~К.\,А.} Решение задачи безусловного экстремума для дробно-\linebreak
\\[-12pt]
\hspace*{23pt}линейного интегрального функционала, зависящего от параметра&2&\hphantom{1}98--103\\
\Avtors{Шнурков~П.\,В., Новиков~Д.\,А.} О концепции стохастической модели с управлением в~моменты выхода процесса на границу заданного подмножества множества\linebreak
\\[-12pt]
\hspace*{23pt}состояний&3&101--108\\
\Avtors{Шоргин~В.\,С.} см.\ Москалева~Ф.\,А.&&\\
\Avtors{Шоргин~С.\,Я.} см.\ Агеев~К.\,А.&&\\
\Avtors{Шоргин~С.\,Я.} см.\ Грушо~А.\,А.&&\\
\Avtors{Шоргин~С.\,Я.} см.\ Харин~П.\,А.&&\\
\Avtors{Щербинина~А.\,А.} см.\ Горшенин~А.\,К.&&\\
\Avtors{Щетинин~Е.\,Ю.} см.\ Севастьянов~Л.\,А.&&\\
\Avtors{Эрлих~Л.\,И.} см.\ Козеренко~Е.\,Б.&&\\
\Avtors{Ядринцев~В.\,В.} см.\ Будзко~В.\,И.&&\\
\Avtors{Яркина~Н.\,В.} см.\ Агеев~К.\,А.&&\\
\end{tabular}
}

%\thispagestyle{myheadings}
\def\leftfootline{\small{\textbf{\thepage}
\hfill ИНФОРМАТИКА И ЕЁ ПРИМЕНЕНИЯ\ \ \ том~14\ \ \ выпуск~4\ \ \ 2020}
}%
 \def\rightfootline{\small{ИНФОРМАТИКА И ЕЁ ПРИМЕНЕНИЯ\ \ \ том~14\ \ \ выпуск~4\ \ \ 2020
 \hfill \textbf{\thepage}}}

 \label{end\stat}

\newpage

\def\stat{cont-e}
{%\hrule\par
%\vskip 7pt % 7pt
\raggedleft\Large \bf%\baselineskip=3.2ex
2\,0\,2\,0\ \ A\,U\,T\,H\,O\,R\ \ I\,N\,D\,E\,X \vskip 17pt
 \hrule
 \par
\vskip 21pt plus 6pt minus 3pt }

\label{st\stat}

\def\tit{\ }

\def\aut{\ }
\def\auf{\ }

\def\leftkol{\ } %2020 AUTHOR INDEX} % ENGLISH ABSTRACTS}

\def\rightkol{\ } %2020 AUTHOR INDEX} %ENGLISH ABSTRACTS}

\titele{\tit}{\aut}{\auf}{\leftkol}{\rightkol}
\addcontentsline{toc}{subsection}{\textrm\textbf 2020 Author Index}

\def\leftfootline{\small{\textbf{\thepage}
\hfill INFORMATIKA I EE PRIMENENIYA~--- INFORMATICS AND APPLICATIONS\ \ \ 2020\
\ \ volume~14\ \ \ issue\ 4}
}%
 \def\rightfootline{\small{INFORMATIKA I EE PRIMENENIYA~--- INFORMATICS AND APPLICATIONS\ \ \ 2020\ \ \ volume~14\ \ \ issue\ 4
\hfill \textbf{\thepage}}}

\vspace*{-24pt}

\noindent
{\tabcolsep=3pt
\begin{tabular}{p{395.89pt}cc}
&\textbf{Issue} & \textbf{Page}\\[6pt]
\Avtors{Abgaryan~K.\,K. and Gavrilov~E.\,S.} Integration platform for multiscale modeling of neuromorphic\linebreak
\\[-12pt]
\hspace*{23pt}systems&2&104--110\\
\Avtors{Abgaryan~K.\,K. and Kolbin~I.\,S.} Application of multiscale approach and data sciences for\linebreak
\\[-12pt]
\hspace*{23pt}modeling thermal conductivity in layered structures&4&91--99\\
\Avtors{Adamova~K.\,A.} see Shnurkov~~P.\,V.&&\\
\Avtors{Agalarov~Ya.\,M.} Optimization of the capacity of the main storage in $G/M/1/K$ queueing system\linebreak
\\[-12pt]
\hspace*{23pt}with an additional storage device&2&72--79\\
\Avtors{Agasandyan~G.\,A.} Computational aspects of optimization on CC-VaR in a complex of markets&3&62--70\\
\Avtors{Ageev~K.\,A., Sopin~E.\,S., Yarkina~N.\,V., Samouylov~K.\,E., and Shorgin~S.\,Ya.} Analysis of the\linebreak
\\[-12pt]
\hspace*{23pt}network slicing mechanisms with guaranteed allocated resources for various traffic types&3&\hphantom{1}94--100\\
\Avtors{Bakhteev~O.\,Yu.} see Grabovoy~A.\,V.&&\\
\Avtors{Bazilevskiy~M.\,P.} Multifactor fully connected linear regression models without constraints to the\linebreak
\\[-12pt]
\hspace*{23pt}ratios of variables errors variances&2&92--97\\
\Avtors{Belenkov~V.\,G.} see Budzko~V.\,I.&&\\
\Avtors{Betelin~V.\,B., Kushnirenko~A.\,G., and Leonov~A.\,G.} Basic concepts of programming expounded\linebreak
\\[-12pt]
\hspace*{23pt}for preschoolers&3&55--61\\
\Avtors{Betelin~V.\,B., Kushnirenko~A.\,G., Semenov~A.\,L., and Soprunov~S.\,F.} About digital literacy and\linebreak
\\[-12pt]
\hspace*{23pt}environments for its development&4&100--107\\
\Avtors{Borisov~A.\,V.} Numerical schemes of Markov jump process filtering given discretized observa-\linebreak
\\[-12pt]
\hspace*{23pt}tions~II: Additive noise case&1&17--23\\
\Avtors{Borisov~A.\,V.} Numerical schemes of Markov jump process filtering given discretized observa-\linebreak
\\[-12pt]
\hspace*{23pt}tions III: Multiplicative noises case&2&10--18\\
\Avtors{Bosov~A.\,V.} Stochastic differential system output control by the quadratic criterion. V. Case of\linebreak
\\[-12pt]
\hspace*{23pt}incomplete state information&2&19--28\\
\Avtors{Bosov~A.\,V., Martyushova~Ya.\,G., Naumov~A.\,V., and Sapunova~A.\,P.} Bayesian approach to the\linebreak
\\[-12pt]
\hspace*{23pt}construction of an individual user trajectory in the system of distance learning&3&86--93\\
\Avtors{Bosov~A.\,V. and Stefanovich~A.\,I.} Stochastic differential system output control by the quadratic\linebreak
\\[-12pt]
\hspace*{23pt}criterion. IV. Alternative numerical decision&1&24--30\\
\Avtors{Briukhov~D.\,O., Stupnikov~S.\,A., Kovalev~D.\,Yu., and Shanin~I.\,A.} Neurophysiology as a subject\linebreak
\\[-12pt]
\hspace*{23pt}domain for~data intensive problem solving&1&40--47\\
\Avtors{Budzko~V.\,I., Yadrintsev~V.\,V., Sochenkov~I.\,V., Korolev~V.\,I., and Belenkov~V.\,G.} Extraction of confidentiality markers from texts under conditions of high uncertainty in systems with\linebreak
\\[-12pt]
\hspace*{23pt}data intensive usage&4&69--76\\
\Avtors{Danilishin~A.\,R. and Golembiovsky~D.\,Yu.} Estimating the fair value of options based on\linebreak
\\[-12pt]
\hspace*{23pt}ARIMA--GARCH models with errors distributed according to the Johnson's $S_u$ law&4&83--90\\
\Avtors{Danilishin~A.\,R. and Golembiovsky~D.\,Yu.} Risk-neutral dynamics for the ARIMA-GARCH\linebreak
\\[-12pt]
\hspace*{23pt}random process with errors distributed according to the Johnson's $S_u$ law&1&48--55\\
\Avtors{Diachenko~Yu.\,G.} see Sokolov~I.\,A.&&\\
\Avtors{Dimentov~A.\,V.} see Krasnov~F.\,V.&&\\
\Avtors{Donskoy~V.\,I.} Optimization models extraction from data&3&109--118\\
\Avtors{Dubnov~Y.\,A.} see Popkov~Y.\,S.&&\\
\Avtors{Dulin~S.\,K., Dulina~N.\,G., and Ermakov~P.\,V.} Information fusion of documents&1&128--135\\
\Avtors{Dulina~N.\,G.} see Dulin~S.\,K.&&\\
\Avtors{Efrosinin~D.\,V.} see Kharin~P.\,A.&&\\
\Avtors{Ehrlich~L.\,I.} see Kozerenko~E.\,B.&&\\
\Avtors{Ermakov~P.\,V.} see Dulin~S.\,K.&&\\
\end{tabular}
}
\pagebreak

\def\leftfootline{\small{\textbf{\thepage}
\hfill INFORMATIKA I EE PRIMENENIYA~--- INFORMATICS AND APPLICATIONS\ \ \ 2020\
\ \ volume~14\ \ \ issue\ 4}
}%
 \def\rightfootline{\small{INFORMATIKA I EE PRIMENENIYA~---
INFORMATICS AND APPLICATIONS\ \ \ 2020\ \ \ volume~14\ \ \ issue\ 4
\hfill \textbf{\thepage}}}

\def\leftkol{2020 AUTHOR INDEX} % ENGLISH ABSTRACTS}

\def\rightkol{2020 AUTHOR INDEX} %ENGLISH ABSTRACTS}


\noindent
{\tabcolsep=3pt
\begin{tabular}{p{395.48108pt}cc}
&\textbf{Issue} & \textbf{Page}\\[6pt]
\Avtors{Fahrutdinov~R.\,Sh.} see Kostina~A.\,A.&&\\
\Avtors{Frenkel~S.\,L. and Zakharov~V.\,N.} Joint assessment of data predictability and quality pre-\linebreak
\\[-12pt]
\hspace*{23pt}dictors&2&40--49\\
\Avtors{Gaidamaka~Yu.\,V.} see Moskaleva~F.\,A.&&\\
\Avtors{Gavrilov~E.\,S.} see Abgaryan~K.\,K.&&\\
\Avtors{Golembiovsky~D.\,Yu.} see Danilishin~A.\,R.&&\\
\Avtors{Golembiovsky~D.\,Yu.} see Danilishin~A.\,R.&&\\
\Avtors{Goncharov~A.\,V. and Strijov~V.\,V.} Alignment of ordered set Cartesian product&1&31--39\\
\Avtors{Goncharov~A.\,A., Zatsman~I.\,M., and Kruzhkov~M.\,G.} Evolution of classifications in supracorpora\linebreak
\\[-12pt]
\hspace*{23pt}databases&4&108--116\\
\Avtors{Gorshenin~A.\,K. and Korolev~V.\,Yu.} Approximation of particle size distributions of lunar regolith\linebreak
\\[-12pt]
\hspace*{23pt}based on the resampling&2&50--57\\
\Avtors{Gorshenin~A.\,K., Korolev~V.\,Yu., and Shcherbinina~A.\,A.} Statistical estimation of distributions\linebreak
\\[-12pt]
\hspace*{23pt}of random coefficients in the Langevin stochastic differential equation&3&\hphantom{1}3--12\\
\Avtors{Gorshenin~A.\,K. and Kuzmin~V.\,Yu.} Analysis of configurations of LSTM networks for medium-\linebreak
\\[-12pt]
\hspace*{23pt}term vector forecasting&1&10--16\\
\Avtors{Grabovoy~A.\,V., Bakhteev~O.\,Yu., and Strijov~V.\,V.} Ordering the set of neural network parameters&2&58--65\\
\Avtors{Grusho~A.\,A., Timonina~E.\,E., Grusho~N.\,A., and Teryokhina~I.\,Yu.} Identifying anomalies using\linebreak
\\[-12pt]
\hspace*{23pt}metadata&3&76--80\\
\Avtors{Grusho~A.\,A., Zabezhailo~M.\,I., Smirnov~D.\,V., and Timonina~E.\,E.} On probabilistic estimates of\linebreak
\\[-12pt]
\hspace*{23pt}the validity of empirical conclusions&4&3--8\\
\Avtors{Grusho~A.\,A., Zabezhailo~M.\,I., and Timonina~E.\,E.} On causal representativeness of training\linebreak
\\[-12pt]
\hspace*{23pt}samples of precedents in diagnostic type tasks&1&80--86\\
\Avtors{Grusho~A.\,A.} see Grusho~N.\,A.&&\\
\Avtors{Grusho~N.\,A., Grusho~A.\,A., Zabezhailo~M.\,I., and Timonina~E.\,E.} Methods of finding the causes\linebreak
\\[-12pt]
\hspace*{23pt}of information technology failures by means of metadata&2&33--39\\
\Avtors{Grusho~N.\,A., Zabezhailo~M.\,I., Smirnov~D.\,V., Timonina~E.\,E., and Shorgin~S.\,Ya.} Mathematical\linebreak
\\[-12pt]
\hspace*{23pt}statistics in the task of identifying hostile insiders&3&71--75\\
\Avtors{Grusho~N.\,A.} see Grusho~A.\,A.&&\\
\Avtors{Kharin~P.\,A., Makeeva~E.\,D., Kochetkova~I.\,A., Efrosinin~D.\,V., and Shorgin~S.\,Ya.} Retrial\linebreak
\\[-12pt]
\hspace*{23pt}queuing model for analyzing joint URLLC and eMBB transmission in 5G networks&4&17--24\\
\Avtors{Khusainov~A.\,A.} Performance of the bounded pipeline&1&87--93\\
\Avtors{Kirikov~I.\,A.} see Rumovskaya~S.\,B.&&\\
\Avtors{Kirilyuk~I.\,L. and Sen'ko~O.\,V.} Selection of optimal complexity models by methods of nonparametric statistics (on the example of production function model of regions of the Russian\linebreak
\\[-12pt]
\hspace*{23pt}Federation)&2&111--118\\
\Avtors{Kochetkova~I.\,A.} see Kharin~P.\,A.&&\\
\Avtors{Kolbin~I.\,S.} see Abgaryan~K.\,K.&&\\
\Avtors{Korolev~V.\,I.} see Budzko~V.\,I.&&\\
\Avtors{Korolev~V.\,Yu.} On the distribution of the ratio of the sum of sample elements exceeding\linebreak
\\[-12pt]
\hspace*{23pt}a threshold to the total sum of sample elements.~I&3&35--43\\
\Avtors{Korolev~V.\,Yu.} On the distribution of the ratio of the sum of sample elements exceeding\linebreak
\\[-12pt]
\hspace*{23pt}a threshold to the total sum of sample elements.~II&4&33--36\\
\Avtors{Korolev~V.\,Yu.} see Gorshenin~A.\,K.&&\\
\Avtors{Korolev~V.\,Yu.} see Gorshenin~A.\,K.&&\\
\Avtors{Kostina~A.\,A., Mirin~A.\,Yu., Moldovyan~D.\,N., and Fahrutdinov~R.\,Sh.} Method for defining finite noncommutative associative algebras of arbitrary even dimension for development of the\linebreak
\\[-12pt]
\hspace*{23pt}postquantum cryptoschemes&1&\hphantom{1}94--100\\
\Avtors{Kovalev~D.\,Yu.} see Briukhov~D.\,O.&&\\
\Avtors{Kozerenko~E.\,B., Mikheev~M.\,Y., Somin~N.\,V., Ehrlich~L.\,I., and Kuznetsov~K.\,I.} Analytical\linebreak
\\[-12pt]
\hspace*{23pt}textology in intelligent processing systems for unstructured data&1&113--120\\
\Avtors{Krasnov~F.\,V., Dimentov~A.\,V., and Shvartsman~M.\,E.} Using topic models for pairwise comparison\linebreak
\\[-12pt]
\hspace*{23pt}of collections of scientific papers&3&129--135\\
\end{tabular}
}
\pagebreak

\def\leftfootline{\small{\textbf{\thepage}
\hfill INFORMATIKA I EE PRIMENENIYA~--- INFORMATICS AND APPLICATIONS\ \ \ 2020\
\ \ volume~14\ \ \ issue\ 4}
}%
 \def\rightfootline{\small{INFORMATIKA I EE PRIMENENIYA~---
INFORMATICS AND APPLICATIONS\ \ \ 2020\ \ \ volume~14\ \ \ issue\ 4
\hfill \textbf{\thepage}}}

\def\leftkol{2020 AUTHOR INDEX} % ENGLISH ABSTRACTS}

\def\rightkol{2020 AUTHOR INDEX} %ENGLISH ABSTRACTS}


\noindent
{\tabcolsep=3pt
\begin{tabular}{p{395.48108pt}cc}
&\textbf{Issue} & \textbf{Page}\\[6pt]
\Avtors{Krivenko~M.\,P.} Sequential analysis of serial measurements based on multivariate reference\linebreak
\\[-12pt]
\hspace*{23pt}regions&2&86--91\\
\Avtors{Kruzhkov~M.\,G.} see Goncharov~A.\,A.&&\\
\Avtors{Kudryavtsev~A.\,A. and Shestakov~O.\,V.} Method of logarithmic moments for estimating the\linebreak
\\[-12pt]
\hspace*{23pt}gamma-exponential distribution parameters&3&49--54\\
\Avtors{Kushnirenko~A.\,G.} see Betelin~V.\,B.&&\\
\Avtors{Kushnirenko~A.\,G.} see Betelin~V.\,B.&&\\
\Avtors{Kuzmin~V.\,Yu.} see Gorshenin~A.\,K.&&\\
\Avtors{Kuznetsov~K.\,I.} see Kozerenko~E.\,B.&&\\
\Avtors{Leonov~A.\,G.} see Betelin~V.\,B.&&\\
\Avtors{Makeeva~E.\,D.} see Kharin~P.\,A.&&\\
\Avtors{Malashenko~Yu.\,E. and Nazarova~I.\,A.} Approximation of the multiuser network feasible\linebreak
\\[-12pt]
\hspace*{23pt}flows set&3&81--85\\
\Avtors{Martyushova~Ya.\,G.} see Bosov~A.\,V.&&\\
\Avtors{Matyushenko~S.\,I. and Razumchik~R.\,V.} Stationary characteristics of discrete-time Geo$/G/1/\infty$\linebreak
\\[-12pt]
\hspace*{23pt}queue with batch arrivals and one queue skipping policy&4&25--32\\
\Avtors{Melnikov~A.\,V.} see Vokhmintcev~A.\,V.&&\\
\Avtors{Melnikov~S.\,Yu. and Samouylov~K.\,E.} Statistical properties of binary nonautonomous shift\linebreak
\\[-12pt]
\hspace*{23pt}registers with internal xor&2&80--85\\
\Avtors{Meykhanadzhyan~L.\,A. and Razumchik~R.\,V.} Stationary characteristics of $M/G/2/\infty$ queue\linebreak
\\[-12pt]
\hspace*{23pt}with identical servers, LIFO service, and resampling policy&2&66--71\\
\Avtors{Mikheev~M.\,Y.} see Kozerenko~E.\,B.&&\\
\Avtors{Milovanova~T.\,A. and Razumchik~R.\,V.} A single-server queueing system with LIFO service,\linebreak
\\[-12pt]
\hspace*{23pt}probabilistic priority, batch Poisson arrivals, and background customers&3&26--34\\
\Avtors{Mirin~A.\,Yu.} see Kostina~A.\,A.&&\\
\Avtors{Moldovyan~D.\,N.} see Kostina~A.\,A.&&\\
\Avtors{Moskaleva~F.\,A., Gaidamaka~Yu.\,V., and Shorgin~V.\,S.} Impact of the isolation parameters on\linebreak
\\[-12pt]
\hspace*{23pt}resource allocation in the network slicing model&4&\hphantom{1}9--16\\
\Avtors{Naumov~A.\,V.} see Bosov~A.\,V.&&\\
\Avtors{Naumov~V.\,A. and Samouylov~К.\,Е.} On Markovian and rational arrival processes.~I&3&13--19\\
\Avtors{Naumov~V.\,A. and Samouylov~K.\,E.} On Markovian and rational arrival processes.~II&4&37--46\\
\Avtors{Nazarova~I.\,A.} see Malashenko~Yu.\,E.&&\\
\Avtors{Novikov~D.\,A.} see Shnurkov~P.\,V.&&\\
\Avtors{Nuriev~V.\,A. and Zatsman~I.\,M.} Reducing the spectrum of translation models in supracorpora\linebreak
\\[-12pt]
\hspace*{23pt}databases&2&119--126\\
\Avtors{Pachganov~S.\,A.} see Vokhmintcev~A.\,V.&&\\
\Avtors{Popkov~A.\,Y.} see Popkov~Y.\,S.&&\\
\Avtors{Popkov~Y.\,S., Popkov~A.\,Y., and Dubnov~Y.\,A.} Deterministic and randomized methods of entropy\linebreak
\\[-12pt]
\hspace*{23pt}projection for dimensionality reduction problems&4&47--54\\
\Avtors{Popov~G.\,A., Simavoryan~S.\,Zh., Simonyan~A.\,R., and Ulitina~E.\,I.} Modeling of monitoring of\linebreak
\\[-12pt]
\hspace*{23pt}information security process on the basis of queuing systems&1&71--79\\
\Avtors{Popov~M.\,V. and Posypkin~M.\,A.} Approximation of the set of solutions of systems of nonlinear\linebreak
\\[-12pt]
\hspace*{23pt}inequalities using graphic accelerators&3&20--25\\
\Avtors{Posypkin~M.\,A.} see Popov~M.\,V.&&\\
\Avtors{Potanin~M.\,S., Vayser~K.\,O., Zholobov~V.\,A., and Strijov~V.\,V.} Deep learning neural network\linebreak
\\[-12pt]
\hspace*{23pt}structure optimization&4&55--62\\
\Avtors{Razumchik~R.\,V.} see Matyushenko~S.\,I.&&\\
\Avtors{Razumchik~R.\,V.} see Meykhanadzhyan~L.\,A.&&\\
\Avtors{Razumchik~R.\,V.} see Milovanova~T.\,A.&&\\
\Avtors{Rogdestvenski~Yu.\,V.} see Sokolov~I.\,A.&&\\
\Avtors{Rumovskaya~S.\,B. and Kirikov~I.\,A.} Conflict visual representation method in hybrid intelligent\linebreak
\\[-12pt]
\hspace*{23pt}multiagent systems&4&77--82\\
\Avtors{Samouylov~K.\,E.} see Ageev~K.\,A.&&\\
\end{tabular}
}
\pagebreak

\def\leftfootline{\small{\textbf{\thepage}
\hfill INFORMATIKA I EE PRIMENENIYA~--- INFORMATICS AND APPLICATIONS\ \ \ 2020\
\ \ volume~14\ \ \ issue\ 4}
}%
 \def\rightfootline{\small{INFORMATIKA I EE PRIMENENIYA~---
INFORMATICS AND APPLICATIONS\ \ \ 2020\ \ \ volume~14\ \ \ issue\ 4
\hfill \textbf{\thepage}}}

\def\leftkol{2020 AUTHOR INDEX} % ENGLISH ABSTRACTS}

\def\rightkol{2020 AUTHOR INDEX} %ENGLISH ABSTRACTS}


\noindent
{\tabcolsep=3pt
\begin{tabular}{p{395.48108pt}cc}
&\textbf{Issue} & \textbf{Page}\\[6pt]
\Avtors{Samouylov~K.\,E.} see Melnikov~S.\,Yu.&&\\
\Avtors{Samouylov~K.\,E.} see Naumov~V.\,A.&&\\
\Avtors{Samouylov~K.\,Е.} see Naumov~V.\,A.&&\\
\Avtors{Sapunova~A.\,P.} see Bosov~A.\,V.&&\\
\Avtors{Satin~Ya.\,A., Zeifman~A.\,I., and Shilova~G.\,N.} On approaches to constructing limiting regimes\linebreak
\\[-12pt]
\hspace*{23pt}for some queuing models&2&3--9\\
\Avtors{Semenov~A.\,L.} see Betelin~V.\,B.&&\\
\Avtors{Sen'ko~O.\,V.} see Kirilyuk~I.\,L.&&\\
\Avtors{Serebryanskii~S.\,M. and Tyrsin~A.\,N.} Improvement of the accuracy of solution of tasks for the\linebreak
\\[-12pt]
\hspace*{23pt}account of the construction of boundary conditions&1&56--62\\
\Avtors{Sevastianov~L.\,A. and Shchetinin~E.\,Yu.} On methods for improving the accuracy of multiclass\linebreak
\\[-12pt]
\hspace*{23pt}classification on imbalanced data&1&63--70\\
\Avtors{Shanin~I.\,A.} see Briukhov~D.\,O.&&\\
\Avtors{Shcherbinina~A.\,A.} see Gorshenin~A.\,K.&&\\
\Avtors{Shchetinin~E.\,Yu.} see Sevastianov~L.\,A.&&\\
\Avtors{Shestakov~O.\,V.} Asymptotic regularity of the wavelet methods of inverting linear homogeneous\linebreak
\\[-12pt]
\hspace*{23pt}operators from observations recorded at random times&1&3--9\\
\Avtors{Shestakov~O.\,V.} Asymptotics of the mean-square risk estimate in the problem of inverting the\linebreak
\\[-12pt]
\hspace*{23pt}Radon transform from projections registered on a random grid&2&29--32\\
\Avtors{Shestakov~O.\,V.} On the statistical properties of risk estimate in the problem of inverting the\linebreak
\\[-12pt]
\hspace*{23pt}Radon transform with a random volume of projection data&3&44--48\\
\Avtors{Shestakov~O.\,V.} see Kudryavtsev~A.\,A.&&\\
\Avtors{Shihiev~F.\,Sh.} see Shihiev~Sh.\,B.&&\\
\Avtors{Shihiev~Sh.\,B. and Shihiev~F.\,Sh.} Incapsulation of semantic representations into elements of\linebreak
\\[-12pt]
\hspace*{23pt}a grammar&1&121--127\\
\Avtors{Shilova~G.\,N.} see Satin~Ya.\,A.&&\\
\Avtors{Shnurkov~~P.\,V. and Adamova~K.\,A.} Solution of the unconditional extremal problem for a~linear-\linebreak
\\[-12pt]
\hspace*{23pt}fractional integral functional dependent on the parameter&2&\hphantom{1}98--103\\
\Avtors{Shnurkov~P.\,V. and Novikov~D.\,A.} On the concept of a stochastic model with control at the\linebreak
\\[-12pt]
\hspace*{23pt}moments of the process at the border of a presented subset of multiple states&3&101--108\\
\Avtors{Shorgin~S.\,Ya.} see Ageev~K.\,A.&&\\
\Avtors{Shorgin~S.\,Ya.} see Grusho~N.\,A.&&\\
\Avtors{Shorgin~S.\,Ya.} see Kharin~P.\,A.&&\\
\Avtors{Shorgin~V.\,S.} see Moskaleva~F.\,A.&&\\
\Avtors{Shvartsman~M.\,E.} see Krasnov~F.\,V.&&\\
\Avtors{Simavoryan~S.\,Zh.} see Popov~G.\,A.&&\\
\Avtors{Simonyan~A.\,R.} see Popov~G.\,A.&&\\
\Avtors{Smirnov~D.\,V.} see Grusho~A.\,A.&&\\
\Avtors{Smirnov~D.\,V.} see Grusho~N.\,A.&&\\
\Avtors{Sochenkov~I.\,V.} see Budzko~V.\,I.&&\\
\Avtors{Sokolov~I.\,A., Stepchenkov~Yu.\,A., Diachenko~Yu.\,G., and Rogdestvenski~Yu.\,V.} Improvement of\linebreak
\\[-12pt]
\hspace*{23pt}self-timed circuit soft error tolerance&4&63--68\\
\Avtors{Somin~N.\,V.} see Kozerenko~E.\,B.&&\\
\Avtors{Sopin~E.\,S.} see Ageev~K.\,A.&&\\
\Avtors{Soprunov~S.\,F.} see Betelin~V.\,B.&&\\
\Avtors{Stefanovich~A.\,I.} see Bosov~A.\,V.&&\\
\Avtors{Stepchenkov~Yu.\,A.} see Sokolov~I.\,A.&&\\
\Avtors{Strijov~V.\,V.} see Goncharov~A.\,V.&&\\
\Avtors{Strijov~V.\,V.} see Grabovoy~A.\,V.&&\\
\Avtors{Strijov~V.\,V.} see Potanin~M.\,S.&&\\
\Avtors{Stupnikov~S.\,A.} see Briukhov~D.\,O.&&\\
\Avtors{Teryokhina~I.\,Yu.} see Grusho~A.\,A.&&\\
\Avtors{Timonina~E.\,E.} see Grusho~A.\,A.&&\\
\end{tabular}
}
\pagebreak

\def\leftfootline{\small{\textbf{\thepage}
\hfill INFORMATIKA I EE PRIMENENIYA~--- INFORMATICS AND APPLICATIONS\ \ \ 2020\
\ \ volume~14\ \ \ issue\ 4}
}%
 \def\rightfootline{\small{INFORMATIKA I EE PRIMENENIYA~---
INFORMATICS AND APPLICATIONS\ \ \ 2020\ \ \ volume~14\ \ \ issue\ 4
\hfill \textbf{\thepage}}}

\def\leftkol{2020 AUTHOR INDEX} % ENGLISH ABSTRACTS}

\def\rightkol{2020 AUTHOR INDEX} %ENGLISH ABSTRACTS}


\noindent
{\tabcolsep=3pt
\begin{tabular}{p{395.48108pt}cc}
&\textbf{Issue} & \textbf{Page}\\[6pt]
\Avtors{Timonina~E.\,E.} see Grusho~A.\,A.&&\\
\Avtors{Timonina~E.\,E.} see Grusho~A.\,A.&&\\
\Avtors{Timonina~E.\,E.} see Grusho~N.\,A.&&\\
\Avtors{Timonina~E.\,E.} see Grusho~N.\,A.&&\\
\Avtors{Tyrsin~A.\,N.} see Serebryanskii~S.\,M.&&\\
\Avtors{Ulitina~E.\,I.} see Popov~G.\,A.&&\\
\Avtors{Vayser~K.\,O.} see Potanin~M.\,S.&&\\
\Avtors{Vokhmintcev~A.\,V., Melnikov~A.\,V., and Pachganov~S.\,A.} Simultaneous localization and mapping method in  three-dimensional space based on the combined solution of the  point--point\linebreak
\\[-12pt]
\hspace*{23pt}variation problem ICP for an affine transformation&1&101--112\\
\Avtors{Yadrintsev~V.\,V.} see Budzko~V.\,I.&&\\
\Avtors{Yarkina~N.\,V.} see Ageev~K.\,A.&&\\
\Avtors{Zabezhailo~M.\,I.} see Grusho~A.\,A.&&\\
\Avtors{Zabezhailo~M.\,I.} see Grusho~A.\,A.&&\\
\Avtors{Zabezhailo~M.\,I.} see Grusho~N.\,A.&&\\
\Avtors{Zabezhailo~M.\,I.} see Grusho~N.\,A.&&\\
\Avtors{Zakharov~V.\,N.} see Frenkel~S.\,L.&&\\
\Avtors{Zatsman~I.\,M.} Problem-oriented verifying the completeness  of~temporal ontologies and\linebreak
\\[-12pt]
\hspace*{23pt}filling~conceptual lacunas&3&119--128\\
\Avtors{Zatsman~I.\,M.} see Goncharov~A.\,A.&&\\
\Avtors{Zatsman~I.\,M.} see Nuriev~V.\,A.&&\\
\Avtors{Zeifman~A.\,I.} see Satin~Ya.\,A.&&\\
\Avtors{Zholobov~V.\,A.} see Potanin~M.\,S.&&\\
\end{tabular}
}

%\thispagestyle{myheadings}
\def\leftfootline{\small{\textbf{\thepage}
\hfill INFORMATIKA I EE PRIMENENIYA~--- INFORMATICS AND APPLICATIONS\ \ \ 2020\
\ \ volume~14\ \ \ issue\ 4}
}%
 \def\rightfootline{\small{INFORMATIKA I EE PRIMENENIYA~---
INFORMATICS AND APPLICATIONS\ \ \ 2020\ \ \ volume~14\ \ \ issue\ 4
\hfill \textbf{\thepage}}}

 \label{end\stat}

\newpage


%\linebreak
%\\[-12pt]
%\hspace*{23pt}
%%Информатика и её применения
%Том 15 Выпуск 1-4 Год 2021

\def\stat{cont}
{%\hrule\par
%\vskip 7pt % 7pt
\raggedleft\Large \bf%\baselineskip=3.2ex
А\,В\,Т\,О\,Р\,С\,К\,И\,Й\ \ У\,К\,А\,З\,А\,Т\,Е\,Л\,Ь\ \ З\,А\ \ 2\,0\,2\,1 г. \vskip 17pt
 \hrule
 \par
\vskip 21pt plus 6pt minus 3pt }

\label{st\stat}

\def\tit{\ }

\def\aut{\ }
\def\auf{\ }

\def\leftkol{\ } % ENGLISH ABSTRACTS}

\def\rightkol{\ } %АВТОРСКИЙ УКАЗАТЕЛЬ ЗА 2021 г.} %ENGLISH ABSTRACTS}

\titele{\tit}{\aut}{\auf}{\leftkol}{\rightkol}
\addcontentsline{toc}{subsection}{\textrm\textbf Авторский указатель за 2021 г.}

\vspace*{-24pt}

\noindent
{\tabcolsep=3pt
\begin{tabular}{p{397pt}cc}
&\textbf{Вып.} & \textbf{Стр.}\\[6pt]
\Avtors{Абгарян~К.\,К., Гаврилов~Е.\,С.} Распределенная информационная система для расчета\linebreak
\\[-12pt]
\hspace*{23pt}структурных свойств композиционных материалов&4&50--58\\
\Avtors{Агаларов~Я.\,М.} Оптимальное пороговое управление доступом в системе $M/M/s$ с не-\linebreak
\\[-12pt]
\hspace*{23pt}однородными приборами и общим накопителем&1&57--64\\
\Avtors{Андрианова~Е.\,Г.} см.\ Сигов~А.\,С.&&\\
\Avtors{Арутюнов~Е.\,Н., Кудрявцев~А.\,А., Недоливко~Ю.\,Н.} Вероятностные характеристики\linebreak
\\[-12pt]
\hspace*{23pt}индекса баланса факторов, имеющих обобщенные гамма-распределения&1&65--71\\
\Avtors{Базилевский~М.\,П.} Метод выпрямления искаженных из-за мультиколлинеарности\linebreak
\\[-12pt]
\hspace*{23pt}коэффициентов в регрессионных моделях&2&60--65\\
\Avtors{Бахтеев~О.\,Ю.} см.\ Гребенькова~О.\,С.&&\\
\Avtors{Бахтеев~О.\,Ю.} см.\ Кузнецова~Р.\,В.&&\\
\Avtors{Борисов~А.\,В., Казанчян~Д.\,Х.} Фильтрация состояний марковских скачкообразных про-\linebreak
\\[-12pt]
\hspace*{23pt}цессов по комплексным наблюдениям I: точное решение задачи&2&12--19\\
\Avtors{Борисов~А.\,В., Казанчян~Д.\,Х.} Фильтрация состояний марковских скачкообразных про-\linebreak
\\[-12pt]
\hspace*{23pt}цессов по комплексным наблюдениям II: численный алгоритм&3&\hphantom{1}9--15\\
\Avtors{Босов~А.\,В.} О некоторых частных случаях в задаче управления выходом стохастической\linebreak
\\[-12pt]
\hspace*{23pt}дифференциальной системы по квадратичному критерию&1&11--17\\
\Avtors{Босов~А.\,В.} Управление линейным выходом марковской цепи по квадратичному кри-\linebreak
\\[-12pt]
\hspace*{23pt}терию&2&\hphantom{1}3--11\\
\Avtors{Босов~А.\,В., Жуков~Д.\,В.} Экспертная система для мониторинга и прогнозирования\linebreak
\\[-12pt]
\hspace*{23pt}процессов распределения ресурсов&3&29--40\\
\Avtors{Босов~А.\,В., Игнатов~А.\,Н., Наумов~А.\,В.} Алгоритмы приближенного решения задачи\linebreak
\\[-12pt]
\hspace*{23pt}назначения <<технологического окна>> на участках железнодорожной сети&4&\hphantom{1}3--11\\
\Avtors{Брюхов~Д.\,О., Ступников~С.\,А., Ковалёв~Д.\,Ю., Шанин~И.\,А.} Архитектура распределенного\linebreak
\\[-12pt]
\hspace*{23pt}решения задач анализа данных в области нейрофизиологии&1&78--85\\
\Avtors{Власкина~А.\,С.} см.\ Кочеткова~И.\,А.&&\\
\Avtors{Ву~Н.\,Н.} см.\ Кочеткова~И.\,А.&&\\
\Avtors{Вышинский~Л.\,Л., Флёров~Ю.\,А.} Информационная модель весового облика летательных\linebreak
\\[-12pt]
\hspace*{23pt}аппаратов&1&50--56\\
\Avtors{Вышинский~Л.\,Л., Флёров~Ю.\,А.} Теоретические основы формирования весового облика\linebreak
\\[-12pt]
\hspace*{23pt}самолета&4&\hphantom{1}93--102\\
\Avtors{Гаврилов~Е.\,С.} см.\ Абгарян~К.\,К.&&\\
\Avtors{Гончаренко~М.\,Б., Захарова~Т.\,В.} Некоторые свойства смесей нормальных распределений\linebreak
\\[-12pt]
\hspace*{23pt}и~их приложения к задачам магнитоэнцефалографии&2&44--51\\
\Avtors{Гончаров~А.\,А., Зацман~И.\,М.} Принципы структуризации статей в электронных словарях&2&89--95\\
\Avtors{Гончаров~А.\,А., Зацман~И.\,М., Кружков~М.\,Г.} Представление новых лексикографических\linebreak
\\[-12pt]
\hspace*{23pt}знаний в динамических классификационных системах&1&86--93\\
\Avtors{Гончаров~А.\,А., Зацман~И.\,М., Кружков~М.\,Г., Лощилова~Е.\,Ю.} Отражение эволюции\linebreak
\\[-12pt]
\hspace*{23pt}лексикографических знаний в~динамических классификационных системах&4&41--49\\
\Avtors{Гончаров~А.\,А., Инькова~О.\,Ю.} Извлечение знаний о средствах выражения логико-\linebreak
\\[-12pt]
\hspace*{23pt}се\-ман\-ти\-че\-ских отношений при помощи надкорпусной базы данных&2&\hphantom{1}96--103\\
\Avtors{Горшенин~А.\,К., Кузьмин~В.\,Ю.} Метод повышения точности нейросетевых прогнозов с использованием смешанных вероятностных моделей и его реализация в виде\linebreak
\\[-12pt]
\hspace*{23pt}цифрового сервиса&3&63--74\\
\Avtors{Гребенькова~О.\,С., Бахтеев~О.\,Ю., Стрижов~В.\,В.} Вариационная оптимизация модели\linebreak
\\[-12pt]
\hspace*{23pt}глубокого обучения с контролем сложности&1&42--49\\
\end{tabular}
}

\pagebreak

\def\leftkol{АВТОРСКИЙ УКАЗАТЕЛЬ ЗА 2021 г.} % ENGLISH ABSTRACTS}

\def\rightkol{АВТОРСКИЙ УКАЗАТЕЛЬ ЗА 2021 г.} %ENGLISH ABSTRACTS}

%\thispagestyle{myheadings}
\def\leftfootline{\small{\textbf{\thepage}
\hfill ИНФОРМАТИКА И ЕЁ ПРИМЕНЕНИЯ\ \ \ том~15\ \ \ выпуск~4\ \ \ 2021}
}%
 \def\rightfootline{\small{ИНФОРМАТИКА И ЕЁ ПРИМЕНЕНИЯ\ \ \ том~15\ \ \ выпуск~4\ \ \ 2021
 \hfill \textbf{\thepage}}}


\noindent
{\tabcolsep=3pt
\begin{tabular}{p{394pt}cc}
&\textbf{Вып.} & \textbf{Стр.}\\[3pt]
\Avtors{Гринченко~С.\,Н.} Антропогенная <<третья>> природа: относительно автономный статус\linebreak
\\[-12pt]
\hspace*{23pt}ее искусственных интеллектуальных субъектов&4&110--114\\
\Avtors{Гринченко~С.\,Н.} О системной иерархии искусственного интеллекта&1&111--115\\
\Avtors{Грушо~А.\,А., Грушо~Н.\,А., Забежайло~М.\,И., Смирнов~Д.\,В., Тимонина~Е.\,Е., Шоргин~С.\,Я.} Статистика и кластеры в~поисках аномальных вкраплений в~условиях больших\linebreak
\\[-12pt]
\hspace*{23pt}данных&4&79--86\\
\Avtors{Грушо~А.\,А., Грушо~Н.\,А., Забежайло~М.\,И., Тимонина~Е.\,Е.} Удаленный мониторинг\linebreak
\\[-12pt]
\hspace*{23pt}рабочих процессов&3&2--8\\
\Avtors{Грушо~А.\,А., Забежайло~М.\,И., Смирнов~Д.\,В., Тимонина~Е.\,Е.} Интеллектуальный анализ\linebreak
\\[-12pt]
\hspace*{23pt}пополняемых коллекций Big Data в режиме процессно-реального времени&2&36--40\\
\Avtors{Грушо~Н.\,А.} см.\ Грушо~А.\,А.&&\\
\Avtors{Грушо~Н.\,А.} см.\ Грушо~А.\,А.&&\\
\Avtors{Дараселия~А.\,В., Сопин~Э.\,С., Молчанов~Д.\,А., Самуйлов~К.\,Е.} Анализ стратегии разгрузки\linebreak
\\[-12pt]
\hspace*{23pt}базовых станций 5G NR с помощью технологии NR-U&3&98--111\\
\Avtors{Дорофеева~А.\,В.} см.\ Королев~В.\,Ю.&&\\
\Avtors{Дьяченко~Ю.\,Г.} см.\ Соколов~И.\,А.&&\\
\Avtors{Дюкова~Е.\,В., Масляков~Г.\,О.} О выборе частичных порядков на множествах значений\linebreak
\\[-12pt]
\hspace*{23pt}признаков в~задаче классификации&4&72--78\\
\Avtors{Егорова~А.\,Ю.} см.\ Нуриев~В.\,А.&&\\
\Avtors{Жуков~Д.\,В.} см.\ Босов~А.\,В.&&\\
\Avtors{Жуков~Д.\,О., Хватова~Т.\,Ю., Зальцман~А.\,Д.} Моделирование стохастической динамики изменения состояний узлов и~перколяционных переходов в~социальных сетях\linebreak
\\[-12pt]
\hspace*{23pt}с~учетом самоорганизации и наличия памяти&1&102--110\\
\Avtors{Забежайло~М.\,И.} см.\ Грушо~А.\,А.&&\\
\Avtors{Забежайло~М.\,И.} см.\ Грушо~А.\,А.&&\\
\Avtors{Забежайло~М.\,И.} см.\ Грушо~А.\,А.&&\\
\Avtors{Зальцман~А.\,Д.} см.\ Жуков~Д.\,О.&&\\
\Avtors{Захарова~Т.\,В.} см.\ Гончаренко~М.\,Б.&&\\
\Avtors{Зацман~И.\,М.} Концепция создания ВОЗ-центра компетенций по пандемиям и~эпиде-\linebreak
\\[-12pt]
\hspace*{23pt}миям: ключевые понятия и~их терминологический анализ&4&103--109\\
\Avtors{Зацман~И.\,М.} Проблемно-ориентированная актуализация словарных статей двуязыч-\linebreak
\\[-12pt]
\hspace*{23pt}ных словарей и медицинской терминологии: сопоставительный анализ&1&\hphantom{1}94--101\\
\Avtors{Зацман~И.\,М.} Формы представления нового знания, извлеченного из текстов&3&83--90\\
\Avtors{Зацман~И.\,М.} см.\ Гончаров~А.\,А.&&\\
\Avtors{Зацман~И.\,М.} см.\ Гончаров~А.\,А.&&\\
\Avtors{Зацман~И.\,М.} см.\ Гончаров~А.\,А.&&\\
\Avtors{Зейфман~А.\,И., Сатин~Я.\,А., Ковалёв~И.\,А.} Об одной нестационарной модели обслужи-\linebreak
\\[-12pt]
\hspace*{23pt}вания с катастрофами и тяжелыми хвостами&2&20--25\\
\Avtors{Игнатов~А.\,Н.} см.\ Босов~А.\,В.&&\\
\Avtors{Инькова~О.\,Ю., Кружков~М.\,Г.} Структурированные определения дискурсивных отно-\linebreak
\\[-12pt]
\hspace*{23pt}шений в~надкорпусной базе данных коннекторов&4&27--32\\
\Avtors{Инькова~О.\,Ю.} см.\ Гончаров~А.\,А.&&\\
\Avtors{Истратов~Л.\,А.} см.\ Сигов~А.\,С.&&\\
\Avtors{Казанчян~Д.\,Х.} см.\ Борисов~А.\,В.&&\\
\Avtors{Казанчян~Д.\,Х.} см.\ Борисов~А.\,В.&&\\
\Avtors{Каменских~А.\,Н.} см.\ Соколов~И.\,А.&&\\
\Avtors{Кириков~И.\,А., Листопад~С.\,В.} Согласование целей агентов сплоченных гибридных\linebreak
\\[-12pt]
\hspace*{23pt}интеллектуальных многоагентных систем&2&66--71\\
\Avtors{Кириков~И.\,А.} см.\ Румовская~С.\,Б.&&\\
\Avtors{Ковалёв~Д.\,Ю.}см.\ Брюхов~Д.\,О.&&\\
\Avtors{Ковалёв~И.\,А.} см.\ Зейфман~А.\,И.&&\\
\Avtors{Ковалёв~С.\,П.} Методы теории категорий в цифровом проектировании гетерогенных\linebreak
\\[-12pt]
\hspace*{23pt}киберфизических систем&1&23--29\\
\end{tabular}
}

\pagebreak

\def\leftkol{АВТОРСКИЙ УКАЗАТЕЛЬ ЗА 2021 г.} % ENGLISH ABSTRACTS}

\def\rightkol{АВТОРСКИЙ УКАЗАТЕЛЬ ЗА 2021 г.} %ENGLISH ABSTRACTS}

%\thispagestyle{myheadings}
\def\leftfootline{\small{\textbf{\thepage}
\hfill ИНФОРМАТИКА И ЕЁ ПРИМЕНЕНИЯ\ \ \ том~15\ \ \ выпуск~4\ \ \ 2021}
}%
 \def\rightfootline{\small{ИНФОРМАТИКА И ЕЁ ПРИМЕНЕНИЯ\ \ \ том~15\ \ \ выпуск~4\ \ \ 2021
 \hfill \textbf{\thepage}}}


\noindent
{\tabcolsep=3pt
\begin{tabular}{p{394pt}cc}
&\textbf{Вып.} & \textbf{Стр.}\\[3pt]
\Avtors{Коновалов~М.\,Г., Разумчик~Р.\,В.} Диспетчеризация в системе с параллельным обслужи-\linebreak
\\[-12pt]
\hspace*{23pt}ванием с помощью распределенного градиентного управления марковской цепью&3&41--50\\
\Avtors{Королев~В.\,Ю., Дорофеева~А.\,В.} О точности нормальной аппроксимации при отсутствии\linebreak
\\[-12pt]
\hspace*{23pt}нормальной сходимости&1&116--121\\
\Avtors{Кочеткова~И.\,А., Власкина~А.\,С., Ву~Н.\,Н., Шоргин~В.\,С.} Система массового обслуживания с управляемым по сигналам перераспределением приборов для анализа\linebreak
\\[-12pt]
\hspace*{23pt}нарезки ресурсов сети 5G&3&91--97\\
\Avtors{Кочеткова~И.\,А., Кущазли~А.\,И., Харин~П.\,А., Шоргин~С.\,Я.} Модель схемы приоритетного доступа 
трафика URLLC и~eMBB в~сети пятого поколения в~виде ресурсной\linebreak
\\[-12pt]
\hspace*{23pt}системы массового обслуживания&4&87--92\\
\Avtors{Кривенко~М.\,П.} Мягкие вычисления в задачах медицинской диагностики&2&52--59\\
\Avtors{Кружков~М.\,Г.} см.\ Гончаров~А.\,А.&&\\
\Avtors{Кружков~М.\,Г.} см.\ Гончаров~А.\,А.&&\\
\Avtors{Кружков~М.\,Г.} см.\ Инькова~О.\,Ю.&&\\
\Avtors{Кудрявцев~А.\,А., Шестаков~О.\,В.} Минимаксные оценки функции потерь, основанной на интегральных вероятностях ошибок при пороговой обработке вейвлет-\linebreak
\\[-12pt]
\hspace*{23pt}коэффициентов&4&12--19\\
\Avtors{Кудрявцев~А.\,А., Шестаков~О.\,В., Шоргин~С.\,Я.} Метод оценивания параметров изгиба,\linebreak
\\[-12pt]
\hspace*{23pt}формы и масштаба гамма-экспоненциального распределения&3&57--62\\
\Avtors{Кудрявцев~А.\,А.} см.\ Арутюнов~Е.\,Н.&&\\
\Avtors{Кузнецова~Р.\,В., Бахтеев~О.\,Ю., Чехович~Ю.\,В.} Методы обнаружения переводных\linebreak
\\[-12pt]
\hspace*{23pt}заимствований в больших текстовых коллекциях&1&30--41\\
\Avtors{Кузьмин~В.\,Ю.} см.\ Горшенин~А.\,К.&&\\
\Avtors{Кущазли~А.\,И.} см.\ Кочеткова~И.\,А.&&\\
\Avtors{Липатьев~А.\,А.} Неасимптотический анализ статистики Бартлетта--Нанда--Пилая для\linebreak
\\[-12pt]
\hspace*{23pt}данных большой размерности&1&72--77\\
\Avtors{Листопад~С.\,В.} см.\ Кириков~И.\,А.&&\\
\Avtors{Лощилова~Е.\,Ю.} см.\ Гончаров~А.\,А&&\\
\Avtors{Малашенко~Ю.\,Е.} Максимальные межузловые потоки при предельной загрузке много-\linebreak
\\[-12pt]
\hspace*{23pt}пользовательской сети&3&24--28\\
\Avtors{Малашенко~Ю.\,Е., Назарова~И.\,А.} Анализ распределения предельных нагрузок в~мно-\linebreak
\\[-12pt]
\hspace*{23pt}гопользовательской сети&4&20--26\\
\Avtors{Масляков~Г.\,О.} см.\ Дюкова~Е.\,В.&&\\
\Avtors{Молчанов~Д.\,А.} см.\ Дараселия~А.\,В.&&\\
\Avtors{Монахов~М.\,М.} Разложения Чебышёва--Эджворта для распределений обобщенных\linebreak
\\[-12pt]
\hspace*{23pt}статистик типа Хотеллинга, построенных по выборкам случайного размера&2&72--81\\
\Avtors{Назарова~И.\,А.} см.\ Малашенко~Ю.\,Е.&&\\
\Avtors{Наумов~А.\,В.} см.\ Босов~А.\,В.&&\\
\Avtors{Недоливко~Ю.\,Н.} см.\ Арутюнов~Е.\,Н.&&\\
\Avtors{Нуриев~В.\,А., Егорова~А.\,Ю.} Методы оценки качества машинного перевода: современное\linebreak
\\[-12pt]
\hspace*{23pt}состояние&2&104--111\\
\Avtors{Павлов~Ю.\,Л.} Связность конфигурационных графов в моделях сложных сетей&1&18--22\\
\Avtors{Разумчик~Р.\,В.} см.\ Коновалов~М.\,Г.&&\\
\Avtors{Рождественский~Ю.\,В.} см.\ Соколов~И.\,А.&&\\
\Avtors{Румовская~С.\,Б., Кириков~И.\,А.} Метод визуализации стимуляции конфликтов в гибрид-\linebreak
\\[-12pt]
\hspace*{23pt}ных интеллектуальных многоагентных системах&3&75--82\\
\Avtors{Самуйлов~К.\,Е.} см.\ Дараселия~А.\,В.&&\\
\Avtors{Сатин~Я.\,А.} см.\ Зейфман~А.\,И.&&\\
\Avtors{Севастьянов~Л.\,А.} см.\ Щетинин~Е.\,Ю.&&\\
\Avtors{Сигов~А.\,С., Андрианова~Е.\,Г., Истратов~Л.\,А.} Стохастическая динамика самоорганизу-\linebreak
\\[-12pt]
\hspace*{23pt}ющихся социальных систем с памятью (электоральные процессы)&2&112--121\\
\Avtors{Синицын~И.\,Н.} Нормальные субоптимальные фильтры для дифференциальных стоха-\linebreak
\\[-12pt]
\hspace*{23pt}стических систем, не разрешенных относительно производных&1&\hphantom{1}3--10\\
\end{tabular}
}

\pagebreak

\def\leftkol{АВТОРСКИЙ УКАЗАТЕЛЬ ЗА 2021 г.} % ENGLISH ABSTRACTS}

\def\rightkol{АВТОРСКИЙ УКАЗАТЕЛЬ ЗА 2021 г.} %ENGLISH ABSTRACTS}

%\thispagestyle{myheadings}
\def\leftfootline{\small{\textbf{\thepage}
\hfill ИНФОРМАТИКА И ЕЁ ПРИМЕНЕНИЯ\ \ \ том~15\ \ \ выпуск~4\ \ \ 2021}
}%
 \def\rightfootline{\small{ИНФОРМАТИКА И ЕЁ ПРИМЕНЕНИЯ\ \ \ том~15\ \ \ выпуск~4\ \ \ 2021
 \hfill \textbf{\thepage}}}


\noindent
{\tabcolsep=3pt
\begin{tabular}{p{394pt}cc}
&\textbf{Вып.} & \textbf{Стр.}\\[3pt]
\Avtors{Смирнов~Д.\,В.} см.\ Грушо~А.\,А.&&\\
\Avtors{Смирнов~Д.\,В.} см.\ Грушо~А.\,А.&&\\
\Avtors{Соколов~И.\,А., Степченков~Ю.\,А., Дьяченко~Ю.\,Г., Рождественский~Ю.\,В., Каменских~А.\,Н.}\linebreak
\\[-12pt]
\hspace*{23pt}Базис реализации сбоеустойчивых электронных схем&4&65--71\\
\Avtors{Сопин~Э.\,С.} см.\ Дараселия~А.\,В.&&\\
\Avtors{Степченков~Ю.\,А.} см.\ Соколов~И.\,А.&&\\
\Avtors{Стрижов~В.\,В.} см.\ Гребенькова~О.\,С.&&\\
\Avtors{Ступников~С.\,А.}см.\ Брюхов~Д.\,О.&&\\
\Avtors{Сушко~Д.\,В.} Алгоритмы сжатия данных массивов силовых кривых I: кодирование\linebreak
\\[-12pt]
\hspace*{23pt}ошибок предсказания&2&82--88\\
\Avtors{Сушко~Д.\,В.} Алгоритмы сжатия данных массивов силовых кривых II: кодирование\linebreak
\\[-12pt]
\hspace*{23pt}компонент вейвлет-преобразования&3&16--23\\
\Avtors{Тимонина~Е.\,Е.} см.\ Грушо~А.\,А.&&\\
\Avtors{Тимонина~Е.\,Е.} см.\ Грушо~А.\,А.&&\\
\Avtors{Тимонина~Е.\,Е.} см.\ Грушо~А.\,А.&&\\
\Avtors{Ушаков~В.\,Г., Ушаков~Н.\,Г.} Многомерные распределения выходящих потоков в системе\linebreak
\\[-12pt]
\hspace*{23pt}с абсолютным приоритетом&2&26--29\\
\Avtors{Ушаков~Н.\,Г.} см.\ Ушаков~В.\,Г.&&\\
\Avtors{Флёров~Ю.\,А.} см.\ Вышинский~Л.\,Л.&&\\
\Avtors{Флёров~Ю.\,А.} см.\ Вышинский~Л.\,Л.&&\\
\Avtors{Харин~П.\,А.} см.\ Кочеткова~И.\,А.&&\\
\Avtors{Хватова~Т.\,Ю.} см.\ Жуков~Д.\,О.&&\\
\Avtors{Чехович~Ю.\,В.} см.\ Кузнецова~Р.\,В.&&\\
\Avtors{Шанин~И.\,А.}см.\ Брюхов~Д.\,О.&&\\
\Avtors{Шестаков~О.\,В.} Анализ несмещенной оценки среднеквадратичного риска метода блоч-\linebreak
\\[-12pt]
\hspace*{23pt}ной пороговой обработки&2&30--35\\
\Avtors{Шестаков~О.\,В.} Пороговые функции в методах подавления шума, основанных на\linebreak
\\[-12pt]
\hspace*{23pt}вейвлет-разложении сигнала&3&51--56\\
\Avtors{Шестаков~О.\,В.} см.\ Кудрявцев~А.\,А.&&\\
\Avtors{Шестаков~О.\,В.} см.\ Кудрявцев~А.\,А.&&\\
\Avtors{Шнурков~П.\,В.} Создание стохастической динамической односекторной экономической модели с~дискретным временем и~анализ соответствующей задачи оптимального\linebreak
\\[-12pt]
\hspace*{23pt}управления&4&33--40\\
\Avtors{Шоргин~В.\,С.} см.\ Кочеткова~И.\,А.&&\\
\Avtors{Шоргин~С.\,Я.} см.\ Грушо~А.\,А.&&\\
\Avtors{Шоргин~С.\,Я.} см.\ Кочеткова~И.\,А.&&\\
\Avtors{Шоргин~С.\,Я.} см.\ Кудрявцев~А.\,А.&&\\
\Avtors{Щетинин~Е.\,Ю., Севастьянов~Л.\,А.} О методах переноса глубокого обучения в~задачах\linebreak
\\[-12pt]
\hspace*{23pt}классификации биомедицинских изображений&4&59--64\\
\end{tabular}
}

%\thispagestyle{myheadings}
\def\leftfootline{\small{\textbf{\thepage}
\hfill ИНФОРМАТИКА И ЕЁ ПРИМЕНЕНИЯ\ \ \ том~15\ \ \ выпуск~4\ \ \ 2021}
}%
 \def\rightfootline{\small{ИНФОРМАТИКА И ЕЁ ПРИМЕНЕНИЯ\ \ \ том~15\ \ \ выпуск~4\ \ \ 2021
 \hfill \textbf{\thepage}}}

 \label{end\stat}

\newpage

\def\stat{cont-e}
{%\hrule\par
%\vskip 7pt % 7pt
\raggedleft\Large \bf%\baselineskip=3.2ex
2\,0\,2\,1\ \ A\,U\,T\,H\,O\,R\ \ I\,N\,D\,E\,X \vskip 17pt
 \hrule
 \par
\vskip 21pt plus 6pt minus 3pt }

\label{st\stat}

\def\tit{\ }

\def\aut{\ }
\def\auf{\ }

\def\leftkol{\ } %2021 AUTHOR INDEX} % ENGLISH ABSTRACTS}

\def\rightkol{\ } %2021 AUTHOR INDEX} %ENGLISH ABSTRACTS}

\titele{\tit}{\aut}{\auf}{\leftkol}{\rightkol}
\addcontentsline{toc}{subsection}{\textrm\textbf 2021 Author Index}

\def\leftfootline{\small{\textbf{\thepage}
\hfill INFORMATIKA I EE PRIMENENIYA~--- INFORMATICS AND APPLICATIONS\ \ \ 2021\
\ \ volume~15\ \ \ issue\ 4}
}%
 \def\rightfootline{\small{INFORMATIKA I EE PRIMENENIYA~--- INFORMATICS AND APPLICATIONS\ \ \ 2021\ \ \ volume~15\ \ \ issue\ 4
\hfill \textbf{\thepage}}}

\vspace*{-24pt}

\noindent
{\tabcolsep=3pt
\begin{tabular}{p{395.89pt}cc}
&\textbf{Issue} & \textbf{Page}\\[6pt]
\Avtors{Abgaryan~K.\,K. and Gavrilov~E.\,S.} Distributed information system for calculating the structural\linebreak
\\[-12pt]
\hspace*{23pt}properties of composite materials&4&50--58\\
\Avtors{Agalarov~Ya.\,M.} Optimal threshold-based admission control in the $M/M/s$ system with hetero-\linebreak
\\[-12pt]
\hspace*{23pt}geneous servers and a common queue&1&57--64\\
\Avtors{Andrianova~E.\,G.} see Sigov~A.\,S.&&\\
\Avtors{Arutyunov~E.\,N., Kudryavtsev~A.\,A., and~Nedolivko~Iu.\,N.} Probabilistic characteristics of balance\linebreak
\\[-12pt]
\hspace*{23pt}index of factors with generalized gamma distribution&1&65--71\\
\Avtors{Bakhteev~O.\,Yu.} see Grebenkova~O.\,S.&&\\
\Avtors{Bakhteev~O.\,Yu.} see Kuznetsova~R.\,V.&&\\
\Avtors{Bazilevskiy~M.\,P.} Method of straightening distorted due to multicollinearity coefficients in\linebreak
\\[-12pt]
\hspace*{23pt}regression models&2&60--65\\
\Avtors{Borisov~A.\,V.\ and~Kazanchyan~D.\,Kh.} Filtering of Markov jump processes given composite\linebreak
\\[-12pt]
\hspace*{23pt}ob\-ser\-va\-tions I: Exact solution&2&12--19\\
\Avtors{Borisov~A.\,V.\ and~Kazanchyan~D.\,Kh.} Filtering of Markov jump processes given composite\linebreak
\\[-12pt]
\hspace*{23pt}observations II: Numerical algorithm&3&\hphantom{1}9--15\\
\Avtors{Bosov~A.\,V.} Linear output control of Markov chains by the quadratic criterion&2&\hphantom{1}3--11\\
\Avtors{Bosov~A.\,V.} On some special cases in the problem of stochastic differential system output control\linebreak
\\[-12pt]
\hspace*{23pt}by the quadratic criterion&1&11--17\\
\Avtors{Bosov~A.\,V., Ignatov~A.\,N., and Naumov~A.\,V.} Algorithms for an approximate solution of the\linebreak
\\[-12pt]
\hspace*{23pt}track possession problem on the railway network segment&4&\hphantom{1}3--11\\
\Avtors{Bosov~A.\,V.\ and~Zhukov~D.\,V.} Expert system for monitoring and forecasting of resource allocation\linebreak
\\[-12pt]
\hspace*{23pt}processes&3&29--40\\
\Avtors{Briukhov~D.\,O., Stupnikov~S.\,A., Kovalev~D.\,Yu., and~Shanin~I.\,A.} An architecture for distributed\linebreak
\\[-12pt]
\hspace*{23pt}data analysis problem solving in neurophysiology&1&78--85\\
\Avtors{Chekhovich~Yu.\,V.} see Kuznetsova~R.\,V.&&\\
\Avtors{Daraseliya~А.\,V., Sopin~E.\,S., Moltchanov~D.\,А., and~Samouylov~K.\,E.} Analysis of 5G NR base\linebreak
\\[-12pt]
\hspace*{23pt}stations offloading by means of NR-U technology&3&98--111\\
\Avtors{Diachenko~Yu.\,G.} see Sokolov~I.\,A.&&\\
\Avtors{Djukova~E.\,V. and Masliakov~G.\,O.} On the choice of partial orders on feature values sets in the\linebreak
\\[-12pt]
\hspace*{23pt}supervised classification problem&4&72--78\\
\Avtors{Dorofeeva~A.\,V.} see Korolev~V.\,Yu.&&\\
\Avtors{Egorova~A.\,Yu.} see Nuriev~V.\,A.&&\\
\Avtors{Flerov~Yu.\,A.} see Vyshinsky~L.\,L.&&\\
\Avtors{Flerov~Yu.\,A.} see Vyshinsky~L.\,L.&&\\
\Avtors{Gavrilov~E.\,S.} see Abgaryan~K.\,K.&&\\
\Avtors{Goncharenko~M.\,B.\ and~Zakharova~T.\,V.} Some properties of Gaussian mixtures and applications\linebreak
\\[-12pt]
\hspace*{23pt}to magnetoencephalography problems&2&44--51\\
\Avtors{Goncharov~A.\,A.\ and~Inkova~O.\,Yu.} Extracting knowledge about means of expression of\linebreak
\\[-12pt]
\hspace*{23pt}logical-semantic relations from the supracorpora database&2&\hphantom{1}96--103\\
\Avtors{Goncharov~A.\,A.\ and~Zatsman~I.\,M.} Structuring principles of electronic dictionary's entries&2&89--95\\
\Avtors{Goncharov~A.\,A., Zatsman~I.\,M., and~Kruzhkov~M.\,G.} Representation of new lexicographical\linebreak
\\[-12pt]
\hspace*{23pt}knowledge in dynamic classification systems&1&86--93\\
\Avtors{Goncharov~A.\,A., Zatsman~I.\,M., Kruzhkov~M.\,G., and Loshchilova~E.\,Yu.} Capturing evolution\linebreak
\\[-12pt]
\hspace*{23pt}of lexicographic knowledge in dynamic classification systems&4&41--49\\

\Avtors{Gorshenin~A.\,K.\ and~Kuzmin~V.\,Yu.} Method for improving accuracy of neural network forecasts\linebreak
\\[-12pt]
\hspace*{23pt}based on probability mixture models and its implementation as a digital service&3&63--74\\
\end{tabular}
}
\pagebreak

\def\leftfootline{\small{\textbf{\thepage}
\hfill INFORMATIKA I EE PRIMENENIYA~--- INFORMATICS AND APPLICATIONS\ \ \ 2021\
\ \ volume~15\ \ \ issue\ 4}
}%
 \def\rightfootline{\small{INFORMATIKA I EE PRIMENENIYA~---
INFORMATICS AND APPLICATIONS\ \ \ 2021\ \ \ volume~15\ \ \ issue\ 4
\hfill \textbf{\thepage}}}

\def\leftkol{2021 AUTHOR INDEX} % ENGLISH ABSTRACTS}

\def\rightkol{2021 AUTHOR INDEX} %ENGLISH ABSTRACTS}


\noindent
{\tabcolsep=3pt
\begin{tabular}{p{395.5pt}cc}
&\textbf{Issue} & \textbf{Page}\\[6pt]
\Avtors{Grebenkova~O.\,S., Bakhteev~O.\,Yu., and~Strijov~V.\,V.} Variational deep learning model optimi-\linebreak
\\[-12pt]
\hspace*{23pt}zation with complexity control&1&42--49\\[-.15pt]
\Avtors{Grinchenko~S.\,N.} Anthropogenic ``third'' nature: The relatively autonomous status of its artificial\linebreak
\\[-12pt]
\hspace*{23pt}intellectual subjects&4&110--114\\[-.15pt]
\Avtors{Grinchenko~S.\,N.} On the system hierarchy of artificial intelligence&1&111--115\\[-.15pt]
\Avtors{Grusho~A.\,A., Grusho~N.\,A., Zabezhailo~M.\,I., Smirnov~D.\,V., Timonina~E.\,E., and Shorgin~S.\,Ya.}\linebreak
\\[-12pt]
\hspace*{23pt}Statistics and clusters for detection of anomalous insertions in Big Data
en\-vi\-ron\-ment&4&79--86\\[-.15pt]
\Avtors{Grusho~A.\,A., Grusho~N.\,A., Zabezhailo~M.\,I., and~Timonina~E.\,E.} Remote monitoring of\linebreak
\\[-12pt]
\hspace*{23pt}workflows&3&2--8\\[-.15pt]
\Avtors{Grusho~A.\,A., Zabezhailo~M.\,I., Smirnov~D.\,V., and~Timonina~E.\,E.} Intelligent analysis of Big\linebreak
\\[-12pt]
\hspace*{23pt}Data extendible collections under the limits of process-real time&2&36--43\\[-.15pt]
\Avtors{Grusho~N.\,A.} see Grusho~A.\,A.&&\\[-.15pt]
\Avtors{Grusho~N.\,A.} see Grusho~A.\,A.&&\\[-.15pt]
\Avtors{Ignatov~A.\,N.} see Bosov~A.\,V.&&\\[-.15pt]
\Avtors{Inkova~O.\,Yu.~and Kruzhkov~M.\,G.} Structured definitions of discourse relations in the\linebreak
\\[-12pt]
\hspace*{23pt}Supracorpora Database of Connectives&4&27--32\\[-.15pt]
\Avtors{Inkova~O.\,Yu.} see Goncharov~A.\,A.&&\\[-.15pt]
\Avtors{Istratov~L.\,A.} see Sigov~A.\,S.&&\\[-.15pt]
\Avtors{Kamenskih~A.\,N.} see Sokolov~I.\,A.&&\\[-.15pt]
\Avtors{Kazanchyan~D.\,Kh.} see Borisov~A.\,V.&&\\[-.15pt]
\Avtors{Kazanchyan~D.\,Kh.} see Borisov~A.\,V.&&\\[-.15pt]
\Avtors{Kharin~P.\,A.} see Kochetkova~I.\,A.&&\\[-.15pt]
\Avtors{Khvatova~T.\,Yu.} see Zhukov~D.\,O.&&\\[-.15pt]
\Avtors{Kirikov~I.\,A.\ and~Listopad~S.\,V.} Coordination of agents' goals in cohesive hybrid intelligent\linebreak
\\[-12pt]
\hspace*{23pt}multiagent systems&2&66--71\\[-.15pt]
\Avtors{Kirikov~I.\,A.} see Rumovskaya~S.\,B.&&\\[-.15pt]
\Avtors{Kochetkova~I.\,A., Kushchazli~A.\,I., Kharin~P.\,A., and Shorgin~S.\,Ya.} Model for analyzing priority 
admission control of URLLC and eMBB communications in 5G networks as a~resource\linebreak
\\[-12pt]
\hspace*{23pt}queuing system&4&87--92\\[-.15pt]
\Avtors{Kochetkova~I.\,A., Vlaskina~A.\,S., Vu~N.\,N., and~Shorgin~V.\,S.} Queuing system with signals for\linebreak
\\[-12pt]
\hspace*{23pt}dynamic resource allocation for analyzing network slicing in 5G networks&3&91--97\\[-.15pt]
\Avtors{Konovalov~M.\,G.\ and~Razumchik~R.\,V.} Routing jobs to heterogeneous parallel queues using\linebreak
\\[-12pt]
\hspace*{23pt}distributed policy grandient algorithm&3&41--50\\[-.15pt]
\Avtors{Korolev~V.\,Yu.\ and~Dorofeeva~A.\,V.} On the accuracy of the normal approximation under the\linebreak
\\[-12pt]
\hspace*{23pt}violation of the normal convergence&1&116--121\\[-.15pt]
\Avtors{Kovalev~D.\,Yu.} see Briukhov~D.\,O.&&\\[-.15pt]
\Avtors{Kovalev~I.\,A.} see Zeifman~A.\,I.&&\\[-.15pt]
\Avtors{Kovalyov~S.\,P.} Methods of the category theory in digital design of heterogeneous cyber-physical\linebreak
\\[-12pt]
\hspace*{23pt}systems&1&23--29\\[-.15pt]
\Avtors{Krivenko~M.\,P.} Soft computing in problems of medical diagnostics&2&52--59\\[-.15pt]
\Avtors{Kruzhkov~M.\,G.} see Goncharov~A.\,A&&\\[-.15pt]
\Avtors{Kruzhkov~M.\,G.} see Goncharov~A.\,A.&&\\[-.15pt]
\Avtors{Kruzhkov~M.\,G.} see Inkova~O.\,Yu.&&\\[-.15pt]
\Avtors{Kudryavtsev~A.\,A. and Shestakov~O.\,V.} Minimax estimates of the loss function based on integral\linebreak
\\[-12pt]
\hspace*{23pt}error probabilities during threshold processing of wavelet coefficients&4&12--19\\[-.15pt]
\Avtors{Kudryavtsev~A.\,A., Shestakov~O.\,V., and~Shorgin~S.\,Ya.} A~method for estimating bent, shape and\linebreak
\\[-12pt]
\hspace*{23pt}scale parameters of the gamma-exponential distribution&3&57--62\\[-.15pt]
\Avtors{Kudryavtsev~A.\,A.} see Arutyunov~E.\,N.&&\\[-.15pt]
\Avtors{Kushchazli~A.\,I.} see Kochetkova~I.\,A.&&\\[-.15pt]
\Avtors{Kuzmin~V.\,Yu.} see Gorshenin~A.\,K.&&\\[-.15pt]
\Avtors{Kuznetsova~R.\,V., Bakhteev~O.\,Yu., and~Chekhovich~Yu.\,V.} Methods of cross-lingual text reuse\linebreak
\\[-12pt]
\hspace*{23pt}detection in large textual collections&1&30--41\\[-.15pt]
\Avtors{Lipatiev~A.\,A.} Nonasymptotic analysis of Bartlett--Nanda--Pillai statistic for high-dimensional\linebreak
\\[-12pt]
\hspace*{23pt}data&1&72--77\\
\end{tabular}
}
\pagebreak

\def\leftfootline{\small{\textbf{\thepage}
\hfill INFORMATIKA I EE PRIMENENIYA~--- INFORMATICS AND APPLICATIONS\ \ \ 2021\
\ \ volume~15\ \ \ issue\ 4}
}%
 \def\rightfootline{\small{INFORMATIKA I EE PRIMENENIYA~---
INFORMATICS AND APPLICATIONS\ \ \ 2021\ \ \ volume~15\ \ \ issue\ 4
\hfill \textbf{\thepage}}}

\def\leftkol{2021 AUTHOR INDEX} % ENGLISH ABSTRACTS}

\def\rightkol{2021 AUTHOR INDEX} %ENGLISH ABSTRACTS}


\noindent
{\tabcolsep=3pt
\begin{tabular}{p{395.5pt}cc}
&\textbf{Issue} & \textbf{Page}\\[6pt]
\Avtors{Listopad~S.\,V.} see Kirikov~I.\,A.&&\\
\Avtors{Loshchilova~E.\,Yu.} see Goncharov~A.\,A.&&\\
\Avtors{Malashenko~Yu.\,E.} Maximum internode flows at peak load of a multiuser network&3&24--28\\
\Avtors{Malashenko~Yu.\,E. and Nazarova~I.\,A.} Analysis of peak load distribution in the multiuser\linebreak
\\[-12pt]
\hspace*{23pt}network&4&20--26\\
\Avtors{Masliakov~G.\,O.} see Djukova~E.\,V.&&\\
\Avtors{Moltchanov~D.\,А.} see Daraseliya~А.\,V.&&\\
\Avtors{Monakhov~M.\,M.} Chebyshev--Edgeworth expansions for distributions of generalised Hotelling-\linebreak
\\[-12pt]
\hspace*{23pt}type statistics based on random size samples&2&72--81\\
\Avtors{Naumov~A.\,V.} see Bosov~A.\,V.&&\\
\Avtors{Nazarova~I.\,A.} see Malashenko~Yu.\,E.&&\\
\Avtors{Nedolivko~Iu.\,N.} see Arutyunov~E.\,N.&&\\
\Avtors{Nuriev~V.\,A.\ and~Egorova~A.\,Yu.} Methods of quality estimation for machine translation:\linebreak
\\[-12pt]
\hspace*{23pt}State-of-the-art&2&104--111\\
\Avtors{Pavlov~Yu.\,L.} Connectivity of configuration graphs in complex network models&1&18--22\\
\Avtors{Razumchik~R.\,V.} see Konovalov~M.\,G.&&\\
\Avtors{Rogdestvenski~Yu.\,V.} see Sokolov~I.\,A.&&\\
\Avtors{Rumovskaya~S.\,B.\ and~Kirikov~I.\,A.} Visual representation method for the conflict stimulation in\linebreak
\\[-12pt]
\hspace*{23pt}hybrid intelligent multiagent systems&3&75--82\\
\Avtors{Samouylov~K.\,E.} see Daraseliya~А.\,V.&&\\
\Avtors{Satin~Ya.\,A.} see Zeifman~A.\,I.&&\\
\Avtors{Sevastianov~L.\,A.} see Shchetinin~E.\,Yu.&&\\
\Avtors{Shanin~I.\,A.} see Briukhov~D.\,O.&&\\
\Avtors{Shchetinin~E.\,Yu. and Sevastianov~L.\,A.} On transfer learning methods in biomedical images\linebreak
\\[-12pt]
\hspace*{23pt}classification tasks&4&59--64\\
\Avtors{Shestakov~O.\,V.} Analysis of the unbiased mean-square risk estimate of the block thresholding\linebreak
\\[-12pt]
\hspace*{23pt}method&2&30--35\\
\Avtors{Shestakov~O.\,V.} Thresholding functions in the noise suppression methods based on the wavelet\linebreak
\\[-12pt]
\hspace*{23pt}expansion of the signal&3&51--56\\
\Avtors{Shestakov~O.\,V.} see Kudryavtsev~A.\,A.&&\\
\Avtors{Shestakov~O.\,V.} see Kudryavtsev~A.\,A.&&\\
\Avtors{Shnurkov~P.\,V.} Creation of a stochastic dynamic one-sector economic model with discrete time\linebreak
\\[-12pt]
\hspace*{23pt}and analysis of the corresponding optimal control problem&4&33--40\\
\Avtors{Shorgin~S.\,Ya.} see Grusho~A.\,A.&&\\
\Avtors{Shorgin~S.\,Ya.} see Kochetkova~I.\,A.&&\\
\Avtors{Shorgin~S.\,Ya.} see Kudryavtsev~A.\,A. &&\\
\Avtors{Shorgin~V.\,S.} see Kochetkova~I.\,A.&&\\
\Avtors{Sigov~A.\,S., Andrianova~E.\,G., and~Istratov~L.\,A.} Stochastic dynamics of self-organizing social\linebreak
\\[-12pt]
\hspace*{23pt}systems with memory (electoral processes)&2&112--121\\
\Avtors{Sinitsyn~I.\,N.} Normal suboptimal filtering for differential stochastic systems with unsolved\linebreak
\\[-12pt]
\hspace*{23pt}derivatives&1&\hphantom{1}3--10\\
\Avtors{Smirnov~D.\,V.} see Grusho~A.\,A.&&\\
\Avtors{Smirnov~D.\,V.} see Grusho~A.\,A.&&\\
\Avtors{Sokolov~I.\,A., Stepchenkov~Yu.\,A., Diachenko~Yu.\,G., Rogdestvenski~Yu.\,V., and Kamenskih~A.\,N.}\linebreak
\\[-12pt]
\hspace*{23pt}The electronic component base of failure resilience digital circuits&4&65--71\\
\Avtors{Sopin~E.\,S.} see Daraseliya~А.\,V.&&\\
\Avtors{Stepchenkov~Yu.\,A.} see Sokolov~I.\,A.&&\\
\Avtors{Strijov~V.\,V.} see Grebenkova~O.\,S.&&\\
\Avtors{Stupnikov~S.\,A.} see Briukhov~D.\,O.&&\\
\Avtors{Sushko~D.\,V.} Compression algorithms for force volume data I: Coding of prediction errors&2&82--88\\
\Avtors{Sushko~D.\,V.} Compression algorithms for force volume data II: Coding of wavelet transform\linebreak
\\[-12pt]
\hspace*{23pt}components&3&16--23\\
\Avtors{Timonina~E.\,E.} see Grusho~A.\,A.&&\\
\end{tabular}
}
\pagebreak

\def\leftfootline{\small{\textbf{\thepage}
\hfill INFORMATIKA I EE PRIMENENIYA~--- INFORMATICS AND APPLICATIONS\ \ \ 2021\
\ \ volume~15\ \ \ issue\ 4}
}%
 \def\rightfootline{\small{INFORMATIKA I EE PRIMENENIYA~---
INFORMATICS AND APPLICATIONS\ \ \ 2021\ \ \ volume~15\ \ \ issue\ 4
\hfill \textbf{\thepage}}}

\def\leftkol{2021 AUTHOR INDEX} % ENGLISH ABSTRACTS}

\def\rightkol{2021 AUTHOR INDEX} %ENGLISH ABSTRACTS}


\noindent
{\tabcolsep=3pt
\begin{tabular}{p{395.5pt}cc}
&\textbf{Issue} & \textbf{Page}\\[6pt]
\Avtors{Timonina~E.\,E.} see Grusho~A.\,A.&&\\
\Avtors{Timonina~E.\,E.} see Grusho~A.\,A.&&\\
\Avtors{Ushakov~N.\,G.} see Ushakov~V.\,G.&&\\
\Avtors{Ushakov~V.\,G.\ and~Ushakov~N.\,G.} The multivariate distributions of output streams in a queueing\linebreak
\\[-12pt]
\hspace*{23pt}system with preemptive repeat priority&2&26--29\\
\Avtors{Vlaskina~A.\,S.} see Kochetkova~I.\,A.&&\\
\Avtors{Vu~N.\,N.} see Kochetkova~I.\,A.&&\\
\Avtors{Vyshinsky~L.\,L.\ and~Flerov~Yu.\,A.} Information model of aircraft weight profile&1&50--56\\
\Avtors{Vyshinsky~L.\,L. and Flerov~Yu.\,A.} Theoretical foundation of formation of aircraft weight\linebreak
\\[-12pt]
\hspace*{23pt}appearance&4&\hphantom{1}93--102\\
\Avtors{Zabezhailo~M.\,I.} see Grusho~A.\,A.&&\\
\Avtors{Zabezhailo~M.\,I.} see Grusho~A.\,A.&&\\
\Avtors{Zabezhailo~M.\,I.} see Grusho~A.\,A.&&\\
\Avtors{Zakharova~T.\,V.} see Goncharenko~M.\,B.&&\\
\Avtors{Zaltcman~A.\,D.} see Zhukov~D.\,O.&&\\
\Avtors{Zatsman~I.\,M.} Forms representing new knowledge discovered in texts&3&83--90\\
\Avtors{Zatsman~I.\,M.} Problem-oriented updating of dictionary entries of bilingual dictionaries and\linebreak
\\[-12pt]
\hspace*{23pt}medical terminology: Comparative analysis&1&\hphantom{1}94--101\\
\Avtors{Zatsman~I.\,M.} The conception of creating WHO Hub for Pandemic and Epidemic Intelligence:\linebreak
\\[-12pt]
\hspace*{23pt}Keywords and their terminological analysis&4&103--109\\
\Avtors{Zatsman~I.\,M.} see Goncharov~A.\,A.&&\\
\Avtors{Zatsman~I.\,M.} see Goncharov~A.\,A.&&\\
\Avtors{Zatsman~I.\,M.} see Goncharov~A.\,A.&&\\
\Avtors{Zeifman~A.\,I., Satin~Ya.\,A., and~Kovalev~I.\,A.} On one nonstationary service model with\linebreak
\\[-12pt]
\hspace*{23pt}catastrophes and heavy tails&2&20--25\\
\Avtors{Zhukov~D.\,O., Khvatova~T.\,Yu., and~Zaltcman~A.\,D.} Modeling of the stochastic dynamics of changes in node states and percolation transitions in social networks with self-organization\linebreak
\\[-12pt]
\hspace*{23pt}and memory&1&102--110\\
\Avtors{Zhukov~D.\,V.} see Bosov~A.\,V.&&\\
\end{tabular}
}

%\thispagestyle{myheadings}
\def\leftfootline{\small{\textbf{\thepage}
\hfill INFORMATIKA I EE PRIMENENIYA~--- INFORMATICS AND APPLICATIONS\ \ \ 2021\
\ \ volume~15\ \ \ issue\ 4}
}%
 \def\rightfootline{\small{INFORMATIKA I EE PRIMENENIYA~---
INFORMATICS AND APPLICATIONS\ \ \ 2021\ \ \ volume~15\ \ \ issue\ 4
\hfill \textbf{\thepage}}}

 \label{end\stat}

\newpage


%\linebreak
%\\[-12pt]
%\hspace*{23pt}

%
   \vspace*{-46pt}

\begin{center}
\vspace*{4pt}
\mbox{%

\epsfxsize=55mm %112.705
\epsfbox{zhur-2.eps}
}
%\end{center}

\vspace*{10pt} 


%   \begin{center}
\fbox{\large\textbf{Академик Юрий Иванович Журавлёв}}\\[10pt]
\textbf{\large 14.01.1935--14.01.2022}
   \end{center}


   %\vspace*{2.5mm}

   \vspace*{5mm}

   \thispagestyle{empty}

%\

%\vspace*{-12pt}
       


В январе этого года ушел из жизни главный научный сотрудник Федерального исследовательского 
центра <<Информатика и управление>> РАН, председатель Редакционного совета журнала 
<<Информатика и~её применения>> академик Юрий Иванович Журавлёв. В~его лице мировая 
наука потеряла одного из своих ярчайших представителей~--- выдающегося ученого-исследователя 
и~талантливого ученого-организатора.

Юрий Иванович родился в Воронеже в 1935~г.\ в семье ученого и врача. Среднее образование 
получил в школе №\,6 г.~Фрунзе (ныне Бишкек) Киргизской ССР. В~1952~г.\ поступил на 
ме\-ха\-ни\-ко-ма\-те\-ма\-ти\-че\-ский факультет МГУ им.\ М.\,В.~Ломоносова. В~1957~г.\ Юрий Иванович 
защищает диплом и продолжает обучение в аспирантуре Московского университета на кафедре 
вычислительной математики (возглавляемой тогда академиком С.\,Л.~Соболевым). После 
успешной защиты кандидатской диссертации (к.ф.-м.н., 1959 г., научный руководитель~--- 
А.\,А.~Ляпунов, оппоненты~--- чл.-корр.\ А.\,А.~Марков, к.ф.-м.н.\ О.\,Б.~Лупанов) и~до 
окончательного переезда в Москву в 1969~г.\ работал в Институте математики Сибирского 
отделения АН СССР, занимая в нем последовательно должности младшего научного сотрудника, 
заведующего отделом, заведующего отделением, заместителя директора по научной работе. 
В~этот период (1954--1966~гг.)\ им был опубликован цикл работ по решению задач алгебры и 
математической логики, причем полученные результаты применялись для создания эффективных 
программ для ЭВМ, конструирования схем и сетей для обработки информации. Наиболее значимый 
результат этого периода научной работы~--- обоснование нового направления исследований, 
общей теории локальных алгоритмов. В~ней были окончательно объединены топологические 
принципы и теория алгоритмов. Эта теория и легла в основу докторской диссертации Юрия 
Ивановича (д.ф.-м.н., 1965~г.)\ по еще тогда новой научной специальности <<Математическая 
кибернетика>>. Оппонировали ему как специалисты по кибернетике~--- академик 
В.\,М.~Глушков, член-корреспондент А.\,А.~Ляпунов и О.\,Б.~Лупанов, так и про\-фес\-сор-ал\-геб\-раист А.\,Д.~Тайманов. 

В 1969~г.\ Юрий Иванович переезжает в Москву и возглавляет в Вычислительном центре АН 
СССР лабораторию проблем распознавания. Впоследствии он~--- заместитель директора по 
научной работе. Научные интересы этого периода связаны с проблемами классификации или 
распознавания образов. В~1976--1978~гг.\ Юрий Иванович публикует цикл работ по ставшему 
вскоре знаменитым алгебраическому подходу к проблеме синтеза корректных алгоритмов. Эти 
работы определили современное состояние всей проблематики распознавания и многих смежных 
областей прикладной математики и информатики. В~своих основополагающих работах Юрий 
Иванович показал, что можно в явном виде строить экстремальные по качеству алгоритмы для 
решения очень широких классов плохо формализованных задач. 
{\looseness=-1

}





Научные заслуги Юрия Ивановича получили широкое признание. В~1966~г.\ он совместно с 
О.\,Б.~Лупановым и чле\-ном-кор\-рес\-пон\-ден\-том АН СССР С.\,В.~Яблонским были удостоены 
звания лауреата Ленинской премии в~об\-ласти науки и техники. В~1984~г.\ Юрий Иванович 
был избран членом-корреспондентом АН СССР (по специальности <<Информатика>>), 
а~в~1992~г.~--- академиком РАН (по той же специальности).\linebreak\vspace*{-12pt}

\pagebreak

\

\vspace*{-46pt}

\noindent
\begin{floatingfigure}{48mm}
\begin{center}
%\vspace*{6pt}
\mbox{%

\epsfxsize=46mm %112.705
\epsfbox{zhur-3.eps}
}
\end{center}
\vspace*{6pt}
\end{floatingfigure}

 \thispagestyle{empty}

\noindent
В~1986~г.\ за цикл прикладных 
работ ему и ряду его учеников была при\-суж\-де\-на премия Совета Министров СССР. Он являлся 
членом иностранных академий наук, председателем секции <<Прикладная математика
 и~информатика>> Отделения математических наук РАН, председателем экспертного совета ВАК 
России по управ\-ле\-нию и информатике, заслуженным профессором нескольких университетов, 
председателем Российской ассоциации <<Распознавание образов и обработка изображений>>, 
членом исполкома Международной ассоциации IAPR (распознавание образов и обработка 
изображений). Был награжден 8-ю орденами и медалями СССР и России.

Юрий Иванович проводил большую научно-литературную работу, являясь, в том числе, главным 
редактором международных научных журналов и членом редколлегий ряда рецензируемых 
научных журналов. 


Параллельно с активной научной деятельностью Юрий Иванович вел и преподавательскую 
работу. С~1961 по~1969~гг.~--- в Новосибирском государственном университете на кафедре 
алгебры и математической логики, которую возглавлял в то время академик А.\,И.~Мальцев. 
С~1970~г., будучи уже профессором (1967~г.),~--- в Московском физико-техническом институте 
на кафедре академика Н.\,Н.~Моисеева. В~1997~г.\ по предложению ректора МГУ им.\ 
М.\,В.~Ломоносова академика В.\,А.~Садовничего Юрий Иванович организовал на факультете 
Вычислительной математики и кибернетики новую кафедру <<Математические методы 
прогнозирования>>, которой и руководил до конца жизни. В~2008~г.\ ему была присуждена 
премия Совета Министров РФ в области образования. С~1965~г.\ Юрий Иванович периодически 
читал курсы лекций за рубежом, в университетах США, Франции, Финляндии, Швеции, Австрии, 
Польши, Болгарии, ГДР и других стран. Эта работа в существенной степени обеспечила широкое 
международное признание советской и российской науки в области дискретной математики и~распознавания образов. 

%\begin{floatingfigure}{60mm}
\begin{figure}[b]
\begin{center}
\vspace*{-6pt}
\mbox{%

\epsfxsize=112mm %90mm %112.705
\epsfbox{zhur-1.eps}
}
\end{center}
\end{figure}
%\end{floatingfigure}

Понимая важность вопроса воспитания подрастающего поколения для развития науки в стране, 
Юрий Иванович вскоре после защиты первой диссертации включился в работу по подготовке 
научных кадров. Им создана большая научная школа: под руководством Юрия Ивановича 
защищены более 100~кандидатских диссертаций по всевозможным разделам естествознания 
(математике, информатике, медицине, технике, экономике, геологии), не один десяток докторов 
наук. Он воспитал академиков и членов-корреспондентов РАН и академий государств СНГ. 
С~большим вниманием и участием Юрий Иванович относился к развитию научных школ страны 
в~об\-ласти обработки изображений, распознавания образов и компьютерной оптики. 

Для всех коллег и учеников Юрия Ивановича он останется примером замечательного человека, 
та\-лант\-ли\-во\-го педагога и выдающегося, преданного служению науке ученого. 


%\def\stat{cont}
{%\hrule\par
%\vskip 7pt % 7pt
\raggedleft\Large \bf%\baselineskip=3.2ex
А\,В\,Т\,О\,Р\,С\,К\,И\,Й\ \ У\,К\,А\,З\,А\,Т\,Е\,Л\,Ь\ \ З\,А\ \ 2\,0\,1\,0 г. \vskip 17pt
    \hrule
    \par
\vskip 21pt plus 6pt minus 3pt }

\label{st\stat}

\def\tit{\ }

\def\aut{\ }
\def\auf{\ }

\def\leftkol{\ } % ENGLISH ABSTRACTS}

\def\rightkol{\ } %АВТОРСКИЙ УКАЗАТЕЛЬ ЗА 2010 г.} %ENGLISH ABSTRACTS}

\titele{\tit}{\aut}{\auf}{\leftkol}{\rightkol}

\vspace*{-12pt}

{\tabcolsep=3pt
\begin{tabular}{p{388pt}rr}
&\textbf{Выпуск} & \textbf{Стр.}\\[6pt]
\hangindent=23pt\noindent\textbf{Арутюнян~А.\,Р.} Моделирование влияния деформаций отпечатков пальцев на 
точность\linebreak
\vspace*{-12pt}\\
\hspace*{23pt}дактилоскопической идентификации$\dotfill$&1&51\\
\hangindent=23pt\noindent\textbf{Архипов~О.\,П., Зыкова~З.\,П.} Интеграция гетерогенной информации о цветных 
пикселях\linebreak
\vspace*{-12pt}\\
\hspace*{23pt}и их цветовосприятии$\dotfill$&4&15\\
\hangindent=23pt\noindent\textbf{Баранов~С.\,И., Френкель~С.\,Л., Захаров~В.\,Н.} Полуформальная верификация 
цифрового устройства с конвейером, основанная на использовании алгоритмических машин\linebreak
\vspace*{-12pt}\\
\hspace*{23pt}состояния$\dotfill$&4&49\\
\textbf{Бекетова~И.\,В.} см.~Каратеев~С.\,Л.&&\\
\textbf{Белоусов~В.\,В.} см.~Синицын~И.\,Н.&&\\
\hangindent=23pt\noindent\textbf{Бенинг~В.\,Е., Королев~Р.\,А.} О предельном поведении мощностей критериев в 
случае\linebreak
\vspace*{-12pt}\\
\hspace*{23pt}распределения Лапласа$\dotfill$&2&63\\
\hangindent=23pt\noindent\textbf{Бенинг~В.\,Е., Сипина~А.\,В.} Асимптотическое разложение для мощности 
критерия,\linebreak
\vspace*{-12pt}\\
\hspace*{23pt}основанного на выборочной медиане, в случае распределения Лапласа$\dotfill$&1&18\\
\textbf{Бондаренко~А.\,В.} см.~Каратеев~С.\,Л.&&\\
\hangindent=23pt\noindent\textbf{Бородина~А.\,В., Морозов~Е.\,В.} Об оценивании асимптотики вероятности 
большого\linebreak
\vspace*{-12pt}\\
\hspace*{23pt}уклонения стационарной регенеративной очереди с одним прибором$\dotfill$&3&29\\
\hangindent=23pt\noindent\textbf{Бунтман~Н.\,В., Минель~Ж.-Л., Ле~Пезан~Д., Зацман~И.\,М.} Типология и 
компьютерное\linebreak
\vspace*{-12pt}\\
\hspace*{23pt}моделирование трудностей перевода$\dotfill$&3&77\\
\textbf{Визильтер~Ю.\,В.} см.~Каратеев~С.\,Л.&&\\
\hangindent=23pt\noindent\textbf{Гавриленко~С.\,В.} Оценки скорости сходимости распределений случайных сумм с 
безгранично делимыми индексами к нормальному закону$\dotfill$&4&81\\
\hangindent=23pt\noindent\textbf{Григорьева~М.\,Е., Шевцова~И.\,Г.} Уточнение неравенства 
Каца--Берри--Эссеена$\dotfill$&2&75\\
\hangindent=23pt\noindent\textbf{Грушо~А.\,А., Грушо~Н.\,А., Тимонина~Е.\,Е.} Поиск конфликтов в политиках 
безопасности: модель случайных графов$\dotfill$&3&38\\
\textbf{Грушо~Н.\,А.} см.~Грушо~А.\,А.&&\\
\hangindent=23pt\noindent\textbf{Гудков~В.\,Ю.} Математические модели изображения отпечатка пальца на основе 
описания линий$\dotfill$&1&58\\
\textbf{Гуртов~А.\,В.} см.~Лукьяненко~А.\,С.&&\\
\textbf{Желтов~С.\,Ю.} см.~Каратеев~С.\,Л.&&\\
\hangindent=23pt\noindent\textbf{Захаров~А.\,А., Серебряков~В.\,А.} Система управления электронной библиотекой 
LibMeta$\dotfill$&4&2\\
\textbf{Захаров~В.\,Н.} см.~Баранов~С.\,И.&&\\
\textbf{Захарова~Т.\,В.} см.~Матвеева~С.\,С.&&\\
\hangindent=23pt\noindent\textbf{Зацаринный~А.\,А., Чупраков~К.\,Г.} Некоторые аспекты выбора технологии для 
постро-\linebreak
\vspace*{-12pt}\\
\hspace*{23pt}ения систем отображения информации ситуационного центра$\dotfill$&3&59\\
\textbf{Зацман~И.\,М.} см.~Бунтман~Н.\,В.&&\\
\hangindent=23pt\noindent\textbf{Зейфман~А.\,И., Коротышева~А.\,В., Сатин~Я.\,А., Шоргин~С.\,Я.} Об 
устойчивости нестаци-\linebreak
\vspace*{-12pt}\\
\hspace*{23pt}онарных систем обслуживания с катастрофами$\dotfill$&3&9\\
\textbf{Зыкова~З.\,П.} см.~Архипов~О.\,П.&&\\
\hangindent=23pt\noindent\textbf{Илюшин~Г.\,Я., Соколов~И.\,А.} Организация управляемого доступа пользователей 
к\linebreak
\vspace*{-12pt}\\
\hspace*{23pt}разнородным ведомственным информационным ресурсам$\dotfill$&1&24\\
\hangindent=23pt\noindent\textbf{Кавагучи~Ю., Ульянов~В.\,В., Фуджикоши~Я.} Приближения для статистик, 
описывающих\linebreak
\vspace*{-12pt}\\
\hspace*{23pt}геометрические свойства данных большой размерности, с оценками 
ошибок$\dotfill$&1&12\\
\hangindent=23pt\noindent\textbf{Каратеев~С.\,Л., Бекетова~И.\,В., Ососков~М.\,В., Князь~В.\,А., 
Визильтер~Ю.\,В., Бондаренко~А.\,В., Желтов~С.\,Ю.} Автоматизированный контроль 
качества цифровых\linebreak
\vspace*{-12pt}\\
\hspace*{23pt}изображений для персональных документов$\dotfill$&1&65\\
\end{tabular}
}

\pagebreak

\def\leftkol{АВТОРСКИЙ УКАЗАТЕЛЬ ЗА 2010 г.} % ENGLISH ABSTRACTS}

\def\rightkol{АВТОРСКИЙ УКАЗАТЕЛЬ ЗА 2010 г.} %ENGLISH ABSTRACTS}

{\tabcolsep=3pt
\begin{tabular}{p{388pt}rr}
&\textbf{Выпуск} & \textbf{Стр.}\\[3pt]
\hangindent=23pt\noindent\textbf{Козеренко~Е.\,Б.} Лингвистические фильтры в статистических моделях машинного\linebreak
\vspace*{-12pt}\\
\hspace*{23pt}перевода$\dotfill$&2&83\\
\hangindent=23pt\noindent\textbf{Козеренко~Е.\,Б., Кузнецов~И.\,П.} Когнитивно-лингвистические представления в 
систе-\linebreak
\vspace*{-12pt}\\
\hspace*{23pt}мах обработки текстов$\dotfill$&3&69\\
\textbf{Князь~В.\,А.} см.~Каратеев~С.\,Л.&&\\
\hangindent=23pt\noindent\textbf{Колесников~А.\,В., Солдатов~С.\,А.} Алгоритм координации для гибридной 
интеллектуальной системы решения сложной задачи оперативно-производственного\linebreak
\vspace*{-12pt}\\
\hspace*{23pt}планирования$\dotfill$&4&61\\
\hangindent=23pt\noindent\textbf{Коновалов~М.\,Г.} О планировании потоков в системах вычислительных 
ресурсов$\dotfill$&2&3\\
\textbf{Конушин~А.\,С.} см.~Конушин~В.\,С.&&\\
\hangindent=23pt\noindent\textbf{Конушин~В.\,С., Кривовязь~Г.\,Р., Конушин~А.\,С.} Алгоритм распознавания людей 
в видео-\linebreak
\vspace*{-12pt}\\
\hspace*{23pt}последовательности по одежде$\dotfill$&1&74\\
\textbf{Корепанов~Э.\, Р.} см.~Синицын~И.\,Н.&&\\
\textbf{Королев~В.\,Ю.} см.~Соколов~И.\,А.&&\\
\textbf{Королев~Р.\,А.} см.~Бенинг~В.\,Е.&&\\
\textbf{Коротышева~А.\,В.} см.~Зейфман~А.\,И.&&\\
\hangindent=23pt\noindent\textbf{Кривенко~М.\,П.} Непараметрическое оценивание элементов байесовского 
клас\-си-\linebreak
\vspace*{-12pt}\\
\hspace*{23pt}фикатора$\dotfill$&2&13\\
\textbf{Кривовязь~Г.\,Р.} см.~Конушин~В.\,С.&&\\
\textbf{Крылов~А.\,С.} см.~Павельева~Е.\,А.&&\\
\hangindent=23pt\noindent\textbf{Крылов~В.\,А.} Моделирование и классификация многоканальных дистанционных\linebreak
\vspace*{-12pt}\\
\hspace*{23pt}изображений с использованием копул$\dotfill$&4&34\\
\hangindent=23pt\noindent\textbf{Крючин~О.\,В.} Разработка параллельных эвристических алгоритмов подбора 
весовых\linebreak
\vspace*{-12pt}\\
\hspace*{23pt}коэффициентов искусственной нейтронной сети$\dotfill$&2&53\\
\hangindent=23pt\noindent\textbf{Кудрявцев~А.\,А., Шоргин~С.\,Я.} Байесовские модели массового обслуживания и 
надеж-\linebreak
\vspace*{-12pt}\\
\hspace*{23pt}ности: характеристики среднего числа заявок в системе $M\vert M \vert 1\vert 
\infty$$\dotfill$&3&16\\
\hangindent=23pt\noindent\textbf{Кузнецов~А.\,А.} Связь между временными и структурно-топологическими 
характери-\linebreak
\vspace*{-12pt}\\
\hspace*{23pt}стиками диаграмм ритма сердца здоровых людей$\dotfill$&4&39\\
\textbf{Кузнецов~И.\,П.} см.~Козеренко~Е.\,Б.&&\\
\textbf{Ле~Пезан~Д.} см.~Бунтман~Н.\,В.&&\\
\hangindent=23pt\noindent\textbf{Лукьяненко~А.\,С., Морозов~Е.\,В., Гуртов~А.\,В.} Анализ сетевого протокола с общей 
функ-\linebreak
\vspace*{-12pt}\\
\hspace*{23pt}цией расширения окна передачи сообщения при конфликтах$\dotfill$&2&46\\
\hangindent=23pt\noindent\textbf{Лямин~О.\,О.} О предельном поведении мощностей критериев в случае обобщенного\linebreak
\vspace*{-12pt}\\
\hspace*{23pt}распределения Лапласа$\dotfill$&3&47\\
\hangindent=23pt\noindent\textbf{Маркин~А.\,В., Шестаков~О.\,В.} Асимптотики оценки риска при пороговой 
обработке\linebreak
\vspace*{-12pt}\\
\hspace*{23pt}вейвлет-вейглет коэффициентов в задаче томографии$\dotfill$&2&36\\
\hangindent=23pt\noindent\textbf{Матвеева~С.\,С., Захарова~Т.\,В.} Сети массового обслуживания с наименьшей 
длиной\linebreak
\vspace*{-12pt}\\
\hspace*{23pt}очереди$\dotfill$&3&22\\
\hangindent=23pt\noindent\textbf{Матюшенко~С.\,И.} Стационарные характеристики двухканальной системы 
обслужива-\linebreak
\vspace*{-12pt}\\
\hspace*{23pt}ния с переупорядочиванием заявок и распределениями фазового типа$\dotfill$&4&68\\
\textbf{Минель~Ж.-Л.} см.~Бунтман~Н.\,В.&&\\
\textbf{Морозов~Е.\,В.} см.~Бородина~А.\,В.&&\\
\textbf{Морозов~Е.\,В.} см.~Лукьяненко~А.\,С.&&\\
\textbf{Ососков~М.\,В.} см.~Каратеев~С.\,Л.&&\\
\hangindent=23pt\noindent\textbf{Павельева~Е.\,А., Крылов~А.\,С.} Поиск и анализ ключевых точек радужной 
оболочки\linebreak
\vspace*{-12pt}\\
\hspace*{23pt}глаза методом преобразования Эрмита$\dotfill$&1&79\\
\textbf{Печинкин~А.\,В.} см.~Френкель~С.\,Л.,&&\\
\hangindent=23pt\noindent\textbf{Протасов~В.\,И.} Составление субъективного портрета с использованием 
эволюционно-\linebreak
\vspace*{-12pt}\\
\hspace*{23pt}го морфинга и квалиметрия метода$\dotfill$&1&83\\
\hangindent=23pt\noindent\textbf{Рудаков~К.\,В., Торшин~И.\,Ю.} Вопросы разрешимости задачи распознавания 
вторичной\linebreak
\vspace*{-12pt}\\
\hspace*{23pt}структуры белка$\dotfill$&2&25\\
\textbf{Сатин~Я.\,А.} см.~Зейфман~А.\,И.&&\\
\hangindent=23pt\noindent\textbf{Сейфуль-Мулюков~Р.\,Б.} Нефть как носитель информации о своем 
происхождении,\linebreak
\vspace*{-12pt}\\
\hspace*{23pt}структуре и эволюции$\dotfill$&1&41\\
\end{tabular}
}

{\tabcolsep=3pt
\begin{tabular}{p{388pt}rr}
&\textbf{Выпуск} & \textbf{Стр.}\\[6pt]
\textbf{Семендяев~Н.\,Н.} см.~Синицын~И.\,Н.&&\\
\textbf{Серебряков~В.\,А.} см.~Захаров~А.\,А.&&\\
\textbf{Синицын~В.\,И.} см.~Синицын~И.\,Н.&&\\
\hangindent=23pt\noindent\textbf{Синицын~И.\,Н., Синицын~В.\,И., Корепанов~Э.\, Р., Белоусов~В.\,В., 
Семендяев~Н.\,Н.} Оперативное построение информационных моделей движения полюса 
Земли\linebreak
\vspace*{-12pt}\\
\hspace*{23pt}методами линейных и линеаризованных фильтров$\dotfill$&1&2\\
\textbf{Сипина~А.\,В.} см.~Бенинг~В.\,Е.&&\\
\hangindent=23pt\noindent\textbf{Соколов~И.\,А.} О работах заслуженного деятеля науки Российской Федерации 
И.\,Н.~Синицына в области информационных технологий и автоматизации (к 70-летию\linebreak
\vspace*{-12pt}\\
\hspace*{23pt}со дня рождения)$\dotfill$&3&84\\
\textbf{Соколов~И.\,А.} см.~Илюшин~Г.\,Я.&&\\
\hangindent=23pt\noindent\textbf{Соколов~И.\,А., Королев~В.\,Ю.} Предисловие$\dotfill$&2&2\\
\textbf{Солдатов~С.\,А.} см.~Колесников~А.\,В.&&\\
\hangindent=23pt\noindent\textbf{Степанов~С.\,Ю.} Использование координатного метода фрагментации 
коммутаторной\linebreak
\vspace*{-12pt}\\
\hspace*{23pt}нейронной сети для сокращения трафика$\dotfill$&2&57\\
\textbf{Тимонина~Е.\,Е.} см.~Грушо~А.\,А.&&\\
\textbf{Торшин~И.\,Ю.} см.~Рудаков~К.\,В.&&\\
\textbf{Ульянов~В.\,В.} см.~Кавагучи~Ю.&&\\
\textbf{Фазекаш~И.} см.~Чупрунов~А.\,Н.&&\\
\textbf{Френкель~С.\,Л.} см.~Баранов~С.\,И.&&\\
\hangindent=23pt\noindent\textbf{Френкель~С.\,Л., Печинкин~А.\,В.} Оценка времени самовосстановления в 
цифровых\linebreak
\vspace*{-12pt}\\
\hspace*{23pt}системах после сбоев, вызываемых переходными помехами$\dotfill$&3&2\\
\textbf{Фуджикоши~Я.} см.~Кавагучи~Ю.&&\\
\hangindent=23pt\noindent\textbf{Цискаридзе~А.\,К.} Математическая модель и метод восстановления позы человека 
по\linebreak
\vspace*{-12pt}\\
\hspace*{23pt}стереопаре силуэтных изображений$\dotfill$&4&27\\
\hangindent=23pt\noindent\textbf{Чупраков~К.\,Г.} К вопросу о размещении коллективных средств отображения в 
ситуа-\linebreak
\vspace*{-12pt}\\
\hspace*{23pt}ционном зале с заданными параметрами$\dotfill$&4&89\\
\textbf{Чупраков~К.\,Г.} см.~Зацаринный~А.\,А.&&\\
\hangindent=23pt\noindent\textbf{Чупрунов~А.\,Н., Фазекаш~И.} Законы повторного логарифма для числа 
безошибочных\linebreak
\vspace*{-12pt}\\
\hspace*{23pt}блоков при помехоустойчивом кодировании$\dotfill$&3&42\\
\textbf{Шевцова~И.\,Г.} см.~Григорьева~М.\,Е.&&\\
\hangindent=23pt\noindent\textbf{Шестаков~О.\,В.} Аппроксимация распределения оценки риска пороговой 
обработки вейвлет-коэффициентов нормальным распределением при использовании 
выбо-\linebreak
\vspace*{-12pt}\\
\hspace*{23pt}рочной дисперсии$\dotfill$&4&73\\
\textbf{Шестаков~О.\,В.} см.~Маркин~А.\,В.&&\\
\textbf{Шоргин~С.\,Я.} см.~Зейфман~А.\,И.&&\\
\textbf{Шоргин~С.\,Я.} см.~Кудрявцев~А.\,А.&&\\
\end{tabular}
}

%\thispagestyle{myheadings}
\def\leftfootline{\small{\textbf{\thepage}
\hfill ИНФОРМАТИКА И ЕЁ ПРИМЕНЕНИЯ\ \ \ том~4\ \ \ выпуск~4\ \ \ 2010}
}%
 \def\rightfootline{\small{ИНФОРМАТИКА И ЕЁ ПРИМЕНЕНИЯ\ \ \ том~4\ \ \ выпуск~4\ \ \ 2010
 \hfill \textbf{\thepage}}}
 \label{end\stat}
%
%Том 10 Выпуск 1-4 Год 2016

\def\stat{cont-e}
{%\hrule\par
%\vskip 7pt % 7pt
\raggedleft\Large \bf%\baselineskip=3.2ex
2\,0\,1\,6\ \ A\,U\,T\,H\,O\,R\ \ I\,N\,D\,E\,X \vskip 17pt
 \hrule
 \par
\vskip 21pt plus 6pt minus 3pt }

\label{st\stat}

\def\tit{\ }

\def\aut{\ }
\def\auf{\ }

\def\leftkol{\ } %2016 AUTHOR INDEX} % ENGLISH ABSTRACTS}

\def\rightkol{\ } %2016 AUTHOR INDEX} %ENGLISH ABSTRACTS}

\titele{\tit}{\aut}{\auf}{\leftkol}{\rightkol}

\def\leftfootline{\small{\textbf{\thepage}
\hfill INFORMATIKA I EE PRIMENENIYA~--- INFORMATICS AND APPLICATIONS\ \ \ 2016\
\ \ volume~10\ \ \ issue\ 4}
}%
 \def\rightfootline{\small{INFORMATIKA I EE PRIMENENIYA~--- INFORMATICS AND APPLICATIONS\ \ \ 2016\ \ \ volume~10\ \ \ issue\ 4
\hfill \textbf{\thepage}}}

\vspace*{-12pt}
\vspace*{-18pt}

{\tabcolsep=2.8pt
\begin{tabular}{p{382pt}cc}
&\textbf{Issue} & \textbf{Page}\\[6pt]
\Avtors{Agalarov~M.\,Ya.} see~Agalarov~Ya.\,M.&&\\
\Avtors{Agalarov~Ya.\,M., Agalarov~M.\,Ya., and
Shorgin~V.\,S.} About the optimal threshold of queue\linebreak
\\[-12pt]
\hspace*{23pt}length in a~particular problem of profit maximization
in the $M/G/1$ queuing system&2&70--79\\
\Avtors{Alexeyevsky~D.\,A.} BioNLP ontology extraction from 
a~restricted language corpus with\linebreak
\\[-12pt]
\hspace*{23pt}context-free grammars&1&119--128\\
\Avtors{Andreev~S.\,D.} see~Gaidamaka~Yu.\,V.&&\\
\Avtors{Andreev~S.\,D.} see~Ometov~A.\,Ya.&&\\
\Avtors{Arkhipov~O.\,P., Arkhipov~P.\,O., and Sidorkin~I.\,I.} The
option to create a~local coordinate\linebreak
\\[-12pt]
\hspace*{23pt}system for synchronization of selected images&3&91--97\\
\Avtors{Arkhipov~P.\,O.} see~Arkhipov~O.\,P.&&\\
\Avtors{Belousov~V.\,V.} see~Shnurkov~P.\,V.&&\\
\Avtors{Belousov~V.\,V.} see~Shnurkov~P.\,V.&&\\
\Avtors{Bening~V.\,E.} Calculation of~the~asymptotic deficiency
of~some statistical procedures based\linebreak
\\[-12pt]
\hspace*{23pt}on~samples with~random sizes&4&34--45\\
\Avtors{Borisov~A.\,V., Bosov~A.\,V., and Miller~G.\,B.} Modeling and
monitoring of VoIP connection&2&\hphantom{1}2--13\\
\Avtors{Bosov~A.\,V.} see~Borisov~A.\,V.&&\\
\Avtors{Briukhov~D.\,O.} see~Stupnikov~S.\,A.&&\\
\Avtors{Callaos~N.\,K.\ and Seyful-Mulyukov~R.\,B.} Complexity and
its information content&1&129--139\\
\Avtors{Chertok~A.\,V., Kadaner~A.\,I., Khazeeva~G.\,T., and
Sokolov~I.\,A.} Regime switching detection\linebreak
\\[-12pt]
\hspace*{23pt}for~the~Levy driven
Ornstein--Uhlenbeck process using CUSUM methods&4&46--56\\
\Avtors{Chichagov~V.\,V.} Asymptotic expansions of mean absolute
error of uniformly minimum variance unbiased and maximum likelihood
estimators on the one-parameter exponential\linebreak
\\[-12pt]
\hspace*{23pt}family model of lattice distributions&3&66--76\\
\Avtors{Danishevsky~V.\,I.} see~Kolesnikov A.\,V.&&\\
\Avtors{Fazliev~A.\,Z.} see~Kalinichenko~L.\,A.&&\\
\Avtors{Fedoseev~A.\,A.} What is behind the concept of ``knowledge in
small packages''&3&105--110\\
\Avtors{Gaidamaka~Yu.\,V., Andreev~S.\,D., Sopin~E.\,S.,
Samouylov~K.\,E., and Shorgin~S.\,Ya.} Interference analysis
of~the~device-to-device communications model with~regard to~a~signal\linebreak
\\[-12pt]
\hspace*{23pt}propagation environment&4&\hphantom{1}2--10\\
\Avtors{Gasilov~A.\,V.} see~Yakovlev~O.\,A.&&\\
\Avtors{Goncharov~A.\,V.\ and Strijov~V.\,V.} Metric time series
classification using weighted dynamic\linebreak
\\[-12pt]
\hspace*{23pt}warping relative to centroids of classes&2&36--47\\
\Avtors{Gordov~E.\,P.} see~Kalinichenko~L.\,A.&&\\
\Avtors{Gorshenin~A.\,K.} Concept of online service for stochastic
modeling of real processes&1&72--81\\
\Avtors{Gorshenin~A.\,K.} see~Shnurkov~P.\,V.&&\\
\Avtors{Gorshenin~A.\,K.} see~Shnurkov~P.\,V.&&\\
\Avtors{Grusho~A.\,A., Grusho~N.\,A., Zabezhailo~M.\,I., and
Timonina~E.\,E.} Integration of statistical and\linebreak
\\[-12pt]
\hspace*{23pt}deterministic methods for
analysis of information security&3&2--8\\
\Avtors{Grusho~A.\,A., Zabezhailo~M.\,I., and Zatsarinny~A.\,A.} On
the advanced procedure to reduce\linebreak
\\[-12pt]
\hspace*{23pt}calculation of Galois closures&4&\hphantom{1}96--104\\
\Avtors{Grusho~N.\,A.} see~Grusho~A.\,A.&&\\
\Avtors{Havanskov~V.\,A.} see~Minin~V.\,A.&&\\
\Avtors{Inkova~O.\,Yu.} see~Zatsman~I.\,M.&&\\
\Avtors{Isachenko~R.\,V.\ and Strijov~V.\,V.} Metric learning in
multiclass time series classification\linebreak
\\[-12pt]
\hspace*{23pt}problem&2&48--57\\
\end{tabular}
}
\pagebreak

\def\leftfootline{\small{\textbf{\thepage}
\hfill INFORMATIKA I EE PRIMENENIYA~--- INFORMATICS AND APPLICATIONS\ \ \ 2016\
\ \ volume~10\ \ \ issue\ 4}
}%
 \def\rightfootline{\small{INFORMATIKA I EE PRIMENENIYA~---
INFORMATICS AND APPLICATIONS\ \ \ 2016\ \ \ volume~10\ \ \ issue\ 4
\hfill \textbf{\thepage}}}

\def\leftkol{2016 AUTHOR INDEX} % ENGLISH ABSTRACTS}

\def\rightkol{2016 AUTHOR INDEX} %ENGLISH ABSTRACTS}


{\tabcolsep=2.83pt
\begin{tabular}{p{382pt}cc}
&\textbf{Issue} & \textbf{Page}\\[6pt]
\Avtors{Kadaner~A.\,I.} see~Chertok~A.\,V.&&\\[.255pt]
\Avtors{Kalinichenko~L.\,A., Volnova~A.\,A., Gordov~E.\,P.,
Kiselyova~N.\,N., Kovaleva~D.\,A., Malkov~O.\,Yu., Okladnikov~I.\,G.,
Podkolodnyy~N.\,L., Pozanenko~A.\,S., Ponomareva~N.\,V.,
Stupnikov~S.\,A.,} \textbf{and Fazliev~A.\,Z.} Data access challenges for data
intensive\linebreak
\\[-12pt]
\hspace*{23pt}research in Russia&1& 2--22\\[.255pt]
\Avtors{Karasikov~M.\,E.\ and Strijov~V.\,V.} Feature-based
time-series classification&4&121--131\\[.255pt]
\Avtors{Khazeeva~G.\,T.} see~Chertok~A.\,V.&&\\[.255pt]
\Avtors{Khokhlov~Yu.\,S.} Multivariate fractional Levy motion and its
applications&2&\hphantom{1}98--106\\[.255pt]
\Avtors{Kirikov~I.\,A., Kolesnikov~A.\,V., Listopad~S.\,V., and
Rumovskaya~S.\,B.} Fine-grained hybrid\linebreak
\\[-12pt]
\hspace*{23pt}intelligent systems. Part 2:
Bidirectional hybridization&1&\hphantom{1}96--105\\[.255pt]
\Avtors{Kirikov~I.\,A., Kolesnikov~A.\,V., Listopad~S.\,V., and
Rumovskaya~S.\,B.} ``Virtual council''~---\linebreak
\\[-12pt]
\hspace*{23pt}source environment
supporting complex diagnostic decision making&3&81--90\\[.255pt]
\Avtors{Kiselyova~N.\,N.} see~Kalinichenko~L.\,A.&&\\[.255pt]
\Avtors{Kolesnikov A.\,V., Listopad~S.\,V., Rumovskaya~S.\,B., and
Danishevsky~V.\,I.} Informal axiomatic\linebreak
\\[-12pt]
\hspace*{23pt}theory of~the~role visual models&4&114--120\\[.255pt]
\Avtors{Kolesnikov~A.\,V.} see~Kirikov~I.\,A.&&\\[.255pt]
\Avtors{Kolesnikov~A.\,V.} see~Kirikov~I.\,A.&&\\[.255pt]
\Avtors{Kolin~K.\,K.} Humanitarian aspects of information
security&3&111--121\\[.255pt]
\Avtors{Konovalov~M.\,G.\ and Razumchik~R.\,V.} Dispatching
to~two parallel nonobservable queues using\linebreak
\\[-12pt]
\hspace*{23pt}only static
information&4&57--67\\[.255pt]
\Avtors{Korchagin~A.\,Yu.} see~Korolev~V.\,Yu.&&\\[.255pt]
\Avtors{Korchagin~A.\,Yu.} see~Korolev~V.\,Yu.&&\\[.255pt]
\Avtors{Korepanov~E.\,R.} see~Sinitsyn~I.\,N.&&\\[.255pt]
\Avtors{Korepanov~E.\,R.} see~Sinitsyn~I.\,N.&&\\[.255pt]
\Avtors{Korolev~V.\,Yu., Korchagin~A.\,Yu., and Zeifman~A.\,I.} The
Poisson theorem for Bernoulli trials\linebreak
\\[-12pt]
\hspace*{23pt}with~a~random probability
of~success and~a~discrete analog of~the~Weibull distribution&4&11--20\\[.255pt]
\Avtors{Korolev~V.\,Yu., Zeifman~A.\,I., and Korchagin~A.\,Yu.}
Asymmetric Linnik distributions as~limit\linebreak
\\[-12pt]
\hspace*{23pt}laws for~random sums
of~independent random variables with~finite variances&4&21--33\\[.255pt]
\Avtors{Koucheryavy~E.\,A.} see~Ometov~A.\,Ya.&&\\[.255pt]
\Avtors{Kovaleva~D.\,A.} see~Kalinichenko~L.\,A.&&\\[.255pt]
\Avtors{Kovalyov~S.\,P.} Metaprogramming to increase
manufacturability of large-scale software-\linebreak
\\[-12pt]
\hspace*{23pt}intensive systems&1&56--66\\[.255pt]
\Avtors{Krivenko~M.\,P.} Significance tests of feature selection for
classification&3&32--40\\[.255pt]
\Avtors{Kruzhkov~M.\,G.} see~Zalizniak~Anna~A.&&\\[.255pt]
\Avtors{Kruzhkov~M.\,G.} see~Zatsman~I.\,M.&&\\[.255pt]
\Avtors{Kudryavtsev~A.\,A.} Bayesian queueing and reliability models:
\textit{A~priori} distributions with\linebreak
\\[-12pt]
\hspace*{23pt}compact support&1&67--71\\[.255pt]
\Avtors{Kudryavtsev~A.\,A.} Characteristics dependent on the balance
coefficient in Bayesian models\linebreak
\\[-12pt]
\hspace*{23pt}with compact support of \textit{a priori}
distributions&3&77--80\\[.255pt]
\Avtors{Kudryavtsev~A.\,A.\ and Palionnaia~S.\,I.} Bayesian recurrent
model of reliability growth:\linebreak
\\[-12pt]
\hspace*{23pt}Parabolic distribution of parameters&2&80--83\\[.255pt]
\Avtors{Kudryavtsev~A.\,A.\ and Titova~A.\,I.} Bayesian queuing
and~reliability models: Degenerate-\linebreak
\\[-12pt]
\hspace*{23pt}Weibull case&4&68--71\\[.255pt]
\Avtors{Leontyev~N.\,D.\ and Ushakov~V.\,G.} Analysis of a queueing
system with autoregressive arrivals\linebreak
\\[-12pt]
\hspace*{23pt}and nonpreemptive priority&3&15--22\\[.255pt]
\Avtors{Listopad~S.\,V.} see~Kirikov~I.\,A.&&\\[.255pt]
\Avtors{Listopad~S.\,V.} see~Kirikov~I.\,A.&&\\[.255pt]
\Avtors{Listopad~S.\,V.} see~Kolesnikov A.\,V.&&\\[.255pt]
\Avtors{Malkov~O.\,Yu.} see~Kalinichenko~L.\,A.&&\\[.255pt]
\Avtors{Markov~A.\,S., Monakhov~M.\,M., and
Ulyanov~V.\,V.} Generalized Cornish--Fisher expansions\linebreak
\\[-12pt]
\hspace*{23pt}for distributions of statistics based on samples
of random size&2&84--91\\[.255pt]
\Avtors{Melnikov~A.\,K.\ and Ronzhin~A.\,F.} Generalized statistical
method of~text analysis based\linebreak
\\[-12pt]
\hspace*{23pt}on~calculation of~probability distributions
of~statistical values&4&89--95\\
\end{tabular}
}
\pagebreak

\def\leftfootline{\small{\textbf{\thepage}
\hfill INFORMATIKA I EE PRIMENENIYA~--- INFORMATICS AND APPLICATIONS\ \ \ 2016\
\ \ volume~10\ \ \ issue\ 4}
}%
 \def\rightfootline{\small{INFORMATIKA I EE PRIMENENIYA~---
INFORMATICS AND APPLICATIONS\ \ \ 2016\ \ \ volume~10\ \ \ issue\ 4
\hfill \textbf{\thepage}}}

\def\leftkol{2016 AUTHOR INDEX} % ENGLISH ABSTRACTS}

\def\rightkol{2016 AUTHOR INDEX} %ENGLISH ABSTRACTS}


{\tabcolsep=3pt
\begin{tabular}{p{381pt}cc}
&\textbf{Issue} & \textbf{Page}\\[6pt]
\Avtors{Meykhanadzhyan~L.\,A.} Stationary characteristics of the finite
capacity queueing system with\linebreak
\\[-12pt]
\hspace*{23pt}inverse service order and generalized
probabilistic priority&2&123--131\\[.23pt]
\Avtors{Miller~G.\,B.} see~Borisov~A.\,V.&&\\[.23pt]
\Avtors{Minin~V.\,A., Zatsman~I.\,M., Havanskov~V.\,A., and
Shubnikov~S.\,K.} Intensity of citation of scientific publications in
inventions on information and computer technologies patented\linebreak
\\[-12pt]
\hspace*{23pt}in Russia by domestic and foreign applicants&2&107--122\\[.23pt]
\Avtors{Monakhov~M.\,M.} see~Markov~A.\,S.&&\\[.23pt]
\Avtors{Naumov~V.\,A.\ and Samouylov~K.\,E.} On relationship
between queuing systems with resources\linebreak
\\[-12pt]
\hspace*{23pt}and Erlang networks&3&\hphantom{1}9--14\\[.23pt]
\Avtors{Okladnikov~I.\,G.} see~Kalinichenko~L.\,A.&&\\[.23pt]
\Avtors{Ometov~A.\,Ya., Andreev~S.\,D., Turlikov~A.\,M., and
Koucheryavy~E.\,A.} Performance analysis of\linebreak
\\[-12pt]
\hspace*{23pt}a wireless data
aggregation system with contention for contemporary sensor
networks&3&23--31\\[.23pt]
\Avtors{Palionnaia~S.\,I.} see~Kudryavtsev~A.\,A.&&\\[.23pt]
\Avtors{Podkolodnyy~N.\,L.} see~Kalinichenko~L.\,A.&&\\[.23pt]
\Avtors{Ponomareva~N.\,V.} see~Kalinichenko~L.\,A.&&\\[.23pt]
\Avtors{Popkova~N.\,A.} see~Zatsman~I.\,M.&&\\[.23pt]
\Avtors{Pozanenko~A.\,S.} see~Kalinichenko~L.\,A.&&\\[.23pt]
\Avtors{Razumchik~R.\,V.} see~Konovalov~M.\,G.&&\\[.23pt]
\Avtors{Ronzhin~A.\,F.} see~Melnikov~A.\,K.&&\\[.23pt]
\Avtors{Rumovskaya~S.\,B.} see~Kirikov~I.\,A.&&\\[.23pt]
\Avtors{Rumovskaya~S.\,B.} see~Kirikov~I.\,A.&&\\[.23pt]
\Avtors{Rumovskaya~S.\,B.} see~Kolesnikov A.\,V.&&\\[.23pt]
\Avtors{Samouylov~K.\,E.} see~Gaidamaka~Yu.\,V.&&\\[.23pt]
\Avtors{Samouylov~K.\,E.} see~Naumov~V.\,A.&&\\[.23pt]
\Avtors{Serebryanskii~S.\,M.} see~Tyrsin~A.\,N.&&\\[.23pt]
\Avtors{Seyful-Mulyukov~R.\,B.} see~Callaos~N.\,K.&&\\[.23pt]
\Avtors{Shestakov~O.\,V.} Statistical properties of the denoising method
based on the stabilized hard\linebreak
\\[-12pt]
\hspace*{23pt}thresholding&2&65--69\\[.23pt]
\Avtors{Shestakov~O.\,V.} The strong law of large numbers for the risk
estimate in the problem of\linebreak
\\[-12pt]
\hspace*{23pt}tomographic image reconstruction from
projections with a correlated noise&3&41--45\\[.23pt]
\Avtors{Shestakov~O.\,V.} see~Zakharova~T.\,V.&&\\[.23pt]
\Avtors{Shnurkov~P.\,V., Gorshenin~A.\,K., and Belousov~V.\,V.}
Analytical solution of~the~optimal control\linebreak
\\[-12pt]
\hspace*{23pt}task of~a~semi-Markov
process with~finite set of~states&4&72--88\\[.23pt]
\Avtors{Shnurkov~P.\,V., Zasypko~V.\,V., Belousov~V.\,V., and
Gorshenin~A.\,K.} Development of the algorithm of numerical solution
of the optimal investment control problem\linebreak
\\[-12pt]
\hspace*{23pt}in the closed dynamical model of three-sector economy&1&82--95\\[.23pt]
\Avtors{Shorgin~S.\,Ya.} see~Gaidamaka~Yu.\,V.&&\\[.23pt]
\Avtors{Shorgin~V.\,S.} see~Agalarov~Ya.\,M.&&\\[.23pt]
\Avtors{Shubnikov~S.\,K.} see~Minin~V.\,A.&&\\[.23pt]
\Avtors{Sidorkin~I.\,I.} see~Arkhipov~O.\,P.&&\\[.23pt]
\Avtors{Sinitsyn~I.\,N.} Analytical modeling of processes in stochastic
systems with complex fractional\linebreak
\\[-12pt]
\hspace*{23pt}order Bessel nonlinearities&3&55--65\\[.23pt]
\Avtors{Sinitsyn~I.\,N.} Orthogonal supoptimal filters for nonlinear
stochastic systems on manifolds&1&34--44\\[.23pt]
\Avtors{Sinitsyn~I.\,N.\ and Korepanov~E.\,R.} Normal Pugachev
conditionally-optimal filters and extra-\linebreak
\\[-12pt]
\hspace*{23pt}polators for state linear stochastic systems&2&14--23\\[.23pt]
\Avtors{Sinitsyn~I.\,N.\ and Sinitsyn~V.\,I.} Analytical modeling of
distributions in stochastic systems on\linebreak
\\[-12pt]
\hspace*{23pt}manifolds based on ellipsoidal approximation&1&45--55\\[.23pt]
\Avtors{Sinitsyn~I.\,N., Sinitsyn~V.\,I., and
Korepanov~E.\,R.} Ellipsoidal suboptimal filters for nonlinear\linebreak
\\[-12pt]
\hspace*{23pt}stochastic systems on manifolds&2&24--35\\[.23pt]
\Avtors{Sinitsyn~V.\,I.} see~Sinitsyn~I.\,N.&&\\[.23pt]
\Avtors{Sinitsyn~V.\,I.} see~Sinitsyn~I.\,N.&&\\[.23pt]
\Avtors{Skvortsov~N.\,A.} see~Stupnikov~S.\,A.&&\\[.23pt]
\Avtors{Sokolov~I.\,A.} see~Chertok~A.\,V.&&\\
\end{tabular}
}
\pagebreak

\def\leftfootline{\small{\textbf{\thepage}
\hfill INFORMATIKA I EE PRIMENENIYA~--- INFORMATICS AND APPLICATIONS\ \ \ 2016\
\ \ volume~10\ \ \ issue\ 4}
}%
 \def\rightfootline{\small{INFORMATIKA I EE PRIMENENIYA~---
INFORMATICS AND APPLICATIONS\ \ \ 2016\ \ \ volume~10\ \ \ issue\ 4
\hfill \textbf{\thepage}}}

\def\leftkol{2016 AUTHOR INDEX} % ENGLISH ABSTRACTS}

\def\rightkol{2016 AUTHOR INDEX} %ENGLISH ABSTRACTS}


{\tabcolsep=3pt
\begin{tabular}{p{382pt}cc}
&\textbf{Issue} & \textbf{Page}\\[6pt]
\Avtors{Sopin~E.\,S.} see~Gaidamaka~Yu.\,V.&&\\
\Avtors{Strijov~V.\,V.} see~Goncharov~A.\,V.&&\\
\Avtors{Strijov~V.\,V.} see~Isachenko~R.\,V.&&\\
\Avtors{Strijov~V.\,V.} see~Karasikov~M.\,E.&&\\
\Avtors{Stupnikov~S.\,A., Briukhov~D.\,O., and Skvortsov~N.\,A.}
Co-lending systemic risk analysis over\linebreak
\\[-12pt]
\hspace*{23pt}heterogeneous data collections&1&23--33\\
\Avtors{Stupnikov~S.\,A.} see~Kalinichenko~L.\,A.&&\\
\Avtors{Suchkov~A.\,P.} see~Zatsarinny~A.\,A.&&\\
\Avtors{Timonina~E.\,E.} see~Grusho~A.\,A.&&\\
\Avtors{Titova~A.\,I.} see~Kudryavtsev~A.\,A.&&\\
\Avtors{Turlikov~A.\,M.} see~Ometov~A.\,Ya.&&\\
\Avtors{Tyrsin~A.\,N.\ and Serebryanskii~S.\,M.} Recognition of
dependences on the basis of inverse\linebreak
\\[-12pt]
\hspace*{23pt}mapping&2&58--64\\
\Avtors{Ulyanov~V.\,V.} see~Markov~A.\,S.&&\\
\Avtors{Ushakov~V.\,G.} Queueing system with working vacations and
hyperexponential input stream&2&92--97\\
\Avtors{Ushakov~V.\,G.} see~Leontyev~N.\,D.&&\\
\Avtors{Volnova~A.\,A.} see~Kalinichenko~L.\,A.&&\\
\Avtors{Yakovlev~O.\,A.\ and Gasilov~A.\,V.} Speeded-up stereo
matching using geodesic support weights&3&\hphantom{1}98--104\\
\Avtors{Zabezhailo~M.\,I.} see~Grusho~A.\,A.&&\\
\Avtors{Zabezhailo~M.\,I.} see~Grusho~A.\,A.&&\\
\Avtors{Zakharova~T.\,V.\ and Shestakov~O.\,V.} Precision analysis of
wavelet processing of aerodynamic\linebreak
\\[-12pt]
\hspace*{23pt}flow patterns&3&46--54\\
\Avtors{Zalizniak~Anna~A.\ and Kruzhkov~M.\,G.} Database
of~Russian impersonal verbal constructions&4&132--141\\
\Avtors{Zasypko~V.\,V.} see~Shnurkov~P.\,V.&&\\
\Avtors{Zatsarinny~A.\,A.\ and Suchkov~A.\,P.} Systems engineering
approaches to~the~establishment of\linebreak
\\[-12pt]
\hspace*{23pt}a~system for~decision support based
on~situational analysis&4&105--113\\
\Avtors{Zatsarinny~A.\,A.} see~Grusho~A.\,A.&&\\
\Avtors{Zatsman~I.\,M., Inkova~O.\,Yu., Kruzhkov~M.\,G., and
Popkova~N.\,A.} Representation of cross-\linebreak
\\[-12pt]
\hspace*{23pt}lingual knowledge about
connectors in supracorpora databases&1&106--118\\
\Avtors{Zatsman~I.\,M.} see~Minin~V.\,A.&&\\
\Avtors{Zeifman~A.\,I.} see~Korolev~V.\,Yu.&&\\
\Avtors{Zeifman~A.\,I.} see~Korolev~V.\,Yu.&&\\
\end{tabular}
}

%\thispagestyle{myheadings}
\def\leftfootline{\small{\textbf{\thepage}
\hfill INFORMATIKA I EE PRIMENENIYA~--- INFORMATICS AND APPLICATIONS\ \ \ 2016\
\ \ volume~10\ \ \ issue\ 4}
}%
 \def\rightfootline{\small{INFORMATIKA I EE PRIMENENIYA~---
INFORMATICS AND APPLICATIONS\ \ \ 2016\ \ \ volume~10\ \ \ issue\ 4
\hfill \textbf{\thepage}}}

 \label{end\stat}

\newpage

%\def\stat{rekl}
%\label{preobr}

%\def\tit{АКАДЕМИК ПУГАЧЁВ  ВЛАДИМИР СЕМЁНОВИЧ\\
%25.03.1911--25.03.1998}


%   \vspace*{-48pt}
%   \begin{center}\LARGE
%Академик Пугачёв  Владимир Семёнович\\ (25.03.1911--25.03.1998)
%   \end{center}
   
   %\vspace*{2.5mm}
   
   \begin{center}

{\prgsh\LARGE
ОБЪЯВЛЕНИЯ О КОНФЕРЕНЦИЯХ}

\end{center}
%\hrule

\vspace*{6pt}

   
   \vspace*{10mm}
   
   \thispagestyle{empty}

\noindent
\begin{tabular}{cc}
%\begin{center}
\multicolumn{1}{c}{\raisebox{-40pt}[0pt][0pt]{\mbox{%
\epsfxsize=33mm
\epsfbox{vspu.eps}
}}}
%\end{center}
&
\tabcolsep=0pt\begin{tabular}{c}
{\prg{\Large\textbf{XII Всероссийское совещание}}}\\[6pt]
{\prg{\Large\textbf{по проблемам управления}}}\\[12pt]
{\prg{\large 16--19 июня 2014~г.}}\\[6pt] 
{\prg{\large Институт проблем управления имени В.\,А.~Трапезникова РАН}}\\[6pt]
{\prg{\large Москва, Россия}}
\end{tabular}
\end{tabular}

\vspace*{60pt}

     
 { %\large    
 XII Всероссийское совещание по проблемам управления (ВСПУ XII), посвященное 75-летию 
Института проблем управления (ИПУ) имени В.\,А.~Трапезникова РАН, проводится 16--19~июня 
2014~г.\ 
в ИПУ РАН (г.~Москва, Россия). ВСПУ XII организуется ИПУ РАН при поддержке РФФИ, Отделения 
энергетики, машиностроения, механики и процессов управления Российской академии наук, 
Российского 
национального комитета по автоматическому управлению, Академии навигации и управ\-ле\-ния 
движением, 
Научного совета РАН по комплексным проблемам управления и автоматизации, Совета по 
мехатронике и робототехнике РАН. Официальный язык Совещания~--- русский.

\vspace*{24pt}
     
     \textbf{Направления работы}
     \begin{enumerate}[1.]
\item Теория систем управления
\item Управление подвижными объектами и навигация
\item Интеллектуальные системы управления
\item Управление в промышленности, транспортом и логистикой
\item Управление системами междисциплинарной природы
\item Средства измерения, вычислений и контроля в управлении
\item Системный анализ и принятие решений в задачах управления
\item Информационные технологии в управлении
\item Проблемы образования в области управления: современное содержание и технологии обучения
\end{enumerate}

\vspace*{24pt}

     Подробная информация о Совещании находится на сайте {\sf http://vspu2014.ipu.ru}. Срок 
окончательной подачи докладов через систему подачи докладов на сайте~--- \textbf{30~ноября} 
2013~г.
}

%\include{rekl-1}

%\end{document}

%\include{nekrolog-rb}


%\end{document}

%\include{IPPM-25}

\def\stat{cont-rus}
{%\hrule\par
%\vskip 7pt % 7pt
\vspace*{-24pt}
\raggedleft\Large \bf%\baselineskip=3.2ex
Правила подготовки рукописей  для публикации в журнале
<<Информатика~и~её~применения>> \vskip 8pt
    \hrule
    \par
\vskip 14pt plus 6pt minus 3pt }

\label{st\stat}

\def\tit{\ }

\def\aut{\ }
\def\auf{\ }

\def\leftkol{\ }
% Правила подготовки рукописей  для публикации в журнале
%<<Информатика и её применения>>

\def\rightkol{\ }
%Правила подготовки рукописей  для публикации в журнале
%<<Информатика и её применения>>}


\titele{\tit}{\aut}{\auf}{\leftkol}{\rightkol}


\vspace*{-60pt}
{ %\small

Журнал <<Информатика и её применения>>
публикует теоретические, обзорные и дискуссионные статьи,
посвященные научным исследованиям и разработкам в области
информатики и ее приложений.

Журнал издается на русском языке. По специальному решению
редколлегии отдельные статьи могут печататься на английском языке.

Тематика журнала охватывает следующие направления:
\begin{itemize}
\item теоретические основы информатики;\\[-15pt]
      \item
математические методы исследования сложных систем и процессов;\\[-15pt]
           \item
информационные системы и сети;\\[-15pt]
                \item
информационные технологии;\\[-15pt]
                     \item
архитектура и программное обеспечение вычислительных комплексов и сетей.\\[-15pt]
\end{itemize}


\noindent
\begin{enumerate}[1.]
\item В журнале печатаются статьи, содержащие результаты, ранее не опубликованные и
не предназначенные к одновременной публикации в других изданиях.

%Публикация не должна нарушать закон об авторских правах.
Публикация предоставленной автором(ами) рукописи не должна нарушать 
положений глав~69, 70 раздела~VII части~IV Гражданского кодекса, 
которые определяют права на результаты интеллектуальной деятельности 
и~средства индивидуализации, в~том числе авторские права, в~РФ.

Ответственность за нарушение авторских прав, в~случае предъявления претензий к~редакции журнала,  
несут авторы статей.



Направляя рукопись в редакцию, авторы сохраняют свои права на данную
рукопись и при этом передают учредителям и редколлегии журнала неисключительные права на
издание статьи на русском языке 
(или на языке статьи, если он отличен от рус\-ско\-го) и~на перевод ее на английский
язык, а~также на
ее распространение в России и за рубежом. 
Каждый автор должен представить в~редакцию подписанный 
с~его стороны <<Лицензионный договор о~передаче неисключительных прав 
на использование произведения>>, текст которого размещен по адресу 
{\sf http://www.ipiran.ru/publications/licence.doc}. 
Этот договор может быть пред\-став\-лен в~бумажном (в~2-х экз.)\ 
или в~электронном виде (отсканированная копия заполненного и~подписанного документа).




Редколлегия вправе запросить у авторов экспертное заключение о возможности
пуб\-ли\-ка\-ции пред\-став\-лен\-ной статьи в открытой печати.\\[-13.5pt]

\item К статье прилагаются данные автора (авторов) (см.\ п.~8). При наличии нескольких
авторов указывается фамилия автора, ответственного за переписку с редакцией.\\[-13.5pt]

\item Редакция журнала осуществляет экспертизу присланных статей в соответствии с
принятой в журнале процедурой рецензирования.

Возвращение рукописи на доработку не означает ее принятия к печати.

Доработанный вариант с ответом на замечания рецензента необходимо прислать в
редакцию.\\[-13.5pt]

\item Решение редколлегии о публикации статьи или ее отклонении сообщается авторам.

Редколлегия может также направить авторам текст рецензии на их статью. Дискуссия по
поводу отклоненных статей не ведется.\\[-13.5pt]

%\pagebreak

\item Редактура статей высылается авторам для просмотра. Замечания к редактуре должны
быть присланы авторами в кратчайшие сроки.\\[-13.5pt]

\item Рукопись предоставляется в электронном виде в форматах MS WORD (.doc или
.docx) или \LaTeX\  (.tex), дополнительно~--- в формате .pdf, на дискете, лазерном диске
или электронной почтой. Предоставление бумажной рукописи необязательно.\\[-13.5pt]

\item При подготовке рукописи в MS Word рекомендуется использовать следующие
настройки.

Параметры страницы:
формат~--- А4; ориентация~--- книжная; поля (см): внутри~--- 2,5, снаружи~--- 1,5,
сверху~--- 2, снизу~--- 2, от края до нижнего колонтитула~--- 1,3.

Основной текст: стиль~--- <<Обычный>>, шрифт~--- Times New Roman, размер~---
14~пунк\-тов, абзацный отступ~--- 0,5~см, 1,5~интервала, выравнивание~--- по ширине.

\pagebreak

\def\leftkol{Правила подготовки рукописей  для публикации в журнале
<<Информатика и её применения>>}

\def\rightkol{Правила подготовки рукописей  для публикации в журнале
<<Информатика и её применения>>}



Рекомендуемый объем рукописи~--- не свыше 10~страниц указанного формата.
При превышении указанного объема редколлегия вправе потребовать от 
автора сокращения объема рукописи.


Сокращения слов, помимо стандартных, не допускаются. Допускается минимальное
количество аббревиатур.


Все страницы рукописи нумеруются.

Шаблоны оформления представлены в интернете:

\noindent
 {\sf
http://www.ipiran.ru/journal/template\_iiep\_ssi\_2024.zip}\\[-14pt]

\item Статья должна содержать следующую информацию на {\bfseries\textit{русском и
английском языках}}:\\[-16pt]

\begin{itemize}
\item название статьи;\\[-15pt]
\item Ф.И.О.\ авторов, на английском можно только имя и фамилию;\\[-15pt]
\item место работы, с указанием почтового адреса организации и электронного адреса каждого
автора;\\[-15pt]
\item сведения об авторах, в соответствии с форматом, образцы которого
представлены на страницах:



\def\leftfootline{\small{\textbf{\thepage}
\hfill ИНФОРМАТИКА И ЕЁ ПРИМЕНЕНИЯ\ \ \ том\ 18\ \ \ выпуск\ 3\ \ \ 2024}
}%
 \def\rightfootline{\small{ИНФОРМАТИКА И ЕЁ ПРИМЕНЕНИЯ\ \ \ том\ 18\ \ \ выпуск\ 3\ \ \ 2024
\hfill \textbf{\thepage}}}



{\sf http://www.ipiran.ru/journal/issues/2013\_07\_01/authors.asp} и

{\sf http://www.ipiran.ru/journal/issues/2013\_07\_01\_eng/authors.asp};
\item аннотация (не менее 100~слов на каждом из языков). Аннотация~--- это краткое
резюме работы, которое может публиковаться отдельно. Она является основным
источником информации в~ин\-фор\-ма\-ци\-он\-ных системах и базах данных. Английская
аннотация должна быть оригинальной, может не быть дословным переводом русского
текста и должна быть написана хорошим английским языком. В~аннотации не должно
быть ссылок на литературу и, по возможности, формул;\\[-15pt]
\item ключевые слова~--- желательно из принятых в мировой
на\-уч\-но-тех\-ни\-че\-ской литературе тематических тезаурусов. Предложения не
могут быть ключевыми словами;\\[-15pt]
\item источники финансирования работы (ссылки на гранты, проекты,
поддерживающие организации и~т.\,п.).
\end{itemize}



%\pagebreak

\item  Требования к спискам литературы.\\[-14pt]

Ссылки на литературу в тексте статьи нумеруются (в квадратных скобках) и
располагаются в каждом из списков литературы в порядке  первых упоминаний. Если источник имеет DOI и/или EDN,
то их необходимо указывать.

Списки литературы представляются в двух вариантах:\\[-14pt]


\noindent
\begin{enumerate}[(1)]
\item \textbf{Список литературы к русскоязычной части}. Русские и английские
работы~---  на языке и в алфавите оригинала;\\[-14.5pt]
\item  \textbf{References}. Русские работы и работы на других языках~--- в латинской
транслитерации с переводом на английский язык; английские работы и работы на других
языках~--- на языке оригинала.
\end{enumerate}

Необходимо для составления списка ``References'' пользоваться размещенной на сайте
{\sf http://www. translit.net/ru/bgn/} бесплатной программой транслитерации русского
 текста в~латиницу. %, при этом в~за\-клад\-ке <<варианты\ldots>> следует выбратьопцию BGN.

Список литературы ``References'' приводится полностью отдельным блоком, повторяя все
позиции из списка литературы к русскоязычной части, независимо от того, имеются или
нет в нем иностранные источники. Если в списке литературы к русскоязычной части есть
ссылки на иностранные публикации, набранные латиницей, они полностью повторяются в
списке ``References''.

Ниже приведены примеры ссылок на различные виды публикаций в списке ``References''.

\def\leftfootline{\small{\textbf{\thepage}
\hfill ИНФОРМАТИКА И ЕЁ ПРИМЕНЕНИЯ\ \ \ том\ 18\ \ \ выпуск\ 3\ \ \ 2024}
}%
 \def\rightfootline{\small{ИНФОРМАТИКА И ЕЁ ПРИМЕНЕНИЯ\ \ \ том\ 18\ \ \ выпуск\ 3\ \ \ 2024
\hfill \textbf{\thepage}}}

{\small

\noindent
\textbf{Описание статьи из журнала:}

\Aue{Zagurenko, A.\,G., V.\,A.~Korotovskikh, A.\,A.~Kolesnikov, A.\,V.~Timonov, and D.\,V.~Kardymon}. 2008.
Tekhniko-ekonomicheskaya optimizatsiya dizayna gidrorazryva plasta [Technical and
economic optimization of the design
of hydraulic fracturing]. \textit{Neftyanoe hozyaystvo} [\textit{Oil Industry}] 11:54--57.

\Aue{Zhang, Z., and D.~Zhu}. 2008. Experimental research on the localized
electrochemical micromachining. \textit{Russ. J.~Electrochem.}  44(8):926--930.
{\sf doi:10.1134/S1023193508080077}.

\noindent
\textbf{Описание статьи из электронного журнала:}

\Aue{Swaminathan, V., E.~Lepkoswka-White, and B.\,P.~Rao}. 1999. Browsers or buyers in cyberspace? An
investigation of electronic factors influencing electronic exchange. \textit{JCMC}
5(2). Available at: {\sf http://www.ascusc.org/jcmc/vol5/issue2/} (accessed April~28, 2011).

\def\leftkol{Правила подготовки рукописей  для публикации в журнале
<<Информатика и её применения>>}

\def\rightkol{Правила подготовки рукописей  для публикации в журнале
<<Информатика и её применения>>}


\noindent
\textbf{Описание статьи из продолжающегося издания (сборника трудов):}

\Aue{Astakhov, M.\,V., and T.\,V.~Tagantsev}. 2006. Eksperimental'noe
issledovanie prochnosti soedineniy ``stal'--kompozit'' [Experimental study of
the strength of joints ``steel--composite'']. \textit{Trudy MGTU
``Matematicheskoe modelirovanie slozhnykh tekh\-ni\-che\-skikh sistem''}
[\textit{Bauman MSTU ``Mathematical Modeling of Complex Technical
Systems'' Proceedings}]. 593:125--130.


\pagebreak



\noindent
\textbf{Описание материалов конференций:}

\Aue{Usmanov, T.\,S., A.\,A.~Gusmanov, I.\,Z.~Mullagalin, R.\,Ju.~Muhametshina, A.\,N.~Chervyakova, and
A.\,V.~Sveshnikov}. 2007. Osobennosti proektirovaniya razrabotki mestorozhdeniy
s primeneniem gidrorazryva
plasta [Features of the design of field development with the use of hydraulic fracturing].
\textit{Trudy 6-go
Mezhdu\-na\-rod\-no\-go Simpoziuma ``Novye resursosberegayushchie tekhnologii nedropol'zovaniya i povysheniya
neftegazootdachi''} [\textit{6th  Symposium (International) ``New Energy Saving Subsoil Technologies and
the Increasing of the Oil and Gas Impact'' Proceedings}]. Moscow. 267--272.



\def\leftfootline{\small{\textbf{\thepage}
\hfill ИНФОРМАТИКА И ЕЁ ПРИМЕНЕНИЯ\ \ \ том\ 18\ \ \ выпуск\ 3\ \ \ 2024}
}%
 \def\rightfootline{\small{ИНФОРМАТИКА И ЕЁ ПРИМЕНЕНИЯ\ \ \ том\ 18\ \ \ выпуск\ 3\ \ \ 2024
\hfill \textbf{\thepage}}}



\noindent
\textbf{Описание книги (монографии, сборники):}



Lindorf, L.\,S., and L.\,G.~Mamikoniants, eds. 1972.
\textit{Ekspluatatsiya turbogeneratorov s neposredstvennym
okhlazhdeniem} [\textit{Operation of turbine generators with direct cooling}].
Moscow: Energy Publs. 352~p.


\Aue{Latyshev, V.\,N.} 2009. \textit{Tribologiya rezaniya. Kn.~1: Friktsionnye protsessy
pri rezanii metallov}
[\textit{Tribology of cutting. Vol.~1: Frictional processes in metal cutting}]. Ivanovo: Ivanovskii
State Univ. 108~p.

\def\leftkol{Правила подготовки рукописей  для публикации в журнале
<<Информатика и её применения>>}

\def\rightkol{Правила подготовки рукописей  для публикации в журнале
<<Информатика и её применения>>}

\noindent
\textbf{Описание переводной книги}
(в списке литературы к русскоязычной части необходимо указать:~/ Пер.\ с англ.~---
после названия книги, а в конце ссылки указать оригинал книги в круглых скобках):
\begin{enumerate}[1.]
\item  В русскоязычной части:

\def\leftfootline{\small{\textbf{\thepage}
\hfill ИНФОРМАТИКА И ЕЁ ПРИМЕНЕНИЯ\ \ \ том\ 18\ \ \ выпуск\ 3\ \ \ 2024}
}%
 \def\rightfootline{\small{ИНФОРМАТИКА И ЕЁ ПРИМЕНЕНИЯ\ \ \ том\ 18\ \ \ выпуск\ 3\ \ \ 2024
\hfill \textbf{\thepage}}}

\Au{Тимошенко С.\,П., Янг Д.\,Х., Уивер~У.}
Колебания в инженерном деле~/ Пер.\ с англ.~--- М.: Машиностроение, 1985. 472~с.
(\Au{Timoshenko~S.\,P., Young~D.\,H., Weaver~W.}
Vibration problems in engineering.~--- 4th ed.~--- New York, NY, USA: Wiley, 1974. 521~p.)\\[-13.5pt]
\item  В англоязычной части:

\Aue{Timoshenko, S.\,P., D.\,H.~Young, and W.~Weaver}.
1974. \textit{Vibration problems in engineering}. 4th ed. New York: 
Wiley. 521~p.
\end{enumerate}

\vspace*{-3pt}


\noindent
\textbf{Описание неопубликованного документа:}


\Aue{Latypov, A.\,R., M.\,M.~Khasanov, and V.\,A.~Baikov}.
2004 (unpubl.). Geologiya i~dobycha (NGT GiD) [Geology and production (NGT GiD)]. Certificate on official registration of the computer program
No.\,2004611198. 

\noindent
\textbf{Описание интернет-ресурса:}


Pravila tsitirovaniya istochnikov [Rules for the citing of sources]. Available at: {\sf
http://www.scribd.com/doc/1034528/} (accessed February~7, 2011).

%\pagebreak

\noindent
\textbf{Описание диссертации или автореферата диссертации:}

\Aue{Semenov, V.\,I.}
2003. Matematicheskoe modelirovanie plazmy v sisteme kompaktnyy tor [Mathematical
modeling of the plasma in the compact torus].  Moscow.  D.Sc.\ Diss. 272~p.

\Aue{Kozhunova, O.\,S.} 2009. Tekhnologiya razrabotki semanticheskogo
slovarya informatsionnogo monitoringa [Technology of development of
semantic dictionary of information monitoring system].  Moscow: IPI RAN. PhD Thesis. 23~p.


\noindent
\textbf{Описание ГОСТа:}

GOST 8.586.5-2005. 2007. Metodika vypolneniya izmereniy. Izmerenie raskhoda i~kolichestva zhidkostey i~gazov
s~pomoshch'yu standartnykh suzhayushchikh ustroystv [Method of measurement.
Measurement of flow rate and volume of liquids and gases by means of orifice devices]. Moscow:
Standardinform  Publs. 10~p.

\noindent
\textbf{Описание патента:}

\Aue{Bolshakov, M.\,V., A.\,V.~Kulakov, A.\,N.~Lavrenov, and M.\,V.~Palkin}.
2006. Sposob orientirovaniya po krenu letatel'nogo
apparata s opti\-che\-skoy golovkoy
samonavedeniya [The way to orient on the roll of aircraft with optical homing head].
Patent RF No.\,2280590.
}

\item Присланные в редакцию материалы авторам не возвращаются.\\[-13.5pt]

\item При отправке файлов по электронной почте просим придерживаться следующих
правил:
\begin{itemize}
\item указывать в поле subject (тема) название журнала и фамилию автора;\\[-13.5pt]
\item указывать в тексте письма название статьи, авторов и~журнал, в~который направляется статья;\\[-13.5pt]
\item использовать attach (присоединение);\\[-13.5pt]
\item в состав электронной версии статьи должны входить: файл, содержащий текст
статьи, и файл(ы), содержащий(е) иллюстрации.\\[-13.5pt]
\end{itemize}

\item Журнал <<Информатика и её применения>> является некоммерческим изданием.
Плата за публикацию не взимается, гонорар авторам не выплачивается.
\end{enumerate}



\def\leftfootline{\small{\textbf{\thepage}
\hfill ИНФОРМАТИКА И ЕЁ ПРИМЕНЕНИЯ\ \ \ том\ 18\ \ \ выпуск\ 3\ \ \ 2024}
}%
 \def\rightfootline{\small{ИНФОРМАТИКА И ЕЁ ПРИМЕНЕНИЯ\ \ \ том\ 18\ \ \ выпуск\ 3\ \ \ 2024
\hfill \textbf{\thepage}}}


\vspace*{-1mm}

\begin{center}

\textbf{Адрес редакции журнала <<Информатика и её применения>>:} \\




Москва 119333, ул.~Вавилова, д.~44, корп.~2, ФИЦ ИУ РАН\\[-10pt]

\

Тел.: +7\,(499)\,135-86-92\ \ Факс:  +7\,(495)\,930-45-05\\[-10pt]

 \

e-mail:   {\sf iiep@frccsc.ru} (Стригина Светлана Николаевна)\\[-10pt]

\

{\sf http://www.ipiran.ru/journal/issues/}
\end{center}
}


\def\leftkol{Правила подготовки рукописей  для публикации в журнале
<<Информатика и её применения>>}

\def\rightkol{Правила подготовки рукописей  для публикации в журнале
<<Информатика и её применения>>}


\def\leftfootline{\small{\textbf{\thepage}
\hfill ИНФОРМАТИКА И ЕЁ ПРИМЕНЕНИЯ\ \ \ том\ 18\ \ \ выпуск\ 3\ \ \ 2024}
}%
 \def\rightfootline{\small{ИНФОРМАТИКА И ЕЁ ПРИМЕНЕНИЯ\ \ \ том\ 18\ \ \ выпуск\ 3\ \ \ 2024
\hfill \textbf{\thepage}}} 
\def\stat{podg-e}
{%\hrule\par
%\vskip 7pt % 7pt
\vspace*{-24pt}
\raggedleft\Large \bf%\baselineskip=3.2ex
Requirements for manuscripts submitted to Journal
``Informatics~and~Applications'' \vskip 8pt
    \hrule
    \par
\vskip 21pt plus 6pt minus 3pt }

\label{st\stat}

\def\tit{\ }

\def\aut{\ }
\def\auf{\ }

\def\leftkol{\ }

\def\rightkol{\ }
%Requirements for manuscripts submitted to Journal
%``Informatics~and~Applications''}

\titele{\tit}{\aut}{\auf}{\leftkol}{\rightkol}

\def\leftfootline{\small{\textbf{\thepage}
\hfill INFORMATIKA I EE PRIMENENIYA~--- INFORMATICS AND APPLICATIONS\ \ \ 2019\
\ \ volume~13\ \ \ issue\ 4}
}%
 \def\rightfootline{\small{INFORMATIKA I EE PRIMENENIYA~--- INFORMATICS AND APPLICATIONS\ \ \ 2019\ \ \ volume~13\ \ \ issue\ 4
\hfill \textbf{\thepage}}}

\vspace*{-60pt}

{\small

\noindent
Journal ``Informatics and Applications'' (Inform.\ Appl.)
publishes theoretical, review, and discussion
articles on the research and development in the
field of informatics and its applications.

The journal is published in Russian.
By a special decision of the editorial
board, some articles can be published in English.


The topics covered include the following areas:
\begin{itemize}
               \item
     theoretical fundamentals of informatics; \\[-14pt]
\item
mathematical methods for studying complex systems and processes; \\[-14pt]
\item
information systems and networks;\\[-14pt]
\item
information technologies; and \\[-14pt]
\item
architecture and software of computational complexes and networks. \\[-14pt]
\end{itemize}

\noindent
\begin{enumerate}[1.]
\item The Journal publishes original articles which have not been published before and are not
intended for simultaneous publication in other editions. An article submitted to the Journal must not violate the
Copyright law. Sending the manuscript to the Editorial Board, the authors retain all rights of the
owners of the manuscript and transfer the nonexclusive rights to publish the article in Russian
(or the language of the article, if not Russian) and its distribution in Russia and abroad to the
Founders and the Editorial Board. Authors should submit a letter to the Editorial Board in the
following form:

{\bfseries\textit{Agreement on the transfer of rights to publish:}}

``\textit{We, the undersigned authors of the manuscript ``\ldots'', pass to the
Founder and the Editorial Board of the Journal ``Informatics and Applications''
the nonexclusive right to publish the manuscript of the article in Russian (or
in English) in both print and electronic versions of the Journal. We affirm
that this publication does not violate the Copyright of other persons or
organizations.}

\textit{Author(s) signature(s): (name(s), address(es), date).}

This agreement should be submitted in paper form or in the form of a scanned copy (signed by
the authors).


%The Editorial Board has the right to request from the authors an official expert conclusion that
%the submitted article has no secret data prohibited for publication. \\[-13.5pt]
\item
A submitted article should be attached with \textbf{the data on the author(s)} (see item~8). If
there are several authors, the contact person should be indicated who is responsible for
correspondence with the Editorial Board and other authors about revisions and final approval
of the proofs.\\[-13.5pt]

\item The Editorial Board of the Journal examines the article according to the established
reviewing procedure. If the authors receive their article for correction after reviewing, it does not
mean that the article is approved for publication. The corrected article should be sent to the
Editorial Board for the subsequent review and approval.\\[-13.5pt]

\item The decision on the article publication or its rejection is communicated to the authors. The
Editorial Board may also send the reviews on the submitted articles to the authors. Any
discussion upon the rejected articles is not possible.\\[-13.5pt]

\item The edited articles will be sent to the authors for proofread. The comments of the authors
to the edited text of the article should be sent to the Editorial Board as soon as possible.\\[-13.5pt]

\item The manuscript of the article should be presented electronically in the MS WORD (.doc or
.docx) or \LaTeX\ (.tex) formats, and additionally in the .pdf format. All documents
 may be sent
by e-mail or provided on a CD or diskette. A~hard copy submission is not necessary.\\[-13.5pt]

\item The recommended typesetting instructions for manuscript.

Pages parameters: format A4, portrait orientation, document margins (cm): left~--- 2.5, right~---
1.5, above~--- 2.0, below~--- 2.0, footer 1.3.

Text: font~---Times New Roman, font size~--- 14, paragraph indent~--- 0.5, line spacing~--- 1.5,
justified alignment.

The recommended manuscript size: not more than 15~pages of the specified format.
If the specified size exceeded, the editorial board is entitled to require the author
to reduce the manuscript.

Use only standard abbreviations. Avoid  abbreviations in the title and
abstract. The full term for which an abbreviation stands should precede
its first use in the text unless it is a standard unit of measurement.

All pages of the manuscript should be numbered.

The templates for the manuscript typesetting are presented on site: {\sf
http://www.ipiran.ru/journal/template.doc}.\\[-13.5pt]


%\def\leftkol{Requirements for manuscripts submitted to Journal
%``Informatics~and~Applications''}

\item The articles should enclose data both in \textbf{Russian and English}:
\begin{itemize}
\item title;\\[-13.5pt]
\item author's name and surname;\\[-13.5pt]
\item affiliation~--- organization, its address with ZIP code, city, country, and
official e-mail address;\\[-13.5pt]
\item data on authors according to the format: (see site)

{\sf http://www.ipiran.ru/journal/issues/2013\_07\_01/authors.asp}  and

{\sf  http://www.ipiran.ru/journal/issues/2013\_07\_01\_eng/authors.asp};\\[-13.5pt]

\pagebreak

\def\leftfootline{\small{\textbf{\thepage}
\hfill INFORMATIKA I EE PRIMENENIYA~--- INFORMATICS AND APPLICATIONS\ \ \ 2019\
\ \ volume~13\ \ \ issue\ 4}
}%
 \def\rightfootline{\small{INFORMATIKA I EE PRIMENENIYA~--- INFORMATICS AND APPLICATIONS\ \ \ 2019\ \ \ volume~13\ \ \ issue\ 4
\hfill \textbf{\thepage}}}


%\def\leftkol{Requirements for manuscripts submitted to Journal
%``Informatics~and~Applications''}

%\def\rightkol{Requirements for manuscripts submitted to Journal
%``Informatics~and~Applications''}



\item abstract (not less than 100 words) both in Russian and in English. Abstract is a short
summary of the article that can be published separately. The abstract is the
main source of information on the article and it could be included in leading information
systems and data bases. The abstract in English has to be an original text and should
not be an exact translation of the Russian one. Good English is required.
In abstracts, avoid references and formulae;\\[-13.5pt]
\item indexing is performed on the basis of keywords. The use of keywords from the
internationally accepted thematic Thesauri is recommended.

%\def\leftkol{Requirements for manuscripts submitted to Journal
%``Informatics~and~Applications''}

%\def\rightkol{Requirements for manuscripts submitted to Journal
%``Informatics~and~Applications''}

Important! Keywords must not be sentences;
\item Acknowledgments.
\end{itemize}

\item References. Russian references have to be presented both in English translation and Latin
transliteration (refer {\sf http://www.translit.net/ru/bgn/}).

Please take into account the following examples of Russian references appearance:

\noindent
\textbf{Article in journal:}

\Aue{Zhang, Z., and D.~Zhu}. 2008. Experimental research on the localized electrochemical
micromachining.
\textit{Rus. J.~Electrochem.}  44(8):926--930. {\sf doi:10.1134/S1023193508080077}.


\noindent
\textbf{Journal article in electronic format:}

\Aue{Swaminathan, V., E.~Lepkoswka-White, and B.\,P.~Rao}. 1999. Browsers or buyers in
cyberspace? An
investigation of electronic factors influencing electronic exchange. \textit{JCMC}
5(2). Available at: {\sf http://www.ascusc.org/jcmc/vol5/issue2/} (accessed April~28, 2011).




\noindent
\textbf{Article from the continuing publication (collection of works, proceedings):}

\Aue{Astakhov, M.\,V., and T.\,V.~Tagantsev}. 2006. Eksperimental'noe
issledovanie prochnosti soedineniy ``stal'--kompozit'' [Experimental study of
the strength of joints ``steel--composite'']. \textit{Trudy MGTU
``Matematicheskoe modelirovanie slozhnykh tekh\-ni\-che\-skikh sistem''}
[\textit{Bauman MSTU ``Mathematical Modeling of Complex Technical
Systems'' Proceedings}]. 593:125--130.

\def\leftfootline{\small{\textbf{\thepage}
\hfill INFORMATIKA I EE PRIMENENIYA~--- INFORMATICS AND APPLICATIONS\ \ \ 2019\
\ \ volume~13\ \ \ issue\ 4}
}%
 \def\rightfootline{\small{INFORMATIKA I EE PRIMENENIYA~--- INFORMATICS AND APPLICATIONS\ \ \ 2019\ \ \ volume~13\ \ \ issue\ 4
\hfill \textbf{\thepage}}}

\def\leftkol{Requirements for manuscripts submitted to Journal
``Informatics~and~Applications''}

\def\rightkol{Requirements for manuscripts submitted to Journal
``Informatics~and~Applications''}

\noindent
\textbf{Conference proceedings:}

\Aue{Usmanov, T.\,S., A.\,A.~Gusmanov, I.\,Z.~Mullagalin, R.\,Ju.~Muhametshina,
A.\,N.~Chervyakova, and
A.\,V.~Sveshnikov}. 2007. Osobennosti proektirovaniya razrabotki mestorozhdeniy
s primeneniem gidrorazryva
plasta [Features of the design of field development with the use of hydraulic fracturing].
\textit{Trudy 6-go
Mezhdu\-na\-rod\-no\-go Simpoziuma ``Novye resursosberegayushchie tekhnologii
nedropol'zovaniya i povysheniya
neftegazootdachi''} [\textit{6th  Symposium (International) ``New Energy Saving Subsoil
Technologies and
the Increasing of the Oil and Gas Impact'' Proceedings}]. Moscow. 267--272.


\noindent
\textbf{Books and other monographs:}




Lindorf, L.\,S., and L.\,G.~Mamikoniants, eds. 1972.
\textit{Ekspluatatsiya turbogeneratorov s neposredstvennym
okhlazhdeniem} [\textit{Operation of turbine generators with direct cooling}].
Moscow: Energy Publs. 352~p.


%\Aue{Latyshev, V.\,N.} 2009. \textit{Tribologiya rezaniya. Kn.~1: Frikcionnye prosessy
%pri rezanii metallov}
%[\textit{Tribology of cutting. Vol.~1: Frictional processes in metal cutting}]. Ivanovo: Ivanovskii
%State Univ. 108~p.


%\noindent
%\textbf{Unpublished material:}

%\Aue{Latypov, A.\,R., M.\,M.~Khasanov, and V.\,A.~Baikov}.
%2004. Geology and production (NGT GiD). Certificate on official registration of the computer
%program
%No.\,2004611198. (In Russian, unpubl.)

%\noindent
%\textbf{Internet-source:}

%APA Style. 2011. Available at: {\sf http://www.apastyle.org/apa-style-help.aspx} (accessed
%February~5, 2011).

%Pravila citirovaniya istochnikov [Rules for the citing of sources]. Available at: {\sf
%http://www.scribd.com/doc/1034528/} (accessed February~7, 2011).


\noindent
\textbf{Dissertation and Thesis:}

%\Aue{Semenov, V.\,I.}
%2003. Matematicheskoe modelirovanie plazmy v sisteme kompaktnyy tor. [Mathematical
%modeling of the plasma in the compact torus]. D.Sc.\ Diss. Moscow. 272~p.

\Aue{Kozhunova, O.\,S.} 2009. Tekhnologiya razrabotki semanticheskogo
slovarya informatsionnogo monitoringa [Technology of development of
semantic dictionary of information monitoring system]. PhD Thesis. Moscow: IPI RAN. 23~p.


\noindent
\textbf{State standards and patents:}

GOST 8.586.5-2005. 2007. Metodika vypolneniya izmereniy. Izmerenie raskhoda i~kolichestva
zhidkostey i gazov 
s~pomoshch'yu standartnykh suzhayushchikh ustroystv [Method of measurement.
Measurement of flow rate and volume of liquids and gases by means of orifice devices]. M.:
Standardinform
Publs. 10~p.

%\noindent
%\textbf{Patent:}

\Aue{Bolshakov, M.\,V., A.\,V.~Kulakov, A.\,N.~Lavrenov, and M.\,V.~Palkin}.
2006. Sposob orientirovaniya po krenu letatel'nogo
apparata s opti\-che\-skoy golovkoy
samonavedeniya [The way to orient on the roll of aircraft with optical homing head].
Patent RF No.\,2280590.

References in Latin transcription are presented in the original language.

References in the text are numbered according to the order of their
first appearance; the number is
placed in square brackets. All items from the reference list should be
cited.\\[-13.5pt]

\item Manuscripts and additional materials are not returned to Authors by the Editorial Board.\\[-13.5pt]

\item Submissions of files by e-mail must include:\\[-13.5pt]
\begin{itemize}
\item   the journal title and author's name in the ``Subject'' field; \\[-13.5pt]
\item   an article and additional materials have to be attached using the ``attach'' function;\\[-13.5pt]
\item   an electronic version of the article should contain the file with the text and a separate file
with figures.\\[-13.5pt]
\end{itemize}

\item ``Informatics and Applications'' journal is not a profit publication. There are no
charges for the authors as well as there are no royalties.\\[-13.5pt]
\end{enumerate}

\def\leftfootline{\small{\textbf{\thepage}
\hfill INFORMATIKA I EE PRIMENENIYA~--- INFORMATICS AND APPLICATIONS\ \ \ 2019\
\ \ volume~13\ \ \ issue\ 4}
}%
 \def\rightfootline{\small{INFORMATIKA I EE PRIMENENIYA~--- INFORMATICS AND APPLICATIONS\ \ \ 2019\ \ \ volume~13\ \ \ issue\ 4
\hfill \textbf{\thepage}}}

\def\leftkol{Requirements for manuscripts submitted to Journal
``Informatics~and~Applications''}

\def\rightkol{Requirements for manuscripts submitted to Journal
``Informatics~and~Applications''}


%\vspace*{5mm}


\begin{center}
\textbf{Editorial Board address:} \\

%ABOUT AUTHORS



FRC CSC RAS, 44, block~2, Vavilov Str., Moscow 119333, Russia\\[-10pt]

\

Ph.: +7\,(499)\,135\,86\,92,\ \ Fax: +7\,(495)\,930\,45\,05\\[-10pt]

\

 e-mail: {\sf rust@ipiran.ru} (to Prof.\ Rustem Seyful-Mulyukov)\\[-10pt]

\

 {\sf http://www.ipiran.ru/english/journal.asp}
\end{center}
 }
%\thispagestyle{myheadings}

\def\leftkol{Requirements for manuscripts submitted to Journal
``Informatics~and~Applications''}

\def\rightkol{Requirements for manuscripts submitted to Journal
``Informatics~and~Applications''}

\def\leftfootline{\small{\textbf{\thepage}
\hfill INFORMATIKA I EE PRIMENENIYA~--- INFORMATICS AND APPLICATIONS\ \ \ 2019\
\ \ volume~13\ \ \ issue\ 4}
}%
 \def\rightfootline{\small{INFORMATIKA I EE PRIMENENIYA~--- INFORMATICS AND APPLICATIONS\ \ \ 2019\ \ \ volume~13\ \ \ issue\ 4
\hfill \textbf{\thepage}}}

 \label{end\stat}

\newpage

%\vspace*{-60pt} {\small
{\baselineskip=9.1pt
\section*{Правила подготовки рукописей статей для публикации в журнале
<<Информатика и её применения>>}

\thispagestyle{empty}

 Журнал <<Информатика и её применения>> публикует
теоретические, обзорные и дискуссионные статьи, посвященные научным
исследованиям и разработкам в области информатики и ее приложений. Журнал
издается на русском языке. По специальному решению редколлегии отдельные статьи,
в виде исключения, могут печататься на английском языке.
Тематика журнала охватывает следующие направления:
\begin{itemize}
\item теоретические основы информатики; %\\[-13.5pt]
\item математические методы исследования сложных систем и процессов; %\\[-13.5pt]
\item информационные системы и сети; %\\[-13.5pt]
\item информационные технологии; %\\[-13.5pt]
\item архитектура и программное
обеспечение вычислительных комплексов и сетей.
\end{itemize}
\begin{enumerate}
\item В журнале печатаются результаты, ранее не
опубликованные и не предназначенные к одновременной публикации в других
изданиях. Публикация не должна нарушать закон об авторских правах. Направляя
свою рукопись в редакцию, авторы автоматически передают учредителям и
редколлегии неисключительные права на издание данной статьи на русском языке и
на ее распространение в России и за рубежом. При этом за авторами сохраняются
все права как собственников данной рукописи. В связи с этим авторами должно
быть представлено в редакцию письмо в следующей форме:
Соглашение о передаче права на публикацию:

\textit{<<Мы, нижеподписавшиеся, авторы рукописи <<$\qquad\qquad$>>, передаем
учредителям и редколлегии журнала <<Информатика и её применения>>
неисключительное право опубликовать данную рукопись статьи на русском языке как
в печатной, так и в электронной версиях журнала. Мы подтверждаем, что данная
публикация не нарушает авторского права других лиц или организаций. Подписи
авторов: (ф.\,и.\,о., дата, адрес)>>.}

Указанное соглашение может быть представлено 
как в бумажном виде, так и в виде отсканированной копии (с подписями авторов).


Редколлегия вправе запросить у авторов экспертное заключение о возможности
опубликования представленной статьи в открытой печати. %\\[-13.5pt]
\item Статья
подписывается всеми авторами. На отдельном листе представляются данные автора
(или всех авторов): фамилия, полные имя и отчество, телефон, факс, e-mail,
почтовый адрес. Если работа выполнена несколькими авторами, указывается фамилия
одного из них, ответственного за переписку с редакцией. %\\[-13.5pt]
\item Редакция журнала
осуществляет самостоятельную экспертизу присланных статей. Возвращение рукописи
на доработку не означает, что статья уже принята к печати. Доработанный вариант
с ответом на замечания рецензента необходимо прислать в редакцию. %\\[-13.5pt]
\item Решение
редакционной коллегии о принятии статьи к печати или ее отклонении сообщается
авторам. Редколлегия не обязуется направлять рецензию авторам отклоненной
статьи. %\\[-13.5pt]
\item Корректура статей высылается авторам для просмотра. Редакция
просит авторов присылать свои замечания в кратчайшие сроки. %\\[-13.5pt]
\item При
подготовке рукописи в MS Word рекомендуется использовать следующие настройки.
Параметры страницы: формат~--- А4; ориентация~--- книжная; поля (см): внутри~---
2,5, снаружи~--- 1,5, сверху~--- 2, снизу~--- 2, от края до нижнего
колонтитула~--- 1,3. Основной текст: стиль~--- <<Обычный>>: шрифт Times New
Roman, размер 14~пунктов, абзацный отступ~--- 0,5~см, 1,5 интервала,
выравнивание~--- по ширине. Рекомендуемый объем рукописи~--- не свыше
25~страниц указанного формата. Ознакомиться с шаблонами, содержащими примеры
оформления, можно по адресу в Интернете:
\textsf{http://www.ipiran.ru/journal/template.doc}.
\item К рукописи, предоставляемой в 2-х
экземплярах, обязательно прилагается электронная версия статьи (как правило, в
форматах MS WORD (.doc) или \LaTeX\ (.tex), а также~--- дополнительно~--- в
формате .pdf) на дискете, лазерном диске или по электронной почте. Сокращения
слов, кроме стандартных, не применяются. Все страницы рукописи должны быть
пронумерованы. %\\[-13.5pt]
\item Статья должна содержать следующую информацию на русском и
английском языках: название, Ф.И.О. авторов, места работы авторов и их
электронные адреса, подробные сведения об авторах, оформленные в соответствии с форматом, 
определяемым файлами {\sf http://www.ipiran.ru/journal/issues/2011\_05\_01/authors.asp} и 
{\sf http://www.ipiran.ru/journal/issues/2011\_01\_eng/authors.asp},
аннотация (не более 100~слов), ключевые слова. Ссылки на
литературу в тексте статьи нумеруются (в квадратных скобках) и располагаются в
порядке их первого упоминания. В~списке литературы не должно быть позиций, на которые нет ссылки в тексте статьи.
Все фамилии авторов, заглавия статей, названия
книг, конференций и~т.\,п.\ даются на языке оригинала, если этот язык
использует кириллический или латинский алфавит. %\\[-13.5pt]
\item Присланные в редакцию материалы авторам не возвращаются.
\item При отправке файлов по электронной
почте просим придерживаться следующих правил:
\begin{itemize}
\item указывать в поле subject (тема) название журнала и фамилию автора; %\\[-13.5pt]
\item использовать attach (присоединение); %\\[-13.5pt]
\item в случае больших объемов информации возможно
использование общеизвестных архиваторов (ZIP, RAR); %\\[-13.5pt]
\item в состав электронной версии статьи должны входить: файл, содержащий текст статьи, и файл(ы),
содержащий(е) иллюстрации. %\\[-13.5pt]
\end{itemize}
\item Журнал <<Информатика и её применения>> является некоммерческим изданием. 
Плата за публикацию с авторов не взимается, гонорар авторам не выплачивается.
\end{enumerate}
\thispagestyle{empty}
\textbf{Адрес редакции:} Москва 119333,
ул.~Вавилова, д.~44, корп.~2, ИПИ РАН\\
\hphantom{\textbf{Адрес редакции:} }Тел.: +7 (499) 135-86-92\ \
Факс:  +7 (495) 930-45-05\ \  E-mail:   rust@ipiran.ru }
}

%\include{ipi-ind}

%\tableofcontents

\end{document}

%\tableofcontents

%\end{document}

%\tableofcontents


\end{document}

\newcommand{\Ack}{\subsection*{\protect\large\bf Acknowledgments}}

\vphantom*{\int\limits_0^T}

{ \begin{center}  %fig1
 \vspace*{1pt}
    \mbox{%
 \epsfxsize=79mm 
 \epsfbox{gru-1.eps}
 }

\end{center}



\noindent
{{\figurename~1}\ \ \small{
@@@@@@@@@@@@@@@@@@@@@@@@@@@@@@@
}}}

\vspace*{6pt}

\addtocounter{figure}{1}

$\acute{\mbox{о}}$

\linebreak