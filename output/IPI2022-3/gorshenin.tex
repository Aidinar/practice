\def\stat{gorshenin}

\def\tit{ПОВЫШЕНИЕ ДОХОДНОСТИ ТОРГОВЛИ НА FOREX С~ПОМОЩЬЮ  
LSTM-ИДЕНТИФИКАЦИИ СВЕЧНЫХ ПАТТЕРНОВ И~ИНДИКАТОРА ТИКОВЫХ 
ОБЪЕМОВ$^*$}

\def\titkol{Повышение доходности торговли на FOREX с~помощью  
LSTM-идентификации свечных паттернов} % и~индикатора тиковых  объемов}

\def\aut{А.\,К.~Горшенин$^1$, Е.\,И.~Гусейнова$^2$}

\def\autkol{А.\,К.~Горшенин, Е.\,И.~Гусейнова}

\titel{\tit}{\aut}{\autkol}{\titkol}

\index{Горшенин А.\,К.}
\index{Гусейнова Е.\,И.}
\index{Gorshenin A.\,K.}
\index{Guseynova E.\,I.}


{\renewcommand{\thefootnote}{\fnsymbol{footnote}} \footnotetext[1]
{Исследование выполнено при поддержке Российского научного фонда, проект 22-11-00212. Для обучения нейронных сетей использовалась инфраструктура Центра коллективного пользования 
<<Высокопроизводительные вычисления и~большие данные>> (ЦКП <<Информатика>>) ФИЦ ИУ РАН 
(г.~Москва).}


\renewcommand{\thefootnote}{\arabic{footnote}}
\footnotetext[1]{Федеральный исследовательский центр <<Информатика и~управление>> Российской академии наук, 
\mbox{agorshenin@frccsc.ru}}
\footnotetext[2]{ Московский государственный университет имени М.\,В.~Ломоносова, 
\mbox{ei.guseynova@yandex.ru}}

\vspace*{-12pt}



\Abst{Статья посвящена исследованию эффективности использования рекуррентных нейронных 
сетей LSTM (Long--Short Term Memory~--- долгая--краткосрочная память) для свечных данных 
и~индикатора технического анализа для большого числа наиболее распространенных валютных пар 
(всего~--- 27) за длительный период времени с~целью построения автоматических торговых 
стратегий. Продемонстрировано, что средняя итоговая и~годовая доходности за 8~лет при 
проведении модельных торгов составили 286\% и~15,4\% соответственно, что более чем в~50~раз 
превышает значения для классической торговой стратегии Buy \& Hold за тот же временной период. 
Кроме того, в~работе предложен новый индикатор технического анализа на основе тиковых объемов, 
который используется как самостоятельно в~качестве альтернативной торговой стратегии (итоговая 
и~годовая доходности LSTM-мо\-де\-лей превосходят ее в~среднем в~7,2 и~2,3~раза), так и~в~качестве 
дополнительного признака для повышения доходности нейросетевой стратегии за счет 
использования ансамблирования. Установлено, что для 37\% анализируемых валютных пар 
использование именно ансамбля нейронных сетей позволяет дополнительно повысить итоговую 
доходность в~среднем на~17,2\%.}

\KW{LSTM; ансамблевое обучение; свечные данные; технический индикатор; FOREX; валютные 
пары}

\DOI{10.14357/19922264220304} 
  
\vspace*{2pt}


\vskip 10pt plus 9pt minus 6pt

\thispagestyle{headings}

\begin{multicols}{2}

\label{st\stat}

\section{Введение}

\vspace*{-4pt}

  С развитием информационных технологий и~их внедрением в~работу бирж различных 
финансовых рынков у трейдеров появилась возможность использовать программы для 
осуществления торговых операций, построенные на основе их собственных методов 
и~правил торговли. 

Ведение торговли с~помощью про\-грамм-ро\-бо\-тов получило название 
автоматической торговой системы. Такие сис\-те\-мы способствовали существенному 
повышению эффективности работы трейдеров. В~отличие от механической торговой 
системы, автоматические торговые системы способны совершать торговые операции по 
купле и~продаже финансовых активов без непосредственного участия человека за счет 
встроенного алгоритма, который отвечает за автоматическую выработку сигнала на  
от\-кры\-тие/за\-кры\-тие позиции, а~также доставку ордера (заявки) на торговую платформу 
брокера. В~основе автоматических торговых сис\-тем чаще всего лежит стратегия на основе 
комбинации различных индикаторов и~паттернов технического анализа.
  
  В последние годы все большее влияние на инструменты подобной автоматизированной 
тор\-гов\-ли, как и~повсеместно в~мировом финансовом секторе~[1], стали оказывать алгоритмы 
на основе \mbox{машинного} обучения и~нейронных сетей. В~част\-ности, одним из направлений 
применения таких подходов стало построение торговых стратегий\linebreak с~более высокой 
доходностью по сравнению с~классическими инструментами. Зачастую для этого 
используется прогнозирование цен различных финансовых инструментов с~помощью 
алгоритмов машинного обучения, причем как на основе исходных ценовых данных, так 
и~расширением признакового пространства различными характеристиками, включая 
различные индикаторы технического анализа (см., например, статью~[2]). 

В~литературе 
наиболее распространены исследования для фондового рынка, однако также 
расcматриваются криптовалюты~[3] и~вы\-со\-ко\-час\-тот\-ная торговля отдельными валютными 
парами, прежде всего евро--дол\-лар~[4].
  
  Целью данной статьи ставится исследование эффективности применения рекуррентных 
нейросетевых архитектур LSTM сразу для~27~разнообразных валютных пар. В~качестве базового уровня до\-ход\-ности 
используются показатели простой тестовой\linebreak стратегии Buy \& Hold~[5], также на\-зы\-ва\-емой 
позиционной торговлей, при которой инвестор приобретает некоторые активы для 
долгосрочного хранения в~ожидании повышения цен на них в~\mbox{пределах} не менее нескольких 
лет. Кроме того, как было упомянуто выше, весьма популярны стратегии, использующие 
индикаторы технического анализа,~--- и~в~статье предложен новый инструмент, 
ориентированный на тиковые объемы. Наконец, представляет интерес построение стратегий 
с~одновременным использованием индикаторов и~нейронных сетей, которое в~данной работе 
реализуется с~помощью ансамблевого подхода к~обучению LSTM-ар\-хи\-тек\-тур.
  
  Статья организована следующим образом. В~разд.~2 приведен обзор известных решений 
на основе машинного обучения для различных финансовых инструментов. Раздел~3 
содержит описание анализируемых данных и~используемый подход для обработки свечных 
данных. В~разд.~4 представлена динамика доходности для каждой из пар, получаемая 
в~рамках использования LSTM-стратегии. Раздел~5 посвящен описанию нового технического 
индикатора и~сравнению его доходности с~нейросетевым подходом. В~разд.~6 
продемонстрирована возможность дополнительного повышения доходности 
для~10~валютных пар в~случае использования ансамблевого обучения для свечных данных 
и~этого индикатора. В~заключительном разделе подводятся краткие итоги и~обсуждаются 
возможные на\-прав\-ле\-ния дальнейших исследований.

\vspace*{-6pt}
  
\section{Машинное обучение в~задачах финансового прогнозирования}

\vspace*{-2pt}

  Одним из популярных подходов к~прогнозированию цен на фондовом рынке 
  с~использованием нейронных сетей стал графический анализ. Так, в~статье~[6] применяется 
сверточная архитектура для обработки исходных финансовых временн$\acute{\mbox{ы}}$х рядов~--- 
15~различных технических индикаторов, преобразованных в~двумерные изображения, 
каж\-дое из которых затем помечается как Buy, Sell или Hold в~зависимости от 
предполагаемых точек входа в~рынок и~выхода из него. Тестирование на данных 
промышленного индекса Доу Джонса и~биржевых инвестиционных фондов показало 
высокую эффективность данного подхода, в~частности по сравнению со стратегией Buy \& 
Hold.
  
  В статье~[7] предложена торговая система, не ограничивающаяся известными 
техническими паттернами. Она позволяет сравнивать текущее рыночное состояние с~более 
ранними похожими паттернами с~целью получения торговых сигналов. Для ее тестирования 
использовались 560~акций Нью-Йорк\-ской фондовой биржи, при этом авторы не проводили 
автоматическую и~динамическую оптимизацию параметров торговой стратегии~--- и~тем не 
менее предложенная система приводила к~92,5\% прибыльных сделок.
  
  Весьма популярны методы на основе анализа свечных данных. В~статье~[8] предложена 
ан\-самб\-ле\-вая нейронная сеть, включающая и~сверточную архитектуру CNN (Convolutional 
Neural Network), которая позволяет автоматически идентифицировать восемь типов свечных 
паттернов со средней точностью 90,7\% в~реальных данных, превосходя LSTM-сеть. Еще 
один пример анализа с~использованием свечных данных для Тайваньской фондовой биржи 
и~индекса фондового рынка Nikkei 225 Токийской фондовой биржи приведен в~статье~[9].
  
  Другая ансамблевая архитектура, сочетающая CNN и~LSTM-се\-ти, предложена 
  в~статье~[10]. Она использует последовательность исторических данных и~опережающие 
индикаторы (опционы и~фьючерсы) для извлечения дополнительных признаков с~по\-мощью 
CNN, а затем использует их в~качестве входных данных для LSTM. Для тестирования авторы 
использовали характеристики десяти американских и~тайваньских акций. 
  
  Эффективным оказалось сочетание алгоритмов машинного обучения и~нейронных сетей 
даже достаточно простых архитектур. Так, в~статье~[5] при построении модели 
прогнозирования учитывались не только ценовые данные по торговым инструментам, но 
и~данные объема торговли, а для получения оптимального набора параметров торговой 
стратегии использована комбинация регрессионного метода опорных векторов 
и~многослойного персептрона. Полученный алгоритм продемонстрировал высокие результаты 
на шести инструментах фондового рынка в~период с~2001 по 2015~гг.

   
  
  В статье~[11] поиск оптимальных параметров технического индикатора осуществляется 
  с~по\-мощью генетических алгоритмов. Затем его значения передаются в~глубокий 
многослойный перцептрон для получения одного из трех торговых сигналов (покупка, продажа, 
удержание текущей позиции). Результаты тестирования на исторических данных цен акций 
промышленного индекса Доу Джонса в~период с~2007 по 2016~гг.\ показывают, что 
оптимизация параметров технического индикатора повышает эффективность торговли 
акциями и~дает сопоставимые или лучшие результаты по сравнению с~Buy \& Hold и~другими торговыми стратегиями. 
  
  В работе~[12] показано, что прогнозирование точки разворота цены акций может быть 
весьма эффективным при использовании только LSTM-се\-тей, для которых на основе 
комбинаций свечных индикаторов и~технических индикаторов строятся наборы 
дополнительных признаков. Для 10 китайских и~10 американских акций результаты 
превзошли и~метод опорных векторов, и~многослойный перцептрон,  
и~CNN-ар\-хи\-тек\-туру.
  
  В статье~[13] предложен метод предварительной обработки исходных ценовых свечей 
  и~значений технических индикаторов для генерации торговых сигналов, используемых для 
обучения нейронной сети LSTM. Такая замена исходных ценовых данных позволила 
повысить точность предсказаний для пяти типов торговых стратегий, однако, как 
и~в~работе~[7], оптимизация их параметров не проводилась.
  
  В статье~[2] на примере фондового рынка Китая с~2000 по 2017~гг.\ получена точность 
прогноза более 60\% для некоторых моделей с~использованием анализа свечных графиков 
и~получения признаков с~помощью ансамблевых методов машинного обуче\-ния. Отмечено, что 
дополнительные технические индикаторы могут в~разной степени повысить точ\-ность 
прогноза. 
  
  Работа с~валютной парой ев\-ро--дол\-лар рас\-смот\-ре\-на в~статье~[4], в~которой 
предлагаются инструменты прогнозирования краткосрочного тренда на валютном рынке 
FOREX с~использованием глубокого обучения и~алгоритмов обучения с~подкреплением для 
высокочастотной торговли.
  
  Таким образом, весьма актуально исследование сразу большого числа валютных пар за 
длительный период на основе свечных данных с~использованием нейронных сетей 
и~инструментов технического анализа, которому и~посвящена данная статья.


   
   \vspace*{-12pt}

\section{Анализируемые данные и~создание пространства признаков}

\vspace*{-2pt}

Тиковые данные, содержащие метки времени и~цены сделки, по наиболее торгуемым 27 валютным парам 
загружены с~помощью программного интерфейса на языке Python с~торговой платформы 
<<Metatrader>>\footnote{{\sf https://www.metaquotes.net}.}. 

Начало доступного периода для каждой валютной пары приведено в~таблице. В~качестве 
конца периода для всех пар использована дата 15~апреля 2022~г., 23:54:00. Таким 
образом, для большинства валютных пар доступный интервал составляет около 10~лет. %\linebreak\vspace*{-12pt}



  
%\begin{table*}\small
   \begin{center}
   \vspace*{-9pt}
   
{\small \begin{tabular}{|c|c|c|c|}
   \multicolumn{2}{p{63mm}}{Анализируемые валютные пары и~даты начала периода для данных}\\
   \multicolumn{2}{c}{\ }\\[-6pt]
   \hline
Валютная пара&Начало периода\\
\hline
AUDCAD&2011-12-19 00:00:00\\ 
AUDCHF&2011-12-19 00:00:00\\ 
AUDJPY&2011-12-19 21:00:00\\ 
AUDNZD&2011-12-20 21:00:00\\ 
AUDUSD&2011-12-19 21:00:00\\ 
CADCHF&2011-12-21 21:00:00\\ 
CADJPY&2013-03-19 00:00:00\\ 
CHFJPY&2011-12-26 21:00:00\\ 
GBPAUD&2013-03-19 00:00:00\\ 
GBPCHF&2012-01-02 00:00:00\\ 
GBPJPY&2012-01-09 00:00:00\\ 
GBPNZD&2014-07-30 13:47:00\\ 
GBPUSD&2011-12-19 21:00:00\\ 
EURAUD&2012-01-13 21:00:00\\
EURCAD&2011-12-28 21:00:00\\  
EURCHF&2011-12-29 21:00:00\\  
EURGBP&2011-12-28 21:00:00\\  
EURJPY&2011-12-29 21:00:00\\  
EURNZD&2012-01-12 21:00:00\\  
EURUSD&2011-12-19 21:00:00\\  
NZDCAD&2012-01-17 20:58:00\\  
NZDCHF&2012-01-17 20:58:00\\  
NZDJPY&2012-01-17 20:58:00\\  
NZDUSD&2011-12-26 21:00:00\\  
USDCAD&2011-12-28 21:00:00\\  
USDCHF&2011-12-26 21:00:00\\  
USDCNH&2015-11-17 07:38:00\\
\hline
  \end{tabular}
  }
\end{center}

\vspace*{3pt}

%\end{table*}
   
   
  %\noindent
   

В~связи со значительным объемом занимаемой 
памяти параллельно с~выгрузкой данные конвертировались в~датафрейм, в~котором одно наблюдение 
соответствовало свече на минутном таймфрейме (торговом периоде~--- интервале времени, используемом для 
группировки котировок). Данные содержат информацию о цене открытия и~закрытия свечи, ее максимуме 
и~минимуме, а~также чис\-ло тиковых колебаний цен Ask (покупка) и~Bid (продажа) (рис.~1).
  
  Метод обработки свечей состоит в~том, чтобы из количественных данных цен открытия, 
закрытия, максимума и~минимума образовать одномерный массив категориальных объектов. 
Такое упрощение структуры данных сокращает объем выборки, ускоряя обучение, 
и~позволяет нейросети находить свечные паттерны, выявляя последовательности в~них, 
используемые в~дальнейшем для формирования сигналов в~рамках торговой стратегии.

 В зависимости от длины тела свечи и~ее хвостов были выделены по десять подклассов для 
интервалов~1 и~2 (рис.~2,\,\textit{а}). Сочетание двух подклассов\linebreak\vspace*{-12pt}

\pagebreak 

\end{multicols}

\begin{figure*} %fig1
   \vspace*{1pt}
  \begin{center}  
    \mbox{%
\epsfxsize=136.465mm
\epsfbox{gor-1.eps}
}

\end{center}
\vspace*{-9pt}
   \Caption{Цены открытия, закрытия, минимума, максимума, тиковые объемы Ask и~Bid}
%   \end{figure*}
 %  \begin{figure*} %fig2
\vspace*{12pt}
  \begin{center}  
    \mbox{%
\epsfxsize=161.084mm
\epsfbox{gor-2.eps}
}

\end{center}
\vspace*{-12pt}
\Caption{Схема преобразования числовых данных в~категориальные: (\textit{а})~объекты для 
определения класса свечи; (\textit{б})~преобразование в~шаблоны из классов свечей и~зависимой 
переменной направления рынка}
\end{figure*}

\begin{multicols}{2}
  
   
 
        %
\begin{figure*} %fig3
  \vspace*{1pt}
  \begin{center}  
    \mbox{%
\epsfxsize=163mm
\epsfbox{gor-3.eps}
}

\end{center}
\vspace*{-12pt}
  \Caption{Схема разбиения данных на обучающую, тестовую и~валидационную выборки}
   \end{figure*}
%
  
  \noindent
   определяло класс 
свечи. Все свечи на пятиминутном таймфрейме были переведены в~категориальные объекты 
  (рис.~2,\,\textit{б}). Задача прогнозирования
   сводится к~определению направления следующего 
категориального объекта в~последовательности, или, другими словами, к~определению 
семантики, т.\,е.\ к~бинарной классификации направления следующей свечи. Для этого 
длинная цепочка категориальных данных была разбита на блоки, состоящие из~50~объектов. 
Каждому блоку соответствовало значение зависимой переменной, характеризующей 
семантику~--- направление рынка на следующей свече. Стоит отметить, что направление 
рынка определялось на более старшем таймфрейме, а~именно на часовом, в~то время как 
объекты блока~--- на пятиминутном. 
  
  Поскольку с~ростом обучающей выборки сеть <<знакомится>> 
  с~б$\acute{\mbox{о}}$льшим числом паттернов, то предпочтительнее обучать сеть на как 
можно большем временн$\acute{\mbox{о}}$м периоде. Одновременно с~этим, чтобы можно 
было судить о состоятельности предложенного метода обучения, тестирование также 
должно проводиться на достаточно продолжительном временн$\acute{\mbox{о}}$м отрезке. 
Учитывая эти два фактора, было решено предварительно обучить сеть на данных, 
соответствующих двум годам, а~затем до\-обучать модель на новых данных с~периодичностью 
в~два с~половиной месяца. Схема разбиения данных на обучающие, валидационные 
и~тестовые выборки изображена на рис.~3. 
  
 
   
  Для каждой валютной пары было сделано около 30~разбиений: при первом 
разбиении из 15~тыс.\  наблюдений была создана валидационная и~тестовая 
выборка, а из следующих 1500~наблюдений~--- тестовая; при втором~--- тестовая 
выборка предыдущего разбиения добавлялась к~обучающей и~валидационной, а~следующие 
1500~наблюдений составляли новую тестовую выборку и~т.\,д.

\vspace*{-6pt}
  
\section{Торговая стратегия для~свечных~данных с~использованием~LSTM-сети}

\vspace*{-3pt}

  Суть предлагаемой стратегии заключается в~следовании сигналам нейронной сети 
(фактически, прогнозам) о по\-куп\-ке/про\-да\-же на основе свечных паттернов. Удержание 
позиции происходит до генерации следующего сигнала через один модельный торговый час. 
В~ходе исследования был проведен ряд экспериментов для определения оптимальной 
конфигурации сети, в~результате которых была выбрана структура, схематично 
представленная на рис.~4 (слева внутри графика).
  
   \begin{figure*} %fig4
   \vspace*{1pt}
  \begin{center}  
    \mbox{%
\epsfxsize=162.4mm
\epsfbox{gor-4.eps}
}

\end{center}
\vspace*{-6pt}
   \Caption{Динамика баланса в~ходе тестирования LSTM-стратегии на тестовом периоде для 27 
валютных пар и~архитектура используемой нейронной сети (на графике слева внутри)}
   \end{figure*}
   
  Слой embedding~--- векторное представление категориальных объектов в~одном 
наблюдении. В~его\linebreak основе лежит векторизация каждой категории, всего~100, так как 
класс свечи определяется де\-сятью\linebreak вариантами интервала~1 и~де\-сятью вариантами 
интервала~2. Веса инициализируются случайным образом, а~затем они корректируются 
с~по\-мощью алгоритма обратного распространения ошибки.

 Слой dropout с~коэффициентом~0,5 
традиционно используется для предотвращения пе\-ре\-обуче\-ния сети. LSTM-слой 
в~архитектуре позволяет сохранять и~передавать информацию от одного шага сети 
к~другому, учитывая последовательность, в~какой данные подаются, что является ключевым 
в~распознавании паттернов. 
Слой dense~--- стандартный полносвязный. 

В~качестве 
активационной функции выходного слоя используется softmax~--- обобщение логистической 
функции для многомерного случая. В~качестве оптимизатора используется Adam, в~качестве 
функции потерь~--- бинарная кросс-энт\-ро\-пия, точ\-ность оценивается в~смысле качества 
распознавания объектов. 

При обуче\-нии экспериментальным путем были выбраны 
сле\-ду\-ющие значения основных па\-ра\-мет\-ров обуче\-ния: общее чис\-ло тренировочных объектов, 
пред\-став\-лен\-ных в~одном пакете,~--- 50; общее чис\-ло эпох обуче\-ния~--- 300. При этом, если 
значение функции потерь не уменьшалось в~течение~7~эпох, лучшая модель сохранялась 
и~обуче\-ние прекращалось. Для программной реализации был выбран язык Python, в~качестве 
основной платформы для обучения нейронных сетей~--- TensorFlow с~программным 
интерфейсом Keras. Эффективность стратегий оценивалась до\-ход\-ностью со\-от\-вет\-ст\-ву\-ющих 
финансовых инструментов.
  
  Результаты тестирования для 27 валютных пар представляют собой динамику изменения 
баланса в~процентах (см.\ рис.~4), а~также итоговую и~годовую доходность в~процентах, 
которая будет обсуждаться ниже. Период тестирования для 23 валютных пар составляет 8 
лет, для четырех он сокращен до 6--7~лет в~силу доступности меньшего объема данных. 
Визуализация динамики баланса на рис.~4 демонстрирует отсутствие существенных 
просадок ниже уровня 100\% на протяжении всего периода, что свидетельствует о 
стабильности предложенной стратегии. Наибольшая доходность у валютной пары NZDCAD 
$- 1570\%$ (41,2\% годовых), а~наименьшая~--- у~AUDJPY: $-0{,}57\%$ ($-0{,}08\%$ годовых). 
Средняя годовая доходность по всем валютным парам со\-став\-ля\-ет 15,4\%. Сравнение с~другими стратегиями будет приведено в~разд.~5 и~6. 

\begin{figure*}[b] %fig5
\vspace*{6pt}
  \begin{center}  
    \mbox{%
\epsfxsize=140.628mm
\epsfbox{gor-5.eps}
}

\end{center}
\vspace*{-6pt}
\Caption{Визуализация сигналов стратегии на основе индикатора тиковых объемов (данные 122 
сделок)}
\end{figure*}

\vspace*{-6pt}
  
\section{Индикатор тиковых объемов}

\vspace*{-2pt}

  Рассмотренная в~предыдущем разделе стратегия, основанная на нейронных сетях, не 
относится к~традиционному техническому анализу, хотя в~ее основе лежит распознавание 
последовательностей свечей. В~данном разделе сравним результаты LSTM-стра\-те\-гии 
и~подхода на основе технического индикатора. Можно отметить, что само по себе значение 
индикатора не является сигналом к~покупке или продаже. Пороговые значения, после 
которых цену можно считать подходящей для совершения сделки, определяются 
трейдерами, а значит, сигналы одного и~того же индикатора можно интерпретировать по-раз\-но\-му. 
При построении торговой стратегии помимо графических паттернов цены важ\-ную 
роль играют уровни поддержки и~сопротивления. Многие трейдеры строят свои сис\-те\-мы 
исключительно на принципах использования ценовых уровней, зон поддержки 
и~сопротивления (в~них сосредоточены заявки крупных участников рынка на покупку 
и~продажу), не уделяя внимания паттернам в~чистом виде~[14].
  
  Предлагаемый в~данной статье индикатор связан с~поиском и~обнаружением ликвидности. 
Он построен на анализе поступающих на биржу заявок с~целью выявления крупных ордеров 
(заявок) или агентов финансового рынка, которые выставляют большие объемы активов на 
покупку или продажу. Всеобщие данные объемов недоступны, но по характеру прошлых 
изменений цен Ask и~Bid можно установить, какие ценовые уровни заставляли цену менять 
свое направление и~хотя бы на время останавливали тренд. Индикатор создан для выявления 
повышенных уровней объема у свечей. Стоит обратить внимание, что если объем Ask 
и~объем Bid одновременно велики, то это значит, что силы покупателей и~продавцов не 
превалируют друг над другом, а значит, смена тренда маловероятна. По этой причине 
в~индикаторе используются не абсолютные значения ценовых объемов, а их разница~--- 
именно повышение разницы объемов Bid и~Ask в~данном индикаторе служит сигналом о 
смене тренда.  Формула индикатора тиковых объемов Bid и~Ask имеет следующий вид: 
  \begin{multline*}
  \mathrm{fracInd}_t= \fr{\mathrm{CountAsk}_t - \mathrm{CountBid}_t}{\mathrm{CountAsk}_t+\mathrm{CountBid}_t} \times{}\\
  \times \fr{(1- (\vert \mathrm{Close}_t-
\mathrm{Open}_t\vert )/(\mathrm{High}_t -\mathrm{Low}_t))}{\vert \mathrm{Close}_t -\mathrm{Open}_t\vert}\,,
  \end{multline*}
где 
\begin{description}
\item[\,] $\mathrm{fracInd}_t$~--- значение индикатора в~момент времени~$t$; 
\item[\,] $\mathrm{CountAsk}_t$~--- объем тиков Ask в~момент времени~$t$; 
\item[\,] $\mathrm{CountBid}_t$~--- объем тиков Bid в~момент времени~$t$; 
\item[\,] $\mathrm{Close}_t$~--- цена закрытия свечи в~момент времени~$t$; 
\item[\,] $\mathrm{Open}_t$~--- цена открытия свечи в~момент времени~$t$; 
\item[\,] $\mathrm{High}_t$~--- максимальное значение цены в~момент времени~$t$; 
$\mathrm{Low}_t$~--- минимальное значение цены в~момент времени~$t$.
\end{description}
 Чем больше разница тиков 
Ask и~Bid, чем больше хвосты у свечи и~чем меньше тело свечи, тем выше значение 
индикатора $\mathrm{fracInd}_t$. Для генерации сигналов $\mathrm{fracInd}_t$ был нормализован в~соответствии с~формулой:
$$
\mathrm{stoch}_t= \fr{\mathrm{fracInd}_t- \min(\mathrm{fracInd}_n)}{\max (\mathrm{fracInd}_k) -\min(\mathrm{fracInd}_n)}\,\cdot 100\,,
$$
где 
\begin{description}
\item[\,] $\mathrm{stoch}_t$~--- значение нормализованного индикатора в~момент времени~$t$; 
\item[\,] $\min(\mathrm{fracInd}_n)$~--- минимальное значение $\mathrm{fracInd}_t$ за $n$ периодов; 
 \item[\,] $\max(\mathrm{fracInd}_k)$~--- максимальное значение $\mathrm{fracInd}_t$ за $k$ периодов.
 \end{description}
  В~зависимости от 
изменения параметров~$n$ и~$k$ интерпретация индикатора может меняться. Чем выше их 
значения, тем большее окно ценовых значений рассматривается для выявления зон с~\mbox{повышенными} объемами ликвидности. Таким образом, для старших таймфреймов 
предпочтительнее использовать более высокие значения, чем для низких.

  Для тестирования была создана простейшая торговая стратегия: при пиковых значениях 
индикатора $\mathrm{stoch}_t$ в~случае восходящего тренда позиция\linebreak менялась на короткую, 
а~в~случае нисходящего~--- на длинную. Индикатор использовался для определения 
ценовых разворотов и~смены тренда. \mbox{С~помощью} созданной функции бэктестирования 
описанная стратегия была проверена на всех валютных парах на протяжении пяти лет 
с~15~апреля 2017~г.\ по 15~апреля 2022~г. Тестирование на более длительных периодах без 
проведения дополнительной оптимизации неэффективно, так как торговля с~индикаторами 
предусматривает периодическое обновление параметров стратегии.
  
  На рис.~5 представлен фрагмент сигналов, вырабатываемых предложенной стратегией: 
черточки поверх свечей обозначают сделки: красные~--- убыточные, зеленые~--- 
прибыльные. Легко заметить, что рекомендуемые цены открытия/закрытия позиции 
совпадают с~пиковыми значениями локальных трендов. Это означает, что сигналы 
индикатора можно использовать как точки входа и~выхода из позиции. Разработанный 
индикатор хорошо предсказывает развороты рынка даже с~учетом неизменных параметров 
на протяжении всего тестируемого периода. В среднем доходность по всем валютным парам 
за 5~лет составила 36,36\%.
  

\begin{figure*} %fig6
\vspace*{1pt}
  \begin{center}  
    \mbox{%
\epsfxsize=151.209mm
\epsfbox{gor-6.eps}
}

\end{center}
\vspace*{-6pt}
\Caption{Результаты сравнения стратегий на основе LSTM~(\textit{1}~--- за~5~лет и~\textit{4}~--- годовая), индикатора тиковых объемов 
(\textit{2}~--- за~5~лет и~\textit{5}~--- годовая) и~Buy \& 
Hold за 5~лет~(\textit{3})}
\end{figure*}

\begin{figure*} %fig7
\vspace*{1pt}
  \begin{center}  
    \mbox{%
\epsfxsize=151.509mm
\epsfbox{gor-7.eps}
}

\end{center}
\vspace*{-6pt}
\Caption{Сравнение доходностей LSTM-стратегии (\textit{1}~--- доходность за 8~лет; \textit{2}~---  
годовая доходность), стратегии на основе ансамблевой LSTM-ар\-хи\-тек\-ту\-ры (\textit{3}~--- доходность за 8~лет; 
\textit{4}~--- годовая доходность) и~базовой стратегии Buy \& Hold~(\textit{5}) за 
восьмилетний период для 27 валютных пар }
\end{figure*}
  
  Чтобы результаты стратегии с~нейронной сетью были сопоставимы с~результатами 
тестирования индикатора, нейронные сети были протестированы на том же временн$\acute{\mbox{о}}$м 
участке: они продемонстрированы на рис.~6. Кроме того, добавлено сравнение 
с~классической стратегией Buy \& Hold.
  
  Средняя годовая доходность LSTM-стра\-те\-гии~--- 18,2\%. Это несколько выше средней 
годовой доходности за восьмилетний период, приведенной в~разд.~4. Причина этого может 
быть связана с~лучшим обучением нейронной сети на более поздних тестируемых периодах. 
За тот же период времени средняя годовая доходность стратегии на основе индикатора 
значительно ниже~--- 6,1\%. При этом стоит отметить, что по валютным парам \mbox{AUDJPY}, 
AUDUSD и~USDCNH годовая доходность индикатора превосходит годовую доходность 
LSTM-стра\-те\-гии, причем для \mbox{AUDJPY} обе <<технические>> стратегии показали 
положительную годовую доходность. 


 \vspace*{-9pt}


\section{Ансамблевая архитектура}

 \vspace*{-2pt}

  Для улучшения результатов, описанных в~предыду\-щих разделах, возможно построение 
стратегии, которая учитывает сигналы, генерируемые и~нейронной сетью, и~индикатором. 




  
  Для этого архитектура базовой нейронной сети (см.\ рис.~4) была изменена на ансамбль из 
двух LSTM-се\-тей: она имеет два входных потока, на вход одного из которых подаются 
прежние величины, соответствующие LSTM-стра\-те\-гии, а на вход второго~--- значения 
индикатора, приведенные к~аналогичному виду категориальных объектов путем отнесения 
значения к~одному из десяти интервалов. Обе части нейронной сети включают в~себя слои 
embedding, dropout и~LSTM~--- эти фрагменты в~точности соответствуют LSTM-стра\-те\-гии. 
Далее выходные потоки попадают в~слой concatenate, который объединяет их и~единым 
потоком передает в~выходной полносвязный слой с~функцией активации softmax. Выходной 
слой генерирует итоговые вероятности, на основе которых создается прогноз движения 
\mbox{цены.}
  
  Модифицированная нейронная сеть была протестирована на всех~27~ранее 
рассмотренных валютных парах (рис.~7) за период в~6--8~лет (по этой причине на 
графиках не приводятся значения индикатора). Для этого искусственные нейронные сети 
были предварительно обучены на данных каждой валютной пары, а затем с~использованием 
полученных сигналов покупки и~продажи было осуществлено бэктестирование 
модифицированной стратегии на исторических данных. Обучение проходило по тому же 
принципу, что и~для исходной LSTM-стра\-те\-гии. 
  
  Можно заметить, что на 10 валютных парах стратегия с~применением ансамбля 
 LSTM-се\-тей превзошла первую LSTM-стра\-те\-гию, продемонстрировав более высокую 
доходность. Для этих валютных пар среднее значение итоговой доходности воз\-рос\-ло на 
17,19\%, а годовая доходность увеличилась на 0,409\%. При этом средняя годовая 
доходность, рассчитанная для всех валютных пар, составила 12,2\%, что ниже аналогичного 
показателя простой LSTM-стра\-те\-гии. Однако можно обратить внимание на ранее 
убыточную пару AUDJPY, для которой получена итоговая доходность 31,2\% и~годовая 
3,3\%, что существенно превосходит значения для базовой стратегии Buy \& Hold. Таким 
образом, в~целом ряде случаев (37\% тестируемых валютных пар) использование 
ансамблевого подхода позволяет значимым образом повысить доходность базовой  
LSTM-стра\-те\-гии. Это обстоятельство необходимо учитывать при разработке реальных 
автоматизированных торговых инструментов, в~рамках которых обычно не используется 
единый алгоритм (метод) сразу для всех объектов из портфеля.

 \vspace*{-9pt}

\section{Заключение}

 \vspace*{-2pt}

  В работе продемонстрирована высокая эффективность использования нейросетевых 
подходов к~построению автоматических торговых стратегий на основе валютных пар. 
Показано, что предложенные методы позволяют многократного превзойти базовые методы, 
в~том числе основанные на классическом техническом анализе.
  %
  При этом можно отметить, что многие валютные пары обладают высокой корреляцией~--- 
как положительной, так и~отрицательной (рис.~8). 
  
  
  
  Можно предположить, что для большей информативности признаков, созданных на 
основе значений индикатора, в~дальнейших исследованиях стоит использовать уникальное 
разбиение на классы для разных валютных пар, а~не унифицированное, как это реализовано 
в~данной работе. Потенциально подбор порогов для разбиения на классы может повысить 
доходность стратегии с~применением ан\-самб\-ля LSTM-се\-тей, что сделает модификацию 
LSTM-стра\-те\-гии перспективной об\-ластью дальнейшего исследования, включая 
использование композиционных и~ан\-самб\-ле\-вых подходов. 
  
  Принципиальное улучшение может быть достигнуто за счет привлечения сложных 
математических методов и~моделей, в~частности использования\linebreak  параметров семейств 
вероятностных распределений для расширения признакового пространства для обучения 
нейронных сетей, как предложено в~\mbox{статье}~[15] на примере различных моделей добавления 
дополнительных признаков~--- четырех первых моментов конечных нормальных смесей. 
Определенные успехи в~этом направлении для ряда валютных пар уже 
продемонстрированы~[16], что свидетельствует в~пользу перспективности подобных 
исследований.

\end{multicols}

\begin{figure*} %fig8
   \vspace*{1pt}
  \begin{center}  
    \mbox{%
\epsfxsize=130.012mm
\epsfbox{gor-8.eps}
}

\end{center}
\vspace*{-6pt}
   \Caption{Корреляционная матрица валютных пар}
 %  \vspace*{6pt}
   \end{figure*}
   
   \vspace*{-9pt}
   
   \begin{multicols}{2}
   


  
{\small\frenchspacing
 {%\baselineskip=10.8pt
 %\addcontentsline{toc}{section}{References}
 \begin{thebibliography}{99}
\bibitem{1-gor}
\Au{Bholat D., Susskind~D.} The assessment: Artificial intelligence and financial services~// 
Oxford Rev. Econ. Pol., 2021. Vol.~37. Iss.~3. P.~417--434.
\bibitem{2-gor}
\Au{Lin Y., Liu S., Yang~H., Wu~H.} Stock trend prediction using candlestick charting and 
ensemble machine learning techniques with a~novelty feature engineering scheme~// IEEE 
Access, 2021. Vol.~9. P.~101433--101446.
\bibitem{3-gor}
\Au{Lahmiri S., Bekiros~S.} Intelligent forecasting with machine learning trading systems in chaotic 
intraday bitcoin market~// Chaos Soliton. Fract., 2020. Vol.~133. Art. No.\,109641. 7~p.
\bibitem{4-gor}
\Au{Rundo F.} Deep LSTM with reinforcement learning layer for financial trend prediction in FX 
high frequency trading systems~// Appl. Sci.~--- Basel, 2019. Vol.~9. Iss.~20. Art. No.\,4460. 18~p.
\bibitem{5-gor}
\Au{Dinh T.-A., Kwon~Y.-K.} An empirical study on importance of modeling parameters and 
trading volume-based features in daily stock trading using neural networks~// Informatics, 
2018. Vol.~5. Iss.~3. Art. No.\,36. 12~p.
\bibitem{6-gor}
\Au{Sezer O.\,B., Ozbayoglu A.\,M.} Algorithmic financial trading with deep convolutional 
neural networks: Time series to image conversion approach~// Appl. Soft Comput., 2018. 
Vol.~70. P.~525--538.
\bibitem{7-gor}
\Au{Tsinaslanidis P., Guijarro~F.} What makes trading strategies based on chart recognition 
profitable?~// Expert Syst., 2020. Vol.~38. Iss.~5. Art. No.\,e12596. 17~p.
\bibitem{8-gor}
\Au{Chen J.\,H., Tsai Y.\,C.} Encoding candlesticks as images for pattern classification using 
convolutional neural networks~// Financial Innovation, 2020. Vol.~6. Art. No.\,26. 19~p.
\bibitem{9-gor}
\Au{Hung C.-C., Chen Y.-J.} DPP: Deep predictor for price movement from candlestick charts~// 
PLoS ONE, 2021. Vol.~16. Iss.~6. Art. No.\,e0252404. 22~p.
\bibitem{10-gor}
\Au{Wu J.\,M.-T., Li Z., Herencsar~N., Vo~B., Lin~J.\,C.-W.} A~graph-based CNN-LSTM stock 
price prediction algorithm with leading indicators~// Multimedia Syst., 2021. 20~p. doi: 
10.1007/s00530-021-00758-w.
\bibitem{11-gor}
\Au{Sezer O.\,B., Ozbayoglu A.\,M., Dogdu E.} A~deep neural-network based stock trading 
system based on evolutionary optimized technical analysis parameters~// Procedia Comput. 
Sci., 2017. Vol.~114. P.~473--480.
\bibitem{12-gor}
\Au{JuHyok U., PengYu~L., ChungSong~K., UnSok~R., KyongSok~P.} A~new LSTM based 
reversal point prediction method using upward/downward reversal point feature sets~// Chaos 
Soliton. Fract., 2020. Vol.~32. Art. No.\,109559. 15~p.
\bibitem{13-gor}
\Au{Tsantekidis A., Tefas~A.} Transferring trading strategy knowledge to deep learning 
models~// Knowl. Inf. Syst., 2021. Vol.~63. P.~87--104.
\bibitem{14-gor}
\Au{Edwards R.\,D., Magee~J., Bassetti~W.\,H.\,C.} Technical analysis of stock trends.~--- 11th 
ed.~--- Boca Raton, FL, USA: CRC Press, 2018. 685~p.
\bibitem{15-gor}
\Au{Gorshenin A.\,K., Kuzmin V.\,Yu}. Statistical feature construction for forecasting accuracy 
increase and its applications in neural network based analysis~// Mathematics, 2022. Vol.~10. 
Iss.~4. Art. No.\,589. 21~p.
\bibitem{16-gor}
\Au{Виляев А.\,Л., Горшенин~А.\,К.} О~моделировании торговых стратегий для валютных пар 
с~использованием глубоких нейронных сетей и~метода скользящего разделения смесей~// 
Интеллектуальные сис\-те\-мы. Тео\-рия и~приложения, 2021. T.~25. Вып.~4. С.~92--\linebreak 95. 

\end{thebibliography}

 }
 }

\end{multicols}

\vspace*{-10pt}

\hfill{\small\textit{Поступила в~редакцию 13.07.22}}

\vspace*{8pt}

%\pagebreak

%\newpage

%\vspace*{-28pt}

\hrule

\vspace*{2pt}

\hrule

%\vspace*{-2pt}

\def\tit{INCREASING FOREX TRADING PROFITABILITY\\ WITH~LSTM CANDLESTICK PATTERN 
RECOGNITION\\ AND~TICK VOLUME INDICATOR}


\def\titkol{Increasing FOREX trading profitability with~LSTM candlestick pattern 
recognition and~tick volume indicator}


\def\aut{A.\,K.~Gorshenin$^1$ and E.\,I.~Guseynova$^2$}

\def\autkol{A.\,K.~Gorshenin and E.\,I.~Guseynova}

\titel{\tit}{\aut}{\autkol}{\titkol}

\vspace*{-17pt}


\noindent
$^1$Federal Research Center ``Computer Science and Control'' of the Russian Academy of Sciences, 
44-2~Vavilov\linebreak
$\hphantom{^1}$Str., Moscow 119133, Russian Federation

\noindent
$^2$M.\,V.~Lomonosov Moscow State University, 1~Leninskie Gory, GSP-1, Moscow 119991, Russian 
Federation


\def\leftfootline{\small{\textbf{\thepage}
\hfill INFORMATIKA I EE PRIMENENIYA~--- INFORMATICS AND
APPLICATIONS\ \ \ 2022\ \ \ volume~16\ \ \ issue\ 3}
}%
 \def\rightfootline{\small{INFORMATIKA I EE PRIMENENIYA~---
INFORMATICS AND APPLICATIONS\ \ \ 2022\ \ \ volume~16\ \ \ issue\ 3
\hfill \textbf{\thepage}}}

\vspace*{2pt} 


\Abste{The paper introduces the research of the effectiveness of using LSTM (Long--Short Term Memory)
for candlestick data 
and a technical analysis indicator for a large number of the most common currency pairs (27 in 
total) over a long period in order to build automatic trading strategies. The achieved average total 
and annual return for 8~years of a model trading were 286\% and 15.4\%, respectively. It is more 
than~50~times higher than the values for the classic Buy \& Hold trading strategy for the same 
period. In addition, the paper introduces a new technical indicator based on tick volumes which is 
an alternative trading strategy (the total and annual returns of LSTM models exceed it by an 
average of~7.2 and 2.3~times) as well as an additional feature to increase the profitability of the 
neural network strategy through the use of ensemble learning. For 37\% of the analyzed currency 
pairs, the use of an ensemble of LSTMs allows one to increase further the total return by an average 
of 17.2\%.}

\KWE{LSTM; ensemble learning; candlestick; technical indicator; FOREX; currency pairs}




\DOI{10.14357/19922264220304} 

\vspace*{-20pt}

\Ack

\vspace*{-5pt}

\noindent
The research was supported by the Russian Science Foundation (grant No.\,22-11-00212). The
 research was carried out using the infrastructure of the Shared Research Facilities ``High 
Performance Computing and Big Data'' (CKP ``Informatics'') of FRC CSC RAS (Moscow). 




\vspace*{6pt}

  \begin{multicols}{2}

\renewcommand{\bibname}{\protect\rmfamily References}
%\renewcommand{\bibname}{\large\protect\rm References}

{\small\frenchspacing
 {%\baselineskip=10.8pt
 \addcontentsline{toc}{section}{References}
 
 \begin{thebibliography}{99}
 
 \vspace*{-3pt}
 
\bibitem{1-gor-1}
\Aue{Bholat, D., and D.~Susskind.} 2021. The assessment: Artificial intelligence and financial 
services. \textit{Oxford Rev. Econ. Pol.} 37(3):417--434.
\bibitem{2-gor-1}
\Aue{Lin, Y., S.~Liu, H.~Yang, and H.~Wu.} 2021. Stock trend prediction using candlestick charting 
and ensemble machine learning techniques with a novelty feature engineering scheme. \textit{IEEE Access} 
9:101433--101446.
\bibitem{3-gor-1}
\Aue{Lahmiri, S., and S.~Bekiros.} 2020. Intelligent forecasting with machine learning trading 
systems in chaotic intraday Bitcoin market. \textit{Chaos Soliton. Fract.} 133:109641.\linebreak 7~p.
\bibitem{4-gor-1}
\Aue{Rundo, F.} 2019. Deep LSTM with reinforcement learning layer for financial trend 
prediction in FX high frequency trading systems. \textit{Appl. Sci.~--- Basel} 9(20):4460. 18~p.
\bibitem{5-gor-1}
\Aue{Dinh, T.-A., and Y.-K.~Kwon.} 2018. An empirical study on importance of modeling 
parameters and trading volume-based features in daily stock trading using neural networks. 
\textit{Informatics} 5(3):36. 12~p.
\bibitem{6-gor-1}
\Aue{Sezer, O.\,B., and A.\,M.~Ozbayoglu.} 2018. Algorithmic financial trading with deep 
convolutional neural networks: Time series to image conversion approach. \textit{Appl. Soft Comput.} 
70:525--538.
\bibitem{7-gor-1}
\Aue{Tsinaslanidis, P., and F.~Guijarro.} 2020. What makes trading strategies based on chart 
recognition profitable? \textit{Expert Syst.} 38(5):e12596. 17~p.
\bibitem{8-gor-1}
\Aue{Chen, J.\,H., and Y.\,C.~Tsai.} 2020. Encoding candlesticks as images for pattern classification 
using convolutional neural networks. \textit{Financial Innovation} 6:26. 19~p.
\bibitem{9-gor-1}
\Aue{Hung, C.-C., and Y.-J.~Chen.} 2021. DPP: Deep predictor for price movement from 
candlestick charts. \textit{PLoS ONE} 16(6):e0252404. 22~p.
\bibitem{10-gor-1}
\Aue{Wu, J.\,M.-T., Z.~Li, N.~Herencsar, B.~Vo, and J.\,C.-W.~Lin.} 2021. A~graph-based CNN-LSTM 
stock price prediction algorithm with leading indicators. \textit{Multimedia Syst}. 20~p.
doi: 10.1007/s00530-021-00758-w.
\bibitem{11-gor-1}
\Aue{Sezer, O.\,B., A.\,M.~Ozbayoglu, and E.~Dogdu.} 2017. A~deep neural-network based stock 
trading system based on evolutionary optimized technical analysis parameters. \textit{Procedia Comput. 
Sci.} 114:473--480.
\bibitem{12-gor-1}
\Aue{JuHyok, U., L.~PengYu, K.~ChungSong, R.~UnSok, and P.~KyongSok.} 2020. A~new LSTM 
based reversal point prediction method using upward/downward reversal point feature sets. \textit{Chaos 
Soliton. Fract.} 32:109559. 15~p.
\bibitem{13-gor-1}
\Aue{Tsantekidis, A., and A.~Tefas.} 2021. Transferring trading strategy knowledge to deep 
learning models. \textit{Knowl. Inf. Syst.} 63:87--104.
\bibitem{14-gor-1}
\Aue{Edwards, R.\,D., J.~Magee, and W.\,H.\,C.~Bassetti.} 2018. \textit{Technical analysis of stock trends}. 
11th ed. Boca Raton, FL: CRC Press. 685~p.
\bibitem{15-gor-1}
\Aue{Gorshenin, A.\,K., and V.\,Yu.~Kuzmin.} 2022. Statistical feature construction for forecasting 
accuracy increase and its applications in neural network based analysis. \textit{Mathematics} 10(4):589. 21~p.
\bibitem{16-gor-1}
\Aue{Vilyaev, A.\,L., and A.\,K.~Gorshenin.} 2021. O modelirovanii torgovykh strategiy dlya 
valyutnykh par s ispol'zovaniem glubokikh neyronnykh setey i~metoda skol'zyashchego razdeleniya 
smesey [On modeling trading strategies for currency pairs using deep neural networks and method 
of moving separation of mixtures]. \textit{Intellektual'nye sistemy. Teoriya i~prilozheniya} [Intelligent 
Systems. Theory and Applications] 25(4):92--95.
\end{thebibliography}

 }
 }

\end{multicols}

\vspace*{-6pt}

\hfill{\small\textit{Received July 13, 2022}}

\Contr

\noindent
\textbf{Gorshenin Andrey K.} (b.\ 1986) ~--- Doctor of Science in physics and mathematics, 
associate professor, head of department, leading scientist, Federal Research Center ``Computer 
Science and Control'' of the Russian Academy of Sciences, 44-2~Vavilov Str., Moscow 119333, 
Russian Federation; \mbox{agorshenin@frccsc.ru}

\vspace*{3pt}

\noindent
\textbf{Guseynova Ekaterina I.} (b.\ 1999)~--- Master of Science, Faculty of Economics, M.\,V.~Lomonosov 
Moscow State University, 1~Leninskie Gory, GSP-1, Moscow 119991, Russian Federation; 
\mbox{ei.guseynova@yandex.ru}

\label{end\stat}

\renewcommand{\bibname}{\protect\rm Литература}    