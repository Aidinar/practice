\def\stat{aliu}

\def\tit{ГИСТЕРЕЗИСНОЕ УПРАВЛЕНИЕ НАГРУЗКОЙ В~БЕСПРОВОДНЫХ СЕНСОРНЫХ 
СЕТЯХ$^*$}

\def\titkol{Гистерезисное управление нагрузкой в~беспроводных сенсорных 
сетях}

\def\aut{Б.~Алию$^1$, Е.\,А.~Мачнев$^2$, Е.\,В.~Мокров$^3$}

\def\autkol{Б.~Алию, Е.\,А.~Мачнев, Е.\,В.~Мокров}

\titel{\tit}{\aut}{\autkol}{\titkol}

\index{Алию Б.}
\index{Мачнев Е.\,А.}
\index{Мокров Е.\,В.}
\index{Aliyu B.}
\index{Machnev E.\,A.}
\index{Mokrov E.\,V.}


{\renewcommand{\thefootnote}{\fnsymbol{footnote}} \footnotetext[1]
{Исследование выполнено за счет 
гранта Российского научного фонда №\,21-79-00157, 
{\sf https://rscf.ru/project/21-79-00157} (получатель Е.\,В.~Мокров, системная и~математическая модели), 
и~при поддержке Программы стратегического академического лидерства РУДН (получатель 
Е.\,А.~Мач\-нев, введение).}}


\renewcommand{\thefootnote}{\arabic{footnote}}
\footnotetext[1]{Российский университет дружбы народов, bashaliyuu@gmail.com}
\footnotetext[2]{Российский университет дружбы народов, 1032143100@rudn.ru}
\footnotetext[3]{Российский университет дружбы народов, mokrov-ev@rudn.ru}


\vspace*{-12pt}
  
  \Abst{Решается задача анализа перегрузок в беспроводной сенсорной сети. Предложен 
механизм гистерезисного управления, отсеивающий часть нагрузки в~случае перегрузки 
системы для стабилизации ее функционирования. Построена математическая модель 
в~виде марковского процесса с~конечным пространством со\-сто\-яний, и~получена формула для 
анализа показателей эффективности беспроводной сенсорной сети. Численно исследованы  
ве\-ро\-ят\-ност\-но-вре\-мен\-ные характеристики гистерезисного управ\-ле\-ния~--- 
вероятность сброса пакета и~средняя длина очереди. Сравнение гистерезисного механизма 
и~алгоритма IRED (improved random early detection~--- улучшенное 
случайное раннее обнаружение)
при варьировании размера окна управ\-ле\-ния показало, что 
гистерезисный механизм дает преимущество в~об\-ласти больших нагрузок по сравнению с~механизмом IRED.}
  
  \KW{беспроводные сенсорные сети; управление нагрузкой; гистерезисное управление; 
марковский процесс; система массового обслуживания}

\DOI{10.14357/19922264220311}
  
\vspace*{-3pt}


\vskip 10pt plus 9pt minus 6pt

\thispagestyle{headings}

\begin{multicols}{2}

\label{st\stat}
  
\section{Введение}

  Автономный характер беспроводных сенсорных сетей  в сочетании с их способностью 
воспринимать, обрабатывать и передавать данные об окружающей среде на базовую 
станцию сотовой сети~[1] способствует их широкому распространению. Согласно 
данным IoT (Internet of Things) Analytics~[2], более 14,4~млрд сенсоров взаимодействуют 
с~использованием беспроводных сенсорных сетей. Такое число устройств IoT может 
привести к перегрузкам в сети (см., например,~[3]), особенно с ростом числа устройств, 
использующих технологию беспроводных сенсорных сетей.
  
  Согласно Cisco Systems~[4] к 2023~г.\ прогнозируется порядка 14,7~млрд 
активных IoT-со\-еди\-не\-ний, включая датчики, носимые устройства, смартфоны, 
беспилотные автомобили и~беспилотные летательные аппараты (БПЛА). В~отсутствие 
управ\-ле\-ния трафиком в~беспроводных сенсорных сетях возможны частые 
и~продолжительные перегрузки. Эта особенность исследуется в~\mbox{статье} с~по\-мощью 
сравнительного анализа про\-из\-во\-ди\-тель\-ности сети на основе гистерезисной схемы 
управления нагрузкой~\cite{6-al, 7-al} и~улучшенной схемы управ\-ле\-ния перегрузкой со 
случайным ранним обнаружением (IRED)~\cite{5-al}. Именно перегрузки в~беспроводной 
сенсорной сети и~стали предметом исследования настоящей работы.
  
  В статье предложена системная модель для наиболее типичной архитектуры 
беспроводной сенсорной сети, в~марковских предположениях по\-стро\-ена математическая 
модель, проведен ее анализ и~предложены формулы для вычисления ее характеристик. 
Численный анализ направлен на сравнение алгоритма гистерезисного управ\-ле\-ния 
нагрузкой с~алгоритмом управ\-ле\-ния IRED~\cite{5-al}. Показано, что в~об\-ласти высокой 
нагрузки гистерезисное управ\-ле\-ние дает преимущество по вероятности сброса, т.\,е.\ 
потере пакета при поступлении, и~средней длине очереди пакетов на шлюзе.

\vspace*{-6pt}
  
\section{Системная модель}

\vspace*{-3pt}

  Рассматривается сценарий, в котором датчики, расположенные в пределах некоторой 
заданной области, передают информацию на центральный узел, отвечающий за ее 
хранение и обработку. Для сбора данных с датчиков используются шлюзы, размещенные 
на БПЛА, патрулирующих территорию аналогично одному из случаев, описанных 
в~\cite{7-al}, и~более подробно рассмотренному в~\cite{8-al}.
  



  Системная модель представляет собой беспроводную сенсорную сеть, в которой 
кластер беспроводных датчиков загружает полученные данные об окружающей среде 
через шлюз в базу данных (БД), размещенную в облаке. Архитектура сети, как по-\linebreak\vspace*{-12pt}

\pagebreak

\end{multicols}

\begin{figure*} %fig1
\vspace*{1pt}
  \begin{center}  
    \mbox{%
\epsfxsize=163mm
\epsfbox{ali-1.eps}
}

\end{center}
\vspace*{-6pt}
\Caption{Схема системной модели беспроводной сенсорной сети}
\end{figure*}

\begin{multicols}{2}

\noindent
 казано 
на рис.~1, включает три уровня: уровень физического восприятия, где датчики собирают 
данные об окружающей среде, например температуру, влажность, освещенность; 
виртуальный облачный уровень, где размещены БД, в которые передается на хранение 
собранная датчиками информация; уровень пользователей, на котором расположены\linebreak  
веб-ин\-тер\-фей\-сы для взаимодействия с~БД. Передача данных между первым и~вторым 
уровнями происходит через шлюз, на котором размещена сис\-те\-ма управ\-ле\-ния нагрузкой, 
осу\-ще\-ст\-в\-ля\-ющая \mbox{мониторинг} и~снижение нагрузки для пред\-от\-вра\-ще\-ния перегрузок. 
  
  В статье рассматривается гистерезисная схема управления нагрузкой с двумя 
порогами~\cite{6-al} для пред\-от\-вра\-ще\-ния перегрузок шлюза, результатом которых может 
стать переполнение памяти \mbox{шлюза}.

\vspace*{-9pt}
  
\section{Математическая модель}

\vspace*{-3pt}

\begin{figure*} %fig2
\vspace*{1pt}
  \begin{center}  
    \mbox{%
\epsfxsize=159.108mm
\epsfbox{ali-2.eps}
}

\end{center}
\vspace*{-6pt}
\Caption{Диаграмма интенсивностей переходов МП $X(t)$ для системы с гистерезисным 
управлением}
\end{figure*}

  Описанная сеть моделируется системой массового обслуживания типа $M/M/1/B$ 
с~гистерезисным управ\-ле\-ни\-ем нагрузкой~\cite{9-al}. На систему\linebreak с~одним прибором, 
накопителем емкости~$B$ и экспоненциальным обслуживанием с~па\-ра\-мет\-ром~$\mu$ 
поступает пуассоновский поток заявок, соответствующих пакетам данных от сенсоров, 
с~ин\-тен\-сив\-ностью~$\lambda$. Считаем, что заявки сохраняют за собой место в~накопителе 
до момента окончания обслуживания. 

Для контроля перегрузок в~сис\-те\-ме используются 
два порога: нижний порог~$L$ и~верхний порог~$H$. Система работает в трех режимах: 
режиме нормальной работы ($s\hm=0$), режиме сниженной нагрузки ($s\hm=1$) 
и~режиме блокировки ($s\hm=2$). Когда чис\-ло заявок в~сис\-те\-ме превышает верхний 
порог~$H$, сис\-те\-ма переходит в~режим сниженной нагрузки ($s\hm=1$). Принятая 
в~сис\-те\-му нагрузка в этом режиме снижается до уровня $\lambda^*\hm= (1\hm- p)\lambda$, 
при этом часть нагрузки $p\lambda$ сбрасывается, таким образом $p$~--- это доля 
сбрасываемой нагрузки при нахождении сис\-те\-мы в~режиме сниженной нагрузки. Сис\-те\-ма 
продолжает находиться в~режиме сниженной нагрузки либо до заполнения сис\-те\-мы, либо 
до достижения чис\-лом заявок в~очереди нижнего порога~$L$. В~первом случае сис\-те\-ма 
перейдет в~режим блокировки ($s\hm=2$) и~перестанет принимать новые, но продолжит 
обслуживать уже на\-хо\-дя\-щи\-еся в~сис\-те\-ме заявки. В~таком состоянии сис\-те\-ма будет 
пребывать до тех пор, пока чис\-ло заявок не упадет до верх\-не\-го порога, после чего 
возвратится в~режим сниженной нагрузки ($s\hm=1$). Во втором случае нагрузка на 
систему восстановится до исходной, и~сис\-те\-ма перейдет в~режим нормальной работы 
($s\hm=0$).
{\looseness=1

}
  
  Функционирование системы описывается марковским процессом (МП) $X(t)$ над 
пространством\linebreak состояний $\mathcal{X}\hm= \{ (n,s): s\hm\in \{0,1,2\},\ n\hm\in \{0,\ldots\linebreak
\ldots , B\}\}$, которое представимо в виде раз\-биения 
$$
\mathcal{X}= \mathcal{X}_0\cup 
\mathcal{X}_1 \cup \mathcal{X}_2\,,
$$
 где $\mathcal{X}_0\hm= \{\{ 0,n): n\hm=0,\ldots , 
H\hm-1\}$~--- множество состояний нормальной работы; $\mathcal{X}_1\hm= \{ (1,n): 
n\hm= L,\ldots , B\hm-1\}$~--- множество состояний сниженной нагрузки; 
$\mathcal{X}_2\hm= \{ (2,n): n\hm= H\hm+1,\ldots , B\}$~--- множество состояний 
блокировки. На рис.~2 проиллюстрирована диаграмма интенсивностей переходов 
состояний МП $X(t)$.
  
  Решив для МП $X(t)$ систему уравнений равновесия, получим стационарное 
распределение $p_{sn}$, $(s,n)\hm\in \mathcal{X}$, с~помощью которого в явном виде 
могут быть представлены следующие ве\-ро\-ят\-но\-ст\-но-вре\-мен\-н$\acute{\mbox{ы}}$е характеристики: 
$B(\mathcal{X}_0)\hm= \sum\nolimits_{n=0}^{H-1} p_{0,n}$~---  вероятность того, что 
система работает в режиме нормальной нагрузки; $B(\mathcal{X}_1)\hm= 
\sum\nolimits_{n=L}^{B-1} p_{1,n}$~--- вероятность того, что сис\-те\-ма работает в режиме 
сниженной нагрузки и часть заявок будет сброшена; $B(\mathcal{X}_2)\hm= 
\sum\nolimits_{n=N+1}^B p_{2,n}$~--- вероятность того, что сис\-те\-ма перегружена и не 
принимает новые заявки; $B\hm= (1\hm-p) B(\mathcal{X}_1) \hm+ B(\mathcal{X}_2)$~--- 
вероятность сброса заявки; $MQ\hm= \sum\nolimits_{(s,n)\in \mathcal{X}} np_{s,n}$~--- 
средняя длина очереди.



  Аналогичные характеристики для алгоритма IRED вычисляются согласно~\cite{5-al}, 
что позволяет провести сравнение двух механизмов защиты от перегрузок.

\vspace*{-6pt}
  
\section{Численный анализ}

\vspace*{-3pt}

  Было проведено численное сравнение вероятностей сброса пакета для обоих 
рассматриваемых алгоритмов в зависимости от параметров $B$, $L$, $H$, $p$ и~$\rho\hm=\lambda/\mu$ при 
значениях исходных данных из~\cite{5-al}. В~результате анализа было показано, что 
алгоритм гистерезисного управ\-ле\-ния более эффективен по сравнению с~алгоритмом IRED 
при больших значениях нагрузки. Заметим, что при этом с~точ\-ки зрения средней длины 
очереди алгоритм гистерезисного управ\-ле\-ния уступает алгоритму IRED. Данные 
результаты пред\-став\-ле\-ны на рис.~3 при фиксированном значении объема накопителя 
$B\hm=50$ и~нижнего порога $L\hm=10$ и~варьировании остальных па\-ра\-мет\-ров 
в~пределах $0{,}1\hm\leq p\hm < 1$ и~$12\hm\leq H\hm\leq 40$. График показывает 
минимальное пороговое значение нагрузки~$\rho_0$, после которого гистерезисный подход 
позволяет успешно передать больше пакетов, чем алгоритм IRED, т.\,е.\ $\rho_0:\ 
\forall\,\rho \hm> \rho_0$ $B_{\mathrm{hyst}}(\rho)\hm< B_{\mathrm{IRED}}(\rho)$. Из рис.~3 можно заключить, 
что пороговое значение нагрузки~$\rho_0$ почти линейно зависит от доли~$p$ 
сбрасываемой нагрузки в~со\-сто\-янии сниженной на\-груз\-ки и~слабо зависит от изменения 
верхнего порога при фиксированном нижнем. Аналогичная картина наблюдается и~при 
других значениях нижнего порога.
  
   На рис.~4 отдельно представлены результаты сравнения производительности сис\-те\-мы 
с гистерезисным управ\-ле\-ни\-ем и~управ\-ле\-ни\-ем с~по\-мощью алгоритма IRED для объема 
накопителя $B\hm=40$,\linebreak\vspace*{-12pt}

\pagebreak

\end{multicols}

\begin{figure*} %fig3
\vspace*{1pt}
  \begin{center}  
    \mbox{%
\epsfxsize=89.36mm
\epsfbox{ali-3.eps}
}

\end{center}
\vspace*{-6pt}
\Caption{Пороговое значение нагрузки}
\vspace*{9pt}
\end{figure*}

\begin{figure*} %fig4
\vspace*{1pt}
  \begin{center}  
    \mbox{%
\epsfxsize=160.278mm
\epsfbox{ali-4.eps}
}

\end{center}
\vspace*{-6pt}
\Caption{Средняя длина очереди~(\textit{а}) и вероятность сброса пакета~(\textit{б}):  
\textit{1}~--- гистерезисное управление; \textit{2}~--- управление с~по\-мощью алгоритма IRED}
\end{figure*}


\begin{multicols}{2}

\noindent
 доли сбрасываемых пакетов $p\hm= 0{,}3$, нижнего и~верхнего 
порогов $L\hm= 10$ и~$H\hm= 30$ со\-от\-вет\-ст\-венно. 

Средняя длина очереди 
в~за\-ви\-си\-мости от предложенной нагрузки, варьирующейся в~пределах  $0\hm<\rho \hm< 2$, 
показана на рис.~4,\,\textit{а}. 
%
На рис.~4,\,\textit{б} в~логарифмической шкале показана 
вероятность сброса пакетов в~за\-ви\-си\-мости от предложенной на\-грузки. 
{\looseness=1

}

Из приведенных 
результатов видно, что пороговое значение нагрузки также существенно зависит от 
объема накопителя, поскольку в~данном случае алгоритм IRED показывает более 
хорошие результаты при низкой нагрузке, а~также при небольшой перегрузке $\rho\hm\leq 
1{,}2$. Однако уже начиная со значения нагрузки $\rho \hm> 1{,}2$ лучшие 
характеристики показывает гистерезисное управ\-ле\-ние. Таким образом, в~проведенном 
эксперименте пороговое значение нагрузки уменьшилось по сравнению с~предыду\-щим 
экспериментом до значения $\rho_0\hm= 1{,}2$ при тех же значениях порогов и~доли 
сбрасываемой нагрузки. Это показывает, что значение порога обратно пропорционально 
размеру накопителя.
   

   На рис.~5 показано сравнение систем в зависимости от размера окна управления ($H-L$) 
   при фиксированном верхнем пороге $H\hm= 35$, накопителе объема $B\hm=40$, доле 
сбрасываемых пакетов $p\hm= 0{,}3$ и изменении нижнего порога~$L$ от~5 до~35. 

Из 
графиков видно, что, хотя пороговое значение нагрузки достаточно слабо зависит от 
значения нижнего порога, значение порога достаточно сильно влияет на среднюю длину 
очереди. 
   
\begin{figure*} %fig5
\vspace*{1pt}
  \begin{center}  
    \mbox{%
\epsfxsize=160.311mm
\epsfbox{ali-5.eps}
}

\end{center}
\vspace*{-6pt}
\Caption{Средняя длина очереди~(\textit{а}) и вероятность сброса пакета~(\textit{б}) в 
зависимости от размера окна управления: \textit{1}~--- гистерезисное управление; \textit{2}~---
управление с~по\-мощью алгоритма  IRED}
\end{figure*}


\section{Заключение}

  Численный эксперимент показал, что наиболее значимыми параметрами, влияющими 
на пороговое значение нагрузки, при превышении которого гистерезисный подход дает 
лучшие результаты по сравнению с алгоритмом IRED, с точки зрения вероятности сброса 
являются объем накопителя и~доля сбрасываемой нагрузки. Также было показано, что эти 
два па\-ра\-мет\-ра наиболее значимы и~для фиксированного порогового значения нагрузки.
  
  Помимо этого, для типичного набора исходных данных исследована зависимость 
показателей качества функционирования системы от размера окна управления. Показано, 
что для рассмотренного набора данных при размере окна $H\hm- L\hm \leq 3$ 
гистерезисное управление дает более хорошие результаты, чем IRED, во всех случаях, 
когда $\rho\hm> 1$, т.\,е.\ при перегрузках системы. При $\rho\hm= 0{,}8$ в IRED длина 
очереди меньше, но разница полученных значений для обоих алгоритмов мала, а 
поведение гистерезисной сис\-те\-мы объясняется ее недогруженностью. Таким образом, 
размер окна управ\-ле\-ния, хоть и~в~меньшей степени, но также влияет на пороговое 
значение нагрузки.
  
  В работе для модели беспроводной облачной сенсорной сети с механизмом 
гистерезисного управ\-ле\-ния нагрузкой исследованы основные показатели сис\-те\-мы~--- 
вероятность сброса пакета и~средняя длина очереди. Проведено численное сравнение 
гистерезисного подхода с~алгоритмом IRED и~показано, что при перегрузках 
гистерезисный подход показывает качественно лучшую картину, чем IRED, с~точки 
зрения вероятности сброса пакета. Также введено понятие порогового значения нагрузки, 
при превышении которого гистерезисный подход более эффективен, чем алгоритм IRED, 
и~показана за\-ви\-си\-мость порогового значения нагрузки от па\-ра\-мет\-ров сис\-те\-мы. Для 
нагрузки ниже порога гистерезисный подход незначительно проигрывает алгоритму типа 
IRED в некоторых диапазонах значений параметров системы. При этом алгоритм IRED 
реализуется на сетевом уровне, а~гистерезисное управ\-ле\-ние нагрузкой~--- на при\-клад\-ном 
уровне, где эти различия оказываются несущественными.

\vspace*{-7pt}

{\small\frenchspacing
 {%\baselineskip=10.8pt
 %\addcontentsline{toc}{section}{References}
 \begin{thebibliography}{9}
 
 \vspace*{-2pt}
 
\bibitem{1-al}
\Au{Yadav S.\,L., Ujjwal R.\,L.} Mitigating congestion in wireless sensor networks through clustering 
and queue assistance: A~survey~// J.~Intell. Manuf., 2021. Vol.~32. P.~2083--2098. 
doi: 10.1007/s10845-020-01640-8.
\bibitem{2-al}
\Au{Hasan M.} State of IoT 2022: Number of connected IoT devices growing 18\% to 14.4~billion 
globally. Pr., 2022. {\sf https://iot-analytics.com/number-connected-iot-devices}.
\bibitem{3-al}
\Au{Shiny S.\,S.\,G., Murugan~K.} TSDN-WISE: Automatic threshold-based low control-flow 
communication protocol for SDWSN~// IEEE Sens.~J., 2021. Vol.~21. No.\,17.  
P.~19560--19569. doi: 10.1109/JSEN.2021.3088604.
\bibitem{4-al}
Cisco Annual Internet Report (2018--2023) white paper, 2020. {\sf 
https://www.cisco.com/c/en/us/solutions/\linebreak collateral/executive-perspectives/annual-internet-report/ white-paper-c11-741490.html}.


\bibitem{6-al} %5
\Au{Bashir A., Machnev~E., Mokrov~E.} Queueing model of hysteretic congestion control for cloud 
wireless sensor networks~// 13th Congress (International) on Ultra Modern Telecommunications 
and Control Systems and Workshops.~--- Piscataway, NJ, USA: IEEE, 2021. P.~104--108. doi: 
10.1109/ICUMT54235.2021.9631576.
\bibitem{7-al} %6
\Au{Uyoata U., Mwangama~J., Adeogun~R.} Relaying in the Internet of Things (IoT): A~survey~// 
IEEE Access, 2021. Vol.~9. P.~132675--132704. doi: 10.1109/ACCESS.\linebreak 2021.3112940.

\bibitem{5-al} %7
\Au{Huang J., Du~D., Duan~Q., Zhang~Y., Zhao~Y., Luo~H., Mai~Z., Liu~Q.} Modeling and analysis 
on congestion con-\linebreak\vspace*{-12pt}

\pagebreak

\noindent
trol for data transmission in sensor clouds~// Int. J. Distrib. Sens.~N., 
2014. Vol.~2014. Art. ID 453983. P.~1--9. doi: 10.1155/2014/453983.
 
\bibitem{8-al}
\Au{Petrov V., Samuylov~A., Begishev~V., Moltchanov~D., And\-re\-ev~S., Samouylov~K., 
Koucheryavy~Y.} Vehicle-based relay assistance for opportunistic crowdsensing over 
Narrowband IoT (NB-IoT)~// IEEE Internet Things, 2018. Vol.~5. No.\,5. P.~3710--3723. 
doi: 10.1109/JIOT.2017.2670363.
\bibitem{9-al}
\Au{Красносельский М.\,А., Покровский~А.\,В.} Системы с гистерезисом.~--- М. Наука, 1983. 
272~с.

\end{thebibliography}

 }
 }

\end{multicols}

\vspace*{-6pt}

\hfill{\small\textit{Поступила в~редакцию 01.07.22}}

\vspace*{8pt}

%\pagebreak

%\newpage

%\vspace*{-28pt}

\hrule

\vspace*{2pt}

\hrule

%\vspace*{-2pt}

\def\tit{HYSTERETIC CONGESTION CONTROL IN~WIRELESS~CLOUD~SENSOR NETWORKS}


\def\titkol{Hysteretic congestion control in~wireless cloud sensor networks}


\def\aut{B.~Aliyu, E.\,A.~Machnev, and E.\,V.~Mokrov}

\def\autkol{B.~Aliyu, E.\,A.~Machnev, and E.\,V.~Mokrov}

\titel{\tit}{\aut}{\autkol}{\titkol}

\vspace*{-8pt}


\noindent
Peoples' Friendship University of Russia (RUDN University), 6~Miklukho-Maklaya Str., Moscow 
117198, Russian Federation


\def\leftfootline{\small{\textbf{\thepage}
\hfill INFORMATIKA I EE PRIMENENIYA~--- INFORMATICS AND
APPLICATIONS\ \ \ 2022\ \ \ volume~16\ \ \ issue\ 3}
}%
 \def\rightfootline{\small{INFORMATIKA I EE PRIMENENIYA~---
INFORMATICS AND APPLICATIONS\ \ \ 2022\ \ \ volume~16\ \ \ issue\ 3
\hfill \textbf{\thepage}}}

\vspace*{3pt} 




\Abste{The paper considers the problem of congestion analysis in a wireless sensor network. 
A~hysteresis control mechanism is proposed which screens out part of the load in the event of an 
overload of the system in order to stabilize its operation. A~mathematical model is constructed in the 
form of a~Markov process  with a~finite state space and a~formula is obtained for analyzing the 
performance indicators of a~wireless sensor network. The probabilistic-temporal characteristics of 
hysteresis control, the packet drop probability, and the average queue length are studied numerically. The 
hysteresis mechanism is compared with the IRED (improved random early detection) algorithm with different control window sizes. It is 
shown that the hysteresis mechanism provides an advantage in the region of high loads compared to the 
IRED mechanism.}

\KWE{wireless sensor networks; load control; hysteresis control; Markov process; queuing system}




\DOI{10.14357/19922264220311}

\vspace*{-16pt}

\Ack
\noindent
The reported study was funded by the Russian Science Foundation, project number 21-79-00157 (E.\,V.~Mokrov, system model 
and mathematical model) and by the RUDN University Strategic Academic Leadership Program 
(E.\,A.~Machnev, introduction). 



%\vspace*{4pt}

  \begin{multicols}{2}

\renewcommand{\bibname}{\protect\rmfamily References}
%\renewcommand{\bibname}{\large\protect\rm References}

{\small\frenchspacing
 {%\baselineskip=10.8pt
 \addcontentsline{toc}{section}{References}
 \begin{thebibliography}{9}
\bibitem{1-al-1}
\Aue{Yadav, S.\,L., and R.\,L.~Ujjwal.} 2021. Mitigating congestion in wireless sensor networks 
through clustering and queue assistance: A~survey. \textit{J.~Intell. Manuf.}  
32:2083--2098. doi: 10.1007/s10845-020-01640-8.
\bibitem{2-al-1}
\Aue{Hasan, M.} 2022.
State of IoT 2022: Number of connected IoT devices growing 18\% to 14.4~billion globally. Available 
at: {\sf https://iot-analytics.com/number-connected-iot-devices/} (accessed July~18, 2022).
\bibitem{3-al-1}
\Aue{Shiny, S.\,S.\,G., and K.~Murugan.} 2021.TSDN-WISE: Automatic threshold-based low  
control-flow communication protocol for SDWSN. \textit{IEEE Sens.~J.} 21(17):19560--19569. doi: 
10.1109/JSEN.2021.3088604.
\bibitem{4-al-1}
Cisco Annual Internet Report (2018--2023) white paper. 2020. Available at: {\sf 
https://www.cisco.com/c/en/us/\linebreak solutions/collateral/executive-perspectives/annual-internet-report/white-paper-c11-741490.html} 
(accessed July~18, 2022).

\bibitem{6-al-1} %5
\Aue{Bashir, A., E.~Machnev, and E.~Mokrov.} 2021. Queueing model of hysteretic congestion control 
for cloud wireless sensor networks. \textit{13th Congress (Internatioanl) on Ultra Modern 
Telecommunications and Control Systems Proceedings}. Piscataway, NJ: IEEE. 104--108. doi: 
10.1109/ICUMT54235.2021.9631576.
\bibitem{7-al-1} %6
\Aue{Uyoata, U., J.~Mwangama, and R.~Adeogun}. 2021. Relaying in the Internet of Things (IoT): 
A~survey. \textit{IEEE Access} 9:132675--132704. doi: 10.1109/ACCESS.2021.3112940.

\bibitem{5-al-1} %7
\Aue{Huang, J., D.~Du, Q.~Duan, Y.~Zhang, Y.~Zhao, H.~Luo, Z.~Mai, and Q.~Liu.} 2014. Modeling 
and analysis on congestion control for data transmission in sensor clouds. \textit{Int. J.~Distrib. 
Sens.~N.} 10(3):453983. doi: 10.1155/2014/ 453983.

\bibitem{8-al-1}
\Aue{Petrov, V., A.~Samuylov, V.~Begishev, D.~Moltchanov, S.~And\-re\-ev, K.~Samouylov, and 
Y.~Koucheryavy.} 2017. Vehicle-based relay assistance for opportunistic crowd sensing over 
Narrowband IoT (NB-IoT). \textit{IEEE Internet Things} 5(5):3710--3723. doi: 10.1109/JIOT.2017.2670363.
\bibitem{9-al-1}
\Aue{Krasnoselskii, M., and A.~Pokrovskii.} 1989. \textit{Systems with hysteresis}.  
Berlin\,--Heidelberg\,--New York\,--\,London\,--\,Paris\,--\,Tokio: Springer Verlag. 410~p.

\end{thebibliography}

 }
 }

\end{multicols}

\vspace*{-6pt}

\hfill{\small\textit{Received July 1, 2022}}

\Contr

\noindent
\textbf{Aliyu Bashir} (b.\ 1975)~--- PhD student, Department of Applied Probability and Informatics, 
Peoples' Friendship University of Russia (RUDN University), 6~Miklukho-Maklaya Str., Moscow 
117198, Russian Federation; \mbox{bashaliyuu@gmail.com}

\vspace*{4pt}

\noindent
\textbf{Machnev Egor A.} (b.\ 1996)~--- PhD student, Department of Applied Probability and 
Informatics, Peoples' Friendship University of Russia (RUDN University), 6~Miklukho-Maklaya Str., 
Moscow 117198, Russian Federation; \mbox{1032143100@rudn.ru}

\vspace*{4pt}

\noindent
\textbf{Mokrov Evgeniy V.} (b.\ 1988)~--- Candidate of Science (PhD) in physics and mathematics, 
assistant professor, Department of Applied Probability and Informatics, Peoples' Friendship University 
of Russia (RUDN University), 6~Miklukho-Maklaya Str., Moscow 117198, Russian Federation; 
\mbox{mokrov-ev@rudn.ru}
\label{end\stat}

\renewcommand{\bibname}{\protect\rm Литература}    