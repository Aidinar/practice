
\def\angstrem{\overset{\circ}{a}}
\def\stat{zatsman}

\def\tit{СРЕДОВЫЕ МОДЕЛИ ИНФОРМАЦИОННЫХ ТЕХНОЛОГИЙ: ТЕОРЕТИЧЕСКИЕ 
ОСНОВАНИЯ ПОСТРОЕНИЯ$^*$}

\def\titkol{Средовые модели информационных технологий: теоретические 
основания построения}

\def\aut{И.\,М.~Зацман$^1$}

\def\autkol{И.\,М.~Зацман}

\titel{\tit}{\aut}{\autkol}{\titkol}

\index{Зацман И.\,М.}
\index{Zatsman I.\,M.}


{\renewcommand{\thefootnote}{\fnsymbol{footnote}} \footnotetext[1]
{Исследование выполнено с~использованием ЦКП <<Информатика>> ФИЦ ИУ РАН при поддержке РФФИ 
(проект 20-012-00166) и~по гранту РФФИ и~Государственного фонда естественных наук (ГФЕН) Китая 
№\,21-57-53018.}}


\renewcommand{\thefootnote}{\arabic{footnote}}
\footnotetext[1]{Федеральный исследовательский центр <<Информатика и~управ\-ле\-ние>> Российской 
академии наук, \mbox{izatsman@yandex.ru}}


%\vspace*{-4pt}


 

  \Abst{Рассматриваются варианты модели ITO (information technology oriented model~--- 
ITO model), которые используются при проектировании информационных технологий (ИТ)
извлечения из текстов лингвистического и~медицинского знания в~рамках двух проектов 
по грантам РФФИ и~ГФЕН (КНР). Варианты модели были созданы в~рамках парадигмы 
деления предметной области информатики на среды различной природы. Цель статьи 
состоит в~описании исходных данных и~тео\-ре\-ти\-че\-ских оснований обобщения этих 
вариантов. Эти основания планируется использовать для создания обобщенной модели 
ИТ, ее видов и~част\-ных случаев, образующих новый класс 
моделей, которые предлагается назвать средовыми. Основная идея 
обобщения состоит в~распределении этапов проектируемой ИТ, их входов и~выходов по выделенным средам предметной об\-ласти информатики 
и~границам между ними в~интересах моделирования ИТ. На границах между средами 
размещаются сущ\-ности, неоднородные по своей природе (семиотические знаки на 
границе между ментальной и~информационной средами, таблицы кодирования Unicode на 
границе между информационной и~цифровой средами и~пр.). Такие ИТ предлагается 
назвать полисредовыми. Средовые модели ориентированы, в~част\-ности, 
на решение актуальной проблемы создания таб\-лиц унифицированного кодирования 
в~базах знаний семантического веба значений многозначных слов при обработке больших 
объемов текс\-тов (по аналогии с~таб\-ли\-ца\-ми Unicode для кодирования литер и~символов).}
    
\KW{средовые модели; среды предметной области информатики; 
варианты модели ITO; теоретические основания; полисредовые информационные 
технологии}

\DOI{10.14357/19922264220308}
  
\vspace*{3pt}


\vskip 10pt plus 9pt minus 6pt

\thispagestyle{headings}

\begin{multicols}{2}

\label{st\stat}
    
\section{Введение}

%\vspace*{-3pt}

  Отчет с~прогнозом будущего технологического развития, который был 
подготовлен под эгидой Всемирного экономического форума (Давос, 
Швейцария), содержит следующие положения~[1]:
  \begin{itemize}
  \item компьютеры и~развитие \textit{цифровой} среды способствуют 
приумножению когнитивного и~креативного потенциала человека (с.~3), 
сочетая возможности \textit{ментальной} и~других сред, в~том\linebreak чис\-ле за счет 
использования \textit{нейрокомпьютерных} интерфейсов (brain-computer 
interfaces) (с.~8);\\[-15pt]
  \item органичная (seamless) интеграция физического и~\textit{цифрового} 
миров с~помощью$\ldots$ программного обеспечения изменяет модели 
промышленного производства (с.~4);\\[-15pt]
  \item формирование институциональных структур общества 
в~значительной степени опре\-де\-ля\-ют\-ся$\ldots$ доступными 
\textit{технологиями} (с.~32).
  \end{itemize}
  
  В подготовке материалов для этого отчета приняли участие около 
800~экспертов и~руководителей отрасли информационных 
и~коммуникационных технологий. В~отчете с~результатами этого прогноза 
приведен перечень новых ИТ, которые 
определяют кардинальный характер преобразования общества и~экономики, 
получившего название <<Четвертая промышленная революция>>~[2]. 
Прогнозируемый характер преобразования во многом будет обуслов\-лен теми 
ИТ, которые охватывают сущности, принадлежащие средам разной природы 
(далее~--- \textit{полисредов$\acute{\mbox{ы}}$е информационные 
технологии}, или ПИТ), включая нейросреду, ментальную, 
информационную, циф\-ро\-вую и~другие среды предметной об\-ласти 
информатики~[3]. Точка зрения экспертов о~формировании 
институциональных структур общества и~определяющей роли в~их развитии 
ИТ расширяет предметную об\-ласть информатики 
и~ставит перед ней задачу создания тео\-ре\-ти\-че\-ских оснований 
проектирования и~моделирования ПИТ, а~также изучения и~применения 
интерфейсов на границах между средами и~в~точках их соприкосновения.
  
  Рисунок~1 иллюстрирует разнообразие интерфейсов, которые 
теоретически могут встретить-\linebreak\vspace*{-12pt}

{ \begin{center}  %fig1
 \vspace*{-1pt}
    \mbox{%
\epsfxsize=79mm
\epsfbox{zac-1.eps}
}

\end{center}

\noindent
{{\figurename~1}\ \ \small{Четыре среды разной природы, границы между ними и~точки 
соприкосновения
}}}

\vspace*{9pt}

\noindent
ся при проектировании ПИТ, охватывающих 
сущности нейросреды, ментальной, информационной и~цифровой сред. На 
границах между двумя смежными средами кружками условно обозначены 
интерфейсы 2-го порядка (они пронумерованы числами от~1 до~6), 
а~в~точках соприкосновения трех сред показаны интерфейсы 3-го порядка 
(7--10), которые, по определению в~работе~[3], обеспечивают связи между 
объектами (сущностями), принадлежащими трем средам (интерфейс 4-го 
порядка не показан).



  
  В информационных системах широко используются кодовые таблицы, 
например Unicode, которые служат наиболее распространенным способом 
реализации интерфейса 2-го порядка №\,3. В~системах организации знания 
интерфейс 2-го порядка №\,2 используется, например, для разрешения 
лексической неоднозначности и~может быть реализован с~помощью 
тезаурусов. Из интерфейсов 3-го порядка №\,7 применяется в~технологиях 
управления роботизированной рукой~[3]. Интерфейс №\,8 явно или неявно 
применяется при кодировании значений языковых единиц 
в~информационных системах.
  


  Цель статьи состоит в~описании исходных данных и~теоретических 
оснований построения средов$\acute{\mbox{ы}}$х моделей ПИТ в~рамках 
парадигмы деления предметной области информатики на среды. В~разд.~2 
рассмотрены исходные данные, с~использованием которых описываются 
парадигма деления предметной области информатики на среды и~затем, 
в~разд.~3, теоретические основания построения средов$\acute{\mbox{ы}}$х 
моделей. При описании теоретических оснований их построения 
используются примеры частных случаев этих моделей, применяемых при 
проектировании ПИТ в~двух проектах: извлечение из текстов новых значений 
модальных глаголов и~медицинских терминов.

\section{Исходные данные обобщения}

  При рассмотрении исходных данных используется характеристика 
информатики как фундаментальной науки из обзора~[4]. На основе этого обзора была разработана Европейская 
стратегия преподавания информатики~[5], которая предусматривает 
диадический подход к~преподаванию: (1)~информатики как самостоятельного 
предмета; (2)~методов и~средств информатики в~предметных областях других 
наук~[6]. Для данной статьи наибольший интерес представляет вторая 
позиция этого подхода. Она обусловливает существенное расширение 
предметной области информатики по сравнению с~традиционными 
границами ее содержания\footnote{В~1967~г.\ А.~Ньюэлл, А.\,Дж.~Перлис 
и~Х.\,А.~Саймон написали в~журнале <<Science>>: <<Объектам и~явлениям соответствуют те науки, 
которые их изучают. Появились компьютеры. Следовательно, назначение информатики~--- это 
изучение компьютеров>>~[7]. Такой подход к~определению ее содержания доминировал долгие 
годы в~информатике как компьютерной науке.}. Причина ее расширения~--- 
необходимость рассматривать и~моделировать информационные процессы не 
только в~искусственных, но и~в~живых и~в~социальных сис\-те\-мах. Такой 
спектр информационных процессов будет охватывать объекты (сущ\-ности) 
нескольких сред предметной об\-ласти информатики и,~что важ\-но отметить, 
\textit{сред принципиально разной природы}.
  
  В обзоре~[4] его авторы предложили характеристику современной 
информатики, отметив, что она кардинально изменилась за последние 
50~лет: <<В~то время как естественные науки определяются применительно 
к~миру (world), в~котором мы живем, информатику как научную дисциплину 
определить сложнее; у~нее нет эмпирических основ, как у~естественных 
наук; это нечто большее, чем фор\-маль\-но-сим\-воль\-ное мышление, как 
в~математике; и~это далеко не просто компиляция инженерных принципов 
и~технологий>>.
  
  Авторы этого обзора не претендуют на строгое и~детальное определение 
предметной области информатики. Они предлагают косвенную ее 
характеристику, используя три отобранных ими ключевые работы~[8--10], 
которые, с~их точки зрения, дополняют друг друга при описании 
информатики как фундаментальной науки.
  
  Первая публикация~--- это доклад Кристена Нюгорда (Университет Осло), 
который был прочитан им на Всемирном конгрессе IFIP (International 
Federation of Information Processing~--- Международная федерация по 
обработке информации) в~Дублине в~1986~г. Второй раздел <<The definition 
of the term ``informatics''>> этого доклада посвящен определению термина 
<<информатика>>. В~этом разделе он пишет:
  %
  <<Термин ``computer science'' следует заменить на ``informatics''. Несколько 
лет назад выбор между этими двумя терминами казался, скорее всего, 
несущественным. Обсуждения терминологии часто считают праздными, но 
иногда они могут отражать ключевые различия во мнениях или по крайней 
мере акценты [в~понимании терминов]. Сегодня, к~сожалению, используется 
термин ``computer science''. Этот термин имеет тенденцию поддерживать 
слишком узкое представление об информационных системах, которые 
в~настоящее время связывают коллективы людей и~интегрируют 
разнообразные средства обработки информации. При этом коллективы 
взаимодействуют как посредством межличностных связей людей, так 
и~с~использованием электронных каналов связи>>~\cite{8-zac}.
  
  После констатации важности четкого определения терминов 
<<информатика>> и~<<информация>> для формирования адекватного 
представления об информационных системах Кристен Нюгорд предлагает 
свою дефиницию первого термина:
  %
  <<Информатика~--- это наука, которая имеет своей областью 
[исследований] информационные процессы и~\textit{связанные с~ними 
феномены в~артефактах, обществе и~природе} (курсив мой~--- 
И.~З.)>>~\cite{8-zac}.
  
  Далее дается ссылка на определение понятия <<феномен>> в~словаре 
Webster 1960~г.\ (<<любой факт, обстоятельство или событие, которые 
сенсорно воспринимаются и~которые могут быть научно описаны или 
оценены>>~\cite{11-zac}). Затем Кристен Нюгорд дает расширенное 
толкование этого понятия, которое кардинально отличается от его 
определения в~словаре Webster 1960~г.:
  %
  <<Важными примерами феноменов являются: живые организмы, 
неоду\-шев\-лен\-ные объекты (включая артефакты, такие как, например, 
машины), события и~процессы (например, выполнение компьютерных 
программ). Мы также можем говорить о~\textit{когнитивных феноменах, 
происходящих в~сознании людей}, в~отличие от явных [сенсорно 
воспринимаемых] \textit{феноменов, находящихся вне сознания} (курсив 
мой~---  И.\,З.)>>~\cite{8-zac}.
  
  Толкование Кристена Нюгорда с~делением феноменов, \textit{связанных 
с~информационными процессами}, на когнитивные, происходящие 
в~сознании людей и~формирующие концепты знания человека, и~сенсорно 
воспринимаемые (например, тексты), существующие вне сознания, является 
ключевым. Такое деление уже неявно, но существенно расширяет 
предметную область информатики и~создает основание для нового ее 
позиционирования в~сис\-те\-ме современного научного знания. Имплицитно 
такое толкование вводит в~пред\-мет\-ную об\-ласть информатики сущ\-ности 
разной природы: ментальной и~информационной (=\;зна\-ко\-вой сенсорно 
воспринимаемой в~данной \mbox{статье}). Отметим, что толкование понятия 
<<феномен>>, расширенное по объему его значения, позже было включено 
и~в~современный он\-лайн-сло\-варь Merriam-Webster\footnote{Современный  
он\-лайн-сло\-варь Merriam-Webster приводит следующую дефиницию термина <<феномен>>: 
<<Факт или событие, представляющие научный интерес, поддающиеся научному описанию 
и~объяснению>>~\cite{12-zac}, которая соответствует толкованию Кристена Нюгорда, но без явного 
деления феноменов в~этом словаре на когнитивные и~сенсорно воспринимаемые.} (в~новом 
определении удалена \textit{сенсорная воспринимаемость} для феноменов, 
поддающихся научному опи\-санию).
{\looseness=1

}
  
  Таким образом, смысловое содержание термина <<информатика>> по 
Нюгорду зависит только от двух понятий: <<феномен>> 
и~<<информационный процесс>>,~--- при этом в~определении второго понятия 
как единого словосочетания Кристен Нюгорд не использует слово 
<<информация>>. Далее он комментирует свою дефиницию термина 
<<информатика>> следующим образом:
  %
%  \noindent
  <<Приведенная выше дефиниция информатики принципиально 
отличается от дефиниций, которые на\-прав\-ле\-ны на определение\linebreak информатики 
как формальной дис\-ципли\-ны, сродни математике. Однако формальные 
задачи могут относиться к~пред\-мет\-ной об\-ласти информатики~--- например, 
доказательство пра\-виль\-ности\linebreak про\-грамм$\ldots$ Формирование дефиниции 
термина <<информатика>> никоим образом не является тривиальным. Когда 
утверждается, что информатика является только формальной дис\-циплиной, 
то <<согласно такому определению, вли\-яние информационной сис\-те\-мы на 
социум, в~который она интегрирована, находится за пределами пред\-мет\-ной 
об\-ласти информатики. Кроме того, исследования того, как осуществляется 
обработка данных в~организациях, так\-же выходят за рамки информатики в~этом уз\-ком смыс\-ле>>~\cite{13-zac}>>~\cite{8-zac}.
  
  Можно предположить, что выбор авторами обзора~\cite{4-zac} доклада 
Кристена Нюгорда, опубликованного более~35~лет назад, был обуслов\-лен, 
скорее всего, включением в~предметную область широкого спектра 
феноменов, связанных с~информационными процессами 
в~\textit{технических, живых и~социальных системах}.
  
  Из трех публикаций, выбранных авторами обзора~\cite{4-zac}, вторая~--- 
это книга Дэвида Харела (Научный институт Вейцмана), опубликованная 
в~1987~г., с~изложением основ информатики, ряда важных и~базовых ее тем 
с~алгоритмической точки зрения. В~книге подчеркивается фундаментальный 
характер этой науки. В~ней анализируются три вида сложности проб\-лем 
информатики: вычислительная, системная и~когнитивная~\cite{9-zac}.
  
  Для обобщения вариантов модели ITO существенный интерес 
представляет сложность третьего\linebreak вида~--- когнитивная, которая порождается 
использованием в~ИТ сущностей, принадлежащих ментальной среде 
(концепты знания в~сознании людей), или сущностей двойной природы, 
\mbox{принадлежащих} ее границам с~нейросредой, циф\-ро\-вой, информационной 
и~другими средами предметной области информатики. Этот вид слож\-ности 
связан с~задачами, которые не удается решить с~применением формальных 
алгоритмов, так как час\-то они не поддаются точной постановке и~четкому 
формальному определению этих сущностей. Примером такой задачи может 
служить специфицирование в~базе знаний дефиниций новых значений 
модальных глаголов в~контекстах, для которых уже известные базе знаний их 
значения оказываются нерелевантными контекстам~\cite{14-zac}.
  
  Дэвид Харел называет такие задачи <<псевдоалгоритмическими>> 
проблемами (pseudo-algorithmic problems), подчеркивая в~своей книге, что он 
следует формальной трактовке термина <<алгоритм>>. Пристальное 
внимание к~таким проблемам\linebreak обуслов\-ле\-но желанием достичь эффекта от их 
компьютерного решения, подобного человеческому. Такое стремление ставит 
принципиально новые задачи~--- разработать системы, которые будут 
демонстрировать <<человекоподобные интеллектуальные действия>>. 
Главный вопрос при решении подобных проб\-лем состоит в~том, чтобы 
описать сложное знание человека так, чтобы его пред\-став\-ле\-ние поддавалось 
алгоритмической обработке, применению сис\-тем и~средств информатики для 
их решения~\cite[с.~402]{9-zac}.
  
  Из трех публикаций, выбранных авторами обзора~\cite{4-zac}, третья 
работа, комплементарная с~\cite{8-zac, 9-zac},~--- это статья Питера 
Деннинга и~Пола Розенблюма~\cite{10-zac}. В~ней предложен новый 
вариант позиционирования информатики в~системе современного научного 
знания и~обосновывается необходимость существенного расширения ее 
предметной области. Один из подходов к~ее расширению и~предлагаемая 
парадигма деления предметной области на среды различной природы 
приведены в~работах~\cite{3-zac, 15-zac, 16-zac, 17-zac}.
  
  В работах~\cite{3-zac, 16-zac} по развитию варианта Деннинга 
и~Розенблюма были описаны пять сред предметной области информатики 
и~границы между ними. Отметим, что предлагаемое развитие этого варианта 
охватывает следующие ключевые идеи и~Кристена Нюгорда, и~Дэвида Харела:
  \begin{itemize}
  \item   деление феноменов, связанных с~информационными процессами, на 
когнитивные, происходящие в~сознании людей (ментальная среда), 
и~сенсорно воспринимаемые (\mbox{информационная} среда), находящиеся вне 
сознания, что имплицитно уже вводит в~предметную область информатики 
сущности разной природы~\cite{8-zac};
  \item определение класса проблем когнитивной сложности, для решения 
которых с~применением систем и~средств информатики необходимо 
представить сложное знание так, чтобы оно поддавалось алгоритмической 
обработке в~циф\-ро\-вой среде~\cite{9-zac}.
  \end{itemize}
  

  
Рассмотренные работы Деннинга, Розенблюма, Нюгорда, Харела, 
Касперсена\footnote{С~2020~г.\ Майкл Касперсен возглавляет Европейский комитет по 
компетенциям в~области информатики (European Informatics Competence Framework Committee). С~2022~г.\ 
он специальный советник по компьютерному образованию вице-президента Европейской комиссии 
Маргрет Вестагер.} и~его соавторов, об\-суж\-да\-ющие вопросы позиционирования 
информатики в~сис\-те\-ме современного научного знания,\linebreak были опубликованы 
в~период с~1986 по~2019~гг. В~настоящее время эти вопросы сохраняют 
свою актуальность и~продолжается их активное обсуждение, в~том числе 
с~позиции компьютерного образования, качество которого существенно 
зависит от их решения~\cite{18-zac}.

 
  
  Контекст рассмотренных работ дает возможность сформулировать 
следующие исходные позиции для разработки теоретических оснований 
созда\-ния обобщенной модели ПИТ, ее видов и~частных случаев:
  \begin{itemize}
  \item  деление предметной области информатики на среды разной 
природы;
  \item распределение объектов одной природы по средам (концепты 
в~ментальной среде, компьютерные коды в~цифровой и~т.\,д.~\cite{3-zac});
  \item  распределение объектов двойной природы по границам между двумя 
средами со\-от\-вет\-ст\-ву\-ющей природы (семиотические знаки на границе между 
ментальной и~информационной средами, таблицы кодирования Unicode на 
границе между информационной и~цифровой средами и~т.\,д.);
  \item распределение объектов тройной природы (и~более высокого 
порядка) в~точках соприкосновения трех и~большего числа 
сред, соответствующих природе объектов; например, таблица кодирования 
значений слов (см.\ интерфейс №\,8 на рис.~1) принадлежит точке 
соприкосновения трех сред: ментальной (значения слов), информационной 
(слова как последовательности литер) и~цифровой (компьютерные коды 
значений слов)~\cite{3-zac, 16-zac}).
  \end{itemize}
  
\section{Теоретические основания}

  В данном разделе на базе рассмотренных исходных позиций опишем 
теоретические основания создания обобщенной модели ПИТ, используя 
ранее разработанные варианты модели  
ITO~\cite{14-zac, 19-zac, 20-zac, 21-zac}. Первый вариант этой модели 
использовался при проектировании технологии извлечения новых значений 
немецких модальных глаголов из параллельных текс\-тов художественных 
произведений \textit{с~\mbox{целью}} пополнения словарных статей  
не\-мец\-ко-рус\-ских словарей. Сначала для каждого этапа технологии были 
определены его вход и~выход. Затем \textit{этапы}, их \textit{входы} 
и~\textit{выходы} были распределены по средам и~границам между ними 
(рис.~2). Описание применения этого варианта модели ITO дано  
в~\cite{14-zac}. Используя рис.~2 и~исходные данные предыдущего раздела, 
перечислим предлагаемые теоретические основания создания обобщенной 
модели ПИТ:
  \begin{itemize}
  \item выделение в~предметной области информатики тех сред разной 
природы, которые необходимы для моделирования проектируемой ПИТ, 
определяется \mbox{целью} ее создания (на рис.~2 показаны три среды, достаточные 
для моделирования технологии пополнения словарных статей);
  \item распределение моноприродных этапов ПИТ, их входов и~выходов по 
средам определяется их природой\footnote{На рис.~2 показаны три моноприродных 
этапа (Б, В и~Г). Строго говоря, только этап~В является таковым. При выполнении этапа~Б эксперт 
использует свое понимание смысла глагола, сформированное им на этапе~А, а при выполнении 
этапа Г эксперты могут обмениваться сообщениями в~процессе согласования личностных 
интерпретаций. Однако для упрощения рис.~2 этапы Б--Г, их вхо\-ды/вы\-хо\-ды считаются 
моноприродными и~обозначены одним цветом (оранжевым или бирюзовым).};
  \item распределение этапов двойной природы, их входов и~выходов по 
границам между двумя средами определяется природой смежных 
сред\footnote{На рис.~2 показаны пять этапов двойной природы. Этапы <<Интерпретация 
текстов дефиниций значения>> и~<<Формирование коллективной дефиниции значения глагола>> 
размещены на границе между ментальной и~информационной средами. Этапы <<Визуализация>> 
и~<<Оцифровка>>~--- между информационной и~цифровой средами. Этап~А размещен 
в~информационной среде, чтобы упростить рис.~2. Его двойная природа условно обозначена 
оранжево-бирюзовой цветовой гаммой.}; 
\end{itemize}

\end{multicols}

 \setcounter{figure}{1}
  \begin{figure*}[h] %fig2
  \vspace*{-10pt}
  \begin{center}  
    \mbox{%
\epsfxsize=163.291mm
\epsfbox{zac-2.eps}
}

\end{center}
\vspace*{-9pt}
  \Caption{Первый вариант модели ITO}
  \vspace*{-12pt}
  \end{figure*}
  
  \begin{multicols}{2}

\begin{itemize}
  \item распределение полиприродных этапов, их входов и~выходов по 
точкам соприкосновения трех и~более сред определяется природой сред, 
которые сходятся в~этих точках (см.\ интерфейсы 7--10 на рис.~1).
  \end{itemize}
  

  
  В первом варианте модели ITO использовались только интерфейсы 
второго порядка и~реа\-ли\-зу\-ющие их этапы двойной природы. Например,\linebreak 
интерфейсу №\,2 на рис.~1 соответствуют этапы <<Интерпретация текс\-тов 
дефиниций значения>> и~<<Формирование коллективной дефиниции 
значения глагола>>, а~интерфейсу №\,3~--- этапы <<Визуализация>> 
и~<<Оциф\-ровка>>.
  
  Второй вариант модели ITO использовался при проектировании 
технологии извлечения новых терминов из текстов медицинских документов с~\mbox{целью}
 формирования и~актуализации в~базе знаний терминологических 
портретов хронических болезней. Графическое пред\-став\-ле\-ние этого варианта 
приведено на рис.~5 в~\cite{20-zac}. Третий вариант модели ITO, 
адаптированный для одновременного извлечения\linebreak новых терминов 
и~индикаторов на\-стро\-ений, названный ITO-Sent, рассмотрен в~работе~[21]. 
Графическое пред\-став\-ле\-ние этого варианта модели приведено на рис.~2 
в~[21]. Со\-по\-ста\-ви\-тель\-ное \mbox{описание} вариантов модели ITO, приведенное 
в~работах~\cite{17-zac, 21-zac}, поз\-во\-ля\-ет определить те их 
характеристики~[22], которые планируется использовать при описании на 
по\-сле\-ду\-ющих этапах обобщенной модели, ее видов и~част\-ных случаев, 
об\-ра\-зу\-ющих новый класс средов$\acute{\mbox{ы}}$х моделей ПИТ.
  
\section{Заключение}

  Формирование класса средов$\acute{\mbox{ы}}$х моделей ПИТ 
планируется осуществить на основе рассмотренных теоретических 
оснований. На первом этапе его формирования основное внимание будет 
уделено виду средов$\acute{\mbox{ы}}$х моделей технологий кодирования 
значений многозначных слов в~базах знаний. Модели этого вида 
ориентированы в~первую очередь на решение актуальной проблемы 
создания (по аналогии с~таб\-ли\-ца\-ми Unicode для кодирования литер 
и~символов) таб\-лиц одновременного кодирования значений многозначных 
языковых единиц (слов, коллокаций и~пр.)\ и~форм представления их 
значений в~информационной среде в~виде последовательностей литер 
(символов).
  
  Сама идея одновременного кодирования значений слов и~форм их 
представления не нова. Например, она используется в~тезаурусе 
RuWordNet~[23] и~проекте анализа процесса возникновения новых значений 
слов английского языка~[24]. Отличительная черта этого вида 
средов$\acute{\mbox{ы}}$х моделей ПИТ состоит в~кодировании кроме 
самого значения степени и~этапов его социализации. Это даст возможность 
кодировать в~базах знаний значения новых слов, различая личностные, 
коллективные и~конвенциональные их  
значения~\cite{14-zac, 17-zac, 19-zac, 20-zac, 21-zac}.
  
  На последующих этапах обобщения предполагается описать еще три вида 
средов$\acute{\mbox{ы}}$х моделей: информационно-технологически 
ориентированные (средов$\acute{\mbox{ы}}$е модели ITO~--- medium ITO 
models), познавательно-объяснительные (образовательные 
средов$\acute{\mbox{ы}}$е модели), а~также прогностические~[25]. 
В~заключение отметим, что частные случаи средов$\acute{\mbox{ы}}$х 
моделей с~ментальной, информационной и~цифровой средами начали 
применяться в~2012~г.\ при проектировании и~моделировании технологий 
формирования индикаторов мониторинга и~оценки на\-уч\-но-ис\-сле\-до\-ва\-тель\-ских программ~[26, 27].
  
 {\small\frenchspacing
 {%\baselineskip=10.8pt
 %\addcontentsline{toc}{section}{References}
 \begin{thebibliography}{99}
\bibitem{1-zac}
Deep shift~--- technology tipping points and societal impact~// World 
Economic Forum.~--- Geneva, Switzerland, 2015. {\sf 
http://www3.weforum.org/docs/WEF\_\linebreak GAC15\_Technological\_Tipping\_Points\_report\_2015.pdf}.
\bibitem{2-zac}
\Au{Шваб К.} Четвертая промышленная революция~/ Пер. c англ.~--- М.: Эксмо, 2018. 
288~с. (\Au{Schwab~K.} The fourth industrial revolution.~--- Geneva, Switzerland: World 
Economic Forum, 2016. 172~p.)
\bibitem{3-zac}
\Au{Зацман И.\,М.} Интерфейсы третьего порядка в~информатике~// Информатика и~её 
применения, 2019. Т.~13. Вып.~3. С.~82--89.
\bibitem{4-zac}
Informatics education in Europe: Are we all in the same boat?~--- New York, NY, USA: ACM, 2017. 
Technical Report of the 
Committee on European Computing Education. 251~p. 
doi: 10.1145/3106077.
\bibitem{5-zac}
\Au{Caspersen M.\,E., Gal-Ezer~J., McGettrick~A., Nardelli~E.} Informatics for all: The 
strategy.~--- New York, NY, USA: ACM, 2018. 16~p.
\bibitem{6-zac}
\Au{Caspersen M.\,E., Gal-Ezer~J., McGettrick~A., Nardelli~E.} Informatics as a~fundamental 
discipline for the 21st sentury~// Commun. ACM, 2019. Vol.~62. No.\,4.  
P.~58--63.
\bibitem{7-zac}
\Au{Newell A., Perlis~A., Simon~H.} Computer science~// Science, 1967. Vol.~157. No.\,3795. 
P.~1373--1374.
\bibitem{8-zac}
\Au{Nygaard K.} Program development as a~social activity~// 10th World Computer Congress Proceedings~/ Ed. \mbox{H.-J.}~Kugler.~--- 
North Holland: Elsevier Science Publs. B.V., IFIP, 1986. 
P.~189--198.
\bibitem{9-zac}
\Au{Harel D.} Algorithmics~--- the spirit of computing.~--- Reading, MA, USA:  
Addison-Wesley, 1987. 514~p.
\bibitem{10-zac}
\Au{Denning~P., Rosenbloom~P.} Computing: The fourth great domain of science~// Commun. 
ACM, 2009. Vol.~52. No.\,9. P.~27--29.
\bibitem{11-zac}
Webster's New World dictionary of the American language~/
Eds.\  D.\,B.~Guralnik, J.\,H.~Friend.~--- New York, NY, USA: The World Publishing 
Co., 1960. 1760~p.
\bibitem{12-zac}
Merriam-Webster's dictionary. Definition of phenomenon. {\sf  
https://www.merriam-webster.com/\linebreak dictionary/phenomenon}.
\bibitem{13-zac}
\Au{Nygaard K., H$\overset{{\hspace*{1pt}\circ}}{a}$ndlykken~P.} The system development process~--- its setting, 
some problems and needs for methods~// Software Engineering Environments  Symposium Proceedings.~--- Amsterdam, 1981.  
P.~157--172.
\bibitem{14-zac}
\Au{Zatsman I.} A model of goal-oriented knowledge discovery based on human--computer 
symbiosis~// 16th Forum (International) on Knowledge Asset Dynamics Proceedings.~--- 
Matera, Italy: Arts for Business Institute, 2021. P.~297--312.
\bibitem{15-zac}
\Au{Зацман И.\,М.} Методология обратимой генерализации в~контексте классификации 
информационных трансформаций~// Системы и~средства информатики, 2018. Т.~28. №\,2. 
С.~128--144.
\bibitem{16-zac}
\Au{Зацман И.\,М.} Кодирование концептов в~цифровой среде~// Информатика и~её 
применения, 2019. Т.~13. Вып.~4. С.~97--106.
\bibitem{17-zac}
\Au{Зацман И.\,М.} Модель процесса извлечения новых терминов и~тональных слов из 
текстов~// Системы и~средства информатики, 2022. Т.~32. №\,2. С.~115--127.
\bibitem{18-zac}
\Au{Tedre M., Pajunen~J.} Grand theories or design guidelines? Perspectives on the role of 
theory in computing education research~// ACM T. Comput. Educ., 2021. 
doi: 10.1145/3487049.
\bibitem{19-zac}
\Au{Zatsman I.} Finding and filling lacunas in linguistic typologies~// 15th  Forum (International) 
on Knowledge Asset Dynamics Proceedings.~--- Matera, Italy: Arts for Business Institute, 2020. 
P.~780--793.
\bibitem{20-zac}
\Au{Zatsman I., Khakimova~A.} New knowledge discovery for creating terminological profiles 
of diseases~// 22nd European Conference on Knowledge Management Proceedings.~--- Reading, 
U.K.: Academic Publishing International Ltd., 2021. P.~837--846.
\bibitem{21-zac}
\Au{Зацман И.\,М. Золотарев~О.\,В., Хакимова~А.\,Х.} Средовые модели извлечения из 
текста новых терминов и~индикаторов настроений~// Информатика и~её применения, 2022. 
Т.~16. Вып.~2. С.~60--67.
\bibitem{22-zac}
\Au{Зацман И.\,М.} Проблемно-ориентированная актуализация словарных статей 
двуязычных словарей и~медицинской терминологии: сопоставительный анализ~// 
Информатика и~её применения, 2021. Т.~15. Вып.~1. С.~94--101.
\bibitem{23-zac}
\Au{Bolshina A., Loukachevitch~N.} All-words word sense disambiguation for Russian using 
automatically generated text collection~// Cybernetics Information Technologies, 2020. 
Vol.~20. Iss.~4. P.~90--107.
\bibitem{24-zac}
\Au{Ramiro C., Srinivasan~M., Malt~B.\,C., Xu~Y.} Algorithms in the historical emergence of 
word senses~// P.~Natl. Acad. Sci. USA, 2018. Vol.~115. Iss.~10. 
P.~2323--2328.
\bibitem{25-zac}
\Au{Зацман И.\,М.} Таблица интерфейсов информатики как  
ин\-фор\-ма\-ци\-он\-но-компью\-тер\-ной науки~// На\-уч\-но-тех\-ни\-че\-ская 
информация. Сер.~1: Организация и~методика информационной работы, 2014. №\,11.  
С.~1--15.
\bibitem{26-zac}
\Au{Zatsman I.} Tracing emerging meanings by computer: Semiotic framework~// 13th European 
Conference on Knowledge Management Proceedings.~--- Reading, U.K.: Academic Publishing 
International Ltd., 2012. Vol.~2. P.~1298--1307.
\bibitem{27-zac}
\Au{Zatsman I.} Denotatum-based models of knowledge creation for monitoring and evaluating 
R\&D program implementation~// 11th IEEE Conference (International) on Cognitive 
Informatics and Cognitive Computing Proceedings.~--- Los Alamitos, CA, USA: IEEE 
Computer Society Press, 2012. P.~27--34.

  \end{thebibliography}

 }
 }

\end{multicols}

\vspace*{-6pt}

\hfill{\small\textit{Поступила в~редакцию 14.07.22}}

\vspace*{8pt}

%\pagebreak

%\newpage

%\vspace*{-28pt}

\hrule

\vspace*{2pt}

\hrule

%\vspace*{-2pt}

\def\tit{INFORMATICS' MEDIUM MODELS OF~INFORMATION TECHNOLOGY: THEORETICAL 
FOUNDATIONS FOR~THEIR~CREATING}


\def\titkol{Informatics' medium models of~information technology: Theoretical 
foundations for~their~creating}


\def\aut{I.\,M.~Zatsman}

\def\autkol{I.\,M.~Zatsman}

\titel{\tit}{\aut}{\autkol}{\titkol}

\vspace*{-8pt}


\noindent
Federal Research Center ``Computer Science and Control'' of the Russian Academy of Sciences,  
44-2~Vavilov Str., Moscow 119333, Russian Federation


\def\leftfootline{\small{\textbf{\thepage}
\hfill INFORMATIKA I EE PRIMENENIYA~--- INFORMATICS AND
APPLICATIONS\ \ \ 2022\ \ \ volume~16\ \ \ issue\ 3}
}%
 \def\rightfootline{\small{INFORMATIKA I EE PRIMENENIYA~---
INFORMATICS AND APPLICATIONS\ \ \ 2022\ \ \ volume~16\ \ \ issue\ 3
\hfill \textbf{\thepage}}}

\vspace*{3pt}
     


\Abste{The variants of the information technology-oriented (ITO) model are considered which are used 
in the design of information technologies (IT) for discovering linguistic and medical knowledge from texts 
within the framework of two projects under the grants from the Russian Foundation for Basic Research 
(RFBR) and the Natural Science Foundation of China (NSFC). The variants of the ITO model were created 
within the paradigm of dividing the subject domain of informatics into media of different nature. The 
aim of the paper is to describe the initial data and the theoretical foundations for the generalization of the 
ITO model. These foundations are planned to be used to create a generalized model of IT, 
its types, and particular cases as constituents of the new class of models that are proposed to 
be called the informatics' medium ones. The principal idea of generalization is to distribute the stages of the 
IT being designed, their inputs, and outputs according to media of the subject 
domain of informatics and the boundaries between them in the interests of IT modeling. The boundaries 
locate entities of\linebreak\vspace*{-12pt}}

\Abstend{multiple nature (semiotic signs on the boundary between the mental and information 
media, Unicode coding tables
on the boundary between the information and digital media, etc.). 
Such IT is proposed to be called the multimedium one. Medium models are focused, first of all, on solving 
the topical problem of creating unified tables for coding meanings (=\;senses) of ambiguous words in 
semantic web knowledge bases when processing large volumes of texts (by analogy with Unicode tables 
for encoding letters and symbols).}

\KWE{medium models in informatics; media of informatics subject domain; ITO model variants; 
theoretical foundations; multimedium information technologies}




\DOI{10.14357/19922264220308}

%\vspace*{-16pt}

\Ack
\noindent
The research was carried out using infrastructure of shared research facilities CKP ``Informatics'' of FRC 
CSC RAS. The reported study was funded by RFBR, project number 20-012-00166, and by RFBR and 
NSFC, project number 21-57-53018.


%\vspace*{4pt}

  \begin{multicols}{2}

\renewcommand{\bibname}{\protect\rmfamily References}
%\renewcommand{\bibname}{\large\protect\rm References}

{\small\frenchspacing
 {%\baselineskip=10.8pt
 \addcontentsline{toc}{section}{References}
 \begin{thebibliography}{99}
\bibitem{1-zac-1}
 Deep shift~--- technology tipping points and societal impact.   2015.
 \textit{World Economic Forum.} Geneva, 
Switzerland. Available at: {\sf 
http://www3.weforum.org/docs/WEF\_ GAC15\_Technological\_Tipping\_Points\_report\_2015.pdf} 
(accessed July~27, 2022).
\bibitem{2-zac-1}
\Aue{Schwab, K.} 2016. \textit{The fourth industrial revolution}. Geneva, Switzerland: World Economic 
Forum. 172~p. 
\bibitem{3-zac-1}
\Aue{Zatsman, I.\,M.} 2019. {Interfeysy tret'ego poryadka v~informatike} [Third-order interfaces in 
informatics]. \textit{Informatika i~ee Primeneniya~--- Inform. Appl.} 13(3):82--89.
\bibitem{4-zac-1}
The Committee on European Computing Education. 2017. Informatics education in Europe: Are we all in 
the same boat? New York, NY: ACM. Technical Report. 251~p. Available at: {\sf  
https://dl.acm.org/citation.cfm? id=3106077} (accessed July~27, 2022).
\bibitem{5-zac-1}
\Aue{Caspersen, M.\,E., J.~Gal-Ezer, A.~McGettrick, and E.~Nardelli.} 2018. \textit{Informatics for all: 
The strategy.} New York, NY: ACM. 16~p.
\bibitem{6-zac-1}
\Aue{Caspersen, M.\,E., J.~Gal-Ezer, A.~McGettrick, and E.~Nardelli.} 2019. Informatics as 
a~fundamental discipline for the 21st century. \textit{Commun. ACM} 62(4):58--63.
\bibitem{7-zac-1}
\Aue{Newell, A., A.~Perlis, and H.~Simon.} 1967. Computer science. \textit{Science} 
157(3795):1373--1374.
\bibitem{8-zac-1}
\Aue{Nygaard, K.} 1986. Program development as a social activity. \textit{10th World Computer 
Congress Proceedings}. Ed. \mbox{H.-J.}~Kugler. North Holland: Elsevier Science Publs. B.V., IFIP.  
189--198.
\bibitem{9-zac-1}
\Aue{Harel, D.} 1987. \textit{Algorithmics~--- the spirit of computing}. Reading, MA: Addison-Wesley. 
514~p.
\bibitem{10-zac-1}
\Aue{Denning, P., and P.~Rosenbloom.} 2009. Computing: The fourth great domain of science. 
\textit{Commun. ACM} 52(9):27--29.
\bibitem{11-zac-1}
Guralnik, D.\,B., and J.\,H.~Friend, eds. 1960. \textit{Webster's New World dictionary of the American 
language}. New York, NY: The World Publishing Co. 1760~p.
\bibitem{12-zac-1}
{Merriam-Webster's dictionary. Definition of phenomenon}. Available at: {\sf 
https://www.merriam-webster.\linebreak com/dictionary/phenomenon} (accessed July~27, 2022).
\bibitem{13-zac-1}
\Aue{Nygaard, K., and P.~H$\overset{{ \hspace*{1pt}\circ}}{\mathrm{a}}$ndlykken.} 1981. The system development process~--- its setting, 
some problems and needs for methods. \textit{Software Engineering Environments Symposium 
Proceedings}. Amsterdam. 157--172.
\bibitem{14-zac-1}
\Aue{Zatsman, I.} 2021. A~model of goal-oriented knowledge discovery based on human--computer 
symbiosis. \textit{16th Forum (International) on Knowledge Asset Dynamics Proceedings.} Matera, 
Italy: Arts for Business Institute. 297--312.
\bibitem{15-zac-1}
\Aue{Zatsman, I.} 2018. Metodologiya obratimoy generalizatsii v~kontekste klassifikatsii 
informatsionnykh trans\-for\-ma\-tsiy [Methodology of reversible generalization in context of classification of 
information transformations]. \textit{Sistemy i~Sredstva Informatiki~--- Systems and Means of 
Informatics} 28(2):128--144.
\bibitem{16-zac-1}
\Aue{Zatsman, I.} 2019. Kodirovanie kontseptov v~tsifrovoy srede [Digital encoding of concepts]. 
\textit{Informatika i~ee Primeneniya~--- Inform. Appl.} 13(4):97--106.
\bibitem{17-zac-1}
\Aue{Zatsman, I.} 2022. Model' protsessa izvlecheniya novykh terminov i~tonal'nykh slov iz tekstov 
[A~model of discovering novel terms and sentiments in texts]. \textit{Sistemy i~Sredstva Informatiki~--- 
Systems and Means of Informatics} 32(2):115--127.
\bibitem{18-zac-1}
\Aue{Tedre, M., and J.~Pajunen.} 2021. Grand theories or design guidelines? Perspectives on the role of 
theory in computing education research. \textit{ACM T. Comput. Educ.} 
doi: 10.1145/3487049.
\bibitem{19-zac-1}
\Aue{Zatsman, I.} 2020. Finding and filling lacunas in linguistic typologies. \textit{15th Forum 
(International) on Knowledge Asset Dynamics Proceedings}. Matera, Italy: Arts for Business Institute. 
780--793.
\bibitem{20-zac-1}
\Aue{Zatsman, I., and A.~Khakimova.} 2021. New knowledge discovery for creating terminological 
profiles of diseases. \textit{22nd European Conference on Knowledge Proceedings}. Reading, U.K.: 
Academic Publishing International Ltd. 837--846.
\bibitem{21-zac-1}
\Aue{Zatsman, I., O.~Zolotarev, and A.~Khakimova.} 2022. Sredovye modeli izvlecheniya iz teksta 
novykh terminov i~indikatorov nastroeniy [Medium models for discovering novel terms and sentiments 
from texts]. \textit{Informatika i~ee Primeneniya~--- Inform. Appl.} 16(2):60--67.
\bibitem{22-zac-1}
\Aue{Zatsman, I.} 2021. Problemno-orientirovannaya ak\-tu\-a\-li\-za\-tsiya slovarnykh statey dvuyazychnykh 
slovarey i~me\-ditsinskoy terminologii: so\-po\-sta\-vi\-tel'\-nyy ana\-liz [Problem-oriented updating of dictionary 
entries of bilingual \mbox{dictionaries} and medical terminology: Comparative analysis]. \textit{Informatika i~ee 
Primeneniya~--- Inform. Appl.} 15(1):94--101.
\bibitem{23-zac-1}
\Aue{Bolshina, A., and N.~Loukachevitch.} 2020. All-words word sense disambiguation for Russian 
using automatically generated text collection. \textit{Cybernetics Information Technologies} 
20(4):90--107.
\bibitem{24-zac-1}
\Aue{Ramiro, C., M.~Srinivasan, B.\,C.~Malt, and Y.~Xu.} 2018. Algorithms in the historical 
emergence of word senses. \textit{P.~Natl. Acad. Sci. USA} 
 115(10):2323--2328.
 
 \columnbreak 
 
\bibitem{25-zac-1}
\Aue{Zatsman, I.} 2014. A~table of interfaces of informatics as computer and information science. 
\textit{Scientific Technical Information Processing} 41(4):233--246.
\bibitem{26-zac-1}
\Aue{Zatsman, I.} 2012. Tracing emerging meanings by computer: Semiotic framework. \textit{13th 
European Conference on Knowledge Management Proceedings}. Reading, U.K.: Academic Publishing 
International Ltd. 2:1298--1307.
\bibitem{27-zac-1}
\Aue{Zatsman, 1.} 2012. Denotatum-based models of knowledge creation for monitoring and evaluating 
R\&D program implementation. \textit{11th IEEE Conference (International) on Cognitive Informatics 
and Cognitive Computing Proceedings}. Los Alamitos, CA: IEEE Computer Society Press. 27--34.
\end{thebibliography}

 }
 }

\end{multicols}

\vspace*{-6pt}

\hfill{\small\textit{Received July 14, 2022}}


\Contrl

\noindent
\textbf{Zatsman Igor M.} (b.\ 1952)~--- Doctor of Science in technology, head of department, Institute 
of Informatics Problems, Federal Research Center ``Computer Science and Control'' of the Russian 
Academy of Sciences, 44-2~Vavilov Str., Moscow 119333, Russian Federation; 
\mbox{izatsman@yandex.ru}

     
\label{end\stat}

\renewcommand{\bibname}{\protect\rm Литература}    