
\def\stat{shvedov}

\def\tit{КРИТЕРИЙ НЕПУСТОТЫ ЭПСИЛОН-ЯДЕР ДЛЯ~НЕЧЕТКИХ ИГР С~НЕТРАНСФЕРАБЕЛЬНОЙ 
ПОЛЕЗНОСТЬЮ И~ВЫЧИСЛИТЕЛЬНЫЕ ПРОЦЕДУРЫ}

\def\titkol{Критерий непустоты эпсилон-ядер для~нечетких НТП-игр % с~нетрансферабельной полезностью 
и~вычислительные процедуры}

\def\aut{А.\,С.~Шведов$^1$}

\def\autkol{А.\,С.~Шведов}

\titel{\tit}{\aut}{\autkol}{\titkol}

\index{Шведов А.\,С.}
\index{Shvedov A.\,S.}


%{\renewcommand{\thefootnote}{\fnsymbol{footnote}} \footnotetext[1]
%{Работа выполнена при поддержке Министерства науки и~высшего образования Российской Федерации (проект 
%075-15-2020-799).}}


\renewcommand{\thefootnote}{\arabic{footnote}}
\footnotetext[1]{Национальный исследовательский университет Высшая школа экономики, \mbox{ashvedov@hse.ru}}


\vspace*{-12pt}



\Abst{Критерий нового типа для проверки непустоты ядер кооперативных игр был 
опубликован Жао в~2001~г. Сначала необходимое и~достаточное условие было получено Жао 
для частного случая, когда полезность трансферабельна. В~этом случае, как показано Жао, 
данный критерий легко может быть использован для построения вычислительной 
процедуры, дающей ответ на вопрос, пусто или не пусто ядро игры, и~позволяющей найти 
дележи, принадлежащие ядру, если ядро не пусто. Затем критерий был обобщен Жао для игр с~нетрансферабельной 
полезностью (НТП-игр). В~настоящей работе эти результаты развиваются 
в~нескольких на\-прав\-ле\-ни\-ях. Изучается вопрос о~непустоте эп\-си\-лон-ядер~--- несколько более 
общий, чем вопрос о~непустоте ядер. Рассматриваются игры с~нечеткими выигрышами. Для 
некоторых классов НТП-игр показана воз\-мож\-ность 
построения вычислительной процедуры, сходной с~вычислительной процедурой для игр с~трансферабельной по\-лез\-ностью (ТП-игр).}
      
      \KW{нечеткое множество; кооперативная игра; эпсилон-ядро; исчерпывающее 
множество}

\DOI{10.14357/19922264220301} 
  
\vspace*{-7pt}


\vskip 10pt plus 9pt minus 6pt

\thispagestyle{headings}

\begin{multicols}{2}

\label{st\stat}
      
\section{Введение}

\vspace*{-3pt}

    Хорошо известно, что дележи, относящиеся к~ядру кооперативной игры, 
во многих случаях представляют собой оптимальное решение и~имеют большое 
практическое значение (см., например,~[1]). Напомним определение ядра НТП-иг\-ры\linebreak
    для случая, когда выигрыши четкие. 
Пусть $N\hm= \{1, \ldots , n\}$~--- множество игроков. Коалицией\linebreak называется 
любое подмножество множества~$N$. Каж\-дой коалиции~$S$ ставится 
в~соответствие допустимое множество полезностей $V(S)$, множество $V(S)$ 
принадлежит евклидову пространству, размерность которого равна числу 
игроков в~коалиции~$S$. Дележ $(x_1, \ldots ,x_n)\hm\in V(N)$ блокируется 
коалицией~$S$, если каждый из игроков, входящих в~коалицию~$S$, может 
получить больше в~случае отделения этой коалиции. Те дележи, которые не 
блокируются ни одной коалицией, относятся к~ядру игры. Ядро игры может 
быть как непустым, так и~пустым. Частный случай НТП-игр~--- 
это ТП-иг\-ры. В~игре с~трансферабельной полезностью каждое из множеств $V(S)$ состоит из таких 
наборов выигрышей, что сумма этих выигрышей не больше некоторого 
значения. Критерий непустоты ядер для ТП-игр~--- это один из классических результатов тео\-рии 
кооперативных игр (см., например,~[1]).

     В книге~[1] приводится также достаточное, но не необходимое условие 
для непустоты ядер НТП-игр. 
При этом в~\cite[с.~148]{1-sh} сказано следующее: <<Тео\-рия НТП-игр 
намного беднее результатами и~технически гораздо сложнее теории ТП-игр>>. 
Вторая часть данного утверждения в~определенной мере опровергается в~[2]. 
В~работе~[2] найдено необходимое и~достаточное условие непустоты ядер для 
НТП-игр, при этом используются относительно 
несложные математические средства. В~работе~[3] данный критерий 
пред\-став\-лен для частного случая, когда полезность трансферабельна, 
и~показано, что этот критерий приводит к~вычислительной процедуре, дающей 
ответ на вопрос, пусто или не пусто ядро игры, и~позволяющей найти дележи, 
принадлежащие ядру, если ядро не пусто. Вопрос сводится к~решению 
некоторой задачи линейного программирования. Другие два критерия 
непустоты ядер для НТП-игр получены 
в~работах~[4, 5]. Отметим, что эти критерии основаны на традиционном подходе 
изучения сбалансированности кооперативной игры и~математически 
значительно более сложные. Однако алгоритмы для применения к~конкретным 
НТП-иг\-рам в~работах~[4, 5] не даются, 
и,~видимо, данные критерии остаются результатами, относящимися 
исключительно к~теоретической математике.

Приведенное выше условие можно переформулировать так: дележ $(x_1, \ldots , 
x_n)\hm \in V(N)$ 0-бло\-ки\-ру\-ет\-ся коалицией~$S$, если каждый из игроков, 
входящих в~коалицию~$S$, может обеспечить себе потери, меньшие~0, 
в~случае отделения этой коалиции. Такая переформулировка позволяет дать 
определение $\varepsilon$-яд\-ра НТП-иг\-ры при 
любом действительном~$\varepsilon$. Во-пер\-вых, в~предыдущей фразе надо 
заменить <<0-бло\-ки\-ру\-ет\-ся>> на <<$\varepsilon$-бло\-ки\-ру\-ет\-ся>> 
и~<<потери, меньшие~0>> на <<потери, меньшие~$\varepsilon$>>. Те дележи, 
которые не $\varepsilon$-бло\-ки\-ру\-ют\-ся ни одной коалицией, относятся 
к~\mbox{$\varepsilon$-яд}\-ру игры. Ясно, что $\varepsilon$-яд\-ра будут непустыми при 
достаточно больших~$\varepsilon$ и~пустыми при достаточно 
малых~$\varepsilon$. Таким образом, все множество действительных чисел 
делится на две полупрямые. Вопрос состоит в~нахождении точки, разделяющей 
эти две полупрямые. (В~частности, вопрос о~непустоте ядра решается тем, 
принадлежит~0 левой или правой полупрямой.) Введены 
$\varepsilon$-яд\-ра в~[6], детально изучаются в~[7], а~в~\cite[с.~307]{7-sh} 
приводится пример игры, для которой $\varepsilon$-яд\-ро непустое при 
отрицательном~$\varepsilon$. Понятие $\varepsilon$-яд\-ра оказывается 
полезным в~приложениях (см., например,~[8]). Для прикладных задач важно не 
усложнять дело вопросом, существуют или не существуют объекты, которые 
предполагается использовать, а знать, что такие объекты существуют, и~уметь 
их находить. То минимальное значение~$\varepsilon$, при котором 
$\varepsilon$-яд\-ро не пусто, несет в~себе дополнительную информацию об 
игре.

     Применения теории нечетких множеств в~теории игр очень разнообразны. 
В~работе~[9] критерий из~\cite{5-sh} обобщается для ядер игр 
с~нетрансферабельной полезностью, четкими выигрышами и~нечеткими 
коалициями. Насколько известно автору, ранее игры с~нетрансферабельной 
полезностью и~с~нечеткими выигрышами не изучались, хотя, скажем, 
в~некооперативной теории игр рассмотрение игр с~нечеткими выигрышами 
широко распространено (см., например,~[10]). Моделирование выигрышей 
нечеткими числами оказывается очень удобным, если результаты игры 
известны лишь при\-бли\-женно.

    В настоящей работе критерий из~[2] обобщается для $\varepsilon$-ядер 
игр с~нетрансферабельной полезностью, нечеткими выигрышами и~четкими 
коалициями. Как и~в~работе~[2], используемые математические средства 
достаточно простые. Основное значение имеет то, что таким образом 
получается рабочий критерий, применимый для решения конкретных задач. 
В~работе [2] используется только понятие множества, исчерпывающего вниз, 
и~не используется понятие множества, исчерпывающего вверх. В~настоящей 
работе используются оба эти понятия, из-за чего доказательство становится 
значительно более прозрачным. В~разд.~2 дается описание игры. В~разд.~3 
приводится критерий непустоты $\varepsilon$-яд\-ра. В~разд.~4 даются 
примеры и~обсуждаются вычислительные процедуры. Показано, что для 
многих конкретных НТП-игр путем численного 
решения некоторой задачи математического программирования может быть дан 
ответ на вопрос, пусто или не пусто $\varepsilon$-яд\-ро, а также найдены 
дележи, принадлежащие $\varepsilon$-яд\-ру, если $\varepsilon$-яд\-ро не пусто.

\section{Описание игры}

\vspace*{-20pt}

    В дальнейшем, говоря о коалиции $S\hm\subseteq N$, будем 
предполагать, что множество~$S$ не является пус\-тым, запись $S\hm\subset N$ 
означает, что $S\not= N$. Через~$\mathbf{R}^S$ обозначим евклидово 
пространство, размерность которого равна числу элементов в~множестве~$S$. 
Под~$\mathbf{R}^N$ будем понимать обычное евклидово 
пространство~$\mathbf{R}^n$; $\mathbf{R}^S$ является 
подпространством~$\mathbf{R}^N$. Координаты векторов, входящих 
в~$\mathbf{R}^S$, имеют те же номера, которые имеют игроки из 
коалиции~$S$. Например, для $S\hm= \{2,3,7\}$ элементами 
пространства~$\mathbf{R}^S$ являются векторы $(x_2,x_3,x_7)$. Для вектора 
$x\hm\in \mathbf{R}^N$ через~$x_S$ обозначается проекция вектора~$x$ 
на~$\mathbf{R}^S$. 
    
В приложениях достаточно часто рассматриваются нечеткие множества 
с~функциями принадлежности следующего вида. Пусть~$a$ и~$b$~--- 
действительные числа, $a\hm\leq b$. Функция принадлежности равна~1 на 
полупрямой $(-\infty, a]$ и~равна~0 на полупрямой $(b,\infty)$. При $a\hm<b$ 
функция принадлежности монотонно убывает на отрезке~$[a,b]$. Если функция 
принадлежности линейна на отрезке $[a,b]$ и~убывает от~1 до~0, то нечеткое 
множество называется трапецеидальным. В~общем случае можно говорить 
о~трапецеидальном нечетком множестве с~криволинейной правой границей. 
Напомним, что $\eta$-сре\-зом нечеткого множества, где $0\hm <\eta \hm \leq 1$, 
называется подмножество, состоящее из тех элементов универсального 
множества (в~данном случае~--- множества действительных чисел), для 
которых значение функции принадлежности больше или равно~$\eta$. Ясно, 
что для описанного трапецеидального нечеткого множества $\eta$-сре\-за\-ми 
являются полупрямые~$\Pi_\eta$. При $\eta_1\hm< \eta_2$ выполняется условие 
$\Pi_{\eta_2}\hm\subseteq \Pi_{\eta_1}$. Но тогда нечеткое множество может 
быть определено путем задания полупрямых~$\Pi_\eta$ или, по-другому, путем 
задания некоторой функции аргумента~$\eta$, $\eta\hm\in [0,1]$, описывающей 
правую границу трапеции. Такой подход широко распространен. Оказывается 
(см., например,~[11]), что в~ряде случаев от условия $\Pi_{\eta_2}\hm\subseteq 
\Pi_{\eta_1}$ при $\eta_1\hm<\eta_2$ целесообразно отказаться. 
     
     Игра определена, если каждой коалиции $S\hm\subseteq N$ и~каждому 
$\eta\hm\in [0,1]$ поставлено в~соответствие множество $V_\eta(S)\hm\subseteq 
\mathbf{R}^S$. Какими свойствами эти множества должны обладать, 
уточняется ниже. При этом допустимое множество полезностей $V(S)$ может 
рассматриваться как нечеткое множество с~\mbox{$\eta$-сре}\-за\-ми~$V_\eta(S)$. Тем 
самым допускается некоторая расплывчатость, неопределенность для 
допустимого множества полезностей коалиции. Можно рассматривать задачи, 
где множество $V(S)$ известно лишь приближенно. Если $V_\eta(S)\hm = 
V_0(S)$ для любой коалиции $S\hm\subseteq N$ и~для любого $\eta\hm\in (0,1]$, 
то игра называется игрой с~четкими выигрышами.

\pagebreak
     
     Для векторов $x,y\hm\in \mathbf{R}^S$ записи $x\hm< y$ и~$x\hm\leq y$ 
означают, что соответствующие неравенства выполняются покоординатно. 
Пусть $M\hm\subseteq \mathbf{R}^S$.
     
     \smallskip
     
    \noindent
\textbf{Определение~1.} Множество $M$ называется исчерпывающим вниз, 
если из условий $x\hm\in M$ и~$y\hm\leq x$ вытекает, что $y\hm\in M$. 

     \smallskip
     
    \noindent
\textbf{Определение~2.} Множество $M$ называется исчерпывающим вверх, 
если из условий $x\hm\in M$ и~$y\hm \geq x$ вытекает, что $y\hm\in M$. 

\smallskip
    
Соотношение между этими понятиями дается следующей теоремой. 
Рассмотрим непустые замк\-ну\-тые множества~$V$ и~$G$, 
принадлежащие~$\mathbf{R}^S$. Обозначим через~$V_0$ внутренность 
множества~$V$. Будем считать, что выполняются следующие условия: 
$$
V_0\bigcap G\hm= \varnothing;\quad V_0\bigcup G\hm= \mathbf{R}^S.
$$
    
     %\smallskip
     
    \noindent
\textbf{Теорема~1.}\ \textit{Если множество}~$V$ \textit{является ис\-чер\-пы\-ва\-ющим 
вниз, то множество}~$G$ \textit{является ис\-чер\-пы\-ва\-ющим вверх}.
    

\smallskip

\noindent
Д\,о\,к\,а\,з\,а\,т\,е\,л\,ь\,с\,т\,в\,о\,.\ \  Предположим, что это не так. Тогда 
существуют~$x$ и~$y$ такие, что $x\hm\in G$, $y\hm\geq x$, $y\hm\in V_0$. 
Поскольку множество~$V_0$ открытое, это означает, что можно найти точку 
$z\hm\in V_0$ такую, что $z\hm> x$. Далее, можно найти открытую 
окрестность~$O$ точки~$x$ такую, что $z\hm>w$ для любой точки $w\hm \in 
O$. Поскольку множество~$V$ является исчерпывающим вниз, $O\hm\subseteq 
V$. Из этого следует, что $O\hm\subseteq V_0$. В~частности, это означает, что 
$x\hm\in V_0$. Но это противоречит условию $x\hm\in G$.

Теорема~1 доказана.

\smallskip

Будем предполагать, что каждое множество $V_{\eta}(S)$ является замкнутым, 
исчерпывающим вниз и~не совпадающим с~$\mathbf{R}^S$.
    
Если каждое из множеств $V_\eta(S)$ состоит из векторов, таких что 
$$
\sum\limits_{i\in S} x_i\leq v_\eta(S)\,,
$$

\vspace*{-2pt}

\noindent
где $v_\eta(S)$~--- действительное число, то говорят, что НТП-иг\-ра 
переходит в~ТП-игру.

\vspace*{-7pt}

\section{Основной результат}

\vspace*{-2pt}

    Обозначим  через $1_S$ единичный вектор 
пространства~$\mathbf{R}^S$, т.\,е.\ вектор, все координаты которого равны~1. 
В~дальнейшем, если не оговорено противное, $\varepsilon\hm\in \mathbf{R}$, 
$\eta\hm\in [0,1]$. 

        \smallskip
     
    \noindent
\textbf{Определение~3.} Вектор $x\hm\in \mathbf{R}^N$ 
$(\varepsilon,\eta)$-бло\-ки\-ру\-ет\-ся коалицией~$S$, если существует вектор 
$x\hm\in V_\eta(S)$ такой, что $x\hm> z_S\hm+ \varepsilon 1_S$.

     \smallskip

Обозначим через $\partial V_\eta(N)$ границу множества~$V_\eta(N)$.

     \smallskip
     
    \noindent
\textbf{Определение~4.} $(\varepsilon,\eta)$-яд\-ром называется множество 
векторов $z\hm\in \partial V_\eta(N)$ таких, что не существует коалиции~$S$, 
которой вектор~$z$ $(\varepsilon,\eta)$-бло\-ки\-ру\-ется. 
    
Пусть $\varepsilon_1<\varepsilon_2$. Если вектор~$z$  
$(\varepsilon_2,\eta)$-бло\-ки\-ру\-ет\-ся коалицией~$S$, то данный вектор 
и~$(\varepsilon_1,\eta)$-бло\-ки\-ру\-ет\-ся коалицией~$S$. Действительно, существует $x\hm\in 
V_\eta(S)$ такой, что $x\hm > z_S\hm+ \varepsilon_2 1_S$. Отсюда вытекает, что 
$x\hm> z_S\hm+ \varepsilon_1 1_S$. Если вектор~$z$ принадлежит  
$(\varepsilon_1,\eta)$-яд\-ру, то вектор~$z$ принадлежит  
и~$(\varepsilon_2,\eta)$-яд\-ру. Действительно, если вектор~$z$ не принадлежит 
$(\varepsilon_2,\eta)$-яд\-ру, то вектор~$z$  
$(\varepsilon_2,\eta)$-бло\-ки\-ру\-ет\-ся некоторой коалицией~$S$. Тогда 
вектор~$z$ $(\varepsilon_1,\eta)$-бло\-ки\-ру\-ет\-ся коалицией~$S$. Но это 
означает, что вектор~$z$ не может принадлежать $(\varepsilon_1,\eta)$-ядру.

     \smallskip
     
    \noindent
\textbf{Определение~5.} Вектор $z\hm\in \mathbf{R}^N$ называется 
$(\varepsilon,\eta)$-до\-пус\-ти\-мым для коалиции~$S$, если он не  
$(\varepsilon,\eta)$-бло\-ки\-ру\-ет\-ся этой коалицией.
    
Обозначим через $H_{\varepsilon,\eta}(S)$~--- множество векторов $z\hm\in 
\mathbf{R}^N$, которые являются  
$(\varepsilon,\eta)$-до\-пус\-ти\-мы\-ми для коалиции~$S$. Очевидно, что 
$H_{\varepsilon,\eta}(S)$ является цилиндрическим множеством с~основанием 
$G_{\varepsilon,\eta}(S)$, где $ G_{\varepsilon,\eta}(S) \hm\subseteq 
\mathbf{R}^S$~--- непустое множество. Пусть
$$
A_{\varepsilon,\eta} = \mathop{\bigcap}\limits_{S\subset N} H_{\varepsilon,\eta} (S)\,.
$$
Очевидно, что $(\varepsilon,\eta)$-яд\-ро~--- это $A_{\varepsilon,\eta} \cap 
\partial V_\eta(N)$.
    
     \smallskip
     
    \noindent
\textbf{Теорема~2.} \textit{При любых $\varepsilon\hm\in \mathbf{R}$,
$\eta\hm\in [0,1]$ и~$S\hm\subset N$ множество $G_{\varepsilon,\eta}(S)$ 
замкнутое}.
    

\smallskip

\noindent
Д\,о\,к\,а\,з\,а\,т\,е\,л\,ь\,с\,т\,в\,о\,.\ \  Предположим, что это не так. Тогда 
существуют последовательность векторов $y_k\hm\in \mathbf{R}^S$ и~вектор 
$y_0\hm\in \mathbf{R}^S$ такие, что $y_k\hm\to y_0$ при $k\hm\to \infty$, 
$y_k\hm\in G_{\varepsilon,\eta}(S)$ при любом натуральном~$k$, $y_0\hm\not\in 
G_{\varepsilon,\eta}(S)$. Последнее условие означает, что существует $x\hm\in 
V_\eta(S)$ такой, что $x\hm> y_0\hm+ \varepsilon 1_S$. Но тогда при достаточно 
больших~$k$ должны выполняться неравенства $x\hm> y_k\hm+ \varepsilon 
1_S$, что является противоречием.

    Теорема~2 доказана.

\smallskip

     
    \noindent
\textbf{Теорема~3.} \textit{При любых $\varepsilon\hm\in \mathbf{R}$, 
$\eta\hm\in [0,1]$ и~$S\hm\subset N$ множество $G_{\varepsilon,\eta}(S)$ 
является исчерпывающим вверх}.

\smallskip

\noindent
Д\,о\,к\,а\,з\,а\,т\,е\,л\,ь\,с\,т\,в\,о\,.\ \. Предположим, что это не так. Тогда 
существуют векторы~$y$ и~$z$ из~$\mathbf{R}^S$ такие, что $z\hm\in 
G_{\varepsilon,\eta}(S)$, $y\hm\geq z$ и~$y\not\in G_{\varepsilon,\eta}(S)$. 
Последнее условие означает, что существует $x\hm\in V_\eta(S)$ такой, что 
$x\hm> y\hm+ \varepsilon 1_S$. Очевидно, отсюда следует, что $x\hm> z\hm+ 
\varepsilon 1_S$. Но это является противоречием.

    Теорема~3 доказана. 

\smallskip

    Основным результатом работы является следующая теорема.

     \smallskip
     
    \noindent
\textbf{Теорема~4.} \textit{Множество $A_{\varepsilon,\eta}\cap \partial 
V_\eta(N)$ непустое тогда и~только тогда, когда существуют $x\hm\in \partial 
V_\eta(N)$ и~$y\hm\in A_{\varepsilon,\eta}$ такие, что $x\hm\geq y$}.

\smallskip

\noindent
Д\,о\,к\,а\,з\,а\,т\,е\,л\,ь\,с\,т\,в\,о\,.\ \  Пусть выполняется первое условие. Тогда 
существует~$z$ такой, что $z\hm\in \partial V_\eta(N)$ и~$z\hm\in 
A_{\varepsilon,\eta}$. Очевидно, что $z\hm\geq z$.

    Пусть выполняется второе условие. В~силу тео\-ре\-мы~3 множество $ 
A_{\varepsilon,\eta}$ является исчерпывающим вверх, поэтому из условия 
$x\hm\geq y$ вытекает, что $x\hm\in A_{\varepsilon,\eta}$. Следовательно, 
$x\hm\in A_{\varepsilon,\eta} \cap \partial V_\eta(N)$. 

    Теорема~4 доказана.

\smallskip

    Для случая $\varepsilon\hm=0$ и~игр с~четкими выигрышами теорема~4 
доказана в~[2]. Отметим, что в~[2] накладываются некоторые условия на 
множества $V(S)$, которые фактически не нужны.

\vspace*{-7pt} 

\section{Примеры}

\vspace*{-2pt}

     \textbf{Пример~1.} Рассмотрим ТП-игру, 
$n\hm=3$. Пусть $\eta$ фиксировано. Тогда

\vspace*{-2pt}

\noindent
     \begin{multline*}
     \partial V_\eta(N) ={}\\
     {}=\left\{ (x_1, x_2, x_3): 
x_1+x_2+x_3=\nu_\eta(\{1,2,3\})\right\}.
     \end{multline*}
     
     \vspace*{-2pt}
     
     \noindent
При любом $\varepsilon\in \mathbf{R}$ имеем

\noindent
\begin{align*}
G_{\varepsilon,\eta}(\{1\}) &= \left\{ x_1: x_1\geq \nu_\eta(\{1\})-
\varepsilon\right\};\\
G_{\varepsilon,\eta}(\{2\}) &=\left\{ x_2: x_2\geq \nu_\eta(\{2\})-
\varepsilon\right\};\\
G_{\varepsilon,\eta}(\{3\}) &= \left\{ x_3: x_3\geq \nu_\eta(\{3\})-
\varepsilon\right\};\\
G_{\varepsilon,\eta}(\{1,2\}) &={}\\
&\hspace*{-5mm}{}= \left\{ (x_1,x_2): x_1+x_2\geq \nu_\eta(\{1,2\}) -
2\varepsilon\right\};\\
G_{\varepsilon,\eta}(\{1,3\}) &={}\\
&\hspace*{-5mm}{}= \left\{ (x_1,x_3): x_1+x_3\geq \nu_\eta(\{1,3\}) -
2\varepsilon\right\};\\
G_{\varepsilon,\eta}(\{2,3\}) &= {}\\
&\hspace*{-5mm}{}=\left\{ (x_2,x_3): x_2+x_3\geq \nu_\eta(\{2,3\}) -
2\varepsilon\right\}.
\end{align*}

\vspace*{-2pt}

\noindent
Рассмотрим задачу линейного программирования 
$$
x_1+x_2+x_3\to \min
$$
при ограничениях
$$
(x_1, x_2, x_3) \in\mathop{\bigcap}\limits_{S\subset N} H_{\varepsilon,\eta} (S)\,,
$$

\vspace*{-2pt}

\noindent
где $H_{\varepsilon,\eta}(\{1\}) \hm= \{ (x_1,x_2,x_3): x_1\hm\in 
G_{\varepsilon,\eta}(\{1\})\}$ и~т.\,д. Если значение целевой функции в~точке 
минимума не превосходит $\nu_\eta(N)$, то выполняется второе условие 
теоремы~4 и~$(\varepsilon,\eta)$-яд\-ро непустое. Если значение целевой 
функции в~точке минимума больше, чем $\nu_\eta(N)$, то второе условие 
теоремы~4 не выполняется и~$(\varepsilon,\eta)$-яд\-ро пустое. Таким образом, 
для ответа на вопрос, пусто или не пусто $(\varepsilon,\eta)$-яд\-ро, достаточно 
решить обычную задачу линейного программирования. 
    
Отметим, что все результаты работы остаются верными (с~очевидными 
изменениями), если считать, что каждое из множеств~$V_\eta(S)$ принадлежит 
неотрицательному ортанту~$\mathbf{R}_+^S$. В~следующем примере для 
наглядности ограничимся этим случаем.

\columnbreak
    
\textbf{Пример~2.} Пусть $\varepsilon\hm=0$, значение~$\eta$ фиксировано. 
Предположим, что существуют функции $f_S:$ $\mathbf{R}_+^S\hm\to 
\mathbf{R}$ такие, что для каждой коалиции~$S$ 

\noindent
$$
V_\eta(S)= \left\{ x\in \mathbf{R}_+^S: f_S(x)\leq a_S\right\}\,,
$$

\vspace*{-2pt}

\noindent
где $a_S$~--- некоторое действительное число. При этом функции~$f_S$ 
являются непрерывными и~монотонно возрастающими по каждой координате 
при любых фиксированных значениях остальных координат. Тогда теорема~4 
показывает, что для ответа на вопрос, пусто или не пусто ядро игры, 
достаточно решить задачу математического программирования

\vspace*{2pt}

\noindent
$$
f_N(x)\to \min
$$

\vspace*{-2pt}

\noindent
при ограничениях

\noindent
$$
f_S(x_S)\geq a_S
$$

\vspace*{-2pt}

\noindent
для всех $S\hm\subset N$. Пусть $f^*$~--- значение целевой функции в~точке 
минимума. Тогда ядро игры непустое при $f^*\hm\leq a_N$ и~пустое 
в~противоположном случае. Трудностью при численном решении приведенной 
задачи математического программирования может оказаться то, что число 
ограничений быстро рас\-тет при увеличении~$n$. В~частности, могут 
рассматриваться функции $f_S(x)\hm= \sum\nolimits_{i\in S} x_i$ при всех 
$S\hm\subseteq N$, тогда это ТП-иг\-ра. Также 
методический интерес представляет случай $f_S(x)\hm= \sum\nolimits_{i\in S} 
c_{iS} x_i^2$, где все числа $c_{iS}$ положительные. По-ви\-ди\-мо\-му, 
данный подход является единственным, который позволяет в~зависимости от 
значений $c_{iS}$ и~$a_S$ дать ответ на вопрос, пусто или не пусто ядро игры, 
а~также найти дележи, принадлежащие ядру, если ядро не пусто.

\vspace*{-10pt}

\section{Заключение}

\vspace*{-2pt}

    Множество $A_{\varepsilon,\eta}\cap \partial V_\eta(N)$ можно 
рассматривать как $\eta$-срез некоторого нечеткого множества. Изучение 
свойств этих нечетких множеств представляет собой предмет будущих 
исследований.

\vspace*{-10pt}

{\small\frenchspacing
 {\baselineskip=10.5pt
 %\addcontentsline{toc}{section}{References}
 \begin{thebibliography}{99}
 
 \vspace*{-2pt}
 
      \bibitem{1-sh}
      \Au{Мулен Э.} Кооперативное принятие решений: Аксиомы и~модели~/ Пер. 
 с~англ.~--- М.: Мир, 1991. 464~с. (\Au{Moulin~H.} Axioms of cooperative decision making.~--- 
Cambridge: Cambridge University Press, 1988. 332~p.)
      \bibitem{2-sh}
      \Au{Zhao J.} New conditions for core existence in coalitional NTU games.~--- Ames, IA, 
USA: Department of Economics Iowa State University, 2001. 16~p.
      \bibitem{3-sh}
      \Au{Zhao J.} The relative interior of the base polyhedron and the core~// Econ. Theory, 
2001. Vol.~18. P.~635--648.
      \bibitem{4-sh}
      \Au{Keiding H., Thorlund-Petersen~L.} The core of a cooperative game without side 
payments~// J.~Optimiz. Theory App., 1987. Vol.~54. P.~273--288.



      \bibitem{5-sh}
      \Au{Predtetchinski~A., Herings~P.\,J.-J.} A~necessary and sufficient condition for  
non-emptiness of the core of a~non-\linebreak\vspace*{-12pt}

\pagebreak

\noindent
transferable utility game~// J.~Econ. Theory, 2004. 
Vol.~116. P.~84--92.
      \bibitem{6-sh}
      \Au{Shapley L.,S., Shubik~M.} Quasi-cores in a monetary economy with nonconvex 
preferences~// Econometrica, 1966. Vol.~34. P.~805--827.
      \bibitem{7-sh}
      \Au{Maschler M., Peleg~B., Shapley~L.\,S.} Geometric properties of the kernel, nucleolus, 
and related solution concepts~// Math.  Oper. Res., 1979. Vol.~4. P.~303--338.
      \bibitem{8-sh}
      \Au{Mochaourab R., Jorswieck~E.} Coalitional games in MISO interference channels: 
Epsilon-core and coalition structure stable set~// IEEE T. Signal Proces., 2014. Vol.~62. 
P.~6507--6520.
      \bibitem{9-sh}
      \Au{Liu J., Liu~X.} A~necessary and sufficient condition for an NTU fuzzy game to have 
a~non-empty fuzzy core~// J.~Math. Econ., 2013. Vol.~49. P.~150--156.
      \bibitem{10-sh}
      \Au{Larbani~M.} Non cooperative fuzzy games in normal form: A~survey~// Fuzzy Set. 
Syst., 2009. Vol.~160. P.~3184--3210.
      \bibitem{11-sh}
      \Au{Shvedov A.\,S.} Instrumental variables estimation of fuzzy regression models~// 
J.~Intell. Fuzzy Syst., 2019. Vol.~36. P.~5457--5462.

\end{thebibliography}

 }
 }

\end{multicols}

\vspace*{-9pt}

\hfill{\small\textit{Поступила в~редакцию 04.07.21}}

\vspace*{6pt}

%\pagebreak

%\newpage

%\vspace*{-28pt}

\hrule

\vspace*{2pt}

\hrule

\vspace*{-4pt}

\def\tit{A CONDITION FOR NON-EMPTINESS OF~THE~EPSILON-CORE  
OF~A~NONTRANSFERABLE UTILITY FUZZY GAME AND~COMPUTATIONAL 
SCHEMES}


\def\titkol{A condition for non-emptiness of~the~epsilon-core  
of~a~nontransferable utility fuzzy game and~computational 
schemes}


\def\aut{A.\,S.~Shvedov}

\def\autkol{A.\,S.~Shvedov}

\titel{\tit}{\aut}{\autkol}{\titkol}

\vspace*{-15pt}


%\noindent
%National Research University Higher School of Economics, 34~Tallinskaya Str., Moscow 123458, 
%Russian Federation

\noindent
National Research University Higher School of Economics, 20~Myasnitskaya Str., Moscow 101000, Russian Federation


\def\leftfootline{\small{\textbf{\thepage}
\hfill INFORMATIKA I EE PRIMENENIYA~--- INFORMATICS AND
APPLICATIONS\ \ \ 2022\ \ \ volume~16\ \ \ issue\ 3}
}%
 \def\rightfootline{\small{INFORMATIKA I EE PRIMENENIYA~---
INFORMATICS AND APPLICATIONS\ \ \ 2022\ \ \ volume~16\ \ \ issue\ 3
\hfill \textbf{\thepage}}}

\vspace*{2pt} 
      
 




\Abste{Zhao (2001) suggested a new condition for non-emptiness of the core of a cooperative 
game. At first, a~necessary and sufficient condition was found by Zhao for a particular case of 
games with transferable utility. In this case, as was shown by Zhao, the condition can be easily used 
for construction of a computational scheme which decides if the core of a game is empty or 
non-empty and gives imputations belonging to the core if the core is non-empty. Then, the condition 
was generalized by Zhao for games with nontransferable utility. In this paper, the results are 
generalized in few directions. The problem of non-emptiness of the epsilon-core which is 
somewhat more general than the problem of non-emptiness of the core is studied. Games with 
fuzzy payoffs are considered. For some classes of games with nontransferable utility, possibility of 
construction of a~computational scheme which is similar to the computational scheme for games 
with transferable utility is established.}

\KWE{fuzzy set; cooperative game; epsilon-core; comprehensive set}

\DOI{10.14357/19922264220301} 

%\vspace*{-16pt}

%\Ack
%\noindent




\vspace*{-4pt}

  \begin{multicols}{2}

\renewcommand{\bibname}{\protect\rmfamily References}
%\renewcommand{\bibname}{\large\protect\rm References}

{\small\frenchspacing
 {\baselineskip=10.6pt
 \addcontentsline{toc}{section}{References}
 \begin{thebibliography}{99}
      \bibitem{1-sh-1}
\Aue{Moulin, H.} 1988. \textit{Axioms of cooperative decision making}. Cambridge: Cambridge 
University Press. 332~p.
      \bibitem{2-sh-1}
\Aue{Zhao, J.} 2001. \textit{New conditions for core existence in coalitional NTU games}. Ames, 
IA: Department of Economics Iowa State University. 16~p.
      \bibitem{3-sh-1}
\Aue{Zhao, J.} 2001. The relative interior of the base polyhedron and the core. \textit{Econ. 
Theory} 18:635--648.
      \bibitem{4-sh-1}
\Aue{Keiding, H., and L.~Thorlund-Petersen.} 1987. The core of a cooperative game without side 
payments.  \textit{J.~Optimiz. Theory App.} 54:273--288.
      \bibitem{5-sh-1}
\Aue{Predtetchinski, A., and P.\,J.-J.~Herings.} 2004. A~necessary and sufficient condition for 
non-emptiness of the core of a non-transferable utility game. \textit{J.~Econ. Theory} 116: 84--92.
      \bibitem{6-sh-1}
\Aue{Shapley, L.\,S., and M.~Shubik.} 1966. Quasi-cores in a~monetary economy with nonconvex 
preferences. \textit{Econometrica} 34:805--827.
      \bibitem{7-sh-1}
\Aue{Maschler, M., B.~Peleg, and L.\,S.~Shapley.} 1979. Geometric properties of the kernel, 
nucleolus, and related solution concepts. \textit{Math. Oper. Res.} 4:303--338.
      \bibitem{8-sh-1}
\Aue{Mochaourab, R., and E.~Jorswieck.} 2014. Coalitional games in MISO interference channels: 
epsilon-core and coalition structure stable set. \textit{IEEE T. Signal Proces.}  
62:6507--6520.
      \bibitem{9-sh-1}
\Aue{Liu, J., and X.~Liu.} 2013. A~necessary and sufficient condition for an NTU fuzzy game to 
have a non-empty fuzzy core. \textit{J.~Math. Econ.} 49:150--156.
      \bibitem{10-sh-1}
      \Aue{Larbani, M.} 2009. Non cooperative fuzzy games in normal form: A~survey. 
\textit{Fuzzy Set. Syst.} 160:3184--3210.
      \bibitem{11-sh-1}
\Aue{Shvedov, A.\,S.} 2019. Instrumental variables estimation of fuzzy regression models. 
\textit{J.~Intell. Fuzzy Syst.} 36:5457--5462.

 \end{thebibliography}

 }
 }

\end{multicols}

\vspace*{-9pt}

\hfill{\small\textit{Received July 4, 2021}}    

\vspace*{-18pt}

\Contrl

\vspace*{-3pt}

\noindent
\textbf{Shvedov Alexey S.} (b.\ 1956)~--- Doctor of Science in physics and mathematics, 
professor, Faculty of Economics, National Research University Higher School of Economics, 
National Research University Higher School of Economics, 20~Myasnitskaya Str., Moscow 101000, Russian Federation;
%34~Tallinskaya Str., Moscow 123458, Russian Federation; 
\mbox{ashvedov@hse.ru}
 



\label{end\stat}

\renewcommand{\bibname}{\protect\rm Литература}    