\def\stat{nuriev}

\def\tit{ПЕРЕВОДЧЕСКИЙ АНАЛИЗ ТЕКСТА С~ПРИМЕНЕНИЕМ ИНФОРМАЦИОННЫХ 
РЕСУРСОВ: РЕДУЦИРОВАНИЕ СПЕКТРА МОДЕЛЕЙ ПЕРЕВОДА 
В~НАДКОРПУСНЫХ БАЗАХ ДАННЫХ$^*$}

\def\titkol{Переводческий анализ текста с~применением информационных 
ресурсов: редуцирование спектра моделей перевода} 
%в~надкорпусных базах данных}

\def\aut{В.\,А.~Нуриев$^1$}

\def\autkol{В.\,А.~Нуриев}

\titel{\tit}{\aut}{\autkol}{\titkol}

\index{Нуриев В.\,А.}
\index{Nuriev V.\,A.}


{\renewcommand{\thefootnote}{\fnsymbol{footnote}} \footnotetext[1]
{Работа выполнена в~Федеральном исследовательском центре <<Информатика и~управление>> Российской 
академии наук с~использованием ЦКП <<Информатика>> ФИЦ ИУ РАН.}}


\renewcommand{\thefootnote}{\arabic{footnote}}
\footnotetext[1]{Федеральный исследовательский центр <<Информатика и~управление>> Российской академии наук,
\mbox{nurieff.v@gmail.com}}


\vspace*{-4pt}

    
    
    
  \Abst{Уточняется подход к~редуцированию спектра моделей перевода в~надкорпусных 
базах данных (НБД). Надкорпусные базы данных~--- это информационный ресурс широкого профиля: они 
обеспечивают решение задач, пред\-став\-ля\-ющих интерес для информатики, компьютерной 
лингвистики, медицины и~т.\,д. В~данной статье НБД рассматриваются с~точки зрения 
корпусного переводоведения. Показано, как этот автоматизированный инструмент 
может применяться для <<пристального и~дальнего чтения>>~--- подхода, вы\-сту\-па\-юще\-го 
за внедрение современных информационных ресурсов в~деятельность литературного 
переводчика. Особое внимание уделено возможностям использования НБД для 
редуцирования спектра моделей перевода. При переводе ввиду синонимического 
потенциала естественных языков часто имеется некоторое множество относительно 
взаимозаменяемых по смыслу эквивалентов (слов, словосочетаний, синтаксических 
конструкций и~т.\,д.). Выбирая один конечный эквивалент, переводчик стремится 
максимально сузить диапазон своего поиска. Цель статьи~--- уточнить подход 
к~применению НБД, позволяющий сузить диапазон поиска релевантной модели перевода.}
  
  \KW{корпусное переводоведение; цифровые гуманитарные науки; компьютерный 
анализ текста; дальнее чтение; параллельные тексты; перевод; модели перевода; 
надкорпусная база данных; многовариантный выбор}

\DOI{10.14357/19922264220309} 
  
\vspace*{-3pt}


\vskip 10pt plus 9pt minus 6pt

\thispagestyle{headings}

\begin{multicols}{2}

\label{st\stat}
  
  \section{Введение}

  \vspace*{-2pt}
  
  Взаимодействие информатики и~языкознания интенсифицировалось 
в~последние несколько десятилетий и~привело к~оформлению целого 
научного направления, получившего название <<\mbox{corpus-based} translation 
studies>> (корпусное переводоведение)~[1--5]. Оно занимается проблемой 
корпусных информационных ресурсов и~возможностями их внедрения 
в~переводческий процесс.
  
  Электронные текстовые корпусы и~базы данных, где хранится информация 
о естественных языках, призваны изменить подходы к~переводу, возникшие 
в~докорпусную эпоху (см., например,~[6]). Новые информационные ресурсы 
открывают иные возможности для переводчиков (исследователей, писателей, 
преподавателей и~т.\,д.), помогая изучать текст-объ\-ект в~большем 
многообразии его уникальных характеристик\footnote[2]{О характеристиках 
текста, подлежащего переводу, а~также о~многообразии факторов, 
определяющих качество переводного текста, см., например,~[7, с.~96--169].} 
путем применения компьютеризированных инструментов.
  
  Один из таких подходов предлагает английский ученый Р.~Юдейл~[8]. 
В~нем сочетается традиционное присталь\-ное/вни\-ма\-тель\-ное чтение 
(close reading) с~квантитативным анализом, куда вовлечены компьютерные 
инструменты и~который обозначается как дальнее чтение (distant 
reading)\footnote[3]{Термин <<distant reading>> заимствован у Ф.~Моретти; перевод термина как
<<дальнее чтение>> предложен в~русскоязычной версии книги~\cite{9-n}.}. Подход получил название <<пристальное 
и~дальнее чтение>> (close and distant reading approach). Он не предполагает 
применения ка\-ких-ли\-бо средств автоматизации перевода, пытаясь при 
этом привлечь внимание к~возможностям внедрения современных 
информационных технологий в~практику художественного перевода.
  
  Внимательное чтение понимается в~привычном ключе: оно является 
непременным условием художественного перевода, выступая основой для 
стилистической оценки текста и~интерпретации его коммуникативной 
интенции. Глубина внимания при чтении зависит от опыта переводчика, 
накопленного им знания о языках и~культурах.
  
  Вместе с~тем в~художественных текстах большой протяженности имеются 
языковые элементы, необходимые для понимания текста и~его организации, 
но труднодоступные для отслеживания человеком <<вручную>>: паттерны 
использования высокочастотных служебных слов, средняя величина 
предложения, принципы кластеризации слов и~конструкций, повторяющиеся 
союзы и~т.\,д. Дальнее чтение, сконцентрированное не на тексте целиком, а 
на конкретных его структурных параметрах и~элементах, с~помощью 
современных автоматизированных инструментов\footnote[1]{О преимуществах 
современного тестового анализа с~применением компьютеризированных 
инструментов см. высказывание пионера в~области цифровых гуманитарных 
наук Р.~Бузы~[10, p.~xxvi].} способно обеспечить их анализ на фоне 
большого числа других текстов. Тем самым осуществляется <<отдаление>> 
переводчика от отдельно взятого текста. Оно выводит на подкрепленные 
количественными данными обобщения, помогая сделать заключения об 
устройстве текста, подлежащего переводу, и~о~том, каким должен быть 
конечный переводной текст: <<Количественные данные могут указать на 
языковые феномены, о которых читатель не задумывается>>~[11, с.~389].
  
  Примером информационного инструмента, помогающего осуществлять 
дальнее чтение, служат НБД (подробнее см.~[12]), функционирующие на 
основе параллельных корпусов Национального корпуса русского языка 
(НКРЯ). Они разрабатываются с~2012~г.\ в~Институте проблем информатики 
ФИЦ ИУ РАН~[13].
  
Данная статья продолжает серию исследований о возможностях применения НБД в~переводной 
деятельности. Ее основная цель~--- уточнить подход к~редуцированию спектра моделей 
перевода\footnote[2]{О моделях перевода см. подробнее разд.~3.}, которые регистрируются и~хранятся в~НБД 
и~которые в~своей деятельности применяет переводчик. Эти возможности иллюстрируются на примере НБД 
коннекторов\footnote[3]{О НБД коннекторов см.~\cite{12-n, 14-n}.} (см.\ разд.~3). Но прежде чем перейти 
к~уточнению указанного подхода, следует уделить внимание некоторым свойствам НБД и~дать краткое 
описание ее структуры (см.\ разд.~2).

    \section{Надкорпусная база данных: свойства и~структурная схема}
    
Наряду с~лексикографическими изданиями и~контрастивными грамматиками спектр моделей для перевода 
разных языковых единиц может регистрироваться и~аннотироваться (описываться в~соответствии 
с~определенным набором рубрик) в~электронных базах данных, например в~НБД 
коннекторов\footnote[4]{Коннектор~--- языковая единица, функция которой состоит в~выражении  
логико-семантических отношений (ЛСО) между двумя соединенными с~ее помощью текстовыми 
фрагментами (подробнее см.~\cite{16-n, 17-n, 18-n}).}.
  
  Надкорпусная база данных функционирует на основе параллельных текстов, и~в ней можно 
интерактивно осуществить обратный переход от модели перевода 
к~примерам ее использования. Для этого в~НБД каждый пример перевода 
связан с~соответствующей динамически сформированной моделью перевода, 
которая имеет гипертекстовую ссылку на список всех примеров ее 
реализации.
  
  Аннотации изучаемых языковых единиц и~их переводных соответствий 
наполняют надкорпусную часть НБД, предназначенную для формирования, 
хранения и~поиска аннотаций. Они интегрированы с~параллельными 
текстами, которые содержатся в~корпусной части НБД, связанной 
с~надкорпусной. Выделяются два основных класса информационных объектов, к~которым обеспечивает доступ НБД:
  \begin{enumerate}[(1)]
\item аннотационные рубрики, аннотации, фрагменты параллельных текстов и~другие данные, хранящиеся в~таблицах НБД;
  \item генерализуемые объекты (содержащиеся в~НБД модели перевода 
коннекторов, сгруппированные по тому или иному параметру).
  \end{enumerate}
  
  Два основных класса информационных объектов обусловливают два типа 
многоуровневых связей в~НБД:
  \begin{enumerate}[(1)]
  \item связи, задаваемые структурой НБД, т.\,е.\ связи между таблицами баз 
данных SQL-сер\-вера;
  \item вычисляемые связи, обеспечивающие многоуровневые переходы от 
аннотаций к~генерализуемым объектам и~обратно.
  \end{enumerate}
  
  
  Подробнее о НБД и~ее структурной схеме см.~\cite{15-n}.
  
  \section{Редуцирование спектра моделей перевода (на материале 
коннектора <<si>> и~его русских эквивалентов)}
  
  В данном разделе предлагается развитие подхода к~редуцированию спектра 
моделей перевода в~НБД. В~основе этого подхода лежит представление о 
переводе как о ситуации многовариантного выбора, где переводчик 
постоянно оперирует некоторым набором относительно равноценных 
языковых альтернатив для оформления переводного варианта. При 
продуцировании конечного текста переводчику нужно добиться максимально 
возможного редуцирования спектра моделей перевода, в~идеале сузив его до 
единичного соответствия. Формирование такого подхода предполагает ответ 
на два главных вопроса:
  \begin{enumerate}[1.]
  \item Насколько детальной в~НБД должна быть классификационная 
система рубрик-признаков при аннотировании переводимого текстового 
фрагмента, чтобы максимально возможно сузить множество альтернатив при 
выборе конечной модели перевода?
  \item Можно ли сформировать универсальный набор рубрик, позволяющий 
редуцировать модели перевода разных языковых единиц, или система 
руб\-рик-при\-зна\-ков для каждого класса языковых единиц должна проходить 
специализацию и~формироваться таргетированно?
  \end{enumerate}
  
Ранее эмпирические данные, на основе которых проводилась 
экспериментальная часть исследования, включали в~себя хранящиеся в~НБД 
коннекторов аннотации параллельных текстовых \mbox{фрагментов}, содержащих 
один из самых частотных\footnote{В НБД для таких коннекторов зафиксировано наибольшее 
число аннотаций.} французских коннекторов <<c'est-$\grave{\mbox{a}}$-dire>> и~его 
переводные соответствия (204~аннотации). Источник текстового  
материала~--- фран\-цуз\-ско-рус\-ский параллельный 
подкорпус\footnote{Аннотации коннекторов в~НБД формируются посредством обработки 
параллельных текстов, включающих исходные текс\-ты (ИТ) и~их переводы. Для настоящего эксперимента 
привлекались переводы, официально опубликованные, выполненные 
 че\-ло\-ве\-ком-про\-фес\-си\-о\-на\-лом.} НКРЯ. Объем фран\-цуз\-ско-рус\-ских 
параллельных текстов~--- 2\,184\,492~словоупотреблений (=\;то\-ке\-нов). 
Анализ этих эмпирических данных позволил сделать ряд 
выводов~\cite[с.~122--123]{19-n}. На данном этапе они подлежат 
верификации и~уточнению. Для этого привлекаются\linebreak хранящиеся в~НБД 
коннекторов аннотации параллельных текстовых фрагментов, содержащих 
другой частотный французский коннектор~--- <<si>> и~его русские 
эквиваленты (всего 255~аннотации). \mbox{Употребление} <<si>> характеризуется 
относительной равномерностью: он реализуется в~разножанровых текстах, 
принадлежащих разным авторам (О.~де~Баль\-зак, Ф.~Бегбедер, Ж.~Верн, 
Р.~Госинни, Ж.~Женетт, Ж.~Кокто, П.~Модиано, Г.~де~Мо\-пас\-сан, 
А.~де~Сент-Эк\-зю\-пе\-ри). Для <<si>> зафиксировано большое 
многообразие моделей перевода (табл.~1, №\,34), этот коннектор может 
сигнализировать о~разных типах ЛСО (услов\-ные\,/\,услов\-ные 
с~ограничительным оттенком, отношения пропозициональной причины, 
временн$\acute{\mbox{ы}}$е, уступительные, сопоставительные).

  На основе аннотаций в~НБД для исследуемого коннектора был 
сформирован экспериментальный массив моделей перевода в~виде кортежей 
$\langle$коннектор; вариант перевода$\rangle$. Для наиболее продуктивной 
модели перевода кортеж имеет вид $\langle$si; если$\rangle$ (см.\ все 
варианты перевода в~табл.~1). Генерализованная модель перевода имеет вид 
$\langle$si; $\{$вариант перевода$\}$$\rangle$, где $\{$вариант перевода$\}$ 
обозначает весь спектр таких вариантов во втором столбце табл.~1.

\setcounter{table}{1}
\begin{table*}\small
\begin{center}
\Caption{Число аннотаций с~моделями перевода для <<si>> (первые два столбца) при 
указанном сочетании рубрик (третий столбец)}
\vspace*{2ex}

\tabcolsep=7.6pt
\begin{tabular}{|c|c|c|c|}
\hline
  ИТ&Вариант перевода&Сочетание рубрик&Число аннотаций\\
  \hline
  si&если&CNT q p\;+\;начальная\;+\;с~предикацией\;+\;условные&57\\
  &если$\|$то&CNT q p\;+\;начальная\;+\;с~предикацией\;+\;условные&35\\
  &если&p CNT q\;+\;начальная\;+\;с~предикацией\;+\;условные&19\\
  &если&CNT q p\;+\;SubCNT\;+\;начальная\;+\;с~предикацией\;+\;условные&18\\
  &если бы&CNT q p\;+\;начальная\;+\;с~предикацией\;+\;условные&10\\
  &---&Другие сочетания рубрик&116\hphantom{9}\\
  \hline
 \multicolumn{3}{|c|}{Итого:}&255\hphantom{9}\\
  \hline
  \end{tabular}
  \end{center}
  \end{table*}

  
  Затем было определено число аннотаций с~различными сочетаниями 
рубрик, использованных для аннотирования текстовых фрагментов, где 
употреблен коннектор (табл.~2). Таблица~2 показы-\linebreak\vspace*{-12pt}

%\begin{table*}

\begin{center}
\vspace*{-3pt}

\parbox{74mm}{{{\tablename~1}\ \ \small{
Варианты перевода французского коннектора <<si>>
}}
}

\vspace*{6pt}

{\small %tabl1
\begin{tabular}{|c|c|c|}
\hline
  №&\tabcolsep=0pt\begin{tabular}{c}Вариант\\ перевода\end{tabular}&\tabcolsep=0pt\begin{tabular}{c}Число\\ аннотаций\end{tabular}\\
  \hline
  \hphantom{9}1&если&123\hphantom{99}\\
  \hphantom{9}2&если$\|$то&56\hphantom{9}\\
  \hphantom{9}3&если бы&21\hphantom{9}\\
  \hphantom{9}4&если только&6\\
  \hphantom{9}5&хотя&4\\
  \hphantom{9}6&если$\vert$и&4\\
  \hphantom{9}7&если бы$\|$то$\vert$бы&4\\
  \hphantom{9}8&только если&3\\
  \hphantom{9}9&zero (\textit{коннектор не переведен})&2\\
  10&если и$\|$то&2\\
  11&если$\|$значит&2\\
  12&если$\vert$и$\|$то&2\\
  13&же&2\\
  14&даже если&2\\
  15&а&2\\
  16&но&2\\
  17&ежели&1\\
  18&если$\|$значит&1\\
  19&если бы только&1\\
  20&если при этом&1\\
  21&если и&1\\
  22&едва$\|$как&1\\
  23&тогда&1\\
  24&и&1\\
  25&в то время как&1\\
  26&когда&1\\
  27&когда$\|$то&1\\
  28&поскольку$\|$то&1\\
  29&пусть$\vert$и$\|$во всяком случае&1\\
  30&особенно&1\\
  31&x\_при N&1\\
  32&xv\_fin\_Imp$\acute{\mbox{e}}$ratif+только&1\\
  33&xv\_fin\_Imp$\acute{\mbox{e}}$ratif&1\\
  34&x\_g$\acute{\mbox{e}}$rondif pass$\acute{\mbox{e}}$&1\\
  \hline
  \multicolumn{2}{|c|}{Всего аннотаций:}&255\hphantom{99}\\
  \hline
  \end{tabular}
  }
  \vspace*{3pt}
  \end{center}
 % \end{table*}

\noindent
вает, какие рубрики 
характеризуют модель перевода в~НБД, а~также насколько продуктивна эта 
модель при том или ином их сочетании. Экспериментальные данные 
включают следующие рубрики: порядок следования связываемых 
коннектором текстовых фрагментов (p~CNT~q, CNT~q~p); 
встраивающий/встроенный коннектор (SubCNT / SuperCNT); маркирует 
часть предложения без пре\-ди\-ка\-ции\,/\,с~пре\-ди\-ка\-ци\-ей; позиция  
(на\-чаль\-ная/не\-на\-чаль\-ная/ко\-неч\-ная); ЛСО, устанавливаемые 
коннектором (условные и~т.\,д.); маркирует вставку и~т.\,д.
  
  
Релевантность приведенных рубрик определялась в~ходе предыдущих 
корпусных контрастивных исследований, где инструментом служила НБД 
коннекторов. Результаты некоторых исследований см.\  
в~\cite{16-n, 17-n, 20-n}. Приведенные в~табл.~2 рубрики, характеризующие 
коннектор и~его контекст, были проанализированы на предмет их 
ре\-ле\-вант\-ности для выбора переводного варианта\footnote{Анализ проводился доктором 
филологических наук В.\,А.~Нуриевым~--- переводчиком, имеющим опубликованные переводы с~французского 
и~английского языков.}. Следовало определить, позволяет ли имеющийся 
в~НБД\footnote{Набор рубрик первоначально формировался лингвистами, исходя из определенных 
задач, напрямую не связанных с~переводческой деятельностью.} набор рубрик сузить область 
многовариантного переводческого поиска, т.\,е.\ максимально редуцировать 
спектр моделей перевода, анализируя сочетания руб\-рик.


  
  Согласно табл.~1, французский коннектор <<si>> имеет более чем одно 
переводное русскоязычное соответствие: для него в~НБД зарегистрированы 
34~варианта перевода. Чтобы переводчик мог редуцировать спектр 
представленных моделей перевода, указанных в~табл.~2 рубрик оказалось 
недостаточно. Исходя из задач переводной деятельности, число рубрик 
необходимо увеличить. Этот вывод сделан на основе нескольких 
наблюдений:
  \begin{itemize}
  \item несмотря на явную продуктивность самого час\-тот\-но\-го 
русскоязычного эквивалента <<si>>~--- <<если>>, число случаев его 
употребления не достигает 50\% (48\%) от общего числа употреблений 
(123~аннотации из~255), причем лишь в~двух переводах этот эквивалент 
занимает ли\-ди\-ру\-ющую позицию (у~Н.~Галь в~<<Маленьком принце>> 
А.~де Сент-Эк\-зю\-пе\-ри~--- 92\% и~у~Ю.\,Яхниной в~<<Утраченном 
мире>> П.~Модиано~--- 81\%),\linebreak в~остальных переводах на него приходится 
самое большее~50\% употреблений, при этом <<если>> отличается 
универсальностью и~востребовано в~текс\-тах любой стилистической 
и~\mbox{жан\-ро\-вой} на\-прав\-лен\-ности;
  \item  в~сборнике научных статей Ж. Женетта <<Фигуры>> 33\% случаев 
перевода <<si>> приходится на <<если>>, что, по-ви\-ди\-мо\-му, 
мотивировано стремлением к~синонимическому варьированию в~подборе 
эквивалентов для оформления научного высказывания;
  \item  вторым по продуктивности русскоязычным эквивалентом <<si>> 
является <<если$\|$то>>, на него приходится 22\% случаев употребления 
(56~аннотаций из~255, зарегистрированных в~НБД), для него не характерна 
такая универсальность,\linebreak как для <<если>>, это эквивалент с~явно выраженной  
функ\-цио\-наль\-но-праг\-ма\-ти\-че\-ской нагрузкой~--- он употребляется 
в~высказываниях, \mbox{имитирующих} живую разговорную речь  
(диа\-ло\-ги\-че\-ские/мо\-но\-ло\-ги\-че\-ские реплики, внутренний монолог), 
где за счет частицы <<то>> нагнетается иллокутивная сила (об этом 
подробнее см.~\cite{20-n}), а~также в~синтаксически насыщенных\linebreak 
предложениях с~развернутой слож\-ной структурой, где <<то>> служит 
средством демаркации,\linebreak помогая лучше обозначить границы разных 
синтаксических блоков (такие предложения прежде всего востребованы 
в~нехудожественных или пограничных с~художественными жан\-рах);
  \item для <<si>> в~НБД зафиксированы 18~единичных русских 
эквивалентов, из них~14 не содержат в~качестве строительного элемента 
<<если>> (что свидетельствует об интерпретативной диффузности 
коннектора <<si>>), а~некоторые имеют очевидные ограничения на 
употребление, например устаревшее <<ежели>> или характерное для 
письменной речи <<едва$\|$как>>;
  \item  при переводе текстов художественных жан\-ров наблюдается 
тенденция к~большей синтаксической вариативности, здесь реализуются 
такие варианты перевода <<si>>, как частица <<же>> (2~вхож\-де\-ния), 
конструкции типа x\_при~N (1~вхож\-де\-ние), 
xv\_fin\_Imp$\acute{\mbox{e}}$ratif+только (1~вхож\-де\-ние),  
xv\_fin\_Imp$\acute{\mbox{e}}$ratif (1~вхож\-де\-ние), 
x\_g$\acute{\mbox{e}}$rondif pass$\acute{\mbox{e}}$ (1~вхож\-де\-ние) и~т.\,д.
  \end{itemize}
  
  Эти наблюдения подтверждают сделанный на предыдущем этапе вывод о 
том, что при простановке рубрик, характеризующих коннектор и~его 
переводной эквивалент в~НБД, необходимо ввести дополнительную рубрику  
<<ху\-до\-жест\-вен\-ный/не\-ху\-до\-жест\-вен\-ный текст>>, позволяющую 
редуцировать спектр моделей перевода.

\vspace*{-9pt}
  
  \section{Заключение}
  
  \vspace*{-3pt}
  
  В заключение на основе приведенных выше наблюдений можно сделать 
два основополагающих вывода, имеющих принципиальное значение для 
редуцирования спектра моделей перевода в~НБД при переводе.
  \begin{enumerate}[1.]
  \item  Разнообразие в~использовании моделей перевода во многом задается 
средовыми условиями. Некоторые переводные эквиваленты в~явном виде 
тяготеют к~реализации в~текстах с~определенными характеристиками, будь то 
тексты, имеющие усложненную синтаксическую организацию типа 
литературоведческих трудов, как\linebreak в~случае Ж.~Женетта, или 
беллетризованные до\-ку\-мен\-таль\-но-днев\-ни\-ко\-вые заметки, как 
в~случае Ж.~Кокто. Например, <<если$\|$то>>, ука\-зы\-ва\-ющее на условные 
и~сопоставительные ЛСО, употребляется в~силу необходимости более 
рель\-еф\-но отделить условие наступления ка\-ко\-го-ли\-бо положения дел от 
самого положения дел, с~по\-мощью час\-ти\-цы <<то>> проводя более 
от\-чет\-ли\-вую границу между двумя синтаксическими частями 
в~сложносоставных пред\-ло\-же\-ни\-ях-вы\-ска\-зы\-ва\-ни\-ях. Как уже 
говорилось выше, в~диалогических репликах <<если$\|$то>> употреб\-ля\-ет\-ся 
с~другой целью~--- чтобы придать сказанному б$\acute{\mbox{о}}$льшую выразительность, 
используя для этого частицу <<то>>. <<Ежели>> имеет явную тенденцию 
употребляться там, где переводчик стремится стилистически <<состарить>> 
текст. А~упо\-треб\-ле\-ние <<поскольку$\|$то>> может отвечать задачам 
наращивания синонимического ряда в~научном тексте (переводы 
Ж.~Женетта).
\item  Первый вывод можно дополнить вторым.\linebreak В~НБД при простановке 
рубрик, характеризующих текстовый фрагмент с~коннектором, наряду 
с~рубрикой <<ху\-до\-жест\-вен\-ный/не\-ху\-до\-жест\-вен\-ный текст>> 
необходимо ввести такие \mbox{рубрики}, как <<дата создания оригинала>> (для 
маркировки пригодных для старения \mbox{текс\-та} переводных вариантов) 
и~<<речь, ими\-ти\-ру\-ющая жи\-вую\,/\,раз\-го\-вор\-ный  
ре\-гистр\,/\,пись\-мен\-ная речь>> (для маркировки контекстов, име\-ющих 
форму диалога, внут\-рен\-не\-го монолога или, наоборот, возможных только 
в~письменной речи)\footnote{Приводимые названия рубрик носят рабочий характер.}. 
О~значимости таких рубрик говорят и~дополнительные наблюдения, 
связанные с~переводом на русский французских языковых единиц, 
функционирующих в~живой разговорной речи,~--- междометий 
и~обращений: существует явная асимметрия в~употреблении этих классов 
речевых единиц во французском и~русском языках. Таким образом, при переводе не 
коннекторов, а~некоторых других языковых единиц рубрики, 
характеризующие принадлежность к~устному/письменному регистру речи, 
будут не опциональны, а обязательны. Также целесообразно предусмотреть 
рубрики, на\-прав\-лен\-ные на нюансирование стилевой и~жанровой амплитуды 
текстовых фрагментов. Таким образом,  дальнейшая детализация 
классификационной системы рубрик при аннотировании в~НБД необходима. 
Дополнительные рубрики позволят более эффективно сужать поле 
переводческого поиска, существенно редуцируя множество альтернатив при 
выборе моделей перевода обрабатываемой языковой единицы.
\end{enumerate}

  Насколько детальной и~универсальной должна быть эта 
классификационная система, в~настоящее время определить сложно. Чтобы 
ответить на этот вопрос, нужно применить предлагаемый подход к~языковым 
единицам, отличным от коннекторов.

\vspace*{-9pt}

{\small\frenchspacing
 {%\baselineskip=10.8pt
 %\addcontentsline{toc}{section}{References}
 \begin{thebibliography}{99}
 
 \vspace*{-3pt}
 
 \bibitem{4-n} %1
Corpus-based translation studies: Research and applications~/ Eds. A.~Kruger, K.~Wallmach, 
J.~Munday.~--- London: Bloomsbury Academic, 2011. 320~p.
\bibitem{1-n} %2
New directions in corpus-based translation studies~/ Eds. C.~Fantinuoli, F.~Zanettin.~--- Berlin: 
Language Science Press, 2015. 168~p.

\bibitem{3-n} %3
\Au{Kaibao Hu.} Introducing corpus-based translation studies.~--- Heidelberg: Springer, 2016. 
245~p.

\bibitem{5-n} %4
New perspectives on corpus translation studies~/ Eds. V.\,X.~Wang, L.~Lim, D.~Li.~--- Singapore: Springer, 2021. 318~p.

\bibitem{2-n} %5
Extending the scope of corpus-based translation studies~/ Eds. S.~Granger, M.-A.~Lefer.~--- 
London: Bloomsbury Academic, 2022. 288~p.

\bibitem{6-n} %6
\Au{Ляшевская О.\,Н.} Корпусные инструменты в~грамматических исследованиях 
русского языка.~--- М.: ЯСК: Рукописные памятники Древней Руси, 2016. 520~с.

\bibitem{8-n} %7
\Au{Bittner H.} Evaluating the evaluator: A~novel perspective on translation quality 
assessment.~--- London, New York: Routledge, 2021. 296~p.

\bibitem{7-n} %8
\Au{Youdale R.} Using computers in the translation of literary style: Challenges and 
opportunities.~--- London, New York: Routledge, 2020. 242~p.

\bibitem{9-n} %9
\Au{Моретти Ф.} Дальнее чтение~/ Пер. с~англ.~--- М.: Изд-во Института Гайдара, 2016. 352~с.
( \Au{Moretti~F.} {Distant reading}.~--- Verso, 2013. 224~p.)

\bibitem{13-n} %10
\Au{Busa R.} Foreword: Perspectives on the digital humanities~// A~companion to digital 
humanities~/ Eds. S.~Schreibman, R.~Siemens, J.~Unsworth.~--- Malden, Oxford, Carlton: Blackwell, 2004. P.~xvi--xxii.

\bibitem{10-n} %11
\Au{Mahlberg M.} Corpus stylistics~// The Routledge handbook of stylistics~/  
Ed. M.~Burke.~--- London, New York: Routledge, 2014. P.~378--392.
\bibitem{11-n} %12
\Au{Егорова А.\,Ю., Зацман~И.\,М., Мамонова~О.\,С.} Надкорпусные базы данных 
в~лингвистических проектах~// Системы и~средства информатики, 2019. Т.~29. №\,3.  
С.~77--91.
\bibitem{12-n} %13
\Au{Зализняк А.\,А., Зацман И.\,М., Инькова~О.\,Ю.} Надкорпусная база данных 
коннекторов: построение системы терминов~// Информатика и~её применения, 2017. 
Т.~11. Вып.~1. С.~100--108.

\bibitem{14-n} %14
\Au{Инькова О.\,Ю.} Надкорпусная база данных как инструмент формальной 
вариативности коннекторов~// Компьютерная лингвистика и~интеллектуальные 
технологии: по мат-лам ежегодной Междунар. конф. <<Диалог>>.~--- М.: РГГУ, 2018. 
Вып.~17(24). С.~240--253.


\bibitem{17-n} %15
Семантика коннекторов: контрастивное исследование~/ Под ред. О.\,Ю.~Иньковой.~--- 
М.: ТОРУС ПРЕСС, 2018. 396~с.
\bibitem{18-n} %16
Структура коннекторов и~методы ее описания~/ Под ред. О.\,Ю.~Иньковой.~--- М.: 
ТОРУС ПРЕСС, 2019. 316~с.

\bibitem{16-n} %17
\Au{Инькова-Манзотти~О.\,Ю.} Коннекторы противопоставления во французском 
и~русском языках: Сопоставительное исследование.~--- М.: Информэлектро, 2001. 429~с.


\bibitem{15-n} %18
\Au{Зацман И.\,М., Кружков~М.\,Г.} Надкорпусная база данных коннекторов: развитие 
системы терминов проектирования~// Системы и~средства информатики, 2018. Т.~28. 
№\,4. С.~156--167.
\bibitem{19-n}
\Au{Нуриев В.\,А., Зацман~И.\,М.} Редуцирование спектра моделей перевода 
в~надкорпусных базах данных~// Информатика и~её применения, 2020. Т.~14. Вып.~2.  
С.~119--126.
\bibitem{20-n}
\Au{Нуриев В.} Коннектор \textit{раз$\ldots$\ то} и~модели его перевода на французский 
язык: между условием, причиной и~следствием~// Съпоставително езикознание 
(Сопоставительное языкознание\,/\,Contrastive Linguistics), 2017. Т.~42. №\,4. С.~63--76.
\end{thebibliography}

 }
 }

\end{multicols}

\vspace*{-9pt}

\hfill{\small\textit{Поступила в~редакцию 14.07.22}}

\vspace*{8pt}

%\pagebreak

%\newpage

%\vspace*{-28pt}

\hrule

\vspace*{2pt}

\hrule

%\vspace*{-2pt}

\def\tit{COMPUTER-ASSISTED TEXTUAL ANALYSIS 
IN~TRANSLATION: REDUCING THE~SPECTRUM 
OF~TRANSLATION MODELS IN~SUPRACORPORA DATABASES}


\def\titkol{Computer-assisted textual analysis 
in~translation: Reducing the~spectrum 
of~translation models in~supracorpora databases}


\def\aut{V.\,A.~Nuriev}

\def\autkol{V.\,A.~Nuriev}

\titel{\tit}{\aut}{\autkol}{\titkol}

\vspace*{-15pt}


\noindent
Federal Research Center ``Computer Science and Control'' of the Russian Academy of Sciences, 
44-2~Vavilov Str., Moscow 119333, Russian Federation


\def\leftfootline{\small{\textbf{\thepage}
\hfill INFORMATIKA I EE PRIMENENIYA~--- INFORMATICS AND
APPLICATIONS\ \ \ 2022\ \ \ volume~16\ \ \ issue\ 3}
}%
 \def\rightfootline{\small{INFORMATIKA I EE PRIMENENIYA~---
INFORMATICS AND APPLICATIONS\ \ \ 2022\ \ \ volume~16\ \ \ issue\ 3
\hfill \textbf{\thepage}}}

\vspace*{3pt} 
     
    
    
    
    \Abste{The paper refines the approach aimed at reducing the 
spectrum of translation models in supracorpora databases (SDBs). Being 
an information resource of broad potential application, SDBs can be 
used to research on problems in the field of information science, 
computer linguistics, medicine, etc. Here, SDBs are regarded from the 
perspective of the corpus-based translation studies. It is shown how this 
automated instrument can be applied in `close and distant reading'~--- an 
approach that advocates the idea of using modern information resources 
in literary translation. The special focus is on opportunities that SDBs 
could offer for reducing the spectrum of translation models. Due to the 
synonymic potential, characteristic of natural languages, in translation, 
instead of the only possible solution, one has to choose between 
relatively interchangeable alternatives (words, collocations, syntactic 
constructions, etc.). Choosing the only one output equivalent, 
a~translator seeks to narrow the choice set. Hence, the goal of the paper 
is to refine the approach that would allow using SDBs for narrowing the 
choice set of relevant translation models.}
    
    \KWE{corpus-based translation studies; digital humanities; 
computer-assisted textual analysis; distant reading; parallel texts; 
translation; translation models; supracorpora database; multiple choice}
    
    
    
\DOI{10.14357/19922264220309} 

%\vspace*{-16pt}

\Ack
    \noindent
    The research was carried out using the infrastructure of the Shared 
Research Facilities ``High Performance Computing and Big Data'' (CKP 
``Informatics'') of FRC CSC RAS (Moscow).



%\vspace*{4pt}

  \begin{multicols}{2}

\renewcommand{\bibname}{\protect\rmfamily References}
%\renewcommand{\bibname}{\large\protect\rm References}

{\small\frenchspacing
 {%\baselineskip=10.8pt
 \addcontentsline{toc}{section}{References}
 \begin{thebibliography}{99}
 
 \bibitem{4-n-1} %1
Kruger, A., K.~Wallmach, and J.~Munday, eds. 2011. \textit{Corpus-based translation studies: Research and applications}. London: 
Bloomsbury Academic. 320~p.

    \bibitem{1-n-1} %2
    Fantinuoli, C., and F.~Zanettin, eds. 2015. \textit{New directions in 
corpus-based translation studies}. Berlin: Language Science Press. 
168~p.
   
\bibitem{3-n-1} %3
\Aue{Hu, K.} 2016. \textit{Introducing corpus-based translation 
studies}. Heidelberg: Springer. 245~p.

\bibitem{5-n-1} %4
    Wang, V.\,X., L.~Lim, and D.~Li, eds. 2021. \textit{New 
perspectives on corpus translation studies}. Singapore: Springer. 318~p.

 \bibitem{2-n-1} %5
    Granger, S., and M.-A.~Lefer, eds. 2022. \textit{Extending the 
scope of corpus-based translation studies}. London: Bloomsbury 
Academic. 288~p.
\bibitem{6-n-1} %6
    \Aue{Lyashevskaya, O.\,N.} 2016. \textit{Korpusnye instrumenty 
v~grammaticheskikh issledovaniyakh russkogo yazyka} [Corpus 
instruments in grammatical studies of Russian]. Moscow: YaSK: 
Rukopisnye pamyatniki Drevney Rusi. 520~p.

\bibitem{8-n-1} %7
    \Aue{Bittner, H.} 2020. \textit{Evaluating the evaluator: A~novel 
perspective on translation quality assessment}. London, New York: 
Routledge. 296~p.

\bibitem{7-n-1} %8
    \Aue{Youdale, R.} 2020. \textit{Using computers in the translation 
of literary style: Challenges and opportunities}. London, New York: 
Routledge. 242~p.

\bibitem{9-n-1} %9
    \Aue{Moretti, F.} 2013. \textit{Distant reading}. 
Verso. 224~p.

\bibitem{13-n-1} %10
    \Aue{Busa, R.} 2004. Foreword: Perspectives on the digital 
humanities. \textit{A~companion to digital humanities}. Eds. 
S.~Schreibman, R.~Siemens, and J.~Unsworth. Malden, Oxford, 
Carlton: Blackwell. xvi--xxii.

\bibitem{10-n-1} %11
    \Aue{Mahlberg, M.} 2014. Corpus stylistics. \textit{The Routledge 
handbook of stylistics}. Ed. M.~Burke. London, New York: Routledge.  
378--392.
\bibitem{11-n-1} %12
    \Aue{Egorova, A.\,Yu., I.\,M.~Zatsman, and O.\,S.~Mamonova.} 
2019. Nadkorpusnye basy dannykh v~lingvisticheskikh proektakh 
[Supracorpora databases in linguistic projects]. \textit{Sistemy 
i~Sredstva Informatiki~--- Systems and Means of Informatics} 
 29(3):77--91.
\bibitem{12-n-1} %13
    \Aue{Zaliznyak, A.\,A., I.\,M.~Zatsman, and O.\,Yu.~In'kova}. 
2017. Nadkorpusnaya basa dannykh konnektorov: postroenie sistemy 
terminov [Supracorpora database of connectives: Term system 
development]. \textit{Informatika i~ee Primeneniya~--- Inform. Appl.} 
11(1):100--108.

\bibitem{14-n-1}
    \Aue{Inkova, O.\,Yu.} 2018. Nadkorpusnaya baza dannykh kak 
instrument formal'noy variativnosti konnektorov [Supra-corpora 
database as an instrument of the study of the formal variability of 
connectives]. \textit{Komp'yuternaya ling\-vi\-sti\-ka i~intellektual'nye 
tekhnologii: po mat-lam ezhegodnoy Mezhdunar. konf. ``Dialog''} 
[Computer Linguistic and Intellectual Technologies: Conference 
(International) ``Dialog'' Proceedings]. Moscow. 17(24):240--253.


\bibitem{17-n-1} %15
    In'kova, O.Yu., ed. 2018. \textit{Semantika konnektorov: 
kontrastivnoe issledovanie} [Semantics of connectives: Contrastive 
study]. Moscow: TORUS PRESS. 396~p.
\bibitem{18-n-1} %16
    Inkova, O.\,Yu., ed. 2019. \textit{Struktura konnektorov i~metody 
ee opisaniya} [Structure of connectives and methods of its description]. 
Moscow: TORUS PRESS. 316~p.

\bibitem{16-n-1} %17
    \Aue{In'kova-Manzotti, O.\,Yu.} 2001. \textit{Konnektory 
pro\-ti\-vo\-po\-stav\-le\-niya vo frantsuzskom i~russkom yazykakh. 
So\-po\-sta\-vi\-tel'\-noe issledovanie} [Connectors of opposition in French and 
Russian: A comparative study]. Moscow: Informelektro. 429~p.

\bibitem{15-n-1} %18
    \Aue{Zatsman, I.\,M., and M.\,G.~Kruzhkov.} 2018. 
Nadkorpusnaya baza dannykh konnektorov: razvitie sistemy terminov 
proektirovaniya [Supracorpora database of connectives: Design-oriented 
evolution of the term system]. \textit{Sistemy i~Sredstva Informatiki~--- 
Systems and Means of Informatics} 28(4):156--167.

\bibitem{19-n-1}
    \Aue{Nuriev, V.\,A., and I.\,M.~Zatsman.} 2020. Redutsirovanie 
spektra modeley perevoda v~nadkorpusnykh bazakh dannykh [Reducing 
the spectrum of translation models in supracorpora databases]. 
\textit{Informatika i~ee Primeneniya~--- Inform. Appl.} 14(2):119--126.
\bibitem{20-n-1}
    \Aue{Nuriev, V.} 2017. Konnektor \textit{raz}$\ldots$\ \textit{to} i~modeli 
ego perevoda na frantsuzskiy yazyk: mezhdu usloviem, prichinoy 
i~sledstviem [The connector \textit{raz}$\ldots$ \textit{to} and variants of rendering it 
into French: Between condition, cause, and effect]. 
\textit{Sopostavitel'noe yazykoznanie} [Contrastive Linguistics] 
42(4):63--76.
\end{thebibliography}

 }
 }

\end{multicols}

\vspace*{-6pt}

\hfill{\small\textit{Received July 14, 2022}}
    
    \Contrl
    
    \noindent
    \textbf{Nuriev Vitaly A.} (b.\ 1980)~--- Doctor of Science in 
philology, leading scientist, Institute of Informatics Problems, Federal 
Research Center ``Computer Science and Control'' of the Russian 
Academy of Sciences, 44-2~Vavilov Str., Moscow 119333, Russian 
Federation; \mbox{nurieff.v@gmail.com}
    

\label{end\stat}

\renewcommand{\bibname}{\protect\rm Литература}    