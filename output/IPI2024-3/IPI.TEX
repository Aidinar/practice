\documentclass[10pt]{book}
\usepackage[utf8]{inputenc}

\usepackage{latexsym,amssymb,amsfonts,amsmath,amsxtra,dsfont,amscd,
indentfirst,shapepar,%fleqn,%
picinpar,shadow,floatflt,enumerate,multicol,colortbl,moreverb,cite,ipi}

\usepackage{rotating}
\usepackage{mathrsfs}
\usepackage[noend]{algorithmic}
\usepackage{ulem}
\usepackage{graphicx}
%\usepackage{algorithm2e}
\usepackage[linesnumbered,boxed,ruled]{algorithm2e}
%\usepackage{xypic}
\usepackage{oldgerm}
\usepackage{epic}
\usepackage{eepic}

\SetAlgorithmName{Algorithm}{алгоритм}{Список алгоритмов}

%из Дюковой

\newcommand{\algKeyword}[1]{{\bf #1}}
\newcommand{\Proc}[1]{\text{\tt #1}}
\def\CALL{\algKeyword{call}~}

\newenvironment{AlgProcedure}[1]
{
\small
\medskip
%    \hrule
\medskip
\algKeyword{PROCEDURE} #1
\begin{algorithmic}[1]}
{\end{algorithmic}
%    \hrule
\bigskip
}

\def\CALL{\algKeyword{call}~}

%конец для Дюковой

%\RequirePackage[ruled]{algorithm}


\input{epsf}

%\nofiles

%\includeonly{avtor}    %pdf
%\includeonly{podgot-rus-site,podgot-eng-site}  
%\includeonly{podgot-rus,podgot-eng}  
%\includeonly{ipi-ind} 
%\includeonly{index-17i}
%\includeonly{toc-rus, toc-en}
%\includeonly{toc-rus}
%\includeonly{toc-en} 
%\includeonly{popravka}



%\includeonly{bosov}      %pdf+авт+
%\includeonly{borisov}    %pdf+авт+
%\includeonly{shestakov}  %pdf+авт+
%\includeonly{malashenko} %pdf+авт+
%\includeonly{sinitsin}   %pdf+авт+
%\includeonly{dukova}     %pdf+авт повт отпр
%\includeonly{zorin}      %pdf+авт+
%\includeonly{listopad}   %pdf+авт+
%\includeonly{flerov}     %pdf+авт+
%\includeonly{sinitsin-2} %pdf+авт+
%\includeonly{shnurkov}   %pdf читает
%\includeonly{zatsman}    %pdf+авт+
%\includeonly{goncharov}  %pdf+авт повт отпр
%\includeonly{nuriev}     %pdf+авт+



%%%%%%%%%%%%%%%%%%%\includeonly{nekrolog-new}



%\includeonly{rekl}




\usepackage{acad}
%\usepackage{courier}
\usepackage{decor}
\usepackage{newton}
\usepackage{pragmatica}
\usepackage{zapfchan}
\usepackage{petrotex}
\usepackage{bm}                     % полужирные греческие буквы
\usepackage{upgreek}                % прямые греческие буквы \upalpha
\usepackage{eufrak}
\usepackage{verbatim}

\renewcommand{\bottomfraction}{0.99}
\renewcommand{\topfraction}{0.99}
\renewcommand{\textfraction}{0.01}

\setcounter{secnumdepth}{1} %здесь - 3 + chapter = 4

\arraycolsep=1.5pt

%\usepackage[pdftex]{graphicx}

%\usepackage{oz}

%NEW COMMANDS



\renewcommand*{\hm}[1]{#1\nobreak\discretionary{}%
            {\hbox{$\mathsurround=0pt #1$}}{}} %% Дублирует знаки операций
                               %при переносе в формуле (перед знаком, который
                               %надо продублировать ставится команда \hm)
                               
                               \newcommand{\PRB}{\begin{picture}(22.5,11)
      \spline(1,8)(4,10)(7,10.5)(10,11)(13,11)(16,10.5)(19,10)(22,8)
               \put(0,0){$P_{i-1}P_{t_{t-1}}$} \end{picture}}

\newcommand{\prb}{\begin{picture}(15.5,9)
      \spline(1,6)(3,8)(5,8.5)(7,9)(9,9)(11,8.5)(13,8)(15,6)
               \put(0,0){$PP_t$} \end{picture}}
               
                 \newcommand{\PRDN}{\begin{picture}(40,11)
      \spline(4,11.5)(7,10.5)(12,10)(16,9)(20,9)(24,10)(29,10.5)(32,11.5)
               \put(0,0){$P_{i-1}P_{t_{t-1}}$} \end{picture}}

\newcommand{\prdn}{\begin{picture}(18,11)
      \spline(3,10.5)(4,10)(6,9)(8,8.5)(10,8.5)(12,9)(14,10)(15,10.5)
               \put(0,0){$PP_t$} \end{picture}}




%\newcommand{\endproof}{\hfill$\Box$}
%\renewcommand{\r}{\mathbb{R}}
%\newcommand{\I}{{\rm I\hspace{-0.7mm}I}}
%\newcommand{\Ikl}{{\tt{1}}\hspace*{-1.44mm}\mathtt{1}}
%\newcommand{\Ik}{\mbox{{\small \tt {1}}\hspace{-1.3mm}{\tt 1}}}
\newcommand{\Ik}{\mbox{{{\tt 1}}\hspace{-1.3mm}{\sf 1}}}
\newcommand{\argmin}{\mathop{\mathrm{arg}\,\mathrm{min}}}
\newcommand{\argmax}{\mathop{\mathrm{arg}\,\mathrm{max}}}
%\newcommand{\capr}{\mathop{\cap\,}}
%\newcommand{\cupr}{\mathop{\cup\,}}
%\def\argmin{\mathop{arg\,min}}

\def\vrp{\varphi}
\def\prt{\partial}
\def\mm{{\sf M}}
\def\modnop#1{\mathop{#1}\limits_{n}}
\def\eam{\mathbin{{\mathop{=}\limits^{\mathrm{def}}}}}
\def\dey#1#2{#1 (#2)}
\def\deyc#1#2{#1 \cdot  #2}
\def\ra#1{\;\mathop{\to}\limits^{#1}\;}
\def\raz#1{\;\mathop{\longrightarrow}\limits^{\!\!\!#1}\;}
\def\ral#1{\;\mathop{\longrightarrow}\limits^{#1}\;}





\newcommand{\il}[2]{\int\limits_{#1}^{#2}}%интеграл с пределами #1 и #2

\def\sm2{\mathop {\sum\limits^{n^\Theta}\sum\limits^{n^\Theta}}}
\def\sss{\sum\limits}
\def\tr{,\,\ldots\,,\,}
\def\rk{\right]}
\def\lk{\left[}
\def\rf{\right\}}
\def\lf{\left\{}
\def\lv{\,\left\vert}
\def\rv{\right\vert\,}
\def\iii{\int\limits}
\def\iin{\int\limits_{-\infty}^\infty}
\def\rrv{\right\vert}


\def\ee{{\cal E}}
\def\ww{{\cal W}}
\def\yy{{\cal Y}}
\def\vv{{\cal V}}

\newcommand{\R}{\mathbb R}
\newcommand{\E}{\mathbb E}
\newcommand{\N}{\mathbb N}
\newcommand{\T}{\mathbb{T}}
\newcommand{\Z}{\mathbb{Z}}

\renewcommand{\P}{\mathbb{P}}

\newcommand{\Nor}{\mathcal{N}}

\newcommand{\h}{{\bf H}}
\newcommand{\p}{{\sf P}}  % вероятность
\newcommand{\e}{{\sf E}}  % мат. ожидание
\newcommand{\D}{{\sf D}}  % дисперсия



\newcommand{\vw}{{\mathbf w}}
\newcommand{\vp}{{\mathbf p}}
\newcommand{\vz}{{\mathbf z}}
\newcommand{\vx}{{\mathbf x}}
\newcommand{\vf}{{\mathbf f}}
\newcommand{\F}{{\mathcal F}}
\def\ap{{\mathrm{ЭР}}}
\newcommand{\ud}{\Delta_n} %uniform ditance
\newcommand{\nud}{\Delta_n(x)}
%\renewcommand{\Re}{\mathrm{Re}\,}

\newcommand{\abs}[1]{\left\vert#1\right\vert}

\newcommand{\norm}[1]{\left\Vert#1\right\Vert}
%\def\da{(\Delta_t,A)}

\newcommand{\corr}{\mathrm{corr}}

\newcommand{\cov}{\mathrm{cov}}
\newcommand{\Expect}{\mathbb{E}}

\def\w{\omega}
\def\W{\Omega}


\def\inh{\int\limits_{nh}^{(n+1)h}}
\def\inhnolim{\int\nolimits_{nh}^{(n+1)h}}

\def\sumin{\sum_{i=1}^N}


\def\bxt{(Y,t)}
\def\xt{(y,t)}

\def\ovth{{\fr{\tau-nh}{h}}}
\def\ov{\overline}
\def\tm{\tilde m}
\def\tl{\tilde\lambda}
\def\tB{\widetilde B}
\def\tb{\tilde b}
\def\ld{\ldots}
\def\cd{\cdots}


\DeclareMathOperator{\sign}{sign}



\newcommand{\g}{\mbox{\textit{g}}}

\renewcommand{\la}{\lambda}
\newcommand{\si}{\sigma}
\newcommand{\eps}{\varepsilon}
\newcommand{\alp}{\alpha}

\newcommand{\pto}{\stackrel{P}{\longrightarrow}} % сходимость по веpоятности

\newcommand{\eqd}{\stackrel{\mathrm{d}}{=}} % равенство по pаспpеделению
\newcommand{\eqdelta}{\stackrel{\triangle}{=}} % равенство по pаспpеделению

\def\be#1{\begin{equation}\label{#1}}
\def\ee{\end{equation}}
\def\re#1{(\ref{#1})}

\def\bn{\begin{enumerate}}
\def\en{\end{enumerate}}
\def\bi{\begin{itemize}}
\def\ei{\end{itemize}}
%\def\i{\item}

%\newcommand{\kp}{\kappa}
%\def\Q{{\cal Q}} \def\H{{\cal H}}
%\newcommand{\bet}{\beta_{2+\delta}}




%\renewcommand{\baselinestretch}{1.2}

%\pagestyle{myheadings}

\setlength{\textwidth}{167mm}      % 122mm
\setlength{\textheight}{658pt}
%\setlength{\textheight}{635.6pt}
\setlength{\columnsep}{4.5mm}

\setcounter{secnumdepth}{4}

%\addtolength{\headheight}{2pt}
%\addtolength{\headsep}{-2mm}

\addtolength{\topmargin}{-7mm}  % for printing


%\hoffset=-30mm  % From Yap
\hoffset=-23mm  % From Acrobat

%\voffset=0mm % From Yap
\voffset=-5mm   % From Acrobat

%\addtolength{\evensidemargin}{-2.5mm} % for printing
%\addtolength{\oddsidemargin}{2.5mm}  % for printing

\addtolength{\evensidemargin}{-12mm} % for printing
\addtolength{\oddsidemargin}{8mm}  % for printing

%\renewcommand{\thefootnote}{\fnsymbol{footnote}}
%\renewcommand{\thefootnote}{\arabic{footnote}}
\renewcommand{\figurename}{\protect\bf Рис.}
\renewcommand{\tablename}{\protect\bf Таблица}

\newcommand{\Caption}[1]{\caption{\protect\small %\baselineskip=2.5ex
#1}}

\renewcommand{\thefigure}{\arabic{figure}}
\renewcommand{\thetable}{\arabic{table}}
\renewcommand{\theequation}{\arabic{equation}}
\renewcommand{\thesection}{\arabic{section}}

\renewcommand{\contentsname}{СОДЕРЖАНИЕ}
\newcommand{\fr}[2]{\displaystyle\frac{\displaystyle #1\mathstrut}{\displaystyle #2\mathstrut}}

%\renewcommand{\thefootnote}{\fnsymbol{footnote}}
%\newcommand{\g}{\mbox{\textit{g}}}

%\newcommand{\Caption}[1]{\caption{\protect\small\baselineskip=2ex #1}}
\newcounter{razdel}
\setcounter{razdel}{0}

\def\god{2024}
\def\tom{18}
\def\vyp{3}


\newcommand{\titel}[4]{%
\

\vspace*{5pt}

\ifodd\therazdel {\raggedright\noindent\Large\textrm\textbf
 \lineskip .75em
  \baselineskip=3.2ex #1 \par}
\vskip 1em {\noindent\large\textrm\textbf #2 \par}
\addcontentsline{toc}{subsection}{{\textrm\textbf #1}\protect\newline #2}
\def\rightheadline{\underline{\noindent\hbox to \textwidth{\hfill\small\textrm{#4}
%\hfill \large\bf\thepage
}}}
\def\leftheadline{\underline{\noindent\parbox{\textwidth}{
%\raggedleft\large\bf\thepage \hfill
\small\textit{#3}\hfill}}}
\def\leftfootline{\small{\textbf{\thepage}
\hfill ИНФОРМАТИКА И ЕЁ ПРИМЕНЕНИЯ\ \ \ том~\tom\ \ \ выпуск~\vyp\ \ \ \god}
}%
 \def\rightfootline{\small{ИНФОРМАТИКА И ЕЁ ПРИМЕНЕНИЯ\ \ \ том~\tom\ \ \ выпуск~\vyp\ \ \ \god
\hfill \textbf{\thepage}}}
\vskip 2em \setcounter{figure}{0}
\setcounter{table}{0}
\setcounter{equation}{0}
\setcounter{section}{0}
\setcounter{subsection}{0}
\setcounter{subsubsection}{0}
\setcounter{footnote}{0}
\setcounter{razdel}{0}
%\end{flushleft}
\else {
 \raggedright\noindent\Large\textrm\textbf
 \lineskip .75em
\baselineskip=3.2ex #1 \par} \vskip 1em
%\begin{flushleft}
{\noindent\large\textrm\textbf #2 \par}
\addcontentsline{toc}{subsection}{{\textrm\textbf #1}\protect\newline #2}
\def\rightheadline{\underline{\noindent\hbox to \textwidth{\hfill\small\textrm{#4}
%\hfill \large\bf\thepage
}}}
\def\leftheadline{\underline{\noindent\parbox{\textwidth}{%\raggedleft\large\bf\thepage \hfill
\small\textit{#3}\hfill}}}
\def\leftfootline{\small{\textbf{\thepage}
\hfill ИНФОРМАТИКА И ЕЁ ПРИМЕНЕНИЯ\ \ \ том~\tom\ \ \ выпуск~\vyp\ \ \ \god}
}%
 \def\rightfootline{\small{ИНФОРМАТИКА И ЕЁ ПРИМЕНЕНИЯ\ \ \ том~18\ \ \ выпуск~\vyp\ \ \ 2024
\hfill \textbf{\thepage}}} \vskip 2em \setcounter{figure}{0}
\setcounter{table}{0} \setcounter{equation}{0} \setcounter{section}{0}
\setcounter{subsection}{0} \setcounter{subsubsection}{0}
\setcounter{footnote}{0}
%\end{flushleft}
\fi}

\newcommand{\titelr}[2]{%
\

\vspace*{5pt}

\ifodd\therazdel {\raggedright\noindent%\Large\textrm\textbf
 \lineskip .75em
  \baselineskip=3.2ex #1 \par}
\vskip 1em {\noindent\normalsize\textrm\textbf #2 \par}
\else {
 \raggedright\noindent\Large\textrm\textbf
 \lineskip .75em
\baselineskip=3.2ex #1 \par} \vskip 1em
%\begin{flushleft}
{\noindent\large\textrm\textbf #2 \par
%\noindent\normalsize\textrm\textbf #2 \par
} \fi}

\newcommand{\titele}[5]{%
\

%\vspace*{5pt}

\ifodd\therazdel {\raggedright\noindent\large
\textrm\textbf
 \lineskip .75em
%  \baselineskip=3.2ex
#1 \par}
\vskip .5em {\noindent\large\textrm\textbf #2 \par}
\vskip .5em
 {\noindent\textrm #3 \par}
\addcontentsline{toc}{subsection}{{\textrm\textbf #1}\protect\newline #2}
\def\rightheadline{\underline{\noindent\hbox to \textwidth{\hfill\small\textrm{#4}
%\hfill \large\bf\thepage
}}}
\def\leftheadline{\underline{\noindent\parbox{\textwidth}{
%\raggedleft\large\bf\thepage \hfill
\small\textrm{#5}\hfill}}}
\def\leftfootline{\small{\textbf{\thepage}
\hfill ИНФОРМАТИКА И ЕЁ ПРИМЕНЕНИЯ\ \ \ том~18\ \ \ выпуск~\vyp\ \ \ 2024}
}%
 \def\rightfootline{\small{ИНФОРМАТИКА И ЕЁ ПРИМЕНЕНИЯ\ \ \ том~18\ \ \ выпуск~\vyp\ \ \ 2024
\hfill \textbf{\thepage}}} \vskip 1em \setcounter{figure}{0}
\setcounter{table}{0} \setcounter{equation}{0} \setcounter{section}{0}
\setcounter{subsection}{0} \setcounter{subsubsection}{0}
\setcounter{footnote}{0} \setcounter{razdel}{0}
%\end{flushleft}
\else {
 \raggedright\noindent\large
 \textrm\textbf
 \lineskip .75em
%\baselineskip=3.2ex
#1 \par} \vskip .5em
%\begin{flushleft}
{\noindent\large\textrm\textbf #2 \par} \vskip .5em
 {\noindent\textrm #3 \par}
\addcontentsline{toc}{subsection}{{\textrm\textbf #1}\protect\newline #2}
\def\rightheadline{\underline{\noindent\hbox to \textwidth{\hfill\small\textrm{#4}
%\hfill \large\bf\thepage
}}}
\def\leftheadline{\underline{\noindent\parbox{\textwidth}{%\raggedleft\large\bf\thepage \hfill
\small\textrm{#5}\hfill}}}
\def\leftfootline{\small{\textbf{\thepage}
\hfill ИНФОРМАТИКА И ЕЁ ПРИМЕНЕНИЯ\ \ \ том~18\ \ \ выпуск~\vyp\ \ \ 2024}
}%
 \def\rightfootline{\small{ИНФОРМАТИКА И ЕЁ ПРИМЕНЕНИЯ\ \ \ том~18\ \ \ выпуск~\vyp\ \ \ 2024
\hfill \textbf{\thepage}}} \vskip 1em \setcounter{figure}{0}
\setcounter{table}{0} \setcounter{equation}{0} \setcounter{section}{0}
\setcounter{subsection}{0} \setcounter{subsubsection}{0}
\setcounter{footnote}{0}
%\end{flushleft}
\fi}

\def\Abst#1{
\begin{center}\small\nwt
\parbox{150mm}{%\baselineskip=2.5ex
\textbf{Аннотация:}\ \
%\hspace*{\parindent}
#1}
\end{center}}
\def\Abste#1{
\begin{center}\small\nwt
\parbox{150mm}{%\baselineskip=2.5ex
\textbf{Abstract:}\ \
%\hspace*{\parindent}
#1}
\end{center}}

%\def\DOI#1{
%\begin{center}\small\nwt
%\parbox{150mm}{%\baselineskip=2.5ex
%\textbf{DOI:}\ \
%%\hspace*{\parindent}
%#1}
%\end{center}}

\def\Abstend#1{
\begin{center}\small\nwt
\parbox{150mm}{%\baselineskip=2.5ex
%\hspace*{\parindent}
#1}
\end{center}}

\newcommand{\DOI}[2]{\begin{center}\small\nwt
\parbox{150mm}{%\baselineskip=2.5ex
\textbf{DOI:}\ \
%\hspace*{\parindent}
#1 \hfill \textbf{EDN:}\ \
#2}
\end{center}}




\def\KW#1{
\begin{center}\small\nwt
\parbox{150mm}{%\baselineskip=2.5ex
\textbf{Ключевые слова:}\ \ #1}
\end{center}}

\def\KWE#1{
\begin{center}\small\nwt
\parbox{150mm}{%\baselineskip=2.5ex
\textbf{Keywords:}\ \ #1}
\end{center}}


\def\KWN#1{
%\begin{center}
%\small
%\parbox{150mm}\end{center}
}

\newcommand{\Avtors}[1]{%\smallskip
%\vspace*{.5pt}
\hangindent=23pt\noindent
%\nwt
{\bfseries#1}\
}


\renewcommand{\thesubsection}{\thesection.\arabic{subsection}\hspace*{-5pt}}
\renewcommand{\thesubsubsection}{\thesubsection\hspace*{5pt}.\arabic{subsubsection}\hspace*{-3pt}}

\newcommand{\Ack}{\section*{\protect\rmfamily Acknowledgments}\noindent}
\newcommand{\Contr}{\section*{\protect\rmfamily Contributors}\noindent}
\newcommand{\Contrl}{\section*{\protect\rmfamily Contributor}\noindent}

\makeindex


\begin{document}
\Rus

\nwt
%\ptb


%\renewcommand{\contentsname}{\protect\Large\bf Содержание}

\setcounter{tocdepth}{2}

%\tableofcontents

\renewcommand{\bibname}{\protect\rmfamily Литература}
  \def\Au#1{{\it #1}}
    \def\Aue#1{{#1}}

%\newcommand{\No}{№}
  \newcommand{\tg}{\,\mathrm{tg}\,}
    \newcommand{\ctg}{\,\mathrm{ctg}\,}
  \newcommand{\arctg}{\,\mathrm{arctg}\,}

\def\forallb{\mathop{\forall}}
\def\cupb{\mathop{\cup}}
\def\existsb{\mathop{\exists}}


\newpage
\addtocounter{razdel}{1}
%\def\razd{РЕГУЛИРУЕМЫЙ ЭЛЕКТРОПРИВОД ДЛЯ ЭЛЕКТРОЭНЕРГЕТИКИ}


\setcounter{page}{2}

%   { %\Large  
   { %\baselineskip=16.6pt
   
   \vspace*{-48pt}
   \begin{center}\LARGE
   \textit{Предисловие}
   \end{center}
   
   %\vspace*{2.5mm}
   
   \vspace*{25mm}
   
   \thispagestyle{empty}
   
   { %\small 

    
Вниманию читателей журнала <<Информатика и её применения>> предлагается 
очередной тематический выпуск <<Вероятностно-статистические методы и 
задачи информатики и информационных технологий>>. Предыдущие тематические 
выпуски журнала по данному направлению вышли в 2008~г.\ (т.~2, вып.~2), 
в 2009~г.\ (т.~3, вып.~3) и в 2010~г.\ (т.~4, вып.~2). 

Статьи, собранные в данном журнале, посвящены разработке новых вероятностно-статистических 
методов, ориентированных на применение к решению конкретных задач информатики и информационных 
технологий, а также~--- в ряде случаев~--- и других прикладных задач. Проблематика, охватываемая 
публикуемыми работами, развивается в рамках научного сотрудничества между Институтом проблем 
информатики Российской академии наук (ИПИ РАН) и Факультетом вычислительной математики и 
кибернетики Московского государственного университета им.\ М.\,В.~Ломоносова в ходе работ 
над совместными научными проектами (в том числе в рамках функционирования 
Научно-образовательного центра <<Вероятностно-статистические методы анализа рисков>>). 
Многие из авторов статей, включенных в данный номер журнала, являются активными участниками 
традиционного международного семинара по проблемам устойчивости стохастических моделей, 
руководимого В.\,М.~Золотаревым и В.\,Ю.~Королевым; регулярные сессии этого семинара 
проводятся под эгидой МГУ и ИПИ РАН (в 2011~г.\ указанный семинар проводится в октябре 
в Калининградской области РФ). 

Наряду с представителями ИПИ РАН и МГУ в число авторов данного выпуска журнала входят 
ученые из Научно-исследовательского института системных исследований РАН, Института 
проблем технологии микроэлектроники и особочистых материалов РАН, Института 
прикладных математических исследований Карельского НЦ РАН, Московского 
авиационного института, Вологодского государственного педагогического университета, 
НИИММ им.\ Н.\,Г.~Чеботарева, Казанского государственного университета, Дебреценского 
университета (Венгрия).

Несколько статей выпуска посвящено разработке и применению стохастических методов и 
информационных технологий для решения различных прикладных задач. В~работе В.\,Г.~Ушакова 
и О.\,В.~Шестакова рассмотрена задача определения вероятностных характеристик случайных 
функций по распределениям интегральных преобразований, возникающих в задачах эмиссионной 
томографии. В~статье Д.\,О.~Яковенко и М.\,А.~Целищева рассмотрены некоторые вопросы 
математической теории риска и предложен новый подход к диверсификации инвестиционных 
портфелей. Работа И.\,А.~Кудрявцевой и А.\,В.~Пантелеева посвящена построению и 
исследованию математической модели, описывающей динамику сильноионизованной плазмы. 
В~статье П.\,П.~Кольцова изучается качество работы ряда алгоритмов сегментации изображений. 
Статья А.\,Н.~Чупрунова и И.~Фазекаша посвящена вероятностному анализу числа без\-оши\-бочных 
блоков при помехоустойчивом кодировании; получены усиленные законы больших чисел для указанных 
величин.

В данном выпуске традиционно присутствует тематика, весьма активно разрабатываемая в течение 
многих лет специалистами ИПИ РАН и МГУ,~--- методы моделирования и управления для 
информационно-телекоммуникационных и вычислительных систем, в частности методы 
теории массового обслуживания. В~статье А.\,И.~Зейфмана с соавторами рассматриваются 
модели обслуживания, описываемые марковскими цепями с непрерывным временем в случае 
наличия катастроф. В~работе М.\,М.~Лери и И.\,А.~Чеплюковой рассматриваются случайные 
графы Интернет-типа, т.\,е.\ графы, степени вершин которых имеют степенные распределения; 
такие задачи находят применение при исследовании глобальных сетей передачи данных. 
Работа Р.\,В.~Разумчика посвящена исследованию систем массового обслуживания специального 
вида~--- с отрицательными заявками и хранением вытесненных заявок.

Ряд статей посвящен развитию перспективных теоретических 
вероятностно-статистических методов, которые находят широкое применение в различных 
задачах информатики и информационных технологий. В~работе В.\,Е.~Бенинга, А.\,К.~Горшенина 
и В.\,Ю.~Королева рассмотрена задача статистической проверки гипотез о числе компонент 
смеси вероятностных распределений, приводится конструкция асимптотически наиболее мощного 
критерия. Результаты этой работы найдут применение в ряде прикладных задач, использующих 
математическую модель смеси вероятностных распределений (в информатике, моделировании 
финансовых рынков, физике турбулентной плазмы и~т.\,д.). В~статье В.\,Ю.~Королева, 
И.\,Г.~Шевцовой и С.\,Я.~Шоргина строится новая, улучшенная оценка точности нормальной 
аппроксимации для пуассоновских случайных сумм; как известно, указанные случайные суммы 
широко используются в качестве моделей многих реальных объектов, в том числе в информатике, 
физике и других прикладных областях. Работа В.\,Г.~Ушакова и Н.\,Г.~Ушакова посвящена 
исследованию ядерной оценки плотности распределения; эти результаты могут применяться, 
в част\-ности, при анализе трафика в телекоммуникационных системах. Серьезные приложения 
в статистике могут получить результаты работы О.\,В.~Шестакова, в которой доказаны оценки 
скорости сходимости распределения выборочного абсолютного медианного отклонения к нормальному 
закону. 

\smallskip

Редакционная коллегия журнала выражает надежду, что данный тематический  выпуск 
будет интересен специалистам в области теории вероятностей и математической статистики 
и их применения к решению задач информатики и информационных технологий.
     
     %\vfill 
     \vspace*{20mm}
     \noindent
     Заместитель главного редактора журнала <<Информатика и её 
применения>>,\\
     директор ИПИ РАН, академик  \hfill
     \textit{И.\,А.~Соколов}\\
     
     \noindent
     Редактор-составитель тематического выпуска,\\
     профессор кафедры математической статистики факультета\\
      вычислительной математики и кибернетики МГУ им.\ М.\,В.~Ломоносова,\\
     ведущий научный сотрудник ИПИ РАН,\\ 
доктор физико-математических наук \hfill
      \textit{В.\,Ю.~Королев}
     
     } }
     }

\def\stat{flerov}

\def\tit{АВТОМАТИЗИРОВАННАЯ СИСТЕМА ВЕСОВОГО 
ПРОЕКТИРОВАНИЯ САМОЛЕТОВ}

\def\titkol{Автоматизированная система весового 
проектирования самолетов}

\def\aut{Л.\,Л.~Вышинский$^1$, Ю.\,А.~Флеров$^2$, Н.\,И.~Широков$^1$}

\def\autkol{Л.\,Л.~Вышинский, Ю.\,А.~Флеров, Н.\,И.~Широков}

\titel{\tit}{\aut}{\autkol}{\titkol}

\index{Вышинский Л.\,Л.}
\index{Флеров Ю.\,А.}
\index{Широков Н.\,И.}
\index{Vyshinsky L.\,L.}
\index{Flerov Yu.\,A.}
\index{Shirokov N.\,I.}




%{\renewcommand{\thefootnote}{\fnsymbol{footnote}} \footnotetext[1]
%{Работа выполнена при финансовой поддержке РФФИ (проект 17-01-00816).}}


\renewcommand{\thefootnote}{\arabic{footnote}}
\footnotetext[1]{Вычислительный центр им.\ А.\,А.~Дородницына Федерального исследовательского 
центра <<Информатика и~управ\-ле\-ние>> Российской академии наук, 
\mbox{Wysh@ccas.ru}}
\footnotetext[2]{Вычислительный центр им.\ А.\,А.~Дородницына Федерального исследовательского 
центра <<Информатика и~управ\-ле\-ние>> Российской академии наук, 
fler@ccas.ru}
%\footnotetext[3]{Вычислительный центр им.\ А.\,А.~Дородницына Федерального исследовательского 
%центра <<Информатика и~управ\-ле\-ние>> Российской академии наук, 
%\mbox{Wysh@ccas.ru}}

%\vspace*{-6pt}


 
  \Abst{Статья посвящена вопросам автоматизации задач весового проектирования 
самолетов. Весовые и~мас\-со\-во-инер\-ци\-он\-ные параметры являются одними из основных 
величин, влияющих на эксплуатационные характеристики самолетов. Информационной 
основой системы служит весовая модель самолета. Описывается структура весовой 
модели и~даны характеристики отдельным ее компонентам. Показана программная 
реализация системы, которая выполнена в~рамках архитектуры кли\-ент--сер\-вер. 
Автоматизированная система весового проектирования (АСВП)
реализована с~использованием 
про\-грам\-мно-ин\-стру\-мен\-таль\-но\-го комплекса <<Генератор проектов>> (технология ГП), 
который был разработан в~Вычислительном центре Российской академии наук. Создание 
ин\-фор\-ма\-ци\-он\-но-вы\-чис\-ли\-тель\-ных сис\-тем в~рамках технологии ГП базируется на так 
называемом <<проектном подходе>>, когда по формальному описанию системы автоматически 
генерируются тексты программного кода для клиентских и~серверных компонент системы.}
   
  \KW{математическое моделирование; автоматизация проектирования; самолет; весовое 
проектирование; весовая модель; дерево конструкции; генератор проектов; генерация 
программного кода; архитектура кли\-ент--сер\-вер}

  \DOI{10.14357/19922264180103} 
  
\vspace*{12pt}


\vskip 10pt plus 9pt minus 6pt

\thispagestyle{headings}

\begin{multicols}{2}

\label{st\stat}
   
\section{Введение}

  Развитие и~повсеместное использование информационных технологий за 
последние несколько десятилетий существенно изменили традиционный 
процесс проектирования и~создания различных инженерных систем, 
сооружений, машин. Во многих проектных организациях давно отказались от 
ко\-гда-то привычных инструментов конструктора~--- кульмана 
и~логарифмической линейки. 
%
Сейчас первые эскизы новых проектов 
появляются чаще не на бумаге, как было всегда, а~на экране монитора. Этому 
способствует широкий спектр имеющихся систем автоматизированного 
проектирования. В~российских авиационных конструкторских бюро, например, уже давно 
применяются такие CAD (computer aided design)
сис\-те\-мы, как NX (Unigraphics), CATIA и~др. 
%
Эти развитые системы геометрического трех\-мер\-но\-го (3D) мо\-де\-ли\-ро\-ва\-ния позволяют 
создавать сложные по\-верх\-ности, конструировать любые детали, осуществлять 
сборку узлов, агрегатов и~сложнейших изделий. Однако построение 
геометрических моделей изделий является финальной стадией проектирования, 
за которой следует их реализация <<в~металле>>. Построению электронных 
геометрических макетов предшествует и~сопутствует решение множества 
расчетных задач, а~также задач анализа и~оптимизации в~разных областях инженерных 
знаний. В~авиастроении это аэродинамика, динамика полета, прочность, 
системы управления, двигателестроение и~пр. Все эти задачи 
требуют разработки разноплановых математических моделей и~специальных 
вычислительных программ. 
  
  Одной из важнейших технических характеристик самолета является его вес. 
При решении подавляющего большинства проектных и~конструкторских задач 
весовые параметры в~том или ином виде участвуют в~расчетах. Необходимость 
проведения весовых расчетов возникает на самых ранних шагах 
проектирования и~сопровождает все дальнейшие стадии разработки 
и~эксплуатации. 

В~процессе создания и~эксплуатации самолетов постоянно 
контролируются вес и~другие мас\-со\-во-инер\-ци\-он\-ные характеристики (МИХ)
всех размещаемых на борту систем, агрегатов, узлов и~деталей. Количество 
агрегатов, узлов и~деталей современных самолетов исчисляется 
десятками тысяч, поэтому в~авиастроении весовые расчеты, весовой анализ, 
весовой контроль выливаются в~сложную инженерную проблему и~выделяются 
в~целое направление инженерной деятельности, которое принято называть 
весовым проектированием~[1].
  
  Данная статья посвящена вопросам автоматизации задач весового 
проектирования самолетов. В~разные годы Вычислительным центром РАН\linebreak был 
разработан и~внедрен в~эксплуатацию ряд \mbox{программ}, решающих отдельные 
задачи весовых рас\-че\-тов летательных аппаратов (ЛА)~[2--4]. В~настоящей статье 
представлено описание интегрированной АСВП, предназначенной для использования на всех 
этапах жизненного цикла изделий. Она разработана как интерактивная 
многопользовательская информационная система кли\-ент-сер\-вер\-ной 
архитектуры с~централизованной базой данных. Информационным ядром 
и~основой АСВП является единая струк\-тур\-но-па\-ра\-мет\-ри\-че\-ская весовая модель 
самолета, описание которой дает довольно полное представление о~задачах, 
решаемых с~помощью АСВП.

\section{Структурно-параметрическая весовая модель самолета}

  Самолет является сложным техническим объ\-ектом, состоящим из множества 
различных \mbox{ком\-понентов}, функционально и~конструктивно связанных между 
собой. Под струк\-тур\-но-па\-ра\-мет\-ри\-че\-ской весовой моделью самолета 
здесь понимается база данных, которая содержит всю необходи\-мую 
информацию для проведения комплекса расчетов 
МИХ и~мас\-со\-во-цент\-ро\-воч\-ных данных (МЦД) 
самолета. Весовая модель состоит из нескольких структур, ориентированных на 
определенные группы параметров и~задач весового проектирования. Ниже 
перечислены основные структуры весовой модели, реализованные в~системе 
АСВП:
  \begin{itemize}
\item дерево конструкции самолета;
\item иерархия систем координат, связанных с~самолетом и~его агрегатами;
\item геометрические структуры весовой модели самолета;
\item каталог целевой нагрузки, размещаемой во внут\-рен\-них отсеках и~на 
подвесках;
\item реестр допустимых вариантов загрузки само\-лета;
\item таблицы тарировочных характеристик топливных баков;
\item таблицы характеристик выработки топлива.
\end{itemize}


  \subsection{Дерево конструкции самолета}

  Дерево конструкции самолета является центральной структурой весовой 
модели, которая отражает членение изделия на его составные части~--- 
системы, агрегаты, узлы, детали. В~базе данных весовой модели эта структура 
представлена в~виде многоуровневого корневого дерева $W \hm= (U, V)$, где 
вершинам $U \hm= \{U_i\}$ соответствуют различные\linebreak
 элементы конструкции. 
Ориентированные дуги дере\-ва, идущие из~$U_i$ в~$U_j$, означают вхождение 
конструкции~$U_j$ в~конструкцию~$U_i$ в~качестве ее составной части. 
Терминальными или висячими вершинами дерева конструкции будем называть 
вершины, у которых нет ни одной подчиненной конструкции.
  
  Многолетний опыт самолетостроения выработал устоявшиеся 
конструктивные схемы самолетов различного назначения. Существуют 
отраслевые стандарты и~классификаторы, которые вводят определения 
основных элементов конструкции самолетов. На рис.~1 показан пример 
представления в~АСВП нескольких верхних уровней дерева конструкции 
самолета. 


    

  Существующие классификаторы отражают лишь самые общие принципы 
построения конструкции самолетов. Разумеется, каждый новый проект 
самолета имеет свои конструктивные особенности, которые отражаются на 
структуре весовой модели. Дерево конструкции строится постепенно, сверху 
вниз, в~течение всего процесса проектирования самолета. 

 { \begin{center}  %fig1
 \vspace*{9pt}
\mbox{%
 \epsfxsize=77.216mm 
 \epsfbox{fle-1.eps}
 }

\vspace*{6pt}


\noindent
{{\figurename~1}\ \ \small{Дерево конструкции самолета}}
\end{center}
}

\addtocounter{figure}{1}
  
  Понятие <<конструкции>> в~данном контексте используется и~для 
обозначения любой вершины графа, и~для всего поддерева подчиненных 
конструкций этой вершине. Каждая конструкция дерева имеет уникальное 
в~рамках весовой модели обозначение, которое может быть шифром, кодом, 
идентификатором или чертежным номером конструкции. Разумеется, для более 
полного и~наглядного вербального представления конструкции  
в~струк\-тур\-но-па\-ра\-мет\-ри\-че\-ской модели можно задать ее текстовое 
описание.
  
  \textbf{Масса конструкции.} Основную содержательную и~необходимую 
информацию весовой модели содержит список значений  
МИХ, соответствующих каждой 
вершине дерева конструкций. Центральным параметром является масса. 
  
  На разных стадиях создания самолета, когда неизвестно точное значение 
массы, прибегают к~различным оценкам.  
В~струк\-тур\-но-па\-ра\-мет\-ри\-че\-ской весовой модели фиксируются 
перечисленные ниже оценки массы, которые соответствуют разным этапам 
проектирования:
  \begin{description}
\item[\,]  $M_{\mathrm{теор}}$~--- теоретическая масса~--- оценка массы, вычисленная на 
основании некоторой математической модели конструкции; 
  
\item[\,]  $M_{\mathrm{лим}}$~--- лимитная масса конструкции, уста\-нав\-ли\-ва\-емая на 
основании теоретических оценок и~используемая для весового контроля 
в~процессе детальной разработки конструкции;
  
\item[\,]  $M_{\mathrm{черт}}$~--- чертежная масса конструкции, рассчитанная по чертежу или по 
электронной гео\-мет\-ри\-че\-ской модели конструкции;
  
\item[\,]  $M_{\mathrm{креп}}$~--- масса крепежа конструкции~--- дополнение к~чертежной массе, 
которое учитывает мелкие детали конструкции, предназначенные для 
соединения подчиненных деталей (заклепки, болты, гайки, шайбы и~т.\,п.). 
Введение такой дополнительной массы позволяет избавить дерево конструкции 
от десятков и~сотен тысяч вершин, которые несут относительно небольшую 
нагрузку в~весовых характеристиках, но тем не менее их учет необходим при 
контроле веса. Масса крепежа распределяется по подчиненным конструкциям;  
\item[\,]  $M_{\mathrm{факт}}$~--- фактическая масса изготовленной 
и~взвешенной конструкции. 
Фактическая масса может задаваться не только для изготавливаемых 
конструкций ЛА, но и~для готовых по\-став\-ля\-емых 
изделий при их установке на борту.
\end{description}
  
  Порядок задания оценок массы диктуется логикой развития проекта. 
В~дереве конструкции все оценки массы, кроме $M_{\mathrm{лим}}$ и~$M_{\mathrm{креп}}$, 
суммируются по подчиненным вершинам снизу вверх. Однако если для 
некоторых терминальных значений одна или несколько оценок не определены, 
например некоторые детали конструкции не изготовлены и, стало быть, 
$M_{\mathrm{факт}}$ не определена, то и~для всех вышестоящих конструкций эти оценки не 
определены. При задании $M_{\mathrm{лим}}$ это правило может не соблюдаться. 
  
  На основании оценок массы определяется то расчетное значение массы 
конструкции, которое используется во всех расчетах на текущей стадии 
проекта: 
  $M$~--- текущая масса конструкции. Значение текущей массы \textit{для 
нетерминальных} конструкций определяется суммированием по подчиненным 
конструкциям. \textit{Для терминальных} вершин дерева конструкций 
применяется процедура определения текущей массы по первому известному 
значению из следующего списка в~указанном порядке: $M_{\mathrm{факт}}$, 
$M_{\mathrm{черт}}$\;+\;$M_{\mathrm{креп}}$, $M_{\mathrm{теор}}$, $M_{\mathrm{лим}}$.
  
  \textbf{Геометрия масс конструкции.} Кроме собственно массы в~весовой 
модели задаются или вычисляются значения характеристик, которые принято 
называть характеристиками геометрии масс: 
  \begin{description}
  \item[\,] $X$, $Y$ и $Z$~--- положение центра масс конструкции; 
  \item[\,] $L_x$, $L_y$ и $L_z$~--- габаритные параметры конструкции;
  \item[\,] $I_x$, $I_y$ и $I_z$~--- полные плоскостные моменты инерции;
  \item[\,]  $I_{xy}$, $I_{xz}$ и $I_{yz}$~--- полные центробежные моменты 
инерции;
  \item[\,] $I^c_x$, $I^c_y$ и  $I^c_z$~--- собственные плоскостные моменты 
инерции:
  \begin{align*}
  I^c_x &= I_x - M X^2\,;\\ 
  I^c_y &= I_y - M Y^2\,;\\ 
  I^c_z &= I_z - M Z^2\,;
 \end{align*}
  \item[\,] $I^c_{xy}$, $I^c_{xz}$ и~$I^c_{yz}$~--- собственные центробежные 
моменты инерции:
 \begin{align*}
  I^c_{xy} &= I_{xy}- M X Y\,;\\
   I^c_{xz} &= I_{xz}- M X Z\,;\\
   I^c_{yz} &= I_{yz}- M Y Z\,;
\end{align*}
  \item[\,] $J_x$, $J_y$ и $J_z$~--- собственные осевые моменты инерции 
конструкции:
  \begin{align*}
  J_x &= I^c_y + I^c_z\,;\\ 
  J_y &= I^c_x + I^c_z\,;\\
   J_z &= I^c_y + I^c_x\,;
  \end{align*}
  \item[\,] СК~--- система координат конструкции, в~которой задаются 
характеристики геометрии масс.
  \end{description}
  
  \begin{figure*} %fig2
  \vspace*{1pt}
 \begin{center}
 \mbox{%
 \epsfxsize=162mm 
 \epsfbox{fle-2.eps}
 }
 \end{center}
\vspace*{-9pt}
  \Caption{Основные параметры конструкций весовой модели самолета}
  \end{figure*}
  
  Каждая конструкция привязывается к~одной из систем координат, которые 
описаны в~весовой модели. В~весовой модели изделия для удобства описания 
различных агрегатов может быть описано несколько систем координат. Все 
описанные сис\-те\-мы координат организованы в~иерархическую структуру. 
Считается предописанной глобальная самолетная система координат, в~которой 
могут быть заданы или вычислены координаты всех объектов весовой модели. 
Каждая система координат в~весовой модели задается уникальным именем, 
положением начала координат относительно вышестоящей системы координат 
и~тремя углами поворота относительно вышестоящей. 

Параметр, 
обозначенный как СК,~--- это имя одной из сис\-тем координат весовой модели. 
Если СК не задано, то считается, что характеристики гео\-мет\-рии масс заданы 
в~глобальной системе координат. Каж\-дая сис\-те\-ма координат весовой модели 
содержит матрицу преобразования координат из самолетной (глобальной) 
системы координат в~данную и~обратно. Эта матрица для каждой системы 
координат есть произведение локальных матриц преобразований 
в~соответствии с~положением данной системы в~иерархии систем координат. 
Любое изменение параметров систем координат требует пе\-ре\-вы\-чис\-ле\-ния 
матриц преобразования как измененной сис\-те\-мы, так и~всех подчиненных. На 
рис.~2 показана панель параметрического пред\-став\-ле\-ния конструкций весовой 
модели.
  
  Так же как и~масса, центры тяжести и~моменты инерции вычисляются снизу
вверх от терминальных конструкций к~вышестоящим. При этом осуществляется 
пересчет характеристик по заданной иерархии систем координат от 
нижестоящих к~вышестоящим и~к~самолетной системе координат. Расчет 
МИХ терминальных конструкций 
осуществляется на основании гео\-мет\-ри\-че\-ских моделей. Геометрические модели 
на этапе рабочего проекта строятся в~системах гео\-мет\-ри\-че\-ско\-го 
моделирования. В~процессе их построения автоматически вычисляются 
объемы, массы, положение центра тяжести и~другие характеристики гео\-мет\-рии 
масс. Рассчитанная в~системах гео\-мет\-ри\-че\-ско\-го моделирования масса 
с~по\-мощью соответствующих интерфейсных средств может быть загружена как 
$M_{\mathrm{черт}}$ в~весовую модель. (Раньше документация была представлена в~виде 
чертежей на бумажных носителях и~$M_{\mathrm{черт}}$ вручную вычислялась по этим 
чертежам.) Однако на более ранних этапах проектирования, когда еще не 
проработана гео\-мет\-рия многих элементов конструкции, весовые расчеты 
проводятся на основании эскизов и~наборов гео\-мет\-ри\-че\-ских и~конструктивных 
параметров агрегатов изделия. Для этого в~весовой модели должны быть 
предусмотрены средства параметрического представления гео\-мет\-рии 
конструкций. Геометрическое пред\-став\-ле\-ние конструкций 
в~автоматизированной системе весового проектирования выполняет 
и~немаловажную функцию визуализации конструкций, их компоновки, 
размещения нагрузки и~т.\,д. В~АСВП реализовано несколько форм 
представления гео\-мет\-ри\-че\-ской информации, ориентированных не только на 
расчет МИХ, но и~на визуализацию выполняемых расчетов. Это чертежи 
геометрических проекций изделия, это таб\-лич\-ное задание типовых 
геометрических конструкций, это каркасное представление трехмерных 
геометрических моделей и, наконец, задание объемных конструкций 
триангуляционной (фасеточной) поверхностью. Последний вид представления 
является наиболее перспективным для точного вычисления МИХ. В~АСВП он 
применяется для расчета тарировочных характеристик топливных баков, о~чем 
будет сказано ниже.
  
  \textbf{Классификационные признаки конструкции.} В~весовой модели 
кроме числовых параметров опре\-делен ряд классификационных признаков 
конструкций, по которым проводится весовой анализ.\linebreak
 Таки\-ми маркерами могут 
быть подразделения, ответст\-вен\-ные за разработку конструкции, поставщики 
или изготовители готовых изделий, принадлежность конструкции 
к~определенным функциональным системам, конструкционные материалы 
и~пр.
  
  \textbf{Функциональные подсистемы изделия.} Конст\-рук\-тив\-ное членение 
самолета может не совпадать с~его функциональной структурой. Отдельные\linebreak 
элементы функциональных подсистем самолета удобнее описывать в~составе 
конструкции ка\-ко\-го-ни\-будь агрегата планера. Например, некоторая деталь 
может конструктивно входить в~состав крыла, а принадлежать 
к~функциональной подсистеме гидравлики или электрооборудования. Для того 
чтобы иметь возможность выполнять весовые расчеты, учитывая разные 
подходы к~классификации конструкции самолета, в~АСВП отдельно от дерева 
конструкции ведется реестр подсистем, для которых может быть проведен 
специальный расчет весовых параметров. В~этом реестре ведется полный 
перечень конструкций весовой модели, входящих в~подсистемы реестра, 
независимо от того, в~какой ветви дерева конструкции они находятся. Любая 
конструкция может быть включена только в~одну из подсистем реестра. 
В~зависимости от режима расчетов МИХ
конструкций, входящих в~под\-сис\-те\-му, будут учтены либо в~со\-ста\-ве 
вышестоящих агрегатов дерева конструкции, либо отдельно в~под\-сис\-теме. 
{\looseness=1

}
  
  \textbf{Распределенные характеристики изделия.} Задача вычисления 
распределенных характеристик изделия является родственной задачей 
вычисления характеристик геометрии масс. Основное отличие состоит в~том, 
что в~данной задаче рассчитываются не интегральные характеристики 
распределения материала, а сама функция распределения массы по объему 
конструкции. Такие функции рассчитываются по заданному геометрическому 
разбиению конструкции на пространственные отсеки. Функции распределения 
массы по объему конструкции в~процессе проектирования используются при 
построении динамически подобных моделей для проведения некоторых видов 
испытаний и~продувок, а~также для выполнения прочностных расчетов. 
  
  Каждый отсек разбиения для расчета распределенных характеристик 
представляет собой вы\-пук\-лый многогранник, ограниченный конечным набором 
плоскостей. Задача построения распределенных весовых характеристик состоит 
в~вычислении для каждого отсека массы и~положения центра тяжести той части 
конструкции самолета, которая геометрически расположена внутри этого 
отсека. Эта задача решается путем нахождения геометрического пересечения 
каждой терминальной конструкции с~каждым отсеком разбиения, и~если такое 
пересечение не пусто, то вычисление массы и~центра тяжести той части 
конструкции, которая попадает в~отсек. Некоторые конструкции могут быть 
объявлены сосредоточенными массами. Использование сосредоточенных масс 
позволяет исключить все подчиненные конструкции из распределения по 
отсекам и~рассматривать их отдельно для задания сосредоточенных нагрузок. 
Список сосредоточенных масс с~уникальными именами представляет собой 
отдельную структуру весовой модели. Каждая сосредоточенная масса содержит 
список ссылок на конструкции весовой модели. Любая конструкция может 
быть включена не более чем в~одну сосредоточенную массу.
  
  \textbf{Весовые сводки.} Одной из основных задач \mbox{АСВП} является 
построение так называемых весовых сводок. Весовые сводки являются 
документами, сопровождающими построение весовой модели самолета 
в~процессе его создания. В АСВП реализовано несколько форм весовых 
сводок, которые с~разных сторон отражают дерево конструкции самолета или 
отдельных ветвей этого дерева. Назначение этих сводок и~форма представления 
зависят от ре\-ша\-емых задач. Весовые данные в~сводках могут быть 
представлены либо в~табличном виде, либо в~виде иерархии конструкций. 
Могут содержать информацию в~детализированном или в~укрупненном виде. 
Отдельные виды весовых сводок пред\-став\-ля\-ют распределенные 
характеристики по отсекам. Весовые сводки предназначены для решения задач 
весового контроля и~весового анализа. 
  
  Весовой контроль при проектировании самолетов представляет собой  
ор\-га\-ни\-за\-ци\-он\-но-тех\-ническую сис\-те\-му, нацеленную на создание 
конструк\-ции минимального веса. Для эффективного \mbox{весового} контроля 
необходима оперативная информация о текущей массе изделия и~любой его 
части. Весовая информация для весового контроля в~АСВП представляется 
в~виде оперативных весовых сводок по отдельным подразделениям 
предприятия. В~таких весовых сводках содержится информация о текущей, 
теоретической, лимитной,\linebreak чертежной и~фактической массах конструкций, 
разрабатываемых данным подразделением. Могут также выпускаться 
оперативные сводки по группе подразделений или по всему проекту. Сводки 
весового контроля предназначены для использования руководителями проекта.
  
  Весовой анализ также связан с~выпуском определенного вида весовых 
сводок. Для решения задач весового анализа в~АСВП осуществляется 
сортировка и~выборки конструкций по определенному классификационному 
признаку. Например, могут быть рассчитаны массы силового и~несилового 
набора конструкции, массы продольного и~поперечного набора, массы 
конструкций определенного материала, массы готовых изделий или изделий 
конкретного поставщика и~т.\,д. Весовой контроль и~анализ позволяют 
выявить резервы конструкции, узкие места, тренды в~изменении веса 
кон\-ст\-рук\-ции.
{\looseness=1

}
  
  \subsection{Постоянные и~переменные структуры весовой модели 
самолета}
  
  Дерево конструкции весовой модели готового изделия не является 
статической структурой. Конфигурация самолета зависит от конкретных 
условий его применения. Мас\-со\-во-инер\-ци\-он\-ные характеристики при 
взлете и~посадке отличаются от тех же характеристик в~полете, когда убраны 
стойки шасси. Конфигурация меняется и~в~полете у~самолетов с~изменяемым 
углом стреловидности или с~измененяемым вектором тяги. Текущая 
конфигурация является одним из параметров весовой модели и~параметров 
весовых расчетов. По самому смыс\-лу создания самолета как транспортного 
средства предполагается, что кроме собственно конструкции, которая 
обеспечивает выполнение основных задач, на его  
МИХ существенным образом влияет 
перевозимая нагрузка. Перевозимая нагрузка есть переменная часть структуры 
дерева конструкции. Самолетные весовые классификаторы выделяют 
постоянную часть массы изделия и~переменную, состоящую из снаряжения, 
топлива и~целевой нагрузки:
  \begin{multline*}
{M} = M_{\mathrm{пустого}} + 
M_{\mathrm{снаряжения}} + {}\\
{}+M_{\mathrm{топлива}} + 
M_{\mathrm{целевой\_нагрузки}}\,.
  \end{multline*}
  
  Все переменные и~постоянные компоненты самолета составляют единое 
целое, и~расчет мас\-со\-во-инер\-ци\-он\-ных и~центровочных характеристик 
допусти\-мых конфигураций является одной из главных задач проектирования 
самолетов любого назначения. Переменные структуры в~весовой модели могут 
задаваться альтернативными конструкциями, когда у некоторой вершины 
дерева есть несколько вариантов поддеревьев и~когда любой из вариантов, но 
только один из них, может быть активирован в~конкретный момент времени. 
Существует своя специфика задания переменных структур весовой модели для 
разных содержательных задач. 
  
  \textbf{Пустой самолет}~--- это постоянная часть конструкции самолета, 
которая не меняется в~процессе эксплуатации готового изделия. Компонентами 
пустого самолета являются конструкция планера самолета, силовая установка 
и~ее системы, другие самолетные системы, обеспечивающие управление 
самолетом, а~также специальные системы бортового оборудования, 
предназначенные для решения целевых задач самолета. В~процессе 
проектирования и~при эксплуатации самолетов рассматриваются различные 
варианты отдельных конструкций планера, а~чаще~--- различные варианты 
по\-став\-ля\-емых готовых изделий. В~связи с~этим в~весовой модели АСВП 
рассматриваются возможные комбинации вариантов пустого самолета, 
вариантов снаряжения и~полезной нагрузки. 

\begin{figure*} %fig3
\vspace*{1pt}
 \begin{center}
 \mbox{%
 \epsfxsize=162mm 
 \epsfbox{fle-3.eps}
 }
 \end{center}
\vspace*{-9pt}
\Caption{Тарировочная таблица топливного бака}
\end{figure*}
  
  \textbf{Снаряжение самолета} устанавливается на борту в~процессе 
предполетной подготовки. Снаряжение самолета принято разделять на 
основное и~дополнительное. Основное снаряжение включает несколько 
позиций. Это экипаж и~системы жизнеобеспечения экипажа, системы 
жизнеобеспечения пассажиров, заправляемые компоненты и~расходуемые 
материалы, несливаемый остаток топлива и~другие возможные компоненты. 
Использование различных вариантов экипажа и~другого снаряжения самолета 
связано с~различным характером выполняемых задач. Как правило, существует 
несколько типовых вариантов комплектации экипажа 
и~элементов снаряжения. Весовая модель должна содержать перечень 
альтернативных вариантов снаряжения и~их характеристик. Естественно, что 
этот перечень может модифицироваться. К~дополнительному снаряжению 
относят временное оборудование и~средства, связанные с~установкой на борту 
и~закреплением на подвесках перевозимых грузов. Временно устанавливаемое 
оборудование, как правило, связано со спецификой полетных заданий. Это 
может быть специальная измерительная аппаратура или оборудование, которое 
необходимо проверить в~условиях реального полета. Перечень такого 
оборудования и~его характеристики в~весовой модели должны быть 
пред\-став\-ле\-ны в~специальном реестре, или в~каталоге. Для установки 
оборудования, размещения любой коммерческой нагрузки и~вооружения в~конструкции самолета
должны быть  предусмотрены специальные места 
размещения и~узлы крепления. Точки размещения оборудования и~любых 
элементов целевой нагрузки задаются своими координатами и~установочными 
углами закрепления. 

\begin{figure*} %fig4
  \vspace*{1pt}
 \begin{center}
 \mbox{%
 \epsfxsize=162mm 
 \epsfbox{fle-4.eps}
 }
 \end{center}
\vspace*{-11pt}
\Caption{Варианты размещения целевой нагрузки самолета на подвесках}
\end{figure*}
  
  \textbf{Топливо}~--- величина переменная и~на земле, при подготовке 
самолета к~вылету, и~в~воздухе, при выработке топлива, и, если это 
предусмотрено, при дозаправке в~воздухе. Одной из самых сложных и~важных 
задач построения весовой модели является отражение изменяющихся в~полете  
МИХ топлива, находящегося 
в~топливных баках. Топливные баки современных ЛА
могут иметь довольно сложные геометрические формы. В~процессе выработки 
топлива все характеристики располагаемого запаса топлива меняются. 
Необходимо отслеживать эти изменения в~процессе произвольных допустимых 
эволюций траектории полета. Функции изменения МИХ в~зависимости от 
объема оставшегося топлива задаются тарировочными характеристиками баков. 
Для расчета тарировочных характеристик топливных баков при произвольных 
углах атаки, углах тангажа и~крена в~весовой модели наиболее удобно 
триангуляционное задание баков. В~тарировочной таблице вычисляется масса 
оставшегося топлива в~зависимости от уровня поверхности жидкости 
в~топливном баке. На рис.~3 приведен пример расчета тарировочной таблицы 
крыльевого топливного бака.



  Если МИХ топлива в~конкретном баке по 
мере его выработки определяются тарировочной характеристикой, то 
зависимость МИХ оставшегося топлива определяется последовательностью, 
в~которой осуществляется выработка из разных баков. Топливная система 
самолета состоит из нескольких баков~--- как внутренних, так и~размещенных 
на подвесках, а~также из системы трубопроводов, перекачивающих насосов и~управляющей автоматики. Основой управления расходом топлива является 
программа, определяющая порядок расходования топлива из разных баков. 
Переключение перекачки топлива между разными баками осуществляется для 
обеспечения центровки самолета в~заданных границах. Одним из критериев при 
разработке алгоритмов перекачки является число переключений и~обеспечение 
бесперебойной подачи топлива при любых допустимых параметрах траектории 
полета. Массово-инерционные характеристики топлива в~процессе тарировки 
баков задаются их разбиением плоскопараллельными сечениями на тонкие 
слои. Для каждого слоя указывается масса, координаты центра тяжести 
и~плоскостные моменты инерции. Программа выработки топлива пред\-став\-ля\-ет 
собой последовательность выработки слоев из разных баков в~соответствии 
с~диаграммой переключений. В~весовой модели может быть задано несколько 
вариантов программ расходования топлива. Разумеется, в~процессе выполнения 
полетного задания программа расходования топлива фиксирована. 
Предварительный расчет характеристик для разных вариантов порядка 
выработки топлива необходим для выбора наилучшего, удовле\-тво\-ря\-юще\-го 
всем ограничениям.
  
  \textbf{Целевая нагрузка} зависит от назначения самолета и~от конкретного 
полетного задания. Для пасса\-жирских самолетов целевая нагрузка~--- это 
пассажи\-ры с~багажом, для транспортных са\-мо\-летов~--- это коммерческие 
грузы, для военных~--- подвесное или размещаемое в~специальных \mbox{отсеках} 
вооружение. В~полете возможен сброс и~десантирование целевой нагрузки. 
Комплектация и~установка целевой нагрузки представляет собой довольно 
сложный процесс. Выбор состава грузов и~их размещение могут проходить 
в~несколько этапов. Сложность выбора обусловлена большим количеством 
типов перевозимой нагрузки, наличием большого числа вспомогательных 
специальных устройств закрепления грузов как во внутренних отсеках 
самолета, так и~на внешних подвесках. На рис.~4 приведена панель 
формирования различных расчетных вариантов целевой нагрузки самолета. 
Визуализация этого процесса существенно облегчает решение различных задач 
анализа допустимой нагрузки как на этапе проектирования самолета, так и~при 
эксплуатации во время подготовки полетных заданий.
  
  \begin{figure*} %fig5
\vspace*{1pt}
 \begin{center}
 \mbox{%
 \epsfxsize=162mm 
 \epsfbox{fle-5.eps}
 }
 \end{center}
\vspace*{-9pt}
\Caption{Область допустимых центровок}
\end{figure*}

  Для удобства выбора и~проведения расчетов множества вариантов загрузки 
самолета в~рамках весовой модели реализованы каталоги нагрузки~--- 
специального оборудования, коммерческой нагрузки, вооружения. В~этих 
каталогах ведутся клас\-си\-фи\-ка\-то\-ры, позволяющие в~громадных переч\-нях 
находить нужные позиции и~их характеристики. Кроме  
МИХ размещаемой нагрузки в~каталогах 
даются ссылки на их геометрические модели, задаются габариты, другие 
геометрические па\-ра\-мет\-ры. Эти данные нужны для визуализации размещения 
и~компоновки, для вычисления их МИХ. 
Как правило, существуют довольно жесткие ограничения на 
размещение нагрузки на борту, а~также на внешних узлах крепления. Эти 
ограничения должны указываться в~каталоге и~учитываться в~процессе 
формирования вариантов загрузки самолета. 
  
  Ограничения, которые проверяются при анализе различных вариантов 
снаряжения самолета, программы выработки топлива и~допустимых вариантов 
целевой нагрузки, задают область допустимых центровок самолета. 
  
  \textbf{Область допустимых центровок} является неотъемлемой частью 
весовой модели и~служит одной из основных весовых характеристик самолета, 
особенно важной и~контролируемой в~процессе его эксплуатации. На рис.~5 
проиллюстрированы ограничения, образующие область допустимых центровок, 
и~приведен график изменения центровки самолета при выработке топлива. 



  По оси абсцисс на этом графике откладывается центровка самолета, которая 
определяется как положение центра тяжести самолета на средней 
аэро\-ди\-на\-ми\-че\-ской хорде в~процентах от ее длины. По оси ординат 
откладывается текущая масса самолета с~учетом массы снаряжения, массы 
целевой нагрузки и~текущего запаса топлива. Точки излома на графиках 
центровки соответствуют моментам переключения подачи топлива с~одного 
бака на\linebreak другой, которые определяются программой выработки топлива или 
моментами сброса целевой нагрузки. Двойной график изменения центровки 
соответствует двум полетным конфигурациям~--- с~убранными 
и~выпущенными стойками шасси. Ограничения, которые обеспечивают 
устой\-чи\-вость и~управ\-ля\-емость полета, задаются предельными значениями 
центровки. Предельно передняя и~предельно задняя центровки на графике 
показаны вертикаль\-ными штриховыми линиями. Горизонтальные линии задают 
ограничения на взлетную и~посадочную массы. Ограничения максимальной 
взлетной и~посадочной массы при определенных условиях могут нарушаться, 
но эти нарушения допускаются в~исключительных условиях и~сказываются на 
ресурсных характеристиках самолета.\linebreak Превышение \textbf{предельных} 
значений взлетной и~посадочной массы не допускается. Наклонные штриховые 
линии на графике задают ограничения, связанные с~максимально допустимыми 
нагрузками на переднюю и~главную опоры шасси.  

\begin{figure*} %fig6
\vspace*{1pt}
 \begin{center}
 \mbox{%
 \epsfxsize=165mm 
 \epsfbox{fle-6.eps}
 }
 \end{center}
\vspace*{-9pt}
\Caption{Архитектура программной реализации исполнительных модулей АСВП}
\end{figure*} 

%\vspace*{-12pt}

\section{Программная реализация автоматизированной системы весового
проектирования}

  Представленная здесь струк\-тур\-но-па\-ра\-мет\-ри\-че\-ская весовая модель 
самолета позволяет решать широкий круг задач весового проектирования. 
Весовая модель составляет информационную основу,\linebreak на базе которой могут 
быть построены различные вычислительные программы и~пользовательские 
модули. Рассматриваемая в~данной работе АСВП построена по 
кли\-ент-сер\-вер\-ной архитектуре, где весовая модель служит единым хранилищем 
информации, базой данных системы. Непосредственно с~информацией, 
хранящейся в~этой базе данных, взаимодействуют различные вычислительные, 
расчетные программы~--- серверы, которые кроме расчетных функций 
обеспечивают информационную связь клиентских модулей с~весовой моделью 
самолета. Непосредственными пользователями клиентских модулей являются 
конструкторы и~проектировщики, решающие различные задачи весового 
проектирования.  Построена АСВП как многопользовательская интерактивная 
система. На рис.~6 представлена архитектура АСВП, ее основные программные 
и~информационные компоненты.




  Ниже перечислены основные функции программных модулей АСВП:
 \begin{description}
 \item[\,] 
Сервер ПУСТОЙ ЛА\;+\;Модуль расчета МИХ пус\-то\-го самолета:
\begin{itemize}
\item создание и~модификация дерева конструкции пустого самолета;
\item расчет МИХ пустого изделия, всех его сис\-тем, узлов, агрегатов и~деталей 
на любых уровнях дерева конструкции;
\item весовой анализ и~контроль текущего состояния проекта, выполнения 
лимитных ограничений по весу, осуществление выборок весовой информации 
по различным признакам~--- сис\-те\-мам, агрегатам, типу конструкции 
(си\-ло\-вая/не\-си\-ло\-вая),  материалу конструкции, подразделениям и~т.\,д.;
\item расчет распределения массы самолета по различным разбиениям на 
отсеки; эта информация используется для построения динамически подобных 
моделей и~при прочностных расчетах;
\item расчет МИХ при различных вариантах полетной конфигурации при 
убранных и~выпущенных стойках шасси, при отклонениях консолей крыла для 
самолетов с~из\-ме\-ня\-емой геометрией, при отклонении органов управления.
\end{itemize}
\begin{figure*} %fig7
\vspace*{1pt}
 \begin{center}
 \mbox{%
 \epsfxsize=155.86mm 
 \epsfbox{fle-7.eps}
 }
 \end{center}
\vspace*{-1pt}
\Caption{Проектный подход~--- технология ГП}
\vspace*{6pt}
\end{figure*}
 \item[\,]
Сервер НАГРУЗКА ЛА\;+\;Модуль расчета МИХ самолета с~переменной 
массой:
\begin{itemize}
\item создание и~модификация реестра допустимых вариантов нагрузки 
самолета;
\item расчеты МИХ снаряженного и~загруженного самолета для разных 
вариантов компоновки и~размещения на борту полезной нагрузки;
\item расчет изменения МИХ самолета в~полете при выработке топлива, 
дозаправке в~воздухе, сбросе нагрузки;
\item расчет МИХ самолета в~виде табличных зависимостей для различных 
вариантов снаряжения и~размещения нагрузки;
\item расчет МИХ самолета в~виде графических зависимостей от массы 
самолета и/или от массы топлива;
\item проверка выполнения установленных эксплуатационных ограничений по 
центровке, взлетной и~посадочной массе, нагрузке на опоры шасси для 
различных вариантов снаряжения и~размещения нагрузки; сигнализация 
в~случае нарушения ограничений, а~также для различных вариантов программ 
выработки топлива.
\end{itemize}

\pagebreak

 \item[\,]
Сервер КАТАЛОГ\;+\;Модуль ведения каталога элементов нагрузки:\\[-9pt]
\begin{itemize}
\item создание и~модификация каталога элементов целевой нагрузки самолета;\\[-9pt]
\item создание и~модификация базы данных вариантов размещения 
и~закрепления элементов нагрузки каталога на борту самолета или на подвесках;\\[-9pt]
\item создание и~модификация базы данных вспомогательных элементов 
конструкции установки элементов нагрузки.\\[-9pt]
\end{itemize}
 \item[\,]
Сервер ТОПЛИВО\;+\;Модуль расчета порядка выработки топлива:\\[-9pt]
\begin{itemize}
\item создание и~модификация базы данных различных вариантов программы 
выработки топлива;\\[-9pt]
\item расчет МИХ и~МЦД для различных вариантов переключения выработки 
топлива из внутренних, закладных и~подвесных баков;\\[-9pt]
\item расчет МИХ и~МЦД при различных программах заливки и~дозаправки 
топлива во внутренние, закладные и~подвесные баки.\\[-9pt]
\end{itemize}
 \item[\,]
Сервер БАКИ\;+\;Модуль расчета тарировки топливных баков:\\[-9pt]
\begin{itemize}
\item создание и~модификация базы данных гео\-мет\-рии топливных баков;\\[-9pt]
\item расчет тарировочных характеристик топливных баков при различных 
углах тангажа и~крена.\\[-9pt]
  \end{itemize}
  \end{description}
  
  Программная реализация АСВП велась с~использованием инструментального комплекса 
<<Генератор проектов>> (технология ГП)~\cite{5-fl}. Технология ГП 
обеспечивает возможность разработки приклад\-ных систем многоуровневой  
кли\-ент-сер\-вер\-ной архитектуры с~использованием реляционных и~сетевых 
баз данных со сложным пользовательским и~межпрограммным интерфейсом. 
Создание ин\-фор\-ма\-ци\-он\-но-вы\-чис\-ли\-тель\-ных сис\-тем в~рамках 
технологии ГП базируется на так называемом <<проектном подходе>>. Под 
проектом здесь понимается пакет документов (файлов), содержащий описание 
структуры проекта, описание логической структуры баз данных, спецификации 
пользовательского интерфейса, перечень команд и~сценарии работы 
пользователей, описание функций и~процедур обработки пользовательских 
запросов. Исходное описание проекта подается на вход <<Генератора 
проекта>>, который строит в~памяти модель проекта, осуществляет ее анализ 
на предмет корректности и~целостности, а затем на основании этой модели 
генерирует тексты программного кода для клиентских и~серверных компонент 
системы, а~так\-же ге\-нерирует утилиты, необходимые для сборки, инсталляции 
и~сопровождения системы. 

На рис.~7 показана общая архитектура 
программной конструкции, связанной с~применением технологии ГП.
  


  В приведенной цепочке разработчик прикладной информационной системы 
имеет дело только с~первым ее звеном~--- проектом системы. При этом он 
избавлен от необходимости иметь дело с~системным программным окружением 
вычислительной среды, в~которой должна функционировать разрабатываемая 
прикладная система. Все связи прикладных информационных процессов 
с~конкретной системной вычислительной средой привносит 
в~результирующую рабочую программу <<Генератор проектов>> на стадии 
анализа и~генерации итогового программного кода. Естественно, что при этом 
объем описания проекта оказывается существенно короче программного кода, 
который создается автоматически. Экономия трудозатрат разработчика 
оказывается существенной. В~частности, объем описания проекта АСВП на 
порядок меньше, чем объем сгенерированного программного кода. Даже если 
предположить, что написанный вручную программный код благодаря 
искусству программистов будет весьма экономным, то все равно трудоемкость 
разработки прикладных систем будет в~разы меньше. 

Но главное даже не 
в~числе строк программ, а~прежде всего в~экономии интеллектуальных затрат 
разработчиков прикладных систем и,~в~итоге, автоматически созданные 
программы более надежны и~свободны от нечаянных ошибок и~опечаток.\linebreak 
И~кроме того, разрабатываемые в~рамках технологии ГП прикладные системы 
обеспечивают-\linebreak\vspace*{-12pt}

\pagebreak

\noindent
ся эффективными средствами сопровождения, т.\,е.\linebreak достаточно 
простой процедурой внесения ис\-прав\-ле\-ний и~развития программ в~процессе их 
эксплу\-а\-тации. 

Прикладные программные комплексы в~рамках технологии ГП 
разрабатываются как автономные системы и~не требуют для своей работы 
специальной среды и~дорогостоящих программных продуктов (кроме 
использующихся систем управления базами данных
(СУБД) и~общесистемного обеспечения). Разрабатываемые 
в~рамках технологии ГП прикладные системы допускают масштабирование 
и~портирование на различные вычислительные платформы и~СУБД.
  
  \bigskip
  
  Как уже говорилось, система АСВП разрабатывалась в~течение ряда лет, 
многие ее компоненты и~версии были апробированы и~использовались 
в~реальном проектировании. 
%
Авторы выражают благодарность 
С.\,И.~Скобелеву, М.\,К.~Курьянскому, Д.\,Ю.~Стрель\-цу, П.\,В.~Плунскому 
и~К.\,Н.~Ерасову за плодотворные обсуждения проблем весового проектирования 
самолетов, за постановку многих задач и~за апробацию разработанных 
программ.

%\vspace*{-12pt}

{\small\frenchspacing
 {%\baselineskip=10.8pt
 \addcontentsline{toc}{section}{References}
 \begin{thebibliography}{9}
\bibitem{1-fl}
\Au{Шейнин В.\,М., Козловский~В.\,И.} Весовое проектирование и~эффективность 
пассажирских самолетов.~--- М.: Машиностроение, 1977.   Т.~1. 343~с.

%\columnbreak

\bibitem{2-fl}
\Au{Скобелев С.\,И., Широков~Н.\,И.} Весовой анализ и~контроль в~САПР ЛА~// Задачи 
и~методы автоматизированного проектирования.~--- М.: ВЦ РАН, 1991. С.~92--100.
\bibitem{3-fl}
\Au{Широков Н.\,И.} Автоматизированная система весовых расчетов в~САПР ЛА~// 
Автоматизация проектирования инженерных и~финансовых информационных систем 
средствами Генератора проектов~/ Отв. ред. Ю.\,А.~Флеров.~--- М.: ВЦ РАН, 
2010. С.~55--66.

\vspace*{6pt}

\bibitem{4-fl}
\Au{Вышинский Л.\,Л., Широков~Н.\,И.} Система автоматизации расчетов 
массово-инерционных характеристик ЛА с~переменной массой~// Развитие и~применение 
инструментального комплекса Генератор проектов~/ Отв. ред. Ю.\,А.~Флеров.~--- 
М.: ВЦ РАН, 2014. С.~20--31.
{\looseness=1

}

\vspace*{6pt}

\bibitem{5-fl}
\Au{Вышинский Л.\,Л., Гринев~И.\,Л., Флеров~Ю.\,А., Широков~А.\,Н., Широков~Н.\,И.} 
Генератор проектов~--- инструментальный комплекс для разработки  
<<кли\-ент-сер\-вер\-ных>> сис\-тем~// Информационные технологии и~вычислительные 
системы, 2003. №\,1-2. С.~6--25.
 \end{thebibliography}

 }
 }

\end{multicols}

\vspace*{-6pt}

\hfill{\small\textit{Поступила в~редакцию 24.05.17}}

\vspace*{8pt}

%\newpage

%\vspace*{-24pt}

\hrule

\vspace*{2pt}

\hrule

%\vspace*{8pt}


\def\tit{COMPUTER-AIDED SYSTEM OF~AIRCRAFT WEIGHT DESIGN}

\def\titkol{Computer-aided system of~aircraft weight design}

\def\aut{L.\,L.~Vyshinsky, Yu.\,A.~Flerov, and~N.\,I.~Shirokov}

\def\autkol{L.\,L.~Vyshinsky, Yu.\,A.~Flerov, and~N.\,I.~Shirokov}

\titel{\tit}{\aut}{\autkol}{\titkol}

\vspace*{-9pt}


\noindent
A.\,A.~Dorodnicyn Computing Centre, Federal Research Center ``Computer Science and 
Control'' of the Russian Academy of Sciences,  40~Vavilov Str., Moscow 119333, Russian 
Federation 



\def\leftfootline{\small{\textbf{\thepage}
\hfill INFORMATIKA I EE PRIMENENIYA~--- INFORMATICS AND
APPLICATIONS\ \ \ 2018\ \ \ volume~12\ \ \ issue\ 1}
}%
 \def\rightfootline{\small{INFORMATIKA I EE PRIMENENIYA~---
INFORMATICS AND APPLICATIONS\ \ \ 2018\ \ \ volume~12\ \ \ issue\ 1
\hfill \textbf{\thepage}}}

\vspace*{3pt}
   

\Abste{The article is devoted to the problems of computer-aided weight design of 
aircraft. Weight and mass-inertial parameters are one of the basic values that affect 
the performance characteristics of aircraft. The informational basis of the system is 
the weight model of the aircraft. The paper describes the structure of the weight 
model and its individual components. The program implementation of the system, 
which is executed within the framework of the client-server architecture, is shown. 
The automated system of weight design is implemented using the software tool 
complex ``Project Generator'' (GP technology), which was developed at the 
Computing Centre of the Russian Academy of Sciences. The creation of information 
and computing systems within the framework of the GP technology is based on the 
so-called ``project approach,'' when the formal description of the system 
automatically generates code for the client and server components of the system.}

\KWE{math modeling; design automation; aircraft; weight design; weighting model; 
design tree; project generator; code generation; client-server architecture}

  \DOI{10.14357/19922264180103} 

%\vspace*{-12pt}

%\Ack
%\noindent




%\vspace*{8pt}

  \begin{multicols}{2}

\renewcommand{\bibname}{\protect\rmfamily References}
%\renewcommand{\bibname}{\large\protect\rm References}

{\small\frenchspacing
 {%\baselineskip=10.8pt
 \addcontentsline{toc}{section}{References}
 \begin{thebibliography}{9} 
 
 %\vspace*{-6pt}
 
 \bibitem{1-fl-1}
\Au{Sheynin, V.\,M., and V.\,I.~Kozlovskiy}. 1977. \textit{Vesovoe 
proektirovanie i~effektivnost' passazhirskikh samoletov} [Weight design and 
efficiency of passenger aircraft]. Moscow: Mechanical Engineering. Vol.~1. 343~p.
\bibitem{2-fl-1}
\Aue{Skobelev, S.\,I., and N.\,I.~Shirokov.} 1991. Vesovoy analiz i~kontrol' v~SAPR 
LA [Weight analysis and control in CAD of aircraft]. \textit{Zadachi i~metody 
avtomatizirovannogo proektirovaniya} [Tasks and methods of computer-aided 
design]. Moscow: Computing Centre of the USSR Academy of Sciences.  
92--100.
\bibitem{3-fl-1}
\Aue{Shirokov, N.\,I.} 2010. Avtomatizirovannaya sistema vesovykh raschetov 
v~SAPR LA [Automated system weight calculations in CAD].  
\textit{Avtomatizatsiya proektirovaniya inzhenernykh i~finansovykh 
informatsionnykh system sredsvami Generatora proektov} [Computer 
aided  design of engineering and financial information systems by the means of the 
Project Generator]. Moscow: Computing Centre of RAS. 
55--66.
\bibitem{4-fl-1}
\Aue{Vyshinskiy, L.\,L., and N.\,I.~Shirokov.} 2014. Sistema avtomatizatsii 
raschetov massovo-inertsionnykh kharakteristik LA s~peremennoy massoy [CAD 
system of calculation  aircraft mass-inertial characteristics with variable mass].  
\textit{Razvitie i~primenenie instrumental'nogo kompleksa Generator proektov} 
[The development and application of a tool set Project Generator]. 
Moscow: Computing Centre of RAS. 20--31.
{\looseness=1

}

\bibitem{5-fl-1}
\Aue{Vyshinskiy, L.\,L., I.\,L.~Grinev, Yu.\,A.~Flerov, A.\,N.~Shirokov, and 
N.\,I.~Shirokov.} 2003. Generator proektov~--- instrumental'nyy kompleks dlya 
razrabotki ``klient--servernykh'' sistem [The project generator~--- tool complex for 
development of ``client--server'' systems]. 
\textit{Informatsionnye tekhnologii i~vychislitel'nye sistemy} [Information 
Technologies and Computer Systems] 1-2:6--25.

\end{thebibliography}

 }
 }

\end{multicols}

\vspace*{-6pt}

\hfill{\small\textit{Received May 24, 2017}}

%\vspace*{-10pt}

\Contr

\noindent
\textbf{Vyshinsky Leonid L.} (b.\ 1941)~--- Candidate of Sciences (PhD) in physics and 
mathematics, Head of Laboratory, A.\,A.~Dorodnicyn Computing 
Centre, Federal Research Center ``Computer Science and Control'' of the Russian 
Academy of Sciences, 40~Vavilov Str., Moscow 119333, Russian Federation; 
\mbox{Wysh@ccas.ru} 

\vspace*{3pt}

\noindent
\textbf{Flerov Yuri A.} (b.\ 1942)~--- Corresponding Member of the Russian 
Academy of Science, Doctor of Science in physics and mathematics, professor, 
Deputy Director, A.\,A.~Dorodnicyn Computing Centre, Federal Research Center 
``Computer Science and Control'' of the Russian Academy of Sciences, 40~Vavilov 
Str., Moscow 119333, Russian Federation; \mbox{fler@ccas.ru}

\vspace*{3pt}

\noindent
\textbf{Shirokov Nikolai I.} (b.\ 1963)~--- Candidate of Sciences (PhD) in physics and 
mathematics, senior scientist, A.\,A.~Dorodnicyn Computing Centre, Federal 
Research Center ``Computer Science and Control'' of the Russian Academy of 
Sciences, 40~Vavilov Str., Moscow 119333, Russian Federation; 
\mbox{Wysh@ccas.ru} 



\label{end\stat}


\renewcommand{\bibname}{\protect\rm Литература}  %1
\def\stat{sinits}

\def\tit{АНАЛИТИЧЕСКОЕ МОДЕЛИРОВАНИЕ  РАСПРЕДЕЛЕНИЙ С~ИНВАРИАНТНОЙ
МЕРОЙ В~СТОХАСТИЧЕСКИХ СИСТЕМАХ С~РАЗРЫВНЫМИ ХАРАКТЕРИСТИКАМИ$^*$}

\def\titkol{Аналитическое моделирование  распределений с~инвариантной
мерой в~стохастических системах} % с~разрывными характеристиками}

\def\autkol{И.\,Н.~Синицын}

\def\aut{И.\,Н.~Синицын$^1$}

\titel{\tit}{\aut}{\autkol}{\titkol}

{\renewcommand{\thefootnote}{\fnsymbol{footnote}}\footnotetext[1]
{Работа выполнена при финансовой поддержке РФФИ
(проект №\,13-07-00036) и программой <<Интеллектуальные информационные 
технологии, системный анализ и автоматизация>> (проект~1.7).}}

\renewcommand{\thefootnote}{\arabic{footnote}}
\footnotetext[1]{Институт проблем информатики Российской академии наук, sinitsin@dol.ru}



\Abst{На базе методов нормальной аппроксимации и статистической линеаризации разработаны 
точные и приближенные алгоритмы аналитического моделирования плотностей стохастических 
режимов с инвариантной мерой в гауссовых и негауссовых стохастических системах (СтС)
с разрывными 
характеристиками. Рассмотрены особенности моделирования в СтС с 
пуассоновскими шумами. На тестовых примерах показана достаточная для многих приложений 
точность алгоритмов.}

\KW{автокоррелированная помеха; аналитическое моделирование;
интегродифференциальные уравнения Пугачёва; метод нормальной аппроксимации;
метод статистической линеаризации; нелинейная гауссовская и негауссовская стохастическая система в смысле Ито;
пуассоновская стохастическая сис\-те\-ма; распределение с инвариантной мерой;
стохастический режим}

\vskip 14pt plus 9pt minus 6pt

      \thispagestyle{headings}

      \begin{multicols}{2}

            \label{st\stat}



\section{Введение}

Следуя [1, 2], будем рассматривать нестационарный стохастических режим $Z\hm=Z(t)$ 
в нелинейной дифференциальной СтС, понимаемой в смысле Ито:
    \begin{equation}
    \dot Z = a(Z,t) + b (Z,t) V\,, \enskip Z(t_0) = Z_0\,.
    \label{e1.1-sin}
    \end{equation}
Здесь $Z$~--- $k$-мер\-ный вектор состояния СтС, $Z\hm\in \Delta$ ($\Delta$~--- 
многообразие состояний); $a\hm=a(Z,t)$ и $b\hm= b(Z,t)$~--- детерминированные  
$(k\times 1)$- и $(k\times m)$-мер\-ные  функции  отмеченных аргументов; 
$V\hm=V(t)$~--- $m$-мер\-ный вектор негауссовских (в общем случае) белых шумов 
с нулевыми математическими ожиданиями и представляющий собой среднеквадратичную 
(с.к.)\ производную процесса с независимыми приращениями  $W\hm=W(t)$, 
$V\hm=\dot W$. Обозначим через $\chi\hm=\chi(\mu;t)$ логарифмическую производную 
одномерной характеристической функции $h_1\hm=h_1(\mu;t)$ процесса $W\hm=W(t)$, определяемую формулой
    \begin{equation}
    \chi(\mu;t)=\fr{\prt \ln h_1 (\mu;t)}{\prt t}=
    \fr{1}{h_1(\mu;t)}\,\fr{\prt h_1(\mu;t)}{\prt t}\,.
    \label{e1.2-sin}
    \end{equation}

Начальное состояние $Z_0$ будем считать случайной величиной (СВ), не зависящей 
от $W(t)$ для $t\hm>t_0$. Предположим, что стохастический режим $Z(t)$ является 
сильным решением~(\ref{e1.1-sin}), а функции $a,b$ и $\chi$ удовлетворяют известным 
условиям существования и единственности~[1, 2].

Пусть существуют одно- и $n$-мерные плот\-ности\linebreak $f_1\hm=f_1(z;t)$ и 
$f_n\hm= f_n(z_1\tr z_n; t_1 \tr t_n)$ и характеристические функции $g_1\hm=g_1(\la;t)$ и 
$g_n\hm=g_n(\la_1\tr \la_n; t_1\tr t_n)$ $(n\hm\ge 2)$, удовлетворяющие 
интегродифференциальным уравнениям\linebreak Пугачева~[1, 2]:
    \begin{multline}
    \fr{\prt f_1(z;t)}{\prt t}+\fr{\prt^{\mathrm{T}}}{\prt z}\lk a(z,t)f_1(z;t)\rk = 
\fr{1}{(2 \pi)^k} \times{}\\
{}\times \iin\iin \chi(b(\xi,t)^{\mathrm{T}}\la;t) e^{i\la^{\mathrm{T}}(\xi-z)} f_1(z;t) \,
    d\xi d\la\,;
    \label{e1.3-sin}
    \end{multline}
   \begin{equation}
    f_1(z;t_0)=f_0(z)\,;\label{e1.4-sin}
    \end{equation}
    
    \vspace*{-12pt}
    
    \begin{multline*}
\fr{\prt f_n(z_1\tr z_n;t_1\tr t_n)}{\prt t_n}+{}\\
{}+\fr{\prt^{\mathrm{T}}}{\prt z_n}\left[
a(z_n, t_n) f_n (z_1\tr z_n; t_1\tr t_n)\right]={}\\
{}= \fr{1}{(2\pi)^{kn}} \iin\iin \chi(b(\xi_n, t_n)^{\mathrm{T}} \la_n;t_n) \times{}\\
{}\times \exp\lf i \sss_{l=1}^n \la_l^{\mathrm{T}} (\xi_l-z_l)\rf \times{}\\
{}\times
f_n (\xi_1\tr \xi_n; t_1\tr t_n)\,d\xi_1\cdots d\xi_n d\la_1\cdots d\la_n\,;
%\label{e1.5-sin}
\end{multline*}

\vspace*{-12pt}

\begin{multline*}
f_n(z_1\tr z_{n-1},z_n;t_1\tr t_{n-1},t_{n})={}\\
{}= f_{n-1} (z_1\tr z_{n-1};t_1\tr t_{n-1})\delta (z_n - z_{n-1})\,;
%\label{e1.6-sin}
\end{multline*}
       
        
\noindent        
\begin{multline}
\fr{\prt g_1 (\la;t)}{\prt t} -{}\\
{}-\fr{1}{(2\pi)^k} \iin \iin i\la^{\mathrm{T}} a (z,t) 
e^{i(\la^{\mathrm{T}} -\mu^{\mathrm{T}})z} g_1 (\mu;t)\, d\mu dz={}\\
{}=\fr{1}{(2\pi)^k} \iin \iin \chi(b(z,t)^{\mathrm{T}} \la^{\mathrm{T}};t) 
e^{i(\la^{\mathrm{T}} -\mu^{\mathrm{T}})z} \times{}\\
{}\times
g_1 (\mu;t)\, d\mu dz\,;
\label{e1.7-sin}
\end{multline}
\begin{equation}
g_1(\la;t_0) = g_0(\la)\,\,; \label{e1.8-sin}
\end{equation}

\vspace*{-12pt}

\begin{multline*}
\fr{\prt g_n (\la_1\tr \la_n; t_1\tr t_n)}{\prt t_n} -{}\\
{}-
\fr{1}{(2\pi)^{kn}} \iin \cdots \iin i\la^{\mathrm{T}} a (z_n,t_n) \times{}\\
{}\times \exp \lk i \sss\limits_{k=1}^n (\la_k^{\mathrm{T}} - \mu_k^{\mathrm{T}}) z_k\rk \times{}\\
{}\times g_n 
(\mu_1\tr \mu_n; t_1\tr t_n)\, d\mu_1 \cdots d \mu_n dz_1\cdots dz_n={}\\
{}= \fr{1}{(2\pi)^{kn}} \iin\cdots \iin \chi (b(z_n;t)^{\mathrm{T}} \la_n;t_n)\times{}\\
{}\times\exp \lk i \sss_{k=1}^n (\la_k^{\mathrm{T}} - \mu_k^{\mathrm{T}}) z_k\rk \times{}\\
{}\times g_n 
(\mu_1\tr \mu_n; t_1\tr t_n) \,d\mu_1 \cdots d \mu_n dz_1\cdots dz_n;
%\label{e1.9-sin}
\end{multline*}

\vspace*{-12pt}

\noindent
\begin{multline*}
g_n (\la_1\tr \la_n; t_1\tr t_{n-1},t_{n-1})= {}\\
{}=
g_{n-1} (\la_1\tr \la_{n-2},\la_{n-1}+\la_n; t_1\tr t_{n-1})\,, %\label{e1.10-sin}
\end{multline*}
 $$
        t_1\le t_2 \le \cdots \le t_n,\enskip n=2,3,\ldots
        $$

При этом одно- и $n$-мер\-ные плотности и характеристические функции связаны 
между собой известными соотношениями:
\begin{equation*}
f_1(z;t) = \fr{1}{(2\pi)^{k}} \iin e^{-i\mu^{\mathrm{T}} z} g_1(\mu;t) d\mu\,; %\label{e1.11-sin}
    \end{equation*}
      \begin{equation*}
   g_1(\la;t) = \iin e^{i\la^{\mathrm{T}} z} f_1(z;t)\, dz\,; %\label{e1.12-sin}
   \end{equation*}
   
   \vspace*{-12pt}

\noindent
\begin{multline}
f_n( z_1\tr z_n; t_1\tr t_n) ={}\\
{}=
\fr{1}{(2\pi)^{kn}} 
\iin\cdots \iin \exp \lf - i \sss_{l=1}^n \la_l^{\mathrm{T}} z_l\rf \times{}\\
{}\times g_n (\la_1\tr \la_n; t_1\tr t_n)\, d\la_1\cdots d\la_n\,;\label{e1.13-sin}
\end{multline}


\vspace*{-12pt}

\noindent
\begin{multline*}
g_n (\la_1\tr \la_n; t_1\tr t_n) ={}\\
{}=\iin\cdots \iin \exp\lf i \sss_{l=1}^n \la_l^{\mathrm{T}} z_l\rf \times{}\\
{}\times f_n (z_1\tr z_n; t_1\tr t_n)\, dz_1\cdots dz_n\,. %\label{e1.14-sin}
\end{multline*}

Для нахождения одномерных плотностей $f_1(z,t) \hm= f_1^* (z)$ и характеристических функций 
$g_1(\la;t) \hm= g_1^* (\la)$ стохастических режимов в стационарных СтС~(\ref{e1.1-sin}) при
    \begin{equation}
    a(z,t) = a^*(z)\,;\ b(z,t)=b^*(z)\,;\ \chi(\mu;t)= \chi^*(\mu)
    \label{e1.15-sin}
    \end{equation}
следует в~(\ref{e1.3-sin}) и~(\ref{e1.7-sin}) положить 
$\prt f_1/\prt t \hm= 0$ и $\prt g_1/ \prt t \hm=0$. В~результате получим соответственно
\begin{multline*}
\fr{\prt^{\mathrm{T}}}{\prt z}\lk a^* (z) f_1^* (z)\rk = {}\\
{}=
\fr{1}{(2\pi)^k} \iin \iin \chi^* (b^*(\xi)^{\mathrm{T}} \la) e^{i\la^{\mathrm{T}}(\xi-z)} f_1^* (\xi)\, d\xi d\la\,;
%\label{e1.16-sin}
\end{multline*}

\vspace*{-12pt}

\noindent
\begin{multline*}
-\fr{1}{(2\pi)^k} \iin  \iin i\la^{\mathrm{T}} a^*(z) e^{i(\la^{\mathrm{T}}-\mu^{\mathrm{T}})z} g_1^*(\mu)\, d\mu dz={}\\
{}=\fr{1}{(2\pi)^k} \iin  \iin \chi^*(b^*(z)^{\mathrm{T}}\la) e^{i(\la^{\mathrm{T}}-\mu^{\mathrm{T}})z} g_1^*(\mu)\, d\mu dz.
%\label{e1.17-sin}
\end{multline*}
Поставим задачу разработки точных и приближенных  алгоритмов
аналитического моделирования распределений (плотностей и
характеристических функций) стохастических режимов  $Z\hm=Z(t)$ в
нелинейных гауссовских и негауссовских СтС~(\ref{e1.1-sin})  с разрывными
характеристиками $a\hm=a(z,t)$ и $b\hm=b(z,t)$, обладающих свойством
сохранения инвариантной меры, т.\,е.\ удовлетворяющих уравнениям~(\ref{e1.3-sin})
и~(\ref{e1.7-sin}) при $\chi\hm=0$.

Условия сохранения инвариантной меры можно представить в следующем развернутом виде:
\begin{equation}
\left.
\begin{array}{c}
\displaystyle\fr{\prt f_1 (z;t)}{\prt t} + A_a f_1 (z;t) =0\,;\\[9pt] 
\hspace*{-4.5mm}\displaystyle A_a f_1(z;t) = 
    \fr{\prt^{\mathrm{T}}}{\prt z} \lk a(z,t) f_1(z;t)\rk = \mathrm{div}\, \pi(z;t)\,;
    \end{array}
    \right\}
    \label{e1.18-sin}
    \end{equation}
\begin{equation}
\left.
\begin{array}{c}
A_a^* f_1^*(z) =0\,;\\[9pt]
\displaystyle A_a^* f_1^* (z) = \fr{\prt^{\mathrm{T}}}{ \prt z} \lk a^* 
(z) f_1^* (z)\rk =\mathrm{div}\, \pi^* (z)\,;
\end{array}
\right\}
\label{e1.19-sin}
\end{equation}
$$
\fr{\prt g_1 (\la;t)}{\prt t} - B_a g_1(\la;t) =0\,;
$$

\vspace*{-14pt}

\noindent
\begin{multline}
B_a g_1(\la;t) ={}\\[2pt]
{}=\fr{1}{(2\pi)^k} \iin\iin i\la^{\mathrm{T}} a(z,t) e^{i(\la^{\mathrm{T}}-\mu^{\mathrm{T}})z}
 g_1(\mu;t)\, d\mu dz={}\\[2pt]
{}= \iin i\la^{\mathrm{T}} a(z,t) e^{i\la^{\mathrm{T}}z} f_1(z;t)\, dz={}\\[2pt]
{}= \iin e^{i\la^{\mathrm{T}} z} i\la^{\mathrm{T}} \pi(z;t)\, dz\,;
\label{e1.20-sin}
\end{multline}

\vspace*{-9pt}

\noindent
\begin{equation}
\left.
\begin{array}{c}
\hspace*{-45mm}B_a^* g_1^* (\la)=0\,;\\[12pt]
\hspace*{-48mm}B_a^* g_1^* (\la) = {}\\[10pt]
\hspace*{-3mm}{}=\fr{1}{(2\pi)^k} \iin\! i\la^{\mathrm{T}} a^* (z) e^{i(\la^{\mathrm{T}} -\mu^{\mathrm{T}})z} g_1^* (\mu)\, d\mu dz={}\\[10pt]
{}=\displaystyle\iin\! i\la^{\mathrm{T}} a^*(z) e^{i\la^{\mathrm{T}}z} f_1^* (z)\, dz = {}\\[10pt]
\displaystyle{}=
\iin e^{i\la^{\mathrm{T}} z} i\la^{\mathrm{T}} \pi^* (z)\, dz\,.
\end{array}
\right\}
\label{e1.21-sin}
\end{equation}
Для гауссовских (нормальных) СтС с гладкими характери\-стиками точные и приближенные 
методы  и алгоритмы аналитического моделирования рассмотрены в~[1--15]. 

Особое внимание 
уделим приближенным методам, основанным на методах нор\-маль\-ной аппроксимации и статистической 
линеаризации. Подробно рассмотрим их применение к пуассоновским СтС.



\section{Точные методы и~алгоритмы аналитического моделирования распределений 
с~инвариантной мерой}

Пусть функция~$a$ в СтС~(\ref{e1.1-sin}) допускает пред\-став\-ле\-ние
\begin{equation}
a= a(z,t) = a_1(z,t) +a_2 (z,t) \label{e2.1-sin}
\end{equation}
такое, что функция  $f_1\hm=f_1(z;t)$ является плот\-ностью инвариантной меры 
невозмущенной шумами системы, описываемой векторным обыкновенным дифференциальным 
уравнением вида
   \begin{equation}
   \dot z = a_1 (z,t)\,,\label{e2.2-sin}
   \end{equation}
т.\,е.\ удовлетворяет условию~(\ref{e1.18-sin}):
\begin{equation}
\fr{\prt f_1 (z;t)}{\prt t}+ \fr{\prt^{\mathrm{T}}}{\prt z} \lk a_1 (z,t) f_1(z;t)\rk =0\,.
\label{e2.3-sin}
\end{equation}

Для гладких функций $a_1\hm=a_1(z,t)$ вопросы существования и основные свойства 
интегральных 
 инвариантов изучены в~\cite{16-sin, 17-sin}. При этом в~(\ref{e2.1-sin}) 
функция $a_2 \hm= a_2(z,t)$ определяется путем решения следующего интегродифференциального 
уравнения:
\begin{multline}
\fr{\prt^{\mathrm{T}}}{\prt z}\lk a_2 (z,t) f_1(z;t) \rk 
=
\fr{1}{(2\pi)^k}\times{}\\
{}\times \iin\iin \chi(b(\xi,t)^{\mathrm{T}} \la;t) 
e^{i\la^{\mathrm{T}}(\xi-z)} f_1(\xi;t)\, d\xi d\la\,.\label{e2.4-sin}
\end{multline}
В общем случае нахождение функций $a_1$ и~$a_2$ в~(\ref{e2.1-sin})~--- такая же
трудная задача, как решение уравнений~(\ref{e1.3-sin}) и~(\ref{e1.4-sin}).

Для стационарных СтС, когда выполнены условия~(\ref{e1.15-sin}), 
уравнения~(\ref{e2.1-sin})--(\ref{e2.4-sin}) имеют вид:
\begin{align}
a(z)&= a_1(z) + a_2(z)\,;\label{e2.5-sin}
\\
\dot z &= a_1(z)\,,\label{e2.6-sin}
\\
\fr{\prt^{\mathrm{T}}}{\prt z}\lk a_2^*(z) f_1^*(z)\rk &= {}\notag\\
&\hspace*{-28mm}{}=
\fr{1}{(2\pi)^k} \!\!\iin \iin\!\! \chi^* (b^*(\xi)^{\mathrm{T}} \la) 
e^{i\la^{\mathrm{T}}(\xi-z)} f_1^*(\xi)\, d\xi d\la\,.\!\!\!\label{e2.7-sin}
\end{align}
В этом случае можно выбирать невозмущенную сис\-те\-му~(\ref{e2.6-sin}) так, чтобы
она имела первые интегралы.

В терминах характеристических функций соотношения~(\ref{e2.3-sin}), (\ref{e2.4-sin})
и~(\ref{e2.7-sin}) могут быть записаны следующим образом:

\noindent
\begin{equation}
\fr{\prt g_1 (\la;t)}{\prt t} - B_{a_1} g_1(\la;t) =0\,;\label{e2.8-sin}
\end{equation}
\begin{equation*}
B_{a_1}^* g_1^*(\la) =0\,. %\label{e2.9-sin}
\end{equation*}
Для составляющих $a_2(z,t)$ и $a_2^*(z)$ имеют место уравнения
\begin{multline}
B_{a_2} g_1(\la;t) 
= \fr{1}{(2\pi)^k} \times{}\\
\hspace*{-2.5mm}{}\times\iin\iin \!\chi(b(z,t)^{\mathrm{T}} \la;t) 
e^{i(\la^{\mathrm{T}}-\mu^{\mathrm{T}})z} g_1(\mu;t) \,d\mu dz;\label{e2.10-sin}
\end{multline}

\vspace*{-16pt}

\noindent
\begin{multline}
B_{a_2}^* g_1^*(\la) 
= \fr{1}{(2\pi)^k} \times{}\\
{}\times\iin\iin 
\chi^*(b^*(z)^{\mathrm{T}} \la) e^{i(\la^{\mathrm{T}}-\mu^{\mathrm{T}})z} g_1^*(\mu)\, d\mu dz\,.
\label{e2.11-sin}
\end{multline}

Отсюда вытекают конструктивные точные алгоритмы аналитического
моделирования распределений с инвариантной мерой. В~их основе лежат
следующие теоремы.

%\pagebreak

\medskip

\noindent
\textbf{Теорема~2.1.} \textit{Функция $f_1\hm=f_1(z;t)$ будет решением}~(\ref{e1.3-sin})
\textit{и}~(\ref{e1.4-sin}) \textit{тогда и только тогда, когда $a\hm=a(z,t)$ допускает
представление}~(\ref{e2.1-sin}) \textit{такое, что $f_1\hm=f_1(z;t)$ является плотностью
инвариантной меры обыкновенного дифференциального уравнения}~(\ref{e2.2-sin}),
\textit{т.\,е.\ удовле\-тво\-ря\-ет условию}~(\ref{e2.3-sin}). \textit{При этом со\-став\-ля\-ющая $a_2$
определяется из решения интегродифференциального уравнения}~(\ref{e2.4-sin}).

\medskip

\noindent
\textbf{Теорема~2.2.} \textit{Функция $f_1^*\hm=f_1^*(z)$ будет решением}~(\ref{e1.3-sin}) 
\textit{тогда и только тогда, когда $a^*\hm=a^*(z)$ допускает
представление}~(\ref{e2.5-sin}) \textit{такое, что $f_1^*\hm=f_1^*(z)$ является плотностью
инвариантной меры}~(\ref{e2.6-sin}). \textit{При этом составляющая $a_2^{*}$
определяется из решения  уравнения}~(\ref{e2.7-sin}).

\medskip

\noindent
\textbf{Теорема~2.3.} \textit{Функция $g_1\hm=g_1(\la;t)$ будет ре\-ше\-нием}~(\ref{e1.7-sin}), 
(\ref{e1.8-sin}) \textit{тогда и только тогда, когда недиф\-фе\-ренцируемая функция
$a\hm=a(z,t)$  допускает пред\-став\-ление}~(\ref{e2.1-sin}) \textit{такое, что
$g_1\hm=g_1(\la;t)$ является ха\-рак\-теристической функцией инвариантной
меры \mbox{уравнения}}~(\ref{e2.2-sin}), \textit{т.\,е.\ удовлетворяет условию}~(\ref{e2.8-sin}). 
\textit{При этом составляющая $a_2$ определяется из уравнения}~(\ref{e2.10-sin}).

\medskip

\noindent
\textbf{Теорема 2.4.} \textit{Функция $g_1^*\hm=g_1^*(\la)$  будет решением}~(\ref{e1.13-sin}) 
\textit{тогда и только тогда, когда недифференцируемая функция $a^*\hm=a^*(z)$  
допускает представление}~(\ref{e2.5-sin}) \textit{такое, что $g_1^*$ является  
характеристической функцией инвариантной меры}~(\ref{e2.2-sin}). 
\textit{При этом $a_2^*$ определяется из решения}~(\ref{e2.11-sin}).

\smallskip

Теоремы~2.1--2.4 легко обобщаются на случай многомерных распределений с инвариантной мерой.

\section{Приближенные методы и~алгоритмы аналитического моделирования распределений 
с~инвариантной мерой, основанные на~нормальной аппроксимации и статистической линеаризации}

Пусть нелинейная СтС~(\ref{e1.1-sin}) допускает применение метода нормальной аппроксимации 
(МНА)~[1, 2]. Тогда одно- и двумерные нормальные плот\-ности $f_1^{\mathrm{МНА}}$,
 $f_2^{\mathrm{МНА}}$ и характеристические функции  $g_1^{\mathrm{МНА}}$,  
 $g_2^{\mathrm{МНА}}$, а также вектор математического ожидания $m_t = M^{\mathrm{МНА}} Z(t)$, 
 ковариационная мат\-ри\-ца $K_t \hm= M^{\mathrm{МНА}} Z^{0\mathrm{T}} Z^0 (t)$ 
 $(Z^0 (t) \hm= Z(t) \hm- m_t)$ и матрица ковариационных функций 
 $K(t_1, t_2) \hm= M^{\mathrm{МНА}} Z^{0\mathrm{T}} (t_1) Z^0 (t_2)$ $(t_1\hm< t_2)$ определяются 
 следующими уравнениями:
    \begin{multline}
    f_1^{\mathrm{МНА}} = f_1^{\mathrm{МНА}} (z;t, m_t, K_t) =
    \lk (2\pi)^k |K_t|\rk^{-1/2}\times{}\\
    {}\times \exp \lf -  \fr{1}{ 2} 
    \left(z^{\mathrm{T}} - m_t^{\mathrm{T}}\right) K_t^{-1}(z-m_t)\rf\,;\label{e3.1-sin}
    \end{multline}
    
    \vspace*{-12pt}
    
    \noindent
\begin{multline}
f_2^{\mathrm{МНА}} ={}\\
= f_2^{\mathrm{МНА}} (z_1, z_2;t_1, t_2, m_{t_1}, m_{t_2}, K_{t_1}, K_{t_2}, K(t_1, t_2))=\\
{}=\lk (2\pi)^k |\bar K_2|\rk^{-1/2}\times{}\\
\hspace*{-2mm}{}\times \exp \lf - 
([z_1^{\mathrm{T}} z_2^{\mathrm{T}}] - \bar m_2^{\mathrm{T}}) 
\bar K_2^{-1}([z_1^{\mathrm{T}} z_2^{\mathrm{T}}]^{\mathrm{T}}-\bar m_2)\rf;
\!\!\label{e3.2-sin}
\end{multline}
\begin{equation}
g_1^{\mathrm{МНА}} (\la;t)=
\exp\lf i\la^{\mathrm{T}} m- \fr{1}{2}\,\la^{\mathrm{T}} K_t \la\rf\,;\label{e3.3-sin}
\end{equation}

\vspace*{-12pt}

\noindent
\begin{multline}
g_2^{\mathrm{МНА}} (\la_1, \la_2; t_1,t_2) ={}\\
{}= \exp \lf i \bar \la^{\mathrm{T}} \bar m_2 - 
    \fr{1}{2} \,\bar \la^{\mathrm{T}} \bar K_2 \bar \la\rf\,;\label{e3.4-sin}
    \end{multline}
$$
    \bar \la =\lk \la_1^{\mathrm{T}} \la_2^{\mathrm{T}}\rk^{\mathrm{T}}\,;\enskip 
    \bar m_2 =\lk m_{t_1}^{\mathrm{T}} m_{t_2}^{\mathrm{T}}\rk^{\mathrm{T}}\,;
    $$
    $$
    \bar K_2 =\begin{bmatrix}
        K(t_1, t_1)&K(t_1, t_2)\\[3pt]
        K(t_2, t_1)& K(t_2, t_2)
        \end{bmatrix}\,;
        $$
  \begin{multline}
  \dot m_t = a_1 (t, m_t, K_t) ={}\\
  {}=\iin a(z,t) f_1^{\mathrm{МНА}} (z; t, m_t, K_t) \,dz\,;
  \label{e3.5-sin}
  \end{multline}

\vspace*{-12pt}

\noindent
\begin{multline}
\dot K_t = a_2(t, m_t, K_t) = a_{21} + a_{12}+a_{22}={}\\
{}=\left[ \iin a(z,t) (z^{\mathrm{T}}-m_t^{\mathrm{T}}) + (z-m_t) a^{\mathrm{T}} (z,t) +{}\right.\\
\left.{}+ \sigma (z,t)
\vphantom{\iin}\right] f_1^{\mathrm{МНА}} (z;t, m_t, K_t)\, dz\,;
\label{e3.6-sin}
\end{multline}

\vspace*{-12pt}

\noindent
\begin{multline}
\fr{\prt K(t_1, t_2)}{\prt t_2} ={}\\
{}= a_3 (t_1, t_2, m_{t_1},m_{t_2}, K_{t_1}, K_{t_2}, K(t_1,t_2))={}\\
{}=\lk (2\pi)^{2k} |\bar K_2|\rk^{-1/2}\times{}\\
{} \times\iin\iin (z_1-m_{t_1}) a(z_2, t_2)
\exp\left\{ - ([z_1^{\mathrm{T}} z_2^{\mathrm{T}}]-\bar m_2^{\mathrm{T}})\times{}\right.\\
\left.{}\times\bar K_2^{-1} 
([z_1^{\mathrm{T}} z_2^{\mathrm{T}}]-\bar m_2)\right\} dz_1 dz_2\,.
\label{e3.7-sin}
\end{multline}
Здесь введены следующие обозначения:
\begin{equation}
\left.
\begin{array}{c}
z_1=z_{t_1}\,;\enskip  z_2=z_{t_2}\,;\enskip \bar m_2 =\lk m_{t_1}^{\mathrm{T}} m_{t_2}^{\mathrm{T}}\rk^{\mathrm{T}}\,;\\[9pt]
\displaystyle \bar K_2 =\begin{bmatrix}
        K(t_1,t_1)&K(t_1, t_2)\\[3pt]
        K(t_2, t_1)& K(t_2, t_2)
        \end{bmatrix}\,,
        \end{array}
        \right\}
        \label{e3.8-sin}
        \end{equation}
\begin{equation}
\sigma(z,t) = b(z,t) \nu(t) b(z,t)^{\mathrm{T}}\,,\label{e3.9-sin}
\end{equation}
где $\nu=\nu(t)$~--- интенсивность негауссовского белого шума $V\hm=V(t)$.

Для стационарных СтС  при $\dot m^* \hm=0$, $\dot K^* \hm=0$, 
$K(t_1, t_2)\hm= k(\tau)$ $(\tau\hm=t_1-t_2)$  соотношения~(\ref{e3.5-sin})--(\ref{e3.9-sin}) 
принимают вид:
\begin{equation}
a_1^* (m^*, K^*) =0\,;\label{e3.10-sin}
\end{equation}
\begin{equation}
    a_2^*(m^*, K^*) =0\,;\label{e3.11-sin}
    \end{equation}
    \begin{equation}
    \fr{dk(\tau) }{d\tau} = a_{11}^{\mathrm{МНА}} (m^*, K^*) k(\tau)\,;\label{e3.12-sin}
    \end{equation}
$$
k(\tau) = k(-\tau^{\mathrm{T}})\,;\enskip k(0)=K\,.
$$
Из уравнения~(\ref{e3.12-sin}) следует, что алгоритм МНА будет устойчивым, если матрица 
$a_{11}^{\mathrm{МНА}} (m_t, K_t, t)$ будет асимптотически устойчива.

Для $m$ и $K$ уравнения метода статистической линеаризации (МСЛ) в 
нелинейных СтС  при аддитивных шумах, когда $b(z,t) \hm= b_0(t)$, $b^*(z)\hm=b_0^*$ 
получаются из~(\ref{e3.5-sin})--(\ref{e3.7-sin}) и (\ref{e3.10-sin})--(\ref{e3.12-sin}) 
как частный случай.

Условия наличия нормального распределения с инвариантной мерой~(\ref{e1.18-sin}) 
и~(\ref{e1.19-sin}), если заменить $a(z,t)$ статистически
линеаризованным выраже\-нием
\begin{equation*}
    a(Z,t)\approx a_{10}^{\mathrm{МНА}} (t, m_t, K_t) + a_{11}^{\mathrm{МНА}} (t, m_t, K_t) 
    (Z-m_t)\,, %\label{e3.13-sin}
    \end{equation*}
где
\begin{equation*}
a_{10}^{\mathrm{МНА}} =a_{10}^{\mathrm{МНА}} (t, m_t, K_t)\equiv a_1\,; %\label{e3.14-sin}
\end{equation*}
    
    
   
    \noindent
    \begin{multline*}
    a_{11}^{\mathrm{МНА}}=a_{11}^{\mathrm{МНА}} (t, m_t, K_t) = {}\\
    {}=\lk \iin a(z,t) (z^{\mathrm{T}}-m_t^{\mathrm{T}}) 
        f_1^{\mathrm{МНА}} (z; t , m_t, K_t)\, dz\rk\times{}\\
        {}\times K_t^{-1} 
=\left(\fr{\prt}{\prt m_t} a_1^{\mathrm{T}}\right)^{\mathrm{T}}\,, %\label{e3.15-sin}
\end{multline*}
приводят к следующим соотношениям:
        \begin{multline}
\fr{\prt f_1^{\mathrm{МНА}} (z; t, m_t, K_t)}{\prt t} +\fr{\prt^{\mathrm{T}}}{ \prt z} 
\left\{ \left[ a_{10}^{\mathrm{МНА}} (t, m_t, K_t) 
+{}\right.\right.\\
\left.{}+ a_{11}^{\mathrm{МНА}} (t, m_t, K_t) (z-m_t) \vphantom{a_{10}^{\mathrm{МНА}}}
\right]\times{}\\
\left.{}\times 
     f_1^{\mathrm{МНА}} ( z; t , m_t, K_t)\right\} =0\,;
     \label{e3.16-sin}
     \end{multline}
     
     
     \noindent
\begin{multline}
\hspace*{-9.81628pt}\fr{\prt^{\mathrm{T}}}{\prt z} \left\{ \left[ a_{10}^{*{\mathrm{МНА}}}(m^*, K^*) + 
 a_{11}^{*{\mathrm{МНА}}}(m^*, K^*) (z-m^*)\right] \times{}\right.\\
\left.{}\times f_1^{*{\mathrm{МНА}}}(z; m^*, K^*)\right\} =0\,,\label{e3.17-sin}
 \end{multline}
где
\begin{multline*}
f_1^{*{\mathrm{МНА}}} (z; m^*, K^*) = \lk (2\pi)^k |K^*|\rk^{-1/2}\times{}\\
{}\times \exp \lf -
    \fr{1}{2} (z^{\mathrm{T}}-m^{*\mathrm{T}})(K^*)^{-1} (z-m^*)\rf\,.
    \end{multline*}

Аналогично в развернутом виде выписываются условия~(\ref{e1.20-sin}) и~(\ref{e1.21-sin}):
\begin{multline}
\fr{\prt g_1^{\mathrm{МНА}} (\la;t)}{\prt t} -\iin i\la^{\mathrm{T}} \left[ a_{10}^{\mathrm{МНА}} 
    (m_t, K_t, t) +{}\right.\\[2pt]
\left.    {}+ a_{11}^{\mathrm{МНА}} (m_t, K_t, t) (z- m_t) \right]\times{}\\[2pt]
{}\times e^{i\la^{\mathrm{T}} z} f_1^{\mathrm{МНА}} (z; m_t, K_t, t)\, dz=0\,;\label{e3.18-sin}
\end{multline}


\noindent
\begin{multline}
\iin i\la^{\mathrm{T}} \left[ a_{10}^{*{\mathrm{МНА}} } (m^*, K^*) 
+{}\right.\\[2pt]
\left.{}+a_{11}^{*{\mathrm{МНА}} } 
    (m^*, K^*) (z-m^*)\right]\times{}\\[2pt]
    {}\times
     e^{i\la^{\mathrm{T}}z} f_1^{*{\mathrm{МНА}} } (z; m^*, K^*)\, dz =0\,.
    \label{e3.19-sin}
    \end{multline}

Отсюда вытекают следующие теоремы.

\bigskip

\noindent
\textbf{Теорема~3.1.}\ \textit{Если существуют одно- и двумерные  плотности
стохастического режима, а  матрица $a_{11}^{\mathrm{МНА}}$ коэффициентов
статистической (нормальной) линеаризации асимптотически устойчива,
то приближенный алгоритм аналитического моделирования МНА
нестационарных стохастических режимов в СтС}~(\ref{e1.1-sin}) \textit{с инвариантной
мерой определяется выражениями}~(\ref{e3.1-sin})--(\ref{e3.7-sin}) и~(\ref{e3.16-sin}).

\bigskip

\noindent
\textbf{Теорема 3.2.}\ \textit{Если существуют стационарные одно- и
двумерные плотности стохастического режима, а матрица
$a_{11}^{*{\mathrm{МНА}}}$  коэффициентов статистической (нормальной)
линеаризации асимптотически устойчива, то приближенный алгоритм
аналитического моделирования стационарных стохастических режимов с
инвариантной мерой в стационарной СтС}~(\ref{e1.1-sin}) \textit{определяется 
выражениями}~(\ref{e3.10-sin})--(\ref{e3.12-sin}) и~(\ref{e3.17-sin}).

\bigskip

Как известно~[1, 2], одно- и двумерные нормальные распределения
определяют и все  $n$-мер\-ные распределения $(n\hm\ge 3)$, поэтому МНА и
МСЛ дают приближенные алгоритмы для любых многомерных плотностей
стохастических режимов, если они существуют. Аналогично
формулируются теоремы~3.3 и~3.4 на основе условий~(\ref{e3.18-sin}) и~(\ref{e3.19-sin}).


\section{О других приближенных методах и~алгоритмах аналитического моделирования 
распределений с~инвариантной мерой}

\vspace*{-2pt}

 Обобщением МНА являются различные
приближенные методы, основанные на параметризации распределений~[1, 2].
Аппроксимируя одномерную характеристическую функцию $g_1 (\la;t)$
и соответствующую плотность $f_1 (z,t)$ известными функциями
 $g_1^* (\la;\theta)$, $f_1^* (z;\theta)$,  зависящими от
конечномерного векторного параметра~$\theta$, можно свести задачу
приближенного определения одномерного распределения к выводу из
уравнения для характеристических функций обыкновенных
дифференциальных уравнений, определяющих~$\theta$ как функцию
времени. Это относится и к остальным многомерным распределениям.
При аппроксимации многомерных распределений целесообразно выбирать
последовательности функций $\{ f_n^* (z_1,\ldots,z_n;\theta_n)\}$ и 
$\{g_n^* (\la_1\tr \la_n;\theta_n)\}$, каждая пара
которых находилась бы в такой  зависимости от векторного параметра~$\theta_n$, 
чтобы при любом~$n$ множество параметров, образующих
вектор~$\theta_n$, включало в качестве подмножества множество
параметров, образующих вектор~$\theta_{n-1}$. Тогда при
аппроксимации $n$-мер\-но\-го распределения придется определять только
те координаты вектора~$\theta_n$, которые не были определены ранее
при аппроксимации функций $f_1, g_1\tr f_{n-1}$, $g_{n-1}$.

В зависимости от того, что представляют собой параметры, от
которых зависят функции $f_n^* (z_1\tr z_n;\theta_n)$ и 
$g_n^* (\la_1\tr \la_n;\theta_n)$, аппрок-\linebreak симирующие неизвестные
многомерные плотности $f_n (z_1,  \ldots,z_n; t_1 \tr t_n)$ и
характеристические функции $g_n (\la_1\tr \la_n; t_1,\ldots,t_n)$,
используются различные приближенные методы решения
 уравнений при условиях~(9)--(12), определяющих\linebreak многомерные
распределения вектора состояния сис\-те\-мы~$X_t$, в частности методы
моментов (ММ), семиинвариантов (МСИ), ортогональных разложений
(МОР), квазимоментов (МКМ) и~др.~[1, 2].

\vspace*{-6pt}


\section{Обобщение на~случай стохастических систем с~автокоррелированными шумами}

\vspace*{-2pt}

Пусть  СтС описывается нелинейным, в общем случае векторным дифференциальным 
стохастическим уравнением Ито~\cite{1-sin, 2-sin, 15-sin, 18-sin}

\noindent
\begin{equation}
\left.
\begin{array}{c}
    \dot Z = a(Z,t) + b_U(Z,t) U\,;\\[6pt] 
\displaystyle    \sss_{i=0}^l \alpha_i U^{(i)} =
\displaystyle\sss_{j=0}^h \beta_j V^{(j)}\enskip (h<l)\,.
\end{array}
\right\}
    \label{e5.1-sin}
    \end{equation}
    Здесь $U=U(t)$~--- векторная помеха размерности  $m\times 1$; $V\hm=V(t)$~--- 
    негауссовский белый шум с нулевым математическим ожиданием и известной функцией  
    $\chi\hm=\chi(\mu;t)$. В~таком случае в за\-ви\-си\-мости от степени <<гладкости>> 
    стохастического режима $Z\hm=Z(t)$ и помехи $U\hm=U(t)$ уравнения~(\ref{e5.1-sin})  
    путем расширения вектора состояния согласно~[1, 2] приводятся к виду~(\ref{e1.1-sin}) 
    для расширенного вектора состояния~$\bar Z$. Тогда, но уже для расширенного вектора 
    состояния СтС, при решении уравнений~(\ref{e5.1-sin}) могут быть использованы точные 
    (разд.~2) и приближенные (разд.~3) методы и алгоритмы аналитического моделирования 
    нестационарных и стационарных распределений с инвариантной мерой.

\section{Особенности аналитического моделирования распределений с~инвариантной мерой 
в~пуассоновских стохастических системах}

Рассмотрим СтС~(\ref{e1.1-sin}) при $b(z,t) \hm=I_m$ для обобщенного пуассоновского 
белого шума  $V^{\mathrm{OP}}\hm=  V^{\mathrm{OP}}(t)$, когда функция~(\ref{e1.2-sin}) 
определяется формулой
\begin{equation*}
\chi^{\mathrm{OP}} (\mu;t) =\lk g_c^{\mathrm{OP}} (\mu) -
1\rk \nu^{\mathrm{OP}} (t)\,, %\label{e6.1-sin}
\end{equation*}
где $g_c^{\mathrm{OP}} \hm=g_c^{\mathrm{OP}} (\mu)$~--- характеристическая 
функция скачков; $\nu^{\mathrm{OP}} \hm= \nu^{\mathrm{OP}} (t)$~--- 
интенсивность пуассоновского белого шума 
$V^{\mathrm{OP}}\hm=V^{\mathrm{OP}} (t)$. Обозначим через $f_c^{\mathrm{OP}} \hm=
 f_c^{\mathrm{OP}} (z)$ плотность скачков обобщенного пуассоновского процесса. 
 Тогда~(\ref{e1.3-sin}) будет представлять собой известное уравнение Фел\-ле\-ра--Кол\-мо\-го\-ро\-ва
\begin{multline}
\fr{\prt f_1(z;t)}{\prt t} + \fr{\prt^{\mathrm{T}}}{\prt z} 
    \lk a(z,t) f_1(z;t)\rk ={}\\
    \hspace*{-3mm}{}= \nu^{\mathrm{OP}} (t) \lk \iin f_c^{\mathrm{OP}} (z-\xi) f_1 (\xi;t)\, d\xi - f_1(z;t)\rk
    \label{e6.2-sin}
    \end{multline}
с начальным условием~(\ref{e1.4-sin}). В~случае простого пуассоновского белого шума 
с единичными скачками $g_c (\mu) \hm= e^{i\mu}$.

Для  стационарной пуассоновской СтС~(\ref{e1.1-sin}) уравнение~(\ref{e6.2-sin}) имеет следующий вид:
\begin{multline}
\fr{\prt^{\mathrm{T}}}{\prt z} \lk a^* (z) f_1^* (z)\rk = {}\\
{}=
\nu^{\mathrm{OP} *} \lk \iin f_c^{\mathrm{OP}} (z-\xi) f_1^* (\xi)\, d\xi- 
f_1^* (z)\rk\,.\label{e6.3-sin}
\end{multline}

Пользуясь уравнениями~(\ref{e6.2-sin}), (\ref{e6.3-sin})  
и результатами разд.~1 и~2, нетрудно сформулировать следующие утверждения.

\medskip

\noindent
\textbf{Теорема 6.1.}\ \textit{Функция $f_1 \hm= f_1(z;t)$ будет
нестационарным решением}~(\ref{e6.2-sin}), (\ref{e1.4-sin}) \textit{тогда и только тогда, 
когда $a$ допускает представление}~(\ref{e2.1-sin}) \textit{такое, что $f_1$ является плот\-ностью
инвариантной меры обыкновенного дифференциального уравнения}~(\ref{e2.2-sin}),
\textit{т.\,е.\ удовле\-тво\-ря\-ет условию}~(\ref{e2.3-sin}), \textit{а составляющая $a_2$ определяется
из решения следующего уравнения}:
\begin{multline*}
    \fr{\prt^{\mathrm{T}}}{\prt z} \lk a_2 (z,t) f_1 (z;t)\rk =
     \fr{1}{(2\pi)^k}\times{}\\
     {}\times \iin\iin \chi^{\mathrm{OP}} 
    \left(b(\xi,t)^{\mathrm{T}} \la;t\right) e^{i\la^{\mathrm{T}}(\xi-z)} f_1(\xi,t)\,d\xi d\la\,.
%    \label{e6.4-sin}
    \end{multline*}

%\smallskip

\noindent
\textbf{Теорема 6.2.}\ \textit{Функция $f_1^* \hm= f_1^* (z)$ будет стационарным 
решением}~(\ref{e6.3-sin}) \textit{тогда и только тогда, когда $a_2^*$ допускает 
представление}~(\ref{e2.5-sin}) \textit{такое, что  $f_1^*$ является плот\-ностью 
инвариантной меры}~(\ref{e2.6-sin}), \textit{а составляющая $a_2^{*}$ определяется 
из решения следующего уравнения}:
\begin{multline*}
\fr{\prt^{\mathrm{T}} }{\prt z} \lk a_2^{*} (z) f_1^* (z)\rk ={}\\
{}=
    \fr{1}{(2\pi)^k} \iin\iin \chi^{\mathrm{OP} *} (b(\xi)^{\mathrm{T}} \la) 
    e^{i\la^{\mathrm{T}}(\xi-z)} f_1^*(\xi)\,d\xi d\la\,.
%    \label{e6.5-sin}
    \end{multline*}

При использовании МНА и МСЛ для пуассоновских СтС непосредственно применяются теоремы~3.1--3.4, 
причем в формулу~(\ref{e3.9-sin}) для  
$\sigma(z,t)$ входит интенсивность 
$\nu^{\mathrm{OP}} (t)$ обобщенного пуассоновского белого шума.

\section{Тестовые примеры}

\noindent
\textbf{Пример~1}. Рассмотрим осциллятор Дуффинга в обобщенной пуассоновской 
стохастической среде:
\begin{equation}
\ddot X +\w^2 X -\mu X^3 =-\delta^{\mathrm{OP}} \dot X + V^{\mathrm{OP}} (t)\,.\label{e7.1-sin}
\end{equation}
Уравнения МСЛ для~(\ref{e7.1-sin}) имеют следующий вид:
\begin{equation}
\dot m_X = m_{\dot X}\,;\enskip 
\dot m_{\dot X} =- \w_{\mathrm{э}}^2 m_X -\delta^{\mathrm{OP}} m_{\dot X}\,;
\label{e7.2-sin}
\end{equation}
    \begin{equation}
    \left.
    \begin{array}{rl}
    \dot D_{X} &= 2 K_{X\dot X}\,;\\[6pt] 
    \dot D_{\dot X} &=\nu^{\mathrm{OP}} - 2 (\w_{1 \mathrm{э}}^2 K_{X\dot X} + 
    \delta^{\mathrm{OP}} D_{\dot X})\,;\\[6pt]
\dot K_{X\dot X} &= D_{\dot X} -\w_{1 \mathrm{э}}^2 D_X - 
\delta^{\mathrm{OP}} K_{X\dot X}\,.
\end{array}
\right\}
 \label{e7.3-sin}
\end{equation}
Здесь кубическая функция $X^3$ была заменена на статистически линеаризованную при 
гауссовом распределении с дисперсией  $D_X$ согласно~[1, 2]:
\begin{equation*}
X^3 \approx k_0 (m_X, D_X) m_X + k_1 (m_X, D_X) X^0\,,\label{e7.4-sin}
\end{equation*}
где
\begin{align*}
k_0 (m_X, D_X) &= m_X^2 + 3 D_X\,;\\ 
k_1 (m_X, D_X) &= 3 (m_X^2 + D_X)\,;\\
%\label{e7.5-sin}
\w_{\mathrm{э}}^2 &=\w^2 \lk 1- \fr{\mu (m_X^2 + 3D_X)}{\w^2}\rk\,;\\
\w_{1 \mathrm{э}}^2 &=\w^2 \lk 1-  \fr{3\mu (m_X^2 + D_X)}{\w^2}\rk \enskip 
(\w_{\mathrm{э}}>\w_{1 \mathrm{э}})\,.
\end{align*}
%\label{e7.6-sin}
Из~(\ref{e7.2-sin}) и~(\ref{e7.3-sin}) в стационарном режиме имеем:
\begin{gather*}
m_X^* =0\,;\enskip 
m_{\dot X}^* =0\,;\enskip 
K_{X\dot X}^* =0\,;\\
D_{\dot X}^* =\vartheta\,;\enskip 
\vartheta =  \fr{\nu^{\mathrm{OP}}}{ 2\delta^{\mathrm{OP}}}\,,
\end{gather*}
%\label{e7.7-sin}
а $D_X^*$ определяется из уравнения:
    \begin{equation*}
    \w_{1 \mathrm{э}}^2 (D_X^*) D_X^* =\vartheta\,. %\label{e7.8-sin}
    \end{equation*}
Условие наличия стационарного распределения с инвариантной мерой~(\ref{e3.17-sin}) 
требует консерватизма линеаризованной левой части~(\ref{e7.1-sin}). 
Процесс установления стационарных стохастических колебаний происходит 
в два этапа: сначала устанавливается $D_{\dot X}^*$, а затем $D_X^*$.

Интересно отметить, что уравнения МСЛ~(\ref{e7.2-sin}) и~(\ref{e7.3-sin}) сохраняют свой
вид и для любого белого шума интенсивности  $\nu(t)$,
представляющего собой с.к., производную от произвольного процесса с
независимыми приращениями~$W(t)$. Для гауссовского белого шума
$\nu\hm=\nu^G$ соответствующие результаты получены в~\cite{1-sin, 2-sin, 15-sin}. Как
показали вычислительные эксперименты для значений~$\mu$, отвечающих
стохастическим колебаниям, точность составляет около 10\%~\cite{15-sin}.

\medskip

\noindent
\textbf{Пример~2}.\  Для осциллятора Дуффинга в автокоррелированной  пуассоновской среде, когда
\begin{equation*}
\ddot X+ \w^2 X -\mu X^3 =-\delta^{\mathrm{OP}} \dot X + U\,;\enskip 
\dot U +\gamma U =V^{\mathrm{OP}} (t)\,, %\label{e7.9-sin}
\end{equation*}
уравнения МСЛ для  $Z\hm= [X\dot X U]^{\mathrm{T}}$ имеют вид~(\ref{e3.5-sin}) и~(\ref{e3.6-sin}) при
    \begin{gather*}
   a_1 = \begin{bmatrix}
        m_{\dot X}\\
        -\w_{ \mathrm{э}}^2 m_X-\delta^{\mathrm{OP}} m_{\dot X}\\
        -m_U\end{bmatrix}\,;\\
    \alpha=  \begin{bmatrix}
            0&1&0\\
            -\w_{1 \mathrm{э}}^2&-\delta^{\mathrm{OP}}&0\\
            0&0&-\gamma\end{bmatrix}\,;\enskip
    \beta= \begin{bmatrix}
        0&0&0\\
        0&0&0\\
        0&0&1\end{bmatrix}\,;
%        \label{e7.10-sin}
\\
a_2 =\alpha K_t+ K_t \alpha^{\mathrm{T}} +\beta \nu^{\mathrm{OP}} \beta^{\mathrm{T}}\,.
        \end{gather*}
Здесь $\nu^{\mathrm{OP}} =\nu^{\mathrm{OP}}(t)$~--- интенсивность белого шума 
$V^{\mathrm{OP}}(t)$. 
Отсюда аналитическим мо\-де\-ли\-ро\-ванием определяются стационарные
режимы, а также режимы их установления. Так же, как в\linebreak случае
автокоррелированных гауссовских белых шумов~\cite{1-sin, 2-sin, 15-sin}, точность МСЛ
за счет <<профильтрованности>> помех значительно повышается и
достигает 2\%--4\%. Результат справедлив и для произвольных
негауссовских белых шумов.

\medskip

\noindent
\textbf{Пример 3}.\  Для релейного осциллятора в гауссовской стохастической среде
\begin{equation}
\ddot X + \w^2 {\mathrm{sgn}} X = -\delta^G \dot X + V^G + U_0\label{e7.11-sin}
\end{equation}
плотность распределения стационарного режима стохастических колебаний при $U_0\hm=0$ 
определяется формулой Гиббса~[1, 2]:
\begin{equation}
f^* (x,\dot x) = c \exp \lf - 
    \fr{H(x,\dot x)}{\vartheta^G}\rf\,,\enskip \vartheta^G = 
    \fr{\nu^G}{ 2\delta^G}\,.\label{e7.12-sin}
    \end{equation}
Здесь через
\begin{equation*}
H(x,\dot x) = \fr{\dot x^2}{2} +\Pi(x)\,,\enskip \Pi (x) =\w^2 |x|\,, %\label{e7.13-sin}
\end{equation*}
обозначена полная энергия осциллятора.

Для~(\ref{e7.11-sin}) при  $U_0\hm\ne 0$, если заменить релейную характеристику 
статистически линеаризованной, согласно~[1, 2]
\begin{equation*}
\mathrm{sgn}\, X = k_0 (m_X, D_X) m_X + k_1 (m_X, D_X) (X^0 - m_X)\,; %\label{e7.14-sin}
\end{equation*}
    $$
    k_0(m_X, D_X) =\fr{2}{ m_X} \Phi \left( \fr{m_X}{\sqrt{D_X}}\right)\,;
    $$
    $$ 
    k_1 (m_X,D_X) = \fr{1}{\sqrt{D_X}} \sqrt{\fr{2}{\pi}}\, \exp \left( -\fr{m_X^2}{2D_X}\right)\,;
    $$
\begin{equation}
\Phi (\tau) = \fr{1}{2\pi} \int\limits_0^\tau e^{-t^2/2} dt\,.\label{e7.15-sin}
\end{equation}
Тогда уравнения МСЛ будут иметь вид:
\begin{equation}
\left.
\begin{array}{rl}
\dot m_X &= m_{\dot X}\,;\\[9pt]
\dot m_X &= U_0 - \w^2 k_0 (m_X, D_X) m_X -\delta m_{\dot X}\,;
\end{array}
\right\}
\label{e7.16-sin}
\end{equation}
    \begin{equation}
\left.
\hspace*{-3.5mm}\begin{array}{c}
    \dot D_X = 2 K_{X\dot X}\,;
\\
    \dot D_{\dot X} = \nu^G - 2\lk \delta D_{\dot X} + \w^2 k_1(m_X,D_X) K_{X\dot X}\rk\,;\\[9pt]
    \dot K_{X\dot X} = D_{\dot X} - \w^2 k_1 (m_X, D_X) D_X - \delta K_{X\dot X}\,,
    \end{array}
    \right\}\!\!
    \label{e7.17-sin}
    \end{equation}
где $\delta \hm= \delta^G$, $\nu\hm=\nu^G$.
Отсюда для стационарных стохастических колебаний имеем связанную систему уравнений:
\begin{equation}
m_{\dot X}^* =0\,;\enskip \w^2 k_0 (m_X^*, D_X^*) = U_0\,;\label{e7.18-sin}
\end{equation}
\begin{equation}
\left.
\begin{array}{c}
K_{X\dot X}^* =0\,;\enskip 
D_X^* =\vartheta=\displaystyle \fr{\nu}{ 2\delta}\,;\\[9pt]
k_1(m_X^*, D_X^*) D_X^* =\rho= \displaystyle \fr{\vartheta}{\w^2} =\fr{\nu}{ 2\delta \w^2}\,.
\end{array}
\right\}
\label{e7.19-sin}
\end{equation}

При $U_0 =0$ из~(\ref{e7.15-sin}), (\ref{e7.18-sin}) и~(\ref{e7.19-sin}) находим:
\begin{equation*}
m_X^* =0\,;\enskip 
m_{\dot X}^* =0\,; \enskip 
D_{\dot X}^* =\vartheta\,;\enskip 
D_X^* =  \fr{\pi}{2}\,\rho^2\,. %\label{e7.20-sin}
\end{equation*}
Отсюда видно, что стационарная дисперсия скорости совпадает с точным
решением~(\ref{e7.12-sin}). Стационарная дисперсия координаты, найденная
согласно МСЛ, отличается от следующего точного решения, полученного
согласно~(\ref{e7.12-sin}). При $\rho\hm \le 1$ относительная ошибка составляет
10\%. Стационарные колебания по~$X$ и $\dot X$ не коррелированы.

Уравнения~(\ref{e7.16-sin}) и~(\ref{e7.17-sin}) показывают, что процесс установления 
режима стохастических колебаний происходит в две стадии: сначала устанавливается 
стационарное распределение по ско\-рости~$\dot X$, а затем по координате~$X$.

\medskip

\noindent
\textbf{Пример 4}.  В~условиях примера~3, но для пуассоновской среды, когда
    \begin{equation*}
    \ddot X +\w^2 {\mathrm{sgn}} X =-\delta^{\mathrm{OP}} \dot X + 
    V^{\mathrm{OP}} + U_0\,,
%    \label{e7.21-sin}
    \end{equation*}
уравнения МСЛ имеют вид~(\ref{e7.16-sin}), (\ref{e7.17-sin}), если принять 
$\delta\hm= \delta^{\mathrm{OP}}$, $ \nu\hm=\nu^{\mathrm{OP}}$, 
$\vartheta\hm=\vartheta^{\mathrm{OP}}\hm=\nu^{\mathrm{OP}}/(2\delta^{\mathrm{OP}})$, 
$\rho \hm=\vartheta^{\mathrm{OP}}/\w^2$. Точного аналитического уравнения 
Фел\-ле\-ра--Кол\-мо\-го\-ро\-ва не обнаружено.

Другие тестовые примеры можно найти в~[10, 12--14].

\section{Заключение}

Дано обобщение точных и приближенных (основанных на параметризации распределений)\linebreak 
методов и алгоритмов теории распределений с инвари\-антной мерой на случай нелинейных 
дифференциальных гауссовых и негауссовых стохастических систем с гладкими и разрывными 
характеристиками.

Особое внимание уделено пуассоновским стохастическим системам с разрывными характеристиками.

На тестовых примерах показана достаточная точность для практических приложений в стохастической 
информатике.

{\small\frenchspacing
{%\baselineskip=10.8pt
\addcontentsline{toc}{section}{Литература}
\begin{thebibliography}{99}
\bibitem{1-sin}
\Au{Пугачёв В.\,С., Синицын И.\,Н.} Стохастические дифференциальные системы. 
Анализ и фильтрация.~--- 2-е изд., доп.~--- М.: Наука, 1990.

\bibitem{2-sin}
\Au{Пугачёв В.\,С., Синицын И.\,Н.} Теория стохастических систем.~--- 2-е изд.~--- М.: Логос,  2004.

\bibitem{3-sin}
\Au{Moshchuk N.\,K., Sinitsyn I.\,N.} On stationary distributions in nonlinear 
stochastic differential systems: Preprint.~--- Coventry, UK: 
University of Warwick, Mathematics Institute, 1989. 15~p.

\bibitem{4-sin}
\Au{Moshchuk N.\,K., Sinitsyn I.\,N.} On stochastic nonholonomic systems: Preprint.~--- 
Coventry, UK: University of Warwick, Mathematics Institute, 1989. 32~p.

\bibitem{5-sin}
\Au{Мощук Н.\,К., Синицын И.\,Н.} О~стохастических неголономных системах~// 
Прикладная механика и математика, 1990. Т.~54. Вып.~2. С.~213--223.

\bibitem{6-sin}
\Au{Moshchuk N.\,K., Sinitsyn I.\,N.} On stationary distributions in 
nonlinear stochastic differential systems~// Quart. J. Mech. Appl. Math., 1991. Vol.~44.  
Pt.~4.  P.~571--579.

\bibitem{7-sin}
\Au{Мощук Н.\,К., Синицын И.\,Н.} О~стационарных и приводимых к стационарным 
режимах в нормальных стохастических системах~// 
Прикладная механика и математика, 1991. Т.~55. Вып.~6. С.~895--903.

\bibitem{8-sin}
\Au{Мощук Н.\,К., Синицын И.\,Н.} Распределения с инвариантной мерой в механических 
стохастических нормальных сис\-те\-мах~// Докл. АН СССР, 1992. Т.~322. №\,4. С.~662--667.

\bibitem{9-sin}
\Au{Синицын И.\,Н.} Конечномерные распределения с инвариантной мерой в стохастических 
механических сис\-те\-мах~// Докл. РАН, 1993. Т.~328. №\,3. С.~308--310.

\bibitem{13-sin} %10
\Au{Soize C.} The Fokker--Plank equation for stochastic dynamical systems 
and its explicit steady state solutions.~--- Singapore: World Scientific,  1994.

\bibitem{10-sin} %11
\Au{Синицын И.\,Н.} Конечномерные распределения с инвариантной мерой в 
стохастических нелинейных дифференциальных системах.~--- М.: Диалог--МГУ, 1997. С.~129--140.

\bibitem{11-sin} %12
\Au{Синицын И.\,Н., Корепанов Э.\,Р., Белоусов~В.\,В.} 
Точные методы расчета стационарных режимов с инвариантной мерой в стохастических 
сис\-те\-мах управ\-ле\-ния~// Кибернетика и технологии XXI~ве\-ка: Тр.\ II Междунар. 
науч.-техн. конф. C\&T'2002.~--- Воронеж: Саквое, 2002. С.~124--131.

\bibitem{12-sin} %13
\Au{Синицын И.\,Н., Корепанов Э.\,Р., Белоусов~В.\,В.} 
Точные аналитические методы в статистической динамике нелинейных 
ин\-фор\-ма\-ци\-он\-но-управ\-ля\-ющих сис\-тем~// Сис\-те\-мы и средства информатики. 
Спец. вып. Математическое и алгоритмическое обеспечение 
ин\-фор\-ма\-ци\-он\-но-те\-ле\-ком\-му\-ни\-ка\-ци\-он\-ных сис\-тем.~--- М.: Наука, 2002. С.~112--121.

\bibitem{14-sin}
\Au{Синицын И.\,Н.} Развитие методов аналитического моделирования распределений с 
инвариантной мерой в стохастических сис\-те\-мах~// Современные проб\-ле\-мы 
прикладной математики, информатики и автоматизации: Тр. Междунар. науч.-техн. семинара.~--- 
Севастополь, 2012. С.~24--35.

\bibitem{15-sin}
\Au{Синицын И.\,Н.} Аналитическое моделирование распределений с инвариантной мерой 
в стохастических сис\-те\-мах с автокоррелированными шумами~// 
Информатика и её применения, 2012. Т.~6. Вып.~4. С.~4--8.

\bibitem{16-sin}
\Au{Немыцкий В.\,В., Степанов В.\,В.} Качественная теория дифференциальных уравнений.~--- 
М.--Л.: Гостехиздат, 1949.


\bibitem{17-sin}
\Au{Козлов В.\,В.} О~существовании интегрального инварианта гладких динамических систем~// 
ПММ, 1987. №\,1. С.~538--545.

\label{end\stat}

\bibitem{18-sin}
\Au{Синицын И.\,Н.} Фильтры Калмана и Пугачёва.~--- 2-е изд.~--- М.: Логос, 2007.
\end{thebibliography}
}
}

\end{multicols} %2
\include{sinitsin-2} %3
%\newcommand {\ff}{{\mathcal F}}
\newcommand {\ebd}{\triangleq}
\newcommand{\me}[2]{\mathbf{E}_{ #1 }\left\{ \mathop{#2} \right\} }



\def\stat{borisov}

\def\tit{ФИЛЬТРАЦИЯ СОСТОЯНИЙ МАРКОВСКИХ СКАЧКООБРАЗНЫХ ПРОЦЕССОВ 
ПО~ДИСКРЕТИЗОВАННЫМ НАБЛЮДЕНИЯМ$^*$}

\def\titkol{Фильтрация состояний марковских скачкообразных процессов 
по~дискретизованным наблюдениям}

\def\aut{А.\,В.~Борисов$^1$}

\def\autkol{А.\,В.~Борисов}

\titel{\tit}{\aut}{\autkol}{\titkol}

\index{Борисов А.\,В.}
\index{Borisov A.\,A.}




{\renewcommand{\thefootnote}{\fnsymbol{footnote}} \footnotetext[1]
{Работа выполнена при частичной поддержке РФФИ (проект 16-07-00677).}}


\renewcommand{\thefootnote}{\arabic{footnote}}
\footnotetext[1]{Институт проблем информатики Федерального исследовательского центра <<Информатика 
и~управление>> Российской академии наук,
\mbox{aborisov@frccsc.ru}}

%\vspace*{8pt}



\Abst{Статья посвящена решению задачи оптимальной 
фильтрации состояний однородного марковского скачкообразного процесса (МСП). 
Наблюдения представляют собой приращения случайных процессов~--- интегральных 
преобразований состояний, зашумленные винеровскими процессами, интенсивность 
которых также зависит от оцениваемого состояния. Оптимальная оценка в~моменты 
получения нового наблюдения вычисляется как функция предыдущей оценки и~новых 
наблюдений, а~между моментами наблюдений~--- простейшим прогнозом в~силу системы 
уравнений Колмогорова. Рекуррентная формула пересчета ресурсозатратна, так как 
содержит  интегралы~--- мас\-штаб\-но-сдви\-го\-вые смеси многомерных гауссиан, 
где в~качестве смешивающих выступают распределения времени пребывания 
состояния в~каждом из возможных значений. Предложены более простые аппроксимации, 
основанные на предположении об ограниченности числа скачков состояния за время между 
наблюдениями. Получены универсальные локальная и~глобальная характеристики точности 
аппроксимаций, зависящие от па\-ра\-мет\-ров оцениваемого процесса, величины 
временн$\acute{\mbox{о}}$го шага  между наблюдениями и~максимального числа учитываемых скачков.}

\KW{марковский скачкообразный процесс; оптимальная фильтрация; мультипликативные 
шумы в~наблюдениях; стохастическое дифференциальное уравнение; численная аппроксимация}

\DOI{10.14357/19922264180316}
  
%\vspace*{4pt}


\vskip 10pt plus 9pt minus 6pt

\thispagestyle{headings}

\begin{multicols}{2}

\label{st\stat}



 \section{Введение}
 
 Фильтр Вонэма~\cite{Won_65}~--- один из редких удачных случаев, когда 
 оценка оптимальной фильтрации состо\-яния стохастической системы наблюдения 
 выражается в~виде решения некоторой замк\-ну\-той\linebreak конечномерной сис\-те\-мы 
 стохастических дифференциальных уравнений. 
 
 Алгоритм данного фильт\-ра 
 позволяет вычислить оценку фильт\-ра\-ции со\-сто\-яния \textit{марковского скачкообразного 
 процесса} с~\mbox{конечным} множеством состояний по наблюдениям в~присутствии 
 аддитивных винеровских шумов. Теоретически оптимальная оценка со\-сто\-яния~--- 
 его условное распределение в~текущий момент времени~--- 
 обладает очевидными свойствами неотрицательности и~нормировки. 
 При чис\-лен\-ной реализации данного фильтра классическим методом 
 Эй\-ле\-ра--Ма\-ру\-ямы~\cite{KP_92} данные свойства могут не сохраняться и~процедура 
 вы\-чис\-ле\-ния становится неустойчивой.  В~связи с~этим обстоятельством разрабатывались 
 другие алгоритмы чис\-лен\-но\-го решения уравнения фильтра Вонэма, обладающие 
 требуемыми свойствами устойчивости (см.~\cite{YZL_04, PR_10} и~библиографию в~них). 
 В~час\-ти этих работ доказана лишь слабая сходимость пред\-ла\-га\-емых аппроксимационных 
 схем к~оценке фильт\-ра Вонэма, в~то время как ка\-кая-ли\-бо 
 характеризация точ\-ности этих приближений отсутствует.
 
 В~\cite{B_18} было представлено распространение фильт\-ра Вонэма на случай 
 наблюдений с~мультипликативными шумами. При этом уравнение обобщенного 
 фильт\-ра содержит в~правой части квадратическую характеристику шумов в~наблюдениях. 
 Данный процесс на практике никогда не наблюдается непосредственно, а~является лишь 
 некоторым нелинейным интегральным преобразованием наблюдений. Очевидно, что 
 имеющиеся в~настоящий момент времени алгоритмы приближенного вычисления оценки 
 фильтрации Вонэма для данной системы не подходят. 
 
 Целью предлагаемой работы является ис\-поль\-зование результатов оптимальной 
 фильтрации со\-стояний сис\-тем с~дискретным временем для аппроксимации решения 
 аналогичной задачи для\linebreak стохастических дифференциальных сис\-тем. 
 
 Статья организована следующим образом. Раздел~2 содержит формальную постановку 
 задачи фильт\-ра\-ции со\-сто\-яний однородного МСП с~конечным множеством со\-сто\-яний 
 по наблюдениям, полученным путем временн$\acute{\mbox{о}}$й дискретизации процессов с~непрерывным 
 временем~--- интегральных преобразований со\-сто\-яния сис\-те\-мы в~присутствии 
 мультипликативных винеровских шумов.\linebreak
  В~разд.~3 пред\-став\-ле\-но решение поставленной 
 задачи фильт\-ра\-ции: пересчет оценок со\-сто\-яний в~момент получения новых 
 дискретизованных наблюдений выполняется в~соответствии с~некоторыми\linebreak 
 рекуррентными интегральными соотношениями, в~то время как между 
 моментами наблюдений оценка корректируется в~соответствии с~прогнозом в~силу 
 сис\-те\-мы уравнений Колмогорова. Вы\-чис\-ли\-тель\-ная слож\-ность 
 упомянутых выше интегральных\linebreak 
 соотношений связана с~тем, что в~расчет принимается воз\-мож\-ность того, что между 
 моментами наблюдений оцениваемое со\-сто\-яние может совершить произвольное чис\-ло 
 скачков. В~разд.~4 пред\-став\-лен более простой алгоритм приближенного вы\-чис\-ле\-ния 
 оценки фильт\-ра\-ции, основанный на ограничении возможного числа учитываемых скачков 
 МСП. Доказана тео\-ре\-ма, опре\-де\-ля\-ющая как\linebreak
  локальную (одношаговую), так и~глобальную 
 (многошаговую) характеристики точ\-ности предложенного при\-бли\-же\-ния~--- 
 $\ell_1$-нор\-мы ошибки аппроксимации. Полученные характеристики являются\linebreak 
 универсальными, т.\,е.\ не асимптотическими по шагу дискретизации, и~зависят от характеристик 
 самого МСП, %\linebreak
  шага временн$\acute{\mbox{о}}$й дискретизации и~чис\-ла
  скачков со\-сто\-яния, учи\-ты\-ва\-емых 
 на шаге. Об\-суж\-де\-ние результатов и~заключительные комментарии пред\-став\-ле\-ны 
 в~разд.~5.
 
 \section{Постановка задачи фильтрации}
 
 На полном вероятностном пространстве с~фильт\-ра\-цией 
 $(\Omega,\mathcal{F},\mathcal{P},\{\mathcal{F}_{t}\}_{t \geqslant 0})$ рассматривается система наблюдений
\begin{equation}
 \left.
 \begin{array}{rl}
 \displaystyle X_t &=X_0 +  \displaystyle
 \int\limits_0^t \Lambda^{\top}X_{s}\,ds + \mu_s\,;  \\[6pt]
 \displaystyle Y_k &=  \displaystyle\int\limits_{t_{k-1}}^{t_k}fX_s\,ds+
 \int\limits_{t_{k-1}}^{t_k} 
 \sum\limits_{n=1}^NX_s^ng_n \,dW_s, \\[6pt]
 &\hspace*{10mm}\{t_k\}_{k \geqslant 0}: \; 0 = t_0 < t_1 < t_2\cdots,
 \end{array}
 \right\}
 \label{eq:obsys_1}
 \end{equation}
 где
  \begin{itemize}
  \item
  $X_t \ebd \mathrm{col}\left(X_t^1,\ldots,X_t^N\right) \hm\in \mathbb{S}^N$~--- 
  ненаблюда\-емое состояние системы, являющееся однородным МСП с~конечным 
  множеством состояний $ \mathbb{S}^N \ebd$\linebreak $\ebd \{e_1,\ldots,e_N\}$ ($\mathbb{S}^N$~--- 
  множество единичных векторов евклидова пространства~$\mathbb{R}^N$), 
  матрицей интенсивностей переходов~$\Lambda$ и~начальным распределением~$\pi$;
  \item
  $\mu_t \ebd \mathrm{col}\left(
  \mu_t^1,\ldots,\mu_t^N\right)\hm\in \mathbb{R}^N$~--- 
  ${\mathcal{F}}_t$-со\-гла\-со\-ван\-ный мартингал;
  \item
  $\{Y_k\}_{k \in \mathbb{N}}:\;  Y_k \ebd \mathrm{col}\left(Y_k^1,\ldots,Y_k^M\right) 
  \hm\in \mathbb{R}^M$~--- последовательность дискретизованных наблюдений, 
  доступных в~известные неслучайные  моменты времени~$\{t_k\}_{k \in \mathbb{N}}$,
в~которых $W_t \ebd$\linebreak $\ebd \mathrm{col}\left(W_t^1,\ldots,W_t^M\right) \hm\in \mathbb{R}^M$
 является ${\mathcal{F}}_t$-со\-гла\-со\-ван\-ным стандартным винеровским процессом, 
 определяющим шумы в~наблюдениях,\linebreak  $f$~--- $(M \times N)$-мер\-ная 
 мат\-ри\-ца плана наблюдений, а~набор мат\-риц~$\{g_n\}_{n=\overline{1,N}}$ 
 характеризует интенсивности шумов в~зависимости от текущего состояния~$X_t$.
  \end{itemize}
  
  Введем также в~рассмотрение неубывающие семейства $\sigma$-ал\-гебр 
  $\mathcal{O}_k \ebd \sigma\{ Y_{\ell}: \; 1 \hm\leqslant \ell \hm\leqslant k\}$ 
  и~$\mathcal{O}_t \ebd  \mathcal{O}_{k(t)}$, где 
  $k(t) \ebd \sum\nolimits_{j \in \mathbb{N}}\mathbf{I}(t-t_{j})$; 
  $\mathcal{O}_0 \ebd \{\varnothing,\; \Omega\}$.
  
   \textit{Задача оптимальной фильтрации состояния~$X$ по наблюдениям~$Y$} 
   заключается в~нахождении \textit{условного математического ожидания} (УМО)
  \begin{equation*}
  \widehat{X}_t \ebd {\sf E}\left\{X_t|\mathcal{O}_{t} \right\}\,.
 % \label{eq:fest_1}
  \end{equation*}
  
  Относительно системы~(\ref{eq:obsys_1})  сделаны следующие предположения:
   \begin{itemize}
 \item[(а)]
 ${\mathcal{F}}_t \equiv {\mathcal{F}}_{t}^X \bigvee 
 {\mathcal{F}}_{t}^W $ для любого $t \hm\geqslant 0$;
 \item[(б)]
 шумы в~наблюдениях равномерно невырожденные, т.\,е.\
  $g_ng_n^{\top} \hm\geqslant \alpha I \hm> 0$ для всех $n\hm=\overline{1,N}$ 
  и~некоторого $\alpha\hm>0$.
% \item
 % Верно неравенство
  %\begin{equation}
  %\min_{1\leqslant k \leqslant N}|\lambda_{kk}| > 0.
  %\label{eq:ineq_0}
  % \end{equation}
 %\item
 %Для любого $t \geqslant 0$ все компоненты вектора $p_t \ebd \me{}{X_t}$ строго %положительны. 
 \end{itemize} 

 \section{Уравнения оптимального фильтра} 
 
 Для получения уравнений оптимального фильт\-ра воспользуемся подходом, 
 применяемым для решения аналогичной задачи в~стохастических сис\-те\-мах 
 наблюдения с~дискретным временем~\cite{BSh_85}. 
 Воспользу\-ем\-ся методом математической индукции. 
 
 При $r=0$ 
 \begin{equation}
 \widehat{X}_{t_0}={\sf E}\{X_0|\mathcal{O}_0\}={\sf E}\{X_0\}=\pi\,.
 \label{eq:in_cond}
 \end{equation} 
 
 Пусть для некоторого $ r \hm\geqslant 0$ известна оценка оптимальной 
 фильтрации~$\widehat{X}_{t_r} \hm= {\sf E}{X_{t_r} |\mathcal{O}_r}$. 
 Определим оценку оптимальной фильтрации~$\widehat{X}_{t} $ для $t\hm \in (t_r,t_{r+1}]$. 
 
 Для произвольного момента $t \hm\in (t_r,t_{r+1})$ в~силу мартингального 
 разложения МСП~$X_t$ и~свойств УМО верна следующая цепочка равенств:
 \begin{multline*}
 \widehat{X}_{t} = {\sf E}\left\{X_t | \mathcal{O}_r\right\}={}\\
 {}=
 {\sf E}\left\{{\cal P}^{\top}(t_r,t)X_{t_r}+
 \int\limits_{t_r}^t{\cal P}^{\top}(t_r,s)\,dM_s\big\vert \mathcal{O}_r\right\} = {}
\end{multline*}

\noindent
   \begin{multline}
 \hspace*{-11.66pt}{}=\mathcal{P}^{\top}(t_r,t)\widehat{X}_{t_r} + {\sf E}\hspace*{-2pt}
 \left\{{\sf E}\hspace*{-2pt}\left\{\int\limits_{t_r}^t\hspace*{-2pt}\mathcal{P}^{\top}(t_r,s)\,dM_s |
 {\mathcal{F}}_{t_r}\right\}\!\big\vert 
 \mathcal{O}_r\!\right\} ={}\hspace*{-4.24124pt}\\
 {}=
  \mathcal{P}^{\top}(t_r,t)\widehat{X}_{t_r}\,,
 \label{eq:bw_obs}
 \end{multline}
 где $\mathcal{P}(s,t)$ $(s \hm\leqslant t)$~--- матрица переходной ве\-ро\-ят\-ности МСП 
 на промежутке $[s,t]$, являющаяся решением сис\-те\-мы дифференциальных 
 уравнений Колмогорова
 \begin{equation*}
 \mathcal{P}'_t(s,t) = \mathcal{P}(s,t) \Lambda, \enskip t > s, \enskip \mathcal{P}(s,s) = I.
 \end{equation*}
 В случае однородного МСП $\mathcal{P}(s,t) \hm= e^{(t-s)\Lambda}$.
 
 Далее необходимо определить совместное распределение $(X_{t_{r+1}},Y_{r+1})$ 
 относительно~$ \mathcal{O}_r$. Из модели наблюдений следует, что 
 распределение~$Y_{r+1}$ относительно 
 $\sigma$-ал\-геб\-ры~$\mathcal{F}^X_{t_{r+1}} \vee \mathcal{O}_r$~---
 гауссовское с~параметрами 
 \begin{align*}
{\sf E}\left\{Y_{r+1}|{\mathcal{F}}^X_{t_{r+1}}\right\}& = f \tau_{r+1}\,; \\[6pt]
 \mathrm{cov} \left(Y_{r+1},Y_{r+1}|{\mathcal{F}}^X_{t_{r+1}}\right) &= 
 \displaystyle\sum\limits_{n=1}^N \tau_{r+1}^n g_ng_n^{\top}\,,
% \label{eq:occup_1}
 \end{align*}
 где $\tau_{r+1} \hm= \tau_{r+1}(X(\omega))=
 \mathrm{col}\left(\tau_{r+1}^1,\ldots,\tau_{r+1}^N\right) \ebd$\linebreak
 $\ebd 
 \int\nolimits_{t_r}^{t_{r+1}}X_s\,ds$~--- случайный вектор, $n$-я 
 компонента которого равна времени пребывания процесса~$X$ в~со\-сто\-янии~$e_n$ 
 на  интервале времени $[t_r, t_{r+1}]$. 
 Обозначим через $\mathcal{D}_{r+1} \ebd \{u=\mathrm{col}\,(u^1,\ldots,u^N):\; 
 u_m \hm\geqslant 0,\; \sum\nolimits_{m=1}^Mu_m\hm= t_{r+1}-t_r\}$ $(M-1)$-мер\-ный 
 симплекс в~пространстве~$\mathbb{R}^M$, являющийся носителем распределения 
 вектора~$\tau_{r+1}$. Пусть $\rho^{k,\ell}_{r+1}(du)$~--- 
 распределение вектора $\tau_{r+1} X_{t_{r+1}}^{\ell}$ при условии $X_{t_r}\hm=e_k$, 
 т.\,е.\ 
 для любого $\mathcal{A} \hm\in \mathcal{B}(\mathbb{R}^M)$ верно тождество:
\begin{multline*}
 \mathbf{P}\left\{\omega: \; X_{t_{r+1}}(\omega)=e_{\ell},\right.\\
 \left. 
 \tau_{r+1}(X(\omega)) \in \mathcal{A}\;|\;X_{t_r}=e_k\right\} \equiv
   \rho^{k,\ell}_{r+1}(\mathcal{A})\,.
\end{multline*}
 
Обозначим через
\begin{multline*}
 \mathcal{N}(y,m,K) \ebd (2\pi)^{-M/2} \mathrm{ det}^{-1/2} K\times{}\\
 {}\times\exp
 \left\{ -\fr{1}{2}\left(y-m\right)^{\top}K^{-1}(y-m)\right\}
\end{multline*}
 $M$-мер\-ную плот\-ность гауссовского распределения с~математическим 
 ожиданием~$m$ и~ковариационной матрицей~$K$.
 
 Из марковского свойства  $\{X_{t_{r}},Y_{r})\}_{r \geqslant 0}$ 
 относительно~${\mathcal{F}}_{t_{r}}$~\cite{ZhSh_95} и~теоремы Фубини следует, что 
 для любого  множества $\mathcal{A} \hm\in \mathcal{B}(\mathbb{R}^M)$ 
 верна следующая цепочка равенств:
 \begin{multline*}
 {\sf E}\left\{X_{t_{r+1}}\mathbf{I}_{\mathcal{A}}
 \left(Y_{r+1}\right)\big|\mathcal{O}_r\right\}={}\\
 {}=
{\sf E}\left\{{\sf E}\left\{X_{t_{r+1}}\mathbf{I}_{\mathcal{A}}
\left(Y_{r+1}\right)\big|
\mathcal{F}^X_{t_{r+1}} \vee \mathcal{O}_r\right\}
 \big|\mathcal{O}_r\right\} = {}
\end{multline*}

\noindent
\begin{multline*}
 %{}=
% {\sf E}\left\{{\sf E}\left\{X_{t_{r+1}}\mathbf{I}_{\mathcal{A}}
% \left(Y_{r+1}\right)\vert X_{t_r}\right\}
% \vert\mathcal{O}_r\right\} = {}\\
% {}=
%{\sf E}\left\{\sum\limits_{k=1}^N {\sf E}\left\{X_{t_{r+1}}\mathbf{I}_{\mathcal{A}}
%\left(Y_{r+1}\right)  \big| X_{t_r}=e_k\right\}X_{t_r}^k
% \big|\mathcal{O}_r\right\} = {}\\ 
% {}=
% \sum\limits_{k=1}^N{\sf E}
% \left\{X_{t_{r+1}}\mathbf{I}_{\mathcal{A}}\left(Y_{r+1}\right)\bigl| X_{t_r}=e_k\right\} 
% \widehat{X}_{t_r}^k ={}\\
% {}=\!
% \sum\limits_{k=1}^N{\sf E}
% \left\{{\sf E}\left\{X_{t_{r+1}}\mathbf{I}_{\mathcal{A}}
% \left(Y_{r+1}\right)\!\bigl| {\mathcal{F}}_{t_{r+1}}\right\}\!\bigl| 
% X_{t_r}\!=e_k\right\} \widehat{X}_{t_r}^k ={}\\
% {}=
% \sum\limits_{k=1}^N {\sf E}\left\{
% \vphantom{\int\limits_A\left(\sum\limits_{p=1}^N\right)}
% X_{t_{r+1}} \times{}\right.\\
% {}\times\int\limits_{\mathcal{A}}  
% \mathcal{N}\left(y,f \tau_{r+1}(X),\sum\limits_{p=1}^N \tau_{r+1}^p(X) g_pg_p^{\top}\right)dy
% \Biggl| X_{t_r}={}\\
%\left. {}=e_k
% \vphantom{\int\limits_A\left(\sum\limits_{p=1}^N\right)}
%\right\} \widehat{X}_{t_r}^k = 
% \sum\limits_{k=1}^N \int\limits_{\mathcal{A}}{\sf E}\left\{ 
% \vphantom{\sum\limits_{p=1}^N}
% X_{t_{r+1}} \times{}\right.\\
% {}\times\mathcal{N}\left(y,f \tau_{r+1}(X),\sum\limits_{p=1}^N \tau_{r+1}^p(X) 
% g_p g_p^{\top}\right)
% \Biggl| X_{t_r}={}\\
%\left. {}=e_k
%\vphantom{\sum\limits^N_{p=1}}
%\right\} \widehat{X}_{t_r}^k\, dy
 %={}\\
 {}=
 \sum\limits_{\ell=1}^N e_{\ell} \int\limits_{\mathcal{A}} 
 \left[ \sum\limits_{k=1}^N 
 \int\limits_{\mathcal{D}_{r+1}} 
 \mathcal{N}\left(y,f u,\sum_{p=1}^N u^p g_pg_p^{\top}\right)\times{}\right.\\
\left. {}\times
 \rho^{k,\ell}_{r+1}(du)\widehat{X}_{t_r}^k
 \vphantom{\int\limits_A\sum\limits_{p=1}^N}
 \right] 
 dy,
 \end{multline*}
 из чего следует, что интегранд в~квадратных скобках в~последнем выражении 
 определяет искомое совместное распределение $(X_{t_{r+1}},Y_{r+1})$ 
 относительно~$ \mathcal{O}_r$. Оценка~$\widehat{X}_{t_{r+1}}$ покомпонентно 
 определяется~\cite{BSh_85} с~помощью обобщенного варианта формулы Байеса:
 \begin{multline}
 \widehat{X}_{t_{r+1}}^j = {}\\
 \hspace*{-1mm}{}=
 \fr{\int\nolimits_{\mathcal{D}_{r+1}}\hspace*{-6mm} 
 \mathcal{N}\left(Y_{r+1},f u,\sum\nolimits_{p=1}^N \hspace*{-2mm}
 u^p g_pg_p^{\top}\!\right)\hspace*{-1mm}
 \sum\nolimits_{k=1}^N \hspace*{-2mm}
 \widehat{X}_{t_r}^k
 \rho^{k,j}_{r+1}(du)
 }
 { \int\nolimits_{\mathcal{D}_{r+1}} \hspace*{-6mm}
 \mathcal{N}\left(Y_{r+1},f v,\sum\nolimits_{q=1}^N \hspace*{-2mm}
 v^q g_qg_q^{\top}\!\right)\hspace*{-1mm}
 \sum\nolimits_{i,\ell=1}^N \hspace*{-2mm}
 \widehat{X}_{t_r}^i
 \rho^{i,\ell}_{r+1}(dv)
  },  \\ 
  j = \overline{1,N}\,.
 \label{eq:filt_1}
 \end{multline}
 Таким образом, доказана следующая
 
 %\smallskip
 
 \noindent
 \textbf{Лемма~1.}
\textit{Если для системы наблюдения}~(\ref{eq:obsys_1}) 
\textit{верны условия~(а) и~(б), то оценка~$\widehat{X}_t$ оптимальной фильтрации 
определяется формулой}~(\ref{eq:in_cond}) 
\textit{при $t\hm=0$, рекуррентным соотношением}~(\ref{eq:filt_1})~---
\textit{в~моменты~$t_{r+1}$ получения наблюдений~$Y_{r+1}$ 
и~формулой}~(\ref{eq:bw_obs})~--- 
\textit{в~промежутках времени между моментами получения наблюдений}.


\smallskip
 

 
 Несмотря на компактную запись~(\ref{eq:filt_1}), их прямая численная реализация 
 ресурсозатратна. Во-пер\-вых, в~(\ref{eq:filt_1}) требуется вычислять 
 распределения мас\-штаб\-но-сдви\-го\-вых смесей многомерных нормальных 
 распределений, что является трудоемкой\linebreak процедурой. Во-вто\-рых, 
 распределения~$\rho^{k,j}_{r+1}$ вре-\linebreak мени пребывания представляют собой 
 сумму\linebreak бесконечного ряда, слагаемые которого вычис\-ляются с~помощью 
 некоторой рекуррентной про\-це\-дуры~\cite{S_00}. В-третьих, 
 распределения~$\rho^{k,j}_{r+1}$ не являются абсолютно непрерывными 
 относительно меры Ле\-бега.
 { %\looseness=1
 
 }
 
 Следующий раздел посвящен численной аппроксимации~(\ref{eq:filt_1}) и~исследованию 
 ее точностных характеристик.
 
 \section{Приближенное вычисление оценки фильтрации}
 
 Без ограничения общности будем считать, что сетка~$\{t_r\}_{r \geqslant 0}$ 
 является равномерной с~шагом~$\Delta$, т.\,е.\ $t_r \hm= r \Delta$ 
 и~$\mathcal{D}_r \hm\equiv \mathcal{D}$.
 Обозначим через~$N_{r+1}$ об-\linebreak\vspace*{-12pt}
 
 \pagebreak
 
 \noindent
 щее число скачков процесса~$X_t$, имевших место 
 на промежутке $(t_r,t_{r+1}]$. Тогда из формулы полной вероятности следует, 
 что~(\ref{eq:filt_1}) представима в~виде:
 \begin{multline}
 \widehat{X}_{t_{r+1}}^j =  \left(
 \int\limits_{\mathcal{D}} 
 \mathcal{N}\left(Y_{r+1},f u,\sum\limits_{p=1}^N u^p g_pg_p^{\top}\right)\times{}\right.\\
\left. {}\times
 \sum\limits_{h=0}^{\infty}\sum\limits_{k=1}^N \widehat{X}_{t_r}^k
 \rho^{k,j,h}_{r+1}(du)
 \right)\Bigg/ \\
 \left(
 \vphantom{\sum\limits_{m=0}^{\infty}
 \sum\limits_{i,\ell=1}^N \widehat{X}_{t_r}^i
 \rho^{i,\ell,m}_{r+1}(dv)}
 \int\limits_{\mathcal{D}} 
 \mathcal{N}\left(Y_{r+1},f v,\sum\limits_{q=1}^N v^q g_qg_q^{\top}\right)\times{}\right.\\
\left.{}\times \sum\limits_{m=0}^{\infty}
 \sum\limits_{i,\ell=1}^N \widehat{X}_{t_r}^i
 \rho^{i,\ell,m}_{r+1}(dv)
 \right)
  \,, \enskip j = \overline{1,N}\,,
  \label{eq:filt_1_1}
 \end{multline}
 где 
 $ \rho^{k,j,h}_{r+1}(du)$~--- распределение вектора 
 $\tau_{r+1}X_{t_{r+1}}^{j}\mathbf{I}_{\{h\}}(N_{r+1})$ при 
 условии $X_{t_r}\hm=e_k$, т.\,е.\ 
 для любого $\mathcal{A} \hm\in \mathcal{B}(\mathbb{R}^M)$ верно тождество
\begin{multline*}
 \mathbf{P}\left\{\omega: \; X_{t_{r+1}}(\omega)=e_{j}, \; N_{r+1} = h,\right.\\ 
\left. \tau_{r+1}(X(\omega)) \in \mathcal{A}\;|\;X_{t_r}=e_k\right\} \equiv
  \rho^{k,j,h}_{r+1}(\mathcal{A}).
\end{multline*}
В качестве аппроксимации оценок можно использовать  
 $\overline{X}_{t_{r+1}}^n \ebd 
 \mathrm{col}\,(\overline{X}_{t_{r+1}}^{n,1},\ldots,\overline{X}_{t_{r+1}}^{n,N})$, 
 полученные из~(\ref{eq:filt_1_1}) путем урезания сумм ряда в~числителе и~знаменателе:
 
 \noindent
 \begin{multline}
 \overline{X}_{t_{r+1}}^{n,j} = 
 \left(
 \int\limits_{\mathcal{D}} 
 \mathcal{N}\left(Y_{r+1},f u,\sum\limits_{p=1}^N u^p g_pg_p^{\top}\right)\times{}\right.\\[-1pt]
\left.{}\times \sum\limits_{h=0}^{n}\sum\limits_{k=1}^N \overline{X}_{t_r}^k
 \rho^{k,j,h}_{r+1}(du)
 \right)\Bigg/ \\[-1pt]
 \left(
 \int\limits_{\mathcal{D}} 
 \mathcal{N}\left(Y_{r+1},f v,\sum\limits_{q=1}^N v^q g_qg_q^{\top}\right)\times{}\right.\\[-1pt]
\left. {}\times
 \sum\limits_{m=0}^{n}
 \sum\limits_{i,\ell=1}^N \overline{X}_{t_r}^i
 \rho^{i,\ell,m}_{r+1}(dv)
  \right)\,, \enskip
   j = \overline{1,N}.
  \label{eq:filt_2}
 \end{multline}
 Ниже по формуле полной вероятности получены интегралы из~(\ref{eq:filt_2}) для 
 $h\hm=0,1,2$:
 
\vspace*{-3pt}

 \noindent
  \begin{multline*}
 \int\limits_{\mathcal{D}}  \mathcal{N}
 \left(Y_{r+1},f u,\sum\limits_{p=1}^N u^p g_pg_p^{\top}\right) 
 \rho^{k,j,0}_{r+1}(du) = {}\\[-1pt]
 {}=
 \delta_{kj}\mathcal{N}\left(Y_{r+1},\Delta f^j,\Delta g_jg_j^{\top}\right)
 e^{\lambda_{jj}\Delta};
 %\label{eq:h0}
\\[-1pt]
 \int\limits_{\mathcal{D}}  \mathcal{N}\left(
 Y_{r+1},f u,\sum\limits_{p=1}^N u^p g_pg_p^{\top}\right) 
 \rho^{k,j,1}_{r+1}(du) ={} 
 \end{multline*}
 
 \noindent
 \begin{multline}
 \hspace*{-6.7pt}{}=\left(1-\delta_{kj}\right)\lambda_{kj}e^{\lambda_{jj}\Delta}
\! \int\limits_0^{\Delta}\!
 e^{(\lambda_{kk}-\lambda_{jj})u^k}
 \mathcal{N}\left(Y_{r+1},u^kf^k +{}\right.\hspace*{-0.28818pt}\\[-1pt]
\hspace*{-3mm}\left. {}+ \left(\Delta - u^k\right)f^j, u^k g_kg_k^{\top}+
 \left(\Delta-u^k\right)g_jg_j^{\top}\right)\,du^k;
 \label{eq:h1}
 \end{multline}
 
 \vspace*{-12pt}
 
 \noindent
 \begin{multline}
 \int\limits_D \mathcal{N}\left( 
Y_{r+1},f u,\sum\limits_{p=1}^N u^p g_pg_p^{\top}\right)du ={}\\[-1pt]
{}=
\sum\limits_{\substack{{\ell:\ell \neq k,}\\ {\ell \neq j}}}
 \lambda_{k\ell}\lambda_{\ell j} e^{\lambda_{jj}\Delta}\times {}\\[-1pt] 
 {}\times
 \int\limits_0^{\Delta} \int\limits_0^{\Delta-u^k} \!
e^{(\lambda_{kk}-\lambda_{\ell\ell})u^k+(\lambda_{\ell\ell}-
 \lambda_{jj})u^{\ell}}\times{} \\[-1pt] 
{}  \times
 \mathcal{N}\left(Y_{r+1},u^k f^k+u^{\ell}f^{\ell}+\left(
 \Delta-u^k-u^{\ell} \right)f^j,\right.\\[-1pt]
 \hspace*{-1mm}\left.
 u^k g_kg_k^{\top}+u^{\ell}g_{\ell}g_{\ell}^{\top}+\left(
 \Delta-u^k-u^{\ell} \right)
 g_jg_j^{\top}
 \right) du^{\ell}du^{k}, \!\!
  \label{eq:h2}
 \end{multline} 
 
\vspace*{-2pt}
 
 \noindent
  где  $\delta_{ij}$~--- символ Кронекера. Интегралы для $h\hm>2$ также могут 
  быть получены в~явном виде, однако их сложность резко возрастает.
 

   Так как система~(\ref{eq:obsys_1}) является автономной, то в~качестве локальной 
   характеристики бли\-зости~$\{\overline{X}_{t_r}\}$ 
   к~$\{\widehat{X}_{t_r}\}$ может быть выбрана величина
   
\noindent
 \begin{multline*}
 \overline{\sigma}(\pi) \ebd {\sf E}\left\{
 \|\widehat{X}_{t_{1}}(\pi, Y_{1}) - \overline{X}_{t_{1}}
 \left(\pi,Y_{1}\right)\|_{1}\right\} = {}\\
 {}=
 \sum\limits_{j=1}^N{\sf E}
 \left\{\left\vert \widehat{X}^j_{t_{1}}\left(\pi, Y_{1}\right) - \overline{X}^{n,j}_{t_{1}}
 \left(\pi,Y_{1}\right)\right\vert\right\}.
 %\label{eq:prec_1}
 \end{multline*}
 При этом начальное распределение $\pi \hm\in \mathcal{D}_1 \ebd $\linebreak $\ebd
 \{\mathrm{col}\,(\pi^1,\ldots,\pi^N):\;\pi^j > 0$, 
 $\sum\nolimits_{j=1}^N\pi^j\hm=1\}$ является начальным условием применения 
 одного шага рекурсии~(\ref{eq:filt_1}) или~(\ref{eq:filt_2}) для вычисления 
 оценки~$\widehat{X}_{t_{1}}$
   или~$\overline{X}_{t_{1}}$ соответственно. Фактически, 
 характеристика~$\overline{\sigma}(\pi)$ определяет, насколько сильно 
 рекурсивные схемы~(\ref{eq:filt_1}) и~(\ref{eq:filt_2}) разойдутся за 
 один шаг, стартуя из общей точки~$\pi$.
 
 Рекуррентные схемы~(\ref{eq:filt_1}) и~(\ref{eq:filt_2}), примененные~$r$~раз, 
 позволяют вычислить оценки~$\widehat{X}_{t_r}$ и~$\overline{X}_{t_r}$ 
 в~точке~$t_r$. В~качестве характеристики точности глобальной аппроксимации в~этом 
 случае естественно рассмотреть величину
 
 \vspace*{-2pt}
 
 \noindent
 \begin{equation*}
 \overline{\Sigma}_{t_r}(\pi) \ebd {\sf E}
 \left\{\|\widehat{X}_{t_{r}} - \overline{X}_{t_{r}}\|_{1}\right\} = 
 \!\sum\limits_{j=1}^N\!{\sf E}
 \left\{\left\vert \widehat{X}^j_{t_{r}} - 
 \overline{X}^{n,j}_{t_{r}}\right\vert \right\}.
% \label{eq:prec_2}
 \end{equation*}
 
 Следующее утверждение определяет оценки локальной и~глобальной 
 точности схемы аппроксимации~(\ref{eq:filt_2}).
 
 %\smallskip
 
 \noindent
 \textbf{Теорема~1.}\
\textit{Выполняются неравенства} 

%\vspace*{-2pt}

\noindent
 \begin{equation}
 \sup_{\pi \in \mathcal{D}_1} \overline{\sigma}(\pi) 
 \leqslant 2 \fr{(\overline{\lambda}\Delta)^{n+1}}{(n+1)!}\,;
 \label{eq:prec_loc}
\end{equation}

\noindent
\begin{align}
  \sup\limits_{\pi \in \mathcal{D}_1} \overline{\Sigma}_{t_r}(\pi)
   &\leqslant 2r \fr{(\overline{\lambda}\Delta)^{n+1}}{(n+1)!} +{}\notag\\[-0.5pt]
   &\hspace*{-20mm}{}+
  r(r-1)\left(
  \fr{(\overline{\lambda}\Delta)^{n+1}}{(n+1)!}
  \right)^2
  \left(
  1-\fr{(\overline{\lambda}\Delta)^{n+1}}{(n+1)!}
  \right)^{r-2},
 \label{eq:prec_glob}
 \end{align}
 
 \vspace*{-2pt}
 
 \noindent
 \textit{где} $\overline{\lambda} \ebd \max_{1 \leqslant j \leqslant N}|\lambda_{jj}|$.


%\smallskip

 Доказательство теоремы~1 приведено в~приложении.
 
 Данное утверждение представляет полезные оценки точности. Во-пер\-вых, 
 они являются равномерными по начальному распределению $\pi \hm\in \mathcal{D}_1$. 
 Во-вто\-рых, оценки носят универсальный, а~не асимптотический характер. Это 
 существенно в~практических задачах оценивания по дискретизованным 
 наблюдениям с~физическими или алгоритмическими ограничениями на шаг 
 по времени. Например, в~случае наблюдаемого процесса восстановления в~силу 
 центральной предельной теоремы для процессов восстановления~\cite{B_80} его
  приращения можно рассматривать как гауссовские случайные величины. 
  Однако данная аппроксимация обладает удовлетворительной точностью 
  только в~случае, когда шаг дискретизации по времени достаточно большой. 
 %
 В-третьих, неравенство~(\ref{eq:prec_glob}) позволяет получить порядок 
 аппроксимации при $\Delta \hm\to 0$. Зафиксируем момент времени $t\hm=T$ и~рассмотрим 
 характеристику $\sup\nolimits_{\pi \in \mathcal{D}_1} 
 \overline{\Sigma}_{T}(\pi)$ при $r\hm={T}/{\Delta}$ и~$\Delta \hm\to 0$. 
 Как только~$\Delta$ становится настолько мало, что 
 $\max\left({(\overline{\lambda}\Delta)^{n+1}}/{(n+1)!}, 
 \Delta ({T\lambda^{n+1}}/{(n+1)!})\right)\hm< 1$, из~(\ref{eq:prec_glob}) 
 следует неравенство
  %\begin{equation}
  $\sup\nolimits_{\pi \in \mathcal{D}_1} \overline{\Sigma}_{T}(\pi) 
  \hm\leqslant  ({3\overline{\lambda}^{n+1}}/{(n+1)!}) T\Delta^n.$
 %\label{eq:prec_asympt}
 %\end{equation}
 Это значит, что с~ростом времени~$T$ 
 ошибка аппроксимации копится пропорционально~$T$ и~при этом порядок точности 
 по~$\Delta$ равен~$n$.
 
 %\vspace*{-7pt}
 
  \section{Заключение}
  
  \vspace*{-4pt}
 
  В работе решена задача оценивания состояния однородного МСП по 
  дискретизованным наблюдениям. Получено аналитическое решение и~его 
  чис\-лен\-ные аппроксимации. Локальные и~глобальные показатели точ\-ности этих 
  приближений в~статье так\-же пред\-став\-ле\-ны. Примечательно, что  част\-ный случай 
  аппроксимаций~(\ref{eq:filt_2}) при $n\hm=0$ и~$\Lambda\hm=0$ был ранее 
  пред\-став\-лен в~\cite{B_17_1,B_17_2} для решения задачи байесовской классификации 
  случайного вектора по непрерывным наблюдениям с~мультипликативными шумами. 
 % 
Алгоритм оптимальной фильт\-ра\-ции и~его субоптимальные версии могут 
рас\-смат\-ри\-вать\-ся в~качестве основы чис\-лен\-ной реализации обобщения фильт\-ра 
Вонэма для сис\-тем с~мультипликативными шумами в~наблюдениях. 
Однако для их непосредственного использования необходимо решить 
следующие проб\-ле\-мы. Во-пер\-вых, в~(\ref{eq:h1}) и~(\ref{eq:h2}) присутствуют
 многомерные интегралы. Следует выяснить, какую результирующую погрешность 
 будут вносить ошибки их вы\-чис\-ле\-ния. Во-вто\-рых, представляется интересным 
 определить характеристики точ\-ности оптимальной фильт\-ра\-ции по дискретизованным 
 наблюдениям по отношению к~оптимальной фильт\-ра\-ции по непрерывным наблюдениям: 
 каков порядок точ\-ности по шагу временной дискретизации~$\Delta$? Для случая 
 вы\-чис\-ле\-ния классического фильт\-ра Вонэма с~по\-мощью алгоритма Эй\-ле\-ра--Ма\-ру\-ямы 
 подобный результат известен: порядок глобальной ошибки равен~${1}/{2}$. 
 Перечисленные задачи являются предметом дальнейших исследований.
 
 
  \vspace*{-10pt}
 
{\small
\subsection*{\raggedleft Приложение} 

\vspace*{-2pt}


\noindent
Д\,о\,к\,а\,з\,а\,т\,е\,л\,ь\,с\,т\,в\,о\ \ теоремы~1.\ \ Введем следующие 
обозначения для случайных величин и~мат\-риц, составленных из них:
\begin{align*}
\xi^{ji}(\ell)&\ebd 
\sum\limits_{h=0}^n \int\limits_{\mathcal{D}} 
 \mathcal{N}\left(Y_{\ell},f u,\sum\limits_{p=1}^N u^p g_pg_p^{\top}\right)
 \rho^{j,i,h}_{1}(du)\,; \\
  \theta^{ji}(\ell)&\ebd 
\sum\limits_{h=n+1}^{\infty} \int\limits_{\mathcal{D}} 
 \mathcal{N}\left(Y_{\ell},f u,\sum\limits_{p=1}^N u^p g_pg_p^{\top}\right)
 \rho^{j,i,h}_{1}(du)\,;
\\
 \xi(\ell)&\ebd \|\xi^{ji}(\ell)\|_{j,i=\overline{1,N}}\,,\quad 
 \Xi(r) \ebd \xi(r) \xi(r-1)\cdots \xi(1)\,;
 \\
 \theta(\ell)&\ebd \|\theta^{ji}(\ell)\|_{j,i=\overline{1,N}}\,, \quad 
 \Theta(r) \ebd \theta(r) \theta(r-1)\cdots \theta(1)\,.
%\label{eq:not_1}
\end{align*}
 
 Рекуррентные формулы~(\ref{eq:filt_1}) и~(\ref{eq:filt_2}) можно записать в~явной 
 форме
 
 
\noindent
\begin{align*}
 \widehat{X}_{t_r}& = \left( \mathbf{1}\left(\Xi(r) + 
 \Theta(r)\right)\pi\right)^{-1} \left(\Xi(r) + \Theta(r)\right)\pi\,;
\\
 \overline{X}_{t_r} &= \left( \mathbf{1}\Xi(r)\pi\right)^{-1} \Xi(r) \pi,
\end{align*}

\vspace*{-2pt}

\noindent
где $\mathbf{1} \ebd (1,\ldots,1)$~--- век\-тор-стро\-ка 
подходящей раз\-мер\-ности, составленная из единиц.

%Далее для краткости записи зависимость от~$r$ в~обозначениях~$\Xi(r)$ 
%и~$\Theta(r)$ будет опущена. 
Верна следующая цепочка неравенств:

 \vspace*{-3pt}

\noindent
\begin{multline}
\overline{\Sigma}_{t_r}(\pi)=%
%\me{}{\left\| 
%\widehat{X}_{t_r}(\pi, Y_1,\ldots,Y_r) - \overline{X}_{t_r}(\pi, Y_1,\ldots,Y_r)
%\right\|_1} =\\=
{\sf E}\left\{\left\| 
\fr{1}{\mathbf{1}\left(\Xi(r) + \Theta(r)\right)\pi} \left(\Xi(r) +{}\right.\right.\right.\\[-1pt]
\left.\left.\left.{}+ \Theta(r)\right)\pi
- \fr{1}{\mathbf{1}\Xi(r)\pi}\,\Xi(r) \pi
\right\|_1\right\} ={} \\[-1pt]
{}=
{\sf E}\left\{\fr{1}{\mathbf{1}\left(\Xi(r) + \Theta(r)\right)\pi \mathbf{1}\Xi(r)\pi}
\left\|
 \mathbf{1}\Xi(r) \pi \Theta(r)\pi -{}\right.\right.\\[-1pt]
\left.\left. {}- \mathbf{1}\Theta(r)\pi \Xi(r) \pi
 \right\|_1
 \vphantom{\fr{1}{\mathbf{1}\left(\Xi(r) + \Theta(r)\right)\pi \mathbf{1}\Xi(r)\pi}}
\right\} \leqslant {}\\[-1pt]
{}\leqslant 
{\sf E}\left\{\fr{1}{\mathbf{1}\left(\Xi(r) + \Theta(r)\right)\pi \mathbf{1}\Xi(r)\pi}
\left(
\mathbf{1}\Xi(r)\pi \| \Theta(r)\pi \|_1 +{}\right.\right.\\[-1pt]
\left.\left.{}+ \mathbf{1}\Theta(r)\pi 
\|
\Xi(r) \pi
\|_1
\right)
 \vphantom{\fr{1}{\mathbf{1}\left(\Xi(r) + \Theta(r)\right)\pi \mathbf{1}\Xi(r)\pi}}
\right\} ={}\\[-1pt]
{}=
2\,{\sf E}\left\{\fr{1}{\mathbf{1}\left(\Xi(r) + \Theta(r)\right)\pi}\mathbf{1}\Theta(r)\pi 
\right\}.
\label{eq:ineq_1}
\end{multline}

 
 \noindent
 Рассмотрим случайные события $a_{\ell} \ebd \{\omega \in \Omega: 
 N_{\ell}(\omega) \hm\leqslant n\}$, $\ell \hm= \overline{1,r}$, и~$A_r \ebd \{
 \omega\hm \in \Omega: \max_{1 \leqslant {\ell} \leqslant r}N_{\ell}(\omega) 
 \hm\leqslant n
 \}\hm=\prod\nolimits_{\ell=1}^r a_{\ell}$ и~оценку 
 $
 \widetilde{X}_{t_r}(\pi, Y_1,\ldots,Y_r)\ebd$\linebreak $\ebd
 {\sf E}\left\{X_{t_r}(\omega)\mathbf{I}_{A_r}(\omega)|\mathcal{O}_r\right\}.
 $
 Используя введенные выше обозначе\-ния и~абстрактный вариант формулы Байеса, 
 получаем, что
 
 \noindent
\begin{align}
\widetilde{X}_{t_r}& = \fr{1}{{\mathbf{1}\left(\Xi(r) + 
 \Theta(r)\right)\pi}}\,\Xi(r)\pi\,;\notag
 \\
\widehat{X}_{t_r} - \widetilde{X}_{t_r} &=
{\sf E}\left\{X_{t_r}(\omega)\mathbf{I}_{\overline{A}_r}(\omega)|\mathcal{O}_r\right\} ={}\notag\\[-1pt]
&\hspace*{17mm}{}= 
\fr{1}{\mathbf{1}\left(\Xi(r) + \Theta(r)\right)\pi}\Theta(r)\pi\,. 
\label{eq:eq_2}
 \end{align}
 Из (\ref{eq:ineq_1}) и~(\ref{eq:eq_2}) для $r\hm=1$ следует, что
 
 \vspace*{-4pt}
 
 \noindent
 \begin{multline}
 \overline{\sigma}(\pi) \leqslant 2\,{\sf E}
 \left\{\|{\sf E}\left\{X_{t_1}(\omega)\mathbf{I}_{\overline{a}_1}(\omega)|\mathcal{O}_1
 \right\}\|_1
 \right\} ={}\\[-1.5pt]
 {}=
 2\,{\sf E}\left\{\sum\limits_{n=1}^N {\sf E}
 \left\{X^n_{t_1}(\omega)\mathbf{I}_{\overline{a}_1}
 (\omega)|\mathcal{O}_1\right\}\right\} ={} \\[-2pt] 
 {}=
  2\,{\sf E}\left\{{\sf E}\left\{\mathbf{I}_{\overline{a}_1}(\omega)|\mathcal{O}_1
  \right\}\right\} =
   2 \mathbf{P}\left\{\overline{a}_1(\omega)\right\}.
\label{eq:ineq_3}
\end{multline}

 \vspace*{-2pt}
 
 \noindent
 Процесс $N^X_t$ общего числа скачков состояния~$X_t$ является считающим, и~его
  квадратическая характеристика равна 
  
\vspace*{-2pt}
  
  \noindent
 $$
 \langle N^X, N^X\rangle_t = - \int\limits_0^t \sum\limits_{n=1}^N \lambda_{nn} X_s^n\,ds\,,
 $$
 поэтому искомая вероятность ограничена сверху:
 $$ 
 \mathbf{P}\left\{\overline{a}_1(\omega)\right\} \leqslant 
 e^{-\overline{\lambda}\Delta}\sum\limits_{k=n+1}^{\infty} 
 \fr{(\overline{\lambda}\Delta)^{k}}{k!} <
 \fr{(\overline{\lambda}\Delta)^{n+1}}{(n+1)!}.
 $$
 
  \vspace*{-2pt}
  
  \noindent
 Из последнего неравенства и~(\ref{eq:ineq_3}) следует, что  для любого 
 начального распределения~$\pi$ выполняется неравенство $\overline{\sigma}(\pi)  
 \hm< 2({(\overline{\lambda}\Delta)^{n+1}}/{(n+1)!})$, т.\,е.\ 
 локальная оценка~(\ref{eq:prec_loc}) верна.
 
 С помощью марковского свойства пары $(X_t, N^X_t)$ и~последнего 
 неравенства можно оценить сверху вероятность 
 $\mathbf{P}\left\{\overline{A}_r(\omega)\right\}$:
 
  \vspace*{-2pt}
 
 \noindent
 \begin{multline*}
 \mathbf{P}\left\{\overline{A}_r(\omega)\right\} \leqslant 1 - \left(
 1- \fr{(\overline{\lambda}\Delta)^{n+1}}{(n+1)!}
 \right)^r \leqslant r \fr{(\overline{\lambda}\Delta)^{n+1}}{(n+1)!} + {}\\[-1pt]
 {}+\left|
 \sum\limits_{k=2}^r C_r^k \left(-\fr{(\overline{\lambda}\Delta)^{n+1}}{(n+1)!}
 \right)^k
 \right| \leqslant
 r \fr{(\overline{\lambda}\Delta)^{n+1}}{(n+1)!} +{}\\[-1pt]
 {}+\fr{r(r-1)}{2}
 \left(
 \fr{(\overline{\lambda}\Delta)^{n+1}}{(n+1)!}
 \right)^2
 \left(
 1-\fr{(\overline{\lambda}\Delta)^{n+1}}{(n+1)!}
 \right)^{r-2},
 \end{multline*} 
 из чего следует истинность глобальной оценки~(\ref{eq:prec_glob}).
Теорема~1 доказана.

}

%\vspace*{-12pt}

{\small\frenchspacing
 {%\baselineskip=10.8pt
 \addcontentsline{toc}{section}{References}
 \begin{thebibliography}{99}

\bibitem{Won_65}
\Au{Wonham W.} 
Some applications of stochastic differential equations to optimal
  nonlinear filtering~//
SIAM~J.~Control, 1965. Vol.~2. P.~347--369. 

\bibitem{KP_92}
\Au{Kloeden P., Platen E.} Numerical solution of stochastic
differential equations.~--- Berlin: Springer, 1992.~636~p.

\bibitem{YZL_04}
\Au{Yin G., Zhang Q., Liu Y.} 
Discrete-time approximation of Wonham filters~//
J.~Control Theory Applications, 2004. Iss.~2. P.~1--10.

\bibitem{PR_10}
\Au{Platen E., Rendek R.}
Quasi-exact approximation of hidden Markov chain filters~//
Communicat.~Stoch.~Analys., 2010. Vol.~4. Iss.~1. P.~129--142.

\bibitem{B_18}
\Au{Борисов А.} Фильтрация Вонэма по наблюдениям с~мультипликативными шумами~// 
Автоматика и~телемеханика, 2018.
№~1. C.~52--65. 
 
  \bibitem{BSh_85} %6
\Au{Бертсекас Д., Шрив С.} Стохастическое оптимальное управление. 
Случай дискретного времени~/ Пер. с~англ.~--- М.: Наука, 1985.~280~c.
(\Au{Betsekas~D.\,P., Shreve~S.\,E.} Stochastic optimal control:
The discrete-time case.~--- Orlando, FL, USA:
Academic Press Inc., 1978. 323~p.)

  \bibitem{ZhSh_95} %7
\Au{Жакод Ж., Ширяев А.} Предельные теоремы для случайных процессов,~I.~/
Пер. с~англ.~--- 
М.: Физматлит, 1995.~544~c.
(\Au{Jacod~J., Shiryaev~A.} Limit theorems for stochastic processes.~---
Berlin: Springer, 2003. 664~p.)

\bibitem{S_00}
\Au{Sericola B.} Occupation times in Markov processes~//
Commun. Stat. Stochastic Models, 2000. Vol.~16. Iss.~5. P.~479--510. 

  \bibitem{B_80}
\Au{Боровков А.} Асимптотические методы в~тео\-рии массового обслуживания.~--- 
М.: Физматлит, 1995.~384~c.

  \bibitem{B_17_1}
\Au{Борисов А.} Классификация по непрерывным наблюдениям с~мультипликативными шумами.~I. 
Формулы байесовской оценки~// Информатика и~её применения, 2017. Т.~11. Вып.~1. C.~11--19.
doi: 10.14357/19922264170102.

  \bibitem{B_17_2}
\Au{Борисов А.} Классификация по непрерывным наблюдениям с~мультипликативными 
шумами.~II. Алгоритм численной реализации оценки~// Информатика и~её 
применения, 2017. Т.~11. Вып.~2. C.~33--41.
doi: 10.14357/19922264170204.

 \end{thebibliography}

 }
 }

\end{multicols}

\vspace*{-4pt}

\hfill{\small\textit{Поступила в~редакцию 10.07.18}}

\vspace*{6pt}

%\pagebreak

%\newpage

%\vspace*{-28pt}

\hrule

\vspace*{2pt}

\hrule

%\vspace*{-2pt}

\def\tit{FILTERING OF~MARKOV JUMP PROCESSES\\ BY~DISCRETIZED OBSERVATIONS}

\def\titkol{Filtering of Markov jump processes by discretized observations}

\def\aut{A.\,V.~Borisov}

\def\autkol{A.\,V.~Borisov}

\titel{\tit}{\aut}{\autkol}{\titkol}

\vspace*{-11pt}


\noindent
Institute of Informatics Problems, Federal Research Center ``Computer Science 
and Control'' of the Russian Academy of Sciences, 44-2~Vavilov Str., Moscow 
119333, Russian Federation


\def\leftfootline{\small{\textbf{\thepage}
\hfill INFORMATIKA I EE PRIMENENIYA~--- INFORMATICS AND
APPLICATIONS\ \ \ 2018\ \ \ volume~12\ \ \ issue\ 3}
}%
 \def\rightfootline{\small{INFORMATIKA I EE PRIMENENIYA~---
INFORMATICS AND APPLICATIONS\ \ \ 2018\ \ \ volume~12\ \ \ issue\ 3
\hfill \textbf{\thepage}}}

\vspace*{6pt}



\Abste{The article is devoted to a~solution of the optimal filtering problem 
of a~homogenous Markov
jump process state. The available observations represent 
time increments of the integral transformations of the Markov\linebreak\vspace*{-12pt}}

\Abstend{state corrupted by 
Wiener processes. The noise intensity is also state-dependent. At the instant of 
the consecutive
observation obtaining, the optimal estimate is calculated recursively 
as a~function of previous estimate and the new observation, meanwhile between 
observations the filtering estimate is a simple forecast by virtue of the Kolmogorov 
differential system. The recursion is rather expensive because of  need to calculate 
the integrals, which are the location-scale mixtures of Gaussians. The mixing 
distributions represent the occupation of the state in each of possible values 
during the mid-observation intervals. The paper contains numerically cheaper 
approximations, based on the restriction of the state transitions number between 
the observations. Both the local and global characteristics of approximation 
accuracy are obtained as functions of the dynamics parameters, mid-observation 
interval length, and upper bound of transitions number.}

\KWE{Markov jump process; optimal filtering; multiplicative observation noises; 
stochastic differential equation; numerical approximation}




\DOI{10.14357/19922264180316}

%\vspace*{-14pt}

\Ack
\noindent
The work was supported in part by the Russian Foundation
for Basic Research (Project No.\,16-07-00677).



%\vspace*{6pt}

  \begin{multicols}{2}

\renewcommand{\bibname}{\protect\rmfamily References}
%\renewcommand{\bibname}{\large\protect\rm References}

{\small\frenchspacing
 {%\baselineskip=10.8pt
 \addcontentsline{toc}{section}{References}
 \begin{thebibliography}{99}
\bibitem{Won_65-1}
\Aue{Wonham, W.} 1965.
Some applications of stochastic differential equations to optimal
  nonlinear filtering.
\textit{SIAM~J.~Control} 2:347--369. 

\bibitem{KP_92-1}
\Aue{Kloeden,~P., and E.~Platen.} 1992. \textit{Numerical solution of stochastic
differential equations.} Berlin: Springer. 636~p.

\bibitem{YZL_04-1}
\Aue{Yin,~G., Q.~Zhang, and Y.~Liu.} 2004.
Discrete-time approximation of Wonham filters.
\textit{J.~Control Theory Applications} 2:1--10.

\bibitem{PR_10-1}
\Aue{Platen, E., and R.~Rendek.} 2010.
Quasi-exact approximation of hidden Markov chain filters.
\textit{Communicat. Stoch. Analys.} 4(1):129--142.

\bibitem{B_18-1}
\Aue{Borisov, A.} 2018. Wonham filtering by observations
with multiplicative noises. \textit{Automat.~Rem.~Contr.} 79(1):39--50.  
doi: 10.1134/ S0005117918010046.
 
  \bibitem{BSh_85-1}
\Aue{Bertsekas, D., and S.~Shreve.} 1996.
\textit{Stochastic optimal control: The discrete-time case}.
Nashua, NH: Athena Scientific. 330~p.
  
  \bibitem{ZhSh_95-1}
  \Aue{Jacod,~J., and A.~Shiryaev.} 2003.
\textit{Limit theorems for stochastic processes.}
Berlin: Springer. 664~p.

\bibitem{S_00-1}
\Aue{Sericola, B.}
2000. Occupation times in Markov processes.
\textit{Commun. Stat.} 16(5):479--510. 

  \bibitem{B_80-1}
\Aue{Borovkov, A.} 1984.
 \textit{Asymptotic methods in queueing theory}. 
 Hoboken, NJ: Wiley-Blackwell.~304~p.

  \bibitem{B_17_1-1}
  \Aue{Borisov, A.} 2017. 
  Klassifikatsiya po ne\-pre\-ryv\-nym nablyu\-de\-miyam s~mul'tiplikativnymi shumami. I. 
  Formuly bayesov\-skoy otsenki [Classification by continuous-time observations
in multiplicative noise. I.~Formulae for Bayesian 
estimate]. \textit{Informatika i~ee Primeneniya~--- Inform.~Appl.}
11(1):11--19. doi: 10.14357/19922264170102.

  \bibitem{B_17_2-1}
\Aue{Borisov, A.} 2017. Klassifikatsiya po nepreryvnym nablyudemiyam 
s~mul'tiplikativnymi summami. II.~Formuly bayesovskoy otsenki 
[Classification by continuous-time observations
in multiplicative noise. II.~Numerical algorithm].
\textit{Informatika i~ee Primeneniya~--- Inform.~Appl.}
11(2):33--41. doi: 10.14357/19922264170204.

\end{thebibliography}

 }
 }

\end{multicols}

\vspace*{-6pt}

\hfill{\small\textit{Received July 10, 2018}}

%\pagebreak

%\vspace*{-18pt}

\Contrl

\noindent
\textbf{Borisov Andrey V.} (b.\ 1965)~--- 
Doctor of Science in physics and mathematics, principal scientist, Institute of
Informatics Problems, Federal Research Center ``Computer Science and Control''
 of the Russian Academy of
Sciences, 44-2 Vavilov Str., Moscow 119333, Russian Federation; 
\mbox{aborisov@frccsc.ru}
\label{end\stat}

\renewcommand{\bibname}{\protect\rm Литература}        %4
\def\stat{bosov+stef}

\def\tit{УПРАВЛЕНИЕ ВЫХОДОМ СТОХАСТИЧЕСКОЙ ДИФФЕРЕНЦИАЛЬНОЙ СИСТЕМЫ 
ПО~КВАДРАТИЧНОМУ КРИТЕРИЮ. I.~ОПТИМАЛЬНОЕ РЕШЕНИЕ МЕТОДОМ 
ДИНАМИЧЕСКОГО ПРОГРАММИРОВАНИЯ$^*$}

\def\titkol{Управление выходом стохастической дифференциальной системы 
по~квадратичному критерию. I}
%.~Оптимальное решение методом 
%динамического программирования}

\def\aut{А.\,В.~Босов$^1$, А.\,И.~Стефанович$^2$}

\def\autkol{А.\,В.~Босов, А.\,И.~Стефанович}

\titel{\tit}{\aut}{\autkol}{\titkol}

\index{Босов А.\,В.}
\index{Стефанович А.\,И.}
\index{Bosov A.\,V.}
\index{Stefanovich A.\,I.}




{\renewcommand{\thefootnote}{\fnsymbol{footnote}} \footnotetext[1]
{Работа выполнена при частичной поддержке РФФИ (проект 16-07-00677).}}


\renewcommand{\thefootnote}{\arabic{footnote}}
\footnotetext[1]{Институт проблем информатики Федерального исследовательского центра <<Информатика 
и~управление>> Российской академии наук, \mbox{AVBosov@ipiran.ru}}
\footnotetext[2]{Институт проблем информатики Федерального исследовательского центра <<Информатика 
и~управление>> Российской академии наук, \mbox{AStefanovich@frccsc.ru}}

%\vspace*{8pt}



  
  \Abst{Решается задача оптимального управления для диффузионного процесса 
Ито и~линейного управ\-ля\-емо\-го выхода. Рассматриваемая постановка близка 
к~классической ли\-ней\-но-квад\-ра\-тич\-ной гауссовской задаче управления 
(linear-quadratic Gaussian (LQG) control). Отличия состоят в~том, что состояние описывается нелинейным 
дифференциальным уравнение Ито $dy_t\hm= A_t(y_t) \,dt\hm+ \Sigma_t(y_t)\,dv_t$ 
и~не зависит от управ\-ле\-ния~$u_t$, оптимизации подлежит управ\-ля\-емый 
линейный выход $dz_t\hm= a_t y_t\,dt\hm+ b_t z_t \,dt\hm+ c_t u_t \,dt\hm+ \sigma_t\, 
dw_t$. Дополнительные обобщения внесены в~квад\-ра\-тич\-ный критерий качества 
с~целью воз\-мож\-ности постановки таких задач, как отслеживание выходом 
состояния или управ\-ле\-ни\-ем~--- линейной комбинации состояния и~выхода. Для 
решения используется метод динамического программирования. Функцию 
Беллмана позволяет найти предположение о~ее структуре вида $V_t(y,z)\hm= 
\alpha_t z^2\hm+ \beta_t(y)z \hm+\gamma_t(y)$. Решение дают три 
дифференциальных уравнения для коэффициентов~$\alpha_t$, $\beta_t(y)$ 
и~$\gamma_t(y)$. Эти уравнения со\-став\-ля\-ют оптимальное решение 
рас\-смат\-ри\-ва\-емой задачи.}
  
  \KW{стохастическое дифференциальное уравнение; оптимальное управ\-ле\-ние; 
динамическое программирование; функция Беллмана; уравнение Риккати; 
линейные уравнения параболического типа}

\DOI{10.14357/19922264180314}
  
%\vspace*{4pt}


\vskip 10pt plus 9pt minus 6pt

\thispagestyle{headings}

\begin{multicols}{2}

\label{st\stat}

\section{Введение}

     Ключевые результаты в~области оптимизации стохастических 
динамических систем, со\-став\-ля\-ющие классическую теорию управления, 
получены более~40~лет назад (такова работа~[1] в~отношении задачи 
управ\-ле\-ния ли\-ней\-но-гаус\-сов\-ски\-ми стохастическими сис\-те\-ма\-ми по 
квад\-ра\-тич\-но\-му критерию). К~классической тео\-рии следует относить 
линейные модели стохастических сис\-тем и~квадратичный критерий качества. 
Это исходный базис, на котором основано множество успешно 
исследованных и~решенных задач стохастического управ\-ле\-ния 
и~оптимизации. 

Дальнейшее развитие~--- это новые модели и~критерии, но 
прежде всего это новые методы: от тео\-рии линейных регуляторов, метода 
динамического программирования и~принципа максимума к~адаптивному 
и~минимаксному подходу, импульсному управ\-ле\-нию и~т.\,д. Множество 
инноваций как в~час\-ти моделей, так и~в~час\-ти математического аппарата, 
имевших мес\-то в~по\-сле\-ду\-ющие годы, существенно обогатили тео\-рию 
управ\-ле\-ния. Но и~до настоящего времени линейные модели и~квадратичный 
критерий, несмотря на всю справедливую критику в~отношении их 
аде\-кват\-ности и~гиб\-кости, сохраняют исследовательский интерес и~находят 
современные области приложения.
     
     Не претендуя на сколь\-ко-ни\-будь полное обосно\-ва\-ние последнего 
тезиса, приведем несколько примеров, показавшихся наиболее ин\-те\-рес\-ными. 

Так, в~[2] решается ли\-ней\-но-квад\-ра\-тич\-ная за\-да\-ча в~игровой 
постановке с~запаздыванием. В~близ\-кой по модели работе~[3] задача 
управ\-ле\-ния ставится в~терминах $H_\infty$-ро\-баст\-ности. Точнее \mbox{называть} 
эту тематику $H_2/H_\infty$-управ\-ле\-ни\-ем, и~работ по этой теме очень 
много. Аккуратности ради следует уточнить, что под линейными 
понимаются модели с~мультипликативными по состоянию воз\-му\-ще\-ниями. 

Совсем другой класс моделей, особо популярных в~по\-след\-ние годы, 
составляют скачкообразные процессы. Например, линейные уравнения 
в~сочетании с~пуассоновскими скачками в~[4] используются в~моделях, 
описывающих различные показатели функционирования сетевых протоколов 
передачи данных транспортного уровня. Телекоммуникации представляют 
в~последние годы самый популярный прикладной материал для 
исследований, работ по этой проб\-ле\-ма\-ти\-ке множество, математические 
техники привлекаются самые разные и~самые современные, но и~линейным 
моделям место находится. Еще один любопытный пример исследования 
скачкообразного процесса и~оптимизации на основе квад\-ра\-тич\-но\-го критерия 
можно найти в~[5] применительно к~задаче инвестирования на финансовом 
рынке. Наконец, упомянем еще работу~[6], подводящую итог исследований 
в~отношении классической детерминированной  
ли\-ней\-но-квад\-ра\-тич\-ной задачи с~использованием техники матричных 
неравенств.
     
     В данной работе также эксплуатируются привлекательные свойства 
линейных моделей и~квад\-ра\-тич\-но\-го критерия, причем в~стохастической 
постановке. На\-прав\-ле\-ни\-ем для обобщения \mbox{выбрана} модель динамики 
сис\-те\-мы: основные усилия на\-прав\-ле\-ны на то, чтобы сделать ее нелинейной. 
Кроме того, пред\-став\-лен\-ная постановка может рас\-смат\-ри\-вать\-ся и~как 
обобщение ранее решенной задачи в~дискретном времени~[7, 8] на время 
непрерывное. В~упомянутых работах помимо собственно модельной 
постановки важна еще и~привлекаемая прикладная об\-ласть~--- 
функционирование сложных программных сис\-тем. Результатов, 
ориентированных непосредственно на такие приложения, к~настоящему 
времени пренебрежимо мало, поэтому~[7, 8]~--- это еще и~прикладное 
обоснование рас\-смат\-ри\-ва\-емой далее задачи.
     
     Оптимизируемая динамическая сис\-те\-ма описывается двумя 
уравнениями. Состояние задается нелинейным стохастическим 
дифференциальным уравнением Ито, не содержащим управ\-ля\-емой 
переменной. Возмущение здесь описывается стандартным винеровским 
процессом, накладываются простые условия существования 
и~един\-ст\-вен\-ности решения. Поскольку состояние не управ\-ля\-ет\-ся, то уместно 
его интерпретировать как слож\-ное внешнее возмущение. Вторая 
переменная~--- управ\-ля\-емый выход~--- задается линейным стохастическим 
дифференциальным уравнением. Цель оптимизации выхода формируется 
квадратичным критерием общего вида. Формальная постановка задачи 
приведена в~сле\-ду\-ющем разделе.
     
     Для решения задачи используется метод динамического 
программирования, решается уравнение Беллмана~[9]. Соответственно, 
в~результате получаются аналитические выражения и~для оптимального 
управ\-ле\-ния, и~для значения функционала качества. Технически 
традиционный, стандартный подход к~задаче обременен, пожалуй, 
единственной проблемой~--- поиском верного пред\-став\-ле\-ния структуры 
функции Беллмана. Справиться с~этой проблемой в~большей степени удается 
за счет результата, полученного при решении дискретного по времени 
аналога рассматриваемой постановки~\cite{8-bos}. Конечные соотношения 
для оптимального решения, как и~во всех подобных задачах, включая 
классическую ли\-ней\-но-квад\-ра\-тич\-ную, содержат решения 
определенных дифференциальных уравнений (обыкновенных и~в~частных 
производных). Вывод этих уравнений и~со\-став\-ля\-ет содержание первой час\-ти 
данной работы. Во второй части будет обсуждаться их приближенное 
чис\-лен\-ное решение и~компьютерные эксперименты.
     
     Кратко обозначим основные положения, при\-вле\-ка\-емые далее 
к~решению задачи, следуя в~основном обозначениям 
и~терминологии~\cite{9-bos}, а~именно: будем рассматривать задачу 
оптимального управления в~стохастической динамической сис\-те\-ме по полной 
информации, применяя метод динамического программирования. В~качестве 
целевого функционала, опре\-де\-ля\-юще\-го качество управ\-ле\-ния $U_0^T\hm= \{ 
u_t,\ 0\leq t\leq T\}$, выступает
     \begin{equation}
     J\left(U_0^T\right)={\sf E}\left\{ \int\limits_0^T L_t \left(x_t, u_t\right)\,dt+ 
l\left(x_T\right)\right\}\,.
     \label{e1-bos}
     \end{equation}
Здесь ${\sf E}\{\cdot\}$~--- оператор математического ожидания; $x_t$~--- 
случайный процесс, описываемый стохастическим дифференциальным 
уравнением Ито
     \begin{equation}
     dx_t=m_t\left( x_t, u_t\right) dt+ \sigma_t\left( x_t\right)dW_t\,,\enskip 
x_0=X\,,
     \label{e2-bos}
     \end{equation}
где $W_t$~--- стандартный винеровский процесс подходящей раз\-мер\-ности; 
$X$~--- случайный вектор.

     $U_0^T$ будем выбирать из класса допустимых неупреждающих (по 
отношению к~$W_t$) управлений~\cite{9-bos}. Соответственно, 
относительно функций сноса и~диффузии~$m_t$ и~$\sigma_t$  
в~(\ref{e2-bos}) будем предполагать выполненными ка\-кие-ли\-бо условия 
существования сильного решения для заданного до\-пус\-ти\-мо\-го управ\-ле\-ния. 
Например, для управ\-ле\-ния с~обратной связью $u_t\hm= u_t(x_t)$ будем 
считать, что $m_t(x,u_t(x))$ и~$\sigma_t(x)$ удовлетворяют условию 
линейного рос\-та и~локальному условию Липшица по~$x$ равномерно 
по~$t$ (т.\,е.\ условиям Ито).
     
     Для поиска оптимального управления, минимизирующего $J(U_0^T)$, 
рас\-смат\-ри\-ва\-ет\-ся функция Беллмана
     \begin{equation}
     V_t(x)=\left.\mathop{\mathrm{inf}}\limits_{U_t^T} {\sf E} \left\{ \int\limits_t^T 
L_t \left( x_t, u_t\right)\,dt+l\left( x_T\right) \right\vert \mathcal{F}_t^x\right\}\,,
     \label{e3-bos}
     \end{equation}
где $\mathcal{F}_t^x$~--- $\sigma$-ал\-геб\-ра, по\-рож\-ден\-ная~$x_\tau$, 
$0\hm\leq \tau\hm\leq t$, ${\sf E}\{\cdot\vert \mathcal{F}\}$~--- оператор условного 
математического ожидания относительно~$\mathcal{F}$. Соответственно, 
в~качестве достаточного условия оп\-ти\-маль\-ности воспользуемся уравнением 
динамического программирования
\begin{multline}
\fr{\partial V_t(x)}{\partial t} +\fr{1}{2}\sum\limits^n_{i,j=1} \sigma^2_{t_{ij}}
\fr{\partial^2 V_t(x)}{\partial x_i \partial x_j}+{}\\
{}+\min\limits_u\left[  
\sum\limits^n_{i=1} m_{t_i} \fr{\partial V_t(x)}{\partial x_i} + L_t(x,u)\right] 
=0\,,\\
V_T(x)=l(x)\,,
\label{e4-bos}
\end{multline}
где $m_{t_i}$~--- $i$-й элемент век\-тор-функ\-ции~$m_t(x,u)$; 
$\sigma^2_{t_{ij}} \hm= \sum\nolimits^m_{k=1} 
\sigma_{t_{ik}}\sigma_{t_{ki}}$, $\sigma_{t_{ij}}$~--- $i$-й по строке, $j$-й 
по столб\-цу элемент мат\-рич\-ной функции~$\sigma_t(x)$; $n$ и~$m$~--- 
размерности~$x_t$ и~$W_t$ соответственно.

     Традиционно в~рамках применения метода динамического 
программирования будем предполагать, что функции~$L_t$, $l$, $m_t$ 
и~$\sigma_t$ обеспечивают существование хотя бы одного решения 
уравнения~(\ref{e4-bos}), а~следовательно, и~оптимального 
управления~$u_t^*$, $0\hm\leq t\hm\leq T$, до\-став\-ля\-юще\-го минимум 
целевому функционалу~(\ref{e1-bos}). Задача оптимизации далее получается 
путем указания конкретных выражений для~$L_t$, $l$, $m_t$ и~$\sigma_t$.

\section{Постановка задачи управления выходом}

     Рассматриваемые далее случайные функции будут предполагаться 
скалярными. Такое упрощение позволит разгрузить выкладки и~итоговые 
выражения от не самых существенных деталей.
     
     Рассмотрим стохастическую дифференциальную сис\-те\-му, со\-сто\-яние 
которой представляет диффузи\-он\-ный процесс~$y_t$, описываемый 
нелинейным стохастическим дифференциальным уравнением Ито
     \begin{equation}
     dy_t=A_t\left( y_t\right) dt +\Sigma_t \left( y_t\right) dv_t\,,\enskip 
y_0=Y\,,
     \label{e5-bos}
     \end{equation}
где $v_t$~--- стандартный (одномерный) винеровский процесс; $Y$~--- 
случайная величина с~конечным вторым моментом; функции~$A_t$ 
и~$\Sigma_t$ удовлетворяют условиям Ито:
\begin{equation*}
\left\vert A_t(y)\right\vert +\left\vert \Sigma_t(y)\right\vert \leq C(1+\vert y\vert )\ 
\mbox{для\ всех } 0\leq t\leq T\,;
\end{equation*}

\vspace*{-12pt}

\noindent
\begin{multline*}
\hspace*{-2.10051pt}\left\vert A_t\left(y_1\right) -A_t \left( y_2\right) \right\vert +\left\vert 
\Sigma_t\left( y_1\right) -\Sigma_t \left(y_2\right)\right\vert \leq
C\left\vert y_1-y_2\right\vert\\
 \mbox{для\ всех\ } 0\leq t\leq T\ \mbox{и } 
y_1,y_2\in \mathbb{R}^1\,,
\end{multline*}
обеспечивающим существование единственного сильного (потраекторного) 
решения уравнения.
     
     Будем считать, что~$y_t$ описывает состояние некоторой 
динамической системы. Соответственно, поведение этой сис\-те\-мы опишем 
выходом, линейно связанным с~со\-сто\-янием:
     \begin{equation}
     dz_t=a_t y_t \,dt+ b_t z_t \,dt+ c_t u_t \,dt+\sigma_t \,dw_t\,,\enskip
     z_0=Z\,.
     \label{e6-bos}
     \end{equation}
Здесь $w_t$~--- не зависящий от~$v_t$, $Y$ и~$Z$ стандартный (одномерный) 
винеровский процесс; $Z$~--- случайная величина с~конечным вторым 
моментом; $u_t$~--- допустимое неупреждающее управ\-ле\-ние, качество 
которого определяется целевым функционалом следующего вида:
\begin{multline}
\!\hspace*{-3.98538pt}J\left( U_0^T\right) ={\sf E}\left\{ \int\limits_0^T \!\left( S_t\left( s_ty_t-g_t z_t -h_t 
u_t\right)^2 +G_t z_t^2+{}\right.\right.\\
\left.\left.{}+ H_t u_t^2
\vphantom{S_t\left( s_ty_t-g_t z_t -h_t 
u_t\right)^2}
\right) dt+S_T\left( s_T y_T -g_T 
z_T\right)^2+G_T z_T^2
\vphantom{\int\limits_0^T}\right\}\,,
\label{e7-bos}
\end{multline}
где $S_t$, $G_t$ и~$H_t$~--- неотрицательные функции\linebreak
$0\hm\leq t\hm\leq T$. 
Такой критерий отражает физический смысл задачи распределения ресурсов 
со\-глас\-но аналогичной~(\ref{e5-bos})--(\ref{e7-bos}) задаче для дис\-крет\-но\-го 
времени, рас\-смот\-рен\-ной в~\cite{7-bos}. В~част\-ности,  
функци\-онал~(\ref{e7-bos}) поз\-во\-ля\-ет ставить задачи отслеживания
 выходом 
со\-сто\-яния сис\-те\-мы, используя сла\-га\-емое $(y_t\hm- z_t)^2$, или 
управлением~--- линейной комбинации со\-сто\-яния и~выхода, сла\-га\-емое типа\linebreak 
$(y_t\hm+ z_t\hm- u_t)^2$. Поскольку задача формулируется 
в~предположении наличия пол\-ной информации о~со\-сто\-янии~$y_t$ 
и~выходе~$z_t$ (соответствующую $\sigma$-ал\-геб\-ру 
обозначим~$\mathcal{F}_t^{y,z}$), то допустимое управ\-ле\-ние ищется 
в~классе~$\mathcal{F}_t^{y,z}$-из\-ме\-ри\-мых неупреждающих функций 
(и,~как будет показано далее, оказывается управ\-ле\-ни\-ем с~обратной связью).

     Функции~$a_t$, $b_t$, $c_t$ и~$\sigma_t$ будем предполагать 
ограниченными: $\vert a_t\vert \hm+ \vert b_t\vert \hm+\vert c_t\vert \hm+ \vert 
\sigma_t \vert \hm\leq C$ для всех $0\hm\leq t\hm\leq T$, процесс  
управления~--- допустимым не\-упреж\-да\-ющим~\cite{9-bos}, обеспечивая, 
таким образом, существование сильного решения урав\-не\-ния~(\ref{e6-bos}) 
для любого допустимого управ\-ления.
     
     Задачу составляет поиск~$u_t^*$~--- допустимого управ\-ле\-ния, 
доставляющего минимум квад\-ра\-тич\-но\-му функционалу~$J(U_0^T)$.
      
     Поставленная задача очевидным образом формулируется в~терминах 
введенных выше в~(\ref{e1-bos})--(\ref{e3-bos}) обозначений, а~именно: 
     требуется обозначить
     \begin{gather*}
      x_t=\begin{pmatrix}
     y_t\\ z_t\end{pmatrix};\quad  m_t(x_t, u_t)=\begin{pmatrix}
     A_t(y_t)\\ a_t y_t +b_t z_t +c_t u_t\end{pmatrix};\\
     \sigma_t(x_t)= \begin{pmatrix}
     \Sigma_t(y_t)& 0\\
     0& \sigma_t\end{pmatrix};\quad W_t=\begin{pmatrix}
     v_t \\ w_t\end{pmatrix}
     %     \label{e8-bos}
     \end{gather*}
для записи уравнения со\-сто\-яния типа~(\ref{e2-bos}) и
\begin{align*}
L_t(x,u)&= L_t(y,z,u) ={}\\
&\hspace*{3mm}{}=S_t\left( s_t y-g_t z -h_t u\right)^2 +G_t z^2 +H_t  u^2\,;\\
l(x)&= l(y,z) =S_T \left( S_T y-g_T z\right)^2 +G_T z^2
%\label{e9-bos}
\end{align*}
для записи целевого функционала в~виде~(\ref{e1-bos}).

     Функция Беллмана~(\ref{e3-bos}) принимает вид 
     $V_t(x)\hm= V_t(y,z)$. Для записи со\-от\-вет\-ст\-ву\-юще\-го~(\ref{e4-bos}) 
уравнения Беллмана для~$V_t(y,z)$ заметим, что
     $$
     \left( \sigma^2_{t_{ij}}\right)_{i,j=1,2}= \begin{pmatrix}
     \Sigma_t^2(y) & 0\\
     0 & \sigma_t^2\end{pmatrix}\,.
     $$
     
     С~учетом перечисленных обозначений урав\-не\-ние динамического 
программирования~(\ref{e4-bos}) принимает вид:
     \begin{multline}
     \fr{\partial V_t(y,z)}{\partial t} +\fr{1}{2}\left( \Sigma_t^2(y) \fr{\partial^2 
V_t(y,z)} {\partial y^2}+\sigma_t^2\fr{\partial^2 V_t(y,z)} {\partial 
z^2}\right)+{}\\
    {}+\min\limits_u\! \left[ A_t(y) \fr{\partial V_t(y,z)}{\partial y}+\left( a_t 
y+b_t z+c_t u\right) \fr{\partial V_t(y,z)}{\partial z} +{}\right.\hspace*{-3pt}\\
\left.{}+ S_t\left( s_t y-g_t z-h_t 
u\right)^2+G_t z^2+H_t u^2
     \vphantom{\fr{\partial V_t(y,z)}{\partial y}}\right] =0\,,\\
     V_T(y,z)=S_T\left( s_T y-g_T z\right)^2+G_T z^2\,.
     \label{e10-bos}
     \end{multline}
     Это и~есть то самое уравнение, которое требуется решить: 
существование решения данного урав\-не\-ния суть достаточное условие 
оптимальности; оптимальное управ\-ле\-ние при этом~--- точ\-ка минимума 
со\-от\-вет\-ст\-ву\-юще\-го сла\-га\-емого.
     
\section{Динамическое программирование и~оптимальное 
управление}

     В рассматриваемой постановке линейность\linebreak выхода и~квадратичность 
критерия дают те же преимущества, что и~в~классической  
ли\-ней\-но-квад\-ра\-тич\-ной задаче управ\-ле\-ния~\cite{1-bos}, а~именно: 
позволяют сразу определить вид оптимального управ\-ле\-ния и~фактические 
условия его существования. Действительно, со\-хра\-няя в~(\ref{e10-bos}) под 
знаком $\min\nolimits_u$ только члены, зависящие от~$u$, получаем
     \begin{multline*}
     \fr{\partial V_t(y,z)}{\partial t} +\fr{1}{2}\left( \Sigma_t^2(y) \fr{\partial^2 
V_t(y,z)} {\partial y^2}+\sigma_t^2\fr{\partial^2 V_t(y,z)} {\partial 
z^2}\right)+{}\\
     {}+A_t(y)\fr{\partial V_t(y,z)}{\partial y}+\left( a_t y+b_t z\right) 
\fr{\partial V_t(y,z)}{\partial z}+{}\\
{}+S_t\left( s_t y-g_t z\right)^2 +G_t z^2+{}
\end{multline*}

\noindent
\begin{multline*}
     {}+\min\limits_u \left[ \left( c_t \fr{\partial V_t(y,z)}{\partial z}-2S_t \left( 
s_t y-g_t z\right) h_t\right)u +{}\right.\\
\left.{}+\left( S_t h_t^2+H_t\right) u^2
\vphantom{\fr{\partial V_t(y,z)}{\partial z}}
\right]=0\,,
     %\label{e11-bos}
     \end{multline*}
откуда в~предположении $S_t h_t^2\hm+ H_t\hm>0$ следует, что существует 
оптимальное управ\-ле\-ние, которое определяется равенством
\begin{multline}
u_t^* = u_t^*(y,z)=-\fr{1}{2}\left( S_t h_t^2 +H_t\right)^{-1} \left( c_t 
\fr{\partial V_t(y,z)}{\partial z}-{}\right.\\
\left.{}-2S_t\left( s_t y-g_t z\right) h_t
\vphantom{\fr{\partial V_t(y,z)}{\partial z}}
\right)
\label{e12-bos}
\end{multline}
и доставляет минимум соответствующему сла\-га\-емо\-му в~урав\-не\-нии Беллмана, 
равный
$-\left( S_t h_t^2\hm+\right.$\linebreak
$\left.{}+H_t\right)^{-1} \left( c_t 
{\partial V_t(y,z)}/{\partial 
z}\hm-2S_t\left( s_t y \hm-g_t z\right) h_t \right)^2/4.
$ 
     
     Отметим, что, как и~в~классической ли\-ней\-но-квад\-ра\-тич\-ной 
задаче, управ\-ле\-ние из класса до\-пус\-ти\-мых не\-упреж\-да\-ющих получилось 
управ\-ле\-ни\-ем с~обратной связью.
     
     Таким образом, функция Беллмана описывается сле\-ду\-ющим 
дифференциальным уравнением:
     \begin{multline}
     \fr{\partial V_t(y,z)}{\partial t} +\fr{1}{2}\left( \Sigma_t^2(y) \fr{\partial^2 
V_t(y,z)} {\partial y^2}+\sigma_t^2\fr{\partial^2 V_t(y,z)} {\partial 
z^2}\right)+{}\\
     {}+ A_t(y) \fr{\partial V_t(y,z)}{\partial y}+\left( a_t y+b_t z\right) 
\fr{\partial V_t(y,z)}{\partial z}+{}\\
{}+ S_t \left( s_t y- g_t z\right)^2 +G_t z^2-
 \fr{1}{4}\left( S_t h_t^2+H_t\right)^{-1}\times{}\\
 {}\times \left( c_t \fr{\partial V_t(y,z)} 
{\partial z}-2S_t\left( s_t y -g_t z\right) h_t \right)^2=0\,.
     \label{e13-bos}
     \end{multline}
     
     Возводя в~квадрат по\-след\-нее сла\-га\-емое в~(\ref{e13-bos}), перепишем 
его в~виде:
     \begin{multline}
     \fr{\partial V_t(y,z)}{\partial t} +\fr{1}{2}\left( \Sigma_t^2(y) \fr{\partial^2 
V_t(y,z)} {\partial y^2}+\sigma_t^2\fr{\partial^2 V_t(y,z)} {\partial 
z^2}\!\right)+{}\\
{}+A_t(y) \fr{\partial V_t(y,z)}{\partial y}
+ \left( 
\vphantom{\left( S_t h_t^2 +H_t\right)^{-1}}
a_t y+b_t z+{}\right.\\
\left.{}+\left( S_t h_t^2 +H_t\right)^{-1}
 c_t S_t \left( s_t y-g_t z\right) h_t
\right) 
     \fr{\partial V_t(y,z)}{\partial z}+{}\\
     {}+\left( S_t-\left( S_t h_t^2 +H_t\right)^{-1} S_t^2 h_t^2\right)\left( s_t y -
g_t z\right)^2+{}\\
     \!\!{}+
     G_t z^2 -\fr{1}{4}\left( S_t h_t^2+H_t\right)^{-1}\! c_t^2
     \left(\fr{\partial V_t(y,z)}{\partial z}\right)^{\!2}=0\,.\!\!
     \label{e14-bos}
     \end{multline}
     
     Рассматривая полученное уравнение, заметим, что его решение может 
быть пред\-став\-ле\-но в~виде:
   \begin{equation}
     V_t(y,z)= \alpha_t z^2+\beta_t(y) z +\gamma_t(y)\,,
     \label{e15-bos}
     \end{equation}
т.\,е.\ будем искать решение при дополнительном предположении 
о~квад\-ра\-тич\-ности функции Белл\-ма\-на по переменной~$z$, и~сведем, таким 
образом, поиск оптимального решения к~уравнениям относительно функций 
$\alpha_t$, $\beta_t(y)$ и~$\gamma_t(y)$. Отметим сразу, что явный вид 
функции~$\gamma_t(y)$ для реализации оптимального управ\-ле\-ния не 
требуется, однако далее будет предложен вариант вы\-чис\-ле\-ния и~этой 
функции, что пред\-став\-ля\-ет\-ся небесполезным, поскольку позволит выполнять 
расчет минимума целевого функционала. Источником для 
предложения~(\ref{e15-bos}) является уже упоминавшаяся аналогичная 
задача для случая дис\-крет\-но\-го времени~\cite{7-bos, 8-bos}. В~той задаче 
выражение для функции Беллмана получается формально без 
дополнительных усилий. При этом форма~(\ref{e15-bos}) обнаруживается 
как свойство оптимального решения. В~рассматриваемом случае 
непрерывного времени~(\ref{e15-bos}) постулируется, а~пра\-виль\-ность 
постулата под\-тверж\-да\-ет\-ся далее ре\-зуль\-ти\-ру\-ющи\-ми уравнениями 
для~$\alpha_t$, $\beta_t(y)$ и~$\gamma_t(y)$ Кроме того, данное 
предположение пред\-став\-ля\-ет\-ся вы\-те\-ка\-ющим из линейной структуры задачи 
в~отношении переменной~$z$, в~част\-ности, тем фактом, что такой вид 
функции Беллмана обеспечивает линейность оптимального 
управ\-ле\-ния~(\ref{e12-bos}) по~$z$.

     Граничное условие при выбранном предположении~(\ref{e15-bos}) 
принимает вид:

\noindent
     \begin{multline*}
     V_T(y,z)= S_T\left( s_T y- g_T z\right)^2+G_T z^2 ={}\\[-0.5pt]
     {}=\alpha_T z^2 
+\beta_T(y) z +\gamma_T(y)\,,
    \end{multline*}
т.\,е.

\noindent
\begin{align*}
\alpha_T&= S_T g_T^2 +G_T\,;\\[-0.5pt]
\beta_T(y)&=-2S_T s_T g_T y\,;\\[-0.5pt]
\gamma_T(y)&=S_T s_T^2 y^2\,.
%\label{e16-bos}
\end{align*}
          При этом само оптимальное управ\-ле\-ние, определенное 
выражением~(\ref{e12-bos}), оказывается управ\-ле\-ни\-ем с~обратной связью 
по~$y_t$ и~$z_t$:

\noindent
     \begin{multline}
     u_t^*=u_t^*(y,z) ={}\\[-0.5pt]
     {}=
     -\fr{1}{2}\left( S_t h_t^2 +H_t\right)^{-1}
     \left( c_t \left( 2\alpha_t z +\beta_t(y)\right) +{}\right.\\[-0.5pt]
    \left. {}+2S_t\left( s_t y-g_t z\right) 
h_t\right)\,.
     \label{e17-bos}
     \end{multline}
          Подставляем $V_t(y,z)\hm= \alpha_t z^2 \hm+ \beta_t(y) 
z\hm+\gamma_t(y)$ в~(\ref{e14-bos}):

\noindent
     \begin{multline*}
     \fr{\partial \alpha_t}{\partial t}\, z^2 +
     \fr{\partial \beta_t(y)}{\partial t}\,z +
     \fr{\partial \gamma_t(y)}{\partial t}+{}\\[-0.5pt]
     {}+\fr{1}{2}\left( \Sigma_t^2(y) \left( 
\fr{\partial^2\beta_t(y)}{\partial y^2}\,z +\fr{\partial^2 \gamma_t(y)}{\partial 
y^2}\right) +2\sigma_t^2\alpha_t\right)+{}\\[-0.5pt]
 {}+A_t(y)\left(\fr{\partial \beta_t(y)}{\partial y}\,z + \fr{\partial 
\gamma_t(y)}{\partial y}\right) +{}\\[-0.5pt]
\hspace*{-0.22987pt}{}+\left( a_t y+b_t z+\left( S_t h_t^2 +H_t\right)^{-1} c_t S_t \left( s_t y-
g_t z\right) h_t\right)\times{}
\end{multline*}

\noindent
\begin{multline*}
         {}\times \left( 2\alpha_t z+\beta_t(y)\right)+{}\\
     {}+\left( S_t-\left( S_t h_t^2 +H_t\right)^{-1} S_t^2 h_t^2\right)\left( s_t y-
g_t z\right)^2+{}\\
     {}+ G_t z^2 -\fr{1}{4}\left( S_t h_t^2 +H_t\right)^{-1} c_t^2 \left( 
2\alpha_t z+\beta_t(y)\right)^2=0\,.
     \end{multline*}
          Далее выделяем слагаемые при~$z^2$, $z$ и~$z^0$
          
          \noindent
     \begin{multline*}
     \fr{\partial \alpha_t}{\partial t}\, z^2 +\fr{\partial \beta_t(y)}{\partial t}\,z +
     \fr{\partial \gamma_t(y)}{\partial 
t}+\fr{1}{2}\,\Sigma_t^2(y)\fr{\partial^2\beta_t(y)}{\partial y^2}\,z+ {}\\
{}+
\fr{1}{2}\,\Sigma_t^2(y)\fr{\partial^2\gamma_t(y)}{\partial 
y^2}+\sigma_t^2\alpha_t+A_t(y)\fr{\partial \beta_t(y)}{\partial y}\,z +{}\\
{}+A_t(y) \fr{\partial 
\gamma_t(y)}{\partial y}+{}\\
{}+ 2\alpha_t \left( b_t -\left( S_t h_t^2+H_t\right)^{-1} c_t 
S_t h_t g_t \right)z^2+{}\\
     {}+
     \left( 2\alpha_t\left( \alpha_t+\left( S_t h_t^2+H_t\right)^{-1} c_t S_t h_t 
s_t\right)y +{}\right.\\
\left.{}+\beta_t(y) \left( b_t-\left( S_t h_t^2+H_t\right)^{-1} c_t S_t h_t 
g_t\right) \right) z+{}\\
     {}+\beta_t(y)\left( a_t +\left( S_t h_t^2+H_t\right)^{-1} c_t S_t h_t s_t\right) 
y+{}\\
{}+ \left( S_t -\left( S_t h_t^2+H_t\right)^{-1} S_t^2 h_t^2\right) g_t^2 z^2-{}\\
     {}- 2\left( S_t -\left( S_t h_t^2+H_t\right)^{-1} S_t^2 h_t^2\right) s_t g_t yz 
+{}\\
{}+
     \left( S_t-\left( S_t h_t^2+H_t\right)^{-1} S_t^2 h_t^2\right) s_t^2 y^2+{}\\
     {}+G_t z^2 -\left( S_t h_t^2 +H_t\right)^{-1} c_t^2 \alpha_t^2 z^2 -{}\\
     {}-\left( 
S_t h_t^2+H_t\right)^{-1} c_t^2 \alpha_t \beta_t(y) z-{}\\
{}-
\fr{1}{4}\left( S_t h_t^2+H_t\right)^{-1}  c_t^2 \beta_t^2(y)=0\,,
     \end{multline*}
группируем их и~получаем сле\-ду\-ющие уравнения:
\begin{itemize}
\item  для~$\alpha_t$:

\noindent
\begin{multline}
\fr{\partial\alpha_t}{\partial t}+2\alpha_t\left( b_t-\left( S_t h_t^2+H_t\right)^{-1} c_t 
S_t h_t g_t\right)+{}\\
{}+ \left( S_t- \left( S_t h_t^2+H_t\right)^{-1} S_t^2 h_t^2\right) 
g_t^2+G_t-{}\\
\hspace*{-8mm}{}-\left( S_t h_t^2+H_t\right)^{-1} c_t^2 \alpha_t^2 =0\,,\enskip \alpha_T=S_T 
g_t^2+G_T\,;\!\!
\label{e18-bos}
\end{multline}
\item для $\beta_t$:

\noindent
\begin{multline}
\fr{\partial\beta_t(y)}{\partial 
t}+\fr{1}{2}\,\Sigma_t^2(y)\fr{\partial^2\beta_t(y)}{\partial y^2} 
+A_t(y)\fr{\partial \beta_t(y)}{\partial y}+{}\\
{}+ 2\alpha_t\left( a_t +\left( S_t h_t^2+H_t\right)^{-1} c_t S_t h_t s_t\right) y+{}\\
{}+
\beta_t(y)\left( b_t -\left( S_t h_t^2 +H_t\right)^{-1} c_t S_t h_t g_t\right)-{}\\
{}-2\left( S_t-\left( S_t h_t^2+H_t\right)^{-1} S_t^2 h_t^2\right) s_t g_t y-{}
\\
{}-
\left( S_t h_t^2+H_t\right)^{-1} c_t^2 \alpha_t \beta_t(y)=0\,,\\
\beta_T(y)=-2S_T s_T g_T y\,;
\label{e19-bos}
\end{multline}
\item  для $\gamma_t$:
\begin{multline}
\hspace*{-0.8pt}\fr{\partial \gamma_t(y)}{\partial t}+\fr{1}{2}\,\Sigma_t^2(y)
\fr{\partial^2 \gamma_t(y)}{\partial y^2} +\sigma_t^2 \alpha_t +A_t(y)
\fr{\partial \gamma_t(y)}{\partial y}+{}\\
{}+ \beta_t(y)\left( a_t +\left( S_t h_t^2+H_t\right)^{-1} c_t S_t h_t s_t\right) y+{}\\
{}+
\left( S_t-\left( S_t h_t^2+H_t\right)^{-1} S_t^2 h_t^2\right)  s_t^2 y^2-{}\\
{}-\fr{1}{4}\left( S_t h_t^2+H_t\right)^{-1} c_t^2 \beta_t^2(y) =0\,,\\
\gamma_T(y)=S_T s_T^2 y^2\,.
\label{e20-bos}
\end{multline}
\end{itemize}
     
     Уравнение~(\ref{e18-bos}), легко заметить, является уравнением 
Риккати, которое в~силу сформулированного выше условия   
имеет единственное неотрицательное решение для всех $0\hm\leq t\hm\leq T$. 
Этот факт требует дополнительного комментария. Для получения 
уравнения~(\ref{e18-bos}) рас\-смот\-рим исходную задачу при дополнительных 
условиях $a_t\hm=0$ и~$s_t\hm=0$ для всех $0\hm\leq t\hm\leq T$. Нетрудно 
видеть, что эти условия рассматриваемую по\-ста\-нов\-ку сводят фактически 
к~классической ли\-ней\-но-квад\-ра\-тич\-ной задаче. Имеющуюся 
в~рассматриваемой формулировке чуть более общую форму целевой 
функции (принципиального значения это обобщение, конечно, не имеет) 
сведем к~классической еще одним предположением: $S_t\hm=0$ для всех 
$0\hm\leq t\hm\leq T$. Теперь уравнение для~$\alpha_t$ принимает хорошо 
известный вид:
     \begin{equation}
     \fr{\partial \alpha_t}{\partial t}+2\alpha_t b_t +G_t- H_t^{-1} c_t^2 
\alpha_t^2=0\,,\enskip \alpha_T=G_T\,.
     \label{e21-bos}
     \end{equation}

     В таком случае, как известно~\cite{10-bos}, существует единственное 
оптимальное управление~--- линейное с~обратной связью по выходу~$z_t$, 
с~коэффициентом усиления, опи\-сы\-ва\-емым уравнением  
Риккати~(\ref{e21-bos}). Именно этот результат дают  
уравнения~(\ref{e18-bos})--(\ref{e20-bos}) и~описываемая ими функция 
Беллмана~(\ref{e15-bos}), так как из $a_t\hm=0$ и~$s_t\hm=0$ немедленно 
следует, что $\beta_t(y)\hm=0$, откуда, в~свою очередь, с~учетом 
не\-за\-ви\-си\-мости решения от~$y_t$ следует, что $\gamma_t(y)\hm=\gamma_t$, 
т.\,е.\ не зависит от~$y$ и~задается уравнением: 
     $$
     \fr{\partial \gamma_t(y)}{\partial t} +\sigma^2_t \alpha_t=0\,,\enskip 
\gamma_T=0\,.
     $$ 
     Оптимальное управ\-ле\-ние при этом 
     $$
     u_t^*= -H_t^{-1} c_t \alpha_t z_t\,,
     $$
      т.\,е.\ все полностью совпадает с~известным классическим решением.
     
     С уравнениями~(\ref{e19-bos}) и~(\ref{e20-bos}) ситуация, естественно, 
обстоит сложнее. Это линейные уравнения второго порядка параболического 
типа, поскольку\linebreak
 $\Sigma_t^2(y)\hm>0$. Фактически отсутствуют 
конструктивные условия, гарантирующие существование их\linebreak
 решений 
(требовать, чтобы все фигурирующие в~уравнениях коэффициенты были 
представлены аналитическими функциями на всем пространстве значений, 
вряд ли целесообразно), поэтому далее будем предполагать, что данные 
уравнения имеют на рас\-смат\-ри\-ва\-емом интервале $0\hm\leq t\hm\leq T$ хотя 
бы одно ограниченное решение и~именно эти условия будем рас\-смат\-ри\-вать 
как достаточные условия существования оптимального решения 
рассматриваемой задачи.
     
     Таким образом, доказана следующая тео\-рема.
     
     \smallskip
     
     \noindent
     \textbf{Теорема.}\ \textit{Пусть для диффузионного 
процесса}~(\ref{e5-bos}) \textit{выполнены условия Ито, для 
     процесса}~(\ref{e6-bos})~--- \textit{ограничены коэффициенты, 
уравнения}~(\ref{e18-bos})--(\ref{e20-bos}) \textit{имеют ограниченные 
решения для $0\hm\leq t\hm\leq T$. Тогда минимум  
функционалу}~(\ref{e7-bos}) \textit{доставляет оптимальное 
управ\-ле\-ние}~(\ref{e17-bos}), \textit{где} $y\hm= y_t$; $z\hm=z_t$.
     
\section{Заключение}

     Рассмотренная задача оптимизации в~целом близка и~по модели, и~по 
критерию к~классической ли\-ней\-но-квад\-ра\-тич\-ной постановке. 
Принципиальным отличием является нелинейная модель для описания 
со\-сто\-яния динамической сис\-те\-мы, в~которой отсутствует управ\-ля\-ющее 
воздействие.\linebreak
 Такую модель наряду с~традиционной интер\-пре\-тацией  
<<со\-сто\-яние--вы\-ход>> мож\-но понимать как\linebreak модель неконтролируемого 
слож\-но\-го внешнего воздействия. Небольшое дополнительное отличие дает 
предложенная форма квад\-ра\-тич\-но\-го критерия, поз\-во\-ля\-ющая, в~част\-ности, 
ставить такие задачи, как отслеживание выходом или управ\-ле\-ни\-ем со\-сто\-яния 
сис\-те\-мы или ее выхода.
     
     Поскольку обсуждать возможности точного решения уравнений, 
определяющих оптимальное управ\-ле\-ние, не имеет смыс\-ла, наиболее 
актуальной далее является задача их приближенного чис\-лен\-но\-го решения 
и~анализа воз\-мож\-ности практической реализации. Этому посвящена вторая 
часть данной работы, пла\-ни\-ру\-емая к~выходу в~ближайшее время.

{\small\frenchspacing
 {%\baselineskip=10.8pt
 \addcontentsline{toc}{section}{References}
 \begin{thebibliography}{99}
\bibitem{1-bos}
\Au{Athans M.} Editorial on the LQG problem~// IEEE~T. Automat. Contr., 1971. Vol.~16. 
No.\,6. P.~528--552. doi: 10.1109/TAC.1971.1099845.
\bibitem{2-bos}
\Au{Wu Z.} Forward-backward stochastic differential equations, linear quadratic stochastic 
optimal control and nonzero sum differential games~// J.~Syst. Sci. Complex., 2005. Vol.~18. 
No.\,2. P.~179--192.
\bibitem{3-bos}
\Au{Chen B.\,S., Zhang~W.} Stochastic H2/H1 control with state-dependent noise~// IEEE 
T.~Automat. Contr., 2004. Vol.~49. No.\,1. P.~45--56. doi: 10.1109/TAC.2003.821400.
\bibitem{4-bos}
\Au{Bohacek S.} A~stochastic model of TCP and fair video transmission~// IEEE 
INFOCOM, 2003. Vol.~2. P.~1134--1144. doi: 10.1109/INFCOM.2003.1208950.
\bibitem{5-bos}
\Au{Домбровский В.\,В., Объедко~Т.\,Ю.} Управление с~прогнозированием системами 
с~марковскими скачками при ограничениях и~применение к~оптимизации 
инвестиционного портфеля~// Автомат. телемех., 2011. №\,5. С.~96--112. doi: 
10.1134/S0005117911050079.
\bibitem{6-bos}
\Au{Баландин Д.\,В., Коган~М.\,М.} Оптимальное линейно-квад\-ра\-тич\-ное управление: от 
матричных уравнений к~линейным матричным неравенствам~// Автомат. телемех., 2011. 
№\,11. С.~60--69. doi: 10.1134/ S0005117911110038.
\bibitem{7-bos}
\Au{Босов А.\,В.} Обобщенная задача распределения ресурсов программной системы~// 
Информатика и~её применения, 2014. Т.~8. Вып.~2. С.~39--47. doi: 
10.14357/19922264140204.
\bibitem{8-bos}
\Au{Босов А.\,В.} Управление линейным выходом дискретной стохастической системы по 
квадратичному критерию~// Изв. РАН. Теория и~системы управления, 2016. №\,3.  
С.~19--35. doi: 10.1134/S1064230716030060.
\bibitem{9-bos}
\Au{Флеминг У., Ришел~Р.} Оптимальное управление детерминированными 
и~стохастическими системами~/ Пер. с~англ.~--- М.: Мир, 1978. 316~с. 
(\Au{Fleming~W.\,H., Rishel~R.\,W.} Deterministic and stochastic optimal control.~--- New 
York, NY, USA: Springer-Verlag, 1975. 222~p.)
\bibitem{10-bos}
\Au{Девис М.\,Х.\,А.} Линейное оценивание и~стохастическое управление~/ Пер. с~англ.~--- 
М.: Наука, 1984. 206~с. (\Au{Davis~M.\,H.\,A.} Linear estimation and stochastic control.~--- 
London: Chapman and Hall, 1977. 224~p.)

 \end{thebibliography}

 }
 }

\end{multicols}

\vspace*{-6pt}

\hfill{\small\textit{Поступила в~редакцию 30.03.18}}

\vspace*{4pt}

%\newpage

%\vspace*{-24pt}

\hrule

\vspace*{2pt}

\hrule

\vspace*{-2pt}


\def\tit{STOCHASTIC DIFFERENTIAL SYSTEM OUTPUT CONTROL 
BY~THE~QUADRATIC CRITERION.~I.~DYNAMIC\\ PROGRAMMING 
OPTIMAL SOLUTION}


\def\titkol{Stochastic differential system output control 
by~the~quadratic criterion. I.~Dynamic programming 
optimal solution}

\def\aut{A.\,V.~Bosov and~A.\,I.~Stefanovich}

\def\autkol{A.\,V.~Bosov and~A.\,I.~Stefanovich}

\titel{\tit}{\aut}{\autkol}{\titkol}

\vspace*{-11pt}


\noindent
Institute of Informatics Problems, Federal Research Center ``Computer Science 
and Control'' of the Russian Academy of Sciences, 44-2~Vavilov Str., Moscow 
119333, Russian Federation


\def\leftfootline{\small{\textbf{\thepage}
\hfill INFORMATIKA I EE PRIMENENIYA~--- INFORMATICS AND
APPLICATIONS\ \ \ 2018\ \ \ volume~12\ \ \ issue\ 3}
}%
 \def\rightfootline{\small{INFORMATIKA I EE PRIMENENIYA~---
INFORMATICS AND APPLICATIONS\ \ \ 2018\ \ \ volume~12\ \ \ issue\ 3
\hfill \textbf{\thepage}}}

\vspace*{3pt}



\Abste{The problem of optimal control for the Ito diffusion 
process and a~controlled linear output is solved. The considered 
statement is close to the classical linear-quadratic Gaussian 
control  (LQG control) problem. Differences consist in the fact 
that the state is described by the nonlinear differential Ito equation  $dy_y = A_t(y_t) 
\,dt+\Sigma_t(y_t)\,dv_t$ and does not depend on the control~$u_t$, 
optimization subject is controlled linear output 
 $dz_t=a_ty_t\,dt +b_tz_t\,dt +c_t u_t\,dt +\sigma_t \,dw_t$. 
Additional generalizations are included in the quadratic 
quality criterion for the purpose of statement such problems 
as state tracking by output or a linear combination of state 
and output tracking by control. The method of dynamic programming 
is used for the solution. 
The assumption about Bellman function in the form  $V_t(y,z)= \alpha_t 
z^2+\beta_t(y) z+\gamma_t(y)$ allows one to find it. 
Three differential equations for the coefficients $\alpha_t$,  $\beta_t(y)$,
and $\gamma_t(y)$ give the solution. 
These equations constitute the optimal solution of the problem under consideration.}

\KWE{stochastic differential equation; optimal control; dynamic programming; 
Bellman function; Riccati equation; linear differential equations of parabolic type}


\DOI{10.14357/19922264180314}

\vspace*{-12pt}

\Ack
\noindent
This work was partially supported by the Russian Science Foundation (grant  
16-07-00677).



%\vspace*{6pt}

  \begin{multicols}{2}

\renewcommand{\bibname}{\protect\rmfamily References}
%\renewcommand{\bibname}{\large\protect\rm References}

{\small\frenchspacing
 {%\baselineskip=10.8pt
 \addcontentsline{toc}{section}{References}
 \begin{thebibliography}{99}
\bibitem{1-bos-1}
\Aue{Athans, M.} 1971. Editorial on the LQG problem. \textit{IEEE~T. 
Automat. Contr.} 16(6):528--552. doi: 10.1109/ TAC.1971.1099845.
\bibitem{2-bos-1}
\Aue{Wu, Z.} 2005. Forward-backward stochastic differential equations, linear 
quadratic stochastic optimal control and\linebreak\vspace*{-12pt}

\columnbreak

\noindent
 nonzero sum differential games. 
\textit{J.~Syst. Sci. Complex.} 18(2):179--192.
\bibitem{3-bos-1}
\Aue{Chen, B.\,S. and W.~Zhang.} 2004. Stochastic H2/H1 control with  
state-dependent noise. \textit{IEEE~T. Automat. Contr.} 49(1):45--56.
doi: 10.1109/TAC.2003.821400.
\bibitem{4-bos-1}
\Aue{Bohacek, S.} 2003. A~stochastic model of TCP and fair video 
transmission. \textit{IEEE INFOCOM}. 2:1134--1144.
doi: 10.1109/INFCOM.2003.1208950.
\bibitem{5-bos-1}
\Aue{Dombrovskii, V.\,V., and T.\,Yu.~Ob''edko.} 2011. Predictive control of 
systems with Markovian jumps under constraints and its application to the 
investment portfolio optimization. \textit{Automat. Rem. Contr.}  
72(5):989--1003.
\bibitem{6-bos-1}
\Aue{Balandin, D.\,V., and M.\,M.~Kogan.} 2011. Optimal linear-quadratic 
control: From matrix equations to linear matrix inequalities. \textit{Automat. 
Rem. Contr.} 72(11):2276--2284.
\bibitem{7-bos-1}
\Aue{Bosov, A.\,V.} 2014. Obobshchennaya zadacha raspredeleniya resursov 
programmnoy sistemy [The generalized problem of software system resources 
distribution]. \textit{Informatika i~ee Primeneniya~--- Inform. Appl.}  
8(2):39--47. doi: 
10.14357/19922264140204.
\bibitem{8-bos-1}
\Aue{Bosov, A.\,V.} 2016. Discrete stochastic system linear output control 
with respect to a quadratic criterion. \textit{J.~Comput. Syst. Sc. 
Int.} 55(3):349--364.
\bibitem{9-bos-1}
\Aue{Fleming, W.\,H., and R.\,W.~Rishel.} 1975. \textit{Deterministic and 
stochastic optimal control.} New York, NY: Springer-Verlag. 222~p.
\bibitem{10-bos-1}
\Aue{Davis, M.\,H.\,A.} 1977. \textit{Linear estimation and stochastic 
control.} London: Chapman and Hall. 224~p.
\end{thebibliography}

 }
 }

\end{multicols}

\vspace*{-6pt}

\hfill{\small\textit{Received March 30, 2018}}

%\pagebreak

%\vspace*{-18pt}
     
     \Contr
     
       \noindent
       \textbf{Bosov Alexey V.} (b.\ 1969)~--- Doctor of Science in technology, 
principal scientist, Institute of Informatics Problems, Federal Research 
Center ``Computer Science and Control'' of the Russian Academy of Sciences, 
44-2~Vavilov Str., Moscow 119333, Russian Federation; 
\mbox{AVBosov@ipiran.ru}
       
       \vspace*{3pt}
       
       \noindent
       \textbf{Stefanovich Alexey I.} (b.\ 1983)~--- principal specialist, 
Institute of Informatics Problems, Federal Research Center ``Computer Science 
and Control'' of the Russian Academy of Sciences, 44-2~Vavilov Str., Moscow 
119333, Russian Federation; \mbox{AStefanovich@frccsc.ru}
\label{end\stat}

\renewcommand{\bibname}{\protect\rm Литература}       

        %5
\def\stat{zorin}

\def\tit{К ОПРЕДЕЛЕНИЮ ПЕРИОДА ЗАНЯТОСТИ ПРИ~НЕЛОКАЛЬНОМ~ОПИСАНИИ ПОТОКОВ}

\def\titkol{К определению периода занятости при нелокальном описании потоков}

\def\aut{А.\,В.~Зорин$^1$}

\def\autkol{А.\,В.~Зорин}

\titel{\tit}{\aut}{\autkol}{\titkol}

\index{Зорин А.\,В.}
\index{Zorine A.\,V.}


%{\renewcommand{\thefootnote}{\fnsymbol{footnote}} \footnotetext[1]
%{Работа 
%выполнена при поддержке Программы развития МГУ, проект №\,23-Ш03-03. При анализе 
%данных использовалась инфраструктура Центра коллективного пользования 
%<<Высокопроизводительные вычисления и~большие данные>> 
%(ЦКП <<Информатика>>) ФИЦ ИУ РАН (г.~Москва)}}


\renewcommand{\thefootnote}{\arabic{footnote}}
\footnotetext[1]{Национальный исследовательский Нижегородский государственный 
университет им.~Н.~И.~Лобачевского,\\  \mbox{andrei.zorine@itmm.unn.ru}}

\vspace*{-12pt}




\Abst{При вероятностном моделировании и~анализе сложных управляющих систем
  массового обслуживания нескольких конфликтных входных потоков в~ряде работ
  успешно применялся подход, одна из черт которого~--- нелокальное
  описание различных составных блоков сис\-те\-мы. При таком описании часто теряется
  информация об истинных моментах прихода и~ухода требований в~сис\-те\-ме. Это
  приводит к~сложностям при определении периода занятости~--- одного из
  классических показателей качества функционирования.  В~работе предлагается
  определение периода занятости управляющей системы массового обслуживания на
  основе выделения тех моментов наблюдения над системой, в~которые очереди
  достигают нулевого значения. На примере циклического алгоритма обслуживания с~фиксированным ритмом переключения и~с~использованием мартингальной техники
  находятся эффективные вычислительные формулы для математического ожидания
  периода занятости отдельной очереди.}

\KW{управляющая система массового обслуживания; нелокальное описание блоков;
  неординарные потоки Пуассона; циклический алгоритм обслуживания; период
  занятости; многомерная счетная цепь Маркова; мартингал; обобщенная теорема
  Руше; интерполяционный полином Лагранжа}
  
  \DOI{10.14357/19922264240306}{YKSIBJ}
  
%\vspace*{-6pt}


\vskip 10pt plus 9pt minus 6pt

\thispagestyle{headings}

\begin{multicols}{2}

\label{st\stat}


\section{Введение}

В настоящее время в~теории массового обслуживания рассматривается большое число
показателей качества обслуживания, таких как распределение времени ожидания
требования, распределения размеров очередей, распределение интервала занятости,
вероятность потери требования в~системе с~потерями и~т.\,д.~\cite{Ivchenko:Kashtanov:Kovalenko}. В~то время как одни характеристики
можно без труда относить и~к моделям с~непрерывным временем, и~к моделям с~дискретным временем, корректное определение других характеристик может встречать
известные затруднения. 

Поясним сказанное на примере периода занятости. Период
за\-ня\-тости для некоторого прибора (если в~системе несколько обслуживающих
приборов) определяется как промежуток времени, в~начале которого прибор выходит
из режима ожидания и~начинает обслуживание требования, а окончание этого
промежутка совпадает с~первым из тех моментов окончания обслуживания, когда
ожидающих обслуживания на этом приборе требований больше нет и~прибор вынужден
перейти в~режим ожидания. Для систем в~дискретном времени~[2--4] вся временн$\acute{\mbox{а}}$я
ось разбивается на временн$\acute{\mbox{ы}}$е слоты фиксированного размера $h\hm>0$ и~истинный
момент поступления требования не наблюдается. Считается, что требование
поступает в~конце слота, а~его обслуживание может начаться только в~сле\-ду\-ющем
временн$\acute{\mbox{о}}$м слоте. Таким образом, период за\-ня\-тости в~модели с~дискретным временем
будет целым кратным величины~$h$ и~будет содержать <<ошибку округления>> порядка~$h$.  

В~монографиях~\cite{Alfa, Bruneel:Kim} период занятости исследован для
прос\-тей\-шей однолинейной сис\-те\-мы и~сис\-те\-мы с~конечным чис\-лом параллельных
приборов с~гео\-мет\-ри\-че\-ским входящим потоком; в~монографии~\cite{Pechinkin:Razumchik} 
период за\-ня\-тости изуча\-ет\-ся для однолинейной системы с~произвольным (це\-ло\-чис\-лен\-ным) временем обслуживания. Достаточное пред\-став\-ле\-ние об
истории и~современном со\-сто\-янии этого круга вопросов мож\-но со\-ста\-вить также по
работам~[5--8].


Для построения адекватных математических моделей сложных управляющих систем
обслуживания приходится применять так называемое нелокальное задание разных
составных блоков \mbox{системы}, таких как входные потоки, потоки насыщения и~некоторые
другие. Напомним, что потоки насыщения определяются~\cite{Fedotkin:1975} как
виртуальные выходные потоки системы при наличии достаточно %\linebreak 
большого запаса
требований в~очередях и~максимальном использовании возможностей обслуживающего
устройства; таким образом, они пред\-став\-ля\-ют альтернативу указания закона
\mbox{обслуживания} каждого отдельного требования. Понятие нелокального описания потока
было введено в~работах~\cite{Fedotkin:1978,Fedotkin:1981}. Приведем здесь
со\-от\-вет\-ст\-ву\-ющее \mbox{определение}: пусть $\{\tau^{\textrm{н}}_i;i=0,1,\ldots\}$~---
последовательность моментов наблюдения на оси~$Ot$, $(M, \mathcal M)$~---
измеримое пространство \textit{меток}, $\eta^{\textrm{н}}_i \hm\in \{0,1,\ldots\}$ и~$\nu_i\hm\in M$ суть число требований и~метка требований потока~$\Pi$, поступивших
на промежутке $(\tau^{\textrm{н}}_i, \tau^{\textrm{н}}_{i+1}]$. \textit{Неполное
  (нелокальное) описание} потока~$\Pi$ неоднородных требований есть векторная
случайная последовательность
%\begin{equation}
 % \label{AZ:eq:flowdescr-incomplete}
  $\left\{(\tau^{\textrm{н}}_i, \eta^{\textrm{н}}_i, \nu_i); i=0, 1, \ldots\right\}.$
%\end{equation}
Существенное отличие использования нелокального описания от упомянутых ранее
моделей в~дискретном времени заключается в~том, что длительности
$(\tau^{\mathrm{н}}_i\hm- \tau^{\mathrm{н}}_{i-1})$ промежутков наблюдения могут
быть случайными и~достаточно большими, а~их законы распределения могут
существенно меняться в~зависимости от состояния системы
обслуживания. В~частности, оказывается возможным применить понятие нелокального
описания для задания свойств процесса обслуживания требований с~помощью потока
насыщения вместо традиционного указания закона распределения длительности
обслуживания каждого отдельного требования.

Использование нелокального описания для входных потоков и~для потоков насыщения
лишает исследователя информации о точных моментах начала и~окончания периода
занятости в~классическом смысле. Поэтому в~данной работе предлагается связать
периоды занятости с~некоторыми из моментов $\tau^{\textrm{н}}_i$, $i\hm=0, 1,
\ldots$, когда одна или несколько очередей опустошаются. На этом пути удается
вычислить среднее время периода занятости, например для системы обслуживания
нескольких конфликтных потоков по алгоритму с~циклическим ритмом переключения и~фиксированными длительностями фаз.
Этот алгоритм часто применяется, например, в~задачах управления транспортными потоками на регулируемых перекрестках.

\vspace*{-6pt}

\section{Общая постановка задачи}

\vspace*{-3pt}

Рассмотрим следующую систему массового обслуживания.  Имеются $m\hm<\infty$ входных
потоков~$\Pi_j$, $j\hm=\overline{1, m}$. Предположим, что поток~$\Pi_j$~--- ста\-ционарный, неординарный, без последействия.\linebreak  Требования потока
$\Pi_j$ помещаются в~накопитель неограниченного объема.  Обслуживающее~устройство имеет $2m$~состояний: 
$\Gamma^{(1)}, \Gamma^{(2)}, \ldots,
\Gamma^{(2m)}$. В~состоянии $\Gamma^{(2j-1)}$ обслуживаются только требования
потоков~$\Pi_j$, $j\hm=\overline{1, m}$. В~состояниях вида $\Gamma^{(2j)}$
требования не обслуживаются. Состояния меняются по циклическому алгоритму:
$\cdots\to\Gamma^{(1)}\hm\to \Gamma^{(2)} \to \cdots \to \Gamma^{(2m)}\hm\to
\Gamma^{(1)} \to \cdots$. Длительность состояния~$\Gamma^{(r)}$ неслучайна и~равна $T_r\hm>0$.  Закон обслуживания потока задается свойствами потока насыщения:
максимально возможное число $\ell_j\hm\in\{0, 1, \ldots\}$ обслуженных требований
из очереди~$O_j$ за время~$T_{2j-1}$ в~состоянии~$\Gamma^{(2j-1)}$
обслуживающего устройства и~число~0 в~любом другом состоянии~$\Gamma^{(r)}$,
$r\hm\neq 2j-1$.

Будем наблюдать за системой в~моменты переключения сигнала светофора. Обозначим
эти моменты через~$\tau_i$, $i\hm=0, 1, \ldots$ (здесь и~далее для краткости не
будем писать верхнюю букву~<<н>> в~обозначении $\tau^{\mathrm{н}}_i$). Пусть
$\tau_0\hm=0$. Обозначим через~$\Gamma_0$ состояние светофора в~момент~$\tau_0$, а
через~$\Gamma_i$~--- состояние светофора на промежутке $(\tau_{i-1}, \tau_i]$.
Далее, пусть~$\kappa_{j,i}$~--- число требований в~очереди~$O_j$ в~момент~$\tau_i$, 
$\eta_{j,i}$~--- число требований потока~$\Pi_j$, поступивших за
промежуток $(\tau_i, \tau_{i+1}]$, $\xi_{j,i}$~--- число требований по потоку
насыщения на промежутке $(\tau_i, \tau_{i+1}]$. Введем векторы
$\eta_i\hm=(\eta_{1,i}, \eta_{2,i}, \ldots, \eta_{m,i})$
и~$\xi_i\hm=(\xi_{1,i}, \xi_{2,i}, \ldots, \xi_{m,i})$, $i\hm=0, 1, \ldots$

Динамика очередей описывается соотноше\-ни\-ями:

\vspace*{-6pt}

\noindent
\begin{multline*}
\kappa_{j,i+1}= \max\left\{0, \kappa_{j,i}+\eta_{j,i}-\xi_{j,i}\right\}, \\ 
j=\overline{1,m};\enskip i=0,1,\ldots
\end{multline*}

\vspace*{-3pt}

\noindent
Описание входных потоков и~потоков насыщения зададим с~помощью указания свойств
маркированного точечного процесса
$\left\{(\tau_i, \kappa_i, \nu_i); i=0, 1, \ldots\right\},
$
где $\nu_i=\Gamma_i$~--- метка требований, поступающих на промежутке
$(\tau_i, \tau_{i+1}]$. Пусть $r \hm\oplus 1$ означает $r+1$\linebreak для $r\hm<2m$ и~принимает
значение~1 для $r\hm=2m$. Пусть далее~$w_j$ и~$y_j$, $j\hm=\overline{1, m}$,~---
произ\-вольные целые неотрицательные числа. Тогда\linebreak при условии
$\nu_{i}\hm=\Gamma^{(r)}$ вероятность события
%\begin{equation}
 % \label{AZ:eq:eta-xi}
$\bigcap\nolimits_{j=1}^m \{ \eta_{i,j}=w_j, \xi_{i,j}=y_j\}
$ %\end{equation}
независимо от $\Gamma_0$, $\kappa_0$, $\kappa_1$, \ldots, $\kappa_{i}$ равна
$
\prod\nolimits_{j=1}^m \varphi_j(w_j; T_{r\oplus1}),
$
если $y_j=0$ при $r\oplus1\hm\neq 2j-1$ и~$y_j\hm=\ell_j$ при $r\hm \oplus1\hm=2j-1$, и~эта
вероятность равна нулю в~остальных случаях. Здесь распределения вероятностей
$\varphi_j(b;u)$, $b\hm=0, 1, \ldots$ ($u\hm>0$ рассматривается как параметр), могут
браться из разложения производящей функции для простого группового
(неординарного) потока~\cite{Ivchenko:Kashtanov:Kovalenko} следующим
образом. Введем производящие функции

\vspace*{-6pt}

\noindent
\begin{multline*}
\!f_j(z)=\sum\limits_{b=1}^\infty g_j(b) z^b;
\enskip
q_j(z;u)= \exp\{ \lambda_j u (
f_j(z) -1) \},\\  |z|\leqslant1,\quad j=\overline{1,  m}\,.
\end{multline*}
Здесь $\lambda_j>0$ задает интенсивность поступления групп машин по потоку~$\Pi_j$; $g_j(b)$ есть вероятность того, что группа по потоку~$\Pi_j$ состоит
из~$b$~автомашин. Пусть $f_j(z)\hm\neq1$ при $z\hm\neq1$, $|z|\hm=1$. Тогда
\begin{equation*}
\sum\limits_{b=0}^\infty z^b \varphi_j(b; u) = q_j(z;u), \enskip |z|\leqslant1\,.
%\label{AZ:eq:phi-def}
\end{equation*}

Введем обозначения:
$$
\bar \lambda_j =  \lambda_j f'_j(1)=\lambda_j \sum_{b=1}^\infty b g_j(b), \enskip
j=\overline{1,m}\,.
$$
В сделанных предположениях многомерная последовательность
$\{(\Gamma_i, \kappa_i);\ i\hm=0, 1, \ldots\}$ при заданном распределении случайного
элемента $(\Gamma_0, \kappa_0)$ будет однородной цепью Маркова, а необходимое и~достаточное условие существования стационарного распределения будет иметь
следующий вид:
\begin{equation}
  \max\left\{ \fr{{\bar\lambda_1}T}{\ell_1}\,,\;
  \frac{{\bar\lambda_2}T}{\ell_2}\,,\; \ldots,\;
  \frac{{\bar\lambda_m}T}{\ell_m}\right\}<1\,,
  \label{AZ:eq:stat}
\end{equation}
где $ T=T_1+T_2+\cdots+T_{2m}$. Содержательно условие~\eqref{AZ:eq:stat}
означает, что по каждому потоку среднее\linebreak чис\-ло поступающих требований за полный
цикл меньше соответствующего потока насыщения. Всюду далее будем предполагать
условие стационарности~\eqref{AZ:eq:stat} выполненным. Кроме того, каждая из
случайных последовательностей $\{(\Gamma_i, \kappa_{j,i});\ i\hm= 0,1, \ldots\}$, $j\hm=\overline{1,  m}$, 
также представляет собой однородную цепь Маркова, при выполнении
условия~\eqref{AZ:eq:stat}~--- положительно возвратную.

\vspace*{-6pt}

\section{Период занятости}

\vspace*{-3pt}

Перейдем теперь к~определению понятия периода занятости. Введем поток
сиг\-ма-алгебр
$$
\mathfrak{F}_{i}\hm= \sigma\{\Gamma_{0}, \kappa_{1,0}, \eta_{1,0}, \eta_{1,1},
\ldots, \eta_{1,i-1}\}.
$$
 Пусть~$\mathfrak s$~--- момент остановки (относительно
потока $\{\mathfrak F_i; i\hm=0, 1, \ldots\}$) такой, что
$\kappa_{j,\mathfrak s}\hm=0$. Введем случайную величину~$\nu(\mathfrak s)$
следующим образом:

\vspace*{-6pt}

\noindent
\begin{multline*}
\nu(\mathfrak s)={}\\
{}=
\begin{cases}
  \infty,& \!\hspace*{-60pt} \mbox{если\ }\kappa_{j,i}>0\ \mbox{для всех}\  i\geqslant \mathfrak s+1;\\
  \min\{i\geqslant 1\colon \kappa_{j,i+\mathfrak s}=0\}& \!\mbox{в\ противном\ случае}.
\end{cases}
\end{multline*}
Анализ классического понятия периода занятости приводит к~двум возможным
определениями. Согласно первому определению назовем \textit{периодом занятости}
для очереди~$O_j$ промежуток
$(\tau_{\mathfrak s}, \tau_{\mathfrak s+\nu(\mathfrak s)}]$ при условии
$\kappa_{j,\mathfrak s}\hm=0$. Здесь не исключен случай, когда не поступает ни
одного требования за следующий такт работы обслуживающего устройства и~никакого
обслуживания фактически не происходит.  Поэтому можно рассмотреть период
занятости, начинающийся с~$x\hm=1, 2, \ldots$ требований в~очереди~$O_j$,
т.\,е.\ распределение случайного промежутка
$(\tau_{\mathfrak s}, \tau_{\mathfrak s+\nu(\mathfrak s)}]$ можно рассматривать
при условии $\kappa_{j,\mathfrak s}\hm=x$, $x\hm>0$.
На событии
$\{\omega\colon \Gamma_{\mathfrak s}\hm=\Gamma^{(r)}\}$ длину этого промежутка
можно выразить формулой
$$
T_{r\oplus1}+T_{r\oplus2}+\cdots+T_{r\oplus \nu(\mathfrak s)}.
$$
Если при заданных свойствах входных потоков, потоков насыщения и~алгоритма
управления все периоды занятости для данной очереди статистически однородны,
можно, не уменьшая общ\-ности, \mbox{изучать} условное распределение величины~$\nu(\mathfrak s)$ или величины~$\tau_{\nu(\mathfrak s)}$, положив
$\mathfrak{s}\hm=0$ и~рассматривая условные распределения при условии
$\kappa_{j,0}\hm=0$. Цель данного исследования~--- вычисление условных \mbox{средних}

\vspace*{-6pt}

\noindent
\begin{multline*}
\mathbb E\left(T_{r\oplus1}+T_{r\oplus2}+\cdots+T_{r\oplus \nu(\mathfrak s)} \Big\vert \right.\\
\left.\left\{\kappa_{j,0}=x, \Gamma_0=\Gamma^{(r)}\right\}\right).
\end{multline*}

\vspace*{-3pt}

Применяемый ниже мартингальный метод восходит в~работам~\cite{Bacelli-1,Bacelli-2}.

Имеют место следующие утверждения. Не уменьшая общности, будем везде далее
полагать $j\hm=1$. Ниже~$I(\cdot)$ означает индикатор события, указанного в~скобках.


\smallskip

\noindent
\textbf{Теорема~1.}
\textit{ При $|z|<1$ последовательность}
  \begin{equation*}
    M_{i}(z)=
    \begin{cases}
      z^{\kappa_{1,0}}, & \hspace*{-33pt}i=0;\\
      z^{\kappa_{1,i}+ \sum\nolimits_{k=0}^{i-1}
        \min\{\kappa_{1,k}+\eta_{1,k},\;\xi_{1,k}\}}\times{}&\\
        {}\times 
     \displaystyle  \prod\limits_{k=0}^{i-1} \displaystyle \sum\limits_{r=1}^{2m}
      \fr{I(\Gamma_{k}=\Gamma^{(r)})}
      {\exp\{\lambda_{1}T_{r\oplus1}(f_{1}(z)-1)\}},&\\
      & \hspace*{-33pt}i=1,2,\ldots,
    \end{cases}
  \end{equation*}
  \textit{представляет собой мартингал относительно потока}
  $\{\mathfrak{F}_{i};i=0,1,\ldots\}$.

\smallskip

\noindent
Д\,о\,к\,а\,з\,а\,т\,е\,л\,ь\,с\,т\,в\,о\ \ состоит в~проверке определения мартингала и~здесь не приводится.

\smallskip

Пусть, по определению, $\mathfrak t(\mathfrak s)= \mathfrak s + \nu(\mathfrak{s})$.

\smallskip

\noindent
\textbf{Теорема~2.}
\textit{Пусть $\mathfrak s$~---  момент остановки относительно потока
  $\{\mathfrak{F}_{i};\ i\hm=0,1,\ldots\}$. Тогда}
  
  \vspace*{-6pt}
  
  \noindent
  \begin{multline}
    \sum\limits_{r=1}^{2m}\mathbb{E}\Bigl(
    I\left(\Gamma_{\mathfrak   s}=\Gamma^{(r)}\right)\times{}\\
    {}\times 
    z^{\xi_{1,\mathfrak s}+\xi_{1,\mathfrak  s+1}+\cdots +\xi_{1,\mathfrak {t}(\mathfrak s)-2}+
      \kappa_{1,\mathfrak t(\mathfrak s)-1}+\eta_{1,\mathfrak t(\mathfrak s)-1}}
    \times{}
    \\ 
    {}\times
    \exp\{\lambda_{1}(T_{r\oplus1}+T_{r\oplus2}+\cdots+
    T_{r\oplus\nu(\mathfrak s)}) (1-f_{1}(z))\}\Big|
    \mathfrak{F}_{\mathfrak s}\!\Bigr)\!={}\\
    {}=
    z^{\kappa_{1,\mathfrak  s}}.
    \label{AZ:eq:main-mart}
  \end{multline}


\noindent
Д\,о\,к\,а\,з\,а\,т\,е\,л\,ь\,с\,т\,в\,о\,.\ \
  Поскольку для момента остановки $\mathfrak s$ случайная величина
  $\mathfrak t(\mathfrak s)$ конечна с~вероятностью единица (как следствие
  положительной возвратности), то из теоремы Дуба об опциональной остановке
  должно следовать равенство
  $M_{\mathfrak s}(z) \hm= \mathbb E(M_{\mathfrak t(\mathfrak s)} \mid \mathfrak
  F_{\mathfrak s})$. Поскольку
  
  \noindent
\begin{multline*}
  \mathbb {E}\left(M_{\mathfrak t(\mathfrak s)} \mid \mathfrak F_{\mathfrak{s}}\right) =
  z^{\sum\nolimits_{k=0}^{\mathfrak{s}-
1}\min\{\kappa_{1,k}+\eta_{1,k},\;\xi_{1,k}\}}\times{}\\
{}\times
  \prod\limits_{k=0}^{\mathfrak s-1}\sum\limits_{r=1}^{2m}
  \fr{I(\Gamma_{k}=\Gamma^{(r)})}{\exp\{\lambda_{1}T_{r+1}(f_{1}(z)-
1)\}}\times{}
\end{multline*}

\noindent
\begin{multline*}
{}  \times\mathbb{E}\biggl(z^{\sum\nolimits_{k=\mathfrak s}^{\mathfrak t(\mathfrak
      s)-1}\min\{\kappa_{1,k}+\eta_{1,k},\;\xi_{1,k}\}}\times{}\\
      {}\times \prod\limits_{k=\mathfrak
    s}^{\mathfrak t(\mathfrak s)-1}\sum\limits_{r=1}^{2m}
  \fr{I(\Gamma_{k}=\Gamma^{(r)})}%
  {\exp\{\lambda_{1}T_{r\oplus1}(f_{1}(z)-1)\}}\,\bigg|\,\mathfrak{F}_{\mathfrak 
s}\biggr),
\end{multline*}
то
\begin{multline*}
  z^{\kappa_{1,\mathfrak s}}=
  \mathbb{E}\biggl(z^{\sum\nolimits_{k=\mathfrak s}^{\mathfrak t(\mathfrak s)-1}
    \min\{\kappa_{1,k}+\eta_{1,k},\;\xi_{k}\}}\times{}\\
    {}\times \prod\limits_{k=\mathfrak s}^{\mathfrak
    t(\mathfrak s)-1}
  \sum\limits_{r=1}^{2m}\fr{I(\Gamma_{k}=\Gamma^{(r)})}%
  {\exp\{\lambda_{1}T_{r\oplus1}(f_{1}(z)-1)\}}\bigg|\mathfrak{F}_{\mathfrak 
s}\bigg).
\end{multline*}
Во-первых, равенство $\mathfrak t(\mathfrak s)\hm=i$ влечет $\kappa_{1,i}\hm=0$,
откуда $\kappa_{1,i-1}\hm+\eta_{1,i-1}\hm\leqslant\xi_{1,i}$. Для всех~$k$,
$\mathfrak {s} \hm\leqslant k\hm<i$, имеем
$\min\{\kappa_{1,k}\hm+\eta_{1,k},\xi_{1,k}\}=\xi_{1,k}$. Значит,
\begin{multline*}
  \sum\limits_{k=\mathfrak s}^{\mathfrak t(\mathfrak s)-1}
  \min\{\kappa_{1,k}+\eta_{1,k},\;\xi_{k}\}={}\\
  {}= \xi_{1,\mathfrak
    s}+\xi_{1,\mathfrak s+1}+\cdots+\xi_{1,\mathfrak t(\mathfrak s)-2}+
  \kappa_{1,\mathfrak t(\mathfrak s)-1}+\eta_{1,\mathfrak t(\mathfrak s)-1}.
\end{multline*}
Чтобы окончательно установить равенство~\eqref{AZ:eq:main-mart}, остается
заметить, что в~силу циклического обслуживания значение~$\Gamma_{\sigma}$
однозначно определяет все последующие состояния среды, поэтому можно избавиться
от внутренних сумм по~$r$ в~пользу одной внешней суммы.


\smallskip

Чтобы получить практические выводы о периоде занятости, надо в~равенстве~\eqref{AZ:eq:main-mart} положить 
$\mathfrak s\hm=0$ и~вычислить условные ожидания
при условиях $\{\kappa_{1,0}\hm=x, \Gamma_0\hm=\Gamma^{(r)}\}$, $x\hm=0, 1, \ldots$ и~$r\hm=\overline{1, 2m}$. 
Результаты содержатся в~следующих ниже теоремах.

\smallskip

\noindent
\textbf{Лемма~1.}\
%  \label{AZ:lem:1}
 \textit{Пусть $u>0$ и~$\bar\lambda_1 u\hm<\ell_1$. Тогда уравнение}
  \begin{equation}
    z^{\ell_1}=\exp\left\{ \lambda_1 u  \left(f_1(z)-1\right)\right\}
    \label{AZ:eq:maineq}
  \end{equation}
 \textit{имеет $\ell_1$ решений в~круге $|z|\hm\leqslant1$, причем ровно одно из них 
$z\hm=1$}.

\noindent
Д\,о\,к\,а\,з\,а\,т\,е\,л\,ь\,с\,т\,в\,о\,.\ \
  Для доказательства воспользуемся обобщением теоремы Руше из работы~\cite{Klimenok}. Представим уравнение~\eqref{AZ:eq:maineq} в~виде
  $$
  \theta_1(z)\hm+\theta_2(z)\hm=0\,,
  $$ 
  где 
  $$
  \theta_1(z)=z^{\ell_1}; \enskip
\theta_2(z)\hm=-\exp\{\lambda_1
  u(f_1(z)-1)\}.
  $$
   Необходимо проверить, что 
   \begin{enumerate}[(1)]
   \item $|\theta_1(z)|\hm>|\theta_2(z)|$ для
  всех $z\hm\neq 1$ таких, что $|z|\hm=1$; 
  \item $\theta_1(1)\hm=-\theta_2(1)$; 
  \item $(\theta_1'(1)+\theta_2'(1))/\theta_1(1)\hm>0$.
  \end{enumerate}
  
   Пункт~1 следует из неравенств
\begin{multline*}
  |\theta_2(z)| = e^{-\lambda_1 u} \left\vert e^{\lambda_1 u f_1(z)} \right\vert <
  e^{-\lambda_1 u}  e^{\lambda_1 u |f_1(z)|}  \leqslant{}\\
  {}\leqslant 1=|\theta_1(z)|.
  \end{multline*}
  Пункт~2 тривиален. Для доказательства пункта~3 заметим, что
  $$
  \hspace{15mm}\fr{\theta_1'(1)+\theta_2'(1)}{\theta_1(1)}=\ell_1-\bar \lambda_1 u>0\,.\hspace{15mm}\square
  $$


Для дальнейшего понадобится обозначение
\begin{multline*}
\mathcal T(r)={}\\
{}=
\begin{cases}
  T_{r+1}+T_{r+2}+\cdots+T_{2m}, &\!\! r=\overline{1, 2m-1};\\
    0, & \! \!r=2m.
\end{cases}
\end{multline*}

\noindent
\textbf{Теорема~3.}\
\textit{Пусть $x\hm=1, 2, \ldots$ Пусть, далее, $\beta_1, \beta_2, \ldots,
\beta_{\ell_1-1}$~--- нули уравнения $z^{\ell_1}\hm=q(z;T)$, лежащие в~круге
  $|z|\hm<1$ и~$\beta_{\ell_1}\hm=1$.  Введем многочлен}
  
  \vspace*{-6pt}
  
  \noindent
  \begin{multline*}
    L_{x,r}(z)=\fr{(z-\beta_{1})(z-\beta_2)\cdots(z-\beta_{\ell_1-1})}
         {(1-\beta_{1})(1-\beta_2)\cdots(1-\beta_{\ell_1-1})}+{}\\
{}    +\sum\limits_{j=1}^{\ell_1-1}(\beta_{j})^{x-1}
    \exp\{\lambda_{1}(T_{1}+\mathcal{T}(r))(f_{1}(\beta_{j})-1)\}
    \fr{z-1}{\beta_{j}-1}\times{}\\
    {}\times \prod\limits_{k\neq j}\fr{z-\beta_{k}}{\beta_{j}-\beta_{k}}\,.
\end{multline*}

\vspace*{-7pt}

\noindent
\textit{Тогда}

\vspace*{-6pt}

\noindent
\begin{multline*}
    L'_{x,r}(1) =\sum\limits_{j=1}^{\ell_1-1} \fr{1}{1-\beta_{j}} +
    \sum\limits_{j=1}^{\ell_1-1}(\beta_{j})^{x-1}\times{}\\
    {}\times
    \exp\{\lambda_{1}(T_{1}+\mathcal T(r))(f_{1}(\beta_{j})-1)\} 
    \prod\limits_{k\neq j}\fr{1-\beta_{k}}{\beta_{j}-\beta_{k}}\,;
   % \label{AZ:eq:E-kappa-eta-1}
    \end{multline*}
    
\vspace*{-12pt}

    \noindent
    \begin{multline}
  \mathbb E\left( \kappa_{1,\nu(0)-1}+\eta_{1,\nu(0)-1} \Big\vert
  \left\{\kappa_{1,0}=x,\Gamma_{0}=\Gamma^{(r)}\right\}\right)={}\\
  {}=1+L'_{x,r}(1);
   \label{AZ:eq:E-nu0-1}
   \end{multline}
   
  \vspace*{-12pt}

    \noindent
    \begin{multline*}
  \mathbb{E}\left(\left[\fr{\nu(0)-
1}{2m}\right]\bigg|\left\{\kappa_{1,0}=x,\Gamma_{0}=\Gamma^{(r)}\right\}\right)={}\\
{}=
  \fr{x+\bar\lambda_1(T_1+\mathcal T(r))-1-L'_{x,r}(1)}{\ell_1-\bar\lambda_1 T}\,;
\end{multline*}
\textit{средний период занятости очереди~$O_1$ при условии
$\{\kappa_{1,0}\hm=x,\ \Gamma_0\hm =\Gamma^{(r)}\}$ равен}

\vspace*{-6pt}

\noindent
  \begin{multline}
    \mathbb E\left( T_{r\oplus1}+\cdots+T_{r\oplus \nu(0)}\big\vert
    \kappa_{1,0}=x,  \Gamma_0=\Gamma^{(r)}\right)={}    \\ 
    {}=
    T_{1}+\mathcal T(r)+{}\\
    {}+
    T     \fr{x+\bar\lambda_1(T_1+\mathcal T(r)-1-L'_{x,r}(1))}
    {\ell_1-\bar{\lambda}_1 T}\,.\!\!
    \label{AZ:eq:busy-x-r-1}
  \end{multline}

\noindent
Д\,о\,к\,а\,з\,а\,т\,е\,л\,ь\,с\,т\,в\,о\,.\ \
  Полагая в~равенстве~\eqref{AZ:eq:main-mart} $\mathfrak {s}\hm=0$, % $x>0$ и~$r\neq 2m$
  получим:
  
  \pagebreak
  
  \noindent
  \begin{multline}
    \mathbb{E}\biggl(
    z^{\xi_{1,0}+\cdots+\xi_{1,\nu(0)-1}+%
      \kappa_{1,\nu(0)-1}+\eta_{1,\nu(0)-1}-\xi_{1,\nu(0)-1}}\times
    \\ \times
    \exp\{\lambda_{1}(T_{r\oplus1}+T_{r\oplus2}+\cdots{}\\
    {}\cdots +T_{r\oplus\nu(0)})
    (1-f_{1}(z))\}\Big|\{\kappa_{1,0}=x,\Gamma_{0}=\Gamma^{(r)}\}\biggr)={}\\
    {}=z^x.
    \label{AZ:eq:e4}
\end{multline}
На событии $\{\omega\colon \kappa_{1,0}=x\}$ при
$x\hm>0$ с~вероятностью единица выполняется равенство
\begin{equation*}
  \xi_{1,0}+\cdots+\xi_{1,\nu(0)-1}=
  \ell_1\left(\left[\fr{\nu(0)-1}{2m}\right]+1\right),
 % \label{AZ:eq:xi4}
\end{equation*}
где квадратные скобки~$[\cdot]$ обозначают целую часть числа.  Кроме того, на
том же событии при $x\hm>0$ с~вероятностью единица
\begin{equation*}
T_{r\oplus1}+T_{r\oplus2}+\cdots+T_{r\oplus\nu(0)}=
T_1+ \mathcal T(r) +T\left[\fr{\nu(0)-1}{2m}\right]\!.
%\label{AZ:eq:T-5}
\end{equation*}
В силу определения величины $\nu(0)$, $\xi_{1,\nu(0)-1}\hm=\ell_1$. Поэтому
равенство~\eqref{AZ:eq:e4} примет вид:
\begin{multline}
  z^{x}\exp\left\{\lambda_{1}(T_{1}+\mathcal{T}(r))(f_{1}(z)-1)\right\}={}  \\
  {} =
  \mathbb{E}\Bigl( z^{\kappa_{1,\nu(0)-1}+\eta_{1,\nu(0)-1}}\times{}\\
  {}\times 
  \left( z^{\ell_1}\exp\{ \lambda_{1}T(1-f_{1}(z))\}
  \right)^{\left[ (\nu(0)-1)/(2m)\right]}\,\Big|\\
  \left\{\kappa_{1,0}=x,\Gamma_{0}=\Gamma^{(r)}\right\}\Bigr).
  \label{AZ:eq:cyc1}
\end{multline}
По лемме~1 уравнение $z^{\ell_1}\hm=q(z;T)$ имеет ровно~$\ell_1$
заявленных в~формулировке теоремы корней $\beta_1, \beta_2, \ldots,
\beta_{\ell_1}$  в~круге $|z|\hm\leqslant1$. Перенумеруем корни так, чтобы
$\beta_{\ell_1}\hm=1$. Из равенства~\eqref{AZ:eq:cyc1} подстановкой $z\hm=\beta_j$
находим:
\begin{multline}
  \label{AZ:eq:bet}
  (\beta_{j})^{x}\exp\{\lambda_{1}(T_{r\oplus1}+\cdots+T_{2m}+T_{1}) %
  (f_{1}(\beta_{j})-1)\}={}  \\ 
  {}=
  \mathbb{E}\left((\beta_{j})^{\kappa_{1,\nu(0)-1}+\eta_{1,\nu(0)-1}}\Big|
 \left \{\kappa_{1,0}=x,\Gamma_{0}=\Gamma^{(r)}\right\}\right),\\
  j=\overline{1,\ell_1-1}\,.
\end{multline}
Введем условные вероятности

\vspace*{-4pt}

\noindent
\begin{multline*}
\alpha_{k}={}\\
{}+\mathbb{P}\left(X_{\nu(0)-1}\hm+\eta_{\nu(0)-1}\hm=k\big\vert 
X_{0}\hm=x,\Gamma_{0}\hm=\Gamma^{(r)}\right),\\
 k=\overline{0, \ell_1}
\end{multline*}

\vspace*{-4pt}

\noindent
(зависимость от~$x$, $r$ подразумевается). Заметим, что
$\kappa_{1,\nu(0)-1}\hm\geqslant1$. Тогда математическое ожидание в~правой части
равенства~\eqref{AZ:eq:bet}

\vspace*{-4pt}

\noindent
\begin{multline*}
\!\mathbb{E}\left((\beta_{j})^{\kappa_{1,\nu(0)-1}+\eta_{1,\nu(0)-1}}
\Big\vert  \left\{\kappa_{1,0}=x,\Gamma_{0}=\Gamma^{(r)}\right\}\right)={}\\
{}=
\alpha_{1}\beta_{j}+\alpha_{2}(\beta_{j})^{2}+\cdots
+\alpha_{\ell}(\beta_{j})^{\ell_1}.
\end{multline*}

\vspace*{-4pt}

\noindent
Для определения неизвестных величин~$\alpha_1, \alpha_2, \ldots\linebreak
\ldots,
\alpha_{\ell_1}$ получаем неоднородные линейные уравнения (с~присоединенным
условием нормировки):

\columnbreak

\noindent
\begin{multline*}
  \alpha_{1}+\alpha_{2}\beta_{j}+\cdots+\alpha_{\ell_1}\beta_{j}^{\ell_1-1}={}\\
  {}=
  (\beta_{j})^{x-1}\exp\{\lambda_{1}(T_{1}+
  \mathcal T(r))(f_{1}(\beta_{j})-1)\},\\
   j=\overline{1,\ell_1-1}\,;
   \end{multline*}
   
   \vspace*{-9pt}
   
   \noindent
   $$
\alpha_{1}+\alpha_{2}+\cdots+\alpha_{\ell}=1.
$$
Поскольку неизвестные выступают как коэффициенты многочлена степени
$(\ell_1-1)$, принимающего в~заданных точках заданные значения, естественно
ввести интерполяционный многочлен
$$
L_{x,r}(z)=\alpha_1\hm+\alpha_2 z+\cdots+ \alpha_{\ell_1} z^{\ell_1-1},
$$
 который 
с~необходимостью будет иметь вид, указанный в~формулировке теоремы. Но тогда

\vspace*{-6pt}

\noindent
\begin{multline*}
   \! \mathbb{E} \left( \kappa_{1,\nu(0)-1}+\eta_{1,\nu(0)-1} \Big\vert
    \left\{\kappa_{1,0}=x,\Gamma_{0}=\Gamma^{(r)}\right\}\right)={}    \\ 
    {}=
    \alpha_1+2\alpha_2+\cdots+\ell_1 \alpha_{\ell_1} =
    \fr{d}{dz}\left(z L_{x,r}(z)\right)\Big|_{z=1} ={}\\
    {}= 1+L'_{x,r}(1).
\end{multline*}

\vspace*{-4pt}


Дифференцируя равенство~(\ref{AZ:eq:cyc1}) по~$z$ в~точке $z\hm=1$, получим:

\vspace*{-6pt}

\noindent
\begin{multline*}
  x+\bar\lambda_{1}(T_{1}+\mathcal T(r))={}\\
  {}=
  \mathbb E\left( \kappa_{1,\nu(0)-1}+\eta_{1,\nu(0)-1} \Big\vert 
 \left\{ \kappa_{1,0}=x,\Gamma_{0}=\Gamma^{(r)}\right\}\right)+{}\hspace*{-0.5pt}  \\ 
  {} +
  (\ell_1-\bar\lambda_{1}T)  \mathbb{E}\biggl(\biggl[\fr{\nu(0)-1}{2m}\biggr]
  \bigg| \left\{\kappa_{1,0}=x,\Gamma_{0}=\Gamma^{(r)}\right\}\biggr).\hspace*{-0.5pt}
\end{multline*}

\vspace*{-4pt}

\noindent
Отсюда приходим к~равенству~\eqref{AZ:eq:E-nu0-1}.  Переходя теперь к~абсолютному времени, можно записать длительность периода занятости как
$$
  T_{1}+\mathcal T(r)+
  T\mathbb{E}\biggl(\biggl[\fr{\nu(0)-1}{2m}\biggr]
  \bigg| \left\{\kappa_{1,0}=x,\Gamma_{0}=\Gamma^{(r)}\right\}\biggr).
$$
Отсюда получается соотношение~\eqref{AZ:eq:busy-x-r-1}.


%\pagebreak


Следующие две теоремы доказываются аналогично.

\noindent
\textbf{Теорема~4.}\
\textit{Пусть при $r\hm<2m$ определен полином}
    \begin{multline*}
      L_{0,r}(z)=(1-e^{-\lambda_1 T_{r+1}})\times{}\\
      {}\times
      \fr{(z-\beta_{1})(z-\beta_2)\cdots(z-\beta_{\ell_1-1})}%
    {(1-\beta_{1})(1-\beta_2)\cdots(1-\beta_{\ell_1-1})}+\\
    +\sum\limits_{j=1}^{\ell_1-1}(\beta_{j})^{-1}
    \left(1-e^{-\lambda_1 T_{r+1}f_1(\beta_j)}\right)\times{}\\
    {}\times
    \exp\{\lambda_{1}(T_{1}+\mathcal T(r))(f_{1}(\beta_{j})-1)\}
    \fr{z-1}{\beta_{j}-1}\prod\limits_{k\neq j}\fr{z-\beta_{k}}{\beta_{j}-\beta_{k}}\,.\hspace*{-2.84pt}
\end{multline*}
\textit{Тогда имеют место соотношения}

\noindent
  \begin{multline*}
    \mathbb E  \left(  I(\{\nu(0) >1\})  \left[ \fr{\nu(0)-1}{2m} \right] 
\Bigg\vert \right.\\
\left. \kappa_{1,0}=0, \Gamma_0=\Gamma^{(r)} 
\vphantom{\fr{\nu(0)-1}{2m}}
\right) ={}\\
{}=
  \left(\ell_1-\bar\lambda_1 T\right)^{-1}  \Big( \bar\lambda_1 T_{r+1} e^{-\lambda_1
      T_{r+1}} +{}\\
      {}+ \bar\lambda_1 \left(T_1+\mathcal T(r)\right)\left(1-e^{-\lambda_1 T_{r+1}}\right)
    -{}\\
 {}-\left(1-e^{-\lambda_1 T_{r+1}}\right)-L_{0,r}'(1)    \Big);
  %\label{AZ:eq:x0r2m-1}
  \end{multline*}
  
  \vspace*{-12pt}
  
  \noindent
  \begin{multline*}
    \mathbb {E}\left(  T_{r\oplus1} +\cdots+T_{r\oplus \nu(0)}\mid \kappa_{1,0}=0,
    \Gamma_0=\Gamma^{(r)}\right)={}\\
    {}= T_{r+1} e^{-\lambda_1 T_{r+1}} +
    \left(T_1+\mathcal T(r)\right)\left(1-e^{-\lambda_1 T_{r+1}}\right) + {}\\
    {}+\mathbb{E}\! \left(
    I(\{\nu(0)>1\}) \left[ \fr{\nu(0)-1}{2m} \right] \Big| \kappa_{1,0}=0,
    \Gamma_0=\Gamma^{(r)} \!\right).\hspace*{-6pt}
  %  \label{AZ:eq:x0r2m-2}
\end{multline*}


\noindent
\textbf{Теорема~5.}\
\textit{Пусть}

\vspace*{-6pt}

\noindent
  \begin{multline*}
    L_{0,2m}(z)=
    \Bigl(1-{}\\
    {}-\sum\limits_{b=0}^{\ell_1} \varphi_1(b; T_1)\Bigr)
      \fr{(z-\beta_{1})(z-\beta_2)\cdots(z-\beta_{\ell_1-1})}%
    {(1-\beta_{1})(1-\beta_2)\cdots (1-\beta_{\ell_1-1})}+{}\\
{} +\sum\limits_{j=1}^{\ell_1-1}(\beta_{j})^{-1}
    \left(
    \vphantom{\sum\limits_{b=0}^{\ell_1}} 
    \exp\{\lambda_{1} T_1(f_{1}(\beta_{j})-1)\}-{}\right.\\
\left.    {}-\sum\limits_{b=0}^{\ell_1} 
\varphi_1(b; T_1) (\beta_j)^b
    \right)
    \fr{z-1}{\beta_{j}-1}\prod\limits_{k\neq j}\fr{z-\beta_{k}}{\beta_{j}-
\beta_{k}}\,.
  \end{multline*}
  
  \vspace*{-4pt}
  
  \noindent
  \textit{Имеют место равенства}
  
\vspace*{-6pt}

\noindent
  \begin{multline*}
        \mathbb{E}  \left(  I(\{\nu(0) >1\})  \left[ \fr{\nu(0)-1}{2m} \right] 
\Big| \right.\\
\kappa_{1,0}=0,
\left. \Gamma_0=\Gamma^{(2m)} 
\vphantom{\left[ \fr{\nu(0)-1}{2m} \right]}
 \right ) =
        \left(\ell_1-\bar\lambda_1 T\right)^{-1} \times{}        \\ 
        {} \times
        \left(
        \bar\lambda_1 T_1 - \sum\limits_{b=0}^{\ell_1} b \varphi_1(b; T_1)
        -L_{0,2m}(1)-L_{0,2m}'(1)    \right);
 % \label{AZ:eq:x0r2m-3}
  \end{multline*}
  
  \vspace*{-12pt}
  
  \noindent
  \begin{multline*}
    \mathbb {E}\left(  T_{r\oplus1} +\cdots+T_{r\oplus \nu(0)}\mid \kappa_{1,0}=0,
    \Gamma_0=\Gamma^{(2m)}\right)={}    \\ 
    {} =
    T_{1}+ T 
    \mathbb {E} \left(
    I(\{\nu(0)>1\}) \left[ \fr{\nu(0)-1}{2m} \right] \Big|\right.\\
 \left.     \kappa_{1,0}=0,    \Gamma_0=\Gamma^{(2m)}
 \vphantom{\left[ \fr{\nu(0)-1}{2m} \right]}
  \right) .
  %\label{AZ:eq:x0r2m-4}
  \end{multline*}

\vspace*{-24pt}



\section{Заключение}

\vspace*{-3pt}

Для управляющих систем массового обслуживания с~несколькими конфликтными
потоками интерпретация периода занятости как суммарного времени функционирования
до опустошения очередей оказывается продуктивным способом переноса понятия на
задачи с~дискретным временем. При этом использование мартингальной техники
позволяет изучить длительность периода занятости очереди при циклическом
обслуживании с~фиксированным ритмом переключения. По-ви\-ди\-мо\-му, 
данный подход в~дальнейшем необходимо опробовать и~на более сложных входных потоках и~алгоритмах
обслуживания.

\vspace*{-8pt}

{\small\frenchspacing
 {\baselineskip=10.8pt
 %\addcontentsline{toc}{section}{References}
 
 \vspace*{-6pt}
 
 \begin{thebibliography}{99}   
  \bibitem{Ivchenko:Kashtanov:Kovalenko}% %1
    \Au{Ивченко Г.\,И., Каштанов~В.\,А., Коваленко~И.\,Н.} Теория массового
    обслуживания.~--- 2-е изд.~--- М.:
    Либроком, 2012. 304~c.
    
  \bibitem{Bruneel:Kim}%2
    \Au{Bruneel H., Kim~B.} Discrete-time models for communication systems
    including ATM.~--- Norwell: Kluwer Academic Publs.,
    1993.  Vol.~205. 210~p.
    
  \bibitem{Alfa}%3
    \Au{Alfa A.\,S.} Applied discrete-time queues.~--- 2nd ed.~--- New York,
    NY, USA: Springer-Verlag, 2016. 400~p.
    
    \bibitem{Pechinkin:Razumchik}%4
    \Au{Печинкин~А.\,В., Разумчик~Р.\,В.} Системы массового обслуживания в~дискретном времени.~--- М.: Физматлит, 2018. 432~с.
    
  
    
  \bibitem{Gergely:Torok:1974} %5
   \Au{Gergely T., T\,$\ddot{\mbox{\!o}}$r\,$\ddot{\mbox{\!o}}$k~T.\,L.} On the busy period of discrete-time
    queues~// J.~Appl. Probab., 1974. Vol.~11. Iss.~4. P.~853--857. doi: 10.2307/3212571.
    
  \bibitem{Chaudhry:Zhan:1994} %6
\Au{Chaudhry M.\,L., Zhao Y.\,Q.} First-passage-time and busy-period
    distributions of discrete-time Markovian queues: $\mathrm{Geom}(n)/\mathrm{Geom}(n)/1/N$~//
    Queueing Syst., 1994. Vol.~18. P.~5--26. doi: 10.1007/BF01158772.
    
  \bibitem{Foss:Sapozhnikov:2004} %7
\Au{Foss S., Sapozhnikov~A.} On the existence of moments for the busy
    period in a~single-server queue~// Math. Oper.
    Res., 2004. Vol. 29. Iss.~3. P.~592--601. doi: 10.1287/moor.1030.0074.
    
  \bibitem{Brown:Balakrishnan:2021} %8
\Au{Brown G.\,B.} Busy periods of discrete-time queues using the Lagrange
    implicit function theorem~// Oper. Res.
    Lett., 2021. Vol.~49. P.~650--654.  doi: 10.1016/j.orl.2021.06.014.
    
  \bibitem{Fedotkin:1975} %9
\Au{Федоткин М.\,А.} Тео\-ре\-ти\-ко-мно\-жест\-вен\-ный подход при анализе
    дискретных нелинейных систем массового обслуживания~// Автоматика 
    и~вычислительная техника, 1975. №\,2.  С.~58--64.
    
  \bibitem{Fedotkin:1978} %10
\Au{Федоткин М.\,А.} Неполное описание квазирегенерирующих входных потоков 
неоднородных требований и~транспортные потоки~// 4-е Всесоюзное совещание по статистическим
    методам тео\-рии управ\-ле\-ния.~--- М.: Наука, 1978. С.~234--236.
    
  \bibitem{Fedotkin:1981} %11
\Au{Федоткин М.\,А.} Неполное описание потоков неоднородных требований~//
     Теория  массового обслуживания.~--- М.: МГУ--ВНИИСИ, 1981. С.~113--118.
     
%\columnbreak
    
  \bibitem{Bacelli-1} %12
\Au{Baccelli F., Makowski~A.\,M.}
    {Direct martingale argument for stability: The $M/G/1$ case}~//
    Syst. Control Lett., 1985. Vol.~6. P.~181--186. doi: 10.1016/0167-6911(85)90038-6.
    
  \bibitem{Bacelli-2} %13
\Au{Baccelli F., Makowski~A.\,M.}
  {Dynamic, transient and stationary behavior of the $M/\mathrm{GI}/1$ queue via 
martingales}~//
   Ann. Probab., 1989. Vol.~17. Iss.~4. P.~1691--1699.
  
\bibitem{Klimenok} %14
\Au{Klimenok V.\,L.} On the modification of Rouche's theorem for the queueing
  theory problems~// Queueing Syst., 2001. Vol.~38. P.~431--434. doi: 10.1023/A:1010999928701.
  
  \end{thebibliography}

 }
 }

\end{multicols}

\vspace*{-9pt}

\hfill{\small\textit{Поступила в~редакцию 04.03.24}}

%\vspace*{8pt}

%\pagebreak

\newpage

\vspace*{-28pt}

%\hrule

%\vspace*{2pt}

%\hrule

\vspace*{-2pt}


\def\tit{TOWARDS A DEFINITION OF~A~BUSY PERIOD\\ UNDER~NONLOCAL DESCRIPTION OF~INPUT FLOWS}


\def\titkol{Towards a definition of~a~busy period under~nonlocal description of~input flows}


\def\aut{A.\,V.~Zorine}

\def\autkol{A.\,V.~Zorine}

\titel{\tit}{\aut}{\autkol}{\titkol}

\vspace*{-15pt}


\noindent
National Research Lobachevsky State University of Nizhny Novgorod, 23~Prosp.\ Gagarina, Nizhni Novgorod 603022, Russian Federation






\def\leftfootline{\small{\textbf{\thepage}
\hfill INFORMATIKA I EE PRIMENENIYA~--- INFORMATICS AND
APPLICATIONS\ \ \ 2024\ \ \ volume~18\ \ \ issue\ 3}
}%
 \def\rightfootline{\small{INFORMATIKA I EE PRIMENENIYA~---
INFORMATICS AND APPLICATIONS\ \ \ 2024\ \ \ volume~18\ \ \ issue\ 3
\hfill \textbf{\thepage}}}

\vspace*{2pt}





\Abste{In course of a probabilistic modeling and analysis of complex controlled
queueing systems with several conflicting input flows, in a~series of papers, an approach
was successfully applied, one of its features being a~nonlocal description of various system
building blocks. In this description, some information about true arrival and leave
times of customers is lost. It leads to difficulties in defining a~busy period but that is one of
classic performance metrics for an operating queueing system. In this paper, a~ controlled
queuing system busy period definition is based on selecting those observation instants
when queues reach zero level. A~cyclic service algorithm with fixed switching times as an
example using a~martingale technique and effective computational formulas are obtained for
the mathematical expectation of busy periods related to individual queues.}


\KWE{controlled queueing system; nonlocal description of blocks; nonordinary
Poisson flows; cyclic service algorithm; busy period; multivariate denumerable Markov
chain; martingale; generalized Rouch\'{e}'s theorem; Lagrange interpolation polynomial}

  \DOI{10.14357/19922264240306}{YKSIBJ}

%\vspace*{-12pt}


    
     % \Ack

%\vspace*{-3pt}

%\noindent



  \begin{multicols}{2}

\renewcommand{\bibname}{\protect\rmfamily References}
%\renewcommand{\bibname}{\large\protect\rm References}

{\small\frenchspacing
 {%\baselineskip=10.8pt
 \addcontentsline{toc}{section}{References}
 \begin{thebibliography}{99}


%1
\bibitem{Ivchenko:Kashtanov:Kovalenko-1}
\Aue{Ivchenko, G.\,I., V.\,A.~Kashtanov, and I.\,N.~Kovalenko.} 2012.
\textit{Teoriya massovogo obsluzhivaniya} [Queueing theory]. 2nd ed.
Moscow: Librokom. 304~p.

%2
\bibitem{Bruneel:Kim-1}
\Aue{Bruneel, H., and B.~Kim}. 1993.
\textit{Discrete-time models for communication systems including ATM}. 
Norwell: Kluwer Academic Publs. Vol.~205. 210~p.
 

%3
\bibitem{Alfa-1}
\Aue{Alfa, A.\,S.} 2016. 
\textit{Applied discrete-time queues}. 2nd ed. 
New York, NY: Springer-Verlag. 400~p.

%4
\bibitem{Pechinkin:Razumchik-1}
\Aue{Pechinkin, A.\,V., and R.\,V.~Razumchik.} 2018. 
\textit{Sistemy massovogo obsluzhivaniya v~diskretnom vremeni} [Discrete time queuing systems]. 
Moscow: Fizmatlit. 432~p.


%5
\bibitem{Gergely:Torok:1974-1} 
\Aue{Gergely, T., and T.\,L.~T$\ddot{\mbox{o}}$r$\ddot{\mbox{o}}$k.} 1974.
On the busy period of discrete-time queues. 
\textit{J.~Appl. Probab.} 11(4):853--857.
doi: 10.2307/3212571.

%6
\bibitem{Chaudhry:Zhan:1994-1} 
\Aue{Chaudhry, M.\,L., and Y.\,Q.~Zhao.} 1994.
First-passage-time and busy-period distributions of discrete-time Markovian queues: $\mathrm{Geom}(n)/\mathrm{Geom}(n)/1/N$. 
\textit{Queueing Syst.} 18:5--26. 
doi: 10.1007/BF01158772.

%7
\bibitem{Foss:Sapozhnikov:2004-1} 
\Aue{Foss, S., and A.~Sapozhnikov.} 2004. 
On the existence of moments for the busy period in a~single-server queue. 
\textit{Math. Oper. Res.} 29(3):592--601. doi: 10.1287/moor. 1030.0074.

%8
\bibitem{Brown:Balakrishnan:2021-1} 
\Aue{Brown, G.\,B.} 2021. 
Busy periods of discrete-time queues using the Lagrange implicit function theorem. 
\textit{Oper. Res. Lett.} 49(5):650--654. doi: 10.1016/j.orl.2021.06.014.

%9
\bibitem{Fedotkin:1975-1} 
\Aue{Fedotkin, M.\,A.} 1975.
Set-theoretic approach in analyzing discrete nonlinear queuing systems.
\textit{Autom. Control  Comp.~S.} 9(2):50--54.
    
%10
\bibitem{Fedotkin:1978-1} 
\Aue{Fedotkin, M.\,A.} 1978.
Nepolnoe opisanie kva\-zi\-re\-ge\-ne\-ri\-ru\-yushchikh vkhodnykh potokov neodnorodnykh trebovaniy i~transportnye potoki 
[Incomplete description of quasi-regenerating input flows of nonhomogeneous customers and transport flows].
\textit{4-e Vsesoyuznoe soveshchanie po statisticheskim metodam teorii upravleniya} 
[4th All-Union Meeting on Statistical Methods of Control Theory]. Moscow: Nauka. 234--236.


%11
\bibitem{Fedotkin:1981-1} 
\Aue{Fedotkin, M.\,A.} 1981. 
Nepolnoe opisanie potokov neodnorodnykh trebovaniy [Incomplete description of flows of inhomogeneous customers].
\textit{Teoriya  massovogo obsluzhivaniya} [Queueing theory].
Moscow: MGU--VNIISI. 113--118.

%12
\bibitem{Bacelli-1-1}
\Aue{Baccelli, F., and A.\,M.~Makowski.} 1985.
Direct martingale argument for stability: The $M/G/1$ case. 
\textit{Syst. Control Lett.} 6(3):181--186.
doi: 10.1016/0167-6911(85)90038-6.

%13
\bibitem{Bacelli-2-1}
\Aue{Baccelli, F., and A.\,M.~Makowski.} 1989.
Dynamic, transient and stationary behavior of the $M/\mathrm{GI}/1$ queue via martingales.
\textit{Ann. Probab.} 17(4):1691--1699.

%14
\bibitem{Klimenok-1}
\Aue{Klimenok, V.\,L.} 2001. 
On the modification of Rouche's theorem for the queueing theory problems. 
\textit{Queueing Syst.} 38:431--434.
doi: 10.1023/A:1010999928701.




\end{thebibliography}

 }
 }

\end{multicols}

\vspace*{-6pt}

\hfill{\small\textit{Received March 4, 2024}} 

\vspace*{-18pt}

\Contrl

\vspace*{-3pt}

\noindent 
\textbf{Zorine Andrei V.} (b.\ 1978)~--- Doctor of Science in physics and mathematics, associate professor, 
head of department,
National Research Lobachevsky State University of Nizhny Novgorod, 
23~Prosp.\ Gagarina, Nizhni Novgorod 603022, Russian Federation; \mbox{andrei.zorine@itmm.unn.ru}


\label{end\stat}

\renewcommand{\bibname}{\protect\rm Литература}  %6
\def\stat{malashenko}

\def\tit{ПОСЛЕДОВАТЕЛЬНЫЙ АНАЛИЗ И~МЕТРИЧЕСКИЕ ОЦЕНКИ ПРЕДЕЛЬНЫХ
РАСПРЕДЕЛЕНИЙ МЕЖУЗЛОВЫХ ПОТОКОВ В~МНОГОПОЛЬЗОВАТЕЛЬСКОЙ СЕТИ}

\def\titkol{Последовательный анализ и~метрические оценки предельных
распределений межузловых потоков в %~многопользовательской 
сети}

\def\aut{Ю.\,Е. Малашенко$^1$}

\def\autkol{Ю.\,Е. Малашенко}

\titel{\tit}{\aut}{\autkol}{\titkol}

\index{Малашенко Ю.\,Е.}
\index{Malashenko Yu.\,E.}


%{\renewcommand{\thefootnote}{\fnsymbol{footnote}} \footnotetext[1]
%{Исследование выполнено при финансовой поддержке Российского научного фонда (проект 
%<<Информатика>> ФИЦ ИУ РАН, Москва).}}


\renewcommand{\thefootnote}{\arabic{footnote}}
\footnotetext[1]{Федеральный исследовательский центр <<Информатика и~управление>> Российской академии 
\mbox{mala-yur@yandex.ru}}


%\vspace*{-6pt}



\Abst{Для оценки функциональных возможностей
многопользовательской сети связи аналилизируется множество векторов межузловых потоков при предельных распределениях ресурсов
сети. В~рамках многопродуктовой модели про\-пуск\-ные спо\-соб\-ности ребер рас\-смат\-ри\-ва\-ют\-ся 
как компоненты вектора ресурсов различных
типов, которые требуются для передачи потоков различных видов.
При проведении вычислительных экспериментов на каждой итерации вычисляются нормы векторов совместно допустимых межузловых
потоков, при передаче которых полностью используется пропускная спо\-соб\-ность всех ребер сети. Полученные последовательности
метрических оценок позволяют анализировать особенности множества до\-сти\-жи\-мости и~эф\-фек\-тив\-ность использования ресурсов сети при
уравнительном распределении про\-пуск\-ной спо\-соб\-ности между корреспондентами.}

\KW{многопродуктовая потоковая сетевая
модель; множество достижимых межузловых потоков; предельные
распределения пропускной способности}

\DOI{10.14357/19922264220306} 
  
%\vspace*{-3pt}


\vskip 10pt plus 9pt minus 6pt

\thispagestyle{headings}

\begin{multicols}{2}

\label{st\stat}

\section{Введение}

Данная работа продолжает исследования функциональных характеристик
сетевых сис\-тем связи~[1]. В~настоящее время математические модели
передачи многопродуктового потока применяются для поиска
распределений потоков и~ресурсов в~многопользовательских
телекоммуникационных\linebreak сетях~[2--4]. Разрабатываются методы анализа
с~учетом вектора требований всех \textit{равноправных} 
и~невзаимозаменяемых корреспондентов~[5]. С~позиций\linebreak методологии
исследования операций изучаются справедливые распределения потоков
и~ресурсов~[6].

Соответствующие \textit{недискриминирующие} правила управления
потоками являются решениями задач на максмин и/или получаются 
в~результате использования процедур последовательной
лексикографически упорядоченной оптимизации~[7].

В~настоящей работе пути соединения корреспондентов прокладываются
через со\-от\-вет\-ст\-ву\-ющие минимальные разрезы. Указанный метод\linebreak \mbox{можно}
рассматривать как возможный вариант решения задачи о~построении
SPLIT-марш\-ру\-тов~[8,~9]. В~рамках вычислительных экспериментов\linebreak на
многопродуктовой модели анализируются распределения межузловых
потоков  и~пропускной способ\-ности сети.  Для оценки функциональных
возможностей многопользовательской сети используется вектор
совместно допустимых межузловых потоков. Под ресурсом, выделяемым
некоторой паре узлов-кор\-рес\-пон\-ден\-тов,  понимается суммарное
значение тре\-бу\-емых пропускных способностей на всех ребрах,
расположенных на всех маршрутах при прохождении межузлового\linebreak потока
данного вида.  Сумма соответствующих реберных потоков трактуется
как полная нагрузка на сеть, возникающая при передаче заданного
межузлового потока. Рас\-смат\-ри\-ва\-ют\-ся распределения пропускной
способности и~межузловых потоков при предельной загрузке сети.
При проведении вычислительных экспериментов на каждой  итерации
вычисляется норма  вектора совместно допустимых межузловых
потоков.   Для оценки величины требуемых ресурсов при соединении
корреспондентов по различным путям для каж\-дой пары узлов
определяется максимальный однопродуктовый поток. Марш\-ру\-ты передачи
всех допустимых межузловых потоков  проходят по ребрам
соответствующих минимальных разрезов. Вычислительные эксперименты
проводились  для получения последовательности  мет\-ри\-че\-ских оценок
векторов межузловых потоков, принадлежащих множеству до\-сти\-жи\-мости
многопользовательской сети.

\section{Математическая модель}

В качестве математической модели для описания
многопользовательской сетевой системы используется следующая
формальная запись условий и~ограничений, которые должны
выполняться при одновременной передаче потоков различных видов
между всеми парами улов-корреспондентов:

Сеть $G(\mathbf{d})$ задается множествами $\langle V,
R,U,P\rangle$:
\begin{itemize}
\item  узлов (вершин) сети 
$$
V=\left \{v_{1}, v_{2},\dots,v_{n},\dots,v_{N}\right\};
$$
\item  неориентированных ребер 
$$
R=\left\{r_{1}, r_{2}, \dots, r_{k}, \dots,
r_{E}\right\}.
$$
\end{itemize}

Ребро $r_{k}$ \textit{соединяет} концевые вершины~$v_{n_k}$ и~$v_{j_k}$. 
Реб\-ру~$r_{k}$ ставятся в~соответствие две
ориентированные дуги $\{u_{k},u_{k+E}\}$ из множества
ориентированных дуг $U\hm=\{u_{1}, u_{2}, \dots, u_{k}, \dots,
u_{2E}\}$. Дуги $\{u_{k}, u_{k+E}\}$ определяют прямое и~обратное
на\-прав\-ле\-ние передачи потока по реб\-ру~$r_{k}$ между концевыми
вершинами $\{v_{n_k}, v_{j_k}\}$.

В многопользовательской сети~$G(\mathbf{d})$ рассматривается
$M\hm=N(N\hm-1)$ независимых, невзаимозаменяемых и~равноправных потоков
различных видов, которые передаются между уз\-ла\-ми-кор\-рес\-пон\-ден\-та\-ми
из множества 
$$
P=\left\{p_{1}, p_{2}, \dots, p_{M}\right\}.
$$

По определению, каждой паре уз\-лов-кор\-рес\-пон\-ден\-тов~$p_{m}$
соответствуют:
\begin{itemize}
\item вершина-ис\-точ\-ник с~номером~$s_{m}$, через которую входной поток
$m$-го вида~$z_{m}$ поступает в~сеть;
\item  вершина-при\-ем\-ник с~номером~$t_{m}$, из которой поток $m$-го
вида~$z_{m}$ покидает сеть.
\end{itemize}

В множестве~$P$ выделяется подмножество $P(R^{+})$ пар
уз\-лов-кор\-рес\-пон\-ден\-тов, расположенных в~концевых вершинах
ребра~$r_{k}$, $k\hm=\overline{1,E}$. Вводятся сле\-ду\-ющие обозначения:
пусть ребро~$r_{k}$  с~номером~$k$ соединяет вершины с~номерами~$n$ и~$j$ такими, что $n\hm< j$. Для со\-от\-вет\-ст\-ву\-ющей пары
уз\-лов-кор\-рес\-пон\-ден\-тов~$p_{k}$, расположенных в~узлах $\{v_{n},
v_{j}\}$, узел~$v_{n}$ считается источником, а узел~$v_{j}$~---
приемником потока $z_{k}$ $k$-го вида, который передается из узла
c номером~$n$ в~узел с~номером~$j$ для пары~$p_{k}$ с~номером~$k$.
Для пары $p^{}_{k+E} \Longleftrightarrow \{v_{j},v_{n}\}$ узел~$v_{j}$ 
считается источником~$s_{k+E}$, а~узел $v_m$~--- приемником~$t_{k+E}$ для пары с~номером~$p_{k+E}$. Формируется
$R^+\hm=\{1,2,3,\dots,E,E+1,\dots,2E\}$~--- список номеров смежных
пар.

Пары $p_{k}$ из подмножества~$P(R^{+})$ называются
\textit{смежными} уз\-ла\-ми-кор\-рес\-пон\-ден\-та\-ми. Все остальные
\textit{несмежные} пары уз\-лов-кор\-рес\-пон\-ден\-тов относятся к~множеству~$P(R^{-})$:
\begin{equation*}
P=P(R^{+})\cup P(R^{-});\quad
P(R^{+}) \cap P(R^{-}) = \emptyset.
\end{equation*}

Введем обозначения:
\begin{description}
\item[\,]
$z_{m}$~--- величина \textit{межузлового} потока $m$-го вида,
который поступает в~сеть из узла с~номером~$s_{m }$ и~покидает из
узла с~номером~$t_{m}$;
\item[\,]
$S(v_{n})$~--- множество номеров исходящих дуг, по которым поток
покидает узел~$v_{n}$;
\item[\,]
$T(v_{n})$~--- множество номеров входящих дуг, по которым поток
поступает в~узел~$v_{n}$.
\end{description}

Во всех узлах $v_{n}\in V$, $n\hm=\overline{1,N}$, для всех видов
потоков должны выполняться условия сохранения потоков:
\begin{multline}
\label{eq1} 
\sum\limits_{i\in S(v_n )} x_{mi}-\sum\limits_{i\in T(v_n )} x_{mi}
={}\\
{}=\begin{cases}
z_m, & \mbox{если } v=v^{}_{S_m}; \\
-z_m,&\mbox{если } v=v_{t_m}; \\
0&\mbox{в остальных случаях}, \\
\end{cases}
\end{multline}
$n=\overline{1,N}$, $m\hm=\overline{1,M}$, $x_{mi}\hm\ge 0$,
$z_{m}\hm\ge0$.

Величина {z}$_{m}$ равна входному потоку $m$-го вида, который
пропускается от источника к~приемнику пары $p_{m}$ при
распределении потоков $x_{mi}$ по дугам сети.

Каждому ребру $r_{k}\hm\in R$ приписывается неотрицательное число~$d_{k}$, 
определяющее суммарный предельно допустимый поток,
который можно передать по реб\-ру~$r_{k}$ в~обоих на\-прав\-ле\-ни\-ях. 
В~исходной сети компоненты вектора про\-пуск\-ных способностей
$\mathbf{d}\hm=(d_{1}, d_{2},\dots, d_{k}, \dots, d_{E})$~--- наперед
заданные положительные числа $d_{k}
\hm> 0$. Вектором $\mathbf{d}$ определяются сле\-ду\-ющие ограничения на сумму
дуговых потоков всех видов, пе\-ре\-да\-ва\-емых по реб\-ру~$r_{k}$:
\begin{multline}
\sum\limits_{m=1}^M (x_{mk}+x_{m(k+E)}) \le d_k,\\
 x_{mk}\ge 0\,,\enskip
 x_{m(k+E)}\ge 0\,, \enskip k=\overline {1,E}\,.
 \label{eq2} 
\end{multline}
В рамках данной модели пропускная спо\-соб\-ность ребер сети~--- вектор~$\mathbf{d}$~--- трактуется как <<\textit{ресурсное ограничение}>>,
а~сумма дуговых
 потоков рас\-смат\-ри\-ва\-ет\-ся как показатель использования
<<\textit{ресурсов}>> сети при передаче межузловых потоков
различных видов.

Для всех $z_{m}$ и~$x_{mi}$, удовлетворяющих
условиям~\eqref{eq1} и~\eqref{eq2}, вычисляются суммарные потоки:
\begin{equation}
 y_{m }=\sum\limits_{i=1}^{2E} {x}_{mi},\quad
m=\overline{1,M}\,.
\label{eq3}
\end{equation}

Суммарный реберный поток~$y_{m}$ характеризует
<<\textit{нагрузку}>> на сеть при передаче межузлового потока
величины $z_{m}$ из уз\-ла-ис\-точ\-ни\-ка~$s_{m}$ в~узел-при\-ем\-ник~$t_{m}$. 
Величина~$y_{m}$ показывает, какой суммарный
\textit{ресурс}~-- пропускная спо\-соб\-ность сети~-- требуется для
передачи межузлового потока~$z_{m}$, а~отношение
$w_{m}\hm={y_m}/{z_m}$,  $m\hm=\overline{1,M},$
показывает, какие \textit{ресурсы} необходимы для передачи
единичного потока $m$-го вида между узлами~$s_{m}$ и~$t_{m}$.

Ограничения~\eqref{eq1}--\eqref{eq3} задают подмножество
допустимых значений компонент вектора межузловых потоков
$\mathbf{z}\hm=\left(z_{1}, z_{2},\dots,z_{m},\dots,z_{M}\right)$:
\begin{equation*}
{Z}(\mathbf{d})=\left\{\mathbf{z} \ge 0 \mid
(\mathbf{z},\mathbf{x},\mathbf{y}) \ \mbox{удовлетворяют~\eqref{eq1}--\eqref{eq3}}
\right\}\!,
\!\!
%\label{eq4} 
\end{equation*}
а все допустимые распределения ресурсов принадлежат подмножеству
\begin{equation*}
{Y}(\mathbf{d})=\left\{\mathbf{y} \ge 0 \mid
(\mathbf{z},\mathbf{x},\mathbf{y}) \ \mbox{удовлетворяют~\eqref{eq1}--\eqref{eq3}}\right\}\!.
%\!\!\!\label{eq5}
\end{equation*}


\section{Метрические оценки предельных распределений}

Для оценки функциональных возможностей сис\-те\-мы рассматриваются
допустимые распределения межузловых потоков при предельных
загрузках ребер сети.

В рамках данного модельного описания монопольным режимом
называется способ управления, при котором все ресурсы сети
используются для передачи потока одной выделенной пары
уз\-лов-кор\-рес\-пон\-ден\-тов $p_{a}\hm\in P(R^-)$, а для всех
остальных потоки полагаются равными нулю.

Предельно допустимый поток, который можно передать между
фиксированной парой уз\-лов-кор\-рес\-пон\-ден\-тов $p_{a}$ в~монопольном
режиме, является решением стандартной, в~данном случае
однопродуктовой, задачи о~максимальном потоке.

\smallskip

\noindent
\textbf{Задача 1.} Найти
$z_a^0\hm=\max\limits_{\langle z,x\rangle \in Z(d)} z_a
$
при условии $z_{i}=0$, $i\hm=\overline{1,M}$, $i\hm\ne a$.

При решении задачи~1 для пары $p_{a}$ вы\-чис\-ля\-ют\-ся: межузловой
поток~$z_a^0$; дуговые потоки $\{x^{0}_{ak};x^{0}_{a(k+E)}\}$,
$k\hm=\overline{1,E}$; суммарное значение реберного
потока~$y_{a}^{0}\hm=\sum\nolimits_{i=1}^{2E} {x}_{ai}^{0}$.

Поток величины $z_a^0$ является \textit{максимальным потоком},
пе\-ре\-да\-ва\-емым в~\textit{монопольном режиме} для пары
уз\-лов-кор\-рес\-пон\-ден\-тов~$p_{a}$, $p_{a}\hm\in P(R^-)$, в~сети~$G(d)$.

Задача~1 решается последовательно для всех $p_{m}\in P(R^-)$,
вы\-чис\-ля\-ют\-ся значения $z_{m}^{0}(t)$.

При проведении вычислительных экспериментов использовалась
итерационная процедура для оценки функциональных возможностей
сис\-те\-мы при передаче межузловых потоков по нескольким маршрутам.
На предварительном этапе шага~$t$ в~сети~$G(t)$ при заданных
значениях пропускной спо\-соб\-ности ребер~$d_k(t)$ для каждой \mbox{пары}
уз\-лов-кор\-рес\-пон\-ден\-тов $p_m\hm\in P(R^-)$ определяется максимально
допустимый однопродуктовый поток~$z^0_m(t)$, со\-от\-вет\-ст\-ву\-ющие
дуговые потоки $(x_{mk}^0(t),x_{m(k+E)}^0(t))$, $p_m\hm\in P(R^-)$, и~коэффициенты нормировки
$\xi_m^0(t)\hm={1}/{z_m^0(t)}$ для всех  $p_m\hm \in P(R^-)$,
таких что $z^0_m(t)\hm>0$, $y_m^0(t)\hm>0$.
Коэффициенты~$\xi_m^0(t)$ используются для поиска текущих
совместно допустимых квот на передачу потоков одновременно между
всеми парами $p_m\in P(R^-)$.

\smallskip

\noindent
\textbf{Задача 2.} Найти $\alpha^*(t)=\max\limits_\alpha \alpha$
при условиях
$$
\alpha\!\!\sum\limits_{m\in R^-}\! \xi_m^0\left(x_{mk}^0(t)+x_{m(k+E)}^0(t)\right)\le d_k(t),\enskip
k=\overline{1,E}\,.
$$

На основании $\alpha^*(t)$ вычисляются совместно допустимые
дуговые потоки:
\begin{multline*}
x_{mk}^*(t)=\alpha^*(t)\xi^0_m(t)x^0_{mk}(t),\\
x^*_{m(k+E)}(t)=\alpha^*(t)\xi^0_m(t)x^0_{m(k+E)}(t),
\\
m=\overline{1,M}\,,\enskip k=\overline{1,E}\,,
\end{multline*}
и остаточная пропускная способность ребер в~сети $G(t+1)$:
\begin{multline*}
d_k(t+1)=d_k(t)-\sum_{m\in R^-} (x_{mk}^*(t)+x_{m(k+E)}(t)),\\
k=\overline{1,E}\,,\enskip p_m\in P(R^-).
\end{multline*}
Формируется вектор допустимых межузловых потоков:
\begin{align*}
z_k^+(t)&=d_k(t+1),\enskip p_k\in P(R^+),\enskip k=\overline{1,E}\,;
\\
z_m^-(t)&=\sum\limits_{\tau=1}^t\alpha^*(\tau)\xi_m^0(\tau) z_m^0(\tau), \enskip p_m\in P(R^-).
\end{align*}

По построению, на шаге~$t$ при передаче вектора межузлового потока
$\mathbf{z}(t)=\{\mathbf{z}^+(t), \mathbf{z}^-(t)\}$ достигается
предельная загрузка, и~пропускная способность всех ребер  сети
используется полностью.

Точка с~координатами $\mathbf{z}(t)$ принадлежит множеству~$Z(d)$.

Расстояние точки от начала координат определяется как норма
соответствующего вектора:
\begin{align*}
\rho^+(t)&=\|\mathbf{z}^+(t)\|=
\left[\,\sum\limits_{k=1}(\mathbf{z}^+(t))^2\right]^{1/2};
\\
\rho^-(t)&=\|\mathbf{z}^-(t)\|= \left[\sum\limits_{p_m\in P(R^-)}(\mathbf{z}_m^-(t))^2\right]^{1/2}.
\end{align*}

Если при выполнении шага $(t+1)$ окажется, что $z_m^0(t+1)=0$ для
всех $p_m\in P(R^-)$, то про\-изойдет останов и~сформируются
массивы финальных данных:
\begin{align*}
z_m^-(T)&=\sum\limits_{\tau=1}^t \alpha^*(\tau)\xi_m^0(\tau) z_m^0(\tau),\enskip 
p_m\in P(R^-),\\
z_k^+(T)&=d_k(t+1),\enskip p_k\in P(R^+),\enskip k=\overline{1,E}\,.
\end{align*}

Вышеописанная вычислительная процедура далее обозначается как
MFPL-про\-це\-ду\-ра (от англ.\ \textit{max-flow-peak-load}).

При проведении второй серии вычислительных экспериментов
MFPL-про\-це\-ду\-ра использовалась для оценки функциональных
характеристик сис\-те\-мы при \textit{уравнительном} поэтапном
распределении пропускной способности между всеми
па\-ра\-ми-кор\-рес\-пон\-ден\-тами.

При реализации MFPL-процедуры выполнение каждого шага разбивается
на несколько этапов. На предварительном этапе шага~$t$ 
в~сети~$G(t)$ при заданных значениях пропускной способности ребер~$d_k(t)$ 
для каждой пары уз\-лов-кор\-рес\-пон\-ден\-тов $p_m\hm\in P(R^-)$
определяется максимально допустимый однопродуктовый
поток~$z_m^0(t)$, соответствующие дуговые потоки
$\left(x_{mk}^0(t),x_{m(k+E)}^0(t)\right)$, $p_m\hm\in P(R^-)$, и~суммарная
реберная нагрузка
$$
y_m^0(t)=\sum\limits_{k=1}^E (x_{mk}^0(t),x_{m(k+E)}^0(t)),\enskip p_m\in P(R^-).
$$

Для каждой пары $p_m\hm\in P(R^-)$ вычисляются коэффициенты
нормировки
$\theta_m^0(t)\hm={1}/{y_m^0(t)}$ для всех  
$p_m\hm\in P(R^-)$, таких что  $z^0_m(t)\hm>0$,
$y_m^0(t)\hm>0$.
Коэффициенты~$\theta_m^0(t)$ используются для поиска совместно
допустимых дуговых потоков для всех $p_m\hm\in P(R^-)$.

\smallskip

\noindent
\textbf{Задача 3.} Найти $\beta^*(t)=\max\nolimits_\beta \beta$ при
условиях
$$
\beta\!\!\!\!\sum\limits_{p_m\in P(R^-)}\!\!
\theta_m^0(x_{mk}^0(t)+x_{m(k+E)}^0(t))\le d_k(t),\enskip
k=\overline{1,E}\,.
$$

 С помощью $\beta^*(t)$ (решения задачи~3) вычисляются текущие допустимые значения дуговых потоков:
\begin{multline*}
x_{mk}^*(t)=\beta^*(t)\theta^0_m(t)x^0_{mk}(t),\\
x^*_{m(k+E)}(t)=\beta^*(t)\theta^0_m(t)x^0_{m(k+E)}(t), \enskip
k=\overline{1,E},
\end{multline*}
и реберных нагрузок при одновременной передаче межузловых потоков:

\noindent
\begin{multline*}
y_m^*(t)=\sum\limits_{i=1}^E
\left[x_{mi}^*(t)+x^*_{m(i+E)}(t)\right]={}\\
{}= \fr{\beta^*(t)}{y_m^0(t)} \sum\limits_{i=1}^E
\left[x_{mi}^0(t)+x^0_{m(i+E)}(t)\right]=\beta^*(t), \\
 p_m\in P(R^-).
\end{multline*}
Таким образом на каждом шаге определенная часть имеющегося ресурса
(пропускной спо\-соб\-ности) делится строго по\-ров\-ну меж\-ду всеми
корреспондентами $p_m\in P(R^-)$, для которых существует путь
передачи в~$G(t)$.

Формируется вектор допустимых межузловых потоков:
\begin{gather*}
\hspace*{-30mm}z_k^{++}(t)=d_k(t+1)={}\hspace*{10mm}\\
{}=d_k(t)-\!\!\! \sum\limits_{p_m\in P(R^-)}\!\!\!
\left(x_{mk}^*(t)+x_{m(k+E)}(t)\right),\\
\hspace*{35mm}k=\overline{1,E}, \enskip
p_k\in P(R^+);\\
z_m^{(=)}(t)\overset{\Delta}{=}\sum\limits_{\tau=1}^t\beta^*(\tau)
\theta_m^0(\tau) z_m^0(\tau), \enskip p_m\in P(R^-).
\end{gather*}

\noindent
Определяются расстояния:
\begin{align*}
\rho^{++}(t)&=\|\mathbf{z}^{++}(t)\|\overset{\Delta}{=}
\left[\sum\limits_{k=1}^E\left(d_k(t+1)\right)^2\right]^{1/2};\\
\rho^{(=)}(t)&=\|\mathbf{z}^{=}(t)\|= \left[\sum\limits_{p_m\in
P(R^-)}\left(z_m^{(=)}(t)\right)^2\right]^{1/2}.
\end{align*}

Если на предварительном этапе на шаге $(t+1)$ окажется, что в~сети~$G(t+1)$ для всех $p_m\hm\in P(R^-)$ все значения
$z_m^0(t+1)\hm=0$, то произойдет останов и~сформируются финальные
массивы:
\begin{align*}
z_k^{(++)}(T)&=d_k(t+1), \enskip
p_k\in P(R^+), \enskip k=\overline{1,E};
\\
z_m^{(=)}(t)&=\sum\limits_{\tau=1}^{t+1}\beta^*(\tau)
\theta_m^0(\tau) z_m^0(\tau), \enskip p_m\in P(R^-).
\end{align*}



\section{Вычислительный эксперимент}

Результаты вычислительных экспериментов, описанные ниже, служат
продолжением исследований, начатых в~[1]. Вычислительные
эксперименты проводились на моделях сетевых сис\-тем, пред\-став\-лен\-ных
на рис.~1 и~2. В~каждой сети~69~узлов. Пропускные спо\-соб\-но\-сти
ребер~-- значения $d_k$~-- выбирались случайным образом из отрезка
$[900,999]$ и~совпадали для ребер, при\-сут\-ст\-ву\-ющих в~обеих сетях.
В~кольцевой сети пропускная спо\-соб\-ность каждого из добавленных
ребер равнялась~900.

\begin{figure*} %fig1
\vspace*{1pt}
\begin{minipage}[t]{80mm}
  \begin{center}  
    \mbox{%
\epsfxsize=69.408mm
\epsfbox{mal-1.eps}
}

\end{center}
\vspace*{-6pt}
\Caption{Базовая сеть}
\end{minipage}
%\end{figure*}
\hfill
%\begin{figure*} %fig2
\vspace*{1pt}
\begin{minipage}[t]{80mm}
  \begin{center}  
    \mbox{%
\epsfxsize=69.408mm
\epsfbox{mal-2.eps}
}

\end{center}
\vspace*{-6pt}
\Caption{Кольцевая сеть}
\end{minipage}
\end{figure*}

\begin{table*}[b]\small %tabl1
\vspace*{-12pt}
\begin{center}

%\renewcommand{\arraystretch}{1.1}
\Caption{Базовая сеть}
\vspace*{2ex}

\begin{tabular}{|c||c|c|c||c|c|c|} 
\hline
&&&&&&\\[-9pt]
$t$  & $\rho^{-}(t)$ & $\rho^{+}(t)$ & $d^{+}(t+1)$ &
$\rho^{=}(t)$ & $\rho^{++}(t)$&  $d^{++}(t+1)$ \\ 
\hline
\hphantom{99}0  & \hphantom{99}0   & 8048&  68256&  \hphantom{9}0   &  8048&   68256\\
1  & \hphantom{9}63  & 4182&  26544&  \hphantom{9}95  &  3881&   24476\\
$\cdots$  & $\cdots$   & $\cdots$   &  $\cdots$    &  $\cdots$   &  $\cdots$   &   $\cdots$\\
11 & \hphantom{9}70  & 3975&  21469&  \hphantom{9}101\hphantom{9} &  3707&   20155\\
$\cdots$& $\cdots$   & $\cdots$   &  $\cdots$    & $\cdots$   &  $\cdots$   &  $\cdots$\\
22 & \hphantom{9}83  & 3861&  19623&  \hphantom{9}122\hphantom{9} &  3586&   18260\\
$\cdots$ & $\cdots$  & $\cdots$   &  $\cdots$   &  $\cdots$   &  $\cdots$  &   $\cdots$\\
33 & \hphantom{9}103\hphantom{9} & 3778&  18827&  \hphantom{9}139\hphantom{9} &  3522&   17601\\
$\cdots$ &$\cdots$  &$\cdots$  & $\cdots$  & $\cdots$   &  $\cdots$  &  $\cdots$\\
44 & \hphantom{9}\bf 190\hphantom{9} & \bf3553&  \bf17503&  \hphantom{9}\bf203\hphantom{9} &  \bf3285&   \bf16201\\
45 & \hphantom{9}\bf1452\hphantom{99}& \bf2166&  \hphantom{9}\bf7069 &  \hphantom{9}\bf1376\hphantom{99}&  \bf2020&   \hphantom{9}\bf6584\\
46 & \hphantom{9}\bf1498\hphantom{99}& \bf2158&  \hphantom{9}\bf6707 &  \hphantom{9}\bf1388\hphantom{99}&  \bf2017&   \hphantom{9}\bf6483\\
$\cdots$ & $\cdots$   & $\cdots$   &  $\cdots$    & $\cdots$   &  $\cdots$   &  $\cdots$\\
52 & \hphantom{9}1535\hphantom{99}& 2155&  \hphantom{9}6413 & \hphantom{9}1442\hphantom{99} &  2011&   \hphantom{9}6059\\
\hline
\end{tabular}
\end{center}
 %\end{table*}
% \begin{table*}\small %tabl2
\begin{center}
\Caption{Кольцевая сеть}
\vspace*{2ex}


\begin{tabular}{|c||c|c|c||c|c|c|} 
\hline
&&&&&&\\[-9pt]
$t$  & $\rho^{-}(t)$ & $\rho^{+}(t)$ & $d^{+}(t+1)$ &
$\rho^{=}(t)$ & $\rho^{++}(t)$&  $d^{++}(t+1)$ \\
 \hline
\hphantom{9}0  &\hphantom{99}0    & 8440  & 75456   &\hphantom{9}0      &8440   &75456\\
\hphantom{9}1  &\hphantom{9}68   & 5317  & 43038   &92     &5045   &40716 \\ 
$\cdots$ &$\cdots$    & $\cdots$     & $\cdots$   &$\cdots$      &$\cdots$      &$\cdots$      \\
11 &\hphantom{9}95   & 3608  & 20459   &124    &3397   &19080  \\
$\cdots$ &$\cdots$   & $\cdots$    & $\cdots$      &$\cdots$     &$\cdots$     &$\cdots$   \\
22 &\hphantom{9}101\hphantom{9}  & 3540  & 19530   &130    &3350   &18338 \\
$\cdots$ &$\cdots$  & $\cdots$   &$\cdots$      &$\cdots$     &$\cdots$   &$\cdots$    \\
33 &\hphantom{9}135\hphantom{9}  & 3346  & 17561   &154    &3220   &17003 \\
$\cdots$  &$\cdots$   & $\cdots$    & $\cdots$      &$\cdots$     &$\cdots$    &$\cdots$    \\
44 &\hphantom{9}234\hphantom{9}  & 3094  & 14881   &269    &2918   &13848 \\
$\cdots$ &$\cdots$   & $\cdots$    &$\cdots$      &$\cdots$     &$\cdots$     &$\cdots$    \\
50 &\hphantom{9}\bf 413\hphantom{9}  & \bf2770  & \bf12901   &\bf329    &\bf2792   &\bf13079 \\
51 &\hphantom{9}\bf1040\hphantom{99} & \bf2299  & \hphantom{9}\bf8801    &\bf334    &\bf2784   &\bf13034 \\
52 &\hphantom{9}\bf1062\hphantom{99} & \bf2297  & \hphantom{9}\bf8672    &\bf974    &\bf2262   &\hphantom{9}\bf8768  \\
$\cdots$ &$\cdots$   &$\cdots$    & $\cdots$      &$\cdots$      &$\cdots$     &$\cdots$    \\
55 &\hphantom{9}1069\hphantom{99} & 2297  & \hphantom{9}8630    &1010\hphantom{9}   &2259   &\hphantom{9}8553  \\
\hline
 \end{tabular}
\end{center}
 \end{table*}




Для базовой сети исходная сумма пропускных способностей:
$D^+(0)\hm=68\,256$, а~для кольцевой сети $D^{++}(0)=75\,456$.
Соответствующие значения $\rho^+(0)$ и~$\rho^{++}(0)$ указаны в~<<нулевой>> строке 
в~табл.~1 и~2, где собраны результаты
вычислительных экспериментов. В~ходе эксперимента при
уравнительном распределении остаточных ресурсов соблюдается
\textit{равномерное} убывание остаточной пропускной спо\-соб\-ности и~<<\textit{длины}>> вектора~$\rho^+(t)$. 
Однако между 44--46
итерациями для базовой и~50--52 для кольцевой сети наблюдается
резкий скачок величин~$\rho^-(t)$, $\rho^{=}(t)$ и~$d^+(t)$,
$d^{++}(t)$.

На указанных шагах полностью используется пропускная способность
ребер в~центральной час\-ти сети. Сеть \textit{распадается} на
несвязные компоненты, и~для $80\%$ корреспондентов пропадают пути
соединения, а~остаточный ресурс распределяется поровну между
оставшимися парами узлов.

Анализ результатов показал, что почти равные значения потоков
достигаются для~80\% корреспондентов и~требуют 60\%--70\%
ресурсов. Однако для~2\% смежных  пар узлов межузловые потоки на
два порядка выше медианных значений, а~затраты пропускной
способности  со\-став\-ля\-ют~20\%--30\%.








\section{Заключение}

Предложенный метод и~проведенные вычислительные эксперименты
показали, что уравнительное поэтапное распределение   приводит 
к~неравномерному  распределению   потоков  для разных групп\linebreak
корреспондентов.    Метрические оценки, полученные  в~ходе
экспериментов, продемонстрировали\linebreak \textit{деформацию} множества
достижимых потоков. В~рамках модели   предполагалось, что  все
корреспонденты  равноправны, а~потоки невзаимозаменяемы,  однако
при уравнительном предельном  распределении  смежные  пары узлов
оказывались в~привилегированном положении при использовании
остаточной пропускной способности. Пропускные способности  ребер
рассматривались  как вектор   ресурсов  различных типов,  которые
распределяются между корреспондентами   при передаче  потоков
различных видов.  По построению, на каж\-дом шаге норма вектора
смежных   межузловых    потоков численно равна   модулю вектора
остаточных  пропускных способностей.   Полученные мет\-ри\-че\-ские
значения  можно использовать  для   оценки функциональных
возможностей сети  в~режиме  предельной загрузки.

{\small\frenchspacing
 {%\baselineskip=10.8pt
 %\addcontentsline{toc}{section}{References}
 \begin{thebibliography}{9}

\bibitem{1-mal}
\Au{Малашенко Ю.\,Е., Назарова И.\,А.} Неоднородность
распределения   потоков при предельной  загрузке
многопользовательской сети~//  Известия РАН. Теория и~сис\-те\-мы
управления,  2022. №\,3. С.~81--96.

\bibitem{4-mal} %2
\Au{Luss H.} Equitable resource allocation: Models,
algorithms, and applications.~--- Hoboken, NJ, USA: John Wiley \& Sons, 2012.
420~p.

\bibitem{2-mal} %3
\Au{Ogryczak W., Luss~H., Pioro~M., Nace~D., Tomaszewski~A.}   Fair
optimization and networks: A~aurvey~// J.~Appl. Math., 2014. Vol.~2014. Art.~ID~612018. 25~p. doi: 10.1155/ 2014/612018.

\bibitem{3-mal} %4
\Au{Salimifard K., Bigharaz~S.} The multicommodity network
flow problem: State of the art classification, applications, and
solution methods~// J.~Oper. Res., 2020. Vol.~18. Iss.~3. P.~1--47.



\bibitem{5-mal}
\Au{Balakrishnan A., Li~G., Mirchandani~P.}  Optimal
network design with end-to-end service requirements~// Oper. Res.,
2017. Vol.~65. Iss.~3. P.~729--750.

\bibitem{6-mal}
\Au{Nace D., Doan~L.\,N., Klopfenstein~O., Bashllari~A.} Max-min
fairness in multicommodity flows~// Comput. Oper. Res., 2008.
Vol.~35. Iss.~2. P.~557--573.

\bibitem{7-mal}
\Au{Ros-Giralt J., Tsai~W.\,K.} A~lexicographic optimization
framework to the flow control problem~// IEEE T.
Inform. Theory, 2010. Vol.~56. Iss.~6. P.~2875--2886.

\bibitem{8-mal}
\Au{Baier G., Kohler~E., Skutella~M.}  The \mbox{k-splittable}
flow problem~//  Algorithmica, 2005. Vol.~42. Iss.~3-4.
P.~231--248.

\bibitem{9-mal}
\Au{Bialon P.\,A.} Randomized rounding approach to 
a~\mbox{k-splittable} multicommodity flow problem with lower path flow
bounds affording solution quality guarantees~// Telecommun. Syst.,
2017. Vol.~64. Iss.~3. P.~525--542.
\end{thebibliography}

 }
 }

\end{multicols}

\vspace*{-6pt}

\hfill{\small\textit{Поступила в~редакцию 10.06.22}}

\vspace*{8pt}

%\pagebreak

%\newpage

%\vspace*{-28pt}

\hrule

\vspace*{2pt}

\hrule

%\vspace*{-2pt}

\def\tit{SEQUENTIAL ANALYSIS AND METRIC ESTIMATES\\ OF~PEAK LOAD FLOWS IN~THE~MULTIUSER NETWORK}


\def\titkol{Sequential analysis and metric estimates of~peak load flows in~the~multiuser network}


\def\aut{Yu.\,E.~Malashenko}

\def\autkol{Yu.\,E.~Malashenko}

\titel{\tit}{\aut}{\autkol}{\titkol}

\vspace*{-8pt}


\noindent
Federal Research Center ``Computer Science and Control'' of the Russian Academy of Sciences, 
44-2~Vavilov Str., Moscow 119333, Russian Federation



\def\leftfootline{\small{\textbf{\thepage}
\hfill INFORMATIKA I EE PRIMENENIYA~--- INFORMATICS AND
APPLICATIONS\ \ \ 2022\ \ \ volume~16\ \ \ issue\ 3}
}%
 \def\rightfootline{\small{INFORMATIKA I EE PRIMENENIYA~---
INFORMATICS AND APPLICATIONS\ \ \ 2022\ \ \ volume~16\ \ \ issue\ 3
\hfill \textbf{\thepage}}}

\vspace*{3pt} 



\Abste{The set of vectors of internodal flows in a~multiuser communication network under peak load is analyzed. Within the framework of
 the multicommodity model, the throughput capacities of edges are considered as the components of a~vector of resources of various types that 
 are required for the transmission of various kinds of
 flows. When conducting computational experiments, at each iteration, the
  norms of vectors of jointly permissible internodal flows are calculated, during the transmission of which the capacity of 
  all network edges is fully used.\linebreak\vspace*{-12pt}}
 
 \Abstend{The proposed method and computational experiments have shown that the equalizing phased 
  distribution leads to an uneven distribution of flows for different groups of correspondents. Metric values obtained during experiments 
  indicate deformation of the sets of accessible flows. Within the framework of the model, all correspondents are tantamount 
  and the flows are noninterchangeable; however, in the case of an equalizing peak load distribution, adjacent pairs 
  of nodes are in a privileged position when using residual capacity. The obtained metric values can be used to 
  evaluate the functional characteristics of the transmission network in the finite capacity loading mode.}

\KWE{multicommodity flow network model; set of achievable internodal flows; peak load distribution}


\DOI{10.14357/19922264220306} 

%\vspace*{-16pt}

%\Ack
%\noindent



%\vspace*{4pt}

  \begin{multicols}{2}

\renewcommand{\bibname}{\protect\rmfamily References}
%\renewcommand{\bibname}{\large\protect\rm References}

{\small\frenchspacing
 {%\baselineskip=10.8pt
 \addcontentsline{toc}{section}{References}
 \begin{thebibliography}{9}
\bibitem{1-mal-1}
\Aue{Malashenko, Yu.\,E., and I.\,A.~Nazarova.}
2022. Heterogeneous flow distribution at the peak load in the multiuser network. \textit{J.~Comput. Sys. Sc. Int.} 61:372--387.

\bibitem{4-mal-1} %2
\Aue{Luss, H.} 2012. \textit{Equitable resource allocation: Models, algorithms, and applications}.
Hoboken, NJ: John Wiley \& Sons. 420~p.

\bibitem{2-mal-1} %3
\Aue{Ogryczak, W., H.~Luss, M.~Pioro, D.~Nace, and A.~Tomaszewski.}
 2014. Fair optimization and networks: A~survey. \textit{J.~Appl. Math.} 2014:612018. 25~p. doi: 10.1155/ 2014/612018.
\bibitem{3-mal-1} %4
\Aue{Salimifard, K., and S.~Bigharaz.}
 2020. The multicommodity network flow problem: State of the art classification, applications, and solution methods. 
 \textit{J.~Oper. Res.} 18(3):\linebreak 1--47.

\bibitem{5-mal-1}
\Aue{Balakrishnan, A., G.~Li, and P.~Mirchandani.} 2017. Optimal network design with end-to-end service requirements. 
\textit{Oper. Res.} 65(3):729--750.
\bibitem{6-mal-1}
\Aue{Nace, D., L.\,N.~Doan, O.~Klopfenstein, and A.~Bashllari.} 2008. Max-min fairness in multicommodity flows. 
\textit{Comput. Oper. Res.} 35(2):557--573.
\bibitem{7-mal-1}
\Aue{Ros-Giralt, J., and W.\,K.~Tsai.} 2010. A~lexicographic optimization framework to the flow control problem. 
\textit{IEEE T.~Inform. Theory} 56(6):2875--2886.
\bibitem{8-mal-1}
\Aue{Baier, G., E.~Kohler, and M.~Skutella.}
 2005. The k-splittable flow problem. \textit{Algorithmica} 42(3-4):231--248.
\bibitem{9-mal-1}
\Aue{Bialon, P.} 2017. A~randomized rounding approach to a~\mbox{k-splittable} multicommodity flow problem with lower path flow bounds affording solution quality guarantees. 
\textit{Telecommun. Syst.} 64(3):525--542.
 \end{thebibliography}

 }
 }

\end{multicols}

\vspace*{-6pt}

\hfill{\small\textit{Received June 10, 2022}}

\Contrl

\noindent
\textbf{Malashenko Yuri E.} (b.\ 1946)~--- 
Doctor of Science in physics and mathematics, principal scientist, Federal Research Center ``Computer Science and Control'' 
of the Russian Academy of Sciences, 44-2~Vavilov Str., Moscow 119333, Russian Federation; \mbox{malash09@ccas.ru} 


\label{end\stat}

\renewcommand{\bibname}{\protect\rm Литература}   %7
\def\stat{dukova}

\def\tit{О ПОИСКЕ МАКСИМАЛЬНЫХ ЧАСТЫХ И~МИНИМАЛЬНЫХ НЕЧАСТЫХ НАБОРОВ ПРОИЗВЕДЕНИЯ ЧАСТИЧНЫХ ПОРЯДКОВ}

\def\titkol{О поиске максимальных частых и~минимальных нечастых наборов произведения частичных порядков}

\def\aut{Н.\,А.~Драгунов$^1$, Е.\,В.~Дюкова$^2$}

\def\autkol{Н.\,А.~Драгунов, Е.\,В.~Дюкова}

\titel{\tit}{\aut}{\autkol}{\titkol}

\index{Драгунов Н.\,А.}
\index{Дюкова Е.\,В.}
\index{Dragunov N.\,A.}
\index{Djukova E.\,V.}


%{\renewcommand{\thefootnote}{\fnsymbol{footnote}} \footnotetext[1]
%{Работа выполнена при поддержке Министерства науки и~высшего образования Российской Федерации (проект 
%075-15-2020-799).}}


\renewcommand{\thefootnote}{\arabic{footnote}}
\footnotetext[1]{Федеральный исследовательский центр <<Информатика 
и~управ\-ле\-ние>> Российской академии наук, \mbox{nikitadragunovjob@gmail.com}}
\footnotetext[2]{Федеральный исследовательский центр <<Информатика и~управ\-ле\-ние>> 
Российской академии наук, \mbox{edjukova@mail.ru}}

\vspace*{-3pt}




\Abst{Исследованы актуальные вопросы снижения временных затрат, возникающие при 
логическом анализе данных с~элементами из декартова произведения конечных час\-тич\-но 
упорядоченных множеств. Для задачи поиска по базе транзакций максимальных час\-тых и~минимальных 
нечастых наборов произведения час\-тич\-ных порядков предложен оригинальный метод, 
основанный на решении слож\-ной дискретной задачи, называемой дуализацией 
над произведением час\-тич\-ных порядков. Метод представляет собой синтез двух других 
известных методов, один из которых достаточно очевиден, а~другой использует идею 
инкрементального пе\-ре\-чис\-ле\-ния искомых наборов и~поэтому пред\-став\-ля\-ет 
в~основном тео\-ре\-ти\-че\-ский интерес. Проведено экспериментальное исследование предложенного 
подхода к~решению рас\-смат\-ри\-ва\-емой задачи в~случае произведения конечных цепей,
 выявлены условия его эф\-фек\-тив\-ности и~для проводимого анализа данных показана 
 це\-ле\-со\-об\-раз\-ность применения асимптотически оптимальных алгоритмов дуализации 
 над произведением час\-тич\-ных порядков.}

\KW{максимальные час\-тые наборы; минимальные не\-час\-тые наборы; дуализация над 
произведением час\-тич\-ных порядков; асимп\-то\-ти\-чески оптимальный алгоритм дуализации}

\DOI{10.14357/19922264220112}
  
%\vspace*{-4pt}


\vskip 10pt plus 9pt minus 6pt

\thispagestyle{headings}

\begin{multicols}{2}

\label{st\stat}

    \section{Введение}
    
    Рас\-смат\-ри\-ва\-емая задача анализа данных занимает важ\-ное мес\-то в~об\-ласти 
    информационного поиска и~в~случае бинарных данных ставится сле\-ду\-ющим образом~\cite{4}.
    
    Дано некоторое множество элементов~$V$. Подмножества $X \hm\subseteq V$ называются наборами. Пусть~$D$~--- 
    база данных, содержащая некоторые, не обязательно различные, наборы. Наборы, 
    содержащиеся в~$D$, называются транз\-ак\-ци\-ями. Под частотой набора~$\nu(X)$ понимается доля транз\-ак\-ций в~$D$, 
    содержащих~$X$. Если $\nu(X) \hm\geq s$, $s \hm\in \left[0, 1\right]$, то набор~$X$ называется $s$-час\-тым, 
    иначе он называется $s$-не\-час\-тым. Если набор частый и~он не содержится ни в~каком другом 
    час\-том наборе, то такой набор называется максимальным час\-тым. Если набор не\-час\-тый 
    и~при этом он не содержит в~себе никакого другого не\-час\-то\-го набора, то такой набор 
    называется минимальным нечастым. Требуется найти все максимальные час\-тые и~минимальные не\-час\-тые 
    наборы при заданном~$s$.
    
    Рас\-смат\-ри\-ва\-емая задача имеет много важных приложений, одним из которых является 
    нахождение ассоциативных правил в~базах данных. В~случае бинарных данных ассоциативное правило~---
     это упорядоченная пара $ \left( X, Y \right)$ непересекающихся подмножеств множества~$V$, обо\-зна\-ча\-емая 
     $X \hm\Rightarrow Y$. Поддержкой правила $X \hm\Rightarrow Y$ называется час\-то\-та набора $Z\hm = X \cup Y$.
      Достоверностью правила $X\hm \Rightarrow Y$ называется доля транзакций, со\-дер\-жа\-щих~$Y$, 
      среди всех транзакций, содержащих~$X$. Требуется \mbox{найти} все ассоциативные правила, 
      удовле\-тво\-ря\-ющие заданным минимальной поддержке $s\hm \in [0, 1]$ и~минимальной 
      достоверности $c \hm\in [0, 1]$.  Впервые задача нахождения ассоциативных правил
       была поставлена в~\cite{1}, где она формулировалась как задача анализа по\-тре\-би\-тель\-ской корзины.

    В случае небинарных данных каждый элемент из~$V$ имеет некоторое множество чис\-ло\-вых значений 
    и~вместо наборов элементов рас\-смат\-ри\-ва\-ют\-ся наборы их значений.

    Поиск ассоциативных правил осуществляется в~два этапа. 
    На первом этапе находятся частые наборы, на втором этапе из найденных час\-тых 
    наборов формируются ассоциативные правила. При формировании правил на втором 
    этапе фактически возникает задача поиска $t$-не\-час\-тых наборов, где $t\hm > s/c$.
    
    С ростом размерности современных баз данных находить все час\-тые и~не\-час\-тые 
    наборы становится неэффективно как по времени, так и~по памяти в~силу 
    экспоненциального рос\-та чис\-ла таких наборов. Одно из решений данной проблемы 
    заключается в~поиске только максимальных час\-тых наборов и~только минимальных 
    нечастых наборов, что позволяет компактно хранить информацию о~всех час\-тых и~не\-час\-тых 
    наборах соответственно. 
    
    
    В~\cite{9} рас\-смот\-ре\-на задача поиска множеств максимальных час\-тых наборов~$X_{\max}$ 
    и~минимальных не\-час\-тых наборов~$Y_{\min}$ в~данных, пред\-став\-лен\-ных в~виде декартова 
    произведения час\-тич\-но упорядоченных множеств. Показано, что в~этом случае 
    при построении тре\-бу\-емых наборов возникают соответственно задача поиска 
    максимальных независимых элементов час\-тич\-ных порядков и~задача поиска минимальных 
    независимых элементов час\-тич\-ных порядков.  Каж\-дая из этих задач называется 
    дуализацией над произведением час\-тич\-ных порядков~\cite{8}. Обе задачи относятся к~одним 
    из цент\-раль\-ных труд\-но\-ре\-ша\-емых пе\-ре\-чис\-ли\-тель\-ных задач дис\-крет\-ной математики.
    
    Существует достаточно очевидный способ поиска максимальных час\-тых и~минимальных
     не\-час\-тых наборов произведения час\-тич\-ных порядков, основанный на по\-сле\-до\-ва\-тель\-ном 
     по\-стро\-ении указанных множеств. Одно из множеств ищется, например, алгоритмом Apriori~\cite{2},
      второе множество получается путем дуализации первого. 
      В~настоящей работе показано, что метод эффективен только в~случае, когда чис\-ло час\-тых 
      наборов существенно меньше или, наоборот, существенно больше чис\-ла не\-час\-тых наборов. 
      В~\cite{9} предложена идея со\-вмест\-но\-го пе\-ре\-чис\-ле\-ния~$X_{\max}$ и~$Y_{\min}$ с~использованием
       инкрементального алгоритма дуализации из~\cite{14}, которая автором экспериментально 
       не исследована.
    
    Основной результат настоящей работы~--- разработка нового подхода к~решению 
    поставленной задачи, который является синтезом последовательного и~совместного подходов. 
    
    Экспериментальные исследования, проведенные в~настоящей работе для случая
     произведения цепей, свидетельствуют о~том, что предложенный по\-сле\-до\-ва\-тель\-но-со\-вмест\-ный 
     метод наиболее эффективен в~случае, когда мощ\-ность множества час\-тых наборов примерно 
     равна мощ\-ности множества не\-час\-тых наборов.
     
     \vspace*{-6pt}
     
    
    \section{Постановка задачи поиска максимальных частых 
    и~минимальных нечастых наборов произведения частичных порядков}
    
         \vspace*{-2pt}
    
    Пусть $\mathcal{P} = \mathcal{P}_1 \times \dots \times \mathcal{P}_n$~--- 
    де\-кар\-то\-во произведение час\-тич\-но упорядоченных множеств. Элементы~$\mathcal{P}$ называются наборами. 
    На множестве~$\mathcal{P}$ определяется отношение частичного порядка~$\preceq$ сле\-ду\-ющим образом: 
    если $p \hm= (p_1, \dots, p_n) \hm\in \mathcal{P}$ и~$q \hm= (q_1, \dots, q_n)\hm \in \mathcal{P}$, 
    то $ p \hm\preceq q$ в~$ \mathcal{P}\hm \Leftrightarrow p_1 \hm\preceq q_1$ 
    в~$\mathcal{P}_1, \dots, p_n \hm\preceq q_n$ в~$ \mathcal{P}_n$.
    
    Пусть $\mathcal{D} (\mathcal{P})$~--- некоторая со\-во\-куп\-ность
     наборов из~$\mathcal{P}$, называемая базой данных. Наборы, на\-хо\-дя\-щи\-еся в~базе 
     данных $\mathcal{D} (\mathcal{P})$, необязательно по\-пар\-но раз\-лич\-ны и~называются транзакциями. 
     
    Введем обозначения: 
    $\vert \mathcal{D} (\mathcal{P}) \vert$~--- чис\-ло транз\-ак\-ций в~$\mathcal{D} (\mathcal{P})$; 
    $\mathcal{S}_\mathcal{D}(p)$~--- число транз\-ак\-ций в~$\mathcal{D} (\mathcal{P})$, 
    сле\-ду\-ющих за $p \hm\in \mathcal{P}$; $s \hm\in [0, 1]$. 
    
    \smallskip
    
    \noindent
    \textbf{Определение~1.}\
     Набор $p \in \mathcal{P}$ называется $s$-час\-тым, 
     если $\mathcal{S}_\mathcal{D}(p) / \vert \mathcal{D} (\mathcal{P}) \vert \hm\geq s$. Иначе набор~$p$ 
     называется $s$-не\-час\-тым.
    
    \smallskip
    
    \noindent
    \textbf{Определение~2.}\
    Набор $p \in \mathcal{P}$ называется максимальным $s$-час\-тым, если 
    он $s$-час\-тый и~никакой сле\-ду\-ющий за ним набор~$z$, $z\hm \neq p$, не является $s$-час\-тым.

    
    \smallskip
    
    \noindent
    \textbf{Определение~3.}\
    Набор $p \in \mathcal{P}$ называется минимальным $s$-не\-час\-тым, если он $s$-не\-час\-тый 
    и~никакой пред\-шест\-ву\-ющий ему набор~$z$, $z \hm\neq p$, не является $s$-не\-час\-тым.


\smallskip
    
    Далее вместо $s$-частый ($s$-не\-час\-тый) набор будем писать час\-тый (не\-час\-тый) набор. 
    Множество всех максимальных час\-тых наборов будем обозначать как $X_{\max}$, 
    а~множество всех минимальных не\-час\-тых наборов как $Y_{\min}$.
    
    Пусть $R \subset \mathcal{P}$, $R^+\hm = \{ x \in \mathcal{P} \vert \exists\, a \hm\in R, a \hm\preceq x \}$, 
    $R^- \hm= \{ x \hm\in \mathcal{P} \vert \exists\, a \hm\in R, x \hm\preceq a \}$.


    \noindent
    \textbf{Определение~4.}\
     Множество $I(R^+)$, со\-сто\-ящее из всех максимальных элементов множества~$\mathcal{P} \setminus R^+$, 
     называется максимальным независимым от~$R$.

\smallskip


   \noindent
    \textbf{Определение~5.}\
     Множество $I(R^-)$, со\-сто\-ящее из всех минимальных элементов множества~$\mathcal{P} \setminus R^-$, 
     называется минимальным независимым от~$R$.

\smallskip
    
    Каждая из задач построения $I(R^+)$ и~$I(R^-)$ 
    при заданном множестве~$R$ называется задачей дуализации над произведением час\-тич\-ных порядков.
    
    \smallskip

    \noindent
    \textbf{Утверждение~1.}\
    Если $X \hm\subset X_{\max}$, а~$y \hm\in I(X^-)$~--- не\-час\-тый набор, 
    то~$y$~--- минимальный не\-час\-тый набор.

\smallskip    
    
    \noindent
    Д\,о\,к\,а\,з\,а\,т\,е\,л\,ь\,с\,т\,в\,о\,.\  \ 
    Пусть $y \hm\notin I(X_{\max}^-)$. Так как~$y$~--- 
    нечастый набор, то в~$\mathcal{P} \setminus X^{-}_{\max}$ найдется минимальный не\-час\-тый набор~$x$ 
    такой, что $x\hm \neq y$ и~$x \hm\preceq y$. Из того, что $\mathcal{P} \setminus X^{-}_{\max} 
    \hm\subseteq \mathcal{P} \setminus X^-$, следует, что $x\hm \in \mathcal{P} \setminus X^-$, 
    что противоречит условию $y \hm\in I(X^-)$.

\smallskip

\noindent
\textbf{Утверждение~2.}\
    Пусть $X \hm\subseteq X_{\max}$, $Y\hm \subseteq Y_{\min}$. 
    Тогда $I(X^-) \hm= Y$ в~том и~только в~том случае, когда $X \hm= X_{\max}$ и~$Y \hm= Y_{\min}$.


\smallskip


  \noindent
    Д\,о\,к\,а\,з\,а\,т\,е\,л\,ь\,с\,т\,в\,о\,.\  \
    Пусть $X\! \subset\! X_{\max}, x \hm\in X_{\max}\!\setminus\!X$.
     Так как множество~$X_{\max}$~--- антицепь, то $x \hm\notin X^-$. 
     Следовательно, $x \hm\in \mathcal{P} \setminus X^{-}$.
      Но тогда существует элемент $ q \hm\in I(X^-) : q \preceq x$, 
      который является час\-тым. Однако во множестве~$Y$ частых наборов нет; следовательно, $I(X^-) \hm\neq Y$. 
      Если же $X \hm= X_{\max}$, то $I(X^-) \hm= Y_{\min}$. Таким образом, $I(X^-) \hm= Y$ тогда и~только
       тогда, когда $X \hm= X_{\max}$ и~$Y\hm = Y_{\min}$.


    
    \section{Методы построения множеств~$X_{\max}$ и~$Y_{\min}$}

    \subsection{Последовательное перечисление $X_{\max}$~и~$Y_{\min}$}

    Достаточно очевиден поиск~$X_{\max}$ и~$Y_{\min}$ при заданной $\mathcal{D} (\mathcal{P})$ 
    путем последовательного по\-стро\-ения множеств~$X_{\max}$ и~$Y_{\min}$. 
    Данный поиск осуществляется в~два этапа. На первом этапе находятся все максимальные частые 
    наборы~$X_{\max}$, например алгоритмом Apriori~\cite{2}. На втором этапе  используется свойство 
    двойственности $I \left(X_{\max}^- \right)\hm = Y_{\min}$. 
    Минимальные нечастые наборы~$Y_{\min}$ находятся путем дуализации найденного на первом этапе 
    множества~$X_{\max}$. Аналогично можно сначала искать~$Y_{\min}$ алгоритмом Apriori, а~затем 
    искать~$X_{\max}$ путем дуализации~$Y_{\min}$.

    Очевидно, что данный подход будет проявлять себя наилучшим образом в~случаях, когда 
    алгоритм Apriori или его модификации могут найти одно из искомых множеств существенно
     быст\-рее, чем другое множество, например когда мощ\-ность~$X_{\max}$ 
     существенно меньше (больше) мощ\-ности~$Y_{\min}$.
    
    \subsection{Совместное перечисление $X_{\max}$ и~$Y_{\min}$}

    В~\cite{9} предложена идея совместного перечисления множеств~$X_{\max}$ и~$Y_{\min}$. 
    На первом шаге рас\-смат\-ри\-ва\-ет\-ся некоторый случайный набор $q \hm\in \mathcal{P}$. Если $q$~--- 
    час\-тый набор, то ищется максимальный час\-тый набор, сле\-ду\-ющий за~$q$, 
    который пополняет множество $X \hm\subseteq X_{\max}$. Если $q$~---
     не\-час\-тый набор, то ищется минимальный не\-час\-тый набор, пред\-шест\-ву\-ющий~$q$, 
     который пополняет множество $Y \hm\subseteq Y_{\min}$. Пусть на шаге~$i$ ($i\hm \geq 1$) 
     построены множества $X \hm\subseteq X_{\max}$ и~$Y \hm\subseteq Y_{\min}$. Если $X \hm\neq \varnothing$, 
     $Y \hm= \varnothing$, то ищется набор~$q$ такой, что $q \hm\npreceq x, \forall x \hm\in X$. Если 
     $X \hm= \varnothing$, $Y \hm\neq \varnothing$, то ищется набор~$q$ такой, что 
     $q \hm\nsucceq y, \forall y \hm\in Y$. Если же и~$X \hm\neq \varnothing$, и~$Y \hm\neq \varnothing$, 
     то ищется набор~$q$ такой, что $q \hm\npreceq x, \forall x \hm\in X, q \hm\nsucceq y, \forall y \hm\in Y$.
      Затем, аналогично первому шагу, находится максимальный частый или минимальный нечастый набор. 
      Однако в~\cite{9} идея совместного перечисления искомых множеств экспериментально 
      не исследована и~не предложены конкретные указания по воз\-мож\-ной ее реализации.
    
    Алгоритм, основанный на совместном пе\-ре\-чис\-ле\-нии множеств~$X_{\max}$ и~$Y_{\min}$,
     реализован в~на\-сто\-ящей работе. Алгоритм строит две последовательности: $X_1 \hm\subset X_2 
     \subset \dots \subset X_{\max}$, $Y_1\hm \subset Y_2 \subset \dots \subset Y_{\min}$. 
     На первом шаге $X_1 \hm= \{x\}$, $Y_1 \hm= \{y\}$, где~$x$ и~$y$ ищутся алгоритмом Apriori.
      На шаге $i \hm+ 1$ ($i\hm \geq 1$) строится либо~$I(X^{-}_{i})$, либо~$I(Y^{+}_{i})$. Пусть на 
      шаге $i \hm+ 1$ ($i \hm\geq 1$) построено множество~$I(X^{-}_{i})$. 
      Согласно утверждениям~1 и~2, множество~$I(X^{-}_{i})$ либо не содержит час\-тых наборов 
      и~совпадает с~множеством~$Y_{\min}$ (в~этом случае $X_i \hm= X_{\max}$ 
      и~алгоритм заканчивает работу), либо~$I(X^{-}_{i})$ содержит как час\-тые, так и~не\-час\-тые наборы. 
      Каждый нечастый набор из~$I(X^{-}_{i})$ является минимальным не\-час\-тым и~пополняет множество~$Y_{i}$, 
      формируя в~результате множество~$Y_{i+1}$. Для каждого час\-то\-го набора находится один содержащий 
      его максимальный час\-тый набор путем последовательного увеличения текущего 
      частого набора в~лексикографическом порядке, который пополняет множество~$X_{i}$, 
      формируя в~результате множество~$X_{i+1}$.
      
    В~экспериментальной части работы (см.\ разд.~4) рас\-смот\-рен случай произведения цепей. 
    Задача дуализации решается с~помощью асимптотически оптимального алгоритма дуализации
     цепей \mbox{RUNC-M}+~\cite{7}. Асимптотически оптимальные алгоритмы дуализации 
     являются лидерами по ско\-рости счета~\cite{6}.

    Очевидно, что время работы совместного алгоритма в~основном зависит от чис\-ла
     минимальных не\-час\-тых и~максимальных час\-тых наборов. На\linebreak каж\-дой новой 
     итерации происходит дуализация\linebreak все б$\acute{\mbox{о}}$льших по мощ\-ности множеств~$X$ или~$Y$.\linebreak 
     Если число итераций становится достаточно\linebreak большим, то ско\-рость работы совместного 
     перечисления существенно снижается, что делает его практически неприменимым для 
     задач большой раз\-мер\-ности.
     { %\looseness=1
     
     }

    \subsection{Последовательно-совместное перечисление~$X_{\max}$ и~$Y_{\min}$}

    Предлагается следующий итеративный метод, который синтезирует идеи последовательного
     и~совместного методов, описанных выше. Положим $X_0 \hm= \varnothing$. 
     Строится одна по\-сле\-до\-ва\-тель\-ность $X_1 \hm\subset X_2 \hm\subset \dots \subset X_{\max}$. 
     На первом шаге $X_1\hm = \{x\}$, где $x$ ищется алгоритмом Apriori. На шаге $i \hm+ 1$ ($i \hm\geq 1$) 
     решается задача дуализации множества $X_{i} \setminus X_{i-1}$.

    
    
   \setcounter{figure}{1}
    \begin{figure*}[b] %fig2
  \vspace*{12pt}
  \begin{center}  
    \mbox{%
\epsfxsize=163mm
\epsfbox{duk-2.eps}
}

\end{center}
\vspace*{-9pt}
    \Caption{Зависимость времени работы алгоритмов от суммы мощностей множеств~$X_{\max}$ и~$Y_{\min}$ 
    для случая~1~(\textit{а}) и~2~(\textit{б}):
    \textit{1}~--- по\-сле\-до\-ва\-тель\-но-со\-вмест\-ный;
    \textit{2}~--- последовательный; \textit{3}~--- совместный; \textit{4}~--- Apriori}
    \label{12}
    \end{figure*}
     
    Пусть множество~$D$ есть результат дуализации $X_{i} \hm\setminus X_{i-1}$. Согласно утверждению~1, 
    множество~$D$ содержит частые наборы. Для каждого час\-то\-го набора из~$D$ 
    находится один содержащий его максимальный час\-тый набор путем последовательного 
    увеличения текущего час\-то\-го набора в~лексикографическом порядке. Все найденные максимальные
     частые наборы, которых нет в~множестве~$X_{i}$, до\-бав\-ля\-ют\-ся к~$X_{i}$, 
     и~таким образом формируется~$X_{i+1}$. Если же все найденные частые наборы уже содержатся в~$X_{i}$, 
     то решается задача дуализации множества~$X_{i}$. Если в~$I(X^{-}_{i})$ нет частых наборов, 
     то $I(X^{-}_{i})\hm = Y_{\min}$, $X_i \hm= X_{\max}$ и~алгоритм завершает работу. 
     Иначе для каждого частого набора из~$I(X^{-}_{i})$ находится один содержащий его максимальный 
     час\-тый набор, который пополняет множество~$X_{i}$, формируя в~результате множество~$X_{i+1}$.

    \section{Экспериментальное исследование}
    
    Рас\-смат\-ри\-вал\-ся случай данных, пред\-став\-лен\-ных в~виде произведения цепей мощ\-ности~5. 
    Для\linebreak таких данных проводился поиск максимальных час\-тых и~минимальных нечастых 
    наборов сле\-ду\-ющи\-ми методами: алгоритмом Apriori, модифицированным для случая 
    цепей; последовательным \mbox{методом}; совместным методом; по\-сле\-до\-ва\-тель\-но-со\-вмест\-ным методом.
    
    Все методы реализованы на языке Python~3. 
    Задача дуализации решалась алгоритмом дуализации цепей RUNC-M+~\cite{7}. 
    Эксперименты проведены на случайных базах данных различной раз\-мер\-ности. 
    Можно выделить два сле\-ду\-ющих случая соотношения мощностей множеств всех час\-тых и~не\-час\-тых наборов.
    \begin{description}
    \item[Случай 1:] мощ\-ность множества частых наборов примерно рав\-на мощ\-ности множества нечастых наборов.
    \item[Случай 2:] мощ\-ность множества частых наборов существенно меньше (больше) мощ\-ности множества 
    не\-час\-тых наборов.
    \end{description}
    
    Описанные случаи схематично изображены на рис.~1. 

    Графики зависимости времени работы тестируемых методов 
    от мощ\-ности множеств~$X_{\max}$ и~$Y_{\min}$ приведены на рис.~2.
    
    

    

    Нетрудно видеть, что в~случае~1 лучше работает по\-сле\-до\-ва\-тель\-но-со\-вмест\-ный алгоритм: 
    множества час\-тых и~не\-час\-тых наборов имеют примерно одинаковую мощ\-ность, 
    поэтому быст\-рее будет обрабатывать их по\-сле\-до\-ва\-тель\-но-со\-вмест\-ным методом. В~случае~2 
    быст\-рее работает последовательный алгоритм: быст\-рее найти множество максимальных час\-тых наборов, 
    обработав множество час\-тых наборов, и~дуализировать результат. Время поиска множеств~$X_{\max}$ 
    и~$Y_{\min}$ совместным методом и~модифицированным алгоритмом Apriori рас\-тет существенно 
    быст\-рее времени поиска по\-сле\-до\-ва\-тель\-но-со\-вмест\-ным методом в~обоих случаях.
    
    { \begin{center}  %fig1
 \vspace*{9pt}
    \mbox{%
\epsfxsize=67.963mm
\epsfbox{duk-1.eps}
}

\end{center}

\noindent
{{\figurename~1}\ \ \small{
Два случая соотношения мощностей множеств час\-тых и~не\-час\-тых наборов
}}}

%\vspace*{6pt}


    \section{Заключение}
    
Рас\-смот\-ре\-на задача поиска максимальных час\-тых и~минимальных не\-час\-тых наборов в~данных, 
представленных в~виде декартова произведения час\-тич\-ных порядков. Актуальны вопросы 
снижения временн$\acute{\mbox{ы}}$х затрат, возникающих при реализации методов нахождения искомых наборов.
 Разработан новый подход к~по\-стро\-ению множества максимальных частых наборов~$X_{\max}$ и~множества 
 минимальных не\-час\-тых наборов~$Y_{\min}$, пред\-став\-ля\-ющий собой синтез двух ранее известных 
 подходов: последовательного и~со\-вмест\-но\-го (первый достаточно очевиден, идея второго предложена в~\cite{9}). 
 Сложность последовательного, совместного и~пред\-ла\-га\-емо\-го по\-сле\-до\-ва\-тель\-но-со\-вмест\-но\-го поиска 
 обуслов\-ле\-на, в~том чис\-ле, не\-об\-хо\-ди\-мостью рас\-смат\-ри\-вать в~процессе поиска 
 труд\-но\-ре\-ша\-емую пе\-ре\-чис\-ли\-тель\-ную задачу дис\-крет\-ной математики, на\-зы\-ва\-емую дуализацией 
 над произведением час\-тич\-ных порядков.

Для случая, когда данные пред\-став\-ле\-ны в~виде произведения конечных цепей, 
приведены результаты экспериментального срав\-не\-ния названных подходов, а~так\-же независимого 
способа \mbox{по\-стро\-ения} множеств~$X_{\max}$ и~$Y_{\min}$, не тре\-бу\-юще\-го решения задачи дуализации. 
Эксперименты проводились на модельных задачах с~применением асимптотически оптимального
 алгоритма дуализации над произведением конечных цепей \mbox{RUNC-M}+~\cite{7}. 
 Результаты исследования свидетельствуют о~том, что по\-сле\-до\-ва\-тель\-но-со\-вмест\-ный 
 метод наиболее эффективен (требует меньших временн$\acute{\mbox{ы}}$х затрат по сравнению с~другими рас\-смот\-рен\-ны\-ми 
 методами) в~случае, когда мощ\-ность множества час\-тых наборов примерно равна мощ\-ности множества
  нечастых наборов. Иначе выигрывает последовательный поиск. Наихудшие показатели 
  у~независимого пе\-ре\-чис\-ле\-ния множеств~$X_{\max}$ и~$Y_{\min}$ с~использованием в~качестве
   базового алгоритма Apriori~\cite{2}, точ\-нее его модификации на тес\-ти\-ру\-емый случай. 
   Таким образом, показана це\-ле\-со\-об\-раз\-ность применения алгоритмов дуализации для 
   по\-стро\-ения множеств~$X_{\max}$ и~$Y_{\min}$.

  
  {\small\frenchspacing
 {%\baselineskip=10.8pt
 %\addcontentsline{toc}{section}{References}
 \begin{thebibliography}{9}  
    \bibitem{4}
    \Au{Aggarwal C.} 
    Frequent pattern mining.~--- Heidelberg: Springer, 2014. 467~p.
    
    \bibitem{1}
    \Au{Agrawal~R., Imielinski~T., Swami~A.} Mining association rules 
    between sets of items in large databases~// \mbox{SIGMOD} Conference (International) on Management of Data
    Proceedings.~--- New York, NY, USA: ACM, 1993. P.~207--216.
    
    \bibitem{9}
    \Au{Elbassioni K.} On finding minimal infrequent elements in multi-dimensional 
    data defined over partially ordered sets~// arXiv.org, 2014. 30~p. arXiv:1411.2275 [cs.DB].
    
    \bibitem{8}
    \Au{Elbassioni K.} Algorithms for dualization over products of partially 
    ordered sets~// SIAM J.~Discrete Math., 2009. Vol.~23. Iss.~1. P.~487--510.
    
    \bibitem{2}
    \Au{Agrawal R., Srikant~R.} 
    Fast algorithms for mining association rules in large databases~// 
    20th Conference (International) on Very Large Data Bases Proceedings.~--- San Francisco, CA, USA: 
    Morgan Kaufmann Publs. Inc., 1994. P.~487--499.
    
    \bibitem{14}
    \Au{Хачиян Л.\,Г.} Избранные труды.~--- М.: МЦНМО, 2009. 520~с.
    
    \bibitem{7}
    \Au{Дюкова Е.\,В., Масляков~Г.\,О., Прокофьев~П.\,А.} 
    О~дуализации над произведением частичных порядков~// Машинное обучение и~анализ данных, 2017. Т.~3. №\,4.  
    C.~239--249.
    
    \bibitem{6}
    \Au{Дюкова Е.\,В., Прокофьев~П.\,А.} Об асимптотически оптимальных алгоритмах дуализации~// 
    Ж.~вычисл. матем. и~матем. физ., 2015. Т.~55. №\,5. С.~895--910.
    \end{thebibliography}

 }
 }

\end{multicols}

\vspace*{-6pt}

\hfill{\small\textit{Поступила в~редакцию 15.01.21}}

\vspace*{8pt}

%\pagebreak

%\newpage

%\vspace*{-28pt}

\hrule

\vspace*{2pt}

\hrule

%\vspace*{-2pt}

\def\tit{FINDING MAXIMAL FREQUENT AND~MINIMAL INFREQUENT SETS IN~PARTIALLY ORDERED DATA}


\def\titkol{Finding maximal frequent and~minimal infrequent sets in~partially ordered data}


\def\aut{N.\,A.~Dragunov and E.\,V.~Djukova}

\def\autkol{N.\,A.~Dragunov and E.\,V.~Djukova}

\titel{\tit}{\aut}{\autkol}{\titkol}

\vspace*{-11pt}


\noindent
Federal Research Center ``Computer Science and Control'' 
of the Russian Academy of Sciences, 44-2~Vavilov Str., Moscow 119333, Russian Federation

\def\leftfootline{\small{\textbf{\thepage}
\hfill INFORMATIKA I EE PRIMENENIYA~--- INFORMATICS AND
APPLICATIONS\ \ \ 2022\ \ \ volume~16\ \ \ issue\ 1}
}%
 \def\rightfootline{\small{INFORMATIKA I EE PRIMENENIYA~---
INFORMATICS AND APPLICATIONS\ \ \ 2022\ \ \ volume~16\ \ \ issue\ 1
\hfill \textbf{\thepage}}}

\vspace*{3pt} 


\Abste{Relevant issues of time costs reducing in the logical analysis of data with elements 
from the Cartesian product of finite partially ordered sets are investigated. 
An original method based on solving a complex discrete problem called dualization
 over the product of partial orders is proposed for the problem of finding maximal 
 frequent and minimal infrequent sets in the transaction database. The proposed method 
 is a~synthesis of two other known methods, one of which is quite obvious and the other uses 
 the idea of an incremental enumeration of target\linebreak\vspace*{-12pt}}
 
 \Abstend{sets and is, therefore, mainly 
 of theoretical interest. An experimental study of the considered approaches in
  the case of the product of finite chains is carried out and conditions for
   their effectiveness are revealed. The expediency of applying 
asymptotically optimal dualization algorithms over the product of partial orders is shown.}

\KWE{maximal frequent sets; minimal infrequent sets; dualization over the product of 
partial orders; asymptotically optimal dualization algorithm}

\DOI{10.14357/19922264220112}

%\vspace*{-16pt}

%\Ack
%\noindent




%\vspace*{6pt}

  \begin{multicols}{2}

\renewcommand{\bibname}{\protect\rmfamily References}
%\renewcommand{\bibname}{\large\protect\rm References}

{\small\frenchspacing
 {%\baselineskip=10.8pt
 \addcontentsline{toc}{section}{References}
 \begin{thebibliography}{9}
\bibitem{1-dr}
\Aue{Aggarwal, C.} 2014. \textit{Frequent pattern mining}. Heidelberg: Springer. 467~p.
\bibitem{2-dr}
\Aue{Agrawal, R., T.~Imielinski, and A.~Swami.}
 1993. Mining association rules between sets of items in large databases. 
 \textit{SIGMOD  Conference (International) on Management of Data Proceedings}. New York, NY:
 ACM. 207--216. 
\bibitem{3-dr}
\Aue{Elbassioni, K.}
 2014. On finding minimal infrequent elements in multidimensional data defined over partially ordered sets. 
 arXiv.org. 30~p. Available at: 
 {\sf https://arxiv.org/\linebreak pdf/1411.2275.pdf} (accessed January~25, 2022).
\bibitem{4-dr}
\Aue{Elbassioni, K.} 2009. Algorithms for dualization over products of partially ordered sets. 
\textit{SIAM J.~Discrete Math.} 23(1):487--510.
\bibitem{5-dr}
\Aue{Agrawal, R., and R.~Srikant.}
 1994. Fast algorithms for mining association rules in large databases. 
 \textit{20th Conference (International) on Very Large Data Bases Proceedings}.
 San Francisco, CA: 
    Morgan Kaufmann Publs. Inc.  487--499.
\bibitem{6-dr}
\Aue{Khachiyan, L.\,G.} 2009. \textit{Izbrannye trudy} [Selected works]. Moscow: MCCME. 520~p.
\bibitem{7-dr}
\Aue{Djukova, E.\,V., G.\,O.~Maslyakov, and P.\,A.~Prokofyev.} 
2017. O~dualizatsii nad proizvedeniem chastichnykh poryadkov [On dualization over the product of 
partial orders]. \textit{Mashinnoe obuchenie i~analiz dannykh} [J.~Machine Learning Data Analysis] 
3(4):239--249.
\bibitem{8-dr}
\Aue{Djukova, E.\,V., and P.\,A.~Prokofyev.}
 2015. Asymptotically optimal dualization algorithms. \textit{Comp. Math.
 Math. Phys.} 55(5):891--905. 
 
 \end{thebibliography}

 }
 }

\end{multicols}

\vspace*{-6pt}

\hfill{\small\textit{Received January 15, 2021}}

%\pagebreak

%\vspace*{-18pt}

\Contr

\noindent
\textbf{Dragunov Nikita A.} (b.\ 1997)~--- 
PhD student, Federal Research Center ``Computer Science and Control'' 
of the Russian Academy of Sciences, 44-2~Vavilov Str., Moscow 119333, Russian Federation; 
\mbox{nikitadragunovjob@gmail.com}

\vspace*{3pt}

\noindent
\textbf{Djukova Elena V.} (b.\ 1945)~--- 
Doctor of Science in physics and mathematics, principal scientist, Federal Research Center
``Computer Science and Control'' of the Russian Academy of Sciences, 44-2~Vavilov Str., Moscow 119333, 
Russian Federation; \mbox{edjukova@mail.ru}




\label{end\stat}

\renewcommand{\bibname}{\protect\rm Литература}  %8
\def\stat{shestakov+vor}

\def\tit{АСИМПТОТИЧЕСКАЯ НОРМАЛЬНОСТЬ И~СИЛЬНАЯ СОСТОЯТЕЛЬНОСТЬ ОЦЕНКИ РИСКА ПРИ~ИСПОЛЬЗОВАНИИ FDR-ПОРОГА В УСЛОВИЯХ СЛАБОЙ ЗАВИСИМОСТИ}

\def\titkol{Асимптотическая нормальность и~сильная состоятельность оценки риска при~использовании FDR-порога} % в~условиях слабой зависимости}

\def\aut{М.\,О.~Воронцов$^1$, О.\,В.~Шестаков$^2$}

\def\autkol{М.\,О.~Воронцов, О.\,В.~Шестаков}

\titel{\tit}{\aut}{\autkol}{\titkol}

\index{Воронцов М.\,О.}
\index{Шестаков О.\,В.}
\index{Vorontsov M.\,O.}
\index{Shestakov O.\,V.}


%{\renewcommand{\thefootnote}{\fnsymbol{footnote}} \footnotetext[1]
%{Работа 
%выполнена при поддержке Программы развития МГУ, проект №\,23-Ш03-03. При анализе 
%данных использовалась инфраструктура Центра коллективного пользования 
%<<Высокопроизводительные вычисления и~большие данные>> 
%(ЦКП <<Информатика>>) ФИЦ ИУ РАН (г.~Москва)}}


\renewcommand{\thefootnote}{\arabic{footnote}}
\footnotetext[1]{Московский государственный университет 
имени~М.\,В.~Ломоносова, факультет вычислительной математики и~кибернетики;  
Московский центр фундаментальной и~прикладной математики, \mbox{m.vtsov@mail.ru}}
\footnotetext[2]{Московский государственный университет 
имени М.\,В.~Ломоносова, факультет вычислительной математики и~кибернетики; 
Федеральный исследовательский центр <<Информатика и~управление>> Российской 
академии наук; Московский центр фундаментальной и~прикладной математики, 
\mbox{oshestakov@cs.msu.ru}}


\vspace*{-12pt}





\Abst{Рассматривается подход к~решению задачи удаления шума в~большом массиве 
разреженных данных, основанный на методе контроля средней доли ложных отклонений 
гипотез (False Discovery Rate, FDR). Данный подход эквивалентен процедурам 
пороговой обработки, обнуляющим компоненты массива, значения которых не 
превосходят некоторого заданного порога.  Наблюдения в~модели считаются слабо 
зависимыми. Для контроля степени зависимости используются ограничения на 
коэффициент сильного перемешивания и~максимальный коэффициент корреляции. 
В~качестве меры эффективности рассматриваемого подхода используется 
среднеквадратичный риск. Вычислить значение риска можно только на тестовых 
данных, поэтому в~работе рассматривается его статистическая оценка и~исследуются 
ее свойства. Показана асимптотическая нормальность и~сильная состоятельность 
оценки риска при использовании FDR-по\-ро\-га в~условиях слабой зависимости в~данных.}

\KW{пороговая обработка; множественная проверка гипотез; 
оценка риска}

\DOI{10.14357/19922264240309}{ZOQVTO}
  
%\vspace*{-6pt}


\vskip 10pt plus 9pt minus 6pt

\thispagestyle{headings}

\begin{multicols}{2}

\label{st\stat}



\section{Введение}

Во многих прикладных областях возникает задача обработки больших массивов 
зашумленных данных. Примерами служат задачи обработки изоб\-ра\-же\-ний с~высоким 
разрешением~\cite{FDRImage}, задачи множественной проверки гипотез, возникающие 
в~\mbox{исследованиях} в~об\-ласти генетики~\cite{MultipleTesting}, и~другие проб\-ле\-мы. 
В~связи с~этим рас\-смот\-рим модель
$$
x_i = \mu_i + z_i, \enskip i=\overline{1,n}\,,
$$
где $\mu_i\in\mathbb{R}$~--- <<полезные>> данные; $z_i \sim N(0,\sigma^2)$~--- 
шум. Задача заключается в~нахождении оценки неизвестного вектора $\mu \hm= 
(\mu_1,\ldots,\mu_n)$ как функции вектора $x \hm= (x_1,\ldots,x_n)$ и~может 
рассматриваться как задача множественной проверки гипотез о~равенстве нулю 
компонент вектора~$\mu$~\cite{AdaptingFDR}. При этом обычно предполагается, что 
вектор~$\mu$ имеет в~определенном смысле <<разреженную>> структуру, т.\,е.\ для 
<<полезных>> данных используется <<экономное>> представление.



В работе~\cite{AdaptingFDR} для решения рассматриваемой задачи в~условиях 
независимости компонент вектора~$x$ и~разреженности вектора~$\mu$ была 
предложена процедура построения оценки~$\hat{\mu}_F$ вектора~$\mu$, основанная 
на методе контроля средней доли ложных отклонений (FDR) 
гипотез при помощи алгоритма Бен\-жа\-ми\-ни--Хох\-бер\-га,
и~было проведено исследование асимптотики ее среднеквадратичного риска. 
В~работах~\cite{ZasShe17,Mathematics2020} была показана состоятельность 
и~асимптотическая нормальность оценки риска данной процедуры. Аналогичные 
результаты для других методов построения~$\hat{\mu}_F$ получены в~работах~\cite{Shestakov2021-1,Shestakov2021-2,Shestakov2022}.

В то же время в~определенных приложениях, например  при анализе полученных 
в~результате использования ДНК-мик\-ро\-чи\-пов данных~\cite{ResultsOnFDRUnderDependence}, исследовании геофизических процессов 
и~анализе помех\linebreak в~телекоммуникационных каналах, условие незави\-си\-мости компонент 
вектора $x$ может не выполняться. Ранее в~работах~\cite{VorontsovShestakov2023,Vorontsov2024} была \mbox{исследована} асимп\-то\-ти\-ка 
среднеквадратичного риска оценки~$\hat{\mu}_F$ \mbox{в~случае}, когда~$\mu$ принадлежит 
одному из классов разреженности
$$
l_0[\eta] = \left\{\mu\,:\, ||\mu||_0 \leq \eta n\right\}, \enskip \eta \in 
(0,1),
$$

\vspace*{-12pt}

\noindent
\begin{multline*}
m_p[\eta] \equiv{}\\
{}\equiv \left\{\mu \in \mathbb{R}^n : |\mu|_{(k)} \leq \eta n^{1/p} 
k^{-1/p},\ k=\overline{1,n}\right\}, \\
 p\in(0, 2),
\end{multline*}
а компоненты вектора~$x$ слабо зависимы~--- имеют достаточно быстро убывающий 
коэффициент сильного перемешивания~\cite{Bosq}

\noindent
\begin{multline*}
\alpha(k) = \sup\limits_{1\leq m\leq n}\alpha\left(\sigma(x_i, i\leq m), 
\sigma(x_i, i\geq m+k)\right), \\ 
k=\overline{1,n-1}\,,
\end{multline*}
где символом $\sigma(x_i, i\in I)$ обозначена сиг\-ма-ал\-геб\-ра, порожденная 
множеством случайных величин $\{x_i, i \hm\in I\}$, а~мера  $\alpha(\cdot, \cdot)$ 
близости двух сиг\-ма-ал\-гебр определяется как
$$
\alpha(\mathcal{B},\mathcal{C}) = \sup\limits_{B\in\mathcal{B}, 
C\in\mathcal{C}} \left|\p(BC)-\p(B)\p(C)\right|.
$$

В настоящей работе показана асимптотическая нормальность и~сильная 
состоятельность оценки риска при применении FDR-про\-це\-ду\-ры в~случае, когда 
компоненты вектора~$x$ слабо зависимы, а~$\mu$ принадлежит одному из классов 
раз\-ре\-жен\-ности: 
$l_0[\eta]$ или $m_p[\eta]$.


\section{Обработка вектора данных с~помощью FDR-процедуры}

Широким классом методов построения оценки~$\hat{\mu}$ стала пороговая обработка 
вектора~$x$ с~некоторым порогом~$T$. Различают жесткую пороговую обработку, при 
которой полагается
\begin{equation*}
\left(\hat{\mu}\right)_i  = p_H(x_i,T) \equiv
 \begin{cases}
   x_i, & |x_i| > T\,;\\
   0, & |x_i| \leq T\,,
 \end{cases}
\end{equation*}
и мягкую пороговую обработку, для которой
\begin{equation*}
(\hat{\mu})_i  = p_S(x_i,T) \equiv
 \begin{cases}
   x_i-T, & \hphantom{\vert\vert}x_i > T;\\
   x_i+T, & \hphantom{\vert\vert}x_i <- T;\\
   0, & |x_i| \leq T.
 \end{cases}
\end{equation*}
Среднеквадратичный риск подобных процедур определяется как
\begin{equation}
\label{riskDef}
R(T) = {\mathsf E} ||\hat{\mu}-\mu||^2 = \sum\limits_{i=1}^n {\mathsf E} \left((\hat{\mu})_i-
\mu_i\right)^2.
\end{equation}
Обозначим через~$T_m$ наилучшее значение порога:
$$
T_m : \, R(T_m) = \min\limits_{T} R(T).
$$

Предложенная в~\cite{AdaptingFDR} процедура заключается в~жесткой пороговой 
обработке компонент вектора~$x$ с~порогом $\hat{t}_F \hm= \hat{t}_F(x)$, и~ее 
результат~--- оценка $\hat{\mu}_F$ вектора~$\mu$ с~компонентами $(\hat{\mu}_F)_i  
\hm= p_H(x_i,\hat{t}_F)$, где
\begin{multline*}
\hat{t}_F = \sigma z\left(\fr{q \hat{k}_F}{2n}\right), \enskip
\hat{k}_F = \max 
\left\{k \, :\, |x|_{(k)} \geq t_k \right\}, \\
 t_k = \sigma z\left(\fr{q  k}{2n}\right);
\end{multline*}
$z(\alpha)$ --- квантиль уровня $(1\hm-\alpha)$ стандартного нормального 
распределения; $|x|_{(k)}$~--- $k$-й элемент вектора, получаемого в~результате 
упорядочения вектора~$|x|$ по невозрастанию:
$$
|x|_{(1)} \geq |x|_{(2)} \geq \cdots \geq |x|_{(n)};
$$
$q\in(0;1)$~--- управ\-ля\-ющий параметр FDR-ме\-то\-да.
Далее полагается, что $q\hm\equiv q_n$ зависит от~$n$. В~\cite{AdaptingFDR} 
показано, что эта процедура эквивалентна множественной проверке гипотез 
о~равенстве нулю компонент наблюдаемого вектора. Также показано, что с~помощью 
метода штрафных функций данную процедуру можно свести к~другим видам пороговой 
обработки, в~част\-ности к~мягкой пороговой обработке.

В работах~\cite{VorontsovShestakov2023, Vorontsov2024} была исследована 
асимптотика среднеквадратичного риска~$R(\hat{t}_F)$ описанной процедуры 
в~случае, когда компоненты вектора $x$ слабо зависимы, а $\mu$ принадлежит классу 
разреженности~$\Theta_n$, где~$\Theta_n$ есть~$l_0[\eta_n]$ или~$m_p[\eta_n]$. 
Было показано, что~$R(\hat{t}_F)$ асимптотически отличается от минимаксного 
риска
$\inf\nolimits_{\hat{\mu}\hm=\hat{\mu}(x)} \sup\nolimits_{\mu\in \Theta_n} {\mathsf E} 
||\hat{\mu}-\mu||^2$
на множитель не более чем логарифмического по\-рядка.

Отметим, что в~выражении для среднеквадратичного риска~(\ref{riskDef}) 
присутствуют неизвестные величины~$\mu_i$, а~потому вычислить~$R(T_m)$ и~$T_m$ 
не представляется возможным. На практике можно пользоваться, например, следующей 
оценкой среднеквадратичного риска~\cite{Mallat}:
$$
\hat{R}(T) = \sum\limits_{i=1}^n F[x_i, T],
$$
где  
\begin{multline*}
F[x_i, T] = {}\\[3pt]
{}=\!\begin{cases}
\left(x_i^2-\sigma^2\right) \Ik(|x_i|\leq T) + \sigma^2 \Ik\left(|x_i|>T\right) &\\[3pt]
&\hspace*{-53mm}\mbox{для\ жесткой\ пороговой\ обработки};\\[3pt]
\left(x_i^2-\sigma^2\right) \Ik\left(|x_i|\leq T\right) + (\sigma^2+T^2) 
\Ik \left(|x_i|>T\right) \hspace*{-11.21576pt}&\\[3pt]
&\hspace*{-51mm}\mbox{для\ мягкой\ пороговой\ обработки}.
\end{cases}\hspace*{-7.17859pt}
\end{multline*}


\noindent
\textbf{Замечание}.\ При пороговой обработке иногда также используется так 
называемый универсальный порог $T_U\hm = \sigma \sqrt{2\ln n}$, предложенный 
в~работе~\cite{spatialAdaptation}. Исследования в~\cite{AdaptingSURE, ExactRisk} 
показали, что порог~$T_U$ в~определенном смысле максимальный, и~рас\-смат\-ри\-вать 
пороги выше него не имеет смысла. Более того, нетрудно показать, что $t_k \hm< T_U$ 
для всех~$k$ и~всех достаточно больших~$n$, в~связи с~чем всюду далее полагаем, 
что порог~$\hat{t}_F$ выбирается на отрезке $[0; T_U]$.

\section{Вспомогательные утверждения}

Кроме коэффициента сильного перемешивания~$\alpha(\cdot)$ также понадобится 
следующее понятие~\cite{Bosq}.

\smallskip

\noindent
\textbf{Определение.} %\label{defRho}
Максимальным коэффициентом корреляции~$\rho(\cdot)$ компонент вектора~$x$ 
называется
\begin{multline*}
\rho (k) \equiv \rho_n (k) = {}\\
{}=\sup\limits_{1\leq m\leq n}\rho\left(\sigma(x_i, 
i\leq m), \sigma(x_i, i\geq m+k)\right), \\
 k=\overline{1,n-1}\,,
\end{multline*}
где мера $\rho(\cdot, \cdot)$ близости двух сиг\-ма-ал\-гебр определяется как
$$
\rho(\mathcal{B},\mathcal{C}) = \sup\limits_{\substack{\xi 
\in\mathcal{L}^2(\mathcal{B}) \\
 \eta \in\mathcal{L}^2(\mathcal{C})}} 
\left|\mathrm{corr}\,(\xi, \eta)\right|.
$$


Введем обозначения:
$$
T_1 = \sqrt{2\ln \eta_n^{-p}};  \,\gamma_n = \fr{1}{\ln\ln n}; \, \kappa_n 
= \fr{n \eta_n^p T_1^{-p}}{1 - q_n - \gamma_n}; 
$$
$$ 
\kappa_n^0 = \fr{[n \eta_n]}{1 - q_n - \gamma_n} ;\, \rho^\star (k) = 
\sup\limits_{n\geq k+1} \rho(k), k \in \mathbb{N} ;
$$
$$
t_{\kappa_n} = \sigma z\left(\fr{q_n \kappa_n }{2n}\right) , \,\, t_{\kappa_n^0} 
= \sigma z\left(\fr{q_n \kappa_n^0 }{2n}\right).
$$


Следующие два утверждения показывают, что случайный порог~$\hat{t}_F$ в~случае 
$\mu\hm\in m_p[\eta_n]$ (соответственно $\mu\hm\in l_0[\eta_n]$) с~большой 
вероятностью будет не меньше~$t_{\kappa_n}$ (соответственно~$ t_{\kappa_n^0}$). 
Их  доказательства приведены в~работах~\cite{VorontsovShestakov2023, Vorontsov2024}.

\smallskip

\noindent
%\begin{lem}\label{lem5}
\textbf{Лемма~1.}\ \textit{Пусть $n^{-\delta_1} \hm\leq \eta_n^p \hm\leq n^{-\delta_2}$, 
$0\hm<\delta_2\hm<\delta_1<1$, $\mathrm{lim\,inf} q_n \ln n \hm\geq C \hm> 0$, 
$m\hm\in[1;n/2]\cap\mathbb{N}$, а $\alpha(\cdot)$~--- коэффициент сильного 
перемешивания компонент вектора~$x$. Для некоторого $N\hm\in\mathbb{N}$ при $n \hm\geq 
N$ справедливо}
\begin{multline*}
\hspace*{-3pt}\sup\limits_{\mu\in m_p[\eta_n]} \p \left(\hat{k}_F \geq \kappa_n \right) \leq 
4 n \exp\left\{-\fr{m}{256n}  \kappa_n q_n \gamma_n^2    \right\}+{}\\
{}+ 22\left(1+\fr{8n}{\kappa_n q_n \gamma_n}\right)^{1/2} n m 
\alpha\left(\left[\fr{n}{2m}\right]\right).
\end{multline*}



\smallskip

\noindent
\textbf{Лемма 2.}\ 
%\label{lem1}
\textit{Пусть $\eta_n \hm\leq b\hm<1$, $m\in[1;n/2]\cap\mathbb{N}$, а~$\alpha(\cdot)$~--- 
коэффициент сильного перемешивания компонент вектора~$x$. Для некоторого 
$N\hm\in\mathbb{N}$ при $n \hm\geq N$ справедливо}
\begin{multline*}
\sup\limits_{\mu\in l_0[\eta_n]} \p \left(\hat{k}_F \geq \kappa_n^0 \right) 
\leq{}\\
{}\leq 4 n \exp\left\{-\fr{(1-b)m}{64n}\,  \kappa_n^0 q_n \gamma_n^2    
\right\}+{}\\
{}+ 22\left(1+\fr{4n}{(1-b)\kappa_n^0 q_n \gamma_n}\right)^{1/2} n m 
\alpha\left(\left[\fr{n}{2m}\right]\right).
\end{multline*}

Следующие два утверждения доказаны в~\cite{Bosq} и~представляют собой аналоги 
неравенств Хеффдинга и~Бернштейна для слабо зависимых случайных величин.


\smallskip

\noindent
\textbf{Лемма 3.}\
\textit{Пусть для набора действительных случайных величин $X_1, \ldots, X_n$ 
с~коэффициентом сильного перемешивания $\alpha(\cdot)$ выполняется ${\mathsf E} X_i \hm=0$, 
$|X_i|\hm\leq b$, $i\hm=\overline{1,n}$. Тогда для любого целого числа $m\hm\in[1; n/2]$ 
и~любого $\eps\hm>0$ справедливо}
\begin{multline*}
\p\left(\left|\sum\limits_{i=1}^n X_i\right| > n\eps \right) \leq 4 
\exp\left\{-\fr{\eps^2 m}{8 b^2}\right\}+ {}\\
{}+
22\left(1+\fr{4b}{\eps}\right)^{1/2} m\, 
\alpha\left(\left[\fr{n}{2m}\right]\right).
\end{multline*}


\smallskip

\noindent
\textbf{Лемма 4.}\
\textit{Пусть для набора действительных случайных величин $X_1, \ldots, X_k$ 
с~коэффициентом сильного перемешивания $\alpha(\cdot)$ выполняется ${\mathsf E} X_i \hm=0$, 
$|X_i|\hm\leq b$, $i\hm=\overline{1,k}$. Тогда для любого целого числа $m\hm\in[1; k/2]$ 
и~любого $\eps\hm>0$ справедливо}
\begin{multline*}
\p\left(\left|\sum\limits_{i=1}^k X_i\right| > \eps \right) \leq 4 
\exp\left\{-\fr{\eps^2 m}{8 v^2 k^2}\right\}+{}\\
{}+ 22\left(1+\fr{4bk}{\eps}\right)^{1/2} m\, 
\alpha\left(\left[\fr{k}{2m}\right]\right),
\end{multline*}
\textit{где $p = k/(2m)$}:
\begin{multline*}
v^2 =
 \fr{b \eps}{2k} + {}\\
 {}+\fr{2}{p^2} \,  \max\limits_{ j\in[0,\,2m-1]} 
{\mathsf E} \big( ([jp]+1-jp)X_{[jp]+1} + X_{[jp]+2}+{}\\
{}+ \cdots +  X_{[(j+1)p]} + ((j+1)p-[(j+1)p])X_{[(j+1)p+1]}\big)^2.
\end{multline*}

\noindent
\textbf{Замечание.}
Если существует такое число $S \hm> 0$, что сразу для всех $i\hm\in[1;k]$  выполняется 
${\mathsf E} X_i^2 \hm\leq S^2$, то в~качестве~$v^2$ можно взять
$$
v^2 = \fr{b \eps}{2k} + 8 S^2.
$$


Д\,о\,к\,а\,з\,а\,т\,е\,л\,ь\,с\,т\,в\,о\ \ сле\-ду\-юще\-го утверж\-де\-ния приведено в~работе~\cite{AdaptingFDR}.

\smallskip

\noindent
\textbf{Лемма 5.}\ 
\textit{Для $y\leq 0{,}01$ справедливы представления}
\begin{multline}
\label{lem1eq1}
z^2(y) = 2 \ln y^{-1} - \ln \ln y^{-1} - r_2(y), \\
 r_2(y) \in [1{,}8; 3];
\end{multline}

\noindent
\begin{equation}
\label{lem1eq2}
z(y) = \sqrt{2 \ln y^{-1}} - r_1(y), \, \, r_1(y) \in [0; 1{,}5].
\end{equation}


\section{Асимптотическая нормальность оценки риска при~применении FDR-процедуры в~условиях слабой зависимости}

Перейдем к~описанию достаточных условий для асимптотической нормальности оценки 
риска $\hat{R}(\hat{t}_F)$ в~случае $\mu \hm\in m_p[\eta_n]$.

\smallskip

\noindent
\textbf{Теорема~1.}\
\textit{Пусть $\mu \hm\in m_p[\eta_n],$ $\eta_n^p \hm\in[n^{-\delta_1}; n^{-\delta_2}],$ $1/2 \hm< 
\delta_2 \hm< \delta_1<1;$ имеются такие константы $c_1, c_2>0$, что для 
коэффициента сильного перемешивания $\alpha(\cdot)$ компонент вектора $x$ 
справедливо  $\alpha(k) \hm\leq c_1 k^{-1-(5/2)\delta_1/(1-\delta_1)-c_2},$ 
$k\hm=\overline{1,n-1};$ $q_n \hm< c_3 \hm< 1;$ $\mathrm{lim\,inf} q_n \ln n \hm= c_4 \hm> 0;$ и,~кроме того, 
для максимального коэффициента корреляции $\rho(\cdot)$ компонент вектора~$x$ 
справедливо}
$$
\sum\limits_{k = 1}^{\infty} \sup\limits_{n\geq k+1} \rho(k) \equiv 
\sum\limits_{k = 1}^{\infty}  \rho^\star (k) = c_5 < \infty. 
$$
\textit{Тогда при $n \to \infty$}
$$
\fr{\hat{R}(\hat{t}_F) - R(T_m)}{C_\rho \sqrt{2n}} \Rightarrow N(0, 1),
$$
\textit{где}
$$
C_\rho = \sigma^2\sqrt{1 +  \lim\limits_{n\to\infty} \fr{1}{n} \sum\limits_{j\neq i} \mathrm{corr}^2 (x_i, x_j)}.
$$

\noindent
Д\,о\,к\,а\,з\,а\,т\,е\,л\,ь\,с\,т\,в\,о\  \
 приводится для метода мягкой пороговой обработки; в~случае жесткой пороговой 
обработки доказательство аналогично. Обозначим
$$
U(T) = \hat{R}(T) -  \hat{R}(T_m) = \sum \limits_{i=1}^n H_i(T, T_m),
$$
где
$$
H_i(T, T_m) = F[x_i, T] - F[x_i, T_m].
$$
Имеем

\vspace*{-3pt}

\noindent
\begin{multline}
\label{D00}
\hat{R}(\hat{t}_F) - R(T_m) + \hat{R}(T_m) - \hat{R}(T_m) ={}\\
{}= \hat{R}(T_m) - 
R(T_m) + U(\hat{t}_F).
\end{multline}
Покажем, что
\begin{equation}
\label{D0}
\fr{\hat{R}(T_m) - R(T_m)}{C_\rho\sqrt{2n}} \Rightarrow N(0, 1).
\end{equation}


Повторяя рассуждения из~\cite{KuShe2016_1,KuShe2016_2,Jansen}, можно показать, 
что $T_m \hm\geq t_{\kappa_n}$. Учитывая также $T_m\hm \leq T_U$, имеем 
$$
C \sqrt{\ln n} \leq T_m \leq C^\prime \sqrt{\ln n}
$$ 
для некоторых положительных констант $C$ и~$C^\prime$.

\columnbreak

В случае мягкой пороговой обработки $\hat{R}(T_m)$ представляет собой 
несмещенную оценку~$R(T_m)$, а~при жесткой пороговой обработке и~выполнении 
условий теоремы смещение стремится к~нулю при делении на $\sqrt{n}$~\cite{Mallat}.

Для дисперсии числителя~(\ref{D0}) имеем:
\begin{multline*}
{\mathsf D} \left(\hat{R}(T_m) - R(T_m)\right) = \sum\limits_{i=1}^n {\mathsf D} F[x_i, T_m] + {}\\
{}+
\sum\limits_{i=1}^n\sum\limits_{\substack{j=1 \\  j\neq i}}^n \mathrm{cov}\left( F[x_i, T_m], F[x_j, 
T_m] \right).
\end{multline*}

Поскольку $\mu \in m_p[\eta_n]$,
\begin{equation}
\left.
\begin{array}{l}
 \displaystyle\sum\limits_{i: |\mu_i| > 1/T_1} {\mathsf D} F[x_i, T_m]  \leq{}\\
 \hspace*{15mm}{}\leq  4\left(\sigma^2 + T_m^2\right)^2 n \eta_n^p 
T_1^p = o(n);
\\[6pt]
\displaystyle \sum\limits_{\substack{{i,j: \max\{|\mu_i|, |\mu_j|\} > 1/T_1,}\\{j\neq i}}}  \hspace*{-12mm}\mathrm{cov}\,(F[x_i, 
T_m],F[x_j, T_m])  \leq{}\\
\hspace*{10mm}{}\leq 16\left(\sigma^2 + T_m^2\right)^2 n \eta_n^p T_1^p c_5 = o(n). 
\end{array}
\right\}    
\label{D2}
\end{equation}
Далее, учитывая что ${\mathsf D} x_i^2 \hm= 2\sigma^4 \hm+ 4\sigma^2 \mu_i^2$, нетрудно 
убедиться, что
\begin{multline}
\label{D3}
\sum\limits_{i: |\mu_i| \leq 1/T_1}\hspace*{-4mm} {\mathsf D} F[x_i, T_m] ={}\\
{}= \sum\limits_{i: |\mu_i| \leq 1/T_1} \hspace*{-4mm} {\mathsf D} 
x_i^2 + o(n) = 2\sigma^4 n + o(n).
\end{multline}


Введем обозначение 
$$
D_n = \left\{(i,j) : \max\left\{|\mu_i|, |\mu_j|\right\}  \leq \fr{1}{T_1}\,, \enskip j\hm\neq i\right\}.
$$
 Для суммы ковариаций аналогично~(\ref{D3}) получим
\begin{multline*}
\sum\limits_{(i,j)\in D_n} \hspace*{-2mm}\mathrm{cov}\left( F[x_i, T_m], F[x_j, T_m] \right) = {}\\
{}=
\sum\limits_{(i,j)\in D_n} \hspace*{-2mm}\mathrm{cov}\left( x_i^2, x_j^2 \right) + o(n).
\end{multline*}
Воспользуемся тождеством~\cite{Eroshenko}
$$
\mathrm{cov}\left (x_i^2, x_j^2\right) = 4 {\mathsf E} x_i {\mathsf E} x_j \mathrm{cov}\left(x_i, x_j\right) + 2 \mathrm{cov}^2 \left(x_i, x_j\right)
$$
для вектора $(x_i, x_j)$, имеющего двумерное нормальное распределение. Заметим, 
что
\begin{gather*}
 \sum\limits_{(i,j)\in D_n} 4 | {\mathsf E} x_i {\mathsf E} x_j \mathrm{cov}\left(x_i, x_j\right)| \leq 8 T_1^{-2} 
\sigma^2 n c_5 = o(n);
\\
\sum\limits_{(i,j)\in D_n} 2 \mathrm{cov}^2 (x_i, x_j)  = 2\sigma^4 \sum\limits_{(i,j)\in D_n} 
\mathrm{corr}^2 (x_i, x_j). 
\end{gather*}
Более того, поскольку  %< 4 \sigma^2 n c_5.$$
\begin{equation*}
\sum\limits_{\substack{{i,j: \max\{|\mu_i|, |\mu_j|\} > 1/T_1} \\ {j\neq i}}}
\hspace*{-10mm}\mathrm{corr}^2 (x_i, x_j)  
\leq  4 n \eta_n^p T_1^p c_5 =  o(n),
\end{equation*}
имеем
\begin{multline*}
\sum\limits_{(i,j)\in D_n} \mathrm{corr}^2 (x_i, x_j) ={}\\
{}= \sum\limits_{j\neq i} \mathrm{corr}^2 (x_i, x_j) 
+o(n)= c_6 n + o(n),
\end{multline*}
где
$$
c_6 = \lim\limits_{n\to\infty} \fr{1}{n} \sum\limits_{j\neq i} \mathrm{corr}^2 (x_i, x_j) 
\leq 2 c_5.
$$
Полагая $C_\rho \hm= \sigma^2\sqrt{1 + c_6}$, получим, наконец,
\begin{equation}
\label{D1}
{\mathsf D} \left(\hat{R}(T_m) - R(T_m)\right)  =  2 n C_\rho^2 + o(n).
\end{equation}
Заметим, что из~(\ref{D2}), (\ref{D3}) и~(\ref{D1}) следует, что
\begin{equation}
\label{D5}
\sup\limits_{n} \fr{\sum\nolimits_{i=1}^n {\mathsf D} F[x_i, T_m]}{V_n^2} < \infty\,,
\end{equation}
где 
$$
V_n^2 = {\mathsf D} \sum\limits_{i=1}^n \left(F[x_i, T_m] \hm- {\mathsf E} F[x_i, T_m]\right).
$$
Кроме того, поскольку $F[x_i, T_m]$ по модулю ограничены величиной $\sigma^2 \hm+ 
T_m^2$, выполнено условие Линдеберга: для любого $\eps\hm>0$ при $n \hm\to \infty$
\begin{multline}
\label{D6}
\!\!\!\fr{1}{V_n^2}\sum\limits_{i=1}^n {\mathsf E} \left( \!\left( F\left[x_i, T_m\right]\! -\! {\mathsf E} F\left[x_i, T_m\right]\right)^2 
\Ik \left(\vert F\left[x_i, T_m\right] -{}\right.\right.\hspace*{-2.69505pt}\\
\left.\left.{}- {\mathsf E} F\left[x_i, T_m\right]\vert >\eps V_n\right)\!
\vphantom{\left( F\left[x_i, T_m\right]\! -\! {\mathsf E} F\left[x_i, T_m\right]\right)^2}
\right) 
\to  0\,.
\end{multline}
Из~(\ref{D1})--(\ref{D6}), очевидного неравенства
$$ 
\lim\limits_{k\to\infty} \sup\limits_{n\geq k+1}\rho(k) \equiv 
\lim\limits_{k\to\infty} \rho^\star (k)  < 1
$$
 и~центральной предельной теоремы для сильно перемешанных случайных величин~\cite{Peligrad} следует~(\ref{D0}).

Перейдем к~доказательству того, что $U(\hat{t}_F) \, n^{-1/2} \overset{\, \p \, }{\to} 0$.
Всюду далее, не ограничивая общности, полагаем $\sigma=1$. 
Введем обозначения:

\noindent
\begin{align*}
S_1(T) &= \sum\limits_{i: |\mu_i| > 1/T_1} H_i(T, T_m); \\
S_2(T) &= \sum\limits_{i: |\mu_i| \leq 1/T_1} H_i(T, T_m); 
\\
N_1(a, b) &= \sum\limits_{i: |\mu_i| > 1/T_1} \Ik (a<|x_i|\leq b); \\ 
N_2(a, b) &= \sum\limits_{i: |\mu_i| \leq 1/T_1} \Ik (a<|x_i|\leq b);
\end{align*}

\noindent
\begin{align*}
Z_l(T) &= S_l(T) - {\mathsf E} S_l(T),\enskip l = 1,2\,; \\  
d_n &= \fr{T_U -  t_{\kappa_n}}{n};\\
T_j^{\prime} &= t_{\kappa_n}+j d_n,\enskip j = \overline{0,n-1}\,.
\end{align*} 

\vspace*{-3pt}

\noindent
Для произвольного $\eps>0$

\vspace*{-3pt}

\noindent
\begin{multline}
\p \left( \fr{|U(\hat{t}_F)|}{\sqrt{n}}> 4\eps \right) \leq 
\p\left(\hat{t}_F \leq t_{\kappa_n}\right) + {}\\
{}+\p \left(\fr{\sup\nolimits_{T\in 
[t_{\kappa_n}, T_U]} |U(T)|}{\sqrt{n}}>4\eps \right)\leq  {}\\
{}\leq \p\left(\hat{t}_F \leq t_{\kappa_n}\right) + \p\left(\fr{\sup\nolimits_{T\in 
[t_{\kappa_n}, T_U]} |{\mathsf E} U(T)|}{\sqrt{n}}>\eps\right)+{}\\
{}+ \p \left(\sup\limits_{T\in [t_{\kappa_n}, T_U]} |Z_1(T)| > 
\eps\sqrt{n}\right) +{}\\
{}+ \p \left(\sup\limits_{j \in [0, n-1]} |Z_2(T_j^{\prime})| > 
\eps\sqrt{n}\right) +{}\\
{}+ \p \left(\sup\limits_{\substack{j \in [0, n-1] \\
 T\in [T_j^{\prime},T_j^{\prime}+d_n]}} |Z_2(T)-Z_2(T_j^{\prime})| > \eps\sqrt{n}\right).
\label{M1}
\end{multline}
Заметим, что $\gamma_n\hm > \ln^{-1} n$, $\kappa_n\hm > n \eta_n^p \ln ^{-1} n \hm\geq 
n^{1-\delta_1} \ln ^{-1} n$ и~$q_n\hm > c_4 \ln ^{-1} n /2$ для всех достаточно 
больших~$n$.
Для первого слагаемого в~(\ref{M1}) по лемме~1 с~$m \hm= n^{\delta_1} \ln 
^7 n$ для  больших~$n$ имеем

\vspace*{-3pt}

\noindent
\begin{multline}
\label{M1next}
\p\left(\hat{t}_F \leq t_{\kappa_n}\right)  = \p \left(\hat{k}_F \geq \kappa_n 
\right) \leq 4 n e^{-\ln^2 n} + {}\\
{}+n^{1+(3/2)\,\delta_1} \ln^9 n \, 
\alpha\left(\left[\fr{n^{1-\delta_1}}{\ln^{7} n}\right]\right) = o(1)
\end{multline}
при $n\to\infty$. 
Для оценки второго слагаемого в~(\ref{M1}) заметим, что при $T \hm\in 
[t_{\kappa_n}, T_U]$ справедливо
\begin{equation}
\label{M2}
{\mathsf E} H_i(T, T_m) \leq T_U^2 + 1.
\end{equation}
Если же кроме $T \hm\in [t_{\kappa_n}, T_U]$ также выполнено $|\mu_i| \hm\leq T_1^{-1}$, то

\vspace*{-6pt}

\noindent
\begin{multline*}
|{\mathsf E} H_i (T, T_m)| \leq 2 T_U^2 \, \p \left(|x_i| > t_{\kappa_n}\right) \leq {}\\
{}\leq2 
T_U^2 \, \p \left(|x_i-\mu_i| > t_{\kappa_n}-T_1^{-1}\right) \leq{}\\
{}\leq 2 T_U^2  \exp\left\{ -\fr{1}{2} \left(t_{\kappa_n} - T_1^{-
1}\right)^2 \right\}  \leq{}\\
{}\leq
 4 (\ln n)  \exp\left\{ -\fr{1}{2} 
\left(z\left(\fr{q_n\kappa_n}{2n}\right)\right)^2 + t_{\kappa_n} T_1^{-
1}\right\},
\end{multline*}

\vspace*{-2pt}

\noindent
где использовано неравенство 

\noindent
$$
2(1-\Phi(x))\hm \leq \fr{e^{-x^2/2}}{x}
$$

\pagebreak


\noindent
 для $x\hm\geq 0$ 
($\Phi(x)$~--- функция распределения $N(0,1)$). Рас\-смот\-рим выражение 
в~экспоненте. Второе слагаемое не превышает $1\hm+o(1)$ при $n\hm\to\infty$, поскольку 
$t_{\kappa_n} \hm\leq T_1 (1+o(1))$ при $\sigma\hm=1$, что нетрудно получить из 
определения~$t_{\kappa_n}$, пред\-став\-ле\-ния~(\ref{lem1eq2}) и~ограничения на~$q_n$ 
из формулировки тео\-ре\-мы. Для первого слагаемого, используя пред\-став\-ле\-ние~(\ref{lem1eq1}) 
и~ограничения, наложенные на~$q_n$, при больших~$n$ получим
\begin{multline*}
-\fr{1}{2}\left(z\left(\fr{q_n \kappa_n}{2n}\right)\right)^2 \leq - \ln 
\fr{2n (1-q_n-\gamma_n)}{q_n n \eta_n^p T_1^{-p}} + {}\\
{}+\fr{1}{2} \ln 
\left((1+o(1)) \ln \eta_n^{-p}\right) + \fr{3}{2} \leq{}\\
{}\leq \ln \fr{c_3}{1-c_3} + \ln \eta_n^p + \ln T_1^{-p} + \ln T_1 + 
\fr{3}{2}+ o(1).
\end{multline*}
Из приведенных соотношений следует, что с~некоторой константой $c_7 = c_7(c_3, 
p, \delta_1, \delta_2, c_4)$
\begin{equation}\label{M3}
\sup\limits_{\substack{i: |\mu_i| \leq 1/T_1 \\ T\in [t_{\kappa_n}, T_U]}} |{\mathsf E} 
H_i (T, T_m)|  \leq c_7 (\ln n)^{(3-p)/2}\eta_n^p.
\end{equation}
Из (\ref{M2}) и~(\ref{M3}) с~учетом $\delta_2 \hm> 1/2$ следует
\begin{multline*}
\sup\limits_{T\in [t_{\kappa_n}, T_U]} |{\mathsf E} U(T)| \leq{}\\
{}\leq 
 n\eta_n^p T_1^p 
(T_U^2+1) + c_7 (\ln n)^{(3-p)/2} n \eta_n^p = o(\sqrt{n})
\end{multline*}
при $n\to\infty$, а следовательно, для любого $\eps\hm>0$ второе слагаемое в~(\ref{M1}) обращается в~ноль для всех достаточно больших~$n$.

Далее, поскольку при $T \hm\leq T_U$ и~$\sigma\hm=1$
$$
|H_i(T, T_m) - {\mathsf E} H_i(T, T_m)| \leq 2 (T_U^2 +2), \enskip i=\overline{1, n}\,,
$$
а число слагаемых в~$Z_1(T)$ не превосходит $n\eta_n^p T_1^p$, имеем
$$
\sup\limits_{T\in [t_{\kappa_n}, T_U]} |Z_1(T)|  \leq 2 n\eta_n^p T_1^p (T_U^2 
+2) = o(\sqrt{n})
$$
при $n\to\infty$, а следовательно, для любого $\eps\hm>0$ и~третье слагаемое в~(\ref{M1}) обращается в~ноль для всех достаточно больших~$n$.

Перейдем к~оценке четвертого слагаемого в~(\ref{M1}). Аналогично~(\ref{M3}) 
можно получить:
\begin{multline}
\label{M10}
\!\!\sup\limits_{\substack{i: |\mu_i| \leq 1/T_1 \\ T\in [t_{\kappa_n}, T_U]}} \!{\mathsf D} 
H_i (T, T_m)  \leq \!\sup\limits_{\substack{i: |\mu_i| \leq 1/T_1 \\ T\in 
[t_{\kappa_n}, T_U]}} \!{\mathsf E} \left(H_i (T, T_m)\right)^2  \leq{}\\
{}\leq 2 c_7 (\ln n)^{(5-p)/2} \eta_n^p.
\end{multline}
По лемме~4 с~$m \hm= \sqrt{n} (\ln n)^3$ и~$k \hm= n-[n\eta_n^p T_1^p]$ 
для четвертого слагаемого в~(\ref{M1}) имеем:

\noindent
\begin{multline}
\p \left(\sup\limits_{j \in [0, n-1]} |Z_2(T_j^\prime)| > \eps\sqrt{n}\right) 
\leq {}\\
{}\leq \sum\limits_{j \in [0, n-1]} \hspace*{-3mm}\p \left( |Z_2(T_j^\prime)| > \varepsilon\sqrt{n}\right)\leq{}\\
{}\leq 4 n \exp \left\{ - \fr{\eps^2 n^{3/2} (\ln n)^3}{n-[n\eta_n^p T_1^p]}\!\Bigg/\! \big( 8 (T_U^2+2)\eps\sqrt{n} +{}\right.\\
\left.{}+ 128 c_7 (\ln n)^{(5-p)/2} \eta_n^p  (n-
[n\eta_n^p T_1^p])\big) 
\vphantom{ \fr{\eps^2 n^{3/2} (\ln n)^3}{n-[n\eta_n^p T_1^p]}}
\right\} +{}\\
{}
+ 22 \left(1+\fr{8(T_U^2+2) (n-[n\eta_n^p T_1^p])}{\eps 
\sqrt{n}}\right)^{1/2}\times{}\\
{}\times n^{3/2} (\ln n)^3 \alpha\left(\left[\fr{n-[n\eta_n^p 
T_1^p]}{2 (\ln n)^3 \sqrt{n}}\right]\right).
\label{M5}
\end{multline}
Используя ограничения $n^{-\delta_1}\hm\leq \eta_n^p \leq n^{-\delta_2}$ 
и~$1/2\hm<\delta_2\hm<\delta_1\hm<1$, из~(\ref{M5}) получим для любого $\eps\hm>0$
$$
\p \left(\sup\limits_{j \in [0, n-1]} |Z_2(T_j^\prime)| > \eps\sqrt{n}\right) 
\to 0
$$
при $n \to \infty$.

Рассмотрим, наконец, пятое слагаемое в~(\ref{M1})). Заметим, что при $0\hm< a \hm< b$ 
справедливо
$$
|Z_2(b)-Z_2(a)| \leq 2 |N_2(a,b)-{\mathsf E} N_2(a,b)| + n (b^2-a^2).
$$
Полагая $a = T_j^\prime$, $b \hm= T \hm\in [T_j^\prime, T_j^\prime+d_n]$ для 
произвольного $j \hm\in [0, n-1]$ и~учитывая, что
$$
(T^2 - (T_j^\prime )^2) = (T - T_j^\prime)(T+ T_j^\prime ) \leq  2 d_n T_U < 2 
T_U^2 n^{-1}; 
$$

\vspace*{-12pt}

\noindent
\begin{multline*}
\p\left(T_j^\prime < |x_i| \leq T \right) \leq \p\left(T_j^\prime < |x_i| \leq 
T_j^\prime+d_n\right) <{}\\
{}< d_n < T_U n^{-1}, 
\end{multline*}
получим  оценку
$$
|Z_2(T)-Z_2(T_j^\prime)| \leq 2 N_2(T_j^\prime, T) +  3 T_U^2 .
$$
Далее, поскольку $N_2 (T_j^\prime, T) \hm\leq N_2 (T_j^\prime, T_j^\prime+d_n)$ и~${\mathsf E} N_2 (T_j^\prime, T_j^\prime+d_n) \hm< T_U^2$,
имеем
\begin{multline*}
\sup\limits_{T \in [T_j^\prime, T_j^\prime+d_n]} |Z_2(T)-Z_2(T_j^\prime)| \leq {}\\
{}\leq
2 \left|N_2 (T_j^\prime, T_j^\prime+d_n) - {\mathsf E} N_2 (T_j^\prime, 
T_j^\prime+d_n)\right| +  5 T_U^2 .
\end{multline*}
Аналогично~(\ref{M3}) показывается, что
\begin{multline}
\label{M11}
\sup\limits_{\substack{i : |\mu_i| \leq 1/T_1 \\ j \in [0, n-1]}} {\mathsf D} \Ik 
(T_j^\prime < |x_i| \leq T_j^\prime + d_n) <{}\\
{}< c_7 (\ln n)^{(1-p)/2} \eta_n^p.
\end{multline}
Пусть $n > N(\eps)$ настолько, что 
$$
\fr{\eps\sqrt{n} - 5 T_U^2}{2} > \fr{\eps \sqrt{n} }{4}\,.
$$
%
 Тогда для пятого слагаемого в~(\ref{M1}) по лемме~4 с~$m \hm= 
\sqrt{n} (\ln n)^2$ и~$k \hm= n\hm-[n\eta_n^p T_1^p]$ имеем
\begin{multline}
\p \left(\sup\limits_{\substack{j \in [0, n-1] \\ T\in 
[T_j^{\prime},T_j^{\prime}+d_n]}} |Z_2(T)-Z_2(T_j^{\prime})| > 
\eps\sqrt{n}\right) \leq{}\\
{}\leq  \sum\limits_{j \in [0, n-1]} \p \left(  \left|N_2 (T_j^\prime, 
T_j^\prime+d_n) -{}\right.\right.\\
\left.\left.{}- {\mathsf E} N_2 (T_j^\prime, T_j^\prime+d_n)\right| > \fr{\eps\sqrt{n}}{4} 
\right) \leq{}\\
{}\leq  4n \exp \left\{ -  \fr{\eps^2 n^{3/2} (\ln n)^2}{(n-[n\eta_n^p T_1^p])^{-1}}\Bigg/ 
\big( 16 \eps \sqrt{n} +{}\right.\\
\left.{}+ 64 c_7 (\ln n)^{(1-p)/2} \eta_n^p (n-[n\eta_n^p 
T_1^p]) \big) 
\vphantom{\fr{\eps^2 n^{3/2} (\ln n)^2}{(n-[n\eta_n^p T_1^p])^{-1}}}
\right\} +{}\\
{}+ 22 \left(1+\fr{16 (n-[n\eta_n^p T_1^p])}{\eps \sqrt{n}}\right)^{1/2}\times{}\\
{}\times 
n^{3/2} (\ln n)^2 \alpha\left(\left[\fr{n-[n\eta_n^p T_1^p]}{2 (\ln n)^2 
\sqrt{n}}\right]\right).
\label{M6}
\end{multline}
Используя ограничения $n^{-\delta_1}\hm\leq \eta_n^p\hm \leq n^{-\delta_2}$ 
и~$1/2\hm<\delta_2\hm<\delta_1<1$, из~(\ref{M6}) получим для любого $\eps\hm>0$
$$
\p \left(\sup\limits_{\substack{j \in [0, n-1] \\ T\in 
[T_j^{\prime},T_j^{\prime}+d_n]}} |Z_2(T)-Z_2(T_j^{\prime})| > 
\eps\sqrt{n}\right) \to 0
$$
при $n \to \infty$.

Таким образом, показано, что для любого $\eps>0$ все слагаемые в~(\ref{M1}) 
стремятся к~нулю при $n\to\infty$. Следовательно,
$$
\fr{|U(\hat{t}_F)|}{\sqrt{n}}  \overset{\, \p \, }{\to} 0 \,,
$$
что вместе с~(\ref{D0}) завершает доказательство тео\-ремы.~\hfill$\square$

\smallskip

Следующая теорема дает достаточные условия для асимптотической нормальности 
оценки риска $\hat{R}(\hat{t}_F)$ в~случае $\mu \hm\in l_0[\eta_n]$.

\smallskip

\noindent
\textbf{Теорема 2.}\ 
\textit{Пусть $\mu \hm\in l_0[\eta_n]$, $\eta_n\hm\in[n^{-\delta_1}, n^{-\delta_2}]$, $1/2\hm < 
\delta_2\hm < \delta_1\hm<1;$ имеются такие константы $c_1, c_2\hm>0$, что для 
коэффициента сильного перемешивания $\alpha(\cdot)$ компонент вектора~$x$ 
справедливо} 
\begin{gather*}
\alpha(k) \leq c_1 k^{-1-(5/2)\delta_1/(1\hm-\delta_1)\hm-c_2},\enskip 
k=\overline{1,n-1};\\
 q_n < c_3 < 1;\enskip \mathrm{lim\,inf} q_n \ln n = c_4 > 0;
\end{gather*}
\textit{для максимального коэффициента корреляции~$\rho(\cdot)$ компонент вектора~$x$ 
справедливо}
$$
\sum\limits_{k = 1}^{\infty} \sup\limits_{n\geq k+1} \rho(k) \equiv 
\sum\limits_{k = 1}^{\infty}  \rho^\star (k) = c_5 < \infty. 
$$
\textit{Тогда при $n \to \infty$}
$$
\fr{\hat{R}(\hat{t}_F) - R(T_m)}{C_\rho \sqrt{2n}} \Rightarrow N(0, 1),
$$
\textit{где}
$$
C_\rho = \sigma^2\sqrt{1 +   \lim\limits_{n\to\infty} \fr{1}{n} 
\sum\limits_{j\neq i} \mathrm{corr}^2 (x_i, x_j)}\,.
$$

\noindent
Д\,о\,к\,а\,з\,а\,т\,е\,л\,ь\,с\,т\,в\,о\  проводится аналогично доказательству теоремы~1. 
Переменная~$D_n$ теперь определяется как $D_n \hm= \{(i,j) : 
|\mu_i|\hm=|\mu_j|=0$, $j\hm\neq i\}$. Условия вида $|\mu_i|\hm<T_1^{-1}$ (вида 
$|\mu_i|\hm\geq T_1^{-1}$) заменяются условиями  $\mu_i\hm=0$ (соответственно 
$|\mu_i|\hm>0$).
Поскольку $\mu \hm\in l_0[\eta_n]$, количество~$i$ таких, что $|\mu_i|\hm>0$ 
(а~значит, и~число слагаемых в~$Z_1(T)$), не превышает~$[n \eta_n]$.

Для оценки первого слагаемого в~(\ref{M1}) используется лемма~2, 
в~которой можно взять, например, $b\hm=1/2$, а~для~$\kappa_n^0$ использовать оценку 
$\kappa_n^0 \hm> n \eta_n$. Формулы (\ref{M3}),  (\ref{M10}) и~(\ref{M11}) 
принимают вид соответственно
\begin{align*}
\sup\limits_{\substack{i: \mu_i =0 \\ T\in [t_{\kappa_n^0}, T_U]}} |{\mathsf E} H_i (T, 
T_m)| & \leq c_8 (\ln n)^{3/2} \eta_n ;
\\
\sup\limits_{\substack{i: \mu_i =0 \\ T\in [t_{\kappa_n^0}, T_U]}} {\mathsf D} H_i (T, 
T_m)  & \leq 2 c_8 (\ln n)^{5/2} \eta_n;
\\
\sup\limits_{\substack{i : \mu_i =0 \\ j \in [0, n-1]}} {\mathsf D} \Ik (T_j^\prime < 
|x_i| \leq T_j^\prime + d_n) &< c_8 (\ln n)^{1/2} \eta_n,
\end{align*}
где $c_8 = c_8(c_3,\delta_1, \delta_2, c_4)$. В~остальном доказательство 
аналогично.~\hfill$\square$

\section{Сильная состоятельность оценки риска при~применении FDR-процедуры 
в~условиях слабой зависимости}

Следующая теорема дает достаточные условия для сильной состоятельности оценки 
риска $\hat{R}(\hat{t}_F)$ в~случаях $\mu \hm\in m_p[\eta_n]$ и~$\mu \hm\in 
l_0[\eta_n]$.

\smallskip

\noindent
\textbf{Теорема 3.}
\textit{Пусть $\mu\hm \in m_p[\eta_n]$, $\eta_n^p\hm\in[n^{-\delta_1}, n^{-\delta_2}]$ либо 
$\mu \hm\in l_0[\eta_n]$, $\eta_n\hm\in[n^{-\delta_1}, n^{-\delta_2}]$; $0 \hm< \delta_2 
\hm< \delta_1<1$; имеются такие константы $c_1, c_2\hm>0$, что для коэффициента 
сильного перемешивания $\alpha(\cdot)$ компонент вектора~$x$ справедливо}  
$\alpha(k) \hm\leq c_1 k^{-2-(7/2)\delta_1/(1\hm-\delta_1)\hm-c_2}$, $k\hm=\overline{1,n-1}$; 
$q_n \hm< c_3 \hm< 1$; $\mathrm{lim\,inf} q_n \ln n \hm= c_4 \hm> 0$. \textit{Тогда при} $n \hm\to \infty$
$$
\fr{\hat{R}(\hat{t}_F) - R(T_m)}{n} \rightarrow 0 \, \, \,\textit{п.~в.}
$$


\noindent
Д\,о\,к\,а\,з\,а\,т\,е\,л\,ь\,с\,т\,в\,о\,.  Воспользуемся представлением~(\ref{D00}).

Покажем, что $(\hat{R}(T_m)-R(T_m))n^{-1}\hm \to 0$ п.~в.\ при $n\hm\to\infty$. 
При мягкой пороговой обработке ${\mathsf E} \hat{R}(T_m) \hm= R(T_m)$, а~при жесткой 
пороговой обработке
\begin{multline*}
\fr{\hat{R}(T_m)-R(T_m)}{n} = {}\\
{}=\fr{\hat{R}(T_m)-{\mathsf E} \hat{R}(T_m)}{n} 
+\fr{{\mathsf E}\hat{R}(T_m)-R(T_m)}{n}\,,
\end{multline*}
где второе слагаемое стремится к~нулю при $n\to\infty$ \cite{Mallat}. 
Следовательно, достаточно показать, что $(\hat{R}(T_m)\hm-{\mathsf E}\hat{R}(T_m))n^{-1} \hm\to 0$ п.~в.

Полагая в~лемме~3 $X_i \hm= F[x_i, T_m] \hm- {\mathsf E} F[x_i, T_m]$, $b \hm= 
2(\sigma^2\hm+T_m^2)$ и~$m \hm= n^{1/4}$ и~учитывая ограничения на $\alpha(\cdot)$ из 
условия, нетрудно убедиться, что для всех~$n$
$$
\p \left(\left| \fr{\hat{R}(T_m)-{\mathsf E} \hat{R}(T_m)}{n}\right| >\eps \right) 
\leq \fr{c_5}{n^{1+c_6}}\,, 
$$
где константы $c_5$, $c_6$ положительны. Отсюда
$$
\sum\limits_{n=1}^{\infty}\p \left(\left|\fr{\hat{R}(T_m)-{\mathsf E} 
\hat{R}(T_m)}{n}\right| >\eps \right) < \infty,
$$
и по теореме~1.3.4 из~\cite{Serfling2002} 
$$
\left(\hat{R}(T_m)-{\mathsf E}\hat{R}(T_m)\right)n^{-1} \to 0~\mbox{п.~в.}
$$



Покажем теперь, что  $U(\hat{t}_F) \, n^{-1}\hm \to 0$ п.~в. Доказательство 
проведено для $\mu \hm\in m_p[\eta_n]$, в~случае $\mu\hm \in l_0[\eta_n]$ 
доказательство аналогично.
Аналогично формуле~(\ref{M1}), для произвольного $\eps\hm>0$ в~терминах тео\-ре\-мы~1 имеем
\begin{multline*}
\p \left( \fr{|U(\hat{t}_F)|}{n}> 4\eps \right) \leq \p\left(\hat{t}_F 
\leq t_{\kappa_n}\right) +{}\\
{}+ \p\left(\fr{\sup\nolimits_{T\in [t_{\kappa_n}, T_U]} |{\mathsf E} 
U(T)|}{n}>\eps\right)+{}\\
{}+ \p \left(\sup\limits_{T\in [t_{\kappa_n}, T_U]} |Z_1(T)| > \eps n\right) +{}
\end{multline*}

\noindent
\begin{multline}
{}+ \p  \left(\sup\limits_{j \in [0, n-1]} |Z_2(T_j^{\prime})| > \eps n\right) +{}\\
{}+ \p \left(\sup\limits_{\substack{j \in [0, n-1] \\ T\in 
[T_j^{\prime},T_j^{\prime}+d_n]}} |Z_2(T)-Z_2(T_j^{\prime})| > \eps n\right).
\label{M1SC}
\end{multline}
Применяя рассуждения, аналогичные приведенным в~доказательстве теоремы~1, можно показать, что
$$
\sup\limits_{T\in [t_{\kappa_n}, T_U]} |{\mathsf E} U(T)| = o(n); \enskip
\sup\limits_{T\in [t_{\kappa_n}, T_U]} |Z_1(T)|  = o(n),
$$
откуда следует, что второе и~третье слагаемые в~(\ref{M1SC}) обращаются в~ноль 
для всех достаточно больших~$n$.

Для некоторых положительных констант  $c_7$ и~$c_8$ первое, четвертое и~пятое 
слагаемые  в~(\ref{M1SC}) не превышают $c_7 n^{-1-c_8}$ для всех достаточно 
боль\-ших~$n$, что можно показать с~помощью ограничения на $\alpha(\cdot)$ из 
условия и~рассуждений, аналогичных приведенным при выводе соответственно формул~(\ref{M1next}), (\ref{M5}) и~(\ref{M6}), с~тем отличием, что при применении 
леммы~4 полагается $m \hm= (\ln n)^3$.

Из доказанного следует, что
$$
\sum\limits_{n=1}^{\infty}\p \left( \fr{|U(\hat{t}_F)|}{n}> 4\eps \right) 
< \infty,
$$
и по теореме~1.3.4 из~\cite{Serfling2002} $U(\hat{t}_F) \, n^{-1} \to 0$ п.~в., 
что завершает доказательство теоремы.~\hfill$\square$



{\small\frenchspacing
 {\baselineskip=11.5pt
 %\addcontentsline{toc}{section}{References}
 \begin{thebibliography}{99}
\bibitem{FDRImage}
\Au{Krylov V.\,A., Moser~G., Serpico~S.\,B., Zerubia~J.}
False discovery rate approach to unsupervised image change detection~// IEEE 
T. Image Process., 2016. Vol.~25. No.\,10. P.~4704--4718. doi: 10.1109/TIP.2016.2593340.

\bibitem{MultipleTesting} %2
\Au{Menyhart~O., Weltz~B., Gyorffy~B.}
MultipleTesting.com: A~tool for life science researchers for multiple hypothesis 
testing correction~// PLoS One, 2021. Vol.~16. No.\,6. Art.~0245824. doi: 10.1371/journal.pone.0245824.

\bibitem{AdaptingFDR} %3
\Au{Abramovich~F., Benjamini~Y., Donoho~D., Johnstone~I.}
Adapting to unknown sparsity by controlling the false discovery rate~// Ann. Stat., 2006. Vol.~34. No.\,2. P.~584--653.
doi: 10.1214/009053606000000074.

\bibitem{ZasShe17} %4
\Au{Заспа~А.\,Ю., Шестаков~О.\,В.}
Состоятельность оценки риска при множественной проверке гипотез с~FDR-по\-ро\-гом~// 
Вестник ТвГУ. Сер. Прикладная математика, 2017. Вып.~1. С.~5--16.
doi: 10.26456/vtpmk119. EDN: YFYJXT.

\bibitem{Mathematics2020} %5
\Au{Palionnaya~S.\,I., Shestakov~O.\,V.}
Asymptotic properties of MSE estimate for the false discovery rate controlling 
procedures in multiple hypothesis testing // Mathematics, 2020. Vol.~8. No.~11. 
Art.~1913. 11~p. doi: 10.3390/ math8111913.

\bibitem{Shestakov2021-1} %6
\Au{Шестаков~О.\,В.}
Анализ несмещенной оценки среднеквадратичного риска метода блочной пороговой 
обработки~// Информатика и~её применения, 2021. Т.~15. Вып.~2. С.~30--35.
doi: 10.14357/19922264210205. EDN: DSQQAU.

\bibitem{Shestakov2021-2} %7
\Au{Шестаков~О.\,В.}
Пороговые функции в~методах подавления шума, основанных на вейв\-лет-раз\-ло\-же\-нии 
сигнала~// Информатика и~её применения, 2021. Т.~15. Вып.~3. С.~51--56.
doi: 10.14357/19922264210307. EDN: WSEAYG.

\bibitem{Shestakov2022} %8
\Au{Шестаков~О.\,В.}
Несмещенная оценка риска пороговой обработки с~двумя пороговыми значениями~// 
Информатика и~её применения, 2022. Т.~16. Вып.~4. С.~14--19.
doi: 10.14357/19922264220403. EDN: \mbox{DZBVLC}.

\bibitem{ResultsOnFDRUnderDependence} %9
\Au{Farcomeni~A.}
Some results on the control of the false discovery rate under dependence~// 
Scand. J. Stat., 2007. Vol.~34. No.\,2. P.~275--297.
doi: 10.1111/j.1467-9469.2006.00530.x.

\bibitem{VorontsovShestakov2023} %10
\Au{Воронцов~М.\,О., Шестаков~О.\,В.}
Среднеквадратичный риск FDR-про\-це\-ду\-ры в~условиях слабой за\-ви\-си\-мости~// 
Информатика и~её применения, 2023. Т.~17. Вып.~2. С.~34--40.
doi: 10.14357/19922264230205. EDN: AVJZDX.

\bibitem{Vorontsov2024} %11
\Au{Воронцов~М.\,О.}
Анализ среднеквадратичного риска при использовании методов множественной 
проверки гипотез для выбора параметров пороговой обработки в~условиях слабой 
зависимости~// Вестник Московского университета. Сер. 15: Вычислительная 
математика и~кибернетика, 2024. №\,2. С.~18--24.

\bibitem{Bosq} %12
\Au{Bosq~D.}
Nonparametric statistics for stochastic processes: Estimation and prediction.~--- 
Lecture notes in statistics ser.~--- New York, NY, USA: Springer, 1996. Vol.~110. 
188~p.

\bibitem{Mallat} %13
\Au{Mallat~S.}
A wavelet tour of signal processing.~--- New York, NY, USA: Academic Press, 1999. 
857~p.

\bibitem{spatialAdaptation} %14
\Au{Donoho~D., Johnstone~I.}
Ideal spatial adaptation via wavelet shrinkage~// Biometrika, 1994. Vol.~81. 
No.\,3. P.~425--455. doi: 10.1093/biomet/81.3.425.

\bibitem{AdaptingSURE} %15
\Au{Donoho D., Johnstone I.\,M.}
Adapting to unknown smoothness via wavelet shrinkage~// J.~Amer. Stat. Assoc., 
1995. Vol.~90. P.~1200--1224.

\bibitem{ExactRisk} %16
\Au{Marron J.\,S., Adak~S., Johnstone~I.\,M., Neumann~M.\,H., Patil~P.}
Exact risk analysis of wavelet regression~// J.~Comput. Graph. Stat., 1998. 
Vol.~7. P.~278--309. doi: 10.1080/ 10618600.1998.10474777.

\bibitem{Jansen} %17
\Au{Jansen~M.}
Noise reduction by wavelet thresholding.~-- Lecture notes in statistics ser.~--- 
New York, NY, USA: Springer, 2001. Vol.~161. 217~p.

\bibitem{KuShe2016_1} %18
\Au{Кудрявцев~А.\,А., Шестаков~О.\,В.}
Асимптотическое поведение порога, минимизирующего усредненную\linebreak вероятность ошибки 
вычисления вейв\-лет-ко\-эф\-фи\-ци\-ен\-тов~// Докл. Акад. наук, 2016. Т.~468. №\,5. 
С.~487--491.

\bibitem{KuShe2016_2} %19
\Au{Кудрявцев~А.\,А., Шестаков~О.\,В.}
Асимптотически оптимальная пороговая обработка вейв\-лет-ко\-эф\-фи\-ци\-ен\-тов в~моделях с~негауссовым распределением шума~// Докл. Акад. наук, 2016. Т.~471. №\,1. 
С.~11--15.



\bibitem{Eroshenko} %20
\Au{Ерошенко~А.\,А.}
Статистические свойства оценок сигналов и~изображений при пороговой обработке 
коэффициентов в~вейв\-лет-раз\-ло\-же\-ни\-ях: Дис.\ \ldots\ канд. физ.-мат. наук.~--- 
М.: МГУ, 2015. 82~с.

\bibitem{Peligrad} %21
\Au{Peligrad~M.}
On the asymptotic normality of sequences of weak dependent random variables~// 
J. Theor. Probab., 1996. Vol.~9. No.\,3. P.~703--715. doi: 10.1007/BF02214083.

\bibitem{Serfling2002} %22
\Au{Serfling~R.\,J.}
Approximation theorems of mathematical statistics.~--- New York, NY, USA: John Wiley \&~Sons, Inc., 2002. 371~p.

\end{thebibliography}

 }
 }

\end{multicols}

\vspace*{-6pt}

\hfill{\small\textit{Поступила в~редакцию 21.05.24}}

\vspace*{8pt}

%\pagebreak

%\newpage

%\vspace*{-28pt}

\hrule

\vspace*{2pt}

\hrule



\def\tit{ASYMPTOTIC NORMALITY AND STRONG CONSISTENCY\\ OF~RISK ESTIMATE WHEN USING THE~FDR THRESHOLD\\ UNDER WEAK DEPENDENCE CONDITION}


\def\titkol{Asymptotic normality and strong consistency of~risk estimate when using the~FDR threshold under weak dependence condition}


\def\aut{M.\,O.~Vorontsov$^{1,2}$ and~O.\,V.~Shestakov$^{1,2,3}$}

\def\autkol{M.\,O.~Vorontsov and~O.\,V.~Shestakov}

\titel{\tit}{\aut}{\autkol}{\titkol}

\vspace*{-13pt}


\noindent
$^{1}$Department of Mathematical Statistics, Faculty of Computational Mathematics and Cybernetics,
 M.\,V.~Lo\-mo-\linebreak
 $\hphantom{^1}$nosov Moscow State University, 1-52~Leninskie Gory, GSP-1, Moscow 119991, Russian Federation

\noindent
$^{2}$Moscow Center for Fundamental and Applied Mathematics, M.\,V.~Lomonosov Moscow State University,\linebreak
$\hphantom{^1}$1~Leninskie Gory, GSP-1, Moscow 119991, Russian Federation

\noindent
$^{3}$Federal Research Center ``Computer Science and Control'' of the Russian Academy of Sciences, 44-2~Vavilov\linebreak
$\hphantom{^1}$Str., Moscow 119333, Russian Federation


\def\leftfootline{\small{\textbf{\thepage}
\hfill INFORMATIKA I EE PRIMENENIYA~--- INFORMATICS AND
APPLICATIONS\ \ \ 2024\ \ \ volume~18\ \ \ issue\ 3}
}%
 \def\rightfootline{\small{INFORMATIKA I EE PRIMENENIYA~---
INFORMATICS AND APPLICATIONS\ \ \ 2024\ \ \ volume~18\ \ \ issue\ 3
\hfill \textbf{\thepage}}}

\vspace*{2pt}






\Abste{An approach to solving the problem of noise removal in a large array of sparse data is considered
 based on the method of controlling the average proportion of false hypothesis rejections (False Discovery Rate, FDR). 
 This approach is equivalent to threshold processing procedures that remove array components whose values do not exceed 
 some specified threshold. The observations in the model are considered weakly dependent. To control the\linebreak\vspace*{-12pt}}
 
 \Abstend{degree of dependence, 
 restrictions on the strong mixing coefficient and the maximum correlation coefficient are used. The mean-square risk is 
 used as a measure of the effectiveness of the considered approach. It is possible to calculate the risk value only on the test data;
  therefore, its statistical estimate is considered in the work and its properties are investigated. The asymptotic normality and
   strong consistency of the risk estimate are proved when using the FDR threshold under conditions of weak dependence in the data.}

\KWE{thresholding; multiple hypothesis testing; risk estimate}

\DOI{10.14357/19922264240309}{ZOQVTO}

%\vspace*{-12pt}


    
   %   \Ack

%\vspace*{-3pt}
%\noindent



  \begin{multicols}{2}

\renewcommand{\bibname}{\protect\rmfamily References}
%\renewcommand{\bibname}{\large\protect\rm References}

{\small\frenchspacing
 {\baselineskip=10.8pt
 \addcontentsline{toc}{section}{References}
 \begin{thebibliography}{99} 

%1
\bibitem{FDRImage-1}
\Aue{Krylov, V.\,A., G.~Moser, S.\,B.~Serpico, and J.~Zerubia.} 2016. 
False discovery rate approach to unsupervised image change detection. 
\textit{IEEE T. Image Process.} 25(10):4704--4718. doi: 10.1109/TIP.2016.2593340.

%2
\bibitem{MultipleTesting-1}
\Aue{Menyhart, O., B.~Weltz, and B.~Gyorffy.} 2021. 
MultipleTesting.com: A~tool for life science researchers for multiple hypothesis testing correction. 
\textit{PLoS One} 16(6):0245824. 
doi: 10.1371/journal.pone.0245824.

%3
\bibitem{AdaptingFDR-1}
\Aue{Abramovich, F., Y.~Benjamini, D.~Donoho, and I.\,M.~Johnstone.} 2006. 
Adapting to unknown sparsity by controlling the false discovery rate. 
\textit{Ann. Stat.} 34(2):584--653. 
doi: 10.1214/009053606000000074.


%4
\bibitem{ZasShe17-1}
\Aue{Zaspa, A.\,Yu., and O.\,V.~Shestakov.} 2017.
Sostoyatel'nost' otsenki riska pri mnozhestvennoy proverke gipotez s~FDR-porogom
 [Consistency of the risk estimate of the multiple hypothesis testing with the FDR threshold]. 
\textit{Vestnik TvGU. Ser.: Prikladnaya matematika} [Herald of Tver State University. Ser. Applied Mathematics] 1:5--16.
doi: 10.26456/vtpmk119. EDN: YFYJXT.

%5
\bibitem{Mathematics2020-1}
\Aue{Palionnaya, S.\,I., and O.\,V.~Shestakov.} 2020. 
Asymptotic properties of MSE estimate for the false discovery rate controlling procedures in multiple hypothesis testing. 
\textit{Mathematics} 8(11):1913. 11~p.
doi: 10.3390/math8111913.

%6
\bibitem{Shestakov2021-1-1}
\Aue{Shestakov, O.\,V.} 2021.
Analiz nesmeshchennoy otsenki srednekvadratichnogo riska metoda blochnoy po\-ro\-go\-voy obrabotki 
[Analysis of the unbiased mean-square risk estimate of the block thresholding method]. 
\textit{Informatika i~ee Primeneniya~--- Inform. Appl.} 15(2):30--35.
doi: 10.14357/19922264210205. EDN: DSQQAU.

%7
\bibitem{Shestakov2021-2-1}
\Aue{Shestakov, O.\,V.} 2021.
Porogovye funktsii v~metodakh podavleniya shuma, osnovannykh na veyvlet-razlozhenii signala 
[Thresholding functions in the noise suppression methods based on the wavelet expansion of the signal]. 
\textit{Informatika i~ee Primeneniya~--- Inform. Appl.} 15(3):51--56.
doi: 10.14357/19922264210307. EDN: WSEAYG.

%8
\bibitem{Shestakov2022-1}
\Aue{Shestakov, O.\,V.} 2022.
Nesmeshchennaya otsenka riska porogovoy obrabotki s dvumya porogovymi znacheniyami [Unbiased thresholding risk estimate with two threshold values]. 
\textit{Informatika i~ee Primeneniya~--- Inform. Appl.} 16(4):14--19.
doi: 10.14357/19922264220403. EDN: DZBVLC.

%9
\bibitem{ResultsOnFDRUnderDependence-1}
\Aue{Farcomeni, A.} 2007. Some results on the control of the false discovery rate under dependence. 
\textit{Scand. J. Stat.} 34(2):275--297. 
doi: 10.1111/j.1467-9469.2006.00530.x.

%10
\bibitem{VorontsovShestakov2023-1}
\Aue{Vorontsov, M.\,O., and O.\,V.~Shestakov.} 2023.
Sred\-ne\-kvad\-ra\-tich\-nyy risk FDR-protsedury v~usloviyakh slaboy za\-vi\-si\-mosti [Mean-square risk of the FDR procedure under weak dependence]. 
\textit{Informatika i~ee Primeneniya~--- Inform. Appl.} 17(2):34--40.
doi: 10.14357/19922264230205. EDN: AVJZDX.

%11
\bibitem{Vorontsov2024-1}
\Aue{Vorontsov, M.\,O.} 2024. 
RMS risk analysis when using multiple hypothesis testing select parameters of thresholding under conditions of weak dependence. 
\textit{Moscow University Computational Mathematics Cybernetics} 48:91--97. 
doi: 10.3103/S027864192470002X.

%12
\bibitem{Bosq-1}
\Aue{Bosq, D.} 1996. 
\textit{Nonparametric statistics for stochastic processes: Estimation and prediction}. 
Lecture notes in statistics ser. New York, NY: Springer Verlag. Vol.~110. 188~p.

%13
\bibitem{Mallat-1}
\Aue{Mallat, S.} 1999. 
\textit{A wavelet tour of signal processing}. New York, NY: Academic Press. 857~p.

%14
\bibitem{spatialAdaptation-1}
\Aue{Donoho, D., and I.\,M.~Johnstone.} 1994. 
Ideal spatial adaptation via wavelet shrinkage. 
\textit{Biometrika} 81(3):425--455. doi: 10.1093/biomet/81.3.425.

%15
\bibitem{AdaptingSURE-1}
\Aue{Donoho, D., and I.\,M.~Johnstone.} 1995. 
Adapting to unknown smoothness via wavelet shrinkage. 
\textit{J. Am. Stat. Assoc.} 90(432):1200--1224. doi: 10.1080/01621459. 1995.10476626.

%16
\bibitem{ExactRisk-1}
\Aue{Marron, J.\,S., S.~Adak, I.\,M.~Johnstone, M.\,H.~Neumann, and P.~Patil.} 1998. 
Exact risk analysis of wavelet regression. 
\textit{J.~Comput. Graph. Stat.} 7(3):278-309. doi: 10.1080/ 10618600.1998.10474777.

%17
\bibitem{Jansen-1}
\Aue{Jansen, M.} 2001. 
\textit{Noise reduction by wavelet thresholding}. Lecture notes in statistics ser. New York, NY: Springer Verlag. Vol.~161. 217~p.

%18
\bibitem{KuShe2016_1-1}
\Aue{Kudryavtsev, A.\,A., and O.\,V.~Shestakov.} 2016. 
Asymptotic behavior of the threshold minimizing the average probability of error in calculation of wavelet coefficients. 
\textit{Dokl. Math.} 93(3):295--299.
doi: 10.1134/S1064562416030212. EDN: WUMUEV. 

%19
\bibitem{KuShe2016_2-1}
\Aue{Kudryavtsev, A.\,A., and O.\,V.~Shestakov.} 2016. 
Asymptotically optimal wavelet thresholding in the models with non-Gaussian noise distributions. 
\textit{Dokl. Math.} 94(3):615--619.
doi: 10.1134/S1064562416060028. EDN: YUYVUP.




%20
\bibitem{Eroshenko-1}
\Aue{Eroshenko, A.\,A.} 2015. Statisticheskie svoystva otsenok signalov i~izobrazheniy pri porogovoy obrabotke ko\-ef\-fi\-tsi\-en\-tov 
v~veyvlet-razlozheniyakh 
[Statistical properties of signal and image estimates under thresholding of coefficients in wavelet decompositions]. Moscow: MSU. PhD Diss. 82~p.

%21
\bibitem{Peligrad-1}
\Aue{Peligrad, M.} 1996. 
On the asymptotic normality of sequences of weak dependent random variables. 
\textit{J. Theor. Probab.} 9(3):703--715. doi: 10.1007/BF02214083.

%22
\bibitem{Serfling2002-1}
\Aue{Serfling, R.\,J.} 2002. 
\textit{Approximation theorems of mathematical statistics}. New York, NY: John Wiley \&~Sons. 371~p.
\end{thebibliography}

 }
 }

\end{multicols}

\vspace*{-6pt}

\hfill{\small\textit{Received May 21, 2024}} 

%\vspace*{-18pt}

\Contr

\vspace*{-3pt}


\noindent
\textbf{Vorontsov Mikhail O.} (b.\ 1996)~--- PhD student, Department of Mathematical Statistics, 
Faculty of Computational Mathematics and Cybernetics, M.\,V.~Lomonosov Moscow State University, 1-52~Leninskie Gory, GSP-1, Moscow 119991, Russian Federation;  
mathematician, Moscow Center for Fundamental and Applied Mathematics, M.\,V.~Lomonosov Moscow State University, 1~Leninskie Gory, GSP-1, Moscow 119991, Russian Federation;
\mbox{m.vtsov@mail.ru}

\vspace*{6pt}

\noindent
\textbf{Shestakov Oleg V.} (b.\ 1976)~--- Doctor of Science in physics and mathematics, professor, Department of Mathematical Statistics,
 Faculty of Computational Mathematics and Cybernetics, M.\,V.~Lomonosov Moscow State University, 1-52~Leninskie Gory, GSP-1, Moscow 119991, Russian Federation; 
 senior scientist, Federal Research Center ``Computer Science and Control'' of the Russian Academy of Sciences, 44-2~Vavilov Str., Moscow 119333, 
 Russian Federation; leading scientist, Moscow Center for Fundamental and Applied Mathematics, M.\,V.~Lomonosov Moscow State University, 
 1~Leninskie Gory, GSP-1, Moscow 119991, Russian Federation; \mbox{oshestakov@cs.msu.su}


\label{end\stat}

\renewcommand{\bibname}{\protect\rm Литература}  %9

\def\stat{shnurkov}

\def\tit{АНАЛИТИЧЕСКОЕ РЕШЕНИЕ ЗАДАЧИ ОПТИМАЛЬНОГО УПРАВЛЕНИЯ ПОЛУМАРКОВСКИМ ПРОЦЕССОМ\\ 
С~КОНЕЧНЫМ МНОЖЕСТВОМ СОСТОЯНИЙ$^*$}

\def\titkol{Аналитическое решение задачи оптимального управления полумарковским 
процессом} %с~конечным множеством состояний}

\def\aut{П.\,В.~Шнурков$^1$, А.\,К.~Горшенин$^2$, В.\,В.~Белоусов$^3$}

\def\autkol{П.\,В.~Шнурков, А.\,К.~Горшенин, В.\,В.~Белоусов}

\titel{\tit}{\aut}{\autkol}{\titkol}

\index{Шнурков П.\,В.}
\index{Горшенин А.\,К.}
\index{Белоусов В.\,В.}
\index{Shnurkov P.\,V.}
\index{Gorshenin A.\,K.}
\index{Belousov V.\,V.}


{\renewcommand{\thefootnote}{\fnsymbol{footnote}} \footnotetext[1]
{Работа выполнена при частичной поддержке РФФИ (проект 15-07-05316).}}


\renewcommand{\thefootnote}{\arabic{footnote}}
\footnotetext[1]{Национальный исследовательский университет <<Высшая школа экономики>>, 
\mbox{pshnurkov@hse.ru}}
\footnotetext[2]{Институт проблем информатики Федерального исследовательского центра <<Информатика 
и~управ\-ле\-ние>> Российской академии наук, \mbox{agorshenin@frccsc.ru}}
\footnotetext[3]{Институт проблем информатики Федерального исследовательского центра <<Информатика 
и~управление>> Российской академии наук, \mbox{vbelousov@ipiran.ru}}

%\vspace*{-6pt}

\Abst{Настоящее исследование посвящено теоретическому обоснованию нового метода 
нахождения оптимальной стратегии управления полумарковским процессом с~конечным 
множеством состояний. Рассматриваются марковские рандомизированные стратегии 
управления, определяемые конечным набором вероятностных мер, соответствующих 
каждому состоянию. Характеристикой качества управления служит стационарный 
стоимостной показатель. Данный показатель представляет собой дроб\-но-ли\-ней\-ный 
интегральный функционал от набора вероятностных мер, задающих стратегию управления. 
Для этого функционала известны явные аналитические представления подынтегральных 
функций числителя и~знаменателя. Дальнейшие результаты основываются на новой 
усиленной и~обобщенной форме теоремы об экстремуме дроб\-но-ли\-ней\-но\-го интегрального 
функционала. Доказывается, что проблемы существования оптимальной стратегии управления 
полумарковским процессом и~ее нахождения сводятся к~задаче численного исследования 
на глобальный экстремум заданной функции от конечного числа вещественных переменных.}

\KW{оптимальное управление полумарковским процессом; стационарный стоимостной 
показатель качества управления; дроб\-но-ли\-ней\-ный интегральный функционал}

\DOI{10.14357/19922264160408} 

\vspace*{9pt}


\vskip 10pt plus 9pt minus 6pt

\thispagestyle{headings}

\begin{multicols}{2}

\label{st\stat}

\section{Введение}

Теория оптимального управления марковскими и~полумарковскими случайными 
процессами интенсивно развивается с~начала 1960-х~гг. Еще в~первых 
основополагающих исследованиях рассматривались не только проблемы существования 
оптимальных стратегий управления, но и~способы нахождения этих стратегий. 

Для решения таких проблем, имеющих алгоритмическое содержание, использовались 
открытые незадолго до этого мощные методы прикладной математики: линейное 
программирование и~динамическое программирование. Отметим, прежде всего, 
классическую работу Р.~Ховарда~\cite{1}, в~которой метод динамического 
программирования был применен для решения проблемы оптимального управления 
марковским процессом с~непрерывным временем. В~дальнейшем В.\,В.~Рыков~\cite{2} 
доказал, что для аналогичной модели управления марковским процессом с~учетом 
переоценки оптимальной стратегией также является стационарная.

Важную роль в~развитии теории управления случайными процессами сыграла работа 
В.~Джевелла~\cite{3}, в~которой были впервые рассмотрены полумарковские модели 
управления для вариантов с~переоценкой и~без переоценки. Данная работа была 
переведена на русский язык и~послужила основой для многих последующих работ 
отечественных и~зарубежных специалистов. В~частности, Б.~Фокс показал~\cite{4}, 
что оптимальной стратегией управления полумарковским процессом в~варианте без 
переоценки является стационарная; аналогичные результаты были получены Э.~Денардо 
и~для варианта с~переоценкой~\cite{5}.

Среди последующих исследований алгоритмической направленности отметим работы 
Р.~Ховарда~\cite{6}, Б.~Фокса~\cite{4}, а также С.~Осаки и~Х.~Майна~\cite{7}. 
В~этих работах для нахождения оптимальных стратегий управления полумарковскими 
процессами использовался метод линейного программирования.

В 1970~г.\ была опубликована фундаментальная монография Х.~Майна и~С.~Осаки~\cite{8}, 
переведенная на русский язык в~1977~г., в~которой были систе\-ма\-ти\-зи\-ро\-ва\-ны и~изложены 
основные результаты по теории оптимального управления марковскими и~полумарковскими 
случайными процессами. Фактически данная книга стала итогом исследований по проблемам 
стохастического управления\linebreak
 за~10~лет. Отметим, что в~этой монографии рас\-смат\-ри\-ва\-лись 
марковские и~полумарковские модели управления с~конечными множествами состояний 
и~допустимых решений, принимаемых \mbox{в~каждом} состоянии. Были получены принципиальные 
тео\-ре\-ти\-че\-ские результаты, заключающиеся в~том, что оптимальные стратегии управ\-ле\-ния 
для основных видов рас\-смат\-ри\-ва\-емых моделей с~переоценкой и~без переоценки являются 
детерминированными и~стационарными. Были разработаны и~обоснованы процедуры нахождения 
оптимальных стратегий управления. В~частности, для модели управления полумарковским 
процессом без переоценки, когда множество со\-сто\-яний образует один эргодический класс, 
а~показатель качества управления пред\-став\-ля\-ет собой стационарный средний удельный 
доход (см.~[8, гл.~5, п.~5.5]), процедура поиска оптимальной рандомизированной 
стратегии осуществлялась методом линейного программирования. Обратим особое внимание 
на данный результат, поскольку аналогичная модель управления полумарковским 
процессом будет рассмотрена в~настоящей работе.

Принципиальную роль в~развитии теории стохастического управления сыграла 
монография И.\,И.~Гихмана и~А.\,В.~Скорохода~\cite{9}. В~этой книге были впервые 
систематически изложены основы теории оптимального управления случайными процессами 
с~дискретным и~непрерывным временем, включая теорию управления процессами, которые 
описываются стохастическими дифференциальными уравнениями. Отдельно были рас\-смот\-ре\-ны 
проблемы управления марковскими процессами с~дискретным временем и~скачкообразными 
марковскими процессами с~непрерывным временем. Роли множеств состояний и~допустимых 
управ\-ле\-ний играли пространства весьма общей структуры. Для широких классов функционалов 
качества управ\-ле\-ния (так называемых эволюционных функционалов в~марковских моделях 
с~дискретным временем и~интегральных функционалов накопления в~марковских моделях 
с~непрерывным временем) были доказаны теоремы о~существовании и~формах пред\-став\-ле\-ния 
оптимальных стратегий управ\-ле\-ния. Было установлено, что для однородных марковских 
моделей оптимальные стратегии управ\-ле\-ния существуют, являются стационарными 
и~детерминированными. Иначе говоря, такие стратегии задаются детерминированными 
функциями, аргументом которых является со\-сто\-яние сис\-те\-мы в~момент принятия решения, 
и~не зависящими от самого момента принятия решения. Что же касается важного вопроса 
о~формах представления этих функций, то их можно охарактеризовать следующим образом. 
Были найдены функциональные уравнения, осложненные условием экстремума, которым 
удовле\-тво\-ря\-ют упомянутые функции. По существу эти соотношения пред\-став\-ля\-ют собой 
уравнения Беллмана для соответствующих динамических стохастических моделей.

Особо отметим, что в~монографии~\cite{9} не рас\-смат\-ри\-ва\-лись проблемы управления 
полумарковскими процессами. Однако дальнейшее развитие общей теории управления 
такими процессами шло по пути, идейно намеченному в~указанной книге.

В последующие годы развитие теории управ\-ле\-ния полумарковскими процессами 
осуществля-\linebreak лось по направлению усложнения моделей и~обобщения исходных предположений. 
Например,\linebreak в~работах~\cite{10, 11} рассмотрены управляемые по\-лумарковские процессы при 
весьма общих предположениях относительно характера пространств состояний и~управлений. 
Проблемы управления исследовались по отношению к~различным видам целевых показателей, 
обобщающих упомянутый выше стационарный показатель средней удельной прибыли. В~этих 
работах доказывается, что оптимальная стратегия управления по отношению к~каж\-до\-му из 
показателей существует и~является одной и~той же стационарной детерминированной 
стратегией, определяемой некоторой функцией, заданной на множестве со\-сто\-яний процесса. 
Об этой функции известно лишь то, что она удовлетворяет некоторому интегральному 
уравнению, которое по содержанию пред\-став\-ля\-ет собой уравнение Бел\-лма\-на для 
соответствующей задачи управ\-ления.

Среди исследований, предшествовавших настоящему, отметим работу 
В.\,А.~Каштанова~[12, гл. 13]. В этом разделе коллективной монографии~\cite{12} 
автором была рассмотрена проблема оптимального управления полумарковским 
процессом с~конечным множеством состояний и~множеством возможных решений, 
которое представляет собой произвольный интервал множества вещественных чисел. 
Модель относится к~виду моделей без переоценки, показателем качества управления 
служит стационарное значение среднего удельного дохода, определяемое аналогично 
классическим работам~\cite{3, 8}. Рандомизированное управление в~каждом состоянии 
определяется в~соответствии с~вероятностным распределением, совокупность которых 
задает\linebreak
 стратегию управления. В.\,А.~Каш\-та\-но\-вым было\linebreak сформулировано утверждение о том, 
что стацио\-нарное значение среднего удельного дохода представляет собой 
дроб\-но-ли\-ней\-ный интегральный функционал от набора вероятностных распределений, 
образующих стратегию управления. При этом\linebreak ранее~[12, гл.~10; 13] было уста\-нов\-ле\-но, 
что дроб\-но-ли\-ней\-ный функционал достигает экстремума на вырожденных распределениях. 
Отсюда естест-\linebreak венно следует, что оптимальная стратегия управ\-ле-ния является 
детерминированной и~должна\linebreak определяться точкой экстремума функции, представляющей 
собой отношение подынтегральных функций чис\-ли\-те\-ля и~знаменателя данного 
дроб\-но-ли\-ней\-но\-го функционала. Однако в~\cite{12} не были получены явные 
представления для указан-\linebreak ных функций. Кроме того, приведенный в~гл.~10 
монографии~\cite{12} вариант теоремы об экстремуме дроб\-но-ли\-ней\-но\-го 
интегрального функционала требовал проверки выполнения условия существования 
этого экстремума. Такие условия указаны не были. В~связи с~этими обстоятельствами 
использовать полученные в~\cite{12} результаты для доказательства существования 
оптимальной детерминированной стратегии управ\-ле\-ния полумарковским процессом и~для 
строгого обоснования способа нахождения такой стратегии оказалось невозможным.

Настоящее исследование посвящено теоретическому обоснованию нового метода 
нахождения\linebreak оптимальной стратегии управления полумарковским процессом с~конечным 
множеством со\-сто\-яний. Рассматриваются марковские рандомизи\-рованные стратегии 
управления, определяемые конеч\-ным набором вероятностных мер, соответствующих 
каждому состоянию. Показателем качества управления служит уже упоминавшийся 
классический  показатель~--- стационарное значение средней удельной прибыли. 
Доказано, что этот показатель представляет собой дроб\-но-ли\-ней\-ный интегральный 
функционал от набора вероятностных мер, задающих стратегию управления. При этом, 
в~отличие от~\cite{12}, получены явные аналитические представления для подынтегральных 
функций числителя и~знаменателя этого дроб\-но-ли\-ней\-но\-го\linebreak
 функционала. Дальнейшие 
результаты основываются на новой усиленной и~обобщенной форме\linebreak
 теоремы об экстремуме 
дроб\-но-ли\-ней\-но\-го интегрального функционала, впервые опубликованной 
в~работе П.\,В.~Шнуркова~\cite{14}. Согласно\linebreak
 утверж\-де\-нию этой теоремы, если 
существует глобальный экстремум так называемой основной функции дроб\-но-ли\-ней\-но\-го 
функционала, которая пред\-став\-ля\-ет собой отношение подынтегральных функций чис\-ли\-те\-ля 
и~знаменателя, то существует безусловный экстремум самого дроб\-но-ли\-ней\-но\-го 
функционала, который достигается на наборе вырожденных вероятностных распределений, 
сосредоточенных в~точке глобального экстремума. В~этом случае оптимальная стратегия 
управ\-ле\-ния существует, является стационарной и~детерминированной и~определяется точкой, 
в~которой основная\linebreak функция достигает глобального экстремума. Таким\linebreak образом, проблемы 
существования оптимальной стратегии управ\-ле\-ния полумарковским процессом и~ее 
нахождения сводятся к~задаче чис\-лен\-но\-го исследования на глобальный экстремум 
заданной функции от конечного чис\-ла вещественных переменных.

\section{Общее описание модели управления полумарковским случайным процессом}

Построим модель управления полумарковским случайным процессом, следуя общему 
подходу, принятому в~классических работах~\cite{3, 8}. Пусть $\xi(t)$~--- 
случайный полумарковский процесс с~конечным множеством состояний
$X\hm=\{1,2,\ldots, N\}$, $N\hm< \infty$. Обозначим через~$t_n$, $n=0,1,2,\ldots$, 
$t_0\hm=0$, случайные моменты изменения состояний данного процесса, 
$\theta_n\hm=t_{n+1}-t_n$, $n\hm=0,1,2,\ldots$, $\xi_n\hm=\xi(t_n)\hm=\xi(t_n+0)$, 
$n\hm=0,1,2,\ldots$ (предполагается, что траектории процесса~$\xi(t)$ 
непрерывны справа). Случайная последовательность~$\{\xi_n\}$
образует цепь Маркова, вложенную в~полумарковский процесс~$\xi(t)$.
Зададим набор измеримых пространств\linebreak $(U_1, \mathscr{B}_1), 
(U_2, \mathscr{B}_2), \ldots, (U_N, \mathscr{B}_N)$, где $U_i$~--- 
множество возможных допустимых управ\-ле\-ний, $\mathscr{B}_i$~--- $\sigma$-ал\-геб\-ра 
подмножеств множества~$U_i$, вклю\-ча\-ющая в~себя все одноточечные подмножества\linebreak  
множества~$U_i$, т.\,е.\ если $u_i \hm\in U_i$, то $\{u_i\} \hm\in \mathscr{B}_i$, 
$i\hm=1,2,\ldots, N$.
Пусть $\Gamma_i$~--- некоторое множество всевозможных вероятностных мер $\Psi_i 
\hm \in \Gamma_i$, заданных на $\sigma$-ал\-геб\-ре~$\mathscr{B}_i$, $i\hm=1,2,\ldots,N$.

Поскольку идейное содержание и~свойства вероятностных мер существенно используются 
в~данной работе, укажем на некоторые фундаментальные издания, в~которых 
изложена соответствующая тео\-рия. Понятие и~основные свойства вероятностной 
меры определены и~подробно проанализированы в~книге А.\,Н.~Ширяева~\cite[гл.~II]{15}. 
Глубокое изложение основ теории вероятностных мер имеется также в~книге 
А.\,А.~Боровкова~\cite{16}. Заметим попутно, что в~книге~\cite{16} имеются разделы, 
посвященные изложению основ теории полумарковских и~регенерирующих случайных процессов. 
Из зарубежных изданий отметим фундаментальную работу П.~Хеннекена и~А.~Тортра~\cite{17}, 
основная часть которой посвящена изложению математических основ теории вероятностей.

Введем специальное понятие вырожденной вероятностной меры, которое будет часто 
использоваться в~дальнейшем. Пусть $(U_0, \mathscr{B}_0)$~--- некоторое измеримое 
пространство, $\mathscr{B}_0$~--- $\sigma$-ал\-геб\-ра подмножеств множества~$U_0$, 
включающая в~себя все одноточечные подмножества этого множества.

\medskip

\noindent
\textbf{Определение 1.}\ Вероятностная мера~$\Psi^*$, заданная 
на $\sigma$-ал\-геб\-рe~$\mathscr{B}_0$, называется вырожденной, если существует 
такой элемент $u^* \hm\in U_0$, для которого выполняются условия $\Psi^*(\{u^*\})\hm=
1$, $\Psi^*(U_0 \setminus \{u^*\})\hm=0$, где $\{u^*\}=u^*$~--- 
множество, состоящее из единственной точки $u^* \hm\in U_0$. Соответствующая 
точка $u^* \hm\in U_0$ будет называться точкой сосредоточения вырожденной 
вероятностной меры~$\Psi^*$.
Таким образом, всякая вырожденная вероятностная мера~$\Psi^*$ определяется 
своей точкой сосредоточения~$u^*$. В~дальнейшем будем использовать 
обозначение~$\Psi_{u^*}^{*}$, имея в~виду, что вырожденная вероятностная мера~$\Psi^*$ 
сосредоточена в~точке~$u^*$.
Отметим также, что вырожденная вероятностная мера~$\Psi_{u^*}^{*}$ соответствует 
детерминированной величине, которая принимает фиксированное значение $u\hm=u^*$ 
с~вероятностью, равной единице.

\medskip

Обозначим через $\Gamma_0$ множество всех  вероятностных мер, заданных 
на измеримом пространстве ($U_0, \mathscr{B}_0$), а через~$\Gamma_0^*$~--- 
множество всех вырожденных вероятностных мер, заданных на этом пространстве, 
$\Gamma_0^*\hm\in \Gamma_0$. Аналогичные обозначения будут использоваться 
и~в~дальнейшем. Заметим, что множество~$\Gamma_0^*$ находится во взаимно
 однозначном соответствии с~множеством точек сосредоточения вырожденных 
 вероятностных мер, т.\,е.\ с~множеством~$U_0$.

Пусть $\Gamma_i^{*}$~--- множество всех вырожденных мер, заданных на 
$\sigma$-ал\-геб\-ре~$\mathscr{B}_i$, $\Gamma_i^{*}\hm\subset \Gamma_i$.
Произвольная вероятностная мера~$\Psi_i$ описывает случайную величину, 
принимающую значения в~$U_i$, а вырожденная мера~$\Psi_i^*$, сосредоточенная 
в~точке~$u_i^*$, соответствует детерминированной величине $u_i^*\hm\in U_i$.
Предполагается, что соответствующие конструкции определены на всех измеримых 
пространствах управлений $(U_1, \mathscr{B}_1), (U_2, \mathscr{B}_2), \ldots, 
(U_N,\mathscr{B}_N)$.

Предположим, что управления случайным полумарковским процессом~$\xi(t)$ 
осуществляются в~моменты времени~$t_n,$ $n\hm=0,1,2,\ldots,$
непосредственно после изменения состояния процесса. Если\linebreak 
$\xi_n\hm=\xi(t_n)\hm=i \hm\in X$, то значение управления представляет 
собой случайную величину~$u_n$, принимающую значения в~множестве допустимых 
управ\-ле\-ний~$U_i$ и~описываемую вероятностной мерой (распределе\-ни\-ем 
вероятностей) $\Psi_i \hm\in \Gamma_i$.
Будем предполагать, что при фиксированном условии $\xi_n\hm=\xi(t_n)=i$ 
управ\-ле\-ние определяется независимо от прошлого поведения процесса~$\xi(t)$ 
и~вероятностная мера~$\Psi_i$,
описывающая стохастическое управление~$u_n$, зависит только от состояния $i\hm\in X$.
Тогда выбор управ\-ле\-ний в~моменты изменения состояний $\{t_n, n\hm=0,1,2,\ldots \}$ 
описывается набором вероятностных мер (распределений вероятностей) 
$(\Psi_1, \Psi_2,\ldots, \Psi_N)$, 
$\Psi_i \hm\in \Gamma_i$, $i\hm=1,2,\ldots,N$.
Назовем любой такой набор стратегией управ\-ле\-ния полумарковским процессом~$\xi(t)$. 
По своим свойствам такая стратегия является марковской, однородной 
и~рандомизированной.

Следуя классической монографии П.~Халмоша~\cite[гл.~VII]{18}, 
рассмотрим декартово произведение пространств $U\hm=U_1\times U_2\times \cdots\times U_N$ 
и~соответствующих $\sigma$-ал\-гебр $\mathscr{B}\hm=\mathscr{B}_1\times \mathscr{B}_2
\times \cdots \times\mathscr{B}_N$. Обозначим через $\Psi\hm=\Psi_1\times \Psi_2\times \cdots
\times \Psi_N$ вероятностную меру на~$(U,\mathscr{B})$, определяемую как 
произведение мер $\Psi_1,\Psi_2, \ldots , \Psi_N$, где $\Psi_i \hm\in \Gamma_i$, 
$i\hm=1,2,\ldots,N$. Обозначим также через~$\Gamma$ множество вероятностных мер~$\Psi$, 
заданных на~$(U,\mathscr{B})$, которые пред\-став\-ля\-ют собой произведение 
мер $\Psi_1,\Psi_2, \ldots , \Psi_N$, где $\Psi_i \hm\in \Gamma_i$, $i\hm=1,2,\ldots,N$.
Множество~$\Gamma$ можно отож\-де\-ст\-вить с~множеством всех стратегий управ\-ле\-ния 
полумарковским процессом~$\xi(t)$.

Проблема оптимального управления полумар\-ковским процессом~$\xi(t)$ будет в~дальнейшем 
сформулирована в~виде задачи безусловного экстремума некоторого функционала 
$I(\Psi)\hm=I(\Psi_1,\Psi_2, \ldots , \Psi_N)$, заданного на множестве 
допустимых стратегий управления. Содержание показателя качества управления~$I(\Psi)$, 
аналитическое представление для него, а~также описание множества допустимых 
стратегий управления будут приведены в~последующих разделах данной работы.

Для получения дальнейших результатов потребуются различные вероятностные 
характеристики управляемого полумарковского процесса~$\xi(t)$. Как известно из
 общей теории полумарковских процессов~\cite{19, 20}, 
 основной вероятностной характеристикой такого процесса является так называемая 
 полумарковская функция. Определим эту функцию для процесса с~управлением 
 (см.~\cite[гл.~5]{8}):
\begin{multline}
Q_{ij}(t,u)=
{\sf P}\left(\xi_{n+1}=j,\theta_n<t \mid \xi_n=i, u_n=u\right)\,,\\
t\in [0,\infty)\,,\ u\in U_i\,;\ i,j\in X=\{1,2,\ldots,N\}\,. \label{e1}
\end{multline}
Используя полумарковские функции, можно получить вероятности перехода 
управляемой цепи Маркова~$\{\xi_n\}$:
\begin{multline}
p_{ij}(u)={\sf P}\left(\xi_{n+1}=j \mid \xi_n=i, u_n=u\right)= {}\\
{}=
\lim\limits_{t\rightarrow\infty}Q_{ij}(t,u)\,,\enskip
u\in U_i\,;\enskip i,j\in X\,, 
\label{e2}
\end{multline}
а также функции распределения длительностей пребывания полумарковского 
процесса~$\xi(t)$ в~соответствующих состояниях:

\noindent
\begin{multline}
H_{i}(t,u)={\sf P}\left(\theta_n<t \mid \xi_n=i, u_n=u\right)={}\\
{}=
\sum\limits_{j\in X}Q_{ij}(t,u)\,,\enskip
t\in [0,\infty)\,,\  u\in U_i\,; \  i\in X\,. 
\label{e3}
\end{multline}

Обозначим через
\begin{multline}
T_{i}(u)=\mathbf{E}\left[\theta_n \mid \xi_n=i, u_n=u\right]={}\\
{}=
\int\limits_0^{\infty}\left[1-H_i(t,u)\right]\,dt\,,\enskip
u\in U_i\,,\ i\in X\,, 
\label{e4}
\end{multline}
математические ожидания длительностей пребывания полумарковского процесса~$\xi(t)$ 
в~каждом из состояний.

Введенные выше характеристики~(1)--(4) определены для случая, когда 
в~момент изменения состояния~$t_n$ процесс оказывается в~состоянии~$i$ 
и~принимается решение $u\hm\in U_i$. При заданной стратегии управления 
$\Psi\hm=\left(\Psi_1,\Psi_2, \ldots , \Psi_N\right)$ можно записать 
соответствующие вероятностные характеристики без условия на управление, а~именно:
\begin{multline*}
Q_{ij}(t)={\sf P}\left(\xi_{n+1}=j,\theta_n<t \mid \xi_n=i\right)={}\\
{}=
\int\limits_{U_i}Q_{ij}(t,u) \,d\Psi_i(u)\,,\enskip 
t\in [0,\infty)\,,\ i,j\in X\,; %\label{e5}
\end{multline*}

\vspace*{-12pt}

\noindent
\begin{multline}
p_{ij}={\sf P}\left(\xi_{n+1}=j \mid \xi_n=i\right)=
\int\limits_{U_i}p_{ij}(u)\, d\Psi_i(u)\,,\\  
i,j\in X\,; 
\label{e6}
\end{multline}

\vspace*{-9pt}

\noindent
\begin{equation}
T_{i}=\mathbf{E}\left[\theta_n \mid \xi_n=i\right]=
\int\limits_{U_i}T_{i}(u)\,d\Psi_i(u)\,,\enskip i\in X\,. 
\label{e7}
\end{equation}
В дальнейшем будем предполагать, что для рас\-смат\-ри\-ва\-емой 
полумарковской модели заданы вероятностные характеристики 
$p_{ij}(u)$, $u\hm\in U_i$, $i,j\hm\in X$, и~$T_i(u)$, $u\hm\in U_i$, $i\hm\in X$, 
определяемые соотношениями~(\ref{e2}) и~(\ref{e4}). 
Для фиксированной стратегии управления $\Psi\hm=(\Psi_1, \Psi_2,\ldots, \Psi_N)$ 
соответствующие вероятностные характеристики~$p_{ij}$ и~ $T_i$, $i,j\hm\in X,$ 
определены равенствами~(\ref{e6}) и~(\ref{e7}) без условий на управление.

\section{Стационарный стоимостной показатель качества управления}

Определим некоторый стоимостной аддитивный функционал, связанный 
с~рассматриваемым полумарковским процессом~$\xi(t)$. По содержанию этот функционал 
представляет собой случайный\linebreak доход или прибыль, накопленную за период времени $[0,t]$. 
Определения такого функционала приведены в~основополагающих работах~[3; 8, гл.~5].\linebreak 
Обозначим через $\widetilde{v}(t)$, $t\hm\geq 0$, значение этого аддитивного 
функционала в~момент времени~$t$; $\widetilde{v}_n\hm=\widetilde{v}(t_n\hm+0)$~--- 
соответствующее значение непосредственно после очередного момента изменения\linebreak 
состояния~$t_n$, $n\hm=0,1,2,\ldots$; $\widetilde{v}_0\hm=v_0$~--- 
заданное начальное значение в~момент $t\hm=0$. Рассмотрим величину
\begin{multline}
d_i(u)=\mathbf{E}\left[\widetilde{v}_{n+1}-\widetilde{v}_n \mid \xi_n=i\,, 
u_n=u\right]\,,\\
u\in U_i\,, \enskip i\in X\,, \label{e8}
\end{multline}
представляющую собой математическое ожидание приращения стоимостного 
аддитивного функционала за период времени между последовательными 
изменениями состояния полумарковского процесса~$\xi(t)$. Тогда соответствующее 
математическое ожидание, вычисляемое без условия на решение, 
принимаемое в~момент времени~$t_n$, представляется в~виде:
\begin{equation*}
d_i=\mathbf{E}\left[\widetilde{v}_{n+1}-\widetilde{v}_n \mid \xi_n=i\right]=
\!\int\limits_{U_i}\!d_i(u)\,d\Psi_i(u)\,,\ i\in X \,. %\label{e9}
\end{equation*}

Предположим, что для заданной стратегии управ\-ле\-ния 
$\Psi\hm=(\Psi_1,\Psi_2,\ldots,\Psi_N)$ вложенная цепь Маркова~$\{\xi_n\}$ 
имеет ровно один класс возвратных положительных состояний (по терминологии, 
принятой в~\cite{8}, такое множество состояний называется эргодическим классом). 
Как известно~\cite[гл.~VIII]{15}, данное условие является необходимым 
и~достаточным для существования единственного\linebreak стационарного распределения. 
Обозначим это стационарное распределение цепи Маркова~$\{\xi_n\}$ через 
$\pi\hm=(\pi_1, \pi_2,\ldots, \pi_N)$. Заметим, что данное\linebreak распределение зависит  
от стратегии управления $\Psi\hm=(\Psi_1,\Psi_2,\ldots,\Psi_N)$. При указанном 
условии имеет место следующий результат, называемый эргодической теоремой 
для аддитивного стоимостного функционала:
\begin{equation}
I=\lim\limits_{t\rightarrow\infty}\fr{\mathbf{E}\widetilde{v}(t)}{t}=
\fr{\sum\nolimits_{i=1}^N d_i\pi_i}{\sum\nolimits_{i=1}^N T_i\pi_i}\,. 
\label{e10}
\end{equation}

Соотношение~(\ref{e10}) доказано в~работе~\cite[гл.~5]{8}. Заметим, что аналогичные 
результаты имеют мес\-то для гораздо более общих полумарковских моделей~\cite{10, 11}.

По своему прикладному содержанию величина, определяемая соотношением~(\ref{e10}), 
представляет собой
среднюю удельную прибыль, связанную с~эволюцией системы в~стационарном
режиме. Кроме того, величина~$I$ представляет собой функционал от
набора вероятностных распределений~$\Psi_{i}$, $i\hm\in\lbrace 1,\ldots
,N\rbrace $, определяющих стратегию управле-\linebreak\vspace*{-12pt}

\pagebreak

\noindent
ния системой. 
В~дальнейшем будем рассматривать стационарный стоимостной функционал 
$I\hm=I(\Psi_{1},\Psi_{2},\ldots , \Psi_{N})$ как
показатель качества управ\-ле\-ния системой и~построенным полумарковским
процессом~$\xi (t)$.

\section{Представление стационарного показателя в~форме
дробно-линейного интегрального функционала}

В данном разделе будет приведено утверждение об аналитическом
представлении стационарного стоимостного функционала~(\ref{e10}), 
служащего критерием качества управления в~рассматриваемой задаче управления 
полумарковским процессом.

\smallskip

\noindent
\textbf{Теорема 1.} \textit{Стационарный стоимостной показатель, 
определяемый равенством}~(\ref{e10}), \textit{представляет собой дроб\-но-ли\-ней\-ный
функционал от вероятностных распределений~$\Psi_{i}(u_{i})$,
$i\hm\in\{1,\dots,N\}$. Данный функционал задается
аналитически следующей формулой:}
\begin{multline}
I=I(\Psi_{1},\ldots, \Psi_{N})={}\\
\hspace*{-2mm}{}=\!
\fr{\int\nolimits_{U_1}\!{\cdots\! 
\int\nolimits_{U_N}\!{A(u_{1},\ldots ,u_{N})d\Psi_{1}(u_{1})\cdots
\,d\Psi_{N}(u_{N})}}}{\int\nolimits_{U_1}{\!\cdots\! \int\nolimits_{U_N}\!{B(u_{1},\ldots ,u_{N})
\,d\Psi_{1}(u_{1})\ldots
d\Psi_{N}(u_{N})}}},\!\!\! \label{e11}
\end{multline}
\textit{где подынтегральные функции числителя и~знаменателя выражаются
соотношениями}:
\begin{align}
A(u_{1},\ldots
,u_{N})&={}\notag\\
&\hspace*{-20mm}{}=\sum\limits_{i=1}^{N}{d_{i}(u_{i})}{\widehat{D}}^{(i)}(u_{1}, \ldots
,u_{i-1},u_{i+1}, \ldots , u_{N})\,;  \label{e12}\\
 B(u_{1},\ldots
,u_{N})&={}\notag\\
&\hspace*{-20mm}{}=\sum\limits_{i=1}^{N}{T_{i}(u_{i})}{\widehat{D}}^{(i)}(u_{1}, \ldots
,u_{i-1},u_{i+1}, \ldots , u_{N})\,.  \label{e13}
\end{align}
\textit{В свою очередь, функции} ${\widehat{D}}^{(i)}(u_{1}, \ldots
,u_{i-1},u_{i+1}, \ldots$\linebreak $\ldots , u_{N})$, $i\hm\in\{1,\dots,N\}$, 
\textit{входящие в~правые части формул}~(\ref{e12}) и~(\ref{e13}), 
\textit{определяются следующим образом:}

\noindent
\begin{multline}
{\widehat{D}}^{(i)}(u_{1}, \ldots ,u_{i-1},u_{i+1}, \ldots , u_{N})={}
\\
{}=(-1)^{N+i+2}\sum\limits_{\alpha ^{(N),i}}{(-1)}^{\delta (\alpha
^{(N),i}) }{\widehat{D}}_{0}^{(i)}\left(\alpha ^{(N),i},u_{1}, \ldots\right.\\
\left.\ldots , u_{i-1},u_{i+1}, \ldots , u_{N}\right)\,. \label{e14}
\end{multline}
\textit{Здесь} $\alpha ^{(N),i}=(\alpha _{1}, \ldots , \alpha _{i-1},\alpha_{i+1}, \ldots , 
\alpha _{N})$~\textit{--- произвольная
перестановка чисел }$(1, \ldots , i-1, i+1, \ldots , N)$;
$\delta
(\alpha ^{(N),i})$~\textit{--- число инверсий в~перестановке} 
$\alpha ^{(N),i}$;

\noindent
\begin{multline}
{\widehat{D}}_{0}^{(i)}(\alpha ^{(N),i},u_{1}, \ldots ,u_{i-1},u_{i+1},
\ldots , u_{N})={}\\
{} ={\widetilde{p}}_{1,\alpha _{1}}\left(u_{1}\right)\cdots {\widetilde{p}}_{i-1,\alpha
_{i-1}}\left(u_{i-1}\right){\widetilde{p}}_{i+1,\alpha _{i+1}}\left(u_{i+1}\right)\cdots\\
\cdots
{\widetilde{p}}_{N,\alpha _{N}}\left(u_{N}\right)\,, 
\label{e15}
\end{multline}
где
\begin{multline}
 {\widetilde{p}}_{k,\alpha _{k}}(u_{k})=
\begin{cases}
p_{kk}(u_{k})-1,\  & \alpha _{k}=k\,; \\
p_{k,\alpha _{k}}(u_{k}),\  & \alpha _{k}\ne k, \\
\end{cases}\\
 k=1, \ldots , i-1, i+1, \ldots ,N\,. \label{e16}
 \end{multline}
\textit{Функции $p_{ij}(u_i)$, $T_{i}(u_{i})$ и~$d_{i}(u_{i})$,
$u_i\hm\in U_i$, $i,j\hm\in \{1,2,\ldots,N\}$, 
входящие в~соотношения}~(\ref{e12})--(\ref{e16}), 
\textit{определяются равенствами}~(\ref{e2}), (\ref{e4}) \textit{и}~(\ref{e8}) \textit{соответственно.}

\smallskip

\noindent
Д\,о\,к\,а\,з\,а\,т\,е\,л\,ь\,с\,т\,в\,о\ теоремы~1 
в~весьма сжатой форме приведено в~работе~\cite{21}. Читателю, интересующемуся 
более подробным обоснованием данного результата, порекомендуем обратиться к~тексту 
кандидатской диссертации А.\,В.~Иванова~\cite[гл.~3]{22}.

\smallskip

Итак, теорема~1 позволяет получить явное аналитическое представление 
для стационарного стоимостного показателя вида~(\ref{e10}) в~форме 
дроб\-но-ли\-ней\-но\-го интегрального функционала от набора\linebreak вероятностных мер 
$\Psi\hm=(\Psi_{1},\Psi_{2},\ldots , \Psi_{N})$, за\-да\-ющих стратегию управления 
полумарковским процессом~$\xi(t)$. При этом подынтегральные функции числителя 
и~знаменателя задаются формулами~(\ref{e12}), (\ref{e13}) 
и~вспомогательными равенствами~(\ref{e14})--(\ref{e16}). Таким образом, функция
\begin{equation}
C\left(u_1, u_2,\ldots, u_N\right)=\fr{A(u_1, u_2,\ldots, u_N)}{B(u_1, u_2,\ldots, u_N)}\,,
\label{e17}
\end{equation}
которая в~дальнейшем будет называться основной функцией дроб\-но-ли\-ней\-но\-го 
интегрального функционала~(\ref{e11}) и~которая будет играть важную роль 
в~дальнейшем исследовании, также явно определяется формулами~(\ref{e17}), 
(\ref{e12}), (\ref{e13}).

\section{Формальная постановка оптимизационной задачи 
и~условия существования оптимальной стратегии управления полумарковским процессом}

Будем рассматривать проблему управления полумарковским процессом~$\xi(t)$ в~форме 
экстремальной задачи
\begin{multline}
I(\Psi)=I\left(\Psi_1, \Psi_2,\ldots,\Psi_N\right)\rightarrow \mathrm{extr}\,,
\\
\Psi=\left(\Psi_1, \Psi_2,\ldots,\Psi_N\right)\in\Gamma\,. \label{e18}
\end{multline}
При этом показатель качества управления~$I(\Psi)$ представляет собой 
дроб\-но-ли\-ней\-ный интегральный функционал вида~(\ref{e11}).

Для решения экстремальной задачи~(\ref{e18}) воспользуемся некоторым утверждением 
об экстремуме дроб\-но-ли\-ней\-но\-го интегрального функционала. Прежде 
чем сформулировать данное утверждение, отметим, что в~теории оптимизации 
хорошо известны задачи, в~которых целевая функция представляет собой 
отношение двух линейных отображений, а имеющиеся ограничения также линейны. 
Такой раздел называется дроб\-но-ли\-ней\-ным программированием. Основные
 теоретические результаты данного направления изложены в~работе~\cite{23},
  там же приведена подробная библиография. В~дальнейшем потребуется некоторый 
  специальный результат о безусловном экстремуме дроб\-но-ли\-ней\-но\-го 
  интегрального функционала вида~(\ref{e11}), который был впервые сформулирован 
  в~работе~\cite{14}. Заметим, что для использования этого результата необходимо, 
  чтобы выполнялись некоторые предварительные условия, которые в~данном случае 
  можно сформулировать следующим образом:
\begin{enumerate}[1.]
\item Интегральные выражения
\begin{align*}
I_1(\Psi)&=I_1\left(\Psi_1,\Psi_2,\ldots,\Psi_N\right)={}&\\
&\hspace*{-13mm}{}=\int\limits_{U_1}\!\cdots\!
\int\limits_{U_N}\!\!A\left(u_1,\ldots ,u_N\right)\,
d\Psi_1\left(u_1\right) %d\Psi_2\left(u_2\right)
\cdots
 d\Psi_N\left(u_N\right)\,;
\\
I_2(\Psi)&=I_2\left(\Psi_1,\Psi_2,\ldots,\Psi_N\right)={}&\\
&\hspace*{-13mm}{}=\int\limits_{U_1}\!\cdots\!\int\limits_{U_N}\!\!
B\left(u_1,\ldots,u_N\right)\,
d\Psi_1\left(u_1\right)% d\Psi_2\left(u_2\right)\cdots\\
\cdots d\Psi_N\left(u_N\right)
\end{align*}
определены для всех стратегий управления $\Psi\hm=(\Psi_1, \ldots,\Psi_N)
\hm\in \Gamma$.

\item Функционал $I_2(\Psi)=I_2(\Psi_1, \ldots,\Psi_N)\hm\neq 0$ 
для всех $\Psi\hm=(\Psi_1, \ldots,\Psi_N)\hm\in \Gamma$.

\item Множество $\Gamma$ включает в~себя множество всех вырожденных 
вероятностных мер: $\Gamma^* \hm\subset \Gamma$.
\end{enumerate}

Сделаем несколько важных замечаний по поводу введенных предварительных условий.

\smallskip

\noindent
\textbf{Замечание~1.}\ Из условия~2 следует, что функция $B(u_1, u_2,\ldots, u_N)$ 
не может принимать значения разных знаков. С~учетом условия~3 
получаем, что указанная функция должна обладать \mbox{свойством} строгой 
знакопостоянности на всем множестве~$U$. С~другой стороны, если выполняется 
условие строгой знакопостоянности функции $B(u_1, u_2,\ldots, u_N), 
(u_1, u_2,\ldots, u_N)\hm\in U$, то условие~2 выполняется автоматически.

\smallskip

\noindent
\textbf{Замечание~2.}\ Если рассматривать в~качестве целевого функционала 
$I(\Psi_1, \Psi_2,\ldots,\Psi_N)$ экстремальной задачи~(\ref{e18}) 
стационарный стоимостной пока\-затель~(\ref{e10}), то функция $B(u_1,u_2,\ldots,u_N)$ 
имеет\linebreak следующее теоретическое содержание. Данная функция представляет собой условное 
математическое ожидание длительности периода времени между соседними моментами 
изменения со\-сто\-яния полумарковского процесса~$\xi(t)$ при условии, что стратегия 
его управ\-ле\-ния является детерминированной и~задается набором значений аргументов 
$(u_1,u_2,\ldots,u_N)$. Тогда условие строгой положительности функции 
$B(u_1,u_2,\ldots,u_N)$ при всех $(u_1,u_2,\ldots,u_N)\hm\in U$ является естественным 
и~фактически означает, что при любой заданной детерминированной стратегии 
управ\-ле\-ния процесс~$\xi(t)$ не имеет мгновенных со\-сто\-яний, длительность пребывания 
в~которых равна нулю.

\smallskip

\noindent
\textbf{Замечание~3.}\ Сделаем некоторые замечания, связан\-ные с~подынтегральной 
функцией числителя дроб\-но-ли\-ней\-но\-го интегрального функционала~(\ref{e11}). 
Как и~ранее, будем рассматривать в~качестве целевого функционала $I(\Psi_1, \Psi_2,\ldots,\Psi_N)$\linebreak 
экстремальной задачи~(\ref{e18}) стационарный стоимостной показатель~(\ref{e10}). 
Тогда для любого фиксированного набора значений аргументов $(u_1,u_2,\ldots,u_N)\hm\in U$ 
значение функции $A(u_1,u_2,\ldots\linebreak \ldots,u_N)$ представляет собой условное математическое
 ожидание приращения рассматриваемого стоимостного функционала, 
 происшедшее за время пребывания полумарковского процесса~$\xi(t)$ в~некотором 
 фиксированном  состоянии при условии, что стратегия управления является 
 детерминированной и~задается указанным набором $(u_1,u_2,\ldots,u_N)\hm\in U$. 
 Отметим, что в~теореме об экстремуме дроб\-но-ли\-ней\-но\-го интегрального 
 функционала, доказанной в~работе~\cite[гл.~10]{12}, 
 на подынтегральную функцию числителя накладываются условия ограниченности на 
 всем множестве значений аргумента. Для многих математических моделей и~связанных 
 с~ними задач оптимального управления такое условие является излишне ограничительным. 
 В~качестве примера можно привести модели оптимального управления запасом непрерывного 
 продукта, рассмотренные в~работах~\cite{27, 28}. 
 В~настоящем исследовании на функцию $A(u_1,u_2,\ldots,u_N)$ накладывается только 
 условие интегрируемости по любому заданному набору вероятностных мер 
 $\Psi\hm=(\Psi_1, \Psi_2,\ldots,\Psi_N)$, образующему стратегию управления 
 полумарковским процессом~$\xi(t)$ (условие~1 системы предварительных условий).

\smallskip

\noindent
\textbf{Замечание~4.} Условия~1--3 являются необходимыми для корректной 
постановки задачи безусловного экстремума дроб\-но-ли\-ней\-но\-го интегрального 
функционала. Если этот функционал служит показателем качества в~задаче оптимального 
управления случайным процессом, то необходимо добавить к~этим условиям дополнительное, 
связанное с~некоторой регулярностью самого управляемого процесса, а~именно: некоторый 
содержательный показатель, связанный с~поведением этого процесса, должен существовать 
и~быть представимым в~виде дроб\-но-ли\-ней\-но\-го интегрального функционала. 
Если потребовать, чтобы выполнялось эргодическое соотношение~(\ref{e10}), 
то можно использовать\linebreak теорему~1 и~сформулировать задачу оптимального управ\-ле\-ния 
в~виде~(\ref{e18}) для дроб\-но-ли\-ней\-но\-го\linebreak интегрального функционала~(\ref{e11}). 
Таким образом, необходимо ввести условие, обеспечивающее существование единственного 
стационарного распределения вложенной цепи Маркова и~выполнение\linebreak соотношения~(\ref{e10}). 
По аналогии с~[8, гл.~5] сформулируем это дополнительное условие в~следующем виде:
\begin{enumerate}
\setcounter{enumi}{3}
\item Для любой рассматриваемой стратегии управ\-ле\-ния $\Psi\hm=
(\Psi_1, \Psi_2,\ldots,\Psi_N)\hm\in \Gamma$ вложенная цепь Маркова 
полумарковского процесса $\xi(t)$ имеет ровно один класс возвратных 
положительных состояний.
\end{enumerate}

Теперь определим понятие допустимой стратегии управления полумарковским процессом 
с~конечным множеством состояний.

\smallskip

\noindent
\textbf{Определение~2.}\ Назовем стратегию управления 
$\Psi\hm=(\Psi_1, \Psi_2,\ldots,\Psi_N)$ 
допустимой в~данной задаче, если она удовлетворяет условиям~1--4.


\smallskip

\noindent
\textbf{Замечание~5.}\ Как следует из замечания~1, если потребовать, 
чтобы функция $B(u_1, u_2,\ldots,u_N)$ являлась строго знакопостоянной при 
всех $(u_1, u_2,\ldots,u_N)\hm\in U$, то можно считать допустимыми стратегии 
$(\Psi_1, \Psi_2,\ldots,\Psi_N)$, удовлетворяющие условиям~1, 3, 4. С~учетом замечания~2 
о~естественном характере условия строгой знакопостоянности функции $B(u_1,u_2,\ldots,u_N)$ 
при всех значениях аргументов $(u_1, u_2,\ldots,u_N)\hm\in U$ будем требовать 
выполнения этого условия в~формулировке приводимой в~дальнейшем основной 
теоремы об оптимальной стратегии управления полумарковским процессом.

\smallskip

\noindent
\textbf{Замечание~6.}\ Ниже будет сформулирована и~доказана основная 
теорема об оптимальной стра\-тегии управления полумарковским процессом с~конеч\-ным 
множеством состояний. Будем формулировать эту теорему по отношению к~экстремальной 
задаче~(\ref{e18}), в~которой целевой функционал $I(\Psi_1, \Psi_2,\ldots,\Psi_N)$ 
имеет вид дроб\-но-ли\-ней\-но\-го интегрального функционала. 
Это обстоятельство связано с~тем, что целевой функционал в~задаче 
оптимального управления необязательно должен иметь характер стационарного 
стоимостного показателя вида~(\ref{e10}). В~частности, еще в~1983~г.\ П.\,В.~Шнурковым 
было установлено~\cite{24}, что ряд показателей, связанных 
с~временем пребывания управляемого полумарковского процесса в~заданном конечном 
подмножестве состояний, имеет структуру дроб\-но-ли\-ней\-но\-го интегрального 
функционала от набора вероятностных мер, определяющих стратегию управления. 
Таким образом, рассматриваемая задача управления имеет более общий характер, 
чем задача, в~которой целевой функционал представляет собой стационарный 
стоимостной показатель вида~(\ref{e10}).






\smallskip

\noindent
\textbf{Замечание~7.}\ Если рассматривать задачу оптимального управления 
полумарковским процессом, в~кото\-рой целевой функционал не совпадает 
со стационарным стоимостным показателем~(\ref{e10}), то возможно, что могут 
потребоваться другие дополнительные условия, обеспечивающие существование этого 
показателя и~его представление в~форме~(\ref{e11}). В~связи с~этим в~формулировке 
основной теоремы будем использовать термин допустимые стратегии в~широком смысле, 
имея в~виду выполнение всех необходимых условий для каждого рассмат\-ри\-ва\-емо\-го 
показателя качества управления.

\smallskip


\noindent
\textbf{Замечание 8.} Множество допустимых стратегий может 
не совпадать с~множеством всех возможных стратегий управления. 
В~частности, допустимые стратегии могут состоять только из дискретных вероятностных 
мер $\Psi_1, \Psi_2,\ldots,\Psi_N$, т.\,е.\ таких, которые сосредоточены на дискретных 
множествах точек пространств $U_1, U_2,\ldots,U_N$.

\section{Теоретическое решение задачи оптимального управления}

Перейдем к~формулировке и~доказательству тео\-ре\-мы об 
оптимальной стратегии управ\-ле\-ния полумарковским процессом с~конечным 
множеством состояний.

\smallskip

\noindent
\textbf{Теорема~2.} \textit{Рассмотрим проблему оптимального управ\-ле\-ния 
полумарковским процессом~$\xi(t)$ в~виде экстремальной задачи}~(\ref{e18}), 
\textit{определенной на множестве допустимых стратегий $\Gamma$, 
для дроб\-но-ли\-ней\-но\-го 
функционала}~(\ref{e11}). \textit{Пусть функция $B(u_1,u_2,\ldots,u_N)$, 
входящая в~определение функционала}~(\ref{e11}),
\textit{является строго знакопостоянной (строго положительной или строго отрицательной) 
при всех значениях аргументов $(u_1,u_2,\ldots,u_N)\hm\in U$.
Тогда справедливы сле\-ду\-ющие утверждения}:
\begin{enumerate}[1.]
\item \textit{Если функция} $C(u_1,u_2,\ldots,u_N)\hm=A(u_1,u_2,\ldots$\linebreak
$\ldots,u_N)/{B(u_1,u_2,\ldots,u_N)}$ 
\textit{ограничена сверху или снизу и~достигает глобального экст\-ре\-му\-ма на множестве
$U\hm=U_1\times U_2\times \cdots \times U_N$ (максимума или минимума), 
то оптимальная стратегия управления полумарковским процессом~$\xi(t)$ существует, 
является детерминированной и~определяется
вырожденной вероятностной мерой $\Psi^*\hm\in \Gamma^*$, сосредоточенной в~точке, 
в~которой достига\-ет соответствующего экстремума функция $C(u_1,u_2,\ldots,u_N)$,
и~при этом выполняются соотношения}:
\begin{multline}  %{\substack{{i=\overline{1,n}}\\ {j=\overline{1,l}}}}
\max\limits_{\Psi \in \Gamma} I(\Psi)=
\max\limits_{\substack{{\Psi_i \in \Gamma_i\,,}\\ 
{i=\overline{1,N}}}}
I\left(\Psi_1,\Psi_2,\ldots,\Psi_N\right)={}\\
{}=
\max\limits_{\substack{{\Psi_i^* \in \Gamma_i^*,}\\ 
{i=\overline{1,N}}}}
 I\left(\Psi_1^*,\Psi_2^*,\ldots,\Psi_N^*\right)={}\\
{}=\max\limits_{(u_1,u_2,\ldots,u_N)\in U}\fr{A(u_1,u_2,\ldots,u_N)}
{B(u_1,u_2,\ldots,u_N)}\,; \label{e19}
\end{multline}

\vspace*{-12pt}

\noindent
\begin{multline*}
\min\limits_{\Psi \in \Gamma} I(\Psi)=
\min\limits_{\substack{{\Psi_i \in \Gamma_i\,,}\\ 
{i=\overline{1,N}}}} I\left(\Psi_1,\Psi_2,\ldots,\Psi_N\right)={}\\
{}=
\min\limits_{\substack{{\Psi_i^* \in \Gamma_i^*,}\\ 
{i=\overline{1,N}}}}
I\left(\Psi_1^*,\Psi_2^*,\ldots,\Psi_N^*\right)={}\\
{}=\min\limits_{(u_1,u_2,\ldots,u_N)\in U}\fr{A(u_1,u_2,\ldots,u_N)}
{B(u_1,u_2,\ldots,u_N)}\,. %\label{e20}
\end{multline*}
\item \textit{Если функция $C(u_1,u_2,\ldots,u_N)\hm=
{A(u_1,u_2,\ldots,u_N)}/{B(u_1,u_2,\ldots,u_N)}$ ограничена сверху или снизу, 
но не достигает глобального экстремума на множестве $U\hm=U_1\times U_2\times\cdots
\times U_N$,
то для любого $\varepsilon\hm > 0$ можно выбрать $\varepsilon$-оп\-ти\-маль\-ную 
детерминированную стратегию управления полумарковским процессом~$\xi(t)$, 
которая определяется вырожденной
вероятностной мерой $\Psi^{*(+)}(\varepsilon)\hm\in \Gamma^*$ или вырожденной
вероятностной мерой $\Psi^{*(-)}(\varepsilon)\hm\in \Gamma^*$, в~зависимости от 
вида экстремума (максимума или минимума) в~задаче}~(\ref{e18}). 
\textit{При этом вероятностная мера $\Psi^{*(+)}(\varepsilon)\hm\in \Gamma^*$ может быть 
сосредоточена в~любой точке $\left(u_1^{(+)}(\varepsilon),u_2^{(+)}(\varepsilon),\ldots,
u_N^{(+)}(\varepsilon)\right)$, удовлетворяющей соотношению}:
\begin{multline}
\sup\limits_{(u_1,u_2,\ldots,u_N) \in U}
\fr{A(u_1,u_2,\ldots,u_N)}{B(u_1,u_2,\ldots,u_N)}-\varepsilon <{}\\
{}<
\fr{A\left(u_1^{(+)}(\varepsilon),u_2^{(+)}(\varepsilon),\ldots,u_N^{(+)}
(\varepsilon)\right)}
{B\left(u_1^{(+)}(\varepsilon),u_2^{(+)}(\varepsilon),\ldots,u_N^{(+)}
(\varepsilon)\right)}<{}\\
{}<\sup\limits_{(u_1,u_2,\ldots,u_N) \in U}
\fr{A(u_1,u_2,\ldots,u_N)}{B(u_1,u_2,\ldots,u_N)}<\infty\,, 
\label{e21}
\end{multline}
\textit{если функция $C(u_1,u_2,\ldots,u_N)$ ограничена сверху 
и~экстремальная задача}~(\ref{e18}) 
\textit{представляет собой задачу на максимум. Аналогично вероятностная мера 
$\Psi^{*(-)}(\varepsilon)\hm\in \Gamma^*$ может быть сосредоточена в~любой точке 
$\left(u_1^{(-)}(\varepsilon),u_2^{(-)}(\varepsilon),\ldots,u_N^{(-)}(\varepsilon)
\right)$, удовлетворяющей соотношению}:

\noindent
\begin{multline*}
-\infty<\inf\limits_{(u_1,u_2,\ldots,u_N) \in U}\fr{A(u_1,u_2,\ldots,u_N)}
{B(u_1,u_2,\ldots,u_N)} <{}\\
{}<
\fr{A\left(u_1^{(-)}(\varepsilon),u_2^{(-)}
(\varepsilon),\ldots,u_N^{(-)}(\varepsilon)\right)}
{B\left(u_1^{(-)}(\varepsilon),u_2^{(-)}(\varepsilon),\ldots,
u_N^{(-)}(\varepsilon)\right)}<{}\\
{}<\inf\limits_{(u_1,u_2,\ldots,u_N) \in U}
\fr{A(u_1,u_2,\ldots,u_N)}{B(u_1,u_2,\ldots,u_N)}+\varepsilon\,, 
%\label{e22}
\end{multline*}
\textit{если функция $C(u_1,u_2,\ldots,u_N)$ ограничена снизу и~экстремальная 
задача}~(\ref{e18})  \textit{представляет собой задачу на минимум}.
\item \textit{Если функция $C(u_1,u_2,\ldots,u_N)\hm=
{A(u_1,u_2,\ldots,u_N)}/{B(u_1,u_2,\ldots,u_N)}$ не ограничена сверху 
или снизу, то оптимальной стратегии управления в~смысле
соответствующей экстремальной задачи не существует. 
При этом найдется такая последовательность вырожденных вероятностных
мер~$\Psi^{*(+)}(n)$, сосредоточенных в~точках 
$\left(u_1^{(+)}(n),u_2^{(+)}(n),\ldots,u_N^{(+)}(n)\right)$, $n\hm=1,2,\dots $, 
для которых выполняется соотношение}:
\begin{multline*}
I\left(\Psi^*(n)\right)={}\\
{}=
I\left(\Psi_1^{*(+)}(n),\Psi_2^{*(+)}(n),\ldots,\Psi_N^{*(+)}(n)\right)={}\\
{}=\fr{A\left(u_1^{(+)}(n),u_2^{(+)}(n),\ldots,u_N^{(+)}(n)\right)}
{B\left(u_1^{(+)}(n),u_2^{(+)}(n),\ldots,u_N^{(+)}(n)\right)}\to 
\infty\\
\mbox{при}\ n\to\infty\,, 
%\label{e23}
\end{multline*}
\textit{если функция $C(u_1,u_2,\ldots,u_N)$ не ограничена сверху. 
Аналогично найдется такая последовательность вырожденных вероятностных
мер~$\Psi^{*(-)}(n)$, сосредоточенных в~точках 
$\left(u_1^{(-)}(n),u_2^{(-)}(n),\ldots,u_N^{(-)}(n)\right)$, 
$n\hm=1,2,\dots $, для которых выполняется соотношение}:
\begin{multline*}
I\left(\Psi^{*(-)}(n)\right)={}\\
{}= I
\left(\Psi_1^{*(-)}(n),\Psi_2^{*(-)}(n),\ldots,\Psi_N^{*(-)}(n)\right)={}\\
{}=\fr{A\left(u_1^{(-)}(n),u_2^{(-)}(n),\ldots,u_N^{(-)}(n)\right)}
{B\left(u_1^{(-)}(n),u_2^{(-)}(n),\ldots,u_N^{(-)}(n)\right)}\to 
-\infty\\
\mbox{при}~~n\to\infty\,,  
%\label{e24}
\end{multline*}
\textit{если функция $C(u_1,u_2,\ldots,u_N)$ не ограничена \mbox{снизу}}.
\end{enumerate}
\textit{При этом сформулированные утверждения каждого пункта теоремы~$2$ 
могут выполняться как по отдельности, для одного из двух
видов экстремума, так и~совместно, для обоих видов экстремума.}

\smallskip

Прежде чем непосредственно доказывать теорему~2, докажем некоторые 
вспомогательные утверждения.

\smallskip

\noindent
\textbf{Лемма~1.}\ 
\textit{Рассмотрим дроб\-но-ли\-ней\-ный интегральный функционал 
$I(\Psi_1, \Psi_2,\ldots, \Psi_N)$ вида}~(\ref{e11}), 
\textit{заданный на некотором множестве наборов вероятностных мер 
$\Psi\hm=(\Psi_1, \Psi_2,\ldots, \Psi_N)\hm \in \Gamma$. Предположим, что на 
множестве~$\Gamma$ выполняется условие~$1$ из набора предварительных условий 
и~функция $B(u_1, u_2,\ldots, u_N)$  обладает свойством строгой знакопостоянности 
при всех $(u_1, u_2,\ldots, u_N) \hm\in U$. Тогда справедливы следующие утверждения}:
\begin{enumerate}[1.]
\item \textit{Если основная функция 
$C(u_1, u_2,\ldots, u_N)\hm={A(u_1, u_2,\ldots, u_N)}/{B(u_1, u_2,\ldots, u_N)}$ 
ограничена сверху, т.\,е.\ выполняется условие}
\begin{multline}
C\left(u_1, u_2,\ldots, u_N\right)=
\fr{A(u_1, u_2,\ldots, u_N)}{B(u_1, u_2,\ldots, u_N)}\leq {}\\
{}\leq
c_0^{(+)}<\infty \,, \enskip \left(u_1, u_2,\ldots, u_N\right) \in U\,, \label{e25}
\end{multline}
\textit{то имеет место неравенство}:
\begin{equation}
I\left(\Psi_1, \Psi_2,\ldots, \Psi_N\right)\leq c_0^{(+)} 
\label{e26}
\end{equation}
\textit{для всех} $(\Psi_1, \Psi_2,\ldots, \Psi_N) \in \Gamma$.
\item \textit{Если основная функция 
$C(u_1, u_2,\ldots, u_N)\hm={A(u_1, u_2,\ldots, u_N)}/{B(u_1, u_2,\ldots, u_N)}$ 
ограничена снизу, т.\,е.\ выполняется условие}
\begin{multline*}
C\left(u_1, u_2,\ldots, u_N\right)=\fr{A(u_1, u_2,\ldots, u_N)}{B(u_1, u_2,\ldots, 
u_N)}\geq{}\\
{}\geq c_0^{(-)}>-\infty \,, 
\left(u_1, u_2,\ldots, u_N\right) \in U\,, 
%\label{e27}
\end{multline*}
\textit{то имеет место неравенство}:
\begin{equation*}
I\left(\Psi_1, \Psi_2,\ldots, \Psi_N\right)\geq c_0^{(-)} 
%\label{e28}
\end{equation*}
\textit{для всех} $(\Psi_1, \Psi_2,\ldots, \Psi_N) \hm\in \Gamma$.
\end{enumerate}

\noindent
Д\,о\,к\,а\,з\,а\,т\,е\,л\,ь\,с\,т\,в\,о\ \ леммы~1.\ 
Докажем первое утверждение леммы. Предположим сначала, 
что функция $B(u_1, u_2,\ldots,  u_N)$ строго положительна:
\begin{equation}
B\left(u_1, u_2,\ldots, u_N\right)>0\,,\enskip
\left(u_1, u_2,\ldots, u_N\right)\in U\,. \label{e29}
\end{equation}
Заметим, что в~таком случае по свойству интеграла~\cite[гл.~V]{18}
\begin{multline}
\hspace*{-2mm}\int\limits_{U_1}\!\!\cdots\! \!\int\limits_{U_N}\!\!B(u_1, \ldots,u_N) \,
d\Psi_1(u_1)%d\Psi_2(u_2)\cdots\\
\cdots d\Psi_N(u_N)>0 \!\!\!\!\label{e30}
\end{multline}
для любого фиксированного набора $\Psi\hm=(\Psi_1, \ldots, \Psi_N)\hm\in \Gamma$.
Из неравенства~(\ref{e25}) с~уче\-том~(\ref{e29}) получаем:
\begin{multline}
\hspace*{-4mm}A\left(u_1,\ldots, u_N\right)\leq{}\\
\hspace*{-4mm}{}\leq c_0^{(+)} B\left(u_1, \ldots, u_N\right)\,, 
\left(u_1, \ldots, u_N\right)\in U\,. \label{e31}
\end{multline}
В свою очередь, из неравенства~(\ref{e31}) и~свойств интеграла следует:
\begin{multline}
\int\limits_{U_1}\!\!\cdots\! \!\int\limits_{U_N}\!\!A(u_1,\ldots, u_N) \,
d\Psi_1\left(u_1\right)%d\Psi_2\left(u_2\right)\cdots\\
\cdots d\Psi_N\left(u_N\right)\leq\\
\hspace*{-24pt}\leq 
c_0^{(+)}\!\!\int\limits_{U_1}\!\!\cdots\!\! \int\limits_{U_N}\!\!\!B\!\left(u_1,\ldots, u_N\right)
 d\Psi_1\!\left(u_1\right)\!%d\Psi_2\left(u_2\right)\cdots\\
 \cdots d\Psi_N\!\left(u_N\right)\!\! 
 \label{e32}
\end{multline}
для любого фиксированного набора $\Psi\hm=(\Psi_1, \ldots, \Psi_N)\hm\in \Gamma$. 
Но тогда из~(\ref{e32}) с~учетом~(\ref{e30}) получаем:
\begin{multline}
I(\Psi_1, \ldots, \Psi_N)={}\\
{}=
\fr{\int\nolimits_{U_1}\!\cdots\! \int\nolimits_{U_N}\!\!A\left(u_1, \ldots, u_N\right)\,
 d\Psi_1(u_1)\cdots d\Psi_N(u_N)}{
\int\nolimits_{U_1}\!\cdots\! \int\nolimits_{U_N}\!\!B\left(u_1, \ldots, u_N\right)\,
 d\Psi_1(u_1)
 \cdots d\Psi_N(u_N)}\leq{}\\
 {}\leq c_0^{(+)} 
 \label{e33}
\end{multline}
для любого фиксированного набора $(\Psi_1, \ldots\linebreak\ldots, \Psi_N)\hm\in \Gamma$.

Предположим теперь, что функция $B(u_1,\ldots, u_N)$ строго отрицательна:
\begin{equation}
B(u_1,\ldots, u_N)<0 \quad \left(u_1, \ldots, u_N\right)\in U\,. 
\label{e34}
\end{equation}
Тогда
\begin{multline}
\hspace*{-6pt}\int\limits_{U_1}\!\!\cdots\!\! \int\limits_{U_N}\!\!B\!\left(u_1,\ldots, u_N\right)\!
 d\Psi_1(u_1) \cdots d\Psi_N(u_N)<0 \!\!\!
 \label{e35}
\end{multline}
для любого фиксированного набора $(\Psi_1, \ldots\linebreak \ldots, \Psi_N)\hm\in \Gamma$.

Как и~ранее, будем исходить из неравенства~(\ref{e25}). 
При выполнении условий~(\ref{e34}) и~(\ref{e35}) характер неравенств~(\ref{e31}) 
и~(\ref{e32}) меняется на противоположный, но характер неравенства~(\ref{e33}) 
остается неизменным. Таким образом, для любой функции 
$B(u_1, u_2,\ldots, u_N)$, обладающей свойством строгой знакопостоянности, 
из условия~(\ref{e25}) следует выполнение неравенства~(\ref{e33}), 
которое совпадает с~(\ref{e26}). Первое утверждение леммы~1 доказано. 
Второе утверждение доказывается аналогично. Лемма~1 доказана.

\smallskip

\noindent
\textbf{Лемма 2.} \textit{Рассмотрим дроб\-но-ли\-ней\-ный интегральный функционал 
$I(\Psi_1, \Psi_2,\ldots, \Psi_N)$ вида}~(\ref{e11}), 
\textit{заданный на некотором множестве наборов вероятностных мер 
$\Psi\hm=(\Psi_1, \Psi_2,\ldots, \Psi_N)\hm\in \Gamma$. Предпо\-ложим, что на 
множестве~$\Gamma$ выполняется условие~$1$ из набора предварительных условий 
и~функция $B(u_1, u_2,\ldots, u_N)$ обладает свойством строгой знакопостоянности 
при всех $(u_1, u_2,\ldots, u_N)\hm\in U$. Тогда справедливы следующие утверждения}:
\begin{enumerate}[1.]
\item \textit{Если основная функция $C(u_1, u_2,\ldots, u_N)\hm=
{A(u_1, u_2,\ldots, u_N)}/{B(u_1, u_2,\ldots, u_N)}$ ограничена сверху, 
но не достигает своего максимального 
значения, то имеет место неравенство}:
\begin{multline}
I\left(\Psi_1, \Psi_2,\ldots, \Psi_N\right)<{}\\
{}< \sup\limits_{(u_1, u_2,\ldots, u_N)\in U}
 C\left(u_1, u_2,\ldots, u_N\right)<\infty \label{e36}
\end{multline}
\textit{для всех} $(\Psi_1, \Psi_2,\ldots, \Psi_N)\in \Gamma$.
\item \textit{Если основная функция $C(u_1, u_2,\ldots, u_N)\hm=
{A(u_1, u_2,\ldots, u_N)}/{B(u_1, u_2,\ldots, u_N)}$ ограничена снизу, 
но не достигает своего минимального значения, то имеет место неравенство}:
\begin{multline*}
I\left(\Psi_1, \Psi_2,\ldots, \Psi_N\right)>{}\\
{}> \inf\limits_{(u_1, u_2,\ldots, u_N)\in U} 
C\left(u_1, u_2,\ldots, u_N\right)>-\infty 
%\label{e37}
\end{multline*}
\textit{для всех} $(\Psi_1, \Psi_2,\ldots, \Psi_N)\hm\in \Gamma$.
\end{enumerate}

\noindent
Д\,о\,к\,а\,з\,а\,т\,е\,л\,ь\,с\,т\,в\,о\ \ леммы~2. 
Докажем первое утверждение леммы. Поскольку множество значений 
основной функции $C(u_1, u_2,\ldots, u_N)$ ограничено сверху, оно имеет конечную 
верхнюю грань:
$$
\exists \sup\limits_{(u_1, u_2,\ldots, u_N)\in U} 
C\left(u_1, u_2,\ldots, u_N\right)<\infty
$$
(см.~\cite[гл.~1, \S3, п.~3.4, теорема~1]{25}).

По условию функция $C(u_1, u_2,\ldots, u_N)$ не достигает своего максимального 
значения. Следовательно, выполняется неравенство:
\begin{multline}
C(u_1, u_2,\ldots, u_N)=\fr{A(u_1, u_2,\ldots, u_N)}{B(u_1, u_2,\ldots, u_N)}<{}\\
{}< 
\sup\limits_{(u_1, u_2,\ldots, u_N)\in U} C(u_1, u_2,\ldots, u_N)<\infty\,, 
\\
\left(u_1, u_2,\ldots, u_N\right)\in U\,.
\label{e38}
\end{multline}
Взяв за основу строгое неравенство~(\ref{e38}), проведем рассуждения, аналогичные тем, 
которые были проведены в~лемме~1 по отношению к~неравенству~(\ref{e25}). 
В~результате получим строгое неравенство~(\ref{e36}).

Второе утверждение леммы~2 доказывается аналогично. Лемма~2 доказана.

\noindent
Д\,о\,к\,а\,з\,а\,т\,е\,л\,ь\,с\,т\,в\,о\ 
\ теоремы~2.
Начнем с~доказательства утверждения~1. Предположим сначала, что основная 
функция $C(u_1, u_2,\ldots, u_N)={A(u_1, u_2,\ldots, u_N)}/{B(u_1, u_2,\ldots, u_N)}$ 
ограничена сверху и~достигает глобального максимума на множестве~$U$ 
в~некоторой точке $u^{(+)}\hm=\left(u^{(+)}_1,u^{(+)}_2,\ldots,u^{(+)}_N\right)\hm\in U$,
а~именно:
\begin{multline*}
\max\limits_{(u_1, u_2,\ldots, u_N)\in U} C\left(u_1, u_2,\ldots, u_N\right) = {}\\
{}=
C\left(u^{(+)}_1,u^{(+)}_2,\ldots,u^{(+)}_N\right)<\infty\,.
\end{multline*}
Тогда выполняется соотношение:
\begin{multline}
C(u_1, u_2,\ldots, u_N)=\fr{A(u_1, u_2,\ldots, u_N)}{B(u_1, u_2,\ldots, u_N)}
\leq{}\\
{}\leq C\left(u^{(+)}_1,u^{(+)}_2,\ldots,u^{(+)}_N\right)<\infty\,, 
\\
\left(u_1, u_2,\ldots, u_N\right)\in U\,.
\label{e39}
\end{multline}
Условия леммы~1 выполнены, и~можно воспользоваться ее утверждениями. 
Согласно первому из них, если выполняется неравенство~(\ref{e39}), 
то имеет место соотношение:
\begin{equation*}
I(\Psi_1, \Psi_2,\ldots, \Psi_N)\leq 
C\left(u^{(+)}_1,u^{(+)}_2,\ldots,u^{(+)}_N\right)<\infty 
%\label{e40}
\end{equation*}
для всех стратегий управления $\Psi\hm=(\Psi_1, \Psi_2,\ldots\linebreak
\ldots, \Psi_N)\hm\in \Gamma$.

Таким образом, множество значений дроб\-но-ли\-ней\-но\-го интегрального 
функционала $I(\Psi_1, \Psi_2,\ldots, \Psi_N)$ ограничено сверху при всех 
$\Psi\hm=(\Psi_1, \Psi_2,\ldots, \Psi_N)\hm\in \Gamma$. Тогда существует верхняя 
грань этого множества и~выполняется неравенство:
\begin{multline}
\sup\limits_{(\Psi_1, \Psi_2,\ldots, \Psi_N)\in \Gamma} 
I\left(\Psi_1, \Psi_2,\ldots, \Psi_N\right)\leq {}\\
{}\leq
C\left(u^{(+)}_1,u^{(+)}_2,\ldots,u^{(+)}_N\right). \label{e41}
\end{multline}
Рассмотрим детерминированную стратегию управ\-ле\-ния 
$\Psi^{*(+)}\hm=\left(\Psi_1^{*(+)}, \Psi_2^{*(+)},\ldots, \Psi_N^{*(+)}\right)$, 
в~которой каждая вероятностная мера~$\Psi_i^{*(+)}$ является вы\-рож\-ден\-ной 
и~сосредоточена в~точке $u_i^{(+)}$, $i\hm=\overline{1, N}$.
По свойству интеграла
\begin{multline}
I\left(\Psi_1^{*(+)}, \Psi_2^{*(+)},\ldots ,\Psi_N^{*(+)}\right)={}\\
{}=
C\left(u^{(+)}_1,u^{(+)}_2,\ldots,u^{(+)}_N\right). \label{e42}
\end{multline}
Из соотношений~(\ref{e41}) и~(\ref{e42}) получаем:
\begin{multline}
\sup\limits_{(\Psi_1, \Psi_2,\ldots, \Psi_N)\in \Gamma} 
I\left(\Psi_1, \Psi_2,\ldots, \Psi_N\right)\leq{}\\
{}\leq
 I\left(\Psi_1^{*(+)}, 
\Psi_2^{*(+)},\ldots, \Psi_N^{*(+)}\right). \label{e43}
\end{multline}
Заметим дополнительно, что выполняются отношения принадлежности:
\begin{equation}
\Psi^{*(+)}=\left(\Psi_1^{*(+)}, \Psi_2^{*(+)},\ldots, \Psi_N^{*(+)}\right) 
\in \Gamma^* \subset \Gamma\,. \label{e44}
\end{equation}
Из~(\ref{e44}) и~свойства верхней грани следует:
\begin{multline}
\sup\limits_{\left(\Psi_1^{*}, \Psi_2^{*},\ldots, \Psi_N^{*}\right) \in \Gamma^*} 
I\left(\Psi_1^{*}, \Psi_2^{*},\ldots, \Psi_N^{*}\right)\leq {}\\
{}\leq
\sup\limits_{\left(\Psi_1, \Psi_2,\ldots, \Psi_N\right) 
\in \Gamma} I\left(\Psi_1, \Psi_2,\ldots, \Psi_N\right)\,. 
\label{e45}
\end{multline}
Объединяя~(\ref{e42}), (\ref{e43}) и~(\ref{e45}), получаем соотношение:
\begin{multline}
\sup\limits_{\left(\Psi_1^{*}, \Psi_2^{*},\ldots, \Psi_N^{*}\right) 
\in \Gamma^*} I\left(\Psi_1^{*}, \Psi_2^{*},\ldots, 
\Psi_N^{*}\right)\leq{}\\
{}\leq \sup\limits_{\left(\Psi_1, \Psi_2,\ldots, \Psi_N\right) 
\in \Gamma} I\left(\Psi_1, \Psi_2,\ldots, \Psi_N\right)\leq{}\\
{}\leq I\left(\Psi_1^{*(+)}, \Psi_2^{*(+)},\ldots, \Psi_N^{*(+)}\right)={}\\
{}=
\fr{A\left(u^{(+)}_1,u^{(+)}_2,\ldots,u^{(+)}_N\right)}{B\left(u^{(+)}_1,u^{(+)}_2,
\ldots,u^{(+)}_N\right)}\,.
 \label{e46}
\end{multline}
Из соотношения~(\ref{e46}) с~учетом~(\ref{e44}) получаем, что максимум 
функционала $I(\Psi_1, \Psi_2,\ldots, \Psi_N)$ на множестве допустимых стратегий 
$\Psi\hm=(\Psi_1, \Psi_2,\ldots, \Psi_N)\hm\in \Gamma$ существует и~достигается 
на детерминированной стратегии $\left(\Psi_1^{*(+)}, \Psi_2^{*(+)},\ldots, 
\Psi_N^{*(+)}\right)$.

Кроме того, выполняются соотношения~(\ref{e19}). Таким образом, утверждение~1 
в~случае, когда основная функция $C(u_1, u_2,\ldots, u_N)$ достигает глобального 
максимума, доказано. Соответствующее утверждение в~случае, когда основная функция 
$C(u_1, u_2,\ldots, u_N)$ достигает глобального минимума, доказывается аналогично. 
При этом используется второе утверждение леммы~1.

\smallskip

Перейдем к~доказательству второго утверждения теоремы~2. Предположим, что основная 
функция $C(u_1, u_2,\ldots, u_N)\hm=A(u_1, u_2,\ldots$\linebreak
$\ldots, u_N)/{B(u_1, u_2,\ldots, u_N)}$ 
ограничена сверху, но не достигает глобального максимума на множестве 
$U \hm= U_1 \times U_2 \times \cdots \times U_N$. Тогда множество значений 
основной функции имеет конечную верхнюю грань:

\noindent
\begin{multline*}
C\left(u_1, u_2,\ldots, u_N\right)=\fr{A(u_1, u_2,\ldots, u_N)}
{B(u_1, u_2,\ldots, u_N)}<{}\\
{}<
\sup\limits_{(u_1, u_2,\ldots, u_N)\in U} \fr{A(u_1, u_2,\ldots, u_N)}
{B(u_1, u_2,\ldots, u_N)}<\infty\,, 
\\
\left(u_1, u_2,\ldots, u_N\right)\in U\,.
%\label{e47}
\end{multline*}
По определению верхней грани для любого фиксированного $\varepsilon \hm>0$ 
существует точка $(u_1^{(+)}(\varepsilon), u_2^{(+)}(\varepsilon),\ldots, 
u_N^{(+)}(\varepsilon))$ такая, что выполняется двойное неравенство~(\ref{e21}) 
(см.~\cite[гл.~1, \S\,3, п.~3.4]{25}). Иначе говоря, значение основной функции 
в~указанной точке лежит в~левой \mbox{$\varepsilon$-окрест}\-ности верхней грани. 
Рассмотрим детерминированную стратегию управления 
$\Psi^{*(+)}(\varepsilon)\hm=\!\left(\Psi_1^{*(+)}(\varepsilon), 
\Psi_2^{*(+)}(\varepsilon),\ldots, \Psi_N^{*(+)}(\varepsilon)\!\right)$, компонентами\linebreak 
которой являются вырожденные вероятностные меры $\Psi_1^{*(+)}(\varepsilon), 
\Psi_2^{*(+)}(\varepsilon),\ldots, \Psi_N^{*(+)}(\varepsilon)$, причем вырожденная 
мера~$\Psi_i^{*(+)}(\varepsilon)$ сосредоточена в~точке~$u_i^{(+)}(\varepsilon)$,
$i\hm=1,2,\ldots,N$.

По свойству интеграла
\begin{multline}
I\left(\Psi_1^{*(+)}(\varepsilon), \Psi_2^{*(+)}(\varepsilon),\ldots,
 \Psi_N^{*(+)}(\varepsilon)\right)={}\\
 {}=
 C\left(u_1^{(+)}(\varepsilon), u_2^{(+)}(\varepsilon),\ldots, 
 u_N^{(+)}(\varepsilon)\right)\,. 
 \label{e48}
\end{multline}
Из соотношения~(\ref{e48}) с~учетом указанного свойства основной функции получаем:
\begin{multline}
\sup\limits_{(u_1, u_2,\ldots, u_N)\in U} \fr{A(u_1, u_2,\ldots, u_N)}
{B(u_1, u_2,\ldots, u_N)}-\varepsilon<{}\\
{}< I\left(\Psi_1^{*(+)}(\varepsilon), 
\Psi_2^{*(+)}(\varepsilon),\ldots, \Psi_N^{*(+)}(\varepsilon)\right)<{}
\\
{}< \sup\limits_{(u_1, u_2,\ldots, u_N)\in U} \fr{A(u_1, u_2,\ldots, u_N)}
{B(u_1, u_2,\ldots, u_N)}<\infty\,. 
\label{e49}
\end{multline}
Заметим также, что в~рассматриваемом случае выполнены условия леммы~2. 
Воспользуемся первым утверждением этой леммы, а~именно соотношением~(\ref{e36}):
\begin{multline}
I(\Psi_1, \Psi_2,\ldots, \Psi_N)< {}\\
{}<\sup\limits_{(u_1, u_2,\ldots, u_N)
\in U} \fr{A(u_1, u_2,\ldots, u_N)}{B(u_1, u_2,\ldots, u_N)}<\infty 
\label{e50}
\end{multline}
для всех $(\Psi_1, \Psi_2,\ldots, \Psi_N)\in\Gamma$.

Из соотношений~(\ref{e49}) и~(\ref{e50}) следует, что детерминированная стратегия 
$\Psi^{*(+)}(\varepsilon)\hm=\left(\Psi_1^{*(+)}(\varepsilon), \Psi_2^{*(+)}(\varepsilon),
\ldots, \Psi_N^{*(+)}(\varepsilon)\right)$, опре\-де\-ля\-емая набором вырожденных 
вероятностных мер, сосредоточенных в~соответствующих точках 
$\left(u_1^{(+)}(\varepsilon), u_2^{(+)}(\varepsilon),\ldots, 
u_N^{(+)}(\varepsilon)\right)$, является $\varepsilon$-оп\-ти\-маль\-ной. 
Вторая часть утверждения~2 теоремы~2, связанная со свойствами нижней грани, 
доказывается аналогично.

Докажем третье утверждение теоремы~2. Предположим, что множество значений 
основной функции $C(u_1, u_2,\ldots, u_N)\hm=
A(u_1, u_2,\ldots$\linebreak $\ldots, u_N)/{B(u_1, u_2,\ldots, u_N)}$
не является ограниченным сверху на множестве $U\hm=U_1\times U_2 \times \cdots $\linebreak
$\cdots \times U_N$.
Тогда существует последовательность\linebreak точек $\left(u_1^{(+)}(n), u_2^{(+)}(n),
\ldots,u_N^{(+)}(n)\right)\hm\in U$, $n\hm=1,2,\ldots$, для которой
\begin{multline}
C\left(u_1^{(+)}(n), u_2^{(+)}(n),\ldots,u_N^{(+)}(n)\right)={}\\
{}=
\fr{A\left(u_1^{(+)}(n), u_2^{(+)}(n),\ldots,u_N^{(+)}(n)\right)}
{B\left(u_1^{(+)}(n), u_2^{(+)}(n),\ldots,u_N^{(+)}(n)\right)}
\longrightarrow \infty \,,\\
n\rightarrow \infty\,.
\label{e51}
\end{multline}
Зафиксируем некоторую последовательность точек $\left(u_1^{(+)}(n), u_2^{(+)}(n),
\ldots,u_N^{(+)}(n)\right)\hm\in U$, $n\hm=1,2,\ldots$, обладающих указанным свойством, 
и~рассмотрим последовательность детерминированных  стратегий управления 
$\Psi^{*(+)}(n)\hm=\left(\Psi_1^{*(+)}(n), \Psi_2^{*(+)}(n),\ldots, 
\Psi_N^{*(+)}(n)\right)$, $n\hm=1,2,\ldots$, определяемых набором вырожденных 
вероятностных мер, сосредоточенных в~соответствующих точках 
$\left(u_1^{(+)}(n), u_2^{(+)}(n),\ldots,u_N^{(+)}(n)\right)$, $n\hm=1,2,\ldots$ 
По свойству интеграла для любого фиксированного значения $n=1,2,\ldots$ 
выполняется равенство:
\begin{multline}
I \left(\Psi^{*(+)}(n)\right)={}\\
{}=I\left(\Psi_1^{*(+)}(n), \Psi_2^{*(+)}(n),\ldots,
 \Psi_N^{*(+)}(n)\right)={}\\
{}=\fr{A\left(u_1^{(+)}(n), u_2^{(+)}(n),\ldots,u_N^{(+)}(n)\right)}
{B\left(u_1^{(+)}(n), u_2^{(+)}(n),\ldots,u_N^{(+)}(n)\right)}\,. 
\label{e52}
\end{multline}
Из соотношений~(\ref{e51}) и~(\ref{e52}) следует, что
\begin{multline}
I\left(\Psi^{*(+)}(n)\right)={}\\
{}=I\left(\Psi_1^{*(+)}(n), \Psi_2^{*(+)}(n),\ldots, 
\Psi_N^{*(+)}(n)\right)\longrightarrow\infty\,,\\ 
n \rightarrow\infty\,.
 \label{e53}
\end{multline}
Соотношение~(\ref{e53}) означает, что множество значе\-ний дроб\-но-ли\-ней\-но\-го 
интегрального функциона\-ла $I(\Psi_1, \Psi_2,\ldots, \Psi_N)$ вида~(\ref{e11}) 
не ограничено сверху\linebreak на множестве наборов вырожденных вероятностных мер 
$\left(\Psi_1^{*(+)}(n), \Psi_2^{*(+)}(n),\ldots, \Psi_N^{*(+)}(n)\right)\hm\in\Gamma^*$, 
а~следовательно, и~на более широком\linebreak множестве наборов вероятностных 
мер $(\Psi_1, \Psi_2,\ldots$\linebreak $\ldots, \Psi_N)\hm\in\Gamma$. В~таком случае решения экстремальной 
задачи~(\ref{e18}) в~форме задачи на максимум не существует. Соответствующее утвержде\-ние 
для варианта, когда множество значений основной функции $C(u_1, u_2,\ldots,u_N)
\hm=A(u_1, u_2,\ldots$\linebreak $\ldots,u_N)/{B(u_1, u_2,\ldots,u_N)}$ 
не является ограниченным снизу, доказывается аналогично. Третье утверж\-де\-ние теоремы~2 
доказано. Тем самым тео\-ре\-ма~2 доказана полностью.

\smallskip

Применим теорему~2 для решения поставленной задачи оптимального управления. 
Из утверждения этой теоремы следует, что для доказательства су-\linebreak ществования 
оптимального управ\-ле\-ния и~его нахождения необходимо исследовать на 
глобальный экстремум основную функцию дроб\-но-ли\-ней\-но\-го интегрального 
функционала $C(u_1,u_2,\ldots,u_N)$, определяемую формулой~(\ref{e17}) с~учетом 
равенств~(\ref{e12})--(\ref{e16}). В~некоторых случаях, например когда основной 
процесс~$\xi(t)$ является регенерирующим, а~стоимостные характеристики 
модели задаются линейными функциями, такое исследование можно провести 
аналитически. Однако для подавляющего большинства полумарковских моделей 
для этого необходимо использовать численные методы.

\section{Заключение}

В заключительной части работы приведем \mbox{краткое} описание теоретической 
основы метода решения задачи оптимального управления полумарковским 
процессом с~конечным множеством состояний.

\begin{enumerate}[1.]
\item Исходная проблема оптимального управления формулируется в~виде 
экстремальной задачи~(\ref{e18}). Целевым показателем качества управ\-ле\-ния в~данной задаче 
служит величина~(\ref{e10}), которая имеет характер средней удельной прибыли.
\item Доказывается, что стационарный показатель~(\ref{e10}) представим в~виде 
дроб\-но-ли\-ней\-но\-го интегрального функционала~(\ref{e11}), для которого явно 
определяются подынтегральные функции числителя и~знаменателя, а~следовательно, 
и~основная функция данного функционала.
\item Используется теорема об экстремуме дроб\-но-ли\-ней\-но\-го интегрального 
функционала. На основании утверждений этой теоремы уста\-нав\-ли\-ва\-ет\-ся, что 
исходная задача оптимального управления сводится к~исследованию на глобальный 
экстремум основной функции этого функционала, для которой получено явное 
аналитическое представление.
\end{enumerate}

Заметим, что такое исследование задач оптимального управления 
стохастическими системами фактически уже было проведено в~ряде работ П.\,В.~Шнуркова 
и~его соавторов. В~частности, в~работе~\cite{26} была рассмотрена модель 
управления для обрывающегося процесса восстановления, описывающего функционирование 
некоторой технической системы. Задача управления решалась для различных показателей 
эффективности и~надежности этой системы, имеющих структуру дроб\-но-ли\-ней\-но\-го 
интегрального функционала.

В работах~\cite{27, 28} рассматривались модели регенерирующих процессов 
для исследования сис\-тем управления запасами. Различные показатели качества 
управления были представлены в~форме дроб\-но-ли\-ней\-ных интегральных функционалов. 
Основные функции этих функционалов были найде\-ны в~явной форме и~исследовались 
на глобальный экстремум. В~работах~\cite{21,29} рассматривалась достаточно 
сложная полумарковская модель с~конечным множеством состояний, описывающая 
сис\-те\-му управления запасом непрерывного продукта. Показатели качества управления в~этой 
модели также имели структуру дроб\-но-ли\-ней\-ных интегральных функционалов, 
для основных функций которых были найдены явные аналитические представления. 
Упомянем также работы~\cite{30, 31}, в~которых была исследована полумарковская 
модель с~дис\-крет\-но-не\-пре\-рыв\-ным фазовым пространством. Показатели 
качества управления в~этой  модели были найдены в~явной форме как функции от 
двух непрерывных параметров управления.

Фактически во всех упомянутых работах уже был использован метод решения задачи 
оптимального управления регенерирующим или полумарковским случайным процессом, 
основанный на исследовании экстремальных свойств основной функции соответствующего 
дроб\-но-ли\-ней\-но\-го интегрального функционала. Из соображений, изложенных 
во\linebreak введении, следует, что в~период написания и~пуб\-ли\-кации этих работ данный метод 
не имел стро\-гого обоснования. Однако после публикации\linebreak работы~\cite{14} и~настоящего 
исследования можно утверж\-дать, что полученные в~них результаты полностью теоретически 
обоснованы.

Таким образом, изложенный выше метод решения проблемы оптимального управления 
полумарковскими процессами с~конечными множествами состояний может быть успешно 
реализован для многих задач, рассматриваемых в~различных областях прикладной 
теории вероятностей.

Практическая реализация численной процедуры поиска оптимального решения на примере\linebreak 
полумарковской модели управления запасом непрерывного продукта (подробнее 
см.~\cite{21, 29}), ба\-зи\-ру\-юща\-яся на изложенных выше результатах (в~частности, 
теореме~1), была осуществлена А.\,К.~Горшениным и~соавторами 
в~статье~\cite{Gorshenin2015}. Коротко опишем наиболее важные аспекты этой работы.

Для решения поставленной задачи опти\-мального управления была создана 
специальная программа \verb"Inventory" на встроенном языке программирования 
пакета \verb"MATLAB", ее возможности\linebreak кратко представле\-ны в~упомянутой ранее 
\mbox{статье}~\cite{Gorshenin2015}. В~программе \verb"Inventory" реализованы функции 
для оценивания через заданные исходные параметры вероятностных и~стоимостных 
характеристик модели, которые в~дальнейшем используются для поиска значений 
основной функции дроб\-но-ли\-ней\-но\-го функционала~(\ref{e17}). Точка глобального 
экстремума этой функции и~определяет оптимальное управление.

В качестве начальных данных необходимо задание следующих параметров:
\begin{itemize}
\item спрос и~вместимость склада;
\item разбиение множества значений объема запаса;
\item вероятностные характеристики, описывающие модель пополнения запаса;
\item условные математические ожидания длительностей задержек пополнения запаса;
\item функции для характеризации затрат и~доходов.
\end{itemize}

По итогам работы программы \verb"Inventory" ряд вспомогательных функций 
представляется в~аналитической форме (в частности, с~использованием аппарата 
символьных вычислений  \verb"Symbolic Toolbox"\linebreak пакета \verb"MATLAB"), выводится 
точка глобального экстремума функции нескольких вещественных переменных~(\ref{e17}), 
найденная с~помощью применения численных и~при\-бли\-жен\-но-ана\-ли\-ти\-че\-ских\linebreak 
аппроксимаций. 
Также формируются графики оценок значений ве\-ро\-ят\-ност\-но-сто\-и\-мост\-ных 
характеристик 
и~основной функции дроб\-но-ли\-ней\-но\-го функционала~(\ref{e17}), либо трехмерных 
сечений в~случае наличия более трех параметров управления (переменных).

Функциональность пакета \verb"Inventory" может быть расширена для практической 
реализации метода решения задачи поиска оптимального управ\-ле\-ния полумарковскими 
процессами с~конечными множествами состояний, рассмотренного в~данной статье.


 {\small\frenchspacing
 {%\baselineskip=10.8pt
 \addcontentsline{toc}{section}{References}
 \begin{thebibliography}{99}
 \bibitem{1}
\Au{Ховард Р.} Динамическое программирование и~марковские процессы~/ 
Пер. с~англ.~--- М.: Сов. радио, 1964. 189~с.
(\Au{Howard~R.\,A.} Dynamic programming and Markov processes.~--- 
Cambridge, MA, USA: MIT Press, 1960. 136~p.)
\bibitem{2} 
\Au{Рыков В.\,В.} Управляемые марковские процессы с~конечными пространствами 
состояний и~управлений~// Теория вероятностей и~ее применения, 1966. Т.~11. 
Вып.~2. С.~343--351.
\bibitem{3} 
\Au{Джевелл В.} Управляемые полумарковские процессы~// Кибернетич. сборник.~--- 
М.: Мир, 1967. Вып.~4. С.~97--134.
%{\em Jewell W.\,S.} Markov-renewal programming~// Operations Research, 1963. Vol.~11. P.~938--971.
\bibitem{4} 
\Au{Fox B.} Markov renewal programming by linear fractional programming~// 
SIAM J.~Appl. Math., 1966. Vol.~14. P.~1418--1432.
\bibitem{5} 
\Au{Denardo E.\,V.} Contraction mappings in the theory underlying dinamic programming~// 
SIAM Rev., 1967. Vol.~9. P.~165--177.

\bibitem{6} 
\Au{Howard R.\,A.} Research in semi-Markovian decision structures~// 
J.~Oper. Res. Soc. Japan, 1963. Vol.~6. P.~163--199.
\bibitem{7} 
\Au{Osaki S., Mine H.} Linear programming algorithms for Markovian decision processes~//
 J.~Math. Anal. Appl., 1968. Vol.~22. P.~356--381.
\bibitem{8} 
\Au{Майн Х., Осаки С.} Марковские процессы принятия решений~/ Пер. с~англ.~--- 
М.: Наука, 1977. 176~с.
(\Au{Mine~H., Osaki~S.} 
Markovian decision processes.~--- New York, NY, USA: 
American Elsevier Publishing Co., 1970. 142~p.)
\bibitem{9} 
\Au{Гихман И.\,И., Скороход А.\,В.} Управляемые случайные процессы.~--- 
Киев: Наукова думка, 1977. 251~с.
\bibitem{10} 
\Au{Luque-Vasquez F., Herndndez-Lerma~О.} Semi-Markov control models with average costs~// 
Appl. Math., 1999. Vol.~26. No.\,3. P.~315--331.
\bibitem{11} 
\Au{Vega-Amaya O., Luque-Vasquez~F.} Sample-path average cost optimality for 
semi-Markov control processes on Borel spaces: Unbounded costs and mean holding times~// 
Appl. Math., 2000. Vol.~27. No.\,3. P.~343--367.
\bibitem{12} 
Вопросы математической теории надежности~/ Под ред. Б.\,В. Гнеденко.~--- 
М.: Радио и~связь, 1983. 376~с.
\bibitem{13} 
\Au{Барзилович Е.\,Ю., Каштанов~В.\,А.} Некоторые математические вопросы теории 
обслуживания сложных систем.~---  М.: Сов. радио, 1971. 272~с.
\bibitem{14} 
\Au{Шнурков П.\,В.} О~решении проблемы безусловного экстремума для 
дроб\-но-ли\-ней\-но\-го интегрального функционала на множестве вероятностных мер~// 
Докл. РАН. Сер. Математика, 2016. Т.~470. №\,4. C.~387--392.
\bibitem{15} 
\Au{Ширяев А.\,Н.}  Вероятность.~--- М.:~МЦНМО, 2011. Кн.~1. 552~с. Кн.~2. 968~с.
\bibitem{16} 
\Au{Боровков А.\,А.} Теория вероятностей.~--- М.: Либроком, 2009. 656~c.
\bibitem{17} 
\Au{Хеннекен П.\,Л., Тортра А.} Теория вероятностей 
и~некоторые ее приложения.~--- М.: Наука, 1974. 472~c.
\bibitem{18} 
\Au{Халмош П.} Теория меры~/ Пер. с~англ.~--- М.: ИЛ, 1953. 282~c.
(\Au{Halmos~P.} Measure theory.~--- Litton Educational Publishing, Inc. 1950. 304~p.)
\bibitem{19} 
\Au{Королюк В.\,С., Турбин~А.\,Ф.} Полумарковские процессы и~их приложения.~--- 
Киев:~Наукова думка, 1976. 184~с.
\bibitem{20} 
\Au{Janssen J., Manca R.} Applied semi-Markov processes.~--- New York,
NY, USA: Springer, 2006. 309~p.
\bibitem{21} 
\Au{Шнурков П.\,В., Иванов~А.\,В.} Анализ дискретной полумарковской модели
 управления запасом непрерывного продукта при периодическом прекращении потребления~// 
 Дискретная математика, 2014. Т.~26. Вып.~1. С.~143--154.
\bibitem{22} 
\Au{Иванов~А.\,В.} Анализ дискретной полумарковской модели
 управления запасом непрерывного продукта при периодическом прекращении 
 потребления.~--- М.: НИУ ВШЭ, 2014.  Дисс.\ \ldots\ канд. физ.-мат. наук. 120~с.
\bibitem{23}  %23
\Au{Bajalinov~E.\,B.} Linear-fractional programming. 
Theory, methods, applications and software.~--- 
Boston/\linebreak Dordrecht/London: Kluwer Academic Publs., 2003. 423~p.

\bibitem{27} %27
\Au{Шнурков П.\,В., Мельников~Р.\,В.} Оптимальное управление запасом 
непрерывного продукта в~модели регенерации~// Обозрение прикладной 
и~промышленной математики, 2006. Т.~13. Вып.~3. С.~434--452.
\bibitem{28} 
\Au{Шнурков П.\,В., Мельников~Р.\,В.} 
Исследование проб\-ле\-мы управления запасом непрерывного продукта при детерминированной 
задержке поставки~// Автоматика и~телемеханика, 2008. Т.~10. С.~93--113.


\bibitem{24}  %26
\Au{Шнурков П.\,В.} Методы исследования задач оптимального обслуживания 
в~математической теории надежности.~--- 
М.: МИЭМ, 1983.  Дисс.\ \ldots\ канд. физ.-мат. наук.

 \bibitem{25}  %25
\Au{Кудрявцев Л.\,Д.} Курс математического анализа. Т.~1.~--- 
М.: Дрофа, 2006. 704~с.

\bibitem{26} %24
\Au{Шнурков П.\,В.} Оптимальное обслуживание на периоде 
до первого отказа системы~// Применение аналитических методов в~вероятностных
 задачах.~--- Киев: Институт математики АН УССР, 1986. С.~121--129.

\bibitem{29} 
\Au{Шнурков П.\,В., Иванов~А.\,В.} Исследование задачи оптимизации в~дискретной 
полумарковской модели управления непрерывным запасом~// Вестник МГТУ им.\ 
Н.\,Э. Баумана. Сер.\ Естественные науки, 2013. Т.~3. Вып.~50. С.~62--87.
\bibitem{30} 
\Au{Shnourkoff P.\,V.} The two-element system with one 
restoring device optimum maintenance~// Stoch. Anal. Appl., 1997. 
Vol.~15. No.\,5. P.~823--837.
\bibitem{31} 
\Au{Shnourkoff P.\,V.} The two-element system optimum maintenance tills the first fail~// 
Stoch. Anal. Appl., 2001. Vol.~19. No.\,6. P.~1005--1024.
\bibitem{Gorshenin2015} 
\Au{Gorshenin~A.\,K., Belousov~V.\,V., Shnourkoff~P.\,V.,
Ivanov~A.\,V.} Numerical research of the optimal control problem in the semi-Markov 
inventory model~// AIP Conference Proceedings, 2015. Vol.~1648. {250007}. 4~p.
%\bibitem{33} {\em Горшенин А.\,К., Белоусов В.\,В., Шнурков П.\,В.} 2016. Система управления запасами на основе стохастических полумарковских моделей. Свидетельство о государственной регистрации программы для ЭВМ \textnumero 2016619021.
 \end{thebibliography}

 }
 }

\end{multicols}

\vspace*{-6pt}

\hfill{\small\textit{Поступила в~редакцию 15.07.16}}

%\vspace*{8pt}

\newpage

\vspace*{-24pt}

%\hrule

%\vspace*{2pt}

%\hrule

%\vspace*{8pt}


\def\tit{ANALYTICAL SOLUTION OF~THE~OPTIMAL CONTROL TASK OF~A~SEMI-MARKOV 
PROCESS WITH~FINITE SET OF~STATES}

\def\titkol{Analytical solution of~the~optimal control task of~a~semi-Markov 
process with~finite set of~states}

\def\aut{P.\,V.~Shnurkov$^{1}$, A.\,K.~Gorshenin$^{2}$, and~V.\,V.~Belousov$^{2}$}

\def\autkol{P.\,V.~Shnurkov, A.\,K.~Gorshenin, and~V.\,V.~Belousov}

\titel{\tit}{\aut}{\autkol}{\titkol}

\vspace*{-9pt}


    
\noindent
$^1$National Research University Higher School of Economics, 34~Tallinskaya Str., 
Moscow, 123458, Russian\linebreak
$\hphantom{^9}$Federation

\noindent
$^2$Institute of Informatics Problems, Federal Research Center 
``Computer Science and Control'' of the Russian\linebreak
$\hphantom{^9}$Academy of Sciences, 44-2~Vavilova Str., 
Moscow 119333, Russian Federation



\def\leftfootline{\small{\textbf{\thepage}
\hfill INFORMATIKA I EE PRIMENENIYA~--- INFORMATICS AND
APPLICATIONS\ \ \ 2016\ \ \ volume~10\ \ \ issue\ 4}
}%
 \def\rightfootline{\small{INFORMATIKA I EE PRIMENENIYA~---
INFORMATICS AND APPLICATIONS\ \ \ 2016\ \ \ volume~10\ \ \ issue\ 4
\hfill \textbf{\thepage}}}

\vspace*{3pt}


\Abste{The theoretical verification of the new method of finding 
the optimal strategy of control of a~semi-Markov process with finite set of states is 
presented. The paper considers Markov randomized strategies of control, determined by 
a~finite collection of probability measures, corresponding to each state. The quality 
characteristic is the stationary cost index. This index is a~linear-fractional integral 
functional, depending on collection of probability measures, giving the strategy of control. 
Explicit analytical forms of integrands of numerator and denominator of this 
linear-fractional integral functional are known. The basis of consequent results is 
the new generalized and strengthened form of the theorem about an extremum of 
a~linear-fractional integral functional. It is proved that problems of existence 
of an optimal control strategy of a~semi-Markov process and finding this strategy 
can be reduced to the task of numerical analysis of global extremum for 
the given function, depending on finite number of real arguments.}

\KWE{optimal control of a~semi-Markov process; stationary cost index of quality control; 
linear-fractional integral functional}




\DOI{10.14357/19922264160408} 

\vspace*{-16pt}

\Ack
\noindent
The research was partially supported by the Russian Foundation 
for Basic Research (project 15-07-05316).



%\vspace*{3pt}

  \begin{multicols}{2}

\renewcommand{\bibname}{\protect\rmfamily References}
%\renewcommand{\bibname}{\large\protect\rm References}

{\small\frenchspacing
 {%\baselineskip=10.8pt
 \addcontentsline{toc}{section}{References}
 \begin{thebibliography}{99}
\bibitem{1-1}
\Aue{Howard,~R.\,A.} 1960. \textit{Dynamic programming and Markov processes}. 
Cambridge, MA: MIT Press. 136~p.
\bibitem{2-1}
\Aue{Rykov,~V.\,V.} 1966. Upravlyaemye markovskie protsessy 
s~konechnymi prostranstvami sostoyaniy i~upravleniy 
[Controlled Markov processes with finite spaces of states and controls ]. 
\textit{Teoriya veroyatnostey i~ee primeneniya} 
[Theory of Probability and Its Applications] 11(2):343--351.
\bibitem{3-1}
\Aue{Jewell,~W.\,S.} 1963. Markov-renewal programming. 
\textit{Oper. Res.} 11:938--971.
\bibitem{4-1}
\Aue{Fox,~B.} 1966. Markov renewal programming by linear fractional programming. 
\textit{SIAM J.~Appl. Math.} 14:1418--1432.
\bibitem{5-1}
\Aue{Denardo, E.\,V.} 1967. Contraction mappings in the theory underlying dinamic 
programming. \textit{SIAM Rev.} 9:165--177.
\bibitem{6-1}
\Aue{Howard,~R.\,A.} 1963. Research in semi-Markovian decision structures. 
\textit{J.~Oper. Res. Soc. Japan} 6:163--199.
\bibitem{7-1}
\Aue{Osaki,~S., and H.~Mine.} 1968. Linear programming algorithms 
for Markovian decision processes. \textit{J.~Math. Anal. Appl.} 22:356--381.
\bibitem{8-1}
\Aue{Mine,~H., and S.~Osaki.} 1970. 
\textit{Markovian decision processes}. New York, NY: Elsevier. 142~p.
\bibitem{9-1}
\Aue{Gikhman,~I.\,I., and A.\,V.~Skorokhod.} 1977. 
\textit{Upravlyaemye sluchaynye protsessy} 
[Controlled random processes]. Kiev: Naukova Dumka. 251~p.
\bibitem{10-1}
\Aue{Luque-Vasquez,~F., and О.~Herndndez-Lerma.} 1999. 
Semi-Markov control models with average costs. \textit{Appl. Math.} 26(3):315--331.
\bibitem{11-1}
\Aue{Vega-Amaya,~O., and  F.~Luque-Vasquez.} 2000.  
Sample-path average cost optimality for semi-Markov control processes on Borel spaces: 
Unbounded costs and mean holding times. \textit{Appl. Math.} 27(3):343--367.
\bibitem{12-1}
Gnedenko,~B.~V., ed. 1983. 
\textit{Voprosy matematicheskoy teorii nadezhnosti} 
[Problems of the mathematical theory of reliability].  Moscow: Radio i~svyaz'. 376~p.
\bibitem{13-1}
\Aue{Barzilovich,~E.\,Yu., and V.\,A.~Kashtanov.} 1971. 
\textit{Nekotorye matematicheskie voprosy teorii obsluzhivaniya slozhnykh sistem}  
[Some mathematical questions in theory of complex systems maintenance]. 
Moscow: Sovetskoe radio. 272~p.
\bibitem{14-1}
\Aue{Shnurkov,~P.\,V.} 2016. Solution of the unconditional extremum problem for 
a~linear-fractional 
integral functional on a~set of probability measures. 
\textit{Dokl. Math.} 94(2):550--554.
\bibitem{15-1} %14
\Aue{Shiryaev,~A.\,N.} 2016. 
\textit{Probability-1}. Graduate texts in mathematics ser.
New York, NY: Springer. Vol.~95. 503~p.;
2017. \textit{Probability-2.} Vol.~900. 500~p.
\bibitem{16-1}
\Aue{Borovkov,~А.\,А.} 2009. 
\textit{Teoriya veroyatnostey} [Probability theory]. Moscow: Librokom. 656~p.
\bibitem{17-1}
\Aue{Khenneken,~P.\,L., and A.~Tortra.} 1974. 
\textit{Teoriya veroyatnostey i~nekotorye ee prilozheniya} 
[Probability theory and some of its applications]. Moscow: Nauka. 472~p.
\bibitem{18-1}
\Aue{Halmos,~P.} 1950. \textit{Measure theory}. Litton Educational Publishing. 304~p.
\bibitem{19-1}
\Aue{Korolyuk, V.\,S., and A.\,F.~Turbin.} 1976. 
\textit{Polumarkovskie protsessy i~ikh prilozheniya} 
[Semi-Markov processes and their applications]. Kiev: Naukova Dumka. 184~p.
\bibitem{20-1}
\Aue{Janssen,~J., and R.~Manca.} 2006. 
\textit{Applied semi-Markov processes}. New York, NY: Springer. 309~p.
\bibitem{21-1}
\Aue{Shnurkov,~P.\,V, and A.\,V~Ivanov.} 2015. Analysis of a~discrete semi-Markov model of continuous inventory 
control with periodic interruptions of consumption. 
\textit{Discrete Math. \mbox{Appl}.} 25(1):59--67.
\bibitem{22-1} %21
\Aue{Ivanov,~A.\,V.} 2014. Analiz diskretnoy polumarkovskoy modeli upravleniya 
zapasom nepreryvnogo produkta pri periodicheskom prekrashchenii potrebleniya 
[Analysis of a~discrete semi-Markov control model of continuous product inventory 
in a~periodic cessation of consumption].  
Moscow: Natsional'nyy Issledovatel'skiy Universitet ``Vysshaya Shkola Ekonomiki.'' 
PhD Thesis. 120~p.
\bibitem{23-1} %22
\Aue{Bajalinov,~E.\,B.} 2003. 
\textit{Linear-fractional programming. Theory, methods, applications and software}. 
Boston/\linebreak Dordrecht/London: Kluwer Academic Publs. 423~p.
\bibitem{26-1} %24
\Aue{Shnurkov,~P.\,V., and R.\,V.~Mel'nikov.} 2006. Optimal'noe upravlenie 
zapasom nepreryvnogo produkta v modeli regeneratsii [Optimal control of 
a~continuous product inventory in the regeneration model]. 
\textit{Obozrenie prikladnoy i~promyshlennoy matematiki} [Rev. Appl. Ind. Math.]
13(3):434--452.

\bibitem{25-1} %25
\Aue{Shnurkov,~P.\,V., and R.\,V.~Mel'nikov.} 2008. 
Analysis of the problem of continuous-product inventory control under deterministic 
lead time. \textit{Automat. Rem. Contr.} 69(10):1734--1751.

\columnbreak

\bibitem{24-1} %26
\Aue{Shnurkov,~P.\,V.} 1983. Metody issledovaniya zadach optimal'nogo obsluzhivaniya 
v~matematicheskoy teorii nadezhnosti [Research methods of optimal service problems 
in the mathematical theory of reliability].  
Moscow: Moskovskiy Institut Elektronnogo Mashinostroeniya.  PhD Thesis. 


\bibitem{27-1} %27
\Aue{Kudryavtsev,~L.\,D.} 2006. 
\textit{Kurs matematicheskogo analiza} 
[A~course of mathematical analysis]. Vol.~1. Moscow: Drofa. 704~p.

\bibitem{28-1}
\Aue{Shnurkov,~P.\,V.} 1986. Optimal'noe obsluzhivanie na periode do 
pervogo otkaza sistemy [The optimum service period until the first system failure]. 
\textit{Primenenie analiticheskikh metodov v~veroyatnostnykh zadachakh} 
[The application of analytical methods in probabilistic tasks]. Kiev:
Institute of Mathematics of the Academy of Sciences of the USSR. 121--129.

\bibitem{29-1}
\Aue{Shnurkov,~P.\,V., and A.\,V.~Ivanov.} 2013. Issledovanie zadachi optimizatsii 
v~diskretnoy polumarkovskoy modeli upravleniya nepreryvnym zapasom 
[Study of the optimization problem in discrete semi-Markov model of continuous 
inventory control]. \textit{Vestnik MGTU im.\ N.\,E.~Baumana. Ser. 
Estestvennye nauki} [Vestnik of MSTU named after N.\,E.~Bauman. Ser. Natural sciences] 
3(50):62--87.
\bibitem{30-1}
\Aue{Shnourkoff,~P.\,V.} 1997. The two-element system with one restoring device 
optimum maintenance.  \textit{Stoch. Anal. Appl.} 15(5):823--837.
\bibitem{31-1}
\Aue{Shnourkoff,~P.\,V.} 2001. The two-element system optimum maintenance tills 
the first fail. \textit{Stoch. Anal. Appl.} 19(6):1005--1024.
\bibitem{32-1}
\Aue{Gorshenin,~A.\,K., V.\,V.~Belousov, P.\,V.~Shnourkoff, and A.\,V.~Ivanov.}
2015. Numerical research of the optimal control problem in the semi-Markov 
inventory model. \textit{AIP Conference Proceedings} 1648:250007.
\end{thebibliography}

 }
 }

\end{multicols}

\vspace*{-3pt}

\hfill{\small\textit{Received July 15, 2016}}

\Contr

\noindent
\textbf{Shnurkov Peter V.} (b.\ 1953)~---
 Candidate of Science (PhD) in physics and mathematics, 
 associate professor, National Research University Higher School of Economics, 
 34~Tallinskaya Str., Moscow 123458, Russian Federation; \mbox{pshnurkov@hse.ru} 
 
 \vspace*{3pt}
 
 \noindent
\textbf{Gorshenin Andrey K.}  (b.\ 1986)~---
Candidate of Science (PhD) in physics and mathematics, leading scientist, 
Institute of Informatics Problems, Federal Research Center ``Computer Science 
and Control'' of the Russian Academy of Sciences, 44-2~Vavilov Str., Moscow 119333, 
Russian Federation; associate professor, Federal State Budget Educational 
Institution of Higher Education ``Moscow Technological University,'' 
78~Vernadskogo Ave., Moscow 119454, Russian Federation;
\mbox{agorshenin@frccsc.ru}

\vspace*{3pt}

\noindent
\textbf{Belousov Vasiliy V.} (b.\ 1977)~---
Candidate of Science (PhD) in technology, senior scientist, Institute of 
Informatics Problems, Federal Research Center ``Computer Science and Control'' 
of the Russian Academy of Sciences, 44-2~Vavilov Str., Moscow 119333, Russian 
Federation; \mbox{VBelousov@ipiran.ru}
\label{end\stat}


\renewcommand{\bibname}{\protect\rm Литература}  %10
\def\stat{listopad}

\def\tit{ЖИЗНЕННЫЙ ЦИКЛ МЕТОДОЛОГИИ ПОСТРОЕНИЯ РЕФЛЕКСИВНО-АКТИВНЫХ 
СИСТЕМ ИСКУССТВЕННЫХ ГЕТЕРОГЕННЫХ ИНТЕЛЛЕКТУАЛЬНЫХ АГЕНТОВ$^*$}

\def\titkol{Жизненный цикл методологии построения РАСИГИА} %рефлексивно-активных систем искусственных гетерогенных интеллектуальных агентов}

\def\aut{С.\,В.~Листопад$^1$}

\def\autkol{С.\,В.~Листопад}

\titel{\tit}{\aut}{\autkol}{\titkol}

\index{Листопад С.\,В.}
\index{Listopad S.\,V.}


{\renewcommand{\thefootnote}{\fnsymbol{footnote}} \footnotetext[1]
{Исследование выполнено за счет гранта Российского научного фонда №\,23-21-00218, 
{\sf https://rscf.ru/project/23-21-00218/}.}}


\renewcommand{\thefootnote}{\arabic{footnote}}
\footnotetext[1]{Федеральный исследовательский центр <<Информатика и~управ\-ле\-ние>> Российской академии наук, 
\mbox{ser-list-post@yandex.ru}}

%\vspace*{-12pt}

  
  

  \Abst{Представлена темпоральная структура (жизненный цикл) методологии построения 
рефлексивно-активных систем искусственных гетерогенных интеллектуальных агентов (\mbox{РАСИГИА}). 
Такие системы предназначены для компьютерного моделирования процессов и~эффектов, 
возникающих при решении практических проблем коллективами специалистов под 
руководством лица, принимающего решения. Искусственные гетерогенные 
интеллектуальные агенты реф\-лек\-сив\-но-ак\-тив\-ных сис\-тем~--- активные субъекты, 
способные к~рассуждениям, коммуникации и~рефлексии как умению моделировать 
рассуждения других агентов системы и~себя самих. Моделирование рефлексивных процессов 
обеспечивает выработку агентами согласованного представления об объекте управ\-ле\-ния, 
цели коллективной работы и~нормах взаимодействия, позволяя системе в~ходе 
самоорганизации генерировать заново релевантный гибридный интеллектуальный метод 
решения очередной проб\-лемы.} 
  
  \KW{рефлексия; методология; рефлексивно-активная сис\-те\-ма искусственных 
гетерогенных интеллектуальных агентов; гибридная интеллектуальная многоагентная 
система; коллектив специалистов}

\DOI{10.14357/19922264240112}{GUAMVE}
  
%\vspace*{-6pt}


\vskip 10pt plus 9pt minus 6pt

\thispagestyle{headings}

\begin{multicols}{2}

\label{st\stat}

\section{Введение}

  Компьютерное моделирование процессов и~эффектов, возникающих при 
решении практических проблем коллективами специалистов, каждый из 
которых обладает собственным опытом, знаниями и~пониманием предметной 
области,~--- перспективное на\-прав\-ле\-ние научных разработок, которое 
Д.\,А.~Поспелов выделял как одну из десяти горячих точек в~исследованиях по 
искусственному\linebreak интеллекту~[1]. Для компьютерного моделирования 
рас\-суж\-де\-ний коллективов специалистов предлагается создание \mbox{РАСИГИА} 
в~рамках многоагентного подхода~[2] на основе модели 
\mbox{ги\-брид\-ных} интеллектуальных многоагентных сис\-тем~[3]. Агенты 
\mbox{РАСИГИА}~--- активные программные сущности, способные 
рас\-суж\-дать, взаимодействовать и~рефлексировать. Рефлексивное 
моделирование агентами друг друга обеспечивает выработку согласованного 
пред\-став\-ле\-ния об объекте управ\-ле\-ния, \mbox{цели} коллективной работы и~нормах 
взаимодействия, а~также эволюцию \mbox{РАСИГИА} в~ходе 
самоорганизации в~сильном смыс\-ле. В~на\-сто\-ящей работе рас\-смат\-ри\-ва\-ют\-ся 
вопросы создания методологии разработки сис\-тем такого класса, которая 
понимается как учение об организации продуктивной де\-я\-тель\-ности 
в~це\-лост\-ную сис\-те\-му с~чет\-ко определенными характеристиками, логической 
структурой и~процессом ее осуществления (темпоральной структурой)~[4]. 
Характеристики (особенности и~принципы) и~логическая структура (субъект, 
объект, предмет, методы, средства, результат) методологии разработки 
\mbox{РАСИГИА} рас\-смот\-ре\-ны в~[5]. Данная работа по\-свя\-ще\-на разработке 
жизненного цик\-ла (темпоральной структуры) предлагаемой методологии.

\begin{figure*} %fig1
\vspace*{1pt}
      \begin{center}
     \mbox{%
\epsfxsize=148.855mm 
\epsfbox{lis-1.eps}
}
\end{center}
%\vspace*{-9pt}

{\small Темпоральная структура методологии построения РАСИГИА: \textit{1}~--- этап методологии; \textit{2}~--- стадия методологии;
\textit{3}~--- граница фазы методологии; \textit{4}~--- отношение следования при нормальном завершении этапа;  
\textit{5}~--- возврат к~предыдущим этапам при выявлении допущенных на них недочетов}
\end{figure*}

\vspace*{-6pt}
  
\section{Темпоральная структура методологии}

\vspace*{-6pt}

  Укрупненно в~жизненном цикле методологии построения \mbox{РАСИГИА}, 
показанном на рисунке, могут быть выделены проектная, технологическая 
и~рефлексивная фазы, которые со\-сто\-ят из стадий и~этапов. Как видно, 
последовательное выполнение этапов методологии приводит к~же\-ла\-емо\-му 
результату лишь в~идеальном случае, когда проектировщик сразу получает всю 
необходимую достоверную информацию, имеет необходимый арсенал методов, 
не совершает ошибок ни на одном из этапов и,~по сути, заранее знает, какой 
должна быть разрабатываемая \mbox{РАСИГИА}. В~реальности на каждом из 
этапов могут обнаруживаться ранее допущенные недочеты, требующие 
возврата к~соответствующему этапу, их исправления и~повторного выполнения 
проделанной работы с~новыми исходными данными. В~определенном смыс\-ле 
такой подход представляет собой метод проб и~ошибок, и~чем слож\-нее 
проблема, для которой проектируется \mbox{РАСИГИА}, с~точ\-ки зрения 
конкретного коллектива разработчиков, тем больше будет возвратов в~ходе 
проектирования системы~[6]. Рас\-смот\-рим по\-дроб\-нее каждую из фаз 
методологии.



\section{Проектная фаза}

  Проектная фаза включает в~себя стадии концептуального описания проб\-ле\-мы и~моделирования, выполняемые сис\-тем\-ны\-ми аналитиками из коллектива 
разработчиков. В~рамках первой стадии фазы на доформальном, 
содержательном уровне рас\-смат\-ри\-ва\-ет\-ся проб\-ле\-ма как отрицательное 
отношение субъекта к~реальности~[6] и~проблемная ситуация как объективное 
стечение обстоятельств, обуслов\-ли\-ва\-ющее проб\-ле\-му. Данная стадия со\-сто\-ит из 
сле\-ду\-ющих этапов:
  \begin{itemize}
\item формулирование проб\-ле\-мы, ее предварительное описание в~ходе 
интервьюирования лица, при\-ни\-ма\-юще\-го решение, его советников и~активных 
групп на естественном языке с~использованием привычных для них 
определений и~формулировок~[7];
  \item определение проб\-ле\-ма\-ти\-ки, т.\,е.\ комплекса проб\-лем, связанных 
с~рас\-смат\-ри\-ва\-емой~[4], чтобы учесть создаваемые ее решением последствия 
для каж\-дой из них. Необходимо охватить весь круг участников проб\-лем\-ной 
ситуации (стейкхолдеров, заинтересованных лиц): непосредственных 
участников ситуации, пред\-ста\-ви\-те\-лей проб\-ле\-мо\-раз\-ре\-ша\-ющих 
и~проб\-ле\-мо\-со\-дер\-жа\-щих сис\-тем, же\-ла\-емых помощников или союзников, 
субъектов, связанных с~ситуацией юридически, лиц с~возможным негативным 
отношением к~решению проб\-ле\-мы~[6]. Для по\-стро\-ения проб\-ле\-ма\-ти\-ки может 
быть использована, например, технология Дж.~Уор\-фил\-да, подходы 
с~использованием метафор организации, взгляда на проблему стейкхолдером 
с~раз\-ных точек зрения, рас\-смот\-ре\-ния проб\-ле\-мы в~рамках различных парадигм 
(функциональной, объяснительной, освободительной, пост\-мо\-дер\-нист\-ской)~[4, 6]. 
Формируется древовидная или сетевая структура в~виде диаграммы связей, 
концептуальной кар\-ты или аналогичных инструментов;
  \item определение целей проектирования \mbox{РАСИГИА}, 
пред\-по\-ла\-га\-ющее проведение собеседований с~каж\-дым стейк\-хол\-де\-ром, 
выяснение их целей и~пожеланий, формирование и~структурирование 
множества целей в~виде дерева или сетевидной структуры и~его 
визуализация~[4, 6]. Выделяются следующие уровни целей: ожи\-да\-емые 
в~плановом периоде результаты; задачи, которые не будут решены 
в~рас\-смат\-ри\-ва\-емом периоде, но будет достигнут существенный прогресс на 
пути к~ним; не\-до\-сти\-жи\-мые идеалы, к~которым следует стремиться~[8];
  \item выбор критериев, т.\,е.\ до\-ступ\-ных для наблюдения и~измерения 
характеристик, опи\-сы\-ва\-ющих важ\-ные особенности объектов или процессов 
и~поз\-во\-ля\-ющих сравнивать \mbox{пред\-ла\-га\-емые} альтернативы, контролировать 
процесс решения~[6]. Со\-во\-куп\-ность критериев долж\-на быть релевантной 
количественной моделью выделенных ранее качественных целей. Отдельно 
выделяются ограничения, фик\-си\-ру\-ющие условия, которые не могут нарушаться 
при до\-сти\-же\-нии цели;
  \item оценка концептуального описания проб\-ле\-мы в~ходе специально 
спланированного эксперимента. Если существует коллектив специалистов, 
решающий на практике по\-став\-ле\-нную или схожие проб\-ле\-мы, он выступает 
образцом, прототипом создаваемой сис\-те\-мы агентов. В~этом случае 
выполняется наблюдение за работой такого коллектива в~рамках решения 
реальных или тренировочных проб\-лем и~оценка релевантности 
зафиксированной информации сведениям, полученным в~ходе предыду\-щих 
этапов. Если выявлено существенное рас\-хож\-де\-ние, выполняется возврат 
к~этапу, в~рамках которого были получены некорректные сведения. Сведения 
о~составе участников коллектива, вы\-де\-ля\-емых ими под\-проб\-ле\-мах, методах их 
решения используются на по\-сле\-ду\-ющих этапах проектирования 
\mbox{РАСИГИА} при по\-стро\-ении со\-от\-вет\-ст\-ву\-ющих моделей проб\-ле\-мы 
и~сис\-те\-мы <<как есть сейчас>>. Данные о~качестве принятых решений 
и~дли\-тель\-ности их выработки используются в~дальнейшем как показатель 
эффекта от разработки и~внед\-ре\-ния \mbox{РАСИГИА}. Если подобных 
коллективов нет или не\-воз\-мож\-но реализовать со\-от\-вет\-ст\-ву\-ющий эксперимент, 
данный этап отсутствует.
  \end{itemize}
  
  Стадия моделирования предполагает разработку формализованного описания 
проб\-ле\-мы, коллектива специалистов, ре\-ша\-юще\-го проб\-ле\-му на момент 
разработки \mbox{РАСИГИА}, если он существует, и~самой 
\mbox{РАСИГИА}. Модели строятся с~использованием визуального 
метаязыка~[9], что позволяет наглядно их изобразить, а~так\-же поз\-во\-ля\-ет 
с~использованием заранее заданных соответствий однозначно отоб\-ра\-зить 
графическое пред\-став\-ле\-ние моделей в~формальное символьное пред\-став\-ле\-ние, 
пригодное для компьютерной интерпретации. Данная стадия со\-сто\-ит из 
сле\-ду\-ющих этапов:
  \begin{itemize}
  \item моделирование проб\-ле\-мы, которое обеспечивает ее формальное 
пред\-став\-ле\-ние на макро- и~мик\-ро\-уров\-не. Мак\-ро\-уров\-не\-вая модель описывает 
проб\-ле\-му как <<чер\-ный ящик>>, отражая ее место в~ме\-та\-проб\-ле\-ме (проб\-ле\-ме 
более высокого уровня), свойства как целого и~связи с~другими проб\-ле\-ма\-ми 
ме\-та\-проб\-ле\-мы. Атрибуты проблемы на макроуровне~--- цели, критерии 
(включая ограничения), исходные данные и~идентификатор. Мик\-ро\-уров\-не\-вая 
модель раскрывает со\-став и~структуру проб\-ле\-мы, описывает ее под\-проб\-ле\-мы 
и~связи между ними. Для каждой под\-проб\-ле\-мы специфицируются цели, 
критерии, исходные данные и~идентификатор, выполняется поиск релевантных 
методов решения. Если такие методы найдены, дальнейшая декомпозиция 
под\-проб\-ле\-мы не требуется, иначе выполняется по\-стро\-ение ее мик\-ро\-уров\-не\-вой 
модели, т.\,е.\ модели более глубокого уров\-ня иерархии. Таким образом, 
формируется многоуровневая иерархическая структура по\-став\-лен\-ной 
проб\-лемы;
  \item моделирование коллектива, которое отражает ситуацию решения 
проб\-ле\-мы <<как есть сейчас>> со всеми ее преимуществами и~недостатками. 
Модель коллектива~--- основа, образец для проектирования \mbox{РАСИГИА} и~оценки эф\-фек\-тив\-ности 
альтернативных конфигураций \mbox{РАСИГИА}. 
При моделировании коллектива специалистов фиксируется его со\-став в~виде 
множества ролей участников, час\-ти проб\-ле\-мы, ре\-ша\-емые каж\-дым из 
участников с~определенной ролью, знания и~методы, ис\-поль\-зу\-емые 
участниками для решения своей части проб\-ле\-мы, а~так\-же порядок и~нормы 
взаимодействия участников коллектива; 
  \item моделирование \mbox{РАСИГИА}, фор\-ми\-ру\-ющее идеализированное 
пред\-став\-ле\-ние <<как должно стать>> о~коллективе интеллектуальных агентов, 
ре\-ша\-ющих по\-став\-лен\-ную проб\-ле\-му. В~ходе моделирования \mbox{РАСИГИА} 
должны быть специфицированы со\-став и~иерархия ролей агентов, множество 
агентов, ис\-поль\-зу\-емые протоколы взаимодействия, под\-дер\-жи\-ва\-емые языки 
передачи сообщений, базовая онтология как осно\-ва для интерпретации 
семантики пе\-ре\-да\-ва\-емых сообщений, модель окру\-жа\-ющей среды, содержащая 
в~том чис\-ле пул, из которого агенты могут привлекаться сис\-те\-мой по мере 
не\-об\-хо\-ди\-мости и~в~который попадают ис\-клю\-ча\-емые из нее агенты, множество 
моделей архитектур \mbox{РАСИГИА}, множество необходимых моделей 
мак\-ро\-уров\-не\-вых эффектов. В~множестве агентов должны присутствовать 
агенты, пред\-став\-ля\-ющие стейк\-хол\-де\-ров с~их целями, критериями достижения 
цели и~ограничениями. Если на предыду\-щем этапе была по\-стро\-ена модель 
коллектива, то одна из архитектур \mbox{РАСИГИА} долж\-на соответствовать 
данной модели. 
  \end{itemize}
  
\section{Технологическая фаза}

  Технологическая фаза включает в~себя разработку эскизного проекта 
\mbox{РАСИГИА}, ее технического проекта и~программной реализации. 
Стадия разработки эскизного проекта \mbox{РАСИГИА} обеспечивает 
пред\-став\-ле\-ние создаваемой сис\-те\-мы и~ее внеш\-ней среды в~виде 
взаимосвязанных мо\-ду\-лей-бло\-ков в~соответствии с~моделью 
\mbox{РАСИГИА}, по\-стро\-ен\-ной на стадии проектирования. Данная стадия 
со\-сто\-ит из сле\-ду\-ющих этапов:
  \begin{itemize}
  \item разработка функциональной структуры, в~ходе которой строится 
множество взаимосвязанных схем-диа\-грамм, определяющих под\-сис\-те\-мы 
РАСИГИА, распределение агентов по ним, функционал агентов, до\-пус\-ти\-мые 
языки передачи сообщений и~протоколы взаимодействия для каж\-дой пары или 
группы ролей агентов, технологические элементы сис\-те\-мы, потоки 
информации и~управ\-ле\-ния, а~также отношения, воз\-ни\-ка\-ющие между агентами 
в~процессе решения проб\-лем. Для каждой роли указывается множество 
релевантных ей уже существующих (разработанных ранее для других сис\-тем) 
агентов, если таковые имеются. В~случае отсутствия релевантных агентов они 
должны быть разработаны на сле\-ду\-ющих этапах. Кроме того, 
специфицируются функциональные мо\-ду\-ли-бло\-ки, от\-ве\-ча\-ющие за организацию 
макроуровневых эффектов в~\mbox{РАСИГИА};
  \item разработка структуры внешней среды по аналогии с~разработкой 
функциональной структуры \mbox{РАСИГИА} предполагает построение схем-диа\-грамм, описывающих виртуальную внеш\-нюю среду, ее под\-сис\-те\-мы, роли 
агентов и~способы взаимодействия \mbox{РАСИГИА} с~ними, т.\,е.\ языки 
передачи сообщений и~протоколы взаимодействия, отношения, потоки 
информации и~управ\-ле\-ния. Для каж\-дой роли указываются су\-щест\-ву\-ющие 
релевантные ей агенты, если они имеются;
  \item разработка архитектур агентов выполняется для тех ролей 
в~функциональной структуре и~структуре внеш\-ней среды, для которых не 
найдено релевантных реализованных агентов. Архитектура агента~--- схема, 
описывающая со\-став, структуру и~взаимосвязь функ\-ций-бло\-ков, 
ре\-а\-ли\-зу\-емых агентом, обеспечивающая выполнение им своего предназначения. 
Для каждой функ\-ции-бло\-ка указывается метод или алгоритм, с~по\-мощью 
которого она реализуется, в~случае если таковые отсутствуют, они долж\-ны 
быть разработаны в~рамках сле\-ду\-ющей стадии.
  \end{itemize}
  
  Стадия разработки технического проекта \mbox{РАСИГИА} обеспечивает 
создание недостающих блоков для ее агентов или технологических элементов. 
При этом может по\-тре\-бо\-вать\-ся разработка методов решения под\-проб\-лем, 
алгоритмов на основе метода, баз данных, онтологий и~др. Порядок их 
разработки не регламентируется на\-сто\-ящей методологией в~связи 
с~существенным разнообразием и~не\-воз\-мож\-ностью совместного рас\-смот\-ре\-ния. 
На данной стадии должен быть сформирован технический проект, 
опи\-сы\-ва\-ющий для каждого блока со\-став, структуру и~форму пред\-став\-ле\-ния 
входных и~выходных данных, алгоритм его функционирования, спецификацию 
необходимых технических средств~[10].
  
  Стадия программной реализации и~отладки предполагает разработку 
программного кода \mbox{РАСИГИА} и~его тестирование на предмет 
корректной работы с~\mbox{целью} формирования полноценного программного 
продукта, а~так\-же разработку программной документации. Данная стадия 
со\-сто\-ит из сле\-ду\-ющих этапов:
  \begin{itemize}
  \item программная реализация и~разработка документации выполняется 
с~использованием платформы JaCaMo~[11], объединяющей технологию Jason 
для программирования автономных агентов, Cartago для программирования 
элементов внеш\-ней среды и~Moise для программирования многоагентных 
организаций. Кроме того, применяется язык Java для программирования 
отдельных элементов сис\-те\-мы и~тонкой настройки механизмов 
платформы~[12];
  \item тестирование и~отладка обеспечивают выявление и~устранение 
основных дефектов в~сис\-те\-ме. Ввиду того что полное тестирование  
сколь\-ко-ни\-будь слож\-ной программы не\-воз\-мож\-но~[13], выполняется 
выборочное тестирование в~сле\-ду\-ющем порядке: отдельные функ\-ции и~блоки 
из состава аген\-тов и~технологических элементов, межмодульные связи, агенты 
и~технологические элементы в~целом, протоколы взаимодействия агентов, 
\mbox{РАСИГИА} в~целом. В~тес\-ти\-ро\-ва\-нии принимают участие 
представители всех ролей команды разработчиков, так как каж\-дый из них 
выполняет поиск ошибок разного рода~[14]. При этом выделяется отдельная 
роль тестировщика, опре\-де\-ля\-юще\-го стратегию тес\-ти\-ро\-ва\-ния,  
тест-тре\-бо\-ва\-ния и~тест-пла\-ны для каждой из фаз проекта; он выполняет 
тестирование сис\-те\-мы, собирает и~анализирует отчеты о~про\-хож\-де\-нии 
тестирования. 
\end{itemize}

\section{Рефлексивная фаза}

  Рефлексивная фаза предназначена для оценки показателей реализованной 
\mbox{РАСИГИА} и~процесса ее разработки, выявления ее недостатков и~при 
не\-об\-хо\-ди\-мости до\-ра\-бот\-ки как сис\-те\-мы, так и~методологии ее построения. 
Стадия оценки эф\-фек\-тив\-ности \mbox{РАСИГИА} предполагает сбор 
показателей работы сис\-те\-мы и~их сравнение с~целевыми значениями. Если 
выявляется их несоответствие, выполняется анализ причин отклонений, 
переход к~этапу методологии, вызвавшему их, и~повторное выполнение данного и~по\-сле\-ду\-ющих этапов с~учетом тре\-бу\-емых корректировок. Кроме того, на этой 
стадии продолжается отладка сис\-те\-мы. Данная стадия выполняется в~три этапа:
  \begin{enumerate}[(1)]
  \item оценка в~лабораторных условиях командой разработчиков, когда 
система работает в~виртуальной внешней среде, решая тестовые проб\-ле\-мы. На 
данной стадии оценка сис\-те\-мы выполняется вычислительными моделями 
стейкхолдеров, реализованными со\-от\-вет\-ст\-ву\-ющи\-ми агентами виртуальной 
внеш\-ней среды; 
  \item оценка по результатам тестовой эксплуатации, когда \mbox{РАСИГИА} 
функционирует в~реальной внеш\-ней среде параллельно с~традиционным 
методом решения проб\-ле\-мы и~выполняется сравнение их эф\-фек\-тив\-ности 
пользователями и~реальными стейк\-хол\-де\-ра\-ми. Первоначально 
у~\mbox{РАСИГИА} должна быть отключена воз\-мож\-ность оказывать ка\-кое-ли\-бо воздействие на реальную внеш\-нюю среду, а~результатом ее\linebreak
 работы 
долж\-ны стать рекомендации по оказанию таких воздействий. После 
удовлетворительной оценки пользователей и~стейк\-хол\-де\-ров \mbox{РАСИГИА} 
может быть переведена в~\mbox{автоматический} режим взаимодействия со средой, 
а~традиционный метод решения проб\-ле\-мы используется в~качестве резервного 
для проверки ее работы еще в~течение некоторого времени. Длительности 
каждого из этих периодов долж\-ны определяться заказчиками сис\-те\-мы для 
решения конкретной проб\-ле\-мы совместно с~коллективом разработчиков; 
  \item сопровождение после внед\-ре\-ния поз\-во\-ля\-ет собирать жалобы, замечания и~предложения в~процессе эксплуатации \mbox{РАСИГИА}, в~том чис\-ле от 
людей, которые ошибочно не были включены в~со\-став стейкхолдеров.\\[-13pt] 
  \end{enumerate}
  
  Стадия оценки и~корректировки методологии в~определенном смысле длится 
на протяжении всего проекта, так как для ее реализации долж\-ны вес\-тись 
протоколы де\-я\-тель\-ности разработчиков,\linebreak в~которых отмечается дли\-тель\-ность 
реализации каж\-до\-го этапа, выполненные возвраты и~их причины. Однако 
именно по завершении проекта выполняется рефлексия проделанной работы, 
когда разработчики долж\-ны проанализировать удачные и~провальные решения, 
причины рас\-хож\-де\-ния результатов с~планами, возвратов к~предыду\-щим этапам 
и~фазам разработки \mbox{РАСИГИА}, затягивания отдельных этапов 
разработки, из\-бы\-точ\-ность или, наоборот, не\-ин\-фор\-ма\-тив\-ность 
по\-стро\-ений~\cite{4-lis}. По результатам анализа в~методологию вносятся 
изменения в~статусе <<предложение>>, которые после под\-тверж\-де\-ния 
эф\-фек\-тив\-ности в~новых проектах закрепляются в~новой версии методологии.

\vspace*{-9pt}

\section{Заключение}

\vspace*{-3pt}

  В работе представлена темпоральная структура (жизненный цикл) 
разработки \mbox{РАСИГИА}, опи\-сы\-ва\-ющая процессы сис\-тем\-но\-го анализа 
проб\-ле-\linebreak мы, моделирования, эскизного и~технического \mbox{проектирования} сис\-те\-мы, 
ее программной реализации, отладки и~тестирования. 
Основной результат 
организации работ в~соответствии с~предложенной методологией~--- 
программная реализация \mbox{РАСИГИА}, релевантно моделирующая 
коллектив специалистов, со\-вмест\-но ре\-ша\-ющих по\-став\-лен\-ную проб\-ле\-му 
с~учетом ее слабой формализации, не\-од\-но\-род\-ности, сетевого характера условий 
и~целей, не\-опре\-де\-лен\-ности и~ди\-на\-мич\-ности~\cite{5-lis}. Кроме того, в~результате 
рефлексивной стадии методологии формируется ее новая версия или 
подтверждается эф\-фек\-тив\-ность су\-щест\-ву\-ющей, что представляется\linebreak 
дополнительным результатом работ. Таким образом, методология предполагает 
свое развитие, потенциально обеспечивающее ее ре\-ле\-вант\-ность \mbox{актуальным}
подходам к~проектированию и~реализации интеллектуальных информационных 
сис\-тем.

\vspace*{-9pt}
  
{\small\frenchspacing
 { %\baselineskip=10.6pt
 %\addcontentsline{toc}{section}{References}
 \begin{thebibliography}{99}
 
 \vspace*{-3pt}
 
  \bibitem{1-lis}
   \Au{Поспелов Д.\,А.} Десять <<горячих точек>> в~исследованиях по искусственному 
интеллекту~// Искусственный\linebreak\vspace*{-12pt}

\columnbreak

\noindent
 интеллект и~принятие решений, 2019. №\,4. С.~3--9. doi: 
10.14357/20718594190401. EDN: BAUHFV.
  
  \bibitem{2-lis}
\Au{Тарасов В.\,Б.} От многоагентных сис\-тем к~интеллектуальным организациям: 
философия, психология, информатика.~--- М.: Эдиториал УРСС, 2002. 348~с.
  \bibitem{3-lis}
  \Au{Колесников А.\,В., Кириков~И.\,А., Листопад~С.\,В.} Ги\-брид\-ные интеллектуальные 
сис\-те\-мы с~самоорганизацией: координация, со\-гла\-со\-ван\-ность, спор.~--- М.: ИПИ РАН, 2014. 
189~с.
  \bibitem{4-lis}
  \Au{Новиков А.\,М., Новиков~Д.\,А.} Методология.~--- М.: Синтег, 2007. 668~с.
  \bibitem{5-lis}
  \Au{Листопад С.\,В.} Характеристики и~логическая структура методологии по\-стро\-ения  
реф\-лек\-сив\-но-ак\-тив\-ных сис\-тем искусственных гетерогенных интеллектуальных 
агентов~// Сис\-те\-мы и~средства \mbox{информатики}, 2023. Т.~33. №\,4. С.~16--27. doi: 
10.14357/ 08696527230402. EDN: TRTHEI.
  \bibitem{6-lis}
  \Au{Тарасенко Ф.\,П.} Прикладной сис\-те\-мный анализ.~--- М.: 
КНОРУС, 2010. 224~с.
  \bibitem{7-lis}
  \Au{Ларичев О.\,И.} Вербальный анализ решений.~--- М.: Наука, 2006. 181~с.
  \bibitem{8-lis}
  \Au{Акофф Р.} Акофф о менеджменте~/ Пер.\ с~англ.~--- СПб.: Питер, 2002. 448~с.
  (\Au{Akoff~R.\,L.} Ackoff's best: His classic writings on management.~--- New 
York, NY, USA: Wiley, 1999. 368~p.)
  \bibitem{9-lis}
  \Au{Колесников А.\,В., Листопад~С.\,В., Румовская~С.\,Б., Данишевский~В.\,И.} 
Неформальная аксиоматическая тео\-рия ролевых визуальных моделей~// Информатика и~её 
применения, 2016. Т.~10. Вып.~4. С.~114--120.  doi: 10.14357/19922264160412. EDN: XGSIVN.
  \bibitem{10-lis}
  \Au{Черушева Т.\,В.} Проектирование программного обеспечения.~--- Пенза: ПГУ, 2014. 
172~с.
  \bibitem{11-lis}
  \Au{Boissier O., Bordini~R.\,H., Hubnerand~J., Ricci~A.} Multi-agent oriented programming: 
Programming multi-agent systems using JaCaMo.~--- Intelligent robotics and autonomous agents 
series.~--- Cambridge: The MIT Press, 2020. 264~p.
  \bibitem{12-lis}
  \Au{Смирнов С.\,С., Смольянинова~В.\,А.} Введение в~разработку многоагентных сис\-тем 
в~среде Jason. Основы программирования на языке AgentSpeak.~--- М.: \mbox{МИРЭА}, 2009. 136~с.
  \bibitem{13-lis}
  \Au{Канер~С., Фолк~Д., Нгуен~Е.\,К.} Тестирование про\-грам\-мно\-го обеспечения. 
Фундаментальные концепции менеджмента биз\-нес-при\-ло\-же\-ний~/
Пер. с~англ.~--- Киев: ДиаСофт, 
2001. 544~с. (\Au{Kaner~С., Falk~J., Nguyen~H.\,Q.} {Testing computer software}.~--- 
International Thomson Computer Press,  1999. 496~p.)
  \bibitem{14-lis}
  \Au{Романькова Т.\,Л.} Тестирование программного обеспечения. {\sf 
https://elib.gstu.by/bitstream/handle/220612/ 9860/416.pdf}.

\end{thebibliography}

 }
 }

\end{multicols}

\vspace*{-6pt}

\hfill{\small\textit{Поступила в~редакцию 25.11.23}}

%\vspace*{8pt}

%\pagebreak

\newpage

\vspace*{-28pt}

%\hrule

%\vspace*{2pt}

%\hrule



\def\tit{LIFE CYCLE OF METHODOLOGY FOR~CONSTRUCTING REFLEXIVE-ACTIVE SYSTEMS OF~ARTIFICIAL HETEROGENEOUS INTELLIGENT AGENTS}


\def\titkol{Life cycle of methodology for~constructing reflexive-active systems of~artificial heterogeneous intelligent agents}


\def\aut{S.\,V.~Listopad}

\def\autkol{S.\,V.~Listopad}

\titel{\tit}{\aut}{\autkol}{\titkol}

\vspace*{-8pt}


\noindent
Federal Research Center ``Computer Science and Control'' of the Russian Academy of 
Sciences, 44-2~Vavilov Str., Moscow 119333, Russian Federation

\def\leftfootline{\small{\textbf{\thepage}
\hfill INFORMATIKA I EE PRIMENENIYA~--- INFORMATICS AND
APPLICATIONS\ \ \ 2024\ \ \ volume~18\ \ \ issue\ 1}
}%
 \def\rightfootline{\small{INFORMATIKA I EE PRIMENENIYA~---
INFORMATICS AND APPLICATIONS\ \ \ 2024\ \ \ volume~18\ \ \ issue\ 1
\hfill \textbf{\thepage}}}

\vspace*{4pt}
  
  
   
   \Abste{The paper presents the temporal structure (life cycle) of the methodology for 
constructing reflexive-active systems of artificial heterogeneous intelligent agents. These systems 
are designed for computer modeling of processes and effects that arise when solving practical 
problems by teams of specialists under the guidance of a~decision maker. Artificial heterogeneous 
intelligent agents of reflexive-active systems are active subjects capable of reasoning, 
communication, and reflection as the ability to model the reasoning of other agents of the system 
and themselves. Modeling of reflexive processes ensures the development by agents of a~consistent 
understanding of the control object, the purpose of collective work, and the norms of interaction 
allowing the system to self-organize and re-develop a relevant hybrid intelligent method for solving 
the next problem.}
   
   \KWE{reflection; methodology; reflexive-active system of artificial heterogeneous intelligent 
agents; hybrid intelligent multiagent system; team of specialists}
   
 
   
\DOI{10.14357/19922264240112}{GUAMVE}

\vspace*{-8pt}

\Ack

\vspace*{-1pt}


     \noindent
     This work was supported by the Russian Science Foundation, project No.\,23-21-00218.


\vspace*{6pt}

  \begin{multicols}{2}

\renewcommand{\bibname}{\protect\rmfamily References}
%\renewcommand{\bibname}{\large\protect\rm References}

{\small\frenchspacing
 {\baselineskip=11.5pt
 \addcontentsline{toc}{section}{References}
 \begin{thebibliography}{99} 
  \bibitem{1-lis-1}
   \Aue{Pospelov, D.\,A.} 2019. Desyat' ``goryachikh tochek'' v~issledovaniyakh po 
iskusstvennomu intellektu [Ten hot topics in AI research]. \textit{Is\-kus\-stven\-nyy in\-tel\-lekt 
i~pri\-nya\-tie re\-she\-niy} [Artificial Intelligence and Decision Making] 4:3--9. doi: 
10.14357/20718594190401. EDN: BAUHFV.
  \bibitem{2-lis-1}
   \Aue{Tarasov, V.\,B.} 2002. \textit{Ot mnogoagentnykh sis\-tem k~in\-tel\-lek\-tu\-al'\-nym 
or\-ga\-ni\-za\-tsi\-yam: fi\-lo\-so\-fiya, psi\-kho\-lo\-giya, in\-for\-ma\-ti\-ka} [From multiagent systems to intelligent 
organizations: Philosophy, psychology, and computer science]. Moscow: Editorial URSS. 348~p.
  \bibitem{3-lis-1}
   \Aue{Kolesnikov, A.\,V., I.\,A.~Kirikov, and S.\,V.~Listopad.} 2014. \textit{Gib\-rid\-nye 
in\-tel\-lek\-tu\-al'\-nye sis\-te\-my s~sa\-mo\-or\-ga\-ni\-za\-tsiey: ko\-or\-di\-na\-tsiya, so\-gla\-so\-van\-nost', spor} [Hybrid 
intelligent systems with self-organization: Coordination, consistency, and dispute]. Moscow: IPI 
RAN. 189~p.
  \bibitem{4-lis-1}
   \Aue{Novikov, A.\,M., and D.\,A.~Novikov.} 2007. \textit{Me\-to\-do\-lo\-giya} [Methodology]. 
Moscow: SINTEG. 668~p.
  \bibitem{5-lis-1}
   \Aue{Listopad, S.\,V.} 2023. Kharakteristiki i~logicheskaya struk\-tu\-ra me\-to\-do\-lo\-gii po\-stro\-eniya 
refleksivno-aktivnykh sis\-tem is\-kus\-stven\-nykh ge\-te\-ro\-gen\-nykh in\-tel\-lek\-tu\-al'\-nykh agen\-tov 
[Characteristics and logical structure of the methodology for constructing reflexive-active systems 
of artificial heterogeneous intelligent agents]. \textit{Sistemy i~Sredstva Informatiki~--- Systems 
and Means of Informatics} 33(4):16--27. doi: 10.14357/08696527230402. EDN: TRTHEI.
  \bibitem{6-lis-1}
   \Aue{Tarasenko, F.\,P.} 2010. \textit{Pri\-klad\-noy sis\-tem\-nyy ana\-liz} 
[Applied systems analysis]. Moscow: KNORUS. 224~p.
  \bibitem{7-lis-1}
   \Aue{Larichev, O.\,I.} 2006. \textit{Ver\-bal'\-nyy ana\-liz re\-she\-niy} [Verbal analysis of decisions]. 
Moscow: Nauka. 181~p.
  \bibitem{8-lis-1}
   \Aue{Akoff, R.\,L.} 1999. \textit{Ackoff's best: His classic writings on management}. New 
York, NY: Wiley. 368~p.
  \bibitem{9-lis-1}
   \Aue{Kolesnikov, A.\,V., S.\,V.~Listopad, S.\,B.~Rumovskaya, and V.\,I.~Danishevskiy.} 
2016. Ne\-for\-mal'\-naya ak\-sio\-ma\-ti\-che\-skaya teo\-riya ro\-le\-vykh vi\-zu\-al'\-nykh mo\-de\-ley [Informal axiomatic 
theory of the role visual models]. \textit{Informatika i~ee Primeneniya~--- Inform. Appl.} 
10(4):114--120. doi: 10.14357/19922264160412. EDN: XGSIVN.
  \bibitem{10-lis-1}
   \Aue{Cherusheva, T.\,V.} 2014. \textit{Pro\-ek\-ti\-ro\-va\-nie pro\-gram\-mno\-go obes\-pe\-che\-niya} 
[Software design]. Penza: PGU. 172~p.
  \bibitem{11-lis-1}
   \Aue{Boissier, O., R.\,H.~Bordini, J.~Hubnerand, and A.~Ricci}. 2020. \textit{Multi-agent 
oriented programming: Programming multi-agent systems using JaCaMo}. Intelligent robotics and 
autonomous agents ser. Cambridge: The MIT Press. 264~p.
  \bibitem{12-lis-1}
   \Aue{Smirnov, S.\,S., and V.\,A.~Smol'yaninova}. 2009. \textit{Vve\-de\-nie v~raz\-ra\-bot\-ku 
mno\-go\-agent\-nykh sis\-tem v~sre\-de Jason. Osno\-vy pro\-gram\-mi\-ro\-va\-niya na yazy\-ke AgentSpeak} 
[Introduction to the development of multiagent systems in the Jason environment. Fundamentals of 
programming in the AgentSpeak language]. Moscow: MIREA. 136~p.
  \bibitem{13-lis-1}
   \Aue{Kaner, С., J.~Falk, and H.\,Q.~Nguyen}. 1999. \textit{Testing computer software}. 
International Thomson Computer Press. 496~p.
  \bibitem{14-lis-1}
   \Aue{Romankova, T.\,L.} 2014. Tes\-ti\-ro\-va\-nie pro\-gram\-mno\-go obes\-pe\-che\-niya [Software testing]. 
Available at: {\sf https://}\linebreak\vspace*{-12pt}

\columnbreak

\noindent
 {\sf elib.gstu.by/bitstream/handle/220612/9860/416.pdf} (accessed January~16, 
2024).
   
  \end{thebibliography}

 }
 }

\end{multicols}

\vspace*{-6pt}

\hfill{\small\textit{Received November 25, 2023}} 

%\vspace*{-18pt}
     
     \Contrl
     
 %    \vspace*{-3pt}
   
   \noindent
   \textbf{Listopad Sergey V.} (b.\ 1984)~--- Candidate of Science (PhD) in technology, senior 
scientist, Federal Research Center ``Computer Science and Control'' of the Russian Academy of 
Sciences, 44-2~Vavilov Str., Moscow 119133, Russian Federation;  
\mbox{ser-list-post@yandex.ru}
   
    
\label{end\stat}

\renewcommand{\bibname}{\protect\rm Литература}  %11
\def\stat{zatsman}

\def\tit{ТРАНСФОРМАЦИИ ОБЪЕКТОВ ПЕРВОГО И~ВТОРОГО ПОРЯДКА 
В~ЛЕКСИКОГРАФИЧЕСКОЙ ИНФОРМАЦИОННОЙ СИСТЕМЕ$^*$}

\def\titkol{Трансформации объектов первого и~второго порядка 
в~лексикографической информационной системе}

\def\aut{И.\,М.~Зацман$^1$}

\def\autkol{И.\,М.~Зацман}

\titel{\tit}{\aut}{\autkol}{\titkol}

\index{Зацман И.\,М.}
\index{Zatsman I.\,M.}


{\renewcommand{\thefootnote}{\fnsymbol{footnote}} \footnotetext[1]
{Исследование выполнено в~ФИЦ ИУ РАН за счет гранта Российского научного фонда №\,24-18-00155, {\sf 
https://rscf.ru/project/24-18-00155}. Работа выполнялась с~использованием инфраструктуры Центра 
коллективного пользования <<Высокопроизводительные вычисления и~большие данные>> (ЦКП 
<<Информатика>>) ФИЦ ИУ РАН (г.\ Москва).}}


\renewcommand{\thefootnote}{\arabic{footnote}}
\footnotetext[1]{ Федеральный исследовательский центр <<Информатика и~управление>> Российской академии наук, 
\mbox{izatsman@yandex.ru}}

\vspace*{-12pt}


  
  \Abst{Рассматриваются теоретические основания проектирования информационных 
технологий (ИТ) интеграции двуязычных словарей и~параллельных корпусов. Дано описание 
первых результатов создания третьего уровня классификации трансформаций объектов 
предметной области информатики, которую предполагается использовать при создании 
концепции лексикографической информационной системы, обеспечивающей интеграцию. 
Все сущности информатики в~статье разделены на два глобальных класса: объекты и~их 
трансформации. Для каждого такого класса конструируется своя классификация. Ранее были 
описаны два верхних уровня классификации трансформаций объектов предметной области. 
В~данной статье рассматривается третий уровень этой классификации. Основанием для 
построения самого верхнего ее уровня служило деление предметной области информатики 
на среды (ментальная, сенсорно воспринимаемая, цифровая и~ряд других сред), каждая из 
которых по определению включает объекты одной природы. Основанием для построения 
второго уровня классификации трансформаций объектов служила типология знаковых  
сис\-тем А.~Соломоника. Цель статьи состоит в~систематизации трансформаций первого 
и~второго порядка объектов предметной области на третьем уровне этой классификации. 
Основанием для систематизации служит средовая версия иерархии Акоффа.}
  
  \KW{объекты предметной области; трансформации объектов; классификация; данные; 
информация; знание; лексикографическая информационная сис\-тема}

\DOI{10.14357/19922264240211}{VZTGVV}
  
\vspace*{3pt}


\vskip 10pt plus 9pt minus 6pt

\thispagestyle{headings}

\begin{multicols}{2}

\label{st\stat}
  
\section{Введение}

\vspace*{-9pt}

  Возникновение параллельных корпусов, в~которых предложениям 
оригинального текста со\-по\-став\-ле\-ны предложения его перевода, обеспечило 
возможность контрастивного лингвистического\linebreak \mbox{анализа} на принципиально 
новом уровне полноты и~точности, недостижимом в~докорпусную эпоху. 
Пионерскими в~этой области стали работы \mbox{1990-х~гг}. Стига Йоханссона  
с~анг\-ло-нор\-веж\-ским корпусом~[1]. В России параллельные корпусы стали 
формироваться в~начале XXI~века в~рамках Национального корпуса русского 
языка~[2].
  
  Создатели двуязычных словарей используют параллельные корпусы для 
сбора материала и~эмпирической проверки своих гипотез, касающихся 
межъязы\-ко\-вой эквивалентности. Ценность параллельных корпусов 
определяется тем, что в~лингвистике этап сбора исходного материала считается 
наиболее трудоемким и~наименее творческим, а~параллельные корпусы 
позволяют значительно сэкономить время и~силы для творческого этапа 
создания словарей~[3].
 % 
  При этом двуязычные словари, создаваемые на основе исходного материала, 
извлеченного из параллельных корпусов, сейчас формируются без связей с~их 
текстами. Другими словами, онлайновые связи созданных словарей 
с~параллельными корпусами, которые служили источниками исходного 
материала, отсутствуют. 

Параллельные корпусы постоянно пополняются 
новыми текстами, в~предложениях которых можно обнаружить новые значения 
слов и~устойчивых словосочетаний. Однако при этом отсутствуют методы 
и~средства оперативного обновления словарей по корпусным данным. 
В~настоящее время проблема установления связей между двуязычными 
словарями и~параллельными корпусами (далее~--- проблема интеграции) 
находится на стадии поиска концептуальных подходов к~их интеграции на 
уровне значений.
  
  Подход к~решению проблемы интеграции, предлагаемый в~статье, учитывает 
  и~появление новых значений слов и~устойчивых словосочетаний, и~динамику 
смысловых значений, которая обусловлена развитием и~пополнением знания 
лингвистов, фиксирующих эти значения в~результате семантического анализа 
пополняемых корпусных данных. Проведенные эксперименты показали, что 
обнаружение нового лингвистического знания обусловливает и~формирование 
дефиниций новых значений, и~пересмотр уже существующих дефиниций~[4, 5].
  
  Например, в~проведенных экспериментах с~использованием ЦКП 
<<Информатика>> ФИЦ ИУ РАН фиксировалась эволюция значений немецких 
модальных глаголов, исходное состояние значений которых было описано 
в~не\-мец\-ко-рус\-ском словаре. В~экспериментальном массиве текстов как 
потенциальных источниках нового знания 16\,268 предложений содержали 
немецкие модальные глаголы и~в~2041 из них встречался глагол sollen. 
В~начале эксперимента в~словаре были описаны~12~значений этого модального 
глагола. По окончании эксперимента лингвисты обнаружили два новых его 
значения, согласовали их дефиниции и~описали эволюцию дефиниций~[6, 7].
  
  Таким образом, для решения проблемы интеграции требуется фиксировать 
новое знание, обнаруженное лингвистами в~текстовых данных параллельных 
корпусов, отслеживать эволюцию знания, представленного в~виде дефиниций 
значений слов и~устойчивых словосочетаний, и,~соответственно, 
актуализировать электронные двуязычные словари. Предлагаемый 
концептуальный подход к~интеграции, который планируется реализовать 
в~процессе проектирования лексикографической информационной сис\-те\-мы, 
фиксирующей эволюцию лингвистического знания, основан на решении 
следующих задач:\\[-14pt]
  \begin{itemize}
  \item категоризация трех базовых понятий информатики, включенных 
  в~иерархию Акоффа~[8] (данные, информация, знание), на объекты 
проектируемой сис\-те\-мы, которая необходима, чтобы фиксировать 
<<кванты>> нового знания и~отслеживать его эволюцию в~этой сис\-теме;\\[-15pt]
  \item  систематизация трансформаций объектов этой сис\-темы.\\[-14pt]
  \end{itemize}
  
  Цель статьи и~состоит в~решении двух задач: категоризации трех базовых 
понятий информатики на объекты лексикографической информационной  
сис\-те\-мы и~сис\-те\-ма\-ти\-за\-ции трансформаций первого и~второго порядка 
ее объектов.
  
  Трансформациями первого порядка, о которых сказано в~формулировке цели 
статьи, называются взаимные преобразования между двумя объектами  
сис\-те\-мы одной природы. Например, перевод в~сис\-те\-ме текста с~русского 
языка на английский относится к~ним. Трансформациями второго порядка 
и~выше называются взаимные преобразования между двумя и~более объектами 
разной природы. Например, кодирование символов текс\-та компьютерными 
кодами и~их декодирование относятся по определению к~трансформациям 
второго порядка.

%\vspace*{-9pt}
  
\section{Процессы трансформаций в~информатике}

%\vspace*{-3pt}

Процессы трансформаций, рассматриваемые в~статье, относятся к~теоретическому ядру информатики, а~не 
только к~проектированию лексикографической информационной сис\-те\-мы. Например, из трех основных 
подходов к~описанию предметной об\-ласти информатики\footnote{В статье предметная область информатики 
трактуется согласно концепции полиадического компьютинга Пола Розенблума~\cite{9-zac}.} (объектный, 
трансформационный и~синтетический) сис\-те\-ма\-ти\-за\-ция трансформаций ближе всего ко второму 
подходу. Примерами первого подхода, в~рамках которого основное внимание уделяется объектам предметной 
области информатики и~в~меньшей степени отношениям\linebreak между ними, могут служить  
работы~\cite{8-zac, 10-zac, 11-zac}; \mbox{примерами} второго подхода, в~рамках которого основное внимание 
уделяется трансформациям и~в~меньшей степени трансформируемым объектам,~---  
работы~\cite{12-zac, 13-zac}; примерами третьего, синтетического подхода, в~котором уделяется внимание 
и~объектам предметной об\-ласти информатики, и~отношениям между ними, могут служить работы~\cite{14-zac, 
15-zac, 16-zac, 17-zac, 18-zac}.

  Таким образом, для описания трансформаций объектов лексикографической 
информационной\linebreak системы предпочтительнее всего трансформационный 
подход, который упоминается и~в определениях информатики. Например, 
в~2009~г.\ П.~Деннинг и~П.~Розенблум сформулировали суть \mbox{информатики} как 
компьютинга следующим образом: <<$\ldots$информатика~--- это не просто 
алгоритмы и~структуры данных; это преобразования [трансформации] 
представлений>>~\cite{12-zac}. Чуть позже, в~контексте краткого описания 
парадигмы информатики как компьютинга, П.~Деннинг и~П.~Фриман изменили 
эту формулировку на такую: <<Центральный объект внимания в~информатике 
можно определить как информационные процессы~--- \textit{естественные или 
искусственные процессы, преобразующие информацию} (курсив мой~--- 
И.\,З.)>>~\cite{13-zac}. Согласно парадигме, предлагаемой авторами этой 
статьи, на начальном этапе проектирования автоматизированных систем 
базовыми элементами моделей их функционирования служат 
\textit{информационные про\-цессы}.
  
  Однако если 15~лет назад в~формулировке из работы~\cite{13-zac} шла речь 
о~процессах, преобразующих информацию, то в~последние~10~лет в~спектр 
процессов трансформаций все чаще стали включать процессы, преобразующие 
не только информацию, но также и~другие объекты автоматизированных 
систем, в~первую очередь данные и~знания~[19--21]. Например, Виктория 
Стодден, позиционируя науку о~данных как одну из дисциплин информатики, 
говорит, что центральный объект исследований в~науке о~данных~--- это 
<<изучение обобщаемого извлечения знания из данных>>~\cite{21-zac}. 
Увеличение и~чис\-ла объектов, и~спект\-ра процессов их трансформаций 
в~автоматизированных сис\-те\-мах обуслов\-ли\-ва\-ет не\-об\-хо\-ди\-мость 
систематизации и~объектов, и~процессов их трансформаций на начальном этапе 
проектирования сис\-тем.
  
  Для создания концепции лексикографической информационной сис\-те\-мы 
и~проектирования ИТ, обеспечивающих интеграцию 
двуязычных словарей и~параллельных корпусов, сначала выполним 
категоризацию на объекты этой сис\-те\-мы трех базовых понятий информатики 
(данные, информация, знание) в~контексте построения классификаций 
сущностей ее предметной об\-ласти.
  
  Необходимость использования классификаций информатики в~процессе 
создания концепции проиллюстрируем, используя иерархию  
Акоффа~\cite{8-zac}. Он использовал принцип их вертикального размещения 
в~иерархии снизу вверх: данные, информация и~знание. Еще в~ней есть термин 
<<мудрость>>, который в~статье не рассматривается. Такое размещение Акофф 
прокомментировал так: <<Каждое из пе\-ре\-чис\-лен\-ных понятий [кроме данных] 
содержит в~себе нижестоящие$\ldots$>>~\cite{8-zac}.
  
  Этому принципу размещения и~комментарию Акоффа свойственны 
недостатки, проанализированные, в~частности, в~работе~\cite{10-zac}. Главный 
вывод, к~которому пришла Роули после изучения иерархии Акоффа, 
заключается в~следующем: <<$\ldots$информация определяется в~терминах 
данных, знание~--- в~терминах информации$\ldots$ но существует меньше 
консенсуса в~описании трансформаций, которые преобразуют сущности, 
расположенные ниже в~иерархии, в~те, которые находятся над ними, что 
приводит к~их терминологической неопределенности>>~\cite{10-zac}. Причина 
этой неопределенности, скорее всего, в~том, что базовые понятия информатики 
включены в~иерархию Акоффа изолированно от общего контекста 
классификаций сущностей ее предметной об\-ласти.

%\vspace*{-9pt}
  
\section{Классификации сущностей информатики}


%\vspace*{-2pt}

  Все сущности предметной области информатики в~работах~[22, 23] 
разделены на два глобальных класса: ее объекты и~их трансформации. Для 
каждого такого класса была предложена своя классификация. 
В~работе~\cite{22-zac} дано описание классификации объектов предметной 
области информатики, первый уровень которой содержит базовые понятия ее 
предметной области (данные, информация, знания и~др.).  
В~работе~\cite{23-zac} дано описание двух верхних уровней классификации 
трансформаций объектов предметной об\-ласти (см.\ рисунок 
в~работе~\cite{23-zac}). Основанием для построения самого верхнего ее уровня послужило деление 
предметной области информатики на среды\footnote{В~работе~\cite{24-zac} дано описание пяти сред 
предметной области информатики (ментальная; сенсорно воспринимаемая, или информационная; 
цифровая; нейро- и~ДНК-среда), каждая из которых по определению включает объекты одной и~той же 
природы.} и~степень разнообразия природы объектов, вовлеченных в~трансформации:
\begin{itemize}
\item  первый класс верхнего уровня классификации включает 
трансформации объектов в~пределах среды только одной природы 
(трансформации первого порядка);
\item  второй класс включает трансформации объектов, относящихся 
к~двум средам разной природы (трансформации второго порядка);
\item третий и~последующие классы включают трансформации объектов, 
относящихся к~трем и~более средам разной природы (трансформации 
третьего и~более высоких порядков).
\end{itemize}

  В работе~\cite{23-zac} были приведены примеры для трех первых классов 
трансформаций, включая пример трансформаций объектов, относящихся 
к~двум средам разной природы (компьютерное кодирование символов текстов 
с~по\-мощью таб\-лиц Unicode).
  
Основанием для построения второго уровня классификации трансформаций объектов послужила типология 
знаковых сис\-тем А.~Соломоника~\cite[c.~131]{25-zac}: естественные знаковые сис\-те\-мы, образные,  
ес\-тест\-вен\-но-язы\-ко\-в$\acute{\mbox{ы}}$е,  
вер\-баль\-но-не\-сло\-вес\-ные сис\-те\-мы записи\footnote{Под системой записи понимается знаковая 
система, сочетающая вербальные знаки с~несловесными (языки нотной записи, карт, таблиц и~др.).} 
и~формализованные знаковые сис\-те\-мы, включая математические. Введем понятие обобщенного текста~--- 
это текст, который может быть создан в~любой из перечисленных знаковых систем. Тогда обобщенные тексты 
могут быть естественными, образными, ес\-тест\-вен\-но-язы\-ко\-в$\acute{\mbox{ы}}$\-ми,  
вер\-баль\-но-не\-сло\-вес\-ны\-ми и~формализованными. Второй уровень классификации трансформаций 
охватывает не все виды объектов предметной  
об\-ласти информатики, а~только перечисленные~5~видов текс\-тов и~их представления, вовлеченные 
в~процессы трансформаций в~одной или более средах вместе с~данными, знанием и~его концептами.

\begin{figure*}[b] %fig1
\vspace*{6pt}
      \begin{center}
     \mbox{%
\epsfxsize=121.191mm 
\epsfbox{zac-1.eps}
}
\end{center}
\vspace*{-6pt}
\Caption{Средовая версия иерархии Акоффа}
\end{figure*}

\section{Классификация трансформаций: построение~третьего 
уровня}

  Основанием для систематизации трансформаций первого и~второго порядка 
на третьем уровне этой классификации служит иерархия Акоффа~\cite{8-zac}, 
на основе которой и~была создана ее средов$\acute{\mbox{а}}$я версия~[26, 
27]. Для создания средов$\acute{\mbox{о}}$й версии была выполнена 
категоризация трех базовых понятий информатики (данные, информация, 
знания) на объекты лексикографической информационной сис\-те\-мы 
в~процессе создания ее концепции\linebreak (рис.~1).
  


  В отличие от классической иерархии Акоффа, в~ее 
средов$\acute{\mbox{о}}$й версии различаются три вида данных: сенсорно 
воспринимаемые, цифровые и~те данные, которые генерируются 
искусственными нейронными сетями (ИНС) в~системах искусственного интеллекта 
(далее~--- ИИ-дан\-ные). Последний вид данных необходим, например, для 
различения входа и~выхода процесса применения обученной 
ИНС в~цифровой модели генерации знания, описанию которой 
посвящена работа~\cite{27-zac}.
  
  Также предлагается различать два вида информации: сенсорно 
воспринимаемая и~цифровая. Кроме знания в~средов$\acute{\mbox{у}}$ю 
версию добавлены концепты и~ментальные образы сенсорно воспринимаемых 
данных. Последние служат промежуточной сущностью между сенсорно 
воспринимаемыми данными и~генерируемым знанием при описании процессов 
извлечения знания из текстовых данных лексикографической информационной 
системы. Описание объектов средов$\acute{\mbox{о}}$й версии иерархии 
Акоффа (см.\ рис.~1) и~отношений между ними дано в~работах~\cite{26-zac, 28-zac}.
  
  В средов$\acute{\mbox{о}}$й версии число объектов равно восьми. Если 
учитывать направления трансформаций, то между восемью объектами на 
рис.~1 она включает~16 их видов (трансформации на границе между сенсорно 
воспринимаемыми данными и~информацией, обозначенные символом~<<?>>, 
в~статье не рас\-смат\-ри\-ва\-ют\-ся). В~будущем число объектов 
в~средов$\acute{\mbox{о}}$й версии, которая выбрана как основание для 
сис\-те\-ма\-ти\-за\-ции трансформаций первого и~второго порядка, может быть 
увеличено. Для построения классификации трансформаций 
важ\-но не возможное увеличение числа объектов 
и~трансформаций между ними, а то, что их виды в~средов$\acute{\mbox{о}}$й 
версии распределены между трансформациями первого и~второго порядка. Из 
16~видов на рис.~1 шесть относятся к~трансформациям первого порядка, это\linebreak 
виды с~номерами~7, 8, 13--16 (далее~--- типология трансформаций первого 
порядка), а~десять~--- к~трансформациям второго порядка, это виды 
с~\mbox{номерами}~1--6 и~9--12 (далее~--- типология трансформаций второго 
порядка). Разместим обе типологии на третьем уровне классификации (см.\ ее 
схему на рис.~2). Перечислим виды трансформаций первой типологии, вводя 
в~скобках их краткие названия, используемые ниже на рис.~3:
  \begin{description}
  \item[\,] 7~--- членение знания на концепты с~помощью одной или нескольких 
знаковых систем (далее~--- членение знания);
  \item[\,] 8~--- формирование знания на основе концептов (формирование 
знания);
  \item[\,] 13~--- обучение ИНС;
  \end{description}
  
  \vspace*{-6pt}
  
  \pagebreak
  
  \end{multicols}
  
  \begin{figure*} %fig2
\vspace*{1pt}
      \begin{center}
     \mbox{%
\epsfxsize=127.513mm 
\epsfbox{zac-2.eps}
}
\end{center}
\vspace*{-9pt}
\Caption{Схема трех верхних уровней классификации трансформаций объектов (объединены 
по три слоя и~для второго, и~для третьего уровней этой классификации)}
\end{figure*}
  
  \begin{multicols}{2}
  
  \noindent
  \begin{description}
  \item[\,] 14~--- восстановление обучающей информации на основе 
содержания обученной ИНС (обращение ИНС);
  \item[\,] 15~--- использование обученной ИНС (использование ИНС);



  \item[\,] 16~--- восстановление исходных данных, соответствующих 
полученным результатам работы обучен\-ной ИНС (восстановление исходных данных 
по результатам ИНС).
  \end{description}
  
  
  Не все виды трансформаций 13--16 поддерживаются в~конкретных системах 
искусственного интеллекта, но с~теоретической точки зрения все их 
предлагается включить в~первую типологию для полноты спектра видов 
трансформаций.
  
  Перечислим виды трансформаций второй типологии:
  \begin{description}
  \item[\,] 1~--- декодирование цифровых данных в~компьютерных системах 
(декодирование данных);
  \item[\,]  2~--- кодирование сенсорно воспринимаемых данных (кодирование 
данных);
  \item[\,] 3~--- ментальное копирование сенсорно воспринимаемых данных 
(ментальное копирование);
  \item[\,] 4~--- восстановление сенсорно воспринимаемых данных по 
ментальным образам (восстановление по образам);
  \item[\,] 5~--- смысловая интерпретация без деления на концепты ментальных 
образов сенсорно воспринимаемых данных (смысловая интерпретация);
  \item[\,] 6~--- восстановление ментальных образов (восстановление образов);
  \item[\,] 9~--- представление концептов в~виде сенсорно воспринимаемой 
информации, например текс\-та\-ми, формулами, таблицами, рисунками и~т.\,д.\ 
(представление концептов);
  \item[\,] 10~--- понимание смысла сенсорно воспринимаемой информации 
(понимание смысла);
  \item[\,] 11~--- кодирование сенсорно воспринимаемой информации 
(кодирование информации);
\end{description}

\vspace*{-6pt}

\pagebreak

\end{multicols}

\begin{figure*} %fig3
\vspace*{1pt}
      \begin{center}
     \mbox{%
\epsfxsize=163mm 
\epsfbox{zac-3.eps}
}
\end{center}
\vspace*{-9pt}
\Caption{Схема частного случая классификации трансформаций объектов (трансформации 
пронумерованы согласно рис.~1)}
\end{figure*}

\begin{multicols}{2}

\noindent
\begin{description}

  \item[\,] 12~--- декодирование цифровой информации (декодирование 
информации).
  \end{description}
  
  Отметим, что в~существующих ИТ
  и~компьютерных системах наиболее часто используются виды 
трансформаций~13 и~15 типологии первого порядка и~1, 2, 11 и~12 типологии 
второго порядка. На рис.~2 в~первом слое третьего уровня классификации 
показаны типологии первого порядка без указания числа трансформаций в~них 
и~без детализации трансформируемых объектов.
  
  Во втором слое третьего уровня классификации условно (без названий) 
показаны типологии второго порядка. Также на рис.~2 в~третьем слое третьего 
уровня классификации условно (также без названий) показаны типологии 
третьего порядка, которые планируется рассмотреть в~отдельной статье. По 
определению они должны включать трансформации между тремя объектами 
разной природы, но средов$\acute{\mbox{а}}$я версия иерархии Акоффа 
включает трансформации только между двумя объектами разной природы. 
Поэтому потребуется другое основание для их систематизации (ранее были 
рассмотрены отдельные примеры трансформаций третьего 
порядка\footnote{Далеко не всегда трансформации третьего и~более высоких порядков можно 
рассматривать как последовательность трансформаций второго порядка. Примером этого могут 
служить трансформации в~процессе обучения пациента пользованию роботизированной рукой, 
охватывающие личностные концепты пациента, релевантные его намерениям, сигналы активности 
мозга как объекты нейросреды и~компьютерные коды~\cite{29-zac}.}~\cite{29-zac}).

\section{Классификация трансформаций: частный~случай}

  Выше было отмечено, что в~будущем число объектов 
в~средов$\acute{\mbox{о}}$й версии иерархии Акоффа может быть увеличено. 
Это означает, что увеличатся и~чис\-ло объектов, и~чис\-ло трансформаций между 
ними в~классификации трансформаций, так как эта средов$\acute{\mbox{а}}$я 
версия служит по определению основанием для систематизации 
трансформаций первого и~второго порядка. Поэтому на третьем уровне рис.~2 
указаны типологии без детализации объектов и~без указания числа 
трансформаций в~каждой из них. С~одной стороны, при таком подходе 
получаем достаточно общий вид этой классификации, так как она не зависит от 
числа объектов в~том или ином варианте средов$\acute{\mbox{о}}$й версии 
(и~это существенно упрощает рис.~2). С~другой стороны, на третьем уровне 
такой общей классификации подразумевается, но не эксплицируется природа 
трансформируемых объектов и~их возможные сочетания в~трансформациях. 

При проектировании лексикографической информационной системы важно 
эксплицировать природу трансформируемых объектов и~их возможные 
сочетания.
  %
  Поэтому в~парадигму информатики~\cite{30-zac} кроме общей 
классификации трансформаций предлагается включать и~ее частные случаи, 
эксплицирующие природу трансформируемых объектов. 

В~этом разделе 
рассмотрим один частный случай, когда используются только естественные 
знаковые сис\-те\-мы из типологии А.~Соломоника~\cite{25-zac} вместе 
с~данными, знанием и~его концептами. Чис\-ло естественных языков при этом не 
ограничено. И~этот частный случай классификации включает только три 
класса природных трансформаций (первого, второго и~третьего порядка, см.\ 
схему классификации на рис.~3).
  
  Первый и~второй уровни схемы общей классификации (см.\ рис.~2) можно 
объединить в~один уровень в~этом частном случае. Ниже этого уровня 
приведено содержание типологий первого и~второго порядка без содержания 
типологий третьего по\-рядка.




  Наполнение типологий первого и~второго порядка соответствует 
средов$\acute{\mbox{о}}$й версии иерархии Акоффа на рис.~1, содержащей 
6~видов трансформаций типологии первого порядка и~10~видов 
трансформаций типологии второго порядка (на рис.~3 стрелки указывают 
направления трансформаций согласно средов$\acute{\mbox{о}}$й версии на рис.~1).
  
  Таким образом, частный случай классификации содержит для этих двух 
типологий 16~теоретически возможных трансформаций, 6 из которых 
в~настоящее время в~существующих ИТ применяются наиболее часто: виды 
трансформаций~1, 2, 11 и~12 типологии второго порядка реализуются 
с~помощью тех или иных методов ко\-ди\-ро\-ва\-ния/де\-ко\-ди\-ро\-ва\-ния 
(например, с~использованием таблиц Unicode), а~виды трансформаций~13 и~15
 в~типологии первого порядка реализуются полностью с~по\-мощью процессов 
цифровой обработки компьютерами.
  
  Остальные виды трансформаций или применяются намного реже (это 
виды~3, 5, 7, 9 и~10), или находятся в~стадии поиска и~разработки (14 и~16) или 
в~настоящее время носят только теоретический характер, обеспечивая полноту 
первой и~второй типологий (4, 6 и~8). Знаком~<<?>> обозначены те виды 
трансформаций, которые по определению не существуют в~используемой 
парадигме информатики~\cite{30-zac}. Однако возможно, что в~других 
будущих подходах к~построению ее парадигмы эти виды трансформаций будут 
существовать.
  
\section{Заключение}

  На сегодняшний день процесс построения классификаций объектов 
предметной области информатики~\cite{22-zac} и~их  
трансформаций~\cite{23-zac} еще не завершен. Однако первые результаты их 
построения уже используются для создания концепции лексикографической 
информационной сис\-те\-мы, обеспечивающей интеграцию двуязычных 
словарей и~параллельных корпусов.
  
  \bigskip
  
  
  Автор признателен рецензентам за помощь в~улучшении статьи.
  
{\small\frenchspacing
 { %\baselineskip=10.6pt
 %\addcontentsline{toc}{section}{References}
 \begin{thebibliography}{99}
\bibitem{1-zac}
\Au{Aijmer K., Altenberg~B.} Advances in corpus-based contrastive linguistics. Studies in honour 
of Stig Johansson.~--- Amsterdam: John Benjamins, 2013. 295~p.  doi: 10.1075/scl.54.
\bibitem{2-zac}
\Au{Добровольский Д.\,О., Кретов~А.\, А., Шаров~С.\,А.} Корпус параллельных текстов~// 
Научная и~техническая информация. Сер.~2: Информационные процессы и~сис\-те\-мы, 2005. 
№\,6. С.~16--27.
\bibitem{3-zac}
\Au{Добровольский Д.\,О.} Корпус параллельных текстов и~сопоставительная 
лексикология~// Труды Института русского языка им.\ В.\,В.~Виноградова, 2015. №\,6. 
С.~413--449. EDN: VJQBHP.
\bibitem{4-zac}
\Au{Гончаров А.\,А., Зацман~И.\,М., Кружков~М.\,Г.} Эволюция классификаций 
в~надкорпусных базах данных~// Информатика и~её применения, 2020. Т.~14. Вып.~4. 
С.~108--116. doi: 10.14357/19922264200415.  
EDN: \mbox{GKWBZT}.
\bibitem{5-zac}
\Au{Гончаров А.\, А., Зацман И. \,М., Кружков~М.\, Г}. Представление новых 
лексикографических знаний в~динамических классификационных сис\-те\-мах~// 
Информатика и~её применения, 2021. Т.~15. Вып.~1. С.~86--93.  doi: 10.14357/19922264210112. EDN: OPEFXW.
\bibitem{6-zac}
\Au{Zatsman I.} Finding and filling lacunas in linguistic typologies~// 15th Forum (International) 
on Knowledge Asset Dynamics Proceedings.~--- Matera, Italy: Institute of Knowledge Asset 
Management, 2020. P.~780--793.
\bibitem{7-zac}
\Au{Zatsman I.} Three-dimensional encoding of emerging meanings in AI-systems~// 21st 
European Conference on Knowledge Management Proceedings.~--- Reading, U.K.: Academic 
Publishing International Ltd., 2020. P.~878--887.
\bibitem{8-zac}
\Au{Ackoff R.} From data to wisdom~// J.~Applied Systems Analysis, 1989. Vol.~16. No.\,1. P.~3--9.
\bibitem{9-zac}
\Au{Rosenbloom P.\,S.} On computing: The fourth great scientific domain.~--- Cambridge, MA, 
USA: MIT Press, 2013. 307~p.
\bibitem{10-zac}
\Au{Rowley J.} The wisdom hierarchy: Representations of the DIKW hierarchy~// J.~Inf. 
Sci., 2007. Vol.~33. Iss.~2. P.~163--180. doi: 10.1177/0165551506070706.
\bibitem{11-zac} 
\Au{Frick$\acute{\mbox{e}}$~M.\,H.} Data--Information--Knowledge--Wisdom (DIKW) pyramid, 
framework, continuum~// Encyclopedia of big data~/ Eds. L.~Schintler, C.~McNeely.~--- Cham: 
Springer, 2018. 4~p. doi: 10.1007/978-3-319-32001-4\_331-1.
\bibitem{12-zac}
\Au{Denning P., Rosenbloom~P.} Computing: The fourth great domain of science~// Commun. 
ACM, 2009. Vol.~52. Iss.~9. P.~27--29.
\bibitem{13-zac}
\Au{Denning P., Freeman~P.} Computing's paradigm~// Commun.  ACM, 2009. Vol.~52. 
Iss.~12. P.~28--30. doi: 10.1145/ 1610252.1610265.
\bibitem{17-zac} %14
\Au{Farradane J.} Knowledge, information, and information science~// J.~Inf. Sci., 
1980. Vol.~2. Iss.~2. P.~75--80. doi: 10.1177/01655515800020020.

\bibitem{15-zac}
\Au{Шрейдер Ю.\,А.} Информация и~знание~// Сис\-тем\-ная концепция информационных 
процессов.~--- М.: ВНИИСИ, 1988. С.~47--52.
\bibitem{16-zac}
\Au{Ingwersen P.} Information and information science~// Enclyclopaedie of library and 
information science~/ Eds. J.\,D.~McDonald, 
M.~Levine-Clark.~--- New York, NY, USA: Marcel Dekker Inc., 1992. Vol.~56. Sup.~19. 
P.~137--174.

\bibitem{14-zac} %17
Информатика как наука об информации: Информационный, документальный, 
технологический, экономический, социальный и~организационный аспекты~/ Под ред. 
Р.\,С.~Гиляревского.~--- М.: Фаир-Пресс, 2006. 592~с.

\bibitem{18-zac}
\Au{Hjorland B.} Library and information science: practice, theory, and philosophical basis~// 
Inform. Process. Manag., 2000. Vol.~36. Iss.~3. P.~501--531. doi:  
10.1016/S0306-\mbox{4573(99)00038-2}.
\bibitem{19-zac}
Deep shift~--- technology tipping points and societal impact.~--- Geneva: WE Forum, 2015. 44~p. 
{\sf http://www3.weforum.org/docs/WEF\_GAC15\_ Technological\_Tipping\_Points\_report\_2015.pdf}.
\bibitem{20-zac}
\Au{Berman F., Rutenbar~R., Hailpern~B., Christensen~H., Davidson~S., Estrin~D., 
Franklin~M., Martonosi~M., Raghavan~P., Stodden~V., Szalay~A.\,S.} Realizing the potential of 
data science~// Commun.  ACM, 2018. Vol.~61. Iss.~4. P.~67--72. doi: 10.1145/3188721.

\bibitem{21-zac}
\Au{Stodden V.} The data science life cycle: A~disciplined approach to advancing data science as 
a~science~// Commun.  ACM, 2020. Vol.~63. Iss.~7. P.~58--66. doi: 10.1145/ 3360646.


\bibitem{23-zac} %22
\Au{Зацман И.\,М.} Научная парадигма информатики: классификация трансформаций 
объектов предметной об\-ласти~// Системы и~средства информатики, 2023. Т.~33. №\,4. 
С.~126--138. doi: 10.14357/08696527230412. EDN: ZIKUWO.

\bibitem{22-zac} %23
\Au{Зацман И.\,М.} Научная парадигма информатики: классификация объектов предметной  
об\-ласти~// Информатика и~её применения, 2023. Т.~17. Вып.~4. С.~96--103. doi: 
10.14357/19922264230413. EDN: FIUQAT.

\bibitem{24-zac}
\Au{Зацман И.\,М.} О~научной парадигме информатики: верхний уровень классификации 
объектов ее предметной об\-ласти~// Информатика и~её применения, 2022. Т.~16. Вып.~4. 
С.~73--79. doi: 10.14357/ 19922264220411. EDN: XZNKVI.

\bibitem{25-zac}
\Au{Соломоник А.\,Б.} Философия знаковых систем и~язык.~--- М.: ЛКИ, 2011. 408~с.
\bibitem{26-zac}
\Au{Зацман И.\,М.} Трансформация иерархии Акоффа в~научной парадигме информатики~// 
Информатика и~её применения, 2023. Т.~17. Вып.~3. С.~107--113. doi: 
10.14357/19922264230315. EDN: UMVRRV.

\bibitem{27-zac}
\Au{Zatsman I.} Building digital spiral models of knowledge generation~// 19th Forum 
(International) on Knowledge Asset Dynamics Proceedings.~--- Matera, Italy: Arts for Business 
Institute, 2024. P.~2185--2196.
\bibitem{28-zac}
\Au{Zatsman I.} Digital spiral model of knowledge creation and encoding its dynamics~// 18th 
Forum (International) on Knowledge Asset Dynamics Proceedings.~--- Matera, Italy: Arts for 
Business Institute, 2023. P.~581--596.
\bibitem{29-zac}
\Au{Зацман И.\,М.} Интерфейсы третьего порядка в~информатике~// Информатика и~её 
применения, 2019. Т.~13. Вып.~3. С.~82--89. doi: 10.14357/19922264190312. EDN: 
EHRQLF.

\bibitem{30-zac}
\Au{Зацман И.\,М.} Научная парадигма информатики как третьей культуры~//  
На\-уч\-но-тех\-ни\-че\-ская информация. Сер.~1: Организация и~методика информационной 
работы, 2023. №\,11. С.~1--14.

\end{thebibliography}

 }
 }

\end{multicols}

\vspace*{-9pt}

\hfill{\small\textit{Поступила в~редакцию 14.04.24}}

\vspace*{4pt}

%\pagebreak

%\newpage

%\vspace*{-28pt}

\hrule

\vspace*{2pt}

\hrule



\def\tit{OBJECT TRANSFORMATIONS OF~THE~FIRST AND~SECOND ORDER
IN~A~LEXICOGRAPHIC INFORMATION SYSTEM\\[-5pt]}


\def\titkol{Object transformations of~the~first and~second order
in~a~lexicographic information system}


\def\aut{I.\,M.~Zatsman}

\def\autkol{I.\,M.~Zatsman}

\titel{\tit}{\aut}{\autkol}{\titkol}

\vspace*{-13pt}


\noindent
Federal Research Center ``Computer Science and Control'' of the Russian Academy of Sciences, 
44-2~Vavilov Str., Moscow 119133, Russian Federation


\def\leftfootline{\small{\textbf{\thepage}
\hfill INFORMATIKA I EE PRIMENENIYA~--- INFORMATICS AND
APPLICATIONS\ \ \ 2024\ \ \ volume~18\ \ \ issue\ 2}
}%
 \def\rightfootline{\small{INFORMATIKA I EE PRIMENENIYA~---
INFORMATICS AND APPLICATIONS\ \ \ 2024\ \ \ volume~18\ \ \ issue\ 2
\hfill \textbf{\thepage}}}

\vspace*{2pt}



\Abste{The theoretical foundations of the design of information technologies used for 
the integration of bilingual dictionaries and parallel corpora are considered. The 
description of the first outcomes of the creation of the third\linebreak\vspace*{-12pt}}

\Abstend{ level of object 
transformations classification in the subject domain of informatics, which is supposed 
to be used
in creating the lexicographic information system providing integration, is 
given. All the entities of informatics are divided into two global classes: objects and 
their transformations. For each such class, its own classification is constructed. 
Previously, the two upper levels of the object transformation classification in the subject 
domain have been described. The present paper discusses the third level of this classification. The 
basis for the construction of its highest level was the division of the subject domain of 
informatics into media (mental, sensory, digital, and a~number of other media), each 
of which by definition includes objects of the same nature. The Solomonick's 
typology of sign systems served as the basis for constructing the second level of the 
object transformation classification. The aim of the paper is to systematize object 
transformations of the first and second orders at the third level of this classification. 
The basis for systematization is the medium version of the Ackoff's hierarchy.}

\KWE{subject domain objects; object transformations; classification; data; 
information; knowledge; lexicographic information system}


\DOI{10.14357/19922264240211}{VZTGVV}

\vspace*{-12pt}

\Ack

\vspace*{-3pt}


\noindent
The reported study was funded by the Russian Science Foundation, project  
No.\,24-18-00155, {\sf 
https://rscf.ru/project/24-18-00155}. The research was carried out using the infrastructure of the Shared 
Research Facilities ``High Performance Computing and Big Data'' (CKP 
``Informatics'') of FRC CSC RAS (Moscow) .
   


  \begin{multicols}{2}

\renewcommand{\bibname}{\protect\rmfamily References}
%\renewcommand{\bibname}{\large\protect\rm References}

{\small\frenchspacing
 {%\baselineskip=10.8pt
 \addcontentsline{toc}{section}{References}
 \begin{thebibliography}{99} 
\bibitem{1-zac-1}
\Aue{Aijmer, K., and B.~Altenberg.} 2013. \textit{Advances in corpus-based 
contrastive linguistics. Studies in honour of Stig Johansson}. Amsterdam: John 
Benjamins. 295~p. doi: 10.1075/scl.54.
\bibitem{2-zac-1}
\Aue{Dobrovolskiy, D.\,O., A.\,A.~Kretov, and S.\,A.~Sharov.} 2005. Korpus 
parallel'nykh tekstov [Corpus of parallel texts]. \textit{Nauchnaya i~tekhnicheskaya 
informatsiya. Ser. 2. Informatsionnye protsessy i~sistemy} [Scientific and Technical 
Information. Ser.~2: Information Processes and Systems] 6:16--27.
\bibitem{3-zac-1}
\Aue{Dobrovolskiy, D.\,O.} 2015. Korpus parallel'nykh tekstov i~sopostavitel'naya 
leksikologiya [The corpus of parallel texts and contrastive lexicology]. \textit{Trudy 
Instituta russkogo yazyka im. V.\,V.~Vinogradova} [Proceedings of the 
V.\,V.~Vinogradov Russian Language Institute] 6:413--449. EDN: VJQBHP.
\bibitem{4-zac-1}
\Aue{Goncharov, A.\,A., I.\,M.~Zatsman, and M.\,G.~Kruzhkov.} 2020. Evolyutsiya 
klassifikatsiy v~nadkorpusnykh ba\-zakh dannykh [Evolution of classifications in 
supracorpora databases]. \textit{Informatika i~ee Primeneniya~--- Inform. \mbox{Appl.}}  
14(4):108--116. doi: 10.14357/19922264200415.  
EDN: GKWBZT.
\bibitem{5-zac-1}
\Aue{Goncharov, A.\,A., I.\,M.~Zatsman, and M.\,G.~Kruzhkov.} 2021. 
Predstavlenie novykh leksikograficheskikh znaniy v~dinamicheskikh 
klassifikatsionnykh sistemakh [Representation of new lexicographical knowledge in 
dynamic classification systems]. \textit{Informatika i~ee Primeneniya~--- Inform. 
Appl.} 15(1):86--93. doi: 10.14357/19922264210112. EDN: OPEFXW.
\bibitem{6-zac-1}
\Aue{Zatsman, I.} 2020. Finding and filling lacunas in linguistic typologies. 
\textit{15th Forum (International) on Knowledge Asset Dynamics Proceedings}. 
Matera, Italy: Institute of Knowledge Asset Management. 780--793.
\bibitem{7-zac-1}
\Aue{Zatsman, I.} 2020. Three-dimensional encoding of emerging meanings in  
AI-systems. \textit{21st European Conference on Knowledge Management 
Proceedings}. Reading, U.K.: Academic Publishing International Ltd. 878--887.
\bibitem{8-zac-1}
\Aue{Ackoff, R.} 1989. From data to wisdom. \textit{J.~Applied Systems Analysis} 
16(1):3--9.
\bibitem{9-zac-1}
\Aue{Rosenbloom, P.\,S.} 2013. \textit{On computing: The fourth great scientific 
domain}. Cambridge, MA: MIT Press. 307~p.
\bibitem{10-zac-1}
\Aue{Rowley, J.} 2007. The wisdom hierarchy: Representations of the DIKW 
hierarchy. \textit{J.~Inf. Sci.} 33(2):163--180. doi: 10.1177/0165551506070706.
\bibitem{11-zac-1}
\Aue{Frick$\acute{\mbox{e}}$, M.\,H.} 2018.  
Data-Information-Knowledge-Wisdom (DIKW) pyramid, framework, continuum. 
\textit{Encyclopedia of big data}. Eds. L.~Schintler and C.~McNeely. Cham: 
Springer. 4~p. doi: 10.1007/978-3-319-32001- 4\_331-1.
\bibitem{12-zac-1}
\Aue{Denning, P., and P.~Rosenbloom.} 2009. Computing: The fourth great domain 
of science. \textit{Commun. ACM} 52(9):27--29.
\bibitem{13-zac-1}
\Aue{Denning, P., and P.~Freeman.} 2009. Computing's paradigm. \textit{Commun. 
ACM} 52(12):28--30. doi: 10.1145/ 1610252.1610265.

\bibitem{17-zac-1} %14
\Aue{Farradane, J.} 1980. Knowledge, information, and information science. 
\textit{J.~Inf. Sci.} 2(2):75--80. doi: 10.1177/ 01655515800020020.

\bibitem{15-zac-1}
\Aue{Shreyder, Yu.\,A.} 1988. Informatsiya i~znanie [Information and knowledge]. 
\textit{Sistemnaya kontseptsiya in\-for\-ma\-tsi\-on\-nykh protsessov} [System concept of 
information processes]. Moscow: VNIISI. 47--52.
\bibitem{16-zac-1}
\Aue{Ingwersen, P.} 1995. Information and information science. 
\textit{Encyclopedia of library and information science}. Eds. J.\,D.~McDonald and 
M.~Levine-Clark. New York, NY: Marcel Dekker Inc. 56(19):137--174.

\bibitem{14-zac-1} %17
Gilyarevskiy, R.\,S., ed. 2006. \textit{Informatika kak nauka ob informatsii: 
informatsionnyy, dokumental'nyy, tekh\-no\-lo\-gi\-che\-skiy, ekonomicheskiy, sotsial'nyy 
i~organizatsionnyy aspekty} [Informatics as information science: Informational, 
documentary, technological, economic, social, and organizational dimensions]. 
Moscow: FAIR-PRESS. 592~p.

\bibitem{18-zac-1}
\Aue{Hjorland, B.} 2000. Library and information science: Practice, theory, and 
philosophical basis. \textit{Inform. Process. Manag.} 36(3):501--531. doi:  
10.1016/S0306-\mbox{4573(99)00038-2}.
\bibitem{19-zac-1}
Deep shift~--- technology tipping points and societal impact. 2015. \textit{World Economic 
Forum}. Geneva. 44~p. Available at: {\sf 
http://www3.weforum.org/docs/WEF\_ GAC15\_Technological\_Tipping\_Points\_report\_2015.pdf} (accessed May~20, 
2024).
\bibitem{20-zac-1}
\Aue{Berman, F., R.~Rutenbar, B.~Hailpern, H.~Christensen, S.~Davidson, 
D.~Estrin, M.~Franklin, M.~Martonosi, P.~Raghavan, V.~Stodden, and 
A.\,S.~Szalay.} 2018. Realizing the potential of data science. \textit{Commun. ACM} 
61(4):67--72. doi: 10.1145/3188721.
\bibitem{21-zac-1}
\Aue{Stodden, V.} 2020. The data science life cycle: A~disciplined approach to 
advancing data science as a~science. \textit{Commun. ACM} 
 63(7):58--66. doi: 10.1145/3360646.

\bibitem{23-zac-1} %22
\Aue{Zatsman, I.\,M.} 2023. Nauchnaya paradigma informatiki: klassifikatsiya 
transformatsiy ob''ektov predmetnoy oblasti [Scientific paradigm of informatics: 
Transformation classification of domain objects]. \textit{Sistemy i~Sredstva 
Informatiki~--- Systems and Means of Informatics} 33(4):126--138. doi: 
10.14357/08696527230412. EDN: ZIKUWO.

\bibitem{22-zac-1} %23
\Aue{Zatsman, I.\,M.} 2023. Nauchnaya paradigma informatiki: klassifikatsiya 
ob''ektov predmetnoy oblasti [Scientific paradigm of informatics: Classification of 
domain objects]. \textit{Informatika i~ee Primeneniya~--- Inform. Appl.} 
 17(4):96--103. doi: 10.14357/19922264230413. EDN: FIUQAT.
 
\bibitem{24-zac-1}
\Aue{   Zatsman, I.\,M.} 2022. O nauchnoy paradigme informatiki: verkhniy uroven' 
klassifikatsii ob''ektov ee predmetnoy oblasti [On the scientific paradigm of 
informatics: The classification high level of its objects]. \textit{Informatika i~ee 
Primeneniya~--- Inform. Appl.} 16(4):73--79. doi: 10.14357/19922264220411. EDN: 
XZNKVI.
\bibitem{25-zac-1}
\Aue{Solomonick, A.\,B.} 2011. \textit{Filosofiya znakovykh system i~yazyk} 
[Philosophy of sign systems and language]. Moscow: LKI. 408~p.
\bibitem{26-zac-1}
\Aue{Zatsman, I.\,M.} 2023. Transformatsiya ierarkhii Akoffa v~nauchnoy 
paradigme informatiki [Transformation of the Ackoff's hierarchy in the scientific 
paradigm of informatics]. \textit{Informatika i~ee Primeneniya~--- Inform. \mbox{Appl.}} 
17(3):107--113. doi: 10.14357/19922264230315. EDN: UMVRRV.
\bibitem{27-zac-1}
\Aue{Zatsman, I.} 2024. Building digital spiral models of knowledge 
generation. \textit{19th Forum (International) on Knowledge Asset Dynamics 
Proceedings}. Matera, Italy: Arts for Business Institute. 2185--2196.
\bibitem{28-zac-1}
\Aue{Zatsman, I.} 2023. Digital spiral model of knowledge creation and encoding its 
dynamics. \textit{18th Forum (International) on Knowledge Asset Dynamics 
Proceedings}. Matera, Italy: Arts for Business Institute. 581--596.
\bibitem{29-zac-1}
\Aue{Zatsman, I.\,M.} 2019. Interfeysy tret'ego poryadka v~informatike 
 [Third-order interfaces in informatics]. \textit{Informatika i~ee Primeneniya~--- 
Inform. Appl.} 13(3):82--89. doi: 10.14357/19922264190312. EDN: EHRQLF.
\bibitem{30-zac-1}
\Aue{Zatsman, I.} 2023. Scientific paradigm of informatics as a~third culture. 
\textit{Scientific Technical Information Processing} 50(4):246--258. doi: 
10.3103/S0147688223040111. EDN: CKHMYS.

\end{thebibliography}

 }
 }

\end{multicols}

\vspace*{-6pt}

\hfill{\small\textit{Received April 14, 2024}} 


\vspace*{-12pt}


\Contrl

\vspace*{-3pt}

\noindent
\textbf{Zatsman Igor M.} (b.\ 1952)~--- Doctor of Science in technology, head of 
department, Federal Research Center ``Computer Science and Control'' of the 
Russian Academy of Sciences, 44-2~Vavilov Str., Moscow 119333, Russian 
Federation; \mbox{izatsman@yandex.ru}





\label{end\stat}

\renewcommand{\bibname}{\protect\rm Литература}  %12
\def\stat{goncharov}

\def\tit{ВЫРАВНИВАНИЕ ДЕКАРТОВЫХ ПРОИЗВЕДЕНИЙ УПОРЯДОЧЕННЫХ МНОЖЕСТВ$^*$}

\def\titkol{Выравнивание декартовых произведений упорядоченных множеств}

\def\aut{А.\,В.~Гончаров$^1$, В.\,В.~Стрижов$^2$}

\def\autkol{А.\,В.~Гончаров, В.\,В.~Стрижов}

\titel{\tit}{\aut}{\autkol}{\titkol}

\index{Гончаров А.\,В.}
\index{Стрижов В.\,В.}
\index{Goncharov A.\,V.}
\index{Strijov V.\,V.}


{\renewcommand{\thefootnote}{\fnsymbol{footnote}} \footnotetext[1]
{Работа выполнена при частичной финансовой поддержке РФФИ 
(проекты 19-07-1155 и~19-07-00885). Настоящая статья содержит 
результаты проекта <<Статистические методы машинного обучения>>, 
выполняемого в~рамках реализации Программы Центра компетенций 
Национальной технологической инициативы <<Центр хранения 
и~анализа больших данных>>, поддерживаемого Министерством науки 
и~высшего образования Российской Федерации по договору МГУ им.\ 
М.\,В.~Ломоносова  с~Фондом поддержки проектов Национальной 
технологической инициативы от 11.12.2018 №\,13/1251/2018.}}


\renewcommand{\thefootnote}{\arabic{footnote}}
\footnotetext[1]{Московский физико-технический институт, alex.goncharov@phystech.edu}
\footnotetext[2]{Вычислительный центр им.\ А.\,А.~Дородницына Федерального исследовательского 
центра <<Информатика и~управ\-ле\-ние>> Российской академии наук; 
Московский фи\-зи\-ко-тех\-ни\-че\-ский институт, \mbox{strijov@ccas.ru}}

%\vspace*{-12pt}



\Abst{Работа посвящена исследованию метрических методов анализа 
объектов сложной структуры. Предлагается обобщить метод динамического 
выравнивания двух временных рядов на случай объектов, определенных на 
двух и~более осях времени. В~дискретном представлении такие объекты 
являются матрицами. Метод динамического выравнивания временных рядов 
обобщается как метод динамического выравнивания матриц. Предложена 
функция расстояния, устойчивая к~монотонным нелинейным деформациям 
декартова произведения двух и~более временных шкал. Определен выравнивающий 
путь между объектами. В~дальнейшем объектом называется матрица, 
в~которой строки и~столбцы соответствуют осям времени. Исследованы 
свойства предложенной функции расстояния. Для иллюстрации метода 
решаются задачи метрической классификации объектов на модельных 
данных и~данных из набора MNIST.}

\KW{функция расстояния; динамическое выравнивание; расстояние между матрицами; 
нелинейные деформации времени; про\-стран\-ст\-вен\-но-вре\-мен\-ные ряды}

\DOI{10.14357/19922264200105} 
  
\vspace*{-3pt}


\vskip 10pt plus 9pt minus 6pt

\thispagestyle{headings}

\begin{multicols}{2}

\label{st\stat}


\section{Введение}

Временн$\acute{\mbox{ы}}$е ряды представляют собой набор измерений, упорядоченных 
по оси времени. Анализ временн$\acute{\mbox{ы}}$х рядов производится при решении задач, 
связанных с~классификацией активности человека по измерениям акселерометра 
телефона, поиском паттернов в~EEG-сиг\-на\-лах (электроэнцефалограмма), 
кластеризации набора ECoG (электрокортикограмма) данных и~во многих других 
задачах~\cite{0}. Рассматриваются объекты, для которых время между измерениями 
фиксированно. В~данной работе для построения адекватной функции 
расстояния между объектами требуется учесть нелинейные деформации 
относительно оси времени: глобальные и~локальные сдвиги, растяжения 
и~сжатия~\cite{1}.

В~\cite{2} приводятся различные методы решения задач анализа 
временн$\acute{\mbox{ы}}$х рядов: классификации, детектирования паттернов, 
кластеризации и~др. В~\cite{3} описание временных рядов 
строится с~по\-мощью анализа параметров моделей, в~\cite{4} 
используется их признаковое описание, в~\cite{5} анализируется их форма. 
Комбинации этих подходов описаны в~\cite{2}.

Метрические методы находят схожие объекты в~наборе. Используются 
функции расстояния над временн$\acute{\mbox{ы}}$ми рядами: расстояние Хаусдорфа~\cite{10}, 
MODH~\cite{11}, расстояние, основанное на HMM
(hiden Markov model)~\cite{6}, евклидово расстояние 
в~исходном пространстве или в~пространстве сниженной размерности~\cite{5}, 
\mbox{LCSS} (longest common\linebreak subsequence)~\cite{7}. Показано~\cite{8}, что в~случае локальных или глобальных 
деформаций времени при решении задач, требующих анализа исходной формы 
временн$\acute{\mbox{о}}$го ряда, метод динамического выравнивания оси времени 
DTW (Dynamic Time Warping) 
превосходит другие функции расстояния~\cite{9} по качеству итогового 
решения задачи, так как при наличии смещений двух объектов относительно 
друг друга требуется выравнивать их оптимальным образом для вычисления 
расстояния между ними.

В данной работе предлагается перейти от рас\-смот\-ре\-ния объекта~$\textbf{s}(t)$, 
временн$\acute{\mbox{о}}$го ряда, к~более общему случаю $\textbf{s}(\textbf{t})$, 
в~котором компоненты вектора~$\textbf{t}$~--- оси времени. Из-за 
существенного рос\-та вы\-чис\-ли\-тель\-ной слож\-ности при увеличении чис\-ла 
осей времени предлагается рас\-смот\-реть объекты $\textbf{s}(t_1, t_2)$, 
определенные на двух осях времени. Оси времени считаются независимыми. 
В~случае единственной дискретной и~ограниченной сверху шкалы времени 
объект представим вектором фиксированной размерности. 
Аналогично объект настоящего исследования представим мат\-ри\-цей.

Вводятся ограничения на зависимости осей времени в~декартовом 
произведении для таких объектов. Определена гипотеза порождения данных: 
объекты одного класса эквивалентности получены при помощи допустимых 
преобразований, а~именно: локальных деформаций (растяжений и~сжатий) 
каждой из осей времени по отдельности. В~дискретном случае преобразование 
представимо дуп\-ли\-ци\-ро\-ва\-ни\-ем строк и~столбцов матриц. 
В~число допустимых преобразований попадают и~глобальные деформации: 
сдвиги по осям времени, представимые добавлением и~удалением крайних 
строк и~столбцов исходных матриц. Для каждой из осей времени выполняются 
свойства времени: монотонность и~непрерывность. Похожими на описанные 
свойствами обладает, например, частотный спектр сигнала, где одна ось 
определяет время, а другая~--- частоту, величину, обратную времени.


Между двумя объектами, матрицами, в~случае допустимых преобразований 
требуется определить инвариантную к~преобразованиям осей времени функцию 
расстояния, которая сможет выделить классы эквивалентности множества 
преобразованных объектов. Работа посвящена определению такой функции 
расстояния, как обобщения метода динамического выравнивания временных рядов 
DTW для матриц.

Цель данной работы~--- построение метода, основанного на динамическом 
выравнивании осей времени для матриц. Метод динамического выравнивания 
временн$\acute{\mbox{ы}}$х рядов~\cite{33} определен только для объектов с~одной осью времени, 
что делает его неприменимым для описанного случая. Однако концепции, 
используемые на каждой стадии вы\-чис\-ле\-ния оптимального выравнивания, обобщены 
на рассматриваемый случай. Работа исследует свойства предложенного 
метода и~сравнивает результаты применения метода к~задачам классификации 
изображений~\cite{12} с~результатами функции расстояния~$L_2$.

Для иллюстрации и~анализа результатов решается задача метрической 
классификации объектов (матриц низкой размерности). Используются наборы данных: 
модельные данные, которые согласуются с~выдвинутой гипотезой порождения 
данных для временн$\acute{\mbox{ы}}$х рядов, подмножество набора MNIST сниженной 
размерности и~частотный спектр сигнала.

\vspace*{-10pt}

\section{Постановка задачи построения функции расстояния}

\vspace*{-2pt}

Рассмотрим задачу построения функции расстояния между объектами. 
Функция расстояния инвариантна к~допустимым преобразованиям осей времени: 
глобальным и~локальным линейным и~нелинейным деформациям временн$\acute{\mbox{о}}$й шкалы. 
Ниже приведены две постановки задачи, с~помощью которых определены свойства 
предложенной функции расстояния, оценено ее качество и~проведено сравнение 
нескольких функций расстояния: предложенной и~$L_2$.

Первая постановка задачи использует общее свойство функций расстояния: 
объединение схожих объектов и~разделение непохожих объектов. 
Вводится определение свойства инвариантности функции расстояния к~допустимым 
преобразованиям осей времени.
Вторая постановка задачи уточняет первую и~заключается в~проведении метрической 
классификации методом ближайшего соседа.

\textbf{Постановка задачи выбора функции расстояния между двумя объектами.}
На двух временн$\acute{\mbox{ы}}$х осях заданы объекты вида 
$\textbf{A}(t_1,t_2)\hm \in \mathbb{R}^{n \times n}$. 
Функция $G_w(\textbf{A}):\mathbb{R}^{n \times n} \hm\rightarrow 
\mathbb{R}^{\hat{n} \times \hat{n}}$ задает допустимые преобразования 
исходного объекта~$\textbf{A}$: глобальные сдвиги, локальные линейные 
и~нелинейные деформации, а~именно: растяжения и~сжатия оси времени, 
сдвиги значений по оси времени. Скалярный параметр $w \hm\in \mathbb{R}^+$
 функции~$G$ фиксирует набор этих преобразований.

Допустимым элементарным преобразованием матрицы~$\textbf{A}$ назовем 
дуплицирование случайных строк и~столбцов исходной матрицы, добавление 
или удаление крайних строк и~столбцов. Допустимым преобразованием 
примем некоторую последовательность допустимых элементарных 
преобразований матрицы~$\textbf{A}$ и~обозначим как~$G_w(\textbf{A})$.

Будем называть объект~$\textbf{B} \hm\in \mathbb{R}^{\hat{n} \times \hat{n}}$ 
полученным из объекта~$\textbf{A}$ при помощи допустимых 
преобразований~$G_{\hat{w}}$, если существует $\hat{w}\hm\in \mathbb{R}^+ : 
\textbf{B} \hm= G_{\hat{w}}(\textbf{A})$.

Функцию расстояния между двумя объектами $\rho: 
\mathbb{R}^{{n} \times {n}} \times \mathbb{R}^{\hat{n} \times \hat{n}} 
\hm\rightarrow  \mathbb{R}^+$ оценим на выборке $\mathfrak{D } \hm= 
\{ \textbf{A}_i \}_{i=1}^m$ объектов вида $\textbf{A}_i \hm\in 
\mathbb{R}^{n \times n}$.

Для каждого объекта выборки~$\textbf{A}_i$ и~объекта~$\textbf{B}_j$ его 
класса эквивалентности $\{\textbf{B}_j\}_i \hm= \{  \textbf{B} 
\hm\in \mathfrak{D} | \exists w_i,w_j: G_{w_i}(\textbf{A}_i) \hm= G_{w_j}
(\textbf{B}_j)   \}$ заданы допустимые трансформации с~параметрами~$w_i$ 
и~$w_j$, такие что $G_{w_i}(\textbf{A}_i)\hm = G_{w_j}(\textbf{B}_j)$. 
Для каждого объекта выборки~$\textbf{A}_i$ и~объекта~$\textbf{C}_j$ 
из других классов эквивалентности $\{ \textbf{C}_k\}_i \hm= 
\{  \textbf{C} \hm\in \mathfrak{D} | \nexists w_i,w_k: G_{w_i}(\textbf{A}_i)
\hm = G_{w_k}(\textbf{C})   \}$ не существует таких $ w_i, w_k : G_{w_i}
(\textbf{A}_i) \hm= G_{w_k}(\textbf{C}_k)$.

Решается задача поиска функции расстояния~$\rho$, значение
 которой на паре объектов одного класса эквивалентности меньше, 
 чем на любой паре объектов из разных: для любых $i,j,k \hm\in 
 \{1,\dots,m\}$ $\quad \rho(\textbf{A}_i,\textbf{B}_j) \hm< 
 \rho(\textbf{A},\textbf{C}_k)$. Функцию расстояния, обладающую 
 таким свойством, назовем инвариантной на классах эквивалентности.

Критерием качества для функции расстояния~$\rho$ на выборке~$\mathfrak{D}$ 
примем долю объектов, для которых указанное неравенство выполняется:
$$
S_{\rho}(\mathfrak{D}) = \fr{1}{m} \sum\limits_{i=1}^m 
\prod\limits_{\{ \textbf{B}_j\}_i} 
\prod\limits_{\{ \textbf{C}_k\}_i}  
\left[  \rho(\textbf{A}_i,\textbf{B}_j) < \rho(\textbf{A}_i,\textbf{C}_k)  
 \right].
 $$
Постановка задачи выбора функции расстояния~$\rho$ 
сводится к~задаче максимизации критерия качества.

\textbf{Прикладное использование функции расстояния.}
Задана выборка $\mathfrak{D}\hm = \{(\textbf{A}_i,y_i)\}^m_{i=1}$, 
состоящая из пар объ\-ект--от\-вет. Объектами служат объекты сложной 
структуры: $\textbf{A}_i\hm \in \mathbb{R}^{n\times n}$, 
а~ответами выступают метки класса~---~$y_i\hm \in Y \hm= \{1,\ldots,E\}$, 
где $E \hm\ll m$. Выборка разделена на обучение $\mathfrak{D}_l \hm= 
\{(\textbf{A}_i,y_i)\}^{m_1}_{i=1}$ и~контроль $\mathfrak{D}_t \hm= 
\{(\textbf{A}_i,y_i)\}_{m_1}^{m_1+m_2}$.

Модель классификации~$f$ принадлежит множеству моделей метрической 
классификации 1NN, которые классифицируемому объекту ставят 
в~соответствие метку класса ближайшего объекта из обучающей 
выборки по заданной функции расстояния~$\rho$:
$$ 
\hat{y} = f(\textbf{B} | \rho) = y \argmin\limits_{i = 1,\dots, m_1} 
\rho\left(B,A_i\right)\,.
$$
Критерий качества $S$ модели~$f$ для задачи классификации~--- 
доля правильно проставленного класса на контрольной выборке:
 $$ 
 S(f | \rho) = \fr{1}{m_2}\sum\limits_{i=m_1}^{m_1+m_2} 
 \left[f(\textbf{A}_i | \rho) = y_i\right].
 $$

Требуется выбрать функцию расстояния~$\rho$ для модели 
классификации~$f:~\mathbb{R}^{n\times n} \hm\rightarrow~Y$, 
максимизируюшую критерий качества~$S$ на контрольной выборке:
\begin{equation*}
f =  \argmax\limits_{\rho \in \{\mathrm{mDTW}, L_2\}}\left(S(f | \rho)\right).
\end{equation*}

\section{Вычисление матричного расстояния mDTW}

Предлагается использовать функцию расстояния DTW, 
модифицированную для случая выравнивания двойной шкалы времени.

\smallskip

\noindent
\textbf{Определение~1.} {Даны два объекта~$\textbf{A},\textbf{B}\hm \in 
\mathbb{R}^{n\times n}$. Тензор 
невязок~$\boldsymbol{\Omega}^{n \times n \times n \times n}$~--- 
такой тензор, что его элемент~$\boldsymbol{\Omega}(i,j,k,l)$ 
равен квадрату разности между элементами~$\textbf{A}(i,j)$ и~$\textbf{B}(k,l)$:}
\begin{equation*}
\boldsymbol{\Omega}(i,j,k,l)=(\textbf{A}(i,j) - \textbf{B}(k,l))^2.
\end{equation*}

\noindent
\textbf{Определение 2.} {Путем~$\boldsymbol{\pi}$ между двумя 
объектами $\textbf{A},\textbf{B} \hm\in \mathbb{R}^{n\times n}$ 
назовем множество индексов тензора~$\boldsymbol{\Omega}$: }
$$
\boldsymbol{\pi} = \{(i,j,k,l)\},\quad i,j,k,l \in \{1,\ldots,n\} ,
$$
\textit{удовлетворяющее следующим условиям:}

{\bfseries\textit{Частичный порядок.}}
Для элементов пути~$\boldsymbol{\pi}$ с~фиксированными значениями~$i,k$ 
задан порядок: выравнивающий путь для фиксированных строк двух 
матриц упорядочен~--- $\{(i,j_r,k,l_r))\}_{r=1}^{R} \hm\subset 
\boldsymbol{\pi}$ мощностью~$R$. Аналогично для фиксированных столбцов 
с~индексами~$j,l$.

{\bfseries\textit{Граничные условия.}}
 Пусть $(i,j,k,l) \in \boldsymbol{\pi}$, тогда $(1,j,1,l) \hm\in 
 \boldsymbol{\pi}$ и~$(i,1,k,1) \hm\in \boldsymbol{\pi}$.
Путь $\boldsymbol{\pi}$ содержит элементы тензора~$\boldsymbol{\Omega}$: 
$(1,1,1,1) \hm\in \boldsymbol{\pi}$ и~$(n,n,n,n) \hm\in \boldsymbol{\pi}$.

{\bfseries\textit{Непрерывность по направлению.}}
Для упорядоченного подмножества пути $\{(i,j_r,k,l_r)\}_{r=1}^{R}
\hm\subset\boldsymbol{\pi}$ выполняется условие непрерывности:
$$
j_{r}-j_{r-1}\leq1\,,\quad l_r-l_{r-1}\leq1\,, \quad r = 2,\ldots,R\,.
$$
На~шаге пути~$\boldsymbol{\pi}$ по фиксированному направлению времени~$i,k$ 
встречаются только соседние элементы матрицы (включая соседние по диагонали). 
Аналогично для фиксированных~$j,l$.

{\bfseries\textit{Монотонность по направлению.}}
Для упорядоченного подмножества пути  $\{(i,j_r,k,l_r)\}_{r=1}^{R}
\hm\subset\boldsymbol{\pi}$ выполняется хотя бы одно из условий 
монотонности функции выравнивания времени: 
$$
j_{r}-j_{r-1}\geq1\,,\quad l_r-l_{r-1}\geq1\,, \quad r = 2,\ldots,R\,.
$$

Свойства пути между матрицами обобщают свойства пути между двумя 
временными рядами.

\smallskip

\noindent
\textbf{Определение~3.}\ {Стоимость 
$\mathrm{Cost}\,(\textbf{A},\textbf{B},{\boldsymbol{\pi}})$ пути $\boldsymbol{\pi}$ 
между объектами $\textbf{A}, \textbf{B}$:
\begin{equation*}
\mathrm{Cost}\,(\textbf{A},\textbf{B},{\boldsymbol{\pi}}) = 
\sum\limits_{(i,j,k,l) \in \boldsymbol{\pi}}{\boldsymbol{\Omega}}(i,j,k,l).
\end{equation*}}

\noindent
\textbf{Определение~4.}\ 
{Выравнивающий путь~$\hat{\boldsymbol{\pi}}$ между 
объектами $\textbf{A},\textbf{B}$~--- путь наименьшей стоимости 
среди всех возможных путей между объектами:
\begin{equation*}
\hat{\boldsymbol{\pi}} = 
\argmin\limits_{{\boldsymbol{\pi}}} \mathrm{Cost}
\left(\textbf{A},\textbf{B},{\boldsymbol{\pi}}\right).
\end{equation*}}
Функция расстояния~$\rho (\textbf{A},\textbf{B})\hm = \mathrm{mDTW}\,
(\textbf{A},\textbf{B})$ между объектами~$\textbf{A}$ и~$\textbf{B}$ 
рассчитывается как стоимость выравнивающего пути~$\hat{\boldsymbol{\pi}}$:
\begin{equation}
\mathrm{mDTW}(\textbf{A},\textbf{B}) = \mathrm{Cost}\left(\textbf{A},
\textbf{B},\hat{\boldsymbol{\pi}}\right).
\end{equation}

\setcounter{figure}{1}
\begin{figure*}[b] %fig2
{\small 
\begin{center}
\begin{tabular}{l}
\hline
DTW(\textbf{s},\textbf{c}):\\
\hspace*{3mm}$\boldsymbol{D}$(1:n+1,1:m+1) = inf;\\
\hspace*{3mm}$\boldsymbol{D}$(1,1) = 0;\\
\hspace*{3mm}for $i = 2$: $n+1$\\
\hspace*{6mm}for $j = 2$ : $m+1$\\
\hspace*{9mm}$d = (\textbf{s}(i-1)-\textbf{c}(j-1))^2$;\\
\hspace*{9mm}$\boldsymbol{D}(i,j) = d + \min( 
[ \boldsymbol{D}(i-1,j), \boldsymbol{D}(i,j-1), \boldsymbol{D}(i-1,j-1) ])$;\\
return\ sqrt$(\boldsymbol{D}(n+1,m+1))$\\
\hline
\end{tabular}
\end{center}}
\vspace*{-9pt}

\Caption{Алгоритм вычисления DTW для временных рядов
\label{ris:dtwts}}
%\end{figure*}
%\begin{figure*} %fig3
\vspace*{6pt}
{\small 
\begin{center}
\begin{tabular}{l}
\hline
\\[-9pt]
Correction $(\overline{i,j,k,l}, \boldsymbol{\pi}(\overline{i,j,k,l})):$\\
\hspace*{3mm}if $\overline{i,j,k,l} \in \{ (i-1, j, k,l)  ;  
(i, j, k-1, l)  ;  (i-1, j, k-1, l) \}$:\\
\hspace*{6mm}$ \widehat{\pi} = \{ (\overline{i}, r, \overline{k}, f) \in 
\boldsymbol{\pi}(\overline{i, j, k, l}) \vert r, f \in \mathbb{N} \}$\\
\hspace*{3mm}elif $\overline{i,j,k,l}\in \{  
(i, j-1, k, l); (i, j, k, l-1); (i, j-1, k, l-1) \}$:\\
\hspace*{6mm}$\widehat{\pi} = \{ (r, \overline{j}, f, \overline{l}) 
\in \boldsymbol{\pi}(\overline{i, j, k, l}) \vert r, f \in \mathbb{N} \}$\\
\hspace*{3mm}elif $\overline{i,j,k,l} =  i-1,j-1,k-1,l-1:$\\
\hspace*{6mm}$\widehat{\pi} = \{ (\overline{i}, r, \overline{k}, f) 
\in \boldsymbol{\pi}(\overline{i, j, k, l}) \vert r,f \in \mathbb{N} \} \cup$\\
\hspace*{6mm}$\cup \{ (r, \overline{j}, f, \overline{l}) \in \boldsymbol{\pi}
(\overline{i, j, k, l}) \vert r,f \in \mathbb{N} \}$\\
\hspace*{3mm}$\boldsymbol{d\pi} = \{ \mathrm{element} \in \widehat{\pi}: 
\mbox{произведены\ замены\ индексов } 
\overline{i} = i,\ \overline{j} = j,\ \overline{k} = k,\ \overline{l} = l \}$\\
return $\boldsymbol{d\pi}$\\
\hline
\end{tabular}
\end{center}
}
\vspace*{-9pt}

\Caption{Алгоритм вычисления поправки $\boldsymbol{d\pi}$ 
пути $\boldsymbol{\pi}$
\label{ris:codedpi}}
\end{figure*}


\textbf{Алгоритм вычисления значения расстояния~(4).}
Построение алгоритма вычисления значения функции расстояния 
между матрицами основан на алгоритме расчета функции расстояния 
между временн$\acute{\mbox{ы}}$ми рядами. В~случае выравнивания одной\linebreak\vspace*{-12pt}

{ \begin{center}  %fig1
 \vspace*{-3pt}
    \mbox{%
 \epsfxsize=79mm 
 \epsfbox{gon-1.eps}
 }


\end{center}


\noindent
{{\figurename~1}\ \ \small{Матрица стоимости оптимального выравнивания, по обеим 
осям отложены временные отсчеты}}
}

\vspace*{12pt}


\noindent 
временн$\acute{\mbox{о}}$й шкалы
 итоговая матрица расстояний~$\boldsymbol{D}$ (рис.~1) в~каждом 
 элементе~$\boldsymbol{D}(i,j)$ содержит рас\-сто\-яние между подрядом 
 первого временн$\acute{\mbox{о}}$го ряда и~подрядом второго временн$\acute{\mbox{о}}$го ряда. 
 Рас\-смот\-рим алгоритм динамического выравнивания двух временн$\acute{\mbox{ы}}$х 
 рядов $\textbf{s} \hm\in R^n$ и~$\textbf{c} \hm\in R^m$ на рис.~2.
 
 

Элемент $\boldsymbol{D}(i,j)$ матрицы~$\boldsymbol{D}$ соответствует 
стоимости выравнивающего пути между подпоследовательностями 
исходных временн$\acute{\mbox{ы}}$х рядов: $\textbf{s}(1:i) \hm= \textbf{s}(t)$, 
$t \hm= 1,\ldots,i,$ и~$\textbf{c}(1:j) \hm= \textbf{c}(t)$, $t \hm= 1,\ldots,j$. 
Алгоритм построения наилучшего выравнивания времени 
подразумевает, что выравнивающий путь между этими 
подпоследовательностями получен одним из трех способов~--- 
если стоимость выравнивающего пути между 
подпоследовательностями~$\textbf{s}(1:\overline{i}) $ 
и~$\textbf{c}(1:\overline{j})$ минимальна для~$\overline{i,j}$ из множества
$$
\overline{i,j} \in \left\{ \{i-1,j\},\{i,j-1\},\{i-1,j-1\} \right\},$$
тогда выравнивающий путь между $\textbf{s}(1:i)$ и~$\textbf{c}(1:j)$ получен добавлением пары~$(i,j)$ к~выбранному 
выравнивающему пути с~минимальной стоимостью из трех.



Предложенный алгоритм переносит эти рас\-суж\-де\-ния на случай 
выравнивания двух матриц~$\textbf{A}$ и~$\textbf{B}$. 
Элемент~$\boldsymbol{D}(i,j,k,l)$ четырехиндексного
 тензора расстояний~$\boldsymbol{D}$ соответствует стоимости выравнивающего 
 пути между $\textbf{A}(1:i,1:j) \hm= \textbf{A}(t_1,t_2)$, 
 $t_1 \hm= 1,\ldots, i$, $t_2 \hm= 1,\ldots, j,$ 
 и~$\textbf{B}(1:k,1:l) \hm= \textbf{B}(t_1,t_2)$, $t_1 \hm= 1,\ldots, k$,
 $t_2 \hm= 1,\ldots, l$. Выравнивающий путь между этими 
 подматрицами получен одним из семи способов~--- 
 если стоимость выравнивающего пути между 
 подматрицами $\textbf{A}(1:\overline{i},1:\overline{j})$ 
 и~$\textbf{B}(1:\overline{k},1:\overline{l})$ 
 минимальна для~$\overline{i,j,k,l}$ из множества
\begin{multline*} 
\overline{i,j,k,l} \in 
\left\{ \{i-1,j,k,l\},\{i,j-1,k,l\},\right.\\
\{i,j,k-1,l\},
\{i,j,k,l-1\}, \{i-1,j,k-1,l\},\\
\left.
\{i,j-1,k,l-1\},\{i-1,j-1,k-1,l-1\}\right\},
\end{multline*}

\setcounter{figure}{3}
\begin{figure*} %fig4
{\small 
\begin{center}
\begin{tabular}{l}
\hline
$\mathrm{mDTW}\left(\textbf{A},\textbf{B}\right):$\\
\hspace*{3mm}$\textbf{D}(1:n+1,1:n+1, 1:n+1, 1:n+1) = inf$;\\
\hspace*{3mm}$\textbf{D}(1,1,1,1) = 0;$\\
\hspace*{3mm}$\boldsymbol{\pi}(1,1,1,1) = ((1,1),(1,1))$\\
\hspace*{3mm}$for\ i,j,k,l  \in \mathbb{N}^{2 : n+1} \times 
\mathbb{N}^{2 : n+1} \times \mathbb{N}^{2 : n+1} \times \mathbb{N}^{2 : n+1}:$\\
\hspace*{6mm}$\overline{i,j,k,l} = \argmin($ [ \textbf{D}(i-1, j, k, l), 
\textbf{D}(i, j-1, k, l), \textbf{D}(i, j, k-1, l), 
\textbf{D}(i, j, k, l-1),    \\
\hspace*{9mm}$\textbf{D}(i-1, j, k-1, l), \textbf{D}(i, j-1, k, l-1), 
\textbf{D}(i-1, j-1, k-1, l-1) ])$;\\
\hspace*{3mm}$\boldsymbol{d \pi} = \mathrm{Correction}\,(\overline{i,j,k,l}, 
\boldsymbol{\pi}(\overline{i,j,k,l}))$\\
\hspace*{3mm}$\boldsymbol{\pi}(i, j, k, l) = \boldsymbol{d \pi} \cup 
\{(\overline{i,j,k,l})\}$\\
\hspace*{3mm}$\mathrm{cost} = (\textbf{A}(i, j)-\textbf{B}(k, l))^2 + 
\sum\nolimits_{(r,f,t,g) \in \boldsymbol{d \pi}}
(\textbf{A}(r, f)-\textbf{B}(t, g))^2$;\\
\hspace*{3mm}$\textbf{D}(i,j,k,l) = \mathrm{cost} + \textbf{D}
(\overline{i,j,k,l})$\\
return  sqrt$(\textbf{D}(n+1,n+1,n+1,n+1))$\\
\hline
\end{tabular}
\end{center}
}
\vspace*{-9pt}

\Caption{Алгоритм вычисления расстояния между матрицами
\label{ris:matrixdtw}}
\end{figure*}

\begin{table*}[b]\small
\begin{center}
\begin{tabular}{|l|c|c|c|c|}
\multicolumn{5}{c}{Снижение расстояний при выполнении преобразований 
для различных наборов данных}\\
\multicolumn{5}{c}{\ }\\[-6pt]
\hline
 &\multicolumn{4}{c|}{Метод}\\
 \cline{2-5}
\multicolumn{1}{|c|}{Данные}  & \multicolumn{2}{c|}{$L_2$} & \multicolumn{2}{c|}{MatrixDTW} \\
\cline{2-5}
& $S(f|p)$  &  $S_{\rho}(\mathfrak{D})$ &  $S(f|p)$ & $S_{\rho}(\mathfrak{D})$ \\
\hline
Модельные данные без преобразований& 92\% & 78\% & 100\%\hphantom{9} & 85\% \\
Модельные данные с~преобразованиями & 86\% & 65\% &  100\%\hphantom{9} & 82\% \\
Модельные данные с~преобразованиями и~шумом& 69\% & 61\% &  92\% & 78\% \\
MNIST без преобразований& 95\% & --- & 95\% & --- \\
MNIST с~преобразованиями & 53\% & --- & 92\% & --- \\
Спектр сигнала& 83\% & --- & 96\% & --- \\
\hline
\end{tabular}
\end{center}
\end{table*}

\noindent
то к~выравнивающему пути между этими под\-мат\-ри\-ца\-ми 
добавляется элемент пути $(i,j,k,l)$ и~поправка~$\boldsymbol{d\pi} $ 
пути~$\boldsymbol{\pi}$, алгоритм вычисления которой приведен ниже.

Обозначим выравнивающий путь между $\textbf{A}(1:i,\linebreak 1:j)$
 и~$\textbf{B}(1:k,1:l)$ как~$\boldsymbol{\pi}(i,j,k,l)$, тогда 
 поправка~$\boldsymbol{d\pi} $ пути~$\boldsymbol{\pi}(i,j,k,l)$ 
 при фиксированных~$\overline{i,j,k,l}$ вычисляется приведенным на рис.~3 
 образом.





Алгоритм динамического выравнивания двух матриц и~вычисления 
расстояния $\mathrm{mDTW}$ между ними с~учетом приведенного выше 
алгоритма примет вид, представленный на рис.~4.





\begin{figure*} %fig5
\vspace*{1pt}
    \begin{center}  
  \mbox{%
 \epsfxsize=161.412mm 
 \epsfbox{gon-5.eps}
 }
\end{center}
\vspace*{-12.5pt}
\Caption{Выравнивание модельных данных: (\textit{а})~один класс без шума; 
(\textit{б})~разные классы без шума; 
(\textit{в})~один класс с~шумом; (\textit{г})~разные классы с~шумом
\label{ris:random}}
%\end{figure*}
%\begin{figure*} %fig6
\vspace*{1pt}
    \begin{center}  
  \mbox{%
 \epsfxsize=163mm 
 \epsfbox{gon-6.eps}
 }
\end{center}
\vspace*{-12.5pt}
\Caption{Выравнивание данных MNIST: левый столбец~--- один класс; 
правый столбец~--- разные 
классы;
(\textit{а})~$\mathrm{mDTW}\hm=720{,}1$; 
(\textit{б})~948,6;
(\textit{в})~2017,0;
(\textit{г})~$\mathrm{mDTW}\hm=2071{,}4$
\label{ris:mnist}}
\end{figure*}


Следует отметить, что алгоритм~\cite{15} имеет\linebreak высокую сложность 
вычисления~--- $O(n^4)$. Предполагается ускорение метода 
с~использованием ограниче\-ния Sakoe-Chiba band, что сократит 
вычислительную сложность алгоритма до $O(n^2k^2)$, где~$k$~--- 
параметр ограничения.


\section{Вычислительный эксперимент}

Вычислительный эксперимент проведен на модельных данных с~допустимыми 
преобразованиями и~на реальных данных: объектах коллекции MNIST с~допустимыми 
преобразованиями и~на спектрограммах зашумленных сигналов.





Решается задача метрической классификации методом ближайшего соседа. В~таблице 
приведены значения критерия качества функции расстояния 
$S_{\rho}(\mathfrak{D})$ и~критерия качества метрической классификации $S(f|p)$ 
при использовании двух функций расстояния: предложенной в~работе $\mathrm{mDTW}$ 
и~$L_2$.

Модельные данные~--- это нулевые матрицы со случайными ненулевыми 
строками, столбцами, подпрямоугольниками с~наложенным шумом. 
К~ним применены допустимые преобразования, согласованные с~гипотезой 
наличия локальных и~глобальных искажений. На рис.~\ref{ris:random} 
показан пример оптимального выравнивания двух объектов. 
Линиями показаны элементы пути~$\boldsymbol{\pi}$.

Подготовлена подвыборка набора данных MNIST. Она 
состоит из~100 объектов классов 0 и~1 сниженной размерности
 с~допустимыми преобразованиями. На рис.~\ref{ris:mnist} 
 показан пример оптимального выравнивания объектов.


Аналогичный эксперимент проведен для решения задачи метрической 
классификации спектров различных сигналов, пример которых приведен на 
рис.~\ref{ris:spectr}. На рисунке показаны примеры Фурье-спект\-ров 
этих сигналов. Спектр получен путем применения быстрого преобразования 
Фурье к~исходному сигналу для различных окон с~фиксированным размером и~сдвигом. 
Исходные временн$\acute{\mbox{ы}}$е ряды обладали свойством периодичности, период выбирался 
случайным образом.



Тестирование проведено на разного рода данных: исходных 
модельных данных без наложения\linebreak\vspace*{-12pt}

\pagebreak

\end{multicols}

\begin{figure*} %fig7
\vspace*{1pt}
    \begin{center}  
  \mbox{%
 \epsfxsize=149.062mm 
 \epsfbox{gon-7.eps}
 }
\end{center}
\vspace*{-8pt}
\Caption{Данные спектров сигнала: (\textit{а})~класс~1; (\textit{б})~спектр 
класса~1; (\textit{в})~класс~2; (\textit{г})~спектр класса~2; 
(\textit{д})~класс~3; (\textit{е})~спектр класса~3
\label{ris:spectr}}
\vspace*{9pt}
\end{figure*}

\begin{multicols}{2}

\noindent допустимых преобразований, с~ними, а~также 
на модельных данных с~наложенным поверх объектов случайным шумом.



В каждом из проведенных экспериментов была продемонстрирована 
устойчивость предложенного подхода к~допустимым преобразованиям. 
Наилучшее значение критерия качества задачи классификации было 
достигнуто при использовании предложенной функции расстояния.

\vspace*{-5pt}

\section{Заключение}

В работе предложено обобщение метода динамического выравнивания
 временн$\acute{\mbox{ы}}$х рядов для случая объектов, определенных на двух осях времени. 
 Существует теоретическое обобщение предлагаемых методов на случай 
 конечного множества осей времени. Вычислительный эксперимент позволил 
 проанализировать свойства подхода: устойчивость к~допустимым 
 преобразованиям и~разделяющая способность функции расстояния как 
 на реальных, так и~на модельных данных. Качество решения задачи 
 метрической классификации выше решения, основанного на евклидовом 
 расстоянии. Вычислительная сложность метода высокая, что ограничивает 
 его применимость на объектах высокой размерности.

\vspace*{-2pt}

{\small\frenchspacing
 {%\baselineskip=10.8pt
 \addcontentsline{toc}{section}{References}
 \begin{thebibliography}{99}
%\bibitem{Karasikov2016}
%\Au{Карасиков~М.\,Е., Стрижов~В.\,В.} Классификация временных рядов 
%в~пространстве параметров по\-рож\-да\-ющих моделей~// Информатика и~её 
%применения,~2016. T.~10. Вып.~4. С.~121--131.

\bibitem{0}
\Au{Hill~N.\,J., Lal~T.\,N., Schroder~M., Hinterberger~T., 
Wilhelm~B., Nijboer~F., Mochty~U., Widman~G., Elger~C., 
Scholkopf~B., Kubler~A., Birbaumer~N.} Classifying EEG and 
ECoG signals without subject training for fast BCI implementation: 
Comparison of nonparalyzed and completely paralyzed subjects~//  
IEEE~T. Neur. Sys. Reh., 2006. Vol.~14. 
Iss.~2. P.~183--186.

\bibitem{1}
\Au{Sakoe~H., Chiba~S.} 
A~dynamic programming approach to continuous speech recognition~// 
7th  Congress (International) on Acoustics Proceedings, 1971. Vol.~3. P.~65--69.

\bibitem{2} %3
\Au{Aghabozorgi~S., Ali~S.\,S., Wah~T.\,Y.} 
Time-series clustering~--- a~decade review~// Inform. Syst., 
2015. Vol.~53. P.~16--38.

\bibitem{3} %4
\Au{Warrenliao~T.} Clustering of time series data~--- a~survey~// 
Pattern Recogn., 2005. Vol.~38. Iss.~11. P.~1857--1874.



\bibitem{4} %5
\Au{Hautamaki~V., Nykanen~P., Franti~P.} 
Time-series clustering by approximate prototypes~// 
19th  Conference (International) on Pattern Recognition Proceedings, 2008. No.\,D. 
P.~1--4.

\bibitem{5} %6
\Au{Faloutsos~C., Ranganathan~M., Manolopoulos~Y.} 
Fast subsequence matching in time-series databases~// \mbox{SIGMOD} Rec., 1994. 
Vol.~23. Iss.~2. P.~419--429.

\bibitem{10} %7
\Au{Basalto~N., Bellotti~R., Carlo~F.\,D., Facchi~P., 
Pascazio~S.} Hausdorff clustering of financial time series~// 
Physica~A, 2007. Vol.~379. Iss.~2. P.~635--644.

\bibitem{11} %8
\Au{Gorelick~L., Blank~M., Shechtman~E., Irani~M., Basri~R.} 
Actions as space-time shapes~// IEEE~T. Pattern Anal., 
2007. Vol.~29. Iss.~12. P.~2247--2253.

\bibitem{6} %9
\Au{Smyth~P.} Clustering sequences with hidden Markov models~// 
Adv. Neural In., 1997. Vol.~9. P.~648--654.

\bibitem{7} %10
\Au{Banerjee~A., Ghosh~J.} Clickstream clustering using weighted 
longest common subsequences~// 
Workshop on Web Mining, SIAM Conference on Data Mining
Proceedings, 2001. P.~33--40.

\bibitem{8} %11
\Au{Aach~J., Church~G.M.} Aligning gene expression time series
 with time warping algorithms~// Bioinformatics, 2001. Vol.~17. Iss.~6. P.~495--508.

\bibitem{9} %12
\Au{Yi~B.\,K., Faloutsos~C.} Fast time sequence indexing 
for arbitrary $\mathcal{L}_p$ norms~// 
26th  Conference (International) on Very Large Data Bases Proceedings, 2000. P.~385--394.

\bibitem{33} %13
\Au{Goncharov~A.\,V., Strijov~V.\,V.} 
Analysis of dissimilarity set between time series~// Computational 
Mathematics Modeling, 2018. Vol.~29. Iss.~3. P.~359--366.

\bibitem{12} %14
\Au{Alon~J., Athitsos~V., Sclaroff~S.}
 Online and offline character recognition using alignment to prototypes~// 
 8th  Conference (International) on Document Analysis and Recognition, 2005. 
 Vol.~2. P.~839--843.

\bibitem{15} %15
\Au{Гончаров~А.\,В.} 
Выравнивания декартовых произведений упорядоченных множеств mDTW. 
Про\-грам\-мная реализация алгоритма, 2019. 
{\sf https://github.
com/Intelligent-Systems-Phystech/PhDThesis/tree/\linebreak  master/Goncharov2019/MatrixDTW/code}.
 \end{thebibliography}

 }
 }

\end{multicols}

\vspace*{-9pt}

\hfill{\small\textit{Поступила в~редакцию 24.04.19}}

\vspace*{6pt}

%\pagebreak

%\newpage

%\vspace*{-28pt}

\hrule

\vspace*{2pt}

\hrule

\vspace*{-4pt}

\def\tit{ALIGNMENT OF~ORDERED SET CARTESIAN PRODUCT\\[-5pt]}


\def\titkol{Alignment of~ordered set cartesian product}

\def\aut{A.\,V.~Goncharov$^1$ and~V.\,V.~Strijov$^{1,2}$}

\def\autkol{A.\,V.~Goncharov and~V.\,V.~Strijov}

\titel{\tit}{\aut}{\autkol}{\titkol}

\vspace*{-13pt}


\noindent
$^1$ Moscow Institute of Physics and Technology, 
9~Institutskiy Per., Dolgoprudny, Moscow Region 141700, Russian\linebreak
$\hphantom{^1}$Federation


\noindent
$^2$A.\,A.~Dorodnicyn Computing Center, Federal Research Center 
``Computer Science and Control'' of the Russian\linebreak
$\hphantom{^1}$Academy of Sciences, 
40~Vavilov Str., Moscow 119333, Russian Federation

\def\leftfootline{\small{\textbf{\thepage}
\hfill INFORMATIKA I EE PRIMENENIYA~--- INFORMATICS AND
APPLICATIONS\ \ \ 2020\ \ \ volume~14\ \ \ issue\ 1}
}%
 \def\rightfootline{\small{INFORMATIKA I EE PRIMENENIYA~---
INFORMATICS AND APPLICATIONS\ \ \ 2020\ \ \ volume~14\ \ \ issue\ 1
\hfill \textbf{\thepage}}}

\vspace*{2pt} 



\Abste{The work is devoted to the study of metric methods for analyzing 
objects with complex structure. It proposes to generalize the dynamic 
time warping method of two time series for the case of objects defined 
on two or more time axes. Such objects are matrices in the discrete 
representation. The DTW (Dynamic Time Warping) method of time series is generalized as 
a~method of matrices dynamic alignment. The paper proposes 
a~distance function resistant to monotonic nonlinear deformations of the 
Cartesian product of two time scales. The alignment path between objects is 
defined. An object is called a~matrix in which the rows and columns correspond 
to the axes of time. The properties of the proposed distance function 
are investigated. To illustrate the method, the problems of metric 
classification of objects are solved on model data and data from the 
MNIST dataset.}

\KWE{distance function; dynamic alignment; distance between matrices; 
nonlinear time warping; space--time series}



\DOI{10.14357/19922264200105} 

%\vspace*{-14pt}

\Ack
\noindent
This work was supported by the Russian Foundation for Basic
Research (projects 19-07-1155 and 19-07-00885). 
The paper contains results of the project Statistical 
methods of machine learning, which is carried out within the 
framework of the Program ``Center of Big Data Storage and Analysis'' 
of the National Technology Initiative Competence Center. 
It is supported by the Ministry of Science and Higher Education 
of the Russian Federation according to the agreement between the
 M.\,V.~Lomonosov Moscow State University and the Foundation 
 of project support of the National Technology Initiative from 11.12.2018, 
 No.\,13/1251/2018.
 


%\vspace*{6pt}

  \begin{multicols}{2}

\renewcommand{\bibname}{\protect\rmfamily References}
%\renewcommand{\bibname}{\large\protect\rm References}

{\small\frenchspacing
 {%\baselineskip=10.8pt
 \addcontentsline{toc}{section}{References}
 \begin{thebibliography}{99}

 \bibitem{0-1}   
\Aue{Hill, N.\,J., T.\,N.~Lal, M.~Schroder, T.~Hinterberger, B.~Wilhelm, 
F.~Nijboer, U.~Mochty, G.~Widman, C.~Elger, B.~Scholkopf, A.~Kubler, and 
N.~Birbaumer.} 2006. Classifying EEG and ECoG signals without subject 
training for fast BCI implementation: Comparison of nonparalyzed and completely 
paralyzed subjects. \textit{IEEE~T. Neur. Sys. 
Reh.} 14(2):183--186.

\bibitem{1-1}   
\Aue{Sakoe, H., and S.~Chiba.} 1971. A~dynamic programming approach 
to continuous speech recognition. \textit{7th 
 Congress (International) on Acoustics Proceedings}. 3:65--69.

\bibitem{2-1}    %2
\Aue{Aghabozorgi,~S., S.\,S.~Ali, and T.\,Y.~Wah.} 2015. 
Time-series clustering~--- a~decade review.  \textit{Inform. Syst.} 
53:16--38.

\bibitem{3-1}   %4 
\Aue{Warrenliao,~T.} 2005. Clustering of time series data~--- a~survey. 
\textit{Pattern Recogn.}
38(11):1857--1874.



\bibitem{4-1}    %5
\Aue{Hautamaki,~V., P.~Nykanen, and P.~Franti.} 2008. 
Time-series clustering by approximate prototypes. 
 \textit{19th  Conference (International) on Pattern Recognition Proceedings}. 
 D:1--4.

\bibitem{5-1}    %6
\Aue{Faloutsos,~C., M.~Ranganathan, and Y.~Manolopoulos.} 1994. 
Fast subsequence matching in time-series databases.  \textit{SIGMOD Rec}. 
23(2):419--429.

\bibitem{10-1}    %7
\Aue{Basalto, N., R.~Bellotti, F.\,D.~Carlo, P.~Facchi, and S.~Pascazio.} 
2007. Hausdorff clustering of financial time series. 
\textit{Physica~A} 379(2):635--644.

\bibitem{11-1}   %8
\Aue{Gorelick, L., M.~Blank, E.~Shechtman, M.~Irani, and R.~Basri.} 
2007. Actions as space-time shapes.
\textit{IEEE~T. Pattern Anal.} 29(12):2247--2253.

\bibitem{6-1}    %9
\Aue{Smyth, P.} 1997. 
Clustering sequences with hidden Markov models. \textit{Adv. Neural In.} 9:648--654.

\bibitem{7-1}    %10
\Aue{Banerjee,~A., and J.~Ghosh.} 2001. 
Clickstream clustering using weighted longest common subsequences.  
\textit{Workshop on Web Mining, SIAM Conference 
on Data Mining Proceedings.} 33--40.

\bibitem{8-1}    %11
\Aue{Aach, J., and G.\,M.~Church.} 2001. 
Aligning gene expression time series with time warping algorithms. 
\textit{Bioinformatics} 17(6):495--508.

\bibitem{9-1}   %12
\Aue{Yi, B.\,K., and C.~Faloutsos.} 2000. 
Fast time sequence indexing for arbitrary $\mathcal{L}_p$ norms. 
\textit{26th  Conference (International) 
on Very Large Data Bases Proceedings}. 385--394.

\bibitem{33-1}   %13 
\Aue{Goncharov,~A.\,V., and V.\,V.~Strijov.} 2018. 
Analysis of dissimilarity set between time series. 
\textit{Computational Mathematics Modeling } 29(3):359--366.



\bibitem{12-1}    %14
\Aue{Alon, J., V.~Athitsos, and S.~Sclaroff.} 2005.
 Online and offline character recognition using alignment to prototypes. 
 \textit{8th  Conference (International) on Document Analysis and Recognition}. 
 2:839--843.

\bibitem{15-1}    %15
\Aue{Goncharov, A.\,V.} Alignment of 
Ordered Set Cartesian Product mDTW. Software implementation of the algorithm. 
Available at: {\sf https://github.com/Intelligent-\linebreak 
Systems-Phystech/PhDThesis/tree/master/Goncharov\linebreak 2019/MatrixDTW/code} 
(accessed December~27, 2019).
\end{thebibliography}

 }
 }

\end{multicols}

%\vspace*{-7pt}

\hfill{\small\textit{Received April 24, 2019}}

%\pagebreak

%\vspace*{-22pt}



\Contr

\noindent
\textbf{Goncharov Alexey V.} (b.\ 1995)~--- 
PhD student, Moscow Institute of Physics and Technology, 
9~Institutskiy Per., Dolgoprudny, Moscow Region 141701, 
Russian Federation; \mbox{alex.goncharov@phystech.edu}

\vspace*{3pt}

\noindent
\textbf{Strijov Vadim V.} (b.\ 1967)~--- 
Doctor of Science in physics and mathematics, leading scientist, 
A.\,A.~Dorodnicyn Computing Centre, Federal Research Center 
``Computer Science and Control'' of the Russian Academy of Sciences, 
40~Vavilov Str., Moscow 119333, Russian Federation;
 professor, Moscow Institute of Physics and Technology, 
 9~Institutskiy Per., Dolgoprudny, Moscow Region 141701, Russian Federation; 
 \mbox{strijov@ccas.ru}
\label{end\stat}

\renewcommand{\bibname}{\protect\rm Литература} %13
\def\stat{nuriev}

\def\tit{МЕТОДОЛОГИЯ КОРПУСНО-ОРИЕНТИРОВАННОГО ИССЛЕДОВАНИЯ 
В~ОБЛАСТИ КОНТРАСТИВНОЙ ПУНКТУАЦИИ$^*$\\[-5pt]}

\def\titkol{Методология корпусно-ориентированного исследования 
в~области контрастивной пунктуации}

\def\aut{В.\,А.~Нуриев$^1$, В.\,И.~Карпов$^2$}

\def\autkol{В.\,А.~Нуриев, В.\,И.~Карпов}

\titel{\tit}{\aut}{\autkol}{\titkol}

\index{Нуриев В.\,А.}
\index{Карпов В.\,И.}
\index{Nuriev V.\,A.}
\index{Karpov V.\,I.}


{\renewcommand{\thefootnote}{\fnsymbol{footnote}} \footnotetext[1]
{Работа выполнена за счет гранта Российского научного фонда (проект 23-28-00548) с~использованием инфраструктуры 
Центра коллективного пользования <<Высокопроизводительные вычисления и~большие данные>> (ЦКП 
<<Информатика>>) ФИЦ ИУ РАН (г.~Москва).}}


\renewcommand{\thefootnote}{\arabic{footnote}}
\footnotetext[1]{Федеральный исследовательский центр <<Информатика и~управление>> Российской академии наук, 
\mbox{nurieff.v@gmail.com}}
\footnotetext[2]{Институт языкознания Российской академии наук; Федеральный исследовательский центр <<Информатика 
и~управ\-ле\-ние>> Российской академии наук, \mbox{wi.karpow@gmail.com}}

\vspace*{-3pt}

  
  
    
  \Abst{Уточняется  подход к~современным исследованиям 
в~об\-ласти контрастивной пунктуации с~точки зрения методологии. С~учетом новейших достижений информатики, 
компьютерной лингвистики и~теории перевода такие исследования очевидным образом 
должны иметь кор\-пус\-но-ори\-ен\-ти\-ро\-ван\-ный характер. В~данной статье представлена 
методологическая схема подобного исследования, направленного на выявление 
межъязыковой пунктуационной асим\-мет\-рии посредством сравнения функционального 
диапазона одного и~того же знака препинания в~разных языках. Показываются основные 
методологические тенденции, характерные для этой научной об\-ласти. Внимание 
уделяется особенностям корпусной методологии при контрастивном изучении 
пунктуации. В~качестве одного из современных методологических инструментов 
предлагаются надкорпусные базы данных (НБД), раз\-ра\-ба\-ты\-ва\-емые в~ФИЦ ИУ РАН.}

%\vspace*{-6pt}
  
  \KW{контрастивная пунктуация; перевод; корпусное переводоведение; кор\-пус\-но-ори\-ен\-ти\-ро\-ван\-ное 
  исследование; параллельный корпус; надкорпусная база данных; 
межъязыковая асим\-мет\-рия; методология}

%\vspace*{-6pt}

\DOI{10.14357/19922264230213}{VBOZAO} 
  
%\vspace*{-3pt}


\vskip 10pt plus 9pt minus 6pt

\thispagestyle{headings}

\begin{multicols}{2}

\label{st\stat}
    
    \section{Введение}
    
    \vspace*{-3pt}
    
  Важность и~необходимость исследований в~области контрастивной 
пунктуации в~научной литературе отмечалась неоднократно (см., 
например,~[1--7]). Обычно эта необходимость выводится из нужд 
переводческой практики, которая предполагает при обработке письменного 
текста обязательную речемыслительную программу, связанную с~исходным 
пунктуационным компонентом и~его переносом в~сис\-те\-му переводящего 
языка. Так, Ньюмарк в~своем <<Учебнике перевода>> пишет, что 
<<пунктуация может быть мощнейшим инструментом, но ее настолько легко 
упус\-тить из виду, что я~советую переводчикам: специально сравнивайте, где 
у~вас рас\-став\-ле\-ны знаки препинания, а~где они стоят 
в~оригинале>>~\cite[с.~58]{4-nu}. В~работе <<Переводчик в~текс\-те: 
о~чтении русской литературы  
по-анг\-лий\-ски>> значение пунктуации отмечает Мей, критикуя 
англоязычных переводчиков за недостаточное внимание к~межъязыковой 
пунктуационной асим\-мет\-рии~--- за <<игнорирование отличительных 
особенностей, присущих знакам препинания>>~\cite[с.~121]{2-nu}. 
О~пунктуации в~переводе говорит Юдейл, выделяя три аспекта:
%\begin{enumerate}[(1)]
%\item 
(1)~<<знаки препинания~--- важ\-ная часть перевода, но, концентрируясь на 
общем смыс\-ле переводимого, ее час\-то не замечают>>; 
%\item 
(2)~<<изменения 
в~пунктуации при переводе могут значительно по\-вли\-ять на вы\-ра\-зи\-тель\-ность 
текс\-та, его свя\-зан\-ность и~ритм>>; 
%\item 
(3)~<<час\-то возникает впечатление, что 
литературные переводчики наделили себя правом менять границы исходного 
предложения и~пунктуационные знаки, как им 
заблагорассудится>>~\cite[с.~121]{5-nu}.
%\end{enumerate}
 Гораздо реже 
в~специализированной литературе подчеркивается роль, которую 
исследования в~об\-ласти контрастивной пунктуации играют при обучении 
иностранным языкам, в~част\-ности при обуче\-нии иноязычной письменной 
речи~\cite{7-nu}.
  
  Признавая безусловную зна\-чи\-мость данного научного на\-прав\-ле\-ния и~его 
дальнейшего развития, необходимо предметно разрабатывать методологию 
исследования в~об\-ласти контрастивной пунктуации, которая учитывала бы 
новейшие достижения информатики, компьютерной лингвистики 
и~корпусного переводоведения. Пред\-став\-ля\-ет\-ся, что такая методология 
долж\-на основываться на использовании современных информационных 
корпусных инструментов, поз\-во\-ля\-ющих автоматизированным образом 
обрабатывать пред\-ста\-ви\-тель\-ные массивы текс\-то\-вых данных, 
и,~следовательно, носить кор\-пус\-но-ори\-ен\-ти\-ро\-ван\-ный характер 
(о~корпусных данных при контрастивном изуче\-нии пунктуации  
см.~\cite{6-nu}).

%\vspace*{-6pt}
    
    \section{Методологические модели  
корпусно-ориентированного исследования контрастивной 
пунктуации}

\vspace*{-3pt}
  
  В мае 2019~г.\ в~Регенсбурге (Германия) про\-шла научная конференция 
под названием <<Punctuation Seen Internationally. System--Norm--Practice>> 
(<<Пунктуация в~мировом мас\-шта\-бе: 
 сис\-те\-ма--нор\-ма--прак\-ти\-ка>>)~--- первая конференция, пол\-ностью\linebreak 
по\-свя\-щен\-ная проб\-ле\-мам контрастивной пунктуации. Оргкомитет, собирая 
заявки на участие, справедливо отмечал, что до на\-сто\-яще\-го времени 
пунктуации едва ли уделялось внимание в~рамках \mbox{типологии}, контрастивной 
лингвистики, прагмалингвистики, а~так\-же в~исследованиях индивидуальной 
языковой манеры на фоне языкового стандарта. Сейчас появляются 
отдельные работы, где проводится сопоставительное изуче\-ние пунктуации, 
однако по-преж\-не\-му ощущается острая не\-об\-хо\-ди\-мость в~исследованиях по 
контрастивной пунктуации, которые бы учитывали типологические 
(сис\-тем\-ные), социолингвистические (нормативные) и~прагматические 
(речевые) ас\-пекты.
  
  Итогом конференции стала коллективная монография~\cite{8-nu}, 
со\-сто\-ящая из шестнадцати статей, которые пред\-став\-ля\-ют собой пио\-нер\-ские 
исследования, на\-прав\-лен\-ные на формирование целостной па\-ра\-диг\-мы 
контрастивного изучения пунктуации и~борьбу с~маргинализацией важ\-ной 
научной от\-расли. Все статьи услов\-но мож\-но разделить на~4~категории, 
первые две из которых имеют в~большей степени тео\-ре\-ти\-че\-ский характер и~связаны с~сис\-те\-мой и~нормой, а~вторые~--- более практической 
на\-прав\-лен\-ности~--- с~узусом и~освоением пунктуационных навыков. 
В~пред\-став\-лен\-ных работах доминируют два подхода к~исследованию 
конт\-растив\-ной пунк\-ту\-ации:
  \begin{enumerate}[(1)]
\item интралингвистический (контрастивный анализ знаков препинания 
и~кон\-ку\-ри\-ру\-ющих с~ними маркеров синтаксических отношений в~рамках 
одного языка)~\cite[с.~110]{9-nu};
  \item  интерлингвистический (контрастивный анализ знаков препинания 
  и~конкурирующих с~ними средств в~разных языках, конт\-растив\-ная пунктуация 
рас\-смат\-ри\-ва\-ет\-ся в~том чис\-ле как часть методики обуче\-ния неродному языку, 
например при интеграции трудовых мигрантов в~иноязычную 
среду)~\cite[с.~57--73]{10-nu}.
  \end{enumerate}
  
  Интралингвистический подход час\-то носит смешанный характер: если 
речь идет об эволюции пунктуационной сис\-те\-мы отдельно взятого языка на 
фоне развития аналогичных сис\-тем других языков, контрастивный анализ 
со\-про\-вож\-да\-ет\-ся 
 ис\-то\-ри\-ко-эти\-мо\-ло\-ги\-че\-ским~\cite[с.~187--206]{11-nu}. В~рамках 
этого подхода в~указанной монографии имеются психолингвистические 
исследования с~нетривиальным корпусным материалом. Так, 
в~статье~\cite[с.~163--186]{12-nu} корпусные данные привлекаются для 
контрастивного анализа пунктуационных предпочтений двух групп 
ис\-пы\-ту\-емых. Автор использует корпус \mbox{CoPaDocs} (Corpus of Patient 
Documents), основу которого со\-ста\-ви\-ли письма и~другие личные документы 
бывших пациентов психиатрических учреж\-де\-ний Германии на рубеже  
XIX--XX~вв. Корпус поз\-во\-ля\-ет установить, зависит ли языковое оформление 
пись\-ма от лич\-ности адресата~--- происходит ли переключение ре\-гист\-ров 
сознательно. Данный корпус создан с~целью разработки интегративной 
методики анализа языковой ва\-риа\-тивн\-ости, в~том чис\-ле и~в~об\-ласти 
пунктуации. \mbox{Изучив} специфику расстановки~12~знаков препинания, 
Эбер-Хам\-мерль приходит к~выводу, что пациенты, чей род де\-я\-тель\-ности 
прежде не был связан с~письменной сферой, использовали больше 
пунктуационных маркеров (но с~меньшей ва\-риа\-тив\-ностью), чем 
представители второй опытной группы~--- канцелярские служащие. 
В~лич\-ной переписке участники обеих групп к~знакам препинания прибегали 
гораздо реже, чем в~документах, адресованных официальным лицам.
  
 В статье~\cite[с.~57--73]{10-nu} представлено контрастивное исследование, выполненное в~интерлигвистическом 
ключе. Со\-по\-став\-ле\-ние 
пунктуации в~италь\-ян\-ском и~немецком языках здесь проводится на основе 
комплексной методологии, вклю\-ча\-ющей приемы дескриптивного, 
просодического, синтаксического и~ком\-му\-ни\-ка\-тив\-но-текс\-то\-во\-го 
анализа. Примеры приводятся из различных источников, причем 
к~корпусным данным в~статье отсылают не напрямую, а~опосредованно~--- 
через более раннюю работу~\cite{13-nu}. По мнению авторов, 
пунктуирование в~этих языках организовано по-раз\-но\-му, что объясняется 
резкими различиями в~пунктуационном узусе: если в~итальянском знаки 
препинания коммуникативно на\-гру\-же\-ны, то в~немецком они подчинены 
фор\-маль\-но-син\-так\-си\-че\-ско\-му принципу. Иначе говоря, итальянская 
пунктуация выполняет не формальную функцию, а~сигнализиру-\linebreak ет о~тон\-ких 
смыс\-ло\-вых нюансах, которых нельзя\linebreak достичь другими языковыми 
средствами (аргументативный конфликт, полифонические эффекты, 
метатекстовые комментарии). В~этом же духе\linebreak выполнена и~другая 
интерлингвистическая работа~\cite{14-nu}, по\-свя\-щен\-ная контрастивному 
исследованию многоточия и~тире в~италь\-ян\-ском и~анг\-лий\-ском языках 
и~продуктивно ис\-поль\-зу\-ющая \mbox{корпусный} метод сбора и~обработки 
эмпирических данных.
     
     Объединенные в~коллективную монографию рабо\-ты позволяют 
вывести обобщенную ме\-то\-до\-ло\-гическую схему контрастивного изуче\-ния 
пунктуации. Она имеет трехфазную структуру. Первая\linebreak фаза включает 
тео\-ре\-ти\-че\-ское описание пунктуации в~изуча\-емом языке с~привлечением 
исторических и~современных нормативных грамматик и~справочников. 
Вторая фаза на\-прав\-ле\-на на описание трансформаций в~других языках, 
оказавших существенное влияние на статус и~мес\-то пунктуации в~сис\-те\-ме 
конкретного языка. Обе фазы нацелены на создание такого 
исследовательского поля, которое поз\-во\-лит выявить значение пунктуации 
для языковой культуры. Это, в~свою очередь, долж\-но стать задачей треть\-ей 
фазы. Вторая и~\mbox{третья} фазы предполагают межъязыковое сравнение как 
функционального диапазона отдельно взятых знаков препинания, так 
и~пунктуационного репертуара в~целом. На этих стадиях применяется 
корпусный метод. Контрастивный анализ в~за\-ви\-си\-мости от по\-став\-лен\-ных 
целей и~задач наряду со знаками препинания может охватывать 
и~кон\-ку\-ри\-ру\-ющие с~ними языковые средства. На\-прав\-ле\-ние контрастивного 
исследования пунктуации может быть и~синхронным, и~диахроническим.

\vspace*{-6pt}
    
    \section{Методологические особенности  
корпусно-ориентированного исследования в~области 
контрастивной пунктуации}

\vspace*{-3pt}
  
  Особенности методологии при корпусном контрастивном изуче\-нии 
пунктуации, как, впрочем, и~при любом 
 кор\-пус\-но-ори\-ен\-ти\-ро\-ван\-ном исследовании, связаны прежде всего 
со стремлением получить непротиворечивые, валидные и~на\-деж\-ные данные. 
Электронный корпус, будучи методологически новаторским инструментом 
для получения научного знания, поз\-во\-ля\-ет, с~одной стороны, автоматическим 
образом обрабатывать большие массивы данных и~тем самым серьезно 
сокращает временные издержки на поиск эмпирического материала. 
С~другой стороны, электронные корпусные ресурсы имеют свои 
особенности, и~без их над\-ле\-жа\-ще\-го учета пользователь рискует получить 
искаженные результаты.
  
  Например, в~указанной выше работе~\cite[с.~291]{14-nu} авторы, описывая 
методологию своего исследования, отмечают, что итальянские примеры 
заимствованы из корпуса, хранящегося в~Базельском университете 
и~со\-сто\-яще\-го из двух частей~--- 33~современных  
ро\-ма\-на-бест\-сел\-ле\-ра (1~млн словоупотреблений) 
и~нехудожественных текс\-та разной на\-прав\-лен\-ности (1~млн 40~тыс.\ 
словоупотреблений), в~то время как англоязычные примеры извлечены из 
подкорпуса <<Книги и~периодические издания>> Британского 
национального корпуса (80~млн словоупотреблений). Итальянский материал, 
по словам авторов, был проанализирован весь, а~для английского из-за 
гораздо большего объема ограничились анализом случайной выборки, объем 
которой со\-по\-ста\-вим с~выборкой из итальянского корпуса. Очевидным 
образом ва\-лид\-ность выводов по результатам анализа англоязычного 
материала здесь может оказаться под вопросом в~силу методологически 
неоднородных установок применительно к~процедуре обработки данных, 
полученных по двум языкам. Примечательно к~тому же, что базельский 
корпус, в~отличие от британского, за\-крыт для общественного пользования.
  
  О подобных ограничениях рассуждает На\-двор\-ни\-ко\-ва в~своей работе, 
по\-свя\-щен\-ной корпусной методологии контрастивного изучения 
пунктуации~\cite{15-nu}, где анализируется час\-тот\-ность упо\-треб\-ле\-ния шести 
знаков препинания (запятой, точки, двоеточия, точ\-ки с~запятой, 
вопросительного и~восклицательного знака) в~английском, французском 
и~чешском языках. Для сбора данных используются со\-по\-ста\-ви\-мые 
веб-кор\-пу\-сы, моноязычные общие (референтные) и~параллельные корпусы. Цель 
автора~--- определить, какой из трех типов корпусных ресурсов наиболее 
подходит для исследований в~об\-ласти контрастивной пунктуации.
  
  Полученные данные показывают, что при изучении пунктуации показатели 
час\-тот\-ности проявляют высокую чув\-ст\-ви\-тель\-ность к~типу текс\-та; 
следовательно, веб-кор\-пу\-сы, которые, как правило, отличают стихийное 
наполнение, не\-упо\-ря\-до\-чен\-ность и~низ\-кая степень струк\-ту\-ри\-ро\-ван\-ности, не 
могут служить источником до\-сто\-вер\-ной информации об упо\-треб\-ле\-нии 
знаков препинания в~том или ином языке. Моноязычный общий корпус, 
наоборот, содержит специальную раз\-мет\-ку (морфологическую, 
синтаксическую и~т.\,д.)\ и~поз\-во\-ля\-ет гиб\-ко настраивать поиск (в~том чис\-ле 
выбирать соответствующий тип текс\-та) в~за\-ви\-си\-мости от конкретных 
исследовательских задач. Такие корпусы располагают большими массивами 
данных, поскольку призваны пред\-ста\-вить язык во всей его пол\-но\-те 
и~многообразии, что, казалось бы, обеспечивает на\-деж\-ность и~ва\-лид\-ность 
полученных результатов. Меж\-ду тем этот тип корпусов имеет существенный 
недостаток~--- ограниченную межъязыковую со\-по\-ста\-ви\-мость. Как правило, 
моноязычные общие корпусы разных языков разительно отличаются по 
объему данных и~их со\-ста\-ву и~поэтому не подходят в~качестве основного 
инструмента контрастивного исследования, а~могут служить лишь 
референтным (проверочным) источником для дополнительной верификации 
ре\-зуль\-ти\-ру\-ющих данных. Кроме того, со\-по\-ста\-ви\-тель\-ный анализ 
относительной час\-тот\-ности упо\-треб\-ле\-ния знаков препинания в~разных 
языках на основе данных, извлеченных из корпусов этого типа, так\-же имеет 
свои ограничения. Он не применим для изучения пунк\-ту\-а\-ции в~языках 
разного строя, которым для кодирования информации требуется 
количественно больше (аналитические языки типа французского) или 
меньше слов (синтетические языки типа русского). Таким образом, лучше 
всего для контрастивного изуче\-ния пунк\-ту\-а\-ции подходят параллельные 
корпусы, которые, не\-смот\-ря на свой сравнительно небольшой объем, 
пред\-став\-ля\-ют существенно больше воз\-мож\-но\-стей для качественного анализа 
упо\-треб\-ле\-ния знаков препинания и~непосредственного со\-по\-став\-ле\-ния их 
абсолютной час\-тот\-ности в~параллельных текс\-тах~--- оригинале и~переводе. 
Однако и~этот тип информационного ресурса не может служить 
универсальным исследовательским инструментом. При его использовании 
необходимо учитывать, что пунктуационные рас\-хож\-де\-ния в~исходном 
и~переводном текс\-те могут быть не результатом сис\-тем\-ных дифференциаций, 
а~возникнуть под влиянием переводческих предпочтений. Следовательно, 
чтобы избежать искажения ре\-зуль\-ти\-ру\-ющих данных, надо следовать 
некоторым методологическим принципам: %\\[-13pt] 
\begin{enumerate}[(1)]
\item данные собираются в~обоих 
переводных на\-прав\-ле\-ни\-ях; %\\[-13pt] 
\item выявленные тенденции проходят 
обязательную проверку с~по\-мощью референтного моноязычного корпуса; %\\[-13pt] 
\item контрастивное изуче\-ние пунктуации с~применением параллельных 
корпусов требует сис\-тем\-но\-го подхода в~том смыс\-ле, что в~функциональном 
диапазоне разных знаков препинания могут быть общие зоны, ука\-зы\-ва\-ющие 
на их потенциальную внут\-ри\-язы\-ко\-вую и~межъ\-язы\-ко\-вую конкуренцию. %\\[-13pt]
\end{enumerate}
    
 \vspace*{-12pt}
 
    \section{Заключение}
    
    \vspace*{-3pt}
    
  В статье представлена обобщенная методологическая схема 
  кор\-пус\-но-ори\-ен\-ти\-ро\-ван\-но\-го 
  исследования в~об\-ласти контрастивной пунктуации~--- 
от\-расли научного знания, интенсивно \mbox{раз\-ви\-ва\-ющей\-ся} и~при\-вле\-ка\-ющей 
внимание специалистов самого широкого профиля. Несмотря на то что 
появляются работы, где описываются сопоставительные исследования 
пунктуации на примере одного произведения или литературного наследия 
отдельно взятого писателя (см., например,~\cite{16-nu,17-nu}), очевидно, что 
для ка\-ких-ли\-бо существенных, круп\-но\-мас\-штаб\-ных обобщений относительно 
межъязыковой пунктуационной асимметрии и~специфики функционирования 
знаков препинания в~разных языках требуется привлечение корпусного 
материала.
  
  Дальнейшее изучение контрастивной пунктуации видится в~нескольких 
направлениях. Необходимо качественное углубление со\-по\-ста\-ви\-тель\-но\-го 
анализа, чтобы его тонкая нюансировка \mbox{поз\-во\-ли\-ла} установить, в~какой мере 
совпадает и~разнится функциональный диапазон того или иного знака 
препинания в~кон\-так\-ти\-ру\-ющих языках в~за\-ви\-си\-мости от жанровой 
при\-над\-леж\-ности текс\-та. Этот анализ целесообразно проводить комплексно, 
охватывая всю со\-во\-куп\-ность синтаксических изменений, которые влекут за 
собой отказ от исходного пунктуирования при переводе с~одного языка на 
другой. Такая ком\-плекс\-ность поможет выявить и~с~большей пол\-но\-той 
описать су\-щест\-ву\-ющие межъ\-язы\-ко\-вые структурные различия, что 
необходимо и~для переводческой практики, и~для обуче\-ния иностранным 
языкам. Требует дальнейшего уточ\-не\-ния вопрос, как на пунктуационные 
преференции переводчика влияет род\-ная языковая культура, 
пунктуационные уста\-нов\-ки которой могут меняться со временем. По мере 
наращивания опыта и~мастерства могут меняться пунктуационные 
предпочтения и~самого переводчика, и~это так\-же пред\-став\-ля\-ет определенный 
научный интерес.
  
  В заключение следует отметить, что одним из современных 
информационных инструментов корпусного исследования в~об\-ласти 
контрастивной пунктуации могут быть НБД, 
раз\-ра\-ба\-ты\-ва\-емые в~отделе~54 Федерального исследовательского цент\-ра 
<<Информатика и~управ\-ле\-ние>> Российской академии наук (о~возможностях 
НБД см.~\cite{6-nu}). В~данный момент этот методологический инструмент 
проходит апро\-ба\-цию в~контрастивном исследовании двоеточия и~многоточия в~трех языках~--- русском, французском и~немецком.

\vspace*{-9pt}
  
{\small\frenchspacing
 {\baselineskip=11.5pt
 %\addcontentsline{toc}{section}{References}
 \begin{thebibliography}{99}
 
 \vspace*{-3pt}
 
 \bibitem{4-nu} %1
\Au{Newmark P.} A~textbook of translation.~--- New York, London, Toronto, Sydney, Tokyo: Prentice 
Hall, 1988. 402~p.
 

\bibitem{2-nu} %2
\Au{May R.} The translator in the text: On reading Russian literature in English.~--- Evanston, IL, USA: 
Northwestern University Press, 1994. 209 p.
\bibitem{3-nu}
\Au{Munday J.} Systems in translation: A~systemic model for descriptive translation studies~// 
Crosscultural transgressions: Research models in translation studies II~--- historical and 
ideological issues~/ Ed. T.~Hermans.~---  Manchester, U.K.: St.\ Jerome, 2002. P.~76--92.
\bibitem{1-nu} %4
\Au{Baker M.} In other words.~--- 2nd ed.~--- London, New York: Routledge, 2011. 352~p.

\bibitem{7-nu} %5
\Au{Сигал К.\,Я.} Контрастивная пунктуация в~начале XXI века~// Язык. Текст. Дискурс: 
Научный альманах Ставропольского отделения РАЛК.~--- Ставрополь: СКФУ, 
2019.  Вып.~17. С.~69--78.
\bibitem{5-nu} %6
\Au{Youdale R.} Using computers in the translation of literary style: Challenges and 
opportunities.~--- London, New York: Routledge, 2020. 242~p.
\bibitem{6-nu} %7
\Au{Нуриев В.\,А., Кружков~М.\,Г.} Корпусные данные при контрастивном изуче\-нии 
пунктуации~// Сис\-те\-мы и~средства информатики, 2023. Т.~33. №\,1. С.~14--23. doi: 10.14357/08696527230102.

\bibitem{8-nu}
Vergleichende Interpunktion~--- comparative punctuation~/ Eds. P.~R$\ddot{\mbox{o}}$ssler, P.~Besl, A.~Saller.~--- 
Berlin, Boston: De Gruyter, 2021. 454~p.
\bibitem{9-nu}
\Au{Rinas K.} Vom genormten Satzbau zur genormten Interpunktion. Zur Funktion der 
Zeichensetzung in $\ddot{\mbox{a}}$lterer und neuerer Zeit~// Vergleichende Interpunktion~--- comparative 
punctuation~/ Eds. P.~R$\ddot{\mbox{o}}$ssler, P.~Besl, A.~Saller.~---
 Berlin, Boston: De Gruyter, 2021. P.~109--136. doi: 10.1515/9783110756319-006.
\bibitem{10-nu}
\Au{Ferrari~A., Stojmenova Weber R.} Das Komma in kontrastiver Perspektive Italienisch-Deutsch~// Vergleichende Interpunktion~--- 
comparative punctuation / Eds. P.~R$\ddot{\mbox{o}}$ssler, P.~Besl, 
A.~Saller.~--- Berlin, Boston: De Gruyter, 2021. P.~57--73. doi: 10.1515/9783110756319-003.

\columnbreak

\bibitem{11-nu}
\Au{Besch W.} Zur Entwicklung der deutschen Interpunktion seit dem sp$\ddot{\mbox{a}}$ten Mittelalter~// 
Interpretation und Edition deutscher Texte des Mittelalters. Festschrift f$\ddot{\mbox{u}}$r John Asher zum 60. 
Geburtstag~/ Eds. K.~Smits, W.~Besch, V.~Lange.~--- Berlin: Erich Schmidt, 1981. P.~187--206.
\bibitem{12-nu}
\Au{Eber-Hammerl F.} Interpunktion in historischen Patientenbriefen // Vergleichende
Interpunktion~--- comparative punctuation~/ Eds. P.~R$\ddot{\mbox{o}}$ssler, 
P.~Besl, A.~Saller.~--- Berlin, Boston: De Gruyter, 2021. P.~163--186.
\bibitem{13-nu}
\Au{Ferrari A.} Leggere la virgola. Una prima ricognizione~// Chimera Romance Corpora 
Linguistic Studies, 2017. Vol.~4. Iss.~2. P.~145--162. doi: 
10.15366/chimera2017. 4.2.001.
\bibitem{14-nu}
\Au{Pecorari F., Longo~F.} The ellipsis and the dash in Italian and English: A~contrastive 
perspective~// Vergleichende Interpunktion~--- comparative punctuation~/ Eds.
 P.~R$\ddot{\mbox{o}}$ssler, P.~Besl, A.~Saller.~--- Berlin, Boston: De Gruyter, 2021. P.~289--314.
 doi: 10.1515/9783110756319-013.
\bibitem{15-nu}
\Au{N$\acute{\mbox{a}}$dvorn$\acute{{\iota}}$kov$\acute{\mbox{a}}$~O.}
The use of English, Czech and French punctuation marks in reference, 
parallel and comparable web corpora: A~question of methodology~// 
Linguist. Prag.,  2020. Vol.~30. Iss.~2. P.~30--50. doi: 
10.14712/ 18059635.2020.1.2.
\bibitem{16-nu}
\Au{Сигал К.\,Я.} Пунктуация как средство создания эмоционального под\-текс\-та (на 
материале рассказа М.\,А.~Шолохова <<Судьба человека>> и~его переводов на английский 
язык)~// Известия РАН. Серия литературы и~языка, 2014. Т.~73. №\,6. С.~38--50.
\bibitem{17-nu}
\Au{Богданов К.\,А.} Пунктуация как мотив: многоточие и~тире~// НЛО, 2022. №\,2(174). С.~241--253.
doi: 0.53953/ 08696365\_2022\_174\_2\_241.

\end{thebibliography}

 }
 }

\end{multicols}

\vspace*{-8pt}

\hfill{\small\textit{Поступила в~редакцию 15.04.23}}

\vspace*{6pt}

%\pagebreak

%\newpage

%\vspace*{-28pt}

\hrule

\vspace*{2pt}

\hrule

\vspace*{-2pt}

\def\tit{METHODOLOGY OF~THE~CORPUS-BASED STUDIES\\ 
IN~THE~FIELD OF~CONTRASTIVE PUNCTUATION}


\def\titkol{Methodology of~the~corpus-based studies 
in~the~field of~contrastive punctuation}


\def\aut{V.\,A.~Nuriev$^1$ and~V.\,I.~Karpov$^{1,2}$}

\def\autkol{V.\,A.~Nuriev and~V.\,I.~Karpov}

\titel{\tit}{\aut}{\autkol}{\titkol}

\vspace*{-14pt}


\noindent
      $^1$Federal Research Center ``Computer Science and Control'' of the Russian 
Academy of Sciences, 44-2~Vavilov\linebreak
$\hphantom{^1}$Str., Moscow 119333, Russian Federation
      
      \noindent
      $^2$Institute of Linguistics of the Russian Academy of Sciences, 1~bld.~1 
Bolshoy Kislovsky Lane, Moscow 125009,\linebreak
$\hphantom{^1}$Russian Federation

\def\leftfootline{\small{\textbf{\thepage}
\hfill INFORMATIKA I EE PRIMENENIYA~--- INFORMATICS AND
APPLICATIONS\ \ \ 2023\ \ \ volume~17\ \ \ issue\ 2}
}%
 \def\rightfootline{\small{INFORMATIKA I EE PRIMENENIYA~---
INFORMATICS AND APPLICATIONS\ \ \ 2023\ \ \ volume~17\ \ \ issue\ 2
\hfill \textbf{\thepage}}}

\vspace*{3pt}
      
      
    
    \Abste{The paper refines the methodological approach to the contrastive 
studies of punctuation. Given the recent achievements of information science, 
computer linguistics, and translation theory, such studies are most likely to be 
corpus-based. The paper presents a~methodological model of research into 
interlingual punctuation asymmetry, the aim of which is to shed light on the 
functional scope of the same punctuation marks in different languages. It shows 
what methodological trends are characteristic of this research area. The focus is 
also on the specificities of corpus methodology in the contrastive study of 
punctuation. It is argued that one of the methodological tools, tailored specifically 
to the needs of contrastive punctuation research, may be the supracorpora 
databases developed at the Federal Research Center ``Computer Science and 
Control'' of the Russian Academy of Sciences.}
    
    \KWE{contrastive punctuation; translation; corpus-based translation studies; 
corpus-based studies; parallel corpus; supracorpora database; asymmetry between 
languages; methodology}
    
    
    
\DOI{10.14357/19922264230213}{VBOZAO}

%\vspace*{-18pt}

\Ack
    \noindent
    The research was carried out using the infrastructure of the Shared Research 
Facilities ``High Performance Computing and Big Data'' (CKP ``Informatics'') of 
FRC CSC RAS (Moscow). The research was supported by the Russian Science Foundation (project  
No.\,23-28-00548).
 
%\vspace*{4pt}

  \begin{multicols}{2}

\renewcommand{\bibname}{\protect\rmfamily References}
%\renewcommand{\bibname}{\large\protect\rm References}

{\small\frenchspacing
 {%\baselineskip=10.8pt
 \addcontentsline{toc}{section}{References}
 \begin{thebibliography}{99}
 
 \bibitem{4-nu-1} %1
\Aue{Newmark, P.} 1988. \textit{A~textbook of translation}. New York, London, Toronto, Sydney, Tokyo: 
Prentice Hall. 402~p.   

\bibitem{2-nu-1}
\Aue{May, R.} 1994. \textit{The translator in the text: On reading Russian 
literature in English}. Evanston, IL: Northwestern University Press. 209~p.
\bibitem{3-nu-1}
\Aue{Munday, J.} 2002. Systems in translation: A~systemic model for 
descriptive translation studies. \textit{Crosscultural transgressions: Research models in 
translation studies II~--- historical and ideological issues}. Ed. T.~Hermans. 
Manchester, U.K.: St.\ Jerome. 76--92.

\bibitem{1-nu-1} %4
\Aue{Baker, M.} 2011. \textit{In other words}. 2nd ed. London, New York: 
Routledge. 352~p.

\bibitem{7-nu-1} %5
\Aue{Seagal, K.\,Ya.} 2019. Kont\-ras\-tiv\-naya punk\-tu\-a\-tsiya v~na\-cha\-le XXI~veka 
[Contrastive punctuation at the beginning of the XXI century]. \textit{Yazyk. Tekst. 
Diskurs: Nauchnyy al'manakh Stavropol'skogo otdeleniya RALK} [Language. Text. 
Discourse: Scientific almanac of Stavropol Branch of the Russian Cognitive 
Linguists Association].  Stavropol': SKFU. 17:69--78.

\bibitem{5-nu-1} %6
\Aue{Youdale, R.} 2020. \textit{Using computers in the translation of literary style: 
Challenges and opportunities}. London, New York: Routledge. 242~p.
\bibitem{6-nu-1} %7
\Aue{Nuriev, V.\,A., and M.\,G.~Kruzhkov.} 2023. Kor\-pus\-nye dan\-nye pri 
kont\-ras\-tiv\-nom izu\-che\-nii punk\-tu\-a\-tsii [The parallel corpora perspective on studying 
contrastive punctuation]. \textit{Sistemy i~Sredstva Informatiki~--- Systems and Means of 
Informatics} 33(1):14--23. doi: 10.14357/08696527230102.

  \bibitem{8-nu-1}
R$\ddot{\mbox{o}}$ssler, P., P.~Besl, and A.~Saller, eds. 2021. \textit{Vergleichende 
Interpunktion~--- comparative punctuation}. Berlin, Boston: De Gruyter. 454~p.
\bibitem{9-nu-1}
\Aue{Rinas, K.} 2021. Vom genormten satzbau zur genormten interpunktion. 
Zur funktion der zeichensetzung in $\ddot{\mbox{a}}$lterer und neuerer zeit. \textit{Vergleichende 
Interpunktion~--- comparative punctuation}. Eds.\ P.~R$\ddot{\mbox{o}}$ssler, 
P.~Besl, and A.~Saller. 
Berlin, Boston: De Gruyter. 109--136. doi: 10.1515/ 9783110756319-006.
\bibitem{10-nu-1}
\Aue{Ferrari, A., and R.~Stojmenova.} 2021. Weber das komma in kontrastiver 
perspektive Italienisch-Deutsch. \textit{Vergleichende Interpunktion~--- comparative 
punctuation}. Eds. P.~R$\ddot{\mbox{o}}$ssler, P.~Besl, and A.~Saller. Berlin, Boston: De Gruyter.  
57--73. doi: 10.1515/9783110756319-003.
 \bibitem{11-nu-1}
\Aue{Besch, W.} 1981. Zur entwicklung der deutschen interpunktion seit 
dem sp$\ddot{\mbox{a}}$ten mittelalter. \textit{Interpretation und Edition deutscher Texte des Mittelalters. 
Festschrift f$\ddot{\mbox{u}}$r John Asher zum 60. Geburtstag}. Eds. K.~Smits, W.~Besch, and 
V.~Lange. Berlin: Erich Schmidt. 187--206.
 \bibitem{12-nu-1}
\Aue{Eber-Hammerl, F.} 2021. Interpunktion in historischen 
Patientenbriefen. \textit{Vergleichende Interpunktion~--- comparative punctuation}. Eds. 
P.~R$\ddot{\mbox{o}}$ssler, P.~Besl, and A.~Saller. Berlin, Boston: De Gruyter. 163--186.
\bibitem{13-nu-1}
\Aue{Ferrari, A.} 2017. Leggere la virgola. Una prima ricognizione. 
\textit{Chimera Romance Corpora Linguistic Studies} 4(2):145--162. doi: 
10.15366/chimera2017.4.2.001.
\bibitem{14-nu-1}
\Aue{Pecorari, F., and F.~Longo.} 2021. The ellipsis and the dash in Italian 
and English: A~contrastive perspective. \textit{Vergleichende Interpunktion~--- 
comparative punctuation}. Eds. P.~R$\ddot{\mbox{o}}$ssler, P.~Besl, and A.~Saller. Berlin, Boston: 
De Gruyter. 289--314. doi: 10.1515/9783110756319-013.
\bibitem{15-nu-1}
\Aue{N$\acute{\mbox{a}}$dvorn$\!\acute{\mbox{\ptb{\i}}}$kov$\acute{\mbox{a}}$,~O.} 2020. The use of English, Czech and French 
punctuation marks in reference, parallel and comparable web corpora: A~question 
of methodology. \textit{Linguist. Prag.} 30(2):30--50. doi: 
10.14712/18059635.2020.1.2.
\bibitem{16-nu-1}
\Aue{Seagal, K.\,Ya.} 2014. Punk\-tu\-a\-tsiya kak sred\-st\-vo so\-zda\-niya 
emo\-tsi\-o\-nal'\-no\-go pod\-teks\-ta (na ma\-te\-ri\-ale ras\-ska\-za M.\,A.~Sho\-lo\-kho\-va ``Sud'\-ba 
che\-lo\-ve\-ka'' i~ego pe\-re\-vo\-dov na ang\-liy\-skiy yazyk) [Punctuation as a means of 
revealing the emotional subtext (the case of Mikhail Sholokhov's short story ``The 
Fate of a~Man'' and its translations into English)]. \textit{Izvestiya RAN. Seriya literatury i~yazyka}
 [The Bulletin of the Russian Academy of Sciences: Studies in Literature 
and Language]. 73(6):38--50.
\bibitem{17-nu-1}
\Aue{Bogdanov, K.\,A.} 2022. Punk\-tu\-a\-tsiya kak mo\-tiv: mno\-go\-to\-chie i~ti\-re 
[Punctuation as a~motive: The ellipsis and the dash]. \textit{NLO} [New Literary Observer] 
2(174):241--253. doi: 0.53953/08696365\_2022\_174\_2\_241.
\end{thebibliography}

 }
 }

\end{multicols}

\vspace*{-6pt}

\hfill{\small\textit{Received April 15, 2023}} 

\vspace*{-18pt}
    
    
    \Contr
    
    
    \vspace*{-3pt}
    
    \noindent
    \textbf{Nuriev Vitaly A.} (b.\ 1980)~--- Doctor of Science in philology, leading 
scientist, Institute of Informatics Problems, Federal Research Center ``Computer 
Science and Control'' of the Russian Academy of Sciences, 44-2~Vavilov Str., 
Moscow 119333, Russian Federation; \mbox{nurieff.v@gmail.com}
    
    \vspace*{3pt}
    
    \noindent
    \textbf{Karpov Vladimir I.} (b.\ 1971)~--- Candidate of Science (PhD) in 
philology, leading scientist, Institute of Linguistics of the Russian Academy of 
Sciences, 1~bld.~1 Bolshoy Kislovsky lane, Moscow 125009, Russian Federation; 
scientist, Institute of Informatics Problems, Federal Research Center ``Computer 
Science and Control'' of the Russian Academy of Sciences, 44-2~Vavilov Str., 
Moscow 119333, Russian Federation; \mbox{wi.karpow@gmail.com}
     
      
\label{end\stat}

\renewcommand{\bibname}{\protect\rm Литература}  %14

\def\stat{authorsrus}
{%\hrule\par
%\vskip 7pt % 7pt
\raggedleft\Large \bf%\baselineskip=3.2ex
О\,Б\ \ А\,В\,Т\,О\,Р\,А\,Х \vskip 17pt
    \hrule
    \par
\vskip 21pt plus 8pt minus 4pt }


\def\tit{\ }

\def\aut{\ }

\def\auf{\ }

\def\leftkol{\ } % ENGLISH ABSTRACTS}

\def\rightkol{ОБ АВТОРАХ} %ENGLISH ABSTRACTS}

\titele{\tit}{\aut}{\auf}{\leftkol}{\rightkol}
      
            \label{st\stat}



\vspace*{24pt}

\begin{multicols}{2}




\noindent
\textbf{Архипов Олег Петрович} (р.\ 1948)~---
кандидат технических наук, директор Орловского филиала Института проб\-лем информатики
Российской академии наук
%302025, г.Орел, Московское шоссе, д.137

\vspace*{3pt}

\noindent
\textbf{Бирюкова Татьяна Константиновна} (р.\ 1968)~---
кандидат фи\-зи\-ко-ма\-те\-ма\-ти\-че\-ских наук, старший научный сотрудник Института проб\-лем информатики
Российской академии наук

\vspace*{3pt}

\noindent 
\textbf{Бобков  Сергей Геннадьевич} (р.\ 1955)~---
доктор технических наук,  заведующий отделением На\-уч\-но-ис\-сле\-до\-ва\-тель\-ско\-го 
института системных исследований Российской академии наук
%117218, Москва, Нахимовский просп., 36, к.1 

\vspace*{3pt}

\noindent \textbf{Васильев Николай Семенович} (р.\ 1952)~--- доктор 
фи\-зи\-ко-ма\-те\-ма\-ти\-че\-ских наук, профессор, 
МГТУ им.\ Н.\,Э.~Баумана 
%, Москва 105005, 2-я Бауманская ул., д.~5,

\vspace*{3pt}

\noindent
\textbf{Гершкович Максим Михайлович} (р.\ 1968)~---
старший научный сотрудник Института проб\-лем информатики
Российской академии наук

\vspace*{3pt}

\noindent 
\textbf{Дьяченко Юрий Георгиевич} (р.\ 1958)~--- кандидат технических наук, 
старший научный сотрудник Института проб\-лем информатики
Российской академии наук

\vspace*{3pt}

\noindent 
\textbf{Ерошенко Александр Андреевич} (р.\ 1989)~--- аспирант кафедры 
математической статистики факультета вычисли\-тельной математики и кибернетики 
Московского государственного университета им.\ М.\,В.~Ломоносова
%119991, Москва ГСП-1, Ленинские горы, д.\ 1, стр. 52

\vspace*{3pt}
 
\noindent 
\textbf{Захаров Виктор Николаевич} (р.\ 1948)~--- 
доктор технических наук, доцент, ученый секретарь Института проб\-лем информатики
Российской академии наук

\vspace*{3pt}

\noindent
\textbf{Зейфман Александр Израилевич} (р.\ 1954)~---
доктор фи\-зи\-ко-ма\-те\-ма\-ти\-че\-ских наук, профессор, 
заведующий кафедрой Вологодского государственного университета; 
старший научный сотрудник Института проб\-лем информатики
Российской академии наук; главный научный сотрудник ИСЭРТ Российской академии наук

\vspace*{3pt}

\noindent
\textbf{Зыкин Сергей Владимирович} (р.\ 1959)~--- 
доктор технических наук, профессор, заведующий лабораторией Института математики 
им.\ С.\,Л.~Соболева Сибирского отделения Российской академии наук, Новосибирск 
%630090, пр.\ ак.\ Коптюга, 4 

\vspace*{4pt}

\noindent
\textbf{Киреев Владимир Иванович} (р.\ 1938)~---
доктор фи\-зи\-ко-ма\-те\-ма\-ти\-че\-ских наук, профессор Московского 
государственного горного университета
%Адрес: Россия, 119991, г. Москва, Ленинский проспект, д. 6

%\columnbreak

\vspace*{4pt}

\noindent
\textbf{Козеренко Елена Борисовна} (р.\ 1959)~---
кандидат филологических наук, заведующая лабораторией Института проб\-лем информатики
Российской академии наук

\vspace*{4pt}

\noindent
\textbf{Королев Виктор Юрьевич} (р.\ 1954)~--- доктор
фи\-зи\-ко-ма\-те\-ма\-ти\-че\-ских наук, профессор кафедры математической 
статистики факультета вычисли\-тельной математики и кибернетики 
Московского государственного университета; 
ведущий научный сотрудник Института проб\-лем информатики
Российской академии наук

\vspace*{4pt}

\noindent
\textbf{Коротышева Анна Владимировна} (р.\ 1988)~---
старший преподаватель Вологодского государственного университета

\vspace*{4pt}

\noindent 
\textbf{Кун Де Турк} (р.\ 1981)~--- научный сотрудник 
исследовательской группы SMACS факультета телекоммуникаций и обработки информации
Университета Гента, Бельгия
%В-9000 Гент, Бельгия

\vspace*{4pt}

\noindent
\textbf{Лупенцов Олег Сергеевич} (р.\ 1986)~---
аспирант Омского государственного института сервиса
%Омск 644043, ул.\ Певцова 13

\vspace*{4pt}

\noindent
\textbf{Лучко Олег Николаевич} (р.\ 1961)~---
кандидат педагогических наук, профессор, заведующий кафедрой 
Омского государственного института сервиса
%Омск 644043, ул.\ Певцова 13

\vspace*{4pt}

\noindent
\textbf{Малашенко Юрий Евгеньевич} (р.\ 1946)~---
доктор фи\-зи\-ко-ма\-те\-ма\-ти\-че\-ских наук, заведующий сектором 
Вычислительного центра им.\ А.\,А.~Дородницына Российской академии наук
%Адрес: 119333, Москва, ул. Вавилова, 40,

\vspace*{4pt}

\noindent
\textbf{Маньяков Юрий Анатольевич} (р.\ 1984)~---
кандидат технических наук, научный сотрудник Орловского филиала Института проб\-лем информатики
Российской академии наук
%302025, г.Орел, Московское шоссе, д.137

\vspace*{4pt}

\noindent
\textbf{Маренко Валентина Афанасьевна} (р.\ 1951)~---
кандидат технических наук, доцент, старший научный сотрудник 
Института математики им.\ С.\,Л.~Соболева Сибирского отделения Российской академии наук
%Новосибирск 630090, пр. ак. Коптюга, 4 

\vspace*{3pt}

\noindent 
\textbf{Морозов Евсей Викторович} (р.\ 1947)~--- доктор 
фи\-зи\-ко-ма\-те\-ма\-ти\-че\-ских, профессор, ведущий научный сотрудник 
Института прикладных математических исследований Карельского научного центра Российской
академии наук; 
%%185910 Россия, Республика Карелия, г.\ Петрозаводск, ул.\ Пушкинская, 11
профессор Петрозаводского государственного университета, Петрозаводск
%185910 Россия, Республика Карелия, г.\ Петрозаводск, пр.\ Ленина, 33

%\pagebreak

\vspace*{3pt}

\noindent
\textbf{Назарова Ирина Александровна} (р.\ 1966)~---
кандидат фи\-зи\-ко-ма\-те\-ма\-ти\-че\-ских наук, 
научный сотрудник Вычислительного центра им.\ А.\,А.~Дородницына Российской академии наук 
%Адрес: 119333, Москва, ул. Вавилова, 40

\vspace*{3pt}

\noindent
\textbf{Павлов Игорь Валерианович} (р.\ 1945)~--- 
доктор фи\-зи\-ко-ма\-те\-ма\-ти\-че\-ских наук, профессор МГТУ им.\ Н.\,Э.~Баумана 
%Москва 105005, 2-я Бауманская ул., д.~5 

%\pagebreak

\vspace*{3pt}

\noindent 
\textbf{Потахина Любовь Викторовна} (р.\ 1989)~--- аспирантка
Института прикладных математических исследований Карельского научного центра
Российской академии наук; 
%%185910 Россия, Республика Карелия, г.\ Петрозаводск, ул.\ Пушкинская, 11
инженер Петрозаводского государственного университета, Петрозаводск
%185910 Россия, Республика Карелия, г.\ Петрозаводск, пр.\ Ленина, 33

\vspace*{3pt}

\noindent 
\textbf{Рождественский Юрий Владимирович} (р.\ 1952)~--- 
кандидат технических наук, заведующий сектором Института проб\-лем информатики
Российской академии наук

\vspace*{3pt}

\noindent 
\textbf{Синицын Игорь Николаевич} (р.\ 1940)~--- доктор технических наук,
профессор, заслуженный деятель\linebreak\vspace*{-12pt}

\columnbreak

\noindent
 науки РФ, заведующий отделом Института проб\-лем информатики
Российской академии наук

\vspace*{7pt}


\noindent
\textbf{Сиротинин Денис Олегович} (р.\ 1984)~---
кандидат технических наук, научный сотрудник Орловского филиала Института проб\-лем информатики
Российской академии наук
%302025, г.Орел, Московское шоссе, д.137

\vspace*{7pt}

%\columnbreak

\noindent 
\textbf{Соколов  Игорь Анатольевич} (р.\ 1954)~--- академик (действительный член) Российской 
академии наук, доктор технических наук, директор Института проб\-лем информатики
Российской академии наук

\vspace*{7pt}

\noindent
\textbf{Степченков Юрий Афанасьевич} (р.\ 1951)~---
кандидат технических наук, заведующий отделом Института проб\-лем информатики
Российской академии наук

\vspace*{7pt}

\noindent
\textbf{Сурков Алексей Викторович} (р.\ 1978)~--- 
старший научный сотрудник На\-уч\-но-ис\-сле\-до\-ва\-тель\-ско\-го 
института системных исследований Российской академии наук
%117218, Москва, Нахимовский просп., 36, к.1 

\vspace*{7pt}

\noindent 
\textbf{Шестаков Олег Владимирович} (р.\ 1976)~--- доктор 
фи\-зи\-ко-ма\-те\-ма\-ти\-че\-ских, доцент кафедры математической статистики 
факультета вычисли\-тельной математики и кибернетики Московского 
государственного университета им.\ М.\,В.~Ломоносова; 
%119991, Москва ГСП-1, Ленинские горы, д.\ 1, стр. 52
старший научный сотрудник Института проб\-лем информатики
Российской академии наук
%, Москва 119333, ул. Вавилова, д.~44, корп.~2

\vspace*{7pt}

\noindent 
\textbf{Шоргин Сергей Яковлевич} (р.\ 1952.)~--- доктор
фи\-зи\-ко-ма\-те\-ма\-ти\-че\-ских наук, профессор, заместитель директора Института 
проб\-лем информатики Российской академии наук





%%%%%%%%%%%%%%%%%%%%%%%%%%%%%%%%%%%%%%%%%%%%%%%%%%%%%%%%%%%%%%%%%%%%%%%%%%%%%%%




%\def\rightkol{ОБ АВТОРАХ}
%\def\leftkol{ОБ АВТОРАХ}

 \label{end\stat}





%\def\leftfootline{\small{\textbf{\thepage}
%\hfill ИНФОРМАТИКА И ЕЁ ПРИМЕНЕНИЯ\ \ \ том~7\ \ \ выпуск~1\ \ \ 2013}
%}%
% \def\rightfootline{\small{ИНФОРМАТИКА И ЕЁ ПРИМЕНЕНИЯ\ \ \ том~7\ \ \ выпуск~1\ \ \ 2013
%\hfill \textbf{\thepage}}}


%\thispagestyle{myheadings}



\end{multicols}

\newpage  

%\def\stat{cont}
{%\hrule\par
%\vskip 7pt % 7pt
\raggedleft\Large \bf%\baselineskip=3.2ex
А\,В\,Т\,О\,Р\,С\,К\,И\,Й\ \ У\,К\,А\,З\,А\,Т\,Е\,Л\,Ь\ \ З\,А\ \ 2\,0\,0\,7 г. \vskip 17pt
    \hrule
    \par
\vskip 21pt plus 6pt minus 3pt }

\label{st\stat}

\def\tit{\ }

\def\aut{\ }
\def\auf{\ }

\def\leftkol{\ } % ENGLISH ABSTRACTS}

\def\rightkol{\ } %ENGLISH ABSTRACTS}

\titele{\tit}{\aut}{\auf}{\leftkol}{\rightkol}


\contentsline {chapter}{\ }{Выпуск \quad Стр.} 
\contentsline {section}{\textbf{Батракова Д.\,А., Королев В.\,Ю., Шоргин С.\,Я.}\ \ Новый метод вероятностно-ста\-ти\-сти\-че\-ско\-го анализа информационных потоков в\nobreakspace {}телекоммуникационных сетях}{\qquad 1 \qquad 40} 
\contentsline {section}{\textbf{Борисов А.\,В.}\ \ Байесовское оценивание в системах наблюдения с\nobreakspace {}марковскими скачкообразными процессами: игровой подход}{\qquad 2 \qquad 65}
\contentsline {section}{\textbf{Босов А.\,В., Иванов А.\,В.}\ \ Программная инфраструктура информационного Web-пор\-тала}{\qquad 2 \qquad 50}
\contentsline {section}{\textbf{Захаров В.\,Н., Калиниченко Л.\,А., Соколов И.\,А., Ступников С.\,А.}\ \ Конструирование канонических информационных моделей для интегрированных информационных систем}{\qquad 2 \qquad 15}
\contentsline {section}{\textbf{Захаров В.\,Н., Козмидиади В.\,А.}\ \ Средства обеспечения отказоустойчивости при\-ло\-жений}{\qquad 1 \qquad 14} 
\contentsline {section}{\textbf{Иванов А.\,В.}\ \ см. Босов А.\,В.\hfill\hfill\hfill\hfill\hfill\hfill\hfill\hfill\hfill\hfill\hfill\hfill\hfill\hfill\hfill\hfill\hfill\hfill\hfill\hfill\hfill\hfill\hfill\hfill\hfill\hfill\hfill\hfill\hfill\hfill\hfill\hfill\hfill\hfill\hfill}{\ }
\contentsline {section}{\textbf{Ильин В.\,Д., Соколов И.\,А.}\ \ Символьная модель системы знаний информатики в\nobreakspace {}че\-ло\-ве\-ко-автоматной среде}{\qquad 1 \qquad 66} 
\contentsline {section}{\textbf{Калиниченко Л.\,А.}\ \ см. Захаров В.\,Н.\hfill\hfill\hfill\hfill\hfill\hfill\hfill\hfill\hfill\hfill\hfill\hfill\hfill\hfill\hfill\hfill\hfill\hfill\hfill\hfill\hfill\hfill\hfill\hfill\hfill\hfill\hfill\hfill\hfill\hfill\hfill\hfill\hfill\hfill\hfill}{\ }
\contentsline {section}{\textbf{Козеренко Е.\,Б.}\ \ Лингвистическое моделирование для систем машинного перевода и обработки знаний}{\qquad 1 \qquad 54} 
\contentsline {section}{\textbf{Козмидиади В.\,А.}\ \ см. Захаров В.\,Н.\hfill\hfill\hfill\hfill\hfill\hfill\hfill\hfill\hfill\hfill\hfill\hfill\hfill\hfill\hfill\hfill\hfill\hfill\hfill\hfill\hfill\hfill\hfill\hfill\hfill\hfill\hfill\hfill\hfill\hfill\hfill\hfill\hfill\hfill\hfill }{\ } 
\contentsline {section}{\textbf{Королев В.\,Ю.}\ \ см. Батракова Д.\,А.\hfill\hfill\hfill\hfill\hfill\hfill\hfill\hfill\hfill\hfill\hfill\hfill\hfill\hfill\hfill\hfill\hfill\hfill\hfill\hfill\hfill\hfill\hfill\hfill\hfill\hfill\hfill\hfill\hfill\hfill\hfill\hfill\hfill\hfill\hfill}{\ } 
\contentsline {section}{\textbf{Кудрявцев А.\,А., Шоргин С.\,Я.}\ \ Байесовский подход к\nobreakspace {}анализу систем массового обслуживания и\nobreakspace {}показателей надежности}{\qquad 2 \qquad 76}
\contentsline {section}{\textbf{Печинкин А.\,В., Соколов И.\,А., Чаплыгин В.\,В.}\ \ Многолинейная система массового обслуживания с конечным накопителем и ненадежными приборами}{\qquad 1 \qquad 27} 
\contentsline {section}{\textbf{Печинкин А.\,В., Соколов И.\,А., Чаплыгин В.\,В.}\ \ Стационарные характеристики многолинейной\nobreakspace {}системы массового обслуживания с\nobreakspace {}одновременными отказами приборов}{\qquad 2 \qquad 39}
\contentsline {section}{\textbf{Синицын И.\,Н.}\ \ Корреляционные методы построения аналитических информационных моделей флуктуаций полюса Земли по априорным данным}{\qquad 2 \qquad \hphantom{9}2}
\contentsline {section}{\textbf{Синицын И.\,Н.}\ \ Развитие теории фильтров Пугачева для оперативной обработки информации в стохастических системах}{{\qquad 1 \qquad \hphantom{9}3}} 
\contentsline {section}{\textbf{Соколов И.\,А.}\ \ см. Захаров В.\,Н.\hfill\hfill\hfill\hfill\hfill\hfill\hfill\hfill\hfill\hfill\hfill\hfill\hfill\hfill\hfill\hfill\hfill\hfill\hfill\hfill\hfill\hfill\hfill\hfill\hfill\hfill\hfill\hfill\hfill\hfill\hfill\hfill\hfill\hfill\hfill}{\ }
\contentsline {section}{\textbf{Соколов И.\,А.}\ \ см. Ильин В.\,Д.\hfill\hfill\hfill\hfill\hfill\hfill\hfill\hfill\hfill\hfill\hfill\hfill\hfill\hfill\hfill\hfill\hfill\hfill\hfill\hfill\hfill\hfill\hfill\hfill\hfill\hfill\hfill\hfill\hfill\hfill\hfill\hfill\hfill\hfill\hfill}{\ } 
\contentsline {section}{\textbf{Соколов И.\,А.}\ \ см. Печинкин А.\,В.\hfill\hfill\hfill\hfill\hfill\hfill\hfill\hfill\hfill\hfill\hfill\hfill\hfill\hfill\hfill\hfill\hfill\hfill\hfill\hfill\hfill\hfill\hfill\hfill\hfill\hfill\hfill\hfill\hfill\hfill\hfill\hfill\hfill\hfill\hfill}{\ } 
\contentsline {section}{\textbf{Соколов И.\,А.}\ \ см. Печинкин А.\,В.\hfill\hfill\hfill\hfill\hfill\hfill\hfill\hfill\hfill\hfill\hfill\hfill\hfill\hfill\hfill\hfill\hfill\hfill\hfill\hfill\hfill\hfill\hfill\hfill\hfill\hfill\hfill\hfill\hfill\hfill\hfill\hfill\hfill\hfill\hfill}{\ }
\contentsline {section}{\textbf{Ступников С.\,А.}\ \ см. Захаров В.\,Н.\hfill\hfill\hfill\hfill\hfill\hfill\hfill\hfill\hfill\hfill\hfill\hfill\hfill\hfill\hfill\hfill\hfill\hfill\hfill\hfill\hfill\hfill\hfill\hfill\hfill\hfill\hfill\hfill\hfill\hfill\hfill\hfill\hfill\hfill\hfill}{\ }
\contentsline {section}{\textbf{Чаплыгин В.\,В.}\ \ см. Печинкин А.\,В.\hfill\hfill\hfill\hfill\hfill\hfill\hfill\hfill\hfill\hfill\hfill\hfill\hfill\hfill\hfill\hfill\hfill\hfill\hfill\hfill\hfill\hfill\hfill\hfill\hfill\hfill\hfill\hfill\hfill\hfill\hfill\hfill\hfill\hfill\hfill}{\ } 
\contentsline {section}{\textbf{Чаплыгин В.\,В.}\ \ см. Печинкин А.\,В.\hfill\hfill\hfill\hfill\hfill\hfill\hfill\hfill\hfill\hfill\hfill\hfill\hfill\hfill\hfill\hfill\hfill\hfill\hfill\hfill\hfill\hfill\hfill\hfill\hfill\hfill\hfill\hfill\hfill\hfill\hfill\hfill\hfill\hfill\hfill}{\ }
\contentsline {section}{\textbf{Шоргин С.\,Я.}\ \ см. Батракова Д.\,А.\hfill\hfill\hfill\hfill\hfill\hfill\hfill\hfill\hfill\hfill\hfill\hfill\hfill\hfill\hfill\hfill\hfill\hfill\hfill\hfill\hfill\hfill\hfill\hfill\hfill\hfill\hfill\hfill\hfill\hfill\hfill\hfill\hfill\hfill\hfill}{\ } 
\contentsline {section}{\textbf{Шоргин С.\,Я.}\ \ см. Кудрявцев А.\,А.\hfill\hfill\hfill\hfill\hfill\hfill\hfill\hfill\hfill\hfill\hfill\hfill\hfill\hfill\hfill\hfill\hfill\hfill\hfill\hfill\hfill\hfill\hfill\hfill\hfill\hfill\hfill\hfill\hfill\hfill\hfill\hfill\hfill\hfill\hfill}{\ }
%\thispagestyle{myheadings}
\def\leftfootline{\small{\textbf{\thepage}
\hfill ИНФОРМАТИКА И ЕЁ ПРИМЕНЕНИЯ\ \ \ том~1\ \ \ выпуск~2\ \ \ 2007}
}%
 \def\rightfootline{\small{ИНФОРМАТИКА И ЕЁ ПРИМЕНЕНИЯ\ \ \ том~1\ \ \ выпуск~2\ \ \ 2007
 \hfill \textbf{\thepage}}}
 \label{end\stat} 
                     
%\def\stat{cont-e}
{%\hrule\par
%\vskip 7pt % 7pt
\raggedleft\Large \bf%\baselineskip=3.2ex
2\,0\,0\,7\ \ A\,U\,T\,H\,O\,R\ \ I\,N\,D\,E\,X \vskip 17pt
    \hrule
    \par
\vskip 21pt plus 6pt minus 3pt }

\label{st\stat}

\def\tit{\ }

\def\aut{\ }
\def\auf{\ }

\def\leftkol{\ } % ENGLISH ABSTRACTS}

\def\rightkol{\ } %ENGLISH ABSTRACTS}

\titele{\tit}{\aut}{\auf}{\leftkol}{\rightkol}


\contentsline {chapter}{\ }{Issue \quad Page} 
\contentsline {subsection}{\textbf{Batrakova D.\,A., Korolev V.\,Yu., Shorgin S.\,Ya.}\ \ A New Method for the Probabilistic and Statistical Analysis of Information Flows in Telecommunication Networks}{\qquad 1 \qquad 40} 
\contentsline {subsection}{\textbf{Borisov A.\,V.}\ \ Bayesian Estimation in\nobreakspace {}Observation Systems with\nobreakspace {}Markov Jump Processes: Game-Theoretic Approach}{\qquad 2 \qquad 65} 
\contentsline {subsection}{\textbf{Bosov A.\,V., Ivanov A.\,V.}\ \ Linguistic Simulation for Machine Translation and Knowledge Management Systems}{\qquad 2 \qquad 50} 
\contentsline {subsection}{\textbf{Chaplygin V.\,V.} see Pechinkin A.\,V.\hfill\hfill\hfill\hfill\hfill\hfill\hfill\hfill\hfill\hfill\hfill\hfill\hfill\hfill\hfill\hfill\hfill\hfill\hfill\hfill\hfill\hfill\hfill\hfill\hfill\hfill\hfill\hfill\hfill\hfill\hfill\hfill\hfill\hfill\hfill}{\ }
\contentsline {subsection}{\textbf{Chaplygin V.\,V.} see Pechinkin A.\,V.\hfill\hfill\hfill\hfill\hfill\hfill\hfill\hfill\hfill\hfill\hfill\hfill\hfill\hfill\hfill\hfill\hfill\hfill\hfill\hfill\hfill\hfill\hfill\hfill\hfill\hfill\hfill\hfill\hfill\hfill\hfill\hfill\hfill\hfill\hfill}{\ }
\contentsline {subsection}{\textbf{Ilyin V.\,D., Sokolov I.\,A.}\ \ The Symbol Model of Informatics Knowledge System in Human-Automaton Environment}{\qquad 1 \qquad 66} 
\contentsline {subsection}{\textbf{Ivanov A.\,V.} see Bosov A.\,V.\hfill\hfill\hfill\hfill\hfill\hfill\hfill\hfill\hfill\hfill\hfill\hfill\hfill\hfill\hfill\hfill\hfill\hfill\hfill\hfill\hfill\hfill\hfill\hfill\hfill\hfill\hfill\hfill\hfill\hfill\hfill\hfill\hfill\hfill\hfill}{\ }
\contentsline {subsection}{\textbf{Kalinichenko L.\,A.} see Zakharov V.\,N.\hfill\hfill\hfill\hfill\hfill\hfill\hfill\hfill\hfill\hfill\hfill\hfill\hfill\hfill\hfill\hfill\hfill\hfill\hfill\hfill\hfill\hfill\hfill\hfill\hfill\hfill\hfill\hfill\hfill\hfill\hfill\hfill\hfill\hfill\hfill}{\ }
\contentsline {subsection}{\textbf{Korolev V.\,Yu.} see Batrakova D.\,A.\hfill\hfill\hfill\hfill\hfill\hfill\hfill\hfill\hfill\hfill\hfill\hfill\hfill\hfill\hfill\hfill\hfill\hfill\hfill\hfill\hfill\hfill\hfill\hfill\hfill\hfill\hfill\hfill\hfill\hfill\hfill\hfill\hfill\hfill\hfill}{\ }
\contentsline {subsection}{\textbf{Kozerenko E.\,B.}\ \ Linguistic Simulation for Machine Translation and Knowledge Management Systems}{\qquad 1 \qquad 54} 
\contentsline {subsection}{\textbf{Kozmidiady V.\,A.} see Zakharov V.\,N.\hfill\hfill\hfill\hfill\hfill\hfill\hfill\hfill\hfill\hfill\hfill\hfill\hfill\hfill\hfill\hfill\hfill\hfill\hfill\hfill\hfill\hfill\hfill\hfill\hfill\hfill\hfill\hfill\hfill\hfill\hfill\hfill\hfill\hfill\hfill}{\ }
\contentsline {subsection}{\textbf{Kudryavtsev A.\,A., Shorgin S.\,Ya.}\ \ Bayesian Approach to Queueing Systems and Reliability Characteristics}{\qquad 2 \qquad 76} 
\contentsline {subsection}{\textbf{Pechinkin A.\,V., Sokolov I.\,A., Chaplygin V.\,V.}\ \ Multichannel Queuing System with Finite Buffer and Unreliable Servers}{\qquad 1 \qquad 27} 
\contentsline {subsection}{\textbf{Pechinkin A.\,V., Sokolov I.\,A., Chaplygin V.\,V.}\ \ Stationary Characteristics of a Multichannel Queueing System with\nobreakspace {}Simultaneous Refusals of Servers}{\qquad 2 \qquad 39} 
\contentsline {subsection}{\textbf{Shorgin S.\,Ya.} see Batrakova D.\,A.\hfill\hfill\hfill\hfill\hfill\hfill\hfill\hfill\hfill\hfill\hfill\hfill\hfill\hfill\hfill\hfill\hfill\hfill\hfill\hfill\hfill\hfill\hfill\hfill\hfill\hfill\hfill\hfill\hfill\hfill\hfill\hfill\hfill\hfill\hfill}{\ }
\contentsline {subsection}{\textbf{Shorgin S.\,Ya.} see Kudryavtsev A.\,A.\hfill\hfill\hfill\hfill\hfill\hfill\hfill\hfill\hfill\hfill\hfill\hfill\hfill\hfill\hfill\hfill\hfill\hfill\hfill\hfill\hfill\hfill\hfill\hfill\hfill\hfill\hfill\hfill\hfill\hfill\hfill\hfill\hfill\hfill\hfill}{\ }
\contentsline {subsection}{\textbf{Sinitsyn I.\,N.}\ \ Correlational Methods for Analytical Informational Models of the Earth Pole Fluctuations Design Based on a priori Data}{\qquad 2 \qquad \hphantom{9}2}
\contentsline {subsection}{\textbf{Sinitsyn I.\,N.}\ \ Development of Pugachev Filtering for Stochastic Systems}{\qquad 1 \qquad \hphantom{9}3}
\contentsline {subsection}{\textbf{Sokolov I.\,A.} see Ilyin V.\,D.\hfill\hfill\hfill\hfill\hfill\hfill\hfill\hfill\hfill\hfill\hfill\hfill\hfill\hfill\hfill\hfill\hfill\hfill\hfill\hfill\hfill\hfill\hfill\hfill\hfill\hfill\hfill\hfill\hfill\hfill\hfill\hfill\hfill\hfill\hfill}{\ }
\contentsline {subsection}{\textbf{Sokolov I.\,A.} see Pechinkin A.\,V.\hfill\hfill\hfill\hfill\hfill\hfill\hfill\hfill\hfill\hfill\hfill\hfill\hfill\hfill\hfill\hfill\hfill\hfill\hfill\hfill\hfill\hfill\hfill\hfill\hfill\hfill\hfill\hfill\hfill\hfill\hfill\hfill\hfill\hfill\hfill}{\ }
\contentsline {subsection}{\textbf{Sokolov I.\,A.} see Pechinkin A.\,V.\hfill\hfill\hfill\hfill\hfill\hfill\hfill\hfill\hfill\hfill\hfill\hfill\hfill\hfill\hfill\hfill\hfill\hfill\hfill\hfill\hfill\hfill\hfill\hfill\hfill\hfill\hfill\hfill\hfill\hfill\hfill\hfill\hfill\hfill\hfill}{\ }
\contentsline {subsection}{\textbf{Sokolov I.\,A.} see Zakharov V.\,N.\hfill\hfill\hfill\hfill\hfill\hfill\hfill\hfill\hfill\hfill\hfill\hfill\hfill\hfill\hfill\hfill\hfill\hfill\hfill\hfill\hfill\hfill\hfill\hfill\hfill\hfill\hfill\hfill\hfill\hfill\hfill\hfill\hfill\hfill\hfill}{\ }
\contentsline {subsection}{\textbf{Stupnikov S.\,A.} see Zakharov V.\,N.\hfill\hfill\hfill\hfill\hfill\hfill\hfill\hfill\hfill\hfill\hfill\hfill\hfill\hfill\hfill\hfill\hfill\hfill\hfill\hfill\hfill\hfill\hfill\hfill\hfill\hfill\hfill\hfill\hfill\hfill\hfill\hfill\hfill\hfill\hfill}{\ }
\contentsline {subsection}{\textbf{Zakharov V.\,N., Kalinichenko L.\,A., Sokolov I.\,A., Stupnikov S.\,A.}\ \ Development of Canonical Information Models for Integrated Information Systems}{\qquad 2 \qquad 15} 
\contentsline {subsection}{\textbf{Zakharov V.\,N., Kozmidiady V.\,A.}\ \ Means Providing Applications Fault Tolerance}{\qquad 1 \qquad 14} 
\def\leftfootline{\small{\textbf{\thepage}
\hfill ИНФОРМАТИКА И ЕЁ ПРИМЕНЕНИЯ\ \ \ том~1\ \ \ выпуск~2\ \ \ 2007}
}%
 \def\rightfootline{\small{ИНФОРМАТИКА И ЕЁ ПРИМЕНЕНИЯ\ \ \ том~1\ \ \ выпуск~2\ \ \ 2007
 \hfill \textbf{\thepage}}}
 \label{end\stat} 


%\end{document}

%
\def\stat{rekl}
%\label{preobr}

%\def\tit{АКАДЕМИК ПУГАЧЁВ  ВЛАДИМИР СЕМЁНОВИЧ\\
%25.03.1911--25.03.1998}


%   \vspace*{-48pt}
%   \begin{center}\LARGE
%Академик Пугачёв  Владимир Семёнович\\ (25.03.1911--25.03.1998)
%   \end{center}

   %\vspace*{2.5mm}

   \begin{center}

{\prgsh\LARGE
ЮБИЛЕИ}

\end{center}
%\hrule

\vspace*{6pt}


   \vspace*{8mm}

   \thispagestyle{empty}


%\def\stat{emel}


\section*{К 70-летию заместителя директора ИПИ РАН,\\ члена редколлегии журнала
<<Информатика и её применения>>\\ доктора технических наук В.\,И.~Будзко}

\vspace*{18pt}




          \begin{multicols}{2}

%            \label{st\stat}

\begin{center}
\vspace*{1pt}
\mbox{%
\epsfxsize=78mm
\epsfbox{bud-1.eps}
}
\end{center}

\vspace*{12pt}

      14 августа 2014~г.\ исполнилось 70~лет за\-мес\-ти\-те\-лю директора ИПИ РАН по
научной работе доктору технических наук Владимиру Игоревичу Будзко.

      Владимир Игоревич Будзко родился в г.~Москве. Высшее образование получил на факультете
элект\-рон\-но-вы\-чис\-ли\-тель\-ных устройств в Московском
ин\-же\-нер\-но-фи\-зи\-че\-ском институте
(МИФИ), который он окончил в 1968~г., после чего был на\-прав\-лен для прохождения
службы в одну из войс\-ко\-вых частей, где прошел путь от инженера до первого заместителя
командира войсковой части.

      С приходом В.\,И.~Будзко в ИПИ РАН (2001~г.)\ в институте
сформировалось новое научное на\-прав\-ле\-ние теоретических исследований~--- <<Постро\-ение
ин\-фор\-ма\-ци\-он\-но-те\-ле\-ком\-му\-ни\-ка\-ци\-он\-ных\linebreak сис\-тем
высокой до\-ступ\-ности>>. В~рамках этого
направления выполнен широкий круг фундаментальных исследований по поиску подходов и
определению принципов построения средств обеспечения доступности, конфиденциальности
и целостности современных крупномасштабных
ин\-фор\-ма\-ци\-он\-но-те\-ле\-ком\-му\-ни\-ка\-ци\-он\-ных
сис\-тем (ИТС). Разработаны основные сис\-тем\-но-тех\-ни\-че\-ские принципы и базовые
архитектурные решения построения перспективных для условий России ИТС с
централизованной обработкой и хранением информации, сочетающих в себе свойства
высокой доступности, отказо- и катастрофоустойчивости, информационной защищенности.
Определены принципы, методы и математические основы рационального построения и
оптимизации средств восстановления функционирования центров обработки данных (ЦОД)
после возникновения отказов и катастроф, передачи и хранения данных, обеспечения
информационной безопасности при достижении минимальной совокупной стоимости
владения такими системами. Результаты нашли практическое воплощение при реализации
проектов в интересах ряда отечественных государственных и негосударственных
организаций, таких как Банк России (БР), Внешторгбанк, ОАО <<ГМК <<Норильский Никель>>,
<<Газпром>>, Минэкономразвития России, Правительство Москвы, а также ряд силовых
ведомств.

      Под руководством В.\,И.~Будзко начиная с 2001~г.\ выполнен комплекс
      на\-уч\-но-ис\-сле\-до\-ва\-тель\-ских и
      опыт\-но-кон\-ст\-рук\-тор\-ских работ (свыше 100~проектов),
направленных на развитие электронной информационной технологии БР.
Разработаны концепции развития ИТС БР сначала до 2008~г., а затем до 2013~г., которые
были приняты в качестве основы проведения технической политики. За реализацию проекта
<<Катастрофоустойчивая тер\-ри\-то\-ри\-аль\-но-рас\-пре\-де\-лен\-ная
      ин\-фор\-ма\-ци\-он\-но-те\-ле\-ком\-му\-ни\-ка\-ци\-он\-ная сис\-те\-ма централизованной
обработки банковской информации>> В.\,И.~Будзко удостоен Премии Правительства РФ в
области науки и техники за 2010~г.

      В.\,И.~Будзко возглавлял и возглавляет работы по ряду других прикладных проектов,
связанных с созданием, совершенствованием и развитием крупномасштабных ИТС.

      В.\,И.~Будзко~--- генерал-майор, доктор технических наук, член-кор\-рес\-пон\-дент
Академии криптографии РФ, известный ученый в области информатики и применения
информационных технологий при построении территориально распределенных ИТС
различного назначения. Является автором свыше 250~научных работ, опубликованных в
на\-уч\-но-тех\-ни\-че\-ских и специальных изданиях.

    \thispagestyle{empty}

      В.\,И.~Будзко уделяет большое внимание подготовке научных кадров. Под его
руководством защищено 6~диссертаций на соискание ученой степени кандидата
технических наук. Свыше 30~лет он читает лекции в ИКСИ Академии ФСБ, профессор
кафедры НИЯУ МИФИ. Является членом двух диссертационных советов, главным
редактором журнала <<Системы высокой доступности>> и членом редколлегии журнала
<<Информатика и её применения>>.

      \bigskip

      Редакционный совет и Редакционная коллегия журнала <<Информатика и её
применения>> сердечно поздравляют Владимира Игоревича Будзко с 70-ле\-ти\-ем и желают
крепкого здоровья и новых научных достижений.

\end{multicols}


%
\def\stat{cont}
{%\hrule\par
%\vskip 7pt % 7pt
\raggedleft\Large \bf%\baselineskip=3.2ex
А\,В\,Т\,О\,Р\,С\,К\,И\,Й\ \ У\,К\,А\,З\,А\,Т\,Е\,Л\,Ь\ \ З\,А\ \ 2\,0\,2\,3 г. \vskip 17pt
 \hrule
 \par
\vskip 21pt plus 6pt minus 3pt }

\label{st\stat}

\def\tit{\ }

\def\aut{\ }
\def\auf{\ }

\def\leftkol{\ } % ENGLISH ABSTRACTS}

\def\rightkol{\ } %АВТОРСКИЙ УКАЗАТЕЛЬ ЗА 2021 г.} %ENGLISH ABSTRACTS}

\titele{\tit}{\aut}{\auf}{\leftkol}{\rightkol}
\addcontentsline{toc}{subsection}{\textrm\textbf Авторский указатель за 2023 г.}

\vspace*{-24pt}

\noindent
{\tabcolsep=3pt
\begin{tabular}{p{397pt}cc}
&\textbf{Вып.} & \textbf{Стр.}\\[6pt]
\Avtors{Агаларов~Я.\,М.} Об оптимизации работы резервного прибора в~многолинейной 
системе массового обслуживания&\raisebox{-12pt}[0pt][0pt]{1}&\raisebox{-12pt}[0pt][0pt]{89--95}\\
\Avtors{Агаларов~Я.\,М.} Оптимизация схемы распределения буферной памяти узла 
пакетной коммутации&\raisebox{-12pt}[0pt][0pt]{3}&\raisebox{-12pt}[0pt][0pt]{39--48}\\
\Avtors{Агасандян~Г.\,А.} Многомерные баттерфляи в~задачах оптимизации по CC-VaR&1&107--115\\
\Avtors{Аду~К.\,И.\,Б., Маркова~Е.\,В., Гайдамака~Ю.\,В., Шоргин~С.\,Я.} Анализ схемы 
доступа с~прерыванием при нарезке радиоресурсов сети пятого 
поколения&\raisebox{-12pt}[0pt][0pt]{1}&\raisebox{-12pt}[0pt][0pt]{\hphantom{1}96--106}\\
\Avtors{Архипов~П.\,О., Филиппских~С.\,Л., Цуканов~М.\,В.} Разработка новой модели 
ступенчатой сверточной нейронной сети для классификации аномалий на панорамах&\raisebox{-12pt}[0pt][0pt]{1}&\raisebox{-12pt}[0pt][0pt]{50--56}\\
\Avtors{Бегишев~В.\,О.} см.\ Сопин~Э.\,С.&&\\
\Avtors{Берговин~А.\,К., Ушаков~В.\,Г.} Исследование систем обслуживания со 
смешанными приоритетами&\raisebox{-12pt}[0pt][0pt]{2}&\raisebox{-12pt}[0pt][0pt]{57--61}\\
\Avtors{Борисов~А.\,В.} Рынок с~марковской скачкообразной волатильностью 
I:~мониторинг цены риска как задача оптимальной фильтрации&\raisebox{-12pt}[0pt][0pt]{2}&\raisebox{-12pt}[0pt][0pt]{27--33}\\
\Avtors{Борисов~А.\,В.} Рынок с~марковской скачкообразной волатильностью~II: алгоритм 
вы\-чис\-ле\-ния справедливой цены деривативов&\raisebox{-12pt}[0pt][0pt]{3}&\raisebox{-12pt}[0pt][0pt]{18--24}\\
\Avtors{Борисов А.\,В.} Рынок с марковской скачкообразной волатильностью III:  алгоритм 
мониторинга цены риска по дискретным наблюдениям цен активов&\raisebox{-12pt}[0pt][0pt]{4}&\raisebox{-12pt}[0pt][0pt]{\hphantom{9}9--16}\\
\Avtors{Босов~А.\,В.} Исследование робастности численных аппроксимаций фильтра 
Вонэма&2&41--49\\
\Avtors{Босов~А.\,В.} Оптимальная фильтрация состояния нелинейной динамической 
системы по наблюдениям со случайными запаздываниями&\raisebox{-12pt}[0pt][0pt]{3}&\raisebox{-12pt}[0pt][0pt]{\hphantom{1}8--17}\\
\Avtors{Босов~А.\,В., Иванов~А.\,В.} Технология многофакторной классификации 
математического контента электронной системы обучения&\raisebox{-12pt}[0pt][0pt]{4}&\raisebox{-12pt}[0pt][0pt]{32--41}\\
\Avtors{Босов~А.\,В., Игнатов~А.\,Н.} О~задаче оценки и~анализа риска транспортных 
происшествий на рельсовом транспорте&\raisebox{-12pt}[0pt][0pt]{1}&\raisebox{-12pt}[0pt][0pt]{73--82}\\
\Avtors{Вакуленко~В.\,В., Зацман~И.\,М.} Формализованное описание статистической 
обработки информации в~базах данных&\raisebox{-12pt}[0pt][0pt]{3}&\raisebox{-12pt}[0pt][0pt]{93--99}\\
\Avtors{Васильев~Н.\,С.} Композициональное представление структуры игры многих лиц 
в~моноидальной категории бинарных отношений&\raisebox{-12pt}[0pt][0pt]{2}&\raisebox{-12pt}[0pt][0pt]{18--26}\\
\Avtors{Волканов~Д.\,Ю.} см.\ Горшенин~А.\,К.&&\\
\Avtors{Воронцов~М.\,О., Шестаков~О.\,В.} Среднеквадратичный риск FDR-процедуры 
в~условиях слабой зависимости&\raisebox{-12pt}[0pt][0pt]{2}&\raisebox{-12pt}[0pt][0pt]{34--40}\\
\Avtors{Гайдамака~Ю.\,В.} см.\ Аду~К.\,И.\,Б.&&\\
\Avtors{Гайдамака~Ю.\,В.} см.\ Иванова Д.\,В.&&\\
\Avtors{Гайдамака~Ю.\,В.} см.\ Самуйлов~А.\,К.&&\\
\Avtors{Гаримелла~Р.\,М.} см.\ Разумчик~Р.\,В.&&\\
\Avtors{Гончаров~А.\,А.} Аннотирование параллельных корпусов: подходы и направления 
развития&4&81--87\\
\Avtors{Горбунов~С.\,А.} см.\ Горшенин~А.\,К.&&\\
\Avtors{Горшенин~А.\,К., Горбунов~С.\,А., Волканов~Д.\,Ю.} О~кластеризации объектов 
сетевой вы\-чис\-ли\-тель\-ной инфраструктуры на основе анализа статистических аномалий 
в~трафике&\raisebox{-12pt}[0pt][0pt]{3}&\raisebox{-12pt}[0pt][0pt]{76--87}\\
\Avtors{Грушо~А.\,А., Грушо~Н.\,А., Забежайло~М.\,И., Кульченков~В.\,В., 
Тимонина~Е.\,Е., Шоргин~С.\,Я.} Причинно-следственные связи в~задачах 
классификации&\raisebox{-12pt}[0pt][0pt]{1}&\raisebox{-12pt}[0pt][0pt]{43--49}\\
\Avtors{Грушо~А.\,А., Грушо~Н.\,А., Забежайло~М.\,И., Смирнов~Д.\,В., Тимонина~Е.\,Е.} 
Классификация с~помощью причинно-следственных связей&\raisebox{-12pt}[0pt][0pt]{3}&\raisebox{-12pt}[0pt][0pt]{71--75}\\
\Avtors{Грушо~А.\,А., Грушо~Н.\,А., Забежайло~М.\,И., Тимонина~Е.\,Е., Шоргин~С.\,Я.} 
Сложные причинно-следственные связи&\raisebox{-12pt}[0pt][0pt]{2}&\raisebox{-12pt}[0pt][0pt]{84--89}\\
\end{tabular}
}

\pagebreak

\def\leftkol{АВТОРСКИЙ УКАЗАТЕЛЬ ЗА 2023 г.} % ENGLISH ABSTRACTS}

\def\rightkol{АВТОРСКИЙ УКАЗАТЕЛЬ ЗА 2023 г.} %ENGLISH ABSTRACTS}

%\thispagestyle{myheadings}
\def\leftfootline{\small{\textbf{\thepage}
\hfill ИНФОРМАТИКА И ЕЁ ПРИМЕНЕНИЯ\ \ \ том~17\ \ \ выпуск~4\ \ \ 2023}
}%
 \def\rightfootline{\small{ИНФОРМАТИКА И ЕЁ ПРИМЕНЕНИЯ\ \ \ том~17\ \ \ выпуск~4\ \ \ 2023
 \hfill \textbf{\thepage}}}


\noindent
{\tabcolsep=3pt
\begin{tabular}{p{394pt}cc}
&\textbf{Вып.} & \textbf{Стр.}\\[3pt]
\Avtors{Грушо~Н.\,А.} см.\ Грушо~А.\,А.&&\\
\Avtors{Грушо~Н.\,А.} см.\ Грушо~А.\,А.&&\\
\Avtors{Грушо~Н.\,А.} см.\ Грушо~А.\,А.&&\\
\Avtors{Дулин~С.\,К.} см.\ Розенберг~И.\,Н.&&\\
\Avtors{Дулина~Н.\,Г.} см.\ Розенберг~И.\,Н.&&\\
\Avtors{Дюкова~А.\,П.} см.\ Дюкова~Е.\,В.&&\\
\Avtors{Дюкова~Е.\,В., Масляков~Г.\,О., Дюкова~А.\,П.} Логические методы корректной 
классификации данных&\raisebox{-12pt}[0pt][0pt]{3}&\raisebox{-12pt}[0pt][0pt]{64--70}\\
\Avtors{Забежайло~М.\,И.} см.\ Грушо~А.\,А&&\\
\Avtors{Забежайло~М.\,И.} см.\ Грушо~А.\,А.&&\\
\Avtors{Забежайло~М.\,И.} см.\ Грушо~А.\,А.&&\\
\Avtors{Захаров~В.\,Н.} см.\ Сазонтьев В.\,В.&&\\
\Avtors{Захаров В.\,Н.} см.\ Френкель С.\,Л.&&\\
\Avtors{Зацман~И.\,М.} Данные, информация и~знание в~научной парадигме 
информатики&1&116--125\\
\Avtors{Зацман И.\,М.} Научная парадигма информатики: классификация объектов 
предметной области&\raisebox{-12pt}[0pt][0pt]{4}&\raisebox{-12pt}[0pt][0pt]{\hphantom{9}96--103}\\
\Avtors{Зацман~И.\,М.} Трансформация иерархии Акоффа в~научной парадигме 
информатики&3&107--113\\
\Avtors{Зацман~И.\,М.} см.\ Вакуленко~В.\,В.&&\\
\Avtors{Зейфман~А.\,И.} см.\ Усов~И.\,А.&&\\
\Avtors{Иванов~А.\,В.} см.\ Босов~А.\,В.&&\\
\Avtors{Иванова Д.\,В., Маркова Е.\,В., Шоргин~С.\,Я., Гайдамака~Ю.\,В.} Модели 
совместного обслуживания трафика eMBB и URLLC на основе приоритетов в 
промышленных развертываниях 5G NR&\raisebox{-24pt}[0pt][0pt]{4}&\raisebox{-24pt}[0pt][0pt]{64--70}\\
\Avtors{Игнатов~А.\,Н.} см.\ Босов~А.\,В.&&\\
\Avtors{Инькова~О.\,Ю., Кружков~М.\,Г.} Критерии определения семантической близости 
дискурсивных отношений&\raisebox{-12pt}[0pt][0pt]{3}&\raisebox{-12pt}[0pt][0pt]{100--106}\\
\Avtors{Инькова О.\,Ю., Кружков~М.\,Г.} Степень семантической близости дискурсивных 
отношений:  методы и инструменты расчета&\raisebox{-12pt}[0pt][0pt]{4}&\raisebox{-12pt}[0pt][0pt]{88--95}\\
\Avtors{Кабанов~Ю.\,М., Сидоренко~А.\,П.} Аксиоматический взгляд на модели системного 
риска Роджерса--Вераарт и~Судзуки--Эльсингера&\raisebox{-12pt}[0pt][0pt]{1}&\raisebox{-12pt}[0pt][0pt]{11--17}\\
\Avtors{Карпов~В.\,И.} см.\ Нуриев~В.\,А.&&\\
\Avtors{Кириков~И.\,А.} см.\ Листопад~С.\,В.&&\\
\Avtors{Ковалёв~С.\,П.} Монада диаграмм как математическая метамодель системной 
инженерии&2&11--17\\
\Avtors{Королев~Д.\,О., Малеев~О.\,Г.} Исследование эффективности применения бинарных 
нейронных сетей при детектировании объекта на изображении&\raisebox{-12pt}[0pt][0pt]{3}&\raisebox{-12pt}[0pt][0pt]{88--92}\\
\Avtors{Кривенко~М.\,П.} Критерии выбора размерности модели факторизации&2&50--56\\
\Avtors{Кружков~М.\,Г.} см.\ Инькова О.\,Ю.&&\\
\Avtors{Кружков~М.\,Г.} см.\ Инькова~О.\,Ю.&&\\
\Avtors{Кудрявцев~А.\,А., Шестаков~О.\,В.} Метод оценивания параметров 
гамма-экс\-по\-нен\-ци\-аль\-но\-го распределения по выборке со слабо зависимыми компонентами&\raisebox{-12pt}[0pt][0pt]{3}&\raisebox{-12pt}[0pt][0pt]{58--62}\\
\Avtors{Кульченков~В.\,В.} см.\ Грушо~А.\,А.&&\\
\Avtors{Лапко~А.\,В.} см.\ Тубольцев~В.\,П.&&\\
\Avtors{Лапко~В.\,А.} см.\ Тубольцев~В.\,П.&&\\
\Avtors{Лери~М.\,М.} Среднее расстояние в~конфигурационных графах со степенным 
распределением&\raisebox{-12pt}[0pt][0pt]{1}&\raisebox{-12pt}[0pt][0pt]{28--34}\\
\Avtors{Листопад~С.\,В., Кириков~И.\,А.} Метод на основе нечетких правил для 
управления конфликтами агентов в~гибридных интеллектуальных многоагентных 
системах&\raisebox{-12pt}[0pt][0pt]{1}&\raisebox{-12pt}[0pt][0pt]{66--72}\\
\Avtors{Малашенко~Ю.\,Е., Назарова~И.\,А.} Анализ загрузки многопользовательской сети 
при расщеплении потоков по кратчайшим маршрутам&\raisebox{-12pt}[0pt][0pt]{3}&\raisebox{-12pt}[0pt][0pt]{33--38}\\
\Avtors{Малашенко~Ю.\,Е., Назарова~И.\,А.} Оценки распределения ресурсов 
в~многопользовательской сети при равных межузловых нагрузках&\raisebox{-12pt}[0pt][0pt]{1}&\raisebox{-12pt}[0pt][0pt]{83--88}\\
\Avtors{Малеев~О.\,Г.} см.\ Королев~Д.\,О.&&\\
\Avtors{Маркова~Е.\,В.} см.\ Аду~К.\,И.\,Б.&&\\
\Avtors{Маркова Е.\,В.} см.\ Иванова Д.\,В.&&\\
\end{tabular}
}

\pagebreak

\def\leftkol{АВТОРСКИЙ УКАЗАТЕЛЬ ЗА 2023 г.} % ENGLISH ABSTRACTS}

\def\rightkol{АВТОРСКИЙ УКАЗАТЕЛЬ ЗА 2023 г.} %ENGLISH ABSTRACTS}

%\thispagestyle{myheadings}
\def\leftfootline{\small{\textbf{\thepage}
\hfill ИНФОРМАТИКА И ЕЁ ПРИМЕНЕНИЯ\ \ \ том~17\ \ \ выпуск~4\ \ \ 2023}
}%
 \def\rightfootline{\small{ИНФОРМАТИКА И ЕЁ ПРИМЕНЕНИЯ\ \ \ том~17\ \ \ выпуск~4\ \ \ 2023
 \hfill \textbf{\thepage}}}


\noindent
{\tabcolsep=3pt
\begin{tabular}{p{394pt}cc}
&\textbf{Вып.} & \textbf{Стр.}\\[3pt]
\Avtors{Маслов~А.\,Р.} см.\ Сопин~Э.\,С&&\\
\Avtors{Масляков~Г.\,О.} см.\ Дюкова~Е.\,В.&&\\
\Avtors{Мелехин~В.\,Б., Хачумов~В.\,М., Хачумов~М.\,В.} Самообучение автономных 
интеллектуальных роботов в~процессе поисково-исследовательской деятельности&\raisebox{-12pt}[0pt][0pt]{2}&\raisebox{-12pt}[0pt][0pt]{78--83}\\
\Avtors{Назарова~И.\,А.} см.\ Малашенко~Ю.\,Е.&&\\
\Avtors{Назарова~И.\,А.} см.\ Малашенко~Ю.\,Е.&&\\
\Avtors{Нейчев~Р.\,Г., Шибаев~И.\,А., Стрижов~В.\,В.} Восстановление матрицы 
суперпозиции в~задаче символьной регрессии&\raisebox{-12pt}[0pt][0pt]{1}&\raisebox{-12pt}[0pt][0pt]{35--42}\\
\Avtors{Нуриев~В.\,А., Карпов~В.\,И.} Методология корпусно-ориентированного 
исследования в~области контрастивной пунктуации&\raisebox{-12pt}[0pt][0pt]{2}&\raisebox{-12pt}[0pt][0pt]{90--95}\\
\Avtors{Пешкова И.\,В.} Границы незавершенной работы в системе с повторными вызовами 
разных классов и показательным временем обслуживания&\raisebox{-12pt}[0pt][0pt]{4}&\raisebox{-12pt}[0pt][0pt]{57--63}\\
\Avtors{Платонова~А.\,А.} см.\ Самуйлов~А.\,К.&&\\
\Avtors{Рабинович Я.\,И.} Процедура построения множества Парето для дифференцируемых 
критериальных функций&\raisebox{-12pt}[0pt][0pt]{4}&\raisebox{-12pt}[0pt][0pt]{17--22}\\
\Avtors{Разумчик~Р.\,В., Румянцев~А.\,С., Гаримелла~Р.\,М.} Вероятностная модель для 
оценки основных характеристик производительности марковской модели 
суперкомпьютера&\raisebox{-24pt}[0pt][0pt]{2}&\raisebox{-24pt}[0pt][0pt]{62--70}\\
\Avtors{Розенберг~И.\,Н., Дулин~С.\,К., Дулина~Н.\,Г.} Моделирование структуры 
интероперабельности средствами структурной согласованности&\raisebox{-12pt}[0pt][0pt]{1}&\raisebox{-12pt}[0pt][0pt]{57--65}\\
\Avtors{Румовская~С.\,Б.} Подходы к~подбору специалистов при организации 
коллективного решения проблем&\raisebox{-12pt}[0pt][0pt]{2}&\raisebox{-12pt}[0pt][0pt]{\hphantom{1}96--103}\\
\Avtors{Румянцев~А.\,С.} см.\ Разумчик~Р.\,В.&&\\
\Avtors{Сазонтьев В.\,В., Ступников~С.\,А., Захаров~В.\,Н.} Расширяемый подход к слиянию 
данных в распределенных вычислительных средах&\raisebox{-12pt}[0pt][0pt]{4}&\raisebox{-12pt}[0pt][0pt]{42--47}\\
\Avtors{Самуйлов~А.\,К., Платонова~А.\,А., Шоргин~В.\,С., Гайдамака~Ю.\,В.} 
К~моделированию эффектов обслуживания многоадресного трафика в~сетях 5G~NR&\raisebox{-12pt}[0pt][0pt]{2}&\raisebox{-12pt}[0pt][0pt]{71--77}\\
\Avtors{Сатин~Я.\,А.} см.\ Усов~И.\,А.&&\\
\Avtors{Сидоренко~А.\,П.} см.\ Кабанов~Ю.\,М.&&\\
\Avtors{Синицын~И.\,Н.} Аналитическое моделирование распределений с~инвариантной 
мерой в~стохастических системах, не разрешенных относительно 
производных&\raisebox{-12pt}[0pt][0pt]{1}&\raisebox{-12pt}[0pt][0pt]{\hphantom{1}2--10}\\
\Avtors{Смирнов~Д.\,В.} см.\ Грушо~А.\,А.&&\\
\Avtors{Сопин~Э.\,С., Маслов~А.\,Р., Шоргин~В.\,С., Бегишев~В.\,О.} Моделирование 
настойчивого поведения пользователей в~сетях 5G NR с~адаптацией скорости 
и~блокировками&\raisebox{-12pt}[0pt][0pt]{3}&\raisebox{-12pt}[0pt][0pt]{25--32}\\
\Avtors{Степанов~Е.\,П.} см.\ Шестаков~О.\,В.&&\\
\Avtors{Стрижов~В.\,В.} см.\ Нейчев~Р.\,Г.&&\\
\Avtors{Ступников~С.\,А.} см.\ Сазонтьев В.\,В.&&\\
\Avtors{Тимонина~Е.\,Е.} см.\ Грушо~А.\,А.&&\\
\Avtors{Тимонина~Е.\,Е.} см.\ Грушо~А.\,А.&&\\
\Avtors{Тимонина~Е.\,Е.} см.\ Грушо~А.\,А.&&\\
\Avtors{Торшин~И.\,Ю.} О~задачах оптимизации, возникающих при применении 
топологического анализа данных к~поиску алгоритмов прогнозирования с~фиксированными 
корректорами&\raisebox{-24pt}[0pt][0pt]{2}&\raisebox{-24pt}[0pt][0pt]{\hphantom{1}2--10}\\
\Avtors{Торшин~И.\,Ю.} О~формировании множеств прецедентов на основе таблиц 
разнородных признаковых описаний методами топологической теории анализа 
данных&\raisebox{-12pt}[0pt][0pt]{3}&\raisebox{-12pt}[0pt][0pt]{2--7}\\
\Avtors{Тубольцев~В.\,П., Лапко~А.\,В., Лапко~В.\,А.} Непараметрический алгоритм 
автоматической классификации данных дистанционного зондирования&\raisebox{-12pt}[0pt][0pt]{4}&\raisebox{-12pt}[0pt][0pt]{23--31}\\
\Avtors{Усов~И.\,А., Сатин~Я.\,А., Зейфман~А.\,И.} О~скорости сходимости и~предельных 
характеристиках для одного обобщенного процесса рождения и~гибели&\raisebox{-12pt}[0pt][0pt]{3}&\raisebox{-12pt}[0pt][0pt]{49--57}\\
\Avtors{Ушаков~В.\,Г., Ушаков~Н.\,Г.} Критерии нормальности вероятностного 
распределения при округленных данных&\raisebox{-12pt}[0pt][0pt]{1}&\raisebox{-12pt}[0pt][0pt]{18--27}\\
\Avtors{Ушаков~В.\,Г.} см.\ Берговин~А.\,К.&&\\
\Avtors{Ушаков~Н.\,Г.} см.\ Ушаков~В.\,Г.&&\\
\Avtors{Филиппских~С.\,Л.} см.\ Архипов~П.\,О.&&\\
\end{tabular}
}

\pagebreak

\def\leftkol{АВТОРСКИЙ УКАЗАТЕЛЬ ЗА 2023 г.} % ENGLISH ABSTRACTS}

\def\rightkol{АВТОРСКИЙ УКАЗАТЕЛЬ ЗА 2023 г.} %ENGLISH ABSTRACTS}

%\thispagestyle{myheadings}
\def\leftfootline{\small{\textbf{\thepage}
\hfill ИНФОРМАТИКА И ЕЁ ПРИМЕНЕНИЯ\ \ \ том~17\ \ \ выпуск~4\ \ \ 2023}
}%
 \def\rightfootline{\small{ИНФОРМАТИКА И ЕЁ ПРИМЕНЕНИЯ\ \ \ том~17\ \ \ выпуск~4\ \ \ 2023
 \hfill \textbf{\thepage}}}


\noindent
{\tabcolsep=3pt
\begin{tabular}{p{394pt}cc}
&\textbf{Вып.} & \textbf{Стр.}\\[3pt]
\Avtors{Френкель С.\,Л., Захаров В.\,Н.} Модели учета влияния статистических 
характеристик трафика вычислительных сетей на эффективность прогнозирования 
средствами машинного обучения&\raisebox{-24pt}[0pt][0pt]{4}&\raisebox{-24pt}[0pt][0pt]{71--80}\\
\Avtors{Хачумов~В.\,М.} см.\ Мелехин~В.\,Б.&&\\
\Avtors{Хачумов~М.\,В.} см.\ Мелехин~В.\,Б.&&\\
\Avtors{Цуканов~М.\,В.} см.\ Архипов~П.\,О.&&\\
\Avtors{Шестаков~О.\,В., Степанов~Е.\,П.} Нелинейная регуляризация обращения линейных 
однородных операторов с помощью метода блочной пороговой обработки&\raisebox{-12pt}[0pt][0pt]{4}&\raisebox{-12pt}[0pt][0pt]{2--8}\\
\Avtors{Шестаков~О.\,В.} см.\ Воронцов~М.\,О.&&\\
\Avtors{Шестаков~О.\,В.} см.\ Кудрявцев~А.\,А.&&\\
\Avtors{Шибаев~И.\,А.} см.\ Нейчев~Р.\,Г.&&\\
\Avtors{Шнурков П.\,В.} Решение задачи оптимального управления запасом непрерывного 
продукта в стохастической модели регенерации со случайными стоимостными 
характеристиками&\raisebox{-24pt}[0pt][0pt]{4}&\raisebox{-24pt}[0pt][0pt]{48--56}\\
\Avtors{Шоргин~В.\,С.} см.\ Самуйлов~А.\,К.&&\\
\Avtors{Шоргин~В.\,С.} см.\ Сопин~Э.\,С.&&\\
\Avtors{Шоргин~С.\,Я.} см.\ Аду~К.\,И.\,Б.&&\\
\Avtors{Шоргин~С.\,Я.} см.\ Грушо~А.\,А.&&\\
\Avtors{Шоргин~С.\,Я.} см.\ Грушо~А.\,А.&&\\
\Avtors{Шоргин~С.\,Я.} см.\ Иванова Д.\,В.&&\\


\end{tabular}
}

%\thispagestyle{myheadings}
\def\leftfootline{\small{\textbf{\thepage}
\hfill ИНФОРМАТИКА И ЕЁ ПРИМЕНЕНИЯ\ \ \ том~17\ \ \ выпуск~4\ \ \ 2023}
}%
 \def\rightfootline{\small{ИНФОРМАТИКА И ЕЁ ПРИМЕНЕНИЯ\ \ \ том~17\ \ \ выпуск~4\ \ \ 2023
 \hfill \textbf{\thepage}}}

 \label{end\stat}

\newpage

\def\stat{cont-e}
{%\hrule\par
%\vskip 7pt % 7pt
\raggedleft\Large \bf%\baselineskip=3.2ex
2\,0\,2\,3\ \ A\,U\,T\,H\,O\,R\ \ I\,N\,D\,E\,X \vskip 17pt
 \hrule
 \par
\vskip 21pt plus 6pt minus 3pt }

\label{st\stat}

\def\tit{\ }

\def\aut{\ }
\def\auf{\ }

\def\leftkol{\ } %2021 AUTHOR INDEX} % ENGLISH ABSTRACTS}

\def\rightkol{\ } %2021 AUTHOR INDEX} %ENGLISH ABSTRACTS}

\titele{\tit}{\aut}{\auf}{\leftkol}{\rightkol}
\addcontentsline{toc}{subsection}{\textrm\textbf 2023 Author Index}

\def\leftfootline{\small{\textbf{\thepage}
\hfill INFORMATIKA I EE PRIMENENIYA~--- INFORMATICS AND APPLICATIONS\ \ \ 2023\
\ \ volume~17\ \ \ issue\ 4}
}%
 \def\rightfootline{\small{INFORMATIKA I EE PRIMENENIYA~--- INFORMATICS AND APPLICATIONS\ \ \ 2023\ \ \ volume~17\ \ \ issue\ 4
\hfill \textbf{\thepage}}}

\vspace*{-24pt}

\noindent
{\tabcolsep=3pt
\begin{tabular}{p{395.89pt}cc}
&\textbf{Issue} & \textbf{Page}\\[6pt]
\Avtors{Adou~K.\,Y.\,B., Markova~E.\,V., Gaidamaka~Yu.\,V., and~Shorgin~S.\,Ya.} 
Preemption-based prioritization scheme for network resources slicing in 5G 
systems&\raisebox{-12pt}[0pt][0pt]{1}&\raisebox{-12pt}[0pt][0pt]{\hphantom{1}96--106}\\
\Avtors{Agalarov~Ya.\,M.} Optimization of a queue-length dependent additional server in the 
multiserver queue&\raisebox{-12pt}[0pt][0pt]{1}&\raisebox{-12pt}[0pt][0pt]{89--95}\\
\Avtors{Agalarov~Ya.\,M.} Optimization of the buffer memory allocation scheme of the packet 
switching node&\raisebox{-12pt}[0pt][0pt]{3}&\raisebox{-12pt}[0pt][0pt]{39--48}\\
\Avtors{Agasandyan~G.\,A.} Multidimensional butterflies in problems of optimization on CC-VaR&1&107--115\\
\Avtors{Arkhipov~P.\,O., Philippskih~S.\,L., and~Tsukanov~M.\,V.} Development of a~new model 
of a~step convolutional neural network for classification of anomalies on panoramas&\raisebox{-12pt}[0pt][0pt]{1}&\raisebox{-12pt}[0pt][0pt]{50--56}\\
\Avtors{Begishev~V.\,O.} see Sopin~E.\,S.&&\\
\Avtors{Bergovin~A.\,K. and~Ushakov~V.\,G.} Analysis of the queueing systems with mixed 
priorities&2&57--61\\
\Avtors{Borisov~A.\,V.} Market with Markov jump volatility I: Price of risk monitoring as an 
optimal filtering problem&\raisebox{-12pt}[0pt][0pt]{2}&\raisebox{-12pt}[0pt][0pt]{27--33}\\
\Avtors{Borisov~A.\,V.} Market with Markov jump volatility~II: Algorithm of derivative fair price 
calculation&3&18--24\\
\Avtors{Borisov A.\,V.} Market with Markov jump volatility III: Price of risk monitoring algorithm 
given discrete-time observations of asset prices&\raisebox{-12pt}[0pt][0pt]{4}&\raisebox{-12pt}[0pt][0pt]{\hphantom{9}9--16}\\
\Avtors{Bosov~A.\,V.} Nonlinear dynamic system state optimal filtering by observations with 
random delays&\raisebox{-12pt}[0pt][0pt]{3}&\raisebox{-12pt}[0pt][0pt]{\hphantom{1}8--17}\\
\Avtors{Bosov~A.\,V.} Robustness investigation of the numerical approximation of the Wonham 
filter&2&41--49\\
\Avtors{Bosov~A.\,V. and~Ignatov~A.\,N.} On the problem of assessing and analyzing traffic 
accidents risk on the rail transport&\raisebox{-12pt}[0pt][0pt]{1}&\raisebox{-12pt}[0pt][0pt]{73--82}\\
\Avtors{Bosov~A.\,V. and Ivanov~A.\,V.} Multifactor classification technology of mathematical 
content of e-learning system&\raisebox{-12pt}[0pt][0pt]{4}&\raisebox{-12pt}[0pt][0pt]{32--41}\\
\Avtors{Djukova~A.\,P.} see Djukova~E.\,V.&&\\
\Avtors{Djukova~E.\,V., Masliakov~G.\,O., and Djukova~A.\,P.} Logical methods of correct data 
classification&3&64--70\\
\Avtors{Dulin~S.\,K.} see Rozenberg~I.\,N.&&\\
\Avtors{Dulina~N.\,G.} see Rozenberg~I.\,N.&&\\
\Avtors{Frenkel~S.\,L. and Zakharov~V.\,N.} Models for study of the influence of statistical 
characteristics of computer networks traffic on the efficiency of prediction by machine learning 
tools&\raisebox{-12pt}[0pt][0pt]{4}&\raisebox{-12pt}[0pt][0pt]{71--80}\\
\Avtors{Gaidamaka~Yu.\,V.} see Adou~K.\,Y.\,B.&&\\
\Avtors{Gaidamaka~Yu.\,V.} see Ivanova~D.\,V.&&\\
\Avtors{Gaidamaka~Yu.\,V.} see Samouylov~A.\,K.&&\\
\Avtors{Garimella~R.\,M.} see Razumchik~R.\,V.&&\\
\Avtors{Goncharov~A.\,A.} Parallel corpus annotation: Approaches and directions for 
development&4&81--87\\
\Avtors{Gorbunov~S.\,A.} see Gorshenin~A.\,K.&&\\
\Avtors{Gorshenin~A.\,K., Gorbunov~S.\,A., and Volkanov~D.\,Yu.} Toward clustering of 
network computing infrastructure objects based on analysis of statistical anomalies in network 
traffic&\raisebox{-12pt}[0pt][0pt]{3}&\raisebox{-12pt}[0pt][0pt]{76--87}\\
\Avtors{Grusho~A.\,A., Grusho~N.\,A., Zabezhailo~M.\,I., Kulchenkov~V.\,V., Timonina~E.\,E., 
and~Shorgin~S.\,Ya.} Causal relationships in classification problems&\raisebox{-12pt}[0pt][0pt]{1}&\raisebox{-12pt}[0pt][0pt]{43--49}\\
\Avtors{Grusho~A.\,A., Grusho~N.\,A., Zabezhailo~M.\,I., Smirnov~D.\,V., and Timonina~E.\,E.} 
Classification by cause-and-effect relationships&\raisebox{-12pt}[0pt][0pt]{3}&\raisebox{-12pt}[0pt][0pt]{71--75}\\
\Avtors{Grusho~A.\,A., Grusho~N.\,A., Zabezhailo~M.\,I., Timonina~E.\,E., 
and~Shorgin~S.\,Ya.} Complex cause-and-effect relationships&\raisebox{-12pt}[0pt][0pt]{2}&\raisebox{-12pt}[0pt][0pt]{84--89}\\
\Avtors{Grusho~N.\,A.} see Grusho~A.\,A.&&\\
\Avtors{Grusho~N.\,A.} see Grusho~A.\,A.&&\\
\Avtors{Grusho~N.\,A.} see Grusho~A.\,A.&&\\
\end{tabular}
}
\pagebreak

\def\leftfootline{\small{\textbf{\thepage}
\hfill INFORMATIKA I EE PRIMENENIYA~--- INFORMATICS AND APPLICATIONS\ \ \ 2023\
\ \ volume~17\ \ \ issue\ 4}
}%
 \def\rightfootline{\small{INFORMATIKA I EE PRIMENENIYA~---
INFORMATICS AND APPLICATIONS\ \ \ 2023\ \ \ volume~17\ \ \ issue\ 4
\hfill \textbf{\thepage}}}

\def\leftkol{2023 AUTHOR INDEX} % ENGLISH ABSTRACTS}

\def\rightkol{2023 AUTHOR INDEX} %ENGLISH ABSTRACTS}


\noindent
{\tabcolsep=3pt
\begin{tabular}{p{395.5pt}cc}
&\textbf{Issue} & \textbf{Page}\\[6pt]
\Avtors{Ignatov~A.\,N.} see Bosov~A.\,V.&&\\
\Avtors{Inkova O.\,Yu. and Kruzhkov M.\,G.} Evaluating the degree of discourse relations 
semantic affinity: Methods and instruments&\raisebox{-12pt}[0pt][0pt]{4}&\raisebox{-12pt}[0pt][0pt]{88--95}\\
\Avtors{Inkova~O.\,Yu. and Kruzhkov~M.\,G.} Evaluation criteria for discourse relations semantic 
affinity&3&100--106\\
\Avtors{Kruzhkov~M.\,G.} see Inkova~O.\,Yu.&&\\
\Avtors{Ivanov~A.\,V.} see Bosov~A.\,V.&&\\
\Avtors{Ivanova~D.\,V., Markova~E.\,V., Shorgin~S.\,Ya., and~Gaidamaka~Yu.\,V.} Priority-based 
eMBB and URLLC traffic coexistence models in 5G NR industrial deployments&\raisebox{-12pt}[0pt][0pt]{4}&\raisebox{-12pt}[0pt][0pt]{64--70}\\
\Avtors{Kabanov~Yu.\,M. and~Sidorenko~A.\,P.} An axiomatic viewpoint on the Rogers--Veraart 
and Suzuki--Elsinger models of systemic risk&\raisebox{-12pt}[0pt][0pt]{1}&\raisebox{-12pt}[0pt][0pt]{11--17}\\
\Avtors{Karpov~V.\,I.} see Nuriev~V.\,A.&&\\
\Avtors{Khachumov~M.\,V.} see Melekhin~V.\,B.&&\\
\Avtors{Khachumov~V.\,M.} see Melekhin~V.\,B.&&\\
\Avtors{Kirikov~I.\,A.} see Listopad~S.\,V.&&\\
\Avtors{Korolev~D.\,O. and Maleev~O.\,G.} Efficiency of binary neural networks for object 
detection on an image&\raisebox{-12pt}[0pt][0pt]{3}&\raisebox{-12pt}[0pt][0pt]{88--92}\\
\Avtors{Kovalyov~S.\,P.} The monad of diagrams as a mathematical metamodel of systems 
engineering&2&11--17\\
\Avtors{Krivenko~M.\,P.} Criteria for choosing the factorization model dimensionality&2&50--56\\
\Avtors{Kruzhkov M.\,G.} see Inkova O.\,Yu.&&\\
\Avtors{Kudryavtsev~A.\,A. and Shestakov~O.\,V.} A~method for estimating parameters of the 
gamma-exponential distribution from a~sample with weakly dependent components&\raisebox{-12pt}[0pt][0pt]{3}&\raisebox{-12pt}[0pt][0pt]{58--63}\\
\Avtors{Kulchenkov~V.\,V.} see Grusho~A.\,A.&&\\
\Avtors{Lapko~A.\,V.} see Tuboltsev V.\,P.&&\\
\Avtors{Lapko~V.\,A.} see Tuboltsev V.\,P.&&\\
\Avtors{Leri~M.\,M.} An average distance in the power-law configuration graphs&1&28--34\\
\Avtors{Listopad~S.\,V. and~Kirikov~I.\,A.} Fuzzy rules based method for agent conflict 
management in hybrid intelligent multiagent systems&\raisebox{-12pt}[0pt][0pt]{1}&\raisebox{-12pt}[0pt][0pt]{66--72}\\
\Avtors{Malashenko~Yu.\,E. and~Nazarova~I.\,A.} Estimates of the resource distribution in the 
multiuser network with equal internodal loads&\raisebox{-12pt}[0pt][0pt]{1}&\raisebox{-12pt}[0pt][0pt]{83--88}\\
\Avtors{Malashenko~Yu.\,E. and Nazarova~I.\,A.} Multiuser network load analysis by splitting 
flows along the shortest routes&\raisebox{-12pt}[0pt][0pt]{3}&\raisebox{-12pt}[0pt][0pt]{33--38}\\
\Avtors{Maleev~O.\,G.} see Korolev~D.\,O.&&\\
\Avtors{Markova~E.\,V.} see Adou~K.\,Y.\,B.&&\\
\Avtors{Markova~E.\,V.} see Ivanova~D.\,V.&&\\
\Avtors{Masliakov~G.\,O.} see Djukova~E.\,V.&&\\
\Avtors{Maslov~A.\,R.} see Sopin~E.\,S.&&\\
\Avtors{Melekhin~V.\,B., Khachumov~V.\,M., and~Khachumov~M.\,V.} Self-learning of 
autonomous intelligent robots in the process of search and explore activities&\raisebox{-12pt}[0pt][0pt]{2}&\raisebox{-12pt}[0pt][0pt]{78--83}\\
\Avtors{Nazarova~I.\,A.} see Malashenko~Yu.\,E.&&\\
\Avtors{Nazarova~I.\,A.} see Malashenko~Yu.\,E.&&\\
\Avtors{Neychev~R.\,G., Shibaev~I.\,A., and~Strijov~V.\,V.} Optimal spanning tree reconstruction 
in symbolic regression&\raisebox{-12pt}[0pt][0pt]{1}&\raisebox{-12pt}[0pt][0pt]{35--42}\\
\Avtors{Nuriev~V.\,A. and~Karpov~V.\,I.} Methodology of the corpus-based studies in the field of 
contrastive punctuation&\raisebox{-12pt}[0pt][0pt]{2}&\raisebox{-12pt}[0pt][0pt]{90--95}\\
\Avtors{Peshkova~I.\,V.} Bounds of the workload in a~multiclass retrial queue with exponential 
services&\raisebox{-12pt}[0pt][0pt]{4}&\raisebox{-12pt}[0pt][0pt]{57--63}\\
\Avtors{Philippskih~S.\,L.} see Arkhipov~P.\,O.&&\\
\Avtors{Platonova~A.\,A.} see Samouylov~A.\,K.&&\\
\Avtors{Rabinovich Ya.\,I.} Procedure of constructing a~Pareto set for differentiable criteria 
functions&4&17--22\\
\Avtors{Razumchik~R.\,V., Rumyantsev~A.\,S., and~Garimella~R.\,M.} A~queueing system for 
performance evaluation of a~Markovian supercomputer model&\raisebox{-12pt}[0pt][0pt]{2}&\raisebox{-12pt}[0pt][0pt]{62--70}\\
\Avtors{Rozenberg~I.\,N., Dulin~S.\,K., and~Dulina~N.\,G.} Modeling the structure of 
interoperability by means of structural consistency&\raisebox{-12pt}[0pt][0pt]{1}&\raisebox{-12pt}[0pt][0pt]{57--65}\\
\Avtors{Rumovskaya~S.\,B.} Selection of specialists in the organization of collective solving 
problems&2&\hphantom{1}96--103\\
\end{tabular}
}
\pagebreak

\def\leftfootline{\small{\textbf{\thepage}
\hfill INFORMATIKA I EE PRIMENENIYA~--- INFORMATICS AND APPLICATIONS\ \ \ 2023\
\ \ volume~17\ \ \ issue\ 4}
}%
 \def\rightfootline{\small{INFORMATIKA I EE PRIMENENIYA~---
INFORMATICS AND APPLICATIONS\ \ \ 2023\ \ \ volume~17\ \ \ issue\ 4
\hfill \textbf{\thepage}}}

\def\leftkol{2023 AUTHOR INDEX} % ENGLISH ABSTRACTS}

\def\rightkol{2023 AUTHOR INDEX} %ENGLISH ABSTRACTS}


\noindent
{\tabcolsep=3pt
\begin{tabular}{p{395.5pt}cc}
&\textbf{Issue} & \textbf{Page}\\[6pt]
\Avtors{Rumyantsev~A.\,S.} see Razumchik~R.\,V.&&\\
\Avtors{Samouylov~A.\,K., Platonova~A.\,A., Shorgin~V.\,S., and~Gaidamaka~Yu.\,V.} On 
modeling the effects of multicast traffic servicing in 5G NR networks&\raisebox{-12pt}[0pt][0pt]{2}&\raisebox{-12pt}[0pt][0pt]{71--77}\\
\Avtors{Satin~Y.\,A.} see Usov~I.\,A.&&\\
\Avtors{Sazontev V.\,V., Stupnikov~S.\,A., and~Zakharov~V.\,N.} An extensible approach to data 
fusion in~distributed computing environments&\raisebox{-12pt}[0pt][0pt]{4}&\raisebox{-12pt}[0pt][0pt]{42--47}\\
\Avtors{Shestakov~O.\,V. and Stepanov~E.\,P.} Nonlinear regularization of the inversion of linear 
homogeneous operators using the block thresholding method&\raisebox{-12pt}[0pt][0pt]{4}&\raisebox{-12pt}[0pt][0pt]{2--8}\\
\Avtors{Shestakov~O.\,V.} see Kudryavtsev~A.\,A.&&\\
\Avtors{Shestakov~O.\,V.} see Vorontsov~M.\,O.&&\\
\Avtors{Shibaev~I.\,A.} see Neychev~R.\,G.&&\\
\Avtors{Shnurkov P.\,V.} Solution of the problem of~optimal control of~the stock of a~continuous 
product in a~stochastic model of regeneration with random cost characteristics&\raisebox{-12pt}[0pt][0pt]{4}&\raisebox{-12pt}[0pt][0pt]{48--56}\\
\Avtors{Shorgin~S.\,Ya.} see Adou~K.\,Y.\,B.&&\\
\Avtors{Shorgin~S.\,Ya.} see Grusho~A.\,A.&&\\
\Avtors{Shorgin~S.\,Ya.} see Grusho~A.\,A.&&\\
\Avtors{Shorgin~S.\,Ya.} see Ivanova~D.\,V.&&\\
\Avtors{Shorgin~V.\,S.} see Samouylov~A.\,K.&&\\
\Avtors{Shorgin~V.\,S.} see Sopin~E.\,S.&&\\
\Avtors{Sidorenko~A.\,P.} see Kabanov~Yu.\,M.&&\\
\Avtors{Sinitsyn~I.\,N.} Analytical modeling of distributions with invariant measure in stochastic 
systems with unsolved derivatives&\raisebox{-12pt}[0pt][0pt]{1}&\raisebox{-12pt}[0pt][0pt]{\hphantom{1}2--10}\\
\Avtors{Smirnov~D.\,V.} see Grusho~A.\,A.&&\\
\Avtors{Sopin~E.\,S., Maslov~A.\,R., Shorgin~V.\,S., and Begishev~V.\,O.} Modeling insistent 
user behavior in~5G New Radio networks with rate adaptation and blockage&\raisebox{-12pt}[0pt][0pt]{3}&\raisebox{-12pt}[0pt][0pt]{25--32}\\
\Avtors{Stepanov~E.\,P.} see Shestakov~O.\,V.&&\\
\Avtors{Strijov~V.\,V.} see Neychev~R.\,G.&&\\
\Avtors{Stupnikov~S.\,A.} see Sazontev V.\,V.&&\\
\Avtors{Timonina~E.\,E.} see Grusho~A.\,A.&&\\
\Avtors{Timonina~E.\,E.} see Grusho~A.\,A.&&\\
\Avtors{Timonina~E.\,E.} see Grusho~A.\,A.&&\\
\Avtors{Torshin~I.\,Yu.} On optimization problems arising from the application of topological data 
analysis to the search for forecasting algorithms with fixed correctors&\raisebox{-12pt}[0pt][0pt]{2}&\raisebox{-12pt}[0pt][0pt]{\hphantom{1}2--10}\\
\Avtors{Torshin~I.\,Yu.} On the formation of sets of precedents based on tables of heterogeneous 
feature descriptions by methods of topological theory of data analysis&\raisebox{-12pt}[0pt][0pt]{3}&\raisebox{-12pt}[0pt][0pt]{2--7}\\
\Avtors{Tsukanov~M.\,V.} see Arkhipov~P.\,O.&&\\
\Avtors{Tuboltsev V.\,P., Lapko~A.\,V., and~Lapko~V.\,A.} Nonparametric algorithm for 
automatic classification of remote sensing data&\raisebox{-12pt}[0pt][0pt]{4}&\raisebox{-12pt}[0pt][0pt]{23--31}\\
\Avtors{Ushakov~N.\,G.} see Ushakov~V.\,G.&&\\
\Avtors{Ushakov~V.\,G. and~Ushakov~N.\,G.} Tests for normality of the probabilistic distribution 
when data are rounded&\raisebox{-12pt}[0pt][0pt]{1}&\raisebox{-12pt}[0pt][0pt]{18--27}\\
\Avtors{Ushakov~V.\,G.} see Bergovin~A.\,K.&&\\
\Avtors{Usov~I.\,A., Satin~Y.\,A., and Zeifman~A.\,I.} On the rate of convergence and limiting 
characteristics for one quasi-birth--death process&\raisebox{-12pt}[0pt][0pt]{3}&\raisebox{-12pt}[0pt][0pt]{49--57}\\
\Avtors{Vakulenko~V.\,V. and Zatsman~I.\,M.} Formalized description of statistical information 
processing in databases&\raisebox{-12pt}[0pt][0pt]{3}&\raisebox{-12pt}[0pt][0pt]{93--99}\\
\Avtors{Vasilyev~N.\,S.} Multiplayers' games compositional structure in the monoidal category of 
binary relations&\raisebox{-12pt}[0pt][0pt]{2}&\raisebox{-12pt}[0pt][0pt]{18--26}\\
\Avtors{Volkanov~D.\,Yu.} see Gorshenin~A.\,K.&&\\
\Avtors{Vorontsov~M.\,O. and~Shestakov~O.\,V.} Mean-square risk of the FDR procedure under 
weak dependence&\raisebox{-12pt}[0pt][0pt]{2}&\raisebox{-12pt}[0pt][0pt]{34--40}\\
\Avtors{Zabezhailo~M.\,I.} see Grusho~A.\,A.&&\\
\Avtors{Zabezhailo~M.\,I.} see Grusho~A.\,A.&&\\
\end{tabular}
}
\pagebreak

\def\leftfootline{\small{\textbf{\thepage}
\hfill INFORMATIKA I EE PRIMENENIYA~--- INFORMATICS AND APPLICATIONS\ \ \ 2023\
\ \ volume~17\ \ \ issue\ 4}
}%
 \def\rightfootline{\small{INFORMATIKA I EE PRIMENENIYA~---
INFORMATICS AND APPLICATIONS\ \ \ 2023\ \ \ volume~17\ \ \ issue\ 4
\hfill \textbf{\thepage}}}

\def\leftkol{2023 AUTHOR INDEX} % ENGLISH ABSTRACTS}

\def\rightkol{2023 AUTHOR INDEX} %ENGLISH ABSTRACTS}


\noindent
{\tabcolsep=3pt
\begin{tabular}{p{395.5pt}cc}
&\textbf{Issue} & \textbf{Page}\\[6pt]
\Avtors{Zabezhailo~M.\,I.} see Grusho~A.\,A.&&\\
\Avtors{Zakharov~V.\,N.} see Frenkel~S.\,L.&&\\
\Avtors{Zakharov~V.\,N.} see Sazontev V.\,V.&&\\
\Avtors{Zatsman~I.\,M.} On the scientific paradigm of informatics: Data, information, and 
knowledge&1&116--125\\
\Avtors{Zatsman I.\,M.} Scientific paradigm of informatics: Classification of domain 
objects&4&\hphantom{9}96--103\\
\Avtors{Zatsman~I.\,M.} Transformation of the Ackoff's hierarchy in the scientific paradigm of 
informatics&3&107--113\\
\Avtors{Zatsman~I.\,M.} see Vakulenko~V.\,V.&&\\
\Avtors{Zeifman~A.\,I.} see Usov~I.\,A.&&\\
\end{tabular}
}

%\thispagestyle{myheadings}
\def\leftfootline{\small{\textbf{\thepage}
\hfill INFORMATIKA I EE PRIMENENIYA~--- INFORMATICS AND APPLICATIONS\ \ \ 2023\
\ \ volume~17\ \ \ issue\ 4}
}%
 \def\rightfootline{\small{INFORMATIKA I EE PRIMENENIYA~---
INFORMATICS AND APPLICATIONS\ \ \ 2023\ \ \ volume~17\ \ \ issue\ 4
\hfill \textbf{\thepage}}}

 \label{end\stat}

\newpage

%
   \vspace*{-46pt}

\begin{center}
\vspace*{4pt}
\mbox{%

\epsfxsize=55mm %112.705
\epsfbox{zhur-2.eps}
}
%\end{center}

\vspace*{10pt} 


%   \begin{center}
\fbox{\large\textbf{Академик Юрий Иванович Журавлёв}}\\[10pt]
\textbf{\large 14.01.1935--14.01.2022}
   \end{center}


   %\vspace*{2.5mm}

   \vspace*{5mm}

   \thispagestyle{empty}

%\

%\vspace*{-12pt}
       


В январе этого года ушел из жизни главный научный сотрудник Федерального исследовательского 
центра <<Информатика и управление>> РАН, председатель Редакционного совета журнала 
<<Информатика и~её применения>> академик Юрий Иванович Журавлёв. В~его лице мировая 
наука потеряла одного из своих ярчайших представителей~--- выдающегося ученого-исследователя 
и~талантливого ученого-организатора.

Юрий Иванович родился в Воронеже в 1935~г.\ в семье ученого и врача. Среднее образование 
получил в школе №\,6 г.~Фрунзе (ныне Бишкек) Киргизской ССР. В~1952~г.\ поступил на 
ме\-ха\-ни\-ко-ма\-те\-ма\-ти\-че\-ский факультет МГУ им.\ М.\,В.~Ломоносова. В~1957~г.\ Юрий Иванович 
защищает диплом и продолжает обучение в аспирантуре Московского университета на кафедре 
вычислительной математики (возглавляемой тогда академиком С.\,Л.~Соболевым). После 
успешной защиты кандидатской диссертации (к.ф.-м.н., 1959 г., научный руководитель~--- 
А.\,А.~Ляпунов, оппоненты~--- чл.-корр.\ А.\,А.~Марков, к.ф.-м.н.\ О.\,Б.~Лупанов) и~до 
окончательного переезда в Москву в 1969~г.\ работал в Институте математики Сибирского 
отделения АН СССР, занимая в нем последовательно должности младшего научного сотрудника, 
заведующего отделом, заведующего отделением, заместителя директора по научной работе. 
В~этот период (1954--1966~гг.)\ им был опубликован цикл работ по решению задач алгебры и 
математической логики, причем полученные результаты применялись для создания эффективных 
программ для ЭВМ, конструирования схем и сетей для обработки информации. Наиболее значимый 
результат этого периода научной работы~--- обоснование нового направления исследований, 
общей теории локальных алгоритмов. В~ней были окончательно объединены топологические 
принципы и теория алгоритмов. Эта теория и легла в основу докторской диссертации Юрия 
Ивановича (д.ф.-м.н., 1965~г.)\ по еще тогда новой научной специальности <<Математическая 
кибернетика>>. Оппонировали ему как специалисты по кибернетике~--- академик 
В.\,М.~Глушков, член-корреспондент А.\,А.~Ляпунов и О.\,Б.~Лупанов, так и про\-фес\-сор-ал\-геб\-раист А.\,Д.~Тайманов. 

В 1969~г.\ Юрий Иванович переезжает в Москву и возглавляет в Вычислительном центре АН 
СССР лабораторию проблем распознавания. Впоследствии он~--- заместитель директора по 
научной работе. Научные интересы этого периода связаны с проблемами классификации или 
распознавания образов. В~1976--1978~гг.\ Юрий Иванович публикует цикл работ по ставшему 
вскоре знаменитым алгебраическому подходу к проблеме синтеза корректных алгоритмов. Эти 
работы определили современное состояние всей проблематики распознавания и многих смежных 
областей прикладной математики и информатики. В~своих основополагающих работах Юрий 
Иванович показал, что можно в явном виде строить экстремальные по качеству алгоритмы для 
решения очень широких классов плохо формализованных задач. 
{\looseness=-1

}





Научные заслуги Юрия Ивановича получили широкое признание. В~1966~г.\ он совместно с 
О.\,Б.~Лупановым и чле\-ном-кор\-рес\-пон\-ден\-том АН СССР С.\,В.~Яблонским были удостоены 
звания лауреата Ленинской премии в~об\-ласти науки и техники. В~1984~г.\ Юрий Иванович 
был избран членом-корреспондентом АН СССР (по специальности <<Информатика>>), 
а~в~1992~г.~--- академиком РАН (по той же специальности).\linebreak\vspace*{-12pt}

\pagebreak

\

\vspace*{-46pt}

\noindent
\begin{floatingfigure}{48mm}
\begin{center}
%\vspace*{6pt}
\mbox{%

\epsfxsize=46mm %112.705
\epsfbox{zhur-3.eps}
}
\end{center}
\vspace*{6pt}
\end{floatingfigure}

 \thispagestyle{empty}

\noindent
В~1986~г.\ за цикл прикладных 
работ ему и ряду его учеников была при\-суж\-де\-на премия Совета Министров СССР. Он являлся 
членом иностранных академий наук, председателем секции <<Прикладная математика
 и~информатика>> Отделения математических наук РАН, председателем экспертного совета ВАК 
России по управ\-ле\-нию и информатике, заслуженным профессором нескольких университетов, 
председателем Российской ассоциации <<Распознавание образов и обработка изображений>>, 
членом исполкома Международной ассоциации IAPR (распознавание образов и обработка 
изображений). Был награжден 8-ю орденами и медалями СССР и России.

Юрий Иванович проводил большую научно-литературную работу, являясь, в том числе, главным 
редактором международных научных журналов и членом редколлегий ряда рецензируемых 
научных журналов. 


Параллельно с активной научной деятельностью Юрий Иванович вел и преподавательскую 
работу. С~1961 по~1969~гг.~--- в Новосибирском государственном университете на кафедре 
алгебры и математической логики, которую возглавлял в то время академик А.\,И.~Мальцев. 
С~1970~г., будучи уже профессором (1967~г.),~--- в Московском физико-техническом институте 
на кафедре академика Н.\,Н.~Моисеева. В~1997~г.\ по предложению ректора МГУ им.\ 
М.\,В.~Ломоносова академика В.\,А.~Садовничего Юрий Иванович организовал на факультете 
Вычислительной математики и кибернетики новую кафедру <<Математические методы 
прогнозирования>>, которой и руководил до конца жизни. В~2008~г.\ ему была присуждена 
премия Совета Министров РФ в области образования. С~1965~г.\ Юрий Иванович периодически 
читал курсы лекций за рубежом, в университетах США, Франции, Финляндии, Швеции, Австрии, 
Польши, Болгарии, ГДР и других стран. Эта работа в существенной степени обеспечила широкое 
международное признание советской и российской науки в области дискретной математики и~распознавания образов. 

%\begin{floatingfigure}{60mm}
\begin{figure}[b]
\begin{center}
\vspace*{-6pt}
\mbox{%

\epsfxsize=112mm %90mm %112.705
\epsfbox{zhur-1.eps}
}
\end{center}
\end{figure}
%\end{floatingfigure}

Понимая важность вопроса воспитания подрастающего поколения для развития науки в стране, 
Юрий Иванович вскоре после защиты первой диссертации включился в работу по подготовке 
научных кадров. Им создана большая научная школа: под руководством Юрия Ивановича 
защищены более 100~кандидатских диссертаций по всевозможным разделам естествознания 
(математике, информатике, медицине, технике, экономике, геологии), не один десяток докторов 
наук. Он воспитал академиков и членов-корреспондентов РАН и академий государств СНГ. 
С~большим вниманием и участием Юрий Иванович относился к развитию научных школ страны 
в~об\-ласти обработки изображений, распознавания образов и компьютерной оптики. 

Для всех коллег и учеников Юрия Ивановича он останется примером замечательного человека, 
та\-лант\-ли\-во\-го педагога и выдающегося, преданного служению науке ученого. 


%\def\stat{cont}
{%\hrule\par
%\vskip 7pt % 7pt
\raggedleft\Large \bf%\baselineskip=3.2ex
А\,В\,Т\,О\,Р\,С\,К\,И\,Й\ \ У\,К\,А\,З\,А\,Т\,Е\,Л\,Ь\ \ З\,А\ \ 2\,0\,1\,0 г. \vskip 17pt
    \hrule
    \par
\vskip 21pt plus 6pt minus 3pt }

\label{st\stat}

\def\tit{\ }

\def\aut{\ }
\def\auf{\ }

\def\leftkol{\ } % ENGLISH ABSTRACTS}

\def\rightkol{\ } %АВТОРСКИЙ УКАЗАТЕЛЬ ЗА 2010 г.} %ENGLISH ABSTRACTS}

\titele{\tit}{\aut}{\auf}{\leftkol}{\rightkol}

\vspace*{-12pt}

{\tabcolsep=3pt
\begin{tabular}{p{388pt}rr}
&\textbf{Выпуск} & \textbf{Стр.}\\[6pt]
\hangindent=23pt\noindent\textbf{Арутюнян~А.\,Р.} Моделирование влияния деформаций отпечатков пальцев на 
точность\linebreak
\vspace*{-12pt}\\
\hspace*{23pt}дактилоскопической идентификации$\dotfill$&1&51\\
\hangindent=23pt\noindent\textbf{Архипов~О.\,П., Зыкова~З.\,П.} Интеграция гетерогенной информации о цветных 
пикселях\linebreak
\vspace*{-12pt}\\
\hspace*{23pt}и их цветовосприятии$\dotfill$&4&15\\
\hangindent=23pt\noindent\textbf{Баранов~С.\,И., Френкель~С.\,Л., Захаров~В.\,Н.} Полуформальная верификация 
цифрового устройства с конвейером, основанная на использовании алгоритмических машин\linebreak
\vspace*{-12pt}\\
\hspace*{23pt}состояния$\dotfill$&4&49\\
\textbf{Бекетова~И.\,В.} см.~Каратеев~С.\,Л.&&\\
\textbf{Белоусов~В.\,В.} см.~Синицын~И.\,Н.&&\\
\hangindent=23pt\noindent\textbf{Бенинг~В.\,Е., Королев~Р.\,А.} О предельном поведении мощностей критериев в 
случае\linebreak
\vspace*{-12pt}\\
\hspace*{23pt}распределения Лапласа$\dotfill$&2&63\\
\hangindent=23pt\noindent\textbf{Бенинг~В.\,Е., Сипина~А.\,В.} Асимптотическое разложение для мощности 
критерия,\linebreak
\vspace*{-12pt}\\
\hspace*{23pt}основанного на выборочной медиане, в случае распределения Лапласа$\dotfill$&1&18\\
\textbf{Бондаренко~А.\,В.} см.~Каратеев~С.\,Л.&&\\
\hangindent=23pt\noindent\textbf{Бородина~А.\,В., Морозов~Е.\,В.} Об оценивании асимптотики вероятности 
большого\linebreak
\vspace*{-12pt}\\
\hspace*{23pt}уклонения стационарной регенеративной очереди с одним прибором$\dotfill$&3&29\\
\hangindent=23pt\noindent\textbf{Бунтман~Н.\,В., Минель~Ж.-Л., Ле~Пезан~Д., Зацман~И.\,М.} Типология и 
компьютерное\linebreak
\vspace*{-12pt}\\
\hspace*{23pt}моделирование трудностей перевода$\dotfill$&3&77\\
\textbf{Визильтер~Ю.\,В.} см.~Каратеев~С.\,Л.&&\\
\hangindent=23pt\noindent\textbf{Гавриленко~С.\,В.} Оценки скорости сходимости распределений случайных сумм с 
безгранично делимыми индексами к нормальному закону$\dotfill$&4&81\\
\hangindent=23pt\noindent\textbf{Григорьева~М.\,Е., Шевцова~И.\,Г.} Уточнение неравенства 
Каца--Берри--Эссеена$\dotfill$&2&75\\
\hangindent=23pt\noindent\textbf{Грушо~А.\,А., Грушо~Н.\,А., Тимонина~Е.\,Е.} Поиск конфликтов в политиках 
безопасности: модель случайных графов$\dotfill$&3&38\\
\textbf{Грушо~Н.\,А.} см.~Грушо~А.\,А.&&\\
\hangindent=23pt\noindent\textbf{Гудков~В.\,Ю.} Математические модели изображения отпечатка пальца на основе 
описания линий$\dotfill$&1&58\\
\textbf{Гуртов~А.\,В.} см.~Лукьяненко~А.\,С.&&\\
\textbf{Желтов~С.\,Ю.} см.~Каратеев~С.\,Л.&&\\
\hangindent=23pt\noindent\textbf{Захаров~А.\,А., Серебряков~В.\,А.} Система управления электронной библиотекой 
LibMeta$\dotfill$&4&2\\
\textbf{Захаров~В.\,Н.} см.~Баранов~С.\,И.&&\\
\textbf{Захарова~Т.\,В.} см.~Матвеева~С.\,С.&&\\
\hangindent=23pt\noindent\textbf{Зацаринный~А.\,А., Чупраков~К.\,Г.} Некоторые аспекты выбора технологии для 
постро-\linebreak
\vspace*{-12pt}\\
\hspace*{23pt}ения систем отображения информации ситуационного центра$\dotfill$&3&59\\
\textbf{Зацман~И.\,М.} см.~Бунтман~Н.\,В.&&\\
\hangindent=23pt\noindent\textbf{Зейфман~А.\,И., Коротышева~А.\,В., Сатин~Я.\,А., Шоргин~С.\,Я.} Об 
устойчивости нестаци-\linebreak
\vspace*{-12pt}\\
\hspace*{23pt}онарных систем обслуживания с катастрофами$\dotfill$&3&9\\
\textbf{Зыкова~З.\,П.} см.~Архипов~О.\,П.&&\\
\hangindent=23pt\noindent\textbf{Илюшин~Г.\,Я., Соколов~И.\,А.} Организация управляемого доступа пользователей 
к\linebreak
\vspace*{-12pt}\\
\hspace*{23pt}разнородным ведомственным информационным ресурсам$\dotfill$&1&24\\
\hangindent=23pt\noindent\textbf{Кавагучи~Ю., Ульянов~В.\,В., Фуджикоши~Я.} Приближения для статистик, 
описывающих\linebreak
\vspace*{-12pt}\\
\hspace*{23pt}геометрические свойства данных большой размерности, с оценками 
ошибок$\dotfill$&1&12\\
\hangindent=23pt\noindent\textbf{Каратеев~С.\,Л., Бекетова~И.\,В., Ососков~М.\,В., Князь~В.\,А., 
Визильтер~Ю.\,В., Бондаренко~А.\,В., Желтов~С.\,Ю.} Автоматизированный контроль 
качества цифровых\linebreak
\vspace*{-12pt}\\
\hspace*{23pt}изображений для персональных документов$\dotfill$&1&65\\
\end{tabular}
}

\pagebreak

\def\leftkol{АВТОРСКИЙ УКАЗАТЕЛЬ ЗА 2010 г.} % ENGLISH ABSTRACTS}

\def\rightkol{АВТОРСКИЙ УКАЗАТЕЛЬ ЗА 2010 г.} %ENGLISH ABSTRACTS}

{\tabcolsep=3pt
\begin{tabular}{p{388pt}rr}
&\textbf{Выпуск} & \textbf{Стр.}\\[3pt]
\hangindent=23pt\noindent\textbf{Козеренко~Е.\,Б.} Лингвистические фильтры в статистических моделях машинного\linebreak
\vspace*{-12pt}\\
\hspace*{23pt}перевода$\dotfill$&2&83\\
\hangindent=23pt\noindent\textbf{Козеренко~Е.\,Б., Кузнецов~И.\,П.} Когнитивно-лингвистические представления в 
систе-\linebreak
\vspace*{-12pt}\\
\hspace*{23pt}мах обработки текстов$\dotfill$&3&69\\
\textbf{Князь~В.\,А.} см.~Каратеев~С.\,Л.&&\\
\hangindent=23pt\noindent\textbf{Колесников~А.\,В., Солдатов~С.\,А.} Алгоритм координации для гибридной 
интеллектуальной системы решения сложной задачи оперативно-производственного\linebreak
\vspace*{-12pt}\\
\hspace*{23pt}планирования$\dotfill$&4&61\\
\hangindent=23pt\noindent\textbf{Коновалов~М.\,Г.} О планировании потоков в системах вычислительных 
ресурсов$\dotfill$&2&3\\
\textbf{Конушин~А.\,С.} см.~Конушин~В.\,С.&&\\
\hangindent=23pt\noindent\textbf{Конушин~В.\,С., Кривовязь~Г.\,Р., Конушин~А.\,С.} Алгоритм распознавания людей 
в видео-\linebreak
\vspace*{-12pt}\\
\hspace*{23pt}последовательности по одежде$\dotfill$&1&74\\
\textbf{Корепанов~Э.\, Р.} см.~Синицын~И.\,Н.&&\\
\textbf{Королев~В.\,Ю.} см.~Соколов~И.\,А.&&\\
\textbf{Королев~Р.\,А.} см.~Бенинг~В.\,Е.&&\\
\textbf{Коротышева~А.\,В.} см.~Зейфман~А.\,И.&&\\
\hangindent=23pt\noindent\textbf{Кривенко~М.\,П.} Непараметрическое оценивание элементов байесовского 
клас\-си-\linebreak
\vspace*{-12pt}\\
\hspace*{23pt}фикатора$\dotfill$&2&13\\
\textbf{Кривовязь~Г.\,Р.} см.~Конушин~В.\,С.&&\\
\textbf{Крылов~А.\,С.} см.~Павельева~Е.\,А.&&\\
\hangindent=23pt\noindent\textbf{Крылов~В.\,А.} Моделирование и классификация многоканальных дистанционных\linebreak
\vspace*{-12pt}\\
\hspace*{23pt}изображений с использованием копул$\dotfill$&4&34\\
\hangindent=23pt\noindent\textbf{Крючин~О.\,В.} Разработка параллельных эвристических алгоритмов подбора 
весовых\linebreak
\vspace*{-12pt}\\
\hspace*{23pt}коэффициентов искусственной нейтронной сети$\dotfill$&2&53\\
\hangindent=23pt\noindent\textbf{Кудрявцев~А.\,А., Шоргин~С.\,Я.} Байесовские модели массового обслуживания и 
надеж-\linebreak
\vspace*{-12pt}\\
\hspace*{23pt}ности: характеристики среднего числа заявок в системе $M\vert M \vert 1\vert 
\infty$$\dotfill$&3&16\\
\hangindent=23pt\noindent\textbf{Кузнецов~А.\,А.} Связь между временными и структурно-топологическими 
характери-\linebreak
\vspace*{-12pt}\\
\hspace*{23pt}стиками диаграмм ритма сердца здоровых людей$\dotfill$&4&39\\
\textbf{Кузнецов~И.\,П.} см.~Козеренко~Е.\,Б.&&\\
\textbf{Ле~Пезан~Д.} см.~Бунтман~Н.\,В.&&\\
\hangindent=23pt\noindent\textbf{Лукьяненко~А.\,С., Морозов~Е.\,В., Гуртов~А.\,В.} Анализ сетевого протокола с общей 
функ-\linebreak
\vspace*{-12pt}\\
\hspace*{23pt}цией расширения окна передачи сообщения при конфликтах$\dotfill$&2&46\\
\hangindent=23pt\noindent\textbf{Лямин~О.\,О.} О предельном поведении мощностей критериев в случае обобщенного\linebreak
\vspace*{-12pt}\\
\hspace*{23pt}распределения Лапласа$\dotfill$&3&47\\
\hangindent=23pt\noindent\textbf{Маркин~А.\,В., Шестаков~О.\,В.} Асимптотики оценки риска при пороговой 
обработке\linebreak
\vspace*{-12pt}\\
\hspace*{23pt}вейвлет-вейглет коэффициентов в задаче томографии$\dotfill$&2&36\\
\hangindent=23pt\noindent\textbf{Матвеева~С.\,С., Захарова~Т.\,В.} Сети массового обслуживания с наименьшей 
длиной\linebreak
\vspace*{-12pt}\\
\hspace*{23pt}очереди$\dotfill$&3&22\\
\hangindent=23pt\noindent\textbf{Матюшенко~С.\,И.} Стационарные характеристики двухканальной системы 
обслужива-\linebreak
\vspace*{-12pt}\\
\hspace*{23pt}ния с переупорядочиванием заявок и распределениями фазового типа$\dotfill$&4&68\\
\textbf{Минель~Ж.-Л.} см.~Бунтман~Н.\,В.&&\\
\textbf{Морозов~Е.\,В.} см.~Бородина~А.\,В.&&\\
\textbf{Морозов~Е.\,В.} см.~Лукьяненко~А.\,С.&&\\
\textbf{Ососков~М.\,В.} см.~Каратеев~С.\,Л.&&\\
\hangindent=23pt\noindent\textbf{Павельева~Е.\,А., Крылов~А.\,С.} Поиск и анализ ключевых точек радужной 
оболочки\linebreak
\vspace*{-12pt}\\
\hspace*{23pt}глаза методом преобразования Эрмита$\dotfill$&1&79\\
\textbf{Печинкин~А.\,В.} см.~Френкель~С.\,Л.,&&\\
\hangindent=23pt\noindent\textbf{Протасов~В.\,И.} Составление субъективного портрета с использованием 
эволюционно-\linebreak
\vspace*{-12pt}\\
\hspace*{23pt}го морфинга и квалиметрия метода$\dotfill$&1&83\\
\hangindent=23pt\noindent\textbf{Рудаков~К.\,В., Торшин~И.\,Ю.} Вопросы разрешимости задачи распознавания 
вторичной\linebreak
\vspace*{-12pt}\\
\hspace*{23pt}структуры белка$\dotfill$&2&25\\
\textbf{Сатин~Я.\,А.} см.~Зейфман~А.\,И.&&\\
\hangindent=23pt\noindent\textbf{Сейфуль-Мулюков~Р.\,Б.} Нефть как носитель информации о своем 
происхождении,\linebreak
\vspace*{-12pt}\\
\hspace*{23pt}структуре и эволюции$\dotfill$&1&41\\
\end{tabular}
}

{\tabcolsep=3pt
\begin{tabular}{p{388pt}rr}
&\textbf{Выпуск} & \textbf{Стр.}\\[6pt]
\textbf{Семендяев~Н.\,Н.} см.~Синицын~И.\,Н.&&\\
\textbf{Серебряков~В.\,А.} см.~Захаров~А.\,А.&&\\
\textbf{Синицын~В.\,И.} см.~Синицын~И.\,Н.&&\\
\hangindent=23pt\noindent\textbf{Синицын~И.\,Н., Синицын~В.\,И., Корепанов~Э.\, Р., Белоусов~В.\,В., 
Семендяев~Н.\,Н.} Оперативное построение информационных моделей движения полюса 
Земли\linebreak
\vspace*{-12pt}\\
\hspace*{23pt}методами линейных и линеаризованных фильтров$\dotfill$&1&2\\
\textbf{Сипина~А.\,В.} см.~Бенинг~В.\,Е.&&\\
\hangindent=23pt\noindent\textbf{Соколов~И.\,А.} О работах заслуженного деятеля науки Российской Федерации 
И.\,Н.~Синицына в области информационных технологий и автоматизации (к 70-летию\linebreak
\vspace*{-12pt}\\
\hspace*{23pt}со дня рождения)$\dotfill$&3&84\\
\textbf{Соколов~И.\,А.} см.~Илюшин~Г.\,Я.&&\\
\hangindent=23pt\noindent\textbf{Соколов~И.\,А., Королев~В.\,Ю.} Предисловие$\dotfill$&2&2\\
\textbf{Солдатов~С.\,А.} см.~Колесников~А.\,В.&&\\
\hangindent=23pt\noindent\textbf{Степанов~С.\,Ю.} Использование координатного метода фрагментации 
коммутаторной\linebreak
\vspace*{-12pt}\\
\hspace*{23pt}нейронной сети для сокращения трафика$\dotfill$&2&57\\
\textbf{Тимонина~Е.\,Е.} см.~Грушо~А.\,А.&&\\
\textbf{Торшин~И.\,Ю.} см.~Рудаков~К.\,В.&&\\
\textbf{Ульянов~В.\,В.} см.~Кавагучи~Ю.&&\\
\textbf{Фазекаш~И.} см.~Чупрунов~А.\,Н.&&\\
\textbf{Френкель~С.\,Л.} см.~Баранов~С.\,И.&&\\
\hangindent=23pt\noindent\textbf{Френкель~С.\,Л., Печинкин~А.\,В.} Оценка времени самовосстановления в 
цифровых\linebreak
\vspace*{-12pt}\\
\hspace*{23pt}системах после сбоев, вызываемых переходными помехами$\dotfill$&3&2\\
\textbf{Фуджикоши~Я.} см.~Кавагучи~Ю.&&\\
\hangindent=23pt\noindent\textbf{Цискаридзе~А.\,К.} Математическая модель и метод восстановления позы человека 
по\linebreak
\vspace*{-12pt}\\
\hspace*{23pt}стереопаре силуэтных изображений$\dotfill$&4&27\\
\hangindent=23pt\noindent\textbf{Чупраков~К.\,Г.} К вопросу о размещении коллективных средств отображения в 
ситуа-\linebreak
\vspace*{-12pt}\\
\hspace*{23pt}ционном зале с заданными параметрами$\dotfill$&4&89\\
\textbf{Чупраков~К.\,Г.} см.~Зацаринный~А.\,А.&&\\
\hangindent=23pt\noindent\textbf{Чупрунов~А.\,Н., Фазекаш~И.} Законы повторного логарифма для числа 
безошибочных\linebreak
\vspace*{-12pt}\\
\hspace*{23pt}блоков при помехоустойчивом кодировании$\dotfill$&3&42\\
\textbf{Шевцова~И.\,Г.} см.~Григорьева~М.\,Е.&&\\
\hangindent=23pt\noindent\textbf{Шестаков~О.\,В.} Аппроксимация распределения оценки риска пороговой 
обработки вейвлет-коэффициентов нормальным распределением при использовании 
выбо-\linebreak
\vspace*{-12pt}\\
\hspace*{23pt}рочной дисперсии$\dotfill$&4&73\\
\textbf{Шестаков~О.\,В.} см.~Маркин~А.\,В.&&\\
\textbf{Шоргин~С.\,Я.} см.~Зейфман~А.\,И.&&\\
\textbf{Шоргин~С.\,Я.} см.~Кудрявцев~А.\,А.&&\\
\end{tabular}
}

%\thispagestyle{myheadings}
\def\leftfootline{\small{\textbf{\thepage}
\hfill ИНФОРМАТИКА И ЕЁ ПРИМЕНЕНИЯ\ \ \ том~4\ \ \ выпуск~4\ \ \ 2010}
}%
 \def\rightfootline{\small{ИНФОРМАТИКА И ЕЁ ПРИМЕНЕНИЯ\ \ \ том~4\ \ \ выпуск~4\ \ \ 2010
 \hfill \textbf{\thepage}}}
 \label{end\stat}
%
%Том 10 Выпуск 1-4 Год 2016

\def\stat{cont-e}
{%\hrule\par
%\vskip 7pt % 7pt
\raggedleft\Large \bf%\baselineskip=3.2ex
2\,0\,1\,6\ \ A\,U\,T\,H\,O\,R\ \ I\,N\,D\,E\,X \vskip 17pt
 \hrule
 \par
\vskip 21pt plus 6pt minus 3pt }

\label{st\stat}

\def\tit{\ }

\def\aut{\ }
\def\auf{\ }

\def\leftkol{\ } %2016 AUTHOR INDEX} % ENGLISH ABSTRACTS}

\def\rightkol{\ } %2016 AUTHOR INDEX} %ENGLISH ABSTRACTS}

\titele{\tit}{\aut}{\auf}{\leftkol}{\rightkol}

\def\leftfootline{\small{\textbf{\thepage}
\hfill INFORMATIKA I EE PRIMENENIYA~--- INFORMATICS AND APPLICATIONS\ \ \ 2016\
\ \ volume~10\ \ \ issue\ 4}
}%
 \def\rightfootline{\small{INFORMATIKA I EE PRIMENENIYA~--- INFORMATICS AND APPLICATIONS\ \ \ 2016\ \ \ volume~10\ \ \ issue\ 4
\hfill \textbf{\thepage}}}

\vspace*{-12pt}
\vspace*{-18pt}

{\tabcolsep=2.8pt
\begin{tabular}{p{382pt}cc}
&\textbf{Issue} & \textbf{Page}\\[6pt]
\Avtors{Agalarov~M.\,Ya.} see~Agalarov~Ya.\,M.&&\\
\Avtors{Agalarov~Ya.\,M., Agalarov~M.\,Ya., and
Shorgin~V.\,S.} About the optimal threshold of queue\linebreak
\\[-12pt]
\hspace*{23pt}length in a~particular problem of profit maximization
in the $M/G/1$ queuing system&2&70--79\\
\Avtors{Alexeyevsky~D.\,A.} BioNLP ontology extraction from 
a~restricted language corpus with\linebreak
\\[-12pt]
\hspace*{23pt}context-free grammars&1&119--128\\
\Avtors{Andreev~S.\,D.} see~Gaidamaka~Yu.\,V.&&\\
\Avtors{Andreev~S.\,D.} see~Ometov~A.\,Ya.&&\\
\Avtors{Arkhipov~O.\,P., Arkhipov~P.\,O., and Sidorkin~I.\,I.} The
option to create a~local coordinate\linebreak
\\[-12pt]
\hspace*{23pt}system for synchronization of selected images&3&91--97\\
\Avtors{Arkhipov~P.\,O.} see~Arkhipov~O.\,P.&&\\
\Avtors{Belousov~V.\,V.} see~Shnurkov~P.\,V.&&\\
\Avtors{Belousov~V.\,V.} see~Shnurkov~P.\,V.&&\\
\Avtors{Bening~V.\,E.} Calculation of~the~asymptotic deficiency
of~some statistical procedures based\linebreak
\\[-12pt]
\hspace*{23pt}on~samples with~random sizes&4&34--45\\
\Avtors{Borisov~A.\,V., Bosov~A.\,V., and Miller~G.\,B.} Modeling and
monitoring of VoIP connection&2&\hphantom{1}2--13\\
\Avtors{Bosov~A.\,V.} see~Borisov~A.\,V.&&\\
\Avtors{Briukhov~D.\,O.} see~Stupnikov~S.\,A.&&\\
\Avtors{Callaos~N.\,K.\ and Seyful-Mulyukov~R.\,B.} Complexity and
its information content&1&129--139\\
\Avtors{Chertok~A.\,V., Kadaner~A.\,I., Khazeeva~G.\,T., and
Sokolov~I.\,A.} Regime switching detection\linebreak
\\[-12pt]
\hspace*{23pt}for~the~Levy driven
Ornstein--Uhlenbeck process using CUSUM methods&4&46--56\\
\Avtors{Chichagov~V.\,V.} Asymptotic expansions of mean absolute
error of uniformly minimum variance unbiased and maximum likelihood
estimators on the one-parameter exponential\linebreak
\\[-12pt]
\hspace*{23pt}family model of lattice distributions&3&66--76\\
\Avtors{Danishevsky~V.\,I.} see~Kolesnikov A.\,V.&&\\
\Avtors{Fazliev~A.\,Z.} see~Kalinichenko~L.\,A.&&\\
\Avtors{Fedoseev~A.\,A.} What is behind the concept of ``knowledge in
small packages''&3&105--110\\
\Avtors{Gaidamaka~Yu.\,V., Andreev~S.\,D., Sopin~E.\,S.,
Samouylov~K.\,E., and Shorgin~S.\,Ya.} Interference analysis
of~the~device-to-device communications model with~regard to~a~signal\linebreak
\\[-12pt]
\hspace*{23pt}propagation environment&4&\hphantom{1}2--10\\
\Avtors{Gasilov~A.\,V.} see~Yakovlev~O.\,A.&&\\
\Avtors{Goncharov~A.\,V.\ and Strijov~V.\,V.} Metric time series
classification using weighted dynamic\linebreak
\\[-12pt]
\hspace*{23pt}warping relative to centroids of classes&2&36--47\\
\Avtors{Gordov~E.\,P.} see~Kalinichenko~L.\,A.&&\\
\Avtors{Gorshenin~A.\,K.} Concept of online service for stochastic
modeling of real processes&1&72--81\\
\Avtors{Gorshenin~A.\,K.} see~Shnurkov~P.\,V.&&\\
\Avtors{Gorshenin~A.\,K.} see~Shnurkov~P.\,V.&&\\
\Avtors{Grusho~A.\,A., Grusho~N.\,A., Zabezhailo~M.\,I., and
Timonina~E.\,E.} Integration of statistical and\linebreak
\\[-12pt]
\hspace*{23pt}deterministic methods for
analysis of information security&3&2--8\\
\Avtors{Grusho~A.\,A., Zabezhailo~M.\,I., and Zatsarinny~A.\,A.} On
the advanced procedure to reduce\linebreak
\\[-12pt]
\hspace*{23pt}calculation of Galois closures&4&\hphantom{1}96--104\\
\Avtors{Grusho~N.\,A.} see~Grusho~A.\,A.&&\\
\Avtors{Havanskov~V.\,A.} see~Minin~V.\,A.&&\\
\Avtors{Inkova~O.\,Yu.} see~Zatsman~I.\,M.&&\\
\Avtors{Isachenko~R.\,V.\ and Strijov~V.\,V.} Metric learning in
multiclass time series classification\linebreak
\\[-12pt]
\hspace*{23pt}problem&2&48--57\\
\end{tabular}
}
\pagebreak

\def\leftfootline{\small{\textbf{\thepage}
\hfill INFORMATIKA I EE PRIMENENIYA~--- INFORMATICS AND APPLICATIONS\ \ \ 2016\
\ \ volume~10\ \ \ issue\ 4}
}%
 \def\rightfootline{\small{INFORMATIKA I EE PRIMENENIYA~---
INFORMATICS AND APPLICATIONS\ \ \ 2016\ \ \ volume~10\ \ \ issue\ 4
\hfill \textbf{\thepage}}}

\def\leftkol{2016 AUTHOR INDEX} % ENGLISH ABSTRACTS}

\def\rightkol{2016 AUTHOR INDEX} %ENGLISH ABSTRACTS}


{\tabcolsep=2.83pt
\begin{tabular}{p{382pt}cc}
&\textbf{Issue} & \textbf{Page}\\[6pt]
\Avtors{Kadaner~A.\,I.} see~Chertok~A.\,V.&&\\[.255pt]
\Avtors{Kalinichenko~L.\,A., Volnova~A.\,A., Gordov~E.\,P.,
Kiselyova~N.\,N., Kovaleva~D.\,A., Malkov~O.\,Yu., Okladnikov~I.\,G.,
Podkolodnyy~N.\,L., Pozanenko~A.\,S., Ponomareva~N.\,V.,
Stupnikov~S.\,A.,} \textbf{and Fazliev~A.\,Z.} Data access challenges for data
intensive\linebreak
\\[-12pt]
\hspace*{23pt}research in Russia&1& 2--22\\[.255pt]
\Avtors{Karasikov~M.\,E.\ and Strijov~V.\,V.} Feature-based
time-series classification&4&121--131\\[.255pt]
\Avtors{Khazeeva~G.\,T.} see~Chertok~A.\,V.&&\\[.255pt]
\Avtors{Khokhlov~Yu.\,S.} Multivariate fractional Levy motion and its
applications&2&\hphantom{1}98--106\\[.255pt]
\Avtors{Kirikov~I.\,A., Kolesnikov~A.\,V., Listopad~S.\,V., and
Rumovskaya~S.\,B.} Fine-grained hybrid\linebreak
\\[-12pt]
\hspace*{23pt}intelligent systems. Part 2:
Bidirectional hybridization&1&\hphantom{1}96--105\\[.255pt]
\Avtors{Kirikov~I.\,A., Kolesnikov~A.\,V., Listopad~S.\,V., and
Rumovskaya~S.\,B.} ``Virtual council''~---\linebreak
\\[-12pt]
\hspace*{23pt}source environment
supporting complex diagnostic decision making&3&81--90\\[.255pt]
\Avtors{Kiselyova~N.\,N.} see~Kalinichenko~L.\,A.&&\\[.255pt]
\Avtors{Kolesnikov A.\,V., Listopad~S.\,V., Rumovskaya~S.\,B., and
Danishevsky~V.\,I.} Informal axiomatic\linebreak
\\[-12pt]
\hspace*{23pt}theory of~the~role visual models&4&114--120\\[.255pt]
\Avtors{Kolesnikov~A.\,V.} see~Kirikov~I.\,A.&&\\[.255pt]
\Avtors{Kolesnikov~A.\,V.} see~Kirikov~I.\,A.&&\\[.255pt]
\Avtors{Kolin~K.\,K.} Humanitarian aspects of information
security&3&111--121\\[.255pt]
\Avtors{Konovalov~M.\,G.\ and Razumchik~R.\,V.} Dispatching
to~two parallel nonobservable queues using\linebreak
\\[-12pt]
\hspace*{23pt}only static
information&4&57--67\\[.255pt]
\Avtors{Korchagin~A.\,Yu.} see~Korolev~V.\,Yu.&&\\[.255pt]
\Avtors{Korchagin~A.\,Yu.} see~Korolev~V.\,Yu.&&\\[.255pt]
\Avtors{Korepanov~E.\,R.} see~Sinitsyn~I.\,N.&&\\[.255pt]
\Avtors{Korepanov~E.\,R.} see~Sinitsyn~I.\,N.&&\\[.255pt]
\Avtors{Korolev~V.\,Yu., Korchagin~A.\,Yu., and Zeifman~A.\,I.} The
Poisson theorem for Bernoulli trials\linebreak
\\[-12pt]
\hspace*{23pt}with~a~random probability
of~success and~a~discrete analog of~the~Weibull distribution&4&11--20\\[.255pt]
\Avtors{Korolev~V.\,Yu., Zeifman~A.\,I., and Korchagin~A.\,Yu.}
Asymmetric Linnik distributions as~limit\linebreak
\\[-12pt]
\hspace*{23pt}laws for~random sums
of~independent random variables with~finite variances&4&21--33\\[.255pt]
\Avtors{Koucheryavy~E.\,A.} see~Ometov~A.\,Ya.&&\\[.255pt]
\Avtors{Kovaleva~D.\,A.} see~Kalinichenko~L.\,A.&&\\[.255pt]
\Avtors{Kovalyov~S.\,P.} Metaprogramming to increase
manufacturability of large-scale software-\linebreak
\\[-12pt]
\hspace*{23pt}intensive systems&1&56--66\\[.255pt]
\Avtors{Krivenko~M.\,P.} Significance tests of feature selection for
classification&3&32--40\\[.255pt]
\Avtors{Kruzhkov~M.\,G.} see~Zalizniak~Anna~A.&&\\[.255pt]
\Avtors{Kruzhkov~M.\,G.} see~Zatsman~I.\,M.&&\\[.255pt]
\Avtors{Kudryavtsev~A.\,A.} Bayesian queueing and reliability models:
\textit{A~priori} distributions with\linebreak
\\[-12pt]
\hspace*{23pt}compact support&1&67--71\\[.255pt]
\Avtors{Kudryavtsev~A.\,A.} Characteristics dependent on the balance
coefficient in Bayesian models\linebreak
\\[-12pt]
\hspace*{23pt}with compact support of \textit{a priori}
distributions&3&77--80\\[.255pt]
\Avtors{Kudryavtsev~A.\,A.\ and Palionnaia~S.\,I.} Bayesian recurrent
model of reliability growth:\linebreak
\\[-12pt]
\hspace*{23pt}Parabolic distribution of parameters&2&80--83\\[.255pt]
\Avtors{Kudryavtsev~A.\,A.\ and Titova~A.\,I.} Bayesian queuing
and~reliability models: Degenerate-\linebreak
\\[-12pt]
\hspace*{23pt}Weibull case&4&68--71\\[.255pt]
\Avtors{Leontyev~N.\,D.\ and Ushakov~V.\,G.} Analysis of a queueing
system with autoregressive arrivals\linebreak
\\[-12pt]
\hspace*{23pt}and nonpreemptive priority&3&15--22\\[.255pt]
\Avtors{Listopad~S.\,V.} see~Kirikov~I.\,A.&&\\[.255pt]
\Avtors{Listopad~S.\,V.} see~Kirikov~I.\,A.&&\\[.255pt]
\Avtors{Listopad~S.\,V.} see~Kolesnikov A.\,V.&&\\[.255pt]
\Avtors{Malkov~O.\,Yu.} see~Kalinichenko~L.\,A.&&\\[.255pt]
\Avtors{Markov~A.\,S., Monakhov~M.\,M., and
Ulyanov~V.\,V.} Generalized Cornish--Fisher expansions\linebreak
\\[-12pt]
\hspace*{23pt}for distributions of statistics based on samples
of random size&2&84--91\\[.255pt]
\Avtors{Melnikov~A.\,K.\ and Ronzhin~A.\,F.} Generalized statistical
method of~text analysis based\linebreak
\\[-12pt]
\hspace*{23pt}on~calculation of~probability distributions
of~statistical values&4&89--95\\
\end{tabular}
}
\pagebreak

\def\leftfootline{\small{\textbf{\thepage}
\hfill INFORMATIKA I EE PRIMENENIYA~--- INFORMATICS AND APPLICATIONS\ \ \ 2016\
\ \ volume~10\ \ \ issue\ 4}
}%
 \def\rightfootline{\small{INFORMATIKA I EE PRIMENENIYA~---
INFORMATICS AND APPLICATIONS\ \ \ 2016\ \ \ volume~10\ \ \ issue\ 4
\hfill \textbf{\thepage}}}

\def\leftkol{2016 AUTHOR INDEX} % ENGLISH ABSTRACTS}

\def\rightkol{2016 AUTHOR INDEX} %ENGLISH ABSTRACTS}


{\tabcolsep=3pt
\begin{tabular}{p{381pt}cc}
&\textbf{Issue} & \textbf{Page}\\[6pt]
\Avtors{Meykhanadzhyan~L.\,A.} Stationary characteristics of the finite
capacity queueing system with\linebreak
\\[-12pt]
\hspace*{23pt}inverse service order and generalized
probabilistic priority&2&123--131\\[.23pt]
\Avtors{Miller~G.\,B.} see~Borisov~A.\,V.&&\\[.23pt]
\Avtors{Minin~V.\,A., Zatsman~I.\,M., Havanskov~V.\,A., and
Shubnikov~S.\,K.} Intensity of citation of scientific publications in
inventions on information and computer technologies patented\linebreak
\\[-12pt]
\hspace*{23pt}in Russia by domestic and foreign applicants&2&107--122\\[.23pt]
\Avtors{Monakhov~M.\,M.} see~Markov~A.\,S.&&\\[.23pt]
\Avtors{Naumov~V.\,A.\ and Samouylov~K.\,E.} On relationship
between queuing systems with resources\linebreak
\\[-12pt]
\hspace*{23pt}and Erlang networks&3&\hphantom{1}9--14\\[.23pt]
\Avtors{Okladnikov~I.\,G.} see~Kalinichenko~L.\,A.&&\\[.23pt]
\Avtors{Ometov~A.\,Ya., Andreev~S.\,D., Turlikov~A.\,M., and
Koucheryavy~E.\,A.} Performance analysis of\linebreak
\\[-12pt]
\hspace*{23pt}a wireless data
aggregation system with contention for contemporary sensor
networks&3&23--31\\[.23pt]
\Avtors{Palionnaia~S.\,I.} see~Kudryavtsev~A.\,A.&&\\[.23pt]
\Avtors{Podkolodnyy~N.\,L.} see~Kalinichenko~L.\,A.&&\\[.23pt]
\Avtors{Ponomareva~N.\,V.} see~Kalinichenko~L.\,A.&&\\[.23pt]
\Avtors{Popkova~N.\,A.} see~Zatsman~I.\,M.&&\\[.23pt]
\Avtors{Pozanenko~A.\,S.} see~Kalinichenko~L.\,A.&&\\[.23pt]
\Avtors{Razumchik~R.\,V.} see~Konovalov~M.\,G.&&\\[.23pt]
\Avtors{Ronzhin~A.\,F.} see~Melnikov~A.\,K.&&\\[.23pt]
\Avtors{Rumovskaya~S.\,B.} see~Kirikov~I.\,A.&&\\[.23pt]
\Avtors{Rumovskaya~S.\,B.} see~Kirikov~I.\,A.&&\\[.23pt]
\Avtors{Rumovskaya~S.\,B.} see~Kolesnikov A.\,V.&&\\[.23pt]
\Avtors{Samouylov~K.\,E.} see~Gaidamaka~Yu.\,V.&&\\[.23pt]
\Avtors{Samouylov~K.\,E.} see~Naumov~V.\,A.&&\\[.23pt]
\Avtors{Serebryanskii~S.\,M.} see~Tyrsin~A.\,N.&&\\[.23pt]
\Avtors{Seyful-Mulyukov~R.\,B.} see~Callaos~N.\,K.&&\\[.23pt]
\Avtors{Shestakov~O.\,V.} Statistical properties of the denoising method
based on the stabilized hard\linebreak
\\[-12pt]
\hspace*{23pt}thresholding&2&65--69\\[.23pt]
\Avtors{Shestakov~O.\,V.} The strong law of large numbers for the risk
estimate in the problem of\linebreak
\\[-12pt]
\hspace*{23pt}tomographic image reconstruction from
projections with a correlated noise&3&41--45\\[.23pt]
\Avtors{Shestakov~O.\,V.} see~Zakharova~T.\,V.&&\\[.23pt]
\Avtors{Shnurkov~P.\,V., Gorshenin~A.\,K., and Belousov~V.\,V.}
Analytical solution of~the~optimal control\linebreak
\\[-12pt]
\hspace*{23pt}task of~a~semi-Markov
process with~finite set of~states&4&72--88\\[.23pt]
\Avtors{Shnurkov~P.\,V., Zasypko~V.\,V., Belousov~V.\,V., and
Gorshenin~A.\,K.} Development of the algorithm of numerical solution
of the optimal investment control problem\linebreak
\\[-12pt]
\hspace*{23pt}in the closed dynamical model of three-sector economy&1&82--95\\[.23pt]
\Avtors{Shorgin~S.\,Ya.} see~Gaidamaka~Yu.\,V.&&\\[.23pt]
\Avtors{Shorgin~V.\,S.} see~Agalarov~Ya.\,M.&&\\[.23pt]
\Avtors{Shubnikov~S.\,K.} see~Minin~V.\,A.&&\\[.23pt]
\Avtors{Sidorkin~I.\,I.} see~Arkhipov~O.\,P.&&\\[.23pt]
\Avtors{Sinitsyn~I.\,N.} Analytical modeling of processes in stochastic
systems with complex fractional\linebreak
\\[-12pt]
\hspace*{23pt}order Bessel nonlinearities&3&55--65\\[.23pt]
\Avtors{Sinitsyn~I.\,N.} Orthogonal supoptimal filters for nonlinear
stochastic systems on manifolds&1&34--44\\[.23pt]
\Avtors{Sinitsyn~I.\,N.\ and Korepanov~E.\,R.} Normal Pugachev
conditionally-optimal filters and extra-\linebreak
\\[-12pt]
\hspace*{23pt}polators for state linear stochastic systems&2&14--23\\[.23pt]
\Avtors{Sinitsyn~I.\,N.\ and Sinitsyn~V.\,I.} Analytical modeling of
distributions in stochastic systems on\linebreak
\\[-12pt]
\hspace*{23pt}manifolds based on ellipsoidal approximation&1&45--55\\[.23pt]
\Avtors{Sinitsyn~I.\,N., Sinitsyn~V.\,I., and
Korepanov~E.\,R.} Ellipsoidal suboptimal filters for nonlinear\linebreak
\\[-12pt]
\hspace*{23pt}stochastic systems on manifolds&2&24--35\\[.23pt]
\Avtors{Sinitsyn~V.\,I.} see~Sinitsyn~I.\,N.&&\\[.23pt]
\Avtors{Sinitsyn~V.\,I.} see~Sinitsyn~I.\,N.&&\\[.23pt]
\Avtors{Skvortsov~N.\,A.} see~Stupnikov~S.\,A.&&\\[.23pt]
\Avtors{Sokolov~I.\,A.} see~Chertok~A.\,V.&&\\
\end{tabular}
}
\pagebreak

\def\leftfootline{\small{\textbf{\thepage}
\hfill INFORMATIKA I EE PRIMENENIYA~--- INFORMATICS AND APPLICATIONS\ \ \ 2016\
\ \ volume~10\ \ \ issue\ 4}
}%
 \def\rightfootline{\small{INFORMATIKA I EE PRIMENENIYA~---
INFORMATICS AND APPLICATIONS\ \ \ 2016\ \ \ volume~10\ \ \ issue\ 4
\hfill \textbf{\thepage}}}

\def\leftkol{2016 AUTHOR INDEX} % ENGLISH ABSTRACTS}

\def\rightkol{2016 AUTHOR INDEX} %ENGLISH ABSTRACTS}


{\tabcolsep=3pt
\begin{tabular}{p{382pt}cc}
&\textbf{Issue} & \textbf{Page}\\[6pt]
\Avtors{Sopin~E.\,S.} see~Gaidamaka~Yu.\,V.&&\\
\Avtors{Strijov~V.\,V.} see~Goncharov~A.\,V.&&\\
\Avtors{Strijov~V.\,V.} see~Isachenko~R.\,V.&&\\
\Avtors{Strijov~V.\,V.} see~Karasikov~M.\,E.&&\\
\Avtors{Stupnikov~S.\,A., Briukhov~D.\,O., and Skvortsov~N.\,A.}
Co-lending systemic risk analysis over\linebreak
\\[-12pt]
\hspace*{23pt}heterogeneous data collections&1&23--33\\
\Avtors{Stupnikov~S.\,A.} see~Kalinichenko~L.\,A.&&\\
\Avtors{Suchkov~A.\,P.} see~Zatsarinny~A.\,A.&&\\
\Avtors{Timonina~E.\,E.} see~Grusho~A.\,A.&&\\
\Avtors{Titova~A.\,I.} see~Kudryavtsev~A.\,A.&&\\
\Avtors{Turlikov~A.\,M.} see~Ometov~A.\,Ya.&&\\
\Avtors{Tyrsin~A.\,N.\ and Serebryanskii~S.\,M.} Recognition of
dependences on the basis of inverse\linebreak
\\[-12pt]
\hspace*{23pt}mapping&2&58--64\\
\Avtors{Ulyanov~V.\,V.} see~Markov~A.\,S.&&\\
\Avtors{Ushakov~V.\,G.} Queueing system with working vacations and
hyperexponential input stream&2&92--97\\
\Avtors{Ushakov~V.\,G.} see~Leontyev~N.\,D.&&\\
\Avtors{Volnova~A.\,A.} see~Kalinichenko~L.\,A.&&\\
\Avtors{Yakovlev~O.\,A.\ and Gasilov~A.\,V.} Speeded-up stereo
matching using geodesic support weights&3&\hphantom{1}98--104\\
\Avtors{Zabezhailo~M.\,I.} see~Grusho~A.\,A.&&\\
\Avtors{Zabezhailo~M.\,I.} see~Grusho~A.\,A.&&\\
\Avtors{Zakharova~T.\,V.\ and Shestakov~O.\,V.} Precision analysis of
wavelet processing of aerodynamic\linebreak
\\[-12pt]
\hspace*{23pt}flow patterns&3&46--54\\
\Avtors{Zalizniak~Anna~A.\ and Kruzhkov~M.\,G.} Database
of~Russian impersonal verbal constructions&4&132--141\\
\Avtors{Zasypko~V.\,V.} see~Shnurkov~P.\,V.&&\\
\Avtors{Zatsarinny~A.\,A.\ and Suchkov~A.\,P.} Systems engineering
approaches to~the~establishment of\linebreak
\\[-12pt]
\hspace*{23pt}a~system for~decision support based
on~situational analysis&4&105--113\\
\Avtors{Zatsarinny~A.\,A.} see~Grusho~A.\,A.&&\\
\Avtors{Zatsman~I.\,M., Inkova~O.\,Yu., Kruzhkov~M.\,G., and
Popkova~N.\,A.} Representation of cross-\linebreak
\\[-12pt]
\hspace*{23pt}lingual knowledge about
connectors in supracorpora databases&1&106--118\\
\Avtors{Zatsman~I.\,M.} see~Minin~V.\,A.&&\\
\Avtors{Zeifman~A.\,I.} see~Korolev~V.\,Yu.&&\\
\Avtors{Zeifman~A.\,I.} see~Korolev~V.\,Yu.&&\\
\end{tabular}
}

%\thispagestyle{myheadings}
\def\leftfootline{\small{\textbf{\thepage}
\hfill INFORMATIKA I EE PRIMENENIYA~--- INFORMATICS AND APPLICATIONS\ \ \ 2016\
\ \ volume~10\ \ \ issue\ 4}
}%
 \def\rightfootline{\small{INFORMATIKA I EE PRIMENENIYA~---
INFORMATICS AND APPLICATIONS\ \ \ 2016\ \ \ volume~10\ \ \ issue\ 4
\hfill \textbf{\thepage}}}

 \label{end\stat}

\newpage

%\def\stat{rekl}
%\label{preobr}

%\def\tit{АКАДЕМИК ПУГАЧЁВ  ВЛАДИМИР СЕМЁНОВИЧ\\
%25.03.1911--25.03.1998}


%   \vspace*{-48pt}
%   \begin{center}\LARGE
%Академик Пугачёв  Владимир Семёнович\\ (25.03.1911--25.03.1998)
%   \end{center}
   
   %\vspace*{2.5mm}
   
   \begin{center}

{\prgsh\LARGE
ОБЪЯВЛЕНИЯ О КОНФЕРЕНЦИЯХ}

\end{center}
%\hrule

\vspace*{6pt}

   
   \vspace*{10mm}
   
   \thispagestyle{empty}

\noindent
\begin{tabular}{cc}
%\begin{center}
\multicolumn{1}{c}{\raisebox{-40pt}[0pt][0pt]{\mbox{%
\epsfxsize=33mm
\epsfbox{vspu.eps}
}}}
%\end{center}
&
\tabcolsep=0pt\begin{tabular}{c}
{\prg{\Large\textbf{XII Всероссийское совещание}}}\\[6pt]
{\prg{\Large\textbf{по проблемам управления}}}\\[12pt]
{\prg{\large 16--19 июня 2014~г.}}\\[6pt] 
{\prg{\large Институт проблем управления имени В.\,А.~Трапезникова РАН}}\\[6pt]
{\prg{\large Москва, Россия}}
\end{tabular}
\end{tabular}

\vspace*{60pt}

     
 { %\large    
 XII Всероссийское совещание по проблемам управления (ВСПУ XII), посвященное 75-летию 
Института проблем управления (ИПУ) имени В.\,А.~Трапезникова РАН, проводится 16--19~июня 
2014~г.\ 
в ИПУ РАН (г.~Москва, Россия). ВСПУ XII организуется ИПУ РАН при поддержке РФФИ, Отделения 
энергетики, машиностроения, механики и процессов управления Российской академии наук, 
Российского 
национального комитета по автоматическому управлению, Академии навигации и управ\-ле\-ния 
движением, 
Научного совета РАН по комплексным проблемам управления и автоматизации, Совета по 
мехатронике и робототехнике РАН. Официальный язык Совещания~--- русский.

\vspace*{24pt}
     
     \textbf{Направления работы}
     \begin{enumerate}[1.]
\item Теория систем управления
\item Управление подвижными объектами и навигация
\item Интеллектуальные системы управления
\item Управление в промышленности, транспортом и логистикой
\item Управление системами междисциплинарной природы
\item Средства измерения, вычислений и контроля в управлении
\item Системный анализ и принятие решений в задачах управления
\item Информационные технологии в управлении
\item Проблемы образования в области управления: современное содержание и технологии обучения
\end{enumerate}

\vspace*{24pt}

     Подробная информация о Совещании находится на сайте {\sf http://vspu2014.ipu.ru}. Срок 
окончательной подачи докладов через систему подачи докладов на сайте~--- \textbf{30~ноября} 
2013~г.
}


%\end{document}

%\include{nekrolog-rb}



%\include{IPPM-25}

\def\stat{cont-rus}
{%\hrule\par
%\vskip 7pt % 7pt
\vspace*{-24pt}
\raggedleft\Large \bf%\baselineskip=3.2ex
Правила подготовки рукописей  для публикации в журнале
<<Информатика~и~её~применения>> \vskip 8pt
    \hrule
    \par
\vskip 14pt plus 6pt minus 3pt }

\label{st\stat}

\def\tit{\ }

\def\aut{\ }
\def\auf{\ }

\def\leftkol{\ }
% Правила подготовки рукописей  для публикации в журнале
%<<Информатика и её применения>>

\def\rightkol{\ }
%Правила подготовки рукописей  для публикации в журнале
%<<Информатика и её применения>>}


\titele{\tit}{\aut}{\auf}{\leftkol}{\rightkol}


\vspace*{-60pt}
{ %\small

Журнал <<Информатика и её применения>>
публикует теоретические, обзорные и дискуссионные статьи,
посвященные научным исследованиям и разработкам в области
информатики и ее приложений.

Журнал издается на русском языке. По специальному решению
редколлегии отдельные статьи могут печататься на английском языке.

Тематика журнала охватывает следующие направления:
\begin{itemize}
\item теоретические основы информатики;\\[-15pt]
      \item
математические методы исследования сложных систем и процессов;\\[-15pt]
           \item
информационные системы и сети;\\[-15pt]
                \item
информационные технологии;\\[-15pt]
                     \item
архитектура и программное обеспечение вычислительных комплексов и сетей.\\[-15pt]
\end{itemize}


\noindent
\begin{enumerate}[1.]
\item В журнале печатаются статьи, содержащие результаты, ранее не опубликованные и
не предназначенные к одновременной публикации в других изданиях.

%Публикация не должна нарушать закон об авторских правах.
Публикация предоставленной автором(ами) рукописи не должна нарушать 
положений глав~69, 70 раздела~VII части~IV Гражданского кодекса, 
которые определяют права на результаты интеллектуальной деятельности 
и~средства индивидуализации, в~том числе авторские права, в~РФ.

Ответственность за нарушение авторских прав, в~случае предъявления претензий к~редакции журнала,  
несут авторы статей.



Направляя рукопись в редакцию, авторы сохраняют свои права на данную
рукопись и при этом передают учредителям и редколлегии журнала неисключительные права на
издание статьи на русском языке 
(или на языке статьи, если он отличен от рус\-ско\-го) и~на перевод ее на английский
язык, а~также на
ее распространение в России и за рубежом. 
Каждый автор должен представить в~редакцию подписанный 
с~его стороны <<Лицензионный договор о~передаче неисключительных прав 
на использование произведения>>, текст которого размещен по адресу 
{\sf http://www.ipiran.ru/publications/licence.doc}. 
Этот договор может быть пред\-став\-лен в~бумажном (в~2-х экз.)\ 
или в~электронном виде (отсканированная копия заполненного и~подписанного документа).




Редколлегия вправе запросить у авторов экспертное заключение о возможности
пуб\-ли\-ка\-ции пред\-став\-лен\-ной статьи в открытой печати.\\[-13.5pt]

\item К статье прилагаются данные автора (авторов) (см.\ п.~8). При наличии нескольких
авторов указывается фамилия автора, ответственного за переписку с редакцией.\\[-13.5pt]

\item Редакция журнала осуществляет экспертизу присланных статей в соответствии с
принятой в журнале процедурой рецензирования.

Возвращение рукописи на доработку не означает ее принятия к печати.

Доработанный вариант с ответом на замечания рецензента необходимо прислать в
редакцию.\\[-13.5pt]

\item Решение редколлегии о публикации статьи или ее отклонении сообщается авторам.

Редколлегия может также направить авторам текст рецензии на их статью. Дискуссия по
поводу отклоненных статей не ведется.\\[-13.5pt]

%\pagebreak

\item Редактура статей высылается авторам для просмотра. Замечания к редактуре должны
быть присланы авторами в кратчайшие сроки.\\[-13.5pt]

\item Рукопись предоставляется в электронном виде в форматах MS WORD (.doc или
.docx) или \LaTeX\  (.tex), дополнительно~--- в формате .pdf, на дискете, лазерном диске
или электронной почтой. Предоставление бумажной рукописи необязательно.\\[-13.5pt]

\item При подготовке рукописи в MS Word рекомендуется использовать следующие
настройки.

Параметры страницы:
формат~--- А4; ориентация~--- книжная; поля (см): внутри~--- 2,5, снаружи~--- 1,5,
сверху~--- 2, снизу~--- 2, от края до нижнего колонтитула~--- 1,3.

Основной текст: стиль~--- <<Обычный>>, шрифт~--- Times New Roman, размер~---
14~пунк\-тов, абзацный отступ~--- 0,5~см, 1,5~интервала, выравнивание~--- по ширине.

\pagebreak

\def\leftkol{Правила подготовки рукописей  для публикации в журнале
<<Информатика и её применения>>}

\def\rightkol{Правила подготовки рукописей  для публикации в журнале
<<Информатика и её применения>>}



Рекомендуемый объем рукописи~--- не свыше 10~страниц указанного формата.
При превышении указанного объема редколлегия вправе потребовать от 
автора сокращения объема рукописи.


Сокращения слов, помимо стандартных, не допускаются. Допускается минимальное
количество аббревиатур.


Все страницы рукописи нумеруются.

Шаблоны оформления представлены в интернете:

\noindent
 {\sf
http://www.ipiran.ru/journal/template\_iiep\_ssi\_2024.zip}\\[-14pt]

\item Статья должна содержать следующую информацию на {\bfseries\textit{русском и
английском языках}}:\\[-16pt]

\begin{itemize}
\item название статьи;\\[-15pt]
\item Ф.И.О.\ авторов, на английском можно только имя и фамилию;\\[-15pt]
\item место работы, с указанием почтового адреса организации и электронного адреса каждого
автора;\\[-15pt]
\item сведения об авторах, в соответствии с форматом, образцы которого
представлены на страницах:



\def\leftfootline{\small{\textbf{\thepage}
\hfill ИНФОРМАТИКА И ЕЁ ПРИМЕНЕНИЯ\ \ \ том\ 18\ \ \ выпуск\ 3\ \ \ 2024}
}%
 \def\rightfootline{\small{ИНФОРМАТИКА И ЕЁ ПРИМЕНЕНИЯ\ \ \ том\ 18\ \ \ выпуск\ 3\ \ \ 2024
\hfill \textbf{\thepage}}}



{\sf http://www.ipiran.ru/journal/issues/2013\_07\_01/authors.asp} и

{\sf http://www.ipiran.ru/journal/issues/2013\_07\_01\_eng/authors.asp};
\item аннотация (не менее 100~слов на каждом из языков). Аннотация~--- это краткое
резюме работы, которое может публиковаться отдельно. Она является основным
источником информации в~ин\-фор\-ма\-ци\-он\-ных системах и базах данных. Английская
аннотация должна быть оригинальной, может не быть дословным переводом русского
текста и должна быть написана хорошим английским языком. В~аннотации не должно
быть ссылок на литературу и, по возможности, формул;\\[-15pt]
\item ключевые слова~--- желательно из принятых в мировой
на\-уч\-но-тех\-ни\-че\-ской литературе тематических тезаурусов. Предложения не
могут быть ключевыми словами;\\[-15pt]
\item источники финансирования работы (ссылки на гранты, проекты,
поддерживающие организации и~т.\,п.).
\end{itemize}



%\pagebreak

\item  Требования к спискам литературы.\\[-14pt]

Ссылки на литературу в тексте статьи нумеруются (в квадратных скобках) и
располагаются в каждом из списков литературы в порядке  первых упоминаний. Если источник имеет DOI и/или EDN,
то их необходимо указывать.

Списки литературы представляются в двух вариантах:\\[-14pt]


\noindent
\begin{enumerate}[(1)]
\item \textbf{Список литературы к русскоязычной части}. Русские и английские
работы~---  на языке и в алфавите оригинала;\\[-14.5pt]
\item  \textbf{References}. Русские работы и работы на других языках~--- в латинской
транслитерации с переводом на английский язык; английские работы и работы на других
языках~--- на языке оригинала.
\end{enumerate}

Необходимо для составления списка ``References'' пользоваться размещенной на сайте
{\sf http://www. translit.net/ru/bgn/} бесплатной программой транслитерации русского
 текста в~латиницу. %, при этом в~за\-клад\-ке <<варианты\ldots>> следует выбратьопцию BGN.

Список литературы ``References'' приводится полностью отдельным блоком, повторяя все
позиции из списка литературы к русскоязычной части, независимо от того, имеются или
нет в нем иностранные источники. Если в списке литературы к русскоязычной части есть
ссылки на иностранные публикации, набранные латиницей, они полностью повторяются в
списке ``References''.

Ниже приведены примеры ссылок на различные виды публикаций в списке ``References''.

\def\leftfootline{\small{\textbf{\thepage}
\hfill ИНФОРМАТИКА И ЕЁ ПРИМЕНЕНИЯ\ \ \ том\ 18\ \ \ выпуск\ 3\ \ \ 2024}
}%
 \def\rightfootline{\small{ИНФОРМАТИКА И ЕЁ ПРИМЕНЕНИЯ\ \ \ том\ 18\ \ \ выпуск\ 3\ \ \ 2024
\hfill \textbf{\thepage}}}

{\small

\noindent
\textbf{Описание статьи из журнала:}

\Aue{Zagurenko, A.\,G., V.\,A.~Korotovskikh, A.\,A.~Kolesnikov, A.\,V.~Timonov, and D.\,V.~Kardymon}. 2008.
Tekhniko-ekonomicheskaya optimizatsiya dizayna gidrorazryva plasta [Technical and
economic optimization of the design
of hydraulic fracturing]. \textit{Neftyanoe hozyaystvo} [\textit{Oil Industry}] 11:54--57.

\Aue{Zhang, Z., and D.~Zhu}. 2008. Experimental research on the localized
electrochemical micromachining. \textit{Russ. J.~Electrochem.}  44(8):926--930.
{\sf doi:10.1134/S1023193508080077}.

\noindent
\textbf{Описание статьи из электронного журнала:}

\Aue{Swaminathan, V., E.~Lepkoswka-White, and B.\,P.~Rao}. 1999. Browsers or buyers in cyberspace? An
investigation of electronic factors influencing electronic exchange. \textit{JCMC}
5(2). Available at: {\sf http://www.ascusc.org/jcmc/vol5/issue2/} (accessed April~28, 2011).

\def\leftkol{Правила подготовки рукописей  для публикации в журнале
<<Информатика и её применения>>}

\def\rightkol{Правила подготовки рукописей  для публикации в журнале
<<Информатика и её применения>>}


\noindent
\textbf{Описание статьи из продолжающегося издания (сборника трудов):}

\Aue{Astakhov, M.\,V., and T.\,V.~Tagantsev}. 2006. Eksperimental'noe
issledovanie prochnosti soedineniy ``stal'--kompozit'' [Experimental study of
the strength of joints ``steel--composite'']. \textit{Trudy MGTU
``Matematicheskoe modelirovanie slozhnykh tekh\-ni\-che\-skikh sistem''}
[\textit{Bauman MSTU ``Mathematical Modeling of Complex Technical
Systems'' Proceedings}]. 593:125--130.


\pagebreak



\noindent
\textbf{Описание материалов конференций:}

\Aue{Usmanov, T.\,S., A.\,A.~Gusmanov, I.\,Z.~Mullagalin, R.\,Ju.~Muhametshina, A.\,N.~Chervyakova, and
A.\,V.~Sveshnikov}. 2007. Osobennosti proektirovaniya razrabotki mestorozhdeniy
s primeneniem gidrorazryva
plasta [Features of the design of field development with the use of hydraulic fracturing].
\textit{Trudy 6-go
Mezhdu\-na\-rod\-no\-go Simpoziuma ``Novye resursosberegayushchie tekhnologii nedropol'zovaniya i povysheniya
neftegazootdachi''} [\textit{6th  Symposium (International) ``New Energy Saving Subsoil Technologies and
the Increasing of the Oil and Gas Impact'' Proceedings}]. Moscow. 267--272.



\def\leftfootline{\small{\textbf{\thepage}
\hfill ИНФОРМАТИКА И ЕЁ ПРИМЕНЕНИЯ\ \ \ том\ 18\ \ \ выпуск\ 3\ \ \ 2024}
}%
 \def\rightfootline{\small{ИНФОРМАТИКА И ЕЁ ПРИМЕНЕНИЯ\ \ \ том\ 18\ \ \ выпуск\ 3\ \ \ 2024
\hfill \textbf{\thepage}}}



\noindent
\textbf{Описание книги (монографии, сборники):}



Lindorf, L.\,S., and L.\,G.~Mamikoniants, eds. 1972.
\textit{Ekspluatatsiya turbogeneratorov s neposredstvennym
okhlazhdeniem} [\textit{Operation of turbine generators with direct cooling}].
Moscow: Energy Publs. 352~p.


\Aue{Latyshev, V.\,N.} 2009. \textit{Tribologiya rezaniya. Kn.~1: Friktsionnye protsessy
pri rezanii metallov}
[\textit{Tribology of cutting. Vol.~1: Frictional processes in metal cutting}]. Ivanovo: Ivanovskii
State Univ. 108~p.

\def\leftkol{Правила подготовки рукописей  для публикации в журнале
<<Информатика и её применения>>}

\def\rightkol{Правила подготовки рукописей  для публикации в журнале
<<Информатика и её применения>>}

\noindent
\textbf{Описание переводной книги}
(в списке литературы к русскоязычной части необходимо указать:~/ Пер.\ с англ.~---
после названия книги, а в конце ссылки указать оригинал книги в круглых скобках):
\begin{enumerate}[1.]
\item  В русскоязычной части:

\def\leftfootline{\small{\textbf{\thepage}
\hfill ИНФОРМАТИКА И ЕЁ ПРИМЕНЕНИЯ\ \ \ том\ 18\ \ \ выпуск\ 3\ \ \ 2024}
}%
 \def\rightfootline{\small{ИНФОРМАТИКА И ЕЁ ПРИМЕНЕНИЯ\ \ \ том\ 18\ \ \ выпуск\ 3\ \ \ 2024
\hfill \textbf{\thepage}}}

\Au{Тимошенко С.\,П., Янг Д.\,Х., Уивер~У.}
Колебания в инженерном деле~/ Пер.\ с англ.~--- М.: Машиностроение, 1985. 472~с.
(\Au{Timoshenko~S.\,P., Young~D.\,H., Weaver~W.}
Vibration problems in engineering.~--- 4th ed.~--- New York, NY, USA: Wiley, 1974. 521~p.)\\[-13.5pt]
\item  В англоязычной части:

\Aue{Timoshenko, S.\,P., D.\,H.~Young, and W.~Weaver}.
1974. \textit{Vibration problems in engineering}. 4th ed. New York: 
Wiley. 521~p.
\end{enumerate}

\vspace*{-3pt}


\noindent
\textbf{Описание неопубликованного документа:}


\Aue{Latypov, A.\,R., M.\,M.~Khasanov, and V.\,A.~Baikov}.
2004 (unpubl.). Geologiya i~dobycha (NGT GiD) [Geology and production (NGT GiD)]. Certificate on official registration of the computer program
No.\,2004611198. 

\noindent
\textbf{Описание интернет-ресурса:}


Pravila tsitirovaniya istochnikov [Rules for the citing of sources]. Available at: {\sf
http://www.scribd.com/doc/1034528/} (accessed February~7, 2011).

%\pagebreak

\noindent
\textbf{Описание диссертации или автореферата диссертации:}

\Aue{Semenov, V.\,I.}
2003. Matematicheskoe modelirovanie plazmy v sisteme kompaktnyy tor [Mathematical
modeling of the plasma in the compact torus].  Moscow.  D.Sc.\ Diss. 272~p.

\Aue{Kozhunova, O.\,S.} 2009. Tekhnologiya razrabotki semanticheskogo
slovarya informatsionnogo monitoringa [Technology of development of
semantic dictionary of information monitoring system].  Moscow: IPI RAN. PhD Thesis. 23~p.


\noindent
\textbf{Описание ГОСТа:}

GOST 8.586.5-2005. 2007. Metodika vypolneniya izmereniy. Izmerenie raskhoda i~kolichestva zhidkostey i~gazov
s~pomoshch'yu standartnykh suzhayushchikh ustroystv [Method of measurement.
Measurement of flow rate and volume of liquids and gases by means of orifice devices]. Moscow:
Standardinform  Publs. 10~p.

\noindent
\textbf{Описание патента:}

\Aue{Bolshakov, M.\,V., A.\,V.~Kulakov, A.\,N.~Lavrenov, and M.\,V.~Palkin}.
2006. Sposob orientirovaniya po krenu letatel'nogo
apparata s opti\-che\-skoy golovkoy
samonavedeniya [The way to orient on the roll of aircraft with optical homing head].
Patent RF No.\,2280590.
}

\item Присланные в редакцию материалы авторам не возвращаются.\\[-13.5pt]

\item При отправке файлов по электронной почте просим придерживаться следующих
правил:
\begin{itemize}
\item указывать в поле subject (тема) название журнала и фамилию автора;\\[-13.5pt]
\item указывать в тексте письма название статьи, авторов и~журнал, в~который направляется статья;\\[-13.5pt]
\item использовать attach (присоединение);\\[-13.5pt]
\item в состав электронной версии статьи должны входить: файл, содержащий текст
статьи, и файл(ы), содержащий(е) иллюстрации.\\[-13.5pt]
\end{itemize}

\item Журнал <<Информатика и её применения>> является некоммерческим изданием.
Плата за публикацию не взимается, гонорар авторам не выплачивается.
\end{enumerate}



\def\leftfootline{\small{\textbf{\thepage}
\hfill ИНФОРМАТИКА И ЕЁ ПРИМЕНЕНИЯ\ \ \ том\ 18\ \ \ выпуск\ 3\ \ \ 2024}
}%
 \def\rightfootline{\small{ИНФОРМАТИКА И ЕЁ ПРИМЕНЕНИЯ\ \ \ том\ 18\ \ \ выпуск\ 3\ \ \ 2024
\hfill \textbf{\thepage}}}


\vspace*{-1mm}

\begin{center}

\textbf{Адрес редакции журнала <<Информатика и её применения>>:} \\




Москва 119333, ул.~Вавилова, д.~44, корп.~2, ФИЦ ИУ РАН\\[-10pt]

\

Тел.: +7\,(499)\,135-86-92\ \ Факс:  +7\,(495)\,930-45-05\\[-10pt]

 \

e-mail:   {\sf iiep@frccsc.ru} (Стригина Светлана Николаевна)\\[-10pt]

\

{\sf http://www.ipiran.ru/journal/issues/}
\end{center}
}


\def\leftkol{Правила подготовки рукописей  для публикации в журнале
<<Информатика и её применения>>}

\def\rightkol{Правила подготовки рукописей  для публикации в журнале
<<Информатика и её применения>>}


\def\leftfootline{\small{\textbf{\thepage}
\hfill ИНФОРМАТИКА И ЕЁ ПРИМЕНЕНИЯ\ \ \ том\ 18\ \ \ выпуск\ 3\ \ \ 2024}
}%
 \def\rightfootline{\small{ИНФОРМАТИКА И ЕЁ ПРИМЕНЕНИЯ\ \ \ том\ 18\ \ \ выпуск\ 3\ \ \ 2024
\hfill \textbf{\thepage}}} 
\def\stat{podg-e}
{%\hrule\par
%\vskip 7pt % 7pt
\vspace*{-24pt}
\raggedleft\Large \bf%\baselineskip=3.2ex
Requirements for manuscripts submitted to Journal
``Informatics~and~Applications'' \vskip 8pt
    \hrule
    \par
\vskip 21pt plus 6pt minus 3pt }

\label{st\stat}

\def\tit{\ }

\def\aut{\ }
\def\auf{\ }

\def\leftkol{\ }

\def\rightkol{\ }
%Requirements for manuscripts submitted to Journal
%``Informatics~and~Applications''}

\titele{\tit}{\aut}{\auf}{\leftkol}{\rightkol}

\def\leftfootline{\small{\textbf{\thepage}
\hfill INFORMATIKA I EE PRIMENENIYA~--- INFORMATICS AND APPLICATIONS\ \ \ 2019\
\ \ volume~13\ \ \ issue\ 4}
}%
 \def\rightfootline{\small{INFORMATIKA I EE PRIMENENIYA~--- INFORMATICS AND APPLICATIONS\ \ \ 2019\ \ \ volume~13\ \ \ issue\ 4
\hfill \textbf{\thepage}}}

\vspace*{-60pt}

{\small

\noindent
Journal ``Informatics and Applications'' (Inform.\ Appl.)
publishes theoretical, review, and discussion
articles on the research and development in the
field of informatics and its applications.

The journal is published in Russian.
By a special decision of the editorial
board, some articles can be published in English.


The topics covered include the following areas:
\begin{itemize}
               \item
     theoretical fundamentals of informatics; \\[-14pt]
\item
mathematical methods for studying complex systems and processes; \\[-14pt]
\item
information systems and networks;\\[-14pt]
\item
information technologies; and \\[-14pt]
\item
architecture and software of computational complexes and networks. \\[-14pt]
\end{itemize}

\noindent
\begin{enumerate}[1.]
\item The Journal publishes original articles which have not been published before and are not
intended for simultaneous publication in other editions. An article submitted to the Journal must not violate the
Copyright law. Sending the manuscript to the Editorial Board, the authors retain all rights of the
owners of the manuscript and transfer the nonexclusive rights to publish the article in Russian
(or the language of the article, if not Russian) and its distribution in Russia and abroad to the
Founders and the Editorial Board. Authors should submit a letter to the Editorial Board in the
following form:

{\bfseries\textit{Agreement on the transfer of rights to publish:}}

``\textit{We, the undersigned authors of the manuscript ``\ldots'', pass to the
Founder and the Editorial Board of the Journal ``Informatics and Applications''
the nonexclusive right to publish the manuscript of the article in Russian (or
in English) in both print and electronic versions of the Journal. We affirm
that this publication does not violate the Copyright of other persons or
organizations.}

\textit{Author(s) signature(s): (name(s), address(es), date).}

This agreement should be submitted in paper form or in the form of a scanned copy (signed by
the authors).


%The Editorial Board has the right to request from the authors an official expert conclusion that
%the submitted article has no secret data prohibited for publication. \\[-13.5pt]
\item
A submitted article should be attached with \textbf{the data on the author(s)} (see item~8). If
there are several authors, the contact person should be indicated who is responsible for
correspondence with the Editorial Board and other authors about revisions and final approval
of the proofs.\\[-13.5pt]

\item The Editorial Board of the Journal examines the article according to the established
reviewing procedure. If the authors receive their article for correction after reviewing, it does not
mean that the article is approved for publication. The corrected article should be sent to the
Editorial Board for the subsequent review and approval.\\[-13.5pt]

\item The decision on the article publication or its rejection is communicated to the authors. The
Editorial Board may also send the reviews on the submitted articles to the authors. Any
discussion upon the rejected articles is not possible.\\[-13.5pt]

\item The edited articles will be sent to the authors for proofread. The comments of the authors
to the edited text of the article should be sent to the Editorial Board as soon as possible.\\[-13.5pt]

\item The manuscript of the article should be presented electronically in the MS WORD (.doc or
.docx) or \LaTeX\ (.tex) formats, and additionally in the .pdf format. All documents
 may be sent
by e-mail or provided on a CD or diskette. A~hard copy submission is not necessary.\\[-13.5pt]

\item The recommended typesetting instructions for manuscript.

Pages parameters: format A4, portrait orientation, document margins (cm): left~--- 2.5, right~---
1.5, above~--- 2.0, below~--- 2.0, footer 1.3.

Text: font~---Times New Roman, font size~--- 14, paragraph indent~--- 0.5, line spacing~--- 1.5,
justified alignment.

The recommended manuscript size: not more than 15~pages of the specified format.
If the specified size exceeded, the editorial board is entitled to require the author
to reduce the manuscript.

Use only standard abbreviations. Avoid  abbreviations in the title and
abstract. The full term for which an abbreviation stands should precede
its first use in the text unless it is a standard unit of measurement.

All pages of the manuscript should be numbered.

The templates for the manuscript typesetting are presented on site: {\sf
http://www.ipiran.ru/journal/template.doc}.\\[-13.5pt]


%\def\leftkol{Requirements for manuscripts submitted to Journal
%``Informatics~and~Applications''}

\item The articles should enclose data both in \textbf{Russian and English}:
\begin{itemize}
\item title;\\[-13.5pt]
\item author's name and surname;\\[-13.5pt]
\item affiliation~--- organization, its address with ZIP code, city, country, and
official e-mail address;\\[-13.5pt]
\item data on authors according to the format: (see site)

{\sf http://www.ipiran.ru/journal/issues/2013\_07\_01/authors.asp}  and

{\sf  http://www.ipiran.ru/journal/issues/2013\_07\_01\_eng/authors.asp};\\[-13.5pt]

\pagebreak

\def\leftfootline{\small{\textbf{\thepage}
\hfill INFORMATIKA I EE PRIMENENIYA~--- INFORMATICS AND APPLICATIONS\ \ \ 2019\
\ \ volume~13\ \ \ issue\ 4}
}%
 \def\rightfootline{\small{INFORMATIKA I EE PRIMENENIYA~--- INFORMATICS AND APPLICATIONS\ \ \ 2019\ \ \ volume~13\ \ \ issue\ 4
\hfill \textbf{\thepage}}}


%\def\leftkol{Requirements for manuscripts submitted to Journal
%``Informatics~and~Applications''}

%\def\rightkol{Requirements for manuscripts submitted to Journal
%``Informatics~and~Applications''}



\item abstract (not less than 100 words) both in Russian and in English. Abstract is a short
summary of the article that can be published separately. The abstract is the
main source of information on the article and it could be included in leading information
systems and data bases. The abstract in English has to be an original text and should
not be an exact translation of the Russian one. Good English is required.
In abstracts, avoid references and formulae;\\[-13.5pt]
\item indexing is performed on the basis of keywords. The use of keywords from the
internationally accepted thematic Thesauri is recommended.

%\def\leftkol{Requirements for manuscripts submitted to Journal
%``Informatics~and~Applications''}

%\def\rightkol{Requirements for manuscripts submitted to Journal
%``Informatics~and~Applications''}

Important! Keywords must not be sentences;
\item Acknowledgments.
\end{itemize}

\item References. Russian references have to be presented both in English translation and Latin
transliteration (refer {\sf http://www.translit.net/ru/bgn/}).

Please take into account the following examples of Russian references appearance:

\noindent
\textbf{Article in journal:}

\Aue{Zhang, Z., and D.~Zhu}. 2008. Experimental research on the localized electrochemical
micromachining.
\textit{Rus. J.~Electrochem.}  44(8):926--930. {\sf doi:10.1134/S1023193508080077}.


\noindent
\textbf{Journal article in electronic format:}

\Aue{Swaminathan, V., E.~Lepkoswka-White, and B.\,P.~Rao}. 1999. Browsers or buyers in
cyberspace? An
investigation of electronic factors influencing electronic exchange. \textit{JCMC}
5(2). Available at: {\sf http://www.ascusc.org/jcmc/vol5/issue2/} (accessed April~28, 2011).




\noindent
\textbf{Article from the continuing publication (collection of works, proceedings):}

\Aue{Astakhov, M.\,V., and T.\,V.~Tagantsev}. 2006. Eksperimental'noe
issledovanie prochnosti soedineniy ``stal'--kompozit'' [Experimental study of
the strength of joints ``steel--composite'']. \textit{Trudy MGTU
``Matematicheskoe modelirovanie slozhnykh tekh\-ni\-che\-skikh sistem''}
[\textit{Bauman MSTU ``Mathematical Modeling of Complex Technical
Systems'' Proceedings}]. 593:125--130.

\def\leftfootline{\small{\textbf{\thepage}
\hfill INFORMATIKA I EE PRIMENENIYA~--- INFORMATICS AND APPLICATIONS\ \ \ 2019\
\ \ volume~13\ \ \ issue\ 4}
}%
 \def\rightfootline{\small{INFORMATIKA I EE PRIMENENIYA~--- INFORMATICS AND APPLICATIONS\ \ \ 2019\ \ \ volume~13\ \ \ issue\ 4
\hfill \textbf{\thepage}}}

\def\leftkol{Requirements for manuscripts submitted to Journal
``Informatics~and~Applications''}

\def\rightkol{Requirements for manuscripts submitted to Journal
``Informatics~and~Applications''}

\noindent
\textbf{Conference proceedings:}

\Aue{Usmanov, T.\,S., A.\,A.~Gusmanov, I.\,Z.~Mullagalin, R.\,Ju.~Muhametshina,
A.\,N.~Chervyakova, and
A.\,V.~Sveshnikov}. 2007. Osobennosti proektirovaniya razrabotki mestorozhdeniy
s primeneniem gidrorazryva
plasta [Features of the design of field development with the use of hydraulic fracturing].
\textit{Trudy 6-go
Mezhdu\-na\-rod\-no\-go Simpoziuma ``Novye resursosberegayushchie tekhnologii
nedropol'zovaniya i povysheniya
neftegazootdachi''} [\textit{6th  Symposium (International) ``New Energy Saving Subsoil
Technologies and
the Increasing of the Oil and Gas Impact'' Proceedings}]. Moscow. 267--272.


\noindent
\textbf{Books and other monographs:}




Lindorf, L.\,S., and L.\,G.~Mamikoniants, eds. 1972.
\textit{Ekspluatatsiya turbogeneratorov s neposredstvennym
okhlazhdeniem} [\textit{Operation of turbine generators with direct cooling}].
Moscow: Energy Publs. 352~p.


%\Aue{Latyshev, V.\,N.} 2009. \textit{Tribologiya rezaniya. Kn.~1: Frikcionnye prosessy
%pri rezanii metallov}
%[\textit{Tribology of cutting. Vol.~1: Frictional processes in metal cutting}]. Ivanovo: Ivanovskii
%State Univ. 108~p.


%\noindent
%\textbf{Unpublished material:}

%\Aue{Latypov, A.\,R., M.\,M.~Khasanov, and V.\,A.~Baikov}.
%2004. Geology and production (NGT GiD). Certificate on official registration of the computer
%program
%No.\,2004611198. (In Russian, unpubl.)

%\noindent
%\textbf{Internet-source:}

%APA Style. 2011. Available at: {\sf http://www.apastyle.org/apa-style-help.aspx} (accessed
%February~5, 2011).

%Pravila citirovaniya istochnikov [Rules for the citing of sources]. Available at: {\sf
%http://www.scribd.com/doc/1034528/} (accessed February~7, 2011).


\noindent
\textbf{Dissertation and Thesis:}

%\Aue{Semenov, V.\,I.}
%2003. Matematicheskoe modelirovanie plazmy v sisteme kompaktnyy tor. [Mathematical
%modeling of the plasma in the compact torus]. D.Sc.\ Diss. Moscow. 272~p.

\Aue{Kozhunova, O.\,S.} 2009. Tekhnologiya razrabotki semanticheskogo
slovarya informatsionnogo monitoringa [Technology of development of
semantic dictionary of information monitoring system]. PhD Thesis. Moscow: IPI RAN. 23~p.


\noindent
\textbf{State standards and patents:}

GOST 8.586.5-2005. 2007. Metodika vypolneniya izmereniy. Izmerenie raskhoda i~kolichestva
zhidkostey i gazov 
s~pomoshch'yu standartnykh suzhayushchikh ustroystv [Method of measurement.
Measurement of flow rate and volume of liquids and gases by means of orifice devices]. M.:
Standardinform
Publs. 10~p.

%\noindent
%\textbf{Patent:}

\Aue{Bolshakov, M.\,V., A.\,V.~Kulakov, A.\,N.~Lavrenov, and M.\,V.~Palkin}.
2006. Sposob orientirovaniya po krenu letatel'nogo
apparata s opti\-che\-skoy golovkoy
samonavedeniya [The way to orient on the roll of aircraft with optical homing head].
Patent RF No.\,2280590.

References in Latin transcription are presented in the original language.

References in the text are numbered according to the order of their
first appearance; the number is
placed in square brackets. All items from the reference list should be
cited.\\[-13.5pt]

\item Manuscripts and additional materials are not returned to Authors by the Editorial Board.\\[-13.5pt]

\item Submissions of files by e-mail must include:\\[-13.5pt]
\begin{itemize}
\item   the journal title and author's name in the ``Subject'' field; \\[-13.5pt]
\item   an article and additional materials have to be attached using the ``attach'' function;\\[-13.5pt]
\item   an electronic version of the article should contain the file with the text and a separate file
with figures.\\[-13.5pt]
\end{itemize}

\item ``Informatics and Applications'' journal is not a profit publication. There are no
charges for the authors as well as there are no royalties.\\[-13.5pt]
\end{enumerate}

\def\leftfootline{\small{\textbf{\thepage}
\hfill INFORMATIKA I EE PRIMENENIYA~--- INFORMATICS AND APPLICATIONS\ \ \ 2019\
\ \ volume~13\ \ \ issue\ 4}
}%
 \def\rightfootline{\small{INFORMATIKA I EE PRIMENENIYA~--- INFORMATICS AND APPLICATIONS\ \ \ 2019\ \ \ volume~13\ \ \ issue\ 4
\hfill \textbf{\thepage}}}

\def\leftkol{Requirements for manuscripts submitted to Journal
``Informatics~and~Applications''}

\def\rightkol{Requirements for manuscripts submitted to Journal
``Informatics~and~Applications''}


%\vspace*{5mm}


\begin{center}
\textbf{Editorial Board address:} \\

%ABOUT AUTHORS



FRC CSC RAS, 44, block~2, Vavilov Str., Moscow 119333, Russia\\[-10pt]

\

Ph.: +7\,(499)\,135\,86\,92,\ \ Fax: +7\,(495)\,930\,45\,05\\[-10pt]

\

 e-mail: {\sf rust@ipiran.ru} (to Prof.\ Rustem Seyful-Mulyukov)\\[-10pt]

\

 {\sf http://www.ipiran.ru/english/journal.asp}
\end{center}
 }
%\thispagestyle{myheadings}

\def\leftkol{Requirements for manuscripts submitted to Journal
``Informatics~and~Applications''}

\def\rightkol{Requirements for manuscripts submitted to Journal
``Informatics~and~Applications''}

\def\leftfootline{\small{\textbf{\thepage}
\hfill INFORMATIKA I EE PRIMENENIYA~--- INFORMATICS AND APPLICATIONS\ \ \ 2019\
\ \ volume~13\ \ \ issue\ 4}
}%
 \def\rightfootline{\small{INFORMATIKA I EE PRIMENENIYA~--- INFORMATICS AND APPLICATIONS\ \ \ 2019\ \ \ volume~13\ \ \ issue\ 4
\hfill \textbf{\thepage}}}

 \label{end\stat}

\newpage



%\include{ipi-ind}

%\tableofcontents

\end{document}





%%%%%%%%%%%%%%%%%%%%%%

%\newcommand{\Ack}{\subsection*{\protect\large\bf Acknowledgments}}

%\vphantom*{\int\limits_0^T}

{ \begin{center}  %fig1
 \vspace*{6pt}
    \mbox{%
 \epsfxsize=79mm 
 \epsfbox{gru-1.eps}
 }

\end{center}



\noindent
{{\figurename~1}\ \ \small{
}}}

%\vspace*{6pt}

\addtocounter{figure}{1}