\def\da{(\Delta,A)}



\def\stat{sinitsin}

\def\tit{СТАТИСТИЧЕСКОЕ МОДЕЛИРОВАНИЕ ДИФФЕРЕНЦИАЛЬНЫХ~СТОХАСТИЧЕСКИХ СИСТЕМ,  НЕ~РАЗРЕШЕННЫХ~ОТНОСИТЕЛЬНО ПРОИЗВОДНЫХ}

\def\titkol{Статистическое моделирование дифференциальных СтС, %стохастических систем,  
не~разрешенных относительно производных}

\def\aut{И.\,Н.~Синицын$^1$}

\def\autkol{И.\,Н.~Синицын}

\titel{\tit}{\aut}{\autkol}{\titkol}

\index{Синицын И.\,Н.}
\index{Sinitsyn I.\,N.}


%{\renewcommand{\thefootnote}{\fnsymbol{footnote}} \footnotetext[1]
%{Работа 
%выполнена при поддержке Программы развития МГУ, проект №\,23-Ш03-03. При анализе 
%данных использовалась инфраструктура Центра коллективного пользования 
%<<Высокопроизводительные вычисления и~большие данные>> 
%(ЦКП <<Информатика>>) ФИЦ ИУ РАН (г.~Москва)}}


\renewcommand{\thefootnote}{\arabic{footnote}}
\footnotetext[1]{Федеральный исследовательский центр <<Информатика и~управление>> Российской академии наук; Московский авиационный 
институт, \mbox{sinitsin@dol.ru}}


\vspace*{-12pt}



%Работа 
%выполнена при финансовой поддержке РАН (ГЗ 2024--2028~гг.).






\Abst{Статья посвящена методическому обеспечению статистического моделирования 
дифференциальных стохастических сис\-тем (СтС), не разрешенных относительно 
производных (НРОП). Дан обзор работ в~области аналитического моделирования 
стохастических процессов (СтП) в~СтС НРОП. Получены две теоремы приведения 
стохастических функ\-ци\-о\-наль\-но-диф\-фе\-рен\-ци\-аль\-ных уравнений к~дифференциальным. 
Изложен эйлеров метод аппроксимации для дифференциальных уравнений Ито 
с~гауссовскими и~пуассоновскими шумами. Представлены три теоремы, уточ\-ня\-ющие 
алгоритмы Эйлера численного интегрирования в~рамках сильной и~слабой 
аппроксимации распределений. В их основу положена обобщенная формула Ито для 
дифференцирования нелинейных функций, гауссовских и~пуассоновских шумов, а~также 
формулы для плотностей распределений соответствующих случайных величин при 
аппроксимации соответствующих интегралов. Особое внимание уделено подходам к~аппроксимации стохастических интегралов в~случае негладких функций в~СтС НРОП. 
Проведен методологический анализ уточняющих алгоритмов различной степени 
точности для детерминированных и~случайных со\-став\-ля\-ющих. Представлены выводы и~предложены направления дальнейших исследований.}

\KW{аналитическое моделирование; методическое обеспечение; %сильная и~слабая аппроксимации;
сис\-те\-ма, стохастически не разрешенная относительно производных; статистическое моделирование;
стохастическая сис\-те\-ма, не разрешенная относительно производной (СтС НРОП)} %; стохастический процесс (СтП)}

\DOI{10.14357/19922264240302}{WWMEOT}
  
%\vspace*{-6pt}


\vskip 10pt plus 9pt minus 6pt

\thispagestyle{headings}

\begin{multicols}{2}

\label{st\stat}

\section{Введение}

В~[1--4] рассмотрены вопросы аналитического моделирования процессов в~СтС НРОП. 
Особое внимание в~них уделено нормальным (гауссовским) СтП. В~[5] предложены методы нормализации сис\-тем, сто\-ха\-сти\-че\-ски НРОП. 
Тео\-рия распределений с~инвариантной мерой в~СтС НРОП 
развита в~[6].
В~[7] дано обоб\-ще\-ние~[1--4] на случай СтС НРОП со случайными параметрами. Для 
СтС НРОП, до\-пус\-ка\-ющих приведение к~сис\-те\-ме дифференциальных и~конечных 
сто\-ха\-сти\-че\-ских уравнений, могут быть использованы известные методы чис\-лен\-но\-го 
интегрирования~[8--10]. В~[11, 12] предложен ряд чис\-лен\-ных методов 
интегрирования сто\-ха\-сти\-че\-ских дифференциальных уравнений, основанных на 
использовании обобщенной формулы Ито, а~так\-же канонических разложениях СтП.

Рассмотрим методическое обеспечение чис\-ленного интегрирования уравнений 
диф\-фе\-рен\-ци\-альных СтС НРОП, приводимых к~диф\-фе\-рен\-ци\-альным СтС. В~разд.~2 даны 
сведения из теории\linebreak \mbox{приведения} функ\-ци\-о\-наль\-но-диф\-фе\-рен\-ци\-аль\-ных уравнений 
к~дифференциальным. Разделы~3 и~4 содержат основные результаты для
приведенных СтС НРОП. В~разд.~5 приведены основные выводы, а~также обобщения 
на случай общих неявных детерминированных и~стохастических непрерывных, 
дискретных и~не\-пре\-рыв\-но-дис\-крет\-ных сис\-тем.


\section{Дифференциальные стохастические системы, не~разрешенные относительно~производных }

Рассмотрим сначала дифференциальную СтС с~нелинейностями, описываемыми гладкими 
детерминированными скалярными функциями:
   \begin{multline}
   \Phi =\Phi(t, Y_t, \dot Y_t \tr Y_t^{(k)}, U_t) =0, \enskip Y(t_0) = Y_0, 
\\
 \dot Y(t_0) = \dot Y_0 \tr Y^{(k)} (t_0) = Y_0^{(k)}.
\label{e2.1-s}
\end{multline}
При этом уравнение  нелинейного формирующего фильтра (ФФ) для стохастических 
возмущений~$U_t$ возьмем в~виде, разрешенном относительно возмущений
\begin{equation}
\dot U_t = a^U (t, U_t)  + b^U (t, U_t) V_t^U, \enskip U(t_0) = U_0.
\label{e2.2-s}
\end{equation}
Здесь $a^U \hm= a^U(t, U_t)$ и~$b^U(t, U_t)$~--- $(n^Y\times 1)$- и~$(n^Y\times n^V)$-мер\-ные функции; $V_t^U$~--- белый шум в~строгом смысле~\cite{11-s}, допускающий 
представление вида
\begin{equation*}
V_t^U = \dot W_t^U, \enskip W_t^U = W_0^U (t) + \iii_{R_0^q} c^U (\rho) 
P^0 ( t, d\rho),
%\label{e2.3-s}
\end{equation*}
где $\nu_t$--- его интенсивность:
    \begin{equation*}
    \nu_t=\nu_t^W =\nu_t^{W_0} +  \iii_{R_0^q} c^U (\rho) \left[c^U (\rho)\right]^{\mathrm{T}} \nu_P 
(t,\rho) \,d\rho\,;
%\label{e2.4-s}
\end{equation*}
 $c^U \hm= c^U(\rho)$~--- известная векторная функция той же 
размерности, что и~$W_0^U$; интеграл при любом $t\hm\ge t_0$ представляет собой 
стохастический интеграл по центрированной пуассоновской мере~$P^0(t,{\cal A})$, 
независимой от~$W_0^U$ и~име\-ющей независимые значения на попарно 
непересекающихся множествах; ${\cal A}$~--- борелевское множество пространства 
$R_0^q$ с~выколотым началом; $\nu_t^W$, $\nu_t^{W_0}$ и~$\nu_P$~--- интенсивности СтП 
$W_t^U$, $W_{0}^U$ и~$P^0$. Уравнение~(\ref{e2.2-s}) понимается в~смысле Ито и~имеет 
единственное решение в~среднем квадратичном~\cite{11-s}.


Для гладких функций в~(\ref{e2.1-s}), допускающих стохастические производные Ито до $h$-го порядка, выполним  следующие преобразования.
Будем дифференцировать сполна по~$t$ левые части уравнений~(\ref{e2.1-s}) по обобщенной 
формуле Ито~\cite{11-s, 12-s} до тех пор, пока не появятся производные белого шума. 
В~результате получим следующую сис\-те\-му нелинейных дифференциальных уравнений:
    \begin{equation}
    \Phi=0,\enskip \dot \Phi =0\tr \Phi^{(h)}=0\,.
    \label{e2.5-s}
    \end{equation}
Далее введем вектор  $Z_t \hm= \lk Y_t^{\mathrm{T}}\, X_t^{\mathrm{T}}\rk^{\mathrm{T}}$, составленный из $Y_t\hm= \lk 
Y_t^{\mathrm{T}}\, \dot Y_t^{\mathrm{T}}\cdots Y_t^{(k-1)\mathrm{T}}\rk^{\mathrm{T}}$ и~вспомогательного вектора~$X_t$, 
определяемого уравнениями~(\ref{e2.5-s}).
В~результате придем к~уравнениям, разрешенным относительно дифференциалов, 
следующего вида:
\begin{equation}
dZ_t = a^Z  dt + b^Z d W_0 + \iii_{R_0^q} c^Z P^0(t,du),\label{e2.6-s}
\end{equation}
где $a^Z \hm= a^Z(t,Z_t)$; $b^Z\hm=b^Z(t,Z_t)$; $ c^Z\hm=c^Z(t,Z_t, u)$.

Таким образом, имеем следующее утверждение.

\smallskip

\noindent
\textbf{Теорема~2.1.}\ \textit{Пусть нелинейная негауссовская СтС}~(\ref{e2.1-s}), (\ref{e2.2-s}), 
\textit{НРОП $k$-го порядка, удовлетворяет условиям}:
\begin{itemize}
\item[\,] $1^0$~\textit{функции}~(\ref{e2.1-s}) \textit{допускают обобщенные стохастические дифференциалы Ито вплоть 
до $h$-го порядка включительно};

\item[\,] $2^0$~\textit{уравнение ФФ}~(\ref{e2.2-s}) \textit{разрешено относительно возмущений~$U_t$, имеет 
единственное среднеквадратичное решение}.
\end{itemize}
\textit{Тогда система}~(\ref{e2.1-s}), (\ref{e2.2-s}) \textit{приводима к~сис\-те\-ме, разрешенной относительно 
производных}~(\ref{e2.6-s}).

\smallskip


Положим, что в~(\ref{e2.1-s}) стохастическое возмущение~$U_t$ представляет собой 
автокоррелированный СтП и~описывается стохастическим дифференциальным уравнением 
следующего ФФ:
\begin{multline}
U_t^{(n)} + \sss_{k=n-m}^{n-1}\alpha_k^U \left(t,U_t, \dot U_t \tr U_t^{(n-m-1)}\right) U_t^{(k)} + {}\\
{}+
\alpha_0^U \left(t,  U_t, \dot U_t \tr U_t^{(n-1)}\right) ={}\\
{}= \sss_{h=1}^m \beta_h^U \left(t,  U_t, \dot U_t \tr U_t^{(n-m-1)}\right)V_t^{(h)}.\label{e2.7-s}
\end{multline}
Здесь $n=n^U$~--- порядок дифференциального уравнения, причем $0\hm<m\hm<n$; 
$\alp_0^U$, $\alp_k^U$ и~$\beta_k^U$~--- известные функции отмеченных переменных.

Пусть компоненты СтП $Y_t, \dot Y_t\tr Y_t^{(k)}$ более гладкие, чем~$U_t$. 
В~этом случае для приведения~(\ref{e2.1-s}), (\ref{e2.7-s}) к~(\ref{e2.6-s}) можно применить способ, 
основанный на дифференцировании~(\ref{e2.1-s}) и~исключении~$U_t$ и~ее производных, не 
содержащих белого шума из уравнений~(\ref{e2.1-s}) и~уравнений, полученных из него 
дифференцированием~(\ref{e2.7-s}) по обобщенной формуле Ито~\cite{11-s, 12-s}. В~результате придем к~следующему уравнению вида~(\ref{e2.6-s}):
    \begin{equation}
    d\bar Z_t = \bar a^Z dt + \bar b^Z dW_0 + \iii_{R_0^q} \bar c^Z P^0 (t, du),\label{e2.8-s}
    \end{equation}
где $\bar a^Z = \bar a^Z(t,\bar Z_t)$; $\bar b^Z \hm= \bar b^Z(t, \bar Z_t)$; $\bar c^Z \hm= \bar c^Z(t, \bar Z_t, u)$.

Аналогично~[1--6] рассматривается случай векторных дифференциальных СтС НРОП.

Таким образом, получаем следующий результат.

\smallskip

\noindent
\textbf{Теорема~2.2.}\ \textit{Пусть нелинейная негауссовская СтС}~(\ref{e2.1-s}), (\ref{e2.7-s}) \textit{удовлетворяет условиям}:
\begin{itemize}
\item[\,]
 $1^0$~\textit{функции}~(\ref{e2.1-s}) \textit{допускают обобщенные стохастические дифференциалы Ито 
вплоть до $h$-го порядка включительно};
\item[\,]
 $2^0$~\textit{уравнение ФФ}~(\ref{e2.7-s}) \textit{имеет единственное среднеквадратичное решение};
\item[\,]
 $3^0$~\textit{СтП $Y_t, \dot Y_t\tr Y_t^{(k)}$ более гладки, чем возмущение}~$U_t$.
\end{itemize}
\textit{Тогда система}~(\ref{e2.1-s}), (\ref{e2.7-s}) \textit{приводима к~системе}~(\ref{e2.8-s}).


Для гладких  век\-тор-функ\-ций~(\ref{e2.5-s}) конечные уравнения в~(\ref{e2.1-s}) допускают гладкую 
замену переменных\linebreak и~приведение исходных дифференциальных уравнений СтС НРОП 
к~сис\-те\-ме, со\-сто\-ящей из дифференциального векторного дифференциального 
\mbox{стохастического} уравнения Ито и~конечного векторного уравнения вида
    \begin{multline}
    dY_t = a^Y (t,Y_t)\, dt + b^Y (t,Y_t)\, d W_0 + {}\\
    {}+\iii_{R_0^q} c^Y (t,Y_t, u) 
P^0 (t,du)
\label{e2.9-s}
\end{multline}
и конечным уравнениям вида
  \begin{equation}
  \Psi(Y_t, X_t, t) =0\,.
  \label{e2.10-s}
  \end{equation}

\section{Метод Эйлера}

Следуя~\cite{11-s, 12-s}, заменим интеграл по переменной~$u$ в~(\ref{e2.9-s})
соответствующей интегральной суммой. В~результате~(\ref{e2.9-s})
заменится уравнением
\begin{equation}
dY=a(Y,t)\,dt+b(Y,t)\,dW_0+\!\sum\limits_{i=1}^N c_i(Y,t)\,dP^0_i,\label{e3.1-s}
\end{equation}
где $c_i(y,t)$~--- $p$-мер\-ные векторные функции, представляющие собой
значения функции $c(y,t,v)$ в~некоторых средних точках~$u_i$
соответствующих элементов~$A_i$ разбиения $r$-мер\-но\-го шара достаточно
большого радиуса; $u_i\hm\in A_i$ ($i\hm=\overline{1,N}$); $P^0_i(t)$~---
центрированные простые пуассоновские СтП:
    \begin{equation*}
    P^0_i(t)=P^0([0,t),A_i)-\mu([0,t),A_i),\quad i=\overline{1,N}\,.
  %  \label{e(3.2-s}
    \end{equation*}
Интенсивности этих СтП определяются через математическое
ожидание~$\mu\da$  пуассоновской меры~$P\da$ по формуле
\begin{equation*}
\nu_i(t)=\fr{d\mu([\,0,t),A_i)}{dt}\,.
%\label{e3.3-s}
\end{equation*}
Простейший способ замены уравнения~(\ref{e3.1-s}) разностным уравнением состоит в~замене всех дифференциалов
элементами интегральных сумм:
\begin{multline*}
Y((n+1)h)-Y(nh)=a(Y(nh),nh)h+{}\\
{}+
b(Y(nh,nh)) \left[W_0((n+1)h)- W_0(nh)\right]+{}\\
{}+\sum\limits_{i=1}^N c_i(Y(nh),nh)\left[ P_i^0((n+1)h)-P_i^0(nh) \right].
    \end{multline*}
Положив
\begin{align*}
    \bar Y_n&=Y(nh);\\
     \varphi_n(\bar Y_n)&=Y(nh)+a(Y(nh),nh)h; \\
  \psi_{1n}(\bar Y_n)&=b(Y(nh),nh);\\
    \psi_{in}(\bar    Y_n)&=c_{i-1}(Y(nh),nh);\\
     V_{1n}&=W_0((n+1)h)-W_0(nh);\\
    V_{in}&=P_{i-1}^0((n+1)h)-P_{i-1}^0(nh),\  i=\overline{2,N+1}, \\[-20pt]
    %    \label{e3.4-s}
    \end{align*}
получим стохастическое разностное уравнение
    $$
    \bar Y_{n+1}=\varphi_n(\bar Y_n)+\sum\limits_{i=1}^{N+1}\psi_{{in}}(\bar Y_n)V_{in}.
$$

\vspace*{-2pt}

\noindent
Вводя блочную матрицу $p\times(m+N)$
    \begin{equation*}
    \psi_n(\bar Y_n)=\left[\psi_{1n}(\bar Y_n) \cdots \psi_{N+1,n}(\bar
    Y_n)\right] %\label{e3.5-s}
    \end{equation*}
    
    \vspace*{-2pt}

\noindent
и $(m+N)$-мерный случайный вектор
    \begin{equation*}
    V_n=\left[ V_{1n}^{\mathrm{T}}\,\, V_{2n}\,\,\cdots \,\, V_{N+1,n}\right]^{\mathrm{T}},
    %\label{e3.6-s}
    \end{equation*}
    
    \vspace*{-2pt}

\noindent
можем записать полученное разностное уравнение коротко в~виде
    \begin{equation}
    \bar Y_{n+1}=\varphi_n(\bar Y_n)+\psi_n(\bar Y_n)V_n. \label{e3.7-s}
    \end{equation}

\vspace*{-2pt}


Случайные векторы~$V_n$ образуют
последовательность независимых случайных векторов~$\{V_n\}$, причем
блоки~$V_{1n}$ векторов~$V_n$ имеют нормальное (гауссовское) распределение 
$\aleph(0,\bar G_n)$, где
    \begin{equation*}
    \bar G_n =\inh \nu_0(\tau)\,d\tau \cong \nu_0(nh)h; %\label{e3.8-s}
    \end{equation*}
    
    \vspace*{-2pt}

\noindent
$\nu_0(t)$~--- интенсивность винеровского СтП $W_0(t)$;
скалярные блоки $V_{{in}}$ ($i\hm=\overline{2,N+1}$) имеют пуассоновские
распределения с~параметрами
\begin{equation*}
\mu_{{in}}=\inh \nu_i(\tau)\,d\tau \cong \nu_i(nh)h\,. %\label{e3.9-s}
\end{equation*}

\vspace*{-2pt}


Ковариационная матрица~$G_n$ вектора~$V_n$
представляет собой блоч\-но-диа\-го\-наль\-ную мат\-рицу:
\begin{equation*}
G_n=
\begin{bmatrix}
    \bar G_n &0 &0 &\cdots &0\\
    0 &\mu_{2n} &0 &\cdots &0\\
    0&0&\mu_{3n} &0&0\\
    \vdots &\vdots &\ddots &\ddots &\vdots\\
    0 &0 &0 &\cdots &\mu_{N+1,n}\\
\end{bmatrix}.
%\label{e3.10-s}
\end{equation*}

\vspace*{-2pt}


Уравнение~(\ref{e3.7-s}) определяет~$\bar Y_{n+1}$ при данном~$\bar Y_n$ с~точ\-ностью до~$h$ в~детерминированном слагаемом $\varphi_n(\bar Y_n)$ и~с~точ\-ностью до~$\sqrt{h}$ в~случайном слагаемом $\psi_n(\bar Y_n)V_n$ (\textbf{тео\-ре\-ма~3.1}).

\vspace*{-6pt}

\section{Более точные методы}

\vspace*{-3pt}

\textbf{4.1.}\ Методы $h^2$ и~$h^{3/2}$ для приведенных уравнений~(\ref{e2.9-s}) и~(\ref{e2.10-s}) 
с~точ\-ностью до~$h^2$ в~детерминиро-\linebreak\vspace*{-12pt}

\pagebreak

\noindent
ванном слагаемом и~$h^{3/2}$ в~случайном 
слагаемом, следуя~\cite{11-s, 12-s}, дают следующую \textbf{тео\-ре\-му~4.1}:
\begin{equation}
\bar Y_{n+1}=\varphi_n(\bar Y_n)+\psi_n(\bar     Y_n,V_n^{(1)})V_n.
\label{e4.1-s}
\end{equation}
Входящие в~уравнения~(\ref{e4.1-s}) величины определяются формулами:

\vspace*{-6pt}

\noindent
 \begin{multline*}
 \!\varphi_n(\bar Y_n)=\bar Y_n+\left[a(\bar Y_n,nh)-\!\sum_{j=1}^N
    c_j(\bar Y_n,nh)\nu_{jn}\, \right]h+{}\\
{}+\fr{1}{2}\left\{a_t(\bar Y_n,nh)-\sum\limits_{j=1}^N c_{jt}(\bar 
Y_n,nh)\nu_{jn}+   {}\right.\\
{}+ \Biggl[ a_y(\bar Y_n,nh)^{\mathrm{T}}-\sum\limits_{j=1}^N c_{jy}(\bar Y_n,hn)^{\mathrm{T}} \nu_{jn} \Biggr]
    \Biggl[ a(\bar Y_n,nh)-{}\\
    {}- \sum\limits_{j=1}^N c_j(\bar Y_n,hn) \nu_{jn}t\,\Biggr]+
    \fr{1}{3}\Biggl[ a_{yy}(\bar Y_n,nh)-{}\\
   \left. {}-\sum\limits_{j=1}^N
    c_{jyy}(\bar Y_n,nh)\nu_{jn} \Biggr]: \sigma(\bar Y_n,nh)\right\}h^2;
%\label{e4.2-s}
\end{multline*}

\vspace*{-12pt}

\noindent
\begin{multline*}
        \psi_{1n}\left(\bar Y_n\right)=b\left(\bar Y_n,nh\right)+\fr{1}{2}\Biggl[ a_y(\bar 
Y_{nh},nh)^{\mathrm{T}}-  {}\\
{}-  \sum\limits_{j=1}^N c_j \left(\bar Y_n,hn\right)^{\mathrm{T}} \nu_{jn}\Biggr] b\left(\bar     Y_n,nh\right);
%\label{e4.3-s}
    \end{multline*}
    
    \vspace*{-12pt}

\noindent
\begin{multline*}
\psi_{{in}}(\bar Y_n)={}\\
{}=c_{i-1}(\bar Y_n,nh)+\Biggl[ \Delta_{i-1}a_n-\sum\limits_{j=1}^N
    \Delta_{i-1}c_{jn}\nu_{jn} \Biggr]h; %\label{e4.4-s}
    \end{multline*}
    
    \vspace*{-12pt}

\noindent
\begin{multline*}
\psi_{1n}'(Y_n,V_n^{(1)})=\Bigg\{b(\bar Y_n,nh)+
    \Biggl[ a(\bar Y_n,nh)^{\mathrm{T}}-{}\\
{}-\sum\limits_{j=1}^N c_j(\bar Y_n,hn)^{\mathrm{T}} \nu_{jn}\,\Biggr]\fr{\partial}{\partial 
y}\, b(\bar Y_n,nh)+{}\\
{}+V_{1n}^{\mathrm{T}} b(\bar Y_n,nh)^{\mathrm{T}} \fr{\partial}{\partial y}\,b(\bar Y_n,nh)
    \Bigg\}h+\sum\limits_{j=1}^N \Delta_j b_{n}V_{j+1,n}; %\label{e4.5-s}
    \end{multline*}
    
    \vspace*{-12pt}

\noindent
\begin{multline*}
\psi_{in}'\left(\bar Y_n,V_n^{(1)}\right)=\Biggl\{ c_{i-1,t}(\bar
    Y_n,nh)+{}\\
    {}+ c_{i-1,y}(\bar Y_n,nh)^{\mathrm{T}}\!
    \Biggl[a(\bar Y_n,nh)-\!\sum_{j=1}^N \!c_j(\bar 
Y_n,hn)\nu_{jn}  \Biggr]\!\Biggr\}h+{}\\
{}+c_{i-1,y}(\bar Y_n,nh)^{\mathrm{T}} b(\bar Y_n,nh)V_{1,n}+\!
\sum\limits_{j=1}^N
    \Delta_j c_{j-1,n}V_{j+1,n};\hspace*{-1pt} %\label{e4.6-s}
    \end{multline*}
    
    \noindent
    \begin{equation*}
\psi_{1n}''(\bar Y_n)=\fr{1}{2}\left[b_{yy}(\bar Y_n,nh):
    \sigma(\bar Y_n,nh)\right]; %\label{e4.7-s}
    \end{equation*}
    
  
\noindent
\begin{equation*}
    \psi_{{in}}''(\bar Y_n)=\fr{1}{2}\left[c_{i-1,yy}(\bar Y_n,nh):
    \sigma(\bar Y_n,nh)\right];
%    \label{e4.8-s}
    \end{equation*}
    
   %  \columnbreak


\noindent
 \begin{equation*}
 V_{1n}=\Delta W_n=W_0((n+1)h)-W_0(nh); %\label{e4.9-s}
 \end{equation*}
% \vspace*{-12pt}
\begin{equation*}
V_{{in}}=\Delta P_{i-1,n}=P_{i-1}((n+1)h)-P_{i-1}(nh),
    \ \  i=\overline{2,N}; %\label{e4.10-s}
    \end{equation*}
    \begin{equation*}
V_{1n}'=\inh \ovth \,dW_0(\tau);
\end{equation*}
\begin{equation*}
V_{1n}''=
    \inh \left(\ovth\right)^2 dW_0(\tau); % \label{e4.11-s}
    \end{equation*}
    \begin{equation}
    \left.
    \begin{array}{rl}
V_{{in}}'&= \displaystyle \inh \ovth\, dP_{i-1}^0(\tau); 
\\[6pt]
  V_{in}''&=\displaystyle\inh \left(\ovth\right)^2 dP_{i-1}^0(\tau),
    \\[12pt]
    &\hspace*{30mm} i=\overline{2,N+1}\,. %\label{e4.13-s}
    \end{array}
    \right\}
    \label{e4.12-s}
    \end{equation}
Здесь введены следующие обозначения для векторных функций:
   \begin{equation*}
   [\varphi_{yy}:\sigma]_k=\mathrm{tr}
\left [\varphi_{kyy}\sigma\right],\enskip \sigma=b\nu_0(t)b^{\mathrm{T}}.
%\label{e4.14-s}
\end{equation*}
Величина
$\varphi_{yy}:\sigma$ в~случае матричной функции $\varphi$
представляет собой матрицу, элементами которой служат следы
произведений на матрицу~$\sigma$
матриц вторых производных соответствующих элементов
матрицы~$\varphi$ по компонентам вектора~$y$ на матрицу~$\sigma$:
    \begin{equation*}
    \left[\varphi_{yy} : \sigma\right]_{kl}=
    \mathrm{tr}\left[\varphi_{klyy}\sigma\right].
    %\label{e4.15-s}
    \end{equation*}

Далее обозначим
$$
V_n^{(1)}=\left[ V_{1n}^{\mathrm{T}}\,\, V_{2n}\,\,\cdots\,\, V_{n+1}\right]^{\mathrm{T}}
$$
и введем блочную матрицу

\vspace*{-6pt}

\noindent
   \begin{multline*}
   \psi_n\left(\bar Y_n,V_n^{(1)}\right)={}\\
   {}=\left[\psi_{1n}(\bar Y_n)\,\,
    \psi_{1n}'\left(\bar Y_n,V_n^{(1)}\right)\,\, \psi_{1n}''(\bar Y_n) \cdots\right.\\
\left.\cdots     \psi_{N+1,n}\left(\bar Y_n\right)\,\, \psi_{N+1,n}'\left(\bar Y_n,V_n^{(1)}\right)\,\,
    \psi_{N+1,n}''\left(\bar Y_n\right) \right]
    %\label{e4.16-s}
    \end{multline*}
и блочный случайный вектор

\vspace*{-6pt}

\noindent
\begin{multline*}
V_n=\left [ V_{1n}^{\mathrm{T}}\,\, V_{1n}^{\prime\mathrm{T}}\,\, V_{1n}^{\prime\prime\mathrm{T}}\,\,
    V_{2n}^{\mathrm{T}}\,\, V_{2n}^{\prime\mathrm{T}}\,\, V_{2n}^{\prime\prime\mathrm{T}}\,
    \cdots\right.\\
\left.    \cdots\, V_{N+1,n}\,\, V_{N+1,n}'\,\, V_{N+1,n}'' \right]^{\mathrm{T}}. %\label{e4.17-s}
    \end{multline*}

\noindent
\textbf{Замечание~4.1.}\
Для полного определения распределения случайного вектора~$V_n$ в~(\ref{e4.1-s}) достаточно найти ковариационную матрицу нормально
распределенного случайного вектора $[V_{{in}}^{\mathrm{T}}\,\,{V_{{in}}}^{\prime\mathrm{T}}\,\,{V_{{in}}}^{\prime\prime\mathrm{T}}\,]^{\mathrm{T}}$. 
Пользуясь известными
формулами ковариационных и~взаимных ковариационных матриц
стохастических интегралов, находим блоки ковариационной
матрицы~$K_{1n}$ случайного вектора $[V_{{in}}^{\mathrm{T}}\,\,{V_{{in}}'}^{\mathrm{T}}\,\,{V_{{in}}''}^{\mathrm{T}}\,]^{\mathrm{T}}$:
    $$
    K_{1n,11}\!=\mathrm{M} V_{1n}V_{1n}^{\mathrm{T}}\!=\!\!\inh \!\!\nu_0(\tau)\,d\tau\cong \nu_0\left(nh+\fr{h}{2}\right)h;
    $$
    
\noindent
\begin{multline*}
    K_{1n,12}=\mathrm{M} V_{1n} V^{\prime\mathrm{T}}_{1n}=\inh \ovth \nu_0(\tau)\,d\tau
    \cong{}\\
    {}\cong \fr{1}{2}\nu_0\left(nh+\fr{h}{2}\right)h;
 \end{multline*}
 
 \vspace*{-12pt}
 
 \noindent
 \begin{multline*}
K_{1n,13}=\mathrm{M} V_{1n}V^{\prime\prime\mathrm{T}}_{1n}=\!\!\!\!\!\!\!\inh\! \left(\ovth\right)^2 
\nu_0(\tau)\,d\tau
    \cong{}\\
    {}\cong \fr{1}{3}\nu_0\left(nh+\fr{h}{2}\right)h;
    \end{multline*}
    $$ 
    K_{1n,21}= K_{1n,12};\quad K_{1n,31}= K_{1n,13};
    $$
    
   \vspace*{-12pt}
 
 \noindent
 \begin{multline*}
    K_{1n,23}=\mathrm{M} V_{1n}V^{\prime\prime\mathrm{T}}_{1n}=\!\!\!\!\!\inh \!\left(\ovth\right)^{\!3}\! 
\nu_0(\tau)\,d\tau     \cong{}\\
{}\cong \fr{1}{4}\,\nu_0\left(nh+\fr{h}{2}\right)h;
\end{multline*}
    $$
     K_{1n,31}= K_{1n,13};\quad K_{1n,32}= K_{1n,23};
     $$
     
   \vspace*{-12pt}
 
 \noindent
 \begin{multline*}
    K_{1n,33}\!=\mathrm{M} V_{1n}^{\prime\prime} V^{\prime\prime\mathrm{T}}_{1n}\!=\!\!\!\inh\! \left(\ovth\right)^{\!4}\!
    \nu_0(\tau)\,d\tau\cong{}\\
    {}\cong \fr{1}{5}\,\nu_0\left(nh+\fr{h}{2}\right)h.
    \end{multline*}


\noindent
\textbf{Замечание~4.2.}
Практически целесообразно аппроксимировать
стохастические интегралы от неслучайных
функций в~(\ref{e4.12-s}), определяющие величины $V_{in}^\prime$ и~$V_{in}^{\prime\prime}$
($i\hm=\overline{2,N+1}$), с~помощью аналога интегральной теоремы о~ среднем
для стохастических интегралов:
 \begin{multline*}
 V_{{in}}'=\inh \ovth \,dP_{i-1}^0(\tau)\cong{}\\
 {}\cong \fr{\inhnolim \left(
    (\tau- nh)/h\right) \nu_{i-1}(\tau)\,d\tau}{\inhnolim \nu_{i-1}(\tau)\,d\tau}
    \Delta P_{i-1,n}\cong{}\\
    {}\cong \fr{1}{2}\,\Delta P_{i-1,n}=
    \fr{1}{2}\,V_{{in}};\\[-20pt]
    \end{multline*}
    


\noindent
\begin{multline*}
   V_{{in}}''=\inh \left( \ovth\right)^2 dP_{i-1}^0(\tau)
    \cong{}\\
    {}\cong \fr{\inhnolim \left( (\tau-nh)/h\right)^2 \nu_{i-1}(\tau)\,d\tau}{\inhnolim 
\nu_{i-1}(\tau)\,d\tau}\Delta P_{i-1,n}\cong{}\\
{}\cong \fr{1}{3}\,\Delta P_{i-1,n}=
    \fr{1}{3}\,V_{{in}},\enskip i=\overline{2,N+1}\,.
    \end{multline*}


\textbf{4.2.}\ В уточнении метода $h^2$ и~$h^{3/2}$ при выводе уравнения~(\ref{e4.1-s}) были 
допущены две небольшие неточности.
Во-пер\-вых, при замене~$\bar Y_{\tau}$ величиной $\bar Y_n\hm+(\tau\hm-nh)\Delta
Y_n/h$ случайные функции $b(Y_{\tau},\tau)$ и~$c_i(Y_{\tau},\tau )$,
независимые от~$dW_0(\tau)$ и~$dP_i^0(\tau)$ в~силу конструкции
интеграла Ито, были заменены неслучайными функциями, зависящими от
случайного параметра~$\Delta Y_n$, который зависит от значений
$dW_0(\tau)$ и~$dP_i^0(\tau)$ в~интервале $(nh,(n\hm+1)h)$. Во-вто\-рых,
если
$c_i(y,t)\hm\ne 0$ хотя бы при одном~$i$, реализации СтП~$Y(t)$ имеют разрывы первого рода в~случайных точках, несмотря на его
среднеквадратичную непрерывность. Поэтому линейную интерполяцию
данного СтП, строго говоря, проводить нельзя. В~условиях \textbf{теоремы~4.2} 
первую из этих
неточностей можно устранить двумя способами~\cite{11-s, 12-s}. Первый состоит в~замене
интерполяции СтП~$Y(t)$ экстраполяцией, что равноценно замене~$\Delta Y_n$ в~получаемом выражении для~$Y_{\tau}$ величиной $\Delta
Y_{n-1}$. Однако это приведет к~появлению в~правой части разностного
уравнения величин $\bar Y_{n-1}$ и~$V_{n-1}^{(1)}$, т.\,е.\ к~замене
уравнения первого порядка разностным уравнением второго порядка.
Второй способ состоит в~отказе от интерполяции процесса~$Y(t)$ на
интервале $(nh,(n+1)h)$ и~непосредственном выражении приращений
функций $a(Y_{\tau},\tau)$, $b(Y_{\tau},\tau)$ и~$c_i(Y_{\tau},\tau)$
на малом интервале $(nh,(n+1)h)$ по обобщенной формуле Ито с~заменой в~ней дифференциалов приращениями. При этом способе устраняется и~вторая
допущенная неточность. Но полученное таким путем разностное уравнение
будет более сложным. В~него войдут случайные величины, представляющие
собой двойные интегралы по компонентам винеровского СтП $W(t)\hm=W_0(t)$ и~по пуассоновским СтП:
\begin{equation}
\left.
\begin{array}{l}
\hspace*{-3.8mm}\displaystyle \inh \!\int\limits_{nh}^{\tau}\!dW_j(\sigma)dW_j(\tau); \!
\displaystyle  \inh \!\int\limits_{nh}^{\tau}\!dP_i(\sigma)dP_j^0(\tau);\\[6pt]
\hspace*{-3.8mm}\displaystyle    \inh\! \int\limits_{nh}^{\tau}\!dP_i^0(\sigma)dW_j(\tau);\!
\displaystyle  \inh\! \int\limits_{nh}^{\tau}\!dP_j^0(\sigma)dW_i(\tau). 
\end{array}\!\!
\right\}\!\!
\label{e4.18-s}
    \end{equation}

\noindent
\textbf{Замечание~4.3.}
Распределения этих случайных величин найти чрезвычайно сложно, и~только первые две из них легко вычисляются при $j\hm=i$:
   \begin{align*}
    \inh \!\int\limits_{nh}^{\tau}\!dW_i(\sigma)dW_i(\tau)&=
    \fr{[\Delta W_{{in}}]^2-\nu_{ii}(nh+h/2)h}{2};
   \\
    \inh \!\int\limits_{nh}^{\tau}\!dP_i^0(\sigma)dP_i^0(\tau)&=
    \fr{[\Delta P_{{in}}]^2-\Delta P_{{in}}}{2}. 
   \end{align*}


Что касается второй неточности, то она не может существенно повлиять
на результат, так как вероятность появления скачка пуассоновского
процесса на достаточно малом интервале $(nh,(n+1)h)$ ничтожно мала.

\textbf{4.3.}\ В~условиях \textbf{теоремы~4.3} точность аппроксимации стохастического 
дифференциального уравнения разностным можно  повышать и~дальше. В~частности, в~одном из способов
достаточно выразить~$a(Y_{\tau},\tau)$,
$b(Y_{\tau},\tau)$ и~$c_i(Y_{\tau},\tau)$ на интервале $(nh,(n+1)h)$
интегральной формулой Ито, соответствующей дифференциальной формуле:
   \begin{multline}
   a\left(Y_{\tau},\tau\right)=a\left(\bar Y_n,nh\right)+\int\limits_{nh}^{\tau} a_t(Y_s,s)+{}\\
   {}+
    a_y(Y_s,s)^{\mathrm{T}}\Biggl[ a(Y_s,s)-\sum\limits_{i=1}^N c_i(Y_s,\nu_i(s))+{}\\
    {}+\fr{1}{2}\,a_{yy}(Y_s,s):
    \sigma(Y_s,s)\Biggr]ds+{}\\
    {}+\int\limits_{nh}^{\tau}a_y(Y_s,s)^{\mathrm{T}}  b\left(Y_s,s\right)\,dW_0(s)+{}\\
{}+\sum\limits_{i=1}^N \int\limits_{nh}^{\tau}
    \left[a(Y_s+c_i(Y_s,s),s)-a(Y_s,s)\right]dP_i^0(s) \label{e4.19-s}
    \end{multline}
(формулы для $b(Y_{\tau},\tau)$ и~$c_i(Y_{\tau},\tau)$ аналогичны). В~результате 
получим правую
часть разностного уравнения с~точностью до~$h^3$ в~детерминированном
(при данном~$\bar Y_n$) сла\-га\-емом и~$h^{5/2}$ в~случайном сла\-га\-емом.
Процесс уточнения разностного уравнения, соответствующего данному
стохастическому дифференциальному уравнению, можно продолжать
неограниченно. Каждое новое уточнение потребует существования
производных функций~$a$, $b$ и~$c_i$ все более высоких порядков.

\smallskip

\noindent
\textbf{Замечание~4.4.}
Для уточнения разностного уравнения~(\ref{e4.1-s}) можно применить и~другой способ, 
а~именно: можно выразить подынтегральные функции в~(\ref{e4.18-s}) и~аналогичных формулах для $b(Y_{\tau},\tau)$ и~$c_i(Y_{\tau},\tau)$ их
выражениями по обобщенной формуле Ито, заменив в~ней дифференциалы
приращениями. При этом в~разностное уравнение войдут тройные интегралы
по компонентам винеровского СтП~$W_0(t)$ и~пуассоновским СтП~$P_i(t)$. Для дальнейшего уточнения аппроксимации стохастического
дифференциального уравнения разностным в~этом случае следует
подынтегральные функции в~(\ref{e4.19-s}) и~в~со\-от\-вет\-ст\-ву\-ющих формулах для
$b(Y_{\tau},\tau)$ и~$c_i(Y_{\tau},\tau)$, в~свою очередь,
представить интегральной формулой Ито, а~затем уже применять
дифференциальную формулу Ито с~заменой дифференциалов приращениями.
Данный СтП можно продолжать неограниченно, и~в итоге он приведет к~представлению СтП $Y(t)$ на интервале $(nh,\linebreak (n\hm+1)h)$
стохастическими аналогами формулы Тейлора. При этом в~разностное
уравнение войдут кратные стохастические интегралы по компонентам
винеровского СтП~$W_0(t)$ и~по пуассоновским СтП~$P_i(t)$.
Нахождение распределения этих интегралов представляет практически
непреодолимые трудности. И лишь интегралы любой кратности по одной и~той же компоненте винеровского процесса $W_0(t)$ или по одному и~тому же
пуассоновскому процессу~$P_i(t)$ вычисляются очень просто.


\smallskip

\noindent
\textbf{Замечание~4.5.}
Чтобы избежать вычисления производных
функций $a(Y_{\tau},\tau)$, $b(Y_{\tau},\tau)$ и~$c_i(Y_{\tau},\tau)$ при
применении двух изложенных способов аппроксимации стохастического
дифференциального уравнения разностным, можно рекомендовать заменить
их отношениями конечных приращений, например на интервале
$(nh,(n+1)h)$ по $t$ и~на интервалах $(\bar Y_{nk}, \bar
Y_{nk}+a_k(\bar Y_n,nh)h)$ по компонентам вектора~$y$.



\smallskip

\noindent
\textbf{Замечание~4.6.}
Полученные разностные уравнения можно использовать как при
теоретических исследованиях, так и~для численного интегрирования
стохастических дифференциальных уравнений СтС НРОП. При этом нужно знать
распределение всех случайных величин, входящих в~разностные уравнения.
В данном случае разностные уравнения будут представлять собой так называемую
сильную аппроксимацию стохастических дифференциальных уравнений.
При численном интегрировании такая аппроксимация нужна, когда требуется
получать реализации СтП~$Y(t)$. Однако часто нет
нужды в~получении реализаций СтП, а достаточно иметь лишь
оценки моментов или математических ожиданий ка\-ких-ли\-бо функций от значения
СтП~$Y(t)$ в~определенный момент. В~таких случаях можно
отказаться от использования точных распределений входящих в~разностные
уравнения случайных величин, а~заменить их ка\-ки\-ми-ни\-будь более
простыми распределениями с~теми же моментными характеристиками.
Например, нормально распределенную скалярную величину с~нулевым
математическим ожиданием и~дисперсией~${\mathsf D}$ можно заменить дискретной
случайной величиной, принимающей два значения $\pm\sqrt{\mathsf{D}}$ с~вероятностями~$1/2$. При замене случайных величин на величины с~более простыми распределениями разностное уравнение будет представлять
собой слабую аппроксимацию стохастического
дифференциального уравнения.

\vspace*{-6pt}

\section{Выводы и~обобщения}

\vspace*{-3pt}

Методическое обеспечение статистического моделирования для различных уровней 
точности аппроксимации, основанное на численном интегрировании уравнений 
дифференциальных СтС НРОП, приводимых к~дифференциальным системам, применимо 
только к~приведенным системам~(\ref{e2.9-s}) и~(\ref{e2.10-s}) с~гладкими нелинейными функциями.

Для негладких нелинейных функций в~исходных уравнениях~(\ref{e2.1-s}) в~рамках 
корреляционной тео\-рии, как показано в~\cite{3-s, 4-s, 5-s, 6-s}, целесообразно сразу применить 
регрессионную линеаризацию или полиномиальную аппроксимацию. В результате придем к~уравнениям~(\ref{e2.9-s}) и~(\ref{e2.10-s}), линеаризованным или полиномиальным стохастическим 
уравнениям, параметрически зависящим от первых двух вероятностных методов 
переменных~$X$ и~$Y$. Эти вероятностные моменты определяются или непосредственно 
статистическим, или аналитическим моделированием путем численного интегрирования 
обыкновенных дифференциальных уравнений для вероятностных моментов первого 
и~второго порядка~\cite{2-s, 11-s, 12-s}. Для полиномиальных моделей аппроксимации приходится 
решать обыкновенные дифференциальные уравнения для параметризованных одно- и~многомерных плотностей. Такие методы моделирования называют комбинированными 
ве\-ро\-ят\-ност\-но-ста\-ти\-сти\-че\-ски\-ми. К~последним также относят методы моделирования СтС 
НРОП со случайными па\-ра\-мет\-ра\-ми~\cite{7-s} как на основе канонических представлений 
случайных функций, так и~прямого статистического моделирования уравнений для 
параметров одно- и~многомерных распределений. Со\-от\-вет\-ст\-ву\-ющие примеры приведены 
в~\cite{7-s}.

Представляет интерес развитие ве\-ро\-ят\-ност\-но-ста\-ти\-сти\-че\-ских методов моделирования 
для сис\-тем, стохастически НРОП~\cite{5-s}, а~также 
не\-пре\-рыв\-но-дис\-крет\-ных сис\-тем, не разрешенных относительно производных 
(разностей), в~том числе на основе канонических разложений и~интегральных 
канонических представлений~\cite{12-s}. Особый интерес представляют системы с~неявной 
переменной детерминированной и~неявной стохастической структурой.

{\small\frenchspacing
 {\baselineskip=10.8pt
 %\addcontentsline{toc}{section}{References}
 \begin{thebibliography}{99}    
%1
\bibitem{1-s}
\Au{Синицын И.\,Н.}
Аналитическое моделирование широкополосных процессов в~стохастических системах, 
не разрешенных относительно производных~// Информатика и~её применения, 2017. 
Т.~11. Вып.~1. С.~3--10. doi: 10.14357/19922264170101. EDN: YOCMVL.

%2
\bibitem{2-s}
\Au{Синицын И.\,Н.}
Параметрическое аналитическое моделирование процессов в~стохастических  
системах, не разрешенных относительно производных~// Системы и~средства 
информатики, 2017. Т.~27. №\,1. С.~21--45. doi: 10.14357/08696527170102. EDN: YODCZL.

%3
\bibitem{3-s}
\Au{Sinitsyn I.\,N.}
Analytical modeling and estimation of normal processes defined by stochastic 
differential equations with unsolved derivatives~// J.~Mathematics  
Statistical Research, 2021. Vol.~3. Iss.~1. Art.~139. 7~p. doi: 
10.36266/\linebreak JMSR/139.

%4
\bibitem{4-s}
\Au{Синицын И.\,Н.}
Аналитическое моделирование и~оценивание нестационарных нормальных процессов 
в~стохастических сис\-те\-мах, не разрешенных относительно производных~// Системы и~средства 
информатики, 2022. Т.~32. №\,2. С.~58--71. doi: 10.14357/08696527220206. EDN: YMGERJ.


%5
\bibitem{5-s}
\Au{Синицын И.\,Н.}
Нормализация систем, стохастически не разрешенных относительно производных~// 
Информатика и~её применения, 2022. Т.~16. Вып.~1. С.~32--38. doi: 10.14357/19922264220105. EDN: \mbox{LDFJJB}.

%6
\bibitem{6-s}
\Au{Синицын И.\,Н.}
Аналитическое моделирование распределений с~инвариантной мерой в~стохастических 
системах, не разрешенных относительно производных~// Информатика и~её 
применения, 2023. Т.~17. Вып.~1. С.~2--10. doi: 10.14357/19922264230101. EDN: QWXVXC.

%7
\bibitem{7-s}
\Au{Синицын И. Н. }
Аналитическое моделирование стохастических систем, не разрешенных относительно 
производных, со случайными параметрами~// Системы и~средства информатики, 2024. 
Т.~34. №\,1. С.~4--22. doi: 10.14357/08696527240101. EDN: ZPTXJI.

%8
\bibitem{8-s}
\Au{Kloeden P., Platen~E.}
Numerical solution of stochastic differential equations.~--- Berlin: Springer-Verlag, 1992. 636~p.
doi: 10.1007/978-3-662-12616-5.

%9
\bibitem{9-s}
\Au{Артемьев А.\,А., Михайличенко~А.\,М., Синицын~И.\,Н.}
Статистическое моделирование срочных финансовых операций.~--- Новосибирск: ВЦ СО 
РАН, 1996. Кн.~1, 2. 280~с.

%10
\bibitem{10-s}
\Au{Кузнецов Д.\,Ф.}
Численное интегрирование стохастических дифференциальных уравнений.~--- СПб: 
\mbox{СПбГУ}, 2001. 712~с.

%11
\bibitem{11-s}
\Au{Пугачёв В.\,С., Синицын~И.\,Н.}
Теория стохастических систем.~--- М.: Логос, 2000; 2004. 1000~с.


%12
\bibitem{12-s}
\Au{Синицын И.\,Н.}
Канонические представления случайных функций. Теория и~применения.~--- 2-е изд.~--- М.: ТОРУС ПРЕСС, 2023. 816~с.
\end{thebibliography}

 }
 }

\end{multicols}

\vspace*{-9pt}

\hfill{\small\textit{Поступила в~редакцию 18.01.24}}

%\vspace*{10pt}

%\pagebreak

\newpage

\vspace*{-28pt}

%\hrule

%\vspace*{2pt}

%\hrule



\def\tit{STATISTICAL MODELING OF~DIFFERENTIAL STOCHASTIC SYSTEMS WITH~UNSOLVED DERIVATIVES}


\def\titkol{Statistical modeling of~differential stochastic systems with~unsolved derivatives}


\def\aut{I.\,N.~Sinitsyn$^{1,2}$}

\def\autkol{I.\,N.~Sinitsyn}

\titel{\tit}{\aut}{\autkol}{\titkol}

\vspace*{-8pt}


\noindent
$^{1}$Federal Research Center ``Computer Science and Control'' of the Russian Academy of Sciences, 44-2~Vavilov\linebreak
$\hphantom{^1}$Str., Moscow 119333, Russian Federation

\noindent
$^{2}$Moscow State Aviation Institute (National Research University), 4~Volokolamskoe Shosse, Moscow 125933,\linebreak
$\hphantom{^1}$Russian Federation




\def\leftfootline{\small{\textbf{\thepage}
\hfill INFORMATIKA I EE PRIMENENIYA~--- INFORMATICS AND
APPLICATIONS\ \ \ 2024\ \ \ volume~18\ \ \ issue\ 3}
}%
 \def\rightfootline{\small{INFORMATIKA I EE PRIMENENIYA~---
INFORMATICS AND APPLICATIONS\ \ \ 2024\ \ \ volume~18\ \ \ issue\ 3
\hfill \textbf{\thepage}}}

\vspace*{4pt}


\Abste{The paper is dedicated to statistical modeling methodological support for differential stochastic systems with unsolved derivatives (StS USD).
 The basic results are: ($i$)~two theorems concerning reduction of stochastic functional-differential equations to stochastic Ito equations; 
 ($ii$)~Euler approximation method for stochastic differential equations with Gaussian and Poisson noises; 
 ($iii$)~three theorems concerning numerical algorithms of 
 various accuracy for StS USD with smooth nonlinearities; and ($i\nu$)~two algorithms for StS USD with nonsmooth nonlinearities. Special attention is paid 
 to methodological aspects of numerical statistical analysis of deterministic and random components in the cases of weak and strong approximation. 
 Directions for further research are given.}


\KWE{analytical (probabilistic) modeling; methodological support; statistical modeling; stochastic systems (StS); StS with unsolved derivatives (StS USD)}

\DOI{10.14357/19922264240302}{WWMEOT}

\vspace*{-12pt}


    
     % \Ack

%\vspace*{-3pt}

%\noindent



  \begin{multicols}{2}

\renewcommand{\bibname}{\protect\rmfamily References}
%\renewcommand{\bibname}{\large\protect\rm References}

{\small\frenchspacing
 {%\baselineskip=10.8pt
 \addcontentsline{toc}{section}{References}
 \begin{thebibliography}{99} 
%1
\bibitem{1-s-1} 
\Aue{Sinitsyn, I.\,N.} 2017. 
Analiticheskoe modelirovanie shirokopolosnykh protsessov v~stokhasticheskikh sistemakh, ne razreshennykh otnositel'no proizvodnykh  
[Analytical modeling of wide band processes in stochastic systems with unsolved derivatives]. \textit{Informatika i ee Primeneniya --- Inform. Appl.} 11(1):3--10.
doi: 10.14357/19922264170101. EDN: YOCMVL.

%2
\bibitem{2-s-1} 
\Aue{Sinitsyn, I.\,N.}  2017.
Parametricheskoe analiticheskoe modelirovanie protsessov v~stokhasticheskikh sistemakh, ne razreshennykh otnositel'no proizvodnykh 
[Parametric analytical modeling of wide band processes in stochastic systems with unsolved derivatives]. \textit{Sistemy i Sredstva
Informatiki --- Systems and Means of Informatics} 27(1):21--45. doi: 10.14357/08696527170102. EDN: YODCZL.

%3
\bibitem{3-s-1} 
\Aue{Sinitsyn, I.\,N.} 2021. 
Analytical modeling and estimation of normal processes defined by stochastic differential equations with unsolved derivatives. 
\textit{J. Mathematics Statistics Research} 3(1):139. 7~p. doi: 10.36266/JMSR/139.

%4
\bibitem{4-s-1} 
\Aue{Sinitsyn, I.\,N.} 2022. 
Analiticheskoe modelirovanie i~otse\-ni\-va\-nie nestatsionarnykh normal'nykh protsessov v~sto\-kha\-sti\-che\-skikh sistemakh, ne razreshennykh
\mbox{otno\-si\-tel'\-no} proizvodnykh 
[Analytical modeling and estimation of nonstationary normal processors with unsolved derivatives]. 
\textit{Sistemy i~Sredstva Informatiki~--- Systems and Means of Informatics} 32(2):58--71. doi: 10.14357/ 08696527220206. EDN: YMGERJ.

%5
\bibitem{5-s-1} 
\Aue{Sinitsyn, I.\,N.} 2022.
Normalizatsiya sistem, sto\-kha\-sti\-che\-ski ne razreshennykh otnositel'no proizvodnykh [Normalization of systems with stochastically unsolved derivatives]. 
\textit{Informatika i~ee Primeneniya~--- Inform. Appl.} 16(1):32--38.
doi: 10.14357/19922264220105. EDN: LDFJJB.

%6
\bibitem{6-s-1} 
\Aue{Sinitsyn, I.\,N.} 2023. 
Analiticheskoe modelirovanie raspredeleniy s invariantnoy meroy v~stokhasticheskikh sistemakh, ne razreshennykh otnositel'no proizvodnykh 
[Analytical modeling of distributions with invariant measure in stochastic systems with unsolved derivatives]. 
\textit{Informatika i~ee Primeneniya~--- Inform. Appl.}  17(1):2--10. doi: 10.14357/19922264230101. EDN: QWXVXC.

%7
\bibitem{7-s-1} 
\Aue{Sinitsyn, I.\,N.} 2024. 
Analiticheskoe modelirovanie sto\-kha\-sti\-che\-skikh  sistem, ne razreshennykh otnositel'no pro\-iz\-vod\-nykh, so sluchaynymi parametrami 
[Analytical modeling of stochastic systems with random parameters and unsolved derivatives]. 
\textit{Sistemy i~Sredstva Informatiki~--- Systems and Means of Informatics} 34(1):4--22. 
doi: 10.14357/ 08696527240101. EDN: ZPTXJI.

%8
\bibitem{8-s-1} 
\Aue{Kloeden, P.\,E., and E.~Platen.} 1992. 
\textit{Numerical solution of stochastic differential equations}. Berlin--Heidelberg: Springer-Verlag. 636~p.
doi: 10.1007/978-3-662-12616-5.

%9
\bibitem{9-s-1} 
\Aue{Artemyev, A.\,A., A.\,M.~Mikhaylichenko, and I.\,N.~Si\-ni\-tsyn.} 1996. 
\textit{Statisticheskoe modelirovanie srochnykh finansovykh operatsiy} [Statistical modeling of urgent financial transactions]. 
Novosibirsk: CC SB RAS. 280~p.

%10
\bibitem{10-s-1} 
\Aue{Kuznetsov, D.\,F.} 2001.
\textit{Chislennoe integrirovanie sto\-kha\-sti\-che\-skikh differentsial'nykh uravneniy} [Numerical integration of stochastic differential equations]. 
Saint Petersburg: SPbGU. 712~p.

%11
\bibitem{11-s-1} 
\Aue{Pugachev, V.\,S., and I.\,N.~Sinitsyn.} 2001.
\textit{Stochastic systems: Theory and applications}. Singapore: World Scientific. 908~p.

%12
\bibitem{12-s-1} 
\Aue{Sinitsyn, I.\,N.} 2023.
\textit{Kanonicheskie predstavleniya sluchaynykh funktsiy. Teoriya i~primeneniya} [Canonical expansions of random functions. Theory and application]. 
Moscow: TORUS PRESS. 816~p.



\end{thebibliography}

 }
 }

\end{multicols}

\vspace*{-6pt}

\hfill{\small\textit{Received January 18, 2024}} 

%\vspace*{-18pt}

\Contrl

\vspace*{-3pt}

\noindent
\textbf{Sinitsyn Igor N.} (b.\ 1940)~--- Doctor of Science in technology, professor, Honored scientist of RF, principal scientist, 
Federal Research Center ``Computer Science and Control'' of the Russian Academy of Sciences, 44-2~Vavilov Str., Moscow 119333, 
Russian Federation; professor, Moscow State Aviation Institute (National Research University), 4~Volokolamskoe Shosse, Moscow 125933, Russian Federation; 
\mbox{sinitsin@dol.ru}


\label{end\stat}

\renewcommand{\bibname}{\protect\rm Литература} 