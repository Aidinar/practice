\def\stat{flerov}

\def\tit{МОДЕЛИ АНАЛИЗА КОМПОНОВОЧНЫХ СХЕМ В~ЗАДАЧЕ~ФОРМИРОВАНИЯ 
ОБЛИКА САМОЛЕТА}

\def\titkol{Модели анализа компоновочных схем в~задаче формирования 
облика самолета}

\def\aut{Л.\,Л.~Вышинский$^1$, Ю.\,А.~Флёров$^2$}

\def\autkol{Л.\,Л.~Вышинский, Ю.\,А.~Флёров}

\titel{\tit}{\aut}{\autkol}{\titkol}

\index{Вышинский Л.\,Л.}
\index{Флёров Ю.\,А.}
\index{Vyshinsky L.\,L.}
\index{Flerov Yu.\,A.}


%{\renewcommand{\thefootnote}{\fnsymbol{footnote}} \footnotetext[1]
%{Работа 
%выполнена при поддержке Программы развития МГУ, проект №\,23-Ш03-03. При анализе 
%данных использовалась инфраструктура Центра коллективного пользования 
%<<Высокопроизводительные вычисления и~большие данные>> 
%(ЦКП <<Информатика>>) ФИЦ ИУ РАН (г.~Москва)}}


\renewcommand{\thefootnote}{\arabic{footnote}}
\footnotetext[1]{Федеральный исследовательский центр <<Информатика и~управление>> Российской академии наук, 
\mbox{wyshinsky@mail.ru}}
\footnotetext[2]{Федеральный исследовательский центр <<Информатика и~управление>> Российской академии наук, 
\mbox{fler@ccas.ru}}

\vspace*{-2pt}


  
  
  \Abst{Рассматриваются задачи анализа характеристик проектируемого самолета на стадии 
формирования его облика. Характерная особенность этих задач состоит в~отсутствии 
достаточной информации о конструкции самолета, которая появляется только на этапах 
эскизного и~рабочего проектирования. По сути, это задачи технического прогнозирования по 
довольно ограниченному набору параметров, которым может оперировать проектировщик на 
этом этапе. Основная задача формирования облика~--- это синтез компоновочной схемы 
самолета и~построение ее параметрического представления. Компоновочные схемы 
в~практике проектирования представляют собой один из основных проектных документов 
и~служат прообразом разрабатываемого изделия. Настоящая работа посвящена представлению 
математических моделей, предназначенных для построения оценок весовых, 
аэродинамических, лет\-но-тех\-ни\-че\-ских (ЛТХ) и~взлет\-но-по\-са\-доч\-ных (ВПХ) характеристик 
самолета по параметрам его компоновочной схемы и~последующей проверке соответствия 
полученных оценок требованиям, предъявляемым к~проектируемому изделию.}
  
  \KW{математическое моделирование; автоматизация проектирования; самолет; 
компоновочная схема; характеристики самолета}

\DOI{10.14357/19922264240301}{VWQBMD}
  
%\vspace*{3pt}


\vskip 10pt plus 9pt minus 6pt

\thispagestyle{headings}

\begin{multicols}{2}

\label{st\stat}
  
\section{Задача формирования облика самолета}

%\vspace*{3pt}

  Процесс проектирования самолетов (далее для краткости~--- ЛА, 
летательных аппаратов) достаточно протяжен во времени и~представляет собой\linebreak 
последовательность этапов проектирования, отличающихся уровнем 
детализации конструкции и~математических моделей, когда идет речь об 
автоматизации проектирования~[1]. 
Начальный этап \mbox{проектирования} ЛА 
обычно называют формированием облика~[2]. Такое название отражает 
реальные процессы этапа. На стадии этапа появляются первые эскизы будущего 
изделия, которые \mbox{затем} оформляются в~основополагающий проектный 
документ~--- в~компоновочную схему ЛА. Компоновочная схема определяет 
структуру ЛА, состав основных его агрегатов, параметры этих агрегатов и~их 
взаимное расположение. Формирование облика ЛА представляет собой 
сложный итерационный процесс, который включает анализ требований  
так\-ти\-ко-тех\-ни\-че\-ско\-го задания (ТТЗ), построение областей 
существования в~пространстве основных проектных параметров ЛА, синтез 
компоновочной схемы проектируемого изделия и~ее всесторонний анализ 
(рис.~1). 



  В ТТЗ формулируются цели создания ЛА и~определяются конкретные задачи 
и~требования, которые он должен выполнять в~процессе эксплуатации. В~ТТЗ 
могут быть также сформулированы критерии, по которым следует 
оптимизировать разрабатываемый проект. 

На основании анализа ТТЗ 
и~имеющегося опыта проектирования в~пространстве основных проектных 
параметров ЛА (тяговооруженность, удельная на\-груз\-ка\, на\, крыло,\,\, 
максимальное\, аэродинамиче-  %\linebreak\vspace*{-15pt}



\end{multicols}
  
\begin{figure*}[h] %fig1
  %\vspace*{6pt}
  \begin{center}  
    \mbox{%
\epsfxsize=155.078mm 
\epsfbox{fle-1.eps}
}
\end{center}
\vspace*{-11pt}
\Caption{Процесс формирования облика ЛА}
%\vspace*{-6pt}
\end{figure*}

\pagebreak


\begin{multicols}{2}


\noindent
 ское качества и~ряд других   величин) строится 
<<об\-ласть существования>> проектируемого изделия~[3]. Заметим, что 
при\-над\-леж\-ность точки об\-ласти существования еще не гарантирует 
ре\-а\-ли\-зу\-емость проекта, поскольку об\-ласть существования на начальном этапе 
проектирования не учитывает многих ограничений, воз\-ни\-ка\-ющих на более 
позд\-них этапах проектирования. Однако область существования позволяет 
выявить на множестве реально существующих и~экс\-плу\-а\-ти\-ру\-ющих\-ся 
самолетов подмножество ЛА, па\-ра\-мет\-ры которых лежат в~по\-стро\-ен\-ной об\-ласти 
существования. Экземпляры этого подмножества могут служить в~качестве 
потенциальных прототипов про\-ек\-ти\-ру\-емо\-го изделия при решении основной 
задачи формирования облика ЛА~--- синтезе компоновочной схемы. 

В~работах~[4, 5] был представлен ряд математических моделей, связанных 
с~решением этой задачи, которые были реализованы в~модуле формирования 
облика ЛА (далее~--- МФО), разработанном в~составе автоматизированной 
системы весового проектирования~[6]. 

В~на\-сто\-ящей статье рас\-смат\-ри\-ва\-ют\-ся 
математические модели анализа компоновочных схем, реализованные  
с~по\-мощью того же программного \mbox{модуля}. Модели анализа вместе 
с~моделями синтеза образуют единую струк\-тур\-но-па\-ра\-мет\-ри\-че\-скую 
модель ЛА. Цель анализа на этапе формирования облика со\-сто\-ит 
в~последовательном расчете всех па\-ра\-мет\-ров и~характеристик  
струк\-тур\-но-па\-ра\-мет\-ри\-че\-ской модели ЛА.

  
  Анализ компоновочной схемы ЛА на этапе формирования облика решает две 
основных задачи: пер\-вая задача состоит в~анализе возможности компоновки 
в~рамках выбранной схемы всех агрегатов ЛА, сис\-тем и~полезной нагрузки, 
а~вторая задача~--- это рас\-чет так\-ти\-ко-тех\-ни\-че\-ских характеристик (ТТХ) ЛА 
и~проверка их соответствия требованиям, сформулированным в~ТТЗ. Если 
\mbox{рассчитанные} в~процессе анализа компоновочной схемы ТТХ не соответствуют 
требованиям ТТЗ, то приходится возвращаться на стадию синтеза 
и~корректировать основные проектные па\-ра\-мет\-ры и/или па\-ра\-мет\-ры 
компоновочной схемы. 
  
  Модели анализа на этапе формирования облика характерны тем, что они 
строятся по довольно ограниченному набору параметров, которым  может 
оперировать проектировщик на этом этапе. Это,\linebreak как правило, упрощенные 
полуэмпирические, \mbox{полутеоретические} модели, которые, с~одной стороны, 
отражают тенденции и~за\-ви\-си\-мости рас\-смат\-ри\-ва\-емых характеристик от 
па\-ра\-мет\-ров компоновочной схемы ЛА, а~с~другой стороны, не требуют 
тру\-до\-емких расчетов, что поз\-во\-ля\-ет использовать их при расчете большого 
чис\-ла альтернативных вариантов компоновочных схем и~тем более 
в~алгоритмах оптимизации. 

\begin{figure*} %fig2
\vspace*{1pt}
  \begin{center}  
    \mbox{%
\epsfxsize=160mm 
\epsfbox{fle-2.eps}
}
\end{center}
\vspace*{-9pt}
\Caption{Габаритная модель крыльевого топливного бака}
\end{figure*}

%\vspace*{-6pt}

\section{Геометрический анализ компоновочной схемы летательного аппарата}


%\vspace*{-3pt}

  В пространстве конструктивных параметров компоновочной схемы ЛА 
геометрические па\-ра\-мет\-ры занимают цент\-раль\-ное мес\-то. От них зависят 
весовые, аэродинамические и~другие характеристики ЛА~[7]. Гео\-мет\-ри\-че\-ский 
анализ на этапе формирования облика решает две основные задачи.  
Во-пер\-вых, гео\-мет\-ри\-че\-ские па\-ра\-мет\-ры планера служат исходной 
информацией для расчетов весовых и~аэродинамических характеристик ЛА. 
Эта информация должна содержать ряд фор\-мо\-об\-ра\-зу\-ющих па\-ра\-мет\-ров 
компоновочной схемы, таких как площадь основной несущей по\-верх\-ности, 
удлинение, сужение, стреловидность крыла, удлинение фюзеляжа, па\-ра\-мет\-ры 
оперения, а~так\-же ряд расчетных величин. Среди расчетных па\-ра\-мет\-ров на 
этапе формирования облика важную роль играют площадь омываемой 
по\-верх\-ности ($S_{\mathrm{ом}}$), которая определяет аэродинамическое 
сопротивление трения, и~площадь миделя планера ($S_{\mathrm{м}}$), от 
которой зависит сопротивление дав\-ле\-ния на околозвуковых и~сверхзвуковых 
скоростях. Иногда для расчета сопротивления сверхзвуковых ЛА 
рассматривают такую гео\-мет\-ри\-че\-скую характеристику, как <<график 
площадей>>~--- за\-ви\-си\-мость площади поперечного сечения планера от 
продольной координаты: $S_{zy}(x)$. Заметим, что график площадей может 
использоваться и~для решения второй задачи гео\-мет\-ри\-че\-ско\-го анализа~--- 
внутренней компоновки ЛА. Эта задача состоит в~проверке воз\-мож\-ности 
размещения на самолете полезной нагрузки~--- пассажиров, коммерческих 
грузов и~топлива, а~так\-же размещение экипажа, двигателей силовой установки, 
стоек шасси в~убранном положении, самолетного и~специального 
оборудования. Одной из характеристик самолета как транспортного средства 
служит величина располагаемого объема планера, которая может быть 
вы\-чис\-ле\-на интегрированием графика площадей по координате~$x$: 
$$
V_{\mathrm{расп}}= \int\limits_0^{L_{\mathrm{пл}}} S_{zy}(x)\,dx\,.
$$ 
В~процессе решения задач внутренней компоновки формируется разбиение 
агрегатов компоновочной схемы на определенные зоны целевого 
использования~--- пассажирские салоны, грузовые, топ\-лив\-ные, приборные, 
служебные и~другие отсеки. При этом располагаемые объемы отсеков должны 
удовлетворять соотношениям  
$$
V_{\mathrm{расп}}= \sum\limits_i 
V^i_{\mathrm{расп}} \geq k^i_{\mathrm{зап}} V^i_{\mathrm{потр}},
$$
 где~$V^i_{\mathrm{потр}}$~--- по\-треб\-ный объем для размещения $i$-го объекта 
размещения; $ k^i_{\mathrm{зап}}$~--- коэффициент заполнения 
со\-от\-вет\-ст\-ву\-юще\-го отсека, которые определяются нормативными документами 
или практикой проектирования. Одна из важных задач гео\-мет\-ри\-че\-ско\-го анализа 
компоновочной схемы~--- анализ цент\-ров\-ки ЛА в~процессе выработки топ\-ли\-ва, 
которая зависит от гео\-мет\-рии и~размещения топ\-лив\-ных емкостей~[8]. 


  
  В МФО в~качестве основной структуры геометрического анализа 
компоновочной схемы реализован механизм построения па\-ра\-мет\-ри\-че\-ских 
габаритных моделей агрегатов. Габаритные модели (<<габаритки>>)~--- это 
традиционный инструмент, ис\-поль\-зо\-вав\-ший\-ся для внут\-рен\-ней компоновки 
размещаемых на борту объектов. В~докомпьютерные времена <<габаритки>> 
вырезали из бумаги и~манипулировали ими, накладывая на чертежи 
компоновочной схемы. Сейчас габаритные модели всех агрегатов, в~том числе 
крыла, фюзеляжа, оперения, мотогондол, двигателей, топливных баков, можно 
воспроизвести на экране компьютера, управляя их параметрами и~положением 
на компоновочной схеме и~одновременно проводя вы\-чис\-ле\-ния нуж\-ных 
гео\-мет\-ри\-че\-ских характеристик. На рис.~2 приведен пример габаритной модели 
раз\-ме\-ща\-емых в~крыле самолета топливных баков. 



    Габаритная модель строится как многосекционная конструкция, каждая 
секция которой задается положением в~сис\-те\-ме координат агрегата и~состоит из 
упорядоченного набора плос\-ких сечений определенного типа. В~свою очередь, 
сечения задаются положением плос\-кости в~сис\-те\-ме координат секции и~своими 
габаритами. Все сечения секции\linebreak однотипны, а~сам тип и~число точек в~сечении 
задаются в~определении секции. В~трехмерной интерпретации каждая секция 
габаритной модели представляет собой объемное тело или по\-верх\-ность\linebreak 
некоторого многогранника, ограничивающего модели\-ру\-емый агрегат. Такое 
представление габаритной модели допускает относительно прос\-тые алгоритмы 
расчета нужных гео\-мет\-ри\-че\-ских характеристик или их оценку.

\begin{figure*} %fig3
\vspace*{1pt}
  \begin{center}  
    \mbox{%
\epsfxsize=160mm 
\epsfbox{fle-3.eps}
}
\end{center}
\vspace*{-11pt}
\Caption{Дерево конструкции ЛА}
%\vspace*{-pt}
\end{figure*}

%\vspace*{-6pt}

\section{Весовой анализ на~этапе формирования облика}

%\vspace*{-3pt}

  Весовой анализ на этапе формирования облика ЛА представляет собой 
достаточно слож\-ную задачу вы\-чис\-ле\-ния или оценки целого ряда весовых 
характеристик ЛА и~в~первую очередь $m_0$~--- взлетной массы самолета 
и~$m_{\mathrm{т}}$~--- стартовой массы заправляемого топлива. Основным 
соотношением весового анализа на всех этапах проектирования служит так 
на\-зы\-ва\-емое уравнение весового баланса, которое может быть пред\-став\-ле\-но 
в~сле\-ду\-ющем общем \mbox{виде}:
  $$
  m_0=\sum\limits_i m_i(m_0).
  $$
  Здесь суммирование ведется по всем агрегатам ЛА, включая целевую 
нагрузку и~за\-прав\-ля\-емое топ\-ли\-во. Масса многих агрегатов ЛА, в~част\-ности 
масса агрегатов конструкции планера и~потребная масса топ\-ли\-ва, зависит от 
общей массы поднимаемого в~воздух изделия. Априори эти зависимости не 
известны, но име\-ющий\-ся опыт создания ЛА позволяет на начальных этапах 
проектирования оценить характер этих зависимостей и~получить первое 
приближение взлетной массы ЛА и~потребного запаса топ\-ли\-ва. Как правило, 
первое при\-бли\-же\-ние взлетной массы вычисляется по прототипам 
проектируемого изделия. В~[5] была представлена одна из возможных 
реализованных в~МФО схем по\-стро\-ения первого приближения взлетной 
массы~$m_0^*$ путем анализа выборки прототипов из базы данных готовых 
изделий. При этом уравнение весового баланса рас\-смат\-ри\-ва\-лось в~виде
  $$
  m_0^*= m^*_{\mathrm{пуст}} +m^*_{\mathrm{пол\_нагр}}\,,
  $$
  
  \vspace*{-7pt}
  
  \noindent
  где
  
  \vspace*{-3pt}
  
  \noindent
  $$
  m^*_{\mathrm{пол\_нагр}} =m_{\mathrm{нагр}} +m_{\mathrm{сн}} + 
m_{\mathrm{т}}\,.
  $$
  
  \vspace*{-4pt}
  
  \noindent
Здесь $m^*_{\mathrm{пуст}}$~--- масса пустого ЛА; $ 
m^*_{\mathrm{пол\_нагр}}$~--- масса полезной нагрузки, вклю\-ча\-ющая массу 
целевой на\-груз\-ки~$ m_{\mathrm{нагр}}$, за\-да\-ва\-емую в~ТТЗ, массу снаряжения 
ЛА  и~массу топ\-ли\-ва $ m_{\mathrm{т}}$, потребного для полета с~заданной 
на\-груз\-кой на рас\-сто\-яние $L_{\mathrm{крейс}}$ в~крейсерском режиме. При 
этом в~результате анализа выборки прототипов вы\-чис\-ля\-лись 

\vspace*{-4pt}

 \noindent
\begin{align*}
m^*_{\mathrm{пол\_нагр}} &= f\left(L_{\mathrm{крейс}};\ m_{\mathrm{нагр}}\right);\\
 m^*_{\mathrm{пуст}}  &= g(m^*_{\mathrm{пол\_нагр}}).
 \end{align*}

\vspace*{-4pt}
  
 
  На этапе анализа компоновочной схемы ЛА, когда известны конструктивные 
параметры основных агрегатов ЛА, итерационный процесс весового анализа 
позволяет получить второе и,~возможно, по\-сле\-ду\-ющие при\-бли\-же\-ния взлетной 
массы и~других весовых характеристик ЛА. При этом основное соотношение 
весового баланса преобразуется в~сис\-те\-му уравнений, в~которой участвуют 
весовые па\-ра\-мет\-ры всех агрегатов, со\-став\-ля\-ющих компоновочную схему. 
В~МФО такая сис\-те\-ма уравнений реализована в~рамках основной весовой 
модели ЛА~--- дерева конструкции изделия. Дерево конструкции представляет 
собой иерархическую структуру, построенную на множестве агрегатов ЛА 
с~помощью бинарного отношения вхож\-де\-ния. Под агрегатами здесь 
понимаются детали, узлы и~сис\-те\-мы, а~так\-же собранные конструкции 
определенной конфигурации, в~том чис\-ле такие, как планер, пустое изделие, 
самолет во взлетной конфигурации, последний слу-\linebreak\vspace*{-12pt}

\columnbreak

\noindent
жит корнем дерева 
конструкции. Дерево конструкции выстраивается постепенно в~процессе 
развития проекта. 

На рис.~3 показано дерево конструкции ЛА на начальном 
этапе формирования облика. Выделенные серым цветом агрегаты на этом этапе 
не имеют подчиненных вершин, которые называют терминальными агрегатами.
  



  С каждой вершиной дерева конструкции связаны соотношения, 
определяющие массу, статические моменты цент\-ра массы и~моменты инерции 
соответствующего агрегата. На правой панели рис.~3 показаны соотношения 
для масс. Для нетерминальных вершин дерева конструкции эти соотношения 
выстраиваются автоматически суммированием по подчиненным агрегатам. Что 
касается терминальных агрегатов, то весовые па\-ра\-мет\-ры некоторых из них 
задаются или следуют из ТТЗ, как, например, масса целевой нагрузки и~масса 
размещаемого на борту специального оборудования. Это может относиться 
и~к~такому важ\-но\-му агрегату, как силовая установка ЛА, если проект заранее 
ориентирован на определенный тип серийного или раз\-ра\-ба\-ты\-ва\-емо\-го 
двигателя. Весовые характеристики большинства агрегатов конструкции ЛА, 
как правило, заранее не могут быть известны и~должны вы\-чис\-лять\-ся в~процессе 
формирования облика. Существуют различные подходы к~по\-стро\-ению 
алгоритмов их вы\-чис\-ле\-ния. Например, путем анализа прототипов, или по 
габаритным моделям агрегатов, или по специально разработанным так 
называемым <<весовым формулам>>. Многие из этих методов описаны 
в~учебных пособиях по проектированию ЛА (например, в~[9]). Пользователи 
могут выбрать любые из них или использовать свои наработки. В~МФО 
реализованы механизмы включения широкого класса соотношений в~систему 
уравнений весового баланса. Иерархическая структура дерева конструкции 
позволяет свести приведенную выше сис\-те\-му соотношений к~одному 
уравнению весового баланса относительно переменной~$m_0$:
  \begin{multline*}
  m_0=m_{\mathrm{кр}}(m_0) +m_{\mathrm{ф}}(m_0) +
  m_{\mathrm{оп}}(m_0)+ m_{\mathrm{впу}}(m_0)+{}\\
  {}+ m_{\mathrm{об}}(m_0)+ m_{\mathrm{су}}(m_0)+ 
m_{\mathrm{т}}(m_0)+ {}\\
{}+m_{\mathrm{сн}}(m_0, m_{\mathrm{нагр}}) + 
m_{\mathrm{нагр}}.
  \end{multline*}
%
При заданной массе целевой нагрузки $ m_{\mathrm{нагр}}$ и~заданных 
зависимостях остальных слагаемых от~$m_0$ это уравнение может быть 
решено чис\-лен\-но. В~дереве конструкции аналогичные соотношения могут 
быть выписаны и~для статических моментов, и~для моментов инерции. 
Положения цент\-ров масс и~моменты инерции агрегатов дерева конструкции 
могут быть приближенно получены из габаритных\linebreak моделей агрегатов. Заметим, 
что детализация дерева конструкции позволяет, изменяя со\-став терминальных вершин, 
последовательно уточнять значения весовых характеристик ЛА. 
Уточнение \mbox{весовых} характеристик ЛА происходит на всех этапах жизненного 
цикла ЛА. Задача весового анализа на этапе формирования облика состоит 
в~построении информационной весовой модели ЛА~[10, 11], которая может 
быть использована при решении многих задач весового проектирования~[12], 
производства и~эксплуатации~[13].

\section{Аэродинамический анализ на~этапе формирования 
облика}

  Под аэродинамическим анализом на этапе формирования облика понимается 
оценка значений ЛТХ  
и~ВПХ ЛА. Для расчета ЛТХ и~ВПХ 
не требуется моделирование движения ЛА вокруг цент\-ра масс, поэтому могут 
быть использованы упрощенные уравнения движения цент\-ра масс ЛА:
  \begin{equation*}
  \dot{X}=V\cos\theta \cos\Psi\,;\enskip
  \dot{H}=V\sin\theta\,;\enskip \dot{Z}=V\cos \theta\sin\Psi\,;
  \end{equation*}
  
  \vspace*{-12pt}
  
  \noindent
  \begin{multline*}
  m\dot{V}=P(\mathrm{M},H) \aleph \cos\alpha -c_x(\mathrm{M},H,c_y(\mathrm{M},\alpha)) qS -{}\\
  {}-m\g (\sin\theta 
 + f_{\mathrm{тр}} (H));
 \end{multline*}
 
 \vspace*{-12pt}
 
 \noindent
 \begin{multline*}
  mV\dot{\theta} =(P(\mathrm{M},H)\aleph \sin\alpha +c_y(\mathrm{M},\alpha)q S)\cos\gamma -{}\\
  {}- m\g\cos\theta\,;
  \end{multline*}
  
  \vspace*{-12pt}
  
\noindent
  \begin{gather*}
  mV\cos \theta\dot{\Psi}= (P(\mathrm{M},H)\aleph \sin\alpha +c_y(\mathrm{M},\alpha) qS)\sin\gamma\,;\\
  \dot{\eta} m_{\mathrm{т}} =c_p(\mathrm{M},H) P(\mathrm{M},H)\aleph\,.
  \end{gather*}
  Здесь $X$, $H$ и~$Z$~--- координаты цент\-ра масс полета в~земной сис\-те\-ме 
координат; $V$~--- скорость полета; $\theta$~--- угол тангажа траектории 
полета; $\Psi$~--- угол рыскания; $m$~--- текущая масса самолета; $\eta$~--- 
безразмерная фазовая переменная, обозначающая долю выработанного на 
текущий момент топлива ($m\hm= m_0\hm-\eta m_{\mathrm{т}}$, где $m_0$~--- 
взлетная масса самолета; $m_{\mathrm{т}}$~--- стартовая масса расходуемого 
топлива). В~качестве управления траекторией полета в~приведенной сис\-те\-ме 
уравнений приняты угол атаки и~угол крена самолета~--- $\alpha\hm\in [0, \alpha_{\max}]$ и~$\gamma\hm\in [-\pi, \pi]$, а~$\aleph\hm\in [0,1]$~--- 
параметр управления тягой силовой установки; $\mathrm{M}\hm=V/a(H)$~--- число Маха; 
$q\hm= \rho(H)V^2/2$~--- скоростной напор набегающего потока, где $a(H)$ 
и~$\rho(H)$~--- ско\-рость звука и~плот\-ность воздуха на высоте~$H$; 
$f_{\mathrm{тр}}(H)\hm=0$ при $H\hm>0$~--- коэффициент трения при 
движении самолета по земле. Изменяемая в~полете масса самолета здесь 
обозначена маленькой бук\-вой~$m$, чтобы не путать ее с~традиционным 
обозначением числа Маха.
  
  Состав контролируемых ЛТХ зависит от тех задач ЛА, которые определены 
в~ТТЗ, но всегда это диапазон скоростей и~практический потолок возможного 
полета, даль\-ность крейсерского полета, максимальная взлетная масса 
и~максимальная допустимая перегрузка. Для маневренных самолетов это могут 
быть скороподъемность, разгонные характеристики, минимальный радиус 
разворота и~другие производные от этих характеристики. Кроме того, 
обязательно контролируется длина взлета и~посадки ЛА. Средствами МФО все 
эти характеристики могут быть вычислены на основе анализа приведенных 
уравнений движения при заданных режимах полета и~заданных законах 
управ\-ле\-ния. 

\begin{figure*}[b] %fig4
\vspace*{1pt}
  \begin{center}  
    \mbox{%
\epsfxsize=160mm 
\epsfbox{fle-4.eps}
}
\end{center}
\vspace*{-9pt}
\Caption{Пример расчета квадратичной поляры}
%\end{figure*}
%\begin{figure*}[b] %fig5
  \vspace*{9pt}
  \begin{center}  
    \mbox{%
\epsfxsize=160mm 
\epsfbox{fle-5.eps}
}
\end{center}
\vspace*{-9pt}
  \Caption{Формулы для расчета тяги и~расхода топлива силовой установки}
  \end{figure*}
  
  В уравнения движения явно входят такие па\-ра\-мет\-ры, как характерная 
площадь несущей по\-верх\-ности~$S$, взлетная масса~$m_0$ 
и~$m_{\mathrm{т}}$~--- стартовая масса расходуемого топлива. Это 
цент\-раль\-ные па\-ра\-мет\-ры, вокруг которых строится в~дальнейшем целая 
стратегия весового проектирования, в~том чис\-ле весовой контроль и~весовой 
анализ. Кроме этих па\-ра\-мет\-ров в~уравнения ЛА явно входят коэффициент 
подъемной силы $c_y(\mathrm{M},\alpha)$ и~коэффициент лобового сопротивления 
$c_x(\mathrm{M},H,c_y)$, зависящий от действующей подъемной силы. Для 
приближенных расчетов эту за\-ви\-си\-мость принято интерполировать 
<<параболической полярой>>: 
  $$
  c_x(\mathrm{M},H,c_y) \cong c_{x0}(\mathrm{M},H)+A(\mathrm{M}) c_y^2,
  $$
   где $c_{x0}$~--- коэффициент лобового сопротивления при нулевой 
подъемной силе; $A(\mathrm{M})$~--- коэффициент отвала поляры. В~свою очередь, 
лобовое сопротивление при нулевой подъемной силе складывается из 
сопротивления аэродинамического трения и~сопротивления давления: 
  $$
   c_{x0}(\mathrm{M},H) \hm= c_{x0\_{\mathrm{тр}}} (\mathrm{M},H) 
\hm+c_{x0\_{\mathrm{д}}}(\mathrm{M}).
$$
  
    Все агрегаты ЛА, которые по компоновочной схеме находятся 
в~набегающем воздушном потоке, вносят свой вклад в~лобовое сопротивление~ЛА 
$$
c_{x0}(\mathrm{M},H)= \sum\limits_i c_{x0\_i}(\mathrm{M},H,S,\overline{d}_i)
$$ 
и~рассчитываются как функции своих конструктивных 
па\-ра\-мет\-ров~$\overline{d}_i$. Площадь~$S$~--- характерная площадь, 
относительно которой рассчитываются аэродинамические коэффициенты всех 
агрегатов компоновочной схемы. Нужно сделать замечание, что к~вы\-чис\-лен\-ным 
аэродинамическим коэффициентам должны быть при\-бав\-ле\-ны (или удалены) 
поправки, связанные с~влиянием агрегатов друг на друга. 



  
  Коэффициент подъемной силы компоновочной схемы ЛА также может быть 
оценен поагрегатно и~затем просуммирован в~соответствии с~их вкладом 
в~общее значение 
$$
c_y(\mathrm{M})= \sum\limits_i c_{y\_i}\left(\mathrm{M},S,\overline{d}_i\right).
$$ 
Кроме суммарного коэффициента подъемной силы поагрегатное вычисление 
поз\-во\-ля\-ет оценить положение цент\-ра приложения подъемной силы ЛА 
(\mbox{положение} аэродинамического фокуса). В~МФО реализованы алгоритмы 
приближенного расчета аэродинамических коэффициентов, основанные на 
использовании известных из литературы экспериментальных данных, 
проверенных на тео\-ре\-ти\-че\-ских моделях и~апробированных в~реальном 
проектировании. На рис.~4 приведены результаты расчета квад\-ра\-тич\-ной 
по\-ляры.
  


  Энергетические характеристики ЛА в~уравнениях движения задаются 
функциями максимальной тяги и~удельного секундного расхода топлива: 
$P(\mathrm{M},H)$ и~$\mathrm{Cp}\,(\mathrm{M},H)$. Эти характеристики могут быть заранее известны, если 
марка двигателей, уста\-нав\-ли\-ва\-емых на самолете, задана в~ТТЗ как условие 
проектирования. В~противном случае для определения характеристик силовой 
установки нужны специальные модели. Базовыми чис\-ло\-вы\-ми 
характеристиками силовой установки служат максимальная стартовая тяга 
$P_0\hm= P(0,0)$ и~удельный секундный расход топ\-ли\-ва в~крейсерском 
режиме~$\mathrm{Cp}_{\mathrm{кр}}$. Если заданы эти па\-ра\-мет\-ры, то
\begin{align*}
  P(\mathrm{M},H)&=P_0\overline{P}(\mathrm{M},H);\\
\mathrm{Cp}(\mathrm{M}, H)&=\mathrm{Cp}_{\mathrm{кр}}\overline{\mathrm{Cp}}(\mathrm{M},H).
  \end{align*}
  
  В МФО использованы в~качестве одного из возможных вариантов расчета 
характеристик силовой установки формулы, приведенные в~[14], зависящие от 
основных параметров двухконтурных турбореактивных двигателей. Фрагменты 
формул показаны на рис.~5.
  
  
  
\section{Анализ тактико-технических характеристик на~этапе формирования облика летательного аппарата}

   Основной задачей анализа компоновочной схемы ЛА считается проверка 
соответствия ТТХ требованиям, 
сформулированным в~ТТЗ. Расчет ТТХ базируется на моделях геометрического, 
весового и~аэродинамического анализа, о~которых шла речь выше. В~качестве 
примера рассмотрим требования, связанные с~обеспечением равномерного 
горизонтального полета в~диапазоне скоростей на разных высотах 
   $[V^*_{\min}(H), V^*_{\max}(H)]$. Эти требования сводятся к~сле\-ду\-ющим 
соотношениям:
   \begin{align*}
   V_{\min}(H)&= \sqrt{\fr{2m\g}{\rho(H^*) Sc_{y\max}}}\leq V^*_{\min}(H)\,;\\
   V_{\max}(H)&= \sqrt{\fr{2P_{\max}(H)}{\rho(H) Sc_{x0}(H)}}\geq 
V^*_{\max}(H)\,.
   \end{align*}
   
  Требования к~маневренности самолета могут задаваться значениями 
максимально допустимых перегрузок ($n_y ^*$), максимальной угловой 
скорости ($\dot{\Psi}^*_{\max}$) или минимального радиуса разворота 
($R^*_{\min}$) в~горизонтальной плос\-кости:

\noindent
  \begin{align*}
  n_{y\max}(V,H,m) &=\fr{qS}{m\g}\,c_{y\max}(\mathrm{M})\geq n_y^*\,;
  \\
  \dot{\Psi}_{\max} (V,H,m) &=\fr{\g}{V}\sqrt{n_{y\max}^2-1}\geq 
\dot{\Psi}_{\max}^*\,;\\
    R_{\min} (V,H,m) &=\fr{V}{\dot{\Psi}_{\max}(V,H,m)}\leq R^*_{\min}.
    \end{align*}
    
  Требования к~транспортным характеристикам самолета, как правило, 
задаются в~значениях максимальной даль\-ности крейсерского горизонтального 
полета на крейсерской высоте ($H_{\mathrm{кр}}$) и~с~постоянной 
крейсерской ско\-ростью ($V_{\mathrm{кр}}$). Требования даль\-ности 
крейсерского полета соответствуют соотношению 
  $$
  L(V_{\mathrm{кр}}, H_{\mathrm{кр}})= V_{\mathrm{кр}}\int\limits_0^1  
\fr{K(\mathrm{M}_{\mathrm{кр}}, H_{\mathrm{кр}},m) 
m_{\mathrm{т}}\,d\eta}{c_p(\mathrm{M}_{\mathrm{кр}}, H_{\mathrm{кр}}) m\g}\geq 
L_{\mathrm{кр}}^*,
  $$
 где $K(\mathrm{M}_{\mathrm{кр}}, H_{\mathrm{кр}}, m)$~--- так называемое 
аэродинамическое качество, зависящее от текущего режима по\-лета:
$$
K(\mathrm{M}_{\mathrm{кр}}, H_{\mathrm{кр}}, m)=\fr{ c_y(\mathrm{M}_{\mathrm{кр}}, 
H_{\mathrm{кр}}, m)}{c_x(\mathrm{M}_{\mathrm{кр}}, H_{\mathrm{кр}}, 
c_y(\mathrm{M}_{\mathrm{кр}}, H_{\mathrm{кр}}, m))}.
$$
 
  Для обеспечения различных сценариев анализа компоновочных схем 
в~МФО реализованы механизмы параметрических 
расчетов различных функциональных зависимостей, поиска экстремумов 
функций, интегрирования и~ряд других операций. Кроме того, реализована 
визуализации результатов рас\-четов.
  
  Как было сказано выше, в~ТТЗ могут быть сформулированы критерии, по 
которым следует оптимизировать ана\-ли\-зи\-ру\-емую компоновочную \mbox{схему}. 
В~данной работе вопросы методов и~моделей оптимизации не рассматривались, 
хотя инструменты разработанного МФО позволяют 
проводить многовариантный синтез и~анализ компоновочных схем. Отдельные 
параметры струк\-тур\-но-па\-ра\-мет\-ри\-че\-ской модели в~МФО могут быть 
объявлены варь\-и\-ру\-емы\-ми в~определенных диапазонах значений, 
опре\-де\-ля\-ющих об\-ласть поиска наилучших решений. Критерии оптимизации 
также могут быть заданы на уровне описания моделей компоновочных схем. 
Задание критериев оптимизации считается более тонкой задачей, которой 
в~свое время было посвящено много работ. В~част\-ности, для многоцелевых 
маневренных самолетов в~работах~\cite{15-fl, 16-fl, 17-fl} сформулирован ряд 
принципов по\-ста\-нов\-ки задач оптимизации и~выбора критериев. Для 
гражданских самолетов основные критерии лежат в~об\-ласти минимизации 
затрат при их производстве и~экономической эф\-фек\-тив\-ности их эксплуатации. 
На уровне формирования облика такие задачи могут быть по\-став\-ле\-ны, 
и~в~рамках механизмов разработанного модуля, по мнению авторов, могут быть 
реализованы.
  

\section{Заключение}

  Представленные в~данной работе модели анализа компоновочных схем 
ЛА вместе с~описанными в~\cite{5-fl} моделями синтеза 
представляют собой пример единой струк\-тур\-но-па\-ра\-мет\-ри\-че\-ской 
модели для решения задачи формирования облика ЛА. Традиционно 
формирование облика, т.\,е.\ разработка первоначального общего вида 
самолета, считалось исключительно творческим актом, чуть ли не искусством. 
Одним из первых математическую, вы\-чис\-ли\-тель\-ную сущ\-ность этой задачи 
понял замечательный авиаконструктор О.\,С.~Са\-мой\-ло\-вич. Он вместе 
с~математиком академиком П.\,С.~Крас\-но\-ще\-ко\-вым создал направление 
автоматизации задач формирования облика самолетов, которое можно 
определить как математическое проектирование ЛА. Тогда стало понятно, что 
формирование облика ЛА~--- многокритериальная задача математического 
программирования, правда, очень большой раз\-мер\-ности, в~которой 
задействованы несколько моделей из разных пред\-мет\-ных областей. Модуль 
формирования облика ЛА, пред\-став\-лен\-ный авторами на\-сто\-ящей статьи, во 
многом основан на тех идеях, которые были заложены в~работах 
О.\,С.~Самойловича, П.\,С.~Краснощекова, их сотрудников 
и~единомышленников. Модуль формирования облика ЛА был разработан 
с~применением технологии <<Генератор проектов>>, описанной в~\cite{18-fl}.
  
{\small\frenchspacing
 {\baselineskip=10.5pt
 %\addcontentsline{toc}{section}{References}
 \begin{thebibliography}{99}   
\bibitem{1-fl}
\Au{Егер С.\,М., Лисейцев~И.\,К., Самойлович~О.\,С.} Основы автоматизированного 
проектирования самолетов.~--- М.: Машиностроение, 1986. 232~с.
\bibitem{2-fl}
\Au{Вышинский Л.\,Л., Самойлович~О.\,С., Флёров~Ю.\,А.} Программный комплекс 
формирования облика летательных аппаратов~// Задачи и~методы автоматизированного 
проектирования в~авиастроении.~--- М.: ВЦ АН СССР, 1991. С.~24--42.
\bibitem{3-fl}
\Au{Самойлович О.\,С.} Формирование области существования самолета в~пространстве 
обобщенных проектных параметров.~--- М.: МАИ, 1994. 55~с. 
\bibitem{4-fl}
\Au{Вышинский Л.\,Л., Флёров~Ю.\,А.} Теоретические основы формирования весового 
облика самолета~// Информатика и~её применения, 2021. Т.~15. Вып.~4. С.~93--102. doi: 
10.14357/19922264210413. EDN: UGQQLU.
\bibitem{5-fl}
\Au{Вышинский Л.\,Л., Флёров~Ю.\,А.} Модели синтеза компоновочной схемы в~задаче 
формирования облика самолёта~// Информатика и~её применения, 2024. Т.~18. Вып.~1.  
С.~61--70. doi: 10.14357/19922264240109. EDN: DSPGKV.
\bibitem{6-fl}
\Au{Вышинский Л.\,Л., Флёров~Ю.\,А., Широков~Н.\,И.} Автоматизированная система 
весового проектирования самолетов~// Информатика и~её применения, 2018. Т.~12. Вып.~1. 
С.~18--30. doi: 10.14357/19922264180103.  EDN: YTTRBQ.
\bibitem{7-fl}
\Au{Микеладзе В.\,Г., Титов~В.\,М.} Основные геометрические и~аэродинамические 
характеристики самолетов и~ракет.~--- М.: Машиностроение, 1990. 144~с.
\bibitem{8-fl}
\Au{Вышинский Л.\,Л., Флеров~Ю.\,А.} Вычислительные модели в~задачах проектирования 
топливных систем самолетов~// Информационные технологии и~вычислительные системы, 
2022. №\,2. С.~70--83. doi: 10.14357/ 20718632220208. EDN: IHHTIH.
\bibitem{9-fl}
\Au{Шейнин В.\,М., Козловский~В.\,И.} Весовое проектирование и~эффективность 
пассажирских самолетов.~---  М.: Машиностроение, 1977. Т.~1. 343~с.
\bibitem{10-fl}
\Au{Вышинский~Л.\,Л., Флёров~Ю.\,А.} Информационная модель весового облика 
летательных аппаратов~// Информатика и~её применения, 2021. Т.~15. Вып.~1. С.~50--56. 
doi: 10.14357/19922264210107. EDN: BTLLPF.
\bibitem{11-fl}
\Au{Кантимиров С.\,А., Серебрянский~С.\,А.} Весовое проектирование летательного 
аппарата на цифровой платформе в~едином информационном пространстве жизненного 
цикла изделия~// Управ\-ле\-ние развитием крупномасштабных систем: Сб. трудов XIV 
Междунар. конф.~--- М.: ИПУ РАН, 2021. С.~1151--1161. doi: 10.25728/2486.2021.63.53.001. 
EDN: ZLYWZK.
\bibitem{12-fl}
\Au{Skobelev S.\,I., Strelets~D.\,Yu., Kuryanskii~M.\,K., Vyshinskii~L.\,L., Grinev~I.\,L.} Digital 
platform for aircraft weight design~// Aerospace Systems, 2022. Vol.~5. Iss.~4. P.~577--589. doi: 
10.1007/s42401-022-00154-w.
\bibitem{13-fl}
\Au{Вышинский Л.\,Л., Курьянский~М.\,К., Флеров~Ю.\,А.} Циф\-ро\-вая модель весового 
паспорта летательного аппарата~// Информатика и~её применения, 2019. Т.~13. Вып.~4.  
С.~3--10. doi: 10.14357/19922264190401. EDN: XXZPSJ.
\bibitem{14-fl}
Проектирование самолетов~/ Под ред. М.\,А.~Погосяна.~--- 5-е изд.~--- М.: 
Инновационное машиностроение, 2018. 864~с.
\bibitem{15-fl}
\Au{Краснощеков П.\,С., Федоров~В.\,В., Флеров~Ю.\,А.} Элементы математической теории 
принятия проектных решений~// Автоматизация проектирования, 1996. №\,2. С.~15--23.
\bibitem{16-fl}
\Au{Платунов В.\,С.} Методология системных военно-на\-уч\-ных исследований авиационных 
комплексов.~--- М.: Дельта, 2005. 344~с.
\bibitem{17-fl}
\Au{Мышкин Л.\,В.} Прогнозирование развития авиационной техники: теория и~практика.~--- 
М.: Физматлит, 2006. 304~с.
\bibitem{18-fl}
\Au{Флёров Ю.\,А., Вышинский~Л.\,Л.} Автоматизация проектирования прикладных 
информационных вы\-чис\-ли\-тель\-ных сис\-тем~// Информационные технологии 
и~вычислительные сис\-те\-мы, 2018. №\,3. С.~29--41. doi: 10.14357/20718632180303. EDN: 
YCMETJ.
\end{thebibliography}

 }
 }

\end{multicols}

\vspace*{-9pt}

\hfill{\small\textit{Поступила в~редакцию 07.05.24}}

%\vspace*{10pt}

%\pagebreak

\newpage

\vspace*{-28pt}

%\hrule

%\vspace*{2pt}

%\hrule


\def\tit{MODELS FOR ANALYZING LAYOUT SCHEMES
  IN~THE~PROBLEM~OF~AIRCRAFT~DESIGN}


\def\titkol{Models for analyzing layout schemes in the problem of~aircraft design}


\def\aut{L.\,L.~Vyshinsky and~Yu.\,A.~Flerov}

\def\autkol{L.\,L.~Vyshinsky and~Yu.\,A.~Flerov}

\titel{\tit}{\aut}{\autkol}{\titkol}

\vspace*{-8pt}


\noindent
Federal Research Center ``Computer Science and Control'' of the Russian Academy of Sciences, 44-2~Vavilov Str., Moscow 119333, Russian Federation






\def\leftfootline{\small{\textbf{\thepage}
\hfill INFORMATIKA I EE PRIMENENIYA~--- INFORMATICS AND
APPLICATIONS\ \ \ 2024\ \ \ volume~18\ \ \ issue\ 3}
}%
 \def\rightfootline{\small{INFORMATIKA I EE PRIMENENIYA~---
INFORMATICS AND APPLICATIONS\ \ \ 2024\ \ \ volume~18\ \ \ issue\ 3
\hfill \textbf{\thepage}}}

\vspace*{4pt}
    
  
       
      
      \Abste{The problems of analyzing the characteristics of the designed aircraft at the 
stage of its appearance formation are considered. The characteristic feature of these tasks is the lack 
of sufficient information about the aircraft design which appears only at the stages of preliminary 
and detailed design.  In essence, these are the tasks of technical forecasting for a rather limited set 
of parameters that can be operated by the designer at this stage. The main task of image formation is 
synthesis of the aircraft layout scheme and construction of its parametric representation. 
Composition diagrams in design practice represent one of the main design documents and serve as a 
prototype of the developed product. The present work is devoted to the presentation of 
mathematical models designed to build estimates of weight, aerodynamic, flight and technical,  
take-off, and landing characteristics of the aircraft using the parameters of its layout and the 
subsequent verification of compliance of the obtained estimates with the requirements for the 
designed product.}
      
      \KWE{mathematical modeling; design automation; aircraft; aircraft layout; aircraft 
characteristics}
      
\DOI{10.14357/19922264240301}{VWQBMD}

\vspace*{-12pt}


    
     % \Ack

%\vspace*{-3pt}

%\noindent



  \begin{multicols}{2}

\renewcommand{\bibname}{\protect\rmfamily References}
%\renewcommand{\bibname}{\large\protect\rm References}

{\small\frenchspacing
 {%\baselineskip=10.8pt
 \addcontentsline{toc}{section}{References}
 \begin{thebibliography}{99}
      \bibitem{1-fl-1}
      \Aue{Eger, S.\,M., I.\,K.~Liseytsev, and O.\,S.~Samoylovich.} 1986. \textit{Osnovy 
avtomatizirovannogo proektirovaniya samoletov} [Fundamentals of aircraft automated design]. 
Moscow: Mashinostroenie. 232~p.
      \bibitem{2-fl-1}
\Aue{Vyshinsky, L.\,L., O.\,S.~Samoylovich, and Yu.\,A.~Flerov.} 1991. Programmnyy kompleks 
formirovaniya oblika letatel'nykh apparatov [Program complex for forming the appearance of 
aircraft]. \textit{Zadachi i~metody avtomatizirovannogo proektirovaniya v~aviastroenii} [Tasks 
and methods of computer-aided design in aircraft industry]. Moscow: CC USSR AS. 24--42.
      \bibitem{3-fl-1}
      \Aue{Samoylovich, O.\,S.} 1994. \textit{Formirovanie oblasti su\-shchest\-vo\-va\-niya samoleta 
v~prostranstve obobshchennykh pro\-ekt\-nykh parametrov} [Formation of the 
area of existence of the aircraft in the space of generalized design parameters]. 
Moscow: MAI. 55~p. 
      \bibitem{4-fl-1}
\Aue{Vyshinskiy, L.\,L., and Yu.\,A.~Flerov.} 2021. Teoreticheskie osno\-vy formirovaniya 
vesovogo oblika samoleta [Theoretical foundation of formation of aircraft weight appearance]. 
\textit{Informatika i~ee Primeneniya~--- Inform. Appl.} 15(4):93--102. doi: 
10.14357/19922264210413. EDN: UGQQLU.
\bibitem{5-fl-1}
\Aue{Vyshinskiy, L.\,L., and Yu.\,A.~Flerov.} 2024. Modeli sinteza komponovochnoy skhemy 
v~zadache formirovaniya oblika samoleta [Synthesis models of layout scheme in the task of 
forming an aircraft image]. \textit{Informatika i~ee Primeneniya~--- Inform. Appl.} 18(1):61--70. 
doi: 10.14357/ 19922264240109. EDN: DSPGKV.
\bibitem{6-fl-1}
\Aue{Vyshinskiy, L.\,L., Yu.\,A.~Flerov, and N.\,I.~Shirokov.} 2018. Avtomatizirovannaya 
sistema vesovogo proektirovaniya samoletov [Computer-aided system of aircraft weight design]. 
\textit{Informatika i~ee Primeneniya~--- Inform. \mbox{Appl.}} 12(1):18--30. doi: 
10.14357/19922264180103. EDN: \mbox{YTTRBQ}.
\bibitem{7-fl-1}
      \Aue{Mikeladze, V.\,G., and V.\,M.~Titov.} 1990. \textit{Osnovnye geometricheskie 
i~aerodinamicheskie kharakteristiki samoletov i~raket} [Basic geometric and aerodynamic 
characteristics of aircraft and missiles]. Moscow: Mashinostroenie. 144~p.
\bibitem{8-fl-1}
\Aue{Vyshinskiy, L.\,L., and Yu.\,A.~Flerov.} 2022. Vychislitel'nye modeli v~zadachakh 
proektirovaniya toplivnykh sis\-tem samoletov [Computational models in aircraft fuel system 
design problems]. \textit{Informatsionnye tekhnologii i~vychislitel'nye sistemy} [J.~Information 
Technologies Computing Systems] 2:70--83. doi: 10.14357/20718632220208. EDN: \mbox{IHHTIH}.
\bibitem{9-fl-1}
\Aue{Sheynin, V.\,M., and V.\,I.~Kozlovskiy.} 1977. \textit{Vesovoe proektirovanie i~effektivnost' 
passazhirskikh samoletov} [Weight design and efficiency of passenger aircraft]. Moscow: 
Ma\-shi\-no\-stro\-enie.  Vol.~1. 343~p.
\bibitem{10-fl-1}
\Aue{Vyshinskiy, L.\,L., and Yu.\,A.~Flerov.} 2021. Informatsionnaya model' vesovogo oblika 
letatel'nykh apparatov [Information model of aircraft weight profile]. \textit{Informatika i~ee 
Primeneniya~--- Inform. Appl.} 15(1):50--56. doi: 10.14357/ 19922264210107. EDN: BTLLPF.
\bibitem{11-fl-1}
\Aue{Kantimirov, S.\,A., and S.\,A.~Serebryanskiy.} 2021. Vesovoe proektirovanie letatel'nogo 
apparata na tsifrovoy platforme v~edinom informatsionnom prostranstve zhiznennogo tsikla 
izdeliya [Weight design of an aircraft on a~digital platform in a~single information space of the 
product life cycle]. \textit{14th Conference  (International) on Management of Large-Scale System 
Development}. Moscow: IPU RAN. 1151--1161. doi: 10.25728/2486.2021.63.53.001. EDN: 
ZLYWZK.
\bibitem{12-fl-1}
      \Aue{Skobelev, S.\,I., D.\,Yu.~Strelets, M.\,K.~Kuryanskii, L.\,L.~Vyshinskii, and 
I.\,L.~Grinev.} 2022. Digital platform for aircraft weight design. \textit{Aerospace Systems} 
5(4):577--589. doi: 10.1007/s42401-022-00154-w.
\bibitem{13-fl-1}
      \Aue{Vyshinskiy, L.\,L., M.\,K.~Kuryansky, and Yu.\,A.~Flerov.} 2019. Tsifrovaya model' 
vesovogo pasporta letatel'nogo apparata [Digital model of the aircraft's weight passport]. 
\textit{Informatika i~ee Primeneniya~--- Inform. Appl.} 13(4):3--10. doi: 
10.14357/19922264190401. EDN: XXZPSJ.
\bibitem{14-fl-1}
      Pogosyan, M.\,A., ed. 2018. \textit{Proektirovanie samoletov} [Aircraft design]. 5th ed. 
Moscow: Innovatsionnoe ma\-shi\-no\-stro\-enie. 864~p.
\bibitem{15-fl-1}
      \Aue{Krasnoshchekov, P.\,S., V.\,V.~Fedorov, and Yu.\,A.~Flerov.} 1997.  Elementy 
matematicheskoy teorii prinyatiya pro\-ekt\-nykh resheniy [Elements of the mathematical theory of 
design decision making].  \textit{Avtomatizatsiya pro\-ek\-ti\-ro\-va\-niya} [Design Automation] 1:15--18.
\bibitem{16-fl-1}
      \Aue{Platunov, V.\,S.} 2005. \textit{Metodologiya sistemnykh voyenno-nauchnykh 
issledovaniy aviatsionnykh kompleksov} [Methodology of systemic military scientific research of 
aviation complexes]. Moscow: Delta. 344~p.
\bibitem{17-fl-1}
      \Aue{Myshkin, L.\,V.} 2006. \textit{Prognozirovanie razvitiya avia\-tsi\-on\-noy tekhniki: 
teoriya i~praktika} [Forecasting the development of aviation technology: Theory and practice]. 
Moscow: Fizmatlit. 304~p.
\bibitem{18-fl-1}
      \Aue{Flerov, Yu.\,A., and L.\,L.~Vyshinskiy.} 2018. Av\-to\-ma\-tizatsiya proektirovaniya 
prikladnykh informatsionnykh vy\-chis\-li\-tel'\-nykh sis\-tem [Computer-aided design of applied 
information computing systems]. \textit{Informatsionnye tekhnologii i~vychislitel'nye sis\-te\-my} 
[J.~Information Technologies Computing Systems] 3:29--41. doi: 10.14357/ 20718632180303. 
EDN: YCMETJ.
     


\end{thebibliography}

 }
 }

\end{multicols}

\vspace*{-6pt}

\hfill{\small\textit{Received May 7, 2024}} 

\vspace*{-18pt}

\Contr

\vspace*{-3pt}


      \noindent
      \textbf{Vyshinsky Leonid L.} (b.\ 1941)~--- Candidate of Science (PhD) in physics and 
mathematics, leading scientist, Federal Research Center ``Computer Science and Control'' of the 
Russian Academy of Sciences, 44-2~Vavilov Str., Moscow 119333, Russian Federation; 
\mbox{wyshinsky@mail.ru} 
      
      \vspace*{3pt}
      
      \noindent
      \textbf{Flerov Yuri A.} (b.\ 1942)~--- Corresponding Member of the Russian Academy of 
Sciences, Doctor of Science in physics and mathematics, professor, Deputy Director, Federal 
Research Center ``Computer Science and Control'' of the Russian Academy of Sciences,  
44-2~Vavilov Str., Moscow 119333, Russian Federation; \mbox{fler@ccas.ru}
    

\label{end\stat}

\renewcommand{\bibname}{\protect\rm Литература} 
      