%\newcommand{\Var}{\ensuremath{{\rm\mathbb{V}ar}}}

\renewcommand{\figurename}{\protect\bf Figure}
\renewcommand{\tablename}{\protect\bf Table}

\def\stat{lukashenko}


\def\tit{APPLYING COMPUTER-ASSISTED TOOLS TO~LITERARY TRANSLATION: THE CASE 
OF~PUNCTUATION}

\def\titkol{Applying computer-assisted tools to~literary translation: The case 
of~punctuation}

\def\autkol{V.\,A.~Nuriev}

\def\aut{V.\,A.~Nuriev$^1$}

\titel{\tit}{\aut}{\autkol}{\titkol}

%{\renewcommand{\thefootnote}{\fnsymbol{footnote}}
%\footnotetext[1] {The study was carried out under state order to the Karelian Research 
%Centre of the Russian Academy of Sciences (Institute of Applied Mathematical 
%Research KarRC RAS) and supported by the Russian Foundation for Basic Research, 
%projects 18-07-00187, 18-07-00147, 18-07-00156, 19-07-00303.}}

\renewcommand{\thefootnote}{\arabic{footnote}}
\footnotetext[1]{Federal Research Center ``Computer Science and Control'' of the Russian 
Academy of Sciences, \mbox{nurieff.v@gmail.com}}


\index{Nuriev V.\,A.}
\index{Нуриев В.\,А.}


\def\leftfootline{\small{\textbf{\thepage}
\hfill INFORMATIKA I EE PRIMENENIYA~--- INFORMATICS AND
APPLICATIONS\ \ \ 2024\ \ \ volume~18\ \ \ issue\ 3}
}%
 \def\rightfootline{\small{INFORMATIKA I EE PRIMENENIYA~---
INFORMATICS AND APPLICATIONS\ \ \ 2024\ \ \ volume~18\ \ \ issue\ 3
\hfill \textbf{\thepage}}}

\vspace*{-6pt}


    




\Abste{The article highlights the opportunities that the integration of computer-assisted tools 
into the literary translator's workbench provides in navigating the complexities of interlingual 
punctuation asymmetry. Across natural languages, punctuation principles can vary, which results 
in differences in the punctuation repertoire and its usage rules. For literary translators, 
understanding these differences is essential to accurately convey the artistic integrity of the 
source text. While interlingual punctuation asymmetry has long been studied, modern 
advancements in computer and language sciences, particularly, the corpus-based approach, offer 
new theoretical and practical insights that can greatly enhance the translation process. This 
approach allows for a~deeper understanding of punctuation nuances leading to more informed 
translation choices and decisions. Thus, the article contributes to the recent trend of applying 
computer-assisted tools to literary translation.}

\KWE{computer-assisted literary translation; punctuation; asymmetry between languages; 
corpus-based translation studies; parallel corpus; English; French; Russian}

\DOI{10.14357/19922264240314}{MFTPWK}


%\vspace*{1pt}


\vskip 12pt plus 9pt minus 6pt

 \thispagestyle{myheadings}

 \begin{multicols}{2}

 \label{st\stat}

    \section{Introduction}
    
   % \vspace*{-12pt}
    
    \noindent
  Existing scholarship has a twofold understanding of punctuation: as a system of 
symbols employed in written speech and as a set of rules that regulate the 
functioning of this system. The purpose of punctuation marks is to make the 
written text clearer and convey certain traits of oral speech through graphic means. 
Often, this purpose~--- to convey a certain orality in the written text~--- causes the 
discussions about punctuation principles to gravitate to the predominance of 
rhythm and melody. In many countries, such as Russia, France, and others, the 
training of radio and TV presenters follows this principle grounded in auditory 
analogies between punctuation marks and pause and intonation~[1, 2]. However, 
rhythm, melody, and punctuation do not always correlate casting doubt on the 
consistency and appropriateness of this principle. Furthermore, different natural 
languages may use the same punctuation marks differently.
  
  Notably, in punctuation usage, European languages fall into two groups~[3]. 
The first one has a communicative system of punctuation: the use of punctuation 
depends on the meaning of a~given message (French, English, Italian, etc.). The 
second group is characterized by a~morpho-syntactic (grammatical) system, i.\,e., 
the use of punctuation is defined by the formal grammatical parsing of the sentence 
(German, Czech, Polish, Russian, etc.).
  
  Alongside encoded in rules and quite clear-cut preferences, there are `oscillation 
zones' in punctuation where variability becomes possible opening up new stylistic 
opportunities. While individual punctuation has not yet attracted a considerable 
amount of research attention, some studies have explored this question. For 
example, in~[4], I.~\mbox{Ser{\ptb{\hspace*{-0.5pt}\!\c{c}}}\,\!a} approaches punctuation as an art 
and shows how some writers use it unconventionally. In~[5], L.\,C.~Mitchell also 
sees punctuation preferences as a distinct manifestation of individual style.
  
  The individual dimension of punctuation reveals itself in writers' reflections that 
vividly portray one's personal attitude toward a given punctuation mark. Kurt 
Vonnegut advises not to use semicolons as they represent ``absolutely nothing''~[6, 
p.~30]. Julien Gracq has problems with colons, because they have ``an active 
function of elimination'' and ``mark the place of a mini-collapse in discourse''~[7, 
p.~258]. Alexander Genis thinks that the ellipsis is ``almost always'' needless~[8, 
p.~131]. Camille Laurens does not conceive of literature without parentheses both 
arguing that they ``inject the necessary oxygen into the text~--- doubt, hesitation, 
detail,'' and dubbing them ``the apotheosis of nuance''~[9].
  
  All this sums up, firstly, to the conclusion about the significance of punctuation 
for the literary translator's work and, secondly, to the question of whether modern 
computer and language sciences have methods and tools that would simplify this 
work and lead to more informed translation choices.
  
  Furthermore, in its following parts, the article contributes to a growing trend of 
applying computer-assisted tools to literary translation. For more information on 
this trend see the recent collective volume \textit{Computer-assisted literary 
translation}~[10].
  
    \section{Punctuation in Translation Studies: Introducing the~Corpus-Based Approach}
    
    \noindent
    Historically, research on interlingual punctuation asymmetry has been closely 
connected with literary translation. The 1919 Soviet translation requirements, 
favoring ``objective accuracy even in the smallest details,'' declare that ``all the 
punctuation choices of the author must be sacredly preserved by the 
translator''~[11, p.~23]. In 1937, after the critical analysis of translations from 
French, M.~Stolyarov advances these ideas paying special attention to punctuation 
and putting accuracy at the forefront. According to him, the sentence-final 
punctuation in translation should follow the original intent as much as possible, 
especially when the translated writer shows a strong inclination to meticulously 
construct a~paragraph. Within-sentence punctuation should also serve to preserve 
the stylistic integrity of the source text, although here the placement of punctuation 
marks strongly depends on the rules of the target language. The task of the 
translator, therefore, is to study the role that the writer assigns to punctuation in 
their artistic mastery~[12, p.~252].
    
    Both Stolyarov's and Chukovsky's observations, valuable as they might be, 
remain largely speculative without relying on statistics or big data. More recent 
works do not attempt at systematization either; however, they touch upon some 
critical aspects of punctuation study. In~[13],  two chapters are devoted to punctuation. It 
demonstrates how Andrei Sinyavsky expands the semantic and expressive potential 
of punctuation marks. For instance, the ellipsis\footnote{In the specialized 
literature on punctuation, the ellipsis is also referred to as suspensions or 
suspension points.} tends to convey reticence, suspension, and tension, and in his 
work, it becomes a~politically charged sign of unreliability, circumspection, fear, 
(self-)censorship. Sometimes, English translations omit these ellipses in a tangible 
desire to normalize and adjust the translated text to the expectations of the target 
language and audience~[13, p.~130--140].
    
    This argument aligns with findings of K.~Malmkj$\ae$r in~[14] where the contrastive analysis 
involving 9~English versions of Hans Christian Andersen's \textit{The steadfast 
tin soldier} is performed. She  identifies punctuation differences in 
the source and target languages and then shows how translators in their efforts to 
normalize (and simplify) syntactic structure mostly disregard the author's 
punctuation manner, even though it contributes to the narrative. Yet, these 
observations are still very limited: only one example is thoroughly examined.
    
    Some scholars in their studies of punctuation focus on a given mark and its 
specific functions. In so doing, K.~Seagal argues that the Russian ellipsis can 
convey an emotional subtext: he compares Mikhail Sholokhov's \textit{The fate of 
a~man} and its two English translations (by H.\,C.~Stevens and R.~Daglish) to 
reveal the role of the ellipsis in Sholokhov's writing and how the writer's 
punctuation style is interpreted in translation. In the original story, there are 
76~ellipses and its English versions contain~33 and 17~ellipses. Their number 
significantly diminished (by 2.3~times in Stevens's version and by~4.5~times in 
Daglish's one, respectively), which, according to the author, indicates the reticence 
of the English language to openly show emotions~[15, p.~47] and the oppressive 
influence of the target punctuation.
    
    A pivotal contrastive study of punctuation is~[16]. It analyzes the 
comparative use of punctuation marks in Russian and English newspaper 
editorials. For each language, the corpus totals 20,000~words. The corpora are 
compiled manually and the number of punctuation marks is calculated using the 
MS Word search and replace function. Results show that the comma, colon, and 
dash are used more often in Russian. The data are verified in the corpora of literary 
texts where again these punctuation marks are found to occur more frequently in 
Russian. Nonetheless, their frequency is lower in the newspaper corpus for both 
languages. Furthermore, taking the most frequent mark, the colon, as an example, 
the author identifies the reasons for interlingual asymmetry in its use and suggests 
guidelines for its translation from Russian into English. Evidently, the design of 
this study has some shortcomings~--- the corpora of a~very limited size, a small 
repertoire of genres, etc.~[16, p.~137--162]. At the same time, it is the first study 
to have adopted, albeit in its embryonic form, the corpus-based method to examine 
cross-linguistic discrepancies in the use of punctuation marks, which allowed for 
processing a~significantly larger data set compared to the pre-corpus era.
    
     So far, contrastive punctuation studies remain on the periphery of academic 
scholarship: there are very few such works (see [17--20, p.~121--150]. Yet, this 
field of research has recently seen some distinct changes resulting, for instance, in 
the publication of the collective volume \textit{Comparative punctuation}~[21].
     
    \section{Corpus Applications in~Literary~Translation: The~Case~of~Punctuation}
    
    \noindent
    If one asks a practicing translator whether they see any perceived benefits of 
academic advances in the computer-assisted research, corpus-based approach, and 
contrastive punctuation studies, the answer is not immediately obvious. Translators 
do not always feel the need to learn modern research methods or master the latest 
information tools. Though, it is difficult to deny that punctuation asymmetry 
between languages sometimes causes struggles in literary translation, which may 
be illustrated by the words of the eminent French translator 
A.~Cold$\acute{\mbox{e}}$fy-Faucard from her speech at the 2020 International 
Congress of Translators in Moscow: ``Oh, those dashes of Russian literature! 
$<\ldots>$~Starting in the 20th century, dashes became more commonly used in 
French literature but still much less than in Russian. How should we understand 
and translate these dashes?''\footnote[1]{See also reflections on the translation of 
punctuation marks in~[22, p.~140--156].}
    
To answer this question, the translator needs a broad knowledge of punctuation in both the source and target 
languages. Nevertheless, close reading, inherent to literary translation, often does not provide this knowledge. 
Recent advances in computer and language sciences may help to gain it: corpus-based methods and the latest 
information tools enable `distant reading'\footnote[2]{In~[23], F.~Moretti coined the term \textit{distant reading}.}~---
 a~computer-assisted quantitative analysis of information (on punctuation marks, for instance) that 
would otherwise be challenging to obtain. Distant reading gives translators the opportunity to `distance' 
themselves from the text they are working on. Studying a given phenomenon in large electronic text corpora 
allows one to check it against general trends and to see its specificity (for more on this, see~[24]).

    In so doing, B.~Blatt collects big data to test the literary canon and elicit 
whether celebrated writers consistently follow their own writing advices~[25]. Not 
only does he look at individual preferences but also tries to paint a bigger  
picture~--- to see whether books in general (the classics and the bestsellers) share 
a~given trait. Thus, the American writer Elmore Leonard once insisted on using no 
more than two or three exclamation points per 100,000~words of prose. The 
entirety of his literary career includes~45~novels and 3.4~mln words. As it 
turns out, in his works, he used~1,651~exclamation points, which is~16~times as 
many as he recommended. If he had followed his own advice, he would have been 
allowed only~102~exclamation points overall. Yet, compared to his many fellow 
writers, Leonard's usage of exclamation points is quite moderate. Furthermore, his 
overall frequency of this punctuation mark (49 per 100,000~words) ranks first 
lowest, ahead of Ernest Hemingway (59 per 100,000~words) and John Updike (88 
per~100,000 words). On the other end of the spectrum, with career usages  
of 844, 929, and 1005 per~100,000 words, are Sinclair Lewis, Tom Wolfe, and 
James Joyce, respectively~[25, p.~84--85]. Blatt rightfully argues that many style 
guides support Leonard's disdain for the exclamation point. They caution against 
its overuse: it should stress only the moments of a text that deserve extra attention. 
With additional statistical evaluations, Blatt discerns a general trend: in English 
prose, amateur writers use this punctuation mark much more than 
professionals~[25, p.~88--90] and its abundance remains a sign of either a distinct 
style or an insufficient writing experience. It is to note that in his text processing, 
Blatt uses Python and NLTK (Natural Language Toolkit) and he relies 
predominantly on the straightforward procedure of counting words or punctuation.
    
    Another tool for distant reading in literary translation may be supra-corpora 
databases (SCDs) that stock parallel sub-corpora of the Russian National 
Corpus~\cite{28-nr}. Since 2013, they have been developed at the FRC CSC RAS. 
With the aid of these databases, users may collect and process large data sets: the 
French parallel sub-corpus alone exceeds 7.5~mln words. To investigate 
a~given phenomenon, bilingual examples are automatically extracted from the text 
corpus. These examples are then analyzed, annotated, and saved by the user in the 
database. The SCDs have extensive search capabilities, including the ability to 
search for punctuation marks (for more on this, see~\cite{26-nr, 27-nr}). More 
particularly, the SCDs help to clarify what often goes unnoticed or unclear, with 
some translators making transformative choices that, from a certain normative 
perspective, may seem unusual.

 Speaking once more about the dynamics of 
punctuation usage, let us look at the Russian version\footnote[3]{Made by 
Yu.~Kotova and published in 2015.} of the internationally acclaimed French novel 
\textit{R$\acute{\mbox{e}}$parer les vivants} (2014) by Maylis de Kerangal. The 
source text uses zero ellipses, and the target text has as much as~24 of them. 
Nonetheless, before jumping to any conclusions, it is necessary to collect some 
additional statistics. The data from the Frantext corpus~\cite{29-nr}\footnote[4]{It 
totals 159~mln words.} show that French uses only 3.5~ellipses per 
1,000~words, whereas according to the data from~[30, p.~378]\footnote[5]{In~[30], 
the corpus is compiled manually and totals 22,7~mln words. The Russian 
National Corpus, the largest monolingual Russian corpus, does not provide the 
ability to search for the ellipsis. Thus, the data on the use of this punctuation mark 
in Russian are taken from this work.}, Russian uses 10.6~ellipses per 
1,000~words. In de Kerangal's original novel, the ellipsis is absent, which is 
clearly the stylistically significant choice made by the author, widely known for 
her scrupulous writing. In the Russian translation, this punctuation mark is used at 
a rate of~0.4 per 1,000~words, which is approximately 26.5~times less frequently 
than in contemporary Russian literature. Here, its number approaches zero but 
does not equal zero leaving some room to question translation choices. Should the 
translator accurately mirror the author's restraint toward the ellipsis, keeping the 
original ratio of individual preferences to common usage? The answer might lie in 
further quantitate analysis of the writer's other works to discern more clearly her 
punctuation preferences.
    
  In other words, the SCDs and, more generally, the corpus-based tools enable 
translators to determine if there is a punctuation usage asymmetry between the 
source and target languages. They help to learn how to deal with punctuation 
differentiations by studying samples from both monolingual and parallel corpora. 
Therefore, corpus data become a reliable source of specifically targeted 
knowledge: by their means, one can identify the translation patterns for a given 
language pair and translation direction. In this way, modern corpus technologies 
provide a blueprint of translation possibilities that go far beyond punctuation to 
encompass every linguistic element that contributes to writing. Moreover, the 
corpus tools are helpful in research on individual translation decisions and 
constraints. Such is the case, to take a notable example, of the renowned Soviet 
translator N.~Gal', who, in her translation of \textit{Little Prince} (1943) by 
Antoine de Saint-Exup$\acute{\mbox{e}}$ry, went against the general trend, 
decreasing the original number of ellipses from~189 to~165.
  
    \section{Concluding Remarks}
    
    \noindent
     The role of punctuation marks in literary translation is quite evident. 
Alongside other elements, punctuation is responsible for the artistic integrity of the 
source text, and the translator lacking sufficient understanding of interlingual 
punctuation differences may distort this integrity. Modern computer-assisted tools 
and, more specifically, corpus resources provide literary translators with broad 
opportunities to verify and expand existing punctuation knowledge. The obtained 
data highlight the need to continue research in the field of contrastive punctuation. 
This will help to better define functional asymmetries between the same 
punctuation marks in different languages, which, in turn, will lead to better 
translation choices: it would be easier to decide if the source text allows the 
translator to follow a regular program or if there is a need for some specific 
decision.
     
  \Ack
  \noindent
  The research was carried out using the infrastructure of the Shared Research 
Facilities ``High Performance Computing and Big Data'' (CKP ``Informatics'') of 
FRC CSC RAS (Moscow). The research was supported by the Russian Science 
Foundation (project No.\,23-28-00548).



\renewcommand{\bibname}{\protect\rmfamily References}


%\vspace*{-6pt}

{\small\frenchspacing
{ %\baselineskip=10.35pt
\begin{thebibliography}{99}
\bibitem{1-nr-1}
\Aue{Lyashenko, B.} 2007. \textit{Khochu k~mikrofonu: professional'nye sovety 
diktoru} [I~want to go to the microphone: Professional advices for the announcer]. 
Moscow: Aspekt Press. 125~p. EDN: SUESBX.
\bibitem{2-nr-1}
\Aue{Lehtinen, M.} 2007. L'interpr$\acute{\mbox{e}}$tation prosodique des 
signes de ponctuation: L'exemple de la lecture radiophonique de 
l'$\acute{\mbox{E}}$tranger d'Albert Camus. \textit{L'Information 
Grammaticale} 113:23--31.
\bibitem{3-nr-1}
\Aue{Shcherba, L.} 1935. Punktuatsiya [Punctuation]. \textit{Literaturnaya 
entsiklopediya} [Literary encyclopedia]. Moscow: OGIZ RSFSR. 9:366--370.
Available at: {\sf http://feb-web.ru/feb/litenc/encyclop/le9/le9-3661.htm}
(accessed August~27, 2024).
\bibitem{4-nr-1}
\Aue{\mbox{Ser{\ptb{\hspace*{-0.5pt}\!\c{c}}}a}, I.} 2012. 
\textit{Esth$\acute{\mbox{e}}$tique de la ponctuation}. Paris, France: 
Gallimard. 320~p.
\bibitem{5-nr-1}
\Aue{Mitchell, L.\,C.} 2020. \textit{Mark my words: Profiles of punctuation in 
modern literature}. New York,  London: Bloomsbury Academic. 192~p.
\bibitem{6-nr-1}
\Aue{Vonnegut, K.} 2005. \textit{A~man without a country.} New York, NY: 
Seven Stories Press. 74~p.
\bibitem{7-nr-1}
\Aue{Gracq, J.} 1980. \textit{En lisant, en $\acute{\mbox{e}}$crivant}. Mayenne, 
France: Jos$\acute{\mbox{e}}$ Corti. 302~p.
\bibitem{8-nr-1}
\Aue{Genis, A.} 1999. \textit{Dovlatov i~okrestnosti. Filologicheskiy roman} 
[Dovlatov and surroundings. Philological novel]. Moscow: VAGRIUS. 304~p.
\bibitem{9-nr-1}
\Aue{Laurens, C.} 2014. Parenth$\grave{\mbox{e}}$se(s). \textit{La Licorne} 52. 
Available at: {\sf http://licorne.edel.univ-poitiers.fr/index.php?id=\linebreak 5828} (accessed 
July~29, 2024).
\bibitem{10-nr-1}
Rothwell, A., A.~Way, and R. Youdale, eds.  2024. \textit{Computer-assisted 
literary translation}. Abingdon, New York: Routledge. 302~p.
\bibitem{11-nr-1}
\Aue{Chukovsky, K.} 1919. Perevody prozaicheskie [Prose translations]. 
\textit{Printsipy khudozhestvennogo perevoda} [Principles of literary 
translation].  Peterburg: Vsemirnaya li\-te\-ra\-tu\-ra. 7--24.
\bibitem{12-nr-1}
\Aue{Stolyarov, M.} 1937. Iskusstvo perevoda khudozhestvennoy prozy [The art of 
translating fiction]. \textit{Literaturnyy kritik} [Literary Critic] 
5-6:242--254.
\bibitem{13-nr-1}
\Aue{May, R.} 1994. \textit{The translator in the text: On reading Russian 
literature in English}. Evanston, IL: Northwestern University Press. 209~p.
\bibitem{14-nr-1}
\Aue{Malmkj$\ae$r, K.} 1997. Punctuation in Hans Christian Andersen's stories 
and in their translations into English. \textit{Nonverbal communication and 
translation: New perspectives and challenges in literature, interpretation and 
the media}. Ed. F.~Poyatos. Amsterdam, Philadelphia: John Benjamins Publishing 
Co. 151--162. doi: 10.1075/btl.17.13mal.
\bibitem{15-nr-1}
\Aue{Seagal, K.\,Ya.} 2014. Punktuatsiya kak sredstvo so\-zda\-niya 
emotsional'nogo podteksta (na materiale rasskaza M.\,A.~Sho\-lo\-kho\-va 
``Sud'ba cheloveka'' i~ego perevodov na angliyskiy yazyk) [Punctuation as 
a~means of creating of emotional subtext (on the material of the 
M.\,A.~Sholokhov's story ``The fate of a man'' and its translations into 
English)]. \textit{Izvestiya RAN. Ser. literatury i~yazy\-ka} [Bulletin of the 
Russian Academy of Sciences: Studies in Literature and Language] 73(6):38--50. 
EDN: TEQTBF.
\bibitem{16-nr-1}
\Aue{Bystrova-McIntyre, T.} 2007. Looking at the overlooked: A~corpora study 
of punctuation use in Russian and English. \textit{Transl. Interpret. Stu.}  
2(1):137--162. doi: 10.1075/ tis.2.1.04bys.

\bibitem{19-nr-1} %17
\Aue{Wollin, L.} 2018. Punctuation: Providing the setting for translation? 
\textit{Stud. Neophilol.} 90(1):37--49. doi: 10.1080/ 00393274.2018.1531254.

\bibitem{17-nr-1} %18
\Aue{Brusasco, P., and E.~Corino.} 2020. Translating punctuation. 
\textit{Translating and comparing languages: Corpus-based insights}. Eds. 
S.~Granger and M.-A.~Lefer. Louvain-la-Neuve, Belgium: Presses universitaires 
de Louvain. 101--122.
\bibitem{18-nr-1} %19
\Aue{N$\acute{\mbox{a}}$dvorn$\iota$kov$\acute{\mbox{a}}$, O.} 2020. The 
use of English, Czech and French punctuation marks in reference, parallel and 
comparable web corpora: A~question of methodology. \textit{Linguist. Prag.} 
30(2):30--50. doi: 10.14712/18059635.2020.1.2.

\bibitem{20-nr-1}
\Aue{Youdale, R.} 2020. \textit{Using computers in the translation of literary 
style: Challenges and opportunities}. London, New York: Routledge. 
242~p.
\bibitem{21-nr-1}
R$\ddot{\mbox{o}}$ssler, P., P.~Besl, and A. Saller, eds. 2021. 
\textit{Vergleichende Interpunktion~--- comparative punctuation}. Berlin, 
 Boston: De Gruyter. 454~p.
\bibitem{22-nr-1}
\Aue{Babkov, V.\,O.} 2022. \textit{Igra slov: praktika i~ideologiya 
khu\-do\-zhest\-ven\-no\-go perevoda} [Play on words: Practice and ideology of literary 
translation]. Moscow: AST, Corpus. 140--156.
\bibitem{23-nr-1}
\Aue{Moretti, F.} 2013. \textit{Distant reading}. London, U.K.: Verso. 254~p.
\bibitem{24-nr-1}
\Aue{Nuriyev, V.\,A.} 2022. Perevodcheskiy analiz teks\-ta s~primeneniem 
informatsionnykh resursov: redutsirovanie spekt\-ra modeley perevoda 
v~nad\-kor\-pus\-nykh ba\-zakh dan\-nykh [Computer-assisted textual analysis 
in translation: Reducing the spectrum of translation models in supracorpora 
databases]. \textit{Informatika i~ee Primeneniya~--- Inform. Appl.} 16(3):68--74. 
doi: 10.14357/19922264220309. EDN: UUWKDZ.
\bibitem{25-nr-1}
\Aue{Blatt, B.} 2017. \textit{Nabokov's favorite word is mauve: What the numbers 
reveal about the classics, bestsellers, and our own writing}. New York, NY: 
Simon \& Schuster. 84--85.

\bibitem{28-nr-1} %26
\Aue{Nuriyev, V.\,A., and V.\,I.~Karpov.} 2023. Metodologiya  
korpusno-orientirovannogo issledovaniya v~ob\-lasti kontrastivnoy punktuatsii 
[Methodology of the corpus-based studies in the field of contrastive punctuation]. 
\textit{Informatika i~ee Primeneniya~--- Inform. Appl.} 17(2):90--95. doi: 
10.14357/19922264230213. EDN: VBOZAO.

\bibitem{26-nr-1} %27
Natsional'nyy korpus russkogo yazyka [Russian National Corpus]. Available at: 
{\sf https://ruscorpora.ru} (accessed July~29, 2024).
\bibitem{27-nr-1} %28
\Aue{Nuriyev, V.\,A., and M.\,G.~Kruzhkov.} 2023. Korpusnye dannye pri 
kontrastivnom izuchenii punk\-tu\-a\-tsii [The parallel corpora perspective on studying 
contrastive punctuation]. \textit{Sistemy i~Sredstva Informatiki~--- Systems and 
Means of Informatics} 33(1):14--23. doi: 10.14357/ 08696527230102. EDN: 
JOUMFY.

\bibitem{29-nr-1}
\Aue{Inkova, O., and N.~Popkova.} 2017. Statistical data as information source 
for linguistic analysis of Russian connectors. \textit{Informatika i~ee 
Primeneniya~--- Inform. Appl.} 11(3):123--131. doi: 10.14357/19922264170314. 
EDN: ZGIGJZ.
\bibitem{30-nr-1}
\Aue{Zaliznyak, Anna~A., I.\,M.~Zatsman, and O.\,Yu.~Inkova.} 2017. 
Nadkorpusnaya baza dannykh konnektorov: postroenie sistemy terminov 
[Supracorpora database on connectives: Term system development]. \textit{Informatika 
i~ee Primeneniya~--- Inform. Appl.} 11(1):100--108. doi: 
10.14357/19922264170109. EDN: YOCMYN.
\bibitem{31-nr-1}
Frantext. Available at: https://www.frantext.fr (accessed July 29, 2024).
\bibitem{32-nr-1}
\Aue{Mukhin, M.\,Yu.} 2019. Statisticheskaya dinamika mnogotochiya. 
Pozavchera$\ldots$ Vchera$\ldots$ Segodnya$\ldots$ [Statistical dynamics of 
ellipsis. The day before yesterday$\ldots$ Yesterday$\ldots$ Today$\ldots$]. 
\textit{Fenomen nezavershennogo} [The phenomenon of the unfinished].  
Eds. T.\,A.~Snigireva and A.\,V.~Podchinenov. 2nd ed.  Ekaterinburg: Ural University Publs. 
373--384. doi: 10.15826/B978-5-7996-2470-5.16.
     
 
     \end{thebibliography} } }

\end{multicols}

\vspace*{-6pt}

\hfill{\small\textit{Received July 5, 2024}}

\vspace*{-18pt}

 \Contrl
  
  \noindent
  \textbf{Nuriev Vitaly A.} (b.\ 1980)~--- Doctor of Science (PhD) in philology, leading 
scientist, Federal Research Center ``Computer Science and Control'' of the Russian Academy of 
Sciences, 44-2~Vavilov Str., Moscow 119333, Russian Federation; 
\mbox{nurieff.v@gmail.com} 


%\vspace*{8pt}

%\hrule

%\vspace*{2pt}

%\hrule

%\vspace*{-7pt}

\newpage

\vspace*{-28pt}

\def\tit{КОМПЬЮТЕРНЫЕ ИНСТРУМЕНТЫ ДЛЯ~ОБРАБОТКИ ПУНКТУАЦИОННОГО КОМПОНЕНТА\\ 
В~ХУДОЖЕСТВЕННОМ ПЕРЕВОДЕ$^*$}

\def\titkol{Компьютерные инструменты для~обработки пунктуационного компонента 
в~художественном переводе}

\def\aut{В.\,А.~Нуриев}

\def\autkol{В.\,А.~Нуриев}

{\renewcommand{\thefootnote}{\fnsymbol{footnote}} \footnotetext[1]
{Работа выполнена при поддержке гранта РНФ (проект №\,23-28-00548) с использованием 
инфраструктуры Центра коллективного пользования <<Высокопроизводительные 
вычисления и большие данные>> (ЦКП <<Информатика>>) ФИЦ ИУ РАН (г.~Москва).}}



\titel{\tit}{\aut}{\autkol}{\titkol}

\vspace*{-11pt}

\noindent
Федеральный исследовательский центр <<Информатика и управление>> Российской 
академии наук, \mbox{nurieff.v@gmail.com}

\vspace*{1pt}

\def\leftfootline{\small{\textbf{\thepage}
\hfill ИНФОРМАТИКА И ЕЁ ПРИМЕНЕНИЯ\ \ \ том\ 18\ \ \ выпуск\ 3\ \ \ 2024}
}%
 \def\rightfootline{\small{ИНФОРМАТИКА И ЕЁ ПРИМЕНЕНИЯ\ \ \ том\ 18\ \ \ выпуск\ 3\ \ \ 2024
\hfill \textbf{\thepage}}}

\vspace*{-1pt}


  
  
  \Abst{Представлен обзор возможностей применения современных компьютерных 
инструментов для решения проблем межъязыковой пунктуационной асимметрии 
в~художественном переводе. Принципы пунктуирования не обладают универсальностью: 
в~разных естественных языках репертуар знаков препинания неодинаков, как и правила их 
расстановки. Литературному переводчику знание межъязыковых пунктуационных 
различий необходимо: пунктуационная составляющая наравне со всеми прочими отвечает 
за целостность исходного замысла, и переводчик, не имея достаточного представления о 
межъязыковых пунктуационных расхождениях, может эту целостность нарушить. 
Межъязыковые различия в~употреблении знаков препинания регистрируются в~научных 
работах давно, однако сейчас кор\-пус\-но-ори\-ен\-ти\-ро\-ван\-ный подход позволяет 
выйти на новый уровень теоретического и практического обобщения. В~статье 
демонстрируется, как информатика и современные компьютерные ресурсы могут 
обеспечить обработку пунктуационного наполнения в~литературном переводе. Ранее 
исследований, всецело сфокусированных на данном аспекте, не было. Таким образом, 
настоящее исследование вносит вклад в~развитие нового направления, которое 
разрабатывает способы интеграции компьютерных инструментов в художественный 
перевод с целью всесторонней оптимизации переводческой деятельности.}
  
  \KW{художественный перевод с применением компьютерных инструментов; 
пунктуация; межъязыковая асимметрия; корпусное переводоведение; параллельный 
корпус; английский; русский; французский}
  
\DOI{10.14357/19922264240314}{MFTPWK}



%\vspace*{-3pt}


 \begin{multicols}{2}

\renewcommand{\bibname}{\protect\rmfamily Литература}
%\renewcommand{\bibname}{\large\protect\rm References}

{\small\frenchspacing
{\baselineskip=10.5pt
\begin{thebibliography}{99}
%\vspace*{-3pt}
\bibitem{1-nr}
\Au{Ляшенко Б.} Хочу к микрофону: профессиональные советы диктору.~--- 
М.: Аспект Пресс, 2007. 125~c. EDN: SUESBX.
\bibitem{2-nr}
\Au{Lehtinen M.} L'interpr$\acute{\mbox{e}}$tation prosodique des signes de 
ponctuation : L'exemple de la lecture radiophonique de 
l'$\acute{\mbox{E}}$tranger d'Albert Camus~// L'Information Grammaticale, 
2007. No.\,113. P.~23--31.
\bibitem{3-nr}
\Au{Щерба Л.} Пунктуация~// Литературная энциклопедия.~--- М.: 
ОГИЗ РСФСР, 1935. Т.~9. С.~366--370. 
{\sf http://feb-web.ru/feb/litenc/encyclop/le9/le9-3661.htm}.
\bibitem{4-nr}
\Au{\mbox{Ser{\ptb{\hspace*{-1pt}\!\c{c}}}a~I.}} Esth$\acute{\mbox{e}}$tique de la 
ponctuation.~--- Paris, France: Gallimard, 2012. 320~p.
\bibitem{5-nr}
\Au{Mitchell L.\,C.} Mark my words: Profiles of punctuation in modern 
literature.~--- New York, London: Bloomsbury Academic, 2020. 
192~p.
\bibitem{6-nr}
\Au{Vonnegut K.} A~man without a~country.~--- New York, NY, USA: Seven 
Stories Press, 2005. 74~ p.
\bibitem{7-nr}
\Au{Gracq J.} En lisant, en $\acute{\mbox{e}}$crivant.~--- Mayenne, France: 
Jos$\acute{\mbox{e}}$ Corti, 1980. 302~p.
\bibitem{8-nr}
\Au{Генис А.} Довлатов и окрестности. Филологический роман.~--- М.: 
ВАГРИУС, 1999. 304~с.
\bibitem{9-nr}
\Au{Laurens C.} Parenth$\grave{\mbox{e}}$se(s)~// La Licorne, 2014. 
No.\,52. {\sf http://licorne.edel.univ-poitiers.fr/index.php?id=5828}.
\bibitem{10-nr}
Computer-assisted literary translation~/ Eds. A.~Rothwell, A.~Way, 
R.~Youdale.~--- Abingdon, New York: Routledge, 2024. 
302~p.
\bibitem{11-nr}
\Au{Чуковский К.} Переводы прозаические~// Принципы художественного 
перевода.~--- Петербург: Всемирная литература, 1919. С.~7--24.
\bibitem{12-nr}
\Au{Столяров М.} Искусство перевода художественной прозы~// 
Литературный критик, 1937. №\,5-6. С.~242--254.
\bibitem{13-nr}
\Au{May R.} The translator in the text: On reading Russian literature in English.~--- Evanston, IL, USA: Northwestern University Press, 1994. 209~p.
\bibitem{14-nr}
\Au{Malmkj$\ae$r K.} Punctuation in Hans Christian Andersen's stories and in 
their translations into English~// Nonverbal communication and translation: New 
perspectives and challenges in literature, interpretation and the media~/ Ed. 
F.~Poyatos.~--- Amsterdam, Philadelphia: John 
Benjamins Publishing Co., 1997. P.~151--162. doi: 10.1075/ btl.17.13mal.
\bibitem{15-nr}
\Au{Сигал К.\,Я.} Пунктуация как средство создания эмоционального 
подтекста (на материале рассказа М.\,А.~Шолохова <<Судьба человека>> 
и~его переводов на английский язык)~// Известия РАН. Серия литературы 
и~языка, 2014. Т.~73. №\,6. 
 С.~38--50. EDN: TEQTBF.

\bibitem{16-nr}
\Au{Bystrova-McIntyre T.} Looking at the overlooked: A~сorpora study of 
punctuation use in Russian and English~// Transl. Interpret. Stu., 
2007. Vol.~2. No.\,1. P.~137--162. doi: 10.1075/tis.2.1.04bys.

\bibitem{19-nr} %17
\Au{Wollin L.} Punctuation: Providing the setting for translation?~// Stud. 
Neophilol., 2018. No.\,90. Sup.~1. P.~37--49. doi: 
10.1080/00393274.2018.1531254.

\bibitem{17-nr} %18
\Au{Brusasco P., Corino~E.} Translating punctuation~// Translating and 
comparing languages: Corpus-based insights~/ Eds. S.~Granger, M.-A.~Lefer.~--- 
Louvain-la-Neuve, Belgium: Presses universitaires de Louvain, 2020. P.~101--
122.
\bibitem{18-nr} %19
\Au{N$\acute{\mbox{a}}$dvorn$\iota$kov$\acute{\mbox{a}}$~O}. The use of 
English, Czech and French punctuation marks in reference, parallel and 
comparable web corpora: A~question of methodology~// Linguist. Prag., 
2020. Vol.~30. Iss.~2. P.~30--50. doi: 10.14712/ 18059635.2020.1.2.

\bibitem{20-nr}
\Au{Youdale R.} Using computers in the translation of literary style: Challenges 
and opportunities.~--- London, New York: Routledge, 2020. 
242~p.
\bibitem{21-nr}
Vergleichende Interpunktion~--- comparative punctuation~/ Eds. 
P.~R$\ddot{\mbox{o}}$ssler, P.~Besl, A.~Saller.~--- Berlin, Boston: De Gruyter, 2021. 454~ p.
\bibitem{22-nr}
\Au{Бабков В.\,О.} Игра слов: практика и идеология художественного перевода.~--- М.: 
АСТ, Corpus, 2022. С.~140--156.
\bibitem{23-nr}
\Au{Moretti F.} Distant reading.~--- London, U.K.: Verso, 2013. 254~p.
\bibitem{24-nr}
\Au{Нуриев В.\,А.} Переводческий анализ текста с применением 
информационных ресурсов: редуцирование спектра моделей перевода в 
надкорпусных базах данных~// Информатика и~её применения, 2022. Т.~16. 
Вып.~3. С.~68--74. doi: 10.14357/19922264220309. EDN: UUWKDZ.
\bibitem{25-nr}
\Au{Blatt B.} Nabokov's favorite word is mauve: What the numbers reveal about 
the classics, bestsellers, and our own writing.~--- New York, NY, USA: Simon \& 
Schuster, 2017. P.~84--85.

\bibitem{28-nr} %26
\Au{Нуриев В.\,А., Карпов~В.\,И.} Методология кор\-пус\-но-ори\-ен\-ти\-ро\-ван\-но\-го 
исследования в области конт\-рас\-тив\-ной пунктуации~// Информатика и её 
применения, 2023. Т.~17. Вып.~2. С.~90--95. doi: 10.14357/ 19922264230213. 
EDN: VBOZAO.
\bibitem{26-nr} %27
Национальный корпус русского языка (НКРЯ). {\sf https://ruscorpora.ru}.
\bibitem{27-nr} %28
\Au{Нуриев В.\,А., Кружков~М.\,Г.} Корпусные данные при контрастивном 
изучении пунктуации~// Системы и~средства информатики, 2023. Т.~33. №\,1. 
С.~14--23. doi: 10.14357/08696527230102.  EDN: JOUMFY.

\bibitem{29-nr}
\Au{Inkova~O., Popkova~N.} Statistical data as information source for linguistic 
analysis of Russian connectors~// Информатика и~её применения, 2017. Т.~11. 
Вып.~3. С.~123--131. doi: 10.14357/19922264170314. EDN: ZGIGJZ.
\bibitem{30-nr}
\Au{Зализняк Анна~А., Зацман~И.\,М., Инькова~О.\,Ю.} Надкорпусная база 
данных коннекторов: по\-стро\-ение сис\-те\-мы терминов~// Информатика и~её 
применения, 2017. Т.~11. Вып.~1. С.~100--108. doi: 
10.14357/ 19922264170109. EDN: YOCMYN.
\bibitem{31-nr}
Frantext. {\sf https://www.frantext.fr}.
\bibitem{32-nr}
\Au{Мухин М.\,Ю.} Статистическая динамика многоточия. Позавчера$\ldots$ 
Вчера$\ldots$ Сегодня$\ldots$~// Феномен незавершенного~/ Под ред. 
Т.\,А.~Снигиревой, А.\,В.~Подчиненова.~--- 2-е изд.~--- Екатеринбург: Изд-во Урал. ун-
та, 2019. С.~373--384. doi: 10.15826/B978-5-7996-2470-5.16.
    
  
    
\end{thebibliography}
} }

\end{multicols}

 \label{end\stat}

 \vspace*{-9pt}

\hfill{\small\textit{Поступила в~редакцию 05.07.2024}}


%\renewcommand{\bibname}{\protect\rm Литература}
\renewcommand{\figurename}{\protect\bf Рис.}
\renewcommand{\tablename}{\protect\bf Таблица}