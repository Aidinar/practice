\def\ss2{\mathop{\sum\sum}}

\def\stat{sinitsin2}

\def\tit{УСЛОВНО-ОПТИМАЛЬНАЯ ФИЛЬТРАЦИЯ В~СТОХАСТИЧЕСКИХ СИСТЕМАХ СО~СЛУЧАЙНЫМИ
ПАРАМЕТРАМИ И~НЕ~РАЗРЕШЕННЫХ ОТНОСИТЕЛЬНО
ПРОИЗВОДНЫХ}

\def\titkol{Условно-оптимальная фильтрация в~СтС %стохастических системах 
со~случайными параметрами и~не~разрешенных относительно
производных}

\def\aut{И.\,Н.~Синицын$^1$}

\def\autkol{И.\,Н.~Синицын}

\titel{\tit}{\aut}{\autkol}{\titkol}

\index{Синицын И.\,Н.}
\index{Sinitsyn I.\,N.}


%{\renewcommand{\thefootnote}{\fnsymbol{footnote}} \footnotetext[1]
%{Работа 
%выполнена при поддержке Программы развития МГУ, проект №\,23-Ш03-03. При анализе 
%данных использовалась инфраструктура Центра коллективного пользования 
%<<Высокопроизводительные вычисления и~большие данные>> 
%(ЦКП <<Информатика>>) ФИЦ ИУ РАН (г.~Москва)}}


\renewcommand{\thefootnote}{\arabic{footnote}}
\footnotetext[1]{Федеральный исследовательский центр <<Информатика и~управление>> Российской академии наук, %\mbox{sinitsin@dol.ru}; Московский авиационный институт, 
\mbox{kafedra802@yandex.ru}}


\vspace*{-12pt}




\Abst{Для наблюдаемых дифференциальных гауссовских стохастических сис\-тем (СтС), не 
разрешенных относительно производных (НРОП), со случайными па\-ра\-мет\-ра\-ми 
в~виде интегральных канонических представлений (ИКП) и~приводимых 
к~дифференциальным СтС, разработано методическое обеспечение и~алгоритм услов\-но-оп\-ти\-маль\-ной фильт\-ра\-ции и~анализа точ\-ности фильт\-ров.
Дан обзор результатов в~об\-ласти субоптимальных (СОФ) и~услов\-но-оп\-ти\-маль\-ных фильт\-ров (УОФ) и~приведены необходимые сведения из корреляционной 
теории ИКП и~многокомпонентных (МК) ИКП. Особое внимание уделено 
регрессионной линеаризации посредством МК ИКП. Представлено методическое 
обеспечение и~базовые алгоритмы УОФ для приведенных дифференциальных СтС НРОП. 
Для оценки точности УОФ используются среднеквадратичные регрессионные уравнения 
для условных вероятностных характеристик при фиксированном векторе случайных 
параметров, а~МК ИКП применяется для получения безусловных характеристик для 
случайных переменных параметров на основе МКМ. В~качестве примера рас\-смот\-рен УОФ 
для линейной СтС НРОП с~параметрическим шумом. Определены на\-прав\-ле\-ния дальнейших исследований.}


\KW{регрессионная среднеквадратичная линеаризация;
стохастическая система, не разрешенная относительно производных (СтС НРОП);
стохастический процесс (СтП);
услов\-но-оп\-ти\-маль\-ный фильтр (УОФ)}

\DOI{10.14357/19922264240303}{XCXLGD}
  
%\vspace*{-6pt}


\vskip 10pt plus 9pt minus 6pt

\thispagestyle{headings}

\begin{multicols}{2}

\label{st\stat}

\section{Введение}


В~[1] рассмотрены вопросы синтеза нормальных СОФ (НСОФ) для дифференциальных 
СтС НРОП. Представлены уравнения состояния 
и~наблюдения нелинейных дифференциальных СтС НРОП. Синтез НСОФ выполнен при следующих условиях: 
\begin{enumerate}[(1)]
\item отсутствуют 
пуассоновские шумы в~наблюдениях; 
\item коэффициент при гауссовском шуме не зависит 
от состояния.
\end{enumerate}
 Подробно рассмотрен синтез НСОФ при аддитивных шумах в~уравнениях 
состояния и~наблюдения.
%
В~[2] для нелинейных интег\-ро-диф\-фе\-рен\-ци\-аль\-ных\linebreak СтС (ИДСтС), не 
разрешенных относительно производных и~приводимых к~дифференциальным методом 
сингулярных ядер, разработаны алгоритмы аналитического моделирования нормальных 
СтП, при этом нелинейность под интегралом может быть разрывной, а~также синтеза 
НСОФ для он\-лайн-об\-ра\-бот\-ки информации в~ИДСтС. Предложены алгоритмы оценки 
качества НСОФ на основе теории чувствительности.
%
В~[3] разработано методическое обеспечение для негладких правых частей уравнений 
СтС НРОП. Рассмотрены вопросы аналитического моделирования нормальных СтП на 
основе нелинейных регрессионных моделей. Особое внимание уделено методам 
гауссовской фильтрации и~экстраполяции.\linebreak
 Изуче\-ны вопросы услов\-но-оп\-ти\-маль\-ной 
фильт\-ра\-ции и~экстраполяции для СтС НРОП с~па\-ра\-мет\-ри\-че\-ски\-ми шумами.
В~[4] разработано методическое и~алгоритмическое обеспечение аналитического 
моделирования оценивания и~\mbox{идентификации} для существенно нестационарных 
процессов (например, ударных) в~СтС НРОП. Дан обзор профильных пуб\-ли\-ка\-ций 
и~изучены основные классы регрессионных уравнений СтС НРОП. Основные ре\-зуль\-таты: 
\begin{enumerate}[(1)]
\item для общего вида нелинейных СтС НРОП приведены оптимальные алгоритмы 
совместной фильтрации и~распознавания; 
\item для линейных гауссовских СтС НРОП 
получены простые алгоритмы; 
\item для~СтС НРОП, линейных относительно состояния~$X_t$ 
и~нелинейных относительно наблюдений~$Y_t$, получены соответствующие  
алгоритмы; 
\item в случае~3 методом нормальной аппроксимации (МНА) получен простой 
алгоритм.
\end{enumerate}
 Приводится иллюстративный пример скалярной нелинейной гауссовской СтС 
НРОП.

В~[5] для наблюдаемых гауссовских дифференциальных СтС НРОП со случайными 
параметрами в~виде ИКП, приводимых к~дифференциальным СтС, разработано 
методическое обеспечение анализа точности субоптимальной фильтрации, основанное 
на МНА при фиксированном векторе параметров для условных вероятностных 
характеристик и~регрессионной линеаризации посредством ИКП безусловных 
характеристик.

В настоящей статье рассматривается задача разработки методического обеспечения 
и~алгоритмов услов\-но-оп\-ти\-маль\-ной фильт\-ра\-ции процессов в~СтС НРОП со случайными 
переменными параметрами, описываемыми МК ИКП, и~моделирования точ\-ности УОФ. 
В~разд.~2 приводятся сведения по теории МК ИКП 
и~их преобразованиям. Раздел~3 посвящен вопросам приведения дифференциальных СтС 
НРОП к~дифференциальным СтС. Основные результаты по теории УОФ приводятся 
в~разд.~4. В~разд.~5 рассматривается УОФ в~системах с~мультипликативными 
шумами. Приводится иллюстративный пример. Заключение содержит основные выводы и~направления дальнейших исследований.


\section{Многокомпонентные интегральные канонические представления}

Как известно~[6], для векторного СтП $X(t) =[ X_1(t) \cdots X_n(t)]^{\mathrm{T}}$ 
справедливы следующие выражения для однокомпонентных ИКП:
    \begin{multline}
    X(t) = m^x (t) + \displaystyle\int\limits_\Lambda V(\la) x (t,\la) \,d\la\,,\\
    X_h(t) = m_h^x (t) \displaystyle \int\limits_\Lambda V(\la) x_h (t,\la) \,d\la\enskip 
\left(h=\overline{1,n}\right).
          \label{e2.1-s2}
     \end{multline}
%
Здесь 
$$
m^x (t) = \lk m_1^x (t) \cdots m_n^x (t)\rk^{\mathrm{T}}.
$$
 При этом ИКП матрицы 
ковариационных функций $K^x (t, t') \hm= \lk K_{hl}^x (t,t')\rk$ имеет вид
       \begin{multline}
    K_{hl}^x (t,t') = \int\limits_\Lambda G(\la) x_h (t,\la) \overline{x_l(t',\la)} 
\,d\la \\
 \left( h,l=\overline{1,n}\right),
\label{e2.2-s2}
\end{multline}
где белый шум $V(\la)$  определяется формулой
    $$
    V(\la) = \sss_{h=1}^n \int\limits_T \overline{a_h (t,\la)} X_h^0 (t)\,dt,
    $$
а его интенсивность $G(\la)$ равна

\vspace*{-4pt}

\noindent
   \begin{multline*}
    G(\la) ={}\\
    {}= \sss_{h,l=1}^n \int\limits_T \int\limits_T \int\limits_T \overline{a_h (t,\la)} a_l (t',\la') K_{hl}^x (t,t')\, dtdt'd\la'.
\end{multline*}
Для вычисления координатных функций~$x_h (t,\la)$ и~функций~$a_h (t,\la)$ 
получим  уравнения

\vspace*{-4pt}

\noindent
    \begin{multline}
    x_h (t,\la) =\fr{1}{G(\la)} \sss_{l=1}^n \int_T a_l (t',\la) K_{hl}^x 
(t,t') \,dt' \\\
 \left(h=\overline{1,n}\right);
\label{e2.3-s2}
\end{multline}

\vspace*{-12pt}

\noindent
\begin{multline}
\sss_{h=1}^n \int\limits_T \overline{a_h(t,\la)} x_h (t, \la') \,dt = \delta (\la-\la'),\\
    \int\limits_\Lambda \overline{a_l(t',\la)} x_h (t, \la)\, d\la = \delta_{hl} (t-t').
    \label{e2.4-s2}
    \end{multline}
    

Если параметр~$\la$ принимает все возможные значения, принадлежащие нескольким 
областям $\Lambda_1\tr \Lambda_r$, то такое МК ИКП 
векторного СтП $X(t) \hm=\lk X_1(t) \cdots X_n(t)\rk^{\mathrm{T}}$ и~матрицы ковариационных 
функций $K^x (t,t') \hm= \lk K_{hl}^x (t,t')\rk$  будут иметь вид:
  \begin{multline}
    X(t) = m^x (t) +\sss_{\rho =1}^r \int\limits_{\Lambda_\rho} V_\rho (\la) x_\rho 
(t,\la) \,d\la,
\\
    X_h(t) = m_h^x(t) +\sss_{\rho =1}^r \int\limits_{\Lambda_\rho} V_\rho (\la) 
x_{\rho h} (t,\la) \,d\la\\
 \left(h=\overline{1,n}\right);
\label{e2.5-s2}
\end{multline}

\vspace*{-12pt}

\noindent
        \begin{multline}
    K_{hl}(t, t') =\sss_{\rho =1}^r \int\limits_{\Lambda_\rho} G_\rho (\la) x_{\rho h} (t,\la)\overline{x_{\rho l} (t',\la)}\, d\la\\ 
\left(h,l=\overline{1,n}\right),
\label{e2.6-s2}
\end{multline}
где $V_1(\la)\tr V_r(\la)$~--- некоррелированные белые шумы, определяемые 
формулой
        \begin{equation*}
    V_\rho (\la) = \sss_{h=1}^n  \int\limits_{T} \overline{a_{\rho h} (t,\la)} X_h^0 
(t) \,dt\  \left(\la\in \Lambda_\rho; \enskip \rho=\overline{1,n}\right).
%\label{e2.7-s2}
\end{equation*}

Интенсивности белых шумов~$V_\rho (\la)$ определяются формулой:

\vspace*{-4pt}

\noindent
     \begin{multline*}
     G_\rho (\la) = {}\\
     {}=\sss_{h,l=1}^n   \int\limits_{\Lambda_\rho}\int\limits_{T}\int\limits_T 
\overline{a_{\rho h} (t,\la)}a_{\rho l}(t',\la') K_{hl}^x (t,t')\, dt 
dt'd\la'\\
 \left(\la\in \Lambda_\rho; \enskip \rho=\overline{1,r}\right).
%\label{e2.8-s2}
\end{multline*}
Для вычисления координатных функций~$x_{\rho h} (t',\la')$ и~функций~$a_{\rho h} 
(t,\la)$ используются формулы:

\pagebreak

\noindent
    \begin{multline}
    x_{\rho h} (t,\la) =\fr{1}{G_\rho (\la)} \sss_{l=1}^n \int\limits_T a_{\rho l} 
(t',\la)K_{hl}^x(t,t')\,dt'\\
 \left(\la\in \Lambda_\rho; \enskip 
\rho=\overline{1,r};\enskip h=\overline{1,n}\right);
\label{e2.9-s2}
\end{multline}

\vspace*{-12pt}

\noindent
 \begin{multline}
    \sss_{h=1}^n \int\limits_T \overline{a_{\mu h} (t,\la)}a_{\rho h} 
(t,\la')\,dt=\delta_{\rho \mu} \delta (\la-\la')\\[-2pt]
 \left(\la\in \Lambda_\mu; 
\enskip \la'\in \Lambda_\rho;\enskip \rho,\mu=\overline{1,r}\right);
\label{e2.10-s2}
\end{multline}

\vspace*{-6pt}

\noindent
    \begin{equation*}
    \sss_{\rho=1}^r \int\limits_{\Lambda_\rho} \overline{a_{\rho l} (t',\la)}x_{\rho 
h} (t,\la)\,d\la=\delta_{hl} \delta (t-t')\  
\left(h,l=\overline{1,n}\right),
%\label{e2.11-s2}
\end{equation*}
выражающие необходимые и~достаточные условия представления векторного СтП 
посредством МК ИКП.

\smallskip

\noindent
\textbf{Теорема~2.1.}\ \textit{В условиях}~(\ref{e2.3-s2}), (\ref{e2.4-s2}) 
\textit{или}~(\ref{e2.9-s2}), (\ref{e2.10-s2}) \textit{из ИКП}~(\ref{e2.1-s2}) 
\textit{или}~(\ref{e2.5-s2}) \textit{вытекает МК ИКП 
матрицы его ковариационных функций}~(\ref{e2.2-s2}) \textit{или}~(\ref{e2.6-s2}).

\smallskip

\noindent
\textbf{Теорема~2.2.}\ \textit{Если известно ИКП векторного СтП}~(\ref{e2.1-s2}), \textit{то вектор математического ожидания 
и~матрица ковариационных функций линейного преобразования $Y_t (t) \hm= \mathrm{A}_t 
X_t$ допускают МК ИКП, определяемые формулами}~[6]:
    \begin{gather*}
     X_p(t) = m_p^x(t) + \!\sss_{\rho=1}^r \int\limits_{\Lambda_\rho} V_\rho (\la) 
x_{\rho p}(t,\la) \,d\la\enskip \left(p=\overline{1,n}\right);
%\label{e2.12-s2}
\\
     Y_p(s) = m_p^y(s) + \!\sss_{\rho=1}^r \int\limits_{\Lambda_\rho} V_\rho (\la) 
y_{\rho p}(s,\la) \,d\la\enskip \left(p=\overline{1,m}\right);
%\label{e2.13-s2}
\\
     y_{\rho p}(s,\la)) = \!\sss_{h=1}^n  A_{ph} x_{\rho h} (t,\la)\enskip 
\left(p=\overline{1,m}\right);
%\label{e2.14-s2}
\end{gather*}

\vspace*{-12pt}

\noindent
    \begin{multline*}
%    \left.
 %     \begin{array}{rl}
     K^y (s,s') =\lk K_{pq}^y (s,s')\rk,\\
     K_{pq}^y (s,s') = \displaystyle \sss_{\rho=1}^r \int\limits_{\Lambda_\rho} G_\rho (\la) 
y_{\rho p}(s,\la) \overline{y_{\rho p}(s',\la)}\,d\la\\
\left(p,q=\overline{1,m}\right).
%      \end{array}
 %     \right\}
  %    \label{e2.15-s2}
      \end{multline*}

В задачах нелинейной корреляционной теории невырожденные безынерционные 
скалярные и~векторные существенно нелинейные преобразования $Y_t \hm=\varphi_t(X_t)$ 
заменяют оптимальными в~среднеквадратичном смысле линейными регрессионными 
преобразованиями~[7, 8]. Задача эквивалентной регрессионной линеаризации 
детерминированной векторной нелинейной функции $Y\hm=\varphi (X)$ при использовании 
критерия минимума среднеквадратичной ошибки совпадает с~классической задачей 
линейного регрессионного анализа. В~этом случае оптимальная линейная 
среднеквадратичная регрессия вектора~$Y$ на вектор~$X$ определяется формулой
        \begin{equation*}
     m^y(X) =gX+a;
     \end{equation*}
     где
     
     \vspace*{-3pt}
     
     \noindent
\begin{align}
      g&=K^{yx} (K^x)^{-1};     \label{e2.16-s2}\\ 
      a&=m^y-gm^x.\notag
\end{align}

\vspace*{-2pt}


Пусть $f(y,x)$~--- совместная плотность случайных векторов~$Y$ и~$X$; $m^x$ 
и~$K^x$~--- математическое ожидание и~ковариационная матрица вектора~$x$, $\mathrm{det}\,|K^x |\hm\ne 0$.  
Формула~(\ref{e2.16-s2}) при этом принимает вид:

\vspace*{-4pt}

\noindent
        \begin{multline}
    g=K^{yx} (K^x)^{-1} ={}\\
\!\!{}=\!\!\iin \!\lk m^y(x) - m^y\rk \left(x-m^x\right)^{\mathrm{T}} (K^x)^{-1} f_1 (x) \,dx,\!
    \label{e2.17-s2}
        \end{multline}
где  $f_1(x)$~--- плотность случайного вектора~$X$. Эта формула вместе с~приближенной формулой
        \begin{equation*}
    m^y(X) \approx m^y +g\left(X-m^x\right)
   % \label{e2.18-s2}
        \end{equation*}
дает статистическую линеаризацию регрессии~$m^y(X)$ по Казакову.

\smallskip

\noindent
\textbf{Теорема~2.3.}\ \textit{Если существуют конечные моменты первого и~второго порядка векторного СтП 
$X=X(t)$, то векторное нелинейное преобразование $Y\hm=\varphi(X)$ допускает линейную 
среднеквадратичную регрессию $Y\hm=Y(t)$ на $X\hm=X(t)$, определяемую по Казакову 
формулами}~(\ref{e2.16-s2}) \textit{и}~(\ref{e2.17-s2}).

\smallskip

Первый подход к~линеаризации основан на использовании формул для ИКП~$X_t$ 
в~тео\-ре\-ме~2.1 в~случае одной об\-ласти~$\Lambda$, а~второй подход~--- для нескольких 
областей~$\Lambda_\rho$. При этом имеют место сле\-ду\-ющие утверж\-де\-ния.

\smallskip

\noindent
\textbf{Теорема~2.4.}\ \textit{Оптимальная среднеквадратичная линеаризация посредством ИКП (тео\-ре\-ма~$2.1$) 
первого рода определяется сле\-ду\-ющи\-ми формулами}:

\vspace*{-4pt}

\noindent
\begin{multline*}
\varphi_t (X_t) \approx m_t^{(1)y} (X_t) = {}\\
{}=\varphi_{0t}^{(1)} \left(m_t^x, K_t^x\right) + 
g_t^{(1)} \left(m_t^x, K_t^x\right)\times{}\\
{}\times \lk \sss_{\rho=1}^r \int\limits_{\Lambda_\rho} V_\rho (\la) 
x_t (\la) \,d\la-m_t^x\rk; %\label{e2.19-s2}
\end{multline*}

\vspace*{-6pt}

\noindent
\begin{equation*}
%\left.
\begin{array}{rl}
\varphi_{0t}^{(1)} \left(m_t^x, K_t^x\right) &= \mathrm{M}_N \lk \varphi \left(X_t\right)\rk;\\[6pt]
g_{t}^{(1)} \left(m_t^x, K_t^x\right)& = K_t^{yx}\left(K_t^x\right)^{-1}.
\end{array}
%\right\}
%\label{e2.20-s2}
\end{equation*}
\textit{Здесь $\mathrm{M}_N$~--- символ математического ожидания для нормального распределения.}

\smallskip

\noindent
\textbf{Теорема~2.5.}\ \textit{Оптимальная среднеквадратичная линеаризация посредством МК ИКП (теорема~$2.1$) 
второго рода определяется следующими формулами}:
  \begin{multline*}
  \varphi_t (X_t) \approx m_t^{(2)y} \left(X_t\right) = \varphi_{0t}^{(2)} \left(m_t^x, K_{\rho t}^x\right) +{}\\
{}+\sss_{\rho=1}^r \int\limits_{\Lambda_\rho} \!g_{pt}^{(2)} \! \left(m_t^x, K_{\rho t}^x\right) 
V_\rho (\la) x_{\rho t} \,d\la - g_t^{(2)}  \left(m_t^x, K_t^x\right) m_t^x;\hspace*{-7.8623pt}\\[-25pt] %\label{e2.21-s2}
\end{multline*}

\pagebreak

\noindent
  \begin{align*}
  \varphi_{0t}^{(2)}  \left(m_t^x, K_{\rho t}^x\right) &=\mathrm{M}_N \lk \varphi \left(X_t\right)\rk;\\
g_{0t}^{(2)}  \left(m_t^x, K_{\rho t}^x\right) &=K_{\rho t}^{yx} \left(K_{\rho t}^x\right)^{- 1}.
%\label{e2.22-s2}
\end{align*}


\section{Стохастические системы, не~разрешенные относительно~производных}


В качестве исходной СтС НРОП рассмотрим детерминированную  векторную сис\-те\-му 
уравнений
   \begin{equation}
   \Phi = \Phi \left( t, X_t, \bar X_t, Y_t, \Theta_t, U_t\right) =0\,.
    \label{e3.1-s2}
    \end{equation}
Здесь $X_t$~--- вектор состояния; $\bar X_t\hm= [ \dot X_t^{\mathrm{T}} \cdots 
(X^{(l)})^{\mathrm{T}}]^{\mathrm{T}}$~--- расширенный вектор со\-сто\-яния, со\-сто\-ящий из $l$-го порядка 
производных по времени; $Y_t$~--- вектор наблюдений; $\Theta_t$~--- вектор 
случайных па\-ра\-мет\-ров, опи\-сы\-ва\-емый МК ИКП (см.\ разд.~2); $\Phi$~--- нелинейная 
функция переменных  $\bar{\bar X}_t \hm= [ X_t^{\mathrm{T}} \bar X_t^{\mathrm{T}} U_t^{\mathrm{T}}]^{\mathrm{T}}$ 
и~$\Theta_t$, до\-пус\-ка\-ющая при фиксированном векторе случайных па\-ра\-мет\-ров~$\Theta_t$ 
среднеквадратичную регрессионную линеаризацию по~$\bar X_t$  и~$U_t$ 
вида
      \begin{equation*}
    \Phi\approx \Phi_0 + \sss_{j=1} k_{\bar X,j}^\Phi \bar X_t^j + k_U^\Phi U_t^0;
%\label{e3.2-s2}
\end{equation*}
$U_t$~--- вектор возмущений, связанный с~гауссовским белым шумом~$V_0$ линейным 
уравнением формирующего фильтра:
     \begin{equation}
    \dot U_t = a_t^U U_t +  a_{0t}^U + b_t^U V_0,
    \label{e3.3-s2}
    \end{equation}
где $\mathrm{M} V_0 =0$; $ \mathrm{M} \lk \nu_0 (t,\Theta_t) \nu_0(t,\Theta_t)^{\mathrm{T}}\rk\hm = \nu_0 
\delta (t-\tau)$. Тогда при фиксированном~$\Theta$ и~при условиях $\mathrm{det}\ 
k_{\bar X, l}^\Phi\hm \ne 0$ и~$ \mathrm{det}\ K_U^\Phi \hm\ne 0$ дифференциальная СтС НРОП~(\ref{e3.1-s2}) 
приводится к~дифференциальной СтС следующего вида (\textbf{теорема~3.1}):

\noindent
      \begin{multline*}
      \dot{\bar X}_{1t} = \bar X_{2t} \tr {\dot{\bar X}}_{(l-1)t} =\bar 
X_{lt},\\
    \dot{\bar X}_{lt}= -\left(k_{\bar X,l}^\Phi\right)^{-1} \bar X_t^{(l)}-
\left(k_{\bar X, l}^\Phi\right)^{-1}\left(k_{U}^\Phi\right)^{-1} U_t
%\label{e3.4-s2}
\end{multline*}
и~(\ref{e3.3-s2}). Мат\-ри\-цы коэффициентов~$k_{\bar X, l}^\Phi$ и~$k_{U}^\Phi$ неявно зависят 
от первых двух вероятностных моментов~$\bar X_t$.



Пусть объектовая СтС НРОП допускает приведение к~дифференциальной, измерительная 
сис\-те\-ма вполне наблюдаема, наблюдения влияют на объект, а~уравнение наблюдения 
разрешено относительно~$Y_t$. Тогда в~качестве исходных приведенных 
дифференциальных уравнений (объект и~измерительная сис\-те\-ма) можно принять 
следующую:

\noindent
    \begin{multline}
    \dot X_t = A^{\mathrm{п}}\left(X_t, Y_t, \Theta_t, t\right) ={}\\
    {}= a^{\mathrm{п}}\left( X_t, Y_t, \Theta_t, t\right) + b^{\mathrm{п}}\left( X_t, Y_t, \Theta_t, t\right) V_0 
\left(\Theta_t\right);
\label{e3.5-s2}
\end{multline}

%\vspace*{-12pt}

\noindent
\begin{multline}
    Z_t = \dot Y_t= B\left(X_t, Y_t, \Theta_t, t\right) ={}\\
    {}= a_1 \left( X_t, Y_t, \Theta_t, t\right) +
b_1\left( X_t, Y_t, \Theta_t,t\right) V_0 \left(\Theta_t\right).
\label{e3.6-s2}
\end{multline}

\vspace*{-4pt}

\noindent
Здесь $a^{\mathrm{п}}$, $a_1$, $b^{\mathrm{п}}$ и~$b_1$~--- известные век\-тор\-но-мат\-рич\-ные функции; 
$V_0$~--- векторный нормальный (гауссовский) белый шум 
интенсивности $\nu_0 \hm= \nu_0 (\Theta_t)$.


\section{Основные результаты}


\noindent
\textbf{Теорема~4.1.}\ \textit{Пусть составной векторный СтП $Z_t \hm= [ Y_t^{\mathrm{T}} X_t^{\mathrm{T}} \hat X_t^{\mathrm{T}}]^{\mathrm{T}}$  при 
фиксированном векторе случайных параметров~$\Theta_t$ определяется уравнениями}~(\ref{e3.5-s2}), (\ref{e3.6-s2}) 
\textit{и~уравнением УОФ}~[8]:

\vspace*{-6pt}

\noindent
     \begin{multline}
     \dot{\hat X}_t = C\left(Y_t, \hat X_t, \Theta_t, t\right)=\alp_t \xi \left(Y_t, \hat X_t, \Theta_t, t\right) +{}\\
     {}+\beta_t \eta \left(Y_t, \hat X_t, \Theta_t, t\right) \dot Y_t + 
\gamma_t,
\label{e4.1-s2}
\end{multline}
\textit{где}


\vspace*{-6pt}

\noindent
   \begin{multline*}
    \alp_t m_1 + \beta_t m_2 + \gamma_t = m_0,\\ 
    m_0 = M a^{\mathrm{п}},\enskip m_1 = \, \xi, \enskip m_2 = \mathrm{M} \eta a_1;
%\label{e4.2-s2}
\end{multline*}

\vspace*{-12pt}

\noindent
   \begin{multline*}
    \beta_t =\kappa_{02} \kappa_{22}^{-1} \left(\mathrm{det}\ \kappa_{22}\ne  0\right),\\
     \kappa_{02} = \mathrm{M} \left(X_t -\hat X_t\right) a_1^{\mathrm{T}}\eta^{\mathrm{T}} + \mathrm{M} b^{\mathrm{п}} 
\nu_0^{\mathrm{п}}b_1^{\mathrm{T}} \eta^{\mathrm{T}},\\
\kappa_{22} = \mathrm{M} \eta b_1\nu_0^{\mathrm{п}}b_1^{\mathrm{T}} \eta^{\mathrm{T}};
%\label{e4.3-s2}
\end{multline*}

\vspace*{-12pt}

\noindent
\begin{multline*}
    \alp_t \kappa_{11} + \mathrm{M} \lk \left(\hat X_t - X_t\right) \left(\xi^{\mathrm{T}} \alp_t^{\mathrm{T}} 
+\gamma_t^{\mathrm{T}}\right)\rk \fr{\partial  \xi^{\mathrm{T}}}{\partial  \hat X_t} ={}\\
{}=\kappa_{01}^\prime -\beta_t \kappa_{21}^\prime,\\
\kappa_{21} = \mathrm{M} \left(\eta a_1 - m_2\right) \xi^{\mathrm{T}},
\end{multline*}

\vspace*{-12pt}

\noindent
\begin{multline*}
\kappa_{01}' =\kappa_{01} + \mathrm{M} \left( X_t - \hat X_t\right) \fr{\partial  \xi^{\mathrm{T}}}{\partial  
t} +{}\\
{}+\mathrm{M} \lk \left(X_t - \hat X_t\right) a_1^{\mathrm{T}}  + b^{\mathrm{п}}\nu_0^{\mathrm{п}} 
b_1 - \beta_t \eta b_1 \nu_0^{\mathrm{п}} b_1^{\mathrm{T}}\rk \times{}\\
{}\times \left( \fr{\partial }{\partial  y} + \eta^{\mathrm{T}} \beta_t^{\mathrm{T}} \fr{\partial }{\partial  \hat{X}_t}\right)\xi^{\mathrm{T}} +
\fr{1}{2} \mathrm{M} \left(X_t - \hat X^t\right)\times{}
\\
{}\times\biggl\{ \mathrm{tr} \, \lk b_1 \nu_0^{\mathrm{п}} b_1^{\mathrm{T}} 
\left(\fr{\partial }{\partial  y} + 2 \eta^{\mathrm{T}} \beta_t^{\mathrm{T}} \fr{\partial }{\partial  \hat X_t}\right) 
\fr{\partial^{\mathrm{T}}}{\partial  y}\rk+{}\\
{}+ \mathrm{tr}\, \lk \beta_t \eta b_1 \nu_0^{\mathrm{п}} b_1^{\mathrm{T}} \eta^{\mathrm{T}} 
\beta_t^{\mathrm{T}} \fr{\partial }{\partial  \hat X_t}\,\fr{\partial^{\mathrm{T}}}{\partial  \hat X^t}\rk\biggr\} 
\xi^{\mathrm{T}}, \\
 \kappa_{01} =\mathrm{M} (a^{\mathrm{п}}-m_0)\xi^{\mathrm{T}}.
%\label{e4.4-s2}
\end{multline*}
\textit{Тогда совместное распределение составного вектора~$Z_t$  определяется уравнением 
Пугачёва для одномерной характеристической функции относительно} $\la \hm=[ 
\la_1^{\mathrm{T}} \la_2^{\mathrm{T}} \la_3^{\mathrm{T}}]^{\mathrm{T}}$:

\vspace*{-6pt}

\noindent
    \begin{multline*}
    \fr{\partial  g_1 (\la_1, \la_2,\la_3 )}{\partial  t} = \mathrm{M} \biggl [ i \la_1^{\mathrm{T}} a_1 
+ i\la_2^{\mathrm{T}} a^{\mathrm{п}} +{}\\
{}+ i \la_3^{\mathrm{T}} \left(\alp_3 \xi + \beta_t \eta a_1 
+\gamma_t\right)+{}
\end{multline*}

\noindent
\begin{multline*}
{}+\chi_0 \left(b_1^{\mathrm{T}} \la_1 + b^{\mathrm{пT}} \la_2 + b_1^{\mathrm{T}} \eta^{\mathrm{T}} 
\beta_t^{\mathrm{T}} \la_3;t\right)\biggr]\times{}\\
{}\times \exp \lk i\la_1^{\mathrm{T}} Y_t + i\la_2^{\mathrm{T}} X_t + i \la_3^{\mathrm{T}} 
\hat X_t\rk,
%\label{e4.5-s2}
\end{multline*}

\vspace*{-3pt}

\noindent
\textit{где}
\begin{equation*}
\chi_0 (\mu; t) = -\fr{1}{2} \,\mu^{\mathrm{T}} \nu_0\mu, %\label{e4.6-s2}
    \end{equation*}
\textit{и при начальном условии}
\begin{equation*}
g_1 \left(\la_1,\la_2,\la_3;t_0\right) = g_0 \left(\la_1,\la_2,\la_3\right).
%\label{e4.7-s2}
\end{equation*}
\textit{При этом точность УОФ оценивается по следующей формуле для производной 
ковариационной матрицы ошибки}~$R_t$:

\vspace*{-3pt}

\noindent
\begin{multline*}
\dot R_t = D \left(Y_t, X_t, \hat X_t, \Theta_t, t\right) = {}\\
{}=\mathrm{M} \left[ \left(X_t - \hat X_t\right) a^{\mathrm{п}T}+ a^{\mathrm{п}}\left( X_t^{\mathrm{T}} - 
\hat X_t^{\mathrm{T}}\right) -{}\right.\\
\left.{}- \beta_t \eta b_1 \nu_0^{\mathrm{п}} b_1^{\mathrm{T}} \eta^{\mathrm{T}} \beta_t^{\mathrm{T}} + 
b^{\mathrm{п}}\nu_0^{\mathrm{п}} b^{\mathrm{п}T}
\vphantom{\left(X_t - \hat X_t\right) a^{\mathrm{п}T}}
\right].
%\label{e4.8-s2}
\end{multline*}

\vspace*{-3pt}




Применяя регрессионную линеаризацию к~уравнениям~(\ref{e3.5-s2})--(\ref{e4.1-s2}) посредством 
МК ИКП, согласно тео\-ре\-мам~2.4 и~2.5 для условных вероятностных характеристик при 
фиксированном векторе случайных па\-ра\-мет\-ров~$\Theta_t$ придем к~сле\-ду\-ющим 
уравнениям аналитического моделирования УОФ:
  \begin{equation}
  \hat{\tilde X}_t = \hat{\tilde{\tilde X}}_t +\delta \hat{\tilde 
X}_t;\enskip \tilde R_t =\tilde{\tilde R}_t + \delta \tilde R_t.
\label{e4.9-s2}
\end{equation}
Здесь введены следующие обозначения:
\begin{equation}
\left.
\begin{array}{rl}
\fr{d}{dt}\hat{\tilde{\tilde X}}_t &= C_0 \left(Y_t, \hat{\tilde{\tilde X}}_t, O, t\right);\\[6pt]
\fr{d}{dt}\tilde{\tilde R}_t &= D_0 \left(Y_t, \hat{\tilde{\tilde  X}}_t, O, t\right);
\end{array}
\right\}
\label{e4.10-s2}
\end{equation}

\vspace*{-12pt}

\noindent
\begin{multline}
\fr{d}{dt}\delta\hat{\tilde X}_t = C_{1,\delta X} \left(Y_t,  \hat{\tilde{\tilde X}}_t, O, t\right) \hat{\tilde X}_t+{}\\
  C_{1,\delta R} \left(Y_t,  \hat{\tilde{\tilde X}}_t, O, t\right)\delta R_t +{}\\
{}+C_{1,\Theta} \left(Y_t, \hat{\tilde{\tilde X}}_t, O, t\right)\Theta_t;
\label{e4.11-s2}
\end{multline}

\vspace*{-12pt}

\noindent
\begin{multline}
\fr{d}{dt}\delta \tilde R_t = D_{1,\delta X} \left(Y_t, \hat{\tilde{\tilde X}}_t, O, t\right)\delta \hat{\tilde X}_t+ {}\\
{}+
D_{1,\delta R} \left(Y_t, \hat{\tilde{\tilde  X}}_t, O, t\right)\delta R_t+{}\\
{}+ D_{1,\Theta} \left(Y_t, \hat{\tilde{\tilde X}}_t, O, t\right)\Theta_t,
    \label{e4.12-s2}
    \end{multline}
    
    \vspace*{-3pt}

\noindent
где $C_0=C_0(Y_t, \hat{\tilde{\tilde X}}_t, O, t)$, $D_0 \hm= D_0 (Y_t, 
\hat{\tilde{\tilde X}}_t, O, t)$, $C_1 \hm= C_1 (Y_t, \hat{\tilde{\tilde X}}_t, O, 
t)$ и~$D_1 \hm= D_1 (Y_t, \hat{\tilde{\tilde X}}_t, O, t)$ пред\-став\-ля\-ют собой 
коэффициенты среднеквадратичной линеаризации соответствующих функций посредством 
МК ИКП.

\smallskip

\noindent
\textbf{Теорема~4.2.}\ \textit{Пусть в~условиях тео\-ре\-мы~$4.1$ функции~$C$ и~$D$ допускают среднеквадратичную 
линеаризацию посредством МК ИКП, тогда в~основе алгоритмов аналитического 
моделирования будут лежать уравнения}~(\ref{e4.9-s2})--(\ref{e4.12-s2}).


\section{Применение к~линейным~системам с~мультипликативными шумами }


Для приведенных дифференциальных уравнений линейной сис\-те\-мы~(\ref{e3.5-s2}), (\ref{e3.6-s2})  
с~мультипликативными шумами при фиксированном векторе~$\Theta_t$
   \begin{multline}
   \dot X_t = A^{\mathrm{п}}\left(Y_t, X_t, \Theta_t,t\right) = a^{\mathrm{п}}Y_t + 
a_1^{\mathrm{п}}X_t + a_0^{\mathrm{п}}+ {}\\
{}+\lk c_{10}^{\mathrm{п}}+\sss_{r=1}^{n_y} c_{1r}^{\mathrm{п}} Y_r + \sss_{r=1}^{n_x} 
c_{1,n_y+r}^{\mathrm{п}} X_r\rk V_0;
\label{e5.1-s2}
\end{multline}

\vspace*{-12pt}

\noindent
\begin{multline*}
\dot Y_t = B\left(Y_t, X_t, \Theta_t,t\right) = bY_t + b_1 X_t + b_0 + {}\\
{}+\left( c_{20}  + \sss_{r=1}^{n_y} c_{2r} Y_r + \sss_{r=1}^{n_x} c_{2,n_y+r} 
X_r\right)V_0
%\label{e5.2-s2}
\end{multline*}
согласно теореме~4.1 для $\xi\hm=[ Y_t^{\mathrm{T}} \hat X_t^{\mathrm{T}}]^{\mathrm{T}}$ и~$\eta\hm= I_{n_y}$ 
уравнения УОФ имеют вид~(\ref{e4.1-s2}), где
  \begin{equation*}
    \alp_t = \lk \alp_1\,  \alp_2\rk =\lk  a^{\mathrm{п}}- \beta_t b\enskip  
a_1^{\mathrm{п}}- \beta_t b_1 \rk;
%\label{e5.3-s2}
\end{equation*}

\vspace*{-12pt}

\noindent
 \begin{multline*}
 \beta_t = \Biggl\{ R_t b_1^{\mathrm{T}} + \left( c_{10}^\Pi + \sss_{r=1}^{n_y+n_x} 
c_{1r}^{\mathrm{п}}m_r\right) \nu_0^{\mathrm{п}}\times{}\\
{}\times \left(c_{20}^{\mathrm{T}} + 
\sss_{r=1}^{n_y+n_x} c_{2r}^{\mathrm{T}} m_r\right)+\!
\ss2\limits_{r,s=1}^{n_y+n_x} c_{1r}^{\mathrm{п}}\nu_0^{\mathrm{п}} 
c_{2s}^{\mathrm{T}} K_{rs}\Biggr\}\kappa_{11}^{-1};
%\label{e5.4-s2}
\end{multline*}
\begin{equation}
\gamma_t = a_0^{\mathrm{п}}-\beta_t b_0;
\label{e5.5-s2}
\end{equation}
\begin{equation*}
\dot m_t =a^Q m_t + a_0^Q;
%\label{e5.6-s2}
\end{equation*}

\vspace*{-12pt}

\noindent
\begin{multline*}
\dot K_t = a^Q K_t + K_t a^{QT} + c^Q_0 \nu_0^{\mathrm{п}}c_0^{QT} + {}\\
{}+
\sss_{r=1}^{n_y+n_x} \left( c_0^Q \nu_0^{\mathrm{п}} c_r^{QT} + c_r^Q 
\nu_0^{\mathrm{п}}c_0^{QT}\right) m_r +{}\\
{}+ \ss2\limits_{r,s=1}^{n_y+n_x} c_r^Q \nu_0^{\mathrm{п}} c_s^{QT} \left(m_r m_s +K_{rs}\right);
%\label{e5.7-s2}
\end{multline*}

\vspace*{-12pt}

\noindent
\begin{alignat*}{2}
a^Q&=\lk \begin{array}{cc}
    b& b_1\\
    a^{\mathrm{п}}&a_1^{\mathrm{п}}
    \end{array}\rk;&\quad
       a_0^Q&= \lk \begin{array}{c}
        b_0\\
        a_0^{\mathrm{п}}
        \end{array}\rk;
        \\
                  c_0^Q&=  \lk \begin{array}{cc}
            c_{20}\\
            c_{10}^{\mathrm{п}}
            \end{array}\rk;&\quad
               c_r^Q &= \lk \begin{array}{cc}
                c_{2r}\\
                c_{1r}^{\mathrm{п}}
                \end{array}\rk \enskip \left(r=\overline{0, n_y+n_x}\right).
%                \end{array}
%                \right\}
%                \label{e5.8-s2}
                \end{alignat*}
Здесь $m_t$ и~$K_t \hm= \lk K_{lh}\rk$~--- соответственно математическое ожидание 
и~ковариационная матрица составного вектора $Q\hm= [ \hat X_1 \cdots \hat X_{n_x}\,  
Y_1\cdots Y_{n_y}]^{\mathrm{T}}$.
Тогда совместное распределение составного вектора определяется уравнением 
Пугачёва для одномерной характеристической функции:
  \begin{multline*}
  \fr{\partial  g_1}{\partial  t} = \left(\la_1^{\mathrm{T}} b + \la_2^{\mathrm{T}} a^{\mathrm{п}} + 
\la_3^{\mathrm{T}} a^{\mathrm{п}}\right) \fr{\partial  g_1}{\partial  \la_1}+{}\\
{}+
    \left(\la_1^{\mathrm{T}} b_1 + \la_2^{\mathrm{T}} a_1 + \la_3^{\mathrm{T}}\beta_t b_1\right) \fr{\partial  
g_1}{\partial  \la_2}+{}\\
{}+ \la_3^{\mathrm{T}} (a_1^{\mathrm{п}}+\beta_t a^{\mathrm{п}}) \fr{\partial  
g_1}{\partial  \la_3} + i\left( \la_1^{\mathrm{T}} b_0 + \la_2^{\mathrm{T}} a_0^{\mathrm{п}}+ \la_3^{\mathrm{T}} 
a_0^{\mathrm{п}}\right) g_1 +{}\\
{}+\mathrm{M} \chi_0 \Biggl[ \left(c_{20}^{\mathrm{T}} + \sss_{r=1}^{n_y} c_{2r}^{\mathrm{T}} Y_r 
+\sss_{r=1}^{n_x} c_{2, n_x+r}^{\mathrm{T}} X_r\right) \la_1+{}\\
{}+\left( c_{20}^{\mathrm{T}} + \sss_{r=1}^{n_y} c_{1r}^{\mathrm{пT}} Y_r + 
\sss_{r=1}^{n_x} c_{1,n_x+r}^{\mathrm{пT}} X_r \right)\times{}\\
{}\times \left(\la_2+\beta_t^{\mathrm{T}} \la_3\right);t\Biggr] \exp \lf i \la_1^{\mathrm{T}} Y_t + i\la_2^{\mathrm{T}} 
X_t +i \la_3^{\mathrm{T}} \hat X_t\rf.
%\label{e5.9-s2}
\end{multline*}
%
При этом для оценки точности УОФ используется следующее обобщенное уравнение 
Риккати:

\noindent
   \begin{multline*}
    \dot R_t= D(R_t, \Theta_t) = a_1^{\mathrm{п}}R_t + R_t a_1^{\mathrm{пT}}-\left[ 
    \vphantom{\sss_{r=1}^{n_y+n_x}}
    R_t b_1^{\mathrm{T}} +{}\right.\\
   {}+\left(c_{10}^{\mathrm{п}}+\sss_{r=1}^{n_y+n_x} c_{1r}^{\mathrm{п}}m_r\right) \nu_0^{\mathrm{п}} 
\left(c_{20}^{\mathrm{T}}+\sss_{r=1}^{n_y+n_x} c_{2r}^{\mathrm{T}} m_r\right)+ {}\\
\left.{}+\ss2_{r,s=1}^{n_y+n_x} 
c_{1r}^{\mathrm{п}} \nu_0^{\mathrm{п}}c_{2,s}^{\mathrm{T}} K_{rs}\right]\kappa_{11}^{-1} \left[ 
\vphantom{\sss_{r=1}^{n_y+n_x} }
b_1 R_t +{}\right.\\
{}+\left( c_{20} +\sss_{r=1}^{n_y+n_x} c_{2r} m_r\right) 
\nu_0^{\mathrm{п}}
\left( 
\vphantom{\sss_{r=1}^{n_y+n_x} }
c_{10}^{\mathrm{пT}} +{}\right.\\
\left.\left.{}+\sss_{r=1}^{n_y+n_x} 
c_{1r}^{\mathrm{пT}} m_r \right) +\ss2_{r,s=1}^{n_y+n_x}  c_{2r} 
\nu_0^{\mathrm{п}}c_{1s}^{\mathrm{пT}} K_{rs} \right]+{}\\
{}+\left(c_{10}^{\mathrm{п}}+\sss_{r=1}^{n_y+n_x} c_{1r}^{\mathrm{п}} 
m_r\right) \nu_0^{\mathrm{п}} \left(
\vphantom{\sss_{r=1}^{n_y+n_x} }
c_{10}^{\mathrm{пT}} + {}\right.\\
\left.{}+
\sss_{r=1}^{n_y+n_x} c_{1r}^{\mathrm{пT}} m_r\right) +\ss2_{r,s=1}^{n_y+n_x} 
c_{1r}^{\mathrm{п}}\nu_0^{\mathrm{п}} c_{1s}^{\mathrm{пT}}  K_{rs},
%\label{e5.10-s2}
\end{multline*}
где
    \begin{multline*}
    \kappa_{11} ={}\\
    {}=\left(c_{20} + \sss_{r=1}^{n_y+n_x} c_{2r} m_r\right) 
\nu_0^{\mathrm{п}}\left( c_{20}^{\mathrm{T}} + \sss_{r=1}^{n_y+n_x} c_{2r}^{\mathrm{T}} m_r \right)+ {}\\
{}+
\sss_{r=1}^{n_y+n_x} c_{2r} \nu_0^{\mathrm{п}} c_{2s}^{\mathrm{T}} K_{rs}.
%\label{e5.11-s2}
\end{multline*}

По теореме~4.2 оценка точности УОФ при случайных па\-ра\-мет\-рах проводится в~(\ref{e5.1-s2}), 
(\ref{e5.5-s2}) путем линеаризации коэффициентов $ a^{\mathrm{п}}\hm=a^{\mathrm{п}}(\Theta_t, t)$,
$ a_1^{\mathrm{п}}\hm=a_1^{\mathrm{п}}(\Theta_t, t)$,  $ a_0^{\mathrm{п}}\hm=a_0^{\mathrm{п}}(\Theta_t, t)$, 
$ c_{10}^{\mathrm{п}}\hm=c_{10}^{\mathrm{п}}(\Theta_t, t)$, $ c_{12}^{\mathrm{п}}\hm=c_{12}^{\mathrm{п}}(\Theta_t, t)$ 
и~$ \nu_0^{\mathrm{п}}\hm=\nu_0^{\mathrm{п}}(\Theta_t, t)$ посредством МК ИКП 
с~по\-сле\-ду\-ющим решением нелинейных уравнений~(\ref{e4.10-s2}) для средних значений и~линейных 
уравнений~(\ref{e4.11-s2}), (\ref{e4.12-s2}) для отклонений.


\section{Пример}

Рассмотрим систему~[5] при $\gamma\hm=\gamma_0 \Theta_t$ и~$\nu_1\hm =\nu_{10}$, 
$\nu_2\hm = \nu_{20}$:
   \begin{gather}
   \Phi =\Phi_1 (\dot X_{1t}) + \gamma_0 \Theta_t X_{1t} + X_{2t} 
=0\,;\label{e6.1-s2}\\
\dot X_{2t} = a^U X_{2t} + V_1;\enskip Z_t=\dot Y_t = b_1 X_{1t} + 
V_2.\label{e6.2-s2}
\end{gather}
Здесь $[X_{1t} X_{2t}]^{\mathrm{T}}=X_t$~--- вектор состояния; $V_1$ и~$V_2$~--- независимые 
скалярные белые шумы с~интенсивностями~$\nu_1$ и~$\nu_2$; $a^U$ и~$b_1$~--- 
известные количества; $\Phi_1 \hm=\Phi_1 (\dot X_{1t})$~--- функция, допускающая 
регрессионную линеаризацию
   \begin{equation}
   \Phi_1 \approx \Phi_{10} + k_1 \dot X_{1t}^0,\label{e6.3-s2}
   \end{equation}
где $\Phi_{10} \hm=\Phi_{10} (m_{1t}^{\dot X}, D_{1t}^{\dot X})$; $k_1\hm= 
k_1(m_{1t}^{\dot X}, D_{1t}^{\dot X})$. При  $k_1\hm\ne 0$ уравнение~(\ref{e6.1-s2}) с~учетом~(\ref{e6.3-s2}) приводится к~виду:
   \begin{equation}
   \dot X_{1t} = -(\Phi_{10} + \gamma_0 \Theta_t  X_{1t} - X_{2t}) k_1^{-1}. 
\label{e6.4-s2}
\end{equation}
Обозначая
   \begin{alignat*}{2}
    a_{0t} &= \lk \begin{array}{c}
    -\Phi_{10} k_1^{-1}\\
    0
    \end{array}\rk; &\enskip
    a_{1t} &= a_1 (\Theta_t) = \lk \begin{array}{cc}
    \gamma_0 \Theta_t&k_1^{-1}\\
    b_1&0
    \end{array}\rk;
\\
b_{1t} &= [b_1\, 0]; &\enskip \tilde R &= \left[\begin{array}{cc}
    R_{11}& R_{12}\\
    R_{12}& R_{22}
    \end{array}\right],
%    \label{e6.5-s2}
    \end{alignat*}
запишем уравнения~(\ref{e6.2-s2}), (\ref{e6.4-s2}) в~форме
 \begin{align}
 \dot X_t &= a_{1t}(\Theta_t) X_t + a_{0t} + V_1;
 \label{e6.6-s2}
\\
Z_t &=\dot Y_t = b_{1t} X_t + V_2.
\label{e6.7-s2}
\end{align}
Для~(\ref{e6.6-s2}), (\ref{e6.7-s2}) уравнения НСОФ~[5] для условных характеристик УОФ имеют 
сле\-ду\-ющий вид:
    \begin{equation}
    \dot{\hat{\tilde X}}_t = a_{1t} (\Theta_t){\hat{\tilde X}}_t + a_{0t} 
+\tilde \beta_t (Z_t - b_{1t} {\hat{\tilde X}}_t);
\label{e6.8-s2}
\end{equation}
  \begin{equation}
  \tilde \beta_t =\tilde R_t b_{1t}^{\mathrm{T}} \nu_{20t}^{-1};
  \label{e6.9-s2}
  \end{equation}
  
\vspace*{-12pt}
  
\noindent
\begin{multline}
{\dot{\tilde R}}_t = a_{1t}(\Theta_t) \tilde R_t + \tilde R_t 
a_{1t}^{\mathrm{T}}(\Theta_t)-{}\\
{}-\tilde R_t b_{1t}^{\mathrm{T}} \nu_{20t}^{-1} b_{1t} \tilde R_t +\nu_{10}.
\label{e6.10-s2}
\end{multline}
Здесь
    $$
    \Theta_t = m_t^\Theta + \int\limits_\Lambda V(\la) x^\Theta (\la,t) \,d\la;
    $$
    $$
    K_t^\Theta (t_1, t_2) =\int\limits_\Lambda G(\la) x^\Theta (\la,t_1) x^\Theta 
(\la,t_2) \,d\la;
$$
    
\vspace*{-12pt}
    
    \noindent
    \begin{multline*}
    x^\Theta(\la,t) = q_1 (t) q_2^{-1} (\la) {\bf 1} (t-\la),\\
     q_1 (t) = 
e^{-\alp t},\enskip q_2(t) = q_1 (t) \int\limits_{t_0}^t q_1^{-2} (\la)\, d\la;
\end{multline*}

\vspace*{-4pt}

\noindent
    $$
    K^\Theta (t_1, t_2) =\begin{cases}
    q_1(t_2) q_2 (t_1) &  \mbox{при}\  t_1<t_2;\\
    q_1(t_1) q_2 (t_2) &  \mbox{при}\  t_1>t_2.\\
\end{cases}
   $$


Выполняя статистическую линеаризацию функции~$\Theta_t X_{1t}$ в~(\ref{e6.4-s2}) по 
известной формуле~[8]:
\begin{equation*}
\Theta_t X_{1t} \approx m_t^\Theta m_{1t} + K_t^{\Theta X_1} 
+m_{1t}\Theta_t^0 + m_t^\Theta X_{1t}^0,
%\label{e6.14-s2}
\end{equation*}
согласно~(\ref{e6.8-s2})--(\ref{e6.10-s2}) придем к~соответствующим уравнениям для безусловных 
характеристических функций~$\hat{\tilde{\tilde X}}_t$ и~$\hat{\tilde{\tilde R}}_t$, определяющих точ\-ность УОФ.

\vspace*{-4pt}


\section{Заключение}

\vspace*{-2pt}

Для наблюдаемых дифференциальных гауссовских СтС НРОП со случайными параметрами в~виде МК ИКП, приводимых к~дифференциальным уравнениям, разработано 
методическое обеспечение в~виде алгоритмов услов\-но-оп\-ти\-маль\-ной фильт\-ра\-ции 
и~моделирования их точ\-ности. Рассмотрена СтС НРОП, приводимая к~линейной 
дифференциальной СтС с~мультипликативными шумами.
Полученные результаты обобщают~[5] для НСОФ.
Результаты могут быть использованы для наблюда\-емых дифференциальных 
негауссовских СтС НРОП, приводимых к~гауссовским.

Дальнейшее развитие методического обеспечения услов\-но-оп\-ти\-маль\-но\-го оценивания 
(фильтрации, экстраполяции и~интерполяции) для СтС\linebreak НРОП связано 
с~использованием методов па\-ра\-мет\-ри\-за\-ции апостериорных распределений, нелинейных 
регрессионных моделей, а~также комбинированного вероятностного и~статистического 
\mbox{моделирования}.

Важным направлением дальнейших исследований представляется разработка методов 
услов\-но-оп\-ти\-маль\-ной фильт\-ра\-ции для явных и~\mbox{неявных} стохастических функ\-ци\-о\-наль\-но-диф\-фе\-рен\-ци\-аль\-ных включений~[9--14].

{\small\frenchspacing
 {\baselineskip=10.6pt
 %\addcontentsline{toc}{section}{References}
 \begin{thebibliography}{99}   

\vspace*{-2pt}

%1
\bibitem{1-s2}
\Au{Синицын И.\,Н.}
Аналитическое моделирование и~оценивание нестационарных нормальных процессов 
в~стохастических сис\-те\-мах, не разрешенных относительно производных~// Сис\-те\-мы 
и~средства информатики, 2022. Т.~32.  №\,2. С.~58--71. doi: 10.14357/ 08696527220206. EDN: YMGERJ.

%2
\bibitem{2-s2}
\Au{Синицын И.\,Н.}
Аналитическое моделирование и~фильт\-ра\-ция нормальных процессов 
в~интегродифференциальных стохастических сис\-те\-мах, не разрешенных относительно 
производных~// Сис\-те\-мы и~средства информатики, 2021. Т.~31. №\,1. С.~37--56.  doi: 10.14357/08696527210104. EDN: PLYOSF.

%3
\bibitem{3-s2}
\Au{Sinitsyn I.\,N.}
Analytical modeling and estimation of normal processes defined by stochastic differential
equations with unsolved derivatives~// J.~Mathematics Statistics Research, 2021. 
Vol.~3. Iss.~1. Art.~139. 7~p. doi: 10.36266/\linebreak JMSR/139. 

%4
\bibitem{4-s2}
\Au{Синицын И.\,Н.}
Совместная фильт\-ра\-ция и~распознавание нормальных процессов в~стохастических 
сис\-те\-мах, не разрешенных относительно производных~// Информатика и~её 
применения, 2022. Т.~16. Вып.~2. С.~85--93. doi: 10.14357/19922264220211. EDN: SMJCBB.

%5
\bibitem{5-s2}
\Au{Синицын И.\,Н.}
Субоптимальная фильт\-ра\-ция в~стохастических сис\-те\-мах, не разрешенных относительно 
производных, со случайными параметрами~// Информатика и~её применения,  2024. Т.~18. Вып.~1. С.~2--10.
doi: 10.14357/19922264240101. EDN: KUWMKJ.

%6
\bibitem{6-s2}
\Au{Синицын И.\,Н.}
Канонические представления случайных функций. Тео\-рия и~применения.~--- М.: ТОРУС ПРЕСС, 2023. 816~с.



%8
\bibitem{8-s2}
\Au{Пугачёв В.\,С.}
Теория вероятностей и~математическая статистика.~--- 2-е изд.~--- 
М.: Физ\-мат\-лит, 2002. 496~с.

%7
\bibitem{7-s2}
\Au{Синицын И.\,Н.}
Фильтры Калмана и~Пугачёва.~--- 2-е изд.~-- М.: Логос, 2007. 
776~с.

%9
\bibitem{9-s2}
\Au{Толстоногов А.\,А., Финогенко~И.\,А.}
 О~функ\-ци\-о\-наль\-но-диф\-фе\-рен\-ци\-аль\-ных включениях в~банаховых пространствах 
 с~невыпуклой правой частью~// Докл. Акад.\ наук СССР, 1980. Т.~254. №\,1. С.~45--49.

%10
\bibitem{10-s2}
\Au{Финогенко И.\,А.}
 К вопросу о решениях функ\-ци\-о\-наль\-но-диф\-фе\-рен\-ци\-аль\-ных  включений~// Прикладная 
математика и~пакеты при\-клад\-ных программ.~--- Иркутск: СЭИСО АН СССР, 1980. С.~95--107.

%11
\bibitem{11-s2}
\Au{Финогенко И.\,А.}
 Свойства множества решений функ\-ци\-о\-наль\-но-диф\-фе\-рен\-ци\-аль\-ных включений~// Краевые 
задачи.~--- Пермь: ППИ, 1981. С.~145--149.



%13
\bibitem{13-s2}
\Au{Колмановский В.\,Б., Носов~В.\,Р.}
 Устойчивость и~периодические режимы ре\-гу\-ли\-ру\-емых сис\-тем с~последствием.~--- М.: 
Наука, 1981. 448~с.

%12
\bibitem{12-s2}
\Au{Финогенко И.\,А.}
 О неявных функ\-ци\-о\-наль\-но-диф\-фе\-рен\-ци\-аль\-ных уравнениях в~банаховом пространстве~// 
Динамика нелинейных сис\-тем.~-- Новосибирск: Наука, 1983. С.~151--164.

%14
\bibitem{14-s2}
\Au{Азбелев Н.\,В., Максимов~В.\,П., Рахматулина~Л.\,Ф.}
Введение в~тео\-рию функ\-ци\-о\-наль\-но-диф\-фе\-рен\-ци\-аль\-ных уравнений.~--- М.: Наука,  
1991. 280~с.
\end{thebibliography}

 }
 }

\end{multicols}

\vspace*{-9pt}

\hfill{\small\textit{Поступила в~редакцию 30.10.23}}

%\vspace*{10pt}

%\pagebreak

\newpage

\vspace*{-28pt}

%\hrule

%\vspace*{2pt}

%\hrule



\def\tit{CONDITIONALLY OPTIMAL FILTERING IN~STOCHASTIC SYSTEMS WITH~RANDOM PARAMETERS AND~UNSOLVED DERIVATIVES}


\def\titkol{Conditionally optimal filtering in~stochastic systems with~random parameters and~unsolved derivatives}


\def\aut{I.\,N.~Sinitsyn}

\def\autkol{I.\,N.~Sinitsyn}

\titel{\tit}{\aut}{\autkol}{\titkol}

\vspace*{-6pt}


\noindent
Federal Research Center ``Computer Science and Control'' of the Russian Academy of Sciences, 44-2~Vavilov Str., Moscow 119333, Russian Federation

%\noindent
%$^{2}$Moscow State Aviation Institute (National Research University), 4~Volokolamskoe Shosse, Moscow 125933, Russian Federation




\def\leftfootline{\small{\textbf{\thepage}
\hfill INFORMATIKA I EE PRIMENENIYA~--- INFORMATICS AND
APPLICATIONS\ \ \ 2024\ \ \ volume~18\ \ \ issue\ 3}
}%
 \def\rightfootline{\small{INFORMATIKA I EE PRIMENENIYA~---
INFORMATICS AND APPLICATIONS\ \ \ 2024\ \ \ volume~18\ \ \ issue\ 3
\hfill \textbf{\thepage}}}

\vspace*{8pt}



\Abste{For observable differential Gaussian stochastic systems (StS) with random parameters in the form of multicomponent integral canonical 
expansions (MC ICE) and StS with unsolved derivatives (USD), methodological support for synthesis of conditionally optimal 
filters is presented. A~survey in the fields of analytical modeling and sub- and conditionally optimal filtering,
 extrapolation, and identification is presented. Necessary information concerning MC ICE is given. Special attention is 
 paid to mean square regressive linearization including MC ICE. The stochastic systems with USD reducible to differential are considered. 
 Basic results in normal conditionally optimal filtering (COF) are presented for StS USD reducible to differential. 
 The theory of COF application to StS USD with multiplicative noises is developed. An illustrative example for scalar 
 StS USD reducible to differential is given. For future COF generalization, ($i$)~methods of moments, quasi-moments, and
 one- and multidimensional densities 
 of orthogonal expansions and ($ii$)~development for stochastic inclusions  are recommended.}


\KWE{regression linearization; stochastic 
system with unsolved derivatives (StS USD); stochastic process;  conditionally optimal filtering (COF)} 

\DOI{10.14357/19922264240303}{XCXLGD} 

%\vspace*{-12pt}


    
     % \Ack

\vspace*{6pt}

%\noindent



  \begin{multicols}{2}

\renewcommand{\bibname}{\protect\rmfamily References}
%\renewcommand{\bibname}{\large\protect\rm References}

{\small\frenchspacing
 {\baselineskip=11.8pt
 \addcontentsline{toc}{section}{References}
 \begin{thebibliography}{99} 
%1
\bibitem{1-s2-1} 
\Aue{Sinitsyn, I.\,N.} 2022. 
Analiticheskoe modelirovanie i~otsenivanie nestatsionarnykh normal'nykh protsessov v~stokhasticheskikh sistemakh, 
ne razreshennykh otnositel'no proizvodnykh [Analytical modeling and estimation of nonstationary normal processors with unsolved derivatives]. 
\textit{Sistemy i~Sredstva Informatiki~--- Systems and Means of Informatics} 32(2):58--71. doi: 10.14357/ 08696527220206. EDN: YMGERJ.

%2
\bibitem{2-s2-1} 
\Aue{Sinitsyn, I.\,N.} 2021. 
Analiticheskoe modelirovanie i~fil'tratsiya normal'nykh protsessov v~integ\-ro\-dif\-fe\-ren\-tsi\-al'\-nykh stokhasticheskikh sistemakh, 
ne razreshennykh otnositel'no proizvodnykh 
[Analytical modeling and filtering for integrodifferential systems with unsolved derivatives]. 
\textit{Sistemy i~Sredstva Informatiki --- Systems and Means of Informatics} 31(1):37--56.
doi: 10.14357/ 08696527210104. EDN: PLYOSF.

%3
\bibitem{3-s2-1} 
\Aue{Sinitsyn, I.\,N.} 2021. 
Analytical modeling and estimation of normal processes defined by stochastic differential equations with unsolved derivatives. 
\textit{J. Mathematics Statistics Research} 3(1):139. 7~p. doi: 10.36266/JMSR/139.

%4
\bibitem{4-s2-1} 
\Aue{Sinitsyn, I.\,N.} 2022. 
Sovmestnaya fil'tratsiya i~ras\-po\-zna\-va\-nie normal'nykh protsessov v~stokhasticheskikh sistemakh, ne razreshennykh otnositel'no proizvodnykh 
[Joint filtration and recognition of normal prosesses in stochastic systems with unsolved derivatives]. 
\textit{Informatika i~ee Primeneniya~--- Inform. Appl.} 16(2):85--93. 
doi: 10.14357/19922264220211. EDN: SMJCBB.

%5
\bibitem{5-s2-1} 
\Aue{Sinitsyn, I.\,N.} 2024. 
Suboptimal'naya fil'tratsiya v~stokhasticheskikh sistemakh, ne razreshennykh otnositel'no proizvodnykh, so sluchaynymi pa\-ra\-met\-ra\-mi 
[Suboptimal filtering in stochastic systems with random parameters and unsolved derivatives].
\textit{Informatika i~ee Primeneniya~--- Inform. Appl.} 18(1):2--10.
doi: 10.14357/ 19922264240101. EDN: KUWMKJ.

%6
\bibitem{6-s2-1} 
\Aue{Sinitsyn, I.\,N.} 2023. 
\textit{Kanonicheskie predstavleniya sluchaynykh funktsiy. Teoriya i~primeneniya} 
[Canonical expansions of random functions. Theory and application]. 
Moscow: TORUS PRESS. 816~p.




%8
\bibitem{8-s2-1} 
\Aue{Pugachev, V.\,S.} 2002. 
\textit{Teoriya veroyatnostey i~ma\-te\-ma\-ti\-che\-skaya statistika} [Probability theory and mathematical statistics]. 2nd ed. 
Moscow: Fizmatlit. 496~p.

%7
\bibitem{7-s2-1} 
\Aue{Sinitsyn, I.\,N.} 2007. 
\textit{Fil'try Kalmana i~Pugacheva} [Kalman and Pugachev filters]. 2nd ed. 
Moscow: Logos. 776~p.

%9
\bibitem{9-s2-1} 
\Aue{Tolstonogov, A.\,A., and I.\,A.~Finogenko.} 1980.
On functional-differential inclusions in a Banach space with a~nonconvex right-hand side. 
\textit{Soviet Mathematics Doklady} 
22:320--324.

%10
\bibitem{10-s2-1} 
\Aue{Finogenko, I.\,A.} 1980.
K voprosu o~resheniyakh funktsional'no-differentsial'nykh  vklyucheniy [On the issue of solutions of functional-differential inclusions].
\textit{Prikladnaya matematika i pakety prikladnykh programm} [Applied mathematics and application software packages]. 
Irkutsk: SEISO AN SSSR. 95--107.

%11
\bibitem{11-s2-1} 
\Aue{Finogenko, I.\,A.} 1981. 
Svoystva mnozhestva resheniy funktsional'no-differentsial'nykh vklyucheniy [Properties of the solution set of functional differential inclusions]. 
\textit{Kraevye zadachi} [Boundary value problem]. Perm: PPI. 145--149.



%13
\bibitem{13-s2-1} 
\Aue{Kolmanovskiy, V.\,B., and V.\,R.~Nosov.} 1981.
\textit{Ustoychivost' i~periodicheskie rezhimy reguliruemykh sistem s~posledstviem} [Stability and periodic modes of regulated systems with consequences]. 
Moscow: Nauka. 448~p.

%12
\bibitem{12-s2-1} 
\Aue{Finogenko, I.\,A.} 1983.
O neyavnykh funktsional'no-differentsial'nykh uravneniyakh v~banakhovom pro\-stran\-st\-ve [On implicit functional differential equations in a~Banach space].
\textit{Dinamika nelineynykh sistem} [Dynamics of nonlinear systems]. 
Novosibirsk: Nauka.\linebreak 151--164.

%14
\bibitem{14-s2-1} 
\Aue{Azbelev, N.\,V., V.\,P.~Maksimov, and L.\,F.~Rakhmatulina.} 1991.
\textit{Vvedenie v~teoriyu funktsional'no-differentsial'nykh uravneniy} [Introduction to the theory of functional differential equations]. 
Moscow: Nauka. 280~p.
 


\end{thebibliography}

 }
 }

\end{multicols}

\vspace*{-6pt}

\hfill{\small\textit{Received October 30, 2023}} 

\vspace*{-18pt}

\Contrl

\vspace*{-3pt}

\noindent 
\textbf{Sinitsyn Igor N.} (b.\ 1940)~--- 
Doctor of Science in technology, professor, Honored scientist of RF, principal scientist, Federal Research Center ``Computer Science and Control''
 of the Russian Academy of Sciences, 44-2~Vavilov Str., Moscow 119333, Russian Federation; \mbox{sinitsin@dol.ru}


\label{end\stat}

\renewcommand{\bibname}{\protect\rm Литература} 