\def\stat{shnurkov}

\def\tit{ЧИСЛЕННО-АНАЛИТИЧЕСКОЕ РЕШЕНИЕ ЗАДАЧИ О~НАСТРОЙКЕ С~ДИСКРЕТНЫМ ВРЕМЕНЕМ 
ДЛЯ~МОДЕЛИ~ИНТЕРВЕНЦИЙ НА~ВАЛЮТНОМ РЫНКЕ}

\def\titkol{Численно-аналитическое решение задачи о~настройке с~дискретным временем 
для модели интервенций} % на валютном рынке}

\def\aut{П.\,В.~Шнурков$^1$, Д.\,А.~Новиков$^2$}

\def\autkol{П.\,В.~Шнурков, Д.\,А.~Новиков}

\titel{\tit}{\aut}{\autkol}{\titkol}

\index{Шнурков П.\,В.}
\index{Новиков Д.\,А.}
\index{Shnurkov P.\,V.}
\index{Novikov D.\,A.}


%{\renewcommand{\thefootnote}{\fnsymbol{footnote}} \footnotetext[1]
%{Работа 
%выполнена при поддержке Программы развития МГУ, проект №\,23-Ш03-03. При анализе 
%данных использовалась инфраструктура Центра коллективного пользования 
%<<Высокопроизводительные вычисления и~большие данные>> 
%(ЦКП <<Информатика>>) ФИЦ ИУ РАН (г.~Москва)}}


\renewcommand{\thefootnote}{\arabic{footnote}}
\footnotetext[1]{Национальный исследовательский университет 
<<Высшая школа экономики>>, pshnurkov@hse.ru}
\footnotetext[2]{Национальный исследовательский университет <<Высшая школа 
экономики>>, even.he@yandex.ru}


%\vspace*{-12pt}




\Abst{Исследуется проблема оптимизации внешних 
воздействий (управлений) на процесс изменения цены так называемой бивалютной 
корзины на валютном рынке Российской Федерации. Тео\-ре\-ти\-че\-ской основой 
используемого подхода послужило решение стохастической задачи о~настройке 
с~дискретным временем. На основе проведенного ранее ста\-ти\-сти\-че\-ско\-го анализа было 
установлено, что стохастический процесс, ха\-рак\-те\-ри\-зу\-ющий эволюцию цены 
бивалютной корзины, при определенных условиях может быть достаточно адекватно 
описан классической однородной цепью Маркова.  Были получены 
статистические оценки вероятностей перехода указанной цепи.
%
Численно определены необходимые вспомогательные вероятностные 
характеристики марковской модели. Для различных заданных стоимостных 
характеристик проведено исследование стационарного стоимостного показателя 
эффективности управления~--- сред\-ней удельной прибыли. Получены конкретные 
численные решения соответствующей задачи оптимального управ\-ле\-ния, которые мож\-но 
интерпретировать как оптимальные внеш\-ние воздействия (интервенции) на 
исследуемый стохастический процесс.}

\KW{стохастические марковские и~полумарковские модели 
управления; задача о~настройке с~дискретным временем; дроб\-но-ли\-ней\-ные 
интегральные функционалы на дискретных вероятностных распределениях; оптимальное 
управ\-ле\-ние в~стохастических экономических сис\-темах}

\DOI{10.14357/19922264240310}{ZXNZDQ}
  
%\vspace*{-6pt}


\vskip 10pt plus 9pt minus 6pt

\thispagestyle{headings}

\begin{multicols}{2}

\label{st\stat}


\section{Введение}

Решение стохастической задачи о~настройке для моделей с~дискретным и~непрерывным 
временем было впервые в~достаточно сжатой форме изложено в~работе~\cite{A1}.
Следует отметить, что теоретическую основу этого решения составляют результаты 
исследования проблемы безусловного экстремума для дроб\-но-ли\-ней\-но\-го интегрального 
функционала, заданного на множестве конечных наборов вероятностных распределений~\cite{A2, A3}.

В работе~\cite{A4} была проанализирована проблема создания математических 
моделей интервенций на товарных и~валютных рынках. В~этой работе был предложен 
и~теоретически разработан подход к~решению проб\-ле\-мы оптимального управ\-ле\-ния 
интервенциями на основе стохастической задачи о~настройке с~дискретным временем. 
В~дальнейшем в~статье~\cite{A5} этот подход был конкретизирован для 
стохастической модели проведения интервенций на валютном рынке Российской 
Федерации.\linebreak В~этой статье был под\-роб\-но описан способ построения соответствующей 
управляемой полумарковской стохастической модели с~дискретным временем. Отметим, 
что в~на\-сто\-ящей работе будут \mbox{использованы} все основные идеи по\-стро\-ения такой 
модели и~сохранены принятые в~ней обозна\-че\-ния.
{\looseness=1

}

В статье~\cite{A6} был проведен достаточно обстоятельный анализ реальных 
процессов, характеризующих цену так называемой бивалютной корзины (дол\-лар--ев\-ро) 
на валютном рынке Российской Федерации в~течение 2010--2013~гг. Для этого 
анализа были использованы методы математической статистики, основными из которых 
стали методы проверки статистических гипотез. В~результате было установлено, что 
при определенных условиях процесс эволюции цены бивалютной корзины можно 
достаточно адекватно описывать классической моделью однородной марковской цепи 
с~дискретным временем и~дискретным множеством состояний. В~указанной работе были 
также получены статистические оценки вероятностей перехода со\-от\-вет\-ст\-ву\-ющих 
марковских цепей на материале статистических данных, собранных в~определенные 
периоды времени.

В настоящей работе статистические выводы и~оценки, полученные в~статье~\cite{A6}, 
используются для выполнения завершающего этапа решения сто\-ха\-сти\-че\-ской 
задачи о~настройке с~дискретным временем. Исходя из известных оценок элементов 
матрицы вероятностей перехода марковской цепи, получены вспомогательные 
вероятностные характеристики модели, а~именно: матрицы вероятностей по\-гло\-ще\-ния 
в~двух граничных состояниях при определенных условиях на начальные со\-сто\-яния. 
Далее на основе некоторых достаточно естественных качественных соображений 
определены стоимостные характеристики модели: доходы при однократном попадании 
процесса в~произвольное фиксированное состояние и~за\-тра\-ты, характеризующие 
управляющее воздействие, т.\,е.\ перевод процесса из граничного (поглощающего) 
со\-сто\-яния во внутреннее. После задания этих вероятностных и~стоимостных 
характеристик появилась воз\-мож\-ность использовать теоретический результат, 
касающийся решения задачи о~настройке с~дискретным временем. Численное решение 
задачи определяется точкой достижения глобального максимума заданной функции 
двух дискретных аргументов, каждый из которых принимает конечное число значений. 
По своему содержанию эта функция пред\-став\-ля\-ет собой стационарное значение 
средней удельной прибыли, определенное на детерминированных стратегиях 
управ\-ле\-ния.

\section{Основные вероятностные характеристики стохастической~модели}

Полумарковская стохастическая модель с~дискретным временем, на основе которой 
ставится и~решается задача о~настройке, по\-дроб\-но описана в~работах~\cite{A1,A4,A5}.

Основными стохастическими объектами, со\-став\-ля\-ющи\-ми эту модель, являются 
следующие:
\begin{enumerate}[(1)]
\item последовательность независимых однородных цепей Маркова с~поглощениями 
$\{ \xi _{k}^{(n)} \}_{k=0}^{\infty}$, $n\hm=0,1,2,\dots$, име\-ющих 
одинаковые матрицы переходных вероятностей, которые описывают эволюцию 
рассматриваемой стохастической системы на периодах времени между внеш\-ни\-ми 
воздействиями (управ\-ле\-ни\-ями);
\item последовательность случайных величин 
$\{\widehat{\xi}_k\}_{k=0}^{\infty}$, образующих так называемый 
основной процесс. Этот процесс конструируется на основе упомянутой выше 
по\-сле\-до\-ва\-тель\-ности по\-гло\-ща\-ющих марковских цепей и~описывает эволюцию 
рас\-смат\-ри\-ва\-емой стохастической сис\-те\-мы на бесконечном интервале времени $\{0, 1, 2,\dots\}$. 
Данный процесс управ\-ля\-ет\-ся в~моменты времени, когда со\-от\-вет\-ст\-ву\-ющие 
поглощающие цепи Маркова достигают своих граничных по\-гло\-ща\-ющих со\-сто\-яний. По 
своим вероятностным свойствам основной процесс пред\-став\-ля\-ет собой полумарковский 
управ\-ля\-емый процесс с~дискретным временем.
\end{enumerate}

В дальнейшем для определения основных вероятностных характеристик, связанных 
с~рассматриваемой моделью, будем использовать обозначения, введенные в~разд.~4 
работы~\cite{A5}.

Предположим сначала, что для поглощающих марковских цепей $\{ \xi_{k}^{(n)} \}_{k=0}^{\infty}$, $n\hm=0,1,2,\dots$, заданы сле\-ду\-ющие мат\-рич\-ные 
вероятностные характеристики:
\begin{description}
\item[\,]
$\textbf{P}_{00}$~--- матрица вероятностей перехода внут\-ри множества допустимых
состояний $\lbrace 2,3,\ldots ,N\rbrace $, имеет раз\-мер\-ность $(N-1)\times (N-1)$;
\item[\,]
$\textbf{P}_{01}$~--- матрица вероятностей перехода из допустимых состояний $\lbrace 2,3,\ldots ,N\rbrace $ в~поглощающие со\-сто\-яния 
$\lbrace 0,1\rbrace $ за один шаг цепи, имеет раз\-мер\-ность $(N-1)\times 2$;
\item[\,]
$\textbf{P}_{10}$~--- матрица вероятностей перехода из по\-гло\-ща\-ющих состояний 
$\lbrace 0,1\rbrace $ в~допустимые со\-сто\-яния $\lbrace 2,3,\ldots,N\rbrace $. 
Данная мат\-ри\-ца нулевая и~имеет раз\-мер\-ность $2\times (N-1)$;
\item[\,]
$\textbf{P}_{11}$~--- матрица вероятностей перехода внутри множества поглощающих
состояний $\lbrace 0,1\rbrace $. Данная матрица единичная, имеет размерность $2\times 2$.
\end{description}

Тогда матрица вероятностей перехода марковской цепи $\{ \xi _{k}^{(n)} 
\}_{k=0}^{\infty}$ с~произвольным номером~$n$ имеет
сле\-ду\-ющую клеточную структуру:
 $$
 \mathbf{P}=\begin{pmatrix}
\textbf{P}_{11} & \textbf{P}_{10} & \\
\textbf{P}_{01} & \textbf{P}_{00} & \\
\end{pmatrix}.
$$

Вероятности перехода основного процесса 
$\{\widehat{\xi}_k\}_{k=0}^{\infty}$ из граничного состояния~$\{0\}$ 
во внут\-рен\-ние допустимые со\-сто\-яния задаются вектором 
$$
\alpha^{(0)}=\left(\alpha_k^{(0)},\ k=\overline{2, N}\right),\quad
\sum\limits_{k=2}^N\alpha_k^{(0)}=1\,,
$$
 вероятности перехода основного процесса 
$\{\widehat{\xi}_k\}_{k=0}^{\infty}$ из граничного со\-сто\-яния~$\{1\}$ 
во внут\-рен\-ние допустимые со\-сто\-яния состояния задаются вектором 
$$
\alpha^{(1)}=\left(\alpha_k^{(1)},\ k=\overline{2,N}\right),\quad
\sum\limits_{k=2}^N\alpha_k^{(1)}=1\,.
$$
 Распределения вероятностей $\alpha^{(0)}$ 
и~$\alpha^{(1)}$ описывают внеш\-ние управ\-ля\-ющие воздействия.

Предположим также, что заданы следующие стоимостные
характеристики модели.

Обозначим через $c_{l}$ доход при однократном пребывании основного процесса 
$\{\widehat{\xi}_k\}_{k=0}^{\infty}$ в~со\-сто\-янии $l\hm\in\lbrace 
2,3,\ldots ,N\rbrace $ в~период
свободной эволюции (без внеш\-них воздействий). Пусть $\bar{c}\hm=(c_{l},l\hm\in\lbrace 2,3,\ldots ,N\rbrace )^{\mathrm{T}}$~--- век\-тор-стол\-бец
указанных доходов.

Обозначим через $d_{i}^{(s)}$ величину за\-трат, связанных с~переводом основного
процесса из граничного со\-сто\-яния~$s$ во
внутреннее со\-сто\-яние~$i, s\hm\in\lbrace 0,1\rbrace , i\hm\in\lbrace
2,3,\ldots ,N\rbrace $. Данные величины \mbox{характеризуют} расходы, необходимые для 
проведения внеш\-не\-го воздействия, в~результате которого основной процесс 
указанным образом изменяет свое со\-сто\-яние. В~соответствии со своим экономическим 
содержанием эти величины отрицательные.

Теперь приведем представления для некоторых дополнительных вероятностных 
и~стоимостных характеристик рас\-смат\-ри\-ва\-емой модели, основываясь на тео\-рии 
поглощающих марковских цепей.

Пусть $b_{i0}$ и~$b_{i1}$~--- вероятности по\-гло\-ще\-ния марковской цепи $\{ \xi 
_{k}^{(n)}\}_{k=0}^{\infty}$, $n\hm=
0,1,2,\dots$, в~состояниях $\lbrace 0\rbrace $ и~$\lbrace
1\rbrace $ соответственно при условии, что в~начальный момент времени
данный процесс находится в~со\-сто\-янии~$i$; $ \xi _{0}^{(n)}\hm=i$,
$i\hm\in\lbrace 2,3,\ldots ,N\rbrace $.

Пусть, далее, $r_{i}$~--- математическое ожидание дохода, связанного с~поведением
марковской цепи $\{ \xi _{k}^{(n)} \}_{k=0}^{\infty}$, $n \hm= 
0,1,2,\dots$, на всем периоде ее эволюции от начального момента до
поглощения при условии, что в~начальный момент времени данный процесс
находится в~со\-сто\-янии $i$; $ \xi _{0}^{(n)}\hm=i$, $i\hm\in\lbrace 2,3,\ldots
,N\rbrace $.
В~рассматриваемой модели предполагается, что доходы, связанные с~поведением 
марковских цепей $\{\xi_k^{(n)}\}_{k=0}^{\infty}$, $n\hm=0,1,2,\dots $, 
определяемые заданными выше па\-ра\-мет\-ра\-ми 
$\overline{c}\hm=\left(c_l, l\hm\in\{2,3,\dots,N\}\right)^{\mathrm{T}}$, по\-рож\-да\-ют 
доходы на со\-от\-вет\-ст\-ву\-ющих траекториях основного процесса 
$\{\widehat{\xi}_k\}_{k=0}^{\infty}$.

Тогда матрица вероятностей по\-гло\-ще\-ния $\textbf{B}$, состоящая из строк $(b_{i0}, 
b_{i1}), i\hm\in\lbrace 2,3,\ldots ,N\rbrace$, определяется формулой:
$$
\textbf{B}=(\textbf{I}-\textbf{P}_{00})^{-1}\textbf{P}_{01},
$$
где $\textbf{I}$~--- единичная мат\-ри\-ца раз\-мер\-ности $(N-1)\hm\times(N-1)$.

\smallskip

\noindent
\textbf{Замечание~1.} По своему вероятностному содержанию, мат\-ри\-ца~$\textbf{B}$ 
является стохастической:
$$
b_{i0}+b_{i1} = 1\,,\enskip i\in\lbrace 2,3,\ldots ,N\rbrace.
$$

Вектор $\bar{r}\hm=(r_{i},i\in\lbrace 2,3,\ldots ,N\rbrace )^{\mathrm{T}}$ может
быть выражен сле\-ду\-ющим образом:
$$
\bar{r}=(\textbf{I}-\textbf{P}_{00})^{-1}\bar{c}\,.
$$

Таким образом, для получения последующих результатов, ка\-са\-ющих\-ся решения задачи 
оптимального управ\-ле\-ния, в~рас\-смат\-ри\-ва\-емой сто\-ха\-сти\-че\-ской модели необходимо 
задать мат\-ри\-цу \mbox{вероятностей} перехода~$\textbf{P}$, вектор доходов, 
ха\-рак\-те\-ри\-зу\-ющих эволюцию основного процесса на периодах времени без проведения 
внеш\-них воздействий~$\bar{c}$, и~набор величин~$d_i^{(s)}$, 
$i\hm\in\{2,3,\dots,N\}$, $s\hm\in\{0,1\}$, ха\-рак\-те\-ри\-зу\-ющих за\-тра\-ты на проведение 
внешних воздействий или управ\-ле\-ний основным процессом. Остальные необходимые 
вероятностные и~стоимостные характеристики определяются на основе приведенных 
выше аналитических формул.

\section{Решение проблемы оптимального управления в~стохастической 
полумарковской модели с~дискретным временем}

В работах~\cite{A1, A4} доказано сле\-ду\-ющее утверж\-де\-нии о~пред\-став\-ле\-нии 
стационарного стоимостного показателя эф\-фек\-тив\-ности в~рас\-смат\-ри\-ва\-емой модели.

\smallskip

\noindent
\textbf{Теорема 1.}
\textit{Предположим, что в~рассматриваемой стохастической модели выполняются 
условия: 
$$
b_{l,0}>0\,,\enskip b_{l,1}>0\,,\enskip l\in\{2,3,\dots,N\}.
$$
 Тогда имеет мес\-то 
сле\-ду\-ющее пред\-став\-ле\-ние для стационарного стоимостного показателя 
$I\hm=I\left(\alpha^{(0)},\alpha^{(1)}\right)$, который по содержанию пред\-став\-ля\-ет 
собой сред\-нюю удельную прибыль}:
\begin{multline}
I = I\left(\alpha ^{(0)},\alpha
^{(1)}\right)={}\\
{}=\fr{\sum\nolimits_{m_{0}=2}^{N}{\sum\nolimits_{m_{1}=2}^{N}{A(m_{0},
m_{1})\alpha
_{m_{0}}^{(0)}\alpha_{m_{1}}^{(1)}}}}{\sum\nolimits_{m_{0}=2}^{N}{\sum\nolimits_{m_{
1}=2}^{N}{B(m_{0},m_{1})\alpha
_{m_{0}}^{(0)}\alpha _{m_{1}}^{(1)}}}}, \label{e1-sh}
\end{multline}
\textit{где}
\begin{align}
A(m_{0},m_{1})&={}\notag\\
&\hspace*{-20mm}{}=\left[ d_{m_{0}}^{(0)}+r_{m_{0}}\right] b_{m_{1},0} +\left[ d_{m_{1}}^{(1)}+r_{m_{1}}\right] b_{m_{0},1} ; \label{e2-sh}\\[3pt]
B(m_{0},m_{1})&=b_{m_{0},1}+b_{m_{1},0} . \label{e3-sh}
\end{align}

\noindent
\textbf{Замечание~2.} Основу доказательства тео\-ре\-мы~1\linebreak со\-став\-ля\-ют эргодические 
теоремы для стоимостных аддитивных функционалов, связанных с~полумарковскими 
случайными процессами. Основное достаточное условие для применения этих \mbox{тео\-рем} 
в~рас\-смат\-ри\-ва\-емой полумарковской модели заключается в~том, что любая цепь Маркова 
$\{ \xi _{k}^{(n)} \}_{k=0}^{\infty}$, $n\hm=0,1,2,\dots$, опи\-сы\-ва\-ющая 
поведение системы на периоде времени при отсутствии внеш\-них воздействий 
(управлений), должна за конечное время с~ве\-ро\-ят\-ностью, равной единице, достичь 
одного из граничных поглощающих состояний. Условия на вероятности по\-гло\-ще\-ния, 
включенные в~формулировку тео\-ре\-мы, обеспечивают выполнение этого требования в~усиленной форме.

\smallskip


Из утверждения теоремы~1 следует, что в~рас\-смат\-ри\-ва\-емой стохастической модели 
стационарный стоимостный показатель эф\-фек\-тив\-ности\linebreak управ\-ле\-ния 
$I(\alpha^{(0)}, \alpha^{(1)})$, аналитически опре\-де\-ля\-емый формулами~(\ref{e1-sh})--(\ref{e3-sh}), 
пред\-став\-ля\-ет собой дискретный вариант так называемого дроб\-но-ли\-ней\-но\-го 
интегрального функционала. В~этом \mbox{варианте} функционал определен на множестве пар 
дискретных вероятностных распределений $(\alpha^{(0)}, 
\alpha^{(1)})$.
Таким образом, задача оптимального управ\-ле\-ния формулируется в~виде экстремальной 
задачи на максимум для функционала вида~(\ref{e1-sh}).

Решение общей задачи о безусловном экстремуме дроб\-но-ли\-ней\-но\-го интегрального 
функционала, заданного на множестве конечных наборов вероятностных мер, изложено в~работах~\cite{A2, A3}. Отметим также, что применение утверж\-де\-ний об экстремуме 
дроб\-но-ли\-ней\-но\-го интегрального функционала для решения задачи о~настройке 
описано в~работах~\cite{A1, A4}. В~общем варианте решение экстремальной задачи 
для дроб\-но-ли\-ней\-но\-го интегрального функционала пол\-ностью определяется 
экстремальными свойствами так называемой основной функции, которая пред\-став\-ля\-ет 
собой отношение подынтегральных функций чис\-ли\-те\-ля и~знаменателя. В~дискретном 
варианте эти подынтегральные функции совпадают с~функциями под знаком 
многомерных сумм, не зависящими от набора вероятностных распределений, по 
которому осуществляется оптимизация.

Перейдем к~анализу особенностей экстремальной задачи для целевого функционала~(\ref{e1-sh}). 
Роли пространств допустимых решений (управ\-ле\-ний) играют конечные множества 
$$
U_0=U_1=\{2,3,\dots,N\}
$$ 
и~их декартово произведение 
$$
U=U_0\times U_1,
$$
 которое 
представляет собой множество пар 
$$
U=\left\{(m_0,m_1): m_0\in U_0, m_1 \in 
U_1\right\}.
$$
 Дискретные вероятностные распределения 
$$
\alpha^{(0)}=\left(\alpha_l^{(0)}, l=\overline{2,N}\right);\ 
\alpha^{(1)}=\left(\alpha_l^{(1)}, l=\overline{2,N}\right)
$$
 определены на 
множествах $U_0$ и~$U_1$ соответственно. Основная функция дроб\-но-ли\-ней\-но\-го 
интегрального дискретного функционала    $I(\alpha^{(0)}, 
\alpha^{(1)})$ определяется формулой:
\begin{equation}
C(m_{0},m_{1})=\fr{A(m_{0},m_{1})}{B(m_{0},m_{1})}\,, \label{e4-sh}
\end{equation}
где функции $A(m_{0},m_{1})$ и~$B(m_{0},m_{1})$ задаются равенствами~(\ref{e2-sh}) и~(\ref{e3-sh}) 
соответственно.

Заметим, что в~рас\-смат\-ри\-ва\-емой задаче функция $C(m_{0},m_{1})$ определена на 
конечном множестве значений аргументов~$U$.
Таким образом, данная функция достигает на множестве~$U$ своего минимального и~максимального значения (глобальных
экстремумов). Напомним так\-же, что в~рас\-смат\-ри\-ва\-емой сто\-ха\-сти\-че\-ской модели 
вероятностные характеристики $b_{m_0,1}$ и~$b_{m_1,0}$ предполагались строго 
положительными при всех значениях $m_0\hm\in U_0 \hm= \{2,3,\dots,N\}$, $m_1\hm\in U_1 \hm= 
\{2,3,\dots,N\}$ (см.\ формулировку тео\-ре\-мы~1 и~замечание~2). Отсюда следует, что 
выполняется условие
$$
B(m_{0},m_{1})=b_{m_{0},1}+b_{m_{1},0} >0,\enskip (m_{0},m_{1})\in U \,
$$

Таким образом, выполнены все условия тео\-ре\-мы об экстремуме дроб\-но-ли\-ней\-но\-го 
интегрального функционала, заданного на множестве дискретных вероятностных 
распределений. В~соответствии с~этой тео\-ре\-мой решение
экстремальной задачи для функционала~(\ref{e1-sh}) на максимум существует и~достигается на 
вы\-рож\-ден\-ных
дискретных распределениях, сосредоточенных в~точках $m_{0}^{*}$ и~$m_{1}^{*}$. При 
этом $(m_{0}^{*}$, $m_{1}^{*})$ есть точка, в~которой достигается со\-от\-вет\-ст\-ву\-ющий 
глобальный
максимум основной функции $C(m_{0},m_{1})$, определяемой формулами~(\ref{e4-sh}), (\ref{e2-sh}) и~(\ref{e3-sh}).

Таким образом, решение задачи оптимального управления в~рас\-смат\-ри\-ва\-емой 
стохастической модели пред\-став\-ля\-ет собой пару детерминированных значений 
па\-ра\-мет\-ров управ\-ле\-ния $(m_{0}^{*}$, $m_{1}^{*})$, до\-став\-ля\-ющих максимум явно 
заданной функции $C(m_{0},m_{1})$.

\begin{figure*} %fig1
\vspace*{1pt}
  \begin{center}
 \mbox{%
 \epsfxsize=159.942mm 
\epsfbox{shn-1.eps}
 }
\end{center}
\vspace*{-6pt}
\Caption{Исходные матрицы $\textbf{P}$, полученные в~результате оценки: (\textit{а})~период 
11.01.2010--25.09.2011; (\textit{б})~период 11.01.2012--25.12.2013}
\label{fig}
\vspace*{3pt}
\end{figure*}

\vspace*{-9pt}

\section{Численные решения задачи оптимального управления}

\noindent
\textbf{Пример~1.} В этом примере стохастическая полумарковская управ\-ля\-емая 
модель, в~рамках которой рас\-смат\-ри\-ва\-ет\-ся задача о~настройке, строится на базе 
наблюдений за ценой бивалютной корзины на валютном рынке Российской Федерации 
в~пери-\linebreak\vspace*{-12pt}

\pagebreak

\noindent
од с~11.01.2010 по 25.09.2011. Оценки элементов мат\-ри\-цы вероятностей 
перехода для поглощающей цепи Маркова получены в~работе~\cite{A6}. В~рас\-смат\-ри\-ва\-емом варианте предполагается, что $N \hm= 37$. Таким образом, 
в~со\-от\-вет\-ст\-ву\-ющей марковской цепи два со\-сто\-яния $\{0\}$ и~$\{1\}$ являются граничными 
и~по\-гло\-ща\-ющи\-ми, а~остальные~36 образуют множество так называемых внут\-рен\-них 
невозвратных со\-сто\-яний. В~данной работе мат\-ри\-ца оценок вероятностей перехода~P 
преобразована в~стандартную фор\-му для по\-гло\-ща\-ющей марковской цепи. Визуализация 
этой мат\-ри\-цы пред\-став\-ле\-на на рис.~1,\,\textit{а}.



Матрица $\textbf{B}$, элементами которой служат вероятности поглощения 
в~граничных со\-сто\-яни\-ях $\{0\}$ и~$\{1\}$ при фиксированных условиях на начальные 
со\-сто\-яния, приведена на рис.~2,\,\textit{а}. Заметим, что элементы этой мат\-ри\-цы удовле\-тво\-ря\-ют 
условиям, достаточным для выполнения утверж\-де\-ний тео\-ре\-мы~1 и~тео\-ре\-мы об 
экстремуме дроб\-но-ли\-ней\-но\-го интегрального функционала.



Графики функций, определяющих доходы, по\-лу\-ча\-емые при однократном попадании 
в~каж\-дое со\-сто\-яние, и~за\-тра\-ты, связанные с~переводом основного процесса из 
граничных по\-гло\-ща\-ющих \mbox{со\-сто\-яний} во внут\-рен\-ние, пред\-став\-ле\-ны на рис.~3,\,\textit{а} и~4,\,\textit{а}. При 
определении па\-ра\-мет\-ров доходов используется качественное предположение о~том, 
что доходы монотонно воз\-рас\-та\-ют с~рос\-том рас\-сто\-яния от граничного по\-гло\-ща\-юще\-го 
со\-сто\-яния до со\-от\-вет\-ст\-ву\-юще\-го внут\-рен\-не\-го и~достигают максимального значения 
в~середине множества внут\-рен\-них состояний. Относительно функций затрат вводится 
со\-от\-вет\-ст\-ву\-ющее предположение, за\-клю\-ча\-юще\-еся в~том, что они монотонно воз\-рас\-та\-ют 
по абсолютной величине с~рос\-том рас\-сто\-яния от граничного по\-гло\-ща\-юще\-го со\-сто\-яния, 
откуда переводится основной процесс, до внут\-рен\-не\-го со\-сто\-яния, в~которое он 
будет переведен.


Графическое представление основной функции $C(m_0, m_1)$, которая определяет 
оптимальную детерминированную стратегию управ\-ле\-ния, приведено на рис.~5,\,\textit{а}. 
Максимальное значение функции достигается в~точке $m_0^*\hm=14$, $m_1^*\hm=17$, 
со\-от\-вет\-ст\-ву\-ющее максимальное значение функции $C(14;17) \hm= 1561{,}433986$.

\begin{figure*} %fig2
 \vspace*{1pt}
 \begin{center}
    \mbox{%
 \epsfxsize=83.443mm 
\epsfbox{shn-2.eps}
 }
\end{center}
\vspace*{-11pt}
\Caption{Матрицы вероятностей поглощения $\textbf{B}$: (\textit{а})~пример~1; (\textit{б})~пример~2}
%\end{figure*}
%\begin{figure*} %fig3
\vspace*{18pt}
  \begin{center}
 \mbox{%
 \epsfxsize=159.1mm 
\epsfbox{shn-3.eps}
 }
\end{center}
\vspace*{-9pt}
\Caption{Графики функции $c$, определяющие доходы в~каждом состоянии:
(\textit{а})~пример~1; (\textit{б})~пример~2}
%\label{fig}
\end{figure*}


\smallskip

\noindent
\textbf{Пример~2.} В данном примере соответствующая полумарковская управляемая 
модель строится на базе данных наблюдений за ценой бивалютной корзины в~период с~01.01.2012 по 01.12.2013. Как и~в~первом примере, оценки элементов мат\-ри\-цы 
вероятностей перехода для поглощающей цепи Маркова получены в~работе~\cite{A6}. 
При этом вновь предполагается, что параметр $N\hm=37$. Визуализация матрицы оценок 
вероятностей перехода для по\-гло\-ща\-ющей марковской цепи~ $\textbf{P}$ пред\-став\-ле\-на 
на рис.~1,\,\textit{б}. Со\-от\-вет\-ст\-ву\-ющая мат\-ри\-ца оценок вероятностей по\-гло\-ще\-ния $\textbf{B}$ 
приведена на рис.~2,\,\textit{б}. Заметим, что, как\linebreak и~в~предыду\-щем варианте, элементы этой 
матри-\linebreak цы удовле\-тво\-ря\-ют условиям, достаточным для\linebreak выполнения утверж\-де\-ний тео\-ре\-мы~1 
и~теоремы об экстремуме дроб\-но-ли\-ней\-но\-го интегрального функционала. Графики 
функций, опре\-де\-ля\-ющих\linebreak\vspace*{-11.5pt}

\pagebreak

\end{multicols}

\begin{figure*} %fig4
\vspace*{1pt}
  \begin{center}
 \mbox{%
 \epsfxsize=162.901mm 
\epsfbox{shn-4.eps}
 }
\end{center}
\vspace*{-9pt}
\Caption{Графики функций $d^{(0)}$~(\textit{1}) и~$d^{(1)}$~(\textit{2}), определяющих расходы, связанные с~переводом основного процесса из граничных со\-сто\-яний во внут\-рен\-ние:
(\textit{а})~пример~1; (\textit{б})~пример~2}
%\label{fig}
%\end{figure*}
%\begin{figure*} %fig5
\vspace*{11pt}
  \begin{center}
 \mbox{%
 \epsfxsize=157.34mm 
\epsfbox{shn-5.eps}
 }
\end{center}
\vspace*{-9pt}
\Caption{Основные функции $C(m_0,m_1)$, определяющие решение задачи:
(\textit{а})~пример~1; (\textit{б})~пример~2}
%\label{fig}
\vspace*{3pt}
\end{figure*}


\begin{multicols}{2}

\noindent
 стоимостные характеристики модели, пред\-став\-ле\-ны на рис.~3,\,\textit{б} 
и~4,\,\textit{б}. Поведение этих функций
 предполагается аналогичным поведению со\-от\-вет\-ст\-ву\-ющих 
функций в~примере~1.
Графическое представление основной функции $C(m_0, m_1)$, которая определяет 
оптимальную детерминированную стратегию управ\-ле\-ния, приведено на рис.~5,\,\textit{б}. 
Максимальное значение функции достигается в~точке $m_0^*\hm=15$, $m_1^*\hm=17$, 
соответствующее максимальное значение функции $C(15;17)\hm = 1710{,}247851$.
%\linebreak\vspace*{-12pt}

%\pagebreak

%\end{multicols}





%\begin{multicols}{2}

%\noindent
 
%\end{multicols}


\vspace*{-12pt}

\section{Заключение}

\vspace*{-4pt}

Оптимальное управление в~задаче о~настройке с~дискретным временем определено 
в~настоящей работе лишь для двух вариантов задания исходных па\-ра\-мет\-ров модели. 
Число таких вариантов можно увеличить, рас\-смат\-ри\-вая различные соотношения 
параметров. Это позволит выявить влияние каж\-до\-го из них на оптимальное решение. 
Анализ такого влияния может со\-ста\-вить содержание отдельного исследования.

Важное направление дальнейших исследований связано так\-же с~расширением круга 
при\-клад\-ных задач, для которых могут быть использованы тео\-ре\-ти\-че\-ские результаты 
решения задачи о~настройке. Как уже отмечалось в~работах~\cite{A1, A4, A5}, 
стохастические управ\-ля\-емые модели, проблема оптимального управления в~которых 
сводится к~одному из вариантов задачи о~настройке, возникают в~различных 
областях техники и~экономики. Со\-от\-вет\-ст\-ву\-ющее исследование должно иметь 
комплексный характер, в~котором будут существенно использоваться современные 
методы математической ста\-ти\-сти\-ки и~вы\-чис\-ли\-тель\-ной математики.

{\small\frenchspacing
 { %\baselineskip=11.5pt
 %\addcontentsline{toc}{section}{References}
 \begin{thebibliography}{9}
\bibitem{A1} 
\Au{Shnurkov P.\,V.} Optimal control problem in a stochastic 
model with periodic hits on the boundary of a given subset of the state set 
(tuning problem).~--- Cornell University, 2017.  16~p. arXiv: 1709.03442v1.
\bibitem{A2} 
\Au{Шнурков П.\,В.} О решении задачи безусловного экстремума 
для дроб\-но-ли\-ней\-но\-го интегрального функционала на множестве вероятностных мер~// 
Докл. Акад. наук, 2016. Т.~470. №\,4. С.~387--392.
\bibitem{A3} 
\Au{Шнурков П.\,В., Горшенин~А.\,К., Белоусов~В.\,В.} 
Аналитическое решение задачи оптимального управ\-ле\-ния полумарковским процессом 
с~конечным множеством состояний~// Информатика и~её применения, 
2016. Т.~10. Вып.~4. С.~72--88. doi: 10.14357/19922264160408. EDN: XGSITZ.
\bibitem{A4} 
\Au{Shnurkov P.\,V., Novikov~D.\,A.} Analysis of the problem 
of intervention control in the economy on the basis of solving the problem of 
tuning.~--- Cornell University, 2018.  15~p. arXiv: 1811.10993.
\bibitem{A5} 
\Au{Шнурков П.\,В., Новиков~Д.\,А.} О концепции 
стохастической модели с~управ\-ле\-ни\-ем в~моменты выхода процесса на границу 
заданного подмножества множества со\-сто\-яний~// Информатика и~её 
применения, 2020. Т.~14. Вып.~3. С.~101--108. doi: 10.14357/19922264200315. EDN: FSLGBJ.
\bibitem{A6} 
\Au{Шнурков П.\,В., Мигуля~М.\,А.}  Некоторые результаты 
анализа процесса изменения цены бивалютной корзины на основе методов статистики 
случайных процессов~// Информатика и~её применения, 2022. Т.~16. 
Вып.~3. С.~16--25. doi: 10.14357/19922264220303. EDN: QGZCHT.
\end{thebibliography}

 }
 }

\end{multicols}

\vspace*{-6pt}

\hfill{\small\textit{Поступила в~редакцию 15.07.24}}

\vspace*{6pt}

%\pagebreak

%\newpage

%\vspace*{-28pt}

\hrule

\vspace*{2pt}

\hrule

\vspace*{-2pt}

\def\tit{NUMERICAL-ANALYTICAL SOLUTION OF~THE~DISCRETE-TIME TUNING PROBLEM FOR~AN~INTERVENTION MODEL IN~THE~FOREIGN EXCHANGE MARKET}


\def\titkol{Numerical-analytical solution of~the~discrete-time tuning problem for~an~intervention model in~the~foreign exchange market}


\def\aut{P.\,V.~Shnurkov and~D.\,A.~Novikov}

\def\autkol{P.\,V.~Shnurkov and~D.\,A.~Novikov}

\titel{\tit}{\aut}{\autkol}{\titkol}

\vspace*{-10pt}




\noindent
National Research University ``Higher School of Economics,'' 34~Tallinskaya Str., Moscow 123458, Russian Federation


\def\leftfootline{\small{\textbf{\thepage}
\hfill INFORMATIKA I EE PRIMENENIYA~--- INFORMATICS AND
APPLICATIONS\ \ \ 2024\ \ \ volume~18\ \ \ issue\ 3}
}%
 \def\rightfootline{\small{INFORMATIKA I EE PRIMENENIYA~---
INFORMATICS AND APPLICATIONS\ \ \ 2024\ \ \ volume~18\ \ \ issue\ 3
\hfill \textbf{\thepage}}}

\vspace*{4pt}



\Abste{The work examines the problem of optimizing external influences (controls) on the process of changing the price of the so-called 
bi-currency basket on the foreign exchange market of the Russian Federation. The theoretical basis of the approach used is the solution 
of a stochastic tuning problem with discrete time. Based on the previous statistical analysis, it was established that the stochastic 
process characterizing the evolution of the bi-currency basket price, under certain conditions, can be quite adequately described by 
a~classical homogeneous Markov chain. In this case, statistical estimates of the transition probabilities of the specified chain were obtained.
Necessary auxiliary probabilistic characteristics of the Markov model are numerically determined. For various given
 cost characteristics, a~study of the stationary cost indicator of management efficiency-average specific profit was conducted. Specific numerical
  solutions to the corresponding optimal control problem are obtained which can be interpreted as optimal external influences (interventions) 
  on the stochastic process under study.}


\KWE{stochastic Markov and semi-Markov control models; discrete time tuning
problem; fractional linear integral functionals on discrete probability distributions; optimal
control in stochastic economic systems}

\DOI{10.14357/19922264240310}{ZXNZDQ}

%\pagebreak


    
     % \Ack

%\vspace*{10pt}

%\noindent


  \begin{multicols}{2}

\renewcommand{\bibname}{\protect\rmfamily References}
%\renewcommand{\bibname}{\large\protect\rm References}

{\small\frenchspacing
 {%\baselineskip=10.8pt
 \addcontentsline{toc}{section}{References}
 \begin{thebibliography}{9} 
%1
\bibitem{A1-1} 
\Aue{Shnurkov, P.\,V.} 2017. Optimal control problem in a~stochastic model with periodic hits on the boundary
 of a~given subset of the state set (tuning problem).  Cornell University. 16~p. Available at: 
 {\sf https://arxiv.org/abs/1709.03442v1} (accessed July~29, 2024).

%2
\bibitem{A2-1} 
\Aue{Shnurkov, P.\,V.} 2016. 
Solution of the unconditional extremum problem for a linear-fractional integral functional on a~set of probability measures. 
\textit{Dokl. Math.} 94(2):550--554. 
doi: 10.1134/S1064562416050161. EDN:\linebreak XFUJKH.

%3
\bibitem{A3-1} 
\Aue{Shnurkov, P.\,V., A.\,K.~Gorshenin, and V.\,V.~Belousov.} 2016. 
Analiticheskoe reshenie zadachi optimal'nogo upravleniya polumarkovskim protsessom s~konechnym mnozhestvom sostoyaniy 
[Analytical solution of the optimal control task of a~semi-Markov process with finite set of states]. 
\textit{Informatika i~ee Primeneniya~--- Inform. Appl.} 10(4):72--88.
doi: 10.14357/19922264160408. EDN: XGSITZ.

%4
\bibitem{A4-1} 
\Aue{Shnurkov, P.\,V., and D.\,A.~Novikov.} 2018. 
Analysis of the problem of intervention control in the economy on the basis of solving the problem of tuning. Cornell University. 15~p.
Available at: {\sf https://arxiv.org/abs/1811.10993} (accessed July~29, 2024).

%5
\bibitem{A5-1} 
\Aue{Shnurkov, P.\,V., and D.\,A.~Novikov.} 2020.
O~kon\-tsep\-tsii sto\-kha\-sti\-che\-skoy modeli s~uprav\-le\-niem v~mo\-men\-ty vykhoda pro\-tses\-sa na gra\-ni\-tsu 
za\-dan\-no\-go pod\-mno\-zhest\-va mno\-zhest\-va so\-sto\-yaniy
[On the concept of a~stochastic model with control at the moments of the process at the border of a~presented subset of multiple states].
\textit{Informatika i~ee Primeneniya~--- Inform. Appl.} 14(3):101--108.
doi: 10.14357/19922264200315. EDN: FSLGBJ.

%6
\bibitem{A6-1} 
\Aue{Shnurkov, P.\,V., and M.\,A.~Migulya.} 2022. 
Ne\-ko\-to\-rye re\-zul'\-ta\-ty ana\-li\-za pro\-tses\-sa iz\-me\-ne\-niya tse\-ny bi\-va\-lyut\-noy kor\-zi\-ny na osno\-ve me\-to\-dov 
sta\-ti\-sti\-ki slu\-chay\-nykh pro\-tses\-sov 
[Some results of the analysis of the process of changing the price of a~dual currency basket based on random process statistics methods].
\textit{Informatika i~ee Primeneniya~--- Inform. Appl.} 16(3):16--25.
doi: 10.14357/19922264220303. EDN: QGZCHT.


\end{thebibliography}

 }
 }

\end{multicols}

\vspace*{-6pt}

\hfill{\small\textit{Received July 15, 2024}} 

\vspace*{-18pt}

\Contr

\vspace*{-6pt}

\noindent
\textbf{Shnurkov Peter V.} (b.\ 1953)~--- Candidate of Science (PhD) in physics 
and mathematics, associate professor, National Research University ``Higher 
School of Economics,'' 34~Tallinskaya Str., Moscow 123458, Russian Federation; \mbox{pshnurkov@hse.ru}

\vspace*{3pt}

\noindent
\textbf{Novikov Daniil A.} (b.\ 1993)~--- PhD applicant, National Research 
University ``Higher School of Economics,'' 34~Tallinskaya Str., Moscow 123458, 
Russian Federation; \mbox{even.he@yandex.ru}



\label{end\stat}

\renewcommand{\bibname}{\protect\rm Литература} 