\def\stat{zatsman}

\def\tit{МОДЕЛЬ ИЗВЛЕЧЕНИЯ ЗНАНИЯ ИЗ~ПАРАЛЛЕЛЬНЫХ ТЕКСТОВ 
ЛЕКСИКОГРАФИЧЕСКОЙ ИНФОРМАЦИОННОЙ СИСТЕМЫ$^*$}

\def\titkol{Модель извлечения знания из~параллельных текстов 
лексикографической информационной системы}

\def\aut{Д.\,О.~Добровольский$^1$, И.\,М.~Зацман$^2$}

\def\autkol{Д.\,О.~Добровольский, И.\,М.~Зацман}

\titel{\tit}{\aut}{\autkol}{\titkol}

\index{Добровольский Д.\,О.}
\index{Зацман И.\,М.}
\index{Dobrovol'skij D.\,O.}
\index{Zatsman I.\,M.}


{\renewcommand{\thefootnote}{\fnsymbol{footnote}} \footnotetext[1]
{Исследование выполнено в~ФИЦ ИУ РАН за счет гранта Российского научного фонда №\,24-18-00155, {\sf 
https://rscf.ru/project/24-18-00155}, с~использованием с~использованием инфраструктуры Центра коллективного пользования 
<<Высокопроизводительные вы\-чис\-ле\-ния и~большие данные>> 
(ЦКП <<Информатика>>) ФИЦ ИУ РАН (г.~Москва).}}



\renewcommand{\thefootnote}{\arabic{footnote}}
\footnotetext[1]{Институт русского языка Российской академии наук; Институт языкознания Российской академии наук; 
Федеральный исследовательский центр <<Информатика и~управ\-ле\-ние>> Российской академии наук, 
\mbox{dm-dbrv@yandex.ru.}}
\footnotetext[2]{Федеральный исследовательский центр <<Информатика и~управление>> Российской академии наук, 
\mbox{izatsman@yandex.ru}}


%\vspace*{-6pt}

  
  \Abst{Рассматривается проблемно-ориентированная модель извлечения языкового знания 
из параллельных текстов как ключевой теоретический компонент концепции 
лексикографической информационной системы (ЛГИС), обеспечивающей интеграцию электронных 
двуязычных словарей и~параллельных корпусов. Предлагаемый подход к~решению проблемы 
интеграции учитывает появление новых значений слов и~устойчивых словосочетаний, 
которое обусловлено приобретением нового знания экспертами, фиксирующими эти 
значения, в~результате семантического анализа регулярно пополняемых корпусных 
текстовых данных. Предлагаемая модель описывает взаимодействие компьютерных 
и~экспертных процессов, в~том числе поиск фрагментов параллельных текстов как 
потенциальных источников нового языкового знания, его извлечение экспертами из текстов 
и~представление в~ЛГИС. Основанием для 
построения проб\-лем\-но-ори\-ен\-ти\-ро\-ван\-ной модели служит спиральная модель 
генерации знания, которую в~1991~г.\ предложил Икуджиро Нонака. \mbox{Цель} \mbox{статьи} со\-сто\-ит 
в~описании стадий построения модели извлечения языкового знания, используемой при 
проектировании ЛГИС.}
  
  
  \KW{лексикографическая информационная система; параллельные текс\-ты; спиральная 
модель генерации знания; проб\-лем\-но-ори\-ен\-ти\-ро\-ван\-ная модель}

\DOI{10.14357/19922264240312}{NDNPCF}
  
%\vspace*{-2pt}


\vskip 10pt plus 9pt minus 6pt

\thispagestyle{headings}

\begin{multicols}{2}

\label{st\stat}

\section{Введение}

  Лексикографическая информационная система, проектируемая в~ФИЦ ИУ 
РАН, предназначена для решения актуальной, но нерешенной проб\-ле\-мы 
интеграции электронных двуязычных словарей и~параллельных корпусов, 
учитывающей регулярное пополнение корпусов новыми текс\-то\-вы\-ми данными. 
Этап создания концепции этой сис\-те\-мы предполагает решение комплекса 
вопросов, одни\linebreak из которых относятся к~лингвистике, другие~--- 
к~информатике. Ключевым тео\-ре\-ти\-че\-ским компонентом концепции 
ЛГИС служит  
проб\-лем\-но-ори\-ен\-ти\-ро\-ван\-ная модель информатики поэтапного\linebreak 
\mbox{извлечения} языков$\acute{\mbox{о}}$го знания из параллельных текс\-тов.
  
  Теоретическим основанием ее по\-стро\-ения в~данной статье служит 
спиральная модель генерации знания, которую в~1991~г.\ предложил Икуджиро 
Нонака~[1]. В~спиральной модели определены две категории знания: явное 
(explicit) и~неявное (tacit). В~каж\-дой из них знание делится на индивидуальное 
и~коллективное. Таким образом, в~этой модели определены четыре вида знания 
(индивидуальное явное и~неявное, коллективное явное и~неявное). В~модели 
также определены четыре экспертных процесса преобразования этих четырех 
видов знания~[2].
  
  Для решения проблемы интеграции электронных двуязычных словарей 
и~параллельных корпусов в~спиральную модель генерации знания необходимо 
добавить циф\-ро\-вую среду компьютерных кодов для хранения и~обработки 
текстов электронных словарей и~корпусов. Как будет показано ниже, ее 
добавление происходит на одной из первых стадий построения  
проб\-лем\-но-ори\-ен\-ти\-ро\-ван\-ной модели извлечения 
языков$\acute{\mbox{о}}$го знания. Описанию этой и~других стадий 
по\-стро\-ения такой модели и~посвящена данная статья.

\section{Исходные данные}

  В процессе построения проб\-лем\-но-ори\-ен\-ти\-ро\-ван\-ной модели учитывались 
сле\-ду\-ющие исходные данные, сформулированные лингвистами. Во-пер\-вых, 
реальные упо\-треб\-ле\-ния лексических единиц в~текс\-тах параллельного корпуса 
во всем их смыс\-ло\-вом многообразии в~проектируемой сис\-те\-ме необходимо 
со\-по\-ста\-вить с~релевантными элементами\linebreak словарных статей двуязычного 
словаря. Цент\-раль\-ной оказывается при этом проб\-ле\-ма мно\-го\-знач\-ности 
лексических единиц. Если слова и~устойчивые словосочетания (далее~--- 
фраземы), \mbox{об\-ла\-да\-ющие} только одним значением, поз\-во\-ля\-ют постулировать 
связь между элементом словарной статьей и~релевантными корпусными 
примерами на уровне лем\-мы, то все многозначные лексические единицы 
требуют соотнесения корпусных данных со словарной информацией на уровне 
конкретных значений таких лексических единиц~[3, 4]. Игнорирование 
многозначности дает результаты, неудовлетворительные с~точ\-ки зрения 
современных пред\-став\-ле\-ний об эф\-фек\-тив\-ности ресурсов, пред\-по\-ла\-га\-ющих 
интеграцию словарной информации с~корпусными данными.
  
  Безусловно, даже не учитывающие мно\-го\-знач\-ность информационные 
ресурсы оказываются полезными для эмпирически ориентированных 
лингвистических исследований, но предостав\-ля\-емые ими данные требуют от 
пользователя большой и~слож\-ной дополнительной работы. Так, например, 
устроен проект Бер\-лин\-ско-Бран\-ден\-бург\-ской академии наук DWDS (Digitales 
W$\ddot{\mbox{o}}$rterbuch der deutschen\linebreak Sprache) [{\sf 
https://www.dwds.de}]\footnote{Об общих принципах построения этого ресурса  
см.~\cite{5-zac}.}. Этот ресурс предполагает воз\-мож\-ность перехода от леммы 
словарной \mbox{статьи} к~примерам из разных корпусов, интегрированных 
в~структуру ресурса, однако \mbox{сис\-те\-ма} не позволяет найти примеры, 
ил\-люст\-ри\-ру\-ющие конкретное значение данного слова. Этим DWDS 
принципиально отличается от результатов, которые ожидается получить в~ходе 
проектирования ЛГИС. Еще одно 
важное отличие~--- ориентированность DWDS на один язык, а~не на пару 
языков, как в~про\-ек\-ти\-ру\-емой сис\-теме.
  
  Во-вторых, авторы двуязычных словарей уже дав\-но используют 
параллельные корпусы для сбора материала и~эмпирической проверки своих 
гипотез, ка\-са\-ющих\-ся межъязыков$\acute{\mbox{о}}$й эквивалентности. 
Ценность параллельных корпусов определяется тем, что в~лингвистике этап 
сбора исходного материала считается наиболее трудоемким и~наименее 
творческим, а~параллельные корпусы позволяют значительно сэкономить время и~силы для творческого этапа создания словарей~[6, 7].
  
  При этом онлайновые связи созданных двуязычных словарей 
с~параллельными корпусами, которые служили источниками исходного 
материала, в~на\-сто\-ящее время отсутствуют. Параллельные корпусы постоянно 
пополняются новыми текс\-та\-ми, в~которых можно обнаружить новые значения 
слов и~фразем. Однако при этом отсутствуют методы и~средства оперативного 
об\-нов\-ле\-ния словарей по корпусным данным. В~настоящее время проб\-ле\-ма 
уста\-нов\-ле\-ния связей между электронными двуязычными словарями 
и~параллельными корпусами (далее~--- проб\-ле\-ма интеграции) находится на 
стадии поиска концептуальных подходов к~их интеграции на уровне значений.
  
  В-третьих, эксперименты, проведенные с~использованием ЦКП 
<<Информатика>> ФИЦ ИУ РАН, показали, что обнаружение нового 
языков$\acute{\mbox{о}}$го знания обусловливает и~формирование дефиниций 
новых значений слов и~фразем, и~пересмотр уже существующих дефиниций~[8,~9].
  
  Таким образом, в~процессе построения  
проб-\linebreak лем\-но-ори\-ен\-ти\-ро\-ван\-ной модели извлечения 
язы\-ко\-в$\acute{\mbox{о}}$\-го знания из текстов параллельных корпусов 
учитываются:
  \begin{itemize}
  \item потенциальная многозначность лексических единиц;
  \item необходимость установления онлайновых связей электронных 
двуязычных словарей с~текс\-та\-ми параллельных корпусов;
  \item обнаружение экспертами нового языков$\acute{\mbox{о}}$го знания, 
формирование ими дефиниций новых значений слов и~фразем, а~также 
пересмотр уже су\-щест\-ву\-ющих дефиниций. 
  \end{itemize}
  
   \begin{figure*} %fig1
  \vspace*{1pt}
  \begin{center}
 \mbox{%
 \epsfxsize=88.717mm 
\epsfbox{zac-1.eps}
 }
\end{center}
\vspace*{-9pt}
  \Caption{Спиральная модель~[1] (эта диаграмма взята из работы~\cite{14-zac})}
  \vspace*{-6pt}
  \end{figure*}
  
\section{Стадии построения проблемно-ориентированной модели}
  
  В работе~[10] соотнесены базовые понятия информатики, включенные 
в~иерархию Акоффа~[11] (данные, информация, знание), с~объектами 
про\-ек\-ти\-ру\-емой ЛГИС. В~интересах 
описания процессов извлечения из текстов параллельных кор\-пу\-сов новых 
значений ис\-сле\-ду\-емых языков$\acute{\mbox{ы}}$х \mbox{единиц} в~\mbox{статье} 
различаются два вида объектов этой сис\-те\-мы, относящихся в~иерархии Акоффа к~понятию <<данные>>: \textit{сенсорно вос\-при\-ни\-ма\-емые} и~\textit{циф\-ро\-вые}, 
а~также два вида объектов, относящихся в~ней к~понятию <<информация>>: 
\textit{сенсорно вос\-при\-ни\-ма\-емая} и~\textit{циф\-ро\-вая}. В~сис\-те\-ме кроме 
\textit{знания} рас\-смат\-ри\-ва\-ют\-ся еще два ментальных объекта. Во-пер\-вых, это 
\textit{концепты} слов, так как одно и~то же знание в~разных языках членится 
на значения (концептуальные единицы), как правило, по-раз\-но\-му.  
Во-вто\-рых, это ментальные образы сенсорно вос\-при\-ни\-ма\-емых текс\-то\-вых 
данных, на основе которых и~формируется знание в~процессе их интерпретации 
(далее~--- \textit{ментальные данные})~[12]. Таким образом, последние служат
промежуточной сущ\-ностью между сенсорно воспринимаемыми данными 
и~знанием, из\-вле\-ка\-емым из текс\-тов параллельных корпусов 
ЛГИС. Выполненная детализация 
базовых понятий информатики позволила сис\-те\-ма\-ти\-зи\-ро\-вать взаимные 
трансформации перечисленных объектов проектируемой сис\-те\-мы~[10].
  
  Используя результаты детализации и~сис\-те\-ма\-ти\-за\-ции, опишем шесть стадий 
по\-стро\-ения проб\-лем\-но-ори\-ен\-ти\-ро\-ван\-ной модели на основе 
спиральной модели (рис.~1). Отметим, что в~результате первых трех стадий 
получается абстрактная модель извлечения знания, которая, как и~спиральная 
модель, не зависит от решаемой проб\-ле\-мы и~предметной об\-ласти. Эта 
абстрактная модель ранее получила название \textit{циф\-ро\-вой спиральной 
модели}~\cite{13-zac}.
  

 
  
  Как было отмечено выше, в~спиральной модели, на осно\-ве которой строится 
ее цифровая версия, определены четыре вида знания,\linebreak  а~также четыре процесса 
их преобразования, выполняемых экспертами:  
со\-ци\-а\-ли\-за\-ция--экс\-тер\-на\-ли\-за\-ция--со\-че\-та\-ние--интер\-на\-ли\-за\-ция  
(Socialization--Externalization--Combination--Internalization, далее кратко~--- 
SECI). По определению этой модели, каждый виток спирали генерации знания 
включает в~себя по\-сле\-до\-ва\-тель\-ность сле\-ду\-ющих процессов:  
со\-ци\-а\-ли\-за\-ция\;$\to$\;экс\-тер\-на\-ли\-за\-ция\;$\to$\;со\-че\-та\-ние\;$\to$\;ин\-тер\-на\-ли\-за\-ция\;$\to$\;со\-ци\-а\-ли\-за\-ция (как 
\mbox{начало} сле\-ду\-юще\-го витка спирали).
  
  Согласно К.~Братиану~[15], спиральная модель ($=$\;модель SECI) не 
содержит время в~явном виде в~качестве переменной. Модель содержит время 
неявно, поскольку для любого преобразования оно требуется, но это 
абстрактное время без ка\-кой-ли\-бо воз\-мож\-ности его 
измерения\footnote{В~статье К.~Братиану для спиральной модели используется второе ее 
название, SECI model: ``The SECI model does not contain \textit{time} as an explicit variable. [$\ldots$] 
The model contains time implicitly since any transformation needs time, but it is a generic time without any 
possibility of measuring it''~\cite[p.~186]{15-zac}.}.
  
  
  \subsection{Цифровая спиральная модель: три~стадии построения} %3.1
  
  \textbf{Стадия~1.} В~классическую спиральную модель~[1] добавим 
моменты времени начала и~завершения каждого процесса витка спирали. Это 
даст воз\-мож\-ность фиксировать динамику извлечения знания. На этой же стадии 
построения вносится сле\-ду\-ющее изменение в~сис\-те\-му терминов: явное знание 
спиральной модели и~та информация, которая является пред\-став\-ле\-ни\-ем 
неявного знания (если эксперты смогли это сделать) и~которая может быть 
воспринята органами чувств, считаются \textit{синонимами}.

\begin{figure*} %fig2
 \vspace*{1pt}
  \begin{center}
 \mbox{%
 \epsfxsize=128mm 
\epsfbox{zac-2.eps}
 }
\end{center}
\vspace*{-9pt}
\Caption{Три среды и~две границы между ними (моменты времени начала и~завершения 
процессов не указаны)}
\vspace*{-4pt}
\end{figure*}
  
  \textbf{Стадия~2.} Повернем рис.~1 на 90~градусов по часовой стрелке 
и~добавим следующие пять понятий (рис.~2):
  \begin{itemize}
  \item \textit{ментальная среда}, которая включает в~себя неявные знания 
спиральной модели;
  \item \textit{сенсорно воспринимаемая среда} (крат\-ко~--- информационная 
среда), которая включает в~себя явные знания ($=$\;сен\-сор\-но 
воспринимаемая информация) спиральной модели;
  \item \textit{цифровая среда} компьютерных кодов;
 
  \item \textit{верхняя граница}, на которой происходит преобразование 
неявного знания в~сенсорно вос\-при\-ни\-ма\-емую информацию;
  \item \textit{нижняя граница}, на которой сенсорно вос\-при\-ни\-ма\-емые данные 
и~информация кодируются в~компьютерах и~происходит декодирование.
  \end{itemize}
  
  Эти две границы разделяют сущности второго этапа построения на три 
категории: \textit{ментальная} для неявного знания, \textit{информационная} 
для сенсорно вос\-при\-ни\-ма\-емо\-го явного знания и~\textit{цифровая} для 
компьютерных кодов. На этой же стадии определим положение четырех 
процессов преобразования видов знания спиральной модели сле\-ду\-ющим 
образом. Процессы интернализации и~экстернализации начинаются 
и~заканчиваются в~разных средах. Поэтому поместим их на границе между 
ментальной и~информационной средами (см.\ рис.~2).
  



  Процесс социализации разместим в~ментальной среде, так как его начальный и~конечный виды знания относятся к~ней. По этой же причине процесс 
сочетания разместим в~информационной среде. Такое размещение процессов 
в~этих средах соответствует именно начальному и~конечному видам знания без 
учета видов знания, которые служат \textit{промежуточными} в~этих 
процессах (см.\ рис.~2). 
  
\textbf{Стадия~3.} На этой стадии спиральная модель объединяется 
 с~мо\-делью информационной сис\-те\-мы хранения и~обработки циф\-ро\-вых данных как потенциальных 
источников нового знания, и~в~результате получается цифровая спиральная модель. На этой же стадии 
постулируется, что в~общем случае эта сис\-те\-ма включает сле\-ду\-ющие компоненты. Во-пер\-вых, она должна 
содержать циф\-ро\-вые потенциальные источники нового знания, которое извлекается экспертами в~процессе 
их по\-сле\-ду\-юще\-го семантического анализа. Во-вто\-рых, она имеет в~своем составе базу индивидуальных 
знаний (БИЗ), которая заполняется экспертами персонально, и~базу коллективных знаний (БКЗ)\footnote{Базы индивидуальных и~коллективных знаний 
планируется разделять только на концептуальном и~логическом этапах проектирования 
ЛГИС. Физически это будет одна база с~разделением в~ней 
индивидуального и~коллективного знания.}, которая заполняется экспертами в~результате согласования 
концептов их индивидуального извлеченного знания (рис.~3).



  
  В ЛГИС фиксируется состав 
экспертов, согласовавших свои концепты. Другими словами, для любого 
коллективного концепта известен перечень договорившихся между собой 
экспертов о его дефиниции. И,~в-треть\-их, она \textit{в~общем случае} 
включает базу искусственных нейронных сетей (БИНС), для обуче\-ния 
которых процессам извлечения нового знания из данных большого объема 
используется БКЗ (см.\ рис.~3). Однако не\-об\-хо\-ди\-мость использования БИНС 
в~информационной  сис\-те\-ме хранения и~обработки циф\-ро\-вых данных 
решается на по\-сле\-ду\-ющих стадиях построения модели исходя из конкретной 
ре\-ша\-емой задачи.
  
  На рис.~3 показана цифровая спиральная модель, полученная после 
завершения трех стадий ее по\-стро\-ения на основе ее классической версии~[1, 2]. 
У~этих двух спиральных моделей (классической и~цифровой) есть три общих 
свойства. Во-пер\-вых, они практически не зависят от предметной области, 
в~которой необходимо генерировать новые знания. Во-вто\-рых, в~них 
различается индивидуальное и~коллективное знание. И,~в-треть\-их, эти\linebreak\vspace*{-12pt}

\pagebreak

\end{multicols}

  \begin{figure*} %fig3
   \vspace*{1pt}
  \begin{center}
 \mbox{%
 \epsfxsize=128mm 
\epsfbox{zac-3.eps}
 }
\end{center}
\vspace*{-9pt}
  \Caption{Цифровая спиральная модель}
  \vspace*{-6pt}
  \end{figure*}

\begin{multicols}{2}

\noindent
 модели 
имеют в~своем со\-ста\-ве процесс социализации индивидуального знания, 
сформированного одним экспертом, т.\,е.\ генерацию коллективного знания 
группой экспертов, а~также процесс экстернализации неявного знания.

\vspace*{-6pt}
  
  \subsection{Проблемно-ориентированная модель: три стадии 
построения} %3.2

\vspace*{-3pt}
  
  Основой проектирования ЛГИС
служит проб\-лем\-но-ори\-ен\-ти\-ро\-ван\-ная модель, построение которой 
выполняется в~три стадии на основе циф\-ро\-вой спиральной модели (с~чет\-вер\-той 
по шес\-тую).
  
  \textbf{Стадия~4.} Цифровая спиральная модель предполагает наличие 
цифровых данных как потенциальных источников нового знания, необходимых 
для решения по\-став\-лен\-ной задачи, но не содержит их конкретного описания. 
Построение проб\-лем\-но-ори\-ен\-ти\-ро\-ван\-ной модели начинается на этой стадии 
с~описания конкретного массива циф\-ро\-вых данных, вклю\-ча\-юще\-го 
ис\-сле\-ду\-емые языков$\acute{\mbox{ы}}$е единицы и~их контексты. 
В~проектируемой сис\-те\-ме используется массив не\-мец\-ко-рус\-ских 
параллельных текс\-тов из Национального корпуса русского языка, который 
в~на\-сто\-ящее время служит одним из наиболее полных и~регулярно 
по\-пол\-ня\-емых источников нового языков$\acute{\mbox{о}}$го знания для 
решаемой проб\-ле\-мы интеграции.
  
  На этой же стадии в~модель извлечения знания в~ЛГИС добавляются немецкий и~русский языки, 
используемые в~проектируемой сис\-те\-ме. В~по\-сле\-ду\-ющих версиях  
сис\-те\-мы возможно увеличение чис\-ла или изменение набора ис\-поль\-зу\-емых 
языков.
  
  В ЛГИС новое знание извлекается только экспертами. Поэтому для 
построения проб\-лем\-но-ори\-ен\-ти\-ро\-ван\-ной модели нет не\-об\-хо\-ди\-мости 
в~БИНС, которая не используется в~ЛГИС. На этой же стадии фиксируются те 
процессы модели, в~которых используются вербальные знаковые сис\-те\-мы 
(отмечены зеленым цветом на рис.~4).

\smallskip
  
  \textbf{Стадия~5.} Любая компьютерная сис\-те\-ма предполагает 
использование таб\-лиц кодирования символов. Поэтому проблемно-ори\-ен\-ти\-ро\-ван\-ная 
модель долж\-на включать в~себя два сле\-ду\-ющих процесса: 
оциф\-ров\-ку для кодирования и~визуализацию для декодирования (см.\ рис.~4). 
До\-бав\-ле\-ние этих процессов происходит на пятой стадии ее по\-стро\-ения.
  
\smallskip

  \textbf{Стадия~6.} Чтобы извлекать новые знания из\linebreak текс\-то\-вых данных,  
проб\-лем\-но-ори\-ен\-ти\-ро\-ван\-ная модель должна включать в~себя еще 
четыре процесса: \textit{поиск} кон\-текс\-тов ис\-сле\-ду\-емой 
язы\-ко\-в$\acute{\mbox{о}}$й еди-\linebreak\vspace*{-12pt}

\pagebreak

\end{multicols}

\begin{figure*} %fig4
 \vspace*{1pt}
  \begin{center}
 \mbox{%
 \epsfxsize=152.859mm 
\epsfbox{zac-4.eps}
 }
\end{center}
\vspace*{-9pt}
\Caption{Проблемно-ориентированная модель извлечения знания в~ЛГИС}
  \vspace*{-3pt}
\end{figure*}

\begin{multicols}{2}

\noindent
 ницы в~циф\-ро\-вых данных параллельного 
корпуса как потенциальных источников нового знания (процесс поиска 
обозначен треугольником с~бук\-вой~R на рис.~4), \textit{визуализация} 
цифровых данных (т.\,е.\ декодирование текс\-то\-вых данных параллельного 
корпуса, которые хранятся в~ЛГИС в~циф\-ро\-вой форме, что обозначено 
прямоугольником со словами <<визуализация данных>>), 
\textit{концептуализация} (после визуализации цифровых данных) 
и~\textit{определение новизны} концепта, т.\,е.\ извлеченного значения 
исследуемой языков$\acute{\mbox{о}}$й единицы, сформированного 
экспертом в~процессе семантического анализа найден\-ных циф\-ро\-вых данных 
(процесс определения новизны обозначен ромбом с~бук\-вой~N 
и~вопросительным знаком на рис.~4). Отметим, что процесс визуализации 
циф\-ро\-вых данных отличается от двух смеж\-ных процессов визуализации: 
первый визуализирует текс\-то\-вые данные параллельного корпуса, а~смежны\-е 
с~ним процессы визуализации декодируют информацию из БИЗ 
и~БКЗ.
  
  На двух последних процессах остановимся по\-дроб\-нее. После визуализации 
циф\-ро\-вых данных каж\-до\-го текс\-то\-во\-го фрагмента с~контекстом ис\-сле\-ду\-емой 
языков$\acute{\mbox{о}}$й единицы, найденного в~корпусе, эксперт 
анализирует его, чтобы определить, с~каким значением этой единицы (из тех, 
что есть в~электронном словаре) соотнести ее упо\-треб\-ле\-ние в~ана\-ли\-зи\-ру\-емом 
контексте. Если в~словаре уже есть такое значение, то оно не признается новым 
для словаря (см.\ скруг\-лен\-ный прямоугольник со словом <<Стоп>> на рис.~4, 
обозначающий завершение обработки только этого фрагмента, а~не всего 
процесса извлечения знания).
  
  Эксперт может обнаружить в~текстовом фрагменте новое значение 
исследуемой языков$\acute{\mbox{о}}$й единицы, которого нет в~словаре. 
Если обнаружено новое значение (см.\ слово <<да>> на рис.~4), то сначала 
эксперт формирует его дефиницию, которая вместе с~фрагментом передается на 
вход процесса\linebreak интернализации. Одновременно она оциф\-ро\-вы\-ва\-ет\-ся 
и~записывается этим экспертом в~базу индивидуальных знаний ЛГИС, 
а~дефиниция, согласованная в~процессе обсуждения нового значения, 
\mbox{оцифровывается} и~записывается в~БКЗ и~электронный 
словарь ЛГИС. При этом сохраняется связь дефиниции нового значения 
ис\-сле\-ду\-емой языков$\acute{\mbox{о}}$й единицы с~тем текстовым фрагментом 
параллельного корпуса, из которого оно было извлечено. Связи по значениям 
между словарем, БКЗ и~параллельным корпусом 
обозначены на рис.~4 двойными серыми стрелками. Связь процесса 
определения новизны со словарем не показана, чтобы не усложнять рис.~4. 
Отметим, что от процесса интернализации до процесса сочетания 
включительно четыре процесса проб\-лем\-но-ори\-ен\-ти\-ро\-ван\-ной модели 
совпадают с~классической спиральной моделью~[1, 2]. По этой причине эти 
четыре процесса в~статье не рас\-смат\-ри\-ва\-ются.
  
  Процессы определения новизны и~концептуализации выполняются на 
границе между ментальной и~информационной средами с~использованием 
знаковых сис\-тем (отмечены в~этих процессах зеленым цветом). Эти процессы 
помещены в~информационную среду. Такое размещение именно в~этой среде 
обусловлено тем, что к~ней относятся начальный и~конечный объекты этих 
процессов.
  
  Таким образом, добавление четырех процессов (поиск, визуализация 
цифровых данных, концептуализация и~определение новизны) становится 
шес\-той и~завершающей стадией по\-стро\-ения  
проб\-лем\-но-ори\-ен\-ти\-ро\-ван\-ной модели.

\vspace*{-2pt}
  
\section{Заключение}

\vspace*{-2pt}

  В результате первых трех стадий построения получается довольно 
абстрактная модель, которая, как и~классическая спиральная модель, не зависит 
от решаемой проб\-ле\-мы и~предметной об\-ласти, что дает возможность 
отнести ее к~тео\-ре\-ти\-че\-ско\-му яд\-ру информатики. В~результате 
 по\-сле\-ду\-ющих трех стадий получается  
проб\-лем\-но-ори\-ен\-ти\-ро\-ван\-ная модель, нацеленная на решение 
конкретной задачи.
  
  В статье такой задачей выступает проектирование ЛГИС для решения 
актуальной проб\-ле\-мы интеграции электронных двуязычных словарей 
и~параллельных корпусов. Пред\-ла\-га\-емый подход к~\mbox{интеграции} словаря 
и~корпуса условно обозначен в~модели на рис.~4 двойными серыми стрелками. 
Словарь и~корпус связаны не напрямую, а~через БКЗ. 
Если словарь содержит только по\-след\-нюю версию дефиниции значения 
ис\-сле\-ду\-емой языков$\acute{\mbox{о}}$й единицы, то в~базе хранятся еще 
и~предыду\-щие варианты описания ее значения, что дает воз\-мож\-ность 
исследовать динамику лингвистического знания~[16].
  
  Интеграция электронных двуязычных словарей и~параллельных корпусов 
имеет большое значение для развития современной лингвистики. Для 
современного со\-сто\-яния исследований характерен переход от моделей, 
ориентированных на общие правила, ра\-бо\-та\-ющие в~<<идеальных условиях>>, 
к~учету и~объяснению всего многообразия способов реального употреб\-ле\-ния 
языка (empirical turn). Отсюда следует, что и~словарные описания, традиционно 
игнорирующие <<маргинальные случаи>>, перестраиваются на пред\-став\-ле\-ние 
лексического материала, полученное на основе подходов, основанных на 
реальном употреб\-ле\-нии языка (usage-based approaches). С~этой точки зрения 
ЛГИС, ин\-тег\-ри\-ру\-ющая словарные описания с~корпусными данными,~--- это 
новый шаг в~выработке современных способов пред\-став\-ле\-ния знания о~языке. 
Особенно новым представляется учет мно\-го\-знач\-ности опи\-сы\-ва\-емых 
языков$\acute{\mbox{ы}}$х единиц и~выход за пределы ка\-ко\-го-то одного 
конкретного языка. Со\-по\-став\-ле\-ние данных двух разных языков (в~данном 
случае немецкого и~русского) как на уровне словарного пред\-став\-ле\-ния, так и~на 
уровне корпусного материала~--- существенный шаг вперед по срав\-не\-нию 
с~име\-ющи\-ми\-ся на сегодня отечественными и~зарубежными 
исследованиями в~этой об\-ласти.

\vspace*{-7pt}
  
{\small\frenchspacing
 { %\baselineskip=11.5pt
 %\addcontentsline{toc}{section}{References}
 \begin{thebibliography}{99}
 
 \vspace*{-2pt}
 
\bibitem{1-zac}
\Au{Nonaka I.} The knowledge-creating company~// Harvard Bus. Rev., 1991. Vol.~69. 
No.\,6. P.~96--104.
\bibitem{2-zac}
\Au{Нонака И., Такеучи Х.} Компания~--- создатель знания~/ Пер. c англ.~--- М.:  
Олимп-биз\-нес, 2003. 384~с. (\Au{Nonaka~I., Takeuchi~H.} The knowledge-creating 
company.~--- Oxford, NY, USA: Oxford University Press, 1995. 284~p.)
\bibitem{3-zac}
\Au{Добровольский Д.\,О., Зализняк Анна~А.} Немецкие конструкции с~модальными 
глаголами и~их русские соответствия: проект надкорпусной базы данных~//\linebreak Компьютерная 
лингвистика и~интеллектуальные технологии: По мат-лам Междунар. конф. <<Диалог>>.~--- 
М.: РГГУ, 2018. Вып.~17(24). С.~172--184.
\bibitem{4-zac}
\Au{Зацман И.\,М.} Проб\-лем\-но-ори\-ен\-ти\-ро\-ван\-ная актуализация словарных ста\-тей 
двуязычных словарей и~медицинской терминологии: сопоставительный анализ~// 
Информатика и~её применения, 2021. Т.~15. Вып.~1. С.~94--101. doi: 10.14357/19922264210113. 
EDN: \mbox{DMCMSK}.
\bibitem{5-zac}
\Au{Klein W., Geyken~A.} Das Digitale W$\ddot{\mbox{o}}$rterbuch der Deutschen Sprache 
(DWDS)~// Lexicographica, 2010. Vol.~26. No.\,2010. P.~79--96. doi: 
10.1515/ 9783110223231.1.79.
\bibitem{6-zac}
\Au{Добровольский Д.\,О.} Корпус параллельных текс\-тов и~сопоставительная лексикология~// 
Труды Института русского языка им.\ В.\,В.~Виноградова, 2015. Вып.~6. С.~413--449. EDN: 
VJQBHP.
\bibitem{7-zac}
\Au{Гончаров А.\,А.} Аннотирование параллельных кор\-пу\-сов: подходы и~на\-прав\-ле\-ния 
развития~// Информатика и~её применения, 2023. Т.~17.  Вып.~4. С.~81--87. doi: 
10.14357/19922264230411. EDN: GDKDOZ.
\bibitem{8-zac}
\Au{Гончаров А.\,А., Зацман~И.\,М., Кружков~М.\, Г}. Эволюция классификаций 
в~надкорпусных базах дан-\linebreak\vspace*{-12pt}

\pagebreak

\noindent
ных~// Информатика и~её применения, 2020. Т.~14. Вып.~4.  
С.~108--116. doi: 10.14357/19922264200415. EDN: \mbox{GKWBZT}.
\bibitem{9-zac}
\Au{Гончаров А.\,А., Зацман~И.\,М., Кружков~М.\,Г}. Пред\-став\-ле\-ние новых 
лексикографических знаний в~динамических классификационных сис\-те\-мах~// Информатика 
и~её применения, 2021. Т.~15. Вып.~1. С.~86--93. doi: 10.14357/19922264210112. EDN: OPEFXW.
\bibitem{10-zac}
\Au{Зацман И.\,М.} Трансформации объектов первого и~второго порядка 
в~лексикографической информационной сис\-те\-ме~// Информатика и~её применения, 2024. 
Т.~18. Вып.~2. С.~82--91. doi: 10.14357/19922264240211. EDN: VZTGVV.
\bibitem{11-zac}
\Au{Ackoff R.} From data to wisdom~// J.~Applied Systems Analysis, 1989. Vol.~16. P.~3--9.
\bibitem{12-zac}
\Au{Зацман И.\,М.} Данные, информация и~знание в~научной парадигме информатики~// 
Информатика и~её применения, 2023. Т.~17. Вып.~1. С.~116--125. doi: 
10.14357/19922264230115. EDN: CWIROJ.

\bibitem{14-zac} %13
\Au{Wierzbicki A., Nakamori~Y.} Basic dimensions of creative space~// Creative space: Models of 
creative processes for knowledge civilization age~/ Eds. A.~Wierzbicki, Y.~Nakamori.~--- 
Heidelberg: Springer Verlag, 2006.\linebreak P.~59--90.
\bibitem{13-zac} %14
\Au{Zatsman I}. Digital spiral model of knowledge creation and encoding its dynamics~// 18th 
Forum (International) on Knowledge Asset Dynamics Proceedings.~--- Matera, Italy: Arts for 
Business Institute, 2023. P.~581--596. {\sf  
https://www.researchgate.net/publication/371303696\_ Digital\_Spiral\_Model\_of\_Knowledge\_Creation\_and\_ Encoding\_its\_Dynamics}.
\bibitem{15-zac}
\Au{Bratianu C.} A~strategic view on the knowledge dynamics models used in knowledge 
management~// 20th European Conference on Knowledge Management Proceedings.~---  Reading, 
U.K.: Academic Publishing International Ltd., 2019. Vol.~1. P.~185--192.
\bibitem{16-zac}
\Au{Гончаров А.\,А., Зацман~И.\,М., Кружков~М.\,Г., Лощилова~Е.\,Ю.} Отражение 
эволюции лексикографических знаний в~динамических классификационных сис\-те\-мах~// 
Информатика и~её применения, 2021. Т.~15. Вып.~4. С.~41--49. doi: 10.14357/19922264210406. 
EDN: MGORMY.


\end{thebibliography}

 }
 }

\end{multicols}

\vspace*{-6pt}

\hfill{\small\textit{Поступила в~редакцию 13.07.24}}

\vspace*{6pt}

%\pagebreak

%\newpage

%\vspace*{-28pt}

\hrule

\vspace*{2pt}

\hrule


\def\tit{A MODEL FOR~EXTRACTING KNOWLEDGE FROM PARALLEL TEXTS OF~A~LEXICOGRAPHIC 
INFORMATION SYSTEM}


\def\titkol{A model for~extracting knowledge from parallel texts of~a~lexicographic 
information system}


\def\aut{D.\,O.~Dobrovol'skij$^{1,2,3}$ and~I.\,M.~Zatsman$^3$}

\def\autkol{D.\,O.~Dobrovol'skij and~I.\,M.~Zatsman}

\titel{\tit}{\aut}{\autkol}{\titkol}

\vspace*{-8pt}


\noindent
$^1$Vinogradov Russian Language Institute of the Russian Academy of Sciences, 18/2~Volkhonka Str., 
Moscow\linebreak
$\hphantom{^1}$119019, Russian Federation

\noindent
$^2$Institute of Linguistics of the Russian Academy of Sciences, 1-1~Bolshoy Kislovsky Lane, Moscow 
125009,\linebreak
$\hphantom{^1}$Russian Federation

\noindent
$^3$Federal Research Center ``Computer Science and Control'' of the Russian Academy of Sciences,  
44-2~Vavilov\linebreak
$\hphantom{^1}$Str., Moscow 119333, Russian Federation

\def\leftfootline{\small{\textbf{\thepage}
\hfill INFORMATIKA I EE PRIMENENIYA~--- INFORMATICS AND
APPLICATIONS\ \ \ 2024\ \ \ volume~18\ \ \ issue\ 3}
}%
 \def\rightfootline{\small{INFORMATIKA I EE PRIMENENIYA~---
INFORMATICS AND APPLICATIONS\ \ \ 2024\ \ \ volume~18\ \ \ issue\ 3
\hfill \textbf{\thepage}}}

\vspace*{4pt}



\Abste{The problem-oriented model of extracting linguistic knowledge from parallel texts is considered 
to be a~key theoretical component for creating a lexicographic information system that provides integration 
of electronic bilingual dictionaries and parallel corpora. The proposed approach to solving the integration 
problem takes into account the emergence of new meanings of words and phrasemes which is due to the 
acquisition of new knowledge by experts who discover these meanings as a~result of semantic analysis of 
regularly updated corpus data. The proposed model describes the human--computer interaction including the 
search for fragments of parallel texts as potential sources of new linguistic knowledge, its extracting by 
experts from texts, and representation in the lexicographic information system. The basis for building the 
problem-oriented model is the spiral model of knowledge generation which was proposed by Ikujiro Nonaka 
in 1991. The purpose of the paper is to describe the stages of building the model for discovering linguistic 
knowledge used in the lexicographic information system design.}

\KWE{lexicographic information system; parallel texts; spiral model of knowledge generation; problem-
oriented model}



\DOI{10.14357/19922264240312}{NDNPCF}

\vspace*{-12pt}


     \Ack
     
     \vspace*{-3pt}
     
      \noindent
            The study was funded by the Russian Science Foundation, project No.\,24-18-00155, {\sf 
https://rscf.ru/project/24-18-00155}. The research was carried out using the infrastructure of the Shared 
Research Facilities ``High Performance Computing and Big Data'' (CKP ``Informatics'') of FRC CSC RAS 
(Moscow).
     


  \begin{multicols}{2}

\renewcommand{\bibname}{\protect\rmfamily References}
%\renewcommand{\bibname}{\large\protect\rm References}

{\small\frenchspacing
 {%\baselineskip=10.8pt
 \addcontentsline{toc}{section}{References}
 \begin{thebibliography}{99} 
\bibitem{1-zac-1}
\Aue{Nonaka, I.} 1991. The knowledge-creating company. \textit{Harvard Bus. Rev.}  69(6):96--104.
\bibitem{2-zac-1}
\Aue{Nonaka, I., and H.~Takeuchi.} 1995. \textit{The knowledge-creating company}. Oxford, NY: 
Oxford University Press. 284~p.
\bibitem{3-zac-1}
\Aue{Dobrovol'skij, D.\,O., and Anna A.~Zalizniak.} 2018. Nemetskie konstruktsii s~modal'nymi 
glagolami i~ikh russkie sootvetstviya: proekt nadkorpusnoy bazy dannykh [German constructions with 
modal verbs and their Russian correlates: A~supracorpora database project]. \textit{Computer Linguistic 
and Intellectual Technologies: Conference (International) ``Dialog'' Proceedings}. Moscow: Russian State University for the
Humanities.   17(24):172--184.
\bibitem{4-zac-1}
\Aue{Zatsman, I.\,M.} 2021. Problemno-oriyentirovannaya aktualizatsiya slovarnykh statey 
dvuyazychnykh slovarey i~me\-di\-tsin\-skoy terminologii: sopostavitel'nyy analiz [Problem-oriented 
updating of dictionary entries of bilingual \mbox{dictionaries} and medical terminology: Comparative analysis]. 
\textit{Informatika i~ee Primeneniya~--- Inform. \mbox{Appl.}} 15(1):94--101. doi: 10.14357/19922264210113. 
EDN: DMCMSK.
\bibitem{5-zac-1}
\Aue{Klein, W., and A.~Geyken.} 2010. Das Digitale W$\ddot{\mbox{o}}$rterbuch der Deutschen Sprache 
(DWDS).  \textit{Lexicographica} 26(2010):79--96. doi: 10.1515/9783110223231.1.79.
\bibitem{6-zac-1}
\Aue{Dobrovol'skij, D.\,O.} 2015. Korpus parallel'nykh tekstov i~sopostavitel'naya leksikologiya [The 
corpus of parallel texts and contrastive lexicology]. \textit{Trudy Instituta russkogo yazyka im.\ 
V.\,V.~Vinogradova} [Proceedings of the V.\,V.~Vinogradov Russian Language Institute] 6:413--449. 
EDN: VJQBHP.
\bibitem{7-zac-1}
\Aue{Goncharov, A.\,A.} 2023. Annotirovanie parallel'nykh korpusov: podkhody i~napravleniya 
razvitiya [Parallel corpus annotation: Approaches and directions for development]. \textit{Informatika 
i~ee Primeneniya~--- Inform. Appl.} 17(4):81--87. doi: 10.14357/19922264230411. EDN: GDKDOZ.
\bibitem{8-zac-1}
\Aue{Goncharov, A.\,A., I.\,M.~Zatsman, and M.\,G.~Kruzhkov.} 2020. Evolyutsiya klassifikatsiy 
v~nadkorpusnykh ba\-zakh dan\-nykh [Evolution of classifications in supracorpora databases]. 
\textit{Informatika i~ee Primeneniya~--- Inform. \mbox{Appl}.} 14(4):108--116. doi: 
10.14357/19922264200415. EDN: GKWBZT.
\bibitem{9-zac-1}
\Aue{Goncharov, A.\,A., I.\,M.~Zatsman, and M.\,G.~Kruzhkov.} 2021. Predstavlenie novykh 
leksikograficheskikh znaniy v~dinamicheskikh klassifikatsionnykh sis\-te\-makh [Representation of new 
lexicographical knowledge in dynamic classification systems]. \textit{Informatika i~ee Primeneniya~--- 
Inform. Appl.} 15(1):86--93. doi: 10.14357/19922264210112. EDN: OPEFXW.
\bibitem{10-zac-1}
\Aue{Zatsman, I.\,M.} 2024. Transformatsii ob''ektov pervogo i~vtorogo poryadka 
v~leksikograficheskoy informatsionnoy sis\-te\-me [Object transformations of the first and second order 
in a lexicographic information system]. \textit{Informatika i~ee Primeneniya~--- Inform. Appl}. 
18(2):82--91. doi: 10.14357/19922264240211. EDN: VZTGVV.
\bibitem{11-zac-1}
\Aue{Ackoff, R.} 1989. From data to wisdom. \textit{J.~Applied Systems Analysis} 16(1):3--9.
\bibitem{12-zac-1}
\Aue{Zatsman, I.\,M.} 2023. Dannye, informatsiya i~znanie v~na\-uch\-noy paradigme informatiki [On the 
scientific paradigm of informatics: Data, information, and knowledge]. \textit{Informatika i~ee 
Primeneniya~--- Inform. Appl.} 17(1):116--125. doi: 10.14357/19922264230115. EDN: CWIROJ.

\bibitem{14-zac-1}
\Aue{Wierzbicki, A.\,P., and Y.~Nakamori.} 2006. Basic dimensions of creative space. \textit{Creative 
space: Models of creative processes for knowledge civilization age}. Eds. A.\,P.~Wierzbicki and 
Y.~Nakamori. Berlin: Springer Verlag. 59--90.

\bibitem{13-zac-1}
\Aue{Zatsman, I.} 2023. Digital spiral model of knowledge creation and encoding its dynamics. 
\textit{18th Forum (International) on Knowledge Asset Dynamics Proceedings}. Matera, Italy: Arts for 
Business Institute. 581--596. Available at: {\sf  
https://www.researchgate.net/publication/371303696\_ Digital\_Spiral\_Model\_of\_Knowledge\_Creation\_and\_ 
Encoding\_its\_Dynamics} (accessed 
July~30, 2024).

\bibitem{15-zac-1}
\Aue{Bratianu, C.} 2019. A~strategic view on the knowledge dynamics models used in knowledge 
management. \textit{20th European Conference on Knowledge Management Proceedings}. Reading, 
U.K.: Academic Publishing International Ltd. 1:185--192.
\bibitem{16-zac-1}
\Aue{Goncharov, A.\,A., I.\,M.~Zatsman, M.\,G.~Kruzhkov, and E.\,Yu.~Loshchilova}. 2021. 
Otrazhenie evolyutsii leksikograficheskikh znaniy v~dinamicheskikh klassifikatsionnykh sistemakh 
[Capturing evolution of lexicographic knowledge in dynamic classification systems]. \textit{Informatika 
i~ee Primeneniya~--- Inform. Appl.} 15(4):41--49. doi: 10.14357/19922264210406. EDN: MGORMY.

\end{thebibliography}

 }
 }

\end{multicols}

\vspace*{-6pt}

\hfill{\small\textit{Received July 13, 2024}} 

\vspace*{-18pt}


\Contr

\vspace*{-3pt}

\noindent
\textbf{Dobrovol'skij Dmitrij O.} (b.\ 1953)~--- Doctor of Science in philology, principal 
scientist, Vinogradov Russian Language Institute of the Russian Academy of Sciences, 
18/2~Volkhonka Str., Moscow 119019, Russian Federation; principal scientist, Institute of 
Linguistics of the Russian Academy of Sciences, 1-1 Bolshoy Kislovsky Lane, Moscow 125009, 
Russian Federation; principal scientist, Federal Research Center ``Computer Science and Control'' of 
the Russian Academy of Sciences, 44-2~Vavilov Str., Moscow 119333, Russian Federation; 
\mbox{dm-dbrv@yandex.ru}

\vspace*{3pt}

\noindent
\textbf{Zatsman Igor M.} (b.\ 1952)~---  Doctor of Science in technology, head of department, 
Federal Research Center ``Computer Science and Control'' of the Russian Academy of Sciences,  
44-2~Vavilov Str., Moscow 119333, Russian Federation; \mbox{izatsman@yandex.ru}



\label{end\stat}

\renewcommand{\bibname}{\protect\rm Литература} 