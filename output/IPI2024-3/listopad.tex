\def\stat{listopad}

\def\tit{БАЗОВАЯ АРХИТЕКТУРА РЕФЛЕКСИВНО-АКТИВНЫХ СИСТЕМ ИСКУССТВЕННЫХ 
ГЕТЕРОГЕННЫХ ИНТЕЛЛЕКТУАЛЬНЫХ АГЕНТОВ$^*$}

\def\titkol{Базовая архитектура рефлексивно-активных систем искусственных 
гетерогенных интеллектуальных агентов}

\def\aut{С.\,В.~Листопад$^1$}

\def\autkol{С.\,В.~Листопад}

\titel{\tit}{\aut}{\autkol}{\titkol}

\index{Листопад С.\,В.}
\index{Listopad S.\,V.}


{\renewcommand{\thefootnote}{\fnsymbol{footnote}} \footnotetext[1]
{Исследование выполнено за счет гранта Российского научного фонда №\,23-21-00218, {\sf  
https://rscf.ru/project/23-21-00218}.}}


\renewcommand{\thefootnote}{\arabic{footnote}}
\footnotetext[1]{Федеральный исследовательский центр <<Информатика и~управление>> Российской академии наук, \mbox{ser-list-post@yandex.ru}}

  \vspace*{-10pt}
  
  
  \Abst{Статья посвящена разработке базовой архитектуры реф\-лек\-сив\-но-ак\-тив\-ных 
сис\-тем искусственных гетерогенных интеллектуальных агентов (\mbox{РАСИГИА}), включающей типовую 
функциональную структуру сис\-те\-мы и~обобщенную архитектуру агентов. Благодаря 
сочетанию в~сис\-те\-ме разнородных агентов осуществляется гиб\-ри\-ди\-за\-ция различных методов 
формального представления сис\-те\-мы и~знаний специалистов, обеспечивается учет 
инструментальной и~функциональной не\-од\-но\-род\-ности воз\-ни\-ка\-ющих проб\-лем. 
Рефлексивные возможности агентов поз\-во\-ля\-ют им моделировать других агентов, самих себя и~сис\-те\-му 
в~целом, снижая ин\-тен\-сив\-ность конфликтов и~дли\-тель\-ность вы\-стра\-и\-ва\-ния 
взаимодействия при изменении со\-ста\-ва сис\-те\-мы. В~результате становится воз\-мож\-ной 
самоорганизация сис\-те\-мы в~сильном смысле, в~ходе которой для каж\-дой новой практической 
проб\-ле\-мы вырабатывается релевантный ей метод решения.}
  
  \KW{рефлексия; рефлексивно-активная сис\-те\-ма искусственных гетерогенных 
интеллектуальных агентов; гиб\-рид\-ная интеллектуальная многоагентная сис\-те\-ма; 
коллектив специалистов}

\DOI{10.14357/19922264240311}{UNTQBV}

  

\vspace*{9pt}

 

\vskip 10pt plus 9pt minus 6pt

 

\thispagestyle{headings}

 

\begin{multicols}{2}

 

\label{st\stat}

 

\section{Введение}

 

%\vspace*{1pt}

 

  Как показано в~[1], традиционная субъ\-ект-объ\-ект\-ная парадигма 
нерелевантна процессам управле\-ния сложными организационными 
структурами, например логистическими цент\-ра\-ми, энерго\-рас\-пре\-де\-ли\-тель\-ны\-ми
или медицинскими организациями. Такие структуры~--- это не \mbox{пассивные} 
объекты, а~развивающиеся и~эво\-лю\-ци\-о\-ни\-ру\-ющие \mbox{сети} взаимодействующих 
субъектов, которые характеризуются це\-ле\-уст\-рем\-лен\-ностью (ак\-тив\-ностью), 
реф\-лек\-сив\-ностью, ком\-му\-ни\-ка\-тив\-ностью, со\-ци\-аль\-ностью, спо\-соб\-ностью 
к~развитию и~т.\,д. При управ\-ле\-нии и~автоматизированном решении 
проб\-лем, возникающих в~этих структурах, необходимо рассматривать их как 
са\-мо\-раз\-ви\-ва\-ющи\-еся реф\-лек\-сив\-но-ак\-тив\-ные среды~[1]. Для этого в~[2] в~рамках 
муль\-ти\-агент\-ной па\-ра\-диг\-мы~[3--5] предложена концепция 
\mbox{РАСИГИА} как компьютерных мо\-делей 
коллективов, управ\-ля\-ющих слож\-ны\-ми организационными структурами. 
Агенты \mbox{РАСИГИА}~--- автономные программные сущности, 
ха\-рак\-те\-ри\-зу\-ющи\-еся активностью и~ре\-ак\-тив\-ностью, \mbox{способные} рас\-суж\-дать, 
взаимодействовать и~реф\-лек\-си\-ро\-вать.
{\looseness=1

}

  
  Основы математического моделирования реф\-лек\-сив\-ных процессов 
  и~управ\-ле\-ния заложены в~работах В.\,А.~Лефевра, который рассматривал 
реф\-лек\-сию как спо\-соб\-ность объ\-ек\-та-ис\-сле\-до\-ва\-те\-ля моделировать 
другие объекты и~самого себя, свои действия и~мысли~[6]. Новиков 
и~Чхар\-ти\-шви\-ли предложили концепцию равновесия в~рефлексивных 
играх~[7]. В~[8] предложен подход на основе\linebreak нечеткой логики для 
формализации реф\-лек\-сив\-ных процессов вра\-ча-экс\-пер\-та. В~[9] 
рассмат\-ривает\-ся концепция по\-стро\-ения виртуального интеллектуального 
аген\-та-ас\-сис\-тен\-та, реф\-лек\-сив\-но моделирующего своего пользователя. 
\mbox{Статья}[10] посвящена разработке механизмов са\-мо\-обуче\-ния автономных 
интеллектуальных роботов при по\-стро\-ении траектории движения в~проб\-лем\-ной 
среде с~препятствиями, неизвестными априори.
{ %\looseness=1

}
  
  В РАСИГИА рефлексивные способности агентов служат для сокращения 
дли\-тель\-ности переговорных процессов в~ходе выработки согласованного 
пред\-став\-ле\-ния о~проб\-лем\-ной об\-ласти, целях сис\-те\-мы и~порядке 
взаимодействия, что поз\-во\-ля\-ет реализовать принцип от\-кры\-тости (способности 
к~привлечению новых агентов из внеш\-ней среды) \mbox{РАСИГИА} и~ее 
самоорганизацию в~силь\-ном смыс\-ле~[11]. 

\begin{figure*}[b] %fig1
\vspace*{-2pt}
  \begin{center}
 \mbox{%
 \epsfxsize=162.026mm 
\epsfbox{lis-1.eps}
 }
\end{center}
\vspace*{-9pt}
\Caption{Типовая функциональная структура РАСИГИА: \textit{1}~--- отношения агентов (запросы 
информации, помощи в~решении задач, передача результатов их решения); \textit{2}~--- 
взаимодействие (получение сведений из модели, обновление модели)  агентов с~базовой 
онтологией;
\textit{3}~--- отношения агентов (запросы перевода сообщений с~одного языка кодирования 
сообщений на другой); \textit{4}~--- отношения агентов (запросы имен и~адресов агентов 
с~заданными возможностями); \textit{5}~--- отношения фасилитации, ини\-ци\-а\-ли\-зи\-ру\-ющие 
процессы согласования целей, моделей пред\-мет\-ной об\-ласти и~протоколы взаимодействия 
агентами под\-сис\-те\-мы решения проб\-лем; \textit{6}~--- отношения управ\-ле\-ния со\-ста\-вом  
аген\-тов-спе\-ци\-а\-лис\-тов; \textit{7}~--- отношения рефлексии агентов под\-сис\-те\-мы 
решения проблем; \textit{8}~--- отношения агентов с~внеш\-ней средой}
\end{figure*}


В~на\-сто\-ящей работе представлены 
типовая функциональная структура и~обобщенная архитектура агентов 
\mbox{РАСИГИА}, разработанные в~соответствии с~методологией~[12] 
и~формализованной моделью сис\-те\-мы~[2].
    
    \section{Типовая функциональная структура  
рефлексивно-активных систем искусственных гетерогенных 
интеллектуальных агентов}
  
  Типовая функциональная структура \mbox{РАСИГИА} (рис.~1) описывает 
подсистемы агентов и~их функционал, потоки информации и~отношения между 
ними, а~также их взаимодействие с~внеш\-ней средой, которая рассматривается 
как многоагентная сис\-те\-ма более высокого уровня. В~то же время каж\-дый из 
пред\-став\-лен\-ных агентов может быть \mbox{РАСИГИА} более низкого уровня. 
Пред\-ла\-га\-емая структура выступает в~качестве осно\-вы при разработке сис\-те\-мы 
для конкретной проб\-ле\-мы, в~ходе которой специфицируется чис\-ло и~со\-став 
агентов, не предопределенные в~типовой структуре. В~рас\-смат\-ри\-ва\-емой 
типовой функциональной структуре выделяются четыре под\-сис\-те\-мы: 
интерфейсная, технологическая, управ\-ле\-ния и~решения проб\-лем. 
  


  Интерфейсная подсистема~--- это посредник меж\-ду агентами системы и~ее 
пользователями, а~так\-же объектом управ\-ле\-ния. Чис\-ло агентов этой\linebreak под\-сис\-те\-мы 
может меняться и~определяется подходом разработчиков конкретной  
сис\-те\-мы к~распределению функционала между агентами и~спецификой 
проблемы или класса проб\-лем, для \mbox{решения} которых строится 
\mbox{РАСИГИА}. Интерфейсный агент запрашивает у~пользователя входные 
данные и~представляет результат решения проб\-ле\-мы. Кроме того, он может 
обеспечивать на\-строй\-ку сис\-те\-мы пользователем и~визуализировать 
происходящие в~ней процессы~[13]. Взаимодействие \mbox{РАСИГИА} 
с~объектом управ\-ле\-ния также осуществляется через агентов интерфейсной 
под\-сис\-те\-мы, которые получают информацию о~его со\-сто\-янии и~выдают 
управ\-ля\-ющие воздействия через программные ин\-тер\-фейсы.
  
  Элементы и~агенты технологической под\-сис\-те\-мы предоставляют служебные 
функции другим агентам сис\-те\-мы. Аген\-ты-пе\-ре\-вод\-чи\-ки включаются в~процесс 
передачи сообщений меж\-ду парой агентов, если они не поддерживают единый 
язык и~не могут общаться между собой напрямую. Агент-по\-сред\-ник 
предостав\-ля\-ет служ\-бу <<желтых страниц>>, т.\,е.\ выполняет сопоставление 
способностей и~имен агентов, об\-ла\-да\-ющих ими. Базовая онтология~--- 
технологический элемент \mbox{РАСИГИА}, обес\-пе\-чи\-ва\-ющий понимание 
агентами семантики сообщений друг друга в~рамках базовой коммуникации по 
согласованию собственных онтологий, по\-стро\-ен\-ных на ее основе, целей 
и~протокола решения проб\-лемы. 
  
  Агенты подсистемы управ\-ле\-ния обеспечивают эффективное взаимодействие 
остальных агентов \mbox{РАСИГИА} и~ее самоорганизацию. Агент контроля 
протокола отслеживает и~контролирует соблюдение\linebreak договоренностей агентов 
по выработке и~корректировке согласованного протокола решения проб\-ле\-мы. 
Агент-фа\-си\-ли\-та\-тор обеспечивает эффективную совместную работу 
агентов под\-сис\-те\-мы \mbox{решения} проб\-лем, в~част\-ности оценивает текущую 
ситуацию (определяет стадию коллективной работы, отношения между 
агентами, про\-яв\-ля\-ющи\-еся мак\-ро\-уров\-не\-вые процессы и~т.\,п.)\ в~ней, 
инициирует действия агентов, сти\-му\-ли\-ру\-ющие или  
спо\-соб\-ст\-ву\-ющие разрешению воз\-ни\-ка\-ющих меж\-ду ними 
конфликтов~[14]. Агент управ\-ле\-ния составом сис\-те\-мы привлекает 
в~под\-сис\-те\-му решения проб\-лем агентов из пула во внеш\-ней среде, моделируя 
методы подбора специалистов в~реальные коллективы~[15], а~также исключает 
агентов из со\-ста\-ва \mbox{РАСИГИА}, помещая их в~пул. Таким образом, он 
обеспечивает динамику со\-ста\-ва агентов, развитие \mbox{РАСИГИА} и~ее 
ре\-ле\-вант\-ность очередной проб\-леме.
  
  Подсистема решения проблем предназначена для компьютерного 
моделирования групповой работы специалистов за круглым столом над 
решением проб\-ле\-мы. Эта подсистема обеспечивает выполнение принципа 
необходимого разнообразия за счет моделирования рас\-суж\-де\-ний специалистов 
разных на\-прав\-ле\-ний и~использования различных\linebreak методов решения проб\-ле\-мы. 
Агент, при\-ни\-ма\-ющий решения, получив необходимую для решения\linebreak \mbox{проб\-ле\-мы} 
информацию от интерфейсного аген\-та, выполняет декомпозицию проб\-ле\-мы на 
под\-проб\-ле\-мы, распределяет их между аген\-та\-ми-спе\-ци\-а\-ли\-ста\-ми, 
собирает и~оценивает предложенные ими решения проб\-ле\-мы в~целом или ее 
частей, формирует результат работы сис\-те\-мы путем выбора одного из 
предложенных решений проб\-ле\-мы в~целом или интеграции решений  
под\-проб\-лем. Агент-спе\-ци\-а\-лист, моделируя рассуждения реального 
специалиста, решает под\-проб\-ле\-му или проб\-ле\-му целиком в~за\-ви\-си\-мости от 
поручений агента, при\-ни\-ма\-юще\-го решения. Агенты этой под\-сис\-те\-мы могут 
создаваться разными разработчиками, что потенциально приводит к~различиям 
и~противоречиям их онтологий и~целей. Однако благодаря рефлексивному 
моделированию рас\-суж\-де\-ний друг друга обеспечивается снижение 
ин\-тен\-сив\-ности конфликтов и~дли\-тель\-ности переговоров. В~этом смыс\-ле 
рефлексия агентов служит средством координации и~синхронизации, 
поз\-во\-ля\-ющим им для эффективного решения проб\-ле\-мы согласовывать 
собственные онтологии, цели и~варианты протокола решения проб\-ле\-мы по 
запросу аген\-та-фа\-си\-ли\-та\-тора.

  
  
\section{Обобщенная архитектура искусственного 
интеллектуального агента гетерогенной  
рефлексивно-активной системы}
  
  Архитектура агента~--- это схема, описывающая со\-став, структуру 
и~взаимосвязь функ\-ций-бло\-ков, ре\-а\-ли\-зу\-емых агентом, обес\-пе\-чи\-ва\-ющая 
выполнение им своего предназначения, в~част\-ности реализацию роли или 
множества ролей, для которых он проектируется~[2]. Для каждой  
функ\-ции-бло\-ка указывается метод или алгоритм, ре\-а\-ли\-зу\-ющий ее. Если 
таковые отсутствуют, они долж\-ны быть разработаны в~рамках сле\-ду\-ющей 
стадии методологии по\-стро\-ения \mbox{РАСИГИА}~\cite{12-lis}. Архитектура 
агента разрабатывается с~учетом формальной модели 
\mbox{РАСИГИА}~\cite{2-lis} и~самого агента, а~также функциональной 
структуры сис\-те\-мы (см.\ рис.~1).

\begin{figure*} %fig2
\vspace*{1pt}
  \begin{center}
 \mbox{%
 \epsfxsize=146.816mm 
\epsfbox{lis-2.eps}
 }
\end{center}
\vspace*{-9pt}
\Caption{Обобщенная архитектура искусственного интеллектуального агента гетерогенной 
рефлексивно-активной системы}
\end{figure*}
  
  В соответствии с~микроуровневой моделью~\cite{2-lis}, описывающей со\-став 
и~структуру искусственного интеллектуального агента гетерогенной  
реф\-лек\-сив\-но-ак\-тив\-ной сис\-те\-мы, минимально необходимое 
множество действий агента может быть пред\-став\-ле\-но сле\-ду\-ющим образом:
  \begin{equation*}
  \mathrm{ACT}^{\mathrm{ag}} =\mathrm{ACT}_{\mathrm{msg}}^{\mathrm{ag}}\cup 
  \mathrm{ACT}_{\mathrm{fnc}}^{\mathrm{ag}} \cup 
\mathrm{ACT}_{\mathrm{com}}^{\mathrm{ag}}\,,
  \end{equation*}
    где $\mathrm{ACT}_{\mathrm{msg}}^{\mathrm{ag}}$~--- множество стандартных для 
всех агентов служебных действий по получению и~передаче сообщений, их 
интерпретации и~со\-став\-ле-\linebreak нию; $\mathrm{ACT}_{\mathrm{fnc}}^{\mathrm{ag}}$~--- 
множество действий агента по выполнению своего непосредственного 
функционала,\linebreak например перевод сообщений с~одного языка\linebreak передачи 
сообщений на другой в~случае аген\-та-пе\-ре\-вод\-чи\-ка или отслеживание 
информации о~возможностях агентов аген\-том-по\-сред\-ни\-ком; 
$\mathrm{ACT}_{\mathrm{com}}^{\mathrm{ag}}$~--- множество допустимых 
последовательностей действий агента, по\-лу\-ча\-емых в~результате корректной 
интеграции элементарных действий.
  
  Обобщенная архитектура агента, реализующая рас\-смот\-рен\-ное множество 
действий, схематически пред\-став\-ле\-на на рис.~2. 
  
  Подсистема восприятия отслеживает состояние\linebreak артефактов во внеш\-ней среде 
агента, которые подразделяются на артефакты \mbox{РАСИГИА} (базовая 
онтология) и~артефакты ее внеш\-ней среды. В~терминах плат\-фор\-мы 
JaCaMo~[16], на которой \mbox{реализуется} \mbox{РАСИГИА}, и~ее составной 
части~--- под\-сис\-те\-мы CArtAgO~--- под артефактом понимается 
функ\-цио\-наль\-но-ори\-ен\-ти\-ро\-ван\-ная вы\-чис\-ли\-тель\-ная абстракция, 
предостав\-ля\-ющая услуги агентам для использования в~их основной 
деятельности~[17] по\-сред\-ст\-вом множества пуб\-лич\-но до\-ступ\-ных функций 
и~наблюдаемых свойств, на которые может подписаться агент~[18]. В~случае 
изменения свойства артефакта, на которое подписан агент, ему отправляется 
со\-от\-вет\-ст\-ву\-ющее оповещение. Оповещения обрабатываются под\-сис\-те\-мой 
восприятия агента, которая при его получении формирует список перцептов 
(восприятий) и~модифицирует базу убеж\-де\-ний агента. Если агенту требуется 
интерпретировать ка\-кой-ли\-бо концепт из сообщений других агентов, он 
может сделать это, используя пуб\-лич\-ные функции артефакта, ре\-а\-ли\-зу\-юще\-го 
базовую онтологию \mbox{РАСИГИА}.




  База убеждений агента~--- это хранилище представлений агента о~своей 
внешней среде, т.\,е.\ о~\mbox{РАСИГИА} и~ее внеш\-ней среде. База убеждений 
модифицируется либо в~результате наблюдения изменений во внеш\-ней среде 
с~использованием под\-сис\-те\-мы восприятия, либо в~результате рас\-суж\-де\-ний агента и~исполнения намерений со\-от\-вет\-ст\-ву\-ющим методом. При каж\-дой 
корректировке базы убеж\-де\-ний генерируется событие, которое может быть 
внеш\-ним, вызванным в~результате изменения базы убеж\-де\-ний под\-сис\-те\-мой 
восприятия, или внут\-рен\-ним, по\-рож\-да\-емым методом выбора и~исполнения 
намерения (в~этом случае фиксируется так\-же намерение, вызвавшее событие). 
База убеж\-де\-ний реализуется стандартными средствами платформы JaCaMo и~ее 
со\-став\-ной части~--- под\-сис\-те\-мы Jason~[19]. 
  
  Подсистема обмена сообщениями обеспечивает коммуникацию с~другими 
агентами посредством маршрутизатора сообщений. Последний пред\-став\-ля\-ет 
собой под\-сис\-те\-му программной платформы JaCaMo, обес\-пе\-чи\-ва\-ющую 
корректную доставку сообщений их адресатам. При получении сообщения от 
другого агента под\-сис\-те\-ма обмена сообщениями помещает его в~очередь. 
В~каж\-дом цикле рассуждений агент выбирает первое сообщение из очереди 
и~обрабатывает его. Отправка сообщений другим агентам через маршрутизатор 
сообщений \mbox{РАСИГИА} выполняется по запросу метода выбора 
и~исполнения намерения, при этом для формирования семантически 
правильного сообщения используется интерпретатор базовой онтологии. 
Подсистема обмена сообщениями реализуется стандартными средствами 
под\-сис\-те\-мы Jason платформы JaCaMo, которая поддерживает приоритизацию 
и~фильт\-ра\-цию. 
  
  Интерпретатор базовой онтологии, получая тело сообщения от подсистемы 
обмена сообщениями, с~использованием базовой онтологии выполняет его 
семантический анализ, формирует события, содержащие сгенерированные 
в~результате анализа программные объекты, и~помещает их в~очередь. При 
по\-ступ\-ле\-нии запроса от под\-сис\-те\-мы обмена сообщениями интерпретатор 
базовой онтологии генерирует семантически корректное тело сообщения в~соответствии с~намерением, ини\-ци\-иро\-вав\-шим запрос. 
  
  Очередь событий~--- это буфер, содержащий упорядоченное множество пар  
<<из\-ме\-не\-ние--на\-ме\-ре\-ние>>. В~качестве <<изменения>> может 
выступать изменение убеж\-де\-ний как в~результате восприятия внешней среды 
(в~этом случае вторая часть пары~--- <<намерение>>~--- остается пустой) 
самим агентом или посредством других агентов, так и~в~результате выполнения 
со\-от\-вет\-ст\-ву\-ющих <<намерений>>. Порядок следования событий может быть 
настроен разработчиком агента в~за\-ви\-си\-мости от источника и~времени их 
возникновения.
  
  Метод формирования реакции на событие выбирает из очереди первое 
ожи\-да\-ющее событие, генерирует ре\-ле\-вант\-ное ему намерение с~использованием 
биб\-лио\-те\-ки планов, помещая последнее в~\mbox{оче\-редь} намерений. В~соответствии 
с~данным методом из биб\-лио\-те\-ки планов выбираются все планы, име\-ющие 
ини\-ци\-иру\-ющее событие, которое мож\-но объединить с~выбранным событием 
в~соответствии с~механизмом объединения, принятым под\-сис\-те\-мой Jason 
платформы JaCaMo. Из найден\-ных планов выбираются те, контекстная часть 
которых соответствует текущим убеж\-де\-ни\-ям агента~[19]. Если такие планы не 
найдены, выбранное событие перемещается в~конец очереди для последующей 
повторной обработки. Если же осталось более одного плана, для дальнейшей 
обработки выбирается первый в~соответствии с~порядком, в~котором планы 
записаны в~исходном коде агента, иначе выбирается единственный план. 
Выбранный план становится намерением и~помещается в~очередь намерений. 
  
  Библиотека планов содержит алгоритмы действий агента для реакции на 
возникающие события в~определенной ситуации. План со\-сто\-ит из тела 
и~заголовка, в~котором, в~свою очередь, выделяются ини\-ци\-иру\-ющее событие 
и~контекст, оп\-ре\-де\-ля\-ющие условия исполнения плана. Ини\-ци\-иру\-ющее 
событие плана описывает множество реальных событий, для которых план 
должен использоваться. Если ини\-ци\-иру\-ющее событие плана соответствует 
реальному, план считается актуальным и~в~случае ис\-тин\-ности кон\-текс\-та 
с~учетом текущих убеж\-де\-ний агента становится кандидатом на выполнение. 
Тело плана~--- это по\-сле\-до\-ва\-тель\-ность инструкций (формул), позволяющая 
успешно обработать событие, ини\-ци\-иро\-вав\-шее план. Инструкции в~теле плана 
могут представлять собой вызов функций на языке Java, в~част\-ности, 
ре\-а\-ли\-зу\-ющих методы агента по выполнению своего непосредственного 
функционала, прямые действия по изменению объектов внеш\-ней среды или 
отправке сообщений, а~так\-же действия по генерации убеж\-де\-ний или планов. 
  
  Блок <<методы агента по выполнению своего непосредственного 
функционала>> представляет собой множество функции на языке Java, которые 
могут быть вызваны из тела плана. В~част\-ности, \mbox{функция} может запускать на 
исполнение \mbox{РАСИГИА} более низ\-ко\-го уров\-ня, для которой агент 
выступает в~качестве <<оберт\-ки>>, взаимодействуя с~ней через ее 
интерфейсных агентов и~реализуя таким образом принцип ие\-рар\-хич\-ности.
  
  Очередь намерений~--- упорядоченный список планов, принятых 
к~выполнению и~став\-ших намерениями, или их частей. В~общем случае 
в~очереди намерений агента находятся более одного намерения в~наборе 
намерений, каждое из которых конкурирует за <<внимание>> агента. 
В~каж\-дом цик\-ле рас\-суж\-де\-ний агентом выполняется одна инструкция 
(формула) намерения, после чего оно перемещается в~конец очереди. 
В~результате организуется псевдопараллельное исполнение всех планов, 
принятых к~исполнению. Очередь намерений реализуется стандартными 
средствами под\-сис\-те\-мы Jason платформы JaCaMo.
  
  Метод выбора и~исполнения намерения использует в~своей работе 
стандартную функцию выбора намерения под\-сис\-те\-мы Jason, которая 
поддерживает приоритизацию. Метод выбирает первое намерение в~очереди, 
удаляет его из очереди, выполняет одну его инструкцию. Инструкция может 
содержать вызов функций на языке Java, действия по изменению объектов 
внеш\-ней среды, отправке сообщений, генерации убеж\-де\-ний или планов. 
Исполненная инструкция удаляется из со\-ста\-ва намерения. Если после этого 
инструкций не осталось, намерение считается выполненным и~удаляется из 
очереди, в~противном случае скорректированное намерение вставляется в~ее 
конец.
  
  Подсистема действия предназначена для изменения со\-сто\-яния артефактов во 
внешней среде агента с~использованием их пуб\-лич\-но доступных функций.

\vspace*{-3pt} 

\section{Заключение }

\vspace*{-3pt}

  В работе представлены функциональная структура и~обобщенная 
архитектура агентов \mbox{РАСИГИА}, ре\-а\-ли\-зу\-ющие основные принципы по\-стро\-ения 
таких сис\-тем в~соответствии с~методологией~\cite{12-lis}. В~част\-ности, 
благодаря динамическому со\-ста\-ву и~разнообразию агентов под\-сис\-те\-мы 
решения проб\-ле\-мы обеспечивается учет не\-од\-но\-род\-ности и~из\-мен\-чи\-вости 
проблемы. Предложенная типовая архитектура агента реализует такие его 
свойства, как ав\-то\-ном\-ность, ак\-тив\-ность, ре\-ак\-тив\-ность, ком\-му\-ни\-ка\-тив\-ность, 
реф\-лек\-сив\-ность, спо\-соб\-ность к~моделированию предметной об\-ласти 
и~целеполаганию. Агент \mbox{РАСИГИА} может пред\-став\-лять собой\linebreak сис\-те\-му 
агентов более низ\-ко\-го уровня, что обеспечивает принцип ие\-рар\-хич\-ности 
и~поз\-во\-ля\-ет рас\-смат\-ри\-вать проб\-ле\-му на разных уровнях пред\-став\-ле\-ния. 
Механизмы рефлексивного управ\-ле\-ния, \mbox{ис\-поль\-зу\-емые} агентами, обеспечивают 
гомеостаз сис\-те\-мы благодаря согласованию агентами собственных целей, 
онтологий и~протоколов. За счет открытого характера \mbox{РАСИГИА} 
и~механизмов рефлексивного управ\-ле\-ния в~сис\-те\-ме возникает 
самоорганизация агентов в~сильном смыс\-ле~\cite{11-lis}, без 
централизованного управ\-ле\-ния этим процессом одним из них. 

\vspace*{-3pt}
  
{\small\frenchspacing
 {\baselineskip=11.5pt
 %\addcontentsline{toc}{section}{References}
 \begin{thebibliography}{99}
 
 \vspace*{-3pt}
 
  \bibitem{1-lis}
  \Au{Lepskiy V.} Evolution of cybernetics: Philosophical and methodological analysis~// 
Kybernetes, 2018. Vol.~47. Iss.~2. P.~249--261. doi: 10.1108/K-03-2017-0120.
  \bibitem{2-lis}
  \Au{Листопад С.\,В.} Моделирование рефлексивных процессов в~коллективах 
специалистов, решающих проб\-ле\-мы за круглым столом~// Моделирование неравновесных, 
адаптивных и~управляемых систем: Мат-лы XXVI Всеросс. семинара.~--- Красноярск: ИВМ 
СО РАН, 2023. С.~57--66. 
  \bibitem{3-lis}
  \Au{Тарасов В.\,Б.} От многоагентных сис\-тем к~интеллектуальным организациям: 
философия, психология, информатика.~--- М.: Эдиториал УРСС, 2002. 348~с.
  \bibitem{4-lis}
  \Au{Городецкий В.\,И., Карсаев~О.\,В., Самойлов~В.\,В., Се\-реб\-ря\-ков~С.\,В.} 
Инструментальные средства для открытых сетей агентов~// Известия РАН. Теория и~сис\-те\-мы 
управления, 2008. №\,3. С.~106--124.
  \bibitem{5-lis}
  \Au{Wooldridge M.} An introduction to multiagent systems.~--- New York, NY, USA: Wiley, 
2009. 484~p.
  \bibitem{6-lis}
  \Au{Лефевр В.\,А.} Конфликтующие структуры.~--- М.: Советское радио, 1973. 158~с.
  \bibitem{7-lis}
  \Au{Новиков Д.\,А., Чхартишвили~А.\,Г.} Рефлексия и~управ\-ле\-ние: математические 
модели.~--- М.: Физматлит, 2012. 412~с.
  \bibitem{8-lis}
   \Au{Kobrinskii B.} Expert reflection in the process of diagnosis of diseases at the extraction of 
knowledge~// 4th Research Conference (International)  ``Information Technologies in Science, 
Management, Social Sphere and Medicine'' Proceedings.~--- Advances in computer science 
research ser.~--- Shenzhen: Atlantis Press, 2017. Vol.~72. P.~321--323. doi:  
10.2991/itsmssm-17.2017.66.
  
  \bibitem{9-lis}
   \Au{Смирнов И.\,В., Панов~А.\,И., Скрынник~А.\,А., Чистова~Е.\,В.} Персональный 
когнитивный ассистент: концепция и~принципы работы~// Информатика и~её применения, 
2019. Т.~13. Вып.~3. С.~105--113. doi: 10.14357/19922264190315. EDN: QQSTUK.
  
  \bibitem{10-lis}
   \Au{Мелехин В.\,Б., Хачумов~В.\,М., Хачумов~М.\,В.} Самообучение автономных 
интеллектуальных роботов в~процессе по\-иско\-во-ис\-сле\-до\-ва\-тель\-ской 
дея\-тель\-ности~// Информатика и~её применения, 2023. Т.~17. Вып.~2. С.~78--83. doi: 
10.14357/19922264230211. EDN: \mbox{SOFDKW}.
  
  \bibitem{11-lis}
   \Au{Serugendo G.\,D.\,M., Gleizes~M.-P., Karageorgos~A.} Self-organization in multiagent 
systems~// Knowl. Eng. Rev., 2005. Vol.~20. Iss.~2. P.~165--189. doi: 
10.1017/ S0269888905000494.
  
  \bibitem{12-lis}
  \Au{Листопад С.\,В.} Жизненный цикл методологии построения  
реф\-лек\-сив\-но-ак\-тив\-ных сис\-тем искусственных гетерогенных интеллектуальных 
агентов~// Информатика и~её применения, 2024. Т.~18. Вып.~1.\linebreak С.~84--91.  doi: 
10.14357/19922264240112. EDN: \mbox{GUAMVE}.
  \bibitem{13-lis}
   \Au{Румовская С.\,Б., Кириков~И.\,А.} Метод визуального представления конфликтов 
в~гибридных интеллектуальных многоагентных сис\-те\-мах~// Информатика и~её 
применения, 2020. Т.~14. Вып.~4. С.~77--82. doi: 10.14357/19922264200411. EDN: MXQMLK.
  
  \bibitem{14-lis}
   \Au{Листопад С.\,В., Кириков~И.\,А.} Метод на основе нечетких правил для управления 
конфликтами агентов в~гибридных интеллектуальных многоагентных сис\-те\-мах~// 
Информатика и~её применения, 2023. Т.~17. Вып.~1. С.~66--72. doi: 
10.14357/19922264230109. EDN: DCWSOQ.
  
  \bibitem{15-lis}
   \Au{Румовская С.\,Б.} Подходы к~подбору специалистов при организации коллективного 
решения проб\-лем~// Информатика и~её применения, 2023. Т.~17. Вып.~2.\linebreak С.~96--103. doi: 
10.14357/19922264230214. EDN: VJWNOE.
  
  \bibitem{16-lis}
  \Au{Boissier O., Bordini~R.\,H., Hubnerand~J., Ricci~A.} Multi-agent oriented programming: 
Programming multi-agent systems using JaCaMo.~--- Intelligent robotics and autonomous agents 
ser.~--- Cambridge, MA, USA: The MIT Press, 2020. 264~p.
  \bibitem{17-lis}
  \Au{Freitas A., Panisson A.\,R., Hilgert~L.\,W., Meneguzzi~F., Vieira~R., Bordini~R.\,H.} 
Integrating ontologies with multi-agent systems through CArtAgO artifacts~// IEEE/WIC/ACM  
Conference (International) on Web Intelligence and Intelligent Agent Technology Proceedings.~--- Piscataway, 
NJ, USA: IEEE, 2015. P.~143--150. doi: 10.1109/WI-IAT.2015.116.
  \bibitem{18-lis}
  \Au{Ricci A., Piunti~M., Viroli~M.} Environment programming in multi-agent systems: An 
artifact-based perspective~// Auton. Agent. Multi-Ag., 2011. Vol.~23. Iss.~2. 
P.~158--192. doi: 
10.1007/s10458-010-9140-7.
  \bibitem{19-lis}
  \Au{Bordini R.\,H., H$\ddot{\mbox{u}}$bner~J.\,F., Wooldridge~M.} Programming  
multi-agent systems in AgentSpeak using Jason.~--- Chichester: Wiley-Interscience, 2007. 304~p.

\end{thebibliography}

 }
 }

\end{multicols}

\vspace*{-6pt}

\hfill{\small\textit{Поступила в~редакцию 24.06.24}}

\vspace*{10pt}

%\pagebreak

%\newpage

%\vspace*{-28pt}

\hrule

\vspace*{2pt}

\hrule



\def\tit{BASIC ARCHITECTURE OF~REFLEXIVE-ACTIVE SYSTEMS OF~ARTIFICIAL 
HETEROGENEOUS INTELLIGENT AGENTS}


\def\titkol{Basic architecture of~reflexive-active systems of~artificial 
heterogeneous intelligent agents}


\def\aut{S.\,V.~Listopad}

\def\autkol{S.\,V.~Listopad}

\titel{\tit}{\aut}{\autkol}{\titkol}

\vspace*{-10pt}


\noindent
Federal Research Center ``Computer Science and Control'' of the Russian Academy of 
Sciences, 44-2~Vavilov Str., Moscow 119333, Russian Federation

\def\leftfootline{\small{\textbf{\thepage}
\hfill INFORMATIKA I EE PRIMENENIYA~--- INFORMATICS AND
APPLICATIONS\ \ \ 2024\ \ \ volume~18\ \ \ issue\ 3}
}%
 \def\rightfootline{\small{INFORMATIKA I EE PRIMENENIYA~---
INFORMATICS AND APPLICATIONS\ \ \ 2024\ \ \ volume~18\ \ \ issue\ 3
\hfill \textbf{\thepage}}}

\vspace*{4pt}
  
  
   
   
   \Abste{The paper is devoted to the development of the basic architecture of reflexive-active 
systems of artificial heterogeneous intelligent agents including a standard functional structure of 
the system and a generalized architecture of agents. Due to the combination of heterogeneous 
agents in the system, it hybridizes various specialists' knowledge and methods of formal 
representation of the systems taking into account the instrumental and functional heterogeneity of 
emerging problems. The reflexive capabilities of agents allow them to model other agents, 
themselves, and the system as a~whole reducing the intensity of conflicts and the duration of 
building interaction when the composition of the system changes. As a result, self-organization of 
the system in a~strong sense becomes possible during which for each new practical problem, 
a~solution method relevant to it is developed.}
   
   \KWE{reflection; reflexive-active system of artificial heterogeneous intelligent agents; hybrid 
intelligent multiagent system; team of specialists}
   

   
\DOI{10.14357/19922264240311}{UNTQBV}

\vspace*{-16pt}


     \Ack
     
     \vspace*{-3pt}
     
      \noindent
      The work was supported by the Russian Science Foundation, project No.\,23-21-00218 ({\sf  
https://rscf.ru/project/23-21-00218}).

  \begin{multicols}{2}

\renewcommand{\bibname}{\protect\rmfamily References}
%\renewcommand{\bibname}{\large\protect\rm References}

{\small\frenchspacing
 {%\baselineskip=10.8pt
 \addcontentsline{toc}{section}{References}
 \begin{thebibliography}{99} 
  \bibitem{1-lis-1}
   \Aue{Lepskiy, V.} 2018. Evolution of cybernetics: Philosophical and methodological analysis. 
\textit{Kybernetes} 47(2):249--261. doi: 10.1108/K-03-2017-0120.
  \bibitem{2-lis-1}
   \Aue{Listopad, S.\,V.} 2023. Modelirovanie refleksivnykh pro\-tses\-sov v~kollektivakh 
spetsialistov, reshayushchikh problemy za kruglym stolom [Modeling reflexive processes in teams 
of specialists solving problems at a round table]. \textit{Modelirovanie neravnovesnykh, 
adaptivnykh i~uprav\-lya\-emykh sistem: Mat-ly XXVI Vseross. seminara} [Modeling of 
nonequilibrium, adaptive and controlled systems: 26th National Workshop]. Krasnoyarsk: 
IVM SO RAN. 57--66. 
  \bibitem{3-lis-1}
   \Aue{Tarasov, V.\,B.} 2002. \textit{Ot mnogoagentnykh sistem k~intellektual'nym 
organizatsiyam: filosofiya, psikhologiya, informatika} [From multiagent systems to intelligent 
organizations: Philosophy, psychology, and computer science]. Moscow: Editorial URSS. 348~p.
  \bibitem{4-lis-1}
   \Aue{Gorodetskii, V.\,I., O.\,V.~Karsaev, V.\,V.~Samoilov, and S.\,V.~Serebryakov.} 2008. 
Development tools for open agent networks. \textit{J.~Comput. Sys. Sc. Int.} 
47(3):429--446. doi: 10.1134/S1064230708030131. EDN: LLHPLJ.
  \bibitem{5-lis-1}
   \Aue{Wooldridge, M.} 2009. \textit{An introduction to multiagent systems}. New York, NY: 
Wiley. 484~p.
  \bibitem{6-lis-1}
   \Aue{Lefebvre, V.\,A.} 1973. \textit{Konfliktuyushchie struktury} [Conflicting structures]. 
Moscow: Soviet Radio. 158~p.
  \bibitem{7-lis-1}
   \Aue{Novikov, D.\,A., and A.\,G.~Chkhartishvili.} 2012. \textit{Refleksiya i~upravlenie: 
matematicheskie modeli} [Reflexion and control: Mathematical models]. Moscow: 
Fizmstlit. 412~p.
  \bibitem{8-lis-1}
   \Aue{Kobrinskii, B.} 2017. Expert reflection in the process of diagnosis of diseases at the 
extraction of knowledge. \textit{4th Research Conference (International)  ``Information 
Technologies in Science, Management, Social Sphere and Medicine'' Proceedings}. Advances in 
computer science research ser. Shenzhen: Atlantis Press. 72:321--323. doi: 
 10.2991/\linebreak  itsmssm-17.2017.66.
  \bibitem{9-lis-1}
   \Aue{Smirnov, I.\,V., A.\,I.~Panov, A.\,A.~Skrynnik, and E.\,V.~Chistova.} 2019. Personal'nyy 
kognitivnyy assistent: kon\-tsep\-tsiya i~printsipy raboty [Personal cognitive assistant: Concept and 
key principals]. \textit{Informatika i~ee Primeneniya~--- Inform. Appl.} 13(3):105--113. doi: 
10.14357/ 19922264190315. EDN: QQSTUK.
  \bibitem{10-lis-1}
   \Aue{Melekhin, V.\,B., V.\,M.~Khachumov, and M.\,V.~Khachumov.} 2023. Samoobuchenie 
avtonomnykh intellektual'nykh robotov v~protsesse poiskovo-issledovatel'skoy deyatel'nosti [Self-
learning of autonomous intelligent robots in the process of search and explore activities]. 
\textit{Informatika i~ee Primeneniya~--- Inform. Appl.} 17(2):78--83. doi: 
10.14357/19922264230211. EDN: SOFDKW.
  \bibitem{11-lis-1}
   \Aue{Serugendo, G.\,D.\,M., M.-P.~Gleizes, and A.~Karageorgos.} 2005. Self-organization in 
multiagent systems. \textit{Knowl. Eng. Rev.} 20(2):165--189. doi: 10.1017/S0269888905000494.
  \bibitem{12-lis-1}
   \Aue{Listopad, S.\,V.} 2024. Zhiznennyy tsikl metodologii po\-stroeniya refleksivno-aktivnykh 
sistem iskusstvennykh geterogennykh intellektual'nykh agentov [Life cycle of methodology for 
constructing reflexive-active systems of artificial heterogeneous intelligent agents]. 
\textit{Informatika i~ee Primeneniya~--- Inform. Appl.} 18(1):84--91. doi: 
10.14357/19922264240112. EDN: GUAMVE.
  \bibitem{13-lis-1}
   \Aue{Rumovskaya, S.\,B., and I.\,A.~Kirikov.} 2020. Metod vizual'nogo predstavleniya konfliktov 
v~gibridnykh intellektual'nykh mnogoagentnykh sistemakh [Conflict visual representation method in hybrid 
intelligent multiagent systems]. \textit{Informatika i~ee Primeneniya~--- Inform. \mbox{Appl.}} 14(4):77--82. 
doi: 10.14357/19922264200411. EDN: MXQMLK.
 \bibitem{14-lis-1}
   \Aue{Listopad, S.\,V., and I.\,A.~Kirikov.} 2023. Metod na osno\-ve nechetkikh pravil dlya 
upravleniya konfliktami agentov v~gib\-rid\-nykh intellektual'nykh mnogoagentnykh sistemakh [Fuzzy 
rules based method for agent conflict management in hybrid intelligent multiagent systems]. 
\textit{Informatika i~ee Primeneniya~--- Inform. Appl.} 17(1):66--72. doi: 
10.14357/19922264230109. EDN: DCWSOQ.
  \bibitem{15-lis-1}
   \Aue{Rumovskaya, S.\,B.} 2023. Podkhody k podboru spetsialistov pri organizatsii kollektivnogo 
resheniya problem [Selection of specialists in the organization of collective solving problems]. 
\textit{Informatika i~ee Primeneniya~--- Inform. \mbox{Appl.}} 17(2):96--103. doi: 10.14357/ 
19922264230214. EDN: VJWNOE.
  \bibitem{16-lis-1}
   \Aue{Boissier, O., R.\,H.~Bordini, J.~Hubnerand, and A.~Ricci}. 2020. \textit{Multi-agent 
oriented programming: Programming multi-agent systems using JaCaMo}. Intelligent robotics and 
autonomous agents ser. Cambridge, MA: The MIT Press. 264~p.
  \bibitem{17-lis-1}
   \Aue{Freitas, A., A.\,R.~Panisson, L.~Hilgert, F.~Meneguzzi, R.~Vieira and R.\,H.~Bordini}. 
2015. Integrating ontologies with multi-agent systems through CArtAgO artifacts. 
\textit{IEEE/WIC/ACM International Conference on Web Intelligence and Intelligent Agent 
Technology Proceedings}. 143--150. doi: 10.1109/WI-IAT.2015.116.
  \bibitem{18-lis-1}
   \Aue{Ricci, A., M.~Piunti, and M.~Viroli.} 2011. Environment programming in multi-agent 
systems: An artifact-based perspective. \textit{Auton. Agent. Multi-Ag.} 23(2):158--192. doi: 
10.1007/s10458-010-9140-7.
  \bibitem{19-lis-1}
   \Aue{Bordini, R.\,H., J.\,F.~Hьbner, and M.~Wooldridge.} 2007. \textit{Programming  
multi-agent systems in AgentSpeak using Jason}. Chichester: Wiley-Interscience. 304~p.
   
   
  \end{thebibliography}

 }
 }

\end{multicols}

\vspace*{-6pt}

\hfill{\small\textit{Received June 24, 2024}} 

\vspace*{-18pt}


\Contrl

\vspace*{-3pt}
   
   \noindent
   \textbf{Listopad Sergey V.} (b.\ 1984)~--- Candidate of Science (PhD) in technology, senior 
scientist, Federal Research Center ``Computer Science and Control'' of the Russian Academy of 
Sciences, 44-2~Vavilov Str., Moscow 119133, Russian Federation;  
\mbox{ser-list-post@yandex.ru}
   
   
\label{end\stat}

\renewcommand{\bibname}{\protect\rm Литература} 