\newcommand {\ebd}{\triangleq}
\newcommand{\pp}[1]{\mathcal{P}\left\{ #1 \right\}}
\newcommand {\ff}{{\mathcal F}}
\newcommand {\ppp}{{\mathcal P}}
\newcommand {\NII}{{\mathcal N}}
\newcommand {\qqq}{{\mathcal Q}}
\newcommand{\me}[2]{\mathsf{E}_{ #1 }\left\{ \mathop{#2} \right\} }


\def\stat{borisov}

\def\tit{ВЕРОЯТНОСТНЫЙ АНАЛИЗ КЛАССА МАРКОВСКИХ~СКАЧКООБРАЗНЫХ ПРОЦЕССОВ$^*$}

\def\titkol{Вероятностный анализ класса марковских скачкообразных процессов}

\def\aut{А.\,В.~Борисов$^1$,  Ю.\,Н.~Куринов$^2$,  Р.\,Л.~Смелянский$^3$}

\def\autkol{А.\,В.~Борисов,  Ю.\,Н.~Куринов,  Р.\,Л.~Смелянский}

\titel{\tit}{\aut}{\autkol}{\titkol}

\index{Борисов А.\,В.}
\index{Куринов Ю.\,Н.}
\index{Смелянский Р.\,Л.}
\index{Borisov A.\,V.}
\index{Kurinov Yu.\,N.}
\index{Smeliansky R.\,L.}


{\renewcommand{\thefootnote}{\fnsymbol{footnote}} \footnotetext[1]
{Работа 
выполнена при поддержке Программы развития МГУ, проект №\,23-Ш03-03. При анализе 
данных использовалась инфраструктура Центра коллективного пользования 
<<Высокопроизводительные вычисления и~большие данные>> 
(ЦКП <<Информатика>>) ФИЦ ИУ РАН (г.~Москва).}}


\renewcommand{\thefootnote}{\arabic{footnote}}
\footnotetext[1]{Федеральный исследовательский центр <<Информатика 
и~управление>> Российской академии наук; Московский государственный
университет имени М.\,В.~Ломоносова, \mbox{aborisov@frccsc.ru}}
\footnotetext[2]{Московский  государственный университет имени М.\,В.~Ломоносова, \mbox{kurinovurij@gmail.com}}
\footnotetext[3]{Московский государственный университет имени М.\,В.~Ломоносова, \mbox{smel@cs.msu.su}}


\vspace*{-12pt}




 \Abst{Исследован некоторый класс скачкообразных процессов. Их первая блочная 
компонента представляет собой марковский скачкообразный процесс (МСП) с~конечным 
множеством состояний. Вторая блочная компонента изменяется синхронно с~первой 
и~при фиксированной первой компоненте образует последовательность независимых 
векторов. При этом носители условных распределений второй компоненты могут 
пересекаться, что не дает возможности точно восстановить значения первой 
компоненты по наблюдениям второй. Для рассмотренного класса случайных процессов 
доказано марковское его свойство и~получен ряд важных вероятностных 
характеристик. Выведен инфинитезимальный генератор и~сопряженный к~нему 
оператор. Это позволило построить систему уравнений Колмогорова, описывающую 
эволюцию распределения процесса. Предложено мартингальное разложение 
произвольной функции от исследуемого процесса: ее удается характеризовать 
с~по\-мощью сис\-те\-мы линейных стохастических дифференциальных уравнений (СДУ)
с~мартингалами в~правой части. В~случае если функции исследуемого процесса имеют 
конечные моменты второго порядка, получен вид квадратичных характеристик 
мартингалов в~соответствующих разложениях.}

\KW{марковский скачкообразный процесс; инфинитезимальный генератор; 
мартингальное разложение; стохастическое дифференциальное уравнение}

\DOI{10.14357/19922264240304}{XPVTGJ}
  
\vspace*{-6pt}


\vskip 10pt plus 9pt minus 6pt

\thispagestyle{headings}

\begin{multicols}{2}

\label{st\stat}


\section{Введение}

\vspace*{-8pt}

Марковские скачкообразные процессы, вклю\-чая их управ\-ля\-емые версии, служат 
математическими моделями большого числа явлений, наблюдаемых в~различных областях 
техники, биологии и~медицины, социологии, экономики и~пр.~\cite{W_93}. Они так\-же 
используются при описании более сложных объектов: скрытых марковских моделей 
и~стохастических динамических сис\-тем переменной структуры~\cite{EM_02, DLW_23}.

Для МСП существует собственный развитый математический аппарат~\cite{D_63, GS_73_2}, позволяющий решать задачи системного анализа применительно к~данному 
классу процессов. Кроме того, при выполнении некоторых необременительных условий 
эти функции представляют собой семимартингалы с~возможностью использования 
результатов стохастического анализа~\cite{LSh_86}.

При построении моделей чаще используются МСП с~конечным или счетным множеством 
состояний. Это связано с~их простотой, наличием решений различных задач 
оп\-ти\-маль\-но\-го/ро\-баст\-но\-го оценивания и~управления состояниями подобных процессов~\cite{EAM_10}, а~также набора эффективных вычислительных алгоритмов, реализующих 
эти решения. В~то же время решения указанных задач для произвольных МСП 
достаточно сложны для численной реализации~\cite{LSh_86, WD_79, CGT_02}.

Цель работы заключается в~представлении подкласса МСП, обеспечивающего 
компромисс между общностью и~гибкостью для математического моделирования 
различных явлений, простотой вероятностного описания и~решения задач оценивания и~управ\-ле\-ния. Работа имеет следующую организа\-цию. Раздел~2 содержит 
конструктивное представ\-ле\-ние класса процессов на примере описания 
скачкообразного изменения состояния и~па\-ра\-мет\-ров некоторого 
телекоммуникационного соединения. В~разд.~3 доказано марковское свойство этих %\linebreak 
процессов, получен инфинитезимальный генератор, а~также сис\-те\-ма уравнений 
Колмогорова, описывающая эволюцию распределения. Раздел~4 %\linebreak 
посвящен 
мартингальному представлению ис\-сле\-ду\-емо\-го класса МСП и~функций от них. 
Оказывается, что их можно характеризовать с~по\-мощью решений линейных сис\-тем 
СДУ с~мартингалами в~правой части. 
Раздел~5 содержит заключительные замечания.

\section{Конструктивное описание класса процессов}

Для представления класса исследуемых процессов воспользуемся иллюстрацией из 
области телекоммуникаций~\cite{B_14, B_16_1, B_16_2, G_23_1, B_21_1}. Рассмотрим 
состояние некоторого гетерогенного (про\-вод\-но\-го/бес\-про\-вод\-но\-го) сетевого 
соединения. Оно описывается блочным вектором $Z_t \hm\ebd \mathrm{col}(\theta_t, Y_t)$. 
Первая компонента, $\theta_t$, определяет качественное состояние соединения 
и~может принимать значения из конечного множества, например:
\begin{itemize}
\item
$\theta_t=e_1$~--- канал, обеспечивающий
 соединение, загружен умеренно: буфер сетевого устройства~--- <<бутылочного 
горла>> свободен;
\item
$\theta_t=e_2$~--- канал находится в~состоянии, предшествующем перегрузке: буфер 
<<бутылочного горла>> непуст;
\item
$\theta_t=e_3$~--- перегрузка канала: буфер <<бутылочного горла>> заполнен 
полностью;
\item
$\theta_t=e_4$~--- потеря сигнала на беспроводном участке канала.
\end{itemize}
Вторая компонента, $Y_t$, определяет текущие средние числовые параметры 
соединения, например:
\begin{itemize}
\item
$Y_t^1$~--- значение времени кругового обращения сегмента данных (RTT, round-trip time);
\item[--]
$Y_t^2$~--- значение джиттера;
\item
$Y_t^3$~--- долю потерянных пакетов;
\item
$Y_t^4$~--- пропускную способность, обес\-пе\-чи\-ва\-емую соединением.
\end{itemize}
Траектории процесса $Z_t$~--- ку\-соч\-но-по\-сто\-ян\-ные функции, и~скачки компоненты~$\theta_t$ 
синхронизированы со скачками~$Y_t$. Применительно к~данному примеру 
это означает, что со сменой качественного состояния соединения меняются и~его 
числовые параметры и~в~процессе, пока качественное состояние остается 
неизменным, параметры соединения также не меняются. Следует отметить, что 
в~случае возвращения соединения в~некоторое состояние, ранее уже имевшее место, 
его чис\-ло\-вые па\-ра\-мет\-ры могут не совпасть с~прошлыми значениями. Например, после 
возвращения состояния соединения в~$e_1$ значения RTT, джиттера, доли потерянных 
пакетов и~пропускной способности могут отличаться от предыдущих значений 
параметров на предшествующих промежутках пребывания в~состоянии~$e_1$. При этом 
по текущему набору па\-ра\-мет\-ров~$Y_t$ \mbox{нельзя} сделать однозначный вывод о~текущем 
качественном состоянии соединения~$\theta_t$: одни и~те же значения~$Y_t$ могут 
соответствовать различным значениям~$\theta_t$. В~этом заключено ключевое 
отличие представляемого класса процессов от специальных МСП, исследованных 
в~\cite{B_04_1}. Практической целью ставится выделение класса процессов, 
неформально определяемых как МСП с~конечным множеством состояний~--- 
вероятностных распределений. Этот класс, конечно, не охватывает все множество 
МСП~\cite{D_63, GS_73_2}, однако обладают достаточной степенью общности, чтобы 
описывать ряд реальных явлений, включающий в~себя маневрирование целей~\cite{LJ_05}, 
скачки параметров финансового рынка~\cite{mamon2014hidden}, 
изменение состояния и~параметров сетевого трафика~\cite{DPR_08} и~пр.

Итак, первый блочный компонент, $\theta_t$, пред\-став\-ля\-ет собой МСП с~множеством 
состояний 
$$
\mathbb{S}^N \ebd  \left\{e_1, \ldots, e_N\right\},
$$
 матрицей 
интенсивностей переходов 
$$
\Lambda(t)=\|\Lambda_{ij}(t)\|_{i,j=\overline{1,N}}
$$ 
и~начальным распределением 
$$
p_0=\mathrm{col}\left(p_0^1,\ldots,p_0^N\right).
$$
Вторая блочная компонента,  $Y_t \hm\in \mathbb{R}^M$,~--- ку\-соч\-но-по\-сто\-ян\-ный 
процесс, терпящий скачки синхронно с~$\theta_t$. Если $\{\tau_i\}_{i \in 
\mathbb{N}}$ -- последовательность моментов скачков~$\theta_t$, то при известной 
траектории~$\theta_t$ последовательность $\{Y_{\tau_i}\}_{i \in \mathbb{N}}$ 
составлена из независимых случайных векторов, условное распределение которых 
определяется вектором 
$$
\pi(y) = \mathrm{col}\left(\pi_1(y),\ldots,\pi_N(y)\right),
$$
 составленным из 
известных плотностей ве\-ро\-ят\-ности на~$\mathbb{R}^M$:
\begin{multline*}
\pp{Y_{\tau_i}\in B\;|\;\theta_{\tau_i}\! =\! e_n}\! =\! \int\limits_{B} \pi_n(y)\,dy, \forall
\;B \hm\in \mathcal{B}(\mathbb{R}^M),\\
  n=\overline{1,N}\,.
\end{multline*}
Распределение начального значения $Y_0$ определяется аналогично с~по\-мощью 
распределения $\phi(y) \hm= \mathrm{col}\,(\phi_1(y),\ldots,\phi_N(y))$:
\begin{multline*}
\pp{Y_{0} \in B\;|\;\theta_{0} = e_n} = \int\limits_{B} \phi_n(y)\,dy,\ 
\forall\,B \hm\in \mathcal{B}(\mathbb{R}^M),\\
  n=\overline{1,N}\,.
\end{multline*}
В~\cite{IK_06} показано, что существует канонический базис с~фильтрацией 
$(\Omega,\ff,\ppp, \{\ff_t\}_{t \in [0,T]})$, так называемое пространство Ви\-не\-ра--Пуас\-со\-на, 
на котором подобные процессы могут быть заданы корректно. Ниже 
представлены вероятностные свойства этих процессов.

\section{Марковское свойство и~производящий оператор}

В дальнейшем изложении используются следующие обозначения:
\begin{itemize}
\item
$I$~--- единичная матрица подходящей раз\-мер\-ности;
\item
$\mathbf{I}(t)$~--- единичная ступенчатая функция;
\item
$\mathbf{I}_{D}(x)$ -- индикаторная функция множества $D$;
\item
$\NII_t$~--- число скачков $\theta_t$, произошедших на отрезке времени $[0,t]$,
\item
$\mathcal{T}_n(s,t) \ebd \pp{\NII_t-\NII_s=0|\theta_s=e_n} \hm= \exp \left( \int\nolimits_s^t 
\Lambda_{nn}(u)\,du\right)$ ($0 \hm\leqslant s \hm< t$)~--- условная функция 
распределения времени ожидания скачка процесса~$\theta_t$ в~зависимости от его 
состояния, $\mathcal{T}(s,t) \hm\ebd \mathrm{row}\, (\mathcal{T}_1(s,t),\ldots,\mathcal{T}_N(s,t))$;
\item
$\lambda(t) \ebd \mathrm{row}\, (\Lambda_{11}(t),\ldots,\Lambda_{NN}(t))$~--- строка, 
составленная из диагональных элементов матрицы~$\Lambda(t)$, 
$\overline{\Lambda}(t) \ebd \Lambda(t) \hm- \mathrm{diag}\, \lambda(t)$;
\item
$\ppp(s,t) = \|\ppp_{ij}(s,t)\|_{i,j=\overline{1,N}}$~--- матрица переходных 
вероятностей~$\theta$ на отрезке $[s,t]$:
$$
\ppp_{ij}(s,t) \ebd \pp{\theta_t=e_j\; | \; \theta_s=e_i};
$$
 $\ppp(s,t)$ 
представляет собой решение системы дифференциальных уравнений: 
$$
\ppp'_t(s,t)= \ppp(s,t)\Lambda(t),\ 0\leqslant s < t,\ \ppp(s,s) \equiv I\,;
$$
\item
любая функция $f(e,y):\; \mathbb{S}^N \times  \mathbb{R}^M \hm\to \mathbb{R}$ 
представима в~виде $f(e,y) \hm= \overline{f}(y)e$, где
$$
\overline{f}(y) \ebd \mathrm{row}\left(f(e_1,y), \ldots, f(e_N,y)\right);
$$
\item
$ {\sf E}_f^n \ebd \int_{\mathbb{R}^M} f(y,e_n) \pi_n(y)\,dy$, 
${\sf E}_f \hm= \mathrm{col}\, ({\mathsf E}_f^1,\ldots,{\mathsf E}_f^N) = \int_{\mathbb{R}^M} 
\mathrm{diag}\,\overline{f}(y)\pi(y) \,dy$;
\item[--]
любая вероятностная мера $\qqq(\cdot)$, определенная на $(\mathbb{S}^N \times  
\mathbb{R}^M ,2^{\mathbb{S}^N}\times  \mathcal{B}(\mathbb{R}^M))$, может быть 
задана с~помощью распределения 
$$
m(B) = \mathrm{col}\left(m_1(B),\ldots,m_N(B)\right),
$$
 где
$$
m_n(B) \ebd \qqq\{\theta \hm= e_n, Y \hm\in B\}.
$$
\end{itemize}
Важнейшим качеством исследуемых процессов выступает марковское свойство.

\smallskip

\noindent
\textbf{Теорема~1.}
%\label{th:th_1}
\textit{На вероятностном базисе с~фильтрацией $(\Omega,\ff,\ppp, \{\ff_t\}_{t \in 
[0,T]})$ процесс $Z_t$ обладает марковским свойством.
Матрица переходной вероятности} $\mathsf{P}(y,s,B,t) 
=\|\mathsf{P}_{ij}(y,s,B,t)\|_{i,j=\overline{1,N}}$
($\mathsf{P}_{ij}(y,s,B,t) \hm\ebd \pp{\theta_t = e_j, Y_t \in B | \theta_s = e_i, 
Y_s=y}$)
\textit{имеет вид}:
\begin{multline}
\mathsf{P}(y,s,B,t) = \mathbf{I}_B(y)\mathrm{diag}\,\mathcal{T}(s,t) + {}\\
{}+\mathrm{diag}  \int\limits_B 
\pi(u)\,du \left( \ppp(s,t) - \mathrm{diag}\,\mathcal{T}(s,t) \right).
\label{eq:eq_trfunc}
\end{multline}

\noindent
Д\,о\,к\,а\,з\,а\,т\,е\,л\,ь\,с\,т\,в\,о\ тео\-ре\-мы~1 приведено в~приложении.

\smallskip

В \cite{B_04_1} представлен частный случай~$Z_t$: мно-\linebreak жест\-ва-но\-си\-те\-ли 
$\{D_n\}_{n=\overline{1,N}}$ распределений $\{\pi_n(\cdot)\}_{n=\overline{1,N}}$ 
и~$\{\phi_n(\cdot)\}_{n=\overline{1,N}}$ не пересекались. В~этом случае 
компоненту $\theta_t$ можно было восстановить по компоненте~$Y_t$,
и~поэтому компонента~$Y_t$, рассмотренная отдельно, также обладала марковским 
свойством. В~общем же случае~$Y_t$, рассмотренная в~отрыве от~$\theta_t$, 
марковским свойством не обладает.


\smallskip

\noindent
\textbf{Теорема~2.}
%\label{th:th_2} 
\textit{Верны следующие утверждения}.
\begin{enumerate}[1.]
\item \textit{Пусть $f(e,y): \mathbb{S}^N \times  \mathbb{R}^M \hm\to \mathbb{R}$~--- 
ограниченная борелевская функция.
Инфинитезимальный генератор~$\mathcal{A}_t$ процесса~$Z_t$ имеет вид}:

\vspace*{-4pt}

\noindent
\begin{multline*}
\mathcal{A}_t f(e,y) \ebd{}\\
{}\ebd
\lim\limits_{s \downarrow t}\fr{\me{}{f( \theta_s,Y_s)|Y_t = y, \theta_t = e} - 
f(e,y)}{s - t} ={}\\
{}=
\left[\overline{f}(y) \mathrm{diag}\,\lambda(t) + {\mathsf E}_f^{\top} \overline{\Lambda}^{\top}(t)\right]e\,.
\end{multline*}

\item \textit{Пусть $m(B) = \mathrm{col}\,(m_1(B),\ldots,m_N(B))$ -- некоторое вероятностное 
распределение на\linebreak $( \mathbb{S}^N \times  \mathbb{R}^M ,2^{\mathbb{S}^N}\times  
\mathcal{B}(\mathbb{R}^M))$, тогда оператор~$\mathcal{A}_t^*$, сопряженный к~генератору, имеет вид}:

\vspace*{-4pt}

\noindent
\begin{multline*}
\mathcal{A}_t^* m(B)  =
\mathrm{diag}\left(\lambda (t)\right) m(B) +{}\\
{}+
 \mathrm{diag}\left( \int\limits_B \pi(u)\,du \right) 
\overline{\Lambda}^{\top}(t) m(\mathbb{R}^M).
\end{multline*}

\item \textit{Распределение $Q_t(B) \hm= \mathrm{col}\, (Q_t^1(B),\ldots,Q_t^N(B))$ процесса 
$\mathrm{col}\,(\theta_t, Y_t) (Q_t^n(B) \ebd \mathcal{P}\left\{\theta_t \hm= e_n,\right.$\linebreak $\left. Y_t \hm\in B\right\}
\vphantom{\left(Q_t^n(B)\right)\ebd}\!\!
)$ 
определяется решением системы уравнений Колмогорова}:

\noindent
\begin{align*}
\fr{dQ_t(B)}{dt} &= \mathrm{diag}\,(\lambda (t)) Q_t(B) + {}\\
&\hspace*{-5mm}{}+\mathrm{diag}\left( \int\limits_B \pi(u)\,du 
\right) \overline{\Lambda}^{\top}(t) \; Q_t\left(\mathbb{R}^M\right);\\
Q_0(B)& = \mathrm{diag}\left( \int\limits_B \phi(u)\,du \right) p_0.
%\label{eq:sys_1}
\end{align*}

\item \textit{Функция переходной вероятности ${\sf P}(y,s,B,t)$ определяется решением системы 
уравнений}:

\pagebreak

\noindent
\begin{equation}
\left.
\begin{array}{l}
\mathsf{P}'_t(y,s,B,t) = \displaystyle \mathsf{P}(y,s,B,t) \mathrm{diag}\,\lambda(t) +{}\\[9pt]
\hspace*{10mm}{}+ \mathrm{diag}\left(\, 
\displaystyle\int\limits_B \pi(u)\,du \right) \ppp(s,t)\overline{\Lambda}(t); \\[16pt]
\ppp'_t(s,t) = \ppp(s,t)\Lambda (t),\qquad 0 \leqslant s < t; \\[9pt]
\mathsf{P}(y,s,B,s) = \mathbf{I}_B(y)I,\qquad \ppp(s,s) = I.
\end{array}
\right\}
\label{eq:sys_2}
\end{equation}
\end{enumerate}

Истинность утверждений~1 и~2 теоремы~2 следует из определений 
инфинитезимального генератора 
%$$
$\mathcal{A}_t f(e,y)$
% =
%\lim\limits_{s \downarrow t}\fr{\me{}{f( \theta_s,Y_s)|Y_t = y, \theta_t = e} - 
%f(e,y)}{s - t}$$ 
сопряженного оператора
\begin{multline*}
\langle \mathcal{A}_t f,m\rangle = \int\limits_{\mathbb{R}^M} \sum\limits_{n=1}^N\mathcal{A}_t 
f(e_n, y) m_n(dy) ={}\\
{}= \int\limits_{\mathbb{R}^M}
\sum\limits_{n=1}^N f(e_n, y) \mathcal{A}_t^* m_n(dy) = \langle  f,\mathcal{A}_t^* m\rangle.
\end{multline*}
 Истинность утверждения~3 теоремы следует из представления 
распределения~$Z_t$ с~помощью оператора~$\mathcal{A}_t^*$~\cite{GS_73_2}. 
Система~(\ref{eq:sys_2}) (утверждение~4 теоремы) может быть получена путем 
непосредственного дифференцирования переходной вероятности~(\ref{eq:eq_trfunc}).

\section{Мартингальное представление}

Пусть $f=f(e,y): \mathbb{S}^N \times  \mathbb{R}^M \hm\to \mathbb{R}$~--- 
произвольная функция, у которой
$$
\int\limits_{\mathbb{R}^M}\sum\limits_{n=1}^N |f(e_n, y)|(\pi_n(y)+\phi_n(y))\,dy < \infty\,.
$$
Для процесса $f(\theta_t,Y_t)$ верна формула Дынкина: для любых $0\hm \leqslant s \hm< 
t$ выполнено равенство
\begin{multline*}
\me{}{f(\theta_t,Y_t) - \int\limits_s^t \mathcal{A}_uf(\theta_u,Y_u)\,du \; 
\Bigl|\;\ff_s} = {}\\
= {\mathsf E}\left\{f(\theta_t,Y_t) - \int\limits_s^t
 \left[\overline{f}(Y_u) \mathrm{diag}\,\lambda(u) + {}\right.\right.\\
\left.\left. {}+{\mathsf E}_f^{\top} 
\overline{\Lambda}^{\top}(u)\right]\theta_u\,du
\vphantom{\int\limits_s^t}
\right\}= 0 \quad \ppp\mbox{-п.~н.}
\end{multline*}
Это означает, что процесс $f(\theta_t,Y_t)$ допускает разложение
\begin{multline*}
f(\theta_t,Y_t) = f(\theta_0,Y_0) + {}\\
{}+\int\limits_0^t
 \left[\overline{f}(Y_u) \mathrm{diag}\, \lambda(u) + E_f^{\top} 
\overline{\Lambda}^{\top}(u)\right]\theta_u\,du +\mu_t^f,
\end{multline*}
где $\mu_t^f$~--- некоторый $\ff_t$-со\-гла\-со\-ван\-ный мартингал. Используя это 
равенство, можно составить конечную замкнутую систему линейных СДУ, описывающую 
эволюцию процесса $f(\theta_t,Y_t)$. Рассмотрим расширенный процесс 
$$
\mathsf{f}_t \ebd \mathrm{col}\,(\underbrace{\theta_t}_{\ebd \mathsf{f}_t^1}, 
\underbrace{f(\theta_t,Y_t)\theta_t}_{\ebd \mathsf{f}_t^2})\,.
$$

\noindent
\textbf{Теорема~3.}\
%\label{th:th_3}
\textit{Процесс~$\mathsf{f}_t$~--- единственное сильное реше-}\linebreak \textit{ние системы линейных СДУ}
\begin{equation}
\mathsf{f}_t = \mathsf{f}_0 + \int\limits_0^t D^f(s)\mathsf{f}_s\,ds + \mu_t^f,
\label{eq:sys_3}
\end{equation}
\textit{где $\mu_t^f$~--- $\ff_t$-со\-гла\-со\-ван\-ный мартингал; $D^f(s): \mathbb{R}_+ \hm\to 
\mathbb{R}^{2N \times 2N}$-мер\-ная матричнозначная функция}
%\begin{equation*}
$$
D^f(t) \ebd \left[
\begin{array}{ccc}
\Lambda^{\top}(t) & & 0\\
 \mathrm{diag}\, {\mathsf E}_f {\overline{\Lambda}}^{\top}(t)  & &  \mathrm{diag}\,\lambda(t)
\end{array}
\right].
$$
\textit{Если дополнительно}
\begin{equation}
\int\limits_{\mathbb{R}^M}\sum\limits_{n=1}^Nf^2(e_n, y)(\pi_n(y)+\phi_n(y))\,dy < \infty\,,
\label{eq:cond_1}
\end{equation}
\textit{то $\mu_t^f$~--- квадратично интегрируемый мартингал с~квадратичной 
характеристикой}
$$
\langle \mu^f, \mu^f \rangle_t = \left[ 
\begin{array}{cc}
\langle \mathsf{f}^1, \mathsf{f}^1 \rangle_t & \langle \mathsf{f}^1, 
\mathsf{f}^2 \rangle_t \\
\langle \mathsf{f}^1, \mathsf{f}^2 \rangle_t^{\top} & \langle \mathsf{f}^2, 
\mathsf{f}^2 \rangle_t
\end{array}
\right],
$$
\textit{где}

\vspace*{-4pt}

\noindent
\begin{multline}
\langle \mathsf{f}^1, \mathsf{f}^1 \rangle_t =\int\limits_0^t \left[
\mathrm{diag}\left( \Lambda^{\top}(s)\mathsf{f}^1_s\right)-  
\Lambda^{\top}(s)\mathrm{diag}\,(\mathsf{f}^1_s) - {}\right.\\
\left.{}-\mathrm{diag}\,(\mathsf{f}^1_s)\Lambda(s)
\right]ds\,;
\label{eq:sc_11}
\end{multline}

\vspace*{-12pt}

\noindent
\begin{multline}
\langle \mathsf{f}^1, \mathsf{f}^2 \rangle_t={}\\
{}=
\int\limits _0^t \left[
\left(\mathrm{diag}\,( \overline{\Lambda}^{\top}(s)\mathsf{f}^1_s) -   
\mathrm{diag}\,(\mathsf{f}^1_s) \overline{\Lambda}(s)
\right)\mathrm{diag}\, {\mathsf E}_f-{}\right.\\
\left.{}
- \Lambda^{\top}(s) \mathrm{diag}\,(\mathsf{f}^2_s) \right]ds\,;
\label{eq:sc_12}
\end{multline}

\vspace*{-12pt}

\noindent
\begin{multline}
\langle \mathsf{f}^2, \mathsf{f}^2 \rangle_t= \int\limits_0^t\Bigl[
\mathrm{diag}\,{\mathsf E}_{f^2}\mathrm{diag}\,( \overline{\Lambda}^{\top}(s)\mathsf{f}^1_s) -{}\\
{}- \mathrm{diag}\,(\mathsf{f}^2_s) 
\mathrm{diag}\,\lambda(s)  \mathrm{diag}\, (\mathsf{f}^2_s)
-  \mathrm{diag}\,(\mathsf{f}^2_s) \overline{\Lambda}(s)\mathrm{diag}\,{\mathsf E}_f - {}\\
{}-\mathrm{diag}\,{\mathsf E}_f 
\overline{\Lambda}^{\top}(s) \mathrm{diag}\, (\mathsf{f}^2_s)
\Bigl] ds\,.
\label{eq:sc_22}
\end{multline}

\vspace*{-3pt}

\noindent
\textit{Если для функции $g=g(e,y)$ выполнено условие}~(\ref{eq:cond_1}) \textit{и~для процесса 
$g(\theta_t,Y_t)$ верно представление типа}~(\ref{eq:sys_3}), \textit{то совместная квадратичная характеристика}: 

\pagebreak

\noindent
$$
\langle  \mu^f,\mu^g\rangle_t= \left[
\begin{array}{cc}
\langle \mathsf{f}^1, \mathsf{g}^1 \rangle_t & \langle \mathsf{f}^1, 
\mathsf{g}^2 \rangle_t \\
\langle \mathsf{f}^2, \mathsf{g}^1 \rangle_t & \langle \mathsf{f}^2, 
\mathsf{g}^2 \rangle_t
\end{array}
\right],
$$ 
\textit{где}
$\langle \mathsf{f}^1, \mathsf{g}^1 \rangle_t\hm = \langle \mathsf{f}^1, \mathsf{f}^1 \rangle_t$;
$\langle \mathsf{f}^1, \mathsf{g}^2 \rangle_t\hm = \langle \mathsf{g}^1, 
\mathsf{g}^2 \rangle_t$;
$\langle \mathsf{f}^2, \mathsf{g}^1 \rangle_t \hm= \langle \mathsf{f}^2, \mathsf{f}^1 \rangle_t$;

\vspace*{-6pt}

\noindent
\begin{multline}
\langle \mathsf{f}^2, \mathsf{g}^2 \rangle_t=
\int\limits_0^t\Bigl[
\mathrm{diag}\,{\mathsf E}_{fg}\mathrm{diag}\, ( \overline{\Lambda}^{\top}(s)\mathsf{f}^1_s) - {}\\
{}- \mathrm{diag}\,(\mathsf{f}^2_s) 
\mathrm{diag}\,\lambda(s)  \mathrm{diag}\, (\mathsf{g}^2_s)
-  \mathrm{diag}\,(\mathsf{f}^2_s) \overline{\Lambda}(s)\mathrm{diag}\,{\mathsf E}_g - {}\\
{}-\mathrm{diag}\,{\mathsf E}_f 
\overline{\Lambda}^{\top}(s) \mathrm{diag}\,(\mathsf{g}^2_s)
\Bigl]ds\,.
\label{eq:sc_fg}
\end{multline}

\noindent
Д\,о\,к\,а\,з\,а\,т\,е\,л\,ь\,с\,т\,в\,о\ теоремы~3 приведено в~приложении.

\section{Заключение}

В работе представлен класс скачкообразных процессов. Его первая блочная 
компонента~--- МСП с~конечным множеством состояний $\mathbb{S}^N$. Вторая блочная 
компонента синхронизирована по скачкам с~первой и~относительно первой блочной 
компоненты представляет собой последовательность независимых векторов.
Условные распределения второй блочной компоненты при фиксированной первой 
известны. Математическая модель такого рода может трактоваться как МСП 
с~конечным множеством <<сложных>> состояний~--- распределений случайных векторов. 


Для исследуемого класса процессов удалось получить ряд полезных результатов. Во-пер\-вых, 
эти процессы обладают марковским свойством. Во-вто\-рых, для них получены 
инфинитезимальный генератор и~сопряженный к~нему оператор. Эти результаты 
позволили получить мартингальное разложение функций от исследуемого класса 
процессов, а~также обобщения сис\-те\-мы уравнений Колмогорова, описывающей эволюцию 
как функции переходной вероятности, так и~одномерного распределения. В-третьих, в~случае если функция от исследуемого процесса имеет конечный второй момент, 
получены квадратичные характеристики мартингалов в~разложении.

Следует особо упомянуть, что произвольные функции процессов~$Z_t$ исследуемого 
класса удалось выразить через решение конечной сис\-те\-мы линейных СДУ. В~случае же 
МСП общего вида для такого описания потребуется более сложная сис\-те\-ма 
стохастических ин\-тег\-ро-диф\-фе\-рен\-ци\-аль\-ных уравнений.

Предложенные процессы планируется использовать в~качестве уравнений состояния 
в~стохастических сис\-те\-мах наблюдения и~решать задачи оптимальной фильтрации их 
состояний и~па\-ра\-мет\-ров по разнородным наблюдениям. Этому будут посвящены 
последующие работы.


  
{\small \section*{\raggedleft Приложение}
  \noindent
  Д\,о\,к\,а\,з\,а\,т\,е\,л\,ь\,с\,т\,в\,о\  теоремы~1.
Пусть $f(\theta_t,Y_t)$~--- произвольная 
ограниченная борелевская функция. Воспользуемся ее представлением в~виде 
$f(\theta_t,Y_t) \hm= \overline{f}(Y_t)\theta_t$. Для любых моментов времени $0 \hm\leqslant s \hm< t $ верна следующая цепочка равенств:
\begin{multline*}
\me{}{f(\theta_t,Y_t)|\ff_s} =\me{}{\overline{f}(Y_t)\theta_t|\ff_s}={} 
\\
{}={\sf E}\left\{\mathbf{I}(\NII_t - \NII_s = 0)\overline{f}(Y_t)\theta_t|\ff_s\right\} +{}\\
{}+
\me{}{\mathbf{I}(\NII_t - \NII_s > 0)\overline{f}(Y_t)\theta_t|\ff_s} =
\overline{f}(Y_s)\theta_s\mathcal{T}(s, t)\theta_s + {}\\ 
{}+   
    \sum\limits_{j=1}^N {\mathsf E} \left\{\mathbf{I}(\NII_t - \NII_s > 0)\overline{f}(Y_t)\theta_t 
|\ff_s, \theta_t = e_j,\right.\\
 \left.\NII_t - \NII_s > 0\right\} \pp{\NII_t - \NII_s > 0,\; \theta_t = e_j| 
\ff_s} ={} \\
{}=
\overline{f}(Y_s)\theta_s\mathcal{T}(s, t)\theta_s +
    \sum\limits_{j=1}^N  \int\limits_{\mathbb{R}^M} f(y, e_j)\pi_j(y)\,dy \times{}\\
    {}\times 
    \pp{\NII_t - \NII_s > 0,\; \theta_t = e_j | \theta_s} ={}
          \Bigg[ f(\theta_s,Y_s)
    \mathcal{T}(s, t)+{}\\
    {}+\sum\limits_{j=1}^N {\mathsf E}_f^{j} \left(\sum\limits_{i=1}^N 
\mathcal{P}_{ij}(s, t)e_i^\top - \mathcal{T}_j(s, t) e_j^\top\right)
    \Bigg]\theta_s = \mathbf{f}(\theta_s, Y_s).
\end{multline*}

Пусть в~момент времени~$s$ выполнены равенства $\theta_s \hm= e_i$ и~$Y_s\hm=y$. 
Рассмотрим произвольные $e_j \hm\in \mathbb{S}^N$ и~$B \hm\in 
\mathcal{B}(\mathbb{R}^M)$ и~найдем $\pp{\theta_t \hm= e_j, Y_t \hm\in B | \theta_s \hm= 
e_i,\;Y_s=y}$:
\begin{multline*}
\pp{\theta_t = e_j, Y_t \in B | \theta_s = e_i,\;Y_s=y}
={}\\
{}=\pp{\theta_t = e_j, Y_t \in B, \; \NII_t-\NII_s=0 | \theta_s = e_i,\;Y_s=y} + {}\\
{} +
\pp{\theta_t = e_j, Y_t \in B, \; \NII_t-\NII_s>0 | \theta_s = e_i,\;Y_s=y}={} \\ 
{}=
\delta_{ij} \mathcal{T}_i(s, t)\mathbf{I}_B(y) + (1-\delta_{ij}) 
\ppp_{ij}(s,t)\int\limits_B \pi_j(u)\,du +{}\\
{}+
\delta_{ij}\left[ \ppp_{ii}(s,t) -  \mathcal{T}_i(s, t)\right] \int\limits_B 
\pi_i(u)\,du.
%\label{eq:trpr}
\end{multline*}
Первое слагаемое
соответствует отсутствию скачков процесса~$\theta$ на интервале $[s,t]$. 
Вероятность этого события равна~$\mathcal{T}_i(s, t)$. Второе слагаемое 
соответствует переходу $e_i \hm\longrightarrow e_j $ компоненты~$\theta_t$ на 
$[s,t]$, который происходит с~вероятностью $\ppp_{ij}(s,t)$. Третье слагаемое 
соответствует переходу $e_i \hm\longrightarrow e_i $ компоненты~$\theta_t$ на 
$[s,t]$ за более чем один скачок. Вероятность этого события равна 
$\ppp_{ii}(s,t) \hm-  \mathcal{T}_i(s, t)$. Формула~(\ref{eq:eq_trfunc}) 
представляет собой матричную запись последнего выражения.
Теорема~1 доказана.


\smallskip

\noindent
Д\,о\,к\,а\,з\,а\,т\,е\,л\,ь\,с\,т\,в\,о\ теоремы~3.
Система~(\ref{eq:sys_3}) содержит $2N$~уравнений, из которых первые~$N$ образуют 
замкнутую систему и~описывают мартингальное разложение первой блочной компоненты~--- $\theta_t$~--- МСП с~множеством значений~$\mathbb{S}^N$~\cite{EAM_10}.

 Построим инфинитезимальный оператор одной из оставшихся $N$~компонент
   $e_n^{\top} \mathsf{f}^2_t= \overline{f}(Y)_t \mathrm{diag}\,e_n \theta_t
   $:
   $$
   \mathcal{A}_t(\overline{f}(Y)_t \mathrm{diag}\,e_n \theta_t)\hm =
  \left[
  \Lambda_{nn}(t)\overline{f}(Y)_t\hm + {\mathsf E}_f^n e_n^{\top} 
\overline{\Lambda}^{\top}(t)
  \right] \theta_t.
  $$
   Инфинитезимальный генератор компоненты $\mathsf{f}^2_t \hm= 
f(\theta_t,Y_t)\theta_t$ принимает вид:
$$  
 \mathcal{A}_t\mathsf{f}^2_t \hm= \mathrm{diag}\,{\mathsf E}_f \overline{\Lambda}^{\top}(t) 
\mathsf{f}^1_t \hm+ \mathrm{diag}\,\lambda(t) \mathsf{f}^2_t.
$$
   Из формулы Дынкина следует, что процесс
   $   \mathsf{f}^2_t \hm- \mathsf{f}^2_0\hm - \int\nolimits_0^t \mathcal{A}_s\mathsf{f}^2_s \,ds$
   представляет собой $\ff_t$-со\-гла\-со\-ван\-ный мартингал, что доказывает истинность 
представления процесса~$\mathsf{f}_t$ как решения системы уравнений~(\ref{eq:sys_3}).

   Далее докажем истинность формулы~(\ref{eq:sc_fg}): формулы~(\ref{eq:sc_11})--(\ref{eq:sc_22}) представляют собой ее частные случаи. Для этого воспользуемся 
тем фактом, что процессы~$Z_t$, $f(Z_t)$ и~$g(Z_t)$ принадлежат классу 
специальных семимартингалов, мартингальное разложение которых единственно~\cite{LSh_86}.

   Для скалярных процессов~$f(Z_t)$ и~$g(Z_t)$ построим соответствующие 
векторные процессы $\mathsf{f}_t \hm= \mathrm{col}\,(\mathsf{f}_t^1,\mathsf{f}_t^2)$ 
и~$\mathsf{g}_t \hm= \mathrm{col}\,(\mathsf{g}_t^1,\mathsf{g}_t^2)$, описываемые системами типа~(\ref{eq:sys_3}). 
Из определения этих процессов и~формулы Дынкина следует, что
  \begin{multline*}
   \mathsf{f}_t^2 (\mathsf{g}_t^2)^{\top} = \mathrm{diag}\,\mathsf{f}_t^2  \mathrm{diag}\,
\mathsf{g}_t^2 = \mathrm{diag}\, \mathsf{f}_0^2  \mathrm{diag}\, \mathsf{g}_0^2 +{}\\
{}+ \int\limits_0^t
   \left[
  \mathrm{diag}\, \lambda(s) \mathrm{diag}\, \mathsf{f}^2_s \mathrm{diag}\, \mathsf{g}^2_s + \mathrm{diag}\, E_{fg} 
\overline{\Lambda}^{\top}(s) \mathsf{f}^1_s
   \right]ds + \mu_t^1,\hspace*{-2.26256pt}
  % \label{eq:repr_1}
   \end{multline*}
   где $\mu_t^1$~--- некоторый $\ff_t$-со\-гла\-со\-ван\-ный мартингал.

   С другой стороны, мартингальное разложение произведения $\mathsf{f}_t^2 
(\mathsf{g}_t^2)^{\top} $ может быть получено с~по\-мощью обобщенного правила Ито~\cite{LSh_86}:
    \begin{multline*}
   \mathsf{f}_t^2 (\mathsf{g}_t^2)^{\top} =
   \mathrm{diag}\, \mathsf{f}_0^2  \mathrm{diag}\, \mathsf{g}_0^2 + \!\int\limits_0^t\! \mathsf{f}_{s-}^2d(\mathsf{g}_{s-}^2)^{\top} 
   +\!\int\limits_0^t \! d \mathsf{f}_{s}^2(\mathsf{g}_{s-}^2)^{\top} +{}\\
   {}+ \langle 
\mathsf{f}^2, \mathsf{g}^2\rangle_t + \mu_t^2 =
        \mathrm{diag}\, \mathsf{f}_0^2  \mathrm{diag}\, \mathsf{g}_0^2 + {}\\
        {} + \int\limits_0^t
    \Bigl[ \mathrm{diag}\, \mathsf{f}_{s}^2 \overline{\Lambda}(s) \mathrm{diag}\, E_g +
    \mathrm{diag}\, E_f \overline{\Lambda}^{\top}(s) \mathrm{diag}\, \mathsf{g}_{s}^2 +{}\\
    {}+
    2 \mathrm{diag}\, \lambda(s) \mathrm{diag}\, \mathsf{f}^2_s \mathrm{diag}\, \mathsf{g}^2_s
    \Bigl]ds +   \langle \mathsf{f}^2, \mathsf{g}^2\rangle_t + \mu_t^3,
   %\label{eq:repr_2}
   \end{multline*}
   где $\mu_t^2$ и~$\mu_t^3$~--- некоторые $\ff_t$-согласованные мартингалы. Из 
единственности разложения следует $\ppp$-п.~н.\ выполнение равенства:
       \begin{multline*}
   \int\limits_0^t
   \left[
   \mathrm{diag}\, \lambda(s) \mathrm{diag}\, \mathsf{f}^2_s \mathrm{diag}\, \mathsf{g}^2_s + \mathrm{diag}\,{\mathsf E}_{fg} 
\overline{\Lambda}^{\top}(s) \mathsf{f}^1_s
   \right]ds = {}\\
   {}=  \langle \mathsf{f}^2, \mathsf{g}^2\rangle_t +
   \int\limits_0^t
    \Bigl[ \mathrm{diag}\,\mathsf{f}_{s}^2 \overline{\Lambda}(s) \mathrm{diag}\,{\mathsf E}_g +{}\\
    {}+
    \mathrm{diag}\,{\mathsf E}_f \overline{\Lambda}^{\top}(s)\mathrm{diag}\, \mathsf{g}_{s}^2 +
    2\,\mathrm{diag}\, \lambda(s) \mathrm{diag}\,\mathsf{f}^2_s \mathrm{diag}\, \mathsf{g}^2_s
    \Bigl]ds.
   \end{multline*}
   Отсюда в~силу коммутативности диагональных матриц и~тождества 
$\mathsf{f}_{t}^1 \hm\equiv \mathsf{g}_{t}^1\hm\equiv \theta_t$ следует истинность 
равенства~(\ref{eq:sc_fg}).
    Теорема~3 доказана.


}

{\small\frenchspacing
 {\baselineskip=11.5pt
 %\addcontentsline{toc}{section}{References}
 \begin{thebibliography}{99}
 \bibitem{W_93} %1
\Au{White D.}
A~survey of applications of Markov decision processes~//
J.~Oper. Res. Soc., 1993. Vol.~44. No.\,11. P.~1073--1096. doi: 10.1057/jors.1993.181.

\bibitem{EM_02}%2
 \Au{Ephraim Y., Merhav~N.} Hidden Markov processes~// IEEE T. 
Inform. Theory, 2002. Vol.~48. No.~6. P.~1518--1569.
doi: 10.1109/TIT.2002.1003838.

\bibitem{DLW_23}%3
{\it Dong S., Liu M., Wu Z.} A~survey on hidden Markov jump systems: 
Asynchronous control and filtering~// Int. J. Syst. Sci., 
2023. Vol.~54. No.\,6. P.~1360--1376. doi: 10.1109/ TIT.2002.1003838.

   \bibitem{D_63} %4
\Au{Дынкин Е.} Марковские процессы.~--- М.:~Наука, 1963. 862~с.

   \bibitem{GS_73_2} %5
\Au{Гихман И., Скороход~А.} Теория случайных процессов.~--- М.: Наука, 
1973.  Т.~2. 640~с.

   \bibitem{LSh_86} %6
\Au{Липцер Р., Ширяев~А.} Теория мартингалов.~--- М.:~Физматлит, 1986. 512~с.

 \bibitem{EAM_10} %7
\Au{Elliott R.,  Aggoun~L., Moore~J.} Hidden Markov models: Estimation and 
control.~--- New York, NY, USA: Springer, 2010. 382~p.

 \bibitem{WD_79} %8
\Au{Wan C., Davis~M.} Existence of optimal controls for stochastic jump 
processes~// SIAM J. Control Optim., 1979. Vol.~17. No.\,4. P.~511--524. doi: 10.1137/0317037.

\bibitem{CGT_02} %9
\Au{Ceci C., Gerardi~A., Tardelli~P.} Existence of optimal controls for 
partially observed jump processes~// Acta Appl. Math., 2002. Vol.~74. Iss.~2. P.~155--175.
doi: 10.1023/ A:1020669212384.

  \bibitem{B_14} %10
\Au{Борисов A.} Применение алгоритмов оптимальной фильтрации для решения задачи 
мониторинга доступности удаленного сервера~// Информатика и~её применения, 2014. 
Т.~8. Вып.~3. С.~53--69. doi: 10.14375/ 19922264140307. EDN: SMPBCB.


  \bibitem{B_16_1} %11
\Au{Борисов A.} Применение методов оптимальной фильтрации для оперативного 
оценивания состояний сетей массового обслуживания~// Автоматика и~телемеханика, 2016. Вып.~2. С.~115--141. 

  \bibitem{B_16_2} %12
\Au{Борисов A., Босов~А., Миллер~Г.} Моделирование и~мониторинг состояния VoIP-со\-еди\-не\-ния~// 
Информатика и~её применения, 2016. Т.~10. Вып.~2. С.~2--13. doi: 
10.14357/19922264160201. EDN: WCBWUH.

 

  \bibitem{B_21_1} %13
\Au{Borisov A., Bosov~A., Miller~G., Sokolov~I.} Partial diffusion Markov model 
of heterogeneous TCP link: Optimization with incomplete information~// 
Mathematics, 2021. Vol.~9. Art.~1632. 31~p. doi: 10.3390/math9141632.

  \bibitem{G_23_1} %14
\Au{Горшенин A., Горбунов~С., Волканов~Д.} О~кластеризации объектов сетевой 
вычислительной инфраструктуры на основе анализа статистических аномалий в~трафике~// Информатика и~её применения, 2023. Т.~17. Вып.~3. С.~76--87. doi: 
10.14357/19922264230311. EDN: XHTMVI.


  \bibitem{B_04_1} %15
\Au{Борисов A.} Анализ и~оценивание состояний специальных марковских 
скачкообразных процессов. I:~мар-\linebreak\vspace*{-12pt}

\pagebreak

\noindent
тингальное представление~// Автоматика 
и~телемеханика, 2004. Вып.~1. С.~50--65. 

\bibitem{LJ_05} %16
\Au{Rong Li X., Jilkov~V.}
Survey of maneuvering target tracking. Part V. Multiple-model methods~//
IEEE T. Aero.  Elec. Sys., 2005. Vol.~41. No.\,4. 
P.~1255--1321. doi: 10.1109/ TAES.2005.1561886.

  \bibitem{mamon2014hidden} %17
\Au{Mamon R., Elliott~R.}
Hidden Markov models in finance: Further developments and
applications.~--- New York, NY, USA: Springer, 2014.  Vol.~II. 283~p.

  \bibitem{DPR_08} %18
\Au{Dainotti A., Pescap$\acute{\mbox{e}}$~A., Rossi P., \textit{et al.}}
Internet traffic modeling by means of hidden Markov models~//
Comput. Netw., 2008. Vol.~52. P.~2645--2662. doi: 10.1016/j.comnet.2008.05.004.

 \bibitem{IK_06} %19
 \Au{Ishikawa Y., Kunita~H.} Malliavin calculus on the Wiener--Poisson space and 
its application to canonical SDE with jumps~// Stoch. Proc. Appl., 2006. Vol.~116. No.~12. P.~1743--1769. doi: 10.1016/j.spa.2006.04.013.

\end{thebibliography}

 }
 }

\end{multicols}

\vspace*{-6pt}

\hfill{\small\textit{Поступила в~редакцию 19.04.24}}

\vspace*{10pt}

%\pagebreak

%\newpage

%\vspace*{-28pt}

\hrule

\vspace*{2pt}

\hrule



\def\tit{PROBABILISTIC ANALYSIS OF~A~CLASS OF~MARKOV~JUMP~PROCESSES}


\def\titkol{Probabilistic analysis of~a~class of~Markov jump  processes}


\def\aut{A.\,V.~Borisov$^{1,2}$, Yu.\,N.~Kurinov$^{2}$, and~R.\,L.~Smeliansky$^{2}$}

\def\autkol{A.\,V.~Borisov, Yu.\,N.~Kurinov, and~R.\,L.~Smeliansky}

\titel{\tit}{\aut}{\autkol}{\titkol}

\vspace*{-8pt}


\noindent 
$^1$Federal Research Center ``Computer Science and Control'' of the Russian 
Academy of Sciences, 44-2 Vavilov\linebreak
$\hphantom{^1}$Str., Moscow 119333, Russian Federation

\noindent 
$^2$M.\,V.~Lomonosov Moscow State University, 1-52~Leninskie Gory, GSP-1, Moscow
119991, Russian Federation


\def\leftfootline{\small{\textbf{\thepage}
\hfill INFORMATIKA I EE PRIMENENIYA~--- INFORMATICS AND
APPLICATIONS\ \ \ 2024\ \ \ volume~18\ \ \ issue\ 3}
}%
 \def\rightfootline{\small{INFORMATIKA I EE PRIMENENIYA~---
INFORMATICS AND APPLICATIONS\ \ \ 2024\ \ \ volume~18\ \ \ issue\ 3
\hfill \textbf{\thepage}}}

\vspace*{4pt}






\Abste{The paper introduces a class of the jump processes. 
The first compound component represents a Markov jump process with a finite 
state space. The second compound component jumps synchronously with the first 
one.
Given the first component trajectory, the second component forms a sequence of 
independent random vectors.
The corresponding conditional distributions are known and have intersecting 
support sets. This makes impossible the exact recovery of the first process 
component by the second one. The authors prove the Markov property for the considered 
class of random processes and obtain a collection of their probability 
characteristics. It includes the infinitesimal generator and its conjugate 
operator. Their knowledge makes possible the construction of the Kolmogorov 
equation system describing the evolution of the process probability 
distribution. Also,  a~martingale decomposition for an arbitrary 
function of the considered process was derived. It can be characterized by the solution to a 
system of linear stochastic differential equations with martingales on the right 
side. If the functions of the investigated process have finite moments of the 
second order, one may obtain the quadratic characteristics of martingales.}


\KWE{Markov jump process; infinitisemal generator; 
martingale decomposition; stochastic differential equation}



\DOI{10.14357/19922264240304}{XPVTGJ}

\vspace*{-12pt}


    
      \Ack

\vspace*{-3pt}

\noindent
The work was done with the support of MSU 
Program of Development, Project No.\,23-SCH03-03. The research was carried out 
using the infrastructure of the Shared Research Facilities ``High Performance 
Computing and Big Data'' (CKP ``Informatics'') of FRC CSC RAS (Moscow).


  \begin{multicols}{2}

\renewcommand{\bibname}{\protect\rmfamily References}
%\renewcommand{\bibname}{\large\protect\rm References}

{\small\frenchspacing
 {%\baselineskip=10.8pt
 \addcontentsline{toc}{section}{References}
 \begin{thebibliography}{99} 

%1
\bibitem{W_93-1}
\Aue{White, D.} 1993.
A~survey of applications of Markov decision processes. \textit{J. Oper. Res. 
Soc.} 44(11):1073--1096.
doi: 10.1057/jors.1993.181.

%2
\bibitem{EM_02-1}
\Aue{Ephraim, Y., and N.~Merhav.} 2002.
Hidden Markov processes. \textit{IEEE T. Inform. Theory} 48(6):1518--1569. 
doi: 10.1109/TIT.2002.1003838.

%3
\bibitem{DLW_23-1}
\Aue{Dong, S., M.~Liu, and Z.~Wu.} 2023.
A survey on hidden Markov jump systems: Asynchronous control and filtering. 
\textit{Int. J. Syst. Sci.} 54(6):1360--1376.
doi: 10.1080/ 00207721.2023.2171710.

%4
\bibitem{D_63-1}
\Aue{Dynkin, E.} 1963. \textit{Markovskie protsessy} [Markov processes]. Moscow: 
Nauka. 862~p.

%5
\bibitem{GS_73_2-1}
\Aue{Gikhman, I.\,I., and A.\,V.~Skorokhod.} 1973. 
\textit{Teoriya sluchaynykh protsessov} [Theory of random processes]. Moscow: 
Nauka. Vol.~2. 640~p.

%6
\bibitem{LSh_86-1}
\Aue{Liptser, R.\,Sh., and A.\,N.~Shiryayev.} 1989. \textit{Theory of martingales}. New 
York, NY: Springer. 812~p.



%7
\bibitem{EAM_10-1}
\Aue{Elliott, R., L.~Aggoun, and J.~Moore.} 2010. 
\textit{Hidden Markov models: Estimation and control}. New York, NY: Springer. 
382~p.

%8
\bibitem{WD_79-1}
\Aue{Wan, C., and M.~Davis.} 1979. 
Existence of optimal controls for stochastic jump processes. 
\textit{SIAM J. Control Optim.} 17(4):511--524.
doi: 10.1137/0317037.

%9
\bibitem{CGT_02-1}
\Aue{Ceci, C., A.~Gerardi, and P.~Tardelli.} 2002. 
Existence of optimal controls for partially observed jump processes. 
\textit{Acta Appl. Math.} 74(2):155--175.
doi: 10.1023/ A:1020669212384.

%10
\bibitem{B_14-1-1}
\Aue{Borisov, A.} 2014. 
Primenenie algoritmov optimal'noy fil'tratsii dlya resheniya zadachi 
monitoringa dostupnosti udalennogo servera [Monitoring remote server 
accessibility: The optimal filtering approach]. 
\textit{Informatika i ee primeneniya~--- Inform. Appl.} 8(3):53--69. doi: 
10.14375/19922264140307. EDN: SMPBCB.

%11
\bibitem{B_16_1-1-1}
\Aue{Borisov, A.\,V.} 2016. 
Application of optimal filtering methods for online of queueing network states. 
\textit{Automat. Rem. Contr.} 77(2):277--296.
doi: 10.1134/S0005117916020053. EDN: WPMCUD.

%12
\bibitem{B_16_2-1}
\Aue{Borisov, A., A.~Bosov, and G.~Miller.} 2016.
Mo\-de\-li\-ro\-va\-nie i~monitoring sostoyaniya VoIP-soedineniya [Modeling and 
monitoring of VoIP connection]. \textit{Informatika i~ee primeneniya~--- Inform. 
Appl.} 10(2):2--13. 
doi: 10.14357/19922264160201. EDN: WCBWUH.



%13
\bibitem{B_21_1-1}
\Aue{Borisov, A., A.~Bosov, G.~Miller, and I.~Sokolov.} 2021.
Partial diffusion Markov model of heterogeneous TCP link: Optimization with 
incomplete information. 
\textit{Mathematics} 9:1632. 31~p. doi: 10.3390/math9141632.

%14
\bibitem{G_23_1-1}
\Aue{Gorshenin, A.\,K., S.\,A.~Gorbunov, and D.\,Yu.~Volkanov.} 2023.
O~klasterizatsii ob''ektov setevoy vychislitel'noy inf\-ra\-struk\-tu\-ry na osnove 
analiza statisticheskikh anomaliy v~trafike 
[Toward clustering of network computing infrastructure objects based on analysis 
of statistical anomalies in network traffic]. 
\textit{Informatika i~ee primeneniya~--- Inform. Appl.} 17(3):76--87. 
doi: 10.14357/19922264230311. EDN: XHTMVI.

%15
\bibitem{B_04_1-1}
\Aue{Borisov, A.\,V.} 2004.
Analysis and estimation of the states of special jump Markov processes. 
I.~Martingale representation.
\textit{Automat. Rem. Contr.} 65(1):44--57.  
doi: 10.1023/B:AURC.0000011689.11915.24. EDN: MHPDNL.


%16
\bibitem{LJ_05-1}
\Aue{Rong Li, X., and V.~Jilkov.} 2005. 
Survey of maneuvering target tracking. Part~V. Multiple-model methods. 
\textit{IEEE T. Aero. Elec. Sys.} 41(4):1255--1321.
doi: 10.1109/ TAES.2005.1561886.

%17
\bibitem{mamon2014hidden-1}
\Aue{Mamon, R.\,S., and R.~Elliott.} 2014. 
\textit{Hidden Markov models in finance: Further developments and applications}. 
New York, NY: Springer. Vol.~II. 283~p.

%18
\bibitem{DPR_08-1}
\Aue{Dainotti, A., A. Pescap$\acute{\mbox{e}}$, P.~Rossi, \textit{et al.}} 2008.
Internet traffic modeling by means of hidden Markov models. 
\textit{Comput. Netw.} 52(14):2645--2662.
doi: 10.1016/j.\linebreak comnet.2008.05.004.

%19
\bibitem{IK_06-1}
\Aue{Ishikawa, Y., and H.~Kunita.} 2006. 
Malliavin calculus on the Wiener--Poisson space and its application to canonical 
SDE with jumps. 
\textit{Stoch. Proc. Appl.} 116(12):1743--1769.
doi: 10.1016/j.spa.2006.04.013.

\end{thebibliography}

 }
 }

\end{multicols}

\vspace*{-6pt}

\hfill{\small\textit{Received April 19, 2024}} 

\vspace*{-18pt}

\Contr

\vspace*{-3pt}

\noindent
\textbf{Borisov Andrey V.} (b.\ 1965)~--- Doctor of Science in physics and 
mathematics, principal scientist, 
Federal Research Center ``Computer Science and Control'' of the Russian Academy of 
Sciences,  44-2~Vavilov Str., Moscow 119333, Russian Federation; professor, Department of 
Mathematical Statistics, 
Faculty of Computational Mathematics and Cybernetics, M.\,V.~Lomonosov Moscow 
State University, 1-52~Leninskie Gory,  GSP-1, Moscow 119991, Russian Federation; \mbox{aborisov@frccsc.ru}

\vspace*{3pt}

\noindent
\textbf{Kurinov Yuri N.} (b.\ 2002)~--- bachelor student, Department of 
Mathematical Statistics, Faculty of Computational Mathematics and Cybernetics, 
M.\,V.~Lomonosov Moscow State University, 1-52~Leninskie Gory,
GSP-1, Moscow 119991, Russian Federation, \mbox{kurinovurij@gmail.com}

\vspace*{3pt}

\noindent
\textbf{Smeliansky Ruslan L.} (b.\ 1950)~--- Doctor of Science in physics and 
mathematics, professor, Corresponding Member of the Russian Academy of Sciences, head of
department, Faculty of Computational Mathematics and Cybernetics, 
M.\,V.~Lomonosov Moscow State University, 1-52~Leninskie Gory,
GSP-1, Moscow 119991, Russian Federation; \mbox{smel@cs.msu.ru}


\label{end\stat}

\renewcommand{\bibname}{\protect\rm Литература} 