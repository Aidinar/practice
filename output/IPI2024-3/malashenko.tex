\def\stat{malashenko}

\def\tit{СРАВНИТЕЛЬНЫЙ АНАЛИЗ ПОКАЗАТЕЛЕЙ ФУНКЦИОНИРОВАНИЯ СЕТИ ПРИ 
ПОВРЕЖДЕНИИ УЗЛОВ}

\def\titkol{Сравнительный анализ показателей функционирования сети при 
повреждении узлов}

\def\aut{Ю.\,Е.~Малашенко$^1$, И.\,А.~Назарова$^2$}

\def\autkol{Ю.\,Е.~Малашенко, И.\,А.~Назарова}

\titel{\tit}{\aut}{\autkol}{\titkol}

\index{Малашенко Ю.\,Е.}
\index{Назарова И.\,А.}
\index{Malashenko Yu.\,E.}
\index{Nazarova I.\,A.}


%{\renewcommand{\thefootnote}{\fnsymbol{footnote}} \footnotetext[1]
%{Работа 
%выполнена при поддержке Программы развития МГУ, проект №\,23-Ш03-03. При анализе 
%данных использовалась инфраструктура Центра коллективного пользования 
%<<Высокопроизводительные вычисления и~большие данные>> 
%(ЦКП <<Информатика>>) ФИЦ ИУ РАН (г.~Москва)}}


\renewcommand{\thefootnote}{\arabic{footnote}}
\footnotetext[1]{Федеральный исследовательский центр <<Информатика 
и~управление>> Российской академии наук, \mbox{malash09@ccas.ru}}
\footnotetext[2]{Федеральный исследовательский центр <<Информатика 
и~управление>> Российской академии наук, \mbox{irina-nazar@yandex.ru}}


\vspace*{-12pt}



\Abst{На модели  многопользовательской системы связи анализируются 
изменения показателей функционирования при повреждении отдельных узлов сети. 
В~ходе вычислительных экспериментов отслеживается  изменение  удельных затрат 
ресурсов и~загрузка ребер при одновременной передаче межузловых потоков 
в~поврежденной сети. Для оценки последствий каждого повреждения полученные 
значения сравниваются с~исходными. Для каждого поврежденного узла подсчитывается 
увеличение удельных затрат на передачу межузловых потоков. Определяется  число 
корреспондентов, оставшихся без связи.  Формируется  набор гарантированных 
оценок предельно возможных загрузок ребер сети при любом из повреждений. 
Подсчитываются усредненные   показатели для всех повреждений. На основе 
агрегированных расчетных показателей строятся итоговые диаграммы для сетей с~различными структурными особенностями.}
 
 
 \KW{потоковая модель сети связи; оценка повреждений узлов; 
загрузка ребер}


\DOI{10.14357/19922264240307}{YUEGZT}
  
\vspace*{-3pt}


\vskip 10pt plus 9pt minus 6pt

\thispagestyle{headings}

\begin{multicols}{2}

\label{st\stat}



\section{Введение}

В последнее время возрос интерес к~оценке устойчивости функционирования 
многопользовательских сетевых систем связи  при повреждениях и~перегрузках~[1]. 
В рамках предлагаемой в~на\-сто\-ящей работе модели  обеспечение информационного 
обмена между уз\-ла\-ми-кор\-рес\-пон\-ден\-та\-ми рас\-смат\-ри\-ва\-ет\-ся как основная функция сети 
связи, а~соответствующие величины потоков~---  как функциональные 
характеристики сис\-те\-мы~[2--4].

Оценка показателей функционирования в~поврежденной сети проводится исходя из 
изменения удельных затрат на передачу потока и~загрузки ребер. Вычисляются 
удельные затраты и~загрузка ребер при одновременной передаче межузловых потоков 
между всеми парами корреспондентов в~поврежденной сети. Для оценки последствий 
каждого повреждения  полученные значения сравниваются с~исходными.  В~частности, 
в~качестве оценки ущерба используются: доля пар, для которых  соединение в~сети  
отсутствует,  и~среднее значение относительного увеличения удельных затрат на 
передачу межузлового потока.

Для получения многокритериальных и~гарантированных оценок функционирования 
поврежденной сети анализируется изменение загрузки ребер при одновременной 
передаче всех межузловых \mbox{многопродуктовых} потоков. В~рамках модели 
последовательно рассматривается разрушение каждой вершины сети и~формируется  
вектор, ха\-рак\-те\-ри\-зу\-ющий  изменения загрузки всех ребер поврежденной сети. 
Компоненты указанного вектора подсчитываются как отношение значений загрузки 
ребра в~исходной и~по\-вреж\-ден\-ной сетях.

Для всех повреждений на основе лексикографически упорядоченных 
последовательностей стро\-ятся диаграммы, позволяющие получать гарантированные 
оценки  загрузки ребер сети. Далее с~по\-мощью  агрегированных расчетных 
показателей\linebreak строятся недоминируемые многокритериальные оценки ущерба и~изменения 
загрузки при по\-вреж\-де\-ни\-ях, а~так\-же  итоговые диаграммы.
В~на\-сто\-ящей работе вы\-чис\-ли\-тель\-ные эксперименты проводились в~рамках методологии 
исследования \mbox{операций}~[5--7] с~использованием методов оптимизации и~потокового 
программирования~[8--10].

\vspace*{-3pt}

\section{Математическая модель}

Для описания многопользовательской сетевой\linebreak системы связи  воспользуемся 
следующей математической записью модели передачи многопродуктового потока.
Сеть~$G$ задается множествами\linebreak  $\langle V,\ R, \ U,\ P \rangle$:
узлов (вершин) сети  $V \hm= \{ v_1, v_2, \ldots, v_n, \ldots, v_N \}$;
неориентированных ребер $R \hm= \{ r_1, r_2, \ldots, r_k, \ldots, r_E \}$;
ориентированных дуг  $U \hm= \{ u_1, u_2, \ldots, u_k, \ldots, u_{2E}\}$;
пар уз\-лов-кор\-рес\-пон\-ден\-тов $P \hm= \{ p_1, p_2, \ldots, p_M\}$.
Предполагается, что в~сети отсутствуют петли и~сдвоенные ребра.

Ребро $r_k\hm \in R$ соединяет \textit{смежные} вершины~$v_{n_k}$ и~$v_{j_k}$.
Каждому ребру~$r_k$ ставятся в~соответствие две ориентированные дуги~$u_k$  
и~$u_{k+E}$ из множества~$U$.
Дуги $\{u_k, u_{k+E}\}$ определяют прямое и~обратное направление передачи потока 
по  ребру~$r_k$ между концевыми вершинами~$v_{n_k}$ и~$v_{j_k}$. Для каждой 
вершины $v_n$ формируется список $K(n)$ номеров инцидентных ей ребер: $K(n)\hm = 
\{k_1, k_2, \ldots, k_{a(n)}\}$, где $a(n)$~--- число инцидентных ребер для~$v_n$.

В многопользовательской сети~$G$ рассматривается $M \hm= N (N\hm-1)$ независимых, 
невзаимозаменяемых и~равноправных межузловых потоков различных видов.
Каждой паре уз\-лов-кор\-рес\-пон\-ден\-тов~$p_m$ из множества~$P$ соответствуют:
вер\-ши\-на-ис\-точ\-ник с~номером~$s_m$,  из~$s_m$  входной поток $m$-го вида поступает в~сеть;
вер\-ши\-на-при\-ем\-ник с~номером~${t_m}$, из~${t_m}$ поток $m$-го вида покидает сеть.
Для каждой вершины $v_n \hm\in V$, $n \hm= \overline{1, N}$, в~подмножество~$P(v_n)$ входят 
все па\-ры-кор\-рес\-пон\-ден\-ты, для которых~$v_n$ служит уз\-лом-ис\-точ\-ни\-ком:
$$
P(v_n) = \left\{ p_m | s_m= n, t_m \not = n, t_m = \overline{1, N} \right\},
$$
а для каждого~$P(v_n)$~--- список номеров~$M(n)$ пар~$p_m$, входящих в~подмножество~$P(v_n)$:
$M(n) \hm= \{m_1(n), m_2(n), \ldots, m_{N-1}(n)\}.$

Обозначим через $z_m$ величину \textit{межузлового} потока $m$-го вида, 
поступающего в~сеть через узел с~номером~$s_m$ и~покидающего сеть из узла с~номером~$t_m$;
$x_{mk}$ и~$x_{m(k + E)}$~---  поток $m$-го вида, который передается по дугам 
$u_k$ и~$u_{k + E}$ согласно направлению передачи, $x_{mk} \hm\ge 0$, $x_{m(k + 
E)}\hm\ge 0$, $m \hm= \overline{1, M}$, $k \hm= \overline {1, E}$;
$S(v_n)$~--- множество номеров исходящих дуг, по ним поток покидает узел~$v_n$;
$T(v_n)$~--- множество номеров входящих дуг, по ним поток поступает в~узел~$v_n$.
Состав множеств $S(v_n)$ и~$T(v_n)$ однозначно формируется в~ходе выполнения 
следующей процедуры. Пусть некоторое ребро $r_k \hm\in R$ соединяет вершины 
с~номерами~$n$ и~$j$, такими что $n \hm< j$. Тогда ориентированная дуга $u_k\hm = (v_n, 
v_j)$, направленная из вершины~$v_n$ в~$v_j$, считается \textit{исходящей} из 
вершины~$v_{n}$ и~ее номер~$k$ заносится в~множество~$S(v_n)$, 
а~дуга~$u_{k+E}$, направленная из~$v_j$ в~$v_n$,~--- \textit{входящей} для~$v_{n}$ и~ее номер~$k\hm+E$ помещается в~список~$T(v_n)$.
Дуга~$u_k$ является \textit{входящей} для~$v_j$, и~ее номер~$k$ попадает 
в~$T(v_j)$, а~дуга~$u_{k+E}$~--- \textit{исходящей}, и~номер~$k\hm+E$ вносится в~список исходящих дуг~$S(v_j)$.

Во всех узлах сети $v_n \hm\in V$, $n \hm= \overline{1,N}$,  для каждого вида потока должны 
выполняться условия сохранения потоков:

\vspace*{-4pt}

\noindent
\begin{multline}
\sum\limits_{i \in S(v_n)}{x_{mi}} - \sum\limits_{i \in T(v_n)}{x_{mi}} ={}\\
{}=
\begin{cases}
 z_m, & \mbox{если}\ v_n = v_{s_m}; \\
- z_m, & \mbox{если}\  v_n = v_{t_m}; \\
 0 & \mbox{в\ остальных\ случаях,}
\end{cases}\\
n = \overline {1, N}, \ m = \overline {1, M}, \ x_{mi} \ge 0, \ z_m \ge 0.
\label{e1-mal}
\end{multline}

\vspace*{-16pt}

\columnbreak

\noindent
Величина $z_m$ равна входному межузловому потоку $m$-го вида, проходящему от 
источника~$s_m$ к~приемнику~$t_m$ пары~$p_m$ при распределении  потоков~$x_{mi}$ 
по дугам сети.

Каждому ребру $r_k \in R$ приписывается неотрицательное число~$d_k$~--- 
суммарный предельно допустимый поток, который можно передать по реб\-ру~$r_k$ 
в~обоих направлениях. В~исходной сети \mbox{компоненты} вектора пропускных способностей   
$\mathbf{d}\hm = (d_1, d_2, \ldots, d_k, \ldots, d_E)$~---  положительные числа $d_k \hm> 
0$.  Вектор~$\mathbf{d}$ определяет следующие ограничения на сумму потоков всех 
видов, передаваемых по ребру~$r_k$ одновременно:
\begin{multline}
\sum\limits_{m=1}^{M} {(x_{mk}+ x_{m(k+E)})} \le d_k,  \enskip x_{mk} \ge 0, \\
x_{m(k+E)} \ge 0, \enskip  k =\overline{1, E}. \label{e2-mal}
\end{multline}

Ограничения~(\ref{e1-mal}) и~(\ref{e2-mal}) задают множество  допустимых значений вектора межузловых 
потоков
$\mathbf{z} = (z_1, z_2, \ldots, z_m, \ldots , z_M)$:
\begin{multline*}
 \mathcal{Z}(\mathbf{d}) = \{\mathbf{z} \ge 0 \ |\  \e \ \mathbf{x} \ge 0: \\ 
(\mathbf{z}, \mathbf{x})  \mbox{ удовлетворяют\ } (1), (2)\}.
\end{multline*}


\section{Повреждение узлов}

При проведении вычислительного эксперимента вначале подсчитываются межузловые 
потоки\linebreak в~не\-по\-вреж\-ден\-ной сети~$G(0)$ с~пропускными способностями~$d_k(0)$. 
Предполагается, что поток одного вида передается одновременно по всем маршрутам, 
содержащим минимальное число ребер (\mbox{далее}~--- MER-маршруты, от \textit{англ.}\ mi\-ni\-mum 
edge route).
Для оценки величины <<расщепленного>>  потока для каждой пары узлов $p_m \hm= 
(s_m, t_m)$ в~сети~$G(0)$ формируется набор $H_m(1)$ путей, которые далее 
рассматриваются как MER-марш\-ру\-ты передачи $m$-го вида потока:
$$
H_m(1) = \left\{ h_m^1(1), h_m^2(1), \ldots, h_m^j(1), \ldots, h_m^{J_m(1)}(1)
\right\},
$$
где $h_m^j(1)$~--- список номеров дуг в~$j$-м  пути в~сети~$G(0)$ между узлами~$s_m$ 
и~$t_m$;  $\iota_m(1)$~--- число ребер в~MER-марш\-ру\-те $h_m^j(1)$; 
$J_m(1)$~--- число MER-марш\-ру\-тов для $m$-й пары.

Для каждой пары $p_m \hm\in P$ по всем MER-марш\-ру\-там из~$H_m(1)$ передается 
единичный межузловой поток~$z_m$ и~подсчитываются значения индикаторной функции:
$$ 
\eta_k^j(m) = \begin{cases}
 1, & \mbox{если} \ k \in h_m^j(1); \\
 0 & \mbox{в\ остальных\ случаях.}
\end{cases} 
$$
Определяются  дуговые потоки для пары~$p_m$:
\begin{equation}
x_{mk}^0 (1) = \sum\limits_{j=1}^{J_m(1)} \eta_k^j(m), \quad m = \overline{1, M}\,, \ \ k = 
\overline{1, 2E}\,. \label{e3-mal}
\end{equation}

Межузловой поток по MER-марш\-ру\-там (далее~---  MER-по\-ток) $z_{m}^0(1)$ между 
узлами~$s_m$ и~$t_m$ вычисляется  по формулам~(\ref{e1-mal}), (\ref{e3-mal}). Рассчитывается  
нормирующий коэффициент
$$
\omega_m^0(1) = \fr{1}{z_{m}^0(1)}, \quad z_{m}^0(1) \not = 0, \quad \ m = \overline{1, M}\,, 
$$
и дуговые потоки:
\begin{equation}
x_{mk}^0 = \omega_m^0(1) x_{mk}^0(1), \quad m = \overline{1, M}\,, \ \ k = \overline{1, 2E}\,. 
\label{e4-mal}
\end{equation}
При передаче всех потоков~$x_{mk}^0$ по ребрам сети межузловой поток из узла~$s_m$ в~узел~$t_m$ равен единице для всех $p_m \hm\in P$.

На основании~(\ref{e4-mal}) для получения оценки максимального значения равных межузловых 
потоков формируется и~решается задача~1.

\smallskip

\noindent
\textbf{Задача~1.} Найти $\tilde{\alpha} \hm= \max\nolimits_{\alpha} \alpha$
при условиях:
\begin{equation*}
  \alpha \sum\limits_{m = 1}^ M [ x_{mk}^0+  x_{m(k+E)}^0] 
\le d_k(0), \quad
 \alpha \ge 0, \ \  k =\overline{1, E}\,.
\end{equation*}
С помощью решения задачи~1 для всех $p_m \hm\in P$  определяется вектор 
$\tilde{\mathbf{z}}(\tilde{\alpha})$, все компоненты которого равны 
$\tilde{\alpha}$, т.\,е. 
\begin{alignat*}{2}
\tilde{z}_m &= \tilde{\alpha},&\enskip m&=\overline{1, M}\,,  \\[3pt] 
\tilde{x}_{mk} &= \tilde{\alpha}x_{mk}^0, &\enskip m&=\overline{1, M}\,,\ k =\overline{1, 2E}\,.
\end{alignat*}

Для каждой пары узлов-кор\-рес\-пон\-ден\-тов $p_m \hm\in P$, для полученного допустимого 
межузлового потока~$\tilde{z}_m$ и~соответствующих значений дуговых потоков~$\tilde{x}_{mk}$, $k \hm= \overline{1, 2E}$, величина
$$
\tilde{y}_m = \sum\limits_{k=1}^{2E} \tilde{x}_{mk}, \enskip m = \overline{1, M}\,,
$$
характеризует результирующую межузловую \textit{нагрузку} (далее~--- RI-на\-груз\-ка, 
от \textit{англ.}\ resulting internodal) на ребра сети  при передаче  межузлового 
потока~$\tilde{z}_m$ из уз\-ла-ис\-точ\-ни\-ка~$s_m$  в~узел-при\-ем\-ник~$t_m$. Величина~$\tilde{y}_m$ 
показывает, какая суммарная пропускная способность сети 
потребуется для передачи дуговых потоков $\tilde{x}_{mk}$.

В рамках модели отношение RI-нагрузки и~межузлового потока
$$ 
\tilde{w}_m = \fr{\tilde{y}_m}{\tilde{z}_m},  \enskip m = \overline{1, M}\,,
$$
можно трактовать как удельные \textit{затраты}  ресурсов сети при передаче 
единичного   потока $m$-го вида между узлами~$s_m$ и~$t_m$ при  дуговых потоках~$\tilde{x}_{mk}$.  
\textit{Загрузка}

\noindent
$$
\tilde{\Delta}_k = \sum\limits_{m = 1}^M \left[\tilde{x}_{mk} + \tilde{x}_{m(k+E)}\right]  
$$
для каждого ребра~$r_k$ подсчитывается на основании всех дуговых потоков~$\tilde{x}_{mk}$ при одновременной передаче всех потоков в~сети.

Для оценки последствий по\-вреж\-де\-ния узла~$v_j$ на основе исходной сети~$G(0)$ 
формируется сеть~$G(v_j)$, в~которой пропускная способность ребер, инцидентных 
вершине~$v_j$, полагается равной нулю:
$$ 
d_k(v_j) = \begin{cases}
 0, & \mbox{если} \ k \in K(v_j); \\
 d_k(0) & \mbox{в\ остальных\ случаях.}
\end{cases} 
$$

Формально для сети~$G(v_j)$ применяется схема построения MER-марш\-ру\-тов и~со\-глас\-но~(\ref{e4-mal}) 
строятся дуговые потоки~$\{x_{mk}^0\}$, $k \hm= \overline{1, 2E}$, для 
межузловых потоков~$z_m^j$.
Вводятся  обозначения: $P^-(v_j) \hm= \{m_1, m_2, \ldots \}$~--- список пар, для 
которых в~сети $G(v_j)$ не существует пути соединения; $M^-(v_j)$~--- число 
таких пар. Для всех пар $p_m \hm\in P^-(v_j)$ соответствующие потоки~$z_m^j$ и~затраты~$w_m^j$ равны нулю:
 $z_m^j \hm= w_m^j \hm= 0$.
Обозначим через~$P^+(v_j)$ список пар, поток между которыми в~сети~$G(v_j)$ не 
равен нулю, т.\,е.\ для всех пар из списка~$P^+(v_j)$ выполняется
$$ 
z_m^j =1, \enskip w_m^j \ge w_m(0), \enskip m \in P^+(v_j), 
$$
где $w_m(0)$~--- удельные затраты в~исходной сети~$G(0)$ при  условии~(\ref{e4-mal}).

Пусть $M^+(v_j)$~--- число таких пар в~$P^+(v_j)$. При этом
$$ 
P = P^+(v_j) \cup P^-(v_j), \enskip M = M^+(v_j) + M^-(v_j).
$$


Для всех $m \hm\in P^+(v_j)$ вычисляются $\varphi_m^j\hm = w_m^j/w_m(0)$, 
подсчитывается среднее значение относительного увеличения удельных затрат на 
передачу межузлового потока~$z_m$ при по\-вреж\-де\-нии узла~$v_j$:

\noindent
$$
 \varphi_+^j = \fr{1}{M^+(v_j)} \left[ \sum\limits_{p_m \in P^+(v_j)} \varphi_m^j 
\right].
$$
Далее определяется доля пар~$p_m$, для которых путь соединения в~сети~$G(v_j)$ 
отсутствует:
$$  
\psi_-^j = \fr{M^-(v_j)}{M}\,.
$$

Показатели $\varphi_+^j$ и~$\psi_-^j$ используются далее в~качестве оценки 
ущерба при по\-вреж\-де\-нии узла~$v_j$. Значения~$\varphi_+^j$ и~$\psi_-^j$ 
вычисляются последовательно для всех узлов~$v_j$, $j \hm= \overline{1, N}$, 
упорядочиваются по невозрастанию величины, и~строятся диаграммы, которые 
позволяют анализировать ущерб.

\begin{figure*}[b] %fig1
\vspace*{1pt}
  \begin{center}
 \mbox{%
 \epsfxsize=153.408mm 
\epsfbox{mal-1.eps}
 }
\end{center}
\vspace*{-9pt}
\Caption{Базовая~(\textit{а}) и~кольцевая~(\textit{б}) сети}
\end{figure*}


\section{Оценка изменения показателей функционирования }

Для оценки изменения загрузки ребер при по\-вреж\-де\-нии узла~$v_j$ в~каждой сети~$G(v_j)$, $j \hm= \overline{1, N}$, 
согласно~(\ref{e4-mal}) для каждого ребра рассчитывается
$$
\theta_k^j = \fr{\Delta_k(v_j)}{\Delta_k(0)}, \enskip k = \overline{1, E}\,.
$$

Для получения гарантированных оценок загрузок в~по\-вреж\-ден\-ной сети для каждого 
фиксированного~$j$ в~сети~$G(v_j)$ значения~$\theta_k^j$ упорядочиваются по 
величине от большего к~меньшему (по не\-воз\-рас\-та\-нию) и~перенумеровываются:
$$
\left\{\theta_i^j\right\}: \ \theta_{i}^j \ge \theta_{i+1}^j, \enskip  i = \overline{1, E-1}\,.
$$
Для каждого фиксированного~${i}$ среди всех  по\-вреж\-де\-ний во всех сетях~$G(v_j)$ 
определяются
$$
\Theta_i^* = \max_{j} \theta_i^j, \enskip \Theta_i^{**} = \min_{j} \theta_i^j,\enskip i = 
\overline{1, E}\,.
$$
Набор $\{\Theta_i^*\}$,  $i \hm= \overline{1, E}$, представляет собой лексикографически 
упорядоченные верхние оценки загрузки ребер при по\-вреж\-де\-нии узлов~$v_j$, $j \hm= \overline{1, N}$.

\begin{figure*} %fig2
\vspace*{1pt}
  \begin{center}
 \mbox{%
 \epsfxsize=162.817mm 
\epsfbox{mal-2.eps}
 }
\end{center}
\vspace*{-10pt}
\Caption{Усредненные оценки удельных затрат $\varphi^j_+$~(\textit{1}) и~доли пар~$\psi^j$~(\textit{2}) при по\-вреж\-де\-нии узла~$v_j$ в~базовой~(\textit{а}) 
и~кольцевой~(\textit{б}) сетях}    
\end{figure*}

\begin{figure*}[b] %fig3
\vspace*{6pt}
  \begin{center}
 \mbox{%
 \epsfxsize=162.783mm 
\epsfbox{mal-3.eps}
 }
\end{center}
\vspace*{-11pt}
\Caption{Гарантированные оценки загрузки ребер при по\-вреж\-де\-нии в~базовой~(\textit{а}) и~кольцевой~(\textit{б}) сетях}  
\end{figure*}

Для оценки средних значений загрузки ребер относительно  пропускной способности 
для каждой сети~$G(v_j)$, $j \hm= \overline{1, N}$, вычисляется
\begin{multline*}
\nu_k^j = \fr{\Delta_k(v_j)}{d_k(v_j)} \  \mbox{для\ всех}\  k = \overline{1, E}\,,  
\ \mbox{таких\  что}\\
 d_k(v_j) = d_k(0) > 0,
\end{multline*}
и средняя (относительная) загрузка ребра при различных по\-вреж\-де\-ниях:
$$
\nu_k^+ = \fr{1}{N} \left[\sum\limits_{j = 1}^N \nu_k^j \right], \enskip k = \overline{1, E}\,.
$$

Для оценки отклонений~$\nu_k^j$ в~по\-вреж\-ден\-ной сети~$G(v_j)$, исходя из 
начальных  $\nu_k(0) \hm= \Delta_k/d_k(0)$,
определяется среднее значение
$$
\delta_k^+ =\fr{1}{\sqrt{N}} \left[\sum\limits_{j=1}^N(\nu_k^j - \nu_k(0))^2\right]^{1/2}, \enskip k = \overline{1, E}\,.
$$

\section{Вычислительный эксперимент}

Вычислительный эксперимент проводился на моделях сетевых сис\-тем, представленных 
на рис.~1. В~каждой сети 69~узлов. В~ходе вычислительного эксперимента 
проводилась нормировка, и~суммарная пропускная спо\-соб\-ность в~обеих сетях была 
одинакова:
$$
 \sum\limits_{k=1}^{E} d_k(0) = D(0)= 68\,256\,.
 $$


На рис.~2 представлены диаграммы изменения удельных затрат~$\varphi_+^j$ и~доли 
пар~$\psi_-^j$ при по\-вреж\-де\-нии узла $v_j$. Величины~$\varphi_+^j$ и~$\psi_-^j$ 
переупорядочены  от большего к~меньшему и~откладываются по вертикальной оси. 
По горизонтальной оси указаны порядковые номера по\-вреж\-ден\-ных узлов. Кривые~\textit{1}
 на рис.~2 соответствуют средним значениям 
удельных затрат на передачу межузловых потоков при по\-вреж\-де\-нии узла~$v_j$. 
Кривые~\textit{2} указывают долю общего чис\-ла корреспондентов, для 
которых не существует пути передачи.
При по\-вреж\-де\-нии одного узла пропускная спо\-соб\-ность от одного до пяти ребер 
становится рав\-ной нулю. В~среднем по\-вреж\-де\-ние~3\% ребер <<разделяет>> 3\%--4\% 
пар-кор\-рес\-пон\-ден\-тов.

Из рис.~2,\,\textit{б} для кольцевой сети следует, что число по\-вреж\-де\-ний около~20\% 
приводит к~росту удельных затрат на~10\%. Число <<разъединенных>> пар-кор\-рес\-пон\-ден\-тов 
не превышает~3\% при 75\% по\-вреж\-де\-ний. Для базовой сети (см.\ рис.~2,\,\textit{а}) наличие  более~75\% по\-вреж\-де\-ний увеличивает удельные затраты более чем 
на~10\%, а~более~25\% <<разъединяют>> более~5\%~пар. В~обеих сетях при по\-вреж\-де\-нии 
висячих узлов кратчайшие пути для всех кор\-рес\-пон\-ден\-тов остаются неизменными. 
В~результате правые части диаграмм на рис.~2 практически совпадают для номеров 
узлов более~45 как для кольцевой, так и~для базовой сетей.

На рис.~3 для базовой и~кольцевой сетей представлены диаграммы значений 
$\Theta_i^{*}$ и~$\Theta_i^{**}$, которые упорядочены по величине от большего 
к~меньшему, а~номера указаны по горизонтальной оси. Кривые, описывающие~$\Theta_i^{*}$, 
служат верхними огибающими для всех распределений загрузок~$\theta_i^{j}$ 
при по\-вреж\-де\-ни\-ях узлов~$v_j$. По построению~$\Theta_i^{*}$ можно 
рассматривать как гарантированные верхние оценки загрузки ребер при разрушении 
любой вершины графа сети. При по\-вреж\-де\-нии любого узла~$v_j$ лексикографически 
упорядоченное распределение загрузок ребер~--- точки $\theta_i^{j}$~--- лежат 
ниже линии~$\Theta_i^{*}$.

\begin{figure*} %fig4
\vspace*{1pt}
  \begin{center}
 \mbox{%
 \epsfxsize=162.923mm 
\epsfbox{mal-4.eps}
 }
\end{center}
\vspace*{-9pt}
    \Caption{Усредненная   относительная оценка загрузки ребер при 
по\-вреж\-де\-ни\-ях в~базовой~(\textit{а}) и~кольцевой~(\textit{б}) сетях}       
\end{figure*}

Ступенчатые кривые для ребер с~номерами от~1 до~33 соответствуют по\-вреж\-де\-ни\-ям, 
при которых загрузка на некоторых ребрах увеличивается \mbox{в~1,5--2}~ра\-за для базовой 
и~в~1,5--3~ра\-за для кольцевой сети. При этом в~кольцевой сети <<перегрузка>> 
может быть более~300\%, но <<перегрузка>> в~150\% может появляться в~2~раза 
реже, чем в~базовой сети. Большие <<перегрузки>> в~кольцевой сети наблюдаются 
только в~нескольких случаях и~для небольшого числа ребер. Из сравнения диаграмм 
на рис.~3 видно, что в~базовой сети величины <<перегрузок>> меньше, чем 
в~кольцевой, но они наблюдаются для большего числа ребер.
В~кольцевой сети большее число маршрутов соединения, кратчайшие пути короче, чем 
в~базовой. Повреждение некоторых центральных  узлов резко увеличивает  длину 
кратчайших маршрутов и~транзитную нагрузку.



На рис.~3 значения~$\Theta_i^{**}$ формируют нижнюю огибающую для всех 
распределений~$\theta_i^{j}$. Величины~$\Theta_i^{**}$ в~80\% случаев почти 
равны единице, но для пяти ребер как в~базовой, так и~в кольцевой сетях при 
разрушении вершины с~пятью инцидентными ребрами $\Theta_i^{**}\hm = 0$.  Значения~$\Theta_i^{*}$ 
и~$\Theta_i^{**}$ практически совпадают или близки к~единице 
для ребер, инцидентных висячим вершинам. Нагрузка таких ребер меняется 
незначительно при выходе из строя большинства вершин сети. В базовой сети больше 
ребер, ведущих к~висячим вершинам, и~диаграмма рис.~3,\,\textit{а} более <<растянутая>> 
и~<<размытая>>, в~то время как для кольцевой сети на рис.~3,\,\textit{б} максимальное 
значение значительно больше и~кривая резче уходит вниз вдоль вертикальной оси.
В базовой сети по сравнению с~кольцевой  удаление узлов влияет на большее число 
ребер, однако в~кольцевой для некоторых ребер возможные перегрузки могут 
оказаться больше по величине.



На рис.~4 для базовой и~кольцевой  сетей пред\-став\-ле\-ны диаграммы значений 
$\nu_k^+$ и~$\delta_k^+$, которые упорядочены от большего к~меньшему, номера 
указаны по горизонтальной оси; $\nu_k^+$ служат оценками средней загрузки ребра 
при различных по\-вреж\-де\-ниях, а~$\delta_k^+$ используются как оценки отклонения 
загрузок в~по\-вреж\-ден\-ной сети от исходных.

Правые части графиков на рис. 4 практически параллельны оси абсцисс 
и~соответствуют ребрам, инцидентным висячим вершинам. Как уже отмечалось выше, 
загрузка ребер к~висячим вершинам совпадает с~исходной при по\-вреж\-де\-нии 
большинства узлов и~становится равной нулю при разрушении инцидентных висячих 
вершин.

В базовой сети средняя загрузка ребер при по\-вреж\-де\-ни\-ях меньше исходной, но 
отклонения невелики.  Для кольцевой сети даже усредненные показатели для 
некоторых ребер превышают загрузки в~исходной сети и~разброс значений больше, 
чем в~базовой. В~кольцевой сети  через центральные узлы проходят много 
кратчайших маршрутов, и~удаление этих вершин приводит к~резкому увеличению 
транзитных потоков за счет возрастания длины путей соединения.

\vspace*{-4pt}

\section{Заключение}

\vspace*{-2pt}

При построении моделей реальных сис\-тем, в~которых ресурсы ограничены, возникают 
сложности при описании их функционирования и~получении численных результатов.
В~\cite{Gorb-18} приводится обзор работ и~моделей беспроводных сис\-тем связи, 
основанных на оценке ве\-ро\-ят\-ност\-но-вре\-мен\-н$\acute{\mbox{ы}}$х характеристик и~алгоритмах, 
пригодных для получения численных результатов.
В~\cite{Shor-22} предложен\linebreak
 метод, позволяющий оценить минимальную плот\-ность 
развертывания базовых станций для обес\-печения заданной производительности при 
под\-держке ресурсоемких приложений, требующих \mbox{чрезвычайно} высоких скоростей на 
уровне радиоинтерфейса.
В~\cite{Gor-23, Gor-17} рассматриваются задачи \mbox{синтеза} изменения нагрузки на 
узлы сетевой вычислительной инфраструктуры. Рост нагрузки в~реальных системах 
ведет к~необходимости перераспределения ресурсов и~перенаправления потоков 
данных. Предложенные подходы~\cite{Gor-23, Gor-17} могут быть использованы для 
оценки потребности в~ресурсах при изуче\-нии  различных телекоммуникационных 
сис\-тем.

Данная работа предлагает один из возможных подходов к~принятию решений 
в~условиях реально существующей неопределенности о~месте и~цели по\-вреж\-де\-ния 
коммутационных  узлов  систем связи и~управ\-ле\-ния специального назначения~\cite{Starlink}.
Анализ влияния каждого по\-вреж\-де\-ния на информационный обмен позволяет получить 
агрегированный срез данных о~взаимосвязи между жесткой ком\-би\-на\-тор\-но-гра\-фо\-вой 
природой сети и~множеством векторов,  ха\-рак\-те\-ри\-зу\-ющих  изменение нагрузки  на 
сеть  при по\-вреж\-де\-ни\-ях. Способ формирования по\-вреж\-де\-нной сети допускает оценку 
функциональных характеристик, например в~случае стихийных бедствий.

Представленную выше модель можно расценивать как один из вариантов постановки 
традиционных  задач о поиске и~определении критических элементов сети, тогда 
предложенная агрегированная гарантированная  оценка ущерба будет служить 
вариантом  постановки~\cite{Ponton}.  Для поиска критически опасных по\-вреж\-де\-ний 
\cite{Kuhle}, который сводится к~NP-\mbox{труд}\-ной задаче, в~качестве эффективных 
эвристик для метода вет\-вей-и-гра\-ниц подойдут изложенные в~данной статье варианты 
вычисления векторных оценок.  

{\small\frenchspacing
 {\baselineskip=11.5pt
 %\addcontentsline{toc}{section}{References}
 \begin{thebibliography}{99}    
\bibitem{Starlink} %1
\Au{Пехтерев~С.\,В., Макаренко~С.\,И., 
Ковальский~А.\,А.} Описательная модель системы спутниковой связи Starlink~// 
Сис\-те\-мы управ\-ле\-ния, связи и~безопас\-ности, 2022. №\,4. С.~190--255. doi: 10.24412/2410-9916-2022-4-190-255. EDN: QMOLDV.

\bibitem{Mal23-6}  %2
\Au{Малашенко~Ю.\,Е., Назарова~И.\,А.} Анализ критически 
опасных по\-вреж\-де\-ний сети связи. IV.~Многокритериальные оценки уязвимости 
кластеров~// Из\-вес\-тия РАН. Тео\-рия и~сис\-те\-мы управ\-ле\-ния, 2022.  №\,1.\linebreak С.~56--66.
    


\bibitem{Mal23-3}  %3
\Au{Малашенко~Ю.\,Е., Назарова~И.\,А.} Анализ загрузки  
многопользовательской сети при расщеплении потоков по кратчайшим маршрутам~// 
Информатика и~её применения, 2023. Т.~17. Вып.~3. С.~33--38.
doi: 10.14357/19922264230305. EDN: NLUSQJ.

\bibitem{Mal24-1}  %4
\Au{Малашенко~Ю.\,Е., Назарова~И.\,А.} Сравнительный 
анализ узловых мультипотоков в~многопользовательской сетевой сис\-те\-ме~//  
Информатика и~её применения, 2024. Т.~18. Вып.~1. С.~40--45. doi: 10.14357/ 19922264240106. EDN: AKCMCQ.

\bibitem{Dan}  %5
\Au{Данскин Дж.\,М.} Теория максимина и~ее приложение к~задачам распределения вооружения~/
Пер. с~англ.~--- М.: Сов. радио, 1970. 200~с.
(\Au{Danskin~J.\,M.} 
The theory of Max-Min and its application to weapons allocation problems.~--- Berlin: Springer-Verlag, 1970. 128~p. 
doi: 10.1007/978-3-642-46092-0.)


\bibitem{Germ} %6
\Au{Гермейер Ю.\,Б.} Введение в~теорию исследования операций.~--- М.: Наука, 1971. 384~с.

\bibitem{Cl} %7
\Au{Фрэнк Г., Фриш~М.} Сети, связь и~потоки~/ Пер. с~англ.~--- М.: Связь, 
1978. 448~с. (\Au{Frank~H., Frisch~I.}  
Communication, transmission, and transportation networks.~--- Addison-Wesley, 1971. 479~p.)


\bibitem{Yen} %8
\Au{Йенсен П., Барнес~Д.} Потоковое программирование~/ Пер. с~англ.~--- М.:  Радио и~связь, 1984. 392~с.
(\Au{Jensen~P.\,A., Barnes~J.\,W.}  
Network flow programming.~--- New York, NY, USA: Wiley, 1980. 408~p.)


\bibitem{Chankong}  %9
\Au{Chankong V., Haimes~Y.\,Y.}
Multiobjective decision making: Theory and methodology.~--- Mineola, NY, USA: Dover, 2008. 406~p.

\bibitem{Ogryc}  %10
\Au{Ogryczak W., Luss~H., Pioro~M., Nace~D., Tomaszewski~A.} 
Fair optimization and networks: A~survey~// J.~Appl. Math., 2014. Vol.~25. P.~1--25.
doi: 10.1155/2014/612018.



\bibitem{Gorb-18}  %11
\Au{Горбунова~А.\,В.,  Наумов~В.\,А.,  
Гайдамака~Ю.\,В.,  Самуйлов~К.\,Е.}   Ресурсные системы массового обслуживания 
как модели беспроводных сис\-тем связи~//  Информатика и~её применения, 2018. 
Т.~12. Вып.~3. С.~48--55. doi: 10.14357/19922264180307. EDN: YAMDIL.
    
\bibitem{Shor-22}  %12
\Au{Бесчастный~В.\,А., Острикова~Д.\,Ю.,   Шоргин~С.\,Я., Молчанов~Д.\,А., Гайдамака~Ю.\,В.} Анализ плотности базовых 
станций 5G NR для предоставления услуг виртуальной и~дополненной реальности~// 
Информатика и~её применения, 2022. Т.~16. Вып.~2. С.~102--108. doi: 10.14357/19922264220213. EDN: VPIRYN.
    


\bibitem{Gor-17}  %13
\Au{Горшенин~А.\,К.} О некоторых математических и~программных методах построения структурных моделей информационных потоков~// 
Информатика и~её применения, 2017. Т.~11. Вып.~1. С.~58--68. 
doi: 10.14357/19922264170105. EDN: YOCMWZ.

\bibitem{Gor-23}  %14
\Au{Горшенин~А.\,К.,  Горбунов~С.\,А., Волканов~Д.\,Ю.} 
О~клас\-те\-ри\-за\-ции объектов сетевой вычислительной инфраструктуры на основе анализа 
статистических аномалий в~трафике~// Информатика и~её \mbox{применения}, 2023. Т.~17. 
Вып.~3. С.~76--87. doi: 10.14357/ 19922264230311. EDN: XHTMVI.

\bibitem{Ponton} %15
\Au{Ponton J.,  Wei~P., Sun~D.} Weighted clustering 
coefficient maximization for air transportation networks~// European Control Conference 
Proceedings.~--- Zurich, 2013. P.~866--871. doi: 10.23919/ECC.2013.6669250.
 
\bibitem{Kuhle} %16
\Au{Kuhnle A., Nguyen~N.\,P., Dinh~T.\,N., Thai~M.\,T.} 
Vulnerability of clustering under nodes failure in complex networks // Social 
Network Analysis Mining, 2017. Vol.~7. Iss.~1. P.~8--24.

\end{thebibliography}

 }
 }

\end{multicols}

\vspace*{-6pt}

\hfill{\small\textit{Поступила в~редакцию 03.06.24}}

%\vspace*{10pt}

%\pagebreak

\newpage

\vspace*{-28pt}

%\hrule

%\vspace*{2pt}

%\hrule


\def\tit{ANALYSIS OF NETWORK PERFORMANCE INDICATORS IN~CASE~OF~NODE~DAMAGE}


\def\titkol{Analysis of network performance indicators in~case of~node damage}


\def\aut{Yu.\,E.~Malashenko and I.\,A.~Nazarova}

\def\autkol{Yu.\,E.~Malashenko and I.\,A.~Nazarova}

\titel{\tit}{\aut}{\autkol}{\titkol}

\vspace*{-8pt}


\noindent 
Federal Research Center ``Computer Science and Control'' of the Russian 
Academy of Sciences, 44-2 Vavilov Str., Moscow 119333, Russian Federation




\def\leftfootline{\small{\textbf{\thepage}
\hfill INFORMATIKA I EE PRIMENENIYA~--- INFORMATICS AND
APPLICATIONS\ \ \ 2024\ \ \ volume~18\ \ \ issue\ 3}
}%
 \def\rightfootline{\small{INFORMATIKA I EE PRIMENENIYA~---
INFORMATICS AND APPLICATIONS\ \ \ 2024\ \ \ volume~18\ \ \ issue\ 3
\hfill \textbf{\thepage}}}

\vspace*{4pt}



\Abste{On the model of a multiuser communication system, changes in performance indicators are analyzed when  
network nodes are damaged. In the course of computational experiments, changes in the unit cost of resources 
and edge loading are monitored while simultaneously transmitting internodal flows in a damaged network. 
To assess the consequences of each damage, the obtained values are compared with the initial ones. For each 
damaged node, an increase in the unit cost of transmitting internodal flows is calculated. The number of  
source-sink pairs left without connection is determined. A set of guaranteed estimates of the maximum possible
 loads of network edges in case of any damage is formed. The average values for all damages are calculated. 
 Summary diagrams for networks with various structural features are built on the basis of aggregated calculated indicators.}

\KWE{streaming model of the communication network; node damage assessment; edge loading}

\DOI{10.14357/19922264240307}{YUEGZT}

%\vspace*{-12pt}


    
     % \Ack

%\vspace*{-3pt}

%\noindent



  \begin{multicols}{2}

\renewcommand{\bibname}{\protect\rmfamily References}
%\renewcommand{\bibname}{\large\protect\rm References}

{\small\frenchspacing
 {%\baselineskip=10.8pt
 \addcontentsline{toc}{section}{References}
 \begin{thebibliography}{99} 
    
%1
\bibitem{Starlink-1} 
\Aue{Pehterev, S.\,V., S.\,I.~Makarenko, and A.\,A.~Kovalsky.} 2022. 
Opisatel'naya model' sistemy sputnikovoy svyazi Starlink [Descriptive model of Starlink satellite communication system].
\textit{Sistemy upravleniya, svyazi i~bezopas\-nosti} [Systems of Control, Communication and Security] 4:190--255.
doi: 10.24412/2410-9916-2022-4-190-255. EDN: QMOLDV.

%2
\bibitem{Mal23-6-1} 
\Aue{Malashenko, Yu.\,E., and I.\,A.~Nazarova.} 2022.
Analysis of critical damage in the communication network. IV: Multicriteria estimations of cluster vulnerability.
\textit{J.~Comput. Sys. Sc. Int.} 60(6):956--965.
doi: 10.1134/ S1064230721060137. EDN: EJPALH.



%3
\bibitem{Mal23-3-1} 
\Aue{Malashenko, Yu.\,E., and I.\,A.~Nazarova.} 2023.  
Analiz zagruzki  mnogopol'zovatel'skoy seti pri rasshcheplenii potokov po kratchayshim marshrutam 
[Multiuser network load analysis by splitting flows along the shortest routes]. 
\textit{Informatika i~ee Primeneniya~--- Inform. Appl.} 17(3):33--38.
doi: 10.14357/19922264230305. EDN: NLUSQJ.

%4
\bibitem{Mal24-1-1} 
\Aue{Malashenko, Yu.\,E., and I.\,A.~Nazarova.} 2024. 
Srav\-ni\-tel'\-nyy analiz uz\-lo\-vykh mul'\-ti\-po\-to\-kov v~mno\-go\-pol'\-zo\-va\-tel'\-skoy se\-te\-voy sis\-te\-me 
[Analysis of node multiflows in \mbox{a~multiuser} network system].
\textit{Informatika i~ee Primeneniya~--- Inform. Appl.} 18(1):40--45.
doi: 10.14357/ 19922264240106. EDN: AKCMCQ.




%5
\bibitem{Dan-1} 
\Aue{Danskin, J.\,M.} 1970.
\textit{The theory of Max-Min and its application to weapons allocation problems}. Berlin: Springer-Verlag. 128~p. 
doi: 10.1007/978-3-642-46092-0.

%6
\bibitem{Germ-1} 
\Aue{Germeyer, Yu.\,B.} 1971.
\textit{Vvedenie v~teoriyu issledovaniya operatsiy} [Introduction to operations research theory]. Moscow: Nauka. 384~p.

%7
\bibitem{Cl-1} 
\Aue{Frank, H., and I.~Frisch.} 1971. 
\textit{Communication, transmission, and transportation networks}. Addison-Wesley. 479~p.   


%8
\bibitem{Yen-1} 
\Aue{Jensen, P.\,A., and J.\,W.~Barnes.} 1980. 
\textit{Network flow programming}. New York, NY: Wiley. 408~p.



%9
\bibitem{Chankong-1} 
\Aue{Chankong, V., and Y.\,Y.~Haimes.} 2008.
\textit{Multiobjective decision making: Theory and methodology}. Mineola, NY: Dover. 406~p. 

%10
\bibitem{Ogryc-1} 
\Aue{Ogryczak, W., H.~Luss, M.~Pioro, D.~Nace, and A.~Tomaszewski.} 2014. 
Fair optimization and networks: A~survey. 
\textit{J. Appl. Math.} S108:1--25.
doi: 10.1155/2014/ 612018.

%11
\bibitem{Gorb-18-1} 
\Aue{Gorbunova, A.\,V., V.\,A.~Naumov, Yu.\,V.~Gaydamaka, and K.\,E.~Samuylov.} 2018. 
Resursnye sistemy massovogo obsluzhivaniya kak modeli besprovodnykh sistem svyazi [Resource queuing systems as models of wireless communication systems].
\textit{Informatika i~ee Primeneniya~--- Inform. Appl.} 12(3):48--55.
doi: 10.14357/19922264180307. EDN: YAMDIL.

%12
\bibitem{Shor-22-1} 
\Aue{Beschastnyi, V.\,A., D.\,Yu.~Ostrikova, S.\,Ya.~Shorgin, D.\,A.~Moltchanov, and Yu.\,V.~Gaidamaka.} 2022.
Ana\-liz plotnosti bazovykh stantsiy 5G NR dlya predostav\-le\-niya uslug vir\-tu\-al'\-noy i~do\-pol\-nen\-noy real'\-nosti 
[Density analysis of mmWave NR deployments for delivering scalable AR/VR video services].
\textit{Informatika i~ee Primeneniya~--- Inform. Appl.} 16(2):102--108.
doi: 10.14357/ 19922264220213. EDN: VPIRYN.



%13
\bibitem{Gor-17-1} 
\Aue{Gorshenin, A.\,K.} 2017.
O~nekotorykh matematicheskikh i~programmnykh metodakh postroeniya strukturnykh mo\-de\-ley informatsionnykh potokov
 [On some mathematical and programming methods for construction of structural models of information flows].
\textit{Informatika i~ee Primeneniya~--- Inform. Appl.} 11(1):58--68.    
doi: 10.14357/ 19922264170105. EDN: YOCMWZ.

%14
\bibitem{Gor-23-1} 
\Aue{Gorshenin, A.\,K., S.\,A. Gorbunov, and D.\,Yu.~Volkanov.} 2023. 
O~klasterizatsii ob''ektov setevoy vychislitel'noy infra\-struk\-tu\-ry na osnove analiza statisticheskikh ano\-ma\-liy v~trafike 
[Toward clustering of network computing infrastructure objects based on analysis of statistical anomalies in network traffic].
\textit{Informatika i~ee Primeneniya~--- Inform. Appl.} 17(3):76--87.
doi: 10.14357/19922264230311. EDN: XHTMVI.

%15
\bibitem{Ponton-1} 
\Aue{Ponton, J., P.~Wei, and D.~Sun.} 2013.
Weighted clustering coefficient maximization for air transportation networks.
\textit{European Control Conference Proceedings}. Zurich. 866--871.
doi: 10.23919/ECC.2013.6669250.

%16
\bibitem{Kuhle-1} 
\Aue{Kuhnle, A., N.\,P.~Nguyen, T.\,N.~Dinh, and M.\,T.~Thai.} 2017.
Vulnerability of clustering under nodes failure in complex networks. 
\textit{Social Network Analysis Mining} 7(1):\linebreak 8--24.
 
\end{thebibliography}

 }
 }

\end{multicols}

\vspace*{-6pt}

\hfill{\small\textit{Received June 3, 2024}} 

\vspace*{-14pt}

\Contr

\vspace*{-3pt}

\noindent
\textbf{Malashenko Yuri E.} (b.\ 1946)~--- Doctor of Science in physics and mathematics, senior scientist, Federal Research Center 
``Computer Science and Control'' of the Russian Academy of Sciences, 44-2~Vavilov Str., Moscow 119333, Russian Federation; \mbox{malash09@ccas.ru}

\vspace*{6pt}

\noindent
\textbf{Nazarova Irina A.} (b.\ 1966)~--- Candidate of Science (PhD) in physics and mathematics, scientist, Federal Research Center 
``Computer Science and Control'' of the Russian Academy of Sciences, 44-2~Vavilov Str., Moscow 119333, Russian Federation; \mbox{irina-nazar@yandex.ru}


\label{end\stat}

\renewcommand{\bibname}{\protect\rm Литература} 