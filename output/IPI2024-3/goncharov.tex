
\def\stat{goncharov}

\def\tit{ПРИНЦИПЫ АННОТИРОВАНИЯ ИМПЛИЦИТНЫХ ЛОГИКО-СЕМАНТИЧЕСКИХ 
ОТНОШЕНИЙ В~ПАРАЛЛЕЛЬНЫХ~ТЕКСТАХ$^*$}

\def\titkol{Принципы аннотирования имплицитных логико-семантических 
отношений в~параллельных текстах}

\def\aut{А.\,А.~Гончаров$^1$, П.\,В.~Ярошенко$^2$}

\def\autkol{А.\,А.~Гончаров, П.\,В.~Ярошенко}

\titel{\tit}{\aut}{\autkol}{\titkol}

\index{Гончаров А.\,А.}
\index{Ярошенко П.\,В.}
\index{Goncharov A.\,A.}
\index{Iaroshenko P.\,V.}


{\renewcommand{\thefootnote}{\fnsymbol{footnote}} \footnotetext[1]
{Исследование выполнено за счет гранта Российского научного фонда №\,24-28-00527, {\sf 
https://rscf.ru/project/24-28-00527/}.}}


\renewcommand{\thefootnote}{\arabic{footnote}}
\footnotetext[1]{Федеральный исследовательский центр <<Информатика и~управление>> Российской академии наук, 
\mbox{a.gonch48@gmail.com}}
\footnotetext[2]{Федеральный исследовательский центр <<Информатика и~управление>> Российской академии наук, 
\mbox{polina.iaroshenko@yandex.ru}}


\vspace*{-12pt}

 

  \Abst{Рассматривается проблема аннотирования имплицитных ло\-ги\-ко-се\-ман\-ти\-че\-ских 
отношений (ЛСО). Проанализирован мировой опыт аннотирования 
имплицитных ЛСО. Представлены подходы, ориентированные на (1)~анализ глобальной 
структуры текста; (2)~анализ локальной структуры текста; (3)~унификацию данных, 
размеченных в~рамках различных теорий, и~разработку единого стандарта аннотирования. 
Предложены принципы аннотирования имплицитных ЛСО 
в~параллельных текстах, т.\,е.\ в~качестве объекта аннотирования выступает переводное 
соответствие (пара фрагментов текста оригинала и~перевода). Рассматривались такие 
переводные соответствия, где в~русскоязычном тексте показатели ЛСО отсутствуют, 
а~в~тексте на другом языке, напротив, имеются. С учетом специфики имплицитных ЛСО 
сформулированы следующие принципы их аннотирования: (1)~необходимо выделять 
границы аргументов ЛСО (обеспечивается наглядность и~удобство анализа); (2)~признаки 
блоков текста должны образовывать иерархическую структуру (обеспечивается удобство 
работы с~большим числом признаков); (3)~если признак блока текста имеет лексический 
показатель, то этот показатель должен быть выделен (обеспечивается более высокая 
обоснованность решений разметчика).}
  
  \KW{лингвистическое аннотирование; дискурсивные отношения; ло\-ги\-ко-се\-ман\-ти\-че\-ские отношения; имплицитность; параллельные тексты}
  
  \DOI{10.14357/19922264240313}{NPXQNX}

%\vspace*{-6pt}


\vskip 10pt plus 9pt minus 6pt

\thispagestyle{headings}

\begin{multicols}{2}

\label{st\stat}

\section{Введение}

В настоящее время одним из наиболее актуальных направлений в~области 
обработки естественного языка стал анализ языковых единиц, более 
протяженных, чем предложение. Нередко в~качестве единицы анализа 
выступает текст в~целом. Сложность такого анализа очевидна: текст имеет 
многоуровневую структуру, где, с~одной стороны, существуют различные 
связи между единицами каж\-до\-го из уровней, а~с~другой стороны, связи 
между единицами разных уровней (кратко о внутритекстовых связях см.~[1, 
с.~98--99], подробнее~[2]). Эта особенность делает разработку принципов 
и~средств дискурсивного аннотирования текста нетривиальной задачей.

Принципы дискурсивного аннотирования корпусов разрабатываются исходя 
из выбранного создателями корпуса теоретического подхода к~описанию 
структуры дискурса. Чаще всего \mbox{рассматриваются} два уровня разметки~--- 
глобальный и~локальный (подробнее о~противопоставлении локальной 
и~глобальной структуры дискурса см.~[3, с.~49--50; 4, с.~30--35]). В~первом 
случае более пристальное внимание уделяется анализу общей структуры 
документа, во втором~--- анализу отношений между единицами, не слишком 
удаленными друг от друга с~точки зрения линейного расположения (как 
правило, это предложения или клаузы).

Важность создания дискурсивно аннотированных корпусов сложно 
переоценить: размеченные тексты используются в~том числе и~для глубокого 
обучения в~качестве тренировочной выборки при решении различных задач 
по извлечению информации из текста~[5]. Особую ценность представляет 
накопление размеченных данных на разных языках: так, в~[6] отмечается, 
что большое число исследований осуществляется на материале английского 
языка, в~то время как размеченных данных для других языков, в~частности 
для китайского, не хватает.

Отдельная проблема заключается в~том, что при изучении ка\-ко\-го-ли\-бо 
конкретного феномена, относящегося к~уровню дискурса, возникает 
необходимость сочетать <<общую>> разметку с~разметкой, отражающей 
результаты анализа проявлений этого~--- находящегося в~центре  
внимания~--- феномена. К~подобным феноменам можно отнести 
и~\textbf{имплицитные ЛСО}~--- такие отношения, как противопоставление, причина, условие, 
следование во времени и~др., которые чаще всего устанавливаются между 
отдельными предложениями или же между простыми предложениями 
в~составе сложного и~при этом не имеют показателя. Так, в~примере 
<<\textit{Он не смог прийти вовремя, потому что начальник задержал его на 
работе}>> содержится эксплицитное ЛСО причины (выражено при помощи 
\textit{потому что}), а~в~примере <<\textit{Он не смог прийти вовремя, начальник 
задержал его на работе}>>~--- имплицитное (ничем не выражено).

Цель настоящей статьи~--- с~учетом мирового опыта сформулировать 
принципы аннотирования имплицитных ЛСО, которые обеспечили бы 
возможность создания представительной выборки размеченных примеров 
таких ЛСО на русскоязычном материале, подходящей как для чисто 
лингвистических исследований, так и~для использования при создании 
средств автоматической обработки \mbox{текста}.

\section{Мировой опыт аннотирования имплицитных логико-семантических отношений}

\subsection{Имплицитные логико-семантические отношения и~анализ глобальной структуры текста} 
%2.1 

Наиболее известной теорией, ориентированной на анализ глобальной 
структуры текста, пред\-став\-ля\-ет\-ся тео\-рия риторической структуры (Rhetorical 
Structure Theory, RST~[7]). Согласно RST, текст представляет собой единый 
дискурсивный объект, все элементы которого связаны между собой 
\textit{риторическими} (в~терминах RST) отношениями. Следовательно, 
одним из результатов применения этой теории должна стать схема 
анализируемого текста, отражающая то, как он организован. Наличие или 
отсутствие показателей отношений, связывающих части текста, 
в~классической версии RST не принималось во внимание. Позднее, 
с~началом создания корпусов, размеченных в~соответствии с~принципами 
RST, интерес адептов этой теории к~показателям риторических отношений 
повысился, однако фрагменты, где показателей не было, практически не 
исследовались: в~пособиях по аннотированию не использовался даже термин 
<<имплицитное>> отношение~[8, 9]. С~другой стороны, тот факт, что 
в~центре внимания находились не показатели, а непосредственно отношения, 
способствовал тому, что аргументы отношений выделялись исключительно 
исходя из семантики.

В соответствии с~принципами RST размечен русскоязычный корпус 
 Ru-RSTreebank~[10] (об этапах создания корпуса см.~[11, с.~128--150]).

Таким образом, применительно к~аннотированию имплицитных ЛСО 
к~недостаткам глобального подхода к~анализу структуры текста следует 
от\-нес\-ти (1)~интерес к~структуре текста в~целом и~(2)~игнорирование 
показателей внутритекстовых отношений и,~соответственно, фактов их 
отсутствия; а~к~достоинствам подхода~--- сугубо семантические основания 
для выделения аргументов отношения.


\vspace*{-6pt}

\subsection{Имплицитные логико-семантические отношения и~анализ локальной структуры текста} 
%2.2

\vspace*{-2pt}

Говоря об исследованиях, ориентированных на анализ локальной структуры 
текста, в~качестве наиболее известного примера, по всей ви\-ди\-мости, стоит 
назвать Пенсильванский дискурсивно аннотированный корпус (Penn 
Discourse Treebank, PDTB). Он задумывался с~целью аннотировать не 
глобальную структуру текста, а~коннекторы и~выражаемые ими отношения 
(\textit{дискурсивные} в~терминах PDTB)~[12, с.~1]. Именно это обусловило 
большее внимание создателей корпуса к~фрагментам, в~которых коннектор 
отсутствовал. В~соответствии с~исходной целью они разделили коннекторы 
на эксплицитные и~имплицитные. Если понятие эксплицитного коннектора 
не ново, то понятие имплицитного коннектора (\textit{implicit connective}) 
нуждается в~уточнении. Имплицитным в~подходе PDTB считается коннектор, 
который отсутствует в~исходном тексте, но был добавлен туда разметчиком 
при аннотировании как наиболее подходящий по смыс\-лу. Важно, что 
добавление коннектора не должно приводить к~семантической 
избыточности~[13, с.~1200--1201]. Именно фрагменты, куда можно добавить 
коннектор, содержат, согласно последним версиям PDTB, имплицитные 
дискурсивные отношения.

Подобная ориентированность на коннекторы повлекла за собой еще один 
недостаток подхода~--- размывание семантической природы аргументов 
отношения и~использование синтаксических характеристик анализируемого 
фрагмента для выделения аргументов. Так, сами разработчики PDTB при 
описании третьей версии корпуса в~[14, с.~11--12] отмечают, что 
недостатком PDTB-2 было, в~частности, то, что аргументы имплицитных 
ЛСО всегда определялись исключительно исходя из линейного порядка (вне 
за\-ви\-си\-мости от сим\-мет\-рич\-ности отношения). Как следствие, в~примерах 
с~одним и~тем же ЛСО аргументы могли выделяться по-разному, что 
услож\-ня\-ло работу с~получаемой выборкой. Однако решение этой проб\-ле\-мы, 
предложенное в~PDTB-3, не кажется оптимальным:\linebreak чтобы избежать 
повторного \mbox{аннотирования} материала, авторы предлагают определять 
аргументы семантически только в~рамках сложноподчиненных предложений. 
В~сложносочиненных предложениях и~на границе между предложениями 
единственным критерием, обуслов\-ли\-ва\-ющим выбор меток <<аргумент~1>> 
или <<аргумент~2>>, остается линейный порядок.
{\looseness=-1

}

Таким образом, применительно к~аннотированию имплицитных ЛСО 
к~достоинствам локального подхода к~анализу структуры текста следует 
от\-нес\-ти интерес к~связям между единицами дискурса, не слишком 
удаленным друг от друга с~точки зрения линейного расположения; 
а~к~недостаткам~--- (1)~излишнюю ориентированность на показатели ЛСО 
и~(2)~опору на синтаксис при выделении аргументов отношения.

\vspace*{-6pt}

\subsection{Имплицитные логико-семантические отношения и~разработка средств аннотирования, 
не~привязанных к~теории} %2.3

\vspace*{-2pt}

Ряд исследователей заявляют о необходимости разработки системы 
аннотирования, которая могла бы применяться вне зависимости от 
выбранного теоретического подхода. В~[15, с.~43--47] подчеркивается 
важность сопоставления различных способов аннотирования дискурсивных 
корпусов, чтобы впоследствии найти оптимальный способ <<переводить>> 
разметку, выполненную в~рамках одного подхода, таким образом, чтобы этот 
же материал можно было использовать при работе с~другими подходами 
(``translate existing annotations from one framework to another''). В~статье~[16] 
рассматриваются возможные принципы унификации разметки как 
глобальной структуры дискурса, так и~дискурсивных отношений (в~рамках 
RST, теории сегментной репрезентации дискурса 
и~подхода созда\-те\-лей PDTB).

Тенденция к~большей универсальности прослеживается и~в~об\-ласти 
разработки приложений для аннотирования дискурса. Например, при 
создании приложения TIARA~[17] разработчики руководствовались 
следующими принципами: понятный интерфейс; простота использования 
приложения; функция проверки полноты; фиксация и~сохранение действий 
разметчиков (\textit{annotation tracking}); возможность адаптировать 
приложение под свои цели. Разметку можно осуществлять на четырех 
уровнях, первый из которых более общий, а~остальные можно 
охарактеризовать как специализированные.

Как отмечается в~[17], приложения, которые можно использовать для разных 
видов лингвистического аннотирования, зачастую не вполне подходят для 
разметки структуры дискурса. Об этом свидетельствует, к~примеру, 
документация таких инструментов, как BRAT~[18], \mbox{WebAnno}~[19] 
и~\mbox{INCEpTION}~[20]. На основании этого можно констатировать, что 
создание принципов аннотирования имплицитных ЛСО и~приложений для 
поддержки такого аннотирования на сегодняшний день остается актуальным 
направлением работы.

\vspace*{-9pt}

\section{Предлагаемые принципы аннотирования имплицитных 
логико-семантических отношений}

\vspace*{-3pt}

Отправной точкой для разработки предлагаемых в~настоящей статье 
принципов аннотирования имплицитных ЛСО стала, во-пер\-вых, 
классификация ЛСО, предложенная О. Ю. Иньковой и~пред\-став\-лен\-ная в~работе~[21], и,~во-вто\-рых, концепция надкорпусных баз данных
(НБД), 
разработанная коллективом исследователей из ФИЦ ИУ РАН~[22--24]. 
Кроме того, были учтены все предложения по совершенствованию 
функционала для аннотирования текс\-тов, перечисленные в~[25]. Материалом 
исследования послужили параллельные текс\-ты, в~которых были найдены 
пары фрагментов оригинала и~перевода, где в~русскоязычном текс\-те 
показатели ЛСО отсутствуют, а~в~текс\-те на другом языке, напротив, 
имеются\footnote{На текущем этапе работы рассматриваются исключительно примеры ЛСО, 
имеющих два аргумента (бинарных).}. Таким образом, результатом работы 
разметчика оказывается не аннотированный фрагмент с~имплицитным ЛСО, 
а~\textbf{аннотированное переводное соответствие} (АПС). Для хранения 
АПС создана \textbf{НБД ЛСО}, а~для 
формирования АПС и~последующей работы с~ними разработан 
пользовательский интерфейс (веб-при\-ло\-же\-ние)\footnote{Рисунки~1 и~2 
представляют собой скриншоты, сделанные в~этом приложении.}.

Основные структурные элементы АПС с~имплицитными ЛСО совпадают с~элементами АПС, создаваемых в~ходе исследования других языковых 
единиц:
\begin{enumerate}[(1)]
\item \textbf{блоки текста оригинала и~перевода}, в~которых, при наличии, выделено вхождение 
изучаемой языковой единицы и~варианта ее перевода, если изучается единица языка оригинала, 
или вхождение изучаемой языковой единицы и~стимула ее перевода, если изучается единица 
языка перевода;

\end{enumerate}

\end{multicols}

\begin{figure*} %fig1
\vspace*{1pt}
  \begin{center}
 \mbox{%
 \epsfxsize=96.774mm 
\epsfbox{gon-1.eps}
 }
\end{center}
\vspace*{-11pt}
\Caption{Разметка аргументов ЛСО. (Источник примера: [А.\,С.~Пушкин. Капитанская дочка 
(1836)~$\vert$~La fille du capitaine (пер.\ L.~Viardot; 1853)].)}
%\end{figure*}
%\begin{figure*} %fig2
\vspace*{6pt}
  \begin{center}
 \mbox{%
 \epsfxsize=103.489mm 
\epsfbox{gon-2.eps}
 }
\end{center}
\vspace*{-11pt}
\Caption{Пример АПС с~имплицитным ЛСО в~оригинале и~эксплицитным в~переводе 
(Источник примера: [В.\,С.~Гроссман. Жизнь и~судьба (ч.~3) (1960)~$\vert$~Vie et Destin (part.~3) (пер. 
A.~Berelowitch; 1980)].)}
\end{figure*}

\begin{multicols}{2}

\noindent
\begin{enumerate}[(1)]
\setcounter{enumi}{1}
\item \textbf{наборы признаков блоков текста};
\item \textbf{набор признаков переводного соответствия}.
\end{enumerate}

Однако специфика имплицитных ЛСО как объекта исследования оказала 
значительное влияние на внутреннее устройство этих элементов. Не\-смот\-ря 
на значимые результаты, полученные в~ходе изуче\-ния эксплицитных ЛСО и~их показателей 
с~использованием НБД коннекторов, 
и~разнообразие задач, решаемых с~ее по\-мощью (см., напр.,~[26--31]), а также 
результаты, полученные с~использованием надкорпусной базы данных 
иерархии ЛСО~[32], накопленный в~ходе этих исследований опыт не мог 
быть использован для работы с~имплицитными ЛСО в~неизменном виде. Для 
того чтобы АПС содержало достаточный объем сведений об имплицитном 
ЛСО, предлагаются следующие принципы аннотирования:
\begin{enumerate}[(1)]
\item должны быть выделены границы аргументов исследуемого ЛСО 
(обеспечивается наглядность и~удобство анализа);
\item признаки блоков текста должны образовывать иерархическую 
структуру (обеспечивается удобство работы с~большим числом признаков);
\item если признак блока текста имеет лексический показатель, этот 
показатель должен быть выделен (обеспечивается более высокая 
обоснованность решений разметчика).
\end{enumerate}

Принимая во внимание обозначенные принципы, рассмотрим более детально 
каждый из структурных элементов АПС с~имплицитным ЛСО.

\subsection{Блоки текста оригинала и~перевода}

Объем блоков текста оригинала и~перевода должен быть необходимым 
и~достаточным для анализа ЛСО. Внутри каждого блока текста необходимо 
выделить аргументы анализируемого ЛСО~--- отрезки текста, связанные при 
помощи этого ЛСО~--- и~присвоить им метки <<аргумент~1>> 
и~<<аргумент~2>> соответственно\footnote{Выбор нужной метки, с~учетом 
опыта~[33], для асимметричных отношений осуществляется исходя из семантики аргумента 
(линейный порядок не учитывается), а для симметричных~--- напротив, исключительно из 
линейного порядка аргументов. Например, для ЛСО причины (асимметричное ЛСО) меткой 
<<аргумент~1>> всегда обозначается тот, который содержит причину того, что описано во втором 
аргументе, а~для ЛСО противопоставления (симметричное ЛСО) меткой <<аргумент~1>> 
обозначается тот, который находится в~тексте линейно раньше, чем второй аргумент отношения. 
Следует отметить, что такое разграничение представляет собой некоторое упрощение, так как 
примеры из текстов на естественном языке показывают, что даже те отношения, которые 
считаются симметричными, не всегда допускают возможность перестановки аргументов без 
изменения смыс\-ла.}. Для наглядности каждый аргумент выделяется цветом. Так, 
в~примере на рис.~1 зеленым в~обоих случаях выделен аргумент~1 
(в~переводе ЛСО причины выражено показателем \textit{car}), 
а~оранжевым~--- аргумент~2.



Как видно, обозначенные принципы обеспечивают единство подхода к~ЛСО, 
будь они эксплицитными или имплицитными (ср.\ подразд.~2.2 выше).

\subsection{Признаки блоков текста оригинала и~перевода}

После того как были выделены аргументы ЛСО, можно перейти к~основному этапу аннотирования~--- выявлению 
признаков блоков текста.
В~первую очередь признаки делятся на группы в~зависимости от того, к~какой 
именно части блока текста они относятся. Рассмотрим более детально, что 
входит в~каждую из групп.

\subsubsection{Признаки блока текста в~целом}

 Включают признаки, указывающие на тип ЛСО (например, \textit{причина}, \textit{противопоставление}).

\subsubsection{Признаки, относящиеся к~одному из~аргументов}

Признаки, описывающие синтаксическую характеристику аргумента: в~том 
случае, если аргумент представляет собой отдельное предложение, несколько 
предложений или вставную конструкцию, это отмечается в~АПС; если же 
аргумент является клаузой, то меток не ставится.

Признаки, указывающие на то, что аргумент содержит какие-либо 
лексические единицы, значимые для анализа ЛСО: отрицание; некоторые 
формы глагольного наклонения; показатели модальности; субъективно-оценочную лексику.

\subsubsection{Признаки, относящиеся к~обоим аргументам}

 Включают признаки, указывающие на некоторые типы соотношения 
аргументов на лексическом уровне, значимые для анализа ЛСО: разметчик 
может указать на наличие синонимов, лексического повтора и~др. К~этой же 
группе относятся признаки, свидетельствующие о наличии некоторых знаков 
препинания между аргументами.

Таким образом, признаки блока текста весьма разнородны. Обязательным 
для каждого блока текс\-та остается только выбор ЛСО, в~то время как все 
остальные признаки отмечаются при не\-об\-хо\-ди\-мости. Однако если выбран 
признак, имеющий лексические показатели (например, признак наличия  
субъ\-ек\-тив\-но-оце\-ноч\-ной лексики в~ка\-ком-ли\-бо из аргументов), 
разметчик обязан указать, о каких именно лексических единицах идет речь.

\subsection{Признаки переводного соответствия}

Разметчик может присвоить переводному соответствию следующие при\-знаки:
\begin{enumerate}[(1)]
\item <<исключено из рассмотрения>>~--- АПС не должно приниматься 
во внимание при работе с~выборкой размеченных данных (например, 
в~случае очевидно ошибочного перевода);
\item <<изменение порядка пропозиций>>~--- в~переводе изменен 
линейный порядок единиц с~пропозициональным содержанием (т.\,е.\ 
смыс\-ло\-вых частей);
\item <<изменение структуры>>~--- в~переводе изменена структура блока 
текста оригинала;
\item <<опущение>>~--- в~переводе нет соответствия для части блока 
текста оригинала;
\item <<добавление>>~--- в~оригинале нет соответствия для части блока 
текста перевода.
\end{enumerate}

Признаки~(3)--(5) присваиваются только в~тех случаях, если соответствующие 
характеристики значимы для анализа ЛСО.

На рис.~2 представлен пример готового АПС. В~верхней части рисунка 
приведены блоки текста, где цветами выделены аргументы ЛСО, 
а~в~переводе полужирным выделен показатель ЛСО. Ниже~--- метки, 
указывающие на языковые единицы, выражающие анализируемое ЛСО 
(в~оригинале <<NoMarker>>, так как ЛСО имплицитно, в~переводе <<car>>). 
Затем приводятся признаки блоков текста, которые могут быть значимы для 
анализа ЛСО. Так, указано, что (1)~оба аргумента в~обоих языках 
представляют собой отдельные предложения; (2)~оба аргумента~1 содержат 
показатели отрицания; (3)~оба аргумента~2 содержат  
субъ\-ек\-тив\-но-оце\-ноч\-ную лексику; (4)~как в~оригинале, так 
и~в~переводе связь между аргументами поддерживается лексическими 
повторами. Признаки переводного соответствия отсутствуют.



\section{Заключение}

В статье с~учетом мирового опыта были сформулированы принципы 
аннотирования имплицитных ЛСО. Было показано, что предложенные 
принципы обеспечивают комплексный подход к~анализу имплицитных ЛСО 
и~позволяют в~дальнейшем создать представительную выборку размеченных 
примеров. Размеченные данные такого типа могут использоваться как 
в~лингвистических исследованиях, так и~в~об\-ласти автоматической 
обработки текстов в~качестве обучающего множества.

{\small\frenchspacing
 {\baselineskip=11.5pt
 %\addcontentsline{toc}{section}{References}
 \begin{thebibliography}{99}
\bibitem{1-gon}
\Au{Гончаров А.\,А.} Классификации внутритекстовых отношений: основания и~принципы 
структурирования~// Вопросы языкознания, 2021. №\,3. С.~97--119. doi: 10.31857/0373-658X.2021.3.97-119. EDN: OKPZEI.
\bibitem{2-gon}
\Au{Stede M.} Discourse processing.~--- San Rafael, CA, USA: Morgan \& Claypool Publs., 
2012. 155~p.
\bibitem{3-gon}
\Au{Hobbs J.\,R.} Literature and cognition.~--- Stanford, CA, USA: CSLI, 1990. 193~p.
\bibitem{4-gon}
\Au{Кибрик А.\,А.} Анализ дискурса в~когнитивной перспективе: Дис.\ \ldots\ докт. филол. наук.~--- 
М.: Ин-т языкознания РАН, 2003. 90~с.
\bibitem{5-gon}
\Au{Xiang W., Wang~B.} A~survey of implicit discourse relation recognition~// ACM 
Comput. Surv., 2023. Vol.~55. Iss.~12. 34~p. doi: 10.1145/3574134.
\bibitem{6-gon}
\Au{Jiang D., He~J.} Tree framework with BERT word embedding for the recognition of 
Chinese implicit discourse relations~// IEEE Access, 2020. Vol.~8. P.~162004--162011. doi: 
10.1109/ACCESS.2020.3019500.
\bibitem{7-gon}
\Au{Taboada M., Mann W.\,C.} Rhetorical structure theory: Looking back and moving ahead~// 
Discourse Stud., 2006. Vol.~8. Iss.~3. P.~423--459. doi: 10.1177/146144560606188.
\bibitem{8-gon}
\Au{Carlson L., Marcu~D.} Discourse tagging reference manual, 2001.  ISI Technical Report ISI-TR-545. 87~p.
{\sf https://www.isi.edu/content/tr/tr-545.pdf}.
\bibitem{9-gon}
\Au{Das D., Taboada~M.} RST Signalling Corpus annotation manual, 2014. 
{\sf https://www.sfu.ca/$\sim$mtaboada/\linebreak docs/publications/RST\_Signalling\_Corpus\_Annotation\_\linebreak Manual.pdf}.
\bibitem{10-gon}
\Au{Pisarevskaya D., Ananyeva~M., Kobozeva~M., Nasedkin~A., Nikiforova~S., Pavlova~I., 
Shelepov~A.} Towards building a~discourse-annotated corpus of Russian~// Компьютерная 
лингвистика и~интеллектуальные технологии.~--- М.: \mbox{РГГУ}, 2017.  Вып.~16.  Т.~1.   С.~201--212.
\bibitem{11-gon}
\Au{Смирнов И.\,В.} Интеллектуальный анализ текстов на основе методов разноуровневой 
обработки естественного языка.~--- М.: ФИЦ ИУ РАН, 2023. 356~с.
\bibitem{12-gon}
\Au{Prasad R., Miltsakaki~E., Dinesh~N., Lee~A., Joshi~A., Webber~B.\,L.} The Penn Discourse 
TreeBank~1.0 Annotation Manual.~--- IRCS technical reports ser.~--- The PDTB Research Group, 2006. 54~p.
{\sf https://catalog.ldc.upenn.edu/ docs/LDC2008T05/papers/pdtb-1.0-annotation-manual.pdf}.
\bibitem{13-gon}
\Au{Prasad R., Webber~B., Joshi~A.} The Penn Discourse Treebank: An annotated corpus of 
discourse relations~// Handbook of linguistic annotation~/
Eds. N.~Ide, J.~Pustejovsky.~--- Dordrecht: Springer Science\;+\;Business Media, 2017. P.~1197--1217. doi: 10.1007/978-94-024-0881-2\_45.


\bibitem{14-gon}
\Au{Webber B., Prasad R., Lee~A., Joshi~A.} The Penn Discourse Treebank 3.0: Annotation 
Manual, 2019. {\sf https://\linebreak catalog.ldc.upenn.edu/docs/LDC2019T05/PDTB3-Annotation-Manual.pdf}.
\bibitem{15-gon}
\Au{Zufferey S., Degand~L.} Connectives and discourse relations. Key topics in semantics and 
pragmatics.~--- Cambridge: Cambridge University Press, 2024. 268~p.
\bibitem{16-gon}
\Au{Fu Y.} Towards unification of discourse annotation frameworks // 60th 
Annual Meeting of the Association for Computational Linguistics Proceedings.~--- Dublin: Association for 
Computational Linguistics, 2022. P.~132--142. doi: 10.18653/v1/2022.acl-srw.12.
\bibitem{17-gon}
\Au{Putra J.\,W.\,G., Matsumura~K., Teufel~S., Tokunaga~T.} TIARA~2.0: An interactive tool for 
annotating discourse structure and text improvement\,// Lang. Resour. Eval., 
2023. Vol.~57. P.~5--29. doi: 10.1007/s10579-021-09566-0.
\bibitem{18-gon}
Brat rapid annotation tool. {\sf https://brat.nlplab.org/\linebreak index.html}.
\bibitem{19-gon}
WebAnno. A~flexible, web-based and visually supported system for distributed annotations. {\sf 
https://webanno.\linebreak github.io/webanno}.
\bibitem{20-gon}
INCEpTION. A~semantic annotation platform offering intelligent assistance and knowledge 
management. {\sf https://inception-project.github.io}.
\bibitem{21-gon}
\Au{Инькова О.\,Ю.} Ло\-ги\-ко-се\-ман\-ти\-че\-ские отношения: проб\-ле\-мы классификации~// 
Связность текста: мереологические ло\-ги\-ко-се\-ман\-ти\-че\-ские отношения.~--- М.: ЯСК, 2019. 
С.~11--98.
\bibitem{22-gon}
\Au{Зацман И., Кружков~М., Лощилова~Е.} Методы и~средства информатики для описания 
структуры неоднословных коннекторов~// Структура коннекторов и~методы ее описания.~--- 
М.: ТОРУС ПРЕСС, 2019. С.~205--230. doi: 10.30826/SEMANTICS19-06. EDN: YVAJWN.

\bibitem{24-gon} %23
\Au{Гончаров А.\,А., Инькова~О.\,Ю., Кружков~М.\,Г.} Методология аннотирования 
в~надкорпусных базах данных~// Системы и~средства информатики, 2019. Т.~29. №\,2. С.~148--160.
doi: 10.14357/08696527190213. EDN: GNDCJE.

\bibitem{23-gon} %24
\Au{Кружков М.\,Г.} Концепция построения надкорпусных баз данных~// Системы и~средства информатики, 2021. Т.~31. №\,3. С.~101--112.
doi: 10.14357/ 08696527210309. EDN: UMWNIU.
\bibitem{25-gon}
\Au{Гончаров А.\,А.} Аннотирование параллельных корпусов: подходы и~направления 
развития~// Информатика и~её применения, 2023. Т.~17. Вып.~4. С.~81--87. doi: 10.14357/19922264230411. 
EDN: GDKDOZ.
\bibitem{26-gon}
Семантика коннекторов: контрастивное исследование~/ Под ред.\ О.\,Ю.~Иньковой.~--- М.: 
ТОРУС ПРЕСС, 2018. 368~с.
\bibitem{27-gon}
Структура коннекторов и~методы ее описания~/ Под ред.\ О.\,Ю.~Иньковой.~--- М.: ТОРУС 
ПРЕСС, 2019. 316 с. EDN: VVIINM.
\bibitem{28-gon}
\Au{Гончаров А.\,А., Инькова~О.\,Ю.} Методика поиска имплицитных ло\-ги\-ко-се\-ман\-ти\-че\-ских 
отношений в~текс\-те~// Информатика и~её применения, 2019. Т.~13. Вып.~3. С.~97--104. 
doi: 10.14357/19922264190314. EDN: MWGFJW.
\bibitem{29-gon}
\Au{Нуриев В.\,А., Зацман~И.\,М.} Редуцирование спект\-ра моделей перевода в~надкорпусных 
базах данных~// Информатика и~её применения, 2020. Т.~14. Вып.~2. С.~119--126. 
doi:  10.14357/19922264200217. EDN: EBUTTA.
\bibitem{30-gon}
\Au{Гончаров А.\,А., Инькова~О.\,Ю.} Извлечение знаний о~средствах выражения 
логико-семантических отношений при помощи Надкорпусной базы данных~// Информатика и~её 
применения, 2021. Т.~15. Вып.~2. С.~96--103.
doi:  10.14357/19922264210214. EDN: \mbox{GRPWIB}.
\bibitem{31-gon}
\Au{Инькова О.\,Ю., Кружков~М.\,Г.} Структурированные определения дискурсивных 
отношений в~Надкорпусной базе данных коннекторов~// Информатика и~её применения, 
2021. Т.~15. Вып.~4. С.~27--32. doi: 10.14357/19922264210404. EDN: EZJXVI.
\bibitem{32-gon}
\Au{Дурново А.\,А., Инькова~О.\,Ю., Попкова~Н.\,A.} Принципы описания показателей 
логико-семантических отношений и~их иерархии~// Информатика и~её применения, 2022. Т.~16. 
Вып.~2. С.~52--59. doi: 10.14357/19922264220207. EDN: NPFTOH.
\bibitem{33-gon}
\Au{Bunt H., Prasad~R.} ISO DR-Core (ISO 24617-8): Core concepts for the annotation of 
discourse relations~// 12th Joint ACL-ISO Workshop on Interoperable Semantic Annotation 
Proceedings.~--- \mbox{Portoro{\hspace*{-0.6pt}\!\ptb\v{z}}}, Slovenia, 2016.\linebreak P.~45--54.

\end{thebibliography}

 }
 }

\end{multicols}

\vspace*{-6pt}

\hfill{\small\textit{Поступила в~редакцию 12.07.24}}

\vspace*{10pt}

%\pagebreak

%\newpage

%\vspace*{-28pt}

\hrule

\vspace*{2pt}

\hrule



\def\tit{IMPLICIT LOGICAL-SEMANTIC RELATIONS IN~PARALLEL TEXTS: ANNOTATION~PRINCIPLES}


\def\titkol{Implicit logical-semantic relations in~parallel texts: Annotation 
principles}


\def\aut{A.\,A.~Goncharov and~P.\,V.~Iaroshenko}

\def\autkol{A.\,A.~Goncharov and~P.\,V.~Iaroshenko}

\titel{\tit}{\aut}{\autkol}{\titkol}

\vspace*{-8pt}


\noindent 
Federal Research Center ``Computer Science and Control'' of the Russian 
Academy of Sciences, 44-2 Vavilov Str., Moscow 119333, Russian Federation




\def\leftfootline{\small{\textbf{\thepage}
\hfill INFORMATIKA I EE PRIMENENIYA~--- INFORMATICS AND
APPLICATIONS\ \ \ 2024\ \ \ volume~18\ \ \ issue\ 3}
}%
 \def\rightfootline{\small{INFORMATIKA I EE PRIMENENIYA~---
INFORMATICS AND APPLICATIONS\ \ \ 2024\ \ \ volume~18\ \ \ issue\ 3
\hfill \textbf{\thepage}}}

\vspace*{4pt}




\Abste{The problem of implicit logical-semantic relations (LSRs) annotation is considered. 
The state-of-the-art in the annotation of implicit LSRs is analyzed. 
The approaches focused on ($i$)~analysis of the global discourse structure; ($ii$)~analysis of the local 
discourse structure; and ($iii$)~unification of the data annotated within different frameworks and development of 
a~unified annotation standard are presented. The principles for annotating implicit 
LSRs in parallel texts are proposed, i.\,e., target of annotation is a~translated correspondence 
(a~pair of fragments from the original and translated texts). Translation correspondences illustrating 
implicit--explicit mismatch have been studied, i.\,e., where LSR markers are absent in the Russian text while 
in the text in another language, on the contrary, they are present. Taking into account the specificity of implicit LSRs, the 
following principles of their annotation were formulated: ($i$)~it is necessary to determine the 
boundaries of LSR arguments (to ensure clarity and convenience of analysis); ($ii$)~features of 
text blocks should form a hierarchical structure (to ensure convenience of using a large number 
of features); and ($iii$)~if a feature of a text block has a lexical marker, this marker should be indicated 
(to ensure better justification of the annotator's decisions).}


\KWE{linguistic annotation; discourse relations; logical-semantic relations; implicitness; 
parallel texts}



\DOI{10.14357/19922264240313}{NPXQNX}

\vspace*{-12pt}


    
      \Ack

\vspace*{-3pt}

\noindent
The research was supported by the Russian Science Foundation, project No.\,24-28-00527 ({\sf 
https:// rscf.ru/en/project/24-28-00527/}).



  \begin{multicols}{2}

\renewcommand{\bibname}{\protect\rmfamily References}
%\renewcommand{\bibname}{\large\protect\rm References}

{\small\frenchspacing
 {%\baselineskip=10.8pt
 \addcontentsline{toc}{section}{References}
 \begin{thebibliography}{99} 
\bibitem{1-gon-1}
\Aue{Goncharov, A.\,A.} 2021. Klassifikatsii vnutritekstovykh otnosheniy: osnovaniya 
i~printsipy strukturirovaniya [Classifications of intratextual relations: Bases and structuring 
principles]. \textit{Voprosy yazykoznaniya} [Topics in the Study of Language] 3:97--119. doi: 10.31857/0373-658X.2021.3.97-119. EDN: OKPZEI.
\bibitem{2-gon-1}
\Aue{Stede, M.} 2012. \textit{Discourse processing}. San Rafael, CA: Morgan \& Claypool Publs. 
155~p.
\bibitem{3-gon-1}
\Aue{Hobbs, J.\,R.} 1990. \textit{Literature and cognition}. Stanford, CA: CSLI. 193~p.
\bibitem{4-gon-1}
\Aue{Kibrik, A.\,A.} 2003. Analiz diskursa v~kognitivnoy perspektive [Discourse analysis in 
cognitive perspective]. Moscow: Institute of Linguistics RAS. D.Sc. Diss. 90~p.
\bibitem{5-gon-1}
\Aue{Xiang, W., and B.~Wang.} 2023. A~survey of implicit discourse relation recognition. \textit{ACM 
Comput. Surv.} 55(12):258. 34~p. doi: 10.1145/3574134.
\bibitem{6-gon-1}
\Aue{Jiang, D., and J.~He.} 2020. Tree framework with BERT word embedding for the 
recognition of Chinese implicit discourse relations. \textit{IEEE Access} 8:162004--162011. doi: 
10.1109/ACCESS.2020.3019500.
\bibitem{7-gon-1}
\Aue{Taboada, M., and W.~Mann.} 2006. Rhetorical structure theory: Looking back and moving 
ahead. \textit{Discourse Stud.} 8(3):423--459. doi: 10.1177/146144560606188.
\bibitem{8-gon-1}
\Aue{Carlson, L., and D.~Marcu.} 2001. Discourse tagging reference manual. ISI Technical 
Report ISI-TR-545. 87~p.  Available at: {\sf https://www.isi.edu/content/tr/tr-545.pdf} (accessed August~2, 
2024).
\bibitem{9-gon-1}
\Aue{Das, D., and M.~Taboada}. 2014. RST Signalling Corpus annotation manual. Available at: 
{\sf https://www.sfu.ca/\linebreak $\sim$mtaboada/docs/publications/RST\_Signalling\_Corpus\_\linebreak Annotation\_Manual.pdf} 
(accessed August~2, 2024).
\bibitem{10-gon-1}
\Aue{Pisarevskaya, D., M.~Ananyeva, M.~Kobozeva, A.~Nasedkin, S.~Nikiforova, I.~Pavlova, 
and A.~Shelepov.} 2017. Towards building a discourse-annotated corpus of Russian. 
\textit{Komp'yuternaya lingvistika i~intellektual'nye tekhnologii} 
[Computational 
linguistics and intellectual technologies].  Moscow: RSUH. 16(1):201--212.
\bibitem{11-gon-1}
\Aue{Smirnov, I.\,V.} 2023. \textit{Intellektual'nyy analiz tekstov na osno\-ve metodov raznourovnevoy 
obrabotki estestvennogo yazyka} [Intelligent text analysis based on multilevel natural language 
processing methods]. Moscow: FRC CSC RAS. 354~p.
\bibitem{12-gon-1}
\Aue{Prasad, R., E.~Miltsakaki, N.~Dinesh, A.~Lee, A.~Joshi, and B.\,L.~Webber.} 2006. The Penn 
Discourse TreeBank 1.0 annotation manual. IRCS technical reports ser. The PDTB Research Group. 54~p. Available at: 
{\sf https://catalog.\linebreak  ldc.upenn.edu/docs/LDC2008T05/papers/pdtb-1.0-annotation-manual.pdf} (accessed 
August~2, 2024).
\bibitem{13-gon-1}
\Aue{Prasad, R., B.~Webber, and A.~Joshi.} 2017. The Penn Discourse Treebank: An annotated 
corpus of discourse relations. \textit{Handbook of linguistic annotation}. Eds. N.~Ide and J.~Pustejovsky. 
Dordrecht: Springer Science\;+\;Business Media. 1197--1217. doi: 10.1007/978-94-024-0881-2\_45.
\bibitem{14-gon-1}
\Aue{Webber, B., R.~Prasad, A.~Lee, and A.~Joshi.} 2019. The Penn Discourse Treebank 3.0: 
Annotation manual. Available at: {\sf https://catalog.ldc.upenn.edu/docs/\linebreak LDC2019T05/PDTB3-Annotation-Manual.pdf} (accessed August~2, 2024).
\bibitem{15-gon-1}
\Aue{Zufferey, S., and L.~Degand.} 2024. \textit{Connectives and discourse relations. Key topics in 
semantics and pragmatics}. Cambridge: Cambridge University Press. 268~p.
\bibitem{16-gon-1}
\Aue{Fu, Y.} 2022. Towards unification of discourse annotation frameworks. \textit{60th 
Annual Meeting of the Association for Computational Linguistics Proceedings}. Dublin: Association for 
Computational Linguistics. 132--142. doi: 10.18653/v1/2022.acl-srw.12.
\bibitem{17-gon-1}
\Aue{Putra, J.\,W.\,G., K.~Matsumura, S.~Teufel, and T.~Tokunaga.} 2023. TIARA~2.0: An 
interactive tool for annotating discourse structure and text improvement. \textit{Lang. Resour. Eval.} 
57:5--29. doi: 10.1007/s10579-021-09566-0.
\bibitem{18-gon-1}
Brat rapid annotation tool. Available at: {\sf https://brat.\linebreak nlplab.org/index.html} (accessed August~2, 2024).
\bibitem{19-gon-1}
WebAnno. A~flexible, web-based and visually supported system for distributed 
annotations. Available at: {\sf https://\linebreak webanno.github.io/webanno} (accessed August~2, 2024).
\bibitem{20-gon-1}
INCEpTION. A~semantic annotation platform offering intelligent assistance and 
knowledge management. Available at: {\sf https://inception-project.github.io} (accessed August~2, 2024).
\bibitem{21-gon-1}
\Aue{Inkova, O.\,Yu.} 2019. Logiko-semanticheskie otnosheniya: problemy klassifikatsii 
[Logical-semantic relations: Classification problems]. \textit{Svyaznost' teksta: mereologicheskie 
logiko-semanticheskie otnosheniya} [Text coherence: Mereological logical semantic relations]. 
Moscow: LRC Publishing House. 11--98.
\bibitem{22-gon-1}
\Aue{Zatsman, I., M.~Kruzhkov, and E.~Loshchilova.} 2019. Metody i~sredstva informatiki dlya 
opisaniya struktury neodnoslovnykh konnektorov [Methods and means of informatics for 
multiword connectives structure description]. \textit{Struktura konnektorov i~metody ee opisaniya} 
[Connectives structure and methods of its description]. Ed. O.\,Yu.~Inkova. Moscow: TORUS 
PRESS. 205--230. doi: 10.30826/SEMANTICS19-06. EDN: YVAJWN.

\bibitem{24-gon-1}
\Aue{Goncharov, A.\,A., O.\,Yu.~Inkova, and M.~Kruzhkov.} 2019. Metodologiya annotirovaniya 
v nadkorpusnykh bazakh dannykh [Annotation methodology of supracorpora databases]. \textit{Sistemy 
i~Sredstva Informatiki~--- Systems and Means of Informatics} 29(2):148--160. doi: 10.14357/ 08696527190213. EDN: GNDCJE.

\bibitem{23-gon-1} %24
\Aue{Kruzhkov, M.} 2021. Kontseptsiya postroeniya nadkorpusnykh baz dannykh [Conceptual 
framework for supracorpora databases]. \textit{Sistemy i~Sredstva Informatiki~--- Systems and Means of 
Informatics} 31(3):101--112. doi: 10.14357/ 08696527210309. EDN: UMWNIU.

\bibitem{25-gon-1}
\Aue{Goncharov, A.\,A.} 2023. Annotirovanie parallel'nykh kor\-pu\-sov: podkhody i~napravleniya razvitiya 
[Parallel corpus annotation: Approaches and directions for development]. 
\textit{Informatika i~ee Primeneniya~--- Inform. Appl.} 17(4):81--87. doi: 10.14357/19922264230411. 
EDN: GDKDOZ.
\bibitem{26-gon-1}
Inkova, O.\,Yu., ed. 2018. \textit{Semantika konnektorov: kontrastivnoe issledovanie} [Semantics 
of connectives: A~contrastive study]. Moscow: TORUS PRESS. 368~p.
\bibitem{27-gon-1}
Inkova, O.\,Yu., ed. 2019. \textit{Struktura konnektorov i~metody ee opisaniya} [Connectives 
structure and methods of its description]. Moscow: TORUS PRESS.  316~p. EDN: VVIINM.
\bibitem{28-gon-1}
\Aue{Goncharov, A.\,A., and O.\,Yu.~Inkova.} 2019. Metodika poiska implitsitnykh 
logiko-semanticheskikh otnosheniy v~tekste [Methods for identification of implicit logical-semantic 
relations in texts]. \textit{Informatika i~ee Primeneniya~--- Inform. Appl.} 13(3):97--104. doi: 
10.14357/ 19922264190314. EDN: MWGFJW.
\bibitem{29-gon-1}
\Aue{Nuriev, V.\,A., and I.\,M.~Zatsman.} 2020. Re\-du\-tsi\-ro\-va\-nie spektra modeley perevoda 
v~nad\-kor\-pus\-nykh ba\-zakh dan\-nykh [Reducing the spectrum of translation models in supracorpora 
databases]. \textit{Informatika i~ee Primeneniya~--- Inform. Appl.} 14(2):119--126. doi:  10.14357/ 19922264200217. EDN: EBUTTA.

\columnbreak

\bibitem{30-gon-1}
\Aue{Goncharov, A.\,A., and O.\,Yu.~Inkova.} 2021. Izvlechenie znaniy o sredstvakh vyrazheniya 
logiko-semanticheskikh\linebreak otnosheniy pri pomoshchi nadkorpusnoy bazy dannykh [Extracting 
knowledge about means of expression of logical-semantic relations from the supracorpora 
database]. \textit{Informatika i~ee Primeneniya~--- \mbox{Inform.} \mbox{Appl}.} 15(2):96--103. doi:  10.14357/19922264210214. EDN: GRPWIB.
\bibitem{31-gon-1}
\Aue{Inkova, O.\,Yu., and M.\,G.~Kruzhkov.} 2021. Strukturirovannye opredeleniya 
diskursivnykh otnosheniy v~Nadkorpusnoy baze dannykh konnektorov [Structured definitions 
of discourse relations in the Supracorpora Database of Connectives]. \textit{Informatika i~ee 
Primeneniya~--- Inform. Appl.} 15(4):27--32. doi: 10.14357/ 19922264210404. EDN: EZJXVI.
\bibitem{32-gon-1}
\Aue{Durnovo, A.\,A., O.\,Yu.~Inkova, and N.\,A.~Popkova.} 2022. Printsipy opisaniya 
pokazateley logiko-semanticheskikh otnosheniy i~ikh ierarkhii [Principles of describing markers 
of logical-semantic relations and their hierarchy]. \textit{Informatika i ee Primeneniya~--- Inform. Appl.} 
16(2):52--59. doi: 10.14357/19922264220207. EDN: NPFTOH.
\bibitem{33-gon-1}
\Aue{Bunt, H., and R.~Prasad.} 2016. ISO DR-Core (ISO 24617-8): Core concepts for the 
annotation of discourse relations. \textit{12th Joint ACL-ISO Workshop on Interoperable 
Semantic Annotation Proceedings}. Portoro{\hspace*{-0.5pt}\!\ptb\v{z}}, Slovenia. 45--54.



\end{thebibliography}

 }
 }

\end{multicols}

\vspace*{-6pt}

\hfill{\small\textit{Received July 12, 2024}} 

\vspace*{-18pt}

\Contr

\vspace*{-3pt}

\noindent
\textbf{Goncharov Alexander A.} (b.\ 1994)~--- scientist, Federal Research Center ``Computer 
Science and Control'' of the Russian Academy of Sciences, 44-2~Vavilov Str., Moscow 119333, 
Russian Federation; \mbox{a.gonch48@gmail.com}

\vspace*{3pt}

\noindent
\textbf{Iaroshenko Polina V.} (b.\ 1994)~--- Candidate of Science (PhD) in philology, scientist, 
Federal Research Center ``Computer Science and Control'' of the Russian Academy of Sciences, 
44-2~Vavilov Str., Moscow 119333, Russian Federation; \mbox{polina.iaroshenko@yandex.ru}

\label{end\stat}

\renewcommand{\bibname}{\protect\rm Литература} 