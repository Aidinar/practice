\def\stat{bosov}

\def\tit{СТАБИЛИЗАЦИЯ АВТОНОМНОГО ЛИНЕЙНОГО ВЫХОДА МАРКОВСКОЙ ЦЕПИ 
ПО~КВАДРАТИЧНОМУ КРИТЕРИЮ НА~БЕСКОНЕЧНОМ ГОРИЗОНТЕ$^*$}

\def\titkol{Стабилизация автономного линейного выхода марковской цепи 
по~квадратичному критерию} % на бесконечном горизонте}

\def\aut{А.\,В.~Босов$^1$}

\def\autkol{А.\,В.~Босов}

\titel{\tit}{\aut}{\autkol}{\titkol}

\index{Босов А.\,В.}
\index{Bosov A.\,V.}


{\renewcommand{\thefootnote}{\fnsymbol{footnote}} \footnotetext[1]
{Работа выполнялась с~использованием инфраструктуры Центра коллективного пользования 
<<Высокопроизводительные вы\-чис\-ле\-ния и~большие данные>> 
(ЦКП <<Информатика>>) ФИЦ ИУ РАН 
(г.~Москва).}}


\renewcommand{\thefootnote}{\arabic{footnote}}
\footnotetext[1]{Федеральный исследовательский центр <<Информатика и~управление>> Российской академии наук, 
\mbox{ABosov@frccsc.ru}}

\vspace*{-6pt}


\Abst{Решение задачи оптимального управления линейным выходом 
стохастической дифференциальной системы на бесконечном горизонте 
адаптировано для одного частного случая косвенных наблюдений. Роль 
динамической системы выполняет эргодическая цепь Маркова, а~фор\-ми\-ру\-емый 
ею автономный линейный выход обеспечивает косвенные зашумленные 
наблюдения о~состоянии цепи. Цель управления формулируется как стабилизация 
выхода в~положениях, определяемых цепью и~периодически изменяющихся при 
изменениях состояния цепи. Решение, как и~в~аналогичной задаче с~полной 
информацией, получено как предельная форма оптимального управ\-ле\-ния 
в~соответствующей задаче с~конечным горизонтом. Достаточные условия 
существования управления оказываются типовыми для ли\-ней\-но-квад\-ра\-тич\-ных 
задач условиями стационарности, детектируемости и~стабилизируемости. Из-за 
специального вида задачи, обеспечиваемого структурой оптимального фильт\-ра 
Вонэма, в~управ\-ле\-нии присутствуют только линейные компоненты, а~вся 
нелинейность ограничена уравнением оценки фильт\-ра\-ции, поэтому условия 
существования включают только требования к~решению автономного уравнения 
Риккати. Обсуждается чис\-лен\-ный эксперимент для модели механического 
привода, использованной в~предыду\-щих исследованиях. Цель эксперимента~--- 
показать разницу в~использовании оптимального управ\-ле\-ния и~его автономного 
варианта.}

\KW{цепь Маркова с~непрерывным временем; фильтр Вонэма; линейная 
дифференциальная сис\-те\-ма; оптимальное управ\-ле\-ние; стабилизация; 
квадратичный критерий}

\DOI{10.14357/19922264240305}{XVTFLH}
  
\vspace*{-2pt}


\vskip 10pt plus 9pt minus 6pt

\thispagestyle{headings}

\begin{multicols}{2}

\label{st\stat}

\section{Введение}

     Типовым вариантом классической задачи ли\-ней\-но-квад\-ра\-тич\-но\-го 
гауссовского (LQG, linear-quadratic-Gaussian) управ\-ле\-ния~[1] можно считать постановку 
с~бесконечным горизонтом управления~[2, 3], причем отличий между 
случаями с~полной информацией и~косвенными наблюдениями за\linebreak состоянием 
системы нет. Обеспечивается это уникальными свойствами оптимальной 
оценки, описываемой фильтром Кал\-ма\-на--Бью\-си~[4], или, \mbox{другими} 
словами, двойственностью задач LQG-управ\-ле\-ния и~фильтрации, которая 
все вопросы сводит к~анализу поведения решения дифференциального 
уравнения Риккати на бесконечности. В~отсутствие  
ли\-ней\-но-гаус\-сов\-ских условий изучать задачу с~бесконечным 
горизонтом становится сложно из-за трудностей, связанных с~записью 
уравнений для оценки оптимальной фильтрации. Лучше всего для 
формирования модели наблюдения могли бы подойти общие уравнения 
нелинейной фильтрации на основе обновляющих процессов~[5], но их 
конструкция такова, что не дает шансов для практической реализации. 
Занимающим промежуточное положение можно считать фильтр Вонэма~[6], 
описывающий оценку состояния непрерывной цепи Маркова по косвенным 
линейно-гауссовским наблюдениям. Поведение цепи Маркова на 
бесконечности прекрасно изучено~[7], требование эргодичности хорошо 
согласуется с~духом задачи с~непрерывным временем. Таким образом, 
логичным представляется полученное ранее в~[8] оптимальное автономное 
управ\-ле\-ние в~задаче управ\-ле\-ния линейным выходом стохастической 
дифференциальной системы на бесконечном горизонте в~случае полной 
информации дополнить оптимальным автономным управ\-ле\-ни\-ем в~задаче 
стабилизации линейного выхода стохастической дифференциальной сис\-те\-мы 
на бесконечном горизонте в~случае косвенных\linebreak наблюдений за состоянием 
фор\-ми\-ру\-ющей выход дискретной цепи Маркова.

\vspace*{-4pt}
     
\section{Постановка задачи}

     На каноническом вероятностном пространстве $(\Omega, 
\mathcal{F},\mathcal{P}, \mathcal{F}_t)$, $t\hm\in [0,\infty)$, рассмотрим 
автономную линейную дифференциальную стохастическую сис\-те\-му 
с~управ\-ля\-емым вектором выхода $z_t \hm\in \mathbb{R}^{n_z}$:
     \begin{multline}
     dz_t= ay_t\,dt+bz_t\,dt+cu_t\,dt+\sigma \,dw_t,\\ z_0=Z,\enskip t\in [0,\infty).
     \label{e1-bos}
     \end{multline}
      
      В отличие от обычной линейной системы траектории~$z_t$ в~(1) 
имеют систематическую со\-став\-ля\-ющую, которую задает марковский 
скачкообразный процесс~$y_t$~--- цепь с~конечным числом \mbox{состояний} 
и~значениями в~множестве $\{ e_1, \ldots, e_{n_y}\}$, состоящем из 
единичных координатных векторов в~евклидовом 
пространстве~$\mathbb{R}^{n_y}$. Цепь~$y_t$ предполагается 
эргодической, поэтому матрица интенсивностей переходов~$\Lambda$ не 
зависит от времени, предельное распределение обозначается~$\pi_*$, 
начальное распределение~--- $y_0\hm=Y$. Остальные величины 
в~уравнении~(1):
      \begin{itemize}
        \item $w_t\in \mathbb{R}^{n_w}$~--- стандартный векторный винеровский процесс;
        \item $Z\in \mathbb{R}^{n_z}$~--- гауссовский случайный вектор с~известным 
математическим ожиданием и~ковариацией, $w_t$, $y_t$ и~$Z$ независимы в~совокупности;
        \item $u_t\in\mathbb{R}^{n_u}$~--- управление~--- случайный процесс с~конечным 
вторым моментом;
        \item $a\in \mathbb{R}^{n_z\times n_y}$, $b\hm\in \mathbb{R}^{n_z\times n_z}$, $c\hm\in 
\mathbb{R}^{n_z\times n_u}$ и~$\sigma\hm\in \mathbb{R}^{n_z\times n_w}$~--- заданные 
матрицы.
\end{itemize}
     
     Состояние цепи~$y_t$ предполагается неизвестным, так что~$z_t$ 
выполняет роль косвенных наблюдений. Соответственно, допустимые 
управления предполагаются $\mathcal{F}_t^z$-из\-ме\-ри\-мы\-ми 
(через~$\mathcal{F}_t^z$ обозначена $\sigma$-ал\-геб\-ра наблюдений и~выполнено 
$\mathcal{F}_t^z\hm\subseteq \mathcal{F}_t \hm\subseteq \mathcal{F}$). Кроме 
того, c учетом бесконечного времени $t\hm\in[0,\infty)$ и~марковского 
характера задачи класс~$U_0^\infty$ допустимых управлений сформирован 
автономными (не зависящими прямо от времени) управлениями с~полной 
обратной связью, т.\,е.\ функциями вида 
$$
u_t= u(z)\in  \mathbb{R}^{n_u},\enskip z\in \mathbb{R}^{n_z},
$$ в~предположении, что 
соответствующая реализация $u_t\hm= u(z_t)$ обеспечивает выполнение 
условий существования решения~(1) для $u\hm\in U_0^\infty$. Поскольку 
состояние~$y_t$ от управ\-ле\-ния~$u_t$ не зависит, а~выход~$z_t$ 
описывается линейным автономным уравнением с~винеровским процессом, 
то данное формальное требование ограничивает допустимые управления 
процессами второго порядка, что обеспечивает существование решения~(1) 
на любом конечном интервале $t\hm\in [0,T]$. Для управления на интервале 
$[0,\infty)$ дополнительно потребуются типовые условия 
стабилизируемости~[2, 3], обсуждаемые далее.
     
     Качество управления $U_0^\infty$ определяется целевым 
функционалом сле\-ду\-юще\-го вида:
     \begin{multline}
          J\left(U_0^\infty\right) =\lim\limits_{T\to\infty} J\left(U_0^T\right),\enskip
              J\left(U_0^T\right)= {}\\
              {}= \mathbb{E}\left\{ \fr{1}{T} \int\limits_0^T \left\| Py_t+Qz_t+R 
u_t\right\|_s^2\,dt\right\},
 \label{e2-bos}
 \end{multline}

\vspace*{-4pt}

\noindent
где $P\in \mathbb{R}^{n_J\times n_y}$, $Q\hm\in \mathbb{R}^{n_J\times 
n_z}$, $R\hm\in \mathbb{R}^{n_J\times n_u}$ и~$S\hm\in 
\mathbb{R}^{n_J\times n_J}$ ($S\hm\geq 0$, $S\hm= S^\prime$)~--- заданные 
матрицы, весовая функция $\| x\|_S^2\hm= x^\prime Sx$; единичной матрице 
$S\hm=E$ соответствует евклидова норма $\| x\|^2_E\hm= \vert x\vert^2$; 
<<${}^\prime$>>~--- операция транспонирования; $\mathbb{E}\{\cdot\}$~--- 
оператор математического ожидания (далее еще используется обозначение 
$\mathbb{E}\{\cdot\vert\mathcal{F}\}$ для условного математического 
ожидания относительно $\sigma$-ал\-геб\-ры~$\mathcal{F}$). Кроме того, 
предполагается выполненным обычное условие невырожденности, 
в~используемых обозначениях принимающее вид $R^\prime SR>0$.

     В отношении физического смысла критерия~(\ref{e2-bos}) заметим, что 
цепь~$y_t$, изменив состояние, некоторое время сохраняет постоянное 
значение, которое и~вносит в~уравнение~(1) упомянутую систематическую 
составляющую. Аналогично в~целевом функционале присутствуют 
слагаемые, фор\-ми\-ру\-емые <<ступеньками>> на интервалах 
постоянства~$y_t$. Таким образом,~(\ref{e2-bos}) формализует цель управ\-ле\-ния, 
состоящую в~стабилизации выхода~$z_t$ около из\-ме\-ня\-ющих\-ся  
ку\-соч\-но-по\-сто\-ян\-ных положений, задаваемых состоянием цепи~$y_t$. 
Иными словами, эти положения можно назвать на\-прав\-ле\-ни\-ями дрейфа, 
а~целью управ\-ле\-ния~--- обеспечение дрейфа выхода сис\-те\-мы~$z_t$ 
в~заданном и~периодически изменяющемся направлении. Задача со\-сто\-ит, 
таким образом, в~поиске $(U^*)_0^\infty\hm= \{u^*(z), z\hm\in 
\mathbb{R}^{n_z}\}$~--- допустимого управления, реализации $u_t^*\hm= 
u^*(z_t^*)$, $t\hm\in [0,\infty)$, которого доставляют минимум 
квадратичному функционалу~$J(U_0^\infty)$:
     \begin{equation}
     \left( U^*\right)_0^\infty =\argmin\limits_{u\in U_0^\infty} J\left( 
U_0^\infty\right).
     \label{e3-bos}
     \end{equation} 
     
     \vspace*{-4pt}

     
     Далее через~$z_t^*$ обозначается решение~(1), отвечающее~$u_t^*$  
и~учитывается, что~$y_t$ от~$u_t$ не зависит.


\vspace*{-9pt}


\section{Основной результат}

\vspace*{-3pt}

     Как и~в постановке с~полной информацией, рассмотренной в~[8], здесь 
также основу для решения~(\ref{e3-bos}) обеспечивает решение 
соответствующей задачи с~конечным горизонтом, которое получено в~[9]. 
Ключевую роль в~этом решении играет оценка оптимальной фильтрации 
состояния~$y_t$ по наблюдениям $\{z_\tau, 0\hm\leq \tau\hm\leq t\}$, т.\,е.\ 
условное матема-\linebreak\vspace*{-12pt}

\pagebreak

\noindent
тическое ожидание $\hat{y}_t\hm= \mathbb{E}\{ y_t\vert 
\mathcal{F}_t^z\}$, которое задается фильтром Вонэма~[2, 10]:

\vspace*{-6pt}

\noindent
     \begin{multline}
     d\hat{y}_t=\Lambda^\prime \hat{y}_t \,dt +\left( 
\mathrm{diag}\,(\hat{y}_t) - \hat{y}_t \hat{y}_t^\prime\right) 
a^\prime(\sigma\sigma^\prime)^{-1}\times{}\\
     {}\times \left( dz_t-a\hat{y}_t\,dt -bz_t\,dt -cu_t\,dt\right) ,\enskip
     \hat{y}_0=\mathbb{E}\{Y\}.
     \label{e4-bos}
     \end{multline}
     
     \vspace*{-3pt}
     
     Для удобства записи функциональную зависимость оценки от 
наблюдений, определяемую уравнением~(\ref{e4-bos}), обозначим 
$\hat{y}(z)$, так чтобы $\hat{y}_t\hm= \hat{y}(z_t)$.
     
     Для корректного предельного перехода от постановки с~конечным 
горизонтом к~бесконечному времени требуется выполнение ряда условий, 
объединенных в~сле\-ду\-ющем утверж\-де\-нии.
     
     \smallskip
     
     \noindent
     \textbf{Теорема.} \textit{Решение задачи}~(\ref{e3-bos}) \textit{может 
быть записано в~виде}

\vspace*{-4pt}

\noindent
     \begin{multline}
     u^*(z) =-\fr{1}{2} \left( R^\prime SR\right)^{-1} \left( c^\prime 
(2\alpha_*z+\beta_* \hat{y}(z))+{}\right.\\
\left.{}+2R^\prime S \left( P\hat{y}(z)+Qz\right)\right),
     \label{e5-bos}
     \end{multline}
     
          \vspace*{-3pt}
          
          \noindent
\textit{где симметричная неотрицательно определенная мат\-ри\-ца 
$\alpha_*\hm\in \mathbb{R}^{n_z\times n_z}$ и~прямоугольная мат\-ри\-ца 
$\beta_*\hm\in \mathbb{R}^{n_z\times n_y}$ представляют собой решения 
уравнений}

\vspace*{-4pt}

\noindent
\begin{multline}
\left( b^\prime -Q^\prime SR\left(R^\prime SR\right)^{-1} c^\prime\right)\alpha_* 
+{}\\
{}+\alpha_* \left( b-c\left( R^\prime SR\right)^{-1} R^\prime SQ\right)+{}\\
{}+ Q^\prime \left( S-SR\left( R^\prime SR\right)^{-1} R^\prime S\right) Q-{}\\
{}-
\alpha_* c \left( R^\prime SR\right)^{-1} c^\prime \alpha_*=0\,;
\label{e6-bos}
\end{multline}

\noindent
\begin{equation}
\beta_*\Lambda^\prime +M_* -N_*\beta_*=0\,,
\label{e7-bos}
\end{equation}

\vspace*{-3pt}

\noindent
где

\vspace*{-6pt}

\noindent
\begin{multline*}
M_*= 2\left( \left(a^\prime -P^\prime SR\left( R^\prime SR\right)^{-1} 
c^\prime\right) \alpha_* +{}\right.\\
\left.{}+P^\prime\left( S-SR\left( R^\prime SR\right)^{-1} R^\prime S\right) Q\right)\,;
\end{multline*}
$$
N_*= b-c\left( R^\prime SR\right)^{-1} R^\prime SQ -c\left( R^\prime SR\right)^{-
1} c^\prime \alpha_*\,,
$$

\vspace*{-4pt}

\noindent
\textit{если для параметров системы наблюдения~$y_t$, $z_t$ и~целевого 
функционала}~(\ref{e2-bos}) \textit{выполнены следующие условия}:
\begin{enumerate}[(1)]
\item  \textit{матрица~$b$ устойчива};
\item  \textit{матрицы $(K_b, c)$ стабилизируемы}, $K_b\hm= b\hm- 
c(R^\prime SR)^{-1} R^\prime SQ$;
\item  \textit{матрицы $(K_b^\prime, K_Q)$ стабилизируемы}; 
\item  $K_Q\hm= Q^\prime S^{1/2} (E\hm- S^{1/2} R (R^\prime SR)^{-1} 
R^\prime S^{1/2})$.
\end{enumerate}
     
     \noindent
     Д\,о\,к\,а\,з\,а\,т\,е\,л\,ь\,с\,т\,в\,о\,.\ \  Перечисленные в~теореме 
условия обеспечивают существование предельного решения 
соответствующей~(\ref{e3-bos}) задачи с~конечным горизонтом, т.\,е.\ 
управления 
$$
\left(U^\#\right)_0^T =\argmin\nolimits_{u_t\in U_0^T} J\left(U_0^T\right),
$$

\vspace*{-4pt}

\noindent
 где 
$U_0^T\hm= \{u_t(z,T), z\hm\in \mathbb{R}^{n_z}, t\hm\in [0,T]\}$. 
Оптимальное управ\-ле\-ние $u_t^\# \hm= u_t^\#(z_t^\#, T)$ получено в~[9] 
в~результате разделения задач управ\-ле\-ния и~фильт\-ра\-ции в~виде

\vspace*{-2pt}

\noindent
     \begin{multline}
     u_t^{\#} =u_t^{\#}(z,T) =-\fr{1}{2} \left( R^\prime SR\right)^{-1} \left(
 c^\prime \left( 2\alpha_t z+{}\right.\right.\\
\left.\left.{}+\beta_t \hat{y}(z)\right) +2R^\prime S\left( 
P\hat{y}(z) +Qz\right)\right),
     \label{e8-bos}
     \end{multline}
     
     \vspace*{-2pt}

\noindent
где матричные коэффициенты $\alpha_t \hm= \alpha_t(T)\hm\in 
\mathbb{R}^{n_z\times n_z}$ и~$\beta_t\hm= \beta_t(T)\hm\in 
\mathbb{R}^{n_z\times n_y}$ {представляют собой} решения задач 
Коши для уравнения Риккати

\vspace*{-6pt}

\noindent
\begin{multline}
\fr{d\alpha_t}{dt} +\left( b^\prime - Q^\prime SR\left( R^\prime SR\right)^{-1} 
c^\prime\right)\alpha_t+{}\\
{}+ \alpha_t\left( b-c\left( R^\prime SR\right)^{-1} R^\prime SQ\right)+{}\\
{}+ Q^\prime \left( S-SR\left(R^\prime SR\right)^{-1} R^\prime S\right)Q -{}\\
{}-
\alpha_t c \left( R^\prime SR\right)^{-1} c^\prime \alpha_t =0\,,\enskip 
\alpha_T=Q^\prime SQ\,,
\label{e9-bos}
\end{multline}

\vspace*{-4pt}

\noindent
и линейного уравнения
\begin{equation}
\fr{d\beta_t}{dt}= \beta_t\Lambda^\prime +M_t -N_t\beta_t=0\,,\enskip
\beta_T=2Q^\prime_T S_T P_T\,,
\label{e10-bos}
\end{equation}

\vspace*{-2pt}

\noindent
где

\vspace*{-7pt}

\noindent
\begin{multline*}
M_t= M_t(T) =2\left(\left( a^\prime -P^\prime SR \left( R^\prime SR\right)^{-1} 
c^\prime \right) \alpha_t +{}\right.\\[-1pt]
\left.{}+P^\prime \left( S-SR\left( R^\prime SR\right)^{-1} R^\prime S\right) 
Q\right)\,;
\end{multline*}

\vspace*{-14pt}

\noindent
\begin{multline*}
N_t = N_t(T) ={}\\
{}= b - c\left(R^\prime SR\right)^{-1} R^\prime SQ -c\left( R^\prime 
SR\right)^{-1} c^\prime \alpha_t.
%\label{e11-bos}
\end{multline*}

\vspace*{-4pt}

     
     Условия 1--3 теоремы~--- это условия существования предельного 
решения уравнения Рикатти\linebreak $\alpha_*\hm= \lim_{T\to\infty} \alpha_t(T)$ 
и~оптимального решения в~классической задаче  
ли\-ней\-но-квад\-ра\-тич\-но\-го гауссовского управления с~бесконечным 
временем (ис\-поль\-зу\-емые в~рас\-смат\-ри\-ва\-емой задаче обозначения приведены к~формулировкам условий тео\-ре\-мы~12.2 из монографии~[2]). Отсюда 
получаем, что существует мат\-ри\-ца $N_*\hm= \lim_{T\to \infty} N_t(T)$ 
(предел существует и~не зависит от~$t$, так как~$N_t(T)$ выражается 
линейно через $\alpha_t(T)$). Более того, эта же тео\-ре\-ма утверж\-да\-ет, что 
матрица~$N_*$ устойчива, что вместе с~изначально предполагаемой 
эргодичностью~$y_t$ дает устойчивость уравнения~(\ref{e10-bos}), т.\,е.\ 
существование предельной матрицы~$\beta_*$, что завершает 
до\-ка\-за\-тель\-ство.
     
     \smallskip
     
     Отметим, что для предельного перехода $\beta_*\hm= \lim_{T\to\infty} 
\beta_t(T)$ не потребовались дополнительные условия. Это связано со 
специальным видом нелинейной динамики рассматриваемой системы, 
заданным фильт\-ром Вонэма~(\ref{e4-bos}), а~именно: данная 
модель отвечает частному случаю линейного сноса в~задаче управления 
линейным дифференциальным выходом общего вида, рассмотренному 
подробно в~[11]. Более того, снос~$\Lambda^\prime \hat{y}_t$ не просто 
линейный, а аффинный, без свободного члена, что предельно упрощает вид 
управления и~приводит к~тому, что для сходимости достаточно 
эр\-го\-дич\-ности~$y_t$.

%\pagebreak
     
     Также можно заметить, что в~[11] решение\linebreak включает еще и~конечный 
вид функции Беллмана. Нетрудно показать, что эти соотношения \mbox{можно} 
трансформировать и~для рассматриваемой задачи~(\ref{e3-bos}), но кроме 
формального значения практического смысла в~них нет.
     
     Наконец, надо отметить, что по сравнению с~задачей с~полной 
информацией, рассмотренной\linebreak в~[9], здесь оптимальное  
управ\-ле\-ние~(\ref{e5-bos}) нелинейным считается лишь формально, 
а~именно: нелинейность управлению~$u^*$ дает только оценка фильтрации 
$\hat{y}_t\hm= \hat{y}(z_t)$. Если же рассматривать\linebreak \mbox{управление}~$u^*$ как 
функцию двух переменных~$z_t$ и~$\hat{y}_t$, то оно оказывается 
линейным. Такое свойство управ\-ле\-ния в~[12] названо сильным принципом 
разделения.
     
\section{Численный эксперимент}

     Для иллюстрации того, как работает оптимальное автономное 
управление $u^*\hm= u^*(z)$, предлагается сравнить его с~оптимальным 
неавтономным управлением $u_t^\# \hm= u_t^\# (z,T)$, т.\,е.\ 
проанализировать, как происходит переход системы~$y_t, z_t$ 
в~стационарный режим и~как влияет неоптимальность автономного 
управления на результат на конечном горизонте. Для этого использована 
детально исследованная в~[9] модель простого механического привода
     \begin{equation}
     \left.
\begin{array}{rl}
     \hspace*{-2mm}dx_t&=v_t\,dt,\ t\in (0,T],\enskip T=10\,;\\[6pt]
     \hspace*{-2mm}dv_t&=ax_t\,dt +bv_t\,dt +cy_t\,dt+h u_t\,dt +{}\\[6pt]
     &\hspace*{40mm}{}+\sqrt{g}\,dw_t.
     \end{array}
     \!\right\}
     \label{e12-bos}
     \end{equation}
     
     Цепь имеет размерность $n_y\hm=3$, начальное распределение 
     $$
     Y= (1,0,0)^\prime;\enskip
     \lambda = \begin{pmatrix}
     -\fr{1}{2} & \fr{1}{2} &0\\
     \fr{1}{2} & -1 & \fr{1}{2}\\
     0 & \fr{1}{2} & -\fr{1}{2}
     \end{pmatrix}.
     $$
      Начальные условия~$x_0$ и~$v_0$ предполагаются 
независимыми гауссовскими случайными величинами с~нулевым средним 
и~дисперсиями $\sigma_x^2\hm=\sigma_v^2\hm=1$. Остальные параметры: 
$a\hm= -1$; $b\hm= -1/2$; $h\hm=10$; $g\hm= 0{,}01$; $c\hm= (c_1, c_2, 
c_3) \hm= (-1,0,1)$. Устойчивость системы~(\ref{e12-bos}) обеспечивается 
тем, что $b\hm<0$ и~$b^2\hm+4a\hm= -3{,}75\hm<0$, поскольку~$b$ 
и~$b^2\hm+4a$~--- собственные числа матрицы системы $\begin{pmatrix}
     0& 1\\
     a &b\end{pmatrix}$.
     
     Цель управления состоит в~отслеживании приводом формируемого 
цепью дрейфа $cy_t$, для чего используется целевой функционал:
%\noindent
     \begin{equation}
     J\left( U_0^T\right) =\mathbb{E}\left\{ \int\limits_0^T \left( \left\vert C y_t-
x_t \right\vert^2 +R\left\vert u_t\right\vert^2\right) dt\right\},
     \label{e13-bos}
     \end{equation}
где 
$$
C=\left( -\fr{c_1}{a}, -\fr{c_2}{a}, -\fr{c_3}{a}\right) = (-1,0,1) =c;\enskip 
R=0{,}001.
$$

\setcounter{figure}{1}
\begin{figure*} %fig2
\vspace*{1pt}
  \begin{center}
 \mbox{%
 \epsfxsize=163mm 
\epsfbox{bos-2.eps}
 }
\end{center}
\vspace*{-9pt}
\Caption{Формирование $\alpha_t$~(\textit{а}) и~$\beta_t$~(\textit{б}) предельных значений~$\alpha_*$~(\textit{а}) 
и~$\beta_*$~(\textit{б}): 
\textit{1}~--- $(\alpha_t)_{11}$; \textit{2}~--- $(\alpha_t)_{12}$; \textit{3}~--- 
$(\alpha_t)_{22}$; \textit{4}~--- $(\beta_t)_{11}$; \textit{5}~--- $(\beta_t)_{13}$;  
\textit{6}~--- $(\beta_t)_{21}$; \textit{7}~--- $(\beta_t)_{23}$}
%\end{figure*}
%\begin{table*}\small
\vspace*{9pt}
\begin{center}
{\small \begin{tabular}{|c|c|c|c|c|}
\multicolumn{5}{c}{Результаты управления $J(U_0^T)$}\\
\multicolumn{5}{c}{\ }\\[-6pt] 
\hline
$u_t^{(0)}=0$&$u_t^*=u^*(z_t^*)$&$u_t^{**}=u^*(y_t, z_t^{**})$&$u_t^\#= u_t^\# 
(z_t^\#, T)$ &$u_t^\# =u_t^\# (y_t, z_t^\#, T)$\\
\hline
8,418&1,094&0,654&1,093&0,653\\
\hline
\end{tabular}
}
\end{center}
%\end{table*}
\end{figure*}
     
    

     
     Перечисленные параметры формируют самый типовой вариант 
расчета, когда система~(\ref{e12-bos}) обладает устойчивостью, отслеживает 
дрейф цепи и~без управ\-ле\-ния, т.\,е.\ при $u_t\hm=0$, не возникают изуча\-емые 
в~[9] нюансы в~поведении чис\-лен\-ных реализаций фильт\-ра Вонэма, а~само 
оптимальное управ\-ле\-ние действует очень эффективно. Для иллюстрации\linebreak 
последнего утверж\-де\-ния на рис.~1 наряду с~траекториями~$x_t^*$ 
и~$x_t^{(0)}$, фор\-ми\-ру\-емы\-ми соответственно оптимальным автономным 
управ\-ле\-ни\-ем\linebreak $u_t^*\hm= u^*(z_t^*)$ и~нулевым управ\-ле\-нием $u_t^{(0)} \hm= 
u^{(0)}(z_t^{(0)})\hm=0$, приведена еще траектория оптимального 
управ\-ле\-ния с~полной информацией $u_t^{**}\hm= u^{**}(y_t, z_t^{**})$, 
которое вы\-чис\-ля\-ет\-ся так же, как
 управ\-ле\-ние~(\ref{e8-bos}), но с~заменой 
оценки~$\hat{y}_t$ на точ\-ное значение~$y_t$ (оп\-ти\-маль\-ность этого управ\-ле\-ния
в~задаче с~полной информацией и~конечным горизонтом доказана 
в~[9]).




     Рисунок~1 позволяет качественно оценить эффективность управления 
приводом и~показать высокую точность оценивания состояния цепи. Кроме 
того, можно видеть, что и~неуправляемая система
  обозначает своей целью 
слежение за состоянием цепи, т.\,е.\
передвижение привода в~направлении\linebreak\vspace*{-12pt}

{ \begin{center}  %fig1
 \vspace*{9pt}
   \mbox{%
 \epsfxsize=79mm 
\epsfbox{bos-1.eps}
 }

\end{center}

\vspace*{-2pt}

\noindent
{{\figurename~1}\ \ \small{Типовая траектория положения привода~$x_t$ для управлений 
$u_t^*$~(\textit{1}), $u_t^{**}$~(\textit{2}) и~$u_t^{(0)}$~(\textit{3}); \textit{4}~--- 
направление дрейфа $Cy_t$
}}}

%\vspace*{6pt}

\noindent 
дрейфа, но инерция системы при этом слишком велика, чтобы говорить хоть  
о~ка\-кой-то эффективности.
     
     Надо отметить, что траектории на рис.~1 сформированы именно 
автономным управлением, т.\,е.\ 
  вместо коэффициентов~$\alpha_t$ 
и~$\beta_t$  
из~(\ref{e9-bos}) и~(\ref{e10-bos}) использованы предельные 
значения~$\alpha_*$ и~$\beta_*$ из~(\ref{e6-bos}) и~(\ref{e7-bos}). 
Оказалось, что управ\-ле\-ния, учи\-ты\-ва\-ющие переходный процесс 
и~использующие коэффициенты~$\alpha_t$ и~$\beta_t$, визуализировать не 
имеет смысла, так как отличить траектории привода, сформированные 
автономным и~оптимальным управ\-ле\-ни\-ями, невозможно. 
Проиллюстрировать протекание переходного процесса и~формирование 
стационарного режима для автономного управ\-ле\-ния можно графиками для 
коэффициентов~$\alpha_t$ и~$\beta_t$ (рис.~2).


     Не приведены графики двух элементов матрицы~$\beta_t$, для которых 
выполнено $(\beta_t)_{12}\hm=0$ и~$(\beta_t)_{22}\hm=0$. Можно заметить, что 
для расчета управ\-ле\-ния~(\ref{e5-bos}) в~модели~(\ref{e12-bos}) используется 
только часть элементов~$\alpha_*$ и~$\beta_*$, а~именно: вторые столбцы 
этих матриц. Это объясняется вырожденностью модели~(\ref{e12-bos}) 
и~отсутствием управ\-ле\-ния в~первом уравнении для~$v_t$, что дает 
в~каноническом виде~(\ref{e1-bos}) коэффициент $c^\prime\hm= (0,h)$. Это 
замечание не отменяет значимости остальных элементов~$\alpha_*$ 
и~$\beta_*$. Все элементы этих коэффициентов управ\-ле\-ния не зависят от 
того, как в~модели реализовано аддитивное управ\-ле\-ние, но все могут 
оказаться используемыми для стабилизации разных компонентов вектора 
выхода.
     
     Из графиков, показанных на рис.~2, понятно, что переходный процесс 
очень быст\-рый и,~по сути, почти на всей моделируемой траектории в~любом 
случае применяется именно оптимальное автономное управ\-ле\-ние. Для 
формального под\-тверж\-де\-ния этого вывода приведена таб\-ли\-ца, 
в~которой указаны величины целевой функции~(\ref{e13-bos}) для всех 
типов использованных управ\-ле\-ний.

Таким образом, разница по критерию между автономным и~оптимальным управ\-ле\-ни\-ем 
проявляется только в~третьем знаке, а~с~ростом времени, естественно, 
будет уменьшаться дальше. Отметим, что связано это не только с~собственной устой\-чи\-востью сис\-те\-мы~(\ref{e12-bos}), 
но и~с~высокой эф\-фек\-тив\-ностью управ\-ле\-ний~$u_t^\#$  и~$u_t^*$, 
которые, решая задачу отслеживания заданного дрейфа, еще и~ускоряют (практически выполяют мгно\-вен\-но) 
про\-хож\-де\-ние переходного процесса и~переход сис\-те\-мы в~стационарный режим.

 
   
 %  \vspace*{-12pt}  
     
\section{Заключение}

  % \vspace*{-3pt}

     Статья завершает изучение задачи оптимального управ\-ле\-ния линейным 
выходом сто\-ха\-сти\-че\-ской сис\-те\-мы по квад\-ра\-тич\-но\-му критерию 
с~бесконечным временем. Рас\-смот\-рен\-ная в~[8] постановка этой задачи 
с~полной информацией дополнена вариантом с~косвенными наблюдениями. 
В~качестве модели со\-сто\-яния использована эргодическая цепь Маркова. 
Выбор этого част\-но\-го случая продиктован, во-пер\-вых, его практической 
значимостью,\linebreak так как марковская цепь представляет исключительно удобный 
инструмент для описания не\-конт\-ро\-ли\-ру\-емых внеш\-них воздействий, 
имеющих скачкообразный характер. Вторая причина носит \mbox{технический} 
характер. Если бы моделью со\-сто\-яния был произвольный марковский 
процесс, то на первый план при реализации вышла бы задача фильт\-ра\-ции. 
Теоретическому исследованию это бы не помешало. Общие уравнения 
фильт\-ра\-ции поз\-во\-ли\-ли бы применить принцип разделения и~записать 
оптимальное автономное управ\-ле\-ние, выразив его через оптимальную оценку 
фильт\-ра\-ции. Это хорошо продемонстрировано в~ключевой 
в~рас\-смат\-ри\-ва\-емой об\-ласти исследований работе~[12]. Однако 
при\-бли\-зить\-ся к~практической реализации такого результата позволил только 
част\-ный случай марковской цепи, а~точ\-нее, оптимальная оценка фильт\-ра\-ции, 
описываемая фильтром Вонэма. Нелинейная, слож\-ная для компьютерной 
реализации оценка в~рас\-смот\-рен\-ной по\-ста\-нов\-ке обеспечила результат, по 
сути, укла\-ды\-ва\-ющий\-ся в~рамки традиционной LQG-за\-да\-чи, по крайней 
мере достаточными условиями существования оказались обычные условия 
ста\-би\-ли\-зи\-ру\-е\-мости и~де\-тек\-ти\-ру\-е\-мости самого известного в~теории 
управ\-ле\-ния уравнения Риккати.



{\small\frenchspacing
 { %\baselineskip=11.5pt
 %\addcontentsline{toc}{section}{References}
 \begin{thebibliography}{99}
\bibitem{1-bos}
\Au{Athans M.} The role and use of the stochastic linear-quadratic-Gaussian problem in control 
system design~// IEEE T. Automat. Contr., 1971. Vol.~16. No.\,6. P.~529--552.  doi: 
10.1109/TAC.1971.1099818.
\bibitem{2-bos}
\Au{Wonham W.\,M.} Linear multivariable control. A~geometric approach.~--- Lecture notes in 
economics and mathematical systems ser.~--- Berlin: Springer-Verlag, 1974. Vol.~101. 347~p.
\bibitem{3-bos}
\Au{Девис М.\,Х.\,А.} Линейное оценивание и~стохастическое управление~/ Пер. с~англ.~--- 
М.: Наука, 1984. 206~с. (\Au{Davis~M.\,H.\,A.} Linear estimation and stochastic control.~--- 
London: Chapman and Hall, 1977. 224~p.)
\bibitem{4-bos}
\Au{Kalman R.\,E., Bucy~R.\,S.} New results in linear filtering and prediction theory~// J.~Basic Eng.~--- 
T.~ASME, 1965. No.\,83. P.~95--108. doi: 10.1115/1.3658902.
\bibitem{5-bos}
\Au{Липцер Р.\,Ш., Ширяев~А.\,Н.} Статистика случайных процессов (нелинейная 
фильтрация и~смежные вопросы).~--- Серия <<Тео\-рия вероятностей и~математическая статистика>>.~--- М.: Наука, 1974. 696~с.
\bibitem{6-bos}
\Au{Wonham W.\,M.} Some applications of stochastic differential equations to optimal nonlinear 
filtering~// SIAM J. Control, 1965. No.\,2. P.~347--369. doi: 10.1137/0302028.
\bibitem{7-bos}
\Au{Ширяев А.\,Н.} Вероятность.~--- 2-е изд.~--- М.: Наука, 1989. 640~с.
\bibitem{8-bos}
      \Au{Босов А.\,В.} Управление линейным выходом автономной дифференциальной 
системы по квадратичному критерию на бесконечном горизонте~// Информатика и~её 
применения, 2024. Т.~18. Вып.~1. С.~18--25. doi: 10.14357/19922264240103. EDN: UEESFO.

\bibitem{9-bos}
\Au{Босов А.\, В.} Стабилизация и~слежение за траекторией линейной системы со 
скачкообразно изменяющимся дрейфом~// Автоматика и~телемеханика, 2022. №\, 4.  
С.~27--46. doi: 10.31857/S0005231022040031.
\bibitem{10-bos}
\Au{Elliott R.\,J., Aggoun~L., Moore~J.\,B.} Hidden Markov models: Estimation and control.~--- 
New York, NY, USA: Springer-Verlag, 1995. 382~p. doi: 10.1007/978-0-387-84854-9.

\bibitem{12-bos}
\Au{Босов А.\,В.} Задача управления линейным выходом нелинейной неуправляемой 
стохастической дифференциальной системы по квадратичному критерию~// Известия 
РАН. Теория и~системы управления, 2021. №\,5. C.~52--73. doi: 
10.31857/S000233882104003X. EDN: ESNCSX.

\bibitem{11-bos}
\Au{Rishel R.} A~strong separation principle for stochastic control systems driven by a hidden 
Markov model~// SIAM J. Control Optim., 1994. Vol.~32. P.~1008--1020. doi: 
10.1137/S0363012992232233.

\end{thebibliography}

 }
 }

\end{multicols}

\vspace*{-6pt}

\hfill{\small\textit{Поступила в~редакцию 19.03.24}}

\vspace*{6pt}

%\pagebreak

%\newpage

%\vspace*{-28pt}

\hrule

\vspace*{2pt}

\hrule


\def\tit{AUTONOMOUS LINEAR OUTPUT\\ OF~THE~MARKOV CHAIN STABILIZATION\\ 
BY~SQUARE CRITERION ON~AN~INFINITE HORIZON}


\def\titkol{Autonomous linear output of~the~Markov chain stabilization 
by~square criterion on~an~infinite horizon}


\def\aut{A.\,V.~Bosov}

\def\autkol{A.\,V.~Bosov}

\titel{\tit}{\aut}{\autkol}{\titkol}

\vspace*{-8pt}


\noindent
Federal Research Center ``Computer Science and Control'' of the Russian Academy of 
Sciences, 44-2~Vavilov Str., Moscow 119333, Russian Federation

\def\leftfootline{\small{\textbf{\thepage}
\hfill INFORMATIKA I EE PRIMENENIYA~--- INFORMATICS AND
APPLICATIONS\ \ \ 2024\ \ \ volume~18\ \ \ issue\ 3}
}%
 \def\rightfootline{\small{INFORMATIKA I EE PRIMENENIYA~---
INFORMATICS AND APPLICATIONS\ \ \ 2024\ \ \ volume~18\ \ \ issue\ 3
\hfill \textbf{\thepage}}}

\vspace*{4pt}
      
       
      
      \Abste{The solution of the linear output of the stochastic differential system optimal control 
problem on an infinite horizon is adapted for one particular case of indirect observation. The ergodic 
Markov chain plays the role of a~dynamic system and the autonomous linear output formed by it provides 
indirect noisy observations on\linebreak\vspace*{-12pt}}

\Abstend{the state of the chain. The control purpose is formulated as output 
stabilization in positions determined by the chain and periodically changing with chain state changes. 
The solution, as in a~similar problem with complete information, is obtained as the limit form of optimal 
control in the corresponding problem with a finite horizon. Sufficient conditions for the control existence 
turn out to be typical conditions for linear-quadratic problems of structure of the optimal Wonham filter, 
only linear components are present in the control and all nonlinearity is limited by the equation of the 
filtering estimate. Due to this, the existence conditions include only the requirements for the autonomous 
Riccati equation solution. A~numerical experiment for the mechanical drive model used in previous 
studies is discussed. The purpose of the experiment is to show the difference in the use of optimal control 
and its autonomous version.}
      
      \KWE{Markov chain with continuous time; Wonham filter; linear differential system; optimal 
control; stabilization; square criterion}
      
      


\DOI{10.14357/19922264240305}{XVTFLH}

\vspace*{-12pt}


     \Ack
     
     \vspace*{-3pt}
     
      \noindent
      The research was carried out using the infrastructure of the Shared Research Facilities ``High 
Performance Computing and Big Data'' (CKP ``Informatics'') of FRC CSC RAS (Moscow).


  \begin{multicols}{2}

\renewcommand{\bibname}{\protect\rmfamily References}
%\renewcommand{\bibname}{\large\protect\rm References}

{\small\frenchspacing
 {%\baselineskip=10.8pt
 \addcontentsline{toc}{section}{References}
 \begin{thebibliography}{99} 
      \bibitem{1-bos-1}
      \Aue{Athans, M.} 1971. The role and use of the stochastic linear-quadratic-Gaussian problem in 
control system design. \textit{IEEE T. Automat. Contr.} 16(6):529--552. doi: 
10.1109/TAC.1971.1099818.
      \bibitem{2-bos-1}
      \Aue{Wonham, W.\,M.} 1974. \textit{Linear multivariable control. A~geometric approach}. 
Lecture notes in economics and mathematical systems ser. Berlin: Springer-Verlag. 347~p.
      \bibitem{3-bos-1}
      \Aue{Davis, M.\,H.\,A.} 1977. \textit{Linear estimation and stochastic control}. London: 
Chapman and Hall. 224~p.
      \bibitem{4-bos-1}
      \Aue{Kalman, R.\,, and R.\,S.~Bucy.} 1965. New results in linear filtering and prediction theory. 
\textit{J. Basic Eng.~--- T.~ASME}  83(1):95--108. doi: 10.1115/1.3658902.
    \bibitem{5-bos-1}
    \Aue{Liptser, R.\,S., and A.\,N.~Shiryaev.} 1974. \textit{Statistika sluchaynykh protsessov (nelineynaya 
fil'tratsiya i~smezhnye voprosy)} [Statistics of random processes. Nonlinear filtering and related 
problems]. Seriya ``Teorya ve\-ro\-yat\-no\-stey i~matematicheskaya statistika'' [Probability theory and mathematical statistics ser.]. Moscow: Nauka. 696~p.
      \bibitem{6-bos-1}
      \Aue{Wonham, W.\,M.} 1965. Some applications of stochastic differential equations to optimal 
nonlinear filtering. \textit{SIAM J. Control}  2(3):347--369. doi: 10.1137/0302028.
      \bibitem{7-bos-1}
      \Aue{Shiryayev, A.\,N.} 1989. \textit{Veroyatnost'} [Probability]. 2nd ed. Moscow: Nauka. 
640~p.
      \bibitem{8-bos-1}
      \Aue{Bosov, A.\,V.} 2024. Upravlenie lineynym vykhodom avtonomnoy differentsial'noy 
sistemy po kvadratichnomu kriteriyu na beskonechnom gorizonte [Autonomous differential system linear 
output control by square criterion on an infinite horizon]. \textit{Informatika i~ee Primeneniya~--- 
Inform. Appl.} 18(1):18--25. doi: 10.14357/19922264240103. EDN: UEESFO.
      \bibitem{9-bos-1}
      \Aue{Bosov, A.\,V.} 2022. Stabilization and tracking of the trajectory of a linear system with 
jump drift. \textit{Automat. Rem. Contr.} 83(4):520--535. doi: 10.1134/S0005117922040026.
      \bibitem{10-bos-1}
      \Aue{Elliott, R.\,J., L.~Aggoun, and J.\,B.~Moore.} 1995. \textit{Hidden Markov models: 
Estimation and control}. New York, NY: Springer-Verlag. 382~p. doi: 10.1007/978-0-387-84854-9.

\bibitem{12-bos-1}
      \Aue{Bosov, A.\,V.} 2021. The problem of controlling the linear output of a nonlinear 
uncontrollable stochastic differential system by the square criterion. \textit{J.~Comput. Sys. Sc. Int.}  
60(5):719--739. doi: 10.1134/S1064230721040031. EDN: ESNCSX.
      \bibitem{11-bos-1}
      \Aue{Rishel, R.} 1994. A strong separation principle for stochastic control systems driven by 
a~hidden Markov model. \textit{SIAM J.~Control Optim.} 32(4):1008--1020. doi: 
10.1137/S0363012992232233.
      
\end{thebibliography}

 }
 }

\end{multicols}

\vspace*{-6pt}

\hfill{\small\textit{Received March 19, 2024}} 

\vspace*{-18pt}


\Contrl

\vspace*{-3pt}
      
      \noindent
      \textbf{Bosov Alexey V.} (b.\ 1969)~--- Doctor of Science in technology, principal scientist, 
Federal Research Center ``Computer Science and Control`` of the Russian Academy of Sciences,  
44-2~Vavilov Str., Moscow 119333, Russian Federation; \mbox{avbosov@ipiran.ru}

\label{end\stat}

\renewcommand{\bibname}{\protect\rm Литература} 

      