\def\stat{zorin}

\def\tit{К ОПРЕДЕЛЕНИЮ ПЕРИОДА ЗАНЯТОСТИ ПРИ~НЕЛОКАЛЬНОМ~ОПИСАНИИ ПОТОКОВ}

\def\titkol{К определению периода занятости при нелокальном описании потоков}

\def\aut{А.\,В.~Зорин$^1$}

\def\autkol{А.\,В.~Зорин}

\titel{\tit}{\aut}{\autkol}{\titkol}

\index{Зорин А.\,В.}
\index{Zorine A.\,V.}


%{\renewcommand{\thefootnote}{\fnsymbol{footnote}} \footnotetext[1]
%{Работа 
%выполнена при поддержке Программы развития МГУ, проект №\,23-Ш03-03. При анализе 
%данных использовалась инфраструктура Центра коллективного пользования 
%<<Высокопроизводительные вычисления и~большие данные>> 
%(ЦКП <<Информатика>>) ФИЦ ИУ РАН (г.~Москва)}}


\renewcommand{\thefootnote}{\arabic{footnote}}
\footnotetext[1]{Национальный исследовательский Нижегородский государственный 
университет им.~Н.~И.~Лобачевского,\\  \mbox{andrei.zorine@itmm.unn.ru}}

\vspace*{-12pt}




\Abst{При вероятностном моделировании и~анализе сложных управляющих систем
  массового обслуживания нескольких конфликтных входных потоков в~ряде работ
  успешно применялся подход, одна из черт которого~--- нелокальное
  описание различных составных блоков сис\-те\-мы. При таком описании часто теряется
  информация об истинных моментах прихода и~ухода требований в~сис\-те\-ме. Это
  приводит к~сложностям при определении периода занятости~--- одного из
  классических показателей качества функционирования.  В~работе предлагается
  определение периода занятости управляющей системы массового обслуживания на
  основе выделения тех моментов наблюдения над системой, в~которые очереди
  достигают нулевого значения. На примере циклического алгоритма обслуживания с~фиксированным ритмом переключения и~с~использованием мартингальной техники
  находятся эффективные вычислительные формулы для математического ожидания
  периода занятости отдельной очереди.}

\KW{управляющая система массового обслуживания; нелокальное описание блоков;
  неординарные потоки Пуассона; циклический алгоритм обслуживания; период
  занятости; многомерная счетная цепь Маркова; мартингал; обобщенная теорема
  Руше; интерполяционный полином Лагранжа}
  
  \DOI{10.14357/19922264240306}{YKSIBJ}
  
%\vspace*{-6pt}


\vskip 10pt plus 9pt minus 6pt

\thispagestyle{headings}

\begin{multicols}{2}

\label{st\stat}


\section{Введение}

В настоящее время в~теории массового обслуживания рассматривается большое число
показателей качества обслуживания, таких как распределение времени ожидания
требования, распределения размеров очередей, распределение интервала занятости,
вероятность потери требования в~системе с~потерями и~т.\,д.~\cite{Ivchenko:Kashtanov:Kovalenko}. В~то время как одни характеристики
можно без труда относить и~к моделям с~непрерывным временем, и~к моделям с~дискретным временем, корректное определение других характеристик может встречать
известные затруднения. 

Поясним сказанное на примере периода занятости. Период
за\-ня\-тости для некоторого прибора (если в~системе несколько обслуживающих
приборов) определяется как промежуток времени, в~начале которого прибор выходит
из режима ожидания и~начинает обслуживание требования, а окончание этого
промежутка совпадает с~первым из тех моментов окончания обслуживания, когда
ожидающих обслуживания на этом приборе требований больше нет и~прибор вынужден
перейти в~режим ожидания. Для систем в~дискретном времени~[2--4] вся временн$\acute{\mbox{а}}$я
ось разбивается на временн$\acute{\mbox{ы}}$е слоты фиксированного размера $h\hm>0$ и~истинный
момент поступления требования не наблюдается. Считается, что требование
поступает в~конце слота, а~его обслуживание может начаться только в~сле\-ду\-ющем
временн$\acute{\mbox{о}}$м слоте. Таким образом, период за\-ня\-тости в~модели с~дискретным временем
будет целым кратным величины~$h$ и~будет содержать <<ошибку округления>> порядка~$h$.  

В~монографиях~\cite{Alfa, Bruneel:Kim} период занятости исследован для
прос\-тей\-шей однолинейной сис\-те\-мы и~сис\-те\-мы с~конечным чис\-лом параллельных
приборов с~гео\-мет\-ри\-че\-ским входящим потоком; в~монографии~\cite{Pechinkin:Razumchik} 
период за\-ня\-тости изуча\-ет\-ся для однолинейной системы с~произвольным (це\-ло\-чис\-лен\-ным) временем обслуживания. Достаточное пред\-став\-ле\-ние об
истории и~современном со\-сто\-янии этого круга вопросов мож\-но со\-ста\-вить также по
работам~[5--8].


Для построения адекватных математических моделей сложных управляющих систем
обслуживания приходится применять так называемое нелокальное задание разных
составных блоков \mbox{системы}, таких как входные потоки, потоки насыщения и~некоторые
другие. Напомним, что потоки насыщения определяются~\cite{Fedotkin:1975} как
виртуальные выходные потоки системы при наличии достаточно %\linebreak 
большого запаса
требований в~очередях и~максимальном использовании возможностей обслуживающего
устройства; таким образом, они пред\-став\-ля\-ют альтернативу указания закона
\mbox{обслуживания} каждого отдельного требования. Понятие нелокального описания потока
было введено в~работах~\cite{Fedotkin:1978,Fedotkin:1981}. Приведем здесь
со\-от\-вет\-ст\-ву\-ющее \mbox{определение}: пусть $\{\tau^{\textrm{н}}_i;i=0,1,\ldots\}$~---
последовательность моментов наблюдения на оси~$Ot$, $(M, \mathcal M)$~---
измеримое пространство \textit{меток}, $\eta^{\textrm{н}}_i \hm\in \{0,1,\ldots\}$ и~$\nu_i\hm\in M$ суть число требований и~метка требований потока~$\Pi$, поступивших
на промежутке $(\tau^{\textrm{н}}_i, \tau^{\textrm{н}}_{i+1}]$. \textit{Неполное
  (нелокальное) описание} потока~$\Pi$ неоднородных требований есть векторная
случайная последовательность
%\begin{equation}
 % \label{AZ:eq:flowdescr-incomplete}
  $\left\{(\tau^{\textrm{н}}_i, \eta^{\textrm{н}}_i, \nu_i); i=0, 1, \ldots\right\}.$
%\end{equation}
Существенное отличие использования нелокального описания от упомянутых ранее
моделей в~дискретном времени заключается в~том, что длительности
$(\tau^{\mathrm{н}}_i\hm- \tau^{\mathrm{н}}_{i-1})$ промежутков наблюдения могут
быть случайными и~достаточно большими, а~их законы распределения могут
существенно меняться в~зависимости от состояния системы
обслуживания. В~частности, оказывается возможным применить понятие нелокального
описания для задания свойств процесса обслуживания требований с~помощью потока
насыщения вместо традиционного указания закона распределения длительности
обслуживания каждого отдельного требования.

Использование нелокального описания для входных потоков и~для потоков насыщения
лишает исследователя информации о точных моментах начала и~окончания периода
занятости в~классическом смысле. Поэтому в~данной работе предлагается связать
периоды занятости с~некоторыми из моментов $\tau^{\textrm{н}}_i$, $i\hm=0, 1,
\ldots$, когда одна или несколько очередей опустошаются. На этом пути удается
вычислить среднее время периода занятости, например для системы обслуживания
нескольких конфликтных потоков по алгоритму с~циклическим ритмом переключения и~фиксированными длительностями фаз.
Этот алгоритм часто применяется, например, в~задачах управления транспортными потоками на регулируемых перекрестках.

\vspace*{-6pt}

\section{Общая постановка задачи}

\vspace*{-3pt}

Рассмотрим следующую систему массового обслуживания.  Имеются $m\hm<\infty$ входных
потоков~$\Pi_j$, $j\hm=\overline{1, m}$. Предположим, что поток~$\Pi_j$~--- ста\-ционарный, неординарный, без последействия.\linebreak  Требования потока
$\Pi_j$ помещаются в~накопитель неограниченного объема.  Обслуживающее~устройство имеет $2m$~состояний: 
$\Gamma^{(1)}, \Gamma^{(2)}, \ldots,
\Gamma^{(2m)}$. В~состоянии $\Gamma^{(2j-1)}$ обслуживаются только требования
потоков~$\Pi_j$, $j\hm=\overline{1, m}$. В~состояниях вида $\Gamma^{(2j)}$
требования не обслуживаются. Состояния меняются по циклическому алгоритму:
$\cdots\to\Gamma^{(1)}\hm\to \Gamma^{(2)} \to \cdots \to \Gamma^{(2m)}\hm\to
\Gamma^{(1)} \to \cdots$. Длительность состояния~$\Gamma^{(r)}$ неслучайна и~равна $T_r\hm>0$.  Закон обслуживания потока задается свойствами потока насыщения:
максимально возможное число $\ell_j\hm\in\{0, 1, \ldots\}$ обслуженных требований
из очереди~$O_j$ за время~$T_{2j-1}$ в~состоянии~$\Gamma^{(2j-1)}$
обслуживающего устройства и~число~0 в~любом другом состоянии~$\Gamma^{(r)}$,
$r\hm\neq 2j-1$.

Будем наблюдать за системой в~моменты переключения сигнала светофора. Обозначим
эти моменты через~$\tau_i$, $i\hm=0, 1, \ldots$ (здесь и~далее для краткости не
будем писать верхнюю букву~<<н>> в~обозначении $\tau^{\mathrm{н}}_i$). Пусть
$\tau_0\hm=0$. Обозначим через~$\Gamma_0$ состояние светофора в~момент~$\tau_0$, а
через~$\Gamma_i$~--- состояние светофора на промежутке $(\tau_{i-1}, \tau_i]$.
Далее, пусть~$\kappa_{j,i}$~--- число требований в~очереди~$O_j$ в~момент~$\tau_i$, 
$\eta_{j,i}$~--- число требований потока~$\Pi_j$, поступивших за
промежуток $(\tau_i, \tau_{i+1}]$, $\xi_{j,i}$~--- число требований по потоку
насыщения на промежутке $(\tau_i, \tau_{i+1}]$. Введем векторы
$\eta_i\hm=(\eta_{1,i}, \eta_{2,i}, \ldots, \eta_{m,i})$
и~$\xi_i\hm=(\xi_{1,i}, \xi_{2,i}, \ldots, \xi_{m,i})$, $i\hm=0, 1, \ldots$

Динамика очередей описывается соотноше\-ни\-ями:

\vspace*{-6pt}

\noindent
\begin{multline*}
\kappa_{j,i+1}= \max\left\{0, \kappa_{j,i}+\eta_{j,i}-\xi_{j,i}\right\}, \\ 
j=\overline{1,m};\enskip i=0,1,\ldots
\end{multline*}

\vspace*{-3pt}

\noindent
Описание входных потоков и~потоков насыщения зададим с~помощью указания свойств
маркированного точечного процесса
$\left\{(\tau_i, \kappa_i, \nu_i); i=0, 1, \ldots\right\},
$
где $\nu_i=\Gamma_i$~--- метка требований, поступающих на промежутке
$(\tau_i, \tau_{i+1}]$. Пусть $r \hm\oplus 1$ означает $r+1$\linebreak для $r\hm<2m$ и~принимает
значение~1 для $r\hm=2m$. Пусть далее~$w_j$ и~$y_j$, $j\hm=\overline{1, m}$,~---
произ\-вольные целые неотрицательные числа. Тогда\linebreak при условии
$\nu_{i}\hm=\Gamma^{(r)}$ вероятность события
%\begin{equation}
 % \label{AZ:eq:eta-xi}
$\bigcap\nolimits_{j=1}^m \{ \eta_{i,j}=w_j, \xi_{i,j}=y_j\}
$ %\end{equation}
независимо от $\Gamma_0$, $\kappa_0$, $\kappa_1$, \ldots, $\kappa_{i}$ равна
$
\prod\nolimits_{j=1}^m \varphi_j(w_j; T_{r\oplus1}),
$
если $y_j=0$ при $r\oplus1\hm\neq 2j-1$ и~$y_j\hm=\ell_j$ при $r\hm \oplus1\hm=2j-1$, и~эта
вероятность равна нулю в~остальных случаях. Здесь распределения вероятностей
$\varphi_j(b;u)$, $b\hm=0, 1, \ldots$ ($u\hm>0$ рассматривается как параметр), могут
браться из разложения производящей функции для простого группового
(неординарного) потока~\cite{Ivchenko:Kashtanov:Kovalenko} следующим
образом. Введем производящие функции

\vspace*{-6pt}

\noindent
\begin{multline*}
\!f_j(z)=\sum\limits_{b=1}^\infty g_j(b) z^b;
\enskip
q_j(z;u)= \exp\{ \lambda_j u (
f_j(z) -1) \},\\  |z|\leqslant1,\quad j=\overline{1,  m}\,.
\end{multline*}
Здесь $\lambda_j>0$ задает интенсивность поступления групп машин по потоку~$\Pi_j$; $g_j(b)$ есть вероятность того, что группа по потоку~$\Pi_j$ состоит
из~$b$~автомашин. Пусть $f_j(z)\hm\neq1$ при $z\hm\neq1$, $|z|\hm=1$. Тогда
\begin{equation*}
\sum\limits_{b=0}^\infty z^b \varphi_j(b; u) = q_j(z;u), \enskip |z|\leqslant1\,.
%\label{AZ:eq:phi-def}
\end{equation*}

Введем обозначения:
$$
\bar \lambda_j =  \lambda_j f'_j(1)=\lambda_j \sum_{b=1}^\infty b g_j(b), \enskip
j=\overline{1,m}\,.
$$
В сделанных предположениях многомерная последовательность
$\{(\Gamma_i, \kappa_i);\ i\hm=0, 1, \ldots\}$ при заданном распределении случайного
элемента $(\Gamma_0, \kappa_0)$ будет однородной цепью Маркова, а необходимое и~достаточное условие существования стационарного распределения будет иметь
следующий вид:
\begin{equation}
  \max\left\{ \fr{{\bar\lambda_1}T}{\ell_1}\,,\;
  \frac{{\bar\lambda_2}T}{\ell_2}\,,\; \ldots,\;
  \frac{{\bar\lambda_m}T}{\ell_m}\right\}<1\,,
  \label{AZ:eq:stat}
\end{equation}
где $ T=T_1+T_2+\cdots+T_{2m}$. Содержательно условие~\eqref{AZ:eq:stat}
означает, что по каждому потоку среднее\linebreak чис\-ло поступающих требований за полный
цикл меньше соответствующего потока насыщения. Всюду далее будем предполагать
условие стационарности~\eqref{AZ:eq:stat} выполненным. Кроме того, каждая из
случайных последовательностей $\{(\Gamma_i, \kappa_{j,i});\ i\hm= 0,1, \ldots\}$, $j\hm=\overline{1,  m}$, 
также представляет собой однородную цепь Маркова, при выполнении
условия~\eqref{AZ:eq:stat}~--- положительно возвратную.

\vspace*{-6pt}

\section{Период занятости}

\vspace*{-3pt}

Перейдем теперь к~определению понятия периода занятости. Введем поток
сиг\-ма-алгебр
$$
\mathfrak{F}_{i}\hm= \sigma\{\Gamma_{0}, \kappa_{1,0}, \eta_{1,0}, \eta_{1,1},
\ldots, \eta_{1,i-1}\}.
$$
 Пусть~$\mathfrak s$~--- момент остановки (относительно
потока $\{\mathfrak F_i; i\hm=0, 1, \ldots\}$) такой, что
$\kappa_{j,\mathfrak s}\hm=0$. Введем случайную величину~$\nu(\mathfrak s)$
следующим образом:

\vspace*{-6pt}

\noindent
\begin{multline*}
\nu(\mathfrak s)={}\\
{}=
\begin{cases}
  \infty,& \!\hspace*{-60pt} \mbox{если\ }\kappa_{j,i}>0\ \mbox{для всех}\  i\geqslant \mathfrak s+1;\\
  \min\{i\geqslant 1\colon \kappa_{j,i+\mathfrak s}=0\}& \!\mbox{в\ противном\ случае}.
\end{cases}
\end{multline*}
Анализ классического понятия периода занятости приводит к~двум возможным
определениями. Согласно первому определению назовем \textit{периодом занятости}
для очереди~$O_j$ промежуток
$(\tau_{\mathfrak s}, \tau_{\mathfrak s+\nu(\mathfrak s)}]$ при условии
$\kappa_{j,\mathfrak s}\hm=0$. Здесь не исключен случай, когда не поступает ни
одного требования за следующий такт работы обслуживающего устройства и~никакого
обслуживания фактически не происходит.  Поэтому можно рассмотреть период
занятости, начинающийся с~$x\hm=1, 2, \ldots$ требований в~очереди~$O_j$,
т.\,е.\ распределение случайного промежутка
$(\tau_{\mathfrak s}, \tau_{\mathfrak s+\nu(\mathfrak s)}]$ можно рассматривать
при условии $\kappa_{j,\mathfrak s}\hm=x$, $x\hm>0$.
На событии
$\{\omega\colon \Gamma_{\mathfrak s}\hm=\Gamma^{(r)}\}$ длину этого промежутка
можно выразить формулой
$$
T_{r\oplus1}+T_{r\oplus2}+\cdots+T_{r\oplus \nu(\mathfrak s)}.
$$
Если при заданных свойствах входных потоков, потоков насыщения и~алгоритма
управления все периоды занятости для данной очереди статистически однородны,
можно, не уменьшая общ\-ности, \mbox{изучать} условное распределение величины~$\nu(\mathfrak s)$ или величины~$\tau_{\nu(\mathfrak s)}$, положив
$\mathfrak{s}\hm=0$ и~рассматривая условные распределения при условии
$\kappa_{j,0}\hm=0$. Цель данного исследования~--- вычисление условных \mbox{средних}

\vspace*{-6pt}

\noindent
\begin{multline*}
\mathbb E\left(T_{r\oplus1}+T_{r\oplus2}+\cdots+T_{r\oplus \nu(\mathfrak s)} \Big\vert \right.\\
\left.\left\{\kappa_{j,0}=x, \Gamma_0=\Gamma^{(r)}\right\}\right).
\end{multline*}

\vspace*{-3pt}

Применяемый ниже мартингальный метод восходит в~работам~\cite{Bacelli-1,Bacelli-2}.

Имеют место следующие утверждения. Не уменьшая общности, будем везде далее
полагать $j\hm=1$. Ниже~$I(\cdot)$ означает индикатор события, указанного в~скобках.


\smallskip

\noindent
\textbf{Теорема~1.}
\textit{ При $|z|<1$ последовательность}
  \begin{equation*}
    M_{i}(z)=
    \begin{cases}
      z^{\kappa_{1,0}}, & \hspace*{-33pt}i=0;\\
      z^{\kappa_{1,i}+ \sum\nolimits_{k=0}^{i-1}
        \min\{\kappa_{1,k}+\eta_{1,k},\;\xi_{1,k}\}}\times{}&\\
        {}\times 
     \displaystyle  \prod\limits_{k=0}^{i-1} \displaystyle \sum\limits_{r=1}^{2m}
      \fr{I(\Gamma_{k}=\Gamma^{(r)})}
      {\exp\{\lambda_{1}T_{r\oplus1}(f_{1}(z)-1)\}},&\\
      & \hspace*{-33pt}i=1,2,\ldots,
    \end{cases}
  \end{equation*}
  \textit{представляет собой мартингал относительно потока}
  $\{\mathfrak{F}_{i};i=0,1,\ldots\}$.

\smallskip

\noindent
Д\,о\,к\,а\,з\,а\,т\,е\,л\,ь\,с\,т\,в\,о\ \ состоит в~проверке определения мартингала и~здесь не приводится.

\smallskip

Пусть, по определению, $\mathfrak t(\mathfrak s)= \mathfrak s + \nu(\mathfrak{s})$.

\smallskip

\noindent
\textbf{Теорема~2.}
\textit{Пусть $\mathfrak s$~---  момент остановки относительно потока
  $\{\mathfrak{F}_{i};\ i\hm=0,1,\ldots\}$. Тогда}
  
  \vspace*{-6pt}
  
  \noindent
  \begin{multline}
    \sum\limits_{r=1}^{2m}\mathbb{E}\Bigl(
    I\left(\Gamma_{\mathfrak   s}=\Gamma^{(r)}\right)\times{}\\
    {}\times 
    z^{\xi_{1,\mathfrak s}+\xi_{1,\mathfrak  s+1}+\cdots +\xi_{1,\mathfrak {t}(\mathfrak s)-2}+
      \kappa_{1,\mathfrak t(\mathfrak s)-1}+\eta_{1,\mathfrak t(\mathfrak s)-1}}
    \times{}
    \\ 
    {}\times
    \exp\{\lambda_{1}(T_{r\oplus1}+T_{r\oplus2}+\cdots+
    T_{r\oplus\nu(\mathfrak s)}) (1-f_{1}(z))\}\Big|
    \mathfrak{F}_{\mathfrak s}\!\Bigr)\!={}\\
    {}=
    z^{\kappa_{1,\mathfrak  s}}.
    \label{AZ:eq:main-mart}
  \end{multline}


\noindent
Д\,о\,к\,а\,з\,а\,т\,е\,л\,ь\,с\,т\,в\,о\,.\ \
  Поскольку для момента остановки $\mathfrak s$ случайная величина
  $\mathfrak t(\mathfrak s)$ конечна с~вероятностью единица (как следствие
  положительной возвратности), то из теоремы Дуба об опциональной остановке
  должно следовать равенство
  $M_{\mathfrak s}(z) \hm= \mathbb E(M_{\mathfrak t(\mathfrak s)} \mid \mathfrak
  F_{\mathfrak s})$. Поскольку
  
  \noindent
\begin{multline*}
  \mathbb {E}\left(M_{\mathfrak t(\mathfrak s)} \mid \mathfrak F_{\mathfrak{s}}\right) =
  z^{\sum\nolimits_{k=0}^{\mathfrak{s}-
1}\min\{\kappa_{1,k}+\eta_{1,k},\;\xi_{1,k}\}}\times{}\\
{}\times
  \prod\limits_{k=0}^{\mathfrak s-1}\sum\limits_{r=1}^{2m}
  \fr{I(\Gamma_{k}=\Gamma^{(r)})}{\exp\{\lambda_{1}T_{r+1}(f_{1}(z)-
1)\}}\times{}
\end{multline*}

\noindent
\begin{multline*}
{}  \times\mathbb{E}\biggl(z^{\sum\nolimits_{k=\mathfrak s}^{\mathfrak t(\mathfrak
      s)-1}\min\{\kappa_{1,k}+\eta_{1,k},\;\xi_{1,k}\}}\times{}\\
      {}\times \prod\limits_{k=\mathfrak
    s}^{\mathfrak t(\mathfrak s)-1}\sum\limits_{r=1}^{2m}
  \fr{I(\Gamma_{k}=\Gamma^{(r)})}%
  {\exp\{\lambda_{1}T_{r\oplus1}(f_{1}(z)-1)\}}\,\bigg|\,\mathfrak{F}_{\mathfrak 
s}\biggr),
\end{multline*}
то
\begin{multline*}
  z^{\kappa_{1,\mathfrak s}}=
  \mathbb{E}\biggl(z^{\sum\nolimits_{k=\mathfrak s}^{\mathfrak t(\mathfrak s)-1}
    \min\{\kappa_{1,k}+\eta_{1,k},\;\xi_{k}\}}\times{}\\
    {}\times \prod\limits_{k=\mathfrak s}^{\mathfrak
    t(\mathfrak s)-1}
  \sum\limits_{r=1}^{2m}\fr{I(\Gamma_{k}=\Gamma^{(r)})}%
  {\exp\{\lambda_{1}T_{r\oplus1}(f_{1}(z)-1)\}}\bigg|\mathfrak{F}_{\mathfrak 
s}\bigg).
\end{multline*}
Во-первых, равенство $\mathfrak t(\mathfrak s)\hm=i$ влечет $\kappa_{1,i}\hm=0$,
откуда $\kappa_{1,i-1}\hm+\eta_{1,i-1}\hm\leqslant\xi_{1,i}$. Для всех~$k$,
$\mathfrak {s} \hm\leqslant k\hm<i$, имеем
$\min\{\kappa_{1,k}\hm+\eta_{1,k},\xi_{1,k}\}=\xi_{1,k}$. Значит,
\begin{multline*}
  \sum\limits_{k=\mathfrak s}^{\mathfrak t(\mathfrak s)-1}
  \min\{\kappa_{1,k}+\eta_{1,k},\;\xi_{k}\}={}\\
  {}= \xi_{1,\mathfrak
    s}+\xi_{1,\mathfrak s+1}+\cdots+\xi_{1,\mathfrak t(\mathfrak s)-2}+
  \kappa_{1,\mathfrak t(\mathfrak s)-1}+\eta_{1,\mathfrak t(\mathfrak s)-1}.
\end{multline*}
Чтобы окончательно установить равенство~\eqref{AZ:eq:main-mart}, остается
заметить, что в~силу циклического обслуживания значение~$\Gamma_{\sigma}$
однозначно определяет все последующие состояния среды, поэтому можно избавиться
от внутренних сумм по~$r$ в~пользу одной внешней суммы.


\smallskip

Чтобы получить практические выводы о периоде занятости, надо в~равенстве~\eqref{AZ:eq:main-mart} положить 
$\mathfrak s\hm=0$ и~вычислить условные ожидания
при условиях $\{\kappa_{1,0}\hm=x, \Gamma_0\hm=\Gamma^{(r)}\}$, $x\hm=0, 1, \ldots$ и~$r\hm=\overline{1, 2m}$. 
Результаты содержатся в~следующих ниже теоремах.

\smallskip

\noindent
\textbf{Лемма~1.}\
%  \label{AZ:lem:1}
 \textit{Пусть $u>0$ и~$\bar\lambda_1 u\hm<\ell_1$. Тогда уравнение}
  \begin{equation}
    z^{\ell_1}=\exp\left\{ \lambda_1 u  \left(f_1(z)-1\right)\right\}
    \label{AZ:eq:maineq}
  \end{equation}
 \textit{имеет $\ell_1$ решений в~круге $|z|\hm\leqslant1$, причем ровно одно из них 
$z\hm=1$}.

\noindent
Д\,о\,к\,а\,з\,а\,т\,е\,л\,ь\,с\,т\,в\,о\,.\ \
  Для доказательства воспользуемся обобщением теоремы Руше из работы~\cite{Klimenok}. Представим уравнение~\eqref{AZ:eq:maineq} в~виде
  $$
  \theta_1(z)\hm+\theta_2(z)\hm=0\,,
  $$ 
  где 
  $$
  \theta_1(z)=z^{\ell_1}; \enskip
\theta_2(z)\hm=-\exp\{\lambda_1
  u(f_1(z)-1)\}.
  $$
   Необходимо проверить, что 
   \begin{enumerate}[(1)]
   \item $|\theta_1(z)|\hm>|\theta_2(z)|$ для
  всех $z\hm\neq 1$ таких, что $|z|\hm=1$; 
  \item $\theta_1(1)\hm=-\theta_2(1)$; 
  \item $(\theta_1'(1)+\theta_2'(1))/\theta_1(1)\hm>0$.
  \end{enumerate}
  
   Пункт~1 следует из неравенств
\begin{multline*}
  |\theta_2(z)| = e^{-\lambda_1 u} \left\vert e^{\lambda_1 u f_1(z)} \right\vert <
  e^{-\lambda_1 u}  e^{\lambda_1 u |f_1(z)|}  \leqslant{}\\
  {}\leqslant 1=|\theta_1(z)|.
  \end{multline*}
  Пункт~2 тривиален. Для доказательства пункта~3 заметим, что
  $$
  \hspace{15mm}\fr{\theta_1'(1)+\theta_2'(1)}{\theta_1(1)}=\ell_1-\bar \lambda_1 u>0\,.\hspace{15mm}\square
  $$


Для дальнейшего понадобится обозначение
\begin{multline*}
\mathcal T(r)={}\\
{}=
\begin{cases}
  T_{r+1}+T_{r+2}+\cdots+T_{2m}, &\!\! r=\overline{1, 2m-1};\\
    0, & \! \!r=2m.
\end{cases}
\end{multline*}

\noindent
\textbf{Теорема~3.}\
\textit{Пусть $x\hm=1, 2, \ldots$ Пусть, далее, $\beta_1, \beta_2, \ldots,
\beta_{\ell_1-1}$~--- нули уравнения $z^{\ell_1}\hm=q(z;T)$, лежащие в~круге
  $|z|\hm<1$ и~$\beta_{\ell_1}\hm=1$.  Введем многочлен}
  
  \vspace*{-6pt}
  
  \noindent
  \begin{multline*}
    L_{x,r}(z)=\fr{(z-\beta_{1})(z-\beta_2)\cdots(z-\beta_{\ell_1-1})}
         {(1-\beta_{1})(1-\beta_2)\cdots(1-\beta_{\ell_1-1})}+{}\\
{}    +\sum\limits_{j=1}^{\ell_1-1}(\beta_{j})^{x-1}
    \exp\{\lambda_{1}(T_{1}+\mathcal{T}(r))(f_{1}(\beta_{j})-1)\}
    \fr{z-1}{\beta_{j}-1}\times{}\\
    {}\times \prod\limits_{k\neq j}\fr{z-\beta_{k}}{\beta_{j}-\beta_{k}}\,.
\end{multline*}

\vspace*{-7pt}

\noindent
\textit{Тогда}

\vspace*{-6pt}

\noindent
\begin{multline*}
    L'_{x,r}(1) =\sum\limits_{j=1}^{\ell_1-1} \fr{1}{1-\beta_{j}} +
    \sum\limits_{j=1}^{\ell_1-1}(\beta_{j})^{x-1}\times{}\\
    {}\times
    \exp\{\lambda_{1}(T_{1}+\mathcal T(r))(f_{1}(\beta_{j})-1)\} 
    \prod\limits_{k\neq j}\fr{1-\beta_{k}}{\beta_{j}-\beta_{k}}\,;
   % \label{AZ:eq:E-kappa-eta-1}
    \end{multline*}
    
\vspace*{-12pt}

    \noindent
    \begin{multline}
  \mathbb E\left( \kappa_{1,\nu(0)-1}+\eta_{1,\nu(0)-1} \Big\vert
  \left\{\kappa_{1,0}=x,\Gamma_{0}=\Gamma^{(r)}\right\}\right)={}\\
  {}=1+L'_{x,r}(1);
   \label{AZ:eq:E-nu0-1}
   \end{multline}
   
  \vspace*{-12pt}

    \noindent
    \begin{multline*}
  \mathbb{E}\left(\left[\fr{\nu(0)-
1}{2m}\right]\bigg|\left\{\kappa_{1,0}=x,\Gamma_{0}=\Gamma^{(r)}\right\}\right)={}\\
{}=
  \fr{x+\bar\lambda_1(T_1+\mathcal T(r))-1-L'_{x,r}(1)}{\ell_1-\bar\lambda_1 T}\,;
\end{multline*}
\textit{средний период занятости очереди~$O_1$ при условии
$\{\kappa_{1,0}\hm=x,\ \Gamma_0\hm =\Gamma^{(r)}\}$ равен}

\vspace*{-6pt}

\noindent
  \begin{multline}
    \mathbb E\left( T_{r\oplus1}+\cdots+T_{r\oplus \nu(0)}\big\vert
    \kappa_{1,0}=x,  \Gamma_0=\Gamma^{(r)}\right)={}    \\ 
    {}=
    T_{1}+\mathcal T(r)+{}\\
    {}+
    T     \fr{x+\bar\lambda_1(T_1+\mathcal T(r)-1-L'_{x,r}(1))}
    {\ell_1-\bar{\lambda}_1 T}\,.\!\!
    \label{AZ:eq:busy-x-r-1}
  \end{multline}

\noindent
Д\,о\,к\,а\,з\,а\,т\,е\,л\,ь\,с\,т\,в\,о\,.\ \
  Полагая в~равенстве~\eqref{AZ:eq:main-mart} $\mathfrak {s}\hm=0$, % $x>0$ и~$r\neq 2m$
  получим:
  
  \pagebreak
  
  \noindent
  \begin{multline}
    \mathbb{E}\biggl(
    z^{\xi_{1,0}+\cdots+\xi_{1,\nu(0)-1}+%
      \kappa_{1,\nu(0)-1}+\eta_{1,\nu(0)-1}-\xi_{1,\nu(0)-1}}\times
    \\ \times
    \exp\{\lambda_{1}(T_{r\oplus1}+T_{r\oplus2}+\cdots{}\\
    {}\cdots +T_{r\oplus\nu(0)})
    (1-f_{1}(z))\}\Big|\{\kappa_{1,0}=x,\Gamma_{0}=\Gamma^{(r)}\}\biggr)={}\\
    {}=z^x.
    \label{AZ:eq:e4}
\end{multline}
На событии $\{\omega\colon \kappa_{1,0}=x\}$ при
$x\hm>0$ с~вероятностью единица выполняется равенство
\begin{equation*}
  \xi_{1,0}+\cdots+\xi_{1,\nu(0)-1}=
  \ell_1\left(\left[\fr{\nu(0)-1}{2m}\right]+1\right),
 % \label{AZ:eq:xi4}
\end{equation*}
где квадратные скобки~$[\cdot]$ обозначают целую часть числа.  Кроме того, на
том же событии при $x\hm>0$ с~вероятностью единица
\begin{equation*}
T_{r\oplus1}+T_{r\oplus2}+\cdots+T_{r\oplus\nu(0)}=
T_1+ \mathcal T(r) +T\left[\fr{\nu(0)-1}{2m}\right]\!.
%\label{AZ:eq:T-5}
\end{equation*}
В силу определения величины $\nu(0)$, $\xi_{1,\nu(0)-1}\hm=\ell_1$. Поэтому
равенство~\eqref{AZ:eq:e4} примет вид:
\begin{multline}
  z^{x}\exp\left\{\lambda_{1}(T_{1}+\mathcal{T}(r))(f_{1}(z)-1)\right\}={}  \\
  {} =
  \mathbb{E}\Bigl( z^{\kappa_{1,\nu(0)-1}+\eta_{1,\nu(0)-1}}\times{}\\
  {}\times 
  \left( z^{\ell_1}\exp\{ \lambda_{1}T(1-f_{1}(z))\}
  \right)^{\left[ (\nu(0)-1)/(2m)\right]}\,\Big|\\
  \left\{\kappa_{1,0}=x,\Gamma_{0}=\Gamma^{(r)}\right\}\Bigr).
  \label{AZ:eq:cyc1}
\end{multline}
По лемме~1 уравнение $z^{\ell_1}\hm=q(z;T)$ имеет ровно~$\ell_1$
заявленных в~формулировке теоремы корней $\beta_1, \beta_2, \ldots,
\beta_{\ell_1}$  в~круге $|z|\hm\leqslant1$. Перенумеруем корни так, чтобы
$\beta_{\ell_1}\hm=1$. Из равенства~\eqref{AZ:eq:cyc1} подстановкой $z\hm=\beta_j$
находим:
\begin{multline}
  \label{AZ:eq:bet}
  (\beta_{j})^{x}\exp\{\lambda_{1}(T_{r\oplus1}+\cdots+T_{2m}+T_{1}) %
  (f_{1}(\beta_{j})-1)\}={}  \\ 
  {}=
  \mathbb{E}\left((\beta_{j})^{\kappa_{1,\nu(0)-1}+\eta_{1,\nu(0)-1}}\Big|
 \left \{\kappa_{1,0}=x,\Gamma_{0}=\Gamma^{(r)}\right\}\right),\\
  j=\overline{1,\ell_1-1}\,.
\end{multline}
Введем условные вероятности

\vspace*{-4pt}

\noindent
\begin{multline*}
\alpha_{k}={}\\
{}+\mathbb{P}\left(X_{\nu(0)-1}\hm+\eta_{\nu(0)-1}\hm=k\big\vert 
X_{0}\hm=x,\Gamma_{0}\hm=\Gamma^{(r)}\right),\\
 k=\overline{0, \ell_1}
\end{multline*}

\vspace*{-4pt}

\noindent
(зависимость от~$x$, $r$ подразумевается). Заметим, что
$\kappa_{1,\nu(0)-1}\hm\geqslant1$. Тогда математическое ожидание в~правой части
равенства~\eqref{AZ:eq:bet}

\vspace*{-4pt}

\noindent
\begin{multline*}
\!\mathbb{E}\left((\beta_{j})^{\kappa_{1,\nu(0)-1}+\eta_{1,\nu(0)-1}}
\Big\vert  \left\{\kappa_{1,0}=x,\Gamma_{0}=\Gamma^{(r)}\right\}\right)={}\\
{}=
\alpha_{1}\beta_{j}+\alpha_{2}(\beta_{j})^{2}+\cdots
+\alpha_{\ell}(\beta_{j})^{\ell_1}.
\end{multline*}

\vspace*{-4pt}

\noindent
Для определения неизвестных величин~$\alpha_1, \alpha_2, \ldots\linebreak
\ldots,
\alpha_{\ell_1}$ получаем неоднородные линейные уравнения (с~присоединенным
условием нормировки):

\columnbreak

\noindent
\begin{multline*}
  \alpha_{1}+\alpha_{2}\beta_{j}+\cdots+\alpha_{\ell_1}\beta_{j}^{\ell_1-1}={}\\
  {}=
  (\beta_{j})^{x-1}\exp\{\lambda_{1}(T_{1}+
  \mathcal T(r))(f_{1}(\beta_{j})-1)\},\\
   j=\overline{1,\ell_1-1}\,;
   \end{multline*}
   
   \vspace*{-9pt}
   
   \noindent
   $$
\alpha_{1}+\alpha_{2}+\cdots+\alpha_{\ell}=1.
$$
Поскольку неизвестные выступают как коэффициенты многочлена степени
$(\ell_1-1)$, принимающего в~заданных точках заданные значения, естественно
ввести интерполяционный многочлен
$$
L_{x,r}(z)=\alpha_1\hm+\alpha_2 z+\cdots+ \alpha_{\ell_1} z^{\ell_1-1},
$$
 который 
с~необходимостью будет иметь вид, указанный в~формулировке теоремы. Но тогда

\vspace*{-6pt}

\noindent
\begin{multline*}
   \! \mathbb{E} \left( \kappa_{1,\nu(0)-1}+\eta_{1,\nu(0)-1} \Big\vert
    \left\{\kappa_{1,0}=x,\Gamma_{0}=\Gamma^{(r)}\right\}\right)={}    \\ 
    {}=
    \alpha_1+2\alpha_2+\cdots+\ell_1 \alpha_{\ell_1} =
    \fr{d}{dz}\left(z L_{x,r}(z)\right)\Big|_{z=1} ={}\\
    {}= 1+L'_{x,r}(1).
\end{multline*}

\vspace*{-4pt}


Дифференцируя равенство~(\ref{AZ:eq:cyc1}) по~$z$ в~точке $z\hm=1$, получим:

\vspace*{-6pt}

\noindent
\begin{multline*}
  x+\bar\lambda_{1}(T_{1}+\mathcal T(r))={}\\
  {}=
  \mathbb E\left( \kappa_{1,\nu(0)-1}+\eta_{1,\nu(0)-1} \Big\vert 
 \left\{ \kappa_{1,0}=x,\Gamma_{0}=\Gamma^{(r)}\right\}\right)+{}\hspace*{-0.5pt}  \\ 
  {} +
  (\ell_1-\bar\lambda_{1}T)  \mathbb{E}\biggl(\biggl[\fr{\nu(0)-1}{2m}\biggr]
  \bigg| \left\{\kappa_{1,0}=x,\Gamma_{0}=\Gamma^{(r)}\right\}\biggr).\hspace*{-0.5pt}
\end{multline*}

\vspace*{-4pt}

\noindent
Отсюда приходим к~равенству~\eqref{AZ:eq:E-nu0-1}.  Переходя теперь к~абсолютному времени, можно записать длительность периода занятости как
$$
  T_{1}+\mathcal T(r)+
  T\mathbb{E}\biggl(\biggl[\fr{\nu(0)-1}{2m}\biggr]
  \bigg| \left\{\kappa_{1,0}=x,\Gamma_{0}=\Gamma^{(r)}\right\}\biggr).
$$
Отсюда получается соотношение~\eqref{AZ:eq:busy-x-r-1}.


%\pagebreak


Следующие две теоремы доказываются аналогично.

\noindent
\textbf{Теорема~4.}\
\textit{Пусть при $r\hm<2m$ определен полином}
    \begin{multline*}
      L_{0,r}(z)=(1-e^{-\lambda_1 T_{r+1}})\times{}\\
      {}\times
      \fr{(z-\beta_{1})(z-\beta_2)\cdots(z-\beta_{\ell_1-1})}%
    {(1-\beta_{1})(1-\beta_2)\cdots(1-\beta_{\ell_1-1})}+\\
    +\sum\limits_{j=1}^{\ell_1-1}(\beta_{j})^{-1}
    \left(1-e^{-\lambda_1 T_{r+1}f_1(\beta_j)}\right)\times{}\\
    {}\times
    \exp\{\lambda_{1}(T_{1}+\mathcal T(r))(f_{1}(\beta_{j})-1)\}
    \fr{z-1}{\beta_{j}-1}\prod\limits_{k\neq j}\fr{z-\beta_{k}}{\beta_{j}-\beta_{k}}\,.\hspace*{-2.84pt}
\end{multline*}
\textit{Тогда имеют место соотношения}

\noindent
  \begin{multline*}
    \mathbb E  \left(  I(\{\nu(0) >1\})  \left[ \fr{\nu(0)-1}{2m} \right] 
\Bigg\vert \right.\\
\left. \kappa_{1,0}=0, \Gamma_0=\Gamma^{(r)} 
\vphantom{\fr{\nu(0)-1}{2m}}
\right) ={}\\
{}=
  \left(\ell_1-\bar\lambda_1 T\right)^{-1}  \Big( \bar\lambda_1 T_{r+1} e^{-\lambda_1
      T_{r+1}} +{}\\
      {}+ \bar\lambda_1 \left(T_1+\mathcal T(r)\right)\left(1-e^{-\lambda_1 T_{r+1}}\right)
    -{}\\
 {}-\left(1-e^{-\lambda_1 T_{r+1}}\right)-L_{0,r}'(1)    \Big);
  %\label{AZ:eq:x0r2m-1}
  \end{multline*}
  
  \vspace*{-12pt}
  
  \noindent
  \begin{multline*}
    \mathbb {E}\left(  T_{r\oplus1} +\cdots+T_{r\oplus \nu(0)}\mid \kappa_{1,0}=0,
    \Gamma_0=\Gamma^{(r)}\right)={}\\
    {}= T_{r+1} e^{-\lambda_1 T_{r+1}} +
    \left(T_1+\mathcal T(r)\right)\left(1-e^{-\lambda_1 T_{r+1}}\right) + {}\\
    {}+\mathbb{E}\! \left(
    I(\{\nu(0)>1\}) \left[ \fr{\nu(0)-1}{2m} \right] \Big| \kappa_{1,0}=0,
    \Gamma_0=\Gamma^{(r)} \!\right).\hspace*{-6pt}
  %  \label{AZ:eq:x0r2m-2}
\end{multline*}


\noindent
\textbf{Теорема~5.}\
\textit{Пусть}

\vspace*{-6pt}

\noindent
  \begin{multline*}
    L_{0,2m}(z)=
    \Bigl(1-{}\\
    {}-\sum\limits_{b=0}^{\ell_1} \varphi_1(b; T_1)\Bigr)
      \fr{(z-\beta_{1})(z-\beta_2)\cdots(z-\beta_{\ell_1-1})}%
    {(1-\beta_{1})(1-\beta_2)\cdots (1-\beta_{\ell_1-1})}+{}\\
{} +\sum\limits_{j=1}^{\ell_1-1}(\beta_{j})^{-1}
    \left(
    \vphantom{\sum\limits_{b=0}^{\ell_1}} 
    \exp\{\lambda_{1} T_1(f_{1}(\beta_{j})-1)\}-{}\right.\\
\left.    {}-\sum\limits_{b=0}^{\ell_1} 
\varphi_1(b; T_1) (\beta_j)^b
    \right)
    \fr{z-1}{\beta_{j}-1}\prod\limits_{k\neq j}\fr{z-\beta_{k}}{\beta_{j}-
\beta_{k}}\,.
  \end{multline*}
  
  \vspace*{-4pt}
  
  \noindent
  \textit{Имеют место равенства}
  
\vspace*{-6pt}

\noindent
  \begin{multline*}
        \mathbb{E}  \left(  I(\{\nu(0) >1\})  \left[ \fr{\nu(0)-1}{2m} \right] 
\Big| \right.\\
\kappa_{1,0}=0,
\left. \Gamma_0=\Gamma^{(2m)} 
\vphantom{\left[ \fr{\nu(0)-1}{2m} \right]}
 \right ) =
        \left(\ell_1-\bar\lambda_1 T\right)^{-1} \times{}        \\ 
        {} \times
        \left(
        \bar\lambda_1 T_1 - \sum\limits_{b=0}^{\ell_1} b \varphi_1(b; T_1)
        -L_{0,2m}(1)-L_{0,2m}'(1)    \right);
 % \label{AZ:eq:x0r2m-3}
  \end{multline*}
  
  \vspace*{-12pt}
  
  \noindent
  \begin{multline*}
    \mathbb {E}\left(  T_{r\oplus1} +\cdots+T_{r\oplus \nu(0)}\mid \kappa_{1,0}=0,
    \Gamma_0=\Gamma^{(2m)}\right)={}    \\ 
    {} =
    T_{1}+ T 
    \mathbb {E} \left(
    I(\{\nu(0)>1\}) \left[ \fr{\nu(0)-1}{2m} \right] \Big|\right.\\
 \left.     \kappa_{1,0}=0,    \Gamma_0=\Gamma^{(2m)}
 \vphantom{\left[ \fr{\nu(0)-1}{2m} \right]}
  \right) .
  %\label{AZ:eq:x0r2m-4}
  \end{multline*}

\vspace*{-24pt}



\section{Заключение}

\vspace*{-3pt}

Для управляющих систем массового обслуживания с~несколькими конфликтными
потоками интерпретация периода занятости как суммарного времени функционирования
до опустошения очередей оказывается продуктивным способом переноса понятия на
задачи с~дискретным временем. При этом использование мартингальной техники
позволяет изучить длительность периода занятости очереди при циклическом
обслуживании с~фиксированным ритмом переключения. По-ви\-ди\-мо\-му, 
данный подход в~дальнейшем необходимо опробовать и~на более сложных входных потоках и~алгоритмах
обслуживания.

\vspace*{-8pt}

{\small\frenchspacing
 {\baselineskip=10.8pt
 %\addcontentsline{toc}{section}{References}
 
 \vspace*{-6pt}
 
 \begin{thebibliography}{99}   
  \bibitem{Ivchenko:Kashtanov:Kovalenko}% %1
    \Au{Ивченко Г.\,И., Каштанов~В.\,А., Коваленко~И.\,Н.} Теория массового
    обслуживания.~--- 2-е изд.~--- М.:
    Либроком, 2012. 304~c.
    
  \bibitem{Bruneel:Kim}%2
    \Au{Bruneel H., Kim~B.} Discrete-time models for communication systems
    including ATM.~--- Norwell: Kluwer Academic Publs.,
    1993.  Vol.~205. 210~p.
    
  \bibitem{Alfa}%3
    \Au{Alfa A.\,S.} Applied discrete-time queues.~--- 2nd ed.~--- New York,
    NY, USA: Springer-Verlag, 2016. 400~p.
    
    \bibitem{Pechinkin:Razumchik}%4
    \Au{Печинкин~А.\,В., Разумчик~Р.\,В.} Системы массового обслуживания в~дискретном времени.~--- М.: Физматлит, 2018. 432~с.
    
  
    
  \bibitem{Gergely:Torok:1974} %5
   \Au{Gergely T., T\,$\ddot{\mbox{\!o}}$r\,$\ddot{\mbox{\!o}}$k~T.\,L.} On the busy period of discrete-time
    queues~// J.~Appl. Probab., 1974. Vol.~11. Iss.~4. P.~853--857. doi: 10.2307/3212571.
    
  \bibitem{Chaudhry:Zhan:1994} %6
\Au{Chaudhry M.\,L., Zhao Y.\,Q.} First-passage-time and busy-period
    distributions of discrete-time Markovian queues: $\mathrm{Geom}(n)/\mathrm{Geom}(n)/1/N$~//
    Queueing Syst., 1994. Vol.~18. P.~5--26. doi: 10.1007/BF01158772.
    
  \bibitem{Foss:Sapozhnikov:2004} %7
\Au{Foss S., Sapozhnikov~A.} On the existence of moments for the busy
    period in a~single-server queue~// Math. Oper.
    Res., 2004. Vol. 29. Iss.~3. P.~592--601. doi: 10.1287/moor.1030.0074.
    
  \bibitem{Brown:Balakrishnan:2021} %8
\Au{Brown G.\,B.} Busy periods of discrete-time queues using the Lagrange
    implicit function theorem~// Oper. Res.
    Lett., 2021. Vol.~49. P.~650--654.  doi: 10.1016/j.orl.2021.06.014.
    
  \bibitem{Fedotkin:1975} %9
\Au{Федоткин М.\,А.} Тео\-ре\-ти\-ко-мно\-жест\-вен\-ный подход при анализе
    дискретных нелинейных систем массового обслуживания~// Автоматика 
    и~вычислительная техника, 1975. №\,2.  С.~58--64.
    
  \bibitem{Fedotkin:1978} %10
\Au{Федоткин М.\,А.} Неполное описание квазирегенерирующих входных потоков 
неоднородных требований и~транспортные потоки~// 4-е Всесоюзное совещание по статистическим
    методам тео\-рии управ\-ле\-ния.~--- М.: Наука, 1978. С.~234--236.
    
  \bibitem{Fedotkin:1981} %11
\Au{Федоткин М.\,А.} Неполное описание потоков неоднородных требований~//
     Теория  массового обслуживания.~--- М.: МГУ--ВНИИСИ, 1981. С.~113--118.
     
%\columnbreak
    
  \bibitem{Bacelli-1} %12
\Au{Baccelli F., Makowski~A.\,M.}
    {Direct martingale argument for stability: The $M/G/1$ case}~//
    Syst. Control Lett., 1985. Vol.~6. P.~181--186. doi: 10.1016/0167-6911(85)90038-6.
    
  \bibitem{Bacelli-2} %13
\Au{Baccelli F., Makowski~A.\,M.}
  {Dynamic, transient and stationary behavior of the $M/\mathrm{GI}/1$ queue via 
martingales}~//
   Ann. Probab., 1989. Vol.~17. Iss.~4. P.~1691--1699.
  
\bibitem{Klimenok} %14
\Au{Klimenok V.\,L.} On the modification of Rouche's theorem for the queueing
  theory problems~// Queueing Syst., 2001. Vol.~38. P.~431--434. doi: 10.1023/A:1010999928701.
  
  \end{thebibliography}

 }
 }

\end{multicols}

\vspace*{-9pt}

\hfill{\small\textit{Поступила в~редакцию 04.03.24}}

%\vspace*{8pt}

%\pagebreak

\newpage

\vspace*{-28pt}

%\hrule

%\vspace*{2pt}

%\hrule

\vspace*{-2pt}


\def\tit{TOWARDS A DEFINITION OF~A~BUSY PERIOD\\ UNDER~NONLOCAL DESCRIPTION OF~INPUT FLOWS}


\def\titkol{Towards a definition of~a~busy period under~nonlocal description of~input flows}


\def\aut{A.\,V.~Zorine}

\def\autkol{A.\,V.~Zorine}

\titel{\tit}{\aut}{\autkol}{\titkol}

\vspace*{-15pt}


\noindent
National Research Lobachevsky State University of Nizhny Novgorod, 23~Prosp.\ Gagarina, Nizhni Novgorod 603022, Russian Federation






\def\leftfootline{\small{\textbf{\thepage}
\hfill INFORMATIKA I EE PRIMENENIYA~--- INFORMATICS AND
APPLICATIONS\ \ \ 2024\ \ \ volume~18\ \ \ issue\ 3}
}%
 \def\rightfootline{\small{INFORMATIKA I EE PRIMENENIYA~---
INFORMATICS AND APPLICATIONS\ \ \ 2024\ \ \ volume~18\ \ \ issue\ 3
\hfill \textbf{\thepage}}}

\vspace*{2pt}





\Abste{In course of a probabilistic modeling and analysis of complex controlled
queueing systems with several conflicting input flows, in a~series of papers, an approach
was successfully applied, one of its features being a~nonlocal description of various system
building blocks. In this description, some information about true arrival and leave
times of customers is lost. It leads to difficulties in defining a~busy period but that is one of
classic performance metrics for an operating queueing system. In this paper, a~ controlled
queuing system busy period definition is based on selecting those observation instants
when queues reach zero level. A~cyclic service algorithm with fixed switching times as an
example using a~martingale technique and effective computational formulas are obtained for
the mathematical expectation of busy periods related to individual queues.}


\KWE{controlled queueing system; nonlocal description of blocks; nonordinary
Poisson flows; cyclic service algorithm; busy period; multivariate denumerable Markov
chain; martingale; generalized Rouch\'{e}'s theorem; Lagrange interpolation polynomial}

  \DOI{10.14357/19922264240306}{YKSIBJ}

%\vspace*{-12pt}


    
     % \Ack

%\vspace*{-3pt}

%\noindent



  \begin{multicols}{2}

\renewcommand{\bibname}{\protect\rmfamily References}
%\renewcommand{\bibname}{\large\protect\rm References}

{\small\frenchspacing
 {%\baselineskip=10.8pt
 \addcontentsline{toc}{section}{References}
 \begin{thebibliography}{99}


%1
\bibitem{Ivchenko:Kashtanov:Kovalenko-1}
\Aue{Ivchenko, G.\,I., V.\,A.~Kashtanov, and I.\,N.~Kovalenko.} 2012.
\textit{Teoriya massovogo obsluzhivaniya} [Queueing theory]. 2nd ed.
Moscow: Librokom. 304~p.

%2
\bibitem{Bruneel:Kim-1}
\Aue{Bruneel, H., and B.~Kim}. 1993.
\textit{Discrete-time models for communication systems including ATM}. 
Norwell: Kluwer Academic Publs. Vol.~205. 210~p.
 

%3
\bibitem{Alfa-1}
\Aue{Alfa, A.\,S.} 2016. 
\textit{Applied discrete-time queues}. 2nd ed. 
New York, NY: Springer-Verlag. 400~p.

%4
\bibitem{Pechinkin:Razumchik-1}
\Aue{Pechinkin, A.\,V., and R.\,V.~Razumchik.} 2018. 
\textit{Sistemy massovogo obsluzhivaniya v~diskretnom vremeni} [Discrete time queuing systems]. 
Moscow: Fizmatlit. 432~p.


%5
\bibitem{Gergely:Torok:1974-1} 
\Aue{Gergely, T., and T.\,L.~T$\ddot{\mbox{o}}$r$\ddot{\mbox{o}}$k.} 1974.
On the busy period of discrete-time queues. 
\textit{J.~Appl. Probab.} 11(4):853--857.
doi: 10.2307/3212571.

%6
\bibitem{Chaudhry:Zhan:1994-1} 
\Aue{Chaudhry, M.\,L., and Y.\,Q.~Zhao.} 1994.
First-passage-time and busy-period distributions of discrete-time Markovian queues: $\mathrm{Geom}(n)/\mathrm{Geom}(n)/1/N$. 
\textit{Queueing Syst.} 18:5--26. 
doi: 10.1007/BF01158772.

%7
\bibitem{Foss:Sapozhnikov:2004-1} 
\Aue{Foss, S., and A.~Sapozhnikov.} 2004. 
On the existence of moments for the busy period in a~single-server queue. 
\textit{Math. Oper. Res.} 29(3):592--601. doi: 10.1287/moor. 1030.0074.

%8
\bibitem{Brown:Balakrishnan:2021-1} 
\Aue{Brown, G.\,B.} 2021. 
Busy periods of discrete-time queues using the Lagrange implicit function theorem. 
\textit{Oper. Res. Lett.} 49(5):650--654. doi: 10.1016/j.orl.2021.06.014.

%9
\bibitem{Fedotkin:1975-1} 
\Aue{Fedotkin, M.\,A.} 1975.
Set-theoretic approach in analyzing discrete nonlinear queuing systems.
\textit{Autom. Control  Comp.~S.} 9(2):50--54.
    
%10
\bibitem{Fedotkin:1978-1} 
\Aue{Fedotkin, M.\,A.} 1978.
Nepolnoe opisanie kva\-zi\-re\-ge\-ne\-ri\-ru\-yushchikh vkhodnykh potokov neodnorodnykh trebovaniy i~transportnye potoki 
[Incomplete description of quasi-regenerating input flows of nonhomogeneous customers and transport flows].
\textit{4-e Vsesoyuznoe soveshchanie po statisticheskim metodam teorii upravleniya} 
[4th All-Union Meeting on Statistical Methods of Control Theory]. Moscow: Nauka. 234--236.


%11
\bibitem{Fedotkin:1981-1} 
\Aue{Fedotkin, M.\,A.} 1981. 
Nepolnoe opisanie potokov neodnorodnykh trebovaniy [Incomplete description of flows of inhomogeneous customers].
\textit{Teoriya  massovogo obsluzhivaniya} [Queueing theory].
Moscow: MGU--VNIISI. 113--118.

%12
\bibitem{Bacelli-1-1}
\Aue{Baccelli, F., and A.\,M.~Makowski.} 1985.
Direct martingale argument for stability: The $M/G/1$ case. 
\textit{Syst. Control Lett.} 6(3):181--186.
doi: 10.1016/0167-6911(85)90038-6.

%13
\bibitem{Bacelli-2-1}
\Aue{Baccelli, F., and A.\,M.~Makowski.} 1989.
Dynamic, transient and stationary behavior of the $M/\mathrm{GI}/1$ queue via martingales.
\textit{Ann. Probab.} 17(4):1691--1699.

%14
\bibitem{Klimenok-1}
\Aue{Klimenok, V.\,L.} 2001. 
On the modification of Rouche's theorem for the queueing theory problems. 
\textit{Queueing Syst.} 38:431--434.
doi: 10.1023/A:1010999928701.




\end{thebibliography}

 }
 }

\end{multicols}

\vspace*{-6pt}

\hfill{\small\textit{Received March 4, 2024}} 

\vspace*{-18pt}

\Contrl

\vspace*{-3pt}

\noindent 
\textbf{Zorine Andrei V.} (b.\ 1978)~--- Doctor of Science in physics and mathematics, associate professor, 
head of department,
National Research Lobachevsky State University of Nizhny Novgorod, 
23~Prosp.\ Gagarina, Nizhni Novgorod 603022, Russian Federation; \mbox{andrei.zorine@itmm.unn.ru}


\label{end\stat}

\renewcommand{\bibname}{\protect\rm Литература} 