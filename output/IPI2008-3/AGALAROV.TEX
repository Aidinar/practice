\def\stat{aga}


\def\tit{ФУНКЦИЯ СТОИМОСТИ РЕСУРСОВ В ЭКОНОМИЧЕСКОЙ МОДЕЛИ
УПРАВЛЕНИЯ ГРИД}
\def\titkol{Функция стоимости ресурсов в экономической модели
управления ГРИД}

\def\autkol{Я.\,М.~Агаларов}
\def\aut{Я.\,М.~Агаларов$^1$}

\titel{\tit}{\aut}{\autkol}{\titkol}

%{\renewcommand{\thefootnote}{\fnsymbol{footnote}}\footnotetext[1]
%{Работа выполнена при
%поддержке РФФИ, гранты 08-01-00345, 08-01-00363,
%08-07-00152.}}

\renewcommand{\thefootnote}{\arabic{footnote}}
\footnotetext[1]{Институт проблем
информатики Российской академии наук, yagalarov@ipiran.ru}

\Abst{Рассматривается задача максимизации дохода владельца ресурсов локального узла
грид с экономической моделью управления, в которой выделяемые внешнему
пользователю ресурсы оплачиваются, а сумма платы зависит от спроса и предложения на
ресурсы. В рассматриваемой модели очередь глобальных заданий формируется только в
центре планирования ресурсов грид, в котором осуществляется поиск, выбор и
резервирование требуемых ресурсов, и отправка задания на ресурсы происходит
одновременно с их выбором и резервированием. Предлагается функция стоимости
ресурсов, использование которой позволит их владельцу осуществить эффективное
разделение ресурсов локального узла грид между глобальными и локальными
заданиями. Приведены результаты аналитического исследования рассматриваемой
задачи и сравнительного анализа предлагаемых решений с использованием
компьютерного моделирования.}

\KW{грид; модель распределения ресурсов; многопроцессорные задания; владелец
ресурсов; марковский процесс; стратегия}

      \vskip 24pt plus 9pt minus 6pt

      \thispagestyle{headings}

      \begin{multicols}{2}

      \label{st\stat}



\section{Введение}

  Одной из эффективных моделей распределения ресурсов грид считается
экономическая, где цена каждого ресурса определяется потребностями в нем
пользователей и его доступностью~\cite{1ag, 2ag}. Главными субъектами в этой
модели являются пользователи и владельцы ресурсов. Оба субъекта имеют
собственные стратегии. Пользователи ресурсов применяют стратегии решения
своих прикладных задач в зависимости от требуемого времени и наличного
бюджета. Владельцы ресурсов используют стратегию получения наибольшей
выгоды от вложенных средств.

  Задача владельца~--- распределение принадлежащих ему вычислительных
ресурсов (процессоров, оперативной и постоянной памяти, программ и данных)
между глобальными и локальными заданиями. Жесткое разделение ресурсов
вычислительных комплексов (ВК) между локальными и глобальными потоками
приводит к фрагментации ресурсов и их простою~\cite{3ag}. Лучше, если
обоим потокам доступны все ресурсы (часть ресурсов), но тогда необходим
механизм, который бы, сохраняя права владельцев ресурсов, регулировал
интенсивность поступления глобальных и локальных заданий.

  В качестве такого механизма в экономической модели управления грид
предлагается конкуренция локальных и глобальных заданий на основе
сравнения их локальных приоритетов, назначаемых локальной системой
управления ресурсами (ЛСУР). Вычисление локального приоритета
глобального задания предлагается производить (например, как в~\cite{4ag,
5ag}) функцией, зависящей от потребляемых ресурсов и платы за задание.
Конкретный вид зависимости устанавливается владельцем каждого ВК.
В~частности, внешний поток может быть перекрыт полностью. Однако
остается вопрос о виде зависимости стоимости предоставления ресурсов от
типа задания.

  Ниже рассматривается модель грид с не\-от\-чуж\-да\-емыми ресурсами (совместно
используемыми внешними пользователями и их владельцами), в которой
реализована двухуровневая организация управления ресурсами 
(рис.~1)~\cite{1ag}. 

Основные функции верхнего (глобального)
уровня~--- это формирование очереди глобальных заданий и в соответствии с
приоритетами и запросами пользователей осуществление поиска и выделение
требуемых ресурсов. 
К основным функциям нижнего (локального) уровня
относятся: формирование ответа на запрос верхнего уровня, мониторинг
состояния ВК, взаимодействие с владельцем ресурсов и, после того как задание
доставлено на ВК, осуществление контроля над его выполнением.


  Глобальное задание в рассматриваемой модели отправляется на ВК только
после выбора ему верхним и нижним уровнями управления требу\-емых
вычислительных ресурсов (процессоров). Выбор ресурсов осуществляется в
соответствии с требованиями задания и согласно установленному в %\linebreak
 локальном
узле соглашению между владельцем и пользователем (посредством ЛСУР).
Глобальные задания могут быть как однопроцессорные, так и
многопроцессорные, причем одновременно используемые
многопроцессорными заданиями вычислительные ресурсы могут принадлежать
разным ВК. 
\begin{figure*} %[h] %fig1
\vspace*{1pt}
\begin{center}
\mbox{%
\epsfxsize=153.317mm %153.317mm 
\epsfbox{aga-1.eps} }
\end{center}
\vspace*{-9pt} 
\Caption{Структурная схема грид
  \label{f1ag}}
  \end{figure*}
Локальные задания в данной модели всегда однопроцессорные.
Задания различаются по локальным приоритетам, которые учитываются при
распределении ресурсов следующим образом. Доставленное на локальный
уровень глобальное задание сразу же занимает все выделенные ему ресурсы ВК
и после завершения выполнения освобождает все занятые ресурсы
одновременно. Если в момент поступления глобального задания выделенные
ему ресурсы заняты локальными заданиями, то последние освобождают эти
ресурсы, переходят в состояние ожидания и продолжают выполнение после
освобождения ресурсов. Приоритеты глобальных заданий по отношению друг к
другу определяются величинами платы за требуемые ресурсы, определяемыми
на верхнем уровне, и учитываются на стадии выделения ресурсов при
формировании очереди на глобальном уровне. Локальное задание, в отличие от
глобальных, при отсутствии свободного ресурса требуемого типа становится в
очередь в накопителе ВК (при наличии вакантного места) и при появлении
свободного ресурса требуемого типа занимает его и выполняется согласно
выбранной в ВК дисциплине обслуживания. При отсутствии свободных мест в
накопителе локальное задание теряется.
{\looseness=1

}

  В рассматриваемой модели принято следующее соглашение между
глобальными пользователями и владельцами ресурсов:
  \begin{enumerate}[(1)]
\item если в момент поступления глобального задания накопитель (постоянная
память) хотя бы у одного требуемого типа процессоров пол\-ностью занят
(отсутствует требуемый объем свободной памяти) или нет в наличии
необходимого чис\-ла свободных (или занятых локальными заданиями)
процессоров требуемого типа, то оно не получает вычислительные ресурсы в
данном ВК;
\item если нет ограничений, указанных в п.~1, то глобальное задание
получают процессоры данного ВК (в первую очередь, свободные) согласно
выбранной владельцем ресурсов функции стоимости, если в данный момент
времени плата пользователя не меньше значения функции стоимости, иначе
получает отказ;
\item если глобальное задание получают процессоры, занятые локальными
заданиями, то они освобождаются, а занявшие их локальные задания занимают
свободные места в соответствующем накопителе. После этого поступившая
глобальная заявка занимает выделенные ей ресурсы, а выбывшее локальное
задание дообслуживается первым же освободившимся процессором требуемого
типа.
\end{enumerate}

  Предлагается функция стоимости предоставления ресурсов для глобальных и
локальных заданий, увеличивающая доход владельца ВК. Данная функция
определяет зависимость требуемой платы за выполнение задания от текущего
состояния ресурсов. Предлагаемая функция стоимости может быть
использована владельцем ресурсов ВК для выгодного разделения
вычислительных ресурсов между внешними и внутренними пользователями.


\section{Постановка задачи}

  В качестве модели ВК рассматривается сис\-те\-ма массового обслуживания (СМО)
  с $I$ различными
пуассоновскими потоками заявок (заданий), $M$ различными типами приборов
(ресурсов) и $B_l$ приборами $l$-го типа, $l = (\overline{1,\,M})$. Пусть потоки
заявок пронумерованы числами 1, \ldots , $I$, типы приборов числами 1, \ldots ,
$M$, $I\geq M$.

  Введены следующие предположения:
  \begin{enumerate}[1.]
\item Интенсивности поступающих на СМО потоков равны $0 < \lambda_i < \infty$, 
$i=(\overline{1,\,I})$. Для каждого потока с номером $i$ задано
множество типов приборов~$L_i$, на котором он может обслуживаться.
Заявка потока с номером $i$ ($i$-заявка), где $i \in (\overline{1,\,M})$
(локальные заявки), может занимать один любой свободный прибор $i$-го
типа, т.\,е.\ $L_i = \{i\}$ для $i\in (\overline{1,\,M})$. Заявки остальных
потоков (с номерами $i\in (\overline{M+1,\,I})$) (глобальные заявки) могут
требовать для обслуживания одновременно несколько приборов, причем
обязательно различного типа (т.\,е.\ множество $L_i$ может состоять из
более чем одного элемента).
\item Приборы $l$-го типа, $l\in (\overline{1,\,M})$, имеют общий накопитель
емкости $A_l$, который могут занимать только локальные $l$-заявки.
\item Если в момент поступления локальной $i$-заявки, $i\in
(\overline{1,\,M})$, в накопителе приборов $i$-го типа нет свободных мест,
она сразу покидает узел, в противном случае при наличии свободного
прибора соответствующего типа сразу поступает на обслуживание, а при
отсутствии~--- занимает место в накопителе.
\item Если в момент поступления глобальной $i$-за\-яв\-ки, $i\in
(\overline{M+1,\,I})$, хотя бы у одного требуемого типа приборов в
накопителе отсутствует свободное место или все приборы заняты
глобальными заявками, она сразу покидает систему.
\item Если в момент поступления глобальной за\-явки нет ограничений,
указанных в п.~4, то заявка допускается или не допускается на выполнение
согласно выбранной стратегии распределения ресур\-сов, зависящей от
текущего состояния системы (числа заявок каждого типа в системе).
\item Если глобальная заявка допускается в систему и все приборы
некоторого требуемого типа заняты заявками, то одна из локальных заявок,
занимающая требуемый прибор, освобождает его и занимает свободное
место в соответствующем накопителе. Сразу после этого поступившая
глобальная заявка занимает по одному свободному прибору требуемого типа,
а каждая выбывшая заявка дообслуживается первым же освободившимся
прибором требуемого типа.
\item Время обслуживания $i$-заявки (суммарное время занятия приборов
$i$-заявкой)~--- экспоненциально распределенная случайная величина с
параметром $0 < \mu < \infty$, независимая от других случайных событий в
узле, $i = (\overline{1,\,I})$.
\item Выполненная заявка освобождает одновременно все занятые ею
приборы и покидает систему навсегда.
\item Известна плата за выполнение локальной $i$-за\-яв\-ки, $i= (\overline{1,\,M})$.
\end{enumerate}

  Введем обозначения:

  $d_i >0$~--- плата за выполнение $i$-заявки, $i =$\linebreak 
  $=(\overline{1,\,I})$;

  $k_i$~--- число $i$-заявок в системе в некоторый момент времени, $i =
(\overline{1,\,I})$;

  $\overline{k} = (k_1, \ldots , k_I)$~--- вектор-столбец, описы\-ва\-ющий
состояние системы;

  $\overline{b}_l^T = (b_{1l},\ldots , b_{Il})$~---  вектор-строка, где
$b_{il}=0$, если приборы $l$-го типа не требуются для выполнения $i$-заявки,
и $b_{il}=1$ в противном случае;

  $m_l = \overline{b}_l^T \overline{k} = \sum\limits_{i=1}^I b_{il} k_i$~--- суммарное
число заявок в приборах $l$-го типа и в их очереди;

  $N = \{ \overline{k}:\ 0\leq \overline{b}_l^T \overline{k} \leq B_l+A_l$ при
$l=1,\ldots , M$ и $0\leq k_i\leq \underset{l\in L_i}{\min} \{B_l\}$ при $i=M+1,\ldots
, I\}$~--- пространство состояний системы (множество всех возможных
состояний системы);

  $D_l = B_l - \sum\limits_{i=M+1}^I b_{il} k_i$~--- число приборов $l$-го типа, не
занятых глобальными заявками;

  $N_i = \{ \overline{k}:\ \overline{k}\in N,\ k_i < D_i+A_i\}$~--- множество
состояний, при которых в системе есть хотя бы один свободный прибор для
$i$-заявок, $i\in (\overline{1,\,M})$;

  $N_i = \{\overline{k}:\ \overline{k}\in N_l, \, D_l>0\ \mbox{для всех}\ l\in
L_i\}$~---  множество состояний, при которых в сис\-те\-ме есть требуемое число
приборов для $i$-заявок, $i\in (\overline{M+1,\,I})$;

  $\overline{N}_i = \{\overline{k}:\ \overline{k}\in N,\,k_i = D_i+A_i\}$~---
множество состояний, при которых в сис\-те\-ме нет требуемого числа приборов
для $i$-заявок, $i\in (\overline{1,\,M})$;

  $\overline{N}_i = \{\overline{k}:\ D_l=0$ или
$\overline{k}\in\overline{N}_l$ хотя бы для одного $l\in L_i\}$~---
множество состояний, при которых в сис\-те\-ме нет требуемого чис\-ла приборов
для $i$-заявок, $i\in (\overline{M+1,\,I})$;

$\chi (A)$~--- функция-индикатор: 
  $$
  \chi (A) = \begin{cases}
  1, \mbox{если выполняется условие\ } A\,,\\
  0, \mbox{если не выполняется условие\ } A \,.
  \end{cases}\,.
  $$

  Пусть $\overline{r}(\overline{k}) = (r_1(\overline{k}),\ldots ,
r_I(\overline{k}))$, где каждая компонента $r(\overline{k})$ может принимать
одно из значений: 0 или 1, если $\overline{k}\in N_i$, и~1, если
$\overline{k} \in \overline{N}_i$. Политику распределения ресурсов ВК
определим с помощью целочисленной функции $\overline{r}(\overline{k})$
сле\-ду\-ющим образом. Если в момент поступления $i$-заявки система находится
в состоянии $\overline{k}$, то она принимается в систему в случае
$r_i(\overline{k})=0$ и не принимается при $r_i(\overline{k})=1$, $i=1,\ldots , ,
I$. Набор $\overline{r} = \{ \overline{r}(\overline{k}),\ \overline{k}\in N\}$ будем
называть \textit{стратегией распределения приборов (ресурсов)}. В дальнейшем будем
рассматривать только стационарные стратегии (постоянные во времени) и под
словом стратегия будем понимать стационарную стратегию. Обозначим через
$g^{\overline{r}}$ среднее суммарного дохода (предполагаемой платы за
выполнение заявок, включая и локальные), теряемого системой в единицу
времени из-за занятости приборов или недопуска в систему заявок. Пусть
  $R$~--- множество всех возможных $\overline{r}$.

  Ставится задача: найти стратегию $\overline{r}^* \in R$ такую, что при
$\overline{r} = \overline{r}^*$ функция $g^{\overline{r}}$ достигает
минимального значения, т.\,е.
  \begin{equation*}
  \underset{\overline{r}\in R}{\min} g^{\overline{r}} = g^{\overline{r}^*}\,.
%  \label{e1ag}
  \end{equation*}

\section{Решение задачи}

  Для решения задачи воспользуемся аппаратом теории марковских процессов
принятия решений, а именно итерационным алгоритмом динамического
программирования, называемым итерационным алгоритмом
  Ховарда~\cite{6ag, 7ag}.

  Отметим, что для каждой фиксированной стратегии $\overline{r}\in R$
процесс перехода рассматриваемой СМО из одного состояния в другое
описывается марковским процессом~\cite{8ag}. Пусть
$\lambda_i(\overline{k})$~--- интенсивность поступающего в систему потока
$i$-заявок в состоянии $\overline{k}$, $\mu(k_i)$~--- интенсивность
обслуженного потока $i$-заявок в состоянии $\overline{k}$ и $\overline{r}$~---
фиксированная стратегия. Тогда из введенных предположений и определения
стратегии $\overline{r}$ следует, что для любого $\overline{k}\in N$
%
%\noindent
  \begin{align*}
  \lambda_i (\overline{k}) &=
  \begin{cases}
  0,\, & \mbox{если}\ r_i(\overline{k})=1\,,\\
  \lambda_i\,, & \mbox{если}\ r_i (\overline{k})=0\,,
  \end{cases}\\
  \mu (k_i) &=
  \begin{cases}
  \mu D_i\,, & \mbox{если}\ k_i>D_i\,,\ i\in (\overline{1,\,M})\,,\\
  \mu k_i,\, & \mbox{если}\ k_i \leq D_i\ \mbox{и}\\
  & i\in (\overline{1,\,M})\ \mbox{или}\ i\in (\overline{M+1,\,I})\,;
  \end{cases}\\
  i & = (\overline{1,\,I})\,.
  \end{align*}

  Как видно, в рассматриваемом марковском процессе переход системы из
одного состояния в другое может произойти только в двух случаях: при
поступлении заявки или при окончании выполнения заявки. Матрица
интенсивностей перехода $A^{\overline{r}} = (a^{\overline{r}}
(\overline{k},\overline{j}))_{(\overline{k},\overline{j})\in N}$ данного
марковского процесса имеет следующий вид. Интенсивность перехода из
состояния $\overline{k}$ в $\overline{j}$, вызванная поступлением или
окончанием выполнения $i$-заявки, равна:
%
%\noindent
  \begin{equation}
a_i^{r_i}(\overline{k},\overline{j}) =
  \begin{cases}
  \lambda_i & \mbox{при}\ \overline{k}\in N_i\,,\ \overline{j}=\overline{k}+1_i\,,\\
  & r_i(\overline{k})=0\,;\\
  \mu (k_i) & \mbox{при}\ k_i\geq 1\,, \overline{j} =\overline{k}-1_i\,,\\
  & \overline{k}\in N\,;\\
  0 & \mbox{в\ остальных\ случаях}
  \end{cases}
\label{e2ag}
  \end{equation}
  при $\overline{k} \not=\overline{j}$, где $1_i$~--- вектор-столбец, у которого
$i$-я компонента равна~1, а остальные равны~0, $i=(\overline{1,\,I})$,
$\overline{k},\overline{j} \in N$. Тогда по определению матрицы
интенсивностей перехода $a^{\overline{r}}(\overline{k},\overline{j}) =
\sum\limits_{i=1}^I a_i^{r_i} (\overline{k},\overline{j})$ при
$\overline{k}\not=\overline{j}$ и $a^{\overline{r}} (\overline{k},\overline{k}) = -
\sum\limits_{\overline{j}\in N,\, \overline{j}\not=\overline{k}}
a^{\overline{r}}(\overline{k},\overline{j})$, $\overline{k},\overline{j}\in N$.

  Отметим, что рассматриваемый марковский процесс всегда имеет
стационарные вероятности состояний (это следует из введенных
предположений и типа рассматриваемой СМО~--- неприводимая марковская
цепь и конечное число состояний)~\cite{8ag}.

  В дальнейшем всюду будем предполагать, что множество $R$ включает
только те стратегии, при которых рассматриваемый марковский процесс имеет
только одно эргодическое множество состояний.

  Определим начальную стратегию $\overline{r}^{\mathrm{н}}$:
  \begin{align*}
  r_i^{\mathrm{н}} (\overline{k}) &=
  \begin{cases}
  0 & \mbox{при}\ \overline{k}\in N_i\,,\ \sum\limits_{l=1}^M b_{il}=1\,,\\
  1 & \mbox{при}\ \overline{k} \in \overline{N}_i\ \mbox{или}\
\sum\limits_{l=1}^M b_{il} >1\,,
  \end{cases}\\
  i & = (\overline{1,\,I})\,.
  \end{align*}

  Заметим, что при начальной стратегии каждая допускаемая в систему заявка
требует для выполнения только один прибор, а все заявки, требующие
одновременно более одного прибора, не допускаются в систему.

  Из~(\ref{e2ag}) следует, что при $\overline{r} = \overline{r}^{\mathrm{н}}$
интенсивности $a_i^{r_i} (\overline{k},\overline{j})$ имеют вид:
  \begin{align}
  a_i^{r_i} (\overline{k},\overline{j}) & =
  \begin{cases}
 \lambda_i & \mbox{при}\ \overline{k}\in N_i\,,\ \overline{j}=\overline{k}+1_i\,,\\
& r_i(\overline{k}) =0\,,\ 1\leq i \leq M\,;\\
 \mu(k_i)& \mbox{при}\ k_i \geq 1\,,\\
 & \overline{j} = \overline{k}-1_i\,,\ \overline{k}\in N\,;\\
 0 & \mbox{в\ остальных\ случаях}\,;
  \end{cases}\label{e3ag}\\
  \overline{k} & \not=\overline{j}\,;\ \ \overline{k}, \overline{j}\in N\,; \
i=(\overline{1,\,I})\,.\notag
  \end{align}

  Обозначим через $q^{\overline{r}}_{\overline{k}}$ норму стоимостных потерь
в состоянии $\overline{k}$ (теряемый в единицу времени доход, когда система
находится в состоянии $\overline{k}$) при стратегии $\overline{r}$, через
$g^{\overline{r}}$~--- средние стационарные стоимостные потери в единицу
времени. Далее для краткости вместо слов стоимостные потери будем писать
просто потери. Заметим, что при $\overline{r}^{\mathrm{н}}$ каж\-дые $i$-й тип
приборов и $i$-й поток заявок образуют СМО типа $M/M/B_i/A_i$ с
ограниченной очередью. Следовательно, имеем:
\begin{align}
q^{\overline{r}}_{\overline{k}} & =  \sum\limits_{i:\overline{k}\in\overline{N}_i} d_i\lambda_i +
\sum\limits_{\substack{{i:\overline{k}\in N_i}\\{r_i(\overline{k})=1}}} d_i\lambda_i\,;\notag\\
q^{\overline{r}^{\mathrm{н}}}_{\overline{k}} & = \sum\limits_{i=M+1}^I d_i\lambda_i+
\sum\limits_{\substack{{i:k_i =D_i+A_i,}\\{i\in (\overline{1,\,M})}}} d_i\lambda_i\,;
\label{e4ag}\\
g^{\overline{r}^{\mathrm{н}}} & =  \sum\limits^I_{i=M+1} d_i\lambda_i +\sum\limits^M_{i=1} d_i\lambda_i
\pi^{\overline{r}^{\mathrm{н}}}_{B_i+A_i} (\rho_i)\,,\quad \overline{k}\in N\,,\notag
\end{align}
  где $\pi^{\overline{r}^{\prime\prime}}_{B_i+A_i}(\rho_i )$~--- вторая формула Эрланга для
сис\-те\-мы с $B_i$ приборами, $A_i$ местами в накопителе и нагрузкой $\rho_i$,
$i = 1, \ldots , M$.

  Согласно итерационной процедуре Ховарда~\cite{6ag, 7ag} для улучшения
стратегии достаточно найти решение $\overline{r} (\overline{k})$ такое, чтобы
хотя бы в одном состоянии~$\overline{k}$ выполнялось условие:
  \begin{equation}
  q^{\overline{r}}_{\overline{k}} +\sum\limits_{\overline{j}\in N}
a^{\overline{r}}(\overline{k},\overline{j})V_{\overline{j}} <
g^{\overline{r}^{\mathrm{н}}}\,,
\label{e5ag}
  \end{equation}
  где  $g^{\overline{r}^{\mathrm{н}}}$ и $V_{\overline{j}}$ ($\overline{j}\in N$)
являются решением системы уравнений:
  \begin{equation}
  g^{\overline{r}^{\mathrm{н}}}=
q_{\overline{k}}^{\overline{r}^{\mathrm{н}}}+
  \sum\limits_{\overline{j}\in N}
  a^{\overline{r}^{\mathrm{н}}}(\overline{k},\overline{j})
  V_{\overline{j}}
  \label{e6ag}
  \end{equation}
  при $\overline{V}_{\overline{0}} =0$, $\overline{0} = (0,\ldots , 0)$~---
  вектор-столбец, соответствующий нулевому состоянию системы.

  Кроме того, если $\overline{r}^\prime$~--- улучшенная стратегия, то:
  $$
  g^{\overline{r}^\prime} - g^{\overline{r}^{\mathrm{н}}} = \sum\limits_{\overline{k}\in
N}\pi_{\overline{k}}^{\overline{r}^\prime} \gamma_{\overline{k}}\,,
  $$
  где $\gamma_{\overline{k}} = q_{\overline{k}}^{\overline{r}^\prime}+
\sum\limits_{\overline{j}\in N} a^{\overline{r}^\prime}
(\overline{k},\overline{j})V_{\overline{j}}-q_{\overline{k}}^{\overline{r}^{\mathrm{н}}} - 
\sum\limits_{\overline{j}\in N}  a^{\overline{r}^{\mathrm{н}}} (\overline{k}, \overline{j})V_{\overline{j}}$.

  Найдем улучшенную стратегию. Обозначим для краткости $\pi_i =
\pi_{B_i+A_i}^{\overline{r}^{\mathrm{н}}}(\rho_i)$. Для начальной стратегии
$\overline{r}^{\mathrm{н}}$ из~(\ref{e6ag}), подставив~(\ref{e3ag})
и~(\ref{e4ag}), получим:
  \begin{multline}
  \sum\limits_{i=1}^M d_i\lambda_i\pi_i +\sum\limits_{i=M+1}^I d_i\lambda_i =
  \sum\limits_{i=M+1}^I d_i\lambda_i +{}
  \\ {}+
  \sum\limits_{\substack{{i:m_i=D_i+A_i}\\{i\in(\overline{1,\,M})}}}d_i\lambda_i 
  + \sum\limits_{\substack{{i:m_i<D_i+A_i}\\
  { i\in(\overline{1,\,M})}}}\lambda_iV_{\overline{k}-1_i}+{}\\
  {}+
  \sum\limits_{i:k_i\geq 1}\mu(k_i)V_{\overline{k}-1_i} -{}\\
  {}-
  \left ( \sum\limits_{\substack{{i:m_i<D_i+A_i}\\{i\in(\overline{1,\,M})}}}
  \lambda_i +\sum\limits_{{i:} k_i\geq 1} \mu(k_i)\right )
V_{\overline{k}}\,.
  \label{e7ag}
  \end{multline}

  Система уравнений~(\ref{e7ag}) эквивалентна системе уравнений:
  \begin{multline}
  \sum\limits_{i=1}^M d_i\lambda_i\pi_i =
  \sum\limits_{\substack{{i:m_i =D_i+A_i}\\
  {i\in (\overline{1,\,M})}}} d_i\lambda_i+{}\\
  {}+
  \sum\limits_{\substack{{i:m_i <D_i+A_i}\\
  { i\in (\overline{1,\,M})}}}
  \lambda_iV_{\overline{k}+1_i}
  +
  \sum\limits_{\substack{{i:k_i\geq 1}\\{i\in (\overline{1,\,M})}}}\mu (k_i)V_{\overline{k}-1_i}-{}\\
  {}-
  \sum\limits_{\substack{{i:k_i\geq1}\\{i\in (\overline{M+1\,,I})}}}
  \mu(k_i)V_{\overline{k}-1_i}
  -
  \left ( \sum\limits_{\substack{{ i:m_i<D_i+A_i}\\
  {i\in (\overline{1,\,M}) }}} \lambda_i +{}\right.{}\\
  {}+\left.
  \sum\limits_{\substack{{i:k_i\geq 1}\\{i\in (\overline{1,\,M})}}} \mu(k_i)+
  \sum\limits_{\substack{{i:k_i\geq 1}\\{i\in (\overline{M+1,\,I})}}} \mu(k_i)\right )
V_{\overline{k}}\,, \ \
 \overline{k}\in N\,.
  \label{e8ag}
  \end{multline}

  \medskip
  \noindent
  \textbf{Утверждение.} Система уравнений~(\ref{e8ag}) имеет решение вида:
\begin{align}
V_{\overline{k}} & = \sum\limits_{i=1}^M v_{m_i}\,;\notag\\
v_{m_i} & = v_{m_i-1} +u_{m_i}\,;\notag\\
u_{m_i} & = \fr{\mu_{m_i-1}}{\lambda_i}\,u_{m_i-1} +d_i \pi_i\,;\notag\\
v_{m_i} & = u_{m_i} =0\ \ \mbox{при}\ m_i=0\,;\notag
\end{align}
\begin{align}
\mu_{m_i} & = 
\begin{cases}
\mu m_i\,, & \mbox{если}\ m_i<B_i\,,\\
\mu B_i\,, &  \mbox{если}\   m_i\geq B_i\,;
\end{cases}\label{e9ag}\\
\overline{k}&\in N\,,\quad  m_i=1,\ldots , B_i+A_i\,,\quad i=1, \ldots , M\,.\notag
\end{align}

  \medskip
\noindent
\textbf{Следствие.} Справедливы равенства:
  \begin{equation}
  u_{m_i} = \fr{d_i \pi_{B_i+A_i}^{\overline{r}^{\mathrm{н}}} (\rho_i )}
  {\pi_{m_i-1}^{\overline{r}^{\mathrm{н}}} (\rho_i )}\,,
  \label{e10ag}
  \end{equation}
  где $\pi_{m_i}(\rho_i)$~--- первая формула Эрланга с $m_i$ приборами и
нагрузкой $\rho_i$ в случае $m_i <B_i$ и вторая формула Эрланга для системы
с $B_i$ приборами, $m_i- B_i$ местами в накопителе и нагрузкой $\rho_i$ в
случае $m_i\geq B_i$, $m_i = 1, \ldots , B_i + A_i$, $i = 1, \ldots , M$.

  Рассмотрим левую часть неравенства~(\ref{e5ag}) для некоторого
фиксированного состояния $\overline{k}$. Подставив~(\ref{e2ag})
и~(\ref{e4ag}) в левую часть~(\ref{e5ag}), получим:
  \begin{multline}
  q_{\overline{k}}^{\overline{r}}+\sum\limits_{\overline{j}\in N}
a^{\overline{r}} (\overline{k},\overline{j}) V_{\overline{j}} =
  \sum\limits_{i:\overline{k}\in\overline{N}_i} d_i\lambda_i +
  \sum\limits_{\substack{{i:\overline{k}\in N_i}\\
  {r_i(\overline{k})=1}}}d_i\lambda_i +{}\\
  {}+
  \sum\limits_{\substack{{i:\overline{k}\in N_i}\\{r_i(\overline{k})=0}}}
\lambda_i V_{\overline{k}+1_i}+\sum\limits_{i:k_i\geq 1}
\mu(k_i)V_{\overline{k}-1_i} -{}\\
{}-\left (
  \sum\limits_{\substack{{i:\overline{k}\in N_i}\\{r_i(\overline{k})=0}}}
\lambda_i +\sum\limits_{i:k_i\geq 1}\mu(k_i)\right ) V_{\overline{k}}={}\\
  {}=
  \sum\limits_{i:\overline{k}\in\overline{N}_i} d_i\lambda_i+
  \sum\limits_{\substack{{i:\overline{k}\in N_i}\\{r_i(\overline{k})=1}}}
d_i\lambda_i +
  \sum\limits_{\substack{{i:\overline{k}\in N_i}\\{r_i(\overline{k})=0}}}
d_i\lambda_i -
  \sum\limits_{\substack{{i:\overline{k}\in N_i}\\{r_i(\overline{k})=0}}}
d_i\lambda_i +{}\\
{}+
\sum\limits_{\substack{{i:\overline{k}\in N_i}\\{r_i(\overline{k})=0}}}\lambda_i
(V_{\overline{k}+1_i}-V_{\overline{k}})+
  \sum\limits_{i:k_i\geq 1}\mu(k_i)(V_{\overline{k}-1_i} -V_{\overline{k}})={}\\
  {}=
  \sum\limits_{i\in(\overline{1,\,I})} d_i \lambda_i -
  \sum\limits_{\substack{{i:\overline{k}\in N_i}\\{r_i(\overline{k})=0}}}
  d_i\lambda_i +
  \sum\limits_{\substack{{i:\overline{k}\in N_i}\\{r_i(\overline{k})=0}}}\lambda_i
(V_{\overline{k}+1_i}-V_{\overline{k}})+{}\\
{}+\sum\limits_{i:k_i\geq 1}\mu(k_i)
  (V_{\overline{k}-1_i} -V_{\overline{k}})={}\\
  {}=
  \sum\limits_{i\in (\overline{1,\,I})}d_i\lambda_i+
  \sum\limits_{\substack{{i:\overline{k}\in N_i}\\{r_i(\overline{k})=0}}}
\lambda_i (V_{\overline{k}+1_i}-V_{\overline{k}} -d_i)+{}\\
{}+ \sum\limits_{i:k_i\geq
1}\mu(k_i) (V_{\overline{k}-1_i}-V_{\overline{k}})\,.
  \label{e11ag}
  \end{multline}
  Найдем стратегию, при которой~(\ref{e11ag}), т.\,е.\ левая часть~(\ref{e5ag}),
достигает минимума. Заметим, что в~(\ref{e11ag}) все суммы, кроме второй, не
зависят от стратегии. Следовательно, стратегия, на которой достигает
минимума вторая сумма, т.\,е.\ выражение:
  \begin{equation}
  \sum\limits_{\substack{{i:\overline{k}\in N_i}\\
  {r_i(\overline{k})=0}}} \lambda_i (V_{\overline{k}+1_i} - V_{\overline{k}} -
d_i )\,,
  \label{e12ag}
  \end{equation}
и будет искомой. Очевидно, что выражение~(\ref{e12ag}) принимает минимальное значение
при стратегии $\overline{r}$ такой, что:
\begin{align*}
r_i (\overline{k} ) &=
\begin{cases}
0\,, & \mbox{если}\ V_{\overline{k}+1_i} - V_{\overline{k}} < d_i\,,\\
1\,, & \mbox{если}\ V_{\overline{k}+1_i } - V_{\overline{k}}\geq d_i\,,
\end{cases}\\
\overline{k} &\in N_i\,,\quad i=1,\ldots ,I\,.
\end{align*}

  
    Как следует из~(\ref{e10ag}), $u_{m_i} <d_i$, $m_i =1,\ldots , B_i+A_i$,
$i=1,\ldots , M$. Следовательно, выражение~(\ref{e12ag}) достигает минимума на стратегии
$\overline{r}$ такой, что
  \begin{align*}
  r_i(\overline{k}) & = 0\,,\quad \overline{k} \in N_i\,,\quad i=1,\ldots , M\,,\\
  r_i (\overline{k}) & = 0\,,\quad \overline{k}\in N_i\,,\quad i=1,\ldots , M\,,\\
  r_i(\overline{k}) & =
  \begin{cases}
  0\,, & \mbox{если}\ d_i(\overline{k}) <d_i\,,\\
  1\,, & \mbox{если}\ d_i(\overline{k}) \geq d_i\,,
  \end{cases}\\
  & \overline{k}\in N_i\,,\quad i=M+1,\ldots , I\,,\
  \end{align*}
  где $d_i(\overline{k}) = \sum\limits_{l:l\in L_i} u_{m_l+1}$~--- функции стоимости
предо\-став\-ле\-ния приборов для $i$-заявки.

\begin{figure*} %fig2 Рис. 2\vspace*{1pt}
\begin{center}
\mbox{%
\epsfxsize=165.741mm
\epsfbox{aga-2.eps}
}
\end{center}
\vspace*{-9pt}
  \Caption{Зависимость суммарного дохода владельца ресурсов от суммарной
  входной нагрузки~(\textit{а}) и интенсивности потока глобальных
  заданий~(\textit{б}): \textit{1}~--- NEDOP; \textit{2}~--- DINPR; \textit{3}~---
PRIOR
  \label{f2ag}}
  \end{figure*}

  Согласно этой стратегии:
  \begin{itemize}
\item локальная заявка при наличии свободного мес\-та в накопителе приборов
требуемого типа допускается в систему;
\item глобальная заявка при наличии требуемых приборов допускается в
систему, если доход за обслуживание пользователя выше значения
соответствующей функции стоимости (величины $d_i (\overline{k})$), в
противном случае получает отказ.
  \end{itemize}
  

  Ниже приведены результаты вычислительного эксперимента, полученные с
использованием имитационной модели описанной выше системы для
сравнительного анализа эффективности сле\-ду\-ющих схем доступа к ресурсам:
  \begin{itemize}
  \item
NEDOP~--- использует функцию стоимости \mbox{вида}
$$
d_i (\overline{k}) =
\begin{cases}
0\,, & \mbox{если} \ i=1,\ldots , M\,,\\
\infty, & \mbox{если} \ i=M+1, \ldots , I
\end{cases}
$$
для всех $\overline{k} \in N$ (глобальные заявки не допускаются в сис\-те\-му~---
стратегия $\overline{r}^{\mathrm{н}}$);
\item PRIOR~--- использует функцию стоимости вида
$$
d_i (\overline{k}) = 0\,,\quad i=1,\ldots , I, \quad \overline{k} \in N
$$
(все заявки допускаются в сис\-те\-му);
\item DINPR~--- использует функцию стоимости вида
$$
d_i (\overline{k}) =\sum \limits_{{l:} l\in L_i}u_{m_l+1}\,,
$$
где $u_{m_l+1}$, $l=1,\ldots , M$, вычисляются по формуле~(9)
(глобальные заявки допускаются в сис\-те\-му в зависимости от состояния
системы~--- стратегия $\overline{r}^\prime$).
  \end{itemize}

  Была рассмотрена система со следующими параметрами: $M = 3$; $I = 6$;
$B_i =10$, $A_i = 10$ при $i = 1$, 2, 3; $\mu = 1$; $d_i = 1$, $i = 1$,\ldots ,6;
$L_1 = 1$; $L_2 = 2$; $L_3 = 3$, $L_4 = \{1,\,2\}$; $L_5 = \{2,\,3\}$, $L_6 =
\{1,\,2,\,3\}$. На рис.~\ref{f2ag},\,\textit{а} приведен график зависимости дохода владельца
ВК от параметра $\lambda$~--- интенсивности суммарного потока заявок для
указанных выше схем распределения ресурсов при $p_1 = p_2 = p_3 = 0{,}2$;
$p_4 = p_5 = 0{,}1$; $p_6 = 0{,}2$, где $p_i$~--- доля потока $i$-го типа, $i =
1$, \ldots , 6; $\lambda = 10$, 20, 30, 40, 50, 60. На рис.~\ref{f2ag},\,\textit{б} показана
зависимость дохода владельца ресурсов от величины нагрузки глобальных
заданий при фиксированных значениях локальных нагрузок $\lambda_1 =
\lambda_2 = \lambda_3 = 8$.


\section{Заключение}

  Полученные результаты приводят к следующим выводам:
  \begin{enumerate}[1.]
\item Предоставление ресурсов глобальным заданиям (схема DINPR) может
существенно повысить величину дохода владельца ресурсов (согласно
вычислительным экспериментам при сильно загруженных локальными
заданиями ресурсах~--- до 10\%).
\item Бесконтрольный доступ многопроцессорных %\linebreak 
зада\-ний (схема PRIOR)
может существенно снизить доход владельца ресурсов (согласно
вы\-чис\-ли\-тель\-ным экспериментам при сильно загружен\-ных локальными
заданиями ресурсах~--- до~6\%).
\item Схема DINPR эффективнее схемы NEDOP (не допускающей в систему
глобальные задания), если существует хотя бы один тип таких заданий, для
которого плата за выполнение выше значения соответствующей функции
стоимости (т.\,е.\ $d_i (\overline{k}) <d_i$) хотя бы в одном состоянии.
\item Использование адаптивной (зависящей от состояния ресурсов) функции
стоимости (схема DINPR) позволяет эффективно регулировать доступ
глобальных заданий к ресурсам ВК (при сильной нагрузке ресурсов
локальными заданиями увеличение нагрузки глобальных заданий, требующих
ресурсов данного ВК, не приводит к снижению дохода владельца ресурсов).
  \end{enumerate}

{\small\frenchspacing
{%\baselineskip=10.8pt
\addcontentsline{toc}{section}{Литература}
\begin{thebibliography}{9}

\bibitem{2ag}
\Au{Buyya~R., Giddy~J., Abramson~D.}
An economy grid architecture for service-oriented grid com\-puting~// 10th IEEE International
Heterogeneous Com\-puting Workshop (HCW 2001). In conjunction with IPDPS 2001. San
Francisco, USA, April, 2001. {\sf http://citeseer.ist.psu.edu/ buyya01economy.html}.

  \bibitem{1ag}
  \Au{Коваленко~В.\,Н., Коваленко~Е.\,И., Корягин~Д.\,А., Любимский~Э.\,З.}
  Основные положения метода опережающего планирования для грид вы\-чис\-ли\-тель\-но\-го
типа~// Вестник СамГу. Ес\-те\-ст\-вен\-но-на\-уч\-ная сер. <<Ин\-фор\-ма\-ци\-он\-но-вы\-чис\-ли\-тель\-ные
системы>>, 2006. №\,4(44). С.~238--264. {\sf
http://vestnik.ssu.samara.ru/est/2006web4/ivs/ 200642001.pdf}.

\bibitem{3ag}
\Au{Snell~Q., Clement~M., Jackson~D., Gregory~Ch}.
The per\-formance impact of advance reservation meta-scheduling~// Computer Science Department
Brigham Young University Provo, Utah, 2000. {\sf
http://supercluster.org/research/ papers/ipdps2000.pdf}.

\bibitem{4ag}
\Au{Коваленко~В.\,Н., Семячкин~Д.\,А.}
Использование алгоритма Backfill в грид~// Тр.\ Международной конференции
<<Распределенные вычисления и Грид-тех\-но\-ло\-гии в науке и образовании>>. Дубна,
29~июня\,--\,2~июля 2004~г. Дубна: 11-2004-205, ОИЯИ, 2004. С.~139--144.

\bibitem{5ag}
\Au{Агаларов~Я.\,М.}
Динамическая стратегия распределения вычислительных ресурсов локального узла GRID~//
Системы и средства информатики.~--- М.: Наука, 2007. Вып.~17. С.~17--29.

\bibitem{6ag}
\Au{Ховард Р.}
Динамическое программирование и марковские процессы.~--- М.: Советское радио, 1964.
158~с.
\bibitem{7ag}
\Au{Майн Х., Осаки~С.}
Марковские процессы принятия решений.~--- М.: Наука, 1977. 176~с.

\label{end\stat}

\bibitem{8ag}
\Au{Карлин С.}
Основы теории случайных процессов.~--- М.: Мир, 1971. 536~с.

\end{thebibliography}

}
}
\end{multicols}