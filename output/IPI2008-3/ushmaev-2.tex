\def\stat{ushmaev-2}

\def\tit{СЕРВИСНО-ОРИЕНТИРОВАННЫЙ ПОДХОД К РАЗРАБОТКЕ 
МУЛЬТИБИОМЕТРИЧЕСКИХ ТЕХНОЛОГИЙ$^*$}
\def\titkol{Сервисно-ориентированный подход к разработке мультибиометрических технологий}


\def\autkol{О.\,С. Ушмаев}
\def\aut{О.\,С. Ушмаев$^1$}

\titel{\tit}{\aut}{\autkol}{\titkol}

{\renewcommand{\thefootnote}{\fnsymbol{footnote}}\footnotetext[1]
{Работа поддержана грантом РФФИ (проект 07-07-00031) и Программой фундаментальных исследований 
ОНИТ РАН (проект~1.5).}}

\renewcommand{\thefootnote}{\arabic{footnote}}
\footnotetext[1]{Институт проблем информатики Российской академии наук, oushmaev@ipiran.ru
}


\Abst{В настоящее время значительное внимание уделяется технологиям мультибиометрической 
идентификации, т.\,е.\ идентификации человека одновременно по нескольким биометрическим 
признакам. В первую очередь такие технологии востребованы в перспективных системах 
гражданской идентификации, в частности в биометрическом паспорте. В статье предложен подход к 
созданию высокопроизводительных мультибиометрических технологий и систем на базе 
сервисно-ориентированной архитектуры. Приведены результаты разработки программного обеспечения на 
основе новых подходов.}

\KW{биометрические технологии; мультибиометрическая идентификация; многозвенная 
архитектура; аппаратная независимость; сервисно-ориентированная архитектура}

      \vskip 48pt plus 9pt minus 6pt

      \thispagestyle{headings}

      \begin{multicols}{2}

      \label{st\stat}


  \section{ВВЕДЕНИЕ}
  
  Биометрические технологии получили достаточное распространение в 
различных областях применения информационных технологий~--- от электронной 
коммерции до правоохранительной %\linebreak 
деятель\-ности и 
  пас\-порт\-но-ви\-зо\-во\-го сегмента~[1--7]. Суммируя накопленный опыт 
использования биометрических технологий, область их применения можно 
разделить на следующие значительные направления:
  \begin{enumerate}[1.]
  \item Коммерческие приложения:
  \begin{itemize}
\item разграничение физического доступа;
\item учет рабочего времени;
\item управление корпоративной сетью.
\end{itemize}
  \item  Доступ к секретной информации.
  \item Системы идентификации личности:
  \begin{itemize}
\item системы гражданской идентификации (паспорт, визы, 
пользование общественными благами);
\item криминалистические системы.
  \end{itemize}
  \end{enumerate}
  
  Критерии качества биометрической системы в значительной степени зависят 
от назначения и условий эксплуатации. В задачах контроля доступа в 
помещение критичным требованием будет время создания шаблона при 
предъявлении биометрического измерения. Для биометрических систем 
гражданской идентификации (Civil ID) предъявляются максимально жесткие 
требования к времени сравнения, ошибкам 2-го рода и к ошибкам отказа в 
регистрации.
  
  Учитывая опыт многочисленных испытаний биометрических 
  технологий~[8--14], основными критериями качества биометрических 
алгоритмов являются следующие статистические показатели:
  \begin{itemize}
\item ошибка 1-го рода (False Rejection Rate, далее FRR, или False Non 
Match Rate);
\item ошибка 2-го рода (False Acceptance Rate, далее FAR, или False 
Match Rate);
\item среднее время сравнения 1 к 1;
\item среднее время регистрации;
\item доля отказов в регистрации (Fail to Enroll Rate).
\end{itemize}
  
\begin{figure*} %fig1
\vspace*{1pt}
\begin{center}
\mbox{%
\epsfxsize=166.213mm  
\epsfbox{us2-1-2.eps}}
\end{center}
\vspace*{-9pt}
\begin{minipage}[t]{79mm}
  \Caption{Операционное тестирование биометрических систем~[9], 
ошибки 1-го и 2-го рода: \textit{1} и~\textit{2}~--- форма 
лица Visionics~--- FaceIT~(\textit{1}) и (\textit{2}); 
\textit{3} и~\textit{4}~--- отпечаток пальца~--- VeriTouch vr-3~(\textit{1}) и(\textit{2});
\textit{5}~--- голос 
OTG~--- SecurPBX; \textit{6}~--- радужная оболочка глаза~--- Iridian IriScan 
system~2200;
  \textit{7}~--- рисунок вен~--- Neusciences-Biometrics; \textit{8}~--- форма 
ладони~--- HandKey~II 
  \label{f1ush2}}
  \end{minipage}
  \hfill
  \begin{minipage}[t]{79mm}
  \Caption{Технологическое тестирование биометрических алгоритмов 
(2007~г.). Ошибки 1-го и 2-го рода: \textit{1}~--- отпечаток пальца; 
  \textit{2}~--- изображение лица; \textit{3}~--- голос; \textit{4}~--- почерк
  \label{f2ush2}}
  \end{minipage}
  \end{figure*}
  
  Уверенный прирост производительности вы\-чис\-ли\-тель\-ных средств позволяет в 
значительной степени пренебрегать временными характеристиками в 
большинстве функционирующих или перспективных приложений. Поэтому 
типовой оценкой качества биометрической системы является %\linebreak
соот\-но\-ше\-ние 
ошибок 1-го и 2-го рода. На рис.~\ref{f1ush2} приведены результаты измерения 
ошибок распознавания в ходе операционного тестирования различных 
биометрических систем~[9]. На рис.~\ref{f2ush2}  приведены результаты 
технологического тестирования современных российских биометрических 
технологий (взяты SDK производства 2007~г.).
  

%\medskip

%\begin{figure*} %fig3
%\vspace*{12pt}
\begin{center}
\mbox{%
\epsfxsize=78.826mm  
\epsfbox{us2-3.eps} }
\end{center}
\vspace*{3pt}
%\Caption{
{{\figurename~3}\ \ \small{Требования к ошибкам 1-го и 2-го рода в типовых приложениях}}
%\label{f1s}}
%\end{figure*}
\bigskip
\addtocounter{figure}{1}  

  Одной биометрики достаточно для решения большей части задач типовых 
коммерческих приложений, таких как разграничение доступа в корпоративной сети, 
электронная коммерция, контроль доступа в помещение и~т.\,д. В то же время 
большая часть государственных проектов предъявляет предельно жесткие 
требования (рис.~3), которые практически недостижимы при 
использовании единственной биометрики.

 
  В такой ситуации естественным направлением развития биометрических 
технологий является создание мультибиометрических систем, использующих 
одновременно несколько биометрических измерений. На рис.~\ref{f4ush2} 
и~5~[4, 9, 11, 15] показан значительный прирост в качестве 
распознавания при одновременном использовании нескольких образцов одной 
биометрики или нескольких биометрических идентификаторов.
  
  
  
  Если с алгоритмической точки зрения методы интеграции биометрических 
технологий частично проработаны~[3--5, 15--18], то опыт разработки 
мультибиометрических приложений достаточно ограничен. 
  
  В ходе работ по созданию первой российской полнофункциональной 
системы мультибиометрической идентификации АМИС~[6, 7] для 
исполь\-зо\-ва\-ния в аналитических и криминалистических %\linebreak
прило\-же\-ни\-ях были 
сформированы сле\-ду\-ющие основные требования к создаваемому решению:
  \begin{itemize}
\item возможность интеграции в рамках одной сис\-те\-мой максимального 
числа методов био\-мет\-ри\-че\-ской идентификации;
\item возможность одновременной комплексной идентификации по 
нескольким биометрическим признакам;
\item возможность модернизации и замены библиотек, реализующих 
отдельные методы биометрической идентификации;
\item гибкость в конфигурировании;
\item гибкость в распределении вычислений;
\item обеспечение комплексной информационной безопасности и защиты 
биометрической информации;
\item поддержка существующих российских стандартов по биометрическим 
технологиям.
\end{itemize}

 \end{multicols}

  \begin{figure} %fig4
\vspace*{1pt}
\begin{center}
\mbox{%
\epsfxsize=165.49mm  
\epsfbox{us2-4.eps} }
\end{center}
\vspace*{-9pt}
  \Caption{Ошибки 1-го и 2-го рода различных биометрических систем по 
данным тестирования U.K.\ Biometrics Working Group~[9] (\textit{а}): \textit{1}--\textit{8}~---
см.\ рис.~1,~--- и
теоретические ошибки 1-го и 2-го рода многофакторных биометрических 
систем~[4]~(\textit{б}):
\textit{1}~--- форма лица\;+\;отпечаток пальца (ПП)\;+\;голос; 
\textit{2}~--- форма лица\;+\;отпечаток пальца (ПП);
\textit{3}~--- форма лица\;+\;радужная оболочка глаза;
\textit{4}~--- отпечаток пальца (ПП)\;+\;радужная оболочка глаза;
\textit{5}~--- отпечаток пальца (ПП)\;+\;форма ладони;
\textit{6}~--- отпечаток пальца (ПП)\;+\;голос;
\textit{7}~--- радужная оболочка глаза\;+\;форма ладони;
\textit{8}~--- форма лица\;+\;форма ладони;
\textit{9}~--- радужная оболочка глаза;
\textit{10}~--- отпечаток пальца (NIST VTB)
  \label{f4ush2}}
  \vspace*{12pt}
  \end{figure}
  
  \begin{multicols}{2}
  
%  \medskip

%\begin{figure*} %fig5
%\vspace*{-12pt}
\begin{center}
\mbox{%
\epsfxsize=80mm  
\epsfbox{us2-5.eps} }
\end{center}
\vspace*{3pt}
%\Caption{
{{\figurename~5}\ \ \small{Ошибки 1-го и 2-го рода распознавания по нескольким отпечаткам 
пальцев (NIST SD14)~[15]: 
\textit{1}~--- 4~отпечатка;
\textit{2}~--- 2~отпечатка (указательные);
\textit{3}~---  2~отпечатка (большие);
\textit{4}~--- правый указательный;
\textit{5}~---  правый большой;
\textit{6}~---  левый указательный;
\textit{7}~--- левый большой}}
%\label{f1s}}
%\end{figure*}
\bigskip
\addtocounter{figure}{1}  

  
Далее в статье изложен подход к разработке мультибиометрических
технологий, учитывающий перечисленные требования.  
     В разделе~2 изложена методологическая основа разработки 
муль\-ти\-биомет\-ри\-че\-ских технологий. Подход к созданию аппаратной 
архитектуры и программной инфраструктуры мультибиометрической системы, 
удов\-ле\-тво\-ря\-ющей перечисленным требованиям, изложен в разделе~3. 
В~разделе~4 изложены результаты разработки программного обеспечения, 
реа\-ли\-зу\-юще\-го данный подход. 
{ %\looseness=-1

}

  
  \section{Интеграция биометрических технологий}
  
 \vspace*{-18pt}
  
   \begin{figure*} %fig6
\vspace*{1pt}
\begin{center}
\mbox{%
\epsfxsize=143.132mm  
\epsfbox{us2-6.eps} }
\end{center}
\vspace*{-9pt}
  \Caption{Схема биометрической интеграции в паспортной задаче
  \label{f6ush2}}
  \end{figure*}
  
   Разработка интегрированных биометрический (многофакторных или 
мультибиометрических) %\linebreak 
сис\-тем имеет два существенных аспекта~--- 
технический и методологический. Технически интеграция заключается в 
объединении нескольких програм\-мно-ап\-па\-рат\-ных комплексов единым 
интерфейсом. С~методологической точки зрения интеграция является 
достаточно сложной прикладной задачей принятия решения об идентичности 
наборов биометрических  образцов. 
 
Основой для разработки произвольной био\-мет\-ри\-че\-ской, в том числе и 
муль\-ти\-био\-мет\-ри\-че\-ской, системы являются требования фактически 
сложившихся отраслевых стандартов.\linebreak
 Наиболее систематизированным 
документом является стандарт bioAPI\footnote{BioAPI~--- Biometric Application
Programming Interface~--- биометрический программный интерфейс приложений.} 
(основа серии стандартов 
ГОСТ~Р~ИСО/МЭК 19784). Согласно стандарту bioAPI, базовыми функциями 
произвольной биометрической системы являются регистрация\linebreak (enroll) и 
сравнение (match). В ходе регистрации  информация, полученная  при помощи 
биометрических сканеров, преобразуется в циф\-ро\-вой шаб\-лон.


  
  На этапе сравнения предъявляемые биометрические данные сравниваются с 
шаблоном регистрации. Результатом сравнения биометрических %\linebreak
 данных 
является число, в том числе и для многофакторной биометрической системы. 
На рис.~\ref{f6ush2} пред\-став\-ле\-на схема простейшей, с алгоритмической точки 
зрения, многофакторной  биометрической системы~--- биометрический 
паспорт. 
  
  Отличительной упрощающей интеграцию чертой паспортной задачи (или 
систем гражданской идентификации в целом) является структурированный и 
контролируемый ввод биометрической информации. В шаблоне регистрации 
фиксируется стандартизированный набор биометрических измерений, 
например пара отпечатков пальцев и фотография лица. Соответственно в 
штатном режиме функционирования для верификации или идентификации 
личности предъявляется такой же ограниченный и полный набор данных. Такая 
схема оставляет минимальную свободу действий при интеграции. 

 
  Так как паспортные системы ориентированы на достижение минимальной 
ошибки 2-го рода, целесообразно использовать максимально возможную 
информацию о сравниваемых образцах, т.\,е.\ дожидаться ответа от всех 
используемых биометрических технологий. Данный подход позволяет 
максимально оптимизировать ошибки 1-го и 2-го рода за счет уменьшения 
производительности системы (рис.~7).
{ %\looseness=1

}
 
  Более сложным примером многофакторной биометрической идентификации 
являются криминалистические системы. В данном случае нет серь\-ез\-ных 
ограничений по времени. Главным критерием являются ошибки 1-го и 2-го 
родов. Однако в отличие от паспортной задачи в криминалистической системе 
не структурирован шаблон регистрации, в системе может храниться 
произвольное коли\-че\-ст\-во биометрических образцов, принадлежащих одному 
субъекту. Также криминалистическая система может использовать
произвольное число био\-мет\-ри\-че\-ских алгоритмов, использующих одну и ту же 
биометрию. В отличие от паспортной задачи, где замена алгоритмической 
начинки технологии максимально затруднена, криминалистические сис\-те\-мы 
при разумном проектировании позволяют в штатном порядке обновлять или 
добавлять био\-мет\-ри\-че\-ские алгоритмы.
  
  Соответственно, в криминалистической задаче могут быть реализованы все 
уровни интеграции~\mbox{[5--7]:}
  \begin{itemize}
\item на уровне модальностей (multimodal biometrics);
\item на уровне биометрических образцов (multisample);
\item на уровне алгоритмов (multialgorithm).
  \end{itemize}
  
  Интеграция на уровне образцов в данном случае предполагает 
систематические решения, улуч\-ша\-ющие качество распознавания при 
одновременном предъявлении нескольких образцов одного био\-мет\-ри\-че\-ско\-го 
измерения. На рис.~8 представлены графики ошибок распознавания 
в зависимости от чис\-ла предъявляемых для идентификации образцов одного 
отпечатка пальца.

\end{multicols}

\begin{figure*} %fig7 fig8
\vspace*{1pt}
\begin{center}
\mbox{%
\epsfxsize=166.3mm  
\epsfbox{us2-7-8.eps} }
\end{center}
\vspace*{-9pt}
\begin{minipage}[t]{82mm}
\Caption{
Теоретические ошибки 1-го и 2-го рода многофакторных 
биометрических систем на основе современных алгоритмов (2007):
\textit{1}~--- отпечаток пальца;
\textit{2}~--- два образца отпечатка пальца;
\textit{3}~--- отпечаток пальца\;+\;голос;
\textit{4}~--- отпечаток пальца\;+\;почерк;
\textit{5}~--- лицо\;+\;голос\;+\;почерк;
\textit{6}~--- лицо\;+\;отпечаток пальца
\label{f7ush2}}
\end{minipage}
\hfill
\begin{minipage}[t]{82mm} %fig8
\Caption{
Ошибки 1-го и 2-го рода распознавания по отпечаткам пальцев в 
зависимости от числа образцов (FVC2002 DB1):
\textit{1}~---  1~образец; 
\textit{2}~--- 2; \textit{3}~--- 3; \textit{4}~--- 4; \textit{5}~--- 5;
\textit{6}~--- 6;
\textit{7}~---  7~образцов
\label{f8ush2}}
\end{minipage}
\vspace*{6pt}
\end{figure*}

 \begin{figure*} %fig9
\vspace*{1pt}
\begin{center}
\mbox{%
\epsfxsize=143.132mm  
\epsfbox{us2-9.eps} }
\end{center}
\vspace*{-9pt}
  \Caption{ Схема биометрической интеграции в общем случае
  \label{f9ush2}}
  \end{figure*}
  
\begin{multicols}{2}  

  Схематично поток данных в многофакторной криминалистической системе 
представлен на рис.~\ref{f9ush2}. 
  
 
  В качестве математического обеспечения блока интеграции могут 
использоваться произвольные алгоритмы, позволяющие достичь необходимого 
уровня статистических показателей качества биометрической системы. Более 
подробно алгоритмические проблемы биометрической интеграции изложены 
в~[4, 15--18].
  
\vspace*{-6pt}
  \section{Программная архитектура}
  
  Принимая во внимания тенденцию к созданию крупномасштабных 
распределенных био\-мет\-ри\-че\-ских систем, следует изначально создавать 
масштабируемые и распределенные архитектуры. В~рам-\linebreak ках распространенной 
на сегодняшний день концепции сер\-ви\-со-ори\-ен\-ти\-ро\-ван\-ной 
  архитектуры~[19, 20] следует, в первую очередь, разделить внутреннюю 
логику биометрических приложений на элементарные сервисы. Учитывая опыт 
разработки высо\-ко\-про\-из\-во\-ди\-тель\-ных биометрических приложений~[5], 
функционально можно выделить сле\-ду\-ющие сервисы:
\begin{itemize}
\item вычислительные сервисы, отвечающие за выполнение функций 
биометрических библиотек, ядро системы;
\item сервисы бизнес-логики приложения;
\item сервисы хранилища;
\item клиентские приложения, <<тонкий>> клиент терминальных 
станций;
\item вспомогательные сервисы управления, мониторинга, 
диагностики;
\item сервисы сообщений/предоставления интерфейса, отвечающие 
за обмен информацией между узлами системы;
\item сервисы операционной системы, распределенная среда 
исполнения.
\end{itemize}
  
   
  На рис.~\ref{f10ush2} приведена схема взаимодействия перечисленных групп 
сервисов. С точки зрения аппаратных средств перечисленные сервисы могут 
исполняться как на одном вычислительном средстве, так и на отдельных 
специализированных серверах или кластерах. На рис.~\ref{f11ush2} приведен 
пример комплекса аппаратных средств системы гражданской идентификации.

  \end{multicols}
  
\begin{figure}[b] %fig10
\vspace*{1pt}
\begin{center}
\mbox{%
\epsfxsize=164.07mm  
\epsfbox{us2-10.eps} }
\end{center}
\vspace*{-9pt}
  \Caption{Программная архитектура 
  \label{f10ush2}}
  \end{figure}
  
    \begin{figure*} %fig11
\vspace*{1pt}
\begin{center}
\mbox{%
\epsfxsize=149.649mm  
\epsfbox{us2-11.eps} }
\end{center}
\vspace*{-9pt}
  \Caption{Комплекс аппаратных средств системы гражданской идентификации
  \label{f11ush2}}
  \vspace*{6pt}
  \end{figure*}

  \begin{multicols}{2}


  
  На рис.~\ref{f12ush2} приведен пример распределенной архитектуры 
системы интеллектуального видеонаблюдения~[21]. 
  
    \begin{figure*} %fig12
\vspace*{1pt}
\begin{center}
\mbox{%
\epsfxsize=143.44mm  
\epsfbox{us2-12.eps} }
\end{center}
\vspace*{-9pt}
  \Caption{Комплекс аппаратных средств системы видеонаблюдения
  \label{f12ush2}}
  \vspace*{6pt}
  \end{figure*}
    
  По организации вычислений и логике по\-стро\-ения информационных потоков 
приведенные примеры (рис.~11 и~12) 
отличаются только источником биометрической 
информации. Если следовать основным %\linebreak 
 отрас\-ле\-вым стандартам, созданным по 
результатам разработки bioAPI, то биометрические приложения на уровне 
прикладного программного интерфейса устроены одинаково независимо 
от выбранного метода биометрической идентификации. Несмотря на вполне 
стандартизированные вызовы и протоколы обмена данными внутри 
биометрических приложений, работа с терминальными устройствами может 
требовать специальной организации вычислительного процесса. Поэтому 
разделение терминальных устройств и основного вычислительного узла при 
помощи сервера сообщений, как это приведено на рис.~\ref{f11ush2} 
и~\ref{f12ush2}, является оправданным. Такой подход позволяет отделить 
работу с внешними устройствами, чей интерфейс зачастую не
стандартизирован, от основного вычислительного узла, который, в свою очередь, 
должен быть реализован с учетом требований российских и международных 
стан\-дартов. 
  
  Согласно bioAPI и российским стандартам по биометрии, реализация 
биометрической системы предполагает три типа данных:
  \begin{enumerate}[(1)]
\item исходные данные, полученные с терминальных устройств;
\item данные, обработанные функциями специализированных библиотек 
(фильтрация, шумоочистка и~т.\,д.);
\item цифровой шаблон/модель биометрического образца.
\end{enumerate}
  Функции биометрической системы:
  \begin{itemize}
\item регистрация, получение шаблона на основе исходных данных;
\item сравнение двух шаблонов, определение меры сходства; 
\item нестандартные функции дополнительной обработки 
биометрической информации.
\end{itemize}

  При дополнительной обработке данных помимо функций возможно более 
сложное взаимодействие с биометрическими библиотеками, реализованное в 
виде контролов или ActiveX компонентов.
  
  Соответственно функционал базовых сервисов, работающих в рамках 
мультибиометрического приложения, должен поддерживать перечисленные %\linebreak
\linebreak{типы} данных и вызовы функций. В случае биометрических приложений более 
сложной задачей является работа с перечисленными типами данных и их 
передача внутри системы. 
  
  Основой для протокола передачи исходных биометрических данных 
являются стандарты по обмену биометрической информации. Наиболее 
общими с точки зрения применимости к мультибиометрическим технологиям 
являются стандарты ITL-1-200x. Стандарты данной серии предполагают 
следующий состав необходимой информации:
  \begin{itemize}
\item заголовок;
\item сопроводительная текстовая информация;
\item биометрические образцы;
\item сопроводительная информация по каждому биометрическому 
образцу.
\end{itemize}

  \begin{table*}\small
  \begin{center}
  \Caption{Список биометрических технологий АМИС
  \label{t1ush2}}
  \vspace*{2ex}
  
  \begin{tabular}{|c|p{70mm}|c|}
  \hline
\multicolumn{1}{|c|}{Биометрия}&\multicolumn{1}{c|}{Основные источники}&\multicolumn{1}{|c|}{Точность идентификации}\\
\hline
Отпечаток пальца &
\vspace*{-12pt}
\begin{enumerate}[1.]
\item Данные обязательной дактилоскопической регистрации
\item Следы отпечатков пальцев
\item Системы контроля доступа
\item Биометрические паспорта нового поколения
\end{enumerate}
\vspace*{-14pt} &  Очень высокая\\
\hline
Форма лица &
\vspace*{-12pt}
\begin{enumerate}[1.]
\item Фотографии; видеозаписи
\item Камеры наружного наблюдения
\item Системы контроля доступа
\item Биометрические паспорта нового поколения \end{enumerate}
\vspace*{-14pt}  & Высокая\\
\hline
Голос  & 
\vspace*{-12pt}
\begin{enumerate}[1.]
\item Телефонные линии
\item Записи разговоров
\end{enumerate}
\vspace*{-14pt}
&  Низкая\\
\hline
Почерк & 
Произвольные рукописные образцы & Низкая\\
\hline
\end{tabular}
\end{center}
\vspace*{-3pt}
\end{table*}

  В настоящей редакции серия стандартов ITL 1-200x ориентирована, в первую 
очередь, на правоохранительную деятельность. Поэтому в данном формате 
данных стандартные контейнеры предназначены для хранения цифровых 
фотографий и дактилоскопического материала. Только в редакции 2007~г.\ 
данный стандарт пополнился контейнерами для геометрии руки и радужной 
оболочки глаза. Данное обстоятельство обусловлено тем, что на сегодняшний 
день разработаны следующие стандарты форматов обмена биометрической 
ин\-фор\-ма\-цией:
  \begin{itemize}
\item ANSI INCIST 377-2004, Information Technology~--- Finger Pattern 
Based Interchange Format;
\item ANSI INCIST 378-2004, Information Technology~--- Finger 
Minutiae Format for Interchange;
\item ANSI INCIST 379-2004, Information Technology~--- Iris Image 
Interchange Format;
\item ANSI INCIST 381-2004, Information Technology~--- Finger 
Image-Based Data Interchange Format;
\item ANSI INCIST 385-2004, Information Technology~--- Face 
Recognition Format for Data Interchange;
\item ANSI INCIST 395-2005, Information Technology~--- 
Signature/Sign Data Interchange Format;
\item ANSI INCIST 396-2005, Information Technology~--- Hand 
Geometry Interchange Format.
  \end{itemize}
  
 
  Как видно из списка, в стандартах представлены отпечатки пальцев, форма 
лица, геометрия руки, почерк/подпись и радужная оболочка глаза. При этом 
отсутствует стандарт на мультибиометрические данные. В такой ситуации 
целесообразным является развитие подхода ITL 1-200x. А именно: предлагается 
дополнить формат данных недостающими контейнерами для хранения данных, 
отличных от перечисленного списка стандартов. Для описания данных внутри 
контейнера можно использовать структуры CBEFF\footnote{CBEFF~--- Common Biometric Exchange File Format~---
единый формат представления биометрических данных.} 
и bioAPI. К настоящему 
времени в bioAPI зарегистрировано более 20~методов (против 
5~стандартизированных) биометрической идентификации. 


  Для работы с обработанными данными и цифровыми биометрическими 
шаблонами целесообразно использовать схожую структуру CBEFF. Однако 
здесь следует отметить, что представленные в таком виде цифровые шаблоны 
слабо пригодны для обмена со сторонними автоматизированными информационными системами по причине отсутствия 
стандартов по форматам биометрических цифровых шаблонов для всех 
биометрик, кроме отпечатков пальцев.

\vspace*{-8pt}
  
  \section{Программная реализация}
  
  В данном разделе приведены результаты разра\-бот\-ки программного 
комплекса, реализующего изложенную в предыдущих разделах концепцию %\linebreak 
мультибиометрической идентификации (Auto\-matic Multibiometric Identification 
System, AMIS или АМИС).


  \begin{figure*} %fig13
\vspace*{1pt}
\begin{center}
\mbox{%
\epsfxsize=163.817mm  
\epsfbox{us2-13.eps} }
\end{center}
\vspace*{-6pt}
  \Caption{Программная архитектура АМИС
  \label{f13ush2}}
%  \vspace*{6pt}
  \end{figure*}
  
    В ходе работ по созданию АМИС было решено интегрировать основные 
систематически доступные биометрики, а именно: отпечатки пальцев, 
цифровое изображение лица, голос и образцы рукописного почерка. В 
табл.~\ref{t1ush2} представлен список биометрических технологий АМИС и 
потенциальные источники их получения.
  
  
  Функционально АМИС предназначена для решения следующих задач:
  \begin{itemize}
\item импорт, экспорт и хранение биометрических образцов;\\[-13pt]
\item учет и обработка биометрических образцов;\\[-13pt]
\item оперативная обработка запросов на биометрическую 
идентификацию.\\[-13pt]
\end{itemize}
  
  С точки зрения реальных сценариев эксплуатации АМИС перечисленные 
задачи решаются различ\-ны\-ми группами пользователей. Учет и первичная 
обработка биометрической информации %\linebreak
требует специальной квалификации и 
дополнительного аппаратного обеспечения, что приводит к необходимости 
выделения программного обеспечения отдельных автоматизированных рабочих 
мест (ПО АРМ) биометриста. 
  
  Подготовка запроса на биометрическую иден\-ти\-фи\-ка\-цию в большинстве 
случаев не связана со %\linebreak
специ\-фи\-че\-ской обработкой биометрической информации 
и может быть выполнена обычными пользователями. В частности, это 
позволяет использовать сервисы АМИС посредством web-ин\-тер\-фей\-са.
  
  При разработке аппаратной архитектуры также приходилось учитывать, что 
биометрическая идентификация является трудоемкой процедурой для 
вычислительных средств общего назначения. Например, запрос на 
идентификацию голоса требует передачи больших объемов информации и 
может обрабатываться значительное время, измеряемое десятками минут или 
даже часами при значительном размере базы дикторов. Такая низкая 
производительность требует использования для выполнения вычислений 
максимально производительных и бесперебойно работающих ресурсов. 
  
  Для эффективного решения поставленных задач в состав комплекса 
необходимо включить высокопроизводительный сервер и специально 
оборудованные АРМ биометриста, оснащенные специализированным 
аппаратным обеспечением. С учетом вышеперечисленного АМИС 
разрабатывалась как клиент-серверное приложение (рис.~\ref{f13ush2}). 
  
  Учитывая функциональное назначение компонентов, мы предлагаем 
следующее базовое распределение сервисов и вычислительной нагрузки \mbox{между} 
клиентским и серверным компонентами %\linebreak 
сис\-темы. 

В состав клиентского 
компонента АМИС включены:
  \begin{itemize}
\item интерфейс пользователя;\\[-13pt]
\item модуль управления запросами пользователя;\\[-13pt]
\item биометрические сервисы;\\[-13pt]
\item интерфейс администратора.\\[-13pt]
\end{itemize}

 
  В состав серверного компонента АМИС включены:
  \begin{itemize}
\item сервис сообщений;\\[-13pt]
\item сервис хранилища;\\[-13pt]
\item биометрические сервисы.\\[-13pt]
  \end{itemize}
  

  \begin{table*}\small
  \begin{center}
  \Caption{Характеристики биометрических шаблонов 
  \label{t2ush2}}
  \vspace*{2ex}
  
  \begin{tabular}{|l|c|c|c|c|}
  \hline
\multicolumn{1}{|c|}{Технология}&
\multicolumn{1}{c|}{\tabcolsep=0pt\begin{tabular}{c}Среднее\\ время\\ создания\\ шаблона, мс\end{tabular}} 
&\multicolumn{1}{c|}{\tabcolsep=0pt\begin{tabular}{c}Средний\\ размер\\ исходных \\ данных, КБ\end{tabular}} &
\multicolumn{1}{c|}{\tabcolsep=0pt\begin{tabular}{c}Средний\\ размер\\ шаблона, Б\end{tabular}}&
\multicolumn{1}{c|}{\tabcolsep=0pt\begin{tabular}{c}Степень\\ сжатия\\ информации\end{tabular}}\\
  \hline
  Отпечатки пальцев. Десятипальцевая дактокарта&2349 
  &700  &26170 &27\\
%  \hline
  Отпечатки пальцев. След&\hphantom{9}363 &\hphantom{9}35 &\hphantom{9}2617 &\hphantom{5,}13,5\\
 % \hline
  Изображение лица&\hphantom{9}595 &\hphantom{9}50 &\hphantom{9}2933 &17\\
  %\hline
  Голос&2933 &2000\hphantom{9} &54370 &40\\
%  \hline
  Почерк&3290 &780 &20528 &38\\
  \hline
  \end{tabular}
  \end{center}
%  \vspace*{-9pt}
  \end{table*}
  
  Конечному пользователю, оператору АМИС, основной функционал 
предоставляется посредством или Интранет-сервера АМИС (технология 
<<тонкого клиента>>), или специализированного ПО терминальных станций 
(<<толстый клиент>>). Пользовательские функции АМИС можно представить 
в виде комбинации следующих обращений:
\begin{itemize}
\item Enroll, регистрация объекта учета в хранилище;\\[-13pt]
\item Modify, изменение информации об объекте %\linebreak
\mbox{учета};\\[-13pt]
\item Identify, биометрическая идентификация;\\[-13pt]
\item Select, запрос на выборку данных из храни\-лища.\\[-13pt]  
  \end{itemize}
  
  Перечисленные  обращения схематично представлены на рис.~14.
    \medskip

%\begin{figure*} %fig14
\vspace*{1pt}
\begin{center}
\mbox{%
\epsfxsize=75mm   %.801
\epsfbox{us2-14.eps} }
\end{center}
\vspace*{2pt}
%\Caption{
{{\figurename~14}\ \ \small{Обмен данными между интерфейсом пользователя и сервисом 
сообщений АМИС}}
%\label{f1s}}
%\end{figure*}

\addtocounter{figure}{1}    
 
 
%    \medskip

%\begin{figure*} %fig15
\vspace*{1pt}
\begin{center}
\mbox{%
\epsfxsize=79.801mm  
\epsfbox{us2-15.eps} }
\end{center}
\vspace*{3pt}
%\Caption{
{{\figurename~15}\ \ \small{Обмен данными между интерфейсом пользователя и 
биометрическими библиотеками клиентской части АМИС}}
%\label{f1s}}
%\end{figure*}
\bigskip
\medskip
\addtocounter{figure}{1}  

  Для большинства биометрических библиотек цифровой шаблон, полученный 
при вызове функции Enroll биометрической библиотеки, по размеру 
значительно компактней исходных необработанных данных. Сравнительный 
размер шаблонов и данных представлен в табл.~\ref{t2ush2}. В частности, из 
таблицы видно, что хранение шаблонов предпочтительнее.
  

  
  Для снижения нагрузки на сеть при передаче данных возможно частичное 
перераспределение нагрузки между клиентским АРМ и сервером. Поскольку 
операции, выполняемые при идентификации, требуют обращения к 
биометрическому хранилищу, которое является общим ресурсом для всех 
пользователей АМИС, целесообразным является перераспределение 
вычислений на клиентское вычислительное средство только на этапе 
подготовки данных и формирования шаблона. Схема обращений интерфейса 
пользователя к биометрическим библиотекам представлена на 
рис.~15.

 
  Приведенная реализация клиент-серверного приложения потенциально 
является независимой от операционной системы. Однако, рассматривая 
реализацию серверного компонента, приходится учитывать специфику 
общесистемной части. АМИС была разработана на базе следующего 
общесистемного ПО производства Microsoft:
  \begin{itemize}
  \item Операционная система~--- MS Windows Server 2003/ XP; 
  \item СУБД~--- MS SQL Server 2005;
\item Транспорт~--- Windows Communication Foundation (WMC) из 
состава комплекта библиотек .NET Framework~3.0.
  \end{itemize}
  

  \begin{figure*} %[b] %fig16
\vspace*{1pt}
\begin{center}
\mbox{%
\epsfxsize=165.842mm  
\epsfbox{us2-16.eps} }
\end{center}
\vspace*{-6pt}
  \Caption{Обмен данными между сервисом сообщений  и (\textit{а})~сервисом хранилища 
АМИС; (\textit{б})~вычислительным сервисом АМИС
  \label{f16ush2}}
  \vspace*{6pt}
  \end{figure*}

  
  Данная конфигурация ПО позволяет пол\-ностью переложить трудоемкие в 
реализации механизмы информационной безопасности и разграничения 
доступа на операционную сис\-те\-му. А именно: встро\-ен\-ные механизмы СУБД и 
транспортной\linebreak сис\-те\-мы позволяют перейти на сквозную идентификацию 
посредством инструментов Windows\linebreak Au\-thenti\-cation. Соответственно, 
разграничение полномочий по использованию общих ресурсов осуществляется 
административными средствами операционной системы.  
{ %\looseness=1

}

  
  Внутри серверного компонента управление потоками данных 
осуществляет сервис сообщений, который обращается к общим ресурсам 
(хранилищу) и вычислительным библиотекам (рис.~\ref{f16ush2}). 

 
  Ядром алгоритмического обеспечения АМИС являются биометрические 
библиотеки, входящие в состав вычислительного сервиса АМИС.  В состав 
биометрической составляющей АМИС входят модули однофакторной 
биометрической идентификации, модули экспертной (первичной) обработки, 
модуль анализа видеоряда и модуль комплексной обработки результатов 
биометрической идентификации. Обращения к модулям однофакторной 
биометрической идентификации (рис.~17) стандартизированы 
ГОСТ~9784.


  
  Модуль комплексной биометрической идентификации вычислительного 
сервиса АМИС, в котором производится оценка меры сходства наборов 
биометрических образцов по результатам идентификации по отдельным 
признакам, реализуется согласно общей схеме рис.~\ref{f9ush2}. Внутри могут 
быть использованы произвольные алгоритмы, обеспечивающие требуемые 
ошибки 1-го и 2-го рода. В~частности, в~[4, 15--18] приведен широкий класс\linebreak 
алгоритмов биометрической интеграции, предназначенные для оптимизации 
ошибок распознавания.

  \medskip

%\begin{figure*} %fig17
\vspace*{1pt}
\begin{center}
\mbox{%
\epsfxsize=79.801mm  
\epsfbox{us2-18.eps} }
\end{center}
\vspace*{3pt}
%\Caption{
{{\figurename~17}\ \ \small{Обмен данными между вычислительным сервером и 
биометрическими библиотеками серверной части АМИС}}
%\label{f1s}}
%\end{figure*}
\bigskip
\addtocounter{figure}{1}  
  
  \section{Заключение}
  
  В статье предложен новый подход к созданию мультибиометрических 
технологий на базе сервисно-ориентированной архитектуры. Ранее 
преобладали принципы разработки биометрических технологий в рамках 
единого приложения. В качестве основных преимуществ выработанного 
подхода отметим:
  \begin{itemize}
\item программная архитектура позволяет разделить общесистемную 
часть  и работу с конкретными биометрическими библиотеками, 
реализующими методы биометрической идентификации;
\item реализация общесистемной части независима от состава 
биометрических библиотек;
\item разработанная архитектура позволяет модернизировать и заменять 
отдельные биометрические библиотеки;
\item программная архитектура обеспечивает комплексную 
идентификация по нескольким биометрическим признакам 
одновременно;
\item программная архитектура обеспечивает воз\-мож\-ность 
распределения основных вы\-чис\-литель\-ных узлов, хранилища данных и 
интерфейса пользователей между различными %\linebreak
вычис\-ли\-тель\-ны\-ми  средствами;
\item полное соответствие требованиям основных международных и 
российских стандартов к биометрическим системам и обмену 
биометрической информации;
\item защита информационных ресурсов осуществляется механизмами 
операционной сис\-темы.
\end{itemize}

  Направлениями для дальнейшего развития разработанного подхода с целью 
обеспечения большей универсальности, увеличения производительности и 
надежности являются:
  \begin{itemize}
\item обеспечение масштабируемости;
\item обеспечение отказоустойчивости;
\item реализация механизмов распараллеливания вычислений на 
биометрических узлах.
\end{itemize}
  
{\small\frenchspacing
{%\baselineskip=10.8pt
\addcontentsline{toc}{section}{Литература}
\begin{thebibliography}{99}
  
\bibitem{7ush2} %1
\Au{Beardsley, Ch.\,T.}
Is your computer insecure?~// IEEE Spectrum, 1972. P.~67--78.

\bibitem{6ush2} %2
\Au{Woodward~J.\,D., Jr.} 
Biometrics: Facing up to terrorism~// The Biometric Consortium Conference 
2002.  Arlington, February, 2002.

\bibitem{1ush2} %3
\Au{Wayman~J., Jain A., Maltoni~D., Maio~D.} 
Biometric systems: Technology, design and performance evaluation.~--- 
Springer Verlag, 2004.

\bibitem{2ush2} %4
\Au{Синицын И.\,Н., Новиков С.\,О., Ушмаев~О.\,С.}
Развитие технологий интеграции биометрической информации~// 
Системы и средства информатики, 2004. Вып.~14. С.~5--36.

\bibitem{3ush2}%5
\Au{Ушмаев~О.\,С., Синицын~И.\,Н.}
Опыт проектирования многофакторных биометрических систем~// Тр.\ 
VIII международной научно-технической конференции <<Кибернетика и 
высокие технологии XXI~века>>, 2007.  Т.~1. С.~17--28.

\bibitem{5ush2} %6
\Au{Ушмаев~О.\,С.}
Реализации концепции многофакторной биометрической идентификации 
в правоохранительных системах. Ин\-тер\-по\-ли\-тех-2007.  
{\sf  
http://www.dancom.ru/rus/AIA/Archive/RUVI\_ BioLinkSolutions\_MultimodalBiometricsConcept.pdf}.

\bibitem{4ush2} %7
\Au{Ушмаев~О.\,С., Босов~А.\,В.} 
Реализация концепции многофакторной биометрической идентификации 
в интегрированных аналитических системах~// Бизнес и безопасность в 
России,  2008. №\,49. С.~104--105.


\bibitem{8ush2} %8
First International Competition for Fingerprint Verification Algorithms 
(FVC2000). {\sf  http://bias.csr.unibo.it/ fvc2000/}.

\bibitem{13ush2} %9
\Au{Mansfield~T., Kelly~G., Chandler~D.,~Kane~J.}
Biometric product testing final report. U.K.\ Biometrics Working Group, 2001. 
{\sf 
http://www.cesg.gov.uk/site/ast/ \mbox{biometrics/media/BiometricTestReportpt1.pdf}}.

\bibitem{9ush2} %10
FVC2002, the Second International Competition for Fingerprint Verification 
Algorithms (FVC2002). {\sf http:// bias.csr.unibo.it/fvc2002/}.

\bibitem{14ush2} %11
\Au{Mansfield~A.\,J., Wayman~J.\,L.}
Best practices in testing and reporting performance of biometric devices. U.K. 
Biometrics Working Group, 2002. 

\bibitem{10ush2} %12
Studies of Fingerprint Matching Using the NIST Verification Test Bed (VTB).  
{\sf  ftp://sequoyah.nist.gov/pub/ nist\_internal\_reports/ir\_7020.pdf}.

\bibitem{11ush2} %13
Face Recognition Vendor Test. {\sf  http://www.frvt.org}.

\bibitem{12ush2} %14
Fingerprint Vendors Technology Evaluation.  {\sf  http:// fpvte.nist.gov}.


\bibitem{15ush2}
\Au{Ushmaev~O.\,S., Novikov~S.\,O.} 
Integral criteria for large-scale multiple fingerprint solutions~// Biometric 
Technology for Human Identification~/ Ed. A.\,K.~Jain, N.\,K.~Ratha. 
Proceedings of SPIE. Berlingham, WA: SPIE, 2004. Vol.~5404. P.~534--543.

\bibitem{17ush2} %16
\Au{Пугачев~В.\,С., Синицын~И.\,Н.}
Теория стохастических систем.~--- М.: Логос, 2004. [Stochastic systems: 
Theory and applications.~--- World Scientific. Singapore, 2001].

\bibitem{18ush2} %17
\Au{Ушмаев~О.\,С., Новиков~С.\,О., Синицын~И.\,Н.}
Стохастические проблемы интегрированной обработки информации~// 
Проблемы и методы информатики. II Научная сессия Института проблем 
информатики РАН, Москва, 18--22~апреля 2005. Тезисы докладов.~--- М.: 
ИПИ РАН, 2005. С.~77--79.

\bibitem{16ush2} %18
\Au{Ushmaev~O., Novikov~S.}
Biometric fusion: Robust approach~// MMUA06 Proceedings, 2006. Toulose, 
France.


\bibitem{19ush2}
Reference Model for Service-Oriented Architecture 1.0.
{\sf http://www.oasis-open.org/committees/ \mbox{download.php/19679/soa-rm-cs.pdf}}.

\bibitem{20ush2}
\Au{Cherbakov~L., Galambos~G., Harishankar~R., Kalyana~S., 
Rackham~G.}
 Impact of service orientation at the business level~// IBM Systems J. SOA, 
2005. Vol.~44. No.\,4. P.~653--668.

\label{end\stat}

\bibitem{21ush2}
\Au{Ушмаев~О.\,С., Синицын~И.\,Н.} 
Информационные технологии распознавания лиц в потоке~// 8-я 
международная научно-техническая конференция 
<<Распознавание-2008>>. 13--15~мая~2008~г. В печати.

\end{thebibliography}

}
}
\end{multicols}



%SERVICE-ORIENTED APPROACH TO MULTIMODAL BIOMETRICS 
%DESIGNING
%
%O.S. Ushmaev, oushmaev@ipiran.ru
%(Institute for Informatics Problems, Russian Academy of Sciences)
%
%Abstract. Novadays, multimodal biometrics is rapidly replacing tedious procedures of identification. 
%Particularly operating and perspective civil ID systems use multimodal approach. The formal method 
%for designing high-speed multibiometric technologies and systems in the article
%is suggested. The effectiveness 
%of the approach by an example of developed experimental software with service-oriented architecture
%is shown.   

%Key words. biometric identification; multimodal biometrics; platform independent; service-oriented 
%architecture 

 
 
 
 