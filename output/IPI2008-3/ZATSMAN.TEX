\def\stat{zats}


\def\tit{
ЗАДАЧИ ПРЕДСТАВЛЕНИЯ ЛИЧНОСТНЫХ И КОЛЛЕКТИВНЫХ
КОНЦЕПТОВ В ЦИФРОВОЙ СРЕДЕ}
\def\titkol{
Задачи представления личностных и коллективных
концептов в цифровой среде}

\def\autkol{И.\,М.~Зацман, В.\,В.~Косарик, О.\,А.~Курчавова}
\def\aut{И.\,М.~Зацман$^1$, В.\,В.~Косарик$^2$, О.\,А.~Курчавова$^3$}

\titel{\tit}{\aut}{\autkol}{\titkol}

%{\renewcommand{\thefootnote}{\fnsymbol{footnote}}\footnotetext[1]
%{Работа выполнена при
%поддержке РФФИ, гранты 08-01-00345, 08-01-00363,
%08-07-00152.}}

\renewcommand{\thefootnote}{\arabic{footnote}}
\footnotetext[1]{Институт проблем информатики Российской академии наук, im@a170.ipi.ac.ru}
\footnotetext[2]{Институт проблем информатики Российской академии наук, valery@a170.ipi.ac.ru}
\footnotetext[3]{Институт проблем информатики Российской академии наук, koa@a170.ipi.ac.ru}

\Abst{В статье анализируются ключевые положения документов 7-й Рамочной программы
Европейского Союза, принятой на период 2007--2013~гг., содержащие формулировки ряда новых
направлений и задач, относящихся к проблематике представления знаний в информационных
системах долговременного использования. Результаты анализа этих документов позволяют
утверждать, что одновременно с традиционной проблемой информатики, охватывающей вопросы
представления в цифровой среде уже имеющихся знаний, ориентированных на удовлетворение
технологических, экономических, образовательных и других социально значимых потребностей
общества, их хранения и извлечения, становится актуальной задача направляемой генерации
новых целевых систем знаний в тех случаях, когда име\-ющие\-ся системы знаний не удовлетворяют
этим потребностям и с этой точки зрения являются неполными. Предложен новый подход к
отражению в цифровой среде личностных и коллективных концептов, а также стадий их
эволюции в контексте генерации целевых систем знаний.}

\KW{личностные, коллективные и конвенциональные концепты; стабильные и нестабильные
концепты; представление личностных и коллективных концептов в цифровой среде}

      \vskip 24pt plus 9pt minus 6pt

      \thispagestyle{headings}

      \begin{multicols}{2}

      \label{st\stat}


\section{Введение}

   Согласно Ю.\,А.~Шрейдеру, специфические проблемы информатики возникают там, где
встают задачи информационного представления знаний в форме, удобной для обработки,
передачи и творческого реконструирования знаний в результате %\linebreak 
усилий пользователя~[1, с.~51]. При этом ин\-фор\-мати\-ка рассматривалась им 
согласно парадигме Горна~\cite{2za, 3za} как информационно-компьютерная наука. 
В настоящее время наряду с традиционной для информатики проблемой представления 
в цифровой среде уже имеющихся знаний, их хранения и извлечения становится 
актуальной задача на\-прав\-ля\-емой генерации новых целевых систем знаний в 
тех случаях, когда имеющиеся системы не удовлетворяют технологическим, 
экономическим, образовательным и другим социально значимым потребностям и с 
этой точки зрения являются неполными. {\looseness=1

}

   В процессе описания традиционной проблемы представления в цифровой среде уже
имеющихся знаний явно или неявно присутствует положительный или отрицательный ответ
на следующий основополагающий для информатики и других областей знаний вопрос:
<<Является ли система знаний субъекта (субъектов) и порождаемые им концепты в разные
моменты времени самотождественными?>> В~случае отрицательного ответа формулируемая
проблема  является нестационарной, а в случае положительного~--- стационарной.

   Если задавать этот вопрос в рамках других областей знаний, то, согласно
М.\,К.~Мамардашвили, в классических основаниях точных и естественных наук заложен, как
правило, положительный ответ на этот вопрос. Положительный ответ, который может
приниматься в явной или неявной форме, позволяет не учитывать в проблематике
пред\-став\-ле\-ния знаний точных и естественных наук эволюцию знаний субъектов в процессе
их познавательной деятельности. В основаниях гуманитарных и социальных наук этот ответ
часто бывает отрицательным, так как в широком спектре проблем этих областей знаний
требуется учитывать эволюцию знаний субъектов~\cite{4za}.

   В работе~\cite{5za} система базовых терминов и семиотические основания информатики
как ин\-фор\-ма\-ционно-компьютерной науки сформулированы таким образом, чтобы охватить
оба возможных варианта ответа в предметной области представления знаний. Поэтому в
процессе постановки каждой задачи представления знаний в рамках предложенного подхода
к описанию семиотических оснований информатики имеется возможность явного или
неявного выбора положительного или отрицательного ответа, исходя из специфики задачи.

   Что касается проблемы направляемой гене\-рации новых целевых систем знаний, то в
любом %\linebreak 
варианте ее описания явно или неявно будет присутствовать отрицательный ответ 
на сформулированный вопрос, так как само название этой проблемы говорит о том, 
что в ее постановке и методах %\linebreak 
реше\-ния необходимо учитывать эволюцию 
системы знаний субъектов и порождаемых ими концептов во времени, а также 
эволюцию во времени форм представления концептов в среде социальных 
коммуникаций и в цифровой среде\footnote{Цифровая среда определяется в статье 
как сочетание элементов цифровой вычислительной техники, средств 
телекоммуникации, информационно-компьютерных систем, иных цифровых средств 
ввода, хранения, поиска, передачи и других процессов обработки данных.}. Иначе 
говоря, проблема направляемой генерации новых целевых систем знаний является 
нестационарной.
{ %\looseness=1

}

   В статье будут приведены два примера  на\-прав\-ляемой генерации целевых систем знаний,
которые иллюстрируют необходимость учета фактора нестационарности. Первый пример~---
это формирование и ведение патентной классификации с учетом эволюции во времени ее
семантики и смыс\-ло\-во\-го содержания терминов, используемых в патентных рубриках.
Второй пример~--- это информационный мониторинг, анализ и индикаторное оценивание
научной деятельности с учетом эволюции во времени смыслового содержания индикаторов,
используемых в процессе оценивания.

   В настоящее время обозначилась потребность в разработке информационных систем
долговременного использования, в процессе функционирования которых необходимо
учитывать и фиксировать результаты разных стадий генерации и использования целевых
систем знаний. Предполагается, что в информационных системах на разных стадиях их
генерации могут находить свое отражение личностные и коллективные знания (личностные
и коллективные концепты) создателей целевых систем знаний.

   Здесь возникает естественный вопрос: чем личностные и коллективные знания и
концепты отличаются от конвенциональных (общепринятых) знаний и, соответственно, от
конвенциональных концептов? Как сформулировать эти отличия в такой конструктивной
форме, которая позволила бы разработчикам информационных систем зафиксировать и
использовать эти признаки с целью различения разных категорий концептов?

   Если отвлечься от задачи формулировки в конструктивной форме признаков отличий
личностных и коллективных знаний и концептов от конвенциональных и обратиться к
примерам описания самого феномена личностного знания, то такое описание можно найти,
например, в монографии М.~Полани. В его теории неявного знания (tacit knowledge) ряд
положений посвящен личностному означиванию, т.\,е.\ формированию личностных значений
(концептов). В них говорится: <<\ldots личностное значение является самообосновывающим,
если только признается его личностный характер. Оно как бы выдает <<вексель>> на
определенные условия артикуляции\footnote[2]{Артикуляция здесь трактуется как членение системы
знаний на концепты.}, которые обязаны стать явными в ходе наших рефлексивных
размышлений о самом этом процессе <<субсидирования>> доверия>>~\cite{6za}.

   Теория неявного знания сама по себе не \mbox{дает} %\linebreak 
   разра\-бот\-чи\-кам информационных систем
четких признаков отличий личностных и коллективных концептов от конвенциональных, но
ее можно попытаться использовать как отправную точку для формулировки этих признаков в
конструктивной форме. 
В статье предпринята попытка разделить концепты на три
категории~--- личностные, кол\-лек\-тив\-ные и конвенциональные, сформулировав их отличия
на основе ряда признаков, и \mbox{описать} терми\-но\-ло\-ги\-че\-ски предлагаемую ка\-те\-го\-ри\-за\-цию
концептов. Такое терминологическое описание предпринято в интересах создания тех %\linebreak
инфор\-ма\-ци\-он\-ных систем долговременного использования, в задачи которых входит
представление концептов всех трех категорий и целевых сис\-тем знаний в циф\-ро\-вой среде, а
также отражение их эволюции во времени.

   Рассматриваются только те информационные системы долговременного использования,
которые создаются в интересах обеспечения функционирования тех или иных
институциональных систем. Термин \textit{институциональная сис\-те\-ма} трактуется в
соответствии с монографией Г.\,Б.~Клейнера как совокупность взаимосвязанных
институтов\footnote[3]{Институты Г.\,Б.~Клейнер определяет как относительно устойчивые по отношению
к изменению поведения или интересов отдельных субъектов и их групп, а также продолжающие действовать
в течение значимого периода времени формальные и неформальные нормы либо системы норм, регулирующие
принятие решений, деятельность и взаимодействие социально-экономических субъектов (физических и
юридических лиц, организаций) и их групп~[7, с.~19].}~[7, с.~7]. Одной из задач подобных
информационных систем часто является представление в цифровой среде целевых систем
знаний, формируемых в интересах обеспечения функционирования соответствующих
институциональных систем.

   Например, в интересах патентной сферы как институциональной системы была
сформирована и ведется Международная патентная классификация (МПК) как целевая и
эволюционирующая сис\-те\-ма знаний. При этом патентные электронные библиотеки и другие
информационные системы, обеспечивающие представление и хранение МПК в цифровой
среде, являются ключевыми компонентами патентной инфраструктуры, так как в
соответствии с действующими нормативными актами любому решению о принятии или
отклонении патентной заявки должен предшествовать патентный поиск по ее тематике, в
процессе которого используется МПК.

   Другим примером институциональной системы может служить сфера академических
исследований, выполняемых государственными академиями наук в рамках единой
Программы фундаментальных научных исследований в соответствии с Федеральным
законом от 23~августа 1996~г.\ №\,127-ФЗ <<О науке и государственной
   научно-технической политике>> (в редакции Федерального закона от 04.12.2006
   №\,202-ФЗ). Ключевым субъектом этой институциональной системы является
Координационный совет, возглавляемый президентом РАН, а клю\-чевым компонентом
инфраструктуры этой сферы~--- Информационно-технологическая система мониторинга
Программы, разрабатываемая в настоящее время специалистами РАН. Одной из задач этой
информационной системы является пред\-став\-ле\-ние экспертных знаний о новых индикаторах
и процессе индикаторного оценивания результатов и эффективности выполнения
Программы, в том числе представление в цифровой среде личностных и коллективных
концептов в рамках целенаправленно формируемой системы экспертных знаний.
{\looseness=1

}

   Если обратиться к зарубежному опыту, то проблема генерации целевых систем знаний,
пред\-став\-ле\-ния личностных и коллективных концептов в цифровой среде нашла свое
отражение в документах 7-й Рамочной программы Европейского Союза. Эти документы
содержат формулировки новых направлений и задач, относящихся к проблематике
генерации, представления и эволюции систем знаний, и будут проанализированы в статье.

   Задачи представления личностных и коллективных концептов рассматриваются в
контексте %\linebreak 
создания информационных систем, являющихся ключевыми компонентами
инфраструктуры, обеспечивающей функционирование институциональных сис\-тем. Такие
информационные системы %\linebreak 
\mbox{будем} называть институциональными информационными
системами (ИИС). Институциональные инфор\-ма\-ци\-он\-ные
сис\-те\-мы долговременного использования имеют характерные черты,
предъявляющие принципиально новые требования к их раз\-ра\-ботке.
{\looseness=1

}

   Во-первых, период хранения документов в таких ИИС может многократно превышать
цикл жизни одного поколения аппаратно-программных средств их создания и поддержки.
Например, выполнение Программы фундаментальных научных исследований
государственных академий наук было начато в 2008~г., и она будет регулярно обновляться
не реже, чем один раз в 5~лет. За время действия этой долгосрочной программы произойдет
смена нескольких поколений аппаратно-программных средств, что необходимо учитывать
при проектировании Информационно-технологической системы мониторинга этой
программы.

   Во-вторых, во время длительного хранения документов могут существенно
эволюционировать классификационные системы, тезаурусы и онтологии, используемые в
процессах генерации, представления и сохранения систем знаний в ИИС. При этом в
процессе их изменения иногда необходимо различать эволюцию \textit{личностных,
коллективных} и конвенциональных концептов, проводя различие между
\textit{стабильными} (\textit{lasting concepts}) и \textit{нестабильными концептами}
(\textit{volatile concepts}) в системах знаний ИИС.

   Одна из новых задач, относящихся к проблематике представления знаний, связана
именно с необходимостью категоризации концептов в ИИС долговременного использования. В
данной статье задача категоризации решается на основе определения выделенных выше
курсивом словосочетаний как терминов, дополняющих систему базовых терминов из
работы~\cite{5za}. Основная цель расширения этой системы терминов заключается в том,
чтобы в явном виде описать признаки и нормативные аспекты, позволяющие разработчикам
ИИС зафиксировать и использовать отличия личностных, коллективных и
конвенциональных концептов.

\section{Эволюционирующие системы знаний}

   Наглядным примером эволюционирующей сис\-те\-мы технических знаний может служить
МПК. Перечислим все ее редакции с указанием
даты введения в действие и отметим момент разделения всей совокупности рубрик МПК на
базовый и расширенный уровни:
   \begin{itemize}
\item первая редакция МПК~--- 01.09.1968;
\item вторая редакция МПК~--- 01.07.1974;
\item третья редакция МПК~--- 01. 01.1980;
\item четвертая редакция МПК~--- 01.01.1985;
\item пятая редакция МПК~---  01.01.1990;
\item шестая редакция МПК~---  01.01.1995;
\item седьмая редакция МПК~---  01.01.2000;
\item восьмая редакция МПК~---  01.01.2006 (базовый и расширенный уровни);
\item девятая редакция МПК~---  01.01.2009.
\end{itemize}

   Приведенный перечень свидетельствует о том, что с 1968 по 2006~г.\ изменения
вносились один раз в 5--6~лет, а разделение на базовый и расширенный уровни
отсутствовало. Начиная с 2006~г. в патентной сфере вступила в силу очередная (восьмая)
редакция МПК (далее по тексту~--- МПК-8). В процессе подготовки этой редакции МПК как
основного средства классифицирования патентных документов были внесены существенные
изменения и в ее структуру, и в регламент ее ведения~\cite{8za}.

   В отличие от предыдущих 7~редакций, МПК-8 была разделена на рубрики базового и
расширенного уровня. Рубрики базового уровня, вы\-ра\-жа\-ющие стабильные концепты, могут
пересматриваться не чаще, чем раз в три года, а рубрики расширенного уровня могут
пересматриваться значительно чаще. Начиная с января 2007~г. вновь вводимые или
изменяемые рубрики расширенного уровня подготавливаются Специальным подкомитетом
Всемирной организации интеллектуальной собственности (ВОИС) по пересмотру
расширенного уровня МПК с регулярностью один раз в квартал и передаются на
утверждение в Международное бюро ВОИС. Оно за три месяца до ввода в действие доводит
их до сведения патентных ведомств, с тем чтобы последние могли принять обеспечительные
меры (перевод новых рубрик и подготовку их к публикации, ознакомление экспертов, начало
их простановки на публикуемых документах и~т.\,п.). Важно отметить, что для повышения
эффективности поиска, наряду с обязательным рубрицированием изобретений в целом,
МПК-8 может использоваться и для кодирования отдельных содержательных аспектов
описаний изобретений~\cite{9za}.

   Рубрики расширенного уровня на этапе рас\-смот\-ре\-ния Специальным подкомитетом ВОИС
и согласования их содержания выражают личностные и/или коллективные концепты
участников обсуж\-де\-ния. После принятия согласованного решения и его утверждения
Международным бюро ВОИС, ознакомления экспертов со вновь вводимыми или
измененными рубриками, начала их использования в процессе рубрицирования и
кодирования документов они приобретают конвенциональный характер в пределах
патентной сферы как институциональной системы и, следовательно, в рамках патентных
информационных систем.

   При простановке рубрик расширенного уровня на публикуемых документах
одновременно указывается дата (год и месяц), когда данная рубрика была введена в
действие. По мере не\-обхо\-ди\-мости рубрики расширенного уровня, выражающие
нестабильные концепты, могут пересматриваться чаще, чем раз в три года, пока им не будет
присвоен статус рубрик базового уровня.

   Например, с января 2007~г.\ была введена рубрика расширенного уровня с кодом
A62D~3/00 под названием <<Способы обезвреживания или уменьшения вредности
химических отравляющих веществ путем их химического изменения>>, а в рамках этой
рубрики с января 2007~г.\ был определен термин \textit{вредные химические вещества} как
вещества химических отходов, которые являются опасными или токсичными для сброса их в
обычные муниципальные захоронения.

   Итак, начиная с 2006~г.\ можно наблюдать существенное ускорение темпов эволюции
МПК, что выразилось в разделении всех ее рубрик на две категории в зависимости от
степени стабильности. При этом почти в два раза сократился период пересмотра стабильных
рубрик базового уровня (с 5--6 до~3~лет).

   В завершение рассмотрения этого примера отметим, что в МПК как эволюционирующей
системе знаний не отражаются личностные и коллективные концепты, которыми оперируют
эксперты Специального подкомитета ВОИС на этапе рас\-смот\-ре\-ния и согласования
содержания рубрик МПК. Отражаются и используются в патентных информационных
системах только те конвенциональные кон\-цеп\-ты, которые сформировались как результат %\linebreak
обсужде\-ния личностных и коллективных концептов экспертов. Порядок и стадии
рассмотрения и согласования содержания рубрик расширенного уровня определяются
нормативными документами, что по определению относится к институциональному фактору
эволюции МПК. Однако направленность эволюции МПК и сама потребность в
существенном изменении регламента ведения МПК начиная с 2006~г. являются следствием
ускорения темпов эволюции систем технических знаний.

   В литературе уже можно найти примеры описания задач представления личностных и
коллективных концептов в виде авторских классификационных систем и онтологий в
предметной области наук о Земле~[10--13]. Были рассмотрены задачи
согласования нескольких авторских и экспертной онтологий, результаты решения которых
предлагается использовать в процессе создания Геопространственного семантического веба
(Geospatial Semantic Web~--- GSW). В этих задачах экспертная онтология является
представлением знаний коллектива экспертов, а авторские онтологии~--- представлением
личностных знаний обычных пользователей, не являющихся экспертами в науках о Земле.
Предложен подход, разработаны алгоритмы и программы, позволяющие фиксировать
различия между авторскими и экспертной онтологиями. Эти различия систематизированы по
нескольким основаниям: различия в научной парадигме, в покрытии целевой предметной
области, в объемах используемых концептов (объемах значений), в дефинициях терминов и
семантических отношениях между терминами. Возможности предлагаемого подхода были
продемонстрированы на примере интерпретации смысла термина ``region'', различия в
понимании которого разными пользователями были выявлены с помощью специальной
анкеты~\cite{12za, 13za}.

\section{Проблематика представления знаний в 7-й Рамочной программе}

   Приведенные примеры иллюстрируют ак\-ту\-альность категоризации концептов в
интересах поста\-нов\-ки и решения проблемы направляемой %\linebreak 
генерации новых целевых систем
знаний и их пред\-став\-ле\-ния в цифровой среде. Эта проблема нашла свое отражение в
документах 7-й Рамочной программы ЕС. Настоящий раздел статьи содержит краткое
описание приоритетных направлений и проблемы генерации целевых систем знаний в том
виде, как они нашли свое отражение в этих документах.

   В области ин\-фор\-ма\-ци\-он\-но-ком\-му\-ни\-ка\-ци\-он\-ных технологий (ИКТ) в этих документах
сформулировано восемь приоритетных направлений исследований и разработок, включая в
качестве %\linebreak 
отдель\-ных направлений <<Электронные библиотеки и их содержание>> и
<<Перспективные ИКТ>>~\cite{14za, 15za}. Именно в этих двух направлениях нашли
отражение новые грани проблематики представления и сохранения знаний в цифровой среде.
Однако в описаниях приоритетных направлений исследований и разработок, как правило, не
определены ключевые термины.

   В качестве примера рассмотрим несколько положений, взятых из формулировок
приоритетных направлений в области ИКТ. Долгосрочные цели проектов по приоритетному
направлению электронных библиотек в <<Программе работ по ИКТ на 2007--2008~годы>>
сформулированы следующим образом~[16, с.~36]:

   <<Создание новых подходов к сохранению информации и представлению знаний
человека в циф\-ро\-вой среде на основе перспективных технологий по управлению
динамически изменяемыми большими объемами данных, гарантирующих сохранение
цифрового контента, выявление и \textbf{экспликацию эволюции его семантики}>>.

   Отметим, что в приведенной формулировке говорится о <<семантике цифрового
контента>>, но при этом в цитируемом документе отсутствует явное или контекстное
определение смысла этого словосочетания.

   Проблема представления знаний упоминается еще в одном приоритетном
   направлении~--- <<Перспективные ИКТ>> в теме <<ИКТ долговременного
применения>>. Ключевые положения этой темы, имеющие непосредственное отношение к
проблеме представления знаний, сформулированы следующим образом~[16, с.~63]:

   <<Разработать новые подходы к представлению и сохранению знаний, ориентированные
на долговременный и безотказный доступ к ним в условиях \textbf{локальной генерации
концептов}, их интеграции, а также глобального использования сис\-тем пред\-став\-ле\-ния и
сохранения знаний с учетом контекста и временной эволюции систем. При этом должна быть
обеспечена \textbf{долговременная устойчивость} сис\-тем представления и сохранения
знаний в условиях многообразия их использования и \textbf{эволюции семантики во
времени}>>.

   Таким образом, кроме выявления и экспликации эволюции <<семантики цифрового
контента>> ставится задача учета <<локальности генерации концептов>>, обеспечения
<<долговременной устой\-чи\-вости систем представления и сохранения знаний>> и
<<интеграции локально сгенерированных концептов>>, но опять нигде не определено
смысловое содержание используемых словосочетаний. Отметим, что формулировки
<<Программы работ по ИКТ на 2007--2008~годы>> весьма лаконичны, что является
следствием жанра этого документа. Иногда смысл целого ряда ключевых положений трудно
понять из контекста этого программного документа. Поэтому потенциальные заявители
проектов нередко вы\-нуж\-де\-ны обращаться в соответствующий директорат 7-й Рамочной
программы ЕС с просьбой пояснить смысл положений программных документов.

   Что касается других направлений исследований в рамках проблемы представления
знаний, то более подробное их описание можно найти в трудах семинара ``Knowledge
Anywhere Anytime: ``The Social Life of Knowledge'', который состоялся 29--30~апреля
2004~г.\ в Брюсселе~\cite{17za}. Материалы этого семинара в редуцированном виде
использовались при формировании упомянутой <<Программы работ по ИКТ>>.

   В этих материалах  отмечается, что исследование процессов генерации знаний и
образования конвенциональных систем знаний, а также связанных с ними процессов
является актуальной проблемой, которая остается во многом нерешенной. Одновременно
фиксируется то, что содержание самой этой проблемы со временем эволюционирует.
Участники семинара определили четыре актуальных на\-прав\-ле\-ния исследований в рамках
этой эволюционирующей проблемы.

   \textbf{Задачей первого направления} является формирование научного понимания
того, как знания появляются, каким образом на этот процесс и его резуль\-та\-ты влияет
совместная деятельность, как %\linebreak 
формируются конвенциональные системы знаний. Одна из
задач этого направления заключается в определении и фиксировании различий в понимании
идентичных информационных объектов разными участниками совместной деятельности.

   \textbf{Задачей второго направления} является исследование многообразия форм
представления одних и тех же концептов, одной и той же системы знаний. Кроме форм
представления конвенциональных и стабильных концептов предметом исследования
являются формы представления личностных и коллективных концептов, а также процессы
возникновения и эволюции нестабильных концептов. В~рамках этого направления
предполагается выполнение исследований процессов образования конвенциональных
концептов на основе личностных и коллективных концептов.

   \textbf{Задачей третьего направления} является создание нового поколения
интеллектуальных систем, которые должны обеспечить семантическую интероперабельность
в процессе совместной работы пользователей этих систем. Если следовать описанию
семиуровневой модели интероперабельности Р.~Будденберга, то семантическая
интероперабельность имеет отношение к следующим двум уровням этой модели:
когнитивная интероперабельность (\mbox{6-й} уровень семиуровневой модели) и доктринальная
интероперабельность (7-й уровень). В соответствии с определением Р.~ Будденберга
реализация когнитивной интероперабельности в процессе совместной работы предполагает
обеспечение согласованного понимания пользователями идентичных информационных
объектов, являющихся формами представления концептов. Реализация доктринальной
интероперабельности предполагает не только согласованное понимание пользователями
информационных объектов, но и обеспечение принятия ими согласованных решений на
основе идентичной информации. Естественно, что в интеллектуальных системах нового
поколения должна обеспечиваться и технологическая интероперабельность на первых пяти
уровнях семиуровневой модели, в том чис\-ле на уровнях разделяемых процедур, процессов и
данных~\cite{18za, 19za}.

   В настоящее время доктринальная интероперабельность достигается в основном за счет
использования организационных процедур и регламентов согласования личностных
концептов в процессе принятия совместных решений.

   В рамках третьего направления кроме создания методов и средств поддержки
семантической интероперабельности в интеллектуальных сис\-те\-мах планируется исследовать
вопросы выявления и экспликации основных стадий эволюции систем знаний,
представленных в виде классификационных систем, тезаурусов и онтологий, в процессе
совместной работы пользователей. При этом не предполагается, что пользователи заранее
будут владеть согласованной между ними системой терминов и единым пониманием
принципов построения эволюционирующих систем знаний.

   Степень новизны интеллектуальных систем представления знаний предлагается
оценивать на основе их сравнения с системами <<управления знаниями>> (knowledge
management systems), основанными на редукционистском подходе к трактовке знаний
человека, в котором эволюция систем знаний во времени не учитывается, личностные,
коллективные и конвенциональные концепты, стабильные и нестабильные концепты не
различаются.

   \textbf{Задачей четвертого направления} является исследование принципиальных
возможностей и средств влияния на формирование необходимых целевых систем знаний в
процессе совместной дея\-тель\-ности. При этом предполагается, что имеет место ситуация
недостаточности уже имеющихся знаний. Речь идет о формировании новых систем знаний,
ориентированных на удовлетворение технологических, экономических, образовательных и
других социально значимых потребностей общества. В рамках этого направления
предлагается исследовать, какими видами перспективных ИКТ и до какой степени можно
оказывать влияние на процесс генерации целевых систем знаний, отвечающих социально
значимым потребностям и необходимых для получения запланированных результатов
совместной деятельности. Именно возможность оказывать влияние на процесс
формирования новых целевых систем знаний является, по мнению участников семинара
``Knowledge Anywhere Anytime: ``The Social Life of Knowledge'', характерной чертой
общества, основанного на знаниях.

   Приведенный перечень из четырех направлений исследований говорит о том, что
участники семинара существенно расширяют границы предметной области представления
знаний, сформулированной в рамках редукционистского подхода, за счет рассмотрения
процессов целенаправленной генерации и эволюции систем знаний во времени.
Проведенный в этом разделе анализ позволяет сделать вывод о том, что предлагаемое
расширение предметной области происходит за счет следующих направлений исследований:
   \begin{itemize}
\item разработка методов тестирования уже име\-ющих\-ся систем знаний с точки зрения
проверки их полноты, анализируемой с позиции удовлетворения тех или иных явно
выраженных социально значимых потребностей общества;
\item целенаправленное воздействие на процессы генерации личностных и
коллективных концептов (\textit{направляемое развитие}, в терминах
Н.\,Н.~Моисеева~\cite{20za});
\item направляемая генерация новых целевых систем знаний на основе интеграции
личностных и коллективных концептов;
\item представление в цифровой среде личностных и коллективных концептов,
целевых систем знаний и отражение стадий их эволюции во вре\-мени;
\item разработка методов распространения и ориентированного использования
целевых систем знаний, сгенерированных в интересах удовлетворения
соответствующих социально значимых потребностей общества.
\end{itemize}

   В заключение раздела отметим, что в ци\-ти\-ру\-емых материалах семинара акцентируется
институциональная и социальная обусловленность про\-цессов направляемой генерации
целевых систем %\linebreak 
знаний, что необходимо учитывать разработчикам в процессе создания ИИС.
Это существенно усложняет создание ИИС, целью которых является воздействие на
процессы генерации целевых систем знаний средствами ИКТ, так как для разработчиков
ИИС институциональные и социальные факторы являются, как правило, внешними
ограничениями, которые они не могут изменять~\cite{17za}.

   Определяя ключевые направления исследований по тематике представления знаний, в
том числе приводя описание проблемы направляемой генерации целевых систем знаний,
участники семинара не предложили согласованную систему терминов для описания новых
направлений исследований. Пока новые идеи только обсуждаются и определяются
направления исследований, подобная ситуация, как правило, неизбежна, так как система
терминов для их описания только начинает формироваться. Затем, по мере расширения
границ предметной об\-ласти представления знаний, часто возникает необходимость
определять новые термины, используемые для описания новых направлений исследований,
и/или уточнять содержание ранее введенных терминов.

\section{Категоризация концептов}

   Сначала рассмотрим систему базовых терминов информатики, построенную на основе
результатов работ~[21--25] и описанную в~[5, 26]. В~рамках этой
системы терминов разные категории концептов лексически не различались. Основная цель
данного раздела статьи заключается в расширении системы базовых терминов с целью
разделения концептов на три категории~--- личностные, коллективные и конвенциональные.

   Основываясь на результатах работ~[21--25] и понимании семиотического термина
\textit{знак} в рамках лингвоцентрического подхода~[27, 28], сначала уточним дефиниции
системы базовых терминов, которая была определена в работах~[5, 26]. При этом будем
нумеровать описываемые далее термины в соответствии с рис.~\ref{f1za} против часовой
стрелки, выделяя их в тексте курсивом.
   \begin{enumerate}[1.]
   \item Термин \textit{знания} будем трактовать как результаты познавательной и
креативной целеустремленной деятельности человека, носителем которых может быть
только человек и которые могут быть разделены на отдельные <<кванты>> знаний (в
программных документах 7-й Рамочной программы используется словосочетание
``knowledge parts'')~[23, с.~33; 24, с.~185; 29, с.~814].
   \item В процесах представления и сохранения знаний основное внимание будет уделяться
тем <<квантам>> знаний, называемым \textit{концептами}, которые являются
элементарными единицами или сочетаниями элементарных единиц плана
содержания\footnote{Иначе говоря, план содержания естественного языка (знаковой системы)
представляет собой совокупность тех <<квантов знаний>>, которые могут быть выражены средствами
этого языка (знаковой системы).}, выражаемого в рамках некоторого естественного языка (в
общем случае, в рамках той или иной знаковой системы).

   Приведенное определение термина \textit{концепты} предполагает, что они являются
результатом процесса членения знаний человека на <<кванты>>, которые могут быть
выражены в рамках некоторой знаковой системы. Процесс членения неразрывно связан с
процессом выражения знаний человека в сенсорно воспринимаемой и от\-чуж\-ден\-ной от
человека форме, например в виде текста на естественном языке, диаграммы или в виде
геоизображения на языке карты.
   \begin{figure*} %fig1
\vspace*{1pt}
\begin{center}
\mbox{%
\epsfxsize=113.873mm
\epsfbox{zat-1.eps}
}
\end{center}
\vspace*{-9pt}
   \Caption{Система базовых терминов~[5]
   \label{f1za}}
   \end{figure*}

   Система знаний человека может включать один или более планов содержания, что
определяется числом естественных языков или иных знаковых систем, которыми он владеет
и пользуется для представления своих знаний в отчужденной форме. Согласно
атомистическому подходу к описанию семантики, минимальные единицы плана содержания,
имеющие значение в рамках некоторого естественного языка или иной знаковой системы,
будем называть \textit{элементарными концептами} этой знаковой системы~\cite{30za}.

   Отметим следующую особенность описываемой системы терминов: выделение в системе
знаний человека нескольких планов содержания позволяет учесть различия в членении
знаний человека в разных естественных языках и других знаковых системах, что
необходимо, например, в процессе проектирования многоязычных семантических словарей и
тезаурусов.

  Здесь также необходимо отметить разницу %\linebreak 
  между элементарным концептом и значением
слова или устойчивого словосочетания, когда речь идет о естественном языке. В процессах %\linebreak
представле\-ния научно-технических знаний в %\linebreak 
от\-чуж\-денной форме рассматривается только
сигнификативный аспект значения. Экс\-прес\-сив\-но-эмо\-цио\-наль\-ные оценки, идиомы и
коннотации не рассматриваются, так как речь идет о кодировании на\-уч\-но-тех\-ни\-че\-ских
зна\-ний в циф\-ро\-вой среде ИИС, а не о созда\-нии ху\-до\-жест\-вен\-ных произведений.
   Следова\-тельно, %\linebreak 
   термины \textit{концепт} и \textit{элементарный концепт} используются
для обозначения только сигнификативной составляющей значений слов и
словосочетаний~\cite{32za, 31za}.
{\looseness=1

}
   \item Термин \textit{информация} определим как формы эксплицитного и отчужденного
от человека представления его знаний, предназначенные для %\linebreak 
пере\-да\-чи, непосредственного
сенсорного восприятия и понимания их другими людьми. Такое определение, по сути,
является выбором из некоторого множества одной дефиниции в рамках позиционирования
слова \textit{информация} как омонима, охватывающего широкий спектр феноменов,
соответствующих им дефиниций и смысловых значений~\cite{33za}.
   \item Согласно П.~Ингверсену, термин \textit{знаковая ин\-формация} определим как
результат процесса представления концептов человеком в плане выражения\footnote{Иначе
говоря, план содержания представляет собой последовательности литер слов, фраз и текста на
естественном языке, рисунки и другие формы представления концептов в среде социальных коммуникаций.} в
любой знаковой форме, которая\linebreak
 является сенсорно воспринимаемой и понимаемой другими
участниками коммуникаций~\cite{34za}. Отметим, что при таком определении термин
\textit{знаковая информация} имеет отношение только к знаковым формам представления
концептов, а введенный ранее термин \textit{информация}~--- к любым формам
представления любых <<квантов>> знаний, включая концепты.
   \item Существительные \textit{информация} и \textit{знаковая ин\-формация} являются
неисчисляемыми, что не %\linebreak
 всегда удобно для описания процессов пред\-став\-ле\-ния знаний в
цифровой среде ИИС. Поэтому определим следующие термины с использованием
исчисляемых существительных. Представление в плане выражения эле\-мен\-тар\-ных концептов
в виде сенсорно вос\-при\-ни\-ма\-емых знаковых форм будем называть \textit{эле\-ментар\-ны\-ми
информационными объектами}. Сен\-сорно воспринимаемые знаковые формы пред\-став\-ле\-ния
любых концептов, включая элементарные, будем называть \textit{информационными
объ\-ек\-тами}.
{\looseness=1

}
   \item Термин \textit{коды} определим как компьютерные эквиваленты двоичных цифр
(или их последовательностей), которые могут представлять собой намагниченность или ее
отсутствие, наличие электрического тока или его отсутствие, способность к отражению света
или ее отсутствие, в цифровой среде~[24, с.~86]. Отметим, что литеры <<0>> и
<<1>>, о которых говорится в определении термина <<коды>>, по определению являются
сущностями среды социальных коммуникаций ИИС, а их компьютерные эквиваленты~---
циф\-ро\-вой среды ИИС.

   В соответствии с работой~[26] предлагается выделять среди всех возможных кодов
цифровой среды ИИС три следующие категории:
   \begin{itemize}
\item коды, соотнесенные с концептами знаний че\-ло\-ве\-ка-пользователя ИИС
(\textit{коды первой категории ИИС});
\item коды, соотнесенные с эксплицитными и отчужденными от человека сенсорно
воспринимаемыми знаковыми формами пред\-став\-ле\-ния концептов в плане выражения
среды социальных коммуникаций ИИС (\textit{коды второй категории ИИС});
\item коды, идентифицирующие программы и используемые ими информационные ресурсы,
а также другие категории денотатов цифровой среды ИИС (\textit{коды третьей
категории ИИС}).
\end{itemize}

   Например, для кодирования смысла индикаторов в системе информационного
мониторинга сферы науки, которая является примером ИИС, предполагается использовать
коды первой категории, для кодирования названий индикаторов~--- коды второй категории, а
для кодирования программ вычисления значений индикаторов и используемых этими
программами информационных ресурсов~--- коды третьей категории.
   \item В соответствии с работой~[26] определим еще одну \textit{нулевую
категорию кодов}, которую будем называть \textit{цифровыми данными} и к которой будем
относить все коды цифровой среды ИИС, не относящиеся к трем выше определенным
категориям. Таким образом, к цифровым данным будем относить, в частности, результаты
любых измерений, полученных с помощью цифровых технических систем (устройств) и
хранящихся в ИИС, и результаты любых вычислений, не являющихся итогом представления
знаний человека-поль\-зо\-ва\-те\-ля ИИС.
   \item Представление цифровых данных в среде социальных коммуникаций ИИС в
сенсорно воспринимаемой форме будем называть \textit{данными}.
   \item Смысл термина \textit{ментальные образы данных} следует из его названия и
рис.~\ref{f1za}.
   \end{enumerate}

   Разделим рассмотренные термины на три группы в зависимости от природы
обозначаемых ими сущностей, т.\,е.\ среды или сферы, к которой они принадлежат
(ментальная сфера, социальная или цифровая среда)~\cite{25za}:
   \begin{itemize}
\item знания, ментальные образы сенсорно воспринимаемых данных, концепты и
элементарные концепты (сущности ментальной сферы пользователей ИИС);
\item информация, данные, знаковая информация, информационные объекты,
элементарные информационные объекты (сущности среды социальных коммуникаций
пользователей ИИС);
\item цифровые данные и три категории кодов (сущности цифровой среды ИИС).
\end{itemize}

   В указанной системе терминов нас интересуют концепты и элементарные концепты,
обозначенные на рис.~1 номером 2, которые предлагается разделить на три разные
категории. Прежде чем провести категоризацию концептов, отметим, что дефиниция термина
концепт существенно зависит от термина \textit{знаковая система}, так как по определению
любой концепт соотнесен с той или иной знаковой системой. Однако по определению
се\-мио\-ти\-че\-ско\-го знака и знаковой системы две стороны знака (форма~--- материально
выраженная его составляющая~---  и значение~--- его идеальная со\-став\-ля\-ющая), будучи
поставлены в отношение \textbf{постоянной связи}, опосредованной сознанием, составляют
\textbf{устойчивое} единство, которое посредством сенсорно воспринимаемой формы знака
репрезентирует \textbf{конвенционально} приданное ему значение~\cite{35za}.

   Следовательно, в рамках системы базовых терминов из работы~[26] приведенное
определение термина \textit{концепты} по своему генезису (про\-ис\-хож\-де\-нию) предполагает,
что они являются, с одной стороны, конвенциональными, с другой стороны, образуют
устойчивое единство с формой их представления. Таким образом, чтобы определить
личностные и коллективные концепты, а также %\linebreak
 неста\-биль\-ные концепты пользователей ИИС,
сначала необходимо расширить понятие знаковой сис\-те\-мы, сняв обязательные условия
кон\-вен\-цио\-наль\-ности значения и устойчивого единства кон\-цеп\-тов и форм их представления.
Одним из возможных способов расширения понятия знаковой системы является введение
понятий \textit{авторского и коллективного знаков}.

   \textit{Авторский знак} отличается от традиционного семиотического знака тем, что две
стороны авторского знака~--- форма и значение~--- могут со\-став\-лять стабильное или
нестабильное единство, т.\,е.\ находиться в отношении долгосрочной или временной связи,
опосредованной сознанием \textbf{одного пользователя ИИС}, которая посредством
сенсорно воспринимаемой формы знака этим пользователем репрезентирует
\textbf{персонально} приданное ему значение в течение периода времени, указанного при
регистрации авторского знака в ИИС.

   \textit{Коллективный знак} отличается от авторского тем, что две стороны коллективного
знака могут находиться в отношении долгосрочной или временной связи, опосредованной
сознанием каждого из участников совместной деятельности (членов коллектива),
являющихся пользователями ИИС, и согласованной между ними, т.\,е.\ используют и
\textbf{согласованно понимают} коллективные знаки как минимум \textbf{два человека}.

   Таким образом, одним из ключевых признаков отличия коллективного знака от
авторского является согласованность его смыслового содержания между всеми членами
коллектива. Оставим пока открытым вопрос о признаках отличия коллективного знака
(концепта) от конвенционального и стабильного знака (концепта) от нестабильного.

   Важно отметить, что, допуская возможность временной связи формы и значения у одного
пользователя или ограниченной группы пользователей, т.\,е.\ снимая обязательные условия
конвенциональности значения и устойчивого единства концеп\-тов (значений знаков) и форм
их пред\-став\-ле\-ния, мы тем самым говорим об их недоступности для всех, кроме их авторов.
Чтобы расширить круг пользователей авторских и коллективных знаков, предлагается
регистрировать их в ИИС, фиксируя значения и формы авторских и коллективных знаков с
по\-мощью средств лингвистического обеспечения ИИС. Например, в тезаурусе ИИС можно
указывать их авторов, а также период времени, в течение которого сенсорно воспринимаемая
форма знака репрезентирует персонально или коллективно приданное ему значение. Если
авторы проделают эту работу, то для других пользователей ИИС станут доступны их формы,
описание их значений, информация об авторах, а также период времени, в %\linebreak 
течение которого
установлена связь между формой и содержанием для авторских и коллективных %\linebreak 
\mbox{знаков.}


   Авторские и коллективные знаки не возникают и тем более не функционируют раздельно.
В своей совокупности с традиционными семиотическими знаками они образуют единую
систему, которую будем называть \textit{знаковой системой ИИС}. Иначе говоря, авторские
и коллективные знаки регистрируются в ИИС ее пользователями и означиваются в
совокупности с традиционными конвенциональными знаками, являющимися составными
элементами естественного языка или иной знаковой сис\-те\-мы\footnote{Приведенные определения
авторского и коллективного знаков основаны на результатах работы~\cite{35za}.}.

   Используя определения авторского и коллективного знаков, можно предложить
следующий подход к категоризации концептов. Определим \textit{элементарный
личностный концепт} как значение авторского знака, а \textit{личностный концепт}~--- как
значение выражения на естественном языке или в рамках некоторой знаковой системы,
содержащего хотя бы один авторский знак, либо новое значение выражения без авторских
знаков, смысл которого определен автором в явном виде и зарегистрирован в ИИС.

   Определим \textit{элементарный коллективный концепт} как значение коллективного
знака, а \textit{коллективный концепт}~--- как значение выражения на естественном языке
или некоторой знаковой сис\-те\-мы, содержащего хотя бы один коллективный знак, либо новое
значение выражения без коллективных знаков, смысл которого определен в явном виде,
согласованно понимается как минимум двумя участниками совместной деятельности и
зарегистрирован в ИИС.

   \textit{Представлением личностного концепта} в циф\-ро\-вой среде будем называть код(ы)
первой категории ИИС, которые автор этого концепта выбрал и поставил в соответствие
этому концепту в результате выполнения процедуры регистрации соответствующего
авторского знака в ИИС.

   \textit{Представлением коллективного концепта} в циф\-ро\-вой среде будем называть
код(ы) первой категории ИИС, которые участники совместной дея\-тельности согласованно
выбрали и поставили в %\linebreak 
соот\-вет\-ст\-вие этому концепту в результате выполнения процедуры
регистрации соответствующего коллективного знака в ИИС.

   Рассмотренные дефиниции трех категорий кодов, личностного и коллективного
концептов могут служить основой для формирования тех циф\-ро\-вых сущностей, которые и
являются <<представителями>> личностных и коллективных концептов в циф\-ро\-вой среде.
Это дает возможность использовать предлагаемую категоризацию концептов при решении
задач формирования целевых систем знаний и отражать эволюцию семантики личностных и
коллективных концептов этой системы в цифровой среде, используя результаты выполнения
процедуры их регистрации в ИИС. Пример прикладного использования категоризации и
регистрации личностных и коллективных концептов приводится в следующем разделе.

   Теперь определим границу между \textit{стабильными} (lasting) и
\textit{нестабильными концептами} (volatile concepts). В дефинициях авторских и
коллективных знаков говорится, что их формы и значения могут составлять стабильное или
нестабильное единство. При этом период времени долгосрочной или временной связи между
ними указывается при регистрации в ИИС. В качестве основного признака отличия
стабильных концептов от нестабильных возьмем критерий, который использовался для
разделения МПК-8 на рубрики базового и расширенного уровней. Как отмечалось во втором
разделе, рубрики базового уровня, выражающие стабильные концепты, могут
пересматриваться не чаще, чем раз в три года, а рубрики расширенного уровня могут
пересматриваться значительно чаще.

   Взяв за основу этот критерий, будем говорить, что в рамках некоторой
институциональной сис\-те\-мы существует граница между стабильными и нестабильными
концептами, если выполнены следующие условия:
   \begin{itemize}
\item в этой институциональной системе имеется нормативный документ,
определяющий период времени, превышение которого для зарегистрированной связи
между формой и значением (концептом) знака говорит о стабильности этого знака и
соответствующего ему концепта, или содержащий перечень стабильных знаков,
например в виде списка рубрик базового уровня системы классификации или базовых
дескрипторов тезауруса;
\item средства лингвистического обеспечения ИИС содержат информацию о периоде
времени зарегистрированной связи между формой и значением (концептом) знака.
   \end{itemize}

   Иначе говоря, должна быть определена хотя бы одна система классификации (или
тезаурус), отражающие эти концепты, в которой разделены стабильные и нестабильные
рубрики системы классификации (стабильные и нестабильные дескрипторы тезауруса).

   Осталась неопределенной граница между коллективными и традиционными
конвенциональными знаками, а также соответствующими им концептами. По своей сути эта
граница является весьма размытой. Однако для разработчиков ИИС важно иметь некоторый
признак, позволяющий различать коллективные и конвенциональные знаки и кон\-цеп\-ты.
Поэтому при создании конкретных систем предлагается эту границу фиксировать с помощью
нормативного документа.

\vspace*{-6pt}

\section{Примеры описания личностных концептов}

   Категоризация концептов, которая предложена в предыду\-щем разделе, в настоящее время
используется в процессе создания системы индикаторов для оценивания результатов и
эффективности реализации НИОКР Программы фундаментальных научных исследований
РАН на 2008--2012~гг., а также потенциала научных коллективов, вы\-пол\-ня\-ющих НИОКР.

   Создаваемая система включает широкий спектр индикаторов, традиционно используемых
в сфере науки, а также ряд новых. Пример традиционного индикатора приведен на
рис.~\ref{f2za}, на котором дано возрастное распределение сотрудников одного из научных
коллективов ИПИ РАН. На этом рисунке приведены значения индикатора в виде графика
процентного распределения сотрудников по 14 возрастным группам (20--24, 25--29 и далее
до группы 85--89~лет). Отметим, что все графики на рис.~\ref{f2za} и~\ref{f3za} по
определению являются дискретными, но для наглядности все 14~значений соединены
отрезками.

\begin{figure*} %fig2
\vspace*{1pt}
\begin{center}
\mbox{%
\epsfxsize=101.611mm
\epsfbox{zat-2.eps}
}
\end{center}
\vspace*{-9pt}
\Caption{График возрастного распределения сотрудников научного коллектива
\label{f2za}}
\end{figure*}

   Для каждой группы сотрудников вы\-чис\-ля\-ется ее доля в общей численности научного
коллектива. На рисунке видно, что для этого коллектива наибольшую долю имеет возрастная
группа 45--49~лет (24\%). Далее по убыванию следуют шесть возрастных групп: 40--44 года
(20,6\%), 25--29~лет (10,3\%), 30--34~года, 50--54~года, 55--59~лет и 75--79~лет (по 6,9\%).
Затем идут пять возрастных групп с долей в 3,5\%: 35--39~ лет, 60--64~года, 65--69~лет,
   70--74~года и 80--84~ года. Представители двух оставшихся возрастных групп в этом
коллективе отсутствуют: 20--24~года и 85--89~лет.

   Для определения публикационной активности каждой из возрастных групп была
предпринята попытка создать новый индикатор и разработать программу вычисления
значений этого индикатора. С~этой целью были привлечены специалисты, которые имели
разные представления о со\-зда\-ва\-емом индикаторе.

   Первое определение индикатора, предложенное одним из авторов этой статьи (далее по
тексту~--- первый специалист), охватывало все публикации каждой возрастной группы.
Учитывались все пуб\-ли\-ка\-ции, зарегистрированные в течение заданного периода времени в
базе данных проектируемой Информационно-технологической системы  мониторинга
Программы фундаментальных научных исследований. Если у публикации было несколько
авторов, то соответствующим возрастным группам добавлялось по единице. На основе
личностного понимания нового индикатора первым специалистом была разработана
программа вычисления его значений (см.\ рис.~\ref{f3za}) для сотрудников научного
коллектива, возрастное распределение которого приведено на рис.~\ref{f2za}.

\begin{figure*} %fig3
\vspace*{1pt}
\begin{center}
\mbox{%
\epsfxsize=101.886mm
\epsfbox{zat-3.eps}
}
\end{center}
\vspace*{-9pt}
\Caption{График распределения публикационной активности возрастных групп
\label{f3za}}
\end{figure*}

   Для каждой группы сотрудников была вы\-чис\-ле\-на ее доля статей в общей численности
статей научного коллектива с учетом того, что в случае нескольких авторов у статьи
соответствующим возрастным группам добавлялось по единице. На рисунке видно, что для
этого коллектива наибольшую долю статей имеет возрастная группа 75--79~лет (39,7\%).
Далее по убыванию следуют семь возрастных групп: 40--44~года (24,1\%), 45--49~лет
(13,1\%), 55--59~лет (10,1\%), 25--29~лет (6,5\%), 30--34~года, 35--39~лет и 50--54~года (по
1,5\%). Затем идут две возрастные группы с долей в 1,0\%: 70--74~года и 80--84~года.
Представители двух возрастных групп в коллективе отсутствуют: 20--24~года и 85--89~лет, а
две возрастные группы не имеют публикаций: 60--64~года и 65--69~лет.

   На следующем этапе в процесс создания нового индикатора включился второй
специалист, который предложил внести следующее изменение в определение
публикационной активности возрастных групп: если в публикации указано $N$ авторов, то
соответствующим возрастным группам добавлялось не по единице, как на рис.~\ref{f3za}, а
по $1/N$. Этот график в статье не приводится. Укажем только его значения для первых пяти
возрастных групп из 14: 75--79~лет (47\%), 40--44~года (22,6\%), 55--59~лет (9,5\%),
   45--49~лет (9,1\%), 25--29~лет (6,3\%). Из сравнения этих значений с графиком на
рис.~\ref{f3za} следует, что из пяти перечисленных возрастных групп доля первой
увеличилась, а остальных четырех групп уменьшилась. Причина в том, что в первой группе
значительная часть публикаций не имеет соавторов, а в остальных четырех группах~---
наоборот.

   Предлагаемое изменение программы вычисления значений нового индикатора
мотивировалось следующим примером: если два автора, составляющих группу
   50--54~года\footnote{Предполагается, что в возрастной группе 50--54~года присутствуют только два
человека.}, в течение заданного периода времени опубликуют вдвоем в соавторстве 5~статей,
а один автор в группе 55--59~лет\footnote{Предполагается, что в возрастной группе 55--59~лет
присутствует только один человек.} опубликует без соавторов тоже 5~статей, то в соответствии с
первым вариантом программы публикационная активность первой возрастной группы будет
в два раза выше, чем второй группы, хотя в обеих группах будет опубликовано одинаковое
число статей~--- по~5.

   Однако первый специалист не согласился с этим доводом и привел следующий пример:
если два соавтора в течение заданного периода времени опуб\-ли\-ку\-ют вдвоем в соавторстве
5~статей, при этом первый соавтор относится к группе 45--49~лет, второй соавтор~--- к
группе 50--54~года, а еще один автор в группе 55--59~лет\footnote{Предполагается, что в
каждой из трех возрастных групп присутствует по одному человеку.} опубликует без соавторов тоже
5~статей, то в соответствии со вторым вариантом программы публикационная активность
первых двух возрастных группы будет в два раза ниже, чем третьей группы, хотя во всех
трех группах будет опубликовано одинаковое число статей~--- по~5.

   Итогом второго этапа создания нового индикатора стали два разных личностных
понимания этого индикатора каждым из двух специалистов. При этом было отмечено, что
предложенные личностные концепты обладают общим недостатком, так как в обоих
вариантах программы их вычисления не учитывается численность возрастных групп, для
которых сопоставляется публикационная активность. Например, наибольшую численность
сотрудников имеет возрастная группа 45--49~лет (24\%), а доля группы 75--79~лет
составляет 6,9\%. В соответствии с первым личностным концептом публикационная
активность возрастной группы 75--79~лет равна 39,7\%, а группы 45--49~лет~--- 13,1\%.
Таким образом, если не учитывать численность возрастных групп, то публикационная
активность первой группы превышает активность второй группы более чем в 3~раза. Однако
с учетом численности это превышение составит 10,5~раз. Еще один отмеченный недостаток
заключается в том, что не учитывается рейтинг (импакт-факторы) источников публикаций.

   На последующих этапах построения этого индикатора планируется разработать еще
несколько вариантов программы вычисления его значений, устраняющие эти недостатки, что
может привести
к увеличению числа разных личностных концептов, соответствующих
различным вариантам понимания нового индикатора. Важно отметить, что %\linebreak 
различным
вариантам понимания могут соответствовать не только разные варианты алгоритма %\linebreak
вы\-чис\-ле\-ния значений индикатора, например при учете соавторов и/или численности
возрастной группы, но и разные информационные ресурсы, например списки журналов ВАК
или Российского индекса научного цитирования.

   В процессе совместной деятельности и расширения группы специалистов, участвующих в
создании нового индикатора, могут появляться и коллек\-тив\-ные концепты.
Конвенциональный %\linebreak 
концепт появится только в тот момент, когда в рамках соответствующей
институциональной системы будет принят нормативный документ, опре\-де\-ля\-ющий именно
тот вариант определения индикатора, который и должен будет применяться для оценивания
публикационной активности научных коллективов и организаций. При этом в ИИС
сохранится вся история эволюции личностных, коллективных и конвенциональных
концептов, име\-ющих отношение к созданию нового индикатора.

   Ранее на примере патентной сферы уже была описана аналогичная процедура
формирования конвенциональных концептов рубрик МПК. В этой процедуре документ
Международного бюро ВОИС определяет дату ввода в действие новых и измененных
расширенных рубрик МПК и связанных с ними терминов, а за три месяца до этой даты
доводит их до сведения патентных ведомств, с тем чтобы последние могли принять
обеспечительные меры. Однако в соответствующих патентных ИИС фиксируется история
эволюции только конвенциональных концептов. История эволюции личностных и
коллективных концептов в них не отражается.

\section{Отражение эволюции концептов в ИИС}

   В данной статье предлагается следующий подход к отражению эволюции личностных и
коллективных концептов в ИИС. С целью регистрации трех категорий концептов
(личностные, коллективные и конвенциональные) все дескрипторы тезауруса ИИС
предлагается разделить также на три категории. Что касается создания конвенциональных
дескрипторов, то они традиционно формируются лингвистами, как правило, на основе
корпуса текс\-тов нормативных документов, толковых и других видов словарей.

   Предполагается, что личностные концепты будут описывать и представлять в виде
дескрипторов тезауруса  их авторы в соответствии с регламентом ИИС. При этом каждый
автор будет иметь возможность дополнительно отразить эволюцию личностных концептов
во времени в виде серии вариантов авторских дескрипторов тезауруса ИИС. Отметим, что в
примере с созданием нового индикатора эти дескрипторы должны быть связаны с
вариантами программы его вычисления и используемыми этой программой
информационными ресурсами~\cite{36za, 37za}.

   Чтобы отличить авторские дескрипторы от\linebreak других категорий дескрипторов, предлагается
помечать их как персональные, т.\,е.\ ввести для них атрибут <<автор дескриптора>>.
Кроме того, предлагается использовать атрибуты, фиксирующие моменты времени:
   \begin{itemize}
\item включения авторских дескрипторов и их вариантов в тезаурус ИИС;
\item установления их связей с другими дескрипторами, программами,
информационными ресурсами и компонентами других видов обеспечения ИИС.
   \end{itemize}

   Это позволит, анализируя варианты и значения атрибутов, определять авторов новых
индикаторов, находить варианты программ вычисления их значений и оценивать эволюцию
семантики дескрипторов во времени.

   Аналогично в процессе совместной дея\-тель\-ности предлагается описывать и помечать
коллективные дескрипторы с помощью атрибутов коллективного авторства, фиксировать их
варианты, моменты времени включения в тезаурус ИИС и варианты программ вычисления
их значений.

   С методологической точки зрения приведенное описание процедуры построения
авторских и коллективных дескрипторов тезауруса ИИС и отражения эволюции концептов
во многом основано на концепции противопоставления и связи двух планов, или аспектов,
изучения мышления~--- плана образов (или знаний) и плана <<процессов>> (или деятельности),
предложенной Г.\,П.~Щедровицким и Н.\,Г.~Алексеевым. Ими была сформулирована задача
<<операционально-деятельностного анализа понятий и знаний, позволяющего, исходя из
формы какого-либо сложившегося понятия или знания, сводить его к системе операций и
действий, порождающих содержание этого понятия или знания>>~\cite{38za, 39za}.

   Предлагаемый в статье подход к описанию концептуализации денотатов, например
программ вычисления индикаторов, с помощью авторских и коллективных дескрипторов
тезауруса ИИС имеет два существенных отличия от концепции Г.\,П.~Щед\-ро\-виц\-ко\-го и
Н.\,Г.~ Алексеева. Во-первых, в нем учитывается то, что для личностных и коллективных
концептов могут отсутствать конвенциональные эксплицитные формы представления, что
компенсируется их описанием в ИИС. Во-вторых, в нем учитывается то, что объемы
значений личностных и коллективных концептов могут эволюционировать во времени, и
предложен подход к отражению этой эволюции в эксплицитной форме с помощью средств
лингвистического обеспечения ИИС.

   Отметим еще один существенный аспект. Приведенные примеры использования
предлагаемой системы терминов в процессе проектирования новых индикаторов
иллюстрируют необходимость введения кодов третьей категории для денотатов. Как видно
из примеров раздела~5, новые индикаторы на начальных стадиях их разработки еще не
имеют общепринятых названий, отсутствует согласованное понимание их экспертами и
другими пользователями ИИС. При этом уже разработаны варианты программ их
вычисления. Поэтому в тех случаях, когда нет согласованных названий и концептов, для
представления личностных и коллективных концептов в цифровой среде ИИС, а также для
экспликации эволюции семантики новых индикаторов удобно использовать в качестве кодов
денотатов \textbf{уникальные идентификаторы} вариантов программ для вычисления их
значений и используемых этими программами других ресурсов ИИС.

   Другим примером, иллюстрирующим возможности практического использования кодов
денотатов в задачах формирования целевых систем знаний, является Каталог программных
ошибок (The Common Weakness Enumeration~--- CWE), записи которого связаны между
собой тезаурусными отношениями. Каждая запись этого каталога, как правило, включает
дефиницию, фрагмент программного кода, который иллюстрирует смысловое содержание
ошибки, родовидовые и иные тезаурусные отношения с другими описаниями ошибок. В
процессе разработки подобных каталогов идентификаторы фрагментов программного кода,
иллюстрирующего дефекты программ, могут использоваться в качестве кодов денотатов.
Иначе говоря, для идентификации программных ошибок на начальной стадии их описания,
когда они еще не имеют общепринятых названий, удобно использовать коды денотатов в
процессе описания смысла этих ошибок и выбора для них названий~\cite{40za}.

   В заключение отметим, что новые грани проблематики представления знаний в цифровой
среде, для описания которых предложены новые термины, формируются под воздействием
общественно значимых потребностей в формировании целевых систем знаний в тех случаях,
когда отсутствуют готовые системы знаний. Процессы направляемой генерации целевых
систем знаний можно наблюдать в сфере науки, индустрии программных средств и
патентной сфере как ответную реакцию на новые вызовы и общественно значимые
потребности.

\section{Заключение}

   Предложен новый подход к отражению в циф\-ро\-вой среде процессов генерации
личностных и коллективных концептов, а также стадий их эволюции. Согласно
Н.\,Н.~Моисееву, использование цифровой среды в процессах генерации и эволюции систем
знаний рождает новые системные свойства, не выводимые из свойств отдельных разумов. Их
потенциальные возможности не зависят от желания и действий отдельных людей, это~---
результат самоорганизации и направляемого развития~\cite{20za}.

   Вопросы самоорганизации и направляемого развития целевых систем знаний выходят за
рамки настоящей статьи. В ней были рассмотрены только вопросы категоризации концептов
и сформулированы признаки, позволяющие разработчикам ИИС зафиксировать и
использовать отличия личностных, коллективных и конвенциональных концептов при
описании процессов генерации и эволюции целевых систем знаний.

   Приведенное в разделах~5 и~6 описание процедуры построения новых индикаторов и
отражения эволюции соответствующих концептов в виде совокупности вариантов авторских
и коллективных дескрипторов тезауруса ИИС иллюстрирует возможности практического
применения пред\-ла\-га\-емо\-го подхода, описанного с помощью терминов, лексически
фиксирующих предлагаемое разделение концептов на три категории. Это разделение
концептов увеличивает число базовых терминов с~9 (см.\ рис.~\ref{f1za}) до~12, так как
кроме традиционного термина \textit{концепт} появляются еще три термина~---
\textit{личностный, коллективный и конвенциональный концепты}.

   Предлагаемое расширение числа базовых терминов не является чисто
терминологическим вопросом, так как появление новых понятий и терминов, которые они
обозначают, является следствием постановки новой и актуальной проблемы генерации
целевых систем знаний, а также становления новых направлений исследований и разработок,
относящихся к проблематике представления знаний в цифровой среде.

   Главный результат предлагаемой категоризации концептов заключается в том, что
лексически акцентируются различия между тремя категориями концептов, их отличительные
признаки и особенности стадий формирования, эволюции концептов и целевых систем
знаний, ориентированных на удовлетворение технологических, экономических,
образовательных и других социально значимых потребностей общества. Иначе говоря,
наряду с традиционной проблемой извлечения уже имеющихся знаний, отвечающих
социально значимым потребностям, лексически акцентируется внимание на новой проблеме
формирования целевых систем знаний в тех случаях, когда отсутствуют готовые системы
знаний. При этом в интересах описания данной проблемы предложено развитие ранее
рассмотренной системы базовых терминов, дано описание признаков, разделяющих
концепты на три категории, и процедуры отражения стадий эволюции концептов в цифровой
среде ИИС.

{\small\frenchspacing
{%\baselineskip=10.8pt
\addcontentsline{toc}{section}{Литература}
\begin{thebibliography}{99}
\bibitem{1za}
\Au{Шрейдер~Ю.\,А.}
Информация и знание~// Системная концепция информационных процессов.~--- М.:
ВНИИСИ, 1988.  С.~47--52.
\bibitem{2za}
\Au{Gorn S.}
The computer and information sciences: A new basic discipline~// SIAM Review, 1963. Vol.~5.
No.\,2. P.~150--155.
\bibitem{3za}
\Au{Gorn S.}
Informatics (computer and information science): Its ideology, methodology, and sociology~// The
studies of information: Interdisciplinary messages~/ Eds. F.~Machlup, U.~Mansfield.~--- N.
Y.: Wiley, 1983. P.~121--140.

\bibitem{4za}
\Au{Мамардашвили М.\,К.}
Классический и неклассический идеалы рациональности.~--- Тбилиси: Мецниереба, 1984.

\bibitem{5za}
\Au{Зацман И.\,М.}
Семиотические основания и элементарные технологии информатики~// Информационные
технологии, 2005. №\,7. С.~18--31.

\bibitem{6za}
\Au{Полани М.}
Личностное знание.~--- М.: Прогресс, 1985. С.~257--258.

\bibitem{7za}
\Au{Клейнер Г.\,Б.}
Эволюция институциональных сис\-тем.~--- М.: Наука, 2004.

\bibitem{8za}
Введение в МПК-8. {\sf  http://www.fips.ru/ipc8/}\linebreak
{\sf intro/mpk8.htm}.

\bibitem{9za}
О новом порядке пересмотра и реализации МПК расширенного уровня.  {\sf
http://} {\sf www.fips.ru/russite/classificators/new.htm}.

\bibitem{11za}
\Au{Зацман И.\,М.}
Вербально-образное представление знаний в электронных библиотеках (Часть~II)~// 
Научно-техническая информация (серия~2 <<Информационные процессы и системы>>), 2001. №\,12.
С.~10--17.
{\looseness=-1

}

\bibitem{10za}
\Au{Зацман~И.\,М.}
Концептуальный поиск и качество информации.~--- М.: Наука, 2003.  271~с.

\bibitem{12za}
\Au{Agarwal P.}
Contested nature of ``place'': Knowledge \mbox{mapping} for resolving ontological distinctions between
geo\-graphical concepts~// Eds. M.~Egenhofer, C.~Freksa, M.~Harvey. 3rd International
Conference ``GIScience 2004''. LNCS 3234.~--- Berlin: Springer-Verlag, 2004.  P.~1--21.

\bibitem{13za}
\Au{Agarwal~P., Huang~Y., Dimitrova~V.}
Formal approach to reconciliation of individual ontologies for personalisation of geospatial
semantic Web~// Eds. M.~Rodriguez, I.~Cruz, M.~Egenhofer, S.~Levashkin. 1st International
Conference ``GeoS 2005''. LNCS 3799.~--- Berlin: Springer-Verlag, 2005. P.~195--210.

\bibitem{14za}
Decision No.\,1982/2006/EC of the European Parliament and of the Council of 18~December 2006
concerning the Seventh Framework Programme of the European Community for research,
technological development and demonstration activities (2007--2013)~// Official J.\ of the
European Union L412, 30.12.2006.  P.~1--41.

\bibitem{15za}
CORDIS ICT Programme Home. {\sf http://} {\sf cordis.europa.eu/fp7/ict/programme/home\_en.html}.

\bibitem{16za}
ICT FP7 Work Programme.
{\sf ftp://ftp.cordis.} {\sf europa.eu/pub/fp7/ict/docs/ict-wp-2007-08\_en.pdf}.

\bibitem{17za}
FP7 Exploratory Workshop 4 ``Knowledge Anywhere Anytime.''
{\sf http://cordis.europa.eu/ist/directorate\_f/} {\sf f\_ws4.htm}.

\bibitem{18za}
\Au{Buddenberg~R.}
Toward an interoperability reference model.
{\sf  http:// web1.nps.navy.mil/\~budden/}\linebreak {\sf lecture.notes/interop\_RM.html}.

\bibitem{19za}
\Au{Buddenberg R.}
FORCENet: We've been here before. {\sf
http://web1.nps.navy.mil/\~budden/lecture.notes/}\linebreak {\sf it\_arch/large\_info\_systems.html}.

\bibitem{20za}
\Au{Моисеев Н.\,Н.}
Универсум. Информация. Общество.~---  М.: Устойчивый мир, 2001.

\bibitem{21za}
\Au{Nonaka I.}
The knowledge-creating company~// Harvard Business Review, 1991. Vol.~69. No.\,6.
P.~96--104.

\bibitem{22za}
\Au{Nonaka~I., Takeuchi~H.}
The knowledge-creating company.~--- N. Y.: Oxford University Press, 1995. (Пер.: 
\Au{Нонака~И., Такеучи~Х.} Компания~--- создатель знания.~--- М.: ЗАО <<Олимп-
бизнес>>, 2003.)

\bibitem{23za}
\Au{Шемакин Ю.\,И., Романов~А.\,А.}
Компьютерная семантика.~--- М.: НОЦ <<Школа Китайгородской>>, 1995.

\bibitem{24za}
\Au{McArthur D.}
Information, its forms and functions: The elements of semiology.~--- Lewinton: The Edwin Mellen
Press, Ltd., 1997.

\bibitem{25za}
\Au{Колин К.\,К.}
Становление информатики как фундаментальной науки и комплексной научной проблемы~//
Системы и средства информатики. Спец. вып.\ <<Научно-методологические проблемы
информатики>>~/ Под ред. К.\,К.~Колина.~--- М.: ИПИ РАН, 2006. С.~7--58.

\bibitem{28za}
\Au{Зацман И.\,М.}
Концептуализация данных науко\-мет\-ри\-ческих исследований в научных электронных
биб\-лиотеках~// Труды 10-й Всероссийской конференции <<Электронные библиотеки:
перспективные методы и технологии, электронные коллекции>>.~--- Дубна: ОИЯИ, 2008 (в
печати).

\bibitem{26za}
\Au{Eco U.}
A theory of semiotics.~--- Bloomington: Indiana University Press, 1976.

\bibitem{27za}
\Au{Чебанов С.\,В.}
Новый этап становления общей семиотики: вклад техно- и биосемиотики~// Вестник \mbox{РФФИ},
2003. №\,4(34). С.~65--71.


\bibitem{29za}
\Au{Marwick A.\,D.}
Knowledge management technology~// IBM Systems Journal, 2001. Vol.~40. No.\,4.
P.~814--830.

\bibitem{30za}
\Au{Баранов А.\,Н., Добровольский Д.\,О.}
Аспекты теории фразеологии.~---  М.: Знак, 2008.

\bibitem{32za}
\Au{Гак~В.\,Г.}
Лексическое значение слова~// Большой энциклопедический словарь
<<Языкознание>>.~--- М.: Большая российская энциклопедия, 1998. С.~261--263.

\bibitem{31za}
\Au{Баранов А.\,Н.}
Против <<разложения смысла>>: узнавание в семантике идиом~// Компьютерная
лингвистика и интеллектуальные технологии: По материалам ежегодной Международной
конференции <<Диалог>>. Вып.~7(14).~--- М.: РГГУ, 2008.  С.~39--44.

\bibitem{33za}
\Au{Newman J.}
Some observations on the semantics of ``Information''~// Information Systems Frontiers, 2001.
Vol.~3. No.\,2. P.~155--167.

\bibitem{34za}
\Au{Ingwersen P.}
Information and information science~// Encyclopaedia of Library and Information Science.
Vol.~56. Sup.~19.~--- N. Y.: Marcel Dekker Inc., 1992.  P.~137--174.

\bibitem{35za}
\Au{Уфимцева А.\,А.}
Знак языковой~// Большой энциклопедический словарь <<Языкознание>>.~--- М.: Большая
российская энциклопедия, 1998.  С.~167.

\bibitem{36za}
\Au{Зацман И.\,М., Кожунова О.\,С.}
Семантический словарь системы информационного мониторинга в сфере науки: задачи и
функции~// Системы и средства информатики. Вып.~17.~--- М.: Наука, 2007.  С.~124--141.

\bibitem{37za}
\Au{Кожунова О.\,С., Зацман И.\,М.}
Прагматические аспекты создания семантического словаря терминов информационного
мониторинга~// Труды международной конференции Диалог-2007 <<Компьютерная\linebreak
лингвистика и интеллектуальные технологии>>.~--- М.: Издательский центр РГГУ, 2007.
С.~278--285.


\bibitem{38za}
\Au{Щедровицкий Г.\,П., Алексеев Н.\,Г.}
О возможных путях исследования мышления как деятельности~// Докл. АПН РСФСР, 1957.
№\,3; 1958. №№\,1, 4; 1959. №№\,1, 2, 4; 1960. №№\,2, 4--6; 1961. №№\,4, 5; 1962.\linebreak №№\,2--6.


\bibitem{39za}
\Au{Пископпель А.\,А.}
К творческой биографии Г.\,П.~Щед\-ро\-виц\-ко\-го (1929--1994)~// В кн.: Щедровицкий~Г.\,П.
Избранные труды.~--- М.: Шк. Культ. Полит., 1995. С.~XIII--XXXVII.

\label{end\stat}

\bibitem{40za}
Common Weakness Enumeration.  {\sf http://cwe.mitre.org/} {\sf index.html}.
\end{thebibliography}
}
}
\end{multicols}