
\renewcommand{\vec}{\boldsymbol}

\def\stat{korol}

\def\tit{МНОГОЛИНЕЙНАЯ СИСТЕМА МАССОВОГО ОБСЛУЖИВАНИЯ
С~КОНЕЧНЫМ НАКОПИТЕЛЕМ, БЛОКИРОВКОЙ ПОЛУМАРКОВСКОГО ПОТОКА
ЗАЯВОК И ВЫБИВАНИЕМ ЗАЯВОК ИЗ НАКОПИТЕЛЯ$^*$}
\def\titkol{Многолинейная система массового обслуживания
с~конечным накопителем}
%, блокировкой полумарковского потока
%заявок и выбиванием заявок из накопителя}
% к декомпозиции волатильности финансовых индексов}
\def\autkol{В.\,В.~Чаплыгин}
\def\aut{В.\,В.~Чаплыгин$^1$}

\titel{\tit}{\aut}{\autkol}{\titkol}

{\renewcommand{\thefootnote}{\fnsymbol{footnote}}\footnotetext[1]
{Работа выполнена при поддержке
Российского фонда фундаментальных исследований,
гранты 06-07-89056 и 08-07-00152.}}

\renewcommand{\thefootnote}{\arabic{footnote}}
\footnotetext[1]{Институт проблем
информатики Российской академии наук, vchaplygin@ipiran.ru}

\Abst{Рассматривается многолинейная система массового обслуживания (СМО) с
конечным накопителем, блокировкой полумарковского потока заявок и выбиванием
заявок из накопителя первой заявкой, поступившей в систему на периоде времени,
когда поток разблокирован. Периоды блокировки входящего потока и периоды, когда
входящий поток разблокирован, распределены по экспоненциальному закону с
разными интенсивностями. Найдены основные стационарные характеристики системы:
распределение очереди, вероятность потери заявки, среднее время пребывания
заявки в системе.}

\KW{система массового обслуживания; полумарковский поток заявок;
выбивание заявок}

      \vskip 36pt plus 9pt minus 6pt

      \thispagestyle{headings}

      \begin{multicols}{2}

      \label{st\stat}



\section{Описание системы}

В последние годы построению систем распределенных вычислений как
одному из перспективных направлений развития современных инфотелекоммуникационных систем
уделяется большое внимание.
Нередки случаи, когда доступ к ресурсам, выделенным системе распределенных вычислений, оказывается ограниченным:
например, когда обслуживающие устройства, на которых функционирует система,
задействованы для выполнения других задач.
Организация доступа к этим ресурсам может быть самой разнообразной:
приоритетный доступ, интервальный доступ, разделение ресурса между пользователями и др.
В частности, настоящая работа посвящена модели с интервальным доступом к
нескольким обслуживающим устройствам и является развитием работы \cite{C08},
в которой рассмотрена СМО SM/PH/$n$/$r$ с блокировкой полумарковского потока заявок.
Методы, используемые для отыскания стационарных характеристик рассмотренной ниже
системы с блокировкой полумарковского потока заявок
и выбиванием заявок из накопителя, опираются на идеи, заложенные в работах~[2--4]
для СМО с рекуррентным и полумарковским входящими потоками и марковским обслуживанием.

Рассмотрим многолинейную СМО с накопителем конечной емкости.
В системе имеется $n$ работающих независимо друг от друга идентичных приборов,
которые обслуживают поступающие на них однотипные заявки.

Каждый из приборов может находиться на одной из $J$, $1\le J <\infty$, фаз обслуживания.
Время обслуживания заявки на каждом приборе распределено
по закону фазового типа с параметрами $\vec h$ и $H$,
где $\vec h$~--- вектор-строка размерности $J$, а $H$~--- квадратная матрица порядка $J$.
Функция распределения фазового типа времени обслуживания заявки записывается в виде
(ниже через $\vec 1$ будем обозначать вектор-столбец из единиц, через $\vec 0$~--- нулевую вектор-строку,
через $O$~--- нулевую матрицу, а через $E$~--- единичную матрицу,
размерность и порядок которых определяются нижним индексом или из контекста):
\begin{equation}
\label{PH-Distribution}
H(x) = 1-\vec h\, e^{Hx} \vec1\,.
\end{equation}

Если в систему поступает заявка, но все $n$ приборов заняты,
то эта заявка поступает в накопитель емкостью $r$.
Будем считать, что $r\ge 2$; если $r=0$ или~1,
то часть полученных формул потребует упрощения.
Если накопитель полон,
то заявка, не обслуживаясь, покидает систему (теряется).
Заявки из накопителя обслуживаются в порядке их поступления в систему.
Обозначим: $R=n+r$.

Опишем входящий в систему поток заявок.

Рассмотрим полумарковским процесс с конечным множеством состояний $\{1,2,\ldots,I\}$, $1\le I<$\linebreak
$<\infty$.
В каждый момент изменения состояния полумарковского процесса генерируется новая заявка,
которая готова поступить в систему на обслуживание.
Вероятность того, что полумарковский процесс за время меньше $x$ перейдет из состояния~$i$
сразу в состояние~$j$, $i,j=\overline{1,I}$, равна $A_{ij}(x)$.
Среднее время между изменениями состояний полумарковского процесса в стационарном режиме можно записать в виде
%
%\noindent
\begin{equation}
a = \vec\pi_a \int\limits_0^\infty x\, dA(x)\, \vec1\,,
\label{a}
\end{equation}
где $\vec\pi_a$~--- вектор-строка стационарных вероятностей вложенной цепи Маркова полумарковского процесса,
$A(x)$~--- матрица из элементов $A_{ij}(x)$.

Более подробное описание полумарковского потока,
а также некоторые естественные дополнительные предположения относительно его параметров,
которые будем считать выполненными, можно найти в работе~\cite{PCh04}.

Будем рассматривать стационарный режим функционирования
полумарковского процесса генерации заявок. За <<малое>> время
$\Delta$ c вероятностью $\alpha \Delta+o(\Delta)$ происходит
блокировка потока заявок, поступающих в систему, а именно с
этого момента заявки, генерируемые полумарковским процессом, в
систему не попадают, а теряются. Заявки, находящиеся в системе,
систему не покидают, а продолжают обслуживаться (заявки на приборах)
или ожидать обслуживания (заявки в накопителе) до момента
поступления первой заявки после разблокировки потока.

Если поток
заблокирован, то за <<малое>> время~$\Delta$ c вероятностью $\beta
\Delta+o(\Delta)$ поток разблокируется и заявки, которые будут
сгенерированы после этого момента, вновь будут поступать в систему
на обслуживание. Причем первая заявка, поступившая в период времени,
когда поток разблокирован, выбивает все заявки из
накопителя, если таковые имеются, а сама занимает первое место в
очереди. Заявки на приборах продолжают обслуживаться.
Первая заявка, поступившая в период, когда входящий поток разблокирован,
и заставшая полный накопитель, не выбивает заявки из накопителя и сама
покидает систему, не обслужившись. Заявки, поступающие в
систему до момента блокировки процесса генерации,
также заявок из накопителя не выбивают,
а встают в очередь, если есть свободные места.

Генерация заявок полумарковским потоком не зависит от того, заблокировано поступление заявок в систему или нет.

В настоящей работе на основе методов, подробно изложенных в работах \cite{PCh03,PCh04}, получены
математические соотношения для расчета стационарных характеристик системы.

\section{Стационарное распределение числа заявок в системе}

Воспользуемся некоторыми построениями,
полученными для многолинейной СМО с конечным накопителем,
полумарковским потоком заявок и марковским обслуживанием \cite{PCh04}, а именно тем,
что процесс обслуживания всеми приборами системы,
обслуживание на каждом из которых распределено по закону фазового типа,
может быть описан в виде марковского процесса обслуживания следующим образом.

Если в системе находится $k$,\ \ $0\le k\le n+r$, заявок,
то процесс обслуживания может находиться в одном из $l_k$,\ \ $l_k<\infty$, состояний (фаз обслуживания),
причем интенсивность смены фаз марковского процесса определяется элементами
матриц $\Lambda_k$, $k=\overline{0,n+r}$, если ни одна заявка не обслужилась,
и элементами матриц $N_k$, $k=\overline{1,n+r}$, если заявка обслужилась.
Предполагается, что $l_k=l$ при $k=\overline{n,n+r}$,
$\Lambda_k = \Lambda$ при $k=\overline{n,n+r}$,
а $N_k = N$ при $k=\overline{n+1,n+r}$.
Матрицу $\Lambda+N$ будем предполагать неразложимой,
а матрицу $N$ --- ненулевой.

Если в системе находится $k$,\ \ $k=\overline{0,n-1}$, заявок, будем предполагать, что
при поступлении новой заявки в систему марковский процесс обслуживания переходит на фазу,
которая определяется элементами матриц $\Omega_k$.

Более подробное описание структуры такого марковского процесса,
а также способ формирования его инфинитезимальной матрицы по
заданному $PH$-распределению обслуживания заявки на каждом приборе
можно найти в работах \cite{PCh03,PCh04}.

Рассмотрим вложенную цепь Маркова, определяемую моментами смены фаз полумарковского процесса генерации заявок.

Обозначим через
$p^m_{ik}$, \ $m=0,1$,  $i=l_k(u-1)+v$,  $u=\overline{1,I}$,
$v=\overline{1,l_k}$,  $k=\overline{0,R}$,
стационарную вероятность того, что сразу после смены фаз полумарковского процесса в системе находится $k$ заявок,
фаза полумарковского процесса генерации заявок находится на фазе $u$ и
марковский процесс обслуживания находится на фазе $v$ и если $m=0$, то поток заявок заблокирован,
а если $m=1$, то поток заявок разблокирован.
Положим $\vec{p}^m_k=(p^m_{1k},\ldots,p^m_{Il_k,k})$,
$\vec{p}_k = (\vec{p}^0_k,\vec{p}^1_k)$,
 $m=0,1$,  $k=\overline{0,R}$,  $\vec{p}=(\vec{p}_0,\ldots,\vec{p}_R)$.

Для вектора $\vec{p}$ справедлива система уравнений равновесия (СУР):
\begin{equation}
\label{ES}
\vec{p}=\vec{p}P\,,
\end{equation}
в которой матрица $P$ переходных вероятностей вложенной цепи Маркова
представима в блочном виде:
\end{multicols}

\begin{gather*}
P=
\begin{pmatrix}
P_{00}    & P_{01}    & O         & O         & \ldots & O           & O\\
P_{10}    & P_{11}    & P_{12}    & O         & \ldots & O           & O\\
\vdots    & \vdots    & \vdots    & \vdots    & \ldots & O           & O\\
P_{R-1,0} & P_{R-1,1} & P_{R-1,2} & P_{R-1,3} & \ldots & P_{R-1,R-1} & P_{R-1,R}\\
P_{R0}    & P_{R1}    & P_{R2}    & P_{R3}    & \ldots & P_{R,R-1}   & P_{RR}\\
\end{pmatrix}\,,
\end{gather*}

\begin{multicols}{2}

\noindent
где $P_{i,j}$, $i=\overline{0,R}$, $j=\overline{0,\min\{i+1,R\}}$,~--- матрицы размера $l_i\times l_j$.

Чтобы получить математические соотношения для матриц $P_{ij}$, введем некоторые дополнительные обозначения.

Обозначим через $q_{ij}(x)$,  $i, j=0,1$, вероятность того,
что через время $x$ поступление заявок будет заблокировано, если $j=0$,
и поступление заявок будет разблокировано,
если $j=1$, при условии, что в начальный момент времени поступление заявок
заблокировано, если $i=0$, и разблокировано, если $i=1$.

Можно показать~\cite{C08}, что
\begin{gather*}
q_{00}(x)=\fr{\alpha}{\alpha+\beta}+\fr{\beta}{\alpha+\beta}\,
e^{-(\alpha+\beta)x}\,;\\
q_{01}(x)=\fr{\beta}{\alpha+\beta}-\fr{\beta}{\alpha+\beta}\,e^{-(\alpha+\beta)x}\,;\\
q_{10}(x)=\fr{\alpha}{\alpha+\beta}-\fr{\alpha}{\alpha+\beta}\,e^{-(\alpha+\beta)x}\,;\\
q_{11}(x)=\fr{\beta}{\alpha+\beta}+\fr{\alpha}{\alpha+\beta}\,e^{-(\alpha+\beta)x}.
\end{gather*}

Обозначим через $\Tilde q_1(x)$ вероятность того,
что за время $x$ состояние блокировки входящего потока заявок ни разу не поменяется при условии,
что в начальный момент времени поток заявок был разблокирован.
Через $\Tilde q_{11}(x)$ обозначим вероятность того,
что за время $x$ состояние блокировки поменяется хотя бы раз,
но к моменту $x$ останется разблокированным при условии,
что в начальный момент времени поток заявок был разблокирован.
Для функций $\Tilde q_1(x)$ и $\Tilde q_{11}(x)$ нетрудно получить следующие выражения:
\begin{align*}
\Tilde q_1(x)&=e^{-\alpha x}\,;\\
\Tilde q_{11}(x)&=\fr{\beta}{\alpha + \beta} +\fr{\alpha}{ \alpha +\beta}e^{-(\alpha+\beta)x} - e^{-\alpha x}\,.
\end{align*}
Очевидно, что
$$
\Tilde q_1(x)+ \Tilde q_{11}(x)= q_{11}(x)\,.
$$

Определим следующие вспомогательные матрицы (здесь $\otimes$ --- символ кронекерова произведения матриц):

\noindent
\begin{align}
A^{ij}_{k} &=
\int\limits^\infty_0 q_{ij}(x)\,d A(x)
\otimes F_{k}(x)\,,\notag\\
&\ \ \ \ \ \ \ \ \ \ \ \ \ \hspace*{20mm}  k\ge0, \  i,j=0,1\,;\label{A_1}\\
\Tilde A^{11}_k &=
\int\limits^\infty_0 \Tilde q_{11}(x)\,d A(x)
\otimes F_{k}(x),
\ \ k\ge0\,;\notag\\
\Tilde A^1_k & =
\int\limits^\infty_0 \Tilde q_1(x)\,dA(x)
\otimes F_{k}(x),
\ \ k\ge0\,;\notag
\\
A^{i1}_{kw} & =
\int\limits^\infty_0 q_{i1}(x)\,d A(x)
\otimes (F_{k,w}(x)\, \Omega_{w})\,,\notag\\
&\ \ \ \ \ \ \ \ \ \hspace*{8mm}  w=\overline{0,n}\,, \ \ k \ge w\,, \ \ i=0,1\,;\notag
\\
\Tilde A^{i1}_{kw} & =
\int\limits^\infty_0 q_{i1}(x)\,d A(x)
\otimes F_{k,w}(x)\,,\notag\\
& \ \ \ \ \ \ \ \ \ \hspace*{8mm} w=\overline{0,n}\,, \ \ k \ge w\,, \ \ i=0,1\,;\label{A_3}\\
A^{i0}_{kw} & =
\int\limits^\infty_0 q_{i0}(x)\,d A(x)
\otimes F_{k,w}(x)\,,\notag\\
& \ \ \ \ \ \ \ \ \hspace*{8mm} w=\overline{0,n}\,, \ \ k \ge w\,, \ \ i=0,1\,, \label{A_4}
\end{align}
где матрицы $F_k(x)$, $k\ge 0$, определяются соотношениями
\begin{align}
F_{0}(x)&= e^{\Lambda x}\,; \label{F_0} \\
F_{k}(x) & =
\int\limits^x_0
F_{k-1}(y)\,N\,F_{0}(x-y)\, dy,\ \ k\ge1\,,\label{F_k}
\end{align}
а $F_{kw}(x)$~--- матрица, элементы которой представляют собой вероятности того,
что за время $x$ обслужится ($k-w$) заявок при условии,
что в начальный момент в системе было ровно $k$ заявок
с учетом соответствующей смены фаз процессом обслуживания.

Для матриц $F_{kw}(x)$ также можно выписать рекуррентные соотношения,
аналогичные со\-от\-но\-шениям~(\ref{F_0}) и~(\ref{F_k}) для матриц $F_k(x)$,
которые в\linebreak точности будут совпадать с формулами,
полученными для  матриц $F_{kw}(x)$ при исследовании системы SM/MSP/$n$/$r$ в работе \cite{PCh04},
и поэтому они здесь не приводятся.
В работах~[2--4]
можно познакомиться с алгоритмическими методами численного расчета матриц $F_k(x)$ и $F_{kw}(x)$ и матриц,
аналогичных $A^{ij}_k(x)$ и $A^{ij}_{kw}(x)$.

Теперь матрицы $P_{ij}$ можно представить в виде
\begin{align}
\label{p_ij}
P_{i,i+1}&=
\begin{pmatrix}
O & A^{01}_{ii}\\[6pt]
O & A^{11}_{ii}\\
\end{pmatrix},\ \ i=\overline{0,n-1}\,,
\\
P_{i0}&=
\begin{pmatrix}
A^{00}_{i0} & O \\[6pt]
A^{10}_{i0} & O \\
\end{pmatrix}, \ \ i=\overline{0,R}\,;\\
P_{ij}&=
\begin{pmatrix}
A^{00}_{ij} & A^{01}_{i,j-1} \\[6pt]
A^{10}_{ij} & A^{11}_{i,j-1} \\
\end{pmatrix}\,,\notag \\
&\ \ \ \ \ \ \ \ \ i=\overline{1,R}, \ \ j=\overline{1,\min\{i,n-1\}}\,;
\end{align}
\begin{gather}
P_{n,n+1}=
\begin{pmatrix}
O & A^{01}_{0}\\[6pt]
O & A^{11}_{0}\\
\end{pmatrix}, \ \ \ \ \
P_{i,i+1}=
\begin{pmatrix}
O & O\\[6pt]
O & \Tilde A_0^1 \\
\end{pmatrix}, \notag\\
\ \ \ \  \  \ \ \ \hspace*{20mm} i=\overline{n+1,R-1}\,;
\end{gather}
\begin{align}
P_{i,n}& =
\begin{pmatrix}
A^{00}_{i-n} & A^{01}_{i,n-1} \\[6pt]
A^{10}_{i-n} & A^{11}_{i,n-1} \\
\end{pmatrix}\,, \ \ i=\overline{n,R}\,;\\
P_{i,n+1}&=
\begin{pmatrix}
A^{00}_{i-n-1} & \sum\limits_{j=0}^{i-n} A^{01}_{j}\\[6pt]
A^{10}_{i-n-1} & A^{11}_{i-n} + \sum\limits_{j=0}^{i-n-1} \Tilde A^{11}_{j}\\
\end{pmatrix}\,,\notag\\
&\ \ \ \ \ \ \ \ \ \hspace*{20mm} i=\overline{n+1,R-1}\,;\\
P_{R,n+1} &=
\begin{pmatrix}
A^{00}_{R-n-1} & \sum\limits_{j=1}^{R-n} A^{01}_{j}\\[6pt]
A^{10}_{R-n-1} & A^{11}_{R-n} + \sum\limits_{j=1}^{R-n-1} \Tilde A^{11}_{j}\\
\end{pmatrix}\,;
\\
P_{ij} &=
\begin{pmatrix}
A^{00}_{i-j} & O \\[6pt]
A^{10}_{i-j} & \Tilde A^{1}_{i-j+1} \\
\end{pmatrix}\,,\notag \\ 
 i&=\overline{n+2,R}, \ \ j=\overline{n+2,\min\{i,R-1\}}\,;
\\
P_{RR} &=
\begin{pmatrix}
A^{00}_{0} & A^{01}_{0}\\[6pt]
A^{10}_{0} & \Tilde A^{1}_{1}+A^{11}_{0}\\
\end{pmatrix}\,. \label{p_ij_RR}
\end{align}

Можно показать, что матрица $P$, образованная из
заданных по формулам (\ref{p_ij})--(\ref{p_ij_RR}) матриц $P_{i,j}$,
является неразложимой и непериодической.
Методы решения СУР с матрицей такого вида изложены в работах \cite{BDPS03,BDPS04}.


Обозначим через
$q^m_{ik}$, $m=0,1$, $i=l_k(u-1)+v$, $u=\overline{1,I}$, $v=\overline{1,l_k}$, $k=\overline{0,R}$,
стационарную вероятность того, что непосредственно до момента смены фаз процесса генерации заявок
в системе было $k$ других заявок,
полумарковский процесс генерации заявок находился на фазе $u$,
марковский процесс обслуживания --- на фазе $v$ и если $m=0$, то поступление заявок в систему было заблокировано,
а если $m=1$, то поступление заявок в систему было разблокировано.
Положим $\vec{q}^{m}_k=(q^{m}_{1k},\ldots,q^{m}_{Il_k,k})$,
$\vec{q}_k = (\vec{q}^0_k,\vec{q}^1_k)$, \ $m=0,1$, \ $k=\overline{0,R}$.

Для векторов $\vec{q}_k$, \ $k=\overline{0,R}$, можно записать следующие соотношения:
\begin{align}
\label{q_1}
\vec{q}_k &=\sum\limits_{j=k}^{R} \vec{p}_j \Tilde P_{jk},\ \ k=\overline{0,n-1}\,;
\\
\label{q_2}
\vec{q}_k &= \sum\limits_{j=k}^{R} \vec{p}_j \Tilde P_{j-k},\ \ k=\overline{n,R}\,,
\end{align}
где
\begin{align*}
\Tilde P_{ij} &=
\begin{pmatrix}
A_{ij}^{00} & \Tilde A_{ij}^{01} \\[6pt]
A_{ij}^{10} & \Tilde A_{ij}^{11} \\
\end{pmatrix},
\ \ i=\overline{0,R},\ \ j=\overline{0,\min\{i,n-1\}}\,;
\\
\Tilde P_k &=
\begin{pmatrix}
A_k^{00} & A_k^{01} \\
A_k^{10} & A_k^{11} \\
\end{pmatrix},
\ \ k\ge 0\,,
\end{align*}
а матрицы $A_k$, $\Tilde A^{i1}_{kw}$  и $A^{i0}_{kw}$ определяются, в свою очередь,
соотношениями (\ref{A_1})--(\ref{A_4}).

Отыскав векторы $\vec{q}_k$, $k=\overline{0,R}$, можно вычислить вероятность $\pi_1$
потери новой заявки, а именно вероятность того, что заявка в момент генерации
не встанет на прибор и не попадет в накопитель, а сразу покинет систему.
Заявка в момент поступления теряется, если входящий в систему поток заблокирован или
если в накопителе отсутствуют свободные места. Поэтому
\begin{equation}
\label{pi_1}
\pi_1 = \sum\limits_{i=0}^{R-1}\vec{q}^0_k {\bf 1}+ \vec{q}_R {\bf 1}\,.
\end{equation}
Для стационарной вероятности $\pi_2$ того, что в момент генерации заявка не попадет в накопитель,
можно получить следующее выражение:
\begin{equation}
\label{pi_2}
\pi_2 =\sum\limits_{i=0}^{n-1}\vec{q}_k {\bf 1}+\sum\limits_{i=n}^{R-1}\vec{q}^0_k {\bf 1}+\vec{q}_R {\bf 1}\,.
\end{equation}

Найдем вероятность $\psi_1$ того, что заявка будет обслужена при условии,
что она попала в систему (на приборы или в накопитель).
Если заявка попадает сразу на прибор, то она не может быть потеряна и будет обслужена в любом случае.
Если же заявка попадает в накопитель, то она должна успеть обслужиться до тех пор, пока в
систему после следующего момента разблокировки входящего потока не придет очередная заявка.

Обозначим через
$\hat p_{ik}^m(j)$, \ $i=l(u-1)+v$, \ $u=\overline{1,I}$,  $v=\overline{1,l}$, \ $k=\overline{1,r}$, \ $j\ge 0$,
стационарную вероятность того, что произвольная заявка
попадет в накопитель и после $j$-го момента смены фаз полумарковского процесса
останется в накопителе на $k$-м месте, причем
полумарковский процесс генерации заявок будет находиться на фазе $u$,
марковский процесс обслуживания~--- на фазе $v$ и
если $m=0$, то поступление заявок в систему было заблокировано,
а если $m=1$, то поступление заявок в систему было разблокировано.
Введем вектор-строки $\vec{\hat p}_k(j)$, \ $k=\overline{1,r}$, \ $j\ge 0$, следующим образом:
$$
\vec{\hat p}_k(j)=(\hat p_{1k}^0(j),\ldots,\hat p_{Il,k}^0(j),\hat p_{1k}^1(j),\ldots,\hat p_{Il,k}^1(j))\,.
$$

Обозначим через $\vec{\hat p}(j)$, \ $j\ge 0$,
вектор-строку $\vec{\hat p}(j)=(\vec{\hat p}_{1}(j),\ldots,\vec{\hat p}_{r}(j))$.
Заметим, что вектор $\vec{\hat p}(0)$ представляет собой вектор стационарных вероятностей того,
что новая заявка попадет в накопитель при соответствующих значениях
фаз процессов генерации и обслуживания.

Рассмотрим блочную матрицу
\begin{multline}
\label{hat_P}
\hat P={}\\
{}=
\begin{pmatrix}
\hat P_{n+1,n+1}\!\! &\!\! O      \!\!          &\!\! \ldots\!\! &\!\! O \!\!               &\!\! O\\
\hat P_{n+2,n+1}\!\! &\!\! \hat P_{n+2,n+2}\!\! &\!\! \ddots\!\! &\!\! O   \!\!             &\!\! O\\
\vdots          \!\! &\!\! \vdots          \!\! &\!\! \ddots\!\! &\!\! \ddots\!\!                &\!\! \vdots\\
\hat P_{R-1,n+1}\!\! & \!\! \hat P_{R-1,n+2}\!\! &\!\! \ldots\!\! &\!\! \hat P_{R-1,R-1}\!\! &\!\! O\\
\hat P_{R,n+1}  \!\! & \!\! \hat P_{R,n+2}  \!\! &\!\! \ldots\!\! &\!\! \hat P_{R,R-1}  \!\! &\!\! \hat P_{RR}\\
\end{pmatrix},
\end{multline}
где
\begin{align*}
\hat P_{i,j}&=
\begin{pmatrix}
A^{00}_{i-j} & O \\[6pt]
A^{10}_{i-j} & \Tilde A^{1}_{i-j} \\
\end{pmatrix}\,, 
\\
&\ \ \ i=\overline{n+1,R}, \ \ j=\overline{n+1,\min\{i,R-1\}}\,;
\\
\hat P_{R,R}&=
\begin{pmatrix}
A^{00}_{0} & A^{01}_{0}\\[6pt]
A^{10}_{0} & \Tilde A^{1}_{1}+A^{11}_{0}\\
\end{pmatrix}\,.
\end{align*}

Пусть очередная заявка, поступив в систему, попала в накопитель.
Место в накопителе, на которое попадет заявка, определяется векторами, составляющими вектор $\vec{\hat p}(0)$, ---
вероятность того, что она окажется в накопителе на $k$-м месте,
определяется вектором $\vec{\hat p}_{k}(0)$, \ $k=\overline{1,r}$.
Тогда вероятность того, что сразу после следующего момента смены фазы процесса генерации
она останется в накопителе (не будет выбита и не попадет на прибор), равна
$\vec{\hat p}(1)=\vec{\hat p}(0)\hat P$,
причем место, которое будет занимать заявка в накопителе, фазы процессов генерации и обслуживания,
определяются координатой вектора $\vec{\hat p}(1)$.
Вероятность того, что эта заявка останется в накопителе и после $j$-й, $j\ge 1$, смены фаз процесса генерации, равна
\begin{gather*}
\vec{\hat p}(j)=\vec{\hat p}(0)\hat P^j, \ \ j\ge 1\,.
\end{gather*}
Заметим, что матрица $\hat P$ является неотрицательной и
сумма элементов в каждой строке строго меньше единицы.
Поэтому максимальное по модулю собственное значение этой матрицы также меньше единицы,
что гарантирует существование матрицы $(E-\hat P)^{-1}$.

Пусть заявка после $j$-й смены фазы процесса генерации находится на $k$-м месте в накопителе
(фазы процессов генерации и обслуживания, а также состояние блокировки процесса генерации
определяются соответствующими координатами вектора $\vec{\hat p}_{k}(j)$, \ $k=\overline{1,r}$, $j\ge0$).
Чтобы эта заявка попала на прибор до очередного момента смены фаз процесса генерации,
за время между сменами фаз должно обслужиться не менее $k$ заявок.
Вероятность этого события равна
$
\vec{\hat p}_{k}(j) \vec{p}_k^*
$, где
$$
\vec{p}_k^*=
\begin{pmatrix}
\sum\limits_{i=0}^{n}(A_{n+k,i}^{00}{\bf 1}+A_{n+k,i}^{01}{\bf 1})\\[6pt]
\sum\limits_{i=0}^{n}(A_{n+k,i}^{10}{\bf 1}+A_{n+k,i}^{11}{\bf 1})
\end{pmatrix}\,.
$$
Вектор $\vec{p}_k^*$ представляет собой вектор стационарных вероятностей того,
что заявка попадет на прибор до очередного момента смены фаз процесса генерации заявок при условии,
что сразу после предыду\-ще\-го момента смены фаз она находилась на $k$-м месте в накопителе.
Обозначая через $\vec{p}^*$ вектор $\vec{p}^*=(\vec{p}_1^*,\ldots,\vec{p}_r^*)$,
получаем, что вероятность того, что заявка, попав в накопитель, затем попадет на прибор
между $j$-м и $(j+1)$-м моментами смены фаз процесса генерации, равна
$$
\vec{\hat p}(j)\vec{p}^*=\vec{\hat p}(0)\hat P^{j}\vec{p}^*\,.
$$

Вероятность того, что заявка, попав в накопитель, затем (когда-нибудь) попадет на прибор
и не будет выбита другой заявкой, равна
$$
\sum\limits_{j=0}^{\infty} \vec{\hat p}(j) \vec{p}^*=
\sum\limits_{j=0}^{\infty} \vec{\hat p}(0) \hat P^{j} \vec{p}^*=
\vec{\hat p}(0) (E-\hat P)^{-1} \vec{p}^*\,.
$$

Теперь осталось найти векторы $\vec{\hat p}_k(0)$, $k=\overline{1,r}$,
тем самым определив вектор $\vec{\hat p}(0)$.
При $k=\overline{1,r-1}$ имеет место соотношение
\begin{align}
\label{hat_vec_p_1}
\vec{\hat p}_k(0)&=({\vec 0}_{Il},\vec{p}^1_{n+k}), \ k=\overline{1,r-1}\,;
\\
\label{hat_vec_p_2}
\vec{\hat p}_r(0)&=({\vec 0}_{Il},\vec{q}^1_{n+r-1})\,,
\end{align}
где векторы $\vec{p}^1_{n+k}$ являются составляющими векторов $\vec{p}_{n+k}$,
которые определяются из СУР (\ref{ES}) для вложенной цепи Маркова,
а вектор $\vec{q}^1_{n+r-1}$ --- из формулы (\ref{q_2}).

Вероятность $\psi_1$ того, что заявка, попав в систему, будет обслужена, равна
\begin{equation}
\label{psi_1}
\psi_1=\fr{1}{1-\pi_1}\left(\sum\limits_{k=1}^{n} \vec{p}^1_{k}{\bf 1}+
\vec{\hat p}(0) (E-\hat P)^{-1} \vec{p}^*\right)\,.
\end{equation}

Найдем вероятность $\psi_2$ того, что заявка, попав в систему, будет выбита другой заявкой.
Вероятность того, что заявка, находящаяся в накопителе на $k$-м месте после $j$-й смены фаз процесса генерации
с момента ее поступления в систему, будет выбита заявкой,
поступившей в следующий момент смены фаз процесса генерации, равна
$
\vec{\hat p}_{k}(j) \bar{\vec{p}}_k$, где
\begin{align*}
\bar{\vec{p}}_k&=
\begin{pmatrix}
\sum\limits_{j=0}^{k-1} A^{01}_{j}{\bf 1}\\[6pt]
\sum\limits_{j=0}^{k-1} \Tilde A^{11}_{j} {\bf 1}\\
\end{pmatrix}\,,\\
&\ \ \ \ \  k=\overline{1,r-1};\\
 \bar{\vec{p}}_r &=
\begin{pmatrix}
\sum\limits_{j=1}^{k-1} A^{01}_{j}{\bf 1}\\[6pt]
\sum\limits_{j=1}^{k-1} \Tilde A^{11}_{j} {\bf 1}\\
\end{pmatrix}\,.
\end{align*}

Обозначим через $\bar{\vec{p}}$ вектор-столбец 
$$
\bar{\vec{p}}= (\bar{\vec{p}}_1^{T}, \ldots, \bar{\vec{p}}_r^{T})^{T}
$$
и через $\bar v_{j+1}=\vec{\hat p}(j) \bar{\vec{p}}$, $j\ge 0$, вероятность того,
что заявка будет выбита другой заявкой в $(j+1)$-й момент смены фаз процесса генерации.
Тогда для стационарной вероятности $\psi_2$ того, что заявка будет выбита другой заявкой при условии,
что она вообще попадет в систему, можно выписать соотношение
\begin{multline}
\psi_2=\fr{1}{1-\pi_1}\sum\limits_{j=0}^{\infty} \bar v_{j+1}={}\\
{}=
\fr{1}{1-\pi_1}\,\vec{\hat p}(0) (E-\hat P)^{-1} \bar{\vec{p}}\,.\label{psi_2}
\end{multline}

Нетрудно показать, что вероятности $\psi_1$ и $\psi_2$ связывает очевидное соотношение
\begin{equation}
\label{psi_12}
\psi_1+\psi_2=1\,.
\end{equation}
Действительно, складывая соотношения (\ref{psi_1}) и (\ref{psi_2}), получаем
\begin{multline}
\psi_1+\psi_2={}\\
\hspace*{-5pt} {}= \fr{1}{ 1-\pi_1}\left(\sum\limits_{k=1}^{n} \vec{p}^1_{k}{\bf 1}+
\vec{\hat p}(0) (E-\hat P)^{-1} (\vec{p}^*+\bar{\vec{p}})\right).\!\!\!\label{psi_12_2}
\end{multline}
Вспоминая выражение (\ref{hat_P}) для матрицы $\hat P$ и определения векторов $\vec{p}^*$ и $\bar{\vec{p}}$,
а также учитывая стохастичность матрицы $P$ переходных вероятностей вложенной цепи Маркова,
можно записать следующее соотношение:
$$
\vec{p}^* + \bar{\vec{p}}=(E-\hat P){\bf 1}\,.
$$
Подставляя эту формулу в соотношение~(\ref{psi_12_2}) и учитывая формулы~(\ref{q_1})--(\ref{pi_1}),
нетрудно получить требуемое равенство~(\ref{psi_12}).


Весьма полезной характеристикой системы является время, потраченное выбитой заявкой
на ожидание обслуживания в накопителе.
Для среднего времени $\bar v$ пребывания заявки в накопителе с момента ее поступления в систему при условии,
что она будет выбита другой заявкой, справедливо соотношение
\begin{multline*}
\bar v =\fr{a}{1-\pi_2} \sum_{j=1}^{\infty}j\bar v_j =
\fr{a}{ 1-\pi_2} \vec{\hat p}(0) \sum_{j=1}^{\infty} j\hat P^{j-1} \bar{\vec{p}}={}\\
{}=
\fr{a}{ 1- \pi_2} \vec{\hat p}(0) (E-\hat P)^{-2} \bar{\vec{p}}\,,
\end{multline*}
где среднее время $a$ между соседними сменами фаз процесса генерации заявок определяется формулой~(\ref{a}),
матрица $\hat P$~--- формулой (\ref{hat_P}), вектор $\vec{\hat p}(0)$~---
формулами~(\ref{hat_vec_p_1}) и~(\ref{hat_vec_p_2}), а вероятность $\pi_2$~---  формулой~(\ref{pi_2}).

Найдем среднее время $w^*$ пребывания заявки в накопителе до момента ее поступления на прибор.
Вероятность того, что заявка окажется на $k$-м месте в накопителе
после \mbox{$j$-го} момента смены фаз процесса генерации,
определяется координатами вектора $\vec{\hat p}_k(j)$. Тогда среднее время,
проведенное заявкой в накопителе до этого момента, определяется координатами вектора $ja\vec{\hat p}_k(j)$.

Введем вектор-столбец 
$$
\vec{a}=(\vec{a}_1^T,\ldots,\vec{a}_r^T)^T\,,
$$ 
где
$\vec{a}_k$, $k=\overline{1,r}$,~--- вектор-столбец, координатами которого являются средние времена,
проведенные до попадания на прибор заявкой, находившейся в накопителе на $k$-м месте в очередной момент смены фаз процесса генерации с учетом фаз процессов генерации и обслуживания,
при условии, что она попадет на прибор до следующего момента смены фаз процесса генерации.
Для векторов $\vec{a}_k$ справедливы соотношения
$$
\vec{a}_k=\int\limits_{0}^{\infty} (E-A(x))\otimes xF_{k-1}(x) N dx \vec{1}, \ \ k=\overline{1,r}\,,
$$
где матрицы $F_k(x)$, $k\ge 0$, вычисляются с помощью формул~(\ref{F_0}) и~(\ref{F_k}).

Теперь можем записать выражение для среднего времени $w^*$ ожидания начала обслуживания
попавшей в накопитель заявки при условии, что она не будет выбита:

\noindent
\begin{multline*}
w^*=\fr{1}{1-\pi_2}\sum\limits_{j=0}^{\infty} \vec{\hat p}(j) (ja{\bf 1}+\vec{a})={}\\
{}=
\fr{1}{1-\pi_2}\vec{\hat p}(0)(E-\hat P)^{-1} \bigl( a\hat P(E-\hat P)^{-1}{\bf 1}+ \vec{a}\bigr)\,.
\end{multline*}

Поскольку время ожидания начала обслуживания заявки, попавшей сразу на приборы, равно нулю, то
среднее время $w$ ожидания начала обслуживания попавшей в систему заявки, не выбитой другими заявками,
можно записать в виде
$$
w=\fr{1}{ 1-\pi_1}
\vec{\hat p}(0)(E-\hat P)^{-1} \bigl( a\hat P(E-\hat P)^{-1}{\bf 1}+ \vec{a}\bigr)\,.
$$

Время пребывания попавшей в систему заявки, не выбитой другими заявками,
представляет собой сумму времени ожидания начала обслуживания и времени обслуживания заявки,
при условии ее попадания на прибор. Поэтому среднее время $v$ пребывания заявки в системе
можно отыскать по формуле
\begin{multline*}
v=\fr{1}{ 1-\pi_1}
\left[b\sum\limits_{k=1}^{n} \vec{p}^1_{k}{\bf 1}+{}\right.\\
{}+\left. \vec{\hat p}(0)(E-\hat P)^{-1} \bigl( a\hat P(E-\hat P)^{-1}{\bf 1}+ \vec{a}+ b{\bf 1}\bigr)
\right ]\,,
\end{multline*}
где $b = -\vec{h} H^{-1}{\bf 1}$ --- среднее время обслуживания заявки,
а $\vec{h}$ и $H$ --- параметры распределения (\ref{PH-Distribution}) фазового типа на отдельном приборе.

Таким образом, в настоящей работе найдены следующие показатели функционирования
в стационарном режиме СМО с прерывающимся блокировкой полумарковским входящим потоком %\linebreak 
заявок,
конечным накопителем, обслуживанием\linebreak
 фазового типа на каждом приборе и выбиванием заявок из накопителя:
\begin{itemize}
\label{end\stat}
\item распределение числа заявок в системе;
\item
 вероятность непопадания заявки в систему;
\item
 вероятность того, что заявка, попав в накопитель, будет обслужена;
\item
 среднее время пребывания выбитой заявки в накопителе;
\item
 среднее время ожидания начала обслуживания заявки,
попавшей в систему и не выбитой другой заявкой;
\item
 среднее время пребывания в системе этой за\-явки.
\end{itemize}

%\renewcommand{\refname}{Литература}

{\small\frenchspacing
{%\baselineskip=10.8pt
\addcontentsline{toc}{section}{Литература}
\begin{thebibliography}{99}


\bibitem{C08}
\Au{Чаплыгин В.\,В.}
Многолинейная система массового обслуживания
с конечным накопителем и блокировкой полумарковского потока заявок~//
Информационные процессы, 2008. Т.~8. №\,1.  С.~1--9.

\bibitem{PCh03}
\Au{Печинкин А.\,В., Чаплыгин В.\,В.}
Стационарные характеристики системы массового обслуживания G/MSP/$n$/$r$~//
Вестник РУДН, сер. <<Прикладная математика и информатика>>,
2003. №\,1. С.~119--143.

\bibitem{BDPS03}
\textit{Бочаров П.\,П., Д'Апиче~Ч., Печинкин~А.\,В., Салерно~С.}
Система массового обслуживания G/MSP/1/$r$~//
Автоматика и телемеханика, 2003. №\,2. С.~127--143.

\bibitem{PCh04}
\Au{Печинкин А.\,В., Чаплыгин В.\,В.}
Стационарные характеристики системы массового обслуживания SM/MSP/$n$/$r$~//
Автоматика и телемеханика, 2004. №\,9. С.~85--100.


\bibitem{BDPS04}
\textit{Bocharov~P.\,P., D'Apice~C., Pechinkin~A.\,V., Salerno~S.}
Queueing theory.~--- Utrecht, Boston: VSP, 2004.

\end{thebibliography}
}
}
\end{multicols}