\def\stat{abstr}
{%\hrule\par
%\vskip 7pt % 7pt
\raggedleft\Large \bf%\baselineskip=3.2ex
A\,B\,S\,T\,R\,A\,C\,T\,S \vskip 17pt
    \hrule
    \par
\vskip 21pt plus 6pt minus 3pt }

\def\tit{STRUCTURAL MATRIX SYSTEMS DECOMPOSITION}

%1
\def\aut{A.~Olenin}

\def\auf{IPI RAN, aolenin@yandex.ru}

\def\leftkol{\ } % ENGLISH ABSTRACTS}

\def\rightkol{\ } %ENGLISH ABSTRACTS}

\titele{\tit}{\aut}{\auf}{\leftkol}{\rightkol}


\noindent
One of possible approaches to matrix systems parallelizing by their structural
decomposion on the set of  subsystems independent at the certain stage of calculations is
considered. The constructive algorithm of the decomposition method is formulated and the
estimation of its computing expenses is given.

\label{st\stat}

 \KWN{matrix system; band matrix; full matrix; triangular matrix; block tridiagonal
 matrix; decomposition; partition vector; partition interval; factorization;
 LU-algorithm; parallelizing
}

\vskip 14pt plus 6pt minus 3pt


\def\tit{CONCURRENT DESIGN AND VERIFICATION OF DIGITAL HARDWARE}

%2
\def\aut{S.~Baranov$^1$,  S.~Frenkel$^2$,
V.~Sinelnikov$^3$, and V.~Zakharov$^4$}
\def\auf{$^1$Holon Institute of Technology, Holon, Israel, samary@012.net.il\\[1pt]
$^2$IPI RAN\\[1pt]
$^3$Holon Institute of Technology, Holon, Israel\\[1pt]
$^4$IPI RAN, VZakharov@ipiran.ru}

\def\leftkol{\ } % ENGLISH ABSTRACTS}

\def\rightkol{\ } %ENGLISH ABSTRACTS}

\titele{\tit}{\aut}{\auf}{\leftkol}{\rightkol}

\noindent
The main goal of this paper is to present a new design verification methodology for complicated digital
systems, designed by high-level synthesis. This methodology is based on Algorithmic State Machine 
transformations (composition, minimization, extraction, etc.), special algorithms for Data Path
and Control Unit 
design, and very fast optimizing synthesis of finite state machines  and combinational circuits with hardly any
constraints on their size, that is, the number of inputs, outputs,
and states. Design tools supporting this methodology
allow very fast implement, check and estimate many possible design versions, to find an optimized decision of the
design problem and to simplify the verification problem for digital systems.
 In contrast to existent semi-formal approaches to verification of industrial systems, based on combination of
simulation and formal verification approaches,  a formalized method based on concurrency of synthesis
and verification that is providing regular efficient way to verify the system designed properties starting from its
semi-formal specification up to field programmable gate array 
implementation is considered.

%\label{st\stat}

\KWN{digital systems design; formal verification; finite state machine
}

%\pagebreak

% \thispagestyle{headings}

\vskip 14pt plus 6pt minus 3pt

%\vfil

%3
\def\tit{COST FUNCTION OF RESOURCES IN ECONOMICAL MODEL 
OF~GRID CONTROL}

\def\aut{Y.\,M. Agalarov}
\def\auf{IPI RAN, YAgalarov@ipiran.ru}

\titele{\tit}{\aut}{\auf}{\leftkol}{\rightkol}

\noindent 
A problem of profit maximization for owner of GRID local node resources with economical
model of control, in which the assigned resources for external user are paid and payment amount is dependent
on supply and demand of resources is considered. In the model, the queue of global tasks is 
formed only in GRID resources programming center
that performs search and selection of resources, and task sending to
resources is performed simultaneously with it reservation and selection. 
A cost function of resources that allows the
owner to efficiently allocate the resources of GRID local node between global
and local tasks is proposed. The results of analytical investigation of the
considered problem and computer-modeling-based comparative
analysis of the proposed approach are presented.



\KWN{GRID; model resourse allocation; multiprocessor tasks; owner of resources; Markov process;
strategy}
%\pagebreak


%\vfil
 \vskip 18pt plus 6pt minus 3pt
% \vskip 24pt plus 9pt minus 6pt

%4
\def\tit{MULTICHANNEL QUEUEING SYSTEM WITH A FINITE BUFFER, A LOCKOUT OF AN INPUT FLOW,
AND~A~KNOCKOUT OF CUSTOMERS FROM THE BUFFER
}

\def\aut{V.~Chaplygin}
\def\auf{IPI RAN, vchaplygin@ipiran.ru
}


%\def\leftkol{ENGLISH ABSTRACTS}

%\def\rightkol{ENGLISH ABSTRACTS}

\titele{\tit}{\aut}{\auf}{\leftkol}{\rightkol}

\noindent
A multichannel queueing system with a finite buffer, a lockout of a semi-Markovian input
flow, and a knockout of customers from the buffer by a first customer arrived into the system over the
period when the input flow is unlocked is considered. The periods of lockout and the periods when
the input flow is unlocked have an exponential distribution with different intensities. The main
stationary characteristics such as a queue distribution, a loss probability, and a mean sojourn time are
found.

\KWN{queueing system; semi-Markovian input flow;
knockout of customers
}
%\pagebreak

%\vful

 \vskip 12pt plus 6pt minus 3pt

% \vskip 24pt plus 9pt minus 6pt
%\vskip 6pt plus 3pt minus 3pt
%\vspace*{12pt}

%5
\def\tit{SERVICE-ORIENTED APPROACH TO MULTIMODAL BIOMETRICS 
DESIGNING}
%INFORMATION TECHNOLOGY OF USING FACIAL BIOMETRICS FOR~INCREASING~AFIS~THROUGHPUT}

\def\aut{O.\,S.~Ushmaev}

\def\auf{IPI RAN, oushmaev@ipiran.ru}

\def\leftkol{ENGLISH ABSTRACTS}

\def\rightkol{ENGLISH ABSTRACTS}

\titele{\tit}{\aut}{\auf}{\leftkol}{\rightkol}

\noindent
Novadays, multimodal biometrics is rapidly replacing tedious procedures of identification. 
Particularly operating and perspective civil ID systems use multimodal approach. The formal method 
for designing high-speed multibiometric technologies and systems 
is suggested. The effectiveness 
of the approach is shown by an example of developed experimental software with service-oriented architecture.
   
%The task of designing of multimodal face-finger biometric system is considered. 
%Technique for increasing of throughput of matching procedures is developed. The conducted experiments reveal high
%effectiveness of the proposed techniques.

%\KWN{biometric identification; multimodal biometrics; fingerprint biometrics;
%facial  biometrics; throughput optimization}
\KWN{biometric identification; multimodal biometrics; platform independent; service-oriented 
architecture}

%\vskip 18pt plus 6pt minus 3pt

 \vskip 12pt plus 6pt minus 3pt

% \pagebreak

%6
\def\tit{PERSONAL AND COLLECTIVE CONCEPTS REPRESENTATION IN THE DIGITAL SPHERE
}

\def\aut{I.\,M.~Zatsman$^1$, V.\,V.~Kosarik$^2$, and O.\,A.~Kurchavova$^3$}
\def\auf{$^1$IPI RAN, im@a170.ipi.ac.ru\\[1pt]
$^2$IPI RAN, valery@a170.ipi.ac.ru\\[1pt]
$^3$IPI RAN, koa@a170.ipi.ac.ru}

\titele{\tit}{\aut}{\auf}{\leftkol}{\rightkol}

\noindent
Key issues of the 7th Framework program documents of the European
Union accepted for the period 2007--2013 are analyzed. This program contains
formulations of some new directions referring to the knowledge representation
problem in information systems of long-term use. The results of the analysis allow
us to assert that simultaneously with a traditional informatics problem covering
representation, storage, and extraction in the digital sphere of already available
knowledge and meeting technological economic, educational, and other socially significant
challenges, there is a problem of generation of new goal-oriented knowledge systems
when available knowledge systems do not satisfy these challenges and
from this point of view are incomplete. The new way to personal and
collective concepts respresentation in the digital sphere and also stages of concepts
evolution in the process of generation of new goal-oriented  knowledge systems is
offered.

\KWN{personal, collective and conventional concepts; lasting and volatile
concepts; personal and collective concepts representation in
the digital sphere}
\pagebreak

\vskip 12pt plus 6pt minus 3pt

%7
\def\tit{TOWARDS THE 25TH ANNIVERSARY OF IPI RAN
}
\def\aut{I.\,A.~Sokolov$^1$ and V.\,N.~Zakharov$^2$}

\def\auf{$^1$IPI RAN, isokolov@ipiran.ru\\[1pt]
$^2$IPI RAN, vzakharov@ipiran.ru}

%\def\leftkol{ENGLISH ABSTRACTS}

%\def\rightkol{ENGLISH ABSTRACTS}

\titele{\tit}{\aut}{\auf}{\leftkol}{\rightkol}

\noindent
The history of IPI RAN creation, its general characteristics,
main stages of development are presented. The evolution of the
main directions of scientific researches is considered. Main
fundamental and applied results obtained during 25 years are
described.

\KWN{Institute of Informatics Problems RAS; 25 years;
history; directions of scientific researches
}

 \label{end\stat}
 %\pagebreak