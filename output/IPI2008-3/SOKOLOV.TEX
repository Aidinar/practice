\def\stat{sokolov}


\def\tit{К 25-ЛЕТИЮ ИНСТИТУТА ПРОБЛЕМ ИНФОРМАТИКИ РАН$^*$}
\def\titkol{К 25-летию института проблем информатики РАН}
\def\autkol{И.\,А. Соколов, В.\,Н. Захаров}
\def\aut{И.\,А. Соколов$^1$, В.\,Н. Захаров$^2$}

\titel{\tit}{\aut}{\autkol}{\titkol}

{\renewcommand{\thefootnote}{\fnsymbol{footnote}}\footnotetext[1]
{Сокращенный вариант статьи, опубликованной в журнале <<История науки и
техники>>, 2008, №\,7.}}

\renewcommand{\thefootnote}{\arabic{footnote}}
\footnotetext[1]{Институт проблем
информатики Российской академии наук, isokolov@ipiran.ru}
\footnotetext[2]{Институт проблем
информатики Российской академии наук, vzakharov@ipiran.ru}

\Abst{Представлена история создания ИПИ РАН, дана общая
характеристика института, отражены основные этапы его
развития. Рассмотрена эволюция основных направлений
исследований института, охарактеризованы важнейшие
полученные за 25~лет фундаментальные и прикладные
результаты.}

\KW{ИПИ РАН; 25~лет; история; направления исследований}

      \vskip 24pt plus 9pt minus 6pt

      \thispagestyle{headings}

      \begin{multicols}{2}

      \label{st\stat}


\section{История создания и краткая справка об институте.
Основные научные направления}

В середине 1983~г.\ руководством СССР было принято решение о расширении и 
усилении исследований в области информатики и вычислительной техники в рамках 
Академии наук СССР. С целью реализации данного решения в составе Академии наук 
СССР было образовано Отделение информатики, вычислительной техники и 
автоматизации. Главными инициаторами подготовки и принятия решений по 
организации нового Отделения были президент АН СССР Анатолий Петрович 
Александров и ви\-це-пре\-зи\-дент АН СССР Евгений Павлович Велихов, который и 
возглавил новое Отделение. В~состав Отделения вошли Институт прикладной 
математики АН СССР, Вычислительный центр АН СССР, Институт проблем передачи 
информации АН СССР, а также несколько вновь образованных институтов. 29~июля 
1983~г.\ было принято постановление ЦК КПСС и Совета Министров СССР об 
образовании Института проблем информатики АН СССР (ИПИАН). Соответствующее 
Распоряжение Президиума АН СССР было датировано 2~августа 1983~г.

     В начале 1980-х гг.\ появляются и стремительно завоевывают
новые области применения массовые средства вычислительной
техники~--- персональные компьютеры, существенное развитие
получают научные основы информатики. Однако
<<ведомственность>> организаций, занимающихся проблемами
информатизации, служила тормозом для развития информатики в
СССР.

     Именно с целью преодоления <<ведомственных>> барьеров
для развития методов и средств информатизации и было образовано
новое Отделение АН СССР, а член-кор\-рес\-пон\-дент АН СССР Б.\,Н.~Наумов, 
возглавлявший Институт электронных управ\-ля\-ющих
машин Минприбора СССР (ИНЭУМ), был назначен ди\-рек\-то\-ром-ор\-га\-ни\-за\-то\-ром 
одного из новых институтов~--- ИПИАН. Основная
задача ИПИАН была определена как <<проведение
фундаментальных и прикладных исследований в области
технических и программных средств массовой вычислительной
техники и систем на их основе>>, а интеллектуальной базой первых
научных подразделений ИПИАН стали коллективы ряда научных
отделов ИНЭУМ, переведенные в ИПИАН в начале 1984~г.

      В 1984--85~гг.\ под руководством директора \mbox{ИПИАН}
Б.\,Н.~Наумова (который в 1984~г.\ был избран действительным
членом АН СССР) и в значительной степени благодаря его
инициативе и настойчивости коллектив специалистов,
пред\-став\-ляв\-ших институты Академии наук СССР, Академий наук
союзных республик СССР, Академий наук стран Восточной Европы,
разработал Концепцию новых поколений вы\-чис\-ли\-тель\-ных сис\-тем.
Разработка Концепции послужила реакцией на стратегическую
программу создания ЭВМ пятого поколения, объявленную в начале
1980-х~гг.\ в Японии и расценивавшуюся в мире как <<Японский
вызов>>. 

В~Концепции были представлены главные направления
исследований и разработок с целью получения принципиально
нового качества информационных и вычислительных систем.
Концепция предполагала, что повышение качества систем может
быть достигнуто на основе сочетания новых методов представления
и обработки информации (данных и знаний) и возможностей
технических и программных средств вычислительной техники и
передачи данных. Следует особо отметить, что Концепция была
направлена прежде всего на создание систем новых поколений, а не
на разработку ЭВМ пятого поколения. Сейчас, спустя более 20~лет
после разработки Концепции, можно сказать, что именно системный
подход, положенный в ее основу, был полностью подтвержден
мировой практикой. Концепция создания новых поколений
вычислительных систем сопровождалась детальной проработкой
необходимой тематики исследований и разработок.

Концепция  новых поколений вычислительных сис\-тем была
реализована в программе на\-уч\-но-тех\-ни\-че\-ско\-го сотрудничества
стран~--- членов СЭВ. В состав программы, разработанной под
руководством Б.\,Н.~Наумова, были включены фундаментальные и
прикладные исследования, осуществлявшиеся в рамках следующих
комплексных научных проектов:
      \begin{itemize}
\item системы обработки знаний;
\item системы обработки изображений и машинной графики;
\item системы автоматизации проектирования вы\-чис\-ли\-тель\-ных
сис\-тем и сверхбольших интегральных схем (СБИС);
\item сети ЭВМ;
\item системы персональных компьютеров;
\item отказоустойчивые вычислительные системы;
\item новые принципы хранения информации (новые внешние
запоминающие устройства);
\item технологии программирования;
\item новые алгоритмы и архитектуры обработки информации;
\item учебная информатика.
\end{itemize}

     Фундаментальные исследования в начальный период
деятельности ИПИАН велись в основном в рамках работ по этим
направлениям.

     Институт в первые годы своего существования быстро
развивался~--- в его состав вошли филиалы в Бердянске, Казани,
Орле. В 1990~г.\ был образован совместный отдел с
Радиотехническим институтом в Таганроге. Общая численность
сотрудников института в 1988--1989~гг.\ превышала 1000~чел.

     Начиная с 1992~г.\ институт называется Институтом проблем
информатики Российской академии наук (ИПИ РАН). Институт
входит в состав Отделения нанотехнологий и информационных
технологий РАН (до 2002~г.~--- Отделения информатики,
вычислительной техники и автоматизации, в 2002--2007~гг.~---
Отделения информационных технологий и вычислительных систем).

     В настоящее время решением Президиума РАН утверждены
следующие основные научные на\-прав\-ле\-ния работ института:
     \begin{itemize}
\item интегрированные информационно-те\-ле\-ком\-му\-ни\-ка\-ционные системы и сети, информатизация
общества;
\item теоретические основы информатики и информационных
технологий, включая математические модели и методы,
стохастические технологии и системы;
\item информационные технологии накопления, хранения,
поиска, обработки, преобразования, отображения, защиты и
передачи информации, когнитивные технологии;
\item архитектура, системные решения и про\-грам\-мное
обеспечение (ПО) вычислительных комплексов и сетей новых
поколений.
\end{itemize}

     В состав института входят филиалы в Орле и Калининграде.

     В институте с 1985~г.\ работает диссертационный совет,
которому дано право проведения защит диссертаций на соискание
ученых степеней доктора и кандидата наук по четырем
специальностям: 05.13.11, 05.13.13, 05.13.15 и 05.13.17. С 1984~г.\
институт имеет базовую кафедру в Московском институте
радиотехники, электроники и автоматизации (техническом
университете)~--- МИРЭА, на которой ведут преподавание
сотрудники института; многие выпускники кафедры работают в
ИПИ РАН.
{\looseness=-1

}

     Институтом в течение многих лет осуществляется выпуск
ежегодных сборников трудов <<Системы и средства
информатики>>; последний по времени выпуск ежегодника за
2007~г.\ имеет №\,17 (<<номерные>> сборники выходят с 1989~г.;
до этого было выпущено еще 3~сборника под другими названиями).
Начиная с 2001~г.\ практически ежегодно до\-пол\-нитель\-но к
основному сборнику издаются его специальные тематические
выпуски. С 2007~г.\ по поручению Отделения нанотехнологий и
информационных технологий РАН институт осуществляет\linebreak издание
ежеквартального научного журнала Отделения <<Информатика и её
применения>>. Сотрудниками института издано значительное число
монографий, в том числе и в зарубежных издательствах.

     \section{Важнейшие практические результаты работ
института}

     С первых дней существования института большинство
научных исследований было связано с решением важнейших
практических проблем, име\-ющих стратегический характер. В 1980-х~гг.\ 
такой задачей являлась организация работ по созданию и
использованию нового класса вычислительной техники~---
персональных ЭВМ. В 1984~г.\ (буквально в первые месяцы
существования института) по заданию Президиума АН СССР,
Госкомитета по науке и технике и Госплана СССР были определены
основные направления разработки и создания ПЭВМ в стране. В
1986--87~гг.\ в рамках сотрудничества Академий наук
социалистических стран и Межправительственной комиссии по
вычислительной технике была разработана Концепция и программа
развития ПЭВМ в СССР и странах СЭВ на период до 1995~г. В
ней были определены основные стратегические цели и выбраны
технические пути их достижения. Выполнение значительной части
работ в рамках этой программы было возложено на ИПИАН.

     Во исполнение принятых решений были осуществлены работы
по созданию отечественной 32-раз\-ряд\-ной ПЭВМ (был выполнен
логический проект микропроцессора, изготовлен, отлажен и
испытан опытный образец машины). Комплект производственной
документации был передан для внед\-ре\-ния на ряд заводов. К
сожалению, большинство этих заводов находилось на территории
Украины, и в результате произошедшего впоследствии распада
СССР выпуск этих машин так и не начался.

     В рамках программы работ по ПЭВМ, утвержденной
Постановлением ЦК КПСС и Совета Министров СССР от 23~января
1986~г., институт выполнил значительную часть работ по
созданию средств базового ПО. В частности,
была разработана операционная сис\-те\-ма МИКРОС
(функциональный аналог СР/М), поставлявшаяся в составе первых
отечественных ПЭВМ ЕС-1840, НЕЙРОН И9.66, ИСКРА-1030,
Электроника-85, а также микроЭВМ СМ~1800. Для тех же типов
отечественных ПЭВМ была разработана мобильная операционная
система ИНМОС (русскоязычный <<клон>> ОС UNIX), за создание
и внедрение которой 5~сотрудников института были в 1988~г.\
удостоены премии Совета Министров СССР.

     Разработанные в институте компиляторы с языков Бейсик,
Паскаль, СИ, ФОРТРАН прошли %\linebreak 
испытания и были переданы в
тиражирующие организации. Кроме того, были созданы базовые
функциональные пакеты для основных операционных систем
отечественных ПЭВМ: текстовый редактор ТЕКСТ-ОС,
реляционная база данных РБД ОС, пакеты обработки текстовой
информации, графической информации, табличной информации,
интегрированный пакет прикладных программ анализа и обработки
данных ИМАД, пакет статистической обработки.
{\looseness=-1

}

     Широкое практическое применение получили работы
института по средствам цифровой обработки изображений и
созданию ап\-па\-рат\-но-про\-грам\-мных персональных видеосистем
<<Микровидео>>, а также системы дактилоскопической
идентификации личности, использовавшейся в органах внут\-рен\-них
дел.

Приход в институт в качестве директора в 1989~г.\ 
члена-корреспондента АН СССР Игоря Александровича Мизина, много
лет до этого проработавшего в оборонной промышленности, совпал
по времени с началом коренных изменений в жизни страны и
Академии наук. По инициативе Игоря Александровича в институте
стали активно развиваться работы в области построения
интегрированных ин\-фор\-ма\-ци\-он\-но-те\-ле\-ком\-му\-ни\-ка\-ци\-он\-ных сетей и
систем. Признанием научных дости\-же\-ний школы Мизина стало его
избрание в 1997~г.\ действительным членом Российской академии
наук. Важнейшим практическим результатом работ института в
начале 1990-х~гг.\ стала разработка Концепции создания и развития
телекоммуникационных систем, содержащей обоснование и выбор
международных стандартов, методов и средств, на которые
предложено ориентировать развитие инфра\-струк\-ту\-ры
телекоммуникаций в регионах России. Был разработан
функционально полный комп\-лекс ап\-па\-рат\-но-про\-грам\-мных средств
(на %\linebreak
 мик\-ро\-про\-цес\-сор\-ной базе) построения сетей пере\-да\-чи данных с
коммутацией пакетов и с интегральной коммутацией, включающий
высокопроизводительный центр коммутации пакетов, пакетные
адаптеры данных, средства управления и абонентского сопряжения с
сетью пе\-ре\-да\-чи данных, средства электронной поч\-ты X.400.
Разработанный комплекс средств нашел применение при создании
ин\-фор\-ма\-ци\-он\-но-те\-ле\-ком\-му\-ни\-ка\-ци\-он\-ной сис\-те\-мы в Псковской
области, ставшей прообразом ряда проектов по созданию подобных
региональных сис\-тем.
{\looseness=-1

}

     С середины 1990-х гг.\ институт активно участвует в работах
по созданию систем информационного обеспечения процессов
управления для органов государственной власти РФ. Разработаны
типовые средства и комплексы информационной поддержки
процессов принятия решений руководством страны в нормальных
условиях и в случае чрезвычайных ситуаций. Институтом
вы\-пол\-нены крупные системные разработки для Министерства по
чрезвычайным ситуациям РФ~--- по %\linebreak
 развитию автоматизированной
информационной управ\-ля\-ющей сис\-те\-мы единой государственной
сис\-те\-мы предупреждения и ликвидации чрезвычайных ситуаций в
час\-ти обработки данных и обеспечения ин\-фор\-ма\-ци\-он\-но-те\-ле\-ком\-му\-ни\-ка\-ци\-он\-но\-го 
обмена единой диспетчерской системы города.

     С 1999~г.\ директором института является Игорь
Анатольевич Соколов, работавший до этого в институте
заведующим отделом и заместителем директора (в 2003~г.\ избран
чле\-ном-кор\-рес\-пон\-ден\-том РАН, в 2008~г.~--- академиком). Он
продолжил разработку проектов, начатых при И.\,А.~Мизине.
%
     Начиная с этого времени институт стал более активно
участвовать в работах по информатизации Президиума РАН и его
учреждений (институт был назначен головным исполнителем по
целевой программе). Вышли на новый уровень работы института в
интересах Банка России (участие в разработке и создании
ин\-фор\-ма\-ци\-он\-но-те\-ле\-ком\-му\-ни\-ка\-ци\-он\-ной системы обеспечения
электронных расчетов, создание поч\-то\-вых служб ряда
территориальных управлений и~др.).
{\looseness=-1

}

     В настоящее время институтом ведутся крупные проекты по
созданию ин\-фор\-ма\-ци\-он\-но-те\-ле\-ком\-му\-ни\-ка\-ци\-он\-ных систем с ГУСП,
Федеральной службой безопасности и Министерством внутренних
дел~РФ.\\[-28pt]



\section{Основные фундаментальные результаты прошлых лет}

     В первые годы существования института выделялись три
основных направления проводимых исследований:\\[-13pt]
     \begin{enumerate}[(1)]
\item разработка архитектурных решений и технических средств
ЭВМ массового применения;
\item разработка ПО ЭВМ массового
применения;
\item исследование и разработка вопросов системного применения
ЭВМ.
\end{enumerate}

     В области \textbf{разработки технических средств} были
выполнены работы по разработке архитектуры %\linebreak 
семейства
универсальных 32-разрядных ЭВМ, %\linebreak 
которые завершились созданием
прототипа, переданного в производство. Совместно со
специалистами из Чехословакии были разработаны спецпроцессоры
с RISC\footnote{RISC~--- Reduced Instruction Set Computer~--- компьютер
с уменьшенным набором команд.}-ар\-хи\-тек\-ту\-рой, ориентированные на решение задач
искусственного интеллекта. Полученные результаты по
моделированию циф\-ро\-вых устройств, системам тестирования и
оценкам характеристик средств вычислительной %\linebreak 
техники
(диаг\-нос\-ти\-че\-ский комбайн, программные модели 386 и
387~процессоров, наборы функциональных тестов) в настоящее
время используются и развиваются в работах института.
{\looseness=-1

}

     В 1980-е гг.\ институт проводил исследования в области
\textbf{архитектурных решений и управляющих программных
средств} для построения распределенных
     ин\-фор\-ма\-ци\-он\-но-вы\-чис\-ли\-тель\-ных систем на базе ЭВМ общего
назначения и персональных ЭВМ. Результатом работ стало создание
(совместно с НИИ ЭВМ г.~Минска) одной из последних машин
класса мэйнфрейм в СССР~--- ЕС-1130. За работы по созданию
управляющих вычислительных комплексов (совместно с заводом
вы\-чис\-ли\-тель\-ных и управ\-ля\-ющих машин (ВУМ) в Киеве) коллектив разработчиков, в состав которого входили
Б.\,Н.~Наумов и другие сотрудники ИПИАН, был удос\-то\-ен
Государственной премии Украинской ССР. А работа, результатом
которой стало производство таких комплексов в Воронеже, была
удостоена Государственной премии СССР (в составе коллектива
лауреатов был академик Б.\,Н.~Наумов).

     Исследования и разработки в области \textbf{про\-грам\-мно\-го
обеспечения} традиционно представляют одно из основных
направлений деятельности института. Коллектив ученых ИПИАН
разрабатывал базовые программные средства ПЭВМ. К
достижениям того времени следует отнести создание
унифицированной операционной системы УОС для
микропроцессора К1810ВМ86, обладавшей уникальными для того
времени возможностями~--- обеспечивался мультипрограммный
режим, поддержка работы в локальных сетях, многооконный
интерфейс, динамическая смена кодовых таблиц и совместимость с
наиболее распространенными операционными сис\-те\-ма\-ми для
данного класса машин.

     В институте велись работы по \textbf{созданию научно
обоснованной технологии программирования}, охватывающей
все этапы жизненного цикла про\-грам\-мных средств для
вычислительных систем новых поколений и создающей основу
промышленного производства программ и программной %\linebreak
до\-ку\-мен\-тации. 

Были предложены новые подходы к ав\-то\-ма\-ти\-за\-ции
разработки прикладных программ с за\-дан\-ны\-ми характеристиками,
разработан метод по\-рож\-де\-ния целевых программных сис\-тем,
определены со\-став и структура инстру\-мен\-таль\-но-тех\-но\-ло\-ги\-че\-ских
средств поддержки новой технологии %\linebreak 
(сис\-те\-ма ГЕНПАК),
проведены исследования и практические разработки по
обеспечению мобильности ПО, разработана и успешно применена
методика переноса программных средств.
{\looseness=-1

}

     Были проведены исследования по моделированию
технологических этапов проектирования программных средств с
использованием сетей Пет\-ри и созданию инструментальных средств
объ\-ект\-но-ори\-ен\-ти\-ро\-ван\-но\-го программирования. Значительная часть
работ по практической реализации технологии программирования,
созданию базовых %\linebreak
пакетов прикладного ПО и полиграфическому
сопровождению работ института велась в Бердянском филиале
ИПИАН, численность сотрудников которого доходила до
200~чел. К сожалению, после распада СССР Бердянский филиал
прекратил свое существование.
{\looseness=-1

}
%\pagebreak

     В 1980-е\,--\,начале 1990-х гг.\ в институте активно проводились
разработки сис\-тем автоматизированного проектирования (САПР) 
машиностроения и мик\-ро\-электроники. Наиболее
значительным достижением в этой области было создание системы
проектирования печатных плат МАГИСТР-2, в которой были
реализованы принципиально новые, превосходящие тогдашние
зарубежные аналоги,\linebreak алгоритмы трассировки двусторонних и
многослойных печатных плат. В рамках междуна-\linebreak родного проекта
КНП-3 в институте были разра-\linebreak ботаны принципы организации САПР
СБИС\linebreak  методом кремниевого компилирования и элементы
ПО такого компилятора. Работы в области
САПР машиностроения активно велись в тесной связи с
промышленными предприятиями в Казанском %\linebreak
 филиале института (в
частности, была создана интерактивная графическая система для
конструирования и оформления чертежей~--- САПР МАШ). 
В~1997~г.\ Казанский филиал ИПИ РАН был преобразован в
Институт проблем информатики Академии наук Татарстана.

% \bigskip
     С 1999 по 2005~гг.\ в институте работала группа ученых,
которой руководил академик Всеволод Сергеевич Бурцев.
Проводились исследования в области нетрадиционных архитектур и
системного ПО высокопроизводительных
ин\-фор\-ма\-цион\-но-вы\-чис\-ли\-тель\-ных комплексов, основывающихся на
ассоциативной памяти. Были разработаны программные модели
новой архитектуры и под\-го\-тов\-ле\-на реализация ее элементов <<в
железе>>.

     В институте в течение ряда лет велись работы в области
геоинформационных технологий. Была разработана система сбора,
структуризации, семантического моделирования и кодирования,
хранения, анализа, поиска, обработки, отображения и передачи
многоаспектной и разной по форме пространственной информации о
Земле, обеспечивающая синтез электронного образа Земли как
многомерного представления планеты. Разработана технология
ввода, обработки и выдачи пространственных данных в виде
электронных карт и текстовой информации, а также формирование и
ведение базы метаданных электронных карт. С активнейшим
участием сотрудников ИПИ РАН впервые в мировой практике
разработаны Государственные (национальные) стандарты ГОСТ
Р~52055-2003 <<Геоинформационное картографирование.
Пространственные модели местности. Общие требования>> и ГОСТ
Р~52293-2004 <<Геоинформационное картографирование. Система
электронных карт. Карты электронные топографические. Общие\linebreak
требования>>. Разработана методология по\-стро\-ения
     объектно-ориентированных геоинформационных систем (ГИС).
Созданы методы хранения и поиска информации в
     текстово-графической базе данных, которые легли в основу
комплекса программ ГИС Objectland, принятой в качестве базовой
для ведения автоматизированного земельного кадастра в Российской
Федерации и используемой во многих областях страны.

\vspace*{-7pt}

\section{Фундаментальные исследования, проводящиеся 
в~настоящее время}
\vspace*{-3pt}

     В данном разделе статьи кратко представлены некоторые
направления современных исследований в ИПИ РАН.

     В конце 1980-х~гг.\ в институте начались работы в области
самосинхронной схемотехники, ини\-ци\-иро\-ван\-ные контактами с
исследовательской группой из Ленинградского электротехнического
института. Институт стал ведущим в стране центром исследований в
этом направлении схемотехники. Были проведены исследования по
теоретическому обоснованию методов строгой самосинхронизации
(ССС), разработаны инструментальные средства и системы для
проектирования ССС-схем, созданы библиотеки базовых элементов.
Совместно с Технологическим центром МИЭТ удалось впервые в
отечественной и зарубежной практике изготовить на базе
полузаказных БИС тестовые кристаллы, реализующие
вычислительное ядро 8-разрядного микроконтроллера. Проведенные
испытания подтвердили теоретические выводы о значительном
расширении диапазона работоспособности ССС-схем в области
температур и напряжений по сравнению с синхронной реализацией,
а также о возможности ускорения работы этих устройств. %\linebreak
Работы в
этой области продолжаются и пред\-став\-ля\-ют\-ся весьма
перспективными; уже получен ряд патентов на изобретения.
{\looseness=-1

}

     В институте достаточно давно проводятся работы в области
\textbf{периферийных устройств ПЭВМ}. Этим в основном
занимается Орловский филиал института, в котором были
разработаны ми\-ни-прин\-те\-ры и прин\-те\-ры-плот\-те\-ры, превосходившие
на момент создания все имевшиеся отечественные изделия. Это
направление включает в себя и теоретические исследования в
области цветообразования и стереоскопического воспроизведения
изображений; работы в данном направлении активно развиваются.

     В институте в течение многих лет ведутся работы в области
разработки методов интеграции неоднородных информационных
ресурсов. Были разработаны теоретические основы, методология и
архитектура интероперабельных сред неоднородных
информационных ресурсов, методы и средства спецификации и
повторного совместного использования ресурсов при
проектировании информационных систем~--- язык СИНТЕЗ, метод
композиционного проектирования информационных %\linebreak 
сис\-тем,
основанный на композиционном исчислении спецификаций. Эти
фундаментальные результаты являются базисом проводимых работ
по %\linebreak
 созданию слоя предметных посредников в электронных
биб\-лио\-те\-ках, позволяющего унифицировать доступ к разнородным
электронным коллекциям. После прекращения в 1991~г.\
сотрудничества в об\-ласти интеграции неоднородных
информационных ресурсов в рамках комиссии по вычислительной
технике соцстран усилиями сотрудников института была
организована московская секция ACM \mbox{SIGMOD}, ведущая активную
работу и в настоящее время.
{\looseness=-1

}

     С первых дней существования института большое внимание
уделялось исследованиям, связанным с \textbf{системными
применениями ЭВМ} в области созда\-ния методов, алгоритмов и
программ для %\linebreak
 анализа процессов и для обработки информации в
стоха\-сти\-че\-ских системах. Эти исследования выполнялись под
руководством выдающегося отечественного ученого академика
Владимира Семеновича Пугачева, пришедшего в ИПИАН с группой
своих учеников и сотрудников вскоре после образования института.
Научным коллективом были разработаны методы, алгоритмы и
программы статистического анализа и услов\-но-оп\-ти\-маль\-но\-го
оценивания случайных процессов и последовательностей,
алгоритмы автоматического составления и решения уравнений с
помощью ЭВМ для вероятностных характеристик процессов в
нелинейных стохастических системах по исходным уравнениям
модели. Академик Пугачев своей важнейшей задачей считал
доведение теоретических разработок до практики их использования
инженерами и исследователями. Коллективом, которым он
руководил, были разработаны основы новой информационной
технологии построения математических моделей сложных систем и
процессов по экспериментальным данным при наличии
неопределенностей, вклю\-ча\-ющей тео\-рию быстрого
     услов\-но-оп\-ти\-маль\-но\-го оценивания и идентификации,
управления и планирования экспериментов, проверки
достоверности, принятия решений. Полученные результаты
позволили существенно расширить круг специалистов, активно
использующих воз\-мож\-но\-сти современных средств вычислительной
техники и методов математического моделирования, за счет
устранения необходимости заниматься собственно
программированием. Были разработаны принципы создания
интел\-лек\-ту\-а\-ли\-зи\-ро\-ван\-ных пакетов прикладных программ для
исследования стохастических моделей, реализованные
программными средствами. За работы в области стохастических
систем в 1990~г.\ академик В.\,С.~Пугачев был удостоен
Ленинской премии, причем премию он получил индивидуально и
был последним, кто получил такую премию.
{\looseness=-1

}

     В институте проводятся работы в области \textbf{методов
представления и обработки знаний}. Были разработаны пакеты
прикладных программ, обеспечивающие морфологический анализ
русскоязычных текстов, в частности пакет РАМОС, включающий
словарь русского языка Ожегова. Развитием работ, связанных с
обработкой текстов, стали сис\-те\-мы анализа и рубрицирования
текстов, системы построения тезаурусов. Разработана
     ког\-ни\-тив\-но-линг\-ви\-сти\-че\-ская модель и система структурных
правил для алгоритмов машинного перевода и обработки знаний,
извлекаемых из текстов на ряде европейских языков. Построен и
реализован в виде действующих систем аппарат функ\-цио\-наль\-но-се\-ман\-ти\-че\-ской 
анг\-ло-рус\-ской грамматики. Разработаны
ассоциативные модели представления знаний на базе семантических
сетей и фреймовых моделей и создан язык представления знаний
ДЕКЛ. Разработаны методы логического вывода, основанные на
представлении ассоциативной модели и позволяющие извлекать
семантическую информацию из текстов русского языка с
автоматическим формированием структур баз знаний (БЗ) и их использованием в
задачах фактографического поиска и экспертных решений. На
основе этих методов создана ло\-ги\-ко-ана\-ли\-ти\-че\-ская система
\mbox{АНАЛИТИК}, настроенная на задачи ло\-ги\-ко-ана\-ли\-ти\-че\-ской
обработки сводок криминальной милиции и обеспечивающая
автоматическое извлечение значимой информации из текстовых
сообщений и ее использование для нечеткого поиска.
{\looseness=-1

}

     В 1989~г.\ под руководством И.\,А.~Мизина в %\linebreak 
инсти\-ту\-те
активизировались работы в области %\linebreak 
разви\-тия \textbf{сетевых
информационных технологий и технологии построения
крупномасштабных ин\-фор\-ма\-ци\-он\-но-ком\-му\-ни\-ка\-ци\-он\-ных
систем}. Основным фундаментальным результатом этих
исследований стала разработка взаимоувязанного комплекса
математических моделей, методов и программных средств оценки
общесистемных характеристик телекоммуникационных систем,
использующих современные технологии передачи данных.
Впоследствии под руководством И.\,А.~Соколова была решена
актуальная задача описания класса крупномасштабных
телекоммуникационных систем двойного применения как
полносвязных, территориально-струк\-ту\-ри\-ро\-ван\-ных, мультисетевых
систем общенационального масштаба, совмещающих функции
специальных систем и систем общего пользования.

     Начиная с 1992~г.\ в институте большое внимание уделяется
развитию \textbf{математических методов исследования сложных
информационных и телекоммуникационных систем (ИТС)}.
Осуществлены фундаментальные теоретические исследования,
разработка моделей, математических и программных средств для
анализа и оптимизации характеристик реальных ИТС (процессов
доставки информации с разными дисциплинами обслуживания и
входными потоками, процессов управления и комплексной
обработки информации). Новизна методов и результатов,
полученных в этой области в ИПИ РАН, заключается в:
     \begin{itemize}
\item разработке методов расчета сетей с учетом реальных
технологий передачи, определяемых действующими
международными стандартами и рекомендациями, и реальных
вероятностных характеристик информационных потоков;
\item комбинировании методов аналитического и имитационного
моделирования, что позволяет корректировать естественные
погрешности аналитических методов и учитывать специфику
реальных систем;
\item доведении работы до программной реализации,
позволяющей проводить расчеты для реальных сетей
большой размерности.
     \end{itemize}

     Важным этапом работ ИПИ РАН в этой области стало
выполнение в 1999--2002~гг.\ проекта <<Разработка модели
коммутируемой телефонной сети ОАО <<Ростелеком>>. Основной
целью работы была разработка методов и средств моделирования
коммутируемой сети ОАО <<Ростелеком>>, предназначенных для
обеспечения информационной поддержки пользователей АСУ
цифровой сети в процессе принятия ими решений по оперативному
управлению вторичной телефонной сетью ОАО <<Ростелеком>>
при возникновении на ней нештатных ситуаций (в режиме,
приближенном к реаль\-ному времени). В ходе работы выяснилось,
что специфика данной сети не позволяла напрямую использовать ни
одну из существующих в научной литературе моделей телефонных
сетей. В связи с этим пришлось создавать сложную
многоуровневую модель, включающую имитационную модель
собственно телефонной сети с учетом всей специфики ее
деятельности и алгоритмы динамического управ\-ле\-ния сетью,
использующие как имитационную, так и упрощенную
аналитическую модель сети.
{\looseness=-1

}

    С 2000~г.\ в ИПИ РАН активно развивается направление работ,
связанное с исследованием сис\-тем и сетей массового обслуживания
(СМО и \mbox{СеМО}), моделирующих современные и перспективные информационно-технические средства (ИТС)
и их узлы. В рамках этого направления получены существенные
теоретические результаты: разработаны новые математические
методы, позволяющие вычислять на основе аналитических
соотношений главные показатели функционирования СМО с
различными дисциплинами обслуживания и сложными входящими
потоками, получены новые результаты в части исследования СеМО,
в том числе СеМО с отрицательными заявками (G-сети) и
зависимыми длинами заявок на разных этапах обслуживания;
разработана методика расчетов СМО и СеМО большой размерности.
Результаты исследований были применены в ряде НИР и ОКР,
связанных с разработкой средств моделирования реальных ИТС.

     Наряду с работами по созданию математических и
программных средств <<непосредственного>> моделирования ИТС,
в ИПИ РАН осуществляются исследования по разработке
теоретических ве\-роят\-но\-ст\-но-ста\-тис\-ти\-че\-ских методов,
ориентированных на перспективное применение в об\-ласти
моделирования ИТС. В частности, начиная с 1994~г.\ ведутся
исследования в области разработки методов анализа случайных
процессов сложной структуры. Основным направлением работ в
данной об\-ласти стало в последние годы моделирование
информационных рисков и разработка методов вычисления
надежностных характеристик ИТС и их элементов в интересах
проектирования и прогнозирования функционирования систем в
сущест\-венно стохастической среде, обеспечения их
катастрофоустойчивости и информационной безопасности.
{ %\looseness=1

}

     С самого начала существования института большое внимание
уделялось работам в области учебной информатики. Институт
принимал активное участие в разработке ап\-па\-рат\-но-про\-грам\-мных
комплек\-сов и пакетов прикладных программ для учебной
информатики. На него была возложена головная роль в выработке
требований, отборе и %\linebreak
 обеспечении поставки в образовательные
учреждения первых учебных вычислительных комплексов, а также в
подготовке экспериментальных учебных пособий по курсу
школьной информатики. 
В институте были разработаны пакеты
прикладных программ для обучения машинописи, основам
информатики и вычислительной техники, для создания прикладных
программных средств обучения. 
%
Фундаментальные результаты в
области дистанционного образования на практике были опробованы
в ряде международных проектов. Институт тесно сотрудничает с
организациями ЮНЕСКО.
{ %\looseness=1

}


Институт проблем информатики РАН встречает свое 25-летие как признанный центр
исследований и разработок в об\-ласти информатики, сочетающий
фундаментальные научные исследования с практикой построения
важнейших крупномасштабных
     ин\-фор\-ма\-ци\-он\-но-те\-ле\-ком\-му\-ни\-ка\-ци\-он\-ных систем.
\label{end\stat}


\end{multicols}