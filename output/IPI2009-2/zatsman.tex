\def\stat{zatsman}

\def\tit{СЕМИОТИЧЕСКАЯ МОДЕЛЬ ВЗАИМОСВЯЗЕЙ КОНЦЕПТОВ, 
ИНФОРМАЦИОННЫХ ОБЪЕКТОВ И КОМПЬЮТЕРНЫХ КОДОВ$^*$}
\def\titkol{Семиотическая модель взаимосвязей концептов, 
информационных объектов и компьютерных кодов} 

\def\autkol{И.\,М.~Зацман}
\def\aut{И.\,М.~Зацман$^1$}

\titel{\tit}{\aut}{\autkol}{\titkol}

{\renewcommand{\thefootnote}{\fnsymbol{footnote}}\footnotetext[1]
{Работа выполнена при частичной поддержке РФФИ, грант № 09-07-00156.}}

\renewcommand{\thefootnote}{\arabic{footnote}}
\footnotetext[1]{Институт проблем информатики Российской академии наук, 
iz\_ipi@a170.ipi.ac.ru}


\Abst{Рассматривается семиотическая модель, которая была разработана в процессе 
исследования проблем генерации и эволюции целевых систем знаний (ЦСЗ), а также 
отображения в электронных библиотеках и других видах информационных систем процессов их 
эволюции во времени и пространстве знаковых систем. Эти проблемы относятся к новому 
направлению исследований, получившему название <<когнитивная информатика>>. Предлагаемая в 
статье семиотическая модель предназначена для описания взаимосвязей денотатов, концептов, 
информационных объектов и компьютерных кодов. Эта модель позиционируется как 
теоретическая основа для разработки методов трехкомпонентной кодировки денотатов, 
концептов и информационных объектов, в том числе построения взаимно однозначных 
отношений между концептами систем знаний человека и компьютерными кодами.}

\KW{семиотическая модель; целевые системы знаний; денотаты; концепты; информационные 
объекты; компьютерные коды; трехкомпонентная кодировка; отношения между концептами и 
компьютерными кодами}

 \vskip 24pt plus 9pt minus 6pt

      \thispagestyle{headings}

      \begin{multicols}{2}

      \label{st\stat}
      
\section{Введение}

   Анализ работ~[1--3] позволяет утверждать, что в настоящее время в информатике 
как в ин\-фор\-ма\-ци\-он\-но-компью\-тер\-ной науке~\cite{4zat} формируется новая область 
исследований, целью которых является разработка теоретических основ\linebreak
 создания 
информационно-коммуникационных тех\-но\-ло\-гий (ИКТ), обеспечивающих процессы 
целенаправленного формирования (или другими словами, генерации) новых систем знаний. 
Целенаправленно формируемые новые системы знаний предлагается называть 
\textit{целевыми}. Эта новая область исследований относится к когнитивной 
информатике\footnote[2]{Когнитивная информатика~--- направление в информатике как в информационно-компьютерной науке, 
которое при исследовании вычислительных процессов и разработке информационных 
систем использует методы когнитивной науки, изучающей ментальные процессы (познавательные и 
креативные) и ментальные объекты (концепты), а при исследовании форм представления концептов, их 
эволюции, познавательных и креативных процессов использует методы информатики.}, которая 
находится в начальной стадии описания ее предметной области и составляющих ее 
проблем~[5--7].
   
   Целью создания ИКТ, которые должны обеспечивать процессы формирования  
ЦСЗ, является решение как минимум трех актуальных проблем. Это:
   \begin{enumerate}[(1)]
\item идентификация стадий формирования и эволюции ЦСЗ~--- \textit{проблема 
идентификации ЦСЗ};
\item обеспечение анализа и оценивания степени релевантности разных вариантов 
ЦСЗ технологическим и другим общественно значимым потребностям, в 
интересах которых они были целенаправленно сформированы~--- \textit{проблема 
релевантности ЦСЗ};
\item обеспечение средствами ИКТ целенаправленного влияния на процессы 
эволюции целевых систем знаний, формируемых в интересах технологических или 
других общественно значимых потребностей~--- \textit{проблема направляемого 
развития ЦСЗ}.
\end{enumerate}

   В процессе постановки проблем идентификации, релевантности и направляемого 
развития ЦСЗ, а также исследования процессов генерации и эволюции ЦСЗ, ключевыми 
являются следующие вопросы:
\begin{enumerate}[(1)]
\item категоризация концептов и формулировка признаков, позволяющих 
зафиксировать границы между категориями личностных, коллективных и 
конвенциональных концептов, а затем использовать их при описании процессов 
генерации и эволюции ЦСЗ;
\item разработка моделей и методов установления отношений между денотатами, 
концептами, информационными объектами и компьютерными кодами.
\end{enumerate}

   Статья посвящена второму вопросу. Первый вопрос~--- категоризация концептов~--- был 
рассмотрен в работах~\cite{8zat, 9zat}. В~этих работах в процессе описания трех 
категорий концептов использовалась система базовых терминов информатики, построенная 
на основе результатов работ~[10--14]. В~предложенном подходе к 
категоризации концептов лек\-си\-чески акцентированы различия между этими тремя 
категориями и сформулированы их отличительные признаки. Предложенная категоризация 
концептов на личностные, коллективные и конвенциональные не является только 
терминологическим результатом, так как определение трех категорий концептов и их 
признаков необходимо для описания новых направлений исследований и разработок, 
относящихся к проблематике генерации и эволюции~ЦСЗ.
{\looseness=1

}
   
   Целью статьи является описание взаимосвязей денотатов, концептов, информационных 
объектов и компьютерных кодов в виде семиотической модели с использованием системы 
базовых терминов, определенной в работах~\cite{8zat, 9zat}. В процессе создания этой 
модели была использована идея распределения объектов предметной области информатики 
по разным сферам (средам) в зависимости от их природы (например, ментальной и 
материальной сфер, социальной и цифровой сред). Разные формы изложения этой идеи, 
использованные в широком спектре проблем информатики, можно найти в работах~[15--18].
   
   Основная идея статьи заключается в том, что для каждой из трех вершин семиотического 
треугольника Г.~Фреге\footnote{Три вершины семиотического треугольника Г.~Фреге~--- это значение 
знака (его концепта), формы знака (как частный случай информационного объекта) и денотата знака 
(материальной, цифровой или иной природы).} [19--21] предлагается 
использовать свою систему кодировки, в том числе и для вершины концептов. Предлагаемое 
введение специальной системы кодировки для концептов, которые в семиотическом 
треугольнике являются значениями знаков, является необходимым условием по\-стро\-ения 
взаимно однозначных отношений между концептами и компьютерными кодами. Построение 
таких отношений является основой для решения проблем идентификации, релевантности и 
на\-прав\-ля\-емо\-го развития ЦСЗ, а также для решения задач концептуального индексирования и 
семантического поиска~\cite{17zat, 22zat}.
   
Семиотическая модель, предлагаемая в статье, позиционируется как теоретическая 
основа для разработки методов трехкомпонентной кодировки\footnote{Под трехкомпонентной 
кодировкой, которая далее будет рассмотрена подробнее, понимается присвоение компьютерных кодов трем 
вершинам семиотического треугольника (денотату, концепту как значению знака и информационному 
объекту как форме знака).} всех трех вершин семиотического треугольника, в том числе 
построения взаимно однозначных отношений между концептами систем знаний человека и 
компьютерными кодами.
   
Взаимно однозначные отношения между концептами и компьютерными кодами 
существенно отличаются от взаимно однозначных отношений между литерами и цифровыми 
двоичными кодами, которые устанавливаются с помощью различных таблиц кодировки 
множества литер (ASCII~--- American Standard Code for
Information Interchange, EBCDIC~--- Extended Binary Coded Decimal
Interchange Code, UNICODE и~т.\,д.). Это отличие проиллюстрируем на 
примере слов естественного языка. Если нужно закодировать слово с помощью таблицы 
кодировки, то оно разделяется на литеры, которые заменяются кодами в соответствии с 
выбранной таблицей. Отметим, что каждая литера слова в отдельности не является 
носителем смыслового содержания. Однако вся последовательность литер, образующих одно 
слово, например <<поле>>, может иметь несколько смысловых значений (концептов) в 
зависимости от контекста (ржаное поле, футбольное поле, магнитное поле и~т.\,д.).
   
   Следовательно, с помощью таблиц кодировки множества литер могут кодироваться 
только формы знаков, но не их значения, т.\,е.\ таблицы кодировки применимы только к 
одной вершине семиотического треугольника Г.~Фреге. В информационных системах для 
обозначения именно концептов, но не форм их представления используется, как правило, 
следующий подход. Сначала (первый этап) концепт как значение некоторого знака 
обозначают с помощью формы этого знака, а затем (второй этап) форме знака ставится в 
соответствие некоторый компьютерный код.
   
   Так как естественные языки как знаковые сис\-те\-мы обладают свойством 
асимметрии~\cite{23zat} и построенное на первом этапе соответствие концептов и форм 
их представления в общем случае является многозначным, то задача построения взаимно 
однозначных отношений между концептами и компьютерными кодами \textit{через 
кодировку форм пред\-став\-ле\-ния концептов} (но не самих концептов напрямую) в общем виде 
становится трудноразрешимой.
   
   Однако при решении прикладных задач явление асимметрии знаковых систем может 
быть частично учтено, если на первом этапе концепту как значению знака с помощью 
программы семантического анализа текста ставится в соответствие не форма некоторого 
знака, а тот дескриптор тезауруса, который соответствует именно этому 
концепту~\cite{22zat}.
   
   Это возможно реализовать в том случае, если в информационных системах для кодировки 
концептов используются тезаурусы, поддерживающие 
\begin{figure*}[b] %fig1
\vspace*{1pt}
\begin{center}
\vspace*{1pt}
\mbox{%
\epsfxsize=150.154mm
\epsfbox{zat-1.eps}
}
\end{center}
\vspace*{-9pt}
\Caption{Система базовых терминов (определения всех перечисленных 12~терминов 
приведены в работах~\cite{8zat, 9zat})
\label{f1zat}}
\end{figure*}
семантические отношения между 
дескрипторами и фиксирующие разные виды знаковой асимметрии. Например, в тезаурусе с 
помощью дескрипторов могут быть обозначены (представлены) несколько концептов, 
имеющих одну и ту же форму пред\-став\-ле\-ния (например, слово <<поле>>). При этом каждый 
концепт будет представлен только одним дескриптором, соответствующим \textit{только 
одному значению слова <<поле>>}. Таким образом, использование тезаурусов при решении 
прикладных задач в явно определенной предметной области позволяет устанавливать 
взаимно однозначные отношения между концептами и компьютерными кодами.
   
   Предлагаемая семиотическая модель при ее прикладном применении ориентирована в 
основном на использование тезаурусного подхода к кодировке концептов с помощью 
компьютерных кодов, но с теоретической точки зрения она допускает использование иных 
подходов, а не только тезаурусного (см.\ описание способа кодировки в микрокомпьютере 
Altair~8800 в разд.~3).

\section{Система базовых терминов}
   
   В работах~\cite{8zat, 9zat} дано описание тех 12~базовых терминов информатики, 
которые необходимы для этой статьи и перечислены на рис.~\ref{f1zat}. На этом рисунке они 
пронумерованы против часовой стрелки. Для целей настоящей статьи понадобятся 
сле\-ду\-ющие четыре ключевые положения из этого описания.
   
   Во-первых, в этих работах термин \textit{коды} был определен как компьютерные 
эквиваленты литер двоичных цифр (или их последовательностей), которые могут 
представлять собой намагниченность или ее отсутствие в цифровой среде, наличие 
электрического тока или его отсутствие, способность к отражению света или ее отсутствие. 
Литеры <<0>> и~<<1>>, о которых говорится в дефиниции термина \textit{коды}, по 
определению являются сущностями \textit{среды социальных коммуникаций} между 
пользователями информационной системы (ИС), а их компьютерные эквиваленты~--- 
сущностями \textit{цифровой среды} ИС.
   
   Во-вторых, среди всех возможных компьютерных кодов цифровой среды в 
работах~\cite{8zat, 9zat} были выделены три следующие категории, необходимые для 
этой статьи:
   \begin{enumerate}[(1)]
\item коды, соотнесенные в явном виде с концептами знаний пользователей ИС,~--- 
\textit{коды первой категории, или семантические коды};
\item коды, соотнесенные с эксплицитными и от\-чуж\-ден\-ны\-ми от человека сенсорно 
воспринимаемыми знаковыми формами представления элементарных концептов 
(формы знаков) и сложных концептов (например, фразы на естественном языке) в 
среде социальных коммуникаций ИС,~--- \textit{коды второй категории, или 
информационные коды};
\item коды, соотнесенные с материальными объектами и явлениями, а также с 
алгоритмами, программами и другими видами денотатов; эти коды являются 
компьютерными идентификаторами денотатов в цифровой среде ИС~--- \textit{коды 
третьей категории, или объектные коды}.
\end{enumerate}


   В-третьих, была определена еще одна \textit{нулевая категория кодов}, названная 
\textit{цифровыми данными}, к которой были отнесены все компьютерные коды цифровой 
среды ИС, не относящиеся в явном виде\linebreak
 к трем вышеопределенным категориям. К~циф\-ро\-вым 
данным относятся результаты любых измерений, полученных с помощью цифровых 
технических систем (устройств) и хранящихся в ИС, а\linebreak
 также \textit{результаты любых 
вычислений}, не яв\-ля\-ющиеся итогом генерации и представления знаний пользователями ИС.
   
   И последнее, четвертое, положение заключается в том, что все 12~терминов разделены на 
три группы в зависимости от природы обозначаемых ими сущностей, ментальной, 
социальной или циф\-ро\-вой~\cite{24zat}:
   \begin{enumerate}[(1)]
\item знания, ментальные образы сенсорно воспринимаемых данных, концепты как 
родовой \mbox{термин}, личностные, коллективные и конвенциональные концепты как виды 
этого родового термина (сущности ментальной сферы, т.\,е.\ сферы знаний 
пользователей ИС);
\item информация, данные, знаковая информация, информационные объекты 
(сущности среды социальных коммуникаций пользователей ИС);
\item компьютерные коды всех категорий (сущности цифровой среды ИС).
\end{enumerate}

   Четыре рассмотренных положения дополним двумя небольшими пояснениями. 
   Во-первых, \textit{денотатами материальной природы} являются любые физические 
объекты, названия которых присутствуют в естественных языках, в то время как 
\textit{денотатами цифровой природы} являются такие сочетания компьютерных кодов, 
которые имеют уникальные названия в рамках ИС (например, цифровые программные 
объекты), а эти названия могут использоваться разработчиками ИС при профессиональных 
коммуникациях. Во-вторых, некоторый физический объект может одновременно являться 
еще и носителем формы знака, например круг из жести или картона с дорожным знаком 
<<проезд запрещен>>.
   
   Таким образом, с одной стороны, перечисленные 12~терминов разделены на три группы в 
зависимости от природы обозначаемых ими сущностей, с другой стороны, они разделены на 
два класса относительно первоначального источника их возникновения: человека, 
создающего знания, и технической системы как генератора цифровых данных. Первый класс 
терминов обозначен на рис.~\ref{f1zat} белым фоном, второй класс~--- серым.
   
   Разделение 12~терминов на два класса основано на следующем теоретическом 
положении Брукса: информация не является идентичной сенсорным данным. Процесс 
создания информации может зависеть от результатов наблюдений или измерений, но 
данные, полученные таким образом, должны быть сначала интерпретированы когнитивными 
структурами знаний человека, чтобы затем стать концептами\footnote{Результаты 
интерпретации данных наблюдений или измерений Брукс называет не концептами, а ментальной информацией 
как структурной составляющей систем знаний человека. Однако это различие в названиях является не 
понятийным, а чисто лексическим, т.\,е.\ одно и то же рассматриваемое понятие по-разному называется в 
этой статье и в работе Брукса.}~\cite{25zat, 26zat}.
   
   Таким образом, данные как результат наблюдений или измерений по определению не 
являются знаковой информацией, так как не являются формами вербальных и/или 
невербальных знаков или знаковых образований, отчужденными от участников социальных 
коммуникаций, т.\,е.\ не являются формами представления знаний человека в сфере 
социальных коммуникаций.
   
   Однако данные могут быть \textit{интерпретированы} или \textit{концептуализированы} 
с помощью системы знаний человека. Иначе говоря, на основе анализа данных человеком 
сначала могут быть сгенерированы концепты в рамках некоторой системы его знаний, 
используемой для их интерпретации, а затем эти концепты могут быть выражены в виде 
отчужденных форм представления концептов, т.\,е.\ в виде знаковой информации.
   
   Приведенные в этом разделе положения позволяют уточнить формулировку цели статьи: 
предлагается семиотическая модель для описания взаимосвязей денотатов, концептов, 
информационных объектов и компьютерных кодов всех категорий, \textit{кроме нулевой}. 
Нулевая категория компьютерных\linebreak
кодов отсутствует в семиотической модели по следующей 
причине. Основная цель по\-стро\-ения семиотической модели заключается в создании 
теоретической основы для разработки методов\linebreak
 трехкомпонентной кодировки денотатов, 
концептов и информационных объектов, которым по определению соответствуют только 
коды трех категорий, т.\,е.\ нулевая категория кодов для них не\linebreak\vspace*{-12pt}
\pagebreak

\end{multicols}

\begin{figure} %fig2
\vspace*{2pt}
\begin{center}
\vspace*{1pt}
\mbox{%
\epsfxsize=163.053mm
\epsfbox{zat-2.eps}
}
\end{center}
\vspace*{-6pt}
\Caption{Два интерфейса между сущностями ментальной сферы, социальной и цифровой сред
\label{f2zat}}
\vspace*{12pt}
\end{figure}

\begin{multicols}{2}


\noindent
 применяется. Иначе говоря, 
в предлагаемой семиотической модели, являющейся развитием семиотического 
треугольника Г.~Фреге, коды нулевой категории исключены из рассмотрения, так как для 
трех вершин треугольника достаточно иметь только три категории кодов для денотатов, 
концептов и информационных объектов\footnote[1]{Однако для постановки и решения задач 
формирования концептов на основе уже имеющихся цифровых данных~\cite{8zat}, которые здесь не 
рассматриваются, понадобится другая семиотическая модель, включающая коды всех категорий, в том числе 
нулевой.}.


\vspace*{-6pt}
\section{Концепты, информационные объекты и компьютерные коды}

   Традиционные таблицы кодировки множества литер непосредственно применимы только 
к тем информационным объектам, которые являются \textit{линейными конкатенациями 
литер или символов}. Если концепты могут быть представлены в виде таких 
информационных объектов как форм знаков и разделены на литеры (I~этап), то затем этим 
литерам могут быть поставлены в соответствие компьютерные коды (II~этап).
   
   Рисунок~2 иллюстрирует двухэтапный процесс назначения компьютерных кодов 
концептам, которые являются значениями вербальных знаков. Сначала концепты 
выражаются с помощью форм знаков (реализация интерфейса~1), т.\,е.\ словами или 
устойчивыми словосочетаниями. Затем слова делятся на литеры и каждой литере ставится в 
соответствие компьютерный код этой литеры (реализация интерфейса~2). Знаки как 
двуединые сущности, каждая из которых включает концепт и форму его представления, 
обозначены в виде кругов, размещенных на линии интерфейса~1 (см.\ рис.~2). При 
таком двухэтапном процессе необходимо учитывать явление асимметрии, о чем говорилось в 
разд.~1. 

Рассмотрим более подробно интерфейсы~1 и~2.
\pagebreak
   
   На рис.~\ref{f1zat} и~2 показаны два интерфейса между сущностями ментальной 
сферы, социальной и цифровой сред:
   \begin{itemize}
\item интерфейс~1 между значениями знаков (концептами ментальной сферы) и 
формами знаков (информационными объектами среды социальных коммуникаций) 
условно обозначен штриховой линией;
\item интерфейс~2 между литерами, формами знаков (информационными объектами) 
и компьютерными кодами цифровой среды обозначен штрихпунктирной линией.
\end{itemize}
   
   На рис.~2 выше штриховой линии изображена ментальная сфера, ниже 
   штрихпунктирной линии~--- цифровая среда. Между ними~--- среда социальных 
коммуникаций. Слева на этом рисунке изображен компьютер. Пользователь компьютера 
представляет свои концепты в виде вербальных текс\-тов, а компьютер переводит литеры слов 
в коды~ИС.

   
   Первый интерфейс традиционно исследуют в таких областях знаний, как семиотика, 
лингвистика, информационная наука, информатика как информационно-компьютерная 
наука~[4, 27--31]. Согласно Шрейдеру, <<наиболее 
принципиальные вопросы информатики всегда возникали \textit{на стыке информации и 
знания} (курсив~--- \textbf{И.\,З.})~--- там, где речь шла о превращении одного в 
дру-\linebreak гое>>~[29, с.~50].
   
   Описание первого интерфейса в семиотике и лингвистике часто ведется в терминах 
<<план содержания>>, <<план выражения>>, <<знак>>, <<значение %\linebreak
 знака>>, <<форма 
знака>> и <<знаковая система>>. В~качестве примера описания первого интерфейса\linebreak 
процитируем основное положение концепции, лежащей в основе 
   <<Толково-комбинаторного словаря современного русского языка>> 
(ТКС)\footnote{ТКС~--- это словарь синтеза, его цель~--- помочь построить текст, т.\,е., во-первых, дать 
читателю словаря возможность найти все мыслимые формы выражения нужной ему идеи (концептов) и 
отобрать среди них те, которые наилучшим образом подходят к данному контексту, а во-вторых, указать 
правильный способ сочетания выбранных форм. ТКС можно использовать и в обратном направлении, т.\,е.\ 
от текста к смыслу (концептам), но его организующий принцип~--- это движение от смысла к тексту 
(сочетанию информационных объектов)~[32, с.~5].}. Главная идея %\linebreak
 создания этого словаря 
состоит в том, что <<естест\-венный язык есть система, уста\-нав\-ли\-ва\-ющая соответствие между 
заданным смыслом и всеми выражающими его текстами; соответственно, %\linebreak
 лингвистическое 
описание некоторого языка должно представлять собой множество правил, ставящих в 
соответствие \textit{всякому смыслу все тексты данного языка, несущие этот смысл}>> 
(курсив~--- \textbf{И.~З.})~[32, с.~4].
{\looseness=-1

}
   
   Естественный язык по определению является знаковой системой, но все естественные 
языки являются только одной из категорий знаковых сис\-тем. В качестве примера еще одной 
категории знаковых систем можно назвать язык карты и другие геоязыки~\cite{33zat}. 
Согласно Барту, знаковые системы всех категорий объединяет их возможность членить 
знания человека на концепты и устанавливать соответствие между множеством концептов и 
множеством форм их представления в виде вербальных и/или невербальных 
текстов~\cite{34zat}.
   
   В общем случае соответствие между этими двумя множествами, обеспечиваемое 
знаковой системой, обладает двумя особенностями. \textit{Во-первых}, со временем может 
изменяться <<объем значения>> любого концепта, например выражаемого словом или 
устойчивым словосочетанием естественного языка. \textit{Во-вторых}, разные знаковые 
системы различаются по способам и правилам членения систем знаний на концепты. В силу 
этих особенностей <<объемы значений>> практически любых соотносимых в двуязычных 
словарях пар слов не совпадают~\cite{35zat}.
   
   Таким образом, именно знаковые системы являются основой для реализации первого 
интерфейса. Именно они традиционно обеспечивают между концептами и 
информационными объектами как формами знаков взаимосвязи, обладающие двумя 
перечисленными особенностями. Отметим, что содержание рис.~2 имеет два 
существенных отличия от содержания приведенного положения концепции ТКС.
   
   Во-первых, в общем случае при реализации первого интерфейса в качестве 
информационных\linebreak
объектов могут выступать не только слова и словосочетания текстов на 
естественных языках, но и изображения (образные тексты)~--- графики, диаграммы, рисунки, 
карты и~т.\,д.
   
   Во-вторых, на рис.~2 первый интерфейс рас\-смат\-ри\-ва\-ет\-ся в сочетании со вторым 
интерфейсом~--- между литерами и компьютерными кодами цифровой среды.
   
   Второй интерфейс в информатике как компьютерной науке часто изучается независимо 
от первого. При этом информационные объекты в\linebreak
 процессе изучения, как правило, 
изначально определяются не как формы представления концептов, а как последовательности 
литер некоторого алфавита или заданного множества символов. Эти последовательности в 
частном случае могут представлять собой фразы на естественном языке. 
При постановке и решении 
широкого спектра проблем информатики 
как  компьютерной науки последовательности литер используются в качестве исходных данных  
\linebreak\vspace*{-12pt}

\pagebreak

\end{multicols}

\begin{figure} %fig3
\vspace*{1pt}
\begin{center}
\vspace*{1pt}
\mbox{%
\epsfxsize=163.053mm
\epsfbox{zat-3.eps}
}
\end{center}
\vspace*{-9pt}
\Caption{Два интерфейса, система формокодов и знаковая система
\label{f3zat}}
\end{figure}

\begin{multicols}{2}

\noindent
как в 
классических работах~\cite{36zat}, так и в современных~\cite{37zat}.
   
   Для любой фразы естественного языка литеры составляющих ее слов хранятся в памяти 
компьютера в виде кодов. При отображении слов на экране монитора и/или при печати их на 
бумаге используются таблицы кодировки, которые обеспечивают соответствие между этими 
компьютерными кодами и литерами слов. Иначе говоря, эти таблицы обеспечивают второй 
интерфейс методом кодировки литер. Естественно, что для кодирования лексических 
значений слов и смысла фраз естественного языка эти таблицы кодировки не предназначены.
   
   Теоретически имеется возможность кодировать не отдельные литеры вербального текста, 
а каждое слово (форму вербального знака) этого текста как единое целое, если в ИС в 
дополнение к таблицам кодировки литер использовать систему формокодов, подробное 
описание которой дано в работе~\cite{38zat}.
   
   Отметим, что по определению из работы~\cite{38zat} \textit{формокодом} называется 
общепринятое или нор\-ма\-тив\-но-за\-дан\-ное сочетание кода ИС и информационного объекта, 
являющегося формой знака. На рис.~3 показано, что каждый знак является 
дву\-еди\-ной сущностью, включающей концепт как значение знака и форму знака, а каждый 
формокод является дву\-еди\-ной сущностью, включающей форму знака и код этой формы. 
Знаки обозначены в виде кругов, размещенных на линии интерфейса~1, а формокоды~--- в 
виде кругов, размещенных на линии интерфейса~2. Формокодовая 
кодировка, т.\,е.\ кодировка форм знаков компьютерными кодами, принципиально 
отличается от кодировки литер, так как формокодовая кодировка применима к более 
широкому спектру знаковых систем, в том числе к образным знаковым системам, что будет 
показано в разд.~5. Таким образом, рис.~3 отличается от рис.~2 
использованием формокодовой системы для кодировки форм знаков, с помощью которой 
могут быть сформированы информационные коды форм знаков.
   
   Если бы в знаковых системах отсутствовало явление знаковой асимметрии, то каждому 
элементарному концепту соответствовала бы только одна форма его представления, и 
наоборот. В таком случае уникальный в пределах ИС код формы знака можно было бы 
использовать и как код соответствующего этой форме концепта. Следовательно, можно было 
бы поставить во взаимно однозначное соответствие концепт как значение знака и 
компьютерный код. Однако из-за асимметрии знаковых систем (например, явления 
синонимии и полисемии~\cite{39zat} в естественных языках) соответствие концептов и 
форм их представления в общем случае является многозначным.
  
   Как уже было отмечено, отсюда следует важный вывод о том, что задача построения 
взаимно однозначных отношений между концептами и компьютерными кодами \textit{через 
кодирование форм представления концептов} (но не самих концептов или дескрипторов 
тезауруса ) в общем случае является \textit{трудноразрешимой}.
   
   Если в ИС используется тезаурус, то система формокодов может быть построена на его 
основе c использованием дескрипторов тезауруса. Этот способ построения систем 
формокодов, использующий тезаурус, является более универсальным, так как он применим 
для форм вербальных и образных знаков, не являющихся линейными последовательностями 
литер, символов и других знаковых примитивов~\cite{38zat}.

\section{Третий интерфейс и~семиотическая модель}
   
   Обсудив вопросы использования формокодовой системы для кодировки форм знаков, с 
по\-мощью которой могут быть сформированы информационные коды форм знаков, а также 
ее отличий от таблиц кодировок множества литер, рассмотрим задачу формирования 
семантических кодов концептов. Для решения этой задачи необходимо рассмотреть еще 
один интерфейс между концептами как значениями знаков и компьютерными кодами.
   
   До сих пор рассмотрение двух интерфейсов велось с использованием рис.~2 и~3, 
   с помощью которых иллюстрировались отличия формокодовой системы для 
кодировки форм знаков от таблиц кодировок множества литер. Однако существует еще один 
интерфейс~--- третий~--- между концептами ментальной сферы как значениями знаков и 
компьютерными кодами, который на этих рисунках не был изображен. На рис.~4 
приведены все три интерфейса, среди них и интерфейс~3 между ментальной сферой и 
цифровой средой, который обозначен сплошной двойной линией.
   
   На рис.~\ref{f4zat} показано, что, кроме информационных кодов форм знаков, 
формируются семантические коды концептов. Две стрелки, расположенные ниже 
компьютера, условно обозначают процесс кодировки форм знаков и концептов. Для 
формирования кодов концептов необходимо определить еще одну двуединую сущность, 
аналогичную формокоду, что будет сделано чуть позже. Как уже отмечалось, по 
определению коды первой категории ИС соответствуют концептам, коды второй категории 
ИС~--- формам представления концептов.
   
   Для формирования кодов первой и второй категории используется тезаурус. Описания 
нескольких тезаурусов и особенностей их формирования можно найти в работах~[40--44]. 
Отметим, что на рис.~\ref{f4zat} денотаты и коды третьей категории не 
показаны.
   
   Примером реализации третьего интерфейса без использования тезауруса является первый 
коммерчески доступный микрокомпьютер MITS \mbox{Altair}~8800. В базовом комплекте поставки 
у него не было никаких периферийных устройств в современном понимании. На 
микрокомпьютере можно было программировать, задавая определенные пользователем наборы двоичных 
по\-сле\-до\-ватель\-ностей вручную. Для их ввода в этот\linebreak микрокомпьютер использовались 
двухпозиционные переключатели, расположенные на передней панели компьютера.
 С~помощью одного переключателя можно было ввести только одну двоичную цифру (ноль или 
единицу). Результат своей работы микрокомпьютер показывал, включая или выключая 
лампочки на передней панели\footnote{Двухпозиционные переключатели можно трактовать как 
промежуточную среду перехода от концептов пользователя микрокомпьютера MITS Altair~8800 к кодам. 
Однако эту среду не будем учитывать при рассмотрении третьего интерфейса.}. С по\-мощью 
двухпозиционных переключателей и лампочек в этом микрокомпьютере был реализован 
простейший вариант третьего интерфейса\footnote{Описание микрокомпьютера MITS Altair 8800 и его 
цветную фотографию можно найти по адресу {\sf http://historywired.si.edu/\linebreak object.cfm?ID=339}.}.
   
   Этот пример человеко-машинного интерфейса интересен тем, что человек мог почти 
напрямую взаимодействовать с кодами компьютера, используя только двоичные 
переключатели и лампочки на передней панели как формы представления компьютерных 
кодов вне цифровой среды микрокомпьютера.

  
   Вернемся к рис.~\ref{f4zat}, на котором третий интерфейс обозначен двойной сплошной 
линией. Выше этой двойной линии находится ментальная сфера, ниже~--- цифровая среда. 
Для описания третьего интерфейса предлагается применить систему семо-\linebreak\vspace*{-12pt}
\pagebreak

\end{multicols}
\begin{figure} %fig4
\vspace*{1pt}
\begin{center}
\vspace*{1pt}
\mbox{%
\epsfxsize=163.074mm
\epsfbox{zat-4.eps}
}
\end{center}
\vspace*{-9pt}
\Caption{Три интерфейса
\label{f4zat}}
\end{figure}

\begin{multicols}{2}

\noindent
кодов 
(рис.~\ref{f5zat}). Отметим, что по определению из работы~\cite{38zat} \textit{семокодом} 
называется общепринятое или нормативно-заданное сочетание в виде бинарного отношения 
концепта как значения знака и компьютерного кода первой категории. Семокодовая 
кодировка используется для формирования семантических кодов концептов ИС (см.\ 
рис.~\ref{f4zat} и~\ref{f5zat}).
   
   Подводя итоги рассмотрения трех интерфейсов ИС, процессов формирования 
информационных и семантических кодов, показанных на рис.~\ref{f4zat} и~\ref{f5zat}, 
построим табл.~\ref{t1zat}, которая содержит:
   \begin{itemize}
\item названия ментальной сферы, социальной и циф\-ро\-вой сред (в первой строке и 
в первой ко\-лонке);
\item моносферные\footnote{Моносферными будем называть те термины, которые 
обозначают сущности, относящиеся только к одной сфере или среде.} термины <<концепт 
(значение знака)>>, <<информационный объект (форма знака)>> и 
<<компьютерный код>> (расположены на главной диагонали, кроме первой 
ячейки первой строки);
\item полисферные\footnote{Полисферными будем называть те термины, которые 
обозначают сущности, относящиеся более чем к одной сфере или среде.} термины <<знак>>, 
<<формокод>> и <<семокод>>, которые являются по своей природе двуедиными 
сущностями (в остальных ячейках таблицы).
\end{itemize}
 
   Таблица~\ref{t1zat} построена только в целях компактного описания принадлежности 
моносферных терминов к одной сфере (среде), а полисферных~--- к~нескольким.
   
   Для построения семиотической модели будем дополнительно рассматривать сферу 
материальных объектов и денотаты, а также учитывать определенное в разд.~2 разделение 
компьютерных кодов концептов, информационных объектов и денотатов на три категории. 
Рассмотрение сферы материальных объектов и принадлежащих ей денотатов обусловлено 
тем, что в ряде задач возникает необходимость формирования кодов денотатов в дополнение 
к кодам форм знаков и концептов.

\end{multicols}
\begin{figure} %fig5
\vspace*{1pt}
\begin{center}
\vspace*{1pt}
\mbox{%
\epsfxsize=163.074mm
\epsfbox{zat-5.eps}
}
\end{center}
\vspace*{-9pt}
\Caption{Три интерфейса и три двуединых сущности: знак, семокод и формокод
\label{f5zat}}
\vspace*{-6pt}
\end{figure}


\begin{table}\small %tabl1
\begin{center}
\Caption{Моносферные и полисферные термины
\label{t1zat}}
\vspace*{2ex}

\begin{tabular}{|c|c|c|c|}
\hline
\tabcolsep=0pt\begin{tabular}{c}Названия одной\\ сферы и двух сред $\rightarrow$\\ $\downarrow$\end{tabular}&
\tabcolsep=0pt\begin{tabular}{c}Ментальная сфера\\ (знаний человека)\end{tabular}&
\tabcolsep=0pt\begin{tabular}{c}Среда  социальных\\ коммуникаций\end{tabular}&
Цифровая среда\\
\hline
\tabcolsep=0pt\begin{tabular}{c}Ментальная сфера\\ (знаний человека)\end{tabular}&
\tabcolsep=0pt\begin{tabular}{c}\textit{Концепт}\\ \textit{(значение знака)}\end{tabular}&Знак&Семокод\\
\hline
\tabcolsep=0pt\begin{tabular}{c}Среда социальных\\ коммуникаций\end{tabular}&
Знак&
\tabcolsep=0pt\begin{tabular}{c}\textit{Информационный объект}\\ \textit{(форма знака)}\end{tabular}&Формокод\\
\hline
Цифровая среда&Семокод&Формокод&
\tabcolsep=0pt\begin{tabular}{c}\textit{Компьютерный}\\ \textit{код}\end{tabular}\\
   \hline
   \end{tabular}
   \end{center}
   \vspace*{-6pt}
   \end{table}
   
\begin{multicols}{2}

   В процессе построения семиотической модели для описания взаимосвязей денотатов, 
концептов, информационных объектов и компьютерных кодов всех категорий, кроме 
\textit{нулевой}, будем использовать рис.~\ref{f6zat}. На нем кроме ментальной сферы, 
социальной и цифровой сред изображена сфера материальных объектов и явлений, а также 
один денотат как физический объект или явление.


   
   Отметим, что в общем случае денотат может принадлежать и к другим средам, например 
циф\-ро\-вой. Так в описании проблем направляемой генерации и эволюции ЦСЗ 
рассматриваются денотаты, представляющие собой совокупности циф\-ро\-вых 
данных~\cite{8zat, 9zat}, которые по определению из разд.~2 относятся к цифровой 
среде.
%\pagebreak
   
   Таким образом, предлагаемая в статье семиотическая модель включает пять 
составляющих. Это:

\noindent
   \begin{enumerate}[(1)]
\item ментальная сфера, материальная сфера физических объектов и явлений, 
социальная и циф\-ро\-вая среды;
\end{enumerate}

\end{multicols}

\begin{figure} %fig6
\vspace*{1pt}
\begin{center}
\vspace*{1pt}
\mbox{%
\epsfxsize=138.729mm
\epsfbox{zat-6.eps}
}
\end{center}
\vspace*{-9pt}
\Caption{Семиотическая модель для описания взаимосвязей денотатов, концептов, 
информационных объектов как форм знаков и компьютерных кодов
\label{f6zat}}
\end{figure}


\begin{multicols}{2}

\noindent
\begin{enumerate}[(1)]
\setcounter{enumi}{1}
\item денотат, соответствующие ему концепт (значение знака) ментальной сферы и 
информационный объект (форма знака) социальной среды, а также компьютерные 
коды трех категорий цифровой среды для денотата, концепта и информационного 
объекта;
\item знак, объединяющий концепт как значение знака и информационный объект 
как форму знака, являющийся элементом знаковой сис\-те\-мы ИС;
\item формокод, объединяющий информационный объект как форму знака и 
компьютерный код второй категории, являющийся элементом формокодовой 
кодировки для формирования информационных кодов форм знаков ИС;
\item семокод, объединяющий концепт как значение знака и компьютерный код 
первой категории, являющийся элементом семокодовой кодировки для 
формирования семантических кодов концептов ИС.
\end{enumerate}
  
   Естественно, что рассматриваемая семиотическая модель основана на треугольнике 
Г.~Фреге, тремя вершинами которого являются значение знака (концепт), форма знака (как 
частный случай информационного объекта) и денотат знака (ма\-териальной, цифровой или 
иной природы). Новыми компонентами в этой модели являются\linebreak формокод и семокод как 
ключевые элементы формокодовой и семокодовой кодировок, а также три категории %\linebreak
 кодов. 
Они образуют в цифровой среде треугольник, который предлагается называть 
<<компьютерным треугольником>>. Включение в модель формокода, семокода и кодов трех 
категорий %\linebreak
 позволяет разрабатывать принципиально новые методы трехкомпонентной 
кодировки, обеспечивающие в дополнение к литерной (символьной) кодировке 
формирование:
   \begin{itemize}
\item информационных кодов форм знаков с по\-мощью формокодовой кодировки;
\item семантических кодов концептов с помощью семокодовой кодировки;
\item объектных кодов денотатов (в статье определены коды денотатов, но система 
их кодировки не рассматривается).
\end{itemize}

   В следующем разделе приводится пример построения фрагментов формокодовой и 
семо\-ко\-довой кодировок для электронной библиотеки\linebreak геоизображений из 
работы~\cite{38zat}. Эти фрагменты 
 применяются для формирования кодов форм 
вербальных и образных знаков с помощью формокодовой кодировки и кодов концептов 
геообъектов, являющихся устьевыми областями рек, с помощью семокодовой кодировки. С 
помощью семокодовой кодировки в пределах электронной библиотеки геоизображений 
обеспечивается взаимно однозначное отношение между концептами геообъектов, 
явля\-ющих\-ся устьевыми областями рек, и компьютерными кодами.

\section{Пример трехкомпонентной кодировки}

   Рассмотрим пример построения трехкомпонентной кодировки, основанной на 
рассмотренной семиотической модели. Эта кодировка была разработана для электронной 
библиотеки гео\-изоб\-ра\-же\-ний в целях описания взаимосвязей шести геообъектов (денотатов), 
соответствующих им концептов и информационных объектов с компьютерными кодами трех 
категорий~\cite{38zat}.
   
   Для шести геообъектов, являющихся устьевыми областями рек, был разработан фрагмент 
вербально-образного тезауруса. В нем каждый из шести геообъектов (денотатов) и 
соответствующий ему концепт были представлены \textit{тремя дескрипторами}: для 
названия геообъекта на русском языке, названия на английском языке и для изображения 
геообъекта, которое является стилизованной пиктограммой соответствующей устьевой 
области реки.
   
   В соответствии с определением семиотическая модель для этого примера включает  такие 
компоненты, как:
   \begin{itemize}
\item ментальная сфера знаний пользователей электронной библиотеки 
геоизображений, материальная сфера физических объектов и явлений (шесть 
геообъектов, являющихся устьевыми областями рек), социальная и цифровая 
среды электронной библиотеки;
\item геообъект (денотат), соответствующие ему \textit{один концепт} ментальной 
сферы и \textit{три информационных объекта} социальной среды (название на 
русском языке, название на английском языке и изображение геообъекта), а также 
компьютерные коды трех категорий цифровой среды для денотата, концепта и 
информационных объектов;
\item знак, объединяющий концепт как значение знака и информационный объект 
как форму знака, являющийся элементом знаковой системы электронной 
библиотеки;
\item формокод, объединяющий информационный объект как форму знака и 
компьютерный код второй категории, а также используемый для формирования 
\textit{трех видов информационных кодов} форм знаков электронной библиотеки;
\item семокод, объединяющий концепт как значение знака и компьютерный код 
первой категории, а также используемый для формирования \textit{семантических 
кодов} концептов электронной биб\-лио\-те\-ки.
\end{itemize}
   
   Коды денотатов в этом примере не используются. Для построения трех видов 
информационных кодов форм знаков (для названий на русском языке, названий на 
английском языке и изображений геообъектов) и семантических кодов концептов 
электронной библиотеки в качестве исходных данных будем использовать классификацию 
усть\-евых областей рек и их частей по морфологическим признакам из 
работы~[45]. Отметим, что на основе именно этой классификации в 
   вербально-образном тезаурусе электронной библиотеки ранее были построены 
вербальные и образные дескрипторы для шести рассматриваемых геообъектов, в том числе 
их связи с другими дескрипторами тезауруса (рис.~\ref{f7zat}).
   
   Основные 6~видов устьевых областей рек приведены в табл.~\ref{t2zat}, в третьем 
столбце которой их названия выделены курсивом. Кроме этой таблицы в классификации 
используются 6~стилизованных пиктограмм устьевых областей рек, которые приведены в 
нижней части рис.~\ref{f7zat}. Построение трехкомпонентной кодировки осуществляется в 
три этапа.
   
   На первом этапе в целях формирования трех видов информационных кодов форм знаков 
образуем группу из 12~вербальных и шести образных знаков для описания шести 
геообъектов, явля\-ющих\-ся устьевыми областями рек. В примере будет использоваться только 
эта группа из 18~знаков, а не вся знаковая система электронной библиотеки. 
Рисунок~\ref{f7zat} и табл.~\ref{t2zat} позволяют получить первое представление о 
рассматриваемой группе знаков, используемых для описания геообъектов. В общем случае 
\textit{первый принцип} построения трехкомпонентной кодировки заключается в 
конструктивном описании знаковой системы электронной библиотеки, включающем ответы 
на следующие вопросы:
   \begin{enumerate}[1.]
\item Какие естественные языки используются для названий денотатов (в примере 
используются русский и английский языки)?
\item Что должен включать словарь электронной биб\-лио\-те\-ки для каждого 
естественного языка (в примере словарь каждого языка включает шесть названий 
устьевых областей рек)?
\end{enumerate}
\pagebreak

\end{multicols}

      \begin{figure} %fig7
   \vspace*{1pt}
\begin{center}
\vspace*{1pt}
\mbox{%
\epsfxsize=156.591mm
\epsfbox{zat-7.eps}
}
\end{center}
\vspace*{-9pt}
\Caption{Фрагмент вербально-образного тезауруса
\label{f7zat}}
\vspace*{-2pt}
\end{figure}

   \begin{table}\small %tabl2
   \begin{center}
   \Caption{Классификация устьевых областей рек~[45]
   \label{t2zat}}
   \vspace*{2ex}
   
   \begin{tabular}{|p{55mm}|p{40mm}|p{50mm}|}
   \hline
\multicolumn{1}{|c|}{Устьевой участок реки}&\multicolumn{1}{|c|}{Устьевое 
взморье}&\multicolumn{1}{|c|}{Устьевая область реки}\\
\hline
\multicolumn{1}{|l|}{\raisebox{-28pt}[0pt][0pt]{Однорукавный (бездельтовый)}}&Открытое без блокирующей косы&\textit{Простая без 
блокирующей косы}\\
\cline{2-3}
&Полузакрытое
\begin{itemize}
\item без блокирующей косы
\vspace*{-2pt}

\item с блокирующей косой
\end{itemize}&\textit{Эстуарная}
\begin{itemize}
\item \textit{без блокирующей косы}
\vspace*{-2pt}

\item \textit{с блокирующей косой}
\end{itemize}\\[-9pt]
\hline
\multicolumn{1}{|l|}{\raisebox{-22pt}[0pt][0pt]{Мало- и многорукавный (дельтовый)}}&Полузакрытое
\begin{itemize}
\item без блокирующей косы
\vspace*{-2pt}

\item с блокирующей косой
\end{itemize}&\textit{Эстуарно-дельтовая}
\begin{itemize}
\item \textit{без блокирующей косы}
\vspace*{-2pt}

\item \textit{с блокирующей косой}
\end{itemize}\\[-9pt]
\cline{2-3}
&Открытое&\textit{Дельтовая с дельтой выдвижения}\\
\hline
\end{tabular}
\end{center}
\vspace*{-4pt}
\end{table}

\begin{multicols}{2}


\noindent
\begin{enumerate}[1.]
\setcounter{enumi}{2}
\item Будут ли использоваться изображения денотатов (в примере используются 
стилизованные пиктограммы)?\\[-15pt]
\item Что должен включать словарь изображений денотатов (в примере словарь 
изображений включает шесть стилизованных пиктограмм, приведенных на 
рис.~\ref{f7zat})?\\[-15pt]
   \end{enumerate}
   
   На втором этапе будем использовать вербально-образный тезаурус электронной 
библиотеки, в рамках которого сформируем группу из 12~вербальных и 6~образных 
дескрипторов, используемых для описания 6~геообъектов, являющихся устьевыми 
облас-\linebreak\vspace*{-9pt}\columnbreak

\noindent
тями рек (см.\ рис.~\ref{f7zat}). Интересующие нас 6~форм представления образных 
дескрипторов этой группы расположены в нижней части рис.~\ref{f7zat}, на ко\-тором 
приведены названия только для двух из\linebreak
 двенадцати вербальных дескрипторов, 
соответствующих одному геообъекту (эстуарно-дельтовая устьевая область с блокирующей 
косой). В общем случае \textit{второй принцип} построения трехкомпонентной кодировки 
заключается в конструктивном описании вербальных и/или образных дескрипторов 
вербально-образного тезауруса электронной биб\-лио\-те\-ки, определяемых на основе группы 
вербальных и/или образных знаков, сформированной на первом этапе.
%\pagebreak
 
     \begin{table*}\small %tabl3
   \begin{center}
\parbox{142mm}{\Caption{Коды 6 концептов, 12~названий вербальных дескрипторов и 6 пиктограмм образных 
дескрипторов вербально-образного тезауруса электронной библиотеки
   \label{t3zat}}
   }
   
   
   \vspace*{2ex}
   
   \begin{tabular}{|c|l|c|}
   \hline
\tabcolsep=0pt\begin{tabular}{c}Коды концептов\\ геообъектов,\\ являющихся устьевыми\\ областями рек\end{tabular}&
\tabcolsep=0pt\begin{tabular}{c}Названия устьевых  областей на русском\\
и английском языках и коды названий\\ (даны в скобках)\end{tabular}&
\tabcolsep=0pt\begin{tabular}{c}Коды форм \\ образных\\ дескрипторов\\ (см.\ рис.~\ref{f7zat})\end{tabular}\\
\hline
001к&
\tabcolsep=0pt\begin{tabular}{l}Простая без блокирующей косы (001р)~---\\ Simple river mouth without blocked spit 
(001а)\end{tabular}&001о\\
\hline
010к&
\tabcolsep=0pt\begin{tabular}{l}Эстуарная без блокирующей косы (010р)~---\\ Estuary without blocked spit (010а)\end{tabular}&010о\\
\hline
011к&
\tabcolsep=0pt\begin{tabular}{l}Эстуарная с блокирующей косой (011р)~---\\ Estuary with blocked spit (011а)\end{tabular}&011о\\
\hline
100к&\tabcolsep=0pt\begin{tabular}{l}Эстуарно-дельтовая без блокирующей косы (100р)~---\\ Silt delta without blocked spit (100а)\end{tabular}&100о\\
\hline
101к&\tabcolsep=0pt\begin{tabular}{l}Эстуарно-дельтовая с блокирующей косой (101р) ~---\\ Silt delta with blocked spit 
(101а)\end{tabular}&101о\\
\hline
110к&\tabcolsep=0pt\begin{tabular}{l}Дельтовая с дельтой выдвижения (110р)~---\\ Protruding delta (110а)\end{tabular}&110о\\
\hline
\end{tabular}
\end{center}
\end{table*}

   На третьем этапе всем 12~вербальным названиям и шести пиктограммам для 6~геообъектов 
   и соответствующим 6~концептам в электронной библиотеке назначаются 
компьютерные коды (табл.~\ref{t3zat})\footnote{Отметим, что в таблице приведены не сами коды цифровой среды, а их 
алфавитно-цифровые обозначения в среде социальных коммуникаций между пользователями электронной 
библиотеки.}: 6~кодов для концептов (001к--110к, где литера~<<к>> 
обозначает концепт), 6~кодов для русскоязычных названий геообъектов (001р--110р, где 
литера~<<р>> обозначает русский язык), 6~кодов для англоязычных названий (001а--110а, 
где литера~<<а>> обозначает английский язык) и 6~кодов для пиктограммам гео\-объек\-тов 
(001о--110о, где литера~<<о>> обозначает визуальные образы денотатов).
   
   В общем случае \textit{третий принцип} построения трехкомпонентной кодировки 
заключается в назначении компьютерных кодов денотатам (в примере не рассматриваются), 
соответствующим им концептам и информационным объектам. Значения кодов 
определяются на основе идентификаторов соответствующих дескрипторов 
   вербально-образного тезауруса (см.\ коды в табл.~\ref{t3zat} и на рис.~\ref{f7zat}).
 
 
   На основе шести концептов геообъектов, объединенных с образными формами их 
пред\-став\-ле\-ния (пиктограммами), ранее были построены образные дескрипторы 
   вербально-образного тезауруса~\cite{38zat}. В~соответствии с определением 
формокода из разд.~3 формы образных знаков, которые объединены с соответствующими им 
кодами, указанными в третьей колонке табл.~\ref{t3zat}, образуют группу \textit{образных 
формокодов}. На основе этой группы формируется \textit{первый вид информационных 
кодов} форм образных знаков электронной библиотеки, а именно коды 001о--110о.
   
   На основе шести концептов, объединенных с названиями геообъектов на русском языке, 
ранее были построены вербальные дескрипторы русскоязычной части вербально-образного 
тезауруса~\cite{38zat}. В~соответствии с определеним формокода из разд.~3 названия 
шести устьевых областей рек на русском языке, которые объединены с соответствующими 
им кодами, указанными во второй колонке табл.~\ref{t3zat}, образуют группу 
\textit{вербальных формокодов} русскоязычной формокодовой системы электронной 
биб\-лио\-те\-ки. На основе этой группы формируется \textit{второй вид информационных кодов} 
форм русскоязычных знаков электронной библиотеки, а именно коды 001р--110р.
   
   На основе шести концептов, объединенных с названиями геообъектов на английском 
языке, ранее были построены вербальные дескрипторы англоязычной части 
   вербально-образного тезауруса~\cite{38zat}. В соответствии с определеним 
формокода из разд.~3 названия шести устьевых областей рек на английском языке, которые 
объединены с соответствующими им кодами, указанными во второй колонке 
табл.~\ref{t3zat}, образуют группу \textit{вербальных формокодов} англоязычной 
формокодовой системы электронной библиотеки. На основе этой группы формируется 
\textit{третий вид информационных кодов} форм англоязычных знаков электронной 
библиотеки, а именно коды 001а--110а.
   
   В соответствии с определением семокода из разд.~3 концепты, которые объединены с 
соответствующими им кодами, указанными в первой колонке табл.~\ref{t3zat}, образуют 
группу семокодов электронной библиотеки. На основе этой группы формируются 
\textit{семантические коды} концептов (значений) знаков электронной библиотеки, а именно 
коды 001к--110к.
   
   Семантические отношения между дескрипторами вербально-образного тезауруса на 
рис.~\ref{f7zat} были сформированы на основе системы отношений между геообъектами, 
используемой в системе классификации из работы~[45]. Точечными стрелками 
условно обозначены родовидовые отношения в тезаурусе, сплошными~--- отношения 
<<часть--целое>>, штриховыми~--- ассоциативные отношения, штрихпунктирными~--- 
отношения между дескрипторами разных модальностей (вербальной и образной), но с 
одинаковыми концептами (значениями). Стрелка <<штрих-двойной-пунктир>> обозначает 
отношения между дескрипторами разных естественных языков (русский и английский), но с 
одинаковыми концептами (значениями).
   
   В результате выполнения трех этапов были построены следующие коды:
   \begin{itemize}
\item информационные коды форм образных знаков электронной библиотеки 
001о--110о;
\item информационные коды форм русскоязычных знаков электронной библиотеки 
001р--110р;
\item информационные коды форм англоязычных знаков электронной библиотеки 
001а--110а;
\item семантические коды концептов (значений) знаков электронной библиотеки 
001к--110к.
\end{itemize}
   
   Как отмечалось ранее, объектные коды в этом примере не рассматривались. 
Семантические коды концептов обеспечивают \textit{взаимно однозначное соответствие} 
между концептами и компьютерными кодами, т.\,е.\ каждому концепту соответствует один 
код, а каждому семантическому коду~--- один концепт.
   
   Рассмотренный пример построения трехкомпонентной кодировки, основанной на 
рассмотренной семиотической модели, наглядно иллюстрирует следующие ее отличия от 
кодировки символов или литер:
   \begin{itemize}
\item при кодировании текстов и изображений в электронных библиотеках и 
других ИС имеется возможность однозначной идентификации и кодирования 
концептов с помощью семантических кодов;
\item коды концептов, полученные в результате трехкомпонентной кодировки, не 
зависят от модальности форм представления концептов (вербальной или 
образной), что позволяет\linebreak
 организовать концептуальное индексирование и 
семантический поиск в электронных гео\-биб\-лио\-те\-ках~\cite{38zat};
\item при кодировании слов и словосочетаний текс\-тов на разных естественных 
языках в электронных библиотеках и других ИС совокупность информационных 
кодов форм вербальных знаков разных языков, по сути, представляет собой индекс 
межязыковых переходов~[46, 47].
\end{itemize}

\section{Заключение}

   Важно отметить, что частным случаем построенной семиотической модели, в которой 
отсутствуют формокод, семокод и компьютерные коды всех трех категорий, является 
семиотический треугольник Г.~Фреге. Спектр приложений, в которых рассмотренная 
семиотическая модель может быть использована, во многом будет определяться уровнем 
и/или средствами ее реализации в ИС: прикладная задача, среда разработки прикладных 
задач ИС, сис\-те\-ма управления базами данных, форматы данных ИС, язык 
программирования, операционная система и~т.\,д.
   
   Однако существует два основных ограничения на использование рассмотренной 
семиотической модели, которые не зависят от уровня и средств ее реализации в ИС. Первое 
ограничение является следствием стационарности семиотической модели, т.\,е.\ эту модель 
можно применять только в случаях стационарных взаимосвязей денотатов, концептов, 
информационных объектов и компьютерных кодов.
   
   Второе ограничение, которое уже упоминалось, является следствием исключения из 
рассмотрения кодов нулевой категории. Эту модель можно применять только в тех случаях, 
когда денотаты, концепты, информационные объекты, их бинарные отношения с 
компьютерными кодами, а также знаки, формокоды и семокоды ИС не зависят от кодов 
нулевой категории.
   
   Разработка методов трехкомпонентной кодировки, основанных на рассмотренной 
семиоти\-ческой модели, в том числе построение формокодовых и семокодовых кодировок, 
как правило,\linebreak существенно зависит от назначения тех ИС, для которых строится 
трехкомпонентная кодировка. От назначения ИС в первую очередь зависит число видов 
информационных, семантических и объектных кодов, которые должны формироваться в 
процессе трехкомпонентной кодировки.
   
   Главный результат описания взаимосвязей денотатов, концептов, информационных 
объектов и компьютерных кодов в виде семиотической модели заключается в том, что 
предложена теоретическая основа для разработки методов трехкомпонентной кодировки 
денотатов, концептов и информационных объектов, в том числе для построения взаимно 
однозначных отношений между концептами систем знаний человека и компьютерными 
кодами. Иначе говоря, использование предлагаемой семиотической модели позволяет 
исключить \textit{влияние асимметрии знаковых систем} на отношения между концептами 
систем знаний человека и компьютерными кодами.
   
   Полученные результаты были сформулированы в процессе развития идеи семиотического 
треугольника Г.~Фреге. Предлагаемое развитие этой идеи позволяет позиционировать 
компьютерные коды трех категорий как <<представителей>> денотатов, концептов и 
информационных объектов в цифровой среде электронных библиотек и других видов~ИС.
   
   В заключение отметим, что необходимость в дальнейшем развитии семиотической 
модели связана с разделением всех концептов на три категории (см.\ рис.~\ref{f1zat}) в 
контексте постановки актуальных проблем генерации ЦСЗ с учетом их эволюции во 
времени. Поэтому категоризация концептов и их эволюция во времени должны в будущем 
найти свое отражение в \textit{нестационарной семиотической модели} описания 
взаимосвязей денотатов, авторских, коллективных, конвенциональных концептов, 
информационных объектов и компьютерных кодов всех категорий.


{\small\frenchspacing
{%\baselineskip=10.8pt
\addcontentsline{toc}{section}{Литература}
\begin{thebibliography}{99}    
\bibitem{1zat}
FP7 Exploratory Workshop~4 ``Knowledge Anywhere Anytime.'' 
{\sf http://cordis.europa.eu/ist/directorate\_f/ f\_ws4.htm}.

     \bibitem{2zat}
     CORDIS ICT Programme Home. {\sf 
http://cordis.europa.\newline eu/fp7/ict/programme/home\_en.html}.
     
     \bibitem{3zat}
     ICT FP7 Work Programme. 
     {\sf ftp://ftp.cordis.europa.eu/ pub/fp7/ict/docs/ict-wp-2007-08\_en.pdf}.
     
     \bibitem{4zat}
     \Au{Gorn~S.}
     Informatics (computer and information science): Its ideology, methodology, and sociology~// 
The studies of information: Interdisciplinary messages~/ Eds.\ F.~Machlup, 
     U.~Mansfield.~--- N.Y.: John Wiley and Sons, Inc., 1983. P.~ 121--140.
     
     \bibitem{5zat}
     \Au{Wang~Y.}
     Cognitive informatics: A new transdisciplinary research field~// Brain and Mind, 2003. 
Vol.~4. No.\,2. P.~115--127.
     
     \bibitem{6zat}
     \Au{Wang Y.}
     On cognitive informatics~// Brain and Mind, 2003. Vol.~4. No.\,2. P.~151--167.
     
     \bibitem{7zat}
     \Au{Bryant A.}
     Cognitive informatics, distributed representation and embodiment~// Brain and Mind, 2003. 
Vol.~4. No.\,2. P.~215--228.
     
     \bibitem{8zat}
     \Au{Зацман~И.\,М.} 
     Концептуализация данных наукометрических исследований в научных электронных 
биб\-лио\-те\-ках~// Труды X Всероссийской конференции <<Электронные библиотеки: 
перспективные методы и технологии, электронные коллекции>>.~--- Дубна: \mbox{ОИЯИ}, 2008. 
С.~45--54.
     
     \bibitem{9zat}
     \Au{Зацман~И.\,М., Косарик~В.\,В., Курчавова~О.\,А.}
     Задачи представления личностных и коллективных концептов в цифровой среде~// 
Информатика и её применения, 2008. Т.~2. Вып.~3. С.~54--69.
     
     \bibitem{10zat}
     \Au{Nonaka~I.}
     The knowledge-creating company~// Harvard Business Review, 1991. Vol.~69. No.\,6. 
     P.~96--104.
     
     \bibitem{11zat}
     \Au{Nonaka~I., Takeuchi~H.}
     The knowledge-creating company.~--- N. Y.: Oxford University Press, 1995. [Пер.: 
     Нонака~И., Такеучи~Х. Компания~--- создатель знания.~--- М.: ЗАО 
<<Олимп-бизнес>>, 2003.]
     
     \bibitem{12zat}
     \Au{Шемакин~Ю.\,И., Романов~А.\,А.}
     Компьютерная семантика.~--- М.: НОЦ <<Школа Китайгородской>>, 1995.
     
     \bibitem{13zat}
     \Au{McArthur~D.}
     Information, its forms and functions: The elements of semiology.~--- Lewinton: The Edwin 
Mellen Press, Ltd., 1997.
     
     \bibitem{14zat}
     \Au{Колин~К.\,К.}
     Становление информатики как фундаментальной науки и комплексной научной 
проб\-ле\-мы~// Системы и средства информатики. Спец. вып. <<Научно-методологические 
проблемы информатики>>~/ Под ред. К.\,К.~Колина.~--- М.: ИПИ РАН, 2006. С.~7--58.
     
     \bibitem{15zat}
     \Au{Колин~К.\,К.}
     О структуре научных исследований по комплексной проблеме <<Информатика>>~// 
Социальная информатика.~--- М.: ВКШ при ЦК ВЛКСМ, 1990. С.~19--33.
     
     \bibitem{16zat}
     \Au{Колин~К.\,К.}
     Эволюция информатики и проблемы формирования нового комплекса наук об 
информации~// Научно-техническая информация, 1995. Сер.~1. №\,5. С.~1--7.
     
     \bibitem{17zat}
     \Au{Зацман~И.\,М.}
     Концептуальный поиск информационных объектов в электронных библиотеках 
научных документов~// Компьютерная лингвистика и интеллектуальные технологии. Труды 
международной конференции Диалог-2003.~--- М.: Наука, 2003. С.~710--716.
     
     \bibitem{18zat}
     \Au{Gladney H.\,M., Bennet~J.\,L.}
     What do we mean by authentic? What's the real McCoy?~// D-Lib Magazine, 2003. Vol.~9. 
No.\,7/8.
     
     \bibitem{19zat}
     \Au{Успенский~В.\,А.}
     К публикации статьи Г.~Фреге <<Смысл и денотат>>~// В кн.: Семиотика и 
информатика. Вып.~35.~--- М.: Языки русской культуры, 1997. С.~351--352.
     
     \bibitem{20zat}
     \Au{Фреге~Г.}
     Смысл и денотат~// В кн.: Семиотика и информатика. Вып.~35.~--- М.: Языки русской 
культуры, 1997. С.~352--379.
     
     \bibitem{21zat}
     \Au{Фреге~Г.}
     Понятие и вещь~// В кн.: Семиотика и информатика. Вып.~35.~--- М.: Языки русской 
культуры, 1997. С.~380--396.
     
     \bibitem{22zat}
     \Au{Добров~Б.\,В., Лукашевич~Н.\,В.}
     Тезаурус и автоматическое концептуальное индексирование в университетской 
информационной системе <<Россия>>~// Труды III Всероссийской конференции 
<<Электронные библиотеки: перспективные методы и технологии, электронные 
коллекции>>.~--- Петрозаводск: КарНЦ РАН, 2001. С.~78--82.
     
     \bibitem{23zat}
     \Au{Гак~В.\,Г.}
     Асимметрия~// Большой энциклопедический словарь <<Языкознание>>.~--- М.: 
Большая российская энциклопедия, 1998. С.~47.
     
     \bibitem{24zat}
     \Au{Зацман~И.\,М.}
     Семиотические основания и элементарные технологии информатики~// 
Информационные технологии, 2005. №\,7. С.~18--31.
     
     \bibitem{25zat}
     \Au{Brookes~B.\,C.}
     The foundations of information science. Part~I. Philosophical aspects~// J.\ Inf.\ 
Sci., 1980. No.\,2. P.~125--133.
     
     \bibitem{26zat}
     \Au{Зацман~И.\,М., Кожунова~О.\,С.}
     Предпосылки и факторы конвергенции информационной и компьютерной наук~// 
Информатика и её применения, 2008. Т.~2. Вып.~1. С.~77--97.
     
     \bibitem{27zat}
     \Au{Eco~U.}
     A theory of semiotics.~--- Bloomington: Indiana University Press, 1976.
     
          \bibitem{31zat} %28
     \Au{Шрейдер~Ю.\,А.}
     ЭВМ как средство представления знаний~// Природа, 1986. №\,10. С.~14--22.
  
   
     \bibitem{30zat} %29
     \Au{Шрейдер~Ю.\,А.}
     Информация и знание~// В кн.: Сис\-тем\-ная концепция информационных процессов.~--- 
М.: ВНИИСИ, 1988. С.~47--52.

     \bibitem{28zat} %30
     \Au{Гиляревский Р.\,С.}
     Основы информатики.~--- М.: Экзамен, 2003.
     
     \bibitem{29zat} %31
Информатика как наука об информации: Инфор\-мационный, документальный, 
технологический, экономический, социальный и организационный аспекты~/ Под ред. 
Р.\,С.~Гиляревского.~--- М.: ФАИР-ПРЕСС, 2006.
     
    
     \bibitem{32zat}
     \Au{Мельчук~И.\,А.}
     Русский язык в модели <<Смысл$\Leftrightarrow$Текст>>.~--- Москва--Вена: Школа 
<<Языки русской культуры>>, Венский славистический альманах, 1995.
     
     \bibitem{33zat}
     \Au{Лютый~А.\,А.}
     Язык карты: сущность, система, функция. 2-е изд., испр.~--- М.: ИГ РАН, 2002.
     
     \bibitem{34zat}
     \Au{Барт~Р.}
     Основы семиологии~// В кн.: Французская семиотика: От структурализма к 
постструктурализму.~--- М.: Прогресс, 2000. С.~247--310.
     
     \bibitem{35zat}
     \Au{Кибрик~А.\,Е.}
     Язык~// Большой энциклопедический словарь <<Языкознание>>.~--- М.: Большая 
российская энциклопедия, 1998. С.~604--606. 
     
     \bibitem{36zat}
     \Au{Turing~A.\,M.}
     On computable numbers, with an application to the Entscheidungsproblem. 
1936 ({\sf http://www. abelard.org/turpap2/tp2-ie.asp}).
     
     \bibitem{37zat}
     \Au{Fox~E.\,A.}
     Digital libraries of the future: Integration through the 5S Framework~// Труды VI 
Всероссийской научной конференции <<Электронные библиотеки: перспективные методы и 
технологии, электронные коллекции>>.~--- Пущино: Ин-т математических проблем 
биологии РАН, 2004. С.~1--2.
     
     \bibitem{38zat}
     \Au{Зацман~И.\,М.}
     Концептуальный поиск и качество информации.~--- М.: Наука, 2003.
     
     \bibitem{39zat}
     \Au{Шмелев~Д.\,Н.}
     Полисемия~// Большой энциклопедический словарь <<Языкознание>>.~--- М.: Большая 
российская энциклопедия, 1998. С.~382.

    
     \bibitem{40zat}
     Roget's international thesaurus.~--- New York: Thomas Y.~Crowell Co., 1954.
     
          \bibitem{43zat}
     \Au{Морковкин В.\,В.}
     Идеографические словари.~--- М.: Изд-во МГУ, 1970.

          \bibitem{41zat}
     Тезаурус научно-технических терминов~/ Под ред.\ Ю.\,И.~Шемакина.~--- М.: 
Воениздат, 1972.     
    
     \bibitem{42zat}
     \Au{Баранов О.\,С.}
     Идеографический словарь русского языка.~--- М.: ЭТС, 1995.
          
             
     \bibitem{44zat}
     \Au{Лукашевич~Н.\,В., Добров~Б.\,В.}
     Тезаурус для автоматического концептуального индексирования как особый вид 
лингвистического ресурса~// Компьютерная лингвистика и ее приложения. Тр.\ Междунар. 
семинара Диалог'2001. В 2-х т. Т.~2~/ Под ред. А.\,С.~Нариньяни.~--- М.: РосНИИ ИИ, 2001. 
С.~273--279.

     \bibitem{45zat} 
     \Au{Михайлов В.\,Н.}
     Гидрология устьев рек: Методическое пособие.~--- М.: Изд-во МГУ, 1996.
     
     \bibitem{46zat} 
     Vossen~P., ed. EuroWordNet General Document (Version~3) ({\sf 
http://www.illc.uva.nl/EuroWordNet/docs/\linebreak GeneralDoc}).

\label{end\stat}
     
     \bibitem{47zat} 
          \Au{Кожунова О.\,С.}
     Eurowordnet: задачи, структура и отношения~// Информатика и её применения, 2008. 
Т.~2. Вып.~4. С.~85--92.
     
\end{thebibliography}
}
}
\end{multicols} 
 