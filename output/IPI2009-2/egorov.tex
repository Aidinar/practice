\def\stat{egorov}

\def\tit{ВОПРОСЫ РЕАЛИЗАЦИИ ОБЪЕДИНЯЮЩЕЙ СРЕДЫ 
В~АРХИТЕКТУРЕ ДЕЦЕНТРАЛИЗОВАННОЙ ПАКЕТНОЙ~КОММУТАЦИИ}
\def\titkol{Вопросы реализации объединяющей среды 
в~архитектуре децентрализованной пакетной коммутации} 

\def\autkol{В.\,Б.~Егоров}
\def\aut{В.\,Б.~Егоров$^1$}

\titel{\tit}{\aut}{\autkol}{\titkol}

%{\renewcommand{\thefootnote}{\fnsymbol{footnote}}\footnotetext[1]
%{Работа выполнена при поддержке РФФИ, проекты 08--07--00152 и 08--01--00567.}}

\renewcommand{\thefootnote}{\arabic{footnote}}
\footnotetext[1]{Институт проблем
информатики Российской академии наук, vegorov@ipiran.ru}

\vspace*{6pt}

\Abst{Сформулированы требования к объединяющей среде архитектуры 
децентрализованной пакетной коммутации, рассмотрены способы ее реализации, 
особое внимание уделено использованию в качестве объединяющей среды 
высокоскоростных последовательных интерфейсов, предложен редуцированный 
вариант такого интерфейса, специализированный для пакетной коммутации.}

\vspace*{4pt}

\KW{пакетный коммутатор; децентрализованная коммутация; интегрированный 
коммуникационный микроконтроллер; сверхлокальная сеть; высокоскоростной 
последовательный интерфейс}

      \vskip 24pt plus 9pt minus 6pt

      \thispagestyle{headings}

      \begin{multicols}{2}

      \label{st\stat}

\section{Концепция децентрализованной пакетной~коммутации}
     
     Одним из эффективных и доступных отечественным разработчикам средств, 
облегчающих создание оригинальных устройств пакетной коммутации и маршрутизации, 
может стать широкое использование интегрированных коммуникаци\-он\-ных 
микроконтроллеров (ИКМ)~\cite{1eg}. Главное достоинство ИКМ заключается в удачном 
сочетании\linebreak аппаратных коммуникационных адаптеров и высокопроизводительного 
программируемого процессорного ядра, интегрированных на одном кристалле. 
Разнообразные коммуникационные адаптеры, универсальные и специализированные, 
предос\-тав\-ля\-ют пользователю аппаратную поддержку практически всех используемых в 
современной телекоммуникации интерфейсов и протоколов канального уровня. В то же 
время наличие свободно программируемого ядра дает возможность реализации\linebreak 
программными средствами на одном и том же приборе, по существу <<системе на 
кристалле>>, как телекоммуникационных протоколов более высоких уровней, так и других 
сопряженных задач, например маршрутизации и обеспечения сетевой безопас\-ности. Таким 
образом, ИКМ оказывается привлекательным универсальным средством для разработки 
разнообразных пакетных коммутаторов, маршрутизаторов, шлюзов и устройств сетевой 
защиты, целевая ориентация и функциональные возможности которых могут варьироваться 
и наращиваться соответствующим программным обеспечением~\cite{2eg}.
     
     К сожалению, у ИКМ как <<системы на кристалле>> имеется и обратная сторона. 
Однокристальным приборам органически присущи принципиальные ограничения по числу 
встроенных телекоммуникационных портов, суммарной про\-из\-во\-ди\-тель\-ности и пропускной 
способности. Эти ограничения накладываются числом выводов корпуса, полосой 
пропускания внутренних трактов данных и производительностью встроенных процессоров. 

На основе ИКМ, с учетом как их многочисленных достоинств, так и отмеченных 
недостатков, удобно строить интеллектуальные малопортовые устройства средней 
производительности,\linebreak
 например маршрутизаторы и устройства сетевой без\-опас\-ности. Такие 
устройства востребованы на обширном рынке средств подключения локальных\linebreak
 сетей~--- 
домашних, офисных и корпоративных~--- к сетям провайдеров. В~настоящее время ИКМ 
применя\-ются для целей пакетной коммутации и маршрутизации в основном в 
разнообразных ADSL (Asymmetric Digital Subscriber Line) 
модемах и недорогих малопортовых маршрутизаторах, часто со 
встроенными средствами сетевой защиты.
     
     Между тем свойственные ИКМ принципиальные ограничения по числу 
поддерживаемых коммуникационных портов и суммарной пропускной способности могут 
быть преодолены специальными архитектурными решениями, в частности в\linebreak архитектуре 
децентрализованной пакетной коммутации~\cite{3eg, 4eg}, что открывает ИКМ дорогу в 
область многопортовых высокоскоростных пакетных коммутаторов. 
%
Обобщенно концепция 
децентра\-ли\-зованной коммутации предполагает некоторое множество коммутирующих узлов 
и некую объединяющую их среду, функционирующую на прин\-ципах сверхлокальной сети 
(рис.~\ref{f1eg}). При этом\linebreak кон\-цеп\-ция допускает как гомогенную, так и гетерогенную 
организацию системы~\cite{4eg}. В~гомогенной системе множество функционально 
идентичных равноправных универсальных узлов,\linebreak
 обладающих достаточной 
производительностью, совокупно выполняют все функции по коммутации, маршрутизации и 
сетевой защите. При гетерогенной организации выполняемые системой функции 
распределяются между специализированными узлами. Например, $K$ узлов (узлы~$1\ldots 
K$ на рис.~\ref{f1eg}) поддерживают основные внешние коммуникационные порты, 
выполняют фильтрацию и коммутацию входящих в них пакетов (функции data plane). Из 
множества остальных узлов (узлы $K+1\ldots N$ на рис.~\ref{f1eg}) одни выполняют 
функции маршрутизации и управления трафиком (control plane), а другие~--- общего 
управления системой, тарификации и учета (management plane).
\begin{figure*} %fig1
\vspace*{1pt}
\begin{center}
\vspace*{1pt}
\mbox{%
\epsfxsize=88.426mm
\epsfbox{ego-1.eps}
}
\end{center}
\vspace*{-9pt}
\Caption{Концепция децентрализованной пакетной коммутации
     \label{f1eg}}
%     \vspace*{3pt}
     \end{figure*}


     Архитектура децентрализованной пакетной коммутации, ориентированная на ИКМ и в 
максимальной степени использующая их преимущества, предполагает, что узлы 
децентрализованного коммутатора реализуются на основе ИКМ. Гомогенная организация 
системы предполагает во всех узлах однотипные высокопроизводительные ИКМ с широким 
набором функциональных возможностей. При гетерогенной организации функционально 
различные узлы могут реализовываться на ИКМ разных типов. В частности, в 
маршрутизирующих узлах требуется, как правило, ИКМ с высокопроизводительным 
программируемым ядром и поддержкой больших объемов оперативной памяти, в то время 
как узлы сбора статистики и тарификации предполагают не столько 
высокопроизводительное ядро ИКМ, сколько достаточные объемы полупостоянной памяти. 
В любом случае использование ИКМ <<автоматически>> обеспечивает всем узлам 
децентрализованного пакетного коммутатора необходимые внешние порты: основные 
коммуникационные в коммутирующих узлах и служебные для целей настройки и 
управления~--- в маршрутизирующих и вспомогательных.

\vspace*{-6pt}

\section{Требования к объединяющей среде}

\vspace*{-3pt}
     
     Ключевым элементом архитектуры децентрализованной коммутации, в том числе 
ориентированной на использование ИКМ, является объединяющая среда. Хотя 
количественно ее параметры зависят от организации системы~--- гомогенной или 
гетерогенной~--- и характеристик узлов, на качественном уровне можно сформулировать 
общие требования:
     \begin{itemize}
\item функционирование на сетевых принципах и реализация в виде сверхлокальной сети;
\item достаточная суммарная пропускная способность;
\item неблокируемость отдельных трафиков внутри среды;
\item удобная практическая реализуемость.
\end{itemize}

     Функционирование объединяющей среды на сетевых принципах в широком плане 
предполагает:
     \begin{itemize}
\item обмен информацией между узлами системы в виде пакетов (кадров, ячеек);
\item сетевые методы адресации узлов;
\item наличие средств подтверждения целостности и достоверности передаваемых данных;
\item предоставление заданного уровня качества обслуживания, quality of service (QoS).
\end{itemize}

     Благодаря сетевой организации системы упрощается и унифицируется программное 
обеспечение узлов, поскольку все коммуникационные порты узлов, как внешние, так и 
внутренний к объединя\-ющей среде, функционируют на унифицированных принципах и 
обслуживаются одинаковым образом. В случае реализации узлов на ИКМ такая унификация 
дает дополнительный эффект, позволяя использовать для подключения узлов к 
объединяющей среде стандартные встроенные коммуникационные интерфейсы ИКМ. 
Важным требованием к объединяющей среде, легко реализуемым при сетевой организации, 
является предоставление заданного уровня QoS и возможность организации между узлами 
системы динамических виртуальных соединений с гарантированным временем доставки 
критических данных.
     
     Сверхлокальность объединяющей среды, организованной на сетевых принципах, 
может проявляться в нескольких аспектах:
     \begin{itemize}
\item физической и конструктивной ограниченности среды в пределах одного шасси, а в 
идеальном варианте~--- одной платы;
\item очень высоких скоростях передачи данных благодаря конструктивной компактности и 
высококачественным линиям связи;
\item повышенной достоверности передачи данных вследствие коротких расстояний, 
качественной среды передачи данных и возможности сведения к минимуму внешних помех;
\item постоянстве конфигурации системы и неизменности параметров сети.
\end{itemize}

     Требование достаточности пропускной способности объединяющей среды само по себе 
очевид\-но, но неочевидны критерии этой достаточности.\linebreak В~качестве грубой верхней оценки 
можно принять, что объединяющая среда должна быть способной одномоментно пропустить 
через себя сумму всех входных потоков на коммуникационных портах всех %\linebreak
 узлов системы. 
На самом деле в худшем случае пиковые нагрузки могут оказаться еще выше, поскольку 
помимо коммутируемых системой потоков данных в объединяющей среде циркулирует 
служебная информация, обеспечивающая взаимодействие узлов. С другой стороны, в общем 
случае некоторые входящие пакеты могут коммутироваться локально в пределах одного узла 
и не попадать в объединяющую среду, а некоторые вообще не подлежат коммутации. 
Таким образом, в пакетных коммутаторах с большим числом внешних коммуникационных 
портов возможно некое статистическое усреднение суммарного потока через объеди\-ня\-ющую 
среду на более низких уровнях. Оценка уровня минимальной достаточности пропускной 
способности для объединяющей среды представляет собой самостоятельную задачу, которая 
должна учитывать различные факторы:
     \begin{itemize}
\item гомогенную или гетерогенную организацию сис\-те\-мы;
\item скорости передачи данных и временн$\acute{\mbox{о}}$е распределение потоков на 
входных коммуникационных портах;
\item возможности буферирования данных в узлах;
\item жесткость требований QoS для тех или иных трафиков.
\end{itemize}

     Основная задача объединяющей среды~--- коммутация потоков данных и сообщений 
между узлами. С этой точки зрения к ней в принципе применимо требование 
не\-бло\-ки\-ру\-емости, характерное для классических коммутационных структур. Однако с 
практической точки зрения требовать от объединяющей среды стопроцентной 
не\-бло\-ки\-ру\-емости трафиков внутри нее вряд ли целесообразно. В современных коммутаторах 
с обязательными развитыми средствами обеспечения QoS и, как следствие, большими 
объемами буферов, в типичном случае секционированных по приоритетам, порядок 
прохождения пакетов через коммутационную структуру слабо коррелирован как с порядком 
их\linebreak
поступления в коммутатор, так и с порядком их\linebreak
 отправки из коммутатора. С этой точки 
зрения\linebreak более важным представляется обеспечение достаточной суммарной пропускной 
способности объеди\-ня\-ющей среды, а не достижение в ней не\-бло\-ки\-ру\-емости как таковой. 
Наглядно этот тезис подтверждается при шинной реализации объединяющей среды. Хотя 
любая параллельная шина, разделяемая множеством абонентов, являет собой яркий пример 
целиком и полностью блокирующей среды, различные шины с успехом применялись в 
коммутаторах первых поколений до тех пор, пока их пропускной способности хватало для 
суммарного коммутируемого системой трафика. И если сегодня классическая параллельная 
шина в качестве объединяющей среды архитектуры децентрализованной коммутации 
представляет лишь академический интерес, то причина этого отнюдь не в ее блокируемости, 
а в возросших скоростях передачи данных на внешних коммуникационных портах, при 
кото-\linebreak\vspace*{-12pt}

\pagebreak 

%\begin{figure*}[b] %fig2
%\vspace*{1pt}
\begin{center}
\vspace*{1pt}
\mbox{%
\epsfxsize=64.267mm
\epsfbox{ego-2.eps}
}
\end{center}
%\vspace*{6pt}
%\label{f1ush}}
%\end{figure*}
{{\figurename~2}\ \ \small{Блочная шина в качестве объединяющей среды}}

\bigskip
%\medskip
\addtocounter{figure}{1}



\noindent
рых параллельная шина не в состоянии обеспечить необходимые пропускные 
способности.
     
     Вряд ли требует особых комментариев требование удобной реализуемости 
объединяющей среды. Однако оно заставляет уделить внимание возможным вариантам ее 
практической реализации.

    
Первая и самая простая реализация объединяющей среды может представлять собой, 
как уже было замечено выше, некую параллельную шину. Правда, требование 
функционирования на сетевых принципах сразу исключает в качестве объединяющей среды 
типичные компьютерные шины~--- от ISA (Industry Standard Architecture)
до PCI-X (Peripheral Component Interconnect Extended). Тем не менее параллельная шина не 
только может функционировать на сетевых принципах, но и оказывается при этом проще 
компьютерных шин, выполняя функции объединяющей среды более эффективно. 
Рисунок~2 иллюстрирует использование в качестве концептуальной объединяющей 
среды архитектуры децентрализованной коммутации предложенной автором блочной 
шины~\cite{5eg}. Как и всякая другая, блочная шина не обеспечивает неблокируемость 
трафиков, однако она гарантирует своевременную доставку данных из узла в узел в пределах 
своей пропускной способности и с учетом QoS. К~сожалению, суммарная пропускная 
способность блочной шины не превышает 2--4~Гбит/с, поэтому она применима лишь в 
многопортовых децентрализованных коммутаторах со скоростями передачи данных на 
коммуникационных портах, не превышающих 100~Мбит/с, но практически непригодна для 
современных коммутаторов с гигабитными скоростями.

     
     Более производительную альтернативу шинам представляют собой коммутационные 
структуры. Одной из первых появилась технология коммутации Raceway, возникшая как 
расширение шины VME (VersaModule Eurocard)
и унаследовавшая ряд ее особенностей, в частности 32-разрядные 
параллельные порты коммутационных структур. Эта технология была реализована в 
6-портовым интегральном коммутаторе Raceway crossbar компании Cypress 
Semiconductor~\cite{6eg}. При максимальной частоте 40~МГц каждый 32-разрядный порт 
коммутатора Raceway crossbar обеспечивал пропускную способность 1,28~Гбит/с, а 
суммарная пропускная способность коммутатора с учетом неблокируемости коммутируемых 
трафиков составляла 7,68~Гбит/с. Очевидно, что один такой прибор в качестве 
объединяющей среды архитектуры децентрализованной коммутации мог обслуживать 
максимум шесть узлов (рис.~3) и обеспечить пропускную способность, лишь 
ненамного превышающую, например, возможности блочной шины. Кроме того, в отличие от 
блочной шины, технология Raceway, унаследовавшая логическую организацию 
компьютерной шины VME, базировалась отнюдь не на сетевых принципах и, следовательно, 
не отвечала основному требованию к объеди\-ня\-ющей среде архитектуры децентрализованной 
коммутации. Зато, в отличие от параллельной шины, она позволяла строить сложные 
коммутационные структуры с использованием множества интегральных коммутаторов, что в 
принципе открывало возможность неограниченного наращивания числа подключаемых 
узлов и суммарной пропускной способности объединяющей среды. В частности, 
интегральные коммутаторы Raceway crossbar позволяли строить коммутирующие структуры 
с самомаршрутизирующими и неблокирующими топологиями, в частности баньян-сети, сети 
Клоза, Бенеша~\cite{7eg} и~другие. К~сожалению, спроектированные на таких
интегральных 
коммутаторах топологически сложные коммутационные структуры, такие как сети\linebreak

%\begin{figure*} %fig3
%\vspace*{1pt}
\begin{center}
%\vspace*{1pt}
\mbox{%
\epsfxsize=63.994mm
\epsfbox{ego-3.eps}
}
\end{center}
{{\figurename~3}\ \ \small{Параллельный коммутатор Raceway crossbar в качестве объединяющей среды}}

\bigskip
\addtocounter{figure}{1}
%\vspace*{-9pt}
%\Caption{Параллельный коммутатор Raceway crossbar в качестве объединяющей среды
%\label{f3eg}}

\begin{table*}[b]\small
\vspace*{-12pt}
\begin{center}
\Caption{Сравнение различных средств реализации объединяющей среды
\label{t1eg}}
\vspace*{2ex}

\begin{tabular}{|l|c|c|c|}
\hline
\multicolumn{1}{|c|}{Характеристика}&
\tabcolsep=0pt\begin{tabular}{c}Параллельная\\ блочная\\  шина\end{tabular}&
\tabcolsep=0pt\begin{tabular}{c}Коммутатор\\ с параллельными\\ портами\end{tabular} &
\tabcolsep=0pt\begin{tabular}{c}Коммутатор\\ с последовательными\\ портами\end{tabular} \\
\hline
Число объединяемых узлов&$8\ldots 12$&$4\ldots 6$&$8\ldots 64$\\
Ширина интерфейса к узлу&до 64 разрядов&до 32 разрядов&$1\ldots 4$ разряда\\
Скорость обмена данными с узлом&до 4 Гбит/с&до 2 Гбит/с&до 20 Гбит/с (дуплекс)\\
Общая пропускная способность&до 4 Гбит/с&до 12 Гбит/с&до 1280 Гбит/с\\
Масштабируемость&плохая&средняя&хорошая\\
Стоимость подключения узла&средняя&высокая&низкая\\
\hline
\end{tabular}
\end{center}
\end{table*}


\noindent
Клоза 
или Бенеша, с большим числом коммутаторов и, следовательно, еще 
б$\acute{\mbox{о}}$льшим числом параллельных шин, не только не отвечают требованию 
удобства реализуемости объединяющей среды, но и вообще вряд ли реализуемы на 
практике.


\vspace*{-6pt}
\section{Высокоскоростные последовательные интерфейсы~в~качестве объединяющей~среды}

\vspace*{-3pt}
     
     Существенное упрощение реализации топологически сложных коммутационных 
структур стало возможным с появлением локальных высокоскоростных последовательных 
интерфейсов, в которых переход к последовательным методам передачи данных не только 
резко сократил число сигнальных линий интерфейса, но и заставил сблизить принципы их 
функционирования с традиционно сетевыми.
     
     Современные методы последовательной передачи данных позволяют при относительно 
небольших длинах линий связи достигать скоростей в\linebreak
несколько гигабит в секунду на 
типовой низковольтной дифференциальной паре проводников. Переход к последовательной 
передаче данных с резким сокращением числа сигнальных линий коммуникационного порта 
предоставляет возможность существенного увеличения числа портов в одном\linebreak отдельно 
взятом интегральном коммутаторе, а общее сокращение сигнальных линий упрощает\linebreak 
реализацию топологически сложных коммутационных структур из множества интегральных 
коммутаторов. Все это в совокупности открывает\linebreak практически неограниченные перспективы 
наращивания числа объединяемых абонентов и пропускной способности коммутационных 
структур, в частности используемых для создания объеди\-ня\-ющей среды в архитектуре 
децентрализованной коммутации. В~табл.~\ref{t1eg} приведено сравнение различных 
средств реализации объеди\-ня\-ющей среды: блочной шины и интегральных коммутаторов с\linebreak 
параллельными и последовательными высокоскоростными интерфейсами портов. Сравнение 
демонстрирует очевидные преимущества последовательных высокоскоростных интерфейсов 
при реализации сложных коммутационных структур.\linebreak Однако для использования их в 
объединяющих средах архитектуры децентрализованной коммутации этого недостаточно. 
Как было подчеркнуто выше, необходимо, чтобы такие интерфейсы функционировали в 
объеди\-ня\-ющей среде на сетевых прин\-ципах.


     Первые высокоскоростные последовательные интерфейсы появились как замена 
компьютерных параллельных шин, пропускная способность которых перестала отвечать 
возросшим скоростям передачи данных внутри компьютеров, и поэтому в\linebreak
целом сохранили 
логическую организацию компьютерных шин. Так, появившиеся одними из первых 
интерфейсы VXS (VMEbus Switched Serial) и StarFabric унаследовали общую организацию, 
адресацию абонентов и принципы взаимодействия между ними от своих предшественников, 
шин VME и PCI соответственно. Тем не менее, уже в технологии StarFabric появились такие 
чисто сетевые понятия, как пакетная передача данных, разделение сервисов по уровням 
(физический, звеньевой и транспортный), классы трафика, избыточное кодирование 
и контроль данных циклическими кодами. Дальнейшее развитие с улучшением всех 
количественных характеристик этот подход получил в интерфейсах InfiniBand и PCI Express 
(PCIe). 

Однако все эти интерфейсы по-прежнему оставались компьютерными в главном: 
изначально предназначенные для использования в качестве сис\-тем\-но\-го интерфейса 
компьютера, они были ориентированы на древовидную топологию с единственным 
центральным процессором (<<корень>> древа), централизованно управляющим самим 
интерфейсом и под\-клю\-ча\-емы\-ми к нему периферийными устройствами (<<стволом>> и 
<<кроной>>).
     
     Принципиально новое качество в этом плане привнес в новом тысячелетии 
высокоскоростной интерфейс ввода-вывода RapidIO. Хотя он специфицирован в двух 
вариантах, параллельном и\linebreak
 последовательном, широкое практическое применение нашла 
именно последовательная его реализация~--- serial RapidIO (SRIO). Хотя интерфейс SRIO 
все еще несет в себе некоторые <<атавизмы>> компьютерных интерфейсов, в частности 
возможность адресного обращения к памяти по чтению или записи, в нем появились новые 
чисто сетевые черты, которые, в частности, позволяют эффективно использовать его в 
качестве объединяющей среды. Главные среди них:
     \begin{itemize}
\item возможность объединения равноправных (peer-to-peer) абонентов;
\item сетевая адресация конечных абонентов;
\item сервис инкапсуляции сквозных трафиков со сторонними протоколами;
\item развитые механизмы обмена сообщениями между абонентами;
\item наличие средств поддержки QoS;
\item сегментация и сборка пакетов, segmentation and reassembly (SAR).
\end{itemize}

     Возможность объединения равноправных абонентов, средства инкапсуляции сквозных 
трафиков и обмена сообщениями в совокупности с поддержкой QoS позволяют 
естественным образом строить с помощью интерфейса SRIO сверхлокальные сети и 
использовать их в качестве объединяющей среды архитектуры децентрализованной 
коммутации, как это было показано на рис.~\ref{f1eg}. Механизм SAR дает возможность 
объединяющей среде работать с короткими, длиной не более 256~байт, пакетами, что 
облегчает достижение требуемого качества обслуживания. Кроме того, сегментация пакетов 
упрощает реализацию интегральных коммутаторов и, как следствие, объединяющей среды в 
целом.
     
     Достоинством SRIO является использование на физическом уровне стандарта XAUI 
(X-Attachment Unit Interface), давно применяемого в таком широко распространенном 
интерфейсе, как 10~Gigabit Ethernet, что способствует упрощению и уде\-шев\-ле\-нию 
реализации абонентских адаптеров SRIO. Но это достоинство может обернуться 
недостатком. Предельная полезная скорость передачи данных на одной дифференциальной 
паре XAUI равна 2,5~Гбит/с, а на максимально широком звене из четырех пар составляет 
10~Гбит/с. Для сравнения, спецификация PCIe-II допускает вдвое большие скорости на 
дифференциальной паре и ширину звена до~32~пар. Впрочем, скорости передачи данных на 
интерфейсе SRIO также будут расти в дальнейшем, в частности уже имеется стандарт CEI 
(Common Electrical Interface) со скоростями 6 и 11~Гбит/с на одной паре при сигнализации, 
совместимой с XAUI.
     
     В 2007~г.\ SRIO обрел опасного конкурента~--- интерфейс ASI (Advanced Switching 
Interconnect), который, по существу, представляет собой надстройку на стандартном 
интерфейсе PCIe. Эта надстройка призвана компенсировать как раз те недостатки PCIe, 
которые препятствуют его использованию в сверхлокальных сетях и тем самым в качестве 
объединяющей среды. Используя физический уровень и организацию звена PCIe, ASI 
добавляет такие важные сетевые качества, как:
     \begin{itemize}
\item равноправие подключаемых абонентов;
\item возможность обмена сообщениями между абонентами;
\item туннелирование трафиков со сторонними протоколами;
\item многоуровневая поддержка QoS.
\end{itemize}

     При этом ASI воспринимает все возможности и преимущества интерфейса PCIe, в 
частности хорошо отработанную технологию физического уровня и высокие скорости 
передачи данных на звене. Сохраняя совместимость <<сверху вниз>>, ASI может 
взаимодействовать со стандартными сегментами PCIe через прозрачные мосты. В результате 
технология ASI, совмещая <<два в одном>>, смотрится очень привлекательно в качестве 
объединяющей среды архитектуры децентрализованной коммутации. Однако оборотная 
сторона  такой всеобъемлющей универсальности~--- большая избыточность в\linebreak
каждом 
конкретном применении и, главное, сложность реализации с неизбежно высокой ценой 
результата. На сегодняшний день имелись объявления о планах выпуска коммутаторов ASI 
от\linebreak
компаний StarGen\footnote{StarGen прекратила свое существование, ее преемницей стала 
компания Dolphinics, в планах которой коммутаторы ASI вообще не числятся.} и Xyratex, 
однако на рынке подобных продуктов так и не появилось. Поддержку технологии ASI 
рекламирует компания Xilinx, но реально эта поддержка сводится к введению в серию 
программируемых микросхем этой компании Virtex-II, трансиверов PCIe и процессорного 
ядра IBM PowerPC 405 для программной реализации всех <<надстроек>> ASI на 
стандартном интерфейсе PCIe. Таким образом, перспективы технологии ASI пока довольно 
туманны. И даже если в будущем интегральные коммутаторы ASI все же появятся на рынке, 
вряд ли они будут конкурентоспособны по ценам с коммутаторами SRIO.
     
     Но и коммутаторы SRIO отнюдь не дешевы: цена интегрального коммутатора SRIO 
сопоставима, например, с ценой ИКМ высокой производительности. Конечно, если узлы 
децентрализованного пакетного коммутатора строятся на традиционных ИКМ, то 
применение в объединяющей среде интерфейса существующих интегральных коммутаторов 
SRIO выглядит вполне технически сбалансированным и экономически оправданным. Но 
если основой узлов коммутатора будут специальные ориентированные на пакетную 
коммутацию и потому упрощенные и более дешевые ИКМ~\cite{8eg}, то логично дополнить 
упрощенные ИКМ в узлах более дешевыми интегральными коммутаторами в объединяющей 
среде. Удешевления интегральных коммутаторов можно было бы достичь за счет некоего 
более простого последовательного интерфейса,  
специализированного, подобно упрощенным ИКМ, под пакетную коммутацию.

\vspace*{-6pt}

\section{Упрощение последовательного интерфейса для~пакетной 
коммутации}

\vspace*{-4pt}
     
     Интерфейс SRIO специфицирован на трех уровнях: логическом, транспортном и 
звеньевом\footnote{Приблизительные параллели в семиуровневой эталонной модели 
взаимодействия открытых систем ISO: транспортный, сетевой и канальный уровни 
соответственно.}. Каж\-дый из этих уровней обладает избыточностью при использовании 
интерфейса SRIO в качестве объединяющей среды архитектуры децентрализованной 
коммутации.
     
     Логический уровень SRIO предлагает пользователю три сервиса:\\[-14pt]
     \begin{enumerate}[(1)]
     \item адресный доступ к памяти и вводу-выводу (input/output);\\[-14pt]
     \item  передачу сообщений (message passing);\\[-14pt]
\item  инкапсуляцию сквозных потоков данных (data streaming).
\end{enumerate}

     Исходя из требований к объединяющей среде, адресный доступ вообще следует 
признать ненужным и даже вредным сервисом. Он явно противоречит сетевым принципам 
взаимодействия узлов, поскольку не только заставляет программное обеспечение узлов 
децентрализованного коммутатора <<опускаться>> до уровня чтения/записи ячеек памяти и 
адресуемых регистров, но и предполагает неконтролируемое вторжение в память других 
узлов. Два других сервиса~--- передача сообщений и инкапсуляция сквозных потоков 
данных~--- релевантны сверхлокальной сети и вполне отвечают требованиям к 
объединяющей среде. Оба эти сер-\linebreak\vspace*{-12pt}
\columnbreak 


\noindent
виса включают SAR, но, к сожалению, в силу непонятных 
причин реализуются на интерфейсе SRIO различно. В частности, при передаче 
сегментированного сообщения все его сегменты нумеруются, что дает возможность 
передавать и транспортировать их в произвольном порядке, а в конечном 
     абоненте-приемнике восстанавливать сообщение независимо от порядка поступления 
сегментов. Однако при инкапсуляции сквозных потоков данных такая возможность не 
предусмотрена, и сегменты инкапсулированных блоков данных должны транспортироваться 
и доставляться получателю в естественном порядке. 

С точки зрения объединяющей среды 
усложнения в конечном абоненте, связанные с произвольным порядком передачи сегментов 
относительно коротких сообщений, при жестком порядке транспорта сегментов 
относительно длинных инкапсулированных блоков данных выглядят совершенно 
неоправданными. Логично было бы пожертвовать свободным порядком транспорта 
сегментов сообщений, которые зачастую и состоят-то из одного-единственного сегмента, во 
имя унификации с механизмом транспорта инкапсулированных блоков данных и тем самым 
существенного упрощения всего механизма SAR у конечного абонента.
     
     На транспортном уровне интерфейс SRIO предо\-став\-ля\-ет выбор 8- или 16-разрядных 
адресов\linebreak
конечных абонентов, но адресация должна быть единой в пределах всей сети, 
объединяемой интерфейсом SRIO. Возможно, в каких-то приложениях, например связанных 
с массовой цифровой обработкой сигналов, где конечным абонентом интерфейса может 
быть одна-единственная микросхема
 аналого-цифрового (АЦП) или цифро-аналогового\linebreak преобразователя (ЦАП), 
такие абоненты могут исчисляться 
сотнями. Но ИКМ, хотя и позиционируемый как <<система на кристалле>>, будучи\linebreak  основой 
узла децентрализованного пакетного коммутатора, <<обрастает>> внешними памятями, 
транси\-ве\-ра\-ми и другим вспомогательным оборудованием, вследствие чего узел представляет 
собой отнюдь не одинокую мик\-ро\-схе\-му, а, скорее, плату или модуль. Число таких модулей в 
границах физической реализуемости высокоскоростных интерфейсов класса SRIO вряд ли 
перевалит за несколько десятков. Соответственно, для целей децентрализованной пакетной 
коммутации можно было бы ограничиться 8-разрядным адресом конечного абонента.

\begin{figure*} %fig4
\vspace*{1pt}
\begin{center}
\vspace*{1pt}
\mbox{%
\epsfxsize=130.163mm
\epsfbox{ego-4.eps}
}
\end{center}
\vspace*{-9pt}
\Caption{Форматы пакетов для стандартного SRIO и примерного редуцированного интерфейсов
\label{f4eg}}
\end{figure*}
     
     
     На звеньевом уровне интерфейса SRIO каждый сегмент сообщения или 
инкапсулированного блока данных оформляется отдельным пакетом и сопровождается своей 
контрольной суммой. По ней все звеньевые контроллеры, как в конечных абонентах, так и в 
элементах коммутационных структур, контролируют и подтверждают друг другу 
правильность передачи на звене отдельного сегмента и самостоятельно выполняют повтор 
передачи в случае ошибки. Конечно, в целом такой механизм повышает достоверность 
передачи сообщения или инкапсулированного блока данных, но вряд ли сопровождающие 
его издержки оправданы в объединяющей среде децентрализованного пакетного 
коммутатора. Сверхлокальность объединяющей среды предполагает, в частности, короткие и 
высококачественные линии связи на звене с высокой достоверностью передачи данных. С 
другой стороны, сетевая организация объединяющей среды гарантирует надежный контроль 
целостности и достоверности доставленных сообщений и инкапсулированных блоков 
данных у конечных абонентов. Поэтому отказ от контроля и повтора передач сегментов на 
звене позволил бы без ощутимых потерь не только упростить контроллер звена интерфейса, 
но и, главное, ускорить передачу данных за счет исключения контрольных сумм сегментов и 
обмена подтверждениями на уровне звена.
     
     Таким образом, высокоскоростной последовательный интерфейс, упрощенный для 
целей пакетной коммутации, который в дальнейшем будем называть 
\textbf{редуцированным интерфейсом}, мог бы отличаться от стандартного интерфейса 
SRIO:
     \begin{itemize}
\item  исключением сервиса адресного доступа;
\item  унификацией сервисов сообщений и инкапсуляции;
\item  ограничением сетевых адресов конечных абонентов одним байтом;
\item  отказом от контроля передач и подтверждений на уровне звена и, соответственно, 
исключением контрольных сумм сегментов сообщений и инкапсулированных блоков 
данных.
\end{itemize}
     
     Совокупность предложенных упрощений должна обеспечить редуцированному 
интерфейсу по сравнению со стандартным SRIO б$\acute{\mbox{о}}$льшую эффективность 
за счет повышения пропускной способности и упрощения интерфейсных адаптеров.
     
     Выигрыш в пропускной способности редуцированного интерфейса за счет укорочения 
     заголовков пакетов и отказа от контрольных сумм иллюстрирует рис.~\ref{f4eg}, где 
показаны форматы пакетов с сегментами сообщений и инкапсулированных блоков данных 
для стандартного интерфейса SRIO и примерного редуцированного интерфейса, в котором 
упрощения коснулись заголовков всех трех уровней, но не затронули дисциплину SAR.

\begin{figure*} %fig5
\vspace*{1pt}
\begin{center}
\vspace*{1pt}
\mbox{%
\epsfxsize=102.381mm
\epsfbox{ego-5.eps}
}
\end{center}
\vspace*{-9pt}
\Caption{Обобщенная структура интегрального коммутатора SRIO
\label{f5eg}}
\vspace*{3pt}
\end{figure*}

     В примерном редуцированном интерфейсе уровень звена настолько упрощается, что 
звеньевой заголовок оказывается вовсе ненужным. Из двух основных полей звеньевого 
заголовка стандартного SRIO одно~--- идентификатор подтверждения~--- исчезает 
вследствие отказа от подтверждений на\linebreak
уровне звена, а второе~--- приоритет пакета~--- 
пе\-ремещается в транспортный заголовок. В транспортном заголовке, поскольку узлы 
децентрализованного коммутатора могут иметь по нескольку\linebreak коммуникационных 
интерфейсов, для каждой пары узлов предусматривается возможность организации между 
ними нескольких виртуальных каналов, например, для различных сочетаний их входных и 
выходных интерфейсов. С этой целью в транспортный заголовок редуцированного 
интерфейса вводится дополнительное поле номера виртуального канала, которое 
параллельно выполняет функции поля класса обслуживания в логическом заголовке пакета 
стандартного SRIO и в этом качестве\linebreak попутно задает приоритет пакета. Заголовок 
логического уровня редуцированного интерфейса ограничивается двумя тривиальными 
полями: тип пакета (сообщение/инкапсулированные данные,\linebreak запрос/подтверждение и~т.\,п.) 
и фактическая длина сегмента данных.
     
     Таким образом, пакет редуцированного интерфейса может стать на 48~разрядов, т.\,е.\ 
14,3\% исходной длины, короче пакета стандартного SRIO при одной и той же полезной  
нагрузке. Правда, на редуцированном интерфейсе ошибка передачи данных в любом 
сегменте потребует повтора передачи всего сообщения или инкапсулируемого блока 
данных, поэтому упрощение интерфейса оправдано только в сетях с высокой 
достоверностью передачи данных, что как раз характерно для сверхлокальных сетей 
объединяющих сред.
     
     Интерфейсные адаптеры редуцированного интерфейса упростятся как у конечных 
абонентов, так и в элементах коммутационных структур.
     
     У конечных абонентов интерфейсный адаптер редуцированного интерфейса получится 
более компактным и дешевым за счет сокращения числа и унификации поддерживаемых 
сервисов, а также вследствие упрощения функций, поддерживаемых контроллером звена 
интерфейса. Это особенно важно для упрощенных специализированных под пакетную 
коммутацию ИКМ~\cite{8eg}, в которых интерфейсный адаптер объединяющей среды 
может составлять заметную долю аппаратуры.
     
     Еще б$\acute{\mbox{о}}$льшим окажется влияние вносимых в интерфейс упрощений 
на интегральные коммутаторы. Коммутаторы SRIO в общем случае структурно включают 
адаптеры порта~--- по адаптеру на каждый внешний порт коммутатора~--- и некий блок 
коммутации для обмена пакетами между портами~\cite{9eg}. Адаптер порта, в свою очередь, 
состоит из звеньевого и транспортного контроллеров, входных и выходных буферов 
коммутируемых пакетов и некоего хранилища маршрутной информации со средствами 
доступа к ней (рис.~\ref{f5eg}).


     Редукция интерфейса упрощает адаптер порта интегрального коммутатора сразу по 
нескольким направлениям. Во-первых, ограничение адресов конечных абонентов восемью 
разрядами позволяет контроллеру однозначно и быстро маршрутизировать все входящие 
пакеты с помощью простой полноразмерной маршрутной таблицы объемом всего на 
256~входов без вовлечения сложных алгоритмов и механизмов преобразования сетевых 
адресов (хеширование адресов, многоуровневые маршрутные таблицы и~т.\,п.). Маршрутная 
информация может устанавливаться статически вследствие относительной неизменности 
конфигурации сверхлокальной сети и задаваться централизованно, но возможно и 
динамическое заполнение маршрутных таблиц с использованием обычным методов 
<<самообучения>>. Во-вторых, отказ от подтверждений на уровне звена заметно упрощает 
аппаратуру контроллера звена. В-третьих, исключение повторов передачи на звене устраняет 
принципиальную необходимость в выходных буферах порта для хранения уже переданных 
пакетов на случай возникновения необходимости повторной передачи. В результате при 
прочих равных условиях интегральные коммутаторы редуцированного интерфейса будут 
проще и дешевле коммутаторов SRIO, не говоря уже о боле дорогих технологиях вроде ASI.

\vspace*{-4pt}

\section{Заключение}

\vspace*{-2pt}
     
Интегрированные коммуникационные микроконтроллеры зарекомендовали себя хорошим <<строительным материалом>> для недорогого 
интеллектуального коммуникационного оборудования среднего класса, т.\,е.\ 
оборудования пусть не передовых рубежей телекоммуникации, но зато активно 
востребованного и массового выпускаемого. На основе ИКМ строятся различные 
     ADSL-мо\-де\-мы, маршрутизаторы для офисов и домашнего\linebreak применения, небольшие 
цифровые АТС, шлюзы из сетей ISDN (Integrated Services Digital Network)
в пакетные сети и другое оборудование для схожих 
приложений.
     
     К сожалению, ИКМ пока еще недостаточно используются в отечественных 
разработках, хотя потенциально весьма привлекательны многими своими достоинствами, в 
том числе встроенными\linebreak программируемыми процессорными ядрами и аппаратными 
адаптерами с поддержкой телекоммуникационных интерфейсов и протоколов. Эти качества 
ИКМ особенно важны для отечественных разработчиков, традиционно вынужденных решать 
программными средствами задачи, которые за рубежом решаются аппаратно благодаря 
доступности новейших достижений технологии сверхбольших интегральных схем.
     
     Архитектура децентрализованной коммутации дает возможность еще больше 
расширить сферу потенциальной применимости ИКМ, включив в нее высокоскоростные 
многопортовые пакетные коммутаторы. Децентрализованные пакетные коммутаторы могут 
строиться на существующих ИКМ, при этом в качестве объединяющей среды архитектуры 
децентрализованной коммутации можно использовать современные высокоскоростные 
последовательные интерфейсы. Особенно привлекательно в этом плане выглядит интерфейс 
SRIO, адаптеры которого интегрируются в некоторые современные ИКМ.
     
     Однако ИКМ, подходящие для реализации узлов децентрализованного коммутатора, т.\,е.\
      с производительными процессорными ядрами, высокоскоростными 
коммуникационными портами и\linebreak
 поддержкой интерфейса SRIO, недешевы. Соизмеримы с 
ними по цене и реализующие объединяющую среду интегральные коммутаторы SRIO. 
Поэтому в целом децентрализованному пакетному коммутатору, реализуемому на основе 
существующих ИКМ и интерфейсе SRIO, будет трудно конкурировать с массово 
выпускаемыми аппаратными коммутаторами, несмотря на потенциально более широкие 
функциональные возможности и гибкость. Более эффективно и, следовательно, 
конкурентоспособно задача построения децентрализованных высокоскоростных 
многопортовых пакетных коммутаторов могла бы решаться путем упрощения ИКМ и 
высокоскоростного последовательного интерфейса специализацией их для задач пакетной 
коммутации и, в частности, ориентацией на архитектуру децентрализованной пакетной 
коммутации. В этом плане предложенный в статье редуцированный интерфейс естественно 
дополняет ранее выдвинутую автором концепцию создания отечественных упрощенных 
ИКМ для пакетной коммутации~\cite{8eg}.

\vspace*{-3pt}

{\small\frenchspacing
{%\baselineskip=10.8pt
\addcontentsline{toc}{section}{Литература}
\begin{thebibliography}{9}    
\bibitem{1eg}
\Au{Егоров В.\,Б.}
Принципы создания коммутационной аппаратуры на основе специализированных 
микроконтроллеров~// Системы и средства автоматики.~--- М.: Наука, 1999. Вып.~9. С.~44--55.

\bibitem{2eg}
\Au{Егоров В.\,Б.}
Интегрированные коммуникационные микроконтроллеры Freescale Semiconductor: из прош\-ло\-го в 
будущее~// Электронные компоненты, 2008. №\,7. С.~31--40.

\bibitem{3eg}
\Au{Соколов И.\,А., Егоров~В.\,Б.}
Дезинтеграционный подход к архитектуре универсального процессора коммутации пакетов~// 
Информационные технологии и вычислительные системы, 2005. №\,2. С.~76--85.

\bibitem{4eg}
\Au{Соколов И.\,А., Егоров~В.\,Б.}
Дезинтегрированная архитектура пакетной коммутации~// Информатика и её применения, 2008. Т.~2. 
Вып.~4. С.~2--11.

\bibitem{5eg}
\Au{Егоров В.\,Б., Полухин~А.\,Н.}
Принципы создания сис\-тем\-ной шины многопортовых пакетных коммутаторов~// Системы и средства 
информатики.~--- М.: Наука, Физматлит, 2000. Вып.~10. С.~80--90.

\bibitem{6eg}
\Au{Robinson~J.}
RACEway interlink adoption and growth~// VITA J., September 1997. P.~12--16. {\sf 
http://www.vita.com/ sept97vj/rway.pdf}.

\bibitem{7eg}
\Au{Богданов А., Станкова~Е., Корхов~В., Мареев~В.}
Ар\-хи\-тек\-ту\-ры и топологии многопроцессорных вы\-чис\-ли\-тель\-ных систем.~--- М.: ИНТУИТ.ру, 
Открытые системы, 2004. Гл.~5. {\sf http://www.informika.ru/text/ teach/topolog/5.htm}.

\bibitem{8eg}
\Au{Егоров В.\,Б.}
Концепция создания отечественных интегрированных коммуникационных микроконтроллеров для 
пакетной коммутации // Информатика и её применения, 2009. Т.~3. Вып.~1. С.~34--46.

\label{end\stat}

\bibitem{9eg}
\Au{Егоров~В.\,Б.}
Последовательный интерфейс RapidIO и его применение в пакетной коммутации~// Электронные 
компоненты, 2008. №\,12. С.~69--76.

\end{thebibliography}
}
}
\end{multicols}
  
 
 
 