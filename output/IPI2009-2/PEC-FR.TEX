

\def\t{\tau}
\def\v{\vec}
\def\o{\overline}
\def\n{\nu}

\def\stat{pech-fr}

\def\tit{ВЕРОЯТНОСТНЫЙ АНАЛИЗ ВРЕМЕНИ ПРОЯВЛЕНИЯ НЕИСПРАВНОСТИ В~СЕТИ АВТОМАТОВ}
\def\titkol{Вероятностный анализ времени проявления неисправности в сети автоматов}

\def\autkol{А.\,В.~Печинкин, С.\,Л.~Френкель}
\def\aut{А.\,В.~Печинкин$^1$, С.\,Л.~Френкель$^2$}

\titel{\tit}{\aut}{\autkol}{\titkol}

%{\renewcommand{\thefootnote}{\fnsymbol{footnote}}\footnotetext[1]
%{Работа выполнена при поддержке РФФИ, проекты 08--07--00152 и 08--01--00567.}}

\renewcommand{\thefootnote}{\arabic{footnote}}
\footnotetext[1]{Институт проблем
информатики Российской академии наук, apechinkin@ipiran.ru}
\footnotetext[2]{Институт проблем
информатики Российской академии наук, fsergei@mail.ru}



\Abst{Предложен способ распространения математической модели
оценки распределения вероятностей времени обнаружения неисправности
в некоторой цифровой системе, описанной как конечный автомат, на
сеть подавтоматов, в которую декомпозирован данный исходный автомат.
Под <<латентностью>> понимается время (число тактов работы автомата),
которое проходит с момента проявления неисправности в той или иной
переменной, опи\-сы\-ва\-ющей модель автомата, до проявления ее в выходной
последовательности автомата.
Латентность вычисляется как число шагов до попадания цепи Маркова (ЦМ)
с множеством состояний, представляющим собой произведение множеств
состояний исправного и неисправного автоматов, в поглощающее состояние
(на случайных входных наборах автомата).
Предлагаемый подход к оценке распределения вероятностей
латентности основан на расширении пространства состояний описываемой
системы, что позволяет строить ЦМ для пар подавтоматов
(в общем зависимых) исправного и неисправного автоматов и выразить
это распределение в терминах вероятностных характеристик подавтоматов.}

\KW{тестирование цифровых схем; конечные автоматы; цепи Маркова}

 \vskip 24pt plus 9pt minus 6pt

 \thispagestyle{headings}

 \begin{multicols}{2}

 \label{st\stat}

\section{Введение}

 При проектировании современных высоконадежных цифровых систем часто
применяются различные <<толерантные к ошибкам>> (fault-tolerant) решения,
например самопроверяемость (self-checking), самовосстановление
(self-recovery), самокоррекция (self-reconfiguration)~\cite{1}.
Для таких систем исключительно важным является вопрос о времени обнаружения
ошибки функционирования.
Под <<обнаружением>> неисправности понимают
несовпадение значений выходов исправной системы и системы с неисправностью,
что означает, например, некорректное функционирование по сравнению с
заданной спецификацией проектируемой системы.
В общем случае под <<выходом>> понимают некоторые переменные, с которыми
связывают же-\linebreak
\begin{center} %fig1
\vspace*{12pt}
\mbox{%
\epsfxsize=80mm %.391mm
\epsfbox{pech-1.eps}
}
\end{center}
%\vspace*{6pt}
%\label{f1ush}}
%\end{figure*}
{{\figurename~1}\ \ \small{Явление латентности обнаружения неисправ-\linebreak ности, появляющейся при
выполнении некоторой операции в момент $\t$ и проявляющейся на выходе в
момент~$\t + k$}}

%\bigskip
\medskip
%\medskip
\addtocounter{figure}{1}



\noindent
лаемый результат.
Время от момента появления неисправности, приводящей к тому, что при
определенном возможном значении входных переменных происходит несовпадение
значений отдельных переменных (например, переменных, опи\-сы\-ва\-ющих
внутренние состояния) исправной сис\-те\-мы и сис\-те\-мы
с неисправностью, до момента ее проявления в хотя бы одной из выходных
переменных (момента обнаружения неисправности) называют <<{\it латентностью
обнаружения} соответствующей {\it не\-ис\-прав\-ности}>>, или FDL
(Fault Detection Latency, см.\ рис.~1, на котором $\mathrm{FDL} = k$).
Как далее будет показано
на примерах, несовпадение внутренних
состояний исправной сис\-те\-мы и сис\-те\-мы с неисправностью еще не обязательно
ведет к несовпадению выходных переменных.


 Будем рассматривать цифровые системы, представленные моделью конечного
автомата (FSM~--- Finite State Machine), в частности автоматом Мили (Mealy)
с множеством состояний ${\cal A}\;=$\linebreak $=\;\{a_1,\ldots,a_r\}$, входов
${\cal X} = \{x_1,\ldots,x_n\}$ и выходов ${\cal Y} = \{y_1,\ldots,y_m\}$,
где $x_i$,\ $i=\o{1,n}$, и $y_j$,\ $j=\o{1,m}$,~--- булевы переменные.
В этой модели внутреннее состояние $a(t+1)\in \cal A$ в момент $t+1$ и
вектор $\v y(t) = (y_1(t),\ldots,y_m(t))$ выходных переменных в момент $t$
определяются через внутреннее состояние $a(t)\in \cal A$ и вектор
$\v x(t) = (x_1(t),\ldots,x_n(t))$ входных переменных в момент $t$
функциями $\delta$ (переходов состояний) и $\lambda$ (выходов) по формулам

\noindent
\begin{align}
a(t+1) &=\delta(a(t),\v x(t))\,;\notag\\[-6pt]
&\label{1}\\[-6pt]
\v y(t) &=
\lambda(a(t),\v x(t))\,.\notag
\end{align}

 Подробно некоторые модели латентности обнаружения неисправностей
(связанные,
ес\-тес\-т\-вен\-но, с моделями самих неисправностей) будут рас\-смот\-ре\-ны далее.


 Необходимость оценки возможной ла\-тент\-ности обнаружения неисправностей
на ранних этапах проектирования цифровых устройств (например, на автоматном
уровне~\cite{2}) диктуется следующими причинами:
\begin{itemize}
 \item высокой сложностью современных систем, делающей практически
невозможным полное исключения имеющихся дефектов при производственном
тестировании;
 \item влиянием внешних помех (космических лучей, излучения бытовой
электроники и т.\ п.) на современные сложные системы, особенно созданные с
использованием наноэлектроники~\cite{3}.
\end{itemize}
Поэтому современные системы должны включать в себя те или иные средства
самотестирования и самокоррекции~\cite{1} возможных ошибок, влияние которых на
производительность и надежность проектируемых систем зависит в том числе и
от времени между проявлением, обнаружением и коррекцией неисправности.
При этом в большинстве случаев значительная (относительно тактовых частот
работы системы) латентность может рассматриваться как фактор, снижающий
надежность~\cite{13, 12}.

 Поскольку латентность зависит от структуры автомата, опи\-сы\-ва\-юще\-го
проектируемую систему, ее оценку желательно проводить на достаточно ранних
этапах проектирования, чтобы эффективно управлять выбором проектных решений
на как можно более ранних стадиях.

 Будем считать, что адекватной мерой ла\-тент\-ности является ее функция
распределения вероятностей (ФР) $F_{\mathrm{FDL}}(x)$, т.\,е.\ ФР
$F_{\mathrm{FDL}}(x)={\bf Pr}(k\le x)$
числа $k$ тактов (см.\ рис.~1) от появления неисправности до
несовпадения хотя бы одной выходной переменной исправной и неисправной
систем на входной последовательности двоичных независимых векторов
(обоснованность данной меры в настоящей статье не рассматривается, отметим
только, что эта мера широко используется в литературе, см., например,~\cite{4, 7}).
При этом проявление неисправности моделируется как попадание в поглощающее
состояние ЦМ, соответствующей паре <<исправный\,--\,неисправный
автоматы>> при случайных независимых входах и использовании определенного
способа сравнения поведения этих автоматов.

 Для одиночного конечного автомата Мили, представленного своей таблицей
переходов, данную задачу можно считать принципиально решенной~\cite{4},
хотя, разумеется, остаются проблемы,\linebreak связанные с затратами ресурсов по мере
роста размерности автомата.
Однако в случае, когда конечный автомат представлен как сеть подавтоматов,
применение метода~\cite{4} для вычисления указанной выше ФР латентности в терминах
тех или иных характеристик подавтоматов наталкивается на серьез\-ные трудности,
связанные со сложностью описания таких систем в терминах ЦМ.

 В предлагаемой статье рассматривается возможный подход к вычислению ФР
латентности автомата, декомпозированного на три взаимо\-дей\-ст\-ву\-ющих подавтомата.
Эта ФР вычисляется в терминах вероятностей переходов подавтоматов с учетом
специфики конкретного алгоритма взаимодействия.

\section{Основной подход к оценке латентности обнаружения неисправности}

\subsection{Модель неисправности}

 Сделаем прежде всего несколько необходимых замечаний о понятии <<автомат с
неисправностью>> (модели неисправности).
Под неисправностью $f$ будем понимать некоторую трансформацию (относительно
проектной
спецификации) структуры или функции устройства, вызванную производственными,
проектными или индуцированными в процессе эксплуатации факторами.

 Наиболее простая классификация неисправностей состоит в разбиении множества
возможных неисправностей на перманентные (permanent, см.~\cite{1}) неисправности,
действующие (в смысле указанных трансформаций) в течение всего времени
наблюдения за работой устройства, и переходные (transient, см.~\cite{1, 3})
неисправности, представляющие собой лишь кратковременное (по сравнению с
интервалом наблюдения) отличие от спецификации.

 Наиболее распространенной моделью перманентной неисправности является
<<одиночная \mbox{константная} неисправность>>~\cite{8, 9}.
\begin{table*}[b]\small %tabl1
\vspace*{6pt}
\begin{center}
\parbox{272pt}{\Caption{Поведение исправного автомата и автомата с неисправностью $x_2 \equiv
1$
\label{t1pf}}
}

\vspace*{2ex}

\tabcolsep=7pt
\begin{tabular}{|c|c|c|c|c|c|c|}
\hline
\multicolumn{1}{|c|}{ } &
\multicolumn{3}{c|}{Исправный} &
\multicolumn{3}{c|}{С неисправностью} \\
\cline{2-7}
\multicolumn{1}{|c|}{\raisebox{8pt}[0pt][0pt]{$a_t$}} & $a_s$ & $X(a_t,a_s)$ & $Y(a_t,a_s)$ & $a_s$ & $X(a_t,a_s)$ &
$Y(a_t,a_s)$\\
\hline
 & $a_1$ & $\o x_1$ & $y_1,y_4$ & $a_1$ & $\o x_1$ & $y_1,y_4$
\\
%\hline
$a_1$ & $a_1$ & $x_1,\o x_2$ & $y_2,y_3$ & $a_2$ & $x_1,\o x_2$ & $y_2,y_4$
\\
%\hline
 & $a_2$ & $x_1,x_2$ & $y_2,y_4$ & $a_2$ & $x_1,x_2$ & $y_2,y_4$
\\
%\hline
\hline
 & $a_1$ & $\o x_2$ & $y_2,y_4$ & $a_1$ & $\o x_2,\o x_3$& $y_1,y_4$
\\
%\hline
\multicolumn{1}{|c|}{\raisebox{-6pt}[0pt][0pt]{$a_2$}} & & & & $a_2$ & $\o x_2,x_3$ & $y_2,y_4$
\\
%\hline
 & $a_1$ & $x_2,\o x_3$ & $y_1,y_4$ & $a_1$ & $x_2,\o x_3$ & $y_1,y_4$
\\
%\hline
 & $a_2$ & $x_2,x_3$ & $y_2,y_4$ & $a_2$ & $x_2,x_3$ & $y_2,y_4$
\\
\hline
\end{tabular}
\end{center}
\end{table*}
Пусть проектируемое устройство должно реализовать\linebreak булеву функцию
$f(x_1,\ldots,x_n)$, где $x_1,\ldots,x_n$~--- некоторые булевы переменные.
Неисправность устройства может быть задана, например, как
$f(x_1,\ldots,x_{i-1},x_i,x_{i+1},\ldots,x_n)\;\equiv$\linebreak
$\equiv\;f(x_1,\ldots,x_{i-1},1,x_{i+1},\ldots,x_n)$,
или, короче, $x_i\;\equiv$\linebreak $\equiv\;1$, что означает, что неисправное устройство работает
так, как если бы переменная $x_i$ принимала значение логической~<<1>> вне
зависимости от комбинаций входных наборов (в англоязычной литературе такая
неисправность называется\linebreak ``stuck-at-1'').
Аналогично можно рассмотреть неисправность $x_i \equiv 0$ (``stuck-at-0'').

 Что касается автоматных моделей устройств, то данные неисправности
можно связать с булевыми аргументами функций $\delta$ и $\lambda$ в~(\ref{1}).

 Заметим, что проявление неисправности $f$ в конечном автомате можно
задавать еще и в виде неправильных переходов
$$
\left \{a_i\mathop{\longrightarrow}\limits^{\v x} a_j\right \} \to
\left \{a_i\mathop{\longrightarrow}\limits^{\v x,f} a_k\right \}\,,
\quad k\ne j\,, % \eqno(2)
$$
на входном векторе $\v x$ из состояния~$a_i$ в состояние~$a_k$ вместо $a_j$.
Таким образом можно моделировать проявление как константных, так и других
неисправностей.

 Очевидно, что неисправность может приводить к неправильному
функционированию автомата, проявляющемуся непосредственно в момент появления
неисправности или через какое-то время после ее появления, а также вообще не
проявляться ни при каких наборах входных сигналов.
Под <<проявлением>> неисправности здесь понимается изменение состояния и/или
значений одной или нескольких выходных переменных в неисправном устройстве
относительно состояния и/или значений тех же переменных в исправном устройстве
при одинаковых значениях входных переменных.
В~частности, напомним, что модель константной неисправности определяется
неисправностью $x_i \equiv 0$ (``stuck-at-0'') или неисправностью $x_i \equiv 1$
(``stuck-at-1'').

\medskip

\noindent
{\bf Пример 1.} В качестве примера рассмотрим неисправность $x_2 \equiv 1$
в простейшем автомате, пред\-став\-лен\-ном табл.~\ref{t1pf}.


В этой таблице $a_t$ и $a_s$~--- предыдущее и последующее состояния
автомата,
$X(a_t,a_s)$~--- входной вектор, при котором происходит переход из состояния~$a_t$
в состояние $a_s$, и $Y(a_t,a_s)$~--- выходной вектор.
Напомним, что, как обычно при использовании кубической формы представления
автоматов (см., например,~\cite{2, 8}), обозначения $x_i$ и $\o x_i$ в столбце входов
``$X(a_t,a_s)$'' означают равенство единице и нулю $i$-й координаты вектора
$\v x$, отсутствующие в этом столбце переменные могут принимать любые значения
(0 или~1), а в столбце выходов ``$Y(a_t,a_s)$'' указываются только биты,
принимающие единичное значение на соответствующих переходах (остальные биты
нулевые), причем полагающиеся при векторной записи скобки опускаются.

 Из табл.~\ref{t1pf} видно, что входной вектор $\o x_1$ остав\-ля\-ет как исправный,
так и неисправный автоматы в состоянии $a_1$ и при этом
выходной вектор равен $y_1,y_4$, т.\,е.\ неисправность не проявляется.
Аналогично при входном векторе $x_1,x_2$ оба автомата из состояния $a_1$
переходят в состояние $a_2$ при выходном векторе $y_2,y_4$.
Однако при входном векторе $x_1,\o x_2$ исправный автомат остается в состоянии~$a_1$
при выходном векторе $y_2,y_3$, в то время как
неисправный автомат переходит из состояния~$a_1$ в состояние~$a_2$ при
выходном векторе $y_2,y_4$ (неправильный переход из состояния~$a_1$ при входном
наборе $(x_1=1, x_2=0)$ произойдет по условию $(x_1=1, x_2=1)$).
Это означает, что при данных предыдущем состоянии и входном векторе
неисправность проявится как в переходе состояний, так и в выходном векторе.

 Далее, при входном векторе $x_2,\o x_3$ из состояния~$a_2$ оба автомата
переходят в со\-сто\-яние~$a_1$ при выходном векторе $y_1,y_4$.
Аналогично при входном векторе $x_2,x_3$ оба автомата из состояния~$a_2$
переходят в со\-сто\-яние~$a_2$ при выходном векторе $y_2,y_4$.
Однако при входном векторе $\o x_2,\o x_3$ из со\-сто\-яния~$a_2$ исправный автомат
переходит в со\-сто\-яние~$a_1$ при выходном векторе $y_2,y_4$, а неисправный
автомат при этом же входном векторе также переходит из со\-сто\-яния~$a_2$
в со\-сто\-яние~$a_1$, но при выходном векторе~$y_1,y_4$.
Таким образом, неисправность проявляется только в выходном векторе.

Наконец, при входном векторе $\o x_2,x_3$ из со\-сто\-яния~$a_2$ исправный
автомат переходит в со\-сто\-яние~$a_1$ при выходном векторе $y_2,y_4$,
а неисправный автомат переходит из со\-сто\-яния~$a_2$ в со\-сто\-яние~$a_2$,
правда, при том же выходном векторе $y_2,y_4$.
Теперь неисправность проявится только в переходе состояний, но не в
выходном векторе.

В заключение приведем значения функций $\delta$ и~$\lambda$ для
исправного автомата:
\begin{align*}
\delta(a_1,\v x) & =
\begin{cases}
a_1\,, & \o x_1 \vee (x_1 \wedge \o x_2) = 1\,, \\
a_2\,, & \text{иначе};
\end{cases}\\
\delta(a_2,\v x) &=
\begin{cases}
a_1, & \o x_2 \vee (x_2 \wedge \o x_3) = 1\,, \\
a_2, & \text{иначе};
\end{cases}\\
\end{align*}


\vspace*{-24pt}

\noindent
%%%%%%%%%%%%%%%%%%%%%
\begin{align*}
\lambda_1(a_1,\v x) & = \o x_1\,;
&
\lambda_2(a_1,\v x) &= x_1\,;
\\
\lambda_3(a_1,\v x) &= x_1 \wedge \o x_2\,;
&
\lambda_4(a_1,\v x) &= \o x_1 \vee (x_1 \wedge x_2)\,;\\
\lambda_1(a_2,\v x) &= x_2 \wedge \o x_3\,;
&
\lambda_2(a_2,\v x) &= \o x_2 \vee (x_2 \wedge x_3)\,;\\
\lambda_3(a_2,\v x) & \equiv 0\,;
&
\lambda_4(a_2,\v x) & \equiv 1\,,
\end{align*}
а также функций $\delta^f$ и $\lambda^f$ для неисправного автомата:
\begin{align*}
\delta^f(a_1,\v x) &=
\begin{cases}
a_1, & \o x_1 = 1\,, \\
a_2, & \text{иначе}\,;
\end{cases}
\\
\delta^f(a_2,\v x) &=
\begin{cases}
a_1, & \o x_3 = 1\,, \\
a_2, & \text{иначе}\,;
\end{cases}\\
\end{align*}

\vspace*{-24pt}

\noindent
\begin{align*}
\lambda_1^f(a_1,\v x) &= \o x_1\,;
&
\lambda_2^f(a_1,\v x) &= x_1\,;\\
%\quad
\lambda_3^f(a_1,\v x) &\equiv 0\,;
&
\lambda_4^f(a_1,\v x) & \equiv 1\,;
\\
\lambda_1^f(a_2,\v x) &= \o x_3\,;
&
\lambda_2^f(a_2,\v x) &= x_3\,;\\
\lambda_3^f(a_2,\v x) & \equiv 0\,;
&
\lambda_4^f(a_2,\v x) &\equiv 1\,.
\end{align*}

%\vspace*{-6pt}

%\hfill $\blacktriangleleft$

\vspace*{-6pt}

%%% 2.2.
\subsection{Латентность обнаружения неисправности}

Проиллюстрируем простыми примерами явление латентности обнаружения
неисправности в автомате для рассмотренной выше модели неисправности.

Поскольку, как отмечалась во введении, представляет интерес некоторая
вероятностная мера латентности, то необходимо определиться с источником
случайности в поведении автоматов.
Будем считать, что им является случайность входных сигналов,
характеризуемая вероятностью единичного значения для каждого входа
(бита входного век\-тора).

\medskip

\noindent
{\bf Пример 2.}
Явление латентности обнаружения неисправности в автомате покажем на
автомате из примера~1.

Обозначим через $p_i={\bf Pr}(x_i=1)$ вероятность единичного значения
для $i$-го входа ($i$-го бита входного вектора) независимо от значений
остальных входов.
Соответственно, вероятность $q_i={\bf Pr}(x_i=0)$ нулевого значения этого
входа равна $q_i=1-p_i$.

 Пусть в некоторый момент (который будем отож\-дествлять с моментом~0) в
реальном автомате, находившемся в со\-сто\-янии~$a_1$, возникла неисправность
$x_2 \equiv 1$.
Вычислим значения латентности (FDL) на некоторых последовательностях
поступающих сигналов.
\begin{enumerate}[(1)]
\item Первая последовательность~---
$\o x_1 \to x_1,x_2\; \to$\linebreak $\to\; x_2,\o x_3 \to x_1,\o x_2$ (вероятность возникновения
такой последовательности $q_1 \cdot p_1 p_2 \cdot p_2 q_3\; \times$\linebreak $\times\; p_1 q_2$).
Тогда на этой последовательности\linebreak
 по\-сту\-па\-ющих сигналов последовательность
состояний и выходных сигналов (в скобках) в исправном автомате будет
$a_1 \to a_1(y_1,y_4)\; \to$\linebreak
$\to\;a_2(y_2,y_4) \to a_1(y_1,y_4) \to a_1(y_1,y_4)$, а в
неисправном автомате~---
$a_1 \to a_1(y_1,y_4)\; \to$\linebreak
$\to\;a_2(y_2,y_4) \to a_1(y_1,y_4) \to a_2(y_2,y_4)$,
т.\,е.\ на четвертом шаге впервые на исправном и неисправном автоматах появились
разные значения внутренних переменных, причем одновременно различными стали
и состояния, и выходные сигналы.
Видно, что на этой последовательности входных наборов $\mathrm{FDL} = 4$.
\item Следующая последовательность~---
$\o x_1\;\to$\linebreak
$\to\;x_1,x_2 \to x_2,x_3 \to \o x_2,\o x_3$
(вероятность $q_1\; \times$\linebreak $\times\;p_1 p_2 \cdot p_2 p_3 \cdot q_2 q_3$).
Тогда в исправном автомате
$a_1 \to a_1(y_1,y_4) \to a_2(y_2,y_4) \to a_2(y_2,y_4)\;\to$\linebreak
$\to\;a_1(y_2,y_4)$,
а в неисправном~---
$a_1\;\to$\linebreak $\to\;a_1(y_1,y_4) \to a_2(y_2,y_4) \to a_2(y_2,y_4)\; \to$\linebreak
$\to\;a_1(y_1,y_4)$.
Здесь также $\mathrm{FDL} = 4$, но\linebreak
 неисправность проявляется только в выходном сигнале.
\item Наконец, пусть поступила последовательность
$\o x_1 \to x_1,x_2 \to \o x_2,x_3 \to x_1,x_2,x_3 \to
x_2,x_3\; \to$\linebreak
 $\to\; x_2,\o x_3 \to x_1,\o x_2$
(вероятность $q_1 \cdot p_1 p_2\; \times$\linebreak
$\times\; q_2 p_3 \cdot
p_1 p_2 p_3 \cdot p_2 p_3 \cdot p_2 q_3 \cdot p_1 q_2$).
Тогда переходы исправного и неисправного автоматов
$a_1 \to a_1(y_1,y_4) \to a_2(y_2,y_4) \to a_1(y_2,y_4)\;\to$\linebreak
$\to\; a_2(y_2,y_4) \to a_2(y_2,y_4) \to a_1(y_1,y_4)\;\to$\linebreak
$\to\; a_1(y_1,y_4)$
и
$a_1 \to a_1(y_1,y_4) \to a_2(y_2,y_4)\; \to$\linebreak
$\to\; a_2(y_2,y_4) \to a_2(y_2,y_4)
\to a_2(y_2,y_4)\;\to$\linebreak
$\to\;a_1(y_1,y_4) \to a_2(y_2,y_4)$ и $\mathrm{FDL} = 7$.
Интересно отметить, что в данном случае на третьем шаге на исправном и
неисправном автоматах состояния были различными (при одинаковых выходных
сигналах), но затем на
\begin{table*}\small %tabl2
\begin{center}
\Caption{Пример не обнаруживающей себя неисправности
\label{t2pf}}
\vspace*{2ex}

\begin{tabular}{|c|c|c|c|c|c|}
\hline
 & &
\multicolumn{2}{|c|}{ Исправный} &
\multicolumn{2}{|c|}{С неисправностью} \\
\cline{3-6}
\multicolumn{1}{|c|}{\raisebox{6pt}[0pt][0pt]{\ \ \ $a_t$\ \ \ }} &
\multicolumn{1}{|c|}{\raisebox{6pt}[0pt][0pt]{$X(a_t,a_s)$}}
&\ \ \ $a_s$\ \ \ &\ \ $Y(a_t,a_s)$\ \ &
 \ \ \ $a_s$\ \ \ & $Y(a_t,a_s)$ \\
\hline
%\hline
 & $\o x_1$ & $a_1$ & $y_1,y_2$ & $a_1$ & $y_1,y_2$ \\
%\hline
 & $x_1,\o x_3$ & $a_1$ & $y_3$ & $a_1$ & $y_3$ \\
%\hline
$a_1$ & $x_1,x_3,\o x_4$ & $a_1$ & & $a_1$ & \\
%\hline
 & $x_1,x_3,x_4,\o x_5$ & $a_2$ & $y_2,y_3$ & $a_3$ & $y_2,y_3$ \\
%\hline
 & $x_1,x_3,x_4,x_5$ & $a_3$ & $y_2,y_3$ & $a_2$ & $y_2,y_3$ \\
\hline
%\hline
 & $\o x_1,x_4$ & $a_1$ & $y_1$ & $a_1$ & $y_1$ \\
%\hline
\multicolumn{1}{|c|}{\raisebox{-6pt}[0pt][0pt]{$a_2$ }}& $\o x_4$ & $a_2$ & $y_2$ & $a_3$ & $y_2$ \\
%\hline
 & $x_1,\o x_2,x_4$ & $a_2$ & $y_2$ & $a_2$ & $y_2$ \\
%\hline
 & $x_1,x_2,x_4$ & $a_3$ & $y_2$ & $a_2$ & $y_2$ \\
\hline
%\hline
 & $\o x_1,x_4$ & $a_1$ & $y_1$ & $a_1$ & $y_1$ \\
%\hline
\multicolumn{1}{|c|}{\raisebox{-6pt}[0pt][0pt]{$a_3$}} & $x_1,x_2,x_4$ & $a_2$ & $y_2$ & $a_3$ & $y_2$ \\
%\hline
 & $\o x_4$ & $a_3$ & $y_2$ & $a_2$ & $y_2$ \\
%\hline
 & $x_1,\o x_2,x_4$ & $a_3$ & $y_2$ & $a_2$ & $y_2$ \\
\hline
\end{tabular}
\end{center}
\end{table*}
четвертом шаге неисправный автомат <<восстановился>>
и работал в точности, как исправный, до седьмого шага.
\end{enumerate}

%\hfill
%\rightline{ $\blacktriangleleft$}

\medskip

\noindent
{\bf Пример~3.} Приведем пример, когда неисправность вообще не
обнаруживается по выходу (табл.~\ref{t2pf}).
Ошибочный переход $a_1 \to a_3$ вместо $a_1 \to a_2$ может произойти,
например, в результате некоторой переходной (transient) неисправности.
При этом, однако, поскольку выходы исправного и <<сбойного>> автоматов
совпадают, причем и при следующих переходах, рассогласование в выходных
переменных автоматов не происходит.

%\rightline{ $\blacktriangleleft$}

%%% 2.3.
\subsection{Модель латентности в терминах произведения исправного
и~неисправного автоматов}

 Поскольку латентность определяется по факту несовпадения выходов
автомата при наличии и отсутствии неисправности, указанная информация в том
или ином виде должна содержаться в математической модели вычисления ФР
латентности.
Такая модель предложена в~\cite{4}.
Ниже приводится несколько отличное от \cite{4} описание этой модели,
предложенное в~\cite{5}.

Пусть $\v x(t) = (x_1(t),\ldots,x_n(t))\in {\cal X}$~--- вектор двоичных
входных переменных в момент $t$,
$a(t)\in {\cal A}$ и $a^f(t)\in {\cal A}$~--- состояния исправ\-ного и
$f$-неисправ\-ного автоматов Мили в этот момент,
$\v y(t)=(y_1(t),\ldots,y_m(t))\in {\cal Y}$ и
$\v y^f(t)\;=$\linebreak $=\;(y^f_1(t),\ldots,y^f_m(t))\in {\cal Y}$~---
соответствующие векторы двоичных выходных переменных.
Предполагается, что входные векторы представляют
собой независимые (во времени и побитно) случайные последовательности.

%\begin{figure*}[b]
%fig2
%\vspace*{1pt}
\begin{center}
\vspace*{6pt}
\mbox{%
\epsfxsize=73.815mm
\epsfbox{pech-2.eps}
}
\end{center}
%\vspace*{6pt}
%\label{f1ush}}
%\end{figure*}
{{\figurename~2}\ \ \small{Произведение исправного и неисправного конечных автоматов}}

\bigskip
\medskip
%\medskip
\addtocounter{figure}{1}

 На рис.~2 изображен автомат~--- произведение исправного и
неисправного автоматов.
Выходной вектор $(\v y,\v y^f)$ представляет собой вектор раз\-мер\-ности $2m$,
координатами которого являются выходные векторы $\v y$ и $\v y^f$
исправного и неисправного автоматов.
Соответственно, пространство состояний такого автомата (произведения
автоматов)~--- прямое произведение множеств $a(t)\in {\cal A}$ и
$a(t)\in {\cal A}$, т.\,е.\ множество пар $(a_k,a^f_l)$,\ $k,l=\o{1,r}$.

В~\cite{4} вероятность того, что хотя бы на одном входном наборе значение хотя бы
одного выходного бита в присутствии неисправности $f$ будет отлично от
исправного автомата, выражается как вероятность поглощения ЦМ,
соответствующей переходам произведения исправного и неисправного
автоматов (рис.~2) на указанной случайной входной последовательности.
Поглощающее состояние определяется подмножеством состояний, при которых
выходные векторы обоих автоматов не совпадают хотя бы в одном бите.
Обозначим указанную
цепь как ЦМПА (цепь Маркова произведения автоматов).


\begin{table*}[b]\small %tabl3
\begin{center}
\Caption{Матрица $P$ вероятностей переходов ЦМПА
\label{t3pf}}
\vspace*{2ex}

\begin{tabular}{|c||c|c|c|c|c|}
\hline
Элемент& $(1,1)$ & $(1,2)$ & $(2,1)$ & $(2,2)$ & Поглощающее \\
%\hline
\hline
$(1,1)$ &\ \ 0,900\ \ \ &\ \ 0,000\ \ \ &\ \ 0,000\ \ \ &\ \ 0,020\ \ \ &\ \
0,080\ \ \ \\
%\hline
$(1,2)$ & 0,630 & 0,000 & 0,000 & 0,006 & 0,364 \\
%\hline
$(2,1)$ & 0,126 & 0,080 & 0,000 & 0,006 & 0,788 \\
%\hline
$(2,2)$ & 0,140 & 0,240 & 0,000 & 0,060 & 0,560 \\
%\hline
Поглощающее & 0,000 & 0,000 & 0,000 & 0,000 & 1,000 \\
\hline
\end{tabular}
\end{center}
\end{table*}

Кратко модель~\cite{4} может быть описана сле\-ду\-ющим образом:
\begin{itemize}
\item
 входные переменные исправного и неисправного автоматов представляют
собой двоичные сигналы с вероятностями
$p_i = {\bf Pr}(x_i=1)$ и $q_i = {\bf Pr}(x_i=0) = 1-p_i$,\ $i = \o{1,n}$,
одновременно подаваемые на входы обоих автоматов.
Соответственно, вероятности переходов как в исходном автомате, так и в
автомате с неисправностью $f$, определяются формулой вы\-чис\-ле\-ния вероятности
единичного значения булевой функции по известным вероятностям единиц ее
аргументов.
Например, переходы из со\-сто\-яния~$a_1$ в со\-сто\-яние~$a_1$ в исправном автомате
табл.~\ref{t1pf} происходят с вероятностью
${\bf Pr}(x_1=0)+{\bf Pr}(x_1=1,x_2=0) = q_1+p_1 q_2$;
\item
 начальные состояния у обоих автоматов одинаковы, поскольку различие
в их поведении может проявиться только после первого такта работы;
\item
 неисправности не добавляют новых состояний, и множество состояний
исправного и неисправного автоматов совпадают;
\item
 пространство состояний указанной ЦМПА представляет собой множество
${\cal W} = \{w_i,\ i\;=$\linebreak
$=\;\o{1,r^2+1}\} = \{(a_k,a_l),\ \ k,l=\o{1,r}\}\cup s_A$
из $(r^2 + 1)$ элементов, где $s_A$~--- поглощающее состояние, а $r$~---
число состояний исходного автомата.
Под поглощающим со\-сто\-яни\-ем~$s_A$ понимается фиктивное состояние пары
автоматов, появляющееся при первом несовпадении хотя бы в одном бите
выходных векторов обоих автоматов.
\end{itemize}

 Переходы в ЦМПА~--- пары $(w_1,w_2)$~--- отражают возможные изменения
пар состояний в исправном и неисправном автоматах под действием случайных
входов автомата.
Соответственно, матрица $P = (p_{w_1,w_2})$,\ $w_1,w_2\in{\cal W}$,
вероятностей переходов за один такт
задает вероятности того, что под влиянием случайно выбранного (на основе
распределений $p_i$,\ $i=\o{1,n}$) входного вектора в исправном автомате
будет иметь место переход $a_i\to a_j$, тогда как в неисправном~--- $a_k\to a_l$.
Иными словами, эта матрица описывает вероятности всех возможных переходов
пар автоматов, причем те из них, при которых есть несовпадение хотя бы на
одном из выходов, рассматриваются как переходы в поглощающее со\-сто\-яние~$s_A$.
Вероятности переходов, в свою очередь, вычисляются по вероятностям входных
сигналов согласно пункту~(1).

\bigskip

\noindent
 {\bf Пример 4.} Таблица~\ref{t3pf} соответствует матрице~$P$ вероятностей переходов
ЦМПА для автоматов из примера~1, где в каждой паре $(i,j)$ левый номер
соответствует со\-сто\-янию~$a_i$ в исправном автомате, а правый~--- состоянию
$a_j$ в неисправном автомате.
Расчеты проводились с помощью программы, написанной В.\,В.~Чаплыгиным (ИПИРАН).
При расчетах были использованы следующие значения вероятностей $p_i$:
$p_1 = 0{,}1$,\ $p_2 = 0{,}2$,\ $p_3 = 0{,}3$.


%\rightline{ $\blacktriangleleft$}
\medskip
 С учетом сказанного ФР $F_{\mathrm{FDL}}(k)={\bf Pr}(\mathrm{FDL}\le$\linebreak $\le\; k)$,\ \ $k \ge 1$,
латентности обнаружения не\-ис\-прав\-ности $f$, т.\ е.\ вероятность события
<<латентность не превосходит $k$ тактов>> можно записать в виде
\begin{multline}
F_{\mathrm{FDL}}(k)
={}\\
{}=
{\bf Pr}(\min\{m > \tau: \v y(\tau + m) \ne \v y^f(\tau + m)\} \le
 k)={}\\
 {}=
\sum_{w\in {\cal W}} p_w(0) p_{w,s_A}(k)\,,
\label{3}
\end{multline}
где
$\v y(t)$, $\v y^f(t)$~--- векторы выходов исправного и
$f$-неисправного автоматов;
$m$~--- число тактов, прошедшее с момента появления неисправности
(например, для неисправности $x_i \equiv 1$~--- с момента <<залипания>>
переменной $x_i$ в состоянии <<1>> в неисправном автомате после такта $\tau$);
$p_w(0)$,\ $w\in{\cal W}$,~--- координаты вектора $\v p(0)$
(размерности $r^2 + 1$) вероятностей состояний ЦМПА в начальный момент
(на такте~$\tau$, после которого возникла неисправность, см.\ рис.~1).
Поскольку в начальный момент оба автомата исправны, причем имеют одинаковые
состояния (и, соответственно, одинаковые вероятности состояний), то
координаты $p_w(0)$ с номерами $(i-1)r+i$,\ $i=\o{1,r}$, совпадают с
$i$-ми координатами вектора из вероятностей состояний исправного
\begin{figure*} %fig3
\vspace*{1pt}
\begin{minipage}[t]{81mm}
\begin{center}
\vspace*{1pt}
\mbox{%
\epsfxsize=72.248mm
\epsfbox{pech-3.eps}
}
\end{center}
\vspace*{-9pt}
\Caption{Вероятность попадания в поглощающее состояние из состояния
$(a_1,a_1)$
\label{f3pf}}
%\end{figure*}
\end{minipage}
\hfill
\begin{minipage}[t]{81mm}
%\begin{figure*} %fig4
%\vspace*{1pt}
\begin{center}
\vspace*{1pt}
\mbox{%
\epsfxsize=72.51mm
\epsfbox{pech-4.eps}
}
\end{center}
\vspace*{-9pt}
\Caption{Вероятность попадания в поглощающее состояние из состояния
$(a_2,a_1)$
\label{f4pf}}
%\end{figure*}
\end{minipage}
\end{figure*}
автомата
на такте $\tau$, а остальные координаты~$p_w(0)$ равны нулю;
$p_{w_1,w_2}(k)$,\ $k \ge 1$,\ $w_1,w_2\in {\cal W}$,~---
элемент матрицы $P(k) = P^k$ вероятностей переходов ЦМПА за $k$ тактов.

 Матричная запись равенства~(\ref{3}) имеет вид
$$
{\bf Pr}(\mathrm{FDL} \le k) = \v p(0) P^{k} \v 1_e\,,
$$
где $\v 1_e$~--- вектор размерности $(r^2 + 1)$, все координаты которого,
кроме последней, равны нулю, а последняя~--- единице.

\medskip
\noindent
{\bf Пример 5.} На рис.~\ref{f3pf} и~\ref{f4pf} представлены графики ФР латентности
для пары автоматов из примера~1 (см.\ также пример~4).
При этом считалось, что момент~$\tau$ возникновения неисправности совпадал
с начальным моментом 0 функционирования автомата, причем в первом случае
(рис.~\ref{f3pf}) предполагалось, что в этот момент исправный (неисправный также)
автомат находился в состоянии $a_1$ (вектор начальных вероятностей
$\v p(0)=(1,0,0,0,0))$.
Предположение второго случая (рис.~\ref{f4pf}) о том, что в момент~0 исправный
автомат находился в со\-сто\-янии~$a_2$, а неисправный~--- в со\-сто\-янии~$a_1$
(вектор начальных вероятностей $\v p(0)=(0,0,1,0,0))$,
кажется на первый взгляд несколько искусственным, но оказывается весьма
полезным при дальнейшем изучении латентности.

%\rightline{ $\blacktriangleleft$}

\vspace*{-6pt}
%%% 3
\section{Декомпозиция конечного автомата в сеть подавтоматов}

\vspace*{-3pt}

%%% 3.1
\subsection{Проблема декомпозиции конечного автомата}
\vspace*{-1pt}

Поскольку целью настоящей статьи является демонстрация возможности использования
модели произведения исправного и неисправного автоматов для вычисления
латентности обнаружения неисправности в случае, когда при проектировании
системы, специфицированной некоторым автоматом, он должен быть декомпозирован
в некоторую сеть подавтоматов, то в этом пункте будет кратко изложена постановка
проблемы такой декомпозиции.
Подробное описывание алгоритмов декомпозиции автомата по критерию сокращения
латентности обнаружения неисправности можно найти в~\cite{6}.

В настоящей работе для упрощения изложения предлагается декомпозиция исходного
автомата в сеть трех взаимодействующих подавтоматов, хотя результаты работы
элементарно переносятся на декомпозицию в сеть любого числа подавтоматов.

При декомпозиции в сеть подавтоматов (рис.~5) каждый из подавтоматов
реализует свою
часть таб\-ли\-цы переходов исходного автомата (все его состояния и переходы
между ними представляются в таблице переходов соответствующего подавтомата).
Координирует работу подавтоматов управляющий блок, включающий в себя,
в частности, автомат-супервизор (SFSM~--- Supervisor FSM).



В каждый момент времени один из подавтоматов находится в рабочем состоянии,
когда его  выходы используются согласно функциональному
назначению (спецификации), а два других~--- в тес-\linebreak

\begin{center} %fig5
\vspace*{6pt}
\mbox{%
\epsfxsize=75.494mm
\epsfbox{pech-5.eps}
}
\end{center}
%\vspace*{6pt}
%\label{f1ush}}
%\end{figure*}
{{\figurename~5}\ \ \small{Декомпозиция конечного автомата в сеть подавтоматов}}

\bigskip
\medskip
%\medskip
\addtocounter{figure}{1}


\noindent
то\-вом, когда их выходы только проверяются на совпадение
с исправными (на соответствие спецификации), но ни в каких действиях с
окружением (например, в управлении некоторым операционным блоком, для которого
проектируется автомат) они не участвуют.

Переменные выходов супервизора также присутствуют в подавтоматах, увеличивая
число входов по сравнению с рабочими.
Существенно, что переходы внутри одного подавтомата не изменяют состояния
супервизора.
При этом вектор $\v x^i$,\ $i=\o{1,3}$, входных переменных $i$-го подавтомата
имеет вид $\v x^i=(\v x_i,\v p_i)$, где
$\v x_i \in {\cal X}_i$,\ $\v p_i \in {\cal P}_i$,\ ${\cal X}_i$~---
подмножество входных наборов, существенное в состояниях подавтомата $c_i$,
${\cal P}_i$~--- множество сигналов супервизора, поступающих на этот подавтомат.
Очевидно, что
$\bigcup\limits_{i=1}^3 {\cal X}_i = {\cal X}$.
Аналогично векторы $\v y^i$,\ $i=\o{1,3}$, выходов каждого подавтомата
представимы в виде $\v y^i=(\v y_i,\v q_i)$, где
$\v y_i \in {\cal Y}_i$,\ $\v q_i \in {\cal Q}_i$,\ ${\cal Y}_i$~---
множество выходов подавтомата $c_i$, ${\cal Q}_i$~--- множество
дополнительных выходов подавтомата $c_i$, передающих информацию
о его текущем состоянии супер\-ви-\linebreak зору.


Тестирование сети автоматов с целью обнаружения неисправности производится
на основе выходных сигналов (``to test'' на рис.~5) с подавтоматов.


\vspace*{6pt}
\subsection{Модель декомпозиции исправного автомата}

\vspace*{2pt}
%%% 3.2

Опишем теперь модель декомпозиции автомата в сеть подавтоматов, для
которой в следующем разделе будет построен алгоритм вычисления ФР
латентности обнаружения неисправности.
С точки зрения вероятностных характеристик латентности эта модель достаточно
адекватно отражает функционирование приведенного выше декомпозированного
конечного автомата.

Декомпозиция производится в соответствии с таблицей переходов исходного
автомата и сле\-ду\-ющи\-ми правилами функционирования подавтоматов и
автомата-супервизора (управляющего блока на рис.~5):\\[-9pt]
\begin{itemize}
\item
если переходы происходят внутри одного под\-ав\-то\-ма\-та (компонента сети), то
переходы и выходы этого компонента соответствуют переходам и выходам
исходного автомата;\\[-9pt]
\item
если начальное и конечное состояния данного перехода принадлежат разным
подавтоматам, то подавтомат, содержащий начальное состояние, становится
тестируемым и переходит в некоторое (определенное таблицей переходов
данного подавтомата в рабочем режиме) начальное состояние, а подавтомат,
содержащий конечное состояние перехода, инициируется специальным сигналом
автомата-суперви-\linebreak зора;
\item
конечное состояния перехода тестируемого под\-ав\-то\-ма\-та (до момента его
перехода в рабочий режим) определяется таблицей переходов данного
подавтомата в тестовом режиме);\\[-9pt]
\item
выходом сети подавтоматов являются вектор $\v y=(\v y_1,\v y_2,\v y_3)$,
состоящий из векторов $\v y_1$, $\v y_2$ и $\v y_3$ выходов подавтоматов,
и номер $k$ рабочего подавтомата.\\[-9pt]
\end{itemize}

Заметим, что в соответствии с приведенными правилами функционирования
сети подавтоматов, если в исправном режиме функционирования рассматривать
только вектор $\v y_k$ выходов рабочего под\-ав\-то\-ма\-та, то он будет совпадать
с вектором выходов исходного автомата.

\medskip
\noindent
{\bf Пример 6.} Рассмотрим (исправный) автомат табл.~\ref{t4pf}.

Предположим, что произведена декомпозиция этого автомата на подавтоматы
$c_1$, $c_2$ и $c_3$ с множествами состояний ${\cal A}_1 = \{a_1,a_2,a_5\}$,\
${\cal A}_2\; =$\linebreak $=\; \{a_6,a_7,a_8\}$ и ${\cal A}_3 = \{a_3,a_4,a_9\}$.
Декомпозиция произведена в соответствии с правилами функционирования
подавтоматов и автомата-супервизора (таблица переходов которого здесь не
приводится, см.~\cite{6}), определяемыми табл.~\ref{t5pf}.


Приведем комментарии к табл.~\ref{t4pf} и~\ref{t5pf}.

Если рабочий подавтомат находится в состоянии $a_i$ (столбец ``$a_t$'') и
приходит сигнал~$\v x$ (столбец ``$X(a_t,a_s)$''), переводящий исходный
автомат (табл.~\ref{t4pf}) в со\-сто\-яние~$a_j$ (столбец~``$a_s$'')
того же под\-ав\-то\-ма\-та, то этот подавтомат остается рабочим при новом
состоянии $a_j$ (см.\ табл.~\ref{t4pf} и~\ref{t5pf}).
Если же рабочий подавтомат находится в со\-сто\-янии~$a_i$ и приходит сигнал
$\v x$, переводящий исходный автомат в со\-сто\-яние~$a_j$ другого подавтомата,
то рабочим становится новый подавтомат при со\-сто\-янии~$a_j$ (столбец~``$a_s$''
табл.~\ref{t4pf}), а прежний рабочий подавтомат переходит в режим тестирования при
со\-сто\-янии~$a_k$ (столбец~``$a_s$'' табл.~\ref{t5pf}).
Более того, в это же состояние переходит и тестируемый подавтомат,
если меняются ролями два других.
Наконец, переходы тестируемого подавтомата (до момента любого изменения
рабочего подавтомата) происходят в соответствии с табл.~\ref{t5pf}).


Например, если работает подавтомат~$c_1$, находящийся в со\-сто\-янии~$a_1$,
и приходит сигнал $x_3,x_4,x_5$, %\linebreak\vspace*{-12pt}
 то подавтомат~$c_1$ остается рабочим
при том же со\-сто\-янии~$a_1$ (см.\ табл.~\ref{t4pf} и~\ref{t5pf}).
Но ес-\linebreak\vspace*{-12pt}
\pagebreak

\end{multicols}

\begin{table*}\small %tabl4-5
\begin{center}
\begin{minipage}[t]{190pt}
%\begin{center}
\Caption{Таблица переходов 
\label{t4pf} автомата\protect\newline
}
\vspace*{2ex}
\vspace*{10pt}

\tabcolsep=12pt
\begin{tabular}{|c|c|c|c|}
\hline
$a_t$ & $a_s$ & $X(a_t,a_s)$ & $Y(a_t,a_s)$ \\
%\hline
\hline
& $a_1$ & $x_3,x_4,x_5$ & $-$\\
%\cline{2-4}
%\hline
\multicolumn{1}{|c|}{\raisebox{-6pt}[0pt][0pt]{$a_1$}}  & $a_5$ & $x_3,x_4,\o x_5$ & $y_1,y_2$\\
%\cline{2-4}
%\hline
 & $a_7$ & $x_3,\o x_4$ & $y_1,y_3$\\
%\cline{2-4}
%\hline
 & $a_9$ & $\o x_3$ & $y_2$\\
\hline
%\hline
 & $a_1$ & $x_5,x_6,x_7$ & $y_2,y_3,y_4$\\
%\cline{2-4}
%\hline
\multicolumn{1}{|c|}{\raisebox{-6pt}[0pt][0pt]{$a_2$}} & $a_7$ & $x_5,x_6,\o x_7$ & $y_3$\\
%\cline{2-4}
%\hline
 & $a_6$ & $x_5,\o x_6$ & $y_2,y_3$\\
%\cline{2-4}
%\hline
 & $a_6$ & $\o x_5$ & $y_2$\\
\hline
%\hline
 & $a_6$ & $x_5,x_6$ & $y_2,y_3$\\
%\cline{2-4}
%\hline
\multicolumn{1}{|c|}{\raisebox{-6pt}[0pt][0pt]{$a_3$}} & $a_7$ & $x_5,\o x_6$ & $y_3,y_4$\\
%\cline{2-4}
%\hline
 & $a_8$ & $\o x_5,\o x_6$ & $y_3,y_4$\\
%\cline{2-4}
%\hline
 & $a_2$ & $\o x_5,x_6$ & $y_2$\\
\hline
%\hline
 & $a_3$ & $x_6,x_7$ & $y_2$\\
%\cline{2-4}
%\hline
\multicolumn{1}{|c|}{\raisebox{-6pt}[0pt][0pt]{$a_4$}} & $a_5$ & $x_6,\o x_7$ & $y_2,y_3$\\
%\cline{2-4}
%\hline
 & $a_7$ & $\o x_6,x_7$ & $y_3$\\
%\cline{2-4}
%\hline
 & $a_8$ & $\o x_6,\o x_7$ & $y_3,y_4$\\
\hline
%\hline
   & $a_2$ & $x_1,x_2$ & $y_3,y_4$ \\
%\cline{2-4}
%\hline
$a_5$ & $a_3$ & $x_1,\o x_2$ & $y_4,y_5$ \\
%\cline{2-4}
%\hline
 & $a_4$ & $\o x_1$ & $y_3,y_5$ \\
%\hline
\hline
  & $a_2$ & $x_4,x_5$ & $y_1,y_3$ \\
%\cline{2-4}
%\hline
$a_6$ & $a_5$ & $x_4,\o x_5$ & $y_2$ \\
%\cline{2-4}
%\hline
 & $a_7$ & $\o x_4$ & $y_1,y_3$ \\
%\hline
\hline
  & $a_4$ & $x_2,x_3$ & $y_2$ \\
%\cline{2-4}
%\hline
$a_7$ & $a_5$ & $\o x_2,x_3$ & $y_1,y_3$ \\
%\cline{2-4}
%\hline
 & $a_7$ & $\o x_3$ & $y_3,y_4$ \\
\hline
%\hline
 & $a_8$ & $x_3,x_4$ & $y_2$ \\
%\cline{2-4}
%\hline
\multicolumn{1}{|c|}{\raisebox{-6pt}[0pt][0pt]{$a_8$}} & $a_6$ & $x_3,\o x_4$ & $y_1,y_3$ \\
%\cline{2-4}
%\hline
  & $a_7$ & $\o x_3,x_5$ & $y_1,y_3$ \\
%\cline{2-4}
%\hline
 & $a_8$ & $\o x_3,\o x_5$ & $-$ \\
\hline
%\hline
  & $a_3$ & $x_1,x_3$ & $y_3$ \\
%\cline{2-4}
%\hline
$a_9$ & $a_4$ & $x_1,\o x_3$ & $y_3,y_5$ \\
%\cline{2-4}
%\hline
 & $a_5$ & $\o x_1$ & $y_3,y_4$ \\
\hline
\end{tabular}
%\end{center}
\end{minipage}
%\end{center}
\hspace*{17mm} 
%\hfill
\begin{minipage}[t]{190pt}
%\end{table*}
%\vspace*{12pt}
%\begin{table*}\small
%\begin{center}
\Caption{Таблица переходов подавтоматов сети автоматов
\label{t5pf}}
\vspace*{2ex}

\tabcolsep=12pt
\begin{tabular}{|c|c|c|c|}
\hline
 $a_t$ & $a_s$ & $X(a_t,a_s)$ & $Y(a_t,a_s)$ \\
%\hline
\hline
 & $a_1$ & $x_3,x_4,x_5$ & $-$\\
%\hline
\multicolumn{1}{|c|}{\raisebox{-6pt}[0pt][0pt]{$a_1$}} & $a_5$ & $x_3,x_4,\o x_5$ & $y_1,y_2$\\
%\hline
 & $a_5$ & $x_3,\o x_4$ & $y_1,y_3$\\
%\hline
 & $a_5$ & $\o x_3$ & $y_2$\\
\hline
%\hline
 & $a_1$ & $x_5,x_6,x_7$ & $y_2,y_3,y_4$\\
%\hline
\multicolumn{1}{|c|}{\raisebox{-6pt}[0pt][0pt]{$a_2$}} & $a_5$ & $x_5,x_6,\o x_7$ & $y_3$ \\
%\hline
 & $a_5$ & $x_5,\o x_6$ & $y_2,y_3$\\
%\hline
 & $a_5$ & $\o x_5$ & $y_2$\\
\hline
%\hline
 & $a_9$ & $x_5,x_6$ & $y_2,y_3$\\
 %\hline
\multicolumn{1}{|c|}{\raisebox{-6pt}[0pt][0pt]{$a_3$}} & $a_9$ & $x_5,\o x_6$ & $y_3,y_4$\\
%\hline
 & $a_9$ & $\o x_5,\o x_6$ & $y_3,y_4$\\
%\hline
 & $a_9$ & $\o x_5,x_6$ & $y_2$\\
\hline
%\hline
 & $a_3$ & $x_6,x_7$ & $y_2$\\
%\hline
\multicolumn{1}{|c|}{\raisebox{-6pt}[0pt][0pt]{$a_4$}} & $a_9$ & $x_6,\o x_7$ & $y_2,y_3$\\
%\hline
 & $a_9$ & $\o x_6,x_7$ & $y_3$\\
%\hline
 & $a_9$ & $\o x_6,\o x_7$ & $y_3,y_4$\\
%\hline
\hline
  & $a_2$ & $x_1,x_2$ & $y_3,y_4$ \\
%\hline
 $a_5$& $a_5$ & $x_1,\o x_2$ & $y_4,y_5$ \\
%\hline
 & $a_5$ & $\o x_1$ & $y_3,y_5$ \\
\hline
%\hline
  & $a_8$ & $x_4,x_5$ & $y_1,y_3$ \\
%\hline
 $a_6$& $a_8$ & $x_4,\o x_5$ & $y_2$ \\
%\hline
 & $a_7$ & $\o x_4$ & $y_1,y_3$ \\
\hline
%\hline
  & $a_8$ & $x_2,x_3$ & $y_2$ \\
%\hline
$a_7$ & $a_8$ & $\o x_2,x_3$ & $y_1,y_3$ \\
%\hline
 & $a_7$ & $\o x_3$ & $y_3,y_4$ \\
\hline
%\hline
  & $a_8$ & $x_3,x_4$ & $y_2$ \\
%\hline
\multicolumn{1}{|c|}{\raisebox{-6pt}[0pt][0pt]{$a_8$}} & $a_6$ & $x_3,\o x_4$ & $y_1,y_3$ \\
%\hline
 & $a_7$ & $\o x_3,x_5$ & $y_1,y_3$ \\
%\hline
 & $a_8$ & $\o x_3,\o x_5$ & $-$ \\
%\hline
\hline
  & $a_3$ & $x_1,x_3$ & $y_3$ \\
%\hline
 $a_9$& $a_4$ & $x_1,\o x_3$ & $y_3,y_5$ \\
%\hline
 & $a_9$ & $\o x_1$ & $y_3,y_4$ \\
\hline
\end{tabular}
%\end{center}
\end{minipage}
\end{center}
\end{table*}



\begin{multicols}{2}


\noindent
ли при том же начальном условии приходит сигнал $\o x_3$,
то рабочим становится подавтомат~$c_3$ при со\-сто\-янии~$a_9$ (см.\  табл.~\ref{t4pf}),
а подавтомат~$c_1$
переходит в режим тестирования при со\-сто\-янии~$a_5$
(см.\ табл.~\ref{t5pf}).
Наконец, если тестируются подавтоматы~$c_1$ и~$c_2$, находящиеся в любых
(возможных для них) состояниях, рабочий подавтомат~$c_3$ находится в
со\-сто\-янии~$a_4$ и приходит сигнал $\o x_6,x_7$, то подавтомат~$c_2$
становится рабочим (при со\-сто\-янии~$a_7$, см.\ табл.~\ref{t4pf}), подавтомат~$c_3$
начинает тестироваться (из со\-сто\-яния~$a_9$, см.\ табл.~\ref{t5pf}), а подавтомат~$c_1$
переходит в со\-сто\-яние~$a_5$ (см.\ табл.~\ref{t5pf}).

Отметим, что в данном примере при переходе подавтомата из рабочего
состояния в режим тестирования его тестирование начинается из одного и
того же состояния.
Более того, в это же состояние переходит и тестируемый подавтомат,
если меняются ролями два других.
Так, подавтомат~$c_1$ при всех таких изменениях переходит в
со\-сто\-яние~$a_5$, подавтомат~$c_2$~--- в со\-сто\-яние~$a_8$ и подавтомат~$c_3$~--- в~со\-сто\-яние~$a_9$.


%\rightline{ $\blacktriangleleft$}

%%% 3.3
\subsection{Учет неисправности в модели декомпозиции}

Обратимся теперь к автомату с некоторой неисправностью~$f$.
Он также может быть декомпозирован на три подавтомата с теми же самыми
множествами состояний.
Однако переходы и выходные векторы как исходного $f$-неисправного автомата,
так и его подавтоматов, будут, вообще говоря, отличаться от переходов
и выходных векторов исправного автомата.

Контроль правильности функционирования (например, с использованием механизма
самопроверки~\cite{1}) проводится как на тестируемом, так и на работающем
подавтоматах.
Если хотя бы по одному из выходов зафиксирована ошибка, работа системы
останавливается (например, для проведения самовосстановления).

Поскольку порядок тестирования рабочих под\-ав\-то\-ма\-тов (фактически
совпадающий с тестированием всего неисправного автомата) пол\-ностью
определяется моделью произведения автоматов пункта~2.3, кратко остановимся
на процедуре тес\-ти\-ро\-ва\-ния неисправного подавтомата.

Пусть в некоторый момент $k$-й неисправный подавтомат, находившийся в
рабочем режиме, из со\-сто\-яния~$a_j$ при входном сигнале $\v x$
переходит в режим тестирования, которое начинается из со\-сто\-яния~$a_j$.
Поставим ему в соответствие $k$-й исправный подавтомат в режиме тестирования
с со\-сто\-янием~$a_j$.
Теперь, воспользовавшись моделью произведения автоматов пункта~2.3, будем
наблюдать за выходами $\v y$ и $\v y^f$ исправного и неисправного
$k$-х подавтоматов.
Наблюдение проводится до того момента, когда впервые произойдет
несовпадение выходных сигналов или неисправный автомат снова станет
рабочим.

В ряде случаев для снижения латентности оказывается более эффективным
проводить контроль подавтоматов по схеме, когда в каждый момент времени
неработающие подавтоматы тестируются (независимо друг от друга и от
работающего под\-ав\-то\-ма\-та) на сигналах от некоторого встроенного
генератора (для простоты можно считать, что данный генератор находится
в управляющем блоке и передает сигналы для тестирования в выходных
векторах $\v p_i$,\ $i=\o{1,3}$, см.\ рис.~5).
Именно для такой схемы декомпозиции в следующем разделе будет построена
математическая модель латентности.

Данный подход может оказаться эффективным, если, например, проектируемое
устройство содержит встроенный генератор для других целей~[1,~9].
Заметим, что использование независимого тестирования во многих случаях
может улучшить качество тестирования, например благодаря выбору
вероятностного распределения тестовых сигналов по критерию минимизации
латентности обнаружения неисправности~[4].

В заключение этого раздела отметим, что в рассмотренной модели
декомпозиции переходы каж\-до\-го подавтомата в общем случае не образуют
ЦМ, поскольку изменение состояния на очередном шаге зависит
не только от его состояния на предыду\-щем шаге, но и от значений сигналов
супервизора на данном шаге, которое, вообще говоря, зависит от состояний
не только данного, но и других двух под\-ав\-то\-ма\-тов на предыдущем шаге.
Тем не менее, как только что было показано, и в этом случае можно использовать
модель произведения исправного и неисправного автоматов пункта~2.3 и
выразить ФР латентности обнаружения неисправности в терминах матрицы
переходных вероятностей под\-ав\-то\-ма\-тов сети, но с учетом режимов
(рабочего или тестового) их функционирования.

%%% 4
\section{Алгоритм вычисления функции распределения латентности обнаружения
неисправности}

%%% 4.1
\subsection{Общие положения}

 В этом разделе предлагается математическая модель для описанной
выше декомпозиции автомата в сеть независимо тестируемых подавтоматов и
алгоритм вычисления ФР латентности обнаружения неисправности для этой модели,
базирующийся на модели произведения исправного и неисправного автоматов,
описанной в пункте~2.3.

 Пусть первый подавтомат имеет $n_1$ со\-сто\-яний, второй~--- $n_2$
со\-сто\-яний и третий~--- $n_3$ со\-сто\-яний.
Всего у автомата $n=n_1+n_2+n_3$ со\-сто\-яний.
Сделаем следующие дополнительные предположения:
\begin{itemize}
\item
 в момент любого выхода каждого подавтомата из рабочего режима его
тестирование начинается из одного и того же состояния.
Более того, в этот момент так же заново из одного и того же состояния
начинается тестирование того под\-ав\-то\-ма\-та, который продолжает тестироваться;
\item
 тестирование подавтоматов производится независимыми сигналами,
т.\,е.\ тестовые сигналы каждого подавтомата статистически независимы и от
информационных сигналов (входных сигналов исходного декомпозируемого
автомата), и от тестовых сигналов второго тес\-ти\-ру\-емо\-го подавтомата;
\item
 в подавтомате-супервизоре, управляющем переключением режимов
подавтоматов, неисправности не возникают.
\end{itemize}

Сделанные предположения позволяют в качестве математической модели
процесса функционирования неисправного декомпозированного автомата с
тестированием подавтоматов рассмотреть ЦМ, в которой зависимость
ФР латентности обнаружения неисправности от дополнительного тестирования
подавтоматов определяется только через ФР времени обнаружения
неисправности, которую можно вычислить заранее.

%%% 4.2
\subsection{Вычисление функции распределения времени обнаружения
неисправности при тестировании подавтомата}

Рассмотрим $k$-й, $k=\o{1,3}$, подавтомат неисправного автомата.
Предполагая, что в начальный момент он начинает тестироваться
(на основе сравнения его выходных сигналов с выходными сигналами $k$-го
исправного подавтомата), причем в этот момент информационные сигналы
прекращаются и остальные подавтоматы останавливаются, найдем ФР $F^f_k(m)$
числа шагов до проявления неисправности~$f$ в этом подавтомате в тестовом режиме.
Для этого воспользуемся моделью пункта~2.3 и рассмотрим ЦМ с множеством
состояний ${\cal Y}_k =\{(i,j),\ i,j=\o{1,n_k}\}\cup s_A$, где состояние
$y=(i,j)$ означает, что подавтомат $k$ исправного автомата
находится в со\-сто\-янии~$i$, а со\-от\-вет\-ст\-ву\-ющий подавтомат неисправного
автомата~--- в со\-сто\-янии~$j$.
Положим $N_k=n_k^2+1$.
Переход из со\-сто\-яния $y_1=(i_1,j_1)$ в со\-сто\-яние $y_2=(i_2,j_2)$
осуществляется при поступлении входного набора (сигнала), переводящего
исправный подавтомат из со\-сто\-яния~$i_1$ в со\-сто\-яние~$i_2$, а подавтомат
неисправного автомата~--- из со\-сто\-яния~$j_1$ в со\-сто\-яние~$j_2$, причем
выходные сигналы обоих подавтоматов совпадают.
Попадание в поглощающее состояние~$s_A$ ЦМ на очередном шаге при нахождении на
предыду\-щем шаге в со\-сто\-янии~$y_1=(i_1,j_2)$ происходит в том случае, когда
поступает любой сигнал, переводящий подавтомат исправного автомата из
со\-сто\-яния~$i_1$ в со\-сто\-яние~$i_2$, подавтомат автомата с неисправностью~---
из со\-сто\-яния~$j_1$ в со\-сто\-яние~$j_2$, и выходные сигналы этих подавтоматов
при таком переходе не совпадают.

 Для определения мат\-ри\-цы~$P_k$ переходных вероятностей ЦМ необходимо
знать также вероятности входных сигналов при тестировании.
Эти вероятности задаются условиями тестирования подавтомата.

Таким образом, находим мат\-ри\-цу~$P_k$ переходных вероятностей ЦМ.
Отметим, что если вероятность обнаружения неисправности при
тестировании\linebreak
$k$-го подавтомата отлична от нуля, то мат\-ри\-ца~$P_k$ является
полустохастической, т.\,е.\ хотя бы для одной строки сумма элементов
меньше единицы.

 Вектор $\v p(m)=(p_1(m),\ldots,p_{N_k}(m))$ вероятностей состояний ЦМ
после $m$-го шага определяется по формуле
$$
\v p(m) = \v p(0) P^m_k\,,
$$
а начальное условие~$\v p(0)$ определяется со\-сто\-яни\-ем~$i_0$, с которого
начинается тестирование подавтомата, и имеет вид
$$
p_{(i,j)}(0)
=
\begin{cases}
1\,, & i=j=i_0\,, \\
0\,, & \text{иначе}\,.
\end{cases}
$$
Поэтому

\vspace*{-2pt}

\noindent
$$
F_k(m)=\v p(m) \v 1_e = \v p(0) P_k^{m} \v 1_e\,.
$$


\vspace*{-9pt}

%%% 4.3
\subsection{Вычисление функции распределения латентности обнаружения
неисправности декомпозированного автомата}

\vspace*{-2pt}

 Построим обрывающуюся ЦМ, опи\-сы\-ва\-ющую совместное функционирование до
момента обнаружения неисправности (момента обрыва ЦМ) как исправного
автомата, так и $f$-неисправного автомата и его декомпозиции на три подавтомата.

Свяжем с работой декомпозированного автомата функционирующий в дискретном
времени случайный процесс $\n(t)=(\n_1(t),\n_2(t),\n_2(t),\n_4(t))$, где
$\n_2(t)$~--- номер состояния, в которое переходит на такте $t$ исправный
автомат,
$\n_1(t)$ и $\n_3(t)$~--- номера работающего после этого перехода
подавтомата неисправного автомата и его состояния,
$\n_4(t)$~--- число тактов, прошедших с момента последней передачи
управления от одного подавтомата неисправного автомата к другому.
Процесс обрывается в момент обнаружения неисправности.

Учитывая теперь предположения пункта~4.1, получаем, что случайный процесс
$\n(t)$ является ЦМ с множеством состояний
${\cal X}={\cal X}_1\cup{\cal X}_2\cup{\cal X}_3$,
где

\noindent

\vspace*{-10pt}

\begin{multline*}
{\cal X}_k
=\left \{(i,j,m)\,,\ \  i=\o{1,n}\,,\   j=\o{1,n_k}\,,\  m\ge0\right \}\,,\\
\ \ \ \ \ \ \ \ \ \ \ \ \ \ k=\o{1,3}\,.
\end{multline*}

\vspace*{-6pt}

\noindent
При этом состояние~$(i,j,m)$ множества~${\cal X}_k$ означает, что исправный
автомат находится в со\-сто\-янии~$i$, работает $k$-й подавтомат, который
пребывает в со\-сто\-янии~$j$, и с момента последней передачи управ\-ле\-ния прошло
$m$~шагов.

 Определим матрицу~$Q$ переходных вероятностей построенной ЦМ.
Для этого приведем все возможные (ненулевые) переходы этой цепи и
вычислим вероятности таких переходов.

Из (непоглощающего) состояния
$x_1\;=$\linebreak $=\;(i_1,j_1,m)$,\ $x_1\in{\cal X}_k$,\ $k=\o{1,3}$, может произойти
переход в (непоглощающее) состояние $x_2\;=$\linebreak $=\;(i_2,j_2,m+1)$,\ $x_2\in{\cal X}_k$.
Такой переход возникает в том случае, когда передача управления от одного
подавтомата автомата с неисправностью к другому не происходит, причем
выходные сигналы рабочего подавтомата автомата с неисправностью
и исправного автомата совпадают, и, кроме того, не происходит обнаружения
неисправности ни в одном из двух тестируемых подавтоматов.
Для того чтобы определить вероятность $q(x_1,x_2)$ данного перехода, найдем
сначала условную вероятность этого перехода при условии, что проявление
неисправности в тестируемых подавтоматах не происходит.
Условная ве\-ро\-ят\-ность перехода будет совпадать с ве\-ро\-ят\-ностью
$q_{(i_1,j_1),(i_2,j_2)}$ перехода\linebreak
 исправного и неисправного автоматов из
со\-сто\-яний~$i_1$ и $j_1$ в со\-сто\-яния~$i_2$ и~$j_2$ (без тестирования
подавтоматов).
При тес\-ти\-ро\-ва\-нии под\-ав\-то\-ма\-тов вероятность $q_{(i_1,j_1),(i_2,j_2)}$
необходимо\linebreak
 дополнительно умножить на условные ве\-ро\-ят\-ности
$F_s(m+1)/F_s(m)$ и $F_t(m+1)/F_t(m)$ (где $s\ne k$,\ $t\ne k$ и $s\ne t$)
того, что неисправность не будет обнаружена при тестировании нерабочих
под\-ав\-то\-ма\-тов на $(m+1)$-м шаге при условии, что она не была обнаружена на
$m$-м шаге (напомним, что функции $F_k(m)$,\ $k=\o{1,3}$, были определены
в предыду\-щем пункте).
Таким образом, элемент~$q(x_1,x_2)$ мат\-ри\-цы~$Q_{kk}$ вероятностей переходов
из состояний подмножества~${\cal X}_k$ в состояния этого же подмножества
имеет вид
$$
q(x_1,x_2)
=
\fr{F_s(m+1)}{F_s(m)} \cdot \fr{F_t(m+1)}{F_t(m)}\,
q_{(i_1,j_1),(i_2,j_2)}\,.
$$

 Кроме того, из (непоглощающего) состояния
$x_1=(i_1,j_1,m)$,\ $x_1\in{\cal X}_k$,\ $k=\o{1,3}$, может
про\-изойти переход в (непоглощающее) состояние
$x_2\;=$\linebreak
$=\;i_2,j_2,0)$,\ $x_2\in{\cal X}_s$,\ $s=\o{1,3}$,\ $s\ne k$,
(в неисправном автомате происходит передача управления).
Переход происходит в том случае, когда при входном сигнале исправный автомат
переходит из состояния~$i_1$ в состояние~$x_2$, неисправный автомат из
состояния~$j_1$ $k$-го подавтомата переходит в состояние~$j_2$ $s$-го
подавтомата, выходные сигналы рабочего подавтомата автомата с
неисправностью и исправного автомата совпадают и не происходит обнаружения
неисправности ни в одном из двух тестируемых подавтоматов.
Так же, как и преж\-де, элемент $q(x_1,x_2)$ мат\-ри\-цы~$Q_{ks}$ вероятностей
переходов из состояний подмножества~${\cal X}_k$ в состояния подмножества~${\cal X}_s$
получается умножением вероятности $q_{(i_1,j_1),(i_2,j_2)}$
на условные вероятности $F_s(m+1)/F_s(m)$ и $F_t(m+1)/F_t(m)$
(где $t\ne k$ и $t\ne s$):
$$
q(x_1,x_2)
=
\fr{F_s(m+1)}{F_s(m)} \cdot \fr{F_t(m+1)}{ F_t(m)}\,
q_{(i_1,j_1),(i_2,j_2)}\,.
$$

Вероятности остальных переходов (остальные элементы матриц
$Q_{ks}$,\ $k,s=\o{1,3}$) равны нулю.

 Матрица $Q$ переходных вероятностей общей ЦМ, описывающей совместное
функционирование как виртуального, так и рабочего автоматов, представима
в виде
$$
Q
=
\begin{pmatrix}
 Q_{11} & Q_{12} & Q_{13} \\
 Q_{21} & Q_{22} & Q_{23} \\
 Q_{31} & Q_{32} & Q_{33} \\
\end{pmatrix}\,.
$$

 Обозначим через $\v q(n)=(\v q_1(n),\v q_2(n),\v q_2(n))$
вектор вероятностей состояний ЦМ после $n$-го шага.
Здесь $\v q_k(n)$,\ $k=\o{1,3}$,\ $n\ge 0$,~--- вектор, координатой
$q_{k,x}(n)=q_{k,(i,j,m)}(n)$,\ $x\in {\cal X}_k$, которого является
вероятность того, что после $n$-го шага ЦМ будет находиться в состоянии
$x=(i,j,m)\in {\cal X}_k$.
Вектор~$\v q(n)$ определяется по формуле
$$
\v q(n) = \v q(n-1) Q = \v q(0) Q^n\,.
$$
Вектор $\v q(0)$ начальных вероятностей определяется условиями пуска системы.


 Окончательно вероятность того, что неисправность проявится до момента~$n$
 (ФР латентности обнаружения неисправности декомпозированного автомата),
определяется выражением
\begin{multline*}
F_{FDL}(n)
=
{\bf Pr}(FDL\le n)
=
1 - \v q(n) \v1 ={}\\
{}=
1 - \sum\limits_{k=1}^3 \sum\limits_{(i,j,m)\in {\cal X}_k} q_{k,(i,j,m)}(n)\,.
\end{multline*}

Таким образом, удалось представить ФР латентности декомпозированного
автомата в терминах ФР латентности такого же, но не декомпозированного
автомата и ФР времен обнаружения не\-ис\-прав\-ности при тестировании подавтоматов,
на которые автомат был декомпозирован в процессе проектирования.

%%% 4.4
\subsection{Анализ возможности применения предложенной модели}

Скажем несколько слов о сфере возможного применения алгоритма
вычисления ФР ла\-тент\-ности обнаружения неисправности декомпозированного
конечного автомата на основе рассмотренной выше модели.

На первый взгляд кажется, что, поскольку обрывающаяся ЦМ, моделирующая
неисправный автомат, имеет бесконечное число состояний (время до
обнаружения неисправности тес\-ти\-ру\-емо\-го под\-ав\-то\-ма\-та, вообще говоря,
может быть сколь угодно большим), то возникают серьезные трудности
с практической реализацией алгоритма.
Однако здесь нужно учесть следующие обстоятельства.

При декомпозиции автомата на подавтоматы с небольшим числом состояний
у каждого подавтомата ФР $F^f_k(m)$ числа шагов до проявления
неисправности в $k$-м подавтомате в тестовом режиме имеет простой
аналитический вид (смесь небольшого числа геометрических
распределений или, более обще, распределений Паскаля), что
фактически сводит трудоемкость расчетов декомпозированного автомата
к трудоемкости расчетов недекомпозированного автомата.

Часто встречается такая ситуация, когда времена между передачами
управления от одного подавтомата к другому достаточно малы.
Это позволяет при расчетах общей латентности автомата использовать
небольшое число значений ФР~$F^f_k(m)$.

Наконец, во многих случаях можно существенно уменьшить размерность ЦМ
при большом числе состояний подавтоматов, воспользовавшись следующими
двумя принципами.

 Первый принцип связан с тем, что, как правило, функционирование
реального и виртуального автоматов начинается из одного и того же
непоглощающего со\-сто\-яния~$(\v i,\v i)$, а из всех состояний $(\v i,\v i)$
достижимы далеко не все со\-сто\-яния~$(\v i_1,\v i_2)$.
Не\-достижимые состояния можно исключить из рас\-смотре\-ния, решая задачу
построения дерева\linebreak достижимых вершин ориентированного графа, порожденного
переходами с ненулевыми ве\-ро\-ят\-ностя\-ми~ЦМ.

 Второй принцип основан на том, что для многих состояний $(\v i_1,\v i_2)$
ЦМ, имеющих ненулевые вероятности перехода в поглощающее состояние, вероятности
таких переходов за один шаг в точности равны~1, а значит, их можно объединить
в одно состояние, из которого переход в поглощающее состояние обязательно
происходит на первом же шаге.
Аналогично можно объединить в одно состояние все такие состояния, переход из
которых в поглощающее состояние происходит ровно на втором шаге, и~т.\,д.

\vspace*{-6pt}

%%% 5
\section{Заключение}
\vspace*{-3pt}

 Поскольку декомпозиция некоторого исходного автомата в сеть подавтоматов
широко применяется в современном проектировании сложных цифровых устройств
(не только для снижения латентности обнаружения неисправностей, но и для
минимизации потребления энергии в проек\-ти\-ру\-емых системах~--- так называемое
Low Power Design, см.~\cite{11}), а оценка латентности обнаружения неисправности
важна для многих приложений~\cite{13, 12},\linebreak
распространение методов предсказания
ФР латентности~\cite{4} на сети подавтоматов, получаемых
в результате декомпозиции конечно-автоматной\linebreak
 спецификации проектируемого
устройства, пред\-став\-ля\-ет\-ся весьма актуальной.

 Предложенный здесь подход к оценке ФР латентности основан на
расширении пространства состояний описываемой системы, что позволяет строить
ЦМ для пар подавтоматов (в общем зависимых) исправного и неисправного
устройств и выражать распределение вероятностей латентности в терминах
вероятностных характеристик подавтоматов.

 Отметим, что в~\cite{7} рассматривается иной подход к вычислению ФР
латентности времени обнаружения неисправности.
Он основан не на явном использовании произведения двух автоматов, а на
отдельном рассмотрении траекторий автомата, на которых неисправность не
проявляется, и траекторий, соответствующих наличию искаженных выходов.
При этом вероятность обнаружения неисправности на $i$-м шаге равна
вероятности того, что переходы автомата до ($i-1$)-го шага соответствовали
первому классу траекторий, а на $i$-м шаге оказались во втором классе
траекторий.
Траектории рас\-смат\-ри\-ва\-ют\-ся как марковские.
Трудность использования такого подхода состоит в том, что подобное
разбиение на траектории в общем случае не удается автоматизировать.

\vspace*{-6pt}

{\small\frenchspacing
{\baselineskip=10.6pt
\addcontentsline{toc}{section}{Литература}
\begin{thebibliography}{99}
\bibitem{1}
\Au{Lala~P.}
Self-checking and fault-tolerant digital design.~--- Morgan Kaufmann
Publ., 2000.

\bibitem{2}
\Au{Baranov~S.}
Logical synthesis of digital systems.~---
Tallin: TTU Publs., 2008.

\bibitem{3}
\Au{Baumann~R.}
Soft errors in advanced computer systems~//
IEEE Design and Test., 2005 (May--June). P.~258--266.

\bibitem{13}%4
\Au{Soh B.\,C., Dillon~T.\,S.}
Incorporation of multiple errors in reliability modeling of fault-tolerant
systems~//
IEEE Fault Tolerant Systems, Pacific Rim International
Symposium Proceedings, 1991. P.~148--153.

\bibitem{12}%5
\Au{Hagbae K., Kang~G.~S., Chuck~R.}
On reconfiguration latency in fault-tolerant systems~//
IEEE Aerospace Applications Conference Proceedings, 1995. P.~287--301.


\bibitem{4}%6
\Au{Shedletsky~J., McCluskey~E.}
The error latency of fault in a sequential digital circuit~//
IEEE Transaction on Computers, 1976. Vol.~25. No.\,6. P.~655--659.

\bibitem{7}
\Au{Goot~R., Levin~I., Ostanin~S.}
Fault latencies of concurrent checking FSMs~// Euromicro
Symposium on Digital System Design (DSD'02) Proceedings, 2002.

\bibitem{8}
\Au{Карибский~В.\,В., Пархоменко~П.\,П., Согомонян~Е.\,С., Халчев~В.\,Ф.}
Основы технической диагностики.~--- М.: Энергоиздат, 1976.

\bibitem{9}
\Au{Bennets~R.\,G.}
Design of testable logic circuits.~--- Addison-Wesley Publ.\ Co., 1984.

\bibitem{5}
\Au{Frenkel~S., Pechinkin~A., Chaplygin~V., Levin~I.}
A mathematical tool for support of fault-tolerant embedded systems design~//
ERCIM/DECOS Dependable Smart Systems: Research, Industrial Applications,
Standardization, Certification and Education.
Workshop on Dependable Embedded Systems. Luebeck, Germany, 2007.

\bibitem{6}
\Au{Levin~I., Abramov~B., Ostrovsky~B.}
Reduction of fault latency in sequential circuits by using decomposition~//
22nd IEEE International Symposium on Defect and Fault Tolerance in VLSI
Systems, 2007. P.~261--272.


\label{end\stat}

\bibitem{11}
\Au{Sudnitson~A.}
Computational kernel extraction for synthesis of power-managed
sequential components~// IEEE 9th Conference (International) on Electronics,
Circuits and Systems (ICECS 2002) Proceedings. Dubrovnik, Croatia, 2002.
P.~749--752.
 \end{thebibliography}
}
}
\end{multicols}