\documentclass[10pt]{book}
\usepackage[utf8]{inputenc}

\usepackage{latexsym,amssymb,amsfonts,amsmath,indentfirst,shapepar,%fleqn,%
picinpar,shadow,floatflt,enumerate,multicol,ipi}

\input{epsf}

%\nofiles

\usepackage{acad}
\usepackage{courier}
\usepackage{decor}
\usepackage{newton}
\usepackage{pragmatica}
\usepackage{zapfchan}
\usepackage{petrotex}
\usepackage{bm}                     % полужирные греческие буквы
\usepackage{upgreek}                % прямые греческие буквы
%\usepackage{verbatim}

\renewcommand{\bottomfraction}{0.99}
\renewcommand{\topfraction}{0.99}
\renewcommand{\textfraction}{0.01}

%NEW COMMANDS
\renewcommand{\r}{{\rm I\hspace{-0.7mm}\rm R}}

\newcommand{\il}[2]{\int\limits_{#1}^{#2}}%интеграл с пределами #1 и #2

%\pagestyle{myheadings}

\setlength{\textwidth}{167mm}      % 122mm
\setlength{\textheight}{658pt}
%\setlength{\textheight}{635.6pt}
\setlength{\columnsep}{4.5mm}

\setcounter{secnumdepth}{4}

%\addtolength{\headheight}{2pt}
%\addtolength{\headsep}{-2mm}

%\addtolength{\topmargin}{-20mm}  % for printing


\hoffset=-30mm  % From Yap
%\hoffset=-20mm  % From Acrobat

%\voffset=0mm % From Yap
%\voffset=-15mm   % From Acrobat

\addtolength{\evensidemargin}{-9.5mm} % for printing
\addtolength{\oddsidemargin}{9.5mm}  % for printing

\renewcommand\labelitemi{$\bullet$}

\begin{document}
\Rus

\nwt
%\ptb

%\vspace*{-12pt}

\begin{center}

{\prgsh\LARGE
ОБЪЯВЛЕНИЯ О КОНФЕРЕНЦИЯХ}

\end{center}
%\hrule

\vspace*{6pt}
%\begin{center}
%\mbox{%
%\epsfxsize=167mm
%\epsfbox{recl-dia-b.eps}
%}
%\end{center}
\begin{flushright}
{\prg http://www.tvp.ru/conferen/20091001\_1.htm}
\end{flushright}

%\vspace*{6pt}

\begin{center}\prg
\Large
X Всероссийский симпозиум\\ по прикладной и промышленной математике\\
(осенняя открытая сессия)

\end{center}

\begin{center}\prg
%Очередная 15-я международная конференция <<Диалог>> состоится c
1--8 октября 2009~г., Сочи--Дагомыс
\end{center}


%Всероссийские симпозиумы по прикладной и промышленной математике проводятся ежегодно с 2000~года. 

\smallskip

{\centering Симпозиум проводится по следующим направлениям:

\begin{multicols}{2}
\begin{itemize}
\item Безопасность компьютерных систем\item 
Геометрическая нелинейная оптика\item 
Инженерно-технологическая математика\item 
Информационные технологии и задачи связи\item 
Квантовые вычисления\item 
Математические методы биологических и экологических систем\item 
Математические модели в жидких кристаллах\item 
Математические методы в педагогических исследованиях\item 
Математические модели в теории оболочек\item 
Математическое моделирование процессов рассеяния примесей в турбулентной атмосфере\item 
Математическое моделирование свойств материалов и конструкций\item 
Математическое образование\item  
Медицина\item  
Метод конечных элементов\item 
Механика жидкости и газа\item  
Механика природных процессов\item  
Механика разрушения\item 
Модели горения и взрыва\item  
Нанотехнологии: математические модели\item 
Науки о Земле, геология, геофизика\item 
 Неклассические задачи для уравнений математической физики
  \item Нелинейное моделирование и управление\item 
  Обработка данных, анализ и обработка изображений\item 
  Прикладная вероятность и статистика\item 
  Прикладная геометрия\item 
  Обработка и распознавание образов\item 
  Прикладная дискретная математика\item 
  Обработка и защита информации\item 
  Системы поддержки принятия решений для регионального управления\item 
  Социология\item 
  Психология\item 
  Специальные функции и ортогональные многочлены\item 
  Супер-, нейро-, биокомпьютеры\item
  Эволюционные и мембранные вычисления\item
  Теория управления и системные исследования\item
Процессы принятия решений\item 
Тепло- и массоперенос\item 
Физика океана и атмосферы\item 
Фракталы и масштабный эффект\item 
Экономика, страховая и финансовая матема\-тика\item 
Энергетика и передача энергии\item
Юриспруденция\item 
Криминалистика
\end{itemize}
\end{multicols}
}

\smallskip

{\centering В программу симпозиума входят также:
\begin{itemize}
\item учредительное собрание общероссийского <<Общества прикладной и промышленной математики>> (ОППМ);\\[-13pt] 
\item минисимпозиумы по предложенным позднее темам; \\[-13pt]
\item круглые столы по продвижению современных фундаментальных математических методов в различные сферы науки и технологий, по междисциплинарному сотрудничеству; \\[-13pt]
\item выставка-продажа научных изданий, демонстрация программного обеспечения.
\end{itemize}

{\large Организаторы симпозиума: }
%\noindent
%\begin{tabular}{p{7mm}p{420pt}}
%\hspace*{7mm}
\begin{itemize}
\item Управление по науке и образованию г.~Сочи
\item Сочинский государственный университет туризма и курортного дела
\item Академия криптографии Российской Федерации
\item Институт проблем информатики Российской академии наук
\item Редакции журналов <<Информатика и её применения>>, <<Обозрение прикладной и промышленной математики>> (ОПиПМ), <<Прикладная информатика>>, <<Теория вероятностей и ее применения>> (Научное издательство <<ТВП>>) 
\item Экономический факультет Санкт-Петербургского государственного университета (ЭФ СПбГУ)
\end{itemize}
%\end{tabular}

%\vspace*{6pt}
\bigskip

{\large Организационный комитет:}
\vspace*{6pt}

Академик Ю.\,В.~Прохоров (председатель)
\vspace*{6pt}

\tabcolsep=15pt
\begin{tabular}{p{60mm}p{60mm}}
академик В.\,А.~Бабешко\newline
академик А.\,А.~Боровков\newline 
академик С.\,С.~Григорян\newline 
академик И.\,А.~Ибрагимов\newline 
академик В.\,И.~Колесников\newline 
академик А.\,Б.~Куржанский\newline 
академик В.\,П.~Маслов\newline
академик И.\,А.~Соколов\newline
академик В.\,П.~Шорин\newline 
член-корр.\ РАН А.\,Б.~Жижченко\newline
член-корр.\ РАН С.\,В.~Кисляков\newline 
член-корр.\ РАН В.\,В.~Русанов\newline 
член-корр.\ РАН Б.\,А.~Севастьянов 
&
член-корр.\ РАН В.\,А.~Сойфер\newline
член-корр.\ РАН А.\,Н.~Ширяев\newline
И.\,П.~Бойко\newline 
А.\,А.~Емельянов\newline
А.\,М.~Зубков\newline 
В.\,Ф.~Колчин\newline
 А.\,С.~Максимов\newline
О.\,Н.~Медведева\newline 
Г.\,М.~Романова\newline 
В.\,В.~Сапожников\newline 
А.\,Р.~Симонян\newline 
В.\,И.~Хохлов\newline
С.\,Я.~Шоргин
\end{tabular}


}
\end{document}


%\noindent
%\begin{tabular}{p{7mm}p{420pt}}
%\hspace*{7mm}
%&\begin{itemize}
%\item Филологический факультет МГУ
%\item Институт лингвистики РГГУ
%\item Институт проблем информатики РАН
%\item Институт проблем передачи информации РАН
%\item Российский НИИ искусственного интеллекта
%\item Яндекс (Москва)
%\end{itemize}
%\end{tabular}

%Конференция проводится по следующим направлениям, сочетающим теоретические
%исследования и приложения:
%\vspace*{-6pt}

%\noindent
%\begin{tabular}{p{7mm}p{420pt}}
%\hspace*{7mm}&
%\begin{itemize}
%\item Лингвистическая семантика и семантический анализ
%\item Формальные модели языка и их применение
%\item Теоретическая и компьютерная лексикография
%\item Создание и применение компьютерных лексических ресурсов
%\item Корпусная лингвистика. Создание, применение, оценка корпусов
%\item Интернет как лингвистический ресурс. Лингвистические технологии в
%интернете
%\item Извлечение знаний из текстов
%\item Модели общения. Коммуникация, диалог и речевой акт
%\item Анализ и синтез речи
%\item Компьютерный анализ документов: реферирование, классификация,
%поиск
%\item Машинный перевод
%\item Вопросно-ответные системы
%\end{itemize}
%\end{tabular}

%Сайт конференции: {\prg http://www.dialog-21.ru/ }

%\newpage

\begin{center}

{\prgsh\LARGE
ОБЪЯВЛЕНИЯ О КОНФЕРЕНЦИЯХ}

\end{center}
%\hrule

\vspace*{12pt}

%\hrule
\begin{center}
\mbox{%
\epsfxsize=167mm
\epsfbox{recl-RL-b.eps}
}
\end{center}
\begin{flushright}
{\prg http://rcdl2009.krc.karelia.ru/}
\end{flushright}

%\vspace*{6pt}

{\begin{center}\prg
{\Large
XI Всероссийская научная конференция RCDL 2009\\
Электронные библиотеки: перспективные методы и технологии,\\ электронные
коллекции}
\end{center}}

{\begin{center}\prg
17--21 сентября 2009 г.,
Петрозаводск, Россия
\end{center}}

\vspace*{12pt}

Электронные библиотеки~--- область исследований и разработок, направленных 
на развитие теории и практики обработки, распространения, хранения, поиска 
и анализа цифровых данных различной природы. 

Основная цель серии конференций RCDL заключается в том, чтобы способствовать 
формированию сообщества специалистов России, ведущих исследования и разработки 
в области электронных библиотек. Конференция также способствует изучению 
зарубежного опыта, развитию международного сотрудничества в области электронных 
библиотек. 

Значительное внимание в тематике RCDL уделяется практическим проектам, в 
которых решаются сложные задачи. RCDL придает большое значение исследованиям 
в области создания крупномасштабных электронных библиотек (Very Large Digital 
Libraries~--- VLDL), включая использование сервисных архитектур, архитектур, 
основанных на грид, и обеспечение их качества, развитие техники 
интероперабельности и устойчивости VLDL, а также разработке организационных 
моделей крупных электронных библиотек. Особый интерес представляет применение 
современных научных подходов в контексте высоких нагрузок: сотни тысяч 
пользователей, десятки гигабайт данных, терабайты трафика.


%RCDL'2009~---  одиннадцатая конференция в серии Всероссийских научных
%конференций <<Электронные библиотеки: перспективные методы и технологии,
%электронные коллекции>>.
За 10 лет проведения RCDL в работе конференции приняло участие несколько
сотен ученых из ведущих российских и зарубежных научных центров Австрии,
Германии, Греции, Италии, Новой Зеландии, США, Украины и других стран.

Традиционно совместно с RCDL проводятся Всероссийские научные семинары по оценке
методов текстового поиска РОМИП. В 2009 году планируется совмещение с RCDL
Семинара РОМИП и\linebreak
 Третьей Российской летней школы по информационному поиску RuSSIR'2009,
во время которой ведущие российские и зарубежные ученые прочитают
обзорные лекции по актуальным проблемам развития поиска цифровых данных для
решения фундаментальных и прикладных задач.
\vspace*{6pt}

Организаторы конференции RCDL'2009:
\vspace*{-6pt}

\noindent
\begin{tabular}{p{7mm}p{420pt}}
\hspace*{7mm}&
\begin{itemize}
\item Российская академия наук
\item Российский фонд фундаментальных исследований
\item Карельский научный центр РАН
\item Институт прикладных математических исследований
\item Петрозаводский государственный университет
\item Институт проблем информатики РАН
\item Московская секция АСМ SIGMOD
\end{itemize}
\end{tabular}

%Сайт конференции: {\prg http://rcdl2009.krc.karelia.ru/}

\end{document}