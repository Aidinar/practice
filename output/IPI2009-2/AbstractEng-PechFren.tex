                A Probabilistic Analysis of Fault Detection Latency in a Network of FSMs

Keywords: testing; Finite State Machine; Markov Chains 

    This paper suggests an approach to the computation of time probability distribution 
function (PDF) of Fault Detection Latency (FDL) in case, when a system is modeled as a 
combination of interacting finite state machines (FSM) under random inputs, where the 
interactions deal with switching each of the FSMs from a working mode to a testing one. FDL is 
a period of fault detection after it occurs in certain inner states. Traditionally, the FDL of an 
FSM is modeled as the time (numbers of its state transition steps) to absorption for a Markov 
Chan with the state space generated by a product of fault-free and faulty (i.e. corrupted by a 
fault) FSMs. The principal problem of using this model for the networks of sub-FSMs is that 
random transitions of the product of the fault-free and faulty networked automata even under 
independent inputs random vectors are not Markovian ones. Thanks to an extension of the 
transition space of the networked FSMs by some additional states corresponding to the number 
of steps between transitions to the modes mentioned above for each of sub-FSMs, we extend this 
model to the case of an FSM decomposed (in a designing process) into a number of components 
of sub-FSMs. We show a way to compute the FDL PDF in terms of FDL PDF of initial FSM 
(that is not decomposed), and the FSMs of corresponding sub-FSMs.   