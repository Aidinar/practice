
\def\stat{ushmaev}


\def\tit{АДАПТАЦИЯ БИОМЕТРИЧЕСКОЙ СИСТЕМЫ К ИСКАЖАЮЩИМ 
ФАКТОРАМ НА ПРИМЕРЕ ДАКТИЛОСКОПИЧЕСКОЙ 
ИДЕНТИФИКАЦИИ$^*$}
\def\titkol{Адаптация биометрической системы к искажающим 
факторам на примере дактилоскопической 
идентификации} 

\def\autkol{О.\,С.~Ушмаев}
\def\aut{О.\,С.~Ушмаев$^1$}

\titel{\tit}{\aut}{\autkol}{\titkol}

{\renewcommand{\thefootnote}{\fnsymbol{footnote}}\footnotetext[1]
{Работа поддержана грантами РФФИ (проект 07-07-00031) и Программой ОНИТ РАН 
<<Информационные технологии и методы анализа сложных систем>>. Работа выполнена в рамках НОЦ 
ИПИ РАН~--- ВМК МГУ <<Биометрическая информатика>>.}}

\renewcommand{\thefootnote}{\arabic{footnote}}
\footnotetext[1]{Институт проблем
информатики Российской академии наук, oushmaev@ipiran.ru}

%\vspace*{12pt}

\Abst{Рассмотрены проблемы учета искажающих факторов при биометрической 
идентификации. На примере дактилоскопической идентификации проанализировано влияние 
искажений на качество идентификации. Предложены методы учета искажений на основе 
нормальной аппроксимации факторов. Проведенные эксперименты показали эффективность 
предложенного подхода к учету искажений. На основе разработанных методов предложена 
технология адаптации биометрической системы к ис\-ка\-жа\-ющим факторам.}

%\vspace*{4pt}

\KW{биометрическая идентификация; автоматическая дактилоскопическая идентификация; 
искажающие факторы}

 \vskip 36pt plus 9pt minus 6pt

      \thispagestyle{headings}

      \begin{multicols}{2}

      \label{st\stat}
      

\section{Введение}

     На сегодняшний день технологии биометрической идентификации получают 
широкое распространение в различных системах гражданской идентификации. Это и 
паспортно-ви\-зо\-вые документы нового поколения, и идентификация получателей 
социальных услуг, и идентификационные карты работников отдельных категорий 
(государственные служащие, транспортная безопасность) \mbox{и~пр.~[1--5]}. 
{ %\looseness=1

}
     
     Длительное накопление биометрических массивов приводит к ситуации, когда 
одновременно обрабатывается информация, собранная с использованием разных 
технических средств, в разное время и при разных условиях. Постепенное отклонение 
характеристик биометрической информации от эталонных значений приводит к 
отклонению показателей качества биометрической идентификации от проектных 
значений. 
     
     В качестве примеров постепенной деградации биометрической идентификации 
можно привести следующие:
     \begin{itemize}
\item возрастные изменения лицевой биометрии ведут к тому, что качество 
идентификации при увеличении временного лага между регистрацией в системе и 
идентификацией падает;
\item сезонные колебания качества дактилоскопической идентификации: зима и лето 
характеризуются принципиально различными условиями получения отпечатков 
пальцев;
\item изменение в качестве идентификации отпечатков пальцев, полученных из разных 
источников: бумажных дактокарт, <<живым>> сканированием, следов отпечатков 
и~т.\,д. 
\end{itemize}

     В~[6--9] приведены примеры деградации биометрической идентификации для 
широкого круга технологий и систем.
     
     Производители биометрических технологий\linebreak постоянно совершенствуют 
алгоритмы распознавания с целью повышения устойчивости к иска\-жа\-ющим факторам. 
Однако многие проблемы устранить невозможно. 
     
     В этой связи актуальной проблемой проектирования и эксплуатации 
биометрических технологий является учет влияния искажающих факторов. Проблема 
особенно актуальна в случаях, когда искажения невозможно подавить штатными 
средствами биометрической системы. 
    
     Далее статья организована следующим образом. Раздел~2 посвящен оценке 
влияния искажающих\linebreak
факторов при дактилоскопической идентификации. В разд.~3 
изложен подход к адаптации био\-мет\-ри\-ческой системы к искажениям. В разд.~4 
\begin{figure*}  %fig1
\vspace*{1pt}
\begin{center}
\vspace*{6pt}
\mbox{%
\epsfxsize=165.353mm
\epsfbox{ush-1.eps}
}
\end{center}
\vspace*{-9pt}
\Caption{Процесс создания шаблона
\label{f1ush}}
\vspace*{6pt}
\end{figure*}
приведены результаты экспериментов. В разд.~5\linebreak рассмотрены отдельные вопросы 
практического применения изложенного подхода. Основные выводы представлены в 
заключении.
%\pagebreak

    
\section{Влияние искажающих факторов}

     В своей методологической основе большинство биометрических технологий 
являются технологиями распознавания образов. При регистрации человека 
биометрическая информация оцифровывается в количественные данные, пригодные 
для дальнейшей идентификации (рис.~\ref{f1ush})~\cite{7ush}.
  
     На этапе идентификации шаблоны сравниваются. Далее преимущественно 
используются пороговые методы принятия решения. Если результат сравнения 
биометрических шаблонов выше некоторого порогового значения, то принимается 
гипотеза о принадлежности образцов одному человеку. В обратном случае считается, 
что образцы принадлежат разным людям. Изменяя порог, можно управлять 
соотношением ошибок 1-го и 2-го рода, FRR (False Rejection Rate)
и FAR (False Accptance Rate) (рис.~2).

     Соответственно, с точки зрения ошибок принятия решения об идентификации 
биометрическая\linebreak\vspace*{-12pt}
\begin{center} %fig2
\vspace*{12pt}
\mbox{%
\epsfxsize=80mm %.951mm
\epsfbox{ush-2.eps}
}
\end{center}
{{\figurename~2}\ \ \small{Пороговое принятие решения: \textit{1}~--- FRR, \textit{2}~--- FAR}}

\bigskip
\addtocounter{figure}{1}
    
\noindent
система характеризуется двумя распределениями: результаты 
сравнений в <<своих>> сравнениях и результаты сравнения в <<чужих>> сравнениях. 

 
     Под воздействием внешних систематических факторов распределения могут 
претерпевать определенные изменения. На рис.~3 приведен пример 
изменения распределений под влиянием эластичных деформаций отпечатков 
пальцев~\cite{9ush}, достаточно распространенного и значимого искажающего фактора 
в дактилоскопической идентификации. 
   
     Как видно из рис.~3, искажающий фактор вносит смещения и изменения 
дисперсии, при этом общий вид распределений не изменяется. Аналогичная ситуация 
наблюдается для более широкого класса биометрик и искажающих факторов. Среди 
публично доступных данных по проблеме следует выделить протоколы тестирований 
(NIST FRVT, NIST PVT, FpVTE\footnote{NIST~--- National Institute of Standards and Technology;
FRVT~--- Face Recognition Vendor Test;
PVT~--- PHIGS Validation Tests;
PHIGS~--- Programmer's Hierarchical Interactive Graphics System;
FpVTE~--- Fingerprint Vendor Technology Evaluation.})~\cite{10ush, 11ush}.

     
     В то же время незначительное на первый взгляд различие приводит к 
существенным изменениям в качестве распознавания (рис.~4) и в 
зависимостях FAR и FRR от порога (рис.~\ref{f5ush}). Такие изменения необходимо 
учитывать при проектировании и эксплуатации биометрической системы.

\begin{figure*} %fig3-4
\vspace*{1pt}
%\begin{figure*} %fig3
%\vspace*{1pt}
\begin{center}
\vspace*{1pt}
\mbox{%
\epsfxsize=166.178mm %807mm
\epsfbox{ush-3-4.eps}
}
\end{center}
\vspace*{-9pt}
\begin{minipage}[t]{81mm}
\Caption{Изменение распределений: \textit{1}~--- без компенсации деформаций, \textit{2}~--- с 
компенсацией деформаций
\label{f3ush}}
\end{minipage}
\hfill
\begin{minipage}[t]{81mm}
%\begin{figure} %fig4
\Caption{Изменение качества распознавания: \textit{1}~---  с учетом деформаций, \textit{2}~--- 
без учета деформаций
\label{f4ush}}
\end{minipage}
\end{figure*}

\begin{figure*} %fig5
\vspace*{1pt}
\begin{center}
\vspace*{1pt}
\mbox{%
\epsfxsize=165.791mm
\epsfbox{ush-5.eps}
}
\end{center}
\vspace*{-9pt}
\Caption{Изменение ошибок распознавания в зависимости от порога: \textit{1}~--- без 
компенсации деформаций, \textit{2}~--- с компенсацией деформаций (исходная шкала)
\label{f5ush}}
%\end{figure*}
% \begin{figure*} %fig6
     \vspace*{18pt}
\begin{center}
\vspace*{1pt}
\mbox{%
\epsfxsize=151.739mm
\epsfbox{ush-6.eps}
}
\end{center}
\vspace*{-9pt}
     \Caption{Процесс биометрической идентификации
\label{f6ush}}
\end{figure*}

     
     Необходимость учета изменения ошибок связана с типичной логикой принятия 
решения о био\-мет\-ри\-ческой идентификации. Согласно bioAPI (Biometric Application Programming Interface), основному 
био\-мет\-ри\-че\-скому стандарту на API био\-мет\-ри\-ческих библиотек, при идентификации 
предъявляемый образец последовательно сравнивается с хранимыми образцами. 
Полученный результат~--- мера сходства био\-мет\-ри\-ческих образцов~--- приводится к 
единой шкале по таблице ошибок 1-го и~2-го рода (рис.~\ref{f6ush}). Ответ в таком 
случае может быть сформулирован как точные значения ошибок идентификации.
     
     Под влиянием сильных внешних искажений однородность биометрической базы 
нарушается, что приводит к невозможности адекватно оценить ожидаемые ошибки 
\begin{table*}[b]\small %tabl1
\begin{center}
\Caption{Статистики влияния деформаций отпечатков пальцев
\label{t1ush}}
\vspace*{2ex}

\begin{tabular}{|l|c|c|c|c|c|c|c|c|}
\hline
\multicolumn{1}{|c|}{Фактор}&$\mu$&$\sigma$&$\gamma_3$&$\gamma_4$&$\gamma_5$&$\gamma_6$&$\gamma_7$
&$\gamma_8$\\
\hline
C компенсацией деформации (свои)&280&143&0,41&2,73&3,33&14,41&35,71&147,52\\
Без компенсации деформации  
(свои)&265&156&0,84&3,12&4,87&18,36&49,45&208,34\\
C компенсацией деформации 
(чужие)&\hphantom{9}48&\hphantom{9}24&1,07&4,72&15,80\hphantom{9}&77,22&430,55\hphantom{9}&2869,27\hphantom{9}\\
Без компенсации деформации 
(чужие)&\hphantom{9}46&\hphantom{9}24&1,14&4,96&16,35\hphantom{9}&83,01&470,90\hphantom{9}&3027,34\hphantom{9}\\
\hline
\end{tabular}
\end{center}
\end{table*}
распознавания. Как видно из рис.~\ref{f5ush}, различия в ожидаемых уровнях ошибок 
могут быть очень значительными.
     
     
\vspace*{6pt}
\section{Оценка искажающих факторов}
\vspace*{3pt}

     Для количественной оценки влияния искажающего фактора изучим моменты 
распределений результатов сравнения: математическое ожидание, дисперсию, моменты 
больших порядков. Чтобы изолировать влияние математического ожидания и 
дисперсии, заменим моменты больших порядков на производные статистики
     \begin{equation}
     \gamma_k = \mathbf{E}\left [ \left ( \fr{s-\mu}{\sigma}\right )^k\right ]\,,
     \label{e1ush}
     \end{equation}
     где $\mu$~--- математическое ожидание, $\sigma$~--- сред\-не\-квад\-ра\-ти\-ческое 
отклонение.

     Как видно из~(\ref{e1ush}), статистики $\gamma_k$ инварианты относительно 
линейной замены аргументов.
     
     Данные по изменению статистик искажающего фактора приведены в 
табл.~\ref{t1ush}.


   
     Анализ данных табл.~\ref{t1ush} показывает, что при грубой оценке искажающего 
фактора можно ограничиться моментами первых двух порядков. Слабая вариация 
статистик~(\ref{e1ush}) высших порядков подтверждает предположение о том, что 
характер распределения не претерпевает принципиальных изменений. 
     
     Обозначим первые моменты распределений в своих и чужих сравнениях через 
($\mu_g,\sigma_g^2$) и ($\mu_i,\sigma_i^2$). Смещения моментов, сопряженные с 
воздействием искажающего фактора, обозначим через $(\Delta m_g, \Delta s_g^2$) и 
($\Delta m_i, \Delta s_i^2$) соответственно.
     
     Такая оценка искажений на уровне смещений моментов первых двух порядков 
имеет явное практическое преимущество. Пусть имеется эталонная таблица 
соответствий ошибок 1-го рода $\mathrm{FRR}(x)$\linebreak и~2-го рода $\mathrm{FAR} (x)$ в 
зависимости от порога~$x$ принятия решения. Тогда скорректированные с учетом 
искажающего фактора ошибки распознавания определяются следующим образом:

\noindent
     \begin{align*}
     \mathrm{FRR}^c (x) & = \mathrm{FRR}\left ( \alpha_{\mathrm{FRR}} x 
+\beta_{\mathrm{FRR}}\right )\,;\\
     \mathrm{FAR}^c(x) & = \mathrm{FAR}\left ( \alpha_{\mathrm{FAR}} 
x+\beta_{\mathrm{FAR}}\right )\,,
     \end{align*}
где коэффициенты $\alpha$ и $\beta$ являются коэффициентами линейного 
преобразования 
$$
\fr{x-\Delta m -\mu}{\sqrt{\sigma^2+\Delta s^2}} \rightarrow \fr{x-\mu}{\sigma}\,,
$$
которые прямо вычисляются через моменты по следующим формулам:
\begin{align*}
\alpha_{\mathrm{FAR}} & = \fr{\sigma_i}{\sqrt{\sigma_i^2+\Delta s_i^2}}\,;\\
\beta_{\mathrm{FAR}} & = \mu_i -\fr{\sigma_i}{\sqrt{\sigma_i^2+\Delta s_i^2}}\left ( 
\mu_i+\Delta m_i\right )\,;\\
\alpha_{\mathrm{FRR}} & = \fr{\sigma_g}{\sqrt{\sigma_g^2+\Delta s_g^2}}\,;\\
\beta_{\mathrm{FRR}}& = \mu_g -\fr{\sigma_i}{\sqrt{\sigma_g^2+\Delta s_g^2}}\left ( 
\mu_g+\Delta m_g\right )\,.
\end{align*}



     В условиях воздействия нескольких факторов с параметрами ($\Delta m_{jg}, 
\Delta s_{jg}^2$), ($\Delta m_{ji}, \Delta s_{ji}^2$) итоговое воздействие ($\Delta m_g, 
\Delta s_g^2$) и ($\Delta m_i, \Delta s_i^2$) получается суммой отдельных факторов:
     \begin{align*}
     \Delta m_g & = \sum\limits_j \Delta m_{jg}\,;\quad \Delta s_g^2 = \sum\limits_j \Delta 
s_{jg}^2\,;\\
     \Delta m_i & = \sum\limits_j \Delta m_{ji}\,;\quad \Delta s_i^2 = \sum\limits_j \Delta 
s_{ji}^2\,.
     \end{align*}
     
     Сдвиги ошибок $\mathrm{FRR}(x)$ и $\mathrm{FAR}(x)$ вычисляются по всей 
совокупности факторов. Формально можно ожидать, что некоторые факторы уменьшат 
дис\-пер\-сию, в таком случае величина $\Delta s^2$ окажется отрицательной. Примером 
подобного фактора является, например, уменьшение окна сканирования. Уменьшение 
дисперсии в таком случае обусловлено общим снижением информативности 
получаемых биометрических образцов.

\begin{figure*} %fig7
\vspace*{1pt}
\begin{center}
\vspace*{1pt}
\mbox{%
\epsfxsize=161.708mm
\epsfbox{ush-7.eps}
}
\end{center}
\vspace*{-9pt}
\Caption{Гистограммы распределений (слева~--- распределения в <<своих>> 
сравнениях, справа~--- распределения в <<чужих>> сравнениях): оптический~(\textit{1}) и емкостной~(\textit{2})
сканеры; с компенсацией~(\textit{3}) и без компенсации~(\textit{4}) деформаций;
контрольные~(\textit{5}) и откатанные~(\textit{6}) отпечатки
\label{f7ush}}
\end{figure*}

\begin{table*}\small %tabl2
\begin{center}
\Caption{Численные оценки моментов
\label{t2ush}}
\vspace*{2ex}

\begin{tabular}{|c|c|c|c|c|c|c|c|c|}
\hline
&&&&&&&&\\[-8pt]
\multicolumn{1}{|c|}{Эксперименты}&$\mu_g$ &$\sigma_g^2$&$\mu_i$
&$\sigma_i^2$&$\Delta m_g$&$\Delta s_g^2$&$\Delta m_i$&$\Delta s_i^2$\\
\hline
I&265&24\,366&46&580&$-16$&\hphantom{$-$}4\,039&$-2$&\hphantom{$-$}309\\
II&265&24\,366&46&580&\hphantom{$-$}15&$-3\,887$&\hphantom{$-$}2&\hphantom{$-9$}21\\
IIIa&934&168\,434\hphantom{9}&40&3362\hphantom{9}&$-126$\hphantom{9}&$-56\,192$\hphantom{9}&$-14$\hphantom{9}&$-139$\\
IIIb&902&134\,658\hphantom{9}&32&2765\hphantom{9}&$-110$\hphantom{9}&$-47\,744$\hphantom{9}&$-11$\hphantom{9}&$-560$\\
\hline
\end{tabular}
\end{center}
\end{table*}
     
\section{Эксперименты}

     В данном разделе рассмотрим факторы, влияющие на качество распознавания 
отпечатков пальцев. Выделим следующие:
     \begin{itemize}
\item шумы;
\item внешние условия (температура, влажность);
\item временной лаг между регистрацией и идентификацией;
\item деформации отпечатков пальцев;
\item характеристики типичного пользователя системы; 
\item различия в способе получения отпечатков пальцев (например, след отпечатка, 
<<живое>> сканирование и оцифровка бумажных носителей).
\end{itemize}

     Можно предположить, что перечисленные факторы по своему воздействию 
независимы. Поэтому можно проводить их оценку по отдельности.
     
     Базовой технологией распознавания пальцев выберем технологию 
AMIS (Automated Multibiometric Information System)~\cite{12ush}, в качестве основного источника биометрических данных~--- 
оптический сканер. Приведем результаты следующих экспериментов, которые могут 
быть повторены на публично доступных биометрических базах:
     \begin{enumerate}[I.]
     \item Изменение характера шумов: переход от оп\-ти\-че\-ского к емкостному.
     \item Учет деформаций: сравнение качества распознавания AMIS с качеством 
распознавания AMIS, усиленной методами устранения деформаций отпечатков 
пальцев~\cite{12ush, 13ush}.
     \item Различия в способе получения отпечатков: использование для 
идентификации откатанных отпечатков.
     \end{enumerate}
     
     Эксперимент~III проведен раздельно для левого и правого больших пальцев (IIIa 
и~IIIb). 
     
     В качестве тестовых баз использовались: FVC2002 (эксперименты~I 
и~II)~\cite{14ush} и база SD~29 (эксперименты~III)~\cite{15ush}. На рис.~\ref{f7ush} 
приведены гис\-то\-грам\-мы распределений с учетом влияния искажающих факторов.


 {\small \begin{center}
{{\tablename~3}\ \ \small{Поправочные коэффициенты}}

\vspace*{2ex}

\tabcolsep=5.5pt
\begin{tabular}{|c|c|c|c|c|}
\hline
Эксперименты&$\alpha_{\mathrm{FRR}}$ &$\beta_{\mathrm{FRR}}$ &$\alpha_{\mathrm{FAR}}$ &$\beta_{\mathrm{FAR}}$ \\
\hline
I&0,926&$\hphantom{-}34{,}402$&0,807&$10{,}460$\\
II&1,091&$-40{,}454$&0,982&$-1{,}154\hphantom{,}$\\
IIIa&1,225&$-55{,}803$&1,021&$13{,}445$\\
IIIb&1,244&$-83{,}817$&1,030&$10{,}365$\\
\hline
\end{tabular}
\end{center}
}

\bigskip
\addtocounter{table}{1}
    
     Численные результаты оценки моментов искажающих факторов и поправочных 
коэффициентов приведены в табл.~\ref{t2ush} и~3. На 
     рис.~\ref{f8ush}--\ref{f10ush} приведены графики нормальной аппроксимации 
искажающих факторов. 



В частности, из таблиц видно, что влияние ис\-кажающих факторов на 
распределения в чужих сравнениях значительно меньше влияния на распределения в 
своих сравнениях. Поэтому больший интерес представляет оценка ошибки 1-го рода с 
учетом искажающего фактора. На рис.~\ref{f12ush}--\ref{f14ush} приведены примеры 
точности оценки ошибок 1-го рода на основе нормальной аппроксимации ис\-ка\-жа\-юще\-го 
фактора. 

     Как видно из рис.~\ref{f12ush}--\ref{f14ush}, предложенный метод позволяет 
производить достаточно точное приведение шкалы ошибок к эталонному варианту. 
     
     \section{Влияние на качество идентификации}

     Основным практическим применением разработанного подхода к учету 
искажающих факторов является адаптация процедуры идентификации в 
биометрической системе при неоднородных исходных данных. 
     
     В качестве примера рассмотрим адаптацию дактилоскопической системы при 
использовании двух источников биометрических данных: откатанных и контрольных 
отпечатков. Предположим, что на этапе идентификации используются только 
контрольные отпечатки, а база данных собрана с использованием двух типов 
сканирования в равных пропорциях. В качестве данных возьмем эксперименты~IIIa 
и~IIIb. 
\pagebreak

\end{multicols}
     
%\linebreak\vspace*{-12pt}
%\pagebreak


\begin{figure*} %fig8
\vspace*{1pt}
\begin{minipage}[t]{83mm}
\begin{center}
\vspace*{1pt}
\mbox{%
\epsfxsize=82mm %.878mm
\epsfbox{ush-8.eps}
}
\end{center}
\vspace*{-9pt}
\Caption{Нормальная аппроксимация факторов~I (изменение типа сканирования): 
\textit{1}~--- оптический сканер (свои), 
\textit{2}~--- емкостной  сканер (свои), 
\textit{3}~--- оптический сканер (чужие),  
\textit{4}~--- емкостной сканер (чужие)
\label{f8ush}}
%\end{figure*}
\end{minipage}
\hfill
\begin{minipage}[t]{81mm}
%\begin{figure*} %fig9
%\vspace*{1pt}
\begin{center}
\vspace*{1pt}
\mbox{%
\epsfxsize=79.76mm
\epsfbox{ush-9.eps}
}
\end{center}
\vspace*{-9pt}
\Caption{Нормальная аппроксимация факторов~II (учет деформаций): \textit{1}~--- с 
компенсацией деформаций, \textit{2}~--- без компенсации деформаций, \textit{3}~--- чужие 
сравнения
\label{f9ush}}
\end{minipage}
%\vspace*{15pt}
\end{figure*}


\begin{figure*} %fig10
\vspace*{1pt}
\begin{center}
\vspace*{1pt}
\mbox{%
\epsfxsize=166.3mm %792mm
\epsfbox{ush-10.eps}
}
\end{center}
\vspace*{-9pt}
\Caption{Нормальная аппроксимация факторов IIIa (переход от сравнения контрольных 
отпечатков к сравнению контрольных с откатанными, левый большой палец)~(\textit{a})
и IIIb (переход от сравнения контрольных 
отпечатков к сравнению контрольных с откатанными, правый большой палец)~(\textit{б}): \textit{1}~--- 
контрольные отпечатки (свои), \textit{2}~--- 
откатанные отпечатки (свои),  \textit{3}~--- контрольные отпечатки (чужие), 
\textit{4}~--- откатанные отпечатки (чужие) 
\label{f10ush}}
\end{figure*}

\begin{multicols}{2}

     Возможны два основных сценария эксплуатации системы с неоднородностями 
такого сорта.

%\noindent
 Первый заключается в наличии двух функций сравнения: <<контрольные 
с контрольными>> и <<контрольные с откатанными>>. Соответственно, учет\linebreak 
различного характера биометрической информации будет осуществлен силами 
производителя\linebreak алгоритмов распознавания. Но для этого необходима нестандартная 
реализация программного интерфейса биометрических библиотек. К тому же такой 
подход требует постоянной модернизации библиотек при каждом новом искажающем 
факторе. 

Второй сценарий заключается в работе с одной функцией сравнения 
(рис.~\ref{f6ush}), но с учетом предложенного подхода к учету искажений.

     
     Рассмотрим два варианта принятия решения: 
     \begin{enumerate}[(1)]
     \item смешивание двух баз без учета сдвигов в ошибках распознавания;
\item приведение к единой шкале по ошибке 1-го рода или по ошибке 2-го рода.
\end{enumerate}

     Основным критерием качества разработанной методики приведения к единой 
шкале является точность оценки ошибок 1-го и 2-го рода, которая продемонстрирована 
в~разд.~3. 
     
 
%\end{multicols}

\begin{figure*} %fig12  %fig11n
\vspace*{1pt}
\begin{minipage}[t]{81.5mm}
\begin{center}
\vspace*{1pt}
\mbox{%
\epsfxsize=80.497mm
\epsfbox{ush-12.eps}
}
\end{center}
\vspace*{-9pt}
\Caption{Пример приведения шкалы ошибки 1-го рода в эксперименте~I: \textit{1}~--- 
оптический сканер, \textit{2}~--- емкостной сканер (исходная шкала), \textit{3}~--- емкостной 
сканер (приведенная шкала)
\label{f12ush}}
%\end{figure*}
\end{minipage}
\hfill
\begin{minipage}[t]{82mm}
%\begin{figure*} %fig13 %fig12n
%\vspace*{1pt}
\begin{center}
\vspace*{1pt}
\mbox{%
\epsfxsize=81.284mm
\epsfbox{ush-13.eps}
}
\end{center}
\vspace*{-9pt}
\Caption{Пример приведения шкалы ошибки 1-го рода в эксперименте~II: \textit{1}~--- без 
компенсации деформаций,   
\textit{2}~--- с компенсацией деформаций (исходная шкала), 
\textit{3}~--- с компенсацией деформаций (приведенная шкала)
\label{f13ush}}
\end{minipage}
\vspace*{9pt}
%\end{figure*}
\vspace*{2pt}
%\begin{figure*} %fig14 %fig13n
%\vspace*{1pt}
\begin{center}
%\vspace*{1pt}
\mbox{%
\epsfxsize=166.2mm % .912mm
\epsfbox{ush-14.eps}
}
\end{center}
\vspace*{-9pt}
\Caption{Пример приведения шкалы ошибки 1-го рода в экспериментах IIIa~(\textit{а}) и
IIIb~(\textit{б}): \textit{1}~--- 
контрольные, \textit{2}~--- откатанные (исходная шкала), \textit{3}~--- откатанные 
(приведенная шкала) отпечатки
 \label{f14ush}}
%\end{figure*}
\vspace*{2pt}
%\begin{figure*} %fig16 %fig14n
%\vspace*{1pt}
\begin{center}
%\vspace*{1pt}
\mbox{%
\epsfxsize=166.2mm %.638mm
\epsfbox{ush-16.eps}
}
\end{center}
\vspace*{-9pt}
     \Caption{Изменение качества идентификации в экспериментах IIIa~(\textit{а}) и IIIb~(\textit{б}): 
     \textit{1}~--- исходная шкала, 
\textit{2}~--- скорректированная шкала, \textit{3}~--- контрольные отпечатки, \textit{4}~--- откатанные отпечатки
\label{f16ush}}
\end{figure*}

    Дополнительным эффектом от учета искажений является улучшение качества 
идентификации за счет адекватного устранения расслоения базы по неоднородности 
искажающих воздействий.
     
     На рис.~\ref{f16ush} приведены сравнительные графики ошибок 
     1-го и 2-го рода (для простого смешивания результатов сравнения и 
корректировки шкал по ошибке 2-го рода). Оценки идентификации на смешанном 
массиве зависят от доли контрольных и откатанных отпечатков в общем потоке 
обращений к биометрической системе. В данном модельном примере были взяты 
значения 50\% для каждого класса.
     
     Как видно из рисунков, корректировка шкалы позволяет прийти к оптимальному 
решению (примерно среднее арифметическое по ошибкам). В~то же время 
использование ненормированной шкалы ухудшает качество распознавания ниже 
уровня худшего из класса.
     
     
\section{Заключение}
     
     В статье изложены подходы к адаптации биометрических систем к воздействию 
искажающих факторов. Предложен подход к декомпозиции и учету искажающих 
факторов на основе метода нормальной аппроксимации. Преимуществами подхода 
являются:
     \begin{itemize}
     \item учет каждого фактора в отдельности;
\item аддитивность характеристик искажающего фактора.
\end{itemize}

     Разработанные на основе предложенного подхода методы и технологии 
адаптации к ис\-ка\-жа\-ющим факторам позволяют в значительной степени 
компенсировать влияние искажений в задаче дактилоскопической идентификации.
     
     Направлением дальнейших исследований является анализ искажающих факторов 
для других биометрических методов, таких как изображение лица, изображение 
радужной оболочки глаза, рукописный почерк и~т.\,д. 

     
{\small\frenchspacing
{%\baselineskip=10.8pt
\addcontentsline{toc}{section}{Литература}
\begin{thebibliography}{99}    
     
\bibitem{1ush}
\Au{Ушмаев О.\,С.}
Применение биометрии в аэропортах~// Biometrics TTS 2007. 22~ноября 2007~г. {\sf 
http:// www.dancom.ru/rus/AIA/Archive/RUVII\_BioLinkSolu tions\_BiometricsInAirports.pdf}.

\bibitem{2ush}
\Au{Ушмаев О.\,С.}
Реализация концепции многофакторной биометрической идентификации в пра\-во\-ох\-ра\-ни\-тель\-ных 
системах~// Интерполитех-2007. {\sf 
http:// www.dancom.ru/rus/AIA/Archive/RUVI\_BioLinkSolu
tions\_MultimodalBiometricsConcept.pdf}.

\bibitem{3ush}
\Au{Синицын~И.\,Н., Губин~А.\,В., Ушмаев~О.\,С.}
Метрологические и биометрические технологии и системы~// История науки и техники, 2008. №~7. 
С.~41--44.

\bibitem{4ush}
\Au{Ушмаев~О.\,С.}
Сервисно-ориентированный подход к разработке мультибиометрических технологий~// 
Информатика и её применения, 2008. Т.~2. Вып.~3. С.~41--53.

\bibitem{5ush}
\Au{Ушмаев О.\,С.}
Концепция мультибиометрической идентификации в информационно-аналитических системах~// 
Паспортные и правоохранительные сис\-темы-2008. Интерполитех-2008. {\sf 
http://www.dancom. ru/rus/AIA/Archive/RUXIX-IPIRAN-Ushmaev-Multi modalBiometricsFramework.ppt}.

\bibitem{7ush} %6
\Au{Bolle~R.\,M., Connell~J.\,H., Pankanti~S., Ratha~N.\,K.,  Senior~A.\,W.}
Guide to biometrics.~--- New-York: Springer-Verlag, 2003.

\bibitem{9ush} %7
\Au{Novikov~S.\,O., Ushmaev~O.\,S.}
Efficiency of elastic deformation registration for fingerprint identification~// 7th Conference  (International) 
on Pattern Recognition and Image Analysis: New Information Technologies (PRIA-7-2004) Proceedings. St.\ Petersburg, 
October 18--23, 2004. Vol.~III.~--- St.\ Petersburg: \mbox{SPbETU}, 2004. 
P.~833--836.

\bibitem{8ush} %8
Wayman~J., Jain~A., Maltoni~D., Maio~D., eds.
Biometric systems: Technology, design and performance evaluation.~--- London: Springer-Verlag, 2004.

\bibitem{6ush} %9
\Au{Dessimoz~D., Champod~C., Richiadi~J., Drygajlo~A.}
Multimodal biometrics for identity documents. Research Report, PFS 314-08.05. UNIL, June 2006.


\bibitem{10ush}
Face recognition vendor test. {\sf http://www.frvt.org}.  

\bibitem{11ush}
Fingerprint vendor technology evaluation. {\sf http:// fpvte.nist.gov}.

\bibitem{12ush}
\Au{Ушмаев О.\,С., Босов~А.\,В.}
Реализация концепции многофакторной биометрической идентификации в интегрированных 
аналитических системах~// Системы высокой доступности, 2007. Т.~3. Вып.~4. С.~13--23.

\bibitem{13ush}
\Au{Ushmaev O.\,S., Novikov~S.\,O.}
Integral criteria for large-scale multiple fingerprint solutions~/ Biometric Technology for Human 
Identification~// Eds.\ A.\,K.~Jain, and N.\,K.~Ratha. Proceedings of SPIE. Vol.~5404.~--- SPIE, Bellingham, 
WA, 2004. P.~534--543.

\bibitem{14ush}
Second fingerprint verification competition. FVC 2002. {\sf http://bias.csr.unibo.it/fvc2002/ }

\label{end\stat}

\bibitem{15ush}
NIST SD~29. NIST Special Database~29 ``Plain and Rolled Images from Paired Fingerprint Cards.''

 \end{thebibliography}
}
}

\end{multicols} 
 
 
 