\def\stat{sukhomlin}

\def\tit{ТЕХНОЛОГИЧЕСКАЯ СИСТЕМА ДЛЯ~ПОСТРОЕНИЯ ПРОГРАММНЫХ КОМПЛЕКСОВ АВТОМАТИЗАЦИИ 
ОБРАБОТКИ ТРЕХМЕРНЫХ ДАННЫХ ЛАЗЕРНОГО СКАНИРОВАНИЯ}
\def\titkol{Технологическая система для~построения программных комплексов автоматизации 
обработки трехмерных данных}
% лазерного сканирования} 

\def\autkol{В.\,А.~Сухомлин, И.\,Н.~Горькавый}
\def\aut{В.\,А.~Сухомлин$^1$, И.\,Н.~Горькавый$^2$}

\titel{\tit}{\aut}{\autkol}{\titkol}

%{\renewcommand{\thefootnote}{\fnsymbol{footnote}}\footnotetext[1]
%{Работа выполнена при поддержке РФФИ (грант 08-07-00152).}}

\renewcommand{\thefootnote}{\arabic{footnote}}
\footnotetext[1]{Факультет вычислительной математики и кибернетики МГУ; 
Институт проблем информатики Российской академии наук, sukhomlin@mail.ru}
\footnotetext[2]{Факультет вычислительной математики и кибернетики МГУ, ilya\_gor@rambler.ru}
   
\Abst{Рассмотрена технологическая система для создания комплексов 
автоматической обработки данных лазерного сканирования. Приведены разработанные 
и применяемые авторами алгоритмы для классификации трехмерных данных и 
получения высококачественных моделей земного рельефа. Описан инструментарий для 
визуализации, оценки и коррекции полученных моделей. Об\-суж\-да\-ют\-ся достигнутые 
результаты и направление дальнейших исследований.}

\KW{автоматизация обработки данных; алгоритмы классификации; трехмерные 
модели; лидар; лазерное сканирование}

%\vskip 12pt plus 9pt minus 6pt
\vskip 18pt plus 9pt minus 6pt

      \thispagestyle{headings}

      \begin{multicols}{2}

      \label{st\stat}
   
   
\section{Введение}

   Традиционные аэрокосмические данные и существующие программы их 
обработки не дают возможности создать трехмерные модели земной\linebreak 
поверхности высокого разрешения. Стереоскопическая обработка космических 
снимков со спутников SPOT Image приводит к модели с 20-метровым 
горизонтальным разрешением и вертикальной точ\-ностью в 10--20~м при 
уровне достоверности 90\%~[1]. Спутниковая программа радарного 
зондирования Shuttle Radar Topographic Mission, реализованная в 2000~г., 
получила модель земной поверхности с разрешением в 30~м и вертикальной 
точностью $\sim  4$~м~[2].
   
   В 1993~г.\ появился первый коммерческий лидар (LIght Detection And 
Ranging, {\sf http://www.csc. noaa.gov/crs/rs\_apps/sensors/lidar.htm)} 
авиационного базирования. Фактически лидарные авиасистемы совершили 
переворот в генерации трехмерных моделей земной поверхности, позволив 
получать большие объемы лазерных измерений с горизонтальным разрешением 
1--2~м и вертикальной точностью 10--15~см при уровне достоверности 
95\%. На данный момент лазерное зондирование~--- это наиболее эффективный 
и точный способ создания трехмерной модели земного рельефа.
   
   Современные лидарные системы, установленные на самолетах или 
вертолетах, позволяют\linebreak
 сканировать земную поверхность лазерными 
инфракрасными лучами (длина волны $\sim  1$~мкм) с час\-то\-той выстрелов 
   100--150~кГц. Для каждого лазерного импульса фиксируется время одного 
или нескольких отражений от поверхности. Таким образом определяются 
трехмерные координаты точек отражения луча (наклон лазерного луча к 
по\-верх\-ности известен, координаты самолета определяются с помощью 
GPS (Global Positioning System) и инерционного измерительного устройства). Сканер быст\-ро 
перемещает лазерный луч поперек траектории самолета, летящего на высоте 
около километра. Плотность распределения зондируемых точек по земной 
поверхности (post spacing) обычно близка к одному выстрелу на квад\-рат\-ный метр. 
Абсолютная вертикальная точность измерений равна 10--15~см, относительная 
(по отношению к соседней точке)~--- около 5~см. Горизонтальная точность 
зависит от высоты полета и от расхождения луча и обычно близка к 50~см. 
Кроме этого, определяется интенсивность отражения луча от поверхности. 

В  настоящее время по всему миру работает несколько сотен таких авиационных 
сис\-тем. Полученные данные применяются для точного топографирования 
местности, для мониторинга состояния старых и планирования новых нефте- и 
газопроводов, для анализа распространения радиоволн и звука, для 
моделирования стихийных бедствий и других прикладных задач.
   
   Новый уровень трехмерных данных потребовал новых методов обработки и 
соответствующих программных средств, которые могли бы наиболее полно 
использовать информативность лидарных данных. Комплексы для 
автоматической обработки таких данных стали развиваться только\linebreak\vspace*{-12pt}
\pagebreak

\noindent
 в последние годы. Они заметно сложнее своих предшественников, так как умеют 
распознавать и классифицировать объекты на поверхности. Кроме того, 
предъявляются повышенные требования к точ\-ности получаемых моделей 
земного рельефа. 
   
   Целью данной работы является решение задачи построения комплексов 
автоматической обработки данных лазерного сканирования с использованием 
расширяемой компонентной архитектуры.
   
   Статья состоит из восьми разделов. В разд.~2 обсуждаются задачи, 
возникающие при обработке лидарных данных высокого разрешения, и 
приведены ключевые особенности трехмерных данных в сравнении с двумерными 
аналогами; рассмотрено несколько возможных подходов к построению 
комплексов автоматизации обработки. Раздел~3 посвящен требованиям, 
которые предъявляются к комплексам обработки трехмерных данных и к 
получаемым с их помощью моделям рельефа. В~разд.~4 рас\-смот\-ре\-на модель 
системы на основе базовой архитектуры и перечислены ее основные 
компоненты. В~разд.~5 приводится формализованная схема представления 
данных. Раздел~6 посвящен использованным в работе методам классификации 
данных. В~разд.~7 описан инструментарий интерфейса: способы визуализации 
и интерактивной коррекции данных. В~разд.~8 
сформулированы основные результаты работы, отмечены особенности 
предлагаемого решения, указано направление дальнейших исследований.


\section{Задачи}


   Развитие технологий получения высококачественных трехмерных 
изображений обширных районов земной поверхности привело к быстрому 
рос\-ту объема трехмерных данных и одновременно расширило спектр задач, 
возникающих при их обработке. В настоящее время лидарные данные 
используются для получения высокоточных цифровых моделей рель\-ефа (ЦМР, 
Digital Terrain Model) с разрешением 2~м и выше; для извлечения моделей 
трехмерных городов, которые используются, например, в Google Earth; для 
создания контурных, затененных и других типов топографических карт; для 
генерации инфракрасных изображений местности высокого разрешения; для 
исследования характеристик лесных массивов и других практических 
применений~[3--7].
   
   При автоматизации решения этих задач используются схожие алгоритмы 
распознавания и клас\-сификации объектов, фильтрации артефактов, 
интерполяции и триангуляции поверхностей, преобразования координатных 
систем. Кроме того, при решении любой из них понадобится ввод/вывод 
данных, визуализация результата, конвертор форматов и другие системные и 
интерфейсные средства. Разумно было бы создать некоторую базовую 
архитектуру комплексов по обработке трехмерных данных, и в ее рамках реализовать 
необходимые программные модули для решения подзадач обработки. Такой 
метасистемный переход позволил бы создавать целевые решения, 
обеспечивающие весь жизненный цикл автоматической обработки, не 
перегруженные лишней функциональностью и при этом повторно 
использующие код большинства модулей.
   
   Существующие продукты не предусматривают комплексного подхода к этой 
проблеме, обеспечивая решение некоторых распространенных задач, но не 
всего цикла обработки данных. Например, программные средства, 
поставляемые с ALTM (Airborne Laser Terrain Mapper) сис\-те\-ма\-ми Optech (такие как ZinView, DASHMap, 
IVS3D Fledermaus), обеспечивают визуализацию больших массивов данных, 
растеризацию, сбор статистической информации, но не включают 
интерполирование данных, их классификацию (на рель\-еф, здания, 
растительность) и валидацию. Похожая ситуация с ESRI ArcGIS, который хотя 
и является индустриальным стандартом, но не имеет встроенных алгоритмов 
классификации и предназначен для работы с уже готовыми решениями.
   
   Принципиальными требованиями к комплексам по автоматизации обработки 
трехмерных данных лазерного сканирования являются скорость обработки и точность 
алгоритмов распознавания и классификации объектов. Лидарная аэросъемка 
генерирует 100--150~ГБ данных на каждую тысячу квад\-рат\-ных километров. Поэтому уже 
для проектов среднего размера общий объем данных составляет терабайты. 
Эффективная обработка таких объемов возможна только при наличии 
интеллектуальных алгоритмов распознавания и классификации объектов.
   
   Имеющиеся решения для потоковой обработки и распознавания двумерных 
изображений малоприменимы к трехмерным из-за следующих особенностей 
3D-данных~\cite{7su}:
   \begin{itemize}
\item принципиального отличия значимости высоты объекта от значимости его 
цвета и разной сферы их применения в практических задачах;
\item наличия множества значений высот для каждой точки поверхности, 
обусловленных последовательными отражениями луча от частично 
проницаемых поверхностных объектов, тогда как на двумерных изображениях 
цвет имеет один слой;
\item разреженности (неполноты) трехмерных изоб\-ра\-же\-ний, вызванной их 
векторным пред\-став\-ле\-ни\-ем и наличием пробелов в данных из-за сложностей в 
функционировании лидаров, тогда как растровые изображения, как правило, 
полны;
\item неоднородности трехмерных изображений, связанной со сбоями в работе 
оборудования или лидарной аэросъемкой, проведенной в разные времена года.
\end{itemize}

Трехмерные данные более информативны, чем их двумерные аналоги, однако их 
обработка связана с дополнительными сложностями и необходимы 
специальные алгоритмы, которые бы смогли максимально использовать 
имеющуюся информацию об объектах для их автоматической классификации. 
Возникла потребность в создании высокоавтоматизированных комплексов для 
обработки больших массивов трехмерных данных. Существует несколько возможных 
путей построения подобных программных решений, а именно:
   \begin{enumerate}[(1)]
   \item \textbf{создание полнофункциональной системы}, обеспечивающей 
весь жизненный цикл обработки данных и включающей в себя решения всех 
известных задач~--- классификации объектов, выделения контурных карт, 
трехмерного моделирования и~т.\,д. Реализация такого комплекса с нуля 
представляет собой дорогостоящий и сложный проект, а его сопровождение и 
развитие потребует значительных человеческих ресурсов;
\item \textbf{расширение имеющейся геоинформационной сис\-те\-мы (ГИС).} При этом за основу берется 
коммерческая система и не\-дос\-та\-ющие функции реализуются и интегрируются в 
нее. К~сожалению, встроенные программные средства, такие как скрипт-язык 
Avenue в ESRI ArcGIS, предназначены больше для развития уже имеющихся в 
ГИС решений, чем для реализации новых. Получение же исходного кода таких 
систем, как правило, невозможно. Более перспективным выглядит 
использование ГИС с открытым кодом, такой как GRASS (Geographic Resources Analysis Support System), но требования 
GPL (General Public License) не всегда приемлемы;
\item \textbf{создание базовой технологической системы}, включающей 
набор готовых системных и функциональных компонентов, с по\-сле\-ду\-ющим ее 
расширением и конкретизацией к требованиям заданного класса задач. Это 
позволяет при разработке значительно сократить этапы анализа, 
проектирования и тестирования системы и получить программное решение с 
воз\-мож\-ностью адаптации к изменяющимся условиям его функционирования. С 
развитием сис\-те\-мы чис\-ло функциональных компонентов и решаемых ими задач 
может расти.
   \end{enumerate}
   
   В данной работе избран третий вариант решения проблемы~--- 
конкретизация базовой архитектуры. Ее использование обусловлено 
технологичностью, сбалансированностью стоимости и времени на разработку и 
сопровождение. При этом сохраняется возможность для развития системы и 
применения ее к другим типам задач.
   
\section{Требования}
   
   Проектируемые комплексы по обработке трехмерных данных лазерного 
сканирования должны отвечать ряду основных требований:
   \begin{itemize}
\item обеспечивать высокий уровень автоматизации с минимальным участием 
человека;
\item создавать модели рельефа и наземных объектов максимально возможной 
вертикальной и горизонтальной точности;
\item обладать достаточной функциональностью для решения заданного класса 
задач;
\item давать пользователю возможность контроля обработки данных;
\item адаптироваться к различному качеству и формату входных данных;
\item предоставлять возможность оперативно изменять вид и формат конечных 
моделей.
\end{itemize}

Требование вертикальной точности поддается формализации: пользуясь 
общепринятыми стандартами индустрии, а именно среднеквадратическим 
отклонением в сантиметрах, взятым по 95\% контрольных измерений, имеем:
\begin{equation}
\sigma (z)  < 18{,}5~\mbox{см}\,,\quad \sigma(z) = 
\sqrt{\fr{1}{n}\sum\limits_{i=1}^n (z_i -z_i^\prime )^2}\,,
\label{e1su}
\end{equation}
где $z_i^\prime$~--- высота контрольного измерения, а $z_i$~--- 
интерполированная высота ЦМР в точке измерения. По рекомендациям FEMA (Federal
Emergency Management Agency)
и ASPRS (American Society for Photogrammetry and Remote 
Sensing)~[8] отклонение не более чем в 18,5~см считается достаточным для 
построения контурных 60-сантиметровых карт, популярных в индустрии. С 
развитием лидарной технологии требования к точности будут, вероятно, 
повышаться. Количество контрольных измерений, как правило, слишком мало, 
чтобы выявить все ошибки классификации. Поэтому кроме количественной 
оценки точности необходима визуальная верификация трехмерной модели 
оператором. Остальные требования формализуются исходя из характера и 
объема планируемой работы, сроков ее выполнения, пожеланий пользователей 
данных.
   
   Анализ перечисленных выше требований и типовых задач обработки 
позволил сформулировать основные критерии, которым должна удовлетворять 
базовая технологическая система:
   \begin{itemize}
\item иметь минимальный набор функций, при этом обеспечивающий решение 
задач обработки;
\item обладать модульной расширяемой архитектурой;
\item обеспечивать простоту реализации целевых комплексов обработки;
\item минимизировать затраты на сопровождение комплексов.
\end{itemize}

   Архитектура базовой системы, созданной в рамках данной работы, 
рассмотрена ниже.
   
\section{Модель системы}

\subsection{Архитектура базовой технологической системы}
   
   На основе описанной выше концепции была разработана архитектура 
базовой технологической системы, состоящей из трех основных виртуальных 
машин или ядер:
   \begin{enumerate}[(1)]
\item системного ядра, отвечающего за организацию вычислений, поддержку 
управления процессом обработки, управление ресурсами, доставку, подготовку 
и хранение данных;
\item функционального ядра, реализующего математические методы и 
алгоритмические решения подзадач обработки;
\item интерактивного ядра, позволяющего пользователю оценивать и 
оперативно корректировать процесс автоматической обработки данных.
   \end{enumerate}
   
   Структура такой системы и состав ее базовых компонентов показаны на 
рис.~1. Системное ядро состоит из блока ввода/вывода данных, 
конвертора форматов, менеджера данных, менеджера ресурсов и 
профилировщика. Функциональное ядро содержит блоки преобразования 
координатных систем, фильтрации артефактов, калибрации, растеризации, 
интерполяции, классификации объектов, векторизации, моделирования 
поверхностей и валидации. Интерактивное ядро состоит из компонента %\linebreak\vspace*{-12pt}
визуализации 3D, визуализатора 2D, блока статистического анализа и 
графического интерфейса.
   
   Каждый компонент ядер решает определенную задачу на некотором наборе 
входных данных и является необходимым элементом для построения 
комплексов автоматизации обработки. В компоненты выделены только те 
задачи, которые являются общими для большинства типов приложений и, 
соответственно, могут быть повторно использованы.
   
   Применение такой трехъядерной архитектуры даст возможность создавать 
автоматизированные комплексы для широкого спектра приложений. При этом 
каждый целевой продукт будет включать только те компоненты ядер, которые 
необходимы для решения поставленных задач. Реализация выбранных 
компонентов может быть изменена с учетом требований среды, таких как 
специфический формат входных данных или изменение интерфейса 
управления. Компоненты представлены в виде библиотек (динамических и 
статических) или в виде исполнимых файлов.
   

   Анализ показывает, что подобные работы ведутся в настоящее время в ряде 
стран. Например, комплекс TerraScan финской компании Terrasolid, среда 
SCOP++~\cite{3su} немецкой фирмы Inpho или трехстадийная архитектура, 
предложенная итальянской группой в~\cite{4su}.


\noindent
\begin{center}
\vspace*{18pt}
\mbox{%
\epsfxsize=78.59mm
\epsfbox{suh-1.eps}
}
\end{center}
\vspace*{12pt}
%\label{f1ush}}
%\end{figure*}
{{\figurename~1}\ \ \small{Базовая архитектура комплексов обработки 
трехмерных данных}}

\bigskip
%\medskip
\addtocounter{figure}{1}
  
\subsection{Компоненты системного ядра} %4.2  
  
   Системное ядро представляет собой про\-грам\-мный модуль, постоянно 
загруженный в память и работающий в скрытом режиме без прямого общения с 
пользователем. Реализация ядра предоставляет другим компонентам 
программный интерфейс для унифицированного доступа к данным, запускает 
на выполнение функциональные компоненты, следит за использованием 
аппаратных ресурсов и ресурсов операционной системы, ведет статистику времени выполнения 
компонентов. Реализация выполнена на языке С++ в среде Visual Studio. 
Рассмотрим подробней функциональность компонентов или модулей ядра.
   
   \textbf{Ввод}/\textbf{вывод} обеспечивает считывание данных с носителей 
(сотни гигабайт, как правило) и запись готовых результатов. Для упрощения 
предварительной подготовки данных модуль рекурсивно обходит дерево 
директорий файловой системы и автоматически распознает пригодные для 
чтения файлы с данными. Число успешно прочитанных записей лидарных 
импульсов и записей с поврежденным форматом передается в компонент 
статистического анализа интерактивного модуля для генерации отчета о 
входных данных проекта. Скорость операций чтения/записи составляет не 
менее 10~МБ/с, что позволяет за сутки прочитать и подготовить к обработке 
терабайт данных. Подробное описание входного формата лидарных данных 
есть в разд.~5.
{\looseness=1

}
   
   С вводом/выводом тесно связан \textbf{конвертор форматов}, который 
используется для перевода одних форм представления данных в другие, в том 
чис\-ле во внутреннее представление. Этот компонент обеспечивает единый 
метод работы с данными, име\-ющи\-ми различную организацию и структуру. 
Список поддерживаемых форматов в реализации включает такие популярные 
форматы, как бинарные BIL (``band interleaved by line''), LAS
(``LIDAR data exchange''), GeoTIFF (``geographic tagged image file format''), 
текстовые ARC/INFO ASCII GRID, 
ASCII XYZ, и может быть расширен. Практика показала, что узким местом в 
про\-из\-во\-ди\-тель\-ности компонента часто становятся такие базовые 
преобразования, как переводы $\mathrm{STRING}\leftrightarrow 
\mathrm{INT}$, $\mathrm{STRING}\leftrightarrow \mathrm{FLOAT}$. Анализ 
функций atol(\,) и strtol(\,) из библиотеки GNU glibc выявил, что в этих 
функциях б$\acute{\mbox{о}}$льшую часть кода занимают операции, связанные с 
распознаванием недесятичных чисел, использованием языковых локалей и 
проверкой переполнений. Дополнительные действия, условные переходы и 
вызовы других функций значительно замедляют код стандартной библиотеки. 
Поскольку при вводе данных их формат заранее известен, то эти проверки 
можно опустить и использовать более эффективный код.
   
   Организацию данных во внутреннем пред\-став\-ле\-нии и сохранение 
промежуточных результатов вычислений осуществляет \textbf{менеджер 
данных}. Он сортирует прочитанные входные данные на отдельные участки. 
Чтобы определить, к какому участку принадлежит лидарный импульс, 
используется сетка разбиения, которая может быть как равномерной 
прямоугольной, так и произвольно заданной. Если система координат 
(картографическая проекция), в которой определена сетка, не совпадает с 
системой координат входных данных, то производится \textbf{преобразование 
координатных систем} применительно к сетке, так как в ней меньше точек. 
Каждый готовый участок выводится в отдельный файл. На этом этапе 
обработки хранение промежуточных результатов неизбежно, иначе пришлось 
бы считывать всю неструктурированную базу данных проекта при каждом 
обращении к участку. Здесь под участком понимается совокупность лидарных 
точек, расположенных в пространственном параллелепипеде, размеры которого 
задаются с учетом цели обработки и возможностей применяемых компьютеров.
   
   Менеджер данных полагается в своей работе на \textbf{менеджер ресурсов}, 
который контролирует использование ресурсов системы, таких как оперативная 
память, место на жестких дисках, количество одновременно открытых 
дескрипторов файлов, и, по возможности, освобождает ресурсы, занятые редко 
используемыми данными, для более новых.
   
   \textbf{Профилировщик} служит для сбора статистики о 
производительности вычислений и отдельных функциональных компонентов. 
Он имеет точность работы в 1~мс и использует механизм 
timeGetTime(\,) как имеющий низкие накладные расходы и наиболее надежный 
и устойчивый на разных типах систем, включая многопроцессорные. На основе 
собранной статистики периодически обновляется прогноз о завершении 
обработки (это может быть несколько часов или несколько дней), а также 
выдается итоговый отчет. Профилировщик полезен при тестировании 
различных компонентов комплекса и при поиске узких мест в 
последовательной работе компонентов (конвейера обработки).
   
\subsection{Компоненты функционального ядра} %4.3  

   Функциональное ядро представляет собой набор программных модулей, 
которые реализуют алгоритмы конвейера автоматической обработки  данных. 
Число модулей в конвейере и порядок их\linebreak следования могут варьироваться, но в 
общем случае они соответствуют порядку, указанному на рис.~1 в 
функциональном ядре. Модули загружаются в память и исполняются по мере 
надобности. Отделение их разработки от остальных частей комплекса 
позволило сделать алгоритмы более адаптивными к различным задачам и дало 
возможность быстрой перенастройки комплексов. Модули используют 
системное ядро для доступа к данным и интерактивное ядро для визуализации и 
обратной связи с пользователем. Модули были реализованы на языках С++ и 
Fortran~77, а некоторые были портированы на UNIX и использовались как 
самостоятельные утилиты.
   
   \textbf{Преобразование координатных систем} выполняет перевод между 
картографическими проекциями. Типичная задача~--- перевод из 
универсальной равноугольной поперечно-цилиндрической проекции Меркатора 
в локальную систему. На первых этапах реализации была сделана попытка 
использовать для этой цели сторонний продукт Corpscon~5, но его 
производительность и устаревший код ввода-вывода не позволяли выполнить 
задачу за приемлемое время. Поэтому был написан собственный модуль, 
основанный на коде Corpscon, который был переработан для достижения более 
высокой производительности: около 1~млн преобразований координат в 
секунду, что в несколько раз быстрее, чем при использовании оригинального 
кода.
   
   \textbf{Фильтры артефактов} на разных этапах работы проверяют 
соответствие данных набору определенных условий~--- от простейших 
требований к корректности записи до выявления аномальных значений высоты 
и интенсивности, которые могут быть признаками сбоя в оборудовании или 
наличия посторонних объектов между лазером и поверхностью земли. 
Лидарные отражения, не удовлетворяющие условиям, помечаются как 
ошибочные и в дальнейшем не участвуют в построении цифровых моделей и не 
входят в конечный результат. Более сложные ошибки, такие как смещение 
всего массива отражений по одной из координатных осей, выявляются на этапе 
\textbf{валидации} при сравнении с контрольными измерениями. Иногда 
ошибки приводят к пропускам в данных, в таком случае решением является 
\textbf{интерполяция}.
   
   Модуль \textbf{калибрации}, используя нормализацию по выбранному 
эталону, выравнивает значения интенсивности (яркости) для данных, 
собранных разной аппаратурой или в разное время года. Описание алгоритма и 
примеры его работы можно найти в~\cite{5su}.
   
   Лидарные данные после сортировки на участки представляют собой кластер 
трехмерных точек с некоторым набором атрибутов у каждой. 
\textbf{Растеризация} переводит такие данные в набор двумерных матриц, что 
позволяет значительно повысить скорость работы и на порядок понизить 
требования к объему оперативной памяти. При правильном выборе размера 
элемента растра потеря вертикальной точности практически отсутствует. 
Горизонтальная точность лидарных данных изначально не очень высока из-за 
расхождения луча лазера.
   
   Модуль \textbf{интерполяции} заполняет пустоты в двух- или трехмерных данных, 
как правило, связанные с ошибками при сборе данных. При этом используются 
многопроходные алгоритмы, позволяющие\linebreak пробелы в данных отличать от 
пустот, со\-от\-вет\-ст\-ву\-ющих водным массивам, которые не дают стабильного 
отраженного луча. Пробелы заполняются прос\-той интерполяцией информации 
из соседних\linebreak непустых элементов растра. Число пробелов может достигать 10\% 
от всей площади участка. Описание реализованных методов интерполяции есть 
в~\cite{5su}.
   
   Модуль \textbf{классификации объектов} сортирует данные на несколько 
типов (земля, растительность, здания), используя совокупность оригинальных 
математических методов~\cite{6su, 7su}, позволяющих достичь высокой 
точности классификации (1~ошибка на 10$^3$--10$^4$~точек) при сохранении 
приемлемой производительности ($\sim 30$~тыс.\ классификаций в секунду). 
Методам классификации, положенным в основу данного модуля, посвящен 
разд.~6.
   
   Растр используется обычно в качестве внутреннего представления для 
распознавания и классификации, при этом итоговые модели должны быть 
векторными. Поэтому модуль \textbf{векторизации} делает обратное 
преобразование отклассифицированных на растре моделей в векторную форму. 
Далее вступает в работу компонент \textbf{моделирования поверхностей}, 
строящий триангуляционную сетку на основе векторной модели. Проверка 
точности полученной трехмерной модели осуществляется в модуле 
\textbf{валидации}, который формирует набор контрольных измерений и 
вычисляет среднеквадратическое отклонение~(1).
   

\subsection{Компоненты интерактивного ядра} %4.4  
   
   Интерактивное ядро~--- это интерфейсная часть комплекса, которая 
предоставляет пользователю средства визуализации, контроля и активного 
вмешательства в процесс обработки. Основные компоненты ядра реализованы 
на языке С++ и используют кросс-платформенный стандарт \mbox{OpenGL~2.1}.
   
   Компонент \textbf{визуализации 3D} позволяет отобразить трехмерную 
модель рельефа с возможностью наложения текстурных карт; модель можно 
произвольно вращать и приближать. В компоненте реализовано большое число 
оптимизационных техник для обеспечения максимальной про\-из\-во\-ди\-тель\-ности. 
Модуль \textbf{визуализации 2D} отвечает за отоб\-ра\-же\-ние двумерных карт 
высот и ин\-тен\-сив\-ностей и имеет стандартные функции сдвига и увеличения 
изображения.
   
   \textbf{Средства статистического анализа} позволяют узнать характерные 
свойства распознаваемых объектов~--- занимаемую площадь, высоту над 
уровнем земли. Эта информация полезна для классификации с участием 
оператора. Оболочкой для интерактивного ядра служит \textbf{графический 
интерфейс} (GUI, Graphical User Interface). С его помощью пользователь загружает данные участка с 
диска, управляет их отображением в 2D- или 3D-виде, следит за качеством 
автоматической классификации объектов и визуально оценивает получившуюся 
модель. При обнаружении недостатков оператор может про\-вес\-ти 
интерактивную коррекцию: выделить мышкой проблемную область и 
пересчитать ее с новыми параметрами классификации. Более подробно этот 
инструментарий описан в разд.~7.
   
   Контроль обработки человеком улучшает качество моделей, так как создать 
идеальные методы распознавания и классификации пока не пред\-став\-ля\-ет\-ся 
возможным. Однако уже существующий комплекс может провести обработку в 
автономном режиме и получить модели, удовле\-тво\-ря\-ющие стандартам 
точности. Для описания пакетной обработки используется скриптовое 
представление на основе языка Perl и библиотеки, обес\-пе\-чи\-ва\-ющей доступ к 
компонентам системы. Выбор языка обусловлен доступностью 
интерпретаторов Perl для всех основных платформ, его гибкостью при работе 
со строками и легкостью написания на нем сценариев. Элементы управления 
интерфейса во многом зависят от применяемых функциональных компонентов 
и их конфигурационных параметров, поэтому графический интерфейс 
подвержен частым изменениям при построении целевых решений.

  \section{Схема данных}
   
   Лидарные данные имеют высокую информативность, позволяя передать не 
просто трехмерные координаты конкретной точки земной поверхности, но и ее 
цвет в инфракрасном диапазоне, сколько отражающих поверхностей встретил 
лазерный луч на пути к земле, каковы их координаты, под каким углом и в 
какое время был выпущен луч и многое другое. Сохранение в процессе 
обработки этих дополнительных параметров позволяет использовать их на 
этапе классификации объектов и получать более точные цифровые модели. В 
общем виде лидарные данные можно представить множеством точек в 
трехмерном пространстве с некоторым набором обязательных и 
необязательных атрибутов у каждой точки. Формальную схему можно 
определить так:
   
   \end{multicols}
   
   \noindent
%   \begin{center}
   \begin{verbatim}
    <xs:schema>
       <xs:element name="PointData">
          <xs:complexType>
          <xs:sequence>
             <xs:element name="CoordX" type="xs:double" />
             <xs:element name="CoordY" type="xs:double" />
             <xs:element name="HeightZ" type="xs:float" />
             <xs:element name="Intensity" type="xs:float" />
             <xs:element name="ReturnNumber" type="xs:byte" />
             <xs:element name="ScanDirection" minOccurs="0">
               <xs:simpleType>
               <xs:restriction base="xs:string">
                 <xs:enumeration value="Left to Right" />
                 <xs:enumeration value="Right to Left" />
               </xs:restriction>
               </xs:simpleType>
             </xs:element>
             <xs:element name="ScanAngle" type="xs:byte"
                         minOccurs="0" />
             <xs:element name="FlightlineEdgeFlag" type="xs:boolean"
                         minOccurs="0" />
             <xs:element name="ClassType" minOccurs="0">
               <xs:simpleType>
               <xs:restriction base="xs:string">
                 <xs:enumeration value="Noise" />
                 <xs:enumeration value="Unclassified" />
                 <xs:enumeration value="Ground" />
                 <xs:enumeration value="Low Vegetation" />
                 <xs:enumeration value="Medium Vegetation" />
                 <xs:enumeration value="High Vegetation" />
                 <xs:enumeration value="Building" />
                 <xs:enumeration value="Water" />
               </xs:restriction>
               </xs:simpleType>
             </xs:element>
             <xs:element name="SourceID" type="xs:string"
                         minOccurs="0" />
             <xs:element name="GPS Time" type="xs:time" minOccurs="0" />
             <xs:element name="RedChannel" type="xs:integer"
                         minOccurs="0" />
             <xs:element name="GreenChannel" type="xs:integer"
                         minOccurs="0" />
             <xs:element name="BlueChannel" type="xs:integer"
                         minOccurs="0" />
          </xs:sequence>
          </xs:complexType>
       </xs:element>
    </xs:schema>
\end{verbatim}
%\end{center}

\begin{multicols}{2}

Здесь \verb"CoordX", \verb*"CoordY" и \verb*"HeightZ" задают координаты 
точки; \verb*"Intensity"~---интенсивность отражения лазерного луча; 
\verb*"ReturnNumber" определяет, к какому по счету отражению импульса 
относится данная точка; поля \verb*"ScanDirection", \verb*"ScanAngle" и 
\verb*"FlightlineEdgeFlag" характеризуют направление движения лазерного 
луча во время импульса. Поле \verb*"ClassType" содержит класс объекта, к 
которому относится точка (Unclassified~--- для еще не классифицированных). 
Поле \verb*"SourceID" определяет источник данных~--- это может быть имя 
файла, название проекта или любой другой текс\-то\-вой идентификатор. Три 
цветовых компоненты \verb*"RedChannel", \verb*"GreenChannel" и 
\verb*"BlueChannel" позволяют окрасить точку, если лидарный сканер 
использовался совместно с фотокамерой.

   Пример записи, соответствующей изложенной схеме данных:
   
   \noindent
   \begin{verbatim}
<PointData>
  <CoordX> 598867.72 </CoordX>
  <CoordY> 4500731.25 </CoordY>
  <HeightZ> 0.06 </HeightZ>
  <Intensity> 248.92 </Intensity>
  <ReturnNumber> 1 </ReturnNumber>
  <ClassType> Water </ClassType>
  <SourceID> 062603_f1_297 </SourceID>
</PointData>
   \end{verbatim}
   
   Учитывая то, что проект может содержать более 10$^{10}$~точек, 
использование XML (eXtensible Markup Language) формата в реализации сопряжено с большими затратами 
аппаратных ресурсов, поэтому данные хранятся в двоичной форме. При этом 
текстовые поля заменяются чис\-ло\-вы\-ми и максимум полей упаковывается 
побитно. Формат и длина упакованной записи, включая наличие 
необязательных полей, определяются в заголовке. Туда же выносятся 
глобальные свойства данных, такие как информация о картографической 
проекции, текс\-то\-вый комментарий к проекту и~т.\,п. Форма XML может быть 
использована при работе с небольшим подмножеством данных, например при 
просмотре пользователем точек, принадлежащих к конкретному объекту на 
поверхности.

\section{Методы классификации}
     
   Важным компонентом функционального ядра базовой системы является 
модуль <<классификации данных>> (см.\ рис.~1). Для создания 
качественной модели земной поверхности лидарные данные должны быть 
классифицированы на отражения от земли и на отражения от растительности 
или домов, которые для ЦМР являются помехами. Проб\-ле\-ма классификации 
огромного массива данных (около миллиона точек на квадратный километр) 
пред\-ставляет собой нетривиальную алгоритмическую проб\-ле\-му. Рассмотрим 
некоторые методы, которые могут быть положены в основу компонента 
классификации данных.

   В статье~\cite{9su} был предложен алгоритм фильтрации, основанный на 
анализе наклона поверхности (slope-based filter или SB-фильтр). Он состоит в 
том, что для каждой точки лидарного отражения строится фильтрующий 
элемент~--- выпуклая поверхность в виде перевернутой чаши, например 
гиперболоида с вершиной в анализируемой точке. Если в полости гиперболоида 
нет других точек, то рассматриваемая точка отражения считается точкой 
земной поверхности. По набору лидарных отражений от земной поверхности 
методом триангуляции строится модель рельефа.
   
   Такой метод имеет ряд недостатков~\cite{10su}, например он неприменим к 
крупным зданиям. Кроме того, для земной поверхности с большим наклоном 
значительная часть точек может быть неправильно отфильтрована как помехи 
(гиперболоид, построенный в точке крутого склона, легко захватывает соседние 
точки отражений). В статье~\cite{10su} предложен адаптивный SB-фильтр, где 
в качестве фильт\-ру\-юще\-го элемента взят конус с углом, зависящим от наклона 
поверхности, что частично устранило недостатки метода.

   Одним из интересных подходов является <<метод линейного 
предсказания>>~\cite{11su}, совмещающий классификацию и интерполяцию. 
Это статистический метод, основанный на корреляции между соседними 
точками.
   
В статье~\cite{12su} предложен IPF (Interrelated Proc\-esses Forecasting)
метод, который также совмещает\linebreak 
фильтрацию и интерполирование. По точкам лидарных отражений строится 
усредненная по\-верх\-ность первого приближения, и точки, расположенные 
значительно выше этой по\-верх\-ности,\linebreak отбрасываются. Оставшиеся точки 
используются для генерации модели поверхности второго приближения, и 
алгоритм повторяется. После ряда итераций получается земная поверхность 
приемлемого качества.
   
   Авторами настоящей статьи был разработан новый математический метод 
классификации лазерных отражений, соответствующих земной 
   поверхности,~--- метод виртуальных поверхностей (МВП). В качестве 
фильтрующих элементов он использует поверхности с динамически 
меняющейся степенью выпуклости. Этот метод обладает положительными 
качествами известных методов: SB-фильтра~\cite{9su} и его адаптивной 
версии~\cite{10su}, а также IPF-метода~\cite{12su}. При этом метод 
виртуальной поверхности избегает наиболее серьезных недостатков этих 
алгоритмов.\linebreak
 Метод виртуальных поверхностей совмещает фильт\-ра\-цию и интерполирование, как и 
   IPF-метод, получая в результате виртуальную поверхность, близкую к 
реальной земной поверхности. Метод\linebreak
 виртуальных поверхностей, в отличие от IPF-ме\-то\-да, не является 
итеративным и делает классификацию поверхности за один проход, что 
значительно ускоряет обработку данных. Детальное описание МВП можно 
найти в~\cite{6su}. Для эффективной работы метода необходима растеризация 
лидарных данных: перенос на прямоугольную сетку с пикселом определенного 
размера. С этой целью первичное множество точек лазерных отражений 
рассортировывается в три двумерные матрицы:
   \begin{enumerate}[(1)]
   \item
   $Z_{\min\,ik}$~--- матрицу минимального значения высот отражений;
   \item $Z_{\max\,ik}$~--- матрицу максимального значения высот отражений;
   \item $I_{\mathrm{av}\,ik}$~--- матрицу среднего значения интенсивности отражения.
   \end{enumerate}
   
   Для формализованного представления процесса в~\cite{13su} изложен метод 
матричного описания обработки данных (МООД), который сопоставляет 
каждой процедуре обработки оператор, дейст\-ву\-ющий над матрицей значений.
   
   На рис.~\ref{f2su} приведен результат работы классифицирующих 
алгоритмов комплекса АКС-ЛИДАР-3D, реализованного авторами в рамках 
предложенной архитектуры, в сравнении с модулем \mbox{RASCOR}~\cite{14su}, 
разработанным в Ганноверском университете (Германия).
   
   Видно, что RASCOR не смог полностью извлечь растительность и здания, 
оставив многочисленные пики, в то время как МВП построил достаточно 
гладкую модель рельефа, убрав всю растительность.

\section{Инструменты}

   В рамках предложенной архитектуры разработан удобный инструментарий 
активного вмешательства в процесс обработки, который позволяет в течение 
   1--2~мин пересчитать модель земного рельефа на участке $\sim 1$~км$^2$ 
с уточненными параметрами классификации. Для обнаружения проб\-лем\-ной 
области в рельефе, связанной с ошибкой в исходных данных или с неточностью 
автоматической обработки, используется оригинальный 
   3D-визуализатор~\cite{7su}. Регулярная структура ЦМР, полученная на 
этапе классификации, позволяет воспользоваться при отображении 
иерархической структурой данных.
   
   Для эффективной визуализации необходимо обращаться к ресурсам 
современных графических адаптеров (GPU, Graphics Processing Unit), которые превосходят CPU 
(Central Processing Unit) по 
вычислительной мощности. Для ускорения отображения в модуле 
используются: группировка вершин\linebreak\vspace*{-12pt}
\pagebreak

\end{multicols}

   \begin{figure} %fig2
   \vspace*{1pt}
\begin{center}
\vspace*{1pt}
\mbox{%
\epsfxsize=158mm %160mm
\epsfbox{suh-2.eps}
}
\end{center}
\vspace*{-9pt}
   \Caption{Работа алгоритмов классификации: (\textit{1}) и~(\textit{2})~--- исходные данные и 
результат работы RASCOR; (\textit{3}) и~(\textit{4})~--- исходные данные и результат работы 
комплекса АКС-ЛИДАР-3D
   \label{f2su}}
      \vspace*{9pt}
   \end{figure}
   
   \begin{multicols}{2}

\noindent
 в участки, отсечение по усеченной 
пирамиде видимости, динамическая детализация, оптимизация по 
предварительному кэшу и кэшу трансформированных вершин.
  
   
   Качество и скорость визуализации сравнимы с такими популярными 
инструментами ГИС, как ESRI ArcGIS Desktop и APL Quick Terrain Modeler. 
Пример визуализации трехмерных рельефов с помощью разработанного 
инструмента показан на рис.~\ref{f3su}, где изображен карьер глубиной более 
300~м, имеющий ступенчатый рельеф стен.
 
   В зависимости от сложности моделируемой области до 90\% обработанных 
участков оказываются хорошего качества. Остальные участки обычно содержат 
один--два локальных артефакта. После визуализации полученной ЦМР и 
выявления ошибки в рельефе, оператор может воспользоваться инструментом 
активной коррекции трехмерных данных, например:
   \begin{itemize}
\item задать более агрессивную фильтрацию в рамках метода виртуальной 
поверхности и выгладить любую плохо отфильтрованную зону;
\end{itemize}
%\pagebreak

\end{multicols}

\begin{figure} %fig3
\vspace*{1pt}
\begin{center}
\vspace*{1pt}
\mbox{%
\epsfxsize=155.984mm
\epsfbox{suh-3.eps}
}
\end{center}
\vspace*{-9pt}
   \Caption{Окно интерфейса: трехмерная модель земной поверхности
   \label{f3su}}
   \end{figure}
   
   \begin{multicols}{2}
   
\begin{itemize}
\item отменить фильтрацию в определенной зоне и сохранить все имеющиеся 
там объекты;
\item отрезать лидарные отражения выше или ниже заданного уровня;
\item усилить функцию заклеивания точечных артефактов в исходных данных 
и~т.\,д.
  \end{itemize}

  \section{Заключение}
   
   Методы и идеи, на которые опирается данная работа, разрабатывались 
авторами на протяжении последних шести лет. За это время на основе 
предложенной технологической системы было создано два типа комплексов 
автоматизации обработки трехмерных данных лазерного сканирования:
   \begin{itemize}
\item АКС-ЛИДАР-3D для получения трехмерной цифровой модели рельефа и 
распознавания поверхностных объектов;
\item АКС-ЛИДАР для получения двумерных инфракрасных изображений 
поверхности.
\end{itemize}

Для каждого типа было сделано несколько реализаций под конкретные проекты 
обработки (примеры работы комплексов были приведены на рис.~\ref{f2su}
и~\ref{f3su}).

   В частности, комплексы использовались для обработки лидарных данных 
территории штата Мэриленд (США). Результаты в виде инфракрасных 
изображений, цифровых моделей рель\-ефа и сопутствующие метаданные о 
точ\-ности полученных моделей доступны по адресам: {\sf 
http://dnrweb.dnr.state.md.us/gis/data/lidar/ и http://maps.csc.noaa.gov/TCM/}.

   Общий объем работы, выполненной с помощью комплексов, составил свыше 
20\,000~км$^2$ земной поверхности с разрешением 2~м. Скорость полного 
цикла обработки составляла 500~км$^2$ в неделю, при этом полученная модель 
рельефа имела типичное среднеквадратическое отклонение по высоте 
в~14,3~см, а для качественных данных точность достигала 9,3~см (по данным 
независимой экспертизы, проведенной Dewberry LLC). В проекте с 
высококачественными лидарными авиаданными и разрешением в 1,2~м 
была достигнута точность по высоте в 6--8~см. Точность автоматической 
классификации превосходила аналогичные программные продукты~\cite{11su}.

   
   Основными особенностями предлагаемого подхода и реализации 
программных комплексов, описанных в данной статье, являются:
   \begin{itemize}
\item использование трехъядерной архитектуры, включающей системное, 
функциональное и интерактивное ядра, что позволяет повторно использовать 
максимум кода и сократить трудозатраты на разработку;
\item создание базовой технологической системы, включающей необходимый 
набор основных компонентов и упрощающей реализацию и сопровождение 
комплексов обработки данных;
\item применение оригинальных алгоритмических решений, таких как метод 
виртуальной поверхности для классификации объектов и матричное 
представление данных, что обеспечивает высокую точность и скорость 
обработки;
\item реализация собственного визуализатора трехмерных поверхностей, 
который использует современные техники оптимизации, что позволило 
работать с моделями, состоящими из\linebreak
миллионов точек;
\item наличие инструментария для управления цик\-лом обработки~--- как в виде 
интерактивных средств, реализованных в графическом интерфейсе, так и в виде 
скрипт-языка;
\item применение открытого бинарного формата данных, разработанного в 
соответствии с требованиями индустриального стандарта (LAS~1.2/2008). Этот 
формат показал высокую эффективность при хранении и преобразовании 
данных.
\end{itemize}

   Дальнейшее усовершенствование базовой технологической системы ведется 
в направлении расширения возможностей функционального ядра и повышения 
качества работы его алгоритмов. В перспективе возможно создание полностью 
автоматизированного полнофункционального комплекса по обработке 
трехмерных данных лазерного сканирования.
      
{\small\frenchspacing
{%\baselineskip=10.8pt
\addcontentsline{toc}{section}{Литература}
\begin{thebibliography}{99}    
\bibitem{1su}
SPOT DEM product description, 2005. 
{\sf http://www. spotimage.fr/automne\_modules\_files/standard/public/\linebreak 
p807\_fileLINKEDFILE\_SPOT\_DEM\_Product\_Descrip\linebreak tion\_v1-2.pdf}.

\bibitem{2su}
\Au{Rodriguez~E., Morris~C., Belz~J., Chapin~E., Martin~J., Daffer~W., Hensley~S.}
An assessment of the SRTM topographic products. Technical Report JPL D-31639.~--- Pasadena, 
California, 2005. 143~p. {\sf http:// www2.jpl.nasa.gov/srtm/SRTM\_D31639.pdf}.

\bibitem{3su}
\Au{Pfeifer~N., Stadler~P., Briese~C.}
Derivation of digital terrain models in the SCOP++ environment~// OEEPE Workshop on 
Airborne Laserscanning and Interferometric SAR for Digital Elevation Models, Stockholm, 2001.

\bibitem{6su} %4
\Au{Горькавый И.\,Н.}
Метод виртуальной поверхности для классификации данных LIDAR и генерации 
трехмерной модели земного рельефа~// Труды I~Меж-\linebreak дународной 
научно-практической конференции <<Современные информационные технологии и 
ИТ-об\-ра\-зо\-вание>>~/ Под ред.\ В.\,А.~Сухомлина.~--- М.: ВМК МГУ, 2005. С.~583--597.

\bibitem{4su} %5
\Au{Forlani G., Nardinocchi~C., Scaioni~M., Zingaretti~P.}
Complete classification of raw LIDAR data and 3D reconstruction of buildings~// Pattern 
Analysis \& Applications, 2006. Vol.~8. No.\,4. P.~357--374.

\bibitem{5su} %6
\Au{Горькавый И.\,Н.}
Комплексный подход к автоматизации процесса обработки данных LIDAR для получения 
инфракрасных изображений высокого разрешения~// Известия высших учебных заведений. 
Геодезия и аэрофотосъемка, 2007. Вып.~5. С.~148--162.

\bibitem{7su}
\Au{Горькавый И.\,Н.}
Автоматизированные программные средства обработки трехмерных данных лазерного 
сканирования~// Известия высших учебных заведений. Геодезия и аэрофотосъемка, 2008. 
Вып.~4. С.~22--34.

\bibitem{8su}
ASPRS guidelines: Vertical accuracy reporting for lidar data, 2004. {\sf 
http://www.asprs.org/society/committees/\linebreak lidar/Downloads/Vertical\_Accuracy\_Reporting\_for\_\linebreak 
Lidar\_Data.pdf}.

\bibitem{9su}
\Au{Vosselman G.}
Slope based filtering of laser altimetry data~// Int. Arch. Photogrammetry Remote Sensing, 2000. 
Vol.~33.

\bibitem{10su}
\Au{Sithole G.}
Filtering of laser altimetry data using a slope adaptive filter~// Int. Arch. Photogrammetry Remote 
Sensing, 2001. Vol.~34.

\bibitem{11su}
\Au{Lohmann P., Koch~A., Schaeffer~M.}
Approaches to the filtering of laser scanner data // Int. Arch. Photogrammetry Remote Sensing, 2000. 
Vol.~33.

\bibitem{12su}
\Au{Kraus~K., Pfeifer~N.} 
Advanced DTM generation from \mbox{LIDAR} data // Int. Arch. Photogrammetry Remote Sensing, 2001. 
Vol.~34.

\bibitem{13su}
\Au{Горькавый И.\,Н.}
Программные средства и мате\-матические методы обработки и классификации трех\-мерных 
данных~// Труды III~Международной научно-прак\-ти\-че\-ской конференции 
<<Современные информационные технологии и ИТ-образование>>~/ Под ред.\ 
В.\,А.~Сухомлина.~--- М.: ВМК МГУ, 2008. С.~297--313.

\label{end\stat}

\bibitem{14su}
\Au{Jacobsen K., Lohmann~P.}
Segmented filtering of laser scanner DSMs~// ISPRS WG III/3 Workshop. Dresden, 2003.
 \end{thebibliography}
}
}
\end{multicols}