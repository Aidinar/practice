\def\stat{abstr}
{%\hrule\par
%\vskip 7pt % 7pt
\raggedleft\Large \bf%\baselineskip=3.2ex
A\,B\,S\,T\,R\,A\,C\,T\,S \vskip 17pt
    \hrule
    \par
\vskip 21pt plus 6pt minus 3pt }


%1
\def\tit{A PROBABILISTIC ANALYSIS OF FAULT DETECTION LATENCY IN A NETWORK OF FINITE STATE MACHINES}

\def\aut{A.\,V.~Pechinkin$^1$  and S.\,L.~Frenkel$^2$}
\def\auf{$^1$IPI RAN,  apechinkin@ipiran.ru\\[1pt]
$^2$IPI RAN, fsergei@mail.ru}

\def\leftkol{\ } %ENGLISH ABSTRACTS}

\def\rightkol{\ } %ENGLISH ABSTRACTS}


\titele{\tit}{\aut}{\auf}{\leftkol}{\rightkol}

\noindent
This paper suggests an approach to the computation of time probability distribution 
function (PDF) of Fault Detection Latency (FDL) in case, when a system is modeled as a 
combination of interacting finite state machines (FSMs) under random inputs, where the 
interactions deal with switching each of the FSMs from a working mode to a testing one. Fault detection latency is 
a period of fault detection after it occurs in certain inner states. Traditionally, the FDL of an 
FSM is modeled as the time (numbers of its state transition steps) to absorption for a Markov 
chain with the state space generated by a product of fault-free and faulty (i.e., corrupted by a 
fault) FSMs. The principal problem of using this model for the networks of sub-FSMs is that 
random transitions of the product of the fault-free and faulty networked automata even under 
independent inputs random vectors are not the Markovian ones. Thanks to an extension of the 
transition space of the networked FSMs by some additional states corresponding to the number 
of steps between transitions to the modes mentioned above for each of sub-FSMs, this 
model is extended to the case of an FSM decomposed (in a designing process) into a number of components 
of sub-FSMs. A way to compute the FDL PDF in terms of FDL PDF of initial FSM 
(that is not decomposed) and the FSMs of corresponding sub-FSMs is shown.

\KWN{testing; Finite State Machine; Markov chains}

\vskip 14pt plus 6pt minus 3pt

%2
\def\tit{BACKUP USING SNAPSHOTS}


\def\aut{V.\,A.~Kozmidiady}
\def\auf{IPI RAN, v.kozmidiady@gmail.com}


%\def\leftkol{\ } %ENGLISH ABSTRACTS}

%\def\rightkol{\ } %ENGLISH ABSTRACTS}


\titele{\tit}{\aut}{\auf}{\leftkol}{\rightkol}

\noindent
Complication of the reserve copying (backup) using differences
between two successive snapshots is examined. A~formal model
in which it is possible exactly to put a problem is offered. An algorithm
which solves this problem for almost linear time of the volume of
differences is offered.

\KWN{backup; snapshots; file system recovery; journaling file system
}

\vskip 14pt plus 6pt minus 3pt

%2
\def\tit{REGISTRATION OF DISTORTIONS IN AUTOMATIC FINGERPRINT 
IDENTIFICATION}


\def\aut{O.\,S. Ushmaev}
\def\auf{IPI RAN, oushmaev@ipiran.ru}


%\def\leftkol{\ } %ENGLISH ABSTRACTS}

%\def\rightkol{\ } %ENGLISH ABSTRACTS}


\titele{\tit}{\aut}{\auf}{\leftkol}{\rightkol}

\noindent
The problem of biometric system adjusting to the variable operational environment is considered. 
The impact of operational distortion is analyzed with automatic fingerprint 
identification as an example. The general approach for operation distortion registration is suggested. Based on this 
approach, the method for adjusting of a biometric system was developed. The author's experiments showed 
the effectiveness of the suggested approach.

\KWN{biometrics; automatic fingerprint identification; recognition distortions
}

%\vskip 14pt plus 6pt minus 3pt

%2
%\def\tit{MULTI-CHANNEL QUEUEING SYSTEM WITH REFUSALS OF SERVERS GROUPS}


%\def\aut{A.~Pechinkin$^1$, I.~Sokolov$^2$, and V.~Chaplygin$^1$}
%\def\auf{$^1$IPI RAN,apechinkin@ipiran.ru\\[1pt]
%$^2$IPI RAN, isokolov@ipiran.ru\\[1pt]
%$^3$IPI RAN, VasilyChaplygin@mail.ru
%}


%\def\leftkol{\ } %ENGLISH ABSTRACTS}

%\def\rightkol{\ } %ENGLISH ABSTRACTS}


%\titele{\tit}{\aut}{\auf}{\leftkol}{\rightkol}

%\noindent
%The multi-channel queueing system SM/PH/n/r ($r\leq \infty$) with unreliable servers and 
%their group refusals is under consideration. The refusals and the restorations of servers groups 
%occur with a constant intensity, the number of servers refusing simultaneously is a stochastic 
%value, and customers with the interrupted servicing begin its servicing anew after server 
%restoration. The methods are offered to calculate the stationary distribution of the number of the 
%customers in the system under different variants of the functioning of the system.

%\KWN{multi-channel queueing systems; unreliable servers; refusals and restorations of 
%servers groups}

\pagebreak

%\vskip 8pt plus 6pt minus 3pt


%3
\def\tit{APPROXIMATE METHOD OF CALCULATION OF NODE 
CHARACTERISTICS IN~TELECOMMUNICATION NETWORK 
WITH REPETITIVE TRANSMISSIONS}

\def\aut{Ya.\,M.~Agalarov}
\def\auf{IPI RAN, agglar@yandex.ru}

\def\leftkol{\ } %ENGLISH ABSTRACTS}

\def\rightkol{\ } %ENGLISH ABSTRACTS}

\titele{\tit}{\aut}{\auf}{\leftkol}{\rightkol}

\vspace*{-2pt}

\noindent
A model of packet commutation node with repetitive 
transmissions for two schemes of buffer memory distributions~--- fully accessible and 
fully shared~--- is considered. The approximate method of calculation of stream intensity and 
probabilities of node blocking is proposed.
  The necessity and sufficiency conditions for existence and unicity of 
solution of equation were derived for streams in node when operating regime became 
established, and the convergence of iteration method for solving proposed 
equation was proved. 

\KWN{packet commutation node; buffer memory; repetitive transmissions; 
probabilities of blocking; iteration method}

\vskip 8pt plus 6pt minus 3pt


%
\def\tit{SOME IMPLEMENTATION ASPECTS OF THE CONNECTING MEDIUM IN 
THE DECENTRALIZED PACKET SWITCHING ARCHITECTURE}

\def\aut{V.\,B. Egorov}
\def\auf{IPI RAN, vegorov@ipiran.ru}


%\def\leftkol{\ } %ENGLISH ABSTRACTS}

%\def\rightkol{\ } %ENGLISH ABSTRACTS}

\titele{\tit}{\aut}{\auf}{\leftkol}{\rightkol}

\vspace*{-2pt}

\noindent
Specific requirements to the connecting medium in the decentralized packet 
switching architecture are defined. Some medium implementation aspects are discussed. 
Special attention is paid to high-speed serial interfaces used as a connecting medium, with a 
reduced version of such an interface assigned to packet switching being proposed.

\KWN{packet switch; decentralized switching; integrated communications controller; 
superlocal network; high-speed serial interface}
%\pagebreak

%\vful

 \vskip 8pt plus 6pt minus 3pt

%5
\def\tit{TECHNOLOGICAL SYSTEM FOR AUTOMATIC PROCESSING OF~THREE-DIMENSIONAL LIDAR DATA}

\def\aut{V.\,A.~Sukhomlin$^1$ and I.\,N.~Gorkavyi$^2$}

\def\auf{$^1$IPI RAN, sukhomlin@mail.ru\\[1pt]
$^2$Faculty of  Computational Mathematics and Cybernetics,\\
$\hphantom{^1}$M.\,V.~Lomonosov Moscow State University, ilya\_gor@rambler.ru}

\def\leftkol{ENGLISH ABSTRACTS}
%
%\def\rightkol{ENGLISH ABSTRACTS}

%\def\leftkol{\ } %ENGLISH ABSTRACTS}

%\def\rightkol{\ } %ENGLISH ABSTRACTS}

\titele{\tit}{\aut}{\auf}{\leftkol}{\rightkol}

\vspace*{-2pt}

\noindent
New approach to development of software suites for automatic processing of 
LIDAR data is considered. The elaborated algorithms for classification of three-dimensional (3D) data 
and generation of high quality models of bare-Earth relief are described, tools for 
visualization of 3D models and interactive correction of models by operator are 
presented. The paper discusses goals, achieved results, and perspectives of new 
approach.

\KWN{automatic data processing; classification methods; three-dimensional 
models; LIDAR; laser scanning}

%\vskip 18pt plus 6pt minus 3pt

\pagebreak

% \vskip 8pt plus 6pt minus 3pt


\def\tit{SEMIOTIC MODEL FOR COMPUTER CODING CONCEPTS AND INFORMATION OBJECTS}

%6
\def\aut{I.\,M.~Zatsman}

\def\auf{IPI RAN, iz\_ipi@a170.ipi.ac.ru}


\def\leftkol{ENGLISH ABSTRACTS}

\def\rightkol{ENGLISH ABSTRACTS}

\titele{\tit}{\aut}{\auf}{\leftkol}{\rightkol}

\vspace*{-2pt}

\noindent
The semiotic model, which has been developed during research of problems of 
generation and evolution of goal-oriented knowledge systems in digital libraries 
and other kinds of information systems, is considered. These problems concern to 
the new direction of the researches which has been named ``Cognitive Informatics.'' 
The semiotic model suggested is intended for the description of computer coding 
concepts and information objects. This model is positioned as theoretical 
foundations for development of methods of the three-component coding denotata, 
concepts, and information objects, including construction of one-to-one 
correspondence between concepts of goal-oriented knowledge systems and 
computer codes.

\label{st\stat}


\KWN{semiotic model; goal-oriented knowledge systems; denotata; concepts; information 
objects; computer codes; three-component coding; 
one-to-one correspondence between concepts and computer codes}




%\vfil


% \pagebreak

% \label{end\stat}