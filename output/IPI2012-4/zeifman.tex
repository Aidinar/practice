%\newcommand{\vp}{{\mathbf p}}
%\newcommand{\vz}{{\mathbf z}}
%\newcommand{\vf}{{\mathbf f}}
%\newcommand{\vx}{{\mathbf x}}
%\newcommand{\A}{{\mathbf A}}
%\newcommand{\Pr}{\mathrm{Pr}}

\def\stat{zeifman}

\def\tit{ОЦЕНКИ В НУЛЬ-ЭРГОДИЧЕСКОМ СЛУЧАЕ ДЛЯ~НЕКОТОРЫХ~СИСТЕМ ОБСЛУЖИВАНИЯ$^*$}

\def\titkol{Оценки в нуль-эргодическом случае для некоторых систем обслуживания}

\def\autkol{А.\,И.~Зейфман,  А.\,В.~Коротышева, Я.\,А.~Сатин, С.\,Я.~Шоргин}

\def\aut{А.\,И.~Зейфман$^1$,  А.\,В.~Коротышева$^2$, Я.\,А.~Сатин$^3$, С.\,Я.~Шоргин$^4$}

\titel{\tit}{\aut}{\autkol}{\titkol}

{\renewcommand{\thefootnote}{\fnsymbol{footnote}}\footnotetext[1]
{Исследование поддержано РФФИ, гранты 11-01-12026-офи-м, 12-07-00115-а, 12-07-00109-а.}}

\renewcommand{\thefootnote}{\arabic{footnote}}
\footnotetext[1]{Вологодский государственный
педагогический университет;  Институт проблем информатики Российской
академии наук и ИСЭРТ РАН, a\_zeifman@mail.ru}
\footnotetext[2]{Вологодский государственный педагогический
университет, a\_korotysheva@mail.ru}
\footnotetext[3]{Вологодский государственный педагогический
университет, yacovi@mail.ru}
\footnotetext[4]{Институт проблем информатики Российской академии наук, SShorgin@ipiran.ru}


\Abst{Рассматриваются модели обслуживания с групповым поступлением и обслуживанием требований. 
Получены оценки скорости сходимости в нуль-эр\-го\-ди\-че\-ском случае. Рассмотрен пример конкретного 
класса таких сис\-тем обслуживания.}


\KW{нестационарные системы обслуживания с групповым поступлением и обслуживанием требований; 
нуль-эр\-го\-дич\-ность; оценки} 

\vskip 14pt plus 9pt minus 6pt

      \thispagestyle{headings}

      \begin{multicols}{2}

            \label{st\stat}


\section{Введение}

Напомним, что марковская цепь $X\hm=X(t)$, $t\hm\geq 0$, с непрерывным временем и 
дискретным пространством состояний называется нуль-эр\-го\-дич\-ной, если при любых 
начальных условиях и каждом~$i$ выполняется условие 
$\mathrm{Pr}\left\{ X(t)\hm=i\right\} \hm\to 0$ при $t \hm\to \infty$.

Понятие нуль-эргодичности (null ergodicity) и соответствующие оценки
скорости сходимости изуча\-лись для моделей массового обслуживания,
описываемых процессами рождения и гибели начиная с 1990-х~гг.~[1--5].

В настоящей работе оценки, относящиеся к нуль-эр\-го\-ди\-че\-ской ситуации,
будут исследованы для более общего класса марковских систем
обслуживания  с групповым поступлением и обслуживанием требований.
Изучение этого класса моделей начато в работах~\cite{z11, z12}.
Отметим еще, что в таких системах, как будет описано далее, возможны
и катастрофические сбои,  такого рода системы массового обслуживания
(СМО c катастрофами, queues with disasters, queues with
catastrophes) в разных ситуациях и при разных предположениях
изучались ранее  (см., например,~[8--17]).

Пусть $X=X(t)$, $t\geq 0$,~--- число требований в системе обслуживания 
($0 \hm\le X(t)  \hm< \infty $).

Обозначим через 
\begin{multline*}
p_{ij}(s,t)=\mathrm{Pr}\left\{ X(t)=j\left| X(s)=i\right.
\right\},\\
i,j \ge 0\,,\quad 0\leq s\leq t\,,
\end{multline*}
переходные вероятности
процесса $X=X(t)$, а через  $p_i(t)\hm=\mathrm{Pr}\left\{ X(t) \hm=i \right\}$~---
его вероятности состояний.

Будем всюду далее предполагать, что
\begin{multline*}
\mathrm{Pr}\left(X\left( t+h\right) =j/X\left( t\right) =i\right) = {}\\
{}=
\begin{cases}
q_{ij}\left( t\right)  h+\alpha_{ij}\left(t, h\right) & \mbox {при }j\neq i\,;\\ 
1-\sum\limits_{k\neq i}q_{ik}\left( t\right)  h+\alpha_{i}\left(
t,h\right) & \mbox {при } j=i,
\end{cases}
%\label{0101}
\end{multline*}
причем все $\alpha_{i}(t,h)$ есть $o(h)$ равномерно по~$i$, т.\,е.\ $\sup\limits_i
|\alpha_i(t,h)| = o(h)$.

При этом предполагаем, что при $k\hm>0$ $q_{i,i+k}\left( t\right)
\hm=\lambda_k(t)$, $q_{i+k,i}\left( t\right) \hm= \mu_k(t)$.

Другими словами, будем исследовать системы обслуживания, 
в которых интенсивности поступления и обслуживания $k$ требований в момент~$t$ 
в системе обслуживания ($\lambda_{k}(t)$ и  $\mu_{k}(t)$ соответственно)  
не зависят от числа требований, находящихся в системе в момент~$t$, 
причем $\lambda_{k+1}(t) \hm\le \lambda_{k}(t)$ и  $\mu_{k+1}(t) \hm\le \mu_{k}(t)$ при всех~$k$ 
и почти при всех $t \hm\ge 0$.

Отметим, что в рассматриваемой системе, как легко видеть,  
интенсивности обслуживания $\mu_{k}(t)$ являются и интенсивностями катастроф, 
так что если при некоторых $k,t$ выполняется неравенство $\mu_{k}(t) \hm> 0$, то при этих $k,t$ 
возможен катастрофический сбой системы обслуживания с потерей всех требований.

Далее в соответствии со стандартным подходом (см.\ подробное описание
в работах~[3--5, 18] предположим дополнительно, что
все интенсивности являются линейными комбинациями конечного числа
локально интегрируемых на $[0,\infty)$ функций. Кроме того, будем
предполагать, что  почти при всех $t\hm\ge 0$
\begin{equation*}
L_{\lambda}(t)+L_{\mu}(t) =L(t) < \infty\,,
\end{equation*}
 где 
 \begin{equation*}
L_{\lambda}(t)=\sum\limits_{i=1}^{\infty}\lambda_i(t)\,;\enskip  
L_{\mu}(t)=\sum\limits_{i=1}^{\infty}\mu_i(t)\,.
%\label{0102}
\end{equation*}

Тогда для описания вероятностной динамики процесса получаем прямую систему 
Колмогорова в виде дифференциального уравнения в пространстве последовательностей~$l_1$:
\begin{equation} 
\label{ur01}
\fr{d\vp}{dt}=A(t)\vp(t)\,,
\end{equation}
где
{\small %footnotesize
\begin{multline*}
A(t)={}\\
{}=\left(
\begin{array}{cccccccc}
a_{00}(t) & \mu_1(t)  & \mu_2(t)   & \mu_3(t)  &  \cdots & \mu_r(t) & \cdots\\
\la_1(t)   & a_{11}(t)  & \mu_1(t)  & \mu_2(t)   &  \cdots & \mu_{r-1}(t) & \cdots\\
\la_2(t)  & \la_1(t)    & a_{22}(t)& \mu_1(t)  &   \cdots & \mu_{r-2}(t) & \cdots\\
\cdots &\cdots&\cdots&\cdots&\cdots&\cdots&\cdots\\
\la_r(t) & \la_{r-1}(t) & \la_{r-2}(t) & \cdots &  \la_1 (t)   &  a_{rr}(t) & \cdots\\
\cdots &\cdots&\cdots&\cdots&\cdots&\cdots&\cdots\\
\end{array}
\right)\,,
\end{multline*}
}
причем  $a_{ii}(t)=-\sum\limits_{k=1}^{i}\mu_k(t) \hm- \sum\limits_{k=1}^{\infty} \la_{k}(t)$.

Далее будем обозначать через $\|\bullet\|$  $l_1$-нор\-му, т.\,е.\ 
$\|{\vx}\|\hm=\sum|x_i|$, а $\|B\| \hm= \sup\limits_j \sum\limits_i |b_{ij}|$, 
если $B \hm= (b_{ij})_{i,j=0}^{\infty}$, и пусть $\Omega$~--- множество всех векторов из~$l_1$ 
с неотрицательными координатами и единичной нормой.

Тогда имеем $\|A(t)\| \hm= 2\sum\limits_{k=1}^{\infty}(\la_{k}(t)\hm+ \mu_k(t)) \hm\le 2 L(t)$ 
почти при всех $t\hm \ge 0$, следовательно, операторная функция~$A(t)$ ограничена почти при 
всех $t \hm\ge 0$ и локально интегрируема на $[0;\infty)$.

Как известно (см., например,~\cite{DK}), задача Коши для уравнения~(\ref{ur01}) 
тогда имеет единственное решение при любых начальных
условиях, и из $\vp(s) \hm\in \Omega$ вытекает  $\vp(t)\hm \in \Omega$ при
$t \hm\ge s \hm\ge 0$.

Через $E(t,k) = E\left\{X(t)\left|X(0)=k\right.\right\}$ будем далее
обозначать математическое  ожидание процесса (среднее число
требований) в момент~$t$ при условии, что в нулевой момент времени
он находится в состоянии~$k$.

\section{Общий подход и оценки}

Положим
\begin{multline}
\nu(t) ={}\\
\hspace*{-1mm}{}= \inf\limits_{i \ge 0} 
\left(\! |a_{ii}(t)| -\!
\sum\limits_{k=1}^{i}\fr{d_{i-k}}{d_i}\,\mu_k(t) -\! 
\sum\limits_{k=1}^{\infty} \fr{d_{i+k}}{d_i}\,\la_{k}(t)
 \!\right)\!.\hspace*{-2mm}
\label{2001}
\end{multline}

\smallskip


\noindent
\textbf{Теорема~1.}
\textit{Пусть для процесса $X\left( t\right) $ можно найти последовательность $\{d_i\}$
положительных чисел такую, что  $d_{-1}\hm=d_{0}\hm=1$, $\sup\limits_{i \ge
1} d_i \hm= {\rm d}\hm<\infty$, и выполнено условие}

\noindent
\begin{equation*}
 \int\limits_{0}^{\infty} \nu(t) \, dt = + \infty\,.
%\label{2002}
\end{equation*}
\textit{Тогда  $X\left( t\right) $ нуль-эр\-го\-ди\-чен и при любых $0 \hm\le s \hm\le t$ 
и любом~$n$ выполняется неравенство}
\begin{equation*}
\sum_{i=0}^{\infty} d_i p_i (t) \le
 {\rm d} e^{-\int\limits_s^t \nu(\tau) \, d\tau} \,.  
% \label{2003}
\end{equation*}


\medskip

\noindent
Д\,о\,к\,а\,з\,а\,т\,е\,л\,ь\,с\,т\,в\,о\,.\ \
Введем диагональную матрицу следующего вида:
\begin{equation*}
D= \mathrm{diag} \left(d_0, d_1, \dots \right) \,, 
%\label{2004}
\end{equation*}
тогда, очевидно,
\begin{equation*}
D^{-1}= \mathrm{diag} \left(d_0^{-1}, d_1^{-1}, \dots \right)\,.  
%\label{2005}
\end{equation*}

Рассмотрим теперь пространство последовательностей~$\cal B$ 
таких, что 
$$
\|{\bf x}\|_{\cal B} \hm= \|D{\bf x}\|_{1} \hm= \sum\limits_{i=0}^{\infty} d_i
|x_i| \hm< \infty\,,
$$
и прямую систему Колмогорова~(\ref{ur01}) как
уравнение в~$\cal B$. Имеем 
$$
\|A(t)\|_{\cal B} \hm= \|DA(t)D^{-1}\|_{1}
$$ 
и с учетом~(\ref{2001}) и структуры матрицы~$A(t)$ получаем:
\begin{multline*}
\|A(t)\|_{\cal B} ={}\\
{}= \sup\limits_{i \ge 0} \!\left(\! |a_{ii}(t)| +
\sum\limits_{k=1}^{i}\fr{d_{i-k}}{d_i}\,\mu_k(t) + 
\sum\limits_{k=1}^{\infty} \fr{d_{i+k}}{d_i}\,\la_{k}(t)
 \!\right) = {}\\
 {}
= \sup\limits_{i \ge 0} \left( a_{ii}(t) +
\sum\limits_{k=1}^{i}\fr{d_{i-k}}{d_i}\,\mu_k(t) + \sum\limits_{k=1}^{\infty} 
\fr{d_{i+k}}{d_i}\,\la_{k}(t) +{}\right.\\
\hspace*{-6pt}\left.{}+ 2|a_{ii}(t)|
\vphantom{\sum\limits_{k=1}^\infty}\! \right) 
 \le -\nu(t) + 2 \sup\limits_{i \ge 0} \left| a_{ii}(t) \right| = 
 -\nu(t)+\|A(t)\|_{1}.\hspace*{-7.18808pt}
% \label{2006}
\end{multline*}
Следовательно, операторная функция~$A(t)$ ограничена почти при всех $t \hm\ge 0$ и в пространстве
последовательностей~$\cal B$.

Для получения нужных оценок потребуется понятие логарифмической
нормы операторной функции и связанные с ней оценки оператора Коши
линейного дифференциального уравнения в банаховом пространстве.
Соответствующий подход к исследованию в случае процессов рождения и
гибели подробно описан в~\cite{z95, z08b}, а для конечномерного
стационарного случая см.\ также обсуждение и интересные исторические
замечания в~\cite{dzp}.

Рассмотрим логарифмическую норму
$\gamma \left(A(t)\right)$ в~$\cal B$:
\begin{multline*}
\gamma \left(A(t)\right)_{1} = 
\gamma \left(DA(t)D^{-1}\right)_{\cal B} = {} \\ 
{}=  \sup\limits_{i \ge 0} \left( a_{ii}(t) +\sum\limits_{k=1}^{i}\fr{d_{i-k}}{d_i}\,\mu_k(t) 
+ \sum\limits_{k=1}^{\infty} \fr{d_{i+k}}{d_i}\,\la_{k}(t)  \right) ={}\\
{}= - \nu (t)\,. 
%\label{2007}
\end{multline*}

Как известно (см., например,~\cite{z08b}), отсюда вытекает оценка:
\begin{equation*}
\|U(t,s)\|_{\cal B} \le e^{-\int\limits_s^t \nu(\tau) \, d\tau} \,,
%\label{2008}
\end{equation*}
где $U(t,s)$~--- оператор Коши уравнения~(\ref{ur01}).
А тогда при любых $0 \hm\le s \hm\le t$, если $\vp(s) \hm\in \Omega$, получаем неравенство:
\begin{multline*}
\|{\bf p} (t)\|_{\cal B} = \sum\limits_{i=0}^{\infty} d_i p_i (t) \le{}\\
{}\le
\|{\bf p} (s)\|_{\cal B} e^{-\int\limits_s^t \nu(\tau) \, d\tau} \le 
{d} e^{-\int\limits_s^t \nu(\tau) \, d\tau} \,.  
%\label{2009}
\end{multline*}

\bigskip

\noindent
\textbf{Следствие~1.}
Пусть выполнены условия теоремы~1. Тогда при любых $0 \hm\le s \hm\le t$ и любых $n, k$ 
справедливы следующие оценки для вероятности числа требований в системе обслуживания:
\begin{align}
\mathrm{Pr} \left(X(t) \le n \right) &\le 
\fr{d}{\partial_n }\,e^{-\int\limits_s^t \nu(\tau) \, d\tau}\,;
\notag\\
\mathrm{Pr} \left(X(t)  \le n / X(s) = k \right) &\le \fr{d_k}{\partial_n}\,
e^{-\int\limits_s^t \nu(\tau) \, d\tau}\,,
\label{2011}
\end{align}
где ${\partial}_n = \min\limits_{i \le n} d_i$.

\medskip

Для доказательства достаточно отметить, что
\begin{equation*}
{{\partial}_n} \mathrm{Pr} \left(X(t)  \le n \right)  \le  \sum\limits_{i=0}^{n} d_i p_i (t) \le
\|{\bf p} (t)\|_{\cal B}  
%\label{2012}
\end{equation*}
при любых начальных условиях и
\begin{equation*}
\|{\bf p} (s)\|_{\cal B} = \sum\limits_{i=0}^{\infty} d_i p_i (s) = d_k\,, 
%\label{2013}
\end{equation*}
если $X(s) = k$, соответственно.

\bigskip

\noindent
\textbf{Следствие~2.}
Пусть выполнены условия теоремы~1. Тогда при любом $t \hm\ge 0$ и любых $k,n$ 
справедлива следующая оценка для среднего количества требований в системе обслуживания:
\begin{equation*}
E(t;k) \ge \left(n+1\right) \left(1 - \fr{d_k}{{\partial}_n}
e^{-\int\limits_0^t \nu(\tau) \, d\tau}\right)\,.  
%\label{2014}
\end{equation*}


\noindent
Д\,о\,к\,а\,з\,а\,т\,е\,л\,ь\,с\,т\,в\,о\,.\
Из~(\ref{2011}) получаем оценку:
\begin{equation*}
\sum_{i \ge n+1} p_i (t) = 1 - \sum\limits_{i=0}^{n } p_i (t) \ge 1 -
\fr{d_k}{{\partial}_n}\, e^{-\int\limits_0^t \nu(\tau) \, d\tau}\,, 
%\label{2015}
\end{equation*}
и теперь остается лишь заметить, что  
$$
E(t;k) \ge \left(n+1\right) \sum\limits_{i \ge n+1} p_i (t)\,.
$$

\smallskip

Интересно отметить, что при выполнении прос\-то\-го дополнительного условия математическое 
ожидание числа требований можно ограничить и сверху.

\medskip

\noindent
\textbf{Следствие~3.}
Пусть почти при всех $t \hm\ge 0$ выполнено условие
\begin{equation*}
\sum\limits_{k=1}^{\infty}k\la_k(t) = {\sf L}(t) < \infty\,.
\end{equation*}
Тогда при любом $t \hm\ge 0$ и любом~$k$ справедлива следующая оценка для среднего количества требований в системе обслуживания:
\begin{equation*}
E(t;k) \le E(0;k) + \int\limits_0^t \textsf{L}(\tau) \, d\tau\,.  
%\label{20141}
\end{equation*}

\noindent
Д\,о\,к\,а\,з\,а\,т\,е\,л\,ь\,с\,т\,в\,о\,.\ \ 
С учетом сходимости рас\-смат\-ри\-ва\-емых рядов из системы~(\ref{ur01}) получаем неравенство:
\begin{multline*}
\fr{dE(t;k)}{dt} ={}\\
{}= p_0\sum\limits_{k=1}^{\infty}k\la_k(t) + 
p_1 \left(-\mu_1(t)+\sum\limits_{k=1}^{\infty}k\la_k(t)\right) +{} \\
{}+p_2 \left(-2\mu_2(t)-\mu_1(t)+\sum\limits_{k=1}^{\infty}k\la_k(t)\right) + 
\cdots\le{}\\
{}\le \sum\limits_{k=0}^{\infty}p_k {\sf L}(t) = {\sf L}(t)\,, 
%\label{20151}
\end{multline*}
откуда и вытекает требуемая оценка.

\smallskip

Рассмотрим отдельно простейший стационарный случай.

\smallskip


\noindent
\textbf{Следствие~4.}
Пусть интенсивности поступления и обслуживания требований не зависят от времени. 
Пусть для процесса $X\left( t\right) $ можно найти последовательность $\{d_i\}$
положительных чисел такую, что  $d_{-1}\hm=d_{0}\hm=1$, $\sup\limits_{i \ge
1} d_i \hm= {d}\hm<\infty$, и выполнено условие
\begin{equation*}
\nu = \inf\limits_{i \ge 0} \left( |a_{ii}| -\sum\limits_{k=1}^{i}\fr{d_{i-k}}{d_i}\,\mu_k 
- \sum\limits_{k=1}^{\infty} \fr{d_{i+k}}{d_i}\,\la_{k}
 \right) > 0\,.
%\label{2021}
\end{equation*}
Тогда  $X\left( t\right) $ нуль-эр\-го\-ди\-чен и при любых $t \hm\ge 0$, $n$ и~$k$
выполняются оценки:
\begin{align*}
\sum\limits_{i=0}^{\infty} d_i p_i (t) &\le
 {\rm d} e^{-\nu t}\, ,  
% \label{2022}
\\
\mathrm{Pr} \left(X(t)  \le n \right) &\le \fr{{d}}{{\partial}_n}e^{-\nu t}\,; 
\\
\enskip \mathrm{Pr} \left(X(t)  \le n / X(s) = k \right) &\le \fr{{d_k}}{{\partial}_n}e^{-\nu t}
%\label{2023}
\\
E(t;k) &\ge \left(n+1\right) \left(1 - \fr{d_k}{{\partial}_n}\,
e^{-\nu t}\right)\,.  
%\label{2024}
\end{align*}


\section{Система с групповым поступлением и обслуживанием в специальном случае}

Рассмотрим здесь более подробно свойства эргодичности для сис\-те\-мы массового 
обслуживания следующего типа.

Будем предполагать, что требования в систему поступают группами, интенсивность 
поступления новых требований (не более~$m$ одновременно) равна~$\lambda(t)$, а 
обслуживаются также группами (не более $n$ требований одновременно) с интенсивностями~$\mu (t)$.

Тогда число требований в СМО описывается рассмотренной в предыдущем параграфе марковской 
цепью~$X(t)$ с интенсивностями
\begin{equation*}
\lambda_{k}(t)= \begin{cases}
\lambda\left( t\right)   & \mbox {при } k \le m\,;  \\
0 & \mbox {при } k > m\,;
\end{cases}
\label{3001}
\end{equation*}
\begin{equation*}
\mu_{k}(t)=  \begin{cases}
\mu\left( t\right)   & \mbox {при } k \le n\,;  \\
0 & \mbox {при } k > n\,.
\end{cases}
% \label{3002}
\end{equation*}

%\smallskip

\noindent
\textbf{Теорема~2.}
\textit{Пусть для процесса $X\left( t\right) $ можно найти число  $\delta < 1$ такое, что}
\begin{equation}
 \int\limits_{0}^{\infty} \nu^*(t) \, dt = + \infty\,,
\label{3003}
\end{equation}
\textit{где}
\begin{multline*}
\nu^*(t) =  \left( 1- \delta\right)\left(\vphantom{\fr{\delta^n}{\delta^n}}
\left\{ \vphantom{\delta^2} 1 + 
\left(1+\delta\right)+\cdots{}\right.\right.\\
\left.{}\cdots +\left(
1+\delta+ \cdots +\delta^{m-1}\right)\right\}\la(t) -{}\\  
{}- \left. \left\{\fr{1}{\delta} + \fr{1+\delta}{\delta^2}+
\dots+\fr{1+\delta+\dots+\delta^{n-1}}{\delta^n}\right\}\mu(t) \right)\,.
%\label{3004}
\end{multline*}
\textit{Тогда  $X\left( t\right) $ нуль-эр\-го\-ди\-чен и при любых $0 \hm\le s \hm\le t$ и 
любых $k, r$ выполняются неравенства}
\begin{equation*}
\sum_{i=0}^{\infty} d_i p_i (t) \le
e^{-\int\limits_s^t \nu^*(\tau) \, d\tau} \,,  
%\label{3005}
\end{equation*}
\textit{а также}
\begin{align*}
\mathrm{Pr} \left(X(t)  \le k \right) &\le \delta^{-k}e^{-\int\limits_s^t \nu^*(\tau) \, d\tau}\,;
\\
\mathrm{Pr} \left(X(t)  \le k / X(s) = r \right) &\le \delta^{r-k}e^{-\int\limits_s^t \nu^*(\tau) \, d\tau}\,.
%\label{3007}
\end{align*}


\noindent
Д\,о\,к\,а\,з\,а\,т\,е\,л\,ь\,с\,т\,в\,о\,.\ \ 
Достаточно положить $d_k \hm= \delta ^k$, тогда выполнены условия теоремы~1 и следствий, 
причем  ${d}\hm= 1$,  ${\partial}_k \hm= \delta ^{k}$.

\medskip

\noindent
\textbf{Замечание~1.}
Если интенсивности поступления и обслуживания требований не зависят от 
времени (стационарный случай), то условие~(\ref{3003}), гарантирующее 
нуль-эр\-го\-дич\-ность, равносильно выполнению неравенства
\begin{equation}
 m\left(m+1\right)\la - n\left(n+1\right)\mu > 0\,,
\label{3008}
\end{equation}
а при $1$-периодических интенсивностях~--- тому, что
\begin{equation}
\hspace*{-4mm}m\left(m+1\right)\int\limits_0^1\la(t) \, dt  - 
 n\left(n+1\right)\int\limits_0^1\mu(t) \, dt > 0.
\label{3009}
\end{equation}
Действительно, при выполнении неравенств~(\ref{3008}) и~(\ref{3009}) условие~(\ref{3003}) 
заведомо будет выполнено при достаточно малых $1\hm-\delta \hm>0$.

\smallskip

Отметим теперь, что полученные условия нуль-эр\-го\-дич\-ности рас\-смот\-рен\-ной сис\-те\-мы 
оказываются достаточно точными, как показывает следующее утверждение.


\medskip

\noindent
\textbf{Теорема~3.}
\textit{Пусть для процесса $X\left( t\right) $ можно найти число  $d\hm > 1$ такое, что
\begin{equation}
 \int\limits_{0}^{\infty} \beta^*(t) \, dt = + \infty,
\label{3011}
\end{equation}
где}
\begin{multline*}
\beta^*(t) =  \left( d-1\right)\left(\left\{\fr{1}{d} + \fr{1+d}{d^2} + 
\cdots{}\right.\right.\\
\left.{}\cdots +\fr{1+d+\dots+d^{n-1}}{d^n}\right\}\mu(t) -{} \\ 
{} - \left. \left\{1 + \left(1+d\right)+\dots+\left(1+d+\dots+d^{m-1}\right)\right\}\la(t) 
\vphantom{\fr{1}{d^2}}\right).\hspace*{-4.43994pt}
%\label{3012}
\end{multline*}
\textit{Тогда  $X\left( t\right) $ слабо эргодичен и}
\begin{multline}
\|\vp^*(t)-\vp^{**}(t)\|_{1} \le{}\\
{}\le 4 e^{-\int\limits_s^t \!{\beta^*(u)\,du}}
\sum\limits_{i \ge 1} g_i\left|p_i^*(s)-p_i^{**}(s)\right|\hspace*{-0.34201pt}
 \label{3012'}
\end{multline}
\textit{при любых начальных условиях ${\bf p^*}(s), {\bf p^{**}}(s)$ и любых $s,t$, 
$0\hm\le s\hm\le t$, где $g_i\hm=\sum\limits_{n=1}^{i}d^{n-1}$.}


\smallskip

\noindent
Д\,о\,к\,а\,з\,а\,т\,е\,л\,ь\,с\,т\,в\,о\,.\ \
Воспользуемся методикой, описанной в~\cite{z11,z12}.

Полагая $p_0\hm=1-\sum\limits_{i \ge 1}{p_i}$,  из уравнения~(\ref{ur01}) получим
\begin{equation}
\fr{d\vz}{dt}= B(t)\vz(t)+\vf(t)\,, 
\label{3021}
\end{equation}
где вектор $\vf(t)=\left(\la, \cdots, \la, 0,\dots \right)^{\mathrm{T}}$ имеет $m$ 
первых ненулевых координат, элементы первых $m$ строк матрицы 
$B(t)\hm=\left(b_{ij}(t)\right)_{i,j=1}^{\infty}$ получаются вычитанием $\lambda(t)$ из 
соответствующих элементов~$A(t)$, а элементы остальных строк совпадают с со\-от\-вет\-ст\-ву\-ющи\-ми 
элементами матрицы исходной сис\-темы.
{\looseness=1

}

Положим $d_k = d^{k-1}$ и рассмотрим треугольную матрицу вида
\begin{equation*}
H=\left(
\begin{array}{ccccccc}
d_1   & d_1 & d_1 & \cdots  \\
0   & d_2  & d_2  &   \cdots  \\
0   & 0  & d_3  &   \cdots  \\
\ddots & \ddots & \ddots &\ddots\\
%0& 0 & \ddots & \ddots \\
\end{array}
\right)\,,
\end{equation*}
а также пространство последовательностей  
$$
l_{1H}=\left\{{\bf z} = (p_1,p_2,\dots)^{\mathrm{T}} / \|{\bf z}\|_{1H}\equiv \|H {\bf z}\|_1 <\infty
\right\}\,.
$$

Рассмотрим~(\ref{3021}) как дифференциальное уравнение в пространстве $l_{H}$, 
при этом несложно убедиться, что $B(t)$ и ${\bf f}(t)$ локально интегрируемы  при $t \hm\ge 0$.

Оценим логарифмическую норму
\begin{equation*}
\gamma \left(B(t)\right)_{1H} = \gamma \left(H B(t)H^{-1}\right)_{1} \,. 
%\label{3022}
\end{equation*}

Имеем
{\scriptsize
\begin{multline*}
H B H^{-1}={}\\
\!\!{}=\left(\!
\begin{array}{cccccc}
a_{11}  &  (\mu_1-\mu_2) \fr{d_1}{d_2}  & (\mu_2-\mu_3)\fr{d_1}{d_3}  & \cdots &  (\mu_{r-1}-\mu_r)\fr{d_1}{d_r} & \cdots \\
\la_1 \fr{d_2}{d_1} &  a_{22}  &(\mu_1-\mu_3)\fr{d_2}{d_3}  & \cdots &  (\mu_{r-2}-\mu_r)\fr{d_2}{d_r} & \cdots \\
\la_2 \fr{d_3}{d_1} &  \la_1\fr{d_3}{d_2}   &a_{33}   & \cdots &  (\mu_{r-3}-\mu_r)\fr{d_3}{d_r}  & \cdots \\
\cdots & \cdots & \cdots& \cdots & \cdots &\cdots\\
\la_{r-1} \fr{d_r}{d_1} & \la_{r-2} \fr{d_r}{d_2}  & \la_{r-3} \fr{d_r}{d_3}  & \cdots & a_{rr} & \cdots\\
 \cdots & \cdots & \cdots& \cdots& \cdots& \cdots \\
\end{array}\!\!
\right), \hspace*{-7.94606pt}
%\label{3023}
\end{multline*}}

\noindent
где $\la_k(t)=\la(t)$, $1 \hm\le k \hm\le m$,  $\mu_k(t)\hm=\mu(t)$, $1 \hm\le k \hm\le n$, 
а остальные $\la_k(t)$, $\mu_k(t)$~--- тождественные нули.

\columnbreak

Тогда получаем:

\noindent
\begin{multline*}
\gamma \left(B(t)\right)_{1H} = \sup \left(\textsf{b}_{ii}(t)  + 
\sum\limits_{j \neq i}\textsf{b}_{ji}(t)\right) = {}  \\[3pt]
{}= -\left( d-1\right)\left(\left\{\fr{1}{d} + \fr{1+d}{d^2} + \cdots{}\right.\right.\\[4pt]
\left.{}\cdots +
\fr{1+d+\cdots+d^{n-1}}{d^n}\right\}\mu(t) - 
 \left\{ \vphantom{d^{m-1}}
 1 + \left(1+d\right)+\cdots{}\right.\\[4pt]
\left.\left. {}\cdots+\left(1+d+\cdots+d^{m-1}\right)\right\}\la(t) 
\vphantom{\fr{1}{d}}\right) = -\beta^*(t)\,.  %\label{3024}
\end{multline*}

Как известно (см., например~\cite{z12}), отсюда вытекает оценка:
\begin{equation*}
\|\vp^*(t)-\vp^{**}(t)\|_{1H} \le e^{-\int\limits_s^t {\beta^*(u)du}}\|\vp^*(s)-\vp^{**}(s)\|_{1H}\,,
% \label{3025} 
 \end{equation*}
справедливая при любых $s,t$, $0\hm\le s\hm\le t,$ и любых начальных условиях 
${\bf p^*}(s), {\bf p^{**}}(s)$. Отсюда с учетом неравенства 
$\|{\bf z}\|_{1} \hm\le 2\|{\bf z}\|_{1H}$, полученного ранее при сравнении норм, и 
получаем требуемую оценку~(\ref{3012'}).

\medskip


\noindent
\textbf{Замечание~1.}
Если интенсивности поступления и обслуживания требований не зависят от времени 
(стационарный случай), то условие~(\ref{3011}), гарантирующее в этом случае 
сильную эргодичность, равносильно выполнению неравенства:
\begin{equation*}
 m\left(m+1\right)\la - n\left(n+1\right)\mu < 0\,.
%\label{3031}
\end{equation*}
При $1$-периодических интенсивностях условие~(\ref{3011}) гарантирует 
слабую эргодичность,  существование предельного $1$-пе\-рио\-ди\-че\-ско\-го 
режима и $1$-пе\-рио\-ди\-че\-ско\-го предельного среднего и эквивалентно 
выполнению неравенства
\begin{equation*}
 m\left(m+1\right)\int\limits_0^1\la(t) \, dt  - n\left(n+1\right)\int\limits_0^1\mu(t) \, dt < 0\,.
%\label{3032}
\end{equation*}

\medskip

\noindent
\textbf{Пример~1.}

Рассмотрим конкретный пример системы этого класса при конкретных значениях интенсивностей 
и допустимых размерах групп требований с оценками в нуль-эр\-го\-дич\-ном случае и слабо 
эргодичном случае.

Пусть интенсивности поступления и обслуживания требований есть 
$\lambda(t)\hm=1\hm+\sin2\pi t$ и $\mu(t)\hm=1\hm+\cos2\pi t$ соответственно.

Рассмотрим два случая.

{\bf Случай 1.} Пусть $m \hm= 3$, $n\hm=2$. Полагая, как описано ранее,  $d_k\hm=\delta^{k}$, 
получаем:
\begin{equation*}
\nu^*(t) =  \left(3  -\delta -\delta^2-\delta^3\right)\la(t)  
+ \left(2 -\delta^{-1} -\delta^{-2}\right)\mu(t)\,.
%\label{3041}
\end{equation*}
Отметим, что наибольшее значение выражения $f(\delta)\hm=5-\delta \hm-\delta^2\hm-
\delta^3\hm-\delta^{-1} \hm-\delta^{-2}$
на нужном проме-\linebreak\vspace*{-12pt}

\pagebreak

\noindent
жутке достигается при $\delta^* \hm\approx 0{,}84$ и равно $f^* \hm\approx 0{,}254$.
При этом получаем:
\begin{multline*}
\nu^*(t) \approx 0{,}254 +\left(3  -\delta^* -(\delta^*)^2-(\delta^*)^3\right)\sin2\pi t  +{}\\
{}+ \left(2 -(\delta^*)^{-1} -(\delta^*)^{-2}\right)\cos2\pi t\,,
%\label{3042}
\end{multline*}
причем
\begin{multline}
\int\limits_0^t\nu^*(\tau)\,d\tau \ge  0{,}25 t  +\left(3  -\delta^* -
(\delta^*)^2-(\delta^*)^3\right)\times{}\\
{}\times 
\left(\int\limits_0^{[t]}\sin2\pi u \, du + 
\int\limits_{[t]}^{t}\sin2\pi u \, du \right) + {} \\ 
{}+\left(2 -(\delta^*)^{-1} -(\delta^*)^{-2}\right)\left(\int\limits_0^{[t]}
\cos 2\pi u \, du +{}\right.\\
\left.{}+ \int\limits_{[t]}^{t}\cos 2\pi u \, du \right) \ge  0{,}25 t  - \left(5  -\delta^* -(\delta^*)^2-{}\right.\\
\left.{}-(\delta^*)^3- (\delta^*)^{-1} 
-(\delta^*)^{-2}\right) \ge 0{,}25t - 0{,}3\,,
\label{3042}
\end{multline}
откуда заведомо
\begin{equation*}
e^{-\int\limits_0^t\nu^*(\tau)\,d\tau} \le 2 e^{- 0{,}25 t}\,.
%\label{3043}
\end{equation*}

Следовательно, процесс $X(t)$ (число требований в системе) нуль-эр\-го\-ди\-чен 
и при любых $t \hm\ge 0$ и любых $n, r$
выполняются неравенства
\begin{equation*}
\sum\limits_{i=0}^{\infty} {\delta^*}^k p_k (t) \le
2e^{-0{,}25 t}\,,  
%\label{3044}
\end{equation*}
а также
\begin{align*}
\mathrm{Pr} \left(X(t)  \le n \right) &\le 2{\delta^*}^{-n}e^{-0{,}25 t}\,;\\
%\label{3045}
\mathrm{Pr} \left(X(t)  \le n / X(0) = r \right) &\le 2{\delta^*}^{r-n}e^{-0{,}25 t}\,.
%\label{3046}
\end{align*}

%\medskip

{\bf Случай~2.} Пусть $m \hm= 2$, $n\hm=3$.

Аналогично предыдущему случаю рассмотрим теперь выражение
\begin{multline*}
\beta^*(t) =  {}\\
{}=\left(3  -d -d^2\right)\la(t)  + \left(2 -d^{-1} -d^{-2} - d^{-3}\right)\mu(t)\,.
%\label{3051}
\end{multline*}
Отметим, что наибольшее значение выражения 
$g(d)\hm=5\hm-d \hm- d^2 \hm- d^{-1} \hm- d^{-2} \hm- d^{-3}$
на нужном промежутке достигается при $d^* \hm= (\delta^*)^{-1}\hm \approx 1{,}2$ 
и равно $g^* \hm= f^* \hm\approx 0{,}254$.
С учетом оценки~(\ref{3042}) получаем теперь
\begin{equation*}
e^{-\int\limits_0^t\beta^*(\tau)\,d\tau} \le 2 e^{- 0{,}25 t}\,.
%\label{3052}
\end{equation*}
Тогда  $X\left( t\right) $ слабо эргодичен  и
\begin{equation*}
\|\vp^*(t)-\vp^{**}(t)\|_{1} \le 8 e^{-0{,}25 t} 
\sum\limits_{i \ge 1} g_i\left|p_i^*(0)-p_i^{**}(0)\right| 
%\label{3053}
\end{equation*}
при любых начальных условиях ${\bf p^*}(0), {\bf p^{**}}(0)$ и 
любом $t \hm\ge 0$, где $g_i\hm=\sum\limits_{n=1}^{i}(d^*)^{n-1}$.

Кроме того, $X\left( t\right) $ имеет предельный $1$-пе\-рио\-ди\-че\-ский режим и 
$1$-пе\-рио\-ди\-че\-ское предельное среднее, которые можно построить, пользуясь 
методикой, описанной в~\cite{z12, z06}.

{\small\frenchspacing
{%\baselineskip=10.8pt
\addcontentsline{toc}{section}{Литература}
\begin{thebibliography}{99}



\bibitem{z91}  %1
\Au{Zeifman A.\,I.}  Some estimates of the rate of
convergence for birth and death processes~// J.~Appl.
Prob., 1991. Vol.~28. P.~268--277.

\bibitem{z95a}   %2
\Au{Zeifman A.\,I.}  On the estimation of probabilities for
birth and death processes~// J.~Appl. Prob., 1995.
Vol.~32. P.~623--634.

\bibitem{z95}   %3
\Au{Zeifman A.\,I.} Upper and lower bounds on the rate of
convergence for nonhomogeneous birth and death processes~//  Stoch.
Proc. Appl., 1995. Vol.~59. P.~157--173.

\bibitem{gz04} %4
\Au{Granovsky B.\,L., Zeifman A.\,I.} Nonstationary queues:
Estimation of the rate of convergence~// Queueing Syst., 2004.
Vol.~46. P.~363--388.

\bibitem{z08b} 
\Au{Зейфман А.\,И., Бенинг В.\,Е., Соколов~И.\,А.} 
Марковские цепи и модели с непрерывным временем.~--- М.: Элекс-КМ, 2008.

\bibitem{z11} 
\Au{Сатин Я.\,А., Зейфман А.\,И., Коротышева~А.\,В.,  Шоргин~С.\,Я.} 
Об одном классе марковских систем обслуживания~//  Информатика и её применения, 2011. Т.~5. Вып.~4. С.~18--24.

\bibitem{z12} %7
\Au{Сатин Я.\,А., Зейфман А.\,И., Коротышева А.\,В.} 
О~ско\-рости сходимости и усечениях для одного класса марковских систем
обслуживания~//  Теория вероятностей и ее применения, 2012 (в
печати).


\bibitem{du1}  %8
\Au{Dudin~A., Nishimura~S.}  A~BMAP/SM/1 queueing system with Markovian arrival input
of disasters~//
J.~Appl. Prob., 1999. Vol.~36. P.~868--881.

\bibitem{du2} %9
\Au{Dudin~A., Karolik~A.} BMAP/SM/1 queue with Markovian input of disasters 
and non-instantaneous recovery~//
Perform. Eval., 2001. Vol.~45. P.~19--32.

\bibitem{Di}  %10
\Au{Di~Crescenzo~A., Giorno~V., Nobile~A.\,G., Ricciardi~L.\,M.}  On
the $M$/$M$/1 queue with catastrophes and its continuous approximation~// Queueing Syst., 2003. 
Vol.~43. P.~329--347.

\bibitem{du3} 
\Au{Dudin~A., Semenova~O.} Stable algorithm for stationary distribution calculation 
for a BMAP|SM|1 queueing system with markovian input of disasters~// J.~Appl.
Prob.,  2004.  Vol.~42.   P.~547--556.

\bibitem{z08} 
\Au{Zeifman~A., Satin~Ya., Chegodaev~A., Bening~V., Shorgin~V.}
Some bounds for $M(t)/M(t)/S$ queue with catastrophes~// 
4th  Conference (International) on Performance Evaluation
Methodologies and Tools Proceedings.  Athens, Greece, 2008.

\bibitem{z09a} 
\Au{Зейфман А.\,И., Сатин Я.\,А., Чегодаев~А.\,В.} 
О~нестационарных системах обслуживания с катастрофами~// Информатика и её применения, 2009. 
Т.~3. Вып.~1. С.~47--54.

\bibitem{z09b} 
\Au{Зейфман А.\,И., Сатин Я.\,А., Коротышева~А.\,В., Терешина~Н.\,А.} 
О~предельных характеристиках системы обслуживания $M(t)/M(t)/S$ с катастрофами~// 
Информатика и её применения, 2009. Т.~3. Вып.~3. С.~16--22.

\bibitem{z09c} 
\Au{Zeifman A., Satin Ya., Shorgin~S., Bening~V.} 
On $M_n(t)/M_n(t)/S$ queues with catastrophes~//  
4th  Conference (International) on Performance Evaluation Methodologies and Tools Proceedings.  
Pisa, Italy, 2009.

\bibitem{z11b} 
\Au{Зейфман А.\,И., Коротышева А.\,В., Панфилова~Т.\,Л., Шоргин~С.\,Я.} 
Оценки устойчивости для некоторых систем обслуживания с катастрофами~//  
Информатика и её применения, 2011. Т.~5. Вып.~3. С.~27--33.

\bibitem{z12b} 
\Au{Zeifman A., Korotysheva A.} Perturbation bounds for $M_t/M_t/N$ queue with
catastrophes~// Stochastic Models, 2012. Vol.~28. P.~49--62.

\bibitem{z06} 
\Au{Zeifman~A., Leorato~S., Orsingher~E., Satin~Ya., Shilova~G.}
Some universal limits for nonhomogeneous birth and death processes~// 
Queueing Syst., 2006. Vol.~52. P.~139--151.

\bibitem{DK} 
\Au{Далецкий Ю.\,Л., Крейн М.\,Г.} Устойчивость решений
диф\-фе\-рен\-циальных уравнений в банаховом пространстве.~--- М.:
Наука, 1970.

\label{end\stat}

\bibitem{dzp} 
\Au{Ван Доорн~Э.\,А., Зейфман~А.\,И., Панфилова~Т.\,Л.}  
Оценки и асимптотика скорости сходимости для процессов рождения и гибели~// 
Теория вероятностей и ее применения, 2009. Т.~54. С.~18--38.
\end{thebibliography}
}
}


\end{multicols}