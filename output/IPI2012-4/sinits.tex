\def\stat{sinit}

\def\tit{АНАЛИТИЧЕСКОЕ МОДЕЛИРОВАНИЕ РАСПРЕДЕЛЕНИЙ 
С~ИНВАРИАНТНОЙ МЕРОЙ В~СТОХАСТИЧЕСКИХ СИСТЕМАХ 
С~АВТОКОРРЕЛИРОВАННЫМИ ШУМАМИ$^*$}

\def\titkol{Аналитическое моделирование распределений 
с~инвариантной мерой в~стохастических системах 
с %автокоррелированными 
шумами}

\def\autkol{И.\,Н.~Синицын}

\def\aut{И.\,Н.~Синицын$^1$}

\titel{\tit}{\aut}{\autkol}{\titkol}

{\renewcommand{\thefootnote}{\fnsymbol{footnote}}\footnotetext[1]
{Работа поддержана РФФИ (проект № 10-07-00021) и Программой <<Интеллектуальные информационные 
технологии, системный анализ и автоматизация>> (проект~1.7).}}

\renewcommand{\thefootnote}{\arabic{footnote}}
\footnotetext[1]{Институт проблем информатики Российской академии наук, sinitsin@dol.ru}     


\vspace*{-9pt}
     
  \Abst{Для многомерных нелинейных гауссовских (нормальных) 
дифференциальных сис\-тем с некоррелированными и автокоррелированными 
помехами на базе метода нормальной аппроксимации разработаны 
корреляционные алгоритмы аналитического моделирования стохастических 
режимов с инвариантной мерой. На тес\-то\-вых примерах с по\-мощью 
инструментального программного обеспечения в среде MATLAB показана 
достаточная для многих приложений точность алгоритмов.}

\vspace*{-3pt}
  
  \KW{автокоррелированная помеха; аналитическое моделирование; 
корреляционный алгоритм; метод нормальной аппроксимации; многомерная 
нелинейная дифференциальная стохастическая сис\-те\-ма; распределение с 
инвариантной мерой}

\vspace*{-6pt}

\vskip 14pt plus 9pt minus 6pt

      \thispagestyle{headings}

      \begin{multicols}{2}

            \label{st\stat}
     
\section{Введение}
     
    Будем рассматривать в общем случае неста\-ционарный стохастический 
режим $Z\hm=Z(t)$,\linebreak явля\-ющий\-ся сильным решением следующего нормального 
(гауссовского) стохастического дифференциального урав\-не\-ния Ито~[1, 2]:
    \begin{equation}
    \dot{Z}=a(Z,t)+b(Z,t)V\,,\enskip Z(t_0)=Z_0\,.
    \label{e1-sin}
    \end{equation}
Здесь $Z$~--- $k$-мер\-ный вектор со\-сто\-яния $Z\hm\in \Delta$ ($\Delta$~--- 
многообразие со\-сто\-яний), $a\hm=a(Z,t)$ и $b\hm=b(Z,t)$~--- 
детерминированные ($k\times1$)- и ($k\times m)$-функ\-ции отмечен\-ных 
аргументов, $V\hm=V(t)$~---\linebreak $m$-мер\-ный вектор нормально распределенных 
\mbox{белых} шумов с нулевыми математическими ожиданиями и 
($m\times m$)-мат\-ри\-цей интенсивностей $v\hm=v(t)$ и пред\-став\-ля\-ющий 
собой среднеквадратичную\linebreak производную винеровского процесса $W\hm=W(t)$, %\linebreak 
$V\hm=\dot{W}$. Начальное со\-сто\-яние~$Z_0$ будем считать нормальной 
(гауссовой) случайной величиной, не зависящей от приращений винеровского 
процесса $W(t)$ для $t\hm>t_0$.
    
    Если существуют все многомерные плот\-ности вектора со\-сто\-яния~$Z$, то, 
определив сначала одномерную плот\-ность $f_1\hm=f_1(z;t)$ и переходную 
плот\-ность $f\hm=f(z;t\vert \xi;\tau)$ путем интегрирования урав\-не\-ния 
    Фок\-ке\-ра--План\-ка--Кол\-мо\-го\-ро\-ва (ФПК) с соответствующими 
начальными услови\-ями~[1, 2]:

\noindent
    \begin{multline}
%    \left.
%    \begin{array}{c}
    \fr{\partial f_1}{\partial t} = -\fr{\partial^{\mathrm{T}}}{\partial z}\left( af_1\right) 
+\fr{1}{2}\,\mathrm{tr}\left[ \fr{\partial}{\partial z}\,\fr{\partial^{\mathrm{T}}}{\partial z}\left( 
\sigma f_1\right) \right]\,,\\
    \sigma=b\nu b^{\mathrm{T}}\,,\enskip f_1(z;t_0)=f_0(z)\,;
 %   \end{array}
%    \right\}
    \label{e2-sin}
    \end{multline}
    
%    \vspace*{-9pt}
    
    \noindent
    \begin{multline}
%    \left.
%    \begin{array}{c}
    \fr{\partial f}{\partial t} = -\fr{\partial^{\mathrm{T}}}{\partial z}\left( af\right) 
+\fr{1}{2}\,\mathrm{tr}\left[ \fr{\partial}{\partial z}\,\fr{\partial^{\mathrm{T}}}{\partial z}\left( 
\sigma f\right)\right]\,,\\
    f(z;t\vert \xi;\tau)=\delta(z-\xi)\,,
 %   \end{array}
%    \right\}
    \label{e3-sin}
    \end{multline}
можно найти все многомерные плот\-ности $f_n\hm= f_n(z_1, \ldots , z_n; 
t_1,\ldots , t_n)$ по рекуррентной фор\-муле:

\vspace*{-3pt}

\noindent
\begin{multline*}
f_n=f_n(z_1,\ldots ,z_n; t_1,\ldots ,t_n) ={}\\
{}=f_1(z_1;t_1)f(z_2;t_2\vert z_1;t)\cdots f(z_n;t_n\vert z_{n-1};t_{n-1})\,,\\
t_1\leq t_2\leq \cdots \leq t_n\,, \ 
n=2,3,\ldots
%\label{e4-sin}
\end{multline*}


    
    Для нахождения стационарных в узком смысле одно- и многомерных 
распределений стохастических режимов в стохастических сис\-те\-мах (СтС), 
определяемых~(\ref{e2-sin}), в~(\ref{e3-sin}) следует положить $\partial  
f_1/\partial  t\hm=0$. 
    
    В~[1--12] рассмотрены точные методы аналитического моделирования 
одно- и многомерных плот\-но\-стей стохастических стационарных и 
нестационарных режимов, основанные на построении интегральных 
инвариантов специально подобранных обыкновенных дифференциальных 
уравнений. В~[13] на основе теории потенциала предложены методы расчета 
стационарных распределений с инвариантной мерой для гамильтоновых СтС в 
стохастической среде.
    
    Стохастические сис\-те\-мы с автокоррелированными помехами обычно 
описывают следующими уравнениями Ито~[1, 2]:
%\noindent
    \begin{equation}
    \dot{Z}=a(Z,t)+b_U(z,t)U\,.
    \label{e5-sin}
    \end{equation}
    
%    \pagebreak
    
    
    \noindent
    Здесь векторная помеха~$U$ удовлетворяет следующему стохастическому 
линейному дифференциальному уравнению фильтра, формирующего~$U$ из 
гауссового (нормального) белого шума~$V$ интенсивности $\nu\hm=\nu(t)$:
    \begin{equation}
    \sum\limits_{i=0}^l \alpha_i U^{(i)} =\sum\limits_{j=0}^l \beta_j 
V^{(j)}\enskip (h<l)\,,
    \label{e6-sin}
    \end{equation}
где $\alpha_i$ и $\beta_j$~--- коэффициенты формирующего фильт\-ра (ФФ), в 
общем случае зависящие от времени.
    
    Поставим задачу разработки на основе метода нормальной аппроксимации 
(МНА)~[1, 2] корреляционных алгоритмов аналитического моделирования 
одно- и двумерных нормальных (гауссовских) плот\-но\-стей стохастических 
режимов $Z\hm=Z(t)$ в СтС~(1) и~(\ref{e5-sin}) с инвариантной мерой, т.\,е.\ 
удовлетворяющих условию: 
    \renewcommand{\theequation}{\arabic{equation}$^\prime$}
    \begin{equation}
    \fr{\partial  f_1(z;t)}{\partial t} +\fr{\partial^{\mathrm{T}}}{\partial z}\left[ 
a(z,t)f_1(z;t)\right]=0\,,
     \label{e7-1-sin}
     \end{equation}
     \setcounter{equation}{5}
     
\noindent
где $f_1=f_1(z;t)$~--- одномерная плот\-ность распределения. Для стационарных 
режимов условие~(\ref{e7-1-sin}) принимает вид 
    \renewcommand{\theequation}{\arabic{equation}$^{\prime\prime}$}
    \begin{equation}
    \fr{\partial^{\mathrm{T}}}{\partial z} \left[ a^*(z)f_1^*(z)\right] =0\,,
    \label{e7-2-sin}
    \end{equation}
    \renewcommand{\theequation}{\arabic{equation}}
    \setcounter{equation}{6}
  
  Рассмотрим случаи некоррелированной помехи (разд.~2) и 
автокоррелированной помехи, явля\-ющей\-ся менее (более) гладкой, чем 
стохастический режим (разд.~3).
  
\section{Аналитическое моделирование при~некоррелированных 
помехах}
  
  Пусть нелинейная СтС (1) допускает применение МНА~[1, 2]. Тогда 
одномерная нормальная плот\-ность $f_1^N(z;t)$, вектор математического 
ожидания $m_t\hm={\sf M}Z(t)$, ковариационная мат\-ри\-ца $K_t\hm= {\sf M}X^{0\mathrm{T}}(t) 
Z^0(t)$ и мат\-ри\-ца ковариационных\linebreak функций 
$K(t_1,t_2)\hm={\sf M}Z^{0\mathrm{T}}(t_1)Z^0(t_2)$ $(t_1\hm<t_2)$ опре\-де\-ля\-ют\-ся 
следующими урав\-не\-ни\-ями~[1, 2]:
  \begin{multline}
  f_1^N(z;t,m_t,K_t) =\left[ (2\pi)^k\vert K_t\vert \right]^{-1/2}\times{}\\
  {}\times \exp \left\{ -
\fr{1}{2} \left( z^{\mathrm{T}}-m_t^{\mathrm{T}}\right) K_t^{-1}(z-m_t)\right\}\,;\label{e8-sin}
  \end{multline}
  
  \vspace*{-12pt}
  
  \noindent
  \begin{multline}
     \dot{m}_t=a_1(m_t,K_t,t)={}\\
     {}=\int\limits_{-\infty}^\infty a(z,t)f_1^N 
(z;t,m_t,K_t)\,dz\,;\label{e9-sin}
     \end{multline}
     
     \noindent
     \begin{multline}
     \dot{K}_t =a_2(m_t,K_t,t)=\int\limits_{-\infty}^\infty \left[ a(z,t)(z^{\mathrm{T}}-
m_t^{\mathrm{T}})+{}\right.\\
\hspace*{-1mm}{}+(z-m_t)a^{\mathrm{T}}(z,t)+ %{}\\
\left.\sigma(z,t)\right] f_1^N(z,t,m_t,K_t)\,dz,\!
     \label{e10-sin}
     \end{multline}
     где
     \begin{equation*}
     \sigma(z,t)=b(z,t)\nu(t) b(z,t)^{\mathrm{T}}\,;
     \end{equation*}
     
     \vspace*{-12pt}
     
     \noindent
     \begin{multline}
     \fr{\partial K(t_1,t_2)}{\partial t_2}=a_3(m_{t_1},m_{t_2}, K(t_1,t_2))={}\\
     {}=\left[ 
(2\pi)^{2k}\vert \overline{K}_2\vert \right]^{-1/2}\int\limits_{-\infty}^\infty  
\int\limits_{-\infty}^\infty (z_1-m_{t_1})a(z_{t_1},t_2)\times{}\\
     {}\times 
     \exp \left\{ -\left(\left [ z_1^{\mathrm{T}}\ z_2^{\mathrm{T}}\right] -
     \overline{m}_2^{\mathrm{T}}\right) 
\overline{K}_2^{-1} \times{}\right.\\
\left.{}\times\left( \left[ z_1^{\mathrm{T}}\ z_2^{\mathrm{T}}\right]^{\mathrm{T}} -\overline{m}_2\right) 
\right\}dz_1dz_2\,,
     \label{e11-sin}
     \end{multline}
где
\begin{gather*}
z_1= z_1(t_1)\,;\  z_2=z_2(t_2)\,;\  \overline{m}_2=\left[ 
m^{\mathrm{T}}_{t_1},m^{\mathrm{T}}_{t_2}\right]\,;\\
\overline{K}_2=\begin{bmatrix}
K(t_1,t_1) & \ \ K(t_1,t_2)\\[6pt]
K(t_2,t_{1})&\ \  K(t_2,t_2)
\end{bmatrix}\,.
\end{gather*}
     
     Для стационарной сис\-те\-мы~(1) уравнения для $m_t\hm=m^*\hm=const$, 
$K_t\hm=K^*\hm=const$ получаются из (\ref{e9-sin})--(\ref{e11-sin}) при 
усло\-ви\-ях $\dot{m}_t\hm=0$, $\dot{K}_t\hm=0$:
     \begin{equation}
     a_1^* (m^*,K^*)=0\,;\quad a_2^*(m^*,K^*)=0\,;\label{e12-sin}
     \end{equation}
     \begin{equation}
     \left.
     \begin{array}{c}
     \fr{dk(\tau)}{d\tau} =k_1^a(m^*, K^*)k(\tau)\,;\\[9pt]
      k(\tau)=k(-\tau)^{\mathrm{T}}\,;\ 
k(0)=K\,.
\end{array}
\right\}
\label{e13-sin}
     \end{equation}
    Здесь $k_1^a=k_1^a(m_t,K_t)$ представляет собой коэффициент 
статистической гауссовской линеаризации нелинейной функ\-ции
  \begin{equation*}
  a(Z,t) =a_1(m_t,K_t)+k_1^a(m_t,K_t)(Z-m_t)\,,
%  \label{e14-sin}
  \end{equation*}
причем при $m_t=m^*$, $K_t\hm=K^*$ он определяется формулой:
\begin{equation*}
k_1^a=a_{21}(m_t,K_t) K_t^{-1}\,,
%\label{e15-sin}
\end{equation*}
где
\begin{multline*}
a_{21}=a_{21}(m_t,K_t) ={}\\
{}=\!\!\!\int\limits_{-\infty}^\infty\!\! a(z,t)(z^{\mathrm{T}}-m_t^{\mathrm{T}}) 
f_1^N(z;t,m_t,K_t)\,dz%={}\\
= \!\left[ \fr{\partial}{\partial m_t}\,a_1^{\mathrm{T}}\right]^{\mathrm{T}}\!\!\!\!.\hspace*{-7.34363pt}
%\label{e16-sin}
\end{multline*}
      \renewcommand{\theequation}{\arabic{equation}$^{\prime}$}
  
  \noindent
  При этом условия (\ref{e7-1-sin}) и~(\ref{e7-2-sin}) примут следующий вид:
  \pagebreak
  
  \noindent
  \begin{multline}
  \fr{\partial f_1^N (z;t,m_t,K_t)}{\partial t}+\fr{\partial^{\mathrm{T}}}{\partial z}\left\{ \left[ 
a_1(t,m_t,K_t, t) +{}\right.\right.\\
\hspace*{-2.5mm}\left.\left.{}+k_1^a(t,m_t,K_t,t)(z-m_t)\right] f_1^N (z; t,m_t,K_t)\right\}=0,\!\!\!\!\!
  \label{e17-1-sin}
  \end{multline}
    \renewcommand{\theequation}{\arabic{equation}$^{\prime\prime}$}
    \setcounter{equation}{12}
    
\vspace*{-12pt}

    \noindent
    \begin{multline}
    \fr{\partial^{\mathrm{T}}}{\partial z}\left\{\left[ a_1^*(m^*,K^*)+{}\right.\right.\\
\hspace*{-3mm}\left.\left.    {}+k_1^a(m^*,K^*)(z-
m^*)\right] f_1^N (z;m^*,K^*)\right\}=0.
    \label{e17-2-sin}
  \end{multline}
    \renewcommand{\theequation}{\arabic{equation}}
    \setcounter{equation}{13}
  
  Таким образом, имеет место утверждение.
  
  \medskip
  
  \noindent
  \textbf{Теорема 1.} \textit{Если мат\-ри\-ца $k_1^a\hm=k_1^a(m_t,K_t)$ 
асимп\-то\-тически устойчива, то корреляционный алгоритм аналитического 
моделирования нестационарных режимов МНА в СтС}~(1) \textit{определяется 
уравнениями}~(\ref{e8-sin})--(\ref{e12-sin}), (\ref{e17-1-sin}), \textit{а для 
стационарных режимов~--- урав\-не\-ни\-ями} (\ref{e8-sin})--(\ref{e10-sin}), 
(\ref{e13-sin}), (\ref{e17-2-sin}).
  

\section{Аналитическое моделирование при~автокоррелированных 
помехах}
  
  Сначала рассмотрим СтС~(\ref{e5-sin}) и (\ref{e6-sin}) при 
  условиях~(\ref{e7-1-sin}) и~(\ref{e7-2-sin}) в предположении, что 
стохастический режим~$Z(t)$ является более гладким, чем помеха (имеет 
производные более высокого порядка, чем помеха). Преобразуем 
урав\-не\-ние~(\ref{e5-sin}), записанное в виде:
  \begin{equation}
  \mathrm{Э}\left( Z,U,t\right) =\dot{Z}=a(Z,t)+b_U U\,,
  \label{e18-sin}
  \end{equation}
сис\-те\-мой, обратной ФФ~(\ref{e6-sin}). Как известно~[1, 2], сис\-те\-ма, обратная к 
ФФ~(\ref{e6-sin}), представляет собой параллельное соединение сис\-те\-мы, 
осуществляющей линейную дифференциальную операцию
\begin{equation*}
L=\sum\limits_{k=0}^{l-h} \gamma_k D^k\,,\enskip D=\fr{d}{dt}\,,
%\label{e19-sin}
\end{equation*}
и сис\-те\-мы, описываемой дифференциальным урав\-не\-нием
\begin{equation}
\sum\limits_{i=0}^h \beta_i y^{(i)} =\sum\limits_{j=0}^{h-l} \tilde{\alpha}_j 
x^{(j)}\,.
\label{e20-sin}
\end{equation}
    Здесь введены следующие обозначения:
  \begin{align*}
%  \left.
%  \begin{array}{c}
  \gamma_{l-h}&=\beta_h^{-1}\alpha_l\,;\\
  \gamma_k&=\beta_h^{-1}\left[ 
\alpha_{h+k}-\hspace*{-5mm}\hspace*{-9.52328pt}\sum\limits^{h-1}_{r=\max(0,k-l+2h)} 
\sum\limits_{j=0}^{h-r} 
C^r_{r+j}\beta_{r+j}\gamma^{(j)}_{h+k-r}\right] \\
&\hspace*{35mm}(k=0,1,\ldots , l-h-1)\,;\\
  \tilde{\alpha}_k&=\alpha_k- \hspace*{-2mm}\sum\limits^k_{r=\max(0,k-l+2h)} \sum\limits_{j=0}^{h-r}
C^r_{r+j} \beta_{r+j} \gamma^{(j)}_{k-r}\\
&\hspace*{35mm}(k=0,1,\ldots , l-h-1)\,.
%  \end{array}
%  \right\}
 % \label{e21-sin}
  \end{align*}
  
  Пропустив сигнал $\mathrm{Э}(Z,t)$, определяемый 
  урав\-не\-ни\-ем~(\ref{e18-sin}), через сис\-те\-му, со\-сто\-ящую из усилителя с 
коэффициентом~$b_U^{-1}$ и сис\-те\-мы, обратной ФФ~(\ref{e6-sin}), получим 
на выходе сигнал
  \begin{equation*}
  \mathrm{Э}_1=L\left[ b_U^{-1} a(Z,t)\right] +Z_1^\prime +V\,,
%  \label{e22-sin}
  \end{equation*}
где $Z_1^\prime$~--- выходной сигнал сис\-те\-мы~(\ref{e20-sin}) при 
$\mathrm{Э}(Z,t)=b_U^{-1} a(Z,t)$:
\begin{equation}
\sum\limits_{i=0}^h \beta_i Z_1^{\prime(i)}= \sum\limits_{j=0}^{h-1} 
\tilde{\alpha}_j\left( b_U^{-1} a(Z,t)\right)^{(i)}\,.
\label{e23-sin}
\end{equation}
  
  Приведя уравнение~(\ref{e23-sin}) к сис\-те\-ме урав\-не\-ний первого порядка 
согласно~[1, 2], получим дифференциальное урав\-не\-ние, определяющее век\-тор 
$Z^\prime\hm= \left[ Z_1^{\prime \mathrm{T}}\cdots Z_h^{\prime \mathrm{T}}\right]^{\mathrm{T}}$:
  \begin{equation*}
  \dot{Z}^\prime = c(Z,Z^\prime)\,.
%  \label{e24-sin}
  \end{equation*}
  
  Наконец, введя расширенный век\-тор со\-сто\-яния $\overline{Z}\hm= \left[ Z^{\mathrm{T}}\, 
Z^{\prime T}\right]^{\mathrm{T}}$, получим урав\-не\-ние вида~(\ref{e1-sin}):
  \begin{equation}
  \dot{\overline{Z}}=\overline{a}\left(\overline{Z},t\right) +\overline{b}V\,,
  \label{e25-sin}
  \end{equation}
где
\begin{equation}
\overline{a}=\begin{bmatrix} a\\ c\end{bmatrix}\,;\qquad
\overline{b}= \begin{bmatrix} b_U\\ 0\end{bmatrix}\,.
\label{e26-sin}
\end{equation}
Таким образом, имеем следующий результат.
  
  \medskip
  
  \noindent
  \textbf{Теорема~2.} \textit{Если мат\-ри\-ца $k_1^{\overline{a}}$ 
асимптотически устойчива, а стохастический режим в}~(\ref{e5-sin}) 
\textit{является более гладким, чем помеха, то корреляционный алгоритм 
аналитического моделирования МНА определяется теоремой}~1 \textit{для 
уравнения}~(\ref{e25-sin}) \textit{при условии}~(\ref{e26-sin}).
  
  \smallskip
  
  В случае, когда стохастический режим не менее гладкий, чем помеха~$U$, 
достаточно применить процедуру расширения вектора со\-сто\-яния путем 
дифференцирования~(\ref{e18-sin}) и исключения помехи~$U$ и ее 
производных, не содержащих белого шума~$V$, из урав\-не\-ний 
  ФФ~(\ref{e6-sin}) с по\-мощью урав\-не\-ния~(\ref{e18-sin}) и урав\-не\-ний, 
полученных из него дифференцированием.
  
  С этой целью продифференцируем уравнение~(\ref{e18-sin}), умноженное 
слева на $b_U^{-1}$, до появления в нем белого шума. Тогда получим:
  \begin{multline}
  \fr{d^i}{dt^i}\left[ b_U^{-1}(t) \mathrm{Э}(Z,U,t)\right] ={}\\
  {}=\fr{d^i}{dt^i}\left[ 
b_U^{-1} (t) a(Z,t)\right] + U_{i+1}\enskip (i=0,1,\ldots ,s)\,,
  \label{e27-sin}
  \end{multline}
если
$$
\fr{d^s}{dt^s}\left[ b_U^{-1}(t) a(Z,t)\right] \ \mbox{при}\ s<l-h
$$
содержит белый шум, и
\begin{equation}
\left.
\begin{array}{rl}
\fr{d^i}{dt^i}\left[ b_U^{-1}(t)\mathrm{Э}(Z,U,t)\right] &={}\\[9pt]
&\hspace*{-56.57764pt}{}= \fr{d^i}{dt^i}\left[ 
b_U^{-1}(t)a(Z,t)\right]+U_{i+1}\\[9pt]
&\hspace*{-19pt}(i=0,1,\ldots , l-h-1)\,;\\[9pt]
\fr{d^{l-h}}{dt^{l-1}}\left[ b_U^{-1}(t)\mathrm{Э}(Z,U,t)\right] &={}\\
&\hspace*{-106.57764pt}{}= \fr{d^{l-
h}}{dt^{l-h}}\left[ b_U^{-1}(t)a(Z,t)\right]+U_{l-h+1}+q_{l-h}V\,,
  \end{array}
  \right\}
  \label{e28-sin}
  \end{equation}
если ни при каком $s\hm<l\hm-h$ производная $d^s\left[ b_U^{-1}(t) 
a(Z,t)\right]/dt^s$ не содержит белого шума~$V$.
  
  В первом случае, если согласно~[1, 2] при\-вес\-ти~(\ref{e6-sin}) к сис\-те\-ме 
уравнений первого порядка:
  \begin{equation}
  \left.
  \begin{array}{rl}
  U_1&=U\,;\\[9pt]
  \dot{U}_i &=U_{i+1}\enskip (i=1,\ldots , l-h-1)\,;\\[9pt]
  \dot{U}_i &= U_{i+1}+q_iV \enskip (i=l-h,\ldots , l-1)\,;\\[9pt]
  \dot{U}_l &=\displaystyle -\alpha_i^{-1}\sum\limits_{i=1}^{l-1} \alpha_i U_{i+1}+q_l V\,,
  \end{array}
  \right\}
  \label{e29-sin}
  \end{equation}
где $q_i$ и $q_l$ определены в~[1, 2] и зависят от~$\alpha_i$ и~$\beta_j$, 
можно исключить помехи $U_1,\ldots , U_s$ с по\-мощью~(\ref{e18-sin}) и 
урав\-не\-ний, полученных ($s-1$)-крат\-ным дифференцированием.
  
  Во втором случае следует из~(\ref{e29-sin}) исключить помехи $U_1, \ldots , 
U_{l-h-1}$ и белый шум~$V$ с по\-мощью~(\ref{e18-sin}) и всех урав\-не\-ний, 
полученных из него дифференцированием.
  
  В обоих случаях получим уравнения $U_{s+1}, \ldots , U_l$ ($s\hm\leq l\hm-
h$). При этом можно рас\-чле\-нить каждую из этих переменных на две, одна из 
которых зависит от~$Z$, а другая зависит от $\mathrm{Э}(Z,t)$ и его 
производных, и тогда будем иметь
  \begin{equation}
  U_{s+k} =Z^\prime_k-Y_k\,,
  \label{e30-sin}
  \end{equation}
где $Z_1^\prime, \ldots , Z^\prime_{l-s}$ определяются дифференциальными 
уравнениями~(\ref{e25-sin}):
\begin{equation}
\dot{Z}^\prime =c^\prime\,,\ c^\prime=c^\prime(Z,t)\,,\ Z^\prime=\left[ Z_1^\prime, 
\ldots , Z^\prime_{l-s}\right]^{\mathrm{T}}\,,
\label{e31-sin}
\end{equation}
а $Y_1^\prime, \ldots Y^\prime_{l-s}$~--- урав\-не\-ни\-ями
\begin{equation}
\left.
\begin{array}{c}
\dot{Y}^\prime=c^{\prime\prime}\,,\enskip c^{\prime\prime}= 
c^{\prime\prime}\left(\mathrm{Э}, \dot{\mathrm{Э}}, \ldots , \mathrm{Э}^{(l-
s)}\right)\,;\\[9pt]
Y^\prime= \left[ Y_1^\prime, \ldots , Y^\prime_{l-s}\right]^{\mathrm{T}}\,.
\end{array}
\right\}
\label{e32-sin}
\end{equation}
Правые части $c^\prime$ и $c^{\prime\prime}$ определяются 
согласно~(\ref{e29-sin}) и~(\ref{e30-sin}).

  Расширив вектор со\-сто\-яния~$Z$ для $\overline{Z}\hm=\left[ Z^{\mathrm{T}}\ Z_1^{\prime 
\mathrm{T}} \cdots Z_{l-s}^{\prime \mathrm{T}}\right]^{\mathrm{T}}$, придем к окончательным 
урав\-не\-ни\-ям~(\ref{e25-sin}), (\ref{e26-sin}) при $c\hm=c^\prime$.
  
  Таким образом, получен следующий результат.
  
  \medskip
  
  \noindent
  \textbf{Теорема~3.} \textit{Если мат\-ри\-ца $k_1^{\overline{a}}$ 
асимптотически устойчива, а стохастический режим в}~(\ref{e5-sin}) 
\textit{не менее гладкий, чем помеха, то корреляционный алгоритм 
аналитического моделирования МНА определяется уравнениями тео\-ре\-мы}~1 
\textit{для сис\-те\-мы}~(\ref{e25-sin}) \textit{при 
  условиях}~(\ref{e27-sin})--(\ref{e32-sin}).
  
\section{Тестовые примеры}
     
     
     \noindent
1.~Стохастическое уравнение Дуффинга~[1, 2]
\begin{equation}
\ddot{X}+\omega^2 X-\mu X^3 =-\delta \dot{X}+U_0+V
\label{e33-sin}
\end{equation}
при $U_0=0$ допускает режим стационарных стохастических колебаний, 
определяемых формулой Гиббса 
\begin{equation*}
f_1(x,\dot{x})=C\exp \left[ -\fr{2\delta}{\gamma}\,H(x,\dot{x})\right]\,,
%\label{e34-sin}
\end{equation*}
где $H(x,\dot{x})\hm= \dot{x}^2/2\hm+\omega^2x^2/2\hm- \mu x^4/4$. В~основе 
корреляционного алгоритма аналитического моделирования стационарных и 
нестационарных режимов лежат следующие урав\-не\-ния~[1, 2]:
\begin{multline*}
X^3\approx m_X(m_X^2+3D_X)+3(m^2_X+D_X)X^0={}\\
{}=k_0 m_X+k_1 X^0\enskip 
(X^0=X-m_X)\,;
%\label{e35-sin}
\end{multline*}
\begin{equation*}
\dot{m}_X = m_{\dot{X}}\,,\enskip \dot{m}_X=U_0-\omega^2_{\mathrm{э}} 
m_X-\delta m_{\dot{X}}\,;
%\label{e36-sin}
\end{equation*}
      \begin{gather*}
     \dot{D}_X=2K_{X\dot{X}}\,,\enskip \dot{D}_X =\nu -2(\omega^2_{1\mathrm{э}} 
K_{X\dot{X}}+\delta D_{\dot{X}})\,;\\
 \dot{K}_{X\dot{X}}=D_{\dot{X}}- 
\omega^2_{1\mathrm{э}} D_X-\delta K_{X\dot{X}}\,,
    \end{gather*}
где
\begin{equation}
\hspace*{-4mm}\left.
\begin{array}{rl}
\omega^2_{\mathrm{э}} &=\omega^2\left[ 1-
\fr{\mu(m_X^2+3D_X)}{\omega^2}\right]\,;\\[9pt]
\omega^2_{1\mathrm{э}} &=\omega^2\left[ 1-
\fr{3\mu(m^2_X+D_X)}{\omega^2}\right] \enskip 
(\omega_{\mathrm{э}}>\omega_{1\mathrm{э}})\,.
\end{array}
\right\}
\label{e37-sin}
\end{equation}

  При $U_0=0$ в стационарном случае $m_X^*\hm=0$, 
$m^*_{\dot{X}}\hm=0$, $K^*_{X\dot{X}}\hm=0$, $D^*_{\dot{X}}\hm = 
\nu/(2\delta)$, а $D_X^*$ определяются путем решения урав\-не\-ния 
$\nu/(2\delta)\hm= \omega_{1\mathrm{э}}^2(D^*_{X})D^*_{X}$.
  
  Таким образом, МНА для $D_{\dot{X}}^*$ дает решение, совпадающее с 
точным, а для $D^*_{{X}}$~--- приближенное, точ\-ность которого 
зависит от коэффициента~$\mu$. Процесс установления стационарных 
\mbox{колебаний} происходит в два этапа: сначала устанавливается $D_{\dot{X}}^*$, а 
затем $D_{{X}}^*$.

\pagebreak
  
  Условие (\ref{e7-2-sin}) требует консерватизма статистически 
линеаризованной сис\-те\-мы левой час\-ти~(\ref{e33-sin}). Для значений~$\mu$, 
отвечающих колебаниям, точ\-ность со\-став\-ля\-ет~10\%. 

\noindent
2.~Для системы 
\begin{equation}
\ddot{X}+\omega^2 X -\mu X^3=-\delta \dot{X}+U_0+U\,,\ \dot{U}=-\gamma 
U+V
\label{e38-sin}
\end{equation}
уравнения МНА для $Z=\left[ X\, \dot{X}\, U\right]^{\mathrm{T}}$ имеют вид~(\ref{e9-sin}) 
и~(\ref{e10-sin}) при
\begin{gather*}
a_1=\begin{bmatrix}
m_{\dot{X}}\\ -\omega^2_{\mathrm{э}} m_X-\delta m_{\dot{X}}+U_0\\ -m_U
\end{bmatrix};\enskip
\alpha = \begin{bmatrix}
0 & 1 & 0\\
-\omega^2_{1\mathrm{э}} & -\delta & 0\\
0 & 0& -\gamma
\end{bmatrix};\hspace*{-0.26428pt}\\
\beta=\begin{bmatrix} 0&0&0\\
0&0&0\\
0&0&1
\end{bmatrix}\,;\  
a_2=\alpha K_t+K_t\alpha^{\mathrm{T}} +\beta\nu \beta^{\mathrm{T}}\,,
\end{gather*}
где $\nu$~--- интенсивность белого шума~$V$; $\omega_{\mathrm{э}}$, 
$\omega_{1\mathrm{э}}$ определяются~(\ref{e37-sin}). Отсюда аналитическим 
моделированием определяются стационарные режимы стохастических 
колебаний, а также режимы установления. Условие~(\ref{e7-2-sin}) 
выполняется в силу консерватизма левой час\-ти~(\ref{e38-sin}).
  
  Сравнивая оба примера, заключаем, что точность метода за счет 
<<профильтрованности>> помех значительно повышается и достигает 
  2\%--4\%.
  
  \noindent
  3.~В инструментальном программном обеспечении CStS-Analysis тестовые 
примеры описаны в~[14, 15].
  
\section{Заключение}
  
  Для многомерных нелинейных дифференциальных нормальных 
(гауссовских) сис\-тем с некоррелированными и автокоррелированными 
гауссовскими помехами на базе метода нормальной аппроксима\-ции 
разработаны корреляционные ал\-го\-рит\-мы аналитического моделирования 
стохастических режимов с инвариантной мерой. Результаты допускают 
обобщение на случай негауссовских помех, а также недифференцируемых 
функций $a\hm=a(Z,t)$ в урав\-не\-ни\-ях~(1) и~(\ref{e5-sin}).
  
  С помощью разработанного в среде MATLAB инструментального 
программного обеспечения на тестовых примерах показана достаточная для 
многих приложений точность корреляционных ал\-го\-ритмов.
  
  
  {\small\frenchspacing
{%\baselineskip=10.8pt
\addcontentsline{toc}{section}{Литература}
\begin{thebibliography}{99}

     \bibitem{1-sin}
     \Au{Пугачев В.\,С., Синицын И.\,Н.} Стохастические дифференциальные 
сис\-те\-мы. Анализ и фильтрация.~--- 2-е изд., доп.~--- М.: Наука, 1990.
     \bibitem{2-sin}
     \Au{Пугачев В.\,С., Синицын И.\,Н.} Теория стохастических сис\-тем.~--- 
     2-е изд.~--- М.: Логос, 2004.
     \bibitem{3-sin}
     \Au{Moshchuk N.\,K., Sinitsyn I.\,N.} On stationary distributions in nonlinear 
stochastic differential systems: Preprint.~--- Coventry, CV4 7AL, UK: Mathematics 
Institute, University of Warwick, 1989. 15~p. 
     \bibitem{4-sin}
     \Au{Moshchuk N.\,K., Sinitsyn I.\,N.} On stochastic nonholonomic systems: 
Preprint.~--- Coventry, CV4 7AL, UK: Mathematics Institute University of Warwick, 
1989. 32~p. 
     \bibitem{5-sin}
     \Au{Мощук Н.\,К., Синицын И.\,Н.} О~стохастических неголономных 
сис\-те\-мах~// Прикладная механика и математика, 1990. Т.~54. Вып.~2. 
     С.~213--223.
     \bibitem{6-sin}
     \Au{Moshchuk N.\,K., Sinitsyn I.\,N.} On stationary distributions in nonlinear 
stochastic differential systems~// Quart. J.~Mech. Appl. Math., 1991. Vol.~44. Pt.~4. 
P.~571--579.
     \bibitem{7-sin}
     \Au{Мощук Н.\,К., Синицын И.\,Н.} О~стационарных и приводимых к 
стационарным режимах в нормальных стохастических сис\-те\-мах~// Прикладная 
механика и математика, 1991. Т.~55. Вып.~6. С.~895--903.
     \bibitem{8-sin}
     \Au{Мощук Н.\,К., Синицын И.\,Н.} Распределения с инвариантной мерой 
в механических стохастических нормальных сис\-те\-мах~// Докл. АН СССР, 1992. 
Т.~322. №\,4. С.~662--667.
     \bibitem{9-sin}
     \Au{Синицын И.\,Н.}
     Конечномерные распределения с инвариантной мерой в стохастических 
механических сис\-те\-мах~// Докл. РАН, 1993. Т.~328. №\,3. С.~308--310.
     \bibitem{10-sin}
     \Au{Синицын И.\,Н.} Конечномерные распределения с инвариантной 
мерой в стохастических нелинейных дифференциальных сис\-те\-мах.~--- М.: 
Диалог МГУ, 1997. С.~129--140.
     \bibitem{11-sin}
     \Au{Синицын И.\,Н., Корепанов~Э.\,Р., Белоусов~В.\,В.} Точные методы 
рас\-че\-та стационарных режимов с инвариантной мерой в стохастических 
сис\-те\-мах управ\-ле\-ния~// Кибернетика и технологии XXI~века (C\&T'2002): Тр. 
II Междунар. научно-технич. конф.~--- Воронеж: Саквое, 2002. С.~124--131.
     \bibitem{12-sin}
     \Au{Синицын И.\,Н., Корепанов Э.\,Р., Белоусов~В.\,В.} Точные 
аналитические методы в статистической динамике нелинейных 
ин\-фор\-ма\-ци\-он\-но-управ\-ля\-ющих сис\-тем~// Сис\-те\-мы и средства информатики. 
Спец. вып. Математическое и алгоритмическое обеспечение 
     ин\-фор\-ма\-ци\-он\-но-те\-ле\-ком\-му\-ни\-ка\-ци\-он\-ных сис\-тем.~--- 
М.: Наука, 2002. С.~112--121.
     \bibitem{13-sin}
     \Au{Soize C.} The Fokker--Plank equation for stochastic dynamical systems 
and its explicit steady state solutions.~--- Singapore: World Scientific, 1994.
     \bibitem{14-sin}
     \Au{Sinitsyn I.\,N.} Lectures on PC-based nonlinear stochastic mechanical 
systems research: U$\hat{\mbox{c}}$ebni Texty $\acute{\mbox{u}}$snavu 
Termomechaniky.~--- Praha: {\ptb {\v{C}}}AV, 1992. 63~p.

\label{end\stat}
     \bibitem{15-sin}
     \Au{Синицын И.\,Н.} Стохастические информационные технологии для 
исследования нелинейных круговых сис\-тем~// Информатика и её применения, 
2011. Т.~5. Вып.~4. С.~78--99.



%     \bibitem{16-sin}
%     \Au{Синицын И.\,Н.} Математическое обеспечение для анализа 
%нелинейных многоканальных круговых сис\-тем, основанное на па\-ра\-мет\-ри\-за\-ции 
%распределений~// Информатика и её применения, 2012. Т.~6. Вып.~1. С.~11--17.
\end{thebibliography}
}
}


\end{multicols}     
     
     
     
     