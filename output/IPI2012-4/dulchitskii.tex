\newcommand{\La}{\Lambda}

\def\stat{dulch}

\def\tit{О ТОЧНОСТИ НЕКОТОРЫХ МАТЕМАТИЧЕСКИХ МОДЕЛЕЙ КАТАСТРОФИЧЕСКИ
НАКАПЛИВАЮЩИХСЯ ЭФФЕКТОВ ПРИ ПРОГНОЗИРОВАНИИ РИСКА ЭКСТРЕМАЛЬНЫХ
СОБЫТИЙ$^*$}

\def\titkol{О точности некоторых математических моделей катастрофически
накапливающихся эффектов} % при прогнозировании риска экстремальных событий}

\def\autkol{И.\,А.~Дучицкий, В.\,Ю.~Королев, И.\,А.~Соколов}

\def\aut{И.\,А.~Дучицкий$^1$, В.\,Ю.~Королев$^2$, И.\,А.~Соколов$^3$}

\titel{\tit}{\aut}{\autkol}{\titkol}

{\renewcommand{\thefootnote}{\fnsymbol{footnote}}\footnotetext[1]
{Работа поддержана Российским фондом фундаментальных
исследований (проекты 11-01-00515а, 11-01-12026-офи-м,
12-07-00109a, 12-07-00115a) и Министерством образования и науки (госконтракт
16.740.11.0133).}}

\renewcommand{\thefootnote}{\arabic{footnote}}
\footnotetext[1]{Факультет вычислительной
математики и кибернетики Московского государственного университета
им.\ М.\,В.~Ломоносова; duchik@gmail.com}
\footnotetext[2]{Факультет вычислительной математики и кибернетики
Московского государственного университета им.\ М.\,В.~Ломоносова;
Институт проблем информатики Российской академии наук;
vkorolev@cs.msu.su}
\footnotetext[3]{Институт проблем 
информатики Российской академии наук; ipiran@ipiran.ru}

\vspace*{-6pt}


\Abst{Построены оценки точности приближения
распределений экстремумов специальных случайных сумм масштабными
смесями полунормальных законов и обсуждается возможность
использования этих результатов при прогнозировании риска
экстремальных событий, вызванных катастрофически накапливающимися
неблагоприятными эффектами.}

\KW{неоднородные потоки событий; дважды
стохастический пуассоновский процесс; отрицательное биномиальное
распределение; гам\-ма-рас\-пре\-де\-ле\-ние; оценка скорости сходимости}

\vskip 14pt plus 9pt minus 6pt

      \thispagestyle{headings}

      \begin{multicols}{2}

            \label{st\stat}

\section{Введение}

По мере возрастания технической и информационной оснащенности
человечества вопросы\linebreak оценивания и прогнозирования надежности
технических и информационных систем, а стало быть, и рисков,
связанных с их отказами, приобретают все более важное значение.
Цивилизация становит\-ся\linebreak все более и более зависимой от возможностей,\linebreak
предоставляемых современными техническими (транспортными,
энергетическими, оборонными) и информационными
(телекоммуникационными, вычислительными) сис\-те\-ма\-ми. Одновременно
возрастает и зависимость человечества от {\it на\-деж\-ности и
безопасности} технических и информационных систем. В~силу этих
обстоятельств, естественно, возрастают требования к адекватности
математических моделей, используемых для вычисления соответствующих
показателей надежности и возможных рисков.

Многие классические методы оценки риска или показателей надежности,
разработанные, как правило, в середине XX~в., основаны на
идеальных\linebreak предположениях о том, что параметры, характеризующие,
скажем, воздействие внешней среды,\linebreak имеют нормальное распределение, а
параметры,\linebreak
 характеризующие на\-деж\-ность составных час\-тей изучаемой
системы, например время жизни (на\-ра\-бот\-ки на отказ), имеют
показательное (экс-\linebreak по\-нен\-ци\-аль\-ное) распределение и более общее
распределе\-ние Вей\-бул\-ла--Гне\-ден\-ко. Однако, к сожалению, за\-час\-тую
применение классических методов приводит к недооценке риска
катастроф или отказов. Причины иногда имеющей место
несостоятельности классических моделей могут быть разными. 
К~примеру, если показатели на\-деж\-ности вычисляются на осно\-ве
статистических данных, накопленных за определенное время, то, как
показано, например, в~\cite{BKSSh2007}, существенную роль играет
однородность потока событий, в результате которых накапливаются
статистические данные. 

Другими словами, критичным для адекват\-ности
классических моделей является асимптотическое постоянство отношения
количества зарегистрированных в течение определенного интервала
времени экстремальных событий к длине этого интервала времени при
неограниченном увеличении по\-следней. 

Если асимптотическое
постоянство указанного отношения, т.\,е.\ его сближение с некоторым
чис\-лом, имеет место, то классические модели могут давать адекватные
результаты. Однако если такого сближения не наблюдается и указанное
отношение сильно колеблется, оставаясь случайным (т.\,е.\
непредсказуемым), то классические модели неадекватны и могут
приводить к весьма существенной недооценке риска. В~частности,
вместо ожидаемого в соответствии с классической теорией нормального
закона в подобных ситуациях (например, если упомянутое выше
отношение ведет себя как гам\-ма-рас\-пре\-де\-лен\-ная случайная величина)
могут возникать, скажем, функции распределения ущерба типа
распределения Стьюдента с произвольно малым числом степеней свободы
или так называемые дисперсионные гам\-ма-рас\-пре\-де\-ле\-ния, в част\-ности
распределение Лапласа~[2--4].

Негативный эффект нежелательных воздействий может проявиться
мгновенно, а может накапливаться постепенно. Это приводит к
не\-обхо-\linebreak ди\-мости наряду с самим потоком экстремальных\linebreak событий
рассматривать также и процесс, описывающий <<накопленные>>,
негативные воздействия. 

Под {\it катастрофой} (катастрофическим
событием) разумно подразумевать превышение модельным процессом
некоторого критического уровня, который, например, может быть
определен как минимальный уровень, превышение которого накопленным
негативным воздействием на рассматриваемую систему (информационную,
техническую, экологическую, социальную) ведет к необратимым\linebreak ее
изменениям, в частности к невозможности выполнять системой свои
функции в прежнем ре\-жиме.

Как показано в~\cite{KorolevSokolov2008}, если интенсивность пото\-ка
информативных событий случайна и имеет, скажем, стандартное
экспоненциальное распределение, то величина максимума накопленных
эффектов также имеет экспоненциальное распределение (но с параметром
$\sqrt{2}$), а не ожидаемое в соответствии с классической теорией
распределение $G(x)\hm=2\Phi(\max\{0,x\})-1$ максимума винеровского
процесса на единичном отрезке, хвост которого убывает при
$x\hm\to\infty$ как $\sqrt{2/\pi}x^{-1}e^{-x^2/2}$.  Хвост же
упомянутой выше экспоненциальной функции распределения, естественно,
убывает существенно медленнее~--- как $e^{-\sqrt{2}x}$. Это приводит
к существенному различию в оценках вероятностей катастрофических
воздействий и размеров самих катастрофических воздействий,
получаемых на основе этих двух функций распределения. К~примеру,
квантиль порядка 0,99 функции распределения $G(x)$ равна 2,576. В~то
же время квантиль того же порядка экспоненциальной функции
распределения с па\-ра\-мет\-ром~$\sqrt{2}$ примерно равна 3,256.
Вероятность превышения порога~2,576, <<критического>> при функции
распределения~$G(x)$, максимальным суммарным воздействием, име\-ющим
указанную показательную функцию распределения, превышает 0,026, т.\,е.\ 
оказывается более чем в два с половиной раза выше, чем
предполагаемая по классической модели. 

Этот %\linebreak 
пример является еще
одной наглядной иллюстрацией того, насколько можно недооценить риск
ка\-та\-ст\-роф, вызванных накапливающимися эффектами неблагоприятных
воздействий, если не принимать во внимание стохастический характер
интенсивности потока информативных со-\linebreak бытий.

Естественно предположить, что негативные эффекты накапливаются как
результат неких событий, хаотически рассредоточенных по
временн$\acute{\mbox{о}}$й оси, т.\,е.\ образующих так называемый хаотический
поток событий. Как известно (см., например,~\cite{KorolevSokolov2008}),\linebreak 
наилучшей моделью однородного
хаотического\linebreak потока событий является пуассоновский процесс,
характеризуемый тем обстоятельством, что интервалы времени между
событиями потока независимы и имеют одинаковое показательное
распределение. Привлекательность пуассоновского процесса\linebreak в качестве
модели однородного дискретного хаоса обусловлена как минимум двумя
обстоятельствами. 

Во-пер\-вых, показательное распределение ин\-тер\-ва\-лов
времени между событиями пуассоновского потока обладает максимальной
дифференциальной энтропией среди всех абсолютно непрерывных
распределений вероятностей, сосредоточенных на всей положительной
полуоси и имеющих конечное математическое ожидание, а энтропия, как
известно, является очень удобной численной характеристикой
неопределенности. 

Во-вто\-рых, точки (события) пуассоновского потока
равномерно распределены на оси времени в том смысле, что для любого
конечного интервала времени $[t_1,t_2]$ условное совместное
распределение точек пуассоновского потока, попавших в интервал
$[t_1,t_2]$, при условии, что в этот интервал попало фиксированное
число, скажем $n$, точек, совпадает с совместным распределением
вариационного ряда, построенного по независимой однородной выборке
объема $n$ из равномерного на $[t_1,t_2]$ распределения. Равномерное
же распределение обладает\linebreak максимальной дифференциальной энтропией
среди всех абсолютно непрерывных распределений вероят\-ностей,
сосредоточенных на конечных интервалах, и очень хорошо соответствует
общепринятому представлению об абсолютно непредсказуемой
ограниченной случайной величине.

Однако в реальных <<хаотических>> системах хаос практически никогда
не бывает однородным в пространстве или времени. Неоднородный и даже
стохастический характер интенсивности потока информативных событий
может быть обусловлен случайно возникающими (не поддающимися
абсолютно надежному прогнозированию) причинами. Как известно,
наиболее разумными стохастическими моделями неоднородных хаотических
точечных процессов являются {\it дважды стохастические пуассоновские
процессы}, иначе называемые {\it процессами Кокса} (см., например,~\cite{BeningKorolev2002}).

Для эффективной реализации мер, направленных на повышение надежности
и катастрофоустойчивости технических и информационных сис\-тем,
необходимо уметь вычислять вероятностные характеристики возможных
экстремальных воздействий на рассматриваемую систему. Как уже
отмечалось выше, при непостоянной (и тем более при стохастической)
интенсивности потока экстремальных событий статистические
закономерности нежелательных воздействий существенно отличны от
того, какими они были бы в однородной ситуации, описываемой
классической теорией экстремальных значений. Для вычисления таких
мер риска, как квантильные (типа показателей VaR~--- <<Value at
Risk>>), необходимо иметь более или менее точные аппроксимации для
вероятностных распределений величин экстремальных воздействий.

\section{Предельное поведение экстремумов обобщенных дважды стохастических
пуассоновских процессов}

В качестве базовой модели <<накапливающегося\linebreak
 воздействия>> в данной
статье рассматриваются экстремумы обобщенных дважды стохастических
пуассоновских процессов, представляющих собой суммы случайного числа
независимых одинаково распреде\-лен\-ных случайных величин, в которых
число слагаемых определяется значением дважды стохастического
пуассоновского процесса. При этом под катастрофическим
неблагоприятным воздействием будет пониматься превышение случайным
блужданием, порожденным базовой последовательностью независимых
случайных величин, некоторого заданного уровня. Простейшей задачей
оценивания или прогнозирования рисков, связанных с такими событиями,
является задача вычисления ве\-ро\-ят\-ности превышения возможным
значением максимума накапливающихся эффектов критического уровня в
течение некоторого фиксированного интервала времени.
Целесообразность рассмотрения подобных моделей при прогнозировании
рисков катастроф диктуется следующими примерами.

\medskip

\noindent
\textbf{Пример 1}. Эволюция финансовых индексов хорошо описывается
(неоднородным) случайным блуж\-да\-ни\-ем. При этом, как известно, если
изменение этого индекса в течение биржевого дня будет слишком
большим, то, чтобы избежать слишком больших потерь (т.\,е.\
финансовых катастроф), торги автоматически прекращаются. Другими
словами, если экстремум процесса, описывающего динамику финансового
индекса, достигает критического значения, то торги прекращаются. 
В~такой задаче временн$\acute{\mbox{ы}}$м горизонтом, на который строится прогноз,
равен операционному дню, а элементарные слагаемые в сумме~--- это
приращения процесса за единицы времени (например, минуты). На
сегодняшний день весьма актуальным примером подобной задачи является
вопрос принятия Банком России (Центробанком) решения о проведении
валютной (долларовой) интервенции, чтобы остановить процесс падения
курса рубля относительно доллара при опасном приближении курса к
верхней границе объявленного допустимого коридора. Здесь
представляет интерес возможность приближения курса к границе не в
какой-то определенный момент времени, а в течение операционного дня,
при этом риск, связанный с таким приближением, оценивается заранее
(например, после получения информации об эволюции курса в течение
предыду\-ще\-го операционного дня).

\medskip

\noindent
\textbf{Пример 2}. Для снабжения некоторой отрасти некоторого региона в
течение фиксированного периода времени (например, квартала или
зимнего периода), на склад (хранилище) выделяется определенное
количество некоторого ресурса (скажем, топлива). Если {\it
суммарные, накопленные} расходы этого ресурса в течение указанного
времени превысят выделенный лимит, то в данной отрасти в данном
регионе наступит катастрофический коллапс. При этом естественно
предположить, что ресурс отпускается потребителям партиями, вообще
говоря, случайного объема согласно запросам, возникающим, вообще
говоря, в случайные моменты времени. При этом горизонтом
прогнозирования, естественно, считается интервал времени между
поставками (например, квартал).

\medskip

\noindent
\textbf{Пример 3}. При проектировании дамб и водохранилищ необходимо
учитывать то обстоятельство, что количество воды в рассматриваемом
резервуаре (водохранилище, бассейне реки, озере и~т.\,п.)\ изменяется
случайным образом: оно увеличивается за счет выпадения осадков (в
случайные моменты времени) и уменьшается за счет испарения. Если
экстремум уровня превысит критический уровень, то происходят события
катастрофического характера. Избыток воды вызывает наводнения, ее
недостаток~--- засуху. Ясно, что в силу естественных циклических
причин в качестве горизонта прогнозирования разумно взять год или
солнечный цикл (11~лет). При этом уровень воды в каждый момент
времени является суммой приращений этого уровня за каждый из
предыдущих дней.

\medskip

\noindent
\textbf{Определение~1.} Случайный процесс $\Lambda(t)$, $t\hm\geqslant 0$, с
неубывающими непрерывными справа траекториями, удовлетворяющий
условиям $\Lambda(0)\hm=0$, ${\sf P}(\Lambda(t)\hm<\infty)=1$ $(0\hm<t\hm<\infty)$,
называется {\it случайной мерой}.

\smallskip

\noindent
\textbf{Определение 2.} Пусть $N_1(t)$~--- стандартный пуассоновский
процесс, $\Lambda(t)$~--- случайная мера, независимая от $N_1(t)$.
Случайный процесс $N(t)\hm=N_1(\Lambda(t))$ называется {\it дважды
стохастическим пуассоновским процессом} (или {\it процессом Кокса}).
В~таком случае говорят, что процесс Кокса $N(t)$ управляется
процессом $\La(t)$ (или что процесс $\La(t)$ контролирует процесс
Кокса $N(t)$).

\smallskip

В частности, если процесс $\Lambda(t)$ допускает представление
$$
\Lambda(t)=\il{0}{t}\lambda(\tau)\,d\tau\,,\enskip t\geqslant 0\,,
$$
в котором $\lambda(t)$~--- положительный случайный процесс с
интегрируемыми траекториями, то $\lambda(t)$ можно
интерпретировать как мгновенную стохастическую интенсивность
процесса $N(t)$. Поэтому \mbox{иногда} процесс $\Lambda(t)$, управ\-ля\-ющий
процессом Кокса $N(t)$, будет называться {\it накопленной
ин\-тен\-сив\-ностью} процесса~$N(t)$.

Несложно убедиться, что для процесса Кокса $N(t)$, управляемого
процессом $\Lambda(t)$, справедливы соотношения
$$
{\sf E}N(t)={\sf E}\Lambda(t)\,,\enskip{\sf D}N(t)={\sf
E}\Lambda(t)+{\sf D}\Lambda(t)\,.
$$

\smallskip

\noindent
\textbf{Определение 3.} Пусть $X_1,X_2,\ldots$~--- одинаково
распределенные случайные величины. Предположим, что при каждом $t\hm\geqslant
 0$ случайные величины $N(t),X_1,X_2,\ldots$ независимы. Процесс
\begin{equation}
S(t)=\sum\limits_{j=1}^{N(t)}X_j\,,\enskip t\geqslant 0\,,
\label{e1-dul}
\end{equation}
назовем обобщенным процессом Кокса (при этом для определенности
считаем, что $\sum\limits_{j=1}^{0} =0$).

\smallskip

Процессы вида~(\ref{e1-dul}) играют чрезвычайно важную роль во многих
прикладных задачах. Достаточно сказать, что при $\Lambda(t) \hm\equiv
\lambda t$ с $\lambda \hm> 0$ процесс $S(t)$ превращается в
классический обобщенный пуассоновский процесс, широко используемый
при моделировании многих явлений в физике, теории надежности,
финансовой и актуарной деятельности, биологии и~т.\,д. Большое число
разнообразных прикладных задач, приводящих к обобщенным
пуассоновским процессам, описано в книгах~[3, 6--8].

Для вычисления риска катастрофических превышений критического уровня
случайным блуж\-да\-ни\-ем в течение рассматриваемого интервала времени
(горизонта планирования) необходимо знать распределение вероятностей
максимума суммы приращений блуждания. К~сожалению, практически
никогда распределение элементарных приращений не известно, поэтому
точное вычисление этого распределения невозможно. Даже когда есть
основания принять определенную модель такого распределения,
вычисление исключительно трудоемко. Поэтому на практике точное
распределение заменяют его аппроксимацией, в качестве которой
рассматривается асимптотическая аппроксимация, основанная на
(функциональной) центральной предельной теореме,~--- так на\-зы\-ва\-емое
полунормальное распределение~--- распределение модуля нормально
распределенной случайной величины, совпадающее с распределением
максимума стандартного винеровского процесса на единичном отрезке.

Итак, рассмотрим обобщенный процесс Кокса, определяемый соотношением~(\ref{e1-dul}) 
с ${\sf E}X_1\hm=0$, $0\hm<\sigma^2\hm={\sf D}X_1\hm<\infty$. Обозначим
$$
\overline S(t)=\max\limits_{0\le\tau\le t}S(\tau)\,;\enskip \underline
S(t)=\min\limits_{0\le\tau \le t}S(\tau)\,.
$$
Приведем необходимые и достаточные условия слабой сходимости
одномерных распределений случайных процессов $\overline S(t)$ и
$\underline S(t)$, скачки которых обладают указанными выше
свойствами.

Стандартную нормальную функцию распределения и ее плотность будем
обозначать $\Phi(x)$ и $\phi(x)$ соответственно:
$$
\Phi(x)=\int\limits_{-\infty}^{x}\phi(z)\,dz\,,\ 
\phi(x)=\fr{1}{\sqrt{2\pi}}e^{-x^2/2}\,,\ \ \ x\in\r\,.
$$
Функцию распределения максимума стандартного винеровского процесса
на отрезке $[0,1]$ обозначим $G(\cdot)$,
$$
G(x)=2\Phi(\max\{0,x\})-1\,, \enskip  x\in\r\,.
$$
Несложно видеть, что если $X$~--- случайная величина со стандартным
нормальным распределением, то $G(x)\hm={\sf P}(|X|<x)$. Символ
$\Longrightarrow$ будет обозначать сходимость по распределению.

Рассмотрим для начала независимые необязательно одинаково
распределенные случайные величины $Y_1,Y_2,\ldots$ с ${\sf
E}Y_i\hm=0$ и $0\hm<\sigma^2_i\hm={\sf D}Y_i\hm<\infty$, $i\hm\geqslant 1$. Для $k\hm\geqslant
1$ обозначим
$$
S_k=Y_1+\ldots+Y_k\,, \enskip B^2_k=\sigma^2_1+\ldots+\sigma^2_k\,,
$$
$$
\overline S_k=\max\limits_{1\le i\le k}S_i\,,\enskip \underline
S_k=\min_{1\le i\le k}S_i\,.
$$
Предположим, что случайные величины $Y_1,Y_2,\ldots$ удовлетворяют
условию Линдеберга: для любого $\alpha\hm>0$
$$
\lim\limits_{k\to\infty}\fr{1}{B^2_k}\sum\limits_{i=1}^{k}\int\limits_{|x|\geqslant \alpha
B_k}^{} x^2\,d{\sf P}(Y_i<x)=0\,.
$$

Хорошо известно, что при таких предполо\-же\-ниях
\begin{align*}
{\sf P}\left(\fr{\overline S_k}{B_k}<x\right)&\Longrightarrow
G(x)\,,\\
{\sf P}\left(\fr{\underline
S_k}{B_k}<x\right)&\Longrightarrow 1-G(-x)\,,\ \ k\to\infty
\end{align*}
(это одно из проявлений так называемого принципа инвариантности).
Приведем аналог этого результата для обобщенных дважды
стохастических пуассоновских процессов с нулевым средним.

Пусть $d(t)$~--- положительная функция, неограниченно возрастающая
при $t\hm\to\infty$.

\medskip

\noindent
\textbf{Теорема~1.} \textit{Предположим, что $\Lambda
(t)\hm\longrightarrow\infty$ по вероятности при $t\hm\to\infty$.
Одномерные распределения нормированного процесса экстремумов
обобщенного процесса Кокса слабо сходятся к некоторому
распределению, т.\,е.}
$$
\fr{\overline S(t)}{\sigma\sqrt{d(t)}} \Longrightarrow\ 
\overline Z\,,\enskip \fr{\underline S(t)}{\sigma\sqrt{d(t)}}
\Longrightarrow  \underline Z \ (t\to\infty)\,,
$$
\textit{тогда и только тогда, когда существует неотрицательная случайная
величина $U$ такая, что}
\begin{equation}
\fr{\Lambda(t)}{d(t)}\Longrightarrow U\enskip
(t\to\infty)\,.
\label{e2-dul}\end{equation}
\textit{При этом}
\begin{gather*}
{\sf P}(\overline Z<x) = {\sf E}G\left(\fr{x}{\sqrt{U}}\right)\,,\\ 
 {\sf
P}(\underline Z<x) =1-{\sf E}G\left(-\fr{x}{\sqrt{U}}\right)\ \ \ x\in\r\,.
\end{gather*}

\smallskip

\noindent
Д\,о\,к\,а\,з\,а\,т\,е\,л\,ь\,с\,т\,в\,о~теоремы~1 приведено в 
книге~\cite{KorolevSokolov2008}.

\smallskip

\noindent
\textbf{Следствие 1.} \textit{В условиях теоремы~$1$
$$
{\sf P}\left(\fr{\overline S(t)}{\sigma \sqrt{d(t)}} <x\right)
\Longrightarrow G(x)
$$ 
и 
$${\sf P}\left(\fr{\underline
S(t)}{\sigma \sqrt{d(t)}} <x\right) \Longrightarrow 1-G(-x)
$$ при
$t\hm\to\infty$ тогда и только тогда, когда}
$$
\fr{\Lambda(t)}{d(t)} \Longrightarrow   1 \enskip (t\to\infty)\,.
$$

\smallskip

Это утверждение является непосредственным следствием теоремы~1 с
учетом идентифици\-ру\-емости семейства масштабных смесей функций
распределения $G$ (и, следовательно, $1\hm-G(-x)$).

\section{Оценки скорости сходимости экстремумов обобщенных дважды стохастических
пуассоновских процессов}

В нестационарных потоках экстремальных событий, описываемых
обобщенными дважды сто\-хастическими процессами, согласно теореме~1
полунор\-мальное распределение $G(x)$ трансформируется в его
масштабную смесь, в которой сме\-ши\-ва\-ющее распределение отражает
статистические закономерности поведения случайной интенсивности
потока событий. При замене точного распределения указанной
асимптотической аппроксимацией, естественно, возникает погрешность,
без точного знания которой невозможно {\it гарантированно} оценить
риски, которые, как правило, характеризуются его {\it квантильной
мерой} VaR (Value at risk). При этом гарантированные риски
вычисляются как квантили уточненного порядка, равного базовой
допустимой вероятности нежелательного экстремального события плюс
погрешность аппроксимации <<истинного>> распределения максимума
обобщенного дважды стохастического пуассоновского процесса его
предельным вариантом.

Перейдем к уточнению простейших и наиболее популярных моментных
оценок погрешности, использующих информацию о первых моментах
элементарных слагаемых, которые можно просто вычислить (оценить) на
основе статистической информации. При этом основным объектом
исследования будут модели неоднородных хаотических процессов, в
которых интенсивность потока информативных событий имеет
гам\-ма-рас\-пре\-де\-ле\-ние с параметром формы, меньшим единицы. Именно
такие потоки, к примеру, характеризуют потоки событий на крупных
биржах и в итоге приводят к дисперсионным гам\-ма-рас\-пре\-де\-ле\-ни\-ям
(Variance Gamma distributions) приращений базовых финансовых
индексов~\cite{KorolevSokolov2012}. 

Как отмечено в работе~\cite{Gleser1987}, 
потоки отказов авиационной техники также
характеризуются именно таким распределением интервалов времени между
отказами. Более того, в этой же работе показано, что процесс
восстановления с гам\-ма-рас\-пре\-де\-лен\-ны\-ми интервалами времени между
восстановлениями является смешанным пуассоновским тогда и только
тогда, когда параметр формы гам\-ма-рас\-пре\-де\-ле\-ния не превосходит
единицы.

\smallskip

\noindent
\textbf{Лемма 1.} \textit{Пусть $\beta_3\hm={\sf E}|X_1|^3\hm<\infty$ и
$N_{\lambda}$~--- случайная величина, имеющая пуассоновское
распределение с параметром $\lambda \hm>0$ и независимая от
последовательности $\{X_j\}_{j\ge 1}$. Тогда существует конечная
положительная постоянная $C\hm>0$ такая, что для всех $\lambda \hm\ge 1$
справедливо неравенство}
$$
\sup\limits_{x} \left\vert{\sf P}\left(\fr{1}{\sigma\sqrt{\lambda}}
\max\limits_{1\le i\le N_{\lambda}}\sum\limits_{j=0}^{i}
X_j<x\right)-G(x)\right\vert \le
\fr{C\beta_3}{\sqrt{\lambda}\sigma^3}.
$$

\smallskip

\noindent
Д\,о\,к\,а\,з\,а\,т\,е\,л\,ь\,с\,т\,в\,о\,.\ В~работе~\cite{KorolevSelivanova1995}
доказано утверждение, частным случаем которого является следующая
оценка. Пусть $X_1,X_2,\ldots$~--- независимые одинаково
распределенные случайные величины с ${\sf E}X_1\hm=0$, $0\hm<{\sf
D}X_1\hm=\sigma^2\hm<\infty$ и ${\sf E}|X_1|^3\hm=\beta_3\hm<\infty$. Пусть $N$~--- 
целочисленная неотрицательная случайная величина, независимая от
последовательности $X_1,X_2,\ldots$ Для $k\hm\ge 1$ положим 
$$
\overline
S_k=\max\limits_{1\le n\le k}X_1+ \cdots+X_n\,.
$$ 
Тогда существуют конечные
положительные абсолютные постоянные $C'$ и~$C''$ такие, что
\begin{multline*}
\sup\limits_x \left\vert{\sf P}\left(\fr{\overline S_N}{\sigma\sqrt{{\sf
E}N}}< x\right)-G(x)\right\vert\le{}\\
{}\le\fr{C'}{\sqrt{{\sf
E}N}}\,\fr{\beta_3}{\sigma^3}+ C''{\sf E}\left\vert\fr{N}{{\sf
E}N}-1\right\vert\,.
\end{multline*}
В случае $N=N_{\lambda}$ по неравенству Маркова имеем
$$
{\sf E}\left\vert\fr{N_{\lambda}}{{\sf
E}N_{\lambda}}-1\right\vert\le\fr {\sqrt{{\sf D}N_{\lambda}}}{{\sf
E}N_{\lambda}}=\fr{1}{\sqrt{\lambda}}\,.
$$
Отсюда с учетом всегда имеющего место неравенства
$\beta_3/\sigma^3\hm\ge 1$ вытекает желаемый результат.

\smallskip

Метод оценивания точности аппроксимации распределений экстремумов
обобщенных процессов Кокса масштабными смесями функции распределения
$G(x)$, используемый далее, основан на лемме~1 и следующем
представлении распределения экстремумов обобщенного дважды
стохастического пуассоновского процесса $\overline S(t)$,
управляемого случайной мерой $\Lambda(t)$, справедливом в силу
стохастической независимости всех случайных величин и процессов,
вовлеченных в определение обобщенного процесса Кокса: 
\begin{multline*}
{\sf P}(\overline S(t)<x)={}\\
{}=\il{0}{\infty}{\sf P}\left(\max\limits_{1\le n\le
N_1(\lambda)}\sum\limits_{j=1}^{n}X_j<x\right)
d{\sf P}(\Lambda(t)<\lambda)\,,\\
x\in\r\,.
%\label{e3-dul}
\end{multline*}
Всюду далее будем предполагать, что ${\sf E}X_1\hm=0$, ${\sf
E}X_1^2\hm=1$, ${\sf E}|X_1|^3\hm\equiv\beta^3\hm<\infty$. Предположим, что
функция $d(t)$ положительна и неограниченно возрастает при
$t\hm\to\infty$. Обозначим
\begin{align*}
\rho_t&=\sup\limits_x\left\vert{\sf
P}\left(\!\fr{\overline
S(t)}{\sqrt{d(t)}}<x\!\right)-\!\il{0}{\infty}\!G\left(\!\fr{x}{\sqrt{\lambda}}\!\right)d{\sf
P}(\Lambda<\lambda)\right\vert;\\
\Delta_t&=\sup\limits_x\left\vert{\sf P}\left(\!\fr{\Lambda(t)}{d(t)}<x\!\right)-{\sf
P}(\Lambda<x)\right\vert.
\end{align*}

\smallskip

\noindent
\textbf{Теорема 2.} \textit{Пусть выполнено условие ${\sf
E}|X_1|^3\hm<\infty$. Тогда для любого $t\hm>0$ 
$$
\rho_t\le C\beta^3{\sf E}\left[\Lambda(t)\right]^{-1/2}+
\fr{1}{2}\,\Delta_t\,,
$$ 
где $C$~--- абсолютная постоянная из леммы~$1$}.

\smallskip

\noindent
Д\,о\,к\,а\,з\,а\,т\,е\,л\,ь\,с\,т\,в\,о~теоремы~2 приведено в 
книге~\cite{KorolevSokolov2008}.

\smallskip

В качестве примера применения теоремы~2 рассмотрим ситуацию, когда
при каждом $t\hm>0$ случайная величина $N(t)$ имеет отрицательное
биномиальное распределение с параметрами $r\hm>0$ и $p\hm\in(0,1)$:
\begin{multline}
{\sf P}\left(N(t)=n\right)={}\\
{}= C_{r+n-1}^{n}p^r(1-p)^{n}\,,\enskip
n=0,1,2,\ldots
\label{e4-dul}
\end{multline}
Здесь $r>0$ и $p\hm\in(0,1)$~--- параметры и для нецелых $r$ величина
$C_{r+n-1}^{n}$ определяется как 
$$ 
C_{r+n-1}^{n} = \fr{\Gamma(r+n)}{n!\Gamma(r)}\,. 
$$ 
В~частности, при $r\hm=1$
соотношение~(\ref{e4-dul}) задает геометрическое распределение. Отрицательное
биномиальное распределение с натуральным $r$ допускает наглядную
интерпретацию в терминах испытаний Бернулли. А~именно случайная
величина, име\-ющая отрицательное биномиальное распределение с
параметрами $r\hm>0$ и $p$,~--- это число испытаний Бернулли,
проведенных до осуществления $r$-й по счету неудачи, если
вероятность успеха в одном испытании равна $1-p$.

Плотность гамма-распределения с параметром формы $r\hm>0$ и параметром
масштаба $s\hm>0$, как известно, имеет вид:
$$
g_{r,s}(x)=\fr{s^r}{\Gamma(r)}\,e^{-s x}x^{r-1}\,,\enskip x>0\,.
$$
Как известно, отрицательное биномиальное распределение с параметрами
$r\hm>0$ и $p\hm\in(0,1)$ представ\-ля\-ет собой смешанное пуассоновское
распределение, в котором параметр имеет гам\-ма-рас\-пре\-де\-ле\-ние с
параметром масштаба $s\hm=p/(1\hm-p)$ и параметром формы~$r$.

Функцию гамма-распределения с параметром масштаба~$s$ и параметром
формы~$r$ обозначим $G_{r,s}(x)$,
$$
G_{r,s}(x)=\il{0}{x}g_{r,s}(z)\,dz\,.
$$
Несложно убедиться, что
\begin{equation}
G_{r,s}(x)\equiv G_{r,1}(s x)\,.
\label{e5-dul}
\end{equation}
Случайную величину с функцией распределения $G_{r,s}(x)$ обозначим
$U(r,s)$. Хорошо известно, что 
$$
{\sf E}U(r,s)=\fr{r}{s}\,.
$$

Будем считать, что параметр $r$ фиксирован, и в качестве случайной
величины $\Lambda(t)$ возьмем величину $U(r,s)$, считая, что
$t\hm=s^{-1}$: 
$$
\Lambda(t)=U(r,t^{-1})\,.
$$ 
В~качестве функции $d(t)$
возьмем 
$$
d(t)\equiv{\sf E}\Lambda(t)={\sf E}U(r,t^{-1})\,.
$$ 
Несложно видеть, что в терминах новой параметризации 
$$
{\sf E}U(r,t^{-1})=rt\,.
$$ 
Тогда с учетом~(\ref{e5-dul}) 
\begin{multline*}
{\sf P}\left(\fr{\Lambda(t)}{d(t)}<x\right)={\sf P}(U(r,t^{-1})<xrt)={}\\
{}=
{\sf P}(U(r,1)<xr)={\sf
P}(U(r,r)<x)=G_{r,r}(x)\,.
\end{multline*}
Обратим внимание, что функция
распределения, стоящая в правой части последнего соотношения, не
зависит от~$t$. Поэтому при указанном выборе функции $d(t)$ условие~(\ref{e2-dul}) 
выполняется тривиальным образом, причем для всех $t\hm>0$
$$
\Delta_t=0\,.
$$

Вычислим ${\sf E}[\Lambda(t)]^{-1/2}$, предполагая, что выполнено
условие $r>{1}/{2}$. Имеем
\begin{multline*}
{\sf E}[\Lambda(t)]^{-1/2}={\sf
E}[U(r,t^{-1}]^{-1/2}={}\\
{}=\il{0}{\infty}\fr{e^{-x/t}x^{r-1-1/2}}{t^r\Gamma(r)}\,dx
=
\fr{\Gamma(r-{1}/{2})}{t^{1/2}\Gamma(r)}\,.
\end{multline*}
Таким образом, приходим к следующему утверждению, являющемуся
частным случаем теоремы~2.

\smallskip

\noindent
\textbf{Следствие 2.} \textit{Пусть случайная величина $N(t)$ имеет
отрицательное биномиальное распределение с па\-ра\-мет\-ра\-ми $r\hm>0$ и
$p=(1+t)^{-1}$, где $t>0$. Предположим, что ${\sf
E}|X_1|^3\equiv\beta^3<\infty$. Тогда для каждого $t>0:$}

\noindent $1^{\circ}$. \textit{При} $r>1/2$
\begin{multline*}
\sup\limits_x\left\vert {\sf
P}(\overline S(t)<x\sqrt{rt})-\il{0}{\infty}G\left(\fr{x}{\sqrt{y}}\right)dG_{r,r}(y)
\right\vert\le{}\\
{}\le
C\fr{\Gamma(r-{1}/{2})}{\Gamma(r)}\,\fr{\beta^3}{\sqrt{t}}\,,
\end{multline*}
\textit{где $C$~--- абсолютная постоянная из леммы~$1$.}

\noindent $2^{\circ}$. \textit{При} $r\hm<1/2$
\begin{multline*}
\sup\limits_x\left\vert{\sf
P}(\overline
S(t)<x\sqrt{rt})-\il{0}{\infty}G\left(\fr{x}{\sqrt{y}}\right)dG_{r,r}(y)\right\vert\le{}\\
{}\le
\fr{C^{2r}\beta^{6r}}{\Gamma(r)}\left(\fr{1}{r}+\fr{1}{(1/2-r)}\right)
\fr{1}{t^r}\,,
\end{multline*}
\textit{где $C$~--- абсолютная постоянная из леммы~$1$.}

\noindent $3^{\circ}$. \textit{При} $r\hm=1/2$
\begin{multline*}
\sup\limits_x\left\vert{\sf
P}\left(\overline
S(t)<x\sqrt{\fr{t}{2}}\right)-{}\right.\\
\left.{}-\il{0}{\infty}G\left(\fr{x}{\sqrt{y}}\right)
dG_{1/2,1/2}(y)\right\vert\le{}\\
{}\le
\fr{C\beta^3}{\sqrt{\pi}}
\left(2+\ln\left(1+\fr{t}{C^2\beta^6}\right)\right)\fr{1}{\sqrt{t}}\,,
\end{multline*}
\textit{где $C$~--- абсолютная постоянная из леммы~$1$.}

\smallskip

\noindent
Д\,о\,к\,а\,з\,а\,т\,е\,л\,ь\,с\,т\,в\,о\,.~Пункт~$1^{\circ}$, по сути, уже
доказан. Докажем пункт~$2^{\circ}$. Пусть, как и ранее,
$g_{r,r}(\lambda)$~--- плотность гам\-ма-рас\-пре\-де\-ле\-ния с параметрами
$r,r$; $C$~--- константа из леммы~1. Из приведенных выше рассуждений
следует, что
\begin{multline*}
\hspace*{-6pt}\rho_t=\sup\limits_x\left\vert{\sf P}(\overline
S(t)<x\sqrt{rt})-\!\int\limits_{0}^{\infty}\!
G\left(\fr{x}{\sqrt{y}}\right)dG_{r,r}(y)\right\vert\le{}
\\
\hspace*{-6pt}{}\le \int\limits_0^{\infty}\sup\limits_x \left\vert{\sf
P}\left(\!\fr{1}{\sqrt{\lambda rt}}\sum\limits_{j=1}^{N_{\lambda rt}}X_j<x
\!\right)-G(x)\right\vert g_{r,r}(\lambda)d\lambda,
\end{multline*}
где $N_{\lambda rt}$~--- пуассоновская случайная величина с
параметром $\lambda rt$. Заметим, что оценка равномерного расстояния
между функциями распределения, даваемая леммой~1, при малых
$\lambda$ заведомо больше единицы, что тривиально для равномерного
расстояния между функциями распределения. Используя этот факт,
получаем
\begin{multline*}
\int\limits_0^{\infty}\sup\limits_x \left\vert{\sf P}\left(\fr{1}{\sqrt{\lambda
rt}}\sum\limits_{j=1}^{N_{\lambda rt}}X_j<x
\right)-{}\right.\\
\left.{}-G(x)\vphantom{\sum\limits_{j=1}^N}\right\vert g_{r,r}(\lambda)d\lambda\le{}\\
{}\le
\int\limits_0^{\infty} \min\left\{1,\, \fr{C\beta^3}{\sqrt{\lambda
r t}}\right\} g_{r,r}(\lambda) d\lambda={}
\\
{}=\int\limits_0^{C^2\beta^6/(rt)}g_{r,r}(\lambda)
d\lambda+\int\limits_{C^2\beta^6/(rt)}^{\infty}\fr{C\beta^3}{\sqrt{\lambda
r t}} g_{r,r}(\lambda) d\lambda \equiv{}\\
{}\equiv I_1+I_2\,.
\end{multline*}
Оценим первый интеграл:
\begin{multline*}
I_1 =\int\limits_0^{C^2\beta^6/(rt)}g_{r,r}(\lambda)
d\lambda={}\\
{}=\fr{1}{\Gamma(r)}\int\limits_0^{C^2\beta^6/(rt)}r^r
e^{-r\lambda} \lambda^{r-1}d\lambda={}
\\
{}=\fr{1}{\Gamma(r)}\int\limits_0^{C^2\beta^6/t}e^{-u}u^{r-1}du \le{}\\
{}\le
\fr{1}{\Gamma(r)}\int\limits_0^{C^2\beta^6/t}u^{r-1}\,du
=\fr{1}{r\Gamma(r)}\,\fr{C^{2r}\beta^{6r}}{t^r}\,.
\end{multline*}
Оценим второй интеграл. При $r<1/2$ имеем:
\begin{multline*}
I_2=\fr{C\beta^3}{\Gamma(r)\sqrt{ r
t}}\int\limits_{C^2\beta^6/(rt)}^{\infty}\fr{r^r
e^{-r\lambda}\lambda^{r-1} }{\sqrt{\lambda}}\,d\lambda
={}\\
{}=\fr{C\beta^3}{\Gamma(r)\sqrt{t}}
\int\limits_{C^2\beta^6/t}^{\infty}e^{-u}u^{r-3/2}\,du\le{}\\
{}
\le\fr{C\beta^3}{\Gamma(r)\sqrt{t}}e^{-C^2\beta^6/t}
\int\limits_{C^2\beta^6/t}^{\infty}u^{r-3/2}\,du\le{}\\
{}\le
\fr{C\beta^3}{\Gamma(r)\sqrt{t}}\left(r-\fr{1}{2}\right)^{-1}
u^{r-1/2}\Big|_{u=C^2\beta^6/t}^{u=\infty}={}\\
{}
=\fr{C\beta^3}{\Gamma(r)\sqrt{t}}\fr{1}{1/2-r}\,
\fr{C^{2r-1}\beta^{6r-3}}{t^{r-1/2}}={}\\
{}=
\fr{C^{2r}\beta^{6r}}{\Gamma(r)(1/2-r)}\,\fr{1}{t^r}\,.
\end{multline*}
Таким образом,
\begin{multline*}
 \sup\limits_x\left\vert{\sf P}(\overline S(t)<x\sqrt{rt})-\int\limits_{0}^{\infty}
G\left(\fr{x}{\sqrt{y}}\right)dG_{r,r}(y)\right\vert \le{}\\
{}\le
\fr{C^{2r}\beta^{6r}}{\Gamma(r)}\left(\fr{1}{r}+\fr{1}{(1/2-r)}\right)\fr{1}{t^r}\,,
\end{multline*}
что и требовалось доказать.


Докажем пункт $3^{\circ}$. Несложно видеть, что в этом случае
\begin{multline*}
I_2=\fr{C\beta^3}{\Gamma(r)\sqrt{ r
t}}\int\limits_{C^2\beta^6/(rt)}^{\infty}\fr{r^r
e^{-r\lambda}\lambda^{r-1} }{\sqrt{\lambda}}\,d\lambda
={}\\
{}=\fr{C\beta^3}{\Gamma(r)\sqrt{t}}\int\limits_{C^2\beta^6/t}^{\infty}e^{-u}u^{r-3/2}\,du={}
\\{}
=\fr{C\beta^3}{\Gamma(1/2)\sqrt{t}}\int\limits_{C^2\beta^6/t}^{\infty}e^{-u}u^{-1/2}\,du={}\\
{}=
\fr{C\beta^3}{\sqrt{\pi t}}\,E_1\left(\fr{C^2\beta^6}t\right)\le
\fr{C\beta^3}{\sqrt{\pi t}}\ln\left(1+\frac{t}{C^2\beta^6}\right)\,,
\end{multline*}
где $E_1(\cdot)$~--- интегральная показательная функция. Тогда
\begin{multline*}
\sup\limits_x\left\vert{\sf P}(\overline
S(t)<x\sqrt{rt})-\int\limits_{0}^{\infty}
G\left(\fr{x}{\sqrt{y}}\right)dG_{r,r}(y)\right\vert 
\le{}\\
{}\le
\fr{C\beta^3}{\sqrt{\pi}}\left(2+\ln\left(1+\fr{t}{C^2\beta^6}\right)\right)
\fr{1}{\sqrt{t}}\,.
\end{multline*}
Теорема доказана.

\smallskip

Несложно видеть, что при $r\hm=1$ предельная функция распределения
представляет собой показательное распределение с параметром
$\sqrt{2}$.

\smallskip

С помощью теоремы~3 и следствия~2 можно получить следующую оценку
точности приближения распределения экстремумов геометрических
случайных сумм показательным распределением.

\smallskip

\noindent
\textbf{Следствие 3.} \textit{Пусть случайная величина $N(t)$ имеет
геометрическое распределение с параметром $p\hm=(1\hm+t)^{-1}$, где $t\hm>0$.
Предположим, что ${\sf E}|X_1|^3\hm\equiv\beta^3\hm<\infty$. Тогда для
каждого $t\hm>0$}
 $$
 \sup\limits_{x\ge0}|{\sf P}(\overline
S(t)<x\sqrt{t})-(1-e^{-\sqrt{2}x})|\le
C\sqrt{\pi}\fr{\beta^3}{\sqrt{t}}\,,
$$ 
\textit{где $C$~--- абсолютная
постоянная из леммы}~1.

\section{Обсуждение}

Как показывает следствие~2, при $t\hm\to\infty$
\begin{multline*}
\sup\limits_x\left\vert{\sf P}\left(\overline
S(t)<x\sqrt{rt}\right)-\il{0}{\infty}G\left(\fr{x}{\sqrt{y}}\right)dG_{r,r}(y)\right\vert 
={}\\
{}=\begin{cases}
O\left(\fr{1}{\sqrt{t}}\right)\,,& \mbox{ если } r>\fr{1}{2}\,;\\[9pt]
O\left(\fr{\ln t}{\sqrt{t}}\right)\,,& \mbox{ если } r=\fr{1}{2}\,;\\[9pt]
O\left(\fr{1}{t^r}\right)\,,& \mbox{ если } r<\fr{1}{2}\,.
\end{cases}
\end{multline*}

В работе~\cite{GZK2006} показано, что именно такая зависимость от
параметра~$r$ присуща скорости схо\-ди\-мости распределений
<<асимптотически нормальных>> в классическом смысле статистик к
распределе\-нию Стьюдента с $2r$ степенями свободы при замене объема
выборки случайной величиной с отрицательным биномиальным
распределением~(\ref{e4-dul}) с $p\hm=(1\hm+t)^{-1}$ при $t\hm\to\infty$. Распределение
Стьюдента с $2r$ степенями свободы является масштабной смесью
нормальных законов с нулевым средним, в которой смешивающим является
гам\-ма-рас\-пре\-де\-ле\-ние $G_{r,r}$. В~работе~\cite{Nefedova2011} на
примере сумм случайного чис\-ла независимых случайных величин с
индексом, имеющим указанное отрицательное биномиальное
распределение, показано, что такой порядок скорости сходимости
является правильным. Таким образом, результаты данной статьи вполне
согласуются с упомянутыми работами и распространяют указанную
закономерность на <<асимптотически полунормальные>> статистики,
каковыми являются максимальные суммы.

{\small\frenchspacing
{%\baselineskip=10.8pt
\addcontentsline{toc}{section}{Литература}
\begin{thebibliography}{99}

\bibitem{BKSSh2007} 
\Au{Бенинг В.\,Е., Королев В.\,Ю., Соколов~И.\,А., Шоргин~С.\,Я.}
Рандомизированные модели и методы тео\-рии на\-деж\-ности информационных и
технических сис\-тем.~--- М.: ТОРУС ПРЕСС, 2007. 248~с.

\bibitem{BeningKorolev2004} 
\Au{ Бенинг В.\,Е., Королев В.\,Ю.} Об использовании распределения Стьюдента в
задачах теории вероятностей и математической статистики~// Теория
вероятностей и ее применения, 2004. Т.~49. Вып.~3. С.~417--435.

\bibitem{KorolevBeningShorgin2011} 
\Au{Королев В.\,Ю., Бенинг В.\,Е., Шоргин~С.\,Я.} 
Математические основы теории риска.~--- 2-е изд.,
перераб. и доп.~--- М.: Физматлит, 2011. 620~с.

\bibitem{KorolevSokolov2012} 
\Au{Королев В.\,Ю., Соколов И.\,А.} Скошенные распределения Стьюдента,
дисперсионные гам\-ма-рас\-пре\-де\-ле\-ния и их обобщения как асимптотические
аппроксимации~// Информатика и её примерения, 2012. Т.~6. Вып.~1.
С.~2--10.

\bibitem{KorolevSokolov2008} 
\Au{Королев В.\,Ю., Соколов И.\,А.} Математические модели
неоднородных потоков экстремальных событий.~--- М.: ТОРУС ПРЕСС,
2008.

\bibitem{BeningKorolev2002}
\Au{Bening V., Korolev V.} Generalized poisson models and their
applications in insurance and finance.~--- Utrecht: VSP, 2002.

\bibitem{GnedenkoKorolev1996} 
\Au{Gnedenko B.\,V., Korolev V.\,Yu.} Random summation:
Limit theorems and applications.~--- Boca Raton: CRC Press, 1996.

\bibitem{Korolev2011} 
\Au{Королев В.\,Ю.} Ве\-ро\-ят\-но\-ст\-но-ста\-ти\-сти\-че\-ские методы
декомпозиции волатильности хаотических процессов.~--- М.: Изд-во
Московского ун-та, 2011. 510~с.

\bibitem{Gleser1987} 
\Au{Gleser L.\,J.} The gamma distribution as a mixture of exponential
distributions: Technical Report \# 87-28.~--- West Lafayette: Purdue
University, 1987. 6~p.

\bibitem{KorolevSelivanova1995} 
\Au{Korolev V.\,Yu., Selivanova D.\,O.} 
Convergence rate estimates in some limit theorems for
maximum random sums~// J.~Math. Sci., 1995.
Vol.~76. No.\,1. P.~2163--2168.

\bibitem{GZK2006} 
\Au{Гавриленко С.\,В., Зубов В.\,Н., Королев В.\,Ю.} Оценка
скорости сходимости распределений регулярных статистик, построенных
по выборкам случайного объема с отрицательным биномиальным
распределением, к распределению Стьюдента~// Статистические методы
оценивания и проверки гипотез: Межвузовский сб. научных тр.~--- 
Пермь: ПГУ, 2006. C.~118--134.

\label{end\stat}

\bibitem{Nefedova2011} 
\Au{Нефедова Ю.\,С.} Оценки скорости сходимости в предельной
теореме для отрицательных биномиальных случайных сумм~//
Статистические методы оценивания и проверки гипотез: Межвузовский
сб. научных тр.~--- Пермь: ПГУ, 2011. C.~46--61.
\end{thebibliography}
}
}


\end{multicols}