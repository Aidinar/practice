\def\stat{korolev}

\def\tit{ОБОБЩЕННОЕ РАСПРЕДЕЛЕНИЕ ЛАПЛАСА КАК~ПРЕДЕЛЬНОЕ
ДЛЯ~СЛУЧАЙНЫХ СУММ И~СТАТИСТИК, ПОСТРОЕННЫХ ПО~ВЫБОРКАМ СЛУЧАЙНОГО
ОБЪЕМА$^*$}

\def\titkol{Обобщенное распределение Лапласа как~предельное
для~случайных сумм и~статистик, построенных по~выборкам} % случайного объема}

\def\autkol{В.\,Ю.~Королев, В.\,Е.~Бенинг, Л.\,М.~Закс, А.\,И.~Зейфман}

\def\aut{В.\,Ю.~Королев$^1$, В.\,Е.~Бенинг$^2$, Л.\,М.~Закс$^3$, А.\,И.~Зейфман$^4$}

\titel{\tit}{\aut}{\autkol}{\titkol}

{\renewcommand{\thefootnote}{\fnsymbol{footnote}}\footnotetext[1]
{Работа поддержана Российским фондом фундаментальных
исследований (проекты 11-01-12026-офи-м, 11-01-00515а, 
11-07-00112а и 12-07-00115а), а также Министерством образования и науки РФ в рамках
ФЦП <<Научные и на\-уч\-но-пе\-да\-го\-ги\-че\-ские кадры инновационной России на
2009--2013~годы>>.}}

\renewcommand{\thefootnote}{\arabic{footnote}}
\footnotetext[1]{Факультет вычислительной
математики и кибернетики Московского государственного университета
им.\ М.\,В.~Ломоносова; Институт проблем информатики РАН,
vkorolev@cs.msu.su}
\footnotetext[2]{Факультет вычислительной математики и кибернетики
Московского государственного университета им.\ М.\,В.~Ломоносова;
Институт проблем информатики РАН,
bening@cs.msu.su}
\footnotetext[3]{Альфа-банк, отдел моделирования и математической статистики, lily.zaks@gmail.com}
\footnotetext[4]{Вологодский государственный
педагогический университет;  Институт проблем информатики Российской
академии наук; ВНКЦ ЦЭМИ РАН, a\_zeifman@mail.ru}



\Abst{Доказываются предельные теоремы, устанавливающие
критерии сходимости распределений случайных сумм и статистик,
построенных по выборкам случайного объема, к обобщенному
распределению Лапласа.}

\KW{обобщенное распределение Лапласа; симметричное
устойчивое распределение; одностороннее устойчивое распределение;
масштабная смесь нормальных законов; случайная сумма; выборка
случайного объема; смешанное пуассоновское распределение}

\vskip 14pt plus 9pt minus 6pt

      \thispagestyle{headings}

      \begin{multicols}{2}

            \label{st\stat}

\section{Обобщенное распределение Лапласа}

Пусть $0\hm<\alpha\le 2$. {\it Обобщенным распределением Лапласа}
назовем абсолютно непрерывное распределение вероятностей, задаваемое
плот\-ностью
\begin{equation}
\ell_{\alpha}(x)=\fr{\alpha}{2\Gamma({1}/{\alpha})} \,
e^{-|x|^{\alpha}}\,,\enskip -\infty< x<\infty\,.
\label{e1-kor}
\end{equation}
Для упрощения дальнейших выкладок и обозначений здесь и далее в
представлении~(\ref{e1-kor}) будет использоваться лишь один параметр~$\alpha$,
который является <<характеристическим>>, определяя форму обобщенного
распределения Лапласа. При $\alpha\hm=1$ соотношение~(\ref{e1-kor}) определяет
классическое распределение Лапласа (с нулевым средним и дисперсией~2). 
При $\alpha\hm=2$ соотношение~(\ref{e1-kor}) определяет нормальное (гауссово)
распределение (с нулевым средним и дисперсией $1/2$).

Класс распределений~(\ref{e1-kor}) был введен и изучен М.\,Ф.~Субботиным в
1923~г.~\cite{Subbotin1923}. В~разных источ\-никах распределения этого
класса называются\linebreak по-раз\-но\-му. Например, на с.~74--76 книги~\cite{Evans2000} 
эти распределения называются {\it обобщенными
распределениями ошибок}, в книге~\cite{BoxTiao1973} они названы {\it
экс\-по\-нен\-ци\-аль\-но-сте\-пен\-н$\acute{\mbox{ы}}$ми} (exponential power distributions). 
В~\cite{Morgan1996} используется термин {\it обобщенные показатель\-ные
распределения}, тогда как в статьях~\cite{Varanasi1989, Nadaraja2005} 
эти распределения названы соответственно {\it
обобщенными гауссовыми} и {\it обобщенными нормальными}.
Распределения типа~(1) широко применяются в разнообразных областях
от астрономических измерений и обработки изображений до байесовского
анализа.

В работе~\cite{West1987} было замечено, что при $0\hm<\alpha\hm\le2$
распределения~(1) представимы в виде масштабных смесей
нормальных законов (этот результат также цитируется
в ~\cite{ChoySmith1997}). Для удобства приведем свое до\-ка\-зательство этого
результата.

Функцию распределения и плотность строго устойчивого распределения с
характеристическим показателем~$\alpha$ и параметром~$\theta$,
задаваемого характеристической функцией
\begin{multline}
\mathfrak{f}_{\alpha,\theta}(t)={}\\
{}=\exp\left\{-|t|^{\alpha}\exp\left\{
-\fr{i\pi\theta\alpha}{2}\,\mathrm{sign}\,t\right\}\right\}\,,\enskip 
t\in\r\,,\label{e2-kor}
\end{multline}
где $0<\alpha\le2$,
$|\theta|\le\theta_{\alpha}\hm=\min\{1,{2}/{\alpha}-1\}$ будем
обозначать соответственно $G_{\alpha,\theta}(x)$ и
$g_{\alpha,\theta}(x)$ (см., например,~\cite{Zolotarev1983}).
Функцию распределения и плотность стандартного нормального закона
будем обозначать соответственно $\Phi(x)$ и~$\phi(x)$,
$$
\phi(x)=\fr{1}{\sqrt{2\pi}}\,e^{-x^2/2}\,;\enskip
\Phi(x)=\int\limits_{-\infty}^x\phi(z)\,dz\,.
$$
Символ $\eqd$ будет обозначать совпадение распределений.

\smallskip

\noindent
\textbf{Лемма 1.} \textit{Обобщенное распределение Лапласа является
масштабной смесью нормальных законов.}

\smallskip

\noindent
Д\,о\,к\,а\,з\,а\,т\,е\,л\,ь\,с\,т\,в\,о\,.\ Из~(2) вытекает, что характеристическая
функция симметричного ($\theta\hm=0$) строго устойчивого распределения
имеет вид
\begin{equation}
\mathfrak{f}_{\alpha,0}(t)=e^{-|t|^{\alpha}}\,,\enskip t\in\r\,. \label{e3-kor}
\end{equation}
С другой стороны, хорошо известно, что сим\-мет\-рич\-ное строго
устойчивое распределение с па\-ра\-мет\-ром~$\alpha$ является мас\-штаб\-ной
смесью нормальных законов, в которой смешивающим распределением
является односторонний устойчивый закон ($\theta\hm=1$) с параметром
$\alpha/2$:
\begin{equation}
G_{\alpha,0}(x)=\int\limits_{0}^{\infty}\Phi\left(\fr{x}{\sqrt{z}}\right)dG_{\alpha/2,1}(z)\,,\enskip
x\in\r\label{e4-kor}
\end{equation}
(см., например, теорему~3.3.1 в~\cite{Zolotarev1983}). Запишем
соотношение~(\ref{e4-kor}) в терминах характеристических функций с учетом~(\ref{e3-kor}):
\begin{equation}
e^{-|t|^{\alpha}}=\int\limits_0^{\infty}\exp\left\{-\fr{t^2z}{2}\right\}g_{\alpha/2,1}(z)\,dz\,.
\label{e5-kor}
\end{equation}
Обозначим
$$
h_{\alpha/2}(z)=\fr{\alpha}{\Gamma({1}/{\alpha})}\sqrt{\fr{\pi}{2}}\,
\fr{g_{\alpha/2,1}(z)}{\sqrt{z}}\,,\enskip
z\ge0\,.
$$
Тогда, переобозначив аргумент $t\mapsto x$ и выполнив несколько
формальных тождественных преобразований равенства~(\ref{e5-kor}), будем иметь:
\begin{multline}
\ell_{\alpha}(x)=\fr{\alpha}{2\Gamma({1}/{\alpha})}\,e^{-|x|^{\alpha}}={}\\
{}=
\fr{\alpha}{\Gamma({1}/{\alpha})}\,\sqrt{\fr{\pi}{2}}\,\int\limits_0^{\infty}
\fr{\sqrt{z}}{\sqrt{2\pi}}\,\exp\left\{-\fr{x^2z}{2}\right\}\fr{g_{\alpha/2,1}(z)}{\sqrt{z}}\,dz
={}\\
{}=\int\limits_0^{\infty}\sqrt{z}\phi(x\sqrt{z})h_{\alpha/2}(z)\,dz\,.\label{e6-kor}
\end{multline}
Можно убедиться, что $h_{\alpha/2}(z)$~--- плотность распределения
неотрицательной случайной величины. Действительно,
при каждом $z\hm>0$
$$
\int\limits_{-\infty}^{\infty}\sqrt{z}\phi\left(x\sqrt{z}\right)dx=1\,.
$$
Поэтому из~(\ref{e6-kor}) вытекает, что
\begin{multline*}
1=\int\limits_{-\infty}^{\infty}\ell_{\alpha}(x)\,dx={}\\
{}=
\int\limits_{-\infty}^{\infty}\int\limits_{0}^{\infty}\sqrt{z}\,\phi(x\sqrt{z})h_{\alpha/2}(z)\,dz\,dx=
{}\\
{}=\int\limits_{0}^{\infty}h_{\alpha/2}(z)\left(\int\limits_{-\infty}^{\infty}\sqrt{z}
\phi\left(x\sqrt{z}\right)\,dx\right)dz={}\\
{}=
\int\limits_{0}^{\infty}h_{\alpha/2}(z)\,dz\,.
\end{multline*}
Если $Z_{\alpha}$~--- случайная величина, имеющая обобщенное
распределение Лапласа с параметром~$\alpha$, то соотношение~(\ref{e6-kor})
означает, что
$$
Z_{\alpha}\eqd X \sqrt{U_{\alpha/2}}\,,
$$
где $X$ и $U_{\alpha/2}$~--- независимые случайные величины, причем
$X$ имеет стандартное нормальное распределение,
$$
U_{\alpha/2}\eqd\fr{1}{V_{\alpha/2}}\,,
$$
а $V_{\alpha/2}$~--- неотрицательная абсолютно непрерывная случайная
величина с плотностью $h_{\alpha/2}(z)$. Лемма доказана.

\smallskip

Поскольку функция $h_{\alpha/2}(x)$, введенная выше, является
плотностью, то из ее определения вытекает следующее интересное
утверждение, позволяющее вычислить ${\sf E}U_{\alpha/2,1}^{-1/2}$,
несмотря на то что плот\-ность $g_{\alpha/2,1}(z)$, вообще говоря,
нельзя выписать в явном виде в терминах элементарных функций, однако
не имеющее прямого отношения к теме данной статьи.

\smallskip

\noindent
\textbf{Следствие 1.} {\it Пусть $Y_{\alpha,1}$~--- неотрицательная
случайная величина, имеющая одностороннее устойчивое распределение с
характеристическим показателем $\alpha\hm\in(0,1)$. Тогда}
$$
{\sf E}Y^{-1/2}_{\alpha,1}=\fr{\Gamma({1}/(2\alpha))}{\alpha\sqrt{2\pi}}\,.
$$

\smallskip

\noindent
\textbf{Замечание~1.} Соотношению~(\ref{e6-kor}) можно придать несколько иной вид.
Как несложно убедиться, плотность $w_{\alpha/2}$ случайной величины
$U_{\alpha/2}$ имеет вид:
\begin{multline*}
w_{\alpha/2}(z)=z^{-2}h_{\alpha/2}(z^{-1})={}\\
{}=
\fr{\alpha}{\Gamma({1}/{\alpha})}\sqrt{\fr{\pi}{2}}\,\fr{g_{\alpha/2,1}(z^{-1})}{z^{3/2}}\,,\enskip
z>0\,.
\end{multline*}
Тогда~(\ref{e6-kor}) можно записать в эквивалентном виде
\begin{multline}
\ell_{\alpha}(x)=
\fr{\alpha}{\Gamma({1}/{\alpha})}\sqrt{\fr{\pi}{2}}\int\limits_0^{\infty}
\fr{1}{\sqrt{2\pi
z}}\times{}\\
{}\times\exp\left\{-\fr{x^2}{2z}\right\}\fr{g_{\alpha/2,1}(z^{-1})}{z^{3/2}}\,dz={}\\
{}=
\int\limits_0^{\infty}\fr{1}{\sqrt{z}}\,\phi\left(\fr{x}{\sqrt{z}}\right)w_{\alpha/2}(z)\,dz\,.
\label{e7-kor}
\end{multline}

\smallskip

\noindent
\textbf{Пример~1.} Рассмотрим случай $\alpha\hm=1$. Тогда, как известно,
$G_{1/2,1}(x)$~--- это распределение Леви (распределение момента
времени достижения стандартным винеровским процессом единичного
уровня, являющееся также частным случаем обратного нормального
распределения). Ему соответствует плотность
$$
g_{1/2,1}(z)=\fr{1}{z^{3/2}\sqrt{2\pi}}\,\exp\left\{-\fr{1}{2z}\right\}\,,\enskip
z>0\,.
$$
В этом случае
\begin{multline*}
w_{1/2}(z)=\sqrt{\fr{\pi}{2}}\,\fr{g_{1/2,1}(z^{-1})}{z^{3/2}}={}\\
{}=\fr{\sqrt{\pi}
z^{3/2}e^{-z/2}}{\sqrt{2}\sqrt{2\pi}
z^{3/2}}=\fr{1}{2}\,e^{-z/2}\,,
\end{multline*}
т.\,е.\ в рассматриваемом случае
$w_{1/2}(z)\hm=(1/2)e^{-z/2}$~--- плотность
экспоненциального распределения с параметром 1/2.
При этом в соответствии~(\ref{e7-kor})
$$
\ell_1(x)=\fr{1}{2}\,e^{-|x|}=\int\limits_0^{\infty}\fr{1}{\sqrt{z}}\,\phi\left(\fr{x}{\sqrt{z}}\right)
\fr{e^{-z/2}}{2}\,dz\,,
$$
что согласуется с хорошо известным свойством классического
распределения Лапласа (см., например, лемму~12.7.1 в~\cite{KorolevBeningShorgin2011-k}).

\smallskip

В прикладной теории вероятностей хорошо известен принцип,
восходящий, по-ви\-ди\-мо\-му, к работе~\cite{GnedenkoKolmogorov1949},
согласно которому та или иная модель\linebreak может считаться в достаточной
мере обоснованной только тогда, когда она является {\it
асимптотической аппроксимацией}, т.\,е.\ когда существует довольно
простая предельная теорема, в которой рассматрива\-емая модель
выступает в качестве предельного распределения. 

В~книге~\cite{GnedenkoKorolev1996-k} 
прослежена глубокая связь этого принципа
с универсальным принципом неубывания энтропии в замкнутых системах.
Как было показано, при $0\hm<\alpha\hm\le2$ обобщенное распределение
Лапласа имеет вид масштабных смесей нормальных законов. Как
известно, нормальное распределение обладает максимальной
(дифференциальной) энтропией среди всех распределений, носителем
которых является вся чис\-ло\-вая прямая, и имеющих конечный второй
момент. Если бы моделируемая слож\-ная сис\-те\-ма была информационно
изолирована от окружающей среды, то в соответствии с принципом
неубывания энтропии, который в тео\-рии вероятностей проявляется в
виде предельных тео\-рем~\cite{GnedenkoKorolev1996-k}, наблюдаемые
статистические распределения ее характеристик были бы неотличимы от
нормального. Но поскольку любая математическая модель по своему
определению не может учесть все факторы, влияющие на состояние или
эволюцию моделируемой системы, то параметры этого нормального закона
изменяются в зависимости от состояния среды, внешней по отношению к
моделируемой системе. Другими словами, эти параметры являются
случайными и изменяются под влиянием информационных потоков между
системой и внешней средой. 
%
Таким образом, во многих ситуациях
разумные модели статистических закономерностей изменения параметров
сложных сис\-тем должны иметь вид смесей нормальных законов, част\-ным
случаем которых является обобщенное распределение Лапласа~(1).

По-видимому, до сих пор обобщенное распределение Лапласа
использовалось во многих задачах прежде всего в силу относительной
прос\-то\-ты его аналитического представления. <<Асимптотического>>
обоснования адекватности подобной модели пока дано не было. В~данной
работе будет показано, что обобщенное распределение Лапласа может
выступать в качестве предельного в довольно простых предельных
теоремах для регулярных статистик, в частности в схеме случайного
суммирования случайных величин, и, следовательно, наряду с
нормальным законом  может считаться асимптотической аппроксимацией
для распределений многих процессов, например сходных с неоднородными
случайными блужданиями.

Функции распределения, соответствующие плотностям
$\ell_{\alpha}(x)$, $h_{\alpha/2}(z)$ и $w_{\alpha/2}(z)$, будем
обозначать соответствующими заглавными латинскими буквами:
$L_{\alpha}(x)$, $H_{\alpha/2}(z)$ и $W_{\alpha/2}(z)$. Тогда, как
несложно видеть, соотношения~(6) и~(7) эквивалентны соотношениям:
\begin{align}
L_{\alpha}(x)&=\int\limits_0^{\infty}\Phi\left(x\sqrt{z}\right)dH_{\alpha/2}(z)\,;\label{e8-kor}\\
L_{\alpha}(x)&=\int\limits_0^{\infty}\Phi\left(\fr{x}{\sqrt{z}}\right)dW_{\alpha/2}(z)\,.\label{e9-kor}
\end{align}

\section{Критерий сходимости распределений случайных сумм к~обобщенному распределению Лапласа}

Всюду далее символ $\Longrightarrow$ обозначает сходимость по
распределению.

Рассмотрим последовательность независимых одинаково распределенных
случайных величин $X_1,X_2,\ldots$, заданных на некотором
вероятностном пространстве $(\Omega,\, \mathfrak{A},\,{\sf P})$.
Будем предполагать, что ${\sf E}X_1\hm=0$, $0\hm<\sigma^2\hm={\sf D}X_1\hm<\infty$. 
Для натурального $n\hm\ge1$ положим
$$
S_n=X_1+\cdots+X_n\,.
$$
Пусть $N_1,N_2,\ldots$~--- последовательность целочисленных
неотрицательных случайных величин, заданных на том же самом
вероятностном пространстве так, что при каждом $n\hm\ge1$ случайная
величина $N_n$ независима от последовательности $X_1,X_2,\ldots$
Всюду далее для определенности будем считать, что $\sum\limits_{j=1}^0\hm=0$.

Принято считать, что случайная последовательность $N_1,N_2,\ldots$
неограниченно возрастает ($N_n\hm\longrightarrow\infty$) по
вероятности, если для любого $m\hm\in(0,\infty)$ ${\sf P}(N_n\hm\le
m)\hm\longrightarrow 0$ при $n\to\infty$.

\smallskip

\noindent
\textbf{Лемма 1.} {\it Предположим, что случайные величины
$X_1,X_2,\ldots$ и $N_1,N_2,\ldots$ удовлетворяют указанным выше
условиям, причем $N_n\hm\longrightarrow\infty$ по вероятности при
$n\hm\to\infty$. Для того чтобы существовала такая функция
распределения~$F(x)$, что}
$$
{\sf P}\left(\fr{S_{N_n}}{\sigma\sqrt{n}}<x\right) \Longrightarrow
F(x)\enskip (n\to\infty)\,,
$$
\textit{необходимо и достаточно, чтобы существовала функция распределения
$Q(x)$, удовлетворяющая условиям:}
\begin{gather*}
Q(0)=0\,;\enskip
F(x)=\int\limits_{0}^{\infty}\Phi\left(\fr{x}{\sqrt{y}}\right)dQ(y)\,;\\
x\in\mathbb{R}\,;\enskip {\sf P}(N_n<nx)\Longrightarrow Q(x) \enskip
(n\to\infty)\,.
\end{gather*}


\smallskip

\noindent
Д\,о\,к\,а\,з\,а\,т\,е\,л\,ь\,с\,т\,в\,о\,.\ \ Данная лемма доказана в работе~\cite{Korolev1994}.

\smallskip

\noindent
\textbf{Теорема~2.} {\it Предположим, что случайные величины
$X_1,X_2,\ldots$ и $N_1,N_2,\ldots$ удовлетворяют указанным выше
условиям, причем $N_n\hm\longrightarrow\infty$ по вероятности при
$n\hm\to\infty$. Для того чтобы}
$$
{\sf P}\left(\fr{S_{N_n}}{\sigma\sqrt{n}}<x\right) \Longrightarrow
L_{\alpha}(x)\enskip (n\to\infty)\,,
$$
\textit{необходимо и достаточно, чтобы}
$$
{\sf P}(N_n<nx)\Longrightarrow W_{\alpha/2}(x) \enskip (n\to\infty)\,.
$$


%\smallskip

\noindent
Д\,о\,к\,а\,з\,а\,т\,е\,л\,ь\,с\,т\,в\,о\,.\ \ Данное утверждение является
непосредственным следствием леммы~1 с $Q(x)\hm=W_{\alpha/2}(x)$ и
представления~(\ref{e9-kor}).

\vspace*{-4pt}

\section{Критерий сходимости распределений регулярных статистик,
построенных по~выборкам случайного объема, к~обобщенному
распределению Лапласа}

\vspace*{-1pt}

Рассмотрим традиционную для математической статистики постановку
задачи. Пусть $T_n\hm=T_n(X_1,\ldots,X_n)$~--- некоторая статистика, то
есть измеримая функция от случайных величин $X_1,\ldots,X_n$. Для
каждого $n\hm\ge1$ определим случайную величину $T_{N_n}$, положив
$$
T_{N_n}(\omega) = T_{N_n(\omega)}\left(X_1(\omega),\ldots,X_{N_n(\omega)}(\omega)\right)
$$
для каждого элементарного исхода $\omega\hm\in\Omega$.

Будем говорить, что статистика $T_n$ асимптотически нормальна, если
существуют $\delta\hm>0$ и $\theta\hm\in\r$ такие, что
\begin{equation}
{\sf P}\left(\delta\sqrt{n}\left(T_n-\theta\right)<x\right)\Longrightarrow\Phi(x)
\enskip (n\to\infty)\,.\label{e10-kor}
\end{equation}

\smallskip

\noindent
\textbf{Лемма 2.} {\it Предположим, что $N_n\hm\longrightarrow\infty$ по
вероятности. Пусть статистика $T_n$ асимптотически нормальна в
смысле}~(\ref{e10-kor}). \textit{Для того чтобы существовала такая функция
распределения $F(x)$, что}
$$
{\sf P}\left(\delta\sqrt{n}\left(T_{N_n}-\theta\right)<x\right)
\Longrightarrow F(x)\enskip (n\to\infty)\,,
$$
\textit{необходимо и достаточно, чтобы существовала функция распределения
$Q(x)$, удовлетворяющая условиям:}

\vspace*{-2pt}

\noindent
\begin{gather*}
Q(0)=0\,;\enskip F(x)=\int\limits_{0}^{\infty}\Phi\left(x\sqrt{y}\right)dQ(y)\,;\\
x\in\mathbb{R}\,;\enskip {\sf P}(N_n<nx)\Longrightarrow Q(x) \enskip
(n\to\infty)\,.
\end{gather*}


%\smallskip

\noindent
Д\,о\,к\,а\,з\,а\,т\,е\,л\,ь\,с\,т\,в\,о\,.\ \ Данная лемма, по сути, является
частным случаем теоремы~3 из~\cite{Korolev1995}, доказательство
которой, в свою очередь, основано на общих теоремах о сходимости
суперпозиций независимых случайных последовательностей~\cite{Korolev1994, Korolev1996} 
(см.\ также теорему~3.3.2 в~\cite{GnedenkoKorolev1996-k}).

\smallskip

\noindent
\textbf{Теорема 3.} {\it Предположим, что $N_n\hm\longrightarrow\infty$ по
вероятности. Пусть статистика $T_n$ асимптотически нормальна в
смысле}~(\ref{e10-kor}). \textit{Для того чтобы}
$$
{\sf P}\left(\delta\sqrt{n}\left(T_{N_n}-\theta\right)<x\right)
\Longrightarrow L_{\alpha}(x)\enskip (n\to\infty)\,,
$$
\textit{необходимо и достаточно, чтобы}
$$
{\sf P}(N_n<nx)\Longrightarrow H_{\alpha/2}(x)\enskip (n\to\infty)\,.
$$


\noindent
Д\,о\,к\,а\,з\,а\,т\,е\,л\,ь\,с\,т\,в\,о\,.\ \ Данное утверждение является
непосредственным следствием леммы~2 с $Q(x)\hm=H_{\alpha/2}(x)$ и
представления~(\ref{e8-kor}).

\vspace*{-3pt}

\section{Обсуждение}

\vspace*{-1pt}

В теоремах~2 и~3 главным условием является сходимость распределений
нормированных индексов $N_n$ к распределениям $W_{\alpha/2}$ и
$H_{\alpha/2}$ соответственно. Приведем довольно общий пример,
по\-ка\-зы\-ва\-ющий, когда эти условия можно считать выполненными.

В книге~\cite{GnedenkoKorolev1996-k} предложено моделировать эволюцию
неоднородных хаотических стохастических процессов, в част\-ности
динамику цен финансовых активов, с помощью обобщенных дважды
стохастических пуассоновских процессов (обобщенных процессов Кокса).
Этот подход получил дополнительное обоснование и развитие в 
книгах~[10, 16--18]. 
В~книгах~\cite{Korolev2011-k, KorolevSkvortsova2006} этот подход успешно
применен к моделированию процессов плазменной турбулентности. 
В~соответствии с указанным подходом поток информативных событий, в
результате каждого из которых появляется очередное <<наблюденное>>
значение рассматриваемой характеристики, описывается с помощью
точечного случайного процесса вида $M(\Lambda(t))$, где $M(t)$,
$t\hm\geq0$,~--- однородный пуассоновский процесс с единичной
интенсивностью, а $\Lambda(t)$, $t\hm\geq0$,~--- независимый от $M(t)$
случайный процесс, обладающий следующими свойствами: $\Lambda(0)\hm=0$,
${\sf P}(\Lambda(t)\hm<\infty)=1$ для любого $t\hm>0$, траектории
$\Lambda(t)$ не убывают и непрерывны справа. Процесс
$M(\Lambda(t))$, $t\hm\geq0$, называется дважды стохастическим
пуассоновским процессом (процессом Кокса). В~част\-ности, если процесс
$\Lambda(t)$ допускает представление

\noindent
$$
\Lambda(t)=\int\limits_{0}^{t}\lambda(\tau)\,d\tau\,,\enskip t\ge0\,,
$$
в котором $\lambda(t)$~--- положительный случайный процесс с
интегрируемыми траекториями, то $\lambda(t)$ можно интерпретировать
как мгновенную стохастическую интенсивность процесса Кокса.

В соответствии с такой моделью в каждый момент времени~$t$
распределение случайной величины $M(\Lambda(t))$ является смешанным
пуассоновским. Для большей наглядности рассмотрим случай, когда в
рассматриваемой модели время~$t$ остается фиксированным, а
$\Lambda(t)\hm=nU_{\alpha/2}$, где $n$~--- вспомогательный параметр,
$U_{\alpha/2}$~--- случайная величина c функцией распределения
$W_{\alpha/2}(x)$, независимая от стандартного пуассоновского
процесса $M(t)$, $t\hm\ge0$. При этом асимптотика $n\to\infty$ может
интерпретироваться как то, что (случайная) интенсивность потока
информативных событий считается очень большой. Для каждого
натурального~$n$ положим
$$
N_n=M(nU_{\alpha/2})\,.
$$
Очевидно, что так определенная случайная величина $N_n$ имеет
смешанное пуассоновское распределение
\begin{multline*}
{\sf P}(N_n=k)={\sf P}\left(M(nU_{\alpha/2})=k\right)={}\\
{}=\int\limits_0^{\infty}e^{-nz}\fr{(nz)^k}{k!}\,w_{\alpha/2}(z)\,dz\,,\enskip
 k=0,1,\ldots
\end{multline*}
Так определенная случайная величина $N_n$ может быть
интерпретирована как число событий, зарегистрированных к моменту
времени~$n$ в пуас\-со\-нов\-ском процессе со случайной интенсив\-ностью,
имеющей плотность $w_{\alpha/2}(z)$. Предположим, что случайная
величина~$U_{\alpha/2}$ и пуассоновский процесс~$M(t)$ независимы от
последовательности $X_1,X_2,\ldots$ Тогда, очевидно, при каждом~$n$
случайная величина $N_n$ также будет независима от этой
последовательности.

Обозначим $A_n(z)\hm={\sf P}(N_n\hm<nz)$, $z\hm\ge0$ ($A_n(z)\hm=0$ при $z\hm<0$).
Несложно видеть, что $A_n(x)\hm\Longrightarrow W_{\alpha/2}(x)$
($n\hm\to\infty$). Действительно, как известно, если $\Pi(x;\ell)$~---
функция распределения \mbox{Пуассона} с параметром $\ell\hm>0$ и $E(x;c)$~---
функция распределения с единственным единичным скачком в точке
$c\hm\in\r$, то
$$
\Pi(\ell x;\ell)\Longrightarrow E(x;1)\enskip (\ell\to\infty)\,.
$$
Так как для $x\in\r$
$$
A_n(x)=\int\limits_{0}^{\infty}\Pi\left(n x; n z\right)dW_{\alpha/2}(z)\,,
$$
то по теореме Лебега о мажорируемой сходимости при $n\hm\to\infty$
\begin{multline*}
A_n(x)\Longrightarrow\int\limits_{0}^{\infty}E\left(x/z;1\right)dW_{\alpha/2}(z)={}\\
{}=
\int\limits_{0}^{x}dW_{\alpha/2}(z)=W_{\alpha/2}(x)\,,
\end{multline*}
т.\,е.\ так определенные случайные величины $N_n$ удовле\-тво\-ря\-ют
условию, фигурирующему в леммe~1, с $Q(x)\hm=W_{\alpha/2}(x)$.

Аналогично пусть $V_{\alpha/2}$~--- случайная величина с функцией
распределения $H_{\alpha/2}(x)$, независимая от стандартного
пуассоновского процесса $M(t)$, $t\hm\ge0$. Для натурального~$n$
положим $N_n\hm= M\left(nV_{\alpha/2}\right)$. Несложно видеть, что
распределение случайной величины $N_n$ является смешанным
пуассоновским и имеет вид:
\begin{multline*}
{\sf P}(N_n=k)=\fr{1}{k!}\int\limits_{0}^{\infty}e^{-nz}(nz)^kh_{\alpha/2}(z)\,dz\,,\\
 k=0,1,2,\ldots
\end{multline*}
Так определенная случайная величина $N_n$ может быть
интерпретирована как число событий, зарегистрированных к моменту
времени~$n$ в пуассоновском процессе со случайной интенсивностью,
имеющей плотность $h_{\alpha/2}(z)$. Предположим, что случайная
величина $V_{\alpha/2}$ и пуассоновский процесс $M(t)$ независимы от
последовательности $X_1,X_2,\ldots$ Тогда, очевидно, при каждом~$n$
случайная величина $N_n$ также будет независима от этой
последовательности.

Как и выше, легко убедиться, что ${\sf P}(N_n\hm<nz)\hm\Longrightarrow
H_{\alpha/2}(z)$ ($n\hm\to\infty$), т.\,е.\ так определенные случайные
величины $N_n$ удовлетворяют условию, фигурирующему в леммe~2, с
$Q(x)\hm=H_{\alpha/2}(x)$.


{\small\frenchspacing
{%\baselineskip=10.8pt
\addcontentsline{toc}{section}{Литература}
\begin{thebibliography}{99}

\bibitem{Subbotin1923} 
\Au{Subbotin M.\,T.} On the law of frequency
of error~// Математический сборник, 1923. Т.~31. Вып.~2.
С.~296--301.

\bibitem{Evans2000} 
\Au{Evans M., Hastings N., Peacock~B.} Statistical
distributions.~--- 3rd ed.~--- New York: Wiley, 2000.

\bibitem{BoxTiao1973} \Au{Box G., Tiao~G.} Bayesian inference in statistical analysis.~--- 
Reading: Addison--Wesley, 1973.

\bibitem{Morgan1996} 
RiskMetrics Technical Document.~---
New York: RiskMetric Group, J.\,P.~Morgan, 1996.

\bibitem{Varanasi1989} 
\Au{Varanasi M.\,K., Aazhang~B.} Parametric
generalized Gaussian density estimation~// J.~Acoustic
Soc. Amer., 1989. Vol.~86. No.\,4. P.~1404--1415.

\bibitem{Nadaraja2005}  %6
\Au{Nadaraja S.} A~generalized normal
distribution~// J.~Appl. Stat., 2005. Vol.~32. No.\,7. P.~685--694.

\bibitem{West1987}
\Au{West M.}
On scale mixtures of normal distributions~// Biometrika, 1987. Vol.~74. No.\,3. P.~646--648.

\bibitem{ChoySmith1997} \Au{Choy S.\,T.\,B., Smith~A.\,F.\,F.} Hierarchical
models with scale mixtures of normal distributions~// TEST, 1997.
Vol.~6. P.~205--221.

\bibitem{Zolotarev1983} \Au{Золотарев В.\,М.} Одномерные устойчивые
распределения.~--- М.: Наука, 1983.

\bibitem{KorolevBeningShorgin2011-k} 
\Au{Королев В.\,Ю., Бенинг~В.\,Е., Шоргин~С.\,Я.}
Математические основы теории риска.~--- 2-е изд., перераб. и доп.~--- М.:
Физматлит, 2011. 620~с.

\bibitem{GnedenkoKolmogorov1949} 
\Au{Гнеденко Б.\,В., Колмогоpов А.\,Н.} Пpедельные
pаспpеделения для сумм независимых случайных величин.~--- М.--Л.: ГИТТЛ, 1949.

\bibitem{GnedenkoKorolev1996-k} 
\Au{Gnedenko B.\,V., Korolev V.\,Yu.} Random summation:
Limit theorems and applications.~--- Boca Raton: CRC Press, 1996.

\bibitem{Korolev1994} 
\Au{Королев В.\,Ю.} Сходимость случайных последовательностей с
независимыми случайными индексами. I~// Теория вероятностей и ее
применения, 1994. Т.~39. Вып.~2. С.~313--333.

\bibitem{Korolev1995} 
\Au{Королев В.\,Ю.} Сходимость случайных последовательностей с независимыми
случайными индексами. II~// Теория вероятностей и ее применения,
1995. Т.~40. Вып.~4. С.~907--910.

\bibitem{Korolev1996} 
\Au{Korolev V.\,Yu.} A~general
theorem on the limit behavior of superpositions of independent
random processes with applications to Cox processes~// J.~Math. Sci., 
1996. Vol.~81. No.\,5. P.~2951--2956.

\bibitem{BeningKorolev2002-k} 
\Au{Bening V., Korolev~V.} Generalized Poisson models and their applications in
insurance and finance.~--- Utrecht: VSP, 2002. 434~p.

\bibitem{KorolevSokolov2008-k} 
\Au{Королев В.\,Ю., Соколов И.\,А.} Математические модели
неоднородных потоков экстремальных событий.~--- М.: ТОРУС-ПРЕСС, 2008.

\bibitem{Korolev2011-k} 
\Au{Королев В.\,Ю.} Ве\-ро\-ят\-но\-ст\-но-ста\-ти\-сти\-че\-ские методы
декомпозиции волатильности хаотических процессов.~--- М.: Изд-во
Моск. ун-та, 2011. 510~с.

\label{end\stat}

\bibitem{KorolevSkvortsova2006} 
Stochastic models of structural plasma turbulence~/
Eds.\  V.~Korolev, N.~Skvortsova.~--- Utrecht: VSP, 2006. 400~p.




\end{thebibliography}
}
}


\end{multicols}