\def\stat{nazarov}

\def\tit{НИЖНИЕ ОЦЕНКИ УСТОЙЧИВОСТИ СМЕСЕЙ НОРМАЛЬНЫХ РАСПРЕДЕЛЕНИЙ К ВОЗМУЩЕНИЯМ
СМЕШИВАЮЩИХ РАСПРЕДЕЛЕНИЙ$^*$}

\def\titkol{Нижние оценки устойчивости смесей нормальных распределений к возмущениям
смешивающих распределений}

\def\autkol{А.\,Л.~Назаров}

\def\aut{А.\,Л.~Назаров$^1$}

\titel{\tit}{\aut}{\autkol}{\titkol}

{\renewcommand{\thefootnote}{\fnsymbol{footnote}}\footnotetext[1]
{Работа поддержана Российским фондом фундаментальных
исследований (проекты 11-01-00515а, 11-07-00112а,
11-01-12026-офи-м, 12-07-00115а), Министерством образования и науки (госконтракт
16.740.11.0133).}}

\renewcommand{\thefootnote}{\arabic{footnote}}
\footnotetext[1]{Факультет вычислительной
математики и кибернетики Московского государственного университета
им.\ М.\,В.~Ломоносова, nazarov.vmik@gmail.com}



\Abst{Исследуется устойчивость смесей нормальных
распределений к возмущениям смешивающих распределений.
Рассматриваются неравенства, описывающие близость смешивающих
распределений через близость соответствующих смесей. Доказана
теорема существования оценок устойчивости для подклассов масштабных
и сдвиговых смесей нормальных законов. Оценка для сдвиговых смесей
выписана в явном виде. При этом приведен пример, показывающий, что
полученный результат не может быть принципиально улучшен без
дополнительных предположений.}

\KW{смеси нормальных распределений;  устойчивость
стохастических моделей; преобразование Фурье; теорема Планшереля;
теорема Прохорова; мет\-ри\-ка Леви; нижние оценки устойчивости смесей}

\vskip 14pt plus 9pt minus 6pt

      \thispagestyle{headings}

      \begin{multicols}{2}

            \label{st\stat}

\section{Введение}

Изучение проблем устойчивости стохастических моделей представляет
большой интерес для специалистов, занимающихся теорией вероятностей
и математической статистикой. Данные задачи важны  как для
исследователей, работающих над прикладными задачами, так и для тех,
кто занимается теоретическими вопросами. Зачастую исследование
устойчивости необходимо для обоснования применимости того или иного
метода к анализу реальных данных.

В качестве примера можно привести задачу обоснования применимости
сеточных методов разделения смесей вероятностных распределений~[1--4], 
использующихся для оценки смешивающего распределения по
реализациям смеси. При работе этих алгоритмов оценка смешивающего
распределения ищется в классе  распределений, сосредоточенных в
конечном множестве фиксированных точек, покрывающих область,
содержащую носитель истинного распределения. Для доказательства
допустимости такого подхода используются оценки устойчивости смесей
вероятностных распределений  к возмущениям смешивающих
распределений.

В статье рассматриваются вопросы устойчивости смесей нормальных
распределений. Смеси нормальных распределений используются для
изучения и моделирования многих явлений реального мира. Доказаны
предельные теоремы, описывающие сходимость одномерных распределений
обобщенных процессов Кокса к сдвиг-мас\-штаб\-ным смесям нормальных
законов~\cite{KorolevChaos}. Примеры случайных процессов, имеющих в
качестве одномерных распределений смеси нормальных распределений,
можно найти в~\cite{KorolevChaos} и работах из библиографии к ней.

Оценкам устойчивости различных классов смесей нормальных
распределений к возмущениям смешивающих распределений был посвящен
ряд работ. Результат для частного случая~--- простой модели
загрязнения (контаминации), предложенной Тьюки~\cite{Tukey_1960}, 
был получен в \cite{KorNazGridEng}. 
В~\cite{Hall1979357} можно найти решение этой задачи для оценки
расстояния между нормальным распределением и масштабной\linebreak \mbox{смесью} при
выполнении некоторых условий, наложенных на смешивающее
распределение. Рабо-\linebreak та~\cite{NazStab} посвящена верхним оценкам
\mbox{устойчивости} нормальных смесей. Выписаны верхние \mbox{оценки} близости
сдвиг-мас\-штаб\-ных смесей через близость смешивающих распределений в
метрике Ле\-ви--Про\-хо\-ро\-ва. Кроме этого рассмотрены аналогичные оценки
в мет\-ри\-ке Леви для масштабных и сдвиговых смесей.

Целью данного исследования стала попытка установить, как близость
двух смесей вероятностных распределений влияет на близость
со\-от\-вет\-ст\-вующих им смешивающих распределений в некоторой
вероятностной метрике. Для этого \mbox{рассматривались} нижние оценки
близости смесей через расстояние между смешивающими распределениями.
Полученные результаты используются в~\cite{NazConvUnpubl} для
исследования асимптотических свойств оценок, полученных с помощью
сеточных методов разделения смесей.

Не ограничивая общности, если это специально не оговорено, будем
считать, что все случайные величины и элементы, рассматриваемые в
данной статье, определены на одном вероятностном пространстве
$(\Omega,\mathcal{A},\mathrm{P})$.

\vspace*{-6pt}

\section{Используемые метрики}

Для описания близости  распределений в данной статье используется
несколько   метрик.  Условимся здесь и далее обозначать  метрику
Леви через~$L$. Напомним определение метрики Леви.

Пусть $X$ и $Y$~--- две случайные величины с функциями распределения
$F$ и $G$ соответственно. Расстояние (метрика) Леви $L(X,Y)$ между
случайными величинами~$X$ и~$Y$ определяется следующим образом:

\noindent
\begin{multline*}
L(X, Y) = \inf \{ \varepsilon > 0 | F(x - \varepsilon) - 
\varepsilon \leq G(x) \leq{}\\
{}\leq F(x + \varepsilon) + \varepsilon\,, \ \forall x \in \mathbb{R} \}\,.
\end{multline*}
Расстояние Леви допускает наглядную геометрическую интерпретацию:
оно равно длине стороны наибольшего квадрата, который можно
вписать между графиками функций распределения~$F$ и~$G$ так,
чтобы его стороны были параллельны координатным осям. Так как в
определении метрики Леви задействованы не сами случайные величины~$X$ 
и~$Y$, а только их функции распределения~$F$ и~$G$, то
выражения $L(X,Y)$ и $L(F,G)$ не будут различаться:
$$
L(X,Y)\equiv L(F,G)\,.
$$

Метрику  в $L^p$  будем обозначать через~$\|\cdot \|_p$:
$$
\|g \|_{p} = \left(\,
\int\limits_{-\infty}^{\infty} |g(s)|^p\, ds
\right)^{1/p}\,.
$$
Для оценки близости распределений смесей будет использоваться
расстояние между со\-от\-вет\-ст\-ву\-ющи\-ми функциями распределения в метрике~$L^2$. 
Данное расстояние метризует слабую сходимость для некоторого
подкласса рассматриваемых смесей.
{\looseness=-2

}

Чтобы расстояние между функциями распределения в $L^2$ было конечно,
достаточно потребовать от соответствующих случайных величин
существования конечного первого момента. Действительно, пусть $X_1$,
$X_2$~--- случайные величины, имеющие конечный первый момент, $F_1$,
$F_2$~--- соответствующие функции распределения.

\noindent
\begin{multline*}
\|F_1 - F_2 \|^2_{2} =
\int\limits_{-\infty}^{\infty} \left\vert F_1(s) - F_2(s)\right\vert^2\, ds 
\le {}\\
{}\le \int\limits_{-\infty}^{\infty} \left\vert F_1(s) - F_2(s)\right\vert\, ds \le 
\int\limits_{-\infty}^{0} (F_1(s) + F_2(s)) \, ds +{}\\
{}+
\int\limits_{0}^{\infty}( 1- F_1(s) + 1 - F_2(s)  )\,ds  = 
{\sf E}|X_1| + {\sf E}|X_2| < \infty\,.
\end{multline*}

Очевидно, что из сходимости функций распределений в $L^2$ напрямую
следует сходимость в метрике Леви, которая, в свою очередь,
метризует слабую сходимость. Однако легко видеть, что обратное
утверждение неверно даже для класса распределений, имеющих конечный
первый момент. Наложим на рассматриваемые функции распределения
дополнительные условия. Справедливо следующее утверждение.

\smallskip

\noindent
\textbf{Лемма~1.}
\textit{Рассмотрим класс $K$ случайных величин, имеющих конечный первый
момент. Пусть $\{F_n\}_{n=1}^\infty$~--- последовательность функций
распределения случайных величин из~$K$, слабо сходящихся к~$F$. Если
семейство~$K$ является равномерно интегрируемым, то
последовательность $\{F_n\}_{n=1}^\infty$ сходится к~$F$ в метрике~$L^2$.}

\smallskip

\noindent
Д\,о\,к\,а\,з\,а\,т\,е\,л\,ь\,с\,т\,в\,о\,.\ Воспользовавшись 
методом одного (единого) вероятностного
пространства~\cite{Skorohod1956}  можно показать, что для функций
распределения $\{F_n\}_{n=1}^\infty$  и $F$ существует вероятностное
пространство и последовательность случайных величин
$\{X_n\}_{n=1}^\infty$, заданных на нем, которая сходится почти
наверное к~$X$, заданной на нем же. Далее, воспользовавшись тем, что
$\{X_n\}_{n=1}^\infty$ равномерно интегрируема и сходится почти
всюду к~$X$, получаем (см., например,~\cite{shirVer}), что
$\{X_n\}_{n=1}^\infty$ сходится к~$X$ в среднем.

Заметим, что доказательство существования последовательности
$\{X_n\}_{n=1}^\infty$ в~\cite{shirVer} конструктивно.
Представленная последовательность  обладает тем свойством, что для
любых двух ее элементов~$Y$ и~$Z$, имеющих функции распределения~$G$
и~$H$, случайные величины $\max\{Y,Z\}$ и $\min\{Y,Z\}$ имеют
функции распределения $\min\{G,H\}$ и $\max\{G,H\}$.

Зафиксируем некоторый номер $n\hm\in\mathbb{N}$. 
Справедлива следующая цепочка неравенств:
\begin{multline*}
0\le\left\vert F_n - F  \right\vert^2_{2} =
\int\limits_{-\infty}^{\infty} \left\vert F_n(s) - F (s)\right\vert^2\, ds \le {}\\
{}\le
\int\limits_{-\infty}^{\infty} \left\vert F_n(s) - F (s)\right\vert\, ds ={}\hspace*{10mm}
\end{multline*}

\noindent
\begin{multline*}
{} = \!\int\limits_{-\infty}^{\infty}
 (\max\{F_n(s),F (s)\} - \min\{F_n(s),F (s)\})\, ds  ={}
\\{} =
 - \!\int\limits_{-\infty}^{0}\left(\min\{F_n(s),F (s)\} - \max\{F_n(s),F (s)\}  \right) ds
 +{}\\
{}+\int\limits_{0}^{\infty} \left( (1 - \min\{F_n(s),F (s)\})  -{}\right.\\
\left.{}- ( 1 -
\max\{F_n(s),F (s)\}  ) \right) ds ={}
\\
{}= {\sf E}( \max\{X_n,X\} -\min\{X_n,X\} ) = {\sf E} \left\vert X_n - X\right\vert\,.
\end{multline*}
Учитывая, что последовательность $\{X_n\}_{n=1}^\infty$ сходится в
среднем, получаем утверждение леммы. \hfill~$\square$


\medskip

Таким образом, $\|\cdot \|_2$ метризует слабую сходимость для
равномерно интегрируемых последовательностей случайных величин.

Наиболее интересным для исследования рассматриваемой задачи является
класс сдвиг-мас\-штаб\-ных смесей нормальных законов. В~общем случае
искомых оценок для данного класса не существует. Это связанно с тем,
что без дополнительных ограничений класс сдвиг-мас\-штаб\-ных смесей
нормальных законов не является идентифицируемым~\cite{KorolevChaos}.
Можно привести  пример двух смесей, совпадающих по распределению, но
имеющих различные смешивающие распределения.  Поэтому в работе
рассматриваются нижние оценки для идентифицируемых классов~---
сдвиговых и масштабных смесей нормальных законов.

Выпишем интересующие оценки для класса сдвиговых смесей нормальных
законов, т.\,е.\ для сверток некоторых смешивающих распределений со
стандартным нормальным законом.

\section{Оценки устойчивости для~сдвиговых смесей нормальных распределений}

Рассмотрим класс сдвиговых смесей нормальных законов, имеющих
конечный первый момент. Оценим, насколько близость распределений
случайных величин
$$
Y_1= X+U_1 \ \mbox{и} \  Y_2= X+U_2\,, \ \ \  X\sim\mathcal{N}(0,1)
$$
(случайные величины $X$ и~$U_i$ стохастически независимы для
$i\hm=1,2$), очевидно, являющихся сдвиговыми смесями нормальных
законов, влияет на близость распределений~$U_1$ и~$U_2$.

Докажем сначала вспомогательное утверждение.

\medskip

\noindent
\textbf{Лемма~2.}
\textit{Пусть  $F_1$ и  $F_2$~--- произвольные функции распределения, $G$~---
функция симметричного распределения. Существуют функции
распределения  $\tilde F_1$ и~ $\tilde F_2$, такие что $f(x) \hm=
\tilde F_1(x) \hm- \tilde F_2(x)$, $x \hm\in \mathbb{R}$,  принимает
одинаковые значения в парах точек непрерывности, симметричных
относительно нуля, и}
\begin{multline*}
L(F_1,F_2) \le L(\tilde F_1,\tilde F_2)\,,  \enskip
\|
F_1*G - F_2*G \|_{p} \ge{}\\
{}\ge \|
\tilde F_1*G - \tilde F_2*G
\|_{p}\,,  \enskip p\ge1\,.
\end{multline*}

%\medskip


\noindent
Д\,о\,к\,а\,з\,а\,т\,е\,л\,ь\,с\,т\,в\,о\,.\ 
Ниже будем использовать следующее обозначение. Если некоторая
случайная величина~$Z$ имеет функцию распределения~$H$, то
распределение случайной величины~$-Z$ будем обозначать через
$H'(x)\hm\equiv(1 - H)(-x)$, $x \hm\in \mathbb{R}$.

Пусть $L(F_1,F_2) = \delta$. Так как рассматриваемые расстояния
инвариантны относительно сдвига, без ограничения общности можно считать, что
$$
\sup \left\{ \varepsilon > 0 |
F_1\left(- \fr{\varepsilon}{2}  \right) - \varepsilon \geq F_2\left(
\fr{\varepsilon}{2} \right)  \right\} = \delta\,.
$$
Положим 
$\tilde F_1(x) \hm= (1/2)(F_1(x)  \hm+ (1 - F_2)(-x))$, 
$\tilde F_2(x) \hm= (1/2)(F_2(x) \hm + (1 - F_1)(-x))$, 
$x \hm\in \mathbb{R}$.

Очевидно,
\begin{multline*}
\sup \left\{ \varepsilon > 0 |
(1 - F_2)\left( \fr{\varepsilon}{2}  \right) - \varepsilon \geq {}\right.\\
\left.{}\geq(1 - F_1)\left(
-\fr{\varepsilon}{2} \right)  \right\} =
 \delta\,.
\end{multline*}
Следовательно, $L(\tilde F_1,\tilde F_2) \hm\ge \delta$. Далее,
воспользовавшись неравенством треугольника и линейностью оператора
свертки, получаем
\begin{multline}
\|
\tilde F_1*G - \tilde F_2*G
\|_{p}  ={}\\
{}=
 \fr{1}{2}\,
\|
 F_1*G  + (1-F_2)*G  -  F_2 * G - (1-F_1) * G
\|_{p}\le{}\\
{}\le
\fr{1}{2}\|
 F_1*G    -  F_2 * G
\|_{p}+{}\\
{}+\fr{1}{2}\|
 (1-F_2) * G  - (1-F_1)*G
\|_{p}\,.
\label{symmetrLemm}
\end{multline}
Учитывая, что $G$~--- функция симметричного распределения,

\noindent
\begin{multline*}
 (1-F_i) * G (x)  =
\int\limits_{-\infty}^{\infty} (1-F_i)(-x + y ) \,dG(y) =
{}\\
{}=
1-\int\limits_{-\infty}^{\infty}  F_i(-x - y )  d G(y) ={}\\
{}= (1-F_i  * G) ( - x)\,,\ \
  \forall x \in \mathbb{R},
\end{multline*}
для $i=1,2$. Следовательно:
$$\|
 (1-F_2) * G  - (1-F_1)*G
\|_{p} = \|
  F_1  * G  - F_2 * G
\|_{p}\,.
$$

\pagebreak

\noindent
Подставляя равенство в~\eqref{symmetrLemm}, получаем утверждение леммы. \hfill$\square$

Воспользовавшись результатом леммы~2, получим интересующие оценки
для класса сдвиговых смесей.  

\medskip

\noindent
\textbf{Теорема~1.}
\textit{Пусть $X$, $U_1$, $U_2$~--- случайные величины$;$ $\Phi$, $F_1$,
$F_2$~--- соответствующие функции распределения, пары случайных
величин $X$, $U_1$ и $X$, $U_2$  независимы,
$X \hm\sim \mathcal{N}(0,1)$, у случайных величин $U_1$ и $U_2$
существуют конечные первые моменты. Если $L(F_1,F_2) \hm\ge \delta$,~то}
$$
\|  F_1  * \Phi  - F_2 * \Phi \|_{2}\ge
\sqrt{\fr{ \delta^{3}}{2}}
\exp\left\{-\frac{2}{\pi^2\delta^6}\right\}
\,.
$$

\noindent
Д\,о\,к\,а\,з\,а\,т\,е\,л\,ь\,с\,т\,в\,о\,.\ 
Пусть $L(U_1,U_2) \ge \delta$, тогда $\|  F_1  \hm-F_2
\|_{2}^2\ge\delta^3$. Учитывая результат леммы~2,
без ограничения общности будем считать, что $f( x )\hm= F_1 (x) \hm- F_2
(x)$, $x \hm\in \mathbb{R}$, принимает одинаковые значения в парах
точек непрерывности, симметричных относительно нуля. Так как у
случайных величин~$U_1$ и~$U_2$ существуют конечные первые моменты,
легко показать, что $f(x) \hm\in L_1(-\infty, \infty)$. Следовательно,
существует преобразование Фурье:
$$
g(t) = \int\limits_{-\infty}^{\infty}
f(x) e^{-i t x}\,dx\,,  \enskip t \in \mathbb{R}\,.
$$
Воспользовавшись симметричностью почти всюду и ограниченностью~s$f$,
легко показать, что $|g(t)|\le(2/|t|)$, $t \hm\in \mathbb{R}$.

Из теоремы Планшереля следует, что $ \|g \|^2_{2}\hm\ge 2\pi
\delta^3$. Возьмем $A \hm= {2}/({\pi\delta^3})$. Получим:
$$ 
\int\limits_{-A}^{A}
|g(t)|^2\, dt
\ge
\|g \|^2_{2} -
2 \int\limits_{A}^{\infty}\fr{1}{t^2}\,dt
 =  2\pi \delta^3 -   \pi \delta^3 =   \pi \delta^3\,.
 $$
Далее,
\begin{multline*}
\|  F_1  * \Phi  - F_2 * \Phi \|_{2}^2 =
 \|  f * \Phi    \|_{2}^2 =
\fr{1}{2 \pi} \| g \cdot \varphi    \|_{2}^2 \ge
{}\\
{}
\ge\fr{1}{2 \pi}
\int\limits_{-A}^{A}|g(t)|^2 e^{-t^2}\,dt
\ge {}\\
{}\ge\fr{e^{-A^2}}{2 \pi}
\int\limits_{-A}^{A}
|g(t)|^2  dt \ge \fr{ \delta^3e^{-A^2}}{2}\,.
\end{multline*}
Здесь $\varphi$~--- характеристическая функция стандартного
нормального закона. Подставляя значение $A$ в последнее соотношение,
получаем утверждение теоремы. \hfill~$\square$

Заметим, что требование существования конечных первых моментов
является достаточным, но не необходимым.

Оценка, полученная в теореме~1, на первый взгляд
может показаться достаточно грубой.  Ниже построен пример двух
смесей, показывающий, что без дополнительных условий полученная
оценка не может быть принципиально улучшена. Справедливо следующее
утверждение.

\medskip

\noindent
\textbf{Утверждение~1.}
\textit{Пусть $X\sim\mathcal{N}(0,1)$, $\delta \hm\in (0,({4 - \pi})/({8 + \pi}))$.
Существуют случайные величины $U_1$, $U_2$ такие, что пары случайных
величин $X$, $U_1$ и $X$, $U_2$ независимы, расстояние Леви $L(U_1,
U_2)\hm\ge\delta$,~а}
$$
L(X+U_1, X+U_2)\le
\left(\fr{21 }{(4-\pi)^2}\right)^{1/3}
\delta \exp\left(- {\fr{\pi^2}{3\delta^2}}\right)
\,.
$$

\noindent
Д\,о\,к\,а\,з\,а\,т\,е\,л\,ь\,с\,т\,в\,о\,.\ 
Рассмотрим вероятностное пространство
$(\Omega,\mathcal{F},\mathbb{P})$, где $\Omega$~--- объединение
счетного числа непересекающихся множеств $B_0$, $B_1$, $B_{-1}$,
$A_n$, $n\hm\in\mathbb{Z}$:
\begin{gather*}
\mathcal{F} = \sigma(\{B_0, B_1, B_{-1}, A_n, n\in\mathbb{Z}\})\,;
\\
\mathbb{P}(B_{-1}) = \mathbb{P}(B_{ 1}) = \fr{ \pi \delta}{4-\pi}\,; \enskip
\mathbb{P}(B_{0}) = 1- \delta \frac{8 + \pi}{4 - \pi} \,;
\\
\mathbb{P}(A_{0}) =  \delta\,;\\
\mathbb{P}(A_{n}) = \fr{4 \delta
}{4-\pi} \,\fr{1}{4n^2 - 1 }\,,  \ n = 1, -1,  2, -2, \ldots
\end{gather*}
Пусть на указанном вероятностном пространстве заданы случайные
величины $U_1$ и $U_2$, имеющие функции распределения $F_1$ и $F_2$
соответственно:
$$
U_1(\omega)=
\begin{cases}
    \left(-n+\fr{(-1)^{n+1}}{2}\right)\delta\,,     & \omega\in A_{-n}, n\in\mathbb{N}\,;\\
    -\delta\,,            & \omega\in B_{-1}\,;\\
    -\fr{\delta}{2}\,,  & \omega\in B_0\,;\\
    \fr{\delta}{2}\,,  & \omega\in A_0\,;\\
    \fr{\delta}{2}\,,   & \omega\in B_1\,;\\
    \left(n+\fr{(-1)^{n+1}}{2}\right)\delta\,,      & \omega\in A_n, n\in\mathbb{N}\,;
\end{cases}
$$
$$
U_2(\omega)=
\begin{cases}
    \left(-n+\fr{(-1)^{n}}{2}\right)\delta\,,     & \omega\in A_{-n}, n\in\mathbb{N}\,;\\
    -\fr{\delta}{2}\,,            & \omega\in B_{-1}\,;\\
    -\fr{\delta}{2}\,,  & \omega\in B_0\,;\\
    -\fr{\delta}{2}\,,  & \omega\in A_0\,;\\
    \delta\,,   & \omega\in B_1\,;\\
    \left(n+\fr{(-1)^{n}}{2}\right)\delta\,,      & \omega\in A_n, n\in\mathbb{N}\,.
\end{cases}
$$
Разность функций распределения случайных величин $U_1$ и  $U_2$ имеет вид:
\begin{multline*}
F_1(x) - F_2(x)={}\\
\hspace*{-1mm}{}=
\begin{cases}
    \fr{4 \delta }{4-\pi} \,\fr{(-1)^n}{4n^2 - 1 }\,,
    &\hspace*{-5mm} x\in \left[\delta\left(n-\fr{1}{2}\right),\delta\left(n+\fr{1}{2}\right)\!\right),\hspace*{-5.40523pt}\\[9pt]
    &\hspace*{27mm}  n \le -2 ; \\[9pt]
    - \fr{4 \delta }{4-\pi} \,\fr{1}{3}\,,
    & x\in \left[ -\fr{3\delta}{2},-\delta\right)\,; \\[9pt]
    \fr{ \pi \delta}{4-\pi} - \fr{4 \delta }{4-\pi} \,\fr{1}{3}\,,
    & x\in \left[  -\delta, -\fr{ \delta}{2}\right)\,; \\[9pt]
    -\delta\,,
    & x\in \left[-\fr{ \delta}{2}, \fr{\delta}{2}\right)\,; \\[9pt]
    \fr{ \pi \delta}{4-\pi} - \fr{4 \delta }{4-\pi} \,\fr{1}{3},
    & x\in \left[ \fr{ \delta}{2}, \delta\right)\,; \\[9pt]
    - \fr{4 \delta }{4-\pi} \,\fr{1}{3}\,,
    & x\in \left[\delta, \fr{3\delta}{2}\right)\,; \\[9pt]
    \fr{4 \delta }{4-\pi} \,\fr{(-1)^n}{4n^2 - 1 }\,,
    & \hspace*{-5mm}x\in \left[\delta\left(n-\fr{1}{2}\right),\delta\left(n+\fr{1}{2}\right)\!\right),\hspace*{-5.40523pt}\\[9pt]
    &\hspace*{29mm}  n \ge 2\,. 
\end{cases}
\end{multline*}
Заметим, что расстояние Леви между функциями распределения равно
$\delta$.

Выпишем преобразование Фурье разности функций распределения.
Воспользовавшись тем, что для любого $a\hm>0$
\begin{gather*}
\int\limits_{-a}^{a}  e^{-i t x}\,dx = \fr{2\sin a t}{t}\,; \enskip
\left|\fr{ \sin a t}{t}\right|\le a\,;
\\
\lim\limits_{n \to \infty}\int\limits_{-\infty}^{\infty}
\left|
\sum\limits_{k=n}^{\infty} \fr{(-1)^k}{4k^2 - 1 }
\left( I_{[\delta(-k-{1}/{2}),\delta(-k+{1}/{2}))}(x) +{}\right.\right.\\
\left.\left.{}+I_{[\delta(k-1/2),\delta(k+{1}/{2})) }(x)
\right) \vphantom{\fr{(-1)^k}{k^2}}
\right|dx
\le{}\\
{}\le
 \lim\limits_{n \to \infty} \sum\limits_{k=n}^{\infty} \fr{2\delta}{4k^2 - 1 }  =  0 \,;
\\
\lim\limits_{n \to \infty}\sup\limits_t
\left|\fr{2}{t}\sum\limits_{k=n}^{\infty} \fr{(-1)^k}{4k^2 - 1 }
\left(\sin\left(k+\fr{1}{2}\right)\delta t  -{}\right.\right.\\
\left.\left.{}- \sin\left(k - \fr{1}{2}\right)\delta t\right)
\right|  = {}
\\
{}=\lim\limits_{n \to \infty}\sup\limits_t
\left|\fr{4}{t}\sum\limits_{k=n}^{\infty} \fr{(-1)^k}{4k^2 - 1 }
\left(\sin \fr{\delta}{2}  t  \cos k \delta t\right)
\right| \le{}\\
{}\le
2\delta
\lim\limits_{n \to \infty} \sum\limits_{k=n}^{\infty} \fr{ 1 }{4k^2 - 1 }
    =  0\,,
\end{gather*}
и тем, что из сходимости последовательности функций в
$L_1(-\infty,\infty)$ следует равномерная сходимость
последовательности образов Фурье этих функций~\cite{KolmFomin}, получаем:
\begin{multline*}
f(t) = \int\limits_{-\infty}^{\infty}
\left(F_1(x) - F_2(x)\right) e^{-i t x}\,dx ={}
\\
{}= \fr{2}{t}\left(\!
-\delta\sin \fr{\delta}{2} t +
\fr{\pi \delta }{4-\pi}\left(\!\sin \delta t - \sin \fr{\delta}{2}t\right)+
\fr{4 \delta }{4-\pi} \times{} \right.\\
\!\left.{}\times
\sum\limits_{n=1}^{\infty} \fr{(-1)^n}{4n^2 - 1 }
\left(\sin\left(\!n+\fr{1}{2}\right)\delta t  - \sin\left(\!n - \fr{1}{2}\right)\delta t\right)\!
\right).\hspace*{-8.84514pt}
\end{multline*}
Далее,
\begin{multline*}
f(t)  = \fr{2\delta \sin ({\delta}/{2}) t}{t}\left(- 1
 +
\fr{\pi  }{4-\pi}
\left(2 \cos \fr{\delta}{2} \,t - 1\right) -{}\right.\hspace*{-3.037pt}\\
\left.{}-
\fr{8  }{4-\pi}
\sum\limits_{n=1}^{\infty} \fr{\cos \pi n}{1 - 4n^2 }
 \cos n \delta t
\right)={}
\\{}
= \fr{2\delta \sin ({\delta}/{2}) t}{t}\left(
\fr{\pi  }{4-\pi}\,
 2 \cos \fr{\delta}{2}\, t -  \fr{4  }{4-\pi} -{}\right.\\
\left. {}-
\fr{8  }{4-\pi}
\sum\limits_{n=1}^{\infty} \fr{\cos \pi n}{1 - 4n^2 }
\, \cos n \delta t
\right) ={}
\\{}
=\fr{4 \pi  }{4-\pi}\, \fr{\delta \sin ({\delta}/{2}) t}{t}\left(
  \cos \fr{\delta}{2} \,t -  \fr{2  }{ \pi} -{}\right.\\
\left.  {}-
\fr{4  }{ \pi}
\sum\limits_{n=1}^{\infty} \fr{\cos \pi n}{1 - 4n^2 }\,
 \cos n \delta t
\right)\,.
\end{multline*}

Разложим функцию $g(t) \hm=   \cos ({\delta}/{2}) t$ в ряд Фурье на
отрезке $[0, {\pi}/{\delta}]$ по системе функций
$$
\left\{\sqrt{\fr{\delta}{\pi}}, \sqrt{\fr{2\delta}{\pi}}\, \cos n
\delta t \,,\enskip n=1,2,3, \ldots \right\}\,.
$$
Учитывая, что
$$
\int\limits_0^{{\pi}/{\delta}}  \cos\left(\fr{\delta}{2}\, x\right)  \cos(  n \delta x) \, dx =
\fr{2}{\delta}\,\fr{\cos \pi n}{1 - 4n^2 }\,,\enskip   n\ge0\,,
$$
получаем
$$
\cos \fr{\delta}{2}\, t =   \fr{2  }{ \pi} +
\fr{4  }{ \pi}
\sum\limits_{n=1}^{\infty} \fr{\cos \pi n}{1 - 4n^2 }
\, \cos n \delta t\,, \enskip t \in [0, \fr{\pi}{\delta}]\,.
$$
Так как умножение на ограниченную функцию сохраняет сходимость в
$L_2[0, {\pi}/{\delta}]$,
\begin{multline*}
 \int_0^{{\pi}/{\delta}} (f(t))^2\,dt =
\int\limits_0^{{\pi}/{\delta}} \left(\fr{4 \pi  }{4-\pi}\,
 \fr{\delta \sin ({\delta}/{2}) t}{t}\right)^2\times{}\\
 {}\times
\left(
  \cos \fr{\delta}{2}\, t -  \fr{2  }{ \pi} -
\fr{4  }{ \pi}
\sum\limits_{n=1}^{\infty} \fr{\cos \pi n}{1 - 4n^2 }
\, \cos n \delta t
\right)^2\,dt =0\,.
\end{multline*}
Перейдем теперь к оценке  расстояния между функциями распределения
случайных величин $X\hm+U_1$,  $X\hm+U_2$.  Заметим, что
\begin{multline*}
\|F_1(x) - F_2(x) \|_2^2 = 2\sum\limits_{n=2}^\infty 
\fr{16 \delta^3 }{(4-\pi)^2} \,\fr{1}{(4n^2 - 1 )^2} +{}\\
{}+
\fr{16 \delta^3 }{(4-\pi)^2} \,\fr{1}{9} +
\left(\fr{ \pi \delta}{4-\pi} - \fr{4 \delta }{4-\pi}
\,\fr{1}{3}\right)^2 \delta + \delta^3\le{}
\\
{}\le \sum\limits_{n=1}^\infty
\fr{16 \delta^3 }{(4-\pi)^2} \,
\fr{2}{(4n^2 - 1 )}+
\left(\fr{ 2 }{4-\pi} \right)^2 \delta^3  + \delta^3 \le{}\\
{}\le
\fr{21 \delta^3 }{(4-\pi)^2}\,.
\end{multline*}
Воспользовавшись теоремой Планшереля, оценим расстояние между
функциями распределения смесей
\begin{multline*}
\| \Phi\ast F_1(x) - \Phi\ast F_2(x) \|_2^2 = \fr{1}{2\pi}
\left\| f(t) e^{-{t^2}/{2}} \right\|_2^2 =
{}\\{}
= 2 \fr{1}{2\pi} \int\limits_0^{{\pi}/{\delta}}
(f(t))^2 e^{- {t^2} }\,dt + 2
\fr{1}{2\pi} \int\limits_{{\pi}/{\delta}}^\infty (f(t))^2 e^{- {t^2} }\,dt   \le
{}\\
{}\le 2
\fr{e^{- \pi^2/\delta^2 }}{2\pi}
\int\limits_{{\pi}/{\delta}}^\infty (f(t))^2 \,dt
\le
\fr{e^{- \pi^2/\delta^2 }}{2\pi}
\left\| f(t) \right\|_2^2 \le{}\\
{}\le
e^{- \pi^2/\delta^2}
\fr{21 \delta^3 }{(4-\pi)^2}\,.
\end{multline*}
Далее получаем
\begin{multline*}
L(\Phi\ast F_1, \Phi\ast F_2)
\le
\| \Phi\ast F_1(x) - \Phi\ast F_2(x) \|_2^{{2}/{3}}
\le{}\\
{}\le
\left(\fr{21 }{(4-\pi)^2}\right)^{1/3}
\delta e^{- \pi^2/(3\delta^2)}\,.
\end{multline*}
Заметим, что 
$$
~~~~~~~~~~~~~~~~~~~~\left(\fr{21 }{(4-\pi)^2}\right)^{1/3}\approx 3{,}0545\,.~~~~~~~~~~~~~~~~~~~~\square
$$



Таким образом, вид оценок, полученных выше, не может быть
принципиально улучшен без добавления ограничений на смешивающие
распределения.

\section{Оценки устойчивости для~масштабных смесей нормальных распределений}

Перейдем к исследованию устойчивости масштабных смесей нормальных
законов. Верхние оценки устойчивости, описывающие близость смесей
через близость соответствующих им смешивающих распределений,
получены в работе~\cite{NazStab} с использованием метрики Леви. 
К~сожалению, аналогичные результаты для нижних оценок устойчивости не
могут быть получены без дополнительных ограничений на смешивающее
распределение.

\bigskip

\noindent
\textbf{Лемма~3.} \textit{Пусть $X$,  $V_1$, $V_2$~--- случайные величины, удовлетворяющие
условиям $V_1 > 0$, $V_2 > 0$, пары случайных величин  $X$,  $V_1$ и
$X$,  $V_2$ независимы, случайная величина $X$ имеет стандартное
нормальное распределение. Справедливы следующие утверждения.}
\begin{enumerate}
  \item \textit{Для любых $\varepsilon\in(0,1)$, $\delta>0$ существуют $V_1$, $V_2$, такие, что
  $L(V_1, V_2) \hm\ge \varepsilon$, $L(X V_1, X V_2) \hm<\delta$}.
  \item \textit{Для любых $\varepsilon\in(0,1)$, $\delta>0$ существуют $V_1$, $V_2$, такие, что
  $L(\ln V_1, \ln V_2) \hm\ge \varepsilon$, $L(X V_1, X V_2)\hm <\delta$}.
\end{enumerate}


\medskip

\noindent
Д\,о\,к\,а\,з\,а\,т\,е\,л\,ь\,с\,т\,в\,о\,.\ 


\noindent
\textbf{1.}\ Докажем первое утверждение.
Пусть  $V_1$, $V_2$~--- случайные величины, вы\-рож\-ден\-ные в точках $a$, $a+\varepsilon$ соответственно.
      Очевидно, $L(V_1, V_2) \hm= \varepsilon$. Покажем, что для любого
       $\delta\hm>0$ существует $a \hm= a(\delta)$, такое, что
      $L(X V_1, X V_2)\hm < \delta$. Обозначим $x_{\max}$~--- рациональное число, такое, что
      $2( 1 \hm- \Phi\left(  x_{\max} \right) )\hm<\delta$. Справедлива следующая  цепочка равенств:
      \begin{multline*}
      L(X V_1, X V_2)= {}\\
      {}=\sup\limits_{x \ge 0}\!
      \left\{|\gamma| : \Phi\left(\fr{x}{a}\right) =
      \Phi\left(\fr{x+\gamma}{a+\varepsilon}\right)+\gamma
      \right\}={}
\\{}
      =\!\sup\limits_{x \in [0,(a+\varepsilon)x_{\max}]}\!
      \left\{|\gamma| : \Phi\left(\fr{x}{a}\right) =
      \Phi\left(\fr{x+\gamma}{a+\varepsilon}\right)+\gamma
      \right\}\vee{}\\
{}
      \vee\!\sup\limits_{x \in \left((a+\varepsilon)x_{\max},+\infty\right)}
      \!\left\{|\gamma| : \Phi\left(\fr{x}{a}\right) =
      \Phi\left(\fr{x+\gamma}{a+\varepsilon}\right)+\gamma
      \right\}\,.\hspace*{-7.91489pt}
      \end{multline*}
      Далее,
      \begin{multline*}
      \!\!\!\sup\limits_{x \in [0,(a+\varepsilon)x_{\max}]}
      \left\{|\gamma| : \Phi\left(\fr{x}{a}\right) =
      \Phi\left(\fr{x+\gamma}{a+\varepsilon}\right)+\gamma
      \right\}\le{}\\
      {}\le
      \sup\limits_{x \in [0,(a+\varepsilon)x_{\max}]}
      \left(
      \Phi\left(\fr{x}{a}\right) -
      \Phi\left(\fr{x }{a+\varepsilon} \right)
      \right)\le{}
\\
  {}    \le
      \sup\limits_{x \in [0,(a+\varepsilon)x_{\max}]}
      \left(       \phi(0)       \left(       \fr{x}{a}  -
      \frac{x }{a+\varepsilon}       \right)
      \right)\le {}\\
      {}\le      \fr{ 1}{\sqrt{2 \pi} }\,\fr{x_{\max}\varepsilon }{a}\,.
      \end{multline*}
      Рассмотрим теперь второе выражение:
      \begin{multline*}
      \hspace*{-3mm}\!\!\sup\limits_{x \in ((a+\varepsilon)x_{\max},+\infty)}\!
      \left\{|\gamma| : \Phi\left(\fr{x}{a}\right) =
      \Phi\left(\fr{x+\gamma}{a+\varepsilon}\right)+\gamma
      \right\} \le{}\\
  {}   \le \sup\limits_{x \in ((a+\varepsilon)x_{\max},+\infty)}
      \left( \Phi\left(\frac{x}{a}\right) -
      \Phi\left(\fr{x }{a+\varepsilon} \right)
      \right) \le {}
      \end{multline*}

\noindent
\begin{multline*}
{}\le
      \!\!\sup\limits_{x \in ((a+\varepsilon)x_{\max},+\infty)}
      \left( 1 - \Phi\left(\fr{x}{a}\right)   +{}\right.\\
\left.      {}+
      1 - \Phi\left(\fr{x }{a+\varepsilon} \right)
      \right)=      {}\\
      {}      =   1 - \Phi\left(x_{\max} + \fr{ x_{\max} \varepsilon  }{   a  }\right)   +
      1 - \Phi\left(  x_{\max}  \right)<{}\\
      {}<
      2( 1 - \Phi\left( x_{\max}   \right) )\,.
\end{multline*}
      Получается, что при
      $
      a >  x_{\max}\varepsilon/(\sqrt{2\pi}\,\delta)  $
      $$
       L(X V_1, X V_2) <   \delta   \vee \delta = \delta\,.
            $$

\noindent
\textbf{2.} Докажем второе утверждение. Аналогично пусть  
    $V_1$, $V_2$~--- случайные величины, вырожденные в точках~$a$,
      $a\,\exp(\varepsilon)$ соответственно.
      Очевидно, $L( \ln V_1, \ln V_2) \hm= \varepsilon$. 
      Покажем, что для любого $\delta\hm>0$ существует $a \hm= a(\delta)$, такое, что
      $L(X V_1, X V_2)\hm <\delta$. Пусть $Z$~--- 
      случайная величина, вырожденная в нуле. Воспользуемся неравенством треугольника:
      $$
      L(X V_1, X V_2)\le L(X V_1,Z) +  L(Z, X V_2)\,.
      $$
      Далее,
      \begin{gather*}
      \left(L(X V_1,Z)\right)^2\le
      \int\limits_0^{\infty}
      \left(1 - \Phi\left(\fr{x}{a}
      \right)\right)dx  = \fr{1}{\sqrt{2\pi } }a\,, \\
      \left( L(Z, X V_2)\right)^2\le
       \fr{1}{\sqrt{2\pi } }a\,\exp(\varepsilon)\,.
      \end{gather*}
      Таким образом, при $ a \hm<\sqrt{2\pi }\, (\delta^2/4) \exp(-\varepsilon)$
      $$
       ~~~~~~~~~~~~L(X V_1, X V_2) <  \fr{\delta}{2} +
       \fr{ \delta}{2}\exp\left(-\fr{\varepsilon}{2}\right)
       < \delta\,.~~~~~~~~~~~\square
       $$



Из леммы~3 следует, что расстояние Леви между масштабными смесями
нормальных законов невозможно оценить как через расстояние Леви
между случайными величинами с соответствующими смешивающим
распределениями, так и через расстояние Леви между их логарифмами.
Таким образом, выписать оценки, аналогичные оценкам для сдвиговых
смесей, без дополнительных ограничений не представляется возможным.

\section{О существовании нижних оценок устойчивости
смесей вероятностных распределений}

В утверждениях, рассмотренных выше, на смешивающие распределения не
накладывались никакие дополнительные ограничения, кроме технических
(необходимых для того, чтобы расстояния в используемых метриках были
конечны). Однако если потребовать выполнения некоторых
дополнительных условий, можно доказать существование искомых оценок
для сдвиговых и масштабных смесей нормальных законов.

Приведенное ниже утверждение сформулировано для произвольного класса
смесей, порожденных некоторым ядром. Это связанно с тем, что при
доказательстве никак не используются ни свойства ка\-ко\-го-то
конкретного ядра, ни конкретный вид смешивающих распределений. Смеси
нормальных законов являются частным случаем класса, описываемого в
приведенной теореме.

Рассмотрим семейство смесей $\mathfrak{A} \hm= \{P_t, t\in T\}$ и
семейство соответствующих смешивающих распределений $\mathfrak{B} \hm=
\{Q_t, t\in T\}$, заданных на измеримых пространствах
$(S_{\mathfrak{A}}, \mathcal{F}_{\mathfrak{A}})$,
$(S_{\mathfrak{B}}, \mathcal{F}_{\mathfrak{B}})$ соответственно.
Предполагается, что $T$~--- некоторое параметрическое множество и для
любого $t\hm\in T$ смеси $P_t$ соответствует смешивающее распределение
$Q_t$.  Через $\bar{\mathfrak{B}}$ обозначим семейство, содержащее
все распределения, заданные на $(S_{\mathfrak{B}},
\mathcal{F}_{\mathfrak{B}})$ и являющиеся слабыми пределами
последовательностей из  $\mathfrak{B}$. Ему соответствуют семейство
смесей $\bar{\mathfrak{A}}$ и параметрическое множество~$\bar{T}$.
Пусть~$\alpha$ и~$\beta$ метризуют слабую сходимость в
$\bar{\mathfrak{A}}$ и $\bar{\mathfrak{B}}$ соответственно.

\medskip

\noindent
\textbf{Теорема~2.}
\textit{Пусть $\bar{\mathfrak{A}}$ идентифицируемо, $\mathfrak{B}$ плотно.
Если для  $\bar{\mathfrak{A}}$ существуют верхние оценки
устойчивости вида}
\begin{multline*}
\forall \varepsilon > 0  \ \ \forall t_1, t_2 \in  \bar{T}  \ \
\beta(Q_{t_1}, Q_{t_2}) \le \varepsilon   \Rightarrow {}\\
{}\Rightarrow \alpha(P_{t_1}, P_{t_2}) \le g_u(\varepsilon)\,,
\end{multline*}
\textit{где $g_u$~--- некоторая вещественная функция, определенная на
вещественной положительной полуоси,
$$
\lim\limits_{x\downarrow 0} g_u(x) = 0\,,
$$
то для этого семейства должны существовать нижние оценки устойчивости вида
\begin{multline*}
\forall \varepsilon > 0  \ \ \forall t_1, t_2 \in T \ \
\beta(Q_{t_1}, Q_{t_2}) \ge \varepsilon    \Rightarrow{}\\
{}\Rightarrow \alpha(P_{t_1}, P_{t_2}) \ge g_l(\varepsilon)>0,
\end{multline*}
где $g_l$~--- некоторая вещественная функция, определенная на
вещественной положительной полуоси.}

\medskip

\noindent
\textbf{Замечание~1.}
В условиях теоремы вместо требования существования верхних оценок
можно потребовать от семейства $\bar{\mathfrak{A}}$, чтобы оно было
плотным. Возможно, это условие является избыточным для некоторых
классов смесей. Однако для смесей нормальных распределений указанные
ограничения являются достаточно слабыми.

\medskip

\noindent
Д\,о\,к\,а\,з\,а\,т\,е\,л\,ь\,с\,т\,в\,о\,.\ 
По условию $\mathfrak{A}$ идентифицируемо. Поэтому из того, что
расстояние~$\beta$ между двумя смесями из $\mathfrak{B}$
положительно, следует, что для соответствующих распределений из
$\mathfrak{A}$ расстояние~$\alpha$ также положительно. 

Пусть
доказываемое утверждение неверно. Тогда существует $\varepsilon >
0$, такое, что для множества индексов
$T^2_\varepsilon \hm= \left\{
 t_1, t_2 \hm\in T|
\beta(Q_{t_1}, Q_{t_2})\hm \ge \varepsilon
\right\}$
$$ \inf\limits_{t_1, t_2 \in T^2_\varepsilon}
 \alpha(P_{t_1}, P_{t_2}) =0\,.
$$
Рассмотрим последовательность пар распределений из~$\mathfrak{A}$:
\begin{multline*}
\{(P_{i_n}, P_{j_n})\}^{\infty}_{n=1}\,, \ \  \alpha(P_{i_n}, P_{j_n})
< \fr{1}{n}\,, \\  i_n, j_n \in T^2_\varepsilon\,, \ \ n =
1,2,3,\ldots
\end{multline*}
Заметим, что последовательности $\{ Q_{i_n} \}^{\infty}_{n=1}$ и $\{
Q_{j_n} \}^{\infty}_{n=1}$ являются плотными и по теореме Прохорова
относительно компактными (см., например,~\cite{bilingRus}).
Следовательно, из них можно выделить сходящиеся
подпоследовательности. Поэтому без ограничения общности можем
считать, что сами последовательности   $\{ Q_{i_n}
\}^{\infty}_{n=1}$ и $\{ Q_{j_n} \}^{\infty}_{n=1}$ сходятся  и их
пределы принадлежат~$\bar{\mathfrak{B}}$. Учитывая, что для
$\bar{\mathfrak{A}}$ существуют верхние оценки устойчивости,
соответствующие последовательности смесей будут также сходиться к
соответствующим пределам. Так как~$\alpha$ метризует слабую сходимость
в $\bar{\mathfrak{A}}$, а $\beta$ метризует слабую сходимость в
$\bar{\mathfrak{B}}$, получаем:
\begin{align*}
\lim_{n\rightarrow\infty} P_{i_n} &= \lim_{n\rightarrow\infty} P_{j_n}\,; \\
\lim_{n\rightarrow\infty} Q_{i_n} & \neq \lim_{n\rightarrow\infty} Q_{j_n}\,. 
\end{align*}
Однако в этом случае получается, что $\bar{\mathfrak{A}}$ не
является идентифицируемым. Возникает противоречие, а следовательно,
доказываемое утверждение вер-\linebreak но.\hfill~$\square$

\medskip

\noindent
\textbf{Следствие~1.}
Очевидно, что указанная функция $g_l$ может быть взята неубывающей,
и, следовательно, в условиях теоремы~2
\begin{multline*}
\forall \varepsilon > 0  \ \ \forall t_1, t_2 \in T \ \
\alpha(P_{t_1}, P_{t_2}) \le{}\\
{}\le \varepsilon  \  \Rightarrow
\beta(Q_{t_1}, Q_{t_2}) \le  g^{-1}_l(\varepsilon) \,,
\end{multline*}
где
$$
 g^{-1}_l(y) = \inf\limits_{
 x > 0 }\left\{
 g_l(x) > y
 \right\}  \enskip  \forall y >0\,.
$$


\noindent
\textbf{Следствие~2.}
Из условий теоремы~$2$ напрямую следует, что в случае, когда
семейство смешивающих распределений является плотным, нижние оценки
устойчивости существуют для сдвиговых и масштабных смесей нормальных
распределений.

\medskip

Заметим, что если опираться на результаты работы~\cite{NazStab} для
масштабной смеси вида
$$
Y=V X \,,\  \  X\sim\mathcal{N}(0,1)\,, \ \ V>0
$$
(случайные величины $X$ и  $V$ стохастически независимы),
компактность будет требоваться от семейства смешивающих
распределений, со\-от\-вет\-ст\-ву\-ющих  случайным величинам вида $\ln V$, а
не~$V$.

\section{Заключение}

На практике в большинстве случаев носители рассматриваемых
смешивающих распределений ограничены в совокупности. Более того,
очень час\-то  они сосредоточены в конечном числе точек. 

В~связи с
этим оценки для каждого конкретного случая могут оказаться сильно
лучше тех, что рассмотрены выше. Тем не менее полученные оценки
имеют существенное значение. 

 Доказательство состоятельности оценок,
полученных с помощью сеточных методов разделения\linebreak смесей
вероятностных распределений, опирается на  теоремы существования,
доказанные в данной\linebreak в работе. Помимо этого, рассмотренные примеры\linebreak
позволяют лучше  понять, насколько сильно могут\linebreak изменяться оценки
смешивающего распределе-\linebreak ния при работе с реализациями одной и той же\linebreak
смеси.



{\small\frenchspacing
{%\baselineskip=10.8pt
\addcontentsline{toc}{section}{Литература}
\begin{thebibliography}{99}

\bibitem{KorRibGrid} %1
\Au{Королев В.\,Ю., Непомнящий~Е.\,В., Рыбальченко~А.\,Г., Виноградова~А.\,В.} 
Сеточные методы разделения смесей вероятностных распределений
и их применение к декомпозиции волатильности финансовых индексов~//
Информатика и её применения, 2008. Т.~2. Вып.~2. С.~3--18.

\bibitem{NazGrid} %2
\Au{Назаров~А.\,Л.} Разделение смесей вероятностных распределений
сеточным методом максимального правдоподобия при помощи алгоритма
условного градиента~// Сб. статей молодых ученых факультета ВМиК
МГУ, 2009. Вып.~6. С.~128--135.

\bibitem{KorNazGridEng} %3
\Au{Korolev~V., Nazarov~A.} Separating mixtures of probability
  distributions with the grid method of moments and the grid maximal likelihood
  method~// Autom. Remote Control, 2010. Vol.~71. No.\,3. P.~455--472.
  
  \bibitem{KorolevChaos} %4
\Au{Королев~В.\,Ю.} Ве\-ро\-ят\-но\-ст\-но-ста\-ти\-сти\-че\-ские методы декомпозиции
волатильности хаотических процессов.~--- М.: Изд-во Моск.
ун-та, 2011.

\pagebreak


\bibitem{Tukey_1960}
\Au{Tukey~J.\,W.} A~survey of sampling from contaminated
distributions~// Contributions to Probability and Statistics Essays
in Honor of Harold Hotelling, 1960. P.~448--485.

\bibitem{Hall1979357}
\Au{Hall~P.} On measures of the distance of a mixture from its
parent distribution~// Stochastic Proc. Their Applications,
1979. Vol.~8. No.\,3. P. 357--365.

\bibitem{NazStab}
\Au{Назаров~А.\,Л.} Об устойчивости смесей вероятностных законов к
возмущениям смешивающих распределений~// Статистические методы
оценивания и проверки гипотез: Межвузовский сб. науч. тр.~--- Пермь: ПГУ, 2010. 
Вып.~22. С.~154--172.

\bibitem{NazConvUnpubl}
\Au{Назаров~А.\,Л.} Асимптотические свойства оценок, полученных с
помощью сеточных методов разделения смесей вероятностных
распределений~// Статистические методы оценивания и проверки
гипотез: Межвузовский сб. науч. тр.~--- Пермь: ПГУ, 2012.
Вып.~24. С.~22--35.

\bibitem{Skorohod1956}
\Au{Скороход~А.\,В.} Предельные теоремы для случайных процессов~//
Теория вероятностей и ее применения, 1956. Т.~1. Вып.~3.
С.~289--319.

\bibitem{shirVer}
\Au{Ширяев~А.\,Н.} Вероятность.~--- М.: МЦНМО, 2007.

\bibitem{KolmFomin}
\Au{Колмогоров~А.\,Н., Фомин~С.\,В.} Элементы теории функций и
функционального анализа.~--- М.: Наука, 1976.

\label{end\stat}

\bibitem{bilingRus}
\Au{Биллингсли~П.} Сходимость вероятностных мер.~--- М.: Наука, 1977.
\end{thebibliography}
}
}

\end{multicols}