\def\stat{abstr}
{%\hrule\par
%\vskip 7pt % 7pt
\raggedleft\Large \bf%\baselineskip=3.2ex
A\,B\,S\,T\,R\,A\,C\,T\,S \vskip 17pt
    \hrule
    \par
\vskip 21pt plus 6pt minus 3pt }

\label{st\stat}

%\def\rightmark{\ }

%1
\def\tit{ANALYTICAL MODELING INVARIANT MEASURE DISTRIBUTIONS 
IN~STOCHASTIC SYSTEMS WITH AUTOCORRELATED NOISES}

\def\aut{I.\,N. Sinitsyn}

\def\auf{IPI RAN, sinitsin@dol.ru}

\def\leftkol{\ } % ENGLISH ABSTRACTS}
\def\rightkol{\ } %ENGLISH ABSTRACTS}

\titele{\tit}{\aut}{\auf}{\leftkol}{\rightkol}

%\vspace*{12pt}

\noindent
For multidimensional nonlinear normal (Gaussian) differential systems with un- and 
autocorrelated noises, on the basis of normal approximation, the correlational algorithms 
for analytical modeling of stochastic regimes with invariant measure are considered. 
Special software tools in MATLAB are developed. Test examples confirm practical accuracy.



%\vspace*{-5pt}

\KWN{analytical modeling; autocorrelated noise; correlational algorithm; 
distribution with invariant measure; multidimensional nonlinear differential 
stochastic system; normal approximation method}

%\thispagestyle{myheadings}


\vskip 14pt plus 6pt minus 3pt

%2
\def\tit{ON THE ACCURACY OF~SOME MATHEMATICAL MODELS
 OF~CATASTROPHICALLY ACCUMULATED EFFECTS IN~PREDICTION OF~RISKS OF~EXTREMAL EVENTS}

\def\aut{I.\,A.~Duchitskii$^1$, V.\,Yu.~Korolev$^2$, and~I.\,A.~Sokolov$^3$}

\def\auf{$^1$Faculty of Computational Mathematics and Cybernetics, 
   M.\,V.~Lomonosov Moscow State University,\\
   \hphantom{$^1$}duchik@gmail.com\\[1pt]
$^2$M.\,V.~Lomonosov Moscow State University; IPI RAN, vkorolev@cs.msu.su\\[1pt]
$^3$IPI RAN, isokolov@ipiran.ru}


\def\leftkol{\ } % ENGLISH ABSTRACTS}

\def\rightkol{\ } %ENGLISH ABSTRACTS}

\titele{\tit}{\aut}{\auf}{\leftkol}{\rightkol}

%\vspace*{12pt}

\noindent
Estimates are constructed for the accuracy of approximation 
of the distributions of extrema of special random sums by scale mixtures of 
half-normal laws. The possibility of the application of these results in prediction 
of risks of extremal events due to catastrophically accumulated effects is discussed.

%\vspace*{-5pt}

\KWN{nonhomogeneous flows of events; doubly stochastic Poisson process; negative binomial 
distribution; gamma-distribution; convergence rate estimate}

\vskip 14pt plus 6pt minus 3pt

%\pagebreak


%3
\def\tit{ABOUT ADAPTIVE STRATEGIES AND~THEIR EXISTENCE CONDITIONS}

\def\aut{M.\,G.~Konovalov}

\def\auf{IPI RAN, mkonovalov@ipiran.ru}


\def\leftkol{\ } % ENGLISH ABSTRACTS}

\def\rightkol{\ } %ENGLISH ABSTRACTS}

\titele{\tit}{\aut}{\auf}{\leftkol}{\rightkol}

%\vspace*{-2pt}

\noindent
The optimal control problem is considered under deficiency of \textit{a priori} 
information about a controlled object. The solution of the problem is the construction 
of adaptive strategies on the base of in-control available observations. Some conditions 
of adaptive controllability are studied. Controlled random sequences are used as mathematical 
model.

%\vspace*{-5pt}

\KWN{сontrolled random sequences; adaptive strategies; existence conditions}


%\pagebreak

% \vskip 12pt plus 6pt minus 3pt

\pagebreak

\def\leftkol{\ } % ENGLISH ABSTRACTS}
\def\rightkol{\ } %ENGLISH ABSTRACTS}

 %4
\def\tit{BOUNDS IN NULL ERGODIC CASE FOR~SOME QUEUEING SYSTEMS}

\def\aut{A.\,I.~Zeifman$^1$, A.\,V.~Korotysheva$^2$, Ya.~Satin$^3$, and~S.\,Ya.~Shorgin$^4$}

\def\auf{$^1$Vologda State Pedagogical University; IPI RAN;
VSCC CEMI RAS, a\_zeifman@mail.ru\\[1pt]
$^2$Vologda State Pedagogical University, a\_korotysheva@mail.ru\\[1pt]
$^3$Vologda State Pedagogical University, yacovi@mail.ru\\[1pt]
$^4$IPI RAN, SShorgin@ipiran.ru}



\titele{\tit}{\aut}{\auf}{\leftkol}{\rightkol}

\vspace*{-4pt}
 
\noindent
Markovian queueing models with batch arrivals and group services
are considered. The bounds on the rate of convergence in null ergodic situation  are obtained. 
Also,  a class of such queueing systems is considered.


\vspace*{-6pt}

\KWN{nonstationary queueing systems with batch arrivals and group services; 
null ergodicity; bounds}

 \vskip 8pt plus 6pt minus 3pt

%5
\def\tit{GENERALIZED LAPLACE DISTRIBUTION AS A LIMIT LAW FOR~RANDOM SUMS 
AND~STATISTICS CONSTRUCTED FROM~SAMPLES WITH~RANDOM SIZES}

\def\aut{V.\,Yu.~Korolev$^1$, V.\,E.~Bening$^2$, L.\,M.~Zaks$^3$, and A.\,I.~Zeifman$^4$}

\def\auf{$^1$M.\,V.~Lomonosov Moscow State University; IPI RAN, vkorolev@cs.msu.su\\[1pt]
$^2$Department of Mathematical Statistics, 
Faculty of Computational Mathematics and Cybernetics,\\ 
$\hphantom{^1}$M.\,V.~Lomonosov Moscow State University; IPI RAN, bening@cs.msu.su\\[1pt]
$^3$Department of Modeling and 
Mathematical Statistics, Alpha-Bank, lily.zaks@gmail.com\\[1pt]
$^4$Vologda State Pedagogical University; IPI RAN;
VSCC CEMI RAS, a\_zeifman@mail.ru}

%\def\leftkol{ENGLISH ABSTRACTS}
%\def\rightkol{ENGLISH ABSTRACTS}

\titele{\tit}{\aut}{\auf}{\leftkol}{\rightkol}

\vspace*{-4pt}

\def\leftkol{ENGLISH ABSTRACTS}

\def\rightkol{ENGLISH ABSTRACTS}


\noindent
Limit theorems establishing necessary and sufficient conditions of convergence 
of random sums and statistics constructed from the samples with random sizes to the 
generalized Laplace distribution are proved.


\vspace*{-6pt}

\KWN{generalized Laplace distribution; symmetric stable distribution; 
one-sided stable distribution; scale mixture of normal laws; random sum; 
sample with random size; mixed Poisson distribution}


 \vskip 8pt plus 6pt minus 3pt

%6
\def\tit{LOWER BOUNDS FOR~THE~STABILITY OF~NORMAL MIXTURE MODELS 
WITH~RESPECT~TO~PERTURBATIONS OF~MIXING DISTRIBUTION}

\def\aut{A.~Nazarov}

\def\auf{Department of Mathematical Statistics, Faculty of Computational
Mathematics and Cybernetics, M.\,V.~Lomonosov
Moscow State University nazarov.vmik@gmail.com}

%\def\leftkol{ENGLISH ABSTRACTS}
%\def\rightkol{ENGLISH ABSTRACTS}

\titele{\tit}{\aut}{\auf}{\leftkol}{\rightkol}

\vspace*{-4pt}

\def\leftkol{ENGLISH ABSTRACTS}

\def\rightkol{ENGLISH ABSTRACTS}

\noindent
The stability of normal mixture models with respect
to perturbations of mixing distribution is investigated. Inequality
estimating the distance between two mixing distributions through the
closeness of the corresponding mixtures is presented. Existence
theorem for stability estimates is proved for subclasses of  scale
and shift mixtures of normal distributions. For the class of shift
mixtures, the estimate is obtained in an explicit form. It is shown
that the presented results cannot be radically improved without
additional assumptions.

\vspace*{-6pt}

\KWN{normal distribution mixtures;  stability problems
for stochastic models; Fourier transform; Plancherel theorem;
Prokhorov's theorem; L$\acute{\mbox{e}}$vy metric;  lower bounds}

 \vskip 8pt plus 6pt minus 3pt

%7
\def\tit{PREPROCESSING OF TEXT RECOGNITION UNDER~THE~POOR QUALITY IMAGE}


\def\aut{M.\,P.~Krivenko}

\def\auf{IPI RAN, mkrivenko@ipiran.ru}

%\def\leftkol{ENGLISH ABSTRACTS}
%\def\rightkol{ENGLISH ABSTRACTS}

\titele{\tit}{\aut}{\auf}{\leftkol}{\rightkol}

\vspace*{-4pt}

\def\leftkol{ENGLISH ABSTRACTS}

\def\rightkol{ENGLISH ABSTRACTS}

\noindent
The methods of preprocessing of text images including the skew correction and the line 
segmentation are discussed for the case where the recognizable image 
is of low quality being obtained with high resolution. Provided that the brightness 
of the pixel rows of characters differs, even slightly, from the brightness of the background 
pixels, the procedures for the skew correction and segmentation of the
 text lines  are proposed and analyzed.
 

\vspace*{-6pt}

\KWN{text recognition; image preprocessing; skew correction; text line segmentation}

%\pagebreak

 \vskip 12pt plus 6pt minus 3pt

%8
\def\tit{RANDOM GRAPHS MODEL FOR~DESCRIPTION OF~INTERACTIONS IN~THE~NETWORK}

\def\aut{A.~Grusho$^1$ and E. Timonina$^2$} 


\def\auf{$^1$IPI RAN; Department of Mathematical Statistics, 
Faculty of Computational Mathematics and Cybernetics,\\ 
$\hphantom{^1}$M.\,V.~Lomonosov Moscow State University, grusho@yandex.ru\\[1pt]
$^2$IPI RAN, eltimon@yandex.ru}

%\def\leftkol{ENGLISH ABSTRACTS}
%\def\rightkol{ENGLISH ABSTRACTS}

\titele{\tit}{\aut}{\auf}{\leftkol}{\rightkol}

\vspace*{-2pt}

\def\leftkol{ENGLISH ABSTRACTS}

\def\rightkol{ENGLISH ABSTRACTS}

\noindent
A new class of random graphs urged to simulate network functioning in time is considered. 
It is supposed that observations over a network are carried by means of a ``window'' method. 
To detect the anomalies, normal behavior which can be watched in ``windows'' of a
considered model is studied. The asymptotic value of the maximum degree of vertices in graph 
which is generated by a ``window'' of certain size is analyzed.


\vspace*{-2pt}

\KWN{random graphs; simulation of wide area networks; 
information security; abnormal behavior}


 \vskip 10pt plus 6pt minus 3pt


%9
\def\tit{ON THE OPTIMAL CORRECT RECODING OF INTEGER DATA IN~RECOGNITION}

\def\aut{E.\,V.~Djukova$^1$, A.\,V.~Sizov$^2$, and~R.\,M.~Sotnezov$^3$} 


\def\auf{$^1$Institution of Russian Academy of Sciences Dorodnicyn Computing Centre of RAS, 
edjukova@mail.ru\\[1pt]
$^2$Moscow State University, box.sizov@gmail.com\\[1pt]
$^3$Institution of the Russian Academy of Sciences Dorodnicyn Computing Center of RAS, rom.sot@gmail.com}

%\def\leftkol{ENGLISH ABSTRACTS}
%\def\rightkol{ENGLISH ABSTRACTS}

\titele{\tit}{\aut}{\auf}{\leftkol}{\rightkol}

%\vspace*{-2pt}

\def\leftkol{ENGLISH ABSTRACTS}

\def\rightkol{ENGLISH ABSTRACTS}

\vspace*{-2pt}

\noindent
Questions of application of logical procedure of recognition by precedents 
in the case of float information and high-atomicity integer information are 
investigated. The problem of correct reducing the data atomicity is considered. 
Genetic algorithms for the search of optimal correct recoding of source information 
are developed. Developed algorithms are tested on real data.

\vspace*{-2pt}

\KWN{pattern recognition; correct recoding; covering of the Boolean matrix}

\vskip 10pt plus 6pt minus 3pt



%10
\def\tit{ESTIMATION OF LINEAR MODEL HYPERPARAMETERS FOR~NOISE OR~CORRELATED FEATURE SELECTION PROBLEM}

\def\aut{A.\,A.~Tokmakova$^1$ and V.\,V.~Strijov$^2$}

\def\auf{$^1$Moscow Institute of Physics and Technology, aleksandra-tok@yandex.ru\\[1pt]
$^2$Computing Center RAS,  strijov@ccas.ru}


\def\leftkol{ENGLISH ABSTRACTS}

\def\rightkol{ENGLISH ABSTRACTS}

\titele{\tit}{\aut}{\auf}{\leftkol}{\rightkol}

\vspace*{-2pt}

\noindent
The problem of feature selection in linear regression models
has been solved. To select the features, the authors estimate the covariance matrix 
of the model parameters. Dependent variable and model parameters are assumed to be 
normally distributed vectors. Laplace approximation is used for estimation of the 
covariance matrix: logarithm of the error function is approximated by the normal 
distribution function. The problem of noise or correlated features is also examined, 
since in this case, the covariance matrix of the model parameters becomes singular. An algorithm 
for feature selection is suggested. The results of the study for a time series are given 
in the computational experiment.

\vspace*{-2pt}

\KWN{feature selection; regression; coherent Bayesian inference; covariance matrix; 
model parameters}
%\pagebreak

\vskip 10pt plus 6pt minus 3pt

%11
\def\tit{HOLOGRAPHIC CODING BY WALSH--HADAMARD TRANSFORMATION OF~RANDOMIZED AND~PERMUTED DATA}

\def\aut{S.~Dolev$^1$, S.~Frenkel$^2$, and A.~Cohen$^3$}

\def\auf{$^1$Department of Computer Science, Ben-Gurion University of the Negev, Beer-Sheva,
Israel, dolev@cs.bgu.ac.il\\[1pt]
$^2$IPI RAN; Moscow Institute of Radio, Electronics, and Automation (MIREA), fsergei@mail.ru\\[1pt]
$^3$Department of Communication Systems Engineering, 
Ben-Gurion University of the Negev, Beer-Sheva, Israel,\\
$\hphantom{^1}$coasaf@cse.bgu.ac.il}


\def\leftkol{ENGLISH ABSTRACTS}

\def\rightkol{ENGLISH ABSTRACTS}

\titele{\tit}{\aut}{\auf}{\leftkol}{\rightkol}

%\vspace*{-2pt}

\noindent
Holographic coding has the very appealing property of obtaining partial information
on data, from any part of the
coded information. Holographic coding schemes are studied based on
the Walsh--Hadamard orthogonal codes.
It is proposed to randomize the data so that the coefficient of the
Walsh--Hadamard code would be approximately uniform in order to ensure, with
high probability, a monotonic gain of information. The data are
xored with randomly chosen bits from random data that have been
stored during a preprocessing stage or pseudorandom data produced
by a pseudorandom generator.
Statistical properties of the
truncated sums of Inverse Walsh--Hadamard Transformation (WHT),
taking into account the ``white-noise nature'' and the mentioned above  holographic properties
of this encoding method,  and the performance of the method is considered
based on the theoretic Shannon bound.
Using this performance measure, an enhancement for the authors' previous 
WHT-based holographic coding method is suggested.
This enhancement is based on a random  permutation.

 
%\vspace*{-2pt}

\KWN{holographic coding; Walsh--Hadamard transformation; Shannon bound}

%\pagebreak

\vskip 12pt plus 6pt minus 3pt

% \vskip 12pt plus 6pt minus 3pt

%12
\def\tit{MATHEMATICAL FOUNDATION, APPLICATION, AND~COMPARISON OF~GENERAL DATA 
ASSIMILATION METHOD BASED ON~DIFFUSION APPROXIMATION WITH~OTHER DATA ASSIMILATION SCHEMES}

\def\aut{K.\,P.~Belyaev$^1$, C.\,A.\,S.~Tanajura$^2$, and~N.\,P.~Tuchkova$^3$}

\def\auf{$^1$Shirshov Institute of Oceanology, Russian Academy of Sciences, 
Moscow, Russia, kb@sail.msk.ru\\[1pt]
$^2$Federal University of Bahia, Salvador, Brazil, cast@ufba.br\\[1pt]
$^3$Institution of the Russian Academy of Sciences Dorodnicyn Computing Center of RAS,
Moscow, Russia,\\
$\hphantom{^1}$tuchkova@ccas.ru}


\def\leftkol{ENGLISH ABSTRACTS}

\def\rightkol{ENGLISH ABSTRACTS}

\titele{\tit}{\aut}{\auf}{\leftkol}{\rightkol}

%\vspace*{-2pt}

\noindent 
Data assimilation methods commonly used in numerical ocean and atmospheric 
circulation models for weather and climate prediction produce approximations of state 
variables in terms of stochastic processes. This approximation consists of random sequences 
of Markov chains, which converge to a diffusion-type process. The conditions for this 
convergence are investigated. The optimization problem associated with the search of the 
best possible approximation of the state variable and the results of a numerical experiment 
are discussed. It is shown that the data assimilation method can be used in practical 
applications in meteorology and oceanography. Several applications of the methods as an 
example of the modern operational data processing system with the ocean circulation model 
HYCOM and data from ARGO drifters are performed and the results as well as comparisons with 
other assimilation schemes are presented.


%\vspace*{-2pt}

\KWN{sequence of Markov chains; diffusion stochastic process; 
data assimilation methods; HYCOM; ARGO drifters}

\vskip 12pt plus 6pt minus 3pt

%\pagebreak
  
  %13
\def\tit{COMPLETE CONVERGENCE FOR~ARRAYS OF~NEGATIVELY DEPENDENT RANDOM VARIABLES}

\def\aut{S.\,H.~Sung$^1$, K.~Budsaba$^{2}$, and~A.~Volodin$^{3}$}

\def\auf{$^1$Department of Applied Mathematics, Pai Chai University, Taejon, South Korea, 
sungsh@pcu.ac.kr\\[1pt]
$^2$Center of Excellence in Mathematics, CHE, Bangkok, Thailand;
Department of Mathematics and Statistics,\\
$\hphantom{^1}$Thammasat University Rangsit Center, 
Pathumthani, Thailand, kamon@mathstat.sci.tu.ac.th\\[1pt]
$^3$School of Mathematics and Statistics, University of Western Australia, 
Crawley, Australia; University of Regina,\\
$\hphantom{^1}$Canada, 
Andrei.Volodin@uregina.ca}

\def\leftkol{ENGLISH ABSTRACTS}

\def\rightkol{ENGLISH ABSTRACTS}

\titele{\tit}{\aut}{\auf}{\leftkol}{\rightkol}

%\vspace*{-2pt}

\noindent
A general result establishing complete convergence for the row sums 
of an array of row-wise negatively dependent random variables is presented. From this result, 
a number of complete convergence results is obtained
for weighted sums of negatively dependent 
random variables.


 \label{end\stat}

%\vspace*{-2pt}

\KWN{complete convergence; negatively dependent; weighted sums; arrays}




\newpage