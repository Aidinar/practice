%Том 6   Выпуск 1-4   Год 2012

\def\stat{cont}
{%\hrule\par
%\vskip 7pt % 7pt
\raggedleft\Large \bf%\baselineskip=3.2ex
А\,В\,Т\,О\,Р\,С\,К\,И\,Й\ \ У\,К\,А\,З\,А\,Т\,Е\,Л\,Ь\ \ З\,А\ \ 2\,0\,1\,2 г. \vskip 17pt
    \hrule
    \par
\vskip 21pt plus 6pt minus 3pt }

\label{st\stat}

\def\tit{\ }

\def\aut{\ }
\def\auf{\ }

\def\leftkol{\ } % ENGLISH ABSTRACTS}

\def\rightkol{\ } %АВТОРСКИЙ УКАЗАТЕЛЬ ЗА 2012 г.} %ENGLISH ABSTRACTS}

\titele{\tit}{\aut}{\auf}{\leftkol}{\rightkol}

\vspace*{-12pt}

{\tabcolsep=3pt
\begin{tabular}{p{388pt}rr}
&\textbf{Выпуск} & \textbf{Стр.}\\[6pt]
\hangindent=23pt\noindent\textbf{Агаларов Я.\,М.} Алгоритм вычисления характеристик модели
телекоммуникационной\linebreak
\vspace*{-12pt}\\
\hspace*{23pt}сети с повторами передач и неполнодоступной схемой управления
буферами$\dotfill$&2&2\\
\hangindent=23pt\noindent\textbf{Белых И.\,Н., Капустин А.\,И., Козлов А.\,В., Лоханова А.\,И., Матвеев Ю.\,Н.,
Пехов\-ский~Т.\,С., Симончик К.\,К., Шулипа А.\,К.} Система идентификации
дикторов\linebreak
\vspace*{-12pt}\\
\hspace*{23pt}по голосу для конкурса \textit{NIST SRE 2010}$\dotfill$&1& 91\\
\hangindent=23pt\noindent\textbf{Беляев К.\,П., Танажура К.\,А.\,С., Тучкова Н.\,П.} Математическое
обоснование, применение и сравнение обобщенного метода усвоения данных наблюдений,
основанного на методах диффузионной аппроксимации, с другими методами усвоения\linebreak
\vspace*{-12pt}\\
\hspace*{23pt}данных$\dotfill$&4&84\\
\hangindent=23pt\noindent\textbf{Бендерина М.\,В., Борохов С.\,В., Будзко В.\,И., Степанов П.\,В., Сучков
А.\,П.} Управление учетными записями и правами доступа пользователей в центрах обработки
данных\linebreak
\vspace*{-12pt}\\
\hspace*{23pt}высокой доступности$\dotfill$&1&59\\
\hangindent=23pt\noindent\textbf{Бенинг В.\,Е., Закс Л.\,М., Королев В.\,Ю.} Оценки скорости сходимости
распределений\linebreak
\vspace*{-12pt}\\
\hspace*{23pt}случайных сумм к дисперсионным гамма-распределениям$\dotfill$&3&69\\
\textbf{Бенинг В.\,Е.} см.~Королев В.\,Ю.&&\\
\textbf{Боков М.\,В.} см.~Гудков В.\,Ю.&&\\
\textbf{Боровикова О.\,И.} см.~Загорулько Ю.\,А.&&\\
\textbf{Борохов С.\,В.} см.~Бендерина М.\,В.&&\\
\hangindent=23pt\noindent\textbf{Босов А.\,В.} Задачи анализа и оптимизации для модели пользовательской
активности.\linebreak
\vspace*{-12pt}\\
\hspace*{23pt}Часть~2. Оптимизация внутренних ресурсов$\dotfill$&1&19\\
\hangindent=23pt\noindent\textbf{Босов А.\,В.} Задачи анализа и оптимизации для модели пользовательской
активности.\linebreak
\vspace*{-12pt}\\
\hspace*{23pt}Часть~3. Оптимизация внешних ресурсов$\dotfill$&2&14\\
\textbf{Будзко В.\,И.} см.~Бендерина М.\,В.&&\\
\textbf{Будсаба К.} см.~Санг С.\,Х.&&\\
\hangindent=23pt\noindent\textbf{Визильтер Ю.\,В., Горбацевич В.\,С., Каратеев С.\,Л., Костромов Н.\,А.}
Обучение алгорит-\linebreak
\vspace*{-12pt}\\
\hspace*{23pt}мов выделения кожи на цветных изображениях лиц$\dotfill$&1&108\\
\textbf{Володин А.} см.~Санг С.\,Х.&&\\
\hangindent=23pt\noindent\textbf{Гайдамака Ю.\,В., Ефимушкина Т.\,В., Самуйлов А.\,К., Самуйлов К.\,Е.}
Задачи оптималь-\linebreak
\vspace*{-12pt}\\
\hspace*{23pt}ного планирования межуровневого интерфейса в беспроводных
сетях$\dotfill$&3&74\\
\textbf{Горбацевич В.\,С.} см.~Визильтер Ю.\,В.&&\\
\hangindent=23pt\noindent\textbf{Горшенин А.\,К.} Об устойчивости сдвиговых смесей нормальных законов по
отношению\linebreak
\vspace*{-12pt}\\
\hspace*{23pt}к изменениям смешивающего распределения$\dotfill$&2&22\\
\hangindent=23pt\noindent\textbf{Грушо А.\,А., Тимонина Е.\,Е.} Модель случайных графов для описания
взаимодействий\linebreak
\vspace*{-12pt}\\
\hspace*{23pt}в сети$\dotfill$&4&57\\
\hangindent=23pt\noindent\textbf{Гудков В.\,Ю., Боков М.\,В.} Быстрая обработка изображений отпечатков
пальцев$\dotfill$&1&99\\
\hangindent=23pt\noindent\textbf{Де~Никола К., Хохлов Ю.\,С., Пагано М., Сидорова О.\,И.} Дробное движение
Леви с~зависимыми приращениями и~его~приложение к~моделированию сетевого\linebreak
\vspace*{-12pt}\\
\hspace*{23pt}трафика$\dotfill$&3&59\\
\hangindent=23pt\noindent\textbf{Долев Ш., Френкель С., Коен А.} Голографическое кодирование на основе
преобразо-\linebreak
\vspace*{-12pt}\\
\hspace*{23pt}вания Уолша--Адамара рандомизированных и перемешанных данных$\dotfill$&4&76\\
\hangindent=23pt\noindent\textbf{Дулин С.\,К., Розенберг И.\,Н., Уманский В.\,И.} Обработка геопространственной
инфор-\linebreak
\vspace*{-12pt}\\
\hspace*{23pt}мации на базе репозитория геоинформационной системы$\dotfill$&2&29\\
\hangindent=23pt\noindent\textbf{Дучицкий И.\,А., Королев В.\,Ю., Соколов И.\,А.} О точности некоторых
математических моделей катастрофически накапливающихся эффектов при прогнозировании\linebreak
\vspace*{-12pt}\\
\hspace*{23pt}риска экстремальных событий$\dotfill$&4&9\\
\end{tabular}
}

\pagebreak

\def\leftkol{АВТОРСКИЙ УКАЗАТЕЛЬ ЗА 2012 г.} % ENGLISH ABSTRACTS}

\def\rightkol{АВТОРСКИЙ УКАЗАТЕЛЬ ЗА 2012 г.} %ENGLISH ABSTRACTS}

{\tabcolsep=3pt
\begin{tabular}{p{388pt}rr}
&\textbf{Выпуск} & \textbf{Стр.}\\[3pt]
\hangindent=23pt\noindent\textbf{Дюкова Е.\,В., Сизов А.\,В., Сотнезов Р.\,М.} Об оптимальном корректном
перекодирова-\linebreak
\vspace*{-12pt}\\
\hspace*{23pt}нии целочисленных данных в распознавании$\dotfill$&4&61\\
\textbf{Ефимушкина Т.\,В.} см.~Гайдамака Ю.\,В.&&\\
\hangindent=23pt\noindent\textbf{Жевнерчук Д.\,В., Николаев А.\,В.} Методика моделирования нагрузки на сервер
в откры-\linebreak
\vspace*{-12pt}\\
\hspace*{23pt}тых системах облачных вычислений$\dotfill$&2&43\\
\hangindent=23pt\noindent\textbf{Желенкова О.\,П.} Исследование радиоисточников средствами виртуальной
об\-сер\-ва-\linebreak
\vspace*{-12pt}\\
\hspace*{23pt}тории$\dotfill$&3&5\\
\textbf{Жижимов О.\,Л.} см.~Скачков Д.\,М.&&\\
\hangindent=23pt\noindent\textbf{Загорулько Ю.\,А., Боровикова О.\,И., Кононенко И.\,С., Соколова Е.\,Г.}
Методологические аспекты разработки электронного русско-английского тезауруса по
компьютер-\linebreak
\vspace*{-12pt}\\
\hspace*{23pt}ной лингвистике$\dotfill$&3&22\\
\textbf{Закс Л.\,М.} см.~Бенинг В.\,Е.&&\\
\textbf{Закс Л.\,М.} см.~Королев В.\,Ю.&&\\
\textbf{Захаров В.\,Н.} см.~Калиниченко Л.\,А.&&\\
\hangindent=23pt\noindent\textbf{Зейфман А.\,И., Коротышева А.\,В., Сатин Я.\,А., Шоргин С.\,Я.} Оценки в
нуль-эрго\-ди\-че-\linebreak
\vspace*{-12pt}\\
\hspace*{23pt}ском случае для некоторых систем обслуживания$\dotfill$&4&27\\
\textbf{Зейфман А.\,И.} см.~Королев В.\,Ю.&&\\
\hangindent=23pt\noindent\textbf{Калёнов Н.\,Е.} Задачи и функции библиотек РАН в современных
условиях$\dotfill$&2&51\\
\hangindent=23pt\noindent\textbf{Калиниченко Л.\,А., Ступников С.\,А.} Унификация языков систем на правилах
для обес-\linebreak
\vspace*{-12pt}\\
\hspace*{23pt}печения интероперабельности декларативных программ$\dotfill$&2&59\\
\hangindent=23pt\noindent\textbf{Калиниченко Л.\,А., Ступников С.\,А., Захаров В.\,Н.} Развитие технологий
интеграции\linebreak
информации для решения задач над~неоднородными информационными
ре\-сур-\linebreak
\vspace*{-12pt}\\
\hspace*{23pt}сами$\dotfill$&1&70\\
\textbf{Капустин А.\,И.} см.~Белых И.\,Н.&&\\
\textbf{Каратеев С.\,Л.} см.~Визильтер Ю.\,В.&&\\
\hangindent=23pt\noindent\textbf{Карпов А.\,А.} Когнитивные исследования ассистивного многомодального
интерфейса\linebreak
\vspace*{-12pt}\\
\hspace*{23pt}для бесконтактного человеко-машинного взаимодействия$\dotfill$&2&77\\
\hangindent=23pt\noindent\textbf{Когаловский М.\,Р., Паринов С.\,И.} Классификация и использование
семантических\linebreak
\vspace*{-12pt}\\
\hspace*{23pt}связей между информационными объектами в научных электронных
библио-\linebreak
\vspace*{-12pt}\\
\hspace*{23pt}теках$\dotfill$&3&32\\
\textbf{Коен А.} см.\ Долев Ш.&&\\
\textbf{Козлов А.\,В.} см.~Белых И.\,Н.&&\\
\hangindent=23pt\noindent\textbf{Коновалов М.\,Г.} Об адаптивных стратегиях и условиях их
существования$\dotfill$&4&18\\
\hangindent=23pt\noindent\textbf{Коновалов М.\,Г.} Оптимизация работы вычислительного комплекса с помощью
ими-\linebreak
\vspace*{-12pt}\\
\hspace*{23pt}тационной модели и адаптивных алгоритмов$\dotfill$&1&37\\
\textbf{Кононенко И.\,С.} см.~Загорулько Ю.\,А.&&\\
\hangindent=23pt\noindent\textbf{Королев В.\,Ю., Бенинг В.\,Е., Закс Л.\,М., Зейфман А.\,И.} Обобщенное
распределение Лап\-ла\-са как предельное для случайных сумм и статистик, построенных по
выборкам случайного объема$\dotfill$&4&34\\
\hangindent=23pt\noindent\textbf{Королев В.\,Ю., Соколов И.\,А.} Скошенные распределения Стьюдента,
дисперсионные\linebreak
\vspace*{-12pt}\\
\hspace*{23pt}гамма-распределения и их обобщения как асимптотические
аппроксимации$\dotfill$&1&3\\
\textbf{Королев В.\,Ю.} см.~Бенинг В.\,Е.&&\\
\textbf{Королев В.\,Ю.} см.~Дучицкий И.\,А.&&\\
\textbf{Королев В.\,Ю.} см.~Соколов И.\,А.&&\\
\textbf{Коротышева А.\,В.} см.~Зейфман А.\,И.&&\\
\textbf{Костромов Н.\,А.} см.~Визильтер Ю.\,В.&&\\
\hangindent=23pt\noindent\textbf{Кривенко М.\,П.} Предварительная обработка при распознавании текстов по
изображе-\linebreak
\vspace*{-12pt}\\
\hspace*{23pt}нию низкого качества$\dotfill$&4&49\\
\textbf{Кузнецов В.\,В.} см.~Ушмаев О.\,С.&&\\
\hangindent=23pt\noindent\textbf{Кузнецов И.\,П., Сомин Н.\,В.} Выявление имплицитной информации из текстов
на\linebreak
\vspace*{-12pt}\\
\hspace*{23pt}естественном языке: проблемы и методы$\dotfill$&1&49\\
\textbf{Кузнецов И.\,П.} см.~Шарнин М.\,М.&&\\
\hangindent=23pt\noindent\textbf{Куракин А.\,В.} Распознавание жестов ладони в реальном времени на основе
плоских\linebreak
\vspace*{-12pt}\\
\hspace*{23pt}и пространственных скелетных моделей$\dotfill$&1&114\\
\end{tabular}
}

\pagebreak

\def\leftkol{АВТОРСКИЙ УКАЗАТЕЛЬ ЗА 2012 г.} % ENGLISH ABSTRACTS}

\def\rightkol{АВТОРСКИЙ УКАЗАТЕЛЬ ЗА 2012 г.} %ENGLISH ABSTRACTS}

{\tabcolsep=3pt
\begin{tabular}{p{388pt}rr}
&\textbf{Выпуск} & \textbf{Стр.}\\[3pt]
\textbf{Лоханова А.\,И.} см.~Белых И.\,Н.&&\\
\hangindent=23pt\noindent\textbf{Лукашенко О.\,В., Морозов Е.\,В.} Асимптотика максимума процесса нагрузки для
неко-\linebreak
\vspace*{-12pt}\\
\hspace*{23pt}торого класса гауссовских очередей$\dotfill$&3&81\\
\hangindent=23pt\noindent\textbf{Маркова Н.\,А.} Логика биографических фактов$\dotfill$&2&87\\
\textbf{Матвеев Ю.\,Н.} см.~Белых И.\,Н.&&\\
\hangindent=23pt\noindent\textbf{Морозов Е.\,В., Некрасова Р.\,С.} Об оценивании вероятности переполнения
конечного\linebreak
\vspace*{-12pt}\\
\hspace*{23pt}буфера в регенеративных системах обслуживания$\dotfill$&3&90\\
\hangindent=23pt\noindent\textbf{Морозов Е.\,В., Румянцев А.\,С.} Вероятностные модели многопроцессорных
систем:\linebreak
\vspace*{-12pt}\\
\hspace*{23pt}стационарность и моментные свойства$\dotfill$&3&99\\
\textbf{Морозов Е.\,В.} см.~Лукашенко О.\,В.&&\\
\hangindent=23pt\noindent\textbf{Мурашов Д.\,М.} Комбинированный подход к локализации различий
многомодальных\linebreak
\vspace*{-12pt}\\
\hspace*{23pt}изображений$\dotfill$&1&122\\
\hangindent=23pt\noindent\textbf{Назаров А.\,Л.} Нижние оценки устойчивости смесей нормальных
распределений к\linebreak
\vspace*{-12pt}\\
\hspace*{23pt}воз\-му\-щениям смешивающих распределений$\dotfill$&4&40\\
\textbf{Некрасова Р.\,С.} см.~Морозов Е.\,В.&&\\
\textbf{Николаев А.\,В.} см.~Жевнерчук Д.\,В.&&\\
\hangindent=23pt\noindent\textbf{Павлов И.\,В.} Расчет и оптимизация некоторых характеристик для модели
вычисли-\linebreak
\vspace*{-12pt}\\
\hspace*{23pt}тельного комплекса$\dotfill$&2&97\\
\textbf{Пагано М.} см.~Де~Никола К.&&\\
\textbf{Паринов С.\,И.} см.~Когаловский М.\,Р.&&\\
\textbf{Пеховский Т.\,С.} см.~Белых И.\,Н.&&\\
\hangindent=23pt\noindent\textbf{Печинкин А.\,В., Соколов И.\,А., Шоргин С.\,Я.} Ограничение на суммарный
объем заявок\linebreak
\vspace*{-12pt}\\
\hspace*{23pt}в дискретной системе Geo$/G/1/\infty$$\dotfill$&3&107\\
\hangindent=23pt\noindent\textbf{Попов С.\,В.} Уточнение неравномерных оценок скорости сходимости в
центральной\linebreak
\vspace*{-12pt}\\
\hspace*{23pt}предельной теореме при существовании моментов не выше второго$\dotfill$&1&32\\
\textbf{Розенберг И.\,Н.} см.~Дулин С.\,К.&&\\
\hangindent=23pt\noindent\textbf{Рудаков К.\,В., Торшин И.\,Ю.} Анализ информативности мотивов на основе
критерия\linebreak
\vspace*{-12pt}\\
\hspace*{23pt}разрешимости в задаче распознавания вторичной структуры белка$\dotfill$&1&79\\
\textbf{Румянцев А.\,С.} см.~Морозов Е.\,В.&&\\
\textbf{Самуйлов А.\,К.} см.~Гайдамака Ю.\,В.&&\\
\textbf{Самуйлов К.\,Е.} см.~Гайдамака Ю.\,В.&&\\
\hangindent=23pt\noindent\textbf{Санг С.\,Х., Будсаба К., Володин А.} Полная сходимость сумм в схеме серий
отрицательно\linebreak
\vspace*{-12pt}\\
\hspace*{23pt}зависимых случайных величин$\dotfill$&4&95\\
\textbf{Сатин Я.\,А.} см.~Зейфман А.\,И.&&\\
\hangindent=23pt\noindent\textbf{Семенов К.\,К.} Нечеткие переменные как способ формализации характеристик
погреш-\linebreak
\vspace*{-12pt}\\
\hspace*{23pt}ности в задачах математической обработки$\dotfill$&2&101\\
\textbf{Сидорова О.\,И.} см.~Де~Никола К.&&\\
\textbf{Сизов А.\,В.} см.~Дюкова Е.\,В.&&\\
\textbf{Симончик К.\,К.} см.~Белых И.\,Н.&&\\
\hangindent=23pt\noindent\textbf{Синицын И.\,Н.} Аналитическое моделирование распределений с инвариантной
мерой\linebreak
\vspace*{-12pt}\\
\hspace*{23pt}в стохастических системах с автокоррелированными шумами$\dotfill$&4&4\\
\hangindent=23pt\noindent\textbf{Синицын И.\,Н.} Математическое обеспечение для анализа нелинейных
много\-ка\-наль-\linebreak
\vspace*{-12pt}\\
\hspace*{23pt}ных круговых стохастических систем, основанное на параметризации
рас\-пре\-де-\linebreak
\vspace*{-12pt}\\
\hspace*{23pt}лений$\dotfill$&1&12\\
\hangindent=23pt\noindent\textbf{Скачков Д.\,М., Жижимов О.\,Л.} Об интеграции географических метаданных
посред-\linebreak
\vspace*{-12pt}\\
\hspace*{23pt}ством ретроспективного тезауруса$\dotfill$&3&43\\
\textbf{Соколов И.\,А., Королев В.\,Ю.} Предисловие$\dotfill$&4&3\\
\textbf{Соколов И.\,А.} см.~Дучицкий И.\,А. &&\\
\textbf{Соколов И.\,А.} см.~Королев В.\,Ю.&&\\
\textbf{Соколов И.\,А.} см.~Печинкин А.\,В.&&\\
\textbf{Соколова Е.\,Г.} см.~Загорулько Ю.\,А.&&\\
\textbf{Сомин Н.\,В.} см.~Кузнецов И.\,П.&&\\
\textbf{Сотнезов Р.\,М.} см.~Дюкова Е.\,В.&&\\
\textbf{Степанов П.\,В.} см.~Бендерина М.\,В.&&\\
\end{tabular}
}

\pagebreak

\def\leftkol{АВТОРСКИЙ УКАЗАТЕЛЬ ЗА 2012 г.} % ENGLISH ABSTRACTS}

\def\rightkol{АВТОРСКИЙ УКАЗАТЕЛЬ ЗА 2012 г.} %ENGLISH ABSTRACTS}

{\tabcolsep=3pt
\begin{tabular}{p{388pt}rr}
&\textbf{Выпуск} & \textbf{Стр.}\\[3pt]
\textbf{Стрижов В.\,В.} см.~Токмакова А.\,А.&&\\
\textbf{Ступников С.\,А.} см.~Калиниченко Л.\,А.&&\\
\textbf{Ступников С.\,А.} см.~Калиниченко Л.\,А.&&\\
\textbf{Сучков А.\,П.} см.~Бендерина М.\,В.&&\\
\textbf{Танажура К.\,А.\,С.} см.~Беляев К.\,П.&&\\
\textbf{Тимонина Е.\,Е.} см.~Грушо А.\,А.&&\\
\hangindent=23pt\noindent\textbf{Токмакова А.\,А., Стрижов В.\,В.} Оценивание гиперпараметров линейных
регрессионных\linebreak
\vspace*{-12pt}\\
\hspace*{23pt}моделей при отборе шумовых и коррелирующих признаков$\dotfill$&4&66\\
\textbf{Торшин И.\,Ю.} см.~Рудаков К.\,В.&&\\
\textbf{Тучкова Н.\,П.} см.~Беляев К.\,П.&&\\
\textbf{Уманский В.\,И.} см.~Дулин С.\,К.&&\\
\hangindent=23pt\noindent\textbf{Ушаков А.\,В.} Анализ системы обслуживания с гиперэкспоненциальным
входящим\linebreak
\vspace*{-12pt}\\
\hspace*{23pt}потоком в условиях критической загрузки$\dotfill$&3&114\\
\hangindent=23pt\noindent\textbf{Ушаков А.\,В.} О виртуальном времени ожидания в системе с относительным
приорите-\linebreak
\vspace*{-12pt}\\
\hspace*{23pt}том и гиперэкпоненциальным входящим потоком$\dotfill$&1&27\\
\hangindent=23pt\noindent\textbf{Ушмаев О.\,С., Кузнецов В.\,В.} Алгоритмы защищенной биометрической
верификации\linebreak
\vspace*{-12pt}\\
\hspace*{23pt}на основе бинарного представления топологии отпечатков пальцев$\dotfill$&1&132\\
\textbf{Френкель С.} см.~Долев Ш.&&\\
\textbf{Хохлов Ю.\,С.} см.~Де~Никола К.&&\\
\textbf{Шарапов Р.\,В.} см.~Шарапова Е.\,В.&&\\
\hangindent=23pt\noindent\textbf{Шарапова Е.\,В., Шарапов Р.\,В.} Универсальная система проверки текстов на
плагиат\linebreak
\vspace*{-12pt}\\
\hspace*{23pt}<<Автор.NET>>$\dotfill$&3&52\\
\hangindent=23pt\noindent\textbf{Шарнин М.\,М., Кузнецов И.\,П.} Особенности семантического поиска
информационных\linebreak
\vspace*{-12pt}\\
\hspace*{23pt}объектов на основе технологии баз знаний$\dotfill$&2&113\\
\hangindent=23pt\noindent\textbf{Шестаков О.\,В.} О скорости сходимости оценки риска пороговой обработки
вейвлет-коэффициентов к нормальному закону при использовании робастных
оценок\linebreak
\vspace*{-12pt}\\
\hspace*{23pt}дисперсии$\dotfill$&2&122\\
\textbf{Шоргин С.\,Я.} см.~Зейфман А.\,И.&&\\
\textbf{Шоргин С.\,Я.} см.~Печинкин А.\,В.&&\\
\textbf{Шулипа А.\,К.} см.~Белых И.\,Н.&&\\
\hangindent=23pt\noindent\textbf{Янушкявичене О.\,Л., Янушкявичюс Р.} О скорости сходимости некоторой
U-ста\-ти\-сти\-ки$\dotfill$&3&64\\
\textbf{Янушкявичюс Р.} см.~Янушкявичене О.\,Л.&&\\
\end{tabular}
}

%\thispagestyle{myheadings}
\def\leftfootline{\small{\textbf{\thepage}
\hfill ИНФОРМАТИКА И ЕЁ ПРИМЕНЕНИЯ\ \ \ том~6\ \ \ выпуск~4\ \ \ 2012}
}%
 \def\rightfootline{\small{ИНФОРМАТИКА И ЕЁ ПРИМЕНЕНИЯ\ \ \ том~6\ \ \ выпуск~4\ \ \ 2012
 \hfill \textbf{\thepage}}}
 \label{end\stat}