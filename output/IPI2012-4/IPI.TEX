\documentclass[10pt]{book}
\usepackage[utf8]{inputenc}

\usepackage{latexsym,amssymb,amsfonts,amsmath,indentfirst,shapepar,%fleqn,%
picinpar,shadow,floatflt,enumerate,multicol,colortbl,ipi}

\usepackage{rotating}
\usepackage{mathrsfs}
\usepackage[noend]{algorithmic}

\input{epsf}

%\nofiles

%\includeonly{obchak,avtor,avtor-eng} %+pdf
%\includeonly{obchak,avtor}
%\includeonly{pred}      %+pdf
%\includeonly{podgot-1str}  %+
%\includeonly{ocherk} %+


%\includeonly{sinits}         %1+pdf+obch+авт
%\includeonly{dulchitskii}    %2+pdf+obch+авт
%\includeonly{konovalov}      %+pdf %3+obch+авт
%\includeonly{zeifman}        %+pdf  %4+obch+авт
%\includeonly{korolev}        %+pdf  %5+obch+авт
%\includeonly{nazarov}        %+pdf  %6+obch+авт
%\includeonly{krivenko}       %7+pdf+obch+авт
%\includeonly{grusho}         %+pdf  %8+obch+авт
%\includeonly{dukova}         %9+pdf+obch+авт
%\includeonly{tokmakova}      %+ 10pdf+obch+авт
%\includeonly{frenkel}        %11+pdf+obch+авт
%\includeonly{belyaev}        %12+pdf+obch+авт
%\includeonly{volodin}        %13+pdf+obch+авт
 


%\includeonly{toc-rus, toc-en}
%\includeonly{obchak} %,toc-en}

%\includeonly{obchak}
%\includeonly{reshal}  %pdf
%\includeonly{eng-index}
%\includeonly{cover3}

\usepackage{acad}
%\usepackage{courier}
\usepackage{decor}
\usepackage{newton}
\usepackage{pragmatica}
\usepackage{zapfchan}
\usepackage{petrotex}
\usepackage{bm}                     % полужирные греческие буквы
\usepackage{upgreek}                % прямые греческие буквы
\usepackage{eufrak}
%\usepackage{verbatim}

\renewcommand{\bottomfraction}{0.99}
\renewcommand{\topfraction}{0.99}
\renewcommand{\textfraction}{0.01}

\setcounter{secnumdepth}{1} %здесь - 3 + chapter = 4

\arraycolsep=1.5pt

%\usepackage[pdftex]{graphicx}

%\usepackage{oz}

%NEW COMMANDS


\renewcommand*{\hm}[1]{#1\nobreak\discretionary{}%
            {\hbox{$\mathsurround=0pt #1$}}{}} %% Дублирует знаки операций
                               %при переносе в формуле (перед знаком, который 
                               %надо продублировать ставится команда \hm)

%\renewcommand{\endproof}{\hfill$\Box$}
\renewcommand{\r}{\mathbb{R}}
\newcommand{\I}{{\rm I\hspace{-0.7mm}I}}
%\newcommand{\Ikl}{{\tt{1}}\hspace*{-1.44mm}\mathtt{1}}
\newcommand{\Ik}{\mbox{{\small \tt {1}}\hspace{-1.5mm}{\tt 1}}}
\newcommand{\argmin}{\mathop{\mathrm{arg}\,\mathrm{min}}}
\newcommand{\argmax}{\mathop{\mathrm{arg}\,\mathrm{max}}}
%\newcommand{\capr}{\mathop{\cap\,}}
%\newcommand{\cupr}{\mathop{\cup\,}}
%\def\argmin{\mathop{arg\,min}}

\def\vrp{\varphi}
\def\prt{\partial}
\def\mm{{\rm M}}

\newcommand{\il}[2]{\int\limits_{#1}^{#2}}%интеграл с пределами #1 и #2


\def\sss{\sum\limits}
\def\tr{\,,\,\ldots\,,\,}
\def\rk{\right]}
\def\lk{\left[}
\def\rf{\right\}}
\def\lf{\left\{}

\def\ee{{\cal E}}
\def\ww{{\cal W}}
\def\yy{{\cal Y}}
\def\vv{{\cal V}}

\newcommand{\R}{\mathbb R}
\newcommand{\N}{\mathbb N}

\newcommand{\h}{{\bf H}}
\newcommand{\p}{{\sf P}}  % вероятность
%\newcommand{\P}{\mathbb{P}}
\newcommand{\e}{{\sf E}}  % мат. ожидание
\newcommand{\D}{{\sf D}}  % дисперсия
\newcommand{\eps}{\varepsilon}
\newcommand{\vp}{{\mathbf p}}
\newcommand{\vz}{{\mathbf z}}
\newcommand{\vx}{{\mathbf x}}
\newcommand{\vf}{{\mathbf f}}
%\newcommand{\vp}{\mathrm{v.p.}}
\newcommand{\F}{{\mathcal F}}
\def\ap{{\mathrm{ЭР}}}

\newcommand{\abs}[1]{\left\vert#1\right\vert}
\def\w{\omega}
\def\W{\Omega}
\def\iii{\int\limits}
\def\iin{\int\limits_{-\infty}^\infty}

\DeclareMathOperator{\sign}{sign}

%\newcommand{\gr}{{\geqslant}}

\newcommand{\g}{\mbox{\textit{g}}}

\renewcommand{\la}{\lambda}
\newcommand{\si}{\sigma}
\newcommand{\alp}{\alpha}

%\newcommand{\pto}{\stackrel{P}{\longrightarrow}} % сходимость по веpоятности

\newcommand{\eqd}{\stackrel{d}{=}} % равенство по pаспpеделению

%\newcommand{\kp}{\kappa}
%\def\Q{{\cal Q}} \def\H{{\cal H}}
%\newcommand{\bet}{\beta_{2+\delta}}


%\newtheorem{definition}{Определение}
%\renewcommand{\thedefinition}{\arabic{definition}.}
%END NEW COMMANDS

%\renewcommand{\baselinestretch}{1.2}

%\pagestyle{myheadings}

\setlength{\textwidth}{167mm}      % 122mm
\setlength{\textheight}{658pt}
%\setlength{\textheight}{635.6pt}
\setlength{\columnsep}{4.5mm}

\setcounter{secnumdepth}{4}

%\addtolength{\headheight}{2pt}
%\addtolength{\headsep}{-2mm}

%\addtolength{\topmargin}{-20mm}  % for printing


\hoffset=-30mm  % From Yap
%\hoffset=-20mm  % From Acrobat

%\voffset=0mm % From Yap
%\voffset=-15mm   % From Acrobat

\addtolength{\evensidemargin}{-9.5mm} % for printing
\addtolength{\oddsidemargin}{9.5mm}  % for printing

%\renewcommand{\thefootnote}{\fnsymbol{footnote}}
%\renewcommand{\thefootnote}{\arabic{footnote}}
\renewcommand{\figurename}{\protect\bf Рис.}
\renewcommand{\tablename}{\protect\bf Таблица}

\newcommand{\Caption}[1]{\caption{\protect\small %\baselineskip=2.5ex
#1}}

\renewcommand{\thefigure}{\arabic{figure}}
\renewcommand{\thetable}{\arabic{table}}
\renewcommand{\theequation}{\arabic{equation}}
\renewcommand{\thesection}{\arabic{section}}

\renewcommand{\contentsname}{СОДЕРЖАНИЕ}
\newcommand{\fr}[2]{\displaystyle\frac{\displaystyle #1\mathstrut}{\displaystyle #2\mathstrut}}

%\renewcommand{\thefootnote}{\fnsymbol{footnote}}
%\newcommand{\g}{\mbox{\textit{g}}}

%\newcommand{\Caption}[1]{\caption{\protect\small\baselineskip=2ex #1}}
\newcounter{razdel}
\setcounter{razdel}{0}


\newcommand{\titel}[4]{%
\

\vspace*{5pt}

\ifodd\therazdel {\raggedright\noindent\Large\textrm\textbf
 \lineskip .75em
  \baselineskip=3.2ex #1 \par}
\vskip 1em {\noindent\large\textrm\textbf #2 \par}
\addcontentsline{toc}{subsection}{{\textrm\textbf #3}\protect\newline #1}
\def\rightheadline{\underline{\noindent\hbox to \textwidth{\hfill\small\textrm{#4}
%\hfill \large\bf\thepage
}}}
\def\leftheadline{\underline{\noindent\parbox{\textwidth}{
%\raggedleft\large\bf\thepage \hfill
\small\textit{#3}\hfill}}}
\def\leftfootline{\small{\textbf{\thepage}
\hfill ИНФОРМАТИКА И ЕЁ ПРИМЕНЕНИЯ\ \ \ том~6\ \ \ выпуск 4\ \ \ 2012}
}%
 \def\rightfootline{\small{ИНФОРМАТИКА И ЕЁ ПРИМЕНЕНИЯ\ \ \ том~6\ \ \ выпуск~4\ \ \ 2012
\hfill \textbf{\thepage}}} 
\vskip 2em \setcounter{figure}{0}
\setcounter{table}{0} 
\setcounter{equation}{0} 
\setcounter{section}{0}
\setcounter{subsection}{0} 
\setcounter{subsubsection}{0}
\setcounter{footnote}{0} 
\setcounter{razdel}{0}
%\end{flushleft}
\else {
 \raggedright\noindent\Large\textrm\textbf
 \lineskip .75em
\baselineskip=3.2ex #1 \par} \vskip 1em
%\begin{flushleft}
{\noindent\large\textrm\textbf #2 \par}
\addcontentsline{toc}{subsection}{{\textrm\textbf #3}\protect\newline #1}
\def\rightheadline{\underline{\noindent\hbox to \textwidth{\hfill\small\textrm{#4}
%\hfill \large\bf\thepage
}}}
\def\leftheadline{\underline{\noindent\parbox{\textwidth}{%\raggedleft\large\bf\thepage \hfill
\small\textit{#3}\hfill}}}
\def\leftfootline{\small{\textbf{\thepage}
\hfill ИНФОРМАТИКА И ЕЁ ПРИМЕНЕНИЯ\ \ \ том~6\ \ \ выпуск~4\ \ \ 2012}
}%
 \def\rightfootline{\small{ИНФОРМАТИКА И ЕЁ ПРИМЕНЕНИЯ\ \ \ том~6\ \ \ выпуск~4\ \ \ 2012
\hfill \textbf{\thepage}}} \vskip 2em \setcounter{figure}{0}
\setcounter{table}{0} \setcounter{equation}{0} \setcounter{section}{0}
\setcounter{subsection}{0} \setcounter{subsubsection}{0}
\setcounter{footnote}{0}
%\end{flushleft}
\fi}

\newcommand{\titelr}[2]{%
\

\vspace*{5pt}

\ifodd\therazdel {\raggedright\noindent\large\textrm\textbf
 \lineskip .75em
  \baselineskip=3.2ex #1 \par}
\vskip 1em {\noindent\normalsize\textrm\textbf #2 \par}
\else {
 \raggedright\noindent\large\textrm\textbf
 \lineskip .75em
\baselineskip=3.2ex #1 \par} \vskip 1em
%\begin{flushleft}
{\noindent\normalsize\textrm\textbf #2 \par}
\fi}

\newcommand{\titele}[5]{%
\

%\vspace*{5pt}

\ifodd\therazdel {\raggedright\noindent%\large
\textrm\textbf
 \lineskip .75em
%  \baselineskip=3.2ex
#1 \par}
\vskip .5em {\noindent\large\textrm\textbf #2 \par}
\vskip .5em
 {\noindent\textrm #3 \par}
\addcontentsline{toc}{subsection}{{\textrm\textbf #1}\protect\newline #2}
\def\rightheadline{\underline{\noindent\hbox to \textwidth{\hfill\small\textrm{#4}
%\hfill \large\bf\thepage
}}}
\def\leftheadline{\underline{\noindent\parbox{\textwidth}{
%\raggedleft\large\bf\thepage \hfill
\small\textrm{#5}\hfill}}}
\def\leftfootline{\small{\textbf{\thepage}
\hfill ИНФОРМАТИКА И ЕЁ ПРИМЕНЕНИЯ\ \ \ том~6\ \ \ выпуск~4\ \ \ 2012}
}%
 \def\rightfootline{\small{ИНФОРМАТИКА И ЕЁ ПРИМЕНЕНИЯ\ \ \ том~6\ \ \ выпуск~4\ \ \ 2012
\hfill \textbf{\thepage}}} \vskip 1em \setcounter{figure}{0}
\setcounter{table}{0} \setcounter{equation}{0} \setcounter{section}{0}
\setcounter{subsection}{0} \setcounter{subsubsection}{0}
\setcounter{footnote}{0} \setcounter{razdel}{0}
%\end{flushleft}
\else {
 \raggedright\noindent%\large
 \textrm\textbf
 \lineskip .75em
%\baselineskip=3.2ex
#1 \par} \vskip .5em
%\begin{flushleft}
{\noindent\large\textrm\textbf #2 \par} \vskip .5em
 {\noindent\textrm #3 \par}
\addcontentsline{toc}{subsection}{{\textrm\textbf #1}\protect\newline #2}
\def\rightheadline{\underline{\noindent\hbox to \textwidth{\hfill\small\textrm{#4}
%\hfill \large\bf\thepage
}}}
\def\leftheadline{\underline{\noindent\parbox{\textwidth}{%\raggedleft\large\bf\thepage \hfill
\small\textrm{#5}\hfill}}}
\def\leftfootline{\small{\textbf{\thepage}
\hfill ИНФОРМАТИКА И ЕЁ ПРИМЕНЕНИЯ\ \ \ том~6\ \ \ выпуск~4\ \ \ 2012}
}%
 \def\rightfootline{\small{ИНФОРМАТИКА И ЕЁ ПРИМЕНЕНИЯ\ \ \ том~6\ \ \ выпуск~4\ \ \ 2012
\hfill \textbf{\thepage}}} \vskip 1em \setcounter{figure}{0}
\setcounter{table}{0} \setcounter{equation}{0} \setcounter{section}{0}
\setcounter{subsection}{0} \setcounter{subsubsection}{0}
\setcounter{footnote}{0}
%\end{flushleft}
\fi}

\def\Abst#1{
\begin{center}\small\nwt
\parbox{150mm}{%\baselineskip=2.5ex
\textbf{Аннотация:}\ \
%\hspace*{\parindent}
#1}
\end{center}}
\def\Abste#1{
\begin{center}\small\nwt
\parbox{150mm}{%\baselineskip=2.5ex
\textbf{Abstract:}\ \
%\hspace*{\parindent}
#1}
\end{center}}

\def\KW#1{
\begin{center}\small\nwt
\parbox{150mm}{%\baselineskip=2.5ex
\textbf{Ключевые слова:}\ \ #1}
\end{center}}

\def\KWE#1{
\begin{center}\small\nwt
\parbox{150mm}{%\baselineskip=2.5ex
\textbf{Keywords:}\ \ #1}
\end{center}}


\def\KWN#1{
%\begin{center}
%\small
%\parbox{150mm}\end{center}
}

\renewcommand{\thesubsection}{\thesection.\arabic{subsection}\hspace*{-5pt}}
\renewcommand{\thesubsubsection}{\thesubsection\hspace*{5pt}.\arabic{subsubsection}\hspace*{-3pt}}

\begin{document}
\Rus

\nwt
%\ptb

%\renewcommand{\contentsname}{\protect\Large\bf Содержание}

\setcounter{tocdepth}{2}

%\tableofcontents

\renewcommand{\bibname}{\protect\rmfamily Литература}
  \def\Au#1{{\it #1}}

%\newcommand{\No}{№}
  \newcommand{\tg}{\,\mathrm{tg}\,}
    \newcommand{\ctg}{\,\mathrm{ctg}\,}
  \newcommand{\arctg}{\,\mathrm{arctg}\,}
  
\def\forallb{\mathop{\forall}}
\def\cupb{\mathop{\cup}}
\def\existsb{\mathop{\exists}}

\setcounter{page}{1}

\newpage
\addtocounter{razdel}{1}
%\def\razd{РЕГУЛИРУЕМЫЙ ЭЛЕКТРОПРИВОД ДЛЯ ЭЛЕКТРОЭНЕРГЕТИКИ}
%\newpage
%\def\stat{zakh}
\def\tit{СРЕДСТВА ОБЕСПЕЧЕНИЯ ОТКАЗОУСТОЙЧИВОСТИ ПРИЛОЖЕНИЙ}
\def\titkol{Средства обеспечения отказоустойчивости приложений}

\def\aut{В.\,Н.~Захаров$^1$, В.\,А.~Козмидиади$^2$}
\titel{\razd}{\tit}{\aut}{\titkol}


\Abst{Рассмотрены проблемы построения отказоустойчивых серверов, возникающие в связи с недетерминированностью поведения приложений. Предложена формальная модель, описывающая поведение приложения, основными объектами которой являются ресурсы и события. Предложены алгоритмы протоколирования работы приложения на резервном узле кластера, а также восстановления и продолжения его работы при отказе основного узла. При этом для клиентов сбой остается незаметным, за исключением некоторого увеличения времени обслуживания.}

\KW{сервер приложений $\bullet$ прозрачная отказоустойчивость $\diamond$
 процесс $\diamond$ ресурс $\diamond$ событие $\diamond$ контрольная точка
$\bullet$ детерминированность}

\vskip 12pt plus 6pt minus 3pt

\begin{multicols}{2}

\section*{ВВЕДЕНИЕ}

Средства вычислительной техники стали использоваться в областях,
требующих безотказной работы систем в течение многих лет (24 часа
в сутки, 365 дней в году).

\label{st\stat}

\footnotetext{$^1$ФГУП Центральный институт авиационного моторостроения
им. П.И. Баранова, Москва, Россия}
\footnotetext{$^2$ФГУП Центральный институт авиационного моторостроения
им. П.И. Баранова, Москва, Россия}

К таким областям относятся, например, центры хранения и обработки данных  в сетях (системы резервирования билетов, биллинговые,  банковские и т.д.), массированные распределенные вычисления (GRID-вычисления) и другие.

\thispagestyle{headings}

Обычно в подобных системах применяются частные решения, ориентированные в основном на обеспечение надежного хранения данных (например, файловые серверы, использующие для хранения RAID-контроллеры) и корректного их состояния при отказах (серверы баз данных с транзакционным выполнением запросов). Однако большинство приложений не гарантируют, что не произойдет потери части данных при отказе системы. Обычно предполагается, что клиентские средства должны повторять запросы после восстановления серверов, для того, чтобы данные не были потеряны, или что можно сделать возврат по времени на некоторое время назад и повторить работу с этого места. Однако далеко не все клиентские средства и условия применения приложений допускают это.

Отказоустойчивые системы для критически важных приложений, корректно решающие проблемы восстановления после сбоев,   предлагаемые ведущими производителями, как правило, дороги. Кроме того, они включают специфические серверные и клиентские приложения, не совместимые со стандартными приложениями, не обеспечивающими отказоустойчивость. Примером такого подхода к решению проблемы отказоустойчивости  хранения данных являются системы NetApp FAS компании Network Appliance, работающие на базе специализированной операционной системы Data ONTAP [1].

Построение отказоустойчивых систем, использующих серверы со стандартными приложениями, в свете вышесказанного, является актуальной проблемой, вызывающей значительный интерес. Рассмотрение методов достижения прозрачной отказоустойчивости таких систем и является предметом статьи.
\begin{figure*} %fig1
\vspace*{1pt}
\begin{center}
\mbox{%
\epsfxsize=1.6in
\epsfxsize=100mm
\epsfbox{BbR-1.eps}
}
\end{center}
\vspace*{-9pt}
\Caption{Базовый вариант трубы с разными выходными устройствами
(цилиндрическое, расширяющееся и сужающееся сопло)
\label{f1bab}}
\vspace*{-3pt}
\end{figure*}


\section{ОСНОВНЫЕ ПОНЯТИЯ И ПОДХОДЫ}

Под сервером в данной работе понимается вычислительный центр
(отдельный компьютер или кластер) в сети, предоставляющий клиентам
(пользователям, клиентским компьютерам) определенные услуги, разделяя
между ними свои ресурсы. Подобные серверы названы серверами приложений.
Широко распространенным примером сервера такого типа является файловый сервер, обеспечивающий удаленный коллективный доступ к файловой системе. Часто используются вычислительные серверы, предоставляющие клиентам возможность выполнять на них свои программы (например, в центрах коллективного пользования).


Обычно приложение представляет собой программу или группу программ, работающих в операционной среде, создаваемой операционной системой (в другой терминологии - один или несколько взаимодействующих процессов или потоков (threads)), которые реализуют функциональность сервера. Для построения отказоустойчивых серверов приложений широко используется кластерная технология. Следуя [2], кластером, названа разновидность параллельной или распределенной системы, которая:
\begin{itemize}
\item состоит из нескольких компьютеров (узлов кластера), связанных как минимум необходимыми коммуникационными каналами;
\item используется как единый, унифицированный компьютерный ресурс.
\end{itemize}

Прозрачная отказоустойчивость (Transparent Fault Tolerance, TFT) сервера приложений - это такое его поведение при возникновении аппаратных или программных отказов либо отказов в сети, при котором:
\begin{itemize}
\item отказ не вызывает потери или искажения данных, находящихся в базе данных сервера;
\item сервер продолжает нормально функционировать, несмотря на имевшие место отказы.
\end{itemize}

Клиенты сервера "не замечают" произошедших отказов. Единственным\footnote{допустимым
отклонением сервера от нормального поведения с точки зрения клиента является
некоторое увеличение времени обслуживания} (на несколько секунд или десятков секунд).

Обычно приложения, работающие на серверах приложений, не ориентированы на прозрачную отказоустойчивость. Они "заботятся" лишь о собственной целостности (например, состояния файловой системы или базы данных). Восстановление работоспособности сервера приводит к разрыву соединений с клиентами и потере их запросов. Это замечают клиенты - запросы следует повторять, на что клиентские приложения далеко не всегда рассчитаны. В данной работе предполагается, что приложения (прикладные программные средства), выполняемые на сервере, являются стандартными, то есть не имеют специальных средств, обеспечивающих отказоустойчивость.
\begin{figure*}[b] %fig1
\vspace*{1pt}
\begin{center}
\mbox{%
\epsfxsize=1.6in
\epsfxsize=100mm
\epsfbox{BbR-1.eps}
}
\end{center}
\vspace*{-9pt}
\Caption{Базовый вариант трубы с разными выходными устройствами
(цилиндрическое, расширяющееся и сужающееся сопло)
\label{f1bab}}
\vspace*{-3pt}
\end{figure*}

Серьезные исследования в области обеспечения отказоустойчивости серверов были развернуты после создания вычислительных серверов, предназначенных для решения задач, требующих больших вычислительных ресурсов. Решение этих задач выполняется на суперкомпьютерах, обеспечивающих массово-параллельные вычисления и представляющих собой кластеры из сотен и тысяч узлов (процессоров). Однако даже на этих "монстрах" решение может требовать десятков или сотен часов, и одиночный сбой, если не предприняты специальные меры, может привести к необходимости начинать работу сначала. Обычно решение вычислительной задачи в таких случаях осуществляется в модели относительно редко взаимодействующих между собой процессов, выполняемых на разных узлах кластера. Эти взаимодействия нужны для координации работы процессов, в частности, для обмена данными и промежуточными результатами. Взаимодействия опираются на специальный протокол, называемый MPI (Message-Passing Interface) и представляющий собой стандарт "de facto" [3].

Для преодоления последствий сбоя достаточно давно была разработана и широко применяется технология, опирающаяся на механизм контрольных точек (checkpoints) [4-6]. По этой технологии система должна иметь стабильную память, которая не меняется при отказах. Соответствующие программные средства периодически сохраняют информацию о состоянии процессов приложения в стабильной памяти. Все процессы также имеют доступ к устройству стабильной памяти.  В случае отказа или сбоя, записанная в стабильную память информация используется для повторения вычисления с момента, когда была записана эта информация, то есть выполняется откат назад по времени. Данные, сохранение которых позволяет выполнить откат, называются контрольной точкой. В качестве устройства стабильной памяти может использоваться дисковый том, энергонезависимая оперативная память, память другого узла или узлов кластера. В последнем случае узел, которому требуется сохранить информацию, пересылает ее через быстрый канал связи на другой узел. Стабильная память после отказа одного из узлов должна быть доступной узлу, на котором делается повтор.

Однако решение, опирающееся только на контрольные точки, не является прозрачным, поскольку не скрывает от клиентов факт отказа системы и требует от них выполнения определенных действий. Так как при работе процессы обмениваются сообщениями, возможны два варианта решения проблемы. Первый - все процессы выполняют записи контрольных точек одновременно, что затруднительно. Второй вариант, при несоблюдении синхронности, - возврат в каждом процессе к такому скоординированному набору контрольных точек, при котором невозможна противоречивая ситуация. Такая ситуация возникает, когда один процесс вернулся к контрольной точке, после которой он должен получить сообщение от другого процесса, а этот другой процесс вернулся к точке, которая следует за выдачей этого сообщения. Однако при повторе ожидаемое первым процессом сообщение не поступит. В этом случае  возможен эффект домино, в результате процессы оказываются отброшены как угодно далеко назад.

В этом состоит первая проблема, которую необходимо преодолеть.

Если нужно, чтобы последствия отказа узла не были видны клиенту,  это означает:
\begin{itemize}
\item клиент не должен терять и потом восстанавливать соединения с сервером;
\item клиент не должен повторять свои запросы;
\item клиент не должен повторно получать сообщения, которые он уже получил.
\end{itemize}

Вторая проблема, которую надо решать, связана с недетерминированностью поведения сервера приложений. Приведем пример.  Пусть имеется система продажи билетов на самолеты. Два клиента одновременно обратились к системе с запросом билета на один и тот же рейс. Клиентам безразлично, какие места им зарезервирует система. Система выполняет запросы клиентов параллельно, поэтому в какой-то момент между процессами, обрабатывающими эти запросы, может возникнуть конкуренция за ресурс - в данном случае, скажем, рейс. Один из процессов захватывает ресурс первым, резервирует место и освобождает ресурс. Потом второй процесс проделывает то же самое.

Порядок, в котором в этом примере процессы захватили ресурс, зависит от многих факторов и, в конечном счете, случаен. Однако  это не мешает правильному функционированию системы, поскольку клиентам важно одно - получить билеты, причем на разные места. Однако отсутствие детерминизма в поведении приложения приводит к тому, что при повторном выполнении могут быть получены другие результаты: например, клиенту уже сообщено, что ему зарезервировано место №5, а при повторе может получиться, что зарезервировано место №6. Система должна устранить это несоответствие и сделать его невидимым для клиента.
\begin{figure*} %fig1
\vspace*{1pt}
\begin{center}
\mbox{%
\epsfxsize=1.6in
\epsfxsize=100mm
\epsfbox{BbR-1.eps}
}
\end{center}
\vspace*{-9pt}
\Caption{Базовый вариант трубы с разными выходными устройствами
(цилиндрическое, расширяющееся и сужающееся сопло)
\label{f1bab}}
\vspace*{-3pt}
\end{figure*}

Недетерминированность поведения системы это следствие, по крайней мере, двух обстоятельств. Во-первых, это присущая системам с разделением времени неопределенность в порядке выполнения процессов. Во-вторых, это конкуренция процессов за общие ресурсы. Перечислим некоторые причины недетерминированного поведения приложений:
\begin{itemize}
\item синхронизация процессов с помощью семафоров или атомарных операций над операндами в общей памяти процессов;
\item зависимость от порядка получения клиентских запросов;
\item время, затраченное процессом на обработку полученного запроса;
\item генераторы случайных чисел;
\item системное управление процессами и потоками;
\item локальные таймеры;
\item доступ к реальному времени.
\end{itemize}

По различным  причинам время, которое тратится на выполнение вычислительной задачи с одними и теми же исходными данными, не является константой, то есть повторное выполнение может дать другое время. Процессы используют общие ресурсы, обращение к которым требует организации очередности выполнения (сериализации) - первым пришел, первым захватил. И, наконец,  результат работы процесса может зависеть от состояния ресурса, а это состояние может изменить другой процесс, ранее захвативший ресурс. Все это создает значительные трудности при попытках воспроизведения поведения процессов с сохраненной контрольной точки.

Прозрачная отказоустойчивость серверов приложений обычно осуществляется переносом приложения на другой узел кластера, идентичный первому по конфигурации аппаратных средств и операционной среды. Это делается методом, называемым snapshot/restore. На основном узле (оригинале)  периодически фиксируется состояние приложения на этом узле кластера (так называемый снимок или snapshot). После отказа оригинала на резервном узле (копии) делается восстановление (restore), то есть восстанавливается последнее зафиксированное состояние приложения. Операционная среда при этом приводится в состояние, которое соответствует моменту изготовления снимка. После этого узел-копия продолжает работу с зафиксированного места. Сравнение метода  snapshot/restore с другими подходами приведено в [7].

Ниже рассматриваются информационные  технологии, позволяющие решить ряд принципиальных вопросов, связанных с реализацией прозрачной отказоустойчивости серверов приложений. Ими являются:
\begin{itemize}
\item виртуализация операционной среды, в которой работает серверное приложение;
\item отказоустойчивая реализация протокола TCP;
\item создание контрольных точек состояния приложения и файловой системы, которые делаются внешним по отношению к приложению образом;
\item восстановление серверного приложения на основании контрольной точки.
\end{itemize}
\begin{figure*} %fig1
\vspace*{1pt}
\begin{center}
\mbox{%
\epsfxsize=1.6in
\epsfxsize=100mm
\epsfbox{BbR-1.eps}
}
\end{center}
\vspace*{-9pt}
\Caption{Базовый вариант трубы с разными выходными устройствами
(цилиндрическое, расширяющееся и сужающееся сопло)
\label{f1bab}}
\vspace*{-3pt}
\end{figure*}

\section{МОДЕЛЬ ОПИСАНИЯ ПОВЕДЕНИЯ ПРИЛОЖЕНИЯ}

Предлагаемый подход опирается на построение модели вычислений, связанной с использованием понятия времени в многопроцессных приложениях. Впервые подобные проблемы были изучены в классической работе Л. Лампорта [8].

Многопроцессными приложения называются потому, что в них параллельно работают несколько процессов. Процесс ведет себя детерминированно, пока в предписанном кодом порядке выполняет процессорные инструкции. Конечно, его работа может быть прервана практически в любой момент и процессор передан другому процессу или ядру. Поэтому абсолютное время, которое затрачивает процесс на выполнение определенной работы, не  константа, а случайная  величина. То же  относится к относительному времени, то есть времени, которое процесс занимал процессор,  поскольку одни и те же обращения к операционной среде могут вызвать работы разной длительности, а значит потребовать разное время на свое выполнение.

Кэшированность инструкций и данных, а также длина хэш-списков влияют на действительное время пребывания в операционной среде. Утрачивает смысл понятие одновременность действий, поскольку  нельзя установить, выполнили ли два разных процесса какие-либо действия одновременно или одно из них предшествовало другому. Таким образом, с процессом можно связать только его локальное время, которое линейно упорядочивает события,  происходившие в этом процессе.  Глобальное время, линейно упорядочивающее действия во всех процессах, отсутствует. Расстояние (в этом качестве используется время) между действиями оказывается случайной величиной.

Эти соображения важны, поскольку процессы в интересующих нас приложениях взаимодействуют и используют общие ресурсы. Для взаимодействия они используют средства синхронизации, предоставляемые операционной средой - например, наборы семафоров SVR4 (System V Release 4), POSIX-семафоры, бинарные семафоры и другие примитивы взаимного исключения (POSIX- mutual exclusion locks) и т.д. Подобные средства операционной среды, которые позволяют процессам синхронизировать свою деятельность друг с другом или сериализовать обращения к совместно используемым объектам,  будут ниже  называться ресурсами.

С каждым ресурсом связано свое локальное время, линейно упорядочивающее события в жизни ресурса. Например, в случае двоичных семафоров это создание семафора, а также его захват и освобождение процессом. Заметим, что событие - это не намерение процесса (например, захватить бинарный семафор), а сам факт захвата семафора процессом (т.е. успешное выполнение намерения). От изъявления намерения до его осуществления может многое произойти. Например, семафор, который хочет захватить рассматриваемый процесс, принадлежал другому процессу, потом тот процесс его освободил, но семафор был сначала передан операционной средой третьему процессу, который также на него претендовал, и т.д. Поведение рассматриваемого процесса в это время нас не интересует - он ресурсом еще не овладел, а только его захват определяет его дальнейшее поведение. По причинам,  изложенным выше, расстояние между двумя событиями - случайная величина. Однако, события замечательны тем, что они одновременно присутствуют и в локальном времени процесса, и в локальном времени ресурса. Поэтому все, что произошло в истории процесса или/и ресурса до этого события, предшествует ему. Далее  будет считаться, что истории и ресурсов и процессов состоят только из событий, причем между двумя последовательными событиями в жизни процесса последний ведет себя детерминированно.

Это означает, что на  поведении процесса сказывается только его предыдущая история, то есть состояние ресурсов, с которыми он взаимодействовал. Это свойство процессов ниже будет называться локальной детерминированностью. Этим свойством не обладают ресурсы, поскольку - следующее событие в истории ресурса не определяется однозначно по его предыдущей истории. Утверждение, касающееся детерминированного поведения процессов, неявно опирается на предположение,  что учтены все ресурсы, которые могут привести к  недетерминированности процессов.

Таким образом, описанное нами очень неформально время в многопроцессном комплексе представляет собой отношение частичного порядка, введенное на множестве событий. Зная полное состояние комплекса в некоторый момент времени,  нельзя однозначно определить, какое событие в истории ресурса наступит следующим. Можно говорить только о вероятности наступления того или иного события. Недетерминированность поведения есть следствие двух обстоятельств. Во-первых, это неопределенность времени, которое тратит процесс на переход от одного события к другому. Во-вторых, конкуренция процессов за общие ресурсы.

Выполнение приложения, на множестве событий которого введена частичная упорядоченность, можно описать направленным ациклическим графом выполнения. Вершинами этого графа являются события, с каждым  из которых связаны две входящие в него дуги. Одна дуга начинается в событии, которое непосредственно предшествует данному событию в истории процесса, другая - в истории ресурса.

Построение средств обеспечения прозрачной отказоустойчивости приложений опирается на следующее утверждение: для восстановления работы приложения после отказа достаточно располагать:
\begin{itemize}
\item контрольной точкой, которая отражает на некоторый момент времени состояния процессов и других ресурсов, образующих приложение;
\item графом выполнения приложения, который описывает работу приложения, начинающуюся с контрольной точки и заканчивающуюся отказом. Данные, которые нужны для построения графа выполнения, далее называются протоколом.
\end{itemize}
\begin{figure*} %fig1
\vspace*{1pt}
\begin{center}
\mbox{%
\epsfxsize=1.6in
\epsfxsize=100mm
\epsfbox{BbR-1.eps}
}
\end{center}
\vspace*{-9pt}
\Caption{Базовый вариант трубы с разными выходными устройствами
(цилиндрическое, расширяющееся и сужающееся сопло)
\label{f1bab}}
\vspace*{-3pt}
\end{figure*}

Вся эта информация должна находиться в стабильной памяти, не разрушающейся при отказе.

Ниже неформально описан алгоритм восстановления работы приложения после отказа, который опирается на наличие контрольной точки и графа выполнения. Будем считать, что в распоряжении имеются средства, позволяющие остановить процесс в тот момент, когда он намерен совершить некоторую операцию над ресурсом. Заметим, что событие в графе выполнения соответствует не изъявлению намерения, а его удовлетворению, то есть завершению выполнения операции.

Предварительно сделаем следующее:
\begin{itemize}
\item используя контрольную точку, приведем приложение в состояние, соответствующее этой контрольной точке;
\item в графе выполнения пометим все вершины (события) как "не наступившие". У некоторых вершин графа отсутствуют им непосредственно предшествующие; соответствующие события наступили сразу же после создания контрольной точки. Для каждой такой вершины включим в граф дополнительную вершину, ей предшествующую в истории процесса, и отметим эту дополнительную вершину как "наступившую";
\item разрешим процессам приложения выполняться.
\end{itemize}

Пусть некоторый процесс проявляет намерение выполнить операцию над каким-либо ресурсом. Отыщем для этого процесса в его истории последнее наступившее событие. Следующее в его истории событие - это то, которое соответствует требуемой операции. Посмотрим, наступило ли событие в истории ресурса, которое ему предшествует. Если нет, переведем процесс в состояния ожидания, отметив в предшествующем событии, что данный процесс ожидает его наступления. Если да, разрешим процессу выполняться, то есть выполнить операцию над ресурсом.

Пусть некоторый процесс объявляет, что он выполнил операцию над каким-либо ресурсом (это соответствует моменту протоколирования при оригинальном выполнении). Отыщем для этого процесса в его истории последнее наступившее событие и перейдем к следующему событию в его истории. Это опять то событие, которое мы рассматриваем. Отметим его как "наступившее". Если наступления этого события ожидал какой-нибудь процесс, выведем этот процесс из состояния ожидания. Наконец, разрешим процессу, выполнившему операцию, продолжаться дальше.

Когда выясняется, что наступили все события графа выполнения, повторное выполнение считается законченным.

Естественным следствием из сказанного является следующее утверждение: для того, чтобы размер протокола не рос неограниченно, нужно периодически создавать контрольные точки, очищая при этом протокол.

\section{ФОРМАЛЬНОЕ ОПИСАНИЕ МОДЕЛИ ПОВЕДЕНИЯ МНОГОПРОЦЕССНОГО ПРИЛОЖЕНИЯ}
\begin{figure*} %fig1
\vspace*{1pt}
\begin{center}
\mbox{%
\epsfxsize=1.6in
\epsfxsize=100mm
\epsfbox{BbR-1.eps}
}
\end{center}
\vspace*{-9pt}
\Caption{Базовый вариант трубы с разными выходными устройствами
(цилиндрическое, расширяющееся и сужающееся сопло)
\label{f1bab}}
\vspace*{-3pt}
\end{figure*}

Опишем формально поведение приложения, неформальное описание которого было приведено выше. Рассматриваются два типа объектов:
\begin{itemize}
\item ресурсы (r), например, наборы семафоров (POSIX- или SVR4-семафоры), бинарные семафоры (POSIX-mutex's), таймер реального времени, сокеты (sockets), то есть двусторонние виртуальные соединения с внешним миром;
\item процессы (p), например, процессы или потоки (threads) пользователя.
\end{itemize}

\end{multicols}

\label{end\stat}

%\def\stat{batr}

\def\tit{НОВЫЙ МЕТОД ВЕРОЯТНОСТНО-СТАТИСТИЧЕСКОГО\newline
АНАЛИЗА ИНФОРМАЦИОННЫХ ПОТОКОВ
В~ТЕЛЕКОММУНИКАЦИОННЫХ СЕТЯХ$^*$}
\def\titkol{Новый метод вероятностно-статистического
анализа информационных потоков
в~телекоммуникационных сетях}
\def\autkol{Д.\,А.~Батракова, В.\,Ю.~Королев, С.\,Я.~Шоргин}
\def\aut{Д.\,А.~Батракова$^1$, В.\,Ю.~Королев$^2$, С.\,Я.~Шоргин$^3$}

\titel{\tit}{\aut}{\autkol}{\titkol}

{\renewcommand{\thefootnote}{\fnsymbol{footnote}}\footnotetext[1]{Работа 
выполнена при поддержке РФФИ, проекты №№\,04-01-00671, 05-07-90103.} 
\renewcommand{\thefootnote}{\arabic{footnote}}}
 \footnotetext[1]{ИПИ РАН, 
daria.batrakova@gmail.com} \footnotetext[2]{Факультет вычислительной математики 
и кибернетики МГУ им.~М.\,В.~Ломоносова, ИПИ РАН, vkorolev@comtv.ru} 
\footnotetext[3]{ИПИ РАН, sshorgin@ipiran.ru}



\Abst{В данной работе предлагается метод исследования стохастической структуры
хаотических информационных потоков в сложных телекоммуникационных
сетях. Предлагаемый метод основан на стохастической модели
телекоммуникационной сети, в рамках которой она представляется в виде
суперпозиции некоторых простых последовательно-параллельных структур.
Эта модель естественно порождает смеси гамма-распределений для времени
выполнения (обработки) запроса сетью. Параметры получаемой смеси
гамма-распределений характеризуют стохастическую структуру
информационных потоков в сети. Для решения задачи статистического
оценивания параметров смесей экспоненциальных и гамма-распределений
(задачи разделения смесей) используется ЕМ-алгоритм. Чтобы проследить
изменение стохастической структуры информационных потоков во времени,
ЕМ-алгоритм применяется в режиме скользящего окна. Описывается
программный инструментарий для применения полученного решения к
реальным статистическим данным. Приводится интерпретация результатов.}

\KW{телекоммуникационные сети; информационные потоки;
разделение смесей  распределений;
метод скользящего окна;  программа для разделения смесей}

\vskip 24pt plus 9pt minus 6pt

\thispagestyle{headings}

\begin{multicols}{2}


\label{st\stat}

\section{Введение}

Развитие телекоммуникационных сетей, их усложнение поставило перед
инженерами важную прикладную задачу исследования характеристик
информационных потоков, возникающих в этих сетях. Здесь под
информационным потоком мы будем понимать упорядоченное движение
любого вида информации по сети.

Если на заре эры телекоммуникаций, в эпоху первых телефонных линий и
телеграфа эта проблема не была столь насущной, то со временем, при
постепенном охвате мирового пространства сетями возникла необходимость в
построении и исследовании адекватных моделей сетей и процессов,
происходящих в них.

\thispagestyle{headings}


Современные сети~--- это \textit{конвергентные} сети, т.е.\ совокупность крайне
разнородных как по топологии, так и по физической архитектуре сетей, которые
предлагают конечному пользователю самые разнообразные сервисы. Это~--- огромное
виртуальное и физическое пространство, состоящее из миллионов процессоров,
операционных платформ, линий передачи данных и стыковочных узлов.
%
Существует множество классификаций телекоммуникационных сетей по различным
признакам:
\begin{itemize}
\item масштабу (локальные сети~--- LAN, масштаба города~---
MAN, широкого масштаба~--- WAN);
\item топологии, или логической организации (<<звезда>>,
<<кольцо>>, <<шина>>);
\item физической организации (оптические сети, радио);
\item предлагаемым услугам (сотовые сети, для доступа в
Интернет);
\item назначению (военные, гражданские) и~др.
\end{itemize}


Конвергентная сеть входит во все эти классы, причем, как правило,
обладает всеми этими признаками. Поэтому построение модели для ее анализа
является и очень важной, и очень сложной задачей.

Существуют достаточно многочисленные математические методы, ориентированные на
моделирование и анализ телекоммуникационных сетей. В~большинстве своем они
основываются на теории массового обслуживания, то есть разделе теории
вероятностей, посвященном описанию функционирования сложных систем обслуживания
(в том чис\-ле телекоммуникационных сетей и систем) с помощью стохастических
процессов особого вида и анализу таких процессов. Указанная теория является
весьма развитой и широко применяется на практике. Тем не менее, ее применимость
ограничена~--- во-первых, все возрастающей сложностью структур и дисциплин
обслуживания в рас\-смат\-ри\-ва\-емых реальных сетях. Эта сложность во многих
случаях принципиально не может найти адекватного отображения в моделях
массового обслуживания, даже несмотря на постоянно растущую сложность самих
этих моделей. В результате даже модели, допускающие точный математический
анализ, дают возможность расчета всего лишь приближенных значений характеристик
реальных сетей, ибо предположения, принимаемые при построении моделей, во
многих случаях не соответствуют практике. Во-вторых, для описания
телекоммуникационной сети в виде сети массового обслуживания исследователь
должен располагать детальным описанием структуры сети, что далеко не всегда
имеет мес\-то на практике. В-третьих, разработано крайне мало моделей массового
обслуживания, в которых используется в качестве входной информация о
наблюдаемых (статистических) показателях функционирования сети; в то же время,
такая информация очень часто доступна исследователю, и ее использование при
анализе сети весьма целесообразно.

В данной работе предлагается в определенной степени альтернативный подход к
моделированию сложных телекоммуникационных сетей. Строится и исследуется
вероятностная модель сложной телекоммуникационной сети как суперпозиции
достаточно простых структур. При этом практически никакая априорная информация
о структуре исследуемой сети не используется~--- наоборот, в результате
исследования модели исследователь получает приближенное представление об этой
структуре. Характеристики типовых простых структур, составляющих в совокупности
модель сети, и сети в целом при этом подходе описываются
гам\-ма-рас\-пре\-де\-ле\-ни\-я\-ми. Ставится задача оценки параметров модели
на основе статистических данных о функционировании сети, а также предлагается
математическое решение этой задачи. В статье описан также созданный на основе
разработанных математических методов программный инструментарий и приведены
результаты расчетов для реального трафика. {\looseness=-1

}

\section{Математическая модель и~постановка задачи}

\subsection{Логическая модель сети}
 %1.1

Рассмотрим абстрактную \textit{конвергентную телекоммуникационную
сеть}. Это может быть как крупномасштабная транспортная сеть (WAN), сеть
отдельного оператора масштаба города (MAN) с различными сервисами, так и
локальная сеть (LAN).

Любой из этих случаев можно описать как ($p,\,q$)-\textit{сеть}.

\medskip
\textbf{Определение 1.} В теории графов и сетей под ($p,\,q)$-сетью понимается
набор вида $S =$\linebreak $=(G,\,V^\prime ,\,V^{\prime\prime})$, где $G$~---
граф, а $V^\prime$ и $V^{\prime\prime}$~--- выборки из множества $V(G)$ (вершин
графа) длины~$p$ и $q$ соответственно. При этом выборка $V^\prime$
($V^{\prime\prime}$) считается \textit{входной} (\textit{выходной}) выборкой, а
ее $i$-я вершина называется $i$-\textit{м} \textit{входным} (\textit{выходным})
\textit{полюсом} или, иначе, $i$-\textit{м} \textit{входом} (\textit{выходом})
сети~$S$. Вершины, не участвующие во входной и выходной выборках сети,
считаются ее внутренними вершинами~\cite{1bat}.

Сеть $S$ (рис.~\ref{f1bat}) имеет $p$ точек входа~--- точек соединения
с внешней средой (это могут быть точки стыковки разнородных сетей, сетей
различных операторов, физические подключения к интерфейсам
маршрутизаторов и~т.п.). Под \textit{внешней средой} будем понимать другие
сети, которые передают данные в сеть~$S$. Отдельные <<единицы>> данных
(кадры, сообщения, датаграммы, пакеты) поступают на входы сети~$S$,
обрабатываются и подаются на каждый из $q$ выходов, которые могут быть
соединены как с конечными пользователями, так и с другими сетями.
\begin{figure*} %fig1
\vspace*{1pt}
\begin{center}
\mbox{%
\epsfxsize=139.7mm \epsfbox{bat-1.eps}
%\epsfxsize=139.698mm
%\epsfbox{bek-3.eps}
}
\end{center}
\vspace*{-9pt} \Caption{Абстрактная телекоммуникационная сеть \label{f1bat}}
\end{figure*}

Следует отметить, что структура сложных телекоммуникационных сетей обладает
свойством некоторого самоподобия, т.е.\ на каком бы уровне сетевой архитектуры
мы ни рассматривали поведение информационных потоков, мы можем выделить
некоторые базовые структуры, субпотоки, суперпозицией которых мы можем получить
модель конкретной сети, какой бы уровень <<детализации>> сегментов сети мы ни
взяли. Так, например, физические подключения к интерфейсам оптического
кросс-коннекта в этом смысле подобны <<виртуальным>> подключениям к портам TCP
на сервере приложений.

Итак, независимо от уровня сетевой архитектуры мы можем
рассматривать некоторую величину, характеризующую количество каких-либо
ресурсов сети~$S$, занимаемых в процессе передачи и обработки данных.  Это
могут быть ресурсы, относящиеся как к <<объему>> (памяти сетевого
устройства, количеству занятых линий, размеру пакета), так и ко <<времени>>
(времени обслуживания заявки, времени простоя). Эта величина случайна, т.к.\
мы не можем абсолютно точно сказать для сложной телекоммуникационной
сети, какое сообщение на какой из входов поступит и какого размера оно будет.
Таким образом, случайный характер данной величины определяется
случайностью поведения внешней среды.

Пусть $R$~--- это описанная выше случайная величина, $R>0$. Далее, не
ограничивая общности, будем подразумевать под ней время, необходимое для
какой-либо операции сети (обработки <<единицы>> данных), предполагая, что
время обработки прямо зависит от объема сообщения.

\subsection{Вероятностная модель сети} %1.2.

Даже не зная реальной топологии сети, мы можем описать
функционирование некоторых ее участков как процесс выполнения операций
(задач сети) в последовательном  порядке (например, если доступен только
один канал данных) или как процесс одновременного выполнения субопераций
(когда доступно более одного пути выполнения). Это значит, что мы можем
представить функционирование сложной телекоммуникационной сети как
\textit{суперпозицию} таких <<последовательных>> и <<параллельных>>
блоков.

Для построения вероятностной модели распределения~$R$ используется
комбинация асимптотического подхода, основанного на предельных теоремах
теории вероятностей, и принципа максимальной неопределенности (энтропии).

Рассмотрим следующую модель. Предположим, что мы можем разделить
сеть~$S$ на несколько сегментов $S_i$. Пусть $T$~--- случайная величина,
время выполнения операции в отдельно взятом блоке $S_i$ (сегменте сети).

Если операции выполняются \textit{параллельно}, то время, необходимое
для их выполнения~--- это максимальное время, затрачиваемое на какую-либо
субоперацию:
$$
T = \underset{i}{\max}\, T_i\,,
$$
где $T_i$~--- случайные величины для со\-от\-вет\-ст\-ву\-ющих субопераций.
Модель такого выполнения пред\-став\-ле\-на на рис.~\ref{f2bat}.

\begin{figure*} %fig2
\vspace*{1pt}
\begin{center}
\mbox{%
\epsfxsize=117.271mm
\epsfbox{bat-2.eps}
}
\end{center}
\vspace*{-9pt}
\Caption{Параллельное выполнение
\label{f2bat}}
\end{figure*}

Известно, что предельное распределение экстремальных значений для
выборок ~--- это экспоненциальное распределение с плотностью~\cite{2bat}
$$
f(x) =
\begin{cases}
\lambda e^{-\lambda x}\,, & x>0\,,\\
0\,, & x\leq 0\,,
\end{cases}
$$
где $\lambda >0$~--- параметр масштаба.

 Учитывая также энтропийный подход, естественно будет считать
распределение $T$ экспоненциальным, т.к.\ экспоненциальное распределение
обладает наибольшей энтропией среди всех распределений с $x>0$.

Если же операции сети выполняются \textit{последовательно}, то величина
$T$~--- это сумма времен $T_i$, необходимых для выполнения каждой
субоперации:
$$
T = \sum\limits_i T_i\,,
$$
где $T_i$~--- случайные величины для со\-от\-вет\-ст\-ву\-ющих субопераций.
%
Такая модель представлена на рис.~\ref{f3bat}.

\begin{figure*} %fig3
\vspace*{1pt}
\begin{center}
\mbox{%
\epsfxsize=139.592mm
\epsfbox{bat-3.eps}
}
\end{center}
\vspace*{-9pt}
\Caption{Последовательное  выполнение
\label{f3bat}}
\end{figure*}

Это значит, что распределение общей длительности $T$ выполнения
блока~--- это свертка распределений <<элементарных>> времен $T_i$
(экспоненциально распределенных).

Известно, что результатом свертки экспоненциальных распределений
является гамма-распределение, определяемое плотностью
$$
\g_{\lambda , \alpha} (x) =
\begin{cases}
\fr{\lambda_0^{\alpha_0}}{\Gamma (\alpha_0 )}\,x^{\alpha_0-1}
e^{\lambda_0 x}\,, & x>0\,,\\
0,\, & x\leq 0\,,
\end{cases}
$$
где $\alpha >0$~--- параметр формы,  $\lambda >0$  параметр масштаба, а
$\Gamma (z)$~--- гамма-функция Эйлера:
$$
\Gamma (z) = \int\limits_0^\infty x^{z-1} e^{-x}\,dx\,.
$$

\begin{figure*} %fig4
\vspace*{1pt}
\begin{center}
\mbox{%
\epsfxsize=120.831mm
\epsfbox{bat-4.eps}
}
\end{center}
\vspace*{-9pt}
\Caption{Модель пути  обработки сообщения сетью~$S$
\label{f4bat}}
\end{figure*}

Известно~\cite{2bat}, что класс гамма-распределений замкнут над операцией
свертки, поэтому ре\-зуль\-ти\-ру\-ющее распределение случайной величины~$R$
будет также гамма-распределением
$$
\g_{\lambda , \alpha} (x) =
\begin{cases}
\fr{\lambda^{\alpha}}{\Gamma (\alpha )}\,x^{\alpha -1} e^{-\lambda x}\,, &
x>0\,,\\
0,\, & x\leq 0\,.
\end{cases}
$$

В силу случайного характера ввода данных в сеть~$S$ из внешней среды маршрут
передачи данных становится случайным, что представлено на рис.~\ref{f4bat}. Это
означает, что параметры ре\-зуль\-ти\-ру\-юще\-го распределения~$R$ тоже
случайны. Отсюда имеем следующую модель \textit{смеси
гам\-ма-рас\-пре\-де\-ле\-ний}, определяемой плотностью

\begin{equation} %1
p(x) = \iint \g_{\lambda , \alpha}(x)\,dH (\lambda ,\,\alpha )\,,
\end{equation}
где $H(\lambda , \alpha )$~--- смешивающая функция, функция распределения
входных параметров.

Поясним понятие \textit{смеси распределений}.

\medskip
\textbf{Определение~2.} Пусть имеется двух\-па\-ра\-мет\-ри\-че\-ское
семейство $n$-мерных плотностей  распределения
\begin{equation}
F = \{ f_\omega (x;\, \theta (\omega ))\}\,,
\end{equation}
где одномерный (целочисленный или непрерывный) параметр $\omega$ в
качестве нижнего индекса функции $f$ определяет специфику общего вида
каж\-до\-го компонента~--- распределения смеси, а в качестве аргумента при
многомерном, вообще говоря, параметре $\theta$ определяет зависимость
значений хотя бы части компонентов этого параметра от того, в каком именно
составляющем распределении $f_\omega$ он присутствует. Кроме того, пусть
$P = \{P(\omega )\}$~--- \textit{семейство смешивающих функций}
распределения.

Функция плотности распределения
\begin{equation}
f(x) = \int f_\omega (x;\,\theta(\omega ))\,dP (\omega )
\end{equation}
называется $P$-\textit{смесью} (или просто \textit{смесью})
\textit{распределений} семейства~$F$,  интеграл в~(3) понимается в смысле
Лебега--Стильтьеса~\cite{3bat}.

\medskip
\textbf{Определение 3.} Семейство смесей~(3) называется
\textit{идентифицируемым} (\textit{различимым}), если из равенства
$$
\int f_\omega (x;\,\theta(\omega ))\,dP (\omega ) =\int f_\omega
(x,\,\theta(\omega )) dP^*(\omega )
$$
следует, что $P(\omega ) \equiv P^*(\omega )$ для всех $P \in P(\omega
)$~\cite{3bat}.

\subsection{Постановка задачи} %1.3.

Перед нами встает задача \textit{разделения} такой смеси. Вообще говоря,
задача разделения смесей распределений со смешивающими функциями
общего вида является \textit{некорректно поставленной}, т.к.\ она допускает
существование нескольких решений. Поэтому будем искать решение в классе
\textit{конечных идентифицируемых смесей распределений}, где смешивающая
функция дискретна.

Для этого сузим данное выше определение и будем рассматривать в дальнейшем лишь 
случай конечного числа $k$ возможных значений па\-ра\-мет\-ра~$\omega$, что 
соответствует конечному числу скачков смешивающих функций $P(\omega )$.  
Величины этих скачков как раз и будут играть роль \textit{удельных весов} 
(\textit{априорных вероятностей}) $p_j$ компонентов смеси. Более того, в нашем 
случае мы постулируем также однотипность компонентов распределений $f_j$, т.е.\ 
принадлежность всех $f_j$ к одному общему па\-ра\-мет\-ри\-че\-ско\-му 
семейству $\{ f(X;\,\theta )\}$, где $\theta$~--- многомерный, вообще говоря, 
параметр. Так что~(3) в этом случае может быть записано в виде
\begin{equation} %4
p(x) = \sum\limits^k_{j=1} p_j f_j (x;\,\theta_j )\,.
\end{equation}

Переформулируем понятие идентифицируемости (различимости) смесей
специально применительно к такому виду смесей.

\medskip
\textbf{Определение 4.} \textit{Конечная смесь}~(3) называется
\textit{идентифицируемой} (\textit{различимой}), если из равенства
$$
\sum\limits_{j=1}^k p_j f_j (x;\,\theta_j ) = \sum\limits_{l=1}^{k^*} p_l^* f_l
(x;\,\theta_l^* )
$$
следует, что $k=k^*$ и для любого $j$ ($1\leq j \leq k$) найдется такое $l$ 
($1\leq l \leq k^*$), что $p_j = p_l^*$ и $f_j (x;\,\theta_j ) = f_l 
(x;\,\theta_l^* )$~\cite{3bat}.

Решить эту задачу в выборочном варианте~--- значит по выборке
классифицируемых наблюдений
$X_1,\ldots , X_n, $ извлеченной из генеральной совокупности, яв\-ля\-ющей\-ся смесью~(3)
генеральных совокупностей типа~(2) (при заданном общем виде составляющих
смесь функций $f_j (x;\,\theta_j )$), построить статистические оценки для числа
компонентов смеси~$k$, их удельных весов $p_j$ и, главное, для каждого из
компонентов %f_j (x;\,\theta_j )$ анализируемой смеси. Далее будет считать, что
функции $f_j$ однозначно определяются своими параметрами $\theta_j$: $f_j
=f(x;\,\theta_j)$.

Однако не следует ставить знак тождества между задачей разделения смеси
и задачей статистического оценивания параметров в модели~(4) по выборке $
X_1,\ldots , X_n$, поскольку задача разделения сохраняет смысл и
применительно к генеральным совокупностям, т.е.\ в теоретическом
варианте~\cite{3bat}.

Итак, для статистического анализа на основе реальных данных мы
аппроксимируем нашу общую модель~(1) следующей:
$$
p(x) \approx \hat{p}(x) = \sum\limits_{j=1}^k p_j \g_{\lambda_j , \alpha_j}
(x)\,,
$$
где $p_j$~--- дискретные смешивающие параметры, $\g_{\lambda_j , \alpha_j}
(x)$~--- плотности гамма-распределений.

Такая аппроксимация не только позволяет решить поставленную статистическую
задачу, но и полу\-чить наглядную визуализацию результатов. Существуют
достаточно эффективные методики разделения смесей распределений, среди них~---
семейство так называемых \textit{ЕМ-алгоритмов}
(\textit{Expectation-Maximization Algorithms}).

Полученные результаты могут быть достаточно просто интерпретированы. Число
компонентов смеси символизирует число типичных параллельных или
последовательных структур. Значения параметров составляющих смесь
гам\-ма-рас\-пре\-де\-ле\-ний показывают <<степень параллелизма>>
соответствующей структуры: чем ближе параметр формы к~1, тем выше эта
<<степень>>. И наоборот, чем дальше значение параметра формы от~1, тем больше
последовательных операций выполняется в соответствующем блоке.

Веса компонентов характеризуют примерную долю использования
ресурсов для сообщений, соответствующих каждому распределению входных
данных.

Итак, предложенный подход позволяет получить представление о
стохастической структуре телекоммуникационной сети.

\section{ЕМ-алгоритм разделения смесей распределений}

\subsection{Описание алгоритма} %2.1.

Определяемый ниже итерационный алгоритм решения поставленной в
предыдущем разделе задачи относится к процедурам, базирующимся на
\textit{методе максимального правдоподобия}.

Этот алгоритм позволяет находить максимум логарифмической функции
правдоподобия по параметрам $p_1,\,p_2,\ldots ,\,p_k$, $\theta_1 ,\,\theta_2,\ldots ,\,
\theta_k$ при фиксированном $k$ по выборке $X_1, \ldots , X_n$, т.е.\ решение
оптимизационной задачи вида

\begin{equation} \sum\limits_{i=1}^n \ln \left ( \sum\limits_{j=1}^k p_j f_j
(X_i;\,\theta_j )\right ) \rightarrow \underset{p_j,\,\theta_j}{\max}\,.
\end{equation}

Конкретные алгоритмы, построенные по этой схеме, часто называют
\textit{алгоритмами типа ЕМ}, поскольку в каждом из них можно выделить два
этапа, находящихся по отношению друг к другу в последовательности
итерационного взаимодействия: \textit{оценивание} (\textit{Estimation}) и
\textit{максимизация} (\textit{Maximization})~\cite{4bat}.

Введем в рассмотрение так называемые апостериорные вероятности
$\g_{ij}$ принадлежности наблюдения $X_i$ к $j$-му классу:
\begin{equation} %6
\g_{ij} = \fr{p_j f(X_i;\,\theta_j )}{\sum\limits_{l=1}^k p_l f(X_i;\,\theta_l 
)} \ (i=1,\ldots , n;\ j=1,\ldots ,k)\,.\!\!\end{equation} 
Очевидно, что для 
всех $i=1,\ldots ,n$ и $j=1,\ldots ,k$
$$
\g_{ij} \geq 0,\quad \sum_{j=1}^k \g_{ij} =1\,.
$$


Далее обозначим $\Theta = (p_1,\ldots p_k,\,\theta_1,\ldots ,\theta_k )$ и
представим анализируемую логарифмическую функцию правдоподобия
$$
\ln L(\Theta ) = \sum\limits_{i=1}^n \ln \left (\sum\limits_{j=1}^k p_j f_j
(X_i;\,\theta_j )\right )
$$
в виде
\begin{multline}
\ln L (\Theta ) = \sum\limits_{j=1}^k\sum\limits_{i=1}^n \g_{ij} \ln p_j+{}\\
{}+\sum\limits_{j=1}^k\sum\limits_{i=1}^n \g_{ij} f(X_i;\,\theta_j)-
\sum\limits_{j=1}^k\sum\limits_{i=1}^n \g_{ij} \ln \g_{ij}\,.
\end{multline}

Справедливость этого тождества легко проверяется с учетом
$$
\sum\limits_{j=1}^k \g_{ij} =1\,.
$$

Далее идея построения итерационного алгоритма вычисления оценок
$\hat{\Theta} = (\hat{p}_1,\ldots , \hat{p}_k,\
\hat{\theta}_1,\ldots , \hat{\theta}_k)$
для параметров $\Theta = (p_1,\ldots , p_k,\ \theta_1,\ldots , \theta_k)$ состоит в
следующем:
\begin{enumerate}[1.]
\item Выбирается некоторое \textit{начальное приближение}~$\hat{\Theta}^0$.
\item \textbf{E-step:} вычисляются по формулам~(6) начальные приближения
$\g_{ij}^0$ для апостериорных вероятностей $\g_{ij}$~--- \textit{этап
оценивания}.
\item \textbf{M-step:} затем, возвращаясь к~(7), при вычисленных значениях
$\g^0_{ij}$ следует определить значения $\hat{\Theta}^1$ из условия
максимизации отдельно каждого из первых двух слагаемых правой
части~(7), поскольку первое слагаемое
$$
\sum_{j=1}^k \sum_{i=1}^n \g_{ij} \ln p_j
$$
зависит только от параметров $p_j$, а второе слагаемое
$$
\sum_{j=1}^k \sum_{i=1}^n \g_{ij} f(X_i;\,\theta_j )
$$
зависит только от параметров $\theta_j$~--- \textit{этап максимизации}.
\item Проверяется \textit{условие останова}:
$$
\parallel \Theta^{(t)} - \Theta^{t-1}\parallel <\varepsilon\,,
$$
где $t$~--- номер итерации, а
$\parallel\bullet\parallel$~--- евклидова норма, для некоторого $\varepsilon
>0$.
\end{enumerate}

Очевидно, решение оптимизационной задачи
$$
\sum\limits_{j=1}^k\sum\limits_{i=1}^n \g_{ij}^{(t)}\ln p_j \rightarrow
\underset{p_j}{\max}
$$
дается выражением (с учетом $\sum_{j=1}^k p_j =1$):
$$
p_{ij}^{(t+1)} =\fr{1}{n}\,\sum\limits_{i=1}^n \g_{ij}^{(t)}\,,
$$
где $t$~--- номер итерации, $t = 0$, 1, 2,\,\ldots

Решение оптимизационной задачи
$$
\sum\limits_{j=1}^k \sum\limits_{i=1}^n \g_{ij}^{(t)} f(X_i;\,\theta_j )
\rightarrow \underset{\theta_j}{\max}
$$
получить намного проще решения задачи~(5): выражение для $\theta_j$
записывается с учетом знания конкретного вида функций
$f(X,\,\theta)$~\cite{3bat}.

\subsection{О сходимости алгоритма} %2.2.

В работе М.\,И.~Шлезингера~\cite{5bat}, где эта схема (позднее названная
ЕМ-схемой) впервые предложена, установлены и основные свойства
реа\-ли\-зу\-ющих ее алгоритмов. В частности, было доказано, что при достаточно
широких предположениях \textit{предельные точки} всякой последовательности,
порожденной итерациями ЕМ-алгоритма, являются стационарными точками
оптимизируемой логарифмической функции правдоподобия $\ln L(\Theta )$ и что
найдется неподвижная точка алгоритма, к которой будет сходиться каждая из таких
последовательностей. Если дополнительно потребовать положительной
определенности информационной мат\-ри\-цы Фишера для $\ln L(\Theta )$ при
истинных зна\-че\-ни\-ях па\-ра\-мет\-ра $\Theta$, то можно показать, что
асимптотически по $n\rightarrow\infty$ (т.е.\ при больших выборках) существует
единственное сходящееся (по веро\-ят\-но\-сти) решение $\hat{\Theta}(n)$
уравнений метода максимального правдоподобия и, кроме того, существует в
пространстве параметров $\Theta$ норма, в которой последовательность
$\Theta^{(t)}(n)$, порожденная ЕМ-ал\-го\-рит\-мом, сходится к $\hat{\Theta}
(n)$, если только начальная аппроксимация $\hat{\Theta}^0$ не была слишком
далека от~$\hat{\Theta} (n)$. {%\looseness=1

}

Таким образом, результаты исследования свойств ЕМ-алгоритмов метода
максимального правдоподобия разделения смеси и их практическое
использование показали, что они являются достаточно работоспособными (при
известном чис\-ле компонентов смеси) даже при большом чис\-ле $k$ компонентов и
при высоких размерностях анализируемого признака~$X$~\cite{3bat}.

\subsection{Уравнения для смеси экспоненциальных распределений}
%2.3.

Применим описанный выше алгоритм к разделению смеси
экспоненциальных распределений:
$$
p(x) = \sum\limits_{j=1}^k p_j \lambda_j e^{-\lambda_j x}\,.
$$
Получаем следующие итерационные уравнения:
\begin{align*}
\g_{ij}^{(t+1)} & = \fr{p_j^{(t)}\lambda_j^{(t)}e^{-
\lambda_j^{(t)}X_i}}{\sum\limits_{l=1}^k p_l^{(t)}\lambda_l^{(t)}
e^{-\lambda_l^{(t)}X_i}}\,,\\
p_j^{(t+1)} & = \fr{1}{n}\,\sum\limits_{i=1}^n \g_{ij}^{(t)}\,.
\end{align*}

Чтобы найти  оценки $\lambda_j$, подсчитаем первую производную функции
$$\sum_{j=1}^k\sum_{i=1}^n \g_{ij}^{(t)} \ln (\lambda_j e^{-\lambda_j X_i}):$$
\vspace*{-8pt}
\begin{multline*}
\left ( \sum\limits_{j=1}^k \sum\limits_{i=1}^n
\g_{ij}^{(t)}\ln \left ( \lambda_j
e^{-\lambda_j X_i} \right ) \right )^\prime \lambda_j =\\[-3pt]
{}= \left (
\sum\limits_{j=1}^k\sum\limits_{i=1}^n \g_{ij}^{(t)}\ln (\lambda_j -\lambda_j X_i )
\right )^\prime \lambda_j =\\[-3pt]
{}= \sum\limits_{i=1}^n \g_{ij}^{(t)}\left (
\fr{1}{\lambda_j} - X_i \right )\,.
\end{multline*}

Разрешая уравнение
$$
\sum\limits_{i=1}^n \g_{ij}^{(t)}\left ( \fr{1}{\lambda_j} -X_i\right ) =0
$$
относительно $\lambda_j$, получаем следующее итерационное уравнение:
$$
\lambda_j^{(t+1)} = \fr{\sum\limits_{i=1}^n
\g_{ij}^{(t)}}{\sum\limits_{i=1}^n \g_{ij}^{(t)} X_i}\,.
$$

\subsection{Уравнения для смеси гамма-распределений } %2.4.

Применим теперь ЕМ-алгоритм к смеси гам\-ма-рас\-пре\-де\-ле\-ний вида
$$
p(x) = \sum\limits_{j=1}^k p_j \fr{\alpha_j^{\alpha_j} x^{\alpha_j -
1}}{\lambda_j^{\alpha_j} \Gamma (\alpha_j )}\,e^{-(\alpha_j / \lambda_j)x}\,.
$$

Такая параметризация удобна для нахождения
оценок~$\alpha_j$~\cite{6bat}.

Аналогичным способом выписываются итерационные уравнения:
\begin{align*}
\g_{ij}^{(t+1)} & =   \fr{p_j^{(t)}\fr{(\alpha_j^{\alpha_j} )^{(t)}
x^{\alpha_j - 1}}{(\lambda_j^{\alpha_j} )^{(t)}\Gamma (\alpha_j)}\,
e^{-(\alpha_j /\gamma_j)^{(t)}x}}{\sum\limits_{l=1}^k
p_l^{(t)}\fr{(\alpha_l^{\alpha_l})^{(t)} x^{\alpha_l -
1}}{(\lambda_l^{\alpha_l})^{(t)}\Gamma (\alpha_l )}\,
e^{-(\alpha_l /\lambda_l)^{(t)} x}}\,,\\
p_j^{(t+1)} & = \fr{1}{n}\,\sum\limits_{i=1}^n \g_{ij}^{(t)}\,.
\end{align*}

Далее найдем оценки $\lambda_j$ для данного случая, приравнивая
производную
\begin{equation} %8
\sum\limits_{j=1}^k \sum\limits_{i=1}^n \g_{ij}^{(t)} \ln \left (
\fr{\alpha_j^{\alpha_j} x^{\alpha_j -1}}{\lambda_j^{\alpha_j}\Gamma
(\alpha_j)}\,e^{-(\alpha_j /\lambda_j) x}\right )
\end{equation}
по $\lambda_j$ к нулю и разрешая относительно~$\lambda_j$ уравнение:
$$
\sum\limits_{i=1}^n \g_{ij}^{(t+1)}\left ( \fr{\alpha_j^{(t)}}{\lambda_j^{(t)}}
- \fr{\alpha_j^{(t)}X_i}{\left ( \lambda_j^{(t)}\right )^2}\right ) =0 \,.
$$
Получаем
$$
\lambda_j^{(t+1)} = \fr{\sum\limits_{i=1}^n \g_{ij}^{(t)}
X_i}{\sum\limits_{i=1}^n \g_{ij}^{(t)}}\,.
$$

Для того чтобы получить итерационные уравнения для $\alpha_j$, найдем
первую производную~(8):
\begin{multline*}
\left ( \sum\limits_{j=1}^k\sum\limits_{i=1}^n \g_{ij}^{(t)}\ln \left (
\fr{\alpha_j^{\alpha_j} x^{\alpha_j -1}}{\lambda_j^{\alpha_j}\Gamma (\alpha_j
)}\,e^{-(\alpha_j /\lambda_j ) x} \right ) \right )^\prime \alpha_j ={}\\[-3pt]
{}=\left ( \sum\limits_{j=1}^k\sum\limits_{i=1}^n \g_{ij}^{(t)}\ln \left (
\fr{\alpha_j^{\alpha_j}}{\lambda_j^{\alpha_j}}\right ) - \ln \Gamma (\alpha_j )+{} \right.\\[-3pt]
{}+\left.
(\alpha_j -1 )\ln X_i - \fr{\alpha_j}{\lambda_j}\,X_i \right )^\prime \alpha_j =\\[-3pt]
{}=\sum\limits_{i=1}^n \g_{ij}^{(t)} \left ( \ln \alpha_j +1-\ln \lambda_j -
\fr{\Gamma^\prime (\alpha_j )}{\Gamma (\alpha_j)}\right.+\\[-3pt]
{}+\left. \ln X_i - \fr{X_i}{\lambda_j}\right )\,;
\end{multline*}
\begin{multline*}
\sum\limits_{i=1}^n \g_{ij}^{(t)} \left(  \ln \alpha_j +1 -\ln \lambda_j -{}\right. \\[-3pt]
\left. {}-\fr{\Gamma^\prime (\alpha_j )}{\Gamma (\alpha_j )}+\ln X_i 
-\fr{X_i}{\lambda_j} \right) =0\,;
\end{multline*}
\begin{multline}
\fr{\Gamma^\prime (\alpha_j )}{\Gamma (\alpha_j )} ={}\\[-3pt]
{}= \fr{\sum\limits_{i=1}^n \g_{ij}^{(t)} \left ( \ln \alpha_j +1-\ln\lambda_j 
+\ln X_i -\fr{X_i}{\lambda_j} \right )}{\sum\limits_{i=1}^n \g_{ij}^{(t)}}\,.
\end{multline}
%
Здесь $\Gamma^\prime (\alpha_j ) / \Gamma (\alpha_j )$~--- это
\textit{логарифмическая производная гамма-функции}. Для нее существует так
называемое \textit{разложение Абрамовитца}--\textit{Стигана}~\cite{4bat}:
$$
\fr{\Gamma^\prime (\alpha ) }{ \Gamma (\alpha )} = \mathrm{log}\,\alpha -
\fr{1}{2\alpha }-\fr{1}{12\alpha^2 }+\ldots
$$

Подставим первые три члена разложения в~(9) и разрешим это уравнение
относительно~$\alpha_j$:
$$
\alpha_{ij}^{(t+1)} = \fr{\sum\limits_{i=1}^n
\g_{ij}^{(t+1)}}{2\sum\limits_{i=1}^n \g_{ij}^{(t +1)}\left ( \fr{X_i}{\lambda_j^{(t)}} -
\ln \fr{X_i}{\lambda_j^{(t)}} -1\right )}\,.
$$
В итоге получаем итерационные уравнения для ~$\alpha_j$.

\section{Описание программного обеспечения (программа~ЕМ)}

\subsection{Назначение программы} %3.1.

Разработанная авторами статьи программа ЕМ предназначена для решения задачи
разделения смесей экспоненциальных и гамма-распределений, поставленной в
разд.~2, с использованием ЕМ-ал\-го\-рит\-ма и формул, описанных в разд.~3.

\subsection{Инструменты разработки} %3.2.

Для создания программы была использована среда разработки Microsoft
Visual Studio .NET 2005 и объектно-ориентированный язык C\#. Для
визуализации результатов была использована свободно распространяемая
графическая библиотека ZedGraph~\cite{7bat}.


\subsection{Возможности  программы} %3.3.

\noindent
\begin{itemize}
\item Загрузка выборочных данных из текстового файла
\item Оценивание по выборке параметров смеси экспоненциальных
распределений
\item Оценивание по выборке параметров смеси гамма-распределений
\item Отслеживание изменений параметров смесей распределений во
времени в режиме <<скользящего окна>>
\item Построение гистограммы по выборке
\end{itemize}

\subsection{Входные и выходные данные. Функционирование
программы} %3.4.

В качестве \textit{входных данных} программа ЕМ получает:
\begin{itemize}
\item выборочные данные из текстового файла;
\item число компонентов смеси;
\item размер <<скользящего окна>>;
\item размер класса гистограммы.
\end{itemize}

На \textit{выходе} мы получаем:
\begin{itemize}
\item точечные оценки параметров смеси экспоненциальных
распределений;
\item точечные оценки параметров смеси гамма-распределений;
\item графическое изображение результирующей смеси распределения;
\item графическое изображение компонентов каж\-дой смеси;
\item графическое изображение того, как меняются параметры смесей
распределений с течением времени в режиме <<скользящего окна>>;
\item гистограмма, построенная по выборке;
\item значение статистического теста.
\end{itemize}

Выборочные данные загружаются из текстового файла в память программы и подаются
на вход двум независимо работающим реализациям ЕМ-алгоритма~--- для
идентификации смеси экспоненциальных распределений и для идентификации смеси
гамма-распределений. Результатом их работы являются наборы значений оцениваемых
параметров модели, предложенной в разд.~2. Кроме того, результирующие
распределения визуализируются в виде графиков. В программе можно запустить
режим <<скользящего окна>>, который для всех подвыборок заданного
размера с помощью ЕМ-алгоритма оценивает параметры смесей распределений этих
подвыборок. Все действия программы документируются в окне информации.

\section{Описание тестовых расчетов}

С использованием разработанной программы были проведены тестовые
расчеты на выборочных данных реального сетевого трафика.

На вход программы EM были поданы выборки трафика:
\begin{enumerate}[I]
\item Между лабораторией Lawrence Berkeley (Berkeley, California) и
внешним миром размера примерно 7000~\cite{8bat}~--- \textit{выборка~1}.
\item
Сети радиодоступа ЗАО <<Синтерра>> размера примерно 1000~\cite{9bat}~---
 \textit{выборка~2}.
\end{enumerate}

\subsection{Выборка 1 ``Berkeley''} %5.1.

При числе компонентов смеси~5 и случайном начальном приближении
были получены результаты, представленные в табл.~\ref{t1bat}.


Данные результаты иллюстрирует рис.~\ref{f5bat}.

Гистограмма  на рис.~\ref{f6bat} показывает статистическую значимость
полученных результатов.

Данная выборка обладает той особенностью, что она собиралась в течение
достаточно длительного времени и в ней агрегирован самый разнородный
трафик. Поэтому в ней присутствует не только большое количество
<<коротких>> сообщений (что обычно для выборок из телетрафика), но и
некоторый массив сообщений средней длины, а также определенный
<<выброс>> больших сообщений. Это свидетельствует о \textit{пиковости}
телетрафика на довольно больших промежутках времени.

Как мы видим, ЕМ-алгоритм удачно справился с задачей идентификации
смеси.

\subsection{Выборка~2 ``Synterra''} %5.2.

Результаты применения ЕМ-алгоритма к выборке ``Synterra''
представлены в табл.~\ref{t2bat}.
\begin{table*}\small
\begin{minipage}[t]{76mm}
\begin{center}
\Caption{Результаты применения ЕМ-алго\-рит\-ма к выборке~1 ``Berkeley'' 
\label{t1bat}} \vspace*{2ex}

\tabcolsep=8.7pt
\begin{tabular}{|c|c|c|}
\hline
№&Начальное приближение&Результат\\
\hline
\multicolumn{3}{|c|}{$P$}\\
\hline
0&0,2&0,1896\\
1&0,2&0,1858\\
2&0,2&0,1830\\
3&0,2&0,2259\\
4&0,2&0,2154\\
\hline
\multicolumn{3}{|c|}{$\alpha$}\\
\hline
0&2,7028&10,9783\hphantom{9}\\
1&3,6273&5,8621 \\
2&5,7598&2,7092\\
3&0,2315&1,0235\\
4&0,9110&0,4772\\
\hline
\multicolumn{3}{|c|}{$\lambda$}\\
\hline
0&85,2066&137,1714  \\
1&23,9592&136,7349\\
2&63,8425&132,6482\\
3&\hphantom{9}1,8026&116,7317\\
4&98,3882&102,5278\\
\hline
\end{tabular}
\end{center}
\end{minipage}\hfill
\begin{minipage}[t]{76mm}
%\end{table*}
%\begin{table*}\small
\begin{center}
\Caption{Результаты применения ЕМ-алго\-рит\-ма к выборке~2 ``Synterra'' 
\label{t2bat}} \vspace*{2ex}

\tabcolsep=8.7pt
\begin{tabular}{|c|c|c|}
\hline
№&Начальное приближение&Результат\\
\hline
\multicolumn{3}{|c|}{$P$}\\
\hline
0&0,2&$0{,}3815\hphantom{{}\cdot 10^{-9}}$\\
1&0,2&$0{,}3594\hphantom{{}\cdot 10^{-9}}$\\
2&0,2&$0{,}2589\hphantom{{}\cdot 10^{-9}}$\\
3&0,2&$0{,}4401\cdot 10^{-9}$\\
4&0,2&$0{,}0\hphantom{{}\cdot 10^{-9}999}$\\
\hline
\multicolumn{3}{|c|}{$\alpha$}\\
\hline
0&6,0804&1,5833\\
1&3,1838&0,8554\\
2&1,4886&0,4557\\
3&4,6407&0,2278\\
4&3,7843&0,1139\\
\hline
\multicolumn{3}{|c|}{$\lambda$}\\
\hline
0&17,3387&15,8682\\
1&47,8294&16,9150\\
2&54,1984&19,2866\\
3&\hphantom{1}8,6254&19,2866\\
4&\hphantom{1}5,7252&19,2866\\
\hline
\end{tabular}
\end{center}
\end{minipage}
\end{table*}


Данные результаты иллюстрируют рис.~\ref{f7bat}.


Эти результаты также отражают действительную картину, как показано на
рис.~\ref{f8bat}.


Этот трафик был снят с базовой станции <<Лукойл-Юго-Запад>> сети
широкополосного радиодоступа ЗАО <<Синтерра>>. Сеть радиодоступа
является реализацией так называемой <<последней мили>>, переносящей два
разных вида трафика: данные (Ethernet пакеты) и голос (IP-телефония, VoIP).
Поэтому здесь присутствуют в качестве основной массы короткие, но
интенсивные сообщения (пакеты SIP и голосовые фреймы), а также длинные
сообщения, содержащие данные.

Как мы видим, программная реализация ЕМ-ал\-го\-рит\-ма успешно справилась с
задачей разделения смесей распределений для этих двух выборок, что делает
данную программу удобным инструментом построения стохастической картины
конкретной сети. По полученным данным, используя метод интерпретации,
предложенный в разд.~2, можно получить представление о количестве
последовательных и параллельных структур вероятностной модели сети.

\subsection{Режим <<скользящего окна>>} %5.3.

Результаты для выборки
``Berkeley'' в режиме <<скользящего окна>>  представлены
на рис.~\ref{f9bat}.


Данные графики показывают изменение параметров распределений подвыборок выборки 
``Berkeley''. Видно, что параметры распределений подвыборок не остаются 
неизменными во времени, наоборот, они имеют внешне случайный характер. На 
рис.~\ref{f9bat},\,\textit{в} видна даже своеобразная пульсация первой 
компоненты.
%
На основании расчетов можно сделать вывод о том, что пиковость трафика
обусловливается как формой, так и интенсивностью сообщений.

\section{Заключение}

В данной работе исследована вероятностная модель  информационных потоков,
возникающих в сложных телекоммуникационных конвергентных сетях, построенная с
помощью асимптотического и энтропийного подходов. Эта модель предполагает, что
функционирование сложной телекоммуникационной сети можно представить в виде
суперпозиции довольно простых стохастических структур~--- последовательных и
параллельных, которые по\-рож\-да\-ют смеси гамма-распределений для случайной
величины времени обработки и передачи сообщений в сети. Предложена простая
интерпретация параметров данной модели.
\begin{figure*} %fig5
\vspace*{1pt}
\begin{center}
\mbox{%
\epsfxsize=130mm %145.109mm 
\epsfbox{bat-5.eps} }
\end{center}
\vspace*{-13pt} \Caption{Компоненты смеси начального приближения~(\textit{а}) и 
результата~(\textit{б}) для выборки~1 ``Berkeley'' \label{f5bat}}
%\end{figure*}
%\begin{figure*} %fig6
\vspace*{12pt}
\begin{center}
\mbox{%
\epsfxsize=130mm %148.256mm 
\epsfbox{bat-7.eps} }
\end{center}
\vspace*{-13pt} \Caption{График смеси распределений~(\textit{1}) и гистограмма 
для выборки~1 ``Berkeley''~(\textit{2}) \label{f6bat}}
\end{figure*}



\begin{figure*} %fig7
\vspace*{1pt}
\begin{center}
\mbox{%
\epsfxsize=130mm %144.283mm 
\epsfbox{bat-8.eps} }
\end{center}
\vspace*{-16pt} \Caption{Компоненты смеси начального приближения~(\textit{а}) и 
результата~(\textit{б}) для выборки~2 ``Synterra'' \label{f7bat}}
%\end{figure*}
%\begin{figure*} %fig8
\vspace*{12pt}
\begin{center}
\mbox{%
\epsfxsize=130mm %148.256mm 
\epsfbox{bat-10.eps} }
\end{center}
\vspace*{-11pt} \Caption{График смеси распределений~(\textit{1}) и гистограмма
для выборки~2 ``Synterra''~(\textit{2}) \label{f8bat}}
\end{figure*}

\begin{figure*} %fig9
\vspace*{1pt}
\begin{center}
\mbox{%
\epsfxsize=119.041mm
\epsfbox{bat-11.eps} }
\end{center}
\vspace*{-9pt} \Caption{Изменение  смешивающих параметров~(\textit{а}), 
параметров формы~(\textit{б}) и параметров масштаба~(\textit{в}) во времени для 
выборки~1 ``Berkeley'' \label{f9bat}}
\end{figure*}

Для решения вытекающей из модели задачи предложен итерационный алгоритм,
базирующийся на методе максимального правдоподобия~--- ЕМ-ал\-го\-ритм, для
которого получены формулы для конкретного вида смесей~--- экспоненциальных и
гамма-распределений.
%
Кроме того, разработан программный инструментарий для оценки параметров 
предложенной модели на выборках из реальных трафиковых данных. Проведены 
исследования, которые подтвердили предположения вероятностной модели. 


Получение информации о стохастической структуре
телекоммуникационных сетей и наличие программных инструментов для
выявления более или менее стабильных структур позволит понять причины
возникновения неожиданных больших нагрузок, предотвратить такие нагрузки,
а также поможет в будущем в проектировании надежных, оптимальных по
стоимости и уровню сервиса телекоммуникационных сетей нового поколения.

%\vspace*{-15pt} 
{\small\frenchspacing
{%\baselineskip=10.8pt
\addcontentsline{toc}{section}{Литература}
\begin{thebibliography}{9}
\bibitem{1bat}
Teletraffic Engeneering Handbook. International Telecommunication Union, 
Geneva, 2005 {\sf http://www.itu.int}. \vspace*{5pt} 
\bibitem{2bat}
\Au{Севастьянов~Б.\,А.} Курс теории вероятностей и математической статистики. 
М., 2004. \vspace*{5pt} 
\bibitem{3bat}
\Au{Айвазян~C.\,А., Бухштабер~В.\,М., Енюков~И.\,С, Мешалкин~Л.\,Д.} Прикладная 
статистика. Классификация и снижение размерности~// Финансы и статистика. М., 
1989. \vspace*{5pt} 
\bibitem{4bat}
\Au{Bilmes~J.\,A.} A gentle tutorial of the EM algorithm and its application to 
parameter estimation for Gaussian mixture and hidden Markov models. Berkeley, 
CA, USA: International Computer Science Institute,  1998. \vspace*{5pt} 
\bibitem{5bat}
\Au{Шлезингер~М.\,И.} О самопроизвольном различении образов~// Шлезингер~М.\,И. 
Читающие. автоматы. Киев: Наукова думка, 1965. С.~38--45. \vspace*{5pt} 
\bibitem{6bat}
\Au{Hsiao~I.-T., Rangarajan~A., Gindi~G.}. Joint-MAP 
reconstruction/segmentation for transmission tomography using mixture-models as 
priors. Yale University, 1998. \vspace*{5pt} 
\bibitem{7bat}
{\sf http://zedgraph.org}. \vspace*{4pt} 
\bibitem{8bat}
{\sf http://ita.ee.lbl.gov/html/contrib/LBL-PKT.html}. \vspace*{5pt} 
\bibitem{9bat}
{\sf http://www.synterra.ru}.
\end{thebibliography}

} } \label{end\stat}
\end{multicols}


%\addtocounter{razdel}{1}
%\def\razd{НЕРЕГУЛИРУЕМЫЙ ЭЛЕКТРОПРИВОД ДЛЯ ЭЛЕКТРОЭНЕРГЕТИКИ}

\setcounter{page}{3}

   { %\Large  
   { %\baselineskip=16.6pt
   
   \vspace*{-48pt}
   \begin{center}\LARGE
   \textit{Предисловие}
   \end{center}
   
   %\vspace*{2.5mm}
   
   \vspace*{25mm}
   
   \thispagestyle{empty}
   
   { %\small 

    
Вниманию читателей журнала <<Информатика и её применения>> предлагается 
очередной тематический выпуск <<Вероятностно-статистические методы и 
задачи информатики и информационных технологий>>. Предыдущие тематические 
выпуски журнала по данному направлению вышли в 2008~г.\ (т.~2, вып.~2), 
в 2009~г.\ (т.~3, вып.~3) и в 2010~г.\ (т.~4, вып.~2). 

Статьи, собранные в данном журнале, посвящены разработке новых вероятностно-статистических 
методов, ориентированных на применение к решению конкретных задач информатики и информационных 
технологий, а также~--- в ряде случаев~--- и других прикладных задач. Проблематика, охватываемая 
публикуемыми работами, развивается в рамках научного сотрудничества между Институтом проблем 
информатики Российской академии наук (ИПИ РАН) и Факультетом вычислительной математики и 
кибернетики Московского государственного университета им.\ М.\,В.~Ломоносова в ходе работ 
над совместными научными проектами (в том числе в рамках функционирования 
Научно-образовательного центра <<Вероятностно-статистические методы анализа рисков>>). 
Многие из авторов статей, включенных в данный номер журнала, являются активными участниками 
традиционного международного семинара по проблемам устойчивости стохастических моделей, 
руководимого В.\,М.~Золотаревым и В.\,Ю.~Королевым; регулярные сессии этого семинара 
проводятся под эгидой МГУ и ИПИ РАН (в 2011~г.\ указанный семинар проводится в октябре 
в Калининградской области РФ). 

Наряду с представителями ИПИ РАН и МГУ в число авторов данного выпуска журнала входят 
ученые из Научно-исследовательского института системных исследований РАН, Института 
проблем технологии микроэлектроники и особочистых материалов РАН, Института 
прикладных математических исследований Карельского НЦ РАН, Московского 
авиационного института, Вологодского государственного педагогического университета, 
НИИММ им.\ Н.\,Г.~Чеботарева, Казанского государственного университета, Дебреценского 
университета (Венгрия).

Несколько статей выпуска посвящено разработке и применению стохастических методов и 
информационных технологий для решения различных прикладных задач. В~работе В.\,Г.~Ушакова 
и О.\,В.~Шестакова рассмотрена задача определения вероятностных характеристик случайных 
функций по распределениям интегральных преобразований, возникающих в задачах эмиссионной 
томографии. В~статье Д.\,О.~Яковенко и М.\,А.~Целищева рассмотрены некоторые вопросы 
математической теории риска и предложен новый подход к диверсификации инвестиционных 
портфелей. Работа И.\,А.~Кудрявцевой и А.\,В.~Пантелеева посвящена построению и 
исследованию математической модели, описывающей динамику сильноионизованной плазмы. 
В~статье П.\,П.~Кольцова изучается качество работы ряда алгоритмов сегментации изображений. 
Статья А.\,Н.~Чупрунова и И.~Фазекаша посвящена вероятностному анализу числа без\-оши\-бочных 
блоков при помехоустойчивом кодировании; получены усиленные законы больших чисел для указанных 
величин.

В данном выпуске традиционно присутствует тематика, весьма активно разрабатываемая в течение 
многих лет специалистами ИПИ РАН и МГУ,~--- методы моделирования и управления для 
информационно-телекоммуникационных и вычислительных систем, в частности методы 
теории массового обслуживания. В~статье А.\,И.~Зейфмана с соавторами рассматриваются 
модели обслуживания, описываемые марковскими цепями с непрерывным временем в случае 
наличия катастроф. В~работе М.\,М.~Лери и И.\,А.~Чеплюковой рассматриваются случайные 
графы Интернет-типа, т.\,е.\ графы, степени вершин которых имеют степенные распределения; 
такие задачи находят применение при исследовании глобальных сетей передачи данных. 
Работа Р.\,В.~Разумчика посвящена исследованию систем массового обслуживания специального 
вида~--- с отрицательными заявками и хранением вытесненных заявок.

Ряд статей посвящен развитию перспективных теоретических 
вероятностно-статистических методов, которые находят широкое применение в различных 
задачах информатики и информационных технологий. В~работе В.\,Е.~Бенинга, А.\,К.~Горшенина 
и В.\,Ю.~Королева рассмотрена задача статистической проверки гипотез о числе компонент 
смеси вероятностных распределений, приводится конструкция асимптотически наиболее мощного 
критерия. Результаты этой работы найдут применение в ряде прикладных задач, использующих 
математическую модель смеси вероятностных распределений (в информатике, моделировании 
финансовых рынков, физике турбулентной плазмы и~т.\,д.). В~статье В.\,Ю.~Королева, 
И.\,Г.~Шевцовой и С.\,Я.~Шоргина строится новая, улучшенная оценка точности нормальной 
аппроксимации для пуассоновских случайных сумм; как известно, указанные случайные суммы 
широко используются в качестве моделей многих реальных объектов, в том числе в информатике, 
физике и других прикладных областях. Работа В.\,Г.~Ушакова и Н.\,Г.~Ушакова посвящена 
исследованию ядерной оценки плотности распределения; эти результаты могут применяться, 
в част\-ности, при анализе трафика в телекоммуникационных системах. Серьезные приложения 
в статистике могут получить результаты работы О.\,В.~Шестакова, в которой доказаны оценки 
скорости сходимости распределения выборочного абсолютного медианного отклонения к нормальному 
закону. 

\smallskip

Редакционная коллегия журнала выражает надежду, что данный тематический  выпуск 
будет интересен специалистам в области теории вероятностей и математической статистики 
и их применения к решению задач информатики и информационных технологий.
     
     %\vfill 
     \vspace*{20mm}
     \noindent
     Заместитель главного редактора журнала <<Информатика и её 
применения>>,\\
     директор ИПИ РАН, академик  \hfill
     \textit{И.\,А.~Соколов}\\
     
     \noindent
     Редактор-составитель тематического выпуска,\\
     профессор кафедры математической статистики факультета\\
      вычислительной математики и кибернетики МГУ им.\ М.\,В.~Ломоносова,\\
     ведущий научный сотрудник ИПИ РАН,\\ 
доктор физико-математических наук \hfill
      \textit{В.\,Ю.~Королев}
     
     } }
     }



\def\stat{sinits}

\def\tit{АНАЛИТИЧЕСКОЕ МОДЕЛИРОВАНИЕ
НОРМАЛЬНЫХ ПРОЦЕССОВ В~СТОХАСТИЧЕСКИХ СИСТЕМАХ СО~СЛОЖНЫМИ~НЕЛИНЕЙНОСТЯМИ}

\def\titkol{Аналитическое моделирование
нормальных процессов в~стохастических системах со~сложными нелинейностями}

\def\aut{И.\,Н.~Синицын$^1$, В.\,И.~Синицын$^2$}

\def\autkol{И.\,Н.~Синицын, В.\,И.~Синицын}

\titel{\tit}{\aut}{\autkol}{\titkol}

\renewcommand{\thefootnote}{\arabic{footnote}}
\footnotetext[1]{Институт проблем
информатики Российской академии наук, sinitsin@dol.ru}
\footnotetext[2]{Институт проблем
информатики Российской академии наук, vsinitsin@ipiran.ru}


\Abst{Рассматриваются конечномерные дифференциальные стохастические системы
(ДСтС) и эредитарные (интегродифференциальные) стохастические системы  (ЭСтС)
с винеровскими и пуассоновскими шумами, приводимые к ДСтС со сложными конечными,
дифференциальными и интегральными нелинейностями. Такие модели функционирования
описывают поведение многих современных нано- и кван\-то\-во-оп\-ти\-че\-ских
технических средств информатики. Приводятся уравнения методов нормальной
аппроксимации (МНА) и статистической линеаризации (МСЛ) для аналитического
моделирования нестационарных и стационарных нормальных (гауссовских) процессов
в нелинейных ДСтС и  нелинейных ЭСтС путем аппроксимации эредитарных ядер
линейными операторными уравнениями для дифференцируемых нелинейностей и
сингулярными ядрами для недифференцируемых нелинейностей. Рассматриваются
методы вычисления типовых интегралов МНА (МСЛ) для сложных (многомерных и
векторного аргумента) конечных и дифференциальных нелинейностей. Особое
внимание уделяется иррациональным и дробно-рациональным нелинейностям
скалярного аргумента. Приводятся примеры вычисления интегралов. Подробно
рассматриваются вопросы вычисления типовых интегралов МНА (МСЛ) для сложных
интегральных нелинейностей.}

\KW{аналитическое моделирование;
дифференциальные стохастические системы с винеровскими и пуассоновскими шумами (ДСтС);
метод нормальной аппроксимации (МНА);
метод статистической линеаризации (МСЛ);
сложные иррациональные нелинейности;
сложные конечные, дифференциальные и интегральные нелинейности;
эредитарные стохастические системы (ЭСтС), приводимые к дифференциальным}

\DOI{10.14357/19922264140302}

\vspace*{9pt}

\vskip 16pt plus 9pt minus 6pt

\thispagestyle{headings}

\begin{multicols}{2}

\label{st\stat}


\section{Введение}


Моделями функционирования многих современных технических сис\-тем информатики
служат стохастические системы (СтС), описываемые дифференциальными, интегральными
и интегродифференциальными уравнениями со сложными дроб\-но-ра\-ци\-о\-наль\-ны\-ми,
иррациональными и интегральными нелинейностями. В~[1] дано систематическое
изложение МНА и МСЛ для ДСтС и ЭСтС, приводимых к дифференциальным.

Обобщая~[2--7], рассмотрим развитие МНА и МСЛ для аналитического моделирования
нормальных стохастических процессов (СтП) на случай СтС со сложными конечными,
дифференциальными и интегральными нелинейностями.

Как показано в~\cite{4-sin}, альтернативным подходом к аналитическому моделированию
СтП в ДСтС и ЭСтС служит подход, основанный на дискретизации стохастических
дифференциальных уравнений на основе использования обобщенной формы Ито и
кратных стохастических интегралов от винеровских и пуассоновских СтП с
последующим применением дискретных версий МНА (МСЛ).

Статья состоит из введения, пяти разделов и заключения.

В~разд.~2 и~3
приводятся уравнения МНА и МСЛ для аналитического моделирования одно- и
двумерных распределений стационарных и нестационарных СтП в ДСтС и ЭСтС,
приводимых к ДСтС.

Типовые интегралы МНА и МСЛ рассматриваются в разд.~4.

Особенности аналитического моделирования в ДСтС со сложными конечными и
дифференциальными нелинейностями обсуждаются в разд.~5.

Раздел~6
посвящен аналитическому моделированию СтП в ДСтС со сложными интегральными
нелинейностями.

Приводятся примеры.


\section{Уравнения методов нормальной~аппроксимации и~статистической
линеаризации для~дифференциальных стохастических систем}

Как известно~\cite{2-sin, 3-sin},  уравнения конечномерных непрерывных нелинейных сис\-тем
со стохастическими возмущениями путем расширения вектора состояния ДСтС
могут быть записаны в виде следующего векторного стохастического
дифференциального уравнения Ито:
    \begin{multline}
    dY_t = a(Y_t, t)\, dt + b (Y_t, t) \,dW_0+{}\\
    {}+ \iii_{R_0} c (Y_t, t, v) P^0
    (dt, dv)\,,\enskip Y(t_0) = Y_0\,.\label{e2.1-sin}
    \end{multline}
Здесь $a=a(Y_t, t)$ и $b\hm=b(y_t, t)$~--- известные
$(p\times 1)$-мер\-ная и  $(p\times m)$-мер\-ная функции~$Y_t$ и~$t$;
$W_0\hm= W_0(t)$~--- $r$-мер\-ный винеровский СтП интенсивности
$\nu_0 \hm= \nu_0(t)$; $c(Y_t, t, v)$~--- $(p\times 1)$-мер\-ная функция  $Y_t, t$
и вспомогательного $(q\times 1)$-мер\-но\-го па\-ра\-мет\-ра~$v$;
$\iii_{\Delta} dP^0 (t, A)$~--- центрированная пуассоновская мера,
определяемая
\begin{equation*}
\iii_{\Delta} dP^0 (t, A) = \iii_{\Delta} dP (t,A) =
\iii_{\Delta} \nu_P (t,A)\, dt\,. %\label{e2.2-sin}
\end{equation*}
В~(\ref{e2.1-sin}) принято: $\iii_{\Delta}$~-- число скачков пуассоновского
СтП в интервале времени  $\Delta \hm= (t_1, t_2]$; $\nu_P (t, A)$~---
интенсивность пуассоновского СтП  $P(t,A)$; $A$~--- некоторое борелевское
множество пространства  $R_0^q$ с выколотым началом.
Начальное значение~$Y_0$ представляет собой случайную величину, не зависящую
от приращений СтП  $W_0(t)$ и $P(t,A)$ на интервалах времени, следующих
за~$t_0$, $t_0 \hm\le t_1\hm\le t_2$ для любого множества~$A$.

В случае аддитивных нормальных (гауссовских) и обобщенных
пуассоновских возмущений уравнение~(\ref{e2.1-sin}) имеет вид:
\begin{equation}
\dot Y_t = a(Y_t,t)+ b_0 (t) V\,, \enskip
V = \dot W\,,\enskip Y(t_0) = Y_0\,.\label{e2.3-sin}
\end{equation}
Здесь $W$~--- СтП с независимыми приращениями, представляющий собой
смесь нормального и обобщенного пуассоновского СтП.

Если предположить существование конечных вероятностных
моментов второго порядка для моментов времени~$t_1$ и~$t_2$, то уравнения
МНА примут следующий вид~\cite{2-sin, 3-sin}:
\begin{itemize}
\item  для характеристических функций
    \begin{equation}
    g_1^N (\la;t) =\exp \lk i\la^{\mathrm{T}} m_t - \fr{1}{2}\, \la^{\mathrm{T}} K_t \la\rk\,;\label{e2.4-sin}
    \end{equation}
\begin{equation}
\hspace*{-7.5mm}g_{t_1, t_2}^N (\la_1, \la_2;t_1, t_2 ) =\exp \lk i\bar \la^{\mathrm{T}} \bar m_2 -
\fr{1}{2}\, \bar \la^{\mathrm{T}} \bar K_2 \la\rk\,,\!\!\label{e2.5-sin}
\end{equation}
где
    \begin{gather*}
    \bar \la =\lk \la_1^{\mathrm{T}}\la_2^{\mathrm{T}}\rk^{\mathrm{T}}\,; \quad
        \bar m_2 = \lk m_{t_1}^{\mathrm{T}} m_{t_2}^{\mathrm{T}}\rk^{\mathrm{T}}\,;\\
        \bar K_2= \begin{bmatrix}
    K(t_1, t_1)& K(t_1, t_2)\\
    K(t_2, t_1)& K(t_2, t_2)
    \end{bmatrix}\,;
    \end{gather*}

\item для математических ожиданий  $m_t$, ковариационной матрицы~$K_t$ и
матрицы ковариационных функций $K(t_1, t_2)$:
    \begin{equation}
    \dot m_t = a_1 (m_t, K_t, t)\,,\enskip m_0 = m(t_0)\,;\label{e2.6-sin}
    \end{equation}
\begin{equation}
\dot K_t = a_2 (m_t, K_t, t)\,,\enskip K_0 = K(t_0)\,;\label{e2.7-sin}
\end{equation}

\vspace*{-12pt}

\noindent
\begin{multline}
\fr{\prt K(t_1, t_2)}{\prt t_2 }= K(t_1, t_2) a_{21} (m_{t_2}, K_{t_2}, t_2)^{\mathrm{T}}\,;\\
K(t_1, t_1) = K_{t_1}\,.
\label{e2.8-sin}
\end{multline}
    \end{itemize}
Здесь приняты следующие обозначения:
\begin{equation}
a_1 = a_1 (m_t, K_t, t) = M_N a (Y_t, t)\,;\label{e2.9-sin}
\end{equation}

\vspace*{-12pt}

\noindent
\begin{multline}
a_2 = a_2 (m_t, K_t, t) = a_{21} (m_t, K_t, t)+{}\\
{}+ a_{21} (m_t, K_t, t)^{\mathrm{T}} +
a_{22}(m_t, K_t, t)\,;\label{e2.10-sin}
\end{multline}

\vspace*{-12pt}

\noindent

\begin{equation}
a_{21} = a_{21}(m_t, K_t, t)=  M_N a(Y_t, t) Y_{t}^{0\mathrm{T}}\,;\label{e2.11-sin}
\end{equation}
\begin{equation*}
a_{22} = a_{22}(m_t, K_t, t)= M_N \sigma (Y_t, t)\,;
%\label{e2.12-sin}
\end{equation*}

\vspace*{-12pt}

\noindent
\begin{multline*}
\sigma (Y_t, t) = b(Y_t, t) \nu_0(t) b(Y_t, t)^{\mathrm{T}} +{}\\
{}+
\iii_{R_0^q} c (Y_t, t, v) c(Y_t, t,v)^{\mathrm{T}}
\nu_P (t, dv)\,; %\label{e2.13-sin}
\end{multline*}

\vspace*{-12pt}

\begin{gather*}
m_t = MY_t\,,\quad Y_t^0 = Y_t - m_t\,,\\
K_t = M_N Y_0^0 Y_t^{0\mathrm{T}}\,,\quad K(t_1, t_2) =
M_N Y_{t_1}^0 Y_{t_2}^0\,; %\label{e2.14-sin}
\end{gather*}
$M_N$~--- символ вычисления математического ожидания для нормальных
распределений~(\ref{e2.4-sin}) и~(\ref{e2.5-sin}).

Для стационарных ДСтС нормальные стационарные СтП~--- если они существуют,
то  $m_t \hm=\bar m$, $ K_t \hm=\bar K$, $K(t_1, t_2) \hm= k(\tau)$
$(\tau \hm= t_1\hm-t_2)$,~--- определяются уравнениями~\cite{2-sin, 3-sin}:
   \begin{equation}
   a_1 (\bar m, \bar K) =0\,;\enskip a_2 (\bar m, \bar K)=0\,;\label{e2.15-sin}
   \end{equation}
   \begin{equation}
   \left.
   \hspace*{-2.8mm}\begin{array}{l}
  \dot k_\tau (\tau) = a_{21} (\bar m, \bar K)\bar K^{-1} k(\tau)\,;\\[9pt]
  k(0) =\bar K \enskip (\forall \tau >0)\,, \
  k(\tau) = k(-\tau)^{\mathrm{T}} \enskip
  (\forall\tau <0)\,.
  \end{array}\!\!
  \right\}\!\!
  \label{e2.16-sin}
  \end{equation}
При этом необходимо, чтобы матрица  $a_{21} (\bar m, \bar K)\hm=\bar a_{21}$
была бы асимптотически устойчивой.

Для ДСтС~(\ref{e2.3-sin}) уравнения МНА переходят в уравнения МСЛ
Казакова~\cite{2-sin, 3-sin}, если принять
\begin{equation}
a(Y_t,t) = a_1 (m_t, K_t) + k_1^a (m_t, K_t) Y_t^0\,;\label{e2.17-sin}
\end{equation}
\begin{equation}\left.
\begin{array}{rl}
b(Y_t,t) &= b_0 (t)\,;\\[9pt]
    \si(Y_t, t)&= b_0(t) \nu(t) b_0(t)^{\mathrm{T}} = \si_0(t)\,,
    \end{array}
    \right\}\label{e2.18-sin}
    \end{equation}
    \begin{equation}
k_1^a (m_t, K_t, t) =\lk \left(\fr{\prt}{\prt m_t} \right)
    a_0 (m_t, K_t, t)^{\mathrm{T}}\rk^{\mathrm{T}}\,;\label{e2.19-sin}
    \end{equation}
    \begin{equation}
\dot m_t = a_1 (m_t, K_t, t) \,,\enskip m_0 = m(t_0)\,,\label{e2.20-sin}
\end{equation}

\vspace*{-12pt}

\noindent
\begin{multline}
\dot K_t = k_1^a (m_t, K_t, t) K_t + K_t k_1^a (m_t, K_t, t)^{\mathrm{T}}
    +\si_0(t)\,;\\
    K_0 = K(t_0)\,;
    \label{e2.21-sin}
    \end{multline}

    \vspace*{-12pt}

    \noindent
\begin{multline}
\fr{\prt K(t_1, t_2)}{\prt t_2} =
    K(t_1, t_2) k_{t_2} k_1^a (m_{t_2}, K_{t_2}, t_2)^{\mathrm{T}}\,;\\
    K(t_1, t_2) = K_{t_1}\,.
    \label{e2.22-sin}
\end{multline}

Для стационарных ДСтС~(\ref{e2.3-sin})
при условии асимптотической устойчивости матрицы $k_1^a (\bar m, \bar K)$
в основе МСЛ лежат уравнения~(\ref{e2.15-sin}), записанные в виде:
    \begin{gather}
    a_1 (\bar m, \bar K) =0\,; \label{e2.23-sin}\\
k_1^a (\bar m, \bar K) \bar K + \bar K k_1^a
(\bar m, \bar K)^{\mathrm{T}} +\bar \si_0 =0\,;\label{e2.24-sin}
\end{gather}

\vspace*{-12pt}

\noindent
\begin{multline}
k_\tau (\tau) = k_1^a (\bar m, \bar K)k(\tau)\,,\enskip
k(0) =\bar K \enskip (\forall \tau >0)\,,\\
k(\tau) = k (-\tau)^{\mathrm{T}} \enskip (\forall \tau <0)\,.
\label{e2.25-sin}
\end{multline}

Уравнения~(\ref{e2.4-sin})--(\ref{e2.8-sin})
лежат в основе МНА для ДСтС~(\ref{e2.1-sin}), а уравнения~(\ref{e2.17-sin})--(\ref{e2.22-sin})~---
в основе МСЛ для ДСтС~(\ref{e2.3-sin}). Для определения стационарных СтП
согласно МНА служат соотношения~(\ref{e2.15-sin}) и~(\ref{e2.16-sin}),
а МСЛ~--- (\ref{e2.17-sin})--(\ref{e2.25-sin}).

\section{Уравнения методов нормальной~аппроксимации и~статистической линеаризации
для~эредитарных стохастических систем, приводимых к~дифференциальным}

Рассмотрим ЭСтС, описываемую интегродифференциальным уравнением Ито
следующего вида~\cite{7-sin}:

\noindent
\begin{multline}
dX_t = \lk a(X_t, t) +\iii_{t_0}^t a_1 (X(\tau) ,\tau, t)\,d\tau\rk dt+{}\\
{}+\lk b(X_t, t) +\iii_{t_0}^t b_1 (X(\tau) ,\tau, t)\,d\tau\rk dW_0+{}\\
\hspace*{-1.5mm}{}+\!\!\iii_{R_0^q}\!\!\lk c(X_t, t,v) +\!\iii_{t_0}^t\! c_1 (X(\tau) ,\tau, t,v)\,d\tau\!\rk\! dP^0 (t, dv)
\!\!\!\!\label{e3.1-sin}
\end{multline}
с начальным условием  $X(t_0) = X_0$. В~(\ref{e3.1-sin})
сохранены обозначения разд.~2.

Функции $a=a(X_t, t)$, $a_1\hm = a_1(X (\tau),\tau, t)$,
$b\hm=b(X_t, t)$, $b_1\hm = b_1(X (\tau),\tau, t)$,
$c\hm=c(X_t,t,v)$ и $c_1\hm = c_1(X (\tau),\tau, t,v)$ имеют
соответственно размерности $p\times 1$, $p\times 1$, $p\times r$,
$p\times r$, $p\times 1$ и $p\times 1$ и допускают представления следующего вида:
\begin{equation}
\left.
\begin{array}{rl}
a_1&=A(t,\tau) \vrp (X(\tau) , \tau)\,;\\[9pt]
b_1&=B(t,\tau) \psi (X(\tau) ,  \tau)\,;\\[9pt]
c_1&=C(t,\tau) \chi (X(\tau) ,  \tau, v)\,.
\end{array}
\right\}
\label{e3.2-sin}
\end{equation}
Здесь эредитарные ядра $A\hm=A(t,\tau)\hm=\lk A_{ij}(t,\tau)\rk$
$(i,j\hm=\overline{1,p})$,
$B\hm=B(t,\tau)=\lk B_{i l}(t,\tau)\rk$ $(i\hm=\overline{1,p}$;
$l\hm=\overline{1,r})$ и $C\hm=C(t,\tau)=\lk C_{ij}(t,\tau)\rk$
$(i,j\hm=\overline{1,p})$ имеют соответственно размерности
$p\times p$, $p\times r$ и $p\times p$. Они удовлетворяют следующим условиям
физической реализуемости и асимптотического затухания:
\begin{multline}
A_{ij}(t,\tau)=0;\enskip B_{i l}(t,\tau)=0;\\[1pt]
C_{ij}(t,\tau)=0\enskip \forall \tau >t;\label{e3.3-sin}
\end{multline}

\vspace*{-12pt}

\begin{equation}
\left.
\hspace*{-3mm}\begin{array}{c}
\displaystyle\iin\! \lv A_{ij} (t,\tau) \rv d\tau <\infty\,;\
\displaystyle\iin\! \lv B_{i l} (t,\tau) \rv d\tau <\infty \,;\\[9pt]
\displaystyle\iin \!\lv C_{ij} (t,\tau) \rv d\tau <\infty\,.
\end{array}\!
\right\}\!
\label{e3.4-sin}
\end{equation}

В дальнейшем ограничимся случаем, когда эредитарные ядра удовлетворяют
линейным операторным уравнениям~\cite{6-sin, 5-sin, 7-sin}.

Нелинейные в общем случае функции $\vrp\hm=\vrp(X(\tau),\tau)$,
$\psi \hm=\psi(X(\tau), \tau)$, $\chi \hm=\chi (X(\tau),  \tau, v)$
отражают нелинейные свойства ЭСтС, зависят от  $X(\tau)$ и имеют размерности
$p\times 1$, $p\times p$, $p\times 1$ соответственно.

Важный класс  эредитарных ядер представляют собой
сингулярные (вырожденные) ядра, когда имеют место представления:
\begin{equation}
\left.
\hspace*{-3mm}\begin{array}{rl}
A_{ij} (t,\tau) &= A_{ij}^+(t) A_{ij}^-(\tau)\,;\\[9pt]
B_{i l} (t,\tau)& = B_{il}^+(t) B_{il}^-(\tau)\,;\\[9pt]
C_{ij} (t,\tau) &= C_{ij}^+ ( t) C_{ij}^- (\tau)\
(i,l= \overline{1,p}, j=\overline{1,r}).
\end{array}\!
\right\}\!\!
\label{e3.5-sin}
\end{equation}

В~\cite{6-sin, 5-sin, 7-sin} показано, что для дифференцируемых нелинейных
функций~$\vrp$, $\psi$, $\chi$ путем расширения вектора состояния за счет
инструментальных переменных, аппроксимируемых линейными операторными уравнениями,
определяющими эредитарные ядра в ЭСтС, (\ref{e3.1-sin})--(\ref{e3.4-sin})
приводятся к ДСтС вида~(\ref{e2.1-sin}) или~(\ref{e2.3-sin}).
В~случае недифференцируемых нелинейных функций~$\vrp$, $\psi$, $\chi$
ЭСтС~(\ref{e3.1-sin})--(\ref{e3.4-sin}) приводятся к~(\ref{e2.1-sin}) или~(\ref{e2.3-sin})
на основе аппроксимации вырожденными (сингулярными) ядрами~\cite{6-sin, 5-sin, 7-sin}.

Таким образом, после приведения ЭСтС~(\ref{e3.1-sin}) к ДСтС~(\ref{e2.1-sin})
или~(\ref{e2.3-sin}) можно воспользоваться уравнениями МНА и МСЛ разд.~2.

\section{Типовые интегралы методов нормальной аппроксимации и~статистической
линеаризации}

Как следует из уравнений~(\ref{e2.9-sin})--(\ref{e2.11-sin}),
для МНА необходимо уметь вычислять следующие интегралы:
\begin{multline}
I_0^a = I_0^a (m_t, K_t, t) = a_1 (m_t, K_t, t)={}\\
{}= M_N a(Y_t, t)\,;
\label{e4.1-sin}
\end{multline}

\vspace*{-12pt}

\noindent
\begin{multline}
I_1^a = I_1^a (m_t, K_t, t)= a_{21}(m_t, K_t, t)= {}\\
{}=M_N a(Y_t , t) Y_t^{0\mathrm{T}}\,;\label{e4.2-sin}
\end{multline}

\vspace*{-12pt}

\noindent
\begin{multline}
I_0^{\bar \si} = I_0^{\bar \si} (m_t, K_t, t) = a_{22}(m_t, K_t, t) ={}\\
{}= M_N \bar \si (Y_t, t)\,.\label{e4.3-sin}
\end{multline}
Для МСЛ достаточно вычислить интеграл~(\ref{e4.1-sin}),
причем интеграл~$I_1^a$ вычисляется по формуле~\cite{2-sin, 3-sin, 4-sin}:
\begin{equation*}
k_1^a = k_1^a (m_t, K_t, t)=\lk \left( \fr{\prt}{\prt m_t}\right)
I_0^a (m_t, K_t, t)^{\mathrm{T}}\rk^{\mathrm{T}}. %\label{e4.4-sin}
\end{equation*}

\medskip

\noindent
\textbf{Пример 1.} В~[1] для типовых степенных, тригоно\-мет\-ри\-че\-ских,
показательных и ку\-соч\-но-по\-сто\-ян\-ных нелинейностей $Z_t \hm=\vrp (Y_t, t)$
скалярного и векторного аргумента приведены формулы для интегралов
$I_0^\vrp \hm= I_0^\vrp (m_t^y, K_t^y, t)$, а также
$k_1^\vrp \hm= k_1^\vrp (m_t^y, K_t^y, t)$.

\medskip

\noindent
\textbf{Замечание.}
 Важно иметь в виду, что уравнения МНА (МСЛ) содержат интегралы
 $I_0^a$, $I_1^a$, $I_0^\si$ в виде соответствующих коэффициентов.
 Поэтому процедура вычисления интегралов должна быть согласована с
 методом численного решения обыкновенных дифференциальных уравнений для
 $m_t$, $K_t$ и $K(t_1, t_2)$. Эти коэффициенты допускают дифференцирование
 по~$m_t$ и~$K_t$, так как под интегралом стоит сглаживающая нормальная плотность.

\section{Сложные конечные и~дифференциальные нелинейности}

Важный класс сложных конечных нелинейностей (многомерных и векторного аргумента)
представляют собой сложные функции вида:
    \begin{equation*}
    \xi =\vrp (X_t, Y_t, t)\,,\enskip X_t =\psi (Y_t, t)\,. %\label{e5.1-sin}
    \end{equation*}
В~этом случае вычисление интегралов (см.\ разд.~4) проводится по совокупности
переменных  $\lk X_t^{\mathrm{T}} Y_t^{\mathrm{T}}\rk^{\mathrm{T}}$.
К таким нелинейностям, например, относятся дроб\-но-ра\-ци\-о\-наль\-ные,
иррациональные  нелинейности, выражаемые специальными функциями, многозначные
нелинейности, зависящие от СтП~$X_t$ и его производных~$\dot X_t$,  $\ddot X_t$
и~др.

\medskip

\noindent
\textbf{Пример 2.}
Рассмотрим вычисление интегралов~(\ref{e4.1-sin}) и~(\ref{e4.2-sin})
для сложных одномерных иррациональных нелинейностей скалярного аргумента
\begin{equation}
\vrp (Y_t, t) =\lv Y_t\rrv^{\alpha-1}\, \mathrm{sgn}\, Y_t
\label{e5.2-sin}
\end{equation}
($\alpha$~--- нецелый показатель).

Пользуясь~(\ref{e2.16-sin}) и~(\ref{e2.19-sin}), представим~(\ref{e5.2-sin}) в виде
\begin{equation*}
\vrp(Y_t, t) = \vrp_0 (m_t, D_t, t) + k_1^\vrp(m_t, D_t, t) Y_t^0. %\label{e5.3-sin}
\end{equation*}
Здесь введены следующие обозначения:
\begin{gather*}
\vrp_0(m_t, D_t, t) =\Gamma(\alpha) D_t^{1/2} e^{-\xi^2/4} D_{-\alpha} (\xi)\,;%\label{e5.4-sin}
\\
k_1^a (m_t, D_t, t) =\fr {\prt \vrp_0(m_t, D_t, t)}{\prt m_t}\,,%\label{e5.5-sin}
\end{gather*}
где  $\Gamma(\alpha)$~--- гамма-функция,  $\xi \hm= m_t/\sqrt{D_t}$~---
отношение <<сиг\-нал--шум>>; $D_{-\alpha} (\xi)$~---
функция параболического цилиндра~\cite{9-sin}.
При вычислении были учтены следующие соотношения~\cite{9-sin, 8-sin}:
\begin{multline}
\iii_0^\infty x^{\alpha-1} e^{-\beta x^2 - \gamma x} \,dx ={}\\
{}=
(2\beta)^{-\alpha/2} \Gamma(\alpha) \exp \left(\fr{\gamma^2}{8\beta}\right)
D_{-\alpha} \left(\fr{\gamma}{\sqrt{2\beta}}\right)\,;\label{e5.6-sin}
\end{multline}

\vspace*{-12pt}

\noindent
\begin{multline}
\fr{dD_\rho(\xi)}{d\xi} =
   -\fr{\xi}{2}\, D_\rho (\xi) -\rho D_{\rho-1} (\xi) =
   \fr{\xi}{2}\, D_\rho (\xi) -{}\\
   {}- D_{\rho+1} (\xi) \enskip
   (\mathrm{Re}\, \beta>0\,,\enskip \mathrm{Re}\,\alpha>0\,,\enskip
   \rho=-\alpha)\,.\label{e5.7-sin}
   \end{multline}

Соотношения~(\ref{e5.6-sin}) и~(\ref{e5.7-sin})
могут быть использованы также для вычисления интегралов~(\ref{e4.3-sin}).

\medskip

\noindent
\textbf{Замечание.}
Для вычисления интегралов $I_0^a$, $I_1^a$ и $I_0^{\bar \si}$
применительно к типовым иррациональным нелинейностям вида
    $\lv Y_t\rrv^{\alp-1} e^{\delta Y_t}$, $\lv Y_t\rrv^{\alp-1}  \cos \w Y_t$,
    $\lv Y_t\rrv^{\alp-1}  \sin \w Y_t$
и более общим нелинейностям \mbox{вида}
    \begin{equation*}
    \vrp (Y_t, t) =\Phi^\vrp \left( \lv Y_t\rrv^{\alpha-1}, t\right) %\label{e5.8-sin}
    \end{equation*}
можно рекомендовать известные численные методы вычисления функций на ЭВМ~\cite{8-sin}.

\smallskip

\noindent
\textbf{Пример 3.}
Для нелинейной дроб\-но-ра\-ци\-о\-наль\-ной функции

\noindent
\begin{equation*}
\vrp (Y_t, t) = \fr{a}{(b+Y_t)^2} %\label{e5.9-sin}
\end{equation*}
имеем

\vspace*{-3pt}

\noindent
\begin{gather*}
\vrp_0 (m_t, D_t, t) =a b^{-2} \lk 1+ \chi (m_t, D_t, t)\rk\,; %\label{e5.10-sin}
\\
k_1^\vrp (m_t, D_t, t) =  a b^{-2}\fr{\prt \chi (m_t, D_t, t)}{\prt m_t}\,. %\label{e5.11-sin}
\end{gather*}
Здесь

\vspace*{-3pt}

\noindent
\begin{multline*}
\chi (m_t, D_t, t) ={}\\
{}=\sss_{n=1}^\infty \sss_{l=0}^{E(n/2)}
\fr{(-1)^n (n+1) n!}{(n-2l)! (2l)!}\, b^{-n} m_t^n \left( \fr{D_t}{ 2 m_t^2}
\right)^l, %\label{e5.12-sin}
\end{multline*}
где  $E(n/2)$~--- целая часть~$n/2$; $a\hm=a(t)$; $b\hm= b(t)$.

\vspace*{-6pt}

\section{Сложные интегральные нелинейности}

\vspace*{-2pt}

Пусть сначала векторно-матричная нелинейность имеет эредитарный характер, т.\,е.\
\begin{equation}
\underline{\vrp} (Y_t, t) =\iii_{t_0}^t A(t,\tau) \vrp (Y(\tau), \tau) \,d\tau\,.
\label{e6.1-sin}
\end{equation}
Тогда, как показано в~\cite{6-sin, 5-sin, 7-sin}, следует соответст\-ву\-ющие
интегродифференциальные соотношения путем введения  инструментальных
переменных привести к дифференциальным соотношениям.  Для
дифференцируемых функций~$\vrp$ и асимптотически устойчивых ядер
$A(t,\tau)$ зависимость~(\ref{e3.5-sin}) имеет следующий дифференциальный вид:
\begin{equation*}
F^A (t, D) \underline{\vrp} (Y_t, t) = H^A (t, D) \vrp (Y_t, t)\,. %\label{e6.2-sin}
\end{equation*}
Здесь $F^A (t, D)$ и  $H^A (t, D)$~--- линейные дифференциальные операторы $(D\hm= d/dt)$.

Для недифференцируемых функций~$\vrp$ и асимптотически устойчивых
сингулярных ядер~(\ref{e3.5-sin}) используются соотношения:
\begin{equation*}
\underline{\vrp} (Y_t, t) = A^+ Z\,,\enskip
\dot Z = A^- \vrp\,,\enskip
Z(t_0)=0\,. %\label{e6.3-sin}
\end{equation*}

Многочисленные примеры аналитического моделирования ЭСтС можно найти
в~[1--3, 5, 7, 10, 11].

Как отмечалось в~\cite{3-sin}, часто наряду с интегральными
нелинейностями~(\ref{e6.1-sin}) рассматривают нелинейности вида:

\columnbreak

\noindent
\begin{equation*}
Z_s =\sss_{\rho=1}^R \mathcal{A}_\rho \vrp_\rho (Y_{t_1}\tr Y_{t_r})\,, %\label{e6.2-sin}
\end{equation*}
где $\mathcal{A}_1 \tr \mathcal{A}_R$~--- произвольные линейные операторы,
действующие над функциями~$r$ переменных  $t_1\tr t_r$; $\vrp_\rho
\hm=\vrp_\rho (Y_{t_1} \tr Y_{t_r})$~--- линейные функции отмеченных
переменных. Такие нелинейности называются приводимыми к линейным.
Важным частным случаем~(\ref{e6.1-sin}) являются интегральные нелинейности вида:

\noindent
\begin{gather}
Z_s =\iii_T \vrp (Y_t, t, s)\, dt\,; \notag%\label{e6.3-sin}
\\
Z_s =\!\iii_T \!\cdots\!\iii_T\! \vrp (Y_{t_1}\tr Y_{t_r}; t_1\tr t_r, s)\,dt_1
\ldots dt_r,\notag %\label{e6.4-sin}
\end{gather}
В этом случае используется МСЛ по совокупности переменных  $Y_{t_1} \tr Y_{t_r}$.

\vspace*{-9pt}

\section{Заключение}

\vspace*{-2pt}

Разработаны методы и алгоритмы МНА и МСЛ для ДСтС и ЭСтС,
приводимых к ДСтС со сложными конечными, дроб\-но-ра\-ци\-о\-наль\-ны\-ми,
иррациональными, а также дифференциальными и интегральными нелинейностями.
Приведены примеры.

Результаты допускают обобщение на случай ДСтС и ЭСтС со
стохастическими нелинейностями, заданными каноническими разложениями и
интегральными каноническими  представлениями~\cite{1-sin, 3-sin, 11-sin}.

\vspace*{-9pt}

{\small\frenchspacing
 {%\baselineskip=10.8pt
 \addcontentsline{toc}{section}{References}
 \begin{thebibliography}{99}

 \vspace*{-2pt}

\bibitem{1-sin}
\Au{Синицын И.\,Н.,  Синицын~В.\,И.}
Лекции по нормальной и эллипсоидальной аппроксимации распределений в
стохастических сис\-те\-мах.~--- М.: ТОРУС ПРЕСС, 2013. 488~с.

\bibitem{6-sin} %2
\Au{Синицын И.\,Н. }
Stochastic hereditary control systems~// Проблемы управления и
теории информации, 1986. Т.~15. №\,4. С.~287--298.

\bibitem{2-sin} %3
\Au{Пугачев В.\,С., Синицын~И.\,Н.}
Стохастические дифференциальные сис\-те\-мы. Анализ и фильтрация.~--- М.:
Наука,  1990.  632~с. [Англ. пер.
 Stochastic differential systems.
Analysis and filtering.~--- Chichester, New York: Jonh Wiley, 1987.
549~p.].

\bibitem{5-sin} %4
\Au{Синицын И.\,Н. }
Конечномерные распределения процессов в стохастических интегральных
и интегродифференциальных системах~// Preprints of the 2nd IFAC
Symposium on Stochastic Control.~--- Vilnius: Pergamon Press,
1987.  Vol.~1. P.~144--153.

\bibitem{3-sin} %5
\Au{Пугачев В.\,С., Синицын~И.\,Н.}
Теория стохастических систем.~--- М.: Логос, 2000; 2004. 1000~с.
[Англ. пер.\linebreak\vspace*{-12pt}

\pagebreak

\noindent Stochastic systems. Theory and  applications.~---
Singapore: World Scientific, 2001. 908~p.].

\bibitem{4-sin} %6
\Au{Синицын И.\,Н.}
Параметрическое статистическое и аналитическое моделирование распределений
в нелинейных стохастических сис\-те\-мах на многообразиях~//
Информатика и её применения, 2013. Т.~7. Вып.~2. С.~4--16.

\bibitem{7-sin} %7
\Au{Синицын И.\,Н. }
Анализ и моделирование распределений в эредитарных стохастических
сис\-те\-мах~// Информатика и её применения, 2014. Т.~8. Вып.~1.\linebreak
С.~2--11.



\bibitem{9-sin} %8
\Au{Градштейн И.\,С., Рыжик~И.\,М.}
Таблицы интегралов, сумм, рядов и произведений.~--- М.: ГИФМЛ, 1963. 1100~с.

\bibitem{8-sin} %9
\Au{Попов Б.\,А., Теслер~Г.\,С. }
Вычисление функций на ЭВМ: Справочник.~--- Киев: Наукова Думка, 1984. 599~с.


\bibitem{11-sin} %10
\Au{Синицын И.\,Н.}
Канонические представления случайных функций и их применение в
задачах компьютерной поддержки научных исследований.~--- М.: ТОРУС
ПРЕСС, 2009. 768~с.

\bibitem{10-sin} %11
\Au{Синицын И.\,Н., Синицын~В.\,И., Корепанов~Э.\,Р., Белоусов~В.\,В.,
Сергеев~И.\,В., Басилашвили~Д.\,А.}
Опыт моделирования эредитарных стохастических сис\-тем~//
Кибернетика и высокие технологии XXI века: Сб. докл.  XIII Междунар.
науч.-технич. конф.~--- Воронеж: Саквоее, 2012. Т.~2. C.~346--357.

 \end{thebibliography}

 }
 }

\end{multicols}

\vspace*{-9pt}

\hfill{\small\textit{Поступила в редакцию 05.05.14}}

%\newpage

\vspace*{12pt}

\hrule

\vspace*{2pt}

\hrule

\vspace*{12pt}

\def\tit{ANALYTICAL MODELING OF NORMAL PROCESSES
 IN~STOCHASTIC SYSTEMS WITH~COMPLEX NONLINEARITIES}

\def\titkol{Analytical modeling of normal processes
 in~stochastic systems with~complex nonlinearities}

\def\aut{I.\,N.~Sinitsyn and V.\,I.~Sinitsyn}

\def\autkol{I.\,N.~Sinitsyn and V.\,I.~Sinitsyn}

\titel{\tit}{\aut}{\autkol}{\titkol}

\vspace*{-9pt}

\noindent
Institute of Informatics Problems, Russian Academy of Sciences,
44-2 Vavilov Str., Moscow 119333, Russian Federation


\def\leftfootline{\small{\textbf{\thepage}
\hfill INFORMATIKA I EE PRIMENENIYA~--- INFORMATICS AND
APPLICATIONS\ \ \ 2014\ \ \ volume~8\ \ \ issue\ 3}
}%
 \def\rightfootline{\small{INFORMATIKA I EE PRIMENENIYA~---
INFORMATICS AND APPLICATIONS\ \ \ 2014\ \ \ volume~8\ \ \ issue\ 3
\hfill \textbf{\thepage}}}

\vspace*{6pt}

\Abste{Differential stochastic systems (DStS) with Wiener and Poisson
noises and complex finite, differential, and  integral nonlinearities and
hereditary StS reducible to DStS are considered. Equations and algorithms
of analytical modeling based on the normal approximation method (NAM) and the
statistical linearization method (SLM) are given. The case of complex
continuous and discontinuous nonlinearities of scalar and vector arguments
is considered. Special attention is paid to NAM (SLM) typical integrals
for finite rational and irrational nonlinear and integral scalar and vector
nonlinear functions. The general case of integral nonlinearities reducible to
linear is considered. Test examples are given.}

\KWE{analytical modeling;
complex finite differential and integral nonlinearities;
complex irrational nonlinerarites
differential stochastic system with Wiener and Poisson noises;
method of normal approximation;
method of statistical linearization;
hereditary stochastic systems reducible to differential}

\DOI{10.14357/19922264140302}

  \begin{multicols}{2}

\renewcommand{\bibname}{\protect\rmfamily References}
%\renewcommand{\bibname}{\large\protect\rm References}

{\small\frenchspacing
 {%\baselineskip=10.8pt
 \addcontentsline{toc}{section}{References}
 \begin{thebibliography}{99}



\bibitem{1-sin-1}
\Aue{Sinitsyn, I.\,N., and  V.\,I.~Sinitsyn}.  2013.
Lektsii po normal'noy i ellipsoidal'noy approksimatsii raspredeleniy
v stokhasticheskikh sistemakh [Lectures on normal and ellipsoidal
approximation of distributions in stochastic systems].
Moscow: TORUS PRESS. 488~p.

\bibitem{6-sin-1} %2
\Aue{Sinitsyn, I.\,N.}  1986.
{Stochastic hereditary control systems}.
\textit{Problems Control Inform. Theory} 15(4):287--298.

\bibitem{2-sin-1} %3
\Aue{Pugachev, V.\,S., and  I.\,N.~Sinitsyn}.  1987.
\textit{Stochastic differential systems. Analysis and filtering.}
Chichester, New York: Jonh Wiley. 549~p.

\bibitem{5-sin-1} %4
\Aue{Sinitsyn, I.\,N.}  1987.
Konechnomernye raspredeleniya protsessov v stokhasticheskikh integral'nykh
i in\-teg\-ro\-dif\-fe\-ren\-tsial'nykh sistemakh [Finite dimensional distributions
of processes in stochastic integral and integrodifferential systems].
\textit{2nd  Symposium (International) IFAC on Stochastic Control
Preprints}. Vilnius: Pergamon Press. 1:144--153.

\bibitem{3-sin-1} %5
\Aue{Pugachev, V.\,S., and I.\,N.~Sinitsyn}. 2001.
\textit{Stochastic systems. Theory and  applications}.
Singapore: World Scientific. 908~p.

\bibitem{4-sin-1} %6
\Aue{Sinitsyn, I.\,N.}  2013.
Parametricheskoe statisticheskoe i analiticheskoe modelirovanie
raspredeleniy v nelineynykh stokhasticheskikh sistemakh na mnogoobraziyakh
[Parametric statistical and analytical modeling of distributions in
stochastic systems on manifolds].
\textit{Informatika i ee Primeneniya}~--- \textit{Inform. Appl.} 7(2):4--16.


\bibitem{7-sin-1} %7
\Aue{Sinitsyn, I.\,N.}  2014.
Analiz i modelirovanie raspredeleniy v ereditarnykh stokhasticheskikh sistemakh
[Analysis and modeling of distributions in hereditary stochastic systems].
\textit{Informatika i ee Primeneniya}~--- \textit{Inform. Appl.} 8(1):2--11.

\bibitem{9-sin-1} %8
\Aue{Gradshteyn, I.\,S., and I.\,M.~Ryzhik}.  1963.
\textit{Tablitsy integralov, summ, ryadov i proizvedeniy}
[Tables of integrals, sums, series, and products]. Moscow:  GIFML.   1100~p.

\pagebreak

\bibitem{8-sin-1} %9
\Aue{Popov, B.\,A., and G.\,S.~Tesler}.  1984.
\textit{Vychislenie funktsiy na EVM}. Spravochnik [Computing of functions].
Kiev: Naukova Dumka.  599~p.


\bibitem{11-sin-1} %10
\Au{Sinitsyn, I.\,N.} 2009.
\textit{Kanonicheskie predstavleniya sluchaynykh funktsiy i ikh primenenie v
zadachakh komp'yuternoy podderzhki nauchnykh issledovaniy}
[Canonical expansions of random functions and its application to
scientific computer-aided support]. Moscow: TORUS PRESS. 768~p.

\bibitem{10-sin-1} %11
\Aue{Sinitsyn, I.\,N., V.\,I.~Sinitsyn, E.\,R.~Korepanov,
V.\,V.~Belousov, I.\,V.~Sergeev, and D.\,A.~Basilashvili}.
2012. Opyt modelirovaniya ereditarnykh stokhasticheskikh sistem
[Experience of modeling in hereditary stochastic systems].
\textit{Kibernetika i Vysokie Tekhnologii XXI~Veka:
Sbornik dokladov  XIII Mezhdunar. nauch.-tekhnich. konf.}
[Cybernatics ans High Technologies of the XXI Century: Materials of
XIII  Scientific and Technological Conference (International)].
Voronezh: Sakvoee. 2:346--357.

\end{thebibliography}

 }
 }

\end{multicols}

\vspace*{-6pt}

\hfill{\small\textit{Received May 05, 2014}}

\vspace*{-18pt}

\Contr

\noindent
\textbf{Sinitsyn Igor N.} (b.\ 1940)~---
Doctor of Science in technology, professor, Honored scientist of RF, Head of Department, Institute of
Informatics Problems, Russian Academy of Sciences,
44-2 Vavilov Str., Moscow 119333, Russian
Federation; sinitsin@dol.ru

\vspace*{3pt}

\noindent
\textbf{Sinitsyn Vladimir I.} (b.\ 1968)~--- Doctor of Science in physics
and mathematics, associate professor, Head of Department, Institute of
Information Problems, Russian Academy of Sciences,
44-2 Vavilov Str., Moscow 119333, Russian Federation; VSinitsin@ipiran.ru




\label{end\stat}

\renewcommand{\bibname}{\protect\rm Литература} %1
\newcommand{\La}{\Lambda}

\def\stat{dulch}

\def\tit{О ТОЧНОСТИ НЕКОТОРЫХ МАТЕМАТИЧЕСКИХ МОДЕЛЕЙ КАТАСТРОФИЧЕСКИ
НАКАПЛИВАЮЩИХСЯ ЭФФЕКТОВ ПРИ ПРОГНОЗИРОВАНИИ РИСКА ЭКСТРЕМАЛЬНЫХ
СОБЫТИЙ$^*$}

\def\titkol{О точности некоторых математических моделей катастрофически
накапливающихся эффектов} % при прогнозировании риска экстремальных событий}

\def\autkol{И.\,А.~Дучицкий, В.\,Ю.~Королев, И.\,А.~Соколов}

\def\aut{И.\,А.~Дучицкий$^1$, В.\,Ю.~Королев$^2$, И.\,А.~Соколов$^3$}

\titel{\tit}{\aut}{\autkol}{\titkol}

{\renewcommand{\thefootnote}{\fnsymbol{footnote}}\footnotetext[1]
{Работа поддержана Российским фондом фундаментальных
исследований (проекты 11-01-00515а, 11-01-12026-офи-м,
12-07-00109a, 12-07-00115a) и Министерством образования и науки (госконтракт
16.740.11.0133).}}

\renewcommand{\thefootnote}{\arabic{footnote}}
\footnotetext[1]{Факультет вычислительной
математики и кибернетики Московского государственного университета
им.\ М.\,В.~Ломоносова; duchik@gmail.com}
\footnotetext[2]{Факультет вычислительной математики и кибернетики
Московского государственного университета им.\ М.\,В.~Ломоносова;
Институт проблем информатики Российской академии наук;
vkorolev@cs.msu.su}
\footnotetext[3]{Институт проблем 
информатики Российской академии наук; ipiran@ipiran.ru}

\vspace*{-6pt}


\Abst{Построены оценки точности приближения
распределений экстремумов специальных случайных сумм масштабными
смесями полунормальных законов и обсуждается возможность
использования этих результатов при прогнозировании риска
экстремальных событий, вызванных катастрофически накапливающимися
неблагоприятными эффектами.}

\KW{неоднородные потоки событий; дважды
стохастический пуассоновский процесс; отрицательное биномиальное
распределение; гам\-ма-рас\-пре\-де\-ле\-ние; оценка скорости сходимости}

\vskip 14pt plus 9pt minus 6pt

      \thispagestyle{headings}

      \begin{multicols}{2}

            \label{st\stat}

\section{Введение}

По мере возрастания технической и информационной оснащенности
человечества вопросы\linebreak оценивания и прогнозирования надежности
технических и информационных систем, а стало быть, и рисков,
связанных с их отказами, приобретают все более важное значение.
Цивилизация становит\-ся\linebreak все более и более зависимой от возможностей,\linebreak
предоставляемых современными техническими (транспортными,
энергетическими, оборонными) и информационными
(телекоммуникационными, вычислительными) сис\-те\-ма\-ми. Одновременно
возрастает и зависимость человечества от {\it на\-деж\-ности и
безопасности} технических и информационных систем. В~силу этих
обстоятельств, естественно, возрастают требования к адекватности
математических моделей, используемых для вычисления соответствующих
показателей надежности и возможных рисков.

Многие классические методы оценки риска или показателей надежности,
разработанные, как правило, в середине XX~в., основаны на
идеальных\linebreak предположениях о том, что параметры, характеризующие,
скажем, воздействие внешней среды,\linebreak имеют нормальное распределение, а
параметры,\linebreak
 характеризующие на\-деж\-ность составных час\-тей изучаемой
системы, например время жизни (на\-ра\-бот\-ки на отказ), имеют
показательное (экс-\linebreak по\-нен\-ци\-аль\-ное) распределение и более общее
распределе\-ние Вей\-бул\-ла--Гне\-ден\-ко. Однако, к сожалению, за\-час\-тую
применение классических методов приводит к недооценке риска
катастроф или отказов. Причины иногда имеющей место
несостоятельности классических моделей могут быть разными. 
К~примеру, если показатели на\-деж\-ности вычисляются на осно\-ве
статистических данных, накопленных за определенное время, то, как
показано, например, в~\cite{BKSSh2007}, существенную роль играет
однородность потока событий, в результате которых накапливаются
статистические данные. 

Другими словами, критичным для адекват\-ности
классических моделей является асимптотическое постоянство отношения
количества зарегистрированных в течение определенного интервала
времени экстремальных событий к длине этого интервала времени при
неограниченном увеличении по\-следней. 

Если асимптотическое
постоянство указанного отношения, т.\,е.\ его сближение с некоторым
чис\-лом, имеет место, то классические модели могут давать адекватные
результаты. Однако если такого сближения не наблюдается и указанное
отношение сильно колеблется, оставаясь случайным (т.\,е.\
непредсказуемым), то классические модели неадекватны и могут
приводить к весьма существенной недооценке риска. В~частности,
вместо ожидаемого в соответствии с классической теорией нормального
закона в подобных ситуациях (например, если упомянутое выше
отношение ведет себя как гам\-ма-рас\-пре\-де\-лен\-ная случайная величина)
могут возникать, скажем, функции распределения ущерба типа
распределения Стьюдента с произвольно малым числом степеней свободы
или так называемые дисперсионные гам\-ма-рас\-пре\-де\-ле\-ния, в част\-ности
распределение Лапласа~[2--4].

Негативный эффект нежелательных воздействий может проявиться
мгновенно, а может накапливаться постепенно. Это приводит к
не\-обхо-\linebreak ди\-мости наряду с самим потоком экстремальных\linebreak событий
рассматривать также и процесс, описывающий <<накопленные>>,
негативные воздействия. 

Под {\it катастрофой} (катастрофическим
событием) разумно подразумевать превышение модельным процессом
некоторого критического уровня, который, например, может быть
определен как минимальный уровень, превышение которого накопленным
негативным воздействием на рассматриваемую систему (информационную,
техническую, экологическую, социальную) ведет к необратимым\linebreak ее
изменениям, в частности к невозможности выполнять системой свои
функции в прежнем ре\-жиме.

Как показано в~\cite{KorolevSokolov2008}, если интенсивность пото\-ка
информативных событий случайна и имеет, скажем, стандартное
экспоненциальное распределение, то величина максимума накопленных
эффектов также имеет экспоненциальное распределение (но с параметром
$\sqrt{2}$), а не ожидаемое в соответствии с классической теорией
распределение $G(x)\hm=2\Phi(\max\{0,x\})-1$ максимума винеровского
процесса на единичном отрезке, хвост которого убывает при
$x\hm\to\infty$ как $\sqrt{2/\pi}x^{-1}e^{-x^2/2}$.  Хвост же
упомянутой выше экспоненциальной функции распределения, естественно,
убывает существенно медленнее~--- как $e^{-\sqrt{2}x}$. Это приводит
к существенному различию в оценках вероятностей катастрофических
воздействий и размеров самих катастрофических воздействий,
получаемых на основе этих двух функций распределения. К~примеру,
квантиль порядка 0,99 функции распределения $G(x)$ равна 2,576. В~то
же время квантиль того же порядка экспоненциальной функции
распределения с па\-ра\-мет\-ром~$\sqrt{2}$ примерно равна 3,256.
Вероятность превышения порога~2,576, <<критического>> при функции
распределения~$G(x)$, максимальным суммарным воздействием, име\-ющим
указанную показательную функцию распределения, превышает 0,026, т.\,е.\ 
оказывается более чем в два с половиной раза выше, чем
предполагаемая по классической модели. 

Этот %\linebreak 
пример является еще
одной наглядной иллюстрацией того, насколько можно недооценить риск
ка\-та\-ст\-роф, вызванных накапливающимися эффектами неблагоприятных
воздействий, если не принимать во внимание стохастический характер
интенсивности потока информативных со-\linebreak бытий.

Естественно предположить, что негативные эффекты накапливаются как
результат неких событий, хаотически рассредоточенных по
временн$\acute{\mbox{о}}$й оси, т.\,е.\ образующих так называемый хаотический
поток событий. Как известно (см., например,~\cite{KorolevSokolov2008}),\linebreak 
наилучшей моделью однородного
хаотического\linebreak потока событий является пуассоновский процесс,
характеризуемый тем обстоятельством, что интервалы времени между
событиями потока независимы и имеют одинаковое показательное
распределение. Привлекательность пуассоновского процесса\linebreak в качестве
модели однородного дискретного хаоса обусловлена как минимум двумя
обстоятельствами. 

Во-пер\-вых, показательное распределение ин\-тер\-ва\-лов
времени между событиями пуассоновского потока обладает максимальной
дифференциальной энтропией среди всех абсолютно непрерывных
распределений вероятностей, сосредоточенных на всей положительной
полуоси и имеющих конечное математическое ожидание, а энтропия, как
известно, является очень удобной численной характеристикой
неопределенности. 

Во-вто\-рых, точки (события) пуассоновского потока
равномерно распределены на оси времени в том смысле, что для любого
конечного интервала времени $[t_1,t_2]$ условное совместное
распределение точек пуассоновского потока, попавших в интервал
$[t_1,t_2]$, при условии, что в этот интервал попало фиксированное
число, скажем $n$, точек, совпадает с совместным распределением
вариационного ряда, построенного по независимой однородной выборке
объема $n$ из равномерного на $[t_1,t_2]$ распределения. Равномерное
же распределение обладает\linebreak максимальной дифференциальной энтропией
среди всех абсолютно непрерывных распределений вероят\-ностей,
сосредоточенных на конечных интервалах, и очень хорошо соответствует
общепринятому представлению об абсолютно непредсказуемой
ограниченной случайной величине.

Однако в реальных <<хаотических>> системах хаос практически никогда
не бывает однородным в пространстве или времени. Неоднородный и даже
стохастический характер интенсивности потока информативных событий
может быть обусловлен случайно возникающими (не поддающимися
абсолютно надежному прогнозированию) причинами. Как известно,
наиболее разумными стохастическими моделями неоднородных хаотических
точечных процессов являются {\it дважды стохастические пуассоновские
процессы}, иначе называемые {\it процессами Кокса} (см., например,~\cite{BeningKorolev2002}).

Для эффективной реализации мер, направленных на повышение надежности
и катастрофоустойчивости технических и информационных сис\-тем,
необходимо уметь вычислять вероятностные характеристики возможных
экстремальных воздействий на рассматриваемую систему. Как уже
отмечалось выше, при непостоянной (и тем более при стохастической)
интенсивности потока экстремальных событий статистические
закономерности нежелательных воздействий существенно отличны от
того, какими они были бы в однородной ситуации, описываемой
классической теорией экстремальных значений. Для вычисления таких
мер риска, как квантильные (типа показателей VaR~--- <<Value at
Risk>>), необходимо иметь более или менее точные аппроксимации для
вероятностных распределений величин экстремальных воздействий.

\section{Предельное поведение экстремумов обобщенных дважды стохастических
пуассоновских процессов}

В качестве базовой модели <<накапливающегося\linebreak
 воздействия>> в данной
статье рассматриваются экстремумы обобщенных дважды стохастических
пуассоновских процессов, представляющих собой суммы случайного числа
независимых одинаково распреде\-лен\-ных случайных величин, в которых
число слагаемых определяется значением дважды стохастического
пуассоновского процесса. При этом под катастрофическим
неблагоприятным воздействием будет пониматься превышение случайным
блужданием, порожденным базовой последовательностью независимых
случайных величин, некоторого заданного уровня. Простейшей задачей
оценивания или прогнозирования рисков, связанных с такими событиями,
является задача вычисления ве\-ро\-ят\-ности превышения возможным
значением максимума накапливающихся эффектов критического уровня в
течение некоторого фиксированного интервала времени.
Целесообразность рассмотрения подобных моделей при прогнозировании
рисков катастроф диктуется следующими примерами.

\medskip

\noindent
\textbf{Пример 1}. Эволюция финансовых индексов хорошо описывается
(неоднородным) случайным блуж\-да\-ни\-ем. При этом, как известно, если
изменение этого индекса в течение биржевого дня будет слишком
большим, то, чтобы избежать слишком больших потерь (т.\,е.\
финансовых катастроф), торги автоматически прекращаются. Другими
словами, если экстремум процесса, описывающего динамику финансового
индекса, достигает критического значения, то торги прекращаются. 
В~такой задаче временн$\acute{\mbox{ы}}$м горизонтом, на который строится прогноз,
равен операционному дню, а элементарные слагаемые в сумме~--- это
приращения процесса за единицы времени (например, минуты). На
сегодняшний день весьма актуальным примером подобной задачи является
вопрос принятия Банком России (Центробанком) решения о проведении
валютной (долларовой) интервенции, чтобы остановить процесс падения
курса рубля относительно доллара при опасном приближении курса к
верхней границе объявленного допустимого коридора. Здесь
представляет интерес возможность приближения курса к границе не в
какой-то определенный момент времени, а в течение операционного дня,
при этом риск, связанный с таким приближением, оценивается заранее
(например, после получения информации об эволюции курса в течение
предыду\-ще\-го операционного дня).

\medskip

\noindent
\textbf{Пример 2}. Для снабжения некоторой отрасти некоторого региона в
течение фиксированного периода времени (например, квартала или
зимнего периода), на склад (хранилище) выделяется определенное
количество некоторого ресурса (скажем, топлива). Если {\it
суммарные, накопленные} расходы этого ресурса в течение указанного
времени превысят выделенный лимит, то в данной отрасти в данном
регионе наступит катастрофический коллапс. При этом естественно
предположить, что ресурс отпускается потребителям партиями, вообще
говоря, случайного объема согласно запросам, возникающим, вообще
говоря, в случайные моменты времени. При этом горизонтом
прогнозирования, естественно, считается интервал времени между
поставками (например, квартал).

\medskip

\noindent
\textbf{Пример 3}. При проектировании дамб и водохранилищ необходимо
учитывать то обстоятельство, что количество воды в рассматриваемом
резервуаре (водохранилище, бассейне реки, озере и~т.\,п.)\ изменяется
случайным образом: оно увеличивается за счет выпадения осадков (в
случайные моменты времени) и уменьшается за счет испарения. Если
экстремум уровня превысит критический уровень, то происходят события
катастрофического характера. Избыток воды вызывает наводнения, ее
недостаток~--- засуху. Ясно, что в силу естественных циклических
причин в качестве горизонта прогнозирования разумно взять год или
солнечный цикл (11~лет). При этом уровень воды в каждый момент
времени является суммой приращений этого уровня за каждый из
предыдущих дней.

\medskip

\noindent
\textbf{Определение~1.} Случайный процесс $\Lambda(t)$, $t\hm\geqslant 0$, с
неубывающими непрерывными справа траекториями, удовлетворяющий
условиям $\Lambda(0)\hm=0$, ${\sf P}(\Lambda(t)\hm<\infty)=1$ $(0\hm<t\hm<\infty)$,
называется {\it случайной мерой}.

\smallskip

\noindent
\textbf{Определение 2.} Пусть $N_1(t)$~--- стандартный пуассоновский
процесс, $\Lambda(t)$~--- случайная мера, независимая от $N_1(t)$.
Случайный процесс $N(t)\hm=N_1(\Lambda(t))$ называется {\it дважды
стохастическим пуассоновским процессом} (или {\it процессом Кокса}).
В~таком случае говорят, что процесс Кокса $N(t)$ управляется
процессом $\La(t)$ (или что процесс $\La(t)$ контролирует процесс
Кокса $N(t)$).

\smallskip

В частности, если процесс $\Lambda(t)$ допускает представление
$$
\Lambda(t)=\il{0}{t}\lambda(\tau)\,d\tau\,,\enskip t\geqslant 0\,,
$$
в котором $\lambda(t)$~--- положительный случайный процесс с
интегрируемыми траекториями, то $\lambda(t)$ можно
интерпретировать как мгновенную стохастическую интенсивность
процесса $N(t)$. Поэтому \mbox{иногда} процесс $\Lambda(t)$, управ\-ля\-ющий
процессом Кокса $N(t)$, будет называться {\it накопленной
ин\-тен\-сив\-ностью} процесса~$N(t)$.

Несложно убедиться, что для процесса Кокса $N(t)$, управляемого
процессом $\Lambda(t)$, справедливы соотношения
$$
{\sf E}N(t)={\sf E}\Lambda(t)\,,\enskip{\sf D}N(t)={\sf
E}\Lambda(t)+{\sf D}\Lambda(t)\,.
$$

\smallskip

\noindent
\textbf{Определение 3.} Пусть $X_1,X_2,\ldots$~--- одинаково
распределенные случайные величины. Предположим, что при каждом $t\hm\geqslant
 0$ случайные величины $N(t),X_1,X_2,\ldots$ независимы. Процесс
\begin{equation}
S(t)=\sum\limits_{j=1}^{N(t)}X_j\,,\enskip t\geqslant 0\,,
\label{e1-dul}
\end{equation}
назовем обобщенным процессом Кокса (при этом для определенности
считаем, что $\sum\limits_{j=1}^{0} =0$).

\smallskip

Процессы вида~(\ref{e1-dul}) играют чрезвычайно важную роль во многих
прикладных задачах. Достаточно сказать, что при $\Lambda(t) \hm\equiv
\lambda t$ с $\lambda \hm> 0$ процесс $S(t)$ превращается в
классический обобщенный пуассоновский процесс, широко используемый
при моделировании многих явлений в физике, теории надежности,
финансовой и актуарной деятельности, биологии и~т.\,д. Большое число
разнообразных прикладных задач, приводящих к обобщенным
пуассоновским процессам, описано в книгах~[3, 6--8].

Для вычисления риска катастрофических превышений критического уровня
случайным блуж\-да\-ни\-ем в течение рассматриваемого интервала времени
(горизонта планирования) необходимо знать распределение вероятностей
максимума суммы приращений блуждания. К~сожалению, практически
никогда распределение элементарных приращений не известно, поэтому
точное вычисление этого распределения невозможно. Даже когда есть
основания принять определенную модель такого распределения,
вычисление исключительно трудоемко. Поэтому на практике точное
распределение заменяют его аппроксимацией, в качестве которой
рассматривается асимптотическая аппроксимация, основанная на
(функциональной) центральной предельной теореме,~--- так на\-зы\-ва\-емое
полунормальное распределение~--- распределение модуля нормально
распределенной случайной величины, совпадающее с распределением
максимума стандартного винеровского процесса на единичном отрезке.

Итак, рассмотрим обобщенный процесс Кокса, определяемый соотношением~(\ref{e1-dul}) 
с ${\sf E}X_1\hm=0$, $0\hm<\sigma^2\hm={\sf D}X_1\hm<\infty$. Обозначим
$$
\overline S(t)=\max\limits_{0\le\tau\le t}S(\tau)\,;\enskip \underline
S(t)=\min\limits_{0\le\tau \le t}S(\tau)\,.
$$
Приведем необходимые и достаточные условия слабой сходимости
одномерных распределений случайных процессов $\overline S(t)$ и
$\underline S(t)$, скачки которых обладают указанными выше
свойствами.

Стандартную нормальную функцию распределения и ее плотность будем
обозначать $\Phi(x)$ и $\phi(x)$ соответственно:
$$
\Phi(x)=\int\limits_{-\infty}^{x}\phi(z)\,dz\,,\ 
\phi(x)=\fr{1}{\sqrt{2\pi}}e^{-x^2/2}\,,\ \ \ x\in\r\,.
$$
Функцию распределения максимума стандартного винеровского процесса
на отрезке $[0,1]$ обозначим $G(\cdot)$,
$$
G(x)=2\Phi(\max\{0,x\})-1\,, \enskip  x\in\r\,.
$$
Несложно видеть, что если $X$~--- случайная величина со стандартным
нормальным распределением, то $G(x)\hm={\sf P}(|X|<x)$. Символ
$\Longrightarrow$ будет обозначать сходимость по распределению.

Рассмотрим для начала независимые необязательно одинаково
распределенные случайные величины $Y_1,Y_2,\ldots$ с ${\sf
E}Y_i\hm=0$ и $0\hm<\sigma^2_i\hm={\sf D}Y_i\hm<\infty$, $i\hm\geqslant 1$. Для $k\hm\geqslant
1$ обозначим
$$
S_k=Y_1+\ldots+Y_k\,, \enskip B^2_k=\sigma^2_1+\ldots+\sigma^2_k\,,
$$
$$
\overline S_k=\max\limits_{1\le i\le k}S_i\,,\enskip \underline
S_k=\min_{1\le i\le k}S_i\,.
$$
Предположим, что случайные величины $Y_1,Y_2,\ldots$ удовлетворяют
условию Линдеберга: для любого $\alpha\hm>0$
$$
\lim\limits_{k\to\infty}\fr{1}{B^2_k}\sum\limits_{i=1}^{k}\int\limits_{|x|\geqslant \alpha
B_k}^{} x^2\,d{\sf P}(Y_i<x)=0\,.
$$

Хорошо известно, что при таких предполо\-же\-ниях
\begin{align*}
{\sf P}\left(\fr{\overline S_k}{B_k}<x\right)&\Longrightarrow
G(x)\,,\\
{\sf P}\left(\fr{\underline
S_k}{B_k}<x\right)&\Longrightarrow 1-G(-x)\,,\ \ k\to\infty
\end{align*}
(это одно из проявлений так называемого принципа инвариантности).
Приведем аналог этого результата для обобщенных дважды
стохастических пуассоновских процессов с нулевым средним.

Пусть $d(t)$~--- положительная функция, неограниченно возрастающая
при $t\hm\to\infty$.

\medskip

\noindent
\textbf{Теорема~1.} \textit{Предположим, что $\Lambda
(t)\hm\longrightarrow\infty$ по вероятности при $t\hm\to\infty$.
Одномерные распределения нормированного процесса экстремумов
обобщенного процесса Кокса слабо сходятся к некоторому
распределению, т.\,е.}
$$
\fr{\overline S(t)}{\sigma\sqrt{d(t)}} \Longrightarrow\ 
\overline Z\,,\enskip \fr{\underline S(t)}{\sigma\sqrt{d(t)}}
\Longrightarrow  \underline Z \ (t\to\infty)\,,
$$
\textit{тогда и только тогда, когда существует неотрицательная случайная
величина $U$ такая, что}
\begin{equation}
\fr{\Lambda(t)}{d(t)}\Longrightarrow U\enskip
(t\to\infty)\,.
\label{e2-dul}\end{equation}
\textit{При этом}
\begin{gather*}
{\sf P}(\overline Z<x) = {\sf E}G\left(\fr{x}{\sqrt{U}}\right)\,,\\ 
 {\sf
P}(\underline Z<x) =1-{\sf E}G\left(-\fr{x}{\sqrt{U}}\right)\ \ \ x\in\r\,.
\end{gather*}

\smallskip

\noindent
Д\,о\,к\,а\,з\,а\,т\,е\,л\,ь\,с\,т\,в\,о~теоремы~1 приведено в 
книге~\cite{KorolevSokolov2008}.

\smallskip

\noindent
\textbf{Следствие 1.} \textit{В условиях теоремы~$1$
$$
{\sf P}\left(\fr{\overline S(t)}{\sigma \sqrt{d(t)}} <x\right)
\Longrightarrow G(x)
$$ 
и 
$${\sf P}\left(\fr{\underline
S(t)}{\sigma \sqrt{d(t)}} <x\right) \Longrightarrow 1-G(-x)
$$ при
$t\hm\to\infty$ тогда и только тогда, когда}
$$
\fr{\Lambda(t)}{d(t)} \Longrightarrow   1 \enskip (t\to\infty)\,.
$$

\smallskip

Это утверждение является непосредственным следствием теоремы~1 с
учетом идентифици\-ру\-емости семейства масштабных смесей функций
распределения $G$ (и, следовательно, $1\hm-G(-x)$).

\section{Оценки скорости сходимости экстремумов обобщенных дважды стохастических
пуассоновских процессов}

В нестационарных потоках экстремальных событий, описываемых
обобщенными дважды сто\-хастическими процессами, согласно теореме~1
полунор\-мальное распределение $G(x)$ трансформируется в его
масштабную смесь, в которой сме\-ши\-ва\-ющее распределение отражает
статистические закономерности поведения случайной интенсивности
потока событий. При замене точного распределения указанной
асимптотической аппроксимацией, естественно, возникает погрешность,
без точного знания которой невозможно {\it гарантированно} оценить
риски, которые, как правило, характеризуются его {\it квантильной
мерой} VaR (Value at risk). При этом гарантированные риски
вычисляются как квантили уточненного порядка, равного базовой
допустимой вероятности нежелательного экстремального события плюс
погрешность аппроксимации <<истинного>> распределения максимума
обобщенного дважды стохастического пуассоновского процесса его
предельным вариантом.

Перейдем к уточнению простейших и наиболее популярных моментных
оценок погрешности, использующих информацию о первых моментах
элементарных слагаемых, которые можно просто вычислить (оценить) на
основе статистической информации. При этом основным объектом
исследования будут модели неоднородных хаотических процессов, в
которых интенсивность потока информативных событий имеет
гам\-ма-рас\-пре\-де\-ле\-ние с параметром формы, меньшим единицы. Именно
такие потоки, к примеру, характеризуют потоки событий на крупных
биржах и в итоге приводят к дисперсионным гам\-ма-рас\-пре\-де\-ле\-ни\-ям
(Variance Gamma distributions) приращений базовых финансовых
индексов~\cite{KorolevSokolov2012}. 

Как отмечено в работе~\cite{Gleser1987}, 
потоки отказов авиационной техники также
характеризуются именно таким распределением интервалов времени между
отказами. Более того, в этой же работе показано, что процесс
восстановления с гам\-ма-рас\-пре\-де\-лен\-ны\-ми интервалами времени между
восстановлениями является смешанным пуассоновским тогда и только
тогда, когда параметр формы гам\-ма-рас\-пре\-де\-ле\-ния не превосходит
единицы.

\smallskip

\noindent
\textbf{Лемма 1.} \textit{Пусть $\beta_3\hm={\sf E}|X_1|^3\hm<\infty$ и
$N_{\lambda}$~--- случайная величина, имеющая пуассоновское
распределение с параметром $\lambda \hm>0$ и независимая от
последовательности $\{X_j\}_{j\ge 1}$. Тогда существует конечная
положительная постоянная $C\hm>0$ такая, что для всех $\lambda \hm\ge 1$
справедливо неравенство}
$$
\sup\limits_{x} \left\vert{\sf P}\left(\fr{1}{\sigma\sqrt{\lambda}}
\max\limits_{1\le i\le N_{\lambda}}\sum\limits_{j=0}^{i}
X_j<x\right)-G(x)\right\vert \le
\fr{C\beta_3}{\sqrt{\lambda}\sigma^3}.
$$

\smallskip

\noindent
Д\,о\,к\,а\,з\,а\,т\,е\,л\,ь\,с\,т\,в\,о\,.\ В~работе~\cite{KorolevSelivanova1995}
доказано утверждение, частным случаем которого является следующая
оценка. Пусть $X_1,X_2,\ldots$~--- независимые одинаково
распределенные случайные величины с ${\sf E}X_1\hm=0$, $0\hm<{\sf
D}X_1\hm=\sigma^2\hm<\infty$ и ${\sf E}|X_1|^3\hm=\beta_3\hm<\infty$. Пусть $N$~--- 
целочисленная неотрицательная случайная величина, независимая от
последовательности $X_1,X_2,\ldots$ Для $k\hm\ge 1$ положим 
$$
\overline
S_k=\max\limits_{1\le n\le k}X_1+ \cdots+X_n\,.
$$ 
Тогда существуют конечные
положительные абсолютные постоянные $C'$ и~$C''$ такие, что
\begin{multline*}
\sup\limits_x \left\vert{\sf P}\left(\fr{\overline S_N}{\sigma\sqrt{{\sf
E}N}}< x\right)-G(x)\right\vert\le{}\\
{}\le\fr{C'}{\sqrt{{\sf
E}N}}\,\fr{\beta_3}{\sigma^3}+ C''{\sf E}\left\vert\fr{N}{{\sf
E}N}-1\right\vert\,.
\end{multline*}
В случае $N=N_{\lambda}$ по неравенству Маркова имеем
$$
{\sf E}\left\vert\fr{N_{\lambda}}{{\sf
E}N_{\lambda}}-1\right\vert\le\fr {\sqrt{{\sf D}N_{\lambda}}}{{\sf
E}N_{\lambda}}=\fr{1}{\sqrt{\lambda}}\,.
$$
Отсюда с учетом всегда имеющего место неравенства
$\beta_3/\sigma^3\hm\ge 1$ вытекает желаемый результат.

\smallskip

Метод оценивания точности аппроксимации распределений экстремумов
обобщенных процессов Кокса масштабными смесями функции распределения
$G(x)$, используемый далее, основан на лемме~1 и следующем
представлении распределения экстремумов обобщенного дважды
стохастического пуассоновского процесса $\overline S(t)$,
управляемого случайной мерой $\Lambda(t)$, справедливом в силу
стохастической независимости всех случайных величин и процессов,
вовлеченных в определение обобщенного процесса Кокса: 
\begin{multline*}
{\sf P}(\overline S(t)<x)={}\\
{}=\il{0}{\infty}{\sf P}\left(\max\limits_{1\le n\le
N_1(\lambda)}\sum\limits_{j=1}^{n}X_j<x\right)
d{\sf P}(\Lambda(t)<\lambda)\,,\\
x\in\r\,.
%\label{e3-dul}
\end{multline*}
Всюду далее будем предполагать, что ${\sf E}X_1\hm=0$, ${\sf
E}X_1^2\hm=1$, ${\sf E}|X_1|^3\hm\equiv\beta^3\hm<\infty$. Предположим, что
функция $d(t)$ положительна и неограниченно возрастает при
$t\hm\to\infty$. Обозначим
\begin{align*}
\rho_t&=\sup\limits_x\left\vert{\sf
P}\left(\!\fr{\overline
S(t)}{\sqrt{d(t)}}<x\!\right)-\!\il{0}{\infty}\!G\left(\!\fr{x}{\sqrt{\lambda}}\!\right)d{\sf
P}(\Lambda<\lambda)\right\vert;\\
\Delta_t&=\sup\limits_x\left\vert{\sf P}\left(\!\fr{\Lambda(t)}{d(t)}<x\!\right)-{\sf
P}(\Lambda<x)\right\vert.
\end{align*}

\smallskip

\noindent
\textbf{Теорема 2.} \textit{Пусть выполнено условие ${\sf
E}|X_1|^3\hm<\infty$. Тогда для любого $t\hm>0$ 
$$
\rho_t\le C\beta^3{\sf E}\left[\Lambda(t)\right]^{-1/2}+
\fr{1}{2}\,\Delta_t\,,
$$ 
где $C$~--- абсолютная постоянная из леммы~$1$}.

\smallskip

\noindent
Д\,о\,к\,а\,з\,а\,т\,е\,л\,ь\,с\,т\,в\,о~теоремы~2 приведено в 
книге~\cite{KorolevSokolov2008}.

\smallskip

В качестве примера применения теоремы~2 рассмотрим ситуацию, когда
при каждом $t\hm>0$ случайная величина $N(t)$ имеет отрицательное
биномиальное распределение с параметрами $r\hm>0$ и $p\hm\in(0,1)$:
\begin{multline}
{\sf P}\left(N(t)=n\right)={}\\
{}= C_{r+n-1}^{n}p^r(1-p)^{n}\,,\enskip
n=0,1,2,\ldots
\label{e4-dul}
\end{multline}
Здесь $r>0$ и $p\hm\in(0,1)$~--- параметры и для нецелых $r$ величина
$C_{r+n-1}^{n}$ определяется как 
$$ 
C_{r+n-1}^{n} = \fr{\Gamma(r+n)}{n!\Gamma(r)}\,. 
$$ 
В~частности, при $r\hm=1$
соотношение~(\ref{e4-dul}) задает геометрическое распределение. Отрицательное
биномиальное распределение с натуральным $r$ допускает наглядную
интерпретацию в терминах испытаний Бернулли. А~именно случайная
величина, име\-ющая отрицательное биномиальное распределение с
параметрами $r\hm>0$ и $p$,~--- это число испытаний Бернулли,
проведенных до осуществления $r$-й по счету неудачи, если
вероятность успеха в одном испытании равна $1-p$.

Плотность гамма-распределения с параметром формы $r\hm>0$ и параметром
масштаба $s\hm>0$, как известно, имеет вид:
$$
g_{r,s}(x)=\fr{s^r}{\Gamma(r)}\,e^{-s x}x^{r-1}\,,\enskip x>0\,.
$$
Как известно, отрицательное биномиальное распределение с параметрами
$r\hm>0$ и $p\hm\in(0,1)$ представ\-ля\-ет собой смешанное пуассоновское
распределение, в котором параметр имеет гам\-ма-рас\-пре\-де\-ле\-ние с
параметром масштаба $s\hm=p/(1\hm-p)$ и параметром формы~$r$.

Функцию гамма-распределения с параметром масштаба~$s$ и параметром
формы~$r$ обозначим $G_{r,s}(x)$,
$$
G_{r,s}(x)=\il{0}{x}g_{r,s}(z)\,dz\,.
$$
Несложно убедиться, что
\begin{equation}
G_{r,s}(x)\equiv G_{r,1}(s x)\,.
\label{e5-dul}
\end{equation}
Случайную величину с функцией распределения $G_{r,s}(x)$ обозначим
$U(r,s)$. Хорошо известно, что 
$$
{\sf E}U(r,s)=\fr{r}{s}\,.
$$

Будем считать, что параметр $r$ фиксирован, и в качестве случайной
величины $\Lambda(t)$ возьмем величину $U(r,s)$, считая, что
$t\hm=s^{-1}$: 
$$
\Lambda(t)=U(r,t^{-1})\,.
$$ 
В~качестве функции $d(t)$
возьмем 
$$
d(t)\equiv{\sf E}\Lambda(t)={\sf E}U(r,t^{-1})\,.
$$ 
Несложно видеть, что в терминах новой параметризации 
$$
{\sf E}U(r,t^{-1})=rt\,.
$$ 
Тогда с учетом~(\ref{e5-dul}) 
\begin{multline*}
{\sf P}\left(\fr{\Lambda(t)}{d(t)}<x\right)={\sf P}(U(r,t^{-1})<xrt)={}\\
{}=
{\sf P}(U(r,1)<xr)={\sf
P}(U(r,r)<x)=G_{r,r}(x)\,.
\end{multline*}
Обратим внимание, что функция
распределения, стоящая в правой части последнего соотношения, не
зависит от~$t$. Поэтому при указанном выборе функции $d(t)$ условие~(\ref{e2-dul}) 
выполняется тривиальным образом, причем для всех $t\hm>0$
$$
\Delta_t=0\,.
$$

Вычислим ${\sf E}[\Lambda(t)]^{-1/2}$, предполагая, что выполнено
условие $r>{1}/{2}$. Имеем
\begin{multline*}
{\sf E}[\Lambda(t)]^{-1/2}={\sf
E}[U(r,t^{-1}]^{-1/2}={}\\
{}=\il{0}{\infty}\fr{e^{-x/t}x^{r-1-1/2}}{t^r\Gamma(r)}\,dx
=
\fr{\Gamma(r-{1}/{2})}{t^{1/2}\Gamma(r)}\,.
\end{multline*}
Таким образом, приходим к следующему утверждению, являющемуся
частным случаем теоремы~2.

\smallskip

\noindent
\textbf{Следствие 2.} \textit{Пусть случайная величина $N(t)$ имеет
отрицательное биномиальное распределение с па\-ра\-мет\-ра\-ми $r\hm>0$ и
$p=(1+t)^{-1}$, где $t>0$. Предположим, что ${\sf
E}|X_1|^3\equiv\beta^3<\infty$. Тогда для каждого $t>0:$}

\noindent $1^{\circ}$. \textit{При} $r>1/2$
\begin{multline*}
\sup\limits_x\left\vert {\sf
P}(\overline S(t)<x\sqrt{rt})-\il{0}{\infty}G\left(\fr{x}{\sqrt{y}}\right)dG_{r,r}(y)
\right\vert\le{}\\
{}\le
C\fr{\Gamma(r-{1}/{2})}{\Gamma(r)}\,\fr{\beta^3}{\sqrt{t}}\,,
\end{multline*}
\textit{где $C$~--- абсолютная постоянная из леммы~$1$.}

\noindent $2^{\circ}$. \textit{При} $r\hm<1/2$
\begin{multline*}
\sup\limits_x\left\vert{\sf
P}(\overline
S(t)<x\sqrt{rt})-\il{0}{\infty}G\left(\fr{x}{\sqrt{y}}\right)dG_{r,r}(y)\right\vert\le{}\\
{}\le
\fr{C^{2r}\beta^{6r}}{\Gamma(r)}\left(\fr{1}{r}+\fr{1}{(1/2-r)}\right)
\fr{1}{t^r}\,,
\end{multline*}
\textit{где $C$~--- абсолютная постоянная из леммы~$1$.}

\noindent $3^{\circ}$. \textit{При} $r\hm=1/2$
\begin{multline*}
\sup\limits_x\left\vert{\sf
P}\left(\overline
S(t)<x\sqrt{\fr{t}{2}}\right)-{}\right.\\
\left.{}-\il{0}{\infty}G\left(\fr{x}{\sqrt{y}}\right)
dG_{1/2,1/2}(y)\right\vert\le{}\\
{}\le
\fr{C\beta^3}{\sqrt{\pi}}
\left(2+\ln\left(1+\fr{t}{C^2\beta^6}\right)\right)\fr{1}{\sqrt{t}}\,,
\end{multline*}
\textit{где $C$~--- абсолютная постоянная из леммы~$1$.}

\smallskip

\noindent
Д\,о\,к\,а\,з\,а\,т\,е\,л\,ь\,с\,т\,в\,о\,.~Пункт~$1^{\circ}$, по сути, уже
доказан. Докажем пункт~$2^{\circ}$. Пусть, как и ранее,
$g_{r,r}(\lambda)$~--- плотность гам\-ма-рас\-пре\-де\-ле\-ния с параметрами
$r,r$; $C$~--- константа из леммы~1. Из приведенных выше рассуждений
следует, что
\begin{multline*}
\hspace*{-6pt}\rho_t=\sup\limits_x\left\vert{\sf P}(\overline
S(t)<x\sqrt{rt})-\!\int\limits_{0}^{\infty}\!
G\left(\fr{x}{\sqrt{y}}\right)dG_{r,r}(y)\right\vert\le{}
\\
\hspace*{-6pt}{}\le \int\limits_0^{\infty}\sup\limits_x \left\vert{\sf
P}\left(\!\fr{1}{\sqrt{\lambda rt}}\sum\limits_{j=1}^{N_{\lambda rt}}X_j<x
\!\right)-G(x)\right\vert g_{r,r}(\lambda)d\lambda,
\end{multline*}
где $N_{\lambda rt}$~--- пуассоновская случайная величина с
параметром $\lambda rt$. Заметим, что оценка равномерного расстояния
между функциями распределения, даваемая леммой~1, при малых
$\lambda$ заведомо больше единицы, что тривиально для равномерного
расстояния между функциями распределения. Используя этот факт,
получаем
\begin{multline*}
\int\limits_0^{\infty}\sup\limits_x \left\vert{\sf P}\left(\fr{1}{\sqrt{\lambda
rt}}\sum\limits_{j=1}^{N_{\lambda rt}}X_j<x
\right)-{}\right.\\
\left.{}-G(x)\vphantom{\sum\limits_{j=1}^N}\right\vert g_{r,r}(\lambda)d\lambda\le{}\\
{}\le
\int\limits_0^{\infty} \min\left\{1,\, \fr{C\beta^3}{\sqrt{\lambda
r t}}\right\} g_{r,r}(\lambda) d\lambda={}
\\
{}=\int\limits_0^{C^2\beta^6/(rt)}g_{r,r}(\lambda)
d\lambda+\int\limits_{C^2\beta^6/(rt)}^{\infty}\fr{C\beta^3}{\sqrt{\lambda
r t}} g_{r,r}(\lambda) d\lambda \equiv{}\\
{}\equiv I_1+I_2\,.
\end{multline*}
Оценим первый интеграл:
\begin{multline*}
I_1 =\int\limits_0^{C^2\beta^6/(rt)}g_{r,r}(\lambda)
d\lambda={}\\
{}=\fr{1}{\Gamma(r)}\int\limits_0^{C^2\beta^6/(rt)}r^r
e^{-r\lambda} \lambda^{r-1}d\lambda={}
\\
{}=\fr{1}{\Gamma(r)}\int\limits_0^{C^2\beta^6/t}e^{-u}u^{r-1}du \le{}\\
{}\le
\fr{1}{\Gamma(r)}\int\limits_0^{C^2\beta^6/t}u^{r-1}\,du
=\fr{1}{r\Gamma(r)}\,\fr{C^{2r}\beta^{6r}}{t^r}\,.
\end{multline*}
Оценим второй интеграл. При $r<1/2$ имеем:
\begin{multline*}
I_2=\fr{C\beta^3}{\Gamma(r)\sqrt{ r
t}}\int\limits_{C^2\beta^6/(rt)}^{\infty}\fr{r^r
e^{-r\lambda}\lambda^{r-1} }{\sqrt{\lambda}}\,d\lambda
={}\\
{}=\fr{C\beta^3}{\Gamma(r)\sqrt{t}}
\int\limits_{C^2\beta^6/t}^{\infty}e^{-u}u^{r-3/2}\,du\le{}\\
{}
\le\fr{C\beta^3}{\Gamma(r)\sqrt{t}}e^{-C^2\beta^6/t}
\int\limits_{C^2\beta^6/t}^{\infty}u^{r-3/2}\,du\le{}\\
{}\le
\fr{C\beta^3}{\Gamma(r)\sqrt{t}}\left(r-\fr{1}{2}\right)^{-1}
u^{r-1/2}\Big|_{u=C^2\beta^6/t}^{u=\infty}={}\\
{}
=\fr{C\beta^3}{\Gamma(r)\sqrt{t}}\fr{1}{1/2-r}\,
\fr{C^{2r-1}\beta^{6r-3}}{t^{r-1/2}}={}\\
{}=
\fr{C^{2r}\beta^{6r}}{\Gamma(r)(1/2-r)}\,\fr{1}{t^r}\,.
\end{multline*}
Таким образом,
\begin{multline*}
 \sup\limits_x\left\vert{\sf P}(\overline S(t)<x\sqrt{rt})-\int\limits_{0}^{\infty}
G\left(\fr{x}{\sqrt{y}}\right)dG_{r,r}(y)\right\vert \le{}\\
{}\le
\fr{C^{2r}\beta^{6r}}{\Gamma(r)}\left(\fr{1}{r}+\fr{1}{(1/2-r)}\right)\fr{1}{t^r}\,,
\end{multline*}
что и требовалось доказать.


Докажем пункт $3^{\circ}$. Несложно видеть, что в этом случае
\begin{multline*}
I_2=\fr{C\beta^3}{\Gamma(r)\sqrt{ r
t}}\int\limits_{C^2\beta^6/(rt)}^{\infty}\fr{r^r
e^{-r\lambda}\lambda^{r-1} }{\sqrt{\lambda}}\,d\lambda
={}\\
{}=\fr{C\beta^3}{\Gamma(r)\sqrt{t}}\int\limits_{C^2\beta^6/t}^{\infty}e^{-u}u^{r-3/2}\,du={}
\\{}
=\fr{C\beta^3}{\Gamma(1/2)\sqrt{t}}\int\limits_{C^2\beta^6/t}^{\infty}e^{-u}u^{-1/2}\,du={}\\
{}=
\fr{C\beta^3}{\sqrt{\pi t}}\,E_1\left(\fr{C^2\beta^6}t\right)\le
\fr{C\beta^3}{\sqrt{\pi t}}\ln\left(1+\frac{t}{C^2\beta^6}\right)\,,
\end{multline*}
где $E_1(\cdot)$~--- интегральная показательная функция. Тогда
\begin{multline*}
\sup\limits_x\left\vert{\sf P}(\overline
S(t)<x\sqrt{rt})-\int\limits_{0}^{\infty}
G\left(\fr{x}{\sqrt{y}}\right)dG_{r,r}(y)\right\vert 
\le{}\\
{}\le
\fr{C\beta^3}{\sqrt{\pi}}\left(2+\ln\left(1+\fr{t}{C^2\beta^6}\right)\right)
\fr{1}{\sqrt{t}}\,.
\end{multline*}
Теорема доказана.

\smallskip

Несложно видеть, что при $r\hm=1$ предельная функция распределения
представляет собой показательное распределение с параметром
$\sqrt{2}$.

\smallskip

С помощью теоремы~3 и следствия~2 можно получить следующую оценку
точности приближения распределения экстремумов геометрических
случайных сумм показательным распределением.

\smallskip

\noindent
\textbf{Следствие 3.} \textit{Пусть случайная величина $N(t)$ имеет
геометрическое распределение с параметром $p\hm=(1\hm+t)^{-1}$, где $t\hm>0$.
Предположим, что ${\sf E}|X_1|^3\hm\equiv\beta^3\hm<\infty$. Тогда для
каждого $t\hm>0$}
 $$
 \sup\limits_{x\ge0}|{\sf P}(\overline
S(t)<x\sqrt{t})-(1-e^{-\sqrt{2}x})|\le
C\sqrt{\pi}\fr{\beta^3}{\sqrt{t}}\,,
$$ 
\textit{где $C$~--- абсолютная
постоянная из леммы}~1.

\section{Обсуждение}

Как показывает следствие~2, при $t\hm\to\infty$
\begin{multline*}
\sup\limits_x\left\vert{\sf P}\left(\overline
S(t)<x\sqrt{rt}\right)-\il{0}{\infty}G\left(\fr{x}{\sqrt{y}}\right)dG_{r,r}(y)\right\vert 
={}\\
{}=\begin{cases}
O\left(\fr{1}{\sqrt{t}}\right)\,,& \mbox{ если } r>\fr{1}{2}\,;\\[9pt]
O\left(\fr{\ln t}{\sqrt{t}}\right)\,,& \mbox{ если } r=\fr{1}{2}\,;\\[9pt]
O\left(\fr{1}{t^r}\right)\,,& \mbox{ если } r<\fr{1}{2}\,.
\end{cases}
\end{multline*}

В работе~\cite{GZK2006} показано, что именно такая зависимость от
параметра~$r$ присуща скорости схо\-ди\-мости распределений
<<асимптотически нормальных>> в классическом смысле статистик к
распределе\-нию Стьюдента с $2r$ степенями свободы при замене объема
выборки случайной величиной с отрицательным биномиальным
распределением~(\ref{e4-dul}) с $p\hm=(1\hm+t)^{-1}$ при $t\hm\to\infty$. Распределение
Стьюдента с $2r$ степенями свободы является масштабной смесью
нормальных законов с нулевым средним, в которой смешивающим является
гам\-ма-рас\-пре\-де\-ле\-ние $G_{r,r}$. В~работе~\cite{Nefedova2011} на
примере сумм случайного чис\-ла независимых случайных величин с
индексом, имеющим указанное отрицательное биномиальное
распределение, показано, что такой порядок скорости сходимости
является правильным. Таким образом, результаты данной статьи вполне
согласуются с упомянутыми работами и распространяют указанную
закономерность на <<асимптотически полунормальные>> статистики,
каковыми являются максимальные суммы.

{\small\frenchspacing
{%\baselineskip=10.8pt
\addcontentsline{toc}{section}{Литература}
\begin{thebibliography}{99}

\bibitem{BKSSh2007} 
\Au{Бенинг В.\,Е., Королев В.\,Ю., Соколов~И.\,А., Шоргин~С.\,Я.}
Рандомизированные модели и методы тео\-рии на\-деж\-ности информационных и
технических сис\-тем.~--- М.: ТОРУС ПРЕСС, 2007. 248~с.

\bibitem{BeningKorolev2004} 
\Au{ Бенинг В.\,Е., Королев В.\,Ю.} Об использовании распределения Стьюдента в
задачах теории вероятностей и математической статистики~// Теория
вероятностей и ее применения, 2004. Т.~49. Вып.~3. С.~417--435.

\bibitem{KorolevBeningShorgin2011} 
\Au{Королев В.\,Ю., Бенинг В.\,Е., Шоргин~С.\,Я.} 
Математические основы теории риска.~--- 2-е изд.,
перераб. и доп.~--- М.: Физматлит, 2011. 620~с.

\bibitem{KorolevSokolov2012} 
\Au{Королев В.\,Ю., Соколов И.\,А.} Скошенные распределения Стьюдента,
дисперсионные гам\-ма-рас\-пре\-де\-ле\-ния и их обобщения как асимптотические
аппроксимации~// Информатика и её примерения, 2012. Т.~6. Вып.~1.
С.~2--10.

\bibitem{KorolevSokolov2008} 
\Au{Королев В.\,Ю., Соколов И.\,А.} Математические модели
неоднородных потоков экстремальных событий.~--- М.: ТОРУС ПРЕСС,
2008.

\bibitem{BeningKorolev2002}
\Au{Bening V., Korolev V.} Generalized poisson models and their
applications in insurance and finance.~--- Utrecht: VSP, 2002.

\bibitem{GnedenkoKorolev1996} 
\Au{Gnedenko B.\,V., Korolev V.\,Yu.} Random summation:
Limit theorems and applications.~--- Boca Raton: CRC Press, 1996.

\bibitem{Korolev2011} 
\Au{Королев В.\,Ю.} Ве\-ро\-ят\-но\-ст\-но-ста\-ти\-сти\-че\-ские методы
декомпозиции волатильности хаотических процессов.~--- М.: Изд-во
Московского ун-та, 2011. 510~с.

\bibitem{Gleser1987} 
\Au{Gleser L.\,J.} The gamma distribution as a mixture of exponential
distributions: Technical Report \# 87-28.~--- West Lafayette: Purdue
University, 1987. 6~p.

\bibitem{KorolevSelivanova1995} 
\Au{Korolev V.\,Yu., Selivanova D.\,O.} 
Convergence rate estimates in some limit theorems for
maximum random sums~// J.~Math. Sci., 1995.
Vol.~76. No.\,1. P.~2163--2168.

\bibitem{GZK2006} 
\Au{Гавриленко С.\,В., Зубов В.\,Н., Королев В.\,Ю.} Оценка
скорости сходимости распределений регулярных статистик, построенных
по выборкам случайного объема с отрицательным биномиальным
распределением, к распределению Стьюдента~// Статистические методы
оценивания и проверки гипотез: Межвузовский сб. научных тр.~--- 
Пермь: ПГУ, 2006. C.~118--134.

\label{end\stat}

\bibitem{Nefedova2011} 
\Au{Нефедова Ю.\,С.} Оценки скорости сходимости в предельной
теореме для отрицательных биномиальных случайных сумм~//
Статистические методы оценивания и проверки гипотез: Межвузовский
сб. научных тр.~--- Пермь: ПГУ, 2011. C.~46--61.
\end{thebibliography}
}
}


\end{multicols}   %2

\def\stat{konovalov}

\def\tit{ОБ АДАПТИВНЫХ СТРАТЕГИЯХ И~УСЛОВИЯХ~ИХ~СУЩЕСТВОВАНИЯ$^*$}

\def\titkol{Об адаптивных стратегиях и~условиях их 
существования}

\def\autkol{М.\,Г.~Коновалов}

\def\aut{М.\,Г.~Коновалов$^1$}

\titel{\tit}{\aut}{\autkol}{\titkol}

{\renewcommand{\thefootnote}{\fnsymbol{footnote}}\footnotetext[1]
{Работа выполнена при поддержке РФФИ, грант № 11-07-00112.}}

\renewcommand{\thefootnote}{\arabic{footnote}}
\footnotetext[1]{Институт проблем информатики Российской академии наук, mkonovalov@ipiran.ru}



\Abst{Рассматривается задача оптимального управления в отсутствие априорной 
информации об управляемом объекте. Решением задачи является построение адаптивных 
стратегий на основе наблюдений, доступных в процессе управления. Изучаются 
некоторые условия адаптивной управляемости объекта. В~качестве математической 
модели используются управляемые случайные последовательности.}

\KW{управляемые случайные последовательности; адаптивные стратегии; условия 
существования}

\vskip 14pt plus 9pt minus 6pt

      \thispagestyle{headings}

      \begin{multicols}{2}

            \label{st\stat}


\section{Введение}

  Тема статьи относится к области адаптивных методов обработки информации с целью 
принятия оптимальных решений. Потребность в адаптивном\linebreak
подходе возникает в задачах 
с большой информационной неопределенностью, что наиболее характерно для 
телекоммуникационных систем, автоматизированных производственных процессов, 
робототех\-ни\-ки и других сфер, неразрывно связанных с компьютерной обработкой 
информации. Понятие неопределенности многозначно и связано с отсутствием априорных 
сведений, недетерминированностью, а также с неполнотой наблюдений. 
К~перечисленным факторам в нарастающей степени добавляется <<избыточность>> 
информации, которая порождается чрезмерно прогрессирующими объемами 
передаваемой и хранимой информации и обусловлена экспоненциальным ростом 
пропускной способности телекоммуникационных сетей, а также емкостей носителей 
информации.
  
  Идея адаптации (приспособления, самоорганизации), заимствованная из 
биологического мира, начала активно эксплуатироваться в науке примерно с середины 
прошлого века. Кратко, она заключается в том, чтобы, целенаправленно взаимодействуя с 
окружающей средой, отбирать и использовать поступающую информацию, необходимую 
для принятия оптимальных решений с точки зрения поставленной цели.
  
  Данная статья посвящена теоретическим аспектам адаптации. В~качестве исходного 
пред\-став\-ле\-ния использована схема, которая опирается на пред\-став\-ление о паре 
  <<объект--субъект>>, взаимодействующей в дискретном времени путем 
попеременного обмена сигналами. При этом субъект воздействует на объект с помощью 
управлений, получая в ответ сигналы, называемые наблюдениями. Действия субъекта 
преследуют цель, выраженную в наличии определенных свойств у траектории 
наблюдений.
  
  Основная отличительная особенность заключается в предположении, что действия 
субъекта происходят при недостаточной информации об объекте. В~качестве 
математической модели объекта взята конструкция управляемой случайной 
последовательности. В~терминах этого аппарата легко очерчиваются четыре аспекта 
информационной неопределенности:
  \begin{enumerate}[(1)]
\item недетерминированность понимается как стохастичность;
\item недостаток информации об объекте трактуется как неполное знание вероятностного 
распределения, задающего процесс;
  \item неполнота наблюдений означает, что состояния процесса наблюдаются лишь 
частично;
  \item недостаток знаний выражается в неумении \mbox{найти} или рассчитать ту или иную 
характеристику, связанную со случайной последовательностью, даже при наличии 
априорной информации о распределении процесса и полной его наблюдаемости.
  \end{enumerate}
  
  Субъект ассоциируется с алгоритмом, согласно которому выбираются управления, 
регулирующие траекторию случайной последовательности. Такой алгоритм принято 
называть стратегией управ\-ле\-ния. Задача заключается в том, чтобы выбрать стратегию, 
достигающую цели в ситуации, когда информация субъекта об объекте ограничена. 
По-дру\-го\-му можно сказать, что речь идет о построении стратегии, достигающей цели (в 
данном случае~--- максимизации предельного среднего дохода) для любого процесса из 
некоторого заданного класса объектов. Такие стратегии называют адаптивными по 
отношению к заданному классу объектов~[1].
  
  В разд.~2 даются формальные определения объекта, цели и адаптивной стратегии 
управления.
  
  В разд.~3 анализируются условия существования адаптивной стратегии. В~качестве 
необходимых условий обсуждаются два требования, которые, как представляется, должны 
выполняться из интуитивных соображений.
  
  Первое из необходимых условий связано с принципиальной особенностью адаптивных 
стратегий, которые, прежде чем выйти на <<оптимальный режим>>, должны затратить 
некоторое время на <<обуче\-ние>>. (На самом деле в рассматриваемой постановке процесс 
обучения для адаптивных стратегий длится даже неограниченно долго.) Естественно 
предположить, что подобные стратегии могут реализоваться, только если в процессе 
обучения не будут совершены <<непоправимые ошибки>>. Это соображение 
раскрывается на примерах и получает формальное описание.
  
  Второе необходимое условие является менее очевидным. Оно связано с гипотезой о 
том, что адаптивная стратегия управления классом случайных последовательностей 
существует лишь тогда, когда для данного класса возможно построение так называемой 
адаптивной стратегии перебора. Это выражается в том, что существует и заранее известно 
некоторое счетное множество вариантов поведения, среди которого для данного класса 
обязательно найдется оптимальный или близкий к нему вариант. Данное соображение 
также иллюстрировано примерами и приведена теорема о критерии существования 
адаптивной стратегии для определенного класса объектов.
  
  Подход, использованный в статье, а также полученные результаты являются 
продолжением направления, представленного в работе~[2].
  
\section{Постановка задачи адаптивного управления}
  
  Пусть  время $t$ пробегает значения 0, 1, \ldots\ и пусть заданы измеримые 
пространства $(X,\mathbf{X})$, $(Y,\mathbf{Y})$, $(Z,\mathbf{Z})$ (соответственно 
пространства \textit{состояний}, \textit{управлений} и \textit{наблюдений}).
  
  Общая траектория процесса упорядочена в виде последовательности $x_0, y_1, 
z_1,x_1,\ldots$\linebreak $\ldots , x_{t-1},y_t,z_t,x_t,\ldots$ Предыстория процесса до момента~$t$ 
включительно обозначается как

\noindent
  \begin{gather*}
 \! x^t=x_0^t=(x_0,\ldots , x_{t-1});\ \ \ y^t=y_1^t=(y_1, \ldots , y_{t-1});\\
  z^t=z_1^t=(z_1,  \ldots , z_{t-1})\,.
  \end{gather*}
  
  Траектории процесса определяются последовательностями условных вероятностных 
распределений~$\mu$, $\nu$ и~$\sigma$.
  
  Последовательность $\mu\hm=(\mu_0,\mu_1,\ldots ,\mu_t, \ldots)$ задает механизм 
смены состояний. В~этой последовательности $\mu_0$~--- вероятностное распределение 
на $(X,\mathbf{X})$; $\mu_t=\mu_t(A\vert x^{t-1},y^t)$, $t\hm>0$~---  условная 
(переходная) вероятность, которая при любых наборах $(x^{t-1},y^t)$ является 
вероятностной мерой на $(X,\mathbf{X})$ и при любом $A\hm\in X$ является измеримой 
функцией относительно $x^{t-1},y^t$.
  
  Последовательность $\nu\hm=(\nu_1, \ldots , \nu_t, \ldots)$ задает механизм появления 
наблюдений. В~этой последовательности каждый элемент $\nu_t\hm=\nu_t(C\vert x^{t-1}, 
y^t)$, $t\hm>0$, представляет собой условное распределение, которое при любом условии 
является вероятностной мерой на $(Z,\mathbf{Z})$ и для любого $C\hm\in Z$ является 
измеримой функцией относительно переменных, стоящих в условии. Пара $o\hm= 
(\mu,\nu)$ называется объектом.
  
  Последовательность $\sigma\hm= (\sigma_1, \ldots , \sigma_t. \ldots)$ называется 
(допустимой) \textit{стратегией} и определяет выбор управлений. В~этой 
последовательности:
%\smallskip
   $\sigma_1\hm=\sigma_1(\cdot)$~--- вероятностная мера на $(Y,\mathbf{Y})$; 
      $\sigma_{t+1}\hm=\sigma_{t+1}(B\vert y^t,z^t)$, $t\hm>0$,~--- условная вероятность, 
которая при любых $y^t,z^t$ является вероятностной мерой на $(Y,\mathbf{Y})$ и при 
любом $B\hm\in Y$ является измеримой функцией относительно $y^t,z^t$. Элементы 
последовательности~$\sigma$ называются (допустимыми) \textit{правилами}.

%\smallskip
  
  Введем обозначение для прямых произведений множеств:
  $$
  \Omega_0=X\,;\enskip \Omega_t=X^{t+1}\times Y^t\times Z^t\,,\enskip t>0\,,
  $$
а также для наименьших $\sigma$-ал\-гебр, порожденных соответствующими 
$\sigma$-ал\-геб\-рами:
$$
\mathbf{F}_0=\mathbf{X}\,;\enskip \mathbf{F}_t=\mathbf{X}\otimes \mathbf{Y}\otimes 
\mathbf{Z}\otimes \mathbf{X}\otimes \cdots \otimes \mathbf{Y}\otimes \mathbf{Z}\otimes 
\mathbf{X}
$$
($\mathbf{X}$ повторяется $t+1$ раз, $\mathbf{Y}$ и $\mathbf{Z}$~--- $t$ раз, $t\hm>0$).
  
  Положим
  
  \vspace*{3pt}
  
  \noindent
  $$
  \Omega =\prod\limits_{t\geq 0}\Omega_t\,;\enskip 
\mathbf{F}=\mathop{\otimes}\limits_{t\geq0}\mathbf{F}_t\,.
  $$ 
  
  Согласно общей теории~\cite{3-kon} последовательности $o\hm=(\mu,\nu)$ и~$\sigma$ 
порождают на пространстве $(\Omega, \mathbf{F})$ вероятностную меру $\mathbf{P}\hm= 
\mathbf{P}_{o,\sigma}\hm=\mathbf{P}_{\mu,\nu,\sigma}$, которая согласована с 
элементами этих последовательностей следующим образом. Случайные 
последова\-тель\-ности

\vspace*{-3pt}

\noindent
  \begin{gather*}
  x_t=x_t(\omega)\,;\enskip  
  y_{t+1}=y_{t+1}(\omega)\,;\\
  z_{t+1}= z_{t+1}(\omega)\,,\enskip  \omega\in \Omega\,,\  t\geq 0\,,
  \end{gather*}
удовлетворяют соотношениям:

\pagebreak

\noindent
$$
\mathbf{P}(x_0(\omega)\in A_0)=\int\limits_{A_0} \mu_0(dx_0)\,;
$$

\vspace*{-12pt}

\noindent
\begin{multline*}
\mathbf{P}\left(x_0(\omega)\in A_0\,,\  y_1(\omega)\in B_1\,,\ 
z_1(\omega)\in C_1, \ldots \right.\\[1pt]
\left.{}\ldots\,,
y_t(\omega)\in B_t\,,\  z_t(\omega)\in C_t\,,\  x_t(\omega)\in A_t\right)={}\\[1pt]
{}=\int\limits_{A_0}\mu_0(dx_0)\int\limits_{B_1}\sigma_1(dy_1)\int\limits_{C_1}\nu_1(dz_1
\vert x_0, y_1)\cdots{}\\[1pt]
{}\cdots
\int\limits_{B_t}\sigma_t\left(dy_t\vert y^{t-1},z^{t-1}\right) 
\int\limits_{C_t} \nu_t\left( dz_t\vert x^{t-
1},y^t\right) \times{}\\[1pt]
{}\times
\int\limits_{A_t} \mu_t\left( dx_t\vert x^{t-1},y^t\right)
\end{multline*}
для любых $A_t\in X$, $B_{t+1}\hm\in Y$, $C_{t+1}\hm\in Z$, $t\hm\geq 0$.
  
  По определению стратегии, ее правила зависят от предыдущих управлений и 
наблюдений, но не от предыдущих состояний. Это соответствует предположению о том, 
что состояния объекта не наблюдаемы в ходе процесса управления. В~частных случаях 
объект $o\hm=(\mu,\nu)$ может, конечно, описывать полностью наблюдаемый процесс. 
Например, если все множества $X_t$ содержат один и тот же единственный элемент. 
Другой простой пример~--- когда наблюдения тождественны состояниям. Однако на 
самом деле, как показывает лемма~1, с формальной точки зрения рассмотрение объекта с 
<<ненаблюдаемой>> компонентой всегда можно заменить изучением полностью 
наблюдаемого процесса.
  
  \medskip
  
  \noindent
  \textbf{Лемма 1.} \textit{Для любого объекта $o\hm=(\mu,\nu)$ условная вероятность 
$\mathbf{P}\left(dz_t\vert y^t,z^{t-1}\right)$ не зависит от стратегии~$\sigma$ при любых 
$t\hm>0$.}
  
  \medskip
  
  \noindent
  Д\,о\,к\,а\,з\,а\,т\,е\,л\,ь\,с\,т\,в\,о\,.\ Согласно отмеченной выше согласованности 
условных распределений $\mu,\nu,o$ и порождаемой ими меры~\textbf{P} имеем 
соотношения:
  \begin{multline*}
  I_1=\mathbf{P}\left(
  y_1(\omega)\in B_1,\ z_1(\omega)\in C_1\right) ={}\\[1pt]
  {}=
  \mathbf{P}\left( x_0(\omega)\in X_0\,,\ y_1(\omega)\in B_1\,,\ z_1(\omega)\in 
C_1\right)={}\\[1pt]
  {}=\int\limits_{X_0} \int\limits_{B_1} \int\limits_{C_1} \mu_0\left(dx_0\right) 
\sigma_1\left(dy_1\right) \nu_1\left(dz_1\vert x_0,y_1\right)={}\\[1pt]
  {}= \int\limits_{B_1}\int\limits_{C_1}\sigma_1\left(dy_1\right) \int\limits_{X_0}\mu_0\left( 
dx_0\right) \nu_1\left( dz_1\vert x_0,y_1\right)\,,
  \end{multline*}
справедливые при любых $B_1\hm\in Y$ и $C_1\in Z$. Кроме того, по определению 
условной вероятности
$$
I_1=\int\limits_{B_1}\int\limits_{C_1}\sigma_1\left(dy_1\right) \mathbf{P}\left(dz_1\vert 
y_1\right)\,.
$$
  
  Сравнивая оба выражения для~$I_1$, получаем, что
  $$
  \mathbf{P}\left( dz_1\vert 
y_1\right)=\int\limits_{X_0}\mu_0\left(dx_0\right)\nu_1\left(dz_1\vert x_0, y_1\right)\,,
  $$
т.\,е.\ утверждение леммы справедливо для $t\hm=1$. Пусть оно верно для $n\hm=1, 2, 
\ldots , t\hm-1$. Для любых $B_1\hm\in Y$, $C_1\hm\in Z$, \ldots , $B_{t-1}\hm\in Y$, 
$C_t\hm\in Z$ имеем:

\noindent
\begin{multline*}
I_t=\mathbf{P}\left( y_1(\omega)\in B_1\,,\ z_1(\omega)\in C_1, \ldots{}\right.\\[1pt]
\left.{}\ldots , y_t(\omega)\in B_t\,,\ 
z_t(\omega) \in C_t\right)={}\\[1pt]
{}=
\mathbf{P}\left( x_0(\omega)\in X\,,\ y_1(\omega)\in B_1\,,\ z_1(\omega)\in C_1\,, \ldots\right.\\[1pt]
\left.{}\ldots , x_{t-
1}(\omega)\in X\,,\ y_t(\omega)\in B_t\,,\ z_t(\omega)\in C_t\right)={}\\[1pt]
{}=
\int\limits_X \int\limits_{B_1} \int\limits_{C_1}\ldots \\[1pt]
\ldots\int\limits_X \int\limits_{B_t} 
\int\limits_{C_t} \mu_0\left( dx_0\right) \sigma_1\left( dy_1\right) \nu_1\left( dz_1\vert 
x_0,y_1\right)\cdots{}\\[1pt]
\cdots \mu_{t-1}\left( dx_{t-1}\vert x^{t-2} y^{t-1}\right) \sigma_t \left( dy_t\vert y^{t-
1},z^{t-1}\right)\times{}\\[1pt]
{}\times \nu_t\left( dz_t\vert x^{t-1},y^t\right)={}\\[1pt]
{}=\int\limits_{B_1} \sigma_1\left( dy_1\right) \int\limits_{C_1} \int\limits_{B_2} 
\sigma_2\left( dy_2\vert z_1\right)\cdots\\[1pt]
\cdots \int\limits_{C_{t-1}}\int\limits_{B_t} \sigma_t \left( 
dy_t\vert y^{t-1},z^{t-1}\right)\times{}\\[1pt]
{}\times \int\limits_X \mu_0\left(dx_0\right) \nu_1\left( dz_1\vert 
x_o,y_1\right)\cdots{}\\[1pt]
{}\cdots \int\limits_{X_{t-1}}\mu_{t-1}\left( dx_{t-1}\vert x^{t-2} y^{t-1}\right) \nu_t \left( 
dz_t\vert x^{t-1}, y^t\right)={}\\[1pt]
{}=\int\limits_{B_1} \sigma_1\left( dy_1\right) \int\limits_{C_1} 
\int\limits_{B_2}\sigma_2\left( dy_2\vert z_1\right)\cdots\\[1pt]
\cdots \int\limits_{C_{t-1}} 
\int\limits_{B_t} \sigma_t\left( dy_t\vert y^{t-1},z^{t-1}\right) \int\limits_{C_t} 
\mathbf{P}\left( dz_1\vert y_1\right)\ldots{}\\[1pt]
{}\cdots \mathbf{P}\left( dz_{t-1}\vert y^{t-1},z^{t-2}\right) \mathbf{P}\left( dz_t\vert y^t, 
z^{t-1}\right)\,.
\end{multline*}
Отсюда получаем, что

\noindent
  \begin{multline*}
\hspace*{-6.95218pt}\mathbf{P}\left( dz_1\vert y_1\right)\cdots \mathbf{P}\left( dz_{t-1}\vert y^{t-1},z^{t-
2}\right) \mathbf{P}\left( dz_t\vert y^t,z^{t-1}\right)={}\\[1pt]
  {}=\int\limits_X \mu_0\left( dx_0\right) \nu_1\left( dz_1\vert x_o,y_1\right)\cdots \\[1pt]
  \cdots
\int\limits_X \mu_{t-1}\left( dx_{t-1}\vert x^{t-2}y^{t-1}\right) \nu_t\left( dz_t\vert x^{t-
1},y^t\right)\,.
  \end{multline*}
  
  Следовательно, по предположению индукции $\mathbf{P}\left( dz_t\vert y^t,z^{t-
1}\right)$ не зависит от~$\sigma$.
  
  Таким образом, не уменьшая общности, можно ограничиться (что и будет сделано в 
оставшейся части текста) рассмотрением полностью наблюда-\linebreak\vspace*{-12pt}

\pagebreak

\noindent
емых объектов $o\hm=\mu$, 
управляемых (допустимыми) стратегиями~$\sigma$ c правилами вида
  $$
  \sigma_1=\sigma_1\left(\cdot\right)\,;\enskip \sigma_{t+1}=\sigma_{t+1}\left( \cdot \vert 
y^t,x^t\right)\,,\enskip t>0\,.
  $$
(Множество всех таких стратегий при заданных пространствах состояний и управлений 
далее обозначается через~$\Sigma$.) В~этом случае вероятностная мера 
$\mathbf{P}\hm=\mathbf{P}_{\mu,\sigma}$ определена на пространстве $(\Omega, 
\mathbf{F})$, в котором $\Omega\hm=\prod\limits_{t\geq0} X^{t+1}\times Y^t$, 
$\mathbf{F}\mathop{\otimes}\limits_{t\geq0} \mathbf{F}_t$, где $\mathbf{F}_0\hm=\mathbf{X}$; 
$\mathbf{F}_t=\mathbf{X}\otimes \mathbf{Y}\otimes \mathbf{X}\otimes \cdots \otimes 
\mathbf{Y}\otimes \mathbf{X}$ и согласована с последовательностями~$\mu$ и~$\sigma$. 
Через $\mathbf{F}_t$ обозначена $\sigma$-ал\-геб\-ра, порожденная предысторией 
$(x^t,y^t)$ до момента~$t$ включительно.
  
  В то же время необходимо заметить, что предположение о наличии 
<<двухступенчатой>> структуры у объектов (со\-сто\-яние--наблю\-де\-ние) может 
принести пользу при их изучении. Так происходит, например, в теории частично 
наблюдаемых управляемых марковских процессов.
  
  Предположим далее, что на наблюдаемой части траектории процесса задан 
одношаговый доход (в момент~$t$), и будем считать, что этот доход имеет вид 
$g_t\hm=g(x_t)$, где $g:\ X\rightarrow (0,\,1)\subset \mathbb{R}$~--- измеримая числовая 
функция со значениями из интервала (0,\,1).
  
  Обозначим через $v_{t,s}\hm=s^{-1}\sum\limits_{n=1}^s g_{t+n}$ среднее 
арифметическое доходов на промежутке от $t+1$ до $t\hm+s$ ($t\hm\geq0$, $s\hm\geq 1$).
  
  Если объект~$\mu$ управляется согласно стратегии~$\sigma$, то число
  $$
  w_t(\mu,\sigma) =\sup \left\{ c:\ \mathbf{P}_{\mu,\sigma} \left( 
\lim\limits_{\overline{s\rightarrow\infty}} v_{t,s}>c\right) =1\right\}
  $$
характеризует получаемый при этом гарантированный предельный средний доход 
начиная с момента $t=1$. Поскольку $\lim\limits_{\overline{s\rightarrow\infty}} v_{t,s}$ не 
зависит от~$t$, то $w_0(\mu,\sigma)\hm=w_1(\mu,\sigma)\hm=w_2(\mu,\sigma)\hm=\cdots$. 
Величина $w(\mu,\sigma)\hm=w_0(\mu,\sigma)$ играет в дальнейшем роль целевой 
функции и называется просто \textit{доходом} (при управлении объектом~$\mu$ с 
помощью стратегии~$\sigma$).
  
  Из определения дохода следует, что для любого $t>0$ выполняется условие
  $$
  \mathbf{P}_{\mu,\sigma}\left( \lim\limits_{\overline{s\rightarrow\infty}} v_{t,s}\geq 
w(\mu,\sigma)\vert \mathbf{F}_{t-1}\right)=1
  $$
почти наверное.
  Столь общее определение дохода, без предположений об эргодичности, оказывается 
полезным в теоретических рассмотрениях, однако на практике все же среднее 
арифметическое ведет себя более или менее регулярным образом. Поэтому введем 
следующее определение.
{ %\looseness=1

}
  
  Стратегия~$\sigma$ называется \textit{эргодической} по отношению к классу~$M$, 
если для любого объекта $\mu\hm\in M$ и любого $\varepsilon\hm>0$ выполняется 
условие $\sum\limits_{s=1}^\infty a_s\hm<\infty$, где $a_s\hm= 
a_s(\mu,\sigma,\varepsilon)\hm=\sup\limits_{t\geq0} \mathbf{P}_{\mu,\sigma}\left( \left\vert 
v_{t,s}-w(\mu,\sigma)\right\vert >\varepsilon\vert \mathbf{F}_t\right)$. Обозначим еще
  $$
  W=W(\mu) =\sup\limits_\sigma w(\mu,\sigma)\,,
  $$
где точная верхняя грань берется по всем допустимым стратегиям. Стратегия~$\sigma$ 
называется $\varepsilon$-\textit{оп\-ти\-маль\-ной}, если выполняется неравенство
$$
w(\mu,\sigma)\geq W-\varepsilon\,,\enskip \varepsilon\geq 0\,.
$$
  
  Далее объекты будут объединяться в множества объектов (классы объектов). При этом 
без дополнительных оговорок всюду предполагается, что
  \begin{itemize}
  \item все объекты из класса имеют одинаковые пространства состояний, управлений (и 
наблюдений);
  \item в качестве множества допустимых стратегий берется определенное выше 
множество~$\Sigma$;
  \item функция одношаговых доходов~$g$ одна и та же для всех объектов.
  \end{itemize}
  
  Пусть $M$~--- класс объектов. Стратегия~$\sigma$ является равномерно 
  $\varepsilon$-оп\-ти\-маль\-ной относительно этого класса, если последнее неравенство 
выполняется для всех $\mu\hm\in M$. Такую стратегию будем называть также 
  $\varepsilon$-\textit{адап\-тив\-ной} по отношению к классу~$M$. Класс объектов, для 
которого существует $\varepsilon$-адап\-тив\-ная стратегия, называется 
  $\varepsilon$-\textit{адап\-тив\-но управ\-ля\-емым}. (Если $\varepsilon\hm=0$, то 
приставка <<$\varepsilon$->> в этих определениях опускается.)
  
  Основная задача адаптивного управления заключается в построении адаптивных 
стратегий для различных классов объектов. 

К~настоящему вре\-ме\-ни получено много 
решений для многочисленных вариантов этой задачи. Подобные результаты являются 
фактически достаточными условиями адаптивной управ\-ля\-емости. Ниже, однако, будет 
уделено внимание также необходимым условиям существования адаптивных стратегий. 
Подчеркнем, что рассматриваемая постановка задачи предполагает, по сути, наличие 
лишь минимальной априорной информации об объекте управления~--- необходимо знать 
множество управлений~$Y$.

\section{Некоторые условия адаптивной управляемости}

  Пусть $\mu\in M$~--- фиксированный объект, а $\sigma\hm\in \Sigma$~--- 
фиксированная стратегия из некоторой среды. Набор, состоящий из первых $t$ правил 
стратегии~$\sigma$, будем обозначать через $\sigma^t\hm=(\sigma_1, \ldots , \sigma_t)$. 
Таким образом, $\sigma\hm=(\sigma^t, \sigma_{t+1},\sigma_{t+2}, \ldots)$. Положим
  $$
  w_t^*(\mu,\sigma) =w_t^*(\mu,\sigma^t)=\sup\limits_{\sigma_{t+1},\sigma_{t+2}, \ldots} 
w_t(\mu,\sigma)\,,
  $$
где верхняя грань берется по всем допустимым правилам начиная с момента $t\hm+1$. В 
этих обозначениях $w_0^*(\mu,\sigma) \hm=W(\mu)$. Ясно, что $W(\mu)\hm\geq 
w_1^*(\mu,\sigma)\hm\geq w_2^*(\mu,\sigma)\geq \cdots$
  
  Стратегию~$\sigma$ назовем $\varepsilon$-\textit{по\-вреж\-да\-ющей} для 
объекта~$\mu$, если
  $$
  \inf\left\{ t:\ w_t^*(\mu,\sigma)<W(\mu)-\varepsilon\right\} <\infty\,,\enskip \varepsilon>0\,.
  $$
  
  Пример~1 показывает, что существуют объекты, для которых каждая стратегия~--- 
$\varepsilon$-по\-вреж\-да\-ющая (с разными значениями~$\varepsilon$).
  
  \medskip
  
  \noindent
  \textbf{Пример~1.} Множество~$X$ состояний объекта~$\mu$ образовано точками с 
неотрицательными целочисленными координатами на плоскости, $X\hm=\{ (i,j), 
i\hm\geq0,\ j\hm\geq0\}$. Множество управлений $Y\hm=\{1;2\}$. Начальное состояние 
$x_0=(0,\,0)$. Детерминированные переходы между состояниями заданы следующим 
образом ($t\hm>0$, $i\hm\geq0$):
  \begin{align*}
  \mu_t\left( x_t=(i+1{,}0)\vert x_{t-1}=(i,0),y_t=1\right)&=1\,;\\
  \mu_t\left( x_t=(i,j+1)\vert x_{t-1}=(i,j),y_t=1\right)&=1\,,\ j>0\,;\\
  \mu_t\left( x_t=(i,j+1)\vert x_{t-1}=(i,j),y_t=2\right)&=1\,, j\geq 0\,.
  \end{align*}
  
  Одношаговые доходы определены как $g(i,0)\hm=0$, $g(i,j)\hm=1-2^{-i}$ для $i\geq 0$, 
$j\hm>0$.
  
  Стратегия, состоящая из бесконечного повторения управления~1, приносит доход~0. 
Стратегия, в которой управление~2 первый раз применяется (детерминировано) в 
момент~$t$, приносит доход $1\hm-2^{t-1}$, что меньше максимально возможного 
на~$2^{t-1}$. Рандомизация правил и их зависимость от предыстории не вносит 
принципиальных изменений~--- каждая стратегия остается 
  $\varepsilon$-по\-вреж\-да\-ющей относительно предельно наибольшего, но 
недостижимого значения~1.
  
  В примере~2 оптимальная стратегия для любого объекта из класса является 
повреждающей для остальных объектов.
  
  \medskip
  
  \noindent
  \textbf{Пример~2.} Пусть $X\hm= \{0, 1, 2, \ldots\}\cup \{a,b\}$; $Y\hm=\{0;\,1\}$; 
$g(a)\hm=1$; $g(b)\hm=g(i)\hm=0$, $i\hm\geq0$. Зададим счетное множество объектов 
$M\hm=\{\mu^{(k)},\ k\hm=0, 1, \ldots\}$. Пусть для всех~$k$:
  \begin{align*}
  \mu^{(k)}(x_0=0)&=1\,;\\
  \mu^{(k)}(x_{t+1}=i+1\vert x_t=i, y_t=0)&=1\,,\enskip i\geq0\,;\\
     \mu^{(k)}(x_{t+1}=a\vert x_t=k,y_t=1)&=1\,;\\
     \mu^{(k)}(x_{t+1}=b\vert x_t=i,y_t=1) &=1\,,\enskip i\not=k\,;\\
     \mu^{(k)}(x_{t+1}=a\vert x_t=a,y_t=j)&={}\\
&\hspace*{-45mm}{}=\mu^{(k)}(x_{t+1}=b\vert 
x_t=b,y_t=j)=1\,,\enskip j=0\vee 1\,.
     \end{align*}
  
  Таким образом, состояния $a$ и $b$~--- погло\-ща\-ющие, причем в состояние~$a$, 
приносящее максимальный доход, объект~$\mu^{(k)}$ может попасть, только если 
применить управление~1, находясь в со\-сто\-янии~$k$. Первые (существенные) правила 
оптимальной стратегии для объекта~$\mu^{(k)}$ требуют применения управления~0 до 
достижения состояния~$k$, а затем применения в этом состоянии управления~1. Однако 
такая стратегия является повреждающей для всех остальных объектов. Следовательно, для 
класса~$M$ не существует равномерно оптимальной стра\-тегии.
{\looseness=1

}
  
  Пусть $M$~--- класс объектов. Обозначим через $\Sigma_\varepsilon(\mu)$ множество 
$\varepsilon$-по\-вреж\-да\-ющих стратегий для объекта~$\mu$, $\mu\hm\in M$. Положим 
$\Sigma_\varepsilon(M)\bigcap\limits_{\mu\in M}\left( \Sigma\backslash 
\Sigma_\varepsilon(\mu)\right)$.
  
  \medskip
  
  \noindent
  \textbf{Лемма~2.} \textit{Для того чтобы существовала $\varepsilon$-адап\-тив\-ная 
стратегия, необходимо, чтобы $\Sigma_\varepsilon(M)\not=\emptyset$.}
  \medskip
  
  \noindent
  Д\,о\,к\,а\,з\,а\,т\,е\,л\,ь\,с\,т\,в\,о\,.\ Если $\Sigma_\varepsilon\not= \emptyset$, то любая 
допустимая стратегия хотя бы для одного из объектов является 
  $\varepsilon$-по\-вреж\-да\-ющей и, следовательно, не является 
  $\varepsilon$-оп\-ти\-маль\-ной, а потому не может быть равномерно 
  $\varepsilon$-оп\-ти\-маль\-ной по отношению к классу~$M$.
  
  В примере~3, несмотря на наличие по\-вреж\-да\-ющих стратегий, адаптивная стратегия 
существует.
  
  \medskip
  
  \noindent
  \textbf{Пример~3.} Пусть $X\hm=Y\hm=\{1, \ldots , K\}$ и пусть задана 
детерминированная функция~$f:\ X\hm\rightarrow X$, которая представляет собой 
циклическую подстановку на множестве~$X$,  т.\,е.\ $f(i)\not= f(j)$, если $i\not= j$; 
$i,j\hm=1, \ldots , K$. Рассмотрим следующий неоднородный во времени 
детерминированный объект. Положим
  \begin{align*}
  \mu_0(x_0=1)&=1\,;\\
  \mu_t(x_t=f(k)\vert x^{t-1},y^t) &= I_{\{y_t=k\}}\,,\ 0<k\,,\ t\leq K\,;\\
  \mu_t(x_t=f(k)\vert x^{t-1},y^t) &=I_{\{y_{K+1}=k}\,,\\
  & \hspace*{10mm}0<k\leq K\,,\enskip t>K
  \end{align*}
($I_A$~--- индикатор события~$A$).
  
  Одношаговые доходы определим как $g(i)\hm=i$, $i\hm\in X$.
  
  Так определенный объект обозначим через~$\mu^f$. Ясно, что для этого объекта 
траектория управ\-ля\-емо\-го процесса, начиная с момента $K+1$, и, следовательно, доход 
зависят исключительно от управ\-ле\-ния, примененного в момент $K+1$. Доход будет 
максимален (и равен~$K$) тогда и только тогда, когда $y_{K+1}\hm= k^\prime \hm= 
k^*(f)\hm=\argmax\limits_{1\leq k\leq K} f(k)$.
  
  Пусть $M=\{\mu^f\}$~--- совокупность всех объектов данного вида (которая содержит 
$K!$ элементов). Очевидно, для класса~$M$ существует равномерно оптимальная 
стратегия, доставляющая доход, равный~$K$. Например, достаточно вначале в моменты 
$t\hm=1, \ldots , K$ по одному разу применить каждое из управлений, а затем в момент 
$K+1$ применить управление~$k^*$, которое будет выявлено путем наблюдения за 
полученными одношаговыми доходами. Таким образом, на первых тактах необходимо совершить 
<<обучение>>~--- выявить управление, приносящее наибольший одношаговый доход. 
В~то же время существуют и повреждающие стратегии. Например, стратегия, в которой 
первые $K$ правил заключаются в применении управления~1. Правило~$\sigma_{K+1}$ 
такой стратегии может быть построено только в виде зависимости от управления~1 и от 
значения $f(1)$, поэтому при любом его определении найдется объект~$\mu^f$, для 
которого в момент $K+1$ будет с положительной вероятностью предписано применение 
неоптимального управления, и, следовательно, доход будет меньше~$K$.
  
  В примере 3 <<обучение>> оказалось возможным только благодаря знанию структуры 
процессов. Если бы заранее не было известно, что необходимо на первых тактах по разу 
<<испробовать>> все управ\-ле\-ния, то легко можно было пропустить период, когда 
возможно обучение, и совершить тем самым <<непоправимую ошибку>>. Следовательно, 
для того чтобы конструктивно построить равномерно оптимальную стратегию, 
необходима дополнительная информация. Это противоречит избранному принципу 
постановки задачи~--- минимальности априорной информации об объекте. 

Введем более 
жесткое определение адаптивной стратегии, которое, в част\-ности, устраняет указанное 
несоответствие.
  
  Пусть $M$~--- некоторый класс объектов. Эргодическая стратегия~$\sigma$ (ее 
определение дано в конце разд.~2) называется \textit{устойчивой} по отношению к 
классу~$M$, если для любого объекта $\mu\hm\in M$ стратегия~$\tilde{\sigma}$, 
полученная из стратегии~$\sigma$ путем произвольной (допустимой) замены конечного 
числа правил, (1)~имеет одинаковый со стратегией доход 
$w(\mu,\sigma)\hm=w(\mu,\tilde{\sigma})$ и (2)~является эргодической по отношению к 
классу~$M$.

%\columnbreak
  
  Адаптивная стратегия для класса~$M$ называется \textit{строго адаптивной}, если она 
устойчивая по отношению к этому классу.
  
  \medskip
  
  \noindent
  \textbf{Пример~4.} Легко показать, что строго адаптивными являются 
многочисленные адаптивные стратегии для класса управляемых конечных связных 
марковских цепей~[1, 2].
  
  Рассмотрим еще один мотив, выдвигаемый в качестве необходимого условия 
адаптивной управ\-ля\-емости.
  
  \medskip
  
  \noindent
  \textbf{Пример~5.} Пусть класс объектов состоит из функций вещественного 
аргумента~$u$ вида $\mu^y\hm=\mu^y(u)\hm=I_{\{u=y\}}$, $y\hm\in [0,\,1]$. (В~терминах 
управляемых случайных последовательностей: $X\hm= \{0;1]\}$, $Y\hm=[0,1]$; 
$\mu_t(x_t\vert x^{t-1},y^t)\hm=x_t I_{\{y_t=y\}}+ (1-x_t)I_{\{y_t=y\}}$; $g(x)\hm=x$, 
$x\hm\in X$.) Интуитивно представляется очевидным, что невозможно найти максимум 
такой функции за счетное число шагов, если не знать значение, в котором она обращается 
в единицу. В~то же время формально для каждого объекта~$\mu^y$ существует 
оптимальная стратегия. Например, можно постоянно повторять управление~$y$. Однако 
не существует стратегии, равномерно оптимальной по отношению к классу 
$M\hm=\{\mu^y\}$. В~такой стратегии для каждого $y\hm\in [0,\,1]$ необходимо должно 
было бы выполняться следующее условие: $\sigma_t(y_t=y\vert \cdot)>0$ хотя бы для 
одного значения~$t$. Но это невозможно, поскольку для фиксированного значения~$t$ 
данное неравенство может быть выполнено лишь для счетного множества значений~$y$, а 
$t$ также пробегает счетное множество значений. Счетное объединение счетных 
множеств само счетно, поэтому необходимое неравенство не может быть выполнено для 
всех точек на отрезке [0,\,1].
  
  Аналогичные рассуждения показывают, что в данном примере не существует счетного 
множества стратегий, обладающего тем свойством, что для любого объекта найдется 
$\varepsilon$-оп\-ти\-маль\-ная стратегия из этого множества.
  
  Конечное или счетное множество стратегий $\Sigma\hm=\{\sigma(1),\sigma(2), \ldots \}$ 
назовем \textit{базовым} по отношению к классу объектов $M\hm\in \mathcal{M}$, если:
  \begin{enumerate}[(1)]
  \item для любого объекта из $M$ и любого $\varepsilon\hm>0$ существует оптимальная 
стратегия из множества~$\Sigma$;
  \item любая стратегия $\sigma(i)$ является устойчивой по отношению к классу~$M$.
  \end{enumerate}
  
  \smallskip
  
  \noindent
  \textbf{Теорема.} \textit{Строго адаптивная стратегия для класса объектов~$M$ 
существует тогда и только тогда, когда для этого класса существует базовое 
множество стратегий~$\Sigma$.}


%\hfill {\large Приложение~1}

\bigskip

%\pagebreak

\noindent
Д\,о\,к\,а\,з\,а\,т\,е\,л\,ь\,с\,т\,в\,о\ \ теоремы.

Необходимость условий в данном случае является тривиальной, поскольку строго 
адаптивная стратегия, если она существует, образует базовое множество 
стратегий~$\Sigma$, состоящее из одного элемента.
  
  Докажем достаточность. Определим с по\-мощью стратегий из~$\Sigma$ новую 
стратегию $a$ следующим образом. Обозначим
  $$
  \theta_{t,n}=\mathrm{Int}\left(\left( 1-v_{t,n}\right)^{-n}\right)\,,
  $$
где $\mathrm{Int}\left(a\right)$ означает целую часть числа~$a$, и зададим 
последовательность марковских моментов $\tau\hm=\{\tau_n\}$ с помощью рекуррентных 
соотношений

\pagebreak

\noindent
$$
\tau_0=0\,,\enskip \tau_n=\tau_{n-1}+n+\theta_n\,,
$$
где $\theta_n\hm=\theta_{\tau_{n-1},n}$. Соответствующие $\sigma$-ал\-геб\-ры обозначим 
$\mathbf{F}_{(n)}\hm=\mathbf{F}_{\tau_{n-1}}$.
  
  Будем считать, что на пространстве $(\Omega,\mathbf{F})$ задана последовательность 
случайных величин $\beta\hm=\{\beta_n\}$, независимых 
относительно~$\mathbf{F}_{(n)}$. Каждая случайная величина имеет одно и то же 
невырожденное распределение $\{b_i\}$ на множестве номеров стратегий из~$\Sigma$.
  
  Определим правила стратегии $a\hm=a(\Sigma,\beta)$ формулой
  $$
  a_t=\sum\limits_{n=1}^\infty \sigma_t(\beta_n) I_{\{\tau_{n-1}<t\leq \tau_n\}}\,,
  $$
где $\sigma_t(\beta_n)$~--- правило стратегии $\sigma(i)\hm\in\Sigma$ в момент~$t$, если 
$\beta_n\hm=i$.
  
  Наглядно работа стратегии~$a$ выглядит следующим образом. Процесс управления 
разбивается на этапы. Этап с номером $n$ начинается в момент $\tau_{n-1}+1$ и 
оканчивается в момент~$\tau_n;\tau_0\hm=0$. В~момент, предшествующий началу 
очередного этапа, определяется номер стратегии в множестве~$\Sigma$, из которой будут 
взяты правила для применения на данном этапе. Этот номер равен значению случайной 
величины~$\beta_n$. Продолжительность $n$-го этапа равна $n\hm+\theta_n$ и зависит, 
следовательно, от номера этапа и от оценки качества применяемой стратегии, полученной 
в течение первых $n$ тактов этапа. Стратегия~$a$ называется стратегией перебора~[2]. 
Таким образом, последовательность~$\beta$ определяет на каждом этапе выбор стратегии 
из множества~$\Sigma$, правила из которой применяются на этом этапе.
  
  Пусть задан объект $\mu\hm\in M$ и пусть $W\hm=W(\mu)$~--- точная верхняя грань 
доходов для этого объекта, взятая по всем допустимым стратегиям, и пусть %также
  \begin{alignat*}{2}
  W_i&=w(\mu,\sigma(i))\,; &\enskip v_n^{(1)}&=v_{\tau_{n-1},n}\,;\\
  v_n^{(2)}&=v_{\tau_{n-1},n+\theta_n}\,; &\enskip \Delta_n&=\tau_n-\tau_{n-1}=n+\theta_n\,.
  \end{alignat*}
  
  Для произвольного $\varepsilon>0$ определим множества
  $$
  A_n^{(k)}(\varepsilon)=\left\{ v_n^{(k)}\geq W-\varepsilon\right\}\,,
  $$
обозначая их дополнения $\overline{A_n^{(k)}(\varepsilon)}$, $k=1, 2$.
  
  Обозначим
  \begin{align*}
  s_n^{(1)} &= \sum\limits_{l=1}^n I_{A_l^{(1)}(\varepsilon)\cap 
{A_l^{(2)}(2\varepsilon)}} \Delta_l\,;\\
  s_n^{(2)} &= \sum\limits_{l=1}^n I_{A_l^{(1)}\cap 
\overline{A_l^{(2)}(2\varepsilon)}}\Delta_l\,;\\
  s_n^{(3)} &= \sum\limits_{l=1}^n I_{\overline{A_l^{(1)}(\varepsilon)}}\Delta_l\,,
  \end{align*}
так что $\tau_n\hm=\sum\limits_{l=1}^n \Delta_l\hm= s_n^{(1)}\hm+ s_n^{(2)}\hm+ 
s_n^{(3)}$.

\columnbreak

  
  С~помощью введенных обозначений запишем оценку для усредненного дохода к 
моменту~$\tau_n$:
  \begin{multline}
  w_n=\fr{1}{\tau_n}\sum\limits_{t=1}^{\tau_n} g_t=\fr{\sum\limits_{l=1}^n 
v_l^{(2)}\Delta_l} {\sum\limits_{l=1}^n \Delta_l}\geq{}\\
{}\geq (W-2\varepsilon) \fr{s_n^{(1)}} 
{s_n^{(1)}+s_n^{(2)}+s_n^{(3)}}\,.
  \label{e1-kon}
  \end{multline}
  
  Для оценки суммы $s_n^{(1)}$ запишем неравенство
  $$
  s_n^{(1)}\geq \Delta_{v_n}\,,
  $$
в котором обозначено
$$
v_n=\max\left\{ l:\ l\leq n,\ A_l^{(1)}(\varepsilon)\cap A_l^{(2)}(2\varepsilon)\right\}\,.
$$
  
  Оценим вероятность события $B_n\hm=\{v_n\hm\leq n-\ln n\}$, для которого выполняется 
включение
  $$
  B_n\subset \bigcap\limits_{n-\ln n<l\leq n} 
  \overline{A_l^{(1)}(\varepsilon)}\cap \overline{A_l^{(2)}(2\varepsilon)}\,.
  $$
  
  Согласно определениям эргодической стратегии, базового множества стратегий и 
семейства случайных величин~$\beta$ имеем:
  \begin{multline*}
  \mathbf{P}_{a} \left( \overline{A_l^{(1)}(\varepsilon)}\cup\overline{A_l^{(2)} 
(2\varepsilon)}\,\Big\vert \mathbf{F}_{(l)}\right)\leq{}\\
  {}\leq
  \sum\limits_{\substack{{i\in \mathcal{I};}\\ {W_i\leq W-\varepsilon/2}}}\!\!\!\!
   \mathbf{P}_{a}\left(\beta_l=i\vert 
\mathbf{F}_{(l)}\right)+{}\\
{}+  %\substack{{i=\overline{1,n}}\\ {j=\overline{1,l}}}
\sum\limits_{\substack{{i\in \mathcal{I};}\\ {W_i\leq W-\varepsilon/2}}}\!\!\!\!
\mathbf{P}_{a}\left( \overline{A_l^{(1)}(\varepsilon)}, \ \beta_l=i
\vert \mathbf{F}_{(l)}\right)\leq{}\\
  {}\leq \sum\limits_{\substack{{i\in \mathcal{I};}\\ {W_i\leq W-\varepsilon/2}}}\!\!\!\!
  \mathrm{P}_{a}(\beta_l=i)+{}\\
{}+\sum\limits_{\substack{{i\in \mathcal{I};}\\ {W_i> W-
\varepsilon/2}}}
\!\!\!\!\mathbf{P}_{a}\left( v_l^{(1)}\leq W_i-\fr{\varepsilon}{2}, \beta_l=i\vert 
\mathbf{F}_{(l)}\right) \leq{}\\
  {}\leq \sum\limits_{\substack{{i\in \mathcal{I};}\\ {W_i\leq W-\varepsilon/2}}}\!\!\!\!
   b_i+a_l\left( 
\fr{\varepsilon}{2}\right) \leq q<1
  \end{multline*}
при всех достаточно больших~$l$. Отсюда следует, что для всех достаточно больших 
значений~$n$ выполняется неравенство
$$
\mathbf{P}_a(B_n)\leq q^{n-\ln n}\,.
$$
  
  Следовательно, согласно лемме Бо\-ре\-ля--Кан\-тел\-ли
  \begin{equation}
  \mathbf{P}_{a}\left( \overline{\lim\limits_{n\rightarrow\infty}} B_n\right)=0\,.
  \label{e2-kon}
  \end{equation}
  
  Это означает, что
  $$
  s_n^{(1)}\geq \Delta_{v_n}\geq (1-W-\varepsilon)^{-n+\ln n}\,.
  $$
  
  Оценим сумму $s_n^{(2)}$. Обозначив 
$C_n\hm=A_n^{(1)}(\varepsilon)\cap$\linebreak 
$\cap\overline{A_n^{(2)}(2\varepsilon)}$ и $W_{(n)}\hm=\sum\limits_{i\in 
I} W_i I_{\{\beta_n=i\}}$, получим:
  \begin{multline*}
  \mathrm{P}_{a}\left(C_n\vert \mathrm{ F}_{(n)}\right)=
  \mathrm{P}_{a|} \left( C_n, W_{(n)}<W-\fr{3\varepsilon}{2}\vert \mathrm{
  F}_{(n)}\right) +{}\\
  {}+ \mathrm{P}_{a}\left( 
  C_n, W_{(n)}\geq W-\fr{3\varepsilon}{2}\vert \mathrm{
  F}_{(n)}\right)\leq{}\\
  {}\leq \mathrm{P}_{a}\left( v_n^{(1)}>W-\varepsilon,\, W_{(n)}<W-\fr{3\varepsilon}{2}\vert \mathrm{
  F}_{(n)}\right)+{}\\
  {}+
  \mathrm{P}_{a} \left( v_n^{(2)}\leq W-2\varepsilon,\, W_{(n)}\geq W-
\fr{3\varepsilon}{2}\vert \mathrm{
  F}_{(n)}\right)\leq{}\\
  {}\leq \sum\limits_{i\in \mathcal{I}; W_i\leq W- \varepsilon/2} \mathrm{P}_{a}\left(
  v_{\tau_n,n}>W_i+\fr{\varepsilon}{2},\, \beta_l=i\vert\mathrm{F}_{(n)}\right)+{}\\
  {}+\sum\limits_{\substack{{i\in \mathcal{I};}\\ {W_i> W- 3\varepsilon/2}}}\!\!\!\!
   \mathbf{P}_{a} \left( 
v_{\tau_n,n+\theta_n}\leq W_i-\fr{\varepsilon}{2},\,\beta_l=i\vert\mathbf{F}_{(n)}\right)\leq {}\\
{}\leq
a_n\left( \fr{\varepsilon}{2}\right)\,.
  \end{multline*}
  
  Из определения базового множества стратегий следует, что
  $$
  \sum\limits_{n=1}^\infty \mathbf{P}_{a} (C_n)<\infty\,,
  $$
поэтому согласно лемме Бо\-ре\-ля--Кан\-тел\-ли полу\-чаем:
\begin{equation}
\mathbf{P}_{a}\left( \overline{\lim\limits_{n\rightarrow\infty}} C_n\right) =0\,.
\label{e3-kon}
\end{equation}
  
  Отсюда следует, что
  $$
  \sup\limits_n s_n^{(2)}\leq c<\infty\,.
  $$
  
  Для суммы $s_n^{(3)}$ имеем следующую оценку:
  $$
  s_n^{(3)}\geq \sum\limits_{l=1}^n \left(n+(1-W+\varepsilon)^{-l}\right)< n^2+n(1-
W+\varepsilon)^{-n}.
  $$
  
  Подставляя оценки, полученные для сумм $s_n^{(k)}$, в неравенство~(\ref{e1-kon}), 
получаем:
  \begin{multline*}
  w_n\geq (W-\varepsilon) \left( 1+\fr{s_n^{(2)}+s_n^{(3)}}{s_n^{(1)}}\right)^{-1}\geq 
{}\\
  {}\geq (W-\varepsilon)\left( 1+\fr{c+n^2+n(1-W+\varepsilon)^{-n}}{(1-W-\varepsilon/2)^{-
n+\ln n}}\right)^{-1}\geq{}\\
{}\geq W-3\varepsilon
  \end{multline*}
для всех достаточно больших значений~$n$. Отсюда
\begin{equation}
\lim\limits_{\overline{n\rightarrow\infty}} w_n\geq W\,.
\label{e4-kon}
\end{equation}
  
  Рассмотрим далее множество
  $$
  \Omega^\prime =\left\{ \lim\limits_{n\rightarrow\infty} w_n =W\right\}\cap 
\overline{B}\cap\overline{C}\,,
  $$
где $\overline{B}$ и $\overline{C}$ означают соответственно дополнения к множествам 
$B\hm= \overline{\lim\limits_{n\rightarrow\infty}} B_n$ и $C\hm= 
\overline{\lim\limits_{n\rightarrow\infty}} C_n$.
  
  Согласно формулам~(\ref{e2-kon})--(\ref{e4-kon})
  $$
  \mathbf{P}_{a}\left(\Omega^\prime\right) =1\,.
  $$
  
  Определим следующие события:
  
  \noindent
  \begin{align*}
  D_{n,t}^{(1)} &= \left\{ \tau_{n-1}<t\leq \tau_{n-1}+n\right\} \cap \Omega^\prime\,;\\
  D_{n,t}^{(2)} &= \left\{\tau_{n-1}+n<t\leq \tau_n\right\}\cap \Omega^\prime\,;\\
  D_{n,t}^{(3)} &= \left\{ \tau_{n-1}<t\leq \tau_n\right\} \cap \Omega^\prime\,.
  \end{align*}
  
  На множестве $D_{n,t}^{(1)}$ усредненный доход $v_t\hm=v_{0,t}\hm=
  t^{-1}\sum\limits_{s=1}^t g_s$ оценивается с помощью формулы~(\ref{e1-kon}) как
  
    \noindent
  $$
  v_t\geq \fr{\tau_{n-1} w_n}{\tau_{n-1}+n+\theta_n}\geq W-\varepsilon_n^{(1)}\,,
  $$
где $\varepsilon_n^{(1)}\hm\rightarrow0$ при $n\hm\rightarrow\infty$.
  
  Пусть событие $D_{n,t}^{(2)}$ имеет место. Тогда $\theta_n\geq (1\hm- 
W\hm+\varepsilon)^{-n}$. Кроме того, из определения событий $B_n$, $B$, 
$D_{n,t}^{(2)}$ следует, что для всех достаточно больших значений~$n$ выполняется 
неравенство $v_n\hm> n-\ln n$. Следовательно, на множестве~$D_n^{(2)}$ справедлива 
оценка

  \noindent
  $$
  v_t\geq \fr{\tau_{n-1} w_n}{\tau_{n-1}+n+\theta_n}\geq W-\varepsilon_n^{(2)}\,,
  $$
где $\varepsilon_n^{(2)}\hm\rightarrow0$ при $n\hm\rightarrow\infty$.
  
  Из определения событий $C_n$, $C$, $D_{n,t}^{(3)}$ вытекает, что
  
    \noindent
  $$
  D_{n,t}^{(3)} \subset \left\{ \min\limits_{n<m\leq n+\theta_n} v_{n,m}\geq W-
2\varepsilon\right\}\,,
  $$
поэтому на множестве $D_n^{(3)}$ справедливы неравенства:

  \noindent
\begin{multline*}
\!\!v_t\geq \fr{\tau_{n-1} w_n}{t}+\left(1- \fr{\tau_{n-1}}{t}\right) \left( 1-\tau_n\right)^{-1} 
\!\!\sum\limits_{s=\tau_{n-1}+1}^t \!\!\!\!g_s\geq{}\\
{}\geq \fr{\tau_{n-1} w_n}{t}+\left( 1-\fr{\tau_{n-1}}{t}\right)\left( W-2\varepsilon\right) \geq 
W-2\varepsilon -\varepsilon_n^{(3)},
\end{multline*}
где $\varepsilon_n^{(3)}\rightarrow0$ при $n\hm\rightarrow\infty$.

\pagebreak
  
  Таким образом, на множестве
  $$
  D_{n,t}=\bigcup\limits_{k=1}^3 D_{n,t}^{(k)} = \left\{ \tau_{n-1}<t\leq \tau_n\right\} \cap 
\Omega^\prime
  $$
имеет место оценка $v_n\hm\geq W-\varepsilon-\varepsilon_n$, где 
$\varepsilon_n\hm\rightarrow 0$ при $n\hm\rightarrow\infty$. Достаточность утверждения 
теоремы следует из соотношений $\Omega\hm= \bigcup\limits_{n=1}^\infty \left\{ \tau_{n-
1}\hm<t\hm\leq \tau_n\right\}$ и $\lim\limits_{t\rightarrow\infty} I_{D_{n,t}}\hm=0$.

\section{Заключение}

  Адаптивные стратегии, позволяющие достигать цели в условиях информационной 
неопреде\-лен\-ности, основываясь на <<обучении>> в процессе взаимодействия с объектом, 
находят все более широкое практическое применение. 

В~этой работе было уделено 
внимание теоретическим аспектам адаптивного подхода. Сформулированы определения 
адаптивных стратегий и приведена формальная постановка задачи адаптивного 
управления. Сформулированы и доказаны некоторые утверждения о необходимых 
условиях и достаточных условиях адап\-тив\-ной управляемости. 

Продолжение исследований 
в данном на\-прав\-ле\-нии позволит найти ответы на принципиальные вопросы, в каких 
ситуациях можно рассчитывать на <<приспособление к неизвестной среде>> и сколь 
универсальными могут быть <<обучающиеся>> алгоритмы.



{\small\frenchspacing
{%\baselineskip=10.8pt
\addcontentsline{toc}{section}{Литература}
\begin{thebibliography}{9}


  \bibitem{1-kon}
  \Au{Sragovich~V.\,G.}
  Mathematical theory of adaptive control.~--- Singapore: World Scientific, 2006.
  \bibitem{2-kon}
  \Au{Коновалов~М.\,Г.}
  Методы адаптивной обработки информации и их приложения.~--- М.: ИПИ РАН, 2007.
  
  \label{end\stat}
  
  \bibitem{3-kon}
  \Au{Неве~Ж.}
  Математические основы теории вероятностей.~--- М.: Мир, 1969.
\end{thebibliography}
}
}


\end{multicols}%3
%\newcommand{\A}{{\mathbf A}}
%\newcommand{\B}{{\mathbf B}}
%\newcommand{\la}{{\lambda}}
%\newcommand{\be}{\begin{equation}}
%\newcommand{\ee}{\end{equation}}
%\newcommand{\ber}{\begin{eqnarray}}
%\newcommand{\eer}{\end{eqnarray}}

%\newcommand{\nin}{\noindent}
%\newcommand{\non}{\nonumber}
%\newcommand{\half}{\frac{1}{2}}
%\newcommand{\quarter}{\frac{1}{4}}

\def\stat{zeifman}

\def\tit{ОБ ОДНОМ КЛАССЕ МАРКОВСКИХ СИСТЕМ ОБСЛУЖИВАНИЯ$^*$}

\def\titkol{Об одном классе марковских систем обслуживания}

\def\autkol{Я.\,А.~Сатин, А.\,И.~Зейфман, А.\,В.~Коротышева, С.\,Я.~Шоргин}
\def\aut{Я.\,А.~Сатин$^1$, А.\,И.~Зейфман$^2$, А.\,В.~Коротышева$^3$, С.\,Я.~Шоргин$^4$}

\titel{\tit}{\aut}{\autkol}{\titkol}

{\renewcommand{\thefootnote}{\fnsymbol{footnote}}\footnotetext[1]
{Исследование поддержано РФФИ, гранты 11-07-00112-а и 11-01-12026-офи-м.}}


\renewcommand{\thefootnote}{\arabic{footnote}}
\footnotetext[1]{Вологодский государственный педагогический
университет, yacovi@mail.ru}
\footnotetext[2]{Вологодский государственный педагогический университет;  
Институт проблем информатики Российской академии наук; 
Институт социально-экономического развития территорий Российской академии наук,  a\_zeifman@mail.ru}
\footnotetext[3]{Вологодский государственный педагогический
университет,  a\_korotysheva@mail.ru}
\footnotetext[4]{Институт проблем информатики Российской академии наук, SShorgin@ipiran.ru}


\Abst{Рассматриваются модели обслуживания, описываемые конечными марковскими 
цепями с непрерывным временем. При этом предполагается,  что интенсивности 
поступления и обслуживания требований не зависят от числа требований в сис\-те\-ме. 
Получены оценки скорости сходимости и устойчивости различных характеристик таких сис\-тем.}

\KW{нестационарные марковские системы
обслуживания; скорость сходимости; устойчивость; оценки}

 \vskip 14pt plus 9pt minus 6pt

      \thispagestyle{headings}

      \begin{multicols}{2}
      
            \label{st\stat}

\section{Введение}

Классы систем массового обслуживания, описываемых процессами
рождения и гибели (стационарными и нестационарными, с катастрофами)
изучались начиная с 1970-х~гг.\ многими авторами
(см., например,~[1--6]). С~помощью методов,
разработанных одним из авторов настоящей \mbox{статьи}\linebreak (подробное изложение
этих методов приведено в~[7--9]), для таких сис\-тем
удалось получить точные оценки скорости сходимости и устойчивости.

Оказывается, этот же подход можно применить и к существенно более 
общему классу систем обслуживания.

Рассмотрим систему массового обслуживания, число требований в которой 
описывается нестационарной марковской цепью с непрерывным временем и 
конечным пространством состояний, причем требования могут поступать и 
обслуживаться группами.

Пусть $X=X(t)$, $t\geq 0$,~--- число требований в системе обслуживания ($0 \hm\le X(t) \hm\le r$).

Обозначим через 
\begin{gather*}
p_{ij}(s,t)=\mathrm{Pr}\left\{ X(t)=j\left| X(s)=i\right.
\right\}\,,\\
i,j \ge 0\,,\ 0\leq s\leq t\,,
\end{gather*}
переходные вероятности
процесса $X\hm=X(t)$, а через  $p_i(t)\hm=\mathrm{Pr}\left\{ X(t) \hm=i \right\}$~---
его вероятности состояний.

Будем предполагать, что интенсивности поступления и обслуживания $k$ требований в 
момент~$t$ в сис\-те\-ме об\-слу\-жи\-ва\-ния ($\lambda_{k}(t)$ и  $\mu_{k}(t)$ соответственно)  
не зависят от числа требований, находящихся в системе в момент~$t$, являются локально 
интегрируемыми на $[0,\infty)$ функциями времени~$t$ и, кроме того, 
$\lambda_{k+1}(t) \hm\le \lambda_{k}(t)$ и  $\mu_{k+1}(t) \hm\le \mu_{k}(t)$ при всех~$k$ 
и почти при всех $t \hm\ge 0$.

Тогда для описания вероятностной динамики процесса получаем прямую систему Колмогорова в виде
\begin{equation} 
\fr{d\vp}{dt}=A(t)\vp(t)\,,
\label{ur_1}
\end{equation}
 где
 {\footnotesize
\begin{multline*}
A(t)={}\\
{}=
\begin{pmatrix}
a_{00}(t) & \mu_1(t)  & \mu_2(t)   & \mu_3(t)  & \mu_4(t) & \cdots & \mu_r(t) \\
\la_1(t)   & a_{11}(t)  & \mu_1(t)  & \mu_2(t)   & \mu_3(t)  & \cdots & \mu_{r-1}(t) \\
\la_2(t)  & \la_1(t)    & a_{22}(t)& \mu_1(t)  & \mu2(t)    &  \cdots & \mu_{r-2}(t) \\
\cdots&\cdots&\cdots&\cdots&\cdots&\cdots&\cdots \\
\la_r(t) & \la_{r-1}(t) & \la_{r-2}(t) & \cdots & \la_2(t)  & \la_1 (t)   &  a_{rr}(t)
\end{pmatrix}\,,
\end{multline*}}
причем  
$$
a_{ii}(t)=-\sum\limits_{k=1}^{i}\mu_k(t) - \sum\limits_{k=1}^{r-i} \la_{r-k}(t)\,.
$$

Далее будем обозначать через $\|\bullet\|$  $l_1$-нор\-му, т.\,е.\ 
$\|{\vx}\|\hm=\sum|x_i|$, а $\|B\| \hm= \max\limits_j \sum\limits_i |b_{ij}|$, 
если $B \hm= (b_{ij})_{i,j=0}^{r}$.
%
Тогда, в частности, имеем 
$$
\|A(t)\| \le 2\sum\limits_{k=1}^{r}(\la_{k}(t)+ \mu_k(t))
$$ 
при  всех $t \hm\ge 0$.

Через 
$$
E(t,k) = E\left\{X(t)\left|X(0)\hm=k\right.\right\}
$$ 
будем далее обозначать математическое ожидание процесса (среднее число требований) в момент~$t$ 
при условии, что в нулевой момент времени он находится в состоянии~$k$, 
а через $E_{\bf p}(t)$ обозначим математическое ожидание процесса в момент~$t$ 
при начальном распределении вероятностей состояний $\mathbf{p}(0) \hm= \mathbf{p}$.

\section{Оценки скорости сходимости}

Рассмотрим вспомогательную последовательность положительных чисел $\{d_i\}$, $i\hm=1, \dots,r$.

Положим
\begin{equation*}
d=\min\limits_{1 \le i \le r} d_i\,; \enskip 
G=\sum\limits_{i=1}^r d_i\,; \enskip W=\min\limits_k \fr{d_k}{k}\,.
%\label{2.01}
\end{equation*}

Рассмотрим величины
\begin{multline*}
\alpha_i(t)= -a_{ii}(t)+\la_{r-i+1}(t)-\sum\limits_{k=1}^{i-1}(\mu_{i-k}(t)-{}\\
{}-
\mu_i(t))\fr{d_k}{d_i}-\sum\limits_{k=1}^{r-i}(\la_k(t)-\la_{i+r-1}(t))\fr{d_{k+i}}{d_i}\,,
%\label{2.02}
\end{multline*}

\noindent
\begin{equation*}
\alpha(t)=\min\limits_{1 \le i \le r}\alpha_i(t)\,.
%\label{2.03}
\end{equation*}

\smallskip

\noindent
\textbf{Теорема~1.} \textit{Пусть существует последовательность положительных 
чисел  $\{d_j\}$ такая, что}
\begin{equation}
\int\limits_0^{\infty} \alpha(t)\, dt = + \infty\,.
\label{2.031}
\end{equation}
\textit{Тогда $X(t)$ слабо эргодичен, при
любых начальных условиях} $\mathbf{p}^*(s)$, $\mathbf{p}^{**}(s)$ 
\textit{и любых $s$, $t$, $0\le s\le t$, справедлива оценка
\begin{equation} 
\label{2.04}
\|\vp^*(t)-\vp^{**}(t)\| \le \fr{8G}{d}\,e^{-\int\limits_s^t {\alpha(u)\,du}}\,.
\end{equation}
Кроме того,  $X(t)$ имеет предельное среднее $\phi(t)$ и при любых~$k$ и $t \hm\ge 0$ справедливо неравенство}:
\begin{equation}
\label{2.05}
|E(t,k)-\phi(t)|\le \fr{4G}{W}\,e^{-\int\limits_0^t {\alpha(u)\,du}}\,.
\end{equation}


\smallskip


\noindent
Д\,о\,к\,а\,з\,а\,т\,е\,л\,ь\,с\,т\,в\,о\,.\

Пользуясь предложенным в предыдущих работах способом, 
выразим 
$$
p_0=1-\sum\limits_{1\le i \le r}{p_i}\,.
$$

Тогда получим неоднородное уравнение:
\begin{equation} 
\label{ur_per}
\fr{d\vz}{dt}= B(t)\vz(t)+\vf(t)\,, 
%\label{2.06}
\end{equation}
\noindent
где $\vf(t)=\left(\la_1, \la_2,\cdots,\la_r \right)^{\mathrm{T}}$;

\end{multicols}


\hrule

\vspace*{6pt}

\begin{equation*}
B = \left(
\begin{array}{cccccccc}
a_{11}- \la_1   & \mu_1 - \la_1   & \mu_2 - \la_1   & \mu_3 -\la_1   & \cdots& \cdots & \mu_{r-1}- \la_1  \\
\la_1 -\la_2    & a_{22} -\la_2  & \mu_1-\la_2   & \mu_2 -\la_2     & \cdots&  \cdots & \mu_{r-2} -\la_2 \\
\la_2 -\la_3    & \la_1 -\la_3   & a_{33} -\la_3  & \mu_1-\la_2   & \cdots&  \cdots & \mu_{r-3} -\la_3 \\
\cdots&\cdots&\cdots&\cdots&\cdots&\cdots&\cdots \\
\la_{r-1} -\la_r  &\la_{r-2} -\la_r & \cdots & \cdots & \la_2 -\la_r   & \la_1 -\la_r     &  a_{rr} -\la_r
\end{array}
\right)\,.
%\label{2.07}
\end{equation*}

Рассмотрим треугольную матрицу
\begin{equation*}
D=\begin{pmatrix}
d_1   & d_1 & d_1 & \cdots & d_1 \\
0   & d_2  & d_2  &   \cdots & d_2 \\
\cdots&\cdots&\cdots&\cdots&\cdots \\
0  & 0 & \cdots & 0 &  d_r
\end{pmatrix}
%\label{2.08}
\end{equation*}
и соответствующую норму $\|{\bf z}\|_{D}\hm=\|D {\bf z}\|_1$.

Тогда имеем:
\begin{equation*}
 D BD^{-1}=\left(
\begin{array}{ccccccc}
a_{11}-\la_r  &  (\mu_1-\mu_2) \fr{d_1}{d_2}  & (\mu_2-\mu_3)\fr{d_1}{d_3}  & \cdots &  (\mu_{r-1}-\mu_r)\fr{d_1}{d_r} \\
(\la_1-\la_r) \fr{d_2}{d_1} &  a_{22}-\la_{r-1}  &(\mu_1-\mu_3)\fr{d_2}{d_3}  & \cdots &  (\mu_{r-2}-\mu_r)\fr{d_2}{d_r} \\
(\la_2-\la_r) \fr{d_3}{d_1} &  (\la_1-\la_{r-1})\fr{d_3}{d_2}   &a_{33}-\la_{r-2}   & \cdots &  (\mu_{r-3}-\mu_r)
\fr{d_3}{d_r}  \\
\cdots&\cdots&\cdots&\cdots&\cdots \\
(\la_{r-1} -\la_r) \fr{d_r}{d_1} & (\la_{r-2} -\la_{r-1}) \fr{d_r}{d_2}  & (\la_{r-3} -\la_{r-2}) \fr{d_r}{d_3}  & \cdots & a_{rr}-\la_1 \\
\end{array}
\right)\,.
%\label{2.09}
\end{equation*}


\begin{multicols}{2}


Далее, оценивая логарифмическую норму оператора~$B(t)$ (см., например, 
подробное рассмотрение в~[8--10]), получаем
\begin{multline*}
\gamma \left(B(t)\right)_{1D} = \gamma \left(DB(t)D^{-1}\right)_{1}={}\\
{}=
\max \left(\vphantom{\sum\limits_{k=1}^{i-1}}
a_{ii}(t) - \la_{r-i+1}(t) + \sum\limits_{k=1}^{i-1}\left(\mu_{i-k}(t)-{}\right.\right.\\
\left.\left.{}-\mu_i(t)\right)
\fr{d_k}{d_i} +
\sum\limits_{k=1}^{r-i}(\la_k(t)-\la_{i+r-1}(t))\fr{d_{k+i}}{d_i}\right) ={}\\
{}=
 - \min \alpha_i(t) = - \alpha(t)\,.
% \label{2.10}
\end{multline*}
Тогда\\[-7.9pt]
\begin{equation*}
\|\vz^*(t)-\vz^{**}(t)\|_{1D}\le  e^{-\int\limits_s^t {\alpha(u)du}}\|\vz^*(s)-\vz^{**}(s)\|_{1D}
%\label{2.11}
\end{equation*}
для всех $0 \le s \le t$ и любых начальных условий $\vz^*(s)$, $\vz^{**}(s)$.

Теперь, учитывая оценки для сравнения норм (см., например,~\cite{z08b}), получаем:
\begin{multline*}
\|\vp^*(t)-\vp^{**}(t)\| \le 2\|\vz^*(t)-\vz^{**}(t)\| \le{}\\
{}\le  \fr{4}{d}\|\vz^*(t)-\vz^{**}(t)\|_{1D}\le{} \\
{} \le \fr{4}{d}\,e^{-\int\limits_s^t {\alpha(u)\,du}}\|\vz^*(s)-\vz^{**}(s)\|_{1D} 
\le{}\\
{}\le
 \fr{4G}{d}\,e^{-\int\limits_s^t {\alpha(u)\,du}}\|\vz^*(s)-\vz^{**}(s)\| \le{} \\
{} \le  \fr{4G}{d}\,e^{-\int\limits_s^t {\alpha(u)\,du}}\|\vp^*(s)-\vp^{**}(s)\| \le 
\fr{8G}{d}\,e^{-\int\limits_s^t {\alpha(u)\,du}} 
%\label{2.11-a}
\end{multline*}
для любых начальных условий ${\bf p^*}(s)$, ${\bf p^{**}}(s)$ и любых $s,t$, $0\hm\le s\hm\le t$.

Из слабой эргодичности процесса с конечным пространством состояний 
вытекает существование предельного среднего, начальные условия для которого можно 
в общем случае выбрать произвольно.
Для оценки средних воспользуемся неравенством, приведенным в параграфе~2.3 из~\cite{z08b}:
\begin{multline*}
\|{\bf z}\|_{1D} = d_0 \left|\sum\limits_{i=1}^{\infty} p_i \right|
+ d_1 \left|\sum\limits_{i=2}^{\infty} p_i \right| + \dots \ge{}\\
{}\ge 
 W \sum\limits_{k \ge 1} k \left|\sum\limits_{i \ge k} p_i\right| \ge \fr{W}{2}
\sum\limits_{k \ge 1} k \left|p_k\right|\,.  
%\label{2.12}
\end{multline*}
Получаем теперь
\begin{multline*}
|E(t,k)-\phi(t)|\le \fr{2}{W}\,\|\vp^*(t)-\vp^{**}(t)\|_{1D}\le {} \\
{}\le\fr{2}{W}\,e^{-\int\limits_0^t {\alpha(u)\,du}}\|{\bf e}_k -
\vp^{**}(0)\|_{1D} \le \frac{4G}{W}e^{-\int\limits_0^t
{\alpha(u)\,du}}\,,
%\label{2.13}
\end{multline*}
что и требовалось доказать.
\columnbreak

%\smallskip

\noindent
\textbf{Замечание~1.} {Положим в условиях теоремы~1 
$$
\beta(t)=\max\limits_{1 \le i \le r}\alpha_i(t)\,.
$$ 
Тогда, пользуясь внедиагональной неотрицательностью матрицы $DB(t)D^{-1}$ 
с помощью методики, описанной в~\cite{z08b, z95b}, получаем справедливость неравенства

\noindent
\begin{equation*} 
%\label{2.14}
\|\vp^*(t)-\vp^{**}(t)\| \ge \fr{d}{8G}\,e^{-\int\limits_s^t {\beta(u)\,du}}
\end{equation*}
при любых $s$, $t$, $0\le s\le t$ и уже не при любых начальных условиях~${\bf p^*}(s)$, 
${\bf p^{**}}(s)$, а таких, что  $D\left({\bf p^*}(s) \hm-{\bf p^{**}}(s)\right) \hm\ge 0.$ 
Следовательно, оценки тео\-ре\-мы~1 будут заведомо иметь точный по времени порядок, если удастся 
выбрать вспомогательную последовательность $\{d_i\}$ так, что $\alpha(t)\hm=\beta(t)$, т.\,е.\ 
все $\alpha_i(t)$ одинаковы (не зависят от индекса~$i$)}.



\smallskip

Введем теперь в рассмотрение величины

\vspace*{-1pt}

\noindent
\begin{multline*}
\zeta_i(t)= -a_{ii}(t)+\la_{r-i+1}(t)+{}\\
{}+\sum\limits_{k=1}^{i-1}\left(\mu_{i-k}(t)-
\mu_i(t)\right) \fr{d_k}{d_i}+{}\\
{}+\sum\limits_{k=1}^{r-i}\left(\la_k(t)-\la_{i+r-1}(t)\right)\fr{d_{k+i}}{d_i}\,;
%\label{2.0211}
\end{multline*}
\begin{equation*}
\chi(t)=\max\limits_{1 \le i \le r}\zeta_i(t)\,.
%\label{2.0311}
\end{equation*}

\noindent
\textbf{Замечание 2.} {В условиях теоремы~1 при любых начальных условиях 
${\bf p^*}(s)$, ${\bf p^{**}}(s)$ и любых $s,t$,  $0\le s\le t$, 
справедлива следующая двухсторонняя оценка скорости сходимости:

\vspace*{-1pt}

\noindent
\begin{multline*} 
%\label{2.041}
\!\!\!\fr{d}{4G}\,e^{-\int\limits_s^t {\chi(u)\,du}}\|\vp^*(s)-\vp^{**}(s)\| \le
 \|\vp^*(t)-\vp^{**}(t)\| \le {}\\
 {}\le\fr{4G}{d}\,e^{-\int\limits_s^t {\alpha(u)\,du}}\|\vp^*(s)-\vp^{**}(s)\|.
\end{multline*}
Таким образом, можно оценить и сверху и снизу время  вхождения 
сис\-те\-мы обслуживания в предельный режим. Более подробно о получении 
нижних оценок см., например, в~\cite{z95b, gz05}.}

\smallskip

Рассмотрим два частных случая теоремы.

\smallskip

\noindent
\textbf{Следствие 1}. \textit{Пусть при выполнении остальных условий теоремы~1 
вместо}~(\ref{2.031}) \textit{выполняется условие $\alpha(t) \hm\ge \alpha \hm> 0$ 
почти при всех $t \hm\ge 0$. Тогда вместо}~(\ref{2.04}) \textit{и}~(\ref{2.05}) 
\textit{справедливы оценки}:

\vspace*{-1pt}

\noindent
\begin{align*} 
%\label{2.15}
\|\vp^*(t)-\vp^{**}(t)\| &\le \fr{8G}{d}\,e^{-\alpha \left(t-s\right)}\,;
\\
%\label{2.16}
|E(t,k)-\phi(t)|&\le \fr{4G}{W}\,e^{- \alpha t}\,.
\end{align*}

\pagebreak

%\smallskip

Положим 
\begin{gather*}
M_0=\max\limits_{|t-s|\le 1}\int\limits_s^t \alpha(u)\,du;\\
\alpha^* = \int\limits_0^1 \alpha(t)\, dt\,; \quad
M=e^{M_0+\alpha^*}\,.
\end{gather*}
С учетом неравенства 
$$
e^{-\int\limits_s^t {\alpha(u)\,du}} \hm\le M e^{-\alpha^* (t-s)}
$$ 
получаем следующее утверждение.

\smallskip

\noindent
\textbf{Следствие~2.} \textit{Пусть все $\lambda_k(t)$ и $\mu_k(t)$ 1-пе\-ри\-одич\-ны,  
а при выполнении остальных условий теоремы~1 вместо}~(\ref{2.031}) 
\textit{выполняется условие  $\alpha^* \hm> 0$.  Тогда предельный режим (скажем, $\vp^*(t)$) 
и соответствующее ему предельное среднее $\phi^*(t)$ можно выбрать 
1-пе\-ри\-оди\-че\-ски\-ми, а вместо}~(\ref{2.04}) \textit{и}~(\ref{2.05}) 
\textit{справедливы оценки}:
\begin{equation*} 
%\label{2.17}
\|\vp(t) - \vp^*(t)\| \le \fr{8GM}{d}\,e^{-\alpha^*t}
\end{equation*}
\textit{и, кроме того,}
\begin{equation*}
|E(t,k)-\phi^*(t)|\le \fr{4GM}{W}\,e^{-\alpha^*t}
%\label{2.18}
\end{equation*}
\textit{при любом $k$ и $t \ge 0$}.



\section{Устойчивость}

Рассмотрим также <<возмущенный>> процесс обслуживания $\bar{X}\hm=\bar{X}(t)$, $t\hm\geq 0$, 
в котором интенсивности поступления и обслуживания требований также не зависят от чис\-ла 
требований в системе, обозначая его соответствующие характеристики теми же буквами с 
чертой сверху. Для прос\-то\-ты записи оценок будем предполагать, что возмущения 
<<равномерно малы>>, т.\,е.\ выполняется неравенство $\| A(t)-\bar{A}(t)\| \hm\le \varepsilon$. 
Первые результаты для нестационарных цепей с непрерывным временем получены в~\cite{z85}, 
а детальное рассмотрение для более общего случая неравномерных оценок можно без труда 
провести так же, как это сделано в~\cite{z98, ae}. Для получения требуемых равномерных 
оценок устойчивости необходима экспоненциальная эргодичность соответствующего процесса, 
т.\,е.\ существование положительных констант $N$, $a$ таких, что  для правой части~(\ref{2.04}) 
справедливо неравенство:
\begin{equation}
e^{-\int\limits_s^t {\alpha(u)\,du}} \le Ne^{-a\left(t-s\right)}\,.
\label{3.01}
\end{equation}
Оценка~(\ref{3.01}) заведомо имеет место, в частности, если выполнены условия одного из следствий 
предыду\-ще\-го параграфа.

\smallskip

\noindent
\textbf{Теорема~2.}
\textit{Пусть выполнены условия теоремы~1 и}~(\ref{3.01}). \textit{Тогда при
 любых начальных условиях ${\bf p}(s)$ и ${\bar{\bf p}}(s)$ для процессов~$X(t)$ 
 и $\bar{X}(t)$ соответственно справедливы следующие оценки устойчивости:}
\begin{align*} 
%\label{3.02}
\limsup_{t \to \infty}  \|{\bf p}(t)- \bar{\bf p}(t)\| &\le
\fr{\varepsilon(1+\ln(4GN/d))}{a}\,;
\\
% \label{3.03}
\limsup\limits_{t \to \infty}   |E_{\bf p}(t)- \bar{E}_{\bar{\bf p}(t)}|&\le 
\fr{r \varepsilon(1+\ln(4GN/d))}{a}\,.
\end{align*}


\smallskip

\noindent
Д\,о\,к\,а\,з\,а\,т\,е\,л\,ь\,с\,т\,в\,о\ основано на подходе, 
введенном для стационарных процессов в~\cite{mit03} и описанном для нестационарной 
ситуации в~\cite{z11}.
Если  при любых начальных условиях для исходного процесса справедлива оценка
\begin{equation*} 
%\label{3.04}
\|\vp(t) - \vp^*(t)\| \le ce^{-b\left(t-s\right)}\,,
\end{equation*}
то, полагая
\begin{multline*}
\beta (t, s)=\sup\limits_{ \| {\bf v} \| =1, \sum {v_i}=0}
{\|V(t,s){\bf v}(t,s)\|} ={}\\
{}= \fr{1}{2} \max_{i,j} \sum\limits_k {|p_{ik}(t,
s)-p_{jk}(t, s)|}\,, 
\end{multline*}
где $V(t, s)$~--- матрица Коши
уравнения~(\ref{ur_1}), получаем в итоге следующее неравенство:
\begin{equation*}
\|{\bf p}(t)-\bar{\bf p}(t)\| \le{}
\begin{cases}
\|{\bf p}(s)-{\bf \bar{p}}(s)\|+ (t-s)\varepsilon \,, &\\
&\hspace*{-35mm} 0<t< b^{-1} \ln \left(\fr{c}{2}\right)\,; \\
b^{-1}\left(\ln \fr{c}{2} +1-\fr{c}{2}\,e^{-b(t-s)}\right)\varepsilon +{}&\\
{}+
\fr{c}{2}\,e^{-b(t-s)} \|{\bf p}(s)-{\bf \bar{p}}(s)\|\,, &\\
&\hspace*{-30mm}t\ge b^{-1}\ln \left(\fr{c}{2}\right)
\end{cases}
%\label{3.05}
\end{equation*}
для любых начальных условий ${\bf p}(s)$ и $\bar{\bf p}(s)$.
Из неравенств~(\ref{2.04}) и~(\ref{3.01}) вытекает, что $b=a$, $c={8GN}/{d}$.  
Устремив $t \hm\to \infty$ и взяв $s\hm=0$, получаем требуемые оценки.


\smallskip

\noindent
\textbf{Замечание~3.} 
В полученную оценку устойчивости для математического ожидания процесса 
в качестве множителя входит размерность~$r$, поэтому иногда лучший результат 
удается получить при помощи другого подхода, описанного в работе~\cite{z11}.

\smallskip

Положим 
$$
S=\max\limits_{{1 \le i, j \le r}} \fr{d_i}{d_j}\,,
$$ 
и пусть числа $K, L$ таковы, что 

\noindent
$$
d_1\la_1(t) + (d_1+d_2)\la_2(t) + \dots + 
\left(\sum\limits_{1 \le i \le r}d_i\right) \la_r(t) \le K\,,
$$ 
а 

\noindent
\begin{multline*}
d_1(\la_1(t)-\bar{\la}_1(t)) + (d_1+d_2)(\la_2(t)-\bar{\la}_2(t)) + \dots\\
\dots + 
\left(\sum\limits_{1 \le i \le r}d_i\right) (\la_r(t)-\bar{\la}_r(t)) \le 
L\varepsilon
\end{multline*} 
почти при всех $t \ge 0.$

\smallskip

\noindent
\textbf{Теорема~3.}
\textit{Пусть  выполнены условия теоремы~2 и, кроме того, при всех~$k$ 
и почти всех $t \hm\ge 0$ $\la_k(t) \hm< \infty$. Тогда при любых начальных условиях 
${\bf p}(s)$ и ${\bar{\bf p}}(s)$ для процессов $X(t)$ и $\bar{X}(t)$ 
соответственно справедливо неравенство}

\noindent
\begin{equation*}
\limsup\limits_{t \to \infty}   |E_{\bf p}(t)- \bar{E}_{\bar{\bf p}(t)}|\le 
\fr{ N\varepsilon\left(L a+ 2KNS\right)}{W a \left(a-2\varepsilon S\right)}\,.
\end{equation*}


\smallskip

\noindent
Д\,о\,к\,а\,з\,а\,т\,е\,л\,ь\,с\,т\,в\,о.\
 Перепишем исходную систему~(\ref{ur_per}) для невозмущенного процесса в следующем виде:
 \noindent
 
\begin{equation*}
\fr{d\vp}{dt}=\bar{B}(t)\vp(t) + {\bf f}(t)+\left(B(t)-\bar{B}(t)\right)\vp(t)\,.
%\label{eq112-n}
\end{equation*}
Тогда

\noindent
\begin{multline*}
\vp(t)=\bar{U}(t,0)\vp(0)+\int\limits_0^t \bar{U}(t,\tau){\bf{f}}(\tau) \, d\tau+{}\\
{}+\int\limits_0^t \bar{U}(t,\tau) \left(B(\tau)-\bar{B}(\tau)\right)\vp(\tau)\, d\tau\,;
\end{multline*}

\vspace*{-9pt}

\begin{equation*}
\hspace*{-15mm}\bar{\vp}(t)=\bar{U}(t,0)\bar{\vp}(0)+\int\limits_0^t \bar{U}(t,\tau){\bf{f}}(\tau) \, d\tau,
\end{equation*}
где $U(t,s)$~--- матрица Коши для уравнения~(\ref{ur_per}).
В любой норме при одинаковых начальных условиях получаем следующую оценку:
%\noindent
\begin{multline}
 \label{3000}
\!\!\!\!\!\!\left\|\vp(t)-\bar{\vp}(t)\right\|\le \!\!\int\limits_0^t \!\!\|\bar{U}(t,\tau)\|
\left(\| B(\tau)-\bar{B}(\tau)\| \|\vp(\tau)\| +\right.\\
\left.{}+ \| \vf(\tau)-\bar{\vf}(\tau)\|\right)\,d\tau\,.\!
\end{multline}
Имеем почти при всех $t \ge 0$:
\begin{equation*}
\|B(t)-\bar{B}(t)\|_{1D}=\|D(B(t)-\bar{B}(t))D^{-1}\| \le 2S\varepsilon\,;
%\label{3002}
\end{equation*}
%
%\vspace*{-14pt}
%
%\noindent
\begin{multline*}
\|{\bf f}(t)\|_{1D} \le d_1\la_1(t) + (d_1+d_2)\la_2(t) + \dots + {}\\
{}+
\left(\sum\limits_{1 \le i \le r}d_i\right) \la_r(t) \le K\,, 
\quad \|\vf(\tau)-\bar{\vf}(\tau)\|_{1D} \le L\varepsilon\,.
%\label{3002-a}
\end{multline*}
А тогда
\begin{multline*}
\gamma(\bar{B}(t))_{1D} \le \gamma(DB(t)D^{-1})+\|B(t)-\bar{B}(t)\|_{1D} \le  {}\\
{}\le -
\alpha(t)+2S \varepsilon \,.
% \label{3003}
\end{multline*}

Оценим теперь
\begin{multline*} 
%\label{8402}
\!\|{\bf p}(t)\|_{1D} \le
\|U(t){\bf p}(0) \|_{1D} +
 \int\limits_0^t \!\!\| U(t,\tau){\bf f}(\tau)\, d\tau \|_{1D} \le {}\\
 {}\le
 N e^{-a t} \| \vp(0)\|_{1D}  + \fr{K N}{a}.
\end{multline*}

 Тогда с учетом~(\ref{3000}) получаем:
\begin{multline*} 
%\label{3004}
\left\|\vp(t)-\bar{\vp}(t)\right\|_{1D}\le N\int\limits_0^t e^{-(a - 2\varepsilon S)(t-\tau)}\times{}\\
{}\times
\left(2S\varepsilon (N e^{-a \tau} \| \vp(0)\|_{1D}  + \fr{K N}{a}) +  L\varepsilon \right)\, d\tau  \le {} \\
{}\le  o(1)+\fr{ N\varepsilon(L+{2KNS}/{a})}{a-2\varepsilon S}\,. 
\end{multline*}

\vspace*{-9pt}

\section{Примеры}

\noindent
\textbf{Пример 1.}

Рассмотрим исходный процесс обслуживания с интенсивностями 
$\la_1(t)\hm=\la_2(t)\hm=\la_3(t)\hm=\la(t) \hm= 3\hm+\sin{2\pi t}$, 
$\mu_1(t)\hm=\mu_2(t)\hm=  \mu(t) \hm= 2\hm+\cos{2\pi t}$, 
$\la_4(t)=\ldots=\la_r(t)\hm=\mu_3(t)=\ldots=\mu_r(t)\hm=0$. Выберем последовательность  
$d_k\hm=h^k$, где $0{,}82 \hm< h \hm<1$. Тогда имеем
$$
d=h^r\,; \quad G \le \fr{h}{1-h}\,; \quad W=\fr{h^r}{r}\,.
$$

Будем предполагать, что возмущенный процесс имеет такую же структуру 
мат\-ри\-цы интенсивностей, причем $|\la(t)\hm-\bar{\la}(t)| \hm\le \varepsilon$ 
и  $|\mu(t)\hm-\bar{\mu}(t)| \hm\le \varepsilon$ почти при всех $t \hm\ge 0$. 
Отметим кстати, что при этом $\| A(t)\hm-\bar{A}(t)\| \hm\le 10 \varepsilon$ почти при 
всех $t \hm\ge 0$. Рассмотрим дальнейшие оценки:
$$
S=\fr{1}{h^2}\,; \ K=4 \left(3h+2h^2+h^3\right)\,; \ L=3h+2h^2+h^3\,;
$$
$$
\alpha(t) \ge \la(t)\left(3 - h - h^2 -h^3\right)-\mu(t)\left(\fr{1}{h^2}+\fr{1}{h}-2\right)\,;
$$
$$
\alpha^*= 3\left(3 - h - h^2 -h^3\right)-2\left(\fr{1}{h^2}+\fr{1}{h}-2\right)\,;
$$


\noindent
\begin{multline*}
M_0 \le \int\limits_0^1 |\alpha(t)|\, dt \le 4\left(3 - h - h^2 -h^3\right)+{}\\
{}+
3\left(\fr{1}{h^2}+\fr{1}{h}-2\right)\,;
\end{multline*}

\vspace*{-9pt}

\noindent
$$
M=e^{\alpha^*+M_0}\,.
$$

Если, например, взять 
$h\hm=0{,}9$, то $\alpha^*\hm=0{,}992$, $M_0\hm=3{,}281$, $M\hm=71{,}737$.

Тогда получаем следующие оценки.

По следствию~2
\begin{align*}
 \|{\bf p}(t)- {\bf p^{*}}(t)\| &\le \fr{8Me^{-\alpha^*t}}{h^{r-1}(1-h)}\,;\\
|E_{\bf p}(t)-\phi^*(t)| &\le  \fr{4Mre^{-\alpha^*t}}{h^{r-1}(1-h)}\,.
\end{align*}

По теореме~2 ($N=M$, $a=\alpha^*$) с использованием оценок следствия~2
\begin{align*}
\limsup\limits_{t \to \infty} \|{\bf p}(t)- \bar{\bf p}(t)\| &\le{} \notag\\
&\hspace*{-15mm}{}\le \fr{\varepsilon(1+\ln({4M}/({h^{r-1}(1-h)})))}{\alpha^*}\,;\\
\limsup\limits_{t \to \infty}   |E_{\bf p}(t)- \bar{E}_{\bar{\bf p}(t)}| &\le \notag\\
&\hspace*{-15mm}{}\le\fr{r\varepsilon(1+\ln(4M/(h^{r-1}(1-h))))}{\alpha^*}\,.
\end{align*}

По теореме~3 с использованием оценок следствия~2
\begin{multline*}
\limsup\limits_{t \to \infty}   |E_{\bf p}(t)- \bar{E}_{\bar{\bf p}(t)}| \le {}\\
{}\le
\fr{rM\varepsilon(3h+2h^2+h^3)(\alpha^* h^2+8M)}{h^r\alpha^*(\alpha^* h^2-2\varepsilon)}\,.
\end{multline*}

\noindent

\textbf{Пример 2.}

Рассмотрим процесс с интенсивностями 
$\la_1(t)\hm=\la_2(t)\hm=\ldots=\la_r(t) \hm= \la(t) \hm= 3\hm+\sin{2\pi t}$; 
$\mu_1(t)\hm=\mu_2(t)\hm= \mu(t) \hm= 2+\cos{2\pi t}$;
$\mu_3(t)=\ldots=\mu_r(t)=0$.

Будем предполагать, что возмущенный процесс имеет такую же структуру 
мат\-ри\-цы интен\-сив\-ностей, причем $|\la(t)-\bar{\la}(t)| \hm\le \varepsilon$ и  
$|\mu(t)-\bar{\mu}(t)| \hm\le \varepsilon$ почти при всех $t \hm\ge 0$. 
При этом будем иметь $\| A(t)\hm-\bar{A}(t)\| \hm\le 2r \varepsilon$ почти при всех $t \hm\ge 0$.

Выберем последовательность $d_k\hm=1$. Тогда  
\begin{gather*}
d=1\,; \enskip G=r\,; \enskip W=\fr{1}{r}\,; \enskip S=1\,; \\
K=\fr{4r(1+r)}{2}\,; \quad L=\fr{r(1+r)}{2}\,;
\\
\alpha(t)=\la(t)\,; \ \alpha=2\,; \ \alpha^*=3\,; M_0 \le 4\,; \ M \le  e^{7}\,.
\end{gather*}

И получаем следующие оценки.

\columnbreak

По следствию~1
\begin{align*}
 \|{\bf p^*}(t)- {\bf p^{**}}(t)\| &\le 8re^{-2t}\,;\\
|E_{\bf p}(t)- \phi(t)|&\le  4r^2 e^{-2t}\,.
\end{align*}

По следствию~2
\begin{align*}
\|{\bf p}(t)- {\bf p^{*}}(t)\| &\le 8re^{7-3t}\,;
\\[6pt]
|E_{\bf p}(t)- \phi^*(t)| &\le 4r^2 e^{7-3t}\,.
\end{align*}

По теореме~2 ($N=1$, $a=\alpha$) с учетом оценок следствия~1
\begin{align*}
\limsup\limits_{t \to \infty} \|{\bf p}(t)- \bar{\bf p}(t)\| &\le 
\fr{\varepsilon(1+\ln{4r})}{2}\,;
\\[6pt]
\limsup\limits_{t \to \infty}   |E_{\bf p}(t)- \bar{E}_{\bar{\bf p}(t)}|
&\le \fr{r\varepsilon(1+\ln{4r})}{2}\,.
\end{align*}

По теореме~2 ($N=M$, $a=\alpha^*$) с учетом оценок следствия~2
\begin{align*}
\limsup\limits_{t \to \infty} \|{\bf p}(t)- \bar{\bf p}(t)\| &\le 
\fr{\varepsilon(8+\ln{4r})}{3}\,;
\\
\limsup\limits_{t \to \infty}   \left|E_{\bf p}(t)- \bar{E}_{\bar{\bf p}(t)}\right| &\le 
\fr{r\varepsilon(8 + \ln{4r})}{3}\,.
\end{align*}

По теореме~3 с учетом оценок следствия~1
\begin{equation*}
\limsup\limits_{t \to \infty}   \left|E_{\bf p}(t)- \bar{E}_{\bar{\bf p}(t)}\right| \le 
\fr{5 \varepsilon r^2 (1+r)}{4(1- \varepsilon)}\,.
\end{equation*}

По теореме~3 с учетом оценок следствия~2
\begin{equation*}
\limsup\limits_{t \to \infty}   \left|E_{\bf p}(t)- \bar{E}_{\bar{\bf p}(t)}\right| \le 
\fr{\varepsilon e^{7} r^2 (1+r) (3+8e^{7})}{6(3-2\varepsilon)}\,.
\end{equation*}

{\small\frenchspacing
{%\baselineskip=10.8pt
\addcontentsline{toc}{section}{Литература}
\begin{thebibliography}{99}

 \bibitem{b} %1
\Au{Баруча-Рид~А.\,Т.} Элементы теории марковских процессов и их
приложения.~--- М.: Наука, 1969.

\bibitem{gm}  %2
\Au{Гнеденко~Б.\,В., Макаров~И.\,П.} Свойства решений задачи с потерями
в случае периодических интенсивностей~// Дифф. уравнения, 1971.
Вып.~7. С.~1696--1698.

\bibitem{g1}   %3
\Au{Gnedenko~D.\,B.} On a generalization of Erlang formulae~// 
Zastosow. Mat., 1971. Vol.~12. P.~239--242.

\bibitem{S}  %4
\Au{Саати~Т.\,Л.} Элементы теории массового обслуживания
 и ее приложения.~--- М.: Сов. радио, 1971.

\bibitem{g}  %5
\Au{Gnedenko~B., Soloviev~A.} On the conditions of the
existence of final probabilities for a Markov process~// Math.
Operations. Stat., 1973. P.~379--390.

\bibitem{gk} %6
\Au{Гнеденко~Б.\,В., Коваленко~И.\,Н.} Введение в теорию массового
обслуживания.~--- М.: Наука, 1987.
\pagebreak

\bibitem{gz00}   %7
\Au{Granovsky~B.\,L., Zeifman~A.\,I.}  The N-limit of spectral gap of 
a class of birth-death Markov chains~//
 Appl. Stoch. Models Business Ind., 2000. Vol.~16. P.~235--248.

\bibitem{z08b}  %8
\Au{Зейфман~А.\,И., Бенинг~В.\,Е., Соколов~И.\,А.} 
Марковские цепи и модели с непрерывным временем.~--- М.: Элекс-КМ, 2008.

\bibitem{dzp} %9
\Au{Van Doorn~E.\,A., Zeifman~A.\,I., Panfilova~T.\,L.}  
Bounds and asymptotics for the rate of convergence of birth-death processes~//  
Th. Prob. Appl., 2010. Vol.~54. P.~97--113.

\bibitem{z95b}   %10
\Au{Zeifman~A.\,I.} Upper and lower bounds on the rate of
convergence for nonhomogeneous birth and death processes~//  Stoch.
Proc. Appl., 1995. Vol.~59. P.~157--173.

\bibitem{gz05}  %11
\Au{Granovsky~B.\,L., Zeifman~A.\,I.} On the lower bound of the spectrum
 of some mean-field models~// Theory Prob. Appl., 2005. Vol.~49. P.~148--155.
 
\bibitem{z85}  %12
\Au{Zeifman~A.\,I.} Stability for contionuous-time
nonhomogeneous Markov chains~// Lect. Notes Math.,  1985. Vol.~1155.
P.~401--414.

\bibitem{z98} %13
\Au{Zeifman~A.} Stability of birth and death processes~// 
J.~Math. Sci., 1998. Vol.~91. P.~3023--3031.

\bibitem{ae} %14
\Au{Андреев~Д., Елесин~М., Кузнецов~А., Крылов~Е., Зейфман~А.}
Эргодичность и устойчивость нестационарных систем обслуживания~//
Теория вероятностей и математическая статистика, 2003. Т.~68.
С.~1--11.

\bibitem{mit03} %15
\Au{Mitrophanov~A.\,Yu.} Stability and exponential convergence of continuous-time 
Markov chains~//  J. Appl. Prob., 2003. Vol.~40. P.~970--979.

\label{end\stat} 

\bibitem{z11} %16
\Au{Зейфман~А.\,И., Коротышева~А.\,В., Панфилова~Т.\,Л., Шоргин~С.\,Я.} 
Оценки устойчивости  для некоторых систем обслуживания с катастрофами~//  
Информатика и её применения, 2011. Т.~5. Вып.~3. С.~27--33.
 \end{thebibliography}
}
}


\end{multicols}          %4????+
\def\stat{kor-kor}



\def\tit{МОДИФИЦИРОВАННЫЙ СЕТОЧНЫЙ МЕТОД РАЗДЕЛЕНИЯ ДИСПЕРСИОННО-СДВИГОВЫХ
СМЕСЕЙ НОРМАЛЬНЫХ ЗАКОНОВ$^*$}



\def\titkol{Модифицированный сеточный метод разделения дисперсионно-сдвиговых
смесей нормальных законов}

\def\aut{В.\,Ю.~Королев$^1$,  А.\,Ю.~Корчагин$^2$}

\def\autkol{В.\,Ю.~Королев,  А.\,Ю.~Корчагин}

\titel{\tit}{\aut}{\autkol}{\titkol}

{\renewcommand{\thefootnote}{\fnsymbol{footnote}} \footnotetext[1]
{Работа поддержана Российским научным фондом (проект 14-11-00364).}}


\renewcommand{\thefootnote}{\arabic{footnote}}
\footnotetext[1]{Факультет
вычислительной математики и кибернетики Московского государственного
университета им.\ М.\,В.~Ломоносова; Институт проблем информатики
Российской академии наук; victoryukorolev@yandex.ru}
\footnotetext[2]{Факультет вычислительной математики и кибернетики
Московского государственного университета им.\ М.\,В.~Ломоносова;
sasha.korchagin@gmail.com}

%\vspace*{2pt}



\Abst{Описывается модифицированный двухэтапный
сеточный метод разделения дис\-пер\-си\-он\-но-сдви\-го\-вых смесей нормальных
законов, представляющий собой альтернативу чистому ЕМ (expectation-maximization)
ал\-го\-рит\-му. На
первом этапе этого алгоритма строится дискретная аппроксимация для
смешивающего распределения, на втором этапе подбирается абсолютно
непрерывное распределение из заранее заданного семейства, например,
обобщенных обратных гауссовских законов, ближайшее к~дискретному
распределению, полученному на первом этапе. Обсуждаются вопросы
сходимости этого двухэтапного алгоритма. Доказана монотонность
сеточного итерационного метода, используемого на первом этапе.
Подробно обсуждается вопрос оптимального выбора параметров метода,
прежде всего сетки, накидываемой на носитель смешивающего
распределения. С~этой целью предложены статистические оценки
квантилей смешивающего распределения. Эффективность метода
иллюстрируется примерами конкретных вычислений оценок параметров
обобщенных гиперболических распределений.}

\KW{смесь распределений вероятностей;
дис\-пер\-си\-он\-но-сдви\-го\-вая смесь нормальных законов; обобщенное
гиперболическое распределение; ЕМ-ал\-го\-ритм; сеточный метод
разделения смесей}

\vspace*{1pt}

%\vspace*{2pt}

\DOI{10.14357/19922264140402}


\vskip 12pt plus 9pt minus 6pt

\thispagestyle{headings}

\begin{multicols}{2}

\label{st\stat}

\section{Введение}

При {\it практическом} решении задачи моделирования и исследования
волатильности (изменчивости) хаотических стохастических процессов
ключевым этапом является статистическое разделение смесей
вероятностных распределений. Задача разделения смесей~---
статистического оценивания параметров смесей вероятностных
распределений~--- в~деталях разобрана, например, в~книге~\cite{k2011}.

Для решения задачи разделения смесей вероятностных распределений
традиционно используются итерационные процедуры типа ЕМ-ал\-го\-рит\-ма.
К~сожалению, классический ЕМ-ал\-го\-ритм обладает рядом серьезных
недостатков при его применении к~смесям нормальных законов, а~именно:
он демонстрирует крайнюю неустойчивость по отношению к~исходным
данным и~начальным приближениям.

Для преодоления этих недостатков
предложено много модификаций ЕМ-ал\-го\-рит\-ма (см., например,~\cite{k2011}).
Вместе с тем в~указанной книге предложен и~исследован
принципиально новый~--- сеточный~--- метод приближенного решения
задачи разделения смесей. В~работе~\cite{n2013} подробно исследованы
вопросы сходимости сеточных методов разделения смесей.

В соответствии с подходом к~статистическому анализу хаотических
стохастических процессов, в~частности к~решению задачи декомпозиции
волатильности таких процессов, развитом в~книге~\cite{k2011},
в~общем случае на практике приходится решать задачу разделения
конечных смесей нормальных законов с~произвольно большим числом
неизвестных параметров (параметров компонент и~их весов).
И~хотя в~большинстве приложений возникают смеси не более чем с~пятью--семью
компонентами, даже при использовании таких смесей, скажем, в~задачах
анализа и~прогнозирования финансовых рисков приходится моделировать
траекторию движения точки в~пространствах, размерность которых
соответственно лежит в~пределах от~14 (для пятикомпонентных смесей)
до~20 (для семикомпонентных смесей), что существенно увеличивает
вычислительные и~временн$\acute{\mbox{ы}}$е ресурсы, необходимые для практического
решения указанных задач.

Поскольку во многих ситуациях (например,
при прогнозировании на основе высокочастотных данных) эти задачи
необходимо решать в~режиме, близком к~реальному времени, для
создания эффективных методов статистического анализа на основе
смешанных моделей на первый план выходит проб\-ле\-ма снижения
размерности решаемой задачи, т.\,е.\ параметрического пространства.

Одним из возможных подходов к~снижению размерности является
априорное сужение классов допусти\-мых смесей. К~примеру, при решении
многих задач, связанных с~анализом процессов атмосферной или
плазменной турбулентности, а~так\-же процессов, описывающих эволюцию
различных финансовых индексов, высочайшую адекватность
продемонстрировали модели, основанные на дис\-пер\-си\-он\-но-сдви\-го\-вых
смесях нормальных законов. Класс таких смесей очень обширен
и,~в~част\-ности, включает в~себя обобщенные гиперболические распределения,
которые были введены О.-Е.~Барн\-дорфф-Ниль\-се\-ном в~1977--1978~гг.\ как
класс специальных сдвиг-мас\-штаб\-ных смесей нормальных законов~\cite{BN1977, BN1978}.
Пусть $\alpha\hm\in\r$, $\beta\hm\in\r$. Если
функцию распределения обобщенного гиперболического закона
с~параметрами~$\alpha$, $\beta$, $\nu$, $\mu$, $\lambda$ обозначить
$P_{GH}(x;\alpha,\beta,\nu,\mu,\lambda)$, то по определению
\begin{multline}
P_{GH}(x;\alpha,\beta,\nu,\mu,\lambda)={}\\
{}=
\int\limits_{0}^{\infty}\Phi\left(\fr{x-\beta-\alpha
z}{\sqrt{z}}\right)\,p_{GIG}(z;\nu,\mu,\lambda)\,dz\,,\\
x\in\r\,,
\label{e1-kor}
\end{multline}
где $\Phi(x)$~--- стандартная нормальная функция распределения:
$$
\Phi(x)=\int\limits_{-\infty}^{x}\varphi(z)\,dz\,,\enskip
\varphi(x)=\fr{1}{\sqrt{2\pi}}e^{-x^2/2}\,,\enskip  x\in\mathbb{R}\,;
$$
$p_{GIG}(x;\nu,\mu,\lambda)$~--- плот\-ность обобщенного обратного
гауссовского распределения:
\begin{multline*}
p_{GIG}(x;\nu,\mu,\lambda)={}\\
{}=\fr{\lambda^{\nu/2}}{2\mu^{\nu/2}
K_{\nu}\left(\sqrt{\mu\lambda}\right)}\,
x^{\nu-1}\exp\left\{-\fr{1}{2}\left(\fr{\mu}{x}+\lambda
x\right)\right\}\,,\\ x>0\,.
\end{multline*}
Здесь $\nu\in\r$;
$$
\begin{array}{lll}
\mu>0\,, & \lambda\geqslant0\,, & \mbox{если }\nu<0\,;\\[6pt]
\mu>0\,, & \lambda>0\,, & \mbox{если }\nu=0\,;\\[6pt]
\mu\geqslant0\,, & \lambda>0\,, & \mbox{если }\nu>0\,;
\end{array}
$$
$K_{\nu}(z)$~--- модифицированная бесселева функция третьего рода
порядка~$\nu$:

\noindent
\begin{multline*}
K_{\nu}(z)=\fr{1}{2}\int\limits_{0}^{\infty}y^{\nu-1}\exp
\left\{-\fr{z}{2}\left(y+\fr{1}{y}\right)\right\}\,dy\,,\\
z\in\mathbb{C}\,,\enskip \mathrm{Re}\,z>0\,.
\end{multline*}
Обратим внимание, что в~(1) смешивание происходит одновременно и~по
параметру сдвига, и~по параметру масштаба, но так как эти параметры
в~(1)  связаны жесткой зависимостью, так что параметр сдвига
смешиваемого распределения пропорционален его дисперсии, то
фактически смесь~(1) является {\it однопараметрической} и~поэтому
называется {\it дис\-пер\-си\-он\-но-сдви\-го\-вой} (см., например,~\cite{BN1982}).

Другим примером дис\-пер\-си\-он\-но-сдви\-го\-вых смесей нормальных законов
являются обобщенные дисперсионные гам\-ма-рас\-пре\-де\-ле\-ния, в~которых
смешивающими являются обобщенные гам\-ма-рас\-пре\-де\-ле\-ния~\cite{ks2012, zk2013}.

В указанных семействах смесей число неизвестных параметров равно
пяти или шести (если\linebreak учитывать неслучайный сдвиг). Вместе
с~тем у~подоб\-ных моделей имеются довольно серьезные тео\-ре\-ти\-че\-ские
обоснования: в~работах~\cite{zk2013, k2013} показано, что указанные
модели являются асимптотическими аппроксимациями в~простой
предельной схеме случайного суммирования и~потому могут успешно
применяться для анализа процессов типа остановленных случайных
блужданий. Эти выводы подтверждены статистическим анализом
вы\-со\-ко\-час\-тот\-ных финансовых данных, в~результате которого выявлен
синхронизированный характер изменения интенсивностей потоков заявок
в~сис\-те\-мах электронных торгов, что естественно приводит к~синхронизированному
поведению па\-ра\-мет\-ров сдвига и~диффузии в~соответствующих моделях вида смесей
нормальных законов~\cite{kckg2013}.

\section{Описание моди\-фи\-ци\-ро\-ван\-но\-го
сеточного ме\-то\-да разделения дисперсионно-сдвиговых смесей
нормальных законов и~его свойства}

Оказывается, что сеточные методы разделения смесей довольно
эффективны не только при разделении конечных смесей нормальных
законов, но и~при разделении произвольных дис\-пер\-си\-он\-но-сдви\-го\-вых
смесей нормальных законов. Поясним сказанное на примере задачи
оценивания па\-ра\-мет\-ров обобщенных гиперболических распределений.

Для решения задачи оценивания параметров обобщенных гиперболических
распределений традиционно используется метод, предложенный в~статье~\cite{p2004}
и~по сути являющийся классическим ЕМ-ал\-го\-рит\-мом,
приспособленным к~конкретной задаче, и,~соответственно, наследующий
присущие ЕМ-ал\-го\-рит\-мам недостатки.

Рассмотрим следующий альтернативный двухэтапный метод. На первом
этапе на поло\-жи\-тельной полупрямой выделим основную часть носителя
смешивающего распределения, т.\,е.\ \mbox{ограниченный} интервал,
вероятность которого, вычисленная в~соответствии со смешивающим
распределением, практически равна единице. На этот интервал накинем
конечную сетку, содержащую, возможно, очень много {\it известных}
узлов $u_1,\ldots,u_K$. Считая параметр сдвига~$\beta$ равным нулю,
приблизим искомое обобщенное гиперболическое распределение конечной
смесью нормальных законов:

\noindent
\begin{multline}
P_{GH}(x;\,\alpha,0,\nu,\mu,\lambda)\approx{}\\
{}\approx \sum\limits_{i=1}^K
p_i\Phi\left(\fr{x-\alpha u_i}{\sqrt{u_i}}\right)\,,\enskip
x\in\mathbb{R}\,.\label{e2-kor}
\end{multline}
В смеси, стоящей в~правой части соотношения~(2), неизвестными
являются только параметры $p_1,\ldots,p_{K-1}$ и~$\alpha$. Пусть
$x_1,\ldots,x_n$~--- анализируемая выборка значений случайной
величины с~оцениваемым обобщенным гиперболическим распределением.
Итерационный процесс, определяющий сеточный ЕМ-ал\-го\-ритм для данной
задачи, задается следующим образом. Пусть
$p_1^{(m)},\ldots,p_{K-1}^{(m)}$ и~$\alpha^{(m)}$~--- оценки параметров
$p_1,\ldots,p_{K-1}$ и~$\alpha$ на $m$-й итерации,
$p_K^{(m)}\hm=1\hm-p_1^{(m)}-\cdots-p_{K-1}^{(m)}$. Обозначим

\noindent
\begin{align*}
\varphi_{ij}^{(m)}&=\fr{1}{\sqrt{u_i}}\varphi\left(\fr{x_j-\alpha^{(m)}u_i}{\sqrt{u_i}}\right)\,;
\\
g_{ij}^{(m)}&=\fr{p_i^{(m)}\varphi_{ij}^{(m)}}{\sum\limits_{r=1}^K
p_r^{(m)}\varphi_{rj}^{(m)}}\,,\\
&\hspace*{14mm}i=1,\ldots,K\,;\enskip j=1,\ldots,n\,.
\end{align*}
Тогда, используя стандартные рассуждения, определяющие
вычислительные формулы EM-ал\-го\-рит\-ма для параметров конечной смеси
нормальных законов (см, например,~[1, разд.~5.3.7--5.3.8]),
следует положить

\noindent
\begin{equation}
p_i^{(m+1)}=\fr{1}{n}\sum\limits_{j=1}^n g_{ij}^{(m)}\,, \enskip
i=1,\ldots,K\,.\label{e3-kor}
\end{equation}
Обозначим $\overline{x}=(1/n)\sum\limits_{j=1}^nx_j$. Используя
соотношение~(5.3.24) в~\cite{k2011}, с~учетом очевидного равенства
$\sum\limits_{i=1}^K g_{ij}^{(m)}\hm=1$ можно заметить, что уточненная
оценка параметра~$\alpha$ имеет вид:

\columnbreak

\noindent
\begin{equation}
\alpha^{(m+1)}=\fr{\overline{x}}{\sum\limits_{i=1}^K u_ip_i^{(m+1)}}\,,
\label{e4-kor}
\end{equation}
т.\,е.\ равна отношению генерального выборочного среднего и~текущего
эмпирического среднего смешивающего распределения, что вполне
согласуется с~тем, что в~соответствии с~приводимым ниже соотношением~(\ref{e5-kor})
в~данном случае ${\sf E}X\hm=\alpha{\sf E}U$.

В силу монотонности классического ЕМ-ал\-го\-рит\-ма справедливо следующее
утверждение.

\smallskip

\noindent
\textbf{Теорема~1.} {\it Пусть узлы $u_1,\ldots,u_K$ сетки различны,
неотрицательны и~известны. Тогда итерационный процесс $(3)$--$(4)$
является монотонным, т.\,е.\ каждая его итерация не уменьшает
целевую сеточную функцию правдоподобия}
\begin{multline*}
L(p_1,\ldots,p_K,\alpha;x_1,\ldots,x_n)={}\\
{}=
\prod\nolimits_{j=1}^n\left[\sum\nolimits_{i=1}^K
\fr{p_i}{\sqrt{u_i}}\,\varphi\left(\fr{x_j-\alpha^{(m)}u_i}{\sqrt{u_i}}\right)\right].
\end{multline*}

\smallskip

\noindent
\textbf{Замечание~1.} В~разд.~5.7.4 книги~\cite{k2011} показано, что
при каждом фиксированном значении параметра~$\alpha$ сеточная
функция правдоподобия\linebreak
$L(p_1,\ldots,p_{K-1},\alpha;\,x_1,\ldots,x_n)$ вогнута по
аргументам $p_1,\ldots,p_{K-1}$. Поэтому на каждом шаге
итерационного процесса вместо соотношения~(3) можно\linebreak использо\-вать
любой более быстрый алгоритм максимизации функции
$L(p_1,\ldots,p_{K-1},\alpha^{(m)};\,x_1,\ldots$\linebreak $\ldots,x_n)$ по переменным
$p_1,\ldots,p_{K-1}$. Например, оценки весов $p_1,\ldots,p_K$ можно
искать методом условного градиента~\cite{k2011, kn2010}.

\smallskip

Таким образом, на первом этапе получаются оценки параметра~$\alpha$
и~весов всех узлов~$u_i$ конечной сетки, накинутой на носитель
смешивающего обобщенного обратного гауссовского распределения
$P_{\mathrm{GIG}}(z;\,\nu,\mu,\lambda)$.

На втором этапе остается применить ка\-кой-ли\-бо стандартный метод
подгонки обобщенного обратного гауссовского распределения
$P_{\mathrm{GIG}}(z;\,\nu,\mu,\lambda)$ к~эмпирическим данным типа
гистограммы $(u_1, p_1),\ldots, (u_K, p_K)$. Например, параметры~$\nu$,
$\mu$ и~$\lambda$ можно оценить, минимизируя соответствующую
статистику хи-квад\-рат. Или же, например, можно решить задачу
наименьших квад\-ратов:
\begin{multline*}
(\nu^*,\mu^*,\lambda^*)={}\\
{}=\arg\min\limits_{\nu,\mu,\lambda}\sum\limits_{i=1}^K
\left[p_i- \!\!\!\!\!
\int\limits_{(1/2)\left(u_{i-1}+u_i\right)}^{(1/2)(u_i+u_{i+1})}\!\!\!\!\!\!\!\!\!\!\!\!\!\!\!
p_{GIG}(u;\,\nu,\mu,\lambda)\,du\right]^2,
\end{multline*}
где $u_0=0$; $u_{K+1}\hm=\infty$.

На практике хорошие результаты показал подход с решением задачи
наименьших квадратов. Для поиска параметров использовался алгоритм
ns2sol, описанный в~книге~\cite{DSch1983}. Указанный алгоритм
доступен во многих статистических пакетах, отличается высоким
быстродействием и~возможностью при желании задавать разумные
интервалы для поиска параметров.

%\vspace*{-9pt}

\section{О практическом выборе сетки
на~первом этапе моди\-фи\-ци\-ро\-ван\-но\-го
сеточного метода разделения дисперсионно-сдвиговых смесей нормальных
законов}

Естественно, что при использовании указанного двухэтапного метода
в~динамическом режиме крайне важным становится вопрос о~выборе
наиболее эффективных и~быстродействующих численных процедур и~их
параметров. В~частности, исключительную важность приобретает
правильный выбор сетки на первом этапе. Рассмотрим этот вопрос
подробнее.

Формально рассматриваемая задача выглядит так: по наблюдаемым
значениям $x_1,\ldots,x_n$ требуется построить статистическую оценку
верхней границы квантилей заданного порядка сме\-ши\-ва\-юще\-го закона так,
чтобы как можно точнее оценить носитель смешивающего распределения.

В дальнейшем будем считать, что $x_1,\ldots,x_n$~--- независимые
реализации случайной величины $X\hm=Y\sqrt{U}+\alpha U$, где $Y$~---
случайная величина со стандартным нормальным распределением, а~$U$~---
независимая от нее случайная величина с~обобщенным обратным
гауссовским распределением. Тогда, очевидно, распределение случайной
величины~$X$ имеет вид~(1). Предположим, что у~случайной величины~$U$
существуют моменты первых двух порядков. Тогда, как несложно видеть,
\begin{equation}
{\sf E}X={\sf E}Y\cdot{\sf E}\sqrt{U}+\alpha{\sf E}U=\alpha{\sf
E}U\,.\label{e5-kor}
\end{equation}
При этом по усиленному закону больших чисел с~вероятностью единица
$\overline x\hm\longrightarrow {\sf E}X$ $(n\hm\to\infty)$, так что при
больших~$n$ справедливо приближенное равенство ${\sf E}X\hm\approx\overline x$
и~с учетом~(\ref{e5-kor})
\begin{equation}
{\sf E}U\approx\fr{\overline x}{\alpha}\,.\label{e6-kor}
\end{equation}
Далее, очевидно,

\columnbreak

\noindent
\begin{multline}
{\sf E}X^2={\sf E}Y^2\cdot{\sf E}U+2\alpha{\sf E}X\cdot{\sf E}U^{3/2}+{}\\
{}+
\alpha^2{\sf E}U^2={\sf E}U+\alpha^2{\sf E}U^2\,.
\label{e7-kor}
\end{multline}

\noindent
Поэтому, обозначив
$$
m^2=\fr{1}{n}\sum\limits_{i=1}^nx_i^2\,,
$$
получаем приближенное равенство ${\sf E}X^2\hm\approx m^2$, так что
с~учетом~(\ref{e6-kor}) и~(\ref{e7-kor}) имеем:
\begin{equation}
{\sf E}U^2\approx\fr{1}{\alpha^2}\left(m^2-\fr{\overline
x}{\alpha}\right)\,.\label{e8-kor}
\end{equation}
Если параметр~$\alpha$ известен, то для определения верхней границы~$u^*$
сетки, накидываемой на носитель распределения случайной
величины~$U$, можно задать малое положительное число~$\varepsilon$
и~воспользоваться требованием
\begin{equation}
{\sf P}(U\geqslant u^*)\leqslant\varepsilon\,.\label{e9-kor}
\end{equation}
А~для гарантированного выполнения требования~(\ref{e9-kor}) можно использовать
неравенство Маркова:
$$
{\sf P}(U\geqslant u^*)\leqslant\fr{{\sf E}U^2}{(u^*)^2}\leqslant \varepsilon\,,
$$
откуда с учетом~(\ref{e8-kor})
$$
(u^*)^2\geqslant\fr{{\sf E}U^2}{\varepsilon}\approx
\fr{1}{\alpha^2\varepsilon}\left( m^2-\fr{\overline x}{\alpha}\right)
$$
или
\begin{equation}
u^*\approx\fr{1}{\alpha\sqrt{\varepsilon}}\sqrt{m^2-
\fr{\overline x}{\alpha}}\,.\label{e10-kor}
\end{equation}

\begin{figure*}[b] %fig1
\vspace*{1pt}
 \begin{center}
 \mbox{%
 \epsfxsize=161.718mm
 \epsfbox{kor-1.eps}
 }
 \end{center}
 \vspace*{-9pt}
\Caption{Примеры применения модифицированного двухэтапного сеточного
ЕМ-ал\-го\-рит\-ма для подгонки обобщенного гиперболического распределения
к искусственным данным, $\beta\hm=0$: (\textit{a})~$n\hm=1000$, $\alpha\hm=0{,}3$,
$\nu\hm=1{,}3$, $\mu\hm=1{,}6$, $\lambda\hm=0{,}2$;
(\textit{б})~$n\hm=1000$, $\alpha\hm=0{,}5$, $\nu\hm=1$, $\mu\hm=1$,
$\lambda\hm=3$;
(\textit{в})~$n\hm=1000$, $\alpha\hm=3$,
 $\nu\hm=1{,}3$, $\mu\hm=1{,}6$, $\lambda\hm=2$;
(\textit{г})~$n\hm=10\,000$,
$\alpha\hm=0{,}3$, $\nu\hm=1{,}3$, $\mu\hm=1{,}6$, $\lambda\hm=0{,}2$}
\end{figure*}


Если же параметр~$\alpha$, определяющий асим\-мет\-рию распределения
случайной величины~$X$, неизвестен, то можно воспользоваться
следующими рассуждениями. Обозначим
$$
q_n=\fr{1}{n}\sum\limits_{i=1}^n{\bf 1}(x_i<0)\,,
$$
где ${\bf 1}(A)$~--- индикаторная функция множества (события)~$A$.
При этом по усиленному закону больших чисел с~вероятностью единица
$q_n\hm\longrightarrow {\sf P}(X\hm<0)$ $(n\hm\to\infty)$, так что при
больших~$n$ справедливо приближенное равенство
\begin{equation}
q_n\approx{\sf P}(X<0)\,.\label{e11-kor}
\end{equation}
Но
\begin{multline}
{\sf P}(X<0)=\int\limits_{0}^{\infty}\Phi
\left(-\alpha\sqrt{u}\right) p_{\mathrm{GIG}}(u;\nu,\mu,\lambda)\,du={}\\
{}=
{\sf E}\Phi\left(-\alpha\sqrt{U}\right)\,.\label{e12-kor}
\end{multline}

\pagebreak

\noindent
Предположим сначала, что $q_n\hm<1/2$. Если~$n$ достаточно велико,
то можно с~большой степенью
 уверенности утверж\-дать, что тогда
$\overline x\hm>0$ и~$-\alpha\hm<0$, т.\,е.
 $\alpha\hm>0$ и,~стало быть, на
положительной полуоси значений аргумента~$u$ функция $\Phi(\alpha u)$
вогнута, т.\,е.\ выпукла вверх. Тогда из~(\ref{e11-kor}) и~(\ref{e12-kor}), дважды
применяя неравенство Иенсена, в~силу монотонности функции~$\Phi$
получаем:
\begin{multline}
1-q_n\approx 1-{\sf E}\Phi\left(-\alpha\sqrt{U}\right)=
          {\sf E}\Phi\left(\alpha\sqrt{U}\right)\leqslant{}\\
          {}\leqslant\Phi
          \left(\alpha{\sf E}\sqrt{U}\right)\leqslant
          \Phi\left(\alpha\sqrt{{\sf E}U}\right)\,.\label{e13-kor}
\end{multline}
Если теперь для $t\hm\in(0,1)$ символом~$v_t$ обозначить $t$-кван\-тиль
стандартного нормального закона, то из~(\ref{e13-kor}) и~(\ref{e6-kor}) вытекает
<<приближенное неравенство>>
$$
v_{1-q_n}\hm\leqslant \alpha\sqrt{{\sf E}U}\,,
$$
т.\,е.
$$
\alpha\geqslant\fr{v_{1-q_n}}{\sqrt{{\sf E}U}}\approx
\fr{v_{1-q_n}\sqrt{\alpha}}{\sqrt{\overline x}}\,,
$$
откуда получаем, что при достаточно больших~$n$
\begin{equation}
\alpha\geqslant\fr{v_{1-q_n}^2}{\overline x}\,.\label{e14-kor}
\end{equation}
Если теперь задать малое положительное число~$\varepsilon$, то
для определения верхней границы~$u^*$ сетки, накидываемой на
носитель распределения случайной величины~$U$, можно воспользоваться
требованием~(\ref{e9-kor}), для гарантированного выполнения которого
с~учетом~(\ref{e6-kor}) и~(\ref{e14-kor}) можно использовать неравенство Маркова:
$$
{\sf P}(U\geqslant u^*)\leqslant \fr{{\sf E}U}{u^*}\approx\fr{\overline
x}{\alpha u^*}\leqslant \fr{(\overline x)^2}{v_{1-q_n}^2 u^*}\leqslant
\varepsilon\,,
$$
откуда окончательно вытекает оценка
\begin{equation}
u^*\approx\fr{(\overline x)^2}{v_{1-q_n}^2 \varepsilon}\,.\label{e15-kor}
\end{equation}

\begin{figure*}[b] %fig2
\vspace*{18pt}
 \begin{center}
 \mbox{%
 \epsfxsize=162.433mm
 \epsfbox{kor-3.eps}
 }
 \end{center}
 \vspace*{-9pt}
\Caption{Примеры применения модифицированного двухэтапного
сеточного ЕМ-ал\-го\-рит\-ма для подгонки обобщенного гиперболического
распределения к~искусственным данным, $n=10\,000$, $\beta\hm=0$:
(\textit{а})~$\alpha\hm=0{,}3$,
$\nu\hm=2$, $\mu\hm=2$, $\lambda\hm=2{,}5$;
(\textit{б})~$\alpha\hm=0{,}5$,  $\nu\hm=1$, $\mu\hm=1$, $\lambda\hm=3$;
(\textit{в})~$\alpha\hm=0{,}8$,
$\nu\hm=1{,}3$, $\mu\hm=1{,}6$, $\lambda\hm=2$;
(\textit{г})~$\alpha\hm=1{,}3$, $\nu\hm=2$, $\mu\hm=2$, $\lambda\hm=2{,}5$}
\end{figure*}



В случае $q_n\hm\geqslant1/2$, если $n$ достаточно велико, то можно
с~большой степенью уверенности утверж\-дать, что $\overline x\hm\leqslant 0$
и~$-\alpha\hm\geqslant 0$, т.\,е.\ на положительной\linebreak\vspace*{-12pt}

\pagebreak

%\end{multicols}


%\begin{multicols}{2}

\noindent
 полуоси значений аргумента~$u$
функция $\Phi(-\alpha u)$ вогнута, т.\,е.\ выпукла вверх. Тогда
из~(\ref{e11-kor}) и~(\ref{e12-kor}), дважды применяя неравенство Иенсена, в~силу
монотонности функции~$\Phi$ получаем
$$
q_n\approx {\sf E}\Phi\left(-\alpha\sqrt{U}\right)\leqslant
\Phi\left(-\alpha\sqrt{{\sf E}U}\right)\,,
$$
откуда вытекает <<приближенное неравенство>> $v_{q_n}\hm \leqslant
-\alpha\sqrt{{\sf E}U}$,
т.\,е.
$$
-\alpha\geqslant\fr{v_{q_n}}{\sqrt{{\sf E}U}}\approx
\fr{v_{q_n}\sqrt{|\alpha|}}{\sqrt{|\overline x|}}
$$
и при достаточно больших~$n$
\begin{equation}
|\alpha|\geqslant\fr{v_{q_n}^2}{|\overline x|}\,.\label{e16-kor}
\end{equation}
Для определения верхней границы~$u^*$ сетки, накидываемой на
носитель распределения случайной величины~$U$, снова зададим малое
положительное число~$\varepsilon$ и~потребуем, чтобы было
справедливо условие~(\ref{e9-kor}), для гарантированного выполнения которого
с~учетом~(\ref{e6-kor}) и~(\ref{e16-kor}) используем неравенство Маркова и~тот факт, что
$\mathrm{sign}\, \overline x\hm=\mathrm{sign}\,\alpha$ при достаточно
больших~$n$:
\begin{multline}
{\sf P}(U\geqslant u^*)\leqslant \fr{{\sf E}U}{u^*}\approx
\fr{\overline x}{\alpha u^*}=
\fr{|\overline x|}{|\alpha| u^*} \leqslant{}\\
{}\leqslant
\fr{(\overline x)^2}{v_{q_n}^2 u^*}\leqslant
\varepsilon\,.\label{e17-kor}
\end{multline}
В силу симметричности нормального распределения $v_{t}\hm=-v_{1-t}$ для
любого $t\hm\in(0,1)$, поэтому $v_{q_n}^2\hm=v_{1-q_n}^2$ и~в~случае
$q_n\hm\geqslant1/2$ соотношение~(\ref{e17-kor}) снова приводит к~оценке~(\ref{e15-kor}).

Справедливости ради необходимо отметить, что оценки~(\ref{e10-kor}) и~(\ref{e15-kor})
являются завышенными, но они гарантируют, что
$(1-\varepsilon)$-почти-весь носитель распределения случайной
величины~$U$ будет лежать внутри интервала $[0, u^*]$.

\section{Результаты численных экспериментов}

Приводимые в~данном разделе графики иллюстрируют качество работы
модифицированного сеточного метода разделения дис\-пер\-си\-он\-но-сдви\-го\-вых
смесей нормальных законов на примере его\linebreak применения к~оцениванию
параметров обоб\-щенных гиперболических распределений с~ис\-поль\-зованием
указанного алгоритма выбора сетки\linebreak с~умеренным чис\-лом узлов $K\hm=40$.
Для вы\-чис\-ле\-ний использовались искусственно сгенерированные выборки
объемов $n\hm=1000$ и~$n\hm=10\,000$ с~разными наборами параметров, значения
которых указаны на рисунках. На рис.~1 и~2 изображены гистограммы
(серые столбики) и~графики
истинной плот\-ности (штриховые линии), промежуточной
оценки, полученной сеточным ЕМ-ал\-го\-рит\-мом (пунктирные линии)
и~итоговой оценки (непрерывные линии). На рис.~1 и~2 так\-же указаны
значения полученных оценок параметров. Как видно из приводимых
рисунков, параметры~$\alpha$ оцениваются очень точно. Точность
оценок остальных параметров удовлетворительная и~может быть повышена
за счет использования более частых сеток и~более чувствительных
критериев остановки ЕМ-ал\-го\-рит\-ма на первом этапе. Следует отметить,
что даже в~тех случаях, в~которых наблюдаются заметные расхождения
оценок параметров и~их точных значений, оценки самих плотностей
довольно \mbox{точны}.




{\small\frenchspacing
 {%\baselineskip=10.8pt
 \addcontentsline{toc}{section}{References}
 \begin{thebibliography}{99}
\bibitem{k2011}
\Au{Королев В.\,Ю.} Ве\-ро\-ят\-но\-ст\-но-ста\-ти\-сти\-че\-ские методы
декомпозиции волатильности хаотических процессов.~--- М.: Изд-во
Московского ун-та, 2011.

\bibitem{n2013}
\Au{Назаров А.\,Л.} Приближенные методы разделения смесей
вероятностных распределений: Дисс.\ \ldots\  канд. физ.-мат. наук.~--- М.:
МГУ им.\ М.\,В.~Ломоносова, 2013.

\bibitem{BN1977}
\Au{Barndorff-Nielsen~O.-E.} Exponentially decreasing distributions
for the logarithm of particle size~// Proc. Roy. Soc. Lond.~A,
1977. Vol.~353. P.~401--419.

\bibitem{BN1978}
\Au{Barndorff-Nielsen~O.-E.} Hyperbolic distributions and
distributions of hyperbolae~// Scand. J. Statist., 1978. Vol.~5.
P.~151--157.

\bibitem{BN1982}
\Au{Barndorff-Nielsen~O.-E., Kent~J., S\!{\!\ptb{\!\o}}\,rensen~M.} Normal
variance-mean mixtures and $z$-distributions~// Int. Statist. Rev.,
1982. Vol.~50. No.\,2. P.~145--159.

\bibitem{ks2012}
\Aue{Королев В.\,Ю., Соколов И.\,А.} Скошенные распределения
Стьюдента, дисперсионные гам\-ма-рас\-пре\-де\-ле\-ния и~их обобщения как
асимптотические аппроксимации~// Информатика и~её применения, 2012.
Т.~6. Вып.~1. С.~2--10.

\bibitem{zk2013}
\Au{Закс Л.\,М., Королев В.\,Ю.} Обобщенные дисперсионные
гам\-ма-рас\-пре\-де\-ле\-ния как предельные для случайных сумм~// Информатика
и её применения, 2013. Т.~7. Вып.~1. С.~105--115.

\bibitem{k2013}
\Au{Королев В.\,Ю.} Обобщенные гиперболические
распределения как предельные для случайных сумм~// Тео\-рия
вероятностей и~ее применения, 2013. Т.~58. Вып.~1. С.~117--132.

\bibitem{kckg2013}
\Au{Королев В.\,Ю., Черток А.\,В., Корчагин~А.\,Ю.,
Горшенин~А.\,К.} Ве\-ро\-ят\-но\-ст\-но-ста\-ти\-сти\-че\-ское моделирование
информационных потоков в~сложных финансовых системах на основе
высокочастотных данных~// Информатика и~её применения, 2013. Т.~7.
Вып.~1. С.~12--21.

\bibitem{p2004}
\Au{Protassov R.\,S.} EM-based maximum likelihood parameter
estimation for a~multivariate generalized hyperbolic distribution
with fixed~$\lambda$~// Statistics Computing, 2004. Vol.~14.
P.~67--77.

\bibitem{kn2010}
\Au{Королев В.\,Ю., Назаров А.\,Л.} Разделение смесей
вероятностных распределений при помощи сеточных методов моментов и~максимального правдоподобия~//
Автоматика и~телемеханика, 2010. Вып.~3. С.~98--116.

\bibitem{DSch1983}
\Au{Dennis J.\,E., Schnabel R.\,B.} Numerical methods for
unconstrained optimization and nonlinear equations.~--- Englewood
Cliffs: Prentice-Hall, 1983. 378~p.
 \end{thebibliography}

 }
 }

\end{multicols}

\vspace*{-6pt}

\hfill{\small\textit{Поступила в редакцию 01.10.14}}

\newpage

%\vspace*{12pt}

%\hrule

%\vspace*{2pt}

%\hrule

%\vspace*{12pt}

\def\tit{A MODIFIED GRID METHOD FOR~STATISTICAL SEPARATION
OF~NORMAL VARIANCE-MEAN MIXTURES}

\def\titkol{A modified grid method for statistical separation
of~normal variance-mean mixtures}

\def\aut{V.\,Yu.~Korolev$^{1,2}$ and~A.\,Yu.~Korchagin$^1$}

\def\autkol{V.\,Yu.~Korolev and~A.\,Yu.~Korchagin}

\titel{\tit}{\aut}{\autkol}{\titkol}

\vspace*{-9pt}


\noindent
$^1$Faculty of Computational Mathematics and Cybernetics,
M.\,V.~Lomonosov Moscow State University,\linebreak
$\hphantom{^1}$1-52 Leninskiye Gory, GSP-1, Moscow 119991, Russian Federation


\noindent
$^2$Institute of Informatics Problems, Russian Academy of Sciences,
44-2~Vavilov Str., Moscow 119333, Russian\linebreak
$\hphantom{^1}$Federation

\def\leftfootline{\small{\textbf{\thepage}
\hfill INFORMATIKA I EE PRIMENENIYA~--- INFORMATICS AND
APPLICATIONS\ \ \ 2014\ \ \ volume~8\ \ \ issue\ 4}
}%
 \def\rightfootline{\small{INFORMATIKA I EE PRIMENENIYA~---
INFORMATICS AND APPLICATIONS\ \ \ 2014\ \ \ volume~8\ \ \ issue\ 4
\hfill \textbf{\thepage}}}

\vspace*{3pt}

\Abste{A~modified two-stage grid method for
statistical separation of normal variance-mean mixtures is described
as an alternative to a pure EM (expectation-maximization) algorithm.
At the first stage of this
algorithm, a~discrete approximation is constructed to the mixing
distribution. At the second stage, the obtained discrete
distribution is approximated by an absolutely continuous
distribution from a~predetermined family, say, by a generalized
inverse Gaussian distribution. The convergence of this two-stage
procedure is discussed. The monotonicity of the grid procedure used
at the first stage is proved. The problem of the optimal choice of
the parameters of the method is discussed in detail. First of all,
the problem of the optimal choice of the grid thrown on the support
of the mixing distribution is considered. Statistical estimators are
proposed for the quantiles of the mixing law. The efficiency of the
method is illustrated by examples of its application to the
estimation of the parameters of generalized hyperbolic
distributions.}

\smallskip

\KWE{mixture of probability distributions; normal
variance-mean mixture; generalized hyperbolic distribution;
EM-algorithm; grid method of separation of mixtures}

\DOI{10.14357/19922264140402}

\Ack
\noindent
The research was supported by the Russian Science Foundation (project 14-11-00364).

%\vspace*{3pt}

  \begin{multicols}{2}

\renewcommand{\bibname}{\protect\rmfamily References}
%\renewcommand{\bibname}{\large\protect\rm References}



{\small\frenchspacing
 {%\baselineskip=10.8pt
 \addcontentsline{toc}{section}{References}
 \begin{thebibliography}{99}
 \bibitem{k2011eng}
 \Aue{Korolev, V.\,Yu.} 2011.
\textit{Veroyatnostno-statisticheskie metody dekompozitsii
volatil'nosti khaoticheskikh protsessov}
[Probabilistic and statistical methods for the decomposition of volatility
of chaotic processes].
Moscow: Moscow University Press. 510~p.

\bibitem{n2013eng}
\Aue{Nazarov, A.\,L.} 2013.
{Priblizhennye metody razdeleniya smesey veroyatnostnykh raspredeleniy}
[Approximate methods for the decomposition of volatility of chaotic processes].
Ph.D. Thesis. Moscow: Moscow State University.

\bibitem{BN1977eng}
\Aue{Barndorff-Nielsen, O.\,E.} 1977.
Exponentially decreasing distributions for the logarithm of particle size.
\textit{Proc. Roy. Soc. Lond. A} 353:401--419.

\bibitem{BN1978eng}
\Aue{Barndorff-Nielsen, O.\,E.} 1978.
Hyperbolic distributions and distributions of hyperbolae.
\textit{Scand. J. Statist.} 5:151--157.

\bibitem{BN1982eng}
\Aue{Barndorff-Nielsen, O.\,E., J.~Kent, and M.~S\!{\ptb{\o}}rensen}. 1982.
Normal variance-mean mixtures and $z$-distributions.
\textit{Int. Statist. Rev.} 50(2):145--159.

\bibitem{ks2012eng}
\Aue{Korolev, V.\,Yu., and I.\,A. Sokolov}. 2012.
{Skoshennye raspredeleniya St'yudenta, dispersionnye
gam\-ma-ras\-pre\-de\-le\-niya i~ikh obobshcheniya kak asimptoticheskie
approksimatsii}
[Skewed Student's distributions, variance gamma distributions, and their
generalizations as asymptotic approximations].
\textit{Informatika i ee Primeneniya}~--- \textit{Inform. Appl.} 6(1):2--10.

\bibitem{zk2013eng}
\Aue{Korolev, V.\,Yu., and L.\,M.~Zaks}. 2013.
{Obobshchennye dispersionnye gam\-ma-ras\-pre\-de\-le\-niya kak
predel'nye dlya sluchaynykh summ}
[Generalized variance gamma distributions as limiting for random sums].
\textit{Informatika i ee Primeneniya}~--- \textit{Inform. Appl.} 7(1):105--115.

\bibitem{k2013eng} \Aue{Korolev, V.\,Yu.} 2013.
{Obobshchennye giperbolicheskie raspredeleniya kak predel'nye dlya sluchaynykh summ}
[Generalized hyperbolic distributions as limiting for random sums]
\textit{Theory Probab. Appl.} 58(1):117--132.

\bibitem{kckg2013eng}
\Aue{Korolev, V.\,Yu., A.\,V. Chertok, A.\,Yu.~Korchagin, and A.\,K.~Gorshenin}.
2013. {Ve\-ro\-yat\-no\-st\-no-sta\-ti\-sti\-che\-skoe
mo\-de\-li\-ro\-va\-nie informatsionnykh potokov v~slozhnykh finansovykh sistemakh
na osnove vysokochastotnykh dannykh}
[Probability and statistical modeling of information flows in complex
financial systems from high-frequency data].
\textit{Informatika i~ee Primeneniya}~--- \textit{Inform.  Appl.} 7(1):12--21.

\bibitem{p2004eng-1}
\Aue{Protassov, R.\,S.} 2004.
EM-based maximum likelihood parameter estimation for a multivariate
generalized hyperbolic distribution with fixed~$\lambda$.
\textit{Statistics Computing} 14:67--77.

\bibitem{kn2010eng-1}
\Aue{Korolev, V.\,Yu., and A.\,L.~Nazarov}. 2010.
{Razdelenie smesey veroyatnostnykh raspredeleniy pri pomoshchi
setochnykh metodov momentov i~maksimal'nogo pravdopodobiya}
[Separation of mixtures using grid moment-based methods and maximum likelihood].
\textit{Avtomatika i~Telemekhanika} [Automatics and Telemechanics] 3:98--116.

\bibitem{DSch1983eng}
\Aue{Dennis, J.\,E., and R.\,B.~Schnabel}. 1983.
\textit{Numerical methods for unconstrained optimization and nonlinear equations}.
Englewood Cliffs: Prentice-Hall. 378~p.


\end{thebibliography}

 }
 }

\end{multicols}

\vspace*{-6pt}

\hfill{\small\textit{Received October 01, 2014}}

\vspace*{-18pt}

\Contr

\noindent
\textbf{Korolev Victor Yu.} (b.\ 1954)~---
Doctor of Science in physics and mathematics, professor,
Department of Mathematical Statistics, Faculty of Computational Mathematics
and Cybernetics, M.\,V.~Lomonosov Moscow State University,
1-52 Leninskiye Gory, GSP-1, Moscow 119991, Russian Federation;
leading scientist, Institute of Informatics Problems,
Russian Academy of Sciences, 44-2~Vavilov Str., Moscow 119333, Russian
Federation; victoryukorolev@yandex.ru

\vspace*{3pt}

\noindent
\textbf{Korchagin Alexander Yu.} (b.\ 1989)~---
PhD student, Faculty of Computational Mathematics and Cybernetics,
M.\,V.~Lomonosov Moscow State University,
1-52 Leninskiye Gory, GSP-1, Moscow 119991, Russian Federation;
sasha.korchagin@gmail.com


\label{end\stat}

\renewcommand{\bibname}{\protect\rm Литература} %5
\def\stat{nazarov}

\def\tit{ВЕРОЯТНОСТНАЯ МОДЕЛЬ ВЛИЯНИЯ КНИГИ ЗАКАЗОВ
НА~ПРОЦЕСС ЦЕНЫ}

\def\titkol{Вероятностная модель влияния книги заказов
на~процесс цены}

\def\aut{Е.\,В.~Быковец$^1$, В.\,В.~Лаврентьев$^2$,  Л.\,В.~Назаров$^3$}

\def\autkol{Е.\,В.~Быковец, В.\,В.~Лаврентьев,  Л.\,В.~Назаров}

\titel{\tit}{\aut}{\autkol}{\titkol}

\index{Быковец Е.\,В.}
\index{Лаврентьев В.\,В.}
\index{Назаров Л.\,В.}
\index{Nazarov L.\,V.}
\index{Lavrentyev V.\,V.}
\index{Bykovets E.\,V.}




%{\renewcommand{\thefootnote}{\fnsymbol{footnote}} \footnotetext[1]
%{Работа поддержана РНФ (проект 16-11-10227).}}


\renewcommand{\thefootnote}{\arabic{footnote}}
\footnotetext[1]{Московский государственный университет им.\ М.\,В.~Ломоносова, 
факультет вычислительной математики и~кибернетики, 
\mbox{eugene.bykovets@stud.cs.msu.su}}
\footnotetext[2]{Московский государственный университет им.\ М.\,В. Ломоносова, 
факультет вычислительной математики и~кибернетики, \mbox{lavrent@cs.msu.ru}}
\footnotetext[3]{Московский государственный университет им.\ М.\,В. Ломоносова, 
факультет вычислительной математики и~кибернетики, 
\mbox{nazarov@cs.msu.ru}}

\vspace*{-9pt}


   

\Abst{Рассматривается модель книги заказов, в~которой заказы на покупку 
и~продажу образуют два независимых процесса Кокса. Предложен механизм
        влияния поступающих заказов на цену актива на основе физической модели 
        абсолютно упругого соударения. В~этой модели цена представляет собой 
        материальную точку с~некоторой массой, движущуюся по прямой без трения. 
        Приходящие заказы на покупку и~уходящие заказы на продажу упруго 
        сталкиваются с~ней и~придают дополнительный импульс в~одном 
        направлении, а~приходящие заказы на продажу и~уходящие заказы на покупку~--- 
        в~противоположном. Получена функциональная предельная теорема для процесса 
        цены при высокой интенсивности входящего потока заказов, позволяющая 
        аппроксимировать его некоторым процессом Леви.}

\KW{лимитные заявки; абсолютно упругий удар; модель книги заказов; процесс цены; процесс Кокса; 
функциональная предельная тео\-рема}

\DOI{10.14357/19922264180205}
  
\vspace*{-3pt}


\vskip 10pt plus 9pt minus 6pt

\thispagestyle{headings}

\begin{multicols}{2}

\label{st\stat}

\section{Введение}

Рассмотрим некоторый торгуемый на бирже актив, в~отношении которого 
могут приходить два вида запросов: на покупку и~на продажу. 
Список таких запросов формирует книгу заказов для данного актива. 
Информация, содержащаяся в~книге, позволяет делать прогнозы относительно 
возможного движения цены рассматриваемого актива. 
Особенный интерес эта информация начала представлять с~развитием высокочастотной
 торговли.

В работе рассматривается модель, которая описывает влияние книги 
заказов на цену актива. Базовой моделью для исследования была выбрана модель 
книги заказов, близкая к~описанной в~\cite{first}. Основное отличие состоит
 в~следующем: на бирже торгуемый актив имеет цену, которая размещается в~узлах 
 сетки~$nh$, где $n$~--- некоторое целое число; $h$~--- тик, т.\,е.\ 
 минимальное изменение цены. Однако высокая частота узлов сетки позволяет считать, 
 что рассматриваемый актив может иметь произвольную цену, равно как и~заявки на 
 покупку и~продажу торгуемого актива. С~учетом этого рассматриваем следующую модель 
 влияния заявок на цену актива, используя физическую аналогию.  
 %
 Рассмотрим 
 материальную точку массой~$M$, которая может двигаться по прямой (числовой оси) 
 в~любом на\-прав\-ле\-нии без трения. При этом, связывая модель 
 физическую и~математическую, будем считать, что текущее положение на оси~--- 
 это текущая цена~$X(t)$.  Будем далее для краткости называть ценой и~саму 
 указанную материальную точку массой~$M$, т.\,е.\ 
 будем говорить о~ско\-рости цены, импульсе цены и~т.\,п. 
 Каждый заказ на продажу (поступающий по цене~$A_{i}$
не ниже, чем~$X(t)$) придает цене дополнительный импульс в~направлении от~$A_{i}$ 
к~$X(t)$. Заказы живут экспоненциальное время, после чего уходят из книги 
за счет исполнения или отмены. Уход заказа из книги  придает цене 
дополнительный импульс той же абсолютной величины, что и~при ее 
поступлении, но противоположного направления. С~заказами на продажу все аналогично, 
только приходят они с~ценой, не превосходящей~$X(t)$.

Цель данной работы состоит в~том, чтобы выяснить, какой процесс движения цены 
порождает такая сис\-те\-ма при интенсивном потоке приходящих заказов.
Близкая задача решалась авторами в~работе~\cite{second}, но там рас\-смат\-ри\-вал\-ся 
другой механизм влияния по\-сту\-па\-ющих заказов на цену. Здесь следует упомянуть 
и~работу~\cite{Korolev1}, в~которой также изуча\-ет\-ся связь механизма 
функционирования книги заказов на микроуровне с~процессом цены.

\vspace*{-12pt}

\section{Описание модели}

\vspace*{-4pt}

Будем рассматривать работу книги заказов на временн$\acute{\mbox{о}}$м интервале $t\hm\in[0,T]$. 
В~начальный момент времени $t\hm=0$ заказов в~книге нет. Считаем, что поток 
приходящих заказов является процессом Кокса следующего вида:
\begin{equation*}
\left\{N(\Lambda(t))=N_1(\Lambda(t)), t\geqslant0\right\}\,,
\end{equation*}
где
$N_1$~--- пуассоновский процесс, интенсивность которого равна~1; 
$\Lambda(t)$~--- стартующий из нуля случайный процесс, у~которого траектории 
являются неубывающими и~непрерывными справа функциями, а~также справедливо 
$\mathbb{P}(\Lambda(t)\hm<\infty)$.

Каждый заказ, приходящий в~книгу, находится в~ней некоторое случайное время. 
Более точно, время нахождения конкретного заказа в~книге является случайной 
величиной, распределенной по экспоненциальному закону.

Для каждого приходящего заказа определен набор параметров $(h_i, \gamma_i, \eta_i)$, 
где 
$h_i$~--- абсолютное значение разности между ценой заказа и~текущей ценой; 
$\gamma_i$~--- разность между ско\-ростью приходящего заказа и~текущей ско\-ростью цены; 
$\eta_i$~--- время пребывания заказа в~книге. 

Случайные величины $h_i$, $\gamma_i$ и~$\eta_i$, $i\hm=1, 2, \ldots,$ независимы 
в~совокупности и~не зависят от потока заявок, a~$\eta_i$ распределены 
экспоненциально с~параметром~$\mu$. Параметр~$\gamma_i$ фактически определяет 
тип заказа. Для заказов на покупку~$\gamma_i$ положительны, для заказов на 
продажу~--- отрицательны.

В качестве физической модели влияния заказа на цену возьмем модель абсолютно 
упругого удара.  Считаем, что $i$-й заказ, поступающий в~момент~$t$,~--- 
это материальная точка массой $m_0(h_i)\hm>0$, которая движется по той же прямой, 
что и~цена, и~имеет в~момент времени~$t$ скорость, равную $u_i \hm= v_{i-1}\hm+\gamma_i$, 
где~$v_{i-1}$~--- скорость цены до столкновения с~$i$-м заказом. В~момент~$t$ 
происходит их упругое соударение. На распределения~$h_i$ и~$\gamma_i$ 
наложим следующие ограничения:
\begin{equation}
\mathbb{E}\gamma_i = 0;\enskip
\mathbb{E}\gamma_i^2 = \overline{\gamma}<\infty;\enskip
\mathbb{E}m_0(h_i)^2 = \overline{m} < \infty\,.
\label{e1-naz}
\end{equation}
Смысл первого условия заключается в~том, что разности между скоростями 
приходящих заказов и~текущей скоростью актива для заказов на покупку и~продажу 
в~среднем равны. Остальные ограничения имеют технический характер и~лишь постулируют 
конечность соответствующих моментов.

Функция $m_0$ является убывающей на интервале~$(0, \infty)$, поскольку 
воздействие заказа на цену тем больше, чем ближе его цена к~текущей цене актива.
Это соответствует реальному положению дел на рынке, где заказы на уровнях, 
близких к~текущей цене, выставляются более ответственно, так как могут быть 
тут же удовлетворены. В~то же время заказы на более удаленных уровнях чаще 
ставятся для дезориентации других участников  рынка и~снимаются\linebreak\vspace*{-12pt}

\columnbreak

\noindent
 при приближении к~ним 
цены. Иными словами, они не отражают реальный спрос. 

\vspace*{-7pt}

\section{Процесс цены}

Рассмотрим точку на числовой прямой, которая представляет собой текущую цену 
актива. Положим 
$M$~--- масса точки; 
$v_{i}$~--- текущая скорость цены, полученная после удара $i$-й частицы, 
полагаем $v_ {0}=0$; 
$u_{i}$~--- ско\-рость $i$-го заказа до соударения; 
$m_0(h_i)$~--- масса $i$-го заказа. 
Как было сказано в~предыду\-щем разделе, данная точ\-ка (исследуемая цена) 
в~определенные моменты времени абсолютно упруго соударяется с~другими 
частицами (заказами). В~этом случае есть возможность выразить ско\-рость точки после 
удара $i$-й час\-ти\-цы через массу точ\-ки и~массу час\-ти\-цы, а~также их ско\-рости 
до столк\-но\-ве\-ния (это следует из закона сохранения импульса и~закона сохранения 
энергии, см.~\cite[гл.~4, \S\,28]{third}):
\begin{equation*}
v_i = -v_{i-1} +2\fr{Mv_{i-1}+ m_0(h_i)u_i}{M+ m_0(h_i)}\,.
\end{equation*}
Обозначим $\Delta v_i \hm= v_i - v_{i-1}$, тогда

\noindent
\begin{multline*}
\Delta v_i = -2v_{i-1} +2\fr{Mv_{i-1}+ m_0\left(h_i\right)u_i}{M+ m_0\left(h_i\right)} = {}\\
{}=
\fr{2 m_0\left(h_i\right)}{M+ m_0\left(h_i\right)}\left(u_{i}-v_{i-1}\right).
\end{multline*}
Фактически $\Delta v_i$ показывает изменение ско\-рости цены после соударения 
с~$i$-м заказом.

Пусть в~начальный момент книга заказов пус\-та, а~начальная скорость $v_0 \hm= 0$. 
Далее полагаем, что заказ с~номером~$i$ приходит в~момент времени~$\tau_{i0}$ и~уходит 
в~момент времени~$\tau_{i1}$. В~итоге получаем, что скорость цены является случайным 
процессом $\{V(t), t\hm\geqslant0\}$ с~ку\-соч\-но-по\-сто\-ян\-ны\-ми траекто\-риями:

\noindent
\begin{equation*}
V(t) = \sum\limits_{i=1}^{N_1(\Lambda(t))}\Delta v_i\mathbb{I}_{\{\tau_{i0}\le t\le 
\tau_{i1}\}}(t)\,.
\end{equation*}
Тогда изменение цены за время~$T$ будет иметь вид:
\begin{equation}
X(T) = \int\limits_{0}^{{T}} V(t)\, dt = \sum\limits_{i=1}^{N_1(\Lambda(T))}X_i(T)\,,
\label{e2-naz}
\end{equation}
где $X_i(T)$~--- изменение цены на интервале [0,T] за счет удара $i$-го заказа:

\vspace*{-2pt}

\noindent
\begin{multline*}
X_i(T) = \int\limits_{0}^{{T}} \Delta v_i\mathbb{I}_{\{\tau_{i0}\leqslant 
t\leqslant \tau_{i1}\}}(t)\,dt ={}\\
{}= \fr{2 m_0(h_i)}{M+ m_0(h_i)}
\int\limits_{0}^{{T}}(u_{i}-v_{i-1})\mathbb{I}_{\{\tau_{i0}\leqslant t
\leqslant \tau_{i1}\}}(t)\,dt\,.
\end{multline*}
Поскольку по определению $u_{i}\hm-v_{i-1} \hm= \gamma_i$ , то последнее 
выражение можем переписать в~виде ($a \wedge b\hm = \min(a,b)$):
\begin{multline*}
X_i(T)=\fr{2 m_0(h_i)\gamma_i}{M+ m_0(h_i)}\int\limits_{0}^{{T}}
\mathbb{I}_{\{\tau_{i0}\leqslant t\leqslant \tau_{i1}\}}(t)\,dt = {}\\
{}=
\fr{2 m_0(h_i)\gamma_i}{M+ m_0(h_i)}\left(T\wedge\tau_{i1}-
T\wedge\tau_{i0}\right) ={}\\
{}=\fr{2 m_0(h_i)\gamma_i}{M+ m_0(h_i)}
\left(T\wedge\left(\tau_{i0}+\eta_i\right)-T\wedge\tau_{i0}\right).
\end{multline*}
Строго говоря, случайные величины $\{X_i(T)$, $i\hm=1,2,\dots\}$ не являются 
независимыми, но суммы в~(\ref{e2-naz}) можно представить в~виде сумм 
независимых случайных величин. 

Рассмотрим распределение вектора 
моментов прихода заявок $\tau_0\hm=\{\tau_{10},\ldots ,\tau_{n0}\}$. По свойству 
пуассоновского потока при $N_1(\Lambda(T))\hm=n$ распределение~$\tau_0$ 
есть распределение вариационного ряда выборки из~$n$~независимых случайных 
величин,\linebreak равномерно распределенных на $[0, \Lambda(T)]$.
Поскольку значение конечной суммы при перестановке\linebreak слагаемых не меняется, 
далее будем считать, что в~каждой из сумм~(\ref{e2-naz})~$\tau_{i0}$ 
независимы и~равномерно распределены на $[0, \Lambda(T)]$, а~следовательно, 
случайные величины~$\{X_i(T)$, $i\hm=1,2,\dots\}$ также независимы.

Изучим асимптотические свойства моментов~$X_i(T)$.

\smallskip

\noindent
\textbf{Лемма~1.}\ \textit{ Пусть случайная величина~$\xi$ 
равномерно распределена на $[0,T]$, $\eta_0$ не зависит от~$\xi$ и~имеет 
экспоненциальное распределение с~параметром~$\mu$ и}
\begin{equation*}
s = T\wedge\left(\xi+\eta_0\right)-\xi\,.
\end{equation*}

\vspace*{-8pt}

\noindent
\textit{Тогда}
\begin{enumerate}[(1)]
\item $s\stackrel{d}=\xi\wedge\eta_0$;
\item \textit{моменты случайной величины s обладают следующими асимптотическими 
свойствами}: 
\begin{equation*}
\lim\limits_{\mu\rightarrow\infty}\mu\mathbb{E}s = 1;\
 \lim\limits_{\mu\rightarrow\infty}\mu^2\mathbb{E}s^2 = 2;\
  \lim\limits_{\mu\rightarrow\infty}\mu^2\mathbb{D}s = 1.
\end{equation*}
\end{enumerate}

\noindent
{Д\,о\,к\,а\,з\,а\,т\,е\,л\,ь\,с\,т\,в\,о\,.}\ \ 
Вычислим математическое ожидание случайной величины~$s$ с~учетом независимости~$\xi$ 
и~$\eta_0$. По определению
\begin{equation*}
s = T\wedge\left(\xi+\eta_0\right)-\xi = (T-\xi)\wedge\eta_0\,.
\end{equation*}
Справедливость первого утверждения леммы следует из независимости~$\xi$ и~$\eta_0$ 
и~одинаковой распределенности~$\xi$ и~$T\hm-\xi$. Таким образом, математическое 
ожидание~$s$ есть
\begin{equation*}
\mathbb{E}s  = \mathbb{E}\left(\xi\wedge\eta_0\right).
\end{equation*}
При вычислении моментов неоднократно будет требоваться значение интеграла
\begin{equation*}
\int\limits_{0}^{{T}}y^ne^{-\mu y}\,dy = \fr{n!}{\mu^{n+1}}\,F_{n+1}(T)\,,
\end{equation*}
где $F_{n+1}$~--- 
функция распределения Эрланга $(n+1)$-го порядка:
$$
F_{n+1}(x) = 1 - e^{-\mu x}\sum\limits_{i=1}^{n}\fr{\mu^iT^i}{i!}\,.
$$ 
Вычислим 
$\mathbb{E}(\xi\wedge\eta_0)$:
\begin{multline*}
\mathbb{E}\left(\xi\wedge\eta_0\right) =
\fr{\mu}{T}\int\limits_{0}^{\infty}\!e^{-\mu y}\,dy\int\limits_{0}^{T}(x\wedge y)\,dx 
 ={}\\
 {}=\fr{\mu}{T}\int\limits_{0}^{\infty}\!e^{-\mu y}\,dy\left\{
 \mathbb{I}_{\{y<T\}}\left[\int\limits_{0}^{y}x\,dx+
 \int\limits_{y}^{T}y\,dx\right] + {}\right.\\
\left. {}+
 \mathbb{I}_{\{y\geqslant T\}}\int\limits_{0}^{T}x\,dx\right\} = \fr{\mu}{T}\left[T\int\limits_{0}^{T}\!ye^{-\mu y}\,dy -{}\right.\\
\left.{}-
\fr{1}{2}\int\limits_{0}^{T}\!y^2e^{-\mu y}\,dy + 
\fr{T^2}{2}\int\limits_{T}^{\infty}\!e^{-\mu y}\,dy \right]={} \\
{}=\left[\fr{1}{\mu}-\fr{1}{\mu^2 T}\right]+\left[\fr{T}{2}+\fr{1}{\mu^2 T}\right]
e^{-\mu T}.
\end{multline*}
И,~соответственно,
\begin{multline*}
\lim\limits_{\mu \rightarrow \infty}\mu\mathbb{E}s ={}\\
{}= 
\lim\limits_{\mu \rightarrow \infty}\left\{\left[1-\fr{1}{\mu T}\right]+
\left[\fr{T}{2}+\fr{1}{\mu^2 T}\right]\mu e^{-\mu T}\right\}=1. 
\end{multline*}
Вычислим $\mathbb{E}s^2\hm=\mathbb{E}(\xi\wedge\eta_0)^2$:
\begin{multline*}
\mathbb{E}\left(\xi\wedge\eta_0\right)^2=
\fr{\mu}{T}\int\limits_{0}^{\infty}\!e^{-\mu y}\,dy
\int\limits_{0}^{T}\left(x\wedge y\right)^2\,dx = {}\\
{}=\fr{\mu}{T}\int\limits_{0}^{\infty}\!e^{-\mu y}\,dy\left\{
\mathbb{I}_{\{y<T\}}\left[{\int\limits_{0}^{y}\!x^2\,dx+
\int\limits_{y}^{T}\!y^2\,dx}\right] + {}\right.\\
\left.{}+
\mathbb{I}_{\{y\geqslant T\}}\int\limits_{0}^{T}\!x^2\,dx\right\}= 
\fr{\mu}{T}\left[T\int\limits_{0}^{T}\!y^2e^{-\mu y}\,dy -{}\right.\\
\left.{}-
\fr{2}{3}\int\limits_{0}^{T}\!y^3e^{-\mu y}dy + 
\fr{T^3}{3}\int\limits_{T}^{\infty}\!e^{-\mu y}dy \right]= {}\\
{}=\left[\fr{2}{\mu^2}-\fr{4}{\mu^3 T}\right]+\left[
\fr{2}{\mu^2}+\fr{4}{\mu^3 T}\right]e^{-\mu T}.
\end{multline*}
Отсюда получаем асимптотику второго момента
\begin{multline*}
\lim\limits_{\mu \rightarrow \infty}\mu^2\mathbb{E}s^2 = {}\\
{}=
\lim\limits_{\mu \rightarrow \infty}\left\{\left[2-\fr{4}{T\mu}\right]+
\left[\fr{2}{\mu^2}+\fr{4}{\mu^3 T}\right]\mu^2e^{-\mu T}\right\} = 2
\end{multline*}
и дисперсии
\begin{multline*}
\lim\limits_{\mu \rightarrow \infty}\mu^2\mathbb{D}s =  
\lim\limits_{\mu \rightarrow \infty}\mu^2\left[\mathbb{E}s^2-(\mathbb{E}s)^2\right] ={}\\
{}=
 \lim\limits_{\mu \rightarrow \infty}\mu^2\mathbb{E}s^2 - 
  \lim\limits_{\mu \rightarrow \infty}(\mu\mathbb{E}s)^2 =1\,.
\end{multline*}
Утверждение леммы доказано.

\smallskip

Рассмотрим следующую последовательность:
\begin{equation}
\left\{X_{n}(t) = \sum\limits_{i=1}^{N_1(\Lambda_n(t))}X_{ni}(t),\ 
 t \in [0,T]\right\}.
 \label{e3-naz}
\end{equation}
При этом каждому члену~$\{X_n\}$ соответствует процесс $\{\Lambda_n(t)$, 
$t\hm\in [0,T]\}$, параметр~$\mu_n$ и~функция массы ударяющей частицы (приходящего 
заказа)~$m_{n0}$. При увеличении~$n$ будем увеличивать интенсивность входящего 
потока заявок~$\Lambda_n(t)$ и~уменьшать время пребывания заказов в~книге 
посредством увеличения~$\mu_n$ ($\Lambda_n(t)\hm\Rightarrow\infty$, 
$\mu_n\hm\rightarrow\infty$ при $n \hm\rightarrow \infty$). Будем также 
уменьшать влияние отдельного заказа на цену:
\begin{equation*}
m_{n0} = \alpha_n m_0,\enskip
 \alpha_n >0\,,\enskip
  \alpha_n \rightarrow 0\,,\enskip
   n\rightarrow\infty\,.
\end{equation*}
Получим асимптотические свойства моментов случайных величин $X_{n1}(T)$ 
при установленных параметрических зависимостях. Аргумент~$T$ у~них одинаков и~для 
краткости будем его опускать.

\smallskip

\noindent
\textbf{Лемма~2.}\  \textit{Пусть $\mu_n\hm \rightarrow \infty$, 
$\alpha_n \hm\rightarrow 0$, $k_n \hm= {\mu_n^2}/{\alpha_n^2}$.  Тогда}
\begin{enumerate}[(1)]
\item $k_n\mathbb{E}X_{n1}\hm\rightarrow 0$, 
$k_n\mathbb{D}X_{n1}\hm\rightarrow {8\overline{m}\overline{\gamma}}/{M^2}$,
$n\hm\rightarrow\infty$;
\item \textit{Выполняется условие Линдеберга, т.\,е.\ для любого} $\varepsilon \hm>0$
\begin{equation*}
\lim\limits_{n\rightarrow\infty}k_n
\mathbb{E}\left[X_{n1}^2\mathbb{I}(|X_{n1}|>\varepsilon)\right]=0\,.
\end{equation*}
\end{enumerate}

\noindent
Д\,о\,к\,а\,з\,а\,т\,е\,л\,ь\,с\,т\,в\,о\,.\ \
Как было показано выше, 
$$
X_{n1} = \fr{2 m_{n0}(h_1)\gamma_{1}}{M+ m_{n0}(h_1)}s_n\,,
$$
 где $s_n\stackrel{d} =\xi\wedge\eta_{n0}$ и~$\eta_{n0}$ распределена 
 экспоненциально с~параметром~$\mu_n$, а~величины $h_1$, $\xi$, $\eta_{n0}$
и~$\gamma_{1}$ независимы. Так как в~соответствии с~(\ref{e1-naz}) 
 $\mathbb{E}\gamma_1 \hm= 0$, то $k_n\mathbb{E}X_{n1}\hm=0$ для любого~$n$. 
 Проверим соотношение для дисперсии:
 
 \noindent
\begin{multline*}
k_n\mathbb{D}X_{n1} = \fr{\mu_n^2}{\alpha_n^2}\, \mathbb{E} 
\left[\fr{2 m_{n0}(h_1)\gamma_{1}}{M+ m_{n0}(h_1)}\,s_n\right]^2={}\\
{}=4\mathbb{E}\gamma_{1}^2\mathbb{E}\left[
\fr{m_{n0}(h_1)}{\alpha_n}\,\fr{1}{M+m_{n0}(h_1)}\right]^2\mu_n^2\mathbb{E}s_n^2 = {}\\
{}=
4\overline{\gamma}\mathbb{E}\left[\fr{m_0(h_1)}{M+m_{n0}(h_1)}\right]^2\mu_n^2
\mathbb{E}s_n^2.
\end{multline*}

Последовательность случайных величин 
$\{[{m_0(h_1)}/({M+m_{n0}(h_1)})]^2\}$ мажорируется интегрируемой случайной 
величиной $[{m_0(h_1)}/{M}]^2$ и~поточечно сходится к~ней, так как $\alpha_n 
\hm\rightarrow 0$, $n\hm\rightarrow\infty$. Поэтому

\vspace*{-6pt}

\noindent
\begin{multline*}
\lim\limits_{n\rightarrow\infty}k_n\mathbb{D}X_{n1} =  
\lim\limits_{n\rightarrow\infty}4\overline{\gamma}\mathbb{E}
\left[\fr{m_0(h_1)}{M}\right]^2\mu_n^2\mathbb{E}s_n^2 ={}\\
{}=
\fr{4\overline{m}\overline{\gamma}}{M^2}
\lim\limits_{n\rightarrow\infty}\mu_n^2\mathbb{E}s_n^2 = 
\fr{8\overline{m}\overline{\gamma}}{M^2}\,.
%\label{e4-naz}
\end{multline*}

Докажем справедливость условия Линдеберга. Рассмотрим

\vspace*{-6pt}

\noindent
\begin{multline*}
\fr{\mu_n}{\alpha_n}\left\vert X_{n1}\right\vert = 
 \fr{\mu_n}{\alpha_n}\left\vert \fr{2 m_{n0}(h_1)\gamma_{1}}{M+ m_{n0}
 \left(h_1\right)}\,s_n\right\vert = {}\\
 {}=
 2\fr{m_{n0}(h_1)}{\alpha_n}\,\fr{1}{M+ m_{n0}(h_1)}\left\vert \gamma_1\right\vert
 \mu_ns_n \leqslant {}\\
{}\leqslant 2\fr{m_0(h_1)}{M}\left\vert \gamma_1\right\vert \mu_ns_n
\leqslant {}\\
{}\leqslant \fr{2m_0(h_1)|\gamma_1|\mu_n\eta_{n0}}{M} \stackrel{d}= 
\fr{2m_0(h_1)|\gamma_1|\hat{\eta}}{M}\,,
\end{multline*}
где $\hat{\eta}$ распределена экспоненциально с~па\-ра\-мет\-ром~1. 
Распределение случайной величины 
$$
Y_n =   \fr{{2m_0(h_1)|\gamma_1|\mu_n\eta_{n0}}}{M}
$$ 
не зависит от~$n$ и~согласно~(1) имеет конечный второй момент, поэтому

\noindent
\begin{multline*}
k_n\mathbb{E}\left[X_{n1}^2\mathbb{I}(|X_{n1}|>\varepsilon)\right] ={}\\
{}= 
k_n\mathbb{E}\left[X_{n1}^2\mathbb{I}(\sqrt{k_n}|X_{n1}|>
\sqrt{k_n}\varepsilon)\right] \leqslant {} \\
{}\leqslant \mathbb{E}\left[Y_{n}^2\mathbb{I}(|Y_{n}|>
\sqrt{k_n}\varepsilon)\right]
= \mathbb{E}\left[Y_{1}^2\mathbb{I}(|Y_{1}|>
\sqrt{k_n}\varepsilon)\right].\hspace*{-3.79228pt}
\end{multline*}
Последнее математическое ожидание стремится к~нулю при $n\hm\rightarrow\infty$ по 
тео\-ре\-ме Лебега о~ма\-жо\-ри\-ру\-емой сходимости. Утверждение леммы доказано.

\vspace*{2pt}

Сформулируем доказанную в~работе~\cite{fourth} функ\-циональную центральную 
предельную тео\-ре\-му, устанавливающую условия, при которых процессы вида~(\ref{e3-naz}) 
сходятся к~некоторому предельному процессу~$X$ в~про\-стран\-ст\-ве Скорохода 
$\mathcal{D}\hm = \mathit{(D[0,1], d_0)}$ (см.~\cite[гл.~3]{five}). 
Позднее были получены более сильные результаты, касающиеся схо\-ди\-мости обобщенных 
процессов Кокса~(\ref{e3-naz}) (см., на\-при\-мер,~\cite{Korolev_FLT}), 
но достаточно будет приводимого ниже утверж\-де\-ния.
{ %\looseness=1

}

\smallskip

\noindent
\textbf{Теорема}~\cite{fourth}.\ 
\textit{Пусть для некоторой неограниченно возрастающей последовательности 
чисел~$\{k_n\}_{n\geqslant1}$ выполнены условия}:
\begin{enumerate}[(1)]
\item \textit{существуют числа $a \hm\in \mathbb{R}$ и~$\sigma \hm> 0$ такие, что}
\begin{equation*}
k_n\mathbb{E}X_{n1} \rightarrow a;\enskip
 k_n\mathbb{D}X_{n1} \rightarrow \sigma^2 (n\rightarrow \infty);
\end{equation*}
\item \textit{условие Линдеберга, т.\,е.\ для любого} $\varepsilon\hm>0$
\begin{equation*}
\lim\limits_{n\rightarrow\infty}k_n\mathbb{E}\left[(X_{n1}-a_n)^2\mathbb{I}
\left(\left\vert X_{n1}-a_n\right\vert >\varepsilon\right)\right]=0\,,
\end{equation*}
\textit{где $\mathbb{I}(A)$~--- индикатор события}~$A$; $a_n \hm= \mathbb{E}X_{n1}$;
\item \textit{существует безгранично делимая случайная величина~$U$ такая, 
что $\mathbb{P}(U=0) \hm< 1$, $\mathbb{P}(U\geqslant0) \hm= 1$, 
$\mathbb{E}U^2 \hm< \infty$ и}
\begin{equation*}
k_n^{-1}\Lambda_n(1)\Rightarrow U,n\rightarrow\infty;
\end{equation*}
\item
$\displaystyle \sup\limits_nk_n^{-2}\mathbb{E}\Lambda_n(1)^2<\infty.
$
\end{enumerate}
\textit{Тогда обобщенные процессы Кокса $\{X_n\}$ 
слабо сходятся в~пространстве Скорохода~$\mathcal{D}$ к~процессу Леви~$X$ такому, что}
\begin{equation*}
X(1) \stackrel{d}=\sigma\sqrt{U}N(0,1)+aU\,,
\end{equation*}
\textit{где $N(0,1)$~--- случайная величина, имеющая стандартное нормальное 
распределение, при этом не зависящая от}~$U$.

Для последовательности $\{k_n = {\mu_n^2}/{\alpha_n^2}\}$ при $a\hm=0$ первые 
два условия теоремы выполняются по лемме~2. Таким образом, достаточно наложить 
определенные условия на ин\-тен\-сив\-ность входящего потока заявок, чтобы была 
справедлива сле\-ду\-ющая теорема.

\smallskip

\noindent
\textbf{Теорема 1.}  \textit{Пусть $\mu_n \hm\rightarrow \infty$, 
$\alpha_n \hm\rightarrow 0$, $k_n = {\mu_n^2}/{\alpha_n^2}$, 
$\sup_nk_n^{-2}\mathbb{E}\Lambda_n(1)^2\hm<\infty$ и~существует 
безгранично делимая случайная величина~$U$ такая, что}
\begin{equation*}
\mathbb{P}(U=0) < 1\,;\enskip
\mathbb{P}(U\geqslant0) = 1\,;\enskip
 \mathbb{E}U^2 < \infty
\end{equation*}
и

\noindent
\begin{equation*}
k_n^{-1}\Lambda_n(1)\Rightarrow U\,,\enskip n\rightarrow\infty\,.
\end{equation*}
\textit{Тогда обобщенные процессы Кокса~$\{X_n\}$ слабо сходятся в~пространстве 
Скорохода $\mathcal{D}$ к~процессу Леви~$X$ такому, что}
\begin{equation*}
X(1) \stackrel{d}=\sigma\sqrt{U}N(0,1)\,,
\end{equation*}

%\columnbreak

\noindent
\textit{где $\sigma = {8\overline{m}\overline{\gamma}}/{M^2}$, а $N(0,1)$~--- 
случайная величина со стандартным нормальным распределением, независимая от}~$U$. 

%\vspace*{-24pt}

\section{Заключение}

В настоящей работе была предложена модель механизма влияния по\-сту\-па\-ющих 
заказов на цену актива на основе физической модели абсолютно упругого 
соударения час\-тиц. 

Была установлена справедливость функциональной предельной 
тео\-ре\-мы, на основании результатов которой можно аппроксимировать процесс 
цены при интенсивном потоке приходящих заявок процессом Леви, приращения 
которого являются смесью нормальных законов и~поддаются более точному анализу. 
Такая аппроксимация дает также воз\-мож\-ность оценки риска динамических
 стратегий~\cite{six}.


%\vspace*{-48pt}

    {\small\frenchspacing
 {%\baselineskip=10.8pt
 \addcontentsline{toc}{section}{References}
 \begin{thebibliography}{9}
\bibitem{first} 
\Au{Kukanov A.} Stochastic models of limit order markets.~--- 
Columbia University, 2013. Ph.D. Thesis. 131~p.

\bibitem{second} 
\Au{Лаврентьев В.\,В., Назаров~Л.\,В.} 
Процесс движения цены, порожденный непрерывной моделью книги заказов~// 
Вестн. Тверского государственного ун-та. Сер. Прикладная математика, 2015. 
№~4. С.~55--63.

\bibitem{Korolev1}
\Au{Korolev V.\,Yu., Chertok~A.\,V., Korchagin~A.\,Yu, Zeifman~A.\,I.} 
Modeling high-frequency order flow imbalance by functional limit theorems for 
two-sided risk processes~// Appl. Math. Comput., 2015. Vol.~253. P.~224--241.

\bibitem{third} 
\Au{Сивухин Д.\,В.} Общий курс физики.~--- 
В~5 т.~--- Т.~1. Механика.~--- 4-е изд.~--- М.: МФТИ, 2005. 560~с.

\bibitem{fourth} 
\Au{Кащеев Д.\,Е.} Моделирование динамики финансовых временных рядов и~оценивание 
производных ценных бумаг: Дис.\ \ldots\ канд. физ.-мат. наук.~--- 
Тверь: ТвГУ, 2001. 191~c.

\bibitem{five} 
\Au{Биллингсли П.} Сходимость вероятностных мер~/
Пер. с~англ.~--- М.: Наука, 1977. 353~с.
(\Au{Billingsley~P.}  
{Convergence of probability measures}.~--- New York, NY, USA: John Wiley \& Sons, Inc., 
1977. 277~p.)

\bibitem{Korolev_FLT} 
\Au{Korolev V.\,Yu., Chertok~A.\,V., Korchagin~A.\,Yu, Kossova~E.\,V., Zeifman~A.\,I.} 
A~note on functional limit theorems for compound Cox processes~// 
J.~Math. Sci., 2016. Vol.~218. No.\,2. P.~182--194.

\bibitem{six} 
\Au{Balasanov~Y., Doynikov~A., Lavrent'ev~V., Nazarov~L.} 
Estimating risk of dynamic trading strategies from high frequency data flow~// 
Advances in data mining: Applications and theoretical aspects~/
 Ed.\ P.~Perner.~---
Lecture notes in computer science ser.~--- Springer, 2015.  
 Vol.~9165. P.~153--165.
 \end{thebibliography}

 }
 }

\end{multicols}

\vspace*{-3pt}

\hfill{\small\textit{Поступила в~редакцию 07.12.17}}

%\vspace*{6pt}

\newpage

\vspace*{-28pt}

%\hrule

%\vspace*{2pt}

%\hrule

%\vspace*{8pt}


\def\tit{A~PROBABILITY MODEL OF~THE~INFLUENCE\\ OF~THE~ORDER BOOK ON~THE~PRICE PROCESS}

\def\titkol{A probability model of the influence of the order book on the price process}

\def\aut{L.\,V.~Nazarov, V.\,V.~Lavrentyev, and~E.\,V.~Bykovets}

\def\autkol{L.\,V.~Nazarov, V.\,V.~Lavrentyev, and~E.\,V.~Bykovets}

\titel{\tit}{\aut}{\autkol}{\titkol}

\vspace*{-9pt}


\noindent
Faculty of Computational Mathematics and Cybernetics, 
M.\,V.~Lomonosov Moscow State University, 1-52~Leninskiye Gory, GSP-1, Moscow 119991, 
Russian Federation 


\def\leftfootline{\small{\textbf{\thepage}
\hfill INFORMATIKA I EE PRIMENENIYA~--- INFORMATICS AND
APPLICATIONS\ \ \ 2018\ \ \ volume~12\ \ \ issue\ 2}
}%
 \def\rightfootline{\small{INFORMATIKA I EE PRIMENENIYA~---
INFORMATICS AND APPLICATIONS\ \ \ 2018\ \ \ volume~12\ \ \ issue\ 2
\hfill \textbf{\thepage}}}

\vspace*{3pt} 
 


\Abste{The Limit Order Book model is considered, with buy and sell orders arriving 
as two independent Cox processes. It includes the price impact model built on the 
basis of a physical model of perfectly elastic collision. Price is treated as 
a~particle of some mass, moving along a~straight line without friction. The 
incoming buy orders and outgoing sell orders hit the price giving it additional 
momentum in one direction, while incoming sell orders and outgoing buy orders do 
the same in the opposite direction. A~functional limit theorem for the price 
process is obtained at a~high intensity 
of incoming order flow, which allows approximation by some L$\acute{\mbox{e}}$vy process}

\KWE{limit orders; perfectly elastic collision; limit order book model; 
price process; Cox process; functional limit theorem}

 
\DOI{10.14357/19922264180205} %

%\vspace*{-14pt}

  %\Ack
  % \noindent
  


%\vspace*{-3pt}

  \begin{multicols}{2}

\renewcommand{\bibname}{\protect\rmfamily References}
%\renewcommand{\bibname}{\large\protect\rm References}

{\small\frenchspacing
 {%\baselineskip=10.8pt
 \addcontentsline{toc}{section}{References}
 \begin{thebibliography}{9}

\bibitem{1-naz}
\Aue{Kukanov, A.} 2013. Stochastic models of limit order markets. 
Columbia University. Ph.D. Thesis.  131~p.

\bibitem{2-naz}
\Aue{Lavrent'ev, V.\,V., and L.\,V.~Nazarov.} 2015. Protsess dvizheniya tseny, 
porozhdennyy nepreryvnoy model'yu knigi zakazov 
[Price process, generated by the continuous model of the order book]. 
\textit{Vestnik Tverskogo gosudarstvennogo un-ta. Ser. 
Prikladnaya matematika} [Bull. of the Tverskoy State University. Ser. 
Appl. Math.] 4:55--63.

\bibitem{3-naz}
\Aue{Korolev, V.\,Yu., A.\,V.~Chertok, A.\,Yu.~Korchagin, and A.\,I.~Zeifman.} 
2015. Modeling high-frequency order flow imbalance by functional limit theorems
 for two-sided risk processes. \textit{Appl. Math. Comput.} 253:224--241.

\bibitem{4-naz}
\Aue{Sivukhin, D.\,V.} 2005. 
\textit{Obshchiy kurs fiziki. Mekhanika}
[General course of physics. Mechanics].
4~ed. Moscow: MIPT Publs. Vol.~1.  560~p. 

\bibitem{5-naz}
\Aue{Kashcheev, D.\,E.} 2001. Modelirovanie dinamiki finansovykh vremennykh ryadov
 i~otsenivanie proizvodnykh tsennykh bumag [Modeling of dynamics of financial time series and 
 estimation of derivative securities].  
 Tver'. PhD Thesis. 191~p.

\bibitem{6-naz}
\Aue{Billingsley, P.} 1977. 
\textit{Convergence of probability measures}. New York, NY: John Wiley \& Sons, Inc. 
277~p.

\bibitem{7-naz}
\Aue{Korolev, V.\,Yu., A.\,V.~Chertok, A.\,Yu.~Korchagin, E.\,V.~Kossova, 
and A.\,I.~Zeifman.} 2016. 
A~note on functional limit theorems for compound Cox processes. 
\textit{J.~Math. Sci.} 218(2):182--194. 

\bibitem{8-naz}
\Aue{Balasanov, Y., A.~Doynikov, V.~Lavrent'ev, and L.~Nazarov}. 
2015. Estimating risk of dynamic trading strategies from high frequency data flow.
\textit{Advances in data mining: Applications and theoretical aspects.} 
Ed.\ P.~Perner.  Lecture notes in computer science ser.  
Springer. 9165:153--165.
\end{thebibliography}

 }
 }

\end{multicols}

\vspace*{-3pt}

\hfill{\small\textit{Received December 7, 2017}}

%\vspace*{-24pt}




\Contr

\noindent
\textbf{Bykovets Eugene V.} (b.\ 1994)~--- MSc student,  
Faculty of Computational Mathematics and Cybernetics, M.\,V.~Lomonosov Moscow 
State University, 1-52~Leninskiye Gory, GSP-1, Moscow 119991, Russian Federation; 
\mbox{eugene.bykovets@stud.cs.msu.su}

\vspace*{3pt}

\noindent
\textbf{Lavrentyev Victor V.} (b.\ 1955)~---  
Candidate of Science (PhD) in physics and mathematics, scientist, 
Faculty of Computational Mathematics and Cybernetics, M.\,V.~Lomonosov Moscow 
State University, 1-52~Leninskiye Gory, GSP-1, Moscow 119991, Russian Federation; 
\mbox{lavrent@cs.msu.ru}

\vspace*{3pt}

\noindent
\textbf{Nazarov Leonid V.} (b.\ 1957)~--- 
Candidate of Science (PhD) in physics and mathematics, senior scientist, 
Faculty of Computational Mathematics and Cybernetics, M.\,V.~Lomonosov Moscow 
State University, 1-52~Leninskiye Gory, GSP-1, Moscow 119991, Russian Federation; 
\mbox{nazarov@cs.msu.ru}
\label{end\stat}


\renewcommand{\bibname}{\protect\rm Литература}    %6
\def\stat{krivenko}

\def\tit{МНОГОМЕРНЫЙ РЕФЕРЕНСНЫЙ РЕГИОН\\ ВЫСОКОЙ ПЛОТНОСТИ}

\def\titkol{Многомерный референсный регион высокой плотности}

\def\aut{М.\,П.~Кривенко$^1$}

\def\autkol{М.\,П.~Кривенко}

\titel{\tit}{\aut}{\autkol}{\titkol}

\index{Кривенко М.\,П.}
\index{Krivenko M.\,P.}


%{\renewcommand{\thefootnote}{\fnsymbol{footnote}} \footnotetext[1]
%{Работа выполнена при финансовой поддержке РФФИ (проекты 16-07-00677 
%и~15-37-20611-мол\_а\_вед).}}


\renewcommand{\thefootnote}{\arabic{footnote}}
\footnotetext[1]{Институт проблем информатики Федерального исследовательского центра <<Информатика и~управление>> Российской академии наук,
\mbox{mkrivenko@ipiran.ru}}

\vspace*{4pt}



\Abst{Рассматриваются принципы построения многомерных референсных регионов
(MRR~--- multivariate reference region). 
Предложен оригинальный метод построения региона на основе областей с~высокой 
плотностью точек и~аппроксимации распределения данных с~помощью смеси нормальных 
распределений. Для оценки порога для плотности распределения используется  
бут\-стреп-ме\-тод. В~качестве эксперимента рассмотрена задача построения 
и~использования эталонной области для прогнозирования типа мочевого камня. Обработка 
реальных данных продемонстрировала преимущества предлагаемых решений.}

\KW{многомерный референсный регион; область высокой плотности; бут\-стреп-ме\-тод; 
смесь многомерных нормальных распределений}

\vspace*{6pt}

\DOI{10.14357/19922264170207} 


\vskip 10pt plus 9pt minus 6pt

\thispagestyle{headings}

\begin{multicols}{2}

\label{st\stat}

\section{Введение}

     Многомерный референсный регион 
был предложен в~литературе по клинической химии в~начале 1970-х~гг.\ как 
альтернатива одномерным референсным интервалам~[1]. Там излагались 
преимущества предлагаемых множественных тестов, хоть и~имеющих 
упрощенный вид, но снижающих (по отношению к~одномерным вариантам) 
число ложных положительных результатов. Появление MRR оказалось 
особенно привлекательным для интерпретации результатов наборов 
медицинских тестов. Тем не менее возникали трудности в~построении 
и~использовании процедур многомерного анализа (см., например,~[2]), 
связанные, в~частности, с~быстрым увеличением числа параметров, которые 
должны быть оценены. Немногие лаборатории использовали MRR в~своей 
практике, причем в~экспериментальном режиме, и,~как следствие, на 
сегодняшний день имеется относительно малое количество соответствующих 
публикаций. 

\vspace*{-6pt}

\section{Многомерный референсный регион на основе расстояния Махалонобиса}

\vspace*{-2pt}

     Одномерный референсный интервал, полученный статистическим путем, 
использует центральную часть значений анализируемого показателя, обычно 
соответствующую~95\% некоторой популяции~--- совокупности особей 
определенного вида (например, здоровой части населения определенного пола 
из некоторого диапазона возрастов). Одномерные референсные интервалы 
применялись в~течение многих лет в~качестве стандартного приема 
интерпретации лабораторных данных. Они легко формируются, хранятся, 
извлекаются и~передаются в~лабораторных информационных системах, просты 
в~понимании, хорошо воспринимаются медицинским сообществом в~ходе 
длительного использования. Тем не менее одномерные референсные интервалы 
при классификации данных могут дать большое число ложно аномальных 
результатов. Этот далеко не единственный недостаток однофакторного 
референсного интервала может быть полностью или частично устранен 
с~помощью MRR.
     
     Простейшим и~весьма распространенным способом построения MRR 
является использование прямого произведения отдельных референсных 
интервалов в~предположении, что они статистически независимы. Пусть 
$(1\hm-\alpha)$~--- вероятность попадания в~MRR, а~$p_0$~--- вероятность 
попадания в~референсный интервал для любого из~$d$~признаков, тогда 
$p_0\hm= \sqrt[d]{1-\alpha}$. С~ростом размерности~$d$ значения~$p_0$ 
быстро приближаются к~1, что фактически лишает смысла применение MRR.
     
     Как и~в одномерном случае, отправной точкой для построения MRR 
может стать нормальное распределение. Идеи центрального расположения 
референсного региона и~заданной вероятности попадания в~него приводят для 
$d$-мер\-но\-го нормального распределения, имеющего плотность 
распределения
     \begin{multline*}
     \varphi(y,\mu,\Sigma) ={}\\
     {}=(2\pi)^{-d/2}\vert\Sigma\vert^{-1/2}\exp \left( -\fr{\left(y-
\mu\right)^{\mathrm{T}} \Sigma^{-1}(y-\mu)}{2}\right),
   \end{multline*}
где величина $(y-\mu)^{\mathrm{T}} \Sigma^{-1} (y-\mu)$ есть квадрат так 
называемого расстояния Махаланобиса между~$y$ и~$\mu$, к~использованию 
многомерного эллипсоида
\begin{multline*}
(2\pi)^{-d/2}\vert\Sigma\vert^{-1/2}\exp \left( -\fr{\left(y-\mu\right)^{\mathrm{T}}
\Sigma^{-1} 
(y-\mu)}{2}\right) ={}\\
{}=const
\end{multline*}
или, что то же самое, 
$$ 
(y-\mu)^{\mathrm{T}} \Sigma^{-1}(y-\mu)=const\,.
$$
Его называют эллипсоидом равной плотности распределения (или просто 
эллипсоидом равной вероятности). 
     
     Если задаться вероятностью $(1\hm-\alpha)$ попадания в~эллипсоид 
равной вероятности вида $(y\hm-\mu)^{\mathrm{T}}\Sigma^{-1} (y\hm-\mu)\hm= 
\rho$, то параметр~$\rho$ удовлетворяет уравнению $\mathrm{Pr}\left\{ 
\chi_d^2\leq \rho\right\} \hm=1\hm-\alpha$.
     
     Использование эллипсоида в~качестве MRR будет оправдано только 
тогда, когда исходное распределение данных есть многомерное нормаль-\linebreak ное. 
Поэтому становятся актуальными критерии\linebreak подгонки, а~также использование 
процедур норма\-ли\-зации распределения данных в~многомерном\linebreak случае.
 Если 
с~помощью тестов выявляется, что распределение не является нормальным, то 
Международная федерация клинической химии и~лабораторной медицины 
рекомендует, согласно~[3], использовать двухступенчатую процедуру 
нормализации. Следует обратить внимание, что многошаговость здесь 
относится не к~многомерности, а касается лишь покоординатного 
преобразования распределения данных к~нормальному.
     
     Первые же попытки применения MRR на основе расстояния 
Махалонобиса (фактически это означает принятие модели нормального 
распределения референсных значений) выявили ряд недостатков (более 
подробно смотри в~\cite[разд.~6.2]{4-kri}):
     \begin{itemize}
\item проявление <<проклятий>> размерности при механическом 
увеличении~$d$, в~особенности если игнорируется этап анализа состава 
признаков~[1, 5, 6];
\item из-за небольших объемов обучающей выборки невысокая устойчивость 
при применении, в~частности чувствительность к~увеличению неточностей 
измерений после того, как регион был установлен~\cite{5-kri, 7-kri}. 
\item предположение о нормальном распределении и~попытки <<подправить>> 
действительность с~помощью преобразований реальных данных для их 
нормализации при увеличении размерности данных становятся все более 
шаткими~\cite{5-kri};
\item представление и~интерпретация выводов на основе MRR трудно 
понимаемы не только для специалистов в~предметной области~[8].
\end{itemize}

\vspace*{-9pt}

\section{Многомерный референсный регион высокой плотности}

\vspace*{-2pt}

     Заметим, что в~случае нормального распределения референсных значений 
для точек внут\-ри построенного эллипсоида значения плотности\linebreak распределения 
больше, чем на границе, а~вне~--- меньше. Это замечание позволяет 
предложить другой подход к~построению MRR.
     
     \smallskip
     
     \noindent
     \textbf{Определение.}\ Eсли плотность распределения референсных 
значений есть $f(y)$, то MRR есть область $A_t\hm= \left\{ y\in 
\mathcal{R}^d\vert f(y)\hm\geq t\right\}$ для некоторого порогового 
значения~$t$. 
     
     \smallskip
     
     Для нормального распределения это уже упомянутый эллипсоид равной 
вероятности. Если задается вероятность $(1\hm-\alpha)$ попадания в~$A_t$, то 
пороговое значение~$t$ есть решение уравнения $\int\nolimits_{A_t} 
f(u)\,du\hm=1\hm-\alpha$, получить которое аналитически в~случае 
произвольной плотности распределения вряд ли возможно. Здесь присутствуют 
две проблемы: вычисление многомерного интеграла и~зависимость области 
интегрирования от неизвестного значения. Для решения их предлагается 
привлечь метод моделирования.
     
     Сгенерируем выборку из $f(y)$, которую обозначим как $Y^f\hm= \left\{ 
y_1^f, \ldots, y_m^f\right\}$. Для оценки $\int\nolimits_{A_t} f(u)\,du$ 
используем отношение:

\noindent
\begin{multline*}
     \fr{\left\vert \left\{ y_i^f\vert y_i^f\in A_t\right\}\right\vert }{m} =
      \fr{\left\vert\left\{ y_i^f\vert 
f\left(y_i^f\right) \geq t\right\}\right\vert }{m} ={}\\
{}= 1-\fr{\left\vert \left\{ y_i^f\vert f(y_i^f)<t\right\}\right\vert }{m}=1-
F_m(t)\,,
     \end{multline*}
где $F_m(t)$~--- эмпирическая функция распределения случайной 
величины~$f(y)$, т.\,е.\ случайной величины, являющейся результатом 
преобразования с~помощью функции~$f(\cdot)$ случайной величины, име\-ющей 
плотность распределения~$f(u)$.

     Таким образом, искомая оценка~$t^*$ должна удовле\-тво\-рять уравнению 
$F_m(t^*)\hm=\alpha$ и~может быть получена как непараметрическая оценка 
квантиля\linebreak\vspace*{-12pt}

\pagebreak

\noindent
 порядка~$\alpha$ из распределения $F_m(\cdot)$. Если обозначить 
$f_i\hm= f(y_i^f)$, то~$t^*$ есть~$f_{(r)}$, где
     $$
     r= \begin{cases}
     m\alpha, &\ m\alpha~\mbox{---~целое}\,;\\
     \lfloor m\alpha+1\rfloor\,, & m\alpha~\mbox{--- не целое}\,.
     \end{cases}
     $$
     Заметим, что для такой оценки можно указать доверительный интервал.
     
     Для построения MRR необходимо знать распределение данных. При 
реализации принципа точек высокой плотности в~первую очередь следует 
обратиться к~параметрическим моделям, в~част\-ности к~смеси нормальных 
распределений, име\-ющей плотность распределения
     $$
     f(u) =\sum\limits_{j=1}^k p_j \varphi\left (u,\mu_j, \Sigma_j\right)\,.
     $$
Если $\hat{f}(u)$~--- оценка смеси, то~$t^*$ строится сле\-ду\-ющим образом:
\begin{itemize}
\item генерируется выборка $\left\{ y_1^f,\ldots , y_m^f\right\}$ из $\hat{f}(u)$ и~
для каждого ее $i$-го элемента подсчитывается значение $\hat{f}\left( 
y_i^f\right)$;
\item в~качестве~$t^*$ берется непараметрическая оценка квантиля 
порядка~$\alpha$ (в случае необходимости дополнительно находится 
непараметрическая оценка доверительного интервала для~$t^*$, что 
может характеризовать правильность выбранного объема для 
генерируемой выборки).
\end{itemize}

     Пусть для $f(u)$ имеется~$A_t$, а также получена $\hat{f}(u)$ 
и~соответствующий MRR вида~$\hat{A}_t$. Качество аппроксимации~$A_t$ 
с~по\-мощью~$\hat{A}_t$ можно оценить с~по\-мощью вероятности совпадения 
этих областей, т.\,е. 
     $$
     P_c= \int\limits_{\{ u\in A_t\}\cup \{u\in \hat{A}_t\}} \hspace*{-6mm}
f(u)\,du+\int\limits_{\{u\not\in A_t\} \cup\{ u\not\in \hat{A}_t\}}\hspace*{-6mm} f(u)\,du\,.
     $$
     
     Для оценки  $P_c$ можно использовать величину
     \begin{multline*}
     \hat{P}_c= \fr{\left\vert \left\{ 
     y_i^f\vert y_i^f \in \left\{\left\{ y_i^f\in A_t\right\}\cup \left\{y_i^f\in 
\hat{A}_t\right\}\right\}\right\}\right\vert}{m}+{}\\
{}+\fr{\left\vert \left\{ y_i^f\vert y_i^f \in \left\{\left\{ y_i^f\not\in A_t\right\}\cup 
\left\{ y_i^f\not\in \hat{A}_t\right\}\right\}\right\}\right\vert}{m}\,.
     \end{multline*}
     
     Использование MRR высокой плотности для диагностирования сводится 
к~реализации так называемого слабого критерия значимости для наблюденного 
значения~$x$: нулевая гипотеза заключается в~том, что $x\hm\in A_t$, 
статистика критерия есть $\hat{f}(x)$ и~решение о~принадлежности 
критической об\-ласти~$A_t$ принимается при больших значениях~$\hat{f}(x)$.
     
     Для медицинской практики важна возможность использования 
референсного региона при интерпретации результатов обследования 
некоторого пациента с~вектором признаков~$x$. В~подобных случаях 
сложившейся практикой для слабых критериев значимости является 
использование критического уровня~$\alpha_{\mathrm{cr}}$ (более распространенным 
в~медицине является употребление термина $p$-зна\-че\-ние)  $\alpha_{\mathrm{cr}}\hm= 
\mathrm{Pr}\left\{ \hat{f}(y)\hm\leq \hat{f}(x)\right\}$, где $y$~--- случайная 
величина, имеющая плотность распределения~$\hat{f}(u)$, а $\hat{f}(x)$~--- 
значение плотности распределения~$\hat{f}(u)$ в~точке~$x$. Эта 
характеристика дает представление о~том, насколько сильно данное 
наблюденное значение~$x$ противоречит гипотезе (или подкрепляет ее) 
о~принадлежности данных MRR. При выбранном же заранее уровне 
значимости с~помощью~$\alpha_{\mathrm{cr}}$ сразу же можно принять конкретное 
решение. 

\vspace*{-9pt}

\section{Эксперименты}

\vspace*{-2pt}

     Для демонстрации возможностей MRR использовались данные по 
прогнозу химического состава мочевых камней по метаболическим 
показателям мочи и~сыворотки крови, а также антропологическим 
характеристикам пациентов~[9]. В качестве исходной классификации камней 
рассматривалась следующая: чисто оксалатные (далее обозначены как O), чисто 
уратные (U), чисто фосфатные (P), смесь только оксалатных и~уратных (OU), 
смесь только оксалатных и~фосфатных (OP), смесь только уратных 
и~фосфатных (UP), все остальные. Данная классификация была построена 
в~[10] на основе доминирующих частот встречаемости основных компонентов. 
В~качестве референсных значений рассматривались наборы метаболических 
и~антропологических показателей (их всего было~14), соответствующих 
определенному классу камней.

\begin{table*}\small
\begin{center}


\begin{tabular}{|c|c|c|c|c|c|c|}
\multicolumn{7}{c}{Качество классификации с~помощью MRR}\\
\multicolumn{7}{c}{\ }\\[-6pt]
\hline
\multicolumn{1}{|c|}{\raisebox{-6pt}[0pt][0pt]{\tabcolsep=0pt\begin{tabular}{c}Тип\\ камня\end{tabular}}}&
\multicolumn{1}{c|}{\raisebox{-6pt}[0pt][0pt]{$N$}}&$(1-\alpha)$, 
&\multicolumn{2}{c|}{MRR(5)}&\multicolumn{2}{c|}{MRR(1)}\\
\cline{4-7}
&&&&&&\\[-9pt]
&&\%&$(1-\hat{\alpha})$, \%&$\hat{\beta}$, \%&$(1-\hat{\alpha})$, \%&$\hat{\beta}$, \%\\
\hline
\multicolumn{1}{|c|}{\raisebox{-18pt}[0pt][0pt]{O}}&
\multicolumn{1}{c|}{\raisebox{-18pt}[0pt][0pt]{82}}
&95&100\hphantom{9}&71&90&24\\
&&85&96&78&89&36\\
&&75&91&85&77&44\\
&&65&76&88&74&50\\
\hline
\multicolumn{1}{|c|}{\raisebox{-18pt}[0pt][0pt]{U}}&
\multicolumn{1}{c|}{\raisebox{-18pt}[0pt][0pt]{76}}&95&100\hphantom{9}&75&91&24\\
&&85&99&85&80&35\\
&&75&82&89&74&48\\
&&65&71&91&68&56\\
\hline
\multicolumn{1}{|c|}{\raisebox{-18pt}[0pt][0pt]{P}}&
\multicolumn{1}{c|}{\raisebox{-18pt}[0pt][0pt]{83}}&95&100\hphantom{9}&66&87&25\\
&&85&94&78&86&33\\
&&75&86&82&82&41\\
&&65&77&87&75&47\\
\hline
\end{tabular}
\end{center}
\end{table*}
     
     
     Для каждого из основных классов O, U, P, OU, OP и~UP перед построением 
MRR проводилась селекция признаков и~принималось то значение размерности 
признакового пространства~$d$ и~соответствующий набор показателей, 
которые позволяли прогнозировать состав камней без потери качества 
(методика описана в~\cite{9-kri} и~привела к~значению $d\hm=9$). В~качестве 
модели данных в~первую очередь рассматривалась смесь многомерных 
нормальных распределений из пяти элементов (подбор числа элементов смеси 
проводился с~по\-мощью AIC~--- Akaike information criterion), для соответствующего региона было принято 
обозначение MRR(5). Для сравнения также использовалась модель 
нормального распределения, которой соответствовал MRR(1). Полученные 
результаты приводятся час\-тич\-но в~таблице, где $N$~--- объем 
классифицируемых данных; $\hat{\alpha}$~--- оценка для~$\alpha$; 
$\hat{\beta}$~--- оценка мощности критерия при определении типа камня на 
основании MRR.


     Одной из базовых характеристик является вероятность попадания в~MRR 
$(1\hm-\alpha)$ и~ее оценка $(1\hm-\hat{\alpha})$. Сравнение соответствующих 
столбцов с~учетом значений~$N$ и~ориентировочных значений разброса 
(стандартные отклонения на основе биномиального распределения) не 
позволило выявить явных отклонений. Необходимо, правда, отметить, что во 
всех проанализированных случаях для MRR(5) оказалось, что $1\hm-
\hat{\alpha}\hm\geq 1\hm-\alpha$.
     
     Назначение MRR, заключающееся в~сжатом представлении референсных 
значений, в~многомерном случае практически не проявляется. Для задания 
MRR(5) необходимо указать следующие величины: $1\hm-\alpha$, $t$, 
$p_1,\ldots, p_{k-1}$, $\mu_1, \Sigma_1,\ldots , \mu_k,\Sigma_k$, общее 
количество которых равно  $[2\hm+ (k\hm-1)\hm+ k(d\hm+ d(d\hm+1)/2)]$ 
и,~в~частности, в~рассматриваемых экспериментах~--- 276. Для MRR(1) это 
значение меньше и~равно~56. При этом для обрабатываемой обучающей 
выборки в~зависимости от класса камней речь идет о~порядка~10$^2$ векторах 
данных (см.\ столбец со значениями~$N$), что приблизительно 
дает~10$^3$~скалярных величин.
     
     Другое назначение MRR состоит в~его использовании для 
диагностирования (классификации). В~этой связи в~первую очередь 
проводился сравнительный анализ MRR(1) (фактически это означает, что 
построение региона осуществляется на основе расстояния Махаланобиса) 
и~MRR(5) (модель смеси нормальных распределений и~предложенный 
в~данной работе метод оценивания па\-ра\-мет\-ров региона). Показателем 
информативности метода построения многомерного региона выступала 
мощность соответствующего слабого критерия значимости, а~именно: 
вероятность не попасть в~MRR при условии, что данные берутся из дополнения 
к~классу, для которого построена MRR. Сравнение соответствующих столбцов 
говорит о~явном преимуществе двух предложенных моментов: усложнение 
модели данных путем перехода от нормального распределения к~смеси 
нормальных распределений и~построение региона высокой плотности.
     
     Использование критического уровня можно продемонстрировать  
с~по\-мощью зависимости результатов сравнения двух классов от того, какой 
класс взять за основу. Введем для возможных значений $p$-ве\-ли\-чи\-ны три 
интервала: $(-\infty, 1\%)$, $[1\%, 5\%)$, $[5\%, 100\%)$ с~соответствующей 
интерпретацией положения наблюденного набора показателей для пациента 
относительно построенного MRR: уверенное непопадание, неуверенное 
попадание, уверенное попадание. Если MRR построить для оксалатных камней, 
то результаты для анализа пациентов с~фосфатными камнями дадут следующий 
вектор относительных частот попадания $p$-ве\-ли\-чин в~указанные 
интервалы: $(60\%, 18\%, 22\%)$. Если же MRR строить для фосфатных 
камней, то получим $(71\%, 5\%, 24\%)$. Таким образом, для классификации 
указанных камней при приблизительно одинаковых частотах попадания в~MRR 
(22\% или~24\%) уверенный отказ от референсного региона происходит чаще, 
если принять за базовый MRR регион для фосфатных камней. Построение 
шкалы, подобной рассмотренной, является прерогативой специалистов 
в~предметной области, в~данной работе она использовалась только для 
иллюстрации. 

\vspace*{-6pt}

\section{Заключение}

\vspace*{-2pt}

     На настоящий момент имеется относительно мало примеров применения 
MRR в~клинической практике. Тому есть несколько причин. Математическое 
обеспечение, необходимое для получения и~применения MRR, не отвечает 
возможностям большинства клинических лабораторий. Лаборатории слабо 
оснащены программными средствами\linebreak для реализации достаточно сложного 
математического аппарата многомерного анализа, а~еще важнее, что 
отсутствуют методики, инструкции по\linebreak использованию соответствующих 
средств. Лишь немногие клинические применения демонстрируют 
преимущества MRR, хотя свидетельств неудачных попыток больше.
     
     Несмотря на сложности внедрения мно\-го\-мерно\-го анализа референсных 
значений, можно сформулировать некоторые рекомендации по иссле\-до\-ва\-нию 
и~разработке MRR. Во-пер\-вых, эффективная размерность в~MRR должна 
быть как можно меньше, чтобы избежать затенения диагностически полезной 
информации тестами, со\-зда\-ющи\-ми шум. Низкая размерность также должна 
уменьшить неблагоприятные последствия увеличения неточности результатов 
в~связи с~ростом числа анализируемых показателей. Во-вто\-рых, показатели 
(тес\-ты), включенные в~MRR, должны быть физиологически релевантными 
исследуемому кругу расстройств, чтобы максимизировать информацию, 
полученную от MRR. В-треть\-их, чтобы учесть эффекты долгосрочной 
лабораторной из\-мен\-чи\-вости, данные, используемые для получения MRR, 
долж\-ны быть собраны и~проанализированы в~течение достаточно большого 
периода времени (от нескольких недель до нескольких месяцев).  
В-чет\-вер\-тых, представление результатов лабораторных исследований 
следует осуществлять в~графическом виде, чтобы помочь врачам лучше понять 
MRR. Различные подходы к~уменьшению размерности помогут выполнить это 
требование.
     
     Необходима дальнейшая разработка пояснительных инструментов, 
способных воспринять результаты анализа MRR. При этом дополнительно 
необходима информация о~том, какие именно тес\-ты являются важнейшими 
факторами нарушения нормы. Надо признать, что соответствующий 
математический аппарат еще предстоит разработать. Решение перечисленных 
вопросов играет важную роль для обеспечения постоянного клинического 
применения MRR. 

\vspace*{-6pt}
     
{\small\frenchspacing
 {%\baselineskip=10.8pt
 \addcontentsline{toc}{section}{References}
 \begin{thebibliography}{99}
 
 \vspace*{-2pt}
 
\bibitem{1-kri}
\Au{Boyd J.\,C.} Reference regions of two or more dimensions~// Clin. Chem. Lab. 
Med., 2004. Vol.~42. No.\,7. P.~739--746.
\bibitem{2-kri}
\Au{Winkel P.} Patterns and clusters~--- multivariate approach for interpreting 
clinical chemistry results~// Clin. Chem., 1973. Vol.~19. No.\,12. P.~1329--1333.
\bibitem{3-kri}
IFCC. Expert panel on theory of reference values. Approved recommendation on the 
theory of reference values. Part~5. Statistical treatment of collected reference values. 
Determination of reference limits~// J.~Clin. Chem. Clin. Biochem., 1987. Vol.~25. 
No.\,9. P.~645--656.
\bibitem{4-kri}
\Au{Кривенко М.\,П.} Статистические методы представления и~предварительной 
обработки референсных значений.~--- М.: ФИЦ ИУ РАН, 2016. 160~с.
\bibitem{5-kri}
\Au{Boyd J.\,C., Lacher~D.\,A.} The multivariate reference range: An alternative 
interpretation of multi-test profiles~// Clin. Chem., 1982. Vol.~28. No.\,2.  
P.~259--265.
\bibitem{6-kri}
\Au{Albert A., Harris~E.\,K.} Multivariate interpretation of clinical laboratory  
data.~--- New York, NY, USA: CRC Press, 1987. 328~p.
\bibitem{7-kri}
\Au{Linnet K.} Influence of sampling variation and analytical errors on the 
performance of the multivariate reference region~// Meth. Inf. Med., 1988. Vol.~27. 
No.\,1. P.~37--42.
\bibitem{8-kri}
\Au{Durbridge T.\,C.} Clinical acceptance of a multi-test reference region for 
biochemical-panel results~// Clin. Chem., 1983. Vol.~29. No.\,10. P.~1724--1726.
\bibitem{9-kri}
\Au{Кривенко М.\,П.} Критерии значимости отбора признаков классификации~// 
Информатика и~её применения, 2016. Т.~10. Вып.~3. С.~32--40.
\bibitem{10-kri}
\Au{Кривенко М.\,П., Голованов~С.\,А., Сивков~А.\,В.} Анализ однородности 
данных о химическом составе камней при уролитиазе~// Информатика и~её 
применения, 2013. Т.~7. Вып.~4. С.~94--104.
 \end{thebibliography}

 }
 }

\end{multicols}

\vspace*{-10pt}

\hfill{\small\textit{Поступила в~редакцию 5.12.16}}

\vspace*{4pt}

%\newpage

%\vspace*{-24pt}

\hrule

\vspace*{2pt}

\hrule

\vspace*{-3pt}


\def\tit{HIGH-DENSITY MULTIVARIATE REFERENCE REGION\\[-5pt]}

\def\titkol{High-density multivariate reference region}

\def\aut{M.\,P.~Krivenko\\[-7pt]}

\def\autkol{M.\,P.~Krivenko}

\titel{\tit}{\aut}{\autkol}{\titkol}

\vspace*{-16pt}


\noindent
Institute of Informatics Problems, Federal Research Center 
``Computer Science and Control'' of the Russian
Academy of Sciences,  44-2~Vavilov Str., Moscow 119333, Russian Federation



\def\leftfootline{\small{\textbf{\thepage}
\hfill INFORMATIKA I EE PRIMENENIYA~--- INFORMATICS AND
APPLICATIONS\ \ \ 2017\ \ \ volume~11\ \ \ issue\ 2}
}%
 \def\rightfootline{\small{INFORMATIKA I EE PRIMENENIYA~---
INFORMATICS AND APPLICATIONS\ \ \ 2017\ \ \ volume~11\ \ \ issue\ 2
\hfill \textbf{\thepage}}}

\vspace*{2pt}




\Abste{The paper considers the principles of construction of multivariate 
reference regions. An original method of construction of 
a~region on the basis of areas of high density of points and approximation 
of data distribution with a~mixture of normal distributions is suggested. 
To estimate the threshold for the probability density, the bootstrap method is used. 
As an experiment, the paper considers the problem of description and use of 
the reference region for predicting the type of urinary stones. 
Real data treatment demonstrated the benefits of the proposed solutions.}

\KWE{multivariate reference region; high-density region; bootstrap method; 
multivariate normal mixture}

\DOI{10.14357/19922264170207} 

%\vspace*{-18pt}

%\Ack
%\noindent



%\vspace*{3pt}

  \begin{multicols}{2}

\renewcommand{\bibname}{\protect\rmfamily References}
%\renewcommand{\bibname}{\large\protect\rm References}

{\small\frenchspacing
 {%\baselineskip=10.8pt
 \addcontentsline{toc}{section}{References}
 \begin{thebibliography}{99}
\bibitem{1-kri-1}
\Aue{Boyd, J.\,C.} 2004. Reference regions of two or more dimensions. \textit{Clin. 
Chem. Lab. Med.} 42(7):739--746.

\bibitem{2-kri-1}
\Aue{Winkel, P.} 1973. Patterns and clusters~--- multivariate approach for interpreting 
clinical chemistry results. \textit{Clin. Chem.} 19(12):1329--1333.
\bibitem{3-kri-1}
IFCC. 1987. Expert panel on theory of reference values. Approved recommendation on the 
theory of reference values. Part~5. Statistical treatment of collected reference values. 
Determination of reference limits. \textit{J.~Clin. Chem. Clin. Biochem.} 
25(9):645--656.
\bibitem{4-kri-1}
\Aue{Krivenko, M.\,P.} 2016. \textit{Statisticheskie metody predstavleniya 
i~predvaritel'noy obrabotki referensnykh znacheniy}
[Statistical methods for representation and preliminary processing of
reference values]. Moscow: FRC CSC RAS. 160~p.

\bibitem{5-kri-1}
\Aue{Boyd, J.\,C., and D.\,A.~Lacher.} 1982. The multivariate reference range: An 
alternative interpretation of multi-test profiles. \textit{Clin. Chem.}  
28(2):259--265.
\bibitem{6-kri-1}
\Aue{Albert, A., and E.\,K.~Harris.} 1987. \textit{Multivariate interpretation of 
clinical laboratory data}. New York, NY: CRC Press. 328~p.
\bibitem{7-kri-1}
\Aue{Linnet, K.} 1988. Influence of sampling variation and analytical errors on the 
performance of the multivariate reference region. \textit{Meth. Inf. Med.}  
27(1):37--42.
\bibitem{8-kri-1}
\Aue{Durbridge, T.\,C.} 1983. Clinical acceptance of a multi-test reference region 
for biochemical-panel results. \textit{Clin. Chem.} 29(10):1724--1726.
\bibitem{9-kri-1}
\Aue{Krivenko, M.\,P.} 2016. Kriterii znachimosti otbora priznakov klassifikatsii
[Significance tests of feature selection for~classification]. \textit{Informatika i~ee 
Primeneniya~--- Inform. Appl.} 10(3):32--40.
\bibitem{10-kri-1}
\Aue{Krivenko, M.\,P., S.\,A.~Golovanov, and A.\,V.~Sivkov}. 2013. Analiz 
odnorodnosti dannykh o~khimicheskom sostave kamney pri urolitiaze
[Analysis of data homogeneity of~the~chemical compositions 
of~stones in~case of~urolithiasis]. \textit{Informatika i~ee Primeneniya~---
Inform Appl.} 7(4):94--104.
\end{thebibliography}

 }
 }

\end{multicols}

\vspace*{-3pt}

\hfill{\small\textit{Received December 5, 2016}}


\Contrl

\noindent
\textbf{Krivenko Michail P.} (b.\ 1946)~--- Doctor of Science in technology, 
professor, leading scientist, Institute of Informatics Problems, Federal Research 
Center ``Computer Science and Control'' of the Russian Academy of Sciences, 
\mbox{44-2}~Vavilov Str., Moscow 119333, Russian Federation; \mbox{mkrivenko@ipiran.ru}

\label{end\stat}


\renewcommand{\bibname}{\protect\rm Литература}  %7
\def\stat{grusho}

\def\tit{АРХИТЕКТУРНЫЕ РЕШЕНИЯ В~ЗАДАЧЕ ВЫЯВЛЕНИЯ МОШЕННИЧЕСТВА ПРИ~АНАЛИЗЕ 
ИНФОРМАЦИОННЫХ ПОТОКОВ В~ЦИФРОВОЙ ЭКОНОМИКЕ$^*$}

\def\titkol{Архитектурные решения в~задаче выявления мошенничества при~анализе 
информационных потоков в
%~цифровой 
экономике}

\def\aut{А.\,А.~Грушо$^1$, М.\,И.~Забежайло$^2$, Н.\,А.~Грушо$^3$, 
Е.\,Е.~Тимонина$^4$}

\def\autkol{А.\,А.~Грушо, М.\,И.~Забежайло, Н.\,А.~Грушо, 
Е.\,Е.~Тимонина}

\titel{\tit}{\aut}{\autkol}{\titkol}

\index{Грушо А.\,А.}
\index{Забежайло М.\,И.}
\index{Грушо Н.\,А.}
\index{Тимонина Е.\,Е.}
\index{Grusho A.\,A.}
\index{Zabezhailo M.\,I.}
\index{Grusho N.\,A.}
\index{Timonina E.\,E.}


{\renewcommand{\thefootnote}{\fnsymbol{footnote}} \footnotetext[1]
{Работа частично поддержана РФФИ (проекты 18-29-03081 и~18-07-00274).}}


\renewcommand{\thefootnote}{\arabic{footnote}}
\footnotetext[1]{Институт проблем информатики Федерального исследовательского центра <<Информатика и~управление>> 
Российской академии наук, grusho@yandex.ru}
\footnotetext[2]{Институт проблем информатики Федерального исследовательского центра <<Информатика и~управление>> 
Российской академии наук, m.zabezhailo@yandex.ru}
\footnotetext[3]{Институт проблем информатики Федерального исследовательского центра <<Информатика и~управление>> 
Российской академии наук, info@itake.ru}
\footnotetext[4]{Институт проблем информатики Федерального исследовательского центра <<Информатика и~управление>> 
Российской академии наук, eltimon@yandex.ru}

\vspace*{-12pt}
   

 
  
  \Abst{Cформулирован подход к~исследованию некоторых видов мошенничества в~цифровой 
экономике с~использованием причинно-следственных связей. Во всех видах рассматриваемых 
мошенничеств должно наблюдаться несоответствие между целями финансовых транзакций 
и~реальной стоимостью достижения этих целей. Данные о транзакциях можно собирать, 
наблюдая информационные потоки, в~которых отражаются эти транзакции. Архитектура сбора 
данных и~их анализа может быть организована с~помощью распределенных реестров 
с~централизованным консенсусом, что позволяет создать аналог электронной бухгалтерской 
книги, фиксирующей финансово-экономическую деятельность субъектов цифровой экономики в~регионе. 
  Рассматриваемые методы выявления мошенничества основаны на противоречиях 
между действиями, описанными в~транзакциях, и~информацией, содержащейся в~планах, 
стандартах, прецедентах и~др. Рассмотрен метод, основанный на некоторой упрощенной схеме 
реализации абстрактного проекта. Для выявления противоречий необходимо проводить анализ 
от следствия к~причине, т.\,е.\ искать аномалии в~информации, описывающей порождение 
наблюдаемых следствий. 
  Показано, как в~реализации проекта можно выделять простые <<необходимые условия>> 
нарушения при\-чин\-но-след\-ст\-вен\-ных связей, т.\,е.\ множество <<необходимых условий>>, 
нарушение которых свидетельствует о наличии мошенничества. Это множество <<необходимых 
условий>> можно назвать метаданными для контроля проекта на выявление мошенничества.} 
 
 
  \KW{цифровая экономика; информационные потоки; при\-чин\-но-след\-ст\-вен\-ные связи; 
выявление мошеннических схем} 

\DOI{10.14357/19922264190204}
  
\vspace*{-4pt}


\vskip 10pt plus 9pt minus 6pt

\thispagestyle{headings}

\begin{multicols}{2}

\label{st\stat}

\section{Введение}

\vspace*{3pt}

  В работе сформулирован подход к~исследованию некоторых видов 
мошенничества в~цифровой экономике с~использованием  
при\-чин\-но-след\-ст\-вен\-ных связей. Рассматриваются три вида мошенничества, 
а именно:
  \begin{enumerate}[(1)]
\item отмыв денег; 
\item обман при выполнении договорных обязательств при реализации 
технических проектов (строительные проекты и~др.); 
\item незаконный вывод денег. 
\end{enumerate}

  Названные виды мошенничества могут быть сведены к~решению одного типа 
задач. Для отмывания денег источник должен заключать фиктивные контракты, 
в~соответствии с~которыми будут переводиться средства за заведомо ненужную 
работу и~материалы. 
  
  Мошенничество, связанное с~невыполнением договорных обязательств, связано 
со снижением качества услуг, качества и~количества закупаемых 
материалов, выполнением работ с~ненадлежащим качеством. 
  
  Вывод денег связан с~переводом средств фир\-мам-од\-но\-днев\-кам, которые 
заведомо не могут выполнить обязательства по контрактам, за которые им 
переводятся средства. 
  
  Таким образом, во всех трех видах рассматриваемых мошенничеств должно 
наблюдаться несоответствие между целями финансовых транзакций и~реальной 
стоимостью достижения этих целей. Данные о транзакциях можно собирать, 
наблюдая информационные потоки, в~которых отражаются эти транзакции. 
  
  Однако для наблюдения таких информационных потоков необходимо создавать 
архитектуру\linebreak телекоммуникационной системы, позволяющей перехватывать 
и~собирать данные о всех транзакциях. Например, такая архитектура может быть 
организована с~помощью распределенных реестров с~централизованным 
консенсусом, т.\,е.\ все информационные потоки, сформированные в~цифровой 
экономике и~несущие информацию о транзакциях, проходят через некоторый 
центральный узел, запоминающий их в~форме распределенного реестра. Такие 
реестры могут дублироваться в~аналогичных центрах различных регионов, что 
позволяет создать аналог электронной бухгалтерской книги, фиксирующей 
фи\-нан\-со\-во-эко\-но\-ми\-че\-скую деятельность субъектов цифровой экономики. Такой 
подход предложено реализовать на базе системы ситуационных центров, что 
отражено в~работах~[1, 2].
  
  Собранная из информационных потоков информация о~транзакциях, т.\,е.\ 
о~контрактах, договорах, платежах, отчетах, закупленных материалах, 
характеристиках исполнителей работ и~др., собирается в~базе данных в~указанном 
центре. Согласно теории интеллектуальных сис\-тем~[3], эту базу данных можно 
называть базой фактов (БФ). Базу фактов можно представить как бинарную мат\-ри\-цу, 
строки которой описывают характеристики, входящие в~транзакции, а столбцы 
нумеруются характеристиками. Строки матрицы будем называть 
\textit{объектами}~[4, 5]. 
  
  Рассматриваемые в~работе методы выявления мошенничества будут основаны 
на противоречиях между действиями, описанными в~транзакциях, и~информацией, 
содержащейся в~планах, стандартах, прецедентах и~др. Для нахождения 
противоречий в~архитектуре центра предусмотрена другая база данных~--- база 
знаний (БЗ)~\cite{3-gr, 6-gr}, которая устроена так же, как БФ. 
  
  Информация в~БЗ собирается на основе положительного опыта или расчетов. 
Используя БЗ, можно выводить факты нарушения при\-чин\-но-след\-ст\-вен\-ных 
связей. Нарушения при\-чин\-но-след\-ст\-вен\-ных связей будем называть 
\textit{аномалиями}. 
  
  Для упрощения дальнейшее изложение будет вестись в~рамках поиска 
противоречий при выполнении некоторого абстрактного проекта. Выявление 
аномалий будет происходить на основе фактов из БФ с~помощью знаний из БЗ 
методами искусственного интеллекта и~интеллектуального анализа 
данных~\cite{6-gr}. 

\vspace*{-10pt}
  
  \section{Модели}
  
  \vspace*{-3pt}
  
  Наиболее сложная из рассмотренных выше задач~--- выявление противоречий, 
т.\,е.\ использование БЗ для получения новых знаний и~выявление аномалий из 
полученных фактов. 
  
  Все способы выявления противоречий основаны на определении 
  причинно-следственных связей. При этом противоречия в~параметрах транзакций по 
отношению к~требуемым в~БЗ составляют сущность аномалий. 
  
   Далее будет рассмотрен метод, основанный на некоторой упрощенной схеме 
реализации абстрактного проекта. 
  
  Каждый проект имеет цель: например, цель представляет собой построение 
некоторой системы. Воспользуемся структурным подходом, который позволяет 
строить проект на основе разбиения системы на подсистемы и~определения 
взаимодействий подсистем~\cite{7-gr}. При этом каждая подсистема также 
представима структурной моделью. 
  
  Как сама система, так и~каждая ее подсистема имеют свой функционал 
и~спецификацию, па\-ра\-мет\-ры настройки и~домены параметров настройки. Кроме 
этих характеристик существует множество характеристик, связанных 
с~<<жизненным циклом>> создания системы. Сюда входят работы, ресурсы, 
сроки выполнения работ по созданию подсистем и~самой системы, стоимости 
компонентов и~материалов, стоимости работ, схемы поставок, договорные 
обязательства и~др. Все характеристики связаны между собой, поэтому можно 
говорить о стоимости и~времени изготовления структурных компонентов системы. 
  
  Одной из важнейших характеристик является смета (система смет для 
подсистем). Смета сопоставляет каждому компоненту системы стоимость его 
изготовления и~настройки. 
  
  Схема построения системы может быть пред\-став\-ле\-на диаграммой, 
изображенной на рис.~1. 

{ \begin{center}  %fig1
 \vspace*{9pt}
   \mbox{%
 \epsfxsize=79mm 
 \epsfbox{gru-1.eps}
 }


\vspace*{9pt}


\noindent
{{\figurename~1}\ \ \small{Диаграмма достижения цели}}
\end{center}
}

\vspace*{9pt}

\addtocounter{figure}{1}
  
  


  Представленная на рис.~1 диаграмма позволяет описать основные классы 
возможных противоречий при достижении цели. Противоречия возникают, когда 
данные БФ не соответствуют требуемым характеристикам. 
  
  
  \section{Потенциальные классы аномалий при~достижении цели}
  
  Выделим четыре потенциальных класса противоречий, которые показывают, 
каким образом нужно искать эти противоречия.
  
 
  Противоречие цели и~проекта (рис.~2) возникает при отсутствии обоснования 
или в~случае логического противоречия между возможностями проектируемого 
функционала и~целью системы. Отметим, что в~проект входят сроки, перечень 
работ, материалы, настройки, которые описываются соответствующими 
параметрами и~допустимыми значениями этих параметров. Проект формируется 
на основе БЗ и~расчетов, исходя из информации, полученной по аналогии 
с~другими проектами и~решениями, которые считаются апробированными. 
  
  Отметим, что цель порождает проект и~в этом смысле является причиной 
проекта. Однако для анализа противоречий необходимо двигаться по штриховой 
стрелке диаграммы (см.\ рис.~2) от проекта к~цели. В~самом деле, любой компонент 
проекта направлен на теоретическое достижение цели. Цель~--- сложный объект, 
поэтому в~проекте могут возникнуть характеристики, противоречащие хотя бы 
некоторым характеристикам цели. Это делает проект противоречивым, но вывод 
об этом может быть сделан только на уровне описания цели. 
  

  Противоречия между проектом и~его реализацией, исключая настройки 
(рис.~3), могут возникать, например, при закупке исполнителем материалов более 
низкого качества по более низким ценам, при попытках достижения требуемых 
сроков работы за счет снижения качества выполнения работ, за счет нахождения 
<<объективных>> причин для увеличения сроков работы и,~следовательно, 
увеличения цены реализации проекта. 


  Для выявления указанных противоречий необходимо двигаться по диаграмме 
(см.\ рис.~3) в~обратную сторону в~соответствии со~штриховыми стрелками. 
Действительно, выявить противоречия между характеристиками закупленных 
материалов и~требуемыми по проекту можно только при обращении к~проекту 
и~его спецификациям. Манипуляции со сроками работы также можно выявить 
только при обращении к~соответствующим расчетам в~проекте. Задержки в~сроках 
работы, связанные с~поставками материалов, можно определить только на 
предыдущем этапе диаграммы (см.\ рис.~3) в~описании проекта. 


  


  Противоречия между реализацией проекта и~его настройкой (рис.~4) возникает, 
когда не удается добиться требуемых значений параметров функционала, не 
удается обеспечить необходимый уровень\linebreak\vspace*{-12pt}

{ \begin{center}  %fig2
 \vspace*{-6pt}
   \mbox{%
 \epsfxsize=16mm 
 \epsfbox{gru-2.eps}
 }


\vspace*{6pt}


\noindent
{{\figurename~2}\ \ \small{Противоречия цели и~проекта}}
\end{center}
}

%\vspace*{9pt}

\addtocounter{figure}{1}

{ \begin{center}  %fig3
 \vspace*{6pt}
    \mbox{%
 \epsfxsize=79mm 
 \epsfbox{gru-3.eps}
 }


\end{center}

\vspace*{-2pt}


\noindent
{{\figurename~3}\ \ \small{Противоречия проекта и~его реализации (без настройки)}}
}

\vspace*{6pt}

\addtocounter{figure}{1}

{ \begin{center}  %fig4
 \vspace*{1pt}
   \mbox{%
 \epsfxsize=54.5mm 
 \epsfbox{gru-4.eps}
 }


\end{center}


\noindent
{{\figurename~4}\ \ \small{Противоречия реализации проекта и~его на\-стройки}}
}

%\vspace*{9pt}

\addtocounter{figure}{1}

{ \begin{center}  %fig5
 \vspace*{5pt}
    \mbox{%
 \epsfxsize=79mm 
 \epsfbox{gru-5.eps}
 }


\end{center}



\noindent
{{\figurename~5}\ \ \small{Противоречия цели и~достигнутой реализации проекта}}
}

\vspace*{6pt}

\addtocounter{figure}{1}

\noindent
 качества реализации проекта. Для 
определения противоречия в~настройках надо опять же двигаться по диаграмме 
(см.\ рис.~4) в~обратную сторону по штриховым стрелкам, так как для выявления 
характеристик результатов работы, которые не дают возможности реализации 
определенного функционала, необходимо иметь информацию о результатах этой 
работы. 


  



  Противоречие между целью и~достигнутой реализацией проекта (рис.~5) 
возникает, когда реализованная система не позволяет достичь цели. В~этом случае 
опять противоречие нужно искать, двигаясь от цели к~реальному достигнутому 
функционалу по штриховой стрелке (см.\ рис.~5).
  
  Суммируя положения, изложенные в~данном разделе, приходим к~выводу, что 
для выявления противоречий необходимо проводить анализ от следствия 
к~причине, т.\,е.\ искать аномалии в~информации, описывающей порождение 
наблюдаемых следствий. 
  
  
  \section{Связь противоречий и~причин}
  
  Прежде чем построить связь между причинами и~противоречиями, кратко 
опишем простейшую модель связи этих понятий. Причины и~противоречия будут 
сформулированы для представления компонентов системы как объектов, 
обладающих наборами известных характеристик~\cite{4-gr, 5-gr}. 
  
  Пусть $U\hm=\{\alpha, \beta, \ldots\}$~--- совокупность характеристик 
(пространство характеристик). Согласно~\cite{4-gr} \textit{объектом}~$O$ 
называется любое подмножество характеристик $O\hm\subseteq U$. Рассмотрим 
последовательность объектов, возможно в~различных пространствах 
характеристик. 
  
  \smallskip
  
  \noindent
  \textbf{Определение~1.}\ Объект~$P$ с~числом характеристик, большим или 
равным~2, является \textit{причиной} объекта (\textit{свойства})~$B$ в~цепочке 
наблюдаемых объектов тогда и~только тогда, когда выполнены следующие 
условия:
  \begin{enumerate}[(1)]
\item для каждого объекта~$C$, если $P\hm\subseteq C$, то $C\mapsto B$, где 
$C\mapsto B$ означает, что объект~$B$ присутствует в~объекте, следующем за 
объектом~$C$;
\item объект~$P$ является минимальным объектом, удовлетворяющим 
условию~1, а~именно: $\forall \alpha\hm\in P$ объект~$P\backslash \{\alpha\}$ 
не является причиной, т.\,е.\ $\exists C:\ \alpha\not\in C$, $P\backslash 
\{\alpha\}\hm\subseteq C$ и~$C\not\mapsto B$, где $C\not\mapsto B$ означает, 
что~$B$ не может содержаться в~объекте, следующем за объектом~$C$. 
\end{enumerate}

  Приведенное определение причины является упрощением причин, 
возникающих в~реальном мире. Например, реальные причины могут возникать\linebreak 
как совокупность характеристик из разных пространств. Одно следствие может 
порождаться разными причинами или возникать из внешних\linebreak и~ненаблюдаемых 
характеристик. Однако пред\-став\-лен\-ная далее формализация позволяет доступно 
изложить при\-чин\-но-след\-ст\-вен\-ные истоки противоречий, которые 
инициируют в~дальнейшем глубокое исследование рассматриваемых процессов.
  
  Будем считать, что для любого интересующего нас свойства~$B$ существует 
причина. Тогда справедлива следующая теорема.
  
  \smallskip
  
  \noindent
  \textbf{Теорема~1.}\ \textit{Для любого свойства~$B$ существует 
единственная причина}. 
  
  \smallskip
  
  \noindent
  Д\,о\,к\,а\,з\,а\,т\,е\,л\,ь\,с\,т\,в\,о\,.\ \ Доказательство будем вести от противного, 
т.\,е.\ предположим, что существуют две причины свойства~$B$: $P$ 
и~$P^\prime$, $P\hm\not= P^\prime$. Тогда существует $\alpha\hm\in U$, которое 
удовлетворяет одному из двух условий:
  \begin{itemize}
\item[(а)] $\alpha\in P$, $\alpha\notin P^\prime$;
\item[(б)] $\alpha\notin P$, $\alpha \in P^\prime$.
\end{itemize}

  Пусть выполняется условие~(б). Тогда $P^\prime\backslash \{\alpha\}$ не 
является причиной по условию~2 определения~1, т.\,е.\ $\exists C$ такое, что 
$\alpha\notin C$, $P^\prime\backslash \{\alpha\}\hm\subseteq C$ и~$C\not\mapsto B$. 
Но если~$B$ произошло и~$P$ его причина, то $C\mapsto B$, что противоречит 
предположению. Теорема~1 доказана.
  
  \smallskip
  
  \noindent
  \textbf{Лемма.} \textit{Если $P$~--- причина появления свойства~$B$, то 
объект~$B$ определяет существование свойства~$P$ в~объекте, 
предшествующем~$B$. }
  
  \smallskip
  
  \noindent
  Д\,о\,к\,а\,з\,а\,т\,е\,л\,ь\,с\,т\,в\,о\,.\ \ Из предположения, что у~каж\-до\-го 
свойства~$B$ есть причина, и~условия, что~$P$ является причиной~$B$, следует, 
что при появлении в~данных свойства~$B$ объект~$C$, предшествующий 
появлению~$B$, содержит как часть объект~$P$. Это следует из теоремы~1 
и~определения причины. 
  
  Докажем принцип <<необходимого условия>>, который, несмотря на простоту 
доказательства, будет играть в~дальнейшем существенную роль.
  
  \smallskip
  
  \noindent
  \textbf{Теорема~2.} \textit{Если~$P$~--- причина появления свойства~$B$ 
и~$A\hm\subseteq P$, то объект~$B$ определяет наличие свойства~$A$ 
в~объекте, предшествующем~$B$}. 
  
  \smallskip
  
  \noindent
  Д\,о\,к\,а\,з\,а\,т\,е\,л\,ь\,с\,т\,в\,о\,.\ \ Пусть в~данных имеется объект~$B$ 
и~$P\mapsto B$, тогда в~силу существования и~единственности причины~$B$ 
в~данных должен существовать объект~$C$, предшествующий~$B$ 
и~содержащий причину~$P$. Поскольку $A\hm\subseteq P$ и~$B$ содержит 
причину~$P$, то $B\mapsto A$. С~учетом леммы теорема~2 доказана.
  
  \smallskip
  
  Пусть даны пространства $U_1, U_2,\ldots$ и~имеется последовательность 
данных (процесс выполнения этапов проекта в~соответствии с~рис.~1) $A, B, 
\ldots$, где каждый объект является подмножеством некоторого 
пространства~$U_i$, $i\hm=1,\ldots$ Тогда в~объекте~$A$ присутствует 
причина~$P$ появления интересующего нас свойства~$C$ в~объекте~$B$. Пусть 
$P\hm\subseteq A$, тогда по теореме~2 $\forall \alpha\hm\in P$:  
$C\mapsto \{\alpha\}$, т.\,е.\ из появления~$C$ следует появление 
характеристики~$\alpha$ в~предшествующем объекте. Это необходимое условие 
того, что~$C$ удовлетворяет причинно-следственным связям развития процесса 
выполнения проекта. Если для~$C$ нет характеристики~$\alpha$, которую можно 
отнести к~причине~$C$, то можно считать, что нарушена  
при\-чин\-но-след\-ст\-вен\-ная связь и~$C$~--- аномальный объект. 
  
  \smallskip
  
  \noindent
  \textbf{Пример.} Если объект~$C$ состоит в~получении суммы~$a$ 
фирмой~$K$, то согласно теореме~2 в~пред\-шест\-ву\-ющем объекте должна 
существовать причина перевода суммы~$a$ на фирму~$K$. Если эта причина 
в~проекте отсутствует, то это можно считать признаком мошеннической схемы. 
Все проекты по предположению собираются из <<кубиков>>, содержащихся в~БЗ. 
Тогда можно сравнить цену объекта~$C$, породившего получение суммы~$a$, 
и~сумму, присутствующую в~смете проекта. Если разница велика, то это либо 
ошибка проекта, либо признак мошеннической схемы.
  
  \section{Поиск противоречий на~основе~принципа <<необходимых~условий>>}
   
  Как было показано в~разд.~3, нахождение противоречий соответствуют 
движению от следствия к~причине. Для каждого объекта в~наблюдаемых данных 
выявление причин его появления является трудоемкой задачей. Кроме того, при 
реализации контроля соблюдения при\-чин\-но-след\-ст\-вен\-ных связей на 
большом множестве участников экономической деятельности задача анализа 
причин становится трудоемкой. Поэтому процедуру контроля необходимо разбить 
на два этапа, где первый этап состоит в~анализе простых <<необходимых 
условий>> проявления мошенничества, когда используется хотя бы одна 
известная характеристика причины. Второй этап (в~режиме офлайн) состоит 
в~выявлении причин, позволяющих провести анализ источников мошеннических 
схем. 
  
  Один из подходов к~выбору <<необходимых условий>> состоит в~построении 
множества подцелей исходной цели проекта (структурный метод построения 
проекта~\cite{7-gr}). Каждая подцель описывается диаграммой на рис.~1, 
и~реализации подцелей должны образовывать полный функционал цели. Это 
является необходимым, но не достаточным условием достижения цели, так как 
при таком подходе отсутствует компонент согласования всех подцелей в~единую 
систему. Однако такой подход значительно упрощает анализ выполнения проекта 
на предмет поиска мошенничества. Если признаки мошенничества будут 
обнаружены в~реализации хотя бы одной из подцелей, то это значит, что 
мошенничество присутствует в~реализации всего проекта. 
  
  Аналогично в~реализации каждого этапа в~любой из подцелей можно выделять 
простые <<необходимые условия>> нарушения при\-чин\-но-след\-ст\-венн\-ых 
связей. 
  
  Таким образом, получается множество <<необходимых условий>>, нарушение 
которых свидетельствует о наличии мошенничества. Это множество 
<<необходимых условий>> можно назвать метаданными~[8, 9] для контроля 
проекта на выявление мошенничества. 
  
  
  \section{Заключение }
  
  В поиске противоречий необходимо от транзакций, соответствующих 
следствиям при\-чин\-но-след\-ст\-вен\-ных связей, переходить к~анализу причин 
наблюдаемых следствий. Это сложная задача, которая связана с~описанием причин 
определенных свойств. 
  
  В работе представлена модель, позволяющая строить множество необходимых 
условий соответствия наблюдаемого следствия вызвавшей его причине. Этот 
подход делает поиск противоречий вполне вычислимой задачей, но не гарантирует 
успех. 
  
  {\small\frenchspacing
 {%\baselineskip=10.8pt
 \addcontentsline{toc}{section}{References}
 \begin{thebibliography}{9}
\bibitem{1-gr}
\Au{Грушо А.\,А., Зацаринный~А.\,А., Тимонина~Е.\,Е.} Блокчейны цифровой экономики на базе 
системы ситуационных центров и~централизованного консенсуса~// Радиолокация, навигация, 
связь: Мат-лы XXV Междунар. научн.-технич. конф.~---
Воронеж: Издательский дом ВГУ, 2019. Т.~6. С.~183--191. 
\bibitem{2-gr}
\Au{Grusho A., Zatsarinny~A., Timonina~E.} A~system approach to information security in 
distributed ledgers on the situational centers platform.~---
Lecture notes in computer science ser.~--- Springer, 2019 
(in press).
\bibitem{3-gr}
\Au{Финн В.\,К.} Искусственный интеллект: Методология, применения, философия.~--- М.: 
Красанд, 2011. 448~с.

\bibitem{5-gr} %4
\Au{Аншаков~О.\,М., Фабрикантова~Е.\,Ф.} ДСМ-ме\-тод автоматического порождения 
гипотез: Логические и~эпистемологические основания.~--- М.: Либроком, 2009. 432~с.

\bibitem{4-gr} %5
\Au{Poelmans J., Elzinga~P., Viaene~S., Dedene~G.} Formal concept analysis in knowledge 
discovery: A~survey~// Conceptual structures: From information to intelligence~/ Eds.\ M.~Croitoru, 
S.~Ferr$\acute{\mbox{e}}$, and D.~Lukose.~--- Lecture notes in computer science 
ser.~--- Berlin--Heidelberg: Springer, 2010. Vol.~6208.  P.~139--153.

\bibitem{6-gr}
\Au{Панкратова~Е.\,С., Финн~В.\,К.} Автоматическое по\-рож\-де\-ние гипотез в~интеллектуальных 
системах.~--- М.: Либроком, 2009. 528~с. 
\bibitem{7-gr}
\Au{Денисов А.\,А., Колесников~Д.\,Н.} Теория больших систем управления.~--- Л.: Энергоиздат, 1982. 488~с.

\bibitem{9-gr}
\Au{Грушо А.\,А., Грушо Н.\,А., Забежайло~М.\,И., Смирнов~Д.\,В., Тимонина~Е.\,Е.} 
Параметризация в~прикладных задачах поиска эмпирических причин~// Информатика и~её 
применения, 2018. Т.~12. Вып.~3. С.~62--66.

\bibitem{8-gr}
\Au{Грушо А.\,А., Грушо Н.\,А., Левыкин~М.\,В., Тимонина~Е.\,Е.} Методы идентификации 
захвата хоста в~распределенной ин\-фор\-ма\-ци\-он\-но-вы\-чис\-ли\-тель\-ной сис\-те\-ме, 
защищенной с~помощью метаданных~// Информатика и~её применения, 2018. Т.~12. Вып.~4. 
С.~41--45.

 \end{thebibliography}

 }
 }

\end{multicols}

\vspace*{-3pt}

\hfill{\small\textit{Поступила в~редакцию 03.04.19}}

%\vspace*{8pt}

%\pagebreak

\newpage

\vspace*{-28pt}

%\hrule

%\vspace*{2pt}

%\hrule

%\vspace*{-2pt}

\def\tit{ARCHITECTURAL DECISIONS IN~THE~PROBLEM 
OF~IDENTIFICATION OF~FRAUD IN~THE~ANALYSIS 
OF~INFORMATION FLOWS IN~DIGITAL ECONOMY\\[-5pt]}


\def\titkol{Architectural decisions in~the~problem 
of~identification of~fraud in~the~analysis 
of~information flows in~digital economy}

\def\aut{A.\,A.~Grusho, M.\,I.~Zabezhailo, N.\,A.~Grusho, and~E.\,E.~Timonina}

\def\autkol{A.\,A.~Grusho, M.\,I.~Zabezhailo, N.\,A.~Grusho, and~E.\,E.~Timonina}

\titel{\tit}{\aut}{\autkol}{\titkol}

\vspace*{-13pt}


 \noindent
   Institute of Informatics Problems, Federal Research Center ``Computer Sciences and 
Control'' of the Russian Academy of Sciences; 44-2~Vavilov Str., Moscow 119133, 
Russian Federation

\def\leftfootline{\small{\textbf{\thepage}
\hfill INFORMATIKA I EE PRIMENENIYA~--- INFORMATICS AND
APPLICATIONS\ \ \ 2019\ \ \ volume~13\ \ \ issue\ 2}
}%
 \def\rightfootline{\small{INFORMATIKA I EE PRIMENENIYA~---
INFORMATICS AND APPLICATIONS\ \ \ 2019\ \ \ volume~13\ \ \ issue\ 2
\hfill \textbf{\thepage}}}

\vspace*{3pt}


   
     
   \Abste{An approach to a~research of some types of fraud in digital economy with the usage of relationships of 
cause and effect is formulated. In all types of the considered frauds, the discrepancy between the 
purposes of financial transactions and actual cost of achievement of these purposes
has to be observed. Data on 
transactions can be collected by observing information flows in which these transactions are reflected. 
The architecture of data collection and their analysis can be organized by means of the distributed 
ledgers with the centralized consensus that allows creating an analog of the electronic account book 
fixing financial and economic activity of subjects of digital economy in the region. 
   The methods of fraud identification considered are based on the contradictions 
between actions described in transactions and information, which is contained in plans, standards, 
precedents, etc. 
   The method based on a~simplified scheme of implementation of the abstract project is considered. 
For identification of contradictions, it is necessary to carry out the analysis from the effect to the cause, 
i.\,e., to look for anomalies in information describing the generation of the observed effects. 
   It is shown how in implementation of the project it is possible to allocate simple ``necessary 
conditions'' of violation of cause and effect relationships, i.\,e., a~set of ``necessary conditions'' 
violation of which demonstrates fraud existence. It is possible to call this set of "necessary conditions" 
by metadata for control of the project for fraud identification.} 
   
   \KWE{digital economy; information flows; relationships of reason and effect; detection of 
fraudulent schemes}
   
  

 \DOI{10.14357/19922264190204}

\vspace*{-20pt}

 \Ack
   \noindent
   The work was partially supported by the Russian Foundation for Basic Research (projects  
18-29-03081 and 18-07-00274).



%\vspace*{6pt}

  \begin{multicols}{2}

\renewcommand{\bibname}{\protect\rmfamily References}
%\renewcommand{\bibname}{\large\protect\rm References}

{\small\frenchspacing
 {\baselineskip=10.5pt
 \addcontentsline{toc}{section}{References}
 \begin{thebibliography}{9}
\bibitem{1-gr-1}
\Aue{Grusho, A.\,A., A.\,A.~Zatsarinny, and E.\,E.~Timonina.} 2019. Blokcheyny tsifrovoy ekonomiki 
na baze sistemy situatsionnykh tsentrov i~tsentralizovannogo konsensusa [Blockchains of digital 
economy on the basis of the system of the situational centres and the centralized consensus]. 
\textit{25th Scientific and Technical Conference (International) ``Radar-Location, Navigation, 
Communication'' Proceedings}. Voronezh: VSU Publs. 6:183--191.
\bibitem{2-gr-1}
\Aue{Grusho, A., A.~Zatsarinny, and E.~Timonina.} 2019 (in press). 
A~system approach to information security 
in distributed ledgers on the situational centers platform. 
Lecture notes in computer science ser. Springer.
\bibitem{3-gr-1}
\Aue{Finn, V.\,K.} 2011. \textit{Iskusstvennyy intellekt: Metodologiya, primeneniya, filosofiya} 
[Artificial intelligence: Methodology, applications, philosophy]. Moscow: KRASAND. 448~p.

\bibitem{5-gr-1}
\Aue{Anshakov, O.\,M., and E.\,F.~Fabrikantova}. 2009. \textit{DSM-metod avtomaticheskogo porozhdeniya gipotez: Logicheskie 
i~epistemologicheskie osnovaniya} [JSM-method of automatic hypothesis generation: Logical and 
epistemological]. Moscow: KD LIBROKOM. 432~p.
\bibitem{4-gr-1} %5
\Aue{Poelmans, J., P.~Elzinga, S.~Viaene, and G.~Dedene.} 2010. Formal concept analysis in 
knowledge discovery: A~survey. \textit{Conceptual structures: From information to intelligence}. 
Eds.\ M.~Croitoru, S.~Ferr$\acute{\mbox{e}}$, and D.~Lukose. Lecture notes in 
computer science ser. Berlin--Heidelberg: Springer. 6208:139--153.

\bibitem{6-gr-1}
\Aue{Pankratov, E.\,S., and V.\,K.~Finn}. 
2009. \textit{Avtomaticheskoe porozhdenie gipotez v~intellektual'nykh 
sistemakh} [Automatic hypotheses generation in intelligent systems]. Moscow: KD 
\mbox{LIBROKOM}.  528~p. 
\bibitem{7-gr-1}
\Aue{Denisov, A.\,A., and D.\,N.~Kolesnikov.} 1982. \textit{Teoriya bol'shikh 
sistem upravleniya} [Theory of big control systems]. Leningrad: Energoizdat. 488~p.

\bibitem{9-gr-1}
\Aue{Grusho, A.\,A., N.\,A.~Grusho, M.\,I.~Zabezhailo, D.\,V.~Smirnov, and 
E.\,E.~Timonina.} 2018. 
Parametrizatsiya v~prikladnykh zadachakh poiska empiricheskikh prichin 
[Parametrization in applied 
problems of search of the empirical reasons]. 
\textit{Informatika i~ee Primeneniya~--- 
Inform. Appl.} 12(3):62--66.

\bibitem{8-gr-1}
\Aue{Grusho, A.\,A., N.\,A.~Grusho, M.\,V.~Levykin, and E.\,E.~Timonina.} 2018. Metody 
identifikatsii zakhvata khosta v~raspredelennoy informatsionno-vychislitel'noy sisteme, 
zashchishchennoy s~pomoshch'yu metadannykh [Methods of identification of host capture 
in the  distributed information system which is protected on the base of meta data].
\textit{Informatika i~ee 
Primeneniya~--- Inform. Appl.} 12(4):41--45.
{ %\looseness=1

}

\end{thebibliography}

 }
 }

\end{multicols}

\vspace*{-12pt}

\hfill{\small\textit{Received April 3, 2019}}

%\pagebreak

%\vspace*{-18pt}

\Contr

\noindent
\textbf{Grusho Alexander A.} (b.\ 1946)~--- Doctor of Science in physics and 
mathematics, professor, principal scientist, Institute of Informatics Problems, 
Federal Research Center ``Computer Sciences and Control'' of the Russian 
Academy of Sciences; 44-2~Vavilov Str., Moscow 119133, Russian Federation; 
\mbox{grusho@yandex.ru} 

\vspace*{3pt}

\noindent
\textbf{Zabezhailo Michael I.} (b.\ 1956)~--- Doctor of Science in physics and 
mathematics, principal scientist, Institute of Informatics Problems, Federal Research 
Center ``Computer Sciences and Control'' of the Russian Academy of Sciences;  
44-2~Vavilov Str., Moscow 119133, Russian Federation; 
\mbox{m.zabezhailo@yandex.ru} 

\vspace*{3pt}


\noindent
\textbf{Grusho Nikolai A.} (b.\ 1982)~--- Candidate of Science (PhD) in physics 
and mathematics, senior scientist, Institute of Informatics Problems, Federal 
Research Center ``Computer Sciences and Control'' of the Russian Academy of 
Sciences; 44-2~Vavilov Str., Moscow 119133, Russian Federation; 
\mbox{info@itake.ru} 

\vspace*{3pt}


\noindent
\textbf{Timonina Elena E.} (b.\ 1952)~--- Doctor of Science in technology, 
professor, leading scientist, Institute of Informatics Problems, Federal Research 
Center ``Computer Sciences and Control'' of the Russian Academy of Sciences;  
44-2~Vavilov Str., Moscow 119133, Russian Federation; 
\mbox{eltimon@yandex.ru} 

\label{end\stat}

\renewcommand{\bibname}{\protect\rm Литература}   %8
\def\stat{dukova}

\def\tit{О ПОИСКЕ МАКСИМАЛЬНЫХ ЧАСТЫХ И~МИНИМАЛЬНЫХ НЕЧАСТЫХ НАБОРОВ ПРОИЗВЕДЕНИЯ ЧАСТИЧНЫХ ПОРЯДКОВ}

\def\titkol{О поиске максимальных частых и~минимальных нечастых наборов произведения частичных порядков}

\def\aut{Н.\,А.~Драгунов$^1$, Е.\,В.~Дюкова$^2$}

\def\autkol{Н.\,А.~Драгунов, Е.\,В.~Дюкова}

\titel{\tit}{\aut}{\autkol}{\titkol}

\index{Драгунов Н.\,А.}
\index{Дюкова Е.\,В.}
\index{Dragunov N.\,A.}
\index{Djukova E.\,V.}


%{\renewcommand{\thefootnote}{\fnsymbol{footnote}} \footnotetext[1]
%{Работа выполнена при поддержке Министерства науки и~высшего образования Российской Федерации (проект 
%075-15-2020-799).}}


\renewcommand{\thefootnote}{\arabic{footnote}}
\footnotetext[1]{Федеральный исследовательский центр <<Информатика 
и~управ\-ле\-ние>> Российской академии наук, \mbox{nikitadragunovjob@gmail.com}}
\footnotetext[2]{Федеральный исследовательский центр <<Информатика и~управ\-ле\-ние>> 
Российской академии наук, \mbox{edjukova@mail.ru}}

\vspace*{-3pt}




\Abst{Исследованы актуальные вопросы снижения временных затрат, возникающие при 
логическом анализе данных с~элементами из декартова произведения конечных час\-тич\-но 
упорядоченных множеств. Для задачи поиска по базе транзакций максимальных час\-тых и~минимальных 
нечастых наборов произведения час\-тич\-ных порядков предложен оригинальный метод, 
основанный на решении слож\-ной дискретной задачи, называемой дуализацией 
над произведением час\-тич\-ных порядков. Метод представляет собой синтез двух других 
известных методов, один из которых достаточно очевиден, а~другой использует идею 
инкрементального пе\-ре\-чис\-ле\-ния искомых наборов и~поэтому пред\-став\-ля\-ет 
в~основном тео\-ре\-ти\-че\-ский интерес. Проведено экспериментальное исследование предложенного 
подхода к~решению рас\-смат\-ри\-ва\-емой задачи в~случае произведения конечных цепей,
 выявлены условия его эф\-фек\-тив\-ности и~для проводимого анализа данных показана 
 це\-ле\-со\-об\-раз\-ность применения асимптотически оптимальных алгоритмов дуализации 
 над произведением час\-тич\-ных порядков.}

\KW{максимальные час\-тые наборы; минимальные не\-час\-тые наборы; дуализация над 
произведением час\-тич\-ных порядков; асимп\-то\-ти\-чески оптимальный алгоритм дуализации}

\DOI{10.14357/19922264220112}
  
%\vspace*{-4pt}


\vskip 10pt plus 9pt minus 6pt

\thispagestyle{headings}

\begin{multicols}{2}

\label{st\stat}

    \section{Введение}
    
    Рас\-смат\-ри\-ва\-емая задача анализа данных занимает важ\-ное мес\-то в~об\-ласти 
    информационного поиска и~в~случае бинарных данных ставится сле\-ду\-ющим образом~\cite{4}.
    
    Дано некоторое множество элементов~$V$. Подмножества $X \hm\subseteq V$ называются наборами. Пусть~$D$~--- 
    база данных, содержащая некоторые, не обязательно различные, наборы. Наборы, 
    содержащиеся в~$D$, называются транз\-ак\-ци\-ями. Под частотой набора~$\nu(X)$ понимается доля транз\-ак\-ций в~$D$, 
    содержащих~$X$. Если $\nu(X) \hm\geq s$, $s \hm\in \left[0, 1\right]$, то набор~$X$ называется $s$-час\-тым, 
    иначе он называется $s$-не\-час\-тым. Если набор частый и~он не содержится ни в~каком другом 
    час\-том наборе, то такой набор называется максимальным час\-тым. Если набор не\-час\-тый 
    и~при этом он не содержит в~себе никакого другого не\-час\-то\-го набора, то такой набор 
    называется минимальным нечастым. Требуется найти все максимальные час\-тые и~минимальные не\-час\-тые 
    наборы при заданном~$s$.
    
    Рас\-смат\-ри\-ва\-емая задача имеет много важных приложений, одним из которых является 
    нахождение ассоциативных правил в~базах данных. В~случае бинарных данных ассоциативное правило~---
     это упорядоченная пара $ \left( X, Y \right)$ непересекающихся подмножеств множества~$V$, обо\-зна\-ча\-емая 
     $X \hm\Rightarrow Y$. Поддержкой правила $X \hm\Rightarrow Y$ называется час\-то\-та набора $Z\hm = X \cup Y$.
      Достоверностью правила $X\hm \Rightarrow Y$ называется доля транзакций, со\-дер\-жа\-щих~$Y$, 
      среди всех транзакций, содержащих~$X$. Требуется \mbox{найти} все ассоциативные правила, 
      удовле\-тво\-ря\-ющие заданным минимальной поддержке $s\hm \in [0, 1]$ и~минимальной 
      достоверности $c \hm\in [0, 1]$.  Впервые задача нахождения ассоциативных правил
       была поставлена в~\cite{1}, где она формулировалась как задача анализа по\-тре\-би\-тель\-ской корзины.

    В случае небинарных данных каждый элемент из~$V$ имеет некоторое множество чис\-ло\-вых значений 
    и~вместо наборов элементов рас\-смат\-ри\-ва\-ют\-ся наборы их значений.

    Поиск ассоциативных правил осуществляется в~два этапа. 
    На первом этапе находятся частые наборы, на втором этапе из найденных час\-тых 
    наборов формируются ассоциативные правила. При формировании правил на втором 
    этапе фактически возникает задача поиска $t$-не\-час\-тых наборов, где $t\hm > s/c$.
    
    С ростом размерности современных баз данных находить все час\-тые и~не\-час\-тые 
    наборы становится неэффективно как по времени, так и~по памяти в~силу 
    экспоненциального рос\-та чис\-ла таких наборов. Одно из решений данной проблемы 
    заключается в~поиске только максимальных час\-тых наборов и~только минимальных 
    нечастых наборов, что позволяет компактно хранить информацию о~всех час\-тых и~не\-час\-тых 
    наборах соответственно. 
    
    
    В~\cite{9} рас\-смот\-ре\-на задача поиска множеств максимальных час\-тых наборов~$X_{\max}$ 
    и~минимальных не\-час\-тых наборов~$Y_{\min}$ в~данных, пред\-став\-лен\-ных в~виде декартова 
    произведения час\-тич\-но упорядоченных множеств. Показано, что в~этом случае 
    при построении тре\-бу\-емых наборов возникают соответственно задача поиска 
    максимальных независимых элементов час\-тич\-ных порядков и~задача поиска минимальных 
    независимых элементов час\-тич\-ных порядков.  Каж\-дая из этих задач называется 
    дуализацией над произведением час\-тич\-ных порядков~\cite{8}. Обе задачи относятся к~одним 
    из цент\-раль\-ных труд\-но\-ре\-ша\-емых пе\-ре\-чис\-ли\-тель\-ных задач дис\-крет\-ной математики.
    
    Существует достаточно очевидный способ поиска максимальных час\-тых и~минимальных
     не\-час\-тых наборов произведения час\-тич\-ных порядков, основанный на по\-сле\-до\-ва\-тель\-ном 
     по\-стро\-ении указанных множеств. Одно из множеств ищется, например, алгоритмом Apriori~\cite{2},
      второе множество получается путем дуализации первого. 
      В~настоящей работе показано, что метод эффективен только в~случае, когда чис\-ло час\-тых 
      наборов существенно меньше или, наоборот, существенно больше чис\-ла не\-час\-тых наборов. 
      В~\cite{9} предложена идея со\-вмест\-но\-го пе\-ре\-чис\-ле\-ния~$X_{\max}$ и~$Y_{\min}$ с~использованием
       инкрементального алгоритма дуализации из~\cite{14}, которая автором экспериментально 
       не исследована.
    
    Основной результат настоящей работы~--- разработка нового подхода к~решению 
    поставленной задачи, который является синтезом последовательного и~совместного подходов. 
    
    Экспериментальные исследования, проведенные в~настоящей работе для случая
     произведения цепей, свидетельствуют о~том, что предложенный по\-сле\-до\-ва\-тель\-но-со\-вмест\-ный 
     метод наиболее эффективен в~случае, когда мощ\-ность множества час\-тых наборов примерно 
     равна мощ\-ности множества не\-час\-тых наборов.
     
     \vspace*{-6pt}
     
    
    \section{Постановка задачи поиска максимальных частых 
    и~минимальных нечастых наборов произведения частичных порядков}
    
         \vspace*{-2pt}
    
    Пусть $\mathcal{P} = \mathcal{P}_1 \times \dots \times \mathcal{P}_n$~--- 
    де\-кар\-то\-во произведение час\-тич\-но упорядоченных множеств. Элементы~$\mathcal{P}$ называются наборами. 
    На множестве~$\mathcal{P}$ определяется отношение частичного порядка~$\preceq$ сле\-ду\-ющим образом: 
    если $p \hm= (p_1, \dots, p_n) \hm\in \mathcal{P}$ и~$q \hm= (q_1, \dots, q_n)\hm \in \mathcal{P}$, 
    то $ p \hm\preceq q$ в~$ \mathcal{P}\hm \Leftrightarrow p_1 \hm\preceq q_1$ 
    в~$\mathcal{P}_1, \dots, p_n \hm\preceq q_n$ в~$ \mathcal{P}_n$.
    
    Пусть $\mathcal{D} (\mathcal{P})$~--- некоторая со\-во\-куп\-ность
     наборов из~$\mathcal{P}$, называемая базой данных. Наборы, на\-хо\-дя\-щи\-еся в~базе 
     данных $\mathcal{D} (\mathcal{P})$, необязательно по\-пар\-но раз\-лич\-ны и~называются транзакциями. 
     
    Введем обозначения: 
    $\vert \mathcal{D} (\mathcal{P}) \vert$~--- чис\-ло транз\-ак\-ций в~$\mathcal{D} (\mathcal{P})$; 
    $\mathcal{S}_\mathcal{D}(p)$~--- число транз\-ак\-ций в~$\mathcal{D} (\mathcal{P})$, 
    сле\-ду\-ющих за $p \hm\in \mathcal{P}$; $s \hm\in [0, 1]$. 
    
    \smallskip
    
    \noindent
    \textbf{Определение~1.}\
     Набор $p \in \mathcal{P}$ называется $s$-час\-тым, 
     если $\mathcal{S}_\mathcal{D}(p) / \vert \mathcal{D} (\mathcal{P}) \vert \hm\geq s$. Иначе набор~$p$ 
     называется $s$-не\-час\-тым.
    
    \smallskip
    
    \noindent
    \textbf{Определение~2.}\
    Набор $p \in \mathcal{P}$ называется максимальным $s$-час\-тым, если 
    он $s$-час\-тый и~никакой сле\-ду\-ющий за ним набор~$z$, $z\hm \neq p$, не является $s$-час\-тым.

    
    \smallskip
    
    \noindent
    \textbf{Определение~3.}\
    Набор $p \in \mathcal{P}$ называется минимальным $s$-не\-час\-тым, если он $s$-не\-час\-тый 
    и~никакой пред\-шест\-ву\-ющий ему набор~$z$, $z \hm\neq p$, не является $s$-не\-час\-тым.


\smallskip
    
    Далее вместо $s$-частый ($s$-не\-час\-тый) набор будем писать час\-тый (не\-час\-тый) набор. 
    Множество всех максимальных час\-тых наборов будем обозначать как $X_{\max}$, 
    а~множество всех минимальных не\-час\-тых наборов как $Y_{\min}$.
    
    Пусть $R \subset \mathcal{P}$, $R^+\hm = \{ x \in \mathcal{P} \vert \exists\, a \hm\in R, a \hm\preceq x \}$, 
    $R^- \hm= \{ x \hm\in \mathcal{P} \vert \exists\, a \hm\in R, x \hm\preceq a \}$.


    \noindent
    \textbf{Определение~4.}\
     Множество $I(R^+)$, со\-сто\-ящее из всех максимальных элементов множества~$\mathcal{P} \setminus R^+$, 
     называется максимальным независимым от~$R$.

\smallskip


   \noindent
    \textbf{Определение~5.}\
     Множество $I(R^-)$, со\-сто\-ящее из всех минимальных элементов множества~$\mathcal{P} \setminus R^-$, 
     называется минимальным независимым от~$R$.

\smallskip
    
    Каждая из задач построения $I(R^+)$ и~$I(R^-)$ 
    при заданном множестве~$R$ называется задачей дуализации над произведением час\-тич\-ных порядков.
    
    \smallskip

    \noindent
    \textbf{Утверждение~1.}\
    Если $X \hm\subset X_{\max}$, а~$y \hm\in I(X^-)$~--- не\-час\-тый набор, 
    то~$y$~--- минимальный не\-час\-тый набор.

\smallskip    
    
    \noindent
    Д\,о\,к\,а\,з\,а\,т\,е\,л\,ь\,с\,т\,в\,о\,.\  \ 
    Пусть $y \hm\notin I(X_{\max}^-)$. Так как~$y$~--- 
    нечастый набор, то в~$\mathcal{P} \setminus X^{-}_{\max}$ найдется минимальный не\-час\-тый набор~$x$ 
    такой, что $x\hm \neq y$ и~$x \hm\preceq y$. Из того, что $\mathcal{P} \setminus X^{-}_{\max} 
    \hm\subseteq \mathcal{P} \setminus X^-$, следует, что $x\hm \in \mathcal{P} \setminus X^-$, 
    что противоречит условию $y \hm\in I(X^-)$.

\smallskip

\noindent
\textbf{Утверждение~2.}\
    Пусть $X \hm\subseteq X_{\max}$, $Y\hm \subseteq Y_{\min}$. 
    Тогда $I(X^-) \hm= Y$ в~том и~только в~том случае, когда $X \hm= X_{\max}$ и~$Y \hm= Y_{\min}$.


\smallskip


  \noindent
    Д\,о\,к\,а\,з\,а\,т\,е\,л\,ь\,с\,т\,в\,о\,.\  \
    Пусть $X\! \subset\! X_{\max}, x \hm\in X_{\max}\!\setminus\!X$.
     Так как множество~$X_{\max}$~--- антицепь, то $x \hm\notin X^-$. 
     Следовательно, $x \hm\in \mathcal{P} \setminus X^{-}$.
      Но тогда существует элемент $ q \hm\in I(X^-) : q \preceq x$, 
      который является час\-тым. Однако во множестве~$Y$ частых наборов нет; следовательно, $I(X^-) \hm\neq Y$. 
      Если же $X \hm= X_{\max}$, то $I(X^-) \hm= Y_{\min}$. Таким образом, $I(X^-) \hm= Y$ тогда и~только
       тогда, когда $X \hm= X_{\max}$ и~$Y\hm = Y_{\min}$.


    
    \section{Методы построения множеств~$X_{\max}$ и~$Y_{\min}$}

    \subsection{Последовательное перечисление $X_{\max}$~и~$Y_{\min}$}

    Достаточно очевиден поиск~$X_{\max}$ и~$Y_{\min}$ при заданной $\mathcal{D} (\mathcal{P})$ 
    путем последовательного по\-стро\-ения множеств~$X_{\max}$ и~$Y_{\min}$. 
    Данный поиск осуществляется в~два этапа. На первом этапе находятся все максимальные частые 
    наборы~$X_{\max}$, например алгоритмом Apriori~\cite{2}. На втором этапе  используется свойство 
    двойственности $I \left(X_{\max}^- \right)\hm = Y_{\min}$. 
    Минимальные нечастые наборы~$Y_{\min}$ находятся путем дуализации найденного на первом этапе 
    множества~$X_{\max}$. Аналогично можно сначала искать~$Y_{\min}$ алгоритмом Apriori, а~затем 
    искать~$X_{\max}$ путем дуализации~$Y_{\min}$.

    Очевидно, что данный подход будет проявлять себя наилучшим образом в~случаях, когда 
    алгоритм Apriori или его модификации могут найти одно из искомых множеств существенно
     быст\-рее, чем другое множество, например когда мощ\-ность~$X_{\max}$ 
     существенно меньше (больше) мощ\-ности~$Y_{\min}$.
    
    \subsection{Совместное перечисление $X_{\max}$ и~$Y_{\min}$}

    В~\cite{9} предложена идея совместного перечисления множеств~$X_{\max}$ и~$Y_{\min}$. 
    На первом шаге рас\-смат\-ри\-ва\-ет\-ся некоторый случайный набор $q \hm\in \mathcal{P}$. Если $q$~--- 
    час\-тый набор, то ищется максимальный час\-тый набор, сле\-ду\-ющий за~$q$, 
    который пополняет множество $X \hm\subseteq X_{\max}$. Если $q$~---
     не\-час\-тый набор, то ищется минимальный не\-час\-тый набор, пред\-шест\-ву\-ющий~$q$, 
     который пополняет множество $Y \hm\subseteq Y_{\min}$. Пусть на шаге~$i$ ($i\hm \geq 1$) 
     построены множества $X \hm\subseteq X_{\max}$ и~$Y \hm\subseteq Y_{\min}$. Если $X \hm\neq \varnothing$, 
     $Y \hm= \varnothing$, то ищется набор~$q$ такой, что $q \hm\npreceq x, \forall x \hm\in X$. Если 
     $X \hm= \varnothing$, $Y \hm\neq \varnothing$, то ищется набор~$q$ такой, что 
     $q \hm\nsucceq y, \forall y \hm\in Y$. Если же и~$X \hm\neq \varnothing$, и~$Y \hm\neq \varnothing$, 
     то ищется набор~$q$ такой, что $q \hm\npreceq x, \forall x \hm\in X, q \hm\nsucceq y, \forall y \hm\in Y$.
      Затем, аналогично первому шагу, находится максимальный частый или минимальный нечастый набор. 
      Однако в~\cite{9} идея совместного перечисления искомых множеств экспериментально 
      не исследована и~не предложены конкретные указания по воз\-мож\-ной ее реализации.
    
    Алгоритм, основанный на совместном пе\-ре\-чис\-ле\-нии множеств~$X_{\max}$ и~$Y_{\min}$,
     реализован в~на\-сто\-ящей работе. Алгоритм строит две последовательности: $X_1 \hm\subset X_2 
     \subset \dots \subset X_{\max}$, $Y_1\hm \subset Y_2 \subset \dots \subset Y_{\min}$. 
     На первом шаге $X_1 \hm= \{x\}$, $Y_1 \hm= \{y\}$, где~$x$ и~$y$ ищутся алгоритмом Apriori.
      На шаге $i \hm+ 1$ ($i\hm \geq 1$) строится либо~$I(X^{-}_{i})$, либо~$I(Y^{+}_{i})$. Пусть на 
      шаге $i \hm+ 1$ ($i \hm\geq 1$) построено множество~$I(X^{-}_{i})$. 
      Согласно утверждениям~1 и~2, множество~$I(X^{-}_{i})$ либо не содержит час\-тых наборов 
      и~совпадает с~множеством~$Y_{\min}$ (в~этом случае $X_i \hm= X_{\max}$ 
      и~алгоритм заканчивает работу), либо~$I(X^{-}_{i})$ содержит как час\-тые, так и~не\-час\-тые наборы. 
      Каждый нечастый набор из~$I(X^{-}_{i})$ является минимальным не\-час\-тым и~пополняет множество~$Y_{i}$, 
      формируя в~результате множество~$Y_{i+1}$. Для каждого час\-то\-го набора находится один содержащий 
      его максимальный час\-тый набор путем последовательного увеличения текущего 
      частого набора в~лексикографическом порядке, который пополняет множество~$X_{i}$, 
      формируя в~результате множество~$X_{i+1}$.
      
    В~экспериментальной части работы (см.\ разд.~4) рас\-смот\-рен случай произведения цепей. 
    Задача дуализации решается с~помощью асимптотически оптимального алгоритма дуализации
     цепей \mbox{RUNC-M}+~\cite{7}. Асимптотически оптимальные алгоритмы дуализации 
     являются лидерами по ско\-рости счета~\cite{6}.

    Очевидно, что время работы совместного алгоритма в~основном зависит от чис\-ла
     минимальных не\-час\-тых и~максимальных час\-тых наборов. На\linebreak каж\-дой новой 
     итерации происходит дуализация\linebreak все б$\acute{\mbox{о}}$льших по мощ\-ности множеств~$X$ или~$Y$.\linebreak 
     Если число итераций становится достаточно\linebreak большим, то ско\-рость работы совместного 
     перечисления существенно снижается, что делает его практически неприменимым для 
     задач большой раз\-мер\-ности.
     { %\looseness=1
     
     }

    \subsection{Последовательно-совместное перечисление~$X_{\max}$ и~$Y_{\min}$}

    Предлагается следующий итеративный метод, который синтезирует идеи последовательного
     и~совместного методов, описанных выше. Положим $X_0 \hm= \varnothing$. 
     Строится одна по\-сле\-до\-ва\-тель\-ность $X_1 \hm\subset X_2 \hm\subset \dots \subset X_{\max}$. 
     На первом шаге $X_1\hm = \{x\}$, где $x$ ищется алгоритмом Apriori. На шаге $i \hm+ 1$ ($i \hm\geq 1$) 
     решается задача дуализации множества $X_{i} \setminus X_{i-1}$.

    
    
   \setcounter{figure}{1}
    \begin{figure*}[b] %fig2
  \vspace*{12pt}
  \begin{center}  
    \mbox{%
\epsfxsize=163mm
\epsfbox{duk-2.eps}
}

\end{center}
\vspace*{-9pt}
    \Caption{Зависимость времени работы алгоритмов от суммы мощностей множеств~$X_{\max}$ и~$Y_{\min}$ 
    для случая~1~(\textit{а}) и~2~(\textit{б}):
    \textit{1}~--- по\-сле\-до\-ва\-тель\-но-со\-вмест\-ный;
    \textit{2}~--- последовательный; \textit{3}~--- совместный; \textit{4}~--- Apriori}
    \label{12}
    \end{figure*}
     
    Пусть множество~$D$ есть результат дуализации $X_{i} \hm\setminus X_{i-1}$. Согласно утверждению~1, 
    множество~$D$ содержит частые наборы. Для каждого час\-то\-го набора из~$D$ 
    находится один содержащий его максимальный час\-тый набор путем последовательного 
    увеличения текущего час\-то\-го набора в~лексикографическом порядке. Все найденные максимальные
     частые наборы, которых нет в~множестве~$X_{i}$, до\-бав\-ля\-ют\-ся к~$X_{i}$, 
     и~таким образом формируется~$X_{i+1}$. Если же все найденные частые наборы уже содержатся в~$X_{i}$, 
     то решается задача дуализации множества~$X_{i}$. Если в~$I(X^{-}_{i})$ нет частых наборов, 
     то $I(X^{-}_{i})\hm = Y_{\min}$, $X_i \hm= X_{\max}$ и~алгоритм завершает работу. 
     Иначе для каждого частого набора из~$I(X^{-}_{i})$ находится один содержащий его максимальный 
     час\-тый набор, который пополняет множество~$X_{i}$, формируя в~результате множество~$X_{i+1}$.

    \section{Экспериментальное исследование}
    
    Рас\-смат\-ри\-вал\-ся случай данных, пред\-став\-лен\-ных в~виде произведения цепей мощ\-ности~5. 
    Для\linebreak таких данных проводился поиск максимальных час\-тых и~минимальных нечастых 
    наборов сле\-ду\-ющи\-ми методами: алгоритмом Apriori, модифицированным для случая 
    цепей; последовательным \mbox{методом}; совместным методом; по\-сле\-до\-ва\-тель\-но-со\-вмест\-ным методом.
    
    Все методы реализованы на языке Python~3. 
    Задача дуализации решалась алгоритмом дуализации цепей RUNC-M+~\cite{7}. 
    Эксперименты проведены на случайных базах данных различной раз\-мер\-ности. 
    Можно выделить два сле\-ду\-ющих случая соотношения мощностей множеств всех час\-тых и~не\-час\-тых наборов.
    \begin{description}
    \item[Случай 1:] мощ\-ность множества частых наборов примерно рав\-на мощ\-ности множества нечастых наборов.
    \item[Случай 2:] мощ\-ность множества частых наборов существенно меньше (больше) мощ\-ности множества 
    не\-час\-тых наборов.
    \end{description}
    
    Описанные случаи схематично изображены на рис.~1. 

    Графики зависимости времени работы тестируемых методов 
    от мощ\-ности множеств~$X_{\max}$ и~$Y_{\min}$ приведены на рис.~2.
    
    

    

    Нетрудно видеть, что в~случае~1 лучше работает по\-сле\-до\-ва\-тель\-но-со\-вмест\-ный алгоритм: 
    множества час\-тых и~не\-час\-тых наборов имеют примерно одинаковую мощ\-ность, 
    поэтому быст\-рее будет обрабатывать их по\-сле\-до\-ва\-тель\-но-со\-вмест\-ным методом. В~случае~2 
    быст\-рее работает последовательный алгоритм: быст\-рее найти множество максимальных час\-тых наборов, 
    обработав множество час\-тых наборов, и~дуализировать результат. Время поиска множеств~$X_{\max}$ 
    и~$Y_{\min}$ совместным методом и~модифицированным алгоритмом Apriori рас\-тет существенно 
    быст\-рее времени поиска по\-сле\-до\-ва\-тель\-но-со\-вмест\-ным методом в~обоих случаях.
    
    { \begin{center}  %fig1
 \vspace*{9pt}
    \mbox{%
\epsfxsize=67.963mm
\epsfbox{duk-1.eps}
}

\end{center}

\noindent
{{\figurename~1}\ \ \small{
Два случая соотношения мощностей множеств час\-тых и~не\-час\-тых наборов
}}}

%\vspace*{6pt}


    \section{Заключение}
    
Рас\-смот\-ре\-на задача поиска максимальных час\-тых и~минимальных не\-час\-тых наборов в~данных, 
представленных в~виде декартова произведения час\-тич\-ных порядков. Актуальны вопросы 
снижения временн$\acute{\mbox{ы}}$х затрат, возникающих при реализации методов нахождения искомых наборов.
 Разработан новый подход к~по\-стро\-ению множества максимальных частых наборов~$X_{\max}$ и~множества 
 минимальных не\-час\-тых наборов~$Y_{\min}$, пред\-став\-ля\-ющий собой синтез двух ранее известных 
 подходов: последовательного и~со\-вмест\-но\-го (первый достаточно очевиден, идея второго предложена в~\cite{9}). 
 Сложность последовательного, совместного и~пред\-ла\-га\-емо\-го по\-сле\-до\-ва\-тель\-но-со\-вмест\-но\-го поиска 
 обуслов\-ле\-на, в~том чис\-ле, не\-об\-хо\-ди\-мостью рас\-смат\-ри\-вать в~процессе поиска 
 труд\-но\-ре\-ша\-емую пе\-ре\-чис\-ли\-тель\-ную задачу дис\-крет\-ной математики, на\-зы\-ва\-емую дуализацией 
 над произведением час\-тич\-ных порядков.

Для случая, когда данные пред\-став\-ле\-ны в~виде произведения конечных цепей, 
приведены результаты экспериментального срав\-не\-ния названных подходов, а~так\-же независимого 
способа \mbox{по\-стро\-ения} множеств~$X_{\max}$ и~$Y_{\min}$, не тре\-бу\-юще\-го решения задачи дуализации. 
Эксперименты проводились на модельных задачах с~применением асимптотически оптимального
 алгоритма дуализации над произведением конечных цепей \mbox{RUNC-M}+~\cite{7}. 
 Результаты исследования свидетельствуют о~том, что по\-сле\-до\-ва\-тель\-но-со\-вмест\-ный 
 метод наиболее эффективен (требует меньших временн$\acute{\mbox{ы}}$х затрат по сравнению с~другими рас\-смот\-рен\-ны\-ми 
 методами) в~случае, когда мощ\-ность множества час\-тых наборов примерно равна мощ\-ности множества
  нечастых наборов. Иначе выигрывает последовательный поиск. Наихудшие показатели 
  у~независимого пе\-ре\-чис\-ле\-ния множеств~$X_{\max}$ и~$Y_{\min}$ с~использованием в~качестве
   базового алгоритма Apriori~\cite{2}, точ\-нее его модификации на тес\-ти\-ру\-емый случай. 
   Таким образом, показана це\-ле\-со\-об\-раз\-ность применения алгоритмов дуализации для 
   по\-стро\-ения множеств~$X_{\max}$ и~$Y_{\min}$.

  
  {\small\frenchspacing
 {%\baselineskip=10.8pt
 %\addcontentsline{toc}{section}{References}
 \begin{thebibliography}{9}  
    \bibitem{4}
    \Au{Aggarwal C.} 
    Frequent pattern mining.~--- Heidelberg: Springer, 2014. 467~p.
    
    \bibitem{1}
    \Au{Agrawal~R., Imielinski~T., Swami~A.} Mining association rules 
    between sets of items in large databases~// \mbox{SIGMOD} Conference (International) on Management of Data
    Proceedings.~--- New York, NY, USA: ACM, 1993. P.~207--216.
    
    \bibitem{9}
    \Au{Elbassioni K.} On finding minimal infrequent elements in multi-dimensional 
    data defined over partially ordered sets~// arXiv.org, 2014. 30~p. arXiv:1411.2275 [cs.DB].
    
    \bibitem{8}
    \Au{Elbassioni K.} Algorithms for dualization over products of partially 
    ordered sets~// SIAM J.~Discrete Math., 2009. Vol.~23. Iss.~1. P.~487--510.
    
    \bibitem{2}
    \Au{Agrawal R., Srikant~R.} 
    Fast algorithms for mining association rules in large databases~// 
    20th Conference (International) on Very Large Data Bases Proceedings.~--- San Francisco, CA, USA: 
    Morgan Kaufmann Publs. Inc., 1994. P.~487--499.
    
    \bibitem{14}
    \Au{Хачиян Л.\,Г.} Избранные труды.~--- М.: МЦНМО, 2009. 520~с.
    
    \bibitem{7}
    \Au{Дюкова Е.\,В., Масляков~Г.\,О., Прокофьев~П.\,А.} 
    О~дуализации над произведением частичных порядков~// Машинное обучение и~анализ данных, 2017. Т.~3. №\,4.  
    C.~239--249.
    
    \bibitem{6}
    \Au{Дюкова Е.\,В., Прокофьев~П.\,А.} Об асимптотически оптимальных алгоритмах дуализации~// 
    Ж.~вычисл. матем. и~матем. физ., 2015. Т.~55. №\,5. С.~895--910.
    \end{thebibliography}

 }
 }

\end{multicols}

\vspace*{-6pt}

\hfill{\small\textit{Поступила в~редакцию 15.01.21}}

\vspace*{8pt}

%\pagebreak

%\newpage

%\vspace*{-28pt}

\hrule

\vspace*{2pt}

\hrule

%\vspace*{-2pt}

\def\tit{FINDING MAXIMAL FREQUENT AND~MINIMAL INFREQUENT SETS IN~PARTIALLY ORDERED DATA}


\def\titkol{Finding maximal frequent and~minimal infrequent sets in~partially ordered data}


\def\aut{N.\,A.~Dragunov and E.\,V.~Djukova}

\def\autkol{N.\,A.~Dragunov and E.\,V.~Djukova}

\titel{\tit}{\aut}{\autkol}{\titkol}

\vspace*{-11pt}


\noindent
Federal Research Center ``Computer Science and Control'' 
of the Russian Academy of Sciences, 44-2~Vavilov Str., Moscow 119333, Russian Federation

\def\leftfootline{\small{\textbf{\thepage}
\hfill INFORMATIKA I EE PRIMENENIYA~--- INFORMATICS AND
APPLICATIONS\ \ \ 2022\ \ \ volume~16\ \ \ issue\ 1}
}%
 \def\rightfootline{\small{INFORMATIKA I EE PRIMENENIYA~---
INFORMATICS AND APPLICATIONS\ \ \ 2022\ \ \ volume~16\ \ \ issue\ 1
\hfill \textbf{\thepage}}}

\vspace*{3pt} 


\Abste{Relevant issues of time costs reducing in the logical analysis of data with elements 
from the Cartesian product of finite partially ordered sets are investigated. 
An original method based on solving a complex discrete problem called dualization
 over the product of partial orders is proposed for the problem of finding maximal 
 frequent and minimal infrequent sets in the transaction database. The proposed method 
 is a~synthesis of two other known methods, one of which is quite obvious and the other uses 
 the idea of an incremental enumeration of target\linebreak\vspace*{-12pt}}
 
 \Abstend{sets and is, therefore, mainly 
 of theoretical interest. An experimental study of the considered approaches in
  the case of the product of finite chains is carried out and conditions for
   their effectiveness are revealed. The expediency of applying 
asymptotically optimal dualization algorithms over the product of partial orders is shown.}

\KWE{maximal frequent sets; minimal infrequent sets; dualization over the product of 
partial orders; asymptotically optimal dualization algorithm}

\DOI{10.14357/19922264220112}

%\vspace*{-16pt}

%\Ack
%\noindent




%\vspace*{6pt}

  \begin{multicols}{2}

\renewcommand{\bibname}{\protect\rmfamily References}
%\renewcommand{\bibname}{\large\protect\rm References}

{\small\frenchspacing
 {%\baselineskip=10.8pt
 \addcontentsline{toc}{section}{References}
 \begin{thebibliography}{9}
\bibitem{1-dr}
\Aue{Aggarwal, C.} 2014. \textit{Frequent pattern mining}. Heidelberg: Springer. 467~p.
\bibitem{2-dr}
\Aue{Agrawal, R., T.~Imielinski, and A.~Swami.}
 1993. Mining association rules between sets of items in large databases. 
 \textit{SIGMOD  Conference (International) on Management of Data Proceedings}. New York, NY:
 ACM. 207--216. 
\bibitem{3-dr}
\Aue{Elbassioni, K.}
 2014. On finding minimal infrequent elements in multidimensional data defined over partially ordered sets. 
 arXiv.org. 30~p. Available at: 
 {\sf https://arxiv.org/\linebreak pdf/1411.2275.pdf} (accessed January~25, 2022).
\bibitem{4-dr}
\Aue{Elbassioni, K.} 2009. Algorithms for dualization over products of partially ordered sets. 
\textit{SIAM J.~Discrete Math.} 23(1):487--510.
\bibitem{5-dr}
\Aue{Agrawal, R., and R.~Srikant.}
 1994. Fast algorithms for mining association rules in large databases. 
 \textit{20th Conference (International) on Very Large Data Bases Proceedings}.
 San Francisco, CA: 
    Morgan Kaufmann Publs. Inc.  487--499.
\bibitem{6-dr}
\Aue{Khachiyan, L.\,G.} 2009. \textit{Izbrannye trudy} [Selected works]. Moscow: MCCME. 520~p.
\bibitem{7-dr}
\Aue{Djukova, E.\,V., G.\,O.~Maslyakov, and P.\,A.~Prokofyev.} 
2017. O~dualizatsii nad proizvedeniem chastichnykh poryadkov [On dualization over the product of 
partial orders]. \textit{Mashinnoe obuchenie i~analiz dannykh} [J.~Machine Learning Data Analysis] 
3(4):239--249.
\bibitem{8-dr}
\Aue{Djukova, E.\,V., and P.\,A.~Prokofyev.}
 2015. Asymptotically optimal dualization algorithms. \textit{Comp. Math.
 Math. Phys.} 55(5):891--905. 
 
 \end{thebibliography}

 }
 }

\end{multicols}

\vspace*{-6pt}

\hfill{\small\textit{Received January 15, 2021}}

%\pagebreak

%\vspace*{-18pt}

\Contr

\noindent
\textbf{Dragunov Nikita A.} (b.\ 1997)~--- 
PhD student, Federal Research Center ``Computer Science and Control'' 
of the Russian Academy of Sciences, 44-2~Vavilov Str., Moscow 119333, Russian Federation; 
\mbox{nikitadragunovjob@gmail.com}

\vspace*{3pt}

\noindent
\textbf{Djukova Elena V.} (b.\ 1945)~--- 
Doctor of Science in physics and mathematics, principal scientist, Federal Research Center
``Computer Science and Control'' of the Russian Academy of Sciences, 44-2~Vavilov Str., Moscow 119333, 
Russian Federation; \mbox{edjukova@mail.ru}




\label{end\stat}

\renewcommand{\bibname}{\protect\rm Литература}    %9

%\definecolor{KwColor}{rgb}{0,0,0.6}
%\newcommand{\vkKw}[1]{{\bf\color{KwColor} #1}}

\renewcommand{\algorithmicrequire}{\rule{0pt}{2.5ex}\bf{Вход:}}
\renewcommand{\algorithmicif}{\bf{если}}
\renewcommand{\algorithmicthen}{\bf{то}}
\renewcommand{\algorithmicelse}{\bf{иначе}}
\renewcommand{\algorithmicelsif}{\algorithmicelse\ \algorithmicif}
\renewcommand{\algorithmicendif}{\algorithmicend\ \algorithmicif}
\renewcommand{\algorithmicfor}{\bf{для}}
\renewcommand{\algorithmicforall}{\bf{для всех}}
\renewcommand{\algorithmicdo}{}
\renewcommand{\algorithmicendfor}{\algorithmicend\ \algorithmicfor}
\renewcommand{\algorithmicwhile}{\bf{пока}}
\renewcommand{\algorithmicendwhile}{\algorithmicend\ \algorithmicwhile}
\renewcommand{\algorithmicloop}{\bf{цикл}}
\renewcommand{\algorithmicendloop}{\algorithmicend\ \algorithmicloop}
% Мои дополнительные команды для описания алгоритмов
\newcommand{\BEGIN}{\\[1ex]\hrule\vskip 1ex}
\newcommand{\END}{\vskip 1ex\hrule\vskip 1ex}
\newcommand{\vkReturn}{\bf{вернуть} }
\newcommand{\OUT}{\STATE\bf{конец}}
\newcommand{\RET}{\STATE\vkReturn}
\newcommand{\EXIT}{\STATE\bf{выход}}
\newcommand{\CONTINUE}{\STATE\bf{следующий} }
\newcommand{\IFTHEN}[1]{\STATE\algorithmicif\ #1 {\algorithmicthen}}
\newcommand{\vkProcedure}[1]{\text{#1}\:}
\newcommand{\vkProc}[1]{\text{#1}\:}
\newcommand{\PROCEDURE}[1]{\medskip\STATE\bf{ПРОЦЕДУРА} \vkProcedure{#1}}

\def\stat{tokmakova}

\def\tit{ОЦЕНИВАНИЕ ГИПЕРПАРАМЕТРОВ ЛИНЕЙНЫХ РЕГРЕССИОННЫХ МОДЕЛЕЙ ПРИ~ОТБОРЕ
ШУМОВЫХ~И~КОРРЕЛИРУЮЩИХ ПРИЗНАКОВ$^*$}

\def\titkol{Оценивание гиперпараметров линейных регрессионных моделей при отборе
шумовых и коррелирующих признаков}

\def\autkol{А.\,А.~Токмакова, В.\,В.~Стрижов}

\def\aut{А.\,А.~Токмакова$^1$, В.\,В.~Стрижов$^2$}

\titel{\tit}{\aut}{\autkol}{\titkol}

{\renewcommand{\thefootnote}{\fnsymbol{footnote}}\footnotetext[1]
{Работа выполнена при поддержке РФФИ, грант № 10-07-00422.}}

\renewcommand{\thefootnote}{\arabic{footnote}}
\footnotetext[1]{Московский физико-технический институт, aleksandra-tok@yandex.ru}
\footnotetext[2]{Вычислительный центр Российской академии наук, strijov@ccas.ru}

\vspace*{-6pt}


\Abst{Решается задача отбора признаков при восстановлении линейной регрессии. 
Принята гипотеза о нормальном распределении вектора зависимой переменной 
и~параметров модели. Для оценки ковариационной матрицы параметров используется 
аппроксимация Лапласа: логарифм функции ошибки приближается функцией плот\-ности 
нормального распределения. Исследуется проблема присутствия в~выборке шумовых 
и~коррелирующих признаков, так как при их наличии матрица ковариаций параметров 
модели становится вырожденной. Предлагается алгоритм, производящий отбор информативных 
признаков. В~вычислительном эксперименте приводятся результаты исследования 
на~временн$\acute{\mbox{о}}$м  ряде.}

\KW{байесовский вывод; ковариационная матрица; гиперпараметры модели; 
отбор признаков; регрессия}

\vskip 14pt plus 9pt minus 6pt

      \thispagestyle{headings}

      \begin{multicols}{2}

            \label{st\stat}


%\vspace*{-12pt}


\section{Введение}

Часто при анализе временных рядов требуется рассмотрение большого
числа признаков.  В~связи с этим возникают проблемы, связанные с
наличием в~выборке большого количества мультикоррелирующих признаков
или с высокой зашумленностью выборки. В~работе выдвинута  гипотеза о
нормальном распределении вектора зависимой переменной и~вектора
параметров модели~\cite{strijov1, weber}. Необходимо оценить
ковариационные матрицы этих распределений и~установить связь между
пространством данных и~пространством параметров, что позволит
произвести отбор шумовых и~коррелирующих признаков.

Развитие методов отбора признаков имеет богатую историю. Начиная с
1960~г.\ активно развивались шаговые методы (Stepwise
Regression)~\cite{stepwise}. Главная идея этих методов состоит
в~отборе признаков, вносящих наибольший вклад в~зависимую
переменную. Вводится критерий, на~основании которого алгоритм
добавляет или удаляет признаки. Широкое применение получили частные
случаи шаговой регрессии~--- алгоритмы  LARS (Least Angle
Regression)~\cite{lars} и~LASSO (Least Absolute Shrinkage and
Selection Operator)~\cite{lasso}. 

Алгоритм LARS заключается
в~последовательном добавлении признаков. На~каж\-дом шаге веса
признаков меняются таким образом, чтобы обеспечить наибольшую
корреляцию восстановленного вектора зависимых переменных с вектором
регрессионных остатков. Алгоритм позволяет сократить количество
свободных переменных и~избежать проб\-ле\-мы неустойчивой оценки весов.

Метод LASSO вводит ограничения на~норму вектора коэффициентов
модели, что приводит к~обращению в~ноль некоторых коэффициентов
модели. Метод приводит к~повышению устойчивости модели и позволяет
отбирать признаки, ока\-зы\-ва\-ющие наибольшее влияние на~вектор ответов.

Одной из причин возникновения задачи отбора признаков является их
мультиколлинеарность. Первые шаги по решению этой проблемы были
сделаны в 1963~г.\ А.\,И.~Тихоновым, который ввел понятие
регуляризации~--- дополнительного ограничения
на~задачу~\cite{regular}. В~работе~\cite{ridzh1} введено понятие
регуляризации и~описан общий метод решения задач. Но поскольку
работы Тихонова были опуб\-ли\-ко\-ва\-ны на~Западе только лишь в
1977~г., в 1970~г.\ Hoerl и Kennard предложили метод
гребневой регрессии~\cite{ridzh2}. В~минимизируемую функцию
вводилось дополнительное слагаемое, что повышало устойчивость
решения~\cite{ridzh3}, однако не~позволяло производить отбор
признаков. 

Позднее стали появляться методы, ис\-поль\-зу\-ющие качественно
иной подход к~решению проб\-ле\-мы муль\-ти\-кол\-ли\-неар\-ности. Например,
Belsley предложил метод для удаления признаков~\cite{belsly},
использующий сингулярное разложение матрицы плана. Алгоритм находит
коэффициент, характеризующий степень зависимости признаков друг от
друга. Позднее появился метод фактора инфляции дис\-пер\-сии (Variance
Inflation Factor)~\cite{vif}, оценивающий увеличение дисперсии
заданного коэффициента регрессии, что свидетельствует о высокой
корреляции данных.

В данной работе для отбора признаков линейной регрессионной модели
предлагается выполнить анализ пространства параметров. Вектор
па\-ра\-мет\-ров рассматривается как многомерная случайная величина.
Оценивается наиболее вероятное значение параметров. При оценке
используется связный байесовский вывод~\cite{nabney, mackay}.

Основываясь на~гипотезе о нормальном распределении параметров
модели~\cite{weber}, оценивается ковариационная матрица
распределения па\-ра\-мет\-ров~\cite{strijov1, strijov2}. На~ее главной
диагонали стоят дис\-пер\-сии случайных величин, что позволяет
установить степень значимости данного конкретного па\-ра\-мет\-ра
в~модели. При таком подходе к~ отбору признаков не~возникает
необходимости разбиения выборки на~обучение и~контроль. Для оценки
ковариационной матрицы в работе используется аппроксимация
Лапласа~\cite{laplace}. Логарифм функции ошибки приближается
функцией плотности нормального распределения, и~появляется
воз\-мож\-ность вы\-чис\-ле\-ния нормировочной константы.



\section{Постановка задачи}

Дана регрессионная выборка: 
$D\hm=\{\mathbf{x}_i,y_i\}_{i=1}^m\hm=(X,\mathbf{y})$, 
где~$\mathbf{x}_i \hm\in \mathbb{R}^n$, $i \hm= 1, \dots, m,$~--- 
векторы независимой переменной, а $y_i \hm\in \mathbb{R}$, $i \hm= 1, \dots, m,$~--- 
значения зависимой переменной. Решается задача восстановления регрессии
\begin{equation}
\label {eq: y}
\mathbf{y}=\mathbf{f}(\mathbf{w}, X)\,,
\end{equation}
 где~$\mathbf{f}(\mathbf{w}, X)$~--- некоторая параметрическая век\-тор-функ\-ция. 
 Пусть многомерная случайная величина~$\mathbf{y}$ имеет нормальное распределение: 
 $$
 \mathbf{y}\sim\mathcal{N}(\mathbf{f},\sigma^2 I_m)\,,
 $$ 
 где~$\mathbf{f}$~--- век\-тор-функ\-ция, $\sigma^2$~--- дисперсия распределения, 
 $I_m$~--- единичная матрица размерности~$m$.

 Требуется приблизить функцию~$\mathbf{f}(\mathbf{w}, X)$ 
 параметрической функцией~$\widehat{\mathbf{f}}(X,\mathbf{w})$ 
 из заданного класса~$\mathcal{F}$ (линейные функции), 
 причем~$|\mathcal{F}|$ конечно. Отображение
 $\mathbf{f}:\mathbb{R}^m \times \mathbb{W}^n \rightarrow \mathbb{R}^m$ будем называть 
 моделью. Здесь~$\mathbb{R}^m$~--- пространство данных,\linebreak 
 а~$\mathbb{W}^n \hm\subseteq \mathbb{R}^n$~--- пространство параметров. 
 В~задаче линейной регрессии задача приближения функции~$\mathbf{f}(\mathbf{w}, X)$ 
 эквивалентна задаче отбора приз\-наков. В~данном случае модель определяется \mbox{параметрами}, 
 которые соответствуют множеству индексов активных признаков 
 $\mathcal{A}\hm\subseteq \mathcal{J} = \{1,2,\ldots ,n\}$. Таким образом, при 
 выборе модели требуется найти такое множество индексов~$\mathcal{A}^{*}$, 
 которое бы обеспечивало минимум функции
 $$
 \mathcal{A}^{*}=\arg \min\limits_{\mathcal{A} 
 \subseteq \mathcal{J}} S(\mathbf{f}_{\mathcal{A}}|\mathbf{w}^{*},D)\,,
 $$
 где $S(\mathbf{f}|\mathbf{w},D)$~--- функция ошибки, 
 $\mathbf{f}_{\mathcal{A}}$~--- па\-ра\-мет\-ри\-че\-ская век\-тор-функ\-ция, вычисляемая 
 только на множестве активных признаков, заданном индексами из множества~$\mathcal{A}$.
 При этом параметры~$\mathbf{w}^{*}$ модели должны обеспечивать минимум функции
 $$
 \mathbf{w}^{*}=\arg \min \limits_{\mathbf{w}\in\mathbb{W}} 
 S(\mathbf{w}|\mathbf{f}_{\mathcal{A}},D)\,.
 $$

\section{Вид функции ошибки}

Пользуясь предположением о том, что вектор зависимой переменной~--- многомерная 
случайная величина с нормальным распределением, запишем конкретный вид функции 
ошибки~$S(\mathbf{w})$ для по\-став\-лен\-ной задачи.

Пусть многомерная случайная величина~$\mathbf{y}$ имеет нормальное распределение. 
Обозначим~$\beta^{-1}\hm=\sigma^{2}$. Тогда распределение зависимой переменной~$\mathbf{y}$ 
можно представить в~виде:
\begin{multline}
\label {eq: NormB}
p(\mathbf{y})={}\\
\hspace*{-2mm}{}=\left(2\pi\beta^{-1}\right)^{-{m}/{2}}
\exp\left(-\fr{1}{2}\left(\mathbf{y}-\mathbf{f}\right)^{\mathrm{T}}\beta 
I(\mathbf{y}-\mathbf{f})\right).\!
\end{multline}
Рассмотрим функцию правдоподобия данных, которая имеет вид:
\begin{equation}
\label {eq: exp(-ED)}
p(\mathbf{y}|X,\mathbf{w},\beta,\mathbf{f})\mathrel{\stackrel{\mathrm{def}}{=}}p(D|\mathbf{w},\beta,
\mathbf{f})=\fr{\exp(-\beta E_D)}{Z_{D}\left(\beta\right)}\,.
\end{equation}
Здесь~$E_D$~--- функция ошибки. Из выражений~(\ref{eq: NormB}) и~(\ref{eq: exp(-ED)}) 
определим ее как
$$
E_D=\fr{1}{2}\left(\mathbf{y}-\mathbf{f}\right)^{\mathrm{T}}(\mathbf{y}-\mathbf{f})\,.
$$
Коэффициент~$Z_D$ нормирует плотность нормального распределения и равен
\begin{equation}
\label {eq: ZD}
Z_{D}(\beta)=(2\pi\beta^{-1})^{{m}/{2}}\,.
\end{equation}

Рассмотрим равенство~(\ref{eq: y}). Слева стоит многомерная случайная величина~$\mathbf{y}$, 
име\-ющая нормальное распределение. Матрица~$X$ не является случайной величиной, 
поэтому предположим, что~$\mathbf{w} \hm\in \mathbb{W}^n$ также является многомерной 
случайной величиной с нормальным распределением. Параметрами этого распределения 
будут математическое ожидание~$\mathbf{w}_0$ и матрица ковариаций~$A^{-1}$:
\begin{equation}
\label {eq: exp(-Ew)}
p(\mathbf{w}|A,\mathbf{f})=\fr{\exp\left(-E_{\mathbf{w}}\right)}{Z_{\mathbf{w}}(A)}\,.
\end{equation}

Определим функцию-штраф за большое значение параметров модели для принятого 
распределения как 
$E_\mathbf{w}\hm=({1}/{2})\left(\mathbf{w} \hm- \mathbf{w}_0\right)^{\mathrm{T}}A(\mathbf{w} 
\hm- \mathbf{w}_0)$. Нормирующая константа~$Z_\mathbf{w}$ в этом случае равна
\begin{equation}
\label {eq: Zw}
Z_\mathbf{w}(A)=(2\pi)^{{n}/{2}}|A^{-1}|^{{1}/{2}}\,.
\end{equation}
Апостериорное распределение параметров модели для заданных~$A$
и~$\beta$ имеет вид:
\begin{align}
\label {eq: 4p}
p(\mathbf{w}|D,A,\beta,\mathbf{f})&=
\fr{p(D|\mathbf{w},\beta,\mathbf{f})p(\mathbf{w}|A,\mathbf{f})}{p(D|A,\beta,\mathbf{f})}\,,
\\
%\label {eq: 4pE}
\fr{p(D|\mathbf{w},\beta,\mathbf{f})p(\mathbf{w}|A,\mathbf{f})}
{p(D|A,\beta,\mathbf{f})}&=\fr{\exp\left(-\beta E_D\right)
\exp\left(-E_\mathbf{w}\right)}{Z_{D}
(\beta)Z_{\mathbf{w}}(A)}={}\notag\\
&\hspace*{5mm}{}=
\fr{\exp\left(-\left(\beta E_D+E_\mathbf{w}\right)\right)}{Z_{D}(\beta)Z_{\mathbf{w}}(A)}\,.\notag
\end{align}
В выражении~(\ref{eq: 4p}) приняты следующие обозначения:
$p(\mathbf{w}|D,A,\beta,\mathbf{f})$~--- апостериорное распределение параметров;
$p(D|\mathbf{w},\beta,\mathbf{f})$~--- функция правдоподобия данных;
$p(\mathbf{w}|A,\mathbf{f})$~--- априорное распределение па\-ра\-мет\-ров;
$p(D|A,\beta,\mathbf{f})$~--- функция правдоподобия модели.
Записывая функцию ошибки как
\begin{multline}
\label {eq: S(w)}
S=E_\mathbf{w}+\beta E_D=\fr{1}{2}\left(\mathbf{w}-\mathbf{w}_0\right)^{\mathrm{T}}
A(\mathbf{w}-\mathbf{w}_0)+{}\\
{}+\fr{1}{2}\left(\mathbf{y}-\mathbf{f}\right)^{\mathrm{T}}
\beta I(\mathbf{y}-\mathbf{f})\,,
\end{multline}
получим следующее выражение для апостериорного распределения параметров: 
$$
p(\mathbf{w}|D,A,\beta,\mathbf{f})=
\fr{\exp\left(-S(\mathbf{w})\right)}{Z_S(A,\beta)}\,,
$$ 
где $Z_S\hm=Z_{S}(A,\beta)$~--- нормирующий коэффициент. Оценка нормировочного 
коэффициента производится с помощью аппроксимации Лапласа.

\section{Аппроксимация Лапласа}

Аппроксимация Лапласа позволяет оценить нормировочный коэффициент для ненормированной 
плот\-ности вероятности. Пусть задано ненормированное распределение~$p^{*}(\mathbf{w})$. 
Требуется найти нормировочную константу
$$
Z=\int p^{*}\left(\mathbf{w}\right) d\mathbf{w}\,,
$$
при которой распределение $p(\mathbf{w})\hm=Z^{-1}p^{*}(\mathbf{w})$. 
Предположим, что~$p^{*}(\mathbf{w})$ имеет максимум в точке~$\mathbf{w}_0$, т.\,е.\ 
$$
\left. \fr{d p(\mathbf{w})}{d w} \right|_{\mathbf{w}=\mathbf{w}_0}  = 0\,.
$$ 
Прологарифмируем и разложим~$p^{*}(\mathbf{w})$ по Тейлору в~окрестности~$\mathbf{w}_0$:
\begin{multline}
\label {eq: p(w)}
\ln p^{*}(\mathbf{w}) = \ln p^{*}(\mathbf{w}_0) + 0 -{}\\
{}- \fr{1}{2}\left(\mathbf{w}-
\mathbf{w}_0\right)^{\mathrm{T}}A\left(\mathbf{w}-\mathbf{w}_0\right) + \cdots\,,
\end{multline}
где матрица Гессе $A\hm=[\alpha_{ij}]$ определена как
$$
\left. \alpha_{ij}=-\fr{\partial \ln p^{*}(\mathbf{w})}{\partial w_{i}
\partial w_{j}} \right|_{\mathbf{w}=\mathbf{w}_0}\,,
$$ 
т.\,е.\ $A\hm=-\nabla\nabla \ln p^{*}(\mathbf{w})|_{\mathbf{w}\hm=\mathbf{w}_0}$, 
где~$\nabla$~--- градиент функции.

Отбрасывая все члены выше квадратичного в~разложении и~беря экспоненту обеих частей 
выражения~(\ref{eq: p(w)}), получим:

\noindent
$$
p^{*}(\mathbf{w})\approx p^{*}(\mathbf{w}_0)\exp \left( 
-\fr{1}{2}\left(\mathbf{w}-\mathbf{w}_0\right)^{\mathrm{T}}
A\left(\mathbf{w}-\mathbf{w}_0\right) \right)\,.
$$
Тогда нормальное распределение~$\widehat{p}(\mathbf{w})$, приближающее нормированное 
распределение~$p(\mathbf{w})$, имеет вид:

\noindent
\begin{multline*}
\widehat{p}(\mathbf{w}) = \mathcal{N}(\mathbf{w}_0,A^{-1}) = 
\fr{1}{(2\pi)^{{n}/{2}}\left\vert A^{-1}\right\vert^{{1}/{2}}}\times{}\\
{}\times
\exp\left( -\fr{1}{2}\left(\mathbf{w}-\mathbf{w}_0\right)^{\mathrm{T}}A
\left(\mathbf{w}-\mathbf{w}_0\right) \right)\,.
\end{multline*}
Следовательно, нормировочная константа имеет вид:

\noindent
\begin{equation*}
%\label {eq: ZP}
Z=p^{*}(\mathbf{w}_0)\fr{(2\pi)^{{n}/{2}}}{|A|^{{1}/{2}}}\,.
\end{equation*}


\section{Оценка ковариационных матриц}

Анализируя функцию ошибки~$S(\mathbf{w})$, построим алгоритм, позволяющий 
выявлять шумовые и~ коррелирующие признаки.

Пусть нам известен локальный минимум~$S(\mathbf{w})$ и он находится
в~точке~$\mathbf{w}_0$. Рассмотрим матрицу Гессе функции ошибок 
$H\hm= - \nabla \nabla S(\mathbf{w})|_{\mathbf{w=w}_0}$, где $\nabla$~---
градиент функции. При появлении в выборке шумовых или коррелирующих
признаков происходит резкое возрастание некоторых элементов
матрицы~$H$. Необходимо установить связь между компонентами матрицы
Гессе и~ ковариационной матрицей параметров, для того чтобы
произвести отбор активных параметров~$\mathcal{A}$ и повысить
устойчивость решения.

Рассмотрим ряд Тейлора второго порядка логарифма числителя~(\ref{eq: 4p}):

\noindent
\begin{equation}
\label {eq: -S(w)}
-S(\mathbf{w}) \approx -S(\mathbf{w}_0)-\fr{1}{2}\,\Delta\mathbf{w}^{\mathrm{T}}H\Delta\mathbf{w}\,,
\end{equation}
где $\Delta\mathbf{w}\hm=\mathbf{w}\hm-\mathbf{w}_0$.
В~выражении~(\ref{eq: -S(w)}) нет сла\-га\-емо\-го первого порядка, так
как предполагается, что точка~$\mathbf{w}_0$ доставляет локальный
минимум функции~$S(\mathbf{w})$. Следовательно,
$$
\left. \fr{\partial S(\mathbf{w})}{\partial w} \right|_{\mathbf{w} = \mathbf{w}_0}  = 0\,.
$$
Применяя экспоненту к~обеим частям выражения~(\ref{eq: -S(w)}), получим необходимое приближение:
\begin{multline}
\label {eq: exp(-S(w))}
\exp\left(-S(\mathbf{w})\right) \approx {}\\
{}\approx\exp\left(-S(\mathbf{w}_0)\right)
\exp\left(-\fr{1}{2}\,\Delta\mathbf{w}^{\mathrm{T}}H\Delta\mathbf{w}\right)\,.
\end{multline}
При полученном приближении выражение~(\ref{eq: exp(-S(w))}) будет выглядеть следующим образом:
\begin{multline}
\label {eq: p(w|D,A,B)}
p(\mathbf{w}|D,A,\beta) \approx{}\\
{}\approx \fr{\exp\left(-S(\mathbf{w}_0)\right)
\exp\left(-({1}/{2})\Delta\mathbf{w}^{\mathrm{T}}H\Delta\mathbf{w}\right)}{Z_{S}(A,\beta)}\,,
\end{multline}
где~$Z_S(A,\beta)$ выступает в~роли нормировочного коэффициента
плотности вероятностного распределения. Оценка для
коэффициента~$Z_S$ получена с помощью аппроксимации Лапласа
(пояснения см.\ в~гл.~4):
\begin{equation}
\label {eq: ZS}
Z_S=\fr{\exp\left(-S(\mathbf{w}_0)\right)(2\pi)^{{n}/{2}}}{|H|^{{1}/{2}}}\,.
\end{equation}
Подставив~(\ref{eq: ZS}) в~(\ref{eq: p(w|D,A,B)}), получим оценку
правдоподобия модели, на~основании которой будем производить отбор
оптимальных гиперпараметров модели
\begin{equation*}
%\label {eq: approx}
p(\mathbf{w}|D,A,\beta)=\fr{|H|^{{1}/{2}}\exp\left(-({1}/{2})\,\Delta\mathbf{w}^{\mathrm{T}} H 
\Delta \mathbf{w}\right)}{(2\pi)^{{n}/{2}}}\,.
\end{equation*}
Выражение~(\ref{eq: p(w|D,A,B)}) определяет выбор наиболее
правдоподобной модели. Для нахождения ги\-пер\-па\-ра\-мет\-ров воспользуемся
принципом максимума правдоподобия~$p(D|A,\beta)$ относительно~$A$ 
и~$\beta$. Запишем $p(D|A,\beta)$ в~следующем виде:
\begin{equation*}
%\label {eq: 1p(D|A,B)}
p(D|A,\beta)=\int p(D|\mathbf{w},A,\beta)p(\mathbf{w}|A)\,d\mathbf{w}\,.
\end{equation*}
Используя выражения~(\ref{eq: exp(-Ew)}) и~(\ref{eq: exp(-ED)}), 
перепишем функцию правдоподобия в~виде:
\begin{multline}
\label {eq: 2p(D|A,B)}
p(D|A,\beta)={}\\
{}=\fr{1}{Z_{\mathbf{w}}(A)}\,\fr{1}{Z_{D}(\beta)}
\int \exp\left(-S(\mathbf{w})\right)d\mathbf{w}\,.
\end{multline}
Из соображений нормировки интеграл выражения~(\ref{eq: 4p}) равен единице, т.\,е.\
$$
\int p(\mathbf{w}|D,\beta)d\mathbf{w}=\int 
\fr{\exp\left(-S(\mathbf{w})\right)}{Z_{S}(A,\beta)}\,d\mathbf{w}=1\,.
$$
Следовательно, интеграл в~правой части~(\ref{eq: 2p(D|A,B)}) в точ\-ности равен~$Z_S$. Поэтому
\begin{multline}
\label {eq: pdab}
p(D|A,\beta)={}\\
\hspace*{-3mm}{}=\fr{1}{Z_{\mathbf{w}}(A)}\,\fr{1}{Z_{D}(\beta)}
\exp\left(-S(\mathbf{w}_0)\right)(2\pi)^{{n}/{2}}|H|^{-{1}/{2}}\,.\!
\end{multline}
Подставив значение~$Z_\mathbf{w}$ из~(\ref{eq: Zw}) и~$Z_D$ из~(\ref{eq: ZD}) 
в~(\ref{eq: pdab}), получим:
\begin{multline*}
%\label {eq: pdabr}
\hspace*{-2.5mm}p(D|A,\beta)=(2\pi)^{-{n}/{2}}|A^{-1}|^{-{1}/{2}}(2\pi)^{-{m}/{2}}(\beta^{-1})^{{m}/{2}} \times{}\\
{}\times
\exp\left(-S(\mathbf{w}_0)\right)(2\pi)^{{n}/{2}}|H|^{-{1}/{2}}\,.
\end{multline*}
Получим оценку логарифма правдоподобия:
\begin{multline}
\label {eq: lnpdabf}
\ln p(D|A,\beta,\mathbf{f})=-\fr{1}{2}\,\ln|A^{-1}|-\fr{m}{2}\,\ln 2\pi + {}\\
{}+
\fr{m}{2}\,\ln\beta^{-1}-S(\mathbf{w}_0)-\fr{1}{2}\,\ln|H|\,.
\end{multline}
Поочередно приравнивая частные производные по~$A$ и~$\beta$ выражения~(\ref{eq: lnpdabf}) 
к нулю, найдем максимум~(\ref{eq: lnpdabf}) по гиперпараметрам.

Пусть матрица~$A$ диагональна. Введем
обозначение~$\boldsymbol{\alpha}\hm=[\alpha_1, \ldots ,\alpha_n]^{\mathrm{T}}$ для
вектора, состоящего из элементов диагонали матрицы~$A$. Представим
гессиан в~виде:
\begin{multline*}
H=- \nabla \nabla S(\mathbf{w})=-\nabla\nabla\left(\beta E_D + E_\mathbf{w}\right)={}\\
{}=
-\beta\nabla\nabla E_D-\nabla\nabla E_\mathbf{w}=H_D+H_{\mathbf{w}}\,,
\end{multline*}
где~$H_D$ зависит от~$\beta$, а~$H_\mathbf{w}$ зависит от~$A$.
Так как $\nabla\nabla E_{w_i}\hm=({\partial^2}/{\partial w_i}) 
\left((1/2)\alpha_i (w_i\hm-w_{0i})^2\right)\hm=\alpha_i$, то часть гессиана~$H_\mathbf{w}$ 
диагональна. Покажем, что при некоторых допущениях~$H_D$ также будет диагональной матрицей. 
Для этого рассмотрим два случая:
\begin{enumerate}[(1)]
  \item если все признаки независимы, то матрица~$H_D$ будет диагональной, 
  так как недиагональные элементы матрицы Гессе отражают степень зависимости 
  измеряемых величин;
  \item при наличии в~выборке шумовых или коррелирующих признаков будет 
  наблюдаться возрас\-та\-ние диагональных элементов матрицы (дисперсий признаков), 
  в сравнении с которыми недиагональными элементами можно пре\-неб\-речь; т.\,е.\ 
  и в этом случае матрицу~$H_D$ можно считать диагональной (на диагонали~--- 
  собственные числа).
\end{enumerate}

Итак, представим~$H_D$ в следующем виде: $H_D \hm=
\mathrm{diag}\left(h_1, \ldots ,h_n\right).$ Для выявления связи между параметрами
и гиперпараметрами модели рас\-смот\-рим выражение~(\ref{eq: lnpdabf}).
Воспользуемся необходимым условием минимума и приравняем к нулю
первые производные выражения~(\ref{eq: lnpdabf}) по~$\alpha_i$:
\begin{equation*}
%\label {eq: dlnp}
\fr{1}{\alpha_i}-\left(w_{i}-w_{0}\right)^2-\fr{1}{\beta h_i+\alpha_i}=0\,.
\end{equation*}

Данное уравнение имеет два корня. Однако один из них не~имеет
смысла, так как~$A^{-1}$~--- диагональная ковариационная
(положительно определенная) матрица, следовательно, по критерию
Сильвестра (симметричная квадратная матрица является положительно
определенной тогда и~только тогда, когда все ее главные миноры
положительны) не имеет отрицательных компонент:
\begin{equation}
\label {eq: alfa}
\alpha_i=\fr{1}{2}\,\lambda_i\left(\sqrt{1+\fr{4}{(w_{i}-w_{0})^{2}\lambda_i}}-1\right)\,,
\end{equation}
где $\lambda_i\hm=\beta h_i.$

Приравняв производную по~$\beta$ выражения~(\ref{eq: lnpdabf}), найдем оптимальное 
значение~$\beta$:
$$
\fr{m}{2\beta}-E_D-\fr{1}{2\beta}\,\gamma=0\,,
$$
где
$$
\gamma=\sum\limits_{j=1}^W \fr{\lambda_j}{\lambda_j + \alpha_j}\,.
$$
Таким образом,
\begin{equation}
\label {eq: beta}
\beta=\fr{m-\gamma}{2E_D}\,.
\end{equation}

Выражения~(\ref{eq: alfa}) и~(\ref{eq: beta}) не~позволяют явно вы\-чис\-лить 
значения~$\boldsymbol{\alpha}$ и~$\beta$. Поэтому итерационный процесс 
организуется следующим образом. На каж\-дом шаге вычисляем~$\mathbf{w}$ 
(минимизируя функцию\linebreak ошибки из выражения~(\ref{eq: S(w)})), далее, 
используя полученное приближение, находим вектор гиперпараметров~$\boldsymbol{\alpha}$, 
затем значение гиперпараметра~$\beta$. \mbox{Процедура} продолжается до сходимости как па\-ра\-мет\-ров, 
так и~гиперпараметров, т.\,е.\ до сходимости функции правдоподобия 
модели~$p(D|A,\beta,\mathbf{w})$.

При появлении шумовых или коррелирующих признаков происходит возрастание диагональных 
элементов (большое значение дисперсии свидетельствует о неинформативности признака). 
Вследствие этого недиагональные элементы становятся настолько малы, что можно считать 
матрицу~$H_D$ диагональной. Поэтому необходимо принудительно занижать возрастающие 
диагональные элементы, тем самым производя отсеивание шумовых и коррелирующих признаков.

Ниже приведен псевдокод алгоритма оценки гиперпараметров регрессионной модели.

\begin{algorithmic}
\REQUIRE вектор зависимой переменной~$\mathbf{y}$, модель~$\text{mdl}(\mathbf{w}, X)$
\STATE $\mathbf{w}_0=0$;
\STATE $\mathbf{w}=0$;
\STATE $A = \text{diag}(n,1)$;
\STATE $\beta=1$;
\FOR{$k=2,\dots,\mathrm{MaxIterations}$} вычислить $A, \beta, \mathbf{w}:$
    \STATE$\mathbf{w}=\textbf{FindParameters}(S(\mathbf{w}), A, \beta, \mathbf{w}, \mathbf{w}_0, \mathbf{y});$
    \FOR{$j=2,\dots,\mathrm{MaxIterations}$} добиться сходимости $A$ и~$\beta$ при данном векторе $\mathbf{w}$:
            \STATE$H=\textbf{CalcHessian}(S(\mathbf{w}), A, \beta, \mathbf{w}, \mathbf{w}_0, \mathbf{y});$
            \IF{${\max(H)}/{\min(H)}>10^6$}
                \STATE$\mathrm{idx}\;= \text{find} (\max(H));$ \COMMENT {индекс строки/cтолбца (диагональный элемент) с max элементом}
                \STATE занулить строку и~столбец Гессиана, содержащие максимальный элемент;
            \ENDIF
            \EXIT ;
            \STATE$\lambda=\beta*\text{diag}(H);$
            \STATE$A=\fr{1}{2}\,\lambda(\sqrt{1+\fr{4}{(\mathbf{w}-\mathbf{w}_0)^{2}\lambda}}-1);$
            \IF {$\mathrm{idx}\;\neq 0$}
                \STATE занулить соответствующие диагональные элементы матрицы $A$ (необходимо для сходимости гиперпараметра $\alpha$);
            \ENDIF
            \EXIT ;
            \STATE$\gamma=\sum \fr{\lambda_j}{\lambda_j + \alpha_j};$
            \STATE$\mathbf{f}=\text{mdl}(\mathbf{w},X);$
            \STATE$E_D=\fr{1}{2}(\mathbf{y}-\mathbf{f})^{\mathrm{T}}(\mathbf{y}-\mathbf{f});$
            \STATE$\beta = \fr{(m - \gamma)}{2 E_D};$
            \IF {$\sum (\alpha_k - \alpha_{k-1})^2 < \mathrm{Convergency}$ и $(\beta_k-\beta_{k-1})^2 < \mathrm{Convergency}$;}
                \STATE {закончить выполнение цикла на~текущей итерации;}
            \ENDIF
            \EXIT ;
            \IF {$j=\mathrm{MaxIterations}$}
                \STATE {вывести сообщение о величине ошибки и~закончить выполнение программы;}
            \ENDIF
            \EXIT;
    \ENDFOR
    \IF {$\sum (w_k-w_{k-1})^2 < \mathrm{Convergency}$}
        \STATE {закончить выполнение программы;}
    \ENDIF
\ENDFOR
\OUT
\end{algorithmic}

\vspace*{-12pt}

\begin{algorithmic}
    \PROCEDURE {\textbf{FindParameters}$((S\left(\mathbf{w}\right),$} 
    \STATE {\hspace*{50mm}$\left.A, \beta, \mathbf{w}, \mathbf{w}_0, \mathbf{y}\right)$}
    \WHILE {не найден минимум функции $S(\mathbf{w})$ по $\mathbf{w}$}
        \STATE$\mathbf{f} = \text{mdl}(\mathbf{w}, X);$
        \STATE$S(\mathbf{w}) = ({1}/{2})(\mathbf{w}-\mathbf{w}_0)^{\mathrm{T}}
        A(\mathbf{w}-\mathbf{w}_0)$
        \STATE$+({1}/{2})(\mathbf{y}-\mathbf{f})^{\mathrm{T}}\beta I(\mathbf{y}-\mathbf{f});$
    \ENDWHILE
    \EXIT ;
    \STATE{\vkReturn {$\mathbf{w};$}}
\end{algorithmic}

\begin{algorithmic}
\PROCEDURE {\textbf{CalcHessian}$(S(\mathbf{w}), A, \beta, \mathbf{w}, \mathbf{w}_0, \mathbf{y})$}
    \STATE$h=10^{-6};$ \COMMENT {шаг разностной схемы}
    \FOR {$i=1,...,l$}
        \FOR {$j=1,...,l$}
        \STATE посчитать элемент матрицы Гессе:
        \STATE $\mathbf{e}_i=0;$ \COMMENT {вектор приращения}
        \STATE $e_i(i)=1;$
        \STATE $\mathbf{e}_j=0;$
        \STATE $e_j(j)=1;$
        \STATE $H(i,j) =
        \left(S(\mathbf{w}+(\mathbf{e}_i+\mathbf{e}_j)h)-S(\mathbf{w}+\mathbf{e}_{i}h)-{}\right.$
        \STATE $\left.{}-S(\mathbf{w}+\mathbf{e}_{j}h)+S(\mathbf{w})\right)/h^2;$
        \ENDFOR
        \EXIT ;
    \ENDFOR
    \EXIT ;
    \STATE {\vkReturn {$H;$}}
\end{algorithmic}

\section{Алгоритмы отбора признаков}

Для того чтобы подчеркнуть особенности описанного в работе
алгоритма, приведем примеры ранее предложенных методов
регуляризации, приводящих к~повышению устойчивости решения и  отбору
признаков в~задаче линейной регрессии~\cite{lasso, ridzh2}.

\smallskip

\textbf{Гребневая регрессия.}
Запишем функцию ошибки для линейной модели вида~(\ref{eq: y}):
$$
Q(\mathbf{w})=||X\mathbf{w}-\mathbf{y}||^2\,.
$$
Для минимизации функции воспользуемся необходимым условием минимума
$$
\fr{\partial Q}{\partial \mathbf{w}}=2X^{\mathrm{T}}(X\mathbf{w}-\mathbf{y})=0\,,
$$
откуда следует, что $X^{\mathrm{T}}X\mathbf{w}\hm=X^{\mathrm{T}}\mathbf{y}$. 
Если матрица~$X^{\mathrm{T}}X$ не вырождена, то решением системы является вектор
$$
\mathbf{w}^{*}=\left(X^{\mathrm{T}}X\right)^{-1}X^{\mathrm{T}}\mathbf{y}\,.
$$

Если ковариационная матрица~$X^{\mathrm{T}}X$ имеет неполный ранг, то ее
обращение невозможно.  Также выделяют случай мультиколлинеарности:
матрица~$X^{\mathrm{T}}X$ имеет полный ранг, но близка к~некоторой матрице
неполного ранга. В~этом случае увеличивается разброс
коэффициентов~$\mathbf{w}^{*}$, появляются большие по абсолютной
величине коэффициенты. Решение становится неустойчивым (небольшие
изменения матрицы~$X$ ведут к большим изменениям
величины~$\mathbf{w}^{*}$).

Для решения проблемы мультиколлинеарности к функционалу~$Q$
добавляют регуляризатор, штрафующий большие значения нормы
вектора~$\mathbf{w}$:
$Q_\tau\hm=||X\mathbf{w}\hm-\mathbf{y}||^2\hm+\tau||\mathbf{w}||^2$. Решением
полученной задачи является вектор
$$
\mathbf{w}^{*}=\left(X^{\mathrm{T}}X+\tau I_m\right)^{-1}X^{\mathrm{T}}\mathbf{y}\,.
$$

Увеличение~$\tau$ приводит к~уменьшению нормы вектора~$\mathbf{w}$,
однако при этом ни один из параметров не~обращается в~ноль. Это означает, что,
повышая устойчивость модели, гребневая регрессия не~производит отбор
признаков.

\smallskip

\textbf{Лассо Тибширани.}
В данном методе вместо до-\linebreak бав\-ле\-ния штрафного слагаемого к~функционалу
качест\-ва вводится ограничение-неравенство, запрещающее большие
абсолютные значения коэффициентов:
\begin{align*}
Q(\mathbf{w})=||X\mathbf{w}-\mathbf{y}||^2 &\rightarrow \min\limits_\mathbf{w\in\mathbb{W}}\,;\\[9pt]
\sum\limits_{j=1}^W |w_j|&<\theta\,.
\end{align*}

Чем меньше значение~$\theta$, тем больше коэффициентов~$w_j$
обнуляется, таким образом происходит исключение $j$-го признака.
Недостатком этого метода относительно алгоритма, представленного
в работе, является необходимость в разделении выборки на~две части:
для обучения и контроля.

Также при использовании методов регуляризации возникает проблема
выбора константы регуляризации. Для ее вычисления обычно используют
скользящий контроль, что значительно повышает трудоемкость всей
задачи в целом.

\section{Вычислительный эксперимент}

Результатом вычислительного эксперимента является отбор шумовых
и коррелирующих признаков. Тестирование алгоритма производится
на временн$\acute{\mbox{о}}$м ряде продаж нарезного хлеба в зависимости от времени.
Ряд содержит 195~записей. Модель, аппроксимирующая ряд:
$\mathbf{y}\hm=0{,}2256\hm+0{,}1996\boldsymbol{\xi}\hm+0{,}0496\sin(10\boldsymbol{\xi}),$
где~$\boldsymbol{\xi} \hm\in \mathbb{R}^n$~--- регрессионная выборка.
Введем следующие обозначения:~$\boldsymbol{\xi}^0$,
$\boldsymbol{\xi}^1$~--- значение каждого элемента выборки в нулевой
и первой степени соответственно, $\sin(10\boldsymbol{\xi})$~---
поэлементное применение элементарной функции
к вектору~$\boldsymbol{\xi}$. На~ рис.~1 представлены
выборка и аппроксимирующая ее модель.



Пусть матрица плана~$X$ представлена в~сле\-ду\-ющем
виде $X\hm=[\boldsymbol{\chi}_1,\ldots ,\boldsymbol{\chi}_n],$
где $\boldsymbol{\chi}\hm \in \mathbb{R}^m$. В~данном случае она
состоит из трех столбцов: $\boldsymbol{\xi}^0$,
$\boldsymbol{\xi}^1$, $\sin(10\boldsymbol{\xi})$.

\setcounter{figure}{1}
\begin{figure*}[b] %fig2
\vspace*{-3pt}
 \begin{center}
 \mbox{%
 \epsfxsize=161.853mm
 \epsfbox{tok-2.eps}
 }
 \end{center}
 \vspace*{-9pt}
\Caption{Итерационный процесс для матрицы Гессе (случай шумового параметра)}
\label{fig:noise}
\end{figure*}


%\vspace*{2pt}
\begin{center}  %fig1
 \mbox{%
 \epsfxsize=77.76mm
 \epsfbox{tok-1.eps}
 }
 \end{center}
% \vspace*{6pt}
{{\figurename~1}\ \ \small{Данные (точки) и аппроксимирующая модель (кривая)}}



%\pagebreak

\vspace*{12pt}

%\addtocounter{figure}{1}




\smallskip

\textbf{Отбор шумовых признаков.}
Шумовая выборка сформирована при помощи добавления столбца случайных
чисел с нормальным распределением. Модель, аппроксимирующая данные
в эксперименте:
$\mathbf{y}\hm=w_{1}\boldsymbol{\chi}_{1}\hm+w_{2}\boldsymbol{\chi}_{2}
\hm +w_{3}\boldsymbol{\chi}_{3}\hm+w_{4}\boldsymbol{\chi}_{4},$
где $\boldsymbol{\chi}_1\hm=\boldsymbol{\xi}^0$,
$\boldsymbol{\chi}_2\hm\sim\mathcal{N}(0,2)$,
$\boldsymbol{\chi}_3\hm=\boldsymbol{\xi}^1$,
$\boldsymbol{\chi}_4\hm=\sin(10\boldsymbol{\xi})$. При наличии
в выборке шумового элемента процедура сходится за восемь итераций.
На~рис.~\ref{fig:noise} про\-ил\-люст\-ри\-ро\-ва\-ны изменения матрицы
Гессе~$H$ на каждом шаге процедуры.


На 2-й итерации наблюдается резкое отличие диагонального
элемента~(2,\,2). В~течение итераций~2 и~3 он продолжает возрастать,
пока не достигает критической относительной величины\linebreak (принята
эмпирическая оценка отношения максимального элемента матрицы
к минимальному~$10^6$). Далее на 4-й итерации выполняется его
зануление. Таким образом происходит выявление шумового приз\-нака.

На рис.~\ref{fig:A2n}~и~\ref{fig:A134n}~представлены диагональные
элементы матрицы~$A$. Рисунок~3 иллюстрирует изменения второго
диагонального элемента~$\alpha_2$, который соответствует шумовому
параметру модели. Резкий скачок объясняется тем, что на данной
итерации алгоритм находится вблизи локального
минимума~$\mathbf{w}_0$ и, несмотря на возрастание диагональных
элементов матрицы~$H$, знаменатель формулы~(\ref{eq: alfa}) мал.
Далее происходит зануление элементов мат\-ри\-цы Гессе и
соответствующий гиперпараметр~$\alpha$ становится равным нулю.

\begin{figure*} %fig3
\vspace*{1pt}
 \begin{center}
 \mbox{%
 \epsfxsize=162.304mm
 \epsfbox{tok-3.eps}
 }
 \end{center}
 \vspace*{-17pt}
 \begin{minipage}[t]{80mm}
  \Caption{Элемент матрицы~$A$, соответствующий шумовому параметру модели}
  \label{fig:A2n}
  \end{minipage}
  \hfill
  \begin{minipage}[t]{80mm}
  \Caption{Элементы матрицы~$A$, соответствующие нешумовым параметрам модели}
  \label{fig:A134n}
  \end{minipage}
%\end{figure*}
%\begin{figure*} %fig5
\vspace*{1pt}
 \begin{center}
 \mbox{%
 \epsfxsize=161.497mm
 \epsfbox{tok-5.eps}
 }
 \end{center}
 \vspace*{-17pt}
 \begin{minipage}[t]{80mm}
  \Caption{Скалярный гиперпараметр~$\beta$ (случай шумового па\-ра\-мет\-ра)}
  \label{fig:Bn}
\end{minipage}
\hfill
\begin{minipage}[t]{80mm}
  \Caption{Параметры модели~$\mathbf{w}$ (случай шумового па\-ра\-мет\-ра)}
  \label{fig:Wn}
  \end{minipage}
%\end{figure}
%\begin{figure*} %fig7
\vspace*{3pt}
 \begin{center}
 \mbox{%
 \epsfxsize=164.79mm
 \epsfbox{tok-7.eps}
 }
 \end{center}
 \vspace*{-12pt}
\Caption{Итерационный процесс для матрицы Гессе (случай коррелирующих па\-ра\-мет\-ров модели)}
\label{fig:korrelation}
\end{figure*}


На рис.~\ref{fig:Bn}~и~\ref{fig:Wn} представлены скалярный 
гиперпараметр~$\beta$ и процесс изменения параметров модели~$w_i$ соответственно.

\pagebreak


\end{multicols}

\begin{figure} %fig8
\vspace*{1pt}
 \begin{center}
 \mbox{%
 \epsfxsize=162.372mm
 \epsfbox{tok-8.eps}
 }
 \end{center}
 \vspace*{-9pt}
 \begin{minipage}[t]{80mm}
  \Caption{Элементы матрицы~$A$, соответствующие независимым параметрам модели}
  \label{fig:A2k}
  \end{minipage}
  \hfill
\begin{minipage}[t]{80mm}
  \Caption{Элемент матрицы~$A$, соответствующий коррелирующему параметру модели}
  \label{fig:A134k}
  \end{minipage}
%\end{figure*}
%\begin{figure*} %fig10
\vspace*{12pt}
 \begin{center}
 \mbox{%
 \epsfxsize=161.595mm
 \epsfbox{tok-10.eps}
 }
 \end{center}
 \vspace*{-9pt}
 \begin{minipage}[t]{80mm}
  \Caption{Скалярный гиперпараметр~$\beta$ (случай зависимых параметров)}
  \label{fig:Bk}
\end{minipage}
\hfill
\begin{minipage}[t]{80mm}
  \Caption{Вектор параметров модели~$\mathbf{w}$ (случай зависимых параметров)}
  \label{fig:Wk}
  \end{minipage}
\end{figure}

\begin{multicols}{2}



%\smallskip

\textbf{Отбор коррелирующих признаков.}
Выборка с коррелирующими признаками сформирована при помощи
добавления в матрицу плана столбца $1{,}3\boldsymbol{\chi}_2$. 
Таким
образом, модель, аппрок\-си\-ми\-ру\-ющая данные в эксперименте:
$y\hm=w_1\boldsymbol{\chi}_{1}\hm+w_2\boldsymbol{\chi}_{2}\hm+
w_3\boldsymbol{\chi}_{3}\hm+w_4\boldsymbol{\chi}_{4},$
где~$\boldsymbol{\chi}_1\hm=\boldsymbol{\xi}^0$,
$\boldsymbol{\chi}_2\hm=\boldsymbol{\xi}^1$,
$\boldsymbol{\chi}_3\hm=1{,}3\boldsymbol{\xi}^1$,
$\boldsymbol{\chi}_4\hm=\sin(10\boldsymbol{\xi})$.
На~рис.~\ref{fig:korrelation} поэлементно про\-ил\-люст\-ри\-ро\-ва\-на мат\-ри\-ца
Гессе~$H$.


При наличии коррелирующих признаков также наблюдается возрастание
диагональных элементов. Это происходит из-за того, что алгоритм
выбирает ближайший вектор~$\boldsymbol{\chi}$ к вектору~$\mathbf{y}$
(в пространстве векторов матрицы~$X$), а коррелирующий с ним считает
шумовым.

На рис.~\ref{fig:A2k} и~\ref{fig:A134k} 
представлены диагональные элементы матрицы~$A$.



На рис.~\ref{fig:Bk} представлены изменения скалярного
гиперпараметра~$\beta$. На рис.~\ref{fig:Wk} представлены изменения
параметров модели~$w_i$ в течение итерационного процесса.
Коррелирующий параметр~$w_2$ сначала возрастает, а затем стремится
к нулю. Это происходит из-за того, что пространство па\-ра\-мет\-ров
модели многоэкстремально.


\section{Заключение}

В работе предложен способ отсеивания шумовых и коррелирующих
признаков, а также алгоритм оценки ковариационной матрицы параметров
модели. Данный алгоритм имеет следующие преимущества  перед
методами, описанными во введении: (1)~нет необходимости разделения
данных на обуча\-ющую и контрольную выборку; (2)~алгоритм не содержит
никаких параметров, которые необходимо оценивать или задавать
дополнительно (как, например, в методах регуляризации); (3)~добиваясь
сходимости как параметров, так и ги\-пер\-па\-ра\-мет\-ров, предложенный
алгоритм повышает устойчивость выбранной регрессионной модели.

{\small\frenchspacing
{%\baselineskip=10.8pt
\addcontentsline{toc}{section}{Литература}
\begin{thebibliography}{99}


\bibitem{strijov1} 
\Au{Стрижов В.\,В.} Поиск параметрической регрессионной модели в индуктивно заданном
множестве~// Вычислительные технологии, 2007. Т.~1. С.~93--102.

\bibitem{weber} 
\Au{Strijov~V.\,V., Weber~G.-W.} Nonlinear regression model generation using 
hyperparameter optimization~// Computers Math. Appl., 2010. Vol.~60. No.\,4. P.~981--988.

\bibitem{stepwise} 
\Au{Efroymson M.\,A.} Multiple regression analysis~// 
Mathematical methods for digital computers. Vol.~1~/ Eds. A.~Ralston, H.\,S.~Wilf.~--- 
New York: John Wiley and Sons, 1960. P.~191.

\bibitem{lars} 
\Au{Efron B., Hastie T., Johnstone~J., Tibshirani~R.} 
Least angle regression~// Ann. Stat., 2004. Vol.~32. No.\,3. P.~407--499.

\bibitem{lasso} 
\Au{Tibshirani R.} Regression shrinkage and Selection via the Lasso~//
J.~R. Stat. Soc., 1996. Vol.~32. No.\,1.  P.~267--288.

\bibitem{regular} 
\Au{Ильин В.\,А.} О~работах А.\,Н.~Тихонова по методам решения некорректно поставленных задач~//
Успехи математичексих наук, 1997. Т.~1. С.~168--175.

\bibitem{ridzh1} 
\Au{Тихонов А.\,Н.} Решение некорректно поставленных задач и метод регуляризации~//
Докл.\ АН СССР, 1963. Т.~151. С.~501--504.

\bibitem{ridzh2} 
\Au{Hoerl A.\,E., Kennard~R.\,W.} Ridge regression: Biased estimation for nonorthogonal problems~//
Technometrics, 1970. Vol.~3. No.\,12. P.~55--67.

\bibitem{ridzh3} 
\Au{Bjorkstrom A.} Ridge regression and inverse problems. Technical Report.~--- Stockholm: 
Stockholm University, 2001.

\bibitem{belsly} 
\Au{Belsley D.\,A.} Conditioning diagnostics: Collinearity and weak data in regression.~--- 
New York: John Wiley and Sons, 1991.

\bibitem{vif} 
\Au{Marquardt D.\,W.} Generalized inverses, ridge regression, biased linear estimation, 
and nonlinear estimation~// Technometrics, 1996. Vol.~12. No.\,3. P.~605--607.

\bibitem{nabney} %12
\Au{Nabney I.} Bayesian techniques~// Netlab: Algorithms for pattern recognition.~--- 
New York: Springer, 2002. P.~325--366.

\bibitem{mackay}  %13
\Au{MacKay D.} Laplace's method~// Information theory, inference, and learning algotirhms.~--- 
Cambridge: Cambridge University Press, 2005. P.~341--351.


\bibitem{strijov2} 
\Au{Стрижов В.\,В.} Методы выбора регрессионных моделей.~--- М.: ВЦ РАН, 2010.


\label{end\stat}

\bibitem{laplace} 
\Au{Bishop C.\,M.} Linear models for classification~// 
Pattern recognition and machine learning~/ Eds.\ M.~Jordan, J.~Kleinberg, B.~Scholkopf.~--- 
New York: Springer Science\;+\;Business Media, 1960. P.~213--216.
\end{thebibliography}
}
}


\end{multicols}  %10
%\newcommand{\eol}{\end{enumerate}\setlength{\itemsep}{-\parsep}}
%\newcommand{\ang}[1]{\langle{#1}\rangle}
%\newcommand{\infinity}{\infty}
%\newcommand{\mess}[1]{\mbox{\tt #1}}
%\newcommand{\var}[1]{\mbox{\it #1}}
%\newcommand{\order}[1]{\stackrel{#1}\fa}
%\newcommand{\orderr}[1]{\stackrel{#1}\Longrightarrow}
%\newcommand{\infrel}[1]{\stackrel{#1}\Longrightarrow}
%\newcommand{\prog}{\mbox{\tt Prog}}
%\newcommand{\comment}[1]{}
%\newcommand{\set}[1]{\{#1\}}
%\newcommand{\pair}[2]{\langle #1,#2 \rangle}
%\newcommand{\remove}[1]{}
%\renewcommand{\qed}{\hfill\rule{2mm}{2mm}}
%\newcommand{\bull}[1]{\begin{itemize}\item{#1}\end{itemize}}
%\newcommand{\marg}[1]{\marginpar{\small #1}}


\renewcommand{\figurename}{\protect\bf Figure}
\renewcommand{\tablename}{\protect\bf Table}

\def\stat{frenkel}


\def\tit{SEAMLESS ROUTE UPDATES IN SOFTWARE-DEFINED NETWORKING 
VIA QUALITY OF~SERVICE COMPLIANCE VERIFICATION}

\def\titkol{Seamless route updates in software-defined networking via 
quality of service compliance verification}

\def\autkol{S.\,L.~Frenkel and~D.~Khankin}

\def\aut{S.\,L.~Frenkel$^1$ and~D.~Khankin$^2$}

\titel{\tit}{\aut}{\autkol}{\titkol}

%{\renewcommand{\thefootnote}{\fnsymbol{footnote}}
%\footnotetext[1] {The 
%research of Yuri Kabanov was done under partial financial support of the grant 
%of RSF No.\,14-49-00079.}}

\renewcommand{\thefootnote}{\arabic{footnote}}
\footnotetext[1]{Institute of Informatics Problems, Federal Research 
Center ``Computer Science and Control'' of the Russian Academy of Sciences,
 44-2~Vavilov Str., Moscow 119333, Russian Federation, \mbox{fsergei51@gmail.com}}
\footnotetext[2]{Computer Science Department, Ben-Gurion University of the Negev, 
Beer-Sheva 84105, Israel, \mbox{danielkh@post.bgu.ac.il}}


\index{Frenkel S.\,L.}
\index{Khankin D.}
\index{Френкель С.}
\index{Ханкин Д.}

\def\leftfootline{\small{\textbf{\thepage}
\hfill INFORMATIKA I EE PRIMENENIYA~--- INFORMATICS AND
APPLICATIONS\ \ \ 2018\ \ \ volume~12\ \ \ issue\ 4}
}%
 \def\rightfootline{\small{INFORMATIKA I EE PRIMENENIYA~---
INFORMATICS AND APPLICATIONS\ \ \ 2018\ \ \ volume~12\ \ \ issue\ 4
\hfill \textbf{\thepage}}}

\vspace*{4pt}

\Abste{In software-defined networking (SDN), the control plane and the data 
plane are decoupled. This allows high flexibility by providing abstractions 
for network management applications and being directly programmable. 
However, reconfiguration and updates of a~network are sometimes inevitable due 
to topology changes, maintenance, or failures. In the scenario,  
a~current route~$C$ and a set of possible new routes~$\{N_i\}$, where one of the 
new routes is required to replace the current route, are given. There is a chance that 
a~new route $N_i$ is longer than a~different new route $N_j$, but $N_i$ is 
a~more reliable one and it will update faster or perform better after the update 
in terms of quality of service (QoS) demands. 
Taking into account the random nature of the network functioning, 
the present authors supplement the recently proposed algorithm by Delaet
\textit{et al}.\ for route updates with 
a~technique based on Markov chains (MCs). As such, an enhanced algorithm 
for complying QoS demands during route updates is proposed
in a~seamless fashion. First, 
an extension to the update algorithm of Delaet \textit{et al}.\ 
that describes the transmission of packets through a~chosen route and compares 
the update process for all possible alternative routes is suggested. Second, several 
methods for choosing a~combination of preferred subparts of new routes, resulting 
in an optimal, in the sense of QoS compliance, new route is provided.} 

\KWE{software-defined networking; Markov chains; quality of service}

\DOI{10.14357/19922264180408}


\vspace*{8pt}


\vskip 12pt plus 9pt minus 6pt

 \thispagestyle{myheadings}

 \begin{multicols}{2}

 \label{st\stat}

\section{Introduction}
\label{s:Intro}

\noindent
Software-defined networking is an emerging network paradigm, in which the 
control plane is decoupled from the data plane enabling centralized control 
logic. Such a~dynamic network may require frequent modifications and updates to 
the network topology and configuration. 
Also, the network topology is available to the centralized control entity, there, 
due to this flexibility, it is possible to perform offline optimized calculations.

Network functions virtualization (NFV) allows replacing traditional network 
devices with software that is running on commodity servers. This software 
implements the functionality that was previously provided by dedicated hardware. 
Network functions virtualization
 allows services to be composed of virtual network functions (VNF) hosted on 
different data centers. Software-defined networking, 
when applied to NFV, helps in addressing challenges 
of dynamic resource management and intelligent service 
orchestration~\cite{rao_sdn_2014}. Sometimes, traffic is often required to pass 
through and be processed by an ordered sequence of possibly remote 
VNFs~\cite{ghaznavi_service_2016}. For example, traffic may be required to pass 
through intrusion detection system, proxy, load balancer, or a~firewall. 
Such concatenation of services is called \textit{service function chaining} 
(SFC).

Consider, for example, two communicating parties in a~network featuring complex 
network topology (e.\,g., Small-world network), and the communication flow is 
passed over a~series of VNFs. It may be the case that the network operator is 
required to move the communicating flow to a~different path due to QoS 
requirements or other possible cost considerations. We are interested 
to model the anticipated expected number of steps until the update is complete 
given a~possible new route following the required QoS demands, e.\,g., 
delay, communication rounds, cost, etc. 

%Aforesaid dynamic networking requires frequent modifications and updates to the network. 
Let us consider a pair $(C, \{N_i\})$ where a~current route~$C$ from~$s$ to~$d$ 
is scheduled to be replaced by a new route from the set~$\{N_i\}$, each from~$s$ 
to~$d$ either. Let us model each route as an ordered list of network elements, such 
as VNFs (SFCs) or generally saying routers. Each new route~$N_i$ is constructed 
during the update process, and thus, certain delays may be introduced due to
 initial packet processing or due to possible losses. 
 %There, the eventual arrival of packets along the new route during the update process is critical for successful route update. Another possible example is when the routes are SFCs, and the requirement is to update a current chain to a new one, different service chains may exhibit different delays. 

The design goals must be achieved by constructing effective algorithms for 
efficient packet QoS routing in NFV/SDN computer network. Depending on the 
QoS metric, the lower (e.\,g., for reliability) or upper (e.\,g., for a~delay) 
constraints represent the desired bounds that the orchestration must meet. 
Since different configurations could meet these bounds, the designer should also 
optimize against a~specific metric by using these both ends of the extreme. 

Methods based on integer linear programming (ILP) were proposed in several works 
(see section~\ref{sec:related_work}). The difficulty of using tools based on ILP 
 in the operational work of an administrator is that in view of the possible 
 infeasibility of the resulting solution, it may take not a~few resources (time, efforts) 
 until acceptable QoS values can be ensured.

We consider the use of ``design via verification'' approach, suggesting a~method 
for complying QoS demands. The method is based on augmenting the update algorithm with
a~verification logic. Namely, we suggest the use of 
\textit{Probabilistic real-time Computation Tree Logic} 
(PCTL)~\cite{hansson_logic_1994} for expressing real-time and probability in systems. 
Using PCTL, we can express the probability for a~process to complete after 
a~certain number of steps along an execution path and verify the selected route 
for the update.


%Assume that packets are sent from a source node $s$ to a destination node $d$ along the current route. After the update process is finished, packets will be forwarded from $s$ to $d$ along the new route. 
Delaet \textit{et al.}\ proposed a~multicast-based scheme for seamlessly updating 
a~current route to a~new one~\cite{delaet_seamless_2015}. 
According to the multicast scheme, the controller instructs 
a~router to temporarily have two $(s,d)$ entries in the routing table. When 
a~router $r \neq d$ receives a~packet from~$s$ to~$d$, it sends the packet 
according to the forwarding instructions of all of its $(s,d)$ routing 
table entries. When two identical copies of a~packet that was multicasted 
over the current and new portion of a~route arrive, the controller can dismantle 
the current route, as the new route is ready. During the update process, packets 
should not be lost, no cycles should be formed, and communication should not 
be disrupted.

%Taking into account the random nature of the network functioning, we supplement the algorithm for route updates introduced by Delaet et al. in \cite{delaet_seamless_2015}, with a technique based on Markov chains. In our extension of the algorithm, we describe the transmission of packets through a chosen route and compare the update process for all the possible alternative routes that are candidates for replacement. 

Our contribution is a model for a successful route update, including its 
intermediate steps, as MC states, each with 
a~given probability. With our model, we are able to characterize the quality of 
an update by expected number of steps in the~MC. 
%We use Markov chains to characterize the quality of the update service, and represent the expected number of steps in the Markov chain as the quality of a successful update. While, the probability for an update event 

We suggest an enhanced update method for the network administrator to augment 
his decision regarding QoS demands in terms of various network parameters and 
possible failure of the update process. Moreover, in contrast to other works, 
we are able to provide a~version of an algorithm that can perform real-time QoS
 assessment during a~route update, for each subpart of a~route. At last, using 
 our method, it is possible that the active new route will be comprised of subparts 
 of different new routes, providing optimal route update service in regard of 
 required network QoS. 

%We assume that each new route is legal. 
%However, mixing subroutes belonging to different routes may result in inconsistent state or a cycle formed in the network. We use different 
%
%
%
%We model the update process as a service, namely as a VNF, and we use Markov chains to characterize the quality of the update service. Using the expected number of steps in the Markov chain representing the update, we abstract the quality of the update service. We calculate for each possible new (sub-)route the expected number of steps required to update an old (sub-)route successfully. Subsequently, the old route is updated to the new route which requires less number of steps with high probability. We supplement the seamless update algorithm proposed by the authors of \cite{delaet_seamless_2015} with the model in this work.

%The virtualized service implementing the update algorithm will provide a recommendation for an optimal choice of a route, based on the performed calculations. Fundamentally, we create a QoS VNF for seamlessly updating a route, regarding network parameters, and taking into consideration the complexity and possible failures of updating a route. In case there exist several alternatives for a route update, there is a chance that one of the possible new routes is much longer, however, a more reliable one, and as such will update faster. 
%
%
%One of the important requirements to modification process is that the update process should not form congestion in the network, nor result in time delays, and not lose any packets. 
%
%
%Additionally, we provide an enhanced version of an algorithm that can perform the quality of service assessment during the update process, for each subpart of the new route. 
%
%We propose a directed graph $G=(V,E)$, for representing the possible legal combinations of sub-routes. The set of common nodes to $(C, \{N_i\})$ subdivides the old route and each of the new routes to sub-routes. For two sub-routes represented by the nodes $u,v \in V$, the sub-route $v$ can be launched after $u$ if and only if there exists a directed edge $(u,v) \in E$. Otherwise, the launch of $v$ after $u$ is forbidden and can result in a cycle formed in the network.


%The results of this work helped to develop an operating strategy for a network administrator, supporting both, seamlessly updating a route, and providing QoS requirements. 

Extended abstract of this work appeared as a conference paper 
in~\cite{frenkel_predicting_2017} which presented preliminary results. 
In this work, we describe in detail the system settings and bring new results 
by providing two additional algorithms.
{\looseness=1

}

In the following section, we overview the related work. Next, we provide 
the required definitions and the system settings and describe the MC 
characterization of the network. Further, we describe different update setting, 
accordingly accompanying algorithms and data structures, used for QoS assessment 
during route updates.

\vspace*{-9pt}

\section{Related Work}
\label{sec:related_work}

\vspace*{-2pt}
%The design goals must be achieved by constructing effective algorithms for efficient packet QoS routing in NFV/SDN computer network. %These algorithms, which must enable an administrator to orchestrate the existing services exported by remote providers, were considered in \cite{martins_clickos_2014, zaalouk_orchsec:_2014}. Likewise, the functional behavior (e.g., services being deprecated by their providers), as well as changes in the non-functional behavior of the orchestrated services (e.g., an increased execution time) were also considered.

%Depending on the QoS metric, the lower (e.g., for reliability) or upper (e.g., for delay) constraints represent the desired bounds that the orchestration must meet. Since different configurations could meet these bounds, the designer must also optimize against a specific metric by using these both ends of extreme.

\noindent
Quality of service routing using multipath was proposed in~\cite{devi_approach_2015}. 
The routing algorithm, initially, eliminates all links that do not meet the 
bandwidth requirements. Then, it finds disjoint shortest paths based on 
the residual network graph in each iteration.

The work~\cite{egilmez_distributed_2012} proposed a~QoS optimized routing 
over multidomain OpenFlow networks managed by a~distributed control plane, 
where each controller performs optimal routing within its domain. 
The QoS routing problem was posed as a~constrained shortest path (CSP) problem, 
and the proposed solution computes a~near-optimal route, based on LARAC 
(Lagrange relaxation based aggregated cost)
algorithm~\cite{juttner_lagrange_2001}. The proposed algorithm is an approximation 
algorithm; it always gives a~suboptimal solution.

For traditional network architecture, a~routing strategy approach based on 
ILP was introduced in~\cite{yu_efficient_2013}.
 The main disadvantage of using ILP is that the problem is NP-hard. 
 Additionally, ILP cannot be applied to probabilistic values. 
 Using linear programming (not limited to integers) rounded to integer solutions 
 will not yield an optimal solution.
 

Route updates are extensively researched in SDN~\cite{foerster_survey_2016}, 
standing on the work by Reitblatt \textit{et al.}\ where requirements for SDN 
updates were examined. This work focused on per-packet consistency property, 
stating that packets have to be forwarded either using the initial configuration 
or the final configuration but never a~mixture of them, throughout the update 
process~\cite{reitblatt_consistent_2011}. The authors proposed 
a~2-phase commit technique which relies on packets tagging so that either of 
the rules is applied. However, such technique wastes critical network resources 
and complications are formed due to packet tagging~\cite{foerster_survey_2016}. 
Further, Delaet \textit{et al.}\ showed in~\cite{delaet_seamless_2015} 
that using a~careful multicast during route updates provides 
a~better working solution.

Hogan and Esposito propose in~\cite{hogan_stochastic_2017} the use of
 Bayesian networks for delay estimation as a~traffic engineering tool and model 
 the path selection problem using a~risk minimization technique. 
 However, the authors state that the accuracy of their model is limited by its 
 ability to correctly identify dependencies in the data. In our work, 
 we suggest a~general tool for probabilistic verification of any network parameter, 
 which does not depend on variance within the dataset.
 
 

In~\cite{mcgeer_safe_2012}, an update protocol proposed where packets are 
sent to the controller during updates; such approach adds 
a~significant cost to the control plane bandwidth~\cite{delaet_seamless_2015}. 
In~\cite{mcgeer_correct_2013}, an algorithm to find 
a~safe update sequence expressed as a~logic circuit has been proposed. 
However, the algorithm 
requires a~dedicated protocol which is not currently 
supported~\cite{foerster_survey_2016}. The authors 
of~\cite{katta_incremental_2013} propose to perform the 2-phase update 
scheme from~\cite{reitblatt_consistent_2011} incrementally, making the update longer. 
%For a thorough review of route updates, the reader is referred to \cite{foerster_survey_2016}.






Software-defined networking allows the involvement of the network administrator into the network 
management during route udpdates and, in particular, during packet transmission. 
Thus, it would be highly desirable to support the decision making process 
with the right tools. Our novelty is exactly such tool, for augmenting 
online decision making of the network administrator during network management 
in a~stochastic environment.
%In this work, we propose a technique to optimize the update process by selecting the preferred (sub-)route in order to reduce the update time. We use the expected number of steps for successfully completing the update as a QoS metric, and extend the algorithm by Delaet~et~al. with Discrete Time Markov Chains (DTMC) for finding (sub-)routes which are preferred in terms of QoS. % As such, we propose to use the route updates algorithm from \cite{delaet_seamless_2015} as a virtual service for network updates per QoS requirements.

%The interaction of software components have a greater weight in NFV context, which may lead to stochastic-like behavior 

%At present, certain routing algorithms (including $k$ Edge-Disjoint) are based on the shortest path (SP) problem solution \cite{wood_toward_2015}. However, the method proposed by Wood et al. is generic and valuable only in the case of request arrival, and do not consider certain additional important requirements, such as removal or priorities of requests. 

%Several approaches for efficient SP-based QoS routing have been recently proposed in \cite{buchbinder_improved_2006}, where the authors introduce and analyze a centralized algorithm for an online scheduling and routing of arbitrary sequence of communication requests. 

%Unsplittable (single-path) assignment for each request of QoS routing is probably competitive with the best possible splittable (multipath assignment).

The work by Delaet \textit{et al.}~[4] introduced the Make\&Activate-Before-Break 
approach for seamless
route update in SDN. The authors described in a~high-level the multicasting-based 
update, which we
employ in this work. Also, they introduced a~controller-based method for 
verifying the correctness
of a~new route before the traffic redirection. Dinitz \textit{et al.}~[16] 
extended the work~[4] to the general
case of several dependent (via shared links) routes pairs. The routes update 
problem was proved to
be NP-hard~\cite{17-aaa}. The authors of~[16] suggested the use of 
artificial intelligence (AI) methods for 
solving the problem. As a~basis for AI-based solutions, Dinitz 
\textit{et al.}\ proposed a dependence graph model describing the current
state of the problem instance at any replacement stage. 
In addition, route readiness verification similar
to that in~[4] was implemented in~[16] as a high-level network protocol.

In this work, we investigate a different problem; we consider the route updates 
problem from a~QoS
perspective and describe in high-level both the prediction and the update processes.

\vspace*{-9pt}

\section{Preliminaries and Definitions}

\vspace*{-2pt}

\noindent
The basic system settings are as follows. 
For a~(route) sequence~$X$, we denote by~$x_i$ the $i$th element in it.
In a~(directed) communication network, 
we are given a~route~$C$ from source~$s$ to destination~$d$. 
Additionally, we are given a~set of different new routes~$N_i$, each going from~$s$ 
to~$d$. We model each route as an ordered set of network nodes connected by network 
links. We assume that neither of the routes contains cycles. 
Each router in a~route matches a~packet from~$s$ to~$d$ 
and forwards the packet to the next router in order. After the update 
is complete, each router in the new route should forward the packets from~$s$ 
to~$d$ to the next router in order along the new route. 

In our work, we consider the route replacement problem as a~sequence of 
subroutes replacements.
The routes replacement subsystem was in great detail described by Dinitz 
\textit{et al.} in~\cite{dinitz_dependence_2017}. We borrow
from~[16] the relevant parts which we briefly describe here.

\smallskip

\noindent
\textbf{Definition~1.} We  define a~subset from $a\in X$ to $b\in X$ of an ordered
set~$X$, when $a$ precedes~$b$, as~a~subroute from~$a$ to~$b$, and denote such subroute by
$[a,b]$.

\smallskip

 

\textbf{Subroutes.} The current route~$C$ subdivides each new route 
to~$k$~common subroutes (a~subroute may consist of one router in the simplest case) 
and $k-1$ noncommon subroutes. 
For illustration, see Fig.~1.
In Fig.~1 and figures below, the current route is depicted
in a~light grey color full nodes, connected with
solid edges. The new route is depicted in white colored nodes, connected with
dashed edges. The common nodes are depicted as shaded. 
If there are several new
routes, the nodes of each route are filled with a~designating pattern. 
Additionally, for easier reading,
when it is possible, we denote subroutes of some route~$X$ as~$X^\prime$, $X^{\prime\prime}$, 
etc. In other cases, a~subroute~$j$
of a~new (current) route~$i$ is denoted as $N_j^i (C_i^j)$. 
Similarly, routers of some route~$X$ are denoted by~$r^\prime$,
$r^{\prime\prime}$, etc.

 { \begin{center}  %fig1
\vspace*{1pt}
 \mbox{%
 \epsfxsize=78.631mm 
 \epsfbox{fre-1.eps}
 }


\vspace*{3pt}


\noindent
{{\figurename~1}\ \ \small{Route $C$ with two possible new routes sharing a~link}}
\end{center}
}

\vspace*{6pt}






In the example in Fig.~1, 
noncommon new subroutes 
of route~$N_1$ are denoted by~$N^1_1=[s,r_2]$ and~$N^2_1=[r_2,d]$, while the noncommon new 
subroutes of~$N_2$ are denoted by~$N^1_2=[s,r_1]$, $N^2_2=[r_1,r_3]$, 
$N^3_2=[r_3,r_2]$, and~$N^4_2=[r_2,d]$. 

Note that in general, the order of common subroutes along~$C$ and along~$N$ 
can be different. See, for example, the common subroutes of~$C$ and~$N_2$ in 
%Figure \ref{fig:two_routes}.
Fig.~1.

\smallskip

\noindent
\textbf{Definition~2.} A~new noncommon subroute of~$N$ from router~$a$
to router~$b$ is legitimate for update only if~$a$ precedes~$b$ on the route~$C$.

\smallskip

Definition~2 guides us on which subroutes can be launched without creating routing cycles in the
network system. (See~[4] for details.)


When an update of a~subroute~$N^\prime$ from router~$r$ to~$r^\prime$ is finished, 
the update flow goes along~$C$ from~$s$ to~$r$, continues along~$N^\prime$ up to~$r^\prime$, 
and finishes along~$C$ from~$r^\prime$
 to~$d$. 
For illustration, see the result of launching~$N^4_2$ in Fig.~2.

 { \begin{center}  %fig2
\vspace*{-1pt}
 \mbox{%
 \epsfxsize=78.631mm 
 \epsfbox{fre-2.eps}
 }


\vspace*{3pt}


\noindent
{{\figurename~2}\ \ \small{$N^4_2$ was launched}}
\end{center}
}

\vspace*{4pt}


 

 Note that launching a~currently nonlegitimate new subroute, for example,~$N^3_2$ 
 in Fig.~1, is forbidden since it will form a~cycle 
 resulting in packets circulating and overwhelming the network. 

\textbf{Dynamics of the system.}
%\label{sec:dynamics} 
Dinitz \textit{et al.}\ performed a~detailed analysis on the dynamics of a~subroutes
system. After an update of a~subroute is complete, the set of current subroutes~$C$ 
and the set
of new subroutes~$N$ are recalculated. This may result in different system of subroutes. For example,
see Fig.~2 where after the launch of $N^4_2$ from the example in Fig.~1, 
the sets of subroutes are
recalculated. As a~result, we obtain different subroutes (for clarity, the previous labels are kept). See
also~[16] for details and extensive analysis.

\vspace*{-4pt}

\subsection{Markov chain characterization of~the~network~states}

\noindent
We characterize execution of some (sub)route in the network by 
a~packet delay time between the (sub)route's common sender and common destination 
routers as well the probability of a~packet drop. Let us for now define our 
network routing model (conceptual model) informally in the following terms. 
Delay of a~packet is obtained using a~physical delay and the total processing 
time in the router. We consider that transmission of packets in 
a~network can have a~random behavior, caused by the random character of both, 
the input, and possible loss of packets. There we are interested in 
a~probabilistic model, namely, a~Markov model. In order to fully characterize 
the network as an~MC, the internal state of each router 
(and, in particular, the buffer occupancies), as well as the characteristics
 of all flows, need to be expressed as states in the chain. 

However, such approach would result in an enormous and intractable number of states. 
Therefore, to simplify these computations, let us characterize the delay time as 
an abstract variable~$t$. This abstract variable can be interpreted in different ways, 
e.\,g., the current processing queue length and a~packet transmission rate of the link, 
or possibly a~fixed value, such as an interval between the beginning of 
a~packet transmission after being processed in some node and the end of processing 
at the next node. 

We describe the functioning of the network in the transmission of packets 
as transitions of a~discrete-time MC (DTMC). The state space corresponds to the set 
of nodes such that 
the transmission of a~packet from a~node that has finished processing the packet 
to the next node corresponds to the transition of the chain to the next state.


Discrete-time MC is defined as a~tuple $D\linebreak =(S, s_0, P)$. In the tuple, $S$ is 
the finite set of states, $s_0\in S$ is the initial
state, $P:S \times S \rightarrow [0, 1]$ is the transition probability matrix in 
which $\forall s\in S$, $\sum\nolimits_{s' \in S} P(s,s') = 1$. 
For any two states $s, s' \in S$, if $P(s,s')>0$, then~$s'$ is the successor of~$s$. 
For a~subset of states $T \subseteq S$, the probability of moving from a~state~$s$ 
to any state $t \in T$ in a~single step is denoted by $P(s, T)$ and is given by 
$P(s,T)=\sum\nolimits_{t \in T} P(s, t)$. 
%The row $P(s,:)$, in the transition matrix $P$, contains the probabilities of moving from $s$ to its successors, while the column $P(:, s)$ contains the probabilities of entering the state $s$ from any other state.

\vspace*{-6pt}

\subsection{Verification syntax}

\noindent
For implementation of our PCTL-based model, we use PRISM~--- 
probabilistic model checker~\cite{kwiatkowska_prism_2011}. There, we follow 
PRISM property specification language. Here, we briefly describe the essential 
syntax while more details can be found in~\cite{noauthor_prism_nodate}.

Given a property~$\Psi$, we say that~$\Psi$ is true with probability~$p$ 
and write that as
$P_p [ \Psi ]$. If the probability~$p$ is unknown, PRISM allows, for DTMC, 
writing properties queries of the form $P_{=?}[ \Psi ]$, meaning 
``what is the probability that~$\Psi$ is true?''. Additionally, it is possible 
to use a~time bound and write properties queries such as 
$P_{=?}[F^{\leq T} \Psi]$, meaning ``what is the probability that~$\Psi$ 
is true after less than~$T$~steps?''. At last, it is possible to compute 
properties such as expected time or expected number of steps. 
For example, $R_{=?}[F \Psi]$, meaning ``what is the expected number of 
steps until $\Psi$ is true?''. 
%\section{Model Settings}
%, and a subroute of route $X$ from router $a$ to router $b$ is specified by $[a,b]_X$

%When a new subroute of $N$ that is scheduled to update a current sub-route of $C_i$ is launched, the route $C$ is updated such that the updated sub-route is replaced by launched sub-route, and the new sub-route is now part of the current route $C$.

\setcounter{figure}{3}
\begin{figure*}[b] %fig4
\vspace*{-6pt}
 \begin{center}
 \mbox{%
 \epsfxsize=149.177mm 
 \epsfbox{fre-3.eps}
 }
 \end{center}
\vspace*{-9pt}

 \Caption{New routes~$N_1$~(\textit{a}) and $N_2$~(\textit{b}) and
 MC states for~$N_1$~(\textit{c}) 
and~$N_2$~(\textit{d})}
 \label{fig:routes_dtmc_example}
\end{figure*}



\vspace*{-6pt}

\section{Prediction of Preferred Update}
%\section{Prediction of Preferred Update}
\label{sec:dtmc}

\noindent
The states of a~DTMC describe the nodes in the new route and the transition 
probabilities in the chain represent the possible delay or 
a~packet loss in the routers along the new route. The
states are defined as 
$\{s_1, \ldots , s_n\}$ where~$n$ is the number
  of nodes in the new route. 
The network achieves the state~$s_i$ if a packet has reached the $i$th node. 
For example, in Fig.~3, the self-transition 
edge represents the probability for a~delay due to packet loss, rules installation 
at the router, or congestion on the router-controller link, while the 
forward transition edge represents the probability for 
a~successful transition to the next state. These probabilities can be estimated 
from network statistics (see, for example,~\cite{hogan_stochastic_2017}). 
The labels on edges are the probability values, when edge has no label
 means probability~1.
 
 The initial probability distribution of states is given by the vector~$P_0$ of size~$n$. 
We can determine the prob-\linebreak\vspace*{-12pt}
 
 %\linebreak\vspace*{-12pt}

{ \begin{center}  %fig3
\vspace*{-0.5pt}
  \mbox{%
 \epsfxsize=77.518mm 
 \epsfbox{fre-4.eps}
 }


\end{center}

\vspace*{-3pt}

\noindent
{{\figurename~3}\ \ \small{Probability as a~function of number of steps to update routes~$N_1$~(\textit{1})
 and~$N_2$~(\textit{2})}}
}

\vspace*{12pt}



\noindent
ability that a~particular route delays the update process 
by~$k$, that is, the number of steps required for a~successful update is given by 
$p(k)=P_0 P^k$. Using this characteristic, which is, in fact, the 
probability distribution of the number of steps $P(k < x)$, one can 
calculate various properties like average delay time for the new route, 
maximum or minimum number of steps to update, etc.
 
 Consider the example illustrated in Fig.~4. 
Figure~4\textit{a} illustrates the current route~$C$ and a candidate new route~$N_1$. 
Figure~4\textit{b} shows the same current route~$C$ with another candidate 
new route~$N_2$. 
Figures~4\textit{c} and~4\textit{d} 
show the MCs for new routes~$N_1$ and~$N_2$, accordingly, with given transition 
probabilities.

During the update process, packets are sent along the current and the new routes. 
Since the new route is\linebreak\vspace*{-9.5pt}

\columnbreak

\noindent
 not operational yet, packets can be delayed due to 
congestion on certain nodes or due to switch configurations. 
%
For example, if routing rules have not yet been installed in some switch, then an 
arriving packet is sent to the controller~\cite{onf_openflow_2015}. The controller 
then decides reactively on further actions whether to install an appropriate rule 
for the packet. Also, the controller may be busy with other work and not respond 
immediately. Those packet processing actions may delay the update process. 
In the case buffer becomes full, for example, if the network is being congested, 
packets may be dropped. There, the transition to the next state during the 
update process depends on the likelihood of a~delay or a~loss of a~packet in the 
current state. 

In the example, the number of steps required for launching~$N_2$ is smaller than 
the number of steps required for launching~$N_1$. However, due to a higher likelihood 
of delays along the route~$N_2$, it is possible that~$N_1$ is preferred having 
a~higher probability for a~successful update. The network administrator may ask 
which new route is recommended for the update process, considering the expected 
number of steps required for the update. 
%
That is, updating paths requires the operator to decide 
on the possible choice of a~subroute for the next step. 
One should consider the possibility of including a~decision tool augmenting the 
controller during route updates. 

There were many attempts to use the LP/ILP 
approach, as it was already mentioned above (see, e.\,g.,~\cite{juttner_lagrange_2001}), 
but they have encountered the same difficulties, especially when taking 
into account online implementation. We show that it is possible to describe 
the routing process as DTMC. Thus, taking into consideration~$O(n^3)$ worst case 
computation complexity, we consider using the ``design via verification'' 
mentioned above based on PCTL verification, similar to the one used in 
PRISM~\cite{kwiatkowska_prism_2011}.


We have calculated the probability for a~successful update as a~function of 
number of steps for routes~$N_1$ and~$N_2$ from the example in 
Fig.~\ref{fig:routes_dtmc_example}. See Fig.~3 
where this function is shown. Curve~\textit{1}
represents the plot for~$N_1$ and curve~\textit{2} represents
 the plot for~$N_2$. 

Observe that after~20~steps, both new routes will be launched with probability~1 
which can be written as 
$$
P_{1}\left[F^{>20}N_1\right]=P_{1}\left[F^{>20}N_2\right]=1\,.
$$
The expected number of steps required for~$N_1$ is smaller than the required for~$N_2$:
$$
R \left[F~N_1\right] < R \left[F~N_2\right]\,.
$$
However, the probability for successfully updating in less than~15~steps 
is higher for route~$N_2$ ($0.55 \pm 0.040$ for~$N_1$ and 
$0.717 \pm 0.036$ for~$N_2$, based on~99\% confidence level):
$P_{0.717 \pm 0.036}\left[F^{\leq 15} N_2 \right].$

\vspace*{-6pt}


\section{Route Updates per~Quality~of~Service}
\label{sec:updates_qos}

\vspace*{-2pt}

\noindent
In this section, we show algorithm that we propose for various settings. 
First, we show an enhancement for the sequential update algorithm 
from~\cite{delaet_seamless_2015}, which during the update process decides on 
preferred subroute from the set of possible subroutes as part of QoS requirements. 
In the multicast-based update, several methods were proposed 
in~\cite{delaet_seamless_2015} for eliminating duplicated packets. 
In the case the common destination router is not able to immediately eliminate 
duplicated packets, the algorithm begins the update from the end, 
ensuring a~correct update process~[4].



\begin{algorithm*} %alg1
 \setlength{\algowidth}{100mm}
 \setlength{\hsize}{\algowidth}
 \caption{Update per QoS Algorithm}
 \label{alg:update_per_qos}

%\hrule
%\vspace*{2pt}
%\centerline
%{\textbf{Algorithm~1:} Update per QoS Algorithm}\par

%\vspace*{2pt}

%\hrule
 \small
 
 %\Input
 {directed graph $G$} 
 
 \BlankLine
 \tcc{$A$ is a collection of nodes} $A \leftarrow$ choose nodes from $G$ with in-degree $0$ \\
 
 \Repeat {out-degree of node $N^t_i > 0$}
 {
 \ForEach{$v \in A$ \label{alg:inner_loop}}
 {
 calculate $R[F~v]$ \\
% calculate the expected QoS for this node as described in Section \ref{sec:updates_qos} \\
 }\label{alg:end_inner_loop}
 
% $N^t_i \leftarrow$ choose the node that maximizes QoS \label{alg:choose_qos}\\ 
 $N^t_i \leftarrow \argmax_{v} (R[F~v])$ \label{alg:choose_qos} \\
 launch $N^t_i$ \\
 update $C$ accordingly \\
 merge any new and common subroutes as described in section~3 \\ 
 $A \leftarrow$ choose nodes neighboring to $N^t_i$ \\ 
 }
 
 \BlankLine 
 
\end{algorithm*}





 
%The algorithm starts from any node with in-degree 0 since it means that such node has no precedence dependence. Updating is completed when the algorithm arrives to a node with out-degree zero, which would be the last subroute to launch.


After that, we show an algorithm that chooses the subroutes for update arbitrary, 
assuming that the common destination node will not leak duplicated packets. 
However, the packets sending rate along the new subroute need to be temporarily limited~[4].

At last, we present a supplementing algorithm that suggests which subroutes can 
be updated in parallel.

%The set of common nodes for each pair of routes subdivides the routes to sub-routes relatively to each other. 

\vspace*{12pt}

\subsection{Sequential update}

\noindent
Let us begin the update from the end, namely, from the last alternative 
subroute of any new route. Provably, this prevents the formation of 
cycles~\cite{delaet_seamless_2015}. In order to represent all possible choices 
of a~path from a current state of the update process to the end of the update process, 
we propose to use a directed graph which nodes are the new, legitimate for launching, 
subroutes of the network. The edges of the graph represent a~legal order of launching 
new subroutes. Each path in this graph from a~current node to the last node in 
the path represents a~legal combination of chosen subroutes. The update process is 
continued as long as there is a~possible node to transition to. 

Let us examine the two possible new routes~$N_1$ and~$N_2$ that can replace the 
current route~$C$ from the example depicted in Fig.~1. 
The new route~$N_1$ is composed of~$N^1_1$ and~$N^2_1$, while the new route~$N_2$ 
composed of~$N^1_2$, $N^2_2$, $N^3_2$, and~$N^4_2$. Starting from the end, the only 
new subroutes that are allowable to launch are~$N^2_1$ and~$N^4_2$. 
Assume that based on the DTMC calculations performed as described in section~4, 
the subroute~$N^4_2$ is chosen for update. After the update of the subroute is 
complete, the current route~$C$ is composed of not updated yet part of the old 
route and~$N^4_2$. See Fig.~2 where the change in~$C$ 
is depicted.

After the subroute~$N^4_2$ is launched, we arrive at a~smaller problem in which 
less subroutes are left to update. Due to dynamics of the system 
(see section~3), some new subroutes can merge into a~single new subroute.
See Fig.~2 where after~$N^4_2$ was launched, the 
new subroutes~$N^3_2$ and~$N^2_2$ are merged into a~single subroute. Now, one 
can launch either~$N^1_1$ or~$N^2_2$ merged with~$N^3_2$. Assume that we choose to 
launch~$N^1_1$, which launch
 finishes the update. The route~$C$ updated to~$N^1_1$ 
and~$N^4_2$. See Fig.~5 illustrating that.


Figure~6 shows the directed graph that represents 
the possible update sequences. Initially, the subroutes that %\linebreak\vspace*{-12pt}
 are legal 
for launch are~$N^2_1$ and~$N^4_2$. As such, these are
the only subroutes that
 have in-degree~0. Launching~$N^3_2$
 is forbidden; hence, there is no node in the 
 graph~$G$ that represents this subroute. After launching~$N^4_2$, we\linebreak\vspace*{-12pt}
 
 \setcounter{figure}{4}

{ \begin{center}  %fig5
\vspace*{12pt}
 \mbox{%
 \epsfxsize=78.631mm 
 \epsfbox{fre-5.eps}
 }


\vspace*{3pt}


\noindent
{{\figurename~5}\ \ \small{$N^1_1$ was launched}}
\end{center}
}

\vspace*{6pt}

{ \begin{center}  %fig6
\vspace*{1pt}
 \mbox{%
 \epsfxsize=36.428mm 
 \epsfbox{fre-6.eps}
 }


\end{center}


\noindent
{{\figurename~6}\ \ \small{Graph 
representation for possible update paths for routes update example from Fig.~1}}

}

%\vspace*{6pt}

\noindent
  can 
 proceed by launching~$N^1_1$ or~$N^2_2$. However, if~$N^2_1$ was launched first, 
 it would be forbidden to launch~$N^2_2$ since it shares a~common edge with~$N^2_1$. 
 This is reflected in the graph~$G$ by not having a~directed edge from the
  node~$N^2_1$ to the node~$N^2_2$. We finish the update process
 by arriving either 
 to~$N^1_1$ or to~$N^1_2$. Notably, these nodes have out-degree~0.

 Algorithm~1 updates subroutes according to calculated QoS for each new subroute, by
 choosing at each step the new subroute that maximizes QoS.


The algorithm starts by selecting the initial set of subroute nodes. 
These are nodes with in-degree~0. The algorithm continues traversing the graph up 
to arrival at a node with out-degree~0 which would be the last subroute to launch. 
The inner loop at lines~\ref{alg:inner_loop}--\ref{alg:end_inner_loop} 
calculates the QoS for each neighboring node. Afterward, at 
line~\ref{alg:choose_qos}, the algorithm chooses the node that maximizes QoS. 
Then launches this node and updates the route~$C$, accordingly (see 
Figs.~1--5 for illustration). 
Afterward, the algorithm selects the next neighboring nodes.

After execution of Algorithm~1, the resulting new route maximally complies QoS 
requirements.

%\vspace*{12pt}

\subsection{Arbitrary subroutes selection} 
%\label{sec:arbitrary}

%\vspace*{-12pt}

\noindent
In this subsection, we assume that immediate duplicate packets elimination is possible. 
It may be that some of the subroutes are not ready for an update yet. 
Thus, meanwhile, the administrator may want to proceed with the update process 
to other subroutes or see possible variations of the update. 
For such scenario, we provide an algorithm which can select a~subroute for 
update arbitrary and continue the update process from there. 
We create a~forest graph of all possible update combinations from which the 
desired update sequence can be chosen. 
{\looseness=1

}
 


Figure~7 shows all possible combinations from example 
in Fig.~1. Noticeable, as mentioned earlier, some\linebreak\vspace*{-12pt}

{ \begin{center}  %fig7
\vspace*{1pt}
  \mbox{%
 \epsfxsize=71.694mm 
 \epsfbox{fre-7.eps}
 }


\end{center}


\noindent
{{\figurename~7}\ \ \small{Forest graph representing execution combinations for example from 
 Fig.~1}}
}

\vspace*{12pt}


\noindent
 combinations 
exhibit fewer steps, though possible that its QoS compliance is worse than others.



Algorithm~2 starts by iterating over all roots of the forest graph and 
calculating QoS using Algorithm~1 each tree. Afterward, launch the update 
of the tree that maximizes QoS.

\begin{algorithm*} %alg2
\setlength{\algowidth}{100mm}
 \setlength{\hsize}{\algowidth}
 \caption{Arbitrary Selection Update}
 \label{alg:arbitrary_update}
 \small
 
% \Input
{directed graph $G$} 
 
 %\BlankLine
 
 $A_0 \leftarrow$ choose nodes from $G$ with in-degree $0$ \\
 $Q \leftarrow \{\}$ \\
 
 \BlankLine
 \tcc{iterate over all roots of trees in the forest $G$}
 \ForEach{$v_r \in A_0$}
 {
 $q \leftarrow$ get the expected QoS using Algorithm~1 for $v_r$ \\
 $Q \leftarrow Q \cup \{q \rightarrow \mathrm{root} \}$ \\
 }

 \BlankLine
 $q_{\max} \leftarrow \max_{\mathrm{QoS}}(Q)$ \\
 launch maximizing QoS update order in $\mathrm{root}=Q[q_{\max}]$ \\ 
 
 
\end{algorithm*}


%\columnbreak

\vspace*{12pt}





\subsection{Parallel update}

\noindent
In certain cases, it is possible to update in parallel several subroutes 
and, as such, decrease update time. However, launching subroutes in parallel 
is not always possible
 since subroute may share a~link and, thus, leads to congestion 
during the update process, close a~cycle, or lead to an inconsistent state of the 
system. In~\cite{delaet_seamless_2015}, it was shown that two new subroutes~$N'$ 
from~$a$ to~$b$ and~$N''$ from~$c$ to~$d$ can be launched in parallel only if~$c$ 
succeeds~$b$ or~$a$ succeeds~$d$.



%\begin{proposition}
% Let $N'$ from $a$ to $b$ and $N''$ from $c$ to $d$ be two legitimate new subroutes. $N'$ and $N''$ can be launched in parallel only if $c$ succeeds $b$ or $a$ succeeds $d$.
%%Two subroutes that are each legitimate can be launched in parallel only if they share at most one common subroute.
%\end{proposition}
%\begin{proof}
% \textbf{Direction}: $\Rightarrow$ Let $N'$ from router $a$ to $b$ and $N''$ from router $c$ to $d$, be two new legitimate sub-routes. The only way for them to share more than one common sub-route is if $b$ succeeds $c$ on $C$. In such case, launching $N'$ will eliminate the part of $C$ from $c$ to $b$ with no proper connection from $b$ to $c$, which leaves the system in an inconsistent state. The same occurs if $N''$ is launched. \\
% \textbf{Direction}: $\Leftarrow$ Let $N'$ from router $a$ to $b$ and $N''$ from router $c$ to $d$, be two new sub-routes, not necessary part of the same new route, such that $b$ precedes $c$ or $b=c$. If $a$ precedes $b$, than $N'$ is legal for launching independently of $N''$. Similarly, if $c$ precedes $b$, than $N''$ is legal for launching independently of $N'$. Thus, since $N'$ can be launched independently from $N''$, they can be launched in parallel. Symmetric considerations lead to same result in case $a$ succeeds $d$.
% 
%\noindent Generalization to more than two sub-routes is trivial.
%\end{proof}



\begin{algorithm*}[b] %[t] %alg3
\setlength{\algowidth}{100mm}
 \setlength{\hsize}{\algowidth}
 \caption{Parallel Update}
 \label{alg:parallel_update}
 \small
 
 %\Input
 {weighted graph $G_S$} 
 
 \BlankLine
 
 \While{there are still current subroutes to update}
 {
 $A \leftarrow$ find maximum-weight independent set in $G_S$ \\
 
 \BlankLine 
 \tcc{do in parallel} 
 \ForEach{$N^t_i \in A$} 
 { 
 launch $N^t_i$ \\
 }
 }
 
 \vspace*{6pt}
 
\end{algorithm*}

We create a supplementary graph~$G_S$, in which nodes are the new legitimate 
for launching subroutes, and edges represent restrictions on parallel 
launching of subroutes. See Fig.~8 for illustration, 
depicting subroutes from example in Fig.~1 and their parallel 
restrictions. For example, $N^4_2$ and~$N^1_2$ can be launched in parallel since 
there is no edge connecting them.

Clearly, any independent set of subroutes from the supplementary 
graph contains subroutes that can be launched in parallel. 
This can be further enhanced by setting QoS calculated values as weights 
on nodes of the graph and finding the subroutes that can be launched 
in parallel by finding a~maximum-weight independent set of the graph~$G_S$. 
Since~$G_S$ has few
 number of nodes (several tens), it is possible to find 
the
 maximum-weight independent set even by enumerating
 all possible independent 
sets~\cite{wu_review_2015} and comparing their total weights.
{\looseness=-1



{ \begin{center}  %fig8
\vspace*{12pt}
  \mbox{%
 \epsfxsize=36.666mm 
 \epsfbox{fre-8.eps}
 }


\end{center}


\noindent
{{\figurename~8}\ \ \small{Supplementary graph of the example in 
 Fig.~1, showing which subroutes cannot be run in parallel}
}}

%\vspace*{12pt}



} 



Important, the parallel method should not be launched on its own. 
For example, assume that at the first iteration of Algorithm~3, 
the independent sets of nodes are~$A_1$ and~$A_2$. Let us assume that~$A_1$ complies 
better to QoS demands than~$A_2$ and, thus, $A_1$ will be selected. 
Also, let us assume that~$B_1$ is the next independent set in the graph 
if~$A_1$ was selected and~$B_2$ if~$A_2$ was selected. 
Also, let us assume that~$B_1$ is
the next independent set in the graph if~$A_1$ was selected and~$B_2$ if~$A_2$ 
was selected.
It is possible that due to the dynamics of the system (see section~3), 
we could obtain overall higher QoS results if we initially launched the 
subroutes from the sets~$A_2$ and~$B_2$ afterwards than from the sets~$A_1$ and~$B_1$.
 

Therefore, the graph that we create in this section for parallelization constraints 
is a~supplementary graph which must be used in conjunction with the graphs from 
previous sections. Optimal results will be obtained when used in conjunction with 
the forest graph from subsection~5.2.

It is also important to note that, in the worst case, when there are 
no disjoint subroutes, the parallel method is reduced to the sequential 
method thought with a higher running time.

\vspace*{-12pt} 


\section{Implementation}

\noindent
We implemented the update algorithms from~\cite{delaet_seamless_2015} as 
services for our QoS verification module. The update algorithm itself 
was not modified. In other words, we treated the update itself as 
an atomic action. The route updates
 algorithms are implemented as 
applications interacting with the northbound interface of an SDN controller. 
We used POX~\cite{kaur_network_2014} as a~platform for controller development and 
Mininet~\cite{lantz_network_2010} for network topology emulation. 
Figure~9 depicts the schematic arrangement of the 
functional elements. 



We created networks with topology of random graph and small-world features. 
During each simulation trial, a~pair of common source and destination nodes $(s,d)$ 
were selected. A~path connecting~$s$ and~$d$ was selected as a~current route and 
a~set of~4~new routes connecting $(s,d)$, to replace the current route, were 
selected, possibly with shared links among themselves and the current route. 

We considered latency due to the formed congestion as QoS demands for the update, 
implemented by forming congestion on randomly selected subroutes. Route 
update was executed by the update algorithm from~\cite{delaet_seamless_2015} for 
each pair of current and new routes. Further, one of the enhanced versions 
was executed, updating to the
 preferred combination of subroutes, by identifying 
the congested subroutes (e.\,g., by estimating latency).

{ \begin{center}  %fig9
\vspace*{8pt}
  \mbox{%
 \epsfxsize=58.544mm 
 \epsfbox{fre-9.eps}
 }

\vspace*{3pt}


\noindent
{{\figurename~9}\ \ \small{Description of the system}
}
\end{center}}

%\vspace*{12pt}



%\vspace*{-45pt}

\section{Concluding Remarks}

\noindent
The study in this paper illustrates a~feasibility of modeling and 
designing the route update process via verification using DTMC. The goal was to 
strengthen the network administrator involvement in management and decision 
making during route update. In the present model, the network administrator is able 
to consider network parameters such as packet losses, delay, communication 
rounds, flow table updates, congestion, and other inherent unreliabilities of 
the network. 

We extended the updating algorithm with the ability to compute QoS as the 
MC characteristics, where the MC corresponds to the states 
of the update process. Using this MC computation ability, it is 
possible to predict the expected number of steps (delay time) required to 
complete the update process. These prediction results allow the administrator 
to make a~decision whether a~new route can satisfy the user requirements per QoS 
or a~more reliable route will be selected.

We provided sequential update algorithm and an arbitrary order algorithm 
when for the later, it is assumed that immediate duplicate packets elimination 
is possible. Further, we suggest a supplementary graph and algorithm for launching 
updates in parallel when it is possible.

This paper proposes a~conceptual approach. In future research, we will focus 
on optimization of predictions supplementing the network administrator with 
a~powerful tool which will be able to enhance the update process 
with fine grained analysis of the network.

\vspace*{-12pt}


\Ack
\noindent
The first author has partially been supported by the 
Russian Foundation for Basic Research under grants RFBR 18-07-00669 and 18-29-03100. 
The second author has partially been supported by the Rita Altura Trust Chair in
Computer Sciences; The Lynne and William Frankel Center for Computer
Science.

%\bigskip


The authors thank Prof.\ Shlomi Dolev 
for his valuable input and Prof.\ Yefim Dinitz for his comments.
 
\renewcommand{\bibname}{\protect\rmfamily References}

%\vspace*{-6pt}

\vspace*{-6pt}

{\small\frenchspacing
{\baselineskip=10.35pt
\begin{thebibliography}{99}



\bibitem{rao_sdn_2014}  %1
\Aue{Rao, S.\,K.} 2014. SDN and its use-cases~--- NV and NFV:
A~state-of-the-art survey. NEC Technologies India Ltd. 25~p.

\bibitem{ghaznavi_service_2016}  %2
\Aue{Ghaznavi, M., N.~Shahriar, R.~Ahmed, and R.~Boutaba}. 2016. 
Service function chaining simplified. {arXiv.org}. arXiv:1601.00751.

\bibitem{hansson_logic_1994}  %3
\Aue{Hansson, H., and B.~Jonsson}. 
1994. A~logic for reasoning about time and reliability. 
\textit{Form. Asp. Comput.} 6(5):512--535.

\bibitem{delaet_seamless_2015}  %4
\Aue{Delaet, S., S.~Dolev, D.~Khankin, S.~Tzur-David, and T.~Godinger}. 
2015. Seamless SDN route updates. \textit{IEEE 14th Symposium (International)
on Network Computing and Applications}. IEEE. 120--125.

\bibitem{frenkel_predicting_2017} 
\Aue{Frenkel, S., D.~Khankin, and A.~Kutsyy}. 
2017. Predicting and choosing alternatives of route updates per QoS VNF in SDN. 
\textit{IEEE 16th Symposium (International) on Network Computing and Applications}. 
IEEE. 1--6. 

\bibitem{devi_approach_2015} 
\Aue{Devi, G., and S.~Upadhyaya}. 2015. 
An approach to distributed multi-path QoS routing. 
\textit{Indian J.~Sci. Technol.} 8(20):1--14. 
doi: 10.17485/ijst/2015/v8i20/49253.

\bibitem{egilmez_distributed_2012} 
\Aue{Egilmez, H.\,E., S.~Civanlar, and A.\,M.~Tekalp}. 2012. 
A~distributed QoS routing architecture for scalable video streaming over multi-domain 
OpenFlow networks. \textit{19th IEEE Conference (International) on Image Processing}.
IEEE. 2237--2240.

\bibitem{juttner_lagrange_2001} 
\Aue{Juttner, A., B.~Szviatovski, I.~Mecs, and Z.~Rajko}. 2001. 
Lagrange relaxation based method
for the QoS routing problem. \textit{IEEE Conference on Computer Communications. 
20th Annual Joint Conference of the IEEE Computer and Communications Society
 Proceedings}. IEEE. 2:859--868.

\bibitem{yu_efficient_2013} %9
\Aue{Yu, Z., F.~Ma, J.~Liu, B.~Hu, and Z.~Zhang}. 2013. 
An efficient approximate algorithm for disjoint QoS routing.
\textit{Math. Probl. Eng.} 2013:489149. 9~p. 
doi: 10.1155/2013/489149.

\bibitem{foerster_survey_2016} 
\Aue{Foerster, K.-T., S.~Schmid, and S.~Vissicchio} 2016. 
A~survey of consistent network updates. \mbox{Arxiv.org}. \mbox{arXiv}:\linebreak 1609.02305.

\bibitem{reitblatt_consistent_2011} 
\Aue{Reitblatt, M., N.~Foster, J.~Rexford, and D.~Walker}. 
2011. Consistent updates for software-defined networks: Change you can believe in! 
\textit{10th ACM Workshop on Hot Topics in Networks Proceedings}.
New York, NY: ACM. Art.\ No.\,7. doi: 10.1145/2070562.2070569.

\bibitem{hogan_stochastic_2017} 
\Aue{Hogan, M., and F.~Esposito}. 
2017. Stochastic delay forecasts for edge traffic engineering via Bayesian networks. 
\textit{IEEE 16th Symposium (International) on Network Computing and Applications}. 
IEEE. 1--4.

\bibitem{mcgeer_safe_2012} %15
\Aue{McGeer, R.} 2012. A~safe, efficient Update Protocol for Openflow Networks. 
\textit{1st Workshop on Hot Topics in Software Defined Networks Proceedings}. 
New York, NY: ACM. 12:61--66.
\bibitem{mcgeer_correct_2013} 
\Aue{McGeer, R.} 2013. A~correct, zero-overhead protocol for network updates. 
\textit{2nd ACM SIGCOMM Workshop on Hot Topics in Software Defined Networking
Proceedings}. New York, NY: ACM. 13:161--162.
\bibitem{katta_incremental_2013} 
\Aue{Katta, N.\,P., J.~Rexford, and D.~Walker}. 
2013. Incremental consistent updates. \textit{2nd ACM SIGCOMM Workshop on Hot Topics 
in Software Defined Networking Proceedings}.
New York, NY: ACM. 13:49--54.

\bibitem{dinitz_dependence_2017}  %16
\Aue{Dinitz, Y., S.~Dolev, and D.~Khankin}. 
2017. Dependence graph and master switch for seamless dependent routes 
replacement in SDN. \textit{IEEE 16th Symposium 
(International) on Network Computing and Applications}. IEEE. 1--7.

\bibitem{17-aaa}
\Aue{Amiri, S.\,A., S.~Dudycz, S.~Schmid, and S.~Wiederrecht}.
2016. Congestion-free rerouting of flows
on DAGs. \mbox{ArXiv}.org. arXiv:1611.09296.
% [cs, math], Nov. 2016, arXiv: 1611.09296. [Online]. Available:
%http://arxiv.org/abs/1611.09296

\bibitem{kwiatkowska_prism_2011}  %17
\Aue{Kwiatkowska, M., G.~Norman, and D.~Parker}. 2011. 
PRISM~4.0: Verification of probabilistic real-time systems. 
\textit{Computer aided verification}.
Eds. G.~Gopalakrishnan and S.~Qadeer.
Lecture notes in computer science ser. Springer.
6806:585--591.

\bibitem{noauthor_prism_nodate}  %18
\Aue{Kwiatkowska, M., G.~Norman, and D.~Parker}. 2018. 
{PRISM manual}. Available at:
{\sf http://www.\linebreak prismmodelchecker.org/manual/}
(accessed December~10, 2018).

\bibitem{onf_openflow_2015} %19
{Open Networking Foundation}. 2015. 
OpenFlow Switch Specification Ver~1.5.1. 


\bibitem{wu_review_2015}  %20
\Aue{Wu, Q., and J.-K.~Hao}. 2015. 
A~review on algorithms for maximum clique problems. 
\textit{Eur. J.~Oper. Res.} 242(3):693--709.

\bibitem{kaur_network_2014}  %21
\Aue{Kaur, S., J.~Singh, and N.\,S.~Ghumman}. 2014. 
Network programmability using POX controller. 
\textit{Conference (International) on Communication, Computing and Systems}.
138.

\bibitem{lantz_network_2010}  %22
\Aue{Lantz, B., B.~Heller, and N.~McKeown}. 2010. 
A~network in a~laptop: Rapid prototyping for software-defined networks. 
\textit{9th ACM SIGCOMM Workshop on Hot Topics in Networks Proceedings}. 
New York, NY: ACM.  Art.\ No.\,19. doi: 10.1145/1868447.1868466.
\end{thebibliography} } }

\end{multicols}

\vspace*{-9pt}

\hfill{\small\textit{Received October 9, 2018}}

\vspace*{-22pt}

\Contr

\vspace*{-3pt}

\noindent
\textbf{Frenkel Sergey L.} (b.\ 1951)~--- 
Candidate of Science (PhD) in technology, associate professor, 
senior scientist, Institute of Informatics Problems, Federal Research Center 
``Computer Sciences and Control'' of the Russian Academy of Sciences, 
44-2~Vavilov Str., Moscow 119333, Russian Federation; \mbox{fsergei51@gmail.com}

\vspace*{1pt}

\noindent
\textbf{Khankin D.} (b.\ 1983)~--- MSc, doctorate student, Department of Computer 
Science, Ben-Gurion University of the Negev, Beer-Sheva 84105, Israel; 
\mbox{danielkh@post.bgu.ac.il}

\vspace*{4pt}

\hrule

\vspace*{2pt}

\hrule

\vspace*{-7pt}

%\newpage

%\vspace*{-28pt}

\def\tit{НЕПРЕРЫВНЫЕ ОБНОВЛЕНИЯ МАРШРУТА В~SDN С~ИСПОЛЬЗОВАНИЕМ ПРОВЕРКИ СООТВЕТСТВИЯ 
КАЧЕСТВУ~ОБСЛУЖИВАНИЯ$^*$\\[-7pt]}

\def\titkol{Непрерывные обновления маршрута в~SDN с~использованием проверки соответствия 
качеству обслуживания}

\def\aut{С.\,Л.~Френкель$^1$, Д.~Ханкин$^2$\\[-7pt]}

\def\autkol{С.\,Л.~Френкель, Д.~Ханкин}

{\renewcommand{\thefootnote}{\fnsymbol{footnote}} \footnotetext[1]
{Работа была частично поддержана РФФИ (гранты 18-07~00669 и~18-29-03100), 
а~также Rita Altura Trust Chair in
Computer Sciences; The Lynne and William Frankel Center for Computer
Science.}}



\titel{\tit}{\aut}{\autkol}{\titkol}

\vspace*{-22pt}

\noindent
$^1$Институт проблем информатики Федерального исследовательского центра 
<<Информатика и~управление>>\linebreak
$\hphantom{^1}$Российской академии наук
%, fsergei51@gmail.com 

\noindent
$^2$Университет им.\ Бен-Гуриона в Негеве, Беэр-Шева, Израиль
%, danielkh@post.bgu.ac.il 

\vspace*{1pt}

\def\leftfootline{\small{\textbf{\thepage}
\hfill ИНФОРМАТИКА И ЕЁ ПРИМЕНЕНИЯ\ \ \ том\ 12\ \ \ выпуск\ 4\ \ \ 2018}
}%
 \def\rightfootline{\small{ИНФОРМАТИКА И ЕЁ ПРИМЕНЕНИЯ\ \ \ том\ 12\ \ \ выпуск\ 4\ \ \ 2018
\hfill \textbf{\thepage}}}

\vspace*{-1pt}


 
\Abst{В программно-определяемой сети (SDN~--- software-defined networking) 
уровень управ\-ле\-ния 
и~уровень данных разделены. Это обеспечивает высокую гибкость эксплуатации, 
предоставляя абстракции для управления сетью приложений 
и~возможность непосредственного программирования маршрутов.
Однако из-за изменений топологии, процедуры обслуживания или происходящих 
сбоев иногда необходима реконфигурация и~обновление сети. 
В~предлагаемом сценарии рассматривается текущий маршрут~$C$
и~набор возможных новых маршрутов~~$\{N_i\}$, где для замены текущего 
маршрута требуется 
один из\linebreak\vspace*{-12pt}}

\Abstend{новых маршрутов. Существует вероятность того, что новый маршрут~$N_i$ 
окажется длиннее некоторого другого нового маршрута~$N_j$, но при этом~$N_i$ 
будет более надежным и~он будет обновляться быстрее или работать лучше 
после обновления с~точки зрения требований качества обслуживания (QoS~---
quality of service). Принимая 
во внимание случайный характер функционирования сети, авторы дополнили недавно 
предложенный алгоритм обновления маршрута Delaet с~соавт.\ методом оценки соблюдения 
требований QoS во время непрерывного обновления маршрута, основанным на 
использовании цепей Маркова. При этом, во-пер\-вых, предлагается расширить 
алгоритм передачи пакетов по выбранному маршруту, сравнивая процесс обновления 
для возможных альтернатив маршрута. Во-вто\-рых, предлагается несколько 
способов выбора комбинаций предпочтительных отрезков путей новых маршрутов, 
что приводит к оптимальному в~смысле соответствия QoS маршруту.}


\KW{программно-определяемые сети; цепи Маркова; качество обслуживания}

\DOI{10.14357/19922264180408}



%\vspace*{-3pt}


 \begin{multicols}{2}

\renewcommand{\bibname}{\protect\rmfamily Литература}
%\renewcommand{\bibname}{\large\protect\rm References}

{\small\frenchspacing
{\baselineskip=10.5pt
\begin{thebibliography}{99}
%\vspace*{-3pt}


\bibitem{2-fr-1}
\Au{Rao S.\,K.} SDN and its use-cases~--- NV and NFV: A~state-of-the-art survey.~--- 
NEC Technologies India Ltd., 2014. 25~p.
\bibitem{3-fr-1}
\Au{Ghaznavi M., Shahriar~N., Ahmed~R., Boutaba~R.} 
Service function chaining simplified~// Arxiv.org, 2016. \mbox{arXiv}:1601.00751cs.
\bibitem{4-fr-1}
\Au{Hansson H., Jonsson~B.} A~logic for reasoning about time and reliability~// 
Form. Asp. Comput., 1994. Vol.~6. No.\,5. P.~512--535.

\bibitem{1-fr-1} %4
\Au{Delaet S., Dolev~S., Khankin~D., Tzur-David~S., Godinger~T.}
Seamless SDN route updates~// IEEE 14th Symposium (International)
 on Network Computing and Applications.~--- IEEE, 2015. P.~120--125.
 
 
\bibitem{5-fr-1}
\Au{Frenkel S., Khankin D., Kutsyy~A.} Predicting and choosing alternatives 
of route updates per QoS VNF in SDN~// IEEE 16th Symposium (International)
on Network Computing and Applications.~--- IEEE, 2017. P.~1--6.
\bibitem{6-fr-1}
\Au{Devi G., Upadhyaya~S.} An approach to distributed multi-path QoS routing~// 
Indian J.~Sci. Technol., 2015. Vol.~8. Iss.~20. P.~1--14. 
doi: 10.17485/ijst/2015/v8i20/49253.
\bibitem{7-fr-1}
\Au{Egilmez H.\,E., Civanlar S., Tekalp~A.\,M.} 
A~distributed QoS routing architecture for scalable video streaming over multi-domain 
OpenFlow networks~// 19th IEEE Conference (International)
on Image Processing.~--- IEEE, 2012. P.~2237--2240.
\bibitem{8-fr-1}
\Au{Juttner A., Szviatovski B., Mecs~I., Rajko~Z.}
Lagrange relaxation based method for the QoS routing problem~// 
IEEE INFOCOM 2001 Conference on Computer Communications. 20th 
Annual Joint Conference of the IEEE Computer and Communications Society
Proceedings.~--- IEEE, 2001. Vol.~2. P.~859--868.
\bibitem{9-fr-1}
\Au{Yu Z., Ma F., Liu~J., Hu~B., Zhang~Z.}
An efficient approximate algorithm for disjoint QoS routing~// 
Math. Probl. Eng., 2013. Vol.~2013. Art.\ No.\,489149. 9~p. 
doi: 10.1155/2013/489149.
\bibitem{10-fr-1}
\Au{Foerster K.-T., Schmid S., Vissicchio~S.}
A~survey of consistent network updates~// Arxiv.org, 2016. arXiv:1609.02305.
\bibitem{11-fr-1}
\Au{Reitblatt M., Foster N., Rexford J., Walker~D.} 
Consistent updates for software-defined networks: Change you can believe in!~// 
10th ACM Workshop on Hot Topics in Networks Proceedings.~--- New York, NY, USA: ACM, 
2011. Art.\ No.\,7. doi: 10.1145/2070562.2070569.
\bibitem{12-fr-1}
\Au{Hogan M., Esposito F.} Stochastic delay forecasts for edge traffic engineering 
via Bayesian Networks~// IEEE 16th Symposium (International)
on Network Computing and Applications.~--- IEEE, 2017. P.~1--4.
\bibitem{13-fr-1}
\Au{McGeer R.} A~safe, efficient Update Protocol for Openflow Networks~// 
1st Workshop on Hot Topics in Software Defined Networks Proceedings.~--- 
New York, NY, USA: ACM, 2012. Vol.~12. P.~61--66.
\bibitem{14-fr-1}
\Au{McGeer R.} 2013. A~correct, zero-overhead protocol for network updates~// 
2nd Workshop on Hot Topics in Software Defined Networking Proceedings.~--- 
New York, NY, USA: ACM, 2013. Vol.~13. P.~161--162.
\bibitem{15-fr-1}
\Au{Katta N.\,P., Rexford J., Walker~D.} Incremental consistent updates~// 
2nd Workshop on Hot Topics in Software Defined Networking Proceedings.~--- 
New York, NY, USA: ACM, 2013. Vol.~13. P.~49--54.
\bibitem{16-fr-1}
\Au{Dinitz Y., Dolev S., Khankin~D.}
 Dependence graph and master switch for seamless dependent 
 routes replacement in SDN~// IEEE 16th Symposium 
 (International) on Network Computing and Applications.~--- IEEE, 2017. P.~1--7.
 \bibitem{17-aaa-1}
\Au{Amiri~S.\,A., Dudycz~S., Schmid~S., Wiederrecht~S}.
 Congestion-free rerouting of flows
on DAGs~// ArXiv.org, 2016. arXiv:1611.09296.
% [cs, math], Nov. 2016, arXiv: 1611.09296. [Online]. Available:
%http://arxiv.org/abs/1611.09296

\bibitem{17-fr-1}
\Au{Kwiatkowska M., Norman~G., Parker~D.}
 PRISM~4.0: Verification of probabilistic real-time systems~//
 Computer aided verification~/
 Eds. G.~Gopalakrishnan, S.~Qadeer.~---
Lecture notes in computer science ser.~--- Springer, 2011. 
 Vol.~6806. P.~585--591.
\bibitem{18-fr-1}
\Au{Kwiatkowska M., Norman G., Parker~D.}
 PRISM manual, 2018. 
{\sf http://www.prismmodelchecker.org/manual}.
\bibitem{19-fr-1}
Open Networking Foundation. OpenFlow Switch Specification Ver~1.5.1, 2015. 

\bibitem{21-fr-1}
\Au{Wu Q., Hao J.-K.} A~review on algorithms for maximum clique problems~// 
Eur. J.~Oper. Res., 2015. Vol.~242. No.\,3. P.~693--709.

\bibitem{20-fr-1}
\Au{Kaur S., Singh J., Ghumman~N.\,S.}
 Network programmability using POX controller~// Conference
 (International) on Communication, Computing and Systems, 2014. P.~138.
\bibitem{22-fr-1}
\Au{Lantz B., Heller B., McKeown~N.} 
A~network in a~laptop: Rapid prototyping for software-defined networks~// 
9th ACM SIGCOMM Workshop on Hot Topics in Networks Proceedings.~--- 
New York, NY, USA: ACM, 2010. Art.\ No.\,19. doi: 10.1145/1868447.1868466.
\end{thebibliography}
} }

\end{multicols}

 \label{end\stat}

 \vspace*{-9pt}

\hfill{\small\textit{Поступила в~редакцию 09.10.2018}}


%\renewcommand{\bibname}{\protect\rm Литература}
\renewcommand{\figurename}{\protect\bf Рис.}
\renewcommand{\tablename}{\protect\bf Таблица}  %11
\renewcommand{\figurename}{\protect\bf Figure}

\def\stat{belyaev}


\def\tit{ANALYSIS OF SURVEY DATA CONTAINING  ROUNDED CENSORING INTERVALS}

\def\titkol{Analysis of survey data containing  rounded censoring intervals}

\def\autkol{Yu.\,K.~Belyaev and  B.~Kristr$\ddot{\mbox{o}}$m}

\def\aut{Yu.\,K.~Belyaev$^1$ and  B.~Kristr$\ddot{\mbox{o}}$m$^2$}

\titel{\tit}{\aut}{\autkol}{\titkol}

%{\renewcommand{\thefootnote}{\fnsymbol{footnote}}
%\footnotetext[1] {The work of first and second  authors is partially supported by the
%Program of Strategy development of Petrozavodsk State University in
%the framework of the research activity. The third author is a
%postdoctoral fellow with the Research Foundation-Flanders
%(FWO-Vlaanderen).}}

\renewcommand{\thefootnote}{\arabic{footnote}}
\footnotetext[1]{Department of Mathematics and Mathematical Statistics,
\mbox{Ume{\!\!\fontsize{9pt}{9pt}\selectfont\ptb{\!{\r{\hspace*{-2pt}a}}}}}
University,  
\mbox{Ume{\!\!\fontsize{9pt}{9pt}\selectfont\ptb{\!{\r{\hspace*{-2pt}a}}}}} SE-901 87, 
Sweden, yuri.belyaev@umu.se}
\footnotetext[2]{Center for Environmental and Resource Economics (CERE),
Swedish University of Agricultural Sciences,  
\mbox{Ume{\!\!\fontsize{9pt}{9pt}\selectfont\ptb{\!{\r{\hspace*{-2pt}a}}}}} SE-901 83, 
Sweden, bengt.kristrom@umu.se}


\vspace*{-9pt}

\def\leftfootline{\small{\textbf{\thepage}
\hfill INFORMATIKA I EE PRIMENENIYA~--- INFORMATICS AND APPLICATIONS\ \ \ 2015\ \ \ volume~9\ \ \ issue\ 3}
}%
 \def\rightfootline{\small{INFORMATIKA I EE PRIMENENIYA~--- INFORMATICS AND APPLICATIONS\ \ \ 2015\ \ \ volume~9\ \ \ issue\ 3
\hfill \textbf{\thepage}}}



\Abste{This paper makes a~contribution towards the statistical analysis of data sets containing intervals,
that naturally arises in survey contexts. The suggested approach is sufficiently general to cover
most cases where interval data are used. Interval data appear in many contexts, such as in reliability
studies and survival analysis, in medicine and economics, in opinion elicit surveys,
etc. There are
several reasons for the extensive use of interval data, perhaps,
the most common being one of necessity;
exact values of the underlying observations are censored. 
The nature of the intervals analyzed
here is somewhat unusual. The self-selected intervals (SeSeI)
are (freely) chosen by the subjects.
A~generalization of the influential approach has been suggested
to the statistical analysis of general
censoring introduced by B.\,W.~Turnbull. A~key independence assumption
in Turnbull's analysis has been explained and generalized. 
A~sampling stopping rule based on the coverage probability has been suggested
and the properties of a~two-step estimator, based on the idea of asking two questions,
where the second involves a~way of fine-graining the information,
has been discussed. This paper provides several
informatics methods for SeSeI, targeting the problem of partial nonparametric
identification. The properties of the suggested statistical models are stated, including a~recursion for
easy numerical calculations. An extensive simulation study, displaying, inter alia,
the usefulness
of the proposed resampling methods for the situation under study, completes the paper.}

\KWE{elicitation surveys; random sampling;  rounding; anchoring; coverage probability;
likelihood; recursion; maximization; resampling}


\DOI{10.14357/19922264150301} 

%\vspace*{6pt}


\vskip 12pt plus 9pt minus 6pt

      \thispagestyle{myheadings}

      \begin{multicols}{2}

                  \label{st\stat}


\section{Introduction}

\noindent
 Interval data appear in many contexts, such as in reliability studies, survival analysis in medicine,
 and in opinion surveys. There are several reasons for the extensive use of interval data, perhaps,
 the most common being one of necessity; exact values of the underlying observations are not observable.
 In many survey studies, missing information is a~problem. Having an interval rather than a~missing
 observation is valuable. It is known that the SeSeIs studied here, or,
 as they are also known, unfolding brackets, increase response-rates and help reducing a~number
 of well-known problem as described in detail below.
The authors focus on a~case where the exact values are, in principle, observable; yet, for reasons just stated,
 the individual has difficulty pinning down an exact value.

 Because an individual is asked to select any interval he or she finds most suitable, the data obtained
 is necessarily richer and, we argue, provide additional insights relative to points data. The SeSeIs
 studied here are just natural generalizations of a~certain type of open ended survey questions.
 In some ways, the SeSeIs respond to a~point made by  Manski~\cite{BK:MA99} who notes:

\hangindent=3mm\hangafter=0\noindent
\textit{One problem is the fixation in the social sciences on point identification of parameters\ldots
  Weaker and more plausible assumptions often suffice to bound parameters in informative ways}.

\noindent
The idea of SeSeIs is, certainly, not new; the basic idea goes back to at least 
Morgan and
Henrion~\cite{BK:MH90}, who suggested SeSeIs as a~way of overcoming 
``overconfidence'' and also to attack the
anchoring problem (as discussed further below). There is also a~connection to 
symbolic data analysis~\cite{BK:BD06}, in which intervals play an important role. Recent
applications of SeSeIs include Manski and Molinari~\cite{BK:MM10} and 
Johansson and Kristr$\ddot{\mbox{o}}$m~\cite{BK:JK12}.


The present authors' previous research on SeSeIs is reported in~\cite{BK:BK10, BK:BK12},
in which new statistical methods have been developed that cover SeSeIs in some detail. In particular,
consistency of certain parametric models and measures of accuracy (using resampling methods)
has been proved.
In addition, a~sampling stopping rule has been proposed
and the properties of a~2-step estimator have been discussed
based on the idea of asking two questions, where the second involves a~way of fine-graining the information.
This paper provides additional statistical results targeting the problem of nonparametric identification.
A~simple example might be useful to see more clearly what the underlying identification problem is.
Assume that half of the respondents in a~population state the interval $(0, 2]$ 
and that
the other half $(1, 3]$ as their SeSeIs. Clearly, even if one
 knows the SeSeIs for all members of
the population, there cannot be identified the 3~probabilities that the (unknown) exact value is in
either of the three sets $\{(0, 1],(1,2],(2,3])$ without further information (which 
is proposed
then to be obtained in the second step of data collection). A~main contribution in this paper is
to show how the nonparametric estimator generalizes Turnbull~\cite{BK:TU74, BK:TU76} on nonparametric
estimation in  general cases of censoring. In particular, Turnbull's estimator is based on a
particular independency assumption not needed in the considered case. 
The obtained data also suggest that the
independency assumption might be quite stringent. The authors introduce an assumption, of which
Turnbull's independence assumption can be viewed as a~special case. The
 real world example
data illustrate this idea in some detail. Finally, the authors note that the analysis of SeSeIs is
closely related to the econometrics literature on partial identification~\cite{BK:MA03}.

To get some intuition and feel for the theoretical results, in particular,
for Turnbull's assumption,
a~pilot analysis of an illustrative data set is made in section~2.
While the paper focuses on statistical modeling, section~3 discusses
salient assumption  made about the response process used by respondents in the survey
situation studied. Section~4 introduces the basic statistical model,
along with a~discussion about the size  of sampling and the properties of the likelihood in the simplest case.
The present authors propose a~solution  to the identification problem that arises if one
is  interested in nonparametric models in this context in section~5. Here,
a~more general assumption about the response process has been also introduced, 
which has the Turnbull assumption as
a~special case. The properties of statistical models are stated, including a~recursion for easy
numerical calculations. A~final section concludes.

\section{Preliminary Analysis Using Real Data-Set} %\label{sec:pilot}

\noindent
It will be useful before introducing the theoretical details to fix ideas by scrutinizing in some
detail a~real world data-set, where the SeSeIs have been used. 
The authors use contingent valuation,
a widely used survey method to shed light on the value respondent attach to,
for example, environmental
improvements and other nonmarket goods (see~\cite{BK:CA12}). While 
the present approach can be used
in many settings, contingent valuation is a~natural application, not the least because individuals are
typically uncertain about how much they want to pay, say, for nonmarket priced quality improvements.
The interval questions were used within a~study of  the costs and
benefits of changing instead flow for wild salmon at the
Stornorrfors hydropower plant on the Vindel River, in northern
Sweden~\cite{BK:HA08}. 
The respondents were asked about their Willingness-To-Pay (WTP)
for increasing the number of salmon that reach their spawning
grounds in the river each year. The respondents were randomly sampled  from a
general register of the Swedish subpopulation (SPAR) older than~18~years. 
The sample was split into three subsets, each with a~different formulation 
of the valuation question.



In the first sample, denoted $S_1$, a~standard open-ended question was used to
obtain points, i.\,e., the WTP value for each respondent. In the second sample, 
$S_2$, the authors asked for
WTP-intervals, and in the third sample, $S_3$, individuals were free to select either a~point or any interval of choice.
The data are summarized in the table and in Figs.~1 and~\ref{FG-33}.


These figures reveal, inter alia,  that the survival distribution functions (s.d.f.s), of sta\-ted
points in~$S_1$ and the right ends intervals stated 
in~$S_2,$ are nearly the same (see Fig.~1). This (unexpected) coincidence is hardly  related to
an underlying assumption in the Turnbull approach~\cite{BK:TU74, BK:TU76}; the censored value is independent
of the interval ends. In the present case, it appears as if respondents censored values are 
``to the right'' in any given stated interval.  Furthermore,  Fig.~\ref{FG-33} 
suggests that the union
$S_2 \cup S_3$ can be considered as the basic interval-data set. 
This gives~241~intervals in the data collected by\linebreak
\mbox{H{\hspace*{-1.1mm}\fontsize{12pt}{12pt}\selectfont\ptb{\!{\r{\hspace*{-3pt}a}}}}kansson}~\cite{BK:HA08}. 
There are~46~different intervals in $S_2 \cup S_3$ and it is useful to consider
the choice respondents have made in further detail.
%\begin{table*}
{\small
\begin{center}

\tabcolsep=1.4pt
\begin{tabular}{cccccc}
\multicolumn{6}{c}{The results of the study of costs and benefits}\\
\multicolumn{6}{c}{\ }\\[-6pt]
%\multicolumn{5}{c}{\ }\\[6pt]
  \hline
 \multicolumn{1}{c}{\raisebox{-6pt}[0pt][0pt]{Sample}} & \# not  & \# of stated 0 & \# of stated &
   \# of stated &  \multicolumn{1}{c}{\raisebox{-6pt}[0pt][0pt]{Total \#}}  \\
    & answered & & intervals & points &      \\ 
    \hline
  $S_1$ & 97 & 76 & \hphantom{9}0 & 72 & 245 \\ 
%  \hline
  $S_2$ & 97 & 88 & 58 & \hphantom{9}0 & 243 \\ 
%  \hline
  $S_3$ & 527\hphantom{9} & 334\hphantom{9} & 183\hphantom{9} & 148\hphantom{9} & 1192\hphantom{9} \\ 
  \hline
  \end{tabular}
  \end{center}}
%\end{table*}

\begin{center}  %fig1
\vspace*{9pt}
\mbox{%
 \epsfxsize=77.93mm
 \epsfbox{bel-1.eps}
 }

\end{center}

%\vspace*{3pt}

\noindent
{{\figurename~1}\ \ \small{Two empirical distributions corresponding to the stated points in the sample~$S_1$ and
the stated right ends of  WTP-intervals in sample~$S_2$. 
Both distributions are trimmed at 1000~SEK}}

\addtocounter{figure}{1}

\begin{figure*} %fig2
       \vspace*{1pt}
 \begin{center}
 \mbox{%
 \epsfxsize=158.849mm
 \epsfbox{bel-2.eps}
 }
 \end{center}
 \vspace*{-9pt}
 \Caption{Two empirical distributions describing the right ends of the 
 intervals~(\textit{a}) and
the interval lengths~(\textit{b}) in~$S_2$ and~$S_3$.
Both distributions are trimmed at 950~SEK
\label{FG-33}}
\end{figure*}






First notice that  there is considerable heaping on a~certain set of intervals.
Thus, 142 out of~241~respondents stated the following four WTP-in\-ter\-vals: 
$(20, 50]$, $(20, 100]$, $(50, 100]$, and $(100, 200]$. 
These intervals were chosen by~39, 11, 69, and 23~individuals, respectively. Consequently,
there are four ``popular'' intervals that together make up an important part of the data. Of the four
``popular'' intervals, three of them correspond to the face value of money bills in Sweden (the fourth
has a~multiple of a~100~SEK bill). Consequently, even though an individual can state any value
(that he or she can afford), it is clearly the case that rounding is prevalent in the data.
The~21~unique intervals in the data-set are also all possible to obtain by combining existing
coin and bills in Sweden. In short, respondent invariably round their answers, seemingly using
the existing bills and coins nominations in Sweden when reporting their valuations. Let now turn
to some economic and statistical modeling details.

\section{The Economic Model} %\label{sec:economics}

\noindent
Here, a~detailed microeconomic model of the response process
is not presented, because the focus is on
the statistical issues. There are, however, a~number of different existing models that 
can be reinterpreted to fit this particular case. For example, in the McFadden/Manski random utility
maximization (RUM) mo\-del, the essential assumption is that the individual knows 
his/her utility function,
but the analyst cannot observe it completely; hence,
a~random error term is added to the utility function.
It is possible to reinterpret the SeSeIs in terms of the RUM, in the sense that one can assume that
the individual considers the value $Y=\mu + \epsilon$, where $Y$ is the point value 
and $\mu >0$ is the constant and $\epsilon$ is the random errors. In this interpretation, the individual is unable to pin
down the exact value but reports instead a~support of the distribution of~$Y$. 
At any rate,
 this model will not be pursued further, given the focus of the
 present analysis. However, it is important
 to state key assumptions about the response process, to which let now turn.

\subsection*{Basic assumptions}

\noindent
The authors base their statistical models on three basic assumptions 
(implicitly, that the person is also assumed to reveal his/her WTP truthfully):

\smallskip

\noindent
\textbf{Assumption~1.} \textit{Each respondent might not be aware of the exact location of the
true WTP-point. The respondents may freely choose SeSeIs
containing their true WTP-points. The ends of stated intervals may
be rounded, e.\,g., to simple sums of coins or paper values of money.}

\smallskip

\noindent
\textbf{Assumption~2.}
\textit{The true WTP-points are independent of question mode, i.\,e., the
structure of the valuation question does not change the true WTP-points in the
SeSeI.}


\smallskip

\noindent
\textbf{Assumption~3.}
\textit{The pairs of true WTP-points and the stated SeSeIs, corresponding to different sampled
individuals, are the values of independent identically distributed
(i.i.d.)\ random variables (r.v.s).}


The first part of Assumption~1 has ample support in the contingent
valuation literature. For example, the standard approach in this literature is to use a~payment
card (with given brackets), where the individual is to state (for each interval) how certain he
or she is about his/her WTP being in any of the brackets displayed in the card 
(see~\cite{BK:BB08},
a~recent survey of~76~papers is in~\cite{BK:MRKB14}). This is another way to
cater for respondent uncertainty compared to what is suggested here.

 In the second part of Assumption~1, the authors naturally assume (but
do not explicitly state) that the individual chooses an interval
such that WTP does not exceed his or her income. The authors explicitly
allow for rounding in the final part of Assumption~1.

 Assumption 2 is important. From some perspectives,
this assumption is strong, given the evidence that exists on
the differing results between types of valuation questions. 
For example, it has been demonstrated
(e.\,g., in the research by the Nobel Laureate Daniel Kahnemann) that anchoring is prevalent under uncertainty.
Anchoring here means that an individual, in a~situation of uncertainty, uses any information, whether
relevant or not, to form his/her answer. In particular, it is apparent that the actual brackets
being displayed in a~payment card may affect the answers; if so, the answers become ``anchored'' around
the numbers suggested. If we are just asking a~SeSeI question this issue is moot, in the sense that no
numbers are suggested that may be used as anchors. However, as it will be seen, for (partial) nonparametric
identification, it is necessary to ask the second question that does involve suggesting numbers. Fortunately,
if anchoring exists in the suggested two-step approach, it is likely to be less of a~problem, since 
one has
information about the ``undistorted'' interval, elicited in the first stage.

 Assumption 3 is, typically innocuous, in
that most contingent valuation applications are based on large
significant national samples.

Let now turn to the statistical modeling in more detail. It is useful to begin with addressing
the sampling issue first; how many individuals it is necessary to sample 
in order to be sure that
we have, in a~loose sense, ``enough'' information? The proposed solution is based on an estimator obtained
by Good~\cite{BK:GO54}. The estimation which is used in the derivation of a~sample stopping rule, is related to what
is called here a~coverage probability. The end-result is a~stochastic difference equation, which 
will be now derived.

\section{Statistical Modeling of~Self-Selected Intervals: 
Basic~Ideas} %\label{sec:statistics}

\noindent
Let begin the discussion of the statistical modeling with analyzing the sampling problem under SeSeIs,
then turn to a~basic building block called division intervals. Let then introduce similar Turnbull's basic assumption,
derive the likelihood and its properties, and select a~convenient numerical scheme for the implied numerical
maximization problem. The discussion of the basic statistical model 
will be ended by introducing an alternative
to Turnbull's assumption, essentially an assumption suggested by the data. This will lead one naturally to
the two-step extension, introduced in subsequent section~5.
{\looseness=1

}

\subsection{Sampling stopping rule}

\vspace*{-2pt}

\noindent
Consider $n,$ randomly sampled respondents, from  
a~population $\mathfrak{P}$ of interest,
  that have stated SeSeIs $\mathbf{y}_1^n =\{ \mathbf{y}_1, \ldots , \mathbf{y}_n\}$, 
  $ \mathbf{y}_i =(y_{Li}, y_{Ri}].$
  The intervals need not to be unique, because of rounding we do expect to see quite many repeated intervals.
  Yet, an important question is when to stop the sampling process;   which~$n$~may be considered sufficiently
  large? The problem is related to the problem of estimating, e.\,g., the number of species in a~certain area,
  or the number of words in a~given language. It will be addressed using an idea by 
  Good (1954), which   will be generalized
  to a~stochastic difference equation and to a~``coverage probability.'' 
Let first introduce some notation.

    Let  consider $\mathbf{y}_1^n =\{ \mathbf{y}_1,
   \ldots , \mathbf{y}_n \}$ as a~realization of a~multinomial random
   process  $\{ \mathbf{Y}_i\}_{i\ge 1}$ with ``time'' parameter
   $i=1, 2,\ldots$ The r.v.s $\{ \mathbf{Y}_i\}_{i\ge 1}$ are (as noted) i.i.d.\
   and the set of    their values is  all SeSeIs $\mathcal{U}_{\mathrm{all}} =\{ \mathbf{
   u}_\alpha : p_\alpha =P[\mathbf{Y}_i =\mathbf{u}_\alpha ]>0$, $\alpha\in A\}.$
   The set  $\mathcal{U}_{\mathrm{all}}$ and the  discrete distribution $\{ p_\alpha,
   \alpha\in A\} $ are not known.  $\alpha$ is an integer index identifying $\mathbf{
    u}_\alpha, $ i.\,e., $ \mathbf{u}_{\alpha'}\not= \mathbf{u}_{\alpha''}$
    if $ \alpha'\not=\alpha''.$   All $m(n)\le n$ different intervals 
    in~$\mathbf{y}_1^n$ can be ordered
    by their endpoints. Let write $\mathbf{y}_{i_1}\prec  \mathbf{
    y}_{i_2}$ if either $ y_{L i_1}< y_{L i_2},$ or $ y_{L i_1}=
    y_{L i_2}$ but $ y_{R i_1}<  y_{R i_2}.$ Then different intervals 
    $ \mathbf{y}_{i_1'} \prec \mathbf{y}_{i_2'}\prec \cdots 
    \prec \mathbf{y}_{i_{m(n)}'}$ and let $\mathbf{u}_{h n}
    =\mathbf{y}_{i_h'} $ for all $\mathbf{y}_i$ identical with $\mathbf{y}_{i'_h}$ 
    and $\mathbf{t}(n) =\{ t_{1, n}, \ldots, t_{m(n), n}\},$  
    $t_{h, n} =\sum_{i=1}^n I [ \mathbf{y}_i =\mathbf{u}_{h, n}].$
   The collected data    can be written as the following list:
   $$
 \mathbf{d1}_n =\{ \ldots, \{ h, \{  u_{Lh, n} , u_{Rh, n}\}
 , t_{h, n} \} , \ldots \} 
   $$
   where ordering indexes $h=1, \ldots, m(n)$ of different intervals $\mathbf{u}_{h,
   n}=( u_{Lh, n}, u_{Rh, n}]$ depend on the collected data $\mathbf{y}_1^n.$ 
   Finally,  let $\mathcal{U}_{m(n), n}=\{ \mathbf{u}_{1, n} , \ldots , \mathbf{u}_{m(n), n}\}$\linebreak
$\subseteq\mathcal{U}_{\mathrm{all}}$ be the set of all different
SeSeIs stated in~$\mathbf{y}_1^n.$

Consider now the sampling stopping problem. It is 
interesting to estimate the fraction of individuals 
(in~$\mathfrak P$) that will state  WTP-intervals already observed to be in $\mathcal{U}_{m(n), n}.$
  Let $p_c (n)$ be probability of the event that the last
  WTP-interval, $\mathbf{y}_n =\mathbf{u}_{h_n, n},$ in~$\mathbf{y}_1^n$,
$t_{h_n, n} \ge 2.$ Let~$H_n$ denote the
  r.v.\ that the $n$th individual states a~WTP-interval $\mathbf{u}_{H_n, n}$ 
  containing his/her WTP-value~$x_n$.
   Since $E [ I[t_{H_n, n}\ge 2]] =p_c(n)$, one can consider  
   $I[t_{h_n, n} \ge 2]$ as a~value of an unbiased
   estimator of $p_c(n).$ After averaging $I[t_{H_n, n}\ge 2]$ given the 
   sufficient statistic $\mathbf{t}(n)$, one obtains an unbiased
    estimate of $ p_c(n)$:
$$
 \hat p_c (n) =\fr{r(n)}{n} 
 $$
where $r(n) =\sum_{h=1}^{m(n)} t_{h, n} I[t_{h, n}\ge 2],$ and
    cap $\, \hat{\,}\,$ over $p_c(n)$ denotes an estimate~\cite{BK:BK13}.

To proceed, let reduce the problem to a~classical urn problem with $r(i)$
white and $i-r(i)$ black balls.
   Let 
   \pagebreak
   
   \noindent
   $$
   \mathcal{U}_{m(i), i} (1) =
   \{ \mathbf{u}_h :\, \mathbf{u}_h    \in\mathcal{U}_{m(i), i}, t_{hi} =1\}\,; 
$$
$$
\mathcal{U}_{m(i), i} (2) =\{ {\bf u}_h :\, {\bf u}_h
   \in\mathcal{U}_{m(i), i},\enskip t_{hi} \ge 2\}\,;
   $$
$$
\mathcal{U}_{m(i), i} =\mathcal{U}_{m(i), i} (1)\cup\mathcal{U}_{m(i), i}
   (2)\,.
   $$  
    Then
   $$
   \Delta_c(i) = 
   \begin{cases}
   \hat p_c(i+1) -\hat p_c (i) =\fr{i-r(i)}{i(i+1)}&\\[2pt]
&\hspace*{-20mm}     \mbox{ if }
   \mathbf{y}_{i+1} \in \mathcal{U}_{m(i), i}(2)\,;\\[6pt]
\fr{2i-r(i)}{i(i+1)} &\hspace*{-20mm}\mbox{ if } 
\mathbf{y}_{i+1} \in \mathcal{U}_{m(i), i}(1)\,;\\[6pt]
-\fr{r(i)}{i(i+1)} &   \hspace*{-20mm}\mbox{ if } 
 \mathbf{y}_{i+1} \not\in \mathcal{U}_{m(i), i}\,.
 \end{cases}
 $$

   The probability $p_c(n)$ is increasing in~$n.$ Hence, one can use $\hat p_c(n)$ as
an approximation of a~lower bound estimate for $p_c(n')$ for any 
  $n'> n.$ Then, one may   interpret $\hat p_c(n)$ 100\% as a~lower bound estimate of the
  percentage of all individuals in $\mathfrak P_c \subset \mathfrak P$
  who  would claim an interval ${\bf u}_h\in\mathcal{U}_{m(n), n}.$ As indicated above,
  $p_c(n)$ is called here a~\textit{coverage probability.}

    By calculating $\hat p_c(i)$ for each $i\le n$, one can observe
the  evolution of $\hat p_c(i)$ for some empirical data (Fig.~3).

\begin{center}  %fig3
\vspace*{12pt}
\mbox{%
 \epsfxsize=78.234mm
 \epsfbox{bel-3.eps}
 }

\end{center}

%\vspace*{3pt}

\noindent
{{\figurename~3}\ \ \small{The dynamics of the coverage probability
estimates $\hat p_c[i]$ as a~function of~$i$}}

\vspace*{9pt}

\addtocounter{figure}{1}



The decision to stop  collecting data on the first step can be determined by the value of 
$\hat {p}_c(n).$ If this value is not sufficiently close to~1 and if it is
  possible to extend data collection, then sampling continues.  Note that if
  the collection of data  has stopped with $ n_1\ge n$ and 
  $\hat {p}_c(n_1)= r(n_1) /n_1$, then inference about the WTP-distribution
  will correspond to the subset of individuals~$\mathfrak{P}_c$ with
  WTP-intervals already collected in $\mathcal{U}_{m(n_1), n_1}.$

The next step is to introduce the basic building block of the statistical model,
the so-called \textit{division intervals.} The basic idea is to divide each 
interval into smaller pieces
where those pieces are obtained from the responses of the respondents.

\subsection{Division intervals}

\noindent
  Henceforth, the number~$n_1$ of randomly sampled respondents is fixed and 
  it is suppressed
  it in the sequel;  let write $\mathcal{U}_m =\{ \mathbf{u}_1, \ldots, {\mathbf
  u}_m \}$ instead of $\mathcal{U}_{m(n_1), n_1}$\linebreak
  $= \{ \mathbf{u}_{1, n_1}, \ldots, {\mathbf
  u}_{m(n_1), n_1} \}$ and~$t_h$~instead of $ t_{h, n_1}.$
Let simplify and use
 $ \mathcal{U}_m$, $\mathbf{u}_h$,  $\mathbf{v}_j$, $\mathcal{C}_h$,
  and $\mathcal{D}_j$ instead of
  $ \mathcal{U}_{m_{n}, n_1}, \mathbf{u}_{h n_1}, \mathbf{v}_{j n_1}, 
  \mathcal{C}_{h n_1}$, and $\mathcal{D}_{j n_1}.$

  Let $v_0 < v_1 < \cdots < v_{k-1} < v_k$ be ordered values of the end
  points of all intervals $\mathbf{u}_h\in\mathcal{U}_m.$
Let $v_{L j} = v_{j-1} $ and $  v_{Rj}=v_j.$
  Then, the intervals   in the following collection
  $ \mathcal{V}_k =\{ \mathbf{v}_1, \ldots, \mathbf{v}_j, \ldots , \mathbf{v}_k \}$, 
  $\mathbf{v}_j=(v_{Lj}, v_{Rj}]$,   
  $j=1, \ldots, k,$ are called \textit{division intervals}.

    Each interval ${\bf u}_h \in
  \mathcal{U}_m$ is,  thus, the union of disjoint division intervals
 % \begin{equation}\label{FR-XX1}
 $$
   \mathbf{u}_h = \bigcup_{j\in\mathcal{C}_h} \mathbf{v}_j\,;
    \quad \mathcal{C}_h =\left\{ j:\; {\bf v}_j \subseteq \mathbf{u}_h\right\} \,.
    $$
%\end{equation}
Below,  the following sets of indices~$h$ will be also used:
%\begin{equation}\label{FR-XX2}
$$
\mathcal{D}_j= \left\{ h:\; \mathbf{v}_j \subseteq \mathbf{u}_h \right\}\,, \enskip j=1,
    \ldots, k\,. 
    $$
For example, if there are used the data in $S_2\cup S_3$, then
$v1 = (0, 5]$; $v2 = (5, 10]$; $v3 = (10, 15];$  
$ v4$\linebreak $ = (15, 20]$; $v5 = (20, 25]$; $v6 = (25, 30];$ 
$ v7$\linebreak $ = (30, 40]$; $v8 = (40, 50]$;  $v9 = (50, 60];$ 
$  v10 = (60, 70]$; $v11 = (70, 75]$;  $v12 = (75, 80];$ 
$  v13 = (80, 100]$; $v14 = (100, 150];$ 
$  v15 = (150, 170]$; $v16 = (170, 200]$; 
$ v17 = (200, 250]$;  $v18 = (250, 300];$ 
$v19 = (300, 400]$;     $v20 = (400, 500]$;
$ v21 = (500, 600]$;   $v22 = (600, 1000];$ 
$v23 = (1000, 1000]$. 
For $j=21$, $\mathbf{v}_{21} =(500, 600]$.
 Then,
$\mathcal{D}_{21} =\{ 38, 43, 45, 46 \}$; $\mathbf{u}_{38} =( 100, 1000];$ 
$ \mathbf{u}_{43} =( 300, 600]$;
$\mathbf{u}_{45} =( 500, 1000]$; 
$\mathbf{u}_{46} =( 500, 2000]$.
An informative picture of the data was obtained
by considering how each stated interval can be
constructed via properly selecting division intervals (Fig.~4).



Respondents' WTP-points $\{ x_i\}$   by Assumption~3
are values of $\{ X_i\}$ i.i.d.\ r.v.s.   Let define even
$ \{ H_i =h\} \subset\{ X_i\in \mathbf{u}_h\}, \, w_h= P[\{
H_i=h\}\le p_h$\linebreak $= P[X_i \in \mathbf{u}_h]$, $h=1, \ldots, m.$ 
Note that events\linebreak $\{ H_i =h\} $ are
 observable, but not all of  $\{ X_i \in \mathbf{u}_{h'}\}$, $h'\not= h,$
 $\mathbf{u}_{h'}\cap{\bf u}_h \not= \emptyset$ are
observable. The probability, of the number of times $ t_{1 n_1}, \ldots, t_{m n_1}$
that $\mathbf{u}_1, \ldots , \mathbf{u}_m$ had appeared in~$\mathbf{y}_1^n$ 
is proportional to $\prod_{h=1}^m w_h^{t_{h n_1}},$  i.\,e.,
one has a~multinomial distribution. The corresponding
(normed by~$n_1$) log likelihood (llik) is
\begin{align*}
  \mathrm{llik} \left[ w_1, \ldots, w_m \vert t_1, \ldots, t_m \right]&= \sum\limits_{h=1}^m
  \fr{t_{h n_1}}{n_1}\,\mathrm{Log} \left[w_h\right]\,;\\
  \sum\limits_{h=1}^m t_{h n_1} &=n_1\,.
\end{align*}
The maximum of llik over $w_h\ge 0$, $\sum_{h=1}^m w_h =1$,
is attained at $ \check w_h =t_{h n_1}/n_1$, $h=1, \ldots, m.$
 Note that $e[\hat{\mathbf{w}}_1^m ]= -\sum_{h=1}^m \hat{w}_h \mathrm{Log} 
 [\hat w_h]$ is the empirical en-\linebreak\vspace*{-12pt}
 
 \pagebreak
 
 \begin{center}  %fig4
\vspace*{1pt}
\mbox{%
 \epsfxsize=74.792mm
 \epsfbox{bel-4.eps}
 }

\end{center}

%\vspace*{3pt}

\noindent
{{\figurename~4}\ \ \small{The set of all compatible indexes $h, j,$ 
 $ \mathbf{v}_j \subseteq {\bf u}_h, $ $h=1, \ldots , 46$, $j=1,
\ldots , 23$ in the empirical data. The sets $\mathcal{C}_h =
\{ j, \mathbf{v}_j \subseteq \mathbf{u}_h \}$ and $ \mathcal{D}_j=\{ h, 
\mathbf{v}_j \subseteq \mathbf{u}_h \}$ are the $h$-cuts and $j$-cuts of the
shown set, e.\,g., $\mathcal{C}_{10} =\{ 3, 4, 5, 6, 7\}$ and $\mathbf{u}_{10} =
\cup_{j=3}^7 \mathbf{v}_j$}}

\vspace*{9pt}

\addtocounter{figure}{1}
 

\noindent
tropy of the multinomial
distribution with probabilities $ \hat{\mathbf{w}}_1^m=\{ \hat w_1, \ldots, 
\hat w_m \}.$


\subsection{The Turnbull assumption}

\noindent
Let consider statistical models where there are  finite numbers~$m$
of different SeSeIs~$\mathbf{y}_i$, $i=1, \ldots, n_1$, each
contains one unobserved point $X_i=x_i.$ 
The target is nonparametric estimation of the underlying
distribution function. As it is common in the survival analysis, let assume a~kind of
independence. Here, it means that an interval containing value~$x_i$ has been stated
independently of the position of~$x_i$ in the interval.  In statistical models with
exogenously given intervals, these assumptions are called  the \textit{noninformative condition}
(see, e.\,g.,~\cite{BK:GCO04, BK:GCOL09}).
 In the considered context, this  restrictive assumption  on independence of positions (AIP) WTP-points
 included in SeSeIs can be stated as  follows.
 
 \noindent
 \textbf{Assumption~4}.\
\textit{$AIP: $ For each  pair of $\{ h, j\} ,$ $ h=1, ..., m $ and}
$ j\in\mathcal{C}_h$ for events $\{ H_i =h\}$ and $\{ X_i \in {\bf v}_j\}$,
\begin{multline*}
P_o \left[ \left\{ X_i \in {\bf v}_j\right\} \cap \{ H_i =h\}\right] =
q_{oj}w_{oh} I [j\in \mathcal{C}_h ]\,,\\
j\in\mathcal{C}_h, h=1, \ldots, m.
\end{multline*}


  From this assumption, it follows that
  \begin{equation*}
 P_o [ \{ H_i=h, X_i \in \mathbf{y}_i =  \mathbf{u}_h =
 \cup_{j\in\mathcal{C}_h} \mathbf{v}_j\}] =   w_{oh} p_{oh}
 \end{equation*}
  where $ P_o$ is the~true probability; $w_{oh}=P_o[H_i= h]$;
$p_{oh} =\sum_{j\in\mathcal{C}_h} q_{oj}$ with
$q_{oj} =P_o [X_i \in \mathbf{v}_{j'}]$; $\mathbf{v}_j  \subseteq \mathbf{u}_h$.

  This assumption is basic to Turnbull~\cite{BK:TU74, BK:TU76} and corresponds to the case where all end points
  of censoring intervals  are known before realizations of r.v.s~$X_i$, and the conditional
  distribution of~$X_i$, $i=1, 2,\ldots,$ given these points, does not depend
   on the ends of these intervals. Turnbull's assumption
    on conditional probabilities of $ X_i \in u_h, $ given 
    $ \{ \mathbf{Y}_i=(u_{Lh},  u_{Rh}] \} ,$
    implies that $ p_{o h}=P_o\{ X_i \in \mathbf{u}_h \}$. This assumption is
    enough for obtaining a~simplified  version of the llik 
    (see~\cite{BK:GCO04}).

  By AIP,  additional
probabilities~$ w_{oh}$, $h=1$,\linebreak $2, \ldots, m,$ have been
introduced which are the true probabilities that the $i$th respondent selects
$ \mathbf{u}_h \in U_m $ such that $ X_i \in \mathbf{u}_h $. Note that since  
$ I[X_i \in \mathbf{u}_h]=1 $
and due to presence $ \mathbf{u}_h \in \mathbf{y}_1^n$,  the probabilities~$ w_{oh} $ 
should be positive, $h=1, \ldots , m.$
 In addition, all probabilities $p_{oh} =P_o [X_i \in \mathbf{u}_h ]>0$, 
 $h=1, \ldots , m,$ are positive.

We do not know $w_{oh}$ and  parameters~$q_{0j}$ where $ \{ h, j\}$
are the compatible pairs. The natural consistent estimates of~$w_{oh}$ are the
frequencies

\noindent
$$
\hat w_{h}(n_1) = \fr{t_{hn_1}}{n_1}\,, \quad t_{hn_1} =\sum\limits_{i=1}^{n_1} I\left[h_i=h\right]\,.
$$
Such probabilities are the basic ingredients of the likelihood, to which let now turn.

\vspace*{-9pt}

\subsection{The likelihood}

\noindent
The current problem is to find consistent estimates for
points of the true s.d.f.\  $S_{ot} [\cdot ].$
 One has the following normed log likelihood
 $(\mathrm{llik}_{\mathrm{AIP}})$  with parameters $\mathbf{q}_1^k$ and~$\mathbf{w}_1^m$:
 
 \vspace*{-2pt}
 
 \noindent
  \begin{multline*}
 \mathrm{llik}_{\mathrm{AIP}}\! \left[\mathbf{q}_1^k, 
 \mathbf{w}_1^m\mid \mathbf{y}_1^{n_1} \right] =  
 \fr{1}{n_1} \mathrm{Log}\! \left [ \prod\limits_{i=1}^{n_1} \!\left ( \!w_{h_i} \hspace*{-1mm}
 \sum\limits_{j\in\mathcal{C}_{h_i}}\hspace*{-1mm}q_j\!\right )\! \right ] \\
{} = \fr{1}{n_1} \sum\limits_{i=1}^{n_1} \mathrm{Log} \left [
 w_{h_i} \sum\limits_{j\in\mathcal{C}_{h_i}}q_j \right ]= f_{ll} 
 \left[\mathbf{q}_1^k\right] +  \hat{e}\left[\mathbf{w}_1^m\right]
 \end{multline*}
where

\vspace*{-1pt}

\noindent
\begin{multline*}
 f_{ll}  \left[\mathbf{q}_1^k\right]= \sum\limits_{h=1}^m \hat{w}_h (n_1) 
 \mathrm{Log}  \left [ \sum\limits_{j\in\mathcal{C}_h} q_j\right ]\,, \\
\mathbf{q}_1^k =\left\{ q_1, \ldots, q_k\right\}; 
\end{multline*}

\vspace*{-12pt}

\noindent
\begin{equation*}
\hat{e}\left[\mathbf{w}_1^m \right]= \sum\limits_{h=1}^m \hat w_h (n_1) \mathrm{Log}
 \left[ w_h \right]\,,
\enskip  \mathbf{w}_1^m =\left\{ w_1, \ldots,  w_m \right\}\,.
\end{equation*}
To save on notation,  let write $ \mathrm{llik}_{\mathrm{AIP}}$ instead 
of $\mathrm{llik}_{\mathrm{AIP}}[ \mathbf{q}_1^k , \mathbf{w}_1^m\mid \mathbf{y}_1^n ].$
Note that llik$_{\mathrm{AIP}}$ depends on~$\mathbf{q}_1^k$ only thro\-ugh 
 $\mathbf{p}_1^m =\{ p_1, \ldots, p_m \}$.
 Hence, let consider llik$_{\mathrm{AIP}}$ as a~function of~$\mathbf{p}_1^m$ 
 and write it 
 as llik$_{\mathrm{AIP}}[\mathbf{p}_1^m].$
Due to presence of all $\mathbf{u}_h \in\mathcal{U}_m$ in the data~$\mathbf{y}_1^n$, 
there is a~small positive number $\varepsilon >0$ such that  the global maximum of
llik$_{\mathrm{AIP}}$ is contained in the following compact convex multidimensional polyhedron:

\vspace*{-4pt}

\noindent
\begin{multline*}
\mathcal{S}_{k-1} =
\left\{ \mathbf{q}_1^k: 0\le q_j <1,\ \sum\limits_{j=1}^k q_j =1,\right.\\[-3pt] 
p_h =\sum\limits_{j\in\mathcal{C}_h} q_j \ge \varepsilon >0,\  
\left.
h=1, \ldots, 
\vphantom{\sum\limits_{j=1}^k}
m\right\}\,.
  \end{multline*}

 If $k=4$, then   $\mathcal{S}_3 \subseteq \mathbb R^3$ is the~main part of an equiedged pyramid with $k=4$ vertices.
 The case when ${\bf q}_1^k \in \mathcal{S}_{k-1}$ have zero components is not excluded, 
  i.\,e., $q_j =0$ for some of~$j$.

Let now state an important property by means of the following  
lemma and then obtain the concavity of the llik.

\smallskip

\noindent
\textbf{Lemma~4.1.}\ 
\textit{$f_{ll} [{\bf q}_1^k]$ is concave on $\mathcal{S}_{k-1}$}.

\smallskip

\noindent
{P\,r\,o\,o\,f\,.}\ \
For any pair of points $\mathbf{q}_1^k (i)\in \mathcal{S}_{k-1}$, $i=1, 2,$
 $\mathbf{q}_1^k(1) \not= \mathbf{q}_1^k(2), \mathbf{q}_1^k [t] =(1-t) \mathbf{q}_1^k (1) +
 t \mathbf{q}_1^k (2) \in\mathcal{S}_{k-1}$,
 $0\le t\le 1.$ Then, for $p_h [t] =\sum_{j\in\mathcal{C}_h} q_j [t] \ge \varepsilon >0$, 
 $h=1, 2,\ldots, m$, one has:
\begin{align*}
 \fr{ d^2 \mathrm{Log} \left[p_h[t]\right]}{dt^2} &=- \sum\limits_{j\in \mathcal{C}_h}
 \left (\fr{q_j(2)-q_j(1)}{p_h[t]}\right )^2<0\,;
\\
 \fr{d^2 \mathrm{llik} \left[\mathbf{q}_1^k[t]\right]}{dt^2}& = 
 \sum\limits_{h=1}^m \hat{w}_h (n_1) \fr{d^2 \mathrm{Log} \left[p_h[t]\right]}{dt^2}< 0
\end{align*}
 because $\hat w_h(n_1) >0$, $h=1,2,\ldots, m$.

It follows that for any~$\mathbf{q}_1^k(i)$, 
$i=1, 2,\ldots$, on interval $I[q_1^k (1), q_1^k (2)]$
connecting these points, $f_{ll} [q_1^k ]$ is concave.

\smallskip

\noindent
\textbf{Theorem~4.2.}\
\begin{itemize}
\item[$(i)$] \textit{The $\mathrm{llik}_{\mathrm{AIP}} 
[\mathbf{q}_1^k]$ is concave on the}
  $\mathcal{S}_{k-1}.$
\item[$(ii)$] \textit{The $ \mathrm{llik}_{\mathrm{AIP}}$  has a~stationary point 
$\check{\mathbf{q}}_1^k =\{ \check q_1, \ldots$\linebreak $\ldots,
  \check q_k\} \in \mathcal{S}_{k-1}$ where it attains its global
  maximum of  $\mathrm{llik}_{\mathrm{AIP}} [\mathbf{q}_1^k]$ on $\mathcal{S}_{k-1}$
  and
  $\check{\mathbf{p}}_1^m= \{ \check p_1, \ldots, \check p_m\}$, 
  $\check p_h =\sum_{j\in \mathcal{C}_h} \check q_j$,
   is the maximum likelihood (ML) estimate of the true parameters 
   $ \mathbf{p}_{o1}^m$ given $\hat{\mathbf{w}}_1^{m} (n_1)$.}
\end{itemize}

%\smallskip

\noindent
{P\,r\,o\,o\,f\,.}\ \
 The part $e(\mathbf{w}_1^m)$ does not depend on $\mathbf{q}_1^k$; 
 hence, from Lemma~4.1, it follows
 that llik$_{\mathrm{AIP}}$ is concave on $\mathcal{S}_{k-1}.$
At a~stationary point $s\in\mathcal{S}_{k-1}$, all first-order partial derivatives
have to be zero and there are some negative second order derivatives.
To include constrains  $q_j (s)\ge 0$, $\sum_{j=1}^k q_j(s) =1$, 
the method with the following Lagrange function is used:
$$
\varphi \left[\mathbf{q}_1^k (s)\right] =\mathrm{llik}
\left [\mathbf{q}_1^k(s)\right] +\sum_{j=1}^k q_j(s) \left(\mu_j -\mu_o\right)\,.
$$
Here, $\mu_j$ and $\mu_o$ are the
multipliers satisfying the Kuhn--Tucker conditions $\mu_jq_j (s)=0$, 
$\mu_j\ge 0$, $j=1, \ldots , k$~\cite{BK:GG94, BK:BT05}. Then,
one has~$k$~equations, $j=1, \ldots, k$:
\begin{equation*}
 \fr{\partial \varphi [q_1^k(s)]}{\partial q_j(s)} =
\sum\limits_{h=1}^m \hat w_h (n_1) \fr{I[j\in \mathcal{C}_h]}
{\sum\limits_{j'\in \mathcal{C}_h} q_{j'}(s)}
+ (\mu_j -\mu_o) =0\,.
\end{equation*}
By multiplying each of these equations by $q_j(s)$ and sum them using 
$\mu_j q_j (s)=0$, $j=1, \ldots, k,$ one has $\mu_o =1$.
Then, for each~$j$ after multiplying both sides of these equations by $q_j(s)$,
one has
$$
\sum\limits_{h=1}^m \hat{w}_h (n_1) \fr{I[j\subset \mathcal{C}_h] q_j(s)}
{\sum\nolimits_{j'\in\mathcal{C}_h} q_j(s)} -q_j (s) =0\,,\ 
j=1, \ldots, k.
$$

If some of $q_j(s)=0$, $j\in J_o=\{ j':\ q_{j'} (s)=0\}$\linebreak $\not= \varnothing$, then,
 to guarantee that the stationary point $\mathbf{q}_1^k(s)\in\mathcal{S}_{k-1}$,
 it is necessary to check that all
$p_h(s) =\sum_{j\in\mathcal{C}_h} q_j(s) >0$, $h=1, \ldots, m.$
If so, then $\mathbf{q}_1^k(s)$ is placed in the side 
of~$\mathcal{S}_{k-1}$ where $q_j =q_j (s)\equiv 0$, $j\in J_o,$
and $q_j(s)>0, j\in J_p=\{ j':\; q_{j'} (s) >0\}.$
Due to concavity, the stationary point $q_1^k (s)$ is unique and it attains 
the global maximum of the llik.

\smallskip

  This likelihood can be calculated by a~recursion next described.

\subsection{Numerical recursive method}

  


\noindent
The authors suggest a~method efficient  for numerical  calculation of a~sequence
 of $\mathbf{q}_1^{(r)k}$, $r=1, 2, \ldots,$ converging to the point 
 $\mathbf{q}_1^k (s)$ of the global maximum
llik$_{\mathrm{AIP}}[\mathbf{q}_1^k (s)]$
 where $\mathbf{q}_1^k(s)$ is a~stationary point of the llik.
Henceforth, let write $q_j^{(r)}(n_1)$ and $q_j(s, n_1)$ instead of 
$q_j^{(r)}$ and $q_j(s)$ 
to underline  dependence on~$n_1.$
One can consider the following operator, sequentially transforming points
 $\{ q_1^{(r)} (n_1), \ldots, q_k^{(r)}(n_1) \} $ into points 
 $\{ q_1^{(r+1)}(n_1), \ldots, q_k^{(r+1)}(n_1)\} ,$ $k=1, 2,\ldots$
in~$\mathcal{S}_{k-1}$:
$$
q_j^{(r+1)}(n_1) =\sum\limits_{h=1}^m \hat{w}_h (n_1) \fr{q_j^{(r)}(n_1)I[j\in
\mathcal{C}_h]}{\sum\nolimits_{j'\in\mathcal{C}_h} q_{j'}^{(r)}(n_1)}\,.
$$
The set $\mathcal{S}_{k-1}$ is  compact and  
$\sum_{j=1}^k q_j^{(r)}(n_1) =1$ for any~$r$. Hence, by the Banach theorem,
 $\mathcal{S}_{k-1}$ contains at least one limit point of the sequence 
 $\mathbf{q}_1^{(r)k}(n_1) =\{ q_1^{(r)}(n_1), \ldots, q_k^{(r)}(n_1)\}$,
 $r=1, 2, \ldots$ Due to concavity
 of the llik, there  is only one limit point $\mathbf{q}_1^{k(\infty )}(n_1)=
 \{ q_1^{(\infty )}(n_1) , \ldots , q_k^{(\infty )}(n_1) \}$.
 The above equations are satisfied for this stationary point, i.\,e.,
$$
\mathbf{q}_1^{k(\infty )}(n_1) =\mathbf{q}_1^k (s, n_1)\,.
$$

The corresponding recursive method of sequential convergent calculation
   coordinates of the stationary point can be based on this relation. The
   calculation started with~$k$~initial values, e.\,g., $q_j^{(0)} =1/k$,
   $j=1, \ldots, k,$ then next $k$ values $q_j^{(1)}(n_1)$ will be
   calculated and so on after $k$ values $q_j^{(r)}(n_1),$ next~$k$~values 
   $q_j^{(r+1)}(n_1)$ can be calculated. The process is stopped when maximal
   variation of $q_j^{(k+1)}(n_1) - q_j^{(k)}(n_1)$ is negligibly  small.
   The plot in Fig.~5 illustrates convergence of the llik values at
   points  $\mathbf{q}_1^{(r)k}$ to the global maximum.
   
    \begin{center}  %fig5
\vspace*{1pt}
\mbox{%
 \epsfxsize=77.989mm
 \epsfbox{bel-5.eps}
 }
 
\end{center}

%\vspace*{3pt}

\noindent
{{\figurename~5}\ \ \small{Dynamics of llik$_{\mathrm{AIP}}$ values as functions of  recursion's iterations~$rc$
  with two different starting points. The two starting points have~$q_j$ equal~1/23~(\textit{1}), and $ j/276$~(\textit{2}), $1\le j\le k = 23.$ Time needed for 2000~iterations is near
   20~s (Intel processor 3.2~GHz)}}

\vspace*{9pt}

\addtocounter{figure}{1}
 
   
      

 All frequencies $\hat{w}_h(n_1)\to w_{oh}$, $h=1, \ldots, m,$ a.s.\ 
 as $n_1\to\infty.$ Hence,
 at any point $\mathbf{q}_1^k \in\mathcal{S}_{k-1}$, the llik$_{\mathrm{AIP}} 
 [\mathbf{q}_1^k, \hat{w}_1^m (n_1) ]$
 converges a.s.\ to the \textit{limit llik}:
 
 \noindent
$$ 
\mathrm{llik}_{\infty } \left[\mathbf{q}_1^k\right] =
\sum\limits_{h=1}^m  w_{oh} \mathrm{Log} \left [ \sum\limits_{j\in\mathcal{C}_h} q_j\right ]
 +e\left[\mathbf{w}_{o1}^m \right]\,.
 $$
 Due to compactness of $\mathcal{S}_{k-1}$, one has a.s.\ convergence in the uniform metric, i.\,e.,
 
 \noindent
$$
\max\limits_{\mathbf{q}_1^k\in\mathcal{S}_{k-1}} \mid 
\mathrm{llik}_{\mathrm{AIP}} \left[\mathbf{q}_1^k, \hat{\mathbf{w}}_1^m (n_1)\right] - 
\mathrm{llik}_{\infty } [{\bf q}_1^k ] \mid \to 0
$$
 a.s.\ as $ n_1\to\infty$.

   The concavity of the llik$_{\mathrm{AIP}}$  implies that
   
   \noindent
\begin{multline*}
\mathrm{llik}_{\mathrm{AIP}} \left[
{\bf q}_1^{k(\infty )}(n_1)\mid  \hat{\bf w}_1^m (n_1), {\bf y}_1^{n_1}\right]\\
{}=
\max\limits_{\mathbf{q}_1^k\in\mathcal{S}_{k-1}} \mathrm{llik}_{\mathrm{AIP}} 
\left[\mathbf{q}_1^{k}(n_1)\mid  \hat{\mathbf{w}}_1^m (n_1), \mathbf{y}_1^{n_1}\right]\,.
\end{multline*}
   If together with $r\to\infty $ also $n_1\to\infty$, 
   then, from the uniform  convergence, one has:
   
   \noindent
\begin{gather*}
   \mathrm{llik}_{\mathrm{AIP}}\left[
   \mathbf{q}_1^{k}\mid  \hat{\mathbf{w}}_1^m (n_1) \mathbf{y}_1^{n_1}\right]\to 
   \max\limits_{\mathbf{q}_1^k} \mathrm{llik}_\infty \left[\mathbf{q}_1^k \right]\,;\\
\  \sum\limits_{j\in\mathcal{C}_h} q_j^{(r)}(n_1)\to p_{o h} =
     \sum\limits_{j\in\mathcal{C}_h} q_{oj}, 
\mbox{ a.s.\ } r\to\infty,\  n_1\to\infty .
\end{gather*}
 Note that here, it is not stated that  $ q_j^{(r)}(n_1) \to q_{oj}$ as~$r$ 
 and~$n_1$ are growing unboundedly but
 the sums $\sum_{j\in\mathcal{C}_h} q_{oj}$ are consistently estimable if AIP is valid.
Note that the present proof of consistently is different from that
suggested in~\cite{BK:JM03}.

 One may consider $\hat a_{hj}=\hat w_h/\sum_{h'\in\mathcal{D}_j} \hat w_{h'}$ as an 
 \textit{empirical index of
attractiveness} of interval~$\mathbf{u}_h$ with rounded ends. In the data discussed 
in section~2 for $h=27$,
this index is~0.740638;  for $h=32$, it is~0.203977; and for $h=2$, it is~0.015431.
 $a_{ohj}= w_{oh}/p_{oh}$  denote  the \textit{true index} of 
 \textit{attractiveness} of~$\mathbf{u}_h$, $h=1, \ldots, m.$

It is possible to simulate data with~$n$~SeSeIs when AIP is valid. It is necessary to define
 a~c.d.f.\ $F_o[x]$, $x >0$, of i.i.d.\ WTP r.v.s $X_1, \ldots, X_n,$ and to define a~collection
 $\mathcal{U}_m =\{ \mathbf{u}_1, \ldots, \mathbf{u}_m\} $ with SeSeIs. 
 Then, it is possible to find all
 division intervals $V_k =\{ \mathbf{v}_1 , \ldots, \mathbf{v}_k \} $ and 
 all corresponding subsets of indices~$\mathcal{C}_h$, $h=1, \ldots, m,$ 
 and~$\mathcal{D}_j$, $j=1, \ldots, k.$ For each~$i$, $i=1, \ldots, n,$ 
 a~value $X_i=x_i$ is simulated.
 If $X_i =x_i\in \mathbf{v}_{j_i}$, then an index $H_i =h_i$ is sampled randomly 
 from $\mathcal{D}_{j_i}$ with probability proportional to  $w_{h_i}, h_i\in\mathcal{D}_j.$
 The indices $H_i, \ldots, H_n$ are sampled independently and the simulated data,
 $\mathbf{y}_1^n = \{ \mathbf{y}_1, \ldots, \mathbf{y}_n \}$,  
 $\mathbf{y}_i =\mathbf{u}_{h_i},$ are obtained.

It is seen that at least for  data in section~2, the AIP may not necessarily hold. 
Before turning to the more general assumption,
let introduce an assumption on the response process which seems to fit  data better.

\vspace*{-9pt}

\subsection{Another behavioral assumption~--- on preference of~the~right division interval}

\noindent
The pilot statistical analysis (see Fig.~1) 
 suggests that the last division intervals contain most
 of the values of interest~$x_i$, $i=1, \ldots, 46.$ In this case, both 
 rounded ends and positions of~$x_i$ are
 essential for the individual's choice of SeSeI.  If so, the AIP is not valid and another assumption
 is needed.
  To  sharply outline this case, let introduce the following assumption on preference of the right division interval (ARDI).

\smallskip

\noindent
\textbf{Assumption~5.}
\textit{ARDI: In each stated SeSeI 
$\mathbf{y}_i$\linebreak $ =(u_{Lh_i} , u_{Rh_i}] \in \mathcal{U}_m$,
 the value of interest~$x_i$
   is in the right division interval $ \mathbf{v}_{j_{h_i} \mathrm{RD} }
   =(v_{Lj_{h_i}}, v_{Rh_i}]$, $v_{Rj_{h_i}} = u_{Rh_i}$, $i=1, \ldots, n_1$.
}

\smallskip

    Here, one has
    $\{ H_i =h\} \cap \{ X_i \in \mathbf{u}_h\} =\{ H_i$\linebreak $ =h\} \cap\{ X_i \in 
    \mathbf{v} _{j_h \mathrm{RD}} \}$.

    Then, $ P [ \{ H_i =h\} \cap \{ X_i \in \mathbf{v} _{j_h \mathrm{RD}} \} ]= 
    w_h q_{j_h \mathrm{RD}}$, $j_h\in\mathcal{C}_h$.


Note that in the empirical data, there are no intervals 
$\mathbf{u}_h \in\mathcal{U}_m(n_1)$ where~$\mathbf{v}_1$ or~$\mathbf{v}_5$ are 
the last division intervals.
In~Fig.~4, it is seen that there are no pairs $\{ h, 1\}$ and $\{ h, 5\}$ 
where~$\mathbf{v}_1$ and~$\mathbf{v}_5$
 are the end intervals in~$\mathbf{u}_h$, $h=1, \ldots, 46.$
 Here,  the ML-estimates of $q_{oj}, j\not=1, 5,$ have been found.

  Thus, the assumptions AIP and ARDI imply essentially different ML-estimates of several points
  in the estimated s.d.f. Due to consistency of these estimators, if both assumptions
  are true, one should expect that the estimators have to be  similar. But the positions of points on s.d.f.s
 differ essentially.
   One may suspect that either at least one or even both assumptions are erroneous because
  the data correspond to the same sampled respondents.  One possible way to eliminate this
  contradiction is an extension of the used questionnaire; more collected empirical information 
  is needed to overcome the identification problem.
  
  \vspace*{-6pt}

\section{Two-Step Intervals} %\label{sec:two-step}

\noindent
To address the identification problem in a~nonparametric setting,
the authors propose the second-step approach. In addition,
let introduce a~more general assumption that has the Turnbull assumption as a~special case. The data collected with
a questionnaire extended in this way can be applied in
obtaining consistent estimates of points on the s.d.f.\ of interest. The following
assumption on the selection of a~division interval (ASDI) containing our value of interest is proposed.

\smallskip

\noindent
\textbf{Assumption~6.}
\textit{ASDI: Probabilities $w_{hj}, j=1, \ldots, k$, $h\in\mathcal{D}_j$, 
to state an SeSeI~$\mathbf{u}_h,$
 given $X_i \in \mathbf{v}_j\subseteq \mathbf{u}_h$, do not depend on the exact position 
 $X_i \in \mathbf{v}_j\subseteq \mathbf{u}_h$}.

\smallskip


Note that AIP and ARDI are the special cases of \mbox{ASDI}. In the case 
AIP, $w_{h} = w_{h_j}$, $h\in\mathcal{D}_j,$ and in the case \mbox{ARDI},
$w_{h} = w_{h_j\mathrm{RD}}$  
where~$h_{j \mathrm{RD}}$ corresponds to the last right division
interval $\mathbf{v}_{j\mathrm{RD}}$
in the stated SeSeI~$\mathbf{u}_h$. 
In the next subsection, mbox{ASDI} is supposed to be valid.

\vspace*{-6pt}

\subsection{The second step}

\vspace*{-1pt}

\noindent
  Let now introduce the \textit{extended second step\/} of data collection. The authors
  prolong  random sampling of new (not yet sampled) individuals from the population 
  $\mathfrak {P}$. First, each individual is to announce  an
interval containing his/her  WTP-point.
  If the interval does not belong to~$\mathcal{U}_m$, then it is not
  included in the collected data. If the interval~$\mathbf{u}_h$ belongs to~$\mathcal{U}_m$, then
 this respondent is asked to select  from the division
  $\mathcal{V}$  an interval $ \mathbf{v}_j\in\mathcal{V}_k, \mathbf{v}_j\subseteq 
  \mathbf{u}_h$,   containing his/her true  WTP-point.
 The respondents may well abstain from answering this second question, it is 
 just recorded this in the data.
   The collected data will be the list of triples
   
   \vspace*{2pt}
   
   \noindent
$$
%\label{FR-511}
 {\bf z}_i = \{i, {\bf u}_{h_i}, \mathrm{NA}\}\,;
 $$
 
 \vspace*{-2pt}
 
 \noindent
 or
 
 \vspace*{2pt}
 
 \noindent
 $$
\{ i, {\bf u}_{h_i}, {\bf v}_{j_i} \} ,\;\; {\bf d2}_{n_{\cdot 2}}=\{ {\bf z}_1, ...,
{\bf z}_{n_{\cdot 2}}\}
$$

\vspace*{-2pt}

\noindent
where NA is  ``no answer'' to the additional question. For each individual, we thus have an interval and, potentially,
a~selected division interval. These triples have been called
 \textit{singles} and 
\textit{pairs.}, depending on whether the individual reported or do not reported 
a~division interval.

The following notations have been used:

\noindent
\begin{align*}
c_{pj} &=\sum\limits_{i=1}^{n_{\cdot 2}} I [ \mathbf{z}_{i} = \{ i, \mathbf{u}_{h_i},
 \mathbf{v}_j\}]\,;\\ 
c_{phj}&=  \sum\limits_{i=1}^{n_{\cdot 2}} I [ \mathbf{z}_{i} =\{ i,
 \mathbf{u}_{h},  \mathbf{v}_{j} \}]\,;
 \end{align*}
  \begin{align*}
n_{s2} &=\sum\limits_{h=1}^m t_{sh}\,;\enskip n_{p 2} =\sum_{j=1}^k c_{p  j}\,;\\
  t_{sh}&= \sum\limits_{i=1}^{n_{\cdot 2}} I [\mathbf{z}_i =\{ i, \mathbf{u}_{h}, NA\}]\,;
\enskip
n_{\cdot 2}  =n_{s2}+  n_{p2}\,.
\end{align*}

\noindent
  The subindexes $s$ and~$p$ correspond to \textit{singles} and
  \textit{pairs}.
  Henceforth, it is supposed that all $q_j>0$, $j=1, \ldots, k,$ then for all sufficiently 
  large~$n_{\cdot 2}$,
  $c_{pj}>0$ for all $j=1, \ldots, k$.

  The strongly consistent estimates of~$q_{o j}$ and~$w_{hj}$  in pairs
are:
\begin{gather*}
 \check q_{pj}=\fr{c_{pj}}{n_{p2}}\to q_{o j}\,;\\
 \left\{ \hat{w}_{hj}=\fr{c_{phj}}{c_{pj}} \to w_{o hj}\,,  \enskip
 h\in\mathcal{D}_j\,,\enskip    j=1, \ldots, k\right\} \,,
\end{gather*}
  a.s.\ $j=1, \ldots , k,$  as $ n_{\cdot 2}\to\infty$.

\subsection{The likelihood}

\noindent
Let use estimates $\hat{w}_{hj}$ instead of~$w_{hj}$, $h\in\mathcal{D}_j$, 
  $j=1, \ldots, k.$
 The $i$th respondent contributes to  the estimated llik in two different ways
 depending on whether or not the follow-up question was answered:
\begin{multline*}
%\label{FR-521}
\mathrm{ll}_i \left[\mathbf{q}_1^k \mid  {\bf d2}_{n_{\cdot 2}}\right]\\
{} =  I\left[z_i =\left\{  i, \mathbf{u}_{h_i}, \mathrm{NA}\right\}\right] \mathrm{Log} 
\left [ \sum\limits_{j\in\mathcal{C}_{h_i}} \hat{ w}_{h_ij}q_j \right ] \\
{} +  I\left[z_i =\{ i, {\bf u}_{h_i}, {\bf v}_{j_i}\}\right]
      \mathrm{Log} \left[ \hat{w}_{h_i j_i}q_{j_i} \right]\,.
\end{multline*}
Let write the following, normed by $n_{\cdot 2}=n_{s 2}+n_{p 2},$
llik function corresponding to the data $\mathbf{d2}_{n_{\cdot 2}}$  containing both singles and pairs:
\begin{multline}
\label{FR-522}
\mathrm{llik}\left[ \mathbf{q}_1^k \mid  \mathbf{d2}_{n_{\cdot 2}}\right]
    = \fr{1}{n_{\cdot 2}} \sum\limits_{i=1}^{n_{\cdot 2}} \mathrm{ll}_i
    \left[\mathbf{q}_1^k \mid \mathbf{d2}_{n_{\cdot 2}}\right]
    \\
 {}=   \fr{n_{s2}}{n_{\cdot 2}}\sum\limits_{h=1}^m \fr{t_{sh}}{n_{s2}} \,\mathrm{
    Log}  \left [ \sum\limits_{j\in\mathcal{C}_h} \hat{w}_{hj} q_j \right ]\\
{} +
  \fr{n_{p2}}{n_{\cdot 2}} \sum\limits_{j=1}^k \fr{c_{pj}}{n_{p2}}\,\mathrm{Log}
  \left[q_j\right]  \\
{}+ \fr{1}{n_{\cdot 2}}
    \sum\limits_{i=1}^{n_2} \mathrm{Log} 
    \left[ \hat{w}_{h_i j_i}\right] I\left[ \mathbf{z}_i =\left\{ i, \mathbf{u}_{h_i},
    \mathbf{v}_{j_i}\right\} \right]\,.
\end{multline}
This llik
 function of $\mathbf{q}_1^k=\{q_1, \ldots, q_k\}$  corresponds to the all data, 
 with pairs and singles, collected on the second step.

The following properties of this llik are collected in 
the following theorem and corollary:

\pagebreak

\noindent
\textbf{Theorem~5.1.}\
\textit{For every sufficiently large $n_{\cdot 2}$,
 the llik  is concave on $\mathcal{S}_{k-1}.$}
 
 \smallskip
 
 \noindent
 \textbf{Corollary~5.1.}\
 The llik  attains its maximum at  a~point $\check{\mathbf{q}}_{M} =
 \{ \check q_{1M}, \ldots, \check q_{kM}\}\in \mathcal{S}_{k-1}$ and 
 $\check{\mathbf{q}}_{M}$  is its unique stationary point in~$\mathcal{S}_{k-1}.$

\smallskip

It is interesting to find  the  stationary point~$\check{\mathbf{q}}_{M}$
    corresponding to  the maximum value of the llik
      in the multidimensional set
     $\breve{\mathcal{S}}_{k-1}.$
Let suppose that $n_{\cdot 2}$ so large that all $c_{pj}>0$, $j=1, \ldots, k.$
       Then, Lagrange  method with one  multiplier~$\lambda$ can be applied.
 Let consider the following {Lagrange function}
 
 \vspace*{4pt}
 
 \noindent
 $$
    \varphi_L \left[\mathbf{q}_1^k, \lambda \right] =\mathrm{llik} \left[\mathbf{q}_k
   \mid    \mathbf{d2}_{n_{\cdot 2}}\right] +\lambda (q_1+ \cdots + q_1^k)\,.
$$

\vspace*{-2pt}

Components of a~stationary point $q_1, \ldots, q_k$ are the solutions of
the following equations:

\vspace*{-3pt}

\noindent
\begin{multline}
\label{FR-525}
 q_j = \fr{n_{p 2}}{n_{\cdot 2}}\,\fr{t_{pj}}{n_{p 2}}
\\
{} +  \fr{n_{s 2}}{n_{\cdot 2}}  \sum\limits_{h\in\mathcal{D}_j}
   \fr{t_{s h}}{n_{s 2}} \,\fr{\hat w_{hj} q_j}
 {\sum\nolimits_{j'\in\mathcal{C}_h}\hat w_{hj'}q_{j'}}\,,\  j=1, \ldots, k\,.
\end{multline}
Then, all first-order partial derivatives 
${\partial \mathrm{llik} [\mathbf{q}_k]}/{\partial q_j}$\linebreak $=0$, 
$j=1, \ldots, k,$ and $\mathbf{q}_k=\{ q_1, \ldots, q_k\}$ 
together represent a~stationary point. From Corollary~5.1,
one has $\mathbf{q}_1^k=\check{\mathbf{q}}_{M}.$

The proofs of Theorem~5.1 and Corollary~5.1 
are based on ideas similar to that used in Subsection~4.4.

\vspace*{-3pt}

\subsection{Recursive method for numerical analysis}

\noindent
From~(\ref{FR-525}), one may find the following recursion:
\begin{multline}
\label{FR-531}
   q_j^{(r+1)} (n_{\cdot 2}) =\fr{n_{p 2}}{n_{\cdot 2}}\,q_j^{(1)}\\
{}+  \fr{n_{s 2}}{n_{\cdot 2}}  \sum\limits_{h\in\mathcal{D}_j}
  \fr{t_{s h}}{n_{s 2}} \, \fr{\hat{w}_{hj} q_j^{(r)}(n_{\cdot 2})}
 {\sum\nolimits_{j'\in\mathcal{C}_h}\hat{w}_{hj'}q_{j'}^{(r)}(n_{\cdot 2})}\,,\\
r=  1, 2, \ldots ;
\end{multline}
 \begin{equation}\label{FR-56a}
  q_j^{(1)}(n_{\cdot 2}) =\check q_{pj}\,;
  \end{equation}
$$
\hat {w}_{sh}(n_{\cdot 2})=\fr{1}{n_{s2}}
   \sum\limits_{i=1}^{n_{\cdot 2}} I \left[z_i =\left\{  i,
   \mathbf{u}_{h}, \mathrm{NA}\right\}\right]\,.
$$
   As in section~4.5, one has $\sum_{j=1}^k q_j^{(r)} (n_{\cdot 2})=1$ which
   implies the following Corollary.
   
   \smallskip
   
   \noindent
   \textbf{Corollary~5.2.}\
   For any sufficiently large $n_{\cdot 2}$, this
recursion  generates the sequence of points $\mathbf{q}_1^{(r)k}$\linebreak $=\{
q_1^{(r)}, \ldots, q_k^{(r)} \} \in \mathcal{S}_{k-1}$, $r= 1, 2, \ldots, $ which
converge to the unique  stationary point $\check{\mathbf{q}}_{M}= \{
\check{q}_{1M}, \ldots, \check{q}_{kM}\}$.


\smallskip


\noindent
\textbf{Note.} For any $r\ge 1$, $ q_j^{(r)} \ge ({n_{p2}}/{n_{\cdot 2}})q_j^{(1)}$
if $n_{\cdot 2}$ is sufficiently large.

If $n_{\cdot 2}\to\infty$, then all below following estimates a.s.\
 converge to their limits:
 
 \noindent
\begin{gather*}
 \fr{n_{p2}}{n_{\cdot 2}}\to \alpha_{p2}>0\,; \enskip
  \fr{n_{s2}}{n_{\cdot 2}}\to \alpha_{s2}>0\,; \\
 \alpha_{p2}+\alpha_{s2}=1\,;\enskip
\fr{t_{sh}}{n_{s 2}}\to w_{h}\,; \\
\fr{t_{pj}}{n_{p 2}}\to q_{j}; \
 \fr{t_{phj}}{n_{p 2}}\to w_{hj} q_{j},\enskip
 j=1, \ldots,\ k, h \in\mathcal{D}_j\,.
\end{gather*}
 Then, the llik
 on the compact set $\mathcal{S}_{k-1}$ uniformly converges, as
  $n_{\cdot 2}\to\infty ,$ to the nonrandom
 function of~$\mathbf{q}_1^k$:
 
 \noindent
\begin{multline*}
\mathrm{llik}^{(\infty )} \left[\mathbf{q}_1^k\right] = 
\alpha_{p2} \sum\limits_{j=1}^k q_{j} \mathrm{Log} \left[q_j\right]\\
{} + \alpha_{s2} \sum\limits_{h=1}^m w_{h} \mathrm{Log}\left [
\sum\limits_{j\in\mathcal{C}_h} w_{hj}q_j \right ]+ C_p
\end{multline*}
where the constant  $C_p =\alpha_{p2} \sum_{j=1}^k
\sum_{j\in\mathcal{D}_j}q_{j} \mathrm{Log}\,[w_{hj}].$
 Let call the concave function $\mathrm{llik}^{(\infty )}[\mathrm{q}_1^k]$  
 a~\textit{limit llik.}
 Let summarize these results in the following theorem.

\smallskip

\noindent
\textbf{Theorem~5.2.}
\textit{Suppose that   Assumptions~$1$--$3$ and~$6$
 are valid and  the sizes~$n_{p2}$ and~$n_{s2}$ of collected pairs and singles are
growing unboundedly as $n_{\cdot 2}\to\infty$.
  Then, for any sufficiently large~$n_{p2}$ and~$n_{s2}$, the 
  ML-estimator $\check{\bf q}_1^{k} = \{ \check q_{1},
\ldots, \check q_{k}\}$ of $ \check{q}_M$ (based on the data $\mathbf{d2}_{n_{\cdot 2}}$
with singles and pairs) of the llik 
$\mathrm{llik}\,[\mathbf{q}_1^k \mid \mathbf{2}_{n_{\cdot 2}}]$
exists, is strongly consistent, and can be found as the stationary
point of recursion}~(\ref{FR-531}).


\smallskip

      Note  that this inference can  only be applied to the respondents
  in $\mathfrak{P}_c$ who will choose $\mathbf{u}_h \in \mathcal{U}_m$.
  
  \vspace*{-6pt}


\section{Numerical Experiments}

\noindent
To get some insights into the  properties of these estimators, 
some numerical experiments have been carried out.
 In these experiments, simulated data have been used with
  the same set~$\mathcal{U}_m$, $m=46,$ of SeSeI intervals and the set $\mathcal{V}_k$, 
  $k=23$, of division intervals. The authors decided to take the
WTP-distribution to be a~$p_{\mathrm{WE}}$-mixture WE of the Weibull
distribution $W(a, b)$ and the Exponential distribution $ E(m_1)$
with parameters $ p_{\mathrm{WE}}=0.8160$, $a=74.8992$, $b=1.8374$,
and $m1=254.7344$:
\begin{equation}
\label{FR-61}
\mathrm{sf}_{\mathrm{WTP}} [x]= p_{\mathrm{WE}} e^{-(x/a)^b} + (1-p_{\mathrm{WE}})e^{-x/m_1} \,.
\end{equation}
 The motivation for this choice is further explained in~\cite{BK:BK12} 
 where  five  ``behavioral models'' have been also introduced. 
 Each of these provides one possible model of how individuals choose the intervals,
 a~process which is essentially unknown. The authors have decided to use a~behavior model described by the array
$\mathbb W_{ m k} =(w_{hj}),$ for any $j=1, \ldots, k,$ and $h\in
\mathcal{D}_j, w_{hj} =t_h/\left ( \sum_{h'\in \mathcal{D}_j} t_h \right )$
 where~$t_h$ are the numbers of repetitions $\mathbf{u}_h\in\mathcal{U}_m$ 
 in the first step of data collection, $k=1, \ldots, m.$
$t_h$ were taken as they were in the real data.

 Let $\omega_1, \omega_2, \ldots, \omega_{4i-3}, \omega_{4i-2}, 
 \omega_{4i-1}, \omega_{4i}, \ldots$
 be a~sequence of real values generated by a~random number generator. Let use these values as follows.
If  $\omega_{4i-3} \le p_{\mathrm{WE}}$, then~$\omega_{4i-2}$ is transformed to~$x_i$ 
which is a~value of an r.v.~$X_i$ with the Weibull s.d.f.~$e^{-(x/a)b}$. 
Otherwise, if $\omega_{4i-3}> p_{\mathrm{WE}}$, then~$\omega_{4i-3}$
is transformed to r.v.~$X_i$ with the exponential s.d.f.~$e^{-x/m_1}$. 
Further, it will be found which of divisions intervals contains~$x_i.$
 Let $\mathbf{v}_j \ni x_i$, then $\mathcal{D}_j =\{ h_1(j), h_2 (j), \ldots, h_{d_j}(j)\}$ 
 where index $h_d(j)$
 corresponds to probability $w_{hj}$, and~$d_j$ is the number of all indexes 
 in~$\mathcal{D}_j.$
 Remind that $\sum_{h\in\mathcal{D}_j} w_{hj}  =1$. 
 Interval $(0, 1]$ is the union of disjoint intervals
$\left( 0, w_{h_1(j)}, j \right], \left (w_{h_1(j),j}, 
\sum\nolimits_{c=1}^2  w_{h_c (j), j} \right ] , \ldots$\linebreak $\ldots,  \left ( 
\sum\nolimits_{c=1}^{d_j-1}  w_{h_c (j), j} , 1 \right ]$.
 The event $\left\{ \vphantom{\sum_{c=1}^{d_j-1}}\omega_{4i-1}\right.$\linebreak 
 $\left. \in \left ( \sum_{c=1}^{d_j-1}  w_{h_c (j), j} ,
\sum_{c=1}^{d_j}  w_{h_c (j), j} \right]
\right\}$
  means that $\mathbf{u}_{h_c (j)} \supseteq \mathbf{v}_j$ has been stated as SeSeI interval, 
  $h_c(j)\in\mathcal{D}_j.$
  At last, the event $\omega_{4i} \le 2/3$ implies that the $i$th element in the data ${\bf 2}_{t n_{\cdot 2}}$
  is the triplet $\{ i, {\bf u}_h, \mathrm{NA}\}$, i.\,e.,
  the single triplet. If $\omega_{4i} > 2/3$,  the $i$th
  data element  is the triplet $\{ i, \mathbf{u}_h, \mathbf{v}_j\} $ with the pair 
  $\{ \mathbf{u}_h, \mathbf{v}_j\}$, $h=h_c(j)$.

 To illustrate consistency of the suggested estimates, the size of simulated data 
 $\mathbf{d2}_{t n_{\cdot 2}}$
 was selected rather large with $n_{\cdot 2} = 9000$ triplets. There were $n_{s 2} =5856$ singles 
 and $n_{p 2}=3144$  pairs. Let add index~$t$ to underline that data are considered as 
 ``true'' data. By using triplets with pairs
 $\{ i, \mathbf{u}_{h_i}, \mathbf{v}_{j_i}\}$ in $\mathbf{d2}_{t n_{\cdot 2}}$,  
let find estimates $\check q_{tj}>0$ for  the true probabilities 
 $\check q_{oj}=P_o [X_i \in \mathbf{v}_j]$, $j=1, \ldots, 23$. 
 To improve accuracy, one can find,
 by using all data set $\mathbf{d2}_{t n_{\cdot 2}}$ and by applying~$r$ 
 ($r\le 10$) iterations, ML-estimates
 $\check{\mathbf{q}}_t^{(10)} =\{ \check q_{t1}^{(10)}, \ldots, 
 \check q_{tk}^{(10)}\}$, $k=23$, corresponding the
 llik~(\ref{FR-522}). Let use these estimates to illustrate consistency estimates of true points
 $\{ v_{R j}, s_{oj}^k \}$ on s.d.f.~(\ref{FR-61}). 
 Here, true and  estimated points are 
$\left\{ v_{Rj}; s_o^k \right\}$,
$\left\{ v_{R j}, \hat s_{t j}^k \right\}$
$s_o^k =\sum\nolimits_{j'=j}^k q_{oj}$; 
and $\check s_{tj}^k =\sum\nolimits_{j'=j}^k \check q_{tj}^{(10)}$.
Here, the s.d.f.~(\ref{FR-61}) can be used, because  only stated
  $\mathbf{u}_h\in\mathcal{U}_m$ are considered.  It means that
 only intervals in~$\mathcal{U}_m$ will be stated by any respondent in the investigated 
 population~$\mathfrak {P}$. In Fig.~6, it is seen 
 that ML-estimates based on both pairs and singles are rather accurate.
The usage of estimates based only on pairs is less accurate. 
To illustrate utility of usage of both pairs
and singles,  $C=2000$ i.i.d.\ copies of data 
$\mathbf{d2}_{t n_{\cdot 2}}^c$ have been simulated, each such copy had size
$n_{\cdot 2}=9000.$ The same values of parameters $p_w$, $a$, $b$, $m_1$, 
and~$p_{\mathrm{NA}}$ were used.
{\looseness=1

}

The mean value is of central interest in  contingent valuation surveys.
If a~particular parametric distribution is used, one can, of course, derive the estimate of
 the\linebreak\vspace*{-12pt}

    \begin{center}  %fig5
\vspace*{1pt}
\mbox{%
 \epsfxsize=77.817mm
 \epsfbox{bel-6.eps}
 }
 
\end{center}

%\vspace*{-3pt}

\noindent
{{\figurename~6}\ \ \small{The true WTP-survival function is
shown over interval $[ 0, 200]$  together with its true points
  $\{ v_{Rj}^k, q_{j}^k \}$ and the approximated ML-estimated
  points  $\{ v_{Rj}^k,\check q_j^{(10)k}  \}$, 
  $1\le j\le 16$, $n_{\cdot 2}= 9000$}}

\vspace*{14pt}

\addtocounter{figure}{1}
 

\noindent
 mean by estimating parameters.
In a~nonparametric setting, let propose

\noindent
$$
\mathrm{mm}_{o1} =v_{L1} +\sum\limits_{j=1}^k (q_{o j+1}^k +\fr{1}{2}\,q_{0j}) (v_{Rj}-v_{Lj})
$$
 and call it as \textit{medium mean value}.
  The medium mean value~mm$_{o1}$ can be considered as acceptable approximation to the mean value
  $m_{o1} =E [X_i]$ of the WTP-e.d.f.

One has the following strongly consistent and asymptotically unbiased ML-estimate of~mm$_{o1}$:

\vspace*{-6pt}

\noindent
  \begin{multline}
  \label{FR-57a}
\check{\mathrm{mm}}_{1 ps}\\ = v_{L1} 
+ \sum\limits_{j=1}^k \left(\check s_{j+1, ps}^k
 +\fr{1}{2}\,\check q_{j ps} \right ) \left(v_{Rj}-v_{Lj}\right)
\end{multline}
where $\check s_{j+1, ps}^k=\sum_{i=j+1}^k \check q_{i ps},$ $ j=1, \ldots, k$, and
$\check q_{k+1, ps}=0$.
 Note that~$\check{\mathrm{mm}}_{1 ps} $ is the area below the
 broken line, connected by intervals with end points $\{ v_{R j}, s_{j+1, ps}^k \}$, 
 $j=1, \ldots, k-1$,
 and $\{ v_{Ro, 1} \} ,$ $\{ v_{Rk}, 0\}$ on the estimated  WTP-survival function.

\begin{figure*} %fig7
       \vspace*{1pt}
 \begin{center}
 \mbox{%
 \epsfxsize=164.073mm
 \epsfbox{bel-7.eps}
 }
 \end{center}
 \vspace*{-9pt}
 \Caption{Two pairs of histograms,
corresponding to $C= 2000$ deviations dev$_{1ps}^c \hm=
\check{\mathrm{mm}}_{1ps}^c -\mathrm{mm}_{o1}$~(\textit{1}) and
 dev$_{1p}^c$\protect\linebreak $ =
\check{\mathrm{mm}}_{1p}^c -\mathrm{mm}_{o1}$~(\textit{2})~(\textit{a})
 and dev$_{1ps}^{\star c} = \check{\mathrm{mm}}_{1ps}^{\star c} - 
 \check{\mathrm{mm}}_{1ps}^c$~(\textit{1}) and
  dev$_{1p}^{\star c}  = \check{\mathrm{mm}}_{1p}^{\star c}  - \check{\mathrm{mm}}_{1p}^c$~(\textit{2})~(\textit{b}):
\textit{1}~--- triples with both pairs and singles; and 
\textit{2}~--- triples with only pairs, $n_{\cdot 2}= 9000$, $C=C^\star    =2000$} 
   \label{FG-N07}
\end{figure*}

For each copy of data $\mathbf{d2}_{t n_{\cdot 2}}^c$, the authors apply~$r$
iterations of recursion~(\ref{FR-531}) and obtain sufficiently
accurate approximation $\check{\mathbf{q}}_k^{c(r)}$ of the ML-estimate
$\check{\mathbf{q}}_{kM}^{ c}.$ Here, all copies of data have been used, i.\,e.,
both pairs and singles. One can also find the ML-estimates
$\check{\mathbf{q}}_{pk}^c$ of $\mathbf{q}_{k}$ by using~(\ref{FR-56a})
and the part of data $\mathbf{d2}_{n_{\cdot 2}}^c$ with triples
containing pairs~(\ref{FR-531}).
 By substituting $\check{\mathbf{q}}_{pk}^c$
and $\check{\mathbf{q}}_k^{c(r)}$ in~(\ref{FR-57a}), one obtains estimates
of WTP-medium mean values $\check{\mathrm{mm}}_{1p}^c$ and
$\check{\mathrm{mm}}_{1ps}^{c(r)}$, $c=1, \ldots, R.$ To underline the gain of usage
of the whole data $\mathbf{d2}_{n_{\cdot 2}}^c$ compared with only the
part of all pairs,  two lists with  deviations have been calculated:
\begin{equation}
\left.
\begin{array}{c}
    \check{\mathrm{dev}}_{1ps}^{c(r)}= \check{\mathrm{mm}}_{1ps}^{c(r)} - \mathrm{mm}_{o 1}\,;\\[6pt]
     \check{\mathrm{dev}}_{1p}^c= \check{\mathrm{mm}}_{1p}^c - \mathrm{mm}_{o1},\;
    c=1, \ldots, C\,.
    \end{array}
    \right\}
    \label{FR-44}
\end{equation}
The deviations~(\ref{FR-44})   can be considered
as  values  of i.i.d.\ r.v.s. Then, their standard deviation
estimates are:
$$
\check{\mathrm{std}}_{1 ps}^{(r)} =1.780; \quad
\check{\mathrm{std}}_{1p} =2.507\,. 
$$
 Hence, the estimates of~mm$_{o1}$ based on both pairs and singles are more exact
 than based only on pairs.

 It is useful to know the following empirical distribution
 functions  (e.d.f.s)\ of deviations:
\begin{equation}
\left.
\begin{array}{rl}
 \check{\mathrm{edf}}_{1sp}^{(r)}[x] &=\fr{1}{C} \sum\limits_{c=1}^C
I\left[\check{\mathrm{dev}}_{1 sp}^{c(r)}\le x\right]\,;\\[6pt]
  \check{\mathrm{edf}}_{1p}[x] &=\fr{1}{C} \sum\limits_{c=1}^C
I\left[\check{\mathrm{dev}}_{1p}^c\le x\right]\,.
\end{array}
\right\}
\label{FR-46}
\end{equation}
The lists with deviations~(\ref{FR-46}) can be presented as a~paired histograms,
composed from rectangles with low edges intervals $I_{r'} =(2r', 2(r'+1)]$, 
$r'=0, 1,\ldots$ on the $x$-axis. The length of $I_{r'}$ is 
2~SEK. The rectangles side intervals are proportional to the
numbers of $\check{\mathrm{mm}}_{1ps}^{c(r)}\in I_{r'}$ and $\check{\mathrm{mm}}_{1p}^c\in
I_{r'}$, $c=1, \ldots, C.$ Two corresponding components of paired histograms are
shown with the
dashed and solid contour lines, respectively, in Fig.~\ref{FG-N07}\textit{a}.

Note that total size of 2000 samples with size $n_{\dot 2}=9000$ is rather large. Nearly the same
accuracy of pairs histogram one can obtain by using 2000 independently resampled copies from only one
sample~(!) with the same size~9000. Such resampled copies of $\mathbf{d2}_{n_{\cdot 2}}^c$ is denoted
$\mathbf{d2}_{n_{\cdot 2}}^{\star c}$, $c=1, \ldots, 2000,$ and use them farther similarly as if they are common samples
from  a~population~$\mathfrak{P}$. Let apply recursion~(\ref{FR-531}) with~$r$ sufficient  iterations and 
find~$\mathbf{q}_k^{\star c(r)}.$ By substituting~$\check{\mathbf{q}}_{kps}^{\star c}$ 
and~$\check{\bf q}_{kps}^{\star c(r)}$
in relation similar to~(\ref{FR-57a}), one obtains~$\check{\mathrm{mm}}_{1ps}^{\star c(r)}$ 
and~$\check{\mathrm{mm}}_{1p}^c.$
Their deviations are similar to that given in~(\ref{FR-44}):

\noindent
\begin{align*}
\mathrm{dev}_{1 ps}^{\star c(r)}&= \check{\mathrm{mm}}_{1ps}^{\star c(r)}- \check{\mathrm{mm}}_{1ps}^c\,; 
\\
\mathrm{dev}_{1 p}^{\star c}&= \check{\mathrm{mm}}_{1p}^{\star c}- \check{\mathrm{mm}}_{1p}^c\,.
\end{align*}
The pairs histogram obtained by resampled copies of data
  $\mathbf{d2}_{n_{\cdot 2}}$ are shown in Fig.~\ref{FG-N07}\textit{b}. 
  Both paired histograms are similar which reflects consistency of the resamplings  method.
Sufficient and necessary assumptions for consistency the  resampling method were 
given in~\cite{BK:BN97, BK:BE03}.



\vspace*{-6pt}

\section{Concluding Remarks}

\noindent
The present authors have provided an approach to the theory of SeSeIs obtained
in surveys. These SeSeIs should have wide application, not the least in survey research, where
they are already being used.
The drawbacks of alternative elicitation mechanisms are well-documented in the literature and
the SeSeIs can overcome some of them.   Or so, the authors have argued. The new 2-step approach suggested
has not been tested empirically. The present simulation exercise provides some indications that merit
trying this approach in the field. A~key assumption in the 2-step approach (an assumption
currently always made when using any other elicitation mechanisms) is that the suggested
interval/information in the second step does not have any impact on the censored variable.
In other words, the authors assume that anchoring is not a~problem in this context. While anchoring
is less of a~problem in the context compared to, say,  standard bracketing approaches
(because one has information from the ``undistorted by anchoring'' intervals), it is
still an assumption that needs to be scrutinized. In future research,
this assumption will be studied in lab and field experiments.



 A recent bibliography on surveys targeting
contingent valuation has about
7.500 studies from 130~countries~\cite{BK:CA12}. Big data sets are also collected in
surveys health status of individuals, during aging and retirement.
 censoring intervals are common in diversity of biomedical data sets~\cite{BK:KM03}.
 Hopefully, interval data can also be useful in
expert assessments of economic data, in durations of individuals' unemployment, in
usage of futures contracts at bourses, and in psychometrics evaluation of individuals'
relation to different risks.

\renewcommand{\bibname}{\protect\rmfamily References}

\vspace*{-6pt}


{\small\frenchspacing
{%\baselineskip=10.8pt
\begin{thebibliography}{99}

\bibitem{BK:MA99}
\Aue{Manski, C.\,F.} 1999.
\textit{Identification problems in the social sciences.} Harvard University
Press. 194~p.
\bibitem{BK:MH90}
\Aue{Morgan, M.\,G., and M.~Henrion}. 1990. 
\textit{Uncertainty: A~guide to dealing with uncertainty in
quantitative risk and policy analysis}. Cambridge University Press. 
 325~p.
\bibitem{BK:BD06}
\Aue{Billard, L., and E. Diday}. 2006. 
\textit{Symbolic data analysis: Conceptual statistics
and data mining}. Wiley ser. in computational statistics. Wiley. Vol.~636.
330~p.
\bibitem{BK:MM10}
\Aue{Manski, C.\,F., and F. Molinari}.  2010.
Rounding probabilistic expectations in surveys. 
\textit{J.~Bus. Econ. Stat.} 28:219--231.
\bibitem{BK:JK12}
\Aue{Johansson, P.-O., and B. Kristr$\ddot{\mbox{o}}$m}. 2012. 
\textit{The economics of evaluating water projects.
Hydroelectricity versus other uses.} Heidelberg: Springer. 135~p.
\bibitem{BK:BK10} %6
\Aue{Belyaev, Y., and B. Kristr$\ddot{\mbox{o}}$m}. 2010. 
Approach to analysis of self-selected interval data.
\mbox{Ume\!{\fontsize{10pt}{10pt}\selectfont\ptb{\!{\r{\!\!a}}}}}: CERE.
Working Paper 2010:2. 1--34. 
Available at: {\sf  http:// www.cere.se/se/forskning/publikationer/155-approach-to-analysis-of-self-selected-interval-data.html}
(accessed September~8, 2015).
\bibitem{BK:BK12}
\Aue{Belyaev, Y.,  and B. Kristr$\ddot{\mbox{o}}$m}. 2012. 
Two-step approach to self-selected interval
data in elicitation surveys. 
\mbox{Ume\!{\fontsize{10pt}{10pt}\selectfont\ptb{\!{\r{\!\!a}}}}}:
CERE. Working Paper 2012:10. 1--46.
Available at: {\sf  http://www.cere.se/se/forskning/publikationer/386-two-step-approach-to-self-selected-interval-data-in-elicitation-surveys.html}
(accessed September~8, 2015).
\bibitem{BK:TU74}
\Aue{Turnbull, B.\,W.}  1974. 
Nonparametric estimation of a~survivorship function with doubly censored data.
\textit{J.~Am. Stat. Assoc.} 69:169--173.
\bibitem{BK:TU76}
\Aue{Turnbull, B.\,W.} 1976. 
The empirical distribution function with arbitrarily grouped,
censored and truncated data. \textit{J.~Roy. Stat. Soc. B} 38:290--295.
\bibitem{BK:MA03}
\Aue{Manski, C.\,F.} 2003.
\textit{Partial identification of probability distributions.} Springer ser.
in statistics. Springer. 196~p.
\bibitem{BK:CA12}
\Aue{Carson, R.} 2012. \textit{Contingent valuation: A~comprehensive bibliography and history}.
Edward Elgar Publishing. 464~p.
\bibitem{BK:HA08}
\Aue{H\!\!{\fontsize{10pt}{10pt}\selectfont\ptb{\!{\r{\!\!a}}}}kansson, C}. 2008. 
A~new valuation question~--- analysis of and insights from interval
 open ended data in contingent valuation. 
 \textit{Environ. Resour. Econ.} 39(2):175--188.
\bibitem{BK:BB08}
\Aue{Broberg, T., and R. Br$\ddot{\mbox{a}}$nnlund}.  2008.
An alternative interpretation of multiple
bounded WTP data-certainty dependent payment card intervals.
\textit{Energy Resour. Econ.} 30:555--567.
\bibitem{BK:MRKB14}
\Aue{Mahieu, P.,  P.~Riera, B.~Kristr$\ddot{\mbox{o}}$m, R.~Br$\ddot{\mbox{a}}$nnlund, 
and  M.~Giergiczny}. 2014.
Exploring the determinants of uncertainty in contingent valuation surveys. 
\textit{J.~Environ. Econ. Policy} 3(2):186--200.  
Available at: {\sf http://dx.doi.org/10.1080/21606544.2013.876941}
(accessed  September~3, 2015).
\bibitem{BK:GO54} %15
\Aue{Good, I.\,J.} 1953.
The population frequencies of species and the estimation of population
parameters. \textit{Biometrika} 40(3-4):237--264.
\bibitem{BK:BK13}
\Aue{Belyaev, Y., and B.~Kristr$\ddot{\mbox{o}}$m}. 2013.  
Analysis of contingent valuation data
with self-selected rounded WTP-intervals collected by two-steps sampling plans.
\textit{9th Tartu Conference on Multivariate Statistics and 20th IWMS Proceedings}.
Tartu: World Scientific. 48--60.
\bibitem{BK:GCO04}
\Aue{Gomez, J., M. Calle, and R.~Oller}. 2004. Frequentist and Bayesian approaches for
interval-censored data. \textit{Stat. Pap.} 45:139--173.
 \bibitem{BK:GCOL09}
\Aue{Gomez, J., M. Calle, R.~Oller, and K.~Langhor}. 2009. 
Tutorial on methods for interval-censored
data and their implementations in $R$. \textit{Stat. Model.} 9(4):259--297.

\bibitem{BK:GG94} %19
\Aue{Gentleman, R., and C.\,J.~Geyer}. 1994. 
Maximum likelihood for interval censored data:
Consistency and computation. \textit{Biometrika}  81(3):618--623.

\bibitem{BK:BT05} %20
\Aue{Brinkhuis, J., and V. Tihomirov}. 2005.  
\textit{Optimization: Insights and applications.}
Princeton--Oxford: Princeton University Press. 680~p.
\bibitem{BK:JM03} %21
\Aue{Jammalamadaka, S.\,R., and V.~Mangalam}. 2003. 
Non-parametric estimation for middle-censored data.
\textit{J.~Nonparametr. Stat.} 15:253--265.
\bibitem{BK:BN97} %22
\Aue{Belyaev, Y.\,K., and L.~Nilsson}. 1997. Parametric maximum likelihood estimators.
Department of Mathematical Statistics,
\mbox{Ume\!{\fontsize{10pt}{10pt}\selectfont\ptb{\!{\r{\!\!a}}}}} University. 
Research Report 1997-15. 1--28.

\bibitem{BK:BE03} %23
\Aue{Belyaev, Y.\,K.} 2003. 
Necessary and sufficient conditions for consistency of resampling. Sweden:
Centre of Biostochastics, Swedish University of Agricultural
Sciences. Research Report 2003-1. 1--26.
Available at: 
{\sf http://biostochastics.slu.se/publikationer/ dokument/Report2003\_1.pdf}
(accessed September~3, 2015).
\bibitem{BK:KM03} %24
\Aue{ Klein, J.\,P., and M.\,L.~Moeschberger}.  
2003. \textit{Survival analysis: Techniques for censored
 and truncated data}.  New York, N.Y.: Springer-Verlag. 536~p.

\end{thebibliography} } }

\end{multicols}

\vspace*{-10pt}

\hfill{\small\textit{Received June 30, 2015}}

\vspace*{-18pt}

\Contr

\vspace*{-3pt}

\noindent
\textbf{Belyaev Yuri K.} (b.\ 1932)~--- Doctor of Science in physics and mathematics,
professor, 
\mbox{Ume{\!\!\fontsize{12pt}{12pt}\selectfont\ptb{\!{\r{\hspace*{-3pt}a}}}}} 
University,  Petrus Laestadius v$\ddot{\mbox{a}}$g,
\mbox{Ume{\!\!\fontsize{12pt}{12pt}\selectfont\ptb{\!{\r{\hspace*{-3pt}a}}}}} SE-901 87, 
Sweden;  yuri.belyaev@umu.se

%\vspace*{3pt}

\noindent
\textbf{Kristr$\ddot{\mbox{o}}$m Bengt} (b.\ 1960)~--- 
PhD in economics, Director, Department of Forest Economics, Center for Environmental and 
Resource Economics (CERE), Swedish University of Agricultural Sciences,
\mbox{Ume\!\!{\fontsize{12pt}{12pt}\selectfont\ptb{\!{\r{\hspace*{-3pt}a}}}}} SE-901 83, 
Sweden;  bengt.kristrom@umu.se


%\vspace*{8pt}

%\hrule

%\vspace*{2pt}

%\hrule

\newpage

\vspace*{-24pt}




\def\tit{АНАЛИЗ ОБЗОРНЫХ ОБСЛЕДОВАНИЙ, СОДЕРЖАЩИХ ЦЕНЗУРИРОВАННЫЕ ДАННЫЕ В~ОКРУГЛЕННЫХ ИНТЕРВАЛАХ}

\def\aut{Ю.\,К.~Беляев$^1$, Б.~Кристрём$^2$}


\def\titkol{Анализ обзорных обследований, содержащих цензурированные данные 
в округленных интервалах}

\def\autkol{Ю.\,К.~Беляев, Б.~Кристрём}

%{\renewcommand{\thefootnote}{\fnsymbol{footnote}}
%\footnotetext[1]{Работа проводится при финансовой поддержке Программы
%стратегического развития Петрозаводского государственного университета в рамках
%на\-уч\-но-ис\-сле\-до\-ва\-тель\-ской деятельности.}}


\titel{\tit}{\aut}{\autkol}{\titkol}

\vspace*{-12pt}

\noindent
$^1$Университет Умеа, Умеа, Швеция

\noindent
$^2$Центр экономики природных ресурсов и окружающей среды Шведского университета
сельскохозяй-\linebreak
$\hphantom{^1}$ственных наук, Умеа, Швеция

\vspace*{6pt}

\def\leftfootline{\small{\textbf{\thepage}
\hfill ИНФОРМАТИКА И ЕЁ ПРИМЕНЕНИЯ\ \ \ том\ 9\ \ \ выпуск\ 3\ \ \ 2015}
}%
 \def\rightfootline{\small{ИНФОРМАТИКА И ЕЁ ПРИМЕНЕНИЯ\ \ \ том\ 9\ \ \ выпуск\ 3\ \ \ 2015
\hfill \textbf{\thepage}}}




\Abst{Предложены методы статистического анализа данных, собираемых 
в~обзорных обследованиях. Предложенный подход общий и~применим в~большинстве случаев, 
когда в~собранных данных содержатся интервалы. Такие данные типичны во многих 
исследованиях: в~анализе на\-деж\-ности изделий и~систем, продолжительности жизни 
в~демографии, в~медицине и~экономике, в~обзорных обследованиях мнения населения и~др. 
Имеются серьезные причины для интенсивного использования данных с~интервалами. 
Наиболее общей причиной является невозможность наблюдения точных значений. 
Природа исследуемых интервалов необычна. Так называемые самовыбираемые 
интервалы без ка\-ких-ли\-бо ограничений свободно выбираются субьектами обследований. 
Концы таких интервалов могут быть округлены. Предлагается обобщение продуктивного 
подхода к~статистическому анализу в~общей схеме цензурирования, предложенной 
Б.\,В.~Турнбуллом. Объяснено и~обобщено основное условие независимости 
в~анализе Турнбулла. Предложено правило остановки выборочного процесса на 
основе достигнутого значения вероятности покрытия. Введение дополнительного 
(второго) вопроса всем выбранным респондентам дает возможность получения более 
точной оценки характеристик искомого распределения. 
Дано обоснование методов информатики, применяемых для анализа статистических
данных, содержащих самовыбира\-емые интервалы. Эти методы дают возможность частичной 
идентификации искомых непараметрических распределений. Дано описание статистических 
моделей данных, допускающих зависимость выбора интервалов от положения 
в~них точных значений. Получены рекурсии быстрого вычисления оценок 
максимального правдоподобия для характеристик искомых распределений. 
Приведены результаты применения предлагаемых методов, 
подтверждающие их полезность в~анализе смоделированных данных, содержащих 
самовыбираемые интервалы.}



\KW{выявляющие выборочные обследования; случайный выбор; округление; якорность;  
вероятность покрытия; правдоподобие; рекурсия; максимизация; случайный перевыбор}


\DOI{10.14357/19922264150301}

\vspace*{-6pt}


 \begin{multicols}{2}

\renewcommand{\bibname}{\protect\rmfamily Литература}
%\renewcommand{\bibname}{\large\protect\rm References}

{\small\frenchspacing
{%\baselineskip=10.8pt
\begin{thebibliography}{99}

\bibitem{BK:MA99-1}
\Au{Manski C.\,F.} 
{Identification problems in the social sciences.}~--- Harvard University
Press, 1999. 194~p.
\bibitem{BK:MH90-1}
\Au{Morgan M.\,G., Henrion M.} 
{Uncertainty: A~guide to dealing with uncertainty in
quantitative risk and policy analysis}.~--- Cambridge University Press, 1990. 
 325~p.
\bibitem{BK:BD06-1}
\Au{Billard L.,  Diday E.} 
{Symbolic data analysis: Conceptual statistics
and data mining}. Wiley ser. in computational statistics.~--- Wiley, 2006.  Vol.~636.
330~p.
\bibitem{BK:MM10-1}
\Au{Manski C.\,F.,  Molinari F.}  
Rounding probabilistic expectations in surveys~//
{J.~Bus. Econ. Stat.}, 2010. Vol.~28. P.~219--231.
\bibitem{BK:JK12-1}
\Aue{Johansson P.-O.,  Kristr$\ddot{\mbox{o}}$m~B.} 
{The economics of evaluating water projects.
Hydroelectricity versus other uses.}~--- Heidelberg: Springer, 2012.  135~p.
\bibitem{BK:BK10-1}
\Aue{Belyaev Y., Kristr$\ddot{\mbox{o}}$m B.} 
Approach to analysis of self-selected interval data.
\mbox{Ume\!{\fontsize{10pt}{10pt}\selectfont\ptb{\!{\r{\!\!a}}}}}: CERE,  2010.
Working Paper 2010:2. P.~1--34. 
 {\sf  
http://www.cere.se/se/ forskning/publikationer/155-approach-to-analysis-of-self-selected-interval-data.html}.
\bibitem{BK:BK12-1}
\Au{Belyaev Y.,  Kristr$\ddot{\mbox{o}}$m B.} 
Two-step approach to self-selected interval
data in elicitation surveys. 
\mbox{Ume\!{\fontsize{10pt}{10pt}\selectfont\ptb{\!{\r{\!\!a}}}}}:
CERE, 2012. 
Working Paper 2012:10. P.~1--46.
{\sf  http:// www.cere.se/se/forskning/publikationer/386-two-step-approach-to-self-selected-interval-data-in-elicitation-surveys.html}.
\bibitem{BK:TU74-1}
\Au{Turnbull B.\,W.} 
Nonparametric estimation of a~survivorship function with doubly censored data~//
J.~Am. Stat. Assoc., 1974.  Vol.~69. P.~169--173.
\bibitem{BK:TU76-1}
\Au{Turnbull B.\,W.} 
The empirical distribution function with arbitrarily grouped,
censored and truncated data~// J.~Roy. Stat. Soc. B, 1976. Vol.~38. P.~290--295.
\bibitem{BK:MA03-1}
\Au{Manski C.\,F.} 
{Partial identification of probability distributions.} Springer ser.
in statistics.~--- Springer, 2003. 196~p.
\bibitem{BK:CA12-1}
\Au{Carson, R.} 
{Contingent valuation: A~comprehensive bibliography and history}.~---
Edward Elgar Publishing, 2012.  464~p.
\bibitem{BK:HA08-1}
\Au{H\!\!\!\!{\fontsize{10pt}{10pt}\selectfont\ptb{\!{\r{\!a}}}}kansson C}. 
A~new valuation question~--- analysis of and insights from interval
 open ended data in contingent valuation~// 
{Environ. Resour. Econ.}, 2008.  Vol.~39. No.\,2. P.~175--188.
\bibitem{BK:BB08-1}
\Au{Broberg T., Br$\ddot{\mbox{a}}$nnlund R.}
An alternative interpretation of multiple
bounded WTP data-certainty dependent payment card intervals~//
{Energy Resour. Econ.},   2008. Vol.~30. P.~555--567.
\bibitem{BK:MRKB14-1}
\Au{Mahieu P.,  Riera P., Kristr$\ddot{\mbox{o}}$m~B., Br$\ddot{\mbox{a}}$nnlund~R., 
Giergiczny~M.} 
Exploring the determinants of uncertainty in contingent valuation surveys~//
{J.~Environ. Econ. Policy}, 2014.  
{\sf http://dx.doi.org/10.1080/ 21606544.2013.876941}.
\bibitem{BK:GO54-1}
\Au{Good I.\,J.} 
The population frequencies of species and the estimation of population
parameters~// {Biometrika}, 1953. Vol.~40. Iss.\,3-4. P.~237--264.
\bibitem{BK:BK13-1}
\Au{Belyaev Y., Kristr$\ddot{\mbox{o}}$m~B.} 
Analysis of contingent valuation data
with self-selected rounded WTP-intervals collected by two-steps sampling plans~//
9th Tartu Conference on Multivariate Statistics and 20th IWMS Proceedings.~--- 
Tartu: World Scientific, 2013. P.~48--60.
\bibitem{BK:GCO04-1}
\Au{Gomez J.,  Calle M., Oller~R.}  Frequentist and Bayesian approaches for
interval-censored data~// {Stat. Pap.}, 2004. Vol.~45. P.~139--173.
 \bibitem{BK:GCOL09-1}
\Au{Gomez J., Calle M., Oller R., Langhor K.} 
Tutorial on methods for interval-censored
data and their implementations in~$R$~// {Stat. Model.}, 2009. 
Vol.~9. No.\,4. P.~259--297.

\bibitem{BK:GG94-1} %19
\Au{Gentleman R., Geyer C.\,J.} 
Maximum likelihood for interval censored data:
Consistency and computation~// {Biometrika}, 1994.  Vol.~81. No.\,3. P.~618--623.

\bibitem{BK:BT05-1} %20
\Au{Brinkhuis J., Tihomirov V.} 
{Optimization: Insights and applications.}~---
Princeton--Oxford: Princeton University Press,
2005.  680~p.

\bibitem{BK:JM03-1} %21
\Au{Jammalamadaka S.\,R., Mangalam V.} 
Non-parametric estimation for middle-censored data~//
{J.~Nonparametr. Stat.}, 2003.  Vol.~15. P.~253--265.

\bibitem{BK:BN97-1} %22
\Au{Belyaev Y.\,K., Nilsson L.} Parametric maximum likelihood estimators.
Department of Mathematical Statistics,
\mbox{Ume\!{\fontsize{10pt}{10pt}\selectfont\ptb{\!{\r{\!\!a}}}}} University,
1997. Research Report 1997-15. P.~1--28.


\bibitem{BK:BE03-1}
\Au{Belyaev Y.\,K.} %23 
Necessary and sufficient conditions for consistency of resampling.
Sweden: Centre of Biostochastics,  Swedish University of Agricultural
Sciences,  2003. Research Report 2003-1. P.~1--26.
{\sf http://biostochastics. slu.se/publikationer/dokument/Report2003\_1.pdf}.

\bibitem{BK:KM03-1} %24
\Au{Klein, J.\,P., M.\,L.~Moeschberger}.  
2003. {Survival analysis: Techniques for censored
 and truncated data}.~---  New York, N.Y., USA: Springer-Verlag. 536~p.


\end{thebibliography}
} }

\end{multicols}

 \label{end\stat}

 \vspace*{-6pt}

\hfill{\small\textit{Поступила в редакцию  30.06.2015}}
%\renewcommand{\bibname}{\protect\rm Литература}
\renewcommand{\figurename}{\protect\bf Рис.}  %12
% \renewcommand{\figurename}{\protect\bf Figure}
%\renewcommand{\tablename}{\protect\bf Table}
\renewcommand{\bibname}{\protect\rmfamily References}

\def\stat{volodin}

\def\tit{COMPLETE CONVERGENCE FOR~ARRAYS OF~NEGATIVELY DEPENDENT RANDOM VARIABLES}

\def\titkol{Complete convergence for arrays of negatively dependent random variables}

\def\autkol{S.\,H.~Sung, K.~Budsaba, and~A.~Volodin}

\def\aut{S.\,H.~Sung$^1$, K.~Budsaba$^{2}$, and~A.~Volodin$^{3}$}

\titel{\tit}{\aut}{\autkol}{\titkol}

%{\renewcommand{\thefootnote}{\fnsymbol{footnote}}\footnotetext[1]
%{Received by the editors November, 2011. 1991 \textit{Mathematics Subject Classification}.
%Primary 62E20; Secondary 60F05.}}


\renewcommand{\thefootnote}{\arabic{footnote}}
\footnotetext[1]{Department of Applied Mathematics, Pai Chai University, Taejon, South Korea, 
sungsh@pcu.ac.kr}
\footnotetext[2]{Center of Excellence in Mathematics, CHE, Bangkok, Thailand;
Department of Mathematics and Statistics, Thammasat University Rangsit Center, 
Pathumthani, Thailand, kamon@mathstat.sci.tu.ac.th}
\footnotetext[3]{School of Mathematics and Statistics, University of Western Australia, 
Crawley, Australia; University of Regina, Canada,\linebreak 
Andrei.Volodin@uregina.ca}

\Abste{A general result establishing complete convergence for the row sums 
of an array of rowwise negatively dependent random variables is presented. From this result, 
a number of complete convergence results have been obtained
for weighted sums of negatively dependent 
random variables.}


\KWE{complete convergence; negatively dependent; weighted sums; arrays}

\vskip 14pt plus 9pt minus 6pt

      \thispagestyle{headings}

      \begin{multicols}{2}

            \label{st\stat}


\section{Introduction}

\noindent
The concept of complete convergence of a sequence of random variables was 
introduced by Hsu and Robbins~[1]
as follows. A~sequence $\{U_n, n\ge 1\}$ of random variables  converges completely 
to the constant $\theta$ if
$$ 
\sum\limits_{n=1}^{\infty}{\sf P}(|U_n-\theta |>\varepsilon)< \infty \quad 
\mbox{ for all }\quad \varepsilon> 0\,. 
$$

In view of the Borel--Cantelli lemma, this implies that $U_n \rightarrow \theta$ almost surely. 
The converse is true if $\{U_n, n \geq 1\}$ are independent random variables. 
Hsu and Robbins~[1] and Katz~[2] ($p=1$ and
$1< p <2,$ respectively) proved that if $\{X_n, n\ge 1\}$ is a sequence of independent and identically
distributed random variables with mean zero and ${\sf E}|X_1|^{2p}<\infty,$  
then $\sum\limits_{i=1}^n X_i/n^{1/p}$ converges
completely to zero.

The paper~[1] initiated numerous explorations of 
the complete convergence of sums of
independent random variables. The research was continued by Erd$\ddot{\mbox{o}}$s~[3, 4], 
Spitzer~[5], Baum and
Katz~[6], and Gut~[7]. This subject is actively discussed in scientific press during the last few decades.
For example, Hu {\it et al.}~[8] extended the result of Hsu--Robbins--Katz to the case where $\{X_{ni}, 1\le
i\le n, n\ge 1\}$ is an array of rowwise independent random variables which are stochastically dominated 
by a
random variable $X$ satisfying ${\sf E}|X|^{2p}<\infty$ for some $1\le p<2$.

The papers~[9, 10] contain, up to the authors' knowledge, the most general theorems
that provide sufficient conditions for complete convergence for sums of arrays of rowwise independent random
variables.

In the following, let $\{k_n, n\ge 1\}$ be a sequence of positive integers. In general,
the case $k_n=\infty$ is
not precluded. When $k_n=\infty,$ it will be assumed that $\sum\limits_{i=1}^\infty X_{ni}$ converges almost surely. Recall
that an array $\{X_{ni}, 1\le i\le k_n, n\ge 1\}$ of random variables is said to be {\it stochastically
dominated} by a random variable $X$ if there exists a positive constant $C>0$ such that

\noindent
\begin{multline*}
{\sf P}\{|X_{ni}|>x\}\le C {\sf P}\{|X|>x\}\\ \mbox{ for all } x>0, 1\le i\le k_n, \mbox{ and } n\ge 1\,.
\end{multline*}

Recently, some complete convergence theorems for negatively dependent random variables have been obtained by
many authors (see, for example,~[11, 12] and references in
these papers). Taylor \textit{et al}.~[11] extended the result of Hu 
\textit{et al}.~[8] to the array of rowwise negatively
dependent random variables. Giuliano {\it et al.}~[12] considered so-called acceptable random variable, which
is more general notion than negative dependency.

The finite set of random variables $X_1, \cdots, X_n $ is said to be 
{\it negatively dependent} if

\noindent
\begin{multline*}
{\sf P}\{X_1\le x_1, \dots, X_n\le x_n\}\\
{} \le {\sf P}\{X_1\le x_1\} \cdots {\sf P}\{X_n\le x_n\}\,;
\end{multline*}

\vspace*{-12pt}

\noindent
\begin{multline*}
{\sf P}\{X_1> x_1, \dots, X_n> x_n\}\\ \le {\sf P}\{X_1> x_1\} \cdots
{\sf P}\{X_n> x_n\} 
\end{multline*}
for all real $x_1, \dots, x_n$. An infinite sequence $\{X_n, n\ge 1\}$ is said to be negatively dependent if every
finite subset of the sequence $\{X_1, \dots, X_n\}$ is negatively dependent.

In this paper, a general result establishing complete convergence for the row sums of an array of
rowwise negatively dependent random variables is presented. 
It also specifies the corresponding rate of convergence. From
this result, a number of complete convergence\linebreak\vspace*{-12pt}

\pagebreak

\noindent
 results for negatively dependent random variables
have been obtained. As a
corollary, the result of Taylor {\it et al.}~[11] is obtained.

Throughout this paper, $C$ denotes a positive constant which may be different in various places, and
it is convenient to define $\log x =\max\{1, \ln x\}$.

\section{Preliminary Lemmas}

\noindent
To prove the main result, the following lemmas are necessary. The first two lemmas are well known and can be found,
for example, in~[11].

\medskip
\noindent 
\textbf{Lemma~1.} \textit{Let $\{X_n, n\ge 1\}$ be a sequence of negatively dependent  random variables and
$\{f_n, n\ge 1\}$ be a sequence of Borel functions all of which are monotone increasing (or monotone
decreasing), then $\{f_n(X_n), n\ge 1\}$ is a sequence of negatively dependent random variables.}

The second lemma mainly states that negatively dependent random variables are 
negatively correlated.

\medskip

\noindent 
\textbf{Lemma 2.} {\it Let $X_1,\dots, X_n$ be nonnegative negatively dependent integrable random variables. Then}
$$ 
{\sf E} \prod\limits_{i=1}^n X_i \le \prod\limits_{i=1}^n {\sf E}X_i\,. 
$$

\medskip

The following lemma plays an essential role in the main result. Of course, this lemma is of interest only if
positive constants $d_i$, and, hence, second moments ${\sf E}X_i^2, 1\le i \le n$, are close to zero (at least less than
one). Otherwise, there is an alternative so-called subgaussian estimations 
(see, for example,~[12]).

\medskip

\noindent 
\textbf{Lemma~3.} \textit{Let $X_1,\dots, X_n$ be negatively dependent  mean zero random variables such
that}
$$
\left|X_i\right|\le d_i\,, \quad 1\le i\le n\,,
$$
for a sequence of positive constants $d_1,\cdots, d_n$. Then, for any $t>0,$
$$ 
{\sf E} \exp\left\{t\sum\limits_{i=1}^n X_i\right\} \le \exp\left\{\fr{t^2}{2} 
\sum\limits_{i=1}^n e^{td_i} {\sf E}X_i^2 \right\}\,.
$$

\noindent 

P\,r\,o\,o\,f\,.\ From the inequality 
$e^x\le 1+x+({x^2}/{2}) e^{|x|}$, which is true for all $x$, one
has
\begin{multline*}
{\sf E} e^{tX_i} \le 1+ t{\sf E}X_i + \fr{t^2}{2} {\sf E} \left(X_i^2 e^{t|X_i|}\right) \\
{}= 1+ \fr{t^2}{2}\, {\sf E}\left( X_i^2 e^{t|X_i|}\right) \mbox{ (since $X_i$ have mean zero)} \\
{}\le 1+ \fr{t^2}{2}\,e^{td_i}  {\sf E}X_i^2 \le \exp\left\{ \fr{t^2}{2}\,e^{td_i} {\sf E}X_i^2\right\}\,,
\end{multline*}
since $1+x \le e^x$ for all $x$. It follows from Lemmas~1 and~2 that

\columnbreak

\noindent
\begin{multline*}
{\sf E} \exp\left\{t\sum\limits_{i=1}^n X_i\right\} \le \prod\limits_{i=1}^n {\sf E} e^{tX_i} \\
\hspace*{-3pt}{}\le \prod\limits_{i=1}^n \exp\left\{\!\fr{t^2}{2}
\,e^{td_i} {\sf E}X_i^2 \!\right\}
=\exp\left\{\!\fr{t^2}{2} \sum\limits_{i=1}^n e^{td_i} {\sf E}X_i^2\!\right\}.~\Box
\end{multline*}

\section{Main Result}

\noindent
With the preliminary lemmas, the  main result may now be stated and proved.

\medskip

\noindent 
\textbf{Theorem.} \textit{Let $\{X_{ni}, 1\le i \le k_n, n\ge 1\}$ be an array of rowwise negatively
dependent random variables, $\{a_n, n\ge 1\}$ be a sequence of positive constants, and $\{b_n, n\ge 1\}$ be a
sequence of positive constants such that $\lim\limits_{n\to\infty} b_n =\infty$. Suppose that}
\begin{enumerate}[($i$)]
\item $ \sum\limits_{n=1}^\infty a_n \sum\limits_{i=1}^{k_n} 
{\sf P}\{|X_{ni}|>\varepsilon\}<\infty$\ \textit{\ for all}\
$\varepsilon>0;$
\item $\sum\limits_{n=1}^\infty a_n \left(\sum\limits_{i=1}^{k_n}{\sf P}
\{ |X_{ni}|>1/b_n\} \right)^{N_1}\!<\infty$\
\textit{\ for some}\ $N_1>0;$
\item $b_n \sum\limits_{i=1}^{k_n} {\sf E} X_{ni}^2 I\{|X_{ni}|\le 1/b_n\} \to 0$ 
\textit{as}\ $n\to \infty;$ \textit{and}\
\item $\sum\limits_{n=1}^\infty a_n \exp\{-N_2 b_n\}<\infty$\ \textit{\ for some} $N_2>0.$
\end{enumerate}


\noindent
\textit{Then} 
\begin{multline*}
\sum\limits_{n=1}^\infty a_n {\sf P}\left\{\left| 
\sum\limits_{i=1}^{k_n} X_{ni}- {\sf E}X_{ni}I\left\{|X_{ni}|\le \fr{1}{b_n}
\right\}\right|> \varepsilon\right\}\\
{} <\infty
\end{multline*}
\textit{for all} $\varepsilon>0.$

\medskip 

\noindent 
P\,r\,o\,o\,f\,.\ The set of all natural numbers is partitioned into two subsets:
$$ 
A'=\left\{n : \sum\limits_{i=1}^{k_n} {\sf P}\left\{|X_{ni}|>\fr{1}{b_n}\right\}\le 1\right\} \,;
$$
$$
A''=\left\{n : \sum\limits_{i=1}^{k_n} {\sf P} \left\{|X_{ni}|>\fr{1}{b_n}\right\}> 1\right\}\,. 
$$
Applying~($ii$), one obtains
\begin{multline*}
\sum\limits_{n\in A''} \!a_n {\sf P}\left\{\left| \sum\limits_{i=1}^{k_n} \!X_{ni}- 
{\sf E}X_{ni}I\left\{|X_{ni}|\le \fr{1}{b_n}
\right\}\right|> \varepsilon\right\}
\\
\hspace*{-0.55399pt}{}\le \!\sum\limits_{n\in A''} \!a_n\le \!\sum\limits_{n\in A''}\! a_n \left( 
\sum\limits_{i=1}^{k_n} {\sf P}\left\{|X_{ni}|>\fr{1}{b_n}\right\} \right)^{N_1}\!\!\!\!<\infty. 
\end{multline*}
Hence, it is enough to show that
\begin{multline*}
\sum\limits_{n \in A'} a_n {\sf P}\left\{\left|
\sum\limits_{i=1}^{k_n} X_{ni}- {\sf E}X_{ni}I\left\{|X_{ni}|\le \fr{1}{b_n}\right\}\right|>\varepsilon
\right\}\\
<\infty \mbox{ for all } \varepsilon>0\,. 
\end{multline*}
For $1\le i\le k_n$ and $n\ge 1$, define
\begin{align*}
Y_{ni}&= X_{ni}I\left\{|X_{ni}|
\le \fr{1}{b_n}\right\}+ \fr{1}{b_n}\,I\left\{X_{ni}> \fr{1}{b_n}\right\}\\
&\hspace*{38mm}-\fr{1}{b_n}\,I\left\{X_{ni}< -\fr{1}{b_n}\right\}\,;\\
U_{ni}&= \fr{1}{b_n}\left(I\left\{X_{ni}<-\fr{1}{b_n}\right\}-{\sf P}
\left\{X_{ni}<-\fr{1}{b_n}\right\}\right)\,;\\
V_{ni}&= -\fr{1}{b_n}\left(I\left\{X_{ni}>\fr{1}{b_n}\right\}-{\sf P}\left\{X_{ni}> \fr{1}{b_n}\right\}\right)\,;\\
Z_{ni}&= X_{ni} I\left\{\fr{1}{b_n} <|X_{ni}|\le \fr{\varepsilon}{4[N_1+1]}\right\}\,.
\end{align*}
Then, $\{Y_{ni}-{\sf E}Y_{ni}$, $1\le i\le k_n$, $n\ge 1 \}$, $\{U_{ni}$, 
$1\le i\le k_n$, $n\ge 1 \}$, and $\{V_{ni}$, $1\le i\le
k_n$, $n\ge 1 \}$ are the arrays of rowwise negatively dependent  random variables by Lemma~1.

Note that if one defines
$$ 
W_{ni}= \fr{1}{b_n}\left(I\left\{|X_{ni}|>\fr{1}{b_n}\right\}-{\sf P}
\left\{|X_{ni}|>\fr{1}{b_n}\right\}\right)\,,
$$
then it cannot be stated that $\{W_{ni}$, $1\le i\le k_n$, $n\ge 1 \}$ is an array of negatively dependent random
variables. This is a sort of the main disadvantage when one is 
dealing with negatively dependent random variables.

Since $\lim\limits_{n\to\infty}b_n=\infty$, there exists a positive integer $M$ such that
$$
\fr{\varepsilon}{4[N_1+1]}>\fr{1}{b_n}
$$
for all $n>M$. For $n>M$,  one can write that
\begin{multline*} 
\sum\limits_{i=1}^{k_n} X_{ni}- {\sf E}X_{ni}I\left\{|X_{ni}|\le \fr{1}{b_n}\right\}\\
{} = \sum\limits_{i=1}^{k_n}(Y_{ni}-{\sf E}Y_{ni})+ \sum\limits_{i=1}^{k_n} U_{ni} + 
\sum\limits_{i=1}^{k_n} V_{ni}+ \sum\limits_{i=1}^{k_n} Z_{ni}\\
{}+\sum\limits_{i=1}^{k_n} X_{ni} I\left\{|X_{ni}|>\fr{\varepsilon}{4[N_1+1]}\right\}\,. 
\end{multline*}
It follows that

\noindent
\begin{multline*}
\sum\limits_{\substack{{n>M,}\\ {n \in A'}}}  a_n {\sf P}\left\{
\sum\limits_{i=1}^{k_n} X_{ni}- {\sf E}X_{ni}I\left\{|X_{ni}|\le \fr{1}{b_n}\right\}>\varepsilon\right\}\\
{}\le \sum\limits_{n>M, \ n \in A'} a_n {\sf P}\left\{\sum\limits_{i=1}^{k_n} 
Y_{ni}- {\sf E}Y_{ni}>\fr{\varepsilon}{4} \right\}\\
{}+ \sum\limits_{n>M, \ n \in A'} a_n {\sf P}\left\{\sum\limits_{i=1}^{k_n} U_{ni}>
\fr{\varepsilon}{4}\right\}\\
{}+ \sum\limits_{n>M, \ n \in A'} a_n {\sf P}\left\{\sum\limits_{i=1}^{k_n} V_{ni}>
\fr{\varepsilon}{4 }\right\} \\
{}+ \sum\limits_{n>M, \ n \in A'} a_n {\sf P}\left\{\sum\limits_{i=1}^{k_n} Z_{ni}>
\fr{\varepsilon}{4}\right\}\\
{}+ \sum\limits_{\substack{{n>M,}\\ {n \in A'}}} a_n 
{\sf P}\left\{ |X_{ni}|>\fr{\varepsilon}{4[N_1+1]} 
\mbox{ for some }1\le i\le k_n \right\}\\
{}=:I_1+I_2+I_3+I_4+I_5\,.
\end{multline*}

Now, let estimate each sum separately.

For $I_1,$ note that $|Y_{ni}|\le 1/b_n$ and
$$
Y_{ni}^2=X_{ni}^2I\left\{|X_{ni}|\le \fr{1}{b_n} \right\}+\left(\fr{1}{b_n}\right)^2 
I\left\{|X_{ni}|>\fr{1}{b_n}\right\}\,.
$$
Moreover, one has that
$$ 
\fr{1}{b_n} \sum\limits_{i=1}^{k_n} {\sf P}\left\{|X_{ni}|>\fr{1}{b_n}\right \}=o(1) \mbox{ for } n\in A'\,. 
$$
By Lemma~3 with $t=4(N_2+1)b_n/\varepsilon$,  one obtains that for $n\in A'$,
\begin{multline*}
{\sf P}\left\{\sum\limits_{i=1}^{k_n} (Y_{ni}- {\sf E}Y_{ni})>\fr{\varepsilon}{4}\right\}\\
{}\le \exp\left\{-\fr{t\varepsilon}{4}\right\} {\sf E} \exp\left\{t\sum\limits_{i=1}^{k_n} Y_{ni}-{\sf E}Y_{ni}\right\} \\
{}\le \exp\left\{-\fr{t\varepsilon}{4}\right\} \exp\left\{\fr{t^2}{2}\,e^{2t/b_n}
\sum\limits_{i=1}^{k_n} {\sf E}(Y_{ni}-{\sf E}Y_{ni})^2 \right\}\\
{}\le \exp\left\{\!-\fr{t\varepsilon}{4}\!\right\} 
\exp\left\{\fr{t^2}{2}\, e^{2t/b_n}\sum\limits_{i=1}^{k_n} {\sf E}Y_{ni}^2 \right\}
=  \exp\left\{\!
-\fr{t\varepsilon}{4}\!\right\} \\
{}\times \exp\left\{\fr{t^2}{2}\, e^{2t/b_n}\sum\limits_{i=1}^{k_n}
{\sf E}X_{ni}^2I\left\{|X_{ni}|\le \fr{1}{b_n} \right\}\right.\\
\left.{} + \fr{1}{b_n^2} \,{\sf P}\left\{
\left|X_{ni}\right|>\fr{1}{b_n}\right\} \right\}
\le 
\exp\left\{-\vphantom{8\left(N_2 +1\right)^2 e^{8(N_2+1)/\varepsilon}}
\left(N_2+1\right)b_n\right.\\
\left.{}+ {8\left(N_2 +1\right)^2 e^{8(N_2+1)/\varepsilon}}{\varepsilon^{-2}}
o(1)b_n\right\} \mbox{ (by~($iii$))}\\
{}=\exp\left\{-\left(N_2+1 - o(1)\right)b_n \right\} 
\le \exp\{-N_2 b_n\}
\end{multline*}
for all large $n$. Thus, $I_1<\infty$ by~($iv$).

For $I_2,$ it can be observed that $|U_{ni}|\le 1/b_n$ and  ${\sf E}U_{ni}^2\le 
{\sf P}(|X_{ni}|>1/b_n)/b^2_n$. Hence,
\begin{multline*}
\sum\limits_{i=1}^{k_n} {\sf E} U_{ni}^2\le \fr{1}{b^2_n} 
\sum\limits_{i=1}^{k_n} {\sf P}\left\{|X_{ni}|>\fr{1}{b_n}\right\}=\fr{1}{b_n}\, o(1)\\
 \mbox{for } n\in  A'\,. 
\end{multline*}

By Lemma~3 with $t=4(N_2+1)b_n/\varepsilon$,  one obtains that for $n\in A',$
\begin{multline*}
{\sf P}\left\{\sum\limits_{i=1}^{k_n} U_{ni}>\fr{\varepsilon}{4}\right\} 
\le \exp\left\{-\fr{t\varepsilon}{4}\right\}
{\sf E}\exp\left\{t\sum\limits_{i=1}^{k_n}U_{ni}\right\}\\
{}\le\exp\left(-\fr{t\varepsilon}{4}\right) \exp\left\{\fr{t^2}{2}\,e^{t/b_n} 
\sum\limits_{i=1}^{k_n}\sf{E} U_{ni}^2 \right\} \\
\le\exp\left\{-
\vphantom{(N_2 +1)^2 e^{4(N_2+1)/\varepsilon}\varepsilon^{-2}}
(N_2+1)b_n\right.\\
\hspace*{-2.61754pt}\left.{} + 8(N_2 +1)^2 e^{4(N_2+1)/\varepsilon}\varepsilon^{-2}o(1) b_n \right\} \le \exp\{-N_2 b_n\}
\end{multline*}
for all large $n$. Thus, $I_2<\infty$ by~($iv$).

Similarly to $I_2,$ one gets $I_3<\infty.$

For $I_4,$ note that
\begin{equation*}
{\sf P}\left\{\sum\limits_{i=1}^{k_n} Z_{ni}>\fr{\varepsilon}{4}\right\} \le 
{\sf P}\{ \mbox{ at least } 
[N_1 +1] \mbox{ of } Z_{ni}\not= 0\} 
\end{equation*}
because
\begin{multline*}
  Z_{ni}<\fr{\varepsilon}{4[N_1 + 1]}\\
= {\sf P}\left\{\vphantom{\fr{1}{b_n}}\mbox{at least } 
[N_1 +1] \mbox{ of } X_{ni} \mbox{ have the property}\right.\\
\left.\fr{1}{b_n}<|X_{ni}|
\le \fr{\varepsilon}{(4[N_1+1])} \right\} \\
{}\le \!\!\sum\limits_{j_1<\cdots <j_{[N_1+1]}}\!\!\!\! {\sf P}\left\{X_{n,j_1}>\fr{1}{b_n} , \dots, 
X_{n,j_{[N_1+1]}}>\fr{1}{b_n}\right\} \\
\mbox{(where the summation is taken for all }\ [N_1+1]\\
{}- \mbox{tuple } (j_1, \cdots, j_{[N_1+1]}) \\
\mbox{ with } j_1<\cdots <j_{[N_1+1]} \mbox{ and } j_i=1,\dots,k_n \mbox{ for each } i)\\
{}\le\!\! \!\!\!\sum\limits_{j_1<\cdots <j_{[N_1+1]}}\!\!\!\!\!\!\!\! {\sf P}\left\{X_{n,j_1}>\fr{1}{b_n}\right\} 
\cdots {\sf P}\left\{X_{n,j_{[N_1+1]}}>\fr{1}{b_n} \right\}\\
\mbox{(by negative dependence)}\\
{}= \sum\limits_{j_1<\cdots <j_{[N_1+1]}} \prod\limits_{k=1}^{[N_1+1]}  {\sf P}\left\{X_{n,j_k}>\fr{1}{b_n}\right\} \\
{}\le \sum\limits_{j_1, \cdots, j_{[N_1+1]}} \prod\limits_{i=1}^{[N_1+1]}{\sf  P}\left\{X_{n,j_i}>\fr{1}{b_n}\right\} \\
\end{multline*}

\noindent
\begin{multline*}
\mbox{(where the summation is taken for all possible}\\
[N_1+1] -
\mbox{tuple } (j_1, \dots, j_{[N_1+1]})
\\
\mbox{ and}\ j_i=1,\dots,k_n \mbox{ for each } i)\\
{}=\left(\sum\limits_{i=1}^{k_n} {\sf P}\left\{|X_{ni}|>\fr{1}{b_n}\right\} \right)^{[N_1+1]}\,.
\end{multline*}
Thus, $I_4<\infty$ by~($ii$).

Obviously, $I_5<\infty$ by~($i$).

Therefore, one has that
\begin{multline*}
\hspace*{-6.64308pt}\sum\limits_{\substack{{n>M,}\\ {n \in A'}}}\!\!\!  a_n {\sf P}\left(\sum_{i=1}^{k_n} \left(X_{ni}- 
{\sf E}X_{ni}I\left(|X_{ni}|\le \fr{1}{b_n}\right)\right)\right.
\left.{}>\varepsilon\vphantom{\sum_{i=1}^{k_n}}\right)\\ <\infty\,. 
\end{multline*}
Since $\{-X_{ni}\}$ is also an array of rowwise  negatively dependent  random variables, one can replace $X_{ni}$
by $-X_{ni}$ in the above statement. That is,
\begin{multline*}
\hspace*{-3.68776pt}\sum\limits_{\substack{{n>M,}\\ {n \in A'}}} \!\!\! a_n {\sf P}\!
\left(\sum\limits_{i=1}^{k_n}\! \left(X_{ni}- {\sf E}X_{ni}I\!\left(\left|X_{ni}\right|
\le \fr{1}{b_n}\right)\!\right)
<-\varepsilon
\vphantom{\sum\limits_{i=1}^{k_n}}\right)\\
~<\infty\,.~~\square 
\end{multline*}


\noindent 
{\bf Remark 1.} In view of assumption~($iii$), it is interesting to consider sequences $\{b_n, n\ge 1\}$
that increase to infinity as slow as possible for~($iv$) still be true. If the sequence $\{a_n, n\ge 1\}$ has a
polynomial growth or a constant (that is, $a_n=n^t$, $t\ge 0$), then the good choice is $b_n=\log n$, $n\ge 1$,
which has been explored in~[10] for the case of rowwise independent arrays. 
But the present
theorem can be
applied for sequences $\{a_n$, $n\ge 1\}$ with a different than polynomial behavior. 
The main idea is that it is possible to
link sequences $\{a_n, n\ge 1\}$ and $\{b_n, n\ge 1\}$ according to assumption~($iv$).

\section{Corollaries}

\noindent
The theorem presented and proved in the previous section can be applied in different situations for various choices
of weights and moment conditions.

\smallskip

\noindent 
\textbf{Corollary 1.} \textit{Let $\{X_{ni}, 1\le i \le n, n\ge 1\}$ be an array of rowwise negatively
dependent mean zero random variables which are stochastically dominated by a random variable~$X$ with
${\sf E}|X|^{2p}<\infty$ for some $p\ge 1$. Let $\{a_{ni}$, $1\le i\le n$, $n\ge 1\}$ be an array of real numbers and
$\{b_n$, $n\ge 1\}$ be a sequence of positive constants such that}
\begin{itemize}
\item[($a$)] $\lim\limits_{n\to\infty} b_n=\infty$;
\item[($b$)] $b_n= O(n^q)$\ for some $0<q <1/(2p)$;
\item[($c$)] $\sum\limits_{n=1}^\infty \exp\{-N_2 b_n\}<\infty$\ for some $N_2>0;$
\item[($d$)] $b_n \sum\limits_{i=1}^n a_{ni}^2=o(1)$ as $n\to\infty$; and 
\item[($e$)] $\max\limits_{1\le i\le n} |a_{ni}|=O(1/n^{1/p})$.
\end{itemize}
\textit{Then, $ \sum\limits_{i=1}^n a_{ni}X_{ni} \to 0$ completely.}

\medskip

\noindent 
P\,r\,o\,o\,f\,. 
Without loss of generality, one may assume that $a_{ni}\ge 0$  for $1\le i\le n$ and $n\ge
1$. Otherwise, let prove the result separately for two arrays of constants $\{a^+_{ni}$, $1\le i\le n$, 
$n\ge 1\}$ and
$\{a^-_{ni}$, $1\le i\le n$, $n\ge 1\}$, where  the notations $a^+=\max\{a, 0\}$ and $a^-=\max\{-a, 0\}$
are used. Then,
$\{a_{ni} X_{ni}$, $1\le i\le n$, $n\ge 1\}$ is an array of rowwise negatively 
dependent random variables by 
Lemma~1. It can  be also assumed that $\max\limits_{1\le i\le n} a_{ni}\le 1/n^{1/p}$.

Let apply the theorem with $a_n=1$, $n\ge 1$, and $X_{ni}$ replaced by $a_{ni}X_{ni}$, 
$1\le i\le n$, $n\ge 1$.

In order to check condition~($i$) of the theorem, note that by the stochastic domination hypothesis,
\begin{multline*}
\hspace*{-6.44308pt}\sum\limits_{n=1}^\infty  \sum\limits_{i=1}^n {\sf P}
\left\{\left\vert a_{ni}X_{ni}\right\vert>\varepsilon\right\}\le\sum\limits_{n=1}^\infty
\sum\limits_{i=1}^n
{\sf P}\{|X_{ni}|>\varepsilon n^{1/p}\} \\
{}\le C \sum\limits_{n=1}^\infty  n {\sf P}\{|X|>\varepsilon n^{1/p}\}\,.
\end{multline*}
The sum $\sum\limits_{n=1}^\infty  n {\sf P}\{|X|^p>n\}<\infty$ if and only if 
${\sf E}|X|^{2p}<\infty$.
Thus, condition~($i$) of the theorem holds.

For condition~($ii$), taking $N_1>1/(1-2pq)$, one has by Markov's inequality and the stochastic domination
hypothesis that
\begin{multline*}
\sum\limits_{n=1}^\infty \left(\sum\limits_{i=1}^n {\sf P}\left\{|a_{ni}X_{ni}|>\fr{1}{b_n}\right\} \right)^{N_1}\\
{}\le \sum\limits_{n=1}^\infty \left( b_n^{2p} \sum\limits_{i=1}^n |a_{ni}|^{2p} 
{\sf E}|X_{ni}|^{2p} \right)^{N_1} \\
{}\le \sum\limits_{n=1}^\infty \left(C {\sf E}|X|^{2p} \fr{b_n^{2p}}{n} \right)^{N_1} 
\Bigg( \mbox{by assumption~($e$))} \\
{}< \infty \left(
\vphantom{\fr{1}{pq}}\mbox{by assumption~($b$) and the fact}\right.\\
\left.\mbox{that } N_1>\fr{1}{1-2pq}\right)\,.
\end{multline*}
Thus, condition~($ii$) holds.

For condition~($iii$),
\begin{multline*}
 b_n \sum\limits_{i=1}^n {\sf E}(a_{ni}X_{ni})^2I\left(\left|a_{ni}X_{ni}\right|\le \fr{1}{b_n}\right)\\
{}  \le b_n\sum\limits_{i=1}^n a_{ni}^2 {\sf E} X_{ni}^2 
\le C {\sf E} X^2 b_n \sum\limits_{i=1}^n a_{ni}^2\to 0 \mbox{ (by ($d$))}
\end{multline*}
Thus, condition~($iii$) holds.

Condition~($iv$) holds by the assumption~($c$).

By the theorem, one obtains that
\begin{multline*}
\hspace*{-1.55515pt}\sum\limits_{n=1}^\infty {\sf P}\left\{
\left\vert \sum\limits_{i=1}^n a_{ni}\left(X_{ni}- 
{\sf E}X_{ni}I\left\{\left\vert a_{ni}X_{ni}\right\vert \le \fr{1}{b_n}\right\}\right)
\right\vert\right.\\ 
\left.{}>\varepsilon 
\vphantom{\sum\limits_{i=1}^n}\right\}
<\infty
\end{multline*}
for all $\varepsilon>0$.
It remains to show that
$$ 
\sum\limits_{i=1}^n a_{ni}{\sf E}X_{ni}I\left\{|a_{ni}X_{ni}|\le \fr{1}{b_n}\right\}\to 0\,. 
$$
Since ${\sf E}X_{ni}=0$,
$$
{\sf E}X_{ni}I\left\{\!|a_{ni}X_{ni}|\le \fr{1}{b_n}\!\right\}=
-{\sf E}X_{ni}I\left\{\!|a_{ni}X_{ni}|>\fr{1}{b_n}\!\right \}.
$$
It follows that
\begin{multline*}
\left\vert\sum\limits_{i=1}^n a_{ni}{\sf E}X_{ni}I
\left\{\left\vert a_{ni}X_{ni}\right\vert \le \fr{1}{b_n}\right\}\right\vert\\
{}\le\sum\limits_{i=1}^n \left\vert a_{ni}\right\vert {\sf E} \left\vert X_{ni}\right\vert
I\left\{\left\vert a_{ni}X_{ni}\right\vert > \fr{1}{b_n}\right\}\\
{}\le \fr{1}{n^{1/p}} \sum\limits_{i=1}^n {\sf E}\left\vert X_{ni}\right\vert
I\left\{\left\vert X_{ni}\right\vert>\fr{n^{1/p}}{b_n}\right\}\\
\mbox{ (by assumption ($e$))} \\
{}\le C n^{1 - \ 1/p} {\sf E}|X|I\left\{|X|>\fr{n^{1/p}}{b_n}\right\}\\
{}\le C n^{1 - \ 1/p} {\sf E}|X|^{2p} |X|^{1-2p} I\left\{|X|>\fr{n^{1/p}}{b_n}\right\} \\
{}\le C {\sf E}|X|^{2p} n^{1 - \ 1/p}\left(\fr{b_n}{n^{1/p}}\right)^{2p-1} \le C n^{-1/(2p)}\to 0
\end{multline*}
since $b_n<Cn^{1/(2p)}$ for $n$ large enough. Thus, the proof is completed.\hfill $\Box$

\medskip
As a special case of Corollary~1, one gets the following corollary which was proved by 
Taylor \textit{et al}.~[11].

\medskip

\noindent 
\textbf{Corollary 2.} \textit{Let $\{X_{ni}, 1\le i \le n, n\ge 1\}$ be an array of rowwise negatively dependent
mean zero random variables which are stochastically dominated by a random variable~$X$ with 
${\sf E}|X|^{2p}<\infty$
for some $1\le p<2.$ Then,
$\sum\limits_{i=1}^n {X_{ni}}/{n^{1/p}} \to 0$ completely.} 


\noindent 
P\,r\,o\,o\,f\,.\ Let $a_{ni}=1/n^{1/p}$ for $1\le i\le n$ and $n\ge 1.$ 
Then, conditions of Corollary~1 are
trivially satisfied with $b_n= n^q$ for some $0<q <\min\left\{{1}/({2p}), {2}/{p}-1\right\}.\hfill~\Box$

\medskip

\noindent 
\textbf{Corollary 3.} \textit{Let $t>-1, p>0$, and $\beta \in {\mathbb R}$. Denote $\Delta=p(t+\beta+1)$ and
assume that $\Delta\ge 1$. Let $\{X_{ni}, i\ge 1, n\ge 1\}$ be an array of rowwise negatively dependent  mean
zero random variables which are stochastically dominated by a random variable~$X$ with 
${\sf E}|X|^{\Delta}<\infty$.
Let $\{a_{ni}, i\ge 1, n\ge 1\}$ be a bounded array of real numbers such that}
\begin{enumerate}[(1)]
\item  $ \sum\limits_{i=1}^\infty |a_{ni}|^q =O(n^\beta)$\ \textit{for some} $q< \Delta$; \textit{and}
\item  If $\Delta \ge 2$, \textit{then} $\sum\limits_{i=1}^\infty a_{ni}^2=O(n^\gamma )$
\textit{for some} $\gamma<2/p$.
\end{enumerate}
\textit{Then,}
$$ 
\sum\limits_{n=1}^\infty n^t {\sf P}\left\{ 
\fr{\left|\sum\nolimits_{i=1}^\infty a_{ni}X_{ni}\right|}{n^{1/p}}>\varepsilon \right\}
<\infty \mbox{ for all } \varepsilon>0\,.
$$

\noindent 
P\,r\,o\,o\,f\,.\ The same as in the proof of Corollary~1, without loss of generality, 
one may assume that
$a_{ni}\ge 0$ for $i\ge 1, n\ge 1.$ Then, $\{a_{ni}X_{ni}/n^{1/p}$, $i\ge 1$, $n\ge 1\}$ 
is an array of rowwise
negatively dependent random variables by Lemma~1. 
Let apply the theorem with $a_n=n^t$, $n\ge 1$, and $X_{ni}$
replaced by $a_{ni}X_{ni}/n^{1/p}$, $i\ge 1$, $n\ge 1$.

Consider the sequence $b_n=n^{\alpha}, n\ge 1$, 
where $0<\alpha <(t+1)/\Delta$. For the case $\Delta\ge 2$,
let require additionally that $0<\alpha< 2/p - \gamma$.

The fact that
\begin{multline*}
\sum\limits_{n=1}^\infty n^t \sum\limits_{i=1}^\infty 
{\sf P}\left( \left\vert a_{ni}n^{-1/p} X_{ni}\right\vert>
\varepsilon\right)\\
\le C{\sf E}|X|^{p(t+\beta+1)}
< \infty 
\end{multline*}
was established in many papers (see, for example,~[13]) 
(beginning of the proof of Theorem~3.1),
[14] (beginning of the proof of Theorem~3.1), and~[10] 
(beginning of the proof of Theorem~2 and Lemma~3). 
Note also  that the proof presented in~[13] is rather complicated
once it uses the Stieltjes integration technique, summation by parts lemma, and so on. 
The proof presented in~[14] is much more elegant. Also, Hu {\it et al.}~[13] and 
Ahmed {\it et al.}~[14] 
are dealing with an array of constants $\{ a_{ni}X_{ni}$, $i\ge 1$, $n\ge 1\}$ rather 
than the array $\{a_{ni}X_{ni}/n^{1/p}$, $i\ge 1$, $n\ge 1\}$ which is considered in~[10] 
and this paper.

According to the inequality presented above, condition~($i$) of the theorem holds.

For~($ii$), taking $N_1>(t+1)/(t+1 -\alpha\Delta)>0$, one has by Markov's inequality, 
$|a_{ni}|=O(1),$ and~(1)
that
\begin{multline*}
\sum\limits_{n=1}^\infty n^t \left(
\sum\limits_{i=1}^\infty {\sf P}\left\{\left|a_{ni}n^{-1/p}X_{ni}\right|>\fr{1}{b_n}\right\} \right)^{N_1}
\end{multline*}

\noindent
\begin{multline*}
{}\le \sum\limits_{n=1}^\infty n^t \left( b_n^{\Delta} n^{-(t+\beta+1)} 
\sum\limits_{i=1}^\infty \left|a_{ni}\right|^{\Delta}
{\sf E}|X_{ni}|^{\Delta} \right)^{N_1}\\
{}\le C\sum\limits_{n=1}^\infty n^t \left( b_n^{\Delta} n^{-(t+\beta + 1)} 
\sum\limits_{i=1}^\infty \left|a_{ni}\right|^{q} \left|a_{ni}\right|^{\Delta-q} \right)^{N_1} \\
{}\le C \sum\limits_{n=1}^\infty n^{t + \alpha \Delta N_1 -(t+1)N_1} <\infty\,,
\end{multline*}
since  $t +\alpha \Delta N_1 -(t+1)N_1< -1$. Thus, condition~($ii$) of the theorem holds.

For condition~($iii$), let consider two cases. If $1\le \Delta<2$, by~(1), one obtains

\noindent
\begin{multline*}
b_n \sum\limits_{i=1}^\infty {\sf E}\left(a_{ni}n^{-1/p} X_{ni}\right)^2 
I\left\{\left|a_{ni}n^{-1/p} X_{ni}\right|\le \fr{1}{b_n}\right\}\\
{}=b_n \sum\limits_{i=1}^\infty {\sf E}\left|a_{ni}n^{-1/p} X_{ni}\right|^{\Delta} 
\left|a_{ni}n^{-1/p} X_{ni}\right|^{2-\Delta}\\
{}\times I\left\{\left|a_{ni}n^{-1/p} X_{ni}\right|\le \fr{1}{b_n}\right\}\\
\hspace*{-0.36795pt}{}\le b_n^{\Delta-1}\!\sum\limits_{i=1}^\infty {\sf E}\left|a_{ni}n^{-1/p} X_{ni}\right|^{\Delta} 
\!I\!\left\{\!\left|a_{ni}n^{-1/p}X_{ni}\right|\le \fr{1}{b_n}\!\right\} \\
{}\le b_n^{\Delta-1}\sum\limits_{i=1}^\infty {\sf E}\left|a_{ni}n^{-1/p} X_{ni}\right|^{\Delta} \\
{}\le C b_n^{\Delta-1}{\sf E}|X|^{\Delta} \sum\limits_{i=1}^\infty \left|a_{ni}n^{-1/p}\right|^{\Delta} \\
{}\le C n^{\alpha\Delta -\alpha -t -1} < Cn^{-\alpha} \to 0 \mbox{ as } n\to \infty
\end{multline*}
by the choice of~$\alpha$.

If $ \Delta\ge 2$, then by~(2)

\noindent
\begin{multline*}
b_n \sum\limits_{i=1}^\infty {\sf E}\left(a_{ni}n^{-1/p} X_{ni}\right)^2 I
\left\{\left|a_{ni}n^{-1/p} X_{ni}\right|\le\fr{1}{b_n}\right\} \\
{}\le  C b_n {\sf E} X^2 \sum\limits_{i=1}^\infty \fr{a_{ni}^2}{n^{2/p}} 
\le C {\sf E} X^2 n^{\alpha +\gamma- \ 2/p} \to 0
\end{multline*}
as $n\to \infty$
by the choice of~$\alpha$. Thus, condition~($iii$) of the theorem holds.

Condition~($iv$) holds trivially.

Hence, one gets by the theorem that

\noindent
\begin{multline*}
\sum\limits_{n=1}^\infty n^t {\sf P}\left\{ \left|\sum\limits_{i=1}^\infty a_{ni}n^{-1/p} 
\left(X_{ni}\right.\right.\right.\\
\left.\left.\left.{}- {\sf E}X_{ni}I\left\{\left|a_{ni}X_{ni}\right|\le
\fr{n^{1/p}}{b_n}\right\}\right)\right|> \varepsilon \right\}<\infty
\end{multline*}
for all $\varepsilon>0$.
It remains to show that

\noindent
$$ 
\sum\limits_{i=1}^\infty a_{ni}n^{-1/p}  {\sf E}X_{ni}I
\left\{\left|a_{ni}X_{ni}\right|\le \fr{n^{1/p}}{b_n}\right\}\to 0\,. 
$$
Since ${\sf E}X_{ni}=0$,
\begin{multline*}
 {\sf E}X_{ni}I\left(|a_{ni}X_{ni}|\le \fr{n^{1/p}}{b_n}\right)\\=
-{\sf E}X_{ni}I\left\{|a_{ni}X_{ni}|\right.
\left.
>\fr{n^{1/p}}{b_n}\right\}\,.
\end{multline*}
It follows that
\begin{multline*}
\left|\sum\limits_{i=1}^\infty a_{ni}n^{-1/p} {\sf E}
X_{ni}I\left\{\left|a_{ni}X_{ni}\right|\le \fr{n^{1/p}}{b_n}\right\}\right|\\
{}\le n^{-1/p} \sum\limits_{i=1}^\infty   {\sf E}\left|a_{ni}X_{ni}\right| I
\left\{\left|a_{ni}X_{ni}\right|> \fr{n^{1/p}}{b_n}\right\}\\
{}\le n^{-1/p} \left(\fr{b_n}{n^{1/p}}\right)^{\Delta-1}\\
{}\times \sum\limits_{i=1}^\infty  
{\sf E}\left|a_{ni}X_{ni}\right|^{\Delta} I\left\{\left|a_{ni}X_{ni}\right|>
\fr{n^{1/p}}{b_n}\right\}\\
{}\le \fr{C  (b_n)^{\Delta-1} {\sf E}|X|^{\Delta}}{n^{t+1}} \le C n^{\alpha(\Delta-1)-t-1} \to 0
\end{multline*}
by the choice of~$\alpha$.

Thus, the proof is completed.\hfill~$\Box$

\medskip
\noindent 
\textbf{Remark~2.} If $t<-1,$ then the conclusion of Corollary~3 holds trivially. When 
$t\ge -1$, Sung~[10] proved Corollary~3 under the stronger condition that $\{X_{ni}$, $i\ge 1$, $n\ge 1\}$ is an 
array  of rowwise
independent random variables. However, the relatively important case $t=-1$ in 
Corollary~3 cannot be proved by
using the theorem. The present authors left as an open problem whether Corollary~3 holds for $t=-1$.

\medskip
As a special case of Corollary 3, let get the following corollary.

\medskip

\noindent 
\textbf{Corollary~4.} \textit{Let $t>-1$ and $1\le p<2.$ 
Let $\{X_{ni}$, $1\le i \le n$, $n\ge 1\}$ be an array
of rowwise negatively dependent  mean zero random variables which are stochastically dominated by a random
variable~$X$ with ${\sf E}|X|^{p(t+2)}<\infty.$ Then,}
$$
\sum\limits_{n=1}^\infty n^t {\sf P}\left\{ \fr{\left|\sum\nolimits_{i=1}^n X_{ni}\right|}{n^{1/p}}>
\varepsilon\right\}<\infty
\mbox{ for all } \varepsilon>0\,.
$$

\noindent 
P\,r\,o\,o\,f\,.\ Let $a_{ni}=1$ for $1\le i\le n$ and $a_{ni}=0$ for $i>n.$ 
Then, for $q<p(t+2),$
$\sum\limits_{i=1}^\infty |a_{ni}|^q=n.$ Thus, assumption~(1) of Corollary~3 holds for 
$\beta=1$. Since $1\le p<2$,
assumption~(2) holds for $\gamma=1$. Thus, the result follows from Corollary~3.\hfill~$\Box$

\medskip

\noindent 
\textbf{Remark~3.} When $t=0,$  Corollary~4 is the same as Corollary~2.

\section*{Acknowledgments} 

\noindent
This research is partially supported by the 
Center of Excellence in Mathematics, the Commission on Higher Education, Thailand.

  {\small\frenchspacing
{%\baselineskip=10.8pt
\addcontentsline{toc}{section}{Литература}
\begin{thebibliography}{99}

\bibitem{7-vol} %1
\Au{Hsu P.\,L., Robbins~H.}  
Complete convergence and the law of large numbers~// Proc. Nat. Acad. Sci. USA, 1947.
Vol.~ 33. P.~25--31.

\bibitem{10-vol} %2
\Au{Katz M.}  The probability in the tail of a distribution~//
Ann. Math. Stat., 1963. Vol.~34. P.~312--318.

\bibitem{3-vol} %3
\Au{Erd$\ddot{\mbox{o}}$s~P.} 
On a theorem of Hsu and Robbins~// Ann. Math. Statist., 1949. Vol.~20. P.~286--291.

\bibitem{4-vol} %4
\Au{Erd$\ddot{\mbox{o}}$s~P.} 
Remark on my paper ``On a theorem of Hsu and Robbins''~// Ann. Math. Statist., 1950.
Vol.~21. P.~138.

\bibitem{12-vol} %5
\Au{Spitzer F.\,L.}
 A~combinatorial lemma and its applications~//  Trans. Amer. Math. Soc., 1956. Vol.~82.
P.~323--339.


\bibitem{2-vol} %6
\Au{Baum K.\,B., Katz M.} 
Convergence rates in the law of large numbers~// Trans. Amer. Math. Soc., 1965.
Vol.~120. P.~108--123.

\bibitem{6-vol} %7
\Au{Gut A.} Complete convergence for arrays~// 
Periodica Math. Hungarica, 1992. Vol.~25. P.~51--75.


\bibitem{9-vol} %8
\Au{Hu~T.-C., M$\acute{\mbox{o}}$ricz~F., Taylor~R.\,L.}
 Strong laws of large numbers for arrays of rowwise independent
random variables~//  Acta Math. Hung.,  1989. Vol.~54. P.~153--162.

\bibitem{11-vol} %9
\Au{Kruglov V.\,M., Volodin A.\,I., Hu~T.-C.}
 On complete convergence for arrays~//  Stat. Prob. Lett.,  2006.
Vol.~76. P.~1631--1640.

\bibitem{13-vol} %10
\Au{Sung S.\,H.}  Complete convergence for weighted sums of random variables~//
Stat. Prob. Lett., 2007. Vol.~77. P.~303--311.

\bibitem{14-vol} %11
\Au{Taylor R.\,L., Patterson~R.\,F., Bozorgnia~A.} 
A~strong law of large numbers for arrays of rowwise
negatively dependent random variables~// Stochastic Anal. Appl., 2002. Vol.~20. P.~643--656.

\bibitem{5-vol} %12
\Au{Giuliano Antonini~R., Kozachenko~V., Volodin~A.}  Convergence of series of dependent
$\varphi$-subgaussian random variables~// J.~Math. Anal. Appl., 2008.
Vol.~338. P.~1188--1203.

\bibitem{8-vol} %13
\Au{Hu T.-C., Li~D., Rosalsky~A., Volodin~A.}  On the rate of complete convergence for weighted sums of
Banach space valued random elements~//  Theor. Prob. Appl., 2002. Vol.~47. P.~5455--5468.

\bibitem{1-vol} %14
\Au{Ahmed S.\,E., Giuliano Antonini~R., Volodin~A.}
On the rate of complete convergence for weighted sums of
arrays of Banach space valued random elements with application to moving average process~// 
Statist. Prob. Lett.,  2002. Vol.~58. P.~185--194.



\end{thebibliography}
}
}


\end{multicols}

%\vspace*{6pt}

%\hrule

%\vspace*{6pt}


\def\tit{ПОЛНАЯ СХОДИМОСТЬ СУММ В~СХЕМЕ СЕРИЙ ОТРИЦАТЕЛЬНО ЗАВИСИМЫХ СЛУЧАЙНЫХ ВЕЛИЧИН}

\def\aut{С.\,Х.~Санг$^1$, К.~Будсаба$^2$, А.~Володин$^3$}

\titelr{\tit}{\aut}

\vspace*{6pt}


\noindent
$^1$Университет Пай Чай, Республика Корея, sungsh@pcu.ac.kr\\[1pt]
\noindent
$^2$Университет Таммасат, Таиланд, kamon@mathstat.sci.tu.ac.th\\[1pt]
\noindent
$^3$Университет Реджайны, Канада,  Andrei.Volodin@uregina.ca

\vspace*{6pt}

\Abst{Приводится  результат о полной сходимости для сумм в схеме 
серий для отрицательно зависимых случайных величин в весьма общей форме.
Из этого результата следуют многие факты о полной сходимости взвешенных 
сумм отрицательно зависимых случайных величин.}

\label{end\stat}


\KW{полная сходимость; отрицательная зависимость; взвешенные суммы; схема серий}

%\renewcommand{\figurename}{\protect\bf Рис.}
%\renewcommand{\tablename}{\protect\bf Таблица}
\renewcommand{\bibname}{\protect\rmfamily Литература} %13


%\end{document}

%   { %\Large  
   { %\baselineskip=16.6pt
   
   \vspace*{-48pt}
   \begin{center}\LARGE
   \textit{Предисловие}
   \end{center}
   
   %\vspace*{2.5mm}
   
   \vspace*{25mm}
   
   \thispagestyle{empty}
   
   { %\small 

    
Вниманию читателей журнала <<Информатика и её применения>> предлагается 
очередной тематический выпуск <<Вероятностно-статистические методы и 
задачи информатики и информационных технологий>>. Предыдущие тематические 
выпуски журнала по данному направлению вышли в 2008~г.\ (т.~2, вып.~2), 
в 2009~г.\ (т.~3, вып.~3) и в 2010~г.\ (т.~4, вып.~2). 

Статьи, собранные в данном журнале, посвящены разработке новых вероятностно-статистических 
методов, ориентированных на применение к решению конкретных задач информатики и информационных 
технологий, а также~--- в ряде случаев~--- и других прикладных задач. Проблематика, охватываемая 
публикуемыми работами, развивается в рамках научного сотрудничества между Институтом проблем 
информатики Российской академии наук (ИПИ РАН) и Факультетом вычислительной математики и 
кибернетики Московского государственного университета им.\ М.\,В.~Ломоносова в ходе работ 
над совместными научными проектами (в том числе в рамках функционирования 
Научно-образовательного центра <<Вероятностно-статистические методы анализа рисков>>). 
Многие из авторов статей, включенных в данный номер журнала, являются активными участниками 
традиционного международного семинара по проблемам устойчивости стохастических моделей, 
руководимого В.\,М.~Золотаревым и В.\,Ю.~Королевым; регулярные сессии этого семинара 
проводятся под эгидой МГУ и ИПИ РАН (в 2011~г.\ указанный семинар проводится в октябре 
в Калининградской области РФ). 

Наряду с представителями ИПИ РАН и МГУ в число авторов данного выпуска журнала входят 
ученые из Научно-исследовательского института системных исследований РАН, Института 
проблем технологии микроэлектроники и особочистых материалов РАН, Института 
прикладных математических исследований Карельского НЦ РАН, Московского 
авиационного института, Вологодского государственного педагогического университета, 
НИИММ им.\ Н.\,Г.~Чеботарева, Казанского государственного университета, Дебреценского 
университета (Венгрия).

Несколько статей выпуска посвящено разработке и применению стохастических методов и 
информационных технологий для решения различных прикладных задач. В~работе В.\,Г.~Ушакова 
и О.\,В.~Шестакова рассмотрена задача определения вероятностных характеристик случайных 
функций по распределениям интегральных преобразований, возникающих в задачах эмиссионной 
томографии. В~статье Д.\,О.~Яковенко и М.\,А.~Целищева рассмотрены некоторые вопросы 
математической теории риска и предложен новый подход к диверсификации инвестиционных 
портфелей. Работа И.\,А.~Кудрявцевой и А.\,В.~Пантелеева посвящена построению и 
исследованию математической модели, описывающей динамику сильноионизованной плазмы. 
В~статье П.\,П.~Кольцова изучается качество работы ряда алгоритмов сегментации изображений. 
Статья А.\,Н.~Чупрунова и И.~Фазекаша посвящена вероятностному анализу числа без\-оши\-бочных 
блоков при помехоустойчивом кодировании; получены усиленные законы больших чисел для указанных 
величин.

В данном выпуске традиционно присутствует тематика, весьма активно разрабатываемая в течение 
многих лет специалистами ИПИ РАН и МГУ,~--- методы моделирования и управления для 
информационно-телекоммуникационных и вычислительных систем, в частности методы 
теории массового обслуживания. В~статье А.\,И.~Зейфмана с соавторами рассматриваются 
модели обслуживания, описываемые марковскими цепями с непрерывным временем в случае 
наличия катастроф. В~работе М.\,М.~Лери и И.\,А.~Чеплюковой рассматриваются случайные 
графы Интернет-типа, т.\,е.\ графы, степени вершин которых имеют степенные распределения; 
такие задачи находят применение при исследовании глобальных сетей передачи данных. 
Работа Р.\,В.~Разумчика посвящена исследованию систем массового обслуживания специального 
вида~--- с отрицательными заявками и хранением вытесненных заявок.

Ряд статей посвящен развитию перспективных теоретических 
вероятностно-статистических методов, которые находят широкое применение в различных 
задачах информатики и информационных технологий. В~работе В.\,Е.~Бенинга, А.\,К.~Горшенина 
и В.\,Ю.~Королева рассмотрена задача статистической проверки гипотез о числе компонент 
смеси вероятностных распределений, приводится конструкция асимптотически наиболее мощного 
критерия. Результаты этой работы найдут применение в ряде прикладных задач, использующих 
математическую модель смеси вероятностных распределений (в информатике, моделировании 
финансовых рынков, физике турбулентной плазмы и~т.\,д.). В~статье В.\,Ю.~Королева, 
И.\,Г.~Шевцовой и С.\,Я.~Шоргина строится новая, улучшенная оценка точности нормальной 
аппроксимации для пуассоновских случайных сумм; как известно, указанные случайные суммы 
широко используются в качестве моделей многих реальных объектов, в том числе в информатике, 
физике и других прикладных областях. Работа В.\,Г.~Ушакова и Н.\,Г.~Ушакова посвящена 
исследованию ядерной оценки плотности распределения; эти результаты могут применяться, 
в част\-ности, при анализе трафика в телекоммуникационных системах. Серьезные приложения 
в статистике могут получить результаты работы О.\,В.~Шестакова, в которой доказаны оценки 
скорости сходимости распределения выборочного абсолютного медианного отклонения к нормальному 
закону. 

\smallskip

Редакционная коллегия журнала выражает надежду, что данный тематический  выпуск 
будет интересен специалистам в области теории вероятностей и математической статистики 
и их применения к решению задач информатики и информационных технологий.
     
     %\vfill 
     \vspace*{20mm}
     \noindent
     Заместитель главного редактора журнала <<Информатика и её 
применения>>,\\
     директор ИПИ РАН, академик  \hfill
     \textit{И.\,А.~Соколов}\\
     
     \noindent
     Редактор-составитель тематического выпуска,\\
     профессор кафедры математической статистики факультета\\
      вычислительной математики и кибернетики МГУ им.\ М.\,В.~Ломоносова,\\
     ведущий научный сотрудник ИПИ РАН,\\ 
доктор физико-математических наук \hfill
      \textit{В.\,Ю.~Королев}
     
     } }
     }

%%%%%%%%%%%%%%%%%%%%%%%%%%%%%%%%%%%%%%%%%%%%%%%


%\def\stat{rez}
{%\hrule\par
%\vskip 7pt % 7pt
\raggedleft\Large \bf%\baselineskip=3.2ex
Р\,Е\,Ц\,Е\,Н\,З\,И\,И \vskip 17pt
    \hrule
    \par
\vskip 6pt plus 6pt minus 3pt }

%\thispagestyle{headings} %с верхним колонтитулом
%\thispagestyle{myheadings} %с нижним колонтитулом, но в верхнем РЕЦЕНЗИИ

\def\tit{НОВАЯ КНИГА И.\,Н.~СИНИЦЫНА, А.\,С.~ШАЛАМОВА <<ЛЕКЦИИ ПО ТЕОРИИ 
ИНТЕГРИРОВАННОЙ ЛОГИСТИЧЕСКОЙ ПОДДЕРЖКИ>> (М.: ТОРУС ПРЕСС, 2012. 624~с.)}

%1
\def\aut{Д.ф.-м.н., профессор С.\,Я.~Шоргин}

\def\auf{\ }

\def\leftkol{\ % РЕЦЕНЗИИ
}

\def\rightkol{ \ } 

%\def\leftkol{\ } % ENGLISH ABSTRACTS}

%\def\rightkol{\ } %ENGLISH ABSTRACTS}

%\def\leftkol{РЕЦЕНЗИИ}

%\def\rightkol{РЕЦЕНЗИИ}

\titele{\tit}{\aut}{\auf}{\leftkol}{\rightkol}
\vspace*{-18pt}


     \label{st\stat}

     \begin{multicols}{2}
     {\small
     {\baselineskip=10.1pt
     

      В книге представлено системное изложение теоретических основ одного из новейших 
направлений в \mbox{об\-ласти} экономики послепродажного обслуживания изделий наукоемкой 
продукции (ИНП) длительного пользования~--- интегрированной логистической поддержки
(ИЛП). 
{\looseness=1

}

Приведены также результаты новых работ, выполненных в Институте проблем информатики 
Российской академии наук в рамках научного направления <<Информационные технологии и 
анализ сложных сис\-тем>>.
 {%\looseness=1

}
     
      Излагаемые в книге научные подходы позво\-ляют карди\-наль\-но реформировать 
существующие системы производства и эксплуатации ИНП путем создания и внед\-ре\-ния 
методов рационального и оптимального управ\-ле\-ния процессами расходования 
вре\-мен\-н$\acute{\mbox{ы}}$х, 
мате\-ри\-аль\-ных, трудовых и других ресурсов на всех стадиях жизненного цикла изделий (ЖЦИ) по 
критериям экономической целесообразности и эф\-фек\-тив\-ности.
  {\looseness=1

}
    
      В книге приведен краткий обзор причин возник\-новения и
      развития CALS-методологии как основы 
современных международных стандартов по созданию и функционированию глобальных 
ин\-фор\-ма\-ци\-он\-но-ком\-му\-ни\-ка\-ци\-он\-ных систем, ее ключевых возможностей и эффективности 
результатов ее использования. 
Авторы %\linebreak 
предлагают ряд научных обоснований для разработки 
единой теории проектирования и управления систем ИЛП для полноценного использования 
преимуществ %\linebreak
 суще\-ст\-ву\-ющей методологии, определяют \mbox{общую} структурную схему 
комплексной системы <<ИНП-СППО>> и необходимость разработки для ее описания 
гибридных стохастических моделей.
{%\looseness=1

}

%\columnbreak
      
      Книга состоит из пяти частей, где последовательно излагается материал по каждой из 
следующих тем: <<Интегрированная логистическая поддержка>>, <<Теория гибридных 
стохастических систем и компьютерная поддержка исследований и разработок>>, <<Основы 
математического моделирования, анализа и синтеза систем послепродажного обслуживания>>, 
<<Определение и анализ показателей экспортного потенциала ИНП при проектировании>>, 
<<Задачи управления поддержкой послепродажного обслуживания>>, а также 
<<Моделирование инвестиционных процессов ИЛП в условиях неравновесных финансовых 
рынков>>. 
   
      В конце каждой главы приведены выводы и даны вопросы и задания для 
самоконтроля. В~приложениях содержатся основные определения по программам работ по 
анализу ИЛП, логистическим базам данных и компьютерным решениям, эквивалентной статистической 
линеаризации нелинейных преобразований ИЛП, справочный материал, а также развернутые 
уравнения для вероятностных характеристик.


      \def\leftkol{РЕЦЕНЗИИ}

\def\rightkol{РЕЦЕНЗИИ} 

      
      Книга заинтересует широкий круг специалистов и может быть использована научными 
проектными организациями в сфере промышленного производства ИНП. Большое количество 
иллюстраций, примеров и вопросов, обращенных к читателю, позволяет использовать книгу 
также в качестве учебного пособия для студентов и аспирантов машиностроительных, 
транспортных и~других специальностей, а также для самостоятельного изучения. 
{%\looseness=-1

}

Книга 
представляет несомненный интерес для специалистов и студентов в области прикладной 
математики и информатики.
    

}

}
\end{multicols}

%\newpage

%\end{document}

\include{obchak}

%\end{document}



\def\stat{authorsrus}
{%\hrule\par
%\vskip 7pt % 7pt
\raggedleft\Large \bf%\baselineskip=3.2ex
О\,Б\ \ А\,В\,Т\,О\,Р\,А\,Х \vskip 17pt
    \hrule
    \par
\vskip 21pt plus 8pt minus 4pt }


\def\tit{\ }

\def\aut{\ }

\def\auf{\ }

\def\leftkol{\ } % ENGLISH ABSTRACTS}

\def\rightkol{ОБ АВТОРАХ} %ENGLISH ABSTRACTS}

\titele{\tit}{\aut}{\auf}{\leftkol}{\rightkol}
      
            \label{st\stat}



\vspace*{24pt}

\begin{multicols}{2}




\noindent
\textbf{Архипов Олег Петрович} (р.\ 1948)~---
кандидат технических наук, директор Орловского филиала Института проб\-лем информатики
Российской академии наук
%302025, г.Орел, Московское шоссе, д.137

\vspace*{3pt}

\noindent
\textbf{Бирюкова Татьяна Константиновна} (р.\ 1968)~---
кандидат фи\-зи\-ко-ма\-те\-ма\-ти\-че\-ских наук, старший научный сотрудник Института проб\-лем информатики
Российской академии наук

\vspace*{3pt}

\noindent 
\textbf{Бобков  Сергей Геннадьевич} (р.\ 1955)~---
доктор технических наук,  заведующий отделением На\-уч\-но-ис\-сле\-до\-ва\-тель\-ско\-го 
института системных исследований Российской академии наук
%117218, Москва, Нахимовский просп., 36, к.1 

\vspace*{3pt}

\noindent \textbf{Васильев Николай Семенович} (р.\ 1952)~--- доктор 
фи\-зи\-ко-ма\-те\-ма\-ти\-че\-ских наук, профессор, 
МГТУ им.\ Н.\,Э.~Баумана 
%, Москва 105005, 2-я Бауманская ул., д.~5,

\vspace*{3pt}

\noindent
\textbf{Гершкович Максим Михайлович} (р.\ 1968)~---
старший научный сотрудник Института проб\-лем информатики
Российской академии наук

\vspace*{3pt}

\noindent 
\textbf{Дьяченко Юрий Георгиевич} (р.\ 1958)~--- кандидат технических наук, 
старший научный сотрудник Института проб\-лем информатики
Российской академии наук

\vspace*{3pt}

\noindent 
\textbf{Ерошенко Александр Андреевич} (р.\ 1989)~--- аспирант кафедры 
математической статистики факультета вычисли\-тельной математики и кибернетики 
Московского государственного университета им.\ М.\,В.~Ломоносова
%119991, Москва ГСП-1, Ленинские горы, д.\ 1, стр. 52

\vspace*{3pt}
 
\noindent 
\textbf{Захаров Виктор Николаевич} (р.\ 1948)~--- 
доктор технических наук, доцент, ученый секретарь Института проб\-лем информатики
Российской академии наук

\vspace*{3pt}

\noindent
\textbf{Зейфман Александр Израилевич} (р.\ 1954)~---
доктор фи\-зи\-ко-ма\-те\-ма\-ти\-че\-ских наук, профессор, 
заведующий кафедрой Вологодского государственного университета; 
старший научный сотрудник Института проб\-лем информатики
Российской академии наук; главный научный сотрудник ИСЭРТ Российской академии наук

\vspace*{3pt}

\noindent
\textbf{Зыкин Сергей Владимирович} (р.\ 1959)~--- 
доктор технических наук, профессор, заведующий лабораторией Института математики 
им.\ С.\,Л.~Соболева Сибирского отделения Российской академии наук, Новосибирск 
%630090, пр.\ ак.\ Коптюга, 4 

\vspace*{4pt}

\noindent
\textbf{Киреев Владимир Иванович} (р.\ 1938)~---
доктор фи\-зи\-ко-ма\-те\-ма\-ти\-че\-ских наук, профессор Московского 
государственного горного университета
%Адрес: Россия, 119991, г. Москва, Ленинский проспект, д. 6

%\columnbreak

\vspace*{4pt}

\noindent
\textbf{Козеренко Елена Борисовна} (р.\ 1959)~---
кандидат филологических наук, заведующая лабораторией Института проб\-лем информатики
Российской академии наук

\vspace*{4pt}

\noindent
\textbf{Королев Виктор Юрьевич} (р.\ 1954)~--- доктор
фи\-зи\-ко-ма\-те\-ма\-ти\-че\-ских наук, профессор кафедры математической 
статистики факультета вычисли\-тельной математики и кибернетики 
Московского государственного университета; 
ведущий научный сотрудник Института проб\-лем информатики
Российской академии наук

\vspace*{4pt}

\noindent
\textbf{Коротышева Анна Владимировна} (р.\ 1988)~---
старший преподаватель Вологодского государственного университета

\vspace*{4pt}

\noindent 
\textbf{Кун Де Турк} (р.\ 1981)~--- научный сотрудник 
исследовательской группы SMACS факультета телекоммуникаций и обработки информации
Университета Гента, Бельгия
%В-9000 Гент, Бельгия

\vspace*{4pt}

\noindent
\textbf{Лупенцов Олег Сергеевич} (р.\ 1986)~---
аспирант Омского государственного института сервиса
%Омск 644043, ул.\ Певцова 13

\vspace*{4pt}

\noindent
\textbf{Лучко Олег Николаевич} (р.\ 1961)~---
кандидат педагогических наук, профессор, заведующий кафедрой 
Омского государственного института сервиса
%Омск 644043, ул.\ Певцова 13

\vspace*{4pt}

\noindent
\textbf{Малашенко Юрий Евгеньевич} (р.\ 1946)~---
доктор фи\-зи\-ко-ма\-те\-ма\-ти\-че\-ских наук, заведующий сектором 
Вычислительного центра им.\ А.\,А.~Дородницына Российской академии наук
%Адрес: 119333, Москва, ул. Вавилова, 40,

\vspace*{4pt}

\noindent
\textbf{Маньяков Юрий Анатольевич} (р.\ 1984)~---
кандидат технических наук, научный сотрудник Орловского филиала Института проб\-лем информатики
Российской академии наук
%302025, г.Орел, Московское шоссе, д.137

\vspace*{4pt}

\noindent
\textbf{Маренко Валентина Афанасьевна} (р.\ 1951)~---
кандидат технических наук, доцент, старший научный сотрудник 
Института математики им.\ С.\,Л.~Соболева Сибирского отделения Российской академии наук
%Новосибирск 630090, пр. ак. Коптюга, 4 

\vspace*{3pt}

\noindent 
\textbf{Морозов Евсей Викторович} (р.\ 1947)~--- доктор 
фи\-зи\-ко-ма\-те\-ма\-ти\-че\-ских, профессор, ведущий научный сотрудник 
Института прикладных математических исследований Карельского научного центра Российской
академии наук; 
%%185910 Россия, Республика Карелия, г.\ Петрозаводск, ул.\ Пушкинская, 11
профессор Петрозаводского государственного университета, Петрозаводск
%185910 Россия, Республика Карелия, г.\ Петрозаводск, пр.\ Ленина, 33

%\pagebreak

\vspace*{3pt}

\noindent
\textbf{Назарова Ирина Александровна} (р.\ 1966)~---
кандидат фи\-зи\-ко-ма\-те\-ма\-ти\-че\-ских наук, 
научный сотрудник Вычислительного центра им.\ А.\,А.~Дородницына Российской академии наук 
%Адрес: 119333, Москва, ул. Вавилова, 40

\vspace*{3pt}

\noindent
\textbf{Павлов Игорь Валерианович} (р.\ 1945)~--- 
доктор фи\-зи\-ко-ма\-те\-ма\-ти\-че\-ских наук, профессор МГТУ им.\ Н.\,Э.~Баумана 
%Москва 105005, 2-я Бауманская ул., д.~5 

%\pagebreak

\vspace*{3pt}

\noindent 
\textbf{Потахина Любовь Викторовна} (р.\ 1989)~--- аспирантка
Института прикладных математических исследований Карельского научного центра
Российской академии наук; 
%%185910 Россия, Республика Карелия, г.\ Петрозаводск, ул.\ Пушкинская, 11
инженер Петрозаводского государственного университета, Петрозаводск
%185910 Россия, Республика Карелия, г.\ Петрозаводск, пр.\ Ленина, 33

\vspace*{3pt}

\noindent 
\textbf{Рождественский Юрий Владимирович} (р.\ 1952)~--- 
кандидат технических наук, заведующий сектором Института проб\-лем информатики
Российской академии наук

\vspace*{3pt}

\noindent 
\textbf{Синицын Игорь Николаевич} (р.\ 1940)~--- доктор технических наук,
профессор, заслуженный деятель\linebreak\vspace*{-12pt}

\columnbreak

\noindent
 науки РФ, заведующий отделом Института проб\-лем информатики
Российской академии наук

\vspace*{7pt}


\noindent
\textbf{Сиротинин Денис Олегович} (р.\ 1984)~---
кандидат технических наук, научный сотрудник Орловского филиала Института проб\-лем информатики
Российской академии наук
%302025, г.Орел, Московское шоссе, д.137

\vspace*{7pt}

%\columnbreak

\noindent 
\textbf{Соколов  Игорь Анатольевич} (р.\ 1954)~--- академик (действительный член) Российской 
академии наук, доктор технических наук, директор Института проб\-лем информатики
Российской академии наук

\vspace*{7pt}

\noindent
\textbf{Степченков Юрий Афанасьевич} (р.\ 1951)~---
кандидат технических наук, заведующий отделом Института проб\-лем информатики
Российской академии наук

\vspace*{7pt}

\noindent
\textbf{Сурков Алексей Викторович} (р.\ 1978)~--- 
старший научный сотрудник На\-уч\-но-ис\-сле\-до\-ва\-тель\-ско\-го 
института системных исследований Российской академии наук
%117218, Москва, Нахимовский просп., 36, к.1 

\vspace*{7pt}

\noindent 
\textbf{Шестаков Олег Владимирович} (р.\ 1976)~--- доктор 
фи\-зи\-ко-ма\-те\-ма\-ти\-че\-ских, доцент кафедры математической статистики 
факультета вычисли\-тельной математики и кибернетики Московского 
государственного университета им.\ М.\,В.~Ломоносова; 
%119991, Москва ГСП-1, Ленинские горы, д.\ 1, стр. 52
старший научный сотрудник Института проб\-лем информатики
Российской академии наук
%, Москва 119333, ул. Вавилова, д.~44, корп.~2

\vspace*{7pt}

\noindent 
\textbf{Шоргин Сергей Яковлевич} (р.\ 1952.)~--- доктор
фи\-зи\-ко-ма\-те\-ма\-ти\-че\-ских наук, профессор, заместитель директора Института 
проб\-лем информатики Российской академии наук





%%%%%%%%%%%%%%%%%%%%%%%%%%%%%%%%%%%%%%%%%%%%%%%%%%%%%%%%%%%%%%%%%%%%%%%%%%%%%%%




%\def\rightkol{ОБ АВТОРАХ}
%\def\leftkol{ОБ АВТОРАХ}

 \label{end\stat}





%\def\leftfootline{\small{\textbf{\thepage}
%\hfill ИНФОРМАТИКА И ЕЁ ПРИМЕНЕНИЯ\ \ \ том~7\ \ \ выпуск~1\ \ \ 2013}
%}%
% \def\rightfootline{\small{ИНФОРМАТИКА И ЕЁ ПРИМЕНЕНИЯ\ \ \ том~7\ \ \ выпуск~1\ \ \ 2013
%\hfill \textbf{\thepage}}}


%\thispagestyle{myheadings}



\end{multicols}

\newpage


%%\vspace*{-48pt}
\begin{center}\LARGE
\textit{About Authors}
\end{center}

\thispagestyle{empty}
\def\tit{\ }

\def\aut{\ }

\def\auf{\ }


\def\leftkol{ABOUT AUTHORS}

\def\rightkol{ABOUT AUTHORS}

\vspace*{-18pt}

\titele{\tit}{\aut}{\auf}{\leftkol}{\rightkol}

%\vspace*{36pt}

\def\rightmark{{\noindent\hbox to \textwidth{\hfill\small ABOUT AUTHORS
%\hfill \large\bf\thepage
}}}
\def\leftmark{{\noindent\parbox{\textwidth}{
%\raggedleft\large\bf\thepage \hfill
\small\textrm{ABOUT AUTHORS}\hfill}}}


\def\leftfootline{\small{\textbf{\thepage}
\hfill ИНФОРМАТИКА И ЕЁ ПРИМЕНЕНИЯ\ \ \ том~6\ \ \ выпуск~2\ \ \ 2012}
}%
 \def\rightfootline{\small{ИНФОРМАТИКА И ЕЁ ПРИМЕНЕНИЯ\ \ \ том~6\ \ \ выпуск~2\ \ \ 2012
\hfill \textbf{\thepage}}}


\begin{multicols}{2}

\noindent
\textbf{Agalarov Yaver M.} (b.\ 1952)~--- Candidate of Science (PhD)
in technology, 
leading scientist, Institute of Informatics Problems, Russian Academy of Sciences

\vspace*{5pt}


  \noindent
\textbf{Bosov Alexey V.} (b.\ 1969)~--- Doctor of Science in technology, Head of
Laboratory, Institute of Informatics Problems, Russian Academy of Sciences

\vspace*{5pt}


\noindent
\textbf{Dulin Sergey K.} (b.\ 1950)~--- Doctor of Science in technology, 
professor, senior scientist, Institute of Informatics Problems, Russian Academy of Sciences

\vspace*{5pt}

\noindent
\textbf{Gorshenin Andrey K.}~--- (b.\ 1986)~--- Candidate of Science (PhD)
in physics and mathematics,
senior scientist, Institute of Informatics Problems, Russian Academy of Sciences

\vspace*{5pt}

\noindent
\textbf{Kalenov Nikolay E.}  (b.\ 1945)~--- Doctor of Science in technology,
professor, Director, Library for Natural Sciences,  Russian Academy of Sciences 

\vspace*{5pt}

\noindent
\textbf{Kalinichenko Leonid A.} (b.\ 1937)~--- Doctor of Science in physics and mathematics, 
professor, Honored scientist of RF, 
Head of Laboratory, Institute of Informatics Problems, Russian Academy of Sciences 

\vspace*{5pt}

\noindent
\textbf{Karpov Alexey A.} (b.\ 1978)~--- Candidate of Science (PhD) in technology, 
senior scientist, St.\ Petersburg Institute for
Informatics and Automation,  Russian Academy of Sciences

\vspace*{5pt}

\noindent
\textbf{Kuznetsov Igor P.} (b.\ 1938)~--- Doctor of Science in technology, 
professor, principal scientist, Institute of Informatics Problems, Russian Academy of Sciences

\vspace*{5pt}


\noindent
\textbf{Markova Natalia A.} (b.\ 1950)~--- Candidate of Science (PhD) in
physics and mathematics, leading scientist,  
Institute of Informatics Problems, Russian Academy of Sciences

\vspace*{5pt}

\noindent
\textbf{Nikolaev Andrey V.} (b.\ 1985)~--- Candidate of Science (PhD) in technology, 
senior lecturer, Tchaikovsky Technological Institute, Branch of the Izhevsk State Technical 
University

\vspace*{6pt}

\noindent
\textbf{Pavlov Igor V.} (b.\ 1945)~---  Doctor of Science in physics and mathematics,
professor, Bauman Moscow State Technical University

\vspace*{6pt}

%\columnbreak

\noindent
\textbf{Rozenberg Igor N.} (b.\ 1965)~--- Doctor of Science in technology, 
First Deputy Director General, Research \& Design Institute for Information 
Technology, Signalling and Telecommunications on Railway Transport (JSC NIIAS)

\vspace*{6pt}


\noindent
\textbf{Semenov Konstantin K.} (b.\ 1986)~--- MPhil, 
associate professor, St.\ Petersburg State Polytechnical University

\vspace*{6pt}

\noindent
\textbf{Sharnin Mikhail M.} (b.\ 1959)~--- Candidate of Science (PhD) 
in technology, senior scientist, Institute of Informatics Problems, Russian Academy of Sciences

\vspace*{6pt}

\noindent 
\textbf{Shestakov Oleg V.} (b.\ 1976)~--- Candidate of Science (PhD) in physics and mathematics,
associate professor, Department of Mathematical Statistics, Faculty of Computational Mathematics and Cybernetics,
M.\,V.~Lomonosov Moscow State University; senior scientist, Institute of Informatics Problems, 
Russian Academy of Sciences

\vspace*{6pt}

\noindent
\textbf{Stupnikov Sergey A.} (b.\ 1978)~--- Candidate of Science (PhD) in technology, 
senior scientist, Institute of Informatics Problems, Russian Academy of Sciences 

\vspace*{6pt}

\noindent
\textbf{Umansky Vladimir I.} (b.\ 1954)~--- Candidate of Science (PhD) in technology, 
Director General, ``IntechGeoTrans'' Closed Joint Stock Company

\vspace*{6pt}

\noindent
\textbf{Zhevnerchuk Dmitry V.} (b.\ 1978)~--- Candidate of Science (PhD) in technology, 
associate professor, Tchaikovsky Technological Institute, Branch of the Izhevsk State 
Technical University

%\vspace*{6pt}

\def\leftfootline{\small{\textbf{\thepage}
\hfill ИНФОРМАТИКА И ЕЁ ПРИМЕНЕНИЯ\ \ \ том~6\ \ \ выпуск~2\ \ \ 2012}
}%
 \def\rightfootline{\small{ИНФОРМАТИКА И ЕЁ ПРИМЕНЕНИЯ\ \ \ том~6\ \ \ выпуск~2\ \ \ 2012
\hfill \textbf{\thepage}}}



%\thispagestyle{myheadings}

\end{multicols}
\newpage

%   \vspace*{-48pt}

\begin{center}
\vspace*{6pt}
\mbox{%
\epsfxsize=53.502mm
\epsfbox{foto-1.eps}
}
\end{center}

\vspace*{6pt} %Академик


   \begin{center}
\fbox{\Large\textbf{Профессор Игорь Алексеевич Ушаков}}\\[12pt]
\textbf{\large 22.01.1935--27.02.2015}
   \end{center}


   %\vspace*{2.5mm}

   \vspace*{5mm}

   \thispagestyle{empty}

%\

%\vspace*{-12pt}


Редакционный совет и редакционная коллегия журнала <<Информатика и~её применения>> с~глубоким прискорбием извещают, что 27~февраля 2015~г.\ после тяжелой
и~продолжительной болезни скончался Игорь Алексеевич Ушаков~--- доктор технических наук, профессор, член редколлегии журнала <<Информатика и ее применения>>.

Игорь Алексеевич Ушаков окончил Московский авиационный институт, в~1963~г.\ защитил кандидатскую, а~в~1968~г.~--- докторскую диссертацию. С~1958 по 1989~гг.\ работал в~ряде научно-исследовательских организаций СССР, в~том числе руководил отделами в~НИИ АА и~ВЦ АН СССР; с 1969 по 1989 гг. преподавал в~МФТИ (был профессором, а~затем заведующим кафедрой) и~в~МЭИ. С~1989~г.~---- в~США: являлся профессором университета Дж.\ Вашингтона, университета Дж.\ Мэйсона и~Калифорнийского университета, сотрудником компаний MCI, Qualcomm и Hughes.

И.\,А.~Ушаков с момента основания журнала <<Надежность и~контроль качества>> был заместителем ответственного редактора, а~затем на протяжении многих лет членом редколлегии. В~2006~г.\ основал электронный международный журнал ``Reliability: Theory \& Application'', главным редактором которого оставался до конца жизни.

Учебниками и справочниками по теории надежности, написанными И.\,А.~Ушаковым, пользовались и~пользуются несколько поколений ученых и~специалистов в~разных странах мира.

Игорь Алексеевич всегда уделял огромное внимание работе с~молодежью; более~50 его учеников защитили докторские и~кандидатские диссертации.

И.\,А.~Ушаков вел активную научно-про\-све\-ти\-тель\-скую деятельность. В~частности, он был одним из организаторов и~руководителей Московского кабинета качества и~надежности при Политехническом музее (целью этого Кабинета было оказание консультаций работникам промышленных предприятий и~чтение курсов лекций для инженеров, занимающихся проблемой надежности). Находясь в~США, И.\,А.~Ушаков создал международный ин\-тер\-нет-фо\-рум им.\ Б.\,В.~Гнеденко, объединивший около~400~видных специалистов по приложениям теории вероятностей и~математической статистики, преимущественно в~об\-ласти теории надежности и~анализа риска, из десятков стран мира; коллективным членов этого Форума является и~наш журнал. Цели Форума~--- содействие контактам между специалистами из разных стран, организация обмена профессиональными 
новостями и~информацией (новые публикации, предстоящие события и~др.). Также необходимо отметить большое число на\-уч\-но-по\-пу\-ляр\-ных работ, опубликованных И.\,А.~Ушаковым.

И.\,А.~Ушаков обладал большим личным обаянием, имел широкий круг интересов. Все знавшие И.\,А.~Ушакова всегда будут помнить его как замечательного ученого и~прекрасного человека.

\bigskip

Редакционный совет и редакционная коллегия журнала <<Информатика и~её применения>> 
выражают глубокие соболезнования родным и близким покойного, всем, кто его знал и~работал с~ним.


%\vspace*{-60pt} {\small
{\baselineskip=9.1pt
\section*{Правила подготовки рукописей статей для публикации в журнале
<<Информатика и её применения>>}

\thispagestyle{empty}

 Журнал <<Информатика и её применения>> публикует
теоретические, обзорные и дискуссионные статьи, посвященные научным
исследованиям и разработкам в области информатики и ее приложений. Журнал
издается на русском языке. По специальному решению редколлегии отдельные статьи,
в виде исключения, могут печататься на английском языке.
Тематика журнала охватывает следующие направления:
\begin{itemize}
\item теоретические основы информатики; %\\[-13.5pt]
\item математические методы исследования сложных систем и процессов; %\\[-13.5pt]
\item информационные системы и сети; %\\[-13.5pt]
\item информационные технологии; %\\[-13.5pt]
\item архитектура и программное
обеспечение вычислительных комплексов и сетей.
\end{itemize}
\begin{enumerate}
\item В журнале печатаются результаты, ранее не
опубликованные и не предназначенные к одновременной публикации в других
изданиях. Публикация не должна нарушать закон об авторских правах. Направляя
свою рукопись в редакцию, авторы автоматически передают учредителям и
редколлегии неисключительные права на издание данной статьи на русском языке и
на ее распространение в России и за рубежом. При этом за авторами сохраняются
все права как собственников данной рукописи. В связи с этим авторами должно
быть представлено в редакцию письмо в следующей форме:
Соглашение о передаче права на публикацию:

\textit{<<Мы, нижеподписавшиеся, авторы рукописи <<$\qquad\qquad$>>, передаем
учредителям и редколлегии журнала <<Информатика и её применения>>
неисключительное право опубликовать данную рукопись статьи на русском языке как
в печатной, так и в электронной версиях журнала. Мы подтверждаем, что данная
публикация не нарушает авторского права других лиц или организаций. Подписи
авторов: (ф.\,и.\,о., дата, адрес)>>.}

Указанное соглашение может быть представлено 
как в бумажном виде, так и в виде отсканированной копии (с подписями авторов).


Редколлегия вправе запросить у авторов экспертное заключение о возможности
опубликования представленной статьи в открытой печати. %\\[-13.5pt]
\item Статья
подписывается всеми авторами. На отдельном листе представляются данные автора
(или всех авторов): фамилия, полные имя и отчество, телефон, факс, e-mail,
почтовый адрес. Если работа выполнена несколькими авторами, указывается фамилия
одного из них, ответственного за переписку с редакцией. %\\[-13.5pt]
\item Редакция журнала
осуществляет самостоятельную экспертизу присланных статей. Возвращение рукописи
на доработку не означает, что статья уже принята к печати. Доработанный вариант
с ответом на замечания рецензента необходимо прислать в редакцию. %\\[-13.5pt]
\item Решение
редакционной коллегии о принятии статьи к печати или ее отклонении сообщается
авторам. Редколлегия не обязуется направлять рецензию авторам отклоненной
статьи. %\\[-13.5pt]
\item Корректура статей высылается авторам для просмотра. Редакция
просит авторов присылать свои замечания в кратчайшие сроки. %\\[-13.5pt]
\item При
подготовке рукописи в MS Word рекомендуется использовать следующие настройки.
Параметры страницы: формат~--- А4; ориентация~--- книжная; поля (см): внутри~---
2,5, снаружи~--- 1,5, сверху~--- 2, снизу~--- 2, от края до нижнего
колонтитула~--- 1,3. Основной текст: стиль~--- <<Обычный>>: шрифт Times New
Roman, размер 14~пунктов, абзацный отступ~--- 0,5~см, 1,5 интервала,
выравнивание~--- по ширине. Рекомендуемый объем рукописи~--- не свыше
25~страниц указанного формата. Ознакомиться с шаблонами, содержащими примеры
оформления, можно по адресу в Интернете:
\textsf{http://www.ipiran.ru/journal/template.doc}.
\item К рукописи, предоставляемой в 2-х
экземплярах, обязательно прилагается электронная версия статьи (как правило, в
форматах MS WORD (.doc) или \LaTeX\ (.tex), а также~--- дополнительно~--- в
формате .pdf) на дискете, лазерном диске или по электронной почте. Сокращения
слов, кроме стандартных, не применяются. Все страницы рукописи должны быть
пронумерованы. %\\[-13.5pt]
\item Статья должна содержать следующую информацию на русском и
английском языках: название, Ф.И.О. авторов, места работы авторов и их
электронные адреса, подробные сведения об авторах, оформленные в соответствии с форматом, 
определяемым файлами {\sf http://www.ipiran.ru/journal/issues/2011\_05\_01/authors.asp} и 
{\sf http://www.ipiran.ru/journal/issues/2011\_01\_eng/authors.asp},
аннотация (не более 100~слов), ключевые слова. Ссылки на
литературу в тексте статьи нумеруются (в квадратных скобках) и располагаются в
порядке их первого упоминания. В~списке литературы не должно быть позиций, на которые нет ссылки в тексте статьи.
Все фамилии авторов, заглавия статей, названия
книг, конференций и~т.\,п.\ даются на языке оригинала, если этот язык
использует кириллический или латинский алфавит. %\\[-13.5pt]
\item Присланные в редакцию материалы авторам не возвращаются.
\item При отправке файлов по электронной
почте просим придерживаться следующих правил:
\begin{itemize}
\item указывать в поле subject (тема) название журнала и фамилию автора; %\\[-13.5pt]
\item использовать attach (присоединение); %\\[-13.5pt]
\item в случае больших объемов информации возможно
использование общеизвестных архиваторов (ZIP, RAR); %\\[-13.5pt]
\item в состав электронной версии статьи должны входить: файл, содержащий текст статьи, и файл(ы),
содержащий(е) иллюстрации. %\\[-13.5pt]
\end{itemize}
\item Журнал <<Информатика и её применения>> является некоммерческим изданием. 
Плата за публикацию с авторов не взимается, гонорар авторам не выплачивается.
\end{enumerate}
\thispagestyle{empty}
\textbf{Адрес редакции:} Москва 119333,
ул.~Вавилова, д.~44, корп.~2, ИПИ РАН\\
\hphantom{\textbf{Адрес редакции:} }Тел.: +7 (499) 135-86-92\ \
Факс:  +7 (495) 930-45-05\ \  E-mail:   rust@ipiran.ru }
}


%\end{document}

%\include{IPPM-25}

\def\stat{cont}
{%\hrule\par
%\vskip 7pt % 7pt
\raggedleft\Large \bf%\baselineskip=3.2ex
А\,В\,Т\,О\,Р\,С\,К\,И\,Й\ \ У\,К\,А\,З\,А\,Т\,Е\,Л\,Ь\ \ З\,А\ \ 2\,0\,1\,0 г. \vskip 17pt
    \hrule
    \par
\vskip 21pt plus 6pt minus 3pt }

\label{st\stat}

\def\tit{\ }

\def\aut{\ }
\def\auf{\ }

\def\leftkol{\ } % ENGLISH ABSTRACTS}

\def\rightkol{\ } %АВТОРСКИЙ УКАЗАТЕЛЬ ЗА 2010 г.} %ENGLISH ABSTRACTS}

\titele{\tit}{\aut}{\auf}{\leftkol}{\rightkol}

\vspace*{-12pt}

{\tabcolsep=3pt
\begin{tabular}{p{388pt}rr}
&\textbf{Выпуск} & \textbf{Стр.}\\[6pt]
\hangindent=23pt\noindent\textbf{Арутюнян~А.\,Р.} Моделирование влияния деформаций отпечатков пальцев на 
точность\linebreak
\vspace*{-12pt}\\
\hspace*{23pt}дактилоскопической идентификации$\dotfill$&1&51\\
\hangindent=23pt\noindent\textbf{Архипов~О.\,П., Зыкова~З.\,П.} Интеграция гетерогенной информации о цветных 
пикселях\linebreak
\vspace*{-12pt}\\
\hspace*{23pt}и их цветовосприятии$\dotfill$&4&15\\
\hangindent=23pt\noindent\textbf{Баранов~С.\,И., Френкель~С.\,Л., Захаров~В.\,Н.} Полуформальная верификация 
цифрового устройства с конвейером, основанная на использовании алгоритмических машин\linebreak
\vspace*{-12pt}\\
\hspace*{23pt}состояния$\dotfill$&4&49\\
\textbf{Бекетова~И.\,В.} см.~Каратеев~С.\,Л.&&\\
\textbf{Белоусов~В.\,В.} см.~Синицын~И.\,Н.&&\\
\hangindent=23pt\noindent\textbf{Бенинг~В.\,Е., Королев~Р.\,А.} О предельном поведении мощностей критериев в 
случае\linebreak
\vspace*{-12pt}\\
\hspace*{23pt}распределения Лапласа$\dotfill$&2&63\\
\hangindent=23pt\noindent\textbf{Бенинг~В.\,Е., Сипина~А.\,В.} Асимптотическое разложение для мощности 
критерия,\linebreak
\vspace*{-12pt}\\
\hspace*{23pt}основанного на выборочной медиане, в случае распределения Лапласа$\dotfill$&1&18\\
\textbf{Бондаренко~А.\,В.} см.~Каратеев~С.\,Л.&&\\
\hangindent=23pt\noindent\textbf{Бородина~А.\,В., Морозов~Е.\,В.} Об оценивании асимптотики вероятности 
большого\linebreak
\vspace*{-12pt}\\
\hspace*{23pt}уклонения стационарной регенеративной очереди с одним прибором$\dotfill$&3&29\\
\hangindent=23pt\noindent\textbf{Бунтман~Н.\,В., Минель~Ж.-Л., Ле~Пезан~Д., Зацман~И.\,М.} Типология и 
компьютерное\linebreak
\vspace*{-12pt}\\
\hspace*{23pt}моделирование трудностей перевода$\dotfill$&3&77\\
\textbf{Визильтер~Ю.\,В.} см.~Каратеев~С.\,Л.&&\\
\hangindent=23pt\noindent\textbf{Гавриленко~С.\,В.} Оценки скорости сходимости распределений случайных сумм с 
безгранично делимыми индексами к нормальному закону$\dotfill$&4&81\\
\hangindent=23pt\noindent\textbf{Григорьева~М.\,Е., Шевцова~И.\,Г.} Уточнение неравенства 
Каца--Берри--Эссеена$\dotfill$&2&75\\
\hangindent=23pt\noindent\textbf{Грушо~А.\,А., Грушо~Н.\,А., Тимонина~Е.\,Е.} Поиск конфликтов в политиках 
безопасности: модель случайных графов$\dotfill$&3&38\\
\textbf{Грушо~Н.\,А.} см.~Грушо~А.\,А.&&\\
\hangindent=23pt\noindent\textbf{Гудков~В.\,Ю.} Математические модели изображения отпечатка пальца на основе 
описания линий$\dotfill$&1&58\\
\textbf{Гуртов~А.\,В.} см.~Лукьяненко~А.\,С.&&\\
\textbf{Желтов~С.\,Ю.} см.~Каратеев~С.\,Л.&&\\
\hangindent=23pt\noindent\textbf{Захаров~А.\,А., Серебряков~В.\,А.} Система управления электронной библиотекой 
LibMeta$\dotfill$&4&2\\
\textbf{Захаров~В.\,Н.} см.~Баранов~С.\,И.&&\\
\textbf{Захарова~Т.\,В.} см.~Матвеева~С.\,С.&&\\
\hangindent=23pt\noindent\textbf{Зацаринный~А.\,А., Чупраков~К.\,Г.} Некоторые аспекты выбора технологии для 
постро-\linebreak
\vspace*{-12pt}\\
\hspace*{23pt}ения систем отображения информации ситуационного центра$\dotfill$&3&59\\
\textbf{Зацман~И.\,М.} см.~Бунтман~Н.\,В.&&\\
\hangindent=23pt\noindent\textbf{Зейфман~А.\,И., Коротышева~А.\,В., Сатин~Я.\,А., Шоргин~С.\,Я.} Об 
устойчивости нестаци-\linebreak
\vspace*{-12pt}\\
\hspace*{23pt}онарных систем обслуживания с катастрофами$\dotfill$&3&9\\
\textbf{Зыкова~З.\,П.} см.~Архипов~О.\,П.&&\\
\hangindent=23pt\noindent\textbf{Илюшин~Г.\,Я., Соколов~И.\,А.} Организация управляемого доступа пользователей 
к\linebreak
\vspace*{-12pt}\\
\hspace*{23pt}разнородным ведомственным информационным ресурсам$\dotfill$&1&24\\
\hangindent=23pt\noindent\textbf{Кавагучи~Ю., Ульянов~В.\,В., Фуджикоши~Я.} Приближения для статистик, 
описывающих\linebreak
\vspace*{-12pt}\\
\hspace*{23pt}геометрические свойства данных большой размерности, с оценками 
ошибок$\dotfill$&1&12\\
\hangindent=23pt\noindent\textbf{Каратеев~С.\,Л., Бекетова~И.\,В., Ососков~М.\,В., Князь~В.\,А., 
Визильтер~Ю.\,В., Бондаренко~А.\,В., Желтов~С.\,Ю.} Автоматизированный контроль 
качества цифровых\linebreak
\vspace*{-12pt}\\
\hspace*{23pt}изображений для персональных документов$\dotfill$&1&65\\
\end{tabular}
}

\pagebreak

\def\leftkol{АВТОРСКИЙ УКАЗАТЕЛЬ ЗА 2010 г.} % ENGLISH ABSTRACTS}

\def\rightkol{АВТОРСКИЙ УКАЗАТЕЛЬ ЗА 2010 г.} %ENGLISH ABSTRACTS}

{\tabcolsep=3pt
\begin{tabular}{p{388pt}rr}
&\textbf{Выпуск} & \textbf{Стр.}\\[3pt]
\hangindent=23pt\noindent\textbf{Козеренко~Е.\,Б.} Лингвистические фильтры в статистических моделях машинного\linebreak
\vspace*{-12pt}\\
\hspace*{23pt}перевода$\dotfill$&2&83\\
\hangindent=23pt\noindent\textbf{Козеренко~Е.\,Б., Кузнецов~И.\,П.} Когнитивно-лингвистические представления в 
систе-\linebreak
\vspace*{-12pt}\\
\hspace*{23pt}мах обработки текстов$\dotfill$&3&69\\
\textbf{Князь~В.\,А.} см.~Каратеев~С.\,Л.&&\\
\hangindent=23pt\noindent\textbf{Колесников~А.\,В., Солдатов~С.\,А.} Алгоритм координации для гибридной 
интеллектуальной системы решения сложной задачи оперативно-производственного\linebreak
\vspace*{-12pt}\\
\hspace*{23pt}планирования$\dotfill$&4&61\\
\hangindent=23pt\noindent\textbf{Коновалов~М.\,Г.} О планировании потоков в системах вычислительных 
ресурсов$\dotfill$&2&3\\
\textbf{Конушин~А.\,С.} см.~Конушин~В.\,С.&&\\
\hangindent=23pt\noindent\textbf{Конушин~В.\,С., Кривовязь~Г.\,Р., Конушин~А.\,С.} Алгоритм распознавания людей 
в видео-\linebreak
\vspace*{-12pt}\\
\hspace*{23pt}последовательности по одежде$\dotfill$&1&74\\
\textbf{Корепанов~Э.\, Р.} см.~Синицын~И.\,Н.&&\\
\textbf{Королев~В.\,Ю.} см.~Соколов~И.\,А.&&\\
\textbf{Королев~Р.\,А.} см.~Бенинг~В.\,Е.&&\\
\textbf{Коротышева~А.\,В.} см.~Зейфман~А.\,И.&&\\
\hangindent=23pt\noindent\textbf{Кривенко~М.\,П.} Непараметрическое оценивание элементов байесовского 
клас\-си-\linebreak
\vspace*{-12pt}\\
\hspace*{23pt}фикатора$\dotfill$&2&13\\
\textbf{Кривовязь~Г.\,Р.} см.~Конушин~В.\,С.&&\\
\textbf{Крылов~А.\,С.} см.~Павельева~Е.\,А.&&\\
\hangindent=23pt\noindent\textbf{Крылов~В.\,А.} Моделирование и классификация многоканальных дистанционных\linebreak
\vspace*{-12pt}\\
\hspace*{23pt}изображений с использованием копул$\dotfill$&4&34\\
\hangindent=23pt\noindent\textbf{Крючин~О.\,В.} Разработка параллельных эвристических алгоритмов подбора 
весовых\linebreak
\vspace*{-12pt}\\
\hspace*{23pt}коэффициентов искусственной нейтронной сети$\dotfill$&2&53\\
\hangindent=23pt\noindent\textbf{Кудрявцев~А.\,А., Шоргин~С.\,Я.} Байесовские модели массового обслуживания и 
надеж-\linebreak
\vspace*{-12pt}\\
\hspace*{23pt}ности: характеристики среднего числа заявок в системе $M\vert M \vert 1\vert 
\infty$$\dotfill$&3&16\\
\hangindent=23pt\noindent\textbf{Кузнецов~А.\,А.} Связь между временными и структурно-топологическими 
характери-\linebreak
\vspace*{-12pt}\\
\hspace*{23pt}стиками диаграмм ритма сердца здоровых людей$\dotfill$&4&39\\
\textbf{Кузнецов~И.\,П.} см.~Козеренко~Е.\,Б.&&\\
\textbf{Ле~Пезан~Д.} см.~Бунтман~Н.\,В.&&\\
\hangindent=23pt\noindent\textbf{Лукьяненко~А.\,С., Морозов~Е.\,В., Гуртов~А.\,В.} Анализ сетевого протокола с общей 
функ-\linebreak
\vspace*{-12pt}\\
\hspace*{23pt}цией расширения окна передачи сообщения при конфликтах$\dotfill$&2&46\\
\hangindent=23pt\noindent\textbf{Лямин~О.\,О.} О предельном поведении мощностей критериев в случае обобщенного\linebreak
\vspace*{-12pt}\\
\hspace*{23pt}распределения Лапласа$\dotfill$&3&47\\
\hangindent=23pt\noindent\textbf{Маркин~А.\,В., Шестаков~О.\,В.} Асимптотики оценки риска при пороговой 
обработке\linebreak
\vspace*{-12pt}\\
\hspace*{23pt}вейвлет-вейглет коэффициентов в задаче томографии$\dotfill$&2&36\\
\hangindent=23pt\noindent\textbf{Матвеева~С.\,С., Захарова~Т.\,В.} Сети массового обслуживания с наименьшей 
длиной\linebreak
\vspace*{-12pt}\\
\hspace*{23pt}очереди$\dotfill$&3&22\\
\hangindent=23pt\noindent\textbf{Матюшенко~С.\,И.} Стационарные характеристики двухканальной системы 
обслужива-\linebreak
\vspace*{-12pt}\\
\hspace*{23pt}ния с переупорядочиванием заявок и распределениями фазового типа$\dotfill$&4&68\\
\textbf{Минель~Ж.-Л.} см.~Бунтман~Н.\,В.&&\\
\textbf{Морозов~Е.\,В.} см.~Бородина~А.\,В.&&\\
\textbf{Морозов~Е.\,В.} см.~Лукьяненко~А.\,С.&&\\
\textbf{Ососков~М.\,В.} см.~Каратеев~С.\,Л.&&\\
\hangindent=23pt\noindent\textbf{Павельева~Е.\,А., Крылов~А.\,С.} Поиск и анализ ключевых точек радужной 
оболочки\linebreak
\vspace*{-12pt}\\
\hspace*{23pt}глаза методом преобразования Эрмита$\dotfill$&1&79\\
\textbf{Печинкин~А.\,В.} см.~Френкель~С.\,Л.,&&\\
\hangindent=23pt\noindent\textbf{Протасов~В.\,И.} Составление субъективного портрета с использованием 
эволюционно-\linebreak
\vspace*{-12pt}\\
\hspace*{23pt}го морфинга и квалиметрия метода$\dotfill$&1&83\\
\hangindent=23pt\noindent\textbf{Рудаков~К.\,В., Торшин~И.\,Ю.} Вопросы разрешимости задачи распознавания 
вторичной\linebreak
\vspace*{-12pt}\\
\hspace*{23pt}структуры белка$\dotfill$&2&25\\
\textbf{Сатин~Я.\,А.} см.~Зейфман~А.\,И.&&\\
\hangindent=23pt\noindent\textbf{Сейфуль-Мулюков~Р.\,Б.} Нефть как носитель информации о своем 
происхождении,\linebreak
\vspace*{-12pt}\\
\hspace*{23pt}структуре и эволюции$\dotfill$&1&41\\
\end{tabular}
}

{\tabcolsep=3pt
\begin{tabular}{p{388pt}rr}
&\textbf{Выпуск} & \textbf{Стр.}\\[6pt]
\textbf{Семендяев~Н.\,Н.} см.~Синицын~И.\,Н.&&\\
\textbf{Серебряков~В.\,А.} см.~Захаров~А.\,А.&&\\
\textbf{Синицын~В.\,И.} см.~Синицын~И.\,Н.&&\\
\hangindent=23pt\noindent\textbf{Синицын~И.\,Н., Синицын~В.\,И., Корепанов~Э.\, Р., Белоусов~В.\,В., 
Семендяев~Н.\,Н.} Оперативное построение информационных моделей движения полюса 
Земли\linebreak
\vspace*{-12pt}\\
\hspace*{23pt}методами линейных и линеаризованных фильтров$\dotfill$&1&2\\
\textbf{Сипина~А.\,В.} см.~Бенинг~В.\,Е.&&\\
\hangindent=23pt\noindent\textbf{Соколов~И.\,А.} О работах заслуженного деятеля науки Российской Федерации 
И.\,Н.~Синицына в области информационных технологий и автоматизации (к 70-летию\linebreak
\vspace*{-12pt}\\
\hspace*{23pt}со дня рождения)$\dotfill$&3&84\\
\textbf{Соколов~И.\,А.} см.~Илюшин~Г.\,Я.&&\\
\hangindent=23pt\noindent\textbf{Соколов~И.\,А., Королев~В.\,Ю.} Предисловие$\dotfill$&2&2\\
\textbf{Солдатов~С.\,А.} см.~Колесников~А.\,В.&&\\
\hangindent=23pt\noindent\textbf{Степанов~С.\,Ю.} Использование координатного метода фрагментации 
коммутаторной\linebreak
\vspace*{-12pt}\\
\hspace*{23pt}нейронной сети для сокращения трафика$\dotfill$&2&57\\
\textbf{Тимонина~Е.\,Е.} см.~Грушо~А.\,А.&&\\
\textbf{Торшин~И.\,Ю.} см.~Рудаков~К.\,В.&&\\
\textbf{Ульянов~В.\,В.} см.~Кавагучи~Ю.&&\\
\textbf{Фазекаш~И.} см.~Чупрунов~А.\,Н.&&\\
\textbf{Френкель~С.\,Л.} см.~Баранов~С.\,И.&&\\
\hangindent=23pt\noindent\textbf{Френкель~С.\,Л., Печинкин~А.\,В.} Оценка времени самовосстановления в 
цифровых\linebreak
\vspace*{-12pt}\\
\hspace*{23pt}системах после сбоев, вызываемых переходными помехами$\dotfill$&3&2\\
\textbf{Фуджикоши~Я.} см.~Кавагучи~Ю.&&\\
\hangindent=23pt\noindent\textbf{Цискаридзе~А.\,К.} Математическая модель и метод восстановления позы человека 
по\linebreak
\vspace*{-12pt}\\
\hspace*{23pt}стереопаре силуэтных изображений$\dotfill$&4&27\\
\hangindent=23pt\noindent\textbf{Чупраков~К.\,Г.} К вопросу о размещении коллективных средств отображения в 
ситуа-\linebreak
\vspace*{-12pt}\\
\hspace*{23pt}ционном зале с заданными параметрами$\dotfill$&4&89\\
\textbf{Чупраков~К.\,Г.} см.~Зацаринный~А.\,А.&&\\
\hangindent=23pt\noindent\textbf{Чупрунов~А.\,Н., Фазекаш~И.} Законы повторного логарифма для числа 
безошибочных\linebreak
\vspace*{-12pt}\\
\hspace*{23pt}блоков при помехоустойчивом кодировании$\dotfill$&3&42\\
\textbf{Шевцова~И.\,Г.} см.~Григорьева~М.\,Е.&&\\
\hangindent=23pt\noindent\textbf{Шестаков~О.\,В.} Аппроксимация распределения оценки риска пороговой 
обработки вейвлет-коэффициентов нормальным распределением при использовании 
выбо-\linebreak
\vspace*{-12pt}\\
\hspace*{23pt}рочной дисперсии$\dotfill$&4&73\\
\textbf{Шестаков~О.\,В.} см.~Маркин~А.\,В.&&\\
\textbf{Шоргин~С.\,Я.} см.~Зейфман~А.\,И.&&\\
\textbf{Шоргин~С.\,Я.} см.~Кудрявцев~А.\,А.&&\\
\end{tabular}
}

%\thispagestyle{myheadings}
\def\leftfootline{\small{\textbf{\thepage}
\hfill ИНФОРМАТИКА И ЕЁ ПРИМЕНЕНИЯ\ \ \ том~4\ \ \ выпуск~4\ \ \ 2010}
}%
 \def\rightfootline{\small{ИНФОРМАТИКА И ЕЁ ПРИМЕНЕНИЯ\ \ \ том~4\ \ \ выпуск~4\ \ \ 2010
 \hfill \textbf{\thepage}}}
 \label{end\stat}


%Том 10 Выпуск 1-4 Год 2016

\def\stat{cont-e}
{%\hrule\par
%\vskip 7pt % 7pt
\raggedleft\Large \bf%\baselineskip=3.2ex
2\,0\,1\,6\ \ A\,U\,T\,H\,O\,R\ \ I\,N\,D\,E\,X \vskip 17pt
 \hrule
 \par
\vskip 21pt plus 6pt minus 3pt }

\label{st\stat}

\def\tit{\ }

\def\aut{\ }
\def\auf{\ }

\def\leftkol{\ } %2016 AUTHOR INDEX} % ENGLISH ABSTRACTS}

\def\rightkol{\ } %2016 AUTHOR INDEX} %ENGLISH ABSTRACTS}

\titele{\tit}{\aut}{\auf}{\leftkol}{\rightkol}

\def\leftfootline{\small{\textbf{\thepage}
\hfill INFORMATIKA I EE PRIMENENIYA~--- INFORMATICS AND APPLICATIONS\ \ \ 2016\
\ \ volume~10\ \ \ issue\ 4}
}%
 \def\rightfootline{\small{INFORMATIKA I EE PRIMENENIYA~--- INFORMATICS AND APPLICATIONS\ \ \ 2016\ \ \ volume~10\ \ \ issue\ 4
\hfill \textbf{\thepage}}}

\vspace*{-12pt}
\vspace*{-18pt}

{\tabcolsep=2.8pt
\begin{tabular}{p{382pt}cc}
&\textbf{Issue} & \textbf{Page}\\[6pt]
\Avtors{Agalarov~M.\,Ya.} see~Agalarov~Ya.\,M.&&\\
\Avtors{Agalarov~Ya.\,M., Agalarov~M.\,Ya., and
Shorgin~V.\,S.} About the optimal threshold of queue\linebreak
\\[-12pt]
\hspace*{23pt}length in a~particular problem of profit maximization
in the $M/G/1$ queuing system&2&70--79\\
\Avtors{Alexeyevsky~D.\,A.} BioNLP ontology extraction from 
a~restricted language corpus with\linebreak
\\[-12pt]
\hspace*{23pt}context-free grammars&1&119--128\\
\Avtors{Andreev~S.\,D.} see~Gaidamaka~Yu.\,V.&&\\
\Avtors{Andreev~S.\,D.} see~Ometov~A.\,Ya.&&\\
\Avtors{Arkhipov~O.\,P., Arkhipov~P.\,O., and Sidorkin~I.\,I.} The
option to create a~local coordinate\linebreak
\\[-12pt]
\hspace*{23pt}system for synchronization of selected images&3&91--97\\
\Avtors{Arkhipov~P.\,O.} see~Arkhipov~O.\,P.&&\\
\Avtors{Belousov~V.\,V.} see~Shnurkov~P.\,V.&&\\
\Avtors{Belousov~V.\,V.} see~Shnurkov~P.\,V.&&\\
\Avtors{Bening~V.\,E.} Calculation of~the~asymptotic deficiency
of~some statistical procedures based\linebreak
\\[-12pt]
\hspace*{23pt}on~samples with~random sizes&4&34--45\\
\Avtors{Borisov~A.\,V., Bosov~A.\,V., and Miller~G.\,B.} Modeling and
monitoring of VoIP connection&2&\hphantom{1}2--13\\
\Avtors{Bosov~A.\,V.} see~Borisov~A.\,V.&&\\
\Avtors{Briukhov~D.\,O.} see~Stupnikov~S.\,A.&&\\
\Avtors{Callaos~N.\,K.\ and Seyful-Mulyukov~R.\,B.} Complexity and
its information content&1&129--139\\
\Avtors{Chertok~A.\,V., Kadaner~A.\,I., Khazeeva~G.\,T., and
Sokolov~I.\,A.} Regime switching detection\linebreak
\\[-12pt]
\hspace*{23pt}for~the~Levy driven
Ornstein--Uhlenbeck process using CUSUM methods&4&46--56\\
\Avtors{Chichagov~V.\,V.} Asymptotic expansions of mean absolute
error of uniformly minimum variance unbiased and maximum likelihood
estimators on the one-parameter exponential\linebreak
\\[-12pt]
\hspace*{23pt}family model of lattice distributions&3&66--76\\
\Avtors{Danishevsky~V.\,I.} see~Kolesnikov A.\,V.&&\\
\Avtors{Fazliev~A.\,Z.} see~Kalinichenko~L.\,A.&&\\
\Avtors{Fedoseev~A.\,A.} What is behind the concept of ``knowledge in
small packages''&3&105--110\\
\Avtors{Gaidamaka~Yu.\,V., Andreev~S.\,D., Sopin~E.\,S.,
Samouylov~K.\,E., and Shorgin~S.\,Ya.} Interference analysis
of~the~device-to-device communications model with~regard to~a~signal\linebreak
\\[-12pt]
\hspace*{23pt}propagation environment&4&\hphantom{1}2--10\\
\Avtors{Gasilov~A.\,V.} see~Yakovlev~O.\,A.&&\\
\Avtors{Goncharov~A.\,V.\ and Strijov~V.\,V.} Metric time series
classification using weighted dynamic\linebreak
\\[-12pt]
\hspace*{23pt}warping relative to centroids of classes&2&36--47\\
\Avtors{Gordov~E.\,P.} see~Kalinichenko~L.\,A.&&\\
\Avtors{Gorshenin~A.\,K.} Concept of online service for stochastic
modeling of real processes&1&72--81\\
\Avtors{Gorshenin~A.\,K.} see~Shnurkov~P.\,V.&&\\
\Avtors{Gorshenin~A.\,K.} see~Shnurkov~P.\,V.&&\\
\Avtors{Grusho~A.\,A., Grusho~N.\,A., Zabezhailo~M.\,I., and
Timonina~E.\,E.} Integration of statistical and\linebreak
\\[-12pt]
\hspace*{23pt}deterministic methods for
analysis of information security&3&2--8\\
\Avtors{Grusho~A.\,A., Zabezhailo~M.\,I., and Zatsarinny~A.\,A.} On
the advanced procedure to reduce\linebreak
\\[-12pt]
\hspace*{23pt}calculation of Galois closures&4&\hphantom{1}96--104\\
\Avtors{Grusho~N.\,A.} see~Grusho~A.\,A.&&\\
\Avtors{Havanskov~V.\,A.} see~Minin~V.\,A.&&\\
\Avtors{Inkova~O.\,Yu.} see~Zatsman~I.\,M.&&\\
\Avtors{Isachenko~R.\,V.\ and Strijov~V.\,V.} Metric learning in
multiclass time series classification\linebreak
\\[-12pt]
\hspace*{23pt}problem&2&48--57\\
\end{tabular}
}
\pagebreak

\def\leftfootline{\small{\textbf{\thepage}
\hfill INFORMATIKA I EE PRIMENENIYA~--- INFORMATICS AND APPLICATIONS\ \ \ 2016\
\ \ volume~10\ \ \ issue\ 4}
}%
 \def\rightfootline{\small{INFORMATIKA I EE PRIMENENIYA~---
INFORMATICS AND APPLICATIONS\ \ \ 2016\ \ \ volume~10\ \ \ issue\ 4
\hfill \textbf{\thepage}}}

\def\leftkol{2016 AUTHOR INDEX} % ENGLISH ABSTRACTS}

\def\rightkol{2016 AUTHOR INDEX} %ENGLISH ABSTRACTS}


{\tabcolsep=2.83pt
\begin{tabular}{p{382pt}cc}
&\textbf{Issue} & \textbf{Page}\\[6pt]
\Avtors{Kadaner~A.\,I.} see~Chertok~A.\,V.&&\\[.255pt]
\Avtors{Kalinichenko~L.\,A., Volnova~A.\,A., Gordov~E.\,P.,
Kiselyova~N.\,N., Kovaleva~D.\,A., Malkov~O.\,Yu., Okladnikov~I.\,G.,
Podkolodnyy~N.\,L., Pozanenko~A.\,S., Ponomareva~N.\,V.,
Stupnikov~S.\,A.,} \textbf{and Fazliev~A.\,Z.} Data access challenges for data
intensive\linebreak
\\[-12pt]
\hspace*{23pt}research in Russia&1& 2--22\\[.255pt]
\Avtors{Karasikov~M.\,E.\ and Strijov~V.\,V.} Feature-based
time-series classification&4&121--131\\[.255pt]
\Avtors{Khazeeva~G.\,T.} see~Chertok~A.\,V.&&\\[.255pt]
\Avtors{Khokhlov~Yu.\,S.} Multivariate fractional Levy motion and its
applications&2&\hphantom{1}98--106\\[.255pt]
\Avtors{Kirikov~I.\,A., Kolesnikov~A.\,V., Listopad~S.\,V., and
Rumovskaya~S.\,B.} Fine-grained hybrid\linebreak
\\[-12pt]
\hspace*{23pt}intelligent systems. Part 2:
Bidirectional hybridization&1&\hphantom{1}96--105\\[.255pt]
\Avtors{Kirikov~I.\,A., Kolesnikov~A.\,V., Listopad~S.\,V., and
Rumovskaya~S.\,B.} ``Virtual council''~---\linebreak
\\[-12pt]
\hspace*{23pt}source environment
supporting complex diagnostic decision making&3&81--90\\[.255pt]
\Avtors{Kiselyova~N.\,N.} see~Kalinichenko~L.\,A.&&\\[.255pt]
\Avtors{Kolesnikov A.\,V., Listopad~S.\,V., Rumovskaya~S.\,B., and
Danishevsky~V.\,I.} Informal axiomatic\linebreak
\\[-12pt]
\hspace*{23pt}theory of~the~role visual models&4&114--120\\[.255pt]
\Avtors{Kolesnikov~A.\,V.} see~Kirikov~I.\,A.&&\\[.255pt]
\Avtors{Kolesnikov~A.\,V.} see~Kirikov~I.\,A.&&\\[.255pt]
\Avtors{Kolin~K.\,K.} Humanitarian aspects of information
security&3&111--121\\[.255pt]
\Avtors{Konovalov~M.\,G.\ and Razumchik~R.\,V.} Dispatching
to~two parallel nonobservable queues using\linebreak
\\[-12pt]
\hspace*{23pt}only static
information&4&57--67\\[.255pt]
\Avtors{Korchagin~A.\,Yu.} see~Korolev~V.\,Yu.&&\\[.255pt]
\Avtors{Korchagin~A.\,Yu.} see~Korolev~V.\,Yu.&&\\[.255pt]
\Avtors{Korepanov~E.\,R.} see~Sinitsyn~I.\,N.&&\\[.255pt]
\Avtors{Korepanov~E.\,R.} see~Sinitsyn~I.\,N.&&\\[.255pt]
\Avtors{Korolev~V.\,Yu., Korchagin~A.\,Yu., and Zeifman~A.\,I.} The
Poisson theorem for Bernoulli trials\linebreak
\\[-12pt]
\hspace*{23pt}with~a~random probability
of~success and~a~discrete analog of~the~Weibull distribution&4&11--20\\[.255pt]
\Avtors{Korolev~V.\,Yu., Zeifman~A.\,I., and Korchagin~A.\,Yu.}
Asymmetric Linnik distributions as~limit\linebreak
\\[-12pt]
\hspace*{23pt}laws for~random sums
of~independent random variables with~finite variances&4&21--33\\[.255pt]
\Avtors{Koucheryavy~E.\,A.} see~Ometov~A.\,Ya.&&\\[.255pt]
\Avtors{Kovaleva~D.\,A.} see~Kalinichenko~L.\,A.&&\\[.255pt]
\Avtors{Kovalyov~S.\,P.} Metaprogramming to increase
manufacturability of large-scale software-\linebreak
\\[-12pt]
\hspace*{23pt}intensive systems&1&56--66\\[.255pt]
\Avtors{Krivenko~M.\,P.} Significance tests of feature selection for
classification&3&32--40\\[.255pt]
\Avtors{Kruzhkov~M.\,G.} see~Zalizniak~Anna~A.&&\\[.255pt]
\Avtors{Kruzhkov~M.\,G.} see~Zatsman~I.\,M.&&\\[.255pt]
\Avtors{Kudryavtsev~A.\,A.} Bayesian queueing and reliability models:
\textit{A~priori} distributions with\linebreak
\\[-12pt]
\hspace*{23pt}compact support&1&67--71\\[.255pt]
\Avtors{Kudryavtsev~A.\,A.} Characteristics dependent on the balance
coefficient in Bayesian models\linebreak
\\[-12pt]
\hspace*{23pt}with compact support of \textit{a priori}
distributions&3&77--80\\[.255pt]
\Avtors{Kudryavtsev~A.\,A.\ and Palionnaia~S.\,I.} Bayesian recurrent
model of reliability growth:\linebreak
\\[-12pt]
\hspace*{23pt}Parabolic distribution of parameters&2&80--83\\[.255pt]
\Avtors{Kudryavtsev~A.\,A.\ and Titova~A.\,I.} Bayesian queuing
and~reliability models: Degenerate-\linebreak
\\[-12pt]
\hspace*{23pt}Weibull case&4&68--71\\[.255pt]
\Avtors{Leontyev~N.\,D.\ and Ushakov~V.\,G.} Analysis of a queueing
system with autoregressive arrivals\linebreak
\\[-12pt]
\hspace*{23pt}and nonpreemptive priority&3&15--22\\[.255pt]
\Avtors{Listopad~S.\,V.} see~Kirikov~I.\,A.&&\\[.255pt]
\Avtors{Listopad~S.\,V.} see~Kirikov~I.\,A.&&\\[.255pt]
\Avtors{Listopad~S.\,V.} see~Kolesnikov A.\,V.&&\\[.255pt]
\Avtors{Malkov~O.\,Yu.} see~Kalinichenko~L.\,A.&&\\[.255pt]
\Avtors{Markov~A.\,S., Monakhov~M.\,M., and
Ulyanov~V.\,V.} Generalized Cornish--Fisher expansions\linebreak
\\[-12pt]
\hspace*{23pt}for distributions of statistics based on samples
of random size&2&84--91\\[.255pt]
\Avtors{Melnikov~A.\,K.\ and Ronzhin~A.\,F.} Generalized statistical
method of~text analysis based\linebreak
\\[-12pt]
\hspace*{23pt}on~calculation of~probability distributions
of~statistical values&4&89--95\\
\end{tabular}
}
\pagebreak

\def\leftfootline{\small{\textbf{\thepage}
\hfill INFORMATIKA I EE PRIMENENIYA~--- INFORMATICS AND APPLICATIONS\ \ \ 2016\
\ \ volume~10\ \ \ issue\ 4}
}%
 \def\rightfootline{\small{INFORMATIKA I EE PRIMENENIYA~---
INFORMATICS AND APPLICATIONS\ \ \ 2016\ \ \ volume~10\ \ \ issue\ 4
\hfill \textbf{\thepage}}}

\def\leftkol{2016 AUTHOR INDEX} % ENGLISH ABSTRACTS}

\def\rightkol{2016 AUTHOR INDEX} %ENGLISH ABSTRACTS}


{\tabcolsep=3pt
\begin{tabular}{p{381pt}cc}
&\textbf{Issue} & \textbf{Page}\\[6pt]
\Avtors{Meykhanadzhyan~L.\,A.} Stationary characteristics of the finite
capacity queueing system with\linebreak
\\[-12pt]
\hspace*{23pt}inverse service order and generalized
probabilistic priority&2&123--131\\[.23pt]
\Avtors{Miller~G.\,B.} see~Borisov~A.\,V.&&\\[.23pt]
\Avtors{Minin~V.\,A., Zatsman~I.\,M., Havanskov~V.\,A., and
Shubnikov~S.\,K.} Intensity of citation of scientific publications in
inventions on information and computer technologies patented\linebreak
\\[-12pt]
\hspace*{23pt}in Russia by domestic and foreign applicants&2&107--122\\[.23pt]
\Avtors{Monakhov~M.\,M.} see~Markov~A.\,S.&&\\[.23pt]
\Avtors{Naumov~V.\,A.\ and Samouylov~K.\,E.} On relationship
between queuing systems with resources\linebreak
\\[-12pt]
\hspace*{23pt}and Erlang networks&3&\hphantom{1}9--14\\[.23pt]
\Avtors{Okladnikov~I.\,G.} see~Kalinichenko~L.\,A.&&\\[.23pt]
\Avtors{Ometov~A.\,Ya., Andreev~S.\,D., Turlikov~A.\,M., and
Koucheryavy~E.\,A.} Performance analysis of\linebreak
\\[-12pt]
\hspace*{23pt}a wireless data
aggregation system with contention for contemporary sensor
networks&3&23--31\\[.23pt]
\Avtors{Palionnaia~S.\,I.} see~Kudryavtsev~A.\,A.&&\\[.23pt]
\Avtors{Podkolodnyy~N.\,L.} see~Kalinichenko~L.\,A.&&\\[.23pt]
\Avtors{Ponomareva~N.\,V.} see~Kalinichenko~L.\,A.&&\\[.23pt]
\Avtors{Popkova~N.\,A.} see~Zatsman~I.\,M.&&\\[.23pt]
\Avtors{Pozanenko~A.\,S.} see~Kalinichenko~L.\,A.&&\\[.23pt]
\Avtors{Razumchik~R.\,V.} see~Konovalov~M.\,G.&&\\[.23pt]
\Avtors{Ronzhin~A.\,F.} see~Melnikov~A.\,K.&&\\[.23pt]
\Avtors{Rumovskaya~S.\,B.} see~Kirikov~I.\,A.&&\\[.23pt]
\Avtors{Rumovskaya~S.\,B.} see~Kirikov~I.\,A.&&\\[.23pt]
\Avtors{Rumovskaya~S.\,B.} see~Kolesnikov A.\,V.&&\\[.23pt]
\Avtors{Samouylov~K.\,E.} see~Gaidamaka~Yu.\,V.&&\\[.23pt]
\Avtors{Samouylov~K.\,E.} see~Naumov~V.\,A.&&\\[.23pt]
\Avtors{Serebryanskii~S.\,M.} see~Tyrsin~A.\,N.&&\\[.23pt]
\Avtors{Seyful-Mulyukov~R.\,B.} see~Callaos~N.\,K.&&\\[.23pt]
\Avtors{Shestakov~O.\,V.} Statistical properties of the denoising method
based on the stabilized hard\linebreak
\\[-12pt]
\hspace*{23pt}thresholding&2&65--69\\[.23pt]
\Avtors{Shestakov~O.\,V.} The strong law of large numbers for the risk
estimate in the problem of\linebreak
\\[-12pt]
\hspace*{23pt}tomographic image reconstruction from
projections with a correlated noise&3&41--45\\[.23pt]
\Avtors{Shestakov~O.\,V.} see~Zakharova~T.\,V.&&\\[.23pt]
\Avtors{Shnurkov~P.\,V., Gorshenin~A.\,K., and Belousov~V.\,V.}
Analytical solution of~the~optimal control\linebreak
\\[-12pt]
\hspace*{23pt}task of~a~semi-Markov
process with~finite set of~states&4&72--88\\[.23pt]
\Avtors{Shnurkov~P.\,V., Zasypko~V.\,V., Belousov~V.\,V., and
Gorshenin~A.\,K.} Development of the algorithm of numerical solution
of the optimal investment control problem\linebreak
\\[-12pt]
\hspace*{23pt}in the closed dynamical model of three-sector economy&1&82--95\\[.23pt]
\Avtors{Shorgin~S.\,Ya.} see~Gaidamaka~Yu.\,V.&&\\[.23pt]
\Avtors{Shorgin~V.\,S.} see~Agalarov~Ya.\,M.&&\\[.23pt]
\Avtors{Shubnikov~S.\,K.} see~Minin~V.\,A.&&\\[.23pt]
\Avtors{Sidorkin~I.\,I.} see~Arkhipov~O.\,P.&&\\[.23pt]
\Avtors{Sinitsyn~I.\,N.} Analytical modeling of processes in stochastic
systems with complex fractional\linebreak
\\[-12pt]
\hspace*{23pt}order Bessel nonlinearities&3&55--65\\[.23pt]
\Avtors{Sinitsyn~I.\,N.} Orthogonal supoptimal filters for nonlinear
stochastic systems on manifolds&1&34--44\\[.23pt]
\Avtors{Sinitsyn~I.\,N.\ and Korepanov~E.\,R.} Normal Pugachev
conditionally-optimal filters and extra-\linebreak
\\[-12pt]
\hspace*{23pt}polators for state linear stochastic systems&2&14--23\\[.23pt]
\Avtors{Sinitsyn~I.\,N.\ and Sinitsyn~V.\,I.} Analytical modeling of
distributions in stochastic systems on\linebreak
\\[-12pt]
\hspace*{23pt}manifolds based on ellipsoidal approximation&1&45--55\\[.23pt]
\Avtors{Sinitsyn~I.\,N., Sinitsyn~V.\,I., and
Korepanov~E.\,R.} Ellipsoidal suboptimal filters for nonlinear\linebreak
\\[-12pt]
\hspace*{23pt}stochastic systems on manifolds&2&24--35\\[.23pt]
\Avtors{Sinitsyn~V.\,I.} see~Sinitsyn~I.\,N.&&\\[.23pt]
\Avtors{Sinitsyn~V.\,I.} see~Sinitsyn~I.\,N.&&\\[.23pt]
\Avtors{Skvortsov~N.\,A.} see~Stupnikov~S.\,A.&&\\[.23pt]
\Avtors{Sokolov~I.\,A.} see~Chertok~A.\,V.&&\\
\end{tabular}
}
\pagebreak

\def\leftfootline{\small{\textbf{\thepage}
\hfill INFORMATIKA I EE PRIMENENIYA~--- INFORMATICS AND APPLICATIONS\ \ \ 2016\
\ \ volume~10\ \ \ issue\ 4}
}%
 \def\rightfootline{\small{INFORMATIKA I EE PRIMENENIYA~---
INFORMATICS AND APPLICATIONS\ \ \ 2016\ \ \ volume~10\ \ \ issue\ 4
\hfill \textbf{\thepage}}}

\def\leftkol{2016 AUTHOR INDEX} % ENGLISH ABSTRACTS}

\def\rightkol{2016 AUTHOR INDEX} %ENGLISH ABSTRACTS}


{\tabcolsep=3pt
\begin{tabular}{p{382pt}cc}
&\textbf{Issue} & \textbf{Page}\\[6pt]
\Avtors{Sopin~E.\,S.} see~Gaidamaka~Yu.\,V.&&\\
\Avtors{Strijov~V.\,V.} see~Goncharov~A.\,V.&&\\
\Avtors{Strijov~V.\,V.} see~Isachenko~R.\,V.&&\\
\Avtors{Strijov~V.\,V.} see~Karasikov~M.\,E.&&\\
\Avtors{Stupnikov~S.\,A., Briukhov~D.\,O., and Skvortsov~N.\,A.}
Co-lending systemic risk analysis over\linebreak
\\[-12pt]
\hspace*{23pt}heterogeneous data collections&1&23--33\\
\Avtors{Stupnikov~S.\,A.} see~Kalinichenko~L.\,A.&&\\
\Avtors{Suchkov~A.\,P.} see~Zatsarinny~A.\,A.&&\\
\Avtors{Timonina~E.\,E.} see~Grusho~A.\,A.&&\\
\Avtors{Titova~A.\,I.} see~Kudryavtsev~A.\,A.&&\\
\Avtors{Turlikov~A.\,M.} see~Ometov~A.\,Ya.&&\\
\Avtors{Tyrsin~A.\,N.\ and Serebryanskii~S.\,M.} Recognition of
dependences on the basis of inverse\linebreak
\\[-12pt]
\hspace*{23pt}mapping&2&58--64\\
\Avtors{Ulyanov~V.\,V.} see~Markov~A.\,S.&&\\
\Avtors{Ushakov~V.\,G.} Queueing system with working vacations and
hyperexponential input stream&2&92--97\\
\Avtors{Ushakov~V.\,G.} see~Leontyev~N.\,D.&&\\
\Avtors{Volnova~A.\,A.} see~Kalinichenko~L.\,A.&&\\
\Avtors{Yakovlev~O.\,A.\ and Gasilov~A.\,V.} Speeded-up stereo
matching using geodesic support weights&3&\hphantom{1}98--104\\
\Avtors{Zabezhailo~M.\,I.} see~Grusho~A.\,A.&&\\
\Avtors{Zabezhailo~M.\,I.} see~Grusho~A.\,A.&&\\
\Avtors{Zakharova~T.\,V.\ and Shestakov~O.\,V.} Precision analysis of
wavelet processing of aerodynamic\linebreak
\\[-12pt]
\hspace*{23pt}flow patterns&3&46--54\\
\Avtors{Zalizniak~Anna~A.\ and Kruzhkov~M.\,G.} Database
of~Russian impersonal verbal constructions&4&132--141\\
\Avtors{Zasypko~V.\,V.} see~Shnurkov~P.\,V.&&\\
\Avtors{Zatsarinny~A.\,A.\ and Suchkov~A.\,P.} Systems engineering
approaches to~the~establishment of\linebreak
\\[-12pt]
\hspace*{23pt}a~system for~decision support based
on~situational analysis&4&105--113\\
\Avtors{Zatsarinny~A.\,A.} see~Grusho~A.\,A.&&\\
\Avtors{Zatsman~I.\,M., Inkova~O.\,Yu., Kruzhkov~M.\,G., and
Popkova~N.\,A.} Representation of cross-\linebreak
\\[-12pt]
\hspace*{23pt}lingual knowledge about
connectors in supracorpora databases&1&106--118\\
\Avtors{Zatsman~I.\,M.} see~Minin~V.\,A.&&\\
\Avtors{Zeifman~A.\,I.} see~Korolev~V.\,Yu.&&\\
\Avtors{Zeifman~A.\,I.} see~Korolev~V.\,Yu.&&\\
\end{tabular}
}

%\thispagestyle{myheadings}
\def\leftfootline{\small{\textbf{\thepage}
\hfill INFORMATIKA I EE PRIMENENIYA~--- INFORMATICS AND APPLICATIONS\ \ \ 2016\
\ \ volume~10\ \ \ issue\ 4}
}%
 \def\rightfootline{\small{INFORMATIKA I EE PRIMENENIYA~---
INFORMATICS AND APPLICATIONS\ \ \ 2016\ \ \ volume~10\ \ \ issue\ 4
\hfill \textbf{\thepage}}}

 \label{end\stat}

\newpage


\vspace*{-60pt} {\small
{\baselineskip=9.1pt
\section*{Правила подготовки рукописей статей для публикации в журнале
<<Информатика и её применения>>}

\thispagestyle{empty}

 Журнал <<Информатика и её применения>> публикует
теоретические, обзорные и дискуссионные статьи, посвященные научным
исследованиям и разработкам в области информатики и ее приложений. Журнал
издается на русском языке. По специальному решению редколлегии отдельные статьи,
в виде исключения, могут печататься на английском языке.
Тематика журнала охватывает следующие направления:
\begin{itemize}
\item теоретические основы информатики; %\\[-13.5pt]
\item математические методы исследования сложных систем и процессов; %\\[-13.5pt]
\item информационные системы и сети; %\\[-13.5pt]
\item информационные технологии; %\\[-13.5pt]
\item архитектура и программное
обеспечение вычислительных комплексов и сетей.
\end{itemize}
\begin{enumerate}
\item В журнале печатаются результаты, ранее не
опубликованные и не предназначенные к одновременной публикации в других
изданиях. Публикация не должна нарушать закон об авторских правах. Направляя
свою рукопись в редакцию, авторы автоматически передают учредителям и
редколлегии неисключительные права на издание данной статьи на русском языке и
на ее распространение в России и за рубежом. При этом за авторами сохраняются
все права как собственников данной рукописи. В связи с этим авторами должно
быть представлено в редакцию письмо в следующей форме:
Соглашение о передаче права на публикацию:

\textit{<<Мы, нижеподписавшиеся, авторы рукописи <<$\qquad\qquad$>>, передаем
учредителям и редколлегии журнала <<Информатика и её применения>>
неисключительное право опубликовать данную рукопись статьи на русском языке как
в печатной, так и в электронной версиях журнала. Мы подтверждаем, что данная
публикация не нарушает авторского права других лиц или организаций. Подписи
авторов: (ф.\,и.\,о., дата, адрес)>>.}

Указанное соглашение может быть представлено 
как в бумажном виде, так и в виде отсканированной копии (с подписями авторов).


Редколлегия вправе запросить у авторов экспертное заключение о возможности
опубликования представленной статьи в открытой печати. %\\[-13.5pt]
\item Статья
подписывается всеми авторами. На отдельном листе представляются данные автора
(или всех авторов): фамилия, полные имя и отчество, телефон, факс, e-mail,
почтовый адрес. Если работа выполнена несколькими авторами, указывается фамилия
одного из них, ответственного за переписку с редакцией. %\\[-13.5pt]
\item Редакция журнала
осуществляет самостоятельную экспертизу присланных статей. Возвращение рукописи
на доработку не означает, что статья уже принята к печати. Доработанный вариант
с ответом на замечания рецензента необходимо прислать в редакцию. %\\[-13.5pt]
\item Решение
редакционной коллегии о принятии статьи к печати или ее отклонении сообщается
авторам. Редколлегия не обязуется направлять рецензию авторам отклоненной
статьи. %\\[-13.5pt]
\item Корректура статей высылается авторам для просмотра. Редакция
просит авторов присылать свои замечания в кратчайшие сроки. %\\[-13.5pt]
\item При
подготовке рукописи в MS Word рекомендуется использовать следующие настройки.
Параметры страницы: формат~--- А4; ориентация~--- книжная; поля (см): внутри~---
2,5, снаружи~--- 1,5, сверху~--- 2, снизу~--- 2, от края до нижнего
колонтитула~--- 1,3. Основной текст: стиль~--- <<Обычный>>: шрифт Times New
Roman, размер 14~пунктов, абзацный отступ~--- 0,5~см, 1,5 интервала,
выравнивание~--- по ширине. Рекомендуемый объем рукописи~--- не свыше
25~страниц указанного формата. Ознакомиться с шаблонами, содержащими примеры
оформления, можно по адресу в Интернете:
\textsf{http://www.ipiran.ru/journal/template.doc}.
\item К рукописи, предоставляемой в 2-х
экземплярах, обязательно прилагается электронная версия статьи (как правило, в
форматах MS WORD (.doc) или \LaTeX\ (.tex), а также~--- дополнительно~--- в
формате .pdf) на дискете, лазерном диске или по электронной почте. Сокращения
слов, кроме стандартных, не применяются. Все страницы рукописи должны быть
пронумерованы. %\\[-13.5pt]
\item Статья должна содержать следующую информацию на русском и
английском языках: название, Ф.И.О. авторов, места работы авторов и их
электронные адреса, подробные сведения об авторах, оформленные в соответствии с форматом, 
определяемым файлами {\sf http://www.ipiran.ru/journal/issues/2011\_05\_01/authors.asp} и 
{\sf http://www.ipiran.ru/journal/issues/2011\_01\_eng/authors.asp},
аннотация (не более 100~слов), ключевые слова. Ссылки на
литературу в тексте статьи нумеруются (в квадратных скобках) и располагаются в
порядке их первого упоминания. В~списке литературы не должно быть позиций, на которые нет ссылки в тексте статьи.
Все фамилии авторов, заглавия статей, названия
книг, конференций и~т.\,п.\ даются на языке оригинала, если этот язык
использует кириллический или латинский алфавит. %\\[-13.5pt]
\item Присланные в редакцию материалы авторам не возвращаются.
\item При отправке файлов по электронной
почте просим придерживаться следующих правил:
\begin{itemize}
\item указывать в поле subject (тема) название журнала и фамилию автора; %\\[-13.5pt]
\item использовать attach (присоединение); %\\[-13.5pt]
\item в случае больших объемов информации возможно
использование общеизвестных архиваторов (ZIP, RAR); %\\[-13.5pt]
\item в состав электронной версии статьи должны входить: файл, содержащий текст статьи, и файл(ы),
содержащий(е) иллюстрации. %\\[-13.5pt]
\end{itemize}
\item Журнал <<Информатика и её применения>> является некоммерческим изданием. 
Плата за публикацию с авторов не взимается, гонорар авторам не выплачивается.
\end{enumerate}
\thispagestyle{empty}
\textbf{Адрес редакции:} Москва 119333,
ул.~Вавилова, д.~44, корп.~2, ИПИ РАН\\
\hphantom{\textbf{Адрес редакции:} }Тел.: +7 (499) 135-86-92\ \
Факс:  +7 (495) 930-45-05\ \  E-mail:   rust@ipiran.ru }
}

\end{document}


%\tableofcontents

%\end{document}





%\def\stat{cont}
{%\hrule\par
%\vskip 7pt % 7pt
\raggedleft\Large \bf%\baselineskip=3.2ex
А\,В\,Т\,О\,Р\,С\,К\,И\,Й\ \ У\,К\,А\,З\,А\,Т\,Е\,Л\,Ь\ \ З\,А\ \ 2\,0\,0\,7 г. \vskip 17pt
    \hrule
    \par
\vskip 21pt plus 6pt minus 3pt }

\label{st\stat}

\def\tit{\ }

\def\aut{\ }
\def\auf{\ }

\def\leftkol{\ } % ENGLISH ABSTRACTS}

\def\rightkol{\ } %ENGLISH ABSTRACTS}

\titele{\tit}{\aut}{\auf}{\leftkol}{\rightkol}


\contentsline {chapter}{\ }{Выпуск \quad Стр.} 
\contentsline {section}{\textbf{Батракова Д.\,А., Королев В.\,Ю., Шоргин С.\,Я.}\ \ Новый метод вероятностно-ста\-ти\-сти\-че\-ско\-го анализа информационных потоков в\nobreakspace {}телекоммуникационных сетях}{\qquad 1 \qquad 40} 
\contentsline {section}{\textbf{Борисов А.\,В.}\ \ Байесовское оценивание в системах наблюдения с\nobreakspace {}марковскими скачкообразными процессами: игровой подход}{\qquad 2 \qquad 65}
\contentsline {section}{\textbf{Босов А.\,В., Иванов А.\,В.}\ \ Программная инфраструктура информационного Web-пор\-тала}{\qquad 2 \qquad 50}
\contentsline {section}{\textbf{Захаров В.\,Н., Калиниченко Л.\,А., Соколов И.\,А., Ступников С.\,А.}\ \ Конструирование канонических информационных моделей для интегрированных информационных систем}{\qquad 2 \qquad 15}
\contentsline {section}{\textbf{Захаров В.\,Н., Козмидиади В.\,А.}\ \ Средства обеспечения отказоустойчивости при\-ло\-жений}{\qquad 1 \qquad 14} 
\contentsline {section}{\textbf{Иванов А.\,В.}\ \ см. Босов А.\,В.\hfill\hfill\hfill\hfill\hfill\hfill\hfill\hfill\hfill\hfill\hfill\hfill\hfill\hfill\hfill\hfill\hfill\hfill\hfill\hfill\hfill\hfill\hfill\hfill\hfill\hfill\hfill\hfill\hfill\hfill\hfill\hfill\hfill\hfill\hfill}{\ }
\contentsline {section}{\textbf{Ильин В.\,Д., Соколов И.\,А.}\ \ Символьная модель системы знаний информатики в\nobreakspace {}че\-ло\-ве\-ко-автоматной среде}{\qquad 1 \qquad 66} 
\contentsline {section}{\textbf{Калиниченко Л.\,А.}\ \ см. Захаров В.\,Н.\hfill\hfill\hfill\hfill\hfill\hfill\hfill\hfill\hfill\hfill\hfill\hfill\hfill\hfill\hfill\hfill\hfill\hfill\hfill\hfill\hfill\hfill\hfill\hfill\hfill\hfill\hfill\hfill\hfill\hfill\hfill\hfill\hfill\hfill\hfill}{\ }
\contentsline {section}{\textbf{Козеренко Е.\,Б.}\ \ Лингвистическое моделирование для систем машинного перевода и обработки знаний}{\qquad 1 \qquad 54} 
\contentsline {section}{\textbf{Козмидиади В.\,А.}\ \ см. Захаров В.\,Н.\hfill\hfill\hfill\hfill\hfill\hfill\hfill\hfill\hfill\hfill\hfill\hfill\hfill\hfill\hfill\hfill\hfill\hfill\hfill\hfill\hfill\hfill\hfill\hfill\hfill\hfill\hfill\hfill\hfill\hfill\hfill\hfill\hfill\hfill\hfill }{\ } 
\contentsline {section}{\textbf{Королев В.\,Ю.}\ \ см. Батракова Д.\,А.\hfill\hfill\hfill\hfill\hfill\hfill\hfill\hfill\hfill\hfill\hfill\hfill\hfill\hfill\hfill\hfill\hfill\hfill\hfill\hfill\hfill\hfill\hfill\hfill\hfill\hfill\hfill\hfill\hfill\hfill\hfill\hfill\hfill\hfill\hfill}{\ } 
\contentsline {section}{\textbf{Кудрявцев А.\,А., Шоргин С.\,Я.}\ \ Байесовский подход к\nobreakspace {}анализу систем массового обслуживания и\nobreakspace {}показателей надежности}{\qquad 2 \qquad 76}
\contentsline {section}{\textbf{Печинкин А.\,В., Соколов И.\,А., Чаплыгин В.\,В.}\ \ Многолинейная система массового обслуживания с конечным накопителем и ненадежными приборами}{\qquad 1 \qquad 27} 
\contentsline {section}{\textbf{Печинкин А.\,В., Соколов И.\,А., Чаплыгин В.\,В.}\ \ Стационарные характеристики многолинейной\nobreakspace {}системы массового обслуживания с\nobreakspace {}одновременными отказами приборов}{\qquad 2 \qquad 39}
\contentsline {section}{\textbf{Синицын И.\,Н.}\ \ Корреляционные методы построения аналитических информационных моделей флуктуаций полюса Земли по априорным данным}{\qquad 2 \qquad \hphantom{9}2}
\contentsline {section}{\textbf{Синицын И.\,Н.}\ \ Развитие теории фильтров Пугачева для оперативной обработки информации в стохастических системах}{{\qquad 1 \qquad \hphantom{9}3}} 
\contentsline {section}{\textbf{Соколов И.\,А.}\ \ см. Захаров В.\,Н.\hfill\hfill\hfill\hfill\hfill\hfill\hfill\hfill\hfill\hfill\hfill\hfill\hfill\hfill\hfill\hfill\hfill\hfill\hfill\hfill\hfill\hfill\hfill\hfill\hfill\hfill\hfill\hfill\hfill\hfill\hfill\hfill\hfill\hfill\hfill}{\ }
\contentsline {section}{\textbf{Соколов И.\,А.}\ \ см. Ильин В.\,Д.\hfill\hfill\hfill\hfill\hfill\hfill\hfill\hfill\hfill\hfill\hfill\hfill\hfill\hfill\hfill\hfill\hfill\hfill\hfill\hfill\hfill\hfill\hfill\hfill\hfill\hfill\hfill\hfill\hfill\hfill\hfill\hfill\hfill\hfill\hfill}{\ } 
\contentsline {section}{\textbf{Соколов И.\,А.}\ \ см. Печинкин А.\,В.\hfill\hfill\hfill\hfill\hfill\hfill\hfill\hfill\hfill\hfill\hfill\hfill\hfill\hfill\hfill\hfill\hfill\hfill\hfill\hfill\hfill\hfill\hfill\hfill\hfill\hfill\hfill\hfill\hfill\hfill\hfill\hfill\hfill\hfill\hfill}{\ } 
\contentsline {section}{\textbf{Соколов И.\,А.}\ \ см. Печинкин А.\,В.\hfill\hfill\hfill\hfill\hfill\hfill\hfill\hfill\hfill\hfill\hfill\hfill\hfill\hfill\hfill\hfill\hfill\hfill\hfill\hfill\hfill\hfill\hfill\hfill\hfill\hfill\hfill\hfill\hfill\hfill\hfill\hfill\hfill\hfill\hfill}{\ }
\contentsline {section}{\textbf{Ступников С.\,А.}\ \ см. Захаров В.\,Н.\hfill\hfill\hfill\hfill\hfill\hfill\hfill\hfill\hfill\hfill\hfill\hfill\hfill\hfill\hfill\hfill\hfill\hfill\hfill\hfill\hfill\hfill\hfill\hfill\hfill\hfill\hfill\hfill\hfill\hfill\hfill\hfill\hfill\hfill\hfill}{\ }
\contentsline {section}{\textbf{Чаплыгин В.\,В.}\ \ см. Печинкин А.\,В.\hfill\hfill\hfill\hfill\hfill\hfill\hfill\hfill\hfill\hfill\hfill\hfill\hfill\hfill\hfill\hfill\hfill\hfill\hfill\hfill\hfill\hfill\hfill\hfill\hfill\hfill\hfill\hfill\hfill\hfill\hfill\hfill\hfill\hfill\hfill}{\ } 
\contentsline {section}{\textbf{Чаплыгин В.\,В.}\ \ см. Печинкин А.\,В.\hfill\hfill\hfill\hfill\hfill\hfill\hfill\hfill\hfill\hfill\hfill\hfill\hfill\hfill\hfill\hfill\hfill\hfill\hfill\hfill\hfill\hfill\hfill\hfill\hfill\hfill\hfill\hfill\hfill\hfill\hfill\hfill\hfill\hfill\hfill}{\ }
\contentsline {section}{\textbf{Шоргин С.\,Я.}\ \ см. Батракова Д.\,А.\hfill\hfill\hfill\hfill\hfill\hfill\hfill\hfill\hfill\hfill\hfill\hfill\hfill\hfill\hfill\hfill\hfill\hfill\hfill\hfill\hfill\hfill\hfill\hfill\hfill\hfill\hfill\hfill\hfill\hfill\hfill\hfill\hfill\hfill\hfill}{\ } 
\contentsline {section}{\textbf{Шоргин С.\,Я.}\ \ см. Кудрявцев А.\,А.\hfill\hfill\hfill\hfill\hfill\hfill\hfill\hfill\hfill\hfill\hfill\hfill\hfill\hfill\hfill\hfill\hfill\hfill\hfill\hfill\hfill\hfill\hfill\hfill\hfill\hfill\hfill\hfill\hfill\hfill\hfill\hfill\hfill\hfill\hfill}{\ }
%\thispagestyle{myheadings}
\def\leftfootline{\small{\textbf{\thepage}
\hfill ИНФОРМАТИКА И ЕЁ ПРИМЕНЕНИЯ\ \ \ том~1\ \ \ выпуск~2\ \ \ 2007}
}%
 \def\rightfootline{\small{ИНФОРМАТИКА И ЕЁ ПРИМЕНЕНИЯ\ \ \ том~1\ \ \ выпуск~2\ \ \ 2007
 \hfill \textbf{\thepage}}}
 \label{end\stat}

%\def\stat{cont-e}
{%\hrule\par
%\vskip 7pt % 7pt
\raggedleft\Large \bf%\baselineskip=3.2ex
2\,0\,0\,7\ \ A\,U\,T\,H\,O\,R\ \ I\,N\,D\,E\,X \vskip 17pt
    \hrule
    \par
\vskip 21pt plus 6pt minus 3pt }

\label{st\stat}

\def\tit{\ }

\def\aut{\ }
\def\auf{\ }

\def\leftkol{\ } % ENGLISH ABSTRACTS}

\def\rightkol{\ } %ENGLISH ABSTRACTS}

\titele{\tit}{\aut}{\auf}{\leftkol}{\rightkol}


\contentsline {chapter}{\ }{Issue \quad Page} 
\contentsline {subsection}{\textbf{Batrakova D.\,A., Korolev V.\,Yu., Shorgin S.\,Ya.}\ \ A New Method for the Probabilistic and Statistical Analysis of Information Flows in Telecommunication Networks}{\qquad 1 \qquad 40} 
\contentsline {subsection}{\textbf{Borisov A.\,V.}\ \ Bayesian Estimation in\nobreakspace {}Observation Systems with\nobreakspace {}Markov Jump Processes: Game-Theoretic Approach}{\qquad 2 \qquad 65} 
\contentsline {subsection}{\textbf{Bosov A.\,V., Ivanov A.\,V.}\ \ Linguistic Simulation for Machine Translation and Knowledge Management Systems}{\qquad 2 \qquad 50} 
\contentsline {subsection}{\textbf{Chaplygin V.\,V.} see Pechinkin A.\,V.\hfill\hfill\hfill\hfill\hfill\hfill\hfill\hfill\hfill\hfill\hfill\hfill\hfill\hfill\hfill\hfill\hfill\hfill\hfill\hfill\hfill\hfill\hfill\hfill\hfill\hfill\hfill\hfill\hfill\hfill\hfill\hfill\hfill\hfill\hfill}{\ }
\contentsline {subsection}{\textbf{Chaplygin V.\,V.} see Pechinkin A.\,V.\hfill\hfill\hfill\hfill\hfill\hfill\hfill\hfill\hfill\hfill\hfill\hfill\hfill\hfill\hfill\hfill\hfill\hfill\hfill\hfill\hfill\hfill\hfill\hfill\hfill\hfill\hfill\hfill\hfill\hfill\hfill\hfill\hfill\hfill\hfill}{\ }
\contentsline {subsection}{\textbf{Ilyin V.\,D., Sokolov I.\,A.}\ \ The Symbol Model of Informatics Knowledge System in Human-Automaton Environment}{\qquad 1 \qquad 66} 
\contentsline {subsection}{\textbf{Ivanov A.\,V.} see Bosov A.\,V.\hfill\hfill\hfill\hfill\hfill\hfill\hfill\hfill\hfill\hfill\hfill\hfill\hfill\hfill\hfill\hfill\hfill\hfill\hfill\hfill\hfill\hfill\hfill\hfill\hfill\hfill\hfill\hfill\hfill\hfill\hfill\hfill\hfill\hfill\hfill}{\ }
\contentsline {subsection}{\textbf{Kalinichenko L.\,A.} see Zakharov V.\,N.\hfill\hfill\hfill\hfill\hfill\hfill\hfill\hfill\hfill\hfill\hfill\hfill\hfill\hfill\hfill\hfill\hfill\hfill\hfill\hfill\hfill\hfill\hfill\hfill\hfill\hfill\hfill\hfill\hfill\hfill\hfill\hfill\hfill\hfill\hfill}{\ }
\contentsline {subsection}{\textbf{Korolev V.\,Yu.} see Batrakova D.\,A.\hfill\hfill\hfill\hfill\hfill\hfill\hfill\hfill\hfill\hfill\hfill\hfill\hfill\hfill\hfill\hfill\hfill\hfill\hfill\hfill\hfill\hfill\hfill\hfill\hfill\hfill\hfill\hfill\hfill\hfill\hfill\hfill\hfill\hfill\hfill}{\ }
\contentsline {subsection}{\textbf{Kozerenko E.\,B.}\ \ Linguistic Simulation for Machine Translation and Knowledge Management Systems}{\qquad 1 \qquad 54} 
\contentsline {subsection}{\textbf{Kozmidiady V.\,A.} see Zakharov V.\,N.\hfill\hfill\hfill\hfill\hfill\hfill\hfill\hfill\hfill\hfill\hfill\hfill\hfill\hfill\hfill\hfill\hfill\hfill\hfill\hfill\hfill\hfill\hfill\hfill\hfill\hfill\hfill\hfill\hfill\hfill\hfill\hfill\hfill\hfill\hfill}{\ }
\contentsline {subsection}{\textbf{Kudryavtsev A.\,A., Shorgin S.\,Ya.}\ \ Bayesian Approach to Queueing Systems and Reliability Characteristics}{\qquad 2 \qquad 76} 
\contentsline {subsection}{\textbf{Pechinkin A.\,V., Sokolov I.\,A., Chaplygin V.\,V.}\ \ Multichannel Queuing System with Finite Buffer and Unreliable Servers}{\qquad 1 \qquad 27} 
\contentsline {subsection}{\textbf{Pechinkin A.\,V., Sokolov I.\,A., Chaplygin V.\,V.}\ \ Stationary Characteristics of a Multichannel Queueing System with\nobreakspace {}Simultaneous Refusals of Servers}{\qquad 2 \qquad 39} 
\contentsline {subsection}{\textbf{Shorgin S.\,Ya.} see Batrakova D.\,A.\hfill\hfill\hfill\hfill\hfill\hfill\hfill\hfill\hfill\hfill\hfill\hfill\hfill\hfill\hfill\hfill\hfill\hfill\hfill\hfill\hfill\hfill\hfill\hfill\hfill\hfill\hfill\hfill\hfill\hfill\hfill\hfill\hfill\hfill\hfill}{\ }
\contentsline {subsection}{\textbf{Shorgin S.\,Ya.} see Kudryavtsev A.\,A.\hfill\hfill\hfill\hfill\hfill\hfill\hfill\hfill\hfill\hfill\hfill\hfill\hfill\hfill\hfill\hfill\hfill\hfill\hfill\hfill\hfill\hfill\hfill\hfill\hfill\hfill\hfill\hfill\hfill\hfill\hfill\hfill\hfill\hfill\hfill}{\ }
\contentsline {subsection}{\textbf{Sinitsyn I.\,N.}\ \ Correlational Methods for Analytical Informational Models of the Earth Pole Fluctuations Design Based on a priori Data}{\qquad 2 \qquad \hphantom{9}2}
\contentsline {subsection}{\textbf{Sinitsyn I.\,N.}\ \ Development of Pugachev Filtering for Stochastic Systems}{\qquad 1 \qquad \hphantom{9}3}
\contentsline {subsection}{\textbf{Sokolov I.\,A.} see Ilyin V.\,D.\hfill\hfill\hfill\hfill\hfill\hfill\hfill\hfill\hfill\hfill\hfill\hfill\hfill\hfill\hfill\hfill\hfill\hfill\hfill\hfill\hfill\hfill\hfill\hfill\hfill\hfill\hfill\hfill\hfill\hfill\hfill\hfill\hfill\hfill\hfill}{\ }
\contentsline {subsection}{\textbf{Sokolov I.\,A.} see Pechinkin A.\,V.\hfill\hfill\hfill\hfill\hfill\hfill\hfill\hfill\hfill\hfill\hfill\hfill\hfill\hfill\hfill\hfill\hfill\hfill\hfill\hfill\hfill\hfill\hfill\hfill\hfill\hfill\hfill\hfill\hfill\hfill\hfill\hfill\hfill\hfill\hfill}{\ }
\contentsline {subsection}{\textbf{Sokolov I.\,A.} see Pechinkin A.\,V.\hfill\hfill\hfill\hfill\hfill\hfill\hfill\hfill\hfill\hfill\hfill\hfill\hfill\hfill\hfill\hfill\hfill\hfill\hfill\hfill\hfill\hfill\hfill\hfill\hfill\hfill\hfill\hfill\hfill\hfill\hfill\hfill\hfill\hfill\hfill}{\ }
\contentsline {subsection}{\textbf{Sokolov I.\,A.} see Zakharov V.\,N.\hfill\hfill\hfill\hfill\hfill\hfill\hfill\hfill\hfill\hfill\hfill\hfill\hfill\hfill\hfill\hfill\hfill\hfill\hfill\hfill\hfill\hfill\hfill\hfill\hfill\hfill\hfill\hfill\hfill\hfill\hfill\hfill\hfill\hfill\hfill}{\ }
\contentsline {subsection}{\textbf{Stupnikov S.\,A.} see Zakharov V.\,N.\hfill\hfill\hfill\hfill\hfill\hfill\hfill\hfill\hfill\hfill\hfill\hfill\hfill\hfill\hfill\hfill\hfill\hfill\hfill\hfill\hfill\hfill\hfill\hfill\hfill\hfill\hfill\hfill\hfill\hfill\hfill\hfill\hfill\hfill\hfill}{\ }
\contentsline {subsection}{\textbf{Zakharov V.\,N., Kalinichenko L.\,A., Sokolov I.\,A., Stupnikov S.\,A.}\ \ Development of Canonical Information Models for Integrated Information Systems}{\qquad 2 \qquad 15} 
\contentsline {subsection}{\textbf{Zakharov V.\,N., Kozmidiady V.\,A.}\ \ Means Providing Applications Fault Tolerance}{\qquad 1 \qquad 14} 
\def\leftfootline{\small{\textbf{\thepage}
\hfill ИНФОРМАТИКА И ЕЁ ПРИМЕНЕНИЯ\ \ \ том~1\ \ \ выпуск~2\ \ \ 2007}
}%
 \def\rightfootline{\small{ИНФОРМАТИКА И ЕЁ ПРИМЕНЕНИЯ\ \ \ том~1\ \ \ выпуск~2\ \ \ 2007
 \hfill \textbf{\thepage}}}
 \label{end\stat}


%\tableofcontents


\end{document}