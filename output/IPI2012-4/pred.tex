
   { %\Large  
   { %\baselineskip=16.6pt
   
   \vspace*{-48pt}
   \begin{center}\LARGE
   \textit{Предисловие}
   \end{center}
   
   %\vspace*{2.5mm}
   
   \vspace*{25mm}
   
   \thispagestyle{empty}
   
   { %\small
      Вниманию читателей журнала <<Информатика и её применения>> предлагается 
      очередной тематический выпуск <<Вероятностно-статистические методы и 
      задачи информатики и информационных технологий>>. Предыдущие тематические 
      выпуски журнала по данному направлению вышли в 2008 (т.~2, вып.~2), 
       2009 (т.~3, вып.~3),  2010 (т.~4, вып.~2)  и 2011~гг.\ (т.~5, вып.~3).
      
      Статьи, собранные в данном журнале, посвящены разработке и совершенствованию 
ве\-ро\-ят\-но\-ст\-но-ста\-ти\-сти\-че\-ских и смежных с ними методов, ориентированных 
на применение к решению конкретных задач информатики и информационных 
технологий, а также~--- в ряде случаев~--- и других прикладных задач. Проблематика, 
охватываемая публикуемыми работами, в значительной степени развивается в рамках 
научного сотрудничества между Институтом проблем информатики Российской академии 
наук (ИПИ РАН) и факультетом вычислительной математики и кибернетики Московского 
государственного университета им.\ М.\,В.~Ломоносова в ходе работ над совместными 
научными проектами (в том числе в рамках функционирования 
      На\-уч\-но-обра\-зо\-ва\-тель\-но\-го центра 
      <<Ве\-ро\-ят\-но\-ст\-но-ста\-ти\-сти\-че\-ские методы анализа рисков>>). Многие 
из авторов статей, включенных в данный номер журнала, являются активными 
участниками традиционного международного семинара по проблемам устойчивости 
стохастических моделей, руководимого В.\,М.~Золотаревым и В.\,Ю.~Королевым; 
регулярные сессии этого семинара проводятся под эгидой МГУ и ИПИ РАН (в 2012~г.\ 
указанный семинар проводился в сентябре в г.~Светлогорске Калининградской области 
РФ). 
      
      Наряду с представителями ИПИ РАН и МГУ в число авторов данного выпуска 
журнала входят ученые из Вычислительного центра им.\ А.\,А.~Дородницына Российской 
академии наук, Института океанологии им.\ П.\,П.~Ширшова Российской академии наук, 
Московского фи\-зи\-ко-тех\-ни\-че\-ско\-го института, Вологодского государственного 
педагогического университета, отдела моделирования и математической статистики 
Аль\-фа-бан\-ка, а также Федерального университета штата Баия (Бразилия), Университета 
им.\ Бен-Гу\-рио\-на в Негеве (Израиль), Университета Реджайны (Канада), Университета 
Пай Чай (Республика Корея), Университета Таммасат (Таиланд).
      
      Тематика статей данного выпуска включает вопросы математического 
моделирования и анализа реальных процессов, в том числе моделирование  
стохастических систем с автокоррелированными шумами; анализ катастрофически 
накапливающихся эффектов при прогнозировании риска экстремальных событий; 
построение и исследование моделей некоторых специальных систем и сетей передачи 
информации; вывод оценок точности аппроксимаций распределений, описывающих 
реальные процессы; построение адаптивных стратегий оптимального управления на 
основе наблюдений, доступных в процессе управления; математические методы и 
алгоритмы распознавания образов. 
      
      Редакционная коллегия журнала выражает надежду, что данный тематический  
выпуск будет интересен специалистам в области теории вероятностей и математической 
статистики и их применения к решению задач информатики и информационных 
технологий.
      
        %\vfill 
           \vspace*{20mm}
           \noindent
           Заместитель главного редактора журнала <<Информатика и её 
      применения>>,\\
           директор ИПИ РАН, академик  \hfill
           \textit{И.\,А.~Соколов}\\
           
           \noindent
           Редактор-составитель тематического выпуска,\\
           профессор кафедры математической статистики факультета\\
            вычислительной математики и кибернетики МГУ им.\ М.\,В.~Ломоносова,\\
           ведущий научный сотрудник ИПИ РАН,\\ 
      доктор физико-математических наук \hfill
            \textit{В.\,Ю.~Королев}
           
           } }
           }