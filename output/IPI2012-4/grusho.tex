\def\stat{grusho}

\def\tit{МОДЕЛЬ СЛУЧАЙНЫХ ГРАФОВ ДЛЯ~ОПИСАНИЯ~ВЗАИМОДЕЙСТВИЙ В СЕТИ$^*$}

\def\titkol{Модель случайных графов для описания взаимодействий в сети}

\def\autkol{А.\,А.~Грушо, Е.\,Е.~Тимонина}

\def\aut{А.\,А.~Грушо$^1$, Е.\,Е.~Тимонина$^2$}

\titel{\tit}{\aut}{\autkol}{\titkol}

{\renewcommand{\thefootnote}{\fnsymbol{footnote}}\footnotetext[1]
{Работа выполнена при поддержке РФФИ (проекты №\,10-01-00480, №\,11-07-00112).}}

\renewcommand{\thefootnote}{\arabic{footnote}}
\footnotetext[1]{Институт проблем информатики Российской академии наук; Московский государственный 
университет им.\ М.\,В.~Ломоносова, факультет вычислительной математики и кибернетики, 
grusho@yandex.ru}
\footnotetext[2]{Институт проблем информатики Российской академии наук, eltimon@yandex.ru}

\Abst{Рассматривается новый класс случайных графов, призванный 
моделировать функционирование сети во времени. Предполагается, что наблюдения за 
сетью ведутся с помощью <<оконного>> метода. С~целью выявления аномалий 
исследуется нормальное поведение степеней, которые можно наблюдать в <<окнах>> 
рассматриваемой модели. Исследована асимптотика максимальной степени вершин в 
графе, который порожден <<окном>> данного размера.}

\KW{случайные графы; моделирование глобальных сетей; информационная 
безопасность; аномальное поведение}

\vskip 14pt plus 9pt minus 6pt

      \thispagestyle{headings}

      \begin{multicols}{2}

            \label{st\stat}

\section{Введение}

     Возможны различные способы распространения информации в сети. Наиболее 
известным способом является обращение к информационным ресурсам, выложенным на 
тематических сайтах. Поиск таких сайтов по нужной тематике~--- задача 
     ин\-тер\-нет-по\-иско\-ви\-ков. Другое направление~---\linebreak распространение информации 
по электронной поч\-те или через налаженные и специальным образом формируемые связи. 
К~таким способам относятся спам и управление бот-се\-тями. 
     
     Формирование собрания единомышленников или предупреждение о каком-либо 
событии могут передаваться любым из перечисленных выше методов. В~связи с этим 
представляет интерес моделирование процессов последовательной передачи информации. 
Будем рассматривать второй способ передачи информации. 
     
     В предположении о дискретности времени будем описывать состояние связей хостов 
между\linebreak собой неориентированным случайным графом. Реб\-рам графа отвечают логические 
связи хостов. Вмес\-те с тем последовательности связей вершин не обязательно связаны с 
распространением инфор\-мации в рамках некоторой корпорации. В~этом \mbox{случае} 
последовательности связей формируются случайно и независимо друг от друга. 
Выявление корпоратив\-ных связей в сети или центров управления связано с анализом 
случайных связей в рамках большого случайного графа сети. 
     
     Будем рассматривать изменения в графе сети с течением времени с помощью 
<<скользящего окна>>. Граф, который получается при таком рассмотрении, формируется 
фиксацией логических связей, захватываемых <<окном>> и отображенных на одном 
графе. Такой граф получается наложением всех графов в моменты времени, 
принадлежащие <<окну>>, и объединением параллельных ребер.
     
     В работе определена математическая модель таких случайных графов и исследованы 
характеристики, связанные с указанными выше прикладными задачами. Полученные 
результаты носят асимптотический характер при условии, что число хостов стремится к 
бесконечности. 
     
     Работа имеет следующую структуру. В~разд.~2 приведены некоторые близкие 
модели случайных графов. В~разд.~3 определена основная динамическая модель. 
В~разд.~4 исследована асимптотика максимальной степени в случайном графе, связанном 
с <<окнами>>. В~разд.~5 подведены итоги и намечены дальнейшие задачи. 
     
\section{Модели случайных графов}
     
     В научной литературе рассматривались модели случайных графов, связанных с 
Интернетом. В~работе~[1] в качестве одного из примеров приводится классическая 
модель случайного графа $G_{N,p}$ с независимыми ребрами, появляющимися с одной и 
той же вероятностью~$p$. Этой модели посвящено много работ и книг~[2--13]. И хотя 
осново\-по\-ла\-га\-ющая \mbox{статья} Эрдеша и Реньи~\cite{2-gr} связана с несколько другим классом 
случайных графов, большинство аналогичных результатов было также доказано для 
графов $G_{N,p}$ в работах~[3--6, 8] и~др. Фазовые перехо-\linebreak\vspace*{-12pt}

\pagebreak

\noindent
ды в структуре таких графов 
впервые исследованы в\linebreak работах~[3--5]. Первые модели с неравновероятными ребрами 
исследовались в работах~\cite{7-gr}. В~пе\-ре\-чис\-лен\-ных моделях появление ребра не 
допускало его дальнейшего исчезновения.
     
     Изменение случайного графа во времени в связи с задачей роста сети Интернет 
рассматривалось в работах~[7, 9, 10]. 
     
     Специальный класс случайных графов, посвященный исследованию связей в 
Интернете, объединяет модели графов ин\-тер\-нет-типа~[11--13]. Они определяются 
степенями вершин, являющихся независимыми случайными величинами. При этом 
свободные концы ребер замыкаются друг на друга случайно и равновероятно. 
     
\section{Динамическая модель сетевого взаимодействия}
     
     Определим детально модель сетевого взаимодействия, кратко изложенную во 
введении. 
     
     Рассмотрим дискретное время $t \hm= 0, 1, 2, \ldots$ Множество хостов сети 
обозначим $A\hm \{a_1, \ldots , a_N\}$. Логическая связь хостов~$a_i$ и~$a_j$ в момент 
времени~$t$ означает либо наличие в этот момент времени сеанса связи по протоколу 
ТСР между~$a_i$ и~$a_j$, либо передачу одиночного пакета от одного хоста к другому в 
этот момент времени по любому протоколу без установления соединения. Для простоты 
считаем, что время прохождения пакета по сети равно~1. При этом выбрасываются из 
рассмотрения все промежуточные поддерживающие сетевые службы (провайдеры, 
маршрутизаторы, адресные службы и~т.\,д.). Из этих допущений получаем модель графа 
сети. В~каждый момент времени~$t$ определен неориентированный граф~$G_t$, 
вершины которого совпадают с множеством~$A$, а ребра соответствуют существующим 
в момент времени~$t$ логическим связям. Из определения логической связи следует, что 
в соседние моменты времени~$t$ и $t\hm+1$ существование данного ребра в графе~$G_t$ 
и графе $G_{t+1}$ являются зависимыми событиями. Однако процессы появления и 
исчезновения разных ребер можно считать независимыми. 
     
     В простейшем случае полагаем, что процесс, описывающий возникновение и 
исчезновение одного ребра, является стационарной однородной \mbox{цепью} Маркова с двумя 
состояниями: 1~--- есть ребро, 0~--- нет ребра. 
     
     Наблюдения за графами $\{G_t\}$ происходят с помощью <<оконной>> системы. 
Пусть задано натуральное число~$r$ и для любого момента времени~$t$ рассматриваются 
графы $G_t, \ldots , G_{t+r}$, появляющиеся в моменты времени $[t,\,t+r]$. Определим 
операцию объединения этих графов
     $$
     G_{t,r}=\bigcup\limits_{i=0}^r G_{t+i}\,,
     $$
где из нескольких параллельных ребер оставляется одно ребро. Граф $G_{t.r}$ несет 
информацию об активности любой вершины в заданный промежуток времени. Эти графы 
представляют интерес в задачах информационной безопасности. Например, если вершина 
$a_i$ является центром управления бот-сетью, то использование бот-сети для организации 
DDoS атаки должно порождать в некоторый промежуток времени $[t,\,t+r]$ резкое 
повышение степени вершины~$a_i$ в графе~$G_{t,r}$. При малых~$r$ и очень 
больших~$r$ при неизвестном~$t$ этот всплеск активности может оказаться незаметным 
в масштабах всей сети. Поэтому исследование модели случайных графов может позволить 
оценить возможности по выявлению неслучайных всплесков активности отдельных 
вершин и даже дать оценку для центра управления бот-сетью. 

     Пусть поведение каждого ребра описывается стационарной однородной цепью 
Маркова с мат\-ри\-цей переходных вероятностей
     \begin{multline*}
     P=\begin{pmatrix}
     p &\ \  1-p\\
     q &\ \  1-q
     \end{pmatrix}\,,\\ 1>p=p(N)>0\,,\quad 1>q=q(N)>0\,,
%     \label{e1-gr}
     \end{multline*}
и стационарным распределением 
$\left( p_0\ \  1-p_0\right)$.

     
     Тогда вероятность непоявления данного ребра за промежуток времени $[t,\,t+r]$ 
равна
     \begin{equation}
     1-p_r=(1-p_0)(1-q)^r\,.
     \label{e3-gr}
     \end{equation}
 Эта вероятность не зависит от~$t$, поэтому будем обозначать ее~$p_r$. Из~(\ref{e3-gr}) 
     получаем вероятность появления данного ребра в промежуток времени $[t,\,t+r]$ в 
графе~$G_{t,r}$:
     \begin{equation*}
     p_r=1-(1-p_0)(1-q)^r\,.
%     \label{e4-gr}
     \end{equation*}
     
\section{Асимптотические оценки максимальной степени в~графах~{\boldmath{$G_{t,r}$}}}

     Для $v\in A$ обозначим через $d(v)$ степень вершины~$v$. Определим 
индикаторную функцию события~$B$:
     $$
     I(B)=\begin{cases}
     1\,, &\ \mbox{если событие $B$ призошло};\\
     0 &\ \mbox{в противном случае.}
     \end{cases}
     $$
     
     Ожидаемое число соединений у фиксированной вершины в ограниченный 
промежуток времени мало по сравнению с общим числом вершин. Асимптотически это 
отвечает условию $p_0 N\hm\rightarrow \mu\hm>0$, $N\hm\rightarrow\infty$. Серии единиц 
связаны с режимом установления соединения. Поэтому величина ($1-p$) может не 
стремиться к~1. Пусть $q\hm\rightarrow 0$, $N\hm\rightarrow\infty$, так что 
$qN\hm\rightarrow\lambda$. Из условия стационарности следует соотношение:
     $$
     q=\fr{p_0}{1-p_0}\left( 1-p\right)\,.
     $$
Таким образом, получаем
$$
p_r=\fr{\mu}{N}+\fr{\lambda r}{N}+O\left( \fr{\lambda r}{N^2}\right)\,.
$$
Положим $\alpha_r=\mu+\lambda r$ и будем считать, что
$$
p_r=\fr{\alpha_r}{N}\,.
$$
Обозначим 
$$
X=\sum\limits_{v\in A} I(d(v)>d)\,.
$$
Тогда 
$$
\left\{ \max\limits_{v\in A} d(v)\right\} >d =\left\{ X\geq 1\right\}\,.
$$
Используя неравенство Маркова, получаем:
\begin{multline}
P\left\{ \max\limits_{v\in A} d(v)>d\right\} \leq{}\\
{}\leq N\sum\limits_{k>d}\begin{pmatrix}
N-1\\ k\end{pmatrix} p_r^k(1-p_r)^{N-1-k}\,.
\label{e5-gr}
\end{multline}
     
     Пусть $B(N-1, k, p_r)$~--- функция распределения биномиального закона, 
$\overline{B}(N-1, k, p_r)\hm=1 \hm- B(N-1, k, p_r)$. Заметим, что в формуле~(\ref{e5-gr}) 
справа стоит $N\overline{B} (N-1, d, p_r)$. 
     
     Асимптотические оценки проводим в условиях
     $$
     N\rightarrow\infty\,, \ d=\fr{C\ln N}{\ln\ln N}\,,\ C>0\,,\ p_r=\fr{\alpha_r}{N}\,.
     $$
     
     Для оценки функции $\overline{B} (N-1, d, p_r)$ воспользуемся представлением для 
неполной бе\-та-функ\-ции~\cite{14-gr}:
     $$
     \overline{B}(N-1,k,p_r)=N\begin{pmatrix}
     N-2\\ k \end{pmatrix} \int\limits_0^{p_r} z^k (1-z)^{N-k-2}dz\,.
     $$
     
     Используя формулу Тейлора, получим при некотором $0\hm<\theta\hm<1$ 
следующее представление для математического ожидания~$X$:
     $$
     {\sf E}X=N(N-1)\begin{pmatrix}
     N-2\\ d\end{pmatrix} (p_r\theta)^d p_r(1-p_r\theta)^{N-2-d}\,.
     $$
Отсюда получаем следующую асимптотическую формулу:
\begin{equation}
{\sf E}X=N^{(1-C)(1+o(1))}\alpha_r(1-e^{-\alpha_r\theta})\,.
\label{e6-gr}
     \end{equation}
     
     \noindent
     \textbf{Теорема.} \textit{При} $C\hm>1$, $N\hm\rightarrow\infty$, $d=C\ln N/(\ln \ln 
N)$, $p_r=\alpha_r/N$
     $$
     P\left\{ \max\limits_{v\in A} d(v)>d\right\}\rightarrow 0\,.
     $$
     
     \noindent
     Д\,о\,к\,а\,з\,а\,т\,е\,л\,ь\,с\,т\,в\,о\ следует из~(\ref{e6-gr}). 
     
     \smallskip
     
     Таким образом, установлена граница для максимальной степени вершины в графе 
$G_{t,r}$. На основании этого результата можно построить оценку центра управления 
бот-сетью. Если существует вершина, степень которой превосходит заданную границу, то 
с вероятностью, близкой к~1, высокая степень этой вершины получена вне условий 
стационарности и других допущений, которые были сделаны для нормального поведения 
сети. 
     
     Предположим теперь, что $C\hm<1$ и математическое ожидание 
${\sf E}X\hm\rightarrow\infty$. Построим оценку числа вершин, имеющих степень больше~$d$, 
при условии, что ${\sf E}X\hm\rightarrow\infty$. С~этой целью оценим и сравним дисперсию 
$DX$ случайной величины~$X$ и $({\sf E}X)^2$. Очевидно, что 
     \begin{multline*}
     ({\sf E}X)^2=N^2(1-B(N-1,d,p_r))^2=\\
     {}=N^2\overline{B}^2(N-1,d,p_r)\,.
     \end{multline*}
Случайную величину~$X$ можно представить в виде:
\begin{equation}
X=\sum\limits_{i=1}^N I_i\,,
\label{e7-gr}
\end{equation}
где $I_i$~--- индикатор события $d(i)\hm>d$. Тогда из~(\ref{e7-gr}) следует:
\begin{multline*}
{\sf E}X^2={\sf E}\left( \sum\limits_{i=1}^N I_i\right) +{\sf E}
\left( 2\sum\limits_{i<j}I_iI_j\right) ={}\\
{}=
N\overline{B}(N-1,d,p_r) +2\sum\limits_{i<j}P(I_iI_j=1)\,.
\end{multline*}
По формуле полной вероятности
\begin{multline*}
P(I_1I_2=1)=p_r\overline{B}^2(N-2,d-1, p_r)+{}\\
{}+(1-p_r)\overline{B}^2(N-2,d,p_r)\,.
\end{multline*}
     
     Рассмотрим разность ${\sf E}X^2$ и $({\sf E}X)^2$. Несложные вычисления приводят к 
выражению:

\pagebreak

\noindent
     \begin{multline*}
{\sf E}X^2-({\sf E}X)^2={}\\
     {}=N\overline{B}(N-1,d,p_r)(1-\overline{B}(N-1,d,p_r))+{}\\
     {}+N(N-1)(1-p_r)p_r b^2(N-2,d,p_r)\,,
     \end{multline*}
где 
$$
b(N-2,d,p_r)=\begin{pmatrix}
N-2 \\ d\end{pmatrix} p_r^d(1-p_r)^{N-2-d}\,.
$$
     
     Предполагалось, что ${\sf E}X\rightarrow\infty$. Тогда
     \begin{multline*}
     \fr{{\sf D}X}{({\sf E}X)^2}=\fr{1}{N\overline{B}(N-1,d,p_r)}\left(
     B(N-1,d,p_r)+{}\right.\\
\left.     {}+\alpha_r(1-p_r)\fr{(N-1) b^2(N-2,d,p_r)}{N\overline{B}(N-1,d,p_r)}\right)\,.
     \end{multline*}
Из предыдущих оценок имеем, что 
\begin{align*}
Nb(N-2,d,p_r) &=O\left(N^{1-C}\right)\,;\\
N\overline{B} (N-1,d,p_r) &= N^{(1-C)(1+o(1))}\alpha_r\left( 1-e^{-\alpha_r\theta}\right)\,.
\end{align*}
Отсюда следует, что
$$
\fr{{\sf D}X}{({\sf E}X)^2} =O\left(N^{C-1}\right)\,,\enskip C<1\,.
$$
     
     Воспользуемся следствиями~4.32 и~4.33 из работы~\cite{15-gr}. Получаем, что с 
вероятностью, стремящейся к~1, $0<X$ и отношение $X/({\sf E}X)\rightarrow 1$. 
     Это означает, что при $C<1$ с вероятностью, близкой к~1, существует вершина степени 
больше~$d$ и число таких вершин совпадает с математическим ожиданием~$X$. 
     

\section{Заключение}
     
     В ходе решения поставленных в данной работе задач появилось много новых 
направлений, которые заслуживают отдельного внимания. В~данной работе исследовано 
поведение больших степеней в графе, соответствующем фиксированному <<окну>>. 
Естественно, желательно обобщить эти результаты на случай скользящего <<окна>>. 
     
     Поведение графов ин\-тер\-нет-ти\-па часто нельзя считать стационарным. Возникает 
задача анализа <<оконных>> графов в условиях нестационарного поведения сети. 
Перечень проблем, возникших при данном исследовании, не исчерпывается данными 
двумя задачами.

{\small\frenchspacing
{%\baselineskip=10.8pt
\addcontentsline{toc}{section}{Литература}
\begin{thebibliography}{99}
     
\bibitem{1-gr}
\Au{Kolaczyk E.\,D.} Statistical analysis of network data: Methods and models.~--- Springer 
Science\;+\;Business Media, LLC, 2009. 386~p. 
\bibitem{2-gr}
\Au{Erd$\ddot{\mbox{o}}$s P., R$\acute{\mbox{e}}$nyi~A.} On the evolution of random 
graphs~// Publ. Math. Inst. Hungarian Acad. Sci., Ser.~A, 1960. Vol.~5. P.~17--61.
\bibitem{3-gr}
\Au{Степанов В.\,Е.} О вероятности связности случайного графа $g_m(t)$~// Теория 
вероятностей и ее применения, 1970. Т.~15. №\,1. С.~55--67.
\bibitem{4-gr}
\Au{Степанов В.\,Е.} Фазовый переход в случайных графах~// Теория вероятностей и ее 
применения, 1970. Т.~15. №\,2. С.~187--203.
\bibitem{5-gr}
\Au{Степанов В.\,Е.} Структура случайных графов $g_n(x\vert h)$~// Теория вероятностей 
и ее применения, 1972. Т.~17. №\,3. С.~227--242.

\bibitem{6-gr}
\Au{Bollobas B.} Random graphs.~--- London: Academic Press, 1985.

\bibitem{9-gr} %7
\Au{Kleinberg J., Kumar S., Raghavan~P., Rajagopalan~S., Tomkins~A.} The web as a graph: 
measurements, models, and methods~// Conference (International) on Combinatorics and 
Computing Proceedings ~--- Berlin: Springer, 1999. Lecture Notes in Computer Science. 
Vol.~1627. P.~1--18.

\bibitem{7-gr} %8
\Au{Колчин В.\,Ф.} Случайные графы.~--- М.: Физматлит, 2000. 256~с.



\bibitem{10-gr} %9
\Au{Kumar R., Raghavan P., Rajagopalan~S., Sivakumar~D., Tomkins~A., Upfal~E.}
Stochastic models for the web graph~// 42nd Annual IEEE Symposium on the Foundations of 
Computer Science Proceedings, 2000. Vol.~41. P.~57--65. 

\bibitem{8-gr} %10
\Au{Chung F., Lu L., Dewey~T., Galas~D.} Duplication models for biological networks~// 
J.~Comput. Biology, 2003. Vol.~10. No.\,5. P.~677--687. 

\bibitem{11-gr}
\Au{Павлов Ю.\,Л., Степанов М.\,М.} Об асимптотических свойствах случайных графов 
<<ин\-тер\-нет-типа>>~// Обозрение прикладной и промышленной математики, 2005. 
Т.~12. №\,3. С.~677.
\bibitem{12-gr}
\Au{Степанов М.\,М.} О~предельных распределениях степеней узлов в случайных графах 
ин\-тер\-нет-типа~// Методы математического моделирования и информационные 
технологии: Тр. Института прикладных математических исследований Карельского 
научного центра РАН.~--- Петрозаводск: КарНЦ РАН, 2005. Вып.~6. С.~235--242.
\bibitem{13-gr}
\Au{Павлов Ю.\,Л.} Предельное распределение объема гигантской компоненты в 
случайном графе ин\-тер\-нет-типа~// Дискретная математика, 2007. Т.~19. №\,3. 
С.~22--34.
\bibitem{14-gr}
\Au{Феллер В.} Введение в теорию вероятностей и ее приложения.~--- 2-е изд.~--- 
М.: Мир, 1967. Т.~1.

\label{end\stat}

\bibitem{15-gr}
\Au{Alon N., Spencer~J.} The probabilistic method.~--- 2nd ed.~--- New York: Jonh Wiley \& Sons, 
2000.
\end{thebibliography}
}
}

\end{multicols}