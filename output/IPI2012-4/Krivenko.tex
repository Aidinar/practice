
\def\stat{krivenko}

\def\tit{ПРЕДВАРИТЕЛЬНАЯ ОБРАБОТКА ПРИ~РАСПОЗНАВАНИИ ТЕКСТОВ 
ПО~ИЗОБРАЖЕНИЮ НИЗКОГО КАЧЕСТВА}

\def\titkol{Предварительная обработка при~распознавании текстов 
по~изображению низкого качества}

\def\autkol{М.\,П.~Кривенко}

\def\aut{М.\,П.~Кривенко$^1$}

\titel{\tit}{\aut}{\autkol}{\titkol}

%{\renewcommand{\thefootnote}{\fnsymbol{footnote}}\footnotetext[1]
%{Работа поддержана Российским фондом фундаментальных
%исследований (проекты 11-01-00515а, 11-01-12026-офи-м,
%12-07-00109a, 12-07-00115a), Министерством образования и науки (госконтракт
%16.740.11.0133).}}


\renewcommand{\thefootnote}{\arabic{footnote}}
\footnotetext[1]{Институт проблем информатики Российской академии наук, mkrivenko@ipiran.ru}

       \Abst{Рассматриваются методы предварительной обработки изображений текста, 
включающей решение задач коррекции наклона и выделения строк; при этом распознаваемое 
изображение обладает низким качеством и получено с высоким разрешением. При условии, что 
яркость пикселов строк знаков, хоть и незначительно, но отличается от яркости пикселов фона, 
предлагаются и анализируются процедуры коррекции наклона и выделения строк текста.}
       
       \KW{распознавание текста; предварительная обработка изображения; коррекция наклона; 
выделение строк текста}


\vskip 14pt plus 9pt minus 6pt

      \thispagestyle{headings}

      \begin{multicols}{2}

            \label{st\stat}
            
\section{Введение}


      
      Задача восстановления текста по изображению документа является крайне 
сложной, в связи с чем обычно применяется декомпозиция процесса ре\-ше\-ния задачи 
путем выделения этапов превращения изображения, содержащего текстовые фрагмен\-ты, в 
наборы текстов. На рис.~1 приведена типовая схема обработки данных (см., например,~[1,
под\-разд.~2.1]).



      Далее предполагается, что исходное изображение представляет собой набор 
пикселов, каждый из которых передает градации яркости серого (оттенки серого). Каждая 
подсистема распознавания текста, получая данные от предыдущего этапа, преобразует их 
в выходные, которые обрабатываются на следующем этапе. Такая последовательная схема 
обработки изображений обеспечивает возможность получать наилучшие решения для 
каждого этапа. С~целью создания предпосылок для того, чтобы после объединения всех 
частных, лучших по отдельности подсистем можно было получать эффективные решения, 
вводится ядро системы~--- база знаний.
      
      В данной работе рассматривается этап предобработки, при этом из обычного 
состава этого этапа выделены задачи коррекции наклона (в англоязычной литературе для 
этого вида аномальности изоб\-ра\-же\-ния текста принят термин skew) и выделение 
изображений строк текста. 

\end{multicols}

\begin{figure}[h] %fig1
\vspace*{-9pt}
 \begin{center}
 \mbox{%
 \epsfxsize=116.703mm
 \epsfbox{kri-1.eps}
 }
 \end{center}
 \vspace*{-9pt}
\Caption{Традиционная схема распознавания текста по его изображению}
\end{figure}


      
      \pagebreak
      
      \begin{figure} %fig2
      \vspace*{1pt}
 \begin{center}
 \mbox{%
 \epsfxsize=160mm
 \epsfbox{kri-2.eps}
 }
 \end{center}
 \vspace*{-9pt}
      \Caption{Фрагмент распознаваемого текста}
       \end{figure}
       
       \begin{multicols}{2}
      
      \textbf{Коррекция наклона строк.} Нормальным изображением считается то, в 
котором линии документа (вдоль которых идут отдельные знаки) являются 
горизонтальными или вертикальными в зависимости от языка. Уклонение может быть 
преднамеренным, чтобы подчеркнуть важные детали в документе, либо неумышленным в 
силу тех или иных условий получения изображения. В~любом случае оно должно быть 
устранено, иначе точность последующих процессов~--- сегментации и классификации~--- 
резко упадет. В~данной работе рассматривается случай, когда для всех блоков текста 
характерна единая ориентация и без потери общности можно считать, что речь идет о 
горизонтальных строках. Коррекция уклонения распадается на решение двух задач: 
определение угла наклона и собственно выравнивание изображения. 
      
      \smallskip
      
      \textbf{Выделение изображений строк (сегментация изоб\-ра\-же\-ния).} 
Относительно структуры текстовых фрагментов предполагается следующее:
      \begin{itemize}
\item знаки компонуются в строки~--- удлиненное (вытянутое) в горизонтальном 
направлении изображение, пикселы которого темнее пикселов, лежащих вне строк;
\item строка имеет практически постоянную высоту, которая определяется 
расстоянием между двумя ближайшими переходами от более темных к менее темным 
пикселам;
\item начало и конец строки в горизонтальном на\-прав\-ле\-нии являются переменными 
величи\-нами.
\end{itemize}
      
      Из сделанных предположений следует, что критерием для выделения строк может 
служить яркость пикселов, а наиболее простой техникой для этого\linebreak оказывается пороговая. 
Глобальные алгоритмы порого\-вой обработки используют единственный порог для всего 
изображения. Локальные (перестраиваемые, адаптивные) методы вычисляют 
определяемый порог для каждого пиксела с использованием яркости пикселов по 
соседству. 
      
      
       
      \smallskip
      \textbf{Особенности рассматриваемых задач.} Далее речь пойдет о 
предобработке изображений в условиях их низкого качества. В~качестве примера 
рас\-смот\-рим реальное изображение (на рис.~2 приведен фрагмент одной из страниц 
текста), низкое качество которого иллюстрирует рис.~3: нет ярко выраженных 
экстремумов распределения яр\-кости, соответствующих пикселам знаков и пикселам фона. 
Количественно это можно выразить следующим образом: аппроксимируем плот\-ность 
распределения яркости серого пикселов с помощью смеси двух нормальных 
распределений, а именно: $f^*(u)=0{,}28\varphi (u,180,202)+0{,}72\varphi(u,197,56)$. Здесь
первый элемент смеси~--- плотность нормального распределения $\varphi(u,180,202)$ со 
средним 180 и дисперсией 202~--- соответствует пикселам знаков, а второй~--- пикселам 
фона. Если опереться на это представление и найти порог бинаризации изображения так, 
чтобы минимизировать ошибку классификации пикселов на черные и белые, то получим 
значение порога $t=184$ и оценку ошибки классификации в 14\%, т.\,е.\ достаточно 
большой уровень ошибки.



      Пытаясь скомпенсировать недостатки объекта анализа, подчас идут на повышение 
разрешения  изображения, при этом высота строчных знаков мо-\linebreak\vspace*{-12pt}
\vspace*{18pt}
\begin{center}  %fig3
  \mbox{%
 \epsfxsize=76.298mm %046mm
 \epsfbox{kri-3.eps}
 }
 \end{center}
% \vspace*{6pt}
{{\figurename~3}\ \ \small{Распределение значений яркости серого цвета для реального изображения}}



%\pagebreak

%\vspace*{12pt}

\addtocounter{figure}{1}


\noindent
жет достигать 
50~пикселов. Указанные особенности обрабатываемых изображений сразу же 
ограничивают спектр возможных методов анализа, а также предъявляют повышенные 
требования к чувствительности используемых подходов. 
      
      При построении процедур распознавания предполагается, что яркость пикселов для 
изображений знаков текста все же отличается от фона (далее для конкретности речь 
пойдет о темных пикселах знаков на более светлом фоне), но различие является 
несущественным.
      
      В соответствии с типизацией задач распознавания текстов, а также приводимыми 
примерами в~[2, под\-разд.~1.1] описанные свойства обрабатываемых текстов 
позволяют характеризовать задачи их обработки как крайне сложные. 
\section{Коррекция наклона изображений строк текста}
      
      Обычно различают два типа наклона в документах: глобальный и локальный. 
Первая категория является самой популярной при исследованиях, для нее было 
предложено множество методов, однако далеко не все проблемы решены, особенно для 
документов с рисунками, графиками и фигурами или для шрифтов различных размеров 
(см., например,~\cite{8-kr}). С~другой стороны, некоторые исследователи разработали 
методы для локальных уклонений, которые включались в технологию точного 
восстановления текстов, но они являются весьма затратными с вычислительной точки 
зрения~\cite{7-kr}. 
      
      Выделение глобальных уклонений обычно делят согласно основному 
применяемому подходу на четыре основные категории методов: проекционный профиль 
(projection profile), преобразование Хафа (Hough), группирование ближайших соседей 
(nearest neighbor clustering), кросс-кор\-ре\-ля\-ция между сечениями (interline cross 
correlation).
      
      \smallskip
      
      \textbf{Подход на основе проекционных профилей.} Традиционный и простой 
подход в выявлении угла, искажающего изображение документа. В~диапазоне ожидаемых 
углов вычисляются отдельные проекционные профили, из которых выбирается тот, 
который наиболее ярко отражает различие между особенностями линий с пикселами 
знаков и линий с пикселами фона. Для снижения высоких вычислительных затрат были 
предложены различные модификации, суть которых состоит либо в уменьшении 
количества данных, вовлеченных в вы\-чис\-ле\-ние профиля, либо в оптимизации стратегии 
поиска наилучшего решения.
      
      \smallskip
      
      \textbf{Подход на основе преобразований Хафа.} Он нацелен на обнаружение 
линий и кривых на цифровых изображениях. Используется в предположении, что линии, 
характеризующие направленность текста, содержат достаточно большое количество 
пикселов. Чтобы повысить вы\-чис\-ли\-тель\-ную эффективность, применяются разновидности 
базовой идеи использования преобразования Хафа, которые сокращают число точек в 
пространстве образов.
      
      \smallskip
      
      \textbf{Подход на основе ближайших соседей.} Методы этого класса используют 
общие предположения, что знаки в строке текста выровнены и близки друг к другу. Они 
реализуют восходящий, от частного к общему, процесс формирования представления 
строки из отдельных знаков или их деталей, а затем использования этого представления 
для оценки наклона. 
      
      \smallskip
      
      \textbf{Подход на основе сечений изображения.} Это название подхода 
соответствует англоязычному словосочетанию <<interline cross correlation clustering>> 
которое достаточно трудно перевести коротко. Сам подход основан на предположении, 
что изображение текста без наклона строк представляет однородную горизонтальную 
структуру. Тогда оценка наклона строится путем измерения изменений характеристик 
пикселов вдоль вертикальных линий изображения.
      
      Краткий обзор происхождения и сути пе\-ре\-чис\-лен\-ных подходов, их сравнительный 
анализ достаточно полно представлен в~\cite{2-kr} (54~источника с 1972 по 1997~г.)\ и 
в~\cite{6-kr} (58~источников с 1986 по 2011~гг.).
      
      Несмотря на примеры эффективного использования методов коррекции наклона 
строк, можно выделить ряд общих проблем при их применении: 
      \begin{itemize}
\item необходимость фактически использовать двоичное (бинаризованное) или 
достаточно четкое в оттенках серого изображение; 
\item возможность получать грубые оценки угла наклона; 
\item подчас высокая вычислительная сложность; 
\item явное или неявное использование дополнительных параметров, существенно 
влияющих на качество получаемых решений;
\item ориентация на специфические области применения (в частности, ограничения на 
размер шрифта, требование к определенному разрешению изображения, зависимость 
от языка, невысокая эффективность при работе с рукописным текстом);
\item оценка эффективности предлагаемых методов проводится на специфических 
наборах данных, не являющихся общедоступными, как следствие, затруднительно 
провести сравнительный анализ эффективности, понять место предлагаемых решений 
в ряду уже существующих. 
\end{itemize}

      В данной работе рассматривается использование проекционного профиля по 
следующим основным причинам: 
      \begin{itemize}
\item он связан с аккумулированием информации вдоль горизонтальных линий 
изображения, и это создает предпосылки для того, чтобы <<справиться>> с 
изображением низкого качества;
\item метод оценивания угла наклона на основе профиля хорошо зарекомендовал себя 
с точки зрения чувствительности; 
\item с помощью проекционного профиля можно строить и процедуры выделения 
строк.
\end{itemize}

      Введем основные обозначения. Пусть $I(y,x)$~--- яркость серого для пиксела, 
располагающегося на пересечении $y$-й горизонтальной линии и $x$-й вертикальной 
линии изображения (пиксел в $x$-м столбце на $y$-й линии изображения). Проекционный 
профиль по определению есть $\mathrm{PP}(y)\hm=\sum\limits_x I(y,x)$, для его нормированной 
формы примем следующее обозначение: 
      $$
      p(y)=\fr{\mathrm{PP}\left(y\right)}{w}\,,
      $$
где $w$~--- постоянная ширина строки изображения в пикселах. Для последовательности 
$p(y)$ можно вычислить выборочные характеристики: $\overline{p}\hm=\sum\limits_y 
p(y)$ и $s^2\hm=\sum\limits_y (p(y)-\overline{p})^2$.
      
      Для пояснения того, как работают введенные выборочные характеристики при 
оценивании угла наклона строк текста, на рис.~4 приведен фрагмент графика поведения 
$p(y)-\overline{p}$ для значений $y\hm=250, 251, \ldots , 360$ для реального изображения 
в трех случаях: без наклона строк (кривая~$A$), с углом\linebreak\vspace*{-12pt}

\noindent
\vspace*{1pt}
\begin{center}  %fig4
 \mbox{%
 \epsfxsize=67.136mm
 \epsfbox{kri-4.eps}
 }
 \end{center}
% \vspace*{6pt}
{{\figurename~4}\ \ \small{Зависимость значений нормированного и центрированного проекционного профиля от 
номера линии для изображения без наклона (кривая~$A$), с наклоном в 0,5$^\circ$ (кривая~$B$) и 
с наклоном в 1,0$^\circ$ (кривая~$C$)}}



%\pagebreak

%\vspace*{12pt}

\addtocounter{figure}{1}

%\vspace*{9pt}
% \vspace*{6pt}
\noindent
\begin{center}
{{\tablename~1}\ \ \small{Характеристики проекционных профилей}}
\end{center}
%\vspace*{3pt}

%\begin{table*}
{\small
      \begin{center}
      \tabcolsep=14pt
      \begin{tabular}{|c|c|c|}
      \hline
      &&\\[-9pt]
Угол наклона строк текста&$\overline{p}$&$s^2$\\
\hline
0,0$^\circ$&192,1&32,2\\
0,5$^\circ$&192,1&24,8\\
1,0$^\circ$&192,1&15,9\\
\hline
\end{tabular}
\end{center}}

\vspace*{12pt}
%\end{table*}
\addtocounter{table}{1}


\noindent
 наклона строк в 0,5$^\circ$ 
(кривая~$B$), с углом наклона строк в 1,0$^\circ$ (кривая~$C$). 

      

      
      Для проекционных профилей указанного изоб\-ра\-же\-ния в упомянутых случаях 
табл.~1 содержит значения выборочных характеристик. При увеличении отклонения 
абсолютного значения угла наклона строк от нулевого значения происходит следующее:
    \begin{itemize}
\item среднее значение $\overline{p}$ для последовательности $p(y)$ практически не 
изменяется (незначительное
влияние оказывает дискретизация при повороте 
изображения), поэтому все графики $p(y)$ колеблются около одной и той же прямой;
\item локальные экстремальные значения $p(y)$ на интервалах, отвечающих 
положениям строк ($y\hm=255, 256, \ldots, 305$ на рис.~4) и положениям межстрочных 
интервалов ($y\hm=306, 307, \ldots , 345$ на рис.~4), убывают;
\item поведение $p(y)$ на интервалах, отвечающих положениям строк и положениям 
межстрочных интервалов, по форме меняется от <<прямоугольного>> к 
<<треугольному>>, сливаясь в пределе с осью абсцисс.
\end{itemize}



      Тогда становится понятным, что $s^2$ может стать мерой горизонтальности строк 
текста. Также из приведенной иллюстрации следует, что $s^2$ предпочтительней 
интерпретировать не как рассеяние, а как интегральную мощность некоторого процесса 
$p(y)$, при этом нет необходимости в предварительной бинаризации изображения (для 
сравнения см., например,~[1, п.~2.4.2.1]). 


      
      Таким образом, получаем, что искомый угол $\alpha^*$ поворота для коррекции 
наклона будет решением задачи $\alpha^*\hm=\mathrm{arg}\,\max\limits_\alpha s^2(\alpha)$, 
где мера $s^2(\alpha)$ посчитана для предварительно повернутого на угол~$\alpha$ 
исходного изображения.
\begin{table*}\small
\begin{center}
\Caption{Характеристики оценок угла наклона строк}
\vspace*{2ex}

\begin{tabular}{|c|c|c|c|c|}
\hline
$-\alpha_0$, &\multicolumn{2}{c|}{Псевдореальные данные (10 
изображений)}&\multicolumn{2}{c|}{Реальные данные (10 изображений)}\\
\cline{2-5}
&&&&\\[-9pt]
пиксел&$\mathrm{Ave} \left\{ \alpha^*-(-\alpha_0)\right\}$&$\sqrt{\mathrm{Ave}\left\{ (a^*-(-
\alpha_0))^2\right\}}$&$\mathrm{Ave} \left\{ \alpha^*-(-\alpha_0)\right\}$&$\sqrt{\mathrm{Ave}\left\{ (a^*-(-
\alpha_0))^2\right\}}$\\
\hline
0&0,4&0,5&$-0{,}3$\hphantom{$-$}&0,9\\
1&$-0{,}7$\hphantom{$-$}&0,6&$-1{,}4$\hphantom{$-$}&0,8\\
2&0,0&0,0&0,0&0,0\\
4&0,0&0,0&0,0&0,0\\
8&0,0&0,0&0,0&0,0\\
\hline
\end{tabular}
\end{center}
\end{table*}
      
      Для выяснения возможности практического использования описанной процедуры 
коррекции накло\-на на изображениях с рассматриваемыми свойствами был проведен 
анализ ее качества. Для исследования бралось изображение текста с горизонтальными 
строками текста, оно искусственно искажалось путем поворота на определенный 
угол~$\alpha_0$, а затем сравнивался результат оценивания угла наклона и известное 
значение. Подобные действия проводились для 10 различных изоб\-ра\-же\-ний. В~качестве 
итоговых характеристик качества оценок наклона рассматривались следующие: $\mathrm{Ave} 
\left\{ \alpha^*-(-\alpha_0)\right\}$ и $\sqrt{\mathrm{Ave}\left\{ (a^*-(-\alpha_0))^2\right\}}$. 
При\-нимая 
во внимание, что реальные изображения подвергались коррекции наклона строк текс\-та,\linebreak 
описанный анализ сначала проводился для псев\-дореальных данных~--- искусственно 
созданных изоб\-ра\-же\-ний со строками текста без наклона, характеристики которых в 
смыс\-ле отличия яркостей пикселов знаков и пикселов фона были похожи на реальные. 
Угол поворота измерялся числом пикселов, на которое сдвигался по вертикали самый 
правый пиксел некоторой линии изображения относительно самого левого.


      
      Представленные в табл.~2 результаты позволяют сделать следующие выводы:
      \begin{itemize}
\item описанный метод обработки проекционного профиля приводит к достаточно 
хорошим результатам (заметим, что наклон строки в 1~пиксел соответствует всего 
лишь углу 0,02$^\circ$); 
\item ненулевые значения характеристик оценки наклона для реальных данных при 
$\alpha_0\hm=0$ являются следствием дискретизации при повороте изображения. 
В~частности, для отдельного изображения при его повороте на $-1$, а затем снова на 
1~пиксел количество пикселов, не совпавших по яркости именно из-за дискретизации, 
оказалось равным 6947 из общего числа 10$^7$ (0,07\%). Таким образом, 
дополнительным средством повышения качества процедур коррекции наклона могут 
стать приемы уточнения процедур поворота, хотя соответствующее повышение не 
является столь уж существенным.
      \end{itemize}
      
      Наряду с проанализированной процедурой коррекции наклона строк была сделана 
попытка построить и использовать иные методы:
      \begin{itemize}
\item разделение элементов смеси при аппроксимации распределения $p(y)$ с 
помощью двухэлементной смеси нормальных распределений проекционных профилей. 
При этом в качестве меры отсутствия наклона рассматривалась оценка ошибки 
классификации для двух классов (двух элементов смеси). Основанием для принятия 
решения становилось следующее правило: в случае наилучшего разделения двух 
классов изображение должно иметь ярко выраженную горизонтальную структуру 
текста;
\item сравнительный анализ фрагментов изображения с помощью проверки 
однородности подвыборок вдоль горизонтальных линий изображения с последующим 
анализом числа успехов (принятие предположения об однородности) для 
последовательности испытаний для отдельных горизонтальных линий изображения.
\end{itemize}

      Судя по проведенным автором экспериментам, явное усложнение алгоритмов 
обработки данных не дает очевидных преимуществ по сравнению с базовым методом. 
Более того, оба дополнительных метода (разделение смеси и сравнительный анализ 
фрагментов) проявляют меньшую устойчивость в области малых углов искажения 
наклона (0, $\pm1$, $\pm 2$~пиксела). По-видимому, основная причина здесь в том, что 
обрабатываемые изображения обладают существенной зависимостью значений яркости 
отдельных пикселов, что отражается и на проекционных профилях. Это заставляет с 
осторожностью относиться к обращению с обрабатываемыми данными как с выборкой. 

\renewcommand{\figurename}{\protect\bf Таблица}
\setcounter{figure}{2}
\begin{figure*}[b]
\Caption{Варианты шаблонов при делении изображения на 3 вертикальные полосы}
\vspace*{2ex}

 \begin{center}
 \mbox{%
 \epsfxsize=154.663mm
 \epsfbox{kri-t-3.eps}
 }
 \end{center}
 \vspace*{-9pt}
\end{figure*} 

\addtocounter{table}{1}

\renewcommand{\figurename}{\protect\bf Рис.}
\renewcommand{\tablename}{\protect\bf Таблица}

\section{Выделение изображений строк текста}
      
      Главным образом, существует три основных категории подходов к решению 
рассматриваемой задачи: проекционный профиль; преобразование Хафа; заполнение, или 
замазывание (smearing).
      
      \smallskip
      
      \textbf{Подход на основе проекционных профилей.} Соответствующие методы 
включают нахождение интегральных характеристик отдельных горизонтальных линий~--- 
характеристик, специфичных для строк знаков и межстрочных интервалов. После 
рассмотрения их значений по вертикали (проекции на вертикальную линию) становится 
возможным выявлять позиции и размеры по высоте отдельных строк текста.
      
      \smallskip
      
      \textbf{Подход на основе преобразований Хафа.} Обнаружение линий в 
цифровых изображениях с помощью пространства образов дополняется правилами 
проверки гипотез в пространстве изображений. При этом отбираются такие образования 
(группы) точек в пространстве Хафа, которые являются наиболее представительными с 
точки зрения количества элементов изображения, выстроенных вдоль определенной 
линии.
      
      \smallskip
      
      \textbf{Подход на основе заполнения.} Для бинаризо\-ванных документов метод 
заполнения ярче всего воплощается в процедуре RLSA (Run-Length Smoothing Algorithm), 
когда вдоль горизонтального на\-прав\-ле\-ния последовательность, состоящая из белых 
пикселов между черными, заполняется черными пикселами, если расстояние между 
граничными черными пикселами не превосходит определенного порога. Далее строится 
ограничивающий связанные компоненты прямоугольник, он и определяет текстовую 
строку. Существуют разновидности этого метода, приспособленного к изображениям с 
уровнями яркости.
      
      Краткий обзор происхождения и сути описанных подходов, их сравнительный 
анализ изложен в~\cite{3-kr} (60~источников с 1962 по 2004~гг.)\ и в~\cite{4-kr} 
(42~источника с 1982 по 2008~гг.).
      
      Проблемы при применении описанных подходов практически те же, что указаны в 
связи с коррекцией наклона строк. В~первую очередь к ним следует отнести следующие: 
необходимость этапа бинаризации изображения, привнесение дополнительных 
требующих задания параметров, отсутствие сравнительного анализа методов оценки 
эффективности предлагаемых методов на общедоступных данных. 
      
      В данной работе предложена схема выделения строк текста с помощью следующих 
идей:
      \begin{itemize}
\item основу составляет использование проекционных профилей. Для повышения 
качества при-\linebreak нимаемых решений изоб\-ра\-же\-ние делится (фраг\-мен\-ти\-ру\-ет\-ся) на несколько 
вертикальных полос, для каждой из которых строится свой проекционный профиль. 
С~помощью этого приема удается учитывать возможность появления строк знаков с 
различной структурой (на\-при\-мер, наличие неполных строк). Ширина полос при 
подсчете проекционных индексов принимается одинаковой (если ширина изоб\-ра\-же\-ния 
не кратна числу полос, то либо отдельные пикселы отбрасываются, либо 
обрабатываются несколько различающиеся значения ширины фрагментов);
\item выделение строк осуществляется путем сравнения расстояний между 
проекционными профилями горизонтальных полос изоб\-ра\-же\-ния и эталона, 
задаваемого с помощью шаблона;
\item шаблон описывает априорное представление о строках текста. Он может быть 
простым, состоящим из фиксированного числа полных строк с определенными 
яркостями пикселов в них, или комбинированным, состоящим из набора 
фрагментированных строк с определенными яркостями пикселов во фрагментах.
\end{itemize}

      Варианты шаблонов при делении изображения на три вертикальные полосы, 
рассмотренные в данной работе, схематично представлены в табл.~3. Светлые клетки 
соответствуют пикселам фона, темные~--- пикселам знаков текста. Так, комбинированный 
шаблон из двух строк состоит из строки пикселов фона и строки, содержащей пикселы 
знаков, причем в последнем случае это может быть один из трех вариантов: темный 
фрагмент и два светлых фрагмента, два темных и один светлый, все три темных 
фрагмента. 

      
      
      Шаблон задается с помощью следующих параметров: 
      \begin{itemize}
\item ширина строки, которая определяется размерами изображения и задается до 
распознавания;
\item количество фрагментов, оно определяется сложностью структуры строк текста и 
задается исследователем до распознавания. Увеличение этого параметра повышает 
качество предварительной обработки и сопровождается ростом временной сложности 
алгоритмов обработки данных;
\item характеристики нормированного проекционного профиля $p_C(y)$ строки знаков 
и $p_B(y)$ строки фона, которые подлежат оцениванию по изображениям;
\item высота строки знаков $h_C$ и высота строки фона $h_B$, они подлежат 
оцениванию по изображениям.
\end{itemize}
      
      В данной работе характеристиками нормированного проекционного профиля 
выступали средние значения яркости серого $p_C$ и~$p_B$. Соответствующие оценки 
находятся с помощью представления $p^*(y)$ в виде смеси двух нормальных плотностей 
так, что средние элементов смеси упорядочены по возрастанию $\mu_1^*\hm<\mu_2^*$. 
Тогда $p_C^*\hm=\mu_1^*$ и $p_B^*\hm=\mu_2^*$. Если наряду с изображениями текста 
до распознавания доступны изображения строк только фона, то можно применить более 
тонкую процедуру: представить распределение $p(y)$ в виде смеси более чем двух 
нормальных распределений (тем самым улучшая аппроксимацию реального 
распределения). Затем расщепить эту смесь на две составляющие, одна из которых 
соответствует нормированному проекционному профилю знаков, а другая~--- 
фона~\cite{9-kr}. И,~наконец, в качестве характеристик шаблона взять моменты 
выделенных составляющих смеси.


      
      Оценивание высот строки знаков и строки фона шаблона реализуется с помощью 
следующих шагов:
      \begin{itemize}
\item получение проекционного профиля $p(y)$ по всей ширине изображения текста;
\item бинаризация последовательности $p(y)$ путем аппроксимации распределения 
$p(y)$ с помощью смеси двух нормальных распределений, затем нахождение порога 
для разделения элементов полученной смеси и, наконец, формирование 
последовательности из~0 и~1;
\item выделение серий двух типов (0-сер\-ий и 1-се\-рий) и подсчет частот 
встречаемости длин серий определенного типа;
\item оценивание высот строк знаков и строк фона путем выбора длин серий, частоты 
которых являются наибольшими.
\end{itemize}

      Таким образом, основными шагами обработки исходных изображений становятся 
следующие:
      \begin{itemize}
\item определение характеристик шаблона, для чего проводится оценивание высот 
чередующихся строк из темных (строк знаков) и светлых (строк фона) пикселов. 
Предполагается, что строки имеют одинаковую высоту, но не обязательно 
совпадающую для строк знаков и фона;
\item определение положений строк на отдельных изображениях.
\end{itemize}

      Определение положений строк знаков включает следующие действия:
      \begin{itemize}
\item для каждого возможного положения строки знаков нахождение значения меры 
близости между фрагментом изображения, соответствующим этой строке, и 
выбранным эталоном, характеризующим строчную структуру текста;
\item отбор из полученных на предыдущем шаге значений мер близости наименьшего 
значения так, чтобы сформировалось полное изображение страницы.
      \end{itemize}
      
      В качестве меры близости в случае простого шаблона рассматривается квадрат 
различия значений нормированного проекционного профиля для фрагмента изображения, 
соответствующего по структуре и размеру выбранному шаблону, и средней яркости 
серого для элементов шаблона. Для комбинированного шаблона мера близости 
вы\-чис\-ля\-ет\-ся как минимум мер близости для каждого частного вида шаблона.

 \begin{table*}\small
       \begin{center}
       \Caption{Характеристики процедуры выделения строк}
       \vspace*{2ex}
       
       \begin{tabular}{|c|c|c|c|c|c|c|c|c|}
       \hline
Строк &\multicolumn{4}{c|}{Простой шаблон}&\multicolumn{4}{c|}{Комбинированный 
шаблон}\\
\cline{2-9}
в шаблоне&$A$&$B$&$\delta$&\ $\sigma$\ &\ $A$\ &\ $B$\ &\ $\delta$\ &\ $\sigma$\ \\
\hline
1&35&2&0,3&3,5&9&2&0,3&3,4\\
2&32&1&$-0{,}5$\hphantom{$-$}&2,3&7&1&$-0{,}5$\hphantom{$-$}&2,3\\
3&26&0&0,2&2,1&3&0&0,1&2,1\\
\hline
\end{tabular}
\end{center}
\end{table*}

      
      Наибольшие сложности возникают при реализации этапа отбора наименьших 
значений мер близости, что связано с наличием множества ложных локальных 
минимумов. Правда, их число уменьшается с ростом сложности шаблона. Но в любом 
случае приходится отказываться от последовательной по линиям изображения стратегии 
обнаружения строк знаков и обращаться к процедуре, шаг за шагом выделяющей 
наименьшее значение меры близости из оставшихся и исключением из дальнейшего 
рассмотрения линий изображения, соответствующих найденной строке знаков и линиям 
фона, обрамляющим эту строку. 
      
      В табл.~4 сведены результаты экспериментов для 10~реальных изображений 
страниц, содержащих в общей сложности 430~строк текста. Для оценивания качества 
процедуры выделения позиций строк текста использовались следующие характеристики: 
$A$~--- число не найденных строк, $B$~--- число несуществующих строк (найденных 
процедурой позиций строк, не содержащих текст), $\delta$~--- оценка смещения позиции 
строк текста, $\sigma$~--- оценка стандартного отклонения смещения позиции строк 
текста. 
      
      Полученные результаты позволяют сделать следующие выводы:
      \begin{itemize}
\item несмотря на низкое качество изображения, удается получить высокие показатели 
точности выделения строк текста;
\item усложнение шаблона (как за счет увеличения количества входящих в него строк, 
так и путем фрагментации) приводит к явному улучшению качества выделения 
позиций строк;
\item фрагментация шаблона является мощным средством описания сложной 
структуры строки текста (появление отступов, наличие неполных строк, выравнивание 
заголовков и~т.\,п.).
\end{itemize}


\section{Заключение}
      
      Предложенные методы позволяют эффективно проводить предварительную 
обработку распознаваемого текста, несмотря на низкое качество изображения. 
В~ситуации, когда значения яркости серого для пикселов знаков и фона все же 
различаются, удается использовать идеи проекционных профилей. При этом приходится 
отказываться от бинаризации исходного изображения, усложнять и вводить новые приемы 
обработки данных (использование модели смеси распределений при анализе 
проекционных профилей, введение поэтапного оценивания неизвестных характеристик 
строчной структуры текста, фрагментацию изображения). Важным моментом 
предложенных решений является возможность их обобщения на случай, когда средние 
значения яркости серого для строк текста и фона практически не отличаются и 
приходится обращаться к анализу локальных взаимосвязей яркости пикселов. 

{\small\frenchspacing
{%\baselineskip=10.8pt
\addcontentsline{toc}{section}{Литература}
\begin{thebibliography}{9}
  
  \bibitem{1-kr}
  \Au{Cheriet M., Kharma N., Liu C-L., Suen~C.\,Y.} Character recognition systems: A~guide 
for students and practitioners.~--- Hoboken, New Jersey: Wiley-Interscience, 2007. 326~p.

  \bibitem{5-kr} %2
  \Au{Mori S., Nishida H., Yamada~H.} Optical character recognition.~--- Hoboken, New 
Jersey: Wiley, 1999. 560~p.

  \bibitem{8-kr} %3
  \Au{Sharif A.\,E., Movahhedinia N.} On skew estimation of Persian/Arabic printed 
documents~// J.~Appl. Sci., 2008. Vol.~8. Is.~12. P.~2265--2271.

  \bibitem{7-kr} %4
  \Au{Saragiotis P., Papamarkos N.} Local skew correction in documents~// Int. J.~Pattern 
Recognition Artificial Intelligence, 2008. Vol.~22. No.\,4. P.~691--710.

  \bibitem{2-kr} %5
  \Au{Hull J.\,J.} Document image skew detection: Survey and annotated bibliography~// 
Document Analysis Systems~II.~--- Singapore: World Scientific, 1998. P.~40--64.

  \bibitem{6-kr} %6
  \Au{Rehman A., Saba T.} Document skew estimation and correction: Analysis of techniques, 
common problems and possible solution~// Appl. Artificial Intelligence, 2011.  Vol.~25. 
No.\,9. P.~769--787.

  \bibitem{3-kr} %7
  \Au{Likforman-Sulem L., Zahour A., Taconet~B.} Text line segmentation of historical 
documents: A~survey~// Int. J.~Document Analysis Recognition, 2006. Vol.~9. No.\,2--4. 
P.~123--128.

  \bibitem{4-kr} %8
  \Au{Louloudis G., Gatos B., Pratikakis~I., Halatsis~C.} Text line and word segmentation of 
handwritten documents~// Pattern Recognition, 2009. Vol.~42. Is.~12. P.~3169--3183.


\label{end\stat}

  \bibitem{9-kr}
  \Au{Кривенко М.\,П.} Расщепление смеси вероятностных распределений на две 
составляющие~// Информатика и её применения, 2008. Т.~2. Вып.~4. С.~48--56.
\end{thebibliography}
}
}


\end{multicols}      