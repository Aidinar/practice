\def\stat{burl+yak}

\def\tit{РАЗРАБОТКА МЕТОДА ФОРМИРОВАНИЯ ПРИЗНАКОВОГО ПРОСТРАНСТВА 
И~МОДЕЛИ ДЛЯ~ОЦЕНКИ И~ПРОГНОЗИРОВАНИЯ АНТРОПОГЕННОГО ВЛИЯНИЯ 
НА~ОКРУЖАЮЩУЮ СРЕДУ (НА~ПРИМЕРЕ ЛЕСНОГО ФОНДА 
НЕФТЕДОБЫВАЮЩЕГО РЕГИОНА)$^*$}

\def\titkol{Разработка метода формирования признакового пространства 
и~модели для~оценки и~прогнозирования}
% антропогенного влияния 
%на~окружающую среду (на~примере лесного фонда 
%нефтедобывающего региона)}

\def\aut{В.\,В.~Бурлуцкий$^1$, А.\,В.~Якимчук$^2$, А.\,В.~Мельников$^3$, 
  А.\,Л.~Царегородцев$^4$,\\  С.\,В.~Волошин$^5$}

\def\autkol{В.\,В.~Бурлуцкий, А.\,В.~Якимчук, А.\,В.~Мельников и~др.} 
%  А.\,Л.~Царегородцев$^4$,  С.\,В.~Волошин$^5$}

\titel{\tit}{\aut}{\autkol}{\titkol}

\index{Бурлуцкий В.\,В.}
\index{Якимчук А.\,В.}
\index{Мельников А.\,В.} 
\index{Царегородцев А.\,Л.}
\index{Волошин С.\,В.}
\index{Burlutskiy V.\,V.}
\index{Yakimchuk A.\,V.}
\index{Melnikov A.\,V.} 
\index{Tsaregorodtsev A.\,L.}
\index{Voloshin S.\,V.}


{\renewcommand{\thefootnote}{\fnsymbol{footnote}} \footnotetext[1]
{Работа выполнена при поддержке Научного фонда ЮГУ (проект 13-01-20/25).}}


\renewcommand{\thefootnote}{\arabic{footnote}}
\footnotetext[1]{Югорский НИИ информационных технологий, г.~Хан\-ты-Ман\-сийск, BurlutskyVV@uriit.ru}
\footnotetext[2]{Югорский государственный университет, г.~Хан\-ты-Ман\-сийск, YakimchukAV@uriit.ru}
\footnotetext[3]{Югорский НИИ информационных технологий, г.~Хан\-ты-Мансийск, andmelnikov1956@yandex.ru}
\footnotetext[4]{Югорский НИИ информационных технологий, г.~Хан\-ты-Ман\-сийск, TsaregorodtsevAL@uriit.ru}
\footnotetext[5]{Югорский государственный университет, г.~Хан\-ты-Ман\-сийск, Voloshinsv@uriit.ru}

\vspace*{-18pt}

    
        
  
  
  \Abst{Работа посвящена разработке системного метода оценки и~прогнозирования 
влияния природных и~антропогенных воздействий на окружающую среду, включающего 
процедуры преобразования исходных информационных массивов, формирования 
нейросетевой модели, ее обучения и~тестирования. Метод применен для анализа последствий 
антропогенных воздействий на окружающую среду в~Хан\-ты-Ман\-сий\-ском автономном 
округе~--- Югре.}
  
  \KW{анализ данных; машинное обучение; нейронные сети; пространственный анализ; 
географические информационные системы; риск-ори\-ен\-ти\-ро\-ван\-ный подход}

\DOI{10.14357/19922264190318} 
  
\vspace*{-3pt}


\vskip 10pt plus 9pt minus 6pt

\thispagestyle{headings}

\begin{multicols}{2}

\label{st\stat}

  
\section{Введение}

\vspace*{-6pt}

  Проблема формирования признакового пространства, математической 
модели и~подходящих алгоритмов машинного обучения для оценки 
и~прогнозирования влияния природных и~антропогенных факторов на 
окружающую среду весьма актуальна как с~теоретической, так и~с~прикладной 
точки зрения. В~особенности решение этой проб\-ле\-мы важно для 
нефтедобывающих регионов, так как производственные технологии в~них 
сопряжены с~загрязнением окружающей среды. 
  
  Первая часть исходной проблемы~--- фор\-ми\-рование признакового 
пространства~--- связана\linebreak с~доступными количественными показателями 
техногенных аварий~[1--3]. Здесь важной является трансформация исходных 
информационных массивов в~цифровые характеристики риска влияния 
антропогенных факторов.
  
  Вторая и~третья части исходной проблемы~--- модель и~алгоритм  
обучения~--- связаны друг с~другом и~сводятся к~задачам классификации, 
в~которых\linebreak используется какая-либо нейросетевая модель\linebreak в~сочетании 
с~подходящим алгоритмом обучения. В~настоящее время разработано большое 
число различных алгоритмов классификации: метод~$k$~ближайших 
соседей~[4], случайный лес~[5], стохастический градиентный спуск~[6], метод 
опорных векторов~[7]. 
  
  В данной работе для решения поставленной проблемы предлагается метод 
трансформации исходных разнородных динамически организованных 
информационных массивов, полученных путем дис\-тан\-ци\-он\-но\-го зондирования 
земной поверхности с~использованием средств 
географических информационных сис\-тем (ГИС).
  
  На основе преобразованных информационных массивов формируется 
трехслойная нейронная сеть и~проводится ее обучение и~тестирование. 
Обученная сеть используется для прогнозирования последствий природных 
и~антропогенных воздействий на окружающую среду. 
  
  Разработанные методы были применены для оценивания и~прогнозирования 
антропогенных рисков на территории Хан\-ты-Ман\-сий\-ско\-го автономного 
округа.

\begin{figure*}[b] %fig1
%\vspace*{8pt}
    \begin{center}  
  \mbox{%
 \epsfxsize=165.986mm 
 \epsfbox{bur-1.eps}
 }
\end{center}
\vspace*{-9pt}
\Caption{Распределение аварий по районам}
\end{figure*}
  

\vspace*{-12pt}
  
\section{Структуризация и~предварительная обработка разнородных 
данных}

\vspace*{-4pt}

  Все единицы информационных массивов привязаны к~ячейкам 
поверхностной сетки. Для каждого признака набора данных была определена 
шкала измерения с~учетом метода обработки: текстовые признаки были 
заменены на числовые, непрерывные значения были нормализованы по 
максимальному значению.
  
  Ключевым атрибутом набора данных служат поверхностные координаты 
соответствующего информационного объекта.
  
  Для проверки корректности заполнения базы использовались методы 
визуального анализа: гистограммы и~картографический анализ.
  
  Для визуального анализа использовался boxplot, компактно изображающий 
распределение. Такой вид диаграммы в~удобной форме показывает медиану 
(или, если нужно, среднее), нижний и~верхний квартили, минимальное 
и~максимальное значение выборки и~выбросы.

\vspace*{-6pt}
  
\section{Модель оценки рисков}

\vspace*{-4pt}

  В результате проведенного анализа была по\-строена 4-слой\-ная нейросетевая 
модель оценки рисков~[8]. В~качестве данных для входного слоя 
использовалась векторизация признаков, что позволило существенно сократить 
число входных нейронов.
  
  Число нейронов во втором слое составило~166, в~третьем~--- 83, в~четвертом~--- 40, 
и~выходной слой был с~двумя нейронами, которые фиксировали 
положительную и~отрицательную реакцию модели.
  
  Информационные массивы были разделены на три части, которые 
использовались для обучения модели, тестирования и~прогнозирования.
  
\section{Применение разработанного метода для оценки 
и~прогнозирования рисков антропогенных факторов 
в~нефтедобывающем регионе}

\vspace*{-14pt}

  В качестве подходящего полигона был избран Хан\-ты-Ман\-cий\-ский 
автономный округ~--- Югра, который лидирует среди нефтедобывающих 
регионов по добыче нефти и~газа. При этом более~95\% территории 
автономного округа представляют собой земли лесного фонда, которые 
находятся под постоянным негативным воздействием предприятий 
нефтегазового комплекса. 

\begin{figure*} %fig2
\vspace*{1pt}
    \begin{center}  
  \mbox{%
 \epsfxsize=163mm 
 \epsfbox{bur-2.eps}
 }
\end{center}
\vspace*{-9pt}
\Caption{Визуализация исходного набора данных}
%\vspace*{-6pt}
\end{figure*}




  
  Для построения модели прогнозирования рисков загрязнения 
использовались данные об авариях и~инцидентах, транспортной 
инфраструктуре, населенных пунктах~[9]. Вся информация привязывалась  
к~5-ки\-ло\-мет\-ро\-вой поверхностной сетке, и~процесс интеграции данных 
заключался в~унификации географических координат различных объектов, 
построении общих классификаторов, удалении дублирующей информации 
и~приведении к~единому виду текстовых признаков.
  
  В~результате был получен первичный массив данных о загрязнениях 
окружающей среды за период~2012--2017~гг. Объем данной информативной 
выборки составил 26\,015~записей. Общее число элементарных участков 
составило~22\,054. 
  
  Рисунок~1 изображает распределение аварий по районам, а также показывает 
медиану (или, если нужно, среднее), нижний и~верхний квартили, минимальное и~максимальное значение выборки и~выбросы.


  Объекты, привязанные к~элементарным участкам, представлены на рис.~2. 
  

  
  На следующем этапе набор данных был проанализирован на наличие 
неинформативных признаков, для этого использовался метод главных 
компонент.
  
  В результате проведенного анализа и~предобработки данных был получен 
базовый набор данных, объем которого составил~22\,054.
  
  Для тестирования модели использовался метод кросс-ва\-ли\-да\-ции 
по~$k$~блокам~[10]. Исходный набор данных разбивался на~10~одинаковых 
по размеру блоков. Из~10~блоков один оставлялся для тестирования модели, 
а~оставшиеся~9~блоков использовались как тренировочный набор. Процесс 
повторялся~10~раз, и~каждый из блоков один раз использовался как тестовый 
набор. В~итоге получилось~10~результатов, по одному на каждый блок, они 
усреднялись или комбинировались ка\-ким-либо другим способом и~дали одну 
оценку. Результаты представлены в~таблице.
  
  
  
  Средняя оценка составила~90,68.
  
  Преимущество такого способа перед случайным сэмплированием в~том, что 
все наблюдения используются и~для тренировки, и~для тестирования\linebreak\vspace*{-12pt}
  
  %\begin{table*}
{\small
  \begin{center}
  \begin{tabular}{|c|c|}
  \multicolumn{2}{p{35mm}}{Оценка методом кросс-валидации}\\
  \multicolumn{2}{c}{\ }\\[-6pt]
  \hline
\tabcolsep=0pt\begin{tabular}{c}Номер\\ модели\end{tabular}&
\tabcolsep=0pt\begin{tabular}{c}Полученная\\ оценка\end{tabular}\\
\hline
1&91,5\\
2& 90,3\\
3&90,8\\
4&91,2\\
5&89,9\\
6&90,6\\
7&91,4\\
8&89,5\\
9&91,2\\
10\hphantom{9}&90,4\\
\hline
\end{tabular}
\end{center}
}
%\end{table*}

\vspace*{12pt}
  
  \noindent
     модели, 
при этом каждое наблюдение используется для тестирования только один раз.
  
  Получившаяся карта с~прогнозированными авариями представлена на рис.~3.
  
  
\begin{figure*} %fig3
  \vspace*{1pt}
    \begin{center}  
  \mbox{%
 \epsfxsize=163mm 
 \epsfbox{bur-3.eps}
 }
\end{center}
\vspace*{-9pt}
  \Caption{Визуализация прогноза аварий}
  \end{figure*} 
  
\vspace*{-6pt} 

\section{Заключение}

\vspace*{-4pt}

  Разработан системный метод оценки и~прогнозирования природных 
и~антропогенных воздействий на окружающую среду, включающий процедуры 
преобразования исходных информационных массивов, формирования 
нейросетевой прогнозирующей модели, ее обучения и~тестирования. Метод 
применен для прогнозирования техногенных аварий в~нефтедобывающем 
регионе~--- Югре. 
  
 {\small\frenchspacing
 {%\baselineskip=10.8pt
 \addcontentsline{toc}{section}{References}
 \begin{thebibliography}{99}

\bibitem{2-bur} %1
\Au{Guikema S.\,D.} Natural disaster risk analysis for critical infrastructure systems: 
An approach based on statistical learning theory~// Reliab. Eng. 
Syst. Safe., 2009. Vol.~94. Iss.~4. P.~855--860. doi: 10.1016/j.ress.2008.09.003.
\bibitem{3-bur} %2
\Au{Шокин Ю.\,И., Москвичев~В.\,В., Ничепорчук~В.\,В.} Методика оценки 
антропогенных рисков территорий и~построения картограмм рисков 
с~использованием геоинформационных сис\-тем~// Вычислительные 
технологии, 2010. Т.~15. №\,1. С.~120--131. 
\bibitem{1-bur} %3
\Au{Гуменюк В.\,И., Кармишин~А.\,М., Киреев~В.\,А.} О~количественных 
показателях опасности техногенных аварий~// На\-уч\-но-тех\-ни\-че\-ские 
ведомости СПбГПУ, 2013. №\,2(171). С.~281--288.
\bibitem{4-bur}
\Au{Колесенков А.\,Н., Костров~Б.\,В., Ручкин~В.\,Н.} Нейронные сети 
мониторинга чрезвычайных ситуаций по данным ДЗЗ~// Известия Тульского 
государственного университета. Технические науки, 2014. №\,5. С.~220--225.
\bibitem{5-bur}
\Au{Stone C.\,J.} Consistent nonparametric regression~// Ann. Stat., 1977. 
Vol.~5. Iss.~4. P.~595--620.
\bibitem{6-bur}
\Au{Breiman L.} Random forests~// Mach. Learn., 2001. Vol.~45. Iss.~1. 
P.~5--32.
\bibitem{7-bur}
\Au{Robbins H., Siegmund~D.} A~convergence theorem for non negative almost 
supermartingales and some applications~// Optimizing methods in statistics~/
Ed. J.\,S.~Rustagi.~--- New 
York, NY, USA: Academic Press, 1971. P.~233--257.
\bibitem{8-bur}
\Au{Cortes C., Vapnik~V.} Support-vector networks~// Mach. Learn., 1995. 
Vol.~20. Iss.~3. P.~273--297.
\bibitem{9-bur}
\Au{Hong H., Pradhan~B., Jebur~M.\,N.} Spatial prediction of landslide hazard at 
the Luxi area (China) using support vector machines~// Environ. Earth 
Sci., 2016. Vol.~75. Iss.~1. P.~40.
\bibitem{10-bur}
\Au{Kohavi R.} A~study of cross-validation and bootstrap for accuracy estimation 
and model selection~// 14th Joint Conference (International) on Artificial Intelligence 
Proceedings.~--- Montr$\acute{\mbox{e}}$al, 
Qu$\acute{\mbox{e}}$bec, Canada, 1995. Vol.~2. P.~1137--1143.
 \end{thebibliography}

 }
 }

\end{multicols}

\vspace*{-6pt}

\hfill{\small\textit{Поступила в~редакцию 02.10.18}}

%\vspace*{8pt}

%\pagebreak

\newpage

\vspace*{-28pt}

%\hrule

%\vspace*{2pt}

%\hrule

%\vspace*{-2pt}

\def\tit{DEVELOPMENT OF~A~METHOD FOR~THE~FORMATION OF~ATTRIBUTE SPACE 
AND~A~MODEL FOR~THE~ASSESSMENT AND~PREDICTION OF~ANTHROPOGENIC INFLUENCE 
ON~THE~ENVIRONMENT (ON~THE~EXAMPLE OF~THE~FOREST FUND OF~THE~OIL-PRODUCING 
REGION)}


\def\titkol{Development of~a~method for~the~formation of attribute space 
and~a~model for~the~assessment and~prediction of %~anthropogenic 
influence} 
%on~the~environment (on~the~example of~the~forest Fund of~the~oil-producing  region)}

\def\aut{V.\,V.~Burlutskiy$^1$, A.\,V.~Yakimchuk$^2$, A.\,V.~Melnikov$^1$, 
A.\,L.~Tsaregorodtsev$^1$, and~S.\,V.~Voloshin$^2$}

\def\autkol{V.\,V.~Burlutskiy, A.\,V.~Yakimchuk, A.\,V.~Melnikov, et al.} 
%A.\,L.~Tsaregorodtsev$^1$, and~S.\,V.~Voloshin$^2$}

\titel{\tit}{\aut}{\autkol}{\titkol}

\vspace*{-11pt}


\noindent
  $^1$Yugra Research Institute of Information Technologies, 151~Mira Str., 
Khanty-Mansiysk 628011, Russian\linebreak
$\hphantom{^1}$Federation 
  
  \noindent
  $^2$Yugra State University, 16~Chekhova Str., Khanty-Mansiysk 628012, 
Russian Federation

\def\leftfootline{\small{\textbf{\thepage}
\hfill INFORMATIKA I EE PRIMENENIYA~--- INFORMATICS AND
APPLICATIONS\ \ \ 2019\ \ \ volume~13\ \ \ issue\ 3}
}%
 \def\rightfootline{\small{INFORMATIKA I EE PRIMENENIYA~---
INFORMATICS AND APPLICATIONS\ \ \ 2019\ \ \ volume~13\ \ \ issue\ 3
\hfill \textbf{\thepage}}}

\vspace*{3pt}  
  
  
   
  
\Abste{The work is devoted to the development of a systematic method for assessing 
and predicting the influence of natural and anthropogenic impacts on the 
environment, including the procedures for the transformation of initial data store, the 
formation of the neural network model, its training, and testing. The method is used to 
analyze the consequences of anthropogenic impacts on the environment in the 
Khanty-Mansiysk Autonomous Okrug~--- Yugra.}
  
  \KWE{data analysis; machine learning; neural networks; spatial analysis; 
geographic information systems; risk-based approach; control and supervision}
  
  
 
  
  
\DOI{10.14357/19922264190318} 

%\vspace*{-14pt}

 \Ack
  \noindent
  This work was supported by the Science  Foundation of Yugra State 
University  under grant No.\,13-01-20/25.


%\vspace*{-6pt}

  \begin{multicols}{2}

\renewcommand{\bibname}{\protect\rmfamily References}
%\renewcommand{\bibname}{\large\protect\rm References}

{\small\frenchspacing
 {%\baselineskip=10.8pt
 \addcontentsline{toc}{section}{References}
 \begin{thebibliography}{99}
  

\bibitem{2-bur-1} %1
\Aue{Guikema, S.\,D.} 2009. Natural disaster risk analysis for critical infrastructure 
systems: An approach based on statistical learning theory. \textit{Reliab.
Eng. Syst. Safe.} 94(4):855--860. doi: 10.1016/j.ress.2008.09.003.
\bibitem{3-bur-1} %2
\Aue{Shokin, Y.\,I., V.\,V.~Moskvichev, and V.\,V.~Nicheporchuk.} 2010.Metodika otsenki 
antropogennykh riskov territoriy i~postroeniya kartogramm riskov s~ispol'zovaniem 
geoinformatsionnykh system [Technique for estimation of anthropogenous risks for 
territories and construction of risks cartograms using geoinformation systems]. 
\textit{Computational Technologies} 15(1):120--131.

\bibitem{1-bur-1} %3
\Aue{Gumenyuk, V.\,I., A.\,M.~Karmishin, and V.\,A.~Kireev.} 2013. 
O~kolichestvennykh pokazatelyakh opasnosti tekhnogennykh avariy [About 
quantitative indicators of danger man-made accidents]. 
\textit{Nauchno-tekhnicheskie vedomosti SPbPU} [St.\ Petersburg State
Polytechnic University 
J.~Engineering Science Technology] 2(171):281--288. 

\bibitem{4-bur-1}
\Aue{Kolesenkov, A.\,N., B.\,V.~Kostrov, and V.\,N.~Ruchkin.} 2014. Neyronnye 
seti monitoringa chrezvychaynykh situatsiy po dannym DZZ  [Neural network 
monitoring for emergencies according ERS]. \textit{Izvestiya 
Tul'skogo gosudarstvennogo universiteta. Tekhnicheskie nauki} [Transactions of 
Tula State University. Technical Sciences] 5:220--225.
\bibitem{5-bur-1}
\Aue{Stone, C.\,J.} 1977. Consistent nonparametric regression. \textit{Ann.
Stat.} 5(4):595--620.
\bibitem{6-bur-1}
\Aue{Breiman, L.} 2001. Random forests. \textit{Mach. Learn.} 45(1):5--32.
\bibitem{7-bur-1}
\Aue{Robbins, H., and D.~Siegmund.} 1971. A~convergence theorem for non 
negative almost supermartingales and some applications. \textit{Optimizing methods 
statistics}. Ed.\ J.\,S.~Rustagi.
New York, NY: Academic Press. 233--257.
\bibitem{8-bur-1}
\Aue{Cortes, C., and V.~Vapnik.} 1995. Support-vector networks. \textit{Mach. 
Learn.} 20(3):273--297.
\bibitem{9-bur-1}
\Aue{Hong, H., B.~Pradhan, and M.\,N.~Jebur.} 2016. Spatial prediction of 
landslide hazard at the Luxi area (China) using support vector machines. 
\textit{Environ. Earth Sci.} 75(1):40.
\bibitem{10-bur-1}
\Aue{Kohavi, R.} 1995. A~study of cross-validation and bootstrap for accuracy 
estimation and model selection. \textit{14th  Joint Conference (International) on 
Artificial Intelligence Proceedings}. Montr$\acute{\mbox{e}}$al, 
Qu$\acute{\mbox{e}}$bec, Canada.  2:1137--1143. 
 \end{thebibliography}

 }
 }

\end{multicols}

%\vspace*{-7pt}

\hfill{\small\textit{Received October 2, 2018}}

%\pagebreak

%\vspace*{-22pt} 
  
  
  \Contr
  
  
  \noindent
  \textbf{Burlutskiy Vladimir V.} (b.\ 1975)~--- Candidate of Science (PhD) in 
physics and mathematics, Head of the Center of Information and Analytical Systems, 
Yugra Research Institute of Information Technologies, 151~Mira Str.,  
Khanty-Mansiysk 628011, Russian Federation; \mbox{BurlutskyVV@uriit.ru}
  
  \vspace*{3pt}
  
  \noindent
  \textbf{Yakimchuk Aleksandr V.} (b.\ 1994)~--- PhD student, Yugra State 
University, 16~Chekhova Str., Khanty-Mansiysk 628012, Russian Federation; 
\mbox{YakimchukAV@uriit.ru }
  
  \vspace*{3pt}
  
  \noindent
  \textbf{Melnikov Andrey V.} (b.\ 1956)~--- Doctor of Science in technology, 
professor, Director, Yugra Research Institute of Information Technologies, 151~Mira 
Str., Khanty-Mansiysk 628011, Russian Federation; 
\mbox{andmelnikov1956@yandex.ru} 
  
  \vspace*{3pt}
  
  \noindent
  \textbf{Tsaregorodtsev Aleksandr L.} (b.\ 1979)~--- Candidate of Science (PhD) 
in technology, First Deputy Director, Yugra Research Institute of Information 
Technologies, 151~Mira Str., Khanty-Mansiysk 628011, Russian Federation; 
\mbox{TsaregorodtsevAL@uriit.ru} 
  
  
  \vspace*{3pt}
  
  \noindent
  \textbf{Voloshin Semen V.} (b.\ 1991)~--- PhD student, Yugra State University, 
16~Chekhova Str., Khanty-Mansiysk 628012, Russian Federation; 
\mbox{voloshinsv@uriit.ru}
  
       
   
   
   
\label{end\stat}

\renewcommand{\bibname}{\protect\rm Литература}  