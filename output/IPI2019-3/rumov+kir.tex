 \def\stat{rumov+kir}

\def\tit{МЕТОДЫ МОДЕЛИРОВАНИЯ И ВИЗУАЛЬНОГО ПРЕДСТАВЛЕНИЯ КОНФЛИКТА 
В~МАЛОМ КОЛЛЕКТИВЕ ЭКСПЕРТОВ, РЕШАЮЩИХ ПРОБЛЕМЫ (ОБЗОР)}

\def\titkol{Методы моделирования и~визуального представления конфликта 
в~малом коллективе экспертов} %, решающих проблемы (обзор)}

\def\aut{С.\,Б.~Румовская$^1$, И.\,А.~Кириков$^2$}

\def\autkol{С.\,Б.~Румовская, И.\,А.~Кириков}

\titel{\tit}{\aut}{\autkol}{\titkol}

\index{Румовская С.\,Б.}
\index{Кириков И.\,А.}
\index{Rumovskaya S.\,B.}
\index{Kirikov I.\,A.}


%{\renewcommand{\thefootnote}{\fnsymbol{footnote}} \footnotetext[1]
%{Работа частично поддержана РФФИ (проект 18-07-00274).}}


\renewcommand{\thefootnote}{\arabic{footnote}}
\footnotetext[1]{Калининградский филиал Федерального исследовательского центра <<Информатика и~управ\-ле\-ние>> 
Российской академии наук, \mbox{sophiyabr@gmail.com}}
\footnotetext[2]{Калининградский филиал Федерального исследовательского центра <<Информатика  
и~управ\-ле\-ние>> Российской академии наук, \mbox{baltbipiran@mail.ru}}

\vspace*{-14pt}


  
  
 
      

   \Abst{Малые коллективы экспертов как естественный коллективный интеллект 
поддержки принятия решений (гетерогенный коллектив) эффективно решают сложные 
проблемы. При этом такая форма взаимодействия между экспертами, как конфликт, 
порождает позитивные изменения в~коллективе: развитие группы, диагностику отношений, 
снятие напряжения, сплачивание группы, а~также способствует сохранению коллектива. 
В~человеческом мышлении огромную роль играют за\-го\-тов\-ки-схе\-мы стандартных ситуаций, 
использование которых существенно ускоряет рассуждения. Визуализация конфликтной 
ситуации делает возникшие противоречия контрастными, видимыми, предоставляя новую 
информацию для разрешения конфликтов, делая их легкоуправляемыми и~позволяя 
контролировать влияние на них субъективных предпочтений. Рас\-смот\-ре\-но понятие 
конфликта в~малых коллективах, его особенности, структура и~динамика, а~также подходы 
к~моделированию и~визуальному представлению конфликтологического аспекта групповой 
динамики экспертов, решающих проблемы.}
    
  \KW{малый коллектив экспертов; конфликт; модели конфликта; визуализация 
конфликта}

\DOI{10.14357/19922264190317} 
  
\vspace*{-1pt}


\vskip 10pt plus 9pt minus 6pt

\thispagestyle{headings}

\begin{multicols}{2}

\label{st\stat}

\section{Введение}

  Естественный коллективный интеллект (ге\-терогенный коллектив) поддержки 
принятия ре\-шений~\cite{1-r}~--- малая группа экспертов, которой\linebreak присущи 
неоднородность, разнообразие, со\-труд\-ничество, дополнительность 
и~относительность знаний. Подобные коллективы эффективно решают 
сложные проблемы. Ввиду этого малые группы, проблемы взаимодействия 
людей внутри них, а~также моделирование их взаимодействия занимают особое 
место в~широком спектре направлений современной науки. Часто 
встречающаяся форма организации малых коллективов~--- совещания, 
построенные по принципу <<круглого стола>>~\cite{3-r} \mbox{с~целью} выявления 
и~решения проб\-лем. Разного рода конфликты порождают дискуссии, глубина 
которых позволяет получить более продуманные и~согласованные решения. 
Координация работы экспертов в~группе лицом, принимающим решения (ЛПР), 
позволяет повысить качество решений, а~самоорганизация в~группе определяет 
способность чутко реагировать на изменения во внешней среде, корректируя 
свое функционирование и~при\-ни\-ма\-емые решения. 
  
  В~\cite{2-r} представлена система для опе\-ра\-тив\-но-про\-из\-вод\-ст\-вен\-но\-го 
планирования~--- моделировалась координация ЛПР (начальником 
производственного отдела) коллективной работы главного конструктора, 
главного технолога, начальника отдела материального снабжения, начальника 
электромеханического цеха, начальника отдела продаж \mbox{с~целью} повышения 
качества оперативных план-гра\-фи\-ков мелкосерийного производства. 
Относительная погрешность результатов решения задачи с~координацией~--- 
менее~1\%, а без~--- до~36\%.
  
  В~\cite{2-r} также описана модель самоорганизации\linebreak в~группе для 
транспортной логистики гибридными интеллектуальными многоагентными 
сис\-те\-ма\-ми (ГиМАС). Алгоритм функционирования сис\-те\-мы динамически 
перестраивается, вырабатывая\linebreak
 релевантный сложной задаче метод решения 
и~сокращая среднюю суммарную се\-бе\-сто\-и\-мость и~время доставки грузов 
в~день на~7,2\% и~12,13\% соответственно; среднее время по\-стро\-ения 
маршрутов уменьшилось на~23,14\%.


  
  Долгое время социологи и~психологи считали, что конфликты~--- негативное 
явление и~их надо устранять. Однако работа Г.~Зиммеля~\cite{3-r} 
способствовала развитию идеи наличия позитивных изменений, порождаемых 
конфликтами: сохранение социальной системы, развитие группы, диагностика 
отношений, снятие напряжения, сплачивание группы и~др. 
  
  Таким образом, моделирование развития и~разрешения конфликта в~рамках 
методологии ГиМАС позволит спроектировать функционирование сис\-те\-мы 
релевантно групповой динамике коллектива экспертов, решающих проб\-ле\-му, 
и~тем самым существенно повысить качество принимаемых ре\-шений. При 
этом визуализация конфликтной\linebreak си\-туации сделает возникшие противоречия 
контрастными, видимыми, предоставляя новую информацию для разрешения 
конфликтов в~реальном коллективе экспертов. В~работе рассмотрено\linebreak понятие 
конфликта в~малых коллективах, его особенности, структура и~динамика, 
а~также подходы к~моделированию и~визуальному представлению 
конфликтологического аспекта групповой динамики экспертов, решающих 
проблемы. 

\vspace*{-6pt}
  
\section{Малый коллектив экспертов: понятие и~классификация}

\vspace*{-2pt}
  
  Попытки определить малую группу экспертов сводились к~субъективному 
пониманию и~фокусировке на тех или иных сторонах группового процесса 
(определенных априори либо эмпирическим): М.~Шоу~\cite{4-r} 
(психологическая со\-став\-ля\-ющая), Р.~Браун~\cite{5-r} 
и~Г.\,М.~Андреева~\cite{6-r} (социальная составляющая). 
  
  Приведем определение, включающее в~себя психологическую и~социальную 
составляющие: <<малая группа>>~--- элементарное звено структуры 
социальных отношений, обретающее через не-\linebreak посредственные межличностные 
контакты струк-\linebreak турные, динамические, феноменологические\linebreak характеристики, 
отражающие признаки группы как целостной сис\-те\-мы социальных 
и~психологических отношений. Понятия <<малая группа>> и~<<малый 
коллектив>> идентичны.
  
  Верхняя граница размерности коллектива (нижняя~--- 2~участника) 
определяется с~двух точек зрения:
\begin{enumerate}[(1)]
\item в соответствии с~требованиями 
реализации ее основной функции~\cite{6-r}~--- верхняя граница не может быть 
обозначена априори; 
\item относительно успешного руководства группой~--- 
верхняя граница соответствует <<магическому числу>> Дж.~Миллера ($7\pm 
2$), так как при численности свыше~10~чел.\ возрастает число подгрупп 
и~вероятность противостояния ЛПР, осложняется координация.
\end{enumerate}
  
  Есть различные классификации малых коллективов~\cite{6-r}, в~частности 
английский психолог М.~Аргайл выделяет~\cite{7-r}: \textit{семью; 
подростково-юношеские коллективы; рабочие коллективы}~--- модель малых 
групп с~четкой трудовой направленностью и~доминирующими отношениями 
делового характера; \textit{комитеты} и~\textit{группы по решению  
проб\-лем}~--- модель коммуникативных малых коллективов, задача которых 
принимать эффективные решения, а их участники должны владеть навыками 
организации информационного обмена, достижения внутригруппового 
согласия и~т.\,п.; \textit{тренинговые} и~\textit{терапевтические группы}.


\vspace*{-6pt}
  
\section{Конфликт: понятие, структура и~типология}

\vspace*{-2pt}

  Конфликт~\cite{8-r, 9-r}~--- столкновение противоположно направленных 
интересов, целей, взглядов и~т.\,п.\ (обострение противоречия) при 
взаимодействии и~взаимоотношении сторон, вос\-при\-ни\-ма\-емое субъектом как 
значимая для него психологическая проблема, требующая своего разрешения 
и~вызывающая активность, направленную на его преодоление. Структура 
конфликта представлена на рис.~1~\cite{10-r, 11-r}. 



  Стороны конфликта~--- это субъекты социального взаимодействия: 
в~состоянии конфликта (основные участники, оппоненты); или же явно или 
неявно поддерживающие конфликтующих (группы\linebreak поддержки); или 
оказывающие эпизодическое влияние на конфликт (другие участники~--- 
подстрекатели, организаторы и~т.\,д.). Объект находится на пересечении 
личных или групповых интересов субъектов. Предмет конфликта~--- это 
противоречие, из-за которого и~для разрешения которого возникает конфликт. 
Образ конфликтной ситуации~--- это отображение предмета конфликта 
в~сознании субъектов. Мотивы конфликта~--- это потребности, интересы, цели 
и~т.\,д., подталкивающие субъектов к~конфликту. 
  
  Конфликты классифицируются по различным признакам  
(рис.~2)~\cite{8-r, 10-r}. 
  
  \textit{Конструктивные} возникают при объективных противоречиях, когда 
цели и~потребности сторон едины (развитие коллектива). 
\textit{Деструктивные}~--- причины их субъективны, вызывают напряженность и~разрушают коллектив.
  
  \textit{Конфликт отношений (социальный конфликт)}~--- разногласие между 
членами группы по личным\linebreak вопросам и~проблемам, не относящимся 
к~вы\-пол\-ня\-емой работе (связан с~несовместимостью и~враждебностью между 
субъектами). 



  \textit{Межличностные} конфликты бывают пяти типов (см.\ рис.~2). При этом 
конфликты внутри устойчивой подгруппы более скоротечны и~чаще\linebreak\vspace*{-12pt}

\pagebreak

\end{multicols}

\begin{figure*} %fig1
 \vspace*{1pt}
    \begin{center}  
  \mbox{%
 \epsfxsize=134.001mm 
 \epsfbox{rum-1.eps}
 }
\end{center}
\vspace*{-9pt}
\Caption{Структура конфликта}
%\end{figure*}
%\begin{figure*} %fig2
\vspace*{24pt}
    \begin{center}  
  \mbox{%
 \epsfxsize=157.144mm 
 \epsfbox{rum-2.eps}
 }
\end{center}
\vspace*{-9pt}
\Caption{Типология конфликта в~малом коллективе}
\end{figure*}

\begin{multicols}{2}


\noindent
 имеют 
иное значение и~последствие для их участников~--- расцениваются как 
конфликт <<между\linebreak своими>>, семейный конфликт.  
\textit{Микрогрупповой}~---\linebreak конфликт между индивидом и~неформальной 
подгруппой. \textit{Меж\-мик\-ро\-груп\-по\-вой}~--- конфликт между неформальными 
подгруппами в~группе. \textit{Груп\-по\-вой}~--- конфликт между индивидом 
и~группой.
  
  \textit{Конфликт интересов} обусловлен мотивационными факторами 
и~ситуацией, в~которой цели каж\-до\-го члена не совпадают с~целями других. 
\textit{Конфликт ресурсов} предполагает отсутствие соглашения между 
субъектами по поводу ресурсов. \textit{Когнитивный конфликт}~--- цели всех 
членов группы совпадают, но их позиции различны (фокус на 
интеллектуальных или оценочных проблемах, связи между внутригрупповым 
конфликтом и~дея\-тель\-ностью группы).

\vspace*{5pt}
  
  \textit{Инструментальный (деловой) конфликт}:\\[-16pt]
\begin{enumerate}[(1)]
\item \textit{конфликт 
задачи}~--- связан с~различием точек зрения на групповые цели и~задачи 
(помогают открытое об\-суж\-де\-ние и~споры);\\[-14pt]
\item \textit{конфликт процесса}~--- 
возникает по поводу проблем, возникающих относительно технологии 
и~способов решения поставленной задачи, распределения между субъектами 
ролей и~от\-вет\-ст\-вен\-ности. 
\end{enumerate}

\begin{figure*}[b] %fig3
\vspace*{1pt}
    \begin{center}  
  \mbox{%
 \epsfxsize=156.127mm 
 \epsfbox{rum-3.eps}
 }
\end{center}
\vspace*{-9pt}
    \Caption{Фазы конфликта}
    \end{figure*}
  
\section{Причины, позитивные функции и~динамика конфликта} 
  
  Есть разные точки зрения~\cite{9-r, 12-r} на классификацию причин 
конфликтов. Приведем причины по К.~Левину~\cite{9-r}: 
\begin{itemize}
\item \textit{степень 
удовлетворенности потребностей}~--- неудовлетворенные по\-треб\-но\-сти час\-то 
становятся до\-ми\-ни\-ру\-ющи\-ми, увеличивая ве\-ро\-ят\-ность конфликтов; 
\item <<\textit{пространство свободного движения}>> (достаточное  
пространство~--- условие удовлетворения индивидуальных по\-треб\-но\-стей 
и~адаптации к~группе, а~огра\-ни\-чен\-ность ведет к~рос\-ту на\-пря\-же\-ния); 
\item \textit{внешний барьер}~--- наличие или отсутствие возможностей выйти из 
неприятной ситуации (отсутствие провоцирует конфликт); 
\item \textit{совпадение 
или расхождение целей членов группы}~--- конфликты зависят от степени 
противоречия целей участников и~от их го\-тов\-ности к~компромиссу.
\end{itemize}
  
  Позитивные функции конфликта~\cite{8-r}: 
  \begin{itemize}
  \item обеспечивает уникальность 
и~автономность каждого из взаимодействующих субъектов, а~также развитие 
отношений между ними; 
\item предоставляет информацию о~воз\-мож\-но\-стях 
противодействующих субъектов; 
\item высвобождает накапливающееся внутреннее 
на\-пря\-же\-ние, сохраняя связи; 
\item актуализирует разные позиции и~мнения по 
поводу возникающих проб\-лем и~тем самым способствует поиску оптимальных 
способов их решения; 
\item усиливает груп\-по\-вую/мик\-ро\-груп\-по\-вую 
идентичность и~сплоченность.
\end{itemize}
  
  Динамика конфликта~\cite{10-r, 11-r} отражается в~двух понятиях: этапы 
конфликта и~фазы конфликта. Основные этапы конфликта: 
\begin{enumerate}[(1)]
\item \textit{возникновение объективной проблемной ситуации} (появление 
противоречия); 
\item \textit{осознание проблемы} хотя бы одним из участников 
(если разрешить проблему неконфликтными методами не получается, то 
возникает предконфликтная ситуация); 
\item \textit{начало открытого 
конфликтного взаимодействия (инцидент)}; 
\item \textit{развитие открытого 
конфликта (эскалация)}~--- открыто заявляются позиции и~выдвигаются 
требования. Завершается этот этап \textit{сбалансированным 
противодействием}~--- когда силовые методы не дают результата, 
интенсивность борьбы снижается, но действия по достижению согласия еще не 
предпринимаются; 
\item \textit{разрешение конфликта}. 
\end{enumerate}
  
  Основные фазы конфликта:
  \begin{enumerate}[(1)]
  \item  начальная фаза;
  \item  фаза подъема; 
  \item пик конфликта; 
\item фаза спада.
\end{enumerate}
Фазы могут повторяться циклически (рис.~3)~\cite{10-r}. При этом 
возможности разрешения конфликта в~каж\-дом последующем цикле сужаются.

 \begin{figure*}[b] %fig4
\vspace*{1pt}
    \begin{center}  
  \mbox{%
 \epsfxsize=162.027mm 
 \epsfbox{rum-4.eps}
 }
\end{center}
\vspace*{-9pt}
   \Caption{Карта анализа конфликта~(\textit{а}) и~метод декартовых координат~(\textit{б})}
   \end{figure*}  

  
\section{Методы моделирования конфликта}

  В открытой печати встречаются подходы к~моделированию межгрупповых 
  и~межгосударственных конфликтов~\cite{11-r, 13-r, 14-r, 15-r}, которые 
позволяют заменить непосредственный анализ конфликтов анализом свойств 
и~характеристик их моделей, а также прогнозировать и~оценивать события 
в~реальном времени. Их опыт можно перенести на моделирование конфликтов 
в~малых коллективах при их классификации в~рамках микрогрупповой 
концепции~\cite{8-r}. В~\cite{5-r} представлен развернутый обзор 
моделирования военных конфликтов и~выделяются:
\begin{enumerate}[(1)]
\item \textit{описательные 
модели военных действий}~--- основываются на методах теории вероятностей 
и~статистической теории решений (принятие\linebreak решений в~условиях 
<<природной>> неопределенности), тео\-рии на\-деж\-ности и~тео\-рии массового 
обслуживания, тео\-рии экспертных\linebreak оценок, а~так\-же качественный анализ 
соответствующих динамических систем и~исследование их структурной 
устойчивости; 
\item \textit{имитационные модели}~--- основываются на аппарате 
марковских цепей, дифференциальных уравнений (ланчестеровские модели), 
конечных автоматов или методах распределенного искусственного интеллекта; 
\item \textit{оптимизационные модели военных действий} используют аппарат 
линейного и~динамического программирования, теории оптимального 
управления, дискретной оптимизации; 
\item \textit{модели принятия решений}~--- 
индивидуального (основной акцент обычно делается на многокритериальном 
принятии решений) и~коллективного (акцент на использовании теории игр). 
\end{enumerate}
  
  \textit{Многоагентное моделирование}~--- методология, 
применяемая для поддержки принятия решений, анализа и~изучения сложных 
сис\-тем, со\-сто\-ящих из отдельных, функционирующих независимо друг от друга 
индивидов~\cite{16-r}. Изучается влияние взаимодействий индивидов на сис\-тем\-ные 
характеристики в~целом.
  
  \textit{Вероятностные распределения}~--- способ описания 
переменных через указание доли элементов совокупности с~данным значением 
переменной~\cite{17-r}.
  
  \textit{Модели целенаправленного поведения}~--- использование 
целевых функций для анализа, прогнозирования и~планирования социальных 
процессов~\cite{18-r}. Модели имеют вид задачи математического программирования.
  
  \textit{Статистические исследования зависимостей}~--- прежде 
всего это регрессионные модели, пред\-став\-ля\-ющие связь зависимых 
и~независимых переменных в~виде функциональных отношений~\cite{19-r}.
  
  \textit{Теоретические модели} предназначены для логического 
анализа тех или иных содержательных концепций, когда затруднена 
возможность измерения основных параметров и~переменных (возможные 
межгосударственные конфликты и~др.)~\cite{17-r}.
  
\section{Методы визуализации конфликтов }

  Эффективность разрешения конфликтов во многом определяется 
правильностью и~полнотой их анализа. Рассмотрим основные методы, 
пред\-ла\-га\-емые зарубежными и~отечественными психологами.
  
  \textit{Картографический анализ конфликтов  
Х.~Кор\-не\-ли\-ус и~Ш.~Фейр}~\cite{20-r}. Проблема обычно записывается в~виде 
противоречия в~центре карты (рис.~4,\,\textit{а}), которая делится по числу 
сторон конфликта. У~каж\-до\-го участника конфликта выясняются потребности, 
интересы и~опасения. Все основные элементы конфликта (проблема, позиции, 
условия, образы, интересы, потребности, опасения, исходы и~др.)\ 
упорядочиваются и~систематизируются. 
  
  \textit{Метод декартовых координат}~\cite{21-r}. Р.~Декарт считал, что 
наше сознание имеет определенную структуру и~его можно пред\-ста\-вить как 
про\-стран\-ст\-во (рис.~4,\,\textit{б}), обозначенное математическими\linebreak\vspace*{-12pt}

\pagebreak

\end{multicols}

\begin{figure*} %fig5
\vspace*{1pt}
    \begin{center}  
  \mbox{%
 \epsfxsize=130.703mm 
 \epsfbox{rum-5.eps}
 }
\end{center}
\vspace*{-9pt}
\Caption{Ресурсный анализ конфликта: конфликтограмма и~балансный лист ресурсов}
%\end{figure*}
%\begin{figure*} %fig6
\vspace*{18pt}
    \begin{center}  
  \mbox{%
 \epsfxsize=162.045mm 
 \epsfbox{rum-6.eps}
 }
\end{center}
\vspace*{-9pt}
\Caption{Модели социального поведения (\textit{а}) и~сетка  
К.~То\-ма\-са\,--\,Л.~Кил\-ме\-на~(\textit{б})}
%\vspace*{3pt}
\end{figure*}

\begin{multicols}{2}

\noindent
 символами 
$(a, b)$. Все это пространство поделено на четыре квад\-ра\-та с~по\-мощью двух 
пе\-ре\-се\-ка\-ющих\-ся прямых.
     В~каж\-дый квадрат надо записать не менее десяти ответов, которые 
со\-по\-став\-ля\-ют\-ся и~анализируются для поиска решения выхода из кон\-фликта. 
{\looseness=1

}
  
  \textit{Ресурсный анализ конфликта В.\,Н.~Ковалева (РАКК)}~\cite{22-r} 
базируется на заполнении всеми субъектами конфликтограммы 
и~балансного листа ресурсов сторон (рис.~5), которые позволяют по\-смот\-реть 
на конфликт глазами оппонента, установить воз\-мож\-ность обмена ресурсами 
и~компромисса.
  


  Межличностные отношения в~конфликте в~рамках РАКК представляются 
моделями социального поведения оппонента (рис.~6): четверть~I~--- альтруизм 
(оказание помощи другим); II~--- эгоизм (приобретение всего лучшего без 
обмена ресурсами); III~--- альтруизм (оказание помощи через избавление от 
трудностей других); IV~--- эгоизм (избавление от всего плохого). 
В~квадрате~V в~соизмеримом отношении встречаются все формы поведения. 
Завершающий этап~--- оценка сторонами того, что они отдали и~что приобрели, 
и~взаимное согласие с~полученными результатами. Для анализа стилей 
поведения личности в~конфликте психологи пользуются сеткой К.~Томаса 
и~Р.~Килмена (рис.~6,\,\textit{б})~--- она дает возможность спрогнозировать 
исходы конфликта, остроту его протекания, возможные ресурсные потери. 
  

  Затем анализируются возможные типы поведения и~реакции на них (рис.~7), 
и~каждой стороной заполняется таблица проектов предложений. 
Стороны приступают к~совместному анализу предложений и~заключают 
соглашение.
   
  \begin{figure*} %fig7
  \vspace*{1pt}
    \begin{center}  
  \mbox{%
 \epsfxsize=111.089mm 
 \epsfbox{rum-7.eps}
 }
\end{center}
\vspace*{-9pt}
  \Caption{Комплексная матрица анализа возможных типов поведения и~проекты предложений}
   \end{figure*}
  
  Также для визуализации конфликта можно использовать дерево целей 
и~дерево проблем~\cite{23-r}. Дерево целей~--- это структурированная, 
построенная по иерархическому принципу совокупность целей 
конфликтующей системы. Если представить предмет конфликта как проблему, 
то можно построить дерево проблем, а затем и~дерево решений. В~\cite{24-r} 
рассматривается диаграмма разрешения конфликтов, в~которой совместно 
используются деревья текущей реальности, будущей реальности, перехода, 
разрешения конфликтов <<Грозовая туча>>. 

\section{Заключение}

  Аналитический обзор по материалам открытой печати показал разнообразие 
подходов к~определению понятия конфликта, его классификации, а~так\-же 
причин и~функций конфликта. При этом для анализа конфликтной ситуации 
с~целью ее разрешения нет единого подхода к~визуализации субъектов 
конфликтной ситуации, их характеристик, взаимодействия и~взаимоотношений. 
Модели конфликтов строятся в~рамках различных методов и~подходов: 
описательные, имитационные, оптимизационные и~модели принятия  
решений~--- однако найдены были только модели военных конфликтов. 
Используя опыт моделирования  
последних~\cite{15-r, 16-r, 17-r, 18-r, 19-r, 20-r, 21-r} с~позиции микрогрупповой 
концепции А.\,В.~Сидоренкова~\cite{10-r}, будет разработана модель 
конфликта в~\mbox{ГиМАС.}
  
  {\small\frenchspacing
 {%\baselineskip=10.8pt
 \addcontentsline{toc}{section}{References}
 \begin{thebibliography}{99}
\bibitem{1-r}
\Au{Колесников А.\,В.} Гетерогенные естественные и~искусственные системы~// 
Интегрированные модели и~мягкие вычисления в~искусственном интеллекте~/
Под ред.\ В.\,Б.~Тарасова.~--- М.: Физматлит, 2013. Т.~1. С.~86--103.

\bibitem{3-r}
\Au{Зиммель Г.} Конфликт современной культуры~/ Пер. с~нем.~--- Птг.: Начатки знаний, 
1923. 40~с. (\Au{Simmel~G.} Der Konflikt der modernen Kultur.~--- Munchen: Duncker \& 
Humblot, 1918. 60~p.)
\bibitem{2-r}
\Au{Колесников А.\,В., Кириков~И.\,А., Листопад~С.\,В.} Гиб\-рид\-ные интеллектуальные 
системы с~самоорганизацией: координация, согласованность, спор.~--- М.: ИПИ РАН, 2014. 
189~с.

\bibitem{4-r}
\Au{Shaw M.} Group dynamics: The psychology of small group behavior.~--- New York, NY, 
USA: McGraw-Hill, 1981. 531~p.
\bibitem{5-r}
\Au{Brown R.} Group processes.~--- 2nd ed.~--- Oxford: Wiley-Blackwell, 1996. 442~p.
\bibitem{6-r}
\Au{Андреева Г.\,М.} Социальная психология.~--- М.: Аспект-пресс, 2009. 393~с.
\bibitem{7-r}
Small groups and social interaction~/ Eds. H.~Blumberg, A.~Hare, V.~Kent, M.~Davies.~--- 
Chichester: Wiley, 1983. Vol.~1. 478~p.
\bibitem{8-r}
\Au{Сидоренков А.\,В.} Конфликт в~малой группе: понятие, функции, виды и~модель~//  
Се\-ве\-ро-Кав\-каз\-ский психологический вестник, 2008. Т.~6. №\,4. С.~22--28.
\bibitem{9-r}
\Au{Гришина Н.\,В.} Психология конфликта.~--- 2-е изд.~--- СПб.: Питер, 2008. 544~с.
\bibitem{10-r}
\Au{Емельянов С.\,М.} Практикум по конфликтологии.~--- 3-е изд.~--- СПб.: Питер, 2009. 
384~с.
\bibitem{11-r}
\Au{Анцупов А.\,Я., Баклановский~С.\,В.} Конфликтология в~схемах и~комментариях.~--- 
2-е изд.~--- СПб.: Питер, 2009. 304~с.
\bibitem{12-r}
\Au{Робер М.-А., Тильман~Ф.} Психология индивида и~группы~/ Пер. с~фр.~--- М.: 
Прогресс, 1988. 256~с. (\Au{Ro\-bert~M.\,A., Tilman~F.} Psycho. 
Conna$\hat{\mbox{\!\normalsize\!\ptb{\i}}}$tre l'individu et le groupe aujourd'hui.~--- Bruxelles: Vie ouvriere, 
1980. 307~p.)

\bibitem{14-r} %13
\Au{Epstein J.\,M., Steinbruner~J.\,D., Parker~M.\,T.} Modeling civil violence: An agent-based 
computational approach~// P.~Natl. Acad. Sci. USA, 2002. Vol.~99. Suppl.~3. P.~7243--7250. doi: 
10.1073/pnas.092080199.
\bibitem{15-r} %14
\Au{Новиков Д.\,А.} Иерархические модели военных действий~// Управление большими 
системами, 2012. №\,37. С.~25--62.
\bibitem{13-r} %15
\Au{Клаус Н.\,Г., Свечкарев~В.\,П.} Агентные модели локальных этнических конфликтов (на 
примере осе\-ти\-но-ин\-гуш\-ско\-го конфликта в~селе Тарское)~// Инженерный вестник 
Дона, 2013. №\,4.  С.~138.
{\sf http://www. ivdon.ru/magazine/archive/n4y2013/1983}.

\bibitem{16-r}
\Au{Клаус Н.\,Г., Свечкарев~В.\,П.} Многоагентное моделирование конфликтных  
ситуаций.~--- Ростов на Дону: СКНЦВШ ЮФУ, 2013. 148~с.
\bibitem{17-r}
\Au{Саати Т.\,Л.} Математические модели конфликтных ситуаций~/ Пер. с~англ. под ред. 
И.\,А.~Ушакова.~--- М.: Сов. радио, 1977. 302~с. (\Au{Saaty~T.} Mathematical models of arms 
control and disarmament: Application of mathematical structures in politics.~--- John Wiley\&Sons, 
1968. 190~p.)
\bibitem{18-r}
\Au{Покорная О.\,Ю., Покорная~И.\,Ю., Прядкин~Д.\,В.} Математическое моделирование 
оптимальных стратегий в~условиях конфликта~// Молодой ученый, 2011.  \mbox{№\,4-1}.  
С.~16--19. {\sf https://moluch.ru/archive/27/2896.}
\bibitem{19-r}
\Au{Tibshirani R.} Regression shrinkage and selection via the Lasso~// J.~Roy. 
Stat. Soc.~B Met., 1996. Vol.~58. No.\,1. P.~267--288.
\bibitem{20-r}
\Au{Корнелиус Х., Фейр~Ш.} Выиграть может каждый. Как разрешать конфликты~/ Пер. 
с~англ.~--- Киев: Наукова думка, 2006. 344~с. {\sf  
http://www.conflict-resolve.org/conflictresolve.pdf}.
(\Au{Cornelius~H., Faire~S.} Everyone can win: 
How to resolve conflict.~--- West Roseville, New South Wales, Australia: 
Simon \& Schuster, 1989. 192~p.)
\bibitem{21-r}
\Au{Мириманова М.\,С.} Конфликтология.~--- 
2-е изд.~--- М.: Академия, 2004. 320~с.
\bibitem{22-r}
\Au{Ковалев В.\,Н., Лопатина~Н.\,Н., Лебеденко~Д.\,Н.} Сборник ситуационных задач по 
конфликтологии.~--- Севастополь: СФ МГУ, 2013. 91~с.
\bibitem{23-r}
\Au{Готин С.\,В., Калоша~Л.\,П.} Ло\-ги\-ко-струк\-тур\-ный подход и~его применение для 
анализа и~планирования деятельности.~--- М.: Вариант, 2007. 118~с. 
\bibitem{24-r}
\Au{Детмер У.} Теория ограничений Голдратта: Системный подход к~непрерывному 
совершенствованию~/ Пер. с~англ.~--- 4-е изд.~--- М.: Альпина Паблишер, 2014. 443~с. 
{\sf https://econ.wikireading.ru/64894}. (\Au{Dettmer~W.} Goldratt's theory of constraints: 
A~systems approach to continuous improvement.~--- ASQ Quality Press, 2006. 377~p.) 

 \end{thebibliography}

 }
 }

\end{multicols}

\vspace*{-6pt}

\hfill{\small\textit{Поступила в~редакцию 11.06.19}}

\vspace*{9pt}

%\pagebreak

%\newpage

%\vspace*{-28pt}

\hrule

\vspace*{2pt}

\hrule

%\vspace*{-2pt}

\def\tit{METHODS OF MODELING AND~VISUAL REPRESENTATION OF~A~CONFLICT 
IN~A~SMALL COLLECTIVE OF~EXPERTS SOLVING~PROBLEMS (REVIEW)}


\def\titkol{Methods of modeling and~visual representation of~a~conflict 
in~a~small collective of~experts solving problems (review)}

\def\aut{S.\,B.~Rumovskaya and~I.\,A.~Kirikov}

\def\autkol{S.\,B.~Rumovskaya and~I.\,A.~Kirikov}

\titel{\tit}{\aut}{\autkol}{\titkol}

\vspace*{-11pt}


 \noindent
   Kaliningrad Branch of the Federal Research Center ``Computer Science and 
Control'' of the Russian Academy of Sciences, 5~Gostinaya Str., Kaliningrad 
236022, Russian Federation


\def\leftfootline{\small{\textbf{\thepage}
\hfill INFORMATIKA I EE PRIMENENIYA~--- INFORMATICS AND
APPLICATIONS\ \ \ 2019\ \ \ volume~13\ \ \ issue\ 3}
}%
 \def\rightfootline{\small{INFORMATIKA I EE PRIMENENIYA~---
INFORMATICS AND APPLICATIONS\ \ \ 2019\ \ \ volume~13\ \ \ issue\ 3
\hfill \textbf{\thepage}}}

\vspace*{6pt} 
   
     
  
   \Abste{Small collectives of experts as natural collective decision support intellect (heterogeneous 
collective) solve problems effectively. In addition, the form of interaction between experts as conflict 
generates positive changes in collective such as development of the group, diagnostics of relations, tension 
reduction, and consolidation of the group and inspire saving the collective. Preset sketches of standard 
situations play a~huge role in human reasoning. The use of them highly promote reasoning. Visualization of 
conflict situation makes appeared contradictions contrast and observable, giving a new information of 
resolving the conflict. This makes them easy-to-handle and gives the opportunity to control the impaction on 
the conflict of subjective preference. The notion and particularities of the conflict in 
small collectives, its structure, dynamics, and approaches to modelling and visual presentation of the conflict 
aspect in group dynamics of experts solving problems are reviewed.}
   
   \KWE{small collective of experts; conflict; model of a conflict; visualization of a conflict}
  
\DOI{10.14357/19922264190317} 

%\vspace*{-14pt}

%\Ack
%   \noindent



%\vspace*{-6pt}

  \begin{multicols}{2}

\renewcommand{\bibname}{\protect\rmfamily References}
%\renewcommand{\bibname}{\large\protect\rm References}

{\small\frenchspacing
 {%\baselineskip=10.8pt
 \addcontentsline{toc}{section}{References}
 \begin{thebibliography}{99}
\bibitem{1-r-1}
\Aue{Kolesnikov, A.\,V.} 2013. Geterogennye estestvennye i~is\-kus\-stvennye sistemy 
[Natural and artificial heterogeneous systems]. \textit{Integrirovannye modeli
i~myagkie vychisleniya v~iskusstvennom intellekte}
[Integrated models and 
soft computing in artificial intelligence]. 
Moscow: Fizmatlit. 1:86--103. 
\bibitem{3-r-1} %2
\Aue{Simmel, G.} 1918. Der Konflikt der modernen Kultur. Munchen: Duncker \& 
Humblot. 60~p.
\bibitem{2-r-1} %3
\Aue{Kolesnikov, A.\,V., I.\,A.~Kirikov, and S.\,V.~Listopad.} 2014. 
\textit{Gibridnye intellektual'nye sistemy s~samoorganizatsiey: koordinatsiya, 
soglasovannost', spor} [Hybrid artificial systems with self-organization: 
Coordination, conformance, row]. Мoscow: IPI RAN. 189~p. 

\bibitem{4-r-1}
\Aue{Shaw, M.} 1981. \textit{Group dynamics: The psychology of small group 
behavior}. New York, NY: McGraw-Hill. 531~p.
\bibitem{5-r-1}
\Aue{Brown, R.} 1996. \textit{Group processes}. 2nd ed. Oxford: Wiley-Blackwell. 
442~p.
\bibitem{6-r-1}
\Aue{Andreeva, G.\,M.} 2009. \textit{Sotsial'naya psikhologiya} [Social psychology]. 
Moscow: Aspect-press. 393~p.
\bibitem{7-r-1}
Blumberg, H., A.~Hare, V.~Kent, and M.~Davies, eds. 1983. \textit{Small groups 
and social interaction}. Chichester:\linebreak Wiley. Vol.~1. 478~p.
\bibitem{8-r-1}
\Aue{Sidorenkov, A.\,V.} 2008. Konflikt v maloy gruppe: po\-nya\-tie, funktsii, vidy 
i~model' [Conflict in a~small group: Concept, functions, forms and model].  
\textit{Severo-Kavkazskiy psikhologicheskiy vestnik} [North Caucasian 
Psychological Annals] 6(4):22--28.
\bibitem{9-r-1}
\Aue{Grishina, N.\,V.} 2008. \textit{Psikhologiya konflikta} [Psychology of conflict]. 
SPb.: Piter. 544~p.
\bibitem{10-r-1}
\Aue{Emel'yanov, S.\,M.} 2009. \textit{Praktikum po konfliktologii} [Tutorial at 
conflictology]. SPb.: Piter. 384~p.
\bibitem{11-r-1}
\Aue{Antsupov, A.\,Ya., and S.\,V.~Baklanovskiy.} 2009. \textit{Konfliktologiya 
v~skhemakh i~kommentariyakh} [Conflictology in schemes and 
comments]. SPb.: Piter. 304~p.
\bibitem{12-r-1}
\Aue{Robert, M.\,A., and F.~Tilman.} 1980. \textit{Psycho. 
Conna$\hat{\mbox{\normalsize\!\!\ptb{\i}}}$tre l'individu et le groupe aujourd'hui}. Bruxelles: Vie 
ouvriere. 307~p.

\bibitem{14-r-1} %13
\Aue{Epstein, J.\,M., J.\,D.~Steinbruner, and M.\,T.~Parker.} 2002. Modeling civil 
violence: An agent-based computational approach. \textit{P.~Natl. Acad. Sci. USA} 
99(Suppl.~3):7243--7250. 
doi: 10.1073/pnas.092080199.
\bibitem{15-r-1} %14
\Aue{Novikov, D.\,A.} 2012. Ierarkhicheskie modeli voennykh deystviy [Hierarchical 
models of combat]. \textit{Upravlenie bol'shimi sistemami} [Control of Large 
Systems] 37:25--62.
\bibitem{13-r-1} %15
\Aue{Klaus, N.\,G., and V.\,P.~Svechkarev.} 2013. Agentnye modeli lokal'nykh 
ehtnicheskikh konfliktov (na primere ose\-ti\-no-in\-gush\-sko\-go konflikta v~sele Tarskoe) 
[Agent based modeling of the social conflict: Ossetino-Ingushskiy conflict 
in the area of village Tarskoe]. \textit{Inzhenernyy vestnik Dona} [Engineering Annals of Don] 
4:138. Available at: {\sf http://www.ivdon.ru/magazine/archive/n4y2013/ 1983/}
 (accessed 
May~27, 2019).

\bibitem{16-r-1}
\Aue{Klaus, N.\,G., and V.\,P.~Svechkarev.} 2013. \textit{Mnogoagentnoe 
modelirovanie konfliktnykh situatsiy} [Multiagent modeling of conflict situations]. 
Rostov-on-Don: NCSC HS SFEDU. 148~p.

%\columnbreak

\bibitem{17-r-1}
\Aue{Saaty, T.} 1968. \textit{Mathematical models of arms control and disarmament: 
Application of mathematical structures in politics}. John Wiley\&Sons. 190~p.
\bibitem{18-r-1}
\Aue{Pokornaya, O.\,Yu., I.\,Yu.~Pokornaya, and D.\,V.~Pryadkin.} 2011. 
Matematicheskoe modelirovanie op\-ti\-mal'\-nykh strategiy v~usloviyakh konflikta 
[Mathematical modeling of optimal strategies in conflict]. \textit{Molodoy uchenyy} 
[Young Scientist] 4-1:16--19. Available at: {\sf  
 https:// moluch.ru/archive/27/2896/} (accessed May~27, 2019).
\bibitem{19-r-1}
\Aue{Tibshirani, R.} 1996. Regression shrinkage and selection via the Lasso. 
\textit{J.~Roy. Stat. Soc.~B Met.} 58(1):267--288.
\bibitem{20-r-1}
\Aue{Cornelius, H., and S.~Faire.} 1989. \textit{Everyone can win:
How to resolve conflict}. West Roseville, New South Wales, Australia: 
Simon \& Schuster. 192~p.
\bibitem{21-r-1}
\Aue{Mirimanova, M.\,S.} 2004. \textit{Konfliktologiya} 
[Conflictology]. Moscow: Academy. 320~p.
\bibitem{22-r-1}
\Aue{Kovalev, V.\,N., N.\,N.~Lopatina, and D.\,N.~Lebedenko.} 2013. \textit{Sbornik 
situatsionnykh zadach po konfliktologii} [The collection of situational tasks on 
conflictology]. Se\-va\-sto\-pol:  SB MSU. 91~p.
\bibitem{23-r-1}
\Aue{Gotin, S.\,V., and L.\,P.~Kalosha.} 2007. \textit{Logiko-strukturnyy podkhod 
i~ego primenenie dlya analiza i~planirovaniya deya\-tel'\-nosti} [Logical-structural 
approach and its application for the analysis and planning of activities]. Moscow: 
Variant. 118~p. 
\bibitem{24-r-1}
\Aue{Dettmer, W.} \textit{Goldratt's theory of constraints: A~systems approach to 
continuous improvement.} ASQ Quality Press 2006. 377~p. Available at: {\sf 
https://econ.wikireading.ru/64894} (accessed May~27, 2019).
 \end{thebibliography}

 }
 }

\end{multicols}

%\vspace*{-7pt}

\hfill{\small\textit{Received June 11, 2019}}

%\pagebreak

%\vspace*{-22pt}
 
   
   
   
   \Contr
   
\noindent
\textbf{Rumovskaya Sophiya B.} (b.\ 1985)~--- Candidate of Sciences (PhD) in 
technology, scientist, Kaliningrad Branch of the Federal Research Center 
``Computer Science and Control'' of the Russian Academy of Sciences, 5~Gostinaya 
Str., Kaliningrad 236022, Russian Federation; \mbox{sophiyabr@gmail.com}

\vspace*{3pt}

\noindent
\textbf{Kirikov Igor A.} (b.\ 1955)~--- Candidate of Sciences (PhD) in 
technology; director, Kaliningrad Branch of the Federal Research Center 
``Computer Science and Control'' of the Russian Academy of Sciences, 5~Gostinaya 
Str., Kaliningrad 236022, Russian Federation; \mbox{baltbipiran@mail.ru} 
\label{end\stat}

\renewcommand{\bibname}{\protect\rm Литература}
    