\def\stat{gaidamaka}

\def\tit{ФОРМАЛИЗАЦИЯ МЕТОДА РАНЖИРОВАНИЯ АЛЬТЕРНАТИВ ДЛЯ~ПРОЦЕССА ГРУППОВОГО 
ПРИНЯТИЯ РЕШЕНИЙ ПРИ~АНАЛИЗЕ СОЦИАЛЬНЫХ СЕТЕЙ$^*$}

\def\titkol{Формализация метода ранжирования альтернатив для~процесса CDM % группового принятия решений 
при~анализе социальных сетей}

\def\aut{А.\,А.~Гайдамака$^1$, Н.\,В.~Чухно$^2$, О.\,В.~Чухно$^3$, 
К.\,Е.~Самуйлов$^4$, С.\,Я.~Шоргин$^5$}

\def\autkol{А.\,А.~Гайдамака, Н.\,В.~Чухно, О.\,В.~Чухно и~др.} 
%К.\,Е.~Самуйлов$^4$, С.\,Я.~Шоргин$^5$}

\titel{\tit}{\aut}{\autkol}{\titkol}

\index{Гайдамака А.\,А.}
\index{Чухно Н.\,В.}
\index{Чухно О.\,В.} 
\index{Самуйлов К.\,Е.}
\index{Шоргин С.\,Я.}
\index{Gaidamaka A.\,A.}
\index{Chukhno N.\,V.}
\index{Chukhno O.\,V.} 
\index{Samouylov K.\,E.}
\index{Shorgin S.\,Ya.}


{\renewcommand{\thefootnote}{\fnsymbol{footnote}} \footnotetext[1]
{Публикация подготовлена при поддержке Программы РУДН <<5-100>> и~при финансовой поддержке РФФИ 
(проекты 18-07-00576 и~18-00-01555 (18-00-01685)).}}


\renewcommand{\thefootnote}{\arabic{footnote}}
\footnotetext[1]{Российский университет дружбы народов, aagajdamaka@sci.pfu.edu.ru}
\footnotetext[2]{Российский университет дружбы народов, nvchukhno@gmail.com}
\footnotetext[3]{Российский университет дружбы народов, olgachukhno95@gmail.com}
\footnotetext[4]{Российский университет дружбы народов; Институт проблем информатики Федерального исследовательского 
центра <<Информатика и~управление>> Российской академии наук, \mbox{samuylov-ke@rudn.ru}}
\footnotetext[5]{Институт проблем информатики Федерального исследовательского центра 
<<Информатика и~управление>> Российской академии наук, 
  \mbox{sshorgin@ipiran.ru}}

%\vspace*{-2pt}

   
  
\Abst{Распространение и~доступность интернет-технологий позволили по-новому взглянуть 
на социальные сети. Если раньше этот онлайн-сервис имел в~большей степени 
развлекательный характер, то сегодня, с~повышением скоростей передачи данных 
и~появлением возможности общения пользователей в~реальном времени, социальные сети, 
на платформе которых можно легко организовать опрос или провести голосование, 
становятся мощным механизмом достижения консенсуса в~процессе принятия решений. 
В~статье предложен обзор известных моделей группового принятия решений
(\textit{англ}.\ group decision making, GDM) и~формальное 
описание разработанного на его основе алгоритма принятия решений,
ис\-поль\-зу\-юще\-го большой 
объем данных в~социальной сети. Особенностью алгоритма является возможность выбора 
экспертом из огромного числа альтернатив ограниченного набора, представляющего для 
него интерес, а также учет масштабирования сети с~точки зрения числа экспертов, 
привлекаемых к~процессу группового принятия решений. Представлен метод экстраполяции 
значений оценок при масштабировании сети. Разработан и~исследован кейс для иллюстрации 
работы алгоритма.}

\KW{групповое принятие решений; анализ социальных сетей; нечеткая логика; набор 
языковых выражений}

\DOI{10.14357/19922264190310} 
  
\vspace*{2pt}


\vskip 10pt plus 9pt minus 6pt

\thispagestyle{headings}

\begin{multicols}{2}

\label{st\stat}

\section{Введение}

  С помощью интернет-технологий люди из разных стран мира могут 
присоединяться к~социальным сетям, создавать новые ресурсы и~контент\linebreak 
и~взаимодействовать с~другими пользователями~[1]. Социальные сети~--- одна 
из самых последних тенденций, обеспечивающих общение, взаимодействие, 
обмен информацией и~связь пользователей в~виртуальной среде~[2]. 
  
  В последние годы исследователи в~области группового принятия решений 
 проявили интерес к~анализу социальных 
сетей, так как последние могут быть успешно использованы для моделирования 
процесса взаимодействия между лицами, принимающими решения. Тем не 
менее социальная сеть имеет некоторые особенности, которые отличают ее от 
классических сценариев GDM~[1].

 C~одной стороны, внутри социальной сети 
взаимодействуют тысячи пользователей, однако многие из них не принимают 
непосредственного участия в~процессе голосования. Кроме того, пользователи 
общаются в~режиме реального времени~[3, 4]. С другой стороны, зачастую 
некоторые участники могут мешать достижению консенсуса, настаивая на 
своем варианте и~не уступая другим экспертам. 

Таким образом, групповое 
принятие решений~--- это, во-пер\-вых, процесс, состоящий в~выборе 
наилучшей альтернативы или множества альтернатив из всех возможных, 
учитывая мнения, выраженные группой лиц~\cite{3-gai, 5-gai}. Во-вто\-рых, это 
процесс, ориентированный на людей с~их субъективностью 
и~неопределенностью в~выставлении оценок. Поэтому в~исследованиях GDM, 
как правило, используется теория нечеткой логики (\textit{англ}.\ fuzzy  
logic)~\cite{6-gai, 7-gai, 8-gai}.
  
  Традиционные алгоритмы GDM рассматривают небольшую группу 
экспертов и~альтернатив~\cite{5-gai}. Однако сегодня чрезвычайно важно 
разрабатывать такие методы GDM, которые помогут пользователям принимать 
консенсусные решения~\cite{4-gai, 9-gai, 10-gai}. Такие методы могут быть 
использованы в~новых услугах сети интернет, например при оценке удобства 
пользования социальной сетью~\cite{5-gai, 9-gai}. 
  
  В~\cite{4-gai} выявлены важные направления дальнейших исследований 
в~рассматриваемой области. Одной из новых задач является разработка 
структур предпочтений и~их применение в~социальных сетях. Этой проблеме 
посвящена настоящая статья, которая организована следующим образом: в~разд.~2
с~использованием языка UML (Unified Modeling Language) формально описана модель 
группового принятия решений в~социальной сети~[11]. Раздел~3 
демонстрирует модули алгоритма ранжирования альтернатив. В~разд.~4
предлагается метод оценки альтернатив при масштабировании сети. 
Раздел~5 посвящен анализу кейса при оценке удобства пользования 
социальной сетью. В~заключении подведены итоги и~поставлены задачи 
дальнейших исследований.

\section{Модель группового принятия решений в~социальных сетях}

  Людям комфортнее выражать свои предпочтения с~помощью слов. Однако 
компьютеры работают с~числами. В~связи с~этим возникает проблема 
непонимания между человеком и~вычислительной машиной: человеку может 
быть сложно передать свои идеи компьютеру, а компьютеру понять их 
и~работать с~полученной информацией~\cite{6-gai, 12-gai, 13-gai}. Работа 
с~неопределенностью всегда была сложной задачей, для решения которой 
применяются различные аналитические инструменты. В~статье исследована 
лингвистическая модель принятия решений, в~которой эксперты выставляют 
оценки альтернативам, используя набор языковых выражений (\textit{англ}.\ linguistic 
term set, LTS)~\cite{8-gai, 12-gai, 14-gai}. Подход основан на упорядочении 
лингвистических терминов в~наборе LTS, использует индексирование 
и~предполагает, что все выражения одинаково информативны и~симметрично 
распределены. Каждый эксперт выбирает удобный для собственного 
оценивания набор LTS. В~конечном счете для определения итоговой оценки 
все предоставленные предпочтения выражаются с~помощью одного и~того же 
набора языковых выражений, называемого базовым (\textit{англ}.\ basic LTS, 
BLTS)~\cite{5-gai, 12-gai}.
  
  Обозначим символом~$e$ некоторого эксперта, а~символом~$x$ некоторую 
альтернативу принимаемого решения. Пронумеруем экспертов и~альтернативы. 
Тогда ${{E}}\hm= \{ e_1,\ldots , e_K\}$~--- множество экспертов, такое что 
$\vert {E}\vert \hm=K\hm<\infty$, где $K$~--- число экспертов, и~${X}\hm= 
\{x_1, \ldots ,x_M\}$~--- множество альтернатив, такое что $\vert {X}\vert 
\hm=M\hm<\infty$, где $M$~--- число альтернатив. $X_k\hm\subseteq X$~--- 
подмножество альтернатив, выбранных экспертом~$e_k$, $e_k\hm\in E$. 
Значение предпочтения эксперта~$e_k$ для альтернативы~$x_i$ по отношению 
к~альтернативе~$x_j$ обозначим переменной~$p_{ij}(k)$, $i\not=j$,  $x_i, 
x_j\hm\in X$.
  
  Цель исследуемого процесса GDM~--- упорядочение альтернатив от 
наилучшей к~наихудшей~\cite{4-gai}. В~статье рассматривается формальная 
модель процесса GDM, состоящая из двух модулей, как показано на рис.~1 
в~диаграмме, построенной на языке UML. 
  
\begin{figure*} %fig1
   \vspace*{1pt}
    \begin{center}  
  \mbox{%
 \epsfxsize=156.612mm 
 \epsfbox{gai-1.eps}
 }
\end{center}
\vspace*{-9pt}
\Caption{Модель группового принятия решений~--- ранжирования альтернатив}
\end{figure*}

   \begin{figure*}[b] %fig2
      \vspace*{1pt}
    \begin{center}  
  \mbox{%
 \epsfxsize=77.522mm 
 \epsfbox{gai-2.eps}
 }
\end{center}
\vspace*{-9pt}
\Caption{Иллюстрация величины $\beta(t)$ при переводе оценок из шкалы $S(t)$ 
в~шкалу~$S(t^\prime)$ }
\end{figure*}

\section{Модули алгоритма ранжирования альтернатив}

Модуль~1 алгоритма <<Оценивание альтернатив экспертами>> отвечает за 
вычисление предпочтений экспертов и~включает~7~шагов.
\begin{enumerate}[1.]
\item \textit{Определение множества экспертов $E\hm= \{ e_1,\ldots , e_K\}$ 
и~множества альтернатив} $X\hm= \{ x_1,\ldots ,x_M\}$.
\item \textit{Выбор экспертом подмножества альтернатив}. 

  Модель предусматривает, что эксперт~$e_k$ вы\-став\-ля\-ет оценки для 
некоторого подмножества~$X_k$ из множества альтернатив~$X$.
\item \textit{Сравнение альтернатив.}
\begin{enumerate}[3.1]
  \item \textit{Выбор LTS.} Пусть~$t$~--- уровень иерархии набора языковых 
выражений LTS и~$n(t)$~--- число оценок по шкале оценок уровня~$t$, 
$t\hm=1,\ldots , T$, причем $n(t)\hm< n(t+1)$. Обозначим $S(t)\hm= \left\{ s_1(t), 
\ldots, s_{n(t)}(t)\right\}$ шкалу уровня~$t$, а~$s_l(t)$ символьную оценку из 
набора оценок этой шкалы, $l\hm\in \{1, \ldots, n(t)\}$, $\vert S(t)\vert \hm= n(t)$. 
Считаем оценки упорядоченными по возрастанию значений предпочтений: 
$s_1(t)\prec s_2(t)\prec \cdots \prec s_{n(t)}(t)$. 
  
  Заметим, что каждый эксперт самостоятельно делает выбор шкалы, с~которой 
он будет работать. При этом эксперты, которые выбирают шкалу с~высокой 
сте\-пенью детализации, вероятнее всего, готовы дать более точную оценку, 
а~в~случаях, когда эксперт не чувствует себя уверенным при сравнении 
альтернатив, он может пользоваться шкалами оценок уровня ниже. Тогда 
множество экспертов~$E$ представимо в~виде 
$$
E= \bigcup\limits_{t=1}^T 
E(t)\,,
$$
 где 
\begin{multline*}
\hspace*{-9.5pt}E(t)={}\\
 \hspace*{-3.5pt} {}= \left\{ e_k \in E:\ \mbox{эксперт}~e_k\ \mbox{выбрал\ 
LTS}~S(t)\right\}.\hspace*{-6.24257pt}
\end{multline*}
  
  \item  \textit{Предоставление информации.} Для всех пар ($x_i,x_j$), таких 
что $i\not= j$, $x_i, x_j\hm\in X$,\linebreak эксперт~$e_k$ указывает свою оценку~$s_l(t)$ 
в~качестве значения элемента~$p_{ij}(k)$. Полученные значения формируют 
матрицу предпочтений $\mathbf{P}_k\hm= \left( p_{ij}(k)\right)_{i,j\in \{1,\ldots 
,M\}}$, $p_{ij}(k)\hm\in \{ s_1(t),\ldots , s_{n(t)}(t)\}$. 
  \end{enumerate}
  
\item \textit{Перевод оценок эксперта~$e_k$ в~численные значения}. 

  Для перевода оценок может быть использован один из известных 
методов~\cite{7-gai}, выбор которого не существен с~точки зрения 
представленных в~данной статье результатов. Матрица 
предпочтений~$\mathbf{P}_k$ модифицируется в~матрицу предпочтений 
$\mathbf{V}_k\hm= \left( v_{ij}(k)\right)_{i,j\in \{1,\ldots, M\}}$, например 
путем замены для каждого элемента символьной оценки~$s_l(t)$ на 
значение~$l$.
\item \textit{Выбор базового набора языковых выражений BLTS. }

  В качестве базового может быть выбран любой из имеющихся наборов 
языковых выражений LTS, а по умолчанию выбирается шкала оценок 
с~наибольшей детализацией. В~предположении, что шкалы упорядочены по 
возрастанию числа оценок, т.\,е. $S(1)\prec S(2)\prec \cdots \prec S(T)$, по 
умолчанию базовым набором является набор уровня~$T$, а~именно: 
$S(t^\prime)\hm= S(T)$.
\item \textit{Перевод оценок из LTS в~BLTS. }

  Пусть $B(t)$~--- множество пар вида 
  $(s_l(t),\alpha)$, таких что $s_l(t)\hm\in 
S(t)$ и~$\alpha \hm\in [-0{,}5, 0{,}5)$, т.\,е.
     \begin{equation}
     B(t)=S(t)\times [-0{,}5, 0{,}5)\,.
     \label{e1-gai}
     \end{equation}
  
  Определим функцию~$F$, такую что
  \begin{equation*}
  F:\ B(t)\to B\left( t^\prime\right)\,.
 % \label{e2-gai}
  \end{equation*}
  
  Введем действительную величину $\beta(t^\prime)\hm\in \left( 0, 
n(t^\prime)\right]$. Величина~$\beta(t^\prime)$ необходима для определения 
оценки $s_{l^\prime}(t^\prime)\hm\in S(t^\prime)$, в~которую переводится оценка 
$s_l(t)\hm\in S(t)$, поставленная экспертами по шкале~$S(t)$. Заметим, что 
в~(\ref{e1-gai}) для значений~$\alpha$ выбран интервал $[-0{,}5; 0{,}5)$ 
потому, что для получения числовой оценки~$l^\prime$ 
величина~$\beta(t^\prime)$ будет округляться до ближайшего целого по 
шкале~$S(t^\prime)$. Чтобы оценке~$s_l(t)$ по шкале~$S(t)$ поставить 
в~соответствие некоторое численное значение~$l^\prime$ по 
шкале~$S(t^\prime)$, необходимо определить положение со\-от\-вет\-ст\-ву\-ющей 
величины~$\beta(t^\prime)$ на отрезке $(0,n(t^\prime)]$. При этом 
величиной~$\alpha$ обозначено смещение величины~$\beta(t^\prime)$ 
относительно ближайшего целого числа~$l^\prime$, соответствующего 
символьной оценке~$s_{l^\prime}(t^\prime)\hm\in S(t^\prime)$. Заметим, что для 
шкалы~$S(t)$ величина $\beta(t)\hm= l$ принимает целые значения, 
$\beta(t)\hm\in \{1,\ldots, n(t)\}$, а~смещение~$\alpha$ равно нулю. На рис.~2 
введенные обозначения про\-ил\-люст\-ри\-ро\-ва\-ны для оценки $l\hm=2$ по шкале 
уровня $t\hm=3$, которой в~базовом наборе языковых выражений уровня 
$t^\prime\hm= 5$ будет соответствовать оценка $l^\prime\hm=3$.
   


  Будем строить отображение~$F$ так, чтобы оно было взаимно однозначным. 
Поскольку~$F$ строится для наборов языковых выражений со шкалами~$S(t)$ 
и~$S(t^\prime)$, то смещение будем выбирать таким образом, чтобы 
выполнялось соотношение:
  \begin{equation}
  \fr{\beta(t^\prime)}{\beta(t)}=\fr{n(t^\prime)}{n(t)}\,.
  \label{e3-gai}
  \end{equation}
  
  Значение $\beta(t^\prime)$, соответствующее оценке 
$s_{l^\prime}(t^\prime)\hm\in S(t^\prime)$, должно быть получено  
из~(\ref{e3-gai}) по известной после шага~3 модуля~1 оценке $s_l(t)\hm\in S(t)$. 
  
  Введем отображение~$\Delta$ следующим образом:
  \begin{equation*}
  \Delta\left( \beta(t^\prime)\right)=\left( 
s_{l^\prime}\left(t^\prime\right),\alpha\right)\,.
  %\label{e4-gai}
  \end{equation*}
  
  Очевидно, что $\alpha\hm= \beta(t^\prime)\hm- [ \beta(t^\prime)]$, 
$l^\prime\hm=[\beta(t^\prime)]$, поэтому, положив

\noindent
  \begin{equation*}
  \Delta\left( \beta (t^\prime)\right) =\left( s_{[\beta(t^\prime)]}(t^\prime), 
\beta(t^\prime)-\left[ \beta(t^\prime)\right]\right)\,,
  %\label{e5-gai}
  \end{equation*}
имеем:
\begin{equation*}
\beta(t^\prime)= \Delta^{-1} \left( s_{[\beta(t^\prime)]}(t^\prime), \beta(t^\prime)-
\left[ \beta(t^\prime)\right]\right)\,.
%\label{e6-gai}
\end{equation*}
  
  Поскольку $F\hm= \Delta(\beta(t^\prime))$, $t^\prime\hm= 1,\ldots , T$, то 
получена формула для перевода оценок из шкалы~$S(t)$ в~$S(t^\prime)$ 
и~доказана следующая теорема.
  
  \smallskip
  
  \noindent
  \textbf{Теорема~1.}\ \textit{Если $S(t)\prec S(t^\prime)$ 
и~$\beta(t^\prime)/\beta(t)\hm= n(t^\prime)/n(t)$, то отображение $F:\ B(t)\hm\to 
B(t^\prime)$ является взаимно однозначным и~определяется формулой}:
  \begin{multline*}
  F\left( s_{\beta(t)}(t),0\right)=\Delta 
  \left( \beta(t^\prime)\right) ={}\\
  {}=\left( 
s_{[\beta(t^\prime)]}, \beta(t^\prime)-\left[ \beta(t^\prime)\right]\right)\,,
 % \label{e7-gai}
  \end{multline*}
\textit{где}
\begin{equation*}
\beta(t^\prime) =\fr{\beta(t) n(t^\prime)}{n(t)}\,.
%\label{e8-gai}
\end{equation*}
\item \textit{Обновление матрицы предпочтений экспертов. }

  По матрице предпочтений~$\mathbf{V}_k$ должна быть вычислена матрица 
предпочтений $\tilde{\mathbf{V}}_k\hm= \left( \tilde{v}_{ij}(k)\right)_{i,j\in 
1,\ldots M}$, полученная с~помощью алгоритма перевода численных оценок из 
LTS в~BLTS. Результатом работы модуля~1 алгоритма являются матрицы 
предпочтений экспертов, представленных в~численном виде в~едином базовом 
наборе языковых выражений.
  \end{enumerate}
  
  \smallskip
  
  
  %\begin{figure*}
  \begin{center}
  {\small
  \hrule
\vspace*{6pt}

  \textbf{Алгоритм перевода численных оценок из LTS в~BLTS}
  
  \vspace*{6pt}
  
  \hrule
  
  \vspace*{6pt}
  
  \begin{tabular}{l}
    \textbf{Data:}\    Численные оценки экспертов\\
    \hphantom{\textbf{Data:}\ }$v_{ij}(k)=1,\ldots, n(t)$, 
$e_k\hm\in E(t)$, $S(t)$, $S(t^\prime)$\\
        \hspace*{3mm}\textbf{procedure} $\mathbf{V}_k\to 
\tilde{\mathbf{V}}_k$\\
     \hspace{6mm}\textbf{for} all $e_k$, $k=1$ to~$K$\\
          \hspace*{9mm}\textbf{if} $S(t)\not= S(t^\prime)$ \textbf{do}\\
          \hspace*{12mm}\textbf{for all} $v_{ij}(k)_{i,j\in 1,\ldots M}$ \textbf{do}\\
          \hspace*{12mm}$\beta(t)=v_{ij}(k)$\\
          \hspace*{12mm}$\beta(t^\prime)=\fr{\beta(t)n(t^\prime)}{n(t)}$\\
    \hspace*{12mm}$\tilde{v}_{ij}(k)=\beta(t^\prime)$\\
     \hspace*{12mm}\textbf{end for}\\
       \hspace*{9mm}\textbf{else} $\tilde{v}_{ij}(k)=v_{ij}(k)$\\
          \hspace*{9mm}\textbf{end if}\\
          \hspace{6mm}\textbf{end for}\\
       Вывести новые оценки экспертов $\tilde{v}_{ij}(k)$\\
         \textbf{end procedure} $\mathbf{V}_k\to \tilde{\mathbf{V}}_k$
         \end{tabular}
  
  \vspace*{6pt}
  
  \hrule
}
\end{center}

\smallskip
%\end{figure*}

Модуль~2 алгоритма <<Расчет средних оценок, создание рейтинга альтернатив 
и~анализ степени участия экспертов>> отвечает за вычисление матрицы 
усредненных оценок, полученных на основании матриц предпочтений 
экспертов~$\tilde{\mathbf{V}}_k$, а~также за формирование ранжированного 
списка альтернатив. Матрица $\tilde{\mathbf{V}}\hm= (\tilde{v}_{ij})_{i,j\in 
1,\ldots, M}$ содержит сумму всех значений предпочтений, предоставленных 
экспертами:

\noindent
\begin{equation}
\tilde{\mathbf{V}}=\sum\limits^K_{k=1} \tilde{\mathbf{V}}_k\,.
\label{e9-gai}
\end{equation}
  
  Матрица~$\mathbf{N}_k$ содержит информацию об альтернативах, 
выбранных для сравнения экспертом~$e_k$:
  \begin{equation}
  n_{ij}(k)=\begin{cases}
  1\,, & x_i, x_j\in X_k\,;\\
  0\,, & x_i, x_j\not\in X_k\,.
  \end{cases}
  \label{e10-gai}
  \end{equation}
  
  Тогда матрица $\mathbf{N}\hm= (n_{ij})_{i,j\in 1,\ldots , M}$, содержащая 
информацию о числе экспертов, участвующих в~попарном сравнении 
альтернатив~$x_i$ и~$x_j$, может быть вычислена по формуле:

\noindent
  \begin{equation}
  \mathbf{N}=\sum\limits^K_{k=1} \mathbf{N}_k\,.
  \label{e11-gai}
  \end{equation}
  
  Элементы матрицы усредненных оценок $\overline{\mathbf{V}} \hm= 
(\overline{v}_{ij})_{i,j\in 1,\ldots ,M}$ вычисляются следующим образом:

\vspace*{2pt}

\noindent
  \begin{equation}
  \overline{v}_{ij} = \fr{\tilde{v}_{ij}}{n_{ij}}\,.
  \label{e12-gai}
  \end{equation}
  
  \vspace*{-2pt}
  
  Для ранжирования альтернатив используются, например, операторы $\mathrm{GDD}$ 
(\textit{англ}.\ Guided Dominance Degree) и~$\mathrm{GNDD}$ (\textit{англ}.\ 
Guided 
Non Dominance Degree)~\cite{10-gai, 14-gai}. При этом оператор $\mathrm{GDD}$ 
показывает, насколько оце\-ни\-ва\-емая альтернатива доминирует над всеми 
остальными, т.\,е.\ насколько\linebreak эта альтернатива лучше всех остальных, 
а~оператор $\mathrm{GNDD}$~--- насколько %\linebreak
 оце\-ни\-ва\-емая аль\-тер\-натива не доминируется
остальными. Усредняя значения~$\mathrm{GDD}$ и~$\mathrm{GNDD}$, получаем значение 
ранжирования~$\mathrm{RV}$ (\textit{англ}.\ Ranking Value):
\begin{equation}
\mathrm{RV}_i = \fr{ \mathrm{GDD}_i+\mathrm{GNDD}_i}{2}\,,\enskip i=1,\ldots , M\,,
  \label{e15-gai}
  \end{equation}
  
  \vspace*{-2pt}
  
  \noindent
  где
  
  \vspace*{-6pt}
  
\noindent
  \begin{align*}
  \mathrm{GDD}_i&= \sum\limits^M_{j=1} \overline{v}_{ij}\,,\enskip i=1,\ldots ,M\,;
  %\label{e13-gai}
  \\
  \mathrm{GNDD}_i&=\! \sum\limits^M_{j=1} \max\left\{ \overline{v}_{ji}-
\overline{v}_{ij},1\right\}\,,\enskip  i=1,\ldots , M\,.
 % \label{e14-gai}
    \end{align*}
    
    \vspace*{1pt}
  
  Из значений ранжирования формируется рейтинг альтернатив по 
возрастанию степени значимости~--- предпочтительнее будет альтернатива, 
которая имеет более высокое значение~$\mathrm{RV}$.

 \begin{figure*}[b] %fig3
      \vspace*{1pt}
    \begin{center}  
  \mbox{%
 \epsfxsize=104.051mm 
 \epsfbox{gai-3.eps}
 }
\end{center}
\vspace*{-9pt}
\Caption{Лепестковая диаграмма~--- рейтинг альтернатив}
\end{figure*}
  
  Заключительный шаг модуля~2 процесса GDM посвящен оценке 
достоверности полученных результатов. Коэффициент участия (\textit{англ}.\ 
Participation Rate) служит одним из ключевых показателей эффективности 
процесса группового принятия решений. Высокий коэффициент 
свидетельствует о~том, что полученные оценки достаточно точны, а~результаты 
процесса GDM надежны. Так как в~процессе принятия решения эксперты 
сравнивают не все альтернативы, а только выбранное ими подмножество 
альтернатив, то данный коэффициент необходимо вычислить для каждой 
альтернативы:
  \begin{equation}
 \mathrm{PR}_i=\fr{1}{M}\sum\limits^M_{j=1} n_{ij}\,,\enskip i=1,\ldots , M\,.
  \label{e16-gai}
  \end{equation}
Общий коэффициент участия (\textit{англ}.\ the overall participation rate) определим, как 
среднее по всем~$\mathrm{PR}_i$, т.\,е.
\begin{equation*}
\mathrm{PR}=\fr{1}{M}\sum\limits^M_{i=1} \mathrm{PR}_i\,.
%\label{e17-gai}
\end{equation*}
  
  Результатом работы модуля~2 алгоритма является ранжированный список 
альтернатив и~коэффициенты участия для каждой альтернативы.

\section{Метод оценки альтернатив при~масштабировании сети}

  Привлечение дополнительного эксперта к~процессу оценивания альтернатив 
не вызывает необходимости пересчитывать усредненные оценки по 
формулам~(\ref{e9-gai})--(\ref{e12-gai}). Пусть $\tilde{\mathbf{V}}_{K+1}$~--- 
матрица предпочтений эксперта~$e_{K+1}$. Тогда матрица усредненных 
оценок~$\overline{\mathbf{V}}$ для нового набора экспертов $E_{K+1}\hm= 
E_K\bigcup \left\{ e_{K+1}\right\}$ может быть вычислена по формуле:
\vspace*{-2pt}

\noindent
  \begin{multline*}
  \overline{v}_{ij}(K+1)=\fr{1}{n_{ij}(K+1)}\,\tilde{v}_{ij} (K+1)+{}\\
  {}+\left( 1- \fr{1} 
{n_{ij}(K+1)}\right) \overline{v}_{ij}(K)\,.
 % \label{e18-gai}
  \end{multline*}
  
\vspace*{-16pt}

\section{Численный эксперимент}

\vspace*{-6pt}

  При выполнении численного эксперимента был проведен опрос с~целью 
определения, какая из~27~социальных сетей, представленных на рис.~3, 
наиболее удобна для общения по мнению~20~экспертов. 
  
  С использованием изложенной выше методики и~алгоритма было 
разработано программное средство для расчета значений ранжирования по 
формуле~(\ref{e15-gai}), проиллюстрированных на рис.~3.
   
  

  По формуле~(\ref{e16-gai}) для каждой из~27~альтернатив был рассчитан 
коэффициент участия~$\mathrm{PR}_i$, показанный в~таблице.
    Из таблицы видно, что наиболее удобными для общения, по 
мнению~20~экспертов из России, оказались сети
 Vkontakte (альтернатива~$x_1$)\linebreak\vspace*{-10pt}

\pagebreak

%\begin{table*}
{\small
  \begin{center}
  \tabcolsep=7pt
  \begin{tabular}{|l|c|}
  \multicolumn{2}{p{32mm}}{Коэффициент учас\-тия каждой альтернативы}\\
  \multicolumn{2}{c}{\ }\\[-6pt]
  \hline
$x_i$&$\mathrm{PR}_i$\\
\hline
$x_1$&3\hphantom{,99999}\\
$x_2$&1,33333\\
$x_3$&1,62963\\
$x_4$&3\hphantom{,99999}\\
$x_5$&1,66667\\
$x_6$&0,62963\\
$x_7$&1,07407\\
$x_8$&2,62963\\
$x_9$&2,18519\\
$x_{10}$&0,92593\\
$x_{11}$&0,74074\\
$x_{12}$&0,37037\\
$x_{13}$&0,18519\\
$x_{14}$&0,74074\\
$x_{15}$&0,22222\\
$x_{16}$&0,22222\\
$x_{17}$&0,37037\\
$x_{18}$&0,37037\\
$x_{19}$&0,18519\\
$x_{20}$&1,92593\\
$x_{21}$&2,33333\\
$x_{22}$&0,62963\\
$x_{23}$&1\hphantom{,99999}\\
$x_{24}$&0,22222\\
$x_{25}$&1\hphantom{,99999}\\
$x_{26}$&0,88889\\
$x_{27}$&1,44444\\
  \hline
  \end{tabular}
  \end{center}
  }
%  \end{table*}

\vspace*{9pt}

\noindent
 и~YouTube (альтернатива~$x_4$), а~сеть Kindernet 
(альтернатива~$x_{19}$) наименее удобна. Общий коэффициент участия 
$\mathrm{PR}\hm=1{,}1454$, т.\,е.\ в~среднем всего~1~эксперт предоставил информацию 
для каждой пары альтернатив. Поскольку коэффициент участия достаточно 
низок, полученные результаты не являются надежными. Для получения точных 
результатов следует привлечь к~процессу GDM большее число экспертов.

\section{Заключение}

  С каждым годом пользователи сети интернет становятся более 
вовлеченными в~процесс предо\-став\-ле\-ния информации, что приводит 
к~созданию в~рамках социальных сетей среды хранения огромного объема 
доступных каждому пользователю данных. В~статье представлен новый метод 
группового принятия решений, который предназначен для работы с~такими 
типами сред. Задачей дальнейших исследований является применение метода 
в~масштабных численных экспериментах. При этом помимо вычисления 
усредненных величин планируется также оценивать другие показатели 
эффективности процесса принятия решений, в~том числе старшие моменты 
исследуемых характеристик. Еще одной важной задачей является измерение 
и~оценка достигнутого экспертами консенсуса, поскольку при невысокой 
степени согласованности экспертов полученный результат нельзя считать 
надежным. 
  
{\small\frenchspacing
 {%\baselineskip=10.8pt
 \addcontentsline{toc}{section}{References}
 \begin{thebibliography}{99}
\bibitem{1-gai}
\Au{Wasserman S., Faust~K.} Social network analysis: Methods and applications.~--- 
Cambridge: Cambridge University Press, 1994. Vol.~8. 857~p.
\bibitem{2-gai}
\Au{Wu J., Xiong R., Chiclana~F.} Uninorm trust propagation and aggregation methods for 
group decision making in social network with for tuple information~// Knowl.-Based 
Syst., 2016. Vol.~96. P.~29--39.
\bibitem{3-gai}
\Au{Urea R., Cabrerizo~F.\,J., Morente-Molinera~J.\,A., Herrera-Viedma~E.} GDM-R: 
A~new framework in R to support fuzzy group decision making processes~// Inform. 
Sciences, 2016. Vol.~357. P.~161--181.
\bibitem{4-gai}
\Au{Herrera-Viedma E., Cabrerizo~F., Chiclana~F., Wu~J., Cobo~M., Samuylov~K.} 
Consensus in group decision making and social networks~// Stud. Inform.  Control, 2017. 
Vol.~26. Iss.~3. P.~259--268.
\bibitem{5-gai}
\Au{Herrera-Viedma E., Cabrerizo~F.\,J., Kacprzyk~J., Pedrycz~W.} A~review of soft 
consensus models in a fuzzy environment~// Inform. Fusion, 2014. Vol.~17. P.~4--13.
\bibitem{6-gai}
\Au{Zadeh L.\,A.} Fuzzy sets~// Inform. Control, 1965. Vol.~8. Iss.~3. P.~338--353.

\bibitem{8-gai} %7
\Au{Zadeh L.\,A.} Fuzzy logic\;=\;computing with words~// IEEE T.~Fuzzy Syst., 1996. 
Vol.~4. Iss.~2. P.~103--111.
\bibitem{7-gai} %8
\Au{Torra V.} Hesitant fuzzy sets~// Int. J.~Intell. Syst., 2010. Vol.~25. 
Iss.~6. P.~529--539.

\bibitem{9-gai}
\Au{Kacprzyk J.} Group decision making with a fuzzy linguistic majority~// Fuzzy Set. 
Syst., 1986. Vol.~18. P.~105--118.
\bibitem{10-gai}
\Au{Chiclana F., Tapia Garcia~J., del Moral~M.\,J., Herrera-Viedma~E.} A~statistical 
comparative study of different similarity measures of consensus in group decision making~// 
Inform. Sciences, 2013. Vol.~221. P.~110--123.
\bibitem{11-gai}
\Au{Самуйлов К.\,Е., Серебренникова~Н.\,В., Чукарин~А.\,В., Яркина~Н.\,В.} Основы 
формальных методов описания биз\-нес-про\-цес\-сов.~--- М.: РУДН, 
2011. 130~c.

\bibitem{13-gai} %11
\Au{Tanino T.} Fuzzy preference orderings in group decision making~// Fuzzy Set.  Syst., 
1984. Vol.~12. P.~117--131. 
\bibitem{12-gai} %12
\Au{Herrera F., Martinez~L.\,A.} 2-tuple fuzzy linguistic representation model for computing 
with words~// IEEE T.~Fuzzy Syst., 2000. Vol.~8. Iss.~6. P.~746--752.
\bibitem{14-gai} %13
\Au{Herrera F., Herrera-Viedma~E.} Choice functions and mechanisms for linguistic 
preference relations~// Eur. J.~Oper. Res., 2000. Vol.~120. Iss.~1. P.~144--161.
 \end{thebibliography}

 }
 }

\end{multicols}

\vspace*{-6pt}

\hfill{\small\textit{Поступила в~редакцию 01.07.19}}

%\vspace*{8pt}

%\pagebreak

\newpage

\vspace*{-28pt}

%\hrule

%\vspace*{2pt}

%\hrule

%\vspace*{-2pt}

\def\tit{FORMALIZATION OF~THE~ALTERNATIVES RANKING METHOD FOR~GROUP DECISION MAKING 
IN~SOCIAL NETWORKS}


\def\titkol{Formalization of~the~alternatives ranking method for~group decision making 
in~social networks}

\def\aut{A.\,A.~Gaidamaka$^1$, N.\,V.~Chukhno$^1$, O.\,V.~Chukhno$^1$, 
K.\,E.~Samouylov$^{1,2}$, and~S.\,Ya.~Shorgin$^2$}

\def\autkol{A.\,A.~Gaidamaka, N.\,V.~Chukhno, O.\,V.~Chukhno, et al.} 
%K.\,E.~Samouylov, and~S.\,Ya.~Shorgin}

\titel{\tit}{\aut}{\autkol}{\titkol}

\vspace*{-11pt}


\noindent
$^1$Peoples' Friendship University of Russia (RUDN University), 6~Miklukho-Maklaya Str., 
Moscow 117198, Russian\linebreak
$\hphantom{^1}$Federation

\noindent
$^2$Institute of Informatics Problems, Federal Research Center ``Computer Science and Control'' of the 
Russian\linebreak
$\hphantom{^1}$Academy of Sciences; 44-2~Vavilov Str., Moscow 119333, Russian Federation


\def\leftfootline{\small{\textbf{\thepage}
\hfill INFORMATIKA I EE PRIMENENIYA~--- INFORMATICS AND
APPLICATIONS\ \ \ 2019\ \ \ volume~13\ \ \ issue\ 3}
}%
 \def\rightfootline{\small{INFORMATIKA I EE PRIMENENIYA~---
INFORMATICS AND APPLICATIONS\ \ \ 2019\ \ \ volume~13\ \ \ issue\ 3
\hfill \textbf{\thepage}}}

\vspace*{3pt}  

 

\Abste{The expansion and accessibility of Internet technology has allowed a new look at social 
networks. A few decades ago, this online service was more entertaining in nature. However, today, 
with increasing transmission rates and the possibility of real-time communication, social networks, 
on which platform a~poll or vote can be easily organized, become a~powerful mechanism for 
achieving consensus in decision making process. The paper offers an overview of the
known models of 
group decision making (GDM) and a~formal description of the decision-making algorithm 
developed on the basis of the overview taking into account a~large amount of data in a social 
network. A~feature of the algorithm is the ability for an expert to choose a~limited subset of 
interest from a huge number of alternatives, as well as taking into account network scaling in terms 
of the number of experts involved in the GDM process. The method of 
extrapolating the values of the estimates for network scaling is proposed. 
A~numerical case for an 
illustration of the algorithm is developed and presented.} 


\KWE{group decision making; social network analysis; fuzzy logic; LTS}


  
 
\DOI{10.14357/19922264190310} 

%\vspace*{-14pt}

\Ack
\noindent
The publication has been prepared with the support of the ``RUDN University Program 5-100'' and funded by the 
Russian Foundation for Basic Research according to the research projects
 Nos.\,18-07-00576 and 18-00-01555  
(18-00-01685).


%\vspace*{-6pt}

  \begin{multicols}{2}

\renewcommand{\bibname}{\protect\rmfamily References}
%\renewcommand{\bibname}{\large\protect\rm References}

{\small\frenchspacing
 {%\baselineskip=10.8pt
 \addcontentsline{toc}{section}{References}
 \begin{thebibliography}{99}
\bibitem{1-gai-1}
\Aue{Wasserman, S., and K.~Faust.} 1994. \textit{Social network analysis: Methods and 
applications}. Cambridge: Cambridge University Press.  Vol.~8. 857~p.
\bibitem{2-gai-1}
\Aue{Wu, J., R.~Xiong, and F.~Chiclana.} 2016. Uninorm trust propagation and aggregation 
methods for group decision making in social network with for tuple information. 
\textit {Knowl.-Based Syst.} 96:29--39.
\bibitem{3-gai-1}
\Aue{Urea, R., F.\,J.~Cabrerizo, J.\,A.~Morente-Molinera, and E.~Herrera-Viedma.} 2016. 
GDM-R: A~new framework in~R to support fuzzy group decision making processes. 
\textit{Inform. Sciences} 357:161--181.
\bibitem{4-gai-1}
\Aue{Herrera-Viedma, E., F.~Cabrerizo, F.~Chiclana, J.~Wu, M.~Cobo, and K.~Samuylov}. 
2017. Consensus in group decision making and social networks. \textit{Stud. Inform. 
Control} 26(3):259--268.
\bibitem{5-gai-1}
\Aue{Herrera-Viedma, E., F.\,J.~Cabrerizo., J.~Kacprzyk, and W.~Pedrycz.} 2014. A~review 
of soft consensus models in a~fuzzy environment. \textit{Inform. Fusion} 17:4--13.
\bibitem{6-gai-1}
\Aue{Zadeh, L.\,A.} 1965. Fuzzy sets. \textit{Inform. Control} 8(3):338--353.

\bibitem{8-gai-1}
\Aue{Zadeh, L.\,A.} 1996. Fuzzy logic\;=\;computing with words. \textit{IEEE T.~Fuzzy 
Syst.} 4(2):103--111.
\bibitem{7-gai-1}
\Aue{Torra, V.} 2010. Hesitant fuzzy sets. \textit{Int. J.~Intell. Syst.} 
25(6):529--539.

\bibitem{9-gai-1}
\Aue{Kacprzyk, J.} 1986. Group decision making with a fuzzy linguistic majority. \textit{Fuzzy 
Set. Syst.} 18:105--118.
\bibitem{10-gai-1}
\Aue{Chiclana, F., J.~Tapia Garcia, M.\,J.~del Moral, and E.~Herrera-Viedma.} 2013. 
A~statistical comparative study of different similarity measures of consensus in group decision 
making. \textit{Inform. Sciences} 221:110--123.
\bibitem{11-gai-1}
\Aue{Samouylov, K.\,E., N.\,V.~Serebrennikova, A.\,V.~Chukarin, and N.\,V.~Jarkina.} 2011. 
\textit{Osnovy formal'nykh metodov opisaniya biznes-protsessov} [Basics of 
formal methods of describing business processes].  Moscow: RUDN. 130~p.

\bibitem{13-gai-1} %11
\Aue{Tanino, T.} 1984. Fuzzy preference orderings in group decision making. 
\textit{Fuzzy Set. Syst.} 12:117--131.
\bibitem{12-gai-1} %12
\Aue{Herrera, F., and L.\,A.~Martinez.} 2000. 2-tuple fuzzy linguistic representation model for 
computing with words. \textit{IEEE T.~Fuzzy Syst.} 8(6):746--752.

\bibitem{14-gai-1}
\Aue{Herrera, F., and E.~Herrera-Viedma.} 2000. Choice functions and mechanisms for 
linguistic preference relations. \textit{Eur. J.~Oper. Res.} 120(1):144--161.
\end{thebibliography}

 }
 }

\end{multicols}

%\vspace*{-7pt}

\hfill{\small\textit{Received July 1, 2019}}

%\pagebreak

%\vspace*{-22pt}

\Contr

\noindent
\textbf{Gaidamaka Anna A.} (b.\ 1997)~--- Master Student of 
Applied Probability and Informatics Department, Peoples' 
Friendship University of Russia (RUDN University), 6~Miklukho-Maklaya Str., Moscow 117198, Russian Federation; 
\mbox{aagajdamaka@sci.pfu.edu.ru}

\vspace*{3pt}

\noindent
\textbf{Chukhno Nadezhda V.} (b.\ 1995)~--- Master Student of Applied Probability and 
Informatics Department, Peoples' Friendship University of Russia (RUDN University), 
6~Miklukho-Maklaya Str., Moscow 117198, Russian Federation; \mbox{nvchukhno@gmail.com}

\vspace*{3pt}

\noindent
\textbf{Chukhno Olga V.} (b.\ 1995)~--- Master Student of Applied Probability and 
Informatics 
Department, Peoples' Friendship University of Russia (RUDN University), 6~Miklukho-Maklaya 
Str., Moscow 117198, Russian Federation; \mbox{olgachukhno95@gmail.com}

\vspace*{3pt}

\noindent
\textbf{Samouylov Konstantin E.} (b.\ 1955)~--- Doctor of Science in technology, professor, 
Head of Department, Director of Institute, Peoples' Friendship University of Russia (RUDN 
University), 6~Miklukho-Maklaya Str., Moscow 117198, Russian Federation; senior scientist, 
Institute of Informatics Problems, Federal Research Center ``Computer Science and Control'' of the 
Russian Academy of Sciences, 44-2~Vavilov Str., Moscow 119333, Russian Federation; 
\mbox{samuylov-ke@rudn.ru}

\vspace*{3pt}

\noindent
\textbf{Shorgin Sergey Ya.} (b.\ 1952)~--- 
Doctor of Science in physics and mathematics, professor, principal 
scientist, Institute of Informatics Problems, Federal Research Center ``Computer Science and Control'' of 
the Russian Academy of Sciences; 44-2~Vavilov Str., Moscow 119333, Russian Federation; 
\mbox{sshorgin@ipiran.ru}
\label{end\stat}

\renewcommand{\bibname}{\protect\rm Литература}  